% !TEX root = master.tex


\begin{center}
	{\sffamily \bfseries\Large Appendix}\\
\end{center}%\hspace*{\fill}\\[1.5cm]
\vspace{1cm}
\chapter{Hyperparameters for the Fully Connected Autoencoder}
\label{Ch:ApA}
%\pagenumbering{arabic}

The finding of appropriate hyperparameters for the fully connected autoencoder is described here. The hyperparameters include number of layers i.e. depth, number of nodes per hidden layer i.e. width, batch size and non-linear activation functions, number of epochs for training and learning rate. The experiments are evaluated through the validation error which estimates the model's ability to generalize, the training error which estimates the optimization to training data and the $\L2$ as described in \cref{Ch:ROM}. Both validation- and training error are described in \cref{Ch:DimRedAl} and provide information about under- and overfitting. Moreover the validation error is the essential metric for validating a model's performance. The $\L2$ on the other hand gives an estimate of how well the model performs on the whole dataset hence is used as a comparative metric against POD.\\ 
To start with a working model first a guess about the some initial hyperparameters is done, which are summarized in \Cref{Tab:First Guess}. These include a mini-batch size of 16, the width of the bottleneck layer is 3 and 5  for \(\hy\) and \(\rare\) respectively and a learning rate of 0.0001. Activation LeakyReLU is applied for the outputs of the input- and  any hidden layer which does not output the code, referred to as activations hidden. Tanh is applied for the output of the last hidden layer in the encoder which outputs the code, referred to as activation code. A visualization of the activation scheme is provided in \cref{Fig:ActScheme}. Moreover 2000 initial number of epochs are used. This might seem exaggerated but is justified by the little amount of input data and the small size of the network which yiels a fast training.
\begin{center}
	\begin{figure}[H]
		\centering
		\begin{tikzpicture}[align=center,node distance=4.5cm]
	\node [circ,minimum size=.9cm] (input) {\footnotesize in};
	%\node [circ,right of = input] (act1) {act. a};
	\node [circ,minimum size=.9cm,right of = input] (hid1) {\footnotesize h 1};
	%\node [circ,right of = hid1] (act2) {act. b};
	\node [circ,minimum size=.9cm,right of = hid1] (hid2) {\footnotesize h 2};
	%\node [circ,right of = hid2] (act3) {act. a};
	\node [circ,minimum size=.9cm,right of = hid2] (output) {\footnotesize out};
	\draw [arrow,thick] (input)--(hid1) node [midway,yshift=1ex] {act. a};
	\draw [arrow,thick] (hid1)--(hid2) node [midway,yshift=1ex] {act. b};
	\draw [arrow,thick] (hid2)--(output) node [midway,yshift=1ex] {act. a};
\end{tikzpicture}
		\caption{\footnotesize Scheme of a network with four layers showing where the activations are placed. Activation act. a activates the output of the input layer as well as the output of any other hidden layer except the output of the last hidden layer in the encoder which is activated through act.b and outputs the code.}
		\label{Fig:ActScheme}
		\end{figure}
\end{center}
\begin{table}[H]
	\centering
	\caption{Initial selection for batch size, bottleneck size, number of epochs, learning rate and applied activation functions.}
	\begin{tabular*}{15.5cm}{ @{\extracolsep{\fill}} c c c c c @{} }
		\toprule
		Mini-batch size   & Intrinsic dimensions &   Epochs &Learning rate & Activations hidden/code \\   
		\hline
		16 		&	3/5 &     2000&	    0.0001 & LeakyReLU/Tanh\\
		\bottomrule
	\end{tabular*} \label{Tab:First Guess}
\end{table}
Five designs for finding an optimal number of layers i.e. depth are run. The hidden layers are designed to halve the input at each step. Not within this scope is the bottlenck layer which has a fixed size of 5 and 3 for $\rare$ and $\hy$ respectively and the first and last hidden layer (after the input layer and before the output layer). Those should provide an abstraction without shrinking the incoming data. Note that this design feature was validated in an initial exploration cycle which is not included in this thesis. The five designs range from 10 layers in (a) to 2 layers in (e) always taking one layer away from encoder and decoder hence decreasing by two layers and are as follows:     
\begin{enumerate}
	\item 10 layers with layer widths: 40, 40, 20, 10, 5, 3/5, 5, 10, 20, 40, 40.
	\item 8 layers with layer widths: 40, 40, 20 , 10, 3/5, 10, 20, 40, 40.
	\item 6 layers with layer widths: 40, 40, 20 , 3/5, 20, 40, 40.
	\item 4 layers with layer widths: 40, 40, 3/5, 40, 40.
	\item 2 layers with layer widths: 40, 3/5, 40.
	\end{enumerate}
The model's depth is determined in a primary step, as it sets a consequential part of the model's representational capacity and therefore can initiate over- and underfitting at an early stage in the hyperparameter search. The results of the experimentation are shown in \cref{Fig:Depth} and \cref{Tab:Depth} for both rarefaction levels.\\
\begin{table}[htp]
	\centering
	\caption{Results for the variation of depth. Given are minimum values of training and validation error as well as the \(\L2\). The minima where reached around the last 50 epochs of the training. The \(\L2\) is evaluated with the model at the last epoch.}
	\begin{tabular*}{15cm}{ @{\extracolsep{\fill}} c c c c c c c @{} }
		\toprule
		Depth & \multicolumn{2}{c}{Minimum training error} & \multicolumn{2}{c}{Minimum validation error} & \multicolumn{2}{c}{\(\L2\) }\\ [.5ex]
		 & \(\hy\)&\(\rare\)&\(\hy\)&\(\rare\)&\(\hy\)&\(\rare\)\\
		\hline
		10& \num{1.53e-7} & \num{5.96e-7} & \num{2.22e-7} & \num{5.19e-7} & 0.0048 & 0.0091\\ \hline
		8 & \num{1.17e-7 }& \num{2.05e-7} & \num{1.58e-7} & \num{2.32e-7} & 0.0041 & 0.0054\\ \hline
		6 & \num{9.76e-8} & \num{1.40e-7} & \num{1.49e-7} & \num{1.72e-7} & 0.0038 & 0.0045\\ \hline
		4 & \num{6.29e-8} & \num{1.52e-7} & \num{7.74e-8} & \num{1.61e-7} & 0.0031 & 0.0048 \\ \hline
		2 & \num{1.29e-6} & \num{3.29e-6} & \num{1.37e-6} & \num{3.42e-6} & 0.0136 & 0.0217\\ \hline
	\end{tabular*}\label{Tab:Depth}
\end{table}\noindent
For \(\hy\) the lowest validation error of \num{7.74e-8} and an \(\L2\) of 0.0031 is reached with 4 layers, hence constitutes the best performing design out of the five. Additionally, as seen in \cref{Fig:Depth}(left), does a design exceeding 4 layers results in a slight overfitting after around 500 epochs. Less than 4 layers do not reach the validation error and \(\L2\) of the other designs, yielding the conclusion, that the capacity is too low. Overfitting occurs with 4 layers only after the 1000th epoch and is less than with the other three models that show overfitting.\\
For \(\rare\) the lowest validation error of \num{1.61e-7} is reached again with 4 layers. On the other hand the lowest \(\L2\) of 0.0031 and the lowest training error of \num{1.40e-7} are reached with 6 layers. Contrary to the afore discussed case the training error and \(\L2\) are of lower magnitude for 6 layers, except for the validation error. Looking at \cref{Fig:Depth}(right), we observe that the model with 6 layers starts to overfit after the 1500 epochs, yielding a decreasing training error and a stagnating validation error. Hence the model improved in the optimization task which additionally improves the \(\L2\). It's generalization ability, measured by the validation error, did not improve and is greater than the validation error reached with 4 layers. This concludes, a model with 4 layers constitutes the best performing design out of the five.\\
Qualitatively the overall training for both rarefaction levels is very stable. Training and validation error do not diverge exessively and converge early in training. Separation of training and validation error occurs prominently for the hydrodynamic solution. This is thought to be connected to the emersion of sharp shock fronts towards the end of the simulation. This increases variance in the whole dataset and therefore also in the training- and validation set.\\
The number of epochs is now doubled to 4000 epochs because the lowest validation error was achieved towards the end of the training in the previous experiments. Again this is justifiable as one epoch takes less than 1s to finish and no prominent overfitting is observed. 
\begin{table}[htpb!]
	\centering
	\caption{Results for the variation of width. Given is the minimum value of validation error as well as the \(\L2\). The minima where reached around the last 50 epochs of the training, the \(\L2\) is evaluated with the model at the last epoch.}
	\begin{tabular*}{15cm}{ @{\extracolsep{\fill}} c c c c c c c @{} }
		\toprule
		Hidden units & \multicolumn{2}{c}{Validation error} & \multicolumn{2}{c}{$\L2$} & \multicolumn{2}{c}{Shrinkage factor}\\ [.5ex]
		& \(\hy\)&\(\rare\)&\(\hy\)&\(\rare\)&\(\hy\)&\(\rare\)\\
		\hline
		50 & \num{1.91e-8}  & \num{5.05e-8} & \num{0.0015}  & \num{0.0025} & 0.06  & 0.01\\ \hline
		40 & \num{2.65e-08} & \num{1.65e-8} & \num{0.0018}  & \num{0.0014} & 0.075 & 0.125\\ \hline
		30 & \num{1.77e-08} & \num{3.40e-8} & \num{0.0015}  & \num{0.0021} & 0.015 & 0.0167\\ \hline
		20 & \num{2.50e-08} & \num{5.25e-8} & \num{0.0017}  & \num{0.0027} & 0.1   & 0.25 \\ \hline
		10 & \num{5.11e-08} & \num{3.97e-7} & \num{0.0025}  & \num{0.0077} & 0.3   & 0.5\\ \hline
	\end{tabular*}\label{Tab:Width}
\end{table}
The width of the two remaining hidden layers is examined in the following. For both the hydrodynamic and the rarefied regime five experiments are conducted, lowering the hidden units of the hidden layers from fifty to ten. Note that the decoder is chosen to be structurally a reflection of the encoder. Therefore only one parameter is changed. Results for \(\hy\) and \(\rare\) are shown in \cref{Tab:Width}. Note that the contribution of over- and underfitting is negligible and therefore the training error is omitted. A model with 30 hidden units in encoder and decoder performs best with \(\hy\) and reaches a validation error of \(\num{1.77e-08}\). The corresponding \(\L2 = \num{1.5e-3}\) with a shrinkage factor of 0.015. Overall the loss of each experiment with \(\hy\) is quiet similar an ranges from \(\num{1.77e-8}\) to \(\num{5.11e-8}\). The \(\L2\) behaves in a similar fashion and is even equal for 50 and 30 layers. A model with 40 hidden units performs best for \(\rare\). The corresponding validation error is \(\num{1.65e-8}\) with \(\L2=\num{1.4e-3}\), which is smaller than \(\hy\). The shrinkage factor is 0.125. In all experiments a model with 10 hidden nodes performs worst. Training and validation error over all 4000 epochs for both experiments can be seen in \cref{Fig:Width}. The aforementioned separation of training- and validation error, that was observed solely for \(\hy\), is mitigated when moving away from 40 hidden units for encoder and decoder. Not shrinking the input in the first hidden layer only serves the performance when using \(\rare\).\\
\begin{table}[H]
	\centering
	\caption{Results for the variation of batch sizes. Given is the minimum value of validation error as well as the the corresponding epoch. The \(\L2\) is also given but evaluated with the model at the last epoch.}
	\begin{tabular*}{15cm}{ @{\extracolsep{\fill}} c c c c c c c @{} }
		\toprule
		Batch Size & \multicolumn{2}{c}{Validation error} & \multicolumn{2}{c}{$\L2$} &\multicolumn{2}{c}{Epoch}\\ [.5ex]
		& \(\hy\)&\(\rare\)&\(\hy\)&\(\rare\)&\(\hy\)&\(\rare\)\\
		\hline
		32& \num{5.40e-8} & \num{2.17e-8} & \num{0.0024}  & \num{0.0017}&4998&4992\\ \hline
		16& \num{1.95e-8} & \num{2.06e-8} & \num{0.0015}  & \num{0.0016}&4999&5000\\ \hline
		8 & \num{2.25e-8} & \num{1.03e-8} & \num{0.0017}  & \num{0.0012}&4965&4961\\ \hline
		4 & \num{1.52e-8} & \num{6.30e-9} & \num{0.0013}  & \num{0.0010}&3956&4534\\ \hline
		2 & \num{1.15e-8} & \num{9.18e-9} & \num{0.0012}  & \num{0.0013}&4956&4872\\ \hline
	\end{tabular*}\label{Tab:Batch}
\end{table}
Next the mini-batch size is analysed. Epochs are increased by 1000 epochs, as training- and validation error show potential to decrease further as seen in \cref{Fig:Width} for \(\rare\) with 40 hidden nodes. Results for $\hy$  and  $\rare$ are displayed in \cref{Tab:Batch}. Experiments are conducted with mini-batch sizes of 2, 4, 8, 16, 32. Batch sizes to the power of 2 are typically chosen as to fully exploit computational capabilities of a GPU i.e. aligning the batch size with the way memory is structured within a GPU. The smallest batch size of 2 yields the lowest validation error of \(\num{1.15e-8}\) with corresponding \(\L2 = 0.0012\) at epoch 4956 for \(\hy\). The lowest validation error with \(\num{6.30e-9}\) is achieved for \(\rare\) at epoch 4534 with a batch size of 4. The corresponding \(\L2 = 0.001\). Compared to a batch size of 16 in the previous experiments we can observe that small batch sizes have a regularizing effect on training as described in \cref{Ch:DimRedAl} and therefore are beneficial to generalization. At the same time, the lower the batch sizes are, the more unstable is the training as seen in \cref{Fig:Batch}. The oscillations which begin with batch sizes of 8 and lower, which make the training unstable, can be battled with a lower learning rate as soon as training starts to tremble. Additionally small batch sizes drastically increase training time which is why a batch size as low as 2 will not be used for the next experiments. In conclusion a batch size of 2 is omitted and for both input data a batch size of 4 is chosen. Furthermore a reduction of the learning rate from \(\num{1e-4}\) to \(\num{1e-5}\) is applied after the 3000th epoch.\\
Eight experiments with different activation functions namely ReLU, ELU, Tanh, SiLU and LeakyReLU are performed. The experiment designs and results are given in \cref{Tab:activations} for hidden and code activations. With \(\hy\) a combination of ELU and ELU for hidden and code activation yields the best results in validation error with \num{4.44e-9} and a corresponding \(\L2 = 0.0008\). These values are achieved at the last epoch. For \(\rare\) a combination of ReLU and ReLU for hidden and code activation produces a validation error of \num{7.18e-9} and a corresponding \(\L2 = 0.0009\). Both values are also reached close to the last epoch. Note that all models reach their lowest loss at or very close to the last epoch. The reason is the stable training after the 3000th epoch, where the learning rate is lowered to \num{1e-5} as seen in \cref{Fig:Activations}. This measure shows in all experiments an immediate success for learning. Both validation and training error fall at the 3001st epoch and only decrease slightly thereafter. This behavior clearly shows that the updates to the free parameters \(\frepar\) were too big which prohibitively slowed down or even prevented the learning process. Small updates to \(\frepar\) made all models quickly reach a minimum. Therefore 3000 epochs or even less could have been enough to reach similar results with a lower learning rate.\\
If the learning rate could have been reduced whilst producing similar results remains unanswered here as the results are satisfactory. Note that for \(\rare\) in the previous experiment a validation error of \num{6.30e-9} was achieved which is slightly lower than the current result. Nonetheless it is decided to take the current model as the final result.\\
\begin{table}[H]
	\centering
	\caption{Results for the variation of activations for hidden-/code layers. Given is the minimum value of validation error as well as the the corresponding epoch and the \(\L2\). The \(\L2\) is evaluated with the models saved when the minimum validation error was achieved during training.}
	\begin{tabular*}{15.5cm}{ @{\extracolsep{\fill}} c c c c c c c @{} }
		\toprule
		Activations hidden/code & \multicolumn{2}{c}{Validation error} & \multicolumn{2}{c}{$\L2$} &\multicolumn{2}{c}{Epoch}\\ [.5ex]
		& \(\hy\)&\(\rare\)&\(\hy\)&\(\rare\)&\(\hy\)&\(\rare\)\\
		\hline
		ReLU/ReLU 	       & \num{9.79e-9} & \num{7.18e-9} & \num{0.0010}  & \num{0.0009}&5000 &4998\\ \hline
		ELU/ELU            & \num{4.44e-9} & \num{1.11e-8} & \num{0.0008}  & \num{0.0012}&5000 &5000\\ \hline
		Tanh/Tanh 	       & \num{7.83e-9} & \num{2.58e-8} & \num{0.0011}  & \num{0.0018}&5000 &5000\\ \hline
		SiLU/SiLU 	       & \num{7.69e-9} & \num{1.37e-8} & \num{0.0011}  & \num{0.0013}&5000 &5000\\ \hline
		LeakyReLU/LeakyReLU& \num{1.86e-8} & \num{9.39e-9} & \num{0.0015}  & \num{0.0010}&5000 &4997\\ \hline
		ELU/Tanh           & \num{5.49e-9} & \num{1.87e-8} & \num{0.0008}  & \num{0.0014}&5000 &5000\\ \hline
		LeakyReLU/Tanh     & \num{1.00e-8} & \num{1.42e-8} & \num{0.0010}  & \num{0.0012}&4997 &4992\\ \hline
		ELU/SiLU           & \num{8.11e-9} & \num{1.93e-8} & \num{0.0011}  & \num{0.0015}&5000 &5000\\ \hline
	\end{tabular*}\label{Tab:activations}
\end{table} 
The final hyperparameters for both input data are summarized below in \cref{Tab:Final}. From the initial models to the final models the decrease in validation error gained \(\approx \num{1.5e-7}\) for \(\hy\) and \(\approx \num{7.2e-8}\) for \(\rare\) which is 93\% of the initial values for both models.
\begin{table}[H]
	\centering
	\caption{Summary of the final hyperparameters for both input data.}
	\begin{tabular*}{16cm}{ @{\extracolsep{\fill}} c c c c c c c @{} }
		\toprule
		Input data & Act. hidden/code & Batch size & Width & Depth & Learning rate & Epochs\\ [.5ex]
		\hline
		$\hy$ &  ELU/ELU & 4 & 30 & 4 & \num{e-4}/\num{e-5} & $\approx 3000$\\ \hline
		$\rare$ & ReLU/ReLU & 4 & 40 & 4 & \num{e-4}/\num{e-5} & $\approx 3000$\\ \hline
	\end{tabular*}\label{Tab:Final}
\end{table}
\begin{center}
	\begin{figure}[htbp!]
		\input{Figures/Parameterstudy/Fully_Connected/Depth/hydro_depth.tex}
		% This file was created by tikzplotlib v0.9.6.
\begin{tikzpicture}

\begin{groupplot}[group style={group size=1 by 5},
legend cell align={left},
legend style={fill opacity=1, draw opacity=1, text opacity=1, draw=white},
log basis y={10},
tick align=outside,
tick pos=left,
title style={at={(0.45,0.85)},anchor=north},
x grid style={white!69.0196078431373!black},
xlabel={Epoch},
x label style={yshift=13pt},
xmin=-99.95, xmax=2098.95,
xtick style={color=black},
xtick = {0,500,1500,2000},
y grid style={white!69.0196078431373!black},
ylabel={MSE Loss},
ymode=log,
ytick style={color=black},
width=0.45\textwidth,
height=0.25\textwidth
]
\nextgroupplot[
title={10 Layers $\rare$},
xmin=-99.95, xmax=2098.95,
]
\addplot [semithick, black, dashed]
table {%
0 0.00997787586692721
1 0.0027661414751783
2 0.00218623768771067
3 0.00185343958390877
4 0.00115804947796278
5 0.00036777743147104
6 0.00018824286320887
7 0.000164930714039656
8 0.000153345151928079
9 0.000141543177305721
10 0.000127938025187177
11 0.000112482768548944
12 9.57521400996484e-05
13 7.92569487603032e-05
14 6.42920351929206e-05
15 5.20401717658388e-05
16 4.29740166491683e-05
17 3.62663831710961e-05
18 3.07352183917828e-05
19 2.49385155948403e-05
20 2.06156149879462e-05
21 1.71943852956247e-05
22 1.4385688823495e-05
23 1.20649838572717e-05
24 1.02233446727951e-05
25 8.78601411159252e-06
26 7.68138640933103e-06
27 6.83470541434872e-06
28 6.17248770799961e-06
29 5.6463688954409e-06
30 5.22004835283951e-06
31 4.84264568081016e-06
32 4.50758502950066e-06
33 4.20824098557659e-06
34 3.93843728591037e-06
35 3.69604697584691e-06
36 3.4781861807005e-06
37 3.27953066255304e-06
38 3.102444766796e-06
39 2.94624955324707e-06
40 2.80618139169064e-06
41 2.6809971720354e-06
42 2.57034235244191e-06
43 2.47136843603357e-06
44 2.38247599895658e-06
45 2.3052505364376e-06
46 2.2339168820622e-06
47 2.16824932778081e-06
48 2.10921677893339e-06
49 2.05554289499332e-06
50 2.00520992609654e-06
51 1.96032569562021e-06
52 1.91601798240981e-06
53 1.87716817765704e-06
54 1.84075475442569e-06
55 1.8041044749566e-06
56 1.77211888149031e-06
57 1.74144628215345e-06
58 1.71423015325445e-06
59 1.68766705314738e-06
60 1.66201588962167e-06
61 1.63708562206466e-06
62 1.61347263502876e-06
63 1.5917540299597e-06
64 1.57049718154667e-06
65 1.55085114516851e-06
66 1.53163219016506e-06
67 1.51338286417513e-06
68 1.49671406592233e-06
69 1.47951891813136e-06
70 1.4645103722728e-06
71 1.44944799347968e-06
72 1.43420475933453e-06
73 1.41955645852931e-06
74 1.40517172491172e-06
75 1.39141421908562e-06
76 1.37808219858471e-06
77 1.3657449522384e-06
78 1.3527578432786e-06
79 1.34140151624251e-06
80 1.32919058387415e-06
81 1.31819503866382e-06
82 1.30771289047971e-06
83 1.29784039592096e-06
84 1.28814695580104e-06
85 1.27880892563326e-06
86 1.27049044516525e-06
87 1.26170827212491e-06
88 1.25354349086138e-06
89 1.24524294693629e-06
90 1.23696791416705e-06
91 1.22909348488065e-06
92 1.22303059853834e-06
93 1.21662571478964e-06
94 1.20954619882241e-06
95 1.20269462843225e-06
96 1.19554730241589e-06
97 1.18936523608681e-06
98 1.18438249876363e-06
99 1.17871571490014e-06
100 1.17326355984915e-06
101 1.16720301622308e-06
102 1.16031671996097e-06
103 1.15646743591924e-06
104 1.15072228288682e-06
105 1.1456532542411e-06
106 1.14082713997732e-06
107 1.13676369537075e-06
108 1.13183904920788e-06
109 1.12775252654274e-06
110 1.12097592221971e-06
111 1.11884615537861e-06
112 1.11495200479794e-06
113 1.11073350669244e-06
114 1.10769001994981e-06
115 1.10351587494506e-06
116 1.10006028367593e-06
117 1.09647195202456e-06
118 1.09295321027503e-06
119 1.08959795028341e-06
120 1.08633151981508e-06
121 1.08298288154174e-06
122 1.0800084913285e-06
123 1.07806380844977e-06
124 1.074994691038e-06
125 1.07166258550251e-06
126 1.06880648237961e-06
127 1.06578414178671e-06
128 1.06304140217617e-06
129 1.06044994532795e-06
130 1.05643772593567e-06
131 1.0558744672835e-06
132 1.05326026272223e-06
133 1.04935598690759e-06
134 1.04676802968129e-06
135 1.04461839976011e-06
136 1.04219356757085e-06
137 1.03843377786461e-06
138 1.03679534922208e-06
139 1.03479115048799e-06
140 1.03239273369127e-06
141 1.02880143617767e-06
142 1.02687638050725e-06
143 1.02502206539157e-06
144 1.02148192920026e-06
145 1.02024824695945e-06
146 1.01793011802442e-06
147 1.01475467491241e-06
148 1.01393682172102e-06
149 1.01059575447948e-06
150 1.00912902482264e-06
151 1.0062060501923e-06
152 1.00506513257415e-06
153 1.00228589985818e-06
154 1.00133910484601e-06
155 9.99022535552285e-07
156 9.96578049466734e-07
157 9.95774760042423e-07
158 9.92670811598373e-07
159 9.90552980908888e-07
160 9.90131076179068e-07
161 9.8754623903119e-07
162 9.8556907477132e-07
163 9.85354359244184e-07
164 9.82352091568828e-07
165 9.80439722610527e-07
166 9.78883942138964e-07
167 9.78459102157103e-07
168 9.75541983876837e-07
169 9.74201690695509e-07
170 9.73315617585513e-07
171 9.7158668955899e-07
172 9.71484288783131e-07
173 9.67737080088682e-07
174 9.66393180931391e-07
175 9.64838651981381e-07
176 9.64215281697989e-07
177 9.62576921267555e-07
178 9.63217306946262e-07
179 9.60148133856364e-07
180 9.59013245392271e-07
181 9.56711457007486e-07
182 9.55200339518569e-07
183 9.53850673880652e-07
184 9.52729099026328e-07
185 9.51356085352018e-07
186 9.49982874573152e-07
187 9.48758935265914e-07
188 9.4734960742926e-07
189 9.46268320717536e-07
190 9.45888197207978e-07
191 9.43202051558956e-07
192 9.42285718593894e-07
193 9.4148136670924e-07
194 9.39797778954699e-07
195 9.38851433886612e-07
196 9.36940359082428e-07
197 9.35960853297502e-07
198 9.35100780367293e-07
199 9.33036782470253e-07
200 9.32759092336255e-07
201 9.309670438995e-07
202 9.30231287497918e-07
203 9.28388456145512e-07
204 9.28534422172334e-07
205 9.26417262860468e-07
206 9.25963449361689e-07
207 9.23892555078964e-07
208 9.23097257839345e-07
209 9.21156167748904e-07
210 9.20179683930655e-07
211 9.18629479031097e-07
212 9.17580147955732e-07
213 9.16587913565081e-07
214 9.15249802602602e-07
215 9.14260038939574e-07
216 9.13397146405259e-07
217 9.12211345678315e-07
218 9.10792389674953e-07
219 9.10044973522872e-07
220 9.09023786704211e-07
221 9.08015798756878e-07
222 9.07080938276295e-07
223 9.05977511166611e-07
224 9.05009585721928e-07
225 9.03722100105142e-07
226 9.0283103281763e-07
227 9.01766764314971e-07
228 9.00620563015764e-07
229 8.99610003187945e-07
230 8.98586724986217e-07
231 8.97490580570093e-07
232 8.96478950949131e-07
233 8.95298846074866e-07
234 8.94353904413947e-07
235 8.93204120927749e-07
236 8.923171830304e-07
237 8.90687736870177e-07
238 8.89868796548399e-07
239 8.88946101269994e-07
240 8.87893214382984e-07
241 8.86297618535537e-07
242 8.85070359657902e-07
243 8.84417957706773e-07
244 8.83675672440631e-07
245 8.8251698130648e-07
246 8.81460606791507e-07
247 8.8027023426207e-07
248 8.79355826469919e-07
249 8.78572113578002e-07
250 8.77525471793206e-07
251 8.76351880549464e-07
252 8.75500791465811e-07
253 8.74678740700574e-07
254 8.73476796158457e-07
255 8.72610606649005e-07
256 8.71706194686794e-07
257 8.70708743093473e-07
258 8.70191305182289e-07
259 8.69344236235747e-07
260 8.67789354629167e-07
261 8.66628777828282e-07
262 8.65958915056808e-07
263 8.64537149453781e-07
264 8.63818404411631e-07
265 8.62849739490912e-07
266 8.61764416811184e-07
267 8.60982841373925e-07
268 8.60051689585362e-07
269 8.59030311005427e-07
270 8.58128868173935e-07
271 8.57145231947243e-07
272 8.56545518360008e-07
273 8.55089079635718e-07
274 8.54582956577588e-07
275 8.53439548222923e-07
276 8.52625509651261e-07
277 8.52064497024685e-07
278 8.50905582069572e-07
279 8.50112307404061e-07
280 8.49598813260855e-07
281 8.4875707469223e-07
282 8.4775906213963e-07
283 8.4675666619205e-07
284 8.46309231747e-07
285 8.45124665829644e-07
286 8.44529056251986e-07
287 8.43495970570984e-07
288 8.42961660282526e-07
289 8.41671420289458e-07
290 8.41705411914972e-07
291 8.40604265988532e-07
292 8.39964006132732e-07
293 8.38750670340005e-07
294 8.3819956410025e-07
295 8.37723511324384e-07
296 8.36425500693849e-07
297 8.35761372655952e-07
298 8.35393352815572e-07
299 8.3392940771887e-07
300 8.3388055014666e-07
301 8.32478889208232e-07
302 8.32139470674065e-07
303 8.31678217167564e-07
304 8.30812586201546e-07
305 8.30091150533008e-07
306 8.29310000142414e-07
307 8.28749091709824e-07
308 8.2756600025391e-07
309 8.27514076803482e-07
310 8.26390768622787e-07
311 8.25961606182091e-07
312 8.25162235912558e-07
313 8.24495962177707e-07
314 8.24095709617723e-07
315 8.22966658319046e-07
316 8.21936873251161e-07
317 8.21767470569057e-07
318 8.20922504203736e-07
319 8.20247862861834e-07
320 8.19370113532614e-07
321 8.18635833951475e-07
322 8.17986851899377e-07
323 8.17315450689193e-07
324 8.1666383891843e-07
325 8.16427804124942e-07
326 8.15375180422961e-07
327 8.15192657825037e-07
328 8.14120937434382e-07
329 8.13174602825484e-07
330 8.1293584187847e-07
331 8.11936721930806e-07
332 8.11085818554602e-07
333 8.10528149429501e-07
334 8.09983656409941e-07
335 8.08851426143065e-07
336 8.08121500682546e-07
337 8.08130491662951e-07
338 8.07360241225297e-07
339 8.06781943225587e-07
340 8.06020081228098e-07
341 8.05330066299348e-07
342 8.04786609791108e-07
343 8.0464047090345e-07
344 8.03382348692594e-07
345 8.03091622429974e-07
346 8.03480803682533e-07
347 8.01579474654091e-07
348 8.0091967774365e-07
349 8.01363648946563e-07
350 7.99436845767332e-07
351 7.99345849031852e-07
352 7.98882536855672e-07
353 7.97898566048616e-07
354 7.98836069549225e-07
355 7.98981313806735e-07
356 7.97069268514861e-07
357 7.96280659386639e-07
358 7.96522883803164e-07
359 7.94783968302681e-07
360 7.94133767442418e-07
361 7.94591968116265e-07
362 7.92818375458637e-07
363 7.92365432204178e-07
364 7.91729609915137e-07
365 7.92124033381469e-07
366 7.9021502497767e-07
367 7.89800760259141e-07
368 7.89006110096579e-07
369 7.89422273101081e-07
370 7.87633119500697e-07
371 7.87470235025012e-07
372 7.86781127885661e-07
373 7.87345359412939e-07
374 7.85442462444053e-07
375 7.85028574483704e-07
376 7.84489331238092e-07
377 7.83982934024152e-07
378 7.84435161477859e-07
379 7.82677077467042e-07
380 7.8204591821418e-07
381 7.81801198940002e-07
382 7.81111819321723e-07
383 7.81425274823278e-07
384 7.7998786557032e-07
385 7.79324741330356e-07
386 7.78943699941692e-07
387 7.78382946805323e-07
388 7.7774481707138e-07
389 7.78080912994028e-07
390 7.76618715718769e-07
391 7.76134723423638e-07
392 7.75627843495386e-07
393 7.74918021193116e-07
394 7.74431415038634e-07
395 7.74660446609232e-07
396 7.73305947575409e-07
397 7.72896880562257e-07
398 7.72197305280997e-07
399 7.71702392796669e-07
400 7.71071945763424e-07
401 7.70523813088175e-07
402 7.70097182538621e-07
403 7.7021479600603e-07
404 7.68894578584423e-07
405 7.68398064082021e-07
406 7.67811163484566e-07
407 7.67315257036216e-07
408 7.66777737169377e-07
409 7.66156092197434e-07
410 7.65690696312049e-07
411 7.65130233787659e-07
412 7.6555170792858e-07
413 7.64065328610286e-07
414 7.63505121966546e-07
415 7.63133862619725e-07
416 7.62454071889351e-07
417 7.62014022313906e-07
418 7.6143167035525e-07
419 7.60892476279196e-07
420 7.60415168230111e-07
421 7.59914973372133e-07
422 7.5941622890241e-07
423 7.60389407474804e-07
424 7.58777384561427e-07
425 7.58338006789927e-07
426 7.57724335500143e-07
427 7.57657803433176e-07
428 7.56923721269232e-07
429 7.56353567965107e-07
430 7.56923098862217e-07
431 7.55219046823186e-07
432 7.54618819826192e-07
433 7.54311231389693e-07
434 7.53780425611694e-07
435 7.54422887609962e-07
436 7.52576872230293e-07
437 7.52243586475743e-07
438 7.51672492100397e-07
439 7.52297821776438e-07
440 7.50493718726375e-07
441 7.50091651326557e-07
442 7.49571382499425e-07
443 7.5027601661759e-07
444 7.48633131451015e-07
445 7.48203038369866e-07
446 7.47641490107753e-07
447 7.4706847775019e-07
448 7.47778209245098e-07
449 7.45882051290891e-07
450 7.45447632596097e-07
451 7.45004974220365e-07
452 7.45590436821431e-07
453 7.44075579518722e-07
454 7.43488029229411e-07
455 7.4302814456928e-07
456 7.43562355694394e-07
457 7.42126745706173e-07
458 7.41440271383453e-07
459 7.41023272865959e-07
460 7.42221455794834e-07
461 7.39916425203546e-07
462 7.39315465096979e-07
463 7.38942567977574e-07
464 7.39466763462815e-07
465 7.38114260798284e-07
466 7.37723011752678e-07
467 7.38328398483645e-07
468 7.36677821834064e-07
469 7.36189456091552e-07
470 7.36971248016971e-07
471 7.35291715386666e-07
472 7.34859717340441e-07
473 7.35544233293695e-07
474 7.33993297529878e-07
475 7.33437984763441e-07
476 7.34147434599208e-07
477 7.32526463366412e-07
478 7.32113129146228e-07
479 7.32783056690778e-07
480 7.31102353171309e-07
481 7.30759978779361e-07
482 7.30174026074337e-07
483 7.3101718700741e-07
484 7.29139022212166e-07
485 7.28747309921118e-07
486 7.29511678514427e-07
487 7.27845423909912e-07
488 7.27509910404933e-07
489 7.27898909303804e-07
490 7.25985288198672e-07
491 7.26988534040629e-07
492 7.26433349939271e-07
493 7.24913795409066e-07
494 7.25639360695141e-07
495 7.25110006783325e-07
496 7.23576190097219e-07
497 7.24332362437963e-07
498 7.23792091548603e-07
499 7.22295086035274e-07
500 7.23148807196594e-07
501 7.22562276394001e-07
502 7.21028598093199e-07
503 7.21920047169533e-07
504 7.20162841702177e-07
505 7.21097849861962e-07
506 7.20601577768321e-07
507 7.18964301626102e-07
508 7.19790344447802e-07
509 7.19423874585345e-07
510 7.17816719657094e-07
511 7.18529795136646e-07
512 7.17035474110617e-07
513 7.17842366498189e-07
514 7.173693042688e-07
515 7.15891436129823e-07
516 7.16593398948362e-07
517 7.16193349916239e-07
518 7.15836237418444e-07
519 7.13896747583931e-07
520 7.14747440071051e-07
521 7.14939811359727e-07
522 7.14682133747146e-07
523 7.14109896080117e-07
524 7.12665502788923e-07
525 7.13412247876022e-07
526 7.12840395792114e-07
527 7.1258078847336e-07
528 7.1094341305411e-07
529 7.11800691419739e-07
530 7.11041270108126e-07
531 7.11053048263466e-07
532 7.09551945874409e-07
533 7.10300355095228e-07
534 7.10044622053374e-07
535 7.08126653066188e-07
536 7.09192354179322e-07
537 7.08688509973854e-07
538 7.08473750933081e-07
539 7.0703383852333e-07
540 7.07924062425036e-07
541 7.07072048214741e-07
542 7.05930014561318e-07
543 7.06833243867777e-07
544 7.06151251620213e-07
545 7.05789116310029e-07
546 7.05411095324848e-07
547 7.03952515351602e-07
548 7.04824918869917e-07
549 7.04396891237025e-07
550 7.04121892809439e-07
551 7.03592790472385e-07
552 7.02295799982267e-07
553 7.03159633289374e-07
554 7.02392406822128e-07
555 7.02389151811644e-07
556 7.02069595888588e-07
557 7.00544815430249e-07
558 7.01544812031329e-07
559 7.01071055104308e-07
560 7.00843794945172e-07
561 7.00298507183561e-07
562 6.99028340818586e-07
563 6.9988211890859e-07
564 6.99153588470836e-07
565 6.99094921657206e-07
566 6.98874052901033e-07
567 6.97493815380312e-07
568 6.9732204678985e-07
569 6.96886004035946e-07
570 6.96496134878544e-07
571 6.96340423658626e-07
572 6.9577226418005e-07
573 6.95301015184668e-07
574 6.94970092226299e-07
575 6.94665715130327e-07
576 6.94327597415167e-07
577 6.9397252194392e-07
578 6.93629053131417e-07
579 6.93304834925357e-07
580 6.92961221275823e-07
581 6.92612403483395e-07
582 6.92295116081709e-07
583 6.92008054087978e-07
584 6.91836038868132e-07
585 6.91154041760456e-07
586 6.91039159349316e-07
587 6.90780433998839e-07
588 6.90375913677599e-07
589 6.90120593674237e-07
590 6.89844487624214e-07
591 6.89540247705622e-07
592 6.8927526524476e-07
593 6.88991556458518e-07
594 6.88708317511555e-07
595 6.88413684557077e-07
596 6.88122735454044e-07
597 6.87847310999246e-07
598 6.87655565940304e-07
599 6.873611031466e-07
600 6.8711852216552e-07
601 6.86813203969905e-07
602 6.86521028598008e-07
603 6.86258257005079e-07
604 6.86011237931439e-07
605 6.85736956057781e-07
606 6.8547543475006e-07
607 6.85227141644873e-07
608 6.84974181766052e-07
609 6.84731802891747e-07
610 6.84444454293498e-07
611 6.84201174820487e-07
612 6.83949088440272e-07
613 6.83735282066777e-07
614 6.83597051065021e-07
615 6.83230357168441e-07
616 6.82977046281508e-07
617 6.82744479377107e-07
618 6.82628084632597e-07
619 6.82269299900895e-07
620 6.82057781943968e-07
621 6.81825706507766e-07
622 6.81707034459578e-07
623 6.81389691763457e-07
624 6.81201359242323e-07
625 6.81054129145764e-07
626 6.8075853309324e-07
627 6.80542225936165e-07
628 6.80400173507678e-07
629 6.80067346735314e-07
630 6.79983958491448e-07
631 6.79618614057631e-07
632 6.79393168169895e-07
633 6.79165578233665e-07
634 6.79077912963066e-07
635 6.78772824926455e-07
636 6.78559704212489e-07
637 6.78351354096662e-07
638 6.78243983884386e-07
639 6.77901727698327e-07
640 6.77708650769659e-07
641 6.77525354845443e-07
642 6.77312835492216e-07
643 6.77094021853009e-07
644 6.7693512350786e-07
645 6.76660766927739e-07
646 6.76465998878939e-07
647 6.7620693066317e-07
648 6.76058081367614e-07
649 6.75807865121669e-07
650 6.75675274962373e-07
651 6.75309103939981e-07
652 6.75240928430298e-07
653 6.75009894337109e-07
654 6.74884714669588e-07
655 6.7469262417319e-07
656 6.74492794075832e-07
657 6.74472121843905e-07
658 6.74072660487468e-07
659 6.74126479594861e-07
660 6.73871772562507e-07
661 6.73676320815275e-07
662 6.73381162670239e-07
663 6.73315533632035e-07
664 6.7304010840985e-07
665 6.7279634778572e-07
666 6.72624413653011e-07
667 6.72622655784494e-07
668 6.72328725286775e-07
669 6.72171628352203e-07
670 6.72120275240218e-07
671 6.72108327890442e-07
672 6.71769079815476e-07
673 6.71592175606861e-07
674 6.7153022288835e-07
675 6.71228376674549e-07
676 6.71063214738865e-07
677 6.70862587298871e-07
678 6.70727474243904e-07
679 6.70538763273498e-07
680 6.70397458648608e-07
681 6.70331136205959e-07
682 6.70071903499547e-07
683 6.69920863757056e-07
684 6.69779470129583e-07
685 6.69620603034105e-07
686 6.69458129848977e-07
687 6.6930754940131e-07
688 6.69167233667167e-07
689 6.68956220224004e-07
690 6.68770538680974e-07
691 6.68667876396967e-07
692 6.68520957177066e-07
693 6.68362044635273e-07
694 6.68231087544768e-07
695 6.68063874257996e-07
696 6.67905007389891e-07
697 6.67740694836993e-07
698 6.67615698802138e-07
699 6.67491615800486e-07
700 6.67331059744924e-07
701 6.67199185485856e-07
702 6.67054764520003e-07
703 6.66892555187815e-07
704 6.66757220614045e-07
705 6.6661704566684e-07
706 6.66472319977629e-07
707 6.66348736558575e-07
708 6.66208892795339e-07
709 6.66104842267146e-07
710 6.65947033539283e-07
711 6.6583568202816e-07
712 6.65676333156284e-07
713 6.65566532433104e-07
714 6.65371252381419e-07
715 6.65273846124137e-07
716 6.65145204777673e-07
717 6.6501357501636e-07
718 6.64885187660502e-07
719 6.64747493914319e-07
720 6.64656038921407e-07
721 6.64518116366253e-07
722 6.64384310013588e-07
723 6.6426489875937e-07
724 6.64103254891302e-07
725 6.6400623869356e-07
726 6.6387511243704e-07
727 6.63760887292142e-07
728 6.63594443778948e-07
729 6.6352547675308e-07
730 6.63392976136379e-07
731 6.63245974934057e-07
732 6.63125168372858e-07
733 6.63018874917043e-07
734 6.62882144197852e-07
735 6.62734271159593e-07
736 6.62517572180832e-07
737 6.6241604606887e-07
738 6.62473698085364e-07
739 6.62516929139656e-07
740 6.62498065295836e-07
741 6.61815979839275e-07
742 6.61530765412976e-07
743 6.61442542394752e-07
744 6.61331088323891e-07
745 6.61209408832519e-07
746 6.6109859224639e-07
747 6.6099788502072e-07
748 6.60889263187414e-07
749 6.60740850307207e-07
750 6.60631029802516e-07
751 6.60538422906143e-07
752 6.60393012481109e-07
753 6.60271693462278e-07
754 6.60180221544238e-07
755 6.60043354130835e-07
756 6.59950334664927e-07
757 6.59887545353399e-07
758 6.5971084376315e-07
759 6.59582626255428e-07
760 6.59522363278597e-07
761 6.59398176679815e-07
762 6.59279328061757e-07
763 6.59160926986146e-07
764 6.59018419938207e-07
765 6.58968605165455e-07
766 6.58780548107529e-07
767 6.58640463925053e-07
768 6.58545901032426e-07
769 6.58429381942938e-07
770 6.58341001013696e-07
771 6.58279398663808e-07
772 6.58103613247363e-07
773 6.58032226553473e-07
774 6.5792477641935e-07
775 6.57836251804156e-07
776 6.57792503417909e-07
777 6.57651318391572e-07
778 6.57547617464616e-07
779 6.57399611057485e-07
780 6.57286304786453e-07
781 6.57153251651721e-07
782 6.5710807548669e-07
783 6.56900221926549e-07
784 6.56799596583824e-07
785 6.56696743305929e-07
786 6.56575473740872e-07
787 6.56487851685483e-07
788 6.56370081244972e-07
789 6.56292135957415e-07
790 6.56130607339378e-07
791 6.56064941154e-07
792 6.56011381011012e-07
793 6.55829648934514e-07
794 6.55786200027819e-07
795 6.55607685331461e-07
796 6.55609785056299e-07
797 6.55533220481175e-07
798 6.55394131726439e-07
799 6.55311715917151e-07
800 6.55156520764422e-07
801 6.55102531055718e-07
802 6.55082887604408e-07
803 6.54888119228758e-07
804 6.54802214071992e-07
805 6.5473559047291e-07
806 6.54574609910696e-07
807 6.54500922465218e-07
808 6.54382220616867e-07
809 6.54305573817737e-07
810 6.5420019161877e-07
811 6.54057651630069e-07
812 6.53967765671837e-07
813 6.53819775891407e-07
814 6.53770412583299e-07
815 6.53644654022401e-07
816 6.53591989035362e-07
817 6.53487234302474e-07
818 6.53435297138572e-07
819 6.53194102852694e-07
820 6.53198145187162e-07
821 6.5311538847368e-07
822 6.52892069282984e-07
823 6.52899213022806e-07
824 6.52815578064292e-07
825 6.52532915339066e-07
826 6.5250298719377e-07
827 6.52467241110344e-07
828 6.52352714354265e-07
829 6.52214292244935e-07
830 6.52175464963989e-07
831 6.52048443271269e-07
832 6.51983712870674e-07
833 6.51903975750656e-07
834 6.51790568511501e-07
835 6.51704363789918e-07
836 6.51591102098337e-07
837 6.51535222814914e-07
838 6.5140828783683e-07
839 6.51308421680596e-07
840 6.51213768250614e-07
841 6.51116864844425e-07
842 6.50987768693767e-07
843 6.50862097572258e-07
844 6.50764789384084e-07
845 6.50686730963912e-07
846 6.50605446963937e-07
847 6.50517625658154e-07
848 6.49935367121657e-07
849 6.49951558187922e-07
850 6.49781994908949e-07
851 6.49812969228947e-07
852 6.4962381597411e-07
853 6.49715427286424e-07
854 6.49609696424136e-07
855 6.49503299598564e-07
856 6.49437240753059e-07
857 6.4924598638072e-07
858 6.49154591329193e-07
859 6.49055806263732e-07
860 6.48968337969791e-07
861 6.48860881113933e-07
862 6.48779707205449e-07
863 6.4869935371803e-07
864 6.48607655406863e-07
865 6.48494430663504e-07
866 6.48387495814973e-07
867 6.48355465315831e-07
868 6.4823302085415e-07
869 6.48177903727287e-07
870 6.48068991381479e-07
871 6.48002301730344e-07
872 6.47880534643264e-07
873 6.4779980566243e-07
874 6.47472709502495e-07
875 6.47573727789563e-07
876 6.47521113151583e-07
877 6.47460983415726e-07
878 6.47327359146743e-07
879 6.47256306294253e-07
880 6.47165816758388e-07
881 6.47062579616886e-07
882 6.46873130364156e-07
883 6.46932409637202e-07
884 6.4678064497059e-07
885 6.46794917827265e-07
886 6.4658947853502e-07
887 6.46604447453569e-07
888 6.46413081099695e-07
889 6.46435930761413e-07
890 6.46238205860072e-07
891 6.46160109269545e-07
892 6.46079087459839e-07
893 6.45998969318384e-07
894 6.45898591372429e-07
895 6.45816730695969e-07
896 6.45777476066201e-07
897 6.45704528096758e-07
898 6.45561784040183e-07
899 6.45402865401934e-07
900 6.45395685438643e-07
901 6.45250818593013e-07
902 6.45247586589903e-07
903 6.45016569762902e-07
904 6.45096973528325e-07
905 6.44992183836734e-07
906 6.44879979844859e-07
907 6.44814044164832e-07
908 6.44536055304457e-07
909 6.44651230615523e-07
910 6.44630252224943e-07
911 6.44534238688266e-07
912 6.44403965878837e-07
913 6.44269876133308e-07
914 6.44109464232656e-07
915 6.44056983475139e-07
916 6.44181476729955e-07
917 6.44011751091966e-07
918 6.43851985074662e-07
919 6.43799305493076e-07
920 6.43708418380129e-07
921 6.43616032121486e-07
922 6.43390352038864e-07
923 6.43459086845155e-07
924 6.43066392427727e-07
925 6.43326589468529e-07
926 6.43231603277172e-07
927 6.42995318642647e-07
928 6.43065223883355e-07
929 6.42968633215446e-07
930 6.42634859744362e-07
931 6.42847502362542e-07
932 6.42573602135599e-07
933 6.42576249461513e-07
934 6.42420869482407e-07
935 6.42509780959699e-07
936 6.42260338906908e-07
937 6.42345372426689e-07
938 6.4186560094015e-07
939 6.42131707621729e-07
940 6.41876858665569e-07
941 6.41937691256089e-07
942 6.41715006921117e-07
943 6.41790332380765e-07
944 6.41556473468086e-07
945 6.415050187627e-07
946 6.41448093148256e-07
947 6.41452986968716e-07
948 6.41207829275459e-07
949 6.41160052467171e-07
950 6.41136324560421e-07
951 6.41156462506842e-07
952 6.4103641173574e-07
953 6.40879574191899e-07
954 6.40985946901651e-07
955 6.40758436617261e-07
956 6.40720444295084e-07
957 6.40529553727731e-07
958 6.40474377917144e-07
959 6.40498605775974e-07
960 6.40255618591823e-07
961 6.40319487331453e-07
962 6.40329926909544e-07
963 6.40118906979126e-07
964 6.40069182985315e-07
965 6.39933080719857e-07
966 6.40022533211493e-07
967 6.39830953474529e-07
968 6.39713040214929e-07
969 6.39593101901426e-07
970 6.39774005406935e-07
971 6.39434521559679e-07
972 6.39282294187637e-07
973 6.39523240408835e-07
974 6.39073064718332e-07
975 6.39149466316269e-07
976 6.39264868198097e-07
977 6.38937482264623e-07
978 6.3891833339369e-07
979 6.3900999825961e-07
980 6.38726525863831e-07
981 6.38650845402822e-07
982 6.38661012288821e-07
983 6.38419105449373e-07
984 6.38549863495541e-07
985 6.38299137555975e-07
986 6.38424846982843e-07
987 6.38142738338843e-07
988 6.38220265770428e-07
989 6.38023431044132e-07
990 6.38141142715654e-07
991 6.37875260373733e-07
992 6.37939418027145e-07
993 6.37722096641369e-07
994 6.3776769428614e-07
995 6.37635073331921e-07
996 6.37635214005172e-07
997 6.37495254942166e-07
998 6.3745381473268e-07
999 6.37371561651889e-07
1000 6.36697234590144e-07
1001 6.36720888294917e-07
1002 6.36737104024121e-07
1003 6.36686584201129e-07
1004 6.36586772294834e-07
1005 6.36517155868432e-07
1006 6.36517348780785e-07
1007 6.36441783626651e-07
1008 6.3637201033373e-07
1009 6.36290345994439e-07
1010 6.36226407571883e-07
1011 6.36187908853003e-07
1012 6.36128972693939e-07
1013 6.36090735575578e-07
1014 6.3589234423489e-07
1015 6.35912154805851e-07
1016 6.35904042837865e-07
1017 6.35763122460276e-07
1018 6.35666576634719e-07
1019 6.3564752967693e-07
1020 6.35538853181572e-07
1021 6.35491918373532e-07
1022 6.35451334389359e-07
1023 6.35415595198197e-07
1024 6.35248154431167e-07
1025 6.35240329302178e-07
1026 6.35150410481344e-07
1027 6.35137976871647e-07
1028 6.34976308155899e-07
1029 6.35017855039166e-07
1030 6.3482607657761e-07
1031 6.34881822449529e-07
1032 6.34690630576529e-07
1033 6.34540902638037e-07
1034 6.34473359724552e-07
1035 6.34451190421714e-07
1036 6.34297449984445e-07
1037 6.34270674453319e-07
1038 6.34141945823785e-07
1039 6.34231479885727e-07
1040 6.34136626601389e-07
1041 6.34082943328451e-07
1042 6.3395456306381e-07
1043 6.33960346824836e-07
1044 6.33853544790952e-07
1045 6.33788128311608e-07
1046 6.33753139354098e-07
1047 6.33628680830611e-07
1048 6.33625172490326e-07
1049 6.33516750120577e-07
1050 6.33459435064765e-07
1051 6.33394688364319e-07
1052 6.33326460899752e-07
1053 6.33260135010971e-07
1054 6.33217650971574e-07
1055 6.33124233907267e-07
1056 6.33073544925367e-07
1057 6.32994399367703e-07
1058 6.32937160020219e-07
1059 6.32863034105924e-07
1060 6.32802857673198e-07
1061 6.327211272108e-07
1062 6.32664070593592e-07
1063 6.32609571134424e-07
1064 6.3261294997119e-07
1065 6.32557479264051e-07
1066 6.32472583347976e-07
1067 6.32388676500284e-07
1068 6.3235371713688e-07
1069 6.32262299873787e-07
1070 6.32186852847383e-07
1071 6.32157002080191e-07
1072 6.32051000330591e-07
1073 6.31951863908853e-07
1074 6.31979543818772e-07
1075 6.31833903980805e-07
1076 6.31841321954596e-07
1077 6.31720386728318e-07
1078 6.31670170619714e-07
1079 6.31639167558262e-07
1080 6.31551888616855e-07
1081 6.31490619177555e-07
1082 6.31423811270793e-07
1083 6.3135732366959e-07
1084 6.31311455016714e-07
1085 6.31243087482858e-07
1086 6.31197074852707e-07
1087 6.31128118996571e-07
1088 6.31093993149534e-07
1089 6.31002973804584e-07
1090 6.30930024748011e-07
1091 6.30873843036284e-07
1092 6.30818468749794e-07
1093 6.30774500507414e-07
1094 6.30719061497587e-07
1095 6.30654274829112e-07
1096 6.30573381890542e-07
1097 6.30514150955719e-07
1098 6.30455811105435e-07
1099 6.30417263671745e-07
1100 6.303295307859e-07
1101 6.30289127698802e-07
1102 6.30195487040908e-07
1103 6.30157178008517e-07
1104 6.30050639735202e-07
1105 6.29987743629101e-07
1106 6.29968577783302e-07
1107 6.29872791670039e-07
1108 6.29852591870872e-07
1109 6.29763205395761e-07
1110 6.2972433978814e-07
1111 6.29685811993852e-07
1112 6.29570583363659e-07
1113 6.29555979784868e-07
1114 6.29505421251508e-07
1115 6.29378778313594e-07
1116 6.29395147583978e-07
1117 6.29327122219081e-07
1118 6.29239093782985e-07
1119 6.29181523294164e-07
1120 6.29151311549947e-07
1121 6.29059220806027e-07
1122 6.29024048642179e-07
1123 6.28953817965794e-07
1124 6.28911244270114e-07
1125 6.28805891778938e-07
1126 6.28756066198832e-07
1127 6.28648164230583e-07
1128 6.28306397643996e-07
1129 6.28687460732635e-07
1130 6.2848032885654e-07
1131 6.28135447911404e-07
1132 6.28400187004274e-07
1133 6.28573363961493e-07
1134 6.28527363033982e-07
1135 6.28527280390756e-07
1136 6.28343305770329e-07
1137 6.28284698507287e-07
1138 6.2812800601364e-07
1139 6.28083045576488e-07
1140 6.28054853891058e-07
1141 6.28019043531936e-07
1142 6.27906697488356e-07
1143 6.27742714229385e-07
1144 6.27365253272671e-07
1145 6.27446661880526e-07
1146 6.27488955871058e-07
1147 6.27701524173574e-07
1148 6.27380187211202e-07
1149 6.27639201667307e-07
1150 6.2732651976205e-07
1151 6.27196206636427e-07
1152 6.27453849801896e-07
1153 6.27290091358645e-07
1154 6.27293938102014e-07
1155 6.27100247442058e-07
1156 6.26845826687372e-07
1157 6.27136329399036e-07
1158 6.26958845828085e-07
1159 6.27028462695023e-07
1160 6.26799966369163e-07
1161 6.26518928804387e-07
1162 6.26466177664042e-07
1163 6.26487830309941e-07
1164 6.26460136622597e-07
1165 6.26553522593554e-07
1166 6.2658400442217e-07
1167 6.26421702463631e-07
1168 6.26354288016273e-07
1169 6.26202877917592e-07
1170 6.26266790412444e-07
1171 6.26346333575611e-07
1172 6.26213740538617e-07
1173 6.2591202649287e-07
1174 6.26159682411753e-07
1175 6.259679466325e-07
1176 6.25821323929188e-07
1177 6.25983454000334e-07
1178 6.25770567594941e-07
1179 6.25582370801681e-07
1180 6.25654848079193e-07
1181 6.25532107612514e-07
1182 6.2545592751917e-07
1183 6.25413862280766e-07
1184 6.25349340801051e-07
1185 6.25261259557419e-07
1186 6.25203425371978e-07
1187 6.25166111447584e-07
1188 6.25094666915516e-07
1189 6.25081710083464e-07
1190 6.25006898232527e-07
1191 6.24934165074365e-07
1192 6.24913404614347e-07
1193 6.24911467220102e-07
1194 6.24548951755344e-07
1195 6.24694442166174e-07
1196 6.24773008453872e-07
1197 6.24621071231957e-07
1198 6.2450013277271e-07
1199 6.24621874457887e-07
1200 6.24454692619736e-07
1201 6.24370407393826e-07
1202 6.2447302522628e-07
1203 6.24262143524845e-07
1204 6.24315219837968e-07
1205 6.2430981112982e-07
1206 6.23796075679195e-07
1207 6.24177096320011e-07
1208 6.24049105802271e-07
1209 6.24054203349544e-07
1210 6.24027001990157e-07
1211 6.23865109787403e-07
1212 6.23932852477083e-07
1213 6.23779870181807e-07
1214 6.23787094262696e-07
1215 6.23741608634987e-07
1216 6.23713004550552e-07
1217 6.23544608970405e-07
1218 6.23572001856587e-07
1219 6.23602649540089e-07
1220 6.23225978060304e-07
1221 6.23518592469452e-07
1222 6.23039261583358e-07
1223 6.23288506076847e-07
1224 6.23076261497602e-07
1225 6.23334286977695e-07
1226 6.23120273957056e-07
1227 6.23057992385156e-07
1228 6.2310072766536e-07
1229 6.22905203400137e-07
1230 6.23071342417347e-07
1231 6.22551743603594e-07
1232 6.22838785815816e-07
1233 6.22911618577859e-07
1234 6.22436094097623e-07
1235 6.2257276290012e-07
1236 6.22675055979016e-07
1237 6.22591451126198e-07
1238 6.22482214389208e-07
1239 6.22608929845114e-07
1240 6.22082830332715e-07
1241 6.22355137487318e-07
1242 6.22401920892912e-07
1243 6.22228115553014e-07
1244 6.22168779464971e-07
1245 6.2226912954344e-07
1246 6.21835941025495e-07
1247 6.22107425449769e-07
1248 6.22095544343892e-07
1249 6.21943532365776e-07
1250 6.21878639350371e-07
1251 6.21777342473706e-07
1252 6.21807686485454e-07
1253 6.21633627901019e-07
1254 6.21770785983244e-07
1255 6.21682074758212e-07
1256 6.21532353186183e-07
1257 6.21157554036245e-07
1258 6.21374166421163e-07
1259 6.21383820664789e-07
1260 6.21448532641011e-07
1261 6.21237848690726e-07
1262 6.21411155641738e-07
1263 6.20860587424943e-07
1264 6.2109406439248e-07
1265 6.21135994300914e-07
1266 6.21238524701084e-07
1267 6.20719478952481e-07
1268 6.20863639028357e-07
1269 6.21026825456283e-07
1270 6.20972499071115e-07
1271 6.20547807017147e-07
1272 6.20815331657809e-07
1273 6.20883732374011e-07
1274 6.20434088780542e-07
1275 6.20376438746462e-07
1276 6.2064274709428e-07
1277 6.20666749313159e-07
1278 6.20157986659819e-07
1279 6.20282532104e-07
1280 6.20512788032102e-07
1281 6.20443126081227e-07
1282 6.19954360075781e-07
1283 6.20158316202435e-07
1284 6.20412783675306e-07
1285 6.19787940749461e-07
1286 6.20167513709191e-07
1287 6.19920055640932e-07
1288 6.19938900399575e-07
1289 6.2001241861509e-07
1290 6.19576977349823e-07
1291 6.19768135450727e-07
1292 6.19923710424075e-07
1293 6.19364097445896e-07
1294 6.19723660356897e-07
1295 6.19799817101807e-07
1296 6.19224814414565e-07
1297 6.19575253921312e-07
1298 6.1951230144075e-07
1299 6.19440844417341e-07
1300 6.19388264468057e-07
1301 6.1928821070012e-07
1302 6.18971549577907e-07
1303 6.19309097317e-07
1304 6.18896740419927e-07
1305 6.19233279984144e-07
1306 6.18862542118848e-07
1307 6.1904855721906e-07
1308 6.19084869462938e-07
1309 6.18728734835372e-07
1310 6.19067870623269e-07
1311 6.18833963194731e-07
1312 6.18797829730511e-07
1313 6.18840376517937e-07
1314 6.18354753008532e-07
1315 6.18526108105755e-07
1316 6.18729250248862e-07
1317 6.18316310337263e-07
1318 6.18493220983396e-07
1319 6.1857565151513e-07
1320 6.18553323434412e-07
1321 6.18287408251206e-07
1322 6.18182901995112e-07
1323 6.18168552804832e-07
1324 6.18295116474599e-07
1325 6.18236038974373e-07
1326 6.18178005744596e-07
1327 6.18093473242709e-07
1328 6.17960725605826e-07
1329 6.17843770534421e-07
1330 6.18153380194997e-07
1331 6.1793422157308e-07
1332 6.17757560888776e-07
1333 6.1783743137056e-07
1334 6.17917767534948e-07
1335 6.17540525162497e-07
1336 6.17705096779275e-07
1337 6.17609444908851e-07
1338 6.17570422804192e-07
1339 6.17073546720803e-07
1340 6.17361359793733e-07
1341 6.17218927558838e-07
1342 6.1731549609334e-07
1343 6.16839265717317e-07
1344 6.17278455536052e-07
1345 6.16886698999508e-07
1346 6.17145374242512e-07
1347 6.16674280522034e-07
1348 6.16949876508954e-07
1349 6.16746271418833e-07
1350 6.17063569656295e-07
1351 6.1651876738722e-07
1352 6.16958195287509e-07
1353 6.16493237330928e-07
1354 6.16818991467483e-07
1355 6.16564756263926e-07
1356 6.16465375699704e-07
1357 6.1670159974625e-07
1358 6.16341221103767e-07
1359 6.164012397889e-07
1360 6.16298516355584e-07
1361 6.16462023430131e-07
1362 6.16257752142246e-07
1363 6.16205831555305e-07
1364 6.16200928298838e-07
1365 6.16331213180388e-07
1366 6.15774696939297e-07
1367 6.1614706064006e-07
1368 6.16303828067544e-07
1369 6.15772465287989e-07
1370 6.16175163123955e-07
1371 6.15761831603834e-07
1372 6.16118617230654e-07
1373 6.15417715017941e-07
1374 6.15825617352073e-07
1375 6.15959238160713e-07
1376 6.1554938204722e-07
1377 6.1578529990669e-07
1378 6.15404216532056e-07
1379 6.15805073294951e-07
1380 6.15045303192119e-07
1381 6.15779943856865e-07
1382 6.15239680826107e-07
1383 6.15575325518591e-07
1384 6.15164999260287e-07
1385 6.15541761249006e-07
1386 6.1481267056962e-07
1387 6.15518598799269e-07
1388 6.14974511393029e-07
1389 6.15256523204266e-07
1390 6.1495169554604e-07
1391 6.15171495873312e-07
1392 6.14589822056644e-07
1393 6.15063232331181e-07
1394 6.15107449206675e-07
1395 6.14662685840983e-07
1396 6.1502810259384e-07
1397 6.14590154093264e-07
1398 6.14612481669496e-07
1399 6.14802560768624e-07
1400 6.14702865100014e-07
1401 6.14827589217271e-07
1402 6.14387261038019e-07
1403 6.14532386492783e-07
1404 6.14365300663167e-07
1405 6.14513860909938e-07
1406 6.14463790768127e-07
1407 6.14669436131976e-07
1408 6.14088050312489e-07
1409 6.14400729553211e-07
1410 6.14347149664241e-07
1411 6.14008724106441e-07
1412 6.14388912921982e-07
1413 6.13885905003997e-07
1414 6.14030799852117e-07
1415 6.14084501023626e-07
1416 6.14049639672487e-07
1417 6.14017295546887e-07
1418 6.13979460368341e-07
1419 6.14067722430889e-07
1420 6.1396852149187e-07
1421 6.13854022716964e-07
1422 6.13682396014781e-07
1423 6.13604328492556e-07
1424 6.13541604060686e-07
1425 6.13372681478097e-07
1426 6.13554293892093e-07
1427 6.13180419733794e-07
1428 6.13455641953919e-07
1429 6.1348410481088e-07
1430 6.13415493866398e-07
1431 6.13489517483856e-07
1432 6.12983758479402e-07
1433 6.13290954696311e-07
1434 6.13153399413591e-07
1435 6.13249425313711e-07
1436 6.13075842231581e-07
1437 6.1320476547877e-07
1438 6.13227797749971e-07
1439 6.12789343271913e-07
1440 6.13057139688067e-07
1441 6.13030189320796e-07
1442 6.12636833110969e-07
1443 6.12816545171313e-07
1444 6.12819367759698e-07
1445 6.12868730669902e-07
1446 6.12897299447468e-07
1447 6.12634370639853e-07
1448 6.12377568522504e-07
1449 6.12560680650631e-07
1450 6.1260865675905e-07
1451 6.12693906937523e-07
1452 6.12507754290448e-07
1453 6.12503317547919e-07
1454 6.12373988857939e-07
1455 6.12564684573158e-07
1456 6.12393487315899e-07
1457 6.12160607012413e-07
1458 6.12446222532981e-07
1459 6.12120268648653e-07
1460 6.12264537330987e-07
1461 6.12050654183349e-07
1462 6.12274880353425e-07
1463 6.12251636255223e-07
1464 6.12217257234704e-07
1465 6.11654116092097e-07
1466 6.11951332530225e-07
1467 6.12065274943063e-07
1468 6.11820810711095e-07
1469 6.11882885237947e-07
1470 6.11983560567353e-07
1471 6.11653300381931e-07
1472 6.11477287904449e-07
1473 6.11686302534054e-07
1474 6.11679580714508e-07
1475 6.11734728380497e-07
1476 6.11446988500575e-07
1477 6.11656725389764e-07
1478 6.11418435035205e-07
1479 6.1152755747429e-07
1480 6.11419605277774e-07
1481 6.11340049509579e-07
1482 6.11387434275912e-07
1483 6.11206909837847e-07
1484 6.11388030407056e-07
1485 6.11385448635815e-07
1486 6.11177371865779e-07
1487 6.11203298248597e-07
1488 6.11062275424956e-07
1489 6.1113751262809e-07
1490 6.11060912476091e-07
1491 6.11207886258569e-07
1492 6.1091258017143e-07
1493 6.11062240871263e-07
1494 6.10889090808087e-07
1495 6.1091539818392e-07
1496 6.10902968965377e-07
1497 6.10834144879391e-07
1498 6.10740337897653e-07
1499 6.10833803591504e-07
1500 6.10895269836931e-07
1501 6.10666885577871e-07
1502 6.10610917632925e-07
1503 6.10713644597638e-07
1504 6.10572242344176e-07
1505 6.10609684585484e-07
1506 6.10557814717083e-07
1507 6.10448972544475e-07
1508 6.10527583972953e-07
1509 6.1040572688853e-07
1510 6.1039827343734e-07
1511 6.10298882641302e-07
1512 6.10335146262742e-07
1513 6.10234822481459e-07
1514 6.10242099241987e-07
1515 6.1033837800295e-07
1516 6.10180680752137e-07
1517 6.10083428981056e-07
1518 6.10104180410076e-07
1519 6.09997833088016e-07
1520 6.10046423801691e-07
1521 6.10109085918964e-07
1522 6.09993698589051e-07
1523 6.0983415817617e-07
1524 6.09903624265939e-07
1525 6.09998824259605e-07
1526 6.09884452536846e-07
1527 6.0968484560675e-07
1528 6.09763739340963e-07
1529 6.09665772230983e-07
1530 6.09703086560387e-07
1531 6.09820219096946e-07
1532 6.09501713334737e-07
1533 6.09573474022795e-07
1534 6.09616974934113e-07
1535 6.09569544238298e-07
1536 6.09660148164437e-07
1537 6.09644980698931e-07
1538 6.09337568604929e-07
1539 6.09344284718816e-07
1540 6.09197389067617e-07
1541 6.0940735770032e-07
1542 6.09346155506785e-07
1543 6.09101156697989e-07
1544 6.09007024621633e-07
1545 6.09158799576903e-07
1546 6.09148734696419e-07
1547 6.08959669804676e-07
1548 6.09325477711309e-07
1549 6.09174900532139e-07
1550 6.0887942427712e-07
1551 6.08839884300494e-07
1552 6.08781071470332e-07
1553 6.09002018386207e-07
1554 6.08992753903692e-07
1555 6.08921752103697e-07
1556 6.08550892451376e-07
1557 6.08858571517601e-07
1558 6.08579321387026e-07
1559 6.08931567271043e-07
1560 6.08685558709965e-07
1561 6.086547848696e-07
1562 6.08401901835975e-07
1563 6.08758491502215e-07
1564 6.08654236842199e-07
1565 6.08688211428898e-07
1566 6.08301819781332e-07
1567 6.08271498485635e-07
1568 6.08576608897238e-07
1569 6.08319848055317e-07
1570 6.08480469537653e-07
1571 6.07807451878273e-07
1572 6.08380918549756e-07
1573 6.08300615851931e-07
1574 6.08053642686457e-07
1575 6.08368646929591e-07
1576 6.0812452350234e-07
1577 6.08126570497802e-07
1578 6.07947861496427e-07
1579 6.08041351142674e-07
1580 6.08088597630285e-07
1581 6.07849733981425e-07
1582 6.07615484732094e-07
1583 6.08021072672216e-07
1584 6.07963512450738e-07
1585 6.07767478811638e-07
1586 6.07390047363765e-07
1587 6.07963038447679e-07
1588 6.07980262998353e-07
1589 6.07409527361824e-07
1590 6.07784210387763e-07
1591 6.07840305448803e-07
1592 6.07267094785868e-07
1593 6.07612616228437e-07
1594 6.07545026539924e-07
1595 6.07546535789538e-07
1596 6.07477840198101e-07
1597 6.07406954912904e-07
1598 6.07465134486063e-07
1599 6.07722121941379e-07
1600 6.07360805602752e-07
1601 6.07478799395267e-07
1602 6.06909562463898e-07
1603 6.07380107794597e-07
1604 6.07338322012652e-07
1605 6.07294482563248e-07
1606 6.06812804122114e-07
1607 6.0721268062025e-07
1608 6.07233012985375e-07
1609 6.07195121141046e-07
1610 6.071700273651e-07
1611 6.06819598012009e-07
1612 6.07043475504554e-07
1613 6.0704164012293e-07
1614 6.07128623613562e-07
1615 6.06578756020326e-07
1616 6.06798978871836e-07
1617 6.07025782869641e-07
1618 6.06951957884405e-07
1619 6.06766503608469e-07
1620 6.06614765722213e-07
1621 6.0673492006913e-07
1622 6.06793863688893e-07
1623 6.06836134508626e-07
1624 6.06528794833139e-07
1625 6.06360835142539e-07
1626 6.06481000843928e-07
1627 6.06527924325917e-07
1628 6.06438592249958e-07
1629 6.06078926878695e-07
1630 6.06267785045134e-07
1631 6.06416825974065e-07
1632 6.06345947673503e-07
1633 6.06055386640492e-07
1634 6.06261331746794e-07
1635 6.06446446099085e-07
1636 6.06188030587873e-07
1637 6.05938584328669e-07
1638 6.05899705995228e-07
1639 6.06198064531327e-07
1640 6.06270861851499e-07
1641 6.05708320840392e-07
1642 6.06058464242665e-07
1643 6.06221903296955e-07
1644 6.06069751235339e-07
1645 6.06050152747173e-07
1646 6.05819766406057e-07
1647 6.06043714824978e-07
1648 6.05906337582951e-07
1649 6.05514812804131e-07
1650 6.0590809746941e-07
1651 6.05843198755451e-07
1652 6.05349268901989e-07
1653 6.05659533334801e-07
1654 6.05764375777085e-07
1655 6.05851891698705e-07
1656 6.05333976615441e-07
1657 6.05361576432983e-07
1658 6.05489755962196e-07
1659 6.05671521000772e-07
1660 6.05699834061113e-07
1661 6.05625775143892e-07
1662 6.05465998177124e-07
1663 6.05155967747351e-07
1664 6.05444437752567e-07
1665 6.05435842629731e-07
1666 6.05232719344428e-07
1667 6.05358329359262e-07
1668 6.05482555464221e-07
1669 6.04889914235685e-07
1670 6.05215874934117e-07
1671 6.05486620990803e-07
1672 6.04834402068377e-07
1673 6.05198130770646e-07
1674 6.05046599076786e-07
1675 6.05016511656231e-07
1676 6.04631185808557e-07
1677 6.05005039972184e-07
1678 6.04674635233948e-07
1679 6.04942220718385e-07
1680 6.04700395697932e-07
1681 6.04865811205002e-07
1682 6.05009814954371e-07
1683 6.04889728634816e-07
1684 6.04844792221115e-07
1685 6.04359664151843e-07
1686 6.04609927442823e-07
1687 6.04872898598785e-07
1688 6.04709866742326e-07
1689 6.04630048052002e-07
1690 6.04425468274883e-07
1691 6.04469091577187e-07
1692 6.04706080885364e-07
1693 6.04555926010164e-07
1694 6.04537494460544e-07
1695 6.04278143796932e-07
1696 6.04340501048739e-07
1697 6.04197455913891e-07
1698 6.04441300453118e-07
1699 6.04595920748352e-07
1700 6.03850993272204e-07
1701 6.04438322397982e-07
1702 6.04277625079419e-07
1703 6.0438274098118e-07
1704 6.04373536390312e-07
1705 6.04138261813603e-07
1706 6.03913386768795e-07
1707 6.04411630916957e-07
1708 6.04183150997528e-07
1709 6.04112993976003e-07
1710 6.03658407840157e-07
1711 6.04053948315197e-07
1712 6.03713166043462e-07
1713 6.03930260858476e-07
1714 6.04069584888123e-07
1715 6.03919799146979e-07
1716 6.0421229888874e-07
1717 6.03807143292556e-07
1718 6.03866195504565e-07
1719 6.03754792258826e-07
1720 6.03654986100821e-07
1721 6.03806830298481e-07
1722 6.03825659105439e-07
1723 6.03713809127271e-07
1724 6.03699650824296e-07
1725 6.03421308653651e-07
1726 6.0333590960937e-07
1727 6.03738354897132e-07
1728 6.0345359814562e-07
1729 6.03663957136291e-07
1730 6.0354759295933e-07
1731 6.03492661575444e-07
1732 6.0341275359832e-07
1733 6.03002478229087e-07
1734 6.03387956459756e-07
1735 6.03469406755153e-07
1736 6.03433570702805e-07
1737 6.0356447433918e-07
1738 6.03225716766076e-07
1739 6.03256337910807e-07
1740 6.02846615471719e-07
1741 6.03157913118935e-07
1742 6.03347374195096e-07
1743 6.02996052606386e-07
1744 6.03152832425735e-07
1745 6.0341758439364e-07
1746 6.03047560318259e-07
1747 6.02706116737295e-07
1748 6.03170202509773e-07
1749 6.0302586018679e-07
1750 6.02906125479308e-07
1751 6.02758776139467e-07
1752 6.0285802859994e-07
1753 6.02876531445418e-07
1754 6.03065958493687e-07
1755 6.02700803597145e-07
1756 6.03003898859811e-07
1757 6.02863685159605e-07
1758 6.02832223307814e-07
1759 6.0231757582585e-07
1760 6.02770864048807e-07
1761 6.02698666838819e-07
1762 6.02657274626495e-07
1763 6.02313723852887e-07
1764 6.02691921884002e-07
1765 6.02726068130721e-07
1766 6.02516639780504e-07
1767 6.02727442071682e-07
1768 6.0266054122593e-07
1769 6.02443684655896e-07
1770 6.02768862478342e-07
1771 6.02575757440604e-07
1772 6.02620517042851e-07
1773 6.0223736930709e-07
1774 6.02443618674897e-07
1775 6.02343127688698e-07
1776 6.02535673678517e-07
1777 6.02526791688263e-07
1778 6.02374621024637e-07
1779 6.02588626342993e-07
1780 6.02112564081381e-07
1781 6.02313130720233e-07
1782 6.02444795354984e-07
1783 6.02347095764344e-07
1784 6.02369666538039e-07
1785 6.0209122123922e-07
1786 6.01879268558037e-07
1787 6.02125398557973e-07
1788 6.02146500632728e-07
1789 6.01656791744176e-07
1790 6.01805090859386e-07
1791 6.02002694151338e-07
1792 6.02446760957775e-07
1793 6.02113278262095e-07
1794 6.01991785778466e-07
1795 6.01826009479112e-07
1796 6.01817396692184e-07
1797 6.01982162095283e-07
1798 6.0186446570043e-07
1799 6.01880066668059e-07
1800 6.01632548281827e-07
1801 6.01748339363439e-07
1802 6.01747545857734e-07
1803 6.01840091860595e-07
1804 6.01756409380982e-07
1805 6.01922086502782e-07
1806 6.01554449268349e-07
1807 6.01764710836505e-07
1808 6.01585441891928e-07
1809 6.01620399663716e-07
1810 6.01394540993283e-07
1811 6.01578889998677e-07
1812 6.01465641970833e-07
1813 6.01468468630628e-07
1814 6.0148105585256e-07
1815 6.01621551680864e-07
1816 6.01349377944871e-07
1817 6.01436685400358e-07
1818 6.01112377616175e-07
1819 6.01373257438809e-07
1820 6.01033992737143e-07
1821 6.01447081656659e-07
1822 6.01251637455391e-07
1823 6.01425233682562e-07
1824 6.0108123882685e-07
1825 6.00530437850466e-07
1826 6.00767299317795e-07
1827 6.01164053591674e-07
1828 6.01070501694778e-07
1829 6.01493606254167e-07
1830 6.01099195534971e-07
1831 6.01244900330755e-07
1832 6.01057305814834e-07
1833 6.00740347941553e-07
1834 6.00686449153898e-07
1835 6.00974852872582e-07
1836 6.00784366120877e-07
1837 6.00917683321711e-07
1838 6.00919331205319e-07
1839 6.00305996499628e-07
1840 6.00554979889978e-07
1841 6.00792678504547e-07
1842 6.00769171384741e-07
1843 6.00791608299289e-07
1844 6.00129414905837e-07
1845 6.00422443270077e-07
1846 6.01062785804629e-07
1847 6.00362743504945e-07
1848 6.00542174204577e-07
1849 6.00501732691328e-07
1850 6.00728357547098e-07
1851 6.00461085412007e-07
1852 6.00623625381047e-07
1853 6.0029681075946e-07
1854 6.00380710018555e-07
1855 6.00392723740129e-07
1856 6.00370193353683e-07
1857 6.00772083174661e-07
1858 6.00252417889635e-07
1859 6.0015084682874e-07
1860 5.99909958978628e-07
1861 6.00418678168069e-07
1862 6.00348944090001e-07
1863 6.00298169089797e-07
1864 5.99944616880066e-07
1865 6.00106185999039e-07
1866 5.99787089782922e-07
1867 6.00101847382462e-07
1868 5.99852434874038e-07
1869 5.99963285409899e-07
1870 5.9991632118539e-07
1871 6.00112840409395e-07
1872 5.99863743829587e-07
1873 6.00275846551313e-07
1874 5.9979787373976e-07
1875 5.99840269444485e-07
1876 5.99727364530622e-07
1877 5.99769747580581e-07
1878 5.99911294067112e-07
1879 5.99118779724961e-07
1880 5.9970115111696e-07
1881 6.00002350559237e-07
1882 5.99974184062546e-07
1883 5.99959595589894e-07
1884 5.99087935775344e-07
1885 5.99623948900785e-07
1886 5.99626938665665e-07
1887 5.99830924784328e-07
1888 5.99417291731186e-07
1889 5.9955394310407e-07
1890 5.99298534126547e-07
1891 5.99730335409276e-07
1892 5.99491005132791e-07
1893 5.99656654244995e-07
1894 5.99547013450774e-07
1895 5.98482167085024e-07
1896 5.99093556694186e-07
1897 5.99366488827968e-07
1898 5.9942543583702e-07
1899 5.99619707813304e-07
1900 5.9898585281104e-07
1901 5.99188566802411e-07
1902 5.99299312909807e-07
1903 5.98916520921478e-07
1904 5.99358396179639e-07
1905 5.99035597367958e-07
1906 5.98880591311968e-07
1907 5.991552496738e-07
1908 5.98895090178075e-07
1909 5.99326609851403e-07
1910 5.9899597306412e-07
1911 5.98961957322786e-07
1912 5.99268629031258e-07
1913 5.98984463138663e-07
1914 5.98816665522861e-07
1915 5.99366927694689e-07
1916 5.99037884278175e-07
1917 5.98687263121178e-07
1918 5.99125418858648e-07
1919 5.98926484734363e-07
1920 5.98734453859606e-07
1921 5.98929530852388e-07
1922 5.98456384594215e-07
1923 5.99225023350414e-07
1924 5.98748275706384e-07
1925 5.98757728198507e-07
1926 5.9836822791226e-07
1927 5.98157728056492e-07
1928 5.98544783777299e-07
1929 5.98487580987239e-07
1930 5.98244729438591e-07
1931 5.98312649913169e-07
1932 5.98401026920214e-07
1933 5.98220377682423e-07
1934 5.98461834648845e-07
1935 5.98176858069621e-07
1936 5.98417225589287e-07
1937 5.98470352322522e-07
1938 5.97895304501606e-07
1939 5.97987169612679e-07
1940 5.9813538484832e-07
1941 5.98038864715988e-07
1942 5.98157100114349e-07
1943 5.97956616644524e-07
1944 5.98089292012105e-07
1945 5.97987895616825e-07
1946 5.97809126240634e-07
1947 5.98036604159802e-07
1948 5.97797029627145e-07
1949 5.98056657786117e-07
1950 5.97913980556086e-07
1951 5.97672298432883e-07
1952 5.98109917312684e-07
1953 5.97941035053395e-07
1954 5.9753544955754e-07
1955 5.97756714242337e-07
1956 5.97478627987869e-07
1957 5.97894887576444e-07
1958 5.97648920255267e-07
1959 5.97599577375263e-07
1960 5.97631638868279e-07
1961 5.97793312579142e-07
1962 5.97506529700809e-07
1963 5.97551933459783e-07
1964 5.97903794457011e-07
1965 5.97512350587692e-07
1966 5.97395159424252e-07
1967 5.97335368631491e-07
1968 5.971654629775e-07
1969 5.97480472521283e-07
1970 5.97409489770939e-07
1971 5.97503929363086e-07
1972 5.97378307574559e-07
1973 5.97422970471939e-07
1974 5.972871803408e-07
1975 5.972749112928e-07
1976 5.97247786807031e-07
1977 5.97383588605283e-07
1978 5.97325264600101e-07
1979 5.97643469468778e-07
1980 5.97130998862383e-07
1981 5.97120845668542e-07
1982 5.97042139453663e-07
1983 5.97357733362003e-07
1984 5.97179822371174e-07
1985 5.96607952928707e-07
1986 5.96828548545147e-07
1987 5.97014385220973e-07
1988 5.97011463028707e-07
1989 5.96872746299937e-07
1990 5.96864856504453e-07
1991 5.96725136745135e-07
1992 5.97047547614693e-07
1993 5.96956392918457e-07
1994 5.96588074095905e-07
1995 5.9687866851732e-07
1996 5.9665970918843e-07
1997 5.96951477064067e-07
1998 5.96457130043859e-07
1999 5.97000214682453e-07
};
\addlegendentry{Train}
\addplot [semithick, black]
table {%
0 0.00434749247506261
1 0.00228160619735718
2 0.00205221213400364
3 0.00154257053509355
4 0.000718198833055794
5 0.000226133677642792
6 0.000181061404873617
7 0.000167616395629011
8 0.000156123365741223
9 0.000142964738188311
10 0.000127599021652713
11 0.000110671688162256
12 9.30606038309634e-05
13 7.64159558457322e-05
14 6.21071667410433e-05
15 5.10052486788481e-05
16 4.28141283919103e-05
17 3.68101937056053e-05
18 3.00778083328623e-05
19 2.4529757865821e-05
20 2.01874263439095e-05
21 1.66584795806557e-05
22 1.37592851388035e-05
23 1.14278200271656e-05
24 9.60716533882078e-06
25 8.22439596959157e-06
26 7.18310775482678e-06
27 6.39809377389611e-06
28 5.78356139158132e-06
29 5.30209854332497e-06
30 4.88823434352526e-06
31 4.53025495517068e-06
32 4.21924369220505e-06
33 3.93188747693785e-06
34 3.67658185496111e-06
35 3.44876343660871e-06
36 3.2360949262511e-06
37 3.0473665901809e-06
38 2.88254500446783e-06
39 2.73407295026118e-06
40 2.60622755376971e-06
41 2.49123081630387e-06
42 2.38923371398414e-06
43 2.29731722356519e-06
44 2.21009304368636e-06
45 2.13793896364223e-06
46 2.0732218217745e-06
47 2.0203365238558e-06
48 1.97086956177372e-06
49 1.91087838175008e-06
50 1.87627824743686e-06
51 1.82744747689867e-06
52 1.79010999090679e-06
53 1.74291096755042e-06
54 1.71458736986096e-06
55 1.69069596722693e-06
56 1.65673736773897e-06
57 1.63174570388946e-06
58 1.60747129029915e-06
59 1.58263560479099e-06
60 1.56158080244495e-06
61 1.54273459429533e-06
62 1.5270040876203e-06
63 1.508788159299e-06
64 1.48908816299809e-06
65 1.47157015817356e-06
66 1.45421120123501e-06
67 1.43652664519323e-06
68 1.41859891300555e-06
69 1.40249676405801e-06
70 1.39373685215105e-06
71 1.37832682867156e-06
72 1.36012943130481e-06
73 1.34492074721493e-06
74 1.32760681026411e-06
75 1.31394085656211e-06
76 1.30380783502915e-06
77 1.28918247810361e-06
78 1.27781720493658e-06
79 1.26867018934718e-06
80 1.25864710298629e-06
81 1.24917630728305e-06
82 1.23994277601014e-06
83 1.22975836802652e-06
84 1.21914297324111e-06
85 1.20943730053114e-06
86 1.19995434033626e-06
87 1.19173625989788e-06
88 1.18392813419632e-06
89 1.17324577786349e-06
90 1.1509973774082e-06
91 1.14243664484093e-06
92 1.1369610319889e-06
93 1.12879831704049e-06
94 1.12406837615708e-06
95 1.10950884391059e-06
96 1.10892358407e-06
97 1.10261817098944e-06
98 1.09690631688864e-06
99 1.09067332232371e-06
100 1.08576807633654e-06
101 1.08011545307818e-06
102 1.07385312730912e-06
103 1.06806089661404e-06
104 1.06117443010589e-06
105 1.05581989373604e-06
106 1.05201229416707e-06
107 1.04553146229591e-06
108 1.04055129668268e-06
109 1.03591617062193e-06
110 1.03007459983928e-06
111 1.02811327451491e-06
112 1.02186095318757e-06
113 1.01746786640433e-06
114 1.01230853033485e-06
115 1.01044577149878e-06
116 1.0063632771562e-06
117 1.00233330613264e-06
118 9.98497284854238e-07
119 9.94633751361107e-07
120 9.90572061709827e-07
121 9.86766963251284e-07
122 9.83146946964553e-07
123 9.78703383225366e-07
124 9.74755494098645e-07
125 9.7125587217306e-07
126 9.68463950812293e-07
127 9.65317553891509e-07
128 9.62315880315145e-07
129 9.5937218702602e-07
130 9.55727841756016e-07
131 9.65504568739561e-07
132 9.62504373092088e-07
133 9.57780116550566e-07
134 9.55338009589468e-07
135 9.52584571223269e-07
136 9.47709168030997e-07
137 9.45862950629817e-07
138 9.42336384923692e-07
139 9.39557139645331e-07
140 9.37117363264406e-07
141 9.38099617542321e-07
142 9.34217325720965e-07
143 9.29640691538225e-07
144 9.28908491459879e-07
145 9.27524069993524e-07
146 9.25653353078815e-07
147 9.27854500787362e-07
148 9.20564673378976e-07
149 9.17459090032935e-07
150 9.15438135962177e-07
151 9.14473162083596e-07
152 9.11512699985906e-07
153 9.09493792278226e-07
154 9.04993896710948e-07
155 9.11056986296899e-07
156 9.07219032342255e-07
157 9.03521765849291e-07
158 9.03147849840025e-07
159 9.00471945897152e-07
160 8.98007328942185e-07
161 9.01668158803659e-07
162 8.93996002560016e-07
163 8.92658079010289e-07
164 8.9248140966447e-07
165 8.94123616035358e-07
166 8.9210413989349e-07
167 8.83389816408453e-07
168 8.81091580140492e-07
169 8.84440567006095e-07
170 8.77607078564324e-07
171 8.77156878686947e-07
172 8.69033385697549e-07
173 8.72449504640826e-07
174 8.69888140186958e-07
175 8.75546902534552e-07
176 8.69605116804451e-07
177 8.65835147578764e-07
178 8.59352553561621e-07
179 8.55793132359395e-07
180 8.54213169532159e-07
181 8.59682813825202e-07
182 8.60085890508344e-07
183 8.57171983170701e-07
184 8.55991800108313e-07
185 8.54777340464352e-07
186 8.53036908665672e-07
187 8.51283061820141e-07
188 8.5193028098729e-07
189 8.50044557410001e-07
190 8.41960115849361e-07
191 8.45497083901137e-07
192 8.44312069148145e-07
193 8.46131911202974e-07
194 8.4455683690976e-07
195 8.41735698031698e-07
196 8.41986036448361e-07
197 8.42979375192954e-07
198 8.32292528230028e-07
199 8.36260539927025e-07
200 8.28120391815901e-07
201 8.3328586697462e-07
202 8.31986767479975e-07
203 8.3191019939477e-07
204 8.23791140192043e-07
205 8.29448083550233e-07
206 8.22248750864674e-07
207 8.25539586912782e-07
208 8.20777074750367e-07
209 8.2558011627043e-07
210 8.24426592771488e-07
211 8.23368054625462e-07
212 8.22838956082705e-07
213 8.22967081148818e-07
214 8.2723124705808e-07
215 8.24509356789349e-07
216 8.22551385226689e-07
217 8.19931017304043e-07
218 8.20179309357627e-07
219 8.18713544958882e-07
220 8.18752312170545e-07
221 8.17922341411759e-07
222 8.16474653220212e-07
223 8.14113207070477e-07
224 8.14713530417066e-07
225 8.12543078154704e-07
226 8.12100893199386e-07
227 8.11079814866389e-07
228 8.12268183381093e-07
229 8.11947529655299e-07
230 8.10835672382382e-07
231 8.09050561656477e-07
232 8.08929485174303e-07
233 8.07899596111383e-07
234 8.07337016794918e-07
235 8.0674215041654e-07
236 8.0271234992324e-07
237 8.04715227786801e-07
238 8.03089506007382e-07
239 8.02697798008012e-07
240 8.01223507096438e-07
241 7.98131623014342e-07
242 7.98794815182191e-07
243 7.97887537373754e-07
244 7.96937001723563e-07
245 7.9594468616051e-07
246 7.95331402514421e-07
247 7.93857566350198e-07
248 7.93269009591313e-07
249 7.9232358984882e-07
250 7.91217587448045e-07
251 7.90401657013717e-07
252 7.89184298355394e-07
253 7.86902205618389e-07
254 7.87498265708564e-07
255 7.86780731232284e-07
256 7.85954284765467e-07
257 7.85163308592018e-07
258 7.84254552854691e-07
259 7.83878761012602e-07
260 7.82506845098396e-07
261 7.82381562203227e-07
262 7.80722018589586e-07
263 7.80565869717975e-07
264 7.80142954681651e-07
265 7.79061451794405e-07
266 7.78061462369806e-07
267 7.77561069753574e-07
268 7.76982233219314e-07
269 7.76394358581456e-07
270 7.75580019762856e-07
271 7.75101568706305e-07
272 7.74141142301232e-07
273 7.73420026689564e-07
274 7.72909743318451e-07
275 7.72820385464001e-07
276 7.7215980809342e-07
277 7.70394876781211e-07
278 7.69719520121726e-07
279 7.691998575865e-07
280 7.69394375765842e-07
281 7.67452718264394e-07
282 7.67373762755597e-07
283 7.67319932037935e-07
284 7.65990023410268e-07
285 7.65466893426492e-07
286 7.64721789892064e-07
287 7.64168078148941e-07
288 7.63261141401017e-07
289 7.62611932714208e-07
290 7.61927935855056e-07
291 7.61432772833359e-07
292 7.58391877297981e-07
293 7.58502608277922e-07
294 7.57785414862155e-07
295 7.56770930365747e-07
296 7.56209033170308e-07
297 7.55275493702356e-07
298 7.55437724819785e-07
299 7.53612937387516e-07
300 7.53117149088212e-07
301 7.52098856082739e-07
302 7.53102483486146e-07
303 7.53071276449191e-07
304 7.52046616980806e-07
305 7.50117919778859e-07
306 7.49524758703046e-07
307 7.49935281874059e-07
308 7.4813090122916e-07
309 7.48565071262419e-07
310 7.48480601941992e-07
311 7.48050467791472e-07
312 7.4746179734575e-07
313 7.47299566228321e-07
314 7.44646570183249e-07
315 7.44110764117067e-07
316 7.44592284718237e-07
317 7.44728993140598e-07
318 7.43561713534291e-07
319 7.44146404940693e-07
320 7.42608563086833e-07
321 7.42190707114787e-07
322 7.40962263989786e-07
323 7.42079009796726e-07
324 7.40356142614473e-07
325 7.38976041247952e-07
326 7.3847712656061e-07
327 7.37713548915053e-07
328 7.38197684313491e-07
329 7.37495042812952e-07
330 7.3582208415246e-07
331 7.36848448923411e-07
332 7.35805656404409e-07
333 7.33855188173038e-07
334 7.34642640054517e-07
335 7.33675904029951e-07
336 7.33129013497091e-07
337 7.33792035134684e-07
338 7.32961041194358e-07
339 7.30489091438358e-07
340 7.31641080164991e-07
341 7.30738577203738e-07
342 7.29801229226723e-07
343 7.30696967821132e-07
344 7.28915154013521e-07
345 7.2738930612104e-07
346 7.27571375591651e-07
347 7.27127883237699e-07
348 7.26605151157855e-07
349 7.26159214536892e-07
350 7.24804465335183e-07
351 7.24450444522518e-07
352 7.24580900168803e-07
353 7.23101265975856e-07
354 7.25181280358811e-07
355 7.21174274076475e-07
356 7.21943877124431e-07
357 7.22024424248957e-07
358 7.20705259027454e-07
359 7.21280571269745e-07
360 7.20751643257245e-07
361 7.19062086318445e-07
362 7.19834304163669e-07
363 7.19305432994588e-07
364 7.18614899142267e-07
365 7.16819442914129e-07
366 7.17513614745258e-07
367 7.16848603588005e-07
368 7.16118620402995e-07
369 7.14568159310147e-07
370 7.15248006599722e-07
371 7.14689520009415e-07
372 7.14215104835603e-07
373 7.12544363068446e-07
374 7.13283270670217e-07
375 7.1273029789154e-07
376 7.12003952685336e-07
377 7.11549887455476e-07
378 7.09858170466759e-07
379 7.10545691617881e-07
380 7.09978849045001e-07
381 7.09333903614606e-07
382 7.08702884821832e-07
383 7.07186472936883e-07
384 7.07658841747616e-07
385 7.07052095094696e-07
386 7.06604453171167e-07
387 7.06045170772995e-07
388 7.05637660303182e-07
389 7.039701586109e-07
390 7.04642843629699e-07
391 7.04044452959351e-07
392 7.03616649389005e-07
393 7.03200157659012e-07
394 7.02423676557373e-07
395 7.0100924176586e-07
396 7.01711371675628e-07
397 7.01109911460662e-07
398 7.00724115176854e-07
399 7.0014550601627e-07
400 6.99674785664683e-07
401 6.99185022767779e-07
402 6.98526037012925e-07
403 6.97113023306883e-07
404 6.97783491432347e-07
405 6.97211135047837e-07
406 6.96728591265128e-07
407 6.96238316777453e-07
408 6.95751396051492e-07
409 6.95285109486576e-07
410 6.94797620326426e-07
411 6.94314621796366e-07
412 6.92933042500954e-07
413 6.93487834269035e-07
414 6.92926505507785e-07
415 6.92515527589421e-07
416 6.91978584654862e-07
417 6.91427544552425e-07
418 6.91062837177014e-07
419 6.90576371198404e-07
420 6.90117644808197e-07
421 6.8965192667747e-07
422 6.89073999637912e-07
423 6.87776207541901e-07
424 6.88068894305616e-07
425 6.87523083797714e-07
426 6.87386261688516e-07
427 6.85023621826986e-07
428 6.84430517594592e-07
429 6.83863163430942e-07
430 6.8320713353387e-07
431 6.82995391798613e-07
432 6.82697191223269e-07
433 6.82154791320499e-07
434 6.81752055697871e-07
435 6.80956418364076e-07
436 6.80733080571372e-07
437 6.80360813021252e-07
438 6.79862694141775e-07
439 6.78962749134371e-07
440 6.78817912103113e-07
441 6.78641185913875e-07
442 6.78063713621668e-07
443 6.77342143262649e-07
444 6.77221748901502e-07
445 6.76901947826991e-07
446 6.76410422784102e-07
447 6.76095567087032e-07
448 6.75363082791591e-07
449 6.75353931001155e-07
450 6.74947614243138e-07
451 6.74659304422676e-07
452 6.73899592129601e-07
453 6.73770102821436e-07
454 6.734086355209e-07
455 6.72968440085242e-07
456 6.72458668304898e-07
457 6.72336284424091e-07
458 6.71967143262009e-07
459 6.71548377795261e-07
460 6.70906786126579e-07
461 6.70788836032443e-07
462 6.70526560497819e-07
463 6.70081192311045e-07
464 6.69369001116138e-07
465 6.69240478146094e-07
466 6.68576149109867e-07
467 6.68258337555017e-07
468 6.67785855057446e-07
469 6.67444737700862e-07
470 6.67012272970169e-07
471 6.6678467192105e-07
472 6.66446908326179e-07
473 6.65777065478323e-07
474 6.65613526962261e-07
475 6.65223240048363e-07
476 6.64587219034729e-07
477 6.64233652969415e-07
478 6.6413457489034e-07
479 6.63501566577906e-07
480 6.63182163407328e-07
481 6.62958200337016e-07
482 6.62354466385295e-07
483 6.60027126286877e-07
484 6.59670604363782e-07
485 6.5939826754402e-07
486 6.58044939427782e-07
487 6.57807618154038e-07
488 6.57465989206685e-07
489 6.56983502267394e-07
490 6.56501867979387e-07
491 6.55996586829133e-07
492 6.55533256121998e-07
493 6.55213739264582e-07
494 6.54710731851083e-07
495 6.54216535167507e-07
496 6.53951587992196e-07
497 6.53484278245742e-07
498 6.53008669360133e-07
499 6.52751680263464e-07
500 6.52276696655463e-07
501 6.5184121922357e-07
502 6.51570246645861e-07
503 6.51117773031729e-07
504 6.5081883349194e-07
505 6.52022890790249e-07
506 6.49879950742616e-07
507 6.49665992114024e-07
508 6.5090063117168e-07
509 6.50499202947685e-07
510 6.4851650449782e-07
511 6.49750688808126e-07
512 6.47751960514142e-07
513 6.49042988243309e-07
514 6.48573234229843e-07
515 6.4660542875572e-07
516 6.47726494662493e-07
517 6.47245883556025e-07
518 6.46842124751856e-07
519 6.45048260139447e-07
520 6.46169553419895e-07
521 6.45623003947549e-07
522 6.4509652020206e-07
523 6.44619660761236e-07
524 6.44695546725416e-07
525 6.43876830963563e-07
526 6.43419866719341e-07
527 6.43131784272555e-07
528 6.43173052594648e-07
529 6.42439999865019e-07
530 6.42147369944723e-07
531 6.41780445675977e-07
532 6.41900385289773e-07
533 6.41115093458211e-07
534 6.4080978745551e-07
535 6.40917733107926e-07
536 6.40438145182998e-07
537 6.39871302610118e-07
538 6.39449126538238e-07
539 6.39510460587189e-07
540 6.38805261132802e-07
541 6.38748360870522e-07
542 6.38595395230368e-07
543 6.37944083337061e-07
544 6.36039942492062e-07
545 6.35920059721684e-07
546 6.35439903362567e-07
547 6.35426488315716e-07
548 6.3492728941128e-07
549 6.34660409559729e-07
550 6.34151433587249e-07
551 6.34050991266122e-07
552 6.33901663604775e-07
553 6.33360059509869e-07
554 6.33181912235159e-07
555 6.32837952707632e-07
556 6.32510477771575e-07
557 6.32271166978171e-07
558 6.31998489097896e-07
559 6.3154470808513e-07
560 6.31118950877863e-07
561 6.30975591775496e-07
562 6.30681597613147e-07
563 6.30471504337038e-07
564 6.30188822015043e-07
565 6.2968001657282e-07
566 6.29548821962089e-07
567 6.29241469596309e-07
568 6.31519185390061e-07
569 6.31162606623548e-07
570 6.32410319667542e-07
571 6.30406304935605e-07
572 6.31106388482294e-07
573 6.30769079634774e-07
574 6.30071326668258e-07
575 6.29707074040198e-07
576 6.29350154213171e-07
577 6.28937186775147e-07
578 6.28617044640123e-07
579 6.28280531600467e-07
580 6.27890301529987e-07
581 6.27514793904993e-07
582 6.27131555575033e-07
583 6.26867517894425e-07
584 6.25489747108077e-07
585 6.26254404778592e-07
586 6.25649363428238e-07
587 6.25526411113242e-07
588 6.25109180418804e-07
589 6.24954225258989e-07
590 6.24496919954254e-07
591 6.24377662461484e-07
592 6.2394821043199e-07
593 6.2375812603932e-07
594 6.23469361471507e-07
595 6.23108633135416e-07
596 6.22785250925517e-07
597 6.22474374267767e-07
598 6.2215889329309e-07
599 6.21871151906817e-07
600 6.21596711880557e-07
601 6.21240303644299e-07
602 6.20777086624003e-07
603 6.20643902493612e-07
604 6.2034359871177e-07
605 6.20042044374713e-07
606 6.19535171608732e-07
607 6.19301317783538e-07
608 6.18973842847481e-07
609 6.18767558080435e-07
610 6.18440083144378e-07
611 6.18130741258938e-07
612 6.1789410210622e-07
613 6.17563102878194e-07
614 6.17292528204416e-07
615 6.1700291098532e-07
616 6.16780653217575e-07
617 6.16495356098312e-07
618 6.16232796346594e-07
619 6.15975068285479e-07
620 6.15732574260619e-07
621 6.15474277765315e-07
622 6.15192391251185e-07
623 6.15037265561114e-07
624 6.14808868704131e-07
625 6.14448822489067e-07
626 6.14328769188432e-07
627 6.13985093878e-07
628 6.13715428698924e-07
629 6.13532392890193e-07
630 6.13316728959035e-07
631 6.13079009781359e-07
632 6.12652058862295e-07
633 6.12594021731638e-07
634 6.12387736964592e-07
635 6.12399219335202e-07
636 6.11694133567653e-07
637 6.11771952208073e-07
638 6.11560210472817e-07
639 6.1152513808338e-07
640 6.11347161338927e-07
641 6.11660141203174e-07
642 6.11155314800271e-07
643 6.10916799814731e-07
644 6.11013945217564e-07
645 6.10693632552284e-07
646 6.10434710779373e-07
647 6.10246274845849e-07
648 6.10062386385835e-07
649 6.09915787208593e-07
650 6.09418464136979e-07
651 6.09765777426219e-07
652 6.09581832122785e-07
653 6.09351161529048e-07
654 6.09141125096357e-07
655 6.08950244895823e-07
656 6.085472250561e-07
657 6.08197751716943e-07
658 6.08271875535138e-07
659 6.07433378263522e-07
660 6.07862830293016e-07
661 6.0754024389098e-07
662 6.07187303103274e-07
663 6.07062190738361e-07
664 6.06850790063618e-07
665 6.06425828664214e-07
666 6.06228979904699e-07
667 6.061650879019e-07
668 6.05854040713893e-07
669 6.05838579303963e-07
670 6.05554532739916e-07
671 6.05306752277102e-07
672 6.05045670454274e-07
673 6.04837850914919e-07
674 6.04606952947506e-07
675 6.04415049565432e-07
676 6.04132083026343e-07
677 6.03769819917943e-07
678 6.03557452905079e-07
679 6.03321836933901e-07
680 6.03247144681518e-07
681 6.02994305154425e-07
682 6.0272708424236e-07
683 6.02543366312602e-07
684 6.02343277478212e-07
685 6.02184911713266e-07
686 6.01944407208066e-07
687 6.01665021804365e-07
688 6.01617728079873e-07
689 6.01219767304428e-07
690 6.00887460677768e-07
691 6.00905707415222e-07
692 6.00604494138679e-07
693 6.00356031554838e-07
694 6.00312091592059e-07
695 6.00022190155869e-07
696 5.99793168021279e-07
697 5.9959802456433e-07
698 5.99566533310281e-07
699 5.99347231400316e-07
700 5.99278905610845e-07
701 5.99060456352163e-07
702 5.98752365021937e-07
703 5.98683755015372e-07
704 5.9846331623703e-07
705 5.98331439505273e-07
706 5.98146243646624e-07
707 5.97947916958219e-07
708 5.97794894474646e-07
709 5.97568032389972e-07
710 5.97412906699901e-07
711 5.97270968683006e-07
712 5.97106122768309e-07
713 5.96924110141117e-07
714 5.9662210105671e-07
715 5.96534448504826e-07
716 5.96414622577868e-07
717 5.96107383898925e-07
718 5.9604775515254e-07
719 5.95747792431212e-07
720 5.9588558087853e-07
721 5.95528149460733e-07
722 5.95523147239874e-07
723 5.9525535789362e-07
724 5.95215624343837e-07
725 5.94825507960195e-07
726 5.94862626712711e-07
727 5.94637469930603e-07
728 5.94555558564025e-07
729 5.94398954945063e-07
730 5.94267362430401e-07
731 5.93881566146592e-07
732 5.93856384512037e-07
733 5.93667607518e-07
734 5.93539311921631e-07
735 5.93243896673812e-07
736 5.93078425481508e-07
737 5.93134586779342e-07
738 5.92958997458481e-07
739 5.93333481901936e-07
740 5.93243100865948e-07
741 5.92949277233856e-07
742 5.92959111145319e-07
743 5.92892149597901e-07
744 5.92697404044884e-07
745 5.92455364767375e-07
746 5.92424100887001e-07
747 5.92286198752845e-07
748 5.9222605841569e-07
749 5.91908417391096e-07
750 5.91848959174968e-07
751 5.91848504427617e-07
752 5.91390858062368e-07
753 5.91429852647707e-07
754 5.91258356053004e-07
755 5.91029959196021e-07
756 5.90956858559366e-07
757 5.90754041240871e-07
758 5.90493243635137e-07
759 5.90501429087453e-07
760 5.90355682561494e-07
761 5.90161960190017e-07
762 5.89893147662224e-07
763 5.89851254062523e-07
764 5.89754847624135e-07
765 5.89574028708739e-07
766 5.89307092013769e-07
767 5.88996556416532e-07
768 5.89002411288675e-07
769 5.88712566695904e-07
770 5.8877265018964e-07
771 5.8864389984592e-07
772 5.88516684274509e-07
773 5.88221496400365e-07
774 5.88020327541017e-07
775 5.88047896599164e-07
776 5.87884869673871e-07
777 5.87679551244946e-07
778 5.8722577023218e-07
779 5.87009651553672e-07
780 5.86826899962034e-07
781 5.8662902802098e-07
782 5.8636504718379e-07
783 5.86191674756265e-07
784 5.86051783102448e-07
785 5.85859197599348e-07
786 5.85734937885718e-07
787 5.85549116749462e-07
788 5.8545043657432e-07
789 5.8527427881927e-07
790 5.85170653266687e-07
791 5.85088514526433e-07
792 5.84846191031829e-07
793 5.84664689995407e-07
794 5.84579993301304e-07
795 5.84283213811432e-07
796 5.84113990953483e-07
797 5.83931466735521e-07
798 5.83757753247482e-07
799 5.83531175379903e-07
800 5.83447615554178e-07
801 5.83348423788266e-07
802 5.83087285122019e-07
803 5.82979055252508e-07
804 5.8283575299356e-07
805 5.82582060815184e-07
806 5.82539769311552e-07
807 5.82286759254202e-07
808 5.82203995236341e-07
809 5.82098152790422e-07
810 5.81916026476392e-07
811 5.81640335894917e-07
812 5.81447181957628e-07
813 5.81409494770924e-07
814 5.81189738113608e-07
815 5.81240101382718e-07
816 5.81080541905976e-07
817 5.80977484787581e-07
818 5.80631194679881e-07
819 5.80631365210138e-07
820 5.80515063575149e-07
821 5.80391088078613e-07
822 5.80244318371115e-07
823 5.8011283954329e-07
824 5.80056280341523e-07
825 5.79915479193005e-07
826 5.79844709136523e-07
827 5.79836182623694e-07
828 5.79488300900266e-07
829 5.79433674374741e-07
830 5.79320669658046e-07
831 5.79192374061677e-07
832 5.79054812988034e-07
833 5.78886158564274e-07
834 5.78765877889964e-07
835 5.78607796342112e-07
836 5.78466256229149e-07
837 5.78288506858371e-07
838 5.78147705709853e-07
839 5.77914704535942e-07
840 5.77837909077061e-07
841 5.77751734454068e-07
842 5.77541186430608e-07
843 5.77383673316945e-07
844 5.77222635911312e-07
845 5.7707603673407e-07
846 5.77010325741867e-07
847 5.7684599141794e-07
848 5.76777551941632e-07
849 5.76596960399911e-07
850 5.76569163968088e-07
851 5.76450474909507e-07
852 5.76283468944894e-07
853 5.76215256842261e-07
854 5.76165348320501e-07
855 5.76091679249657e-07
856 5.75992828544258e-07
857 5.75921490053588e-07
858 5.75823150938959e-07
859 5.75850833683944e-07
860 5.75673766434193e-07
861 5.75556725834758e-07
862 5.75454237150552e-07
863 5.75314743400668e-07
864 5.75199635477475e-07
865 5.75039791783638e-07
866 5.75033482164145e-07
867 5.74883699755446e-07
868 5.74740056435985e-07
869 5.74566513478203e-07
870 5.74608634451579e-07
871 5.74399791730684e-07
872 5.74291334487498e-07
873 5.73938848447142e-07
874 5.74760406379937e-07
875 5.74689806853712e-07
876 5.7445640777587e-07
877 5.74297018829384e-07
878 5.7429622302152e-07
879 5.7407169151702e-07
880 5.73915713175666e-07
881 5.73811973936245e-07
882 5.73880925003323e-07
883 5.73503257328412e-07
884 5.73323177377461e-07
885 5.7328327329742e-07
886 5.73158331462764e-07
887 5.7300434264107e-07
888 5.73054876440438e-07
889 5.72743033444567e-07
890 5.72749513594317e-07
891 5.72675332932704e-07
892 5.72499175177654e-07
893 5.72533167542133e-07
894 5.72201543036499e-07
895 5.72212854876852e-07
896 5.72155897771154e-07
897 5.71915791169886e-07
898 5.71812961425167e-07
899 5.71749069422367e-07
900 5.71741225030564e-07
901 5.71565522022865e-07
902 5.71457974274381e-07
903 5.71548014249856e-07
904 5.71273346849921e-07
905 5.71148120798171e-07
906 5.71005500660249e-07
907 5.70824909118528e-07
908 5.70780173347885e-07
909 5.70738393435022e-07
910 5.70573661207163e-07
911 5.70557006085437e-07
912 5.70345093819924e-07
913 5.70334236726922e-07
914 5.70062923088699e-07
915 5.69727262700326e-07
916 5.6982736396094e-07
917 5.69743178857607e-07
918 5.69951509987732e-07
919 5.6976796258823e-07
920 5.69766655189596e-07
921 5.69444921438844e-07
922 5.6939870773931e-07
923 5.69414339679497e-07
924 5.68680320611747e-07
925 5.69195321986626e-07
926 5.69133703720581e-07
927 5.68453515370493e-07
928 5.68841755921312e-07
929 5.68714483506483e-07
930 5.68181917515176e-07
931 5.68457494409813e-07
932 5.68024972835701e-07
933 5.68261668831838e-07
934 5.67837730613974e-07
935 5.68126154121273e-07
936 5.67539700568886e-07
937 5.67876270451961e-07
938 5.6725610875219e-07
939 5.67744507407042e-07
940 5.67003723972448e-07
941 5.67543224860856e-07
942 5.67009010410402e-07
943 5.6725309605099e-07
944 5.66791186429327e-07
945 5.66756170883309e-07
946 5.66663175050053e-07
947 5.66955748126929e-07
948 5.66358494324959e-07
949 5.66299263482506e-07
950 5.66179039651615e-07
951 5.66615483421629e-07
952 5.65943878427788e-07
953 5.65943253150181e-07
954 5.66278743008297e-07
955 5.65680579711625e-07
956 5.6612191201566e-07
957 5.65522270790098e-07
958 5.65498396554176e-07
959 5.65843947697431e-07
960 5.65308255318087e-07
961 5.6528011782575e-07
962 5.65584116429818e-07
963 5.65050072509621e-07
964 5.6495582612115e-07
965 5.64875108466367e-07
966 5.65243624350842e-07
967 5.64882839171332e-07
968 5.64565425520414e-07
969 5.64315087103751e-07
970 5.64583444884192e-07
971 5.64078789011546e-07
972 5.64011827464128e-07
973 5.64442416361999e-07
974 5.63758362659428e-07
975 5.6382509683317e-07
976 5.64192021101917e-07
977 5.63577543744032e-07
978 5.63522860375087e-07
979 5.64155755000684e-07
980 5.63291962407675e-07
981 5.63227217753592e-07
982 5.63337550829601e-07
983 5.6312353535759e-07
984 5.63340904591314e-07
985 5.62987395369419e-07
986 5.63238359063689e-07
987 5.62949310278782e-07
988 5.62952550353657e-07
989 5.62745071874815e-07
990 5.62789978175715e-07
991 5.62636614631629e-07
992 5.62693173833395e-07
993 5.62437037388008e-07
994 5.62487628030794e-07
995 5.62522586733394e-07
996 5.62269406145788e-07
997 5.62440391149721e-07
998 5.62039474516496e-07
999 5.62049649488472e-07
1000 5.62327898023796e-07
1001 5.61948695576575e-07
1002 5.62236323276011e-07
1003 5.6231272083096e-07
1004 5.62121158509399e-07
1005 5.62078412258415e-07
1006 5.62027196338022e-07
1007 5.62081424959615e-07
1008 5.61771059892635e-07
1009 5.61776460017427e-07
1010 5.61645720154047e-07
1011 5.61646629648749e-07
1012 5.61444892355212e-07
1013 5.61618833216926e-07
1014 5.61332967663475e-07
1015 5.61311878755077e-07
1016 5.61199897219922e-07
1017 5.61132253551477e-07
1018 5.61223657769006e-07
1019 5.60880721423018e-07
1020 5.60855426101625e-07
1021 5.60699561447109e-07
1022 5.60838998353574e-07
1023 5.60667388072034e-07
1024 5.60778516955907e-07
1025 5.60423757178796e-07
1026 5.60541707272932e-07
1027 5.60333603516483e-07
1028 5.60415912786993e-07
1029 5.60267608307186e-07
1030 5.60274088456936e-07
1031 5.59966167656967e-07
1032 5.59965599222778e-07
1033 5.5987470659602e-07
1034 5.59819511636306e-07
1035 5.59857767257199e-07
1036 5.59605268790619e-07
1037 5.59757495466329e-07
1038 5.59318152681954e-07
1039 5.59383067866293e-07
1040 5.59297120616975e-07
1041 5.59250111109577e-07
1042 5.59283591883286e-07
1043 5.59262502974889e-07
1044 5.59039449399279e-07
1045 5.58947760964656e-07
1046 5.58852150334133e-07
1047 5.58800991257158e-07
1048 5.58712031306641e-07
1049 5.58629892566387e-07
1050 5.58548606477416e-07
1051 5.58476699552557e-07
1052 5.58409283257788e-07
1053 5.58296619601606e-07
1054 5.58388308036228e-07
1055 5.58459021249291e-07
1056 5.58161616481812e-07
1057 5.58302588160586e-07
1058 5.58039175757585e-07
1059 5.58153146812401e-07
1060 5.5791969089114e-07
1061 5.57840451165248e-07
1062 5.5775001328584e-07
1063 5.578288551078e-07
1064 5.57768089493038e-07
1065 5.57841985937557e-07
1066 5.57548389679141e-07
1067 5.5745113058947e-07
1068 5.57477562779241e-07
1069 5.57445105187071e-07
1070 5.57213923002564e-07
1071 5.57275200208096e-07
1072 5.57214320906496e-07
1073 5.57006899271073e-07
1074 5.57029409264942e-07
1075 5.56877637336584e-07
1076 5.5678549415461e-07
1077 5.56933628104161e-07
1078 5.5649110208833e-07
1079 5.56765940018522e-07
1080 5.56384975425317e-07
1081 5.5659290865151e-07
1082 5.56446309474268e-07
1083 5.5635638318563e-07
1084 5.56294367015653e-07
1085 5.56184886590927e-07
1086 5.56241502636112e-07
1087 5.56167378817918e-07
1088 5.5607046078876e-07
1089 5.56016345854005e-07
1090 5.55972917481995e-07
1091 5.55868666651804e-07
1092 5.55794429146772e-07
1093 5.55761971554602e-07
1094 5.55735084617481e-07
1095 5.55589508621779e-07
1096 5.55628332676861e-07
1097 5.55451890704717e-07
1098 5.55484064079792e-07
1099 5.55284202619077e-07
1100 5.55341898689221e-07
1101 5.55159601844935e-07
1102 5.55160966086987e-07
1103 5.55060125861928e-07
1104 5.55072176666727e-07
1105 5.54883740733203e-07
1106 5.54891983028938e-07
1107 5.54759083115641e-07
1108 5.54750499759393e-07
1109 5.54626183202345e-07
1110 5.54695475329936e-07
1111 5.5438840718125e-07
1112 5.54283701603708e-07
1113 5.54192922663788e-07
1114 5.54094071958389e-07
1115 5.54026939880714e-07
1116 5.54149437448359e-07
1117 5.53880965981079e-07
1118 5.53999882413336e-07
1119 5.53749146092741e-07
1120 5.53860729723965e-07
1121 5.53736526853754e-07
1122 5.53434233552252e-07
1123 5.53385802959383e-07
1124 5.53532515823463e-07
1125 5.53507106815232e-07
1126 5.53170195871644e-07
1127 5.54562461729802e-07
1128 5.53301219952118e-07
1129 5.52984488422226e-07
1130 5.54953487608145e-07
1131 5.54384826045862e-07
1132 5.52592780422856e-07
1133 5.52497510852845e-07
1134 5.52637231976405e-07
1135 5.52627909655712e-07
1136 5.52814526599832e-07
1137 5.52772462469875e-07
1138 5.52656501895399e-07
1139 5.52667586362077e-07
1140 5.52475000858976e-07
1141 5.5259755527004e-07
1142 5.5236154139493e-07
1143 5.54062751234596e-07
1144 5.54685925635567e-07
1145 5.54716393708077e-07
1146 5.51839548279531e-07
1147 5.53834695438127e-07
1148 5.518780312741e-07
1149 5.52175094981067e-07
1150 5.53691734239692e-07
1151 5.51586254005088e-07
1152 5.51709604224015e-07
1153 5.5145540045487e-07
1154 5.51296579942573e-07
1155 5.53362554001069e-07
1156 5.51111668301019e-07
1157 5.51206767340773e-07
1158 5.51122639080859e-07
1159 5.51072901089356e-07
1160 5.53369034150819e-07
1161 5.56741952095763e-07
1162 5.56996269551746e-07
1163 5.56844611310225e-07
1164 5.55987355710386e-07
1165 5.52574022094632e-07
1166 5.52798212538619e-07
1167 5.55685517156235e-07
1168 5.5438539448005e-07
1169 5.55603492102819e-07
1170 5.52334427084133e-07
1171 5.52233871076169e-07
1172 5.56309942112421e-07
1173 5.51936693682364e-07
1174 5.52309586510091e-07
1175 5.56176644295192e-07
1176 5.51776338397758e-07
1177 5.52036453882465e-07
1178 5.55059841644834e-07
1179 5.51439768514683e-07
1180 5.51187042674428e-07
1181 5.51817493033013e-07
1182 5.51541404547606e-07
1183 5.5138355037343e-07
1184 5.51623941191792e-07
1185 5.51431071471598e-07
1186 5.51360756162467e-07
1187 5.51342338894756e-07
1188 5.51302150597621e-07
1189 5.51348648514249e-07
1190 5.51180164620746e-07
1191 5.51298285245139e-07
1192 5.51074833765597e-07
1193 5.53531776859018e-07
1194 5.54342250325135e-07
1195 5.50675679278356e-07
1196 5.50459390069591e-07
1197 5.5618221495024e-07
1198 5.50247079900146e-07
1199 5.50833078705182e-07
1200 5.5449271485486e-07
1201 5.50605250282388e-07
1202 5.50335641946731e-07
1203 5.54076791559055e-07
1204 5.50160450529802e-07
1205 5.52420601707126e-07
1206 5.54401765384682e-07
1207 5.54840880795382e-07
1208 5.60113505798654e-07
1209 5.54287908016704e-07
1210 5.59998852622812e-07
1211 5.54786140583019e-07
1212 5.54895279947232e-07
1213 5.60285911888059e-07
1214 5.54507948891114e-07
1215 5.5476112947872e-07
1216 5.54524092422071e-07
1217 5.59361069463193e-07
1218 5.54273071884381e-07
1219 5.57622229280241e-07
1220 5.54609982827969e-07
1221 5.5756493111403e-07
1222 5.59998511562299e-07
1223 5.55260385226575e-07
1224 5.53460608898604e-07
1225 5.53566962935292e-07
1226 5.59539159894484e-07
1227 5.5329167025775e-07
1228 5.59850775516679e-07
1229 5.53804909486644e-07
1230 5.56530324047344e-07
1231 5.59579007131106e-07
1232 5.53679456061218e-07
1233 5.56555278308224e-07
1234 5.58774843284482e-07
1235 5.53203278741421e-07
1236 5.52445783341682e-07
1237 5.59025636448496e-07
1238 5.52643939499831e-07
1239 5.56332167889195e-07
1240 5.60772775770602e-07
1241 5.52324365798995e-07
1242 5.52788719687669e-07
1243 5.58715100851259e-07
1244 5.5232959539353e-07
1245 5.55641747723712e-07
1246 5.59422062451631e-07
1247 5.52230631001294e-07
1248 5.52144854282233e-07
1249 5.58054239263583e-07
1250 5.54277562514471e-07
1251 5.51853531760571e-07
1252 5.58306794573582e-07
1253 5.51807147530781e-07
1254 5.52910762507963e-07
1255 5.51259574876894e-07
1256 5.55123961021309e-07
1257 5.58230738079146e-07
1258 5.5878291504996e-07
1259 5.51227685718914e-07
1260 5.58208228085277e-07
1261 5.50853201275459e-07
1262 5.5470076176789e-07
1263 5.57640419174277e-07
1264 5.57513374133123e-07
1265 5.50627703432838e-07
1266 5.54343387193512e-07
1267 5.57881492113665e-07
1268 5.57308112547616e-07
1269 5.50659137843468e-07
1270 5.54138068764587e-07
1271 5.57869554995705e-07
1272 5.50140498489782e-07
1273 5.54619305148663e-07
1274 5.53664051494707e-07
1275 5.57698342618096e-07
1276 5.5063509307729e-07
1277 5.54223845483648e-07
1278 5.57409578050283e-07
1279 5.56375653104624e-07
1280 5.50191657566756e-07
1281 5.53499148736591e-07
1282 5.57734892936423e-07
1283 5.49325534393574e-07
1284 5.52491883354378e-07
1285 5.56814029550878e-07
1286 5.51206198906584e-07
1287 5.57138378098898e-07
1288 5.49338551536493e-07
1289 5.52988353774708e-07
1290 5.56883662738983e-07
1291 5.49513288206072e-07
1292 5.53639836198272e-07
1293 5.56071825030813e-07
1294 5.49608444089245e-07
1295 5.52754102045583e-07
1296 5.56039253751806e-07
1297 5.49199171473447e-07
1298 5.56443353616487e-07
1299 5.48363118468842e-07
1300 5.56030386178463e-07
1301 5.51951586658106e-07
1302 5.56212626179331e-07
1303 5.52157416677801e-07
1304 5.56417830921418e-07
1305 5.51793107206322e-07
1306 5.56181191768701e-07
1307 5.55295798676525e-07
1308 5.51449943486659e-07
1309 5.55873270968732e-07
1310 5.49992307696812e-07
1311 5.55647886812949e-07
1312 5.48879256712098e-07
1313 5.521025627786e-07
1314 5.56024247089226e-07
1315 5.55421991066396e-07
1316 5.5179071978273e-07
1317 5.55570863980392e-07
1318 5.55938697743841e-07
1319 5.4852887387824e-07
1320 5.480209210873e-07
1321 5.54650114281685e-07
1322 5.55652093225945e-07
1323 5.5484645145043e-07
1324 5.47702143194329e-07
1325 5.55288465875492e-07
1326 5.4785925840406e-07
1327 5.55772260213416e-07
1328 5.53753523035994e-07
1329 5.55704673388391e-07
1330 5.49690241768985e-07
1331 5.54880500658328e-07
1332 5.54132782326633e-07
1333 5.55154201720143e-07
1334 5.50406866750563e-07
1335 5.55368444565829e-07
1336 5.54580083189649e-07
1337 5.54295354504575e-07
1338 5.48735727079475e-07
1339 5.53631195998605e-07
1340 5.52129563402559e-07
1341 5.4489680678671e-07
1342 5.48830030311365e-07
1343 5.52946062271076e-07
1344 5.47131435268966e-07
1345 5.52686458377138e-07
1346 5.48147568224522e-07
1347 5.53204870357149e-07
1348 5.50737297544401e-07
1349 5.52550773136318e-07
1350 5.46355465758097e-07
1351 5.51519633518183e-07
1352 5.46565729564463e-07
1353 5.51855691810488e-07
1354 5.45953582786751e-07
1355 5.51211201127444e-07
1356 5.51498146705853e-07
1357 5.46586704786023e-07
1358 5.44425972748286e-07
1359 5.51447385532811e-07
1360 5.51455627828545e-07
1361 5.46195281003747e-07
1362 5.51093023659632e-07
1363 5.5128225540102e-07
1364 5.50905724594486e-07
1365 5.46010198831937e-07
1366 5.51944992821518e-07
1367 5.5102526630435e-07
1368 5.4657135706293e-07
1369 5.51151572381059e-07
1370 5.45953241726238e-07
1371 5.51682603600057e-07
1372 5.45627415249328e-07
1373 5.51825678485329e-07
1374 5.50419144929037e-07
1375 5.45646628324903e-07
1376 5.51153789274395e-07
1377 5.45590637557325e-07
1378 5.5039305379978e-07
1379 5.46456817573926e-07
1380 5.51121388525644e-07
1381 5.45763782611175e-07
1382 5.50596269022208e-07
1383 5.46003491308511e-07
1384 5.50527659015643e-07
1385 5.46096771358862e-07
1386 5.50899471818411e-07
1387 5.45397426776617e-07
1388 5.49757771750592e-07
1389 5.45102011528797e-07
1390 5.50728259440803e-07
1391 5.44746569630661e-07
1392 5.51189657471696e-07
1393 5.49959224827035e-07
1394 5.44523516055051e-07
1395 5.49074002265115e-07
1396 5.44741226349288e-07
1397 5.49779258562921e-07
1398 5.50676531929639e-07
1399 5.49603839772317e-07
1400 5.49843605313072e-07
1401 5.44186377737788e-07
1402 5.49139940630994e-07
1403 5.49631295143627e-07
1404 5.49633227819868e-07
1405 5.48624598195602e-07
1406 5.48840944247786e-07
1407 5.43726400792366e-07
1408 5.49098047031293e-07
1409 5.52221877114789e-07
1410 5.43015005405323e-07
1411 5.49530398075149e-07
1412 5.43957014542684e-07
1413 5.48878119843721e-07
1414 5.51041409835307e-07
1415 5.48036837244581e-07
1416 5.48871639693971e-07
1417 5.47786271454243e-07
1418 5.47828733488132e-07
1419 5.49257094917266e-07
1420 5.4148978279045e-07
1421 5.49025003238057e-07
1422 5.48541891021159e-07
1423 5.46375815702049e-07
1424 5.42191173735773e-07
1425 5.46752687569096e-07
1426 5.42005807346868e-07
1427 5.45988257272256e-07
1428 5.45861666978453e-07
1429 5.4529499493583e-07
1430 5.45963416698214e-07
1431 5.4129066029418e-07
1432 5.4539430038858e-07
1433 5.45333307400142e-07
1434 5.45026807685645e-07
1435 5.4548058869841e-07
1436 5.44596275631193e-07
1437 5.45936075013742e-07
1438 5.40557948625064e-07
1439 5.45456941836164e-07
1440 5.47779336557142e-07
1441 5.40061250831059e-07
1442 5.45449381661456e-07
1443 5.45027489806671e-07
1444 5.44868726137793e-07
1445 5.45261798379215e-07
1446 5.43083160664537e-07
1447 5.40202677257184e-07
1448 5.4506745073013e-07
1449 5.44989688933128e-07
1450 5.44411648206733e-07
1451 5.48053662896564e-07
1452 5.44378735867213e-07
1453 5.4687075135007e-07
1454 5.44128056390036e-07
1455 5.45517082173319e-07
1456 5.45153568509704e-07
1457 5.44312683814496e-07
1458 5.47962315522454e-07
1459 5.43742942227254e-07
1460 5.44469628493971e-07
1461 5.44027898286004e-07
1462 5.47333854683529e-07
1463 5.45248212802107e-07
1464 5.42070154097019e-07
1465 5.43886926607229e-07
1466 5.43583553280769e-07
1467 5.44124816315161e-07
1468 5.43650912732119e-07
1469 5.4429204965345e-07
1470 5.42106477041671e-07
1471 5.39000723165373e-07
1472 5.4376761227104e-07
1473 5.43715600542782e-07
1474 5.43311443834682e-07
1475 5.44570241345355e-07
1476 5.43344356174202e-07
1477 5.43739758995798e-07
1478 5.43178430234548e-07
1479 5.45581599453726e-07
1480 5.42523366675596e-07
1481 5.43087310234114e-07
1482 5.43226462923485e-07
1483 5.43352427939681e-07
1484 5.42445434348338e-07
1485 5.44139936664578e-07
1486 5.43019950782764e-07
1487 5.42644784218282e-07
1488 5.4239239943854e-07
1489 5.4503453839061e-07
1490 5.41929694009013e-07
1491 5.43801377261843e-07
1492 5.42298323580326e-07
1493 5.43105500128149e-07
1494 5.42363807198853e-07
1495 5.42279622095521e-07
1496 5.42570660400088e-07
1497 5.425381459645e-07
1498 5.42394445801619e-07
1499 5.42323334684625e-07
1500 5.45925274764159e-07
1501 5.4284350881062e-07
1502 5.4227217560765e-07
1503 5.42550708360068e-07
1504 5.42561679139908e-07
1505 5.42183613561065e-07
1506 5.42560655958368e-07
1507 5.42094483080291e-07
1508 5.44695808457618e-07
1509 5.41909059847967e-07
1510 5.44526812973345e-07
1511 5.41986025837105e-07
1512 5.44580700534425e-07
1513 5.41709653134603e-07
1514 5.44500494470412e-07
1515 5.42867610420217e-07
1516 5.44138458735688e-07
1517 5.41551969490683e-07
1518 5.44187344075908e-07
1519 5.41649058050098e-07
1520 5.44198940133356e-07
1521 5.42424515970197e-07
1522 5.43993735391268e-07
1523 5.41417193744564e-07
1524 5.43967416888336e-07
1525 5.42128248071094e-07
1526 5.44420458936656e-07
1527 5.41245867680118e-07
1528 5.44126976365078e-07
1529 5.41143151622236e-07
1530 5.43888006632187e-07
1531 5.41955103017244e-07
1532 5.41082329164055e-07
1533 5.42945031156705e-07
1534 5.42989084806322e-07
1535 5.43017733889428e-07
1536 5.43816099707328e-07
1537 5.34691196207859e-07
1538 5.35232288711995e-07
1539 5.4130015314513e-07
1540 5.43244993878034e-07
1541 5.34651690031751e-07
1542 5.4194435961108e-07
1543 5.40829887540895e-07
1544 5.41864324077324e-07
1545 5.43624537385767e-07
1546 5.42245345513948e-07
1547 5.42116367796552e-07
1548 5.34160051302024e-07
1549 5.41173108103976e-07
1550 5.42598286301654e-07
1551 5.39698703505564e-07
1552 5.39673862931522e-07
1553 5.42870736808254e-07
1554 5.31654904989409e-07
1555 5.41565441380953e-07
1556 5.41863244052365e-07
1557 5.43249313977867e-07
1558 5.41721362878889e-07
1559 5.33666650426312e-07
1560 5.40696532880247e-07
1561 5.42864654562436e-07
1562 5.41593919933803e-07
1563 5.33316040218779e-07
1564 5.33305865246803e-07
1565 5.34662945028685e-07
1566 5.40985922725667e-07
1567 5.41679639809445e-07
1568 5.33378056388756e-07
1569 5.34101275206922e-07
1570 5.369245741349e-07
1571 5.41210567917005e-07
1572 5.3309145187086e-07
1573 5.40376447588642e-07
1574 5.41741940196516e-07
1575 5.32934961938736e-07
1576 5.33501179234008e-07
1577 5.40536802873248e-07
1578 5.41554982191883e-07
1579 5.32690421550797e-07
1580 5.37482719664695e-07
1581 5.3560228252536e-07
1582 5.41299073120172e-07
1583 5.32904664396483e-07
1584 5.3722249049315e-07
1585 5.36099719283811e-07
1586 5.40739222287812e-07
1587 5.33136983449367e-07
1588 5.40876669674617e-07
1589 5.39963025403267e-07
1590 5.32662625118974e-07
1591 5.38084350409918e-07
1592 5.40150153938157e-07
1593 5.39545453648316e-07
1594 5.40691530659387e-07
1595 5.387648229771e-07
1596 5.39767825102899e-07
1597 5.38058088750404e-07
1598 5.41016845545528e-07
1599 5.31131490788539e-07
1600 5.31619775756553e-07
1601 5.39959728484973e-07
1602 5.39264647159143e-07
1603 5.31255921032425e-07
1604 5.33644765710051e-07
1605 5.35122069322824e-07
1606 5.40198527687608e-07
1607 5.31860848695942e-07
1608 5.34050116129947e-07
1609 5.31023204075609e-07
1610 5.41456529390416e-07
1611 5.40443693353154e-07
1612 5.32048943568952e-07
1613 5.36044581167516e-07
1614 5.36835727871221e-07
1615 5.41006613730133e-07
1616 5.40130884019163e-07
1617 5.33048137185688e-07
1618 5.30314935076603e-07
1619 5.42151440185989e-07
1620 5.40819314664986e-07
1621 5.32111130269186e-07
1622 5.35137303359079e-07
1623 5.33250897660764e-07
1624 5.39037387170538e-07
1625 5.40142764293705e-07
1626 5.32012791154557e-07
1627 5.31575437889842e-07
1628 5.36683501195512e-07
1629 5.36796619599045e-07
1630 5.40993084996444e-07
1631 5.30787190200499e-07
1632 5.35720516836591e-07
1633 5.40705684670684e-07
1634 5.30796341990936e-07
1635 5.31281330040656e-07
1636 5.32274725628668e-07
1637 5.37650578280591e-07
1638 5.39340305749647e-07
1639 5.31279454207834e-07
1640 5.33601848928811e-07
1641 5.39212180683535e-07
1642 5.4022535778131e-07
1643 5.30558054379071e-07
1644 5.30063459791563e-07
1645 5.40389578418399e-07
1646 5.40088421985274e-07
1647 5.3224266594043e-07
1648 5.31803891590243e-07
1649 5.39485483841418e-07
1650 5.32450997070555e-07
1651 5.36578397714038e-07
1652 5.39861275683506e-07
1653 5.39025961643347e-07
1654 5.31079365373444e-07
1655 5.37011260348663e-07
1656 5.34967796284036e-07
1657 5.4023934126235e-07
1658 5.39229176865774e-07
1659 5.39417385425622e-07
1660 5.31377168044855e-07
1661 5.31638988832128e-07
1662 5.36076186108403e-07
1663 5.40187102160417e-07
1664 5.30306579094031e-07
1665 5.40157941486541e-07
1666 5.39402265076205e-07
1667 5.32189744717471e-07
1668 5.36294066932896e-07
1669 5.38893345947145e-07
1670 5.31698105987743e-07
1671 5.36320499122667e-07
1672 5.39382540409861e-07
1673 5.31328851138824e-07
1674 5.30458976299997e-07
1675 5.35927881628595e-07
1676 5.34546074959508e-07
1677 5.3546057188214e-07
1678 5.33779257239075e-07
1679 5.34716548372671e-07
1680 5.38587016762904e-07
1681 5.38046549536375e-07
1682 5.29987858044478e-07
1683 5.30412137322855e-07
1684 5.35579943061748e-07
1685 5.32903129624174e-07
1686 5.38232200142374e-07
1687 5.31557475369482e-07
1688 5.29825570083631e-07
1689 5.3780331654707e-07
1690 5.37943265044305e-07
1691 5.37862604232942e-07
1692 5.30455452008027e-07
1693 5.30118825281534e-07
1694 5.35126105205563e-07
1695 5.338239361663e-07
1696 5.36098582415434e-07
1697 5.37652908860764e-07
1698 5.30214038008125e-07
1699 5.35218873665144e-07
1700 5.3606487426805e-07
1701 5.39160680546047e-07
1702 5.36903598913341e-07
1703 5.30251099917223e-07
1704 5.38207530098589e-07
1705 5.3784668807566e-07
1706 5.37110054210643e-07
1707 5.30348245320056e-07
1708 5.28777661656932e-07
1709 5.35006165591767e-07
1710 5.33207412445336e-07
1711 5.33232309862797e-07
1712 5.36679920060124e-07
1713 5.37413995971292e-07
1714 5.37726293714513e-07
1715 5.37150356194616e-07
1716 5.29316537267732e-07
1717 5.33046488726541e-07
1718 5.28853831838205e-07
1719 5.38570532171434e-07
1720 5.29060116605251e-07
1721 5.3785720410815e-07
1722 5.29633950918651e-07
1723 5.29366332102654e-07
1724 5.34360253823252e-07
1725 5.34473258539947e-07
1726 5.36109439508436e-07
1727 5.37881248874328e-07
1728 5.36019229002704e-07
1729 5.29553346950706e-07
1730 5.28278121691983e-07
1731 5.34453079126251e-07
1732 5.33604975316848e-07
1733 5.37030700797914e-07
1734 5.36453114818869e-07
1735 5.31966918515536e-07
1736 5.25792586358875e-07
1737 5.32839294464793e-07
1738 5.2798037586399e-07
1739 5.33936940882995e-07
1740 5.36383993221534e-07
1741 5.38123799742607e-07
1742 5.37166442882153e-07
1743 5.3568265911963e-07
1744 5.35755532382609e-07
1745 5.27878455613973e-07
1746 5.35781794042123e-07
1747 5.35535775725293e-07
1748 5.31675937054388e-07
1749 5.27552401763387e-07
1750 5.37088794771989e-07
1751 5.36386323801707e-07
1752 5.37777964382258e-07
1753 5.35177491656214e-07
1754 5.36658319560956e-07
1755 5.34852688360843e-07
1756 5.31265072822862e-07
1757 5.27177633102838e-07
1758 5.33535228441906e-07
1759 5.36025083874847e-07
1760 5.28901182406116e-07
1761 5.27549445905606e-07
1762 5.35746949026361e-07
1763 5.3645607067665e-07
1764 5.36880634172121e-07
1765 5.37687242285756e-07
1766 5.37072423867357e-07
1767 5.37344647000282e-07
1768 5.37417463419843e-07
1769 5.3627093166142e-07
1770 5.37302980774257e-07
1771 5.3744605565953e-07
1772 5.3675194067182e-07
1773 5.26279222867743e-07
1774 5.36735058176419e-07
1775 5.37736127625976e-07
1776 5.36431912223634e-07
1777 5.37560140401183e-07
1778 5.37587823146168e-07
1779 5.3816756917513e-07
1780 5.3418420975504e-07
1781 5.37335210992751e-07
1782 5.26917176557617e-07
1783 5.36714424015372e-07
1784 5.28411817413144e-07
1785 5.37475500550499e-07
1786 5.37166954472923e-07
1787 5.36814013685216e-07
1788 5.37378639364761e-07
1789 5.32530691543798e-07
1790 5.36904849468556e-07
1791 5.34178241196059e-07
1792 5.3007659062132e-07
1793 5.27738507116737e-07
1794 5.38125334514916e-07
1795 5.37038317816041e-07
1796 5.37331686700782e-07
1797 5.37553603408014e-07
1798 5.36621655555791e-07
1799 5.37721177806816e-07
1800 5.34069499735779e-07
1801 5.36744323653693e-07
1802 5.36539232598443e-07
1803 5.39897769158415e-07
1804 5.39794257292669e-07
1805 5.37317532689485e-07
1806 5.37082996743266e-07
1807 5.37861581051402e-07
1808 5.3678189715356e-07
1809 5.37132223143999e-07
1810 5.3616213335772e-07
1811 5.36165032372082e-07
1812 5.37589187388221e-07
1813 5.36377172011271e-07
1814 5.39508164365543e-07
1815 5.36975903742132e-07
1816 5.36682989604742e-07
1817 5.36853349331068e-07
1818 5.35694425707334e-07
1819 5.36011611984577e-07
1820 5.35890762876079e-07
1821 5.3709476333097e-07
1822 5.38842755304358e-07
1823 5.36849711352261e-07
1824 5.36123593519733e-07
1825 5.26185260696366e-07
1826 5.31577313722664e-07
1827 5.36167704012769e-07
1828 5.34972741661477e-07
1829 5.29105477653502e-07
1830 5.29719727637712e-07
1831 5.28334510363493e-07
1832 5.36550714969053e-07
1833 5.38542735739611e-07
1834 5.34680111741181e-07
1835 5.36470679435297e-07
1836 5.38230551683228e-07
1837 5.3858121873418e-07
1838 5.36410595941561e-07
1839 5.29207966337708e-07
1840 5.35836875314999e-07
1841 5.32918988938036e-07
1842 5.34849164068874e-07
1843 5.35470235263347e-07
1844 5.29298802121048e-07
1845 5.31733462594275e-07
1846 5.32191450020036e-07
1847 5.28591783677257e-07
1848 5.34805167262675e-07
1849 5.37852486104384e-07
1850 5.3605486982633e-07
1851 5.37793880539539e-07
1852 5.35656340616697e-07
1853 5.37525181698584e-07
1854 5.34067396529281e-07
1855 5.35664128165081e-07
1856 5.35126048362145e-07
1857 5.34241848981765e-07
1858 5.25892971836583e-07
1859 5.35870924522897e-07
1860 5.34071887159371e-07
1861 5.36189304511936e-07
1862 5.26354483554314e-07
1863 5.26963788161083e-07
1864 5.35329036210896e-07
1865 5.27186159615667e-07
1866 5.28977977864997e-07
1867 5.34181140210421e-07
1868 5.34482239800127e-07
1869 5.35256049261079e-07
1870 5.34724279077636e-07
1871 5.35265087364678e-07
1872 5.33688989889924e-07
1873 5.30194597558875e-07
1874 5.29249859937408e-07
1875 5.3399207899929e-07
1876 5.34660784978769e-07
1877 5.3702837021774e-07
1878 5.31034970663313e-07
1879 5.29821477357473e-07
1880 5.33547165559867e-07
1881 5.27026145391574e-07
1882 5.23963763043866e-07
1883 5.31302134731959e-07
1884 5.32643184669723e-07
1885 5.34804541985068e-07
1886 5.34181765488029e-07
1887 5.27600491295743e-07
1888 5.34169259935879e-07
1889 5.35108085841784e-07
1890 5.32919045781455e-07
1891 5.35160779691068e-07
1892 5.2363577651704e-07
1893 5.26413998613862e-07
1894 5.34420564690663e-07
1895 5.28370946994983e-07
1896 5.26979818005202e-07
1897 5.33198488028575e-07
1898 5.27686552231899e-07
1899 5.34406467522786e-07
1900 5.33450759121479e-07
1901 5.27895508639631e-07
1902 5.24262361523142e-07
1903 5.35738081453019e-07
1904 5.34566652277135e-07
1905 5.34593027623487e-07
1906 5.32479930370755e-07
1907 5.35980916538392e-07
1908 5.32519948137633e-07
1909 5.29748035660305e-07
1910 5.32780518369691e-07
1911 5.32746696535469e-07
1912 5.3407194400279e-07
1913 5.36184643351589e-07
1914 5.32783417384053e-07
1915 5.26576059201034e-07
1916 5.33797276602854e-07
1917 5.32182014012506e-07
1918 5.37015864665591e-07
1919 5.33013746917277e-07
1920 5.35178116933821e-07
1921 5.29785893377266e-07
1922 5.31823843630264e-07
1923 5.28333430338535e-07
1924 5.33119475676358e-07
1925 5.35613480678876e-07
1926 5.26591293237288e-07
1927 5.32868739355763e-07
1928 5.33718775841407e-07
1929 5.33481511411082e-07
1930 5.32293995547661e-07
1931 5.32692297383619e-07
1932 5.3315829973144e-07
1933 5.32082253812405e-07
1934 5.36916786586517e-07
1935 5.31796445102373e-07
1936 5.25800260220421e-07
1937 5.32587591806077e-07
1938 5.3312385261961e-07
1939 5.31349428456451e-07
1940 5.33420575266064e-07
1941 5.31739942744025e-07
1942 5.32611863945931e-07
1943 5.31838907136262e-07
1944 5.32011029008572e-07
1945 5.31758814759087e-07
1946 5.32115620899276e-07
1947 5.3250562359608e-07
1948 5.31588625563018e-07
1949 5.32523245055927e-07
1950 5.31810371739994e-07
1951 5.31018997662613e-07
1952 5.25016218944074e-07
1953 5.32621584170556e-07
1954 5.32288254362356e-07
1955 5.31857551777648e-07
1956 5.30761951722525e-07
1957 5.28603266047867e-07
1958 5.32485898929735e-07
1959 5.31645127921365e-07
1960 5.34703190169239e-07
1961 5.32242211193079e-07
1962 5.31911553025566e-07
1963 5.31220223365381e-07
1964 5.20707203577331e-07
1965 5.30569934653613e-07
1966 5.32310309608874e-07
1967 5.31015245996969e-07
1968 5.3100717423149e-07
1969 5.31991133811971e-07
1970 5.30409977272939e-07
1971 5.31482783117099e-07
1972 5.30566808265576e-07
1973 5.31631883404771e-07
1974 5.31012460669444e-07
1975 5.30371607965208e-07
1976 5.31250805124728e-07
1977 5.31595105712768e-07
1978 5.19243599228503e-07
1979 5.22074685704865e-07
1980 5.31388820945722e-07
1981 5.27933877947362e-07
1982 5.31645184764784e-07
1983 5.25972325249313e-07
1984 5.31432533534826e-07
1985 5.30685554167576e-07
1986 5.30523436736985e-07
1987 5.30868703663145e-07
1988 5.30334830273205e-07
1989 5.30618365246482e-07
1990 5.29078590716381e-07
1991 5.30395823261642e-07
1992 5.24439315086056e-07
1993 5.30520765096298e-07
1994 5.3086449725015e-07
1995 5.30554643773939e-07
1996 5.29021122019913e-07
1997 5.26845894910366e-07
1998 5.29878377619752e-07
1999 5.25415714491828e-07
};
\addlegendentry{Test}

\nextgroupplot[
title={8 Layers $\rare$},
ymin=1.18077084319194e-07, ymax=1e-05,
]
\addplot [semithick, black, dashed]
table {%
0 0.0137857501469553
1 0.00308861960936338
2 0.00225937323644757
3 0.00184173461562023
4 0.000878539680619724
5 0.000267233489896171
6 0.000189525146357482
7 0.000177248529340432
8 0.000170382540301944
9 0.000163777170237154
10 0.000156803158293769
11 0.000148787730795448
12 0.000139447957270022
13 0.0001243273577129
14 0.000103127006572322
15 8.09196417794738e-05
16 6.05416532816889e-05
17 4.51169893822225e-05
18 3.565513092326e-05
19 3.03123355188291e-05
20 2.68081642607285e-05
21 2.43131255183471e-05
22 2.22995831736625e-05
23 2.05347641986009e-05
24 1.89343700831159e-05
25 1.74776983494667e-05
26 1.61251455356251e-05
27 1.48562264462271e-05
28 1.36716613037606e-05
29 1.25683485612171e-05
30 1.15524945263132e-05
31 1.06277352474535e-05
32 9.70658624783027e-06
33 8.84267294031815e-06
34 8.13857853154332e-06
35 7.58388431859203e-06
36 7.06517329012968e-06
37 6.48986769624571e-06
38 6.01962051950977e-06
39 5.62589393507551e-06
40 5.28921789339165e-06
41 4.98851873965123e-06
42 4.7175234600445e-06
43 4.47041639392864e-06
44 4.24115160376459e-06
45 4.02530521364497e-06
46 3.81863311099551e-06
47 3.62413326899969e-06
48 3.44267336078019e-06
49 3.26990401589455e-06
50 3.110896988062e-06
51 2.96396403069821e-06
52 2.82879928931834e-06
53 2.70368474275529e-06
54 2.58799466007531e-06
55 2.48464924226255e-06
56 2.39010867852585e-06
57 2.30290369552222e-06
58 2.2216036034024e-06
59 2.14735692520662e-06
60 2.07889566303265e-06
61 2.01642455289175e-06
62 1.95783670858418e-06
63 1.90416846038488e-06
64 1.85607089042605e-06
65 1.811010258848e-06
66 1.77054325018844e-06
67 1.7328962984493e-06
68 1.69725946949484e-06
69 1.66754056448326e-06
70 1.63858871133016e-06
71 1.61155683400693e-06
72 1.58789163265283e-06
73 1.56558429978304e-06
74 1.54554523152228e-06
75 1.52573656549748e-06
76 1.50784072388888e-06
77 1.49023391662695e-06
78 1.4743123068115e-06
79 1.45930793024718e-06
80 1.4446281097662e-06
81 1.43157341540245e-06
82 1.41939965010351e-06
83 1.40667576306441e-06
84 1.39518877335831e-06
85 1.38379590680415e-06
86 1.37327026897083e-06
87 1.36364018470658e-06
88 1.35418702711831e-06
89 1.34522714276386e-06
90 1.33634787925985e-06
91 1.32893943043655e-06
92 1.32093163441027e-06
93 1.3132997762284e-06
94 1.30623804824381e-06
95 1.29929187090738e-06
96 1.29237916951297e-06
97 1.28583184201148e-06
98 1.27959553714163e-06
99 1.2735283934262e-06
100 1.26791898730971e-06
101 1.2627259696103e-06
102 1.25771108588424e-06
103 1.25231368014056e-06
104 1.24631144498721e-06
105 1.24037334367699e-06
106 1.2340648443967e-06
107 1.22893064082064e-06
108 1.22437484989746e-06
109 1.21982722055236e-06
110 1.21481426256764e-06
111 1.2107418234848e-06
112 1.20602616044607e-06
113 1.20285992124991e-06
114 1.19859862485328e-06
115 1.19486205630892e-06
116 1.19111396352878e-06
117 1.18701774613328e-06
118 1.18387937891384e-06
119 1.18013333386102e-06
120 1.17652067552854e-06
121 1.17298610891226e-06
122 1.16920934786435e-06
123 1.16626488301108e-06
124 1.16285108117609e-06
125 1.15968924109211e-06
126 1.15640984193988e-06
127 1.15350303428841e-06
128 1.1503143885534e-06
129 1.14705221400868e-06
130 1.1443641856772e-06
131 1.14111493286373e-06
132 1.13798040635515e-06
133 1.13527754101028e-06
134 1.13251169233308e-06
135 1.12930856332127e-06
136 1.12650074075304e-06
137 1.1235084566863e-06
138 1.12072698141219e-06
139 1.1172398548922e-06
140 1.11442736803724e-06
141 1.11158048696325e-06
142 1.10887788622449e-06
143 1.10616437194722e-06
144 1.10341555770788e-06
145 1.10094259639482e-06
146 1.09825828815246e-06
147 1.09558834907375e-06
148 1.09296457605979e-06
149 1.09061500148755e-06
150 1.08795183314214e-06
151 1.08532106719395e-06
152 1.08324758190292e-06
153 1.08097435804666e-06
154 1.07827333013688e-06
155 1.07617651107716e-06
156 1.07380986366934e-06
157 1.07166150010585e-06
158 1.06948107162452e-06
159 1.06647956860684e-06
160 1.06494708404625e-06
161 1.06256772505731e-06
162 1.06035720727959e-06
163 1.05787372760346e-06
164 1.05588405099866e-06
165 1.05346232155057e-06
166 1.05166728118888e-06
167 1.04964814730124e-06
168 1.04769952940842e-06
169 1.04572185878737e-06
170 1.04260839512449e-06
171 1.04152885015196e-06
172 1.04000041307017e-06
173 1.03666240576672e-06
174 1.03516159231276e-06
175 1.03360050624701e-06
176 1.03188542567523e-06
177 1.02898213611979e-06
178 1.02803581077637e-06
179 1.02606050614895e-06
180 1.02454306204436e-06
181 1.02230507911827e-06
182 1.02055142943414e-06
183 1.01792596737482e-06
184 1.01660407779036e-06
185 1.01529137111811e-06
186 1.01237180382441e-06
187 1.01132905575696e-06
188 1.00956093092464e-06
189 1.00810563901632e-06
190 1.0052739773414e-06
191 1.00424275473188e-06
192 1.003211580894e-06
193 1.00131009619986e-06
194 9.98520134771752e-07
195 9.97664387540453e-07
196 9.94486084778146e-07
197 9.93778501651832e-07
198 9.90873338423626e-07
199 9.89912777669133e-07
200 9.88574339203296e-07
201 9.85725029522655e-07
202 9.84244386330602e-07
203 9.82002565308448e-07
204 9.78508168600456e-07
205 9.77044201306399e-07
206 9.76098156883154e-07
207 9.73493165162154e-07
208 9.71834136123562e-07
209 9.71276284246869e-07
210 9.68079569304336e-07
211 9.67498981026438e-07
212 9.64631265333082e-07
213 9.64166400649447e-07
214 9.61177288900217e-07
215 9.60805732916015e-07
216 9.57779043943674e-07
217 9.57766267873694e-07
218 9.54737937746586e-07
219 9.54498856344799e-07
220 9.52170370197791e-07
221 9.50519594084653e-07
222 9.48759981213243e-07
223 9.47102850858528e-07
224 9.46584362793601e-07
225 9.43646763801098e-07
226 9.42000673859411e-07
227 9.41450925949994e-07
228 9.38804523684666e-07
229 9.38322044589768e-07
230 9.35594229076742e-07
231 9.35247297007891e-07
232 9.32721268839032e-07
233 9.31552578464334e-07
234 9.30774605251372e-07
235 9.28317663891676e-07
236 9.26858944836795e-07
237 9.25179429543732e-07
238 9.23763715633186e-07
239 9.22354167357753e-07
240 9.20841406895079e-07
241 9.19228159915519e-07
242 9.17836204820333e-07
243 9.16312063225178e-07
244 9.14887535600428e-07
245 9.12851918116075e-07
246 9.12209820825183e-07
247 9.10882001278424e-07
248 9.09151504657757e-07
249 9.07929249365225e-07
250 9.06114797231794e-07
251 9.05019653970385e-07
252 9.03325363310614e-07
253 9.02047426535546e-07
254 9.00356864292462e-07
255 8.99421740939488e-07
256 8.9786664139524e-07
257 8.96162501391018e-07
258 8.94824077647627e-07
259 8.93166526680034e-07
260 8.92142999276757e-07
261 8.90646540653961e-07
262 8.89196897702504e-07
263 8.87752759098248e-07
264 8.86358454295078e-07
265 8.85122539898475e-07
266 8.83645790310084e-07
267 8.82326195210226e-07
268 8.80935579488096e-07
269 8.79514878278087e-07
270 8.78102829886984e-07
271 8.76637977967221e-07
272 8.75073412373695e-07
273 8.73695470517077e-07
274 8.7229220960694e-07
275 8.71096794639925e-07
276 8.69752706734062e-07
277 8.68293593612179e-07
278 8.66945744888881e-07
279 8.65692836129028e-07
280 8.6425307256377e-07
281 8.62996386615578e-07
282 8.61844596528272e-07
283 8.60314651902172e-07
284 8.58961779329093e-07
285 8.57684940086756e-07
286 8.56548996125639e-07
287 8.55367062882806e-07
288 8.53921339228236e-07
289 8.52381491938559e-07
290 8.5139164980319e-07
291 8.4983274942374e-07
292 8.48550058634601e-07
293 8.47121616715185e-07
294 8.45827474620364e-07
295 8.44044218467843e-07
296 8.42838108809474e-07
297 8.41480709709686e-07
298 8.40213770260334e-07
299 8.38855070696809e-07
300 8.37798641384779e-07
301 8.36651792894827e-07
302 8.35247779519932e-07
303 8.33991844217508e-07
304 8.3249875874003e-07
305 8.31112765837361e-07
306 8.29763220139057e-07
307 8.28537423330999e-07
308 8.27223779992892e-07
309 8.263019122694e-07
310 8.26237897740611e-07
311 8.24713217156159e-07
312 8.2334283979435e-07
313 8.22015577938373e-07
314 8.20726553371287e-07
315 8.19508391487034e-07
316 8.18137974590627e-07
317 8.17039462958746e-07
318 8.15745480878149e-07
319 8.14700765033649e-07
320 8.13773621985092e-07
321 8.12586493225354e-07
322 8.11482275494768e-07
323 8.10508947694188e-07
324 8.09030218846374e-07
325 8.08108718274525e-07
326 8.07103151188926e-07
327 8.06021113376687e-07
328 8.0490811670586e-07
329 8.03929369041612e-07
330 8.02840521885173e-07
331 8.01824190887146e-07
332 8.006422411313e-07
333 7.99249574953365e-07
334 7.98226580513983e-07
335 7.97378391467873e-07
336 7.96114629920908e-07
337 7.95464953313285e-07
338 7.94460564080168e-07
339 7.93411495550345e-07
340 7.92639923162142e-07
341 7.91717580995055e-07
342 7.90659596972887e-07
343 7.89734646772899e-07
344 7.88910815700206e-07
345 7.87838173835098e-07
346 7.87080269731177e-07
347 7.8613142619588e-07
348 7.85359551954912e-07
349 7.84582459175454e-07
350 7.83921955843425e-07
351 7.8297012908024e-07
352 7.82051770102043e-07
353 7.81296007033916e-07
354 7.80507885721704e-07
355 7.79689925153093e-07
356 7.78886507390553e-07
357 7.78194230775853e-07
358 7.77289544629411e-07
359 7.7656953261851e-07
360 7.75903502997721e-07
361 7.74936200443221e-07
362 7.74296334384417e-07
363 7.73721895654944e-07
364 7.72954912008572e-07
365 7.72319233249164e-07
366 7.71566827864945e-07
367 7.70699998838609e-07
368 7.70222717605407e-07
369 7.69374119812483e-07
370 7.68395611913775e-07
371 7.67787794430319e-07
372 7.67064833169684e-07
373 7.66419722197043e-07
374 7.66022498893904e-07
375 7.65052162719826e-07
376 7.64322403355777e-07
377 7.63627818059831e-07
378 7.63236719933502e-07
379 7.6248213609631e-07
380 7.62032572495741e-07
381 7.61118970331154e-07
382 7.60476678948407e-07
383 7.60098934165399e-07
384 7.592362803166e-07
385 7.586400048325e-07
386 7.58175735228406e-07
387 7.57343428460899e-07
388 7.56684466907131e-07
389 7.56365962160999e-07
390 7.55707249595616e-07
391 7.55030124281575e-07
392 7.54614512032958e-07
393 7.54073880841588e-07
394 7.53244257708729e-07
395 7.52822291843813e-07
396 7.52131151841695e-07
397 7.51477053313465e-07
398 7.50787530094499e-07
399 7.50268754131866e-07
400 7.49655679101124e-07
401 7.49106111442188e-07
402 7.48628529322559e-07
403 7.48065854992319e-07
404 7.47522233382369e-07
405 7.47014343886576e-07
406 7.4674312695322e-07
407 7.46154744589944e-07
408 7.45224265713773e-07
409 7.44781130833871e-07
410 7.44168451475957e-07
411 7.43612403141469e-07
412 7.43074742729277e-07
413 7.4255538692114e-07
414 7.41972566757454e-07
415 7.41491772316749e-07
416 7.40886318837397e-07
417 7.40458591266702e-07
418 7.40117279718788e-07
419 7.39657903153557e-07
420 7.3912421322575e-07
421 7.38795697401429e-07
422 7.38292617114666e-07
423 7.37599005930178e-07
424 7.37452366976754e-07
425 7.36602437115152e-07
426 7.36545854650217e-07
427 7.36174392713451e-07
428 7.35576475562993e-07
429 7.35150057053602e-07
430 7.34723915968516e-07
431 7.34193425984131e-07
432 7.33826684466976e-07
433 7.33313877120167e-07
434 7.32849924304446e-07
435 7.3240356381632e-07
436 7.31977483084734e-07
437 7.3147746320501e-07
438 7.31188059575061e-07
439 7.30770797602531e-07
440 7.3025388996939e-07
441 7.29911231729829e-07
442 7.29591046166433e-07
443 7.29134805823151e-07
444 7.28707190248201e-07
445 7.2811659848071e-07
446 7.2783998621162e-07
447 7.27755489791093e-07
448 7.26801464111304e-07
449 7.26691662777057e-07
450 7.26335354073626e-07
451 7.25658945043506e-07
452 7.25614078646686e-07
453 7.25175011581314e-07
454 7.24782964383053e-07
455 7.2419680768121e-07
456 7.2407520153206e-07
457 7.23398212727489e-07
458 7.23838310904057e-07
459 7.22997384855262e-07
460 7.22640464857705e-07
461 7.22022445884818e-07
462 7.21818079284731e-07
463 7.21388100032527e-07
464 7.20957314285897e-07
465 7.20499894029558e-07
466 7.20118750535903e-07
467 7.1978251421001e-07
468 7.19372024406084e-07
469 7.19226161493225e-07
470 7.18592157170406e-07
471 7.18292583542279e-07
472 7.17921304840274e-07
473 7.17656429145563e-07
474 7.17268838670293e-07
475 7.16870283014259e-07
476 7.16790947947743e-07
477 7.16224187868875e-07
478 7.15926577953496e-07
479 7.15506018508449e-07
480 7.15327944973865e-07
481 7.15154259026463e-07
482 7.14588176222719e-07
483 7.14167662749787e-07
484 7.13947983513208e-07
485 7.13584758472052e-07
486 7.13490786978355e-07
487 7.12802996346795e-07
488 7.1260814823404e-07
489 7.12203429642955e-07
490 7.12049823263783e-07
491 7.11565890298971e-07
492 7.11261462981838e-07
493 7.10956241448457e-07
494 7.10864853687099e-07
495 7.10524007445201e-07
496 7.10050566780751e-07
497 7.10148756851936e-07
498 7.09454082794991e-07
499 7.09290811926167e-07
500 7.08878859995821e-07
501 7.08592461123203e-07
502 7.08402716242063e-07
503 7.07911499688407e-07
504 7.07838705423569e-07
505 7.07405181373133e-07
506 7.07240273371212e-07
507 7.06922514538633e-07
508 7.06553931124176e-07
509 7.06358199010992e-07
510 7.05988017728032e-07
511 7.05820195491924e-07
512 7.05478267803983e-07
513 7.05317981612552e-07
514 7.04972263605441e-07
515 7.04692528387341e-07
516 7.04417077272978e-07
517 7.0408141230871e-07
518 7.03932418971931e-07
519 7.03659394588385e-07
520 7.03295456546016e-07
521 7.02983170214111e-07
522 7.02736174432061e-07
523 7.02738154330973e-07
524 7.02413211371322e-07
525 7.02195987742016e-07
526 7.01834438388005e-07
527 7.015176416445e-07
528 7.01357761215604e-07
529 7.01315112848988e-07
530 7.00866517291843e-07
531 7.00584499909951e-07
532 7.00379912302651e-07
533 7.00104224122811e-07
534 6.99711710453244e-07
535 6.99730678590527e-07
536 6.99416920937779e-07
537 6.99106589507892e-07
538 6.99099857314422e-07
539 6.98789812673795e-07
540 6.98547626143409e-07
541 6.9829937623922e-07
542 6.9803573740046e-07
543 6.97883229818785e-07
544 6.97516356922279e-07
545 6.97531738353518e-07
546 6.97229756156048e-07
547 6.96944444854353e-07
548 6.96735756392286e-07
549 6.96441722340069e-07
550 6.96331431043973e-07
551 6.95992913136934e-07
552 6.95820564317273e-07
553 6.95558284050435e-07
554 6.95345673136671e-07
555 6.9511123327004e-07
556 6.9492508868052e-07
557 6.94684943496782e-07
558 6.94408091291621e-07
559 6.94226875609161e-07
560 6.93976865377977e-07
561 6.93962414473503e-07
562 6.93575890437614e-07
563 6.93363744161957e-07
564 6.93089253118728e-07
565 6.92930680912696e-07
566 6.92638980467564e-07
567 6.92638060669992e-07
568 6.92431998800203e-07
569 6.92011693558925e-07
570 6.91779698058781e-07
571 6.91561712798716e-07
572 6.91245357501202e-07
573 6.90914098456119e-07
574 6.90786418317657e-07
575 6.90630641898338e-07
576 6.90921719240123e-07
577 6.90136385131268e-07
578 6.90016155033391e-07
579 6.89746223571319e-07
580 6.89630235626737e-07
581 6.89517517599825e-07
582 6.89564031830514e-07
583 6.8903582733526e-07
584 6.89009364705839e-07
585 6.88743238015377e-07
586 6.8840332086495e-07
587 6.88202743617694e-07
588 6.88021202506661e-07
589 6.87964407845243e-07
590 6.88042817529322e-07
591 6.87679688496701e-07
592 6.87582603731585e-07
593 6.87585936574919e-07
594 6.87160389531982e-07
595 6.86954137947282e-07
596 6.8681868329179e-07
597 6.86839983941923e-07
598 6.86604555980352e-07
599 6.86168778585738e-07
600 6.85818640121738e-07
601 6.85711275750123e-07
602 6.85531425304475e-07
603 6.85469083933299e-07
604 6.85334695788242e-07
605 6.84826699455243e-07
606 6.84622775608545e-07
607 6.84457398264726e-07
608 6.84304423188564e-07
609 6.84296297464471e-07
610 6.84731524756899e-07
611 6.83619167304528e-07
612 6.83459364040573e-07
613 6.83336111848121e-07
614 6.83173144011562e-07
615 6.83247643792129e-07
616 6.829115751259e-07
617 6.82785364645611e-07
618 6.82823477333727e-07
619 6.82156496722541e-07
620 6.82054949521671e-07
621 6.81831772880059e-07
622 6.81677918009882e-07
623 6.81486279333399e-07
624 6.81359977932061e-07
625 6.81108247150064e-07
626 6.81064127164177e-07
627 6.81072284521633e-07
628 6.80686871632474e-07
629 6.80564016704466e-07
630 6.80698681122749e-07
631 6.80240825744249e-07
632 6.80033357681964e-07
633 6.80167803238874e-07
634 6.79523233444002e-07
635 6.79397790690928e-07
636 6.79198196905872e-07
637 6.79049905556894e-07
638 6.79049523313324e-07
639 6.79142973154967e-07
640 6.78811062925888e-07
641 6.78329988289761e-07
642 6.78268604303867e-07
643 6.78312620792099e-07
644 6.77984985998137e-07
645 6.77676616234635e-07
646 6.77458028803812e-07
647 6.77593978508639e-07
648 6.77711388178182e-07
649 6.77032173726388e-07
650 6.7705669290774e-07
651 6.77168279509033e-07
652 6.76492063206524e-07
653 6.76391153575651e-07
654 6.76132728315793e-07
655 6.76058174207128e-07
656 6.76169795809756e-07
657 6.76121099218108e-07
658 6.75468571330384e-07
659 6.75345675475114e-07
660 6.75184716641297e-07
661 6.75026644856302e-07
662 6.74925118602232e-07
663 6.75125837290125e-07
664 6.74430557722872e-07
665 6.74353537561956e-07
666 6.74566049795544e-07
667 6.74105771835798e-07
668 6.73843414475073e-07
669 6.74108299563159e-07
670 6.73400757577269e-07
671 6.73219013251014e-07
672 6.73141517864906e-07
673 6.73084371044297e-07
674 6.73303882052778e-07
675 6.73119193351113e-07
676 6.72406169627493e-07
677 6.72221007633311e-07
678 6.72262872228657e-07
679 6.71918138493766e-07
680 6.7187868074825e-07
681 6.71670914556444e-07
682 6.71474687848672e-07
683 6.71449357596998e-07
684 6.71642237449532e-07
685 6.71132449625134e-07
686 6.70766620928021e-07
687 6.70696218847411e-07
688 6.70498978038836e-07
689 6.70508200329323e-07
690 6.70917440473318e-07
691 6.70120118655859e-07
692 6.69779609609122e-07
693 6.69663694836231e-07
694 6.69445325542029e-07
695 6.69258560876074e-07
696 6.69263294042821e-07
697 6.69365898531282e-07
698 6.69413571543487e-07
699 6.68658066985017e-07
700 6.6849241932232e-07
701 6.68295025889165e-07
702 6.68346391194063e-07
703 6.68034477683932e-07
704 6.67883314008577e-07
705 6.67871747751292e-07
706 6.67869822521539e-07
707 6.6741284337013e-07
708 6.67211375102283e-07
709 6.66998994347523e-07
710 6.67053436302467e-07
711 6.6682756991554e-07
712 6.66589114103999e-07
713 6.66490238728556e-07
714 6.66292823055414e-07
715 6.66251703336229e-07
716 6.66788175905708e-07
717 6.65793759552002e-07
718 6.65465117947406e-07
719 6.65816963035581e-07
720 6.65758463583188e-07
721 6.65544843045041e-07
722 6.65886225434065e-07
723 6.65198361417651e-07
724 6.64831936816768e-07
725 6.64645499369954e-07
726 6.64471501011121e-07
727 6.64225957706321e-07
728 6.64358892365158e-07
729 6.64192160570565e-07
730 6.64110347983637e-07
731 6.63900382065208e-07
732 6.63378257172553e-07
733 6.63277901182369e-07
734 6.63126197338215e-07
735 6.63272134914905e-07
736 6.63339513778283e-07
737 6.63178494576755e-07
738 6.62740552428431e-07
739 6.62412929628431e-07
740 6.62471505791018e-07
741 6.62118904955378e-07
742 6.6211564454477e-07
743 6.61732920079317e-07
744 6.61794542253347e-07
745 6.61537696160508e-07
746 6.6167947284157e-07
747 6.61282068222135e-07
748 6.60957793272132e-07
749 6.60974708708295e-07
750 6.60737628876973e-07
751 6.60563611177167e-07
752 6.60669549006343e-07
753 6.60142026859489e-07
754 6.59790588628084e-07
755 6.59206425495995e-07
756 6.58697558208132e-07
757 6.58472624053275e-07
758 6.58561325010965e-07
759 6.58079057743066e-07
760 6.58026551377589e-07
761 6.57965502142588e-07
762 6.57523861733012e-07
763 6.57337378598299e-07
764 6.56981951877356e-07
765 6.56761206968781e-07
766 6.56501978454571e-07
767 6.55857276768756e-07
768 6.56530517460396e-07
769 6.56905732810742e-07
770 6.56052742030511e-07
771 6.55603195994559e-07
772 6.55322380296752e-07
773 6.54704865823419e-07
774 6.54461226190506e-07
775 6.54197285342661e-07
776 6.53888960741256e-07
777 6.53602056658542e-07
778 6.53268098702142e-07
779 6.54040435051684e-07
780 6.53235653487627e-07
781 6.52853220628913e-07
782 6.52299065265538e-07
783 6.52097500406512e-07
784 6.51775928290022e-07
785 6.5176647149201e-07
786 6.51328027572617e-07
787 6.51507847905464e-07
788 6.51621266655411e-07
789 6.51120280338091e-07
790 6.50271066035657e-07
791 6.50623032598219e-07
792 6.50225028067553e-07
793 6.50682340662456e-07
794 6.49817614075232e-07
795 6.50719730501237e-07
796 6.49664923798809e-07
797 6.50211584300564e-07
798 6.48809363013925e-07
799 6.49380920918929e-07
800 6.4855317921797e-07
801 6.48718604878695e-07
802 6.48325270674377e-07
803 6.48340940912817e-07
804 6.48384094091625e-07
805 6.47670242756249e-07
806 6.47294677563082e-07
807 6.47407202379213e-07
808 6.47062250948238e-07
809 6.47139711901445e-07
810 6.46794424753239e-07
811 6.46586513980196e-07
812 6.4630621535855e-07
813 6.46143918785924e-07
814 6.46206421080819e-07
815 6.46050809052667e-07
816 6.45811578138478e-07
817 6.45534134704917e-07
818 6.45272071750469e-07
819 6.45084313475763e-07
820 6.44940158494478e-07
821 6.44797928664786e-07
822 6.44664317420052e-07
823 6.44708393238602e-07
824 6.44781474790079e-07
825 6.43953618663318e-07
826 6.4401244547696e-07
827 6.44133531608304e-07
828 6.43848811620273e-07
829 6.43534523021572e-07
830 6.43623536802807e-07
831 6.43355545122404e-07
832 6.43459978064698e-07
833 6.42539122694075e-07
834 6.42423720783825e-07
835 6.42229068049005e-07
836 6.43694028795494e-07
837 6.42312120220367e-07
838 6.42023174947326e-07
839 6.41817638097564e-07
840 6.41963917189514e-07
841 6.41126675986925e-07
842 6.41542135880968e-07
843 6.40884694462329e-07
844 6.40648684807843e-07
845 6.40480053760939e-07
846 6.39854228822401e-07
847 6.39868403069954e-07
848 6.39840600157981e-07
849 6.3893318502295e-07
850 6.38639376944639e-07
851 6.38337542994805e-07
852 6.38013974160856e-07
853 6.37820371764519e-07
854 6.38136522212562e-07
855 6.37338352333927e-07
856 6.36990881744737e-07
857 6.36761553366227e-07
858 6.36636442983729e-07
859 6.36787522850568e-07
860 6.36482686665829e-07
861 6.35810526496528e-07
862 6.35790216250598e-07
863 6.35545664792403e-07
864 6.35076302394566e-07
865 6.35354525677201e-07
866 6.35086437441146e-07
867 6.34891969426121e-07
868 6.33774776630958e-07
869 6.34218968521338e-07
870 6.33500820129029e-07
871 6.33716124205819e-07
872 6.33840579588707e-07
873 6.33507658463373e-07
874 6.32700088168292e-07
875 6.32751752640104e-07
876 6.32394511370649e-07
877 6.324811581635e-07
878 6.32187316163879e-07
879 6.31331189211437e-07
880 6.31645610027931e-07
881 6.30612103350359e-07
882 6.31642888585304e-07
883 6.30301115812415e-07
884 6.30303219345763e-07
885 6.30434531956325e-07
886 6.29825851213184e-07
887 6.3026594636284e-07
888 6.29491433983276e-07
889 6.29894949852883e-07
890 6.29112065524851e-07
891 6.2892201845699e-07
892 6.28567796532309e-07
893 6.28616281083794e-07
894 6.28783369236885e-07
895 6.28419824529658e-07
896 6.28173706630264e-07
897 6.2753141462224e-07
898 6.28429217087501e-07
899 6.27228162862536e-07
900 6.26573483586412e-07
901 6.25917755449734e-07
902 6.26262109051368e-07
903 6.26087741146364e-07
904 6.25708041049222e-07
905 6.25127507547063e-07
906 6.25702879716528e-07
907 6.25510647722649e-07
908 6.24522796002225e-07
909 6.25014120444689e-07
910 6.24146452096852e-07
911 6.24608488436706e-07
912 6.24371818311431e-07
913 6.24132697971902e-07
914 6.22939180331628e-07
915 6.2351004271477e-07
916 6.2332889153538e-07
917 6.23131349392736e-07
918 6.22394030202145e-07
919 6.22486220159146e-07
920 6.22470642809958e-07
921 6.22380071284567e-07
922 6.21532071036768e-07
923 6.21287593048692e-07
924 6.21170952555872e-07
925 6.21255831831036e-07
926 6.20757267839167e-07
927 6.20728286811811e-07
928 6.20359319391639e-07
929 6.20314215325379e-07
930 6.19811376125767e-07
931 6.19809856949871e-07
932 6.19194335051532e-07
933 6.19207079509465e-07
934 6.19016115749105e-07
935 6.18487976225879e-07
936 6.18693006728677e-07
937 6.18447636256292e-07
938 6.17691537371456e-07
939 6.1794909366597e-07
940 6.17617315015195e-07
941 6.17466598178851e-07
942 6.1694945669899e-07
943 6.16808603986385e-07
944 6.16357394129352e-07
945 6.16576176412309e-07
946 6.15661359837816e-07
947 6.15837455470114e-07
948 6.15222403098414e-07
949 6.15210279363509e-07
950 6.14131315380462e-07
951 6.14896761973682e-07
952 6.14312084110224e-07
953 6.13742194374822e-07
954 6.1423430738472e-07
955 6.13934294406704e-07
956 6.13463449120388e-07
957 6.13431861239633e-07
958 6.126252800982e-07
959 6.12680259457932e-07
960 6.12373812849398e-07
961 6.11732409318222e-07
962 6.11996258804481e-07
963 6.11834997698679e-07
964 6.11534490232657e-07
965 6.11279412410681e-07
966 6.10976991652024e-07
967 6.10801607635381e-07
968 6.0993121639541e-07
969 6.10257766908262e-07
970 6.09721131198171e-07
971 6.09514962107482e-07
972 6.10151406853277e-07
973 6.09608913990201e-07
974 6.08623257143392e-07
975 6.08609208669009e-07
976 6.07961926370138e-07
977 6.08075848290923e-07
978 6.07992424576764e-07
979 6.0775241048816e-07
980 6.07132156815737e-07
981 6.07294684876081e-07
982 6.06953180458447e-07
983 6.06453473309898e-07
984 6.05623692209178e-07
985 6.06061935386037e-07
986 6.05816325773389e-07
987 6.05566570953897e-07
988 6.05717543429307e-07
989 6.04947530575828e-07
990 6.04886406833316e-07
991 6.03998137073347e-07
992 6.0427891953907e-07
993 6.04073403579264e-07
994 6.03782304587242e-07
995 6.03477907574756e-07
996 6.03108307771549e-07
997 6.0244759085748e-07
998 6.02023152659115e-07
999 6.02464689706039e-07
1000 6.02305121390145e-07
1001 6.01802074946534e-07
1002 6.00690151784988e-07
1003 6.00747979532912e-07
1004 6.00759271186746e-07
1005 6.0054698502654e-07
1006 5.99763121869046e-07
1007 5.99490849140238e-07
1008 5.9974713006028e-07
1009 5.99658466896358e-07
1010 5.98812800596704e-07
1011 5.98873594029215e-07
1012 5.98376134469447e-07
1013 5.98230488101592e-07
1014 5.98133207567741e-07
1015 5.97380654419055e-07
1016 5.97340932856127e-07
1017 5.97020822851846e-07
1018 5.96662386726621e-07
1019 5.96547044061424e-07
1020 5.95986840195906e-07
1021 5.96448906740932e-07
1022 5.96114958746341e-07
1023 5.95908167888126e-07
1024 5.95399733924751e-07
1025 5.94854680898038e-07
1026 5.94013621309841e-07
1027 5.94053043350584e-07
1028 5.93657127041069e-07
1029 5.9312079419982e-07
1030 5.93351660242547e-07
1031 5.93160193432141e-07
1032 5.92637767212523e-07
1033 5.92132599592787e-07
1034 5.92207692520219e-07
1035 5.91352295039371e-07
1036 5.91074109657086e-07
1037 5.91294802873676e-07
1038 5.90296284478597e-07
1039 5.90281609348153e-07
1040 5.89172957631945e-07
1041 5.89864375598381e-07
1042 5.89333540624182e-07
1043 5.89092944501601e-07
1044 5.8811172851847e-07
1045 5.88118473984878e-07
1046 5.87558959523449e-07
1047 5.87498966091005e-07
1048 5.86975968829506e-07
1049 5.86538542663106e-07
1050 5.85390634384453e-07
1051 5.8604224697234e-07
1052 5.85704977417834e-07
1053 5.84833425804732e-07
1054 5.85203938186396e-07
1055 5.84576740706666e-07
1056 5.84105853000949e-07
1057 5.83624754902701e-07
1058 5.82899904443934e-07
1059 5.82916453005566e-07
1060 5.82865486791206e-07
1061 5.81442056130754e-07
1062 5.80942083999503e-07
1063 5.80710280317476e-07
1064 5.8016460865673e-07
1065 5.80148177206752e-07
1066 5.79519820348651e-07
1067 5.79282890541322e-07
1068 5.78339731006849e-07
1069 5.78075172036563e-07
1070 5.77422741336875e-07
1071 5.76988799444678e-07
1072 5.76913078944585e-07
1073 5.7649559715145e-07
1074 5.7592376823834e-07
1075 5.75118488846726e-07
1076 5.74859645261938e-07
1077 5.74464747032266e-07
1078 5.7391911401794e-07
1079 5.73786442799928e-07
1080 5.73293567576627e-07
1081 5.72705432531961e-07
1082 5.72505290044489e-07
1083 5.72218330383123e-07
1084 5.71506954933909e-07
1085 5.70766419834001e-07
1086 5.70619852211962e-07
1087 5.70089109672267e-07
1088 5.69310650618604e-07
1089 5.69087913817157e-07
1090 5.68740758041031e-07
1091 5.68193936842931e-07
1092 5.67973420743328e-07
1093 5.67342231157397e-07
1094 5.66961273946731e-07
1095 5.66632256052912e-07
1096 5.66003266825987e-07
1097 5.65674219494383e-07
1098 5.65259562122833e-07
1099 5.64769521723463e-07
1100 5.64369430676948e-07
1101 5.63613571578969e-07
1102 5.63189598679514e-07
1103 5.63041100242856e-07
1104 5.62284110486644e-07
1105 5.61835125878929e-07
1106 5.61412824893637e-07
1107 5.60981057205367e-07
1108 5.60571713556612e-07
1109 5.59922194639739e-07
1110 5.59530634561156e-07
1111 5.59002662015473e-07
1112 5.581992556003e-07
1113 5.57639288423673e-07
1114 5.57120050359572e-07
1115 5.56716404425117e-07
1116 5.56288290574969e-07
1117 5.55825715849778e-07
1118 5.55270240283789e-07
1119 5.5472401763268e-07
1120 5.5407207688063e-07
1121 5.53537679024885e-07
1122 5.52935819953859e-07
1123 5.52418589208514e-07
1124 5.5196548998282e-07
1125 5.5140549684296e-07
1126 5.50842811136931e-07
1127 5.50178398597723e-07
1128 5.49853654099763e-07
1129 5.49349440554181e-07
1130 5.48803564406342e-07
1131 5.48427744170965e-07
1132 5.47851732008553e-07
1133 5.47107029639449e-07
1134 5.46832161106181e-07
1135 5.46189866398095e-07
1136 5.45587667346581e-07
1137 5.45162206151417e-07
1138 5.4475243558727e-07
1139 5.43768126320288e-07
1140 5.43458325758195e-07
1141 5.42946222211071e-07
1142 5.42318003155628e-07
1143 5.41685208027332e-07
1144 5.41118284786535e-07
1145 5.40587837761564e-07
1146 5.40002958679509e-07
1147 5.3967720147341e-07
1148 5.3918130650743e-07
1149 5.38497736847887e-07
1150 5.37864707133906e-07
1151 5.37328203797927e-07
1152 5.36580279764109e-07
1153 5.36202094679083e-07
1154 5.35710694869351e-07
1155 5.35006929652582e-07
1156 5.34474729406043e-07
1157 5.33720529190873e-07
1158 5.33245014793238e-07
1159 5.32959159869506e-07
1160 5.33745327018664e-07
1161 5.32485687202211e-07
1162 5.31249801881017e-07
1163 5.30626741706897e-07
1164 5.2979573960954e-07
1165 5.29420067280739e-07
1166 5.28620816382386e-07
1167 5.27807401681457e-07
1168 5.27121304358502e-07
1169 5.26377595562622e-07
1170 5.25672054251913e-07
1171 5.24890954721968e-07
1172 5.2444793212203e-07
1173 5.23721696609414e-07
1174 5.23018498213901e-07
1175 5.2188551669019e-07
1176 5.21017574712346e-07
1177 5.20534954489449e-07
1178 5.19553454381594e-07
1179 5.18730347920382e-07
1180 5.17958713544431e-07
1181 5.17150383586795e-07
1182 5.16437501957512e-07
1183 5.15647556241561e-07
1184 5.15112535509843e-07
1185 5.14169199476555e-07
1186 5.13523359217061e-07
1187 5.12790095356763e-07
1188 5.1223551854207e-07
1189 5.11717524076971e-07
1190 5.10849785044343e-07
1191 5.10033092680828e-07
1192 5.09329841293038e-07
1193 5.08588952371269e-07
1194 5.07864471316566e-07
1195 5.07212841625915e-07
1196 5.06569104132382e-07
1197 5.05559155925539e-07
1198 5.05332368987865e-07
1199 5.04487800398579e-07
1200 5.03532473260293e-07
1201 5.03095116613395e-07
1202 5.02088503367304e-07
1203 5.01136403556757e-07
1204 5.00731445981728e-07
1205 4.99776320992851e-07
1206 4.99325033715081e-07
1207 4.98700373640304e-07
1208 4.97808630200325e-07
1209 4.96947417900628e-07
1210 4.96288867537942e-07
1211 4.95732484836253e-07
1212 4.94955880725456e-07
1213 4.94176946858715e-07
1214 4.93376584756788e-07
1215 4.92612390658564e-07
1216 4.920525770018e-07
1217 4.91067804333056e-07
1218 4.90369304714022e-07
1219 4.89532306190199e-07
1220 4.88886637128871e-07
1221 4.88063364329605e-07
1222 4.87359449877545e-07
1223 4.86462110629304e-07
1224 4.8582851755441e-07
1225 4.8509610044789e-07
1226 4.84401673972457e-07
1227 4.83254481267181e-07
1228 4.82487706861434e-07
1229 4.81833279323496e-07
1230 4.80243687633219e-07
1231 4.78414342282463e-07
1232 4.76838534240187e-07
1233 4.76149318657804e-07
1234 4.74790627791322e-07
1235 4.73768826594778e-07
1236 4.72720507829649e-07
1237 4.71855896634565e-07
1238 4.70534331853401e-07
1239 4.70113000531569e-07
1240 4.69121401650341e-07
1241 4.68302746483573e-07
1242 4.67329256736093e-07
1243 4.66545695545051e-07
1244 4.65720774428746e-07
1245 4.64882776356035e-07
1246 4.64056661272139e-07
1247 4.63339145298391e-07
1248 4.62453502805715e-07
1249 4.6154897869144e-07
1250 4.60787575832455e-07
1251 4.59998585000676e-07
1252 4.59476001552161e-07
1253 4.58542613344548e-07
1254 4.58008765320983e-07
1255 4.57043186344208e-07
1256 4.56172413777267e-07
1257 4.55492208899955e-07
1258 4.54621887328699e-07
1259 4.53731679030511e-07
1260 4.52947662239467e-07
1261 4.52175911945574e-07
1262 4.51259721572228e-07
1263 4.50436129000309e-07
1264 4.49572491348249e-07
1265 4.49066028352263e-07
1266 4.48156541779099e-07
1267 4.47460020353674e-07
1268 4.46502518983038e-07
1269 4.45779370750188e-07
1270 4.45073646687888e-07
1271 4.44745651066114e-07
1272 4.43919204570875e-07
1273 4.43130237655964e-07
1274 4.42290351216457e-07
1275 4.41504531295323e-07
1276 4.40815369401548e-07
1277 4.40169212268415e-07
1278 4.39142476835741e-07
1279 4.38690070765801e-07
1280 4.37918824530925e-07
1281 4.3708706678558e-07
1282 4.36362787709754e-07
1283 4.35320906618131e-07
1284 4.34962962970076e-07
1285 4.34124661254032e-07
1286 4.33472893078601e-07
1287 4.3270664316708e-07
1288 4.3223810664017e-07
1289 4.31506152210659e-07
1290 4.30598854350706e-07
1291 4.29677616750723e-07
1292 4.28937097368021e-07
1293 4.2827022625147e-07
1294 4.27427441493933e-07
1295 4.2679034957871e-07
1296 4.25909393555912e-07
1297 4.25184653096267e-07
1298 4.24548843795947e-07
1299 4.2362739712587e-07
1300 4.23001229847841e-07
1301 4.22171937557891e-07
1302 4.21529943849919e-07
1303 4.20893796700739e-07
1304 4.1996023799129e-07
1305 4.19370747252401e-07
1306 4.18629378344804e-07
1307 4.17800233435628e-07
1308 4.17327775906529e-07
1309 4.16702173822614e-07
1310 4.15864855156656e-07
1311 4.15330635235023e-07
1312 4.14520232155269e-07
1313 4.13829448135061e-07
1314 4.1304789226615e-07
1315 4.12341253692716e-07
1316 4.1216946509337e-07
1317 4.11018312902911e-07
1318 4.10335426906272e-07
1319 4.09523415214608e-07
1320 4.08610692858247e-07
1321 4.07992596990425e-07
1322 4.07032663076734e-07
1323 4.06648582554681e-07
1324 4.05807523293333e-07
1325 4.05085819707551e-07
1326 4.04363589808554e-07
1327 4.03830442820663e-07
1328 4.0298853319598e-07
1329 4.02449150698203e-07
1330 4.02263740824083e-07
1331 4.01028636716205e-07
1332 4.00283313709338e-07
1333 3.99633769589514e-07
1334 3.98941660137098e-07
1335 3.98408138792661e-07
1336 3.9767170837024e-07
1337 3.96993410618052e-07
1338 3.96322553143591e-07
1339 3.95798050206508e-07
1340 3.94935926777862e-07
1341 3.94421753668439e-07
1342 3.9375066577918e-07
1343 3.93084235483343e-07
1344 3.92502506613823e-07
1345 3.91987923293868e-07
1346 3.9124424080228e-07
1347 3.91049823832645e-07
1348 3.90322553116107e-07
1349 3.89593919820186e-07
1350 3.88722149352816e-07
1351 3.88301868838425e-07
1352 3.87433650871571e-07
1353 3.87444912661294e-07
1354 3.86126681021892e-07
1355 3.85559809302549e-07
1356 3.8492804539203e-07
1357 3.84593586048254e-07
1358 3.83902541258863e-07
1359 3.83004335461123e-07
1360 3.82602766521245e-07
1361 3.82011311558017e-07
1362 3.81463873210919e-07
1363 3.8110734190866e-07
1364 3.80366255839704e-07
1365 3.79705994987489e-07
1366 3.79158340081176e-07
1367 3.78491242940981e-07
1368 3.78047957042327e-07
1369 3.77540550999811e-07
1370 3.77056229396544e-07
1371 3.7643638771101e-07
1372 3.7545368427061e-07
1373 3.74841019990413e-07
1374 3.74239669682197e-07
1375 3.73709808172862e-07
1376 3.73041227035742e-07
1377 3.72869718347602e-07
1378 3.72220887982166e-07
1379 3.71513008502689e-07
1380 3.70764545507996e-07
1381 3.70413332248631e-07
1382 3.69776117224774e-07
1383 3.68990465801744e-07
1384 3.68179114161649e-07
1385 3.67674000202101e-07
1386 3.67139926510163e-07
1387 3.67087734161942e-07
1388 3.66082978302984e-07
1389 3.65367626926627e-07
1390 3.6481237800956e-07
1391 3.63644148123399e-07
1392 3.63287133581025e-07
1393 3.63036160877073e-07
1394 3.62126584448674e-07
1395 3.61699762649437e-07
1396 3.61027375987533e-07
1397 3.60369763242829e-07
1398 3.60007189115663e-07
1399 3.59295827124129e-07
1400 3.5775579875974e-07
1401 3.53868214730824e-07
1402 3.41379617296411e-07
1403 3.28579712345345e-07
1404 3.25548541511012e-07
1405 3.24216553693191e-07
1406 3.22977575535788e-07
1407 3.22425292083039e-07
1408 3.21560779539709e-07
1409 3.21061373725229e-07
1410 3.19967045953717e-07
1411 3.19681221583323e-07
1412 3.18886990186229e-07
1413 3.18288745347672e-07
1414 3.17631491043358e-07
1415 3.17402957406898e-07
1416 3.16448183653506e-07
1417 3.16011549529094e-07
1418 3.15193454127893e-07
1419 3.1478528707396e-07
1420 3.14031266398729e-07
1421 3.13261919700381e-07
1422 3.12983140958067e-07
1423 3.1251393924947e-07
1424 3.11998803240954e-07
1425 3.11705126065931e-07
1426 3.1094608770843e-07
1427 3.1091148596829e-07
1428 3.10241547680334e-07
1429 3.09736733072441e-07
1430 3.09197059110033e-07
1431 3.09091740660961e-07
1432 3.0816822710733e-07
1433 3.07916834699995e-07
1434 3.07206576749763e-07
1435 3.06838810715249e-07
1436 3.06620812068559e-07
1437 3.0611676983483e-07
1438 3.05473894016473e-07
1439 3.05116190560284e-07
1440 3.0515102234574e-07
1441 3.04555343959123e-07
1442 3.04196095541442e-07
1443 3.03736167268198e-07
1444 3.040585565941e-07
1445 3.0285602309732e-07
1446 3.02862596157638e-07
1447 3.02560588011147e-07
1448 3.02829190928833e-07
1449 3.01927896686038e-07
1450 3.01475552525687e-07
1451 3.00904842305272e-07
1452 3.00629884698367e-07
1453 3.00555958332893e-07
1454 3.00048228730532e-07
1455 2.99900297065392e-07
1456 2.99528309639641e-07
1457 2.99114135650314e-07
1458 2.98604760402554e-07
1459 2.98240277793127e-07
1460 2.97766990755122e-07
1461 2.98054582771101e-07
1462 2.97141440512405e-07
1463 2.96694996791302e-07
1464 2.96270484767547e-07
1465 2.96184018438339e-07
1466 2.95704940683095e-07
1467 2.95396794122382e-07
1468 2.95091585243767e-07
1469 2.94782432320062e-07
1470 2.94291698516247e-07
1471 2.93947373592118e-07
1472 2.93326942326644e-07
1473 2.93185483691616e-07
1474 2.93146487450713e-07
1475 2.92696370287615e-07
1476 2.92517656269808e-07
1477 2.92087966471399e-07
1478 2.91890827398333e-07
1479 2.91456780672661e-07
1480 2.91003376375443e-07
1481 2.90897139052504e-07
1482 2.90544493950051e-07
1483 2.90321684360606e-07
1484 2.89833084551105e-07
1485 2.89581516007331e-07
1486 2.89212358396185e-07
1487 2.88950256795317e-07
1488 2.88564842733763e-07
1489 2.88200527080562e-07
1490 2.87884401444671e-07
1491 2.87497334070963e-07
1492 2.8721040789037e-07
1493 2.86508051104306e-07
1494 2.86306328078467e-07
1495 2.86144508876873e-07
1496 2.85861089821537e-07
1497 2.85360442411786e-07
1498 2.85058979642372e-07
1499 2.84970874481871e-07
1500 2.84450102817857e-07
1501 2.84078861753301e-07
1502 2.83543574305156e-07
1503 2.83563822584654e-07
1504 2.82890254204915e-07
1505 2.82827177642275e-07
1506 2.82641763888591e-07
1507 2.82033121649761e-07
1508 2.82002064608378e-07
1509 2.81318216082127e-07
1510 2.81270375850795e-07
1511 2.8097148791062e-07
1512 2.80864460734165e-07
1513 2.80383226922254e-07
1514 2.80231873112768e-07
1515 2.79643016561693e-07
1516 2.79513022562128e-07
1517 2.79163273717131e-07
1518 2.7870598270141e-07
1519 2.78674357851116e-07
1520 2.78108519935927e-07
1521 2.78267315479752e-07
1522 2.77627101574751e-07
1523 2.77175951154618e-07
1524 2.77061909592646e-07
1525 2.76829202157103e-07
1526 2.7642390725191e-07
1527 2.76237728002116e-07
1528 2.75949262274366e-07
1529 2.75852835628143e-07
1530 2.75600234033391e-07
1531 2.74864786490525e-07
1532 2.74880136871047e-07
1533 2.74431400029584e-07
1534 2.74260290815675e-07
1535 2.73908545914026e-07
1536 2.73709271660039e-07
1537 2.73536457655155e-07
1538 2.73341923637815e-07
1539 2.73021322414024e-07
1540 2.72803043102954e-07
1541 2.72587641099165e-07
1542 2.72122814891418e-07
1543 2.71993649803903e-07
1544 2.71596017171305e-07
1545 2.71636295664734e-07
1546 2.71166804125755e-07
1547 2.70989261075272e-07
1548 2.70696542713722e-07
1549 2.70436283962283e-07
1550 2.70061753731454e-07
1551 2.6989989021331e-07
1552 2.69763729100703e-07
1553 2.69655354983911e-07
1554 2.69066596061407e-07
1555 2.68926969837935e-07
1556 2.68573427192109e-07
1557 2.68608412383742e-07
1558 2.67920140125e-07
1559 2.67861273933079e-07
1560 2.67640900844413e-07
1561 2.67677652011855e-07
1562 2.66964462625197e-07
1563 2.66822012484624e-07
1564 2.66567217181546e-07
1565 2.66555855290562e-07
1566 2.65938470718652e-07
1567 2.65836721510482e-07
1568 2.65485345245509e-07
1569 2.65497387978542e-07
1570 2.65124400584682e-07
1571 2.64794280695924e-07
1572 2.64446180700872e-07
1573 2.64363307621807e-07
1574 2.63953197340072e-07
1575 2.63414974881471e-07
1576 2.6352014254627e-07
1577 2.6342906294019e-07
1578 2.63256980119309e-07
1579 2.6275006576526e-07
1580 2.62603489957769e-07
1581 2.62322275212057e-07
1582 2.62190380972527e-07
1583 2.61826094835271e-07
1584 2.61762005067112e-07
1585 2.6165635841835e-07
1586 2.61387302224136e-07
1587 2.60961940504956e-07
1588 2.60710632772998e-07
1589 2.60395194366936e-07
1590 2.60170396778392e-07
1591 2.60418973766718e-07
1592 2.59962582262574e-07
1593 2.5969601419007e-07
1594 2.59373768479065e-07
1595 2.59266632525623e-07
1596 2.58953215023894e-07
1597 2.58385050443621e-07
1598 2.58312552205098e-07
1599 2.57958812120762e-07
1600 2.57778561035593e-07
1601 2.57694562435518e-07
1602 2.57359396975687e-07
1603 2.5705239291085e-07
1604 2.56811242699939e-07
1605 2.56461717242473e-07
1606 2.56126585846062e-07
1607 2.55749145210871e-07
1608 2.55653439175774e-07
1609 2.55513533744534e-07
1610 2.55211480364892e-07
1611 2.54804955659438e-07
1612 2.54763174751815e-07
1613 2.54787983493543e-07
1614 2.54455153722688e-07
1615 2.54169908593838e-07
1616 2.53858018467668e-07
1617 2.53570905002221e-07
1618 2.53653469627579e-07
1619 2.53107618163995e-07
1620 2.53153813460472e-07
1621 2.52625748231594e-07
1622 2.52780608676062e-07
1623 2.52202483792985e-07
1624 2.51898197333844e-07
1625 2.51692562230232e-07
1626 2.51686352093827e-07
1627 2.51284400277996e-07
1628 2.510466178407e-07
1629 2.50913202592074e-07
1630 2.50352099349982e-07
1631 2.50282909277644e-07
1632 2.50297362768492e-07
1633 2.49979243882592e-07
1634 2.49957796164324e-07
1635 2.49437291799381e-07
1636 2.49304088775659e-07
1637 2.49228307311e-07
1638 2.48961546077453e-07
1639 2.48945990243499e-07
1640 2.48505118250364e-07
1641 2.48237105296312e-07
1642 2.47971299991434e-07
1643 2.47875569947098e-07
1644 2.4794665956307e-07
1645 2.47326225185418e-07
1646 2.47498117275313e-07
1647 2.46912391794751e-07
1648 2.46693469193815e-07
1649 2.46549616903735e-07
1650 2.46913651224645e-07
1651 2.46156420679711e-07
1652 2.46014868004352e-07
1653 2.46031733666996e-07
1654 2.45937145464836e-07
1655 2.45421275380409e-07
1656 2.4518313143318e-07
1657 2.45077444105846e-07
1658 2.45103997855267e-07
1659 2.44453994220351e-07
1660 2.44352771495926e-07
1661 2.44336961486624e-07
1662 2.43957096735414e-07
1663 2.44000740032391e-07
1664 2.43840326682232e-07
1665 2.43737352008111e-07
1666 2.43350576539569e-07
1667 2.43003017637022e-07
1668 2.42828271026951e-07
1669 2.42498429528837e-07
1670 2.42299023938131e-07
1671 2.42565715481646e-07
1672 2.41789805663473e-07
1673 2.41731107983867e-07
1674 2.4149342952029e-07
1675 2.41314539678683e-07
1676 2.41171244880434e-07
1677 2.40871798965259e-07
1678 2.4079149969225e-07
1679 2.41200686303955e-07
1680 2.40602273073875e-07
1681 2.40352490664009e-07
1682 2.40002927021976e-07
1683 2.40169289128289e-07
1684 2.3982877031159e-07
1685 2.40364217468425e-07
1686 2.39530044183311e-07
1687 2.39584125388603e-07
1688 2.39088032536472e-07
1689 2.3960805847878e-07
1690 2.3883765531707e-07
1691 2.38823247961761e-07
1692 2.38320987165253e-07
1693 2.38512777499977e-07
1694 2.37777303709663e-07
1695 2.37730604681019e-07
1696 2.37934320814759e-07
1697 2.3772690016699e-07
1698 2.37244818436011e-07
1699 2.3747739219715e-07
1700 2.3734742929804e-07
1701 2.3705694976428e-07
1702 2.36548055816854e-07
1703 2.36903177189163e-07
1704 2.36646192711021e-07
1705 2.36336162600992e-07
1706 2.35978233178002e-07
1707 2.36257230675108e-07
1708 2.36247042479931e-07
1709 2.35572928225736e-07
1710 2.3533673609677e-07
1711 2.35295649972045e-07
1712 2.35212478422397e-07
1713 2.3516802806256e-07
1714 2.34991536999019e-07
1715 2.34667007468659e-07
1716 2.34174775656015e-07
1717 2.34407579718265e-07
1718 2.34413436089653e-07
1719 2.34040339790909e-07
1720 2.33942749389371e-07
1721 2.33983239716906e-07
1722 2.33622052000726e-07
1723 2.33831540683127e-07
1724 2.33392370176944e-07
1725 2.33173824241817e-07
1726 2.33226754893678e-07
1727 2.33077691788708e-07
1728 2.32529140163251e-07
1729 2.32279638076704e-07
1730 2.32233921138914e-07
1731 2.32112333932832e-07
1732 2.31883679987277e-07
1733 2.31720178788919e-07
1734 2.32038291564152e-07
1735 2.31718267492909e-07
1736 2.31567941852973e-07
1737 2.31331374052957e-07
1738 2.3112843035733e-07
1739 2.30873782498975e-07
1740 2.30605826487817e-07
1741 2.30518129541224e-07
1742 2.30432836758609e-07
1743 2.3033577322451e-07
1744 2.3007742373693e-07
1745 2.30224105777665e-07
1746 2.30077882129365e-07
1747 2.30050854852948e-07
1748 2.29575847257024e-07
1749 2.2943226482397e-07
1750 2.29557112618295e-07
1751 2.29237574053798e-07
1752 2.29205479875816e-07
1753 2.29170308109872e-07
1754 2.28906373244797e-07
1755 2.28766571844119e-07
1756 2.28544766940786e-07
1757 2.2868126626463e-07
1758 2.27993804685411e-07
1759 2.27960802689608e-07
1760 2.27839590941414e-07
1761 2.27652181507665e-07
1762 2.27542316004303e-07
1763 2.27493458346828e-07
1764 2.27313096146986e-07
1765 2.27199244683618e-07
1766 2.27071835674053e-07
1767 2.27219710694726e-07
1768 2.26996866189211e-07
1769 2.26775318253658e-07
1770 2.26718550969451e-07
1771 2.26813940585657e-07
1772 2.26520079387171e-07
1773 2.26371451148566e-07
1774 2.26167481315542e-07
1775 2.25901802842543e-07
1776 2.25779298538953e-07
1777 2.25720340743862e-07
1778 2.25585753618418e-07
1779 2.25394293686065e-07
1780 2.25186369284813e-07
1781 2.24958091202154e-07
1782 2.24947849019941e-07
1783 2.24971209469516e-07
1784 2.24907599303492e-07
1785 2.24740581238336e-07
1786 2.24558846355194e-07
1787 2.24479246575982e-07
1788 2.24269736492033e-07
1789 2.24457478637419e-07
1790 2.23911447562841e-07
1791 2.23885244651001e-07
1792 2.23665631779113e-07
1793 2.23567293865301e-07
1794 2.23408093113164e-07
1795 2.23337343705055e-07
1796 2.23331608850685e-07
1797 2.23211972425474e-07
1798 2.23090034204176e-07
1799 2.22974817511101e-07
1800 2.23243140624163e-07
1801 2.22694968101678e-07
1802 2.22525880616331e-07
1803 2.22527042673448e-07
1804 2.22235511778024e-07
1805 2.22254597375127e-07
1806 2.22211922931592e-07
1807 2.22038266279867e-07
1808 2.22010197184375e-07
1809 2.22046008950372e-07
1810 2.21756733864709e-07
1811 2.21639085218328e-07
1812 2.21576202640961e-07
1813 2.2141506143214e-07
1814 2.21208061695677e-07
1815 2.21245883274435e-07
1816 2.21013788902269e-07
1817 2.21007236611115e-07
1818 2.20982161629024e-07
1819 2.21036623258897e-07
1820 2.20792780325496e-07
1821 2.20643940423315e-07
1822 2.20531014043956e-07
1823 2.20327503903661e-07
1824 2.20266067884722e-07
1825 2.20109437663041e-07
1826 2.20157255810705e-07
1827 2.19891945270945e-07
1828 2.19910469645868e-07
1829 2.19544526203208e-07
1830 2.19702533073018e-07
1831 2.19582152304554e-07
1832 2.19451375329527e-07
1833 2.19306615527159e-07
1834 2.19331069985174e-07
1835 2.18993384471844e-07
1836 2.19311624746865e-07
1837 2.18863029182614e-07
1838 2.18719483584096e-07
1839 2.1868493919186e-07
1840 2.18531250474996e-07
1841 2.18609444459617e-07
1842 2.18387314674828e-07
1843 2.18589027149108e-07
1844 2.18275262263035e-07
1845 2.18104250727436e-07
1846 2.18145190771679e-07
1847 2.17859501219664e-07
1848 2.17956179199064e-07
1849 2.17677949535755e-07
1850 2.17680827518052e-07
1851 2.17496870781986e-07
1852 2.17543898799022e-07
1853 2.17346247204375e-07
1854 2.17274957748259e-07
1855 2.17114544454944e-07
1856 2.17218183117041e-07
1857 2.17002188492188e-07
1858 2.17044254824827e-07
1859 2.16837378872015e-07
1860 2.16830454448313e-07
1861 2.16572191611419e-07
1862 2.16502173238098e-07
1863 2.1663202711153e-07
1864 2.16744778583688e-07
1865 2.16451098282278e-07
1866 2.16251673386125e-07
1867 2.16361901465234e-07
1868 2.1619253545424e-07
1869 2.15748911813307e-07
1870 2.15781792213932e-07
1871 2.15753512151196e-07
1872 2.15870599582502e-07
1873 2.15580016394767e-07
1874 2.15546139990863e-07
1875 2.15363809992652e-07
1876 2.15192065219583e-07
1877 2.15214828294563e-07
1878 2.15056860753293e-07
1879 2.14895314861963e-07
1880 2.14862649194458e-07
1881 2.14835938393776e-07
1882 2.14952394046009e-07
1883 2.14837238630139e-07
1884 2.14968138479321e-07
1885 2.14349865757413e-07
1886 2.14499318161643e-07
1887 2.14418838048402e-07
1888 2.14321352871139e-07
1889 2.14236844541915e-07
1890 2.13926508664031e-07
1891 2.13951539805635e-07
1892 2.1386986568217e-07
1893 2.14016840480724e-07
1894 2.13743024914947e-07
1895 2.13656738431212e-07
1896 2.13609864360365e-07
1897 2.13321337959371e-07
1898 2.13654979923206e-07
1899 2.13477472989609e-07
1900 2.13193983711335e-07
1901 2.13337714185968e-07
1902 2.13059026258122e-07
1903 2.13127699112192e-07
1904 2.12952673969369e-07
1905 2.12747189493712e-07
1906 2.12548494687326e-07
1907 2.12635293713959e-07
1908 2.12583282369394e-07
1909 2.1251817032919e-07
1910 2.12349659655331e-07
1911 2.1230279109119e-07
1912 2.1227749432029e-07
1913 2.12181868263883e-07
1914 2.12258335267279e-07
1915 2.11983677516514e-07
1916 2.12110384183006e-07
1917 2.11937834038167e-07
1918 2.11879030395323e-07
1919 2.11724512972467e-07
1920 2.1171065207426e-07
1921 2.11468209251109e-07
1922 2.11367545965402e-07
1923 2.11558452598126e-07
1924 2.11234478520339e-07
1925 2.11444637699287e-07
1926 2.11205904605549e-07
1927 2.11092604907037e-07
1928 2.11008436089344e-07
1929 2.11081963037429e-07
1930 2.10939091346063e-07
1931 2.10611467700517e-07
1932 2.10756167959403e-07
1933 2.1079672516322e-07
1934 2.10806285565468e-07
1935 2.10525705782061e-07
1936 2.10441943593764e-07
1937 2.10486054513126e-07
1938 2.10354152585523e-07
1939 2.10237939441527e-07
1940 2.09964359129344e-07
1941 2.09949239419416e-07
1942 2.09824008138071e-07
1943 2.09825059066304e-07
1944 2.09793111508816e-07
1945 2.09700543585711e-07
1946 2.09754597442213e-07
1947 2.09670173497045e-07
1948 2.0964833478132e-07
1949 2.09404227710763e-07
1950 2.09431415065353e-07
1951 2.0933758854369e-07
1952 2.09240786993803e-07
1953 2.09373009319336e-07
1954 2.08994799500317e-07
1955 2.09158107217888e-07
1956 2.08914607270572e-07
1957 2.08826088233138e-07
1958 2.08633518951729e-07
1959 2.08776670888255e-07
1960 2.08618614117029e-07
1961 2.0859886277691e-07
1962 2.08522925710497e-07
1963 2.08265814485742e-07
1964 2.08158993729057e-07
1965 2.0817488879743e-07
1966 2.0809151187251e-07
1967 2.08064630996319e-07
1968 2.0800519364883e-07
1969 2.07907418079856e-07
1970 2.07734938754811e-07
1971 2.07863039548783e-07
1972 2.07707496642229e-07
1973 2.07824264585099e-07
1974 2.07635781336535e-07
1975 2.07450170435663e-07
1976 2.0741758948617e-07
1977 2.07326376830963e-07
1978 2.0727377776808e-07
1979 2.07235073858669e-07
1980 2.07186273755156e-07
1981 2.07117118776523e-07
1982 2.0703223926688e-07
1983 2.06995867870319e-07
1984 2.06980355002884e-07
1985 2.06818433952094e-07
1986 2.06886384596316e-07
1987 2.06680300223638e-07
1988 2.06672409454711e-07
1989 2.06529951896073e-07
1990 2.06364807368686e-07
1991 2.06279887251526e-07
1992 2.0620779382341e-07
1993 2.06011688796082e-07
1994 2.0618447984333e-07
1995 2.0597208754225e-07
1996 2.0588623802098e-07
1997 2.05808801467811e-07
1998 2.05898638348856e-07
1999 2.05915041661342e-07
};
\addlegendentry{Train}
\addplot [semithick, black]
table {%
0 0.00545290252193809
1 0.0023266056086868
2 0.00213515129871666
3 0.0014157728292048
4 0.000410291773732752
5 0.000212521030334756
6 0.000190788457985036
7 0.000183048061444424
8 0.000176143643329851
9 0.000169325852766633
10 0.000161589094204828
11 0.000152619570144452
12 0.000141753058414906
13 0.000121109485917259
14 9.81760822469369e-05
15 7.54592983867042e-05
16 5.62838613404892e-05
17 4.32986780651845e-05
18 3.58362085535191e-05
19 3.13278542307671e-05
20 2.81511838693405e-05
21 2.57360643445281e-05
22 2.36443611356663e-05
23 2.17338802031009e-05
24 2.00070007849718e-05
25 1.84041437023552e-05
26 1.689850614639e-05
27 1.5482781236642e-05
28 1.41530126711586e-05
29 1.29242907860316e-05
30 1.18076804938028e-05
31 1.07529313027044e-05
32 9.72076668404043e-06
33 8.84199744177749e-06
34 8.15555267763557e-06
35 7.60991133574862e-06
36 6.93214678904042e-06
37 6.34968409940484e-06
38 5.87008207730833e-06
39 5.46938144907472e-06
40 5.12396627527778e-06
41 4.81088591186563e-06
42 4.53635993835633e-06
43 4.2803903852473e-06
44 4.04217689720099e-06
45 3.82559846912045e-06
46 3.61594788955699e-06
47 3.42828843713505e-06
48 3.25095720654645e-06
49 3.08387575387314e-06
50 2.93283324026561e-06
51 2.79320988738618e-06
52 2.65996595771867e-06
53 2.53490998147754e-06
54 2.42584314946725e-06
55 2.31981516662927e-06
56 2.22306312025466e-06
57 2.135601334885e-06
58 2.05362357519334e-06
59 1.97951840164023e-06
60 1.91189792531077e-06
61 1.8435728179611e-06
62 1.78373568360257e-06
63 1.73281296156347e-06
64 1.6810982970128e-06
65 1.63556671850529e-06
66 1.59371120389551e-06
67 1.55521115630108e-06
68 1.52398138197896e-06
69 1.49048457842582e-06
70 1.46019203839387e-06
71 1.43245529216074e-06
72 1.40674342219427e-06
73 1.38512632474885e-06
74 1.37241966058355e-06
75 1.35226548536593e-06
76 1.33385049139179e-06
77 1.31669241909549e-06
78 1.3001736078877e-06
79 1.28582246361475e-06
80 1.27185080600611e-06
81 1.25796304928372e-06
82 1.24765301734442e-06
83 1.23596134926629e-06
84 1.22478309094731e-06
85 1.21430900890118e-06
86 1.20441939088778e-06
87 1.1951461829085e-06
88 1.18600303267158e-06
89 1.17842034796922e-06
90 1.16994829113537e-06
91 1.16215221623861e-06
92 1.15464808914112e-06
93 1.14744250367949e-06
94 1.14050396859966e-06
95 1.13397800305393e-06
96 1.12736154278537e-06
97 1.1219150337638e-06
98 1.11617839593237e-06
99 1.11055123852566e-06
100 1.10428436528309e-06
101 1.099929704651e-06
102 1.09671827885904e-06
103 1.09289078409347e-06
104 1.08809376797581e-06
105 1.08510903373826e-06
106 1.08004962839914e-06
107 1.07730784293381e-06
108 1.07301877960708e-06
109 1.06946436062572e-06
110 1.06532627341949e-06
111 1.06181403225492e-06
112 1.05836693364836e-06
113 1.05506535419408e-06
114 1.05125116078852e-06
115 1.04815103441069e-06
116 1.04494552033429e-06
117 1.04213427221111e-06
118 1.03922991456784e-06
119 1.03576292076468e-06
120 1.03284810393234e-06
121 1.02973024240782e-06
122 1.0268323649143e-06
123 1.02429930848302e-06
124 1.02109629551705e-06
125 1.01827993148618e-06
126 1.01537966656906e-06
127 1.01420994269574e-06
128 1.01010982689331e-06
129 1.00738623132202e-06
130 1.0047942851088e-06
131 1.00281795312185e-06
132 1.00005240710743e-06
133 9.9704368494713e-07
134 9.94117840491526e-07
135 9.92156742540828e-07
136 9.89238515103352e-07
137 9.8655209512799e-07
138 9.83898303275055e-07
139 9.81608081929153e-07
140 9.79194737737998e-07
141 9.75912826106651e-07
142 9.73706960394338e-07
143 9.7080430805363e-07
144 9.67710093391361e-07
145 9.64837113315298e-07
146 9.62111926128273e-07
147 9.60053739618161e-07
148 9.57607994678256e-07
149 9.55455334405997e-07
150 9.52439222601242e-07
151 9.49674301864434e-07
152 9.47469459333661e-07
153 9.45233921356703e-07
154 9.42940062031994e-07
155 9.40317022468662e-07
156 9.37940569656348e-07
157 9.35726234274625e-07
158 9.32079558424448e-07
159 9.32240482143243e-07
160 9.29431223539723e-07
161 9.27131225125777e-07
162 9.25108963656385e-07
163 9.22836193240073e-07
164 9.20701950235525e-07
165 9.18571743113716e-07
166 9.1653021172533e-07
167 9.1469149765544e-07
168 9.11959375571314e-07
169 9.08875676941534e-07
170 9.11160384475806e-07
171 9.08082881778682e-07
172 9.03431214283046e-07
173 9.06380250853545e-07
174 9.01317491752707e-07
175 8.9896263943956e-07
176 8.96176061360165e-07
177 8.98884422895208e-07
178 8.96814071893459e-07
179 8.95787934496184e-07
180 8.90062665348523e-07
181 8.8801903075364e-07
182 8.86083284967754e-07
183 8.88968656909128e-07
184 8.8491037786298e-07
185 8.79909293871606e-07
186 8.83470363532979e-07
187 8.79027140854305e-07
188 8.76825197337894e-07
189 8.73556359692884e-07
190 8.77249817676784e-07
191 8.73542774115776e-07
192 8.69007124038035e-07
193 8.70745168413123e-07
194 8.6927383335933e-07
195 8.65830202201323e-07
196 8.66813763877872e-07
197 8.65223910295754e-07
198 8.65343565692456e-07
199 8.63232571646222e-07
200 8.59066119573981e-07
201 8.6080262917676e-07
202 8.58512976265047e-07
203 8.53314816140482e-07
204 8.55455709825037e-07
205 8.53785707022325e-07
206 8.49354989895801e-07
207 8.52364507863967e-07
208 8.50142555464117e-07
209 8.46311422719737e-07
210 8.4794891108686e-07
211 8.44281544232217e-07
212 8.45526699322363e-07
213 8.41456255784578e-07
214 8.42289637148497e-07
215 8.3898174807473e-07
216 8.39697861465538e-07
217 8.36104049994901e-07
218 8.37363586470019e-07
219 8.33189119475719e-07
220 8.33541150768724e-07
221 8.32260582228628e-07
222 8.31093473152578e-07
223 8.29913346933608e-07
224 8.26314078494761e-07
225 8.24172957436531e-07
226 8.266484314845e-07
227 8.22245681320055e-07
228 8.24394078335899e-07
229 8.20174932414375e-07
230 8.21676394480164e-07
231 8.19751448943862e-07
232 8.17600550817588e-07
233 8.18929095203202e-07
234 8.14558745787508e-07
235 8.15629562112008e-07
236 8.12827863683196e-07
237 8.14598877241224e-07
238 8.10266840289842e-07
239 8.12092139312881e-07
240 8.07582466677559e-07
241 8.06569687483716e-07
242 8.05487502475444e-07
243 8.04439650892164e-07
244 8.02184729309374e-07
245 8.03753266609419e-07
246 8.01033991137956e-07
247 8.00644329501665e-07
248 7.98470637164428e-07
249 7.98045675765024e-07
250 7.95695939359575e-07
251 7.96012784576305e-07
252 7.93840911228472e-07
253 7.92932894455589e-07
254 7.99726137756807e-07
255 7.97536699792545e-07
256 7.9637828775958e-07
257 7.9514381923218e-07
258 7.93691469880287e-07
259 7.92549712969048e-07
260 7.91949048561946e-07
261 7.90351464274863e-07
262 7.89906266618345e-07
263 7.88357112924132e-07
264 7.87908049915131e-07
265 7.86467808211455e-07
266 7.8583883578176e-07
267 7.84067367476382e-07
268 7.83190898800967e-07
269 7.82696929491067e-07
270 7.81262031068763e-07
271 7.80052346271987e-07
272 7.80344180384418e-07
273 7.79255685756652e-07
274 7.78398316469975e-07
275 7.7718851798636e-07
276 7.76482295350434e-07
277 7.7534531328638e-07
278 7.74429565808532e-07
279 7.73831459355279e-07
280 7.73414683408191e-07
281 7.72132750626042e-07
282 7.70859628573817e-07
283 7.65755316933792e-07
284 7.60040165914688e-07
285 7.6867144116477e-07
286 7.6819742389489e-07
287 7.66913956340431e-07
288 7.65923687140457e-07
289 7.64631579386332e-07
290 7.6424248618423e-07
291 7.62679974286584e-07
292 7.61422882078477e-07
293 7.63743344123213e-07
294 7.60434716085001e-07
295 7.59350712087326e-07
296 7.58960084112914e-07
297 7.57293321385077e-07
298 7.56752513098036e-07
299 7.56100291710027e-07
300 7.547943141617e-07
301 7.54836037231144e-07
302 7.5402886068332e-07
303 7.53300753331132e-07
304 7.51386835418089e-07
305 7.50217452605284e-07
306 7.48964851027267e-07
307 7.48231798297638e-07
308 7.46197088119516e-07
309 7.46279283703188e-07
310 7.44167550692509e-07
311 7.42736119718757e-07
312 7.41974929496791e-07
313 7.40465623039199e-07
314 7.39693291507137e-07
315 7.38051141979668e-07
316 7.37358220703754e-07
317 7.36179060822906e-07
318 7.34689365344821e-07
319 7.3422683044555e-07
320 7.3301083602928e-07
321 7.32796138436242e-07
322 7.34363652554748e-07
323 7.31105785689579e-07
324 7.30252622815897e-07
325 7.29460623460909e-07
326 7.28848874587129e-07
327 7.28102634184324e-07
328 7.27363214991783e-07
329 7.25585778127424e-07
330 7.25738800610998e-07
331 7.26057407973713e-07
332 7.23037885563826e-07
333 7.21743219855853e-07
334 7.23185678452865e-07
335 7.19886486422183e-07
336 7.19306910923478e-07
337 7.18626495199715e-07
338 7.17854049980815e-07
339 7.17089164936624e-07
340 7.16326042038418e-07
341 7.15051100996789e-07
342 7.13832491783251e-07
343 7.12931807811401e-07
344 7.12334440322593e-07
345 7.11622817561874e-07
346 7.11278175913321e-07
347 7.10306608198152e-07
348 7.09811786236969e-07
349 7.11279938059306e-07
350 7.08268146354385e-07
351 7.07583694747882e-07
352 7.07011736267305e-07
353 7.06253274529445e-07
354 7.05259594724339e-07
355 7.04499143466819e-07
356 7.03719081229792e-07
357 7.03026216797298e-07
358 7.02557599652209e-07
359 7.01623548593489e-07
360 7.00906184647465e-07
361 7.00190014413238e-07
362 6.99776137480512e-07
363 6.99556096606102e-07
364 6.98585381542216e-07
365 6.9796993784621e-07
366 6.97261725690623e-07
367 6.99188262842654e-07
368 6.95628955327265e-07
369 6.94950472279743e-07
370 6.9424356752279e-07
371 6.93555193720385e-07
372 6.92852438533009e-07
373 6.92171965965827e-07
374 6.9069432129254e-07
375 6.89652495111659e-07
376 6.88118404923443e-07
377 6.8726382096429e-07
378 6.86377575220831e-07
379 6.85503607655846e-07
380 6.8448548518063e-07
381 6.83920006849803e-07
382 6.83083499097847e-07
383 6.84049155097455e-07
384 6.83072642004845e-07
385 6.82362156112504e-07
386 6.81618757880642e-07
387 6.8105799755358e-07
388 6.8284236931504e-07
389 6.79476443110616e-07
390 6.78799324305146e-07
391 6.7815244619851e-07
392 6.77845548580081e-07
393 6.7688256422116e-07
394 6.76380921049713e-07
395 6.75689136642177e-07
396 6.75101830438507e-07
397 6.74114460252895e-07
398 6.73615545565553e-07
399 6.73901240588748e-07
400 6.72695762204967e-07
401 6.72247892907762e-07
402 6.71928603424021e-07
403 6.71328791668202e-07
404 6.70939130031911e-07
405 6.69477344672487e-07
406 6.71449413403025e-07
407 6.69191194901941e-07
408 6.68752818455687e-07
409 6.68569953177212e-07
410 6.68404140924395e-07
411 6.67407562104927e-07
412 6.6751169924828e-07
413 6.66729363274499e-07
414 6.65977097469295e-07
415 6.65666050281288e-07
416 6.65311461034435e-07
417 6.64649917325733e-07
418 6.64477511236328e-07
419 6.647004511251e-07
420 6.6377134544382e-07
421 6.63081834773038e-07
422 6.63052787786e-07
423 6.62266018025548e-07
424 6.62137495055504e-07
425 6.61343506180856e-07
426 6.61542742363963e-07
427 6.61179512917442e-07
428 6.60141040498274e-07
429 6.60491707549227e-07
430 6.60130638152623e-07
431 6.59555723814265e-07
432 6.59391389490338e-07
433 6.58980411571974e-07
434 6.58071940051741e-07
435 6.58389296859241e-07
436 6.57987527574733e-07
437 6.56790803077456e-07
438 6.57215821320278e-07
439 6.56909946883388e-07
440 6.56719976177556e-07
441 6.56668589726905e-07
442 6.5640790580801e-07
443 6.55567191643058e-07
444 6.54515929454647e-07
445 6.54193570426287e-07
446 6.56843781143834e-07
447 6.53864674404758e-07
448 6.53301640340942e-07
449 6.5313406594214e-07
450 6.5277481553494e-07
451 6.5202783616769e-07
452 6.52458425065561e-07
453 6.51807567919604e-07
454 6.5159122186742e-07
455 6.5107531099784e-07
456 6.51123741590709e-07
457 6.50599531581975e-07
458 6.49443961719953e-07
459 6.48274181003217e-07
460 6.47570630007976e-07
461 6.46804664938827e-07
462 6.49052708467934e-07
463 6.45696445644717e-07
464 6.45135514787398e-07
465 6.44291105800221e-07
466 6.43622058760229e-07
467 6.43751491224975e-07
468 6.42865359168354e-07
469 6.42117015559052e-07
470 6.41737983642088e-07
471 6.41614292362647e-07
472 6.41441090465378e-07
473 6.4141454458877e-07
474 6.41175120108528e-07
475 6.40506129911955e-07
476 6.39838845017948e-07
477 6.39653990219813e-07
478 6.39815482372796e-07
479 6.41032727344282e-07
480 6.38866424651496e-07
481 6.38831011201546e-07
482 6.38149401765986e-07
483 6.3791719639994e-07
484 6.37861091945524e-07
485 6.37615926279977e-07
486 6.36900097106263e-07
487 6.36524930541782e-07
488 6.365310127876e-07
489 6.36269589904259e-07
490 6.35659432646207e-07
491 6.35828826034412e-07
492 6.35466619769431e-07
493 6.35307401353202e-07
494 6.34728337445267e-07
495 6.34790467302082e-07
496 6.35972526197293e-07
497 6.34012621958391e-07
498 6.33850447684381e-07
499 6.33540821581846e-07
500 6.33391721294174e-07
501 6.33164347618731e-07
502 6.32965452496137e-07
503 6.3221284563042e-07
504 6.32333751582337e-07
505 6.32154183222156e-07
506 6.31997920663707e-07
507 6.31769921710656e-07
508 6.31397767847375e-07
509 6.31238776804821e-07
510 6.30889928743272e-07
511 6.30743727469962e-07
512 6.32443459380738e-07
513 6.3013061435413e-07
514 6.29995042800147e-07
515 6.2981672499518e-07
516 6.29599128387781e-07
517 6.29721682798845e-07
518 6.29218220637995e-07
519 6.28705720373546e-07
520 6.2858026694812e-07
521 6.28379609679541e-07
522 6.28293662430224e-07
523 6.2809175460643e-07
524 6.27842211997631e-07
525 6.27398890173936e-07
526 6.2723250948693e-07
527 6.26937492143043e-07
528 6.28939289981645e-07
529 6.26545329396322e-07
530 6.26345638465864e-07
531 6.26179598839371e-07
532 6.25897371264728e-07
533 6.25672953447065e-07
534 6.25465418124804e-07
535 6.2496366126652e-07
536 6.24926769887679e-07
537 6.24546885319432e-07
538 6.24579399755021e-07
539 6.24918584435363e-07
540 6.2390432731263e-07
541 6.23468963567575e-07
542 6.23123526111158e-07
543 6.22671734618052e-07
544 6.24919437086646e-07
545 6.2325915450856e-07
546 6.22872335043212e-07
547 6.22519849002856e-07
548 6.21752064944303e-07
549 6.217342729542e-07
550 6.21478420725907e-07
551 6.20621221969486e-07
552 6.20860419076052e-07
553 6.20620141944528e-07
554 6.20668402007141e-07
555 6.19987929439958e-07
556 6.19759759956651e-07
557 6.19452805494802e-07
558 6.1928477634865e-07
559 6.19068941887235e-07
560 6.18387105078e-07
561 6.18500280324952e-07
562 6.18287572251575e-07
563 6.18079980085895e-07
564 6.17784280620981e-07
565 6.17953389792092e-07
566 6.1889505786894e-07
567 6.17291220805782e-07
568 6.16302997968887e-07
569 6.16126442309906e-07
570 6.15949431903573e-07
571 6.15786518665118e-07
572 6.15377075519064e-07
573 6.15477460996772e-07
574 6.1502845483119e-07
575 6.15673684478679e-07
576 6.14998782566545e-07
577 6.14452005720523e-07
578 6.14236455476203e-07
579 6.14185807989998e-07
580 6.1378352711472e-07
581 6.14997929915262e-07
582 6.14049497471569e-07
583 6.16638430983585e-07
584 6.13176496244705e-07
585 6.12789108345169e-07
586 6.12657970577857e-07
587 6.12516032560961e-07
588 6.12329870364192e-07
589 6.13257270742906e-07
590 6.12553321843734e-07
591 6.11446353104839e-07
592 6.12476924288785e-07
593 6.11559414664953e-07
594 6.10906283782242e-07
595 6.10410722856614e-07
596 6.115710107224e-07
597 6.1076417523509e-07
598 6.09651294780633e-07
599 6.09507480930915e-07
600 6.0946035773668e-07
601 6.08792504408484e-07
602 6.10265146860911e-07
603 6.11023210694839e-07
604 6.07944400599081e-07
605 6.07807862706977e-07
606 6.07570029842464e-07
607 6.07509605288215e-07
608 6.07709750966023e-07
609 6.08056893725006e-07
610 6.07271772423701e-07
611 6.06553157922463e-07
612 6.06442824846454e-07
613 6.06012690695934e-07
614 6.0714904748238e-07
615 6.06214371146052e-07
616 6.06177366080374e-07
617 6.07457650403376e-07
618 6.05256047947478e-07
619 6.0460774875537e-07
620 6.04324441155768e-07
621 6.04349850163999e-07
622 6.04205752097187e-07
623 6.0389822920115e-07
624 6.03832006618177e-07
625 6.03472017246531e-07
626 6.04423235017748e-07
627 6.03690011757863e-07
628 6.03106968810607e-07
629 6.03620037509245e-07
630 6.05038735557173e-07
631 6.02228055868181e-07
632 6.0297810478005e-07
633 6.02302236529795e-07
634 6.01698047830723e-07
635 6.01424574142584e-07
636 6.01296676450147e-07
637 6.01155193180603e-07
638 6.02180421083176e-07
639 6.01698957325425e-07
640 6.01150418333418e-07
641 6.01528313382005e-07
642 6.01278316025855e-07
643 6.02671434535296e-07
644 6.00150428908819e-07
645 5.99733994022245e-07
646 5.99661120759265e-07
647 6.00461305566569e-07
648 5.99833185788157e-07
649 5.99230077114044e-07
650 6.00096200287226e-07
651 5.99760937802785e-07
652 5.98742701640731e-07
653 5.98508279381349e-07
654 5.98413521402108e-07
655 6.00307600961969e-07
656 5.99208988205646e-07
657 5.9886002645726e-07
658 5.9786253814309e-07
659 5.97718951667048e-07
660 5.97650227973645e-07
661 5.97425298565213e-07
662 5.98509018345794e-07
663 5.97649432165781e-07
664 5.96984591538785e-07
665 5.98047563471482e-07
666 5.99248778598849e-07
667 5.96569009303494e-07
668 5.97545295022428e-07
669 5.9671140206774e-07
670 5.96110908190894e-07
671 5.96000120367535e-07
672 5.95880180753738e-07
673 5.96623237925087e-07
674 5.96380345996295e-07
675 5.95231824718212e-07
676 5.95237111156166e-07
677 5.97152563841519e-07
678 5.94936125253298e-07
679 5.94859727698349e-07
680 5.94716709656495e-07
681 5.94584548707644e-07
682 5.9437667232487e-07
683 5.9537563856793e-07
684 5.95155654536939e-07
685 5.93817958360887e-07
686 5.9363321724959e-07
687 5.9358779935792e-07
688 5.93462743836426e-07
689 5.94861148783821e-07
690 5.95698338656803e-07
691 5.93657432546024e-07
692 5.93692732309137e-07
693 5.92485150718858e-07
694 5.9234383797957e-07
695 5.92119988596096e-07
696 5.92389937992266e-07
697 5.92152161971171e-07
698 5.91555249229714e-07
699 5.91135233207751e-07
700 5.91341006384027e-07
701 5.92960759604466e-07
702 5.91054345022712e-07
703 5.90943955103285e-07
704 5.90820093293587e-07
705 5.91612206335412e-07
706 5.91701848406956e-07
707 5.90149227264192e-07
708 5.9025711607319e-07
709 5.91935020111123e-07
710 5.89968635722471e-07
711 5.89754449720203e-07
712 5.89866260725103e-07
713 5.89776789183816e-07
714 5.89262015182612e-07
715 5.8970869076802e-07
716 5.89638091241795e-07
717 5.89734838740696e-07
718 5.91747323142044e-07
719 5.89610237966554e-07
720 5.89176181620132e-07
721 5.89352396218601e-07
722 5.88278851409996e-07
723 5.8876213415715e-07
724 5.88038915338984e-07
725 5.8802964986171e-07
726 5.88318243899266e-07
727 5.89761896208074e-07
728 5.87426711717853e-07
729 5.8808103631236e-07
730 5.87848489885801e-07
731 5.87884244396264e-07
732 5.86981116157403e-07
733 5.87135389196192e-07
734 5.87126464779431e-07
735 5.87201213875232e-07
736 5.88879629503936e-07
737 5.87199281198991e-07
738 5.86761871090857e-07
739 5.86817293424247e-07
740 5.86555074733042e-07
741 5.86496526011615e-07
742 5.86473561270395e-07
743 5.86287285386788e-07
744 5.86288592785422e-07
745 5.89418164054223e-07
746 5.85832538035902e-07
747 5.8585806073097e-07
748 5.85670420605311e-07
749 5.855621907358e-07
750 5.85262910135498e-07
751 5.85271152431233e-07
752 5.83928908781672e-07
753 5.85360396598844e-07
754 5.81364986373956e-07
755 5.80776543301909e-07
756 5.80145126605203e-07
757 5.81566894197749e-07
758 5.79263712552347e-07
759 5.79808386191871e-07
760 5.81231176965957e-07
761 5.78571416554041e-07
762 5.78457559186063e-07
763 5.7889229765351e-07
764 5.77471666929341e-07
765 5.76756917780585e-07
766 5.76549950892513e-07
767 5.78396111450274e-07
768 5.76447348521469e-07
769 5.77424771108781e-07
770 5.77398566292686e-07
771 5.77254866129806e-07
772 5.76768229620939e-07
773 5.76772606564191e-07
774 5.76198146973184e-07
775 5.76245497541095e-07
776 5.75830767957086e-07
777 5.75928595480946e-07
778 5.72834551348933e-07
779 5.7542240483599e-07
780 5.75229933019727e-07
781 5.74057423818886e-07
782 5.72719102365227e-07
783 5.72651799757296e-07
784 5.72460010062059e-07
785 5.72335011383984e-07
786 5.72119233765989e-07
787 5.71842576846393e-07
788 5.71697853501973e-07
789 5.72739793369692e-07
790 5.73536567571864e-07
791 5.73449256080494e-07
792 5.74572595724021e-07
793 5.72729845771391e-07
794 5.72715634916676e-07
795 5.74202090319886e-07
796 5.72578073843033e-07
797 5.70031659208325e-07
798 5.73329543840373e-07
799 5.69547637496726e-07
800 5.69431279018318e-07
801 5.71507939639559e-07
802 5.73008435367228e-07
803 5.71275677430094e-07
804 5.69388362237078e-07
805 5.69290648400056e-07
806 5.69099540825846e-07
807 5.6869367881518e-07
808 5.68106145237834e-07
809 5.68334144190885e-07
810 5.67995186884218e-07
811 5.6782442925396e-07
812 5.67627978398377e-07
813 5.69372161862702e-07
814 5.66663572953985e-07
815 5.65948880648648e-07
816 5.66036817417626e-07
817 5.66007372526656e-07
818 5.6613936294525e-07
819 5.65882544378837e-07
820 5.65571724564506e-07
821 5.67101153592375e-07
822 5.64328331620345e-07
823 5.63556113775121e-07
824 5.66130836432421e-07
825 5.66371227250784e-07
826 5.65962523069174e-07
827 5.65711673061742e-07
828 5.66119524592068e-07
829 5.65659092899296e-07
830 5.66916526167915e-07
831 5.66714390970446e-07
832 5.66741448437824e-07
833 5.62720458674448e-07
834 5.64456627216714e-07
835 5.65353104775568e-07
836 5.65390564588597e-07
837 5.6497725609006e-07
838 5.64390802537673e-07
839 5.64767617561301e-07
840 5.63056289593078e-07
841 5.64213735287922e-07
842 5.63533774311509e-07
843 5.6307862905669e-07
844 5.6283164440174e-07
845 5.62010143312364e-07
846 5.61933404696902e-07
847 5.60533635507454e-07
848 5.6022844319159e-07
849 5.60009311811882e-07
850 5.59437239644467e-07
851 5.59212821826804e-07
852 5.58256658678147e-07
853 5.59069121663924e-07
854 5.58304577680246e-07
855 5.57865178052452e-07
856 5.56761790448945e-07
857 5.57337102691235e-07
858 5.56612462787598e-07
859 5.56414647689962e-07
860 5.56511110971769e-07
861 5.55176029592985e-07
862 5.55042731775757e-07
863 5.54785401618574e-07
864 5.55442113636673e-07
865 5.54828716303746e-07
866 5.54895279947232e-07
867 5.5367985396515e-07
868 5.53154507088038e-07
869 5.53730615138193e-07
870 5.52109327145445e-07
871 5.53511995349254e-07
872 5.5303115686911e-07
873 5.52392123154277e-07
874 5.52241317564039e-07
875 5.51879850263504e-07
876 5.51945561255707e-07
877 5.52051972135814e-07
878 5.51117921077093e-07
879 5.50995196135773e-07
880 5.50676418242801e-07
881 5.50126912912674e-07
882 5.51181017272029e-07
883 5.50445520275389e-07
884 5.49261244486843e-07
885 5.48857997273444e-07
886 5.49612309441727e-07
887 5.50052504877385e-07
888 5.49227820556553e-07
889 5.45539819540863e-07
890 5.46463809314446e-07
891 5.4634932666886e-07
892 5.48130572042282e-07
893 5.48330831406929e-07
894 5.4727092901885e-07
895 5.47988236121455e-07
896 5.47743127299327e-07
897 5.48136142697331e-07
898 5.47735737654875e-07
899 5.4703644991605e-07
900 5.46721651062398e-07
901 5.46016963198781e-07
902 5.45688521924603e-07
903 5.45699890608375e-07
904 5.45776799754094e-07
905 5.45793909623171e-07
906 5.45984960353962e-07
907 5.45646230420971e-07
908 5.45014813724265e-07
909 5.45249292827066e-07
910 5.45130092177715e-07
911 5.44408010227926e-07
912 5.44778629318898e-07
913 5.4366034873965e-07
914 5.43680584996764e-07
915 5.42547923032544e-07
916 5.44338149666146e-07
917 5.42950374438078e-07
918 5.43864473456779e-07
919 5.42703617156803e-07
920 5.42690543170465e-07
921 5.42456575658434e-07
922 5.42157522431808e-07
923 5.41738302217709e-07
924 5.42770237643708e-07
925 5.41936628906114e-07
926 5.41604151749198e-07
927 5.41298390999145e-07
928 5.41550548405212e-07
929 5.40858309250325e-07
930 5.40623602773849e-07
931 5.40330916010134e-07
932 5.39751169981173e-07
933 5.39850816494436e-07
934 5.3937708344165e-07
935 5.39519589892734e-07
936 5.39133225174737e-07
937 5.39250038400496e-07
938 5.38690642315487e-07
939 5.38805863925518e-07
940 5.38885444711923e-07
941 5.38585823051108e-07
942 5.38507435976499e-07
943 5.38494930424349e-07
944 5.35418791969278e-07
945 5.35390086042753e-07
946 5.35538617896236e-07
947 5.34440005139913e-07
948 5.34343996605458e-07
949 5.34613775471371e-07
950 5.33481227193988e-07
951 5.35864103312633e-07
952 5.36457434918702e-07
953 5.35673450485774e-07
954 5.34892478754045e-07
955 5.35024867076572e-07
956 5.34868718204962e-07
957 5.3515969966611e-07
958 5.34078537839378e-07
959 5.34554658315756e-07
960 5.33747879671864e-07
961 5.33487707343738e-07
962 5.33504362465465e-07
963 5.33262664248468e-07
964 5.33256866219745e-07
965 5.33062234353565e-07
966 5.32688488874555e-07
967 5.33253626144869e-07
968 5.32400065367256e-07
969 5.31611306087143e-07
970 5.32070373537863e-07
971 5.30859381342452e-07
972 5.3179758197075e-07
973 5.29043461483525e-07
974 5.28667442267761e-07
975 5.28044097336533e-07
976 5.29317730979528e-07
977 5.30558850186935e-07
978 5.2702284847328e-07
979 5.27678878370352e-07
980 5.26765177255584e-07
981 5.26251199062244e-07
982 5.27006761785742e-07
983 5.2608476153182e-07
984 5.27299732766551e-07
985 5.29026408457867e-07
986 5.2704626796185e-07
987 5.27916654391447e-07
988 5.24872689311451e-07
989 5.24809308899421e-07
990 5.24568747550802e-07
991 5.26005692336184e-07
992 5.26118583366042e-07
993 5.26642679687939e-07
994 5.23658570728003e-07
995 5.23492019510741e-07
996 5.25036853105121e-07
997 5.24643724020279e-07
998 5.24569941262598e-07
999 5.25122061389993e-07
1000 5.22117659329524e-07
1001 5.21977483458613e-07
1002 5.23466781032766e-07
1003 5.23284256814804e-07
1004 5.23404025898344e-07
1005 5.23230255566887e-07
1006 5.2293518137958e-07
1007 5.22415689374611e-07
1008 5.22477762388007e-07
1009 5.22168363659148e-07
1010 5.21880735959712e-07
1011 5.22270795499935e-07
1012 5.21360561833717e-07
1013 5.21586116519757e-07
1014 5.21204128745012e-07
1015 5.21061565450509e-07
1016 5.20839364526182e-07
1017 5.20530306857836e-07
1018 5.20832259098825e-07
1019 5.19352568062459e-07
1020 5.19526565767592e-07
1021 5.19447951319307e-07
1022 5.19321929459693e-07
1023 5.18613319400174e-07
1024 5.18758611178782e-07
1025 5.18254637427162e-07
1026 5.1774827625195e-07
1027 5.17965304425161e-07
1028 5.1694195235541e-07
1029 5.17087698881369e-07
1030 5.16746808898461e-07
1031 5.16922568749578e-07
1032 5.16591853738646e-07
1033 5.15894782893156e-07
1034 5.16118404902954e-07
1035 5.15423437263962e-07
1036 5.15027522851597e-07
1037 5.15259216626873e-07
1038 5.14519229000143e-07
1039 5.14309647314803e-07
1040 5.14010991992109e-07
1041 5.13902023158153e-07
1042 5.13071654495434e-07
1043 5.13397253598669e-07
1044 5.12905501182104e-07
1045 5.12220140080899e-07
1046 5.1258484745631e-07
1047 5.11283076320979e-07
1048 5.10789334384754e-07
1049 5.10470954395714e-07
1050 5.10602433223539e-07
1051 5.12207975589263e-07
1052 5.11434222971729e-07
1053 5.10805421072291e-07
1054 5.11058203755965e-07
1055 5.10024790401076e-07
1056 5.10170423240197e-07
1057 5.09766380218935e-07
1058 5.09166284246021e-07
1059 5.09166738993372e-07
1060 5.07908112012956e-07
1061 5.07666868543311e-07
1062 5.06759420204617e-07
1063 5.06792787291488e-07
1064 5.04973115766916e-07
1065 5.05727598465455e-07
1066 5.0598282541614e-07
1067 5.05277341744659e-07
1068 5.05202081058087e-07
1069 5.05372042880481e-07
1070 5.04564980019495e-07
1071 5.02111845435138e-07
1072 5.03113653849141e-07
1073 5.03030776144442e-07
1074 5.02835405313817e-07
1075 5.02825969306286e-07
1076 5.02551699810283e-07
1077 5.02539080571296e-07
1078 5.00960368299275e-07
1079 5.00337080211466e-07
1080 5.00399039538024e-07
1081 5.0023061248794e-07
1082 4.99443444823555e-07
1083 4.99284737998096e-07
1084 4.98419353789359e-07
1085 4.9832397053251e-07
1086 4.97914356856199e-07
1087 4.98316921948572e-07
1088 4.97851431191521e-07
1089 4.96367420055321e-07
1090 4.95461108585005e-07
1091 4.95205426886969e-07
1092 4.9499578835821e-07
1093 4.94531434469536e-07
1094 4.9386062528356e-07
1095 4.93459708650335e-07
1096 4.92728474910109e-07
1097 4.92356264203409e-07
1098 4.92240872063121e-07
1099 4.92629737891548e-07
1100 4.90818877096899e-07
1101 4.9005092250809e-07
1102 4.90089860250009e-07
1103 4.89810929593659e-07
1104 4.89001422465662e-07
1105 4.89325941543939e-07
1106 4.8882139935813e-07
1107 4.87994384457124e-07
1108 4.874786441178e-07
1109 4.87335796606203e-07
1110 4.86527653720259e-07
1111 4.86302724311827e-07
1112 4.86134183574904e-07
1113 4.84899828734342e-07
1114 4.84834970393422e-07
1115 4.84820645851869e-07
1116 4.84749421048036e-07
1117 4.84267616229772e-07
1118 4.85831662899727e-07
1119 4.83894837088883e-07
1120 4.82986820316e-07
1121 4.82276732327591e-07
1122 4.82470284168812e-07
1123 4.81771905924688e-07
1124 4.80583878470497e-07
1125 4.80435460303852e-07
1126 4.80222354326543e-07
1127 4.79798018204747e-07
1128 4.79000732411805e-07
1129 4.78867434594576e-07
1130 4.79362370242598e-07
1131 4.77969763323927e-07
1132 4.77722835512395e-07
1133 4.77562480227789e-07
1134 4.76548819960954e-07
1135 4.76601059062887e-07
1136 4.75970381330626e-07
1137 4.75507363262295e-07
1138 4.74559328722535e-07
1139 4.76493539736111e-07
1140 4.76385736192242e-07
1141 4.73184513793967e-07
1142 4.72398113515737e-07
1143 4.71788354161617e-07
1144 4.71515619437923e-07
1145 4.71381184752317e-07
1146 4.73791033073212e-07
1147 4.70198699531466e-07
1148 4.69209339826193e-07
1149 4.68710112500048e-07
1150 4.68338242853861e-07
1151 4.68070936676668e-07
1152 4.69724170670816e-07
1153 4.67215670596488e-07
1154 4.66203033511192e-07
1155 4.65940814819987e-07
1156 4.65313064523798e-07
1157 4.65321903675431e-07
1158 4.64489971818693e-07
1159 4.62793138922279e-07
1160 4.66840305080041e-07
1161 4.66530309495283e-07
1162 4.66073970528669e-07
1163 4.68782616280805e-07
1164 4.65874194333082e-07
1165 4.64393053789536e-07
1166 4.64517540876841e-07
1167 4.63776700598828e-07
1168 4.63780679638148e-07
1169 4.62983365423497e-07
1170 4.66212725314108e-07
1171 4.62343081153449e-07
1172 4.6200273118302e-07
1173 4.61566173726169e-07
1174 4.60347820308016e-07
1175 4.59697844235052e-07
1176 4.59752783399381e-07
1177 4.58124844726626e-07
1178 4.57362745009959e-07
1179 4.57636502915193e-07
1180 4.5688616978623e-07
1181 4.55560154932755e-07
1182 4.55125189091632e-07
1183 4.5521957758865e-07
1184 4.53662721611181e-07
1185 4.53419744417261e-07
1186 4.56499776646524e-07
1187 4.52905425163408e-07
1188 4.51598026529609e-07
1189 4.50540738938798e-07
1190 4.49843327032795e-07
1191 4.49465403562499e-07
1192 4.48656265916725e-07
1193 4.48400300001595e-07
1194 4.48266234798211e-07
1195 4.4717853597831e-07
1196 4.46962530986639e-07
1197 4.46226920303161e-07
1198 4.45323166786693e-07
1199 4.44994867621062e-07
1200 4.44465541704631e-07
1201 4.4320916003926e-07
1202 4.43286182871816e-07
1203 4.42710899051235e-07
1204 4.4185742353875e-07
1205 4.41436100118153e-07
1206 4.4036394797331e-07
1207 4.39772463778354e-07
1208 4.39321439671403e-07
1209 4.38582475226212e-07
1210 4.38284871506767e-07
1211 4.37148742093996e-07
1212 4.36335284348388e-07
1213 4.36207102438857e-07
1214 4.35787995911596e-07
1215 4.34726842968303e-07
1216 4.34000327231843e-07
1217 4.33751665696036e-07
1218 4.37369521932851e-07
1219 4.32166643804521e-07
1220 4.32066514122198e-07
1221 4.31190329663877e-07
1222 4.31046629500997e-07
1223 4.30379287763571e-07
1224 4.29509668720129e-07
1225 4.29207631214013e-07
1226 4.29372221333324e-07
1227 4.28263376761606e-07
1228 4.27632784294474e-07
1229 4.26601985736852e-07
1230 4.24502331952681e-07
1231 4.2150401213803e-07
1232 4.21006092210519e-07
1233 4.2067617300745e-07
1234 4.20541283574494e-07
1235 4.20182942661995e-07
1236 4.1943224005081e-07
1237 4.20269742562596e-07
1238 4.18659539036526e-07
1239 4.18506203914149e-07
1240 4.17842301203564e-07
1241 4.16654586388177e-07
1242 4.15534003650464e-07
1243 4.15885182292186e-07
1244 4.15325075664441e-07
1245 4.14708807738862e-07
1246 4.14176554386358e-07
1247 4.13389869891034e-07
1248 4.12280201089743e-07
1249 4.11475696182606e-07
1250 4.10749578350078e-07
1251 4.10012660267967e-07
1252 4.09822547453587e-07
1253 4.08865417966808e-07
1254 4.08344845936881e-07
1255 4.07931992185695e-07
1256 4.07497253718248e-07
1257 4.06822834975173e-07
1258 4.06216514647895e-07
1259 4.05403142167415e-07
1260 4.05037781092688e-07
1261 4.04498308625989e-07
1262 4.03588586550541e-07
1263 4.03095498313633e-07
1264 4.02425428092101e-07
1265 4.0735289985605e-07
1266 4.01404975036712e-07
1267 4.01029694785393e-07
1268 4.0046052163234e-07
1269 4.0008575297179e-07
1270 3.99715304411075e-07
1271 4.04314874913325e-07
1272 3.98258691802766e-07
1273 3.97776432237151e-07
1274 3.97237471361223e-07
1275 3.9669322404734e-07
1276 3.96671623548173e-07
1277 3.95715744616609e-07
1278 3.95073357140063e-07
1279 3.94516433743775e-07
1280 3.93826780964446e-07
1281 3.93262268971739e-07
1282 3.98402391965647e-07
1283 3.91345565731172e-07
1284 3.91724739756683e-07
1285 3.91059359117207e-07
1286 3.9071142055036e-07
1287 3.90745128697745e-07
1288 3.89517936127959e-07
1289 3.88701749898246e-07
1290 3.88214118629548e-07
1291 3.87777703281245e-07
1292 3.8721310602341e-07
1293 3.92592227171917e-07
1294 3.86495315751745e-07
1295 3.85824193926965e-07
1296 3.853438954593e-07
1297 3.84869508707197e-07
1298 3.90073523703904e-07
1299 3.84068755465705e-07
1300 3.83522859692675e-07
1301 3.83131578018947e-07
1302 3.82578832613945e-07
1303 3.8770809851485e-07
1304 3.81700971274768e-07
1305 3.80980793579511e-07
1306 3.80954276124612e-07
1307 3.80211560013777e-07
1308 3.78298551595435e-07
1309 3.77773488935418e-07
1310 3.772863976792e-07
1311 3.82489474759495e-07
1312 3.76729474282911e-07
1313 3.75962997622992e-07
1314 3.7555793142019e-07
1315 3.7517170881074e-07
1316 3.83774789725067e-07
1317 3.78013623958395e-07
1318 3.77355149794312e-07
1319 3.76612319996639e-07
1320 3.75953419506914e-07
1321 3.75756400217142e-07
1322 3.75069504343628e-07
1323 3.7420718967951e-07
1324 3.73646997786636e-07
1325 3.72221307998188e-07
1326 3.72765953216003e-07
1327 3.76872520746474e-07
1328 3.7231802707538e-07
1329 3.71659467646168e-07
1330 3.74206507558483e-07
1331 3.70487839518319e-07
1332 3.70126230109236e-07
1333 3.7050972423458e-07
1334 3.69750381423728e-07
1335 3.69139939948582e-07
1336 3.68651228654926e-07
1337 3.68355813407106e-07
1338 3.67899644970748e-07
1339 3.71311188018808e-07
1340 3.67156218317177e-07
1341 3.67295854175609e-07
1342 3.6649146295531e-07
1343 3.65888126907521e-07
1344 3.65412688552169e-07
1345 3.68409871498443e-07
1346 3.65654244660618e-07
1347 3.63342962828028e-07
1348 3.62609654303014e-07
1349 3.63219754717647e-07
1350 3.6205551623425e-07
1351 3.65572248028911e-07
1352 3.63465943564734e-07
1353 3.63475407993974e-07
1354 3.60481834604798e-07
1355 3.60907392860099e-07
1356 3.59636629809756e-07
1357 3.62707226031489e-07
1358 3.60358427542451e-07
1359 3.59403543370718e-07
1360 3.58706074621296e-07
1361 3.58987080062434e-07
1362 3.58416684775875e-07
1363 3.60146174216425e-07
1364 3.56991307626231e-07
1365 3.57418684870936e-07
1366 3.56571860038457e-07
1367 3.56010730229173e-07
1368 3.56264962420028e-07
1369 3.56945918156271e-07
1370 3.5532474385036e-07
1371 3.54387509560183e-07
1372 3.54789449374948e-07
1373 3.54587001538675e-07
1374 3.55315819433599e-07
1375 3.55705850552113e-07
1376 3.56713229621164e-07
1377 3.61320928732312e-07
1378 3.54882246256238e-07
1379 3.53664262320308e-07
1380 3.52715318285846e-07
1381 3.54139473301984e-07
1382 3.54049262796252e-07
1383 3.52609475839927e-07
1384 3.51182791291649e-07
1385 3.50862592313206e-07
1386 3.53258855056993e-07
1387 3.5777523521574e-07
1388 3.50606171650725e-07
1389 3.50287649553138e-07
1390 3.49829633705667e-07
1391 3.50177629115933e-07
1392 3.51276952414992e-07
1393 3.50914945101977e-07
1394 3.51544343857313e-07
1395 3.51326349345982e-07
1396 3.50379082192376e-07
1397 3.49774722963048e-07
1398 3.52076057197337e-07
1399 3.50178396502088e-07
1400 3.48762483781684e-07
1401 3.48165997365868e-07
1402 3.39052093067949e-07
1403 3.2739535527071e-07
1404 3.24092439996093e-07
1405 3.2403619343313e-07
1406 3.23904714605305e-07
1407 3.22855953527323e-07
1408 3.22771427363477e-07
1409 3.2492141599505e-07
1410 3.22175367273303e-07
1411 3.21146018222862e-07
1412 3.20926432095803e-07
1413 3.21126094604551e-07
1414 3.20336084769224e-07
1415 3.23812372471366e-07
1416 3.18732162440938e-07
1417 3.18207327154596e-07
1418 3.17596970944578e-07
1419 3.19143055094173e-07
1420 3.19259413572581e-07
1421 3.15251952542894e-07
1422 3.14491472863665e-07
1423 3.13839649379588e-07
1424 3.13119898009973e-07
1425 3.15195194389162e-07
1426 3.14470668172362e-07
1427 3.10664916014503e-07
1428 3.10025427552318e-07
1429 3.10205336973013e-07
1430 3.10150170435008e-07
1431 3.13039038246643e-07
1432 3.12552970171964e-07
1433 3.11505431227488e-07
1434 3.08715186747577e-07
1435 3.0832282504889e-07
1436 3.12469751406752e-07
1437 3.07152760115059e-07
1438 3.07470827465295e-07
1439 3.07650054764963e-07
1440 3.15047458343543e-07
1441 3.07903860630176e-07
1442 3.06180368170317e-07
1443 3.09412001797682e-07
1444 3.11320917489866e-07
1445 3.0536835993189e-07
1446 3.04851766941283e-07
1447 3.04442096421553e-07
1448 3.10539434167367e-07
1449 3.02638540006228e-07
1450 3.02553303299646e-07
1451 3.01856886153473e-07
1452 3.01353367149204e-07
1453 3.05506972608782e-07
1454 3.03932608858304e-07
1455 3.02194564483216e-07
1456 3.00949039910847e-07
1457 2.98981774449203e-07
1458 2.99047911767047e-07
1459 2.98347458738135e-07
1460 2.99041005291656e-07
1461 3.06600071553476e-07
1462 2.98601378290186e-07
1463 2.97722067443829e-07
1464 2.97500861279332e-07
1465 3.02467924484517e-07
1466 2.95946819051096e-07
1467 2.98280156130204e-07
1468 3.00249126894414e-07
1469 2.97170373642075e-07
1470 2.95973762831636e-07
1471 2.96138580324623e-07
1472 2.95408540296194e-07
1473 2.95039541242659e-07
1474 2.98986975622029e-07
1475 2.97696828965854e-07
1476 2.98077935667607e-07
1477 2.97567652296493e-07
1478 2.97064019605386e-07
1479 2.97236596225048e-07
1480 2.9750694352515e-07
1481 2.96847900926878e-07
1482 2.96521136533556e-07
1483 2.96352055784155e-07
1484 2.96158475521224e-07
1485 2.95647367920537e-07
1486 2.95646572112673e-07
1487 2.94617109375395e-07
1488 2.94279629997618e-07
1489 2.94864094030345e-07
1490 2.94116034638137e-07
1491 2.91644141725556e-07
1492 2.93796404093882e-07
1493 2.90044027906333e-07
1494 2.94075732654164e-07
1495 2.93189202693611e-07
1496 2.89180547952128e-07
1497 2.91364898430402e-07
1498 2.89038950995746e-07
1499 2.91220857207009e-07
1500 2.92008593305582e-07
1501 2.91229639515223e-07
1502 2.88157394834343e-07
1503 2.90134465785741e-07
1504 2.87983283442372e-07
1505 2.88827436634165e-07
1506 2.9161247994125e-07
1507 2.8782923777726e-07
1508 2.9026671199972e-07
1509 2.86323114551124e-07
1510 2.86463404108872e-07
1511 2.86728237597345e-07
1512 2.88729324893211e-07
1513 2.90206713771113e-07
1514 2.85888518192223e-07
1515 2.8563971454787e-07
1516 2.85307521608047e-07
1517 2.89015844145979e-07
1518 2.84998378674572e-07
1519 2.84633273395229e-07
1520 2.84288489638129e-07
1521 2.9135452450646e-07
1522 2.84495570213039e-07
1523 2.81698135040642e-07
1524 2.82008670637879e-07
1525 2.8463526291489e-07
1526 2.81646322264351e-07
1527 2.81575239569065e-07
1528 2.81138511581958e-07
1529 2.87055030412375e-07
1530 2.83853580640425e-07
1531 2.82652990790666e-07
1532 2.8040736310686e-07
1533 2.79813150427799e-07
1534 2.79584327245175e-07
1535 2.80780909633904e-07
1536 2.79362751598455e-07
1537 2.80642638017525e-07
1538 2.79810137726599e-07
1539 2.81145759117862e-07
1540 2.82896451153647e-07
1541 2.85783329445621e-07
1542 2.78079227200578e-07
1543 2.78029091305143e-07
1544 2.79566506833362e-07
1545 2.81376202337924e-07
1546 2.77599696119069e-07
1547 2.77200769005503e-07
1548 2.77010087756935e-07
1549 2.81214198594171e-07
1550 2.76981126035025e-07
1551 2.76887220707067e-07
1552 2.76685682365496e-07
1553 2.82949542906863e-07
1554 2.76985247182893e-07
1555 2.77798562819953e-07
1556 2.77602396181464e-07
1557 2.79449636764184e-07
1558 2.76346355576607e-07
1559 2.76083540029504e-07
1560 2.76315148539652e-07
1561 2.82675443941116e-07
1562 2.76397912557513e-07
1563 2.75903772717356e-07
1564 2.75692286777485e-07
1565 2.78152896271422e-07
1566 2.75433848173634e-07
1567 2.74444403203233e-07
1568 2.74770599162366e-07
1569 2.81029485904583e-07
1570 2.75322435072667e-07
1571 2.74407597089521e-07
1572 2.74159219770809e-07
1573 2.76390011322292e-07
1574 2.75044527597856e-07
1575 2.7398601787354e-07
1576 2.73803664185834e-07
1577 2.79039966244454e-07
1578 2.72522271416165e-07
1579 2.71790156602947e-07
1580 2.72118057864645e-07
1581 2.73790618621206e-07
1582 2.71467058610142e-07
1583 2.73080331680831e-07
1584 2.70932986268235e-07
1585 2.73744433343381e-07
1586 2.74283678436404e-07
1587 2.70666049573265e-07
1588 2.7186533202439e-07
1589 2.71693522790883e-07
1590 2.73183530907772e-07
1591 2.74775828756901e-07
1592 2.71458560519022e-07
1593 2.7057848228651e-07
1594 2.71338819857192e-07
1595 2.7050214157498e-07
1596 2.73859370736318e-07
1597 2.73263651706657e-07
1598 2.70668522261985e-07
1599 2.7114180056742e-07
1600 2.71323614242647e-07
1601 2.7248006517766e-07
1602 2.68074131781759e-07
1603 2.6781214046423e-07
1604 2.71541665597397e-07
1605 2.68902226707723e-07
1606 2.70887738906822e-07
1607 2.69225552074204e-07
1608 2.7027158466808e-07
1609 2.71062702950076e-07
1610 2.69505846972606e-07
1611 2.7115478928863e-07
1612 2.71231868964605e-07
1613 2.68487355015168e-07
1614 2.67547079602082e-07
1615 2.6756663373817e-07
1616 2.69691497578606e-07
1617 2.68299345407286e-07
1618 2.66980379137749e-07
1619 2.68291017846423e-07
1620 2.6687305876294e-07
1621 2.66334609477781e-07
1622 2.69509939698764e-07
1623 2.66428202166935e-07
1624 2.65897341478194e-07
1625 2.68495142563552e-07
1626 2.65954298583893e-07
1627 2.68813607817719e-07
1628 2.6563159849502e-07
1629 2.65571941326925e-07
1630 2.65539910060397e-07
1631 2.65879407379543e-07
1632 2.68738858721917e-07
1633 2.66692694594894e-07
1634 2.64942059402529e-07
1635 2.64697575858008e-07
1636 2.6453983537067e-07
1637 2.67798554887122e-07
1638 2.63987203652505e-07
1639 2.64464830479483e-07
1640 2.63896339447456e-07
1641 2.66330829390427e-07
1642 2.65741505245387e-07
1643 2.6403833430777e-07
1644 2.64526022419886e-07
1645 2.63189775751016e-07
1646 2.63221693330706e-07
1647 2.6287074206266e-07
1648 2.63970150626847e-07
1649 2.68606754616485e-07
1650 2.64231289293093e-07
1651 2.62343604617854e-07
1652 2.62865370359577e-07
1653 2.6269111685906e-07
1654 2.65575579305732e-07
1655 2.6341760417381e-07
1656 2.61837442394608e-07
1657 2.66762356204708e-07
1658 2.62477158230467e-07
1659 2.63940762579296e-07
1660 2.62111711890611e-07
1661 2.61666542655803e-07
1662 2.61527560496688e-07
1663 2.63112326592818e-07
1664 2.61093731523943e-07
1665 2.63098542063744e-07
1666 2.61932996181713e-07
1667 2.61923332800507e-07
1668 2.61208271012947e-07
1669 2.60791750861245e-07
1670 2.63263530086988e-07
1671 2.64844544517473e-07
1672 2.61020062453099e-07
1673 2.57566142636279e-07
1674 2.60633157722623e-07
1675 2.6054345880766e-07
1676 2.60576655364275e-07
1677 2.61087905073509e-07
1678 2.62727411382002e-07
1679 2.59971699279049e-07
1680 2.60263448126352e-07
1681 2.60006402186264e-07
1682 2.59832773963353e-07
1683 2.59655536183345e-07
1684 2.6287921173207e-07
1685 2.59297678439907e-07
1686 2.59638284205721e-07
1687 2.61373486409866e-07
1688 2.59006156966279e-07
1689 2.61221174469028e-07
1690 2.5845375262179e-07
1691 2.58118660667606e-07
1692 2.57926387803309e-07
1693 2.58681126297233e-07
1694 2.58738140246351e-07
1695 2.57795875313604e-07
1696 2.57359118904787e-07
1697 2.57378616197457e-07
1698 2.57010867699137e-07
1699 2.56666112363746e-07
1700 2.57703760553341e-07
1701 2.57197541486676e-07
1702 2.56467274084571e-07
1703 2.57050459140373e-07
1704 2.56154237376904e-07
1705 2.55863767506526e-07
1706 2.55724870612539e-07
1707 2.58092796912024e-07
1708 2.5973614015129e-07
1709 2.56365638051648e-07
1710 2.56618250205065e-07
1711 2.55338022725482e-07
1712 2.55043346442108e-07
1713 2.55713558772186e-07
1714 2.57908652656624e-07
1715 2.55483371347509e-07
1716 2.56368707596266e-07
1717 2.54903909535642e-07
1718 2.54819070732992e-07
1719 2.54704673352535e-07
1720 2.54455073900317e-07
1721 2.56484156579972e-07
1722 2.55114770197906e-07
1723 2.57609258369484e-07
1724 2.55386567005189e-07
1725 2.53858758014758e-07
1726 2.56572889156814e-07
1727 2.53999530741567e-07
1728 2.54551792977509e-07
1729 2.53551206697011e-07
1730 2.53396194693778e-07
1731 2.49648138606062e-07
1732 2.54020761758511e-07
1733 2.53546886597178e-07
1734 2.56352961969242e-07
1735 2.53038564324015e-07
1736 2.52538882250519e-07
1737 2.52362553965213e-07
1738 2.51543326612591e-07
1739 2.52366646691371e-07
1740 2.51235860559973e-07
1741 2.54044550729304e-07
1742 2.51270051876418e-07
1743 2.5107783585554e-07
1744 2.51951945529072e-07
1745 2.52177187576308e-07
1746 2.51601960599146e-07
1747 2.50349273756001e-07
1748 2.50298825221762e-07
1749 2.51199168133098e-07
1750 2.53019209139893e-07
1751 2.51094633085813e-07
1752 2.52399473765763e-07
1753 2.50969435455772e-07
1754 2.50418963787524e-07
1755 2.51634617143282e-07
1756 2.48939926450475e-07
1757 2.49465330171006e-07
1758 2.49869600565944e-07
1759 2.49710581101681e-07
1760 2.49616988412527e-07
1761 2.51980338816793e-07
1762 2.49749376735053e-07
1763 2.49597576384986e-07
1764 2.49646092242983e-07
1765 2.50322329975461e-07
1766 2.49415791131469e-07
1767 2.50910460408704e-07
1768 2.50084525532657e-07
1769 2.48415204850971e-07
1770 2.49768788762594e-07
1771 2.49347920089349e-07
1772 2.48981677941629e-07
1773 2.53277818274e-07
1774 2.49216839165456e-07
1775 2.49223319315206e-07
1776 2.48684159487311e-07
1777 2.48089889964831e-07
1778 2.48095489041589e-07
1779 2.4900805328798e-07
1780 2.48000588953801e-07
1781 2.4713341417737e-07
1782 2.48195277663399e-07
1783 2.46798975922502e-07
1784 2.46669912939979e-07
1785 2.47801381192403e-07
1786 2.46351106625298e-07
1787 2.46228921696456e-07
1788 2.46044919549604e-07
1789 2.46388367486361e-07
1790 2.47602486069809e-07
1791 2.46268740511368e-07
1792 2.46042048956951e-07
1793 2.46828335548344e-07
1794 2.45649772523393e-07
1795 2.45419698785554e-07
1796 2.46863635311456e-07
1797 2.45189625047715e-07
1798 2.4504367956979e-07
1799 2.45190676650964e-07
1800 2.45301947643384e-07
1801 2.45715142455083e-07
1802 2.47005090159291e-07
1803 2.46501997480664e-07
1804 2.45786424102334e-07
1805 2.45919494545888e-07
1806 2.45920233510333e-07
1807 2.45745070515113e-07
1808 2.47261425556644e-07
1809 2.47633749950182e-07
1810 2.45451786895501e-07
1811 2.42386619220269e-07
1812 2.44998034304444e-07
1813 2.44952957473288e-07
1814 2.46470989395675e-07
1815 2.42788331661359e-07
1816 2.4480414140271e-07
1817 2.4502966766704e-07
1818 2.44707734964322e-07
1819 2.41932269773315e-07
1820 2.46093378564183e-07
1821 2.44617098132949e-07
1822 2.4433211365249e-07
1823 2.44449324782181e-07
1824 2.44248667513602e-07
1825 2.4423579247923e-07
1826 2.44418089323517e-07
1827 2.45353476202581e-07
1828 2.40531932149679e-07
1829 2.4442243784506e-07
1830 2.4422831756965e-07
1831 2.44139926053322e-07
1832 2.43150793721725e-07
1833 2.44416298755823e-07
1834 2.43259506760296e-07
1835 2.43724088022645e-07
1836 2.45612113758398e-07
1837 2.43773371266798e-07
1838 2.42879792722306e-07
1839 2.42779549353145e-07
1840 2.44273081762003e-07
1841 2.43316634396251e-07
1842 2.43476677042054e-07
1843 2.42456877685981e-07
1844 2.42788132709393e-07
1845 2.42315934428916e-07
1846 2.44108775859786e-07
1847 2.41986157334395e-07
1848 2.42224160729165e-07
1849 2.42075373080297e-07
1850 2.39630793430479e-07
1851 2.41834754888259e-07
1852 2.41784476884277e-07
1853 2.42651594817289e-07
1854 2.43045121806063e-07
1855 2.42715998410858e-07
1856 2.41910669274148e-07
1857 2.41815996560035e-07
1858 2.41410305079626e-07
1859 2.41759948949039e-07
1860 2.42874364175805e-07
1861 2.41664793065866e-07
1862 2.41591834537758e-07
1863 2.40746601321007e-07
1864 2.38654848772057e-07
1865 2.40478357227403e-07
1866 2.41306992165846e-07
1867 2.41398765865597e-07
1868 2.41765434338959e-07
1869 2.41143254697818e-07
1870 2.39346547914465e-07
1871 2.40739353785102e-07
1872 2.40265762840863e-07
1873 2.41458934624461e-07
1874 2.41926159105788e-07
1875 2.38505975858061e-07
1876 2.41078225826641e-07
1877 2.42432918184932e-07
1878 2.40723750266625e-07
1879 2.41129129108231e-07
1880 2.41077430018777e-07
1881 2.41197227524026e-07
1882 2.41061087535854e-07
1883 2.40390136241331e-07
1884 2.40363817738398e-07
1885 2.41020501334788e-07
1886 2.40821719899031e-07
1887 2.39908615640161e-07
1888 2.40554101083035e-07
1889 2.41766912267849e-07
1890 2.39663393131195e-07
1891 2.38389404216832e-07
1892 2.40043590338246e-07
1893 2.3943664473336e-07
1894 2.40164354181616e-07
1895 2.40080282765121e-07
1896 2.39243206578976e-07
1897 2.41806162648572e-07
1898 2.39869024198924e-07
1899 2.37614614206905e-07
1900 2.40141730500909e-07
1901 2.39995671336146e-07
1902 2.39426981352153e-07
1903 2.39668054291542e-07
1904 2.40252717276235e-07
1905 2.4018507360779e-07
1906 2.39772475651989e-07
1907 2.39127871282108e-07
1908 2.392215776581e-07
1909 2.40192662204208e-07
1910 2.39833184423333e-07
1911 2.39707816263035e-07
1912 2.39144554825543e-07
1913 2.39142451619045e-07
1914 2.39493459730511e-07
1915 2.38623528048265e-07
1916 2.39528532119948e-07
1917 2.38662948959245e-07
1918 2.38836918242669e-07
1919 2.38911354699667e-07
1920 2.36791905194877e-07
1921 2.38808496533238e-07
1922 2.39183322037206e-07
1923 2.39632981902105e-07
1924 2.39351379605068e-07
1925 2.39768496612669e-07
1926 2.38634868310328e-07
1927 2.38981755273926e-07
1928 2.38886116221693e-07
1929 2.3771487178692e-07
1930 2.38368286886725e-07
1931 2.36766069861005e-07
1932 2.39010375935322e-07
1933 2.38973882460414e-07
1934 2.38152054521379e-07
1935 2.38142902730942e-07
1936 2.38038765587589e-07
1937 2.38247864103869e-07
1938 2.37797124214012e-07
1939 2.38426764553878e-07
1940 2.3590759212766e-07
1941 2.3816996019832e-07
1942 2.37521248891426e-07
1943 2.36997294678076e-07
1944 2.36767689898443e-07
1945 2.37733800645401e-07
1946 2.36594999591944e-07
1947 2.36374262385652e-07
1948 2.34271070098657e-07
1949 2.3550892080948e-07
1950 2.3550563810204e-07
1951 2.36821009025334e-07
1952 2.36124634511725e-07
1953 2.37201248864949e-07
1954 2.35297477502172e-07
1955 2.34330030934871e-07
1956 2.35159575368016e-07
1957 2.35224604239193e-07
1958 2.35224277389534e-07
1959 2.33781378256026e-07
1960 2.34358950024216e-07
1961 2.35072477039466e-07
1962 2.35004577575637e-07
1963 2.35260202430254e-07
1964 2.34695633594129e-07
1965 2.35336344189818e-07
1966 2.34413008115553e-07
1967 2.34599568216254e-07
1968 2.34594423886847e-07
1969 2.35521142144535e-07
1970 2.3476965793634e-07
1971 2.37900735555741e-07
1972 2.34646307717412e-07
1973 2.34722293157574e-07
1974 2.34584206282307e-07
1975 2.34589421665987e-07
1976 2.34618738659265e-07
1977 2.33673276284208e-07
1978 2.35031635043015e-07
1979 2.36424568811344e-07
1980 2.35132830539442e-07
1981 2.35219673072606e-07
1982 2.35252755942383e-07
1983 2.35213718724481e-07
1984 2.34198381576789e-07
1985 2.35738738751934e-07
1986 2.35067872722539e-07
1987 2.34416447142394e-07
1988 2.32451881743145e-07
1989 2.34372564023033e-07
1990 2.3388844283545e-07
1991 2.33884549061258e-07
1992 2.35097274980944e-07
1993 2.34934645959584e-07
1994 2.34960111811233e-07
1995 2.34157354839226e-07
1996 2.3398304449529e-07
1997 2.35362392686511e-07
1998 2.33298024454598e-07
1999 2.34845217050861e-07
};
\addlegendentry{Test}

\nextgroupplot[
title={6 Layers $\rare$},
ymin=7.92387142645586e-08, ymax=1e-05,
]
\addplot [semithick, black, dashed]
table {%
0 0.0123398416563869
1 0.00360772215947509
2 0.00210050142742693
3 0.00136556964972988
4 0.000442093144753017
5 0.000167869987417362
6 0.000124253437606967
7 0.000105282618002093
8 8.82425548843457e-05
9 7.23779160434788e-05
10 5.86782249138196e-05
11 4.76242621680285e-05
12 3.96155822254514e-05
13 3.40272239263868e-05
14 3.00593712836417e-05
15 2.70074979544006e-05
16 2.46262789687535e-05
17 2.26235702912163e-05
18 2.07551111307112e-05
19 1.89469049219042e-05
20 1.72629617045459e-05
21 1.55224563714e-05
22 1.34426069894289e-05
23 1.11816280505082e-05
24 9.25691841734988e-06
25 7.7850572592979e-06
26 6.70509274459619e-06
27 5.87841706851577e-06
28 5.2445509604695e-06
29 4.72543595117258e-06
30 4.28144817021803e-06
31 3.89996458636688e-06
32 3.57282336398157e-06
33 3.29280803441634e-06
34 3.05551408234805e-06
35 2.85406894386142e-06
36 2.68418047295427e-06
37 2.54031582210246e-06
38 2.41697551297193e-06
39 2.31057065030882e-06
40 2.21868967804539e-06
41 2.14003459518608e-06
42 2.06655918418619e-06
43 1.99900723089286e-06
44 1.93939447126468e-06
45 1.88496475192323e-06
46 1.83620056145628e-06
47 1.79129818366164e-06
48 1.75127847552403e-06
49 1.71515877121919e-06
50 1.68159225785303e-06
51 1.65057209937913e-06
52 1.62135240503858e-06
53 1.5948543779416e-06
54 1.57120115275688e-06
55 1.54925050685506e-06
56 1.52899995879352e-06
57 1.50977627174598e-06
58 1.49110848238365e-06
59 1.47337877172049e-06
60 1.4562377309062e-06
61 1.43941254620472e-06
62 1.42394774411514e-06
63 1.40785664979148e-06
64 1.39268427284378e-06
65 1.37903029195741e-06
66 1.36608645095748e-06
67 1.35362228280655e-06
68 1.34251705645738e-06
69 1.33179243542259e-06
70 1.32129135297987e-06
71 1.31214604346042e-06
72 1.30292061902537e-06
73 1.2934876261852e-06
74 1.28448798386671e-06
75 1.27539245028174e-06
76 1.26704309988668e-06
77 1.25877617620063e-06
78 1.25072259191938e-06
79 1.24272126282676e-06
80 1.23494347258202e-06
81 1.22752130641857e-06
82 1.22015924381458e-06
83 1.2131271732585e-06
84 1.20648502098675e-06
85 1.19962771171345e-06
86 1.19339535402219e-06
87 1.18706668985169e-06
88 1.18075626366476e-06
89 1.17410247565886e-06
90 1.16842396803918e-06
91 1.16273461080141e-06
92 1.15702410062113e-06
93 1.15145153267804e-06
94 1.14598836117352e-06
95 1.14059906488251e-06
96 1.13532604845545e-06
97 1.13018968403367e-06
98 1.12509073164802e-06
99 1.12020114266898e-06
100 1.1151988950644e-06
101 1.11040179075417e-06
102 1.10560109081348e-06
103 1.10082693745994e-06
104 1.09681399024453e-06
105 1.09214436383809e-06
106 1.08762769178838e-06
107 1.08304108175616e-06
108 1.07847577979214e-06
109 1.07345339367271e-06
110 1.06841651171408e-06
111 1.06377246470402e-06
112 1.05899966413858e-06
113 1.05448434158006e-06
114 1.04996136724367e-06
115 1.04546922236182e-06
116 1.04111337768131e-06
117 1.03679478681329e-06
118 1.03259843754699e-06
119 1.02832679502285e-06
120 1.02422271547198e-06
121 1.01994421780205e-06
122 1.01589526281032e-06
123 1.01200209840613e-06
124 1.00812265904437e-06
125 1.00423639901237e-06
126 1.0006183156861e-06
127 9.96908712011191e-07
128 9.93364387824158e-07
129 9.89912331561982e-07
130 9.86483144629347e-07
131 9.83119658712894e-07
132 9.79884162006783e-07
133 9.76767184937444e-07
134 9.7369490725896e-07
135 9.70659178733513e-07
136 9.67568446554878e-07
137 9.6463568991112e-07
138 9.61823987196908e-07
139 9.59254379836238e-07
140 9.56425245192349e-07
141 9.53888832711414e-07
142 9.51217113481562e-07
143 9.4911108229212e-07
144 9.46706313655454e-07
145 9.44331015844568e-07
146 9.41654836225325e-07
147 9.39290385375102e-07
148 9.36975863311318e-07
149 9.3513510938692e-07
150 9.32744906435801e-07
151 9.30246976082572e-07
152 9.27845892064738e-07
153 9.26012294058864e-07
154 9.23915108316464e-07
155 9.22096822364438e-07
156 9.20012056781161e-07
157 9.18176443889251e-07
158 9.16536713688743e-07
159 9.14435626924615e-07
160 9.12789152408777e-07
161 9.10839209907977e-07
162 9.09275346472782e-07
163 9.07440475458543e-07
164 9.05768472264867e-07
165 9.04112482743358e-07
166 9.02640279690559e-07
167 9.00853698112769e-07
168 8.99370258679255e-07
169 8.97756142691719e-07
170 8.96330917015575e-07
171 8.94582797329235e-07
172 8.93360443697588e-07
173 8.91753264767203e-07
174 8.90304370585682e-07
175 8.88850426832732e-07
176 8.87540620254867e-07
177 8.85920816045882e-07
178 8.84618458428577e-07
179 8.83402567168901e-07
180 8.81530133113984e-07
181 8.79997025691637e-07
182 8.78698521418642e-07
183 8.77272766103943e-07
184 8.75888788385737e-07
185 8.74532885106305e-07
186 8.72763017611078e-07
187 8.71939544907718e-07
188 8.70648832801635e-07
189 8.69118241226374e-07
190 8.6761001894331e-07
191 8.65737089952745e-07
192 8.64395940283202e-07
193 8.62972824165809e-07
194 8.61815748592676e-07
195 8.60497948323768e-07
196 8.59068576360755e-07
197 8.57680632492475e-07
198 8.56372815150053e-07
199 8.55482535655483e-07
200 8.54045948813109e-07
201 8.52775942036033e-07
202 8.51577464914044e-07
203 8.50483469648111e-07
204 8.49151376996815e-07
205 8.47892317139554e-07
206 8.46595068665579e-07
207 8.45446409471151e-07
208 8.44041327994205e-07
209 8.42795710951805e-07
210 8.41699379236616e-07
211 8.40528489419512e-07
212 8.39388543099062e-07
213 8.38234008995187e-07
214 8.3714998153539e-07
215 8.35976411082129e-07
216 8.3478770331169e-07
217 8.33695449841798e-07
218 8.32547828380825e-07
219 8.31439585581961e-07
220 8.30390521372237e-07
221 8.29354354237921e-07
222 8.282217890212e-07
223 8.27147324216071e-07
224 8.26108056827479e-07
225 8.24817079390527e-07
226 8.23898754006791e-07
227 8.22899281658351e-07
228 8.21835116312286e-07
229 8.20744178270161e-07
230 8.19813014174997e-07
231 8.18755586379893e-07
232 8.17586724522812e-07
233 8.1664835921913e-07
234 8.1565755986901e-07
235 8.14638599820228e-07
236 8.13224320339145e-07
237 8.12251192542135e-07
238 8.11260346324616e-07
239 8.09920863176217e-07
240 8.08824044881362e-07
241 8.07823984786182e-07
242 8.06854697614767e-07
243 8.05676187496829e-07
244 8.04674441468478e-07
245 8.03679898226051e-07
246 8.02658337235584e-07
247 8.01664605035057e-07
248 8.00574525101183e-07
249 7.99637818502674e-07
250 7.98809292533065e-07
251 7.97785369414328e-07
252 7.96945076189104e-07
253 7.96230267283704e-07
254 7.9528897170178e-07
255 7.94540602896632e-07
256 7.93681930403522e-07
257 7.9284034066518e-07
258 7.91980470310705e-07
259 7.91013223462755e-07
260 7.90579809248015e-07
261 7.89608017612409e-07
262 7.88686629192625e-07
263 7.87752846662215e-07
264 7.87134427440606e-07
265 7.86087624106813e-07
266 7.85193357984326e-07
267 7.84498986888593e-07
268 7.83589775522842e-07
269 7.82732736823277e-07
270 7.81896471011123e-07
271 7.80963110670996e-07
272 7.8019500433868e-07
273 7.79207871445919e-07
274 7.78550015908763e-07
275 7.77850056536522e-07
276 7.76952427344213e-07
277 7.76165107780002e-07
278 7.75373747899266e-07
279 7.74126628414251e-07
280 7.73374092460699e-07
281 7.73038591631803e-07
282 7.72132863346542e-07
283 7.71416821947923e-07
284 7.70579882313882e-07
285 7.6975109843147e-07
286 7.68910115823473e-07
287 7.68078289553387e-07
288 7.67310924700837e-07
289 7.66442670752099e-07
290 7.65728364115148e-07
291 7.64954755595681e-07
292 7.64140844125905e-07
293 7.63392372050475e-07
294 7.62528119707895e-07
295 7.61791034051384e-07
296 7.60882000889751e-07
297 7.60180815021272e-07
298 7.59340204837144e-07
299 7.58593691813303e-07
300 7.57566362750595e-07
301 7.56649841804347e-07
302 7.55804310358599e-07
303 7.55054950190015e-07
304 7.54297078572108e-07
305 7.53630016646412e-07
306 7.52398388925712e-07
307 7.51176752771698e-07
308 7.50289109362257e-07
309 7.49504236182474e-07
310 7.48731221477783e-07
311 7.47967658284665e-07
312 7.46919737309781e-07
313 7.4609817229998e-07
314 7.4530783504656e-07
315 7.44552356849226e-07
316 7.43890507180822e-07
317 7.43151755060012e-07
318 7.4168742656866e-07
319 7.41013039032623e-07
320 7.40306250577305e-07
321 7.39633846762899e-07
322 7.38841126263878e-07
323 7.38145651695277e-07
324 7.37387897970621e-07
325 7.36856808529751e-07
326 7.3580377313931e-07
327 7.35254167366861e-07
328 7.34475432693671e-07
329 7.33676929115745e-07
330 7.32933696028226e-07
331 7.32307781419195e-07
332 7.31489909625793e-07
333 7.30722889855429e-07
334 7.29990452612128e-07
335 7.29176166316847e-07
336 7.28431064572987e-07
337 7.27648908494416e-07
338 7.26864246601622e-07
339 7.26030519984988e-07
340 7.25242693633277e-07
341 7.24513480705014e-07
342 7.23657729764682e-07
343 7.23016394871934e-07
344 7.22276114004217e-07
345 7.21526067252398e-07
346 7.20783161014538e-07
347 7.19575753677759e-07
348 7.18794834341452e-07
349 7.18024103022685e-07
350 7.17308066086275e-07
351 7.16577994779755e-07
352 7.15815944687392e-07
353 7.15017746586e-07
354 7.14531430475063e-07
355 7.13830752886224e-07
356 7.13069146087264e-07
357 7.12275111169447e-07
358 7.11477976366837e-07
359 7.10741716062557e-07
360 7.10119088012107e-07
361 7.0926111885683e-07
362 7.08540684854597e-07
363 7.07925718131719e-07
364 7.06917483185521e-07
365 7.0618375472975e-07
366 7.05466290426671e-07
367 7.04642714083548e-07
368 7.03897245344365e-07
369 7.03121250893446e-07
370 7.02332191238497e-07
371 7.01528421174658e-07
372 7.00759514288052e-07
373 7.00011131144151e-07
374 6.99252337440726e-07
375 6.98545427951558e-07
376 6.97781825394372e-07
377 6.97022407436521e-07
378 6.9629537826188e-07
379 6.95492926993779e-07
380 6.94744812420822e-07
381 6.93794321904306e-07
382 6.93057061639024e-07
383 6.9224964283876e-07
384 6.91527599258279e-07
385 6.90665346141373e-07
386 6.89863102991239e-07
387 6.89120396216936e-07
388 6.88291515643868e-07
389 6.87455651004143e-07
390 6.86665834408018e-07
391 6.85835697822768e-07
392 6.85124239893753e-07
393 6.84195066611437e-07
394 6.83331855086067e-07
395 6.82502957616293e-07
396 6.81661531373834e-07
397 6.8080163742934e-07
398 6.7997826995736e-07
399 6.78877194673078e-07
400 6.78056480893474e-07
401 6.77169937631561e-07
402 6.76364556966291e-07
403 6.75584698569764e-07
404 6.74597394947796e-07
405 6.74284166663597e-07
406 6.73051098772248e-07
407 6.71274516804488e-07
408 6.70739537724785e-07
409 6.69699444983962e-07
410 6.68575668896665e-07
411 6.6775409435138e-07
412 6.66595376159762e-07
413 6.65762377650481e-07
414 6.64780582340541e-07
415 6.63689676187573e-07
416 6.62799664723934e-07
417 6.61747141606384e-07
418 6.6069472818242e-07
419 6.59835131415321e-07
420 6.58589620002203e-07
421 6.57366593671327e-07
422 6.56393701149227e-07
423 6.55332765575167e-07
424 6.54158795470039e-07
425 6.53199885888967e-07
426 6.5208656533855e-07
427 6.51036094112101e-07
428 6.49968469019768e-07
429 6.48844476700106e-07
430 6.47807339561268e-07
431 6.46760100977417e-07
432 6.45596247750291e-07
433 6.44497259088439e-07
434 6.433561753596e-07
435 6.42225539394303e-07
436 6.41133589596166e-07
437 6.39810546459785e-07
438 6.38775579247408e-07
439 6.3756034668927e-07
440 6.36380549906335e-07
441 6.35243672789443e-07
442 6.3430030873235e-07
443 6.32866093894791e-07
444 6.317297352183e-07
445 6.30484729811087e-07
446 6.29462696409178e-07
447 6.28137939628459e-07
448 6.26715918528475e-07
449 6.25391289560184e-07
450 6.24072285546617e-07
451 6.2269883365218e-07
452 6.21436054117908e-07
453 6.20035492971738e-07
454 6.18857657329386e-07
455 6.17431679515334e-07
456 6.16079754337306e-07
457 6.14625397574287e-07
458 6.13345939441956e-07
459 6.11862008142339e-07
460 6.10512485422987e-07
461 6.08994147043518e-07
462 6.07573783710791e-07
463 6.06215617793282e-07
464 6.04699294626698e-07
465 6.03296843863177e-07
466 6.01748067438734e-07
467 6.0031973477237e-07
468 5.9878781857492e-07
469 5.97382314808215e-07
470 5.9577579196457e-07
471 5.94325738532575e-07
472 5.92763503988181e-07
473 5.91197139584665e-07
474 5.89631435119031e-07
475 5.88123770967286e-07
476 5.86450834603625e-07
477 5.84600659109924e-07
478 5.8307851119821e-07
479 5.81452565398877e-07
480 5.79780630587834e-07
481 5.7805403373834e-07
482 5.76431406486222e-07
483 5.74753422341701e-07
484 5.73046665763854e-07
485 5.71425639151357e-07
486 5.697364758106e-07
487 5.67982980527404e-07
488 5.66251877003765e-07
489 5.64536222611878e-07
490 5.62800366395777e-07
491 5.61047294937111e-07
492 5.59088146488307e-07
493 5.57157914883533e-07
494 5.55346546548208e-07
495 5.53520301579624e-07
496 5.51731898355001e-07
497 5.49932541431986e-07
498 5.481023330276e-07
499 5.46358477052422e-07
500 5.44578099322734e-07
501 5.42736495077634e-07
502 5.4095197809545e-07
503 5.39110673003051e-07
504 5.37254583420577e-07
505 5.35501339655298e-07
506 5.33602799720256e-07
507 5.31823470268478e-07
508 5.29969835724842e-07
509 5.28107183100701e-07
510 5.26223815953131e-07
511 5.2437986738596e-07
512 5.22512204554459e-07
513 5.20565799817518e-07
514 5.18808043707963e-07
515 5.16891418897103e-07
516 5.15039281381746e-07
517 5.13168640409845e-07
518 5.11311568615724e-07
519 5.09459701802939e-07
520 5.07520326038957e-07
521 5.0576319736706e-07
522 5.03923775610815e-07
523 5.0212953348705e-07
524 5.00181566636115e-07
525 4.98370244670809e-07
526 4.96724241401125e-07
527 4.94698783086278e-07
528 4.92938319155201e-07
529 4.91134695778328e-07
530 4.892951690465e-07
531 4.87495427947238e-07
532 4.85705594229557e-07
533 4.8392452566759e-07
534 4.81883468395949e-07
535 4.80116066853498e-07
536 4.78364070573889e-07
537 4.76641460252836e-07
538 4.74882958201306e-07
539 4.73087241871895e-07
540 4.71567121820726e-07
541 4.69733296498021e-07
542 4.67951570598757e-07
543 4.66116983744769e-07
544 4.64350625321686e-07
545 4.62599090923277e-07
546 4.60901670138014e-07
547 4.59152168687638e-07
548 4.57410101191158e-07
549 4.55810408070079e-07
550 4.53989538158339e-07
551 4.52248134763522e-07
552 4.50730590813464e-07
553 4.48978235681352e-07
554 4.47445664732982e-07
555 4.45807387450259e-07
556 4.44038560431181e-07
557 4.42562672617441e-07
558 4.41073630838673e-07
559 4.39399267932572e-07
560 4.37707108957852e-07
561 4.36279716751642e-07
562 4.35028157212969e-07
563 4.33443427780844e-07
564 4.31859852270122e-07
565 4.30324944460381e-07
566 4.28819917686951e-07
567 4.27373198931491e-07
568 4.2591783225987e-07
569 4.24509906324033e-07
570 4.23182340497874e-07
571 4.21792737839155e-07
572 4.20201634867112e-07
573 4.18938119139511e-07
574 4.17601124752309e-07
575 4.16363068410419e-07
576 4.15196928898354e-07
577 4.13978935199566e-07
578 4.12758511089351e-07
579 4.11527237361042e-07
580 4.10463838989017e-07
581 4.09194157242609e-07
582 4.08021058248664e-07
583 4.06895374936767e-07
584 4.05731228298123e-07
585 4.04599550734019e-07
586 4.03538217142341e-07
587 4.02434160022835e-07
588 4.01364498074486e-07
589 4.00405549896732e-07
590 3.99126805746164e-07
591 3.98028293105313e-07
592 3.96913842905633e-07
593 3.95838489936295e-07
594 3.94790818290858e-07
595 3.93743444675465e-07
596 3.92679299508814e-07
597 3.91662548409499e-07
598 3.90685843044025e-07
599 3.8968739994516e-07
600 3.88714190407313e-07
601 3.8774346158732e-07
602 3.86812633777822e-07
603 3.8609942130563e-07
604 3.85010267322627e-07
605 3.84061796509627e-07
606 3.83154700855926e-07
607 3.82258825936788e-07
608 3.81347371828156e-07
609 3.80447223264468e-07
610 3.79543508927327e-07
611 3.77813635097368e-07
612 3.76775835093213e-07
613 3.758352107468e-07
614 3.74939281726938e-07
615 3.73915380421863e-07
616 3.72618704687966e-07
617 3.71787549823921e-07
618 3.70949824656464e-07
619 3.70131405972529e-07
620 3.69299709333859e-07
621 3.68528754563613e-07
622 3.67765457568225e-07
623 3.67050118995849e-07
624 3.66250478990082e-07
625 3.65456196860237e-07
626 3.64667937873264e-07
627 3.63882003668436e-07
628 3.6319232349058e-07
629 3.62416382358788e-07
630 3.61619537017077e-07
631 3.60891155636978e-07
632 3.60111200862434e-07
633 3.59350442963091e-07
634 3.58822014248972e-07
635 3.58073606406606e-07
636 3.57403751465313e-07
637 3.56767227387422e-07
638 3.56043164870812e-07
639 3.55311229867539e-07
640 3.54665810746724e-07
641 3.53972306896821e-07
642 3.53376419411688e-07
643 3.52688594318806e-07
644 3.52072322229446e-07
645 3.51431733619734e-07
646 3.50775617008026e-07
647 3.50109221344042e-07
648 3.49560339202526e-07
649 3.48991831529588e-07
650 3.48334807995343e-07
651 3.47670573148662e-07
652 3.46942746375589e-07
653 3.46338182993122e-07
654 3.45625739370803e-07
655 3.4501001623255e-07
656 3.44364744393033e-07
657 3.43907511464181e-07
658 3.43234617332655e-07
659 3.42548892419359e-07
660 3.41984625407576e-07
661 3.41435325950101e-07
662 3.40836041246462e-07
663 3.40272571847322e-07
664 3.39671586488066e-07
665 3.39151766027612e-07
666 3.38610759371249e-07
667 3.38049284167141e-07
668 3.37522407960478e-07
669 3.36930625437049e-07
670 3.36414196794976e-07
671 3.35930577051613e-07
672 3.35386302708685e-07
673 3.34875307089533e-07
674 3.34375966360767e-07
675 3.33864348021962e-07
676 3.33381360022145e-07
677 3.32868674632891e-07
678 3.32386172374299e-07
679 3.31887720022905e-07
680 3.31543922214905e-07
681 3.30974561165931e-07
682 3.30465404701385e-07
683 3.29904953190407e-07
684 3.29479462621407e-07
685 3.28938486774177e-07
686 3.28516838919768e-07
687 3.27953090021538e-07
688 3.27506398008381e-07
689 3.27132376895634e-07
690 3.26671064499351e-07
691 3.26246610015346e-07
692 3.25777489834422e-07
693 3.2533895830511e-07
694 3.2487300535422e-07
695 3.24382418170899e-07
696 3.24012619415726e-07
697 3.23544308798773e-07
698 3.23179552154329e-07
699 3.22747996904127e-07
700 3.22264053664867e-07
701 3.22193268203819e-07
702 3.21524197886447e-07
703 3.21059943487967e-07
704 3.20674994327419e-07
705 3.20205105964533e-07
706 3.1974307613325e-07
707 3.19309772237375e-07
708 3.18932557476614e-07
709 3.185478318386e-07
710 3.18098331447914e-07
711 3.17747517030398e-07
712 3.1727523527536e-07
713 3.16919416164296e-07
714 3.16546084491165e-07
715 3.16140349582383e-07
716 3.15768016179163e-07
717 3.15422173954971e-07
718 3.15053878409799e-07
719 3.14667893221099e-07
720 3.14298181478989e-07
721 3.13905958932992e-07
722 3.13541600974077e-07
723 3.13145476127374e-07
724 3.12762507434172e-07
725 3.12395395056342e-07
726 3.12034190059762e-07
727 3.11666450102166e-07
728 3.11300413315507e-07
729 3.10940502657786e-07
730 3.10583937789488e-07
731 3.10193051035412e-07
732 3.09863917138387e-07
733 3.0948306691414e-07
734 3.09144899077296e-07
735 3.0877280732966e-07
736 3.08453979855017e-07
737 3.08083163417905e-07
738 3.07751636043463e-07
739 3.0740453733813e-07
740 3.07150088119101e-07
741 3.06779843455729e-07
742 3.06414445418568e-07
743 3.06068840288276e-07
744 3.05741716388752e-07
745 3.05390029240016e-07
746 3.05080904894339e-07
747 3.04742232501098e-07
748 3.04396438920662e-07
749 3.04099339189179e-07
750 3.03753318689814e-07
751 3.03359085975785e-07
752 3.03117749282933e-07
753 3.02777930613729e-07
754 3.02418874881027e-07
755 3.02095934841873e-07
756 3.01748356505982e-07
757 3.01433956323649e-07
758 3.01116354805231e-07
759 3.00784066666893e-07
760 3.00475649368082e-07
761 3.00165018870757e-07
762 2.99848480992182e-07
763 2.99526612380419e-07
764 2.99224345880589e-07
765 2.98981931749154e-07
766 2.98556235165393e-07
767 2.98254873229098e-07
768 2.97942650746563e-07
769 2.97628104050318e-07
770 2.97322775679731e-07
771 2.9703486772803e-07
772 2.96730180451732e-07
773 2.96409484903393e-07
774 2.96099737312261e-07
775 2.95690023321527e-07
776 2.95404941766719e-07
777 2.9509974686448e-07
778 2.94777930065493e-07
779 2.94495072893142e-07
780 2.94193936909437e-07
781 2.93867021710525e-07
782 2.93602116798297e-07
783 2.93284856667242e-07
784 2.92973450299883e-07
785 2.92705283399641e-07
786 2.92398243132652e-07
787 2.9206155690531e-07
788 2.91797079228218e-07
789 2.91504276106025e-07
790 2.91194727495281e-07
791 2.90920499452341e-07
792 2.90617249376623e-07
793 2.90361091344948e-07
794 2.90044165694781e-07
795 2.89765170535361e-07
796 2.8949163231573e-07
797 2.89171144459033e-07
798 2.88900393641711e-07
799 2.88604603426279e-07
800 2.88336111864851e-07
801 2.88042118015142e-07
802 2.8776393817509e-07
803 2.87476395939734e-07
804 2.87198647498599e-07
805 2.8688575939384e-07
806 2.86585516377613e-07
807 2.86201213839377e-07
808 2.85805951364182e-07
809 2.85463846537937e-07
810 2.85159966637138e-07
811 2.84865897100417e-07
812 2.8456612227501e-07
813 2.8424992549958e-07
814 2.83938474368028e-07
815 2.83641760745468e-07
816 2.83350172082919e-07
817 2.8306550549928e-07
818 2.82782726358732e-07
819 2.82500149396014e-07
820 2.82221300508922e-07
821 2.81941188291057e-07
822 2.81664812305848e-07
823 2.81332383494259e-07
824 2.81078259689593e-07
825 2.80868430166947e-07
826 2.80554127087385e-07
827 2.80323255410053e-07
828 2.80052333096137e-07
829 2.79764336482913e-07
830 2.79452063864483e-07
831 2.79268711054215e-07
832 2.79001107244881e-07
833 2.78250118824985e-07
834 2.78074789591187e-07
835 2.77819495707377e-07
836 2.77579969790054e-07
837 2.77308883056548e-07
838 2.77014791421948e-07
839 2.76749701598078e-07
840 2.76464072541671e-07
841 2.76196901353387e-07
842 2.75913546843753e-07
843 2.75645876996577e-07
844 2.75345277245265e-07
845 2.75092165992419e-07
846 2.74799840539686e-07
847 2.7454305036656e-07
848 2.74223540941421e-07
849 2.741593150688e-07
850 2.73987148176502e-07
851 2.7365873742724e-07
852 2.73366043302303e-07
853 2.73058126794012e-07
854 2.72782502051427e-07
855 2.72502463566582e-07
856 2.72225507075063e-07
857 2.71953454003437e-07
858 2.71670306908334e-07
859 2.71373237694661e-07
860 2.71096914559621e-07
861 2.70816866560608e-07
862 2.70536950189637e-07
863 2.70269339857521e-07
864 2.69996690711594e-07
865 2.69807924844656e-07
866 2.69595038687953e-07
867 2.69286189414686e-07
868 2.68993537716256e-07
869 2.68714597517317e-07
870 2.68438292522433e-07
871 2.68248457075515e-07
872 2.67958006958224e-07
873 2.67688443173597e-07
874 2.67419705949123e-07
875 2.6716403468896e-07
876 2.66889285434502e-07
877 2.66625092187667e-07
878 2.66342235384798e-07
879 2.66080529335966e-07
880 2.65836040085787e-07
881 2.65577441055598e-07
882 2.65310024765597e-07
883 2.65048144726165e-07
884 2.6478827484766e-07
885 2.64524492621376e-07
886 2.64262244364488e-07
887 2.64022590052093e-07
888 2.63761032030629e-07
889 2.63506283957327e-07
890 2.63256057891681e-07
891 2.63006246193243e-07
892 2.62748202352725e-07
893 2.62518559203784e-07
894 2.62263123133266e-07
895 2.62011357627046e-07
896 2.61752221703659e-07
897 2.61500343064824e-07
898 2.6126369032653e-07
899 2.61011602674444e-07
900 2.60842076862389e-07
901 2.60597423306308e-07
902 2.60331008703929e-07
903 2.60080088786196e-07
904 2.5982042413375e-07
905 2.59638311632671e-07
906 2.59370243298918e-07
907 2.59131393420375e-07
908 2.58878507978011e-07
909 2.586284382744e-07
910 2.58384251310417e-07
911 2.58150277986147e-07
912 2.57769369042649e-07
913 2.57647212521306e-07
914 2.57318894284708e-07
915 2.57098177456783e-07
916 2.56852950926145e-07
917 2.56574635791651e-07
918 2.56370015762286e-07
919 2.56130644032737e-07
920 2.55876666145127e-07
921 2.55613511903618e-07
922 2.55410837588954e-07
923 2.55175364515026e-07
924 2.54937162004865e-07
925 2.54705126891963e-07
926 2.54475290972778e-07
927 2.54200212204125e-07
928 2.54000696308765e-07
929 2.53842744214694e-07
930 2.53598822617107e-07
931 2.53356466210164e-07
932 2.53127537042985e-07
933 2.52811221429283e-07
934 2.52427714258374e-07
935 2.52272973000345e-07
936 2.52026854411724e-07
937 2.51824634936781e-07
938 2.51622281318475e-07
939 2.51395105756558e-07
940 2.51234001552803e-07
941 2.50961631749647e-07
942 2.5076156009618e-07
943 2.50625483992906e-07
944 2.502169812999e-07
945 2.50037652293145e-07
946 2.49802929303655e-07
947 2.49562258574088e-07
948 2.49325444713122e-07
949 2.49144056454043e-07
950 2.48893217900559e-07
951 2.48639447150367e-07
952 2.48441487698869e-07
953 2.48082349934009e-07
954 2.47975479481966e-07
955 2.4768462492375e-07
956 2.47564700870839e-07
957 2.47372659487155e-07
958 2.47132789894522e-07
959 2.46908639624621e-07
960 2.46581308672944e-07
961 2.46477646577148e-07
962 2.46244064932455e-07
963 2.46091606477705e-07
964 2.45829542855347e-07
965 2.45802998563249e-07
966 2.45484997392964e-07
967 2.45284709926352e-07
968 2.45175951413046e-07
969 2.44906190204119e-07
970 2.44714816922453e-07
971 2.4452370713135e-07
972 2.44378766183218e-07
973 2.44098992169484e-07
974 2.43881912744825e-07
975 2.43685883688727e-07
976 2.43533879910274e-07
977 2.4327140785374e-07
978 2.43051741186662e-07
979 2.42839710658416e-07
980 2.42508468005553e-07
981 2.42440107172115e-07
982 2.42097741107727e-07
983 2.4193771949399e-07
984 2.41695055009927e-07
985 2.41478524131367e-07
986 2.4129901852632e-07
987 2.40964668762444e-07
988 2.40853002885899e-07
989 2.407496066823e-07
990 2.40499989459408e-07
991 2.40357873536823e-07
992 2.40099384242853e-07
993 2.39951198352628e-07
994 2.39690216822908e-07
995 2.39610334162421e-07
996 2.39379361929082e-07
997 2.39156125566353e-07
998 2.38952522011004e-07
999 2.38884432562259e-07
1000 2.38585784444467e-07
1001 2.38403106763485e-07
1002 2.38165866953466e-07
1003 2.38104864962452e-07
1004 2.37824222878658e-07
1005 2.37611241431068e-07
1006 2.37382503591732e-07
1007 2.37322196490197e-07
1008 2.36982697174426e-07
1009 2.36816653703897e-07
1010 2.36570283178139e-07
1011 2.36513976759056e-07
1012 2.36228111504033e-07
1013 2.36023585742373e-07
1014 2.35831769991535e-07
1015 2.35772166320203e-07
1016 2.35367355394089e-07
1017 2.35188792736096e-07
1018 2.34997789419822e-07
1019 2.34870412185728e-07
1020 2.34615639222113e-07
1021 2.34439032588796e-07
1022 2.34270544055448e-07
1023 2.34068312252589e-07
1024 2.33870019158644e-07
1025 2.33677045784475e-07
1026 2.33565424935023e-07
1027 2.33333376698397e-07
1028 2.3317471845985e-07
1029 2.33097952914818e-07
1030 2.32894862278954e-07
1031 2.32659006258018e-07
1032 2.32505555374019e-07
1033 2.32347042228298e-07
1034 2.32104471336925e-07
1035 2.31888556953663e-07
1036 2.31614380624023e-07
1037 2.30973740855234e-07
1038 2.30657842692494e-07
1039 2.30479159952779e-07
1040 2.30408432727813e-07
1041 2.30105772274669e-07
1042 2.29909623342905e-07
1043 2.2977640654176e-07
1044 2.29498288874197e-07
1045 2.29089195443066e-07
1046 2.29251926285201e-07
1047 2.28941737148602e-07
1048 2.28751623339463e-07
1049 2.28532904245071e-07
1050 2.28477101146041e-07
1051 2.2819306599331e-07
1052 2.28039037878602e-07
1053 2.27836411724525e-07
1054 2.276378145325e-07
1055 2.27448589953383e-07
1056 2.27246072583398e-07
1057 2.27034318413644e-07
1058 2.26812596935133e-07
1059 2.2680265743702e-07
1060 2.26473747943601e-07
1061 2.2628586471285e-07
1062 2.26071283307761e-07
1063 2.25928652895391e-07
1064 2.25710423435999e-07
1065 2.25475910539785e-07
1066 2.25347102713158e-07
1067 2.2515615727059e-07
1068 2.25137285752908e-07
1069 2.24782535724444e-07
1070 2.24661100659773e-07
1071 2.24427793156678e-07
1072 2.24274829363935e-07
1073 2.24072470061287e-07
1074 2.23925563958005e-07
1075 2.23730815505974e-07
1076 2.23608094430006e-07
1077 2.23366170800432e-07
1078 2.231893802076e-07
1079 2.23023735614447e-07
1080 2.22849949224724e-07
1081 2.22686737018307e-07
1082 2.22520633428758e-07
1083 2.22335005247487e-07
1084 2.22150525303277e-07
1085 2.22055091128937e-07
1086 2.21798048954724e-07
1087 2.21616157134008e-07
1088 2.21456793894959e-07
1089 2.21292662303085e-07
1090 2.21104511666681e-07
1091 2.20957314553516e-07
1092 2.20805131547763e-07
1093 2.20601090703099e-07
1094 2.20364399311279e-07
1095 2.20231934605408e-07
1096 2.20033539775955e-07
1097 2.19860471815991e-07
1098 2.19712636919667e-07
1099 2.19505811827503e-07
1100 2.19337902677808e-07
1101 2.19153206160172e-07
1102 2.1898423165112e-07
1103 2.18753622249324e-07
1104 2.18725089737859e-07
1105 2.18500020665147e-07
1106 2.18370176284566e-07
1107 2.18146936930452e-07
1108 2.18021251143341e-07
1109 2.17804548086065e-07
1110 2.17678016326772e-07
1111 2.17548969217773e-07
1112 2.17279204363763e-07
1113 2.1720147081794e-07
1114 2.17086329115546e-07
1115 2.16860271748942e-07
1116 2.16729995628384e-07
1117 2.16540862169268e-07
1118 2.16407908880001e-07
1119 2.16230325747802e-07
1120 2.16115949555729e-07
1121 2.1595942379804e-07
1122 2.15778585513249e-07
1123 2.15643522565756e-07
1124 2.15532340490654e-07
1125 2.15402897651984e-07
1126 2.15150087115035e-07
1127 2.15088849685685e-07
1128 2.14832940798715e-07
1129 2.1470506757737e-07
1130 2.14599525847348e-07
1131 2.14406687746305e-07
1132 2.14300243968069e-07
1133 2.14102542628325e-07
1134 2.14026653182486e-07
1135 2.13775643317149e-07
1136 2.13782876883784e-07
1137 2.13464270345298e-07
1138 2.13315180580764e-07
1139 2.13209156200378e-07
1140 2.13064019092712e-07
1141 2.12898289568386e-07
1142 2.12745699229799e-07
1143 2.12688383193438e-07
1144 2.12465858410837e-07
1145 2.12385078555144e-07
1146 2.12293176218736e-07
1147 2.12006056983682e-07
1148 2.11916661278622e-07
1149 2.11980058153927e-07
1150 2.11884030861143e-07
1151 2.11617118871743e-07
1152 2.112586061358e-07
1153 2.11171311818248e-07
1154 2.1096920914232e-07
1155 2.1082047308596e-07
1156 2.10660570914456e-07
1157 2.1044929780345e-07
1158 2.10310349650911e-07
1159 2.10165095282377e-07
1160 2.09993791912666e-07
1161 2.09832154190792e-07
1162 2.09673616225814e-07
1163 2.09485054568859e-07
1164 2.09344186686167e-07
1165 2.09171541989406e-07
1166 2.09023334370784e-07
1167 2.08930179695699e-07
1168 2.08753241949466e-07
1169 2.08597197719484e-07
1170 2.08429741647365e-07
1171 2.08267057814737e-07
1172 2.0808713637166e-07
1173 2.079254313756e-07
1174 2.07795689149748e-07
1175 2.07635992175881e-07
1176 2.0749827036326e-07
1177 2.07348709153621e-07
1178 2.07179188585371e-07
1179 2.07169089989634e-07
1180 2.07031340437425e-07
1181 2.06863278172875e-07
1182 2.06569507966492e-07
1183 2.06430397795998e-07
1184 2.06312445925505e-07
1185 2.06330049124404e-07
1186 2.06031201670953e-07
1187 2.05874679060969e-07
1188 2.05710471625764e-07
1189 2.05498436571361e-07
1190 2.05410966415798e-07
1191 2.05203450796887e-07
1192 2.05071617287445e-07
1193 2.04938910805197e-07
1194 2.04828923017431e-07
1195 2.04655279809174e-07
1196 2.04512021227288e-07
1197 2.04365966908426e-07
1198 2.04117742399035e-07
1199 2.03960665942304e-07
1200 2.03814129100977e-07
1201 2.03772647829226e-07
1202 2.03666264788183e-07
1203 2.03508104270611e-07
1204 2.03349147732013e-07
1205 2.03150209621583e-07
1206 2.03126787255314e-07
1207 2.02897390053636e-07
1208 2.02838377447279e-07
1209 2.0267451319711e-07
1210 2.02508657778822e-07
1211 2.02402717150107e-07
1212 2.02260963938272e-07
1213 2.02210161823757e-07
1214 2.02068858584425e-07
1215 2.01892407503124e-07
1216 2.01766325048425e-07
1217 2.01637081630679e-07
1218 2.01501181948061e-07
1219 2.01316639163451e-07
1220 2.01153660086106e-07
1221 2.0104988934122e-07
1222 2.00931154189732e-07
1223 2.00789277052138e-07
1224 2.00659732428221e-07
1225 2.00515913768129e-07
1226 2.0029788255016e-07
1227 2.00344795686647e-07
1228 2.00198685597286e-07
1229 2.00074449246301e-07
1230 1.99938548384182e-07
1231 1.99799755918662e-07
1232 1.99659709956279e-07
1233 1.99559281817585e-07
1234 1.99423893995743e-07
1235 1.99313913419985e-07
1236 1.99170452411579e-07
1237 1.99045591401159e-07
1238 1.98900931358992e-07
1239 1.98819001454353e-07
1240 1.98748176380548e-07
1241 1.98609458145427e-07
1242 1.98477725966484e-07
1243 1.98351796328211e-07
1244 1.9824243444333e-07
1245 1.9811649588064e-07
1246 1.97992539547442e-07
1247 1.97872172876146e-07
1248 1.97749505133515e-07
1249 1.97651940780474e-07
1250 1.97499701720005e-07
1251 1.97405093707914e-07
1252 1.97263954916593e-07
1253 1.97157920325708e-07
1254 1.97048636017882e-07
1255 1.96917926764684e-07
1256 1.96793667932127e-07
1257 1.96715060489794e-07
1258 1.96592137719165e-07
1259 1.96464082051762e-07
1260 1.96466295371067e-07
1261 1.96306776430788e-07
1262 1.96098987416349e-07
1263 1.96080326254844e-07
1264 1.95934903054251e-07
1265 1.95697637273895e-07
1266 1.95718739433914e-07
1267 1.95516850894251e-07
1268 1.9523684051137e-07
1269 1.95158133244888e-07
1270 1.94998478377784e-07
1271 1.94928609154488e-07
1272 1.94832241078302e-07
1273 1.94735734076801e-07
1274 1.94623959380635e-07
1275 1.94509951086275e-07
1276 1.94323232634019e-07
1277 1.94323208695835e-07
1278 1.94214990060004e-07
1279 1.94108267912441e-07
1280 1.93981222565753e-07
1281 1.93884619463347e-07
1282 1.93681123803913e-07
1283 1.93679431752969e-07
1284 1.93581637475404e-07
1285 1.93447677105496e-07
1286 1.93401948173744e-07
1287 1.93253780132352e-07
1288 1.93163767946203e-07
1289 1.93035172891598e-07
1290 1.92941963149451e-07
1291 1.92810677383193e-07
1292 1.9271900448814e-07
1293 1.9256051786698e-07
1294 1.92521624065023e-07
1295 1.92417794039557e-07
1296 1.92320950247904e-07
1297 1.92171312434652e-07
1298 1.92055500178867e-07
1299 1.9193523799288e-07
1300 1.91841630694967e-07
1301 1.91736979637369e-07
1302 1.91579143340448e-07
1303 1.91517686928933e-07
1304 1.9140292111075e-07
1305 1.91393471489221e-07
1306 1.91193711451376e-07
1307 1.91094525185065e-07
1308 1.90995505931824e-07
1309 1.90871194469366e-07
1310 1.90773992294169e-07
1311 1.9065815221353e-07
1312 1.90573563081386e-07
1313 1.90469076322586e-07
1314 1.90359666518702e-07
1315 1.90263919876088e-07
1316 1.90148952889047e-07
1317 1.9005949324935e-07
1318 1.9002257427303e-07
1319 1.8984662683863e-07
1320 1.89768912505883e-07
1321 1.89639773751082e-07
1322 1.8958161913929e-07
1323 1.89467373310492e-07
1324 1.89361519034037e-07
1325 1.89279133095965e-07
1326 1.89166540927488e-07
1327 1.89092633739563e-07
1328 1.88992900717722e-07
1329 1.88886089198093e-07
1330 1.88727552959733e-07
1331 1.885757026443e-07
1332 1.88628137777869e-07
1333 1.88392807288551e-07
1334 1.88314237533405e-07
1335 1.88229621585378e-07
1336 1.8813704353704e-07
1337 1.88059274954355e-07
1338 1.8794331563754e-07
1339 1.87805289812104e-07
1340 1.87810978886205e-07
1341 1.87683005627548e-07
1342 1.87599580641518e-07
1343 1.87517155580963e-07
1344 1.87421135528609e-07
1345 1.87336384335879e-07
1346 1.87230721230947e-07
1347 1.87124710066655e-07
1348 1.87043291028033e-07
1349 1.86953630858966e-07
1350 1.86788690740514e-07
1351 1.86809593436976e-07
1352 1.86687570611355e-07
1353 1.86594739425061e-07
1354 1.86510800453732e-07
1355 1.8641386131435e-07
1356 1.86330591269268e-07
1357 1.86211805548453e-07
1358 1.86773451012812e-07
1359 1.86610066670312e-07
1360 1.86478089624131e-07
1361 1.86342251907945e-07
1362 1.86248776699927e-07
1363 1.8614239683501e-07
1364 1.86041232957734e-07
1365 1.85948041611539e-07
1366 1.85829228584566e-07
1367 1.85737652522278e-07
1368 1.85702878447103e-07
1369 1.85570095929677e-07
1370 1.85473617740683e-07
1371 1.85398439981554e-07
1372 1.85303314985674e-07
1373 1.85224384857463e-07
1374 1.85125399106312e-07
1375 1.8504310826728e-07
1376 1.84879282478789e-07
1377 1.84774766601947e-07
1378 1.84677771812858e-07
1379 1.84606017079147e-07
1380 1.84498252508547e-07
1381 1.84376395438335e-07
1382 1.84166458254253e-07
1383 1.84087084456053e-07
1384 1.83993317691034e-07
1385 1.83936501649384e-07
1386 1.8381967802128e-07
1387 1.83710159213035e-07
1388 1.83637281324422e-07
1389 1.83557458967698e-07
1390 1.83460022803672e-07
1391 1.83418098032462e-07
1392 1.83228504795352e-07
1393 1.83213409563621e-07
1394 1.83167969161957e-07
1395 1.83061302195142e-07
1396 1.8301862804293e-07
1397 1.82805986057133e-07
1398 1.82796682764774e-07
1399 1.82770012898459e-07
1400 1.82691416398484e-07
1401 1.82482264733608e-07
1402 1.82473924780879e-07
1403 1.82460220287339e-07
1404 1.82364723215755e-07
1405 1.82183307138928e-07
1406 1.82189850939096e-07
1407 1.82170219964917e-07
1408 1.81946718292636e-07
1409 1.81958008141692e-07
1410 1.81943611394786e-07
1411 1.81713198934119e-07
1412 1.81737528301085e-07
1413 1.81678990884393e-07
1414 1.81380147168397e-07
1415 1.81389601607407e-07
1416 1.81377186038389e-07
1417 1.81324101035329e-07
1418 1.81069013244439e-07
1419 1.811077833338e-07
1420 1.81097505937089e-07
1421 1.80902542162187e-07
1422 1.80937123133162e-07
1423 1.80877996349693e-07
1424 1.80639470940491e-07
1425 1.80640892942563e-07
1426 1.80591676254949e-07
1427 1.80380011435943e-07
1428 1.80449304380659e-07
1429 1.80427673292627e-07
1430 1.80255391732942e-07
1431 1.80217589075937e-07
1432 1.80152620494312e-07
1433 1.79941077057322e-07
1434 1.79968042992584e-07
1435 1.79901906868452e-07
1436 1.79886583801192e-07
1437 1.79714210815973e-07
1438 1.79711526882897e-07
1439 1.79912878166988e-07
1440 1.79583629865476e-07
1441 1.79499229908231e-07
1442 1.79507814891622e-07
1443 1.79297118506838e-07
1444 1.79244252407784e-07
1445 1.79227847993957e-07
1446 1.78968105707611e-07
1447 1.7900247299707e-07
1448 1.78769750604602e-07
1449 1.78865819201235e-07
1450 1.7880063033715e-07
1451 1.78544633463673e-07
1452 1.78598197145163e-07
1453 1.78401792865657e-07
1454 1.78434065830402e-07
1455 1.78366048331213e-07
1456 1.78140285775896e-07
1457 1.78203515595499e-07
1458 1.77971094807106e-07
1459 1.78050324372236e-07
1460 1.7780971526804e-07
1461 1.77921209989051e-07
1462 1.77666550534639e-07
1463 1.77759898008389e-07
1464 1.77513564764809e-07
1465 1.77602158949242e-07
1466 1.77373802863201e-07
1467 1.77450425105974e-07
1468 1.77199449588272e-07
1469 1.77275257598808e-07
1470 1.7705202498064e-07
1471 1.77119061113729e-07
1472 1.76876179985186e-07
1473 1.7704786913697e-07
1474 1.7675798182637e-07
1475 1.76824736890069e-07
1476 1.76611256925696e-07
1477 1.76728529964976e-07
1478 1.76524182201376e-07
1479 1.76595865724494e-07
1480 1.76404259548235e-07
1481 1.76456340788889e-07
1482 1.76306933823867e-07
1483 1.76336408536315e-07
1484 1.76196714932075e-07
1485 1.76192885668058e-07
1486 1.76075068551995e-07
1487 1.76166157871194e-07
1488 1.75953929272055e-07
1489 1.75986434989284e-07
1490 1.75775858807015e-07
1491 1.75752556025088e-07
1492 1.75652927502767e-07
1493 1.75713564964042e-07
1494 1.75486919360424e-07
1495 1.75520386598293e-07
1496 1.75389211662491e-07
1497 1.75413997467899e-07
1498 1.75264902225081e-07
1499 1.75113237027347e-07
1500 1.75202895363213e-07
1501 1.75074816354481e-07
1502 1.75088000922585e-07
1503 1.74870304853414e-07
1504 1.74747025930344e-07
1505 1.74869502302499e-07
1506 1.7468786565189e-07
1507 1.74707191128221e-07
1508 1.74517389872619e-07
1509 1.74380749129455e-07
1510 1.74522519913012e-07
1511 1.74258786138637e-07
1512 1.74290840931235e-07
1513 1.74103210873966e-07
1514 1.74006209924471e-07
1515 1.74082834810463e-07
1516 1.73903709779211e-07
1517 1.73821828866494e-07
1518 1.73876090848069e-07
1519 1.7366219113768e-07
1520 1.73573281330164e-07
1521 1.7364373422879e-07
1522 1.73455024537361e-07
1523 1.73495697019632e-07
1524 1.73301017866834e-07
1525 1.73244048887966e-07
1526 1.73309364988938e-07
1527 1.73116096327419e-07
1528 1.73142148554462e-07
1529 1.72971993386284e-07
1530 1.72845427748314e-07
1531 1.72923080803855e-07
1532 1.72746219163855e-07
1533 1.72728538672118e-07
1534 1.72655353331663e-07
1535 1.72742867874831e-07
1536 1.725679259863e-07
1537 1.72452967163395e-07
1538 1.72443164309755e-07
1539 1.72546531224782e-07
1540 1.72314361570614e-07
1541 1.72202190125148e-07
1542 1.72209017922142e-07
1543 1.72181644352065e-07
1544 1.71979930136956e-07
1545 1.72043918261977e-07
1546 1.71833369826402e-07
1547 1.7187695480203e-07
1548 1.71645923916941e-07
1549 1.71684587744636e-07
1550 1.7139633613894e-07
1551 1.71512161323051e-07
1552 1.71413870020842e-07
1553 1.7117135160305e-07
1554 1.71218849299493e-07
1555 1.70983939344893e-07
1556 1.71053899542528e-07
1557 1.70960183680791e-07
1558 1.70727139206406e-07
1559 1.70767318806497e-07
1560 1.70574188821604e-07
1561 1.70628794030847e-07
1562 1.70400770500123e-07
1563 1.70497390257651e-07
1564 1.70463938239607e-07
1565 1.70230595731624e-07
1566 1.70264121408081e-07
1567 1.70008069822813e-07
1568 1.70172444079242e-07
1569 1.70100166855036e-07
1570 1.6988550992636e-07
1571 1.69929653889511e-07
1572 1.69738526174967e-07
1573 1.69776392649368e-07
1574 1.69581807437424e-07
1575 1.69631272520121e-07
1576 1.69418131239496e-07
1577 1.69527494897181e-07
1578 1.69480565752167e-07
1579 1.69257869636397e-07
1580 1.69249107376146e-07
1581 1.69183881979507e-07
1582 1.69253553828241e-07
1583 1.6907159711721e-07
1584 1.69020149151322e-07
1585 1.68764463968074e-07
1586 1.68901994175741e-07
1587 1.68739855389788e-07
1588 1.68705280792381e-07
1589 1.6857423607064e-07
1590 1.68598350782645e-07
1591 1.68422638509469e-07
1592 1.68515732823948e-07
1593 1.68336841412042e-07
1594 1.68323941700521e-07
1595 1.68157324111462e-07
1596 1.68186502435219e-07
1597 1.680030032567e-07
1598 1.67946491764326e-07
1599 1.67836108193598e-07
1600 1.67806357509903e-07
1601 1.6771281821093e-07
1602 1.67692770375538e-07
1603 1.67595249145336e-07
1604 1.67589385757339e-07
1605 1.67577544779363e-07
1606 1.67307212027623e-07
1607 1.67382204548261e-07
1608 1.67282151185333e-07
1609 1.67144795625518e-07
1610 1.67130110625635e-07
1611 1.66978778171512e-07
1612 1.67040194678947e-07
1613 1.67072373603361e-07
1614 1.6678319262553e-07
1615 1.66924263844237e-07
1616 1.66684385558113e-07
1617 1.66777319122957e-07
1618 1.66790069190625e-07
1619 1.66676113124709e-07
1620 1.66669280798004e-07
1621 1.66599360799324e-07
1622 1.66385354990695e-07
1623 1.66441217817948e-07
1624 1.66180560462692e-07
1625 1.66261227324327e-07
1626 1.66214648821494e-07
1627 1.6602317053227e-07
1628 1.660935848804e-07
1629 1.65851368365111e-07
1630 1.65938582636471e-07
1631 1.65841543704914e-07
1632 1.6583449624008e-07
1633 1.65639953159769e-07
1634 1.65716209632905e-07
1635 1.65571626588701e-07
1636 1.65703231584047e-07
1637 1.65364120746858e-07
1638 1.65454749780025e-07
1639 1.65241769124691e-07
1640 1.65292669453976e-07
1641 1.65071191609911e-07
1642 1.65131631757021e-07
1643 1.64923105010928e-07
1644 1.64978135810401e-07
1645 1.64768182024488e-07
1646 1.64806550817076e-07
1647 1.64694278051769e-07
1648 1.64660206802836e-07
1649 1.64553321084782e-07
1650 1.64564729409733e-07
1651 1.64391995337354e-07
1652 1.64407669839051e-07
1653 1.64245727148682e-07
1654 1.64271331886567e-07
1655 1.64142924642618e-07
1656 1.64112026027397e-07
1657 1.63970966333693e-07
1658 1.64049510917863e-07
1659 1.6386011062508e-07
1660 1.63831361334132e-07
1661 1.63822453146167e-07
1662 1.63736013355731e-07
1663 1.63693351979077e-07
1664 1.63591714677835e-07
1665 1.63461851585112e-07
1666 1.63546576626317e-07
1667 1.63409231205947e-07
1668 1.63298056961025e-07
1669 1.63327065202168e-07
1670 1.6324493445552e-07
1671 1.63148550395675e-07
1672 1.63144620895395e-07
1673 1.63012900010528e-07
1674 1.62943387298498e-07
1675 1.63073880770526e-07
1676 1.62881854755881e-07
1677 1.62740741185274e-07
1678 1.62784837481667e-07
1679 1.62693943568826e-07
1680 1.62563687993611e-07
1681 1.62424298178365e-07
1682 1.62422038737731e-07
1683 1.6234580522223e-07
1684 1.62199692738341e-07
1685 1.62087934626243e-07
1686 1.62192182358467e-07
1687 1.61979586479788e-07
1688 1.61954984662316e-07
1689 1.61900072740195e-07
1690 1.61794866823328e-07
1691 1.61725607121355e-07
1692 1.61704289347142e-07
1693 1.61592031034274e-07
1694 1.61524338675889e-07
1695 1.6145344923757e-07
1696 1.61361668386206e-07
1697 1.61482540445235e-07
1698 1.61326499998893e-07
1699 1.61207344362424e-07
1700 1.61140864570086e-07
1701 1.61105996056676e-07
1702 1.6113932353079e-07
1703 1.609544891501e-07
1704 1.60849132495144e-07
1705 1.60763313491685e-07
1706 1.6071901994863e-07
1707 1.60810745256867e-07
1708 1.60631775713682e-07
1709 1.60569247718456e-07
1710 1.60477954921134e-07
1711 1.60333691923142e-07
1712 1.60313758264863e-07
1713 1.6037918821965e-07
1714 1.60256751648546e-07
1715 1.60181035173679e-07
1716 1.60118059870484e-07
1717 1.60005627435567e-07
1718 1.59937973496227e-07
1719 1.60019609030115e-07
1720 1.59870893419622e-07
1721 1.59783479364961e-07
1722 1.59671755675106e-07
1723 1.59656329611124e-07
1724 1.59655701214234e-07
1725 1.59502004962064e-07
1726 1.59363465797924e-07
1727 1.59326866366882e-07
1728 1.59304694676621e-07
1729 1.59294036087942e-07
1730 1.59164855404015e-07
1731 1.59147519653402e-07
1732 1.59097798409391e-07
1733 1.59025419442571e-07
1734 1.58970847000006e-07
1735 1.58882711364328e-07
1736 1.58867267138874e-07
1737 1.58771086510967e-07
1738 1.58634808961722e-07
1739 1.5866641301443e-07
1740 1.5857230930294e-07
1741 1.58510713646365e-07
1742 1.58518102281846e-07
1743 1.58398518365743e-07
1744 1.58315463565373e-07
1745 1.5842065335292e-07
1746 1.58169357131754e-07
1747 1.58169894458382e-07
1748 1.58048877544559e-07
1749 1.57964022466928e-07
1750 1.57969891837695e-07
1751 1.57903288538819e-07
1752 1.57906339822489e-07
1753 1.57742084979873e-07
1754 1.57686156079251e-07
1755 1.57654101499816e-07
1756 1.57558567295268e-07
1757 1.57426886932654e-07
1758 1.57345728378999e-07
1759 1.57274391927587e-07
1760 1.57290221146411e-07
1761 1.57236021443907e-07
1762 1.57136638730293e-07
1763 1.57081835634898e-07
1764 1.5702763262837e-07
1765 1.56831978223693e-07
1766 1.56877517909493e-07
1767 1.56808450977763e-07
1768 1.56708983631404e-07
1769 1.56672710083683e-07
1770 1.56606075677246e-07
1771 1.56533712708296e-07
1772 1.56460712084083e-07
1773 1.56332773325829e-07
1774 1.56338967055802e-07
1775 1.56272849725525e-07
1776 1.56208304492367e-07
1777 1.56145347627756e-07
1778 1.56017951461251e-07
1779 1.5601533269205e-07
1780 1.55954051447083e-07
1781 1.55878208921934e-07
1782 1.55821872077411e-07
1783 1.55673321749816e-07
1784 1.55648443659118e-07
1785 1.55610423547614e-07
1786 1.55522806242203e-07
1787 1.55482352134584e-07
1788 1.5536468560029e-07
1789 1.55246340373338e-07
1790 1.55279651131934e-07
1791 1.55227879048425e-07
1792 1.55160593521941e-07
1793 1.54964105707478e-07
1794 1.5506129497922e-07
1795 1.54960544538341e-07
1796 1.54910816196008e-07
1797 1.54783226175681e-07
1798 1.54649468772305e-07
1799 1.54696845754643e-07
1800 1.5458550360492e-07
1801 1.54398250700183e-07
1802 1.543687631127e-07
1803 1.54425852933571e-07
1804 1.54366236245096e-07
1805 1.54253686808659e-07
1806 1.54039521465421e-07
1807 1.5414835046812e-07
1808 1.54069335394524e-07
1809 1.5401819128158e-07
1810 1.53877105766753e-07
1811 1.5389234204477e-07
1812 1.53802611674081e-07
1813 1.53761284419573e-07
1814 1.53710501621163e-07
1815 1.53624744644532e-07
1816 1.53444115564838e-07
1817 1.53506436845419e-07
1818 1.53417284220581e-07
1819 1.53398264679083e-07
1820 1.53183080286112e-07
1821 1.53275587813084e-07
1822 1.53152266783252e-07
1823 1.53061549667655e-07
1824 1.53144484954737e-07
1825 1.53134541204025e-07
1826 1.52996582166054e-07
1827 1.53014466789614e-07
1828 1.5292361226571e-07
1829 1.52746858159247e-07
1830 1.52800923487462e-07
1831 1.52824118085704e-07
1832 1.52600269057501e-07
1833 1.52622743108566e-07
1834 1.52593430772896e-07
1835 1.52509476819773e-07
1836 1.52436520146182e-07
1837 1.52394736058881e-07
1838 1.52312047656977e-07
1839 1.52401552696801e-07
1840 1.52052451984019e-07
1841 1.52073565828914e-07
1842 1.52034193611428e-07
1843 1.52023549720326e-07
1844 1.51928160288861e-07
1845 1.51808649448526e-07
1846 1.51736981933936e-07
1847 1.51831029036487e-07
1848 1.51487091770264e-07
1849 1.51531141352024e-07
1850 1.51449768658551e-07
1851 1.51361571518294e-07
1852 1.51288809064454e-07
1853 1.51212613573648e-07
1854 1.51171464597866e-07
1855 1.51074828359299e-07
1856 1.50993570379399e-07
1857 1.50924360116989e-07
1858 1.50848303746898e-07
1859 1.50793979852182e-07
1860 1.5081039794751e-07
1861 1.50612742530143e-07
1862 1.50564313358359e-07
1863 1.50482031951782e-07
1864 1.5036384795053e-07
1865 1.50299122122277e-07
1866 1.50201364601088e-07
1867 1.50106459464894e-07
1868 1.50020361761705e-07
1869 1.4994802844015e-07
1870 1.49862979533566e-07
1871 1.49695719702692e-07
1872 1.49573582248763e-07
1873 1.49498770394274e-07
1874 1.49368556826346e-07
1875 1.49260665914142e-07
1876 1.49067523139479e-07
1877 1.48949050878144e-07
1878 1.48832889792061e-07
1879 1.48887304934675e-07
1880 1.48697200739178e-07
1881 1.48530737732244e-07
1882 1.48449142979956e-07
1883 1.48426813609603e-07
1884 1.4828651639931e-07
1885 1.48368356335027e-07
1886 1.48142760401981e-07
1887 1.48003868776669e-07
1888 1.47926880888605e-07
1889 1.4786842119463e-07
1890 1.4794813456831e-07
1891 1.47951928326506e-07
1892 1.47707737344405e-07
1893 1.47628438327274e-07
1894 1.47523289008689e-07
1895 1.4749361934463e-07
1896 1.47567437295493e-07
1897 1.47592275247632e-07
1898 1.47526118681185e-07
1899 1.47478364784348e-07
1900 1.47465641745725e-07
1901 1.47467974397131e-07
1902 1.47343922506593e-07
1903 1.47294724953895e-07
1904 1.47234271732799e-07
1905 1.46899602583517e-07
1906 1.46904439489504e-07
1907 1.46938974012301e-07
1908 1.46913332194742e-07
1909 1.46588343469034e-07
1910 1.46729204328011e-07
1911 1.4665558854432e-07
1912 1.4663587132091e-07
1913 1.46575698060758e-07
1914 1.46294625370302e-07
1915 1.46345442480822e-07
1916 1.4633276654763e-07
1917 1.462964117529e-07
1918 1.4626782707694e-07
1919 1.46223712839344e-07
1920 1.45866737504718e-07
1921 1.45951736588046e-07
1922 1.45909657017995e-07
1923 1.45917568719511e-07
1924 1.45817347650734e-07
1925 1.45747875009761e-07
1926 1.45414354136619e-07
1927 1.45655360132224e-07
1928 1.45678890245193e-07
1929 1.45603154887652e-07
1930 1.45277113244191e-07
1931 1.45349349004675e-07
1932 1.45321729355885e-07
1933 1.45322160175709e-07
1934 1.45261342343161e-07
1935 1.45142149147404e-07
1936 1.44809298124216e-07
1937 1.44930775281438e-07
1938 1.44826266868847e-07
1939 1.44845201774046e-07
1940 1.44673734961742e-07
1941 1.44501458766655e-07
1942 1.44621464130523e-07
1943 1.44489650327984e-07
1944 1.44545015512421e-07
1945 1.44345174895477e-07
1946 1.44334716846828e-07
1947 1.44032712452002e-07
1948 1.44092405264473e-07
1949 1.4390325414837e-07
1950 1.43935902872983e-07
1951 1.43592415341232e-07
1952 1.43518938063636e-07
1953 1.4366890491857e-07
1954 1.43299677990427e-07
1955 1.43430625684005e-07
1956 1.43133909411119e-07
1957 1.43261215935553e-07
1958 1.42950438593914e-07
1959 1.43087200562775e-07
1960 1.42798902924568e-07
1961 1.42944530207245e-07
1962 1.42610615959882e-07
1963 1.42768609606492e-07
1964 1.42471424741331e-07
1965 1.42618672796857e-07
1966 1.42286813670012e-07
1967 1.42471753012074e-07
1968 1.42162030321913e-07
1969 1.42281305176795e-07
1970 1.42015292933451e-07
1971 1.42151439103344e-07
1972 1.41847793553751e-07
1973 1.42065655687418e-07
1974 1.4180548988918e-07
1975 1.41894381300744e-07
1976 1.41615082803526e-07
1977 1.41687014341585e-07
1978 1.41543121749521e-07
1979 1.41559467877528e-07
1980 1.41319676998819e-07
1981 1.41492825655121e-07
1982 1.41189431868582e-07
1983 1.41321511108572e-07
1984 1.41052439680323e-07
1985 1.41198892592342e-07
1986 1.4091604703026e-07
1987 1.4101621390239e-07
1988 1.40761741768358e-07
1989 1.40903137229031e-07
1990 1.40619113590645e-07
1991 1.40755386958347e-07
1992 1.40480139776855e-07
1993 1.4058122692262e-07
1994 1.40363122071108e-07
1995 1.40449701838463e-07
1996 1.40166930503227e-07
1997 1.40300006094662e-07
1998 1.40021020158088e-07
1999 1.40150075132794e-07
};
\addlegendentry{Train}
\addplot [semithick, black]
table {%
0 0.00612067896872759
1 0.00241778395138681
2 0.00178416504058987
3 0.000832349178381264
4 0.000228335542487912
5 0.00014477816876024
6 0.000121164084703196
7 0.000103494021459483
8 8.57966078910977e-05
9 7.04385383869521e-05
10 5.69721923966426e-05
11 4.71354069304653e-05
12 4.00381104554981e-05
13 3.50985828845296e-05
14 3.13685777655337e-05
15 2.84520083368989e-05
16 2.60826291196281e-05
17 2.39680339291226e-05
18 2.18886834773002e-05
19 1.99229871213902e-05
20 1.80032729986124e-05
21 1.58681177708786e-05
22 1.3261211279314e-05
23 1.07612322608475e-05
24 8.79775870998856e-06
25 7.35119419914554e-06
26 6.31456668997998e-06
27 5.52303117729025e-06
28 4.91325772600248e-06
29 4.42502278019674e-06
30 4.00938188249711e-06
31 3.65722075912345e-06
32 3.35317690769443e-06
33 3.09605366055621e-06
34 2.87716125058068e-06
35 2.69408224085055e-06
36 2.53938173955248e-06
37 2.4065523120953e-06
38 2.29340457735816e-06
39 2.19512571675295e-06
40 2.10877760764561e-06
41 2.03533863896155e-06
42 1.96494829651783e-06
43 1.9018519878955e-06
44 1.83995848601626e-06
45 1.78743152901006e-06
46 1.73937939962343e-06
47 1.69456484400143e-06
48 1.65588107847725e-06
49 1.62164621997363e-06
50 1.58715079123795e-06
51 1.55689838265971e-06
52 1.52585676005401e-06
53 1.50114351527009e-06
54 1.47890284551977e-06
55 1.45495789638517e-06
56 1.43491479320801e-06
57 1.4158513295115e-06
58 1.3964937579658e-06
59 1.38107691327605e-06
60 1.35885613872233e-06
61 1.34527294903819e-06
62 1.331921680503e-06
63 1.32234060856717e-06
64 1.30978708057228e-06
65 1.29658837977331e-06
66 1.28321948977828e-06
67 1.27069154132187e-06
68 1.25950646179263e-06
69 1.24964321912557e-06
70 1.23907295801473e-06
71 1.23627501125156e-06
72 1.22660162560351e-06
73 1.21684240639297e-06
74 1.20891388633027e-06
75 1.19977960366668e-06
76 1.19025094136305e-06
77 1.18113155167521e-06
78 1.17200113436411e-06
79 1.16267381145008e-06
80 1.15558873403643e-06
81 1.14724366540031e-06
82 1.13976352622558e-06
83 1.13222006348224e-06
84 1.1249246654188e-06
85 1.11671636204846e-06
86 1.11043038941716e-06
87 1.10353198579105e-06
88 1.09810355297668e-06
89 1.09201050690899e-06
90 1.09517111468449e-06
91 1.08929032194283e-06
92 1.08361359707487e-06
93 1.0777050647448e-06
94 1.0721158787419e-06
95 1.06678703559737e-06
96 1.0613766789902e-06
97 1.05652713955351e-06
98 1.05124036053894e-06
99 1.04532693967485e-06
100 1.04044522686308e-06
101 1.03632976333756e-06
102 1.03081254110293e-06
103 1.0251795856675e-06
104 1.02002707080828e-06
105 1.01945295227779e-06
106 1.01436341992667e-06
107 1.00864065188944e-06
108 1.00467559605022e-06
109 9.99179633254244e-07
110 9.93750518318848e-07
111 9.88652686828573e-07
112 9.83828499556694e-07
113 9.78484422375914e-07
114 9.73923192759685e-07
115 9.68674953583104e-07
116 9.64083369581203e-07
117 9.59073418016487e-07
118 9.54890197135683e-07
119 9.50302023738914e-07
120 9.45383703765401e-07
121 9.40708730468032e-07
122 9.36518517846707e-07
123 9.32268790165836e-07
124 9.27552946450305e-07
125 9.2399250206654e-07
126 9.20396757919661e-07
127 9.16979843168519e-07
128 9.12954931209242e-07
129 9.09103448520909e-07
130 9.06748311990668e-07
131 9.0292206778031e-07
132 9.00378836377058e-07
133 8.97348002126819e-07
134 8.9400924707661e-07
135 8.91599597707682e-07
136 8.88320812464372e-07
137 8.84986377513997e-07
138 8.82301833371457e-07
139 8.79404069564771e-07
140 8.76571448316099e-07
141 8.74527188443608e-07
142 8.73877752383123e-07
143 8.72545854235796e-07
144 8.71928762080643e-07
145 8.69197208430705e-07
146 8.66671541643882e-07
147 8.64532296418474e-07
148 8.62905665144353e-07
149 8.597427836321e-07
150 8.5738861343998e-07
151 8.5496407109531e-07
152 8.5280248640629e-07
153 8.51164600135235e-07
154 8.48682248033583e-07
155 8.47462217734574e-07
156 8.45505041979777e-07
157 8.43092038849136e-07
158 8.41253040562151e-07
159 8.3970724062965e-07
160 8.37761092498113e-07
161 8.35601611015591e-07
162 8.32110629289673e-07
163 8.30112185212784e-07
164 8.28036831990175e-07
165 8.26208122362004e-07
166 8.24792323328438e-07
167 8.22960203095136e-07
168 8.20809020751767e-07
169 8.19007425434393e-07
170 8.17531713437347e-07
171 8.15463636172353e-07
172 8.14210409316729e-07
173 8.1287060993418e-07
174 8.11063898709108e-07
175 8.09241896604362e-07
176 8.0800157320482e-07
177 8.06865443792049e-07
178 8.0484699083172e-07
179 8.03186765097053e-07
180 8.02010561073985e-07
181 8.00177076598629e-07
182 7.98577275418211e-07
183 7.97275106378947e-07
184 7.95814514731319e-07
185 7.94058109931939e-07
186 7.92464220467082e-07
187 7.91119248333416e-07
188 7.89581406479556e-07
189 7.88300667409203e-07
190 7.8616568544021e-07
191 7.8476153930751e-07
192 7.84151097832364e-07
193 7.82863878612261e-07
194 7.81554547302221e-07
195 7.80621178364527e-07
196 7.79099082137691e-07
197 7.77554589603824e-07
198 7.82685560807295e-07
199 7.80675009082188e-07
200 7.78894843733724e-07
201 7.84375231432932e-07
202 7.82237918883766e-07
203 7.80753396156797e-07
204 7.7965700029381e-07
205 7.78470564455347e-07
206 7.77032596488425e-07
207 7.75677449382783e-07
208 7.74372551859415e-07
209 7.73012800436845e-07
210 7.71613713368424e-07
211 7.70444842146389e-07
212 7.69236976339016e-07
213 7.68006714224612e-07
214 7.6661314096782e-07
215 7.65343941111496e-07
216 7.64044841616851e-07
217 7.63054117669526e-07
218 7.62001661769318e-07
219 7.60788168463478e-07
220 7.59792897042644e-07
221 7.58698035951966e-07
222 7.57437192078214e-07
223 7.5628673812389e-07
224 7.55063979340775e-07
225 7.5415260880618e-07
226 7.52951109461719e-07
227 7.51998470605031e-07
228 7.51103584661905e-07
229 7.49983541936672e-07
230 7.48921138438163e-07
231 7.47821786717395e-07
232 7.46750572488963e-07
233 7.45710906357999e-07
234 7.44748660963523e-07
235 7.43700638849987e-07
236 7.42794270536251e-07
237 7.41903306789027e-07
238 7.40915822916577e-07
239 7.40154973755125e-07
240 7.39319773401803e-07
241 7.37956895591196e-07
242 7.3696372737686e-07
243 7.34946297598071e-07
244 7.33948468223389e-07
245 7.33070237401989e-07
246 7.32308137685322e-07
247 7.31456850644463e-07
248 7.29256214526686e-07
249 7.2833739750422e-07
250 7.27262886357494e-07
251 7.26602252143493e-07
252 7.25592428807431e-07
253 7.2443037879566e-07
254 7.25299230452947e-07
255 7.24491656001192e-07
256 7.23441587524576e-07
257 7.22375034456491e-07
258 7.21421201888006e-07
259 7.20814966825856e-07
260 7.19227102763398e-07
261 7.17625198376481e-07
262 7.16907550213364e-07
263 7.15876069534715e-07
264 7.15014266461367e-07
265 7.15687122010422e-07
266 7.14868519935408e-07
267 7.13870690560725e-07
268 7.12985524842225e-07
269 7.12199323515961e-07
270 7.11371910711023e-07
271 7.10328038167063e-07
272 7.09600215031969e-07
273 7.08977893282281e-07
274 7.07783669895434e-07
275 7.06517766957404e-07
276 7.06880086909223e-07
277 7.06146124684892e-07
278 7.05109698628803e-07
279 7.04045760357985e-07
280 7.0385391381933e-07
281 7.01727628893423e-07
282 7.01247131473792e-07
283 7.00408747889014e-07
284 6.99655970493041e-07
285 6.98917858699133e-07
286 6.98321514391864e-07
287 6.97350003520114e-07
288 6.96610356953897e-07
289 6.9608228159268e-07
290 6.95221956448222e-07
291 6.94521816058113e-07
292 6.93731351475435e-07
293 6.93315655553306e-07
294 6.92313165018277e-07
295 6.91913953687617e-07
296 6.91290836130065e-07
297 6.90296417360514e-07
298 6.89525109009992e-07
299 6.87396436660492e-07
300 6.86550606587844e-07
301 6.85801012423326e-07
302 6.85096836150478e-07
303 6.8446053091975e-07
304 6.83810753798753e-07
305 6.8306241018945e-07
306 6.82722031797312e-07
307 6.82614938796178e-07
308 6.8223260996092e-07
309 6.81499045640521e-07
310 6.80645086958975e-07
311 6.7981954998686e-07
312 6.79356958244171e-07
313 6.78696494560427e-07
314 6.77967648243794e-07
315 6.77285413530626e-07
316 6.76728802773141e-07
317 6.75963065077667e-07
318 6.75091541779693e-07
319 6.74505088227306e-07
320 6.73943873152894e-07
321 6.7339914266995e-07
322 6.72681721880508e-07
323 6.72151145408861e-07
324 6.68341272103135e-07
325 6.67746007820824e-07
326 6.67249651087332e-07
327 6.66409619043407e-07
328 6.65725394810579e-07
329 6.64600293021067e-07
330 6.63890716623428e-07
331 6.63274249745882e-07
332 6.62432682929648e-07
333 6.61594413031708e-07
334 6.61067133478355e-07
335 6.60314242395543e-07
336 6.59541285585874e-07
337 6.58913222650881e-07
338 6.58446538182034e-07
339 6.56359816275653e-07
340 6.55663484394609e-07
341 6.55010865102668e-07
342 6.56161830647761e-07
343 6.55365909096872e-07
344 6.54590678550449e-07
345 6.5384961089876e-07
346 6.52879407425644e-07
347 6.5229528445343e-07
348 6.51441496302141e-07
349 6.50890797260217e-07
350 6.50128697543551e-07
351 6.49347725811822e-07
352 6.48637580979994e-07
353 6.48106606604415e-07
354 6.47457000013674e-07
355 6.46727926323365e-07
356 6.46014314042986e-07
357 6.45379088837217e-07
358 6.44616079625848e-07
359 6.43985742954101e-07
360 6.43224154828204e-07
361 6.42625138880248e-07
362 6.41554208868911e-07
363 6.414300059987e-07
364 6.39899553789292e-07
365 6.39418317405216e-07
366 6.38502342553693e-07
367 6.37803964309569e-07
368 6.37074833775841e-07
369 6.36381741969672e-07
370 6.36101674444944e-07
371 6.35526532732911e-07
372 6.35112144209415e-07
373 6.34386935871589e-07
374 6.33674972050358e-07
375 6.33032414043555e-07
376 6.32238879916258e-07
377 6.31626221547776e-07
378 6.30848603577761e-07
379 6.30083263786219e-07
380 6.29322016720835e-07
381 6.28665361546155e-07
382 6.2782180521026e-07
383 6.27275255737914e-07
384 6.26625933364267e-07
385 6.25880545612745e-07
386 6.25268683052127e-07
387 6.24555184458586e-07
388 6.23833727786405e-07
389 6.23207824901328e-07
390 6.22476477474265e-07
391 6.21807259904017e-07
392 6.21131221123505e-07
393 6.20355478986312e-07
394 6.19517663835722e-07
395 6.1890801816844e-07
396 6.1819946495234e-07
397 6.17510977463098e-07
398 6.16775309936202e-07
399 6.15964722783247e-07
400 6.15297096828726e-07
401 6.14507371210493e-07
402 6.13792792591994e-07
403 6.13140969107917e-07
404 6.12500002716843e-07
405 6.11954362739198e-07
406 6.11131724781444e-07
407 6.10341714946117e-07
408 6.09706034992996e-07
409 6.09093149250839e-07
410 6.08611344432575e-07
411 6.0767450804633e-07
412 6.07193328505673e-07
413 6.06207720466045e-07
414 6.05498769346013e-07
415 6.04851095431513e-07
416 6.03992589276459e-07
417 6.03166995460924e-07
418 6.02114994308067e-07
419 6.01494946295134e-07
420 5.99974384840607e-07
421 5.99347629304248e-07
422 5.98412498220569e-07
423 5.97589689732558e-07
424 5.96961513110728e-07
425 5.96095333094127e-07
426 5.95412586790189e-07
427 5.94638265738467e-07
428 5.93820232097642e-07
429 5.93038294027792e-07
430 5.91803257066204e-07
431 5.90888362239639e-07
432 5.90297872804513e-07
433 5.89409864915069e-07
434 5.88670957313298e-07
435 5.88027376124955e-07
436 5.87255499340245e-07
437 5.87066381285695e-07
438 5.86064061280922e-07
439 5.85094085181481e-07
440 5.8452280882193e-07
441 5.83635596740351e-07
442 5.82819950523117e-07
443 5.81964968660031e-07
444 5.81012272959924e-07
445 5.83866778924857e-07
446 5.83053576974635e-07
447 5.82231109547138e-07
448 5.81640108521242e-07
449 5.81029496515839e-07
450 5.80856749365921e-07
451 5.80034566155518e-07
452 5.79151844704029e-07
453 5.78199774281529e-07
454 5.76805120999779e-07
455 5.75781996303704e-07
456 5.74860791857645e-07
457 5.74016610244144e-07
458 5.72983708480024e-07
459 5.72074782212439e-07
460 5.71150224004668e-07
461 5.70006250200095e-07
462 5.68985853988124e-07
463 5.68066354844632e-07
464 5.67161123399273e-07
465 5.6614089771756e-07
466 5.65128686957905e-07
467 5.64103402211913e-07
468 5.63198057079717e-07
469 5.62183913643821e-07
470 5.61195406589832e-07
471 5.60195360321813e-07
472 5.59209013317741e-07
473 5.58142005502305e-07
474 5.57008831947314e-07
475 5.55419433112547e-07
476 5.53826680516067e-07
477 5.52969254385971e-07
478 5.51769630874333e-07
479 5.50958588974026e-07
480 5.49931826299144e-07
481 5.48835203062481e-07
482 5.47793035821087e-07
483 5.46432602277491e-07
484 5.453054541249e-07
485 5.44289207482507e-07
486 5.43097712579765e-07
487 5.42020586635772e-07
488 5.41018039257324e-07
489 5.39950008260348e-07
490 5.38885046807991e-07
491 5.37689004431741e-07
492 5.3356245643954e-07
493 5.32714068413043e-07
494 5.31532407421764e-07
495 5.3044584547024e-07
496 5.29346323219215e-07
497 5.28432735791284e-07
498 5.26813607848453e-07
499 5.25641439708124e-07
500 5.24614677033242e-07
501 5.23584162692714e-07
502 5.22469747465948e-07
503 5.2144025630696e-07
504 5.20285311722546e-07
505 5.19329773851496e-07
506 5.18255319548189e-07
507 5.17267551458644e-07
508 5.16245222570433e-07
509 5.1535670309022e-07
510 5.14465682499576e-07
511 5.1350446028664e-07
512 5.12559154230985e-07
513 5.11686153004121e-07
514 5.1058344752164e-07
515 5.09652295477281e-07
516 5.08794244069577e-07
517 5.07823472162272e-07
518 5.06794265220378e-07
519 5.05869934386283e-07
520 5.04981358062651e-07
521 5.04213005569909e-07
522 5.03221599501558e-07
523 5.02111788591719e-07
524 5.01275110309507e-07
525 5.00302633099636e-07
526 4.99211068927252e-07
527 4.98467045417783e-07
528 4.97581595482188e-07
529 4.96725647280982e-07
530 4.95988899729127e-07
531 4.95075596518291e-07
532 4.94234143388894e-07
533 4.93552136049402e-07
534 4.92847050281853e-07
535 4.92140543428832e-07
536 4.91508103550586e-07
537 4.90529657781735e-07
538 4.89974468109722e-07
539 4.89233514144871e-07
540 4.88368698370323e-07
541 4.87732961573784e-07
542 4.86725355131057e-07
543 4.85362761537544e-07
544 4.84586337279325e-07
545 4.8380564976469e-07
546 4.83563383113506e-07
547 4.83164114939427e-07
548 4.82637346976844e-07
549 4.81934080198698e-07
550 4.81186873457773e-07
551 4.80388848700386e-07
552 4.79829736832471e-07
553 4.78453443975013e-07
554 4.7776450173842e-07
555 4.76837726637314e-07
556 4.76004515803652e-07
557 4.74860058830018e-07
558 4.74078404977263e-07
559 4.7361248789457e-07
560 4.72580467203443e-07
561 4.71646984578911e-07
562 4.71612054298021e-07
563 4.70345554504092e-07
564 4.69253762958033e-07
565 4.68242205897695e-07
566 4.67236617396338e-07
567 4.66334086013376e-07
568 4.65339695665534e-07
569 4.64660161014763e-07
570 4.63667788608291e-07
571 4.62888550600837e-07
572 4.62135773204864e-07
573 4.61426651554575e-07
574 4.6065696324149e-07
575 4.59907852246033e-07
576 4.59254664519904e-07
577 4.58386352875095e-07
578 4.57721768043484e-07
579 4.56982860441713e-07
580 4.56015499139539e-07
581 4.55630953410946e-07
582 4.54974355079685e-07
583 4.54291040341559e-07
584 4.5352456368164e-07
585 4.53163494285036e-07
586 4.52559447694512e-07
587 4.51896795539142e-07
588 4.51112157406897e-07
589 4.4950874666938e-07
590 4.48633443284052e-07
591 4.47909911827082e-07
592 4.472888122109e-07
593 4.46739051085387e-07
594 4.46114313490398e-07
595 4.45454190867167e-07
596 4.44780567931957e-07
597 4.44234075303029e-07
598 4.43677521388963e-07
599 4.43126253912851e-07
600 4.42657665189472e-07
601 4.42001493183852e-07
602 4.41742230350428e-07
603 4.39728154333352e-07
604 4.389064258703e-07
605 4.38276032355134e-07
606 4.37718398416109e-07
607 4.37213486748078e-07
608 4.36809926895876e-07
609 4.36498936551288e-07
610 4.36335511722064e-07
611 4.2576780856507e-07
612 4.25780939394826e-07
613 4.25666854653173e-07
614 4.25009432092338e-07
615 4.24948581212448e-07
616 4.2516802523096e-07
617 4.2487411633374e-07
618 4.24610732352448e-07
619 4.24380374397515e-07
620 4.23824360495928e-07
621 4.23647009029082e-07
622 4.23212242139925e-07
623 4.22618683160181e-07
624 4.21872357492248e-07
625 4.21264900296592e-07
626 4.20588264660182e-07
627 4.20041857296383e-07
628 4.19452447886215e-07
629 4.189367359686e-07
630 4.18255467593553e-07
631 4.17520396922555e-07
632 4.16709610817634e-07
633 4.16106360034973e-07
634 4.15469003200997e-07
635 4.14830822137446e-07
636 4.14135911341873e-07
637 4.1356869928677e-07
638 4.13019080269805e-07
639 4.12639934666004e-07
640 4.12191070608969e-07
641 4.11505396868961e-07
642 4.10690716989848e-07
643 4.10037671372265e-07
644 4.09676232493439e-07
645 4.08803401796831e-07
646 4.08269926310822e-07
647 4.07850080819117e-07
648 4.07442996674945e-07
649 4.06568119615258e-07
650 4.05861356966852e-07
651 4.04611796511745e-07
652 4.03487860012319e-07
653 4.02562960744035e-07
654 4.01659491444661e-07
655 4.00977995695939e-07
656 4.00038146608495e-07
657 3.99256066430098e-07
658 3.98581391891639e-07
659 3.97931557927222e-07
660 3.97386827444279e-07
661 3.968299324697e-07
662 3.96253454937323e-07
663 3.9558449316246e-07
664 3.94978656004241e-07
665 3.94573248740926e-07
666 3.94045883922445e-07
667 3.93586503832921e-07
668 3.92960629369554e-07
669 3.92554596828631e-07
670 3.92094960943723e-07
671 3.91499895613379e-07
672 3.91047365155828e-07
673 3.90536513350526e-07
674 3.90065196143041e-07
675 3.89521602528475e-07
676 3.8902535948182e-07
677 3.88569930009908e-07
678 3.88015479302339e-07
679 3.87552006486658e-07
680 3.87025636428007e-07
681 3.86569126931136e-07
682 3.86327229762173e-07
683 3.85670972491425e-07
684 3.85205936481725e-07
685 3.8490878750963e-07
686 3.84393075592016e-07
687 3.83825039307339e-07
688 3.83930938596677e-07
689 3.82921598429675e-07
690 3.82584516955831e-07
691 3.8200323615456e-07
692 3.81687272010822e-07
693 3.81060516474463e-07
694 3.80801338906167e-07
695 3.80275537281705e-07
696 3.79991178078853e-07
697 3.79689737428635e-07
698 3.7928032270429e-07
699 3.78701287218064e-07
700 3.78274819468061e-07
701 3.76717480321531e-07
702 3.76198386220494e-07
703 3.755938280392e-07
704 3.75057766177633e-07
705 3.74590968021948e-07
706 3.72831067352308e-07
707 3.72428900163868e-07
708 3.71900995332908e-07
709 3.71428171774824e-07
710 3.7093036553415e-07
711 3.70467176935563e-07
712 3.70151269635244e-07
713 3.69680947187589e-07
714 3.69314932413545e-07
715 3.68888578350379e-07
716 3.68491043900576e-07
717 3.68234850611771e-07
718 3.67962826430812e-07
719 3.67605395013015e-07
720 3.6730725128109e-07
721 3.66937570106529e-07
722 3.66585709343781e-07
723 3.66201192036897e-07
724 3.65822387493608e-07
725 3.65414194902769e-07
726 3.65021207926475e-07
727 3.64614919590167e-07
728 3.6424751215236e-07
729 3.63836619499125e-07
730 3.63478903864234e-07
731 3.63065851161082e-07
732 3.62964414080125e-07
733 3.62523820740535e-07
734 3.62173096846163e-07
735 3.61761834710705e-07
736 3.61412190841293e-07
737 3.60982852498637e-07
738 3.6060447428099e-07
739 3.60194491122456e-07
740 3.59240061698074e-07
741 3.58721621296354e-07
742 3.58271563527524e-07
743 3.57826394292715e-07
744 3.57444037035748e-07
745 3.57080637058971e-07
746 3.56835073489492e-07
747 3.56448850880042e-07
748 3.56118448507914e-07
749 3.5566756650951e-07
750 3.55280576513906e-07
751 3.5840747614202e-07
752 3.57898244374155e-07
753 3.57299484221585e-07
754 3.56844765292408e-07
755 3.5641252793539e-07
756 3.56070330553848e-07
757 3.55643464899913e-07
758 3.55287454567588e-07
759 3.54977032657189e-07
760 3.5462468872538e-07
761 3.54270298430492e-07
762 3.53912298578507e-07
763 3.53537842556761e-07
764 3.53195019897612e-07
765 3.52797201230715e-07
766 3.52420016724864e-07
767 3.52084725818713e-07
768 3.51753129734789e-07
769 3.51397460462977e-07
770 3.51028717204827e-07
771 3.50628113210405e-07
772 3.50320163988727e-07
773 3.49882697037174e-07
774 3.4941484727824e-07
775 3.4908401858047e-07
776 3.48606135958107e-07
777 3.4824572026082e-07
778 3.47998792449289e-07
779 3.47477907780558e-07
780 3.47165865832721e-07
781 3.46951338769941e-07
782 3.46475530932366e-07
783 3.46275015772335e-07
784 3.45923893974032e-07
785 3.45359836728676e-07
786 3.45009283364561e-07
787 3.44935671137137e-07
788 3.44397392382234e-07
789 3.44078188163621e-07
790 3.43559122484294e-07
791 3.43044973760698e-07
792 3.42838575306814e-07
793 3.42370356065658e-07
794 3.42178026357942e-07
795 3.41853535701375e-07
796 3.41369258194391e-07
797 3.40777063456699e-07
798 3.4031643281196e-07
799 3.399057675324e-07
800 3.39576217811555e-07
801 3.3883046057781e-07
802 3.38627444307349e-07
803 3.38139642508395e-07
804 3.37644252113023e-07
805 3.36940473744107e-07
806 3.36203896722509e-07
807 3.34536508717065e-07
808 3.32764955146558e-07
809 3.31990719359965e-07
810 3.31445164647448e-07
811 3.30965576722519e-07
812 3.30522794911303e-07
813 3.30044230167914e-07
814 3.2956938866846e-07
815 3.29212696215109e-07
816 3.28921800019089e-07
817 3.28637042912305e-07
818 3.28328411569601e-07
819 3.28135399740859e-07
820 3.27852092141256e-07
821 3.27466324279158e-07
822 3.27044375580954e-07
823 3.26585819721004e-07
824 3.26146277984662e-07
825 3.25824828451005e-07
826 3.25355927088822e-07
827 3.25217371255349e-07
828 3.24902345028022e-07
829 3.2459308840771e-07
830 3.21346988130244e-07
831 3.20718186230806e-07
832 3.20259772479403e-07
833 3.19463282494326e-07
834 3.19045710739374e-07
835 3.18272327604063e-07
836 3.17515770120735e-07
837 3.16860706561783e-07
838 3.16280676315728e-07
839 3.15762918035034e-07
840 3.15168534825716e-07
841 3.14844584181628e-07
842 3.14400210754684e-07
843 3.14024873659946e-07
844 3.13667243290183e-07
845 3.13287728204159e-07
846 3.12895394927182e-07
847 3.125422267658e-07
848 3.12294417881276e-07
849 3.11602775582287e-07
850 3.11144106035499e-07
851 3.10699419969751e-07
852 3.10261270897172e-07
853 3.09783217744553e-07
854 3.09341317006329e-07
855 3.08899529954942e-07
856 3.08488949940511e-07
857 3.08103949464567e-07
858 3.0773202297496e-07
859 3.07377064245884e-07
860 3.07039925928621e-07
861 3.06708955122303e-07
862 3.06369116742644e-07
863 3.0604749667873e-07
864 3.05726928218064e-07
865 3.06360476542977e-07
866 3.05952141843591e-07
867 3.05582574355867e-07
868 3.05229065133972e-07
869 3.04888715163543e-07
870 3.04562973951761e-07
871 3.04190905353607e-07
872 3.03839584603338e-07
873 3.03520948818914e-07
874 3.03207428942187e-07
875 3.0291835173557e-07
876 3.0260702033047e-07
877 3.02309388189315e-07
878 3.01965457083497e-07
879 3.0165696784934e-07
880 3.01295983717864e-07
881 3.00962284427442e-07
882 3.00613777426406e-07
883 3.00289173083002e-07
884 2.99962181316005e-07
885 2.99558422511836e-07
886 2.99251695423663e-07
887 2.98937351317363e-07
888 2.98634205364579e-07
889 2.98328643566492e-07
890 2.98034734669272e-07
891 2.9773249821119e-07
892 2.97462094067669e-07
893 2.97208345045874e-07
894 2.96945785294156e-07
895 2.96646902597786e-07
896 2.96332189009263e-07
897 2.96055702619924e-07
898 2.9577049076579e-07
899 2.95476439760023e-07
900 2.96181298153897e-07
901 2.95920870030386e-07
902 2.95619713597262e-07
903 2.95327282628932e-07
904 2.95060260668834e-07
905 2.94766152819648e-07
906 2.94441434789405e-07
907 2.94181376148117e-07
908 2.93907476134336e-07
909 2.93626811753711e-07
910 2.93342736767954e-07
911 2.93071707346826e-07
912 2.92646632260585e-07
913 2.92351131747637e-07
914 2.91929893592169e-07
915 2.91488191805911e-07
916 2.91120130668787e-07
917 2.90796975832563e-07
918 2.9046083227513e-07
919 2.90109966272212e-07
920 2.89811396214645e-07
921 2.89544942688735e-07
922 2.89211527615407e-07
923 2.88894341338164e-07
924 2.8862058343293e-07
925 2.88315533225614e-07
926 2.88037171003452e-07
927 2.87778703977892e-07
928 2.87508299834371e-07
929 2.87106928453795e-07
930 2.86787184222703e-07
931 2.86493275325483e-07
932 2.8621300884879e-07
933 2.85642272501718e-07
934 2.86186605080729e-07
935 2.84621393120688e-07
936 2.85267873323392e-07
937 2.83873760054121e-07
938 2.84623808965989e-07
939 2.83249363519644e-07
940 2.82941073237453e-07
941 2.82707020460293e-07
942 2.82422689679152e-07
943 2.82283281194395e-07
944 2.83517778143505e-07
945 2.81550143199638e-07
946 2.8134496687926e-07
947 2.81075642760698e-07
948 2.82406574569904e-07
949 2.80476626812742e-07
950 2.80265425089965e-07
951 2.80014660347661e-07
952 2.81038268212797e-07
953 2.79377587730778e-07
954 2.79131313618564e-07
955 2.80049761158807e-07
956 2.78551397059346e-07
957 2.78416621313227e-07
958 2.78075702908609e-07
959 2.77842246987348e-07
960 2.77726968533898e-07
961 2.77214581956287e-07
962 2.78740458270477e-07
963 2.76709982927059e-07
964 2.78240548823305e-07
965 2.75812652716922e-07
966 2.75515247949443e-07
967 2.76739740456833e-07
968 2.74919869980295e-07
969 2.74775544539807e-07
970 2.74519123877326e-07
971 2.75931000714991e-07
972 2.74089927643217e-07
973 2.74008840506212e-07
974 2.73710014653261e-07
975 2.74933654509368e-07
976 2.73304550546527e-07
977 2.73068280876032e-07
978 2.72866856221299e-07
979 2.7258190016255e-07
980 2.722710235048e-07
981 2.72165351589138e-07
982 2.73251544058439e-07
983 2.71474931423654e-07
984 2.71301388465872e-07
985 2.71066085133498e-07
986 2.7076458763986e-07
987 2.71778930027722e-07
988 2.70364807875012e-07
989 2.70122569645537e-07
990 2.69943427610997e-07
991 2.69151030352077e-07
992 2.68965436589497e-07
993 2.68613831622133e-07
994 2.70138201585723e-07
995 2.68117673840607e-07
996 2.67824390220994e-07
997 2.6759209958982e-07
998 2.68899754019003e-07
999 2.6674732112042e-07
1000 2.66534271986529e-07
1001 2.66373177737478e-07
1002 2.6802916863744e-07
1003 2.65900581553069e-07
1004 2.65773422825077e-07
1005 2.65574954028125e-07
1006 2.67021988520355e-07
1007 2.65098492491234e-07
1008 2.64954422846131e-07
1009 2.64698371665872e-07
1010 2.66416890326582e-07
1011 2.64163247720717e-07
1012 2.64049276665901e-07
1013 2.63811358536259e-07
1014 2.65489745743253e-07
1015 2.63344986706215e-07
1016 2.63123126842402e-07
1017 2.62893451008495e-07
1018 2.64476312850093e-07
1019 2.62342751966571e-07
1020 2.62170971154774e-07
1021 2.61942716406338e-07
1022 2.61649347521598e-07
1023 2.61395882716897e-07
1024 2.61096886333689e-07
1025 2.62607017020855e-07
1026 2.60574012145298e-07
1027 2.61000678847267e-07
1028 2.60707736288168e-07
1029 2.60701881416026e-07
1030 2.60423945519506e-07
1031 2.60150613939913e-07
1032 2.61637069343124e-07
1033 2.59539859825964e-07
1034 2.59348070130727e-07
1035 2.5900763489517e-07
1036 2.58709945910596e-07
1037 2.56947060961465e-07
1038 2.56720568359015e-07
1039 2.5750270538083e-07
1040 2.56187320246681e-07
1041 2.55952443239948e-07
1042 2.5687597826618e-07
1043 2.55386794378865e-07
1044 2.56192748793183e-07
1045 2.54910162311717e-07
1046 2.54659056508899e-07
1047 2.54441147262696e-07
1048 2.54092867635336e-07
1049 2.54943557820297e-07
1050 2.53581646347811e-07
1051 2.53356773782798e-07
1052 2.53103735303739e-07
1053 2.52944147405287e-07
1054 2.52656462862433e-07
1055 2.52341891382457e-07
1056 2.52095162522892e-07
1057 2.51856789645899e-07
1058 2.52443612680509e-07
1059 2.51315952937148e-07
1060 2.51050011002008e-07
1061 2.50908470889044e-07
1062 2.50803424250989e-07
1063 2.50479729402286e-07
1064 2.51212469493112e-07
1065 2.50162145221111e-07
1066 2.49901830784438e-07
1067 2.49762877047033e-07
1068 2.49498441462492e-07
1069 2.50077476948718e-07
1070 2.49029938004242e-07
1071 2.49004926899943e-07
1072 2.48735744889927e-07
1073 2.4854449520717e-07
1074 2.48377546085976e-07
1075 2.48779997491511e-07
1076 2.48036172934007e-07
1077 2.47890483251467e-07
1078 2.47748374704315e-07
1079 2.47600780767243e-07
1080 2.47531204422557e-07
1081 2.47336259917574e-07
1082 2.47178832069039e-07
1083 2.46968454575835e-07
1084 2.47348111770407e-07
1085 2.46622306576683e-07
1086 2.46427219963152e-07
1087 2.46240631440742e-07
1088 2.46151074634327e-07
1089 2.45858927883091e-07
1090 2.45756098138372e-07
1091 2.45531737164129e-07
1092 2.45463581904914e-07
1093 2.46136039550038e-07
1094 2.45455083813795e-07
1095 2.45094213369157e-07
1096 2.44813747940498e-07
1097 2.44650721015205e-07
1098 2.44457169173984e-07
1099 2.44243125280263e-07
1100 2.43947710032444e-07
1101 2.43785194697921e-07
1102 2.4425176547993e-07
1103 2.44263446802506e-07
1104 2.43158638113528e-07
1105 2.43051573534103e-07
1106 2.42948914319641e-07
1107 2.42764514268856e-07
1108 2.42598559907492e-07
1109 2.42439824660323e-07
1110 2.42302320430099e-07
1111 2.42113998183413e-07
1112 2.41966631620016e-07
1113 2.42297431896077e-07
1114 2.41591891381177e-07
1115 2.41537293277361e-07
1116 2.41363096620262e-07
1117 2.41218231167295e-07
1118 2.41117135146851e-07
1119 2.40934696194017e-07
1120 2.40553134744914e-07
1121 2.40422963315723e-07
1122 2.4022651246014e-07
1123 2.40121863726017e-07
1124 2.4043805524343e-07
1125 2.39843359395309e-07
1126 2.39702956150722e-07
1127 2.3957585426615e-07
1128 2.3941765903146e-07
1129 2.39304171145704e-07
1130 2.3915518454487e-07
1131 2.39008244307115e-07
1132 2.38870939028857e-07
1133 2.38759639614727e-07
1134 2.38638563132554e-07
1135 2.38798605778356e-07
1136 2.38218746062557e-07
1137 2.38360442494923e-07
1138 2.38206567360066e-07
1139 2.38266238739016e-07
1140 2.38002598962339e-07
1141 2.37753127407814e-07
1142 2.37648933421042e-07
1143 2.37725160445734e-07
1144 2.37390509028046e-07
1145 2.37706956340844e-07
1146 2.37090816312957e-07
1147 2.36965732369754e-07
1148 2.37234445421564e-07
1149 2.37066018371479e-07
1150 2.36839824196977e-07
1151 2.36668810771334e-07
1152 2.3638808954729e-07
1153 2.36180838442124e-07
1154 2.35963838690623e-07
1155 2.35775814871886e-07
1156 2.35407497939377e-07
1157 2.35633535794477e-07
1158 2.35009338211967e-07
1159 2.34926218922737e-07
1160 2.34783001928918e-07
1161 2.34425300504881e-07
1162 2.3413370797698e-07
1163 2.3396430037792e-07
1164 2.3384004066429e-07
1165 2.34073866067774e-07
1166 2.33503897106857e-07
1167 2.33344238154132e-07
1168 2.33130322158104e-07
1169 2.32938020872098e-07
1170 2.32705843927761e-07
1171 2.32886662843157e-07
1172 2.32314619097451e-07
1173 2.32228387631039e-07
1174 2.32016532208945e-07
1175 2.31803866768132e-07
1176 2.31687536711433e-07
1177 2.31973075415226e-07
1178 2.31391538818571e-07
1179 2.31378805892746e-07
1180 2.31231751968153e-07
1181 2.31008471018868e-07
1182 2.30961802571983e-07
1183 2.31265417482973e-07
1184 2.30742244866633e-07
1185 2.30612997142998e-07
1186 2.30469666462341e-07
1187 2.30321106187148e-07
1188 2.30613537155477e-07
1189 2.29956114594643e-07
1190 2.29802125772949e-07
1191 2.29657914019299e-07
1192 2.29566751386301e-07
1193 2.29398168016814e-07
1194 2.29295807230301e-07
1195 2.29132737672444e-07
1196 2.29029083698151e-07
1197 2.28877411245776e-07
1198 2.28682992542417e-07
1199 2.29018269237713e-07
1200 2.28464188012367e-07
1201 2.28430451443273e-07
1202 2.28289096071421e-07
1203 2.28119731104925e-07
1204 2.28395876433751e-07
1205 2.27851003842261e-07
1206 2.27792256168868e-07
1207 2.27586269829771e-07
1208 2.27468277103071e-07
1209 2.27825722731723e-07
1210 2.27280381182027e-07
1211 2.27182567869022e-07
1212 2.27048289502818e-07
1213 2.26801347480432e-07
1214 2.27107406658433e-07
1215 2.26565219918484e-07
1216 2.26479301090876e-07
1217 2.26341555276122e-07
1218 2.26214524445822e-07
1219 2.26850687568003e-07
1220 2.25998462610733e-07
1221 2.25935337994088e-07
1222 2.2580674396977e-07
1223 2.25694776645469e-07
1224 2.25604765091703e-07
1225 2.25746077830991e-07
1226 2.25332655645616e-07
1227 2.25326886038602e-07
1228 2.25184365376663e-07
1229 2.25033801370955e-07
1230 2.24908717427752e-07
1231 2.25148681920473e-07
1232 2.24748021082632e-07
1233 2.24667971338022e-07
1234 2.24514508317952e-07
1235 2.2439159863552e-07
1236 2.24292790562686e-07
1237 2.24492325173742e-07
1238 2.24067875365108e-07
1239 2.23984585545622e-07
1240 2.23917041353161e-07
1241 2.23815575850494e-07
1242 2.23730950210665e-07
1243 2.23949371047638e-07
1244 2.23559283085706e-07
1245 2.23482913952466e-07
1246 2.23357602635588e-07
1247 2.2324272208607e-07
1248 2.23154529521707e-07
1249 2.23441048774475e-07
1250 2.22895266688283e-07
1251 2.22867683419281e-07
1252 2.22683652850719e-07
1253 2.22614787048769e-07
1254 2.22543022232458e-07
1255 2.23047152303479e-07
1256 2.22337874333789e-07
1257 2.2227871454561e-07
1258 2.22194515231422e-07
1259 2.22043738062894e-07
1260 2.22010797301664e-07
1261 2.22175316366702e-07
1262 2.21715637849229e-07
1263 2.21716106807435e-07
1264 2.22224329604614e-07
1265 2.21368651409648e-07
1266 2.21708305048196e-07
1267 2.21248015463971e-07
1268 2.21232284047801e-07
1269 2.21492143737123e-07
1270 2.2097010798916e-07
1271 2.20956323460086e-07
1272 2.20814584395157e-07
1273 2.20667061512358e-07
1274 2.20524867700078e-07
1275 2.20751786628171e-07
1276 2.20211077817112e-07
1277 2.2019268897111e-07
1278 2.20076501022959e-07
1279 2.20027914110688e-07
1280 2.19876966411903e-07
1281 2.20165517816895e-07
1282 2.19508308418881e-07
1283 2.19592820371872e-07
1284 2.1948540052108e-07
1285 2.19369141518655e-07
1286 2.19939835233163e-07
1287 2.19056403238937e-07
1288 2.18943412733097e-07
1289 2.18913896787853e-07
1290 2.18775724647458e-07
1291 2.18672255414276e-07
1292 2.1902835101173e-07
1293 2.18396920104169e-07
1294 2.18415721064957e-07
1295 2.18331862811283e-07
1296 2.18259870621296e-07
1297 2.18122124806541e-07
1298 2.18712159494316e-07
1299 2.17877143882106e-07
1300 2.1787046478039e-07
1301 2.17718678641177e-07
1302 2.17577436956162e-07
1303 2.17433765214992e-07
1304 2.18075697944187e-07
1305 2.1712098430271e-07
1306 2.17107498201585e-07
1307 2.16978236267096e-07
1308 2.16889986859314e-07
1309 2.16753647919177e-07
1310 2.16680348330556e-07
1311 2.16506961692176e-07
1312 2.16473125647099e-07
1313 2.16381067730254e-07
1314 2.16262947105861e-07
1315 2.16199637748105e-07
1316 2.16080948689523e-07
1317 2.16982797951459e-07
1318 2.15796148950176e-07
1319 2.15784595525292e-07
1320 2.15701206229824e-07
1321 2.15606064557505e-07
1322 2.1553847773248e-07
1323 2.15340691056554e-07
1324 2.15415838056288e-07
1325 2.15198312503162e-07
1326 2.1506319569653e-07
1327 2.15034674511116e-07
1328 2.14955633737191e-07
1329 2.14838280498952e-07
1330 2.14974164691739e-07
1331 2.15345650644849e-07
1332 2.14171578249989e-07
1333 2.14092523265208e-07
1334 2.14016978361542e-07
1335 2.13887219047137e-07
1336 2.13971873108676e-07
1337 2.13734438148094e-07
1338 2.14751480598352e-07
1339 2.1393550753146e-07
1340 2.13477903798776e-07
1341 2.13400653592544e-07
1342 2.14644103380124e-07
1343 2.13219252032104e-07
1344 2.13182715924631e-07
1345 2.13083154676497e-07
1346 2.1316481024769e-07
1347 2.12877992566973e-07
1348 2.13019021089167e-07
1349 2.14122678698914e-07
1350 2.13114986991059e-07
1351 2.12642689234599e-07
1352 2.12561388934773e-07
1353 2.1264389715725e-07
1354 2.1242831849122e-07
1355 2.12497269558298e-07
1356 2.12296868085105e-07
1357 2.12304527735796e-07
1358 2.13575788166054e-07
1359 2.12222616369218e-07
1360 2.12400848909056e-07
1361 2.12317246450766e-07
1362 2.12243577379922e-07
1363 2.11976626474097e-07
1364 2.12065316418375e-07
1365 2.11988250953254e-07
1366 2.12929208487367e-07
1367 2.12517335285156e-07
1368 2.11593828680634e-07
1369 2.11665039273612e-07
1370 2.11652064763257e-07
1371 2.11594155530292e-07
1372 2.11350311474234e-07
1373 2.11429181717904e-07
1374 2.11351604662013e-07
1375 2.1227961610748e-07
1376 2.11853375731152e-07
1377 2.11744122680102e-07
1378 2.10783724696739e-07
1379 2.10764113717232e-07
1380 2.10406028600119e-07
1381 2.09140225138071e-07
1382 2.07761516435312e-07
1383 2.07638549909461e-07
1384 2.07514403882669e-07
1385 2.08304911097912e-07
1386 2.07881811320476e-07
1387 2.07791785555855e-07
1388 2.07762582249416e-07
1389 2.07682887776173e-07
1390 2.07598603196857e-07
1391 2.07553938480487e-07
1392 2.06942573299784e-07
1393 2.06809374958539e-07
1394 2.06688142156963e-07
1395 2.07470264967924e-07
1396 2.06489062293258e-07
1397 2.06566667770858e-07
1398 2.06400358138126e-07
1399 2.06301407956744e-07
1400 2.06199700869547e-07
1401 2.06265625024571e-07
1402 2.06089623588923e-07
1403 2.05981223189156e-07
1404 2.05885811510598e-07
1405 2.05951366183399e-07
1406 2.0566189107285e-07
1407 2.05537560304947e-07
1408 2.05584782975166e-07
1409 2.05427312494066e-07
1410 2.05376437634186e-07
1411 2.05404916187035e-07
1412 2.06106108180393e-07
1413 2.05414579568242e-07
1414 2.05508612793892e-07
1415 2.05395096486427e-07
1416 2.0529516575607e-07
1417 2.05183908974504e-07
1418 2.06127808155543e-07
1419 2.05101486017156e-07
1420 2.05012597120913e-07
1421 2.05072581138666e-07
1422 2.04951007276577e-07
1423 2.04847609097669e-07
1424 2.04909284207133e-07
1425 2.04768127787247e-07
1426 2.04636720013696e-07
1427 2.04708911155649e-07
1428 2.04488628696708e-07
1429 2.04339158926814e-07
1430 2.04376064516509e-07
1431 2.0431916425423e-07
1432 2.04184004815033e-07
1433 2.04217613486435e-07
1434 2.04087939437159e-07
1435 2.04173403517416e-07
1436 2.04081530341682e-07
1437 2.04042763130019e-07
1438 2.03975687895763e-07
1439 2.03925097252977e-07
1440 2.03791771014039e-07
1441 2.03834474632458e-07
1442 2.0377926546189e-07
1443 2.03724994207732e-07
1444 2.03692863465221e-07
1445 2.03628772510456e-07
1446 2.035312576254e-07
1447 2.03500263751266e-07
1448 2.04259734459811e-07
1449 2.03312069402273e-07
1450 2.03314243663044e-07
1451 2.03212579208412e-07
1452 2.03230413831079e-07
1453 2.03240034579721e-07
1454 2.03112762164892e-07
1455 2.03075870786051e-07
1456 2.03004901777604e-07
1457 2.02979080654586e-07
1458 2.02927395775987e-07
1459 2.02868037035842e-07
1460 2.02798361215173e-07
1461 2.02801075488424e-07
1462 2.02722475250994e-07
1463 2.02894867129544e-07
1464 2.02600475063264e-07
1465 2.02573133378792e-07
1466 2.02531921900118e-07
1467 2.02473259491853e-07
1468 2.03213602389951e-07
1469 2.0313183313192e-07
1470 2.03011694566158e-07
1471 2.02946637273271e-07
1472 2.02836304197263e-07
1473 2.01990602022306e-07
1474 2.01942285116274e-07
1475 2.01890983930753e-07
1476 2.01850383518831e-07
1477 2.01814970068881e-07
1478 2.01787727860392e-07
1479 2.01757160311899e-07
1480 2.01724844828277e-07
1481 2.01687356593538e-07
1482 2.02554574002534e-07
1483 2.02432687501641e-07
1484 2.01460863991088e-07
1485 2.01477618588797e-07
1486 2.01596179749686e-07
1487 2.01653264753077e-07
1488 2.01639394958875e-07
1489 2.01658792775561e-07
1490 2.01658977516672e-07
1491 2.02386061687321e-07
1492 2.01350246697984e-07
1493 2.01412760247877e-07
1494 2.01310839997859e-07
1495 2.01383954845369e-07
1496 2.01300210278532e-07
1497 2.01421627821219e-07
1498 2.01284166223559e-07
1499 2.01207654981772e-07
1500 2.01346864514562e-07
1501 2.01192079885004e-07
1502 2.01305283553666e-07
1503 2.01967495172539e-07
1504 2.00901837388301e-07
1505 2.01095204488411e-07
1506 2.00714779907685e-07
1507 2.01017243739443e-07
1508 2.00627439994605e-07
1509 2.00582348952594e-07
1510 2.00679437512008e-07
1511 2.00515245296629e-07
1512 2.0063468753051e-07
1513 2.00476407030692e-07
1514 2.00377840542387e-07
1515 2.00530834604251e-07
1516 2.00355302126809e-07
1517 2.00186946130998e-07
1518 2.00300945607523e-07
1519 2.0020054591896e-07
1520 2.00062530097966e-07
1521 2.00139638195651e-07
1522 1.99993721139435e-07
1523 2.00074069311995e-07
1524 1.99945546341951e-07
1525 1.99899105268742e-07
1526 1.99974962811211e-07
1527 1.9984452137578e-07
1528 1.99911980303114e-07
1529 1.99788246391108e-07
1530 1.99740284756444e-07
1531 1.99758886765267e-07
1532 2.00405992245578e-07
1533 1.99434694536649e-07
1534 1.99504611941848e-07
1535 1.99674019540907e-07
1536 1.9950871887886e-07
1537 1.98318062416547e-07
1538 1.98298593545587e-07
1539 1.98385492922171e-07
1540 1.98235881043729e-07
1541 1.99092312413995e-07
1542 1.97979531435521e-07
1543 1.98064256551334e-07
1544 1.97925658085296e-07
1545 1.98068747181424e-07
1546 1.97801483636795e-07
1547 1.9788630822859e-07
1548 1.97620124708919e-07
1549 1.9768995684899e-07
1550 1.97419055325554e-07
1551 1.97521998757111e-07
1552 1.97519128164458e-07
1553 1.97223428699544e-07
1554 1.97210511032608e-07
1555 1.96999351942395e-07
1556 1.9713257870535e-07
1557 1.97034822235764e-07
1558 1.96813147113062e-07
1559 1.96869976321068e-07
1560 1.96679962982671e-07
1561 1.96704263544234e-07
1562 1.96479305714092e-07
1563 1.96530024254571e-07
1564 1.96506761085402e-07
1565 1.96251974671213e-07
1566 1.9704948783783e-07
1567 1.95991503915138e-07
1568 1.9615085022906e-07
1569 1.96109013472778e-07
1570 1.95917806422585e-07
1571 1.96001977315063e-07
1572 1.95824071624884e-07
1573 1.95945048631074e-07
1574 1.95738763864028e-07
1575 1.97039966565171e-07
1576 1.95533900182454e-07
1577 1.95646080669576e-07
1578 1.9561170461202e-07
1579 1.95435617911244e-07
1580 1.95462533270074e-07
1581 1.95136649949745e-07
1582 1.95131022451278e-07
1583 1.95047519468972e-07
1584 1.94992821889173e-07
1585 1.94898703398394e-07
1586 1.94874758108199e-07
1587 1.94818952081732e-07
1588 1.94778252193828e-07
1589 1.94703119404949e-07
1590 1.94679785181506e-07
1591 1.94625229710255e-07
1592 1.94676729847743e-07
1593 1.9450622801287e-07
1594 1.94499492067735e-07
1595 1.94410318954397e-07
1596 1.9566206788113e-07
1597 1.95192939145272e-07
1598 1.953025758894e-07
1599 1.94367373751447e-07
1600 1.94985389612157e-07
1601 1.94237145478837e-07
1602 1.95145204884284e-07
1603 1.93569931639104e-07
1604 1.93640445900201e-07
1605 1.93676598314596e-07
1606 1.93628011402325e-07
1607 1.93622710753516e-07
1608 1.93589755781431e-07
1609 1.93547379012671e-07
1610 1.93531164427441e-07
1611 1.93452535768301e-07
1612 1.94513987139544e-07
1613 1.93673884041345e-07
1614 1.94286769783503e-07
1615 1.92832203538273e-07
1616 1.94268935160835e-07
1617 1.928351451852e-07
1618 1.92816813182617e-07
1619 1.92813430999195e-07
1620 1.92828835565706e-07
1621 1.92529128639762e-07
1622 1.92388469599791e-07
1623 1.92471816262696e-07
1624 1.92285440903106e-07
1625 1.9380487970011e-07
1626 1.92193084558312e-07
1627 1.92099150808644e-07
1628 1.92167945556321e-07
1629 1.9204051682209e-07
1630 1.93434118500591e-07
1631 1.92474232107998e-07
1632 1.91858561038316e-07
1633 1.93271077364443e-07
1634 1.93179133134436e-07
1635 1.93100717638117e-07
1636 1.91389702308697e-07
1637 1.91407409033673e-07
1638 1.91496255297352e-07
1639 1.9144347618294e-07
1640 1.91501783319836e-07
1641 1.91405348459739e-07
1642 1.91440904018236e-07
1643 1.91347965028399e-07
1644 1.91369892377224e-07
1645 1.91260127735404e-07
1646 1.92815008404068e-07
1647 1.91644986102801e-07
1648 1.91090748558054e-07
1649 1.90982362369141e-07
1650 1.9108992432848e-07
1651 1.90947588407653e-07
1652 1.90958502344074e-07
1653 1.90844502867549e-07
1654 1.90858770565683e-07
1655 1.9075412183156e-07
1656 1.92298188039786e-07
1657 1.92211004446108e-07
1658 1.90312647418978e-07
1659 1.92139950172532e-07
1660 1.90153514267877e-07
1661 1.91917308711709e-07
1662 1.9000391660029e-07
1663 1.91772954849512e-07
1664 1.89875592582212e-07
1665 1.91784835124054e-07
1666 1.9159293174198e-07
1667 1.89655295912416e-07
1668 1.91559195172886e-07
1669 1.91369139201925e-07
1670 1.89595695587741e-07
1671 1.89686971907577e-07
1672 1.91471485777583e-07
1673 1.91242293112737e-07
1674 1.91168183505397e-07
1675 1.9174485998974e-07
1676 1.89285330520761e-07
1677 1.91080332001548e-07
1678 1.91770610058484e-07
1679 1.89037592690511e-07
1680 1.89128627425816e-07
1681 1.90945513622864e-07
1682 1.9108271942514e-07
1683 1.88753048746548e-07
1684 1.90778223441157e-07
1685 1.91054823517334e-07
1686 1.88748515483894e-07
1687 1.90552924550502e-07
1688 1.88780802545807e-07
1689 1.88700838066325e-07
1690 1.88723419114467e-07
1691 1.88605426387767e-07
1692 1.90409807032665e-07
1693 1.90459573445878e-07
1694 1.88302010428743e-07
1695 1.88366200859491e-07
1696 1.90250830200966e-07
1697 1.90862877502695e-07
1698 1.88123820521469e-07
1699 1.900306187963e-07
1700 1.8806836976637e-07
1701 1.88045717663954e-07
1702 1.88074295692786e-07
1703 1.88012236890245e-07
1704 1.8988092165273e-07
1705 1.89870760891608e-07
1706 1.8778196420044e-07
1707 1.8780866639645e-07
1708 1.8776822230393e-07
1709 1.8776201216042e-07
1710 1.89718420529061e-07
1711 1.89453871257683e-07
1712 1.89596647715007e-07
1713 1.87435418297355e-07
1714 1.87324260991772e-07
1715 1.87374681104302e-07
1716 1.87283106356517e-07
1717 1.89108689596651e-07
1718 1.87126389050718e-07
1719 1.87210531521487e-07
1720 1.87074419955024e-07
1721 1.88947026913411e-07
1722 1.88783047860852e-07
1723 1.86738716934087e-07
1724 1.86785669598066e-07
1725 1.88584692750737e-07
1726 1.88411334534067e-07
1727 1.86385122447064e-07
1728 1.88258098887673e-07
1729 1.88069833484406e-07
1730 1.86145285852035e-07
1731 1.86250829870005e-07
1732 1.86285888048587e-07
1733 1.8811152813214e-07
1734 1.87875286883354e-07
1735 1.8596067263843e-07
1736 1.85971174460065e-07
1737 1.85999084578725e-07
1738 1.87761045822299e-07
1739 1.87516093319573e-07
1740 1.85713503242368e-07
1741 1.87587303912551e-07
1742 1.88511052101603e-07
1743 1.85557126997082e-07
1744 1.85557468057596e-07
1745 1.88244044352359e-07
1746 1.85346621606186e-07
1747 1.85416709541641e-07
1748 1.87151101727068e-07
1749 1.86980528837921e-07
1750 1.85083521841989e-07
1751 1.85080367032242e-07
1752 1.85218340220672e-07
1753 1.87140628327143e-07
1754 1.84912053668995e-07
1755 1.85049344736399e-07
1756 1.85014698672603e-07
1757 1.86454045092432e-07
1758 1.84733735864029e-07
1759 1.84853888640646e-07
1760 1.84898425459323e-07
1761 1.84927642976618e-07
1762 1.84935899483207e-07
1763 1.8497840414966e-07
1764 1.86774641974807e-07
1765 1.84665182700883e-07
1766 1.84716768103499e-07
1767 1.8475935803508e-07
1768 1.84806566494444e-07
1769 1.84876327580241e-07
1770 1.84739363362496e-07
1771 1.84765752919702e-07
1772 1.84753304210972e-07
1773 1.84628390798025e-07
1774 1.84627708676999e-07
1775 1.845859713967e-07
1776 1.84599329600132e-07
1777 1.84574190598141e-07
1778 1.84423626592434e-07
1779 1.84422432880638e-07
1780 1.84374925993325e-07
1781 1.84383338819316e-07
1782 1.84357546118008e-07
1783 1.84157258331652e-07
1784 1.84102106004502e-07
1785 1.84087440402436e-07
1786 1.83992582947212e-07
1787 1.83961375910258e-07
1788 1.85577903266676e-07
1789 1.83668902309364e-07
1790 1.83538347187095e-07
1791 1.83711676982057e-07
1792 1.85143150588374e-07
1793 1.83123930241891e-07
1794 1.83360739924865e-07
1795 1.83173042955787e-07
1796 1.83283717092309e-07
1797 1.84775899469969e-07
1798 1.8277594904248e-07
1799 1.82640533807898e-07
1800 1.83913314799611e-07
1801 1.83796672104108e-07
1802 1.8267040502451e-07
1803 1.82502944312546e-07
1804 1.8234705123632e-07
1805 1.83498613637312e-07
1806 1.82120373892758e-07
1807 1.8210054975043e-07
1808 1.82070763798947e-07
1809 1.8201447460342e-07
1810 1.81905150498096e-07
1811 1.81872167104302e-07
1812 1.8184208272487e-07
1813 1.81781132368997e-07
1814 1.81756163897262e-07
1815 1.830920126622e-07
1816 1.81434103296851e-07
1817 1.81429570034197e-07
1818 1.81392877607323e-07
1819 1.82594163788963e-07
1820 1.81129664156288e-07
1821 1.81251181174957e-07
1822 1.81073502858453e-07
1823 1.80974438990233e-07
1824 1.81145423994167e-07
1825 1.81116789121916e-07
1826 1.80861917442598e-07
1827 1.81047056457828e-07
1828 1.82418204985879e-07
1829 1.80910149083502e-07
1830 1.80752209644197e-07
1831 1.82151794092533e-07
1832 1.80536119387398e-07
1833 1.80550600248353e-07
1834 1.8065004780965e-07
1835 1.80488470391538e-07
1836 1.80499995394712e-07
1837 1.80463487708948e-07
1838 1.80428969542845e-07
1839 1.82781633384366e-07
1840 1.80303345587163e-07
1841 1.80143771899566e-07
1842 1.80097018187553e-07
1843 1.80094247070883e-07
1844 1.80007219796607e-07
1845 1.79996931137794e-07
1846 1.79934588118158e-07
1847 1.8220259789814e-07
1848 1.79741675765399e-07
1849 1.79723727455894e-07
1850 1.79663757648996e-07
1851 1.79538901079468e-07
1852 1.79445279968604e-07
1853 1.79409838096944e-07
1854 1.793391817273e-07
1855 1.79257213517303e-07
1856 1.7916880779012e-07
1857 1.79074078232588e-07
1858 1.78990717358829e-07
1859 1.78938620365443e-07
1860 1.79962839297332e-07
1861 1.78684445018007e-07
1862 1.78653181137634e-07
1863 1.78594049771164e-07
1864 1.78516359028436e-07
1865 1.78452296495379e-07
1866 1.78373028347778e-07
1867 1.78347832502368e-07
1868 1.78303253051126e-07
1869 1.78244320636622e-07
1870 1.78208040324535e-07
1871 1.7838765131728e-07
1872 1.79086669049866e-07
1873 1.78189267785456e-07
1874 1.78097167236047e-07
1875 1.7810124575135e-07
1876 1.77985413074566e-07
1877 1.77917016230822e-07
1878 1.7782578254355e-07
1879 1.78596522459884e-07
1880 1.77891877228831e-07
1881 1.77917442556463e-07
1882 1.77942155232813e-07
1883 1.7785347949939e-07
1884 1.77862958139485e-07
1885 1.78450108023753e-07
1886 1.77347217800161e-07
1887 1.77301558323961e-07
1888 1.7729588819293e-07
1889 1.77290260694463e-07
1890 1.77898911601915e-07
1891 1.77617351937442e-07
1892 1.7731035484303e-07
1893 1.77224194430892e-07
1894 1.77127944311906e-07
1895 1.77067818185606e-07
1896 1.77604960072131e-07
1897 1.77327748929201e-07
1898 1.7736225288445e-07
1899 1.76719680666793e-07
1900 1.76712688926273e-07
1901 1.76895980530389e-07
1902 1.7661500351096e-07
1903 1.76551239405853e-07
1904 1.76730665657487e-07
1905 1.76289077558067e-07
1906 1.76519208139325e-07
1907 1.76364252979511e-07
1908 1.76539700191825e-07
1909 1.76111470295837e-07
1910 1.7676195795957e-07
1911 1.76635751358845e-07
1912 1.76538947016525e-07
1913 1.76834944909388e-07
1914 1.76056943246294e-07
1915 1.7617865921693e-07
1916 1.7586603462405e-07
1917 1.75778637867552e-07
1918 1.75667253188294e-07
1919 1.75702126625765e-07
1920 1.75410178826496e-07
1921 1.75636998278605e-07
1922 1.75495941334702e-07
1923 1.75335287622147e-07
1924 1.75425554971298e-07
1925 1.75398682245032e-07
1926 1.75064457152985e-07
1927 1.75263281221305e-07
1928 1.75198792362607e-07
1929 1.75110244526877e-07
1930 1.7479699465639e-07
1931 1.74938293184823e-07
1932 1.74981025224952e-07
1933 1.74811091824267e-07
1934 1.74817770925983e-07
1935 1.74846434219944e-07
1936 1.74609454006713e-07
1937 1.74733500557522e-07
1938 1.74833417077025e-07
1939 1.74672464936521e-07
1940 1.74864780433381e-07
1941 1.74610306657996e-07
1942 1.74586219259254e-07
1943 1.75269903479602e-07
1944 1.75082590203601e-07
1945 1.75159158288807e-07
1946 1.74808391761871e-07
1947 1.75105668631659e-07
1948 1.74730885760255e-07
1949 1.75052107920237e-07
1950 1.74839556166262e-07
1951 1.7436263988202e-07
1952 1.74885272485881e-07
1953 1.74590240931138e-07
1954 1.74108265582618e-07
1955 1.74553647980247e-07
1956 1.74068887304202e-07
1957 1.74487297499581e-07
1958 1.73979628925736e-07
1959 1.74368395278179e-07
1960 1.73902094502409e-07
1961 1.74313228740175e-07
1962 1.73694388649892e-07
1963 1.74144958009492e-07
1964 1.73523702073908e-07
1965 1.74039911371437e-07
1966 1.73318809970624e-07
1967 1.7387584705375e-07
1968 1.73323883245757e-07
1969 1.73641979017702e-07
1970 1.73158966276787e-07
1971 1.73488984955839e-07
1972 1.72846611690147e-07
1973 1.7723350254073e-07
1974 1.76840018184521e-07
1975 1.76954756625491e-07
1976 1.76475765556461e-07
1977 1.76497835013834e-07
1978 1.7651871075941e-07
1979 1.76553740516283e-07
1980 1.76356834913349e-07
1981 1.76973827592519e-07
1982 1.76189047351727e-07
1983 1.76670454266059e-07
1984 1.76095994675052e-07
1985 1.76584435962468e-07
1986 1.75959058879016e-07
1987 1.76389164607826e-07
1988 1.75815586089811e-07
1989 1.76351917957618e-07
1990 1.755733904929e-07
1991 1.76223252879026e-07
1992 1.75512624878138e-07
1993 1.76027114662247e-07
1994 1.75387569356644e-07
1995 1.75885688236121e-07
1996 1.75229160959134e-07
1997 1.75728189333313e-07
1998 1.75019636117213e-07
1999 1.75665803681113e-07
};
\addlegendentry{Test}

\nextgroupplot[
title={4 Layers $\rare$},
ymin=8.43938809238065e-08, ymax=1e-05,
]
\addplot [semithick, black, dashed]
table {%
0 0.0216106624007225
1 0.00673290289845318
2 0.00242847194336355
3 0.00139278739271685
4 0.000877979734446853
5 0.000474738540826365
6 0.000281054378283443
7 0.00021271777068614
8 0.000184169041807763
9 0.000169510987921967
10 0.000160570089246903
11 0.000153623172860534
12 0.000146821876085596
13 0.000139312337923911
14 0.000130708834607503
15 0.000120814576126577
16 0.000109637301975454
17 9.74004520030576e-05
18 8.46704687937745e-05
19 7.22209239793301e-05
20 6.08980650031299e-05
21 5.13222632107499e-05
22 4.36849454090407e-05
23 3.78588456642319e-05
24 3.35587738209142e-05
25 3.04345663553249e-05
26 2.81585673401423e-05
27 2.64713321785166e-05
28 2.51700294393231e-05
29 2.40990629854423e-05
30 2.31376955125597e-05
31 2.21966665994842e-05
32 2.12135114788907e-05
33 2.01455304168121e-05
34 1.89702411180406e-05
35 1.76808702226481e-05
36 1.62871305710723e-05
37 1.48123448534534e-05
38 1.32958351109664e-05
39 1.17858424450787e-05
40 1.0341288148993e-05
41 9.02308400145557e-06
42 7.88079028029642e-06
43 6.9382161775593e-06
44 6.18492073954258e-06
45 5.61335765496551e-06
46 5.18214040664589e-06
47 4.851168252344e-06
48 4.5908486360986e-06
49 4.37404542492459e-06
50 4.1860971711003e-06
51 4.01634194429334e-06
52 3.85815369747888e-06
53 3.70789047474318e-06
54 3.56210532618206e-06
55 3.42163359334791e-06
56 3.28623007618489e-06
57 3.15185335739443e-06
58 3.0234138083074e-06
59 2.90046724973081e-06
60 2.78323152656412e-06
61 2.67022930574967e-06
62 2.56325708880922e-06
63 2.46282143962162e-06
64 2.36972090749532e-06
65 2.28372049502923e-06
66 2.20383440972682e-06
67 2.12981607876372e-06
68 2.0617129700895e-06
69 1.99959229286151e-06
70 1.94295511562359e-06
71 1.89111669618569e-06
72 1.84340184341636e-06
73 1.79986060237525e-06
74 1.75956733130533e-06
75 1.72281113952977e-06
76 1.68883256932872e-06
77 1.65757721526916e-06
78 1.62893817480381e-06
79 1.60244528660769e-06
80 1.57779333051167e-06
81 1.55488546303673e-06
82 1.53347499124834e-06
83 1.51332984992791e-06
84 1.49426139915931e-06
85 1.47615350221031e-06
86 1.45896057071582e-06
87 1.44265126846221e-06
88 1.42635903176824e-06
89 1.41095506359079e-06
90 1.39608382190204e-06
91 1.38187484546393e-06
92 1.3679947178673e-06
93 1.35463125761248e-06
94 1.34165614971948e-06
95 1.32878456537355e-06
96 1.31631478626559e-06
97 1.30407440371982e-06
98 1.29206818090211e-06
99 1.28030122669998e-06
100 1.26874039784752e-06
101 1.25744272938277e-06
102 1.24637065587763e-06
103 1.23556529459279e-06
104 1.22468484934757e-06
105 1.21391777702229e-06
106 1.20343351045449e-06
107 1.19293890546146e-06
108 1.18257735414318e-06
109 1.17147049024879e-06
110 1.16113315289113e-06
111 1.15103359098612e-06
112 1.14106687107096e-06
113 1.13130547171636e-06
114 1.12164716841789e-06
115 1.11212452972609e-06
116 1.10274103678876e-06
117 1.09347531545723e-06
118 1.08429839545465e-06
119 1.07525509415041e-06
120 1.06630471557878e-06
121 1.05751132139176e-06
122 1.0487439832616e-06
123 1.04011279623251e-06
124 1.03160096955435e-06
125 1.02322621967232e-06
126 1.01492625211108e-06
127 1.00679528438263e-06
128 9.98749102421925e-07
129 9.90614294664738e-07
130 9.82658809448367e-07
131 9.74852750289301e-07
132 9.67159750558721e-07
133 9.59597950853208e-07
134 9.52112600600685e-07
135 9.4466021650419e-07
136 9.37624845334994e-07
137 9.30101956882368e-07
138 9.2261810858929e-07
139 9.15006883133174e-07
140 9.07406760788376e-07
141 9.00099601224724e-07
142 8.92701309766153e-07
143 8.85610241880386e-07
144 8.78771238845388e-07
145 8.72219110959804e-07
146 8.65594072266163e-07
147 8.59149445574303e-07
148 8.52838356223629e-07
149 8.46545972649437e-07
150 8.40331516357651e-07
151 8.34286911057802e-07
152 8.28214898177748e-07
153 8.22116376156146e-07
154 8.16204172863877e-07
155 8.10378548209201e-07
156 8.04656332050513e-07
157 7.99051991762667e-07
158 7.93543244043349e-07
159 7.88110822099952e-07
160 7.82816938922792e-07
161 7.77351807641935e-07
162 7.72005825297129e-07
163 7.66674001397405e-07
164 7.61450602340119e-07
165 7.56337212166613e-07
166 7.5132788563792e-07
167 7.46323605710586e-07
168 7.41613327960522e-07
169 7.36650454257415e-07
170 7.31902793432937e-07
171 7.27025475512733e-07
172 7.22359442249854e-07
173 7.177709151307e-07
174 7.13255188415474e-07
175 7.08742484775371e-07
176 7.03956646006532e-07
177 6.98618378038418e-07
178 6.93606436271921e-07
179 6.88884985976301e-07
180 6.84381060153783e-07
181 6.80145616499317e-07
182 6.75724589797255e-07
183 6.71580819329165e-07
184 6.6738621748641e-07
185 6.6328372876967e-07
186 6.59295098714097e-07
187 6.55370623690032e-07
188 6.51456420968088e-07
189 6.4768271292337e-07
190 6.43534997621487e-07
191 6.39708978283693e-07
192 6.359301758323e-07
193 6.32180300485175e-07
194 6.28438367741069e-07
195 6.24702582612713e-07
196 6.21174068299979e-07
197 6.17387885085918e-07
198 6.1360695735857e-07
199 6.09839239018584e-07
200 6.06011450102528e-07
201 6.02128856996842e-07
202 5.98357907904301e-07
203 5.9438991659988e-07
204 5.90488967759484e-07
205 5.86552403930796e-07
206 5.82735191478889e-07
207 5.7880083262063e-07
208 5.7500634336094e-07
209 5.71188780483567e-07
210 5.67494936362323e-07
211 5.63854762901883e-07
212 5.60381861106407e-07
213 5.57017742153221e-07
214 5.53732854783107e-07
215 5.50645551413709e-07
216 5.47619643711528e-07
217 5.44531466303511e-07
218 5.41552928794431e-07
219 5.38660903046662e-07
220 5.35860492874463e-07
221 5.33061054838413e-07
222 5.30413544254316e-07
223 5.27592542923117e-07
224 5.25051639684193e-07
225 5.22539969892932e-07
226 5.20097590815283e-07
227 5.17746779479467e-07
228 5.15497621826455e-07
229 5.13201537245322e-07
230 5.10952482173366e-07
231 5.08788757684897e-07
232 5.06413795463345e-07
233 5.04197328794476e-07
234 5.02117503565103e-07
235 5.00099920714092e-07
236 4.98156201281574e-07
237 4.96120355620633e-07
238 4.94286574834746e-07
239 4.92361857155288e-07
240 4.90496158661813e-07
241 4.88675696999508e-07
242 4.86913171407366e-07
243 4.851756119848e-07
244 4.83470957902909e-07
245 4.81814274188253e-07
246 4.80199573651419e-07
247 4.78683203567698e-07
248 4.77044099817192e-07
249 4.75468882626728e-07
250 4.73879091153151e-07
251 4.72371058982901e-07
252 4.70842513635716e-07
253 4.6937632956201e-07
254 4.67929217776941e-07
255 4.66513790627232e-07
256 4.65088764485699e-07
257 4.6366257139141e-07
258 4.62334256909003e-07
259 4.60908514426706e-07
260 4.59539256539188e-07
261 4.582419610486e-07
262 4.56966788391355e-07
263 4.55675070149653e-07
264 4.54337393499316e-07
265 4.53071155732232e-07
266 4.51853715723871e-07
267 4.50666905862818e-07
268 4.49487872209886e-07
269 4.48350553412524e-07
270 4.47203686391617e-07
271 4.46084542019776e-07
272 4.44943811899634e-07
273 4.43894103753451e-07
274 4.42781260545644e-07
275 4.4177846840654e-07
276 4.40752303987324e-07
277 4.3963250649881e-07
278 4.38508849981645e-07
279 4.37525406496775e-07
280 4.36493751188038e-07
281 4.35531963482561e-07
282 4.34462244669476e-07
283 4.33431620805891e-07
284 4.32435458542102e-07
285 4.31482845442588e-07
286 4.30554591901e-07
287 4.29638015788214e-07
288 4.28745323013402e-07
289 4.2790234015655e-07
290 4.27039721159872e-07
291 4.26191598307923e-07
292 4.25355553360873e-07
293 4.24506018333659e-07
294 4.23655542931556e-07
295 4.22808301834721e-07
296 4.21996215266063e-07
297 4.21198938262535e-07
298 4.20371942965403e-07
299 4.19587211652583e-07
300 4.18860521804731e-07
301 4.18094755104903e-07
302 4.17349502271236e-07
303 4.16623433551422e-07
304 4.15895910307995e-07
305 4.15184494542586e-07
306 4.14479024385628e-07
307 4.13734667318977e-07
308 4.13033298542587e-07
309 4.12367151128024e-07
310 4.11671337289476e-07
311 4.11018487042725e-07
312 4.10363574829375e-07
313 4.09771400768477e-07
314 4.09135555258899e-07
315 4.08534289007889e-07
316 4.07766801657772e-07
317 4.07205162758828e-07
318 4.06719284313795e-07
319 4.05979226258069e-07
320 4.05369681857337e-07
321 4.04813660580317e-07
322 4.0424221703006e-07
323 4.01775835342733e-07
324 3.9732308940188e-07
325 3.95490883548177e-07
326 3.93727580572545e-07
327 3.9239304535954e-07
328 3.91342631729685e-07
329 3.90330171711639e-07
330 3.89447076273086e-07
331 3.88699826885386e-07
332 3.87784748724584e-07
333 3.87108554747329e-07
334 3.86313853951492e-07
335 3.85547236177786e-07
336 3.84728025963454e-07
337 3.83967091067916e-07
338 3.83136319840105e-07
339 3.82331433002037e-07
340 3.81500545188373e-07
341 3.80685198756225e-07
342 3.80033054341311e-07
343 3.79316259142115e-07
344 3.78621012487201e-07
345 3.7795977845434e-07
346 3.77288775951001e-07
347 3.76653722184983e-07
348 3.75958665728149e-07
349 3.75355292362656e-07
350 3.74698903961246e-07
351 3.74061793934288e-07
352 3.73468358191076e-07
353 3.72811548963625e-07
354 3.72241171824328e-07
355 3.71634515659025e-07
356 3.70998225157848e-07
357 3.70405478889779e-07
358 3.69798000349419e-07
359 3.69194266880868e-07
360 3.68567208624881e-07
361 3.679934441152e-07
362 3.67422204163859e-07
363 3.66914002796648e-07
364 3.66252508641196e-07
365 3.65631194782168e-07
366 3.65033643291213e-07
367 3.6445641813998e-07
368 3.63884713749485e-07
369 3.6333000007005e-07
370 3.62738613333136e-07
371 3.62184976779645e-07
372 3.61609488692238e-07
373 3.61079528076402e-07
374 3.60510225192456e-07
375 3.59959687756373e-07
376 3.59414541676983e-07
377 3.58882117126313e-07
378 3.58320710986959e-07
379 3.57816574023673e-07
380 3.5727768839422e-07
381 3.56719101432645e-07
382 3.56145097327953e-07
383 3.55607541536074e-07
384 3.55104071616097e-07
385 3.54350894710365e-07
386 3.53832532411502e-07
387 3.53362483537012e-07
388 3.52820708798163e-07
389 3.52301250558185e-07
390 3.51839069267612e-07
391 3.51369276515356e-07
392 3.50713634929889e-07
393 3.50198180385064e-07
394 3.42994273879071e-07
395 3.37459253060501e-07
396 3.36289835345838e-07
397 3.35470516276359e-07
398 3.34717947012564e-07
399 3.34036376955282e-07
400 3.33413894921364e-07
401 3.32775672802654e-07
402 3.32222780471625e-07
403 3.31656370008204e-07
404 3.31059441123216e-07
405 3.30471022579104e-07
406 3.29948070614705e-07
407 3.29408178927793e-07
408 3.28830659569235e-07
409 3.28420155582876e-07
410 3.27948691108304e-07
411 3.27466074267591e-07
412 3.26983534591818e-07
413 3.26529947187737e-07
414 3.26056213680204e-07
415 3.25601777944939e-07
416 3.25140297491089e-07
417 3.24708497444703e-07
418 3.24223424328807e-07
419 3.23786114506675e-07
420 3.23332559105438e-07
421 3.22898877939792e-07
422 3.22457123729691e-07
423 3.22002698311508e-07
424 3.21506677451566e-07
425 3.21149641436591e-07
426 3.2071827259017e-07
427 3.20318631850114e-07
428 3.19890345039653e-07
429 3.19473695739703e-07
430 3.18698119528449e-07
431 3.18401096876642e-07
432 3.17974057793435e-07
433 3.17500140937454e-07
434 3.17058012129223e-07
435 3.16648788057705e-07
436 3.16250691852815e-07
437 3.15829835571435e-07
438 3.15423972210738e-07
439 3.15066943116449e-07
440 3.14667640367361e-07
441 3.14271113822429e-07
442 3.13797254975157e-07
443 3.13497316142275e-07
444 3.13078043916448e-07
445 3.12755185831293e-07
446 3.12391271933166e-07
447 3.1198606302496e-07
448 3.1160391039009e-07
449 3.11256480088673e-07
450 3.10887099558954e-07
451 3.10662238675263e-07
452 3.10277672426196e-07
453 3.09896373551055e-07
454 3.09517824987893e-07
455 3.09143135638124e-07
456 3.08770616982201e-07
457 3.08405447086102e-07
458 3.08032719928519e-07
459 3.07656596632455e-07
460 3.07311306372071e-07
461 3.06948595408585e-07
462 3.06615132117827e-07
463 3.06262616817321e-07
464 3.05897978648773e-07
465 3.05542331858533e-07
466 3.05163595825775e-07
467 3.04796437418986e-07
468 3.04440751691004e-07
469 3.04076373026874e-07
470 3.03719222827681e-07
471 3.03353062221845e-07
472 3.02990892649291e-07
473 3.0263301108846e-07
474 3.02271752175898e-07
475 3.01934635672296e-07
476 3.0156328080011e-07
477 3.01179149317932e-07
478 3.00842650105437e-07
479 3.00472060928314e-07
480 3.00037528703001e-07
481 2.99667246835611e-07
482 2.99302288965464e-07
483 2.98947667644711e-07
484 2.98582195583208e-07
485 2.98215010573699e-07
486 2.9785535068072e-07
487 2.97490581374404e-07
488 2.97126231615152e-07
489 2.96774992690985e-07
490 2.96397045090657e-07
491 2.96061390230307e-07
492 2.95681678281312e-07
493 2.9532954511069e-07
494 2.94962441074631e-07
495 2.94596214985177e-07
496 2.94228129320118e-07
497 2.93864728405424e-07
498 2.93508675170528e-07
499 2.93143504279669e-07
500 2.92784303780991e-07
501 2.92416588507649e-07
502 2.92051301343577e-07
503 2.91686755957699e-07
504 2.91287320621336e-07
505 2.90917487433262e-07
506 2.90549759711212e-07
507 2.90181204320561e-07
508 2.89812056109895e-07
509 2.89444514450565e-07
510 2.89075045500908e-07
511 2.88704543976337e-07
512 2.88336096332387e-07
513 2.8796272732734e-07
514 2.87592303521933e-07
515 2.87219270418859e-07
516 2.86848664543982e-07
517 2.86476573634786e-07
518 2.86102311221725e-07
519 2.85724018624478e-07
520 2.85357762265903e-07
521 2.84981645876314e-07
522 2.8461455055151e-07
523 2.8425328021342e-07
524 2.83887521433712e-07
525 2.83510480073801e-07
526 2.83132329769842e-07
527 2.82761600246317e-07
528 2.82383754637294e-07
529 2.82003369363792e-07
530 2.81624365769062e-07
531 2.81253791044378e-07
532 2.80863522320374e-07
533 2.80491489519363e-07
534 2.80108860522432e-07
535 2.79729500917369e-07
536 2.79344856437547e-07
537 2.78964073430643e-07
538 2.78581386027099e-07
539 2.78189427277198e-07
540 2.77800684699514e-07
541 2.77422164160157e-07
542 2.77025935531583e-07
543 2.76675719604214e-07
544 2.7628835702842e-07
545 2.75891499214254e-07
546 2.754996529859e-07
547 2.7510834667055e-07
548 2.7471418212599e-07
549 2.74250312244817e-07
550 2.73937980210803e-07
551 2.73469895518019e-07
552 2.73089561275697e-07
553 2.72698263458437e-07
554 2.72312604806757e-07
555 2.71926184367999e-07
556 2.71536050803434e-07
557 2.71139314662605e-07
558 2.70746581449544e-07
559 2.70352984330202e-07
560 2.69956387015213e-07
561 2.69561034599519e-07
562 2.69184922245813e-07
563 2.68785443537922e-07
564 2.68388820202858e-07
565 2.67988743360092e-07
566 2.67580664214506e-07
567 2.6718646597601e-07
568 2.66785075979215e-07
569 2.66419031149212e-07
570 2.66015551616761e-07
571 2.65620774769104e-07
572 2.65219328838384e-07
573 2.64824864359525e-07
574 2.64434130130553e-07
575 2.64023973286953e-07
576 2.63620257200614e-07
577 2.63219747282051e-07
578 2.62736836447175e-07
579 2.62318192085331e-07
580 2.61909523416648e-07
581 2.61502716469408e-07
582 2.61105529247629e-07
583 2.60702718719585e-07
584 2.6029890841528e-07
585 2.59889832378235e-07
586 2.59486747424376e-07
587 2.59076840677608e-07
588 2.58672434839013e-07
589 2.58264190520663e-07
590 2.57854916384304e-07
591 2.57446017087659e-07
592 2.57047310228131e-07
593 2.56630037597461e-07
594 2.56229556200083e-07
595 2.55818167815391e-07
596 2.55413956594452e-07
597 2.55009213333324e-07
598 2.54611830413864e-07
599 2.54200633449386e-07
600 2.5380529264396e-07
601 2.53393371764332e-07
602 2.52922564399682e-07
603 2.52380531748031e-07
604 2.51938129494533e-07
605 2.5150760473025e-07
606 2.51107723556743e-07
607 2.50692401550623e-07
608 2.50296422933616e-07
609 2.49886340895955e-07
610 2.49492911265747e-07
611 2.49087333429543e-07
612 2.48680714250327e-07
613 2.48284265978782e-07
614 2.4788667600717e-07
615 2.47456508517985e-07
616 2.47051033085199e-07
617 2.46647489134944e-07
618 2.4625732633865e-07
619 2.45860469803461e-07
620 2.45425408920141e-07
621 2.45022914640458e-07
622 2.44652590524197e-07
623 2.44252181644811e-07
624 2.43862469332612e-07
625 2.43488337133613e-07
626 2.43087162303368e-07
627 2.42683210387895e-07
628 2.42284148342264e-07
629 2.4189684177145e-07
630 2.41453110305656e-07
631 2.4105597925228e-07
632 2.40666493098729e-07
633 2.402901326235e-07
634 2.39877828683177e-07
635 2.39510399850928e-07
636 2.39118784641335e-07
637 2.38734965940068e-07
638 2.38333517231126e-07
639 2.37971021149974e-07
640 2.37577484739404e-07
641 2.37201121635167e-07
642 2.36807600217048e-07
643 2.36442187372177e-07
644 2.36048477418649e-07
645 2.35720507248516e-07
646 2.3529685198298e-07
647 2.34926964964188e-07
648 2.34651763484806e-07
649 2.34295619804925e-07
650 2.33907534919808e-07
651 2.33553717706059e-07
652 2.33175521088924e-07
653 2.32825393268854e-07
654 2.32450461304268e-07
655 2.32095179185876e-07
656 2.31748523042086e-07
657 2.31385499631642e-07
658 2.31015129955381e-07
659 2.30663613251636e-07
660 2.30288796053912e-07
661 2.29934373187746e-07
662 2.29571597571976e-07
663 2.29216760359918e-07
664 2.28852116933354e-07
665 2.28495701144027e-07
666 2.28147498418707e-07
667 2.27798743303254e-07
668 2.27462652588883e-07
669 2.27117970155177e-07
670 2.26760281407223e-07
671 2.26418564309938e-07
672 2.26108651581569e-07
673 2.25791860522406e-07
674 2.25436486502417e-07
675 2.25098220347775e-07
676 2.24788045336766e-07
677 2.24459609604821e-07
678 2.24133774658242e-07
679 2.23800147416853e-07
680 2.23474523565415e-07
681 2.23188974956656e-07
682 2.2284166180242e-07
683 2.2254372058228e-07
684 2.22211433126063e-07
685 2.21920742731641e-07
686 2.21595553426823e-07
687 2.21314227367486e-07
688 2.21033817595639e-07
689 2.20716097487639e-07
690 2.20356926739385e-07
691 2.2009933127265e-07
692 2.19786205256867e-07
693 2.19499166888681e-07
694 2.19210622162791e-07
695 2.18931428264568e-07
696 2.18612988660993e-07
697 2.18332271472832e-07
698 2.18054914128629e-07
699 2.17747560832038e-07
700 2.17494242413352e-07
701 2.17189853188415e-07
702 2.16910481917409e-07
703 2.16647721089203e-07
704 2.16394595852876e-07
705 2.1610913848491e-07
706 2.15835460842584e-07
707 2.15616975424382e-07
708 2.15359850656682e-07
709 2.15075565435541e-07
710 2.14831694350437e-07
711 2.14503112900388e-07
712 2.14279591034483e-07
713 2.13987736259469e-07
714 2.13740835725673e-07
715 2.13455635510229e-07
716 2.13242247525614e-07
717 2.12986595641951e-07
718 2.12713128561859e-07
719 2.12486485366981e-07
720 2.12225340455063e-07
721 2.11995605106097e-07
722 2.11794130493104e-07
723 2.11524806914554e-07
724 2.1125673990241e-07
725 2.11031820846586e-07
726 2.1081890638186e-07
727 2.10560273671945e-07
728 2.10315091813129e-07
729 2.10102463988449e-07
730 2.09889895714355e-07
731 2.09609642865871e-07
732 2.09425779999606e-07
733 2.09183616597386e-07
734 2.08990658443042e-07
735 2.08729989289225e-07
736 2.08531867073702e-07
737 2.08309944277119e-07
738 2.08086278355779e-07
739 2.0789459659909e-07
740 2.07680310651881e-07
741 2.0747119165776e-07
742 2.07236777498565e-07
743 2.07018175132134e-07
744 2.06816486773675e-07
745 2.06584763589035e-07
746 2.06387027432697e-07
747 2.06218926180668e-07
748 2.06005806859366e-07
749 2.05752826182959e-07
750 2.05548594330196e-07
751 2.05311775921757e-07
752 2.05142609736697e-07
753 2.0495073572846e-07
754 2.04771630293976e-07
755 2.0461071154898e-07
756 2.04373782693779e-07
757 2.04192100383693e-07
758 2.04031139503513e-07
759 2.03814531289481e-07
760 2.03654313601476e-07
761 2.03442323652325e-07
762 2.03277292598614e-07
763 2.0309212533931e-07
764 2.02949136372865e-07
765 2.02724689088996e-07
766 2.02560473084645e-07
767 2.02350404919116e-07
768 2.02225088422381e-07
769 2.02075813668046e-07
770 2.01815807457706e-07
771 2.01668591600423e-07
772 2.01510255024573e-07
773 2.01304897409216e-07
774 2.01147469894636e-07
775 2.0087547069636e-07
776 2.00708041205644e-07
777 2.00552844987101e-07
778 2.00407669666447e-07
779 2.00212583891357e-07
780 2.00079323661839e-07
781 1.99906171182818e-07
782 1.99735537769641e-07
783 1.99590529170734e-07
784 1.99458949076359e-07
785 1.9929100927385e-07
786 1.99101222690956e-07
787 1.98979387548093e-07
788 1.9883520160846e-07
789 1.98662458501531e-07
790 1.98509994092433e-07
791 1.98378701213642e-07
792 1.98230722041615e-07
793 1.98054001302239e-07
794 1.97913957400431e-07
795 1.97748915866214e-07
796 1.97619246243619e-07
797 1.97458854394483e-07
798 1.9732037737441e-07
799 1.9718832942317e-07
800 1.97041426808653e-07
801 1.96902047633785e-07
802 1.96766920048219e-07
803 1.96604294650626e-07
804 1.96496536219115e-07
805 1.96382572973164e-07
806 1.96219839480705e-07
807 1.96085320837369e-07
808 1.95963470162042e-07
809 1.95823555152685e-07
810 1.95713456996316e-07
811 1.95573616785794e-07
812 1.95453819671343e-07
813 1.95311064544512e-07
814 1.95212965564906e-07
815 1.95100162919459e-07
816 1.94962750249772e-07
817 1.94830665954271e-07
818 1.94717411517331e-07
819 1.94620832374426e-07
820 1.94458740004677e-07
821 1.94354109922301e-07
822 1.94270007597197e-07
823 1.94124745341639e-07
824 1.94010768659325e-07
825 1.93872416154761e-07
826 1.93788308628484e-07
827 1.93651885346924e-07
828 1.93532121421924e-07
829 1.93444749363891e-07
830 1.93306050924491e-07
831 1.93204075095821e-07
832 1.9309011060642e-07
833 1.92993874705394e-07
834 1.92862723380927e-07
835 1.92772320488643e-07
836 1.92665639630718e-07
837 1.92546770804825e-07
838 1.92435405381275e-07
839 1.92352559281517e-07
840 1.92240041741343e-07
841 1.92133511006887e-07
842 1.92020548425376e-07
843 1.919115628084e-07
844 1.91812710554018e-07
845 1.91718293251597e-07
846 1.91616724755761e-07
847 1.91497081054592e-07
848 1.91413476578362e-07
849 1.91311501069436e-07
850 1.91188739904646e-07
851 1.91130697771769e-07
852 1.91013988590782e-07
853 1.90901197811399e-07
854 1.90812814622632e-07
855 1.90715631696037e-07
856 1.90646210654677e-07
857 1.90514590059365e-07
858 1.90444030131687e-07
859 1.90331704054358e-07
860 1.90250029120875e-07
861 1.90143055029068e-07
862 1.90024863513827e-07
863 1.89974942529147e-07
864 1.89862021365173e-07
865 1.89752594678794e-07
866 1.89691171975426e-07
867 1.89590054446853e-07
868 1.89481359015531e-07
869 1.89564498441541e-07
870 1.89266790783904e-07
871 1.89147045318805e-07
872 1.89083462664996e-07
873 1.89025151307476e-07
874 1.88914466036749e-07
875 1.88806591928881e-07
876 1.88748112037729e-07
877 1.88681913918742e-07
878 1.88565463105306e-07
879 1.88473345332341e-07
880 1.88422301071967e-07
881 1.88273862789856e-07
882 1.88229817830177e-07
883 1.88129125866965e-07
884 1.88074489372525e-07
885 1.87977990435684e-07
886 1.87873305492303e-07
887 1.87809109206682e-07
888 1.87705275983774e-07
889 1.87653425115286e-07
890 1.87576660152899e-07
891 1.87455449136564e-07
892 1.8739126468148e-07
893 1.87289279985237e-07
894 1.8724850141183e-07
895 1.87153168901943e-07
896 1.8706308216565e-07
897 1.86987817727413e-07
898 1.86928051377322e-07
899 1.86846638250415e-07
900 1.86764167771969e-07
901 1.86676254728013e-07
902 1.86577274298827e-07
903 1.86536912949009e-07
904 1.8646056668814e-07
905 1.86386593952648e-07
906 1.86291404176586e-07
907 1.86223032379473e-07
908 1.86134376484404e-07
909 1.86082618697014e-07
910 1.85978073758974e-07
911 1.85938494226434e-07
912 1.85864153408488e-07
913 1.85785363839841e-07
914 1.85678791211785e-07
915 1.85629285340383e-07
916 1.85553920864834e-07
917 1.85486233405641e-07
918 1.8540947337442e-07
919 1.85350580267141e-07
920 1.85271162948197e-07
921 1.85202297849685e-07
922 1.85131175925335e-07
923 1.85051926912649e-07
924 1.84978863472907e-07
925 1.8492998484021e-07
926 1.84832688674419e-07
927 1.84801209300645e-07
928 1.84736455373979e-07
929 1.84623548165064e-07
930 1.84608434793176e-07
931 1.84504162469068e-07
932 1.84435943999972e-07
933 1.84360517607729e-07
934 1.84322198350628e-07
935 1.84215514472896e-07
936 1.84190147251684e-07
937 1.84140666071642e-07
938 1.84041638341625e-07
939 1.83968719234429e-07
940 1.83896355849811e-07
941 1.83837361070971e-07
942 1.83839822845755e-07
943 1.83766977229993e-07
944 1.83638415258258e-07
945 1.83566072095687e-07
946 1.83519504091123e-07
947 1.83446028835021e-07
948 1.83395068596326e-07
949 1.83316212236662e-07
950 1.83285562606272e-07
951 1.83199140721513e-07
952 1.83183291241562e-07
953 1.83075190882676e-07
954 1.83062210787455e-07
955 1.82927106422426e-07
956 1.82892595525175e-07
957 1.82834764522966e-07
958 1.82783786037533e-07
959 1.82701763108639e-07
960 1.82646872232795e-07
961 1.82573253077578e-07
962 1.82511307478705e-07
963 1.82460096070258e-07
964 1.82422018170314e-07
965 1.82331821022785e-07
966 1.82296495310652e-07
967 1.82236372658906e-07
968 1.8213447961557e-07
969 1.8208163304223e-07
970 1.82025621178639e-07
971 1.81966136040046e-07
972 1.81946712665138e-07
973 1.81872313675058e-07
974 1.81773238892902e-07
975 1.81758913051056e-07
976 1.8167523741397e-07
977 1.81603671109087e-07
978 1.81574254867201e-07
979 1.81481448350951e-07
980 1.8144391169983e-07
981 1.81378839940294e-07
982 1.81291478270396e-07
983 1.81267837923826e-07
984 1.81212172982725e-07
985 1.81145498459045e-07
986 1.81104971638035e-07
987 1.81015536654172e-07
988 1.81003275883995e-07
989 1.80951426081322e-07
990 1.80898478369329e-07
991 1.80804848220362e-07
992 1.80737904678097e-07
993 1.80692982183928e-07
994 1.8063039879479e-07
995 1.80686602881508e-07
996 1.80568369920309e-07
997 1.80461264832843e-07
998 1.80398518651259e-07
999 1.80379310890544e-07
1000 1.80326845374168e-07
1001 1.80228571188934e-07
1002 1.80212151668968e-07
1003 1.80164973421881e-07
1004 1.80055146493885e-07
1005 1.80016070551403e-07
1006 1.80035480369156e-07
1007 1.79911012459399e-07
1008 1.79842733800228e-07
1009 1.79833708145338e-07
1010 1.79744951807947e-07
1011 1.79718168539011e-07
1012 1.79695514212597e-07
1013 1.79632579659028e-07
1014 1.7958827668707e-07
1015 1.79485900808629e-07
1016 1.79413860877276e-07
1017 1.79370492290332e-07
1018 1.79360341761026e-07
1019 1.79332420671585e-07
1020 1.79230263661623e-07
1021 1.7917569885384e-07
1022 1.79127057556627e-07
1023 1.79241113549722e-07
1024 1.7917623971897e-07
1025 1.79244382202626e-07
1026 1.79210308260735e-07
1027 1.79150409216788e-07
1028 1.79036714634151e-07
1029 1.79027852738045e-07
1030 1.78958927357087e-07
1031 1.7893335813568e-07
1032 1.78864548978197e-07
1033 1.78835367890429e-07
1034 1.78795579536484e-07
1035 1.78755134868425e-07
1036 1.78641936344093e-07
1037 1.78593477691891e-07
1038 1.78590952224056e-07
1039 1.78499820442823e-07
1040 1.78439640791339e-07
1041 1.78394428537842e-07
1042 1.78378901708243e-07
1043 1.78302251427453e-07
1044 1.78258820334065e-07
1045 1.78191099934111e-07
1046 1.78159377298925e-07
1047 1.78104258438339e-07
1048 1.78057871373483e-07
1049 1.78015181447222e-07
1050 1.77944034774669e-07
1051 1.77941957588246e-07
1052 1.77853626396995e-07
1053 1.77814524832343e-07
1054 1.7774813699134e-07
1055 1.77703809889351e-07
1056 1.77675952258483e-07
1057 1.77648358352656e-07
1058 1.77571942579391e-07
1059 1.77527472061456e-07
1060 1.77485991635251e-07
1061 1.7744707773204e-07
1062 1.77386032802929e-07
1063 1.77322587973094e-07
1064 1.77299124445085e-07
1065 1.77234235543722e-07
1066 1.77215259405727e-07
1067 1.77158344335737e-07
1068 1.77124247471738e-07
1069 1.77064489179202e-07
1070 1.76985912368366e-07
1071 1.76983701834388e-07
1072 1.76910187533963e-07
1073 1.76876028099571e-07
1074 1.76824726956681e-07
1075 1.76798676029932e-07
1076 1.76776226531672e-07
1077 1.76694430521707e-07
1078 1.76621351180017e-07
1079 1.76610401460664e-07
1080 1.76574187129575e-07
1081 1.7649910562767e-07
1082 1.76468546882802e-07
1083 1.76431862030313e-07
1084 1.7638737516279e-07
1085 1.76306908748813e-07
1086 1.76277939857528e-07
1087 1.76302373539272e-07
1088 1.7617569829298e-07
1089 1.76178361300572e-07
1090 1.76133429633296e-07
1091 1.76041002141858e-07
1092 1.75977817043815e-07
1093 1.75944437359021e-07
1094 1.75941254248357e-07
1095 1.75881839190595e-07
1096 1.75846790924084e-07
1097 1.75819223144913e-07
1098 1.75713790490306e-07
1099 1.75664444050483e-07
1100 1.75637258344352e-07
1101 1.75566675693517e-07
1102 1.75534115825826e-07
1103 1.75526750261668e-07
1104 1.75438703770681e-07
1105 1.7538309885623e-07
1106 1.75376762328483e-07
1107 1.7530193790094e-07
1108 1.75282633335883e-07
1109 1.75220648891639e-07
1110 1.75184468702128e-07
1111 1.75177809552451e-07
1112 1.75096959537768e-07
1113 1.75071028550633e-07
1114 1.75022347598031e-07
1115 1.7497661561805e-07
1116 1.74940363855569e-07
1117 1.74892061075127e-07
1118 1.74832643715206e-07
1119 1.74859169668196e-07
1120 1.74774426554336e-07
1121 1.7468253720665e-07
1122 1.74662733201103e-07
1123 1.74636310930509e-07
1124 1.74569058856378e-07
1125 1.74553399908461e-07
1126 1.74507346848429e-07
1127 1.74474903275268e-07
1128 1.74449821827238e-07
1129 1.74340132559792e-07
1130 1.74361761217767e-07
1131 1.74297626386988e-07
1132 1.74214229467395e-07
1133 1.74233699652859e-07
1134 1.7419528980156e-07
1135 1.74113174260526e-07
1136 1.74103606482845e-07
1137 1.74103460260255e-07
1138 1.73959555390013e-07
1139 1.73925013555731e-07
1140 1.7386210139847e-07
1141 1.73848528262965e-07
1142 1.73805815336436e-07
1143 1.73731946027544e-07
1144 1.73736144176928e-07
1145 1.73698482896612e-07
1146 1.73662524865392e-07
1147 1.73587405967623e-07
1148 1.73579918822497e-07
1149 1.73576758577099e-07
1150 1.735323319636e-07
1151 1.73485661129291e-07
1152 1.73443705548948e-07
1153 1.73351196750104e-07
1154 1.73381033533815e-07
1155 1.73310837986662e-07
1156 1.73295372221105e-07
1157 1.73259445723772e-07
1158 1.73227503879048e-07
1159 1.73195007207028e-07
1160 1.73187520424278e-07
1161 1.7306360564362e-07
1162 1.73050464432833e-07
1163 1.73035037157376e-07
1164 1.73087016712259e-07
1165 1.72968198214107e-07
1166 1.72881617530152e-07
1167 1.7295136483142e-07
1168 1.72851830512855e-07
1169 1.72818079022363e-07
1170 1.72778573158894e-07
1171 1.72718693640661e-07
1172 1.7270090534538e-07
1173 1.7270916892187e-07
1174 1.72568362536651e-07
1175 1.72577032891752e-07
1176 1.7258109477325e-07
1177 1.7260609226355e-07
1178 1.72448717250973e-07
1179 1.72415710743223e-07
1180 1.72395935955194e-07
1181 1.72397184961426e-07
1182 1.72349606657463e-07
1183 1.72301530355412e-07
1184 1.72214634403645e-07
1185 1.72153765760186e-07
1186 1.72193612456795e-07
1187 1.72204249409447e-07
1188 1.72087060846593e-07
1189 1.72026658063373e-07
1190 1.72075992630027e-07
1191 1.72030504209886e-07
1192 1.71926870820016e-07
1193 1.71936961912422e-07
1194 1.71891174524319e-07
1195 1.71880935219804e-07
1196 1.71827827792015e-07
1197 1.71822777382147e-07
1198 1.71786435757326e-07
1199 1.71768815185658e-07
1200 1.71702090860038e-07
1201 1.71739857854902e-07
1202 1.71617547074732e-07
1203 1.7156320400602e-07
1204 1.71528990875913e-07
1205 1.71517066888782e-07
1206 1.71513961973346e-07
1207 1.71403330170961e-07
1208 1.71346198889921e-07
1209 1.71383523543511e-07
1210 1.71362682102938e-07
1211 1.71336470835115e-07
1212 1.71251969312891e-07
1213 1.71174495115167e-07
1214 1.71233604270071e-07
1215 1.71314036798265e-07
1216 1.71165501981818e-07
1217 1.71071120618649e-07
1218 1.70984958153042e-07
1219 1.7098195895926e-07
1220 1.71015771975647e-07
1221 1.70996565053372e-07
1222 1.70884112144165e-07
1223 1.7089750729582e-07
1224 1.70790641050189e-07
1225 1.70798408198891e-07
1226 1.70828165650505e-07
1227 1.70793065173314e-07
1228 1.70700528443035e-07
1229 1.7067476072441e-07
1230 1.70675086287986e-07
1231 1.70612983040996e-07
1232 1.70565547598756e-07
1233 1.7050324265e-07
1234 1.70488775488309e-07
1235 1.70464546236815e-07
1236 1.70451107408098e-07
1237 1.70438943598583e-07
1238 1.70367497503321e-07
1239 1.70297669853881e-07
1240 1.70287840084882e-07
1241 1.70317440392864e-07
1242 1.70259187768806e-07
1243 1.70169387097019e-07
1244 1.70166060954102e-07
1245 1.7015959914346e-07
1246 1.70072879036809e-07
1247 1.70080203453438e-07
1248 1.69973099353626e-07
1249 1.70045703136168e-07
1250 1.69952917950411e-07
1251 1.69955662016719e-07
1252 1.69859973333075e-07
1253 1.69899801925055e-07
1254 1.69820288007827e-07
1255 1.69868564412923e-07
1256 1.69765343969175e-07
1257 1.69748448229257e-07
1258 1.69705800452391e-07
1259 1.6965306291894e-07
1260 1.69602442973371e-07
1261 1.69608080632599e-07
1262 1.69599682138255e-07
1263 1.69567798202763e-07
1264 1.69505588857533e-07
1265 1.69445060365092e-07
1266 1.69388760596689e-07
1267 1.69400478604587e-07
1268 1.69410320395968e-07
1269 1.69311620716428e-07
1270 1.69259400074395e-07
1271 1.69325856880675e-07
1272 1.69370852745487e-07
1273 1.69243204737768e-07
1274 1.69179619966542e-07
1275 1.69161871589552e-07
1276 1.69083306850837e-07
1277 1.69077230957271e-07
1278 1.69089837299907e-07
1279 1.69023000928803e-07
1280 1.69091059262882e-07
1281 1.68983588977767e-07
1282 1.68936555738242e-07
1283 1.68931953034246e-07
1284 1.68985705300884e-07
1285 1.68975296190865e-07
1286 1.68837971287417e-07
1287 1.68774838392949e-07
1288 1.68735167370926e-07
1289 1.68765960466999e-07
1290 1.68746516699514e-07
1291 1.68695649733763e-07
1292 1.68646468864608e-07
1293 1.68562413257689e-07
1294 1.6858988993107e-07
1295 1.68670126527104e-07
1296 1.68366325972613e-07
1297 1.6832450091897e-07
1298 1.68334291473116e-07
1299 1.68343839312968e-07
1300 1.68317190968992e-07
1301 1.68235172893105e-07
1302 1.68274554212644e-07
1303 1.68399181184498e-07
1304 1.68282554334098e-07
1305 1.68080057100894e-07
1306 1.68025432877528e-07
1307 1.68044797234757e-07
1308 1.68017354866379e-07
1309 1.67995841799495e-07
1310 1.68006798560327e-07
1311 1.67982508379794e-07
1312 1.67921054860187e-07
1313 1.67883917420397e-07
1314 1.67839464054964e-07
1315 1.67829756712479e-07
1316 1.67853974403442e-07
1317 1.67749433863662e-07
1318 1.67708428286062e-07
1319 1.67713444042761e-07
1320 1.67860712927848e-07
1321 1.67689125817105e-07
1322 1.67559621274904e-07
1323 1.67547872322871e-07
1324 1.67564348146243e-07
1325 1.67477222703383e-07
1326 1.67485049338723e-07
1327 1.67526131981788e-07
1328 1.6742998275987e-07
1329 1.67414902726648e-07
1330 1.6736702631448e-07
1331 1.67352104973872e-07
1332 1.67339367976638e-07
1333 1.6726915405485e-07
1334 1.67304407696633e-07
1335 1.67264324431926e-07
1336 1.67214283322892e-07
1337 1.67262341591368e-07
1338 1.67241847648825e-07
1339 1.67110093784117e-07
1340 1.67036057391101e-07
1341 1.67139857801146e-07
1342 1.67201785750137e-07
1343 1.66920922900715e-07
1344 1.66950420755541e-07
1345 1.66943121350016e-07
1346 1.66929704967345e-07
1347 1.66912089149207e-07
1348 1.66969160659392e-07
1349 1.66813377553865e-07
1350 1.66787883031816e-07
1351 1.66778691990999e-07
1352 1.66794166055695e-07
1353 1.66790056709942e-07
1354 1.66844713575642e-07
1355 1.66592245811614e-07
1356 1.66663042314497e-07
1357 1.66594229689565e-07
1358 1.66581775118857e-07
1359 1.66553394748803e-07
1360 1.66581533129317e-07
1361 1.66522043343775e-07
1362 1.66457377808626e-07
1363 1.66468092778871e-07
1364 1.66456248294367e-07
1365 1.66391937916899e-07
1366 1.6636954207172e-07
1367 1.66337186605858e-07
1368 1.66337999544908e-07
1369 1.66300411265752e-07
1370 1.66256585025337e-07
1371 1.66230324694538e-07
1372 1.66262054960953e-07
1373 1.66311842029643e-07
1374 1.6620884812113e-07
1375 1.66082737052875e-07
1376 1.66100138692116e-07
1377 1.66071855289829e-07
1378 1.65991287140343e-07
1379 1.66027292841875e-07
1380 1.66048765017024e-07
1381 1.65978204798023e-07
1382 1.65909461287583e-07
1383 1.65914692715319e-07
1384 1.65921796678958e-07
1385 1.65908610057386e-07
1386 1.6584075878967e-07
1387 1.65783007716414e-07
1388 1.65777221020846e-07
1389 1.65749755105082e-07
1390 1.65692761413538e-07
1391 1.65716249071579e-07
1392 1.65707926868208e-07
1393 1.65664443692037e-07
1394 1.65660392909217e-07
1395 1.6555531041007e-07
1396 1.65595974486621e-07
1397 1.65527711487812e-07
1398 1.65510852603745e-07
1399 1.65448139732405e-07
1400 1.65507440037516e-07
1401 1.65409376414516e-07
1402 1.65359866144854e-07
1403 1.65365921347416e-07
1404 1.65365050953881e-07
1405 1.65445104364892e-07
1406 1.65383935843977e-07
1407 1.65204668604702e-07
1408 1.65154134542433e-07
1409 1.65219209932843e-07
1410 1.65202215775651e-07
1411 1.6515671573103e-07
1412 1.65137949920791e-07
1413 1.65116417022659e-07
1414 1.6504532810302e-07
1415 1.65028859363758e-07
1416 1.6504706556475e-07
1417 1.64975679453505e-07
1418 1.64952036158184e-07
1419 1.64949861805042e-07
1420 1.64942287014469e-07
1421 1.64880639310638e-07
1422 1.64852907623469e-07
1423 1.64815023296683e-07
1424 1.64827068971363e-07
1425 1.6479692638427e-07
1426 1.64800686256683e-07
1427 1.64663229007544e-07
1428 1.64735067940569e-07
1429 1.64699877252872e-07
1430 1.64645984632727e-07
1431 1.64649042353915e-07
1432 1.64592018414567e-07
1433 1.64586802981148e-07
1434 1.64533369286346e-07
1435 1.64534088234802e-07
1436 1.64539180580903e-07
1437 1.64588025249657e-07
1438 1.64528626413585e-07
1439 1.64403049730311e-07
1440 1.64434569235539e-07
1441 1.64330305665317e-07
1442 1.64350557383841e-07
1443 1.64313907802693e-07
1444 1.64288167425752e-07
1445 1.64290333536599e-07
1446 1.64252259480691e-07
1447 1.64279479896834e-07
1448 1.6423641449137e-07
1449 1.64184464686912e-07
1450 1.64129400964441e-07
1451 1.64106395708075e-07
1452 1.64089033887649e-07
1453 1.64048898746216e-07
1454 1.64055080958292e-07
1455 1.64035842338706e-07
1456 1.64035157851572e-07
1457 1.63966343592392e-07
1458 1.6393386583502e-07
1459 1.63920002862028e-07
1460 1.63905486353144e-07
1461 1.63842626747623e-07
1462 1.63854071125513e-07
1463 1.63843182377832e-07
1464 1.63801519676099e-07
1465 1.63759081736714e-07
1466 1.63764028513924e-07
1467 1.63723978161556e-07
1468 1.63663750292642e-07
1469 1.63675580736822e-07
1470 1.63701325028853e-07
1471 1.63652787477986e-07
1472 1.63594848658022e-07
1473 1.63543358503659e-07
1474 1.63574874825656e-07
1475 1.63533181869013e-07
1476 1.63477860958494e-07
1477 1.63506070094854e-07
1478 1.6346910118159e-07
1479 1.63388968786649e-07
1480 1.63403992630151e-07
1481 1.63401504394756e-07
1482 1.63334911896129e-07
1483 1.63362521405475e-07
1484 1.63307964221815e-07
1485 1.63266157784392e-07
1486 1.63244747525937e-07
1487 1.63252576001582e-07
1488 1.63205464630778e-07
1489 1.6317926982623e-07
1490 1.63177952131832e-07
1491 1.63116752538883e-07
1492 1.63130517712773e-07
1493 1.63142071770039e-07
1494 1.63082057653696e-07
1495 1.63039829530476e-07
1496 1.62986418189348e-07
1497 1.63003632266623e-07
1498 1.6297419497846e-07
1499 1.62977809573306e-07
1500 1.62908251610361e-07
1501 1.62888266856953e-07
1502 1.62894972540073e-07
1503 1.6281705941168e-07
1504 1.62824144119611e-07
1505 1.62801411775604e-07
1506 1.62785903242479e-07
1507 1.62790458873019e-07
1508 1.62755779378188e-07
1509 1.62805195941473e-07
1510 1.62741083300943e-07
1511 1.62614733277167e-07
1512 1.62588511336992e-07
1513 1.62593399018363e-07
1514 1.62582695779179e-07
1515 1.62611406956614e-07
1516 1.62567546233561e-07
1517 1.62476563382086e-07
1518 1.62453610897728e-07
1519 1.62472841758188e-07
1520 1.62409258251728e-07
1521 1.62407369678874e-07
1522 1.62439174310691e-07
1523 1.62376711131174e-07
1524 1.62357890829412e-07
1525 1.62330951390288e-07
1526 1.62282029890548e-07
1527 1.62258464939669e-07
1528 1.62386171474793e-07
1529 1.62246081977457e-07
1530 1.62181757545454e-07
1531 1.6214893717148e-07
1532 1.62138087752339e-07
1533 1.6213818561539e-07
1534 1.62098755609463e-07
1535 1.62070517177426e-07
1536 1.62076942373801e-07
1537 1.62076044901482e-07
1538 1.62060141597919e-07
1539 1.61987526915652e-07
1540 1.61971739032651e-07
1541 1.61926232046028e-07
1542 1.61918721609311e-07
1543 1.61893961497128e-07
1544 1.6187512852639e-07
1545 1.61897783129916e-07
1546 1.6183849820095e-07
1547 1.61819013932529e-07
1548 1.61767674043745e-07
1549 1.61779257460637e-07
1550 1.61762891366379e-07
1551 1.61720912878138e-07
1552 1.61708021494178e-07
1553 1.61623495422702e-07
1554 1.61689888656724e-07
1555 1.61645654770837e-07
1556 1.61627561659827e-07
1557 1.61584144386495e-07
1558 1.61566055083995e-07
1559 1.61593238935609e-07
1560 1.6156369655107e-07
1561 1.61485519804216e-07
1562 1.6143838396232e-07
1563 1.61465926417748e-07
1564 1.61439363473903e-07
1565 1.61373608236204e-07
1566 1.61376746753206e-07
1567 1.61372395311332e-07
1568 1.61365052711915e-07
1569 1.61270477434528e-07
1570 1.61305438325599e-07
1571 1.6128386074854e-07
1572 1.6120980218659e-07
1573 1.61217556552629e-07
1574 1.61202849270126e-07
1575 1.61158415636464e-07
1576 1.61146072542806e-07
1577 1.61116594973976e-07
1578 1.61089867823705e-07
1579 1.61058869643682e-07
1580 1.61075818958523e-07
1581 1.60997869258495e-07
1582 1.60995202037384e-07
1583 1.60991695985047e-07
1584 1.60972638340695e-07
1585 1.60917201000643e-07
1586 1.60925989412419e-07
1587 1.6090620315623e-07
1588 1.60876287630174e-07
1589 1.60871587006284e-07
1590 1.608295123674e-07
1591 1.60810746869799e-07
1592 1.60792823564293e-07
1593 1.60731222976551e-07
1594 1.60725678398421e-07
1595 1.60743065173108e-07
1596 1.60692538059948e-07
1597 1.60662292991276e-07
1598 1.60687539874971e-07
1599 1.60647360310406e-07
1600 1.60643607117095e-07
1601 1.60562668689579e-07
1602 1.60560070440852e-07
1603 1.60531671774322e-07
1604 1.605189931837e-07
1605 1.60465122299058e-07
1606 1.60481532560652e-07
1607 1.60452848923853e-07
1608 1.60447297758992e-07
1609 1.60448149877368e-07
1610 1.60383949321385e-07
1611 1.60370203410309e-07
1612 1.60340380546131e-07
1613 1.60316806301353e-07
1614 1.60336975469022e-07
1615 1.60274468434807e-07
1616 1.60265671610205e-07
1617 1.60206146965436e-07
1618 1.6017395471124e-07
1619 1.6019242134746e-07
1620 1.60202111970875e-07
1621 1.60110613911968e-07
1622 1.60101618185138e-07
1623 1.60084186518361e-07
1624 1.60075974164897e-07
1625 1.60056341130144e-07
1626 1.60024469295195e-07
1627 1.60016861734391e-07
1628 1.5996215306302e-07
1629 1.59993971742267e-07
1630 1.59975415030544e-07
1631 1.59897546794241e-07
1632 1.59864979707436e-07
1633 1.5981264795073e-07
1634 1.59833385943386e-07
1635 1.59786259438022e-07
1636 1.5975959943404e-07
1637 1.59836109965283e-07
1638 1.59749508142681e-07
1639 1.59673018117701e-07
1640 1.59672461165883e-07
1641 1.59673905095303e-07
1642 1.59624695783123e-07
1643 1.59605662418016e-07
1644 1.59605513445626e-07
1645 1.59538471798726e-07
1646 1.59518216165111e-07
1647 1.59490265389195e-07
1648 1.59514611830502e-07
1649 1.59506289037381e-07
1650 1.594392940234e-07
1651 1.59437268926865e-07
1652 1.59391223355954e-07
1653 1.59379607872268e-07
1654 1.5933557291703e-07
1655 1.59336109298636e-07
1656 1.59273247241742e-07
1657 1.59280083103397e-07
1658 1.59303168210556e-07
1659 1.59266005447023e-07
1660 1.5923947286467e-07
1661 1.59195379566768e-07
1662 1.59185277276208e-07
1663 1.59162616085951e-07
1664 1.59113985738202e-07
1665 1.59115515032227e-07
1666 1.59078035096627e-07
1667 1.590620035401e-07
1668 1.59055500247973e-07
1669 1.59017629734137e-07
1670 1.59004733653489e-07
1671 1.58984582405708e-07
1672 1.5893618891738e-07
1673 1.58970893295418e-07
1674 1.58974270824785e-07
1675 1.58856075486824e-07
1676 1.58825552816211e-07
1677 1.5882929874067e-07
1678 1.58840414911765e-07
1679 1.58816994570543e-07
1680 1.58759385541885e-07
1681 1.58741842831489e-07
1682 1.587232752982e-07
1683 1.58727088788169e-07
1684 1.5868284273779e-07
1685 1.58663293703398e-07
1686 1.58653893663541e-07
1687 1.58617259465643e-07
1688 1.58564271245609e-07
1689 1.58562940306695e-07
1690 1.5858223643761e-07
1691 1.58529176921718e-07
1692 1.58498967628873e-07
1693 1.58529810825314e-07
1694 1.58439630631335e-07
1695 1.58456428877685e-07
1696 1.58413188358963e-07
1697 1.58366994817527e-07
1698 1.5844551713684e-07
1699 1.58414659182426e-07
1700 1.58323922740067e-07
1701 1.58267001054924e-07
1702 1.58278427797143e-07
1703 1.5828589792477e-07
1704 1.58217034922359e-07
1705 1.58233597289836e-07
1706 1.58217247403059e-07
1707 1.58180190943824e-07
1708 1.58171623702685e-07
1709 1.58144930544779e-07
1710 1.5814085011101e-07
1711 1.5810550590345e-07
1712 1.58071546813687e-07
1713 1.58090500818275e-07
1714 1.5802699835632e-07
1715 1.58003030328757e-07
1716 1.57996284798401e-07
1717 1.57944636079321e-07
1718 1.57902301893387e-07
1719 1.57958363530497e-07
1720 1.57928429686649e-07
1721 1.5788781043824e-07
1722 1.5786845246879e-07
1723 1.57857515972637e-07
1724 1.57811574084121e-07
1725 1.57765941366961e-07
1726 1.57808763667333e-07
1727 1.57781808809432e-07
1728 1.57732858816928e-07
1729 1.57714919623686e-07
1730 1.57751770103687e-07
1731 1.57683196277958e-07
1732 1.57685341640956e-07
1733 1.57640333512177e-07
1734 1.57593496886932e-07
1735 1.57609387201774e-07
1736 1.57577048810253e-07
1737 1.57558252041667e-07
1738 1.57547913936185e-07
1739 1.57577006746124e-07
1740 1.57516168158622e-07
1741 1.57490005882721e-07
1742 1.57429037734857e-07
1743 1.57414762966823e-07
1744 1.57409189966984e-07
1745 1.57392864515771e-07
1746 1.57372618552642e-07
1747 1.57369900080084e-07
1748 1.57315704235828e-07
1749 1.57344287217143e-07
1750 1.57322670489179e-07
1751 1.57240089080801e-07
1752 1.57292073694748e-07
1753 1.57229541215997e-07
1754 1.5717209772248e-07
1755 1.57209250694734e-07
1756 1.57179648994088e-07
1757 1.57179907695593e-07
1758 1.57142846951785e-07
1759 1.57105183667738e-07
1760 1.57075447177135e-07
1761 1.57090822746397e-07
1762 1.57044919340876e-07
1763 1.57105361083154e-07
1764 1.57034673691214e-07
1765 1.56961065272299e-07
1766 1.57016002901855e-07
1767 1.56986994667818e-07
1768 1.56888791920551e-07
1769 1.56876593472077e-07
1770 1.56917016937541e-07
1771 1.56922129733061e-07
1772 1.56847977351049e-07
1773 1.56800327566486e-07
1774 1.5682984567178e-07
1775 1.56779214854907e-07
1776 1.56776219405685e-07
1777 1.56752019925932e-07
1778 1.5677984612239e-07
1779 1.56721015706296e-07
1780 1.56645402846323e-07
1781 1.56664259037598e-07
1782 1.5669497613402e-07
1783 1.56634603719397e-07
1784 1.56614299477553e-07
1785 1.56558301490861e-07
1786 1.56580395156425e-07
1787 1.56591386783589e-07
1788 1.5655352395072e-07
1789 1.56526596931883e-07
1790 1.56484596296025e-07
1791 1.56495141503399e-07
1792 1.56515446228411e-07
1793 1.56453958460645e-07
1794 1.5640798075367e-07
1795 1.56401567515729e-07
1796 1.56359416891405e-07
1797 1.56362065531823e-07
1798 1.56313302149158e-07
1799 1.56343904500034e-07
1800 1.56341277055105e-07
1801 1.56306155076891e-07
1802 1.56228445597151e-07
1803 1.56236447892866e-07
1804 1.56230175768712e-07
1805 1.56186808865755e-07
1806 1.5617487574815e-07
1807 1.56168291802317e-07
1808 1.56181358271112e-07
1809 1.56139952530054e-07
1810 1.56133709339201e-07
1811 1.56110389013975e-07
1812 1.56057965199352e-07
1813 1.56061774760019e-07
1814 1.56010338500323e-07
1815 1.56005747960819e-07
1816 1.56006777984885e-07
1817 1.56005831996708e-07
1818 1.55922724239588e-07
1819 1.55939529996374e-07
1820 1.55891446091516e-07
1821 1.55887576582359e-07
1822 1.5586792800093e-07
1823 1.55863338093809e-07
1824 1.55858713945634e-07
1825 1.55812652685938e-07
1826 1.55833806914529e-07
1827 1.55781838373059e-07
1828 1.55788733756879e-07
1829 1.55721982039836e-07
1830 1.55727386349724e-07
1831 1.55851433291332e-07
1832 1.55649586091045e-07
1833 1.55669927124791e-07
1834 1.55658901377365e-07
1835 1.55604411197885e-07
1836 1.55595658576146e-07
1837 1.55614966381279e-07
1838 1.5569612568811e-07
1839 1.5562331170571e-07
1840 1.55492145026415e-07
1841 1.55512192719698e-07
1842 1.55482258158202e-07
1843 1.55477846043084e-07
1844 1.5547972850527e-07
1845 1.55428200905305e-07
1846 1.55491830078347e-07
1847 1.55435238333723e-07
1848 1.55356304126997e-07
1849 1.55355119474621e-07
1850 1.55342363079569e-07
1851 1.55484200355716e-07
1852 1.55344910034216e-07
1853 1.55267062737607e-07
1854 1.55249093879206e-07
1855 1.55265172224972e-07
1856 1.55231461768324e-07
1857 1.55227018687754e-07
1858 1.55229627608833e-07
1859 1.55182650459551e-07
1860 1.55232340787848e-07
1861 1.55126972622099e-07
1862 1.55099817320092e-07
1863 1.55121186253382e-07
1864 1.5512751639335e-07
1865 1.55050447901317e-07
1866 1.55049290604836e-07
1867 1.55044010597294e-07
1868 1.55025785332441e-07
1869 1.55098577394597e-07
1870 1.54999461358329e-07
1871 1.54985007675634e-07
1872 1.54967518348315e-07
1873 1.54887229378176e-07
1874 1.54899446044965e-07
1875 1.54890364761684e-07
1876 1.5493079467177e-07
1877 1.54870422015563e-07
1878 1.54849424923498e-07
1879 1.54797378826288e-07
1880 1.54824989564872e-07
1881 1.54811023548973e-07
1882 1.54755810577001e-07
1883 1.54771308856994e-07
1884 1.54729633891293e-07
1885 1.54714625892893e-07
1886 1.54732818650416e-07
1887 1.54677888644983e-07
1888 1.54708835552242e-07
1889 1.54635840303285e-07
1890 1.54602401082116e-07
1891 1.54665540129884e-07
1892 1.54590238217622e-07
1893 1.54579007322297e-07
1894 1.54548660638909e-07
1895 1.54497419629251e-07
1896 1.54537538065824e-07
1897 1.54494466642063e-07
1898 1.54535719360638e-07
1899 1.54445405819104e-07
1900 1.5440485092455e-07
1901 1.54465775601409e-07
1902 1.54365925610023e-07
1903 1.54361977244832e-07
1904 1.54492287371966e-07
1905 1.54350942509041e-07
1906 1.54304357273816e-07
1907 1.54337254919312e-07
1908 1.54339452990371e-07
1909 1.54368744645694e-07
1910 1.54261017364377e-07
1911 1.5425102045441e-07
1912 1.54197756522478e-07
1913 1.54183083964199e-07
1914 1.54233713530516e-07
1915 1.54137876023697e-07
1916 1.54145033270936e-07
1917 1.54135794069532e-07
1918 1.54119535679342e-07
1919 1.54138514481872e-07
1920 1.54096629827905e-07
1921 1.5406265574569e-07
1922 1.54060015155721e-07
1923 1.54024330797142e-07
1924 1.54051954218915e-07
1925 1.54064928601372e-07
1926 1.53961385088053e-07
1927 1.53930946780179e-07
1928 1.53962987965883e-07
1929 1.53929885584603e-07
1930 1.53935032173536e-07
1931 1.53870435326553e-07
1932 1.53908900053068e-07
1933 1.53856674970143e-07
1934 1.53846832844806e-07
1935 1.53875489658617e-07
1936 1.53802510780565e-07
1937 1.53803029697031e-07
1938 1.53788990317594e-07
1939 1.537533772904e-07
1940 1.53762179159855e-07
1941 1.53716116081171e-07
1942 1.53707295403649e-07
1943 1.53690711535148e-07
1944 1.53655579865131e-07
1945 1.53632156518313e-07
1946 1.53633985100043e-07
1947 1.53624035995392e-07
1948 1.53726699338108e-07
1949 1.53600323478997e-07
1950 1.53522084893609e-07
1951 1.53526479572008e-07
1952 1.53521940212897e-07
1953 1.53634212182396e-07
1954 1.53472079396977e-07
1955 1.53470182361559e-07
1956 1.53480214308388e-07
1957 1.5347605283722e-07
1958 1.53418531354532e-07
1959 1.53409420832418e-07
1960 1.53398752225087e-07
1961 1.5340143021092e-07
1962 1.53342927447397e-07
1963 1.53396333921307e-07
1964 1.53329097088317e-07
1965 1.53354786689874e-07
1966 1.53264965028654e-07
1967 1.53241224921885e-07
1968 1.53321128870232e-07
1969 1.53190591866803e-07
1970 1.53264055889224e-07
1971 1.53205914649845e-07
1972 1.5320671469965e-07
1973 1.53186781268744e-07
1974 1.53127663807595e-07
1975 1.53145490287443e-07
1976 1.53173064497025e-07
1977 1.53088147015978e-07
1978 1.53112812604661e-07
1979 1.53141912839772e-07
1980 1.53054309471656e-07
1981 1.52986316905412e-07
1982 1.52974265290595e-07
1983 1.52974430363884e-07
1984 1.52938154521109e-07
1985 1.52912020645601e-07
1986 1.52903801485138e-07
1987 1.5294464648008e-07
1988 1.52980406049608e-07
1989 1.52853272858522e-07
1990 1.52853196887293e-07
1991 1.52872027200601e-07
1992 1.52813105628979e-07
1993 1.5282282419804e-07
1994 1.52767130309428e-07
1995 1.52762753117486e-07
1996 1.52713809200122e-07
1997 1.52790601468666e-07
1998 1.52704478431076e-07
1999 1.52719887466901e-07
};
\addlegendentry{Train}
\addplot [semithick, black]
table {%
0 0.011574000120163
1 0.00366706307977438
2 0.00170789565891027
3 0.00111751689109951
4 0.000646881584543735
5 0.000352886680047959
6 0.000245284551056102
7 0.000204639916773885
8 0.000184964592335746
9 0.000173768785316497
10 0.000165942328749225
11 0.000158926923177205
12 0.000151453103171661
13 0.000142998978844844
14 0.000133258625282906
15 0.00012215982133057
16 0.000109797903860454
17 9.66031366260722e-05
18 8.32770165288821e-05
19 7.07173239788972e-05
20 5.97596590523608e-05
21 5.08381381223444e-05
22 4.38926981587429e-05
23 3.86942701879889e-05
24 3.48979883710854e-05
25 3.21265833918005e-05
26 3.00821902783355e-05
27 2.85236837953562e-05
28 2.72683046205202e-05
29 2.61690584011376e-05
30 2.51223918894539e-05
31 2.40484914684203e-05
32 2.28933586186031e-05
33 2.16249518416589e-05
34 2.02315659407759e-05
35 1.87114783329889e-05
36 1.70845323737012e-05
37 1.53870423673652e-05
38 1.36686712721712e-05
39 1.19934302347247e-05
40 1.04305418062722e-05
41 9.04368243936915e-06
42 7.8759376265225e-06
43 6.92470666763256e-06
44 6.19254478806397e-06
45 5.6361955103057e-06
46 5.21145375387277e-06
47 4.87881288790959e-06
48 4.61657509731594e-06
49 4.3916338654526e-06
50 4.20265496359207e-06
51 4.03814738092478e-06
52 3.88146781915566e-06
53 3.73240868611902e-06
54 3.5890325307264e-06
55 3.45053081218794e-06
56 3.31653723151248e-06
57 3.180617113685e-06
58 3.04861714539584e-06
59 2.92030449600134e-06
60 2.7996170501865e-06
61 2.68204576059361e-06
62 2.57092233368894e-06
63 2.46800846070983e-06
64 2.37122208091023e-06
65 2.28195494855754e-06
66 2.20259494199126e-06
67 2.12626673601335e-06
68 2.05498326977249e-06
69 1.98996053768496e-06
70 1.93037385542993e-06
71 1.87562704923039e-06
72 1.82363464773516e-06
73 1.77665367573354e-06
74 1.73253715729516e-06
75 1.69225052104593e-06
76 1.6553547084186e-06
77 1.62138883297303e-06
78 1.59035300839605e-06
79 1.56235898884916e-06
80 1.53617838805076e-06
81 1.51190454289463e-06
82 1.48939363953104e-06
83 1.46829404457094e-06
84 1.44854095651681e-06
85 1.43001614105742e-06
86 1.41249040552793e-06
87 1.39639860208263e-06
88 1.38081929890177e-06
89 1.36582798404561e-06
90 1.3515806358555e-06
91 1.33778735289525e-06
92 1.3242972727312e-06
93 1.31096430777689e-06
94 1.2978662198293e-06
95 1.2850188113589e-06
96 1.27255157167383e-06
97 1.26033637570799e-06
98 1.24842938475922e-06
99 1.23707002330775e-06
100 1.22564551929827e-06
101 1.21429775390425e-06
102 1.20332890674035e-06
103 1.1918360769414e-06
104 1.18086018119357e-06
105 1.17003378363734e-06
106 1.15892203211843e-06
107 1.14825797936646e-06
108 1.13600242457323e-06
109 1.12509349037282e-06
110 1.11494068733009e-06
111 1.10509256501246e-06
112 1.09550103388756e-06
113 1.08612948679365e-06
114 1.07696200757346e-06
115 1.0678922990337e-06
116 1.05891751900344e-06
117 1.0499041991352e-06
118 1.04081368590414e-06
119 1.03180457244889e-06
120 1.02287845038518e-06
121 1.01399359664356e-06
122 1.0052174275188e-06
123 9.9669341580011e-07
124 9.88451802186319e-07
125 9.80408231043839e-07
126 9.72623865891364e-07
127 9.64914420364948e-07
128 9.57318775363092e-07
129 9.49509797010251e-07
130 9.41864186643215e-07
131 9.34281558784278e-07
132 9.26734685435804e-07
133 9.19426440759707e-07
134 9.12015252652054e-07
135 9.04922842437372e-07
136 8.97438610536483e-07
137 8.90073920345458e-07
138 8.82843323779525e-07
139 8.75568218816625e-07
140 8.68446079493879e-07
141 8.61326327594725e-07
142 8.54532061111968e-07
143 8.48092724936578e-07
144 8.41753944769152e-07
145 8.35587854908226e-07
146 8.29325244922074e-07
147 8.23131017568812e-07
148 8.17043883216684e-07
149 8.1094850656882e-07
150 8.04856426839251e-07
151 7.99804638518253e-07
152 7.93898664142034e-07
153 7.88029069553886e-07
154 7.81995822762838e-07
155 7.76178239902947e-07
156 7.70457518228795e-07
157 7.64824449106527e-07
158 7.59258000471164e-07
159 7.53204631109838e-07
160 7.47757894714596e-07
161 7.42383235774469e-07
162 7.37123684757535e-07
163 7.3189210070268e-07
164 7.26665348338429e-07
165 7.21628850897105e-07
166 7.16773115527758e-07
167 7.1098833132055e-07
168 7.06490709490026e-07
169 7.02067382007954e-07
170 6.97423786277795e-07
171 6.93051902089792e-07
172 6.88729130615684e-07
173 6.8440965605987e-07
174 6.79950119319983e-07
175 6.75605576816452e-07
176 6.70747397180094e-07
177 6.65254674458993e-07
178 6.60655985029734e-07
179 6.56397730836034e-07
180 6.52389530841901e-07
181 6.48681634629611e-07
182 6.45296154289099e-07
183 6.41551309854549e-07
184 6.38289577636897e-07
185 6.34649552466726e-07
186 6.31188356692292e-07
187 6.26032715445035e-07
188 6.23962023382774e-07
189 6.20240598436794e-07
190 6.16676913978154e-07
191 6.13114991665498e-07
192 6.09508276738779e-07
193 6.06156334015395e-07
194 6.02723559950391e-07
195 6.00569137532148e-07
196 5.97140285663045e-07
197 5.93776462665119e-07
198 5.90396894040168e-07
199 5.86575083616481e-07
200 5.82730763198924e-07
201 5.78846595544746e-07
202 5.74758303173439e-07
203 5.71062741983042e-07
204 5.6730300457275e-07
205 5.63755065741134e-07
206 5.60191267595656e-07
207 5.56695908926486e-07
208 5.53021493487904e-07
209 5.47904960512824e-07
210 5.44347074082907e-07
211 5.41051235813939e-07
212 5.37774440090288e-07
213 5.3457591775441e-07
214 5.31512796442257e-07
215 5.28517830389319e-07
216 5.25565894804458e-07
217 5.22576385719731e-07
218 5.19615412031271e-07
219 5.16793988936115e-07
220 5.13939539814601e-07
221 5.11242149059399e-07
222 5.08597054249549e-07
223 5.06128117194748e-07
224 5.0382772087687e-07
225 5.01570184496813e-07
226 4.99327484249079e-07
227 4.97108032959659e-07
228 4.94572248044278e-07
229 4.92386675432499e-07
230 4.90237198391696e-07
231 4.88122566366656e-07
232 4.8613941316944e-07
233 4.83994483602146e-07
234 4.82092389120226e-07
235 4.80200071706349e-07
236 4.78123013181175e-07
237 4.76248374070565e-07
238 4.74345398515652e-07
239 4.72492956760107e-07
240 4.70647904649013e-07
241 4.68791654384404e-07
242 4.66988893776943e-07
243 4.65245904024414e-07
244 4.63530795968836e-07
245 4.61884667402046e-07
246 4.60318574369012e-07
247 4.58637714473298e-07
248 4.57168653156259e-07
249 4.54612546718636e-07
250 4.53028235369857e-07
251 4.51211690233322e-07
252 4.4972932755627e-07
253 4.48309435796546e-07
254 4.46909439233423e-07
255 4.45462774223415e-07
256 4.4384333364178e-07
257 4.42337722006414e-07
258 4.40380546251617e-07
259 4.39089689052707e-07
260 4.37857494262062e-07
261 4.366330870198e-07
262 4.35459241998615e-07
263 4.31679268331209e-07
264 4.30122213401773e-07
265 4.28812853670024e-07
266 4.27644152978246e-07
267 4.26482699822373e-07
268 4.25356915911834e-07
269 4.2420649037922e-07
270 4.22662679966379e-07
271 4.21609740897111e-07
272 4.2056589677486e-07
273 4.19518642047478e-07
274 4.18486308717547e-07
275 4.1739392031559e-07
276 4.16438382444539e-07
277 4.15426768540783e-07
278 4.14379883295624e-07
279 4.1346791590513e-07
280 4.12524912007939e-07
281 4.11731662097736e-07
282 4.10663517413923e-07
283 4.0974902049129e-07
284 4.08751930081053e-07
285 4.07973942628814e-07
286 4.07007945568694e-07
287 4.06237774086549e-07
288 4.0542386159359e-07
289 4.04617111371408e-07
290 4.03779665703041e-07
291 4.02873013172211e-07
292 4.01829282736799e-07
293 4.01031030605736e-07
294 4.0031130765783e-07
295 3.99608637735582e-07
296 3.98779576471497e-07
297 3.9822057829042e-07
298 3.97490708792247e-07
299 3.96917897660387e-07
300 3.96119702372744e-07
301 3.95498801708527e-07
302 3.94853771013004e-07
303 3.9425222553291e-07
304 3.93532303633037e-07
305 3.92527141457322e-07
306 3.91253138332104e-07
307 3.90722732390714e-07
308 3.90216598589177e-07
309 3.8958322079452e-07
310 3.88906670423239e-07
311 3.88249361549242e-07
312 3.87616012176295e-07
313 3.87023845860313e-07
314 3.86869260182721e-07
315 3.86024908038962e-07
316 3.85521957468882e-07
317 3.8626055243185e-07
318 3.85887119591644e-07
319 3.85291372140273e-07
320 3.84759573535121e-07
321 3.84187245572321e-07
322 3.83851727292495e-07
323 3.82557658440419e-07
324 3.79693375407442e-07
325 3.78235114339986e-07
326 3.76254604361748e-07
327 3.75335702074153e-07
328 3.74560670479696e-07
329 3.73444891010877e-07
330 3.73139982912107e-07
331 3.72862615449776e-07
332 3.72420856820099e-07
333 3.73164709799312e-07
334 3.73019844346345e-07
335 3.72857385855241e-07
336 3.72445867924398e-07
337 3.71896618389655e-07
338 3.71369708318525e-07
339 3.70636030311289e-07
340 3.70102839042374e-07
341 3.69565583469011e-07
342 3.68768354519489e-07
343 3.68172550224699e-07
344 3.67809093404503e-07
345 3.67119611155431e-07
346 3.66304760746061e-07
347 3.66120275430148e-07
348 3.65379548838973e-07
349 3.65072963859348e-07
350 3.6432285810406e-07
351 3.64246261597145e-07
352 3.63825591875866e-07
353 3.63105527867447e-07
354 3.62947218945919e-07
355 3.62348288263092e-07
356 3.61952089633633e-07
357 3.61228728706919e-07
358 3.60791631237589e-07
359 3.60364708740235e-07
360 3.59777459379984e-07
361 3.5935704545409e-07
362 3.58532417976676e-07
363 3.57953780394382e-07
364 3.57419139618287e-07
365 3.56875688112268e-07
366 3.56552618541173e-07
367 3.55851341282687e-07
368 3.55311158273253e-07
369 3.54814204683862e-07
370 3.54620112830162e-07
371 3.54929824197825e-07
372 3.54227182697286e-07
373 3.53933444330323e-07
374 3.53550319687201e-07
375 3.52759201405206e-07
376 3.52269552195139e-07
377 3.52011284121545e-07
378 3.51253163444198e-07
379 3.51150447386317e-07
380 3.49637161889405e-07
381 3.48938726801862e-07
382 3.48291536056422e-07
383 3.47915374732111e-07
384 3.47847446846572e-07
385 3.47481886819878e-07
386 3.46875168588667e-07
387 3.46714813304061e-07
388 3.46323759004008e-07
389 3.45925883493692e-07
390 3.4565402984299e-07
391 3.45379618238439e-07
392 3.44890224823757e-07
393 3.44146030784032e-07
394 3.36688287916331e-07
395 3.34762262355071e-07
396 3.33707816935203e-07
397 3.32992300400292e-07
398 3.32326521856885e-07
399 3.31464406144732e-07
400 3.31011847265472e-07
401 3.30504462908721e-07
402 3.29830101009065e-07
403 3.29425660083871e-07
404 3.28782817859974e-07
405 3.28433287677399e-07
406 3.27996929172514e-07
407 3.27296447721892e-07
408 3.27268566024941e-07
409 3.26741570688682e-07
410 3.2631803037475e-07
411 3.25972393966367e-07
412 3.25581197557767e-07
413 3.25209157381323e-07
414 3.24919938066159e-07
415 3.24388025774169e-07
416 3.24229290527001e-07
417 3.23735889651289e-07
418 3.23417253866864e-07
419 3.22991155599084e-07
420 3.22784956097166e-07
421 3.22435283806044e-07
422 3.22153937304392e-07
423 3.2172994224311e-07
424 3.21632086297541e-07
425 3.21252741741773e-07
426 3.20962698197036e-07
427 3.20572581813394e-07
428 3.20207647064308e-07
429 3.19756196631715e-07
430 3.19627673661671e-07
431 3.19204815468765e-07
432 3.18889675554601e-07
433 3.18575274604882e-07
434 3.18293984946649e-07
435 3.18343666094734e-07
436 3.17982767228386e-07
437 3.17682406603126e-07
438 3.17364794000241e-07
439 3.16984994697123e-07
440 3.16649732212682e-07
441 3.16126261168392e-07
442 3.16085703389035e-07
443 3.15712867404727e-07
444 3.15370357384381e-07
445 3.15339576673068e-07
446 3.15004001549823e-07
447 3.14620791641573e-07
448 3.14270550916262e-07
449 3.1391988386531e-07
450 3.13089259407207e-07
451 3.12688257508853e-07
452 3.12337419927644e-07
453 3.11977174760614e-07
454 3.11632135208129e-07
455 3.11291870502828e-07
456 3.10953282678383e-07
457 3.10632827904556e-07
458 3.10372001877113e-07
459 3.10032049810616e-07
460 3.09867033365663e-07
461 3.0952986662669e-07
462 3.09208076032519e-07
463 3.08884636979201e-07
464 3.08578222529832e-07
465 3.08301252971432e-07
466 3.08034657336975e-07
467 3.07606256910731e-07
468 3.07291571743917e-07
469 3.06981291942066e-07
470 3.06688605178351e-07
471 3.06369400959738e-07
472 3.06053152598906e-07
473 3.05713712123179e-07
474 3.053798423025e-07
475 3.04861373479071e-07
476 3.05539430200952e-07
477 3.05176399706397e-07
478 3.04749789847847e-07
479 3.04363766190363e-07
480 3.03893244790743e-07
481 3.03489088082642e-07
482 3.03117559496968e-07
483 3.02754301628738e-07
484 3.02415003261558e-07
485 3.02079911307374e-07
486 3.01694171866984e-07
487 3.01341884778594e-07
488 3.0099880632406e-07
489 3.00644757089685e-07
490 3.00288945709326e-07
491 2.99927279456824e-07
492 2.99587242125199e-07
493 2.99238195111684e-07
494 2.98902278927926e-07
495 2.98555534072875e-07
496 2.98258953534969e-07
497 2.97935201842847e-07
498 2.97610142752092e-07
499 2.97283349937061e-07
500 2.96965112056569e-07
501 2.9664417411368e-07
502 2.9630987796736e-07
503 2.9598055562019e-07
504 2.95722486498562e-07
505 2.9538978196797e-07
506 2.95058185884045e-07
507 2.94712322101987e-07
508 2.94369954190188e-07
509 2.94032730607796e-07
510 2.93682688834451e-07
511 2.93338757728634e-07
512 2.92983088456822e-07
513 2.92643278498872e-07
514 2.92304946469812e-07
515 2.91953710984671e-07
516 2.91607619828937e-07
517 2.91251382122937e-07
518 2.90900260324634e-07
519 2.90575172812169e-07
520 2.90223084675745e-07
521 2.89879778847535e-07
522 2.89501599581854e-07
523 2.89185152269056e-07
524 2.88839345330416e-07
525 2.88481629695525e-07
526 2.88134714310218e-07
527 2.87771086959765e-07
528 2.87394243514427e-07
529 2.87049402913908e-07
530 2.86693222051326e-07
531 2.86315412267868e-07
532 2.85956446077762e-07
533 2.85580114223194e-07
534 2.85229361907113e-07
535 2.84853683751862e-07
536 2.84494120705858e-07
537 2.84125604821384e-07
538 2.83870150497023e-07
539 2.83490919628093e-07
540 2.8312533117969e-07
541 2.8274163810238e-07
542 2.82374031712607e-07
543 2.8200162205394e-07
544 2.81621538533727e-07
545 2.81268512480892e-07
546 2.8089388592889e-07
547 2.80520936257744e-07
548 2.80160094234816e-07
549 2.7980559025309e-07
550 2.79421868754071e-07
551 2.79073560705001e-07
552 2.78715475587887e-07
553 2.78348835536235e-07
554 2.78005728659991e-07
555 2.77634768508506e-07
556 2.77269691650872e-07
557 2.768990725599e-07
558 2.76530244036621e-07
559 2.76160193379837e-07
560 2.75797845006309e-07
561 2.75427112228499e-07
562 2.75067179700272e-07
563 2.74701903890673e-07
564 2.7433281957201e-07
565 2.73965270025656e-07
566 2.73609543910425e-07
567 2.73232217296027e-07
568 2.72860347649839e-07
569 2.72434988346504e-07
570 2.72053000571759e-07
571 2.71784244887385e-07
572 2.71395975914857e-07
573 2.71007650098909e-07
574 2.70626401288609e-07
575 2.70254929546354e-07
576 2.69863164703565e-07
577 2.69484672799081e-07
578 2.68867324848543e-07
579 2.68489316113119e-07
580 2.68125802449504e-07
581 2.67730001723976e-07
582 2.67355744654196e-07
583 2.66966026174487e-07
584 2.66580883589995e-07
585 2.66187356601222e-07
586 2.65798206555701e-07
587 2.65406129074108e-07
588 2.65013028410976e-07
589 2.64614101297411e-07
590 2.64223103840777e-07
591 2.63831395841407e-07
592 2.63443297399135e-07
593 2.63062531757896e-07
594 2.62660904581935e-07
595 2.62268031292479e-07
596 2.61867967310536e-07
597 2.61462247408417e-07
598 2.6108338602171e-07
599 2.60689262177038e-07
600 2.60297724707925e-07
601 2.59904197719152e-07
602 2.5932695280062e-07
603 2.58735894931306e-07
604 2.58358767268874e-07
605 2.57999460018254e-07
606 2.57653141488845e-07
607 2.57296079553271e-07
608 2.56944787224711e-07
609 2.56603641446418e-07
610 2.56247972174606e-07
611 2.55876727806026e-07
612 2.55541152682781e-07
613 2.55171727303605e-07
614 2.54801648225111e-07
615 2.54435406077391e-07
616 2.54054526749314e-07
617 2.53687204576636e-07
618 2.53374679459739e-07
619 2.5291257088611e-07
620 2.52521687116314e-07
621 2.52182218218877e-07
622 2.51883733426439e-07
623 2.51409375096046e-07
624 2.510828380764e-07
625 2.50676947644024e-07
626 2.50245761890255e-07
627 2.49866133117393e-07
628 2.49488351755645e-07
629 2.49147149133933e-07
630 2.48713462269734e-07
631 2.48321072149338e-07
632 2.48028442229042e-07
633 2.47545159481888e-07
634 2.47190229174521e-07
635 2.46833224082366e-07
636 2.46446376195308e-07
637 2.46034716155918e-07
638 2.45701102130624e-07
639 2.45358478423441e-07
640 2.44987234054861e-07
641 2.44581343622485e-07
642 2.44356158418668e-07
643 2.43886660200587e-07
644 2.43649793674194e-07
645 2.43248194919943e-07
646 2.42877632672389e-07
647 2.42497151248244e-07
648 2.42231067204557e-07
649 2.41791468624797e-07
650 2.41566596059783e-07
651 2.41110427623425e-07
652 2.40842808807429e-07
653 2.40432285636416e-07
654 2.40140423102275e-07
655 2.39740046481529e-07
656 2.39430335113866e-07
657 2.38958989484672e-07
658 2.38692393850215e-07
659 2.3829443307477e-07
660 2.37951979897844e-07
661 2.37546771586494e-07
662 2.37231517985492e-07
663 2.36864622138455e-07
664 2.36517308849216e-07
665 2.36199028336159e-07
666 2.35781655533174e-07
667 2.35482119137487e-07
668 2.35176500495982e-07
669 2.34811381005784e-07
670 2.3447526587006e-07
671 2.34158619605296e-07
672 2.33794096970996e-07
673 2.33537519989113e-07
674 2.33106590030729e-07
675 2.32907524377879e-07
676 2.3253474523699e-07
677 2.32320743975833e-07
678 2.31974524922407e-07
679 2.31625094215815e-07
680 2.31340337109032e-07
681 2.31092514013653e-07
682 2.30703591341808e-07
683 2.30442054771629e-07
684 2.30081525387504e-07
685 2.29810225960136e-07
686 2.29471496027145e-07
687 2.29219878633558e-07
688 2.28838104021634e-07
689 2.28553830083911e-07
690 2.28224678266997e-07
691 2.27979384703758e-07
692 2.27619921133737e-07
693 2.27384390427687e-07
694 2.27092996851752e-07
695 2.26775085820918e-07
696 2.26479698994808e-07
697 2.26220109311726e-07
698 2.25936602760157e-07
699 2.25703516321119e-07
700 2.254302557958e-07
701 2.25145598165e-07
702 2.24920455593747e-07
703 2.24611724775059e-07
704 2.24347132871117e-07
705 2.2408249833461e-07
706 2.23930527454286e-07
707 2.23703963797561e-07
708 2.23431271706431e-07
709 2.23261721998824e-07
710 2.22965169882627e-07
711 2.22815160100254e-07
712 2.22608775857225e-07
713 2.22415039274892e-07
714 2.22238853098133e-07
715 2.22039901132121e-07
716 2.21726409677103e-07
717 2.21545150225211e-07
718 2.21242103748409e-07
719 2.21089408114494e-07
720 2.20802405692666e-07
721 2.20430663944171e-07
722 2.20185839339138e-07
723 2.1988783771576e-07
724 2.19640952536793e-07
725 2.1947865036509e-07
726 2.19261806932991e-07
727 2.19024485659247e-07
728 2.18756127878805e-07
729 2.18547654640133e-07
730 2.18282409036874e-07
731 2.18030095311406e-07
732 2.17866528373634e-07
733 2.17574907424023e-07
734 2.17409663605395e-07
735 2.1716815012951e-07
736 2.16939668007399e-07
737 2.16778417438945e-07
738 2.1647457515428e-07
739 2.16365833693999e-07
740 2.1602446054203e-07
741 2.15883900978042e-07
742 2.15626442923167e-07
743 2.1546036066411e-07
744 2.15647858681223e-07
745 2.15580712392693e-07
746 2.1533863048262e-07
747 2.15239097656195e-07
748 2.14897241335166e-07
749 2.14720657254475e-07
750 2.14489176642019e-07
751 2.14276852261719e-07
752 2.14003208043323e-07
753 2.13844003837949e-07
754 2.13662602277509e-07
755 2.13403296811521e-07
756 2.13128416248765e-07
757 2.12970675761426e-07
758 2.12726050108358e-07
759 2.12609066352343e-07
760 2.12377727848434e-07
761 2.12120355058687e-07
762 2.11968924190842e-07
763 2.11759243029519e-07
764 2.11521026471928e-07
765 2.11349131973293e-07
766 2.11291904861355e-07
767 2.11021401241851e-07
768 2.10844206094407e-07
769 2.10725843885484e-07
770 2.1056639809558e-07
771 2.10400926903276e-07
772 2.10462459904193e-07
773 2.10355452168187e-07
774 2.1009807937844e-07
775 2.10079107887395e-07
776 2.09863443956237e-07
777 2.09755754099206e-07
778 2.09595981459643e-07
779 2.09487652114149e-07
780 2.09303152587381e-07
781 2.09200294420953e-07
782 2.09147160035172e-07
783 2.08933727208205e-07
784 2.08811343327397e-07
785 2.08629757025847e-07
786 2.08507060506236e-07
787 2.08378494903627e-07
788 2.08252046718371e-07
789 2.08054743211505e-07
790 2.07950833441828e-07
791 2.07836549748208e-07
792 2.07747518743417e-07
793 2.07549149422448e-07
794 2.0740286288401e-07
795 2.07233640026061e-07
796 2.0713740411793e-07
797 2.06974561933748e-07
798 2.06849307460288e-07
799 2.06664665824974e-07
800 2.06617016829114e-07
801 2.06395128543591e-07
802 2.06253346846097e-07
803 2.06123658585966e-07
804 2.0606306350146e-07
805 2.05856323987064e-07
806 2.05803587505216e-07
807 2.05639324235563e-07
808 2.05596649038853e-07
809 2.05340100478679e-07
810 2.05299926392399e-07
811 2.05153853016782e-07
812 2.05055059154802e-07
813 2.0491765440056e-07
814 2.04825468586023e-07
815 2.04694970307173e-07
816 2.04536831915902e-07
817 2.04392691216526e-07
818 2.04289094085652e-07
819 2.04134920522847e-07
820 2.04036624040782e-07
821 2.03928394171271e-07
822 2.0385584775795e-07
823 2.03624736627717e-07
824 2.03557220856965e-07
825 2.03502708018277e-07
826 2.03391437025857e-07
827 2.03236567131171e-07
828 2.03131790499356e-07
829 2.03045601665508e-07
830 2.02918585046064e-07
831 2.02826939244005e-07
832 2.02717544084408e-07
833 2.02631397883124e-07
834 2.02510179292403e-07
835 2.02372760327307e-07
836 2.02273369609429e-07
837 2.02256998704797e-07
838 2.02113625391576e-07
839 2.02047189645782e-07
840 2.01891808160326e-07
841 2.01796694909717e-07
842 2.01682837541739e-07
843 2.01673103106259e-07
844 2.01500583330017e-07
845 2.01454554371594e-07
846 2.01320631276758e-07
847 2.01214504613745e-07
848 2.01135875954606e-07
849 2.01066995941801e-07
850 2.00936852934319e-07
851 2.00893310875472e-07
852 2.00752424461825e-07
853 2.00620874579727e-07
854 2.00612490175445e-07
855 2.00512744186199e-07
856 2.00410980255583e-07
857 2.00197234789812e-07
858 2.00219105295218e-07
859 2.0002691769605e-07
860 2.00006809336628e-07
861 1.99923476884578e-07
862 1.99938796185961e-07
863 1.9984901200587e-07
864 1.99821286628321e-07
865 1.99595632466298e-07
866 1.99543109147271e-07
867 1.99589081262275e-07
868 1.99315977056358e-07
869 1.99874861550597e-07
870 1.99124968958131e-07
871 1.98952335495051e-07
872 1.98890177216526e-07
873 1.98814603891151e-07
874 1.98679771301613e-07
875 1.9858663335981e-07
876 1.98599281020506e-07
877 1.98445079035992e-07
878 1.98312548604918e-07
879 1.98282933183691e-07
880 1.98155206021511e-07
881 1.98103421666929e-07
882 1.97989265871001e-07
883 1.97977669813554e-07
884 1.97866143025749e-07
885 1.97791635514477e-07
886 1.9767000480897e-07
887 1.97611171870449e-07
888 1.97537715962426e-07
889 1.97462696860384e-07
890 1.97347304720097e-07
891 1.97358190234809e-07
892 1.97161526216405e-07
893 1.97272726154551e-07
894 1.97102053789422e-07
895 1.96937989471735e-07
896 1.96997191892478e-07
897 1.9698212838648e-07
898 1.96841313027107e-07
899 1.96691090081913e-07
900 1.96711923194925e-07
901 1.964647111663e-07
902 1.9650111937608e-07
903 1.96484947423414e-07
904 1.96302451627162e-07
905 1.96232093685467e-07
906 1.96223652437766e-07
907 1.96024174670129e-07
908 1.96046713085707e-07
909 1.95862767782273e-07
910 1.95863009366803e-07
911 1.95735879060521e-07
912 1.95735651686846e-07
913 1.95593898411062e-07
914 1.95460131635627e-07
915 1.95363540456128e-07
916 1.95322726881386e-07
917 1.95305474903762e-07
918 1.95179922002353e-07
919 1.95101293343214e-07
920 1.95046681028543e-07
921 1.94923799767821e-07
922 1.94888770010948e-07
923 1.94725558344544e-07
924 1.94748864146277e-07
925 1.94603558156814e-07
926 1.94598399616552e-07
927 1.94555454413603e-07
928 1.94394971231304e-07
929 1.94394630170791e-07
930 1.94340145753813e-07
931 1.94154353039266e-07
932 1.94191883906569e-07
933 1.94030391753586e-07
934 1.94043593637616e-07
935 1.93944103443755e-07
936 1.93980355334133e-07
937 1.93812738302768e-07
938 1.93711201745828e-07
939 1.93652752500384e-07
940 1.93593905351008e-07
941 1.93513187696226e-07
942 1.93841856344079e-07
943 1.93322478025948e-07
944 1.93264284575889e-07
945 1.9320289368352e-07
946 1.93143449678246e-07
947 1.93127647207803e-07
948 1.93043973695239e-07
949 1.93014557225979e-07
950 1.92914484387074e-07
951 1.93018308891624e-07
952 1.92791830500028e-07
953 1.92839024748537e-07
954 1.92746284710665e-07
955 1.92608396787364e-07
956 1.92661985920495e-07
957 1.9261233319412e-07
958 1.92549279631749e-07
959 1.92372937135588e-07
960 1.92306259805264e-07
961 1.92160342749048e-07
962 1.92206542237727e-07
963 1.92132844745174e-07
964 1.91974848462451e-07
965 1.91944423022505e-07
966 1.91913713365466e-07
967 1.91649675684857e-07
968 1.91791414749787e-07
969 1.9164853881648e-07
970 1.91528911841488e-07
971 1.91617800737731e-07
972 1.91595148635315e-07
973 1.91296479101766e-07
974 1.91346558153782e-07
975 1.91279809769185e-07
976 1.91170826724374e-07
977 1.91176596331388e-07
978 1.91074377653422e-07
979 1.90990064652397e-07
980 1.90971675806395e-07
981 1.90922932574722e-07
982 1.90977331726572e-07
983 1.90806602518023e-07
984 1.90839514857544e-07
985 1.90741971550779e-07
986 1.90634395380584e-07
987 1.90601610938756e-07
988 1.90523408605259e-07
989 1.90599635629951e-07
990 1.90436537650385e-07
991 1.9025988251542e-07
992 1.90378585784856e-07
993 1.90374009889638e-07
994 1.90159397561729e-07
995 1.91725717968438e-07
996 1.90063147442743e-07
997 1.90030689850573e-07
998 1.89908291758911e-07
999 1.90066714367276e-07
1000 1.89787556337251e-07
1001 1.89885824397606e-07
1002 1.89933103911244e-07
1003 1.8970401072238e-07
1004 1.89629616897946e-07
1005 1.8965111792113e-07
1006 1.89580063647554e-07
1007 1.89562385344288e-07
1008 1.89442019404851e-07
1009 1.89503708725169e-07
1010 1.89262891581166e-07
1011 1.89505371395171e-07
1012 1.89220514812405e-07
1013 1.90751492823438e-07
1014 1.89250840776367e-07
1015 1.89090798130565e-07
1016 1.88996381211837e-07
1017 1.89064778055581e-07
1018 1.89061040600791e-07
1019 1.89030842534521e-07
1020 1.88960669333937e-07
1021 1.88845106663393e-07
1022 1.88846726700831e-07
1023 1.88214130503184e-07
1024 1.88186575655891e-07
1025 1.87995368605698e-07
1026 1.87960196740278e-07
1027 1.87834601206305e-07
1028 1.87655345484927e-07
1029 1.8779016386361e-07
1030 1.87638093507303e-07
1031 1.87418166319731e-07
1032 1.87404793905444e-07
1033 1.8732265516519e-07
1034 1.87314114441506e-07
1035 1.8723257255715e-07
1036 1.870906203294e-07
1037 1.87144038932274e-07
1038 1.87058219580649e-07
1039 1.86962992643203e-07
1040 1.87062497047918e-07
1041 1.87000722462471e-07
1042 1.86903932331006e-07
1043 1.86924665968036e-07
1044 1.86838306603931e-07
1045 1.86858656547884e-07
1046 1.86695828574557e-07
1047 1.86727220352623e-07
1048 1.86580123795466e-07
1049 1.86655555012294e-07
1050 1.86628668075173e-07
1051 1.86480974662118e-07
1052 1.86351272191132e-07
1053 1.86493409159993e-07
1054 1.86258191092747e-07
1055 1.86303424243306e-07
1056 1.86318075634517e-07
1057 1.86220333375786e-07
1058 1.86081066999577e-07
1059 1.86100521659682e-07
1060 1.86008193736598e-07
1061 1.85983751066487e-07
1062 1.85964452725784e-07
1063 1.8584542260669e-07
1064 1.85839127198051e-07
1065 1.85787357054323e-07
1066 1.85791350304498e-07
1067 1.85781161121668e-07
1068 1.85709282618518e-07
1069 1.85676157116177e-07
1070 1.85548799436219e-07
1071 1.86576457394949e-07
1072 1.85332581281727e-07
1073 1.85433350452513e-07
1074 1.8534117884883e-07
1075 1.85417917464292e-07
1076 1.85295974119981e-07
1077 1.85320885748297e-07
1078 1.85289735554761e-07
1079 1.85208094194422e-07
1080 1.8508397658934e-07
1081 1.85067506208725e-07
1082 1.85080565984208e-07
1083 1.8510115751269e-07
1084 1.84827015914379e-07
1085 1.84958736326735e-07
1086 1.84901125521719e-07
1087 1.85815963504865e-07
1088 1.84767813493636e-07
1089 1.84735853281381e-07
1090 1.85530751650731e-07
1091 1.84529127977839e-07
1092 1.84462408014952e-07
1093 1.84515727141843e-07
1094 1.84482900067451e-07
1095 1.84414219006612e-07
1096 1.84996849839081e-07
1097 1.84284743909302e-07
1098 1.84186973228861e-07
1099 1.84361866217841e-07
1100 1.84043585704785e-07
1101 1.84201539354945e-07
1102 1.84032984407168e-07
1103 1.84077691756102e-07
1104 1.83915901175169e-07
1105 1.84014538717747e-07
1106 1.83855348723228e-07
1107 1.84071552666865e-07
1108 1.83841606826718e-07
1109 1.8376117338903e-07
1110 1.83773110506991e-07
1111 1.83734485403875e-07
1112 1.8369206600255e-07
1113 1.83688399602033e-07
1114 1.83622276495043e-07
1115 1.83969461886591e-07
1116 1.83549161647534e-07
1117 1.83535291853332e-07
1118 1.83494847760812e-07
1119 1.84823903737197e-07
1120 1.83405219900123e-07
1121 1.83314782020716e-07
1122 1.8338616314395e-07
1123 1.83294474709328e-07
1124 1.83377025564369e-07
1125 1.83271680498365e-07
1126 1.8307480331714e-07
1127 1.83328410230388e-07
1128 1.8292064396519e-07
1129 1.83168467060568e-07
1130 1.83083727733901e-07
1131 1.82902581968847e-07
1132 1.82867566422829e-07
1133 1.829336326864e-07
1134 1.82953300509325e-07
1135 1.82758668643146e-07
1136 1.8298173642961e-07
1137 1.82719460894987e-07
1138 1.82594050102125e-07
1139 1.82521759484189e-07
1140 1.82621434419161e-07
1141 1.82701953121978e-07
1142 1.82516359359397e-07
1143 1.82474011012346e-07
1144 1.8249750155519e-07
1145 1.82308795615427e-07
1146 1.82321628017235e-07
1147 1.82332172471433e-07
1148 1.82286044037028e-07
1149 1.82213142352339e-07
1150 1.82185715402738e-07
1151 1.83677997256382e-07
1152 1.82122150249597e-07
1153 1.82006644422472e-07
1154 1.8211254371181e-07
1155 1.81901384621597e-07
1156 1.82055444497564e-07
1157 1.81778986529935e-07
1158 1.81683333266847e-07
1159 1.81827360279385e-07
1160 1.81869197035667e-07
1161 1.81896695039541e-07
1162 1.82440331286671e-07
1163 1.81639649099452e-07
1164 1.81727486392447e-07
1165 1.81627513029525e-07
1166 1.81488346129299e-07
1167 1.82795204750619e-07
1168 1.81510557695219e-07
1169 1.81478569061255e-07
1170 1.81518700514971e-07
1171 1.81142922883737e-07
1172 1.81381352604149e-07
1173 1.81303548174583e-07
1174 1.81222247874757e-07
1175 1.80938059202163e-07
1176 1.81252318043335e-07
1177 1.81822315425961e-07
1178 1.81025427536952e-07
1179 1.8073959040521e-07
1180 1.81078007699398e-07
1181 1.80749125888724e-07
1182 1.80973074748181e-07
1183 1.80892925527587e-07
1184 1.80828536144872e-07
1185 1.80731547061441e-07
1186 1.8071089868954e-07
1187 1.80803326088608e-07
1188 1.80584720510524e-07
1189 1.80779139213882e-07
1190 1.80384489567587e-07
1191 1.80641407609983e-07
1192 1.80451664277825e-07
1193 1.80404811089829e-07
1194 1.8037310667296e-07
1195 1.80663604965048e-07
1196 1.80323297627183e-07
1197 1.80226564339137e-07
1198 1.81846758096071e-07
1199 1.80159318574624e-07
1200 1.80288594719968e-07
1201 1.80242636815819e-07
1202 1.80113900682954e-07
1203 1.79901135766158e-07
1204 1.80144169803498e-07
1205 1.80617405476369e-07
1206 1.80242480496418e-07
1207 1.80142052386145e-07
1208 1.80058179921616e-07
1209 1.79839048541908e-07
1210 1.8004276114425e-07
1211 1.79952166945441e-07
1212 1.79826002977279e-07
1213 1.79858432147739e-07
1214 1.79809561018374e-07
1215 1.80875900923638e-07
1216 1.79592049676103e-07
1217 1.7973616195377e-07
1218 1.79538915290323e-07
1219 1.79571117087107e-07
1220 1.79404835876085e-07
1221 1.794761601559e-07
1222 1.7938387486538e-07
1223 1.79423224722086e-07
1224 1.79462901428451e-07
1225 1.79437833480733e-07
1226 1.79455639681692e-07
1227 1.79442650960482e-07
1228 1.79239194153524e-07
1229 1.79304066705299e-07
1230 1.79607212658084e-07
1231 1.790301098481e-07
1232 1.79042018544351e-07
1233 1.79017291657146e-07
1234 1.78841460751755e-07
1235 1.79632266394947e-07
1236 1.78872369360761e-07
1237 1.78704652853412e-07
1238 1.78885571244791e-07
1239 1.7887290937324e-07
1240 1.78472689071896e-07
1241 1.78852005205954e-07
1242 1.78727816546598e-07
1243 1.78735888312076e-07
1244 1.79011706791243e-07
1245 1.78413671392263e-07
1246 1.78404960138323e-07
1247 1.78690768848355e-07
1248 1.78492811642172e-07
1249 1.78337430156716e-07
1250 1.78357637992121e-07
1251 1.78483276158659e-07
1252 1.78372374648461e-07
1253 1.78282476781533e-07
1254 1.78108919612896e-07
1255 1.79667040356435e-07
1256 1.78074216705681e-07
1257 1.77976943405156e-07
1258 1.77984986748925e-07
1259 1.78031299924442e-07
1260 1.78007084628007e-07
1261 1.77890598251906e-07
1262 1.78073875645168e-07
1263 1.78129411665395e-07
1264 1.77943846324524e-07
1265 1.77811486423707e-07
1266 1.77917627297575e-07
1267 1.77878973772749e-07
1268 1.77870134621116e-07
1269 1.77607063278629e-07
1270 1.77748134433386e-07
1271 1.78912742399007e-07
1272 1.77611667595556e-07
1273 1.775171654117e-07
1274 1.7770254601146e-07
1275 1.77629516429079e-07
1276 1.77571692461242e-07
1277 1.77512347931952e-07
1278 1.77475925511317e-07
1279 1.77320032435091e-07
1280 1.7723651524193e-07
1281 1.77412758262108e-07
1282 1.77462609940449e-07
1283 1.77372044163349e-07
1284 1.77243649090997e-07
1285 1.78109686999051e-07
1286 1.77164025672027e-07
1287 1.77204640294804e-07
1288 1.77085695440837e-07
1289 1.76995868628183e-07
1290 1.76999108703058e-07
1291 1.76939849438895e-07
1292 1.76903881765611e-07
1293 1.77263032696828e-07
1294 1.76810445395859e-07
1295 1.76515200678296e-07
1296 1.76605865931379e-07
1297 1.76525020378904e-07
1298 1.7678694064216e-07
1299 1.76149654862456e-07
1300 1.76472894963808e-07
1301 1.7657184514519e-07
1302 1.77758323616217e-07
1303 1.76409912455711e-07
1304 1.76430489773338e-07
1305 1.76421067976662e-07
1306 1.7624529391469e-07
1307 1.76370861026953e-07
1308 1.75778339439603e-07
1309 1.76284672193106e-07
1310 1.76422943809484e-07
1311 1.76130328100044e-07
1312 1.7588477874142e-07
1313 1.76091376147269e-07
1314 1.76111754512931e-07
1315 1.76088420289489e-07
1316 1.75862268747551e-07
1317 1.76040828137047e-07
1318 1.76079637981275e-07
1319 1.76055607425951e-07
1320 1.75565105564601e-07
1321 1.75961716308848e-07
1322 1.75883016595435e-07
1323 1.75878071217994e-07
1324 1.75420694858985e-07
1325 1.75631811316634e-07
1326 1.75964785853466e-07
1327 1.75928391854541e-07
1328 1.7535859342388e-07
1329 1.75851312178565e-07
1330 1.75633914523132e-07
1331 1.7585780653917e-07
1332 1.75147192749137e-07
1333 1.75812360225791e-07
1334 1.75911182509481e-07
1335 1.75566412963235e-07
1336 1.75205016716973e-07
1337 1.75830180637604e-07
1338 1.75681819314377e-07
1339 1.75440362681911e-07
1340 1.7491549897386e-07
1341 1.76246516048195e-07
1342 1.75308628058701e-07
1343 1.75312365513491e-07
1344 1.74829125398901e-07
1345 1.75325666873505e-07
1346 1.75078938013939e-07
1347 1.76410395624771e-07
1348 1.7491076675924e-07
1349 1.74998660895653e-07
1350 1.75054623241522e-07
1351 1.75192084839182e-07
1352 1.74669381181047e-07
1353 1.75237161670339e-07
1354 1.75300542082368e-07
1355 1.74887333059814e-07
1356 1.74381867168449e-07
1357 1.74896001681191e-07
1358 1.7490624770744e-07
1359 1.74956824139372e-07
1360 1.7453675127399e-07
1361 1.74831711774459e-07
1362 1.7480473957221e-07
1363 1.74656065610179e-07
1364 1.74281950648947e-07
1365 1.74574481093259e-07
1366 1.75144876379818e-07
1367 1.74581884948566e-07
1368 1.74261401753029e-07
1369 1.74665785834804e-07
1370 1.74713804312887e-07
1371 1.74474749314868e-07
1372 1.74110283523987e-07
1373 1.76212793689956e-07
1374 1.74366221017408e-07
1375 1.74398479657611e-07
1376 1.73945394976727e-07
1377 1.74452509327239e-07
1378 1.74354880755345e-07
1379 1.74328263824464e-07
1380 1.74061213442656e-07
1381 1.74219522364183e-07
1382 1.7433744403661e-07
1383 1.74226826743507e-07
1384 1.73804053815729e-07
1385 1.74105792893897e-07
1386 1.7441851696276e-07
1387 1.74075950098995e-07
1388 1.7369450233673e-07
1389 1.74163417909767e-07
1390 1.74298833144348e-07
1391 1.74266688190983e-07
1392 1.73596902186546e-07
1393 1.74213170112125e-07
1394 1.73933599967313e-07
1395 1.73927048763289e-07
1396 1.73521854662795e-07
1397 1.73982755313773e-07
1398 1.73909796785665e-07
1399 1.73941074876893e-07
1400 1.73688448512621e-07
1401 1.73859760366213e-07
1402 1.74013294440556e-07
1403 1.73976758333083e-07
1404 1.73476252030014e-07
1405 1.74111860928861e-07
1406 1.73771468325867e-07
1407 1.7387439754657e-07
1408 1.73742250808573e-07
1409 1.73987288576427e-07
1410 1.73758465393803e-07
1411 1.73862204633224e-07
1412 1.74962366372711e-07
1413 1.7385194439612e-07
1414 1.73645844370185e-07
1415 1.73709395312471e-07
1416 1.73724515661888e-07
1417 1.73608768250233e-07
1418 1.7366807014696e-07
1419 1.7371345961692e-07
1420 1.73671736547476e-07
1421 1.73739294950792e-07
1422 1.73554994375991e-07
1423 1.73554553839494e-07
1424 1.73539390857513e-07
1425 1.73823565319253e-07
1426 1.73426286664835e-07
1427 1.73303334349839e-07
1428 1.73739806541562e-07
1429 1.73376108136836e-07
1430 1.73401161873699e-07
1431 1.73426471405946e-07
1432 1.73302751704796e-07
1433 1.73377131318375e-07
1434 1.73294495198206e-07
1435 1.73428730931846e-07
1436 1.7332151003302e-07
1437 1.735969163974e-07
1438 1.73251777368932e-07
1439 1.72680728383057e-07
1440 1.73319307350539e-07
1441 1.73210906950771e-07
1442 1.73224037780528e-07
1443 1.7301186971963e-07
1444 1.73086107224663e-07
1445 1.7307178268311e-07
1446 1.73090313637658e-07
1447 1.73714497009314e-07
1448 1.72976356793697e-07
1449 1.73036525552561e-07
1450 1.72929347286299e-07
1451 1.73107480350154e-07
1452 1.72843826362623e-07
1453 1.72826233324486e-07
1454 1.72873626524961e-07
1455 1.72906339912515e-07
1456 1.72809592413614e-07
1457 1.72787835595045e-07
1458 1.72803211739847e-07
1459 1.72834646150477e-07
1460 1.72652562469011e-07
1461 1.72748627846886e-07
1462 1.7262392759676e-07
1463 1.72669516018686e-07
1464 1.7262220808334e-07
1465 1.7254559736557e-07
1466 1.72821415844737e-07
1467 1.72569002643286e-07
1468 1.72634031514463e-07
1469 1.72533347608805e-07
1470 1.72794244690522e-07
1471 1.72430233646992e-07
1472 1.72533233921968e-07
1473 1.72629299299842e-07
1474 1.72570352674484e-07
1475 1.72572356404999e-07
1476 1.72361637851282e-07
1477 1.72881868820696e-07
1478 1.72376871887536e-07
1479 1.72366299011628e-07
1480 1.72450668856072e-07
1481 1.72460659086937e-07
1482 1.72133951537035e-07
1483 1.72577358625858e-07
1484 1.72221191974131e-07
1485 1.72350880234262e-07
1486 1.72334836179289e-07
1487 1.72402465636878e-07
1488 1.72095013795115e-07
1489 1.7228987303497e-07
1490 1.72288068256421e-07
1491 1.72264378761611e-07
1492 1.72138015841483e-07
1493 1.72183618474264e-07
1494 1.71976694218756e-07
1495 1.71964174455752e-07
1496 1.7207158009569e-07
1497 1.71839417362207e-07
1498 1.72100669715292e-07
1499 1.7209153213571e-07
1500 1.71793089975836e-07
1501 1.72068894244148e-07
1502 1.71826542327835e-07
1503 1.71907103663216e-07
1504 1.71855788266839e-07
1505 1.71913328017581e-07
1506 1.7174174615775e-07
1507 1.72048856939e-07
1508 1.71569581652875e-07
1509 1.71813255178677e-07
1510 1.71826854966639e-07
1511 1.71759651834691e-07
1512 1.71732622789023e-07
1513 1.71685400118804e-07
1514 1.71876379795322e-07
1515 1.73053336993689e-07
1516 1.71630603063022e-07
1517 1.71644842339447e-07
1518 1.71472137822093e-07
1519 1.71307661389619e-07
1520 1.71374978208405e-07
1521 1.71292938944134e-07
1522 1.71478475863296e-07
1523 1.71240387203397e-07
1524 1.71151285144333e-07
1525 1.71372875001907e-07
1526 1.71225863709878e-07
1527 1.71040454688409e-07
1528 1.71470546206365e-07
1529 1.71406981053224e-07
1530 1.71459745956781e-07
1531 1.71104233004371e-07
1532 1.7151393194581e-07
1533 1.71423280903582e-07
1534 1.71292427353364e-07
1535 1.71475988963721e-07
1536 1.7136076735369e-07
1537 1.71799413806184e-07
1538 1.71348901290003e-07
1539 1.71122593428663e-07
1540 1.71156244732629e-07
1541 1.71167727103239e-07
1542 1.71288803585412e-07
1543 1.70980698044332e-07
1544 1.71240600366218e-07
1545 1.71233409673732e-07
1546 1.70937127563775e-07
1547 1.71003165405637e-07
1548 1.70948027289342e-07
1549 1.70915825492557e-07
1550 1.7093138637847e-07
1551 1.71056228737143e-07
1552 1.71015344108127e-07
1553 1.70754688610941e-07
1554 1.7096333237987e-07
1555 1.70760884543597e-07
1556 1.70990176684427e-07
1557 1.70694121948145e-07
1558 1.70758781337099e-07
1559 1.71745710986215e-07
1560 1.70942414001729e-07
1561 1.70399218291095e-07
1562 1.70658353226827e-07
1563 1.70728640114248e-07
1564 1.70738104543489e-07
1565 1.70530427112681e-07
1566 1.70573628111015e-07
1567 1.70522852727117e-07
1568 1.70571169633149e-07
1569 1.70351327710705e-07
1570 1.70547210132099e-07
1571 1.70312944192119e-07
1572 1.7053545775525e-07
1573 1.70197708371234e-07
1574 1.70418843481457e-07
1575 1.70292793200133e-07
1576 1.70418758216329e-07
1577 1.70181380099166e-07
1578 1.70234272900416e-07
1579 1.70269288446434e-07
1580 1.70314621072976e-07
1581 1.69923410453521e-07
1582 1.70150940448366e-07
1583 1.70156894796492e-07
1584 1.70127279375265e-07
1585 1.69874780908685e-07
1586 1.70108179986528e-07
1587 1.69944527783628e-07
1588 1.70072880223415e-07
1589 1.69921975157195e-07
1590 1.70010139299848e-07
1591 1.69966739349547e-07
1592 1.699506242403e-07
1593 1.69701635854835e-07
1594 1.70051833947582e-07
1595 1.6994393092773e-07
1596 1.69954702755604e-07
1597 1.69803044514083e-07
1598 1.69787142567657e-07
1599 1.70144829780838e-07
1600 1.69827615081886e-07
1601 1.69598806110116e-07
1602 1.69755679735317e-07
1603 1.69697145224745e-07
1604 1.69664090776678e-07
1605 1.69522849091663e-07
1606 1.69643811886999e-07
1607 1.69700228980219e-07
1608 1.69723719523063e-07
1609 1.69596589216781e-07
1610 1.69558958873495e-07
1611 1.69728423315973e-07
1612 1.69601193533708e-07
1613 1.69507671898828e-07
1614 1.69442031960898e-07
1615 1.6954324166818e-07
1616 1.69510016689856e-07
1617 1.69150666806672e-07
1618 1.69324280818728e-07
1619 1.69240479408472e-07
1620 1.69296527019469e-07
1621 1.69031210361936e-07
1622 1.69308137287771e-07
1623 1.69192105659022e-07
1624 1.69330789390187e-07
1625 1.69112212233813e-07
1626 1.6933465474267e-07
1627 1.69268645322518e-07
1628 1.69273320693719e-07
1629 1.6977925554329e-07
1630 1.69205705446984e-07
1631 1.6902103538996e-07
1632 1.69069849675907e-07
1633 1.6887233300622e-07
1634 1.69193810961588e-07
1635 1.68896519880946e-07
1636 1.69149885209663e-07
1637 1.70203762195342e-07
1638 1.68955438084595e-07
1639 1.68779223486126e-07
1640 1.69076372458221e-07
1641 1.68629668451103e-07
1642 1.68882920092983e-07
1643 1.68704957559385e-07
1644 1.69026264984495e-07
1645 1.68626442587083e-07
1646 1.68642714015732e-07
1647 1.68775983411251e-07
1648 1.68898708352572e-07
1649 1.6868821717253e-07
1650 1.68638123909659e-07
1651 1.68667739330886e-07
1652 1.6874137998002e-07
1653 1.68419546753285e-07
1654 1.68521978594072e-07
1655 1.68542825917939e-07
1656 1.68474755923853e-07
1657 1.68410693390797e-07
1658 1.68627877883409e-07
1659 1.68414814538664e-07
1660 1.68854185744749e-07
1661 1.68217866303166e-07
1662 1.68444856285532e-07
1663 1.68451833815197e-07
1664 1.68446348425277e-07
1665 1.68113274412462e-07
1666 1.68436656622362e-07
1667 1.6834117388953e-07
1668 1.68711778769648e-07
1669 1.68170274150725e-07
1670 1.68412825019004e-07
1671 1.68350780427318e-07
1672 1.68286447888022e-07
1673 1.69203829614162e-07
1674 1.68452189086565e-07
1675 1.68168014624825e-07
1676 1.68284344681524e-07
1677 1.68075033002424e-07
1678 1.68234777220277e-07
1679 1.68481079754201e-07
1680 1.68194532079724e-07
1681 1.68028705616052e-07
1682 1.68392418231633e-07
1683 1.68273217582282e-07
1684 1.68120180887854e-07
1685 1.67898676295408e-07
1686 1.68009535173042e-07
1687 1.67987934673874e-07
1688 1.67987380450541e-07
1689 1.67894597780105e-07
1690 1.67981696108654e-07
1691 1.67987437293959e-07
1692 1.68139905554199e-07
1693 1.67758997804413e-07
1694 1.67980346077456e-07
1695 1.68025820812545e-07
1696 1.67899656844384e-07
1697 1.67814732776606e-07
1698 1.69062772670259e-07
1699 1.67942317830239e-07
1700 1.6786688661341e-07
1701 1.67689208296906e-07
1702 1.67870524592217e-07
1703 1.67801132988643e-07
1704 1.67842017617659e-07
1705 1.67627973723938e-07
1706 1.67785643157004e-07
1707 1.67704058640084e-07
1708 1.67795946026672e-07
1709 1.67587259625179e-07
1710 1.67730618727546e-07
1711 1.67700875408627e-07
1712 1.67710282994449e-07
1713 1.67414881957484e-07
1714 1.67702594922048e-07
1715 1.67578875220897e-07
1716 1.67608320111867e-07
1717 1.67418534147146e-07
1718 1.67511103654761e-07
1719 1.67584587984493e-07
1720 1.67582172139191e-07
1721 1.67323861433033e-07
1722 1.67519814908701e-07
1723 1.67445179499737e-07
1724 1.67496921221755e-07
1725 1.67178214383057e-07
1726 1.67680326512709e-07
1727 1.67328579436798e-07
1728 1.67554460972497e-07
1729 1.67165424613813e-07
1730 1.67557203667457e-07
1731 1.67316727583966e-07
1732 1.67481161383876e-07
1733 1.67002966122709e-07
1734 1.67349568869213e-07
1735 1.67272119711015e-07
1736 1.67290053809666e-07
1737 1.66990588468252e-07
1738 1.67298111364289e-07
1739 1.68211670370511e-07
1740 1.67290934882658e-07
1741 1.66889321917552e-07
1742 1.67187593547169e-07
1743 1.67010043128357e-07
1744 1.67218303204208e-07
1745 1.66840038673399e-07
1746 1.67086227520485e-07
1747 1.67014846397251e-07
1748 1.6737791952437e-07
1749 1.66877057949932e-07
1750 1.66938960433072e-07
1751 1.66904953857738e-07
1752 1.67140626672335e-07
1753 1.6659498669469e-07
1754 1.66765047993067e-07
1755 1.66949661206672e-07
1756 1.66886295005497e-07
1757 1.66694263725731e-07
1758 1.68056388361038e-07
1759 1.66807097912169e-07
1760 1.66822175629022e-07
1761 1.66670815815451e-07
1762 1.66763300057937e-07
1763 1.68017152191169e-07
1764 1.66701269677105e-07
1765 1.66503710374855e-07
1766 1.66775848242651e-07
1767 1.66502729825879e-07
1768 1.66601878959227e-07
1769 1.66400781154152e-07
1770 1.66812696988927e-07
1771 1.66452323924204e-07
1772 1.66694633207953e-07
1773 1.66283911084975e-07
1774 1.6661171287069e-07
1775 1.66387138733626e-07
1776 1.66613418173256e-07
1777 1.66249236599469e-07
1778 1.6666059821091e-07
1779 1.66300523574137e-07
1780 1.66305241577902e-07
1781 1.66288941727544e-07
1782 1.66365254017364e-07
1783 1.66213482089006e-07
1784 1.66333165907417e-07
1785 1.66012469549059e-07
1786 1.66280457847279e-07
1787 1.66487851060992e-07
1788 1.6627039656214e-07
1789 1.66028996773093e-07
1790 1.6626316323709e-07
1791 1.66169968451868e-07
1792 1.67271778650502e-07
1793 1.65918038419477e-07
1794 1.66119448863356e-07
1795 1.66175112781275e-07
1796 1.6618386666778e-07
1797 1.6577166661591e-07
1798 1.66075324159465e-07
1799 1.65962873666103e-07
1800 1.67543717566332e-07
1801 1.65766920190435e-07
1802 1.65895400527916e-07
1803 1.65931908213679e-07
1804 1.65880791769268e-07
1805 1.65693279541301e-07
1806 1.6590493601143e-07
1807 1.65837320764695e-07
1808 1.66052529948502e-07
1809 1.6561799043302e-07
1810 1.65948719654807e-07
1811 1.65749625580247e-07
1812 1.65690991593692e-07
1813 1.65631206527905e-07
1814 1.65674478580513e-07
1815 1.65661347750756e-07
1816 1.65896238968344e-07
1817 1.65438436283694e-07
1818 1.656109986925e-07
1819 1.65503294624614e-07
1820 1.65526330420107e-07
1821 1.65329041124096e-07
1822 1.65578995847682e-07
1823 1.65401246476904e-07
1824 1.65679168162569e-07
1825 1.65396500051429e-07
1826 1.65607730195916e-07
1827 1.65353696957027e-07
1828 1.6538622560347e-07
1829 1.6524701607068e-07
1830 1.65423045928037e-07
1831 1.64422274906428e-07
1832 1.65378068572863e-07
1833 1.65199622870205e-07
1834 1.65325772627511e-07
1835 1.65182441946854e-07
1836 1.6532383995127e-07
1837 1.65179585565056e-07
1838 1.64743170216752e-07
1839 1.65137620911082e-07
1840 1.65122074236024e-07
1841 1.65027756793279e-07
1842 1.65064932389214e-07
1843 1.65244344429993e-07
1844 1.65106200711307e-07
1845 1.64854569106865e-07
1846 1.65133229756975e-07
1847 1.64932586699251e-07
1848 1.65052043143987e-07
1849 1.64760834309163e-07
1850 1.64899887522552e-07
1851 1.64225184562383e-07
1852 1.64823816817261e-07
1853 1.64727481433147e-07
1854 1.64850177952758e-07
1855 1.64889911502542e-07
1856 1.64756812637279e-07
1857 1.64688501058663e-07
1858 1.64740683317177e-07
1859 1.64842916205998e-07
1860 1.64684308856522e-07
1861 1.64528429991151e-07
1862 1.6464225893742e-07
1863 1.64679107683696e-07
1864 1.64652149692301e-07
1865 1.6444916184355e-07
1866 1.64574686323249e-07
1867 1.64589721407538e-07
1868 1.64529325274998e-07
1869 1.6565240912314e-07
1870 1.64478464625972e-07
1871 1.64645442168876e-07
1872 1.64409172498381e-07
1873 1.64299294169723e-07
1874 1.64400688618116e-07
1875 1.64630733934246e-07
1876 1.64321562579062e-07
1877 1.64366966259877e-07
1878 1.64215961717673e-07
1879 1.6438553984699e-07
1880 1.64397974344865e-07
1881 1.64023290949444e-07
1882 1.64231408916748e-07
1883 1.64251346745914e-07
1884 1.64229447818798e-07
1885 1.64203186159284e-07
1886 1.64069604124961e-07
1887 1.64387557788359e-07
1888 1.63966504374002e-07
1889 1.64304864824771e-07
1890 1.63921086482333e-07
1891 1.64151103376753e-07
1892 1.6402273672611e-07
1893 1.63993504997961e-07
1894 1.63888273618795e-07
1895 1.63907202477276e-07
1896 1.63938693731325e-07
1897 1.64085264486857e-07
1898 1.63971975553068e-07
1899 1.63837214017803e-07
1900 1.63848937972944e-07
1901 1.63721523449567e-07
1902 1.63883441928192e-07
1903 1.63721182389054e-07
1904 1.63014192366973e-07
1905 1.6377849476612e-07
1906 1.63635448302557e-07
1907 1.63656793006339e-07
1908 1.65147312713998e-07
1909 1.64335403951554e-07
1910 1.63717544410247e-07
1911 1.6355173215743e-07
1912 1.63578050660362e-07
1913 1.63639128913928e-07
1914 1.63561296062653e-07
1915 1.63546545195459e-07
1916 1.6348585063497e-07
1917 1.63565019306589e-07
1918 1.63390112106754e-07
1919 1.64808184877074e-07
1920 1.63359146654329e-07
1921 1.63449854539977e-07
1922 1.63294089361443e-07
1923 1.63496352456605e-07
1924 1.63441157496891e-07
1925 1.63094981076028e-07
1926 1.63316613566167e-07
1927 1.63151597121214e-07
1928 1.63252124707469e-07
1929 1.63261901775513e-07
1930 1.63055673851886e-07
1931 1.63240599704295e-07
1932 1.63385266205296e-07
1933 1.63206323122722e-07
1934 1.63134401987008e-07
1935 1.63000379416189e-07
1936 1.64263084911909e-07
1937 1.62951479865114e-07
1938 1.63046649959142e-07
1939 1.63147760190441e-07
1940 1.62917928037132e-07
1941 1.63058842872488e-07
1942 1.62838304618163e-07
1943 1.62896711231042e-07
1944 1.62888966315222e-07
1945 1.62898217581642e-07
1946 1.6284541004552e-07
1947 1.62888341037615e-07
1948 1.6408964143011e-07
1949 1.62758183819278e-07
1950 1.62784516533065e-07
1951 1.62720681373685e-07
1952 1.62770916745103e-07
1953 1.62138263704037e-07
1954 1.62568724704215e-07
1955 1.6262214330709e-07
1956 1.62595725328174e-07
1957 1.63284425980237e-07
1958 1.62512492352107e-07
1959 1.62791323532474e-07
1960 1.62545305215644e-07
1961 1.62506054834921e-07
1962 1.62718393426076e-07
1963 1.62538725589911e-07
1964 1.62741869758065e-07
1965 1.62544282034105e-07
1966 1.62361260436228e-07
1967 1.62388346325315e-07
1968 1.62352606025706e-07
1969 1.624372742981e-07
1970 1.62694902883231e-07
1971 1.62372245426923e-07
1972 1.62441267548274e-07
1973 1.62228104727546e-07
1974 1.62385703106338e-07
1975 1.62338025688769e-07
1976 1.62201601483503e-07
1977 1.62302328021724e-07
1978 1.62312929319341e-07
1979 1.63489119131555e-07
1980 1.62235551215417e-07
1981 1.62133773073947e-07
1982 1.62201843068033e-07
1983 1.62162308470215e-07
1984 1.62159835781495e-07
1985 1.62045353135909e-07
1986 1.62169783379795e-07
1987 1.62661464742087e-07
1988 1.6193700957956e-07
1989 1.62098956479895e-07
1990 1.62095616929037e-07
1991 1.61803583864639e-07
1992 1.62322194796616e-07
1993 1.61973602530452e-07
1994 1.61886589467031e-07
1995 1.61980281632168e-07
1996 1.6187067330975e-07
1997 1.61874808668472e-07
1998 1.62221496680104e-07
1999 1.61939865961358e-07
};
\addlegendentry{Test}

\nextgroupplot[
title={2 Layers $\rare$},
ymin=2.00962005103304e-06, ymax=1e-04,
]
\addplot [semithick, black, dashed]
table {%
0 0.0632710078060627
1 0.051845450758934
2 0.0432980991452932
3 0.0358776403516531
4 0.0290690961554647
5 0.0229545622691512
6 0.0177181212641299
7 0.0134029685482383
8 0.00932001467421651
9 0.00582119612768292
10 0.00408269998338073
11 0.00310812011640519
12 0.00251322026690468
13 0.00211413772078231
14 0.00183244168199599
15 0.00161229025945067
16 0.00143017171323299
17 0.00127473987545818
18 0.0011382556988392
19 0.00101599448849447
20 0.000905665848404169
21 0.000806275674840435
22 0.00071686326363124
23 0.000636678095906973
24 0.000565266070072539
25 0.000501702402951196
26 0.000445806336938404
27 0.000397627679049037
28 0.000357118795858696
29 0.00032489347649971
30 0.000299998327216599
31 0.000281208114116453
32 0.000267276446160395
33 0.000257046025944874
34 0.0002495395517617
35 0.000243972511438187
36 0.000239745286788093
37 0.000236407682736171
38 0.000233631588111166
39 0.000231188191857655
40 0.000228926038951613
41 0.000226746779982932
42 0.000224590560450451
43 0.000222427553118905
44 0.000220237786095822
45 0.000218012332130456
46 0.000215740327854292
47 0.000213416477650753
48 0.000211034586711321
49 0.000208587526110932
50 0.000201116997253848
51 0.000158018265356077
52 0.000147613224027737
53 0.000142723495948303
54 0.000139244368234358
55 0.000136019374200259
56 0.000132747634241241
57 0.000129427922991454
58 0.000126070258433174
59 0.000122665663249791
60 0.000119213276564551
61 0.000115701421811536
62 0.000112120777172095
63 0.000108473681488249
64 0.000104762014772859
65 0.00010098851814837
66 9.71622276556445e-05
67 9.32928094116505e-05
68 8.93886299527367e-05
69 8.54555976766278e-05
70 8.14965877798386e-05
71 7.75122336417553e-05
72 7.35026992588246e-05
73 6.94685789057985e-05
74 6.54132343734091e-05
75 6.13432748941705e-05
76 5.72745405916066e-05
77 5.32301725179423e-05
78 4.92395857872907e-05
79 4.53403164574411e-05
80 4.1573930386221e-05
81 3.79849034761719e-05
82 3.46141511963651e-05
83 3.14956762558722e-05
84 2.8658379233093e-05
85 2.61181172882061e-05
86 2.38809501188371e-05
87 2.19437158739311e-05
88 2.0293696959925e-05
89 1.89088841780176e-05
90 1.77624986854426e-05
91 1.6823803317493e-05
92 1.60607825182524e-05
93 1.50258497696996e-05
94 1.4460581161984e-05
95 1.40592445336551e-05
96 1.37072536540472e-05
97 1.34257343074751e-05
98 1.31816849047937e-05
99 1.29646241002774e-05
100 1.27662120476089e-05
101 1.2581771216901e-05
102 1.24078030003147e-05
103 1.22411715728958e-05
104 1.20805520891736e-05
105 1.19247015427391e-05
106 1.1773092596286e-05
107 1.16248549329612e-05
108 1.14799404746009e-05
109 1.13379276499472e-05
110 1.11984094673971e-05
111 1.10615233461431e-05
112 1.09273854504863e-05
113 1.07954060031261e-05
114 1.06658476615848e-05
115 1.05386192249171e-05
116 1.04137884209194e-05
117 1.02908117264633e-05
118 1.01702188394484e-05
119 1.00515134595298e-05
120 9.93500102867984e-06
121 9.82061678314494e-06
122 9.70835857060592e-06
123 9.59800334567262e-06
124 9.48999531192385e-06
125 9.38343026655275e-06
126 9.2788389024463e-06
127 9.17616092829121e-06
128 9.07542354798352e-06
129 8.97668589686873e-06
130 8.87972992904906e-06
131 8.78457191856796e-06
132 8.69125409280969e-06
133 8.59979536244282e-06
134 8.51017152263012e-06
135 8.42218346087975e-06
136 8.33635802018762e-06
137 8.25191251988144e-06
138 8.16942619667316e-06
139 8.0887000030998e-06
140 8.00948176038219e-06
141 7.93231499164904e-06
142 7.85655636991578e-06
143 7.78257825504625e-06
144 7.71019455214628e-06
145 7.63938447198598e-06
146 7.57029341457383e-06
147 7.50287921164272e-06
148 7.43711638529021e-06
149 7.37279744384978e-06
150 7.31026423613912e-06
151 7.24875530431746e-06
152 7.18897826664033e-06
153 7.13090271165129e-06
154 7.07441331041991e-06
155 7.01886005163033e-06
156 6.96489755046059e-06
157 6.91240939818272e-06
158 6.86140954894654e-06
159 6.81176477974077e-06
160 6.76338529569875e-06
161 6.71635143635285e-06
162 6.67062341608471e-06
163 6.62643124655915e-06
164 6.58323398965877e-06
165 6.54126916742825e-06
166 6.50058312885449e-06
167 6.46115243034728e-06
168 6.42298308639511e-06
169 6.38602485582851e-06
170 6.35025759902419e-06
171 6.31564215518665e-06
172 6.2821410494962e-06
173 6.24969452883306e-06
174 6.21805136665898e-06
175 6.18742654864946e-06
176 6.15822291456425e-06
177 6.12961435763282e-06
178 6.10204494751088e-06
179 6.07546555693261e-06
180 6.04976868044105e-06
181 6.02458717798982e-06
182 6.00032682541496e-06
183 5.97695071451199e-06
184 5.95447526166026e-06
185 5.93280903376581e-06
186 5.91193096875031e-06
187 5.89195336056036e-06
188 5.87234221757171e-06
189 5.8534633653835e-06
190 5.83506993143601e-06
191 5.81726929294746e-06
192 5.79998251942015e-06
193 5.78317128065464e-06
194 5.76691470541846e-06
195 5.75105928714947e-06
196 5.73594326829152e-06
197 5.72154736437369e-06
198 5.70739826844147e-06
199 5.69361865041174e-06
200 5.68028440807211e-06
201 5.66706610925394e-06
202 5.65437040722827e-06
203 5.64182836365035e-06
204 5.62923450252129e-06
205 5.61731328298265e-06
206 5.60552214392374e-06
207 5.59428438737086e-06
208 5.58346819707367e-06
209 5.57276731524325e-06
210 5.56239524667035e-06
211 5.55222034631697e-06
212 5.54241197096417e-06
213 5.53287029742933e-06
214 5.52343979006764e-06
215 5.51419333123704e-06
216 5.5052181321571e-06
217 5.49665210428429e-06
218 5.48808294684022e-06
219 5.47977340056605e-06
220 5.47153629327113e-06
221 5.46346909595741e-06
222 5.45539895256297e-06
223 5.44765622976229e-06
224 5.44031825961611e-06
225 5.4328157098098e-06
226 5.42541469417301e-06
227 5.4179627427402e-06
228 5.41054173163502e-06
229 5.40329494447178e-06
230 5.39551419979034e-06
231 5.38808617102404e-06
232 5.38088338089437e-06
233 5.3739369534469e-06
234 5.36722018341607e-06
235 5.3601379581778e-06
236 5.35322880318745e-06
237 5.34662611130443e-06
238 5.33928093022951e-06
239 5.3319608723541e-06
240 5.32476024727657e-06
241 5.31745174248499e-06
242 5.3104168794107e-06
243 5.30307639201055e-06
244 5.29598574553347e-06
245 5.28916283337821e-06
246 5.28227127779246e-06
247 5.27531217994692e-06
248 5.26836660651497e-06
249 5.26145600565542e-06
250 5.25481974682407e-06
251 5.24836719614541e-06
252 5.24147814007847e-06
253 5.23453493815396e-06
254 5.22773011539357e-06
255 5.22081494614213e-06
256 5.21390236099251e-06
257 5.20662491862822e-06
258 5.19909029208066e-06
259 5.19156840505275e-06
260 5.1840717428604e-06
261 5.17612017256397e-06
262 5.16830087531162e-06
263 5.1598733487026e-06
264 5.15117472787097e-06
265 5.14193537469509e-06
266 5.13207956760198e-06
267 5.12154601028669e-06
268 5.11089195560999e-06
269 5.10023479159827e-06
270 5.08917057959479e-06
271 5.07760456798678e-06
272 5.06578773297406e-06
273 5.05217641853051e-06
274 5.0364313431146e-06
275 5.02120112696503e-06
276 5.00727311009541e-06
277 4.99260774995491e-06
278 4.97757784273745e-06
279 4.96154125721659e-06
280 4.94340508657842e-06
281 4.92293700722257e-06
282 4.89921165763008e-06
283 4.87050170272596e-06
284 4.83581983303338e-06
285 4.79348267253954e-06
286 4.74225533139361e-06
287 4.67328239142262e-06
288 4.58696049599894e-06
289 4.46936619255212e-06
290 4.31748006303678e-06
291 4.23798107794937e-06
292 4.22272348396291e-06
293 4.2174021655228e-06
294 4.21430222650088e-06
295 4.21172886331078e-06
296 4.20933076134133e-06
297 4.20704505904723e-06
298 4.20480583579774e-06
299 4.20261384238074e-06
300 4.20045522423607e-06
301 4.19840986728559e-06
302 4.19635286948505e-06
303 4.19434635909965e-06
304 4.1923705946374e-06
305 4.19050039226931e-06
306 4.18860404033694e-06
307 4.18674072125214e-06
308 4.18486637317983e-06
309 4.18306126630341e-06
310 4.18127912712407e-06
311 4.17950906694386e-06
312 4.17775834330314e-06
313 4.17607370832229e-06
314 4.17431320079231e-06
315 4.1727198783974e-06
316 4.17104828943593e-06
317 4.16937491081626e-06
318 4.16773654842473e-06
319 4.16610417369156e-06
320 4.16449309932432e-06
321 4.16296918592707e-06
322 4.16134395891277e-06
323 4.15973293388561e-06
324 4.15812554933837e-06
325 4.15651645334947e-06
326 4.15494718413356e-06
327 4.15333152818675e-06
328 4.15174744648539e-06
329 4.15025573261119e-06
330 4.14862696129603e-06
331 4.14689902572718e-06
332 4.14517258332125e-06
333 4.14342047315586e-06
334 4.14178949631605e-06
335 4.1400407767469e-06
336 4.13811117368823e-06
337 4.13610095756667e-06
338 4.13402996832701e-06
339 4.1318420285279e-06
340 4.1294807142549e-06
341 4.12701435834606e-06
342 4.12451515353496e-06
343 4.12194325031123e-06
344 4.11919797215887e-06
345 4.11631795668654e-06
346 4.11326574703708e-06
347 4.11016632801875e-06
348 4.10674409454259e-06
349 4.10305109289766e-06
350 4.09920685819998e-06
351 4.09521432902693e-06
352 4.0910061447903e-06
353 4.08667012720798e-06
354 4.08225326873435e-06
355 4.07795396699839e-06
356 4.07355116385588e-06
357 4.0691161605082e-06
358 4.06490110549385e-06
359 4.06080552897947e-06
360 4.05672974352456e-06
361 4.05283861550743e-06
362 4.04928707007457e-06
363 4.04601418176753e-06
364 4.0429513314848e-06
365 4.04010962870416e-06
366 4.03744269556228e-06
367 4.03506154293609e-06
368 4.03286837035921e-06
369 4.03083124660952e-06
370 4.02896440937184e-06
371 4.02703742270205e-06
372 4.02513960216311e-06
373 4.02338820936166e-06
374 4.02179212824194e-06
375 4.02032433521526e-06
376 4.01882592655056e-06
377 4.01735797845504e-06
378 4.01590311412292e-06
379 4.01456900135599e-06
380 4.01327008967201e-06
381 4.01198213216958e-06
382 4.01074603587404e-06
383 4.0095192375702e-06
384 4.00830898684035e-06
385 4.00709463201565e-06
386 4.0059097862013e-06
387 4.00470182489698e-06
388 4.00351677558319e-06
389 4.00236419886824e-06
390 4.00116998844169e-06
391 3.99998082389175e-06
392 3.99884801413464e-06
393 3.99764587837126e-06
394 3.99651417865243e-06
395 3.99532578603612e-06
396 3.99419785026112e-06
397 3.99303706149112e-06
398 3.99190068537791e-06
399 3.99077415227112e-06
400 3.98964895885001e-06
401 3.98854520904024e-06
402 3.98738429157675e-06
403 3.98626458650142e-06
404 3.98513616232776e-06
405 3.98401351844768e-06
406 3.98287694929422e-06
407 3.98177175156889e-06
408 3.98065694753313e-06
409 3.97955125094995e-06
410 3.97847688532238e-06
411 3.97736476134014e-06
412 3.97623916978773e-06
413 3.97515318627484e-06
414 3.97404352497688e-06
415 3.97295079324067e-06
416 3.97184997132172e-06
417 3.97074828265431e-06
418 3.96966739663185e-06
419 3.96858127305677e-06
420 3.96750045547378e-06
421 3.96643120166118e-06
422 3.96534202013754e-06
423 3.96427063765259e-06
424 3.96322433994101e-06
425 3.96213134354184e-06
426 3.96106135895025e-06
427 3.95999829197535e-06
428 3.95894019357001e-06
429 3.95790915376892e-06
430 3.95684226828053e-06
431 3.95577681206305e-06
432 3.95472414516007e-06
433 3.95362605377159e-06
434 3.95254756790564e-06
435 3.95150437725533e-06
436 3.95043570642883e-06
437 3.94937866462897e-06
438 3.94838115562379e-06
439 3.94730105404051e-06
440 3.94621883810942e-06
441 3.9451766710954e-06
442 3.94410750595853e-06
443 3.94304689325509e-06
444 3.94200809296308e-06
445 3.94095385900073e-06
446 3.93992307658664e-06
447 3.93891124645052e-06
448 3.93786665722473e-06
449 3.93685992435167e-06
450 3.9358440626529e-06
451 3.93480710340555e-06
452 3.93380716809588e-06
453 3.93278114415807e-06
454 3.93176267130002e-06
455 3.93070953191454e-06
456 3.92965003652535e-06
457 3.92862235230496e-06
458 3.92760561658179e-06
459 3.92657984616562e-06
460 3.9255645015146e-06
461 3.92449227001634e-06
462 3.92346870353322e-06
463 3.92245446937522e-06
464 3.92145133218946e-06
465 3.92043095689587e-06
466 3.91943706358688e-06
467 3.91841551595462e-06
468 3.91737812401516e-06
469 3.91636446238408e-06
470 3.91536215056476e-06
471 3.91434734888207e-06
472 3.91334745836502e-06
473 3.91231219873589e-06
474 3.91130924549543e-06
475 3.91030783475799e-06
476 3.90929390709971e-06
477 3.90829968523576e-06
478 3.90728442221189e-06
479 3.9063041681402e-06
480 3.90529386982053e-06
481 3.90429663980285e-06
482 3.90328932166994e-06
483 3.90228188075525e-06
484 3.90130058735849e-06
485 3.90029922914437e-06
486 3.89929259267774e-06
487 3.89830421522674e-06
488 3.89731379686964e-06
489 3.89628200673542e-06
490 3.89534313785589e-06
491 3.89432079941798e-06
492 3.89331514952573e-06
493 3.89229541360692e-06
494 3.89127347989415e-06
495 3.89029763300641e-06
496 3.88925411311902e-06
497 3.88816214581311e-06
498 3.88687702547941e-06
499 3.88587540737717e-06
500 3.88488475755366e-06
501 3.88386549957431e-06
502 3.88286122324644e-06
503 3.88185147426157e-06
504 3.88082735753414e-06
505 3.87983534415071e-06
506 3.87884211681921e-06
507 3.87786378541932e-06
508 3.87690347974967e-06
509 3.87590351806466e-06
510 3.87493697735408e-06
511 3.87393002165481e-06
512 3.87293140079237e-06
513 3.87193410711006e-06
514 3.87096493568606e-06
515 3.86997004397927e-06
516 3.86899167233423e-06
517 3.86810477471045e-06
518 3.86708819746673e-06
519 3.86609972861152e-06
520 3.86511584406435e-06
521 3.86410707824325e-06
522 3.86309811415231e-06
523 3.86210161127565e-06
524 3.86109321630101e-06
525 3.86009254270903e-06
526 3.85912106071373e-06
527 3.85811027194904e-06
528 3.85712550837525e-06
529 3.85611493993565e-06
530 3.85510660271393e-06
531 3.85410345074888e-06
532 3.85311390755305e-06
533 3.85213474737611e-06
534 3.85112630192452e-06
535 3.85012735591772e-06
536 3.84913734296788e-06
537 3.84815157872254e-06
538 3.84717732458739e-06
539 3.84619277247111e-06
540 3.84522993113023e-06
541 3.8442311797553e-06
542 3.84322497643552e-06
543 3.84227112340341e-06
544 3.8412537937802e-06
545 3.84022483763147e-06
546 3.83922470746256e-06
547 3.83821959007946e-06
548 3.83719782143999e-06
549 3.83618380419648e-06
550 3.83516073952705e-06
551 3.83415338933446e-06
552 3.83312635744915e-06
553 3.83206138189962e-06
554 3.83102651903755e-06
555 3.82998564509762e-06
556 3.82895923803517e-06
557 3.82797941483659e-06
558 3.82694782160797e-06
559 3.82592881351229e-06
560 3.82489514527151e-06
561 3.82387169111098e-06
562 3.82288159403288e-06
563 3.82187921854893e-06
564 3.82090538437296e-06
565 3.81990201117333e-06
566 3.81885945239446e-06
567 3.8178061686267e-06
568 3.81674583468339e-06
569 3.81568310513103e-06
570 3.81460992866778e-06
571 3.81355367085234e-06
572 3.81251289786633e-06
573 3.81145554933937e-06
574 3.8104192146875e-06
575 3.8093715679679e-06
576 3.80830061658344e-06
577 3.80721733722567e-06
578 3.80615091921754e-06
579 3.80504406757609e-06
580 3.80396494574597e-06
581 3.80284883271997e-06
582 3.80177165607165e-06
583 3.80069691914287e-06
584 3.79964723424564e-06
585 3.7985892818142e-06
586 3.79753483116474e-06
587 3.79640749065402e-06
588 3.79530364398306e-06
589 3.7941353525639e-06
590 3.79299306950998e-06
591 3.79188061958757e-06
592 3.79073451824752e-06
593 3.78960781699789e-06
594 3.78849956632621e-06
595 3.7874091502772e-06
596 3.78626679844274e-06
597 3.78513176633533e-06
598 3.78400663305456e-06
599 3.78287078638095e-06
600 3.78169056853039e-06
601 3.78051059510653e-06
602 3.77938837505099e-06
603 3.77821030451742e-06
604 3.77709365989176e-06
605 3.77594412566395e-06
606 3.77473977471254e-06
607 3.77356035733101e-06
608 3.77238059650153e-06
609 3.77121746578268e-06
610 3.77010574015912e-06
611 3.76895438330394e-06
612 3.76782894613825e-06
613 3.76667858108704e-06
614 3.76551559270411e-06
615 3.76439185151867e-06
616 3.76326227546997e-06
617 3.76208450234117e-06
618 3.76090089821446e-06
619 3.75969973163137e-06
620 3.75851625665291e-06
621 3.75739111655093e-06
622 3.75627195273864e-06
623 3.75508487604748e-06
624 3.75392496414406e-06
625 3.75275827741461e-06
626 3.75164966681041e-06
627 3.7504619098172e-06
628 3.7492805902275e-06
629 3.74812237578226e-06
630 3.74700240786297e-06
631 3.74586033490232e-06
632 3.74465033678462e-06
633 3.74347466811287e-06
634 3.74222576670036e-06
635 3.74095193990343e-06
636 3.73970642544919e-06
637 3.73849202571819e-06
638 3.73727273176883e-06
639 3.73604903143132e-06
640 3.73481999326941e-06
641 3.73360565822622e-06
642 3.73240932685803e-06
643 3.73127844181909e-06
644 3.73011926103572e-06
645 3.72891290930966e-06
646 3.727741849616e-06
647 3.72646188213821e-06
648 3.72512999217633e-06
649 3.72386121966883e-06
650 3.72253252623977e-06
651 3.72115426330311e-06
652 3.71982453145847e-06
653 3.7184918338653e-06
654 3.71706787746007e-06
655 3.71560938947368e-06
656 3.71417414066855e-06
657 3.71250464706918e-06
658 3.71070990593125e-06
659 3.70882397623973e-06
660 3.70690597890189e-06
661 3.70420897252188e-06
662 3.70078934975027e-06
663 3.69826730502609e-06
664 3.69612601537028e-06
665 3.69425015776415e-06
666 3.69259916294595e-06
667 3.69112746614064e-06
668 3.68980126438601e-06
669 3.6885883330342e-06
670 3.68746814126553e-06
671 3.68644335560475e-06
672 3.68547208643122e-06
673 3.68455293175884e-06
674 3.683675603952e-06
675 3.68285116735478e-06
676 3.6820184672024e-06
677 3.68124188753427e-06
678 3.68046135884015e-06
679 3.67971530249633e-06
680 3.67896949774149e-06
681 3.67823716908333e-06
682 3.67751394207971e-06
683 3.67678034842811e-06
684 3.67607592556851e-06
685 3.67535497991867e-06
686 3.67465632757558e-06
687 3.67394465786219e-06
688 3.6732546661824e-06
689 3.67254644345394e-06
690 3.67185343293386e-06
691 3.67115930930595e-06
692 3.67046825454054e-06
693 3.66977536043578e-06
694 3.66907607531175e-06
695 3.66839212199466e-06
696 3.66770006837669e-06
697 3.6670071425533e-06
698 3.66632715258675e-06
699 3.66563417685484e-06
700 3.66494473416878e-06
701 3.66426724519897e-06
702 3.66358821713675e-06
703 3.66290142187609e-06
704 3.6622132258799e-06
705 3.6615301440861e-06
706 3.66084534357469e-06
707 3.66017243891292e-06
708 3.6594921489268e-06
709 3.65881293635084e-06
710 3.65812857126002e-06
711 3.6574575931354e-06
712 3.65677416755261e-06
713 3.65609544701329e-06
714 3.65541832900362e-06
715 3.65474015052314e-06
716 3.65406903290477e-06
717 3.65339285406208e-06
718 3.65271773671338e-06
719 3.65204745116898e-06
720 3.651368467672e-06
721 3.65069879228486e-06
722 3.65002570674733e-06
723 3.64936534811022e-06
724 3.64868461804235e-06
725 3.64801084799637e-06
726 3.64734410402434e-06
727 3.64667091173487e-06
728 3.6460121403934e-06
729 3.6453348136547e-06
730 3.64466524786167e-06
731 3.6439993359636e-06
732 3.64334011158007e-06
733 3.64267599638879e-06
734 3.64200718195207e-06
735 3.64133908624353e-06
736 3.64068387693806e-06
737 3.64001761658983e-06
738 3.63935759855849e-06
739 3.63869932277794e-06
740 3.6380411635264e-06
741 3.63737760744698e-06
742 3.63650741667243e-06
743 3.63583347075291e-06
744 3.63516651168538e-06
745 3.63449705389485e-06
746 3.63384109414255e-06
747 3.63316582570405e-06
748 3.63250598411469e-06
749 3.63184671437011e-06
750 3.63118088159808e-06
751 3.63051347460441e-06
752 3.62984826938373e-06
753 3.62920208863216e-06
754 3.62853011097286e-06
755 3.62786857806441e-06
756 3.62720777866343e-06
757 3.62655042818005e-06
758 3.62588549194243e-06
759 3.62523838361994e-06
760 3.62457825156071e-06
761 3.62390604811935e-06
762 3.62325307582978e-06
763 3.62259630355766e-06
764 3.62194640536018e-06
765 3.62129782820375e-06
766 3.62063571492399e-06
767 3.61997041034101e-06
768 3.61932698899636e-06
769 3.61866497462415e-06
770 3.6180262252401e-06
771 3.617362689738e-06
772 3.61671298037436e-06
773 3.61607361901406e-06
774 3.61541275435684e-06
775 3.6147658632899e-06
776 3.61411101857811e-06
777 3.61345795158741e-06
778 3.61281008554215e-06
779 3.61216918668106e-06
780 3.6115153557148e-06
781 3.61088205590931e-06
782 3.61022876393236e-06
783 3.6095767355846e-06
784 3.60893789741112e-06
785 3.60828682255487e-06
786 3.60765805169194e-06
787 3.60699846305579e-06
788 3.60636278833226e-06
789 3.60572352553845e-06
790 3.60508044082053e-06
791 3.60444044986252e-06
792 3.60379261326216e-06
793 3.60315890213769e-06
794 3.60252107452652e-06
795 3.60187887156371e-06
796 3.6012473944993e-06
797 3.60060167395204e-06
798 3.59996799465989e-06
799 3.59931844286621e-06
800 3.59868616624226e-06
801 3.59804496190463e-06
802 3.59741519469026e-06
803 3.59679051234707e-06
804 3.59614438116296e-06
805 3.59551126325641e-06
806 3.59488375943329e-06
807 3.59423938243708e-06
808 3.59361888456533e-06
809 3.58856573075173e-06
810 3.46819658852837e-06
811 3.45905467918328e-06
812 3.45808383144686e-06
813 3.45728719207727e-06
814 3.45646615710393e-06
815 3.4557891899567e-06
816 3.45512566559592e-06
817 3.45442840398391e-06
818 3.45375150789096e-06
819 3.45314441744904e-06
820 3.45249377733126e-06
821 3.45187570439975e-06
822 3.4512920127554e-06
823 3.45072540699221e-06
824 3.45013294111141e-06
825 3.44956170420119e-06
826 3.44901385676621e-06
827 3.44852854482269e-06
828 3.44790018584717e-06
829 3.44723031332705e-06
830 3.44663635894449e-06
831 3.44601887513818e-06
832 3.44534294094956e-06
833 3.44472433505416e-06
834 3.44406132182939e-06
835 3.44345207020069e-06
836 3.44280726289981e-06
837 3.4422474728899e-06
838 3.44163713850776e-06
839 3.44108118099484e-06
840 3.44047824160043e-06
841 3.4398152462245e-06
842 3.43910582000717e-06
843 3.43842116046744e-06
844 3.43774710404432e-06
845 3.43702792099521e-06
846 3.43634997943809e-06
847 3.43563018009263e-06
848 3.43499882342257e-06
849 3.43428135010981e-06
850 3.43356308815146e-06
851 3.43284928339926e-06
852 3.43210946982708e-06
853 3.43137287131867e-06
854 3.43064236994906e-06
855 3.42992232879169e-06
856 3.42917290743117e-06
857 3.42838393817146e-06
858 3.42754296923431e-06
859 3.42668330313245e-06
860 3.42588032242475e-06
861 3.42507210848453e-06
862 3.42416718467575e-06
863 3.42335599020771e-06
864 3.42258317823507e-06
865 3.42167720782527e-06
866 3.42081106327896e-06
867 3.41993027393528e-06
868 3.41898223450698e-06
869 3.41806178505522e-06
870 3.41711957901225e-06
871 3.41628901833246e-06
872 3.41533120615622e-06
873 3.41435020516201e-06
874 3.41339523242823e-06
875 3.41245838376381e-06
876 3.4114890966066e-06
877 3.41038971214402e-06
878 3.40855177046251e-06
879 3.40721698319157e-06
880 3.40610965008636e-06
881 3.40513882656524e-06
882 3.40419041424411e-06
883 3.40332646044317e-06
884 3.40240671414449e-06
885 3.4013559078403e-06
886 3.40033198938272e-06
887 3.39924581226114e-06
888 3.39811470360019e-06
889 3.39708300680286e-06
890 3.39614488370898e-06
891 3.39512193158953e-06
892 3.39409216473996e-06
893 3.39310099377599e-06
894 3.39206634293987e-06
895 3.39110838353918e-06
896 3.39012106735481e-06
897 3.3891239560262e-06
898 3.38814063684367e-06
899 3.38709192863007e-06
900 3.38610237486137e-06
901 3.38508347010702e-06
902 3.38407450976774e-06
903 3.38305720777043e-06
904 3.38207762490583e-06
905 3.38106447293285e-06
906 3.38001549198452e-06
907 3.37896211885891e-06
908 3.37797512941052e-06
909 3.37698621433447e-06
910 3.37599381316522e-06
911 3.37501526473716e-06
912 3.37411181067182e-06
913 3.37317492824241e-06
914 3.37220609208089e-06
915 3.3712260875518e-06
916 3.37031230503726e-06
917 3.36942957324027e-06
918 3.36857478896491e-06
919 3.36759936112685e-06
920 3.36662839538349e-06
921 3.36570018896509e-06
922 3.3648025120101e-06
923 3.36392968029031e-06
924 3.3630231520192e-06
925 3.36211414503396e-06
926 3.36128224489585e-06
927 3.36038605507838e-06
928 3.35971276251712e-06
929 3.35890150881824e-06
930 3.35802607730784e-06
931 3.35722394027016e-06
932 3.35641767048855e-06
933 3.35569475134889e-06
934 3.35492151180006e-06
935 3.35416929988241e-06
936 3.35343535334687e-06
937 3.35275004340474e-06
938 3.3520320862408e-06
939 3.35134114561697e-06
940 3.35064550688458e-06
941 3.34997043398744e-06
942 3.34934603586134e-06
943 3.34871437360107e-06
944 3.34808450747914e-06
945 3.34750664876537e-06
946 3.34689332385096e-06
947 3.34634646662835e-06
948 3.34575365366163e-06
949 3.34518739339273e-06
950 3.34464585841943e-06
951 3.34406109845986e-06
952 3.34353921266484e-06
953 3.34303967986216e-06
954 3.34256621943041e-06
955 3.34205329431825e-06
956 3.34155066923358e-06
957 3.34107601361211e-06
958 3.34061679984643e-06
959 3.34015869543691e-06
960 3.33970654855875e-06
961 3.3393236901702e-06
962 3.33886904593328e-06
963 3.33842913983062e-06
964 3.33814169107427e-06
965 3.33774210423599e-06
966 3.3372913213725e-06
967 3.33686178612425e-06
968 3.33645433522634e-06
969 3.33606849665102e-06
970 3.33571005978683e-06
971 3.33536255254785e-06
972 3.33501844590955e-06
973 3.33469381189389e-06
974 3.33438582435974e-06
975 3.33408202118335e-06
976 3.33378812365481e-06
977 3.33350157063705e-06
978 3.33323775805638e-06
979 3.3330256313775e-06
980 3.33272925752226e-06
981 3.33243786371895e-06
982 3.33216787623769e-06
983 3.3319054509775e-06
984 3.33164250014306e-06
985 3.33139780241254e-06
986 3.33115889634428e-06
987 3.33091831510046e-06
988 3.33069397129293e-06
989 3.33046564878714e-06
990 3.33024445262708e-06
991 3.33003096750417e-06
992 3.32982458940023e-06
993 3.32962686138671e-06
994 3.32943600267299e-06
995 3.32926675400813e-06
996 3.32906775338415e-06
997 3.3288823390194e-06
998 3.32862334119e-06
999 3.3284366764974e-06
1000 3.32826502312855e-06
1001 3.32808903544901e-06
1002 3.3279268395745e-06
1003 3.3277572111956e-06
1004 3.32759605612409e-06
1005 3.32745577202331e-06
1006 3.32727387956311e-06
1007 3.32710411589687e-06
1008 3.32693895040848e-06
1009 3.32677818335014e-06
1010 3.32661705760984e-06
1011 3.32647359664406e-06
1012 3.32631457933985e-06
1013 3.32618127072237e-06
1014 3.32603964534428e-06
1015 3.32589951085538e-06
1016 3.32574876642866e-06
1017 3.32563812094122e-06
1018 3.325481228444e-06
1019 3.32533804532886e-06
1020 3.32519677795062e-06
1021 3.32507743246424e-06
1022 3.32495024065338e-06
1023 3.32481203804491e-06
1024 3.32468024942045e-06
1025 3.32454988460995e-06
1026 3.32442952083056e-06
1027 3.32430381183713e-06
1028 3.32418459424844e-06
1029 3.32406650829853e-06
1030 3.32396275587143e-06
1031 3.32385228330168e-06
1032 3.32373832941357e-06
1033 3.32363166569394e-06
1034 3.32353530177443e-06
1035 3.32343474212848e-06
1036 3.32334030849779e-06
1037 3.32324221994895e-06
1038 3.32314650916032e-06
1039 3.32305175118108e-06
1040 3.32298359035121e-06
1041 3.32284780517966e-06
1042 3.32274852030423e-06
1043 3.3226549801384e-06
1044 3.32256201613745e-06
1045 3.32246874370412e-06
1046 3.3223735683805e-06
1047 3.32228958177438e-06
1048 3.32220481595868e-06
1049 3.32208896418251e-06
1050 3.32200541083694e-06
1051 3.32191331938247e-06
1052 3.32182456236296e-06
1053 3.32175064909279e-06
1054 3.3216608036355e-06
1055 3.32158123887893e-06
1056 3.32150148597066e-06
1057 3.3214288405361e-06
1058 3.32134673897144e-06
1059 3.32127205160759e-06
1060 3.3211996552609e-06
1061 3.32111623322362e-06
1062 3.32105358813806e-06
1063 3.32098555361426e-06
1064 3.320912385675e-06
1065 3.32084120304899e-06
1066 3.32077246423523e-06
1067 3.32069897615384e-06
1068 3.32062761867746e-06
1069 3.32055835360734e-06
1070 3.32049213534447e-06
1071 3.32042758327589e-06
1072 3.32035822054877e-06
1073 3.32029157220859e-06
1074 3.3202237336809e-06
1075 3.32016097570431e-06
1076 3.32010107376846e-06
1077 3.32003031508066e-06
1078 3.31996242221066e-06
1079 3.31990452946229e-06
1080 3.31984272509089e-06
1081 3.3197831430698e-06
1082 3.31972265939839e-06
1083 3.31965873988338e-06
1084 3.31960340872683e-06
1085 3.31954086311725e-06
1086 3.31948666712378e-06
1087 3.31942682998942e-06
1088 3.31936341729033e-06
1089 3.31931191681178e-06
1090 3.31925431248692e-06
1091 3.31908182454299e-06
1092 3.31901709694193e-06
1093 3.31896539614718e-06
1094 3.31891274390728e-06
1095 3.31885061757475e-06
1096 3.31879793600365e-06
1097 3.3187468189908e-06
1098 3.31868890623355e-06
1099 3.31862838606867e-06
1100 3.31856975185474e-06
1101 3.31851502880909e-06
1102 3.31847244035544e-06
1103 3.31840572573583e-06
1104 3.3183519180966e-06
1105 3.31829593142174e-06
1106 3.31823610724769e-06
1107 3.31818831443798e-06
1108 3.31813788932322e-06
1109 3.31809355793666e-06
1110 3.31803202232095e-06
1111 3.31797998080674e-06
1112 3.31792680481158e-06
1113 3.31787022776098e-06
1114 3.31782323041807e-06
1115 3.31776938503481e-06
1116 3.31772539641406e-06
1117 3.31767134389338e-06
1118 3.31762536609403e-06
1119 3.31757380968156e-06
1120 3.31752182080436e-06
1121 3.31747144957717e-06
1122 3.31731329424656e-06
1123 3.31725821206419e-06
1124 3.31720646829581e-06
1125 3.31715368929508e-06
1126 3.31710390935314e-06
1127 3.3170535549516e-06
1128 3.31699749733616e-06
1129 3.31695505599328e-06
1130 3.31690347150015e-06
1131 3.31685236983503e-06
1132 3.31680772876552e-06
1133 3.3167499815363e-06
1134 3.31669872764451e-06
1135 3.31664693214861e-06
1136 3.31659686685271e-06
1137 3.31655298123223e-06
1138 3.31649509212184e-06
1139 3.31645859967011e-06
1140 3.31640772060382e-06
1141 3.31635877830649e-06
1142 3.31630854077503e-06
1143 3.3162615816309e-06
1144 3.31620984627534e-06
1145 3.31616943481095e-06
1146 3.31612526088065e-06
1147 3.31607749308205e-06
1148 3.31602828350697e-06
1149 3.31598170487268e-06
1150 3.31593465273272e-06
1151 3.31589305983471e-06
1152 3.31584171124177e-06
1153 3.31579683393102e-06
1154 3.31575542475093e-06
1155 3.31569906404638e-06
1156 3.31566256102178e-06
1157 3.31562098369886e-06
1158 3.31557101537783e-06
1159 3.31552743352859e-06
1160 3.31547878431593e-06
1161 3.31543595132189e-06
1162 3.31539260980662e-06
1163 3.31535449493003e-06
1164 3.31530274024772e-06
1165 3.31526033858154e-06
1166 3.31521684279323e-06
1167 3.31517224708477e-06
1168 3.31512761704289e-06
1169 3.31509071429537e-06
1170 3.31504686755579e-06
1171 3.31499783703748e-06
1172 3.31494618387751e-06
1173 3.31488194569829e-06
1174 3.31483411434874e-06
1175 3.31477687655024e-06
1176 3.31463223449191e-06
1177 3.3145593424706e-06
1178 3.31451699707941e-06
1179 3.31447909991311e-06
1180 3.31444194944197e-06
1181 3.31440136801575e-06
1182 3.3143566832905e-06
1183 3.31432279710953e-06
1184 3.3142888198654e-06
1185 3.31424507578504e-06
1186 3.31421255020814e-06
1187 3.31417136544587e-06
1188 3.31413100582267e-06
1189 3.31409358943802e-06
1190 3.3140541273724e-06
1191 3.31401387495589e-06
1192 3.31398105663538e-06
1193 3.31394617762726e-06
1194 3.31390259668751e-06
1195 3.31387615858603e-06
1196 3.31384136154611e-06
1197 3.31379895965256e-06
1198 3.31376556687246e-06
1199 3.31373439541949e-06
1200 3.31369409514082e-06
1201 3.31365316878873e-06
1202 3.31363031023102e-06
1203 3.31359488518501e-06
1204 3.31356466006127e-06
1205 3.31353000353829e-06
1206 3.3134903616201e-06
1207 3.31346309042146e-06
1208 3.31342015306291e-06
1209 3.31339084232241e-06
1210 3.3133535714569e-06
1211 3.31332341784218e-06
1212 3.31328962545285e-06
1213 3.31325111255865e-06
1214 3.31321978012511e-06
1215 3.31318277721948e-06
1216 3.31314616198597e-06
1217 3.31311287493463e-06
1218 3.31307958356319e-06
1219 3.31304765325058e-06
1220 3.31300194511641e-06
1221 3.31296959791416e-06
1222 3.31293913814079e-06
1223 3.31290463248024e-06
1224 3.31286963171351e-06
1225 3.31283649222769e-06
1226 3.31280664943279e-06
1227 3.31276422059545e-06
1228 3.31273262804643e-06
1229 3.31270691606278e-06
1230 3.31266566593058e-06
1231 3.31263192094866e-06
1232 3.31260459256555e-06
1233 3.31257304628707e-06
1234 3.31253451588509e-06
1235 3.3125027416645e-06
1236 3.31246954056041e-06
1237 3.31243238213119e-06
1238 3.31239868376088e-06
1239 3.31237523766958e-06
1240 3.31232966425432e-06
1241 3.3123012260603e-06
1242 3.31226731339029e-06
1243 3.31223492764821e-06
1244 3.31220416660472e-06
1245 3.3121651907777e-06
1246 3.31213434481015e-06
1247 3.31210411036409e-06
1248 3.31207334511419e-06
1249 3.31204188125866e-06
1250 3.31200534310483e-06
1251 3.31196890385854e-06
1252 3.31193951285513e-06
1253 3.31191298664635e-06
1254 3.31186976950448e-06
1255 3.31183692810555e-06
1256 3.31180187276914e-06
1257 3.31176736733596e-06
1258 3.31174078667118e-06
1259 3.31170079664389e-06
1260 3.31167498484319e-06
1261 3.31163897067199e-06
1262 3.31160367568373e-06
1263 3.31157391030956e-06
1264 3.31153827073649e-06
1265 3.31150821864412e-06
1266 3.31147878000593e-06
1267 3.31144059236976e-06
1268 3.3114125475322e-06
1269 3.31137845250851e-06
1270 3.31135019507656e-06
1271 3.31132036353665e-06
1272 3.31128353366239e-06
1273 3.31124508727498e-06
1274 3.3112155958861e-06
1275 3.31118322765178e-06
1276 3.31115418396166e-06
1277 3.31111870571021e-06
1278 3.31108915679579e-06
1279 3.31105810823829e-06
1280 3.31101666643008e-06
1281 3.31099420361625e-06
1282 3.31095552917304e-06
1283 3.31091766827285e-06
1284 3.31088738289509e-06
1285 3.31086019934901e-06
1286 3.31082479078759e-06
1287 3.3107894528257e-06
1288 3.31076125996788e-06
1289 3.31073088023004e-06
1290 3.31069415051388e-06
1291 3.31065719831258e-06
1292 3.31063165572232e-06
1293 3.31060190774224e-06
1294 3.3105640161466e-06
1295 3.31053720560703e-06
1296 3.31050339457306e-06
1297 3.31046521137068e-06
1298 3.31043459812008e-06
1299 3.31039810896527e-06
1300 3.31036881345881e-06
1301 3.31033730344643e-06
1302 3.310306435651e-06
1303 3.31026991398176e-06
1304 3.31024301567595e-06
1305 3.31020190742493e-06
1306 3.31017149255786e-06
1307 3.3101332599017e-06
1308 3.31010171169055e-06
1309 3.31005946975438e-06
1310 3.31003425515064e-06
1311 3.30999968309698e-06
1312 3.30996857246646e-06
1313 3.30993229317755e-06
1314 3.30990236466278e-06
1315 3.30987621293843e-06
1316 3.30983615754121e-06
1317 3.30979875263893e-06
1318 3.30977489818451e-06
1319 3.30974435814824e-06
1320 3.30970497532235e-06
1321 3.30967777449587e-06
1322 3.30965184900833e-06
1323 3.30961561678578e-06
1324 3.30957849496372e-06
1325 3.30955011418155e-06
1326 3.30952235151472e-06
1327 3.30948825808264e-06
1328 3.30946376834618e-06
1329 3.30942424557179e-06
1330 3.30939567766109e-06
1331 3.30936111083702e-06
1332 3.30933207965245e-06
1333 3.3093084363145e-06
1334 3.30927065931519e-06
1335 3.30923937360694e-06
1336 3.30921289116759e-06
1337 3.30918081397158e-06
1338 3.30914972573737e-06
1339 3.30911605874462e-06
1340 3.30908646151329e-06
1341 3.30905639475532e-06
1342 3.30902801658795e-06
1343 3.30899296102416e-06
1344 3.30896267894332e-06
1345 3.3089325166884e-06
1346 3.30890046916466e-06
1347 3.30886465042113e-06
1348 3.3088395756522e-06
1349 3.30881382774351e-06
1350 3.30877656608664e-06
1351 3.30874721919372e-06
1352 3.30871065659721e-06
1353 3.30867955688063e-06
1354 3.30865487069332e-06
1355 3.30862191162851e-06
1356 3.30859550501827e-06
1357 3.30855776235239e-06
1358 3.30853303728418e-06
1359 3.30850446073327e-06
1360 3.30846531176121e-06
1361 3.30843728659147e-06
1362 3.30840472588534e-06
1363 3.30837749436341e-06
1364 3.30834483236231e-06
1365 3.30830624568534e-06
1366 3.30828763708269e-06
1367 3.30825550838654e-06
1368 3.30822551416077e-06
1369 3.30818973498026e-06
1370 3.30815789163807e-06
1371 3.30812527829494e-06
1372 3.30809859678993e-06
1373 3.30806258318717e-06
1374 3.30803215661035e-06
1375 3.30800900974282e-06
1376 3.30797151002571e-06
1377 3.30794009107649e-06
1378 3.30791404837782e-06
1379 3.30788173698693e-06
1380 3.30785453468252e-06
1381 3.30781790967194e-06
1382 3.30779392106706e-06
1383 3.30776260773291e-06
1384 3.30773389839578e-06
1385 3.30769640322615e-06
1386 3.30766638410296e-06
1387 3.30763812337409e-06
1388 3.30760337055835e-06
1389 3.30757708707097e-06
1390 3.307545913799e-06
1391 3.30751326794143e-06
1392 3.30749232489325e-06
1393 3.30745411440603e-06
1394 3.30735595673559e-06
1395 3.30729904260352e-06
1396 3.30725218407224e-06
1397 3.30722578951281e-06
1398 3.30719073667751e-06
1399 3.30716875021153e-06
1400 3.30714862934656e-06
1401 3.30711684216567e-06
1402 3.3070819894192e-06
1403 3.3070488652811e-06
1404 3.30701422922175e-06
1405 3.30698709717581e-06
1406 3.30695631168965e-06
1407 3.30692265868038e-06
1408 3.30688969722814e-06
1409 3.30686106201483e-06
1410 3.3068254247155e-06
1411 3.3067949505039e-06
1412 3.30675967359184e-06
1413 3.30672607924498e-06
1414 3.30669463016875e-06
1415 3.30665754358961e-06
1416 3.30661870475524e-06
1417 3.30659057021876e-06
1418 3.30655833431592e-06
1419 3.30653160381189e-06
1420 3.30648229748931e-06
1421 3.3064635277924e-06
1422 3.30642847688978e-06
1423 3.30640005802252e-06
1424 3.30635991497275e-06
1425 3.30633839712391e-06
1426 3.30630923531317e-06
1427 3.30627130324501e-06
1428 3.30624613002328e-06
1429 3.30621113914731e-06
1430 3.30618818611583e-06
1431 3.30615022528491e-06
1432 3.30612555853804e-06
1433 3.3060949335777e-06
1434 3.30606805039224e-06
1435 3.30602659039414e-06
1436 3.30600337179021e-06
1437 3.30597711342762e-06
1438 3.30594837578246e-06
1439 3.30591299587013e-06
1440 3.30588416147748e-06
1441 3.30585510812398e-06
1442 3.30582726405737e-06
1443 3.30579878891513e-06
1444 3.30576666067373e-06
1445 3.30573926805755e-06
1446 3.30570304413413e-06
1447 3.30567544233418e-06
1448 3.30564356863761e-06
1449 3.30561513396788e-06
1450 3.3055900224781e-06
1451 3.30555552284295e-06
1452 3.30552562525099e-06
1453 3.30550001433494e-06
1454 3.30546434372536e-06
1455 3.30543629740987e-06
1456 3.30540926006506e-06
1457 3.30537772595108e-06
1458 3.30534853821973e-06
1459 3.30531810948287e-06
1460 3.30529423752068e-06
1461 3.30526512493634e-06
1462 3.30523205786903e-06
1463 3.30520980810434e-06
1464 3.30517069141933e-06
1465 3.30514524932823e-06
1466 3.3051182268764e-06
1467 3.3050883089345e-06
1468 3.30505295073635e-06
1469 3.30502717281433e-06
1470 3.30499619519742e-06
1471 3.30496268304614e-06
1472 3.3049321046974e-06
1473 3.30491160332258e-06
1474 3.30488403938034e-06
1475 3.30484924450047e-06
1476 3.30482337255944e-06
1477 3.3047829225552e-06
1478 3.30475701593969e-06
1479 3.30473347594307e-06
1480 3.30469857590288e-06
1481 3.30467577111904e-06
1482 3.30463922102808e-06
1483 3.30460926977594e-06
1484 3.3045828433842e-06
1485 3.30454228947019e-06
1486 3.30451344598259e-06
1487 3.30448826628071e-06
1488 3.30446151951946e-06
1489 3.30443765540167e-06
1490 3.30440346078831e-06
1491 3.30437982188414e-06
1492 3.30434341208274e-06
1493 3.30430771680312e-06
1494 3.30429122220721e-06
1495 3.30425914978605e-06
1496 3.30423125637935e-06
1497 3.30419332021847e-06
1498 3.30416777239861e-06
1499 3.30414132929491e-06
1500 3.3041072015294e-06
1501 3.30408697664097e-06
1502 3.30404774513227e-06
1503 3.30401728399465e-06
1504 3.30398470123328e-06
1505 3.30396658409882e-06
1506 3.30393206252211e-06
1507 3.30390698763949e-06
1508 3.3038729142163e-06
1509 3.30384653216242e-06
1510 3.30381857452267e-06
1511 3.3037834704146e-06
1512 3.30376238377994e-06
1513 3.30373555277674e-06
1514 3.30370023016258e-06
1515 3.30367066123927e-06
1516 3.30364827868834e-06
1517 3.30361546969016e-06
1518 3.30358942426301e-06
1519 3.30355498942936e-06
1520 3.30352247999599e-06
1521 3.3034975198234e-06
1522 3.30347213048299e-06
1523 3.3034410593018e-06
1524 3.30340980383426e-06
1525 3.30338359890447e-06
1526 3.30335414560068e-06
1527 3.30332708222159e-06
1528 3.30329681514741e-06
1529 3.30327013000442e-06
1530 3.30323686262091e-06
1531 3.30320873774781e-06
1532 3.30318001522301e-06
1533 3.30315449173213e-06
1534 3.30311911397985e-06
1535 3.30308946342939e-06
1536 3.30306302032568e-06
1537 3.30303495979933e-06
1538 3.30300534017169e-06
1539 3.30297712798711e-06
1540 3.30294749460336e-06
1541 3.30292028195345e-06
1542 3.3028948644187e-06
1543 3.30286502901345e-06
1544 3.30283326866265e-06
1545 3.30280487150958e-06
1546 3.30277500813736e-06
1547 3.30274673478925e-06
1548 3.30271867062493e-06
1549 3.30269409198536e-06
1550 3.30266561161352e-06
1551 3.30262623526778e-06
1552 3.30260750922662e-06
1553 3.3025772891051e-06
1554 3.30255014239356e-06
1555 3.30251469154064e-06
1556 3.30248873808614e-06
1557 3.3024674067974e-06
1558 3.30243384246387e-06
1559 3.30240462881193e-06
1560 3.30237546916123e-06
1561 3.30234038585786e-06
1562 3.30231066175202e-06
1563 3.30228913446717e-06
1564 3.30225435584453e-06
1565 3.3022217786538e-06
1566 3.30219976888202e-06
1567 3.30216723705234e-06
1568 3.30214145628815e-06
1569 3.30211137895731e-06
1570 3.30207850458919e-06
1571 3.30205081854729e-06
1572 3.302026115648e-06
1573 3.30199760367123e-06
1574 3.30196539312055e-06
1575 3.3019390057234e-06
1576 3.30191038460725e-06
1577 3.3018855078808e-06
1578 3.30184813685719e-06
1579 3.30180191804175e-06
1580 3.30177034481949e-06
1581 3.3017495417198e-06
1582 3.30171515747679e-06
1583 3.30168458845037e-06
1584 3.30166584637936e-06
1585 3.30163201942923e-06
1586 3.30159827740317e-06
1587 3.3015747395666e-06
1588 3.30153459117355e-06
1589 3.30150503953064e-06
1590 3.30147473721354e-06
1591 3.3014371099398e-06
1592 3.30141293727593e-06
1593 3.30138024446569e-06
1594 3.30134788134728e-06
1595 3.3013217273492e-06
1596 3.30129438600579e-06
1597 3.30125701191264e-06
1598 3.30122357820528e-06
1599 3.30119717989419e-06
1600 3.30116559575799e-06
1601 3.30114111216062e-06
1602 3.30111100208796e-06
1603 3.30107951037917e-06
1604 3.30104800923436e-06
1605 3.30102942348276e-06
1606 3.30099440964204e-06
1607 3.30095793640339e-06
1608 3.3009348513815e-06
1609 3.30090420584384e-06
1610 3.30087502663901e-06
1611 3.30084139932296e-06
1612 3.30081177537522e-06
1613 3.3007786385042e-06
1614 3.30075007127562e-06
1615 3.30071773169038e-06
1616 3.30068771290826e-06
1617 3.30065788398315e-06
1618 3.30062718342106e-06
1619 3.30059937959959e-06
1620 3.30057129747274e-06
1621 3.30054074106556e-06
1622 3.30050786396896e-06
1623 3.30048373109548e-06
1624 3.30044843246924e-06
1625 3.30041834229178e-06
1626 3.30038527124543e-06
1627 3.30035584568122e-06
1628 3.30032886301979e-06
1629 3.30029377244045e-06
1630 3.30026115011606e-06
1631 3.30023468347918e-06
1632 3.30020599756153e-06
1633 3.30017422174933e-06
1634 3.30014696180569e-06
1635 3.30011202709102e-06
1636 3.30009334368242e-06
1637 3.30004895295133e-06
1638 3.30002450380107e-06
1639 3.29999766279343e-06
1640 3.29997183382602e-06
1641 3.29994155310942e-06
1642 3.29990957368409e-06
1643 3.29988495889211e-06
1644 3.29985402834154e-06
1645 3.29982594382727e-06
1646 3.2997962562149e-06
1647 3.29977297496953e-06
1648 3.29973987481935e-06
1649 3.29971252767791e-06
1650 3.2996849652136e-06
1651 3.29965504352003e-06
1652 3.29962673617956e-06
1653 3.29959587395479e-06
1654 3.29956945233789e-06
1655 3.29953468565236e-06
1656 3.29950606544571e-06
1657 3.29947025568345e-06
1658 3.2994390381873e-06
1659 3.29941945165047e-06
1660 3.29939342486796e-06
1661 3.2993645697843e-06
1662 3.29933284365325e-06
1663 3.29930496661746e-06
1664 3.29927175471312e-06
1665 3.29925617563731e-06
1666 3.29922392040771e-06
1667 3.2991881481621e-06
1668 3.29915968620753e-06
1669 3.29913620771549e-06
1670 3.29909874335499e-06
1671 3.29907201569313e-06
1672 3.29904115744739e-06
1673 3.29901682164291e-06
1674 3.29898408233475e-06
1675 3.29896147468389e-06
1676 3.29893686694049e-06
1677 3.29890761008755e-06
1678 3.29887316672739e-06
1679 3.29885079906944e-06
1680 3.29882769165124e-06
1681 3.29879325147431e-06
1682 3.2987617918252e-06
1683 3.29873484895415e-06
1684 3.29869935274019e-06
1685 3.29867526863836e-06
1686 3.29864835384797e-06
1687 3.29862218973176e-06
1688 3.29859079624839e-06
1689 3.29856409382501e-06
1690 3.29853535163238e-06
1691 3.29850868524773e-06
1692 3.29849142360672e-06
1693 3.29845765497794e-06
1694 3.29842601661312e-06
1695 3.29840140443594e-06
1696 3.29837368212793e-06
1697 3.29834100489279e-06
1698 3.29831385715806e-06
1699 3.29828804797216e-06
1700 3.29825782705484e-06
1701 3.2982254365379e-06
1702 3.29819456817404e-06
1703 3.29817430258572e-06
1704 3.29814702399744e-06
1705 3.29811753783815e-06
1706 3.29809323477548e-06
1707 3.29806502418251e-06
1708 3.29804435227743e-06
1709 3.29801751342984e-06
1710 3.29798369352829e-06
1711 3.29794811750617e-06
1712 3.29791816761826e-06
1713 3.29790253829287e-06
1714 3.29787257282987e-06
1715 3.29784812049638e-06
1716 3.29782061146489e-06
1717 3.29780717106587e-06
1718 3.29777353874761e-06
1719 3.29774257522786e-06
1720 3.29772007876272e-06
1721 3.29768729181978e-06
1722 3.2976648666363e-06
1723 3.2976386385144e-06
1724 3.29761890964164e-06
1725 3.29759118449147e-06
1726 3.29756736516629e-06
1727 3.29754246070024e-06
1728 3.29751938181744e-06
1729 3.29750904552384e-06
1730 3.29748445767564e-06
1731 3.29746175452783e-06
1732 3.29743904183033e-06
1733 3.29742131543753e-06
1734 3.29739825508568e-06
1735 3.29737244214812e-06
1736 3.29733628484519e-06
1737 3.29732496516044e-06
1738 3.29729847612725e-06
1739 3.29727213284059e-06
1740 3.29725181234153e-06
1741 3.29722723733994e-06
1742 3.29720896218078e-06
1743 3.29718488944764e-06
1744 3.2971577176113e-06
1745 3.29714096903899e-06
1746 3.29710067023825e-06
1747 3.29706860566148e-06
1748 3.29704713885803e-06
1749 3.29702468445703e-06
1750 3.29699818473728e-06
1751 3.29697867221057e-06
1752 3.29696266067003e-06
1753 3.29693500282247e-06
1754 3.29693543767462e-06
1755 3.29696023811721e-06
1756 3.29695082882608e-06
1757 3.29694183756146e-06
1758 3.29691081867622e-06
1759 3.29688356964652e-06
1760 3.2968626192087e-06
1761 3.29684106566219e-06
1762 3.29682028177558e-06
1763 3.29678278865231e-06
1764 3.29675827799747e-06
1765 3.29673581200041e-06
1766 3.29671340170989e-06
1767 3.29668349399981e-06
1768 3.29665820697755e-06
1769 3.29662746696613e-06
1770 3.29660763202355e-06
1771 3.29657895122182e-06
1772 3.29654578069949e-06
1773 3.29651613321857e-06
1774 3.29649540708488e-06
1775 3.29645523083855e-06
1776 3.29643979955563e-06
1777 3.29641163148153e-06
1778 3.29638665425591e-06
1779 3.29635794275873e-06
1780 3.29633211504188e-06
1781 3.29631347597115e-06
1782 3.29628267604676e-06
1783 3.29625317192495e-06
1784 3.29622374113114e-06
1785 3.29620427964983e-06
1786 3.29617785951086e-06
1787 3.29614272970957e-06
1788 3.29612598181939e-06
1789 3.29610504172706e-06
1790 3.29608197171183e-06
1791 3.29605362776419e-06
1792 3.29602873000567e-06
1793 3.29599734322983e-06
1794 3.29598115251883e-06
1795 3.29595368327773e-06
1796 3.29592873890761e-06
1797 3.29590644400923e-06
1798 3.29587115538743e-06
1799 3.2958679788635e-06
1800 3.29584227199575e-06
1801 3.29581704295379e-06
1802 3.29578775961181e-06
1803 3.29576640160667e-06
1804 3.29573920828352e-06
1805 3.2957079362177e-06
1806 3.29567828396193e-06
1807 3.29566272012016e-06
1808 3.29562991976218e-06
1809 3.29560432305698e-06
1810 3.29557507825484e-06
1811 3.29555203279597e-06
1812 3.29552923381016e-06
1813 3.29549874481927e-06
1814 3.29547894079951e-06
1815 3.29544714759322e-06
1816 3.29542329404831e-06
1817 3.29540431994246e-06
1818 3.29537054847151e-06
1819 3.29534739057635e-06
1820 3.2953360105239e-06
1821 3.29530197461736e-06
1822 3.29527806525221e-06
1823 3.29525011989062e-06
1824 3.29522909112256e-06
1825 3.29519833030645e-06
1826 3.29517955606207e-06
1827 3.29515108592204e-06
1828 3.29512099710882e-06
1829 3.29510926121657e-06
1830 3.2950778175973e-06
1831 3.29505256740958e-06
1832 3.29502160002448e-06
1833 3.29499873896566e-06
1834 3.29498074222556e-06
1835 3.29495600999508e-06
1836 3.2949267698541e-06
1837 3.29490305136915e-06
1838 3.29487930957839e-06
1839 3.29485872146051e-06
1840 3.29484164024052e-06
1841 3.29480794687242e-06
1842 3.29478759601898e-06
1843 3.29475699925297e-06
1844 3.29473340502773e-06
1845 3.29469976156815e-06
1846 3.29467887820556e-06
1847 3.29465848574273e-06
1848 3.29463162881893e-06
1849 3.29460808848125e-06
1850 3.2945863548548e-06
1851 3.29457040265879e-06
1852 3.29454211725988e-06
1853 3.29452215373749e-06
1854 3.29448639865859e-06
1855 3.29446427531366e-06
1856 3.29445120473792e-06
1857 3.29442717281836e-06
1858 3.29439618451488e-06
1859 3.2943695296126e-06
1860 3.29434485365709e-06
1861 3.29431243210365e-06
1862 3.29428772670326e-06
1863 3.29426313135173e-06
1864 3.29424544531776e-06
1865 3.29421646949868e-06
1866 3.29419708020851e-06
1867 3.2941675605116e-06
1868 3.29414240070491e-06
1869 3.29411239692945e-06
1870 3.2940831030146e-06
1871 3.29406201421989e-06
1872 3.29403377043036e-06
1873 3.29401004967167e-06
1874 3.29398750795917e-06
1875 3.29395657308851e-06
1876 3.29392971116249e-06
1877 3.29390835315735e-06
1878 3.29388622515125e-06
1879 3.29386511020857e-06
1880 3.29383555288132e-06
1881 3.29381095195913e-06
1882 3.29378112849099e-06
1883 3.29376096726719e-06
1884 3.29372909925496e-06
1885 3.293708965316e-06
1886 3.29376281285931e-06
1887 3.29372889268598e-06
1888 3.2937001569735e-06
1889 3.29363641480995e-06
1890 3.29355041151302e-06
1891 3.2935141928192e-06
1892 3.2934838931169e-06
1893 3.29344554131694e-06
1894 3.29340811242673e-06
1895 3.29338331562212e-06
1896 3.29335985122725e-06
1897 3.29332913520375e-06
1898 3.29330271426898e-06
1899 3.29327321014716e-06
1900 3.29324075880777e-06
1901 3.2932196161255e-06
1902 3.29319021716401e-06
1903 3.29317063017243e-06
1904 3.29314290615912e-06
1905 3.29311947382394e-06
1906 3.2930885067799e-06
1907 3.2930643766349e-06
1908 3.2930345856812e-06
1909 3.29301663055048e-06
1910 3.29298807730538e-06
1911 3.29296217137198e-06
1912 3.29293132870134e-06
1913 3.29291240620933e-06
1914 3.29288583827747e-06
1915 3.29286125383987e-06
1916 3.29283842313544e-06
1917 3.29280995231329e-06
1918 3.29278601918759e-06
1919 3.29276636114173e-06
1920 3.29274506702859e-06
1921 3.29271799364506e-06
1922 3.29269517703779e-06
1923 3.2926744568158e-06
1924 3.29264867593793e-06
1925 3.29262784600814e-06
1926 3.29259761463163e-06
1927 3.29257348857936e-06
1928 3.29255100427872e-06
1929 3.29252045787598e-06
1930 3.29250063430209e-06
1931 3.29247553599998e-06
1932 3.2924495228599e-06
1933 3.29243755550124e-06
1934 3.29241614235798e-06
1935 3.29238935705689e-06
1936 3.29236846596359e-06
1937 3.29234565685965e-06
1938 3.29231642444938e-06
1939 3.29229275598664e-06
1940 3.2922685821859e-06
1941 3.2922482378126e-06
1942 3.29221790286738e-06
1943 3.29218579645385e-06
1944 3.29216482600714e-06
1945 3.29214116948151e-06
1946 3.29212451060812e-06
1947 3.29209477672521e-06
1948 3.29206849983166e-06
1949 3.29205021364487e-06
1950 3.29202197315226e-06
1951 3.2919974479455e-06
1952 3.29197258179192e-06
1953 3.29194415348866e-06
1954 3.29193105790182e-06
1955 3.29190184550043e-06
1956 3.29187419049504e-06
1957 3.29184715440078e-06
1958 3.29182092411884e-06
1959 3.2917975328246e-06
1960 3.29177147079918e-06
1961 3.29175525337178e-06
1962 3.29172685223966e-06
1963 3.29169656640715e-06
1964 3.29167288430199e-06
1965 3.29165438870405e-06
1966 3.29162446041664e-06
1967 3.29160656929162e-06
1968 3.29158040710809e-06
1969 3.29155870622344e-06
1970 3.29153005418448e-06
1971 3.29150733500683e-06
1972 3.2914792138854e-06
1973 3.29145454725221e-06
1974 3.29142962516471e-06
1975 3.29141113172682e-06
1976 3.29138543429508e-06
1977 3.29136207938063e-06
1978 3.29133796822134e-06
1979 3.29131653199966e-06
1980 3.29130359398278e-06
1981 3.29126347924102e-06
1982 3.29124275287995e-06
1983 3.29122574260055e-06
1984 3.29119736397843e-06
1985 3.29117574744942e-06
1986 3.29114958674381e-06
1987 3.29112714518942e-06
1988 3.29110266591215e-06
1989 3.29107843094789e-06
1990 3.29104970512617e-06
1991 3.29103043191026e-06
1992 3.29101177294433e-06
1993 3.29099244595454e-06
1994 3.29096847815435e-06
1995 3.29094845267264e-06
1996 3.29092066874637e-06
1997 3.29090343745975e-06
1998 3.29087490024449e-06
1999 3.2908550610955e-06
};
\addlegendentry{Train}
\addplot [semithick, black]
table {%
0 0.0568433403968811
1 0.047224972397089
2 0.0394334606826305
3 0.0323285199701786
4 0.0258239563554525
5 0.0201245732605457
6 0.0153471510857344
7 0.01151870097965
8 0.00704659707844257
9 0.00468168454244733
10 0.00342778698541224
11 0.00270166806876659
12 0.00223693437874317
13 0.0019190963357687
14 0.00168418802786618
15 0.00149249017704278
16 0.00133072247263044
17 0.00119001336861402
18 0.00106458319351077
19 0.000951396941673011
20 0.000849163043312728
21 0.00075709872180596
22 0.000674260139930993
23 0.000600156141445041
24 0.000534308957867324
25 0.000475712498882785
26 0.00042486353777349
27 0.000381419900804758
28 0.000346150976838544
29 0.000318669364787638
30 0.000297809601761401
31 0.000282323075225577
32 0.000270986813120544
33 0.00026273270486854
34 0.000256687169894576
35 0.00025216766516678
36 0.000248661614023149
37 0.000245791423367336
38 0.000243296759435907
39 0.000240998648223467
40 0.000238787935813889
41 0.0002365982363699
42 0.000234390259720385
43 0.000232142396271229
44 0.000229843572014943
45 0.000227489785174839
46 0.000225075767957605
47 0.000222603513975628
48 0.000220058398554102
49 0.000217428547330201
50 0.000177573019755073
51 0.000159224990056828
52 0.000152720182086341
53 0.000148640465340577
54 0.000145192869240418
55 0.000141745258588344
56 0.000138252813485451
57 0.000134721660288051
58 0.000131145687191747
59 0.00012752044131048
60 0.000123834834084846
61 0.000120076867460739
62 0.000116245821118355
63 0.000112343361251988
64 0.000108371183159761
65 0.000104338119854219
66 0.000100255645520519
67 9.61318219196983e-05
68 9.19731610338204e-05
69 8.77868515090086e-05
70 8.35747341625392e-05
71 7.93364306446165e-05
72 7.50735416659154e-05
73 7.07870494807139e-05
74 6.64839681121521e-05
75 6.21785657131113e-05
76 5.78911603952292e-05
77 5.36491097591352e-05
78 4.94886917294934e-05
79 4.54529035778251e-05
80 4.15847644035239e-05
81 3.79319535568357e-05
82 3.45293810823932e-05
83 3.14146091113798e-05
84 2.86106769635808e-05
85 2.61222430708585e-05
86 2.39542769122636e-05
87 2.20961428567534e-05
88 2.05277174245566e-05
89 1.92211700777989e-05
90 1.81442792381858e-05
91 1.72656818904215e-05
92 1.65521978487959e-05
93 1.54686113091884e-05
94 1.50075047713472e-05
95 1.46070096889162e-05
96 1.42868821058073e-05
97 1.40113052111701e-05
98 1.37679726321949e-05
99 1.35474856506335e-05
100 1.33430439746007e-05
101 1.31497708935058e-05
102 1.29660156744649e-05
103 1.27881812659325e-05
104 1.26158583952929e-05
105 1.24476573546417e-05
106 1.22833544082823e-05
107 1.21227667477797e-05
108 1.19657133836881e-05
109 1.18119778562686e-05
110 1.1661164535326e-05
111 1.15129769255873e-05
112 1.13678606794565e-05
113 1.1225628441025e-05
114 1.1086508493463e-05
115 1.09500888356706e-05
116 1.08142512544873e-05
117 1.06824345493806e-05
118 1.05532053567003e-05
119 1.0426321750856e-05
120 1.03017418950913e-05
121 1.01792411442148e-05
122 1.00587594715762e-05
123 9.94066340354038e-06
124 9.82464280241402e-06
125 9.71068584476598e-06
126 9.59857243287843e-06
127 9.48829347180435e-06
128 9.38005723583046e-06
129 9.27416840568185e-06
130 9.16939734452171e-06
131 9.06731384020532e-06
132 8.9672239482752e-06
133 8.86923226062208e-06
134 8.77271668286994e-06
135 8.67869130161125e-06
136 8.58776002132799e-06
137 8.49761636345647e-06
138 8.40933535073418e-06
139 8.32121622806881e-06
140 8.23702521302039e-06
141 8.15442490420537e-06
142 8.07356082077604e-06
143 7.99586359789828e-06
144 7.9188757808879e-06
145 7.84351959737251e-06
146 7.77000968810171e-06
147 7.69835423852783e-06
148 7.62856234359788e-06
149 7.5604484663927e-06
150 7.49431092117447e-06
151 7.42947440812713e-06
152 7.36604124540463e-06
153 7.3045184763032e-06
154 7.2447046477464e-06
155 7.18621595297009e-06
156 7.12932796886889e-06
157 7.07407889422029e-06
158 7.02037959854351e-06
159 6.96832694302429e-06
160 6.91793866280932e-06
161 6.8689378167619e-06
162 6.82132531437674e-06
163 6.7760588535748e-06
164 6.73133126838366e-06
165 6.68794655211968e-06
166 6.64575827613589e-06
167 6.60486648484948e-06
168 6.56508791507804e-06
169 6.52656990496325e-06
170 6.48933519187267e-06
171 6.45325781079009e-06
172 6.41903125142562e-06
173 6.38514893580577e-06
174 6.35210426480626e-06
175 6.32019282420515e-06
176 6.28927728030249e-06
177 6.25924849373405e-06
178 6.23040887148818e-06
179 6.20257833361393e-06
180 6.17526529822499e-06
181 6.14904183748877e-06
182 6.12369240116095e-06
183 6.09930202699616e-06
184 6.07589436185663e-06
185 6.05336936132517e-06
186 6.0317129282339e-06
187 6.01082092543948e-06
188 5.9906710703217e-06
189 5.96972313360311e-06
190 5.95070787312579e-06
191 5.93227241552086e-06
192 5.91548996453639e-06
193 5.89933324590675e-06
194 5.8800460465136e-06
195 5.86422947890242e-06
196 5.84905501455069e-06
197 5.83525525144069e-06
198 5.82133179705124e-06
199 5.80777714276337e-06
200 5.79399011257919e-06
201 5.78147864871426e-06
202 5.7704469327291e-06
203 5.75944386582705e-06
204 5.7481074691168e-06
205 5.73671741221915e-06
206 5.72645376450964e-06
207 5.71617420064285e-06
208 5.7060146900767e-06
209 5.69607846045983e-06
210 5.68668974665343e-06
211 5.67699453313253e-06
212 5.66766402698704e-06
213 5.65622303838609e-06
214 5.6471517382306e-06
215 5.63813591725193e-06
216 5.62933382752817e-06
217 5.62074683330138e-06
218 5.61249544261955e-06
219 5.60409989702748e-06
220 5.59573300051852e-06
221 5.58747706236318e-06
222 5.5795112530177e-06
223 5.57149814994773e-06
224 5.56320492250961e-06
225 5.55527913093101e-06
226 5.54666621610522e-06
227 5.53910786038614e-06
228 5.53150357518462e-06
229 5.52354003957589e-06
230 5.51616403754451e-06
231 5.50846289115725e-06
232 5.50108734387322e-06
233 5.49397145732655e-06
234 5.48644311493263e-06
235 5.47872559764073e-06
236 5.47146419194178e-06
237 5.46433193449047e-06
238 5.45686725672567e-06
239 5.44945669389563e-06
240 5.44220119991223e-06
241 5.43520445717149e-06
242 5.42854968443862e-06
243 5.42142834092374e-06
244 5.41384997632122e-06
245 5.40641713087098e-06
246 5.39735810889397e-06
247 5.3902199397271e-06
248 5.38244285053224e-06
249 5.37452251592185e-06
250 5.36666402695118e-06
251 5.35727076567127e-06
252 5.34611262992257e-06
253 5.33688944415189e-06
254 5.32820604348672e-06
255 5.31908654011204e-06
256 5.31049272467499e-06
257 5.30071565663093e-06
258 5.29057615494821e-06
259 5.28079726791475e-06
260 5.27077190781711e-06
261 5.26025678482256e-06
262 5.24798451806419e-06
263 5.23672269991948e-06
264 5.22630489285802e-06
265 5.21684751220164e-06
266 5.20693993166788e-06
267 5.19680315846927e-06
268 5.1861011343135e-06
269 5.17618764206418e-06
270 5.16621958013275e-06
271 5.15666988576413e-06
272 5.147417596163e-06
273 5.13090571985231e-06
274 5.1104088925058e-06
275 5.09282563143643e-06
276 5.07438744534738e-06
277 5.05052730659372e-06
278 5.02868670082535e-06
279 5.00205396747333e-06
280 4.97506653118762e-06
281 4.94714595333789e-06
282 4.9123400458484e-06
283 4.87351417177706e-06
284 4.8323931878258e-06
285 4.787225407199e-06
286 4.7372827793879e-06
287 4.66909796159598e-06
288 4.58819476989447e-06
289 4.48287710241857e-06
290 4.37486096416251e-06
291 4.34577896157862e-06
292 4.33684863310191e-06
293 4.33276909461711e-06
294 4.32978049502708e-06
295 4.32708702646778e-06
296 4.32447268394753e-06
297 4.32206161349313e-06
298 4.31959915658808e-06
299 4.31703347203438e-06
300 4.31476792073227e-06
301 4.31259195465827e-06
302 4.31038051829091e-06
303 4.30828322350862e-06
304 4.30618047175813e-06
305 4.30411228080629e-06
306 4.30214640800841e-06
307 4.30015415986418e-06
308 4.29820056524477e-06
309 4.29637384513626e-06
310 4.29449028160889e-06
311 4.29267083745799e-06
312 4.2909068724839e-06
313 4.28908924732241e-06
314 4.28746625402709e-06
315 4.28572002419969e-06
316 4.28399653173983e-06
317 4.28231714977301e-06
318 4.28064959123731e-06
319 4.27907571065589e-06
320 4.27739223596291e-06
321 4.27574195782654e-06
322 4.27412760473089e-06
323 4.27253326051868e-06
324 4.27092254540185e-06
325 4.26919996243669e-06
326 4.2674905671447e-06
327 4.26588712798548e-06
328 4.26419546784018e-06
329 4.26264841735247e-06
330 4.26119049734552e-06
331 4.25945427195984e-06
332 4.2578167267493e-06
333 4.25598864239873e-06
334 4.25418329541571e-06
335 4.25228972744662e-06
336 4.2504029806878e-06
337 4.24864720116602e-06
338 4.24682366428897e-06
339 4.24478594140965e-06
340 4.24272502641543e-06
341 4.24059498982388e-06
342 4.23830442741746e-06
343 4.23606343247229e-06
344 4.23349820266594e-06
345 4.23091114498675e-06
346 4.22818175138673e-06
347 4.22543871536618e-06
348 4.22244193032384e-06
349 4.21937374994741e-06
350 4.21620325141703e-06
351 4.21293543695356e-06
352 4.20974720327649e-06
353 4.20661581301829e-06
354 4.202670425002e-06
355 4.19863454226288e-06
356 4.19480647906312e-06
357 4.19104162574513e-06
358 4.18737818108639e-06
359 4.18363970311475e-06
360 4.17987575929146e-06
361 4.17618002757081e-06
362 4.17271758124116e-06
363 4.16939792557969e-06
364 4.16632610722445e-06
365 4.16302964367787e-06
366 4.15990734836669e-06
367 4.15709519074881e-06
368 4.15459908253979e-06
369 4.15233398598502e-06
370 4.15026715927524e-06
371 4.14833357353928e-06
372 4.14638589063543e-06
373 4.14477153753978e-06
374 4.14328587794444e-06
375 4.14186933994642e-06
376 4.14051510233548e-06
377 4.13920224673348e-06
378 4.13789530284703e-06
379 4.13672933063935e-06
380 4.13551788369659e-06
381 4.13438783652964e-06
382 4.1332641558256e-06
383 4.13216548622586e-06
384 4.13116686104331e-06
385 4.13001862398232e-06
386 4.12884037359618e-06
387 4.12766848967294e-06
388 4.12661029258743e-06
389 4.12538474847679e-06
390 4.12427607443533e-06
391 4.1231328395952e-06
392 4.12199187849183e-06
393 4.12096369473147e-06
394 4.11973678637878e-06
395 4.11872224503895e-06
396 4.1176153899869e-06
397 4.11637756769778e-06
398 4.11526934840367e-06
399 4.11425025959034e-06
400 4.1131033867714e-06
401 4.11193514082697e-06
402 4.11086602980504e-06
403 4.10972324971226e-06
404 4.10870961786713e-06
405 4.10750726587139e-06
406 4.10644906878588e-06
407 4.10532265959773e-06
408 4.10424945584964e-06
409 4.10317534260685e-06
410 4.10211077905842e-06
411 4.10101802117424e-06
412 4.0999971133715e-06
413 4.09878657592344e-06
414 4.09775748266838e-06
415 4.09670246881433e-06
416 4.09567610404338e-06
417 4.094483301742e-06
418 4.09349331675912e-06
419 4.0924824133981e-06
420 4.09131143896957e-06
421 4.09029598813504e-06
422 4.08925325245946e-06
423 4.08829646403319e-06
424 4.08713640354108e-06
425 4.08610048907576e-06
426 4.08502683058032e-06
427 4.08408504881663e-06
428 4.08306641475065e-06
429 4.0820168578648e-06
430 4.08097184845246e-06
431 4.07991092288285e-06
432 4.07887318942812e-06
433 4.077807716385e-06
434 4.07683410230675e-06
435 4.07579136663117e-06
436 4.07471907237777e-06
437 4.07367633670219e-06
438 4.07267498303554e-06
439 4.07158495363547e-06
440 4.07051129514002e-06
441 4.06947674491676e-06
442 4.0684217310627e-06
443 4.06738354286063e-06
444 4.06640037908801e-06
445 4.06531489716144e-06
446 4.06433218813618e-06
447 4.06334720537416e-06
448 4.06233357352903e-06
449 4.06124536311836e-06
450 4.06029084842885e-06
451 4.05928358304664e-06
452 4.05830905947369e-06
453 4.05728587793419e-06
454 4.05626224164735e-06
455 4.05525361202308e-06
456 4.05425225835643e-06
457 4.05319724450237e-06
458 4.05225819122279e-06
459 4.05121909352602e-06
460 4.05020455218619e-06
461 4.04918637286755e-06
462 4.04817456001183e-06
463 4.04717320634518e-06
464 4.04620186600368e-06
465 4.04520960728405e-06
466 4.04416232413496e-06
467 4.04312095270143e-06
468 4.04213096771855e-06
469 4.04110232921084e-06
470 4.04022011935012e-06
471 4.03914737034938e-06
472 4.03814101446187e-06
473 4.03716876462568e-06
474 4.03614649258088e-06
475 4.03518879465992e-06
476 4.03414969696314e-06
477 4.03314970753854e-06
478 4.03220519729075e-06
479 4.03126932724263e-06
480 4.03023796025082e-06
481 4.02923842557357e-06
482 4.02826435674797e-06
483 4.0272316255141e-06
484 4.02624846174149e-06
485 4.02529076382052e-06
486 4.0243298826681e-06
487 4.02338446292561e-06
488 4.02234809371294e-06
489 4.02110936192912e-06
490 4.02040541302995e-06
491 4.01931993110338e-06
492 4.01828401663806e-06
493 4.01729630539194e-06
494 4.01628221879946e-06
495 4.01531178795267e-06
496 4.01431452701217e-06
497 4.01290753870853e-06
498 4.01184797738097e-06
499 4.01082616008352e-06
500 4.00969611291657e-06
501 4.00863609684166e-06
502 4.00757107854588e-06
503 4.00646467824117e-06
504 4.00541603085003e-06
505 4.0043764784059e-06
506 4.00332692152006e-06
507 4.00238195652491e-06
508 4.00135832023807e-06
509 4.00029102820554e-06
510 3.99923919758294e-06
511 3.99821192331729e-06
512 3.99717919208342e-06
513 3.99615692003863e-06
514 3.9951146391104e-06
515 3.9940905480762e-06
516 3.9931387618708e-06
517 3.99215014112997e-06
518 3.99115833715769e-06
519 3.99014652430196e-06
520 3.9890906009532e-06
521 3.98799647882697e-06
522 3.98712882088148e-06
523 3.98604743168107e-06
524 3.98505335397203e-06
525 3.98403471990605e-06
526 3.98307474824833e-06
527 3.98210158891743e-06
528 3.98107749788323e-06
529 3.98007114199572e-06
530 3.97902795157279e-06
531 3.97806252294686e-06
532 3.9770575313014e-06
533 3.97586836697883e-06
534 3.97478561353637e-06
535 3.97380608774256e-06
536 3.97279791286564e-06
537 3.97179201172548e-06
538 3.97077928937506e-06
539 3.96976929550874e-06
540 3.96890618503676e-06
541 3.96785389966681e-06
542 3.9668234421697e-06
543 3.96581754102954e-06
544 3.96481300413143e-06
545 3.96383848055848e-06
546 3.96281029679812e-06
547 3.96184395867749e-06
548 3.96079258280224e-06
549 3.95980259781936e-06
550 3.95881215808913e-06
551 3.95786992157809e-06
552 3.95688721255283e-06
553 3.95591041524312e-06
554 3.95486904380959e-06
555 3.95391725760419e-06
556 3.95294728150475e-06
557 3.95196229874273e-06
558 3.95099004890653e-06
559 3.94998323827167e-06
560 3.94902781408746e-06
561 3.94804555980954e-06
562 3.94705739381607e-06
563 3.94637527278974e-06
564 3.94534754377673e-06
565 3.9443293644581e-06
566 3.9433784877474e-06
567 3.94233484257711e-06
568 3.94137668990879e-06
569 3.9403812479577e-06
570 3.93937307308079e-06
571 3.93839491152903e-06
572 3.93726350012003e-06
573 3.93624441130669e-06
574 3.93536083720392e-06
575 3.93429536416079e-06
576 3.93335722037591e-06
577 3.93234176954138e-06
578 3.93130312659196e-06
579 3.93029768019915e-06
580 3.92933407056262e-06
581 3.92832998841186e-06
582 3.92729771192535e-06
583 3.92627953260671e-06
584 3.92531228499138e-06
585 3.9253982322407e-06
586 3.92451784136938e-06
587 3.92358924727887e-06
588 3.92271294913371e-06
589 3.92267156712478e-06
590 3.92187757825013e-06
591 3.92106267099734e-06
592 3.92016181649524e-06
593 3.91936828236794e-06
594 3.91860066883964e-06
595 3.91765797758126e-06
596 3.91675393984769e-06
597 3.91575031244429e-06
598 3.91484854844748e-06
599 3.91385901821195e-06
600 3.91289677281748e-06
601 3.91241337638348e-06
602 3.91134426536155e-06
603 3.91038884117734e-06
604 3.90942113881465e-06
605 3.90843297282117e-06
606 3.9074225242075e-06
607 3.90633340430213e-06
608 3.90528566640569e-06
609 3.90428249374963e-06
610 3.90326795240981e-06
611 3.90220657209284e-06
612 3.90120521842618e-06
613 3.90016975870822e-06
614 3.89916021958925e-06
615 3.89808792533586e-06
616 3.89705928682815e-06
617 3.8959828998486e-06
618 3.89491970054223e-06
619 3.8938283069001e-06
620 3.89269143852289e-06
621 3.89153774449369e-06
622 3.89058868677239e-06
623 3.88932448913692e-06
624 3.8881821637915e-06
625 3.88695161746e-06
626 3.88575926990598e-06
627 3.88455828215228e-06
628 3.88347871194128e-06
629 3.88226226277766e-06
630 3.8811062950117e-06
631 3.87992804462556e-06
632 3.87876707463874e-06
633 3.87756836062181e-06
634 3.87646559829591e-06
635 3.87523004974355e-06
636 3.87393629353028e-06
637 3.87255795430974e-06
638 3.87124282497098e-06
639 3.86983174394118e-06
640 3.86849569622427e-06
641 3.86709780286765e-06
642 3.86561441700906e-06
643 3.86423744203057e-06
644 3.86281180908554e-06
645 3.86143665309646e-06
646 3.85995463147992e-06
647 3.85841804018128e-06
648 3.8570178730879e-06
649 3.85553676096606e-06
650 3.85387829737738e-06
651 3.85236080546747e-06
652 3.85077282771817e-06
653 3.84913755624439e-06
654 3.84752320314874e-06
655 3.84600889447029e-06
656 3.84387294616317e-06
657 3.84187251256662e-06
658 3.83996211894555e-06
659 3.83782480639638e-06
660 3.83526230507414e-06
661 3.83203814635635e-06
662 3.8289063013508e-06
663 3.82630469175638e-06
664 3.82399912268738e-06
665 3.82199232262792e-06
666 3.82021153200185e-06
667 3.81862764697871e-06
668 3.81716654374031e-06
669 3.81588051823201e-06
670 3.8146592942212e-06
671 3.81359836865158e-06
672 3.81254335479753e-06
673 3.81157178708236e-06
674 3.81064523935493e-06
675 3.80975689040497e-06
676 3.80890378437471e-06
677 3.80807523470139e-06
678 3.8072862480476e-06
679 3.80653000320308e-06
680 3.80576148018008e-06
681 3.80505093744432e-06
682 3.80431856683572e-06
683 3.80359733753721e-06
684 3.80286201107083e-06
685 3.80213623429881e-06
686 3.801424327321e-06
687 3.80074448003143e-06
688 3.80007259082049e-06
689 3.79935909222695e-06
690 3.79865923605394e-06
691 3.79799598704267e-06
692 3.79731136490591e-06
693 3.79664970751037e-06
694 3.79594553123752e-06
695 3.79531616090389e-06
696 3.79458469978999e-06
697 3.79391303795273e-06
698 3.79326934307755e-06
699 3.79258403881977e-06
700 3.79189964405668e-06
701 3.79125708604988e-06
702 3.79057814825501e-06
703 3.78992331206973e-06
704 3.7892550608376e-06
705 3.78860227101541e-06
706 3.78793765776209e-06
707 3.78729782823939e-06
708 3.78659228772449e-06
709 3.78597610506404e-06
710 3.7853180856473e-06
711 3.78462664230028e-06
712 3.78397862732527e-06
713 3.78331287720357e-06
714 3.78265372091846e-06
715 3.78200934392225e-06
716 3.78134109269013e-06
717 3.78070626538829e-06
718 3.78006097889738e-06
719 3.77938113160781e-06
720 3.77875903723179e-06
721 3.77808737539453e-06
722 3.77745595869783e-06
723 3.77680589735974e-06
724 3.77614264834847e-06
725 3.77549645236286e-06
726 3.7748404793092e-06
727 3.77419246433419e-06
728 3.77356400349527e-06
729 3.77290098185767e-06
730 3.77224455405667e-06
731 3.77160608877602e-06
732 3.77097194359521e-06
733 3.77032870346738e-06
734 3.76968614546058e-06
735 3.76903381038574e-06
736 3.76839307136834e-06
737 3.76775551558239e-06
738 3.76710886484943e-06
739 3.76647722077905e-06
740 3.765845349335e-06
741 3.76534171664389e-06
742 3.76479397345975e-06
743 3.76420757675078e-06
744 3.7636286833731e-06
745 3.76303501070652e-06
746 3.76244861399755e-06
747 3.76184516426292e-06
748 3.76123625756009e-06
749 3.76062621398887e-06
750 3.7600361793011e-06
751 3.75940385310969e-06
752 3.7587833503494e-06
753 3.75818376596726e-06
754 3.75754530068662e-06
755 3.75691729459504e-06
756 3.75634908778011e-06
757 3.75568424715311e-06
758 3.75506351701915e-06
759 3.75451372747193e-06
760 3.75381819139875e-06
761 3.75319382328598e-06
762 3.75258400708844e-06
763 3.75196646018594e-06
764 3.7513316328841e-06
765 3.7507400065806e-06
766 3.75007175534847e-06
767 3.74944124814647e-06
768 3.7488289308385e-06
769 3.74819683202077e-06
770 3.74759724763862e-06
771 3.74693763660616e-06
772 3.74634555555531e-06
773 3.74570413441688e-06
774 3.7451122807397e-06
775 3.74445698980708e-06
776 3.74384308088338e-06
777 3.74321302842873e-06
778 3.74259343516314e-06
779 3.74195542462985e-06
780 3.74136789105251e-06
781 3.74073010789289e-06
782 3.74011096937465e-06
783 3.73950297216652e-06
784 3.73888269677991e-06
785 3.73824695998337e-06
786 3.73760985894478e-06
787 3.73700368072605e-06
788 3.73638044948166e-06
789 3.73578814105713e-06
790 3.73515013052383e-06
791 3.73453349311603e-06
792 3.73392140318174e-06
793 3.73332568415208e-06
794 3.73272996512242e-06
795 3.73209536519425e-06
796 3.73148509424936e-06
797 3.73086777472054e-06
798 3.7302440887288e-06
799 3.72963404515758e-06
800 3.72902309209167e-06
801 3.72840418094711e-06
802 3.72779845747573e-06
803 3.72721547137189e-06
804 3.7265774608386e-06
805 3.72596969100414e-06
806 3.72536624126951e-06
807 3.72475824406138e-06
808 3.72415956917393e-06
809 3.64847437595017e-06
810 3.59642240255198e-06
811 3.59560181095731e-06
812 3.59467708221928e-06
813 3.59391128768038e-06
814 3.59319528797641e-06
815 3.59250680048717e-06
816 3.59184855369676e-06
817 3.59110276804131e-06
818 3.59038335773221e-06
819 3.58960483026749e-06
820 3.58890702045755e-06
821 3.58833676727954e-06
822 3.58765237251646e-06
823 3.58684246748453e-06
824 3.5861871765519e-06
825 3.58602960659482e-06
826 3.58539750777709e-06
827 3.58475494977029e-06
828 3.58411898560007e-06
829 3.58349689122406e-06
830 3.58286570190103e-06
831 3.58237184627797e-06
832 3.58176112058572e-06
833 3.58113493348355e-06
834 3.58080342266476e-06
835 3.5801947433356e-06
836 3.57960243491107e-06
837 3.57904650627461e-06
838 3.57845829057624e-06
839 3.57788758265087e-06
840 3.5771549846686e-06
841 3.57660360350565e-06
842 3.57605154022167e-06
843 3.57550266016915e-06
844 3.57495764546911e-06
845 3.57435942532902e-06
846 3.57378712578793e-06
847 3.57324847755081e-06
848 3.57266935679945e-06
849 3.57209751200571e-06
850 3.57153521690634e-06
851 3.57094063474506e-06
852 3.57054250343936e-06
853 3.57013573193399e-06
854 3.56950386048993e-06
855 3.56886607733031e-06
856 3.56824921254884e-06
857 3.5676912375493e-06
858 3.56706004822627e-06
859 3.56643795385025e-06
860 3.56573059434595e-06
861 3.56524333255948e-06
862 3.56452687810815e-06
863 3.5637958717416e-06
864 3.56305531568069e-06
865 3.56231248588301e-06
866 3.5615876186057e-06
867 3.56086707142822e-06
868 3.56003351953404e-06
869 3.55929728357296e-06
870 3.55856195710658e-06
871 3.55784118255542e-06
872 3.55701968146604e-06
873 3.55619408765051e-06
874 3.5554312489694e-06
875 3.55463066625816e-06
876 3.55383076566795e-06
877 3.55256133843795e-06
878 3.55098836735124e-06
879 3.54978806171857e-06
880 3.54889630216348e-06
881 3.54801682078687e-06
882 3.54751728082192e-06
883 3.54665098711848e-06
884 3.54579765371454e-06
885 3.54480039277405e-06
886 3.54397548107954e-06
887 3.54310350303422e-06
888 3.54229473487067e-06
889 3.54152211912151e-06
890 3.54067833541194e-06
891 3.53981704392936e-06
892 3.53890050064365e-06
893 3.53805421582365e-06
894 3.5371806461626e-06
895 3.53630071003863e-06
896 3.53556470145122e-06
897 3.53462701241369e-06
898 3.53376117345761e-06
899 3.53297923538776e-06
900 3.53260816154943e-06
901 3.53172640643606e-06
902 3.53089353666292e-06
903 3.52999722963432e-06
904 3.52902065969829e-06
905 3.52815732185263e-06
906 3.52735219166789e-06
907 3.52652136825782e-06
908 3.5256937280792e-06
909 3.5248585845693e-06
910 3.52405390913191e-06
911 3.52389520230645e-06
912 3.52304437001294e-06
913 3.52219126398268e-06
914 3.52137249137741e-06
915 3.52055690200359e-06
916 3.51974881596107e-06
917 3.51893891092914e-06
918 3.51824382960331e-06
919 3.51723406311066e-06
920 3.51639914697444e-06
921 3.51553921973391e-06
922 3.51462813341641e-06
923 3.5137652503181e-06
924 3.51280164068157e-06
925 3.51190260516887e-06
926 3.51102767126577e-06
927 3.51017865796166e-06
928 3.50932214132627e-06
929 3.508443114697e-06
930 3.50747109223448e-06
931 3.50665936821315e-06
932 3.50584377883933e-06
933 3.50503910340194e-06
934 3.5042755825998e-06
935 3.50351251654502e-06
936 3.50273921867483e-06
937 3.50191885445383e-06
938 3.50118193637172e-06
939 3.5004645724257e-06
940 3.49979018210433e-06
941 3.49906076735351e-06
942 3.49835090673878e-06
943 3.49767060470185e-06
944 3.49704168911558e-06
945 3.49638980878808e-06
946 3.49573724633956e-06
947 3.49509718944319e-06
948 3.49445576830476e-06
949 3.49382867170789e-06
950 3.49323499904131e-06
951 3.49261608789675e-06
952 3.49204242411361e-06
953 3.49188076143037e-06
954 3.49130027643696e-06
955 3.49074980476871e-06
956 3.49021252077364e-06
957 3.48971411767707e-06
958 3.48919593307073e-06
959 3.48869707522681e-06
960 3.48821276929812e-06
961 3.48774051417422e-06
962 3.48726575793989e-06
963 3.4869842693297e-06
964 3.48615799339314e-06
965 3.48567505170649e-06
966 3.48518642567797e-06
967 3.48474600286863e-06
968 3.48427079188696e-06
969 3.48385879078705e-06
970 3.48344065059791e-06
971 3.48303206010314e-06
972 3.4826371120289e-06
973 3.4822435281967e-06
974 3.48185085385921e-06
975 3.48145817952172e-06
976 3.48107892023108e-06
977 3.48072285305534e-06
978 3.48035950992198e-06
979 3.47990180671331e-06
980 3.47954073731671e-06
981 3.47919421983534e-06
982 3.47885907103773e-06
983 3.47850505022507e-06
984 3.47814489032316e-06
985 3.47778723153169e-06
986 3.47745708495495e-06
987 3.47711443282606e-06
988 3.47679247170163e-06
989 3.47646550835634e-06
990 3.47620175489283e-06
991 3.47589843840979e-06
992 3.4755826163746e-06
993 3.47527634403377e-06
994 3.47493505614693e-06
995 3.47462741956406e-06
996 3.47427271663037e-06
997 3.47399941347248e-06
998 3.47366403730121e-06
999 3.47336936101783e-06
1000 3.47308059645002e-06
1001 3.47278978551913e-06
1002 3.47252125720843e-06
1003 3.47222271557257e-06
1004 3.47194327332545e-06
1005 3.47157492797123e-06
1006 3.47128207067726e-06
1007 3.4709798910626e-06
1008 3.47070181305753e-06
1009 3.47043805959402e-06
1010 3.4701733966358e-06
1011 3.46992169397708e-06
1012 3.46964520758775e-06
1013 3.46940714734956e-06
1014 3.46916544913256e-06
1015 3.46894944414089e-06
1016 3.46870342582406e-06
1017 3.46830188391323e-06
1018 3.46803972206544e-06
1019 3.46778028870176e-06
1020 3.46753108715347e-06
1021 3.46734827871842e-06
1022 3.46709043697047e-06
1023 3.46685646945843e-06
1024 3.46661499861511e-06
1025 3.46639239978686e-06
1026 3.46617730428989e-06
1027 3.46595038536179e-06
1028 3.46575893672707e-06
1029 3.46556316799251e-06
1030 3.46535557582683e-06
1031 3.46518686455966e-06
1032 3.46497859027295e-06
1033 3.46481351698458e-06
1034 3.46462866218644e-06
1035 3.46444949173019e-06
1036 3.46428146258404e-06
1037 3.46412798535312e-06
1038 3.46395700034918e-06
1039 3.4637864700926e-06
1040 3.46360252478917e-06
1041 3.46343972523755e-06
1042 3.46326783073891e-06
1043 3.46310616805567e-06
1044 3.46294041264628e-06
1045 3.4627846616786e-06
1046 3.46263209394237e-06
1047 3.4624911222636e-06
1048 3.4623365081643e-06
1049 3.46220349456416e-06
1050 3.46205001733324e-06
1051 3.46188244293444e-06
1052 3.46175283993944e-06
1053 3.46159595210338e-06
1054 3.46143474416749e-06
1055 3.46129900208325e-06
1056 3.46117508343013e-06
1057 3.4610300190252e-06
1058 3.46089314007259e-06
1059 3.46078195434529e-06
1060 3.46061665368325e-06
1061 3.46049819199834e-06
1062 3.46038041243446e-06
1063 3.46024376085552e-06
1064 3.46011302099214e-06
1065 3.45999001183372e-06
1066 3.45985517924419e-06
1067 3.45973512594355e-06
1068 3.45962348546891e-06
1069 3.45949979418947e-06
1070 3.45938201462559e-06
1071 3.45927628586651e-06
1072 3.45916373589716e-06
1073 3.45904686582799e-06
1074 3.45894613928976e-06
1075 3.45884564012522e-06
1076 3.45871967510902e-06
1077 3.45848570759699e-06
1078 3.45837111126457e-06
1079 3.45827857017866e-06
1080 3.45818875757686e-06
1081 3.45806097357126e-06
1082 3.45797980116913e-06
1083 3.45788293998339e-06
1084 3.45780222232861e-06
1085 3.4576967209432e-06
1086 3.45762964570895e-06
1087 3.45751823260798e-06
1088 3.45742751051148e-06
1089 3.45735247719858e-06
1090 3.45727721651201e-06
1091 3.45740090779145e-06
1092 3.45732860296266e-06
1093 3.45726084560738e-06
1094 3.45718422067876e-06
1095 3.4571032756503e-06
1096 3.45703665516339e-06
1097 3.4569504805404e-06
1098 3.45687203662237e-06
1099 3.45679291058332e-06
1100 3.45670810020238e-06
1101 3.45667535839311e-06
1102 3.4565903206385e-06
1103 3.456525519141e-06
1104 3.45645207744383e-06
1105 3.45637204191007e-06
1106 3.45630019182863e-06
1107 3.45623493558378e-06
1108 3.45616308550234e-06
1109 3.45608373208961e-06
1110 3.45603757523349e-06
1111 3.45596231454692e-06
1112 3.4558995594125e-06
1113 3.45583111993619e-06
1114 3.45576040672313e-06
1115 3.45570288118324e-06
1116 3.45563921655412e-06
1117 3.45557327818824e-06
1118 3.45551143254852e-06
1119 3.45544435731426e-06
1120 3.45539160662156e-06
1121 3.45533499057638e-06
1122 3.45527519129973e-06
1123 3.45523403666448e-06
1124 3.4551949283923e-06
1125 3.4551449061837e-06
1126 3.45509465660143e-06
1127 3.4550421332824e-06
1128 3.45499211107381e-06
1129 3.45493117492879e-06
1130 3.45488638231473e-06
1131 3.45482908414851e-06
1132 3.45477678820316e-06
1133 3.45472699336824e-06
1134 3.45465718964988e-06
1135 3.45460034623102e-06
1136 3.4545764719951e-06
1137 3.45450780514511e-06
1138 3.45446278515738e-06
1139 3.45440957971732e-06
1140 3.45435728377197e-06
1141 3.45430294146354e-06
1142 3.45424791703408e-06
1143 3.45419357472565e-06
1144 3.45415014635364e-06
1145 3.45410035151872e-06
1146 3.45405260304688e-06
1147 3.45398188983381e-06
1148 3.45394391843001e-06
1149 3.45389094036364e-06
1150 3.45383887179196e-06
1151 3.45379703503568e-06
1152 3.45373905474844e-06
1153 3.45369198839762e-06
1154 3.45364333043108e-06
1155 3.45358921549632e-06
1156 3.45354374076123e-06
1157 3.45349349117896e-06
1158 3.45345802088559e-06
1159 3.45338912666193e-06
1160 3.45334115081641e-06
1161 3.45329385709192e-06
1162 3.45325543094077e-06
1163 3.45320881933731e-06
1164 3.45316561833897e-06
1165 3.45310786542541e-06
1166 3.45306239069032e-06
1167 3.45301691595523e-06
1168 3.45295961778902e-06
1169 3.45291505254863e-06
1170 3.45285684488772e-06
1171 3.4527838579379e-06
1172 3.45271746482467e-06
1173 3.45266084877949e-06
1174 3.45251783073763e-06
1175 3.45242051480454e-06
1176 3.45230546372477e-06
1177 3.4522820442362e-06
1178 3.4522690839367e-06
1179 3.45227726938901e-06
1180 3.45225180353737e-06
1181 3.45224066222727e-06
1182 3.45223929798522e-06
1183 3.45221246789151e-06
1184 3.45218927577662e-06
1185 3.45216699315642e-06
1186 3.45217290487199e-06
1187 3.45213084074203e-06
1188 3.45212220054236e-06
1189 3.45209582519601e-06
1190 3.4520644476288e-06
1191 3.45206194651837e-06
1192 3.45202533935662e-06
1193 3.45200965057302e-06
1194 3.45199441653676e-06
1195 3.45197622664273e-06
1196 3.45194666806492e-06
1197 3.45192029271857e-06
1198 3.45191369888198e-06
1199 3.45188595929358e-06
1200 3.45185389960534e-06
1201 3.4518338907219e-06
1202 3.45181001648598e-06
1203 3.451784778008e-06
1204 3.45176454175089e-06
1205 3.45173839377821e-06
1206 3.45171383742127e-06
1207 3.45168041349098e-06
1208 3.45164903592377e-06
1209 3.45163562087691e-06
1210 3.45162197845639e-06
1211 3.45160401593603e-06
1212 3.45156854564266e-06
1213 3.4515403513069e-06
1214 3.45150670000294e-06
1215 3.45147736879881e-06
1216 3.45145167557348e-06
1217 3.45142029800627e-06
1218 3.45139642377035e-06
1219 3.45136481882946e-06
1220 3.45134071721986e-06
1221 3.45129069501127e-06
1222 3.45126431966492e-06
1223 3.45123862643959e-06
1224 3.45122134604026e-06
1225 3.45118041877868e-06
1226 3.45115108757454e-06
1227 3.45111561728118e-06
1228 3.45110083799227e-06
1229 3.45107605426165e-06
1230 3.4510403565946e-06
1231 3.45102034771116e-06
1232 3.45099260812276e-06
1233 3.45095236298221e-06
1234 3.45091234521533e-06
1235 3.45088324138487e-06
1236 3.45085777553322e-06
1237 3.45081616615062e-06
1238 3.4507968393882e-06
1239 3.45077205565758e-06
1240 3.4507199870859e-06
1241 3.45071248375461e-06
1242 3.45068951901339e-06
1243 3.45065291185165e-06
1244 3.45062017004238e-06
1245 3.45057696904405e-06
1246 3.45055832440266e-06
1247 3.45054650097154e-06
1248 3.45050329997321e-06
1249 3.45047124028497e-06
1250 3.45043736160733e-06
1251 3.45041826221859e-06
1252 3.45039484273002e-06
1253 3.45034459314775e-06
1254 3.45031730830669e-06
1255 3.45027592629776e-06
1256 3.45024091075175e-06
1257 3.45021408065804e-06
1258 3.45018179359613e-06
1259 3.45018247571716e-06
1260 3.45015496350243e-06
1261 3.45009379998373e-06
1262 3.45005400959053e-06
1263 3.45006924362679e-06
1264 3.45000466950296e-06
1265 3.45000694323971e-06
1266 3.44993986800546e-06
1267 3.44994555234734e-06
1268 3.44990576195414e-06
1269 3.449884843576e-06
1270 3.44988097822352e-06
1271 3.44982140632055e-06
1272 3.44979548572155e-06
1273 3.44976774613315e-06
1274 3.44974432664458e-06
1275 3.44969930665684e-06
1276 3.4496672469686e-06
1277 3.4496422358643e-06
1278 3.44962404597027e-06
1279 3.44958266396134e-06
1280 3.44953286912641e-06
1281 3.44950035469083e-06
1282 3.44947102348669e-06
1283 3.44944328389829e-06
1284 3.44941236107843e-06
1285 3.44938257512695e-06
1286 3.44933573614981e-06
1287 3.44930504070362e-06
1288 3.44926866091555e-06
1289 3.4492431950639e-06
1290 3.44921022588096e-06
1291 3.44915792993561e-06
1292 3.44912268701592e-06
1293 3.44908539773314e-06
1294 3.44907493854407e-06
1295 3.44903537552455e-06
1296 3.44900718118879e-06
1297 3.44899103765783e-06
1298 3.44894351655967e-06
1299 3.4489116842451e-06
1300 3.44888348990935e-06
1301 3.44884961123171e-06
1302 3.44882391800638e-06
1303 3.44879435942858e-06
1304 3.44876934832428e-06
1305 3.44874774782511e-06
1306 3.44871659763157e-06
1307 3.44868158208556e-06
1308 3.44861973644583e-06
1309 3.44857994605263e-06
1310 3.44855334333261e-06
1311 3.4485497053538e-06
1312 3.44850877809222e-06
1313 3.44849081557186e-06
1314 3.44844761457352e-06
1315 3.44842533195333e-06
1316 3.44836826116079e-06
1317 3.44832983500964e-06
1318 3.44830277754227e-06
1319 3.44827503795386e-06
1320 3.4482450246287e-06
1321 3.44821182807209e-06
1322 3.44818840858352e-06
1323 3.44816976394213e-06
1324 3.44813020092261e-06
1325 3.44809154739778e-06
1326 3.44806994689861e-06
1327 3.44805084750988e-06
1328 3.44801219398505e-06
1329 3.4479885471228e-06
1330 3.44795580531354e-06
1331 3.44790964845743e-06
1332 3.44788509210048e-06
1333 3.4478700854379e-06
1334 3.44782688443956e-06
1335 3.44779186889355e-06
1336 3.44776435667882e-06
1337 3.44773980032187e-06
1338 3.44769205185003e-06
1339 3.44767795468215e-06
1340 3.44765635418298e-06
1341 3.44761338055832e-06
1342 3.44758359460684e-06
1343 3.44755471815006e-06
1344 3.44752993441944e-06
1345 3.44749605574179e-06
1346 3.44746899827442e-06
1347 3.44742693414446e-06
1348 3.44740305990854e-06
1349 3.44736940860457e-06
1350 3.44733666679531e-06
1351 3.44732893609034e-06
1352 3.44731370205409e-06
1353 3.44728346135525e-06
1354 3.44726595358225e-06
1355 3.44721752298938e-06
1356 3.44719433087448e-06
1357 3.44716841027548e-06
1358 3.44712361766142e-06
1359 3.44709792443609e-06
1360 3.44707063959504e-06
1361 3.44705608767981e-06
1362 3.4470126593078e-06
1363 3.446974687904e-06
1364 3.44696559295699e-06
1365 3.44692330145335e-06
1366 3.44689010489674e-06
1367 3.44684985975618e-06
1368 3.44683758157771e-06
1369 3.44680324815272e-06
1370 3.44676072927541e-06
1371 3.44675117958104e-06
1372 3.44670456797758e-06
1373 3.44669456353586e-06
1374 3.44665386364795e-06
1375 3.4466265788069e-06
1376 3.44657701134565e-06
1377 3.44657905770873e-06
1378 3.44654313266801e-06
1379 3.44650379702216e-06
1380 3.44648356076505e-06
1381 3.44643513017218e-06
1382 3.44644195138244e-06
1383 3.44640011462616e-06
1384 3.44638374372153e-06
1385 3.44631666848727e-06
1386 3.44629802384588e-06
1387 3.44626187143149e-06
1388 3.44626755577337e-06
1389 3.44624299941643e-06
1390 3.44620934811246e-06
1391 3.44616023539857e-06
1392 3.44613272318384e-06
1393 3.44609384228534e-06
1394 3.44600448443089e-06
1395 3.44579302691272e-06
1396 3.44572731592052e-06
1397 3.44568707077997e-06
1398 3.44564386978163e-06
1399 3.44563272847154e-06
1400 3.4455479180906e-06
1401 3.44548152497737e-06
1402 3.44544832842075e-06
1403 3.44539944308053e-06
1404 3.44532645613072e-06
1405 3.44531008522608e-06
1406 3.44524983120209e-06
1407 3.44520594808273e-06
1408 3.44516865879996e-06
1409 3.44515410688473e-06
1410 3.44506656801968e-06
1411 3.44504042004701e-06
1412 3.44499949278543e-06
1413 3.44496675097616e-06
1414 3.44493378179322e-06
1415 3.44488626069506e-06
1416 3.44486988979043e-06
1417 3.44483623848646e-06
1418 3.44478644365154e-06
1419 3.4447405141691e-06
1420 3.44472005053831e-06
1421 3.44468389812391e-06
1422 3.44463296642061e-06
1423 3.44459044754331e-06
1424 3.44455770573404e-06
1425 3.44453269462974e-06
1426 3.44449176736816e-06
1427 3.44445970767993e-06
1428 3.44442401001288e-06
1429 3.44438785759849e-06
1430 3.44437353305693e-06
1431 3.44432396559569e-06
1432 3.44428713106026e-06
1433 3.44428644893924e-06
1434 3.44422460329952e-06
1435 3.44419436260068e-06
1436 3.44416207553877e-06
1437 3.44413433595037e-06
1438 3.44410318575683e-06
1439 3.44405816576909e-06
1440 3.44404429597489e-06
1441 3.44399995810818e-06
1442 3.44397153639875e-06
1443 3.44394288731564e-06
1444 3.44391901307972e-06
1445 3.44388513440208e-06
1446 3.44383488481981e-06
1447 3.44381328432064e-06
1448 3.44378349836916e-06
1449 3.44374961969152e-06
1450 3.44370755556156e-06
1451 3.44368413607299e-06
1452 3.44364616466919e-06
1453 3.44362524629105e-06
1454 3.44357931680861e-06
1455 3.44356976711424e-06
1456 3.44353315995249e-06
1457 3.44349768965913e-06
1458 3.44346312886046e-06
1459 3.44343084179854e-06
1460 3.443408104431e-06
1461 3.44337217939028e-06
1462 3.44333648172324e-06
1463 3.44329623658268e-06
1464 3.44327395396249e-06
1465 3.44325189871597e-06
1466 3.44321142620174e-06
1467 3.44317709277675e-06
1468 3.44314798894629e-06
1469 3.44310660693736e-06
1470 3.44308205058041e-06
1471 3.44303998645046e-06
1472 3.44302998200874e-06
1473 3.44298928212083e-06
1474 3.44296995535842e-06
1475 3.44294448950677e-06
1476 3.44290515386092e-06
1477 3.44287241205166e-06
1478 3.44284171660547e-06
1479 3.44282602782187e-06
1480 3.44277032127138e-06
1481 3.44275986208231e-06
1482 3.44272098118381e-06
1483 3.44270461027918e-06
1484 3.44265708918101e-06
1485 3.44262207363499e-06
1486 3.44260251949891e-06
1487 3.44256386597408e-06
1488 3.44253999173816e-06
1489 3.44251247952343e-06
1490 3.44248337569297e-06
1491 3.44245358974149e-06
1492 3.4424201658112e-06
1493 3.44238719662826e-06
1494 3.44237196259201e-06
1495 3.4423433135089e-06
1496 3.44231352755742e-06
1497 3.44227692039567e-06
1498 3.44224713444419e-06
1499 3.44219802173029e-06
1500 3.44216527992103e-06
1501 3.44212799063826e-06
1502 3.44207887792436e-06
1503 3.44204727298347e-06
1504 3.4420118026901e-06
1505 3.44196791957074e-06
1506 3.44194290846644e-06
1507 3.44191266776761e-06
1508 3.44187765222159e-06
1509 3.44184741152276e-06
1510 3.44180989486631e-06
1511 3.44177624356234e-06
1512 3.4417516872054e-06
1513 3.44171780852776e-06
1514 3.44167006005591e-06
1515 3.44164641319367e-06
1516 3.44161753673689e-06
1517 3.44158183906984e-06
1518 3.4415506888763e-06
1519 3.44151567333029e-06
1520 3.44148952535761e-06
1521 3.44145587405364e-06
1522 3.4414201763866e-06
1523 3.44139266417187e-06
1524 3.44136356034141e-06
1525 3.4413321827742e-06
1526 3.44130103258067e-06
1527 3.4412673812767e-06
1528 3.44123623108317e-06
1529 3.44120371664758e-06
1530 3.44117506756447e-06
1531 3.44113846040273e-06
1532 3.441110948188e-06
1533 3.44108184435754e-06
1534 3.44104728355887e-06
1535 3.44101636073901e-06
1536 3.44098293680872e-06
1537 3.44095201398886e-06
1538 3.44091881743225e-06
1539 3.4408942610753e-06
1540 3.44085697179253e-06
1541 3.44084764947183e-06
1542 3.44079580827383e-06
1543 3.4407682960591e-06
1544 3.44075556313328e-06
1545 3.44070235769323e-06
1546 3.44067507285217e-06
1547 3.4406498343742e-06
1548 3.44061413670715e-06
1549 3.44059662893414e-06
1550 3.4405470614729e-06
1551 3.4405202313792e-06
1552 3.44049453815387e-06
1553 3.44045315614494e-06
1554 3.44042405231448e-06
1555 3.44040245181532e-06
1556 3.44036811839032e-06
1557 3.44033901455987e-06
1558 3.4403062727506e-06
1559 3.44028762810922e-06
1560 3.44024442711088e-06
1561 3.44020099873887e-06
1562 3.44017735187663e-06
1563 3.44016075359832e-06
1564 3.44011868946836e-06
1565 3.44009322361671e-06
1566 3.44006230079685e-06
1567 3.44001659868809e-06
1568 3.4400072763674e-06
1569 3.4399683954689e-06
1570 3.43993974638579e-06
1571 3.43989586326643e-06
1572 3.43988017448282e-06
1573 3.43984174833167e-06
1574 3.43981378136959e-06
1575 3.43981537298532e-06
1576 3.43978103956033e-06
1577 3.43974534189329e-06
1578 3.43971260008402e-06
1579 3.43966030413867e-06
1580 3.43960232385143e-06
1581 3.43953865922231e-06
1582 3.43949955095013e-06
1583 3.43946589964617e-06
1584 3.43940996572201e-06
1585 3.43933334079338e-06
1586 3.43929013979505e-06
1587 3.43924466505996e-06
1588 3.43920373779838e-06
1589 3.43917895406776e-06
1590 3.4391239296383e-06
1591 3.43907845490321e-06
1592 3.43903298016812e-06
1593 3.4389979646221e-06
1594 3.43896272170241e-06
1595 3.43893248100358e-06
1596 3.43888177667395e-06
1597 3.43884858011734e-06
1598 3.43875285579998e-06
1599 3.43873125530081e-06
1600 3.43870920005429e-06
1601 3.4386775951134e-06
1602 3.43864303431474e-06
1603 3.43860233442683e-06
1604 3.43856845574919e-06
1605 3.43853730555566e-06
1606 3.43850410899904e-06
1607 3.4384693208267e-06
1608 3.43843476002803e-06
1609 3.43840156347142e-06
1610 3.43837746186182e-06
1611 3.43833721672127e-06
1612 3.43830879501184e-06
1613 3.43826718562923e-06
1614 3.4382258036203e-06
1615 3.43819556292146e-06
1616 3.43816509484896e-06
1617 3.43813303516072e-06
1618 3.43809642799897e-06
1619 3.43807641911553e-06
1620 3.43804845215345e-06
1621 3.43800775226555e-06
1622 3.43798910762416e-06
1623 3.43794158652599e-06
1624 3.43791612067434e-06
1625 3.43789315593312e-06
1626 3.43785063705582e-06
1627 3.43781925948861e-06
1628 3.43779925060517e-06
1629 3.43774695465981e-06
1630 3.43772921951313e-06
1631 3.4376989788143e-06
1632 3.43766873811546e-06
1633 3.43764691024262e-06
1634 3.43761212207028e-06
1635 3.43759211318684e-06
1636 3.43755436915671e-06
1637 3.43753026754712e-06
1638 3.43749252351699e-06
1639 3.43747728948074e-06
1640 3.43746319231286e-06
1641 3.43742067343555e-06
1642 3.43738634001056e-06
1643 3.43736701324815e-06
1644 3.43733017871273e-06
1645 3.43729334417731e-06
1646 3.43727219842549e-06
1647 3.43724627782649e-06
1648 3.43721535500663e-06
1649 3.43718761541822e-06
1650 3.43715691997204e-06
1651 3.43712645189953e-06
1652 3.43711485584208e-06
1653 3.43707051797537e-06
1654 3.43702072314045e-06
1655 3.43699093718897e-06
1656 3.43695592164295e-06
1657 3.43692386195471e-06
1658 3.43690135196084e-06
1659 3.43686815540423e-06
1660 3.43685678672045e-06
1661 3.43682472703222e-06
1662 3.43681267622742e-06
1663 3.43676947522908e-06
1664 3.43676128977677e-06
1665 3.4367249099887e-06
1666 3.43669512403721e-06
1667 3.43666670232778e-06
1668 3.4366369163763e-06
1669 3.43660258295131e-06
1670 3.43659144164121e-06
1671 3.43654460266407e-06
1672 3.43652664014371e-06
1673 3.43652368428593e-06
1674 3.43649048772932e-06
1675 3.43648184752965e-06
1676 3.43645137945714e-06
1677 3.43642614097917e-06
1678 3.43639953825914e-06
1679 3.4363813483651e-06
1680 3.43635338140302e-06
1681 3.43633246302488e-06
1682 3.43630836141529e-06
1683 3.43628744303714e-06
1684 3.43627107213251e-06
1685 3.43624060406e-06
1686 3.4362137739663e-06
1687 3.43619353770919e-06
1688 3.43616943609959e-06
1689 3.43612805409066e-06
1690 3.43610872732825e-06
1691 3.43608621733438e-06
1692 3.43606916430872e-06
1693 3.43604233421502e-06
1694 3.43601777785807e-06
1695 3.43600140695344e-06
1696 3.43596298080229e-06
1697 3.43593251272978e-06
1698 3.43592159879336e-06
1699 3.43592091667233e-06
1700 3.43588294526853e-06
1701 3.4358545235591e-06
1702 3.43585611517483e-06
1703 3.43583292305993e-06
1704 3.43581291417649e-06
1705 3.43578858519322e-06
1706 3.43575698025234e-06
1707 3.43575584338396e-06
1708 3.43571969096956e-06
1709 3.43571014127519e-06
1710 3.43566898663994e-06
1711 3.43566057381395e-06
1712 3.43564238391991e-06
1713 3.43561396221048e-06
1714 3.43557894666446e-06
1715 3.43555848303367e-06
1716 3.43555052495503e-06
1717 3.43553210768732e-06
1718 3.43550004799908e-06
1719 3.43546048497956e-06
1720 3.43545866599015e-06
1721 3.43543069902807e-06
1722 3.43541250913404e-06
1723 3.43538681590871e-06
1724 3.43537067237776e-06
1725 3.43535248248372e-06
1726 3.435342478042e-06
1727 3.43531110047479e-06
1728 3.43528904522827e-06
1729 3.43526835422381e-06
1730 3.43529359270178e-06
1731 3.43528063240228e-06
1732 3.43525744028739e-06
1733 3.4352656257397e-06
1734 3.43525698554004e-06
1735 3.43520491696836e-06
1736 3.43517740475363e-06
1737 3.435161033849e-06
1738 3.43515807799122e-06
1739 3.43513261213957e-06
1740 3.43510760103527e-06
1741 3.43511624123494e-06
1742 3.4350860005361e-06
1743 3.4350100577285e-06
1744 3.43505212185846e-06
1745 3.43499482369225e-06
1746 3.43497663379821e-06
1747 3.43494411936263e-06
1748 3.4349416182522e-06
1749 3.4349322959315e-06
1750 3.43489955412224e-06
1751 3.43488454745966e-06
1752 3.43486135534476e-06
1753 3.43489045917522e-06
1754 3.43494184562587e-06
1755 3.43497163157735e-06
1756 3.43506326316856e-06
1757 3.43505598721094e-06
1758 3.43500528288132e-06
1759 3.43497981702967e-06
1760 3.43493934451544e-06
1761 3.43492843057902e-06
1762 3.43490046361694e-06
1763 3.43485930898169e-06
1764 3.43485112352937e-06
1765 3.43481701747805e-06
1766 3.43478927788965e-06
1767 3.43476426678535e-06
1768 3.43473629982327e-06
1769 3.43471992891864e-06
1770 3.43468559549365e-06
1771 3.43466149388405e-06
1772 3.43463875651651e-06
1773 3.43460374097049e-06
1774 3.43458123097662e-06
1775 3.43456190421421e-06
1776 3.4345350741205e-06
1777 3.43450892614783e-06
1778 3.43448436979088e-06
1779 3.4344554933341e-06
1780 3.43442002304073e-06
1781 3.43442934536142e-06
1782 3.43435999639041e-06
1783 3.43434294336475e-06
1784 3.43432384397602e-06
1785 3.43428632731957e-06
1786 3.43426768267818e-06
1787 3.43423926096875e-06
1788 3.43421925208531e-06
1789 3.43420833814889e-06
1790 3.43416922987672e-06
1791 3.4341619539191e-06
1792 3.43412011716282e-06
1793 3.43408873959561e-06
1794 3.43406804859114e-06
1795 3.43404485647625e-06
1796 3.43402916769264e-06
1797 3.43400006386219e-06
1798 3.4339702779107e-06
1799 3.43394208357495e-06
1800 3.43392798640707e-06
1801 3.43391457136022e-06
1802 3.43387387147231e-06
1803 3.43385067935742e-06
1804 3.43373335454089e-06
1805 3.43371789313096e-06
1806 3.43367673849571e-06
1807 3.43365809385432e-06
1808 3.43362921739754e-06
1809 3.43359147336741e-06
1810 3.43358146892569e-06
1811 3.43355713994242e-06
1812 3.43353599419061e-06
1813 3.4335046166234e-06
1814 3.43347414855089e-06
1815 3.43346596309857e-06
1816 3.43344595421513e-06
1817 3.43342458108964e-06
1818 3.433390702412e-06
1819 3.43338660968584e-06
1820 3.43335932484479e-06
1821 3.43332112606731e-06
1822 3.43329338647891e-06
1823 3.43327224072709e-06
1824 3.43324654750177e-06
1825 3.43322903972876e-06
1826 3.43320675710856e-06
1827 3.43318583873042e-06
1828 3.43317947226751e-06
1829 3.43315491591056e-06
1830 3.43313286066405e-06
1831 3.43310875905445e-06
1832 3.43308829542366e-06
1833 3.43307624461886e-06
1834 3.43304691341473e-06
1835 3.43302349392616e-06
1836 3.43300257554802e-06
1837 3.43298097504885e-06
1838 3.43296824212302e-06
1839 3.43294141202932e-06
1840 3.43291617355135e-06
1841 3.43283113579673e-06
1842 3.432837957007e-06
1843 3.43282158610236e-06
1844 3.43278384207224e-06
1845 3.43276633429923e-06
1846 3.43274109582126e-06
1847 3.43272085956414e-06
1848 3.43270380653848e-06
1849 3.432674020587e-06
1850 3.43264946423005e-06
1851 3.43262468049943e-06
1852 3.43260535373702e-06
1853 3.43259125656914e-06
1854 3.43255032930756e-06
1855 3.43253304890823e-06
1856 3.43251076628803e-06
1857 3.43248620993108e-06
1858 3.43245983458473e-06
1859 3.43243937095394e-06
1860 3.43241322298127e-06
1861 3.43239594258193e-06
1862 3.43237434208277e-06
1863 3.43234955835214e-06
1864 3.43231749866391e-06
1865 3.4323147701798e-06
1866 3.43228589372302e-06
1867 3.43226679433428e-06
1868 3.43225519827683e-06
1869 3.43222995979886e-06
1870 3.43221063303645e-06
1871 3.4321972179896e-06
1872 3.43216061082785e-06
1873 3.43216174769623e-06
1874 3.43213150699739e-06
1875 3.4321092243772e-06
1876 3.43207784680999e-06
1877 3.43206443176314e-06
1878 3.4320555641898e-06
1879 3.43203168995387e-06
1880 3.43194415108883e-06
1881 3.4319166388741e-06
1882 3.43188798979099e-06
1883 3.43188639817527e-06
1884 3.43185138262925e-06
1885 3.43183774020872e-06
1886 3.43174178851768e-06
1887 3.43170768246637e-06
1888 3.43156875715067e-06
1889 3.43153101312055e-06
1890 3.43143187819805e-06
1891 3.43135593539046e-06
1892 3.43128817803517e-06
1893 3.43120700563304e-06
1894 3.43113583767263e-06
1895 3.43107353728556e-06
1896 3.43102510669269e-06
1897 3.430981450947e-06
1898 3.43091642207582e-06
1899 3.43086799148296e-06
1900 3.43080682796426e-06
1901 3.43076430908695e-06
1902 3.43071587849408e-06
1903 3.43066153618565e-06
1904 3.43061765306629e-06
1905 3.43056922247342e-06
1906 3.43052511198039e-06
1907 3.43048918693967e-06
1908 3.43043438988389e-06
1909 3.43040528605343e-06
1910 3.43035685546056e-06
1911 3.43029637406289e-06
1912 3.43026886184816e-06
1913 3.4302249787288e-06
1914 3.43018314197252e-06
1915 3.43014153258991e-06
1916 3.4300935567444e-06
1917 3.43005876857205e-06
1918 3.43003580383083e-06
1919 3.43000101565849e-06
1920 3.42996486324409e-06
1921 3.42992848345602e-06
1922 3.42988369084196e-06
1923 3.42985754286929e-06
1924 3.42983298651234e-06
1925 3.42980843015539e-06
1926 3.4297647744097e-06
1927 3.42973544320557e-06
1928 3.42970270139631e-06
1929 3.4296722333238e-06
1930 3.42964085575659e-06
1931 3.42961038768408e-06
1932 3.42958128385362e-06
1933 3.42955581800197e-06
1934 3.42952603205049e-06
1935 3.42949033438344e-06
1936 3.42946304954239e-06
1937 3.42943963005382e-06
1938 3.42940847986029e-06
1939 3.42939006259257e-06
1940 3.42939711117651e-06
1941 3.42938164976658e-06
1942 3.42934959007835e-06
1943 3.42932071362156e-06
1944 3.42930229635385e-06
1945 3.42927728524955e-06
1946 3.42924658980337e-06
1947 3.42923294738284e-06
1948 3.42919975082623e-06
1949 3.42918406204262e-06
1950 3.42915996043303e-06
1951 3.42913131134992e-06
1952 3.42910129802476e-06
1953 3.42907469530473e-06
1954 3.42904854733206e-06
1955 3.42902922056965e-06
1956 3.42900966643356e-06
1957 3.42896282745642e-06
1958 3.42896441907214e-06
1959 3.42892417393159e-06
1960 3.42891144100577e-06
1961 3.42888461091206e-06
1962 3.42884368365048e-06
1963 3.42881912729354e-06
1964 3.42880252901523e-06
1965 3.4287759262952e-06
1966 3.42874750458577e-06
1967 3.42871658176591e-06
1968 3.4286954360141e-06
1969 3.42867383551493e-06
1970 3.4286481422896e-06
1971 3.42862540492206e-06
1972 3.42859175361809e-06
1973 3.42857174473465e-06
1974 3.42853377333086e-06
1975 3.42853036272572e-06
1976 3.42850989909493e-06
1977 3.42848375112226e-06
1978 3.42847488354892e-06
1979 3.42848011314345e-06
1980 3.42843372891366e-06
1981 3.42840485245688e-06
1982 3.42837938660523e-06
1983 3.42835414812726e-06
1984 3.42833959621203e-06
1985 3.42830867339217e-06
1986 3.42828252541949e-06
1987 3.4282620617887e-06
1988 3.42823273058457e-06
1989 3.42821181220643e-06
1990 3.42818134413392e-06
1991 3.42816224474518e-06
1992 3.42814473697217e-06
1993 3.42812086273625e-06
1994 3.42810631082102e-06
1995 3.42807538800116e-06
1996 3.42805014952319e-06
1997 3.4280203635717e-06
1998 3.42798307428893e-06
1999 3.42794146490633e-06
};
\addlegendentry{Test}
\end{groupplot}

\end{tikzpicture}

		\caption{Five experiments over the different depth with $\hy$ left and $\rare$ right. The number of layers used for every experiment are given. Training and validation loss are shown over 2000 epochs.}
		\label{Fig:Depth}
	\end{figure}
\end{center}
\begin{center}
	\begin{figure}[htbp!]
		% This file was created by tikzplotlib v0.9.6.
\begin{tikzpicture}

\begin{groupplot}[group style={group size=1 by 5},height=.25\textwidth,width=.45\textwidth]
\nextgroupplot[
legend cell align={left},
legend style={fill opacity=0.8, draw opacity=1, text opacity=1, at={(0.91,0.5)}, anchor=east, draw=white!80!black},
log basis y={10},
tick align=outside,
tick pos=left,
title={50 Nodes},
x grid style={white!69.0196078431373!black},
xlabel={Epoch},
xmin=-2.45, xmax=51.45,
xtick style={color=black},
y grid style={white!69.0196078431373!black},
ylabel={MSE Loss},
ymin=1.9888885065468e-08, ymax=1e-05,
ymode=log,
ytick style={color=black},
]
\addplot [semithick, black, dashed]
table {%
0 0.0135999593026936
1 4.2896431104964e-06
2 5.3881981935433e-07
3 3.51886634504694e-07
4 2.82757554970203e-07
5 2.6716137870153e-07
6 2.61151586933295e-07
7 2.57311578558017e-07
8 2.53943267082946e-07
9 2.50290613514892e-07
10 2.45850220892407e-07
11 2.41486959048132e-07
12 2.36548477104748e-07
13 2.30781724837925e-07
14 2.22870224526162e-07
15 2.10778227668129e-07
16 1.90057973632918e-07
17 1.52398310959256e-07
18 1.22657644446633e-07
19 1.02636614442986e-07
20 8.56009278216163e-08
21 7.98480282959702e-08
22 7.58233048685497e-08
23 7.29153675891325e-08
24 7.04776545621399e-08
25 6.77215497066896e-08
26 6.5767227466651e-08
27 6.40355770649137e-08
28 6.23953467453475e-08
29 6.0793110801427e-08
30 5.93609453307664e-08
31 5.77782414890038e-08
32 5.64022187035107e-08
33 5.52282547960914e-08
34 5.39651254687357e-08
35 5.17190636681875e-08
36 5.01611363059595e-08
37 4.8596921590871e-08
38 4.727549480954e-08
39 4.61027602014497e-08
40 4.50549624755325e-08
41 4.41297025588483e-08
42 4.33818541978326e-08
43 4.24839656538722e-08
44 4.14374520651961e-08
45 4.05148022650792e-08
46 3.96116335821262e-08
47 3.84950632543735e-08
48 3.79317882224228e-08
49 3.77106582192255e-08
};
\addlegendentry{Train}
\addplot [semithick, black]
table {%
0 0.00487672025337815
1 3.96470886698808e-06
2 5.73329430153535e-07
3 3.53363446947697e-07
4 2.83720822835676e-07
5 2.66964633510725e-07
6 2.62910276660477e-07
7 2.58003979070054e-07
8 2.53354954793394e-07
9 2.4998857384162e-07
10 2.4655307129251e-07
11 2.41297016145836e-07
12 2.38922552853182e-07
13 2.33224852763669e-07
14 2.26032128125553e-07
15 2.14270428955388e-07
16 1.96679422970192e-07
17 1.64139194680502e-07
18 1.23222406500645e-07
19 9.87034667332409e-08
20 8.52792965133631e-08
21 8.20290964043124e-08
22 7.83909257506821e-08
23 7.70745813838403e-08
24 7.47037915971305e-08
25 7.24093496273781e-08
26 7.04010219010343e-08
27 6.81867788898671e-08
28 6.61188153117109e-08
29 6.33973300523394e-08
30 6.12734325500242e-08
31 5.90240176734369e-08
32 5.74804204234169e-08
33 5.60701707286171e-08
34 5.44361888898948e-08
35 4.94028782327405e-08
36 4.69148204729208e-08
37 4.5969155593184e-08
38 4.46059615910599e-08
39 4.37680469644874e-08
40 4.30825508601629e-08
41 4.24594865933159e-08
42 4.20454995264663e-08
43 4.17319654388848e-08
44 4.1080859602971e-08
45 4.06809768094263e-08
46 4.02437265734079e-08
47 3.99750383905939e-08
48 3.97745623104129e-08
49 3.97504145155381e-08
};
\addlegendentry{Test}

\nextgroupplot[
legend cell align={left},
legend style={fill opacity=0.8, draw opacity=1, text opacity=1, at={(0.91,0.5)}, anchor=east, draw=white!80!black},
log basis y={10},
tick align=outside,
tick pos=left,
title={40 Nodes},
x grid style={white!69.0196078431373!black},
xlabel={Epoch},
xmin=-2.45, xmax=51.45,
xtick style={color=black},
y grid style={white!69.0196078431373!black},
ylabel={MSE Loss},
ymin=1.1663983329682e-08, ymax=1e-05,
ymode=log,
ytick style={color=black}
]
\addplot [semithick, black, dashed]
table {%
0 0.0125584448166192
1 8.39119358761309e-06
2 6.41755197790417e-07
3 5.13980546060111e-07
4 4.69025601248063e-07
5 4.39968163938431e-07
6 4.08462546225508e-07
7 3.66736147768165e-07
8 3.1429498905311e-07
9 2.65593225606153e-07
10 2.36086500912336e-07
11 2.15646079219312e-07
12 1.95410477296321e-07
13 1.72950173663367e-07
14 1.47595372588682e-07
15 1.20782248593798e-07
16 9.68819428592838e-08
17 7.91936696131756e-08
18 6.69768956917949e-08
19 5.94595781606699e-08
20 5.43975741109648e-08
21 5.00402651510967e-08
22 4.66497759887829e-08
23 4.38425063187253e-08
24 4.16350062302229e-08
25 3.98629581788157e-08
26 3.83089183255692e-08
27 3.7032645277435e-08
28 3.59446300794986e-08
29 3.49807069195407e-08
30 3.40846099486924e-08
31 3.32054518477065e-08
32 3.23982269900824e-08
33 3.14881593030947e-08
34 3.08296375628458e-08
35 3.02032561627641e-08
36 2.97007302609842e-08
37 2.91900015927382e-08
38 2.87266109992856e-08
39 2.82643468416666e-08
40 2.78522305343643e-08
41 2.74558967383598e-08
42 2.70493051601761e-08
43 2.66321045447881e-08
44 2.63359815946274e-08
45 2.54246853277351e-08
46 2.48360458847685e-08
47 2.44981202008887e-08
48 2.41806697998115e-08
49 2.39571260323146e-08
};
\addlegendentry{Train}
\addplot [semithick, black]
table {%
0 0.00513229425996542
1 8.07414198789047e-06
2 7.24437256849342e-07
3 5.85084194426599e-07
4 5.40613200428197e-07
5 5.09666108428064e-07
6 4.74801765903976e-07
7 4.28708460731286e-07
8 3.6723892549162e-07
9 3.1078030815479e-07
10 2.70513851319265e-07
11 2.43417133560797e-07
12 2.165828334455e-07
13 1.88775615583836e-07
14 1.57114499188538e-07
15 1.26649737808293e-07
16 1.01935178520307e-07
17 8.35827762557528e-08
18 7.18132682209216e-08
19 6.31637107062488e-08
20 5.86344981456932e-08
21 5.53894317079084e-08
22 5.12468858460124e-08
23 4.64575045100446e-08
24 4.24647801366973e-08
25 4.01569195673801e-08
26 3.80347060513486e-08
27 3.63309169415515e-08
28 3.47720465754264e-08
29 3.31292113742165e-08
30 3.21541797632108e-08
31 3.06859959664507e-08
32 2.95508453262983e-08
33 2.8657513695407e-08
34 2.79865055574646e-08
35 2.7307768490914e-08
36 2.67458197811266e-08
37 2.61310955096405e-08
38 2.57605599074395e-08
39 2.54278766931293e-08
40 2.51365417369698e-08
41 2.43617570561128e-08
42 2.42979272258026e-08
43 2.36703741052224e-08
44 2.33637251767504e-08
45 2.31886421175886e-08
46 2.29908945215129e-08
47 2.26493206412215e-08
48 2.2599000004675e-08
49 2.27724932244655e-08
};
\addlegendentry{Test}

\nextgroupplot[
legend cell align={left},
legend style={fill opacity=0.8, draw opacity=1, text opacity=1, at={(0.91,0.5)}, anchor=east, draw=white!80!black},
log basis y={10},
tick align=outside,
tick pos=left,
title={30 Nodes},
x grid style={white!69.0196078431373!black},
xlabel={Epoch},
xmin=-2.45, xmax=51.45,
xtick style={color=black},
y grid style={white!69.0196078431373!black},
ylabel={MSE Loss},
ymin=4.86483168894878e-08, ymax=1e-05,
ymode=log,
ytick style={color=black}
]
\addplot [semithick, black, dashed]
table {%
0 0.025478036031127
1 1.33583519063905e-05
2 9.55964907092266e-07
3 6.14402871562447e-07
4 5.3983266350599e-07
5 4.97156317905478e-07
6 4.67610916629724e-07
7 4.45276465711686e-07
8 4.27980220678137e-07
9 4.12621700874638e-07
10 3.92474687529898e-07
11 3.74473419171295e-07
12 3.53108027582039e-07
13 3.30294354583316e-07
14 3.03527997459696e-07
15 2.63646417621999e-07
16 2.35456707798676e-07
17 2.12933962394857e-07
18 1.92871126223793e-07
19 1.7963987661318e-07
20 1.66413915287933e-07
21 1.56310051750097e-07
22 1.48926974738117e-07
23 1.43480070832425e-07
24 1.38255350393024e-07
25 1.34502120054947e-07
26 1.31207857194227e-07
27 1.28374828044286e-07
28 1.26024475612496e-07
29 1.23833181682187e-07
30 1.21799470690576e-07
31 1.19884085151511e-07
32 1.18102275131093e-07
33 1.16358609567158e-07
34 1.14627515984012e-07
35 1.1290011470777e-07
36 1.11367041853327e-07
37 1.09768638267838e-07
38 1.08368808731996e-07
39 1.0682430382758e-07
40 1.05212458102244e-07
41 1.03748117467717e-07
42 1.02276644444288e-07
43 1.00761814696426e-07
44 9.92868664866364e-08
45 9.77629893057497e-08
46 9.61213030521435e-08
47 9.44933395992109e-08
48 9.27407635060717e-08
49 9.10763936339265e-08
};
\addlegendentry{Train}
\addplot [semithick, black]
table {%
0 0.0111488904803991
1 1.36065009428421e-05
2 1.0181311154156e-06
3 6.48558341254102e-07
4 5.67693007269554e-07
5 5.22431321314798e-07
6 4.90439845179935e-07
7 4.67743547005739e-07
8 4.48024081833864e-07
9 4.32382165627132e-07
10 4.06875926728389e-07
11 3.89341465734105e-07
12 3.69245270803731e-07
13 3.48364039837179e-07
14 3.26868018873938e-07
15 2.89926134655616e-07
16 2.58585288293034e-07
17 2.26261235525271e-07
18 2.02648479330492e-07
19 1.89335096933974e-07
20 1.83924967700477e-07
21 1.78500542347138e-07
22 1.73562696659246e-07
23 1.68111029097417e-07
24 1.635963968738e-07
25 1.61124305009253e-07
26 1.58203874889296e-07
27 1.56054895228408e-07
28 1.53470040231696e-07
29 1.51372347545475e-07
30 1.49231667023741e-07
31 1.47144632478557e-07
32 1.46412460821921e-07
33 1.45509275739641e-07
34 1.42395990110344e-07
35 1.40358977773758e-07
36 1.38469175681166e-07
37 1.37209667627758e-07
38 1.35216296826002e-07
39 1.33807731117486e-07
40 1.30511537577149e-07
41 1.28476713712189e-07
42 1.28295809531664e-07
43 1.26042507986313e-07
44 1.23777013527615e-07
45 1.20058729180528e-07
46 1.17150271705668e-07
47 1.14646404369978e-07
48 1.12263997209539e-07
49 1.09723373498127e-07
};
\addlegendentry{Test}

\nextgroupplot[
legend cell align={left},
legend style={fill opacity=0.8, draw opacity=1, text opacity=1, at={(0.91,0.5)}, anchor=east, draw=white!80!black},
log basis y={10},
tick align=outside,
tick pos=left,
title={20 Nodes},
x grid style={white!69.0196078431373!black},
xlabel={Epoch},
xmin=-2.45, xmax=51.45,
xtick style={color=black},
y grid style={white!69.0196078431373!black},
ylabel={MSE Loss},
ymin=2.66527487591202e-08, ymax=1e-05,
ymode=log,
ytick style={color=black}
]
\addplot [semithick, black, dashed]
table {%
0 0.0235187841430306
1 1.37682099211816e-05
2 2.70340157902638e-06
3 1.13740216755787e-06
4 6.90035947648937e-07
5 5.7164704746171e-07
6 5.22602940083772e-07
7 4.88889048810392e-07
8 4.6290175908581e-07
9 4.41759331557989e-07
10 4.25426490103575e-07
11 4.1301768712998e-07
12 4.0250036734335e-07
13 3.92775480705154e-07
14 3.82550404516735e-07
15 3.721292500245e-07
16 3.60453047676401e-07
17 3.46861678309551e-07
18 3.32693124413197e-07
19 3.17038284180171e-07
20 2.99811278939188e-07
21 2.82473951106255e-07
22 2.63471540296933e-07
23 2.44260742356062e-07
24 2.08983260549189e-07
25 1.55634577676977e-07
26 1.04804999178754e-07
27 9.27326790645111e-08
28 8.8356485374419e-08
29 8.53180843805035e-08
30 8.24971643709205e-08
31 7.9812434343296e-08
32 7.75735438836023e-08
33 7.54435929728459e-08
34 7.33799041707783e-08
35 7.12709132972833e-08
36 6.93650725196448e-08
37 6.75256942201941e-08
38 6.57833202950542e-08
39 6.4136311159757e-08
40 6.24605962808999e-08
41 6.09519960814708e-08
42 5.95092423880672e-08
43 5.80006523449583e-08
44 5.66446524725706e-08
45 5.53879482456665e-08
46 5.42367885945083e-08
47 5.31399612402339e-08
48 5.2111131786603e-08
49 5.11527850939331e-08
};
\addlegendentry{Train}
\addplot [semithick, black]
table {%
0 0.0174245107918978
1 1.36661974465824e-05
2 2.73788396043528e-06
3 1.16194087240729e-06
4 7.2836280651245e-07
5 6.17571231487091e-07
6 5.67888776004111e-07
7 5.3412082934301e-07
8 5.03541457419487e-07
9 4.79125390029367e-07
10 4.60253005485356e-07
11 4.45415736294308e-07
12 4.32549143170036e-07
13 4.20915426957436e-07
14 4.09082218766343e-07
15 3.98191616568511e-07
16 3.85969826766086e-07
17 3.71930923392938e-07
18 3.57995844524339e-07
19 3.43444867212384e-07
20 3.27713280512398e-07
21 3.09689283994885e-07
22 2.86694984197311e-07
23 2.60396120665973e-07
24 2.25773661099993e-07
25 1.66189394690264e-07
26 1.15193458327667e-07
27 9.86466304198075e-08
28 9.32770802819505e-08
29 8.98784620062543e-08
30 8.72306031851622e-08
31 8.46339176518995e-08
32 8.23477108724546e-08
33 7.98600154894302e-08
34 7.76310571382055e-08
35 7.50397930460167e-08
36 7.28227291801886e-08
37 7.09089249539829e-08
38 6.90276067416562e-08
39 6.68584618779278e-08
40 6.49001705710361e-08
41 6.34140562283392e-08
42 6.14814368304906e-08
43 5.98219926928323e-08
44 5.86763597709705e-08
45 5.75104017741523e-08
46 5.63018609511801e-08
47 5.53297212491088e-08
48 5.40805764615016e-08
49 5.31587858176863e-08
};
\addlegendentry{Test}

\nextgroupplot[
legend cell align={left},
legend style={fill opacity=0.8, draw opacity=1, text opacity=1, at={(0.91,0.5)}, anchor=east, draw=white!80!black},
log basis y={10},
tick align=outside,
tick pos=left,
title={10 Nodes},
x grid style={white!69.0196078431373!black},
xlabel={Epoch},
xmin=-2.45, xmax=51.45,
xtick style={color=black},
y grid style={white!69.0196078431373!black},
ylabel={MSE Loss},
ymin=1.4369403302285e-07, ymax=1e-05,
ymode=log,
ytick style={color=black}
]
\addplot [semithick, black, dashed]
table {%
0 0.033769318677485
1 9.92998830442957e-05
2 1.32467232615454e-05
3 7.58530953362424e-06
4 2.46014565448149e-06
5 1.86969139969051e-06
6 1.30335024147143e-06
7 7.07897941126134e-07
8 4.04523505864063e-07
9 3.15800119345511e-07
10 3.00031082105079e-07
11 2.94931752094385e-07
12 2.92238582211723e-07
13 2.90281306689622e-07
14 2.88452885961021e-07
15 2.86939390491625e-07
16 2.85604886840929e-07
17 2.8431849226962e-07
18 2.83125255826633e-07
19 2.81959824974365e-07
20 2.80855088142573e-07
21 2.7980720599885e-07
22 2.78810039219479e-07
23 2.77773855465568e-07
24 2.76804071205561e-07
25 2.75706285215449e-07
26 2.74428176204822e-07
27 2.72686514882992e-07
28 2.71712300317972e-07
29 2.70899042128292e-07
30 2.70156286173773e-07
31 2.69432236478906e-07
32 2.6873054738985e-07
33 2.68040218841747e-07
34 2.67384429797346e-07
35 2.6674044515218e-07
36 2.66098085234034e-07
37 2.65489013290932e-07
38 2.6489454419476e-07
39 2.64306269645687e-07
40 2.63693070515103e-07
41 2.63215628557134e-07
42 2.62606924028574e-07
43 2.62080116861796e-07
44 2.61588543843061e-07
45 2.61082553173253e-07
46 2.604654106797e-07
47 2.59949038184715e-07
48 2.59455454237667e-07
49 2.58943264107359e-07
};
\addlegendentry{Train}
\addplot [semithick, black]
table {%
0 0.0240651033818722
1 0.00010423400090076
2 1.37047718453687e-05
3 7.8467983257724e-06
4 2.53449934461969e-06
5 1.9290473574074e-06
6 1.35212405893981e-06
7 7.56075337449147e-07
8 4.59982828715511e-07
9 3.77869014300813e-07
10 3.62616987104047e-07
11 3.56840700987959e-07
12 3.53545686948564e-07
13 3.5120021379953e-07
14 3.49296897184104e-07
15 3.47380705534306e-07
16 3.45747224628212e-07
17 3.44153249898227e-07
18 3.42598326596999e-07
19 3.41011855198303e-07
20 3.39508687829948e-07
21 3.38228176133271e-07
22 3.36865809913434e-07
23 3.35403768758624e-07
24 3.33961168053065e-07
25 3.32634272126597e-07
26 3.30745962173751e-07
27 3.28328866316951e-07
28 3.27668317368079e-07
29 3.26464288491479e-07
30 3.2535004379497e-07
31 3.24119071137829e-07
32 3.22868373814345e-07
33 3.21547645398823e-07
34 3.20367405493016e-07
35 3.19376368906887e-07
36 3.1828395208322e-07
37 3.17203472377514e-07
38 3.16154853408079e-07
39 3.15131472916619e-07
40 3.14136741508264e-07
41 3.13256833806008e-07
42 3.12196732465964e-07
43 3.11253160134584e-07
44 3.10280256599071e-07
45 3.09361155359511e-07
46 3.08444498386962e-07
47 3.07571752955482e-07
48 3.0656690341857e-07
49 3.05711381543006e-07
};
\addlegendentry{Test}
\end{groupplot}

\end{tikzpicture}

		% This file was created by tikzplotlib v0.9.6.
\begin{tikzpicture}

\begin{groupplot}[group style={group size=1 by 5},
height=.25\textwidth,
width=.45\textwidth,
legend cell align={left},
legend style={draw=white}
]
\nextgroupplot[
log basis y={10},
tick align=outside,
tick pos=left,
title={50 Nodes},
x grid style={white!69.0196078431373!black},
xlabel={Epoch},
xmin=-4.95, xmax=103.95,
xtick style={color=black},
y grid style={white!69.0196078431373!black},
ylabel={MSE Loss},
ymin=2.10874907053618e-08, ymax=1e-05,
ymode=log,
ytick style={color=black}
]
\addplot [semithick, black, dashed]
table {%
0 0.0135610852343962
1 4.27796535393554e-06
2 1.00142231940481e-06
3 6.29764575592162e-07
4 5.23626780548625e-07
5 4.59680069113233e-07
6 4.13698390758555e-07
7 3.68188925648383e-07
8 3.31994941511482e-07
9 3.00197366641441e-07
10 2.73105488318492e-07
11 2.51274823128256e-07
12 2.33952035323171e-07
13 2.19604983698218e-07
14 2.07005901785351e-07
15 1.95602838211073e-07
16 1.85835880664342e-07
17 1.77288367886774e-07
18 1.69392477069152e-07
19 1.62032798648681e-07
20 1.55212995956333e-07
21 1.49935020502312e-07
22 1.45608609841474e-07
23 1.42028429927166e-07
24 1.38940652277597e-07
25 1.36263131871317e-07
26 1.33630953015995e-07
27 1.31107507712613e-07
28 1.28578016123981e-07
29 1.25867618230302e-07
30 1.22495237352638e-07
31 1.20438302648296e-07
32 1.18318069439738e-07
33 1.16304365960218e-07
34 1.14303555477591e-07
35 1.12075342631357e-07
36 1.09155097497649e-07
37 1.06208147808928e-07
38 1.03539121802498e-07
39 1.01148381872918e-07
40 9.87960232379237e-08
41 9.6499766829794e-08
42 9.42918572164331e-08
43 9.21144306591515e-08
44 9.00940896713109e-08
45 8.81827021697035e-08
46 8.64063908050383e-08
47 8.47100569068005e-08
48 8.31553960871645e-08
49 8.17049510217771e-08
50 8.03165578169285e-08
51 7.90115650985967e-08
52 7.77285172084419e-08
53 7.63843575484202e-08
54 7.50669483338129e-08
55 7.38411144673989e-08
56 7.27241773113008e-08
57 7.16978492096843e-08
58 7.07161216428886e-08
59 6.97357344510152e-08
60 6.88130023984712e-08
61 6.78812821188757e-08
62 6.70483267519728e-08
63 6.62559898376713e-08
64 6.5441946796696e-08
65 6.4718245059936e-08
66 6.38478567953627e-08
67 6.30283434830403e-08
68 6.22086865735838e-08
69 6.14161420795511e-08
70 6.06298544187922e-08
71 5.98259236639365e-08
72 5.90788094534389e-08
73 5.83082105762855e-08
74 5.7570235469484e-08
75 5.67572758320978e-08
76 5.59630042893389e-08
77 5.52207383694991e-08
78 5.44524897918564e-08
79 5.37187348967905e-08
80 5.29312348369615e-08
81 5.2183072696721e-08
82 5.13874467902298e-08
83 5.06841408736136e-08
84 5.00412507040693e-08
85 4.92149066637637e-08
86 4.84792270238898e-08
87 4.77404890251165e-08
88 4.70815780868605e-08
89 4.64262167465534e-08
90 4.57196036407481e-08
91 4.50683804729124e-08
92 4.44007558044746e-08
93 4.36341568583742e-08
94 4.28967212009468e-08
95 4.22500890238098e-08
96 4.16709865653786e-08
97 4.09333017223901e-08
98 4.0368053442208e-08
99 3.98665971879097e-08
};
\addlegendentry{Train}
\addplot [semithick, black]
table {%
0 0.00586127676069736
1 4.09676795243286e-06
2 9.0397287522137e-07
3 5.72559429201647e-07
4 4.75174260827771e-07
5 4.21360510927116e-07
6 3.84699177402581e-07
7 3.46316824106907e-07
8 3.20507496098799e-07
9 2.93328014322469e-07
10 2.73369010983515e-07
11 2.55974100582534e-07
12 2.40245327631783e-07
13 2.24633623702175e-07
14 2.10526209798445e-07
15 1.99566983383193e-07
16 1.87955564001641e-07
17 1.803224449759e-07
18 1.70545206401584e-07
19 1.64018217674311e-07
20 1.57999650696183e-07
21 1.53794857737921e-07
22 1.50695328215988e-07
23 1.48003707067801e-07
24 1.44651338018775e-07
25 1.41613483606307e-07
26 1.39136488996883e-07
27 1.3680362087598e-07
28 1.33496584453496e-07
29 1.3109004726175e-07
30 1.28094356455222e-07
31 1.26978491721275e-07
32 1.24501710274671e-07
33 1.22747096042986e-07
34 1.20568500960871e-07
35 1.185906057799e-07
36 1.15871344519292e-07
37 1.11925061219154e-07
38 1.09209850052139e-07
39 1.07030075469083e-07
40 1.05133345584818e-07
41 1.0333705802168e-07
42 1.01856990397664e-07
43 1.00196956509535e-07
44 9.89866748568602e-08
45 9.78960130737505e-08
46 9.69638591641342e-08
47 9.52483460991971e-08
48 9.44195619467791e-08
49 9.3444413096222e-08
50 9.2409976559793e-08
51 9.1363915544207e-08
52 8.99799132980661e-08
53 8.92917810801919e-08
54 8.85045707832433e-08
55 8.76263612781258e-08
56 8.65752696199706e-08
57 8.5787085879474e-08
58 8.48655830054668e-08
59 8.40280449665443e-08
60 8.32084410262723e-08
61 8.23228347712757e-08
62 8.13359903872879e-08
63 8.06001310138527e-08
64 7.97839234678577e-08
65 7.87436746918502e-08
66 7.82185622938414e-08
67 7.79408964035611e-08
68 7.78167219550596e-08
69 7.7573325540925e-08
70 7.65115828471608e-08
71 7.62559722033984e-08
72 7.46562278663987e-08
73 7.39924246317969e-08
74 7.29371052443639e-08
75 7.21593025332368e-08
76 7.09221481542954e-08
77 6.96617021844759e-08
78 6.84941525719296e-08
79 6.73247200211335e-08
80 6.63008066226212e-08
81 6.47553903831977e-08
82 6.38363388816288e-08
83 6.50027658366525e-08
84 6.21795734900843e-08
85 6.09674799534332e-08
86 5.98512102101267e-08
87 5.88380437704927e-08
88 5.79519117138716e-08
89 5.73894567423849e-08
90 5.70997222837377e-08
91 5.71500393675706e-08
92 5.69619160728507e-08
93 5.56799086837145e-08
94 5.57584485250118e-08
95 5.45296003906515e-08
96 5.50675771648912e-08
97 5.43845679601418e-08
98 5.29509769364722e-08
99 5.15905540510175e-08
};
\addlegendentry{Test}

\nextgroupplot[
legend cell align={left},
legend style={fill opacity=0.8, draw opacity=1, text opacity=1, at={(0.91,0.5)}, anchor=east, draw=white!80!black},
log basis y={10},
tick align=outside,
tick pos=left,
title={40 Nodes},
x grid style={white!69.0196078431373!black},
xlabel={Epoch},
xmin=-4.95, xmax=103.95,
xtick style={color=black},
y grid style={white!69.0196078431373!black},
ylabel={MSE Loss},
ymin=8.21582103668331e-09, ymax=1e-05,
ymode=log,
ytick style={color=black}
]
\addplot [semithick, black, dashed]
table {%
0 0.012309129556641
1 7.99112632694232e-06
2 1.32267544427123e-06
3 8.66152799147812e-07
4 6.43708132798793e-07
5 5.10550929604392e-07
6 4.26813268688875e-07
7 3.73201204340035e-07
8 3.35604560504521e-07
9 3.09558686765854e-07
10 2.88766231889781e-07
11 2.71182032982153e-07
12 2.55632179687382e-07
13 2.41478173059306e-07
14 2.30129777648358e-07
15 2.19535546740701e-07
16 2.1004228734256e-07
17 2.00824852001347e-07
18 1.92493961549189e-07
19 1.84638804640258e-07
20 1.76982908733692e-07
21 1.69516086160115e-07
22 1.62355412385296e-07
23 1.55406151613136e-07
24 1.48683428996321e-07
25 1.42245696402199e-07
26 1.35974309849018e-07
27 1.29865598111678e-07
28 1.238648168993e-07
29 1.18140272867606e-07
30 1.12734074502185e-07
31 1.0763744637643e-07
32 1.02853349591925e-07
33 9.82504521154226e-08
34 9.39559621997432e-08
35 8.98512975702204e-08
36 8.59194344791092e-08
37 8.2084769093882e-08
38 7.83635377707981e-08
39 7.54262949911322e-08
40 7.24981481248221e-08
41 7.00154384993823e-08
42 6.7697134483069e-08
43 6.55393562638551e-08
44 6.35341096604236e-08
45 6.1762916423902e-08
46 5.99629104733879e-08
47 5.80082504235691e-08
48 5.61290219636135e-08
49 5.46393292069069e-08
50 5.32959373202857e-08
51 5.18591296483351e-08
52 5.05115447566595e-08
53 4.92823492308503e-08
54 4.80821300428147e-08
55 4.69476763598209e-08
56 4.58728684797904e-08
57 4.48464896116718e-08
58 4.37934589658795e-08
59 4.29126832592885e-08
60 4.17722703360823e-08
61 4.08673991554309e-08
62 3.93693300093645e-08
63 3.8306875172367e-08
64 3.73168313103633e-08
65 3.62723064668558e-08
66 3.53194267113111e-08
67 3.44789038919657e-08
68 3.36885018015209e-08
69 3.28821846924399e-08
70 3.20236276660779e-08
71 3.12318972017778e-08
72 3.05611810702544e-08
73 2.97143615402007e-08
74 2.89541449305375e-08
75 2.81380678046617e-08
76 2.75491371581893e-08
77 2.680068530303e-08
78 2.6151542282804e-08
79 2.55712042296352e-08
80 2.49679921022761e-08
81 2.43428253092759e-08
82 2.37832156670947e-08
83 2.31548184306263e-08
84 2.25748041202678e-08
85 2.20201048115598e-08
86 2.15417261717477e-08
87 2.0964643994148e-08
88 2.04896032585111e-08
89 2.00404295558343e-08
90 1.95668065003929e-08
91 1.91361253030209e-08
92 1.86852826473682e-08
93 1.82015934102608e-08
94 1.78618102140149e-08
95 1.7452759413672e-08
96 1.70803473240611e-08
97 1.67689061330023e-08
98 1.64616247069205e-08
99 1.61705963312997e-08
};
\addlegendentry{Train}
\addplot [semithick, black]
table {%
0 0.00526050478219986
1 7.97675966168754e-06
2 1.23362929116411e-06
3 8.17785974049912e-07
4 6.32081309959176e-07
5 5.11662051394524e-07
6 4.39763852000397e-07
7 3.90031118513434e-07
8 3.48167674246724e-07
9 3.19157521744273e-07
10 2.96828005730276e-07
11 2.78578596635271e-07
12 2.65951570099787e-07
13 2.52831029001754e-07
14 2.43252600284904e-07
15 2.32202694405714e-07
16 2.23209383420908e-07
17 2.16121847529394e-07
18 2.08023280379166e-07
19 2.00261752070219e-07
20 1.93701069406416e-07
21 1.86955247727383e-07
22 1.79593399707301e-07
23 1.72946087673154e-07
24 1.65462751056111e-07
25 1.58343652856274e-07
26 1.50983353819356e-07
27 1.4330112207972e-07
28 1.36678323769956e-07
29 1.30435424239295e-07
30 1.24558994230028e-07
31 1.19041160928646e-07
32 1.14096096126559e-07
33 1.0971750441513e-07
34 1.05268647132561e-07
35 1.006605288012e-07
36 9.61532009569055e-08
37 9.14338826873973e-08
38 8.6691237299874e-08
39 8.22555392687718e-08
40 7.75123183416326e-08
41 7.45306252269984e-08
42 7.22547497389314e-08
43 7.07018159573636e-08
44 6.80019098808771e-08
45 6.72549091973451e-08
46 6.58686687415866e-08
47 6.47085514060564e-08
48 6.3003028571984e-08
49 6.17911624090084e-08
50 6.0593229989081e-08
51 5.89141855300568e-08
52 5.79470054162812e-08
53 5.56474297752629e-08
54 5.37276356737948e-08
55 5.23844363442549e-08
56 5.14426155007186e-08
57 5.01659584983827e-08
58 4.92068679136537e-08
59 4.80536925806518e-08
60 4.67849723406744e-08
61 4.58896920463303e-08
62 4.35416325217375e-08
63 4.25309210072555e-08
64 4.15047907154076e-08
65 3.9888188752002e-08
66 3.82865366077567e-08
67 3.73067017278572e-08
68 3.61960879047274e-08
69 3.47727890925853e-08
70 3.42698278643638e-08
71 3.34728191830891e-08
72 3.31183471757868e-08
73 3.2614991596347e-08
74 3.21404982628337e-08
75 3.0868982037191e-08
76 2.99613454046721e-08
77 2.98178015611938e-08
78 2.84030470254493e-08
79 2.75213150047193e-08
80 2.67612509929904e-08
81 2.67075606075196e-08
82 2.62034518527798e-08
83 2.57414463078476e-08
84 2.47883829018747e-08
85 2.49897276205502e-08
86 2.43824338497234e-08
87 2.29132464113491e-08
88 2.2588823256342e-08
89 2.21800924293802e-08
90 2.18397211426691e-08
91 2.10919495202688e-08
92 2.03095051887203e-08
93 1.99401171130376e-08
94 1.94134699427195e-08
95 1.8719036987136e-08
96 1.82152160022042e-08
97 1.77198344886165e-08
98 1.73021366123294e-08
99 1.68753970797297e-08
};
\addlegendentry{Test}

\nextgroupplot[
legend cell align={left},
legend style={fill opacity=0.8, draw opacity=1, text opacity=1, at={(0.91,0.5)}, anchor=east, draw=white!80!black},
log basis y={10},
tick align=outside,
tick pos=left,
title={30 Nodes},
x grid style={white!69.0196078431373!black},
xlabel={Epoch},
xmin=-4.95, xmax=103.95,
xtick style={color=black},
y grid style={white!69.0196078431373!black},
ylabel={MSE Loss},
ymin=1.75486322327506e-08, ymax=1e-05,
ymode=log,
ytick style={color=black}
]
\addplot [semithick, black, dashed]
table {%
0 0.0159486631359905
1 8.99847916230101e-06
2 2.00791585211846e-06
3 7.79392788928135e-07
4 5.02840871234866e-07
5 3.73513665053338e-07
6 3.04031239437563e-07
7 2.63506940861191e-07
8 2.36038233538238e-07
9 2.12704392758667e-07
10 1.94037957335524e-07
11 1.77845113519481e-07
12 1.63788078360483e-07
13 1.51564501393864e-07
14 1.41036687644203e-07
15 1.3246952826762e-07
16 1.25596368064862e-07
17 1.19833880035003e-07
18 1.14740991875806e-07
19 1.10143833708776e-07
20 1.06167327302842e-07
21 1.02186003346105e-07
22 9.85647406679391e-08
23 9.51199502061684e-08
24 9.19961535821301e-08
25 8.88095804612021e-08
26 8.62161738481859e-08
27 8.38963989657771e-08
28 8.1848308152388e-08
29 8.00610757671905e-08
30 7.84452752142783e-08
31 7.71416911611311e-08
32 7.55562698984136e-08
33 7.42788647514203e-08
34 7.31286248480956e-08
35 7.21007905362114e-08
36 7.11782180609077e-08
37 7.01337673767455e-08
38 6.93299174230333e-08
39 6.87223202149312e-08
40 6.80287435592675e-08
41 6.73656855916249e-08
42 6.6813550510858e-08
43 6.61568773239196e-08
44 6.56730944683659e-08
45 6.49468777531581e-08
46 6.46068533640687e-08
47 6.39020621591158e-08
48 6.33692149527576e-08
49 6.29948371475564e-08
50 6.2174544126492e-08
51 6.15413521671826e-08
52 6.13322881299894e-08
53 6.07372632970282e-08
54 6.00839160362909e-08
55 5.93305634808416e-08
56 5.88520379309898e-08
57 5.81828700951803e-08
58 5.77420875345069e-08
59 5.71180894333168e-08
60 5.66919391289389e-08
61 5.62335650755585e-08
62 5.57414699287051e-08
63 5.52198706706974e-08
64 5.47791515081997e-08
65 5.41115315826346e-08
66 5.35465943798386e-08
67 5.29151716150977e-08
68 5.26910889426802e-08
69 5.18713512818181e-08
70 5.1354522710767e-08
71 5.08900036315651e-08
72 5.03894484182865e-08
73 4.99758594116884e-08
74 4.86703328022031e-08
75 4.84829759965777e-08
76 4.72148046881671e-08
77 4.65767424255858e-08
78 4.62941417112006e-08
79 4.5372918595632e-08
80 4.46206984676678e-08
81 4.39832813512453e-08
82 4.31707217494193e-08
83 4.25775827288533e-08
84 4.1722159798141e-08
85 4.13515535910847e-08
86 4.07779806028685e-08
87 4.02014689715458e-08
88 3.94814357118634e-08
89 3.84405784927822e-08
90 3.781153739979e-08
91 3.71890879851122e-08
92 3.67306903577003e-08
93 3.6289705809267e-08
94 3.5956254466285e-08
95 3.55405742187287e-08
96 3.5050703532491e-08
97 3.45925201479247e-08
98 3.41652294206796e-08
99 3.37272136405886e-08
};
\addlegendentry{Train}
\addplot [semithick, black]
table {%
0 0.00866654142737389
1 8.84094788489165e-06
2 2.00968838726112e-06
3 7.31279897081549e-07
4 4.73140715939735e-07
5 3.52726146957139e-07
6 2.9388934308372e-07
7 2.60102950733199e-07
8 2.36849501789038e-07
9 2.15041239925995e-07
10 1.97724830286461e-07
11 1.81597201276418e-07
12 1.69184531273459e-07
13 1.58091694402174e-07
14 1.47938195027564e-07
15 1.3811572330269e-07
16 1.32056698021188e-07
17 1.26358045804409e-07
18 1.21331140690017e-07
19 1.16853961174002e-07
20 1.11020746373924e-07
21 1.07112768432671e-07
22 1.03425243480615e-07
23 1.00011774861741e-07
24 9.57444683535869e-08
25 9.22386362844918e-08
26 8.91720759454984e-08
27 8.62698072978674e-08
28 8.39927949414232e-08
29 8.23603372168691e-08
30 8.05492845756817e-08
31 7.90747307632955e-08
32 7.72905721646566e-08
33 7.59045235554368e-08
34 7.41566452688858e-08
35 7.3112715881507e-08
36 7.20636634810035e-08
37 7.16871682016063e-08
38 7.08661289650081e-08
39 7.01247842016528e-08
40 6.96210022965715e-08
41 6.85927759036531e-08
42 6.78009968169135e-08
43 6.76279370281918e-08
44 6.78170621881691e-08
45 6.72923050615282e-08
46 6.69068782599425e-08
47 6.61775487742489e-08
48 6.56809433507988e-08
49 6.56868195392235e-08
50 6.46684839011868e-08
51 6.42714326204441e-08
52 6.35435313256494e-08
53 6.2915084697579e-08
54 6.21831759417546e-08
55 6.07670997965215e-08
56 6.00365055447583e-08
57 6.02183192199846e-08
58 5.96264513319511e-08
59 5.88223159070367e-08
60 5.73836729245158e-08
61 5.66791023004498e-08
62 5.6527479586066e-08
63 5.64799869096078e-08
64 5.34284261277662e-08
65 5.69077123202533e-08
66 5.37667652622531e-08
67 5.30290158451407e-08
68 5.2691582652642e-08
69 5.15518010502092e-08
70 5.09446245189338e-08
71 5.03664594475595e-08
72 4.99260757180764e-08
73 4.87754157063591e-08
74 4.6104307926953e-08
75 4.61453169009474e-08
76 4.39276739427896e-08
77 4.29112532174258e-08
78 4.40228653530994e-08
79 4.25775183998667e-08
80 4.28592166201724e-08
81 4.29245226030162e-08
82 4.22940402700078e-08
83 4.25497823641763e-08
84 4.46413892518649e-08
85 4.23871462373882e-08
86 4.4345203065177e-08
87 4.4329883763794e-08
88 3.79936579975038e-08
89 3.98043766836054e-08
90 3.94742230014344e-08
91 3.68367700787076e-08
92 3.80003619682157e-08
93 3.75974309463345e-08
94 3.74506683442632e-08
95 3.75666644458761e-08
96 3.75200457369829e-08
97 3.68651456028601e-08
98 3.50575000140907e-08
99 3.44604949020777e-08
};
\addlegendentry{Test}

\nextgroupplot[
legend cell align={left},
legend style={fill opacity=0.8, draw opacity=1, text opacity=1, at={(0.91,0.5)}, anchor=east, draw=white!80!black},
log basis y={10},
tick align=outside,
tick pos=left,
title={20 Nodes},
x grid style={white!69.0196078431373!black},
xlabel={Epoch},
xmin=-4.95, xmax=103.95,
xtick style={color=black},
y grid style={white!69.0196078431373!black},
ylabel={MSE Loss},
ymin=2.64075478351392e-08, ymax=1e-05,
ymode=log,
ytick style={color=black}
]
\addplot [semithick, black, dashed]
table {%
0 0.0285557779595256
1 1.24313733272174e-05
2 3.73180864073674e-06
3 2.02751454264671e-06
4 1.56720319429837e-06
5 1.37222908460899e-06
6 1.18034215537932e-06
7 9.79656650201832e-07
8 8.21410678128132e-07
9 7.27556342170033e-07
10 6.62785066907645e-07
11 6.09597446626253e-07
12 5.61295417682572e-07
13 5.11799158573467e-07
14 4.60063071514583e-07
15 4.07748478743031e-07
16 3.61075544986988e-07
17 3.27628658560286e-07
18 3.0689191270028e-07
19 2.91522740269556e-07
20 2.79350848089166e-07
21 2.69433258281992e-07
22 2.60725183999e-07
23 2.52954899941926e-07
24 2.46158084685533e-07
25 2.40099659400528e-07
26 2.34455066035366e-07
27 2.29365717451913e-07
28 2.24079580505077e-07
29 2.1881586435768e-07
30 2.12909608926282e-07
31 2.0725780992592e-07
32 2.01118237242781e-07
33 1.94248465255953e-07
34 1.86631604286447e-07
35 1.78660467952341e-07
36 1.70499039029437e-07
37 1.62370256454381e-07
38 1.54432801110715e-07
39 1.47011297379152e-07
40 1.4015082586738e-07
41 1.33836198333626e-07
42 1.28050318242146e-07
43 1.22534956108922e-07
44 1.17403037162234e-07
45 1.12624497262459e-07
46 1.0813981464608e-07
47 1.03820897006557e-07
48 9.99178418013002e-08
49 9.62101170145502e-08
50 9.28111854996416e-08
51 8.98610721264959e-08
52 8.69502238245445e-08
53 8.44598643432448e-08
54 8.21764680267734e-08
55 8.01269076902145e-08
56 7.82042876750211e-08
57 7.63415219466879e-08
58 7.48069975244903e-08
59 7.36054270831232e-08
60 7.25144668969335e-08
61 7.15145011263019e-08
62 7.05964718967778e-08
63 6.97348077984827e-08
64 6.89312749386772e-08
65 6.81780473463789e-08
66 6.74911075435602e-08
67 6.68074192482493e-08
68 6.58624340523772e-08
69 6.51625024197244e-08
70 6.44667319349423e-08
71 6.37830666327943e-08
72 6.32190559706203e-08
73 6.26423401826059e-08
74 6.20359431167117e-08
75 6.15155080545549e-08
76 6.09859739029162e-08
77 6.04561287929073e-08
78 5.99667109995039e-08
79 5.94896548982149e-08
80 5.90217320386444e-08
81 5.85936645371987e-08
82 5.8116005284603e-08
83 5.76691585223443e-08
84 5.72481959757454e-08
85 5.68783763448266e-08
86 5.64325161676038e-08
87 5.60511138516517e-08
88 5.56530101647468e-08
89 5.52918631626653e-08
90 5.49574885653215e-08
91 5.44825457708953e-08
92 5.40150958912022e-08
93 5.35701648978204e-08
94 5.31544862063527e-08
95 5.27252715691873e-08
96 5.23326152368497e-08
97 5.19475783882228e-08
98 5.16280132707436e-08
99 5.1175227934408e-08
};
\addlegendentry{Train}
\addplot [semithick, black]
table {%
0 0.0194376688450575
1 1.24327843877836e-05
2 3.78485833607556e-06
3 1.99572514247848e-06
4 1.51514564095123e-06
5 1.30737805648096e-06
6 1.12589668788132e-06
7 9.24400978874473e-07
8 7.66611492508673e-07
9 6.7515156843001e-07
10 6.14861562553415e-07
11 5.65051209377998e-07
12 5.21227775607258e-07
13 4.77601929560478e-07
14 4.31805375455951e-07
15 3.88944499718491e-07
16 3.53875520886504e-07
17 3.27364148233755e-07
18 3.10661306457405e-07
19 2.96988901027362e-07
20 2.87041160618173e-07
21 2.79218738796772e-07
22 2.70677304570199e-07
23 2.62544944007459e-07
24 2.55831309914356e-07
25 2.50234194254517e-07
26 2.44182700726014e-07
27 2.38447768197148e-07
28 2.32831936841649e-07
29 2.27068312597112e-07
30 2.21170054715003e-07
31 2.16275310549463e-07
32 2.10863575489384e-07
33 2.06022420456975e-07
34 1.99714349946589e-07
35 1.9348279067799e-07
36 1.86967895388079e-07
37 1.79009234102523e-07
38 1.7134455276846e-07
39 1.63348303772182e-07
40 1.56275902440939e-07
41 1.49735527088524e-07
42 1.43273751973538e-07
43 1.37745260531119e-07
44 1.32442011135936e-07
45 1.27830404039742e-07
46 1.22696462767635e-07
47 1.17844827229874e-07
48 1.13215762098662e-07
49 1.06874942673585e-07
50 1.02567646820262e-07
51 9.88152422110034e-08
52 9.44696481042229e-08
53 9.10729411884859e-08
54 8.75029115832149e-08
55 8.48282510901299e-08
56 8.24514430064482e-08
57 7.99327466438626e-08
58 7.79266997597006e-08
59 7.64381553608473e-08
60 7.52385460600635e-08
61 7.39795638082796e-08
62 7.27737798911221e-08
63 7.16789330112988e-08
64 7.07713070369209e-08
65 6.97976645369636e-08
66 6.88359236278302e-08
67 6.80419915966013e-08
68 6.75262725735593e-08
69 6.69912836315234e-08
70 6.63497132791235e-08
71 6.56517116226496e-08
72 6.49561755494688e-08
73 6.42760440427992e-08
74 6.3704071351367e-08
75 6.31445757903748e-08
76 6.26483185328652e-08
77 6.20820088670371e-08
78 6.15190103303576e-08
79 6.1054564071128e-08
80 6.06974950301264e-08
81 6.00799907601868e-08
82 5.96005271802369e-08
83 5.91055986376432e-08
84 5.87124944217976e-08
85 5.82232715373721e-08
86 5.7914466111697e-08
87 5.72296094958347e-08
88 5.70212712602824e-08
89 5.66141764579697e-08
90 5.62721957919621e-08
91 5.60238468949592e-08
92 5.57856054683725e-08
93 5.50581020775098e-08
94 5.4450797648542e-08
95 5.40146309901957e-08
96 5.42553308946481e-08
97 5.37836335468e-08
98 5.34082396086433e-08
99 5.27496304414399e-08
};
\addlegendentry{Test}

\nextgroupplot[
legend cell align={left},
legend style={fill opacity=0.8, draw opacity=1, text opacity=1, at={(0.91,0.5)}, anchor=east, draw=white!80!black},
log basis y={10},
tick align=outside,
tick pos=left,
title={10 Nodes},
x grid style={white!69.0196078431373!black},
xlabel={Epoch},
xmin=-4.95, xmax=103.95,
xtick style={color=black},
y grid style={white!69.0196078431373!black},
ylabel={MSE Loss},
ymin=2.24292733418304e-07, ymax=1e-05,
ymode=log,
ytick style={color=black}
]
\addplot [semithick, black, dashed]
table {%
0 0.0371665318906307
1 0.000167624883826647
2 2.22783393146528e-05
3 6.15570772015417e-06
4 4.8282220514011e-06
5 4.4403949111711e-06
6 3.74377291757355e-06
7 2.61476984138653e-06
8 1.73129622703527e-06
9 1.49983422718947e-06
10 1.44574769089445e-06
11 1.41964382081028e-06
12 1.4003511713554e-06
13 1.38436814270904e-06
14 1.3702457210627e-06
15 1.35803022138248e-06
16 1.34638348720273e-06
17 1.33476664993282e-06
18 1.32250189076899e-06
19 1.30875981335521e-06
20 1.29186143357174e-06
21 1.26879777801037e-06
22 1.23750506836018e-06
23 1.19521276755563e-06
24 1.1385280517402e-06
25 1.05627279762643e-06
26 9.40154226555023e-07
27 8.00368918334016e-07
28 6.78665345631657e-07
29 5.9768706604757e-07
30 5.52904182171687e-07
31 5.28689670716176e-07
32 5.12822961596271e-07
33 5.01921839187958e-07
34 4.93759307900632e-07
35 4.87960223281902e-07
36 4.8317291401645e-07
37 4.79422243714112e-07
38 4.76048073409174e-07
39 4.73040067205943e-07
40 4.70348837538381e-07
41 4.68038463949938e-07
42 4.65900847800071e-07
43 4.63994507683196e-07
44 4.6221442093497e-07
45 4.60607598071761e-07
46 4.58949615932625e-07
47 4.57449188374426e-07
48 4.55983695218265e-07
49 4.54581277892885e-07
50 4.53137427214756e-07
51 4.51883749022386e-07
52 4.50723465448277e-07
53 4.49726536118078e-07
54 4.48682312338633e-07
55 4.47556290993134e-07
56 4.46613790714423e-07
57 4.45703239506656e-07
58 4.447696553882e-07
59 4.4402129603327e-07
60 4.43431036998732e-07
61 4.42907431008166e-07
62 4.42404822166509e-07
63 4.41943481675366e-07
64 4.41575087691604e-07
65 4.41274567691607e-07
66 4.41013233455578e-07
67 4.40718813607077e-07
68 4.40471063456016e-07
69 4.40257352849471e-07
70 4.40057484411227e-07
71 4.39869070248733e-07
72 4.39687860762206e-07
73 4.39530985104852e-07
74 4.39373276734045e-07
75 4.39211473704404e-07
76 4.39061738561009e-07
77 4.38924750568503e-07
78 4.38795396220826e-07
79 4.38652172718434e-07
80 4.38535388269656e-07
81 4.3841395327604e-07
82 4.38307625771017e-07
83 4.38202796672726e-07
84 4.38095886110546e-07
85 4.37995095794008e-07
86 4.37888355307337e-07
87 4.37789605996386e-07
88 4.37691701392851e-07
89 4.37597221377928e-07
90 4.37508018947597e-07
91 4.37420832057001e-07
92 4.37327752536021e-07
93 4.37243473996318e-07
94 4.37165865832867e-07
95 4.37087153954963e-07
96 4.36978484927408e-07
97 4.36831433177076e-07
98 4.36769714653451e-07
99 4.36703425549467e-07
};
\addlegendentry{Train}
\addplot [semithick, black]
table {%
0 0.0290108826011419
1 0.000175098888576031
2 2.42669611907331e-05
3 6.20924220129382e-06
4 5.01842941957875e-06
5 4.6177551666915e-06
6 3.86226065529627e-06
7 2.66841379925609e-06
8 1.70500720741984e-06
9 1.45336809964647e-06
10 1.39014366595802e-06
11 1.36127277983178e-06
12 1.34281583541451e-06
13 1.32999866764294e-06
14 1.32139382458263e-06
15 1.30984506085952e-06
16 1.29833006212721e-06
17 1.28624594708526e-06
18 1.27308464925591e-06
19 1.2585567219503e-06
20 1.24071459595143e-06
21 1.21881032555393e-06
22 1.18766774903634e-06
23 1.14379565729905e-06
24 1.08491462924576e-06
25 1.00215538623161e-06
26 8.86264956534433e-07
27 7.49797834487254e-07
28 6.32172657333285e-07
29 5.56234169835079e-07
30 5.12609688030352e-07
31 4.87748081923201e-07
32 4.7379265311065e-07
33 4.65045616238058e-07
34 4.59446596323687e-07
35 4.54110647751804e-07
36 4.50179157951425e-07
37 4.46305506329736e-07
38 4.4328496073831e-07
39 4.41358366742861e-07
40 4.38246672729292e-07
41 4.3535669647099e-07
42 4.32203052014302e-07
43 4.29440603966214e-07
44 4.26885435445001e-07
45 4.24835718604299e-07
46 4.22295954649599e-07
47 4.19777109073038e-07
48 4.17637949112759e-07
49 4.15761007843685e-07
50 4.13967285339822e-07
51 4.12328290622099e-07
52 4.10705240483367e-07
53 4.09496465181292e-07
54 4.07888450126848e-07
55 4.06697381549748e-07
56 4.05766769517868e-07
57 4.04747197535471e-07
58 4.03904664381116e-07
59 4.03076853672246e-07
60 4.02392117848649e-07
61 4.01878992306592e-07
62 4.01740095412606e-07
63 4.01309222297641e-07
64 4.00987545390308e-07
65 4.00721404503201e-07
66 4.00519354570861e-07
67 4.00306447545518e-07
68 4.00117414756096e-07
69 3.99882424062525e-07
70 3.99712973830901e-07
71 3.99666163275469e-07
72 3.99526328465072e-07
73 3.99394224359639e-07
74 3.99269282524983e-07
75 3.9911228100209e-07
76 3.98999731032745e-07
77 3.9887274283501e-07
78 3.98728758455036e-07
79 3.98601940787557e-07
80 3.9848805499787e-07
81 3.98401112988722e-07
82 3.98319059513597e-07
83 3.98246271515745e-07
84 3.98169305526608e-07
85 3.98091316355931e-07
86 3.9802151263757e-07
87 3.97999116330539e-07
88 3.97884690528372e-07
89 3.97822589093266e-07
90 3.97824123865576e-07
91 3.97776744875955e-07
92 3.9767022030901e-07
93 3.97682839547997e-07
94 3.97628582504694e-07
95 3.97597290202611e-07
96 3.97583363564991e-07
97 3.97599393409109e-07
98 3.97624205561442e-07
99 3.97516402017573e-07
};
\addlegendentry{Test}
\end{groupplot}

\end{tikzpicture}

		\caption{Five experiments over different width with $\hy$ left and $\rare$ right. The number of nodes used for every experiment are given. Training and validation loss are shown over 4000 epochs.}
		\label{Fig:Width}
	\end{figure}
\end{center}
\begin{center}
	\begin{figure}[htbp!]
		% This file was created by tikzplotlib v0.9.6.
\begin{tikzpicture}

\begin{groupplot}[group style={group size=1 by 5},
legend cell align={left},
legend style={fill opacity=1, draw opacity=1, text opacity=1, draw=white},
log basis y={10},
tick align=outside,
tick pos=left,
title style={at={(0.3,0.85)},anchor=north},
x grid style={white!69.0196078431373!black},
xlabel={Epoch},
x label style={yshift=13pt},
xmin=-99.95, xmax=5098.95,
xtick style={color=black},
xtick = {0,1000,4000,5000},
y grid style={white!69.0196078431373!black},
ylabel={MSE Loss},
ymode=log,
ytick style={color=black},
width=0.45\textwidth,
height=0.25\textwidth
]
\nextgroupplot[
title={Batch Size 32 $\hy$},
ymin=2.19689304011912e-08, ymax=1e-05,
]
\addplot [semithick, black, dashed]
table {%
0 0.018279776699841
1 0.010003516331315
2 0.00595930013433099
3 0.00379689644649625
4 0.00272321451641619
5 0.00220740181207657
6 0.00184055117238313
7 0.00145098311174661
8 0.00105770584614947
9 0.000727647650754079
10 0.00049769380572252
11 0.000359426279435866
12 0.000283846344682388
13 0.000245307347446214
14 0.000226581241353415
15 0.000217578529525781
16 0.000213053132407367
17 0.000210495216568233
18 0.00020878604252357
19 0.000207492929446744
20 0.000206403148273239
21 0.000205393510288559
22 0.000204354272194905
23 0.000203228068683529
24 0.000201971621834673
25 0.000200554712238954
26 0.00019896302028792
27 0.000197168161685113
28 0.000195135599875357
29 0.000192829498584615
30 0.000190202805039007
31 0.000186731915775454
32 0.000181116980616935
33 0.000175283783872146
34 0.000168973797961371
35 0.00016187575069489
36 0.000153807618335122
37 0.000144628381822258
38 0.000133628674317151
39 0.000121016363700619
40 0.000107509779831162
41 9.33101031987462e-05
42 7.89550611953018e-05
43 6.50794441608014e-05
44 5.25026073446497e-05
45 4.1934457491152e-05
46 3.36382959721959e-05
47 2.72036691349058e-05
48 2.34670028548862e-05
49 2.109416665553e-05
50 1.94228632935847e-05
51 1.81368334888248e-05
52 1.70711645623669e-05
53 1.61395108734723e-05
54 1.53003792001982e-05
55 1.45287190935051e-05
56 1.38098775223625e-05
57 1.31355724679452e-05
58 1.2499766820838e-05
59 1.18923849095154e-05
60 1.13065995046782e-05
61 1.07391303863551e-05
62 1.01865025844745e-05
63 9.64752519939793e-06
64 9.12014746609202e-06
65 8.60713798283541e-06
66 8.10884660859301e-06
67 7.62637301158975e-06
68 7.15948304787162e-06
69 6.71028730266698e-06
70 6.27853885634977e-06
71 5.86864928209252e-06
72 5.482248876433e-06
73 5.12185664592835e-06
74 4.78452617880976e-06
75 4.47036846981064e-06
76 4.17668670615967e-06
77 3.908571703505e-06
78 3.66194465186709e-06
79 3.43383670497133e-06
80 3.22597387003043e-06
81 3.03591548208715e-06
82 2.86280222189816e-06
83 2.708047793476e-06
84 2.56665889901342e-06
85 2.43750355775774e-06
86 2.32154996319878e-06
87 2.21773789712643e-06
88 2.1230545655726e-06
89 2.03720387207795e-06
90 1.95616122323372e-06
91 1.87988475113343e-06
92 1.8095799623552e-06
93 1.74439507736679e-06
94 1.6833349513945e-06
95 1.62604826323332e-06
96 1.57204723495852e-06
97 1.51897429304881e-06
98 1.46939654177913e-06
99 1.42250795374821e-06
100 1.37710818739833e-06
101 1.33307942883221e-06
102 1.29247836275681e-06
103 1.25404820801123e-06
104 1.21617418062669e-06
105 1.18090810610738e-06
106 1.14728672838282e-06
107 1.11560941058997e-06
108 1.08483458620867e-06
109 1.05480580691619e-06
110 1.02695428790867e-06
111 1.00121923992447e-06
112 9.77464552306628e-07
113 9.54890225102645e-07
114 9.33209033973981e-07
115 9.13015426021957e-07
116 8.9422697885766e-07
117 8.76947148640284e-07
118 8.61613175288767e-07
119 8.4783891418283e-07
120 8.33655071346584e-07
121 8.19832788579333e-07
122 8.07610620768173e-07
123 7.96093350572846e-07
124 7.86567400837157e-07
125 7.78220704546584e-07
126 7.7130168608619e-07
127 7.6528135400622e-07
128 7.60048918891698e-07
129 7.53699768665683e-07
130 7.45873750702231e-07
131 7.38039693601422e-07
132 7.32381362695378e-07
133 7.27230772326948e-07
134 7.21991306022574e-07
135 7.17503453984136e-07
136 7.13783272544788e-07
137 7.10493326209871e-07
138 7.08098746940777e-07
139 7.06139542330675e-07
140 7.04594898707001e-07
141 7.03261042758641e-07
142 7.02102420063966e-07
143 7.01118758570374e-07
144 7.00222886848678e-07
145 6.99455154176576e-07
146 6.98680325058376e-07
147 6.97923998700389e-07
148 6.97215751529257e-07
149 6.9649489648782e-07
150 6.95870183903935e-07
151 6.95256281005641e-07
152 6.94782263508387e-07
153 6.94202209160721e-07
154 6.93687929697262e-07
155 6.93190502602192e-07
156 6.92716696107709e-07
157 6.92245499294586e-07
158 6.91798356911022e-07
159 6.91376403551658e-07
160 6.90968551452897e-07
161 6.90569527591833e-07
162 6.90184787913495e-07
163 6.8980550474862e-07
164 6.89447199079041e-07
165 6.8906503895505e-07
166 6.88662886773272e-07
167 6.88298930754172e-07
168 6.87958655930743e-07
169 6.87603658207081e-07
170 6.87284928289955e-07
171 6.86911204752505e-07
172 6.86446513213923e-07
173 6.86226743368934e-07
174 6.85781098468397e-07
175 6.85494452909552e-07
176 6.85177626678524e-07
177 6.84892957451666e-07
178 6.84584478449324e-07
179 6.84326157966097e-07
180 6.840496815812e-07
181 6.83704623270387e-07
182 6.83421214716873e-07
183 6.83222082329849e-07
184 6.82921488078136e-07
185 6.82648115230222e-07
186 6.82337593616467e-07
187 6.82123107822008e-07
188 6.8186247972335e-07
189 6.81488661939511e-07
190 6.81231163866869e-07
191 6.80929110785655e-07
192 6.8068866073645e-07
193 6.80437562095904e-07
194 6.80207895698004e-07
195 6.79924521364228e-07
196 6.7970182760746e-07
197 6.79483277394866e-07
198 6.79231108279055e-07
199 6.78973124649929e-07
200 6.7881632355693e-07
201 6.78526332876572e-07
202 6.78257674053384e-07
203 6.78132663324504e-07
204 6.77831195275758e-07
205 6.77797532603108e-07
206 6.77556352570718e-07
207 6.771996995667e-07
208 6.77037663535884e-07
209 6.76851037724191e-07
210 6.76588832220659e-07
211 6.76387339581197e-07
212 6.76128217605765e-07
213 6.75856486964221e-07
214 6.75646405852603e-07
215 6.7537215397806e-07
216 6.7520365053042e-07
217 6.74937086500904e-07
218 6.74722397207006e-07
219 6.74562090921427e-07
220 6.74354300940649e-07
221 6.74068465627897e-07
222 6.7396500992345e-07
223 6.73742448952908e-07
224 6.73666398938622e-07
225 6.7345162563015e-07
226 6.7319510606012e-07
227 6.72972255074455e-07
228 6.72768061463103e-07
229 6.72540881396344e-07
230 6.72333369720945e-07
231 6.72107021273405e-07
232 6.71908179242564e-07
233 6.71678451794833e-07
234 6.71497376401931e-07
235 6.71291256367113e-07
236 6.71083019483376e-07
237 6.70867535859543e-07
238 6.70689538196711e-07
239 6.70484076863431e-07
240 6.70283051590559e-07
241 6.70085005594956e-07
242 6.69883269665661e-07
243 6.69691528401017e-07
244 6.69515282083921e-07
245 6.69341574621285e-07
246 6.69147800294923e-07
247 6.68952831460956e-07
248 6.68795275146294e-07
249 6.68582317643995e-07
250 6.68403884674262e-07
251 6.68218784767305e-07
252 6.68039071456406e-07
253 6.67856823383772e-07
254 6.67677022079261e-07
255 6.67491965714362e-07
256 6.67307971866649e-07
257 6.67134490754506e-07
258 6.66949021251639e-07
259 6.66761671482163e-07
260 6.66584724967834e-07
261 6.66399081183044e-07
262 6.66210527469957e-07
263 6.66155533053825e-07
264 6.65961474737742e-07
265 6.65769100805846e-07
266 6.65567627152086e-07
267 6.65373439005634e-07
268 6.65247565279969e-07
269 6.65060144910967e-07
270 6.64862234543762e-07
271 6.64665966041866e-07
272 6.64468908780691e-07
273 6.64268863260986e-07
274 6.64082755633899e-07
275 6.63885463609404e-07
276 6.63695745743098e-07
277 6.6350276108551e-07
278 6.6331086873106e-07
279 6.63124974494167e-07
280 6.6293640315962e-07
281 6.6274719199555e-07
282 6.6255719423225e-07
283 6.62363613173511e-07
284 6.62175631418904e-07
285 6.62067783309794e-07
286 6.6184950094339e-07
287 6.61658571971202e-07
288 6.61474971934695e-07
289 6.61271849253353e-07
290 6.61119629512541e-07
291 6.60973251683572e-07
292 6.60786693742921e-07
293 6.60589277117651e-07
294 6.60396362377469e-07
295 6.60204763335059e-07
296 6.60010577121284e-07
297 6.59814327718777e-07
298 6.59652099557206e-07
299 6.59563411659292e-07
300 6.59234466070302e-07
301 6.59077077557413e-07
302 6.58853323557196e-07
303 6.58616824466662e-07
304 6.5840417732943e-07
305 6.58182535062224e-07
306 6.57972549220176e-07
307 6.5776246481164e-07
308 6.57519814353691e-07
309 6.57310447763848e-07
310 6.57082445172819e-07
311 6.5685530933024e-07
312 6.56562826634399e-07
313 6.56335915209638e-07
314 6.56098304148145e-07
315 6.55832067991469e-07
316 6.55592338034694e-07
317 6.55331723010022e-07
318 6.55080954174991e-07
319 6.54816772112099e-07
320 6.54563590614998e-07
321 6.54305039688552e-07
322 6.54051211768092e-07
323 6.53788975228053e-07
324 6.53549641810969e-07
325 6.53279336233936e-07
326 6.52993483754472e-07
327 6.5271912035314e-07
328 6.52496250268086e-07
329 6.52231789331381e-07
330 6.52006735776922e-07
331 6.51832702715183e-07
332 6.51545208484094e-07
333 6.51308100941606e-07
334 6.51025243655567e-07
335 6.50782264415284e-07
336 6.50498889058326e-07
337 6.50226360221495e-07
338 6.49982914069369e-07
339 6.49711227197258e-07
340 6.4939026697175e-07
341 6.49152192067959e-07
342 6.48868200073593e-07
343 6.48575120180794e-07
344 6.48273612114281e-07
345 6.48002011757853e-07
346 6.47872142963024e-07
347 6.47542157253156e-07
348 6.47071385287745e-07
349 6.4671516463477e-07
350 6.46318900066945e-07
351 6.45899288997498e-07
352 6.45505970965132e-07
353 6.4510854974742e-07
354 6.44709520656761e-07
355 6.44204217223887e-07
356 6.44004300625056e-07
357 6.43540158989708e-07
358 6.42962507299671e-07
359 6.42493451209702e-07
360 6.42079705471588e-07
361 6.41549503825445e-07
362 6.41078244825621e-07
363 6.40621474872205e-07
364 6.40067534618538e-07
365 6.39543802549269e-07
366 6.39043845808374e-07
367 6.38262963207126e-07
368 6.37732056020468e-07
369 6.37208080320306e-07
370 6.36702198335115e-07
371 6.36186395581717e-07
372 6.3567071288162e-07
373 6.35163309539166e-07
374 6.34657438695285e-07
375 6.34138093801084e-07
376 6.33640875321362e-07
377 6.331091856282e-07
378 6.32570139032396e-07
379 6.32027211622699e-07
380 6.31512860422845e-07
381 6.30971610348752e-07
382 6.30446171726362e-07
383 6.29921015956825e-07
384 6.29375539460852e-07
385 6.28689244649649e-07
386 6.28146176609334e-07
387 6.27377119599259e-07
388 6.26657146426624e-07
389 6.25911498332243e-07
390 6.2518359140995e-07
391 6.24390244411188e-07
392 6.23650040779466e-07
393 6.22489184138431e-07
394 6.21166284531682e-07
395 6.20248354493924e-07
396 6.19194296177739e-07
397 6.18109619949792e-07
398 6.17284297049991e-07
399 6.16190991195253e-07
400 6.14987454127913e-07
401 6.1360258609966e-07
402 6.12137440498373e-07
403 6.11057272635662e-07
404 6.09764981618355e-07
405 6.08355998565457e-07
406 6.06617764788098e-07
407 6.05510870514081e-07
408 6.04173991746393e-07
409 6.02903102731034e-07
410 6.01742744152034e-07
411 6.00469195433107e-07
412 5.99251538346834e-07
413 5.97727611420851e-07
414 5.96587586414898e-07
415 5.95319881426803e-07
416 5.94102120089701e-07
417 5.92897524938962e-07
418 5.91636030435438e-07
419 5.90335467336445e-07
420 5.89168332453482e-07
421 5.87935851172006e-07
422 5.86630537441124e-07
423 5.85415262094102e-07
424 5.84224422823354e-07
425 5.82840813763141e-07
426 5.81535135324884e-07
427 5.80269992269677e-07
428 5.78955600758491e-07
429 5.77645672251492e-07
430 5.76336910512509e-07
431 5.74924817669853e-07
432 5.73551804109229e-07
433 5.72240263181811e-07
434 5.70896664612519e-07
435 5.69438193906535e-07
436 5.6800467893936e-07
437 5.66643562365243e-07
438 5.65144981123922e-07
439 5.63761865350898e-07
440 5.61936078270264e-07
441 5.6048916974305e-07
442 5.58927636689077e-07
443 5.57403165544201e-07
444 5.55851294052445e-07
445 5.54304368506564e-07
446 5.52752455632799e-07
447 5.51165033471079e-07
448 5.49603997342274e-07
449 5.48010702573265e-07
450 5.46104568343253e-07
451 5.44589133482987e-07
452 5.42958454161635e-07
453 5.41361638738635e-07
454 5.39749426366143e-07
455 5.38198043045668e-07
456 5.36579981371688e-07
457 5.34922410565741e-07
458 5.32923600871982e-07
459 5.31027317833832e-07
460 5.29197201103671e-07
461 5.27397769303661e-07
462 5.25670179797544e-07
463 5.23970486256076e-07
464 5.22290491289823e-07
465 5.20620977056296e-07
466 5.19006503111541e-07
467 5.1740494097885e-07
468 5.15757541791118e-07
469 5.14107760864135e-07
470 5.12376183735341e-07
471 5.10717093618496e-07
472 5.08761546711867e-07
473 5.07231552035137e-07
474 5.05577181627359e-07
475 5.03950173253997e-07
476 5.02450479075378e-07
477 5.00893357411769e-07
478 4.99174332048824e-07
479 4.97670495064995e-07
480 4.96140567065595e-07
481 4.94664611778717e-07
482 4.93043593564835e-07
483 4.91436097206588e-07
484 4.89949293864811e-07
485 4.88355588686318e-07
486 4.86763746607721e-07
487 4.8523907668141e-07
488 4.83565420609011e-07
489 4.81970147859556e-07
490 4.80389636720702e-07
491 4.78822035290705e-07
492 4.77315247053411e-07
493 4.75806793815536e-07
494 4.74381948947666e-07
495 4.73025316580333e-07
496 4.71604994856989e-07
497 4.70186020891106e-07
498 4.68727505449351e-07
499 4.67349197151634e-07
500 4.65997041146693e-07
501 4.64633828471506e-07
502 4.63299611453749e-07
503 4.61844435449166e-07
504 4.60593947138932e-07
505 4.59353567975995e-07
506 4.58071291575379e-07
507 4.56806329168558e-07
508 4.5548034188414e-07
509 4.54066638951645e-07
510 4.52910892192904e-07
511 4.51255167547515e-07
512 4.4998300251109e-07
513 4.48717653284803e-07
514 4.47437230718606e-07
515 4.4617439709782e-07
516 4.44751070972416e-07
517 4.43488532596348e-07
518 4.42235589559914e-07
519 4.40997258010611e-07
520 4.39815030631507e-07
521 4.38646985116975e-07
522 4.37451012942347e-07
523 4.36284451609481e-07
524 4.35159710491462e-07
525 4.33980527589028e-07
526 4.32815933038455e-07
527 4.31691750009122e-07
528 4.30601543428111e-07
529 4.29344490953554e-07
530 4.28255228996477e-07
531 4.26924533996953e-07
532 4.25833846975365e-07
533 4.2471321563653e-07
534 4.23658604461252e-07
535 4.22446771153773e-07
536 4.21365733814127e-07
537 4.20301998872219e-07
538 4.19219277432603e-07
539 4.1815615526275e-07
540 4.17094854981315e-07
541 4.16030026684666e-07
542 4.1498089626657e-07
543 4.13956119388104e-07
544 4.13158988322948e-07
545 4.1214198012085e-07
546 4.1129941976692e-07
547 4.10332433062877e-07
548 4.09313589528892e-07
549 4.08295574544582e-07
550 4.07252368347599e-07
551 4.06252030614951e-07
552 4.05331559164779e-07
553 4.04293710175807e-07
554 4.03292469798089e-07
555 4.02335441805235e-07
556 4.014428614596e-07
557 4.00411786500854e-07
558 3.99402252298842e-07
559 3.98415568355404e-07
560 3.97464957842431e-07
561 3.96527602731567e-07
562 3.95607577473811e-07
563 3.94701060145053e-07
564 3.93819654391336e-07
565 3.9291089652238e-07
566 3.92022393242542e-07
567 3.91145141065863e-07
568 3.90277977714959e-07
569 3.89420729334233e-07
570 3.88583693876399e-07
571 3.87754789443306e-07
572 3.86936204733956e-07
573 3.86152636252746e-07
574 3.85326576491707e-07
575 3.8450919674915e-07
576 3.83648321133023e-07
577 3.82857540415671e-07
578 3.82058717264044e-07
579 3.81257377227939e-07
580 3.80437472188078e-07
581 3.79659714212721e-07
582 3.78885122785277e-07
583 3.78161623672213e-07
584 3.77364226920918e-07
585 3.7660212552737e-07
586 3.76067156054205e-07
587 3.75127589791191e-07
588 3.74405723306381e-07
589 3.73810468914826e-07
590 3.72923419490689e-07
591 3.72354541383402e-07
592 3.71409898662023e-07
593 3.70717152350153e-07
594 3.70199493801238e-07
595 3.69286663868706e-07
596 3.68727778237599e-07
597 3.678815260173e-07
598 3.67346682850211e-07
599 3.66590436016168e-07
600 3.6567334694837e-07
601 3.64931990191053e-07
602 3.64209674955873e-07
603 3.63392423395226e-07
604 3.62660903277856e-07
605 3.61948892646069e-07
606 3.6115702411621e-07
607 3.60250761559655e-07
608 3.59460061872596e-07
609 3.58724025886659e-07
610 3.57997952903588e-07
611 3.572850450837e-07
612 3.56625527871302e-07
613 3.55899685928307e-07
614 3.54996032257304e-07
615 3.54318282376198e-07
616 3.53594193285289e-07
617 3.52883889974009e-07
618 3.52177354187688e-07
619 3.5145800646319e-07
620 3.50982269992528e-07
621 3.50367863319434e-07
622 3.49642222460034e-07
623 3.48912194453987e-07
624 3.48204410300923e-07
625 3.47504645191066e-07
626 3.46780459636875e-07
627 3.46072915363038e-07
628 3.45355743974096e-07
629 3.44662569489174e-07
630 3.43982327649428e-07
631 3.43290248565609e-07
632 3.42602098015732e-07
633 3.41914696122103e-07
634 3.41287624706865e-07
635 3.40689489831902e-07
636 3.40028673406323e-07
637 3.39306961876673e-07
638 3.38728873060745e-07
639 3.379884336141e-07
640 3.37314327623517e-07
641 3.36673534121701e-07
642 3.36003247952021e-07
643 3.35427720756343e-07
644 3.34697720660415e-07
645 3.34121017175448e-07
646 3.33488175982666e-07
647 3.32774230940913e-07
648 3.32208508694976e-07
649 3.3158572659886e-07
650 3.30881670294048e-07
651 3.30330980034432e-07
652 3.29715889733961e-07
653 3.29098270185568e-07
654 3.28484387409844e-07
655 3.2787269665846e-07
656 3.27260727516432e-07
657 3.26668662353313e-07
658 3.25973536490665e-07
659 3.25391892744165e-07
660 3.24788339838733e-07
661 3.24183108148191e-07
662 3.23562649612086e-07
663 3.22945400228036e-07
664 3.22341664713122e-07
665 3.21740765315326e-07
666 3.21136782901021e-07
667 3.20534071903467e-07
668 3.19925650927644e-07
669 3.19319156545816e-07
670 3.18715394001856e-07
671 3.18113104242457e-07
672 3.17504016891235e-07
673 3.16906197298294e-07
674 3.16300114519663e-07
675 3.15688127699332e-07
676 3.15072994538923e-07
677 3.14453853377472e-07
678 3.1383195553758e-07
679 3.13209550739657e-07
680 3.12588391409463e-07
681 3.11970789681482e-07
682 3.11351863700793e-07
683 3.10686407004823e-07
684 3.09974503664989e-07
685 3.09309519366252e-07
686 3.08682006078698e-07
687 3.08037082305646e-07
688 3.07357483592341e-07
689 3.06762649245229e-07
690 3.06091261222718e-07
691 3.05442932955202e-07
692 3.04823587413239e-07
693 3.0417218803791e-07
694 3.03523494324054e-07
695 3.02864456074303e-07
696 3.02211043106126e-07
697 3.01558484892439e-07
698 3.00905853777067e-07
699 3.00253850639365e-07
700 2.996028350708e-07
701 2.98994927305785e-07
702 2.98331352126979e-07
703 2.9768059735602e-07
704 2.97022721326812e-07
705 2.96361960437252e-07
706 2.95705307905791e-07
707 2.95039819775411e-07
708 2.94378928771266e-07
709 2.93681511550403e-07
710 2.9302979154977e-07
711 2.92358466907672e-07
712 2.91679019966296e-07
713 2.90996425036383e-07
714 2.90320674764644e-07
715 2.8963620800937e-07
716 2.8893199379354e-07
717 2.88243287997147e-07
718 2.87554928348754e-07
719 2.86865220516574e-07
720 2.86165954975104e-07
721 2.85479076922002e-07
722 2.84899429630059e-07
723 2.84188412763342e-07
724 2.83492930293505e-07
725 2.82784609964892e-07
726 2.82088313639406e-07
727 2.81369296345702e-07
728 2.80660744664374e-07
729 2.79946013421295e-07
730 2.79235128232358e-07
731 2.78519276463385e-07
732 2.77912220497001e-07
733 2.77180078654737e-07
734 2.76452398622951e-07
735 2.75722470576056e-07
736 2.74990925305474e-07
737 2.74258400082772e-07
738 2.735249389616e-07
739 2.7267669437947e-07
740 2.7194774369832e-07
741 2.71122829360593e-07
742 2.70369399515857e-07
743 2.6960436488821e-07
744 2.68846852293336e-07
745 2.68089445569331e-07
746 2.67332171489443e-07
747 2.66575587460238e-07
748 2.65762433258487e-07
749 2.64972399833141e-07
750 2.64208724928494e-07
751 2.63447018909346e-07
752 2.62685118826766e-07
753 2.61920102474278e-07
754 2.61097995178261e-07
755 2.60307950924243e-07
756 2.59494283341155e-07
757 2.58718715400619e-07
758 2.57941570623643e-07
759 2.57164740190774e-07
760 2.56379372444826e-07
761 2.55583662323033e-07
762 2.54800274944955e-07
763 2.54021206899324e-07
764 2.53245317708206e-07
765 2.52457457094124e-07
766 2.51666986002874e-07
767 2.50876615695006e-07
768 2.50136141005441e-07
769 2.49335941134632e-07
770 2.48541143434977e-07
771 2.47740479920822e-07
772 2.46936908609996e-07
773 2.46136728719648e-07
774 2.45331631788304e-07
775 2.445255176724e-07
776 2.43715853343929e-07
777 2.42906595190107e-07
778 2.42099257775408e-07
779 2.41291632221419e-07
780 2.40480254717568e-07
781 2.39669370586171e-07
782 2.38858831522748e-07
783 2.38047400983987e-07
784 2.37236057301971e-07
785 2.36423131042329e-07
786 2.35609553328686e-07
787 2.3479513592406e-07
788 2.33981888413837e-07
789 2.33161268511139e-07
790 2.32346056748156e-07
791 2.31523983330817e-07
792 2.30702647229464e-07
793 2.298755296124e-07
794 2.29054708142939e-07
795 2.28236653128988e-07
796 2.27412348095868e-07
797 2.26587338403306e-07
798 2.257657989162e-07
799 2.24944599835908e-07
800 2.24126864679874e-07
801 2.23306168948056e-07
802 2.22485978582654e-07
803 2.21664820458045e-07
804 2.20845298684935e-07
805 2.20021282558491e-07
806 2.19204534346318e-07
807 2.18385736616256e-07
808 2.17561789270349e-07
809 2.16715974943327e-07
810 2.15888742161496e-07
811 2.15063802102122e-07
812 2.1423865590009e-07
813 2.13412836274074e-07
814 2.12587476198678e-07
815 2.11762182061648e-07
816 2.10938097950475e-07
817 2.10113463566586e-07
818 2.09293214339823e-07
819 2.08471189552029e-07
820 2.07649772761442e-07
821 2.06822869358803e-07
822 2.06001212831097e-07
823 2.05175687767678e-07
824 2.04354342884017e-07
825 2.03533064279782e-07
826 2.02713177969827e-07
827 2.01897347693603e-07
828 2.01083187761242e-07
829 2.00268850164775e-07
830 1.99465010524591e-07
831 1.98656006972442e-07
832 1.97847682613883e-07
833 1.970426222897e-07
834 1.96242611707476e-07
835 1.95445740814648e-07
836 1.94652051590083e-07
837 1.93859478827108e-07
838 1.9306757906179e-07
839 1.92275740261039e-07
840 1.91482385559993e-07
841 1.90693288686816e-07
842 1.89907870492334e-07
843 1.89128600254662e-07
844 1.88352445718465e-07
845 1.87577170322584e-07
846 1.86805876722929e-07
847 1.86033025812549e-07
848 1.85264857464063e-07
849 1.84503470848085e-07
850 1.83740960409295e-07
851 1.82980723309356e-07
852 1.8222433644155e-07
853 1.81472018397244e-07
854 1.8072389954682e-07
855 1.79981433745979e-07
856 1.79243513230176e-07
857 1.78508483458018e-07
858 1.77773001695414e-07
859 1.76759986629804e-07
860 1.73079382506103e-07
861 1.71013251957675e-07
862 1.69419601775189e-07
863 1.68037630373874e-07
864 1.66846723573144e-07
865 1.65774525470397e-07
866 1.64839302527753e-07
867 1.63883686923327e-07
868 1.62993925670207e-07
869 1.62133147370014e-07
870 1.61325800917211e-07
871 1.60542182925383e-07
872 1.59800857261416e-07
873 1.59093990475867e-07
874 1.58416554626228e-07
875 1.57708065870565e-07
876 1.57052994353535e-07
877 1.56429330417041e-07
878 1.55826123233282e-07
879 1.55260081442066e-07
880 1.54633883624911e-07
881 1.54077652496198e-07
882 1.5353287304265e-07
883 1.5301173667126e-07
884 1.52491672508859e-07
885 1.51993094533509e-07
886 1.5148040705526e-07
887 1.5101715922583e-07
888 1.50519393883997e-07
889 1.50026264662984e-07
890 1.49545841793497e-07
891 1.4911787394567e-07
892 1.48670303843801e-07
893 1.48128563552064e-07
894 1.47684719678409e-07
895 1.47251842037122e-07
896 1.4682656060927e-07
897 1.46404654685739e-07
898 1.45986338310422e-07
899 1.45576496805688e-07
900 1.45173823966616e-07
901 1.4477500042176e-07
902 1.44379511539228e-07
903 1.43987754910313e-07
904 1.43600496556928e-07
905 1.43217450698785e-07
906 1.42841775243596e-07
907 1.42468341863378e-07
908 1.42098772755617e-07
909 1.41734790986447e-07
910 1.41373301659087e-07
911 1.41007504367963e-07
912 1.40654108889748e-07
913 1.40336868412305e-07
914 1.39993156523133e-07
915 1.39648253309588e-07
916 1.39313856209355e-07
917 1.38983543706672e-07
918 1.38659226621485e-07
919 1.38341004799258e-07
920 1.38024378287582e-07
921 1.37721830810733e-07
922 1.3741034408099e-07
923 1.37099514233796e-07
924 1.36789941620918e-07
925 1.36491087459945e-07
926 1.36192597892659e-07
927 1.35897411951191e-07
928 1.35605631470526e-07
929 1.35313680885929e-07
930 1.35015430544172e-07
931 1.34727781698984e-07
932 1.3444079647229e-07
933 1.34162066132149e-07
934 1.33882295727972e-07
935 1.33607988942686e-07
936 1.33333773078448e-07
937 1.33065725208326e-07
938 1.32795259560226e-07
939 1.3252778907713e-07
940 1.3226238067432e-07
941 1.31994212324571e-07
942 1.31733686913549e-07
943 1.31473504865198e-07
944 1.31213409446218e-07
945 1.30956984577324e-07
946 1.30701533549882e-07
947 1.3044806988205e-07
948 1.30198441723905e-07
949 1.29949327373424e-07
950 1.29705248099299e-07
951 1.29461863281222e-07
952 1.29220086108717e-07
953 1.28980329549222e-07
954 1.28739634391195e-07
955 1.28502709628719e-07
956 1.28266822741807e-07
957 1.28037034215822e-07
958 1.27807024568938e-07
959 1.27577710657079e-07
960 1.27351957274868e-07
961 1.27128088109885e-07
962 1.26902907112481e-07
963 1.26680191243622e-07
964 1.2645786802068e-07
965 1.26237767858584e-07
966 1.2601876241547e-07
967 1.25804025202569e-07
968 1.2559316013494e-07
969 1.25385756717833e-07
970 1.2518012408691e-07
971 1.24969729967006e-07
972 1.24763671578876e-07
973 1.2455967777214e-07
974 1.24360331824391e-07
975 1.24166551614735e-07
976 1.23974806143679e-07
977 1.23780007214691e-07
978 1.23586830767408e-07
979 1.23398985266476e-07
980 1.23208372997397e-07
981 1.23018439126099e-07
982 1.22828412372655e-07
983 1.22641907466914e-07
984 1.22455016651202e-07
985 1.22272061872764e-07
986 1.22090993670554e-07
987 1.21910636096345e-07
988 1.21731754319399e-07
989 1.2155474414044e-07
990 1.21378296825014e-07
991 1.21203973833417e-07
992 1.21029097158498e-07
993 1.20857798350471e-07
994 1.20686820622495e-07
995 1.20515487310513e-07
996 1.20336668885557e-07
997 1.2017355876992e-07
998 1.20008479711942e-07
999 1.19843109075646e-07
1000 1.19679881180446e-07
1001 1.19499392639e-07
1002 1.19337054627522e-07
1003 1.19179699140659e-07
1004 1.19019260409914e-07
1005 1.18862581842905e-07
1006 1.18708687637081e-07
1007 1.18552838870301e-07
1008 1.18398670622355e-07
1009 1.18244277359736e-07
1010 1.18094280935566e-07
1011 1.17942267166882e-07
1012 1.17793172933034e-07
1013 1.17645211560102e-07
1014 1.17500476306986e-07
1015 1.17353694179201e-07
1016 1.17209909831217e-07
1017 1.1706559138247e-07
1018 1.16922905618821e-07
1019 1.16785299042022e-07
1020 1.16644025354162e-07
1021 1.16504450517141e-07
1022 1.16365010541131e-07
1023 1.16228110528027e-07
1024 1.16092250010524e-07
1025 1.15957806627875e-07
1026 1.15824750906768e-07
1027 1.15693665662775e-07
1028 1.1556024287529e-07
1029 1.15433579850333e-07
1030 1.15306280662253e-07
1031 1.15176904586178e-07
1032 1.15051310388026e-07
1033 1.14925825016599e-07
1034 1.14803055168977e-07
1035 1.146838923205e-07
1036 1.14561077140252e-07
1037 1.1443571173686e-07
1038 1.14311863569583e-07
1039 1.14197352729661e-07
1040 1.14077541780944e-07
1041 1.13959240991335e-07
1042 1.138383433954e-07
1043 1.13717991581552e-07
1044 1.13602133865243e-07
1045 1.13483816932103e-07
1046 1.13368678654524e-07
1047 1.1325600283385e-07
1048 1.1314145393726e-07
1049 1.13029427296851e-07
1050 1.12918213432067e-07
1051 1.1280892934451e-07
1052 1.12697599291778e-07
1053 1.12589872372837e-07
1054 1.12481222544147e-07
1055 1.12375398003905e-07
1056 1.12271984505696e-07
1057 1.12165309104739e-07
1058 1.12061402148811e-07
1059 1.11958632032838e-07
1060 1.11857792177261e-07
1061 1.11758240194604e-07
1062 1.11662537250368e-07
1063 1.11559105846482e-07
1064 1.11454579013071e-07
1065 1.11356613530234e-07
1066 1.11256406256643e-07
1067 1.11161863173947e-07
1068 1.11065380735909e-07
1069 1.10970160875468e-07
1070 1.10876189324927e-07
1071 1.10781885439337e-07
1072 1.10689489105198e-07
1073 1.10597167321203e-07
1074 1.10505567420205e-07
1075 1.10414097520106e-07
1076 1.10326507069658e-07
1077 1.10235595428776e-07
1078 1.10149627062128e-07
1079 1.10062518245968e-07
1080 1.09974232373133e-07
1081 1.09890390973533e-07
1082 1.09791485897404e-07
1083 1.09697297574485e-07
1084 1.09611389206066e-07
1085 1.09525379741626e-07
1086 1.09439978842829e-07
1087 1.09348831784928e-07
1088 1.09257076331915e-07
1089 1.09165026088931e-07
1090 1.09070772452924e-07
1091 1.08975928810651e-07
1092 1.0888371517126e-07
1093 1.08787019286183e-07
1094 1.08691606158118e-07
1095 1.08600420389848e-07
1096 1.08508321659428e-07
1097 1.08420256168529e-07
1098 1.08336072599968e-07
1099 1.08256209557567e-07
1100 1.08192780174932e-07
1101 1.08111283452672e-07
1102 1.08026754674029e-07
1103 1.07943134281641e-07
1104 1.07886956641323e-07
1105 1.07804812103041e-07
1106 1.07720033156511e-07
1107 1.07631849829204e-07
1108 1.07549625255388e-07
1109 1.07464932455059e-07
1110 1.07384126749821e-07
1111 1.07301083005495e-07
1112 1.0722233619731e-07
1113 1.0714345646079e-07
1114 1.0706254531101e-07
1115 1.06985618060662e-07
1116 1.06906576831989e-07
1117 1.06831317225442e-07
1118 1.06752653351805e-07
1119 1.06676842420939e-07
1120 1.06598697044546e-07
1121 1.06523149611348e-07
1122 1.06446293642648e-07
1123 1.06376593862478e-07
1124 1.06299907656648e-07
1125 1.06222667085376e-07
1126 1.06148983945786e-07
1127 1.06075516697501e-07
1128 1.05995923263436e-07
1129 1.05924652785916e-07
1130 1.05852356199421e-07
1131 1.05780040300374e-07
1132 1.05708212970512e-07
1133 1.05634114774489e-07
1134 1.05564701826211e-07
1135 1.05495279754564e-07
1136 1.05424498755724e-07
1137 1.05360116052111e-07
1138 1.05293076870794e-07
1139 1.05228267898383e-07
1140 1.05159611266004e-07
1141 1.05091181268335e-07
1142 1.05024533965548e-07
1143 1.04957917059778e-07
1144 1.04891549526087e-07
1145 1.04828378383104e-07
1146 1.04763991771506e-07
1147 1.04700459516494e-07
1148 1.04637986297007e-07
1149 1.0457691567467e-07
1150 1.04515036056796e-07
1151 1.04457522425605e-07
1152 1.0440033510406e-07
1153 1.04341238042593e-07
1154 1.04270218983515e-07
1155 1.0420973862324e-07
1156 1.04152265436142e-07
1157 1.04095046452812e-07
1158 1.04041742744698e-07
1159 1.03985193177891e-07
1160 1.03932252372374e-07
1161 1.03882279063328e-07
1162 1.038228657535e-07
1163 1.03770360368571e-07
1164 1.03719369192845e-07
1165 1.03665015814158e-07
1166 1.03613348485965e-07
1167 1.0355753950364e-07
1168 1.03506507670659e-07
1169 1.03452139398996e-07
1170 1.0340176490331e-07
1171 1.03349210348824e-07
1172 1.03302000070471e-07
1173 1.03251790392278e-07
1174 1.03204953362024e-07
1175 1.03154366072999e-07
1176 1.03108628351833e-07
1177 1.0305923967735e-07
1178 1.03010982400065e-07
1179 1.02970633292898e-07
1180 1.02919175674288e-07
1181 1.02872230144158e-07
1182 1.02823266445284e-07
1183 1.02772253086414e-07
1184 1.02727010798276e-07
1185 1.02680678779166e-07
1186 1.02631985072321e-07
1187 1.02586308599939e-07
1188 1.02539724437634e-07
1189 1.02492469125082e-07
1190 1.02452829324307e-07
1191 1.02440781432733e-07
1192 1.02395242379316e-07
1193 1.02351660359545e-07
1194 1.02285702169524e-07
1195 1.0228686477376e-07
1196 1.02234482497465e-07
1197 1.02179736003905e-07
1198 1.0212385519992e-07
1199 1.0206892781639e-07
1200 1.02018281481264e-07
1201 1.01981422716335e-07
1202 1.0192842376e-07
1203 1.01867811125089e-07
1204 1.01822194295664e-07
1205 1.01773924413351e-07
1206 1.0172639045436e-07
1207 1.01679684618716e-07
1208 1.01633287854952e-07
1209 1.01586801974918e-07
1210 1.01541396531957e-07
1211 1.0149564948847e-07
1212 1.01450999807184e-07
1213 1.01404815922024e-07
1214 1.0136630923796e-07
1215 1.01318865148414e-07
1216 1.01272705592237e-07
1217 1.01228529459263e-07
1218 1.01184872193016e-07
1219 1.01141285796302e-07
1220 1.0109824461324e-07
1221 1.01055302337727e-07
1222 1.01014653765219e-07
1223 1.00972639472729e-07
1224 1.00931950484551e-07
1225 1.00884066100093e-07
1226 1.00848284134258e-07
1227 1.00806156083877e-07
1228 1.0076906741574e-07
1229 1.00723024431204e-07
1230 1.00686426108609e-07
1231 1.00640665920082e-07
1232 1.00603817529077e-07
1233 1.00558938100903e-07
1234 1.00525035762189e-07
1235 1.00481693394272e-07
1236 1.00446857558723e-07
1237 1.00404933036202e-07
1238 1.00372225816159e-07
1239 1.00331474698123e-07
1240 1.00298026382006e-07
1241 1.0026010357933e-07
1242 1.00226940332959e-07
1243 1.00188396999101e-07
1244 1.00155747730923e-07
1245 1.0011971406243e-07
1246 1.00086770459029e-07
1247 1.00049273157765e-07
1248 1.00019158750797e-07
1249 9.9982942813881e-08
1250 9.99517956046247e-08
1251 9.99155180352318e-08
1252 9.9884548319551e-08
1253 9.98627546664466e-08
1254 9.98268551910542e-08
1255 9.97980044132873e-08
1256 9.97636784063616e-08
1257 9.97345975122243e-08
1258 9.96988538872756e-08
1259 9.96700891278124e-08
1260 9.96386071108191e-08
1261 9.96101607739774e-08
1262 9.95755370780671e-08
1263 9.95503728802305e-08
1264 9.95147473616953e-08
1265 9.94877026840868e-08
1266 9.94545542312153e-08
1267 9.94265987657172e-08
1268 9.93959426978108e-08
1269 9.93634262442811e-08
1270 9.93386677379249e-08
1271 9.93068108101625e-08
1272 9.92815187004226e-08
1273 9.92478010175546e-08
1274 9.92235699897037e-08
1275 9.91892201795963e-08
1276 9.91647131769469e-08
1277 9.91338806386466e-08
1278 9.91068546198903e-08
1279 9.90735796335684e-08
1280 9.90493174413132e-08
1281 9.90152130100341e-08
1282 9.89889021525414e-08
1283 9.89590073885438e-08
1284 9.8932301270338e-08
1285 9.89051632416249e-08
1286 9.88728650810344e-08
1287 9.8849222169406e-08
1288 9.88193685884653e-08
1289 9.87767411118057e-08
1290 9.87509290411026e-08
1291 9.87252119131199e-08
1292 9.86961684077414e-08
1293 9.86673631473423e-08
1294 9.86406902541148e-08
1295 9.86155091311502e-08
1296 9.85880555077756e-08
1297 9.85595199551881e-08
1298 9.85484259103941e-08
1299 9.85217326956445e-08
1300 9.84956967045036e-08
1301 9.84733981965746e-08
1302 9.8444722681279e-08
1303 9.84156480967613e-08
1304 9.83896660358141e-08
1305 9.83644559369168e-08
1306 9.8333610438317e-08
1307 9.83114912997962e-08
1308 9.82838170955347e-08
1309 9.82563532119229e-08
1310 9.82320892006783e-08
1311 9.82079060634078e-08
1312 9.81781170992235e-08
1313 9.81566191882166e-08
1314 9.81358789005071e-08
1315 9.81084567825974e-08
1316 9.80822995160224e-08
1317 9.80575329663225e-08
1318 9.80262614831418e-08
1319 9.80056235988513e-08
1320 9.7978752918948e-08
1321 9.79501995175269e-08
1322 9.79296236494065e-08
1323 9.78995633147406e-08
1324 9.78795089423556e-08
1325 9.78528566975001e-08
1326 9.78265361482045e-08
1327 9.78029824239002e-08
1328 9.77764115077662e-08
1329 9.77505052048855e-08
1330 9.77236224457556e-08
1331 9.77047260306563e-08
1332 9.76748777361536e-08
1333 9.76531742509223e-08
1334 9.76242868802046e-08
1335 9.76031424073653e-08
1336 9.75778562803953e-08
1337 9.75528784294966e-08
1338 9.75264191964698e-08
1339 9.75058946011131e-08
1340 9.74758730905023e-08
1341 9.74571529894774e-08
1342 9.74281691696888e-08
1343 9.74077128574891e-08
1344 9.73809785875801e-08
1345 9.73658923584253e-08
1346 9.73431144331016e-08
1347 9.73229168721446e-08
1348 9.72967447552264e-08
1349 9.72771228759939e-08
1350 9.72548346709345e-08
1351 9.72276873625333e-08
1352 9.72061030495297e-08
1353 9.71874135018425e-08
1354 9.71551781105973e-08
1355 9.7138704632016e-08
1356 9.71093410839785e-08
1357 9.70960046231539e-08
1358 9.7070873124494e-08
1359 9.70452271928934e-08
1360 9.70219965665819e-08
1361 9.69977984226489e-08
1362 9.69725730755044e-08
1363 9.69496628897559e-08
1364 9.69248211362128e-08
1365 9.69083422859285e-08
1366 9.68818512774305e-08
1367 9.68609608946736e-08
1368 9.68358941406677e-08
1369 9.68176384930075e-08
1370 9.67752282576839e-08
1371 9.6749241677685e-08
1372 9.67323677514287e-08
1373 9.67218349217092e-08
1374 9.67467346981721e-08
1375 9.68613575480504e-08
1376 9.66779242048688e-08
1377 9.67838043948177e-08
1378 9.6624101075804e-08
1379 9.67139158518648e-08
1380 9.65783944053555e-08
1381 9.66074738499856e-08
1382 9.65267207959641e-08
1383 9.65692152874453e-08
1384 9.64943542385299e-08
1385 9.6538626110032e-08
1386 9.6454923152578e-08
1387 9.65113901827408e-08
1388 9.64182666223223e-08
1389 9.64714476481277e-08
1390 9.63564610998446e-08
1391 9.6344009463678e-08
1392 9.63247252059318e-08
1393 9.6419004364634e-08
1394 9.62750443278537e-08
1395 9.63292039131147e-08
1396 9.62726881965637e-08
1397 9.62445785717136e-08
1398 9.62210940542718e-08
1399 9.6185784244085e-08
1400 9.61714850973294e-08
1401 9.6138922245359e-08
1402 9.61289690479816e-08
1403 9.60961854872267e-08
1404 9.60769633167047e-08
1405 9.60563364316158e-08
1406 9.60365560729315e-08
1407 9.60172682624716e-08
1408 9.59966942275514e-08
1409 9.59764763308613e-08
1410 9.59560219655486e-08
1411 9.59375983029531e-08
1412 9.59167337839517e-08
1413 9.58930394574509e-08
1414 9.58756104836311e-08
1415 9.58551666769836e-08
1416 9.58317988448698e-08
1417 9.581700615513e-08
1418 9.57934706775632e-08
1419 9.57789950319921e-08
1420 9.57558916923062e-08
1421 9.57355399862081e-08
1422 9.57198517284041e-08
1423 9.57019223903899e-08
1424 9.56825613087631e-08
1425 9.56536491685256e-08
1426 9.56559046727534e-08
1427 9.56161926666255e-08
1428 9.56191320256039e-08
1429 9.55790113010835e-08
1430 9.55808901608179e-08
1431 9.55411861411903e-08
1432 9.55404864555476e-08
1433 9.55050278861336e-08
1434 9.54989436223741e-08
1435 9.54676904996177e-08
1436 9.54594715523172e-08
1437 9.54318501129592e-08
1438 9.54132454324963e-08
1439 9.54054330293275e-08
1440 9.53783617205772e-08
1441 9.5359729101574e-08
1442 9.53424753760146e-08
1443 9.53251413449152e-08
1444 9.53078874204039e-08
1445 9.52904599813564e-08
1446 9.52748882951937e-08
1447 9.52573662544864e-08
1448 9.52406590641885e-08
1449 9.52217676655209e-08
1450 9.52040239354801e-08
1451 9.51854160007315e-08
1452 9.51671169246993e-08
1453 9.51488138838386e-08
1454 9.51309384191745e-08
1455 9.51120677683548e-08
1456 9.50939775492543e-08
1457 9.50759631308529e-08
1458 9.50580653551469e-08
1459 9.50407089987948e-08
1460 9.50221863007528e-08
1461 9.50016798526576e-08
1462 9.49842470703288e-08
1463 9.49669718579571e-08
1464 9.49481050156464e-08
1465 9.49328374417746e-08
1466 9.49141875139503e-08
1467 9.48970569680796e-08
1468 9.48803638038953e-08
1469 9.48628478170121e-08
1470 9.48488002876502e-08
1471 9.48337353747775e-08
1472 9.4819522885814e-08
1473 9.48019142867906e-08
1474 9.47851502957064e-08
1475 9.47658013217278e-08
1476 9.47484848552449e-08
1477 9.47312667278766e-08
1478 9.47137346258842e-08
1479 9.46962031491694e-08
1480 9.46718038505878e-08
1481 9.46414109108673e-08
1482 9.46213811516827e-08
1483 9.46071012322136e-08
1484 9.45886218772785e-08
1485 9.45724190302144e-08
1486 9.45558639671162e-08
1487 9.45390962527881e-08
1488 9.45198115687163e-08
1489 9.45055850962717e-08
1490 9.44859574332213e-08
1491 9.44688489852297e-08
1492 9.44523716839285e-08
1493 9.44354430600924e-08
1494 9.44185552782528e-08
1495 9.44013801813526e-08
1496 9.43845929270992e-08
1497 9.43677866018788e-08
1498 9.43504752797253e-08
1499 9.43335502512355e-08
1500 9.43168299727404e-08
1501 9.42945260362649e-08
1502 9.42827388712431e-08
1503 9.42659680447377e-08
1504 9.42381837774064e-08
1505 9.42335429670038e-08
1506 9.42028919439508e-08
1507 9.42028413817297e-08
1508 9.41703017929285e-08
1509 9.41674241801138e-08
1510 9.41370712865819e-08
1511 9.41334639463776e-08
1512 9.41031871235509e-08
1513 9.40936999000996e-08
1514 9.40720601931844e-08
1515 9.40611499515853e-08
1516 9.40391220325409e-08
1517 9.40273106095901e-08
1518 9.40033801413165e-08
1519 9.39939096156195e-08
1520 9.39722353763273e-08
1521 9.39608633672151e-08
1522 9.39367008356839e-08
1523 9.39277413465334e-08
1524 9.39056548787676e-08
1525 9.38991441756798e-08
1526 9.38753908457102e-08
1527 9.38718802387939e-08
1528 9.38449268943486e-08
1529 9.38363377827045e-08
1530 9.38171731519333e-08
1531 9.380058807551e-08
1532 9.37773815792298e-08
1533 9.37684712312148e-08
1534 9.37452534799377e-08
1535 9.38089272466414e-08
1536 9.37843930159943e-08
1537 9.37842178529991e-08
1538 9.37308590494013e-08
1539 9.37097127007291e-08
1540 9.36896289829292e-08
1541 9.36771264292702e-08
1542 9.36518719782953e-08
1543 9.36441556120826e-08
1544 9.36165689893187e-08
1545 9.36123337993422e-08
1546 9.3593552790594e-08
1547 9.3569076852873e-08
1548 9.35696085520021e-08
1549 9.35433631781279e-08
1550 9.35360642273508e-08
1551 9.35035831446385e-08
1552 9.35028045603303e-08
1553 9.34715147735687e-08
1554 9.34704446535761e-08
1555 9.34384656119391e-08
1556 9.34386032014345e-08
1557 9.34063075135327e-08
1558 9.34046294247537e-08
1559 9.33704956480597e-08
1560 9.3364985374933e-08
1561 9.33365707425082e-08
1562 9.33295070240092e-08
1563 9.33028308622852e-08
1564 9.32926630099473e-08
1565 9.32711163130762e-08
1566 9.32547615377643e-08
1567 9.32381466753895e-08
1568 9.32222822740414e-08
1569 9.32103586279709e-08
1570 9.3194073500058e-08
1571 9.31785823610198e-08
1572 9.31669641346389e-08
1573 9.31518920879171e-08
1574 9.31344844303794e-08
1575 9.31183150498782e-08
1576 9.31008049036564e-08
1577 9.30459699759467e-08
1578 9.30264169198836e-08
1579 9.30143017399132e-08
1580 9.2990368003143e-08
1581 9.29673543197396e-08
1582 9.2856778735495e-08
1583 9.29602372110594e-08
1584 9.29093600063879e-08
1585 9.27996008641685e-08
1586 9.27295179877774e-08
1587 9.26827721769996e-08
1588 9.26497184394748e-08
1589 9.26583417566462e-08
1590 9.25897879540116e-08
1591 9.25207315702892e-08
1592 9.24130962687286e-08
1593 9.27149578302533e-08
1594 9.23106899648474e-08
1595 9.25427249427457e-08
1596 9.22640728902024e-08
1597 9.22630454738282e-08
1598 9.22676573367198e-08
1599 9.23316839589461e-08
1600 9.22428453264956e-08
1601 9.23737288189841e-08
1602 9.21593983349567e-08
1603 9.21506454574228e-08
1604 9.2288642775884e-08
1605 9.20460235533937e-08
1606 9.19970969022188e-08
1607 9.22264686664676e-08
1608 9.1968877512727e-08
1609 9.17193049190246e-08
1610 9.19559239491719e-08
1611 9.19176729894389e-08
1612 9.18864826928711e-08
1613 9.18060305963309e-08
1614 9.20793451939517e-08
1615 9.15827815362036e-08
1616 9.19438322455335e-08
1617 9.17587105533357e-08
1618 9.15739303053442e-08
1619 9.14570664747316e-08
1620 9.18073032636357e-08
1621 9.14084272949367e-08
1622 9.16987489745225e-08
1623 9.16662751677677e-08
1624 9.14523595696437e-08
1625 9.18280544226491e-08
1626 9.13496574810324e-08
1627 9.17453205175889e-08
1628 9.12855208241581e-08
1629 9.15381877462096e-08
1630 9.12530605319262e-08
1631 9.16281225755711e-08
1632 9.14820434161356e-08
1633 9.11463106376687e-08
1634 9.11972695405439e-08
1635 9.10925419930209e-08
1636 9.12702184621139e-08
1637 9.10861023442067e-08
1638 9.1390376610434e-08
1639 9.11198410449288e-08
1640 9.10665865632154e-08
1641 9.10358853758453e-08
1642 9.09800175747932e-08
1643 9.1193055610006e-08
1644 9.09643892867962e-08
1645 9.09391795573811e-08
1646 9.08699901884802e-08
1647 9.10917751468787e-08
1648 9.09352454243617e-08
1649 9.10778720424332e-08
1650 9.08659764462527e-08
1651 9.10126473741002e-08
1652 9.07823245341888e-08
1653 9.07619133556636e-08
1654 9.06650576979473e-08
1655 9.08839560906927e-08
1656 9.08585504646453e-08
1657 9.07151624005564e-08
1658 9.0757662334795e-08
1659 9.08502395589039e-08
1660 9.07385040704867e-08
1661 9.0814276944684e-08
1662 9.05667429549339e-08
1663 9.05546598630735e-08
1664 9.0731190795168e-08
1665 9.07067168185449e-08
1666 9.04811687547635e-08
1667 9.04635572851475e-08
1668 9.04494885105578e-08
1669 9.03936518739101e-08
1670 9.05822596877215e-08
1671 9.0589212049963e-08
1672 9.04338262444071e-08
1673 9.05513576014982e-08
1674 9.06020233344407e-08
1675 9.04463264959077e-08
1676 9.03674329748583e-08
1677 9.04644124943843e-08
1678 9.02548225099054e-08
1679 9.0253239619642e-08
1680 9.02900727197675e-08
1681 9.03968268772814e-08
1682 9.03711191853063e-08
1683 9.02391176254014e-08
1684 9.03438377406474e-08
1685 9.0199227543053e-08
1686 9.02988532658355e-08
1687 9.01599305223044e-08
1688 9.02346204014748e-08
1689 9.02084101426226e-08
1690 9.01896052027951e-08
1691 9.01719182877514e-08
1692 9.01393504335601e-08
1693 8.98984194463992e-08
1694 9.01263160670851e-08
1695 9.00982772407133e-08
1696 9.00671242760609e-08
1697 9.00521765743179e-08
1698 8.98710967192073e-08
1699 8.98505603004196e-08
1700 9.00057324599857e-08
1701 8.99793214159672e-08
1702 8.99579892461588e-08
1703 8.97944400435335e-08
1704 8.98174734089707e-08
1705 8.98993613276389e-08
1706 8.98858331055408e-08
1707 8.98590417079959e-08
1708 8.98389212125039e-08
1709 8.98175207169061e-08
1710 8.9673810961699e-08
1711 8.97768132404053e-08
1712 8.97571756439675e-08
1713 8.97367464887111e-08
1714 8.9713189808549e-08
1715 8.95890109831043e-08
1716 8.96685940716679e-08
1717 8.96559134702102e-08
1718 8.96331288799956e-08
1719 8.96244518031608e-08
1720 8.9595844485757e-08
1721 8.95726708165512e-08
1722 8.95608657742741e-08
1723 8.94208101129834e-08
1724 8.95423956421837e-08
1725 8.9496548341117e-08
1726 8.94720598410004e-08
1727 8.94517906715464e-08
1728 8.94396024193611e-08
1729 8.92741843045997e-08
1730 8.94202470078653e-08
1731 8.93826191514791e-08
1732 8.93636439798229e-08
1733 8.92120653475104e-08
1734 8.93432654720527e-08
1735 8.93085977509145e-08
1736 8.92781749399774e-08
1737 8.91392450910189e-08
1738 8.92521807003277e-08
1739 8.92322722876315e-08
1740 8.92150060707309e-08
1741 8.91096633495181e-08
1742 8.91610146283028e-08
1743 8.91548936579056e-08
1744 8.91276669676699e-08
1745 8.9116889938623e-08
1746 8.91099052893196e-08
1747 8.90744833981216e-08
1748 8.90502202679499e-08
1749 8.90302799518849e-08
1750 8.90161278732648e-08
1751 8.8933390514967e-08
1752 8.89462408792951e-08
1753 8.89557137497832e-08
1754 8.89350241806142e-08
1755 8.8938394270599e-08
1756 8.88987912475159e-08
1757 8.88805533776349e-08
1758 8.88608343529995e-08
1759 8.88436166235351e-08
1760 8.87489534022734e-08
1761 8.87724770848308e-08
1762 8.87027457565637e-08
1763 8.87193925223073e-08
1764 8.87473152033635e-08
1765 8.8747181351323e-08
1766 8.87133394087414e-08
1767 8.86915344864292e-08
1768 8.868621138447e-08
1769 8.86183897961246e-08
1770 8.86610823016554e-08
1771 8.85634495517706e-08
1772 8.85760772462163e-08
1773 8.8572240727558e-08
1774 8.85606990834731e-08
1775 8.8485433252572e-08
1776 8.85142315496523e-08
1777 8.85076837988663e-08
1778 8.84497463999878e-08
1779 8.84782866990008e-08
1780 8.83915770515387e-08
1781 8.84113736390191e-08
1782 8.84061681603043e-08
1783 8.83645361540175e-08
1784 8.83736433365812e-08
1785 8.83691513422491e-08
1786 8.83282370551797e-08
1787 8.82900787360086e-08
1788 8.82925102416721e-08
1789 8.82605528573777e-08
1790 8.81607781906268e-08
1791 8.82204043364254e-08
1792 8.82238261539214e-08
1793 8.81489452666528e-08
1794 8.82023634574125e-08
1795 8.81548698146162e-08
1796 8.81266963972394e-08
1797 8.81144947868506e-08
1798 8.80757193471027e-08
1799 8.80804776244304e-08
1800 8.80346614593464e-08
1801 8.80416168058673e-08
1802 8.79954678367767e-08
1803 8.80034103545313e-08
1804 8.79604181847071e-08
1805 8.79651102678736e-08
1806 8.79236256707827e-08
1807 8.79309938284223e-08
1808 8.78537629063203e-08
1809 8.79065679413316e-08
1810 8.7866431286443e-08
1811 8.78655854137378e-08
1812 8.78464145586122e-08
1813 8.78576716161206e-08
1814 8.77668272494247e-08
1815 8.77409806605556e-08
1816 8.77508945222871e-08
1817 8.77251944046975e-08
1818 8.77299923445207e-08
1819 8.7688636853045e-08
1820 8.76219724119665e-08
1821 8.76728355905243e-08
1822 8.76390188295773e-08
1823 8.75845233991868e-08
1824 8.75798382793391e-08
1825 8.75226481724667e-08
1826 8.75655295402566e-08
1827 8.75712943013696e-08
1828 8.75445485775117e-08
1829 8.7528533398995e-08
1830 8.7525358367202e-08
1831 8.74752350057406e-08
1832 8.74515263262765e-08
1833 8.74295230772759e-08
1834 8.74177912209007e-08
1835 8.73844009987579e-08
1836 8.74030811530702e-08
1837 8.73475344178587e-08
1838 8.73352884838141e-08
1839 8.73010334885294e-08
1840 8.73268911902869e-08
1841 8.72552679709315e-08
1842 8.71852491854952e-08
1843 8.72374909874907e-08
1844 8.72246575198687e-08
1845 8.72106413396523e-08
1846 8.7161245019729e-08
1847 8.71337331602717e-08
1848 8.70953816018982e-08
1849 8.71150655825659e-08
1850 8.70370515855257e-08
1851 8.70738392251269e-08
1852 8.70367067307143e-08
1853 8.69941564474175e-08
1854 8.70051570416308e-08
1855 8.69511591616856e-08
1856 8.69610328919634e-08
1857 8.69105890330957e-08
1858 8.69233169709105e-08
1859 8.69131752239127e-08
1860 8.67421574639593e-08
1861 8.68709640400311e-08
1862 8.68001533547158e-08
1863 8.67908204895684e-08
1864 8.67871873140302e-08
1865 8.67695091102405e-08
1866 8.67412170748594e-08
1867 8.67633167729309e-08
1868 8.67188217483772e-08
1869 8.66871348108589e-08
1870 8.66836323893949e-08
1871 8.66765750373588e-08
1872 8.66531249670288e-08
1873 8.66319075072397e-08
1874 8.66239634262911e-08
1875 8.65845074713434e-08
1876 8.65780346401834e-08
1877 8.65480916445449e-08
1878 8.65394785307672e-08
1879 8.63740577869976e-08
1880 8.63758578333318e-08
1881 8.65073748173017e-08
1882 8.64575207799589e-08
1883 8.64504155089207e-08
1884 8.62798030567547e-08
1885 8.64216575138244e-08
1886 8.63968362097012e-08
1887 8.62213301786596e-08
1888 8.63675642079898e-08
1889 8.6338496473104e-08
1890 8.63227540719436e-08
1891 8.62891711648217e-08
1892 8.62779216248555e-08
1893 8.62551587488269e-08
1894 8.62300713748709e-08
1895 8.62191824353431e-08
1896 8.61733000903087e-08
1897 8.61789470292251e-08
1898 8.61443233759474e-08
1899 8.59796672330049e-08
1900 8.5989305688372e-08
1901 8.59668569574978e-08
1902 8.60894575964721e-08
1903 8.60412175853753e-08
1904 8.60228532388874e-08
1905 8.59973136613235e-08
1906 8.59822752374839e-08
1907 8.59592924484787e-08
1908 8.5938587290002e-08
1909 8.59249853277788e-08
1910 8.57576113872938e-08
1911 8.57527338524733e-08
1912 8.5912096579932e-08
1913 8.58602724207458e-08
1914 8.58179812723847e-08
1915 8.57903506954472e-08
1916 8.56152141892608e-08
1917 8.57720742573065e-08
1918 8.55984750245398e-08
1919 8.57376611236305e-08
1920 8.55429235429028e-08
1921 8.56469531242965e-08
1922 8.55908580774667e-08
1923 8.55626116589292e-08
1924 8.55295070039119e-08
1925 8.55064072737832e-08
1926 8.54834738248655e-08
1927 8.54635622573596e-08
1928 8.54422478795414e-08
1929 8.54226934876579e-08
1930 8.53965411806712e-08
1931 8.53688236048811e-08
1932 8.53306198536075e-08
1933 8.53140748375836e-08
1934 8.53665414410898e-08
1935 8.52998944509409e-08
1936 8.52584151971314e-08
1937 8.53195163301734e-08
1938 8.52379154991922e-08
1939 8.52655042677952e-08
1940 8.52525136707527e-08
1941 8.52073555392963e-08
1942 8.51245712567561e-08
1943 8.51126393826007e-08
1944 8.50806660963599e-08
1945 8.51261979875062e-08
1946 8.50416321185321e-08
1947 8.50105804062196e-08
1948 8.50535471954572e-08
1949 8.50313311389073e-08
1950 8.49522056682872e-08
1951 8.49181306250557e-08
1952 8.49018624364817e-08
1953 8.49462813334867e-08
1954 8.49442302097714e-08
1955 8.49103634408266e-08
1956 8.47963359973392e-08
1957 8.48286066599257e-08
1958 8.48098083281457e-08
1959 8.47414188882567e-08
1960 8.45925561208105e-08
1961 8.46220291919053e-08
1962 8.4600287962644e-08
1963 8.45689621797874e-08
1964 8.45383896006524e-08
1965 8.45115353058645e-08
1966 8.44845897916002e-08
1967 8.44599360050324e-08
1968 8.44323329687313e-08
1969 8.44082251205691e-08
1970 8.43895726490018e-08
1971 8.43674370543113e-08
1972 8.43429353380998e-08
1973 8.42932384443884e-08
1974 8.42668277272196e-08
1975 8.4239560635524e-08
1976 8.41984005290897e-08
1977 8.42734065855666e-08
1978 8.46509657890238e-08
1979 8.41937617082067e-08
1980 8.39951047169052e-08
1981 8.38162007141818e-08
1982 8.36474396237463e-08
1983 8.3492350043457e-08
1984 8.3382579703084e-08
1985 8.31887337255921e-08
1986 8.30634590585078e-08
1987 8.29407036349039e-08
1988 8.2835952994742e-08
1989 8.25945733282651e-08
1990 8.24664508627393e-08
1991 8.23987021476569e-08
1992 8.22937494291409e-08
1993 8.22063747136781e-08
1994 8.21457147566207e-08
1995 8.20529114804458e-08
1996 8.20931614242681e-08
1997 8.19065507613459e-08
1998 8.17751113970644e-08
1999 8.16509468819504e-08
2000 8.15539570737656e-08
2001 8.14656988126217e-08
2002 8.13650396196408e-08
2003 8.12616535625921e-08
2004 8.11763806183308e-08
2005 8.10545487865966e-08
2006 8.10472479741975e-08
2007 8.0910401464962e-08
2008 8.08447898492659e-08
2009 8.0702835248303e-08
2010 8.06814107932041e-08
2011 8.05895975020121e-08
2012 8.05045714713515e-08
2013 8.04251487380725e-08
2014 8.03642135167593e-08
2015 8.02466255436229e-08
2016 8.01732683299861e-08
2017 8.00334300095074e-08
2018 8.00592033129988e-08
2019 8.00562648493042e-08
2020 8.00025631235712e-08
2021 7.99010928318467e-08
2022 7.98306497955537e-08
2023 7.97824843346007e-08
2024 7.97521115316613e-08
2025 7.96486878016367e-08
2026 7.95713483086047e-08
2027 7.95071539272385e-08
2028 7.94341620746764e-08
2029 7.93721452652107e-08
2030 7.93186951426605e-08
2031 7.92676945309267e-08
2032 7.92078894704673e-08
2033 7.90999925044389e-08
2034 7.90130111454346e-08
2035 7.89481355667476e-08
2036 7.88902891031284e-08
2037 7.88200623276225e-08
2038 7.87548024163698e-08
2039 7.86871285782809e-08
2040 7.86212167014355e-08
2041 7.85621382277668e-08
2042 7.84973410929979e-08
2043 7.84352667579924e-08
2044 7.83733210170112e-08
2045 7.83131427368744e-08
2046 7.82484399906025e-08
2047 7.81920624888244e-08
2048 7.81310421302805e-08
2049 7.80772500945659e-08
2050 7.80173183443367e-08
2051 7.79638218233458e-08
2052 7.79006803384164e-08
2053 7.78570584287763e-08
2054 7.78018525835478e-08
2055 7.774323501053e-08
2056 7.76891531728552e-08
2057 7.76311835437582e-08
2058 7.7585594326024e-08
2059 7.75334057294685e-08
2060 7.74725256320608e-08
2061 7.74955094868801e-08
2062 7.73795027271262e-08
2063 7.72939116728821e-08
2064 7.72193209996885e-08
2065 7.71419097844728e-08
2066 7.70872501476561e-08
2067 7.70477222573618e-08
2068 7.6987128167616e-08
2069 7.68882642319113e-08
2070 7.6780163993817e-08
2071 7.66850297821975e-08
2072 7.65952523664737e-08
2073 7.64994900492866e-08
2074 7.64427162778247e-08
2075 7.63487461057366e-08
2076 7.62631759272381e-08
2077 7.61786611889192e-08
2078 7.61004326506054e-08
2079 7.60205575005557e-08
2080 7.5972070888497e-08
2081 7.58839252057442e-08
2082 7.57994264262152e-08
2083 7.57152979531384e-08
2084 7.56185527421849e-08
2085 7.55412816175749e-08
2086 7.5469325778954e-08
2087 7.53705079574729e-08
2088 7.52968370818508e-08
2089 7.52253518072621e-08
2090 7.51511194465593e-08
2091 7.50801934827905e-08
2092 7.50028884226595e-08
2093 7.49364251930729e-08
2094 7.48725464347899e-08
2095 7.48139173794016e-08
2096 7.46854103113037e-08
2097 7.46311749963979e-08
2098 7.45675028497317e-08
2099 7.44947349744507e-08
2100 7.44214799652809e-08
2101 7.43608823512432e-08
2102 7.42833827018785e-08
2103 7.4208559027511e-08
2104 7.41457163542236e-08
2105 7.40643506276228e-08
2106 7.39821984154787e-08
2107 7.39221072336704e-08
2108 7.38748263273692e-08
2109 7.37619538284662e-08
2110 7.36985232521192e-08
2111 7.36189700489831e-08
2112 7.3512349814564e-08
2113 7.34517429634707e-08
2114 7.33720902701407e-08
2115 7.32941730632319e-08
2116 7.32137326764359e-08
2117 7.31454010178823e-08
2118 7.30837311948562e-08
2119 7.30109619837549e-08
2120 7.29290245971015e-08
2121 7.28660756692534e-08
2122 7.27928213706264e-08
2123 7.27118614918254e-08
2124 7.26268078210523e-08
2125 7.25533796668287e-08
2126 7.24766564559332e-08
2127 7.24086113876865e-08
2128 7.23329537919426e-08
2129 7.22655340581468e-08
2130 7.21840309694244e-08
2131 7.21160463967863e-08
2132 7.20514683649753e-08
2133 7.19973650831207e-08
2134 7.19270660738403e-08
2135 7.18573869278316e-08
2136 7.17741525306792e-08
2137 7.16925680990244e-08
2138 7.16152895421374e-08
2139 7.1543305580235e-08
2140 7.1469762531251e-08
2141 7.14112566129188e-08
2142 7.13404699439479e-08
2143 7.12694511264544e-08
2144 7.1189018214568e-08
2145 7.11192297586649e-08
2146 7.1050252941518e-08
2147 7.10295820454121e-08
2148 7.09216722896144e-08
2149 7.08834199940611e-08
2150 7.07432451463319e-08
2151 7.07348876147762e-08
2152 7.06051837084942e-08
2153 7.0538830186706e-08
2154 7.04669163127392e-08
2155 7.0404663432555e-08
2156 7.03448961019149e-08
2157 7.028209553539e-08
2158 7.02251088142702e-08
2159 7.01293097620237e-08
2160 7.00650285381244e-08
2161 6.99956661236456e-08
2162 6.99354297495347e-08
2163 6.98441183004661e-08
2164 6.98217554742087e-08
2165 6.96966443172187e-08
2166 6.9679617453744e-08
2167 6.95441892162307e-08
2168 6.94867855486336e-08
2169 6.94272745960234e-08
2170 6.93583003226195e-08
2171 6.93009676240308e-08
2172 6.92853524810744e-08
2173 6.91290611314344e-08
2174 6.9119125242878e-08
2175 6.91079354737667e-08
2176 6.89397778046441e-08
2177 6.88656155745093e-08
2178 6.87807929722339e-08
2179 6.86971922760904e-08
2180 6.86214139875574e-08
2181 6.86075825910848e-08
2182 6.85094890826576e-08
2183 6.84377288706628e-08
2184 6.83677963166929e-08
2185 6.81571754910237e-08
2186 6.81241855602366e-08
2187 6.80968410478044e-08
2188 6.80934489167839e-08
2189 6.79648407100331e-08
2190 6.78758579368832e-08
2191 6.78307844026449e-08
2192 6.77785798615105e-08
2193 6.7733886638166e-08
2194 6.76558686620865e-08
2195 6.76002217119276e-08
2196 6.75297964534138e-08
2197 6.75273806791665e-08
2198 6.74621401941522e-08
2199 6.73532910440144e-08
2200 6.72261483458669e-08
2201 6.72027804426989e-08
2202 6.70756337797229e-08
2203 6.69561916879502e-08
2204 6.69106554482823e-08
2205 6.69012050025231e-08
2206 6.67822802853379e-08
2207 6.67251508872368e-08
2208 6.67938199114815e-08
2209 6.66351490536954e-08
2210 6.66889001763593e-08
2211 6.64453007601651e-08
2212 6.63613399041196e-08
2213 6.63120739687884e-08
2214 6.63428044518355e-08
2215 6.61443887111091e-08
2216 6.61017800922536e-08
2217 6.60551605307091e-08
2218 6.6069205331587e-08
2219 6.58781641362793e-08
2220 6.58619877498268e-08
2221 6.58102580928244e-08
2222 6.57633350868991e-08
2223 6.56598793398189e-08
2224 6.5666188120872e-08
2225 6.55799571234184e-08
2226 6.55253490151608e-08
2227 6.54398262511791e-08
2228 6.54700024540489e-08
2229 6.54066397629549e-08
2230 6.53839588622418e-08
2231 6.51138014404751e-08
2232 6.51252891685772e-08
2233 6.50965949660076e-08
2234 6.50443405305623e-08
2235 6.499644168656e-08
2236 6.48423639049156e-08
2237 6.47582903710031e-08
2238 6.46709621463515e-08
2239 6.47059434015773e-08
2240 6.45192303423414e-08
2241 6.45470677227422e-08
2242 6.44347668483647e-08
2243 6.44615467706444e-08
2244 6.43123896679754e-08
2245 6.43443485017769e-08
2246 6.4227537968975e-08
2247 6.42162778277111e-08
2248 6.41202579245714e-08
2249 6.4142173634707e-08
2250 6.39865438571974e-08
2251 6.40133261597953e-08
2252 6.39678183844694e-08
2253 6.38311887897203e-08
2254 6.39113149887294e-08
2255 6.37914179080212e-08
2256 6.369419622132e-08
2257 6.37332994202211e-08
2258 6.35315687063098e-08
2259 6.36018465698385e-08
2260 6.34195237267932e-08
2261 6.34313239800122e-08
2262 6.34354877036003e-08
2263 6.32334039707416e-08
2264 6.32681356975695e-08
2265 6.31607104182308e-08
2266 6.30921163846665e-08
2267 6.30907088350341e-08
2268 6.29942701309005e-08
2269 6.29753450951398e-08
2270 6.2927442201044e-08
2271 6.28794700290314e-08
2272 6.26890692529969e-08
2273 6.27292113790645e-08
2274 6.26192020618532e-08
2275 6.26358039141905e-08
2276 6.25474611766208e-08
2277 6.25679292625136e-08
2278 6.24685381609424e-08
2279 6.25502014202084e-08
2280 6.23230816145792e-08
2281 6.24235809354445e-08
2282 6.22446650453412e-08
2283 6.22598397228558e-08
2284 6.20170334144632e-08
2285 6.21012061756687e-08
2286 6.21090566923499e-08
2287 6.19931108403193e-08
2288 6.2064235670789e-08
2289 6.18704769621559e-08
2290 6.20265865620695e-08
2291 6.19483255235309e-08
2292 6.16488100675383e-08
2293 6.17492356553839e-08
2294 6.1537759869168e-08
2295 6.16599019664932e-08
2296 6.14539281968973e-08
2297 6.15701255455292e-08
2298 6.13590370122097e-08
2299 6.14882676117645e-08
2300 6.12895147540371e-08
2301 6.13997112424158e-08
2302 6.12051512902667e-08
2303 6.13254632142457e-08
2304 6.11345403171981e-08
2305 6.12377655144769e-08
2306 6.10481508260818e-08
2307 6.12093713314721e-08
2308 6.1061637417481e-08
2309 6.10864974959213e-08
2310 6.09399439994718e-08
2311 6.09199653780479e-08
2312 6.10433950001266e-08
2313 6.08700027058262e-08
2314 6.08588325192727e-08
2315 6.07007277366733e-08
2316 6.0659837842536e-08
2317 6.06345500671068e-08
2318 6.0775380028133e-08
2319 6.06863712064865e-08
2320 6.06186361977734e-08
2321 6.05208958788239e-08
2322 6.06406025553952e-08
2323 6.05030151419328e-08
2324 6.05085967393393e-08
2325 6.04032423794365e-08
2326 6.04139951718707e-08
2327 6.02858330651657e-08
2328 6.03384229407311e-08
2329 6.0236835906835e-08
2330 6.02724377358754e-08
2331 6.01441764089827e-08
2332 6.01389578207545e-08
2333 6.00084142519108e-08
2334 5.99649897807808e-08
2335 5.99307988977671e-08
2336 5.98946461778382e-08
2337 5.98614259388341e-08
2338 5.98297309437612e-08
2339 5.98114693346474e-08
2340 5.99082098062809e-08
2341 5.97597449925047e-08
2342 5.96877928842332e-08
2343 5.96866765434356e-08
2344 5.95897538318013e-08
2345 5.95752827905471e-08
2346 5.95648784624814e-08
2347 5.95111078283139e-08
2348 5.94994554461437e-08
2349 5.94664791933042e-08
2350 5.94295912534903e-08
2351 5.94150529522608e-08
2352 5.9386108269166e-08
2353 5.93476926553649e-08
2354 5.93225121292562e-08
2355 5.92932829732717e-08
2356 5.92584087328873e-08
2357 5.92215187751322e-08
2358 5.91877696507481e-08
2359 5.9174814523999e-08
2360 5.91475576641187e-08
2361 5.91168298171851e-08
2362 5.90742835839819e-08
2363 5.91198578163699e-08
2364 5.91366095576973e-08
2365 5.90471969132977e-08
2366 5.90664319588541e-08
2367 5.89966332427139e-08
2368 5.89852892503018e-08
2369 5.89066603282618e-08
2370 5.88150612443883e-08
2371 5.8683367250012e-08
2372 5.86706011631577e-08
2373 5.8665904603572e-08
2374 5.86470078687285e-08
2375 5.86168255622965e-08
2376 5.86019286998862e-08
2377 5.86143878891221e-08
2378 5.8542073830381e-08
2379 5.85072002934339e-08
2380 5.86240853408526e-08
2381 5.86558040893692e-08
2382 5.85355964446421e-08
2383 5.84414040005754e-08
2384 5.85127576968603e-08
2385 5.83482872045238e-08
2386 5.84627034143637e-08
2387 5.82586616033609e-08
2388 5.82348910711516e-08
2389 5.82150405890047e-08
2390 5.81170969269351e-08
2391 5.81170806697173e-08
2392 5.81255914227086e-08
2393 5.81166457536142e-08
2394 5.80967991794523e-08
2395 5.80483242416108e-08
2396 5.80225538158174e-08
2397 5.80040612163657e-08
2398 5.79719976627757e-08
2399 5.79501012083483e-08
2400 5.79765299235646e-08
2401 5.79226664214616e-08
2402 5.7917999292556e-08
2403 5.79124160040578e-08
2404 5.7833111071659e-08
2405 5.78479375406005e-08
2406 5.78369188346528e-08
2407 5.78155501145261e-08
2408 5.77872140539171e-08
2409 5.77542184458935e-08
2410 5.77320365238165e-08
2411 5.77062460180855e-08
2412 5.76891009345104e-08
2413 5.76607420015307e-08
2414 5.76254222366401e-08
2415 5.76134753274005e-08
2416 5.76007374704091e-08
2417 5.76124256710386e-08
2418 5.75107642148964e-08
2419 5.75255294137378e-08
2420 5.74892557736462e-08
2421 5.7476051814831e-08
2422 5.74582429706538e-08
2423 5.74873390561947e-08
2424 5.73929885661073e-08
2425 5.74252579212953e-08
2426 5.73546134603475e-08
2427 5.73695162202625e-08
2428 5.73815637849862e-08
2429 5.7290468170379e-08
2430 5.73023649508286e-08
2431 5.73339895950653e-08
2432 5.72160331699934e-08
2433 5.72237891418581e-08
2434 5.71651468490586e-08
2435 5.71856908209156e-08
2436 5.71783681451166e-08
2437 5.71773216719862e-08
2438 5.70961320889296e-08
2439 5.71123864361311e-08
2440 5.71224443959295e-08
2441 5.70430629522889e-08
2442 5.70554309504701e-08
2443 5.70607695067338e-08
2444 5.69800046434921e-08
2445 5.70010340084082e-08
2446 5.70052390997944e-08
2447 5.69353219361801e-08
2448 5.69740267195584e-08
2449 5.68945835439649e-08
2450 5.69371743068814e-08
2451 5.68561785883048e-08
2452 5.68965704843549e-08
2453 5.68236189266713e-08
2454 5.67909745754491e-08
2455 5.68346303708722e-08
2456 5.67556847670403e-08
2457 5.68057143226497e-08
2458 5.68722038352121e-08
2459 5.6769994756678e-08
2460 5.6764126760811e-08
2461 5.67311799599679e-08
2462 5.67184939157528e-08
2463 5.67102851789514e-08
2464 5.667837190515e-08
2465 5.66585077663717e-08
2466 5.66544482509812e-08
2467 5.66281641383171e-08
2468 5.66289535086639e-08
2469 5.66034327746934e-08
2470 5.65791820790196e-08
2471 5.65528023557249e-08
2472 5.65215423620202e-08
2473 5.6520149414041e-08
2474 5.64943880760893e-08
2475 5.64931256405998e-08
2476 5.64657342181363e-08
2477 5.64659445743132e-08
2478 5.64282161832352e-08
2479 5.64358007011378e-08
2480 5.6401548697238e-08
2481 5.64002695071508e-08
2482 5.63754466114119e-08
2483 5.63683409566806e-08
2484 5.6347937849921e-08
2485 5.63417893388873e-08
2486 5.63114447800217e-08
2487 5.63009105860601e-08
2488 5.62672790849206e-08
2489 5.62662426659699e-08
2490 5.62490389341974e-08
2491 5.62390890053166e-08
2492 5.62141163626961e-08
2493 5.62174431806284e-08
2494 5.61779759351566e-08
2495 5.61818795290492e-08
2496 5.61579811702018e-08
2497 5.61640358114346e-08
2498 5.61207981064626e-08
2499 5.61328391484039e-08
2500 5.61580994116184e-08
2501 5.61500676496962e-08
2502 5.61296075645146e-08
2503 5.61100326024189e-08
2504 5.61055937282617e-08
2505 5.6071772050359e-08
2506 5.60582777566765e-08
2507 5.60347713687293e-08
2508 5.60253382602127e-08
2509 5.60100902333716e-08
2510 5.59867549441151e-08
2511 5.59748818673711e-08
2512 5.59624252716162e-08
2513 5.59382253229046e-08
2514 5.59345436954572e-08
2515 5.59203657317653e-08
2516 5.58995630086656e-08
2517 5.58892723176996e-08
2518 5.58780322919006e-08
2519 5.58620120969522e-08
2520 5.58522340341483e-08
2521 5.58373229182507e-08
2522 5.58229190374959e-08
2523 5.58118504017102e-08
2524 5.58008287043776e-08
2525 5.57816033577296e-08
2526 5.57764256754467e-08
2527 5.57568276491338e-08
2528 5.57511478689321e-08
2529 5.57391973075028e-08
2530 5.57182188813954e-08
2531 5.57043163738058e-08
2532 5.5702749669706e-08
2533 5.56852042521427e-08
2534 5.56727622935682e-08
2535 5.56593322187382e-08
2536 5.56457677021172e-08
2537 5.56315433044574e-08
2538 5.56264248317007e-08
2539 5.56045694750651e-08
2540 5.55988623176518e-08
2541 5.55812954416979e-08
2542 5.55693593398132e-08
2543 5.55568999160982e-08
2544 5.55477682127048e-08
2545 5.55293402513257e-08
2546 5.55259062480218e-08
2547 5.55073090424685e-08
2548 5.55045210646199e-08
2549 5.54921855169255e-08
2550 5.54772445156004e-08
2551 5.54627544815389e-08
2552 5.54593075108301e-08
2553 5.54364005154184e-08
2554 5.54361767370892e-08
2555 5.54132685266495e-08
2556 5.54086654176444e-08
2557 5.53925773019159e-08
2558 5.53828471936413e-08
2559 5.53764061166362e-08
2560 5.53691752287477e-08
2561 5.53471946460604e-08
2562 5.53485110970087e-08
2563 5.53264708074153e-08
2564 5.53222559034339e-08
2565 5.530326422587e-08
2566 5.53024324361218e-08
2567 5.5289317252516e-08
2568 5.52754683980083e-08
2569 5.52586849664749e-08
2570 5.52605863859412e-08
2571 5.52479888469293e-08
2572 5.52355760419232e-08
2573 5.52187259188486e-08
2574 5.52178826325189e-08
2575 5.52036460845784e-08
2576 5.51785828193374e-08
2577 5.51761449614219e-08
2578 5.51801370960447e-08
2579 5.51638303463164e-08
2580 5.51582944297024e-08
2581 5.51325135020875e-08
2582 5.51210570733929e-08
2583 5.51317598649348e-08
2584 5.51067409588768e-08
2585 5.50893756923188e-08
2586 5.50846021525331e-08
2587 5.50645514962866e-08
2588 5.50647844619334e-08
2589 5.50489813875288e-08
2590 5.50526840967791e-08
2591 5.50278762005973e-08
2592 5.50264548166979e-08
2593 5.50331720461372e-08
2594 5.50080872301351e-08
2595 5.49946067138762e-08
2596 5.5004753200194e-08
2597 5.49653531791705e-08
2598 5.49774940807879e-08
2599 5.4933882964292e-08
2600 5.49558998486077e-08
2601 5.49205409825504e-08
2602 5.49337456590138e-08
2603 5.49082107781373e-08
2604 5.49009765578035e-08
2605 5.48833624236522e-08
2606 5.48800588902054e-08
2607 5.4864492696538e-08
2608 5.48695586388703e-08
2609 5.48430298010771e-08
2610 5.48431385496428e-08
2611 5.48314971879904e-08
2612 5.48263131392446e-08
2613 5.48122751169444e-08
2614 5.48046459343254e-08
2615 5.48029697213792e-08
2616 5.47924722269499e-08
2617 5.47817800793382e-08
2618 5.47659774028375e-08
2619 5.4764267034102e-08
2620 5.47371743664371e-08
2621 5.47320111863314e-08
2622 5.47314232761664e-08
2623 5.47429751378559e-08
2624 5.47046202754586e-08
2625 5.4703040738957e-08
2626 5.4685591642567e-08
2627 5.46801665208818e-08
2628 5.46778498957679e-08
2629 5.4689129328267e-08
2630 5.46866897082054e-08
2631 5.467590730035e-08
2632 5.46570608577213e-08
2633 5.46507597576351e-08
2634 5.46423733354118e-08
2635 5.46284007327813e-08
2636 5.46228062319187e-08
2637 5.46212767034149e-08
2638 5.45938012805891e-08
2639 5.45866908012727e-08
2640 5.45849512079144e-08
2641 5.45673687071257e-08
2642 5.45744826183636e-08
2643 5.45629160200178e-08
2644 5.45711437851537e-08
2645 5.45495441386379e-08
2646 5.45505721376571e-08
2647 5.45361528239141e-08
2648 5.45327452101674e-08
2649 5.44933172292872e-08
2650 5.44858729440989e-08
2651 5.44743557924221e-08
2652 5.44754114883972e-08
2653 5.44621391540545e-08
2654 5.4457334492497e-08
2655 5.44708839100849e-08
2656 5.44509888555922e-08
2657 5.44564186171215e-08
2658 5.44511917652812e-08
2659 5.44291765294247e-08
2660 5.44240028830245e-08
2661 5.44175264565183e-08
2662 5.43993687642796e-08
2663 5.43720038166384e-08
2664 5.43906226724289e-08
2665 5.43844863685194e-08
2666 5.43472845393467e-08
2667 5.4344055428146e-08
2668 5.43670215904513e-08
2669 5.43590585806442e-08
2670 5.43499012835014e-08
2671 5.43093061153854e-08
2672 5.43028946680124e-08
2673 5.43307161535722e-08
2674 5.43180986340985e-08
2675 5.43150286986815e-08
2676 5.42958195168808e-08
2677 5.42861836905217e-08
2678 5.42878332439045e-08
2679 5.42379730674725e-08
2680 5.42521187512079e-08
2681 5.42334170248182e-08
2682 5.42437366135573e-08
2683 5.42261362141971e-08
2684 5.42348911665158e-08
2685 5.41995364216064e-08
2686 5.42177805442634e-08
2687 5.42045605129715e-08
2688 5.41978103427709e-08
2689 5.41933109090564e-08
2690 5.41764339345718e-08
2691 5.41780274332382e-08
2692 5.41646485743286e-08
2693 5.41669423341773e-08
2694 5.41637933437755e-08
2695 5.41512400360489e-08
2696 5.41352113856419e-08
2697 5.41173976387199e-08
2698 5.41164585641241e-08
2699 5.4099521776152e-08
2700 5.40975416427614e-08
2701 5.40605802399341e-08
2702 5.40867928435773e-08
2703 5.40345348696292e-08
2704 5.40438517191433e-08
2705 5.41328363325988e-08
2706 5.40777339352871e-08
2707 5.40412549696612e-08
2708 5.40354568627777e-08
2709 5.40013044911802e-08
2710 5.40110464513077e-08
2711 5.40019711721129e-08
2712 5.39682864371116e-08
2713 5.40028473068332e-08
2714 5.39680741056259e-08
2715 5.40244550535363e-08
2716 5.3997889537527e-08
2717 5.39925839362354e-08
2718 5.39613293781827e-08
2719 5.39727547561597e-08
2720 5.39540151436313e-08
2721 5.38331739647901e-08
2722 5.37660886550384e-08
2723 5.37858925468981e-08
2724 5.36711132852474e-08
2725 5.37235993718355e-08
2726 5.35994649979443e-08
2727 5.36234271777403e-08
2728 5.35597760062956e-08
2729 5.35183191914257e-08
2730 5.34660780289187e-08
2731 5.3430237755947e-08
2732 5.3417237822373e-08
2733 5.338370959862e-08
2734 5.33297577476333e-08
2735 5.32982990790742e-08
2736 5.33322444198347e-08
2737 5.3264600495595e-08
2738 5.32478057380104e-08
2739 5.32043012739791e-08
2740 5.3156373368779e-08
2741 5.31253308508894e-08
2742 5.31150687379522e-08
2743 5.31423339964476e-08
2744 5.30761930974677e-08
2745 5.30462611081361e-08
2746 5.30352429350955e-08
2747 5.29923428302936e-08
2748 5.29537310214323e-08
2749 5.2915529835218e-08
2750 5.29014466792432e-08
2751 5.28161222561607e-08
2752 5.27910796108699e-08
2753 5.27316102321151e-08
2754 5.27449722653728e-08
2755 5.27007900288368e-08
2756 5.26581948037119e-08
2757 5.26600436359104e-08
2758 5.26202819983723e-08
2759 5.26036441286237e-08
2760 5.25498253054479e-08
2761 5.25742374790639e-08
2762 5.25361106085143e-08
2763 5.25070618664358e-08
2764 5.25060299239044e-08
2765 5.25140748379727e-08
2766 5.24598281472777e-08
2767 5.24405100250647e-08
2768 5.24331498326092e-08
2769 5.24090427873602e-08
2770 5.23810431545257e-08
2771 5.24253375289163e-08
2772 5.23724493817213e-08
2773 5.23576249378266e-08
2774 5.23763444135739e-08
2775 5.23133226124628e-08
2776 5.22835796346044e-08
2777 5.22786057288727e-08
2778 5.23152757949674e-08
2779 5.2244280738023e-08
2780 5.22432521634642e-08
2781 5.22652572314541e-08
2782 5.22369808635403e-08
2783 5.22472622819237e-08
2784 5.21760496994261e-08
2785 5.22301009411308e-08
2786 5.21607086412246e-08
2787 5.22050695295206e-08
2788 5.21036978256006e-08
2789 5.21938025173085e-08
2790 5.20744023120301e-08
2791 5.21593596758407e-08
2792 5.20599122424414e-08
2793 5.2156642240675e-08
2794 5.20481876193912e-08
2795 5.21066489511668e-08
2796 5.1981618867103e-08
2797 5.20793912954787e-08
2798 5.19931081441882e-08
2799 5.20827497254572e-08
2800 5.19355498411755e-08
2801 5.20377196764343e-08
2802 5.19510841030524e-08
2803 5.19998075674266e-08
2804 5.18905986623963e-08
2805 5.19411194588315e-08
2806 5.19357524524366e-08
2807 5.19329671888613e-08
2808 5.18635284976199e-08
2809 5.18481317612895e-08
2810 5.19562578844557e-08
2811 5.1837901914098e-08
2812 5.19033669803548e-08
2813 5.18065182859573e-08
2814 5.18044075974444e-08
2815 5.19226354995794e-08
2816 5.1793253639687e-08
2817 5.18258627622004e-08
2818 5.1799442509548e-08
2819 5.17251954335052e-08
2820 5.17841472387204e-08
2821 5.17933482839794e-08
2822 5.17449350923016e-08
2823 5.16704170863136e-08
2824 5.17693128614383e-08
2825 5.1813226477293e-08
2826 5.17642904540594e-08
2827 5.17248923443958e-08
2828 5.16965724841612e-08
2829 5.16706238897768e-08
2830 5.16798883580805e-08
2831 5.16286443001945e-08
2832 5.16514733561735e-08
2833 5.17072909289595e-08
2834 5.16090315016982e-08
2835 5.16040826852304e-08
2836 5.16339873897209e-08
2837 5.16434839852309e-08
2838 5.1555669500658e-08
2839 5.1553139890359e-08
2840 5.15201010102828e-08
2841 5.15280134365526e-08
2842 5.15904770495013e-08
2843 5.14672794835747e-08
2844 5.15183409888209e-08
2845 5.14859120741562e-08
2846 5.14691369133402e-08
2847 5.15070275071139e-08
2848 5.14499458503792e-08
2849 5.15014608808428e-08
2850 5.13736985396918e-08
2851 5.14172093701859e-08
2852 5.13862762048234e-08
2853 5.14152617654418e-08
2854 5.14258707795534e-08
2855 5.13771848034139e-08
2856 5.1381589692312e-08
2857 5.1349184964522e-08
2858 5.14087845573386e-08
2859 5.14052765083761e-08
2860 5.12648102102276e-08
2861 5.13526265066844e-08
2862 5.13225803757678e-08
2863 5.13485674105141e-08
2864 5.13234341497082e-08
2865 5.12165031025802e-08
2866 5.12514701611622e-08
2867 5.12302774922091e-08
2868 5.12698196146744e-08
2869 5.12973551778373e-08
2870 5.11994473981758e-08
2871 5.12665988239291e-08
2872 5.11817745874055e-08
2873 5.12314003415781e-08
2874 5.12587179173352e-08
2875 5.12411246873512e-08
2876 5.12025586800746e-08
2877 5.11997348269233e-08
2878 5.11494091739451e-08
2879 5.1084724297823e-08
2880 5.12000310024519e-08
2881 5.10872082486458e-08
2882 5.11980485455865e-08
2883 5.1054293841446e-08
2884 5.11824758220314e-08
2885 5.10694109578935e-08
2886 5.1109299079144e-08
2887 5.11298179688424e-08
2888 5.10399576185705e-08
2889 5.10730377314417e-08
2890 5.10377277223029e-08
2891 5.1057939167265e-08
2892 5.10096635579771e-08
2893 5.10697691780138e-08
2894 5.11090764376831e-08
2895 5.10144085765774e-08
2896 5.10067262027292e-08
2897 5.10231967751906e-08
2898 5.09943155151404e-08
2899 5.10012500996027e-08
2900 5.09847950098674e-08
2901 5.09090957834246e-08
2902 5.09650226945268e-08
2903 5.10580969503849e-08
2904 5.09061498519259e-08
2905 5.09156832464441e-08
2906 5.0928909146819e-08
2907 5.09612043089192e-08
2908 5.0938365838249e-08
2909 5.09724449031523e-08
2910 5.08359630870814e-08
2911 5.08669017662555e-08
2912 5.09790877387672e-08
2913 5.08364779747694e-08
2914 5.0858794864439e-08
2915 5.08626910189491e-08
2916 5.08517108315232e-08
2917 5.09141113127498e-08
2918 5.08061570272389e-08
2919 5.08076269412072e-08
2920 5.08708778141909e-08
2921 5.08543345034695e-08
2922 5.07678450318849e-08
2923 5.08599674304833e-08
2924 5.07746872386861e-08
2925 5.07362094381847e-08
2926 5.07134806824183e-08
2927 5.07748874056801e-08
2928 5.08906371052831e-08
2929 5.06885867466167e-08
2930 5.07258087054652e-08
2931 5.08441525894909e-08
2932 5.0772445447933e-08
2933 5.07352311842624e-08
2934 5.07491886807543e-08
2935 5.06871272278886e-08
2936 5.06824670765127e-08
2937 5.06401004827239e-08
2938 5.07268547735862e-08
2939 5.06268373428043e-08
2940 5.07234833264647e-08
2941 5.06301428160327e-08
2942 5.06452846593675e-08
2943 5.07488061529671e-08
2944 5.05920519628944e-08
2945 5.06550953218721e-08
2946 5.05785235347389e-08
2947 5.06065833505431e-08
2948 5.07243922029943e-08
2949 5.05279167128947e-08
2950 5.06264459758654e-08
2951 5.05268813810744e-08
2952 5.06462122302764e-08
2953 5.05237122538915e-08
2954 5.05129294552376e-08
2955 5.06897495142766e-08
2956 5.04967959997771e-08
2957 5.06070822225979e-08
2958 5.04813185386865e-08
2959 5.04795215405807e-08
2960 5.06080787801011e-08
2961 5.04929434583801e-08
2962 5.0475605824829e-08
2963 5.04492815665003e-08
2964 5.05952473943694e-08
2965 5.04617557979259e-08
2966 5.04373672143288e-08
2967 5.04171366699779e-08
2968 5.05766105192151e-08
2969 5.04301542392227e-08
2970 5.03966550127188e-08
2971 5.0394213907623e-08
2972 5.05705103535092e-08
2973 5.03901834747467e-08
2974 5.03663397424248e-08
2975 5.0369742126577e-08
2976 5.03482526710286e-08
2977 5.05182689565231e-08
2978 5.03439154471152e-08
2979 5.03456051745843e-08
2980 5.04818880244784e-08
2981 5.03793677495423e-08
2982 5.03109765119802e-08
2983 5.04378374088787e-08
2984 5.02712143202189e-08
2985 5.03996760770065e-08
2986 5.02870976148984e-08
2987 5.02599521823299e-08
2988 5.0280112716905e-08
2989 5.03998247864956e-08
2990 5.02168864002783e-08
2991 5.03556204520805e-08
2992 5.02455765456489e-08
2993 5.02380493401233e-08
2994 5.03377507001801e-08
2995 5.03491585135407e-08
2996 5.01873786618035e-08
2997 5.01980643434763e-08
2998 5.02859376823039e-08
2999 5.01998415955995e-08
3000 5.01808392030512e-08
3001 5.02791442684725e-08
3002 5.01546386075802e-08
3003 5.01387306712786e-08
3004 5.02858379149984e-08
3005 5.01776764281203e-08
3006 5.01626923608001e-08
3007 5.02508216655428e-08
3008 5.01277946014511e-08
3009 5.01469195910431e-08
3010 5.02894052587521e-08
3011 5.01241327981461e-08
3012 5.01141522661896e-08
3013 5.02312445931352e-08
3014 5.00800081937314e-08
3015 5.01087888267193e-08
3016 5.00903012436993e-08
3017 5.01827817060985e-08
3018 5.00535444629691e-08
3019 5.01939629700132e-08
3020 5.00384399373388e-08
3021 5.00732689445726e-08
3022 5.01792918328192e-08
3023 5.00881993446001e-08
3024 5.01337317402317e-08
3025 5.00399829377329e-08
3026 4.99823544544142e-08
3027 5.00330587058784e-08
3028 4.99737172603432e-08
3029 5.00439641584194e-08
3030 4.99564863289947e-08
3031 5.0037197780739e-08
3032 4.9972364266182e-08
3033 5.00355509700512e-08
3034 4.99389356960478e-08
3035 5.00503501470462e-08
3036 4.99380459828558e-08
3037 5.00177203903718e-08
3038 4.99157317435106e-08
3039 4.99997797973606e-08
3040 4.9907556189055e-08
3041 4.99917273515393e-08
3042 4.99040237826875e-08
3043 4.99732652414764e-08
3044 4.98670184541083e-08
3045 4.99665750766098e-08
3046 4.98650410136747e-08
3047 4.99561170386187e-08
3048 4.98636011911913e-08
3049 4.9966576824545e-08
3050 4.98154868466827e-08
3051 4.99689594661845e-08
3052 4.98285839682922e-08
3053 4.99367090043279e-08
3054 4.97818431597352e-08
3055 4.99175074040181e-08
3056 4.9772548038618e-08
3057 4.99175019257336e-08
3058 4.97791652165347e-08
3059 4.98975493741227e-08
3060 4.97379109631879e-08
3061 4.98874320342679e-08
3062 4.97270111736725e-08
3063 4.98762835690059e-08
3064 4.97431797583658e-08
3065 4.98700930648965e-08
3066 4.97150583242956e-08
3067 4.98558944173055e-08
3068 4.97270664823191e-08
3069 4.98335920511295e-08
3070 4.96785515906595e-08
3071 4.98217810758206e-08
3072 4.96692308971092e-08
3073 4.98082198490124e-08
3074 4.96464737125279e-08
3075 4.97894595881121e-08
3076 4.96540733152528e-08
3077 4.97822568590323e-08
3078 4.96204591300398e-08
3079 4.97709572329086e-08
3080 4.96245299217435e-08
3081 4.97749998444874e-08
3082 4.95955823467398e-08
3083 4.97395625203012e-08
3084 4.957363422875e-08
3085 4.9735038551546e-08
3086 4.95612307958027e-08
3087 4.9723418278802e-08
3088 4.95567494951388e-08
3089 4.96609109887913e-08
3090 4.95816444185948e-08
3091 4.96274028876087e-08
3092 4.95551814481132e-08
3093 4.96565914787084e-08
3094 4.95085009291074e-08
3095 4.96395472282529e-08
3096 4.95147012813391e-08
3097 4.9633545906147e-08
3098 4.94743662002861e-08
3099 4.96371045457522e-08
3100 4.9468592756341e-08
3101 4.96143280628303e-08
3102 4.94535728421397e-08
3103 4.9618041934707e-08
3104 4.94378924145167e-08
3105 4.95527994459621e-08
3106 4.94370097356978e-08
3107 4.95912868601067e-08
3108 4.94405189499503e-08
3109 4.9577241313159e-08
3110 4.93919519755082e-08
3111 4.95525613217751e-08
3112 4.9380705462454e-08
3113 4.95611534745422e-08
3114 4.93662759026847e-08
3115 4.95153232549228e-08
3116 4.93705085702345e-08
3117 4.95167187182233e-08
3118 4.93412552344807e-08
3119 4.94970573683418e-08
3120 4.93295680144001e-08
3121 4.94941199775667e-08
3122 4.93212963803558e-08
3123 4.9474175426667e-08
3124 4.93047955458792e-08
3125 4.94603935976556e-08
3126 4.92925416750722e-08
3127 4.94492499569787e-08
3128 4.92744890507879e-08
3129 4.94356836640009e-08
3130 4.92662484674611e-08
3131 4.94439142570968e-08
3132 4.92565963554625e-08
3133 4.93857050329893e-08
3134 4.92364730959594e-08
3135 4.93834908183999e-08
3136 4.92364930479994e-08
3137 4.93779830037511e-08
3138 4.92151184872114e-08
3139 4.93450452196953e-08
3140 4.9223308522528e-08
3141 4.93390909568348e-08
3142 4.92035192536378e-08
3143 4.93163070700575e-08
3144 4.91757810436866e-08
3145 4.93132536902863e-08
3146 4.91646585913941e-08
3147 4.93003070971554e-08
3148 4.91511779188158e-08
3149 4.92688760331816e-08
3150 4.91430141131843e-08
3151 4.92743768205628e-08
3152 4.91229707932916e-08
3153 4.92523231585551e-08
3154 4.91140594185424e-08
3155 4.91789921355235e-08
3156 4.91625641707572e-08
3157 4.92188052163556e-08
3158 4.90927296468158e-08
3159 4.92038241155512e-08
3160 4.9071141539514e-08
3161 4.9199741170014e-08
3162 4.9066635945394e-08
3163 4.91601259184904e-08
3164 4.90425233437009e-08
3165 4.91661337811422e-08
3166 4.90421672871832e-08
3167 4.91765141816813e-08
3168 4.90164636666179e-08
3169 4.91295623881172e-08
3170 4.90008328490887e-08
3171 4.91243021478738e-08
3172 4.90090761431361e-08
3173 4.91133280391409e-08
3174 4.89743187195302e-08
3175 4.90922767326651e-08
3176 4.89647847814467e-08
3177 4.90796488783474e-08
3178 4.89595978017121e-08
3179 4.90551238812031e-08
3180 4.89404888952549e-08
3181 4.89784760020484e-08
3182 4.9018563309744e-08
3183 4.89658397526682e-08
3184 4.90266607897638e-08
3185 4.89716483897951e-08
3186 4.89570796986527e-08
3187 4.89540304968727e-08
3188 4.89449307963241e-08
3189 4.89181251452919e-08
3190 4.89134158243587e-08
3191 4.89132075642829e-08
3192 4.88903797517537e-08
3193 4.89071728537738e-08
3194 4.89002904302538e-08
3195 4.88689923656693e-08
3196 4.88759332952782e-08
3197 4.88203823181266e-08
3198 4.8810810536537e-08
3199 4.88174555144383e-08
3200 4.88116804042704e-08
3201 4.88182140792048e-08
3202 4.87937232875879e-08
3203 4.87975304324095e-08
3204 4.88181437638957e-08
3205 4.88587571112475e-08
3206 4.8772792453633e-08
3207 4.87959874391208e-08
3208 4.8757216514872e-08
3209 4.87636390467117e-08
3210 4.87835346163479e-08
3211 4.8743934890183e-08
3212 4.87969032221258e-08
3213 4.87778013393836e-08
3214 4.87901854597794e-08
3215 4.87646064115665e-08
3216 4.87485456872605e-08
3217 4.87338576462548e-08
3218 4.87413446563778e-08
3219 4.86952337368507e-08
3220 4.87021771249374e-08
3221 4.87013971692818e-08
3222 4.86850638949932e-08
3223 4.87117363263678e-08
3224 4.86949300650963e-08
3225 4.8668249775119e-08
3226 4.86669699597542e-08
3227 4.8673963028989e-08
3228 4.85993622874048e-08
3229 4.86569355011568e-08
3230 4.85794479843094e-08
3231 4.86645130095553e-08
3232 4.85614805008083e-08
3233 4.85883813965415e-08
3234 4.86561666832586e-08
3235 4.85362955018331e-08
3236 4.86624450459772e-08
3237 4.85497946485225e-08
3238 4.85524812674498e-08
3239 4.86130296586396e-08
3240 4.8516963794043e-08
3241 4.8613456939961e-08
3242 4.84604684487522e-08
3243 4.85421978879685e-08
3244 4.85289530089972e-08
3245 4.84742729440768e-08
3246 4.84860178637803e-08
3247 4.84957198523261e-08
3248 4.85745595675269e-08
3249 4.84700829304074e-08
3250 4.85774925387261e-08
3251 4.84774402735866e-08
3252 4.8442552945005e-08
3253 4.84334486436921e-08
3254 4.84268057761028e-08
3255 4.85373824616886e-08
3256 4.84160899141273e-08
3257 4.84302206196219e-08
3258 4.84104850428935e-08
3259 4.84108010780915e-08
3260 4.83960360320168e-08
3261 4.84322828242512e-08
3262 4.83953744243593e-08
3263 4.8346354120099e-08
3264 4.83811860583216e-08
3265 4.83556149788456e-08
3266 4.83153131298764e-08
3267 4.83133010718007e-08
3268 4.82974099647038e-08
3269 4.82968132686779e-08
3270 4.83393711796509e-08
3271 4.83142187270857e-08
3272 4.82900200857728e-08
3273 4.82778585286781e-08
3274 4.82901978351435e-08
3275 4.82883820893676e-08
3276 4.82768903893316e-08
3277 4.82614095140832e-08
3278 4.82534005818991e-08
3279 4.82356465738576e-08
3280 4.82415205027564e-08
3281 4.82357046678317e-08
3282 4.82166378397153e-08
3283 4.82211504078123e-08
3284 4.82105243300168e-08
3285 4.82024275143544e-08
3286 4.82063782705211e-08
3287 4.81936675136296e-08
3288 4.82017889815722e-08
3289 4.81936957470452e-08
3290 4.81880487406272e-08
3291 4.81769581597291e-08
3292 4.81193722130513e-08
3293 4.81797736888723e-08
3294 4.81311357596326e-08
3295 4.81954311233324e-08
3296 4.8120324777301e-08
3297 4.81266018113047e-08
3298 4.81151931879253e-08
3299 4.81024633636196e-08
3300 4.81049747769191e-08
3301 4.8101718000737e-08
3302 4.80353691827418e-08
3303 4.80306024748245e-08
3304 4.80443453767521e-08
3305 4.80469848547216e-08
3306 4.79983483359092e-08
3307 4.80848342228057e-08
3308 4.80331235124254e-08
3309 4.80024770190823e-08
3310 4.80070265638233e-08
3311 4.79917738616109e-08
3312 4.80319685252084e-08
3313 4.80299254093097e-08
3314 4.79865184139783e-08
3315 4.79784208948786e-08
3316 4.79637757244689e-08
3317 4.79416481375949e-08
3318 4.79456079780505e-08
3319 4.79013816914176e-08
3320 4.7885994003849e-08
3321 4.79151314394244e-08
3322 4.79251325842256e-08
3323 4.79116942955216e-08
3324 4.78903585552359e-08
3325 4.78801324064193e-08
3326 4.78479773278195e-08
3327 4.78608841163464e-08
3328 4.78242267654139e-08
3329 4.7819327534171e-08
3330 4.78344670682418e-08
3331 4.78456379440217e-08
3332 4.78484234847087e-08
3333 4.78328895496816e-08
3334 4.77910334524267e-08
3335 4.7788533841242e-08
3336 4.77886740668509e-08
3337 4.77939488590096e-08
3338 4.77721813751941e-08
3339 4.77517139607642e-08
3340 4.7739070936359e-08
3341 4.77640534768398e-08
3342 4.77574730552988e-08
3343 4.77247611669895e-08
3344 4.77109993610725e-08
3345 4.77238010745396e-08
3346 4.77197449484379e-08
3347 4.77056701129186e-08
3348 4.76704370839798e-08
3349 4.76710843173578e-08
3350 4.76769803370303e-08
3351 4.76986706843263e-08
3352 4.7655606834951e-08
3353 4.76297696252459e-08
3354 4.76347322049264e-08
3355 4.76433739109439e-08
3356 4.76467889143351e-08
3357 4.76290535260659e-08
3358 4.76155980706494e-08
3359 4.76248890848296e-08
3360 4.76005888359055e-08
3361 4.76151596942032e-08
3362 4.76095454260417e-08
3363 4.75797645549392e-08
3364 4.75785125644279e-08
3365 4.75921845932703e-08
3366 4.7554093839608e-08
3367 4.75566575772746e-08
3368 4.7554317980314e-08
3369 4.75427945971774e-08
3370 4.75526890042488e-08
3371 4.75465096698713e-08
3372 4.75158549022581e-08
3373 4.75201210505816e-08
3374 4.75159839155026e-08
3375 4.75091259701799e-08
3376 4.75969126298992e-08
3377 4.75261484211842e-08
3378 4.74943323141019e-08
3379 4.74758187465341e-08
3380 4.74617305528113e-08
3381 4.74756684454292e-08
3382 4.74567139505666e-08
3383 4.74498236826548e-08
3384 4.74538608372654e-08
3385 4.74484597248193e-08
3386 4.74243199590774e-08
3387 4.74225379534232e-08
3388 4.7517900199523e-08
3389 4.74102443845936e-08
3390 4.73995172001196e-08
3391 4.73874085713533e-08
3392 4.73756950256643e-08
3393 4.73761260479932e-08
3394 4.74853263199293e-08
3395 4.7423007778491e-08
3396 4.73782192358385e-08
3397 4.73541918495357e-08
3398 4.73573048083153e-08
3399 4.73415560264812e-08
3400 4.7426898660774e-08
3401 4.74104646386309e-08
3402 4.73352958252349e-08
3403 4.73080213048149e-08
3404 4.73064085539931e-08
3405 4.72876689840973e-08
3406 4.72837700797868e-08
3407 4.72781332589989e-08
3408 4.73245018781654e-08
3409 4.72764318999452e-08
3410 4.72782743869971e-08
3411 4.72620175315797e-08
3412 4.72527058548167e-08
3413 4.72531760280503e-08
3414 4.72428426050442e-08
3415 4.7244692524373e-08
3416 4.73409576073891e-08
3417 4.72538130580347e-08
3418 4.72264021240676e-08
3419 4.72409054950163e-08
3420 4.72104699440479e-08
3421 4.71888838831092e-08
3422 4.73162161753749e-08
3423 4.73317622748937e-08
3424 4.72727126137329e-08
3425 4.73003120546878e-08
3426 4.72800045301369e-08
3427 4.72103612310093e-08
3428 4.72193595371095e-08
3429 4.71899120526587e-08
3430 4.72645026334817e-08
3431 4.72673084104258e-08
3432 4.71655135356741e-08
3433 4.71731094044969e-08
3434 4.72230065575729e-08
3435 4.72294937949869e-08
3436 4.7195624780727e-08
3437 4.71812405251626e-08
3438 4.72055164308927e-08
3439 4.72077615043531e-08
3440 4.71975872642361e-08
3441 4.72064081975532e-08
3442 4.71917261606336e-08
3443 4.71736682676749e-08
3444 4.71592973703139e-08
3445 4.71627713274358e-08
3446 4.7147872635378e-08
3447 4.71564974979799e-08
3448 4.71564501047794e-08
3449 4.7139887065839e-08
3450 4.71328399704873e-08
3451 4.71651280520291e-08
3452 4.71052450983223e-08
3453 4.71249067537371e-08
3454 4.70791300308804e-08
3455 4.71082122572852e-08
3456 4.70883324901195e-08
3457 4.71165803084261e-08
3458 4.7081251082659e-08
3459 4.70926065858635e-08
3460 4.70418755575963e-08
3461 4.7071029914747e-08
3462 4.70580504092766e-08
3463 4.70415727633622e-08
3464 4.70688260350016e-08
3465 4.70539691121985e-08
3466 4.70179744631594e-08
3467 4.70361452933332e-08
3468 4.70013776947553e-08
3469 4.70351527752655e-08
3470 4.7004087537772e-08
3471 4.70274200488063e-08
3472 4.69764649473348e-08
3473 4.69853909983442e-08
3474 4.70024352559051e-08
3475 4.70180622791361e-08
3476 4.70374858210221e-08
3477 4.69860156435686e-08
3478 4.70108807846259e-08
3479 4.70030119714693e-08
3480 4.69823863582519e-08
3481 4.69200706092465e-08
3482 4.70074099077067e-08
3483 4.70137490786726e-08
3484 4.70334162798736e-08
3485 4.6989345115378e-08
3486 4.70021104383989e-08
3487 4.70095297444573e-08
3488 4.69981098518701e-08
3489 4.70070889235785e-08
3490 4.69970937047037e-08
3491 4.69904839803803e-08
3492 4.7006317416276e-08
3493 4.69991386040647e-08
3494 4.69853513180851e-08
3495 4.69832987093355e-08
3496 4.69905070659138e-08
3497 4.69909615006259e-08
3498 4.69751705907129e-08
3499 4.6972504613052e-08
3500 4.69945116634562e-08
3501 4.69680429091568e-08
3502 4.69706880039666e-08
3503 4.69639428004598e-08
3504 4.69530136797403e-08
3505 4.69591494152155e-08
3506 4.69542194210248e-08
3507 4.69607122539628e-08
3508 4.69480595128857e-08
3509 4.694157536278e-08
3510 4.6967790190422e-08
3511 4.6937143025616e-08
3512 4.69334169963531e-08
3513 4.69206993471971e-08
3514 4.6926287652127e-08
3515 4.69157779363627e-08
3516 4.69303537897758e-08
3517 4.69140647396671e-08
3518 4.69170732699808e-08
3519 4.6898006708318e-08
3520 4.6906048758899e-08
3521 4.6895157439053e-08
3522 4.69261227920015e-08
3523 4.68879997868044e-08
3524 4.6890950471834e-08
3525 4.68852354345017e-08
3526 4.68867626182146e-08
3527 4.68750210984581e-08
3528 4.68994170219617e-08
3529 4.68712426737738e-08
3530 4.68760377074773e-08
3531 4.68673954401311e-08
3532 4.68659960972673e-08
3533 4.68695856810086e-08
3534 4.68568441220896e-08
3535 4.68671336832926e-08
3536 4.68500185206722e-08
3537 4.68580158070608e-08
3538 4.6841477342241e-08
3539 4.68533441733143e-08
3540 4.68619515530122e-08
3541 4.68557488915167e-08
3542 4.68293786255458e-08
3543 4.68328756824121e-08
3544 4.68192922369326e-08
3545 4.68334410399507e-08
3546 4.68213614084334e-08
3547 4.68352628644197e-08
3548 4.68283192134322e-08
3549 4.68143447491798e-08
3550 4.68102705397655e-08
3551 4.68150959065383e-08
3552 4.68211533188878e-08
3553 4.682003157086e-08
3554 4.67966193795633e-08
3555 4.67941393438309e-08
3556 4.67945485240762e-08
3557 4.67937242376593e-08
3558 4.68051768294231e-08
3559 4.68004275475664e-08
3560 4.67784096755963e-08
3561 4.67749572052867e-08
3562 4.67837356978862e-08
3563 4.67779814670166e-08
3564 4.67952876448408e-08
3565 4.67691452570307e-08
3566 4.67776285049126e-08
3567 4.67602386962085e-08
3568 4.6773445319559e-08
3569 4.67738243372651e-08
3570 4.67668648909125e-08
3571 4.6764058275528e-08
3572 4.67594844835162e-08
3573 4.67608502177086e-08
3574 4.67602377440812e-08
3575 4.67476201180261e-08
3576 4.67444840026587e-08
3577 4.67596544240223e-08
3578 4.6747138981118e-08
3579 4.67351279667128e-08
3580 4.67386848868045e-08
3581 4.67489064135407e-08
3582 4.67278440865471e-08
3583 4.67263289110065e-08
3584 4.67328002287104e-08
3585 4.6736541399639e-08
3586 4.67307257494554e-08
3587 4.6713512276142e-08
3588 4.67156504910804e-08
3589 4.67113240105732e-08
3590 4.6769561421911e-08
3591 4.66969370549464e-08
3592 4.67048327266184e-08
3593 4.66990494771835e-08
3594 4.66966040093553e-08
3595 4.6702954648481e-08
3596 4.66974106885232e-08
3597 4.67025117103503e-08
3598 4.66992143799416e-08
3599 4.66927999696054e-08
3600 4.67159793942073e-08
3601 4.66778854288918e-08
3602 4.66753531682684e-08
3603 4.66731591330927e-08
3604 4.66843224984359e-08
3605 4.66977579876016e-08
3606 4.66674359884678e-08
3607 4.66676559085499e-08
3608 4.66600286372909e-08
3609 4.66696315797321e-08
3610 4.66707879311912e-08
3611 4.66610190130723e-08
3612 4.66625030739465e-08
3613 4.66519253521369e-08
3614 4.66446120839237e-08
3615 4.66610254861166e-08
3616 4.66577515751965e-08
3617 4.66423368408186e-08
3618 4.66336396200973e-08
3619 4.66387770217125e-08
3620 4.66449219160836e-08
3621 4.66409111083976e-08
3622 4.66232079574524e-08
3623 4.66473036340176e-08
3624 4.66537052901117e-08
3625 4.66208365637044e-08
3626 4.66369264842115e-08
3627 4.66144673012536e-08
3628 4.6614991987326e-08
3629 4.66104142873292e-08
3630 4.66044539635391e-08
3631 4.66234760168049e-08
3632 4.66075901002227e-08
3633 4.66070449505196e-08
3634 4.6616582636716e-08
3635 4.65986990221268e-08
3636 4.66228507534083e-08
3637 4.66009918937971e-08
3638 4.65874554294032e-08
3639 4.65990896856283e-08
3640 4.65889460201652e-08
3641 4.66122949731584e-08
3642 4.65846261832326e-08
3643 4.65827225397675e-08
3644 4.6604827275587e-08
3645 4.66093786570809e-08
3646 4.65769358370949e-08
3647 4.65814132866171e-08
3648 4.65655232204654e-08
3649 4.65917360443768e-08
3650 4.65658467589947e-08
3651 4.65801767290941e-08
3652 4.65707713530605e-08
3653 4.65736131687322e-08
3654 4.65626338836955e-08
3655 4.65779208980166e-08
3656 4.65711214872044e-08
3657 4.65573201893221e-08
3658 4.65620692153834e-08
3659 4.65652906456171e-08
3660 4.65422921891445e-08
3661 4.65806843621408e-08
3662 4.65424765252465e-08
3663 4.65447031885446e-08
3664 4.65586072664337e-08
3665 4.65514551351021e-08
3666 4.65345748921209e-08
3667 4.65365158888176e-08
3668 4.65307655943548e-08
3669 4.6557733121233e-08
3670 4.6529505780768e-08
3671 4.65225883630183e-08
3672 4.65270584584232e-08
3673 4.65172657797552e-08
3674 4.65447626467608e-08
3675 4.65137652625458e-08
3676 4.65180412447808e-08
3677 4.65113445997645e-08
3678 4.65057967957705e-08
3679 4.65321823881482e-08
3680 4.65015969979277e-08
3681 4.65067321968604e-08
3682 4.64969618150235e-08
3683 4.64949229197487e-08
3684 4.65281320742861e-08
3685 4.64858506035171e-08
3686 4.64860281610413e-08
3687 4.65036037340383e-08
3688 4.64701309397242e-08
3689 4.65006384828825e-08
3690 4.64749953863475e-08
3691 4.6485798229412e-08
3692 4.64805630002729e-08
3693 4.64797206518597e-08
3694 4.64671694118124e-08
3695 4.64869542184942e-08
3696 4.64751479327674e-08
3697 4.64691700372555e-08
3698 4.64536571556096e-08
3699 4.64600226095513e-08
3700 4.64829130564226e-08
3701 4.64634483918758e-08
3702 4.64475631574146e-08
3703 4.64474063051057e-08
3704 4.64457624715919e-08
3705 4.64668722131023e-08
3706 4.64479302877407e-08
3707 4.6438454212705e-08
3708 4.64337938410608e-08
3709 4.64412919782831e-08
3710 4.64497329488722e-08
3711 4.64411303653378e-08
3712 4.64227968990372e-08
3713 4.64271032569741e-08
3714 4.64147589838149e-08
3715 4.64423482071652e-08
3716 4.64078900890286e-08
3717 4.64218825371177e-08
3718 4.64313491193025e-08
3719 4.64170858194279e-08
3720 4.64046058326062e-08
3721 4.64134744433409e-08
3722 4.64209715218544e-08
3723 4.64038966043745e-08
3724 4.63980313512025e-08
3725 4.63950110685118e-08
3726 4.64081792088678e-08
3727 4.63887143169472e-08
3728 4.638725007311e-08
3729 4.64009830949408e-08
3730 4.63768293457179e-08
3731 4.64139274569675e-08
3732 4.63793551830349e-08
3733 4.63814730977674e-08
3734 4.63677578608213e-08
3735 4.63853216814414e-08
3736 4.63834044452938e-08
3737 4.6374892718859e-08
3738 4.63679514766113e-08
3739 4.6367666328706e-08
3740 4.63570700262039e-08
3741 4.63761159963383e-08
3742 4.63588706267615e-08
3743 4.63606340090905e-08
3744 4.63578068021775e-08
3745 4.6355476634119e-08
3746 4.63492448474767e-08
3747 4.63614413774849e-08
3748 4.63474599001756e-08
3749 4.6344356235295e-08
3750 4.63375668431354e-08
3751 4.63466459024175e-08
3752 4.6338579501537e-08
3753 4.63430903678841e-08
3754 4.63352891486579e-08
3755 4.6331161541957e-08
3756 4.63410394857533e-08
3757 4.63258666698607e-08
3758 4.63327825244164e-08
3759 4.63267199322104e-08
3760 4.63217764021806e-08
3761 4.63313827268053e-08
3762 4.63185945491773e-08
3763 4.63024153773972e-08
3764 4.63122884397649e-08
3765 4.63051161432304e-08
3766 4.62966256193909e-08
3767 4.6302732805259e-08
3768 4.62943985937159e-08
3769 4.62894258319579e-08
3770 4.6295529060103e-08
3771 4.62863493169152e-08
3772 4.62987360592137e-08
3773 4.62829439413781e-08
3774 4.62801427971726e-08
3775 4.62684166393501e-08
3776 4.62813586494804e-08
3777 4.62759012833658e-08
3778 4.62692492959604e-08
3779 4.62600309418804e-08
3780 4.62655715978144e-08
3781 4.6253400668661e-08
3782 4.62750794056888e-08
3783 4.62420766069727e-08
3784 4.62626667925292e-08
3785 4.62457001759731e-08
3786 4.62610541802633e-08
3787 4.62333476178856e-08
3788 4.62451557439181e-08
3789 4.62505911329458e-08
3790 4.62201076771862e-08
3791 4.62541393204674e-08
3792 4.62174536721705e-08
3793 4.62370347023011e-08
3794 4.62015387157066e-08
3795 4.62197871513581e-08
3796 4.62183832468099e-08
3797 4.62245888002144e-08
3798 4.62143747981258e-08
3799 4.62082738863501e-08
3800 4.62014027746704e-08
3801 4.62055438887887e-08
3802 4.6199753192866e-08
3803 4.62069813096377e-08
3804 4.61878302076002e-08
3805 4.62013185398291e-08
3806 4.62136260850343e-08
3807 4.61796251585156e-08
3808 4.61899804022892e-08
3809 4.61845029988694e-08
3810 4.61837238674434e-08
3811 4.6174706255897e-08
3812 4.61840177550243e-08
3813 4.61901042569934e-08
3814 4.61638466688896e-08
3815 4.6169743534108e-08
3816 4.61734862042817e-08
3817 4.6179553372383e-08
3818 4.61735033212562e-08
3819 4.61680598036196e-08
3820 4.61640029882915e-08
3821 4.61537215983299e-08
3822 4.61567790708273e-08
3823 4.6167264684982e-08
3824 4.61591865033029e-08
3825 4.61681775192346e-08
3826 4.615169400779e-08
3827 4.61531223550082e-08
3828 4.61702537464248e-08
3829 4.61499120021358e-08
3830 4.61292818485504e-08
3831 4.61492834062938e-08
3832 4.61420831641135e-08
3833 4.614463052377e-08
3834 4.61556046289502e-08
3835 4.61313909951855e-08
3836 4.61324365872429e-08
3837 4.61399496245463e-08
3838 4.61445568333829e-08
3839 4.61336989161509e-08
3840 4.61204558916961e-08
3841 4.61226060863851e-08
3842 4.61231383752647e-08
3843 4.61199801549128e-08
3844 4.61033424130619e-08
3845 4.6111790680925e-08
3846 4.61340033837132e-08
3847 4.61066351675754e-08
3848 4.61155500417476e-08
3849 4.61005013150384e-08
3850 4.61045115471848e-08
3851 4.61054220721735e-08
3852 4.61225703460855e-08
3853 4.6092650279661e-08
3854 4.60886039519437e-08
3855 4.6087250616722e-08
3856 4.6096683632868e-08
3857 4.60849555921072e-08
3858 4.60884467656797e-08
3859 4.60820958920749e-08
3860 4.60596141280689e-08
3861 4.60848333645458e-08
3862 4.60798593167056e-08
3863 4.60703818419006e-08
3864 4.60702240658861e-08
3865 4.6077666631561e-08
3866 4.60614220187949e-08
3867 4.60615912061257e-08
3868 4.6061550591503e-08
3869 4.60531981048007e-08
3870 4.60718154258188e-08
3871 4.60511154471988e-08
3872 4.60435203564202e-08
3873 4.60483926190136e-08
3874 4.60502766230775e-08
3875 4.60396044061895e-08
3876 4.6045692627672e-08
3877 4.60416568017763e-08
3878 4.60356317759647e-08
3879 4.60374580839584e-08
3880 4.60268082775883e-08
3881 4.60312790195871e-08
3882 4.60431735902489e-08
3883 4.60211841399882e-08
3884 4.60195553770859e-08
3885 4.60169952205547e-08
3886 4.60241872488609e-08
3887 4.60260829484582e-08
3888 4.60227033158844e-08
3889 4.60250096665504e-08
3890 4.60033512297287e-08
3891 4.60128066208654e-08
3892 4.6011802062651e-08
3893 4.59999051969362e-08
3894 4.60034676450505e-08
3895 4.60091707381594e-08
3896 4.60039092118336e-08
3897 4.59972471347214e-08
3898 4.600694937551e-08
3899 4.59928525415876e-08
3900 4.59984258327495e-08
3901 4.59904621266105e-08
3902 4.59873790035203e-08
3903 4.5991087255004e-08
3904 4.59860376693655e-08
3905 4.59795332190538e-08
3906 4.59704547566275e-08
3907 4.5980404330237e-08
3908 4.59984577219075e-08
3909 4.59714939040623e-08
3910 4.59694398600163e-08
3911 4.59717332716991e-08
3912 4.59629733029487e-08
3913 4.59550436531231e-08
3914 4.59713600520217e-08
3915 4.59566190471605e-08
3916 4.59587301193665e-08
3917 4.59654704840773e-08
3918 4.5962097992458e-08
3919 4.59483790606896e-08
3920 4.59428413748242e-08
3921 4.59353681065977e-08
3922 4.59532682057784e-08
3923 4.59406054531541e-08
3924 4.59375406620666e-08
3925 4.59408135284889e-08
3926 4.59382403903419e-08
3927 4.59478025405247e-08
3928 4.5928683974239e-08
3929 4.59298965509447e-08
3930 4.59391696310263e-08
3931 4.59137819817101e-08
3932 4.59249346320689e-08
3933 4.59467490614429e-08
3934 4.59226414761815e-08
3935 4.59149289682159e-08
3936 4.59095129556886e-08
3937 4.59112807646989e-08
3938 4.59194174950994e-08
3939 4.59184144290248e-08
3940 4.58959452913632e-08
3941 4.59132872308032e-08
3942 4.59108710941791e-08
3943 4.5900684632727e-08
3944 4.59025082975018e-08
3945 4.59147690463624e-08
3946 4.58973940524743e-08
3947 4.58974451689187e-08
3948 4.59024044587863e-08
3949 4.5891703322809e-08
3950 4.59136559527451e-08
3951 4.58921129791179e-08
3952 4.58824725058093e-08
3953 4.58760297021854e-08
3954 4.58863484453786e-08
3955 4.58754779373294e-08
3956 4.58676603969366e-08
3957 4.5887798258093e-08
3958 4.58760715247308e-08
3959 4.58794636202242e-08
3960 4.58692902114421e-08
3961 4.58637432245723e-08
3962 4.58702190471172e-08
3963 4.5886935097883e-08
3964 4.58632685109706e-08
3965 4.58720770666332e-08
3966 4.58507895686466e-08
3967 4.58557629769984e-08
3968 4.58649219297058e-08
3969 4.58259542170936e-08
3970 4.58566746672773e-08
3971 4.5860071672621e-08
3972 4.58360543973413e-08
3973 4.5858708972446e-08
3974 4.58516247263674e-08
3975 4.58410965578082e-08
3976 4.58351205168128e-08
3977 4.5830498940802e-08
3978 4.58591972858358e-08
3979 4.58500875950563e-08
3980 4.58303715689112e-08
3981 4.58225032815562e-08
3982 4.58175453701415e-08
3983 4.58139931964752e-08
3984 4.5839270789827e-08
3985 4.58263925153801e-08
3986 4.58184099727532e-08
3987 4.58310959317032e-08
3988 4.58383769270654e-08
3989 4.57989212847565e-08
3990 4.58185160212565e-08
3991 4.58076849341182e-08
3992 4.58208417271067e-08
3993 4.58053273817427e-08
3994 4.58025351548486e-08
3995 4.58091312225406e-08
3996 4.57994679905482e-08
3997 4.58022140676917e-08
3998 4.58080404683869e-08
3999 4.57947939764836e-08
4000 4.5789922026529e-08
4001 4.57909257036704e-08
4002 4.57910882616375e-08
4003 4.5778956369702e-08
4004 4.57873526116259e-08
4005 4.57861400420256e-08
4006 4.57982184087769e-08
4007 4.57817261292348e-08
4008 4.5781224713437e-08
4009 4.57250410690335e-08
4010 4.57482202804727e-08
4011 4.57454816995551e-08
4012 4.57366656831937e-08
4013 4.57825859854211e-08
4014 4.57758135681274e-08
4015 4.57627746826006e-08
4016 4.57583302875264e-08
4017 4.57556256137082e-08
4018 4.57710771186726e-08
4019 4.57533019755374e-08
4020 4.57520305161552e-08
4021 4.57533257005593e-08
4022 4.57469102101982e-08
4023 4.57564141598255e-08
4024 4.57456612537044e-08
4025 4.57447280126644e-08
4026 4.57397255502201e-08
4027 4.57382139771312e-08
4028 4.5737899668552e-08
4029 4.57322663649506e-08
4030 4.57402651221628e-08
4031 4.5725074429015e-08
4032 4.57311132535665e-08
4033 4.57328364191767e-08
4034 4.57204579689119e-08
4035 4.57220980720763e-08
4036 4.57257710309022e-08
4037 4.57494723420382e-08
4038 4.56988391093205e-08
4039 4.57147486869758e-08
4040 4.57088138574591e-08
4041 4.57063215861808e-08
4042 4.57306162431337e-08
4043 4.56729732363215e-08
4044 4.57336924881702e-08
4045 4.5704480861275e-08
4046 4.57026195306298e-08
4047 4.5705511233507e-08
4048 4.56972201163808e-08
4049 4.56906911381338e-08
4050 4.57103906086331e-08
4051 4.57068857286913e-08
4052 4.5658831609785e-08
4053 4.56569222251346e-08
4054 4.56938711295152e-08
4055 4.56868904308294e-08
4056 4.56841325942037e-08
4057 4.56890598385939e-08
4058 4.56856426751529e-08
4059 4.56393377277209e-08
4060 4.56886737367768e-08
4061 4.56754059214859e-08
4062 4.56369106913712e-08
4063 4.56477012207301e-08
4064 4.56420432541904e-08
4065 4.56403993212007e-08
4066 4.56319136645789e-08
4067 4.56624100948488e-08
4068 4.56604331446897e-08
4069 4.56538726822941e-08
4070 4.56630516509904e-08
4071 4.56544036850914e-08
4072 4.56087201285982e-08
4073 4.56613797936711e-08
4074 4.56078365687063e-08
4075 4.56816478617839e-08
4076 4.5602172392023e-08
4077 4.56223851443838e-08
4078 4.56308354230828e-08
4079 4.56396374346468e-08
4080 4.5622755834529e-08
4081 4.56250916514023e-08
4082 4.56246881697098e-08
4083 4.55807776091888e-08
4084 4.56471725840402e-08
4085 4.5586314044499e-08
4086 4.56040137564173e-08
4087 4.55162390622377e-08
4088 4.55641065855161e-08
4089 4.55935024419318e-08
4090 4.55927265647915e-08
4091 4.56073568457782e-08
4092 4.5607056236463e-08
4093 4.55742507909918e-08
4094 4.55736693467657e-08
4095 4.55629685731651e-08
4096 4.55714424631992e-08
4097 4.55756029111853e-08
4098 4.55507525458643e-08
4099 4.55593632366913e-08
4100 4.55623985260445e-08
4101 4.55300556012617e-08
4102 4.55171336497528e-08
4103 4.55477894760747e-08
4104 4.55559234850966e-08
4105 4.55916670603074e-08
4106 4.55821758009733e-08
4107 4.5572967501073e-08
4108 4.55109114554375e-08
4109 4.55641759771197e-08
4110 4.55279219195859e-08
4111 4.55041555156299e-08
4112 4.55187827554937e-08
4113 4.55371556142836e-08
4114 4.55471117177808e-08
4115 4.55613131506993e-08
4116 4.54897065296223e-08
4117 4.55250713855548e-08
4118 4.54170470121085e-08
4119 4.54606157305193e-08
4120 4.54652825041535e-08
4121 4.54677925958435e-08
4122 4.54715844426801e-08
4123 4.54683175590276e-08
4124 4.54706711749964e-08
4125 4.54972420698141e-08
4126 4.54470134982898e-08
4127 4.54441432609087e-08
4128 4.54370037559215e-08
4129 4.54090519212969e-08
4130 4.53882334880973e-08
4131 4.53980523573705e-08
4132 4.54026908656147e-08
4133 4.53937244557778e-08
4134 4.54230035558112e-08
4135 4.54548210413463e-08
4136 4.53912309055227e-08
4137 4.53971795408847e-08
4138 4.53789554981654e-08
4139 4.53342807276158e-08
4140 4.53205255723788e-08
4141 4.52670066692917e-08
4142 4.52372441301918e-08
4143 4.52219021838118e-08
4144 4.51943350299189e-08
4145 4.52089001399258e-08
4146 4.51909794634275e-08
4147 4.51708370903248e-08
4148 4.51384126378684e-08
4149 4.51067209610301e-08
4150 4.50828401596937e-08
4151 4.50296143625906e-08
4152 4.50072894011555e-08
4153 4.49832238942349e-08
4154 4.49766735073354e-08
4155 4.49582795738479e-08
4156 4.49312940915547e-08
4157 4.49251926113448e-08
4158 4.49131367190603e-08
4159 4.48907735162152e-08
4160 4.48959844590036e-08
4161 4.48998741617856e-08
4162 4.48582743572956e-08
4163 4.4849387990098e-08
4164 4.4859935393049e-08
4165 4.48336685963113e-08
4166 4.48239598966893e-08
4167 4.48193441044964e-08
4168 4.48074851178148e-08
4169 4.48164476267721e-08
4170 4.47862490986495e-08
4171 4.47994614489744e-08
4172 4.47864582255875e-08
4173 4.47801073875098e-08
4174 4.47919584019019e-08
4175 4.47624840020922e-08
4176 4.47181304394917e-08
4177 4.47019370071189e-08
4178 4.47178341005383e-08
4179 4.47192630517179e-08
4180 4.47300834522935e-08
4181 4.4696444021497e-08
4182 4.46825209508006e-08
4183 4.47334799460464e-08
4184 4.47190578753975e-08
4185 4.46668471525413e-08
4186 4.46770428297327e-08
4187 4.46936577702672e-08
4188 4.46823408779551e-08
4189 4.46908604843088e-08
4190 4.46326729033331e-08
4191 4.46719642681614e-08
4192 4.4657993377939e-08
4193 4.46809285961081e-08
4194 4.47037833168906e-08
4195 4.46955758164336e-08
4196 4.46880106963476e-08
4197 4.47042302056388e-08
4198 4.47559269858289e-08
4199 4.47504608729332e-08
4200 4.47183659275652e-08
4201 4.46763340917755e-08
4202 4.46641028659656e-08
4203 4.46470607116112e-08
4204 4.46436195318256e-08
4205 4.46784736993777e-08
4206 4.46212543465663e-08
4207 4.46527076860548e-08
4208 4.45932925572379e-08
4209 4.46081768075146e-08
4210 4.46209906357353e-08
4211 4.45385450049685e-08
4212 4.45167449285577e-08
4213 4.45988847772583e-08
4214 4.45881521642377e-08
4215 4.45819015268967e-08
4216 4.45820520198481e-08
4217 4.45768134014202e-08
4218 4.452347991446e-08
4219 4.45606529524412e-08
4220 4.45458848545854e-08
4221 4.45399119683998e-08
4222 4.45560699802172e-08
4223 4.45297804319011e-08
4224 4.44668139820692e-08
4225 4.45371007486983e-08
4226 4.45165028182259e-08
4227 4.45253177048244e-08
4228 4.4515073561513e-08
4229 4.44917927922006e-08
4230 4.44926687848124e-08
4231 4.44855463541671e-08
4232 4.45049759321137e-08
4233 4.44953232658918e-08
4234 4.44792980118791e-08
4235 4.4449217206477e-08
4236 4.44700325417102e-08
4237 4.4467549116689e-08
4238 4.4475163335278e-08
4239 4.44491966504756e-08
4240 4.44452882319979e-08
4241 4.4432189589827e-08
4242 4.44580243978976e-08
4243 4.43889903607442e-08
4244 4.44525463549894e-08
4245 4.44444592488935e-08
4246 4.44420138805413e-08
4247 4.44261157781511e-08
4248 4.44202012275241e-08
4249 4.43934998983764e-08
4250 4.46138195684398e-08
4251 4.43246305081857e-08
4252 4.44989496841686e-08
4253 4.4482403076529e-08
4254 4.43616497278754e-08
4255 4.43724649485944e-08
4256 4.43113323385091e-08
4257 4.44829274499625e-08
4258 4.44926633278442e-08
4259 4.44567327093637e-08
4260 4.42923005081752e-08
4261 4.44411402966693e-08
4262 4.42784391267992e-08
4263 4.42743370641097e-08
4264 4.42691281961061e-08
4265 4.43236821183746e-08
4266 4.43364939428648e-08
4267 4.43688945779286e-08
4268 4.42846208557057e-08
4269 4.43264949581135e-08
4270 4.43202494864181e-08
4271 4.43259647227023e-08
4272 4.44505042551668e-08
4273 4.44493706339699e-08
4274 4.44161049131253e-08
4275 4.42407458081107e-08
4276 4.43982424727096e-08
4277 4.42394581696703e-08
4278 4.42580605763965e-08
4279 4.42735937085104e-08
4280 4.42726637075452e-08
4281 4.42741622492804e-08
4282 4.42704916991943e-08
4283 4.42800548938749e-08
4284 4.42299249741041e-08
4285 4.42825050797069e-08
4286 4.42835083873661e-08
4287 4.42740837058864e-08
4288 4.42382075291903e-08
4289 4.42638120929928e-08
4290 4.44005860700258e-08
4291 4.42316692712552e-08
4292 4.4238723475587e-08
4293 4.41914156255052e-08
4294 4.42178246444769e-08
4295 4.42020132496168e-08
4296 4.41910947870383e-08
4297 4.42370375637324e-08
4298 4.42278169217047e-08
4299 4.42379677281224e-08
4300 4.42597665468725e-08
4301 4.42053977209866e-08
4302 4.42413546579701e-08
4303 4.43455420793271e-08
4304 4.43097312157192e-08
4305 4.41726482449667e-08
4306 4.43143809150115e-08
4307 4.41482806792237e-08
4308 4.41600077181192e-08
4309 4.43172839936778e-08
4310 4.41375118001019e-08
4311 4.41267259816414e-08
4312 4.41554300962821e-08
4313 4.4148488647977e-08
4314 4.41423204904368e-08
4315 4.4168975207981e-08
4316 4.41277933305173e-08
4317 4.41707804768043e-08
4318 4.41988532031701e-08
4319 4.43071687215024e-08
4320 4.4258898689975e-08
4321 4.40923729385645e-08
4322 4.42766690440521e-08
4323 4.41062048039953e-08
4324 4.41372728658962e-08
4325 4.42688758681697e-08
4326 4.40815239812764e-08
4327 4.40600616329334e-08
4328 4.40599639333072e-08
4329 4.40782453736688e-08
4330 4.40947135160741e-08
4331 4.40808126924708e-08
4332 4.40937858883217e-08
4333 4.4092420381503e-08
4334 4.40997551649502e-08
4335 4.43311271780544e-08
4336 4.41221229223743e-08
4337 4.42368445092711e-08
4338 4.41780581965645e-08
4339 4.40561583872068e-08
4340 4.4037324713031e-08
4341 4.4226500811817e-08
4342 4.40496926117362e-08
4343 4.4083417691354e-08
4344 4.42017632451552e-08
4345 4.40236840191233e-08
4346 4.40068993157183e-08
4347 4.40517907733806e-08
4348 4.40157434766775e-08
4349 4.41094015286581e-08
4350 4.40142379787289e-08
4351 4.40227234221879e-08
4352 4.40516221331677e-08
4353 4.42225224830395e-08
4354 4.4036972482786e-08
4355 4.41987109098818e-08
4356 4.40119845990239e-08
4357 4.4173619798471e-08
4358 4.39988445606332e-08
4359 4.39680585841984e-08
4360 4.41497920675715e-08
4361 4.40101149123961e-08
4362 4.41548039802342e-08
4363 4.39872922086693e-08
4364 4.41414587513123e-08
4365 4.39885193301848e-08
4366 4.41269464843685e-08
4367 4.3958917139264e-08
4368 4.39669259648667e-08
4369 4.41305630474176e-08
4370 4.39642829803688e-08
4371 4.39411906754117e-08
4372 4.39336654451949e-08
4373 4.39552568423096e-08
4374 4.39602650246229e-08
4375 4.41336429801709e-08
4376 4.39924945538905e-08
4377 4.39778665111135e-08
4378 4.41423242705241e-08
4379 4.3953175669742e-08
4380 4.41157137416326e-08
4381 4.3932549630199e-08
4382 4.40896398998802e-08
4383 4.39176291209264e-08
4384 4.40723029271339e-08
4385 4.40774574812508e-08
4386 4.39046374580698e-08
4387 4.40770349214858e-08
4388 4.38862526195294e-08
4389 4.39038152038052e-08
4390 4.40774678338585e-08
4391 4.38869533141428e-08
4392 4.38828622719711e-08
4393 4.40624226172304e-08
4394 4.39248283683469e-08
4395 4.4085964574947e-08
4396 4.39207259219643e-08
4397 4.40465401538859e-08
4398 4.38863796858868e-08
4399 4.40424447702981e-08
4400 4.38618881872799e-08
4401 4.40227378817326e-08
4402 4.40364964049422e-08
4403 4.38644446774106e-08
4404 4.40391088289971e-08
4405 4.38460436384958e-08
4406 4.40219867670066e-08
4407 4.38576708745586e-08
4408 4.40092559585992e-08
4409 4.40213769437037e-08
4410 4.38251318826133e-08
4411 4.40178555862758e-08
4412 4.38470587837969e-08
4413 4.40244257475797e-08
4414 4.38154040622862e-08
4415 4.40052770116495e-08
4416 4.3819167515835e-08
4417 4.39747509233257e-08
4418 4.40016826743772e-08
4419 4.38084652643056e-08
4420 4.39937574512328e-08
4421 4.38020356483548e-08
4422 4.39896042294663e-08
4423 4.3791629074974e-08
4424 4.39777360980997e-08
4425 4.39825850548914e-08
4426 4.37916273554606e-08
4427 4.39774619351851e-08
4428 4.37845936076542e-08
4429 4.3975537558083e-08
4430 4.37714682277601e-08
4431 4.3959209165223e-08
4432 4.39668771292645e-08
4433 4.37754646327448e-08
4434 4.39607023778876e-08
4435 4.37619516304721e-08
4436 4.39520270916205e-08
4437 4.37620960198615e-08
4438 4.39299310812657e-08
4439 4.39437544415e-08
4440 4.37561519603946e-08
4441 4.39471657216473e-08
4442 4.37490345959191e-08
4443 4.39364637685458e-08
4444 4.37523518499461e-08
4445 4.39288584459518e-08
4446 4.39005302936835e-08
4447 4.37148015350886e-08
4448 4.38994157576644e-08
4449 4.39257569979645e-08
4450 4.39045446469777e-08
4451 4.38825607815829e-08
4452 4.38710261576603e-08
4453 4.38570743170885e-08
4454 4.37447216654618e-08
4455 4.3880096747273e-08
4456 4.38571230176876e-08
4457 4.38999945373553e-08
4458 4.36829383616555e-08
4459 4.38465498362461e-08
4460 4.3909933324926e-08
4461 4.38606946460141e-08
4462 4.38438219418913e-08
4463 4.38389137613626e-08
4464 4.38389670165407e-08
4465 4.38264696782653e-08
4466 4.38907400308608e-08
4467 4.37538109423485e-08
4468 4.37521294998078e-08
4469 4.38919654115466e-08
4470 4.38728694618362e-08
4471 4.37164779540922e-08
4472 4.38837826308713e-08
4473 4.38562680784571e-08
4474 4.38503131334755e-08
4475 4.3843257685694e-08
4476 4.38306239587405e-08
4477 4.38243404090599e-08
4478 4.38320245450541e-08
4479 4.380972711715e-08
4480 4.38133329438983e-08
4481 4.37341136816372e-08
4482 4.385824974662e-08
4483 4.38562669344833e-08
4484 4.38561796940462e-08
4485 4.38144374186322e-08
4486 4.37892310429788e-08
4487 4.36641442433938e-08
4488 4.36635318408207e-08
4489 4.38076617896854e-08
4490 4.38049209847691e-08
4491 4.38477358883915e-08
4492 4.36381664385976e-08
4493 4.36355555351042e-08
4494 4.37961323953573e-08
4495 4.37867734177644e-08
4496 4.37658899201665e-08
4497 4.37420516021803e-08
4498 4.36126325169539e-08
4499 4.37456880888476e-08
4500 4.37461371376457e-08
4501 4.36047219238844e-08
4502 4.3751253308244e-08
4503 4.38008252530153e-08
4504 4.35724958052219e-08
4505 4.35766342974375e-08
4506 4.35787331696247e-08
4507 4.35992319012257e-08
4508 4.35967829304218e-08
4509 4.35738248683037e-08
4510 4.3578292974189e-08
4511 4.35969055629926e-08
4512 4.35833684662157e-08
4513 4.35552414685958e-08
4514 4.35826279741036e-08
4515 4.35668386487009e-08
4516 4.35739359971876e-08
4517 4.35782764327541e-08
4518 4.35503886961897e-08
4519 4.35623186945122e-08
4520 4.35558712510442e-08
4521 4.35658973501063e-08
4522 4.35527377646849e-08
4523 4.35307114443617e-08
4524 4.35180639541954e-08
4525 4.35341518283394e-08
4526 4.35398110809615e-08
4527 4.35174982342801e-08
4528 4.35342042166553e-08
4529 4.35257903959041e-08
4530 4.35142472312577e-08
4531 4.35315140094872e-08
4532 4.35382286099184e-08
4533 4.35075089413317e-08
4534 4.35249012014083e-08
4535 4.35048377696035e-08
4536 4.35099602142941e-08
4537 4.35018047184599e-08
4538 4.34993154385666e-08
4539 4.3489410089137e-08
4540 4.34782014195889e-08
4541 4.34669418893918e-08
4542 4.34868489520568e-08
4543 4.34844133962997e-08
4544 4.34590772755428e-08
4545 4.34490773528751e-08
4546 4.34683341978825e-08
4547 4.34436364358248e-08
4548 4.34654150183178e-08
4549 4.34681405181436e-08
4550 4.34466884797757e-08
4551 4.34385482108723e-08
4552 4.34170205565465e-08
4553 4.34489677303418e-08
4554 4.34428003472931e-08
4555 4.3419925233934e-08
4556 4.34320484998807e-08
4557 4.34373255799869e-08
4558 4.34115867591345e-08
4559 4.34050383901763e-08
4560 4.3419153683999e-08
4561 4.34367528328039e-08
4562 4.34100121822212e-08
4563 4.33851229999505e-08
4564 4.34037865559844e-08
4565 4.34093412948755e-08
4566 4.33851182464196e-08
4567 4.3380031982565e-08
4568 4.34131939783811e-08
4569 4.33924230094362e-08
4570 4.33743679124632e-08
4571 4.33571273816824e-08
4572 4.33543018445448e-08
4573 4.33787571481048e-08
4574 4.33530872498977e-08
4575 4.33514058144624e-08
4576 4.33753117974334e-08
4577 4.3365410974161e-08
4578 4.33403115067676e-08
4579 4.33550796543614e-08
4580 4.33525775989096e-08
4581 4.33264548576062e-08
4582 4.33320331225673e-08
4583 4.33393577097263e-08
4584 4.33362923502045e-08
4585 4.33171733220661e-08
4586 4.33467197282766e-08
4587 4.33352698365752e-08
4588 4.33153624683769e-08
4589 4.33148023546437e-08
4590 4.33030340616369e-08
4591 4.32971409480842e-08
4592 4.3305780664582e-08
4593 4.33161835573514e-08
4594 4.32924303552795e-08
4595 4.32957577842785e-08
4596 4.32734408803981e-08
4597 4.32773811596121e-08
4598 4.32902738509711e-08
4599 4.32934632215165e-08
4600 4.32754045647243e-08
4601 4.3286737792414e-08
4602 4.32643051837545e-08
4603 4.3246039389544e-08
4604 4.32658605191705e-08
4605 4.32813205364368e-08
4606 4.32699937462644e-08
4607 4.32578974738362e-08
4608 4.3239407659712e-08
4609 4.32462073618467e-08
4610 4.32184197975971e-08
4611 4.32439855728717e-08
4612 4.32479588923229e-08
4613 4.32450432938936e-08
4614 4.32124914055976e-08
4615 4.32288852394436e-08
4616 4.32374926546686e-08
4617 4.32100818557046e-08
4618 4.32240480563451e-08
4619 4.31867054118129e-08
4620 4.32252515167875e-08
4621 4.32234197802472e-08
4622 4.32002088643912e-08
4623 4.32353490040782e-08
4624 4.31627946682056e-08
4625 4.3211618326211e-08
4626 4.31857505205357e-08
4627 4.31891641241577e-08
4628 4.31746859064219e-08
4629 4.31610825657458e-08
4630 4.31880701583509e-08
4631 4.32017181992705e-08
4632 4.31690783884164e-08
4633 4.31693371538699e-08
4634 4.31352902268145e-08
4635 4.31738735358067e-08
4636 4.31624636121342e-08
4637 4.31629251309573e-08
4638 4.3147932267118e-08
4639 4.31327312355734e-08
4640 4.31480738640744e-08
4641 4.31594509038291e-08
4642 4.31319048743717e-08
4643 4.31431942828908e-08
4644 4.31032985872548e-08
4645 4.31479912492705e-08
4646 4.31375630043362e-08
4647 4.31248581307386e-08
4648 4.31155863651611e-08
4649 4.31033188021956e-08
4650 4.31213864260371e-08
4651 4.3121475840735e-08
4652 4.31055224723309e-08
4653 4.31033152423765e-08
4654 4.3080920875127e-08
4655 4.31135218974532e-08
4656 4.31045605751024e-08
4657 4.30855039681433e-08
4658 4.30781620934795e-08
4659 4.30972606935143e-08
4660 4.31029808538597e-08
4661 4.30726512234969e-08
4662 4.30926048480274e-08
4663 4.30564962599078e-08
4664 4.30825707553595e-08
4665 4.30827543951295e-08
4666 4.30534736253207e-08
4667 4.30625754432867e-08
4668 4.30512071503131e-08
4669 4.30506224873284e-08
4670 4.30588271669308e-08
4671 4.30731581602117e-08
4672 4.30578073107313e-08
4673 4.30476396147128e-08
4674 4.30277035334825e-08
4675 4.30420473094273e-08
4676 4.30115228695627e-08
4677 4.30480096085262e-08
4678 4.30357869518616e-08
4679 4.30105758653099e-08
4680 4.3025070830538e-08
4681 4.30156576598506e-08
4682 4.30008470431176e-08
4683 4.30213547417679e-08
4684 4.30282886441091e-08
4685 4.30076463331375e-08
4686 4.30081273918859e-08
4687 4.30010082652643e-08
4688 4.30087835994186e-08
4689 4.29986839662888e-08
4690 4.29898519058725e-08
4691 4.29618187922642e-08
4692 4.30001471585228e-08
4693 4.29947823405996e-08
4694 4.29775256804987e-08
4695 4.30001294731142e-08
4696 4.29770989356371e-08
4697 4.29685147338432e-08
4698 4.29611607231095e-08
4699 4.29591598773982e-08
4700 4.29384919655718e-08
4701 4.29799128838226e-08
4702 4.29626628175583e-08
4703 4.29583019041502e-08
4704 4.29455422761293e-08
4705 4.29570300539694e-08
4706 4.29324862309954e-08
4707 4.29382108180221e-08
4708 4.29120718337117e-08
4709 4.29502837064888e-08
4710 4.29471350571475e-08
4711 4.29286753131919e-08
4712 4.29090403812893e-08
4713 4.29223368811904e-08
4714 4.29001627395564e-08
4715 4.2926661954823e-08
4716 4.2930852245604e-08
4717 4.29202476723844e-08
4718 4.2911796406031e-08
4719 4.28928703399833e-08
4720 4.29091978162433e-08
4721 4.28881386866919e-08
4722 4.29003332129696e-08
4723 4.28864241612814e-08
4724 4.28954944453608e-08
4725 4.28684130824308e-08
4726 4.28989431071614e-08
4727 4.28934690148708e-08
4728 4.28797349343313e-08
4729 4.28870557271921e-08
4730 4.28660771518707e-08
4731 4.28733341593102e-08
4732 4.28591781300724e-08
4733 4.28405485308758e-08
4734 4.28832222070241e-08
4735 4.28771520759597e-08
4736 4.28513044496981e-08
4737 4.28560878447115e-08
4738 4.28291741343401e-08
4739 4.28501932177028e-08
4740 4.28236990472897e-08
4741 4.28776732448455e-08
4742 4.28499467943766e-08
4743 4.2831831102319e-08
4744 4.28332801476472e-08
4745 4.2820572886626e-08
4746 4.28198334816443e-08
4747 4.28064177029341e-08
4748 4.28262487091047e-08
4749 4.28098955964629e-08
4750 4.28262454832407e-08
4751 4.28170810948814e-08
4752 4.27997828325033e-08
4753 4.28027079095727e-08
4754 4.27946613257291e-08
4755 4.27883242792859e-08
4756 4.28072784188771e-08
4757 4.28070646876222e-08
4758 4.27838733401131e-08
4759 4.28005133414899e-08
4760 4.27816306540763e-08
4761 4.28009149686659e-08
4762 4.27772035678231e-08
4763 4.27860299794247e-08
4764 4.27387640513643e-08
4765 4.27776123004264e-08
4766 4.2777297188934e-08
4767 4.27539969010127e-08
4768 4.27666754774236e-08
4769 4.27691910545036e-08
4770 4.27584076163612e-08
4771 4.27438386054746e-08
4772 4.27565139915487e-08
4773 4.27348958567109e-08
4774 4.27505387747829e-08
4775 4.27395674904574e-08
4776 4.27118130872373e-08
4777 4.27518141208338e-08
4778 4.27468103438855e-08
4779 4.27272051553018e-08
4780 4.27386604400226e-08
4781 4.27206328410534e-08
4782 4.27169511638681e-08
4783 4.27158320590593e-08
4784 4.27310642123757e-08
4785 4.27023811298e-08
4786 4.27019567297293e-08
4787 4.27045646915758e-08
4788 4.27149916362168e-08
4789 4.27014713011431e-08
4790 4.27137734320127e-08
4791 4.26906031449903e-08
4792 4.26890659070978e-08
4793 4.2688757474707e-08
4794 4.27003471088483e-08
4795 4.26756235754056e-08
4796 4.26821950725298e-08
4797 4.26748742583527e-08
4798 4.26896046406e-08
4799 4.26859920423794e-08
4800 4.26800362234303e-08
4801 4.2693520562409e-08
4802 4.26580270698196e-08
4803 4.26700894351484e-08
4804 4.26511646693939e-08
4805 4.26515462734756e-08
4806 4.26504902790725e-08
4807 4.26703180380628e-08
4808 4.26463872784666e-08
4809 4.26396074217905e-08
4810 4.26418264822814e-08
4811 4.26380383515834e-08
4812 4.26366987724691e-08
4813 4.2649084257107e-08
4814 4.26335177721171e-08
4815 4.26499567893757e-08
4816 4.26173111023331e-08
4817 4.263730082954e-08
4818 4.26306612766325e-08
4819 4.26247637719257e-08
4820 4.26210972506169e-08
4821 4.26192461375763e-08
4822 4.26220262994548e-08
4823 4.26085585303326e-08
4824 4.26144201455259e-08
4825 4.26179771864099e-08
4826 4.26047075876568e-08
4827 4.25932670395923e-08
4828 4.25854179084695e-08
4829 4.2597334555694e-08
4830 4.2597556401347e-08
4831 4.25983200784685e-08
4832 4.25900839147175e-08
4833 4.26008241873888e-08
4834 4.25965092603064e-08
4835 4.25848408411866e-08
4836 4.25704996658283e-08
4837 4.25802670065423e-08
4838 4.25689460854528e-08
4839 4.25821085059397e-08
4840 4.25624952669068e-08
4841 4.2570896816585e-08
4842 4.25591190804653e-08
4843 4.25773868641954e-08
4844 4.25474526792868e-08
4845 4.25606598071226e-08
4846 4.25452232875045e-08
4847 4.25869765550146e-08
4848 4.25353505946191e-08
4849 4.25508923953544e-08
4850 4.25360271449904e-08
4851 4.25433661490615e-08
4852 4.2528112778939e-08
4853 4.2521858169664e-08
4854 4.25329618991555e-08
4855 4.25317744614517e-08
4856 4.25271616393275e-08
4857 4.25355358899537e-08
4858 4.25131136978507e-08
4859 4.2523948010853e-08
4860 4.25136261199555e-08
4861 4.2510707643828e-08
4862 4.24912981600301e-08
4863 4.25074300451911e-08
4864 4.24894550690169e-08
4865 4.24865572199451e-08
4866 4.24953835604924e-08
4867 4.25216283304053e-08
4868 4.24995869181544e-08
4869 4.24992914247468e-08
4870 4.24851347204935e-08
4871 4.24912016327994e-08
4872 4.24743206437483e-08
4873 4.24904243843116e-08
4874 4.24797314053649e-08
4875 4.25009794966513e-08
4876 4.24651205008786e-08
4877 4.24788755282179e-08
4878 4.24650749337729e-08
4879 4.24686299211885e-08
4880 4.24544355581702e-08
4881 4.24718340923391e-08
4882 4.24634487075082e-08
4883 4.24572838539916e-08
4884 4.24421771967332e-08
4885 4.24605621347496e-08
4886 4.24543382706588e-08
4887 4.24665713438799e-08
4888 4.24311416793444e-08
4889 4.24470698874302e-08
4890 4.24326530250596e-08
4891 4.24368067797332e-08
4892 4.24223960848735e-08
4893 4.2440266568633e-08
4894 4.24323432710594e-08
4895 4.24282955719946e-08
4896 4.24128418075043e-08
4897 4.24392744164948e-08
4898 4.24215464036593e-08
4899 4.24184494320912e-08
4900 4.24070979008206e-08
4901 4.24122320836773e-08
4902 4.23959066608859e-08
4903 4.24140252590632e-08
4904 4.23968123399732e-08
4905 4.24096308009325e-08
4906 4.23769133277574e-08
4907 4.24066152220348e-08
4908 4.23758376157934e-08
4909 4.23952780153059e-08
4910 4.23648923231212e-08
4911 4.23837936907034e-08
4912 4.23752900360341e-08
4913 4.23848783341896e-08
4914 4.23470618926558e-08
4915 4.24060529553572e-08
4916 4.23864138596741e-08
4917 4.23877485502544e-08
4918 4.23707591465927e-08
4919 4.23639771369722e-08
4920 4.23579671178231e-08
4921 4.23486049641042e-08
4922 4.23403889371343e-08
4923 4.23372620446116e-08
4924 4.23248291099299e-08
4925 4.23153666133658e-08
4926 4.23094892525455e-08
4927 4.23078750912964e-08
4928 4.23040242125694e-08
4929 4.23148204475865e-08
4930 4.22959589556626e-08
4931 4.23038447081581e-08
4932 4.22861912952044e-08
4933 4.229060537142e-08
4934 4.2291319758192e-08
4935 4.22778749680219e-08
4936 4.22671583493184e-08
4937 4.22742193961767e-08
4938 4.22759228868586e-08
4939 4.22601263423417e-08
4940 4.22492092084781e-08
4941 4.2261916298969e-08
4942 4.22608142685021e-08
4943 4.22443294212371e-08
4944 4.22351964672885e-08
4945 4.22443192249489e-08
4946 4.22329271216881e-08
4947 4.22273766531589e-08
4948 4.22209910198035e-08
4949 4.22383358156253e-08
4950 4.22039266538832e-08
4951 4.22132960054e-08
4952 4.22051288566649e-08
4953 4.22298417106504e-08
4954 4.21875172946784e-08
4955 4.21986440528599e-08
4956 4.21919518061031e-08
4957 4.22100912160772e-08
4958 4.21743706908728e-08
4959 4.21863958592894e-08
4960 4.21816493769711e-08
4961 4.21796282310538e-08
4962 4.21711069620301e-08
4963 4.2174961762953e-08
4964 4.21627063929009e-08
4965 4.21699960782007e-08
4966 4.21576518760958e-08
4967 4.21669817569637e-08
4968 4.21474967566837e-08
4969 4.2161323094092e-08
4970 4.21339040173052e-08
4971 4.21538636885543e-08
4972 4.21446531149172e-08
4973 4.21318724761477e-08
4974 4.21300442994266e-08
4975 4.21510051253904e-08
4976 4.21165410742219e-08
4977 4.21189229058427e-08
4978 4.21217483719261e-08
4979 4.21195523756523e-08
4980 4.2124631441709e-08
4981 4.21052065107119e-08
4982 4.21189617938467e-08
4983 4.21037285818215e-08
4984 4.21111890389625e-08
4985 4.20933603209051e-08
4986 4.20960025167005e-08
4987 4.20971385821645e-08
4988 4.21038654110362e-08
4989 4.20709708279787e-08
4990 4.20818645423537e-08
4991 4.20686601501075e-08
4992 4.20826475959757e-08
4993 4.2058874811346e-08
4994 4.20718643354689e-08
4995 4.20561488994053e-08
4996 4.2063875042686e-08
4997 4.20500427154025e-08
4998 4.20532562159792e-08
4999 4.20456807006531e-08
};
\addlegendentry{Train}
\addplot [semithick, black]
table {%
0 0.0130676496773958
1 0.00752035155892372
2 0.00463447393849492
3 0.00311522744596004
4 0.00240469910204411
5 0.00201634503901005
6 0.00165862671565264
7 0.00125592085532844
8 0.000891369243618101
9 0.000613982672803104
10 0.000437169743236154
11 0.000337189849233255
12 0.000284692592686042
13 0.000258796470006928
14 0.000246379553573206
15 0.000240298482822254
16 0.000237015134189278
17 0.000234952676692046
18 0.000233434504480101
19 0.000232178645092063
20 0.000231051177252084
21 0.000229932120419107
22 0.000228722012252547
23 0.00022738472034689
24 0.000225871481234208
25 0.000224171570152976
26 0.000222253656829707
27 0.000220081594306976
28 0.000217619381146505
29 0.000214822037378326
30 0.000211638965993188
31 0.000206259021069854
32 0.000199730493477546
33 0.000192884777789004
34 0.000185268901987001
35 0.000176619068952277
36 0.000166784535394982
37 0.000155638947035186
38 0.000141657801577821
39 0.000126843850011937
40 0.000110981440229807
41 9.46498475968838e-05
42 7.84725052653812e-05
43 6.33380332146771e-05
44 5.01757822348736e-05
45 3.96252653445117e-05
46 3.12891679641325e-05
47 2.58624131674878e-05
48 2.27447035285877e-05
49 2.06756940315245e-05
50 1.91729704965837e-05
51 1.79758808371844e-05
52 1.69573995663086e-05
53 1.60571416927269e-05
54 1.52366774273105e-05
55 1.44789692058112e-05
56 1.37702163556241e-05
57 1.3099603165756e-05
58 1.24609314298141e-05
59 1.18495445349254e-05
60 1.12574844024493e-05
61 1.06791039797827e-05
62 1.01157302196953e-05
63 9.566238077241e-06
64 9.03300315258093e-06
65 8.51417098601814e-06
66 8.01431906438665e-06
67 7.53029962652363e-06
68 7.0651817623002e-06
69 6.6184938987135e-06
70 6.19318643657607e-06
71 5.78848721488612e-06
72 5.40714154340094e-06
73 5.04996569361538e-06
74 4.72123292638571e-06
75 4.41469137513195e-06
76 4.13064344684244e-06
77 3.87234376830747e-06
78 3.63965182259562e-06
79 3.43147144121758e-06
80 3.23879430652596e-06
81 3.06280912809598e-06
82 2.90408229375316e-06
83 2.76005062005424e-06
84 2.63091214947053e-06
85 2.51752703661623e-06
86 2.41070733864035e-06
87 2.31166472985933e-06
88 2.21917298404151e-06
89 2.13368934964819e-06
90 2.055065579043e-06
91 1.9814881397906e-06
92 1.91340063793177e-06
93 1.84920133960986e-06
94 1.78697177943832e-06
95 1.72758791450178e-06
96 1.67124881045311e-06
97 1.61672414833447e-06
98 1.56624776082026e-06
99 1.5154695347519e-06
100 1.46814795698447e-06
101 1.42304293149209e-06
102 1.37860035920312e-06
103 1.33692572035216e-06
104 1.29763736822497e-06
105 1.26096847452573e-06
106 1.22432447824394e-06
107 1.19088315386762e-06
108 1.15860689220426e-06
109 1.12798386453505e-06
110 1.09957784388826e-06
111 1.07277628558222e-06
112 1.04646562704147e-06
113 1.02233002508001e-06
114 9.98686232378532e-07
115 9.76412252384762e-07
116 9.5616769613116e-07
117 9.37478091600497e-07
118 9.20412048799335e-07
119 9.03574175481481e-07
120 8.87623059497855e-07
121 8.73065403084183e-07
122 8.60343220665527e-07
123 8.49315483719693e-07
124 8.40073084873438e-07
125 8.32137629913632e-07
126 8.25664187686925e-07
127 8.21128367078927e-07
128 8.17303316580364e-07
129 8.13096278307057e-07
130 8.07891638032743e-07
131 8.0274577385353e-07
132 7.98655719336239e-07
133 7.94480456534075e-07
134 7.90157741903386e-07
135 7.86650218742579e-07
136 7.8375762768701e-07
137 7.81621281475964e-07
138 7.80041034431633e-07
139 7.78847834226326e-07
140 7.77840170940181e-07
141 7.7702435419269e-07
142 7.76452111495018e-07
143 7.75894079652062e-07
144 7.75410001097043e-07
145 7.74820023252687e-07
146 7.74382158397202e-07
147 7.73809688325855e-07
148 7.73360170569504e-07
149 7.72984151353739e-07
150 7.72617909206019e-07
151 7.72246551150602e-07
152 7.71925726894551e-07
153 7.71528050336201e-07
154 7.71168970459257e-07
155 7.70797043969651e-07
156 7.70434212427062e-07
157 7.70050974097103e-07
158 7.69684106671775e-07
159 7.69316727655678e-07
160 7.68948098084365e-07
161 7.68595327826915e-07
162 7.68231927850138e-07
163 7.67877338603284e-07
164 7.67551625813212e-07
165 7.67187714245665e-07
166 7.6682619010171e-07
167 7.66469611335197e-07
168 7.66108769312268e-07
169 7.65755487464048e-07
170 7.65242702982505e-07
171 7.64648632411991e-07
172 7.64341393733048e-07
173 7.64006131248607e-07
174 7.63790524160868e-07
175 7.63480443310982e-07
176 7.63170532991353e-07
177 7.62875174586952e-07
178 7.62699187362159e-07
179 7.6240377211434e-07
180 7.6211932764636e-07
181 7.61649005198706e-07
182 7.61549188155186e-07
183 7.61183173381141e-07
184 7.61024409712263e-07
185 7.60841146529856e-07
186 7.60519469622523e-07
187 7.60262196308759e-07
188 7.6001515481039e-07
189 7.59573083541909e-07
190 7.59194676902553e-07
191 7.59106853820413e-07
192 7.58743453843636e-07
193 7.58658245558763e-07
194 7.58303997372423e-07
195 7.58104931719572e-07
196 7.57989823796379e-07
197 7.57708392029599e-07
198 7.57503471504606e-07
199 7.57458565203706e-07
200 7.57030306886008e-07
201 7.5662404697141e-07
202 7.56694191750285e-07
203 7.56246777200431e-07
204 7.56202382490301e-07
205 7.5565554880086e-07
206 7.55576536448643e-07
207 7.54780728584592e-07
208 7.54472921471461e-07
209 7.54148970827373e-07
210 7.53774429540499e-07
211 7.53567007905076e-07
212 7.52614710108901e-07
213 7.52303265016963e-07
214 7.52086407374009e-07
215 7.51863126424723e-07
216 7.5169214142079e-07
217 7.51480342842115e-07
218 7.51281106659007e-07
219 7.51097786633181e-07
220 7.50900085222383e-07
221 7.50654180592392e-07
222 7.50614219668932e-07
223 7.5019471523774e-07
224 7.4997467436333e-07
225 7.49802381960762e-07
226 7.49589389670291e-07
227 7.49378898490249e-07
228 7.49186824577919e-07
229 7.48975082842662e-07
230 7.48720026422234e-07
231 7.48552338336594e-07
232 7.4822071383096e-07
233 7.47968783798569e-07
234 7.47739306916628e-07
235 7.47537228562578e-07
236 7.47341971418791e-07
237 7.47104252241115e-07
238 7.46916214211524e-07
239 7.46680825614021e-07
240 7.46444072774466e-07
241 7.46274281482329e-07
242 7.46037756016449e-07
243 7.45827890114015e-07
244 7.45659917811281e-07
245 7.45479042052466e-07
246 7.45224042475456e-07
247 7.44997066703945e-07
248 7.44817157283251e-07
249 7.44580006539763e-07
250 7.44378155559389e-07
251 7.44172439226531e-07
252 7.43972691452655e-07
253 7.43766065625096e-07
254 7.43565635730192e-07
255 7.4334468536108e-07
256 7.43164207506197e-07
257 7.42970257761044e-07
258 7.42766019357077e-07
259 7.42552686006093e-07
260 7.42358338357008e-07
261 7.42122608698992e-07
262 7.41228063816379e-07
263 7.40867221793451e-07
264 7.40534403576021e-07
265 7.40260645670787e-07
266 7.40011842026433e-07
267 7.39777703984146e-07
268 7.39367862934159e-07
269 7.39077222533524e-07
270 7.38808182632056e-07
271 7.38543178613327e-07
272 7.38285393708793e-07
273 7.38065921268571e-07
274 7.3782797471722e-07
275 7.3758246799116e-07
276 7.37344407752971e-07
277 7.37113737159234e-07
278 7.36877041163098e-07
279 7.36636991405248e-07
280 7.36404615508945e-07
281 7.36170591153495e-07
282 7.35936112050695e-07
283 7.35698733933532e-07
284 7.35465164325433e-07
285 7.35194305434561e-07
286 7.3489798069204e-07
287 7.34614218345087e-07
288 7.34329091756081e-07
289 7.34078696495999e-07
290 7.33764863980468e-07
291 7.33506510641746e-07
292 7.33235310690361e-07
293 7.32953196802555e-07
294 7.3271093015137e-07
295 7.3245348630735e-07
296 7.322034889512e-07
297 7.31953264221374e-07
298 7.31525517494447e-07
299 7.31650686702778e-07
300 7.31205716419936e-07
301 7.30867668607971e-07
302 7.30526323877712e-07
303 7.30204703813797e-07
304 7.29907242202898e-07
305 7.2960330044225e-07
306 7.29325904558209e-07
307 7.29524572307128e-07
308 7.29265934751311e-07
309 7.28934253402258e-07
310 7.28619511392026e-07
311 7.28247755432676e-07
312 7.28028851426643e-07
313 7.27704502878623e-07
314 7.27277608802979e-07
315 7.26669497908006e-07
316 7.26245275473048e-07
317 7.2589784849697e-07
318 7.25587767647085e-07
319 7.25282859548315e-07
320 7.249832378875e-07
321 7.2468685630156e-07
322 7.24369272120384e-07
323 7.24088295100955e-07
324 7.23775883670896e-07
325 7.23536516034073e-07
326 7.23226776244701e-07
327 7.22893673810177e-07
328 7.22598713309708e-07
329 7.22244465123367e-07
330 7.2194944777948e-07
331 7.21804497061385e-07
332 7.21497542599536e-07
333 7.21180413165712e-07
334 7.20922400887503e-07
335 7.20594925951445e-07
336 7.20354080385732e-07
337 7.19963679784996e-07
338 7.1965160941545e-07
339 7.19368358659267e-07
340 7.19043271146802e-07
341 7.18750584383088e-07
342 7.18407704880519e-07
343 7.18085630069254e-07
344 7.17675504802173e-07
345 7.17320062904037e-07
346 7.16784938958881e-07
347 7.16413808277139e-07
348 7.15971225417888e-07
349 7.1556848979526e-07
350 7.15016710728378e-07
351 7.14662633072294e-07
352 7.14331292783754e-07
353 7.14248244548799e-07
354 7.14081181740767e-07
355 7.1415587399315e-07
356 7.14413545210846e-07
357 7.14349596364627e-07
358 7.14275472546433e-07
359 7.14145983238268e-07
360 7.13883366643131e-07
361 7.14611530838738e-07
362 7.13989948053495e-07
363 7.14273255653097e-07
364 7.15075543666899e-07
365 7.15136991402687e-07
366 7.15581393251341e-07
367 7.15110218152404e-07
368 7.14740735929809e-07
369 7.14395525847067e-07
370 7.14086411335302e-07
371 7.13723579792713e-07
372 7.13347048986179e-07
373 7.1290241976385e-07
374 7.12439771177742e-07
375 7.11863890501263e-07
376 7.11271979980665e-07
377 7.10811036697123e-07
378 7.10190874997352e-07
379 7.09592768544098e-07
380 7.08981872321601e-07
381 7.08400193616399e-07
382 7.07727679127856e-07
383 7.07138383404526e-07
384 7.06484911461303e-07
385 7.05823140378925e-07
386 7.05429329173057e-07
387 7.05480033502681e-07
388 7.04716683230799e-07
389 7.04142905760818e-07
390 7.03394221091003e-07
391 7.0322022338587e-07
392 7.03000466728554e-07
393 7.03155080827855e-07
394 7.02734439528285e-07
395 7.03796899870213e-07
396 7.03505463661713e-07
397 7.02967668075871e-07
398 7.03522857747885e-07
399 7.02758825354977e-07
400 7.01267310887488e-07
401 7.01171018135938e-07
402 6.99946724580514e-07
403 6.98799567544484e-07
404 6.97252289683092e-07
405 6.95581718446192e-07
406 6.93857714395563e-07
407 6.92169749072491e-07
408 6.90277204284939e-07
409 6.88642330715084e-07
410 6.8695942445629e-07
411 6.85172267367307e-07
412 6.83383973409946e-07
413 6.81587721373944e-07
414 6.7986445628776e-07
415 6.78045978474984e-07
416 6.76284344081068e-07
417 6.74528450872458e-07
418 6.72760393172211e-07
419 6.72115902489168e-07
420 6.70280428494152e-07
421 6.6848019741883e-07
422 6.66884091060638e-07
423 6.65240861508209e-07
424 6.63426817482105e-07
425 6.61688943637273e-07
426 6.60138709918101e-07
427 6.58522424146213e-07
428 6.57065129416878e-07
429 6.55370286040124e-07
430 6.53701135888696e-07
431 6.51959567221638e-07
432 6.50435936222493e-07
433 6.48560103400087e-07
434 6.46936143766652e-07
435 6.45152113065706e-07
436 6.4336904870288e-07
437 6.41302904114127e-07
438 6.3960374063754e-07
439 6.37829316474381e-07
440 6.36213940197194e-07
441 6.34410810107511e-07
442 6.32445676274074e-07
443 6.30119245670357e-07
444 6.27825158971973e-07
445 6.25578593371756e-07
446 6.23434118551813e-07
447 6.2121432620188e-07
448 6.19075819940917e-07
449 6.16675663422939e-07
450 6.15254748481675e-07
451 6.13239251379127e-07
452 6.11204939104937e-07
453 6.09163180342875e-07
454 6.07086235504539e-07
455 6.04964668582397e-07
456 6.02566728957754e-07
457 6.00133489569998e-07
458 5.97780285715999e-07
459 5.95538494962966e-07
460 5.93343202126562e-07
461 5.91360389989859e-07
462 5.89191927247157e-07
463 5.87026079301722e-07
464 5.84862220875948e-07
465 5.82733036935679e-07
466 5.80808261929633e-07
467 5.7861291224981e-07
468 5.76360662307707e-07
469 5.74209764181433e-07
470 5.71901523471752e-07
471 5.69796156923985e-07
472 5.68187374483387e-07
473 5.65772495519923e-07
474 5.64044853490486e-07
475 5.62136165171978e-07
476 5.60118280645838e-07
477 5.58082888346689e-07
478 5.56054942535411e-07
479 5.54003463548725e-07
480 5.51983418972668e-07
481 5.49784772374551e-07
482 5.47604145140212e-07
483 5.45615762348461e-07
484 5.43413932518888e-07
485 5.4144584282767e-07
486 5.39389247933286e-07
487 5.3761419849252e-07
488 5.35679873792105e-07
489 5.33750380782294e-07
490 5.31857324403973e-07
491 5.30009117483132e-07
492 5.28136354205344e-07
493 5.26453050042619e-07
494 5.24503377619112e-07
495 5.22859579632495e-07
496 5.21146660048544e-07
497 5.19308457569423e-07
498 5.17719229264912e-07
499 5.16116472226713e-07
500 5.1440713377815e-07
501 5.12779195105395e-07
502 5.11154098603583e-07
503 5.09885580868286e-07
504 5.08538278154447e-07
505 5.06859691995487e-07
506 5.05440709730465e-07
507 5.04070669649082e-07
508 5.03073863455938e-07
509 5.01405907016306e-07
510 5.00211001508433e-07
511 4.99321345159842e-07
512 4.981199026588e-07
513 4.96815516726201e-07
514 4.95879248774145e-07
515 4.94468679335114e-07
516 4.92928563744499e-07
517 4.9163816129294e-07
518 4.8999703494701e-07
519 4.88761259020976e-07
520 4.87417139538593e-07
521 4.86164879021089e-07
522 4.84684676393954e-07
523 4.83408030049759e-07
524 4.8204424274445e-07
525 4.80845073980163e-07
526 4.79458094559959e-07
527 4.78264553294139e-07
528 4.76884736144711e-07
529 4.75810452371661e-07
530 4.74773202085998e-07
531 4.73689681257383e-07
532 4.72736218171121e-07
533 4.71706584903586e-07
534 4.70636223326437e-07
535 4.69133624392271e-07
536 4.68273327669522e-07
537 4.67181507701753e-07
538 4.66165829493548e-07
539 4.65008298533576e-07
540 4.63751320012307e-07
541 4.6262445607681e-07
542 4.61427447362439e-07
543 4.60247491673726e-07
544 4.59193017832149e-07
545 4.58048589280224e-07
546 4.5713866825281e-07
547 4.5612273424922e-07
548 4.55075166883034e-07
549 4.53988747040057e-07
550 4.53528230082156e-07
551 4.52635816827751e-07
552 4.51738912943256e-07
553 4.50808272489667e-07
554 4.498795078689e-07
555 4.4898482087774e-07
556 4.48299431354826e-07
557 4.47674324277614e-07
558 4.47014286919512e-07
559 4.46325572056594e-07
560 4.45413974148323e-07
561 4.44486317974224e-07
562 4.43504575287079e-07
563 4.42440040160363e-07
564 4.41356007740978e-07
565 4.40272685864329e-07
566 4.39214346670269e-07
567 4.38158423321511e-07
568 4.37147491538781e-07
569 4.36106944334824e-07
570 4.35094364092947e-07
571 4.34102048529894e-07
572 4.33143952704995e-07
573 4.3198357957408e-07
574 4.30979810062126e-07
575 4.2992030557798e-07
576 4.28915996053547e-07
577 4.28026055487862e-07
578 4.27125684154817e-07
579 4.26139280307325e-07
580 4.25281143634493e-07
581 4.24376764840417e-07
582 4.23472840793693e-07
583 4.22563232405082e-07
584 4.21689065888131e-07
585 4.20763143438307e-07
586 4.19894519154695e-07
587 4.18780189193058e-07
588 4.17879590486336e-07
589 4.17138352304391e-07
590 4.16200606423445e-07
591 4.15358130112509e-07
592 4.14456877706471e-07
593 4.1357918689755e-07
594 4.12769537661006e-07
595 4.11812379752519e-07
596 4.11080009143916e-07
597 4.10137204198691e-07
598 4.09150601399233e-07
599 4.08087601044826e-07
600 4.07251661727059e-07
601 4.06355098903077e-07
602 4.05450066409685e-07
603 4.04567543910161e-07
604 4.03703211304673e-07
605 4.02713283165212e-07
606 4.01833631258341e-07
607 4.010233851659e-07
608 4.00185200533087e-07
609 3.993947075287e-07
610 3.98713837057585e-07
611 3.977979758929e-07
612 3.9694069187135e-07
613 3.96239528299702e-07
614 3.95606718939234e-07
615 3.9492084624726e-07
616 3.9421280462193e-07
617 3.9346579683297e-07
618 3.9270685192605e-07
619 3.91947537536907e-07
620 3.91211500527788e-07
621 3.90210402656521e-07
622 3.89294058322776e-07
623 3.88405425155725e-07
624 3.874825154071e-07
625 3.86572736488233e-07
626 3.85779969747091e-07
627 3.84881161608064e-07
628 3.83941141990363e-07
629 3.83066634412899e-07
630 3.82116468244931e-07
631 3.81250913505937e-07
632 3.80365435148633e-07
633 3.79539727646261e-07
634 3.78494036112897e-07
635 3.77581073962574e-07
636 3.76837192561652e-07
637 3.75804177110695e-07
638 3.74936149682981e-07
639 3.74076279285873e-07
640 3.73203988601745e-07
641 3.72336046439159e-07
642 3.71479842442568e-07
643 3.70597007304241e-07
644 3.69759959539806e-07
645 3.68904721881336e-07
646 3.68146601203989e-07
647 3.67268825129941e-07
648 3.66433994258841e-07
649 3.65684826419965e-07
650 3.64802701824374e-07
651 3.63959372862155e-07
652 3.63126900992938e-07
653 3.62277944532252e-07
654 3.61436121920633e-07
655 3.60604360594152e-07
656 3.59784905867855e-07
657 3.59391094661987e-07
658 3.58872455308301e-07
659 3.58097878461194e-07
660 3.5733816616812e-07
661 3.56567142034692e-07
662 3.55643976490683e-07
663 3.54884065245642e-07
664 3.54119350731708e-07
665 3.53364129068723e-07
666 3.52609362153089e-07
667 3.51862297520711e-07
668 3.51105626350545e-07
669 3.50360664924665e-07
670 3.49599304172443e-07
671 3.4883902344518e-07
672 3.48074280509536e-07
673 3.47394887967312e-07
674 3.46620680602427e-07
675 3.45852328109686e-07
676 3.45076529129074e-07
677 3.44322927503526e-07
678 3.4355181810497e-07
679 3.42784801432572e-07
680 3.42015397336581e-07
681 3.41254406066582e-07
682 3.40493670591968e-07
683 3.39633146495544e-07
684 3.38669451593887e-07
685 3.37484067358673e-07
686 3.36482884222278e-07
687 3.36127470745851e-07
688 3.35413574248378e-07
689 3.35021155706272e-07
690 3.342373418036e-07
691 3.33422804033034e-07
692 3.32601331365368e-07
693 3.31793273744552e-07
694 3.30981720253476e-07
695 3.30178096419331e-07
696 3.29376348418009e-07
697 3.28576049923868e-07
698 3.27774728248187e-07
699 3.26977271924989e-07
700 3.26183624110854e-07
701 3.25404840850751e-07
702 3.24679206187284e-07
703 3.2388351201007e-07
704 3.23132667290338e-07
705 3.22288002507776e-07
706 3.21549862292159e-07
707 3.20637866479956e-07
708 3.19880626875602e-07
709 3.18680406508065e-07
710 3.17890027190515e-07
711 3.1709188874629e-07
712 3.16297644076258e-07
713 3.15523180915989e-07
714 3.14737633289042e-07
715 3.13946799224141e-07
716 3.13174609800626e-07
717 3.12392671730777e-07
718 3.11604765101947e-07
719 3.10808275116869e-07
720 3.10051632368413e-07
721 3.09266937392749e-07
722 3.0850148391437e-07
723 3.07723297510165e-07
724 3.06919872627986e-07
725 3.06108006498107e-07
726 3.05222840779606e-07
727 3.04393267924752e-07
728 3.03570232063066e-07
729 3.02734974866326e-07
730 3.0191742439456e-07
731 3.01087510479192e-07
732 3.00292072097363e-07
733 2.99473867926281e-07
734 2.98629544204232e-07
735 2.97798010251427e-07
736 2.96959882462033e-07
737 2.9612576213367e-07
738 2.9529198286582e-07
739 2.9440030857586e-07
740 2.93578153787166e-07
741 2.92831373371882e-07
742 2.92043836225275e-07
743 2.91227735260691e-07
744 2.90416551251838e-07
745 2.89597920755114e-07
746 2.88768859491029e-07
747 2.87931470666081e-07
748 2.87065688553412e-07
749 2.86193625242959e-07
750 2.85330600036104e-07
751 2.84467176925318e-07
752 2.83600968487008e-07
753 2.82731065226471e-07
754 2.81803721691176e-07
755 2.80763316595767e-07
756 2.79886876342061e-07
757 2.79010436088356e-07
758 2.78123650332418e-07
759 2.77232345524681e-07
760 2.76336976412495e-07
761 2.75454340226133e-07
762 2.74565678637373e-07
763 2.73707826181635e-07
764 2.72829652203654e-07
765 2.71943918050965e-07
766 2.71049771072285e-07
767 2.70155453563348e-07
768 2.69204605274354e-07
769 2.68268479430844e-07
770 2.67335479975372e-07
771 2.66394437176132e-07
772 2.65471015836738e-07
773 2.64551090367604e-07
774 2.63648217924128e-07
775 2.62752081425788e-07
776 2.61852079574965e-07
777 2.60949775565678e-07
778 2.60049318967503e-07
779 2.59149516068646e-07
780 2.58248093132352e-07
781 2.57344652254687e-07
782 2.56442177715144e-07
783 2.55539930549276e-07
784 2.54640497132641e-07
785 2.53741632150195e-07
786 2.52842852432877e-07
787 2.51946119078639e-07
788 2.51050295219102e-07
789 2.50151600766912e-07
790 2.49251002060191e-07
791 2.48348953846289e-07
792 2.47477260018059e-07
793 2.4657603603373e-07
794 2.45668502429908e-07
795 2.44760059331384e-07
796 2.4384161179114e-07
797 2.42932031824239e-07
798 2.42020121277164e-07
799 2.41108267573509e-07
800 2.4018112299018e-07
801 2.39255825817963e-07
802 2.38318733636333e-07
803 2.37405913594557e-07
804 2.36499417383129e-07
805 2.35617619637196e-07
806 2.34714121916113e-07
807 2.338088762599e-07
808 2.32899253660435e-07
809 2.3202079546536e-07
810 2.31111769721792e-07
811 2.30196064876509e-07
812 2.29280288976952e-07
813 2.28376421773646e-07
814 2.27474046710086e-07
815 2.2656490727968e-07
816 2.25659647412613e-07
817 2.24749200583574e-07
818 2.23844338620438e-07
819 2.22935042870631e-07
820 2.22047816578197e-07
821 2.21123855226324e-07
822 2.20203062895052e-07
823 2.19283066371645e-07
824 2.18357456560625e-07
825 2.17431846749605e-07
826 2.16511480743975e-07
827 2.15593203733988e-07
828 2.14685883292987e-07
829 2.1380476766808e-07
830 2.12907281138541e-07
831 2.11996635357536e-07
832 2.11066222277623e-07
833 2.10128234812146e-07
834 2.09204188195145e-07
835 2.08285172220712e-07
836 2.07380011829628e-07
837 2.06485907483511e-07
838 2.05598226443726e-07
839 2.04705031592312e-07
840 2.03844464863323e-07
841 2.02951127903361e-07
842 2.02049250219716e-07
843 2.01165121893609e-07
844 2.00300661390429e-07
845 1.99437934611524e-07
846 1.98584544364167e-07
847 1.97746928165543e-07
848 1.96902590232639e-07
849 1.9606652301718e-07
850 1.95221545595814e-07
851 1.94384426777106e-07
852 1.93548487459339e-07
853 1.92730823300735e-07
854 1.91903396284943e-07
855 1.91085291589843e-07
856 1.90249991760538e-07
857 1.8941712198739e-07
858 1.88595620898013e-07
859 1.86982731520402e-07
860 1.84085365617648e-07
861 1.82015185146156e-07
862 1.80250466996767e-07
863 1.78771983883053e-07
864 1.77435651949054e-07
865 1.76675413854355e-07
866 1.75529834223198e-07
867 1.74375941242033e-07
868 1.73357449284595e-07
869 1.72397363940036e-07
870 1.71481076449709e-07
871 1.70611883731908e-07
872 1.69786673609451e-07
873 1.68969023661703e-07
874 1.68175972703466e-07
875 1.6738563601848e-07
876 1.66620523600614e-07
877 1.65878404345676e-07
878 1.65168017929318e-07
879 1.6445190453851e-07
880 1.63762393867728e-07
881 1.63073110570622e-07
882 1.62421116556288e-07
883 1.61783631824619e-07
884 1.6115485834689e-07
885 1.60520116310181e-07
886 1.59858146275837e-07
887 1.59376966735181e-07
888 1.58766582103453e-07
889 1.58158798058139e-07
890 1.57568422309851e-07
891 1.56993081645851e-07
892 1.56440052023754e-07
893 1.55930621303924e-07
894 1.55436993054536e-07
895 1.54940522634206e-07
896 1.54465936930137e-07
897 1.53985439510507e-07
898 1.53522833556963e-07
899 1.53065400354535e-07
900 1.52621950633147e-07
901 1.52197117131436e-07
902 1.51759024902276e-07
903 1.51331761344409e-07
904 1.50895203887558e-07
905 1.50472899917986e-07
906 1.50042225754987e-07
907 1.49620561273878e-07
908 1.49187783904381e-07
909 1.48748839023938e-07
910 1.48289572621252e-07
911 1.47863204347232e-07
912 1.47437873465606e-07
913 1.46868899264518e-07
914 1.46429982805785e-07
915 1.45985936228499e-07
916 1.45564257536535e-07
917 1.45145492069787e-07
918 1.44741605367926e-07
919 1.4434125716889e-07
920 1.43948710729092e-07
921 1.4351543597968e-07
922 1.43138109365282e-07
923 1.42755908427716e-07
924 1.42388898893842e-07
925 1.4202220199877e-07
926 1.41668664355166e-07
927 1.41311673473865e-07
928 1.40893888556093e-07
929 1.40499082590395e-07
930 1.40143029625506e-07
931 1.39786223485316e-07
932 1.39429175760597e-07
933 1.39083880412727e-07
934 1.38732048071688e-07
935 1.38380741532274e-07
936 1.38037592023466e-07
937 1.37692381940724e-07
938 1.37364935426376e-07
939 1.37030113478431e-07
940 1.36708720788192e-07
941 1.36379441073586e-07
942 1.36049493448809e-07
943 1.35719147920099e-07
944 1.35366647668889e-07
945 1.35022659719652e-07
946 1.34657909711677e-07
947 1.34318185018856e-07
948 1.3399960607785e-07
949 1.336940016472e-07
950 1.33399652213484e-07
951 1.33103213784125e-07
952 1.32813312347935e-07
953 1.32519645035245e-07
954 1.32227924609651e-07
955 1.31930775637557e-07
956 1.31640163658631e-07
957 1.31359641386553e-07
958 1.31085940324738e-07
959 1.30821518951052e-07
960 1.30557793909247e-07
961 1.30302382217451e-07
962 1.30050793245573e-07
963 1.29793576775228e-07
964 1.2952703798419e-07
965 1.29254118519384e-07
966 1.28999644743999e-07
967 1.28731954873729e-07
968 1.28484472838863e-07
969 1.28226545825783e-07
970 1.27972313634928e-07
971 1.27690029216865e-07
972 1.27396916127509e-07
973 1.27131926319635e-07
974 1.26874226680229e-07
975 1.26632770047763e-07
976 1.26389366528201e-07
977 1.26152485790954e-07
978 1.25905415870875e-07
979 1.25652334759252e-07
980 1.25398571526603e-07
981 1.25146115692587e-07
982 1.24889055541644e-07
983 1.24653595889868e-07
984 1.24411670299196e-07
985 1.24178114901952e-07
986 1.23941461538379e-07
987 1.23705504506688e-07
988 1.23464701573539e-07
989 1.23231870929885e-07
990 1.22994535445287e-07
991 1.22773130328824e-07
992 1.22545017688935e-07
993 1.22312158623572e-07
994 1.22094903076686e-07
995 1.21837928190871e-07
996 1.2158551498942e-07
997 1.21358255000814e-07
998 1.21141795261792e-07
999 1.2091312839857e-07
1000 1.20691083793645e-07
1001 1.20461649544268e-07
1002 1.20223432986677e-07
1003 1.20004202130986e-07
1004 1.19775961593405e-07
1005 1.19568582590546e-07
1006 1.19363036787945e-07
1007 1.19148857891105e-07
1008 1.18948165095389e-07
1009 1.1873245142624e-07
1010 1.18528205916846e-07
1011 1.18312271979448e-07
1012 1.18108680169371e-07
1013 1.17894224160864e-07
1014 1.17694582968397e-07
1015 1.17493208051656e-07
1016 1.17290007040083e-07
1017 1.17101997432201e-07
1018 1.16907308722602e-07
1019 1.16714574005528e-07
1020 1.16519082382638e-07
1021 1.16337453448523e-07
1022 1.16149060147563e-07
1023 1.15958087576473e-07
1024 1.15778682641121e-07
1025 1.1559897927782e-07
1026 1.15414870549557e-07
1027 1.15240460729638e-07
1028 1.15056295157956e-07
1029 1.14888592861462e-07
1030 1.14713280652268e-07
1031 1.14548925012059e-07
1032 1.14384263838474e-07
1033 1.14228626557633e-07
1034 1.14095499270661e-07
1035 1.13974209625667e-07
1036 1.13827404391031e-07
1037 1.1365743546321e-07
1038 1.13484659891583e-07
1039 1.13314271743548e-07
1040 1.13152417213769e-07
1041 1.13000965029642e-07
1042 1.12842954536063e-07
1043 1.12690138109883e-07
1044 1.12531758134082e-07
1045 1.12379730410339e-07
1046 1.12232420690361e-07
1047 1.12081480097004e-07
1048 1.11931598212323e-07
1049 1.11787983314571e-07
1050 1.1164256363827e-07
1051 1.11498657417997e-07
1052 1.11354623300031e-07
1053 1.11211939213263e-07
1054 1.11069226704785e-07
1055 1.10924460727801e-07
1056 1.10776866790729e-07
1057 1.10637209616016e-07
1058 1.10494781324633e-07
1059 1.10356282334578e-07
1060 1.10220931048843e-07
1061 1.10075873749338e-07
1062 1.09903488976215e-07
1063 1.09747951171357e-07
1064 1.09611320908698e-07
1065 1.0947871942335e-07
1066 1.09345407395267e-07
1067 1.09213722510049e-07
1068 1.09079131505041e-07
1069 1.08950011679099e-07
1070 1.08819577349095e-07
1071 1.08693662070891e-07
1072 1.08563781964222e-07
1073 1.08438634072172e-07
1074 1.08313855662345e-07
1075 1.08188636716022e-07
1076 1.08063836989913e-07
1077 1.07940927307482e-07
1078 1.07814230432268e-07
1079 1.07691725759196e-07
1080 1.07564730456033e-07
1081 1.07430516038676e-07
1082 1.07337001509222e-07
1083 1.07212919431277e-07
1084 1.07087579692688e-07
1085 1.06962630752605e-07
1086 1.06843906166887e-07
1087 1.06718424319752e-07
1088 1.06594434612362e-07
1089 1.06465229521291e-07
1090 1.06336528915563e-07
1091 1.0620792068039e-07
1092 1.06079284023508e-07
1093 1.05957148832658e-07
1094 1.05834708108432e-07
1095 1.05716161158398e-07
1096 1.05598019217723e-07
1097 1.05479067258329e-07
1098 1.05356981805471e-07
1099 1.05235208991417e-07
1100 1.05119589477454e-07
1101 1.05012446738328e-07
1102 1.04910391485191e-07
1103 1.0481100787274e-07
1104 1.04725707217312e-07
1105 1.04630480279866e-07
1106 1.04541818757298e-07
1107 1.04442719361941e-07
1108 1.04334873185508e-07
1109 1.0424158602973e-07
1110 1.04134720402271e-07
1111 1.04043252235897e-07
1112 1.03938582185492e-07
1113 1.03847220600528e-07
1114 1.03743111878885e-07
1115 1.03650933169774e-07
1116 1.0354698787296e-07
1117 1.03452769906198e-07
1118 1.03363170467219e-07
1119 1.03268178008875e-07
1120 1.03179438326606e-07
1121 1.03089718095362e-07
1122 1.03014762942166e-07
1123 1.02917901756427e-07
1124 1.02823555891973e-07
1125 1.02732428786112e-07
1126 1.02643660682133e-07
1127 1.02556910519525e-07
1128 1.02480456121157e-07
1129 1.02393606482565e-07
1130 1.0231348568368e-07
1131 1.02223026487991e-07
1132 1.02147147629239e-07
1133 1.02064703355609e-07
1134 1.01982465139372e-07
1135 1.01901818538863e-07
1136 1.01820688769294e-07
1137 1.01734414670318e-07
1138 1.01658400808446e-07
1139 1.01571458799299e-07
1140 1.01489057158233e-07
1141 1.01410797981316e-07
1142 1.0133041428162e-07
1143 1.0124770000175e-07
1144 1.01164637555939e-07
1145 1.010783989841e-07
1146 1.00992735951877e-07
1147 1.00908998490468e-07
1148 1.00828195570557e-07
1149 1.0075106615659e-07
1150 1.00674547809376e-07
1151 1.0059626021075e-07
1152 1.00530378688291e-07
1153 1.0045393139535e-07
1154 1.00399326186107e-07
1155 1.00338425568225e-07
1156 1.0027530095158e-07
1157 1.00212595555149e-07
1158 1.00127046209764e-07
1159 1.0005913253508e-07
1160 9.99867992845793e-08
1161 9.99129881051886e-08
1162 9.98482363456787e-08
1163 9.97857370066413e-08
1164 9.97121247792165e-08
1165 9.96562263821943e-08
1166 9.95880640175528e-08
1167 9.9529273711596e-08
1168 9.94557041167354e-08
1169 9.9402384989844e-08
1170 9.93292132989154e-08
1171 9.92762352325371e-08
1172 9.92015145584446e-08
1173 9.91449979892423e-08
1174 9.90723734162202e-08
1175 9.90113946386373e-08
1176 9.89474315815642e-08
1177 9.88768604770485e-08
1178 9.88306894100788e-08
1179 9.87695187859572e-08
1180 9.87151338449621e-08
1181 9.86532597835321e-08
1182 9.85910730832984e-08
1183 9.85363399763628e-08
1184 9.84910926149496e-08
1185 9.84241879109504e-08
1186 9.83694761202969e-08
1187 9.83074741611745e-08
1188 9.82329240173385e-08
1189 9.81613297312833e-08
1190 9.80864669486436e-08
1191 9.79614753759961e-08
1192 9.78657581640618e-08
1193 9.7683873434562e-08
1194 9.76131602214991e-08
1195 9.74843956669247e-08
1196 9.73977734020082e-08
1197 9.73452642938355e-08
1198 9.72973168700264e-08
1199 9.72512950170312e-08
1200 9.72039941871117e-08
1201 9.71002762639728e-08
1202 9.70413438494688e-08
1203 9.69876552403548e-08
1204 9.69340376855143e-08
1205 9.68826086023e-08
1206 9.68308739857093e-08
1207 9.67742366242419e-08
1208 9.67210453950429e-08
1209 9.66655164802432e-08
1210 9.6611167066385e-08
1211 9.65518651696584e-08
1212 9.64963575711408e-08
1213 9.63851007895755e-08
1214 9.63316182378549e-08
1215 9.627709829374e-08
1216 9.62181232466719e-08
1217 9.61659196718756e-08
1218 9.61136663590878e-08
1219 9.60582013931344e-08
1220 9.60050812182089e-08
1221 9.59535313427295e-08
1222 9.59029406999434e-08
1223 9.58557251351522e-08
1224 9.58086658897628e-08
1225 9.57593258021916e-08
1226 9.57109946853052e-08
1227 9.56648591454723e-08
1228 9.56184678102545e-08
1229 9.55737249341837e-08
1230 9.55299697125156e-08
1231 9.54899093130734e-08
1232 9.54443351020018e-08
1233 9.54034504729862e-08
1234 9.53578052076409e-08
1235 9.53154781768717e-08
1236 9.52716305846479e-08
1237 9.52312291246926e-08
1238 9.51892928924281e-08
1239 9.5150923584697e-08
1240 9.51080423305939e-08
1241 9.50704404090175e-08
1242 9.50291365597877e-08
1243 9.49903409264152e-08
1244 9.49472322986367e-08
1245 9.49110443571044e-08
1246 9.48706073700123e-08
1247 9.48307814496729e-08
1248 9.47912184301458e-08
1249 9.47514351423706e-08
1250 9.47102876125427e-08
1251 9.4673950457036e-08
1252 9.46314315797281e-08
1253 9.45941991403743e-08
1254 9.45585156841844e-08
1255 9.45199332136326e-08
1256 9.44842071248786e-08
1257 9.44436919780856e-08
1258 9.4409088546854e-08
1259 9.43715221524144e-08
1260 9.43322646662637e-08
1261 9.42919911040008e-08
1262 9.42546449778092e-08
1263 9.42130355952031e-08
1264 9.41750997185409e-08
1265 9.41363396123052e-08
1266 9.40994908660286e-08
1267 9.40589757192356e-08
1268 9.40020257189644e-08
1269 9.39683957312809e-08
1270 9.39300335289772e-08
1271 9.38983433229623e-08
1272 9.38633775149356e-08
1273 9.38288167162682e-08
1274 9.37947461920885e-08
1275 9.37587643079496e-08
1276 9.37232016440248e-08
1277 9.3688989011298e-08
1278 9.36558208763927e-08
1279 9.36210611257593e-08
1280 9.35854060912789e-08
1281 9.35542274760337e-08
1282 9.35220185738217e-08
1283 9.34915149741755e-08
1284 9.34579773570476e-08
1285 9.34277366582137e-08
1286 9.33959540816431e-08
1287 9.33627504196011e-08
1288 9.33339876496575e-08
1289 9.33025106064633e-08
1290 9.3273037293784e-08
1291 9.32471380110655e-08
1292 9.32184462953956e-08
1293 9.31889942989983e-08
1294 9.31534103187914e-08
1295 9.31263457459863e-08
1296 9.30980519342484e-08
1297 9.30718826452903e-08
1298 9.30381958141879e-08
1299 9.30114865127507e-08
1300 9.29776575731012e-08
1301 9.29448447095638e-08
1302 9.29165935303899e-08
1303 9.28824306356546e-08
1304 9.28505912156652e-08
1305 9.28201941974294e-08
1306 9.27869763245326e-08
1307 9.27545116269357e-08
1308 9.27224235169888e-08
1309 9.2687756136911e-08
1310 9.26523995303796e-08
1311 9.26189329675253e-08
1312 9.25829723996685e-08
1313 9.2547196572923e-08
1314 9.25159184816948e-08
1315 9.24828356119178e-08
1316 9.24504419685945e-08
1317 9.24220060483094e-08
1318 9.2390465056269e-08
1319 9.23612120118378e-08
1320 9.23327476698432e-08
1321 9.2302222753915e-08
1322 9.22730905017488e-08
1323 9.22422245253074e-08
1324 9.22139307135694e-08
1325 9.21908522855119e-08
1326 9.21612723914222e-08
1327 9.21334191161804e-08
1328 9.21039600143558e-08
1329 9.20762488476612e-08
1330 9.20464131581866e-08
1331 9.20189151543127e-08
1332 9.19890794648381e-08
1333 9.19611693461775e-08
1334 9.19348863703817e-08
1335 9.19071041494135e-08
1336 9.18803522154121e-08
1337 9.18612670375296e-08
1338 9.18360072432733e-08
1339 9.18119766879499e-08
1340 9.17870508487795e-08
1341 9.17630060826014e-08
1342 9.17360978291981e-08
1343 9.17121951715671e-08
1344 9.16874043355165e-08
1345 9.16548117402272e-08
1346 9.16270650463957e-08
1347 9.16026863251318e-08
1348 9.15757993880106e-08
1349 9.15509161814043e-08
1350 9.15235816023596e-08
1351 9.14958917519471e-08
1352 9.14687916520052e-08
1353 9.14410449581737e-08
1354 9.14115361183576e-08
1355 9.13860844775627e-08
1356 9.1360597309631e-08
1357 9.13404036850807e-08
1358 9.13138293867632e-08
1359 9.12920157247754e-08
1360 9.1271999735909e-08
1361 9.12541580078141e-08
1362 9.12334385816393e-08
1363 9.12151136844841e-08
1364 9.11948134785234e-08
1365 9.11808015757742e-08
1366 9.11593645014364e-08
1367 9.11441873086005e-08
1368 9.11197091113536e-08
1369 9.1096907794963e-08
1370 9.10580268964623e-08
1371 9.10398725295636e-08
1372 9.10094328787636e-08
1373 9.10200910198e-08
1374 9.12837663236132e-08
1375 9.09564192852486e-08
1376 9.13038036287617e-08
1377 9.09311310692829e-08
1378 9.12806115138665e-08
1379 9.08721915493516e-08
1380 9.12469246827641e-08
1381 9.08061892346268e-08
1382 9.12008033537859e-08
1383 9.07804960093017e-08
1384 9.11766164790606e-08
1385 9.07330317545529e-08
1386 9.11444786311222e-08
1387 9.0689418641432e-08
1388 9.11288253746534e-08
1389 9.06816595147575e-08
1390 9.10608193294138e-08
1391 9.06394177491165e-08
1392 9.10159840827873e-08
1393 9.05553889651856e-08
1394 9.09801514126229e-08
1395 9.10256758857031e-08
1396 9.10253064034805e-08
1397 9.10255337771559e-08
1398 9.10199275949708e-08
1399 9.10074646753856e-08
1400 9.09982986740943e-08
1401 9.09816861849322e-08
1402 9.0971568056375e-08
1403 9.09537831716989e-08
1404 9.09361688172794e-08
1405 9.09190802644844e-08
1406 9.09033062157505e-08
1407 9.08852229031254e-08
1408 9.08711399461026e-08
1409 9.08515289665957e-08
1410 9.08336517113639e-08
1411 9.08164139445944e-08
1412 9.07981601017127e-08
1413 9.077920992695e-08
1414 9.07600181676571e-08
1415 9.07439101638374e-08
1416 9.0723396795056e-08
1417 9.0705910338329e-08
1418 9.06844448422817e-08
1419 9.06699000324807e-08
1420 9.06504524778029e-08
1421 9.06331152350504e-08
1422 9.06134900446887e-08
1423 9.05919321780857e-08
1424 9.05740051848625e-08
1425 9.05448800381237e-08
1426 9.05357921965333e-08
1427 9.0505544392272e-08
1428 9.04967123460665e-08
1429 9.04662655898392e-08
1430 9.04584425143184e-08
1431 9.04281804992024e-08
1432 9.04200945228695e-08
1433 9.0389704610061e-08
1434 9.03794159512472e-08
1435 9.03511292449366e-08
1436 9.03393910789418e-08
1437 9.03127101992141e-08
1438 9.0292701315775e-08
1439 9.02819721204651e-08
1440 9.02581973605265e-08
1441 9.02374992506338e-08
1442 9.02213272979679e-08
1443 9.02023700177779e-08
1444 9.01856793689149e-08
1445 9.01680934362048e-08
1446 9.01493066862713e-08
1447 9.01324099800149e-08
1448 9.01144474596549e-08
1449 9.00997108033152e-08
1450 9.00819756566307e-08
1451 9.00634731237915e-08
1452 9.00453756003117e-08
1453 9.00281307281148e-08
1454 9.00095074030105e-08
1455 8.99906638096581e-08
1456 8.99731844583584e-08
1457 8.99564298606492e-08
1458 8.99385739216996e-08
1459 8.99256420439087e-08
1460 8.99134917631272e-08
1461 8.99013912203372e-08
1462 8.98881964417342e-08
1463 8.98737440024888e-08
1464 8.9872514763556e-08
1465 8.98574654684126e-08
1466 8.98432972462615e-08
1467 8.98286813821869e-08
1468 8.98141792049501e-08
1469 8.97978154057455e-08
1470 8.97859990800498e-08
1471 8.97625440643424e-08
1472 8.97429544011175e-08
1473 8.97282603773419e-08
1474 8.97122589549326e-08
1475 8.96948932904706e-08
1476 8.96798084681905e-08
1477 8.96625707014209e-08
1478 8.96476635148247e-08
1479 8.96309160225428e-08
1480 8.9572800732185e-08
1481 8.95631799835428e-08
1482 8.95509941756245e-08
1483 8.95376359721922e-08
1484 8.9526245972138e-08
1485 8.95136196277235e-08
1486 8.95023859470712e-08
1487 8.94882816737663e-08
1488 8.94758045433264e-08
1489 8.94632279369034e-08
1490 8.9448477069709e-08
1491 8.94327172318299e-08
1492 8.94189611244656e-08
1493 8.94043381549636e-08
1494 8.93888767450335e-08
1495 8.93749003694211e-08
1496 8.9358664467909e-08
1497 8.93438709681504e-08
1498 8.93286440373231e-08
1499 8.93137084290174e-08
1500 8.92973801569497e-08
1501 8.92825937626185e-08
1502 8.92680560582448e-08
1503 8.92531204499392e-08
1504 8.92415030762095e-08
1505 8.92276759145716e-08
1506 8.92136924335318e-08
1507 8.91999505370222e-08
1508 8.9186016793974e-08
1509 8.91716069872928e-08
1510 8.91566571681324e-08
1511 8.91422757831606e-08
1512 8.91268427949399e-08
1513 8.91138185465934e-08
1514 8.90984139800821e-08
1515 8.90861286961808e-08
1516 8.90707880785158e-08
1517 8.90565843292279e-08
1518 8.90400428943394e-08
1519 8.90271465436854e-08
1520 8.90115146034987e-08
1521 8.89979929752371e-08
1522 8.89824818273155e-08
1523 8.89673756887532e-08
1524 8.89488873667688e-08
1525 8.89392168801351e-08
1526 8.89308751084172e-08
1527 8.89176234863953e-08
1528 8.88942963683803e-08
1529 8.88804052578962e-08
1530 8.88653417518981e-08
1531 8.88459439352118e-08
1532 8.88294096057507e-08
1533 8.88169751078749e-08
1534 8.88012934296967e-08
1535 8.84224817809809e-08
1536 8.87343816202701e-08
1537 8.86933122501432e-08
1538 8.8658218544424e-08
1539 8.86214408524211e-08
1540 8.86362272467522e-08
1541 8.85943478579065e-08
1542 8.85860629296076e-08
1543 8.85481128420906e-08
1544 8.85526603155995e-08
1545 8.85245441395455e-08
1546 8.8497273509347e-08
1547 8.84996396166571e-08
1548 8.84673596601715e-08
1549 8.84652990862378e-08
1550 8.84059616623745e-08
1551 8.84341062601379e-08
1552 8.83751312130698e-08
1553 8.84025581626702e-08
1554 8.83426451991909e-08
1555 8.83710455923392e-08
1556 8.83166606513441e-08
1557 8.83358950432012e-08
1558 8.82732180684798e-08
1559 8.83075443880443e-08
1560 8.82402915181046e-08
1561 8.82576358662845e-08
1562 8.82071375940541e-08
1563 8.821744756915e-08
1564 8.8187903202197e-08
1565 8.81721362588905e-08
1566 8.8153925048573e-08
1567 8.81375541439411e-08
1568 8.81203874314451e-08
1569 8.81022899079653e-08
1570 8.80843629147421e-08
1571 8.80682122783583e-08
1572 8.80520829582565e-08
1573 8.80319674934071e-08
1574 8.8013287324884e-08
1575 8.79947847920448e-08
1576 8.79785417851053e-08
1577 8.79619932447895e-08
1578 8.79462618286198e-08
1579 8.79177477486337e-08
1580 8.78994228514784e-08
1581 8.78792292269281e-08
1582 8.76788064374523e-08
1583 8.78400356896236e-08
1584 8.78010624205672e-08
1585 8.77551755706918e-08
1586 8.77284804801093e-08
1587 8.76920154269101e-08
1588 8.76550885209326e-08
1589 8.76184031994853e-08
1590 8.76029488949825e-08
1591 8.7577504359615e-08
1592 8.72517702532605e-08
1593 8.75141665801493e-08
1594 8.72201511015191e-08
1595 8.70558736210114e-08
1596 8.72507328608663e-08
1597 8.73272725243623e-08
1598 8.72173302468582e-08
1599 8.71703136340329e-08
1600 8.71420624548591e-08
1601 8.71710383876234e-08
1602 8.7076053034707e-08
1603 8.70392611318493e-08
1604 8.70555538767803e-08
1605 8.69931895408627e-08
1606 8.69130261094142e-08
1607 8.67927525405321e-08
1608 8.68926477437526e-08
1609 8.69249419110929e-08
1610 8.69055298835519e-08
1611 8.68854144187026e-08
1612 8.68723262215099e-08
1613 8.68476348614422e-08
1614 8.65971188090953e-08
1615 8.6696978485179e-08
1616 8.66000036126024e-08
1617 8.67074732013862e-08
1618 8.68273275500542e-08
1619 8.67589733388741e-08
1620 8.68079439442226e-08
1621 8.67788116920565e-08
1622 8.66062634941045e-08
1623 8.67804388349214e-08
1624 8.66894396267526e-08
1625 8.6620062234033e-08
1626 8.66238281105325e-08
1627 8.65578755337992e-08
1628 8.66445049041431e-08
1629 8.64311857640132e-08
1630 8.66059224335913e-08
1631 8.64986517967736e-08
1632 8.64926832377932e-08
1633 8.64073328443737e-08
1634 8.63197158196272e-08
1635 8.62176747773447e-08
1636 8.66976250790685e-08
1637 8.62486260189144e-08
1638 8.67430429707383e-08
1639 8.68153335886745e-08
1640 8.68291891720219e-08
1641 8.68354561589513e-08
1642 8.67957297145949e-08
1643 8.68020038069517e-08
1644 8.68026930334054e-08
1645 8.68037020040902e-08
1646 8.6764138984563e-08
1647 8.6764117668281e-08
1648 8.67237801571719e-08
1649 8.67267146986705e-08
1650 8.66999769755239e-08
1651 8.66911307184637e-08
1652 8.66843592461919e-08
1653 8.66643574681802e-08
1654 8.64999094574159e-08
1655 8.66337614979784e-08
1656 8.61825171227792e-08
1657 8.64851159576574e-08
1658 8.65588134502104e-08
1659 8.6577898628093e-08
1660 8.65509548475529e-08
1661 8.65477431943873e-08
1662 8.64957669932664e-08
1663 8.65083649159715e-08
1664 8.64796732003015e-08
1665 8.64744222894842e-08
1666 8.64263967059742e-08
1667 8.64151772361765e-08
1668 8.64175788706234e-08
1669 8.61052313894106e-08
1670 8.640290616313e-08
1671 8.6395196774447e-08
1672 8.63906208792287e-08
1673 8.63660929439902e-08
1674 8.62100932863541e-08
1675 8.62951949898161e-08
1676 8.6291159107077e-08
1677 8.62861142536531e-08
1678 8.62213909158527e-08
1679 8.62666169609838e-08
1680 8.62115498989624e-08
1681 8.62179092564475e-08
1682 8.62063131989999e-08
1683 8.62078337604544e-08
1684 8.61845705912856e-08
1685 8.61851177091921e-08
1686 8.61462154944093e-08
1687 8.61548201669393e-08
1688 8.61384847894442e-08
1689 8.61206643776313e-08
1690 8.61046487443673e-08
1691 8.60928111023895e-08
1692 8.60825295490031e-08
1693 8.57939781440109e-08
1694 8.60285993553589e-08
1695 8.60062172591824e-08
1696 8.60067217445248e-08
1697 8.60055990870023e-08
1698 8.57588844382917e-08
1699 8.56488497902319e-08
1700 8.59252082818784e-08
1701 8.59078568282712e-08
1702 8.59196944702489e-08
1703 8.57580744195729e-08
1704 8.57637019180402e-08
1705 8.58164739270251e-08
1706 8.5785416104045e-08
1707 8.57809467902371e-08
1708 8.57685051869339e-08
1709 8.57573425605551e-08
1710 8.56933226600631e-08
1711 8.56833466400531e-08
1712 8.56480113498037e-08
1713 8.56360529155609e-08
1714 8.56453326036899e-08
1715 8.56153619110955e-08
1716 8.55784207942634e-08
1717 8.55435402513649e-08
1718 8.55372093155893e-08
1719 8.55175557035182e-08
1720 8.55005382049967e-08
1721 8.54843449360487e-08
1722 8.54917274750733e-08
1723 8.5414747275081e-08
1724 8.54025969942995e-08
1725 8.53785167009846e-08
1726 8.53606252348982e-08
1727 8.53503934195032e-08
1728 8.53566817227147e-08
1729 8.51540704616127e-08
1730 8.52681836249758e-08
1731 8.52572057397083e-08
1732 8.52717789712187e-08
1733 8.50977670552311e-08
1734 8.51852703931399e-08
1735 8.51602663942685e-08
1736 8.51553849656739e-08
1737 8.50110950523231e-08
1738 8.508123272577e-08
1739 8.50529531248867e-08
1740 8.50777865935015e-08
1741 8.5066410804302e-08
1742 8.50035348776146e-08
1743 8.49827301863115e-08
1744 8.49771453204085e-08
1745 8.49636805355658e-08
1746 8.494850334273e-08
1747 8.49326156071584e-08
1748 8.49216021947541e-08
1749 8.49061905228155e-08
1750 8.48969747835326e-08
1751 8.48972021572081e-08
1752 8.48327417202199e-08
1753 8.48107362116934e-08
1754 8.48083772098107e-08
1755 8.4797235899714e-08
1756 8.47925036850938e-08
1757 8.4775557240846e-08
1758 8.47630587941239e-08
1759 8.47626537847646e-08
1760 8.48583994184082e-08
1761 8.46906118567858e-08
1762 8.48044976464735e-08
1763 8.46094323492252e-08
1764 8.4587789217494e-08
1765 8.45871070964677e-08
1766 8.45906029667276e-08
1767 8.45855225861669e-08
1768 8.45751273459427e-08
1769 8.45836609641992e-08
1770 8.45526955117748e-08
1771 8.45251193482e-08
1772 8.44758645257571e-08
1773 8.44709262537435e-08
1774 8.4464218730318e-08
1775 8.44854284309804e-08
1776 8.44067713501317e-08
1777 8.43952605578124e-08
1778 8.44159444568504e-08
1779 8.43872314248983e-08
1780 8.4353565910078e-08
1781 8.43182021981193e-08
1782 8.43144931650386e-08
1783 8.43254355231693e-08
1784 8.42984206883557e-08
1785 8.42771825659838e-08
1786 8.4279868417525e-08
1787 8.42916989540754e-08
1788 8.42572873693825e-08
1789 8.42522140942492e-08
1790 8.41856362399085e-08
1791 8.41892102698694e-08
1792 8.41749709934447e-08
1793 8.41920595462398e-08
1794 8.41518144056863e-08
1795 8.41443394961061e-08
1796 8.41631120351849e-08
1797 8.41296952103221e-08
1798 8.41325444866925e-08
1799 8.40991987161033e-08
1800 8.41000158402494e-08
1801 8.40665563828225e-08
1802 8.40677145674817e-08
1803 8.40353848730047e-08
1804 8.40382483602298e-08
1805 8.40056841866499e-08
1806 8.40173370875164e-08
1807 8.39784206618788e-08
1808 8.39790317286315e-08
1809 8.39333509361495e-08
1810 8.39612397385281e-08
1811 8.39303169186678e-08
1812 8.39407334751741e-08
1813 8.38931200064508e-08
1814 8.39092066939884e-08
1815 8.384313332499e-08
1816 8.37901907857486e-08
1817 8.38044300621732e-08
1818 8.37717522017556e-08
1819 8.37802502928753e-08
1820 8.36403799553409e-08
1821 8.37099563000265e-08
1822 8.37125213593026e-08
1823 8.36750686517007e-08
1824 8.36936990822323e-08
1825 8.3687829999235e-08
1826 8.36159586015128e-08
1827 8.36355198430283e-08
1828 8.36027851391918e-08
1829 8.36243430057948e-08
1830 8.35870039850306e-08
1831 8.35969444779039e-08
1832 8.36050233488095e-08
1833 8.36010727311987e-08
1834 8.35682811839433e-08
1835 8.35779161434402e-08
1836 8.35288744838181e-08
1837 8.35302884638622e-08
1838 8.35364346585266e-08
1839 8.35229840845386e-08
1840 8.34776798797066e-08
1841 8.34756974654738e-08
1842 8.34885227618543e-08
1843 8.34032434227083e-08
1844 8.34175466479792e-08
1845 8.33745801287478e-08
1846 8.33888762485913e-08
1847 8.33879241213253e-08
1848 8.33979996173184e-08
1849 8.33234921060466e-08
1850 8.32942532724701e-08
1851 8.32982962606366e-08
1852 8.32392075267308e-08
1853 8.32711606335579e-08
1854 8.32064515066122e-08
1855 8.32454034593866e-08
1856 8.31845241577867e-08
1857 8.32215008017556e-08
1858 8.3165694775289e-08
1859 8.31587954053248e-08
1860 8.31724804584155e-08
1861 8.31393833777838e-08
1862 8.31434618930871e-08
1863 8.31418489610769e-08
1864 8.31388788924414e-08
1865 8.31331377071365e-08
1866 8.30389907946483e-08
1867 8.30739708135297e-08
1868 8.3069465972585e-08
1869 8.30673840823692e-08
1870 8.30353670266959e-08
1871 8.30499757853431e-08
1872 8.30385502581521e-08
1873 8.30271602580979e-08
1874 8.29949797775953e-08
1875 8.30060855605552e-08
1876 8.29833339821562e-08
1877 8.29959816428527e-08
1878 8.29761859222344e-08
1879 8.29539956725966e-08
1880 8.29295174753497e-08
1881 8.29170545557645e-08
1882 8.29220567766242e-08
1883 8.29028934390408e-08
1884 8.28810868824803e-08
1885 8.28767383609375e-08
1886 8.28593016422019e-08
1887 8.28377082484621e-08
1888 8.28360242621784e-08
1889 8.28383690532064e-08
1890 8.28326918167477e-08
1891 8.28204704816926e-08
1892 8.28075599201838e-08
1893 8.27957649107702e-08
1894 8.27861939001195e-08
1895 8.2782705135287e-08
1896 8.27790813673346e-08
1897 8.27695743055301e-08
1898 8.27615522780434e-08
1899 8.27304020845077e-08
1900 8.27117077051298e-08
1901 8.26977526457995e-08
1902 8.27356672061796e-08
1903 8.27312476303632e-08
1904 8.27279933446334e-08
1905 8.27167880856905e-08
1906 8.27057817787136e-08
1907 8.26929635877605e-08
1908 8.26811259457827e-08
1909 8.26692456712408e-08
1910 8.26329937808623e-08
1911 8.25941128823615e-08
1912 8.26222645855523e-08
1913 8.26124590957988e-08
1914 8.25936297133012e-08
1915 8.25899846290667e-08
1916 8.25496329071029e-08
1917 8.25684693950279e-08
1918 8.25262063131049e-08
1919 8.2549256319453e-08
1920 8.24944521582438e-08
1921 8.24856840608845e-08
1922 8.24954611289286e-08
1923 8.24883272798616e-08
1924 8.24930381781996e-08
1925 8.24938766186278e-08
1926 8.24995893822233e-08
1927 8.24947221644834e-08
1928 8.24840995505838e-08
1929 8.24826642542575e-08
1930 8.24734982529662e-08
1931 8.24930879161911e-08
1932 8.24864798687486e-08
1933 8.24487074169156e-08
1934 8.25227743916912e-08
1935 8.24218702177859e-08
1936 8.24219767991963e-08
1937 8.24923844788827e-08
1938 8.24098620455516e-08
1939 8.24815273858803e-08
1940 8.24686949840725e-08
1941 8.24497590201645e-08
1942 8.23587100740042e-08
1943 8.23485635237375e-08
1944 8.23500201363458e-08
1945 8.24151626943603e-08
1946 8.23239432179435e-08
1947 8.23161769858416e-08
1948 8.23834156449266e-08
1949 8.23778023573141e-08
1950 8.22719101734037e-08
1951 8.22665171540393e-08
1952 8.22550347834294e-08
1953 8.2332199724533e-08
1954 8.23074515210465e-08
1955 8.22903274411146e-08
1956 8.22174541781351e-08
1957 8.22910664055598e-08
1958 8.22687837853664e-08
1959 8.22660481958337e-08
1960 8.22523773535977e-08
1961 8.22534644839834e-08
1962 8.2254210553856e-08
1963 8.22470198613701e-08
1964 8.22382872911476e-08
1965 8.22276149392565e-08
1966 8.22128995991989e-08
1967 8.21968626496528e-08
1968 8.21793335603616e-08
1969 8.21612928803006e-08
1970 8.21525318883687e-08
1971 8.21352799107444e-08
1972 8.21191008526512e-08
1973 8.20025150005677e-08
1974 8.19926597728227e-08
1975 8.19762178139172e-08
1976 8.19317378386586e-08
1977 8.2149796298836e-08
1978 8.23624475287943e-08
1979 8.26238633067078e-08
1980 8.28488424531315e-08
1981 8.29868653795529e-08
1982 8.30752782121635e-08
1983 8.31683450996934e-08
1984 8.33017708146144e-08
1985 8.34308195862832e-08
1986 8.34761237911152e-08
1987 8.3546140672297e-08
1988 8.35710025626213e-08
1989 8.36324787201193e-08
1990 8.37463360880975e-08
1991 8.3782886406425e-08
1992 8.37638793882434e-08
1993 8.38406180037055e-08
1994 8.38045792761477e-08
1995 8.38826252902436e-08
1996 8.41925427153001e-08
1997 8.44078442696627e-08
1998 8.46237711016329e-08
1999 8.47748466981102e-08
2000 8.49194847774015e-08
2001 8.50354524573049e-08
2002 8.51526706924233e-08
2003 8.52259347539075e-08
2004 8.52934007866679e-08
2005 8.5377251934915e-08
2006 8.54416128959201e-08
2007 8.54645065828663e-08
2008 8.53712762705072e-08
2009 8.54769623970242e-08
2010 8.55394404197796e-08
2011 8.54267838690248e-08
2012 8.54219948109858e-08
2013 8.53963726399343e-08
2014 8.53972466074993e-08
2015 8.52994403999219e-08
2016 8.52714947541244e-08
2017 8.53001083100935e-08
2018 8.52637498383046e-08
2019 8.52019894637124e-08
2020 8.51520169931064e-08
2021 8.51131503054603e-08
2022 8.50540331498451e-08
2023 8.50051975476163e-08
2024 8.5005311234454e-08
2025 8.49630623633857e-08
2026 8.49142551828663e-08
2027 8.4922355370054e-08
2028 8.48566230615688e-08
2029 8.48535961495145e-08
2030 8.48477768045086e-08
2031 8.47612042775836e-08
2032 8.4736335281832e-08
2033 8.47756851385384e-08
2034 8.4668400290866e-08
2035 8.46134895482464e-08
2036 8.46064196480256e-08
2037 8.45387972958633e-08
2038 8.45319831910274e-08
2039 8.44683682998948e-08
2040 8.44636502961293e-08
2041 8.44045473513688e-08
2042 8.43654248683379e-08
2043 8.43633358726947e-08
2044 8.42982785798085e-08
2045 8.42730685235438e-08
2046 8.42349479057702e-08
2047 8.42001455225727e-08
2048 8.4162756763817e-08
2049 8.41284233388251e-08
2050 8.40923490841305e-08
2051 8.40564453596926e-08
2052 8.40195042428604e-08
2053 8.39784419781608e-08
2054 8.39524716411688e-08
2055 8.39036076172306e-08
2056 8.3855113075515e-08
2057 8.38008276105029e-08
2058 8.37564400058e-08
2059 8.37104465745142e-08
2060 8.3668318495711e-08
2061 8.36080715771459e-08
2062 8.35456148706726e-08
2063 8.35787972164326e-08
2064 8.35705478152704e-08
2065 8.35653253261626e-08
2066 8.3561161545731e-08
2067 8.36788558444823e-08
2068 8.36889455513301e-08
2069 8.37958893384894e-08
2070 8.38482208109781e-08
2071 8.38743403619446e-08
2072 8.3887442769992e-08
2073 8.38854816720414e-08
2074 8.38883096321297e-08
2075 8.38764151467331e-08
2076 8.3859504229622e-08
2077 8.3839317710499e-08
2078 8.38157347970991e-08
2079 8.37783389329161e-08
2080 8.37693221455993e-08
2081 8.37455260693787e-08
2082 8.37150224697325e-08
2083 8.36698106354561e-08
2084 8.36208258192528e-08
2085 8.35682314459518e-08
2086 8.35428082268663e-08
2087 8.34726066045732e-08
2088 8.34413427241998e-08
2089 8.33895086316261e-08
2090 8.33256237342539e-08
2091 8.3293556940589e-08
2092 8.32292670338575e-08
2093 8.31860873518053e-08
2094 8.32968822805924e-08
2095 8.32047959420379e-08
2096 8.29296737947516e-08
2097 8.30776301086189e-08
2098 8.27796995395147e-08
2099 8.27869115482827e-08
2100 8.28247763706713e-08
2101 8.27877997267024e-08
2102 8.27190334007355e-08
2103 8.27005663950331e-08
2104 8.26976105372523e-08
2105 8.26536563636182e-08
2106 8.25835755335902e-08
2107 8.25586923269839e-08
2108 8.24362871298945e-08
2109 8.23337131805602e-08
2110 8.24795591825023e-08
2111 8.2269650647504e-08
2112 8.21221917135517e-08
2113 8.22033427994029e-08
2114 8.20680270408047e-08
2115 8.19547807395793e-08
2116 8.18934040580643e-08
2117 8.19501622117969e-08
2118 8.19198291424073e-08
2119 8.18490946130623e-08
2120 8.16997811625697e-08
2121 8.16941678749572e-08
2122 8.16523595403851e-08
2123 8.16023444372149e-08
2124 8.15541199017389e-08
2125 8.15167027212738e-08
2126 8.1447979027871e-08
2127 8.14092473433448e-08
2128 8.13038099067853e-08
2129 8.12284142170938e-08
2130 8.11523435118033e-08
2131 8.10570739417926e-08
2132 8.1041363841905e-08
2133 8.09709703730732e-08
2134 8.09209126373389e-08
2135 8.08740878710523e-08
2136 8.07658793178234e-08
2137 8.07477320563521e-08
2138 8.06748730042273e-08
2139 8.059389955406e-08
2140 8.05143258730823e-08
2141 8.04305742008182e-08
2142 8.029601161752e-08
2143 8.01754111989794e-08
2144 8.00920858523568e-08
2145 8.00157380353994e-08
2146 7.98996140360941e-08
2147 8.01292756591465e-08
2148 7.96456589569061e-08
2149 7.985154582002e-08
2150 7.94310608398519e-08
2151 7.97691157572444e-08
2152 7.93523824427211e-08
2153 7.92740095789668e-08
2154 7.91685934586894e-08
2155 7.90769618674858e-08
2156 7.89912419918437e-08
2157 7.89068934636816e-08
2158 7.87631790899468e-08
2159 7.86880534064949e-08
2160 7.85784308732218e-08
2161 7.85130751523866e-08
2162 7.84132225817302e-08
2163 7.83106841595327e-08
2164 7.81180702347228e-08
2165 7.81967344209988e-08
2166 7.81442253128262e-08
2167 7.79580275889202e-08
2168 7.78677033963504e-08
2169 7.7760041961028e-08
2170 7.76811219793672e-08
2171 7.75653674622845e-08
2172 7.79910962478425e-08
2173 7.7510556195648e-08
2174 7.72811716842625e-08
2175 7.76230848487103e-08
2176 7.71603225757644e-08
2177 7.69486803164909e-08
2178 7.67904353438098e-08
2179 7.66522134654224e-08
2180 7.6574792728934e-08
2181 7.63833796213476e-08
2182 7.62856018354796e-08
2183 7.6213680699766e-08
2184 7.65039516181787e-08
2185 7.60609211170049e-08
2186 7.5906633867362e-08
2187 7.57891811531408e-08
2188 7.61459020282018e-08
2189 7.56869411588923e-08
2190 7.55133200414093e-08
2191 7.55332294488653e-08
2192 7.54456905838197e-08
2193 7.52781659230095e-08
2194 7.52449764718222e-08
2195 7.5100203389411e-08
2196 7.50688400330546e-08
2197 7.49172173186707e-08
2198 7.48261967942199e-08
2199 7.46859143418988e-08
2200 7.47462607364469e-08
2201 7.5707539792802e-08
2202 7.51388213870996e-08
2203 7.46386774608254e-08
2204 7.43908472600197e-08
2205 7.49143396205909e-08
2206 7.43188977025966e-08
2207 7.41671186688109e-08
2208 7.48629673807955e-08
2209 7.40278522926019e-08
2210 7.4681288708689e-08
2211 7.39165386676177e-08
2212 7.37660883487479e-08
2213 7.36207610430029e-08
2214 7.42601784509134e-08
2215 7.34939007429602e-08
2216 7.34272092017818e-08
2217 7.32971088268641e-08
2218 7.39118988235532e-08
2219 7.32448057760848e-08
2220 7.31663689634843e-08
2221 7.30488096678528e-08
2222 7.29625426743041e-08
2223 7.28820950257614e-08
2224 7.28641751379655e-08
2225 7.27184499282885e-08
2226 7.27098310449037e-08
2227 7.25062534456811e-08
2228 7.24632727155949e-08
2229 7.23849780115415e-08
2230 7.27592635030305e-08
2231 7.25656121858265e-08
2232 7.23643367450677e-08
2233 7.23339184105498e-08
2234 7.22901631888817e-08
2235 7.21145170246018e-08
2236 7.33221767745817e-08
2237 7.24769506632583e-08
2238 7.2021649089038e-08
2239 7.25376594346017e-08
2240 7.18555455136993e-08
2241 7.23647985978459e-08
2242 7.17541226435969e-08
2243 7.22697777177927e-08
2244 7.16352062113401e-08
2245 7.20528774422746e-08
2246 7.15009917939824e-08
2247 7.19400787829727e-08
2248 7.13555721176817e-08
2249 7.18576202984877e-08
2250 7.13100476446016e-08
2251 7.12415300085922e-08
2252 7.16480670348574e-08
2253 7.11333001390813e-08
2254 7.18726766990585e-08
2255 7.14850330041372e-08
2256 7.08583272057695e-08
2257 7.17531349891942e-08
2258 7.07888290207848e-08
2259 7.16982029302926e-08
2260 7.07267489019614e-08
2261 7.05842992942962e-08
2262 7.153404624205e-08
2263 7.05222333863276e-08
2264 7.17062960120529e-08
2265 7.09573768631344e-08
2266 7.02792277706976e-08
2267 7.08698237872341e-08
2268 7.01050240081713e-08
2269 7.12631305077593e-08
2270 7.02674185504293e-08
2271 7.03846012584108e-08
2272 6.98128701515088e-08
2273 7.02654077144871e-08
2274 6.96535593647241e-08
2275 7.01214446507947e-08
2276 6.95447042176056e-08
2277 7.00093210070918e-08
2278 6.9447509076781e-08
2279 7.01376166034606e-08
2280 6.9399582969254e-08
2281 7.01004339020983e-08
2282 6.97018123219095e-08
2283 7.05191567362817e-08
2284 6.99510707136142e-08
2285 6.92793165057992e-08
2286 7.01585847195929e-08
2287 6.93885979785591e-08
2288 6.98460453918415e-08
2289 6.92874877472605e-08
2290 6.87080756733849e-08
2291 6.97254094461641e-08
2292 6.88261749814956e-08
2293 6.95880260082049e-08
2294 6.87026826540205e-08
2295 6.94787516408724e-08
2296 6.86122163529035e-08
2297 6.93927404427086e-08
2298 6.85189220916982e-08
2299 6.93633452897302e-08
2300 6.84902090597461e-08
2301 6.93514152771968e-08
2302 6.84581493715086e-08
2303 6.93780748406425e-08
2304 6.84157370756111e-08
2305 6.92803752144755e-08
2306 6.83322483041593e-08
2307 6.86946393102517e-08
2308 6.82713832134141e-08
2309 6.86319410192482e-08
2310 6.92866279905502e-08
2311 6.88010359795044e-08
2312 6.84481449297891e-08
2313 6.83593484041012e-08
2314 6.91912376282744e-08
2315 6.94993289585e-08
2316 6.96563446922482e-08
2317 6.95616364509988e-08
2318 6.84863792344004e-08
2319 6.81470098129466e-08
2320 6.89340211579292e-08
2321 6.83114365074289e-08
2322 6.94054946848155e-08
2323 6.83369734133521e-08
2324 6.89862602598623e-08
2325 6.81597853713356e-08
2326 6.91950106102013e-08
2327 6.81370337929366e-08
2328 6.88043684249351e-08
2329 6.80407410413864e-08
2330 6.87029597656874e-08
2331 6.79069174225333e-08
2332 6.81854359640965e-08
2333 6.86552539264085e-08
2334 6.86507917180279e-08
2335 6.86722074760837e-08
2336 6.86577408259836e-08
2337 6.86417038764375e-08
2338 6.84870400391446e-08
2339 6.77188225495229e-08
2340 6.75566056429489e-08
2341 6.73995970146279e-08
2342 6.81396912227683e-08
2343 6.83768632825377e-08
2344 6.82903831261683e-08
2345 6.83080045860152e-08
2346 6.83525840372567e-08
2347 6.85410554979171e-08
2348 6.85384549115042e-08
2349 6.83234375742359e-08
2350 6.82929695017265e-08
2351 6.8514566464728e-08
2352 6.82788936501311e-08
2353 6.84970515862915e-08
2354 6.84388297145233e-08
2355 6.84622349922392e-08
2356 6.84247396520732e-08
2357 6.83715697391563e-08
2358 6.83959555658475e-08
2359 6.83601157902558e-08
2360 6.83596681483323e-08
2361 6.81227021459563e-08
2362 6.75271181194148e-08
2363 6.70227322530081e-08
2364 6.67098376538888e-08
2365 6.68510935497579e-08
2366 6.65653345777173e-08
2367 6.68074093823634e-08
2368 6.65414034983769e-08
2369 6.67126016651309e-08
2370 6.73614835022818e-08
2371 6.75610607459021e-08
2372 6.77249190061957e-08
2373 6.78235423379192e-08
2374 6.77729090625689e-08
2375 6.77775346957787e-08
2376 6.77628193557211e-08
2377 6.77869138598908e-08
2378 6.77686671224365e-08
2379 6.77219773592697e-08
2380 6.69625919158534e-08
2381 6.67237713969371e-08
2382 6.74275000278612e-08
2383 6.67865833747783e-08
2384 6.73626274760863e-08
2385 6.67146480282099e-08
2386 6.66888482214745e-08
2387 6.64416859308403e-08
2388 6.68423538741081e-08
2389 6.73510101023567e-08
2390 6.76325839776837e-08
2391 6.77948577276766e-08
2392 6.78014018262729e-08
2393 6.78597515957335e-08
2394 6.78714187074547e-08
2395 6.76795863796542e-08
2396 6.76503901786418e-08
2397 6.76577229796749e-08
2398 6.76584264169833e-08
2399 6.78787941410519e-08
2400 6.80108129813561e-08
2401 6.78500313711083e-08
2402 6.78163161182965e-08
2403 6.80040130873749e-08
2404 6.76909621688537e-08
2405 6.78987674973541e-08
2406 6.78856650893067e-08
2407 6.7831962269338e-08
2408 6.77595082265725e-08
2409 6.77574902852029e-08
2410 6.77024019068995e-08
2411 6.76999007964696e-08
2412 6.76983376024509e-08
2413 6.76429365853437e-08
2414 6.76198155247221e-08
2415 6.7656294788776e-08
2416 6.77886404787387e-08
2417 6.73222331215584e-08
2418 6.76179254810449e-08
2419 6.75641302905206e-08
2420 6.75890703405457e-08
2421 6.75935822869178e-08
2422 6.76321363357602e-08
2423 6.73310225351997e-08
2424 6.75811691053241e-08
2425 6.77301272844488e-08
2426 6.75566624863677e-08
2427 6.77091449574618e-08
2428 6.73581510568511e-08
2429 6.75303368780078e-08
2430 6.76291662671247e-08
2431 6.75048497100761e-08
2432 6.73886830782067e-08
2433 6.75415492423781e-08
2434 6.72203626095325e-08
2435 6.73258071515193e-08
2436 6.74843434467221e-08
2437 6.71107898142509e-08
2438 6.72753515118529e-08
2439 6.74076616746788e-08
2440 6.70646471689906e-08
2441 6.72393767331414e-08
2442 6.73613982371535e-08
2443 6.70218511800158e-08
2444 6.71811619668006e-08
2445 6.73107436455211e-08
2446 6.69731505809068e-08
2447 6.73071554047056e-08
2448 6.69371686967679e-08
2449 6.72802116241655e-08
2450 6.68960069560853e-08
2451 6.72493953857156e-08
2452 6.68551507487791e-08
2453 6.72160993531179e-08
2454 6.70554030079984e-08
2455 6.71967441689958e-08
2456 6.7028651073997e-08
2457 6.72143514179879e-08
2458 6.70086137688486e-08
2459 6.70752626774629e-08
2460 6.70073561082063e-08
2461 6.70288358151083e-08
2462 6.70937296831653e-08
2463 6.69427464572436e-08
2464 6.68751098942266e-08
2465 6.70804780611434e-08
2466 6.68841124706887e-08
2467 6.7050407892566e-08
2468 6.68734401187976e-08
2469 6.70126070190236e-08
2470 6.68727508923439e-08
2471 6.69997177737969e-08
2472 6.67939517029481e-08
2473 6.69989788093517e-08
2474 6.6789517916277e-08
2475 6.69950281917409e-08
2476 6.67873507609329e-08
2477 6.69671109676528e-08
2478 6.6768336637324e-08
2479 6.69686883725262e-08
2480 6.67669013409977e-08
2481 6.69603466008084e-08
2482 6.67459332248654e-08
2483 6.69551738496921e-08
2484 6.67223432060382e-08
2485 6.69343620529617e-08
2486 6.68575808049354e-08
2487 6.6765515782663e-08
2488 6.68036364004365e-08
2489 6.69240023398743e-08
2490 6.67214408167638e-08
2491 6.68798350034194e-08
2492 6.66845636487778e-08
2493 6.68478108423187e-08
2494 6.66499175849822e-08
2495 6.68356321398278e-08
2496 6.66247359504268e-08
2497 6.68015331939387e-08
2498 6.66018138417712e-08
2499 6.67784973984453e-08
2500 6.6710974522266e-08
2501 6.6761124628556e-08
2502 6.67327810788265e-08
2503 6.67114363750443e-08
2504 6.67306991886107e-08
2505 6.67221087269354e-08
2506 6.67060575665346e-08
2507 6.66970620954999e-08
2508 6.6679938015568e-08
2509 6.66626291945249e-08
2510 6.66509123448122e-08
2511 6.66452919517724e-08
2512 6.66285799866273e-08
2513 6.66222419454243e-08
2514 6.66141986016555e-08
2515 6.65947581524051e-08
2516 6.65933583832157e-08
2517 6.65841497493602e-08
2518 6.65751400674708e-08
2519 6.656357953716e-08
2520 6.65491910467608e-08
2521 6.65397976717941e-08
2522 6.65274271227645e-08
2523 6.65129178401003e-08
2524 6.65003696553867e-08
2525 6.64960921881175e-08
2526 6.64836719010964e-08
2527 6.64755717139087e-08
2528 6.64635138036829e-08
2529 6.64508945646958e-08
2530 6.64296706531786e-08
2531 6.64292372221098e-08
2532 6.64195951571855e-08
2533 6.64002683947729e-08
2534 6.63980586068647e-08
2535 6.6377758400904e-08
2536 6.63634835973426e-08
2537 6.63587158555856e-08
2538 6.63479440277115e-08
2539 6.63094681385701e-08
2540 6.6311194757418e-08
2541 6.6280833266319e-08
2542 6.62895018876952e-08
2543 6.62562129605249e-08
2544 6.62457395605998e-08
2545 6.62353443203756e-08
2546 6.62286296915227e-08
2547 6.62158328168516e-08
2548 6.62102905835127e-08
2549 6.61915819932801e-08
2550 6.61951418123863e-08
2551 6.61758789988198e-08
2552 6.61762626918971e-08
2553 6.61677006519312e-08
2554 6.61462635775933e-08
2555 6.61156747128189e-08
2556 6.61102248500356e-08
2557 6.60960992604487e-08
2558 6.60817036646222e-08
2559 6.60936549934377e-08
2560 6.60969519117316e-08
2561 6.60778596284217e-08
2562 6.60804957419714e-08
2563 6.60558256981858e-08
2564 6.60536585428417e-08
2565 6.60375647498768e-08
2566 6.60331309632056e-08
2567 6.60141736830155e-08
2568 6.60164616306247e-08
2569 6.59980798900506e-08
2570 6.59956143067575e-08
2571 6.59863701457652e-08
2572 6.59801173696906e-08
2573 6.59609185049703e-08
2574 6.5958886352746e-08
2575 6.59445120732016e-08
2576 6.58684697896206e-08
2577 6.59106405009879e-08
2578 6.5910086277654e-08
2579 6.58845706880129e-08
2580 6.58337526715513e-08
2581 6.58704806255628e-08
2582 6.57784440249998e-08
2583 6.58278977994087e-08
2584 6.5756154299379e-08
2585 6.57229151102001e-08
2586 6.57131948855749e-08
2587 6.56968452972251e-08
2588 6.56881624649941e-08
2589 6.566628485416e-08
2590 6.56708252222415e-08
2591 6.56335927828877e-08
2592 6.56345875427178e-08
2593 6.56934489029481e-08
2594 6.56388650099871e-08
2595 6.56236309737324e-08
2596 6.57151559835256e-08
2597 6.56156942113739e-08
2598 6.56338841054094e-08
2599 6.55819718531347e-08
2600 6.55945768812671e-08
2601 6.5570432639106e-08
2602 6.55637037993984e-08
2603 6.55695160389769e-08
2604 6.55529888149431e-08
2605 6.55601084531554e-08
2606 6.55322125453495e-08
2607 6.55412577543757e-08
2608 6.55107967872937e-08
2609 6.5491569500864e-08
2610 6.55211067623895e-08
2611 6.54868301808165e-08
2612 6.55036842545087e-08
2613 6.54685550216527e-08
2614 6.54882299500059e-08
2615 6.54560494695033e-08
2616 6.5490993961248e-08
2617 6.54464997751347e-08
2618 6.54402967370515e-08
2619 6.54309459946489e-08
2620 6.53990070986765e-08
2621 6.53825225072069e-08
2622 6.54333405236684e-08
2623 6.53929319582858e-08
2624 6.53627694191528e-08
2625 6.53666987204815e-08
2626 6.53374172543408e-08
2627 6.53345750833978e-08
2628 6.53376019954521e-08
2629 6.53698037922368e-08
2630 6.53817195939155e-08
2631 6.53575042974808e-08
2632 6.53590461752174e-08
2633 6.53523173355097e-08
2634 6.53474927503339e-08
2635 6.53393854577189e-08
2636 6.53342482337393e-08
2637 6.53580798370967e-08
2638 6.52758060937231e-08
2639 6.52742002671403e-08
2640 6.52734044592762e-08
2641 6.52296634484628e-08
2642 6.52693259439729e-08
2643 6.52588241223384e-08
2644 6.52842260251418e-08
2645 6.52639400300359e-08
2646 6.52781011467596e-08
2647 6.52668745715346e-08
2648 6.52872316209141e-08
2649 6.52059597427979e-08
2650 6.51954579211633e-08
2651 6.51962253073179e-08
2652 6.51920188943222e-08
2653 6.51672849016904e-08
2654 6.51716618449427e-08
2655 6.52123688382744e-08
2656 6.51787033234541e-08
2657 6.51945626373163e-08
2658 6.51460680956006e-08
2659 6.51534293183431e-08
2660 6.51869527246163e-08
2661 6.51554898922768e-08
2662 6.51458620382073e-08
2663 6.5096998014269e-08
2664 6.51349196800766e-08
2665 6.51427640718794e-08
2666 6.50721645456542e-08
2667 6.50779341526686e-08
2668 6.51009628427346e-08
2669 6.50860201290016e-08
2670 6.51092051384694e-08
2671 6.50330349571959e-08
2672 6.50377955935255e-08
2673 6.50500595611447e-08
2674 6.5041639629726e-08
2675 6.50461515760981e-08
2676 6.50229594612028e-08
2677 6.50123084255938e-08
2678 6.50225828735529e-08
2679 6.49805471653053e-08
2680 6.49970814947665e-08
2681 6.49572058364356e-08
2682 6.49850377953953e-08
2683 6.49676792363607e-08
2684 6.49791473961159e-08
2685 6.4954591039168e-08
2686 6.49869349444998e-08
2687 6.49880291803129e-08
2688 6.49732356805544e-08
2689 6.49674518626853e-08
2690 6.49538378638681e-08
2691 6.49459437340738e-08
2692 6.49374456429541e-08
2693 6.48907558797873e-08
2694 6.49532765351069e-08
2695 6.49023021992434e-08
2696 6.48342322051576e-08
2697 6.4905016472494e-08
2698 6.48626823362974e-08
2699 6.47915783247299e-08
2700 6.48400941827276e-08
2701 6.48010356485429e-08
2702 6.4744845928999e-08
2703 6.45994973069719e-08
2704 6.45651212494158e-08
2705 6.48049933715811e-08
2706 6.47023128408364e-08
2707 6.46937863280073e-08
2708 6.47504805328936e-08
2709 6.46742961407654e-08
2710 6.45849169700341e-08
2711 6.46024886918894e-08
2712 6.44945927774643e-08
2713 6.46228528466963e-08
2714 6.5230004508976e-08
2715 6.48491251808991e-08
2716 6.48267857172868e-08
2717 6.48971152372724e-08
2718 6.50578826366655e-08
2719 6.49414886311206e-08
2720 6.49174722866519e-08
2721 6.4826352286218e-08
2722 6.4824185130874e-08
2723 6.46679580995624e-08
2724 6.46918323354839e-08
2725 6.44950546302425e-08
2726 6.45124913489781e-08
2727 6.45587192593666e-08
2728 6.46660112124664e-08
2729 6.46445812435559e-08
2730 6.4623641549133e-08
2731 6.46117044311723e-08
2732 6.45917737074342e-08
2733 6.45245066266398e-08
2734 6.45078159777768e-08
2735 6.45549178557303e-08
2736 6.45124558218413e-08
2737 6.44826201323667e-08
2738 6.45123279241488e-08
2739 6.4463570481621e-08
2740 6.44940882921219e-08
2741 6.45966480306015e-08
2742 6.46301501205926e-08
2743 6.47065476755415e-08
2744 6.47204316805983e-08
2745 6.47450661972471e-08
2746 6.4789752229899e-08
2747 6.4772287089454e-08
2748 6.47271889420153e-08
2749 6.48042046691444e-08
2750 6.48545892545371e-08
2751 6.49000568841984e-08
2752 6.48652047630094e-08
2753 6.47825970645499e-08
2754 6.47593054736717e-08
2755 6.47261231279117e-08
2756 6.47264570829975e-08
2757 6.47219522420528e-08
2758 6.46751843191851e-08
2759 6.46778275381621e-08
2760 6.46528661718548e-08
2761 6.46685407446057e-08
2762 6.45698818857454e-08
2763 6.45457589598664e-08
2764 6.46983195906614e-08
2765 6.45260058718122e-08
2766 6.45039150981574e-08
2767 6.45265529897188e-08
2768 6.45977920044061e-08
2769 6.44640820723907e-08
2770 6.45201581050969e-08
2771 6.44722746301341e-08
2772 6.43786890464071e-08
2773 6.4401397992242e-08
2774 6.46438067519739e-08
2775 6.44330029331286e-08
2776 6.44653255221783e-08
2777 6.45683542188635e-08
2778 6.45197104631734e-08
2779 6.44785984604823e-08
2780 6.45425757284102e-08
2781 6.45855919856331e-08
2782 6.45585700453921e-08
2783 6.4499594998324e-08
2784 6.45456381676013e-08
2785 6.45111626340622e-08
2786 6.44839204255732e-08
2787 6.44863717980115e-08
2788 6.45019042622152e-08
2789 6.44686934947458e-08
2790 6.43969499947161e-08
2791 6.44112105874228e-08
2792 6.44066417976319e-08
2793 6.44098960833617e-08
2794 6.44064570565206e-08
2795 6.43951452161673e-08
2796 6.43061781602228e-08
2797 6.4373580244137e-08
2798 6.44163762331118e-08
2799 6.42955058083317e-08
2800 6.42965645170079e-08
2801 6.4334855665038e-08
2802 6.43357580543125e-08
2803 6.42840944919953e-08
2804 6.42818847040871e-08
2805 6.43596109739519e-08
2806 6.42739550471561e-08
2807 6.43282547230228e-08
2808 6.41493613784405e-08
2809 6.41356621144951e-08
2810 6.415388043024e-08
2811 6.40390425132864e-08
2812 6.40783639482834e-08
2813 6.39931272417016e-08
2814 6.39061923379813e-08
2815 6.40265795937012e-08
2816 6.39189607909429e-08
2817 6.39904342847331e-08
2818 6.39144630554256e-08
2819 6.39144204228614e-08
2820 6.39379535982698e-08
2821 6.39817940850662e-08
2822 6.38747224002145e-08
2823 6.38873132174922e-08
2824 6.39393036294678e-08
2825 6.39694377468913e-08
2826 6.3969231689498e-08
2827 6.38964081645099e-08
2828 6.39427639725909e-08
2829 6.38956336729279e-08
2830 6.38130472907505e-08
2831 6.38384349826993e-08
2832 6.40007016272648e-08
2833 6.39220587572709e-08
2834 6.37808312831112e-08
2835 6.39356301235239e-08
2836 6.38034904909546e-08
2837 6.39255404166761e-08
2838 6.39093329368734e-08
2839 6.40059454326547e-08
2840 6.37881285570074e-08
2841 6.40174988575382e-08
2842 6.38804706909468e-08
2843 6.38386694618021e-08
2844 6.38986819012644e-08
2845 6.38521839846362e-08
2846 6.39836414961792e-08
2847 6.38529726870729e-08
2848 6.38513384387807e-08
2849 6.38582449141722e-08
2850 6.38318482515388e-08
2851 6.37328838593021e-08
2852 6.37692281202362e-08
2853 6.37875317011094e-08
2854 6.38155555066078e-08
2855 6.38657411400345e-08
2856 6.37379073964439e-08
2857 6.38310595491021e-08
2858 6.36831174460895e-08
2859 6.37688515325863e-08
2860 6.3719689080699e-08
2861 6.37824584259761e-08
2862 6.38418029552668e-08
2863 6.36884252003256e-08
2864 6.37889883137177e-08
2865 6.36897183881047e-08
2866 6.36611758864092e-08
2867 6.37015986626466e-08
2868 6.38251833606773e-08
2869 6.36898960237886e-08
2870 6.36815329357887e-08
2871 6.3722985998993e-08
2872 6.36992467661912e-08
2873 6.38326298485481e-08
2874 6.37494395050453e-08
2875 6.37075530107722e-08
2876 6.37720560803245e-08
2877 6.36381116692064e-08
2878 6.35120969150194e-08
2879 6.35809414006872e-08
2880 6.36407406773287e-08
2881 6.35971275642078e-08
2882 6.36115231600343e-08
2883 6.35802521742335e-08
2884 6.36768362483053e-08
2885 6.36601100723055e-08
2886 6.36302388556942e-08
2887 6.36677128795782e-08
2888 6.36107202467429e-08
2889 6.35673913507162e-08
2890 6.36050785374209e-08
2891 6.35341024235458e-08
2892 6.35943848692477e-08
2893 6.36427444078436e-08
2894 6.35427568340674e-08
2895 6.35040322549685e-08
2896 6.35636254742167e-08
2897 6.3517155979298e-08
2898 6.35673487181521e-08
2899 6.35870023302232e-08
2900 6.34475156857661e-08
2901 6.35100079193762e-08
2902 6.36615382632044e-08
2903 6.35558521366875e-08
2904 6.34954844258573e-08
2905 6.36053201219511e-08
2906 6.34682848499324e-08
2907 6.35595114317766e-08
2908 6.3548093010013e-08
2909 6.34990371395361e-08
2910 6.34359267337459e-08
2911 6.33975645314422e-08
2912 6.35106758295478e-08
2913 6.3480378287295e-08
2914 6.35402201965007e-08
2915 6.33959587048594e-08
2916 6.34096863905143e-08
2917 6.34930827914104e-08
2918 6.34332906201962e-08
2919 6.35152517247661e-08
2920 6.34633039453547e-08
2921 6.33895282931007e-08
2922 6.33610355293968e-08
2923 6.34960599654733e-08
2924 6.33848884490362e-08
2925 6.32800691846569e-08
2926 6.33932160098993e-08
2927 6.35484980193723e-08
2928 6.34256522857868e-08
2929 6.33515924164385e-08
2930 6.34630694662519e-08
2931 6.33821741757856e-08
2932 6.33387244874939e-08
2933 6.33782022418927e-08
2934 6.32706402825534e-08
2935 6.32642667142136e-08
2936 6.34322177006652e-08
2937 6.3556896634509e-08
2938 6.35942427607006e-08
2939 6.36330668157825e-08
2940 6.36217052374377e-08
2941 6.3600211319681e-08
2942 6.37826857996515e-08
2943 6.38178079270801e-08
2944 6.35789589864544e-08
2945 6.35724646258495e-08
2946 6.36043182566937e-08
2947 6.37298995798119e-08
2948 6.3773455849514e-08
2949 6.36153885125168e-08
2950 6.3609292055844e-08
2951 6.36448760360508e-08
2952 6.3642197289937e-08
2953 6.36104076079391e-08
2954 6.3594171706427e-08
2955 6.36037924550692e-08
2956 6.35836556739378e-08
2957 6.36263948194937e-08
2958 6.36259258612881e-08
2959 6.35246166780234e-08
2960 6.36510506524246e-08
2961 6.36778523244175e-08
2962 6.34827657108872e-08
2963 6.35831014506039e-08
2964 6.36494803529786e-08
2965 6.36254284813731e-08
2966 6.34667003396316e-08
2967 6.3551532036854e-08
2968 6.36391845887374e-08
2969 6.36093986372543e-08
2970 6.34363175322505e-08
2971 6.36159143141413e-08
2972 6.35952872585221e-08
2973 6.35480361665941e-08
2974 6.34287857792515e-08
2975 6.34726120551932e-08
2976 6.35558876638243e-08
2977 6.35579411323306e-08
2978 6.35774171087178e-08
2979 6.33951486861406e-08
2980 6.36642951690192e-08
2981 6.3481529366527e-08
2982 6.35616714816933e-08
2983 6.35724646258495e-08
2984 6.34809751431931e-08
2985 6.34992289860747e-08
2986 6.34966923485081e-08
2987 6.33210461842282e-08
2988 6.35316084185433e-08
2989 6.35270396287524e-08
2990 6.34510612940176e-08
2991 6.35081391919812e-08
2992 6.34096508633775e-08
2993 6.35230392731501e-08
2994 6.35169925544687e-08
2995 6.35785113445309e-08
2996 6.33928038951126e-08
2997 6.34236627661267e-08
2998 6.34513028785477e-08
2999 6.33847605513438e-08
3000 6.33659453797009e-08
3001 6.34364880625071e-08
3002 6.33908925351534e-08
3003 6.32753156537547e-08
3004 6.35204173704551e-08
3005 6.34468406701671e-08
3006 6.34241175134775e-08
3007 6.34447943070882e-08
3008 6.34426342571714e-08
3009 6.3486041312899e-08
3010 6.34990229286814e-08
3011 6.34540668897898e-08
3012 6.34294181622863e-08
3013 6.34802859167394e-08
3014 6.34357562034893e-08
3015 6.34355856732327e-08
3016 6.34529015997032e-08
3017 6.34590335835128e-08
3018 6.343771730144e-08
3019 6.3449718368247e-08
3020 6.34334469395981e-08
3021 6.34627639328755e-08
3022 6.34241104080502e-08
3023 6.33752890166761e-08
3024 6.33885974821169e-08
3025 6.33575893971283e-08
3026 6.33664214433338e-08
3027 6.32986427717697e-08
3028 6.33245775816249e-08
3029 6.33429735330537e-08
3030 6.33966621421678e-08
3031 6.34302850244239e-08
3032 6.34465848747823e-08
3033 6.33741237265895e-08
3034 6.34042933711498e-08
3035 6.34036751989697e-08
3036 6.33679562156431e-08
3037 6.3373001069067e-08
3038 6.33530490290468e-08
3039 6.33902601521186e-08
3040 6.33596357602073e-08
3041 6.33954400086623e-08
3042 6.33497592161802e-08
3043 6.33744789979573e-08
3044 6.33552232898182e-08
3045 6.3348622347803e-08
3046 6.33373034020224e-08
3047 6.33914964964788e-08
3048 6.33680770079081e-08
3049 6.32799341815371e-08
3050 6.32386587540168e-08
3051 6.33802557103991e-08
3052 6.32563725844193e-08
3053 6.32848227155591e-08
3054 6.32334504757637e-08
3055 6.32887378060332e-08
3056 6.32245829024214e-08
3057 6.33423908880104e-08
3058 6.31967083108975e-08
3059 6.32231333952404e-08
3060 6.31965733077777e-08
3061 6.31967651543164e-08
3062 6.3169316888434e-08
3063 6.3209533607278e-08
3064 6.31474037504631e-08
3065 6.31726138067279e-08
3066 6.31151593211143e-08
3067 6.31849701449028e-08
3068 6.30827230452269e-08
3069 6.30957543990007e-08
3070 6.30645331511914e-08
3071 6.30946601631877e-08
3072 6.30334326956472e-08
3073 6.30592325023827e-08
3074 6.3001557748521e-08
3075 6.30501872933564e-08
3076 6.29900966941932e-08
3077 6.30133385470799e-08
3078 6.29726244483209e-08
3079 6.30289420655572e-08
3080 6.29690575237873e-08
3081 6.30448866445477e-08
3082 6.2928833699516e-08
3083 6.29456593514988e-08
3084 6.2925337829256e-08
3085 6.29349585778982e-08
3086 6.28798915158768e-08
3087 6.29235756832713e-08
3088 6.28933491952921e-08
3089 6.29096135185137e-08
3090 6.29813996511075e-08
3091 6.27917344786511e-08
3092 6.2905627373766e-08
3093 6.27503595751477e-08
3094 6.27495566618563e-08
3095 6.28470004926385e-08
3096 6.27252561002933e-08
3097 6.2760378227722e-08
3098 6.27381950835115e-08
3099 6.27593692570372e-08
3100 6.27137026754099e-08
3101 6.27322052082491e-08
3102 6.27137737296835e-08
3103 6.27180085643886e-08
3104 6.26862330932454e-08
3105 6.26939424819284e-08
3106 6.26524396807326e-08
3107 6.27382803486398e-08
3108 6.26297094186157e-08
3109 6.26531999614599e-08
3110 6.26064320385922e-08
3111 6.26310452389589e-08
3112 6.26115976842812e-08
3113 6.26235987510881e-08
3114 6.25775911089477e-08
3115 6.26516012403044e-08
3116 6.25631813022665e-08
3117 6.25862313086145e-08
3118 6.25405220944231e-08
3119 6.25757508032621e-08
3120 6.25225808903451e-08
3121 6.25960439037954e-08
3122 6.24870395427024e-08
3123 6.25356975092473e-08
3124 6.24685299044359e-08
3125 6.25053502290029e-08
3126 6.24687856998207e-08
3127 6.24844531671442e-08
3128 6.24204830046438e-08
3129 6.24708320628997e-08
3130 6.24495939405278e-08
3131 6.24352267664108e-08
3132 6.23932550070094e-08
3133 6.2386291688199e-08
3134 6.23628508833463e-08
3135 6.24032239215921e-08
3136 6.23650606712545e-08
3137 6.23768912078049e-08
3138 6.23127647259025e-08
3139 6.23653306774941e-08
3140 6.23085796291889e-08
3141 6.23396942955878e-08
3142 6.22875120370736e-08
3143 6.22990228293929e-08
3144 6.22849825049343e-08
3145 6.22845135467287e-08
3146 6.22500024860528e-08
3147 6.22677305273101e-08
3148 6.22077322987025e-08
3149 6.22442399844658e-08
3150 6.21960722924086e-08
3151 6.21975928538632e-08
3152 6.21757934027301e-08
3153 6.21912974452243e-08
3154 6.21421420987645e-08
3155 6.21477624918043e-08
3156 6.21291107449906e-08
3157 6.21613551743394e-08
3158 6.21147364654462e-08
3159 6.21187652427579e-08
3160 6.20847231402877e-08
3161 6.21409697032504e-08
3162 6.20119351424364e-08
3163 6.20952320673496e-08
3164 6.20432842879381e-08
3165 6.20558537889337e-08
3166 6.20315745436528e-08
3167 6.20355606884004e-08
3168 6.19751219232967e-08
3169 6.20237514681321e-08
3170 6.19480786667737e-08
3171 6.20494731151666e-08
3172 6.19266771195726e-08
3173 6.19773032894955e-08
3174 6.19195361650782e-08
3175 6.19323046180398e-08
3176 6.1866678890965e-08
3177 6.19270679180772e-08
3178 6.18971114363376e-08
3179 6.18808613239707e-08
3180 6.18378237504658e-08
3181 6.21213587237435e-08
3182 6.18448652289771e-08
3183 6.19985911498588e-08
3184 6.1910114368402e-08
3185 6.18827655785026e-08
3186 6.17961859461502e-08
3187 6.17953759274315e-08
3188 6.17330684349326e-08
3189 6.17322726270686e-08
3190 6.16992039681463e-08
3191 6.16869684222365e-08
3192 6.16742852344032e-08
3193 6.16587527701995e-08
3194 6.16418773802252e-08
3195 6.16214350657174e-08
3196 6.15432256267923e-08
3197 6.15330222331067e-08
3198 6.15421527072613e-08
3199 6.16796427266308e-08
3200 6.16349140614147e-08
3201 6.17194544361155e-08
3202 6.17019466631064e-08
3203 6.17469666508441e-08
3204 6.1689341634974e-08
3205 6.17200726082956e-08
3206 6.16703346167924e-08
3207 6.17217921217161e-08
3208 6.16457498381351e-08
3209 6.18909936633827e-08
3210 6.17763333821131e-08
3211 6.17942959024731e-08
3212 6.17660518287266e-08
3213 6.17602395891481e-08
3214 6.17593514107284e-08
3215 6.16786692830829e-08
3216 6.16765234440209e-08
3217 6.16467374925378e-08
3218 6.16429360889015e-08
3219 6.16366193639806e-08
3220 6.15921749158588e-08
3221 6.16064994574117e-08
3222 6.17007742675924e-08
3223 6.16401507613773e-08
3224 6.15905264567118e-08
3225 6.15700912476314e-08
3226 6.15481425825237e-08
3227 6.14007689136997e-08
3228 6.13235542346047e-08
3229 6.14126705045237e-08
3230 6.13553226003205e-08
3231 6.14166211221345e-08
3232 6.13597563869916e-08
3233 6.15390902680701e-08
3234 6.141198127807e-08
3235 6.13265456195222e-08
3236 6.14398132370297e-08
3237 6.13102599800186e-08
3238 6.15191524389047e-08
3239 6.13730293252956e-08
3240 6.12703132674142e-08
3241 6.13161432738707e-08
3242 6.14385626818148e-08
3243 6.13233979152028e-08
3244 6.12760899798559e-08
3245 6.13186799114374e-08
3246 6.12871460248243e-08
3247 6.1464973555303e-08
3248 6.12844530678558e-08
3249 6.12275812272856e-08
3250 6.13616535360961e-08
3251 6.11957773344329e-08
3252 6.1155205344221e-08
3253 6.11889561241696e-08
3254 6.12101587194047e-08
3255 6.11970492059299e-08
3256 6.11129777894348e-08
3257 6.11475527989569e-08
3258 6.10807688872228e-08
3259 6.10851387250477e-08
3260 6.11267054750897e-08
3261 6.10725976457616e-08
3262 6.10414829793626e-08
3263 6.10134804901463e-08
3264 6.10316206461903e-08
3265 6.09921784189282e-08
3266 6.10935373401844e-08
3267 6.10755535035423e-08
3268 6.1046854682445e-08
3269 6.10519350630057e-08
3270 6.10398274147883e-08
3271 6.10213248819491e-08
3272 6.09829058362266e-08
3273 6.10217014695991e-08
3274 6.10033552561617e-08
3275 6.10004846635093e-08
3276 6.09934502904252e-08
3277 6.08939316748547e-08
3278 6.10356920560662e-08
3279 6.09929386996555e-08
3280 6.09628614256508e-08
3281 6.09558057362847e-08
3282 6.08798274015498e-08
3283 6.09817476515673e-08
3284 6.08778805144539e-08
3285 6.08864922924113e-08
3286 6.08866201901037e-08
3287 6.09040213817025e-08
3288 6.09174719556904e-08
3289 6.08674781688023e-08
3290 6.08900521115174e-08
3291 6.08294499215845e-08
3292 6.09165979881254e-08
3293 6.07388486173477e-08
3294 6.07420176379492e-08
3295 6.06658119295389e-08
3296 6.08269417057272e-08
3297 6.08114305578056e-08
3298 6.08027121984378e-08
3299 6.07679453423771e-08
3300 6.07658350304519e-08
3301 6.07514252237706e-08
3302 6.05962924282721e-08
3303 6.05732566327788e-08
3304 6.06388326218621e-08
3305 6.06290981863822e-08
3306 6.07491941195804e-08
3307 6.0693750469909e-08
3308 6.05786638629979e-08
3309 6.08751733466306e-08
3310 6.07025611998324e-08
3311 6.05227583605483e-08
3312 6.04802323778131e-08
3313 6.04716205998557e-08
3314 6.06207564146644e-08
3315 6.05132299824618e-08
3316 6.05149921284465e-08
3317 6.0535946033724e-08
3318 6.05532690656219e-08
3319 6.04616872124097e-08
3320 6.04562018224897e-08
3321 6.03800955900624e-08
3322 6.03943490773418e-08
3323 6.04379764013174e-08
3324 6.03700485157788e-08
3325 6.03625593953439e-08
3326 6.03394809672864e-08
3327 6.02986176545528e-08
3328 6.02647745040485e-08
3329 6.02360756829512e-08
3330 6.02719865128165e-08
3331 6.023962839663e-08
3332 6.02933880600176e-08
3333 6.02477996380912e-08
3334 6.02046412723212e-08
3335 6.02376957203887e-08
3336 6.01290608415184e-08
3337 6.02309810915358e-08
3338 6.01916880782483e-08
3339 6.01003264932842e-08
3340 6.00514837856281e-08
3341 6.01008167677719e-08
3342 6.00733685018895e-08
3343 6.0035546312065e-08
3344 6.00541554263145e-08
3345 5.9981104527651e-08
3346 6.00672933614987e-08
3347 5.99545870727525e-08
3348 5.99274585511012e-08
3349 5.98785732108809e-08
3350 5.98766476400669e-08
3351 5.98971610088483e-08
3352 5.98705014454026e-08
3353 5.98085989622632e-08
3354 5.98661387130051e-08
3355 5.976959016607e-08
3356 5.98896150449946e-08
3357 5.97859326489925e-08
3358 5.97963421000713e-08
3359 5.97717928485508e-08
3360 5.97986584693899e-08
3361 5.97030194171566e-08
3362 5.983694961742e-08
3363 5.97054210516035e-08
3364 5.97084408582305e-08
3365 5.97212235220468e-08
3366 5.96672720121205e-08
3367 5.96593068280526e-08
3368 5.96640887806643e-08
3369 5.96559388554851e-08
3370 5.96109401840295e-08
3371 5.97079719000249e-08
3372 5.95621756360742e-08
3373 5.9668771257293e-08
3374 5.95433604644313e-08
3375 5.96403850749994e-08
3376 5.97001630353589e-08
3377 5.96516969153527e-08
3378 5.95269540326626e-08
3379 5.95156244287409e-08
3380 5.95211737675072e-08
3381 5.94719082869233e-08
3382 5.95552691606827e-08
3383 5.94401541320622e-08
3384 5.94671938358715e-08
3385 5.94833693412511e-08
3386 5.94530007447247e-08
3387 5.94249947027947e-08
3388 5.95588076635067e-08
3389 5.94061226877329e-08
3390 5.92883964145585e-08
3391 5.93751252608854e-08
3392 5.93018647521149e-08
3393 5.93459148490183e-08
3394 5.9432679222482e-08
3395 5.94285722854693e-08
3396 5.9295469867493e-08
3397 5.92787152697838e-08
3398 5.92787756659163e-08
3399 5.9242278638294e-08
3400 5.93703575191284e-08
3401 5.93289364303473e-08
3402 5.92471955940255e-08
3403 5.9331959789688e-08
3404 5.92025877210745e-08
3405 5.91887072687314e-08
3406 5.91708193553586e-08
3407 5.91566369223528e-08
3408 5.92993067982661e-08
3409 5.91041526831759e-08
3410 5.90912279108124e-08
3411 5.90966244828905e-08
3412 5.9086012527132e-08
3413 5.90865845140343e-08
3414 5.90838276082195e-08
3415 5.90503148600874e-08
3416 5.92046092151577e-08
3417 5.90172177794557e-08
3418 5.92452842340663e-08
3419 5.90895048446782e-08
3420 5.90296771463272e-08
3421 5.89226765157491e-08
3422 5.91906541558274e-08
3423 5.91099613700408e-08
3424 5.90730628857727e-08
3425 5.91488991119604e-08
3426 5.91115458803415e-08
3427 5.91714552911071e-08
3428 5.90763811203487e-08
3429 5.90656625831798e-08
3430 5.91442841368917e-08
3431 5.90514552811783e-08
3432 5.90402677858037e-08
3433 5.89703077480408e-08
3434 5.90810316225543e-08
3435 5.90856004123452e-08
3436 5.89937663164619e-08
3437 5.90713149506428e-08
3438 5.89681761198335e-08
3439 5.9012126740754e-08
3440 5.89725885902226e-08
3441 5.90391842081317e-08
3442 5.89532263006731e-08
3443 5.89918158766523e-08
3444 5.8923916412823e-08
3445 5.89966759889649e-08
3446 5.89156456953788e-08
3447 5.90343773865243e-08
3448 5.88994595318582e-08
3449 5.89601256706374e-08
3450 5.89403370554464e-08
3451 5.89535709139e-08
3452 5.88716595473215e-08
3453 5.89334021583454e-08
3454 5.88570010506828e-08
3455 5.89388520211287e-08
3456 5.88494479814017e-08
3457 5.89203068557254e-08
3458 5.88299293724504e-08
3459 5.88926170053128e-08
3460 5.87788058226124e-08
3461 5.88883466434709e-08
3462 5.88110147248244e-08
3463 5.87915423011509e-08
3464 5.88024207104354e-08
3465 5.88350452801478e-08
3466 5.87547361874385e-08
3467 5.88333186612999e-08
3468 5.87469237700589e-08
3469 5.89053712474197e-08
3470 5.87733310908334e-08
3471 5.88557362846132e-08
3472 5.87854032119139e-08
3473 5.90783528764405e-08
3474 5.90716062731644e-08
3475 5.88333790574325e-08
3476 5.9180205624898e-08
3477 5.90324660265651e-08
3478 5.87968216336776e-08
3479 5.88484105890075e-08
3480 5.90230975205941e-08
3481 5.90174593639858e-08
3482 5.91627546953077e-08
3483 5.91630815449662e-08
3484 5.91389479609461e-08
3485 5.91244635472776e-08
3486 5.91504445424107e-08
3487 5.91151412265845e-08
3488 5.9134425356433e-08
3489 5.91040461017656e-08
3490 5.90768358676996e-08
3491 5.90954059020987e-08
3492 5.91244564418503e-08
3493 5.90644511078153e-08
3494 5.90568518532564e-08
3495 5.91062736532422e-08
3496 5.90602482475333e-08
3497 5.90408504308471e-08
3498 5.90320290427826e-08
3499 5.90425521806992e-08
3500 5.90544502188095e-08
3501 5.9007557950963e-08
3502 5.90046802528832e-08
3503 5.89947291018689e-08
3504 5.90028434999113e-08
3505 5.89951731910787e-08
3506 5.90364699348811e-08
3507 5.89931481442818e-08
3508 5.89738178291555e-08
3509 5.9061147084094e-08
3510 5.89716542265251e-08
3511 5.89501389924862e-08
3512 5.89793565097807e-08
3513 5.89716684373798e-08
3514 5.89661190986135e-08
3515 5.90109614506673e-08
3516 5.89742121803738e-08
3517 5.89560507080478e-08
3518 5.89493645009043e-08
3519 5.89750683843704e-08
3520 5.89528639238779e-08
3521 5.8997294161145e-08
3522 5.89803725858928e-08
3523 5.89422910479698e-08
3524 5.89379247628585e-08
3525 5.89577311416178e-08
3526 5.89363331471304e-08
3527 5.90405129230476e-08
3528 5.89363331471304e-08
3529 5.8929192192636e-08
3530 5.89336437428756e-08
3531 5.89393991390352e-08
3532 5.89419038021788e-08
3533 5.89219908420091e-08
3534 5.89272097784033e-08
3535 5.89253339455809e-08
3536 5.89356581315315e-08
3537 5.89177844290134e-08
3538 5.89117519211868e-08
3539 5.89531268246901e-08
3540 5.89231845538052e-08
3541 5.8912888789564e-08
3542 5.88883217744751e-08
3543 5.88858704020367e-08
3544 5.88890003427878e-08
3545 5.88967772330307e-08
3546 5.88982409510663e-08
3547 5.89181858856591e-08
3548 5.88807012036341e-08
3549 5.88714392790735e-08
3550 5.88926134525991e-08
3551 5.89287196817168e-08
3552 5.88967878911717e-08
3553 5.88936970302711e-08
3554 5.88648347843446e-08
3555 5.88610795659861e-08
3556 5.88683484181729e-08
3557 5.89058792854757e-08
3558 5.89381983218118e-08
3559 5.88447939264825e-08
3560 5.88438098247934e-08
3561 5.88778341636953e-08
3562 5.88517679034339e-08
3563 5.88701496440081e-08
3564 5.88575943538672e-08
3565 5.88443320737042e-08
3566 5.88367825571368e-08
3567 5.8866135077551e-08
3568 5.88491175790296e-08
3569 5.882327513973e-08
3570 5.88385411504078e-08
3571 5.88247779376161e-08
3572 5.88786264188457e-08
3573 5.88293751491165e-08
3574 5.88122865963214e-08
3575 5.8818805825922e-08
3576 5.88479700525113e-08
3577 5.88115582900173e-08
3578 5.883880405122e-08
3579 5.87999799961381e-08
3580 5.88587774075222e-08
3581 5.88092348152713e-08
3582 5.87876272106769e-08
3583 5.87958162157065e-08
3584 5.88175481652797e-08
3585 5.88057496031524e-08
3586 5.87695367926244e-08
3587 5.87857158507177e-08
3588 5.8770403654762e-08
3589 5.886331422289e-08
3590 5.87790154327195e-08
3591 5.87610955449236e-08
3592 5.87661617146296e-08
3593 5.87819890540686e-08
3594 5.87629465087502e-08
3595 5.87618380620825e-08
3596 5.87728052892089e-08
3597 5.87671600271733e-08
3598 5.8796153723506e-08
3599 5.87886290759343e-08
3600 5.8738084618426e-08
3601 5.87327129153437e-08
3602 5.87397970264192e-08
3603 5.87354875847268e-08
3604 5.87731427970084e-08
3605 5.87241189009546e-08
3606 5.87354804792994e-08
3607 5.87401700613555e-08
3608 5.87342228186571e-08
3609 5.87422626097123e-08
3610 5.87238346838603e-08
3611 5.87182320543889e-08
3612 5.87089452608325e-08
3613 5.8709680672564e-08
3614 5.87051367517688e-08
3615 5.87799071638528e-08
3616 5.87071156132879e-08
3617 5.8693576221458e-08
3618 5.87010227093288e-08
3619 5.87045647648665e-08
3620 5.87227262371925e-08
3621 5.86930326562651e-08
3622 5.87072399582667e-08
3623 5.87452504419161e-08
3624 5.86809214553341e-08
3625 5.86985038353305e-08
3626 5.86968447180425e-08
3627 5.86631010435212e-08
3628 5.86695527715619e-08
3629 5.86641171196334e-08
3630 5.86898529775226e-08
3631 5.86618718045884e-08
3632 5.86708281957726e-08
3633 5.86873056818149e-08
3634 5.86566564209079e-08
3635 5.86750310560546e-08
3636 5.86694248738695e-08
3637 5.86475117358987e-08
3638 5.86689417048092e-08
3639 5.8668213398505e-08
3640 5.86882080710893e-08
3641 5.86505173316709e-08
3642 5.86675312774787e-08
3643 5.86827511028787e-08
3644 5.87323967238262e-08
3645 5.86485953135707e-08
3646 5.86576831551611e-08
3647 5.86504178556879e-08
3648 5.86772976873817e-08
3649 5.86407509217679e-08
3650 5.86611434982842e-08
3651 5.86479877995316e-08
3652 5.8695334814729e-08
3653 5.86513166922487e-08
3654 5.86592854290302e-08
3655 5.86650124034804e-08
3656 5.8639312072728e-08
3657 5.86528692281263e-08
3658 5.86341251107569e-08
3659 5.8626824284147e-08
3660 5.87025787979201e-08
3661 5.86430708438002e-08
3662 5.86325654694519e-08
3663 5.86441188943354e-08
3664 5.86298156690646e-08
3665 5.86061759122458e-08
3666 5.86193813489899e-08
3667 5.86125494805856e-08
3668 5.86639359312358e-08
3669 5.86140060931939e-08
3670 5.86041615235899e-08
3671 5.86102189004123e-08
3672 5.86097499422067e-08
3673 5.86315174189167e-08
3674 5.86108370725924e-08
3675 5.85879611492146e-08
3676 5.86058028773095e-08
3677 5.85992552259995e-08
3678 5.86577435512936e-08
3679 5.8595777119308e-08
3680 5.85784611928375e-08
3681 5.8592181773065e-08
3682 5.85915529427439e-08
3683 5.8612002362679e-08
3684 5.85801380736939e-08
3685 5.85632626837196e-08
3686 5.85880322034882e-08
3687 5.8567096061779e-08
3688 5.8593045082489e-08
3689 5.85630139937621e-08
3690 5.85737467417857e-08
3691 5.85722936818911e-08
3692 5.85712101042191e-08
3693 5.85678208153695e-08
3694 5.85619055470943e-08
3695 5.85824793120082e-08
3696 5.85553436849295e-08
3697 5.85553330267885e-08
3698 5.85517838658234e-08
3699 5.85659023499829e-08
3700 5.85660657748122e-08
3701 5.85336579206341e-08
3702 5.85458117541293e-08
3703 5.85395198982042e-08
3704 5.85568677990977e-08
3705 5.8534400437793e-08
3706 5.85388555407462e-08
3707 5.85290749199885e-08
3708 5.85549244647154e-08
3709 5.85423123311557e-08
3710 5.85388555407462e-08
3711 5.85189638968586e-08
3712 5.8521504087139e-08
3713 5.85100217165291e-08
3714 5.85381343398694e-08
3715 5.85191912705341e-08
3716 5.85151518350813e-08
3717 5.85268402630845e-08
3718 5.85509702943909e-08
3719 5.84911177270442e-08
3720 5.85065400571239e-08
3721 5.85034030109455e-08
3722 5.85049555468231e-08
3723 5.84936934444613e-08
3724 5.85072399417186e-08
3725 5.85100252692428e-08
3726 5.84834971562032e-08
3727 5.84847370532771e-08
3728 5.84963579797204e-08
3729 5.84750310395066e-08
3730 5.84973136597e-08
3731 5.84754147325839e-08
3732 5.84666288716562e-08
3733 5.84681920656749e-08
3734 5.84794719316051e-08
3735 5.84843995454776e-08
3736 5.84713610862764e-08
3737 5.84587418472893e-08
3738 5.8461576912805e-08
3739 5.84513060175595e-08
3740 5.84662913638567e-08
3741 5.84566244299367e-08
3742 5.84514694423888e-08
3743 5.84461901098621e-08
3744 5.84377524148749e-08
3745 5.84311230511503e-08
3746 5.84933026459566e-08
3747 5.84299968409141e-08
3748 5.8443447414902e-08
3749 5.84293502470246e-08
3750 5.84274388870654e-08
3751 5.84249342239218e-08
3752 5.84343986531621e-08
3753 5.84098600597827e-08
3754 5.84124144609177e-08
3755 5.84164112638064e-08
3756 5.84028505556944e-08
3757 5.84054689056757e-08
3758 5.84121089275413e-08
3759 5.83868526859987e-08
3760 5.84334429731825e-08
3761 5.84123718283536e-08
3762 5.83812820309504e-08
3763 5.83783901220158e-08
3764 5.83873571713411e-08
3765 5.83757469030388e-08
3766 5.83863091208059e-08
3767 5.83757540084662e-08
3768 5.83757824301756e-08
3769 5.83641934781554e-08
3770 5.8365159816276e-08
3771 5.84039305806527e-08
3772 5.83829411482384e-08
3773 5.83518300345531e-08
3774 5.83551305055607e-08
3775 5.83360559858193e-08
3776 5.83552015598343e-08
3777 5.83250070462782e-08
3778 5.83412891330681e-08
3779 5.83255435060437e-08
3780 5.83211452465093e-08
3781 5.83383510388558e-08
3782 5.83173864754372e-08
3783 5.83243391361066e-08
3784 5.83107571117125e-08
3785 5.8334620689493e-08
3786 5.82990686837093e-08
3787 5.83122243824619e-08
3788 5.83143560106691e-08
3789 5.83040034030091e-08
3790 5.83279522459179e-08
3791 5.82968198159506e-08
3792 5.82838950435871e-08
3793 5.82975054896906e-08
3794 5.82831916062787e-08
3795 5.82732440079781e-08
3796 5.83222394823224e-08
3797 5.82732440079781e-08
3798 5.82755816935787e-08
3799 5.82688350903027e-08
3800 5.82666039861124e-08
3801 5.8270600789001e-08
3802 5.82658792325219e-08
3803 5.82546704208653e-08
3804 5.8250591905562e-08
3805 5.82856607422855e-08
3806 5.82627563971982e-08
3807 5.82356065592649e-08
3808 5.82567452056537e-08
3809 5.82510004676351e-08
3810 5.82371022517236e-08
3811 5.82692898376536e-08
3812 5.82398484993973e-08
3813 5.82158641293518e-08
3814 5.82305759166957e-08
3815 5.82192853926244e-08
3816 5.82260888393193e-08
3817 5.82536330284711e-08
3818 5.82211896471563e-08
3819 5.82192924980518e-08
3820 5.82074974886382e-08
3821 5.82207526633738e-08
3822 5.82164005891173e-08
3823 5.8215142928475e-08
3824 5.82245647251511e-08
3825 5.81929775478329e-08
3826 5.81975747593333e-08
3827 5.82157468898004e-08
3828 5.82185393227519e-08
3829 5.81684140854577e-08
3830 5.81894887830003e-08
3831 5.8203639241583e-08
3832 5.82026373763256e-08
3833 5.82435717433327e-08
3834 5.81937307231328e-08
3835 5.81667300991739e-08
3836 5.82037209539976e-08
3837 5.81956989265109e-08
3838 5.81650567710312e-08
3839 5.81517021203126e-08
3840 5.81791930187592e-08
3841 5.81868349058823e-08
3842 5.81737218396938e-08
3843 5.81399213217537e-08
3844 5.81665666743447e-08
3845 5.81864405546639e-08
3846 5.81659413967373e-08
3847 5.81344608008294e-08
3848 5.81596850679489e-08
3849 5.81547681122174e-08
3850 5.81581005576481e-08
3851 5.81783190511942e-08
3852 5.81353774009585e-08
3853 5.81155141787804e-08
3854 5.81485686268479e-08
3855 5.81615715589123e-08
3856 5.81338746030724e-08
3857 5.81292347590079e-08
3858 5.81111194719597e-08
3859 5.81268686516978e-08
3860 5.81319028469807e-08
3861 5.81259733678507e-08
3862 5.81063943627669e-08
3863 5.81008947619921e-08
3864 5.81321728532203e-08
3865 5.81154040446563e-08
3866 5.81060675131084e-08
3867 5.81213512873546e-08
3868 5.80968624319667e-08
3869 5.81332422200376e-08
3870 5.81036339042384e-08
3871 5.80835148866754e-08
3872 5.81005394906242e-08
3873 5.81268864152662e-08
3874 5.80766368329932e-08
3875 5.80908618985632e-08
3876 5.80990473508791e-08
3877 5.80901975411052e-08
3878 5.80851029496898e-08
3879 5.80700785235422e-08
3880 5.80775925129728e-08
3881 5.80941481587161e-08
3882 5.80484424972383e-08
3883 5.80598857879977e-08
3884 5.80545673756205e-08
3885 5.80826267082557e-08
3886 5.80578607412008e-08
3887 5.80653924941998e-08
3888 5.80709453856798e-08
3889 5.80380365988731e-08
3890 5.8041720762958e-08
3891 5.80595447274845e-08
3892 5.80379726500269e-08
3893 5.8020162896355e-08
3894 5.80492809376665e-08
3895 5.8022685323067e-08
3896 5.80125387728003e-08
3897 5.80310270947848e-08
3898 5.80048009624079e-08
3899 5.80256376281341e-08
3900 5.80249377435393e-08
3901 5.80200207878079e-08
3902 5.80169796648988e-08
3903 5.80030139474275e-08
3904 5.7997123548148e-08
3905 5.7990089175064e-08
3906 5.80057815113832e-08
3907 5.80718086951038e-08
3908 5.79906149766884e-08
3909 5.8003262637385e-08
3910 5.7999866243108e-08
3911 5.79973331582551e-08
3912 5.79702970071594e-08
3913 5.79806460621057e-08
3914 5.79865009342484e-08
3915 5.79943133516281e-08
3916 5.80084673629244e-08
3917 5.79787098331508e-08
3918 5.79631880270881e-08
3919 5.79658703259156e-08
3920 5.79583527837713e-08
3921 5.79840779835195e-08
3922 5.7983175594245e-08
3923 5.79837795555704e-08
3924 5.79814916079613e-08
3925 5.7971462297246e-08
3926 5.79954217982959e-08
3927 5.79478047768589e-08
3928 5.79554928492598e-08
3929 5.79724428462214e-08
3930 5.79500074593398e-08
3931 5.7934464336995e-08
3932 5.79515244680806e-08
3933 5.7916015094861e-08
3934 5.79112153786809e-08
3935 5.79315155846416e-08
3936 5.79238523812364e-08
3937 5.79171341996698e-08
3938 5.79469805472854e-08
3939 5.79012464640982e-08
3940 5.7900219729845e-08
3941 5.7955258370157e-08
3942 5.79192978023002e-08
3943 5.79087000573963e-08
3944 5.79422021473874e-08
3945 5.79050585258756e-08
3946 5.78971253162308e-08
3947 5.79287053881217e-08
3948 5.79003298639691e-08
3949 5.79131089750717e-08
3950 5.79239305409374e-08
3951 5.79237529052534e-08
3952 5.79012997548034e-08
3953 5.79214223250801e-08
3954 5.79157095614846e-08
3955 5.78972247922138e-08
3956 5.79252272814301e-08
3957 5.79385179833025e-08
3958 5.79044154846997e-08
3959 5.79141854473164e-08
3960 5.79080925433573e-08
3961 5.7915539031228e-08
3962 5.7935139352594e-08
3963 5.79010652757006e-08
3964 5.79259449295932e-08
3965 5.78812944240781e-08
3966 5.79027918945485e-08
3967 5.79226053787352e-08
3968 5.78888048607951e-08
3969 5.78966670161662e-08
3970 5.79260017730121e-08
3971 5.78937360273812e-08
3972 5.78891032887441e-08
3973 5.79091228303241e-08
3974 5.78797560990552e-08
3975 5.78745904533662e-08
3976 5.78685721563943e-08
3977 5.79017829238637e-08
3978 5.7877070247514e-08
3979 5.78751020441359e-08
3980 5.78500412018457e-08
3981 5.78587346922177e-08
3982 5.78485739310963e-08
3983 5.78864174372029e-08
3984 5.78884815638503e-08
3985 5.78588519317691e-08
3986 5.78477568069502e-08
3987 5.79611736384322e-08
3988 5.7845735312867e-08
3989 5.7847028500646e-08
3990 5.78524961269977e-08
3991 5.78566776709977e-08
3992 5.78297587594534e-08
3993 5.78160381792259e-08
3994 5.78601060396977e-08
3995 5.78474015355823e-08
3996 5.7844196987844e-08
3997 5.78493590808193e-08
3998 5.78445025212204e-08
3999 5.78319507837932e-08
4000 5.78299648168468e-08
4001 5.78481085256044e-08
4002 5.78309702348179e-08
4003 5.78412802099137e-08
4004 5.78254528704747e-08
4005 5.7840679801302e-08
4006 5.78171466258937e-08
4007 5.78127803407824e-08
4008 5.78177115073686e-08
4009 5.77844403437666e-08
4010 5.78374788062774e-08
4011 5.77996068784614e-08
4012 5.78482293178695e-08
4013 5.78292009834058e-08
4014 5.78098777737068e-08
4015 5.77912686594573e-08
4016 5.77966829951038e-08
4017 5.78140699758478e-08
4018 5.77980863170069e-08
4019 5.77943346513621e-08
4020 5.78022749664342e-08
4021 5.78038026333161e-08
4022 5.78029002440417e-08
4023 5.7792366447984e-08
4024 5.77974113014079e-08
4025 5.77855985284259e-08
4026 5.77906327237088e-08
4027 5.78038736875897e-08
4028 5.77863978890036e-08
4029 5.77956562608506e-08
4030 5.77822483194268e-08
4031 5.77887817598821e-08
4032 5.77891086095406e-08
4033 5.77846037685958e-08
4034 5.77851970717802e-08
4035 5.77700944859316e-08
4036 5.77803369594676e-08
4037 5.77944199164904e-08
4038 5.77228647102856e-08
4039 5.77868526363545e-08
4040 5.77795766787403e-08
4041 5.77757823805314e-08
4042 5.77607259799606e-08
4043 5.77760843611941e-08
4044 5.77800598478007e-08
4045 5.77741872120896e-08
4046 5.77785392863461e-08
4047 5.77713059612961e-08
4048 5.77682008895408e-08
4049 5.77773846544005e-08
4050 5.77504835064246e-08
4051 5.7757237215128e-08
4052 5.77219552155839e-08
4053 5.77468703966133e-08
4054 5.78123042771495e-08
4055 5.77675152158008e-08
4056 5.77437369031486e-08
4057 5.77507961452284e-08
4058 5.77525050005079e-08
4059 5.77626728670566e-08
4060 5.77668792800523e-08
4061 5.77577736748935e-08
4062 5.77196956896842e-08
4063 5.77237635468464e-08
4064 5.77604701845758e-08
4065 5.77233194576365e-08
4066 5.7738887448977e-08
4067 5.77349652530756e-08
4068 5.77289931413816e-08
4069 5.77224206210758e-08
4070 5.77239376298166e-08
4071 5.77256749068056e-08
4072 5.77260941270197e-08
4073 5.77288439274071e-08
4074 5.77565657522428e-08
4075 5.77197241113936e-08
4076 5.76860159640091e-08
4077 5.77316328076449e-08
4078 5.77002978729979e-08
4079 5.77826035907947e-08
4080 5.77106931132221e-08
4081 5.76961447507074e-08
4082 5.76994949597065e-08
4083 5.77055381256741e-08
4084 5.77061776141363e-08
4085 5.76781040706464e-08
4086 5.76920058392716e-08
4087 5.75849803396977e-08
4088 5.77087178044167e-08
4089 5.76801078011613e-08
4090 5.7701591060777e-08
4091 5.76815324393465e-08
4092 5.77249039679373e-08
4093 5.76546348440843e-08
4094 5.76805767593669e-08
4095 5.76651224548641e-08
4096 5.76842005273193e-08
4097 5.76565533094708e-08
4098 5.76496468340792e-08
4099 5.76377168215458e-08
4100 5.76655416750782e-08
4101 5.76514374017734e-08
4102 5.76326399936988e-08
4103 5.76365728477413e-08
4104 5.76647316563594e-08
4105 5.76444278976851e-08
4106 5.7628799510212e-08
4107 5.7638246175884e-08
4108 5.75797649560172e-08
4109 5.76532883656e-08
4110 5.76605607705005e-08
4111 5.75650851430964e-08
4112 5.76312899625009e-08
4113 5.76125316342768e-08
4114 5.76222447534747e-08
4115 5.76353862413725e-08
4116 5.75235610256186e-08
4117 5.75748764219952e-08
4118 5.74306042722128e-08
4119 5.74855754109649e-08
4120 5.74193173008553e-08
4121 5.73998164554723e-08
4122 5.73839500361828e-08
4123 5.74030067923559e-08
4124 5.74395890851065e-08
4125 5.73541001358535e-08
4126 5.73506397927304e-08
4127 5.73293554850807e-08
4128 5.73266270009753e-08
4129 5.72832021816794e-08
4130 5.72838843027057e-08
4131 5.72345335569935e-08
4132 5.72426479550359e-08
4133 5.72258151976257e-08
4134 5.72886555971763e-08
4135 5.72010812049939e-08
4136 5.72160097078722e-08
4137 5.71776013202907e-08
4138 5.72367540030427e-08
4139 5.71280907024629e-08
4140 5.70025591173362e-08
4141 5.69320626198078e-08
4142 5.68614630935826e-08
4143 5.67818254637587e-08
4144 5.67286377872733e-08
4145 5.67531230899476e-08
4146 5.67149918140331e-08
4147 5.6754391408731e-08
4148 5.66790845368814e-08
4149 5.66164715110062e-08
4150 5.65415803066571e-08
4151 5.64599460517456e-08
4152 5.63788375984586e-08
4153 5.63555957455719e-08
4154 5.62841471207776e-08
4155 5.62517499247406e-08
4156 5.62232500556092e-08
4157 5.61917588015604e-08
4158 5.61507569329933e-08
4159 5.61606405824477e-08
4160 5.6163457884395e-08
4161 5.60873871791046e-08
4162 5.60777628777487e-08
4163 5.60934232396448e-08
4164 5.60633743873495e-08
4165 5.60565069918084e-08
4166 5.60666926219255e-08
4167 5.60541693062078e-08
4168 5.60692789974837e-08
4169 5.60314283859498e-08
4170 5.60751090006306e-08
4171 5.60328388132803e-08
4172 5.60558142126411e-08
4173 5.60276340877408e-08
4174 5.60343771383032e-08
4175 5.59336861272186e-08
4176 5.59538406719184e-08
4177 5.59101245301008e-08
4178 5.59170985070523e-08
4179 5.59478046113782e-08
4180 5.58918422655097e-08
4181 5.58495614200183e-08
4182 5.58437207587303e-08
4183 5.60208199829049e-08
4184 5.59107142805715e-08
4185 5.59128530142061e-08
4186 5.59498793961666e-08
4187 5.58863462174486e-08
4188 5.60375639224731e-08
4189 5.59384609744029e-08
4190 5.59264172750318e-08
4191 5.59976172098686e-08
4192 5.59938015953776e-08
4193 5.62604256515442e-08
4194 5.62370416901103e-08
4195 5.62377842072692e-08
4196 5.62467228348851e-08
4197 5.62373969614782e-08
4198 5.62777522361557e-08
4199 5.63135742481791e-08
4200 5.62917144009134e-08
4201 5.62945210447197e-08
4202 5.62883535337733e-08
4203 5.62949189486517e-08
4204 5.63223387928247e-08
4205 5.63751854087968e-08
4206 5.62995907671393e-08
4207 5.64067477171193e-08
4208 5.62410065185759e-08
4209 5.63273729881075e-08
4210 5.62829747252636e-08
4211 5.62354891542327e-08
4212 5.61553470390663e-08
4213 5.6267982273539e-08
4214 5.62642306078942e-08
4215 5.62667636927472e-08
4216 5.62828113004343e-08
4217 5.62573774232078e-08
4218 5.62465451992011e-08
4219 5.626768384559e-08
4220 5.62357485023313e-08
4221 5.62897106703986e-08
4222 5.62652786584295e-08
4223 5.62572566309427e-08
4224 5.6196054032398e-08
4225 5.62401503145793e-08
4226 5.62409638860117e-08
4227 5.62535049652979e-08
4228 5.62251756264232e-08
4229 5.61975319612884e-08
4230 5.62137252302364e-08
4231 5.62397950432114e-08
4232 5.62217614685778e-08
4233 5.62161552863927e-08
4234 5.62465345410601e-08
4235 5.61737145687857e-08
4236 5.61945441290845e-08
4237 5.62514763657873e-08
4238 5.61909949681194e-08
4239 5.61605055793279e-08
4240 5.61969137891083e-08
4241 5.61736790416489e-08
4242 5.61917943286971e-08
4243 5.61930804110489e-08
4244 5.61778144003711e-08
4245 5.61900996842724e-08
4246 5.61681900990152e-08
4247 5.6188344643715e-08
4248 5.61587043534928e-08
4249 5.61475061999772e-08
4250 5.63218733873327e-08
4251 5.62011273075314e-08
4252 5.61345956384685e-08
4253 5.61700659318376e-08
4254 5.62292008510212e-08
4255 5.63178161883116e-08
4256 5.60961623818912e-08
4257 5.62052591135398e-08
4258 5.61786208663761e-08
4259 5.62195126008191e-08
4260 5.61770576723575e-08
4261 5.61982318458831e-08
4262 5.61595783210578e-08
4263 5.60810811123247e-08
4264 5.60541302263573e-08
4265 5.61464332804462e-08
4266 5.61196387138807e-08
4267 5.61398962872772e-08
4268 5.60454722631221e-08
4269 5.6095124989497e-08
4270 5.60966100238147e-08
4271 5.60846373787172e-08
4272 5.62072628440546e-08
4273 5.61156703327015e-08
4274 5.61160433676378e-08
4275 5.60932242876788e-08
4276 5.61334694282323e-08
4277 5.61316824132518e-08
4278 5.60612321010012e-08
4279 5.61046213931604e-08
4280 5.60867228216466e-08
4281 5.60673889538066e-08
4282 5.60759829681956e-08
4283 5.60559243467651e-08
4284 5.60025164020317e-08
4285 5.60459731957508e-08
4286 5.60536292937286e-08
4287 5.60417525719004e-08
4288 5.59507888908684e-08
4289 5.60152066952924e-08
4290 5.59768622565571e-08
4291 5.60773649738167e-08
4292 5.60208022193365e-08
4293 5.59499611085812e-08
4294 5.60169084451445e-08
4295 5.58907622405513e-08
4296 5.58893979984987e-08
4297 5.59517019382838e-08
4298 5.59671136102224e-08
4299 5.59695898516566e-08
4300 5.59926363052909e-08
4301 5.59441808434258e-08
4302 5.59601360805573e-08
4303 5.59540573874528e-08
4304 5.6078832244566e-08
4305 5.59876376371449e-08
4306 5.60324551202029e-08
4307 5.60005766203631e-08
4308 5.60068755817156e-08
4309 5.60168871288624e-08
4310 5.5961411504768e-08
4311 5.59285133761023e-08
4312 5.6034323847598e-08
4313 5.59440778147291e-08
4314 5.59025394863966e-08
4315 5.59377362208124e-08
4316 5.58737021094657e-08
4317 5.59319097703792e-08
4318 5.59707373781748e-08
4319 5.59959225654438e-08
4320 5.5972837031959e-08
4321 5.59476553974037e-08
4322 5.5984152425026e-08
4323 5.59721193837959e-08
4324 5.59957165080505e-08
4325 5.5985911018297e-08
4326 5.59294974777913e-08
4327 5.58741106715388e-08
4328 5.58701991337784e-08
4329 5.5872028781323e-08
4330 5.58117783100442e-08
4331 5.58139667816704e-08
4332 5.5816393285113e-08
4333 5.58559740682085e-08
4334 5.58344304124603e-08
4335 5.60424808782045e-08
4336 5.59275292744132e-08
4337 5.59672770350517e-08
4338 5.59650885634255e-08
4339 5.59251738252442e-08
4340 5.58993207278036e-08
4341 5.59555779489074e-08
4342 5.5911250740337e-08
4343 5.59323183324523e-08
4344 5.59444082171012e-08
4345 5.58910784320688e-08
4346 5.58539028361338e-08
4347 5.59199655469911e-08
4348 5.57937021028465e-08
4349 5.59440600511607e-08
4350 5.58412374118689e-08
4351 5.58904247327519e-08
4352 5.5792082065409e-08
4353 5.58719257526263e-08
4354 5.58576225273555e-08
4355 5.58930715044426e-08
4356 5.59110162612342e-08
4357 5.59101600572376e-08
4358 5.58526096483547e-08
4359 5.58147128515429e-08
4360 5.59256889687276e-08
4361 5.58484671842052e-08
4362 5.59100143959768e-08
4363 5.58799442273994e-08
4364 5.59257351540055e-08
4365 5.59043051850949e-08
4366 5.59115065357219e-08
4367 5.58469928080285e-08
4368 5.58365407243855e-08
4369 5.59123378707227e-08
4370 5.58443531417652e-08
4371 5.57716219873328e-08
4372 5.57486359298309e-08
4373 5.57914674459425e-08
4374 5.57146080382154e-08
4375 5.58392194704993e-08
4376 5.58173347542379e-08
4377 5.58078205870061e-08
4378 5.58929187377544e-08
4379 5.5873865534295e-08
4380 5.59090409524288e-08
4381 5.58320643051502e-08
4382 5.59404647049178e-08
4383 5.58111139525863e-08
4384 5.58655166571498e-08
4385 5.59469413019542e-08
4386 5.58893873403576e-08
4387 5.59385462395312e-08
4388 5.58529116290174e-08
4389 5.58353505653031e-08
4390 5.58900943303797e-08
4391 5.58053514509993e-08
4392 5.57253123645296e-08
4393 5.59040351788553e-08
4394 5.57873889306393e-08
4395 5.61464439385873e-08
4396 5.5802246379244e-08
4397 5.58914727832871e-08
4398 5.58892416790968e-08
4399 5.58878845424715e-08
4400 5.58127233318828e-08
4401 5.58730377520078e-08
4402 5.59305455283265e-08
4403 5.58941550821146e-08
4404 5.59282149481533e-08
4405 5.58553843177378e-08
4406 5.59202462113717e-08
4407 5.58345831791485e-08
4408 5.59048629611425e-08
4409 5.59216388751338e-08
4410 5.58953416884833e-08
4411 5.58927339966431e-08
4412 5.58839126085786e-08
4413 5.58809851725073e-08
4414 5.58433868036445e-08
4415 5.58711512610444e-08
4416 5.58337909239981e-08
4417 5.58179991116958e-08
4418 5.58690018692687e-08
4419 5.58692150320894e-08
4420 5.5850659208545e-08
4421 5.5844399327043e-08
4422 5.58431487718281e-08
4423 5.58193455901801e-08
4424 5.58477850631789e-08
4425 5.58456889621084e-08
4426 5.58323520749582e-08
4427 5.58280994766847e-08
4428 5.58471739964261e-08
4429 5.58079840118353e-08
4430 5.57972406056706e-08
4431 5.58098065539525e-08
4432 5.58225963231962e-08
4433 5.57780062138136e-08
4434 5.58154944485523e-08
4435 5.5795311482143e-08
4436 5.58198429700951e-08
4437 5.57585515537085e-08
4438 5.57601644857186e-08
4439 5.58361570313082e-08
4440 5.5839720403128e-08
4441 5.58131603156653e-08
4442 5.57788411015281e-08
4443 5.58033654840528e-08
4444 5.58162334129975e-08
4445 5.57769297415689e-08
4446 5.57804931133887e-08
4447 5.57869341832884e-08
4448 5.57386243826841e-08
4449 5.57987966942619e-08
4450 5.58081190149551e-08
4451 5.58588624244294e-08
4452 5.5842729551614e-08
4453 5.58499486658093e-08
4454 5.58577681886163e-08
4455 5.58028752095652e-08
4456 5.58158106400697e-08
4457 5.59521176057842e-08
4458 5.58238788528342e-08
4459 5.58088295576908e-08
4460 5.57946009394072e-08
4461 5.58352226676107e-08
4462 5.5861686831804e-08
4463 5.58546915385705e-08
4464 5.58347359458367e-08
4465 5.58712365261727e-08
4466 5.5770414064682e-08
4467 5.57802657397133e-08
4468 5.56511920990488e-08
4469 5.56618786617946e-08
4470 5.56798234185862e-08
4471 5.56608270585457e-08
4472 5.56175550059379e-08
4473 5.56805161977536e-08
4474 5.56713679600307e-08
4475 5.56635590953647e-08
4476 5.56392585338017e-08
4477 5.56879626856244e-08
4478 5.55831007886809e-08
4479 5.5612996874288e-08
4480 5.5591030445612e-08
4481 5.57150023894337e-08
4482 5.57339276952007e-08
4483 5.5814744825966e-08
4484 5.57247510357683e-08
4485 5.57307409110308e-08
4486 5.57655965849335e-08
4487 5.56428076947668e-08
4488 5.56077814906075e-08
4489 5.56797310480306e-08
4490 5.57077619589563e-08
4491 5.57778321308433e-08
4492 5.55990595785261e-08
4493 5.56690480379984e-08
4494 5.57045325422223e-08
4495 5.56196155798716e-08
4496 5.56488899405849e-08
4497 5.56201165125003e-08
4498 5.55393278034444e-08
4499 5.56457777634023e-08
4500 5.56442394383794e-08
4501 5.55918546751855e-08
4502 5.56202941481843e-08
4503 5.56933699158435e-08
4504 5.55703927318518e-08
4505 5.54475221292705e-08
4506 5.54983543565868e-08
4507 5.54763168736372e-08
4508 5.55314940697826e-08
4509 5.54477850300827e-08
4510 5.54571002453486e-08
4511 5.54636621075133e-08
4512 5.53750965082145e-08
4513 5.54364412153063e-08
4514 5.5389890007973e-08
4515 5.54076926562175e-08
4516 5.54409851361015e-08
4517 5.540326242226e-08
4518 5.54178782863346e-08
4519 5.53996741814444e-08
4520 5.5423161171575e-08
4521 5.54303234423514e-08
4522 5.53399459590764e-08
4523 5.5334066217938e-08
4524 5.5364107964806e-08
4525 5.54022037135837e-08
4526 5.5416325750457e-08
4527 5.54012125064673e-08
4528 5.54000472163807e-08
4529 5.532274016673e-08
4530 5.54570434019297e-08
4531 5.54294850019232e-08
4532 5.54084564896584e-08
4533 5.54572991973146e-08
4534 5.54025305632422e-08
4535 5.53635253197626e-08
4536 5.54362884486181e-08
4537 5.53795089786036e-08
4538 5.53849446305321e-08
4539 5.54105348271605e-08
4540 5.53475736353448e-08
4541 5.53769439193275e-08
4542 5.53490409060942e-08
4543 5.53703074501755e-08
4544 5.53450689722013e-08
4545 5.53534391656285e-08
4546 5.53779067047344e-08
4547 5.53424968074978e-08
4548 5.53771783984303e-08
4549 5.5345580562971e-08
4550 5.54104424566049e-08
4551 5.53518084700499e-08
4552 5.53338814768267e-08
4553 5.5406967902627e-08
4554 5.53153043370003e-08
4555 5.53332171193688e-08
4556 5.53374235323645e-08
4557 5.53856089879901e-08
4558 5.53143273407386e-08
4559 5.53258914237631e-08
4560 5.53643957346139e-08
4561 5.53423866733738e-08
4562 5.53329009278514e-08
4563 5.5337672222322e-08
4564 5.53304033701352e-08
4565 5.53630030708518e-08
4566 5.53156951355049e-08
4567 5.53273657999398e-08
4568 5.53081136445144e-08
4569 5.53695507221619e-08
4570 5.53270460557087e-08
4571 5.53290675497919e-08
4572 5.53573258343931e-08
4573 5.53109238410343e-08
4574 5.530683466759e-08
4575 5.53391537039261e-08
4576 5.53084369414591e-08
4577 5.53058328023326e-08
4578 5.53396617419821e-08
4579 5.52954375621084e-08
4580 5.53405534731155e-08
4581 5.53126469071685e-08
4582 5.52880585757975e-08
4583 5.53248895585057e-08
4584 5.52631576056228e-08
4585 5.52802710274136e-08
4586 5.5326246695131e-08
4587 5.52606458370519e-08
4588 5.52439942680394e-08
4589 5.53139116732382e-08
4590 5.525051349764e-08
4591 5.53018004723071e-08
4592 5.52506413953324e-08
4593 5.52586882918149e-08
4594 5.52826193711553e-08
4595 5.52824097610483e-08
4596 5.52122898511698e-08
4597 5.5259008036046e-08
4598 5.52158851974127e-08
4599 5.52368604189724e-08
4600 5.52755778926439e-08
4601 5.52776064921545e-08
4602 5.51993259989558e-08
4603 5.52550396548668e-08
4604 5.52567236411505e-08
4605 5.52208980764135e-08
4606 5.52353860427957e-08
4607 5.523323665102e-08
4608 5.52343664139698e-08
4609 5.51888454936034e-08
4610 5.51959900008114e-08
4611 5.52165637657254e-08
4612 5.52701671097111e-08
4613 5.51684635752281e-08
4614 5.52429497702178e-08
4615 5.51860779296476e-08
4616 5.51841132789832e-08
4617 5.52158923028401e-08
4618 5.52123857744391e-08
4619 5.52321601787753e-08
4620 5.51572547635715e-08
4621 5.52059589153941e-08
4622 5.52318581981126e-08
4623 5.51442589369344e-08
4624 5.52058878611206e-08
4625 5.51391430292369e-08
4626 5.51579049101747e-08
4627 5.52119985286481e-08
4628 5.51266587933696e-08
4629 5.51853247543477e-08
4630 5.51681651472791e-08
4631 5.51550378702359e-08
4632 5.51740448884175e-08
4633 5.51684351535187e-08
4634 5.5156430533998e-08
4635 5.51467351783685e-08
4636 5.51404788495802e-08
4637 5.51759669065177e-08
4638 5.51133325643605e-08
4639 5.52003101006449e-08
4640 5.51029728512731e-08
4641 5.51196954745592e-08
4642 5.52116254937118e-08
4643 5.50988943359698e-08
4644 5.5144131039242e-08
4645 5.51344179200441e-08
4646 5.51010721494549e-08
4647 5.5158476897077e-08
4648 5.51280976424096e-08
4649 5.51292735906372e-08
4650 5.50870069560006e-08
4651 5.51535705994866e-08
4652 5.50747714100908e-08
4653 5.50815251187942e-08
4654 5.51386598601766e-08
4655 5.51123768843809e-08
4656 5.51189600628277e-08
4657 5.50755530071001e-08
4658 5.51467422837959e-08
4659 5.50674172927756e-08
4660 5.50566525703289e-08
4661 5.5125383369159e-08
4662 5.51076588806154e-08
4663 5.50862928605511e-08
4664 5.50848184843744e-08
4665 5.50420722333911e-08
4666 5.50924603714975e-08
4667 5.51300267659371e-08
4668 5.50334746662884e-08
4669 5.50894974082894e-08
4670 5.507219213996e-08
4671 5.50686429789948e-08
4672 5.51088525924115e-08
4673 5.50254455333743e-08
4674 5.50700995916031e-08
4675 5.50603935778327e-08
4676 5.50784768904578e-08
4677 5.50117498221425e-08
4678 5.50159597878519e-08
4679 5.50809673427466e-08
4680 5.50708811886125e-08
4681 5.50415499844803e-08
4682 5.49946683747748e-08
4683 5.50427472489901e-08
4684 5.50395142795423e-08
4685 5.50366294760352e-08
4686 5.50233529850175e-08
4687 5.50708989521809e-08
4688 5.50007150934562e-08
4689 5.50299183998959e-08
4690 5.50174661384517e-08
4691 5.5024575118523e-08
4692 5.49751604239646e-08
4693 5.49909096037027e-08
4694 5.50214345196309e-08
4695 5.51496448508715e-08
4696 5.50130749843447e-08
4697 5.50384200437293e-08
4698 5.49639089797438e-08
4699 5.49888525824827e-08
4700 5.50075398564331e-08
4701 5.4990518805198e-08
4702 5.50057031034612e-08
4703 5.49623173640157e-08
4704 5.50526415565855e-08
4705 5.49559224793938e-08
4706 5.50009247035632e-08
4707 5.49890728507307e-08
4708 5.50135936805418e-08
4709 5.49971623797774e-08
4710 5.49965335494562e-08
4711 5.49533929472545e-08
4712 5.50098633311791e-08
4713 5.49699628038525e-08
4714 5.49921743697723e-08
4715 5.49863479193391e-08
4716 5.50268026699996e-08
4717 5.49351675260823e-08
4718 5.49857119835906e-08
4719 5.50073941951723e-08
4720 5.49616316902757e-08
4721 5.49528635929164e-08
4722 5.49931300497519e-08
4723 5.49987007048003e-08
4724 5.49307053177017e-08
4725 5.50165246693268e-08
4726 5.49221574885905e-08
4727 5.49820313722194e-08
4728 5.49597487520259e-08
4729 5.49679022299188e-08
4730 5.49299095098377e-08
4731 5.49975283092863e-08
4732 5.49325847032378e-08
4733 5.502965194637e-08
4734 5.50062750903635e-08
4735 5.49677032779528e-08
4736 5.49935847971028e-08
4737 5.49110907854811e-08
4738 5.49621432810454e-08
4739 5.49425536178205e-08
4740 5.49929204396449e-08
4741 5.49041239139569e-08
4742 5.49984662256975e-08
4743 5.48903145158874e-08
4744 5.49529310944763e-08
4745 5.48938459132842e-08
4746 5.49644738612187e-08
4747 5.49208465372431e-08
4748 5.49195178223272e-08
4749 5.49157306295456e-08
4750 5.49184626663646e-08
4751 5.49343610600772e-08
4752 5.48805765276938e-08
4753 5.49380523295895e-08
4754 5.48838450242783e-08
4755 5.49835164065371e-08
4756 5.48641772013525e-08
4757 5.490521814977e-08
4758 5.49310605890696e-08
4759 5.4909985891527e-08
4760 5.48946026412978e-08
4761 5.49058363219501e-08
4762 5.48933378752281e-08
4763 5.48457990134921e-08
4764 5.49111618397546e-08
4765 5.48448539916535e-08
4766 5.49751604239646e-08
4767 5.48855716431262e-08
4768 5.49000205296579e-08
4769 5.50315277791924e-08
4770 5.48907550523836e-08
4771 5.49184271392278e-08
4772 5.48892806762069e-08
4773 5.48946488265756e-08
4774 5.48925385146504e-08
4775 5.49257208604104e-08
4776 5.48876641914831e-08
4777 5.48509717646084e-08
4778 5.49155423357206e-08
4779 5.48632037578045e-08
4780 5.48849818926556e-08
4781 5.4927870252186e-08
4782 5.48848255732537e-08
4783 5.48750485052096e-08
4784 5.49144196781981e-08
4785 5.48560166180323e-08
4786 5.48656018395377e-08
4787 5.48758478657874e-08
4788 5.4859338405322e-08
4789 5.48710161751842e-08
4790 5.49043903674828e-08
4791 5.48750485052096e-08
4792 5.48591572169244e-08
4793 5.49003118521796e-08
4794 5.48481580153748e-08
4795 5.48556151613866e-08
4796 5.49259375759448e-08
4797 5.48349845530538e-08
4798 5.48510818987324e-08
4799 5.48999175009612e-08
4800 5.48499130559321e-08
4801 5.48389103016689e-08
4802 5.48780434428409e-08
4803 5.48367324881838e-08
4804 5.48374785580563e-08
4805 5.48695844315716e-08
4806 5.48218856977201e-08
4807 5.48399903266272e-08
4808 5.48574234926491e-08
4809 5.48410561407309e-08
4810 5.48243441755858e-08
4811 5.48464846872321e-08
4812 5.4821775563596e-08
4813 5.48273497713581e-08
4814 5.4889014222681e-08
4815 5.48080549833685e-08
4816 5.48016743096014e-08
4817 5.48378267239968e-08
4818 5.48098597619173e-08
4819 5.48453655824233e-08
4820 5.48089644780703e-08
4821 5.48345049367072e-08
4822 5.48467760097537e-08
4823 5.47987646370984e-08
4824 5.4842359986651e-08
4825 5.48350094220496e-08
4826 5.4790241676983e-08
4827 5.47925509408742e-08
4828 5.48420970858388e-08
4829 5.47983063370339e-08
4830 5.48075860251629e-08
4831 5.48293250801635e-08
4832 5.47783045590222e-08
4833 5.48256622323606e-08
4834 5.48130572042282e-08
4835 5.47823191254793e-08
4836 5.47667617922798e-08
4837 5.478009867943e-08
4838 5.48211396278475e-08
4839 5.47656711091804e-08
4840 5.47713341347844e-08
4841 5.47748548740401e-08
4842 5.47918226345701e-08
4843 5.47675185202934e-08
4844 5.47492575719843e-08
4845 5.47678311590971e-08
4846 5.4785662229051e-08
4847 5.47717995402763e-08
4848 5.47470548895035e-08
4849 5.47827099239839e-08
4850 5.47461276312333e-08
4851 5.47457688071518e-08
4852 5.47415659468697e-08
4853 5.47987433208164e-08
4854 5.47334266798316e-08
4855 5.473793507349e-08
4856 5.47713980836306e-08
4857 5.47461986855069e-08
4858 5.47226122193933e-08
4859 5.47450795806981e-08
4860 5.47460246025366e-08
4861 5.47088490066017e-08
4862 5.47313803167526e-08
4863 5.47238023784757e-08
4864 5.47141745244062e-08
4865 5.47186900234919e-08
4866 5.47551479712638e-08
4867 5.47318776966677e-08
4868 5.4700372231764e-08
4869 5.47660192751209e-08
4870 5.46946452573138e-08
4871 5.47100604819661e-08
4872 5.47093286229483e-08
4873 5.47348584234442e-08
4874 5.47334977341052e-08
4875 5.47119682892117e-08
4876 5.46937002354753e-08
4877 5.47310854415173e-08
4878 5.46819372004848e-08
4879 5.46925171818202e-08
4880 5.46911955723317e-08
4881 5.47644525283886e-08
4882 5.46736984574636e-08
4883 5.46807434886887e-08
4884 5.46813652135825e-08
4885 5.47108669479712e-08
4886 5.46680460900006e-08
4887 5.46740999141093e-08
4888 5.46694671754722e-08
4889 5.47215002200119e-08
4890 5.46589689065513e-08
4891 5.46677974000431e-08
4892 5.46721885541501e-08
4893 5.47367697834034e-08
4894 5.46615375185411e-08
4895 5.46656053757033e-08
4896 5.46954019853274e-08
4897 5.46746470320159e-08
4898 5.46449498983748e-08
4899 5.46938530021635e-08
4900 5.46520659838734e-08
4901 5.46509788534877e-08
4902 5.46466338846585e-08
4903 5.46770380083217e-08
4904 5.4633222390521e-08
4905 5.4636718260781e-08
4906 5.46393223999075e-08
4907 5.46700498205155e-08
4908 5.46296128334234e-08
4909 5.46271685664124e-08
4910 5.46104104159895e-08
4911 5.46488827524172e-08
4912 5.45772564919389e-08
4913 5.45595035816859e-08
4914 5.45607719004693e-08
4915 5.45833565013254e-08
4916 5.45330216539242e-08
4917 5.4512373282023e-08
4918 5.44855751627438e-08
4919 5.44506519872812e-08
4920 5.44242979572118e-08
4921 5.44284688430707e-08
4922 5.43815588116559e-08
4923 5.44084777232001e-08
4924 5.44164713289774e-08
4925 5.4348628708567e-08
4926 5.4338563870715e-08
4927 5.43527356455797e-08
4928 5.43218554582836e-08
4929 5.43123555019065e-08
4930 5.43337215219708e-08
4931 5.42784057699919e-08
4932 5.42662697000651e-08
4933 5.42990719054615e-08
4934 5.42988551899271e-08
4935 5.42436389139311e-08
4936 5.42403064685004e-08
4937 5.42981624107597e-08
4938 5.42460192320959e-08
4939 5.42255520485924e-08
4940 5.42551816806736e-08
4941 5.42476570331019e-08
4942 5.42097460254354e-08
4943 5.42127907010581e-08
4944 5.42459375196813e-08
4945 5.42266427316918e-08
4946 5.41980433865774e-08
4947 5.42078453236172e-08
4948 5.42634488454041e-08
4949 5.41880211812895e-08
4950 5.41868629966302e-08
4951 5.41997415837159e-08
4952 5.4219206901962e-08
4953 5.4247511371841e-08
4954 5.41690674538131e-08
4955 5.4184265962931e-08
4956 5.4226507728572e-08
4957 5.41712843471487e-08
4958 5.41577023227546e-08
4959 5.4194220666659e-08
4960 5.41548708099526e-08
4961 5.41435305478899e-08
4962 5.41550981836281e-08
4963 5.41565121636722e-08
4964 5.41652518393221e-08
4965 5.41383684549146e-08
4966 5.41566862466425e-08
4967 5.41932010378332e-08
4968 5.41558122790775e-08
4969 5.41320943625578e-08
4970 5.41276143906089e-08
4971 5.42028644190395e-08
4972 5.412238124336e-08
4973 5.41168212464527e-08
4974 5.41690745592405e-08
4975 5.41374660656402e-08
4976 5.41657989572286e-08
4977 5.4111541913926e-08
4978 5.41525047026425e-08
4979 5.41507105822348e-08
4980 5.40939346649338e-08
4981 5.40974411933348e-08
4982 5.41308686763387e-08
4983 5.40789848457734e-08
4984 5.40818092531481e-08
4985 5.407747494246e-08
4986 5.40857669761863e-08
4987 5.42162368333265e-08
4988 5.40749063304702e-08
4989 5.40859659281523e-08
4990 5.41207505477814e-08
4991 5.40694742312553e-08
4992 5.40690443529002e-08
4993 5.40865521259093e-08
4994 5.40700213491618e-08
4995 5.40653424252469e-08
4996 5.4093479917583e-08
4997 5.40493374501239e-08
4998 5.40944249394215e-08
4999 5.40521405412164e-08
};
\addlegendentry{Test}

\nextgroupplot[
title={Batch Size 16 $\hy$},
ymin=8.94702000868848e-09, ymax=1e-05,
]
\addplot [semithick, black, dashed]
table {%
0 0.0137274864278734
1 0.00431716263480485
2 0.00237909571127966
3 0.00191532298689708
4 0.00154156649811193
5 0.00108411967055872
6 0.000676903979619965
7 0.000396826002950547
8 0.000232986924573197
9 0.000187109086960845
10 0.000164858861964603
11 0.000143591264066345
12 0.000120276718764217
13 9.62891611416126e-05
14 7.38634760673449e-05
15 5.4657348740875e-05
16 3.99050011274085e-05
17 2.99829872355986e-05
18 2.40958482354472e-05
19 2.09019726653423e-05
20 1.92171714461438e-05
21 1.82826136579024e-05
22 1.76928175815192e-05
23 1.72516784705294e-05
24 1.68720731517169e-05
25 1.65159503067116e-05
26 1.61647211180025e-05
27 1.58031787113941e-05
28 1.53544017402965e-05
29 1.46881677196689e-05
30 1.41518554237337e-05
31 1.36414861763114e-05
32 1.31194685395712e-05
33 1.25778681981501e-05
34 1.20086648821598e-05
35 1.14099147699562e-05
36 1.07847172662332e-05
37 1.01333905067804e-05
38 9.46252704397921e-06
39 8.78533366176271e-06
40 8.10874567059727e-06
41 7.44334617684217e-06
42 6.79567157317251e-06
43 6.17321029494633e-06
44 5.58164971243968e-06
45 5.02565300666902e-06
46 4.50506579477405e-06
47 4.02443795405816e-06
48 3.58429506638913e-06
49 3.1847501811626e-06
50 2.81504779763964e-06
51 2.46258455740644e-06
52 2.13124292270095e-06
53 1.82635500362949e-06
54 1.56813208531048e-06
55 1.36048875771166e-06
56 1.20179831310452e-06
57 1.08225343844026e-06
58 9.92854343692784e-07
59 9.23994277485463e-07
60 8.69172093516113e-07
61 8.24356094838663e-07
62 7.92375916802257e-07
63 7.66183173325885e-07
64 7.4362045631915e-07
65 7.23852290057891e-07
66 7.06617725938941e-07
67 6.91471299958835e-07
68 6.77673051974637e-07
69 6.6496880113931e-07
70 6.53162623592607e-07
71 6.42082102729091e-07
72 6.31171595529167e-07
73 6.20630909210718e-07
74 6.10587335700075e-07
75 6.00901989884051e-07
76 5.9159817558907e-07
77 5.82503811884294e-07
78 5.73879205461481e-07
79 5.65155023252828e-07
80 5.57031663404928e-07
81 5.49146342976314e-07
82 5.41442691741167e-07
83 5.34300989727399e-07
84 5.27284182211929e-07
85 5.20931869658625e-07
86 5.14867203364133e-07
87 5.09007935605155e-07
88 5.03304501989987e-07
89 4.97759342934501e-07
90 4.91949098062605e-07
91 4.8547303703117e-07
92 4.79309857652765e-07
93 4.73423336373457e-07
94 4.67343124839204e-07
95 4.61478271105875e-07
96 4.55039506363164e-07
97 4.4850045082967e-07
98 4.41832408540677e-07
99 4.35164161146417e-07
100 4.2696416851129e-07
101 4.18180375262978e-07
102 4.12271174496936e-07
103 4.06990096266213e-07
104 4.02601702916172e-07
105 3.98553719904271e-07
106 3.94868042320695e-07
107 3.91282630616274e-07
108 3.88000465079585e-07
109 3.84798651708707e-07
110 3.81782116306795e-07
111 3.79011431391518e-07
112 3.76296654238217e-07
113 3.732706894084e-07
114 3.70572445802964e-07
115 3.68032981711508e-07
116 3.65621566402297e-07
117 3.63273273976006e-07
118 3.61050667507357e-07
119 3.58790809556808e-07
120 3.56689573621338e-07
121 3.54660646934235e-07
122 3.52744051937748e-07
123 3.50798695947674e-07
124 3.48914919456433e-07
125 3.47108600905699e-07
126 3.45365146372956e-07
127 3.43657022057187e-07
128 3.41924667054627e-07
129 3.40357991504447e-07
130 3.38683575577647e-07
131 3.37194948571096e-07
132 3.35731513729343e-07
133 3.34273990517886e-07
134 3.32925119664651e-07
135 3.31498091156845e-07
136 3.30187419422145e-07
137 3.28824048409615e-07
138 3.27565241903471e-07
139 3.2639127208256e-07
140 3.25208446156466e-07
141 3.24096659497286e-07
142 3.23045156349622e-07
143 3.22011035891023e-07
144 3.21090873470098e-07
145 3.20067961567361e-07
146 3.19142574056741e-07
147 3.18195014372691e-07
148 3.17311305593648e-07
149 3.1647693113257e-07
150 3.15708772987477e-07
151 3.14911853990907e-07
152 3.1410559499534e-07
153 3.13397518084457e-07
154 3.12632867348839e-07
155 3.1194274735924e-07
156 3.11181787566284e-07
157 3.10667926868291e-07
158 3.09888446054174e-07
159 3.09367080276957e-07
160 3.08626810507917e-07
161 3.08027277682754e-07
162 3.07564994400877e-07
163 3.06857924115889e-07
164 3.06245265292659e-07
165 3.05703405203417e-07
166 3.05161258339126e-07
167 3.04688700353495e-07
168 3.0412783806355e-07
169 3.03579181661462e-07
170 3.03093525275244e-07
171 3.02564091782642e-07
172 3.02010650216289e-07
173 3.01520979078873e-07
174 3.00970691924363e-07
175 3.00514174369937e-07
176 3.00070867695013e-07
177 2.99607086859055e-07
178 2.99238432944549e-07
179 2.98759716372388e-07
180 2.98355265542227e-07
181 2.97854386786867e-07
182 2.97416586064969e-07
183 2.9697725268818e-07
184 2.96535195218439e-07
185 2.96018159559708e-07
186 2.95601901470377e-07
187 2.95166878927944e-07
188 2.94741055171244e-07
189 2.94315570897652e-07
190 2.93867545892113e-07
191 2.93399155950169e-07
192 2.92967971859071e-07
193 2.92557831500062e-07
194 2.92133439558029e-07
195 2.917087665395e-07
196 2.91305933430408e-07
197 2.90897576491034e-07
198 2.90438663313353e-07
199 2.90020287366133e-07
200 2.89613250188836e-07
201 2.89229411492897e-07
202 2.88804448047131e-07
203 2.88387472934915e-07
204 2.87979536039984e-07
205 2.87603965233529e-07
206 2.87214215418885e-07
207 2.86755049245357e-07
208 2.86378200357262e-07
209 2.85903206261651e-07
210 2.85568181311646e-07
211 2.85175317515041e-07
212 2.84806623028544e-07
213 2.84412818473356e-07
214 2.84001570435066e-07
215 2.83667169199475e-07
216 2.83282086599002e-07
217 2.82900169793265e-07
218 2.82576118209477e-07
219 2.82183548094395e-07
220 2.81888060570168e-07
221 2.81526506313412e-07
222 2.8125334117135e-07
223 2.80874968950684e-07
224 2.80538377339212e-07
225 2.80214264250844e-07
226 2.79937651747275e-07
227 2.79590258770668e-07
228 2.79379765963483e-07
229 2.79104436010869e-07
230 2.78802025547975e-07
231 2.78451185984352e-07
232 2.78202562370211e-07
233 2.77849966387578e-07
234 2.77560876995153e-07
235 2.77251104883192e-07
236 2.76969541012306e-07
237 2.76746430365904e-07
238 2.76661280459223e-07
239 2.76352805130387e-07
240 2.76155963000235e-07
241 2.75729903101762e-07
242 2.75579072038568e-07
243 2.75319795228768e-07
244 2.75069527212679e-07
245 2.74871734276871e-07
246 2.7456078626642e-07
247 2.74342937409244e-07
248 2.74080819039568e-07
249 2.7385803708313e-07
250 2.73675317856714e-07
251 2.7333746044178e-07
252 2.73112856191915e-07
253 2.72933954256871e-07
254 2.72693824541648e-07
255 2.72450924747147e-07
256 2.72217675167496e-07
257 2.71940476785915e-07
258 2.71618396148199e-07
259 2.71509369532907e-07
260 2.71273398880112e-07
261 2.70997184323107e-07
262 2.70695426820566e-07
263 2.70517067860965e-07
264 2.70293748393158e-07
265 2.70061954473988e-07
266 2.69826762902881e-07
267 2.69588034157664e-07
268 2.69347646288054e-07
269 2.69120242975873e-07
270 2.68942758886226e-07
271 2.68721411885053e-07
272 2.68501216503125e-07
273 2.68274119150647e-07
274 2.6802078957644e-07
275 2.6782561451455e-07
276 2.67596137497605e-07
277 2.67371704140373e-07
278 2.67140124982745e-07
279 2.66873521624689e-07
280 2.66664064071165e-07
281 2.66496327363086e-07
282 2.66267156462163e-07
283 2.6605362319998e-07
284 2.65830469707851e-07
285 2.65593565849542e-07
286 2.653637746306e-07
287 2.6513492757374e-07
288 2.64938284004756e-07
289 2.64691264533212e-07
290 2.64449259354649e-07
291 2.64184095961184e-07
292 2.63969086724103e-07
293 2.63589465248515e-07
294 2.63513477086974e-07
295 2.63184247749848e-07
296 2.62962396888611e-07
297 2.62630460348134e-07
298 2.62415228888813e-07
299 2.62186799417918e-07
300 2.61983339783001e-07
301 2.61745508986166e-07
302 2.6149418273036e-07
303 2.61293342404656e-07
304 2.61099690739286e-07
305 2.60862154235042e-07
306 2.60627848824413e-07
307 2.60399447789439e-07
308 2.60180488432127e-07
309 2.59960263072401e-07
310 2.59731980882805e-07
311 2.59555470627504e-07
312 2.59321352167774e-07
313 2.59089444192284e-07
314 2.58859883118134e-07
315 2.58611288252553e-07
316 2.58357919655339e-07
317 2.58131841178511e-07
318 2.57908193162848e-07
319 2.57747713305889e-07
320 2.57520690610136e-07
321 2.57292554294963e-07
322 2.57071540517018e-07
323 2.56520201169508e-07
324 2.56287236716446e-07
325 2.56038774239187e-07
326 2.55784327165998e-07
327 2.55579916888848e-07
328 2.55302003282054e-07
329 2.55057509562562e-07
330 2.54925153548413e-07
331 2.5467535632373e-07
332 2.54447703539995e-07
333 2.54177683260082e-07
334 2.5391945712272e-07
335 2.53650146085249e-07
336 2.53437346323437e-07
337 2.53177861992526e-07
338 2.52891544768374e-07
339 2.52621393322272e-07
340 2.52337953597248e-07
341 2.52081404781279e-07
342 2.51871218253541e-07
343 2.51609448596923e-07
344 2.51361516085069e-07
345 2.51107598991496e-07
346 2.50856397251198e-07
347 2.50611124386069e-07
348 2.50349162364216e-07
349 2.50098037220425e-07
350 2.49845405988935e-07
351 2.49592483868355e-07
352 2.49340479371085e-07
353 2.49093577252779e-07
354 2.48848891352793e-07
355 2.48584423346188e-07
356 2.48334369700842e-07
357 2.48084361381018e-07
358 2.47827622551711e-07
359 2.4757746955828e-07
360 2.47326871416931e-07
361 2.47062957484445e-07
362 2.46816356565205e-07
363 2.46553932910842e-07
364 2.46299302681052e-07
365 2.46043571365817e-07
366 2.45799427958104e-07
367 2.45527353897046e-07
368 2.45283082421111e-07
369 2.4500974530639e-07
370 2.44762084250283e-07
371 2.44490831569522e-07
372 2.44247502799055e-07
373 2.4397645023555e-07
374 2.43722680870917e-07
375 2.43438445664879e-07
376 2.43197867760614e-07
377 2.42933668758383e-07
378 2.42697593499486e-07
379 2.42415532795803e-07
380 2.42148812027665e-07
381 2.41883608239846e-07
382 2.41626759439839e-07
383 2.41372030622244e-07
384 2.41115252961777e-07
385 2.40840137742282e-07
386 2.40577587419466e-07
387 2.40459545388205e-07
388 2.40167345566533e-07
389 2.39913366975486e-07
390 2.3961788229343e-07
391 2.39307461825433e-07
392 2.3904169699307e-07
393 2.38694678472484e-07
394 2.38425203875181e-07
395 2.38175939685448e-07
396 2.37906621748607e-07
397 2.37618422673336e-07
398 2.37336836988788e-07
399 2.3706685819036e-07
400 2.36774036387999e-07
401 2.36513957801776e-07
402 2.36237759757785e-07
403 2.35911934723276e-07
404 2.35639368156626e-07
405 2.35348217053399e-07
406 2.35077083580393e-07
407 2.34801767504678e-07
408 2.34523840624945e-07
409 2.3422835146647e-07
410 2.33954849441886e-07
411 2.3366218783849e-07
412 2.33389521859806e-07
413 2.3310327888737e-07
414 2.32825963514927e-07
415 2.32534337115453e-07
416 2.32266378525026e-07
417 2.31968768723334e-07
418 2.31701812758445e-07
419 2.31409488300471e-07
420 2.3113461271862e-07
421 2.30846515833605e-07
422 2.30578267888859e-07
423 2.30291306934305e-07
424 2.30015279989004e-07
425 2.29701958687656e-07
426 2.29434683930663e-07
427 2.29129784464988e-07
428 2.2884633944642e-07
429 2.28565032081463e-07
430 2.28260833097238e-07
431 2.27989829845399e-07
432 2.2768157769093e-07
433 2.27410900038194e-07
434 2.27123069002744e-07
435 2.26837973059446e-07
436 2.2652846216431e-07
437 2.26265045945695e-07
438 2.25923534884487e-07
439 2.25688729209139e-07
440 2.25343421455193e-07
441 2.25088106354576e-07
442 2.24743013042428e-07
443 2.24499334372297e-07
444 2.24148286179116e-07
445 2.23870552389371e-07
446 2.23554672459159e-07
447 2.23279796024656e-07
448 2.22963624928241e-07
449 2.22683038714422e-07
450 2.22365046205653e-07
451 2.22088719603164e-07
452 2.21779675086964e-07
453 2.21543707169758e-07
454 2.21182618567184e-07
455 2.20947558005946e-07
456 2.20548797500442e-07
457 2.20269817212682e-07
458 2.20079597163192e-07
459 2.19676413856007e-07
460 2.19348591677715e-07
461 2.19044931746737e-07
462 2.18794459030391e-07
463 2.18426107167602e-07
464 2.18113793799546e-07
465 2.17868971873258e-07
466 2.175677625047e-07
467 2.17192505900243e-07
468 2.16869971410461e-07
469 2.16647192054609e-07
470 2.16269524834445e-07
471 2.15967567555708e-07
472 2.15724776865045e-07
473 2.15459875882118e-07
474 2.15140661879332e-07
475 2.14823481648807e-07
476 2.1450611591689e-07
477 2.14190365113609e-07
478 2.13872519360336e-07
479 2.13555062650528e-07
480 2.13246725550675e-07
481 2.1291680879898e-07
482 2.12607096621298e-07
483 2.12282845197365e-07
484 2.11963971963769e-07
485 2.1162596515012e-07
486 2.11305132658879e-07
487 2.10978212393798e-07
488 2.10600347720913e-07
489 2.10341784736556e-07
490 2.10002618281635e-07
491 2.09641964346474e-07
492 2.09355287360324e-07
493 2.09031046907171e-07
494 2.08660130297744e-07
495 2.08371932941986e-07
496 2.0804793426521e-07
497 2.07717599700175e-07
498 2.07301362628698e-07
499 2.07020214197939e-07
500 2.0672976783942e-07
501 2.06351586442111e-07
502 2.06083099591581e-07
503 2.05735360083281e-07
504 2.05293517034022e-07
505 2.05025740712017e-07
506 2.0471952664991e-07
507 2.04331938839175e-07
508 2.03999870329596e-07
509 2.0372727497886e-07
510 2.03368903690659e-07
511 2.03003240777377e-07
512 2.02632826166393e-07
513 2.02301128439331e-07
514 2.01985279716155e-07
515 2.01664773683774e-07
516 2.01300083247702e-07
517 2.0093440554092e-07
518 2.00611610260637e-07
519 2.0024500553717e-07
520 1.99918972136004e-07
521 1.99551662518616e-07
522 1.99226040038525e-07
523 1.98823360726408e-07
524 1.98527554069017e-07
525 1.98143583844512e-07
526 1.97833583833074e-07
527 1.9742047673077e-07
528 1.97087421476283e-07
529 1.96771742196233e-07
530 1.96372767042874e-07
531 1.96061845613826e-07
532 1.95639964040595e-07
533 1.95292495007493e-07
534 1.94933018292431e-07
535 1.94579103094839e-07
536 1.94211533198541e-07
537 1.93829868663897e-07
538 1.93507120322067e-07
539 1.93110418045705e-07
540 1.92797984240656e-07
541 1.92383654251671e-07
542 1.92063353246397e-07
543 1.91688187470618e-07
544 1.91292904723639e-07
545 1.90943842184765e-07
546 1.90607013848876e-07
547 1.90196977143842e-07
548 1.89828616115051e-07
549 1.89447352646255e-07
550 1.89066584020736e-07
551 1.8862402483677e-07
552 1.88243204441108e-07
553 1.87870924136746e-07
554 1.87504632336299e-07
555 1.87108488738374e-07
556 1.86729613581349e-07
557 1.86366781257163e-07
558 1.8597400960374e-07
559 1.85609028655165e-07
560 1.85232912137678e-07
561 1.84851388091545e-07
562 1.84470041808993e-07
563 1.8408775462575e-07
564 1.83707683582668e-07
565 1.83325843366333e-07
566 1.82943164254823e-07
567 1.8255715423976e-07
568 1.82144692757902e-07
569 1.8182059964289e-07
570 1.81434204911568e-07
571 1.81044943438735e-07
572 1.80667015477809e-07
573 1.80271112249386e-07
574 1.79889306757275e-07
575 1.79495509897265e-07
576 1.79112843603946e-07
577 1.78716367152276e-07
578 1.78333767976824e-07
579 1.77937752567914e-07
580 1.77550581362595e-07
581 1.77157570270481e-07
582 1.76767578842885e-07
583 1.76379540398841e-07
584 1.75985049210681e-07
585 1.75592541985736e-07
586 1.75198777526475e-07
587 1.74818108234831e-07
588 1.74410169520911e-07
589 1.74029001762221e-07
590 1.73626839405472e-07
591 1.73220435620181e-07
592 1.72810401373624e-07
593 1.72441349910457e-07
594 1.72019691135006e-07
595 1.71650002037893e-07
596 1.71289723859047e-07
597 1.70812198106773e-07
598 1.70440799841742e-07
599 1.7003471577226e-07
600 1.69638359395208e-07
601 1.69233831883275e-07
602 1.68878835253849e-07
603 1.6846000008286e-07
604 1.68099666609578e-07
605 1.67579318116395e-07
606 1.67241923072936e-07
607 1.66849342036812e-07
608 1.66446142337406e-07
609 1.66043749487699e-07
610 1.65646255027241e-07
611 1.65243891366629e-07
612 1.64860760179408e-07
613 1.64483821578187e-07
614 1.64021702893535e-07
615 1.63627930859889e-07
616 1.63208522778291e-07
617 1.6278969756911e-07
618 1.62398115541862e-07
619 1.61975294069805e-07
620 1.61592992576232e-07
621 1.61194903604667e-07
622 1.60802261532922e-07
623 1.60380912582525e-07
624 1.59956159045294e-07
625 1.59575369771403e-07
626 1.59171893642451e-07
627 1.5878636067157e-07
628 1.58358158550698e-07
629 1.57943072366606e-07
630 1.57573667422639e-07
631 1.57154976484719e-07
632 1.56737598842938e-07
633 1.56340774388752e-07
634 1.55933278854548e-07
635 1.55543305957906e-07
636 1.55144518451777e-07
637 1.5473013903744e-07
638 1.54325522380816e-07
639 1.53943266042234e-07
640 1.53544567787378e-07
641 1.53135508796254e-07
642 1.52730140023039e-07
643 1.52351036305731e-07
644 1.51938583535127e-07
645 1.5153900105247e-07
646 1.51150738254557e-07
647 1.50658691872252e-07
648 1.50388736244622e-07
649 1.4997760697355e-07
650 1.49451682588619e-07
651 1.49178036814135e-07
652 1.48714045323572e-07
653 1.48404624326304e-07
654 1.47904955532852e-07
655 1.47575316397308e-07
656 1.4723739888467e-07
657 1.46710725296373e-07
658 1.46355089498229e-07
659 1.45961519834259e-07
660 1.45580722467287e-07
661 1.4516522767849e-07
662 1.44788464659484e-07
663 1.44402652921372e-07
664 1.43987570346837e-07
665 1.43677572779666e-07
666 1.4319869436008e-07
667 1.42818058485261e-07
668 1.4246067030399e-07
669 1.42053977228329e-07
670 1.41678787990429e-07
671 1.41298572586379e-07
672 1.40913777080698e-07
673 1.4051466420284e-07
674 1.40155477247106e-07
675 1.39751706470292e-07
676 1.39448633909467e-07
677 1.39069091218857e-07
678 1.38717742466099e-07
679 1.38276971121343e-07
680 1.37922032735105e-07
681 1.37567378160952e-07
682 1.37199445774172e-07
683 1.36811556700422e-07
684 1.36458982169074e-07
685 1.36097112672928e-07
686 1.35732451262527e-07
687 1.35339461316164e-07
688 1.35015707456887e-07
689 1.34619066386676e-07
690 1.34241309424965e-07
691 1.33902624035898e-07
692 1.33571713824665e-07
693 1.33215046446367e-07
694 1.32862068788597e-07
695 1.32498647211321e-07
696 1.32133106070853e-07
697 1.31815109867262e-07
698 1.31431751825062e-07
699 1.31097048246431e-07
700 1.30753649102644e-07
701 1.30387444869484e-07
702 1.30044283999098e-07
703 1.29685929771028e-07
704 1.29344457150182e-07
705 1.28963766677259e-07
706 1.28634338473432e-07
707 1.28298173763142e-07
708 1.27903959530329e-07
709 1.27560316094844e-07
710 1.27259029902405e-07
711 1.26916074435712e-07
712 1.26570836428641e-07
713 1.26242448089897e-07
714 1.25942857522432e-07
715 1.25658145954333e-07
716 1.25333819035234e-07
717 1.24976995756754e-07
718 1.2469927335701e-07
719 1.24291260057419e-07
720 1.23951656647137e-07
721 1.23588335185332e-07
722 1.23354299191192e-07
723 1.2305910463084e-07
724 1.22795112680762e-07
725 1.22407181045503e-07
726 1.22122077300446e-07
727 1.21692449788213e-07
728 1.21404389872737e-07
729 1.21121971183413e-07
730 1.20819145841722e-07
731 1.20456011423187e-07
732 1.20153513222476e-07
733 1.19856491068049e-07
734 1.19532518908727e-07
735 1.19183330937744e-07
736 1.18885924194956e-07
737 1.18580042972383e-07
738 1.18395574688179e-07
739 1.18002021430641e-07
740 1.17733016708144e-07
741 1.17388062562895e-07
742 1.17087116208836e-07
743 1.16862637476345e-07
744 1.16545521848366e-07
745 1.16176293573744e-07
746 1.15960495627121e-07
747 1.15727445198388e-07
748 1.15473025743995e-07
749 1.15104726525317e-07
750 1.14855429785621e-07
751 1.14566576755237e-07
752 1.14390812825604e-07
753 1.14053330420916e-07
754 1.13753872234668e-07
755 1.13467354871943e-07
756 1.13200909467537e-07
757 1.12911044645614e-07
758 1.12825508487902e-07
759 1.12508793307597e-07
760 1.12130060003324e-07
761 1.11889274812427e-07
762 1.11746402396307e-07
763 1.11409739488977e-07
764 1.1113746469249e-07
765 1.10985593764212e-07
766 1.10618337537005e-07
767 1.10366557230179e-07
768 1.10274462819859e-07
769 1.0984845675921e-07
770 1.09624351722459e-07
771 1.09413163855265e-07
772 1.09173802790963e-07
773 1.08933405329026e-07
774 1.08688462560735e-07
775 1.08423943650848e-07
776 1.08171369209487e-07
777 1.07934555177991e-07
778 1.07738759460574e-07
779 1.07583577960924e-07
780 1.07309208402739e-07
781 1.0709189565361e-07
782 1.06854206709528e-07
783 1.06668577831925e-07
784 1.06425021595413e-07
785 1.06183751533706e-07
786 1.06047607317805e-07
787 1.05866056038906e-07
788 1.05667148588395e-07
789 1.05310264618907e-07
790 1.05179661801458e-07
791 1.04921896678434e-07
792 1.04701653388162e-07
793 1.0448816604125e-07
794 1.04317877553939e-07
795 1.04099020152404e-07
796 1.03901724401112e-07
797 1.0372642863743e-07
798 1.0344589897926e-07
799 1.033051914483e-07
800 1.03147710589724e-07
801 1.02949534703356e-07
802 1.02780997107033e-07
803 1.02574349000406e-07
804 1.02439439505275e-07
805 1.02193342321755e-07
806 1.0199855167059e-07
807 1.01915736067326e-07
808 1.01622178380723e-07
809 1.01485988945882e-07
810 1.01289536253546e-07
811 1.01103692689009e-07
812 1.00894708303656e-07
813 1.00720193952242e-07
814 1.00607804544239e-07
815 1.00366717195044e-07
816 1.00201602901251e-07
817 1.00024840651969e-07
818 9.97699724898382e-08
819 9.95739786269212e-08
820 9.94168200421086e-08
821 9.9221524354931e-08
822 9.90574403978428e-08
823 9.89885481139652e-08
824 9.88177434564363e-08
825 9.85936399189313e-08
826 9.84379009167924e-08
827 9.8297320015206e-08
828 9.8205699806897e-08
829 9.79757726433661e-08
830 9.78314861370677e-08
831 9.76398887999608e-08
832 9.74578103374313e-08
833 9.72439887299004e-08
834 9.71489310224172e-08
835 9.6969024440341e-08
836 9.68111361423496e-08
837 9.66485084887836e-08
838 9.64936330944965e-08
839 9.634713987694e-08
840 9.62108183379939e-08
841 9.60588867897627e-08
842 9.59235828084104e-08
843 9.57559212970693e-08
844 9.56219040055828e-08
845 9.55613572060088e-08
846 9.53464897897049e-08
847 9.52728056056174e-08
848 9.51704371523476e-08
849 9.49775713330325e-08
850 9.48334547921092e-08
851 9.46312326881582e-08
852 9.45444783084781e-08
853 9.44292809137437e-08
854 9.4364027503957e-08
855 9.41214489635911e-08
856 9.40269046978415e-08
857 9.38575123115015e-08
858 9.3745977142845e-08
859 9.36089745202651e-08
860 9.34789545965486e-08
861 9.34246434809438e-08
862 9.3238003653795e-08
863 9.30721382452759e-08
864 9.29436551544427e-08
865 9.28477345496503e-08
866 9.28251453338191e-08
867 9.26124183777688e-08
868 9.24852978236856e-08
869 9.2357654335018e-08
870 9.21949314403037e-08
871 9.21474649153709e-08
872 9.19663649874281e-08
873 9.19267645862476e-08
874 9.1714904502993e-08
875 9.15880169181094e-08
876 9.14674286853767e-08
877 9.13781838498551e-08
878 9.13099021921937e-08
879 9.11359129780465e-08
880 9.10310626878186e-08
881 9.08934237315862e-08
882 9.08064379672169e-08
883 9.07917099972622e-08
884 9.06575754129335e-08
885 9.05579258869693e-08
886 9.0419134696873e-08
887 9.03300116732453e-08
888 9.02366331345661e-08
889 9.00152758269712e-08
890 9.00693039085354e-08
891 8.98317879070021e-08
892 8.98883477624679e-08
893 8.9576921475043e-08
894 8.94455658730919e-08
895 8.94629070025132e-08
896 8.93330440234763e-08
897 8.93570640734254e-08
898 8.9050069533414e-08
899 8.91171324326479e-08
900 8.88800621190455e-08
901 8.8823233735269e-08
902 8.88037699127153e-08
903 8.86474639649748e-08
904 8.846259999018e-08
905 8.84451073339676e-08
906 8.83092624128778e-08
907 8.82182328290071e-08
908 8.81258327112278e-08
909 8.8063329442889e-08
910 8.78729599591566e-08
911 8.78098459793364e-08
912 8.77080317458478e-08
913 8.76478207381126e-08
914 8.7615152221332e-08
915 8.74033804905139e-08
916 8.74204570031623e-08
917 8.72075509050774e-08
918 8.73203632032471e-08
919 8.69711300488518e-08
920 8.70004266211311e-08
921 8.6928570446787e-08
922 8.68353421878965e-08
923 8.68657824852903e-08
924 8.65399485689977e-08
925 8.65403169250101e-08
926 8.6403141519753e-08
927 8.64528037034518e-08
928 8.62174672526805e-08
929 8.6202224714782e-08
930 8.61174761404015e-08
931 8.60195056731072e-08
932 8.59337276075678e-08
933 8.58748110914576e-08
934 8.58186105432424e-08
935 8.56922882057631e-08
936 8.5608802056214e-08
937 8.55411334796941e-08
938 8.54411283626177e-08
939 8.53523412622792e-08
940 8.53311075097452e-08
941 8.51838940931771e-08
942 8.51054129711315e-08
943 8.50228493796123e-08
944 8.49760510206465e-08
945 8.48726971725e-08
946 8.46609368139184e-08
947 8.47074925083291e-08
948 8.46165065127025e-08
949 8.44173795826464e-08
950 8.44296524356025e-08
951 8.4401982462623e-08
952 8.43235445806556e-08
953 8.41520867425061e-08
954 8.41456795406259e-08
955 8.39612151004587e-08
956 8.40093305676248e-08
957 8.38034040917535e-08
958 8.38335203035001e-08
959 8.37577781531707e-08
960 8.35668080121366e-08
961 8.35376274181954e-08
962 8.34479627087603e-08
963 8.33888383979797e-08
964 8.32384640965245e-08
965 8.31795927958012e-08
966 8.31294540013516e-08
967 8.30709145098751e-08
968 8.29979597725128e-08
969 8.29149942482843e-08
970 8.28349787518334e-08
971 8.27753270158382e-08
972 8.26905002710987e-08
973 8.26221869303367e-08
974 8.25362091383397e-08
975 8.24747581340546e-08
976 8.24094157607647e-08
977 8.23461635057754e-08
978 8.22614503377395e-08
979 8.21755319400097e-08
980 8.20989515908366e-08
981 8.2009273796757e-08
982 8.20041112348235e-08
983 8.19762076638142e-08
984 8.17980200586987e-08
985 8.1835680003195e-08
986 8.17410432034649e-08
987 8.16327636066205e-08
988 8.1560490016841e-08
989 8.15575644210753e-08
990 8.14303888709844e-08
991 8.13206017511448e-08
992 8.12998543899823e-08
993 8.12296466357054e-08
994 8.11347571207932e-08
995 8.10894249454464e-08
996 8.10095008390022e-08
997 8.09707260920334e-08
998 8.08950643715889e-08
999 8.07999656302627e-08
1000 8.07494129340114e-08
1001 8.06435048872345e-08
1002 8.06186811033172e-08
1003 8.05260545035935e-08
1004 8.04333400346025e-08
1005 8.03628120102928e-08
1006 8.02932517913746e-08
1007 8.02248824030016e-08
1008 8.01547034576799e-08
1009 8.00924640138589e-08
1010 8.00176725164192e-08
1011 7.99563234252787e-08
1012 7.98922685838477e-08
1013 7.98292367427678e-08
1014 7.97603862814356e-08
1015 7.96879916826754e-08
1016 7.95915811693249e-08
1017 7.95389130630042e-08
1018 7.94659169969236e-08
1019 7.94096726473015e-08
1020 7.93542128292302e-08
1021 7.93433937928967e-08
1022 7.9250005345699e-08
1023 7.9174605531307e-08
1024 7.91195339679973e-08
1025 7.90526601299746e-08
1026 7.89829632097394e-08
1027 7.89201894981772e-08
1028 7.88666104583058e-08
1029 7.87853403672045e-08
1030 7.87756155489205e-08
1031 7.86651779129954e-08
1032 7.84587692148619e-08
1033 7.85410063599556e-08
1034 7.83606911838319e-08
1035 7.82415667028857e-08
1036 7.81526812669853e-08
1037 7.80228605510302e-08
1038 7.79218334443499e-08
1039 7.78891739958709e-08
1040 7.78916969483845e-08
1041 7.78333876318982e-08
1042 7.77376392750995e-08
1043 7.76782735929515e-08
1044 7.76720546795673e-08
1045 7.75406877835394e-08
1046 7.75270741648626e-08
1047 7.74195809611911e-08
1048 7.73888556491187e-08
1049 7.72601070462287e-08
1050 7.72534418569393e-08
1051 7.7156174143056e-08
1052 7.71229245479788e-08
1053 7.71149362321921e-08
1054 7.70612058165199e-08
1055 7.69981229531425e-08
1056 7.68910978461434e-08
1057 7.67764620146494e-08
1058 7.67320503811675e-08
1059 7.66782150858347e-08
1060 7.66197913755207e-08
1061 7.6504476069772e-08
1062 7.64221657920672e-08
1063 7.64483478672417e-08
1064 7.63451676561999e-08
1065 7.62893861914193e-08
1066 7.62114473342734e-08
1067 7.61547537404539e-08
1068 7.60871494538406e-08
1069 7.60705882925095e-08
1070 7.60207829753057e-08
1071 7.58804837417415e-08
1072 7.58889860570378e-08
1073 7.57842114786911e-08
1074 7.57750083053565e-08
1075 7.5782708581329e-08
1076 7.57290570287949e-08
1077 7.55984567106793e-08
1078 7.55729652706094e-08
1079 7.54949345473932e-08
1080 7.54317630846657e-08
1081 7.54678549430565e-08
1082 7.53737377987562e-08
1083 7.52562606383122e-08
1084 7.53147532464027e-08
1085 7.50915149385634e-08
1086 7.51087142933926e-08
1087 7.50800781901262e-08
1088 7.50327649807758e-08
1089 7.50009233509985e-08
1090 7.49297736106769e-08
1091 7.48229013343149e-08
1092 7.47757772945334e-08
1093 7.47285744306936e-08
1094 7.47108151859521e-08
1095 7.46692460769083e-08
1096 7.46109774265591e-08
1097 7.45898734368922e-08
1098 7.44671453851709e-08
1099 7.44674589494565e-08
1100 7.44247703252654e-08
1101 7.43648627814508e-08
1102 7.43000087002343e-08
1103 7.41735331182269e-08
1104 7.42146372303409e-08
1105 7.41980327028102e-08
1106 7.40671640429014e-08
1107 7.39682569470546e-08
1108 7.40063278374947e-08
1109 7.3914790121421e-08
1110 7.38778579449928e-08
1111 7.38198390113354e-08
1112 7.37711309515277e-08
1113 7.37156285204321e-08
1114 7.36312292133334e-08
1115 7.3603536984379e-08
1116 7.35188671683318e-08
1117 7.34764197147797e-08
1118 7.34664308854605e-08
1119 7.33923400044745e-08
1120 7.33848719960406e-08
1121 7.33478215924066e-08
1122 7.32060440107318e-08
1123 7.32161570393686e-08
1124 7.30840426701462e-08
1125 7.31588005198347e-08
1126 7.30077401982498e-08
1127 7.30516243248047e-08
1128 7.28464096635406e-08
1129 7.29128703707005e-08
1130 7.28162615182271e-08
1131 7.28241790266537e-08
1132 7.28134706999839e-08
1133 7.27467658272474e-08
1134 7.26954738947683e-08
1135 7.26379072446548e-08
1136 7.25905974956476e-08
1137 7.25505777321445e-08
1138 7.2485954895285e-08
1139 7.234630542996e-08
1140 7.24246779082449e-08
1141 7.2371720074571e-08
1142 7.23114512890532e-08
1143 7.22764490479477e-08
1144 7.2227290790039e-08
1145 7.20951982735585e-08
1146 7.21777825791747e-08
1147 7.19547366276885e-08
1148 7.20459417227204e-08
1149 7.19380240994383e-08
1150 7.19873008314664e-08
1151 7.17604205462408e-08
1152 7.18644352346587e-08
1153 7.17548808548685e-08
1154 7.16705423737807e-08
1155 7.17516518964345e-08
1156 7.16408281498104e-08
1157 7.16220193215378e-08
1158 7.15438154124115e-08
1159 7.15051165780522e-08
1160 7.15096483769884e-08
1161 7.14793621039433e-08
1162 7.12951840871057e-08
1163 7.13515627417394e-08
1164 7.12608091593125e-08
1165 7.13249203805333e-08
1166 7.10641790249156e-08
1167 7.10656915234864e-08
1168 7.10763698137384e-08
1169 7.10428538557295e-08
1170 7.10737496891767e-08
1171 7.10159877126415e-08
1172 7.08870389178884e-08
1173 7.0895763458978e-08
1174 7.08237122974964e-08
1175 7.07973537732443e-08
1176 7.07712060101784e-08
1177 7.06301004118615e-08
1178 7.07561131818579e-08
1179 7.06043473321927e-08
1180 7.05150934212639e-08
1181 7.055949516932e-08
1182 7.04179674979599e-08
1183 7.03903222980529e-08
1184 7.04620731610817e-08
1185 7.03900908387567e-08
1186 7.0250944613548e-08
1187 7.03303318001502e-08
1188 7.01739798962819e-08
1189 7.02691429328439e-08
1190 7.02032671799913e-08
1191 7.01082245537776e-08
1192 7.00715721269063e-08
1193 6.99846733080989e-08
1194 6.98686978548579e-08
1195 6.99446575662677e-08
1196 6.99046867023156e-08
1197 6.98848906530714e-08
1198 6.98818136992685e-08
1199 6.98051260812349e-08
1200 6.98138071282273e-08
1201 6.95543591309189e-08
1202 6.95993953261365e-08
1203 6.95222497881787e-08
1204 6.95948797133639e-08
1205 6.96444435543953e-08
1206 6.93914463330003e-08
1207 6.9435407050733e-08
1208 6.93903131203655e-08
1209 6.92170475797838e-08
1210 6.93195316987527e-08
1211 6.93375616176439e-08
1212 6.92876138916887e-08
1213 6.92255903924632e-08
1214 6.92269289661596e-08
1215 6.90042447786254e-08
1216 6.90528989917283e-08
1217 6.91073768077644e-08
1218 6.90539477208318e-08
1219 6.88969283988428e-08
1220 6.89425985491709e-08
1221 6.86714102844377e-08
1222 6.88448204879677e-08
1223 6.87808076644814e-08
1224 6.88266665562765e-08
1225 6.87507967995771e-08
1226 6.86653141155347e-08
1227 6.87229389999544e-08
1228 6.83952675828436e-08
1229 6.85943080220852e-08
1230 6.84578053711249e-08
1231 6.83526336313633e-08
1232 6.82868769796841e-08
1233 6.8339941348583e-08
1234 6.81914030060682e-08
1235 6.81682542467144e-08
1236 6.820187751444e-08
1237 6.8028140120191e-08
1238 6.81064309819845e-08
1239 6.79802996934597e-08
1240 6.80153913332759e-08
1241 6.79235934359923e-08
1242 6.79235409819512e-08
1243 6.78361350257006e-08
1244 6.77824150958628e-08
1245 6.77713534749103e-08
1246 6.77872966292625e-08
1247 6.76140520941004e-08
1248 6.7700021107342e-08
1249 6.76828724213152e-08
1250 6.75395935676448e-08
1251 6.75415028208448e-08
1252 6.74712203014849e-08
1253 6.74316781150708e-08
1254 6.74493741996685e-08
1255 6.7367763525894e-08
1256 6.73198023211086e-08
1257 6.7283801715945e-08
1258 6.72434508555142e-08
1259 6.72133993937507e-08
1260 6.71779401120176e-08
1261 6.71631928685912e-08
1262 6.70814908811934e-08
1263 6.71851924831657e-08
1264 6.69714164374113e-08
1265 6.70927379200492e-08
1266 6.69519864313628e-08
1267 6.69877483083781e-08
1268 6.69349803921193e-08
1269 6.69581483396797e-08
1270 6.68821365188421e-08
1271 6.68138571047194e-08
1272 6.68726631722905e-08
1273 6.67096937458922e-08
1274 6.67142133963239e-08
1275 6.66196776748507e-08
1276 6.67583987112863e-08
1277 6.66058068308217e-08
1278 6.65481793848954e-08
1279 6.64961888201532e-08
1280 6.64853603886684e-08
1281 6.65342722925288e-08
1282 6.64159069643233e-08
1283 6.63308299344578e-08
1284 6.64323944157275e-08
1285 6.63917413330495e-08
1286 6.61740812546441e-08
1287 6.62493996035351e-08
1288 6.62144708982737e-08
1289 6.62723512387942e-08
1290 6.61420772249954e-08
1291 6.60725821806096e-08
1292 6.6043052216358e-08
1293 6.59756287699764e-08
1294 6.60814081587802e-08
1295 6.60696830951935e-08
1296 6.59880898528087e-08
1297 6.59831623845974e-08
1298 6.59874921442594e-08
1299 6.5802881483279e-08
1300 6.58913702018538e-08
1301 6.57962006407331e-08
1302 6.57724410491056e-08
1303 6.56766176678758e-08
1304 6.57957205429938e-08
1305 6.57352443607095e-08
1306 6.55633269719402e-08
1307 6.56247181591851e-08
1308 6.56353911612229e-08
1309 6.56317984866206e-08
1310 6.54223048481839e-08
1311 6.54726224258439e-08
1312 6.53701762249881e-08
1313 6.52837356138747e-08
1314 6.54036729059726e-08
1315 6.52590926257801e-08
1316 6.5320114069678e-08
1317 6.52694570941748e-08
1318 6.52272008121457e-08
1319 6.5228118632632e-08
1320 6.51150707025039e-08
1321 6.50548815297469e-08
1322 6.4932238474924e-08
1323 6.49190922299425e-08
1324 6.48357237498232e-08
1325 6.48373123546264e-08
1326 6.47439496077595e-08
1327 6.47510396163398e-08
1328 6.47046825310582e-08
1329 6.47574799881312e-08
1330 6.46582969316967e-08
1331 6.46756611182297e-08
1332 6.46113183560004e-08
1333 6.45539831118924e-08
1334 6.46472488323724e-08
1335 6.46570764128995e-08
1336 6.44152156379363e-08
1337 6.45557183851508e-08
1338 6.43935108222138e-08
1339 6.44670542868653e-08
1340 6.44287982218827e-08
1341 6.4410001579418e-08
1342 6.43257743622172e-08
1343 6.42401712660501e-08
1344 6.42763754026276e-08
1345 6.42433632531692e-08
1346 6.40998201131282e-08
1347 6.40833083789261e-08
1348 6.40439165007223e-08
1349 6.40517374410621e-08
1350 6.40344818911842e-08
1351 6.39777905409034e-08
1352 6.39776652757718e-08
1353 6.39520861867027e-08
1354 6.38653240621068e-08
1355 6.38062988596033e-08
1356 6.38004955426652e-08
1357 6.39451831467852e-08
1358 6.37143135300278e-08
1359 6.37618674286244e-08
1360 6.35357216456356e-08
1361 6.36275544696474e-08
1362 6.35276927507533e-08
1363 6.35030637514689e-08
1364 6.34863673809605e-08
1365 6.34809829449523e-08
1366 6.34207780354501e-08
1367 6.33917354448954e-08
1368 6.34473527938439e-08
1369 6.33899971820284e-08
1370 6.33727345888957e-08
1371 6.33508582694731e-08
1372 6.33138897470076e-08
1373 6.32693828812592e-08
1374 6.3207354136452e-08
1375 6.31532348247532e-08
1376 6.3182022774555e-08
1377 6.31533166117748e-08
1378 6.31132237689513e-08
1379 6.31136953046507e-08
1380 6.30140235227117e-08
1381 6.30657004556667e-08
1382 6.29600611841141e-08
1383 6.29796722311227e-08
1384 6.29592772956045e-08
1385 6.28798939601438e-08
1386 6.28761435930159e-08
1387 6.29084060683738e-08
1388 6.28572675935857e-08
1389 6.28043135435519e-08
1390 6.28106944180473e-08
1391 6.26891566195553e-08
1392 6.27394762986455e-08
1393 6.26804162173755e-08
1394 6.27274527644772e-08
1395 6.2740240020176e-08
1396 6.26272222064017e-08
1397 6.24908021293891e-08
1398 6.25023237876832e-08
1399 6.25678394197138e-08
1400 6.24959434141203e-08
1401 6.25369810141052e-08
1402 6.2460107573159e-08
1403 6.25329393262319e-08
1404 6.23679804352406e-08
1405 6.24036811860407e-08
1406 6.22817843574097e-08
1407 6.22351795023235e-08
1408 6.22068475450988e-08
1409 6.22041213009794e-08
1410 6.22664553020513e-08
1411 6.22070425269072e-08
1412 6.21175646955408e-08
1413 6.20549359169331e-08
1414 6.20315059638443e-08
1415 6.20665960848754e-08
1416 6.20757413720696e-08
1417 6.2031797336104e-08
1418 6.19590582005003e-08
1419 6.19536113468655e-08
1420 6.20076415511761e-08
1421 6.19261295540241e-08
1422 6.18689851759058e-08
1423 6.1826028153078e-08
1424 6.19208055354648e-08
1425 6.18520118642607e-08
1426 6.17248835208528e-08
1427 6.1802893206675e-08
1428 6.17068142485522e-08
1429 6.16915627045245e-08
1430 6.16567387776712e-08
1431 6.16668488788719e-08
1432 6.16802115889214e-08
1433 6.15769074130412e-08
1434 6.15658013067844e-08
1435 6.16248732470126e-08
1436 6.14891621744107e-08
1437 6.14608707056874e-08
1438 6.14451786269399e-08
1439 6.14355819177348e-08
1440 6.14077755471953e-08
1441 6.14296413736781e-08
1442 6.13970089986537e-08
1443 6.13804152962416e-08
1444 6.13737013033244e-08
1445 6.1262538999074e-08
1446 6.13701174625447e-08
1447 6.12855666144441e-08
1448 6.1195222174959e-08
1449 6.12342441836944e-08
1450 6.11450922090029e-08
1451 6.11980987681449e-08
1452 6.11357222837228e-08
1453 6.10229728792433e-08
1454 6.10374002985026e-08
1455 6.10942005838666e-08
1456 6.10254496358209e-08
1457 6.10406095571392e-08
1458 6.10175440112215e-08
1459 6.09875492383338e-08
1460 6.09176045145432e-08
1461 6.09647031097893e-08
1462 6.09012781804808e-08
1463 6.08128564110189e-08
1464 6.08482118931164e-08
1465 6.08169745586196e-08
1466 6.0853253852855e-08
1467 6.07248810933214e-08
1468 6.06711857731312e-08
1469 6.07058140218442e-08
1470 6.07171320794464e-08
1471 6.06507328857475e-08
1472 6.05790051189103e-08
1473 6.05092922700834e-08
1474 6.05416510293111e-08
1475 6.05060932041823e-08
1476 6.03901202218537e-08
1477 6.03501171738685e-08
1478 6.04025230952487e-08
1479 6.04625073670206e-08
1480 6.03278789217399e-08
1481 6.03538663295211e-08
1482 6.03227165445475e-08
1483 6.0213328524128e-08
1484 6.0201226045109e-08
1485 6.01847671362066e-08
1486 6.01654091383352e-08
1487 6.014508397989e-08
1488 6.01233429495807e-08
1489 6.01248872396098e-08
1490 6.01089693965662e-08
1491 6.0105817278e-08
1492 6.0000928536752e-08
1493 6.00806460155923e-08
1494 6.00507297860986e-08
1495 5.99723805709829e-08
1496 5.99071191462741e-08
1497 5.99083890850949e-08
1498 5.99040784194926e-08
1499 5.99132409941916e-08
1500 5.98530904323269e-08
1501 5.98748511659863e-08
1502 5.99154655986922e-08
1503 5.98588880915685e-08
1504 5.98475480142469e-08
1505 5.97873882259847e-08
1506 5.9731291571552e-08
1507 5.96959711245404e-08
1508 5.96802322387902e-08
1509 5.96461802260251e-08
1510 5.96566538835219e-08
1511 5.96443509994771e-08
1512 5.95717547451358e-08
1513 5.95137158772019e-08
1514 5.959686659196e-08
1515 5.95794134845562e-08
1516 5.94797409601e-08
1517 5.95396524865066e-08
1518 5.94716751898261e-08
1519 5.97946166163155e-08
1520 5.9837887567582e-08
1521 5.9820797128296e-08
1522 5.97260096792951e-08
1523 5.96731224806746e-08
1524 5.96955790364007e-08
1525 5.96529990257721e-08
1526 5.9619664495969e-08
1527 5.96069288878454e-08
1528 5.95045015980844e-08
1529 5.95249735635406e-08
1530 5.9383872148544e-08
1531 5.94413071848265e-08
1532 5.93339268117887e-08
1533 5.93306542278071e-08
1534 5.93074283727901e-08
1535 5.92692220884317e-08
1536 5.92856753751647e-08
1537 5.93554186174572e-08
1538 5.92641916394854e-08
1539 5.93318333113046e-08
1540 5.91967311649455e-08
1541 5.92650876907186e-08
1542 5.91634085278514e-08
1543 5.9196797121075e-08
1544 5.91875115709684e-08
1545 5.91937632332673e-08
1546 5.92028680195256e-08
1547 5.9150001350261e-08
1548 5.91333262640603e-08
1549 5.91312632458596e-08
1550 5.90696964479775e-08
1551 5.90639513813329e-08
1552 5.90868665142352e-08
1553 5.89636118597525e-08
1554 5.90322244864438e-08
1555 5.89159145309992e-08
1556 5.89343164492107e-08
1557 5.89115702229748e-08
1558 5.87961392142233e-08
1559 5.8929762042581e-08
1560 5.88232943510292e-08
1561 5.88922050361873e-08
1562 5.88861295121035e-08
1563 5.86172964673892e-08
1564 5.86649945990558e-08
1565 5.8718637879096e-08
1566 5.87609447322279e-08
1567 5.87051747924505e-08
1568 5.85571262483597e-08
1569 5.86728897573607e-08
1570 5.86644843494355e-08
1571 5.85907780248363e-08
1572 5.86321109601329e-08
1573 5.85898334133361e-08
1574 5.85883252437469e-08
1575 5.85015697467384e-08
1576 5.85448282155454e-08
1577 5.84579239610861e-08
1578 5.85012106366634e-08
1579 5.84556953651116e-08
1580 5.83049609925723e-08
1581 5.83817353270177e-08
1582 5.84006233967926e-08
1583 5.83841783399208e-08
1584 5.83341679671889e-08
1585 5.82988901687287e-08
1586 5.8316455463725e-08
1587 5.83331390817676e-08
1588 5.81804040908906e-08
1589 5.82136462803362e-08
1590 5.82641990440891e-08
1591 5.80710060180678e-08
1592 5.82011788861081e-08
1593 5.8121242568987e-08
1594 5.81866827005229e-08
1595 5.80669022340885e-08
1596 5.80650103483293e-08
1597 5.80542040129473e-08
1598 5.8086535542401e-08
1599 5.79826335229683e-08
1600 5.79942535825495e-08
1601 5.79461422809402e-08
1602 5.79765697867884e-08
1603 5.7955934893883e-08
1604 5.79382814205331e-08
1605 5.7960223934117e-08
1606 5.78662638446303e-08
1607 5.78888092395147e-08
1608 5.78656528880117e-08
1609 5.78373466257887e-08
1610 5.7816928659804e-08
1611 5.77766344953545e-08
1612 5.77002090516032e-08
1613 5.78353792324293e-08
1614 5.76594382604156e-08
1615 5.77152839333905e-08
1616 5.77050096985943e-08
1617 5.76386486859093e-08
1618 5.75331130789891e-08
1619 5.75876199686576e-08
1620 5.75852555328993e-08
1621 5.74973211460161e-08
1622 5.75818900170333e-08
1623 5.75444482482368e-08
1624 5.75184087345804e-08
1625 5.74836534141809e-08
1626 5.74368222814314e-08
1627 5.74238611132927e-08
1628 5.74171671452461e-08
1629 5.73901723086578e-08
1630 5.74379704580963e-08
1631 5.73458928290194e-08
1632 5.73970139399194e-08
1633 5.73745841858653e-08
1634 5.72744749458565e-08
1635 5.72565056895513e-08
1636 5.72551413391409e-08
1637 5.72071138478236e-08
1638 5.71946696705794e-08
1639 5.71987397570695e-08
1640 5.7234945854745e-08
1641 5.71473964114944e-08
1642 5.7133663700526e-08
1643 5.71177499271158e-08
1644 5.70842954328299e-08
1645 5.70492411853252e-08
1646 5.70380310680463e-08
1647 5.69918162227623e-08
1648 5.7041033475258e-08
1649 5.69923012463391e-08
1650 5.69798472245253e-08
1651 5.69699207790819e-08
1652 5.69449175369385e-08
1653 5.69098278138114e-08
1654 5.69499187275113e-08
1655 5.68505189502133e-08
1656 5.6859197728798e-08
1657 5.68559886655606e-08
1658 5.68171476391655e-08
1659 5.68159301543858e-08
1660 5.68048425737544e-08
1661 5.6751532591548e-08
1662 5.68019727236191e-08
1663 5.67193444265257e-08
1664 5.67582746864304e-08
1665 5.67219289013821e-08
1666 5.6725984849848e-08
1667 5.66976989322399e-08
1668 5.66983568877077e-08
1669 5.66502959262749e-08
1670 5.66049141959013e-08
1671 5.65990336003352e-08
1672 5.66041950520457e-08
1673 5.65855718281938e-08
1674 5.65686063147552e-08
1675 5.65555438232224e-08
1676 5.65208707197229e-08
1677 5.64822816606636e-08
1678 5.64966538760814e-08
1679 5.64763086767783e-08
1680 5.64278408141661e-08
1681 5.6431917411004e-08
1682 5.64373989160316e-08
1683 5.63661688168793e-08
1684 5.64039088537527e-08
1685 5.63466530234535e-08
1686 5.63212377020506e-08
1687 5.63024112523181e-08
1688 5.62978523763746e-08
1689 5.62262562802118e-08
1690 5.62895351094994e-08
1691 5.62492424602823e-08
1692 5.61989457654732e-08
1693 5.61890079779914e-08
1694 5.61939509555742e-08
1695 5.61947802495411e-08
1696 5.61773661331699e-08
1697 5.61531530003379e-08
1698 5.61235287666051e-08
1699 5.60940196443482e-08
1700 5.61204899973689e-08
1701 5.60406343428355e-08
1702 5.60643956202256e-08
1703 5.60494919579213e-08
1704 5.59998603950618e-08
1705 5.59864019820111e-08
1706 5.59914939639583e-08
1707 5.59463899438839e-08
1708 5.59803844950579e-08
1709 5.59196386724636e-08
1710 5.5920420274802e-08
1711 5.58580741074621e-08
1712 5.58791146083593e-08
1713 5.58437067130768e-08
1714 5.58421911165397e-08
1715 5.5783852104696e-08
1716 5.57677245911492e-08
1717 5.57753714094389e-08
1718 5.57292996550274e-08
1719 5.57495332742519e-08
1720 5.57000440082334e-08
1721 5.5673648212462e-08
1722 5.57074793476886e-08
1723 5.56354557019745e-08
1724 5.56644631490855e-08
1725 5.57036488864071e-08
1726 5.55856740298566e-08
1727 5.55686554726265e-08
1728 5.55945883391473e-08
1729 5.55561823887274e-08
1730 5.55563986299745e-08
1731 5.55365409322661e-08
1732 5.55880503370076e-08
1733 5.53760209367482e-08
1734 5.5451998006717e-08
1735 5.53872368946884e-08
1736 5.54450427117104e-08
1737 5.54718433303236e-08
1738 5.54353870558799e-08
1739 5.5423333501281e-08
1740 5.54008930588878e-08
1741 5.5398429134712e-08
1742 5.53924214923285e-08
1743 5.53879958999914e-08
1744 5.54124541718437e-08
1745 5.53301252192995e-08
1746 5.52668945896784e-08
1747 5.53462605985544e-08
1748 5.52702616314349e-08
1749 5.53287882354425e-08
1750 5.52555329917936e-08
1751 5.52268163822589e-08
1752 5.52134693538875e-08
1753 5.5133538829466e-08
1754 5.51946055367125e-08
1755 5.51694730912544e-08
1756 5.52048054771603e-08
1757 5.5137500563518e-08
1758 5.50951957585255e-08
1759 5.50981046725241e-08
1760 5.49992534111254e-08
1761 5.49789160775305e-08
1762 5.5103229328779e-08
1763 5.49958714763932e-08
1764 5.49898756521117e-08
1765 5.50356775619321e-08
1766 5.49056398320147e-08
1767 5.49246820380489e-08
1768 5.49204639472833e-08
1769 5.49651988546174e-08
1770 5.48529977226764e-08
1771 5.48326578115876e-08
1772 5.48019908546138e-08
1773 5.48953561576582e-08
1774 5.47445199874375e-08
1775 5.48171828285149e-08
1776 5.47500933123501e-08
1777 5.47052652528635e-08
1778 5.47431274018351e-08
1779 5.4834798385528e-08
1780 5.46739062485813e-08
1781 5.46523622535489e-08
1782 5.46848105287268e-08
1783 5.46258883353801e-08
1784 5.46131483432077e-08
1785 5.47325840400958e-08
1786 5.44927466297196e-08
1787 5.46209065319658e-08
1788 5.45049952513921e-08
1789 5.45092141610581e-08
1790 5.45523490078637e-08
1791 5.45345960976107e-08
1792 5.44524102785715e-08
1793 5.45079805629456e-08
1794 5.44182462398624e-08
1795 5.44705242653265e-08
1796 5.44085561369201e-08
1797 5.43801523775755e-08
1798 5.44027891713483e-08
1799 5.43500502274696e-08
1800 5.44582627561852e-08
1801 5.44397154627063e-08
1802 5.4338718060265e-08
1803 5.44225994314473e-08
1804 5.42171299979799e-08
1805 5.4337201040866e-08
1806 5.41214363440901e-08
1807 5.40811222524695e-08
1808 5.41939004321534e-08
1809 5.40749616977365e-08
1810 5.41500592454724e-08
1811 5.41653052650304e-08
1812 5.40761747682694e-08
1813 5.41268452565191e-08
1814 5.41228220782841e-08
1815 5.41007852170594e-08
1816 5.40502305579338e-08
1817 5.40173954721723e-08
1818 5.41068862141003e-08
1819 5.40045172616743e-08
1820 5.4038346789298e-08
1821 5.39798734706665e-08
1822 5.40564639841534e-08
1823 5.39539658941379e-08
1824 5.38244664003429e-08
1825 5.37857237734585e-08
1826 5.38499908611101e-08
1827 5.38678723689401e-08
1828 5.37573918251155e-08
1829 5.371571719337e-08
1830 5.38871727933099e-08
1831 5.37434551119986e-08
1832 5.3639850435161e-08
1833 5.36533420056884e-08
1834 5.36572051910156e-08
1835 5.36227781235965e-08
1836 5.36104635742163e-08
1837 5.36079381703303e-08
1838 5.36290869792566e-08
1839 5.37385967263049e-08
1840 5.36181679109404e-08
1841 5.36484127451331e-08
1842 5.36799106267694e-08
1843 5.35505471344067e-08
1844 5.35945544264393e-08
1845 5.35271928203684e-08
1846 5.35207890095535e-08
1847 5.35482531542897e-08
1848 5.35642720915774e-08
1849 5.34805623750856e-08
1850 5.3431965918449e-08
1851 5.34549173565324e-08
1852 5.34378152927673e-08
1853 5.34455174676651e-08
1854 5.34056202159405e-08
1855 5.33727079545798e-08
1856 5.33614931992332e-08
1857 5.32851254391176e-08
1858 5.32601642575514e-08
1859 5.33542429703715e-08
1860 5.33212912916525e-08
1861 5.32777262343132e-08
1862 5.32938927335636e-08
1863 5.31562632790639e-08
1864 5.32818992891038e-08
1865 5.33181821875672e-08
1866 5.31972725017482e-08
1867 5.32598656768357e-08
1868 5.31830020396029e-08
1869 5.31817506015386e-08
1870 5.31520006106234e-08
1871 5.31114165447377e-08
1872 5.31451917407111e-08
1873 5.31334311624221e-08
1874 5.30620160237305e-08
1875 5.31187041676873e-08
1876 5.30357756751698e-08
1877 5.3064633961597e-08
1878 5.29748408801822e-08
1879 5.30346630807088e-08
1880 5.30289344684576e-08
1881 5.30416028947656e-08
1882 5.29283992722895e-08
1883 5.29177369550382e-08
1884 5.30045731430562e-08
1885 5.28588517081374e-08
1886 5.28985137240312e-08
1887 5.29475812474089e-08
1888 5.2808779052782e-08
1889 5.28289014880556e-08
1890 5.28440459639512e-08
1891 5.28593501218921e-08
1892 5.27749495660146e-08
1893 5.28735262896873e-08
1894 5.27845853053321e-08
1895 5.27881704002198e-08
1896 5.27028273147323e-08
1897 5.27637186191754e-08
1898 5.27312883527031e-08
1899 5.26931697812927e-08
1900 5.26740447330809e-08
1901 5.26480908487059e-08
1902 5.26531934745833e-08
1903 5.26507988549696e-08
1904 5.25389355399142e-08
1905 5.26488673529002e-08
1906 5.26084224645729e-08
1907 5.25807398918943e-08
1908 5.25613662443192e-08
1909 5.24996321864535e-08
1910 5.24453787544843e-08
1911 5.24613502754789e-08
1912 5.24970087791843e-08
1913 5.24854672256936e-08
1914 5.25649782954218e-08
1915 5.24553833844976e-08
1916 5.24687464693585e-08
1917 5.2441078288723e-08
1918 5.23808334769171e-08
1919 5.24360677065516e-08
1920 5.24426827954727e-08
1921 5.22993853167009e-08
1922 5.23353140611249e-08
1923 5.23437364989832e-08
1924 5.23348198715468e-08
1925 5.23603856787958e-08
1926 5.23062117459006e-08
1927 5.23552979014852e-08
1928 5.22349181277804e-08
1929 5.22702303467071e-08
1930 5.21796671986152e-08
1931 5.22491127341595e-08
1932 5.22685628432384e-08
1933 5.22486192178206e-08
1934 5.22139570815483e-08
1935 5.22113819201309e-08
1936 5.22190183289695e-08
1937 5.2076112757149e-08
1938 5.22017290691679e-08
1939 5.20835124184771e-08
1940 5.2079003518557e-08
1941 5.21372876161053e-08
1942 5.19895417756544e-08
1943 5.20888511612583e-08
1944 5.20971060780084e-08
1945 5.19696601539721e-08
1946 5.20138624420241e-08
1947 5.20468353748527e-08
1948 5.19405348207158e-08
1949 5.19631560536027e-08
1950 5.19260941747035e-08
1951 5.19536004475185e-08
1952 5.19081625416362e-08
1953 5.19320281906488e-08
1954 5.18985728419352e-08
1955 5.1928550446334e-08
1956 5.18961853224198e-08
1957 5.18665934023943e-08
1958 5.18521137067296e-08
1959 5.17538961304354e-08
1960 5.17549131089368e-08
1961 5.18354555332223e-08
1962 5.18148292503184e-08
1963 5.18776375830754e-08
1964 5.16802608956368e-08
1965 5.17854563728548e-08
1966 5.16833070278722e-08
1967 5.16794470772908e-08
1968 5.17138261777461e-08
1969 5.16371921950309e-08
1970 5.16681452165102e-08
1971 5.16478825165478e-08
1972 5.16295489187968e-08
1973 5.1612459754935e-08
1974 5.16259277070219e-08
1975 5.1722021092715e-08
1976 5.15612141409605e-08
1977 5.15435393744212e-08
1978 5.15455119423081e-08
1979 5.16167272301971e-08
1980 5.14827943938201e-08
1981 5.14925190202575e-08
1982 5.15451809768308e-08
1983 5.15657653874513e-08
1984 5.16082813213359e-08
1985 5.14165980707304e-08
1986 5.15504572931036e-08
1987 5.15408258774386e-08
1988 5.13613475838781e-08
1989 5.13847143377433e-08
1990 5.13916178999096e-08
1991 5.13711118443894e-08
1992 5.12695608581026e-08
1993 5.13419150287575e-08
1994 5.12768004998776e-08
1995 5.12772312752929e-08
1996 5.13884468915649e-08
1997 5.13359248621725e-08
1998 5.12340304101855e-08
1999 5.13088399003436e-08
2000 5.11715840261218e-08
2001 5.11890708452256e-08
2002 5.12845055329336e-08
2003 5.108533138376e-08
2004 5.1183148974232e-08
2005 5.11481020950555e-08
2006 5.11348524749877e-08
2007 5.12453937009383e-08
2008 5.10534286082276e-08
2009 5.108819168953e-08
2010 5.10586424873338e-08
2011 5.09971432975931e-08
2012 5.1132785982233e-08
2013 5.10062531038358e-08
2014 5.10258888315462e-08
2015 5.0978704100757e-08
2016 5.08830928040993e-08
2017 5.09942403272845e-08
2018 5.09327370039614e-08
2019 5.08663328737668e-08
2020 5.09860580812926e-08
2021 5.08445735345475e-08
2022 5.08936039746999e-08
2023 5.09094181264658e-08
2024 5.08930317977274e-08
2025 5.08564379124721e-08
2026 5.08189316281715e-08
2027 5.08235516516464e-08
2028 5.07668618627832e-08
2029 5.07491164718488e-08
2030 5.07665752333963e-08
2031 5.07693540523491e-08
2032 5.0771642213121e-08
2033 5.07379351954995e-08
2034 5.07573619419333e-08
2035 5.07073098816591e-08
2036 5.05777827211062e-08
2037 5.06819269290304e-08
2038 5.06514455373974e-08
2039 5.06447307238034e-08
2040 5.07077059541672e-08
2041 5.06020028723242e-08
2042 5.07431037473083e-08
2043 5.06044412471596e-08
2044 5.06480008759524e-08
2045 5.07130630467145e-08
2046 5.05267824237876e-08
2047 5.05867775526525e-08
2048 5.05107546864281e-08
2049 5.04366832085879e-08
2050 5.056566077144e-08
2051 5.04607714102434e-08
2052 5.05076682397743e-08
2053 5.04269229715248e-08
2054 5.04503119600486e-08
2055 5.03873394137599e-08
2056 5.05476441077235e-08
2057 5.0389677031859e-08
2058 5.03973501544408e-08
2059 5.03904498945218e-08
2060 5.04219205712531e-08
2061 5.03719183306828e-08
2062 5.03316593327696e-08
2063 5.03992613225535e-08
2064 5.02578325196623e-08
2065 5.02366449453007e-08
2066 5.0278826837058e-08
2067 5.02935927126913e-08
2068 5.02327572711181e-08
2069 5.02330336864532e-08
2070 5.02750241597738e-08
2071 5.02149371506277e-08
2072 5.01553019542911e-08
2073 5.01657466287497e-08
2074 5.00899905873098e-08
2075 5.01789853917245e-08
2076 5.01529840359893e-08
2077 5.00981354196739e-08
2078 5.01358077773517e-08
2079 5.01251214082288e-08
2080 5.0052278082191e-08
2081 5.00252037802795e-08
2082 5.0062973347309e-08
2083 4.9931246728363e-08
2084 5.00272196948259e-08
2085 4.99955715316247e-08
2086 4.99903041806249e-08
2087 4.99563221705301e-08
2088 4.99299010883192e-08
2089 4.99001988547576e-08
2090 4.993478970583e-08
2091 4.99018632993398e-08
2092 4.9912327186874e-08
2093 4.9875605567351e-08
2094 4.98612064685489e-08
2095 4.98594528721696e-08
2096 4.98579711010194e-08
2097 4.98132633932613e-08
2098 4.98373859301182e-08
2099 4.9786350295733e-08
2100 4.97936768333318e-08
2101 4.97991141905629e-08
2102 4.975003211527e-08
2103 4.97484139501125e-08
2104 4.96887673868684e-08
2105 4.97299567978615e-08
2106 4.97160533008412e-08
2107 4.96852688840477e-08
2108 4.97137560202532e-08
2109 4.96771717983791e-08
2110 4.96851741953464e-08
2111 4.96416248161324e-08
2112 4.96316309490652e-08
2113 4.96335141768611e-08
2114 4.95966623788036e-08
2115 4.9643151699641e-08
2116 4.95735543726283e-08
2117 4.95890670375587e-08
2118 4.95653858312295e-08
2119 4.95542918663716e-08
2120 4.95665911941501e-08
2121 4.94799767736254e-08
2122 4.9475043530478e-08
2123 4.94719850756553e-08
2124 4.96253926769441e-08
2125 4.9470145622621e-08
2126 4.92887689631516e-08
2127 4.93575011066838e-08
2128 4.93663929521659e-08
2129 4.93478784520107e-08
2130 4.93142160440385e-08
2131 4.93155137597512e-08
2132 4.93141739550396e-08
2133 4.92669135212509e-08
2134 4.92621176597652e-08
2135 4.92572295307525e-08
2136 4.9198016810692e-08
2137 4.92057887218067e-08
2138 4.91521048662236e-08
2139 4.92070198490069e-08
2140 4.91211864570573e-08
2141 4.91634815400488e-08
2142 4.91254827945653e-08
2143 4.91519297618481e-08
2144 4.9134670163653e-08
2145 4.9051880704809e-08
2146 4.90395767798191e-08
2147 4.90849227929857e-08
2148 4.90442833829263e-08
2149 4.90445734993017e-08
2150 4.90608726373409e-08
2151 4.9041894992996e-08
2152 4.89982328950589e-08
2153 4.89880221898886e-08
2154 4.89602745616935e-08
2155 4.89568223951409e-08
2156 4.89168783133209e-08
2157 4.89172199156229e-08
2158 4.89685717166566e-08
2159 4.88609757596237e-08
2160 4.89263075031943e-08
2161 4.88273234271475e-08
2162 4.88585390208129e-08
2163 4.88363156652127e-08
2164 4.88911442584339e-08
2165 4.8841016777601e-08
2166 4.88153942335146e-08
2167 4.8813878306575e-08
2168 4.87659374162064e-08
2169 4.87637298025589e-08
2170 4.86988088752582e-08
2171 4.87743254300454e-08
2172 4.87551510452988e-08
2173 4.86789006686195e-08
2174 4.87389658623272e-08
2175 4.87347982467412e-08
2176 4.87038046905752e-08
2177 4.87496624348438e-08
2178 4.85581900999676e-08
2179 4.85927552738019e-08
2180 4.8573652431827e-08
2181 4.86040097520402e-08
2182 4.86440360685236e-08
2183 4.86360532079289e-08
2184 4.85832657410867e-08
2185 4.85571463890722e-08
2186 4.8584650938821e-08
2187 4.85181812823754e-08
2188 4.86955202791961e-08
2189 4.87293292152913e-08
2190 4.84575562555989e-08
2191 4.84634830666408e-08
2192 4.86083591866304e-08
2193 4.86511774244747e-08
2194 4.85244748773539e-08
2195 4.83422507535636e-08
2196 4.84176624482302e-08
2197 4.85656777424737e-08
2198 4.87041562422519e-08
2199 4.8695021380496e-08
2200 4.86750873900377e-08
2201 4.86189788642122e-08
2202 4.86105154013927e-08
2203 4.85637531912886e-08
2204 4.85139982284721e-08
2205 4.84500849680103e-08
2206 4.8507263606723e-08
2207 4.84854814928326e-08
2208 4.8485627401007e-08
2209 4.84758057233137e-08
2210 4.84405148810652e-08
2211 4.83962431641061e-08
2212 4.83855662682942e-08
2213 4.83916726476963e-08
2214 4.83807489235488e-08
2215 4.83411497498309e-08
2216 4.8334528045757e-08
2217 4.83195298190964e-08
2218 4.83272201989848e-08
2219 4.82977144944385e-08
2220 4.82678723301433e-08
2221 4.82376803283557e-08
2222 4.82432055726179e-08
2223 4.824876263676e-08
2224 4.8196495324504e-08
2225 4.82037481415176e-08
2226 4.81614372098704e-08
2227 4.81536547631833e-08
2228 4.81418322788585e-08
2229 4.81312924236477e-08
2230 4.81259781857091e-08
2231 4.81309198825386e-08
2232 4.81011687369914e-08
2233 4.80794940926899e-08
2234 4.80530801567625e-08
2235 4.80365342632183e-08
2236 4.80123732753412e-08
2237 4.82136066803918e-08
2238 4.80063985133228e-08
2239 4.79927257064361e-08
2240 4.79688055232685e-08
2241 4.79559529384943e-08
2242 4.7932399487749e-08
2243 4.79327281084352e-08
2244 4.79032842193305e-08
2245 4.78870530127296e-08
2246 4.7897409759301e-08
2247 4.78899824898349e-08
2248 4.79158671460311e-08
2249 4.78342768719386e-08
2250 4.78356331559127e-08
2251 4.78195494810763e-08
2252 4.78241383188305e-08
2253 4.77886039327302e-08
2254 4.77907047358173e-08
2255 4.77790554036517e-08
2256 4.77225829982331e-08
2257 4.77691799112279e-08
2258 4.77202352975326e-08
2259 4.77231042612658e-08
2260 4.76918225160716e-08
2261 4.76850659509864e-08
2262 4.76653707295327e-08
2263 4.76637645032696e-08
2264 4.76752667903924e-08
2265 4.76522601022822e-08
2266 4.75443320020474e-08
2267 4.76046351725046e-08
2268 4.75899503911137e-08
2269 4.76082751106333e-08
2270 4.75562892869874e-08
2271 4.75460607756162e-08
2272 4.74553906464337e-08
2273 4.75433863300623e-08
2274 4.75082791577819e-08
2275 4.74913049526293e-08
2276 4.7472266818005e-08
2277 4.7514954102823e-08
2278 4.74261439702417e-08
2279 4.74247494057778e-08
2280 4.74223153776876e-08
2281 4.74128068788104e-08
2282 4.7373778746973e-08
2283 4.74615011327728e-08
2284 4.73816435917485e-08
2285 4.72311344186238e-08
2286 4.70894225088614e-08
2287 4.71261871304307e-08
2288 4.7089922624366e-08
2289 4.71243092547979e-08
2290 4.71778768567077e-08
2291 4.7099189941946e-08
2292 4.70951039499568e-08
2293 4.70864535984106e-08
2294 4.70535969849806e-08
2295 4.70482702965569e-08
2296 4.71052526815896e-08
2297 4.70171900257554e-08
2298 4.70092308937353e-08
2299 4.69985473436907e-08
2300 4.6982804040141e-08
2301 4.69694128817366e-08
2302 4.70123206977036e-08
2303 4.69505122211444e-08
2304 4.69312570672287e-08
2305 4.69077716367394e-08
2306 4.69025190241723e-08
2307 4.68731555596236e-08
2308 4.68652332692443e-08
2309 4.69317143814152e-08
2310 4.68382702578651e-08
2311 4.68338990025074e-08
2312 4.68038600232035e-08
2313 4.67976002784809e-08
2314 4.67807933350883e-08
2315 4.67787167153233e-08
2316 4.68322396294241e-08
2317 4.67381676845235e-08
2318 4.6729334806983e-08
2319 4.67070101652922e-08
2320 4.66977876474317e-08
2321 4.66965717595968e-08
2322 4.67161351025425e-08
2323 4.66550702480362e-08
2324 4.66441265789541e-08
2325 4.66310267146497e-08
2326 4.66320685248434e-08
2327 4.65410337415761e-08
2328 4.6547874791969e-08
2329 4.65453979394681e-08
2330 4.65093546555551e-08
2331 4.64722014594798e-08
2332 4.65115457686238e-08
2333 4.64743916808175e-08
2334 4.64605620589254e-08
2335 4.64740945194109e-08
2336 4.64476470121156e-08
2337 4.64249473974832e-08
2338 4.64848014800623e-08
2339 4.64884505788632e-08
2340 4.63842475912912e-08
2341 4.63792031606403e-08
2342 4.63494252986862e-08
2343 4.63379110975382e-08
2344 4.63379693211863e-08
2345 4.64114022538098e-08
2346 4.63126262530977e-08
2347 4.62870884661015e-08
2348 4.62881356604328e-08
2349 4.62744234113188e-08
2350 4.6247423872714e-08
2351 4.62715891984544e-08
2352 4.62288329927674e-08
2353 4.62333089128464e-08
2354 4.62618296968031e-08
2355 4.61937061615458e-08
2356 4.6126346880726e-08
2357 4.61416341064336e-08
2358 4.61463271310691e-08
2359 4.60887011666244e-08
2360 4.61372476774358e-08
2361 4.61707675309953e-08
2362 4.61092239465444e-08
2363 4.60881264903179e-08
2364 4.60690908870021e-08
2365 4.60493507645765e-08
2366 4.60942382094487e-08
2367 4.60283508587622e-08
2368 4.6028404346643e-08
2369 4.60132405244451e-08
2370 4.59628308462356e-08
2371 4.60496629379747e-08
2372 4.59794396814317e-08
2373 4.59833530417342e-08
2374 4.59662994281018e-08
2375 4.59048912073712e-08
2376 4.5935132398256e-08
2377 4.59200639824076e-08
2378 4.59515519022347e-08
2379 4.58770692937094e-08
2380 4.58749974239225e-08
2381 4.58770293718658e-08
2382 4.58547017156974e-08
2383 4.58360697841442e-08
2384 4.59388363296398e-08
2385 4.58157067608767e-08
2386 4.58128742568675e-08
2387 4.57550536392404e-08
2388 4.57336962433885e-08
2389 4.57688764416986e-08
2390 4.57551770853826e-08
2391 4.58180880329451e-08
2392 4.57186970344026e-08
2393 4.57537748026482e-08
2394 4.5656324520138e-08
2395 4.56930174443926e-08
2396 4.56428806394626e-08
2397 4.56939622992536e-08
2398 4.56470841161405e-08
2399 4.56426408472765e-08
2400 4.56515097742027e-08
2401 4.56016143317584e-08
2402 4.560056685321e-08
2403 4.55876009635148e-08
2404 4.56001403250639e-08
2405 4.55553495459782e-08
2406 4.55227308755468e-08
2407 4.55230166487297e-08
2408 4.55012463351778e-08
2409 4.54478410958359e-08
2410 4.55274436408359e-08
2411 4.54643422731493e-08
2412 4.54358375296948e-08
2413 4.54359391159898e-08
2414 4.54435548515875e-08
2415 4.54298440946133e-08
2416 4.54605188959789e-08
2417 4.54074609823607e-08
2418 4.53812884142479e-08
2419 4.53733464667039e-08
2420 4.53692461164223e-08
2421 4.53234984192363e-08
2422 4.53594588876172e-08
2423 4.53265968491934e-08
2424 4.52764114200477e-08
2425 4.52984973744464e-08
2426 4.53130955175851e-08
2427 4.52592561561715e-08
2428 4.53241508342472e-08
2429 4.52526349725701e-08
2430 4.51955605580423e-08
2431 4.52048519683501e-08
2432 4.51978455995317e-08
2433 4.52538862862895e-08
2434 4.51487067323342e-08
2435 4.51207987826052e-08
2436 4.51189000347796e-08
2437 4.51654767417153e-08
2438 4.51162765067181e-08
2439 4.51856078811375e-08
2440 4.50536535456791e-08
2441 4.50368157345338e-08
2442 4.50998288812343e-08
2443 4.50042980926213e-08
2444 4.50083280778557e-08
2445 4.50422306528253e-08
2446 4.49848399561858e-08
2447 4.49772960191552e-08
2448 4.49829808299995e-08
2449 4.50294015710284e-08
2450 4.49538773317215e-08
2451 4.49166493190489e-08
2452 4.49170948257915e-08
2453 4.49589146764851e-08
2454 4.49593229383538e-08
2455 4.49599819773994e-08
2456 4.49344020250209e-08
2457 4.48203933807179e-08
2458 4.49152907986417e-08
2459 4.48805895114646e-08
2460 4.4873780948862e-08
2461 4.48401943220489e-08
2462 4.48429458472788e-08
2463 4.48234535674885e-08
2464 4.48091864519995e-08
2465 4.47731328456769e-08
2466 4.47578983742147e-08
2467 4.47998173740416e-08
2468 4.47304871347143e-08
2469 4.47257060045558e-08
2470 4.46906310411777e-08
2471 4.47007092869711e-08
2472 4.474582092584e-08
2473 4.46378149074889e-08
2474 4.46649903800278e-08
2475 4.46439422532308e-08
2476 4.46238308597913e-08
2477 4.46799980782231e-08
2478 4.46298196283834e-08
2479 4.45875122636608e-08
2480 4.45883893362975e-08
2481 4.45663611188252e-08
2482 4.46358282051307e-08
2483 4.45350027931113e-08
2484 4.45214754947187e-08
2485 4.44995302810725e-08
2486 4.45226918301955e-08
2487 4.44902259850721e-08
2488 4.45270056985692e-08
2489 4.44683895217679e-08
2490 4.44681610272113e-08
2491 4.44464482605156e-08
2492 4.44466555293843e-08
2493 4.45026926776393e-08
2494 4.44055714616809e-08
2495 4.43890110570777e-08
2496 4.43960359302054e-08
2497 4.43674068737465e-08
2498 4.43854575244984e-08
2499 4.43878566471767e-08
2500 4.4334693530601e-08
2501 4.43226212265557e-08
2502 4.43174175437377e-08
2503 4.43000149470407e-08
2504 4.43612127618564e-08
2505 4.42417979424903e-08
2506 4.42709021530874e-08
2507 4.42749768492234e-08
2508 4.42489430430015e-08
2509 4.42752194018681e-08
2510 4.42317287241423e-08
2511 4.42114293477403e-08
2512 4.42021089614997e-08
2513 4.41858280915142e-08
2514 4.4258080540871e-08
2515 4.40748920489398e-08
2516 4.42169541479132e-08
2517 4.41688105929927e-08
2518 4.42127747746213e-08
2519 4.41516358211658e-08
2520 4.41866786271561e-08
2521 4.41536992781266e-08
2522 4.40722750596478e-08
2523 4.41020148738147e-08
2524 4.40915417350141e-08
2525 4.4140714116736e-08
2526 4.40567839170569e-08
2527 4.40755789146152e-08
2528 4.4014898469058e-08
2529 4.40320554861984e-08
2530 4.4016578991446e-08
2531 4.40720677623574e-08
2532 4.40181328240641e-08
2533 4.39891899972622e-08
2534 4.39881869969128e-08
2535 4.39434307590858e-08
2536 4.39752036509589e-08
2537 4.39740482196527e-08
2538 4.39137943359924e-08
2539 4.38996085350141e-08
2540 4.39297043328679e-08
2541 4.3914395515543e-08
2542 4.38457781601897e-08
2543 4.40599233666461e-08
2544 4.39322013043864e-08
2545 4.39237850500263e-08
2546 4.38724331957019e-08
2547 4.38168872811673e-08
2548 4.38477626794054e-08
2549 4.37685006460242e-08
2550 4.377987932358e-08
2551 4.37889589974816e-08
2552 4.37748383319558e-08
2553 4.37841001428296e-08
2554 4.37508283610555e-08
2555 4.37149942342785e-08
2556 4.37059092988079e-08
2557 4.37654103677687e-08
2558 4.36987938208233e-08
2559 4.36504521523773e-08
2560 4.37152641445948e-08
2561 4.3670161705478e-08
2562 4.3705749753542e-08
2563 4.36433246200352e-08
2564 4.35952147697094e-08
2565 4.36090761262165e-08
2566 4.35972378340921e-08
2567 4.35634356428238e-08
2568 4.36074991156943e-08
2569 4.36093887437039e-08
2570 4.3574043360195e-08
2571 4.35307859731893e-08
2572 4.35203256188288e-08
2573 4.35300601857591e-08
2574 4.35412281234449e-08
2575 4.34904087640575e-08
2576 4.34825818551587e-08
2577 4.35002449030719e-08
2578 4.35099506752579e-08
2579 4.34777872975189e-08
2580 4.35317461118245e-08
2581 4.339239613671e-08
2582 4.33642946866541e-08
2583 4.33939773287761e-08
2584 4.33509195225668e-08
2585 4.34785634979562e-08
2586 4.33595924214103e-08
2587 4.33520422404854e-08
2588 4.33541588300557e-08
2589 4.33839433604533e-08
2590 4.33110862836372e-08
2591 4.32655810875815e-08
2592 4.33441077589691e-08
2593 4.33118725808868e-08
2594 4.32626178756834e-08
2595 4.33334973308774e-08
2596 4.32249219777248e-08
2597 4.32433875658944e-08
2598 4.32227574513888e-08
2599 4.32806251389195e-08
2600 4.31760359269617e-08
2601 4.31771476385734e-08
2602 4.32084536399202e-08
2603 4.31917110468305e-08
2604 4.31731805825564e-08
2605 4.32125248082116e-08
2606 4.32084947270539e-08
2607 4.31571591974489e-08
2608 4.31036582675404e-08
2609 4.31148172399531e-08
2610 4.304239680053e-08
2611 4.3047965636589e-08
2612 4.29942999495836e-08
2613 4.29821964047505e-08
2614 4.30472207035848e-08
2615 4.29434737121426e-08
2616 4.29695283195031e-08
2617 4.29583121821508e-08
2618 4.29855424677328e-08
2619 4.30092733338938e-08
2620 4.29834278605767e-08
2621 4.2972142232145e-08
2622 4.29816495390867e-08
2623 4.29151406233785e-08
2624 4.29428832120493e-08
2625 4.29072070247116e-08
2626 4.2860837588421e-08
2627 4.28392195352956e-08
2628 4.28664191503003e-08
2629 4.28897359689984e-08
2630 4.28260200884267e-08
2631 4.28166232904204e-08
2632 4.28364383431301e-08
2633 4.27864860803595e-08
2634 4.28516775947685e-08
2635 4.27739261716908e-08
2636 4.27798561766224e-08
2637 4.27491059902252e-08
2638 4.27467819754668e-08
2639 4.27893762555698e-08
2640 4.27302623613457e-08
2641 4.26600006200317e-08
2642 4.26349246751556e-08
2643 4.2697798456004e-08
2644 4.26739182906033e-08
2645 4.25814334850116e-08
2646 4.26572625187305e-08
2647 4.25875918050878e-08
2648 4.2523748630785e-08
2649 4.26101376067578e-08
2650 4.2519223617532e-08
2651 4.25251839022422e-08
2652 4.24765125206505e-08
2653 4.24563074705731e-08
2654 4.24452623732918e-08
2655 4.24732300601249e-08
2656 4.24404068972706e-08
2657 4.23918555192415e-08
2658 4.24278347015417e-08
2659 4.23710078294448e-08
2660 4.23509109168663e-08
2661 4.23924680958976e-08
2662 4.23557099900051e-08
2663 4.23460798160136e-08
2664 4.23354230534301e-08
2665 4.23278912435876e-08
2666 4.23147547934377e-08
2667 4.22916017353003e-08
2668 4.23036709804592e-08
2669 4.22654448932747e-08
2670 4.22964417428062e-08
2671 4.23038416599297e-08
2672 4.22619165618698e-08
2673 4.22390054151123e-08
2674 4.2231167761031e-08
2675 4.22545751970205e-08
2676 4.22039783174455e-08
2677 4.2170441663103e-08
2678 4.22557111221522e-08
2679 4.21616906294275e-08
2680 4.21383817474918e-08
2681 4.21756820632169e-08
2682 4.21956030365322e-08
2683 4.22181806829514e-08
2684 4.21934143393088e-08
2685 4.21249278090841e-08
2686 4.21748733412386e-08
2687 4.21714877418822e-08
2688 4.21780015020801e-08
2689 4.21348735457627e-08
2690 4.21530981338236e-08
2691 4.21292336181267e-08
2692 4.20266791589796e-08
2693 4.20297632111044e-08
2694 4.20262576916741e-08
2695 4.20172497097582e-08
2696 4.20601054980097e-08
2697 4.2102712194847e-08
2698 4.20249607824275e-08
2699 4.20005142611757e-08
2700 4.19827037507758e-08
2701 4.19857079929642e-08
2702 4.20005150587599e-08
2703 4.1967423154432e-08
2704 4.19934347934969e-08
2705 4.19639129045635e-08
2706 4.19338510582179e-08
2707 4.19513752270007e-08
2708 4.19322376821185e-08
2709 4.19412296501775e-08
2710 4.19313450130687e-08
2711 4.18480363482132e-08
2712 4.19145567249757e-08
2713 4.18953215586271e-08
2714 4.18635203445206e-08
2715 4.18801669717084e-08
2716 4.18771391412776e-08
2717 4.183683376624e-08
2718 4.18244078215224e-08
2719 4.18506502093408e-08
2720 4.17508145797996e-08
2721 4.17558994580958e-08
2722 4.17551599323218e-08
2723 4.17771300185166e-08
2724 4.17231103941873e-08
2725 4.17336745286434e-08
2726 4.17109398771487e-08
2727 4.17711492257666e-08
2728 4.16830219922559e-08
2729 4.16909100735552e-08
2730 4.17526506790722e-08
2731 4.16716246878224e-08
2732 4.16687319120257e-08
2733 4.16601773149949e-08
2734 4.163964181636e-08
2735 4.16390667687949e-08
2736 4.16243318266396e-08
2737 4.16400449054777e-08
2738 4.16354593379964e-08
2739 4.16436754377969e-08
2740 4.1577648222102e-08
2741 4.15957992228044e-08
2742 4.16089030270683e-08
2743 4.15725458147165e-08
2744 4.15682001335682e-08
2745 4.15998178304733e-08
2746 4.15572985161816e-08
2747 4.15420032826574e-08
2748 4.1548291466853e-08
2749 4.15345871989103e-08
2750 4.15566344766916e-08
2751 4.1495907145972e-08
2752 4.15116045875408e-08
2753 4.15067164087901e-08
2754 4.1471340125554e-08
2755 4.15193662224311e-08
2756 4.14777582822268e-08
2757 4.14503438843639e-08
2758 4.14448632160003e-08
2759 4.14355072688721e-08
2760 4.14114701285939e-08
2761 4.14155315446862e-08
2762 4.1414877166801e-08
2763 4.13936822916128e-08
2764 4.14021721493185e-08
2765 4.13593602068829e-08
2766 4.13606769544828e-08
2767 4.13602233688692e-08
2768 4.13701498072072e-08
2769 4.1319337611867e-08
2770 4.13314623184391e-08
2771 4.13398202390169e-08
2772 4.12924948509641e-08
2773 4.12931457134391e-08
2774 4.12818025150585e-08
2775 4.13549950888381e-08
2776 4.12555871616149e-08
2777 4.1289296840219e-08
2778 4.12668411797767e-08
2779 4.12247399115273e-08
2780 4.12075194571315e-08
2781 4.11981789127935e-08
2782 4.11889621556583e-08
2783 4.12048444733415e-08
2784 4.12134100411521e-08
2785 4.11783222968864e-08
2786 4.10742671341779e-08
2787 4.11018271382346e-08
2788 4.10141299944655e-08
2789 4.11413353127443e-08
2790 4.10794592280439e-08
2791 4.10592965920387e-08
2792 4.10784002529141e-08
2793 4.10438949884906e-08
2794 4.10591714974373e-08
2795 4.10407785889078e-08
2796 4.10481886170544e-08
2797 4.10237344237174e-08
2798 4.10192539472831e-08
2799 4.10051062118555e-08
2800 4.10605740857051e-08
2801 4.09916988761694e-08
2802 4.09651121895394e-08
2803 4.10208449430627e-08
2804 4.09581999054609e-08
2805 4.09580821703059e-08
2806 4.09690452904954e-08
2807 4.09349500731793e-08
2808 4.09591874070969e-08
2809 4.09130557095239e-08
2810 4.09029867540767e-08
2811 4.09494223987394e-08
2812 4.09267356200615e-08
2813 4.08957771771412e-08
2814 4.09164576868903e-08
2815 4.08207267366123e-08
2816 4.08502255666576e-08
2817 4.08815810803986e-08
2818 4.08473925634922e-08
2819 4.08397821445305e-08
2820 4.0816812735045e-08
2821 4.08221675343157e-08
2822 4.07889864177946e-08
2823 4.08412460739527e-08
2824 4.07558198780578e-08
2825 4.07587884314609e-08
2826 4.06758467779156e-08
2827 4.07476418597952e-08
2828 4.07182855362009e-08
2829 4.07505779413952e-08
2830 4.06932745633526e-08
2831 4.06385295743661e-08
2832 4.06739666374278e-08
2833 4.06531924568299e-08
2834 4.06517023403552e-08
2835 4.0660528668468e-08
2836 4.06836493169749e-08
2837 4.0574303383778e-08
2838 4.06167152444681e-08
2839 4.05958823801456e-08
2840 4.0638741941379e-08
2841 4.05786806467745e-08
2842 4.06307235483183e-08
2843 4.05147149606933e-08
2844 4.05755729868673e-08
2845 4.05440965334236e-08
2846 4.05473884494967e-08
2847 4.05587761544979e-08
2848 4.05225374731089e-08
2849 4.05396242317835e-08
2850 4.04855535194315e-08
2851 4.05316935019329e-08
2852 4.05117315480652e-08
2853 4.05079059060398e-08
2854 4.04444562427386e-08
2855 4.04716012738504e-08
2856 4.048057598105e-08
2857 4.04364161195048e-08
2858 4.04388050352367e-08
2859 4.04036224033177e-08
2860 4.04473300115171e-08
2861 4.04428318407923e-08
2862 4.04070871677931e-08
2863 4.03892270313122e-08
2864 4.03646683775349e-08
2865 4.03609003249983e-08
2866 4.03686414713889e-08
2867 4.03513585638393e-08
2868 4.03383091214238e-08
2869 4.03436672122837e-08
2870 4.03286756132104e-08
2871 4.03433229365646e-08
2872 4.02205740357431e-08
2873 4.03228761545193e-08
2874 4.02989267822562e-08
2875 4.02352464199396e-08
2876 4.02606288005813e-08
2877 4.02590527457392e-08
2878 4.03195087574915e-08
2879 4.02302880218031e-08
2880 4.02220582635948e-08
2881 4.02324334043413e-08
2882 4.02093748572696e-08
2883 4.02285297695926e-08
2884 4.01761396950917e-08
2885 4.02546639595158e-08
2886 4.01448377171931e-08
2887 4.01439243500334e-08
2888 4.00976123060559e-08
2889 4.01173293909096e-08
2890 4.00444440700198e-08
2891 4.01185193474873e-08
2892 4.01425584577453e-08
2893 4.00350977454167e-08
2894 4.00458801266268e-08
2895 4.00535890570097e-08
2896 4.00400996518613e-08
2897 4.00560480962042e-08
2898 4.00716193862394e-08
2899 4.00237083457711e-08
2900 4.00322760825134e-08
2901 3.99985306689388e-08
2902 4.00323687745896e-08
2903 3.99571848976876e-08
2904 3.9993281632178e-08
2905 3.99590938808814e-08
2906 3.99749517860926e-08
2907 3.99615970252398e-08
2908 3.99397293939785e-08
2909 3.99107424122747e-08
2910 3.99262266288503e-08
2911 3.98859180688049e-08
2912 3.98941117971674e-08
2913 3.98442471816196e-08
2914 3.98462421440371e-08
2915 3.98264940599802e-08
2916 3.98382953417098e-08
2917 3.97912297369629e-08
2918 3.98768044895803e-08
2919 3.97585067162254e-08
2920 3.97743694726671e-08
2921 3.97995625309733e-08
2922 3.97556415308031e-08
2923 3.97976677568579e-08
2924 3.9759742993084e-08
2925 3.97259635107616e-08
2926 3.97546070480814e-08
2927 3.96995953977353e-08
2928 3.97108552103731e-08
2929 3.97124434616813e-08
2930 3.97813494164012e-08
2931 3.96692065169901e-08
2932 3.96453876838621e-08
2933 3.96191837683801e-08
2934 3.96394566131164e-08
2935 3.96545780869673e-08
2936 3.96218489751021e-08
2937 3.9616693140232e-08
2938 3.95934449706203e-08
2939 3.96089121501575e-08
2940 3.95682040998935e-08
2941 3.95714697898342e-08
2942 3.96702490021994e-08
2943 3.95404950435108e-08
2944 3.95379627420311e-08
2945 3.95690944561267e-08
2946 3.95415357505868e-08
2947 3.94864965791442e-08
2948 3.94520316540081e-08
2949 3.94952853870478e-08
2950 3.94508835217522e-08
2951 3.95285712606608e-08
2952 3.94303319133371e-08
2953 3.9492430836674e-08
2954 3.94052027914427e-08
2955 3.94402663292936e-08
2956 3.94275443742487e-08
2957 3.95378555175796e-08
2958 3.93568142822431e-08
2959 3.9355948468156e-08
2960 3.94142859398983e-08
2961 3.94380176711451e-08
2962 3.93222501600121e-08
2963 3.93320663523156e-08
2964 3.93502921056665e-08
2965 3.93798002704671e-08
2966 3.93057112795248e-08
2967 3.93322494822712e-08
2968 3.92725907687463e-08
2969 3.94022600946897e-08
2970 3.92483732731819e-08
2971 3.92802794006997e-08
2972 3.92626436465093e-08
2973 3.92631728072246e-08
2974 3.92891048051069e-08
2975 3.92307526162483e-08
2976 3.92038972663045e-08
2977 3.92870811598556e-08
2978 3.91948489486538e-08
2979 3.92090942789025e-08
2980 3.91936227988055e-08
2981 3.92445703525368e-08
2982 3.9125315314692e-08
2983 3.9161424691514e-08
2984 3.91294900570216e-08
2985 3.91464143145726e-08
2986 3.91717376917455e-08
2987 3.91412886422415e-08
2988 3.90662622020699e-08
2989 3.91349328641155e-08
2990 3.90466119881694e-08
2991 3.90714626128386e-08
2992 3.90880471208277e-08
2993 3.91118726241757e-08
2994 3.90031843995331e-08
2995 3.90435067796346e-08
2996 3.90111516388458e-08
2997 3.9019097425097e-08
2998 3.90340616913676e-08
2999 3.89825730611193e-08
3000 3.90076212966051e-08
3001 3.89707189309974e-08
3002 3.89455614584477e-08
3003 3.89949948065294e-08
3004 3.88964304249839e-08
3005 3.89500574318191e-08
3006 3.88989500788028e-08
3007 3.8987640976984e-08
3008 3.88665668360488e-08
3009 3.88695930499949e-08
3010 3.89331662802306e-08
3011 3.88669394286723e-08
3012 3.89503730335861e-08
3013 3.88776567987747e-08
3014 3.88762924128372e-08
3015 3.88050512576399e-08
3016 3.88415189132019e-08
3017 3.88505294015573e-08
3018 3.89295022902303e-08
3019 3.8826999203323e-08
3020 3.8858988624213e-08
3021 3.87442975657848e-08
3022 3.87597721651645e-08
3023 3.87382070954345e-08
3024 3.87298559054727e-08
3025 3.87146429048357e-08
3026 3.87517218474187e-08
3027 3.87742358789467e-08
3028 3.8745835130527e-08
3029 3.86951195121554e-08
3030 3.8713132532564e-08
3031 3.86999454047299e-08
3032 3.86192847550149e-08
3033 3.8666099886342e-08
3034 3.86348539098691e-08
3035 3.86125138902571e-08
3036 3.86280282160811e-08
3037 3.86518194659402e-08
3038 3.85808859721948e-08
3039 3.86229317221876e-08
3040 3.85563591080995e-08
3041 3.85750563989262e-08
3042 3.85685220525289e-08
3043 3.85526868917907e-08
3044 3.85592096066034e-08
3045 3.85324666503095e-08
3046 3.85145745678273e-08
3047 3.85386547812061e-08
3048 3.85303150256533e-08
3049 3.85003759166835e-08
3050 3.84692031207834e-08
3051 3.84832586881601e-08
3052 3.84585423915951e-08
3053 3.85340481265928e-08
3054 3.84074692174607e-08
3055 3.84219219764503e-08
3056 3.84236056305554e-08
3057 3.85003781655513e-08
3058 3.84238089541356e-08
3059 3.83846665990006e-08
3060 3.8380529838733e-08
3061 3.83683619844533e-08
3062 3.83706644573323e-08
3063 3.83554264331565e-08
3064 3.84099313723851e-08
3065 3.83131286731953e-08
3066 3.84038802803133e-08
3067 3.82906932436811e-08
3068 3.82921577077866e-08
3069 3.84445331604866e-08
3070 3.82478378888607e-08
3071 3.8259341229363e-08
3072 3.82591858674175e-08
3073 3.82619056953359e-08
3074 3.82936125102873e-08
3075 3.82811756747259e-08
3076 3.82056131460473e-08
3077 3.81988997784077e-08
3078 3.82023723002334e-08
3079 3.82130173175454e-08
3080 3.8207517967237e-08
3081 3.81452856892395e-08
3082 3.81431965106316e-08
3083 3.8176277879387e-08
3084 3.81127548081395e-08
3085 3.81858954430214e-08
3086 3.80821896044381e-08
3087 3.81181502859818e-08
3088 3.80716513266321e-08
3089 3.80933230665903e-08
3090 3.81256623889215e-08
3091 3.80440369713853e-08
3092 3.80469873366707e-08
3093 3.80239707329366e-08
3094 3.81089260521605e-08
3095 3.80378684550209e-08
3096 3.79855794161443e-08
3097 3.80045631569459e-08
3098 3.79962434191583e-08
3099 3.80463528504293e-08
3100 3.79401883154884e-08
3101 3.79770533704971e-08
3102 3.79630047220303e-08
3103 3.79232085485626e-08
3104 3.80110076712015e-08
3105 3.79045440155323e-08
3106 3.79838943960209e-08
3107 3.78818991535468e-08
3108 3.79187304346829e-08
3109 3.79350369374976e-08
3110 3.78772678182315e-08
3111 3.78451931961621e-08
3112 3.78718335269923e-08
3113 3.78517470700501e-08
3114 3.78362156379097e-08
3115 3.77936004660739e-08
3116 3.78385889465704e-08
3117 3.7809349690221e-08
3118 3.78879299862689e-08
3119 3.77615689597377e-08
3120 3.78383707815288e-08
3121 3.77595394756014e-08
3122 3.77925241572541e-08
3123 3.77416316315049e-08
3124 3.77953327976854e-08
3125 3.77414149710376e-08
3126 3.77410655971744e-08
3127 3.77277213221561e-08
3128 3.7708651921875e-08
3129 3.77675630680585e-08
3130 3.76821819472184e-08
3131 3.7695384200731e-08
3132 3.76821880543332e-08
3133 3.76713667407103e-08
3134 3.76875171888003e-08
3135 3.77007019416453e-08
3136 3.7614161183086e-08
3137 3.76946558731106e-08
3138 3.75968067434229e-08
3139 3.76605826790666e-08
3140 3.7692126245048e-08
3141 3.76812235209911e-08
3142 3.75138776806949e-08
3143 3.75638502436715e-08
3144 3.75516346782234e-08
3145 3.75704523403186e-08
3146 3.74209326992059e-08
3147 3.7457067330493e-08
3148 3.74434795542555e-08
3149 3.74243955771902e-08
3150 3.75301054660326e-08
3151 3.73946777347811e-08
3152 3.74587499560874e-08
3153 3.73824193360406e-08
3154 3.74254440114186e-08
3155 3.75481438208425e-08
3156 3.75137985937357e-08
3157 3.74531534941269e-08
3158 3.74779549989768e-08
3159 3.73857291986468e-08
3160 3.74231911042244e-08
3161 3.74373169798048e-08
3162 3.74353009355843e-08
3163 3.74183495441827e-08
3164 3.74286942328439e-08
3165 3.73598001104369e-08
3166 3.74691617821554e-08
3167 3.73319494482161e-08
3168 3.73804539748335e-08
3169 3.73379698981324e-08
3170 3.73646098985603e-08
3171 3.74062355135152e-08
3172 3.73171928753635e-08
3173 3.73143978684709e-08
3174 3.7309575921185e-08
3175 3.72901806908743e-08
3176 3.73443237062787e-08
3177 3.72810845519211e-08
3178 3.72801344923346e-08
3179 3.72336541456519e-08
3180 3.73047287673955e-08
3181 3.72024493202616e-08
3182 3.72544668110208e-08
3183 3.71085649799596e-08
3184 3.71846619380278e-08
3185 3.71760251667297e-08
3186 3.72106837804864e-08
3187 3.71334825128855e-08
3188 3.71204186002672e-08
3189 3.71251897419711e-08
3190 3.72298519053516e-08
3191 3.71027589078921e-08
3192 3.70733911818633e-08
3193 3.71529963913986e-08
3194 3.70950633143963e-08
3195 3.71789536970368e-08
3196 3.70831092268986e-08
3197 3.70832108185226e-08
3198 3.70677460157509e-08
3199 3.7066950069331e-08
3200 3.71026507259842e-08
3201 3.69632793777441e-08
3202 3.70307795662939e-08
3203 3.70022772440137e-08
3204 3.70070451189974e-08
3205 3.71078309324702e-08
3206 3.69885761362099e-08
3207 3.7011283032129e-08
3208 3.70045042092926e-08
3209 3.70179107491708e-08
3210 3.7010639577062e-08
3211 3.69715185666308e-08
3212 3.69464301321898e-08
3213 3.69590577342649e-08
3214 3.69364248733461e-08
3215 3.69712595915672e-08
3216 3.69118895893195e-08
3217 3.69342005974715e-08
3218 3.68938221129156e-08
3219 3.6899126916623e-08
3220 3.69230900787443e-08
3221 3.68737325082691e-08
3222 3.68581858687378e-08
3223 3.68453344403719e-08
3224 3.68415241602804e-08
3225 3.69005351732454e-08
3226 3.68153084107092e-08
3227 3.68431162538485e-08
3228 3.68065301472598e-08
3229 3.6833390113955e-08
3230 3.68049060845976e-08
3231 3.68154958785283e-08
3232 3.6758634815115e-08
3233 3.68028323673997e-08
3234 3.67249761250577e-08
3235 3.67782494397062e-08
3236 3.67695229606113e-08
3237 3.67017198428243e-08
3238 3.67161493493029e-08
3239 3.67128483116375e-08
3240 3.67214675449645e-08
3241 3.67550384599014e-08
3242 3.66759300494124e-08
3243 3.66853698565706e-08
3244 3.66716398421119e-08
3245 3.66832027332009e-08
3246 3.66956613344627e-08
3247 3.66379957430496e-08
3248 3.66473771507003e-08
3249 3.66274632970232e-08
3250 3.66125470687706e-08
3251 3.66543051537604e-08
3252 3.65898920353658e-08
3253 3.65903800982892e-08
3254 3.65714469889866e-08
3255 3.6583720241623e-08
3256 3.65534785213839e-08
3257 3.66398093625264e-08
3258 3.65100997026246e-08
3259 3.65046888681775e-08
3260 3.65504763060187e-08
3261 3.6570255808499e-08
3262 3.67442403508278e-08
3263 3.65519251364077e-08
3264 3.65297664721709e-08
3265 3.65040247061188e-08
3266 3.65085310960467e-08
3267 3.65736297887054e-08
3268 3.64231084617472e-08
3269 3.64389140763421e-08
3270 3.63969606809178e-08
3271 3.64235490923903e-08
3272 3.64484110875196e-08
3273 3.63515713672768e-08
3274 3.6335772254148e-08
3275 3.6320275262014e-08
3276 3.63173046125098e-08
3277 3.62965352334044e-08
3278 3.63336648110391e-08
3279 3.62929059551931e-08
3280 3.62418795045727e-08
3281 3.62361000050271e-08
3282 3.62317716806615e-08
3283 3.62706323748796e-08
3284 3.62097811770212e-08
3285 3.61676198021854e-08
3286 3.61814970819552e-08
3287 3.61915691033943e-08
3288 3.62182080912987e-08
3289 3.61607693317723e-08
3290 3.60910288765837e-08
3291 3.60865135942134e-08
3292 3.60860579391442e-08
3293 3.60753215371545e-08
3294 3.60937831160868e-08
3295 3.6032263491137e-08
3296 3.60215743686609e-08
3297 3.60778023313912e-08
3298 3.60981831430962e-08
3299 3.60645244565916e-08
3300 3.60053268781968e-08
3301 3.60162490071758e-08
3302 3.59840082673202e-08
3303 3.60077256544855e-08
3304 3.60034945448007e-08
3305 3.59726113359216e-08
3306 3.59998571468623e-08
3307 3.59429475960127e-08
3308 3.59298545209441e-08
3309 3.58957178292485e-08
3310 3.5971776950916e-08
3311 3.58847809245333e-08
3312 3.58725571807383e-08
3313 3.59075609619453e-08
3314 3.58639047632892e-08
3315 3.58747419966221e-08
3316 3.5848353325818e-08
3317 3.58231424755218e-08
3318 3.5801837782401e-08
3319 3.58270612643707e-08
3320 3.57954503211744e-08
3321 3.57808433584239e-08
3322 3.58331406999923e-08
3323 3.57722842121433e-08
3324 3.57432725639484e-08
3325 3.57611866732555e-08
3326 3.57537986666046e-08
3327 3.57378243514006e-08
3328 3.57252678693243e-08
3329 3.57582142456181e-08
3330 3.56959999070483e-08
3331 3.56904755456355e-08
3332 3.56749589069949e-08
3333 3.56484462447071e-08
3334 3.57335135987569e-08
3335 3.56491026298755e-08
3336 3.56322523948904e-08
3337 3.56304277104869e-08
3338 3.56591422772112e-08
3339 3.55860160734522e-08
3340 3.55713222983667e-08
3341 3.55375763003707e-08
3342 3.55644760485774e-08
3343 3.56325840815686e-08
3344 3.55233229676344e-08
3345 3.54824502650786e-08
3346 3.54876926618175e-08
3347 3.54877149018051e-08
3348 3.55320571649997e-08
3349 3.5449276996502e-08
3350 3.54333238750115e-08
3351 3.54405321658646e-08
3352 3.54722076902902e-08
3353 3.54144836620662e-08
3354 3.53873049885323e-08
3355 3.53912820632019e-08
3356 3.53751403121549e-08
3357 3.54151117409884e-08
3358 3.53614527597301e-08
3359 3.52953181135263e-08
3360 3.53335499916341e-08
3361 3.53088156170855e-08
3362 3.53378476489752e-08
3363 3.526684852595e-08
3364 3.52728519157353e-08
3365 3.52455164005505e-08
3366 3.53025743624613e-08
3367 3.51611028399645e-08
3368 3.51709750034956e-08
3369 3.51752420044704e-08
3370 3.51603289399094e-08
3371 3.51745030187089e-08
3372 3.51677948984275e-08
3373 3.51038661570158e-08
3374 3.51044661286437e-08
3375 3.50664360446729e-08
3376 3.51061225316585e-08
3377 3.50932759651812e-08
3378 3.5078749627715e-08
3379 3.49848376419715e-08
3380 3.50402074786871e-08
3381 3.49813181657055e-08
3382 3.50570899243507e-08
3383 3.49644639943136e-08
3384 3.4958397376883e-08
3385 3.49523262688223e-08
3386 3.50319805217936e-08
3387 3.49236385996932e-08
3388 3.49832714512388e-08
3389 3.49012270959292e-08
3390 3.49425967005601e-08
3391 3.48795354945253e-08
3392 3.48954727193984e-08
3393 3.48894039632341e-08
3394 3.49110428050636e-08
3395 3.48133215108959e-08
3396 3.4829753417398e-08
3397 3.49077898533778e-08
3398 3.48218961345736e-08
3399 3.47686443387119e-08
3400 3.47929188198037e-08
3401 3.48036270096941e-08
3402 3.47594651461947e-08
3403 3.47357051175834e-08
3404 3.47002478715552e-08
3405 3.47148333368352e-08
3406 3.47123924129278e-08
3407 3.46760009612979e-08
3408 3.47068318955479e-08
3409 3.46079414121903e-08
3410 3.45987339631648e-08
3411 3.46076492938607e-08
3412 3.45790130520385e-08
3413 3.46066171470483e-08
3414 3.45244106991771e-08
3415 3.4549327654787e-08
3416 3.4519598532512e-08
3417 3.45257169129809e-08
3418 3.45112342312603e-08
3419 3.44577237676447e-08
3420 3.44450100620008e-08
3421 3.44486022196833e-08
3422 3.4429373680922e-08
3423 3.44335829005615e-08
3424 3.44043910711633e-08
3425 3.44102481939501e-08
3426 3.44112231758231e-08
3427 3.43652824614793e-08
3428 3.43608960733377e-08
3429 3.43966952929264e-08
3430 3.43327735148335e-08
3431 3.43198193757388e-08
3432 3.43073334434507e-08
3433 3.44246725489938e-08
3434 3.43522600942947e-08
3435 3.42669753923275e-08
3436 3.42862200071181e-08
3437 3.42592321782576e-08
3438 3.42867512621581e-08
3439 3.41159322889695e-08
3440 3.41255756008962e-08
3441 3.4170170629011e-08
3442 3.40819832675976e-08
3443 3.4074240936377e-08
3444 3.40760743089419e-08
3445 3.4038570621675e-08
3446 3.40902289632794e-08
3447 3.40283922177775e-08
3448 3.40720054019528e-08
3449 3.41880204679512e-08
3450 3.41082216053223e-08
3451 3.41116427158283e-08
3452 3.40541599825883e-08
3453 3.40081324097241e-08
3454 3.40307040875842e-08
3455 3.39710842087726e-08
3456 3.39839564205136e-08
3457 3.40727653238559e-08
3458 3.38204211978166e-08
3459 3.37956369076409e-08
3460 3.38166920101912e-08
3461 3.38007390023876e-08
3462 3.3867790261155e-08
3463 3.39071508790312e-08
3464 3.37632801894472e-08
3465 3.37596189741163e-08
3466 3.36895416879202e-08
3467 3.37000754626615e-08
3468 3.36360364432409e-08
3469 3.3563316970131e-08
3470 3.35776406839017e-08
3471 3.355234002278e-08
3472 3.35636745401047e-08
3473 3.35173149146328e-08
3474 3.35248343414918e-08
3475 3.34595903641599e-08
3476 3.39324653086948e-08
3477 3.38070005110325e-08
3478 3.37764564886101e-08
3479 3.36662012969668e-08
3480 3.36310967892217e-08
3481 3.35485256890422e-08
3482 3.35769385859663e-08
3483 3.35630731953529e-08
3484 3.36370539990583e-08
3485 3.34609437899758e-08
3486 3.355039298647e-08
3487 3.35539018454511e-08
3488 3.34875383920519e-08
3489 3.345642687691e-08
3490 3.3471073756175e-08
3491 3.34156414698583e-08
3492 3.34028315585044e-08
3493 3.336228920503e-08
3494 3.33249018460435e-08
3495 3.3331128884484e-08
3496 3.32905436479791e-08
3497 3.32476463817954e-08
3498 3.32636738047398e-08
3499 3.32167378349624e-08
3500 3.32325816057022e-08
3501 3.31628658258865e-08
3502 3.31840488900781e-08
3503 3.31326229954243e-08
3504 3.31550810930281e-08
3505 3.30503351673173e-08
3506 3.3109584681057e-08
3507 3.30464289373111e-08
3508 3.30661551615208e-08
3509 3.29723554877148e-08
3510 3.29860818393257e-08
3511 3.29198667596842e-08
3512 3.29264210225944e-08
3513 3.28608929969221e-08
3514 3.29122079882893e-08
3515 3.28086560763552e-08
3516 3.28559012441332e-08
3517 3.27626947793647e-08
3518 3.28235477500272e-08
3519 3.2732612622155e-08
3520 3.27771595767246e-08
3521 3.26828071770535e-08
3522 3.27360085208284e-08
3523 3.2659854444006e-08
3524 3.26911523576001e-08
3525 3.26278891709109e-08
3526 3.26750976054058e-08
3527 3.26093983655795e-08
3528 3.26167283315471e-08
3529 3.2574232820437e-08
3530 3.25735967692253e-08
3531 3.25217819590051e-08
3532 3.2574711223532e-08
3533 3.24713784536357e-08
3534 3.25062801778131e-08
3535 3.2458910036226e-08
3536 3.24864849883255e-08
3537 3.24137950240555e-08
3538 3.25029443235536e-08
3539 3.23819087615362e-08
3540 3.24416963781715e-08
3541 3.23462229605553e-08
3542 3.23930509189552e-08
3543 3.2303193878036e-08
3544 3.22965115628904e-08
3545 3.22540335719879e-08
3546 3.2211430635698e-08
3547 3.21795892475052e-08
3548 3.218947706074e-08
3549 3.21411630164192e-08
3550 3.2140294548455e-08
3551 3.21157088905011e-08
3552 3.21089989956391e-08
3553 3.2056144195991e-08
3554 3.20732937133528e-08
3555 3.20260209889511e-08
3556 3.20382766449967e-08
3557 3.19952966663095e-08
3558 3.20315977067054e-08
3559 3.1953924066741e-08
3560 3.19778375761359e-08
3561 3.19326810398479e-08
3562 3.19552669356682e-08
3563 3.18916310995121e-08
3564 3.19338420986526e-08
3565 3.187422958284e-08
3566 3.18823687752712e-08
3567 3.19339046033207e-08
3568 3.19238250128251e-08
3569 3.18104994789792e-08
3570 3.17931208186906e-08
3571 3.18279093960427e-08
3572 3.18465710709148e-08
3573 3.17757559731291e-08
3574 3.17789406949487e-08
3575 3.16715883155183e-08
3576 3.1724377233644e-08
3577 3.16718177337805e-08
3578 3.17008792567464e-08
3579 3.16751141902216e-08
3580 3.16537483797674e-08
3581 3.18176497557943e-08
3582 3.18257031093339e-08
3583 3.16797518742362e-08
3584 3.16786848379991e-08
3585 3.15705751496864e-08
3586 3.16147073196049e-08
3587 3.15233224146283e-08
3588 3.15297156934236e-08
3589 3.1480178947163e-08
3590 3.14709703381766e-08
3591 3.14386554265411e-08
3592 3.14435584627404e-08
3593 3.14242897250239e-08
3594 3.14528945981607e-08
3595 3.13730575030036e-08
3596 3.14072476435001e-08
3597 3.13294283635912e-08
3598 3.13605261084149e-08
3599 3.13189369389732e-08
3600 3.12773603088345e-08
3601 3.12411182807892e-08
3602 3.12419255958929e-08
3603 3.12005770588541e-08
3604 3.12385763727718e-08
3605 3.11587858146112e-08
3606 3.11781463349092e-08
3607 3.11256327609755e-08
3608 3.11283415417307e-08
3609 3.11049993388934e-08
3610 3.10216871817204e-08
3611 3.10401177134878e-08
3612 3.10160318921504e-08
3613 3.09993382590079e-08
3614 3.10320083194426e-08
3615 3.09520843746469e-08
3616 3.09598535928046e-08
3617 3.0931773718379e-08
3618 3.09627962540304e-08
3619 3.08688891923481e-08
3620 3.08838094387198e-08
3621 3.08566577729152e-08
3622 3.08380020914711e-08
3623 3.08679166423076e-08
3624 3.08145429688267e-08
3625 3.07788601858761e-08
3626 3.07871763993717e-08
3627 3.07374010990458e-08
3628 3.07776144694571e-08
3629 3.07250022011374e-08
3630 3.07104066123998e-08
3631 3.07091586293495e-08
3632 3.06530671618788e-08
3633 3.06507034117942e-08
3634 3.06529543774303e-08
3635 3.06159711076504e-08
3636 3.0664374989442e-08
3637 3.05979668731027e-08
3638 3.05512260485585e-08
3639 3.05684826997776e-08
3640 3.05339978243779e-08
3641 3.05367132789058e-08
3642 3.05015551997911e-08
3643 3.05064533261401e-08
3644 3.04826576833506e-08
3645 3.0460949133726e-08
3646 3.04892183375927e-08
3647 3.04677025582123e-08
3648 3.04481917208221e-08
3649 3.04003390692031e-08
3650 3.03996491215486e-08
3651 3.03402755275073e-08
3652 3.03393582985478e-08
3653 3.03058683357449e-08
3654 3.02944324239718e-08
3655 3.0275584443018e-08
3656 3.02367415567772e-08
3657 3.02049626075274e-08
3658 3.02258955713342e-08
3659 3.01607058403874e-08
3660 3.02350793077721e-08
3661 3.01388740435726e-08
3662 3.01210758966164e-08
3663 3.00824177834613e-08
3664 3.0062836904321e-08
3665 3.0068032179642e-08
3666 3.00481659305518e-08
3667 3.00459627737837e-08
3668 3.00148623200158e-08
3669 2.99825946754595e-08
3670 2.99979317777144e-08
3671 2.99249675492774e-08
3672 2.9949000659002e-08
3673 2.99400416405859e-08
3674 2.98240979379472e-08
3675 2.98114240226965e-08
3676 2.97717524819774e-08
3677 2.97873585619612e-08
3678 2.98227142661034e-08
3679 2.97169274023901e-08
3680 2.96600602496255e-08
3681 2.96794009901902e-08
3682 2.9669057262538e-08
3683 2.97168412348725e-08
3684 2.96210626160587e-08
3685 2.96039915852475e-08
3686 2.96180866889983e-08
3687 2.95977981004114e-08
3688 2.96055684447794e-08
3689 2.9592890214758e-08
3690 2.95143879522897e-08
3691 2.9533065971421e-08
3692 2.95396599003794e-08
3693 2.94316633375757e-08
3694 2.94970130489958e-08
3695 2.94239593880974e-08
3696 2.94415404553661e-08
3697 2.94556165467696e-08
3698 2.94703055701007e-08
3699 2.93634239838525e-08
3700 2.93313541988027e-08
3701 2.93306782719327e-08
3702 2.93385280727421e-08
3703 2.93781766700363e-08
3704 2.93124263333056e-08
3705 2.9284246211958e-08
3706 2.93095703174373e-08
3707 2.9277907662717e-08
3708 2.93168737748317e-08
3709 2.92493124138815e-08
3710 2.92293354853257e-08
3711 2.9269755563277e-08
3712 2.91934206266831e-08
3713 2.91894092434575e-08
3714 2.91999011992061e-08
3715 2.91605945950124e-08
3716 2.91551294733239e-08
3717 2.9145054242008e-08
3718 2.91302405912575e-08
3719 2.91084266859087e-08
3720 2.91593166874549e-08
3721 2.90503708146872e-08
3722 2.90773583504489e-08
3723 2.90282018173826e-08
3724 2.9003090615376e-08
3725 2.90253037196209e-08
3726 2.89808176674455e-08
3727 2.89375927611246e-08
3728 2.89650697524735e-08
3729 2.89280269374359e-08
3730 2.89874980516913e-08
3731 2.89226399257103e-08
3732 2.88776646559796e-08
3733 2.89851727259816e-08
3734 2.88845120053338e-08
3735 2.88910241934559e-08
3736 2.88374668677704e-08
3737 2.88084402022548e-08
3738 2.88940607138244e-08
3739 2.88318990122605e-08
3740 2.88822659779697e-08
3741 2.87712347919467e-08
3742 2.88110209947234e-08
3743 2.87926860593757e-08
3744 2.87632448419117e-08
3745 2.87405277106245e-08
3746 2.87013857889207e-08
3747 2.8702991125229e-08
3748 2.87466090789223e-08
3749 2.86684816774851e-08
3750 2.86425714488558e-08
3751 2.86688751902631e-08
3752 2.87089902446525e-08
3753 2.8553180776214e-08
3754 2.85632454328777e-08
3755 2.85956191969916e-08
3756 2.86225386183503e-08
3757 2.856640378468e-08
3758 2.85486248028377e-08
3759 2.86072034327844e-08
3760 2.85349626665266e-08
3761 2.85168578297856e-08
3762 2.84727395847284e-08
3763 2.84449577900858e-08
3764 2.85304364044947e-08
3765 2.84297146446733e-08
3766 2.84292411372178e-08
3767 2.83783006675975e-08
3768 2.84733715290031e-08
3769 2.83455893859141e-08
3770 2.8376744513281e-08
3771 2.83520258665959e-08
3772 2.83868305288593e-08
3773 2.83030877090695e-08
3774 2.83481603711166e-08
3775 2.83169664321292e-08
3776 2.83299277725746e-08
3777 2.83508347767025e-08
3778 2.82898095305129e-08
3779 2.82565435414384e-08
3780 2.82707914740499e-08
3781 2.82590855791298e-08
3782 2.83425710954077e-08
3783 2.82277961076716e-08
3784 2.82478281379284e-08
3785 2.8189947235191e-08
3786 2.81902141443524e-08
3787 2.81369777290763e-08
3788 2.81708681537651e-08
3789 2.80898534006369e-08
3790 2.81357884981404e-08
3791 2.80938967813782e-08
3792 2.80777201639992e-08
3793 2.80631948585963e-08
3794 2.80510665895406e-08
3795 2.80582568645826e-08
3796 2.79997590109105e-08
3797 2.79632239186256e-08
3798 2.79816890289553e-08
3799 2.79196061034881e-08
3800 2.79893347379101e-08
3801 2.78866773149034e-08
3802 2.79296417691199e-08
3803 2.79283212147874e-08
3804 2.79206575504176e-08
3805 2.78837061822301e-08
3806 2.79108080567525e-08
3807 2.7816893009458e-08
3808 2.80001244670203e-08
3809 2.79297689367297e-08
3810 2.79213463603156e-08
3811 2.78588050077744e-08
3812 2.78060041596362e-08
3813 2.77563372925727e-08
3814 2.77816979430412e-08
3815 2.77168455085075e-08
3816 2.77987133117108e-08
3817 2.77033314546316e-08
3818 2.77202494451956e-08
3819 2.76022909506679e-08
3820 2.76873482460971e-08
3821 2.75568120917669e-08
3822 2.75946439298735e-08
3823 2.7598093994996e-08
3824 2.75713961253032e-08
3825 2.75108201517327e-08
3826 2.75892090240149e-08
3827 2.74676521048178e-08
3828 2.74848462682442e-08
3829 2.74903806367632e-08
3830 2.74105480304598e-08
3831 2.74283435608425e-08
3832 2.73853467804841e-08
3833 2.73402347481522e-08
3834 2.74046246193649e-08
3835 2.73030382373918e-08
3836 2.73468309268665e-08
3837 2.73125667442642e-08
3838 2.73745972227957e-08
3839 2.72525666726864e-08
3840 2.73193429860541e-08
3841 2.716735183661e-08
3842 2.71978553945118e-08
3843 2.71734806229773e-08
3844 2.71795743200798e-08
3845 2.70908363910749e-08
3846 2.71804398090936e-08
3847 2.70983765826571e-08
3848 2.7179055799742e-08
3849 2.70722030561998e-08
3850 2.70771949306692e-08
3851 2.7097846427182e-08
3852 2.70438869760525e-08
3853 2.69806834207742e-08
3854 2.70026851501015e-08
3855 2.69720451466782e-08
3856 2.70087286109444e-08
3857 2.69383342068608e-08
3858 2.69777543770999e-08
3859 2.69012096163479e-08
3860 2.6976800248768e-08
3861 2.69212487449266e-08
3862 2.69112747748323e-08
3863 2.685661875379e-08
3864 2.6876282904098e-08
3865 2.68835281032054e-08
3866 2.68645076992868e-08
3867 2.68773372136266e-08
3868 2.68699479146761e-08
3869 2.68115411063974e-08
3870 2.68803948015872e-08
3871 2.67264992510619e-08
3872 2.68272634773581e-08
3873 2.67206131496778e-08
3874 2.66741228873713e-08
3875 2.66279567666672e-08
3876 2.66829762400889e-08
3877 2.66753817506071e-08
3878 2.66425893160616e-08
3879 2.65550787572622e-08
3880 2.6606525968198e-08
3881 2.65148881881672e-08
3882 2.65984288319032e-08
3883 2.65034119637519e-08
3884 2.64843967183737e-08
3885 2.64699476337782e-08
3886 2.64907031377604e-08
3887 2.64214895695503e-08
3888 2.64754228549435e-08
3889 2.63594209188867e-08
3890 2.64035252675043e-08
3891 2.62553164507295e-08
3892 2.62994954161044e-08
3893 2.62602006406709e-08
3894 2.62743632815088e-08
3895 2.62533441190982e-08
3896 2.62495040841415e-08
3897 2.61857793839582e-08
3898 2.6199117907133e-08
3899 2.61700745163296e-08
3900 2.62750897777053e-08
3901 2.61782971486824e-08
3902 2.62327698905551e-08
3903 2.61086403297028e-08
3904 2.61695302468112e-08
3905 2.61532662646502e-08
3906 2.61536123407069e-08
3907 2.60969419594304e-08
3908 2.61103761447501e-08
3909 2.60986365940852e-08
3910 2.60435728076658e-08
3911 2.59143333032341e-08
3912 2.59734853109705e-08
3913 2.59438492102859e-08
3914 2.59444657855212e-08
3915 2.58638533212618e-08
3916 2.59607900821024e-08
3917 2.58679771736325e-08
3918 2.5970743599224e-08
3919 2.58290660104521e-08
3920 2.58442431828598e-08
3921 2.57744767946377e-08
3922 2.58703586117903e-08
3923 2.57448436133956e-08
3924 2.5766723530829e-08
3925 2.56593807570482e-08
3926 2.5732685153379e-08
3927 2.57517777129124e-08
3928 2.57209541185688e-08
3929 2.56150143194134e-08
3930 2.568792113955e-08
3931 2.55836182585512e-08
3932 2.56997616867594e-08
3933 2.55389633885272e-08
3934 2.5616029133424e-08
3935 2.55757978981919e-08
3936 2.56691045796842e-08
3937 2.55139467331134e-08
3938 2.55898828811496e-08
3939 2.54849797682866e-08
3940 2.55623389877613e-08
3941 2.55155017674369e-08
3942 2.5574027806563e-08
3943 2.54772639882717e-08
3944 2.55479387414326e-08
3945 2.53788933317622e-08
3946 2.54757420732332e-08
3947 2.53765453717136e-08
3948 2.55431316213972e-08
3949 2.53353633246078e-08
3950 2.54525419496332e-08
3951 2.53027600916056e-08
3952 2.54359731606257e-08
3953 2.53385147956919e-08
3954 2.53054118388718e-08
3955 2.52615658675737e-08
3956 2.53198764665896e-08
3957 2.52270450351588e-08
3958 2.52669319102949e-08
3959 2.5143874991862e-08
3960 2.51884926827373e-08
3961 2.50323101260719e-08
3962 2.51255055072619e-08
3963 2.51221094762499e-08
3964 2.50817078297771e-08
3965 2.4975093762869e-08
3966 2.50377729225093e-08
3967 2.49019418161467e-08
3968 2.50437654480962e-08
3969 2.48650715608534e-08
3970 2.49054970318241e-08
3971 2.48043074506654e-08
3972 2.49367450946281e-08
3973 2.48454299933698e-08
3974 2.48415039552086e-08
3975 2.48248269807405e-08
3976 2.48483815461498e-08
3977 2.48263214466249e-08
3978 2.48008931267307e-08
3979 2.47013878382774e-08
3980 2.47487334403829e-08
3981 2.46230729379349e-08
3982 2.47612088735139e-08
3983 2.45922158930156e-08
3984 2.46490979494141e-08
3985 2.45620829684157e-08
3986 2.46250750572941e-08
3987 2.46364507781038e-08
3988 2.4584540963879e-08
3989 2.44879689841326e-08
3990 2.4630029280992e-08
3991 2.44475069592909e-08
3992 2.45428630822886e-08
3993 2.44748151256857e-08
3994 2.44922534617942e-08
3995 2.44195291392302e-08
3996 2.44908300759405e-08
3997 2.43650599704992e-08
3998 2.45249743118237e-08
3999 2.44132092985794e-08
4000 2.45252595201251e-08
4001 2.43758066176269e-08
4002 2.44491801240088e-08
4003 2.431508881795e-08
4004 2.44471537245161e-08
4005 2.43163401316693e-08
4006 2.43895535660954e-08
4007 2.43519996185881e-08
4008 2.43225727931673e-08
4009 2.42821033271667e-08
4010 2.43247819939896e-08
4011 2.42591214947296e-08
4012 2.42990134848853e-08
4013 2.41297395353612e-08
4014 2.42657160329784e-08
4015 2.4157825373905e-08
4016 2.43388905580844e-08
4017 2.41926001152137e-08
4018 2.43838950915176e-08
4019 2.41817190351767e-08
4020 2.43115560616403e-08
4021 2.42103405598826e-08
4022 2.4259671847382e-08
4023 2.41442389485869e-08
4024 2.42586369463282e-08
4025 2.4114502161332e-08
4026 2.42280055493538e-08
4027 2.41419708029156e-08
4028 2.42150829103949e-08
4029 2.40723777542584e-08
4030 2.41698221303821e-08
4031 2.40941380029724e-08
4032 2.41518103605287e-08
4033 2.4014697550534e-08
4034 2.41417433235469e-08
4035 2.40094426446902e-08
4036 2.41551509203575e-08
4037 2.39566371842415e-08
4038 2.41165211480876e-08
4039 2.40086571450249e-08
4040 2.40901577219788e-08
4041 2.4005363303381e-08
4042 2.40425997395377e-08
4043 2.39230732361762e-08
4044 2.40278185517084e-08
4045 2.39228708682759e-08
4046 2.41227430324287e-08
4047 2.3897475309731e-08
4048 2.39572967766222e-08
4049 2.38715500033848e-08
4050 2.39606937384451e-08
4051 2.38953794449159e-08
4052 2.39223018656531e-08
4053 2.38262014846669e-08
4054 2.3962215570883e-08
4055 2.3857919130954e-08
4056 2.38458500261274e-08
4057 2.37921078882763e-08
4058 2.39035587972225e-08
4059 2.37569063017418e-08
4060 2.38460106896099e-08
4061 2.38048543401703e-08
4062 2.39230411764879e-08
4063 2.37449403401868e-08
4064 2.38879586911267e-08
4065 2.36970777507395e-08
4066 2.3863614085684e-08
4067 2.36562875093327e-08
4068 2.37474687594386e-08
4069 2.36455309270411e-08
4070 2.37615729732354e-08
4071 2.36272785247849e-08
4072 2.37555596429573e-08
4073 2.36986802395478e-08
4074 2.37272418575074e-08
4075 2.36357802778642e-08
4076 2.37370576874341e-08
4077 2.35688058412009e-08
4078 2.36411232208411e-08
4079 2.36791129601954e-08
4080 2.35874222367372e-08
4081 2.35897836740051e-08
4082 2.35930902885428e-08
4083 2.35375421286932e-08
4084 2.35838760280771e-08
4085 2.36185059074856e-08
4086 2.35528243619498e-08
4087 2.34767951559078e-08
4088 2.357424707089e-08
4089 2.34430128358554e-08
4090 2.35665607224433e-08
4091 2.34929489240088e-08
4092 2.35451392907038e-08
4093 2.35128573731203e-08
4094 2.34735858093416e-08
4095 2.33864670757811e-08
4096 2.34910400127575e-08
4097 2.34088923303233e-08
4098 2.34379933514717e-08
4099 2.33626763357364e-08
4100 2.34455528946853e-08
4101 2.3377522224699e-08
4102 2.34480074903232e-08
4103 2.3411626675518e-08
4104 2.3414811757938e-08
4105 2.33087950824995e-08
4106 2.34570380586163e-08
4107 2.32939459365866e-08
4108 2.33973183663139e-08
4109 2.3288592023718e-08
4110 2.3356925167306e-08
4111 2.3298171219821e-08
4112 2.33299214489335e-08
4113 2.32487519467028e-08
4114 2.33248211953807e-08
4115 2.32293724913291e-08
4116 2.33578426831471e-08
4117 2.32003839721884e-08
4118 2.32730666871461e-08
4119 2.32033957212607e-08
4120 2.32533562796178e-08
4121 2.32202058025877e-08
4122 2.3211753813257e-08
4123 2.31471383216331e-08
4124 2.32524530092704e-08
4125 2.31655897451333e-08
4126 2.31810648401165e-08
4127 2.31650720055043e-08
4128 2.31725534423077e-08
4129 2.31011373834633e-08
4130 2.31972050634965e-08
4131 2.30893474419958e-08
4132 2.32086183018509e-08
4133 2.30464533448327e-08
4134 2.30893096819784e-08
4135 2.30145909192458e-08
4136 2.305505802358e-08
4137 2.30375723493381e-08
4138 2.30745530291898e-08
4139 2.30270749135286e-08
4140 2.31252836133677e-08
4141 2.29989006923503e-08
4142 2.30621454830882e-08
4143 2.29973272745099e-08
4144 2.31030427890744e-08
4145 2.29671571139178e-08
4146 2.30125583762231e-08
4147 2.29683578787387e-08
4148 2.30882372047603e-08
4149 2.28989361632514e-08
4150 2.29736694601357e-08
4151 2.29242808247321e-08
4152 2.29647627616458e-08
4153 2.2920539541893e-08
4154 2.29414246337711e-08
4155 2.28697398778621e-08
4156 2.29458076503874e-08
4157 2.28604618950357e-08
4158 2.29280830836842e-08
4159 2.28250496654425e-08
4160 2.29144622192479e-08
4161 2.28247154261396e-08
4162 2.28856940625022e-08
4163 2.28303040703537e-08
4164 2.28692754147275e-08
4165 2.27555802059243e-08
4166 2.28758607088153e-08
4167 2.27427832033555e-08
4168 2.28388156058301e-08
4169 2.27211905050595e-08
4170 2.28390399321654e-08
4171 2.2686984295639e-08
4172 2.27929652591996e-08
4173 2.27158201031585e-08
4174 2.27861263315532e-08
4175 2.26733401325063e-08
4176 2.27827882648413e-08
4177 2.26421145095301e-08
4178 2.27579790292864e-08
4179 2.26399872786942e-08
4180 2.27421201151046e-08
4181 2.2565339300229e-08
4182 2.26939357377276e-08
4183 2.25812874301567e-08
4184 2.26623676153892e-08
4185 2.25758313003865e-08
4186 2.26792896524941e-08
4187 2.25389864736414e-08
4188 2.26337349102934e-08
4189 2.254447556016e-08
4190 2.26567988512727e-08
4191 2.25143720369303e-08
4192 2.26232840878637e-08
4193 2.24960317289913e-08
4194 2.26216718992589e-08
4195 2.24745385670744e-08
4196 2.26236666449609e-08
4197 2.24589721415924e-08
4198 2.25780086937633e-08
4199 2.24459100763852e-08
4200 2.25963934425977e-08
4201 2.24306725247203e-08
4202 2.25352254634714e-08
4203 2.24128652721589e-08
4204 2.25927571149143e-08
4205 2.24046114798426e-08
4206 2.25084887777172e-08
4207 2.23795452587439e-08
4208 2.24984856833643e-08
4209 2.23585736298659e-08
4210 2.24901578045333e-08
4211 2.23503522649438e-08
4212 2.24658357970142e-08
4213 2.23298292265639e-08
4214 2.24216775182029e-08
4215 2.23267032080088e-08
4216 2.24257495355928e-08
4217 2.23007859503355e-08
4218 2.24178227297145e-08
4219 2.22654112889131e-08
4220 2.23795656788539e-08
4221 2.22648715126894e-08
4222 2.24155984014374e-08
4223 2.22201540189815e-08
4224 2.23356807262931e-08
4225 2.22255614774625e-08
4226 2.23229934261937e-08
4227 2.22103340847823e-08
4228 2.22968355192421e-08
4229 2.22065338393307e-08
4230 2.22941806145016e-08
4231 2.21596329561535e-08
4232 2.2234126248577e-08
4233 2.21116135552535e-08
4234 2.22525038164889e-08
4235 2.21028498108566e-08
4236 2.21770775024055e-08
4237 2.20724142216966e-08
4238 2.21856823454658e-08
4239 2.20727524222752e-08
4240 2.21658300851857e-08
4241 2.20782804687403e-08
4242 2.21598120839772e-08
4243 2.20336363341289e-08
4244 2.21741303541023e-08
4245 2.20189848896268e-08
4246 2.21143876508734e-08
4247 2.20033203124359e-08
4248 2.2092356346981e-08
4249 2.19589425327627e-08
4250 2.20569746112176e-08
4251 2.19460027830465e-08
4252 2.204266363659e-08
4253 2.19275291843957e-08
4254 2.19992595402729e-08
4255 2.19141904453934e-08
4256 2.20120163225701e-08
4257 2.18838623018414e-08
4258 2.20131177135485e-08
4259 2.18691591982179e-08
4260 2.19777237466801e-08
4261 2.18549340651464e-08
4262 2.19669885144214e-08
4263 2.18639548235089e-08
4264 2.19331292559133e-08
4265 2.18385106824925e-08
4266 2.19198430535528e-08
4267 2.18334856079139e-08
4268 2.19118132758211e-08
4269 2.18127796713929e-08
4270 2.19311156364199e-08
4271 2.17672934290647e-08
4272 2.18892279564287e-08
4273 2.17997020977023e-08
4274 2.18230783914919e-08
4275 2.17779018560904e-08
4276 2.18443049195827e-08
4277 2.1731587896312e-08
4278 2.19225116406818e-08
4279 2.17616488003003e-08
4280 2.18021264988266e-08
4281 2.16999193511924e-08
4282 2.18401368048404e-08
4283 2.17343860189345e-08
4284 2.17963776263375e-08
4285 2.17714414247894e-08
4286 2.17490340945758e-08
4287 2.17299363711732e-08
4288 2.17512149465193e-08
4289 2.16591400477029e-08
4290 2.17194091778339e-08
4291 2.16345863215395e-08
4292 2.1714770048753e-08
4293 2.16916110415966e-08
4294 2.16528068204269e-08
4295 2.15996318067013e-08
4296 2.16898767000373e-08
4297 2.15766739097845e-08
4298 2.16580482943485e-08
4299 2.15696029890466e-08
4300 2.16570939501892e-08
4301 2.15449543699009e-08
4302 2.16378538047124e-08
4303 2.15504094125407e-08
4304 2.16158688175483e-08
4305 2.15178031464092e-08
4306 2.16183930685787e-08
4307 2.15004604484648e-08
4308 2.15983062101799e-08
4309 2.14893071373012e-08
4310 2.15915490500151e-08
4311 2.14712314532406e-08
4312 2.15744822060771e-08
4313 2.14614907845245e-08
4314 2.15753983718869e-08
4315 2.14423653348561e-08
4316 2.15305558786127e-08
4317 2.14363770689729e-08
4318 2.15636675422459e-08
4319 2.1439594160455e-08
4320 2.15225461097646e-08
4321 2.13687505787874e-08
4322 2.14464626120403e-08
4323 2.13874086751886e-08
4324 2.14933869653322e-08
4325 2.13442820768961e-08
4326 2.14030053014014e-08
4327 2.14009889294431e-08
4328 2.14034276222463e-08
4329 2.13785016951462e-08
4330 2.13808752311806e-08
4331 2.13158344326558e-08
4332 2.13738642553807e-08
4333 2.13062359621929e-08
4334 2.13557581973944e-08
4335 2.12361016656004e-08
4336 2.13463119402846e-08
4337 2.12487731774402e-08
4338 2.13120244234588e-08
4339 2.12380657220734e-08
4340 2.12854303587306e-08
4341 2.12267540087296e-08
4342 2.12745040535367e-08
4343 2.11981168032338e-08
4344 2.12811241899757e-08
4345 2.11949595447791e-08
4346 2.12342913021857e-08
4347 2.11839492996191e-08
4348 2.12269850461411e-08
4349 2.11587550076331e-08
4350 2.12060016195892e-08
4351 2.11466478798883e-08
4352 2.11990113339056e-08
4353 2.11280930235702e-08
4354 2.11777152916426e-08
4355 2.11178276083857e-08
4356 2.11743972062806e-08
4357 2.11217918915096e-08
4358 2.11644050169824e-08
4359 2.11042503144299e-08
4360 2.11775523348834e-08
4361 2.10759937617766e-08
4362 2.11509800394083e-08
4363 2.10786557204301e-08
4364 2.11338544513495e-08
4365 2.10694203666151e-08
4366 2.10889242948653e-08
4367 2.10540901610656e-08
4368 2.10950060939297e-08
4369 2.10246654903656e-08
4370 2.10579903017205e-08
4371 2.10263490689755e-08
4372 2.11226580582036e-08
4373 2.10294094715735e-08
4374 2.10798094792963e-08
4375 2.09840983229626e-08
4376 2.10556137369977e-08
4377 2.10056450189455e-08
4378 2.10324007934304e-08
4379 2.09980791900932e-08
4380 2.10067821244664e-08
4381 2.098029522557e-08
4382 2.10242667391114e-08
4383 2.09725459416887e-08
4384 2.1013039091855e-08
4385 2.0944690745317e-08
4386 2.10010129846339e-08
4387 2.0945329271882e-08
4388 2.0985431008036e-08
4389 2.09372710937572e-08
4390 2.09771272245973e-08
4391 2.09150135948022e-08
4392 2.09614188149132e-08
4393 2.09092547551748e-08
4394 2.09524936103378e-08
4395 2.08852172329799e-08
4396 2.09447593189083e-08
4397 2.08646377402033e-08
4398 2.09079662898404e-08
4399 2.0814267220004e-08
4400 2.0895021068057e-08
4401 2.08556998950726e-08
4402 2.088127372879e-08
4403 2.08358239950357e-08
4404 2.08659572882297e-08
4405 2.08286766145704e-08
4406 2.08515279185306e-08
4407 2.08192079202973e-08
4408 2.08435325221856e-08
4409 2.08070234171132e-08
4410 2.08264611583076e-08
4411 2.07777376513008e-08
4412 2.08348925561097e-08
4413 2.07601407193891e-08
4414 2.07970717323036e-08
4415 2.07703462269393e-08
4416 2.07803726954836e-08
4417 2.07621050725137e-08
4418 2.07651091326255e-08
4419 2.07489828438767e-08
4420 2.07461497181427e-08
4421 2.07372798568528e-08
4422 2.07282762287875e-08
4423 2.07349069194507e-08
4424 2.07211140654806e-08
4425 2.07258903452967e-08
4426 2.06989394655466e-08
4427 2.06917949698848e-08
4428 2.07068421111956e-08
4429 2.06735238528211e-08
4430 2.06833527718331e-08
4431 2.06748110826993e-08
4432 2.06658080461608e-08
4433 2.0645664029928e-08
4434 2.06748867386253e-08
4435 2.06301285388122e-08
4436 2.06442240004989e-08
4437 2.06403537870159e-08
4438 2.06291618010113e-08
4439 2.06052200670825e-08
4440 2.06327308527321e-08
4441 2.05929221053225e-08
4442 2.06162922049558e-08
4443 2.05776344728292e-08
4444 2.05986550474435e-08
4445 2.05853044832338e-08
4446 2.05733245381978e-08
4447 2.05543857694224e-08
4448 2.05869628251421e-08
4449 2.05364888810067e-08
4450 2.05532416099885e-08
4451 2.05467085487854e-08
4452 2.05363302780981e-08
4453 2.05139848414859e-08
4454 2.05430154194275e-08
4455 2.0504540102273e-08
4456 2.05109205886345e-08
4457 2.04769359903878e-08
4458 2.05195522857693e-08
4459 2.04660598859263e-08
4460 2.04704765849328e-08
4461 2.04588592014332e-08
4462 2.04783622343996e-08
4463 2.04410912818531e-08
4464 2.04409649784409e-08
4465 2.04408076802665e-08
4466 2.04162778851824e-08
4467 2.04506794094783e-08
4468 2.03904236713015e-08
4469 2.04404547403669e-08
4470 2.03772138647196e-08
4471 2.0397461810262e-08
4472 2.03626968104942e-08
4473 2.03845357384935e-08
4474 2.03636776126004e-08
4475 2.03806906391435e-08
4476 2.03274572170287e-08
4477 2.03797361697511e-08
4478 2.03212614229287e-08
4479 2.03652686661115e-08
4480 2.03016480737617e-08
4481 2.03319338680785e-08
4482 2.02988847259888e-08
4483 2.0329223508142e-08
4484 2.02840644005775e-08
4485 2.02997663718563e-08
4486 2.02698787195033e-08
4487 2.02247388116916e-08
4488 2.01926922542839e-08
4489 2.02057371225806e-08
4490 2.02768650536811e-08
4491 2.02788401155729e-08
4492 2.02328871994339e-08
4493 2.02077029038961e-08
4494 2.01949073259655e-08
4495 2.02080421773942e-08
4496 2.01991423507408e-08
4497 2.01362816429906e-08
4498 2.02008355918437e-08
4499 2.01796983567704e-08
4500 2.01601238467575e-08
4501 2.01253945109059e-08
4502 2.01953088172502e-08
4503 2.01326581610317e-08
4504 2.0173283532543e-08
4505 2.0103715145936e-08
4506 2.01856195500838e-08
4507 2.00698575296698e-08
4508 2.0085474139897e-08
4509 2.00746046941092e-08
4510 2.01267536255045e-08
4511 2.00498822460204e-08
4512 2.00912066548042e-08
4513 2.00596599295721e-08
4514 2.01080750601434e-08
4515 2.00896700288666e-08
4516 2.00521391571229e-08
4517 2.00685159246206e-08
4518 2.00409706989646e-08
4519 2.00405954347005e-08
4520 2.00539090613461e-08
4521 1.99895331611799e-08
4522 2.00089221626953e-08
4523 1.99961165483487e-08
4524 2.00350106371872e-08
4525 1.99934244182742e-08
4526 2.00270137984404e-08
4527 1.99421829485402e-08
4528 2.00169978352704e-08
4529 1.99397443543248e-08
4530 1.99571406866994e-08
4531 1.99859413481107e-08
4532 1.99302692402981e-08
4533 1.99454015561429e-08
4534 1.99758609147338e-08
4535 2.00011161304303e-08
4536 2.00144597934937e-08
4537 1.98834092390854e-08
4538 1.99071154041164e-08
4539 1.98780035578494e-08
4540 1.98947777123593e-08
4541 1.99629890866859e-08
4542 1.9893294815887e-08
4543 1.98491226068853e-08
4544 1.98711870380563e-08
4545 1.98664051112019e-08
4546 1.99017529931567e-08
4547 1.98527258579162e-08
4548 1.98375684323793e-08
4549 1.98190000872955e-08
4550 1.98883928730353e-08
4551 1.98020674293531e-08
4552 1.9889670602069e-08
4553 1.98010420655592e-08
4554 1.98927108048252e-08
4555 1.98250617291507e-08
4556 1.98178020269779e-08
4557 1.97777558597068e-08
4558 1.98427693032244e-08
4559 1.97851483472178e-08
4560 1.97997440452014e-08
4561 1.97998345257133e-08
4562 1.98190908031748e-08
4563 1.97864822855109e-08
4564 1.97894308771041e-08
4565 1.97558913770379e-08
4566 1.97468059006667e-08
4567 1.97343302561492e-08
4568 1.97915197119869e-08
4569 1.97066644798127e-08
4570 1.97797742229611e-08
4571 1.97097214647002e-08
4572 1.97595248776494e-08
4573 1.96854753164999e-08
4574 1.97186851682574e-08
4575 1.97036091416081e-08
4576 1.97105908643636e-08
4577 1.9662482075411e-08
4578 1.97483864488035e-08
4579 1.96580658373691e-08
4580 1.96774347731576e-08
4581 1.96463475035102e-08
4582 1.96412152702052e-08
4583 1.97492787705755e-08
4584 1.96637484739526e-08
4585 1.96384373909453e-08
4586 1.96405782562081e-08
4587 1.96602188902162e-08
4588 1.96593207340001e-08
4589 1.95867624404045e-08
4590 1.96756575698842e-08
4591 1.96445323146222e-08
4592 1.95940069795952e-08
4593 1.96032638397625e-08
4594 1.95906292246306e-08
4595 1.96332411981359e-08
4596 1.96429943679632e-08
4597 1.95551979977893e-08
4598 1.95756329315344e-08
4599 1.95316526001577e-08
4600 1.95646472542776e-08
4601 1.95696052847083e-08
4602 1.95625757450912e-08
4603 1.95266781446435e-08
4604 1.95833388918487e-08
4605 1.95338823472113e-08
4606 1.95656760038077e-08
4607 1.95554140489662e-08
4608 1.95141199554882e-08
4609 1.9564257103255e-08
4610 1.94921862188835e-08
4611 1.94954882539733e-08
4612 1.95276660832633e-08
4613 1.94703588975997e-08
4614 1.95907686331154e-08
4615 1.95159422062829e-08
4616 1.9478965487707e-08
4617 1.94843346790208e-08
4618 1.94697911322095e-08
4619 1.95344498781225e-08
4620 1.94816296001932e-08
4621 1.94530417019223e-08
4622 1.94483733508832e-08
4623 1.9461378952812e-08
4624 1.9562701349507e-08
4625 1.94131158650634e-08
4626 1.94397482067288e-08
4627 1.94404928599567e-08
4628 1.9456782335503e-08
4629 1.94289141965953e-08
4630 1.94373823942939e-08
4631 1.943230357071e-08
4632 1.94339432271207e-08
4633 1.93980579297204e-08
4634 1.94247922360447e-08
4635 1.93830162533715e-08
4636 1.94075395478777e-08
4637 1.9394875469203e-08
4638 1.93860817665481e-08
4639 1.93555139924584e-08
4640 1.94075099368973e-08
4641 1.93295669230764e-08
4642 1.93638812779895e-08
4643 1.93581578233903e-08
4644 1.93274724695769e-08
4645 1.93017588916433e-08
4646 1.9360020007575e-08
4647 1.92926323769882e-08
4648 1.93388201878975e-08
4649 1.9283355904065e-08
4650 1.93416478886377e-08
4651 1.92802774314771e-08
4652 1.92498092150828e-08
4653 1.92783458690116e-08
4654 1.9302881764105e-08
4655 1.92433558012794e-08
4656 1.92564578327392e-08
4657 1.92450255012133e-08
4658 1.92573365520587e-08
4659 1.92554609919071e-08
4660 1.92621444474739e-08
4661 1.92394522402495e-08
4662 1.92682208108863e-08
4663 1.91825229824971e-08
4664 1.92089677373275e-08
4665 1.92251995452253e-08
4666 1.92449018072693e-08
4667 1.91803245339628e-08
4668 1.92116314448043e-08
4669 1.91745481439298e-08
4670 1.91901856450016e-08
4671 1.91529286270864e-08
4672 1.9241621458832e-08
4673 1.91568786434004e-08
4674 1.915630331073e-08
4675 1.91687406445595e-08
4676 1.91493947232502e-08
4677 1.91397932205462e-08
4678 1.91870938772709e-08
4679 1.912053674058e-08
4680 1.91399097699829e-08
4681 1.91143111969438e-08
4682 1.91332459706928e-08
4683 1.90757782299045e-08
4684 1.91620591545316e-08
4685 1.90848857224424e-08
4686 1.91107765292742e-08
4687 1.90796482879563e-08
4688 1.91048928135373e-08
4689 1.90454590738298e-08
4690 1.90873864571728e-08
4691 1.90878891919155e-08
4692 1.9088195855943e-08
4693 1.90628409617588e-08
4694 1.91129144448965e-08
4695 1.90445939596273e-08
4696 1.9058947133388e-08
4697 1.90353330156157e-08
4698 1.90183581354475e-08
4699 1.9033794920631e-08
4700 1.90382351057394e-08
4701 1.900195019644e-08
4702 1.90462304940908e-08
4703 1.89896643174592e-08
4704 1.90630281897697e-08
4705 1.89839117519242e-08
4706 1.90038670400128e-08
4707 1.89935647094686e-08
4708 1.90003683107065e-08
4709 1.89601754367885e-08
4710 1.90143754439376e-08
4711 1.89624411781608e-08
4712 1.89639363918914e-08
4713 1.89787554605303e-08
4714 1.91175265529253e-08
4715 1.892933294112e-08
4716 1.89337438669668e-08
4717 1.89225247870795e-08
4718 1.89510361137124e-08
4719 1.8904739069292e-08
4720 1.89430216703812e-08
4721 1.88773734768333e-08
4722 1.89360278062267e-08
4723 1.88891799695057e-08
4724 1.8919103291104e-08
4725 1.88583042719515e-08
4726 1.89644282162504e-08
4727 1.88538184362486e-08
4728 1.89068434259809e-08
4729 1.88618113172723e-08
4730 1.8875106894356e-08
4731 1.88798072455754e-08
4732 1.89296115831183e-08
4733 1.882819216803e-08
4734 1.88965320608858e-08
4735 1.88107629508494e-08
4736 1.88186730811779e-08
4737 1.88468949664511e-08
4738 1.88325965710945e-08
4739 1.88518189112585e-08
4740 1.88062177581827e-08
4741 1.88368642373149e-08
4742 1.88305125625732e-08
4743 1.88135183600835e-08
4744 1.88009004711276e-08
4745 1.88059792147754e-08
4746 1.88238023444143e-08
4747 1.87248739358381e-08
4748 1.8880701312618e-08
4749 1.87333669190437e-08
4750 1.88048472402613e-08
4751 1.86877075165626e-08
4752 1.88296588055081e-08
4753 1.87272713567666e-08
4754 1.87860403837448e-08
4755 1.87074430328948e-08
4756 1.88078918430534e-08
4757 1.86955296115343e-08
4758 1.87416707300159e-08
4759 1.87031342644417e-08
4760 1.87266991167334e-08
4761 1.8734266928e-08
4762 1.87164256493233e-08
4763 1.86858524218181e-08
4764 1.87509449540713e-08
4765 1.86814850282602e-08
4766 1.874665090984e-08
4767 1.86310049006266e-08
4768 1.86962048518424e-08
4769 1.86382447004974e-08
4770 1.86878677173041e-08
4771 1.8696648556471e-08
4772 1.86560261008495e-08
4773 1.86075113308704e-08
4774 1.86494882497001e-08
4775 1.86635551848724e-08
4776 1.85816681153028e-08
4777 1.87061261485155e-08
4778 1.85802515826339e-08
4779 1.86652787013131e-08
4780 1.86160341968389e-08
4781 1.86490380134074e-08
4782 1.85653500102134e-08
4783 1.86927952823268e-08
4784 1.85665334120344e-08
4785 1.86219379152419e-08
4786 1.85752078829537e-08
4787 1.86390146286186e-08
4788 1.85727053017004e-08
4789 1.86276850611122e-08
4790 1.8498320660143e-08
4791 1.85874805458397e-08
4792 1.851151440313e-08
4793 1.85732734978572e-08
4794 1.8515700113575e-08
4795 1.85926312612494e-08
4796 1.84436986545577e-08
4797 1.85780239529976e-08
4798 1.85251602937697e-08
4799 1.85597988844322e-08
4800 1.84437535262205e-08
4801 1.85480670547022e-08
4802 1.84184698737155e-08
4803 1.8560050055072e-08
4804 1.84228092967587e-08
4805 1.85401671943808e-08
4806 1.84158360623243e-08
4807 1.85585658627474e-08
4808 1.8392745847251e-08
4809 1.85303862290098e-08
4810 1.8450033491213e-08
4811 1.84824378397508e-08
4812 1.83813272780498e-08
4813 1.84914464869124e-08
4814 1.83697577238462e-08
4815 1.85000237387101e-08
4816 1.83667306510316e-08
4817 1.85298735653205e-08
4818 1.83521385377361e-08
4819 1.84831928207174e-08
4820 1.83891771552425e-08
4821 1.84543607382182e-08
4822 1.83355623901704e-08
4823 1.84879454456777e-08
4824 1.83152044241908e-08
4825 1.84419684439163e-08
4826 1.83282064503132e-08
4827 1.84536227552101e-08
4828 1.83344441468947e-08
4829 1.84986371323248e-08
4830 1.83347956195234e-08
4831 1.83775918527473e-08
4832 1.83524769852283e-08
4833 1.83961301889468e-08
4834 1.8311618124045e-08
4835 1.84063779808952e-08
4836 1.830216752019e-08
4837 1.83426880262516e-08
4838 1.83306392491644e-08
4839 1.8391577174981e-08
4840 1.82944583535516e-08
4841 1.83669338778003e-08
4842 1.82758082880596e-08
4843 1.83412237522163e-08
4844 1.82656252949442e-08
4845 1.83678272280829e-08
4846 1.82630285543439e-08
4847 1.83159373783326e-08
4848 1.82306288829537e-08
4849 1.83276758702888e-08
4850 1.82539457949105e-08
4851 1.83196741208036e-08
4852 1.82339837477841e-08
4853 1.83627303274037e-08
4854 1.81866194273184e-08
4855 1.83030235341164e-08
4856 1.82406135218471e-08
4857 1.83204737025378e-08
4858 1.8198094747568e-08
4859 1.82381990541103e-08
4860 1.82174632747945e-08
4861 1.82784466788988e-08
4862 1.81968216628192e-08
4863 1.83130379500795e-08
4864 1.81680852024968e-08
4865 1.82723549562169e-08
4866 1.81304512292257e-08
4867 1.82494225153818e-08
4868 1.82322237787247e-08
4869 1.82602824141398e-08
4870 1.81450669121119e-08
4871 1.82230584728771e-08
4872 1.81405469010798e-08
4873 1.82473860155241e-08
4874 1.81565491645941e-08
4875 1.81818412352541e-08
4876 1.81393116349682e-08
4877 1.81825109297762e-08
4878 1.81275853954332e-08
4879 1.82301773508087e-08
4880 1.81695926277925e-08
4881 1.82009671148009e-08
4882 1.81123045637221e-08
4883 1.81777892507284e-08
4884 1.81931856868545e-08
4885 1.81114652111347e-08
4886 1.81302793258453e-08
4887 1.81166965447233e-08
4888 1.81177384286357e-08
4889 1.81241920476083e-08
4890 1.80921392125555e-08
4891 1.81366472551403e-08
4892 1.8052332770857e-08
4893 1.81182593470552e-08
4894 1.80571500019155e-08
4895 1.81957844214153e-08
4896 1.81028971164565e-08
4897 1.80888474341501e-08
4898 1.8080810201937e-08
4899 1.80729753038733e-08
4900 1.80777151541633e-08
4901 1.80652990211527e-08
4902 1.80362371366982e-08
4903 1.81451390073306e-08
4904 1.7996838371559e-08
4905 1.81274251795926e-08
4906 1.80248001147021e-08
4907 1.80745852969011e-08
4908 1.80083322343094e-08
4909 1.81166246813191e-08
4910 1.8036609780836e-08
4911 1.8079029068474e-08
4912 1.79610062138735e-08
4913 1.80682875949856e-08
4914 1.80362759181207e-08
4915 1.80445589288425e-08
4916 1.79628840166757e-08
4917 1.80524943438343e-08
4918 1.79478572155389e-08
4919 1.80820300617057e-08
4920 1.79281682637367e-08
4921 1.79953301113756e-08
4922 1.80116712753531e-08
4923 1.80429892635559e-08
4924 1.78829044221729e-08
4925 1.80374873011147e-08
4926 1.79320744244649e-08
4927 1.79868674141659e-08
4928 1.79203164671904e-08
4929 1.80832576148759e-08
4930 1.78578714882249e-08
4931 1.80141183925286e-08
4932 1.79405611238082e-08
4933 1.79467479046735e-08
4934 1.79375245705771e-08
4935 1.78928474960927e-08
4936 1.79899802645878e-08
4937 1.78881939483233e-08
4938 1.78959602319395e-08
4939 1.79326826392767e-08
4940 1.79512229436796e-08
4941 1.78977940636926e-08
4942 1.7881761039007e-08
4943 1.78621718704974e-08
4944 1.78932891143901e-08
4945 1.78716096401743e-08
4946 1.78785527671366e-08
4947 1.79092253640434e-08
4948 1.79153621422401e-08
4949 1.78199154960978e-08
4950 1.78797882588455e-08
4951 1.7830190138568e-08
4952 1.7919023028945e-08
4953 1.78235578411901e-08
4954 1.78636216823236e-08
4955 1.78426818280286e-08
4956 1.78778846811056e-08
4957 1.7766563477295e-08
4958 1.78855915349274e-08
4959 1.77658429203476e-08
4960 1.79056136575539e-08
4961 1.77758153494523e-08
4962 1.79089007152911e-08
4963 1.77612688601059e-08
4964 1.78343485686128e-08
4965 1.77622054238213e-08
4966 1.78521564420109e-08
4967 1.77658690629912e-08
4968 1.78362747176308e-08
4969 1.77415628481725e-08
4970 1.78414979448149e-08
4971 1.77733181327966e-08
4972 1.78255256662041e-08
4973 1.77081062409101e-08
4974 1.7849292632377e-08
4975 1.77610120370986e-08
4976 1.77760674100469e-08
4977 1.77363478055526e-08
4978 1.777094374944e-08
4979 1.77601441242459e-08
4980 1.77706227972863e-08
4981 1.77097787243952e-08
4982 1.78130504231433e-08
4983 1.76951319321716e-08
4984 1.77683892985669e-08
4985 1.77186729697354e-08
4986 1.77628276123443e-08
4987 1.76916395639992e-08
4988 1.77333950786007e-08
4989 1.76945024898956e-08
4990 1.77750039558333e-08
4991 1.76778213534234e-08
4992 1.77230496278824e-08
4993 1.77240106999932e-08
4994 1.77258614062481e-08
4995 1.76297306015982e-08
4996 1.77939583956288e-08
4997 1.77002055732345e-08
4998 1.78034812243766e-08
4999 1.76720671696273e-08
};
\addlegendentry{Train}
\addplot [semithick, black]
table {%
0 0.00669908942654729
1 0.0028362029697746
2 0.00207857508212328
3 0.00174908770713955
4 0.00132849195506424
5 0.00087331811664626
6 0.00054455769713968
7 0.00031183153623715
8 0.000224997740588151
9 0.000196434993995354
10 0.000172566768014804
11 0.00014765169180464
12 0.000119943731988315
13 9.33730916585773e-05
14 6.95167109370232e-05
15 5.01536560477689e-05
16 3.63262261089403e-05
17 2.77099316008389e-05
18 2.2924608856556e-05
19 2.04328116524266e-05
20 1.91191138583235e-05
21 1.83570246008458e-05
22 1.78362370206742e-05
23 1.74162087205332e-05
24 1.70350940607022e-05
25 1.66706249729032e-05
26 1.63008917297702e-05
27 1.59176433953689e-05
28 1.52766679093475e-05
29 1.46719439726439e-05
30 1.41377922773245e-05
31 1.36111084430013e-05
32 1.30662238007062e-05
33 1.24939642773825e-05
34 1.1891695066879e-05
35 1.12634706965764e-05
36 1.0610226127028e-05
37 9.9392873380566e-06
38 9.25976473808987e-06
39 8.57516351970844e-06
40 7.89625937613891e-06
41 7.23048651707359e-06
42 6.58703129374771e-06
43 5.97395364820841e-06
44 5.39382153874612e-06
45 4.84784050058806e-06
46 4.33964441981516e-06
47 3.87473573937314e-06
48 3.44811041941284e-06
49 3.06556103168987e-06
50 2.69523070528521e-06
51 2.35605239140568e-06
52 2.03147124011593e-06
53 1.75083437170542e-06
54 1.52131883623952e-06
55 1.34682056796009e-06
56 1.21325263080507e-06
57 1.10951771148393e-06
58 1.03478078017361e-06
59 9.76138949226879e-07
60 9.24407459024223e-07
61 8.86032637481549e-07
62 8.5518860259981e-07
63 8.29590817374992e-07
64 8.06046898560453e-07
65 7.84794849550963e-07
66 7.65256174872775e-07
67 7.48712125187012e-07
68 7.33710862732551e-07
69 7.1940212365007e-07
70 7.06457001342642e-07
71 6.94293532887968e-07
72 6.83229359310644e-07
73 6.73063254907902e-07
74 6.6321547365078e-07
75 6.54339430639084e-07
76 6.44999829546578e-07
77 6.34888181139104e-07
78 6.26187102170661e-07
79 6.1919342897454e-07
80 6.10836764280975e-07
81 6.03793750997283e-07
82 5.97139376168343e-07
83 5.89939645578852e-07
84 5.82728716835845e-07
85 5.75983619910403e-07
86 5.69302642361436e-07
87 5.62703348805371e-07
88 5.56563236386864e-07
89 5.50592062609212e-07
90 5.45277600849658e-07
91 5.39277152711293e-07
92 5.331343686521e-07
93 5.27944507666689e-07
94 5.2180502052579e-07
95 5.15872557116381e-07
96 5.10594077240967e-07
97 5.05306161358021e-07
98 4.98175836582959e-07
99 4.90044385514921e-07
100 4.81046697586862e-07
101 4.7211207743203e-07
102 4.66245751340466e-07
103 4.60996716356021e-07
104 4.56539083870666e-07
105 4.52296944786212e-07
106 4.4828624368165e-07
107 4.44545065647617e-07
108 4.41163223285912e-07
109 4.3774781488537e-07
110 4.34814154459673e-07
111 4.31568366821011e-07
112 4.28442604061274e-07
113 4.25517924895757e-07
114 4.2278387013539e-07
115 4.20117657995434e-07
116 4.18218291997619e-07
117 4.15549237686719e-07
118 4.13188075754078e-07
119 4.10743638212807e-07
120 4.08495424153443e-07
121 4.06304252464906e-07
122 4.040950898343e-07
123 4.01807653815922e-07
124 3.99787325022771e-07
125 3.97823015418908e-07
126 3.95827669308346e-07
127 3.94166193018464e-07
128 3.9244437743946e-07
129 3.90728644106275e-07
130 3.89080383911278e-07
131 3.87562607784275e-07
132 3.86083627290645e-07
133 3.84572445000231e-07
134 3.83149625804435e-07
135 3.81790727033149e-07
136 3.80769932917246e-07
137 3.79501216229983e-07
138 3.7818153941771e-07
139 3.77012810304223e-07
140 3.7584482015518e-07
141 3.74652586287993e-07
142 3.73615051785237e-07
143 3.72579535223849e-07
144 3.71298824575206e-07
145 3.70383759218385e-07
146 3.69365181995818e-07
147 3.68365164149509e-07
148 3.67414770607866e-07
149 3.66457669542797e-07
150 3.6548550497173e-07
151 3.64693903520674e-07
152 3.63826700322534e-07
153 3.63015317361715e-07
154 3.62256173502828e-07
155 3.61458603492792e-07
156 3.60691501555266e-07
157 3.59930112381335e-07
158 3.59169860075781e-07
159 3.58571213610048e-07
160 3.5779572726824e-07
161 3.57064550371433e-07
162 3.56528147449353e-07
163 3.55868507995183e-07
164 3.55203212620836e-07
165 3.54721294115734e-07
166 3.54078679265513e-07
167 3.5339817827662e-07
168 3.52682889115385e-07
169 3.52099760903002e-07
170 3.51592802871892e-07
171 3.50953939687315e-07
172 3.50535430015952e-07
173 3.49906599694805e-07
174 3.49482661476941e-07
175 3.48920224269023e-07
176 3.48338005551341e-07
177 3.47811777601237e-07
178 3.4726721764855e-07
179 3.46703785680802e-07
180 3.46205183632264e-07
181 3.45611397278844e-07
182 3.45053337014178e-07
183 3.44504570648496e-07
184 3.4400292747705e-07
185 3.43187508633491e-07
186 3.42638600159262e-07
187 3.42144375053977e-07
188 3.41662854452807e-07
189 3.41205179665849e-07
190 3.40691087785672e-07
191 3.40263454745582e-07
192 3.39784804737064e-07
193 3.39398383175649e-07
194 3.38945852718098e-07
195 3.3855653214232e-07
196 3.38118212539484e-07
197 3.37684355145029e-07
198 3.37266641281531e-07
199 3.36848557935809e-07
200 3.35801928486035e-07
201 3.35356986624902e-07
202 3.34937169554905e-07
203 3.34501436327628e-07
204 3.34148751335306e-07
205 3.33714780254013e-07
206 3.33270122609974e-07
207 3.32294717964032e-07
208 3.31728358560213e-07
209 3.31162937072804e-07
210 3.30692841998825e-07
211 3.30185912389425e-07
212 3.29715561520061e-07
213 3.29285398947832e-07
214 3.28776394553643e-07
215 3.28243373814985e-07
216 3.27704412939056e-07
217 3.27231362007296e-07
218 3.26858383914441e-07
219 3.26336788702974e-07
220 3.26036939668484e-07
221 3.25519465604884e-07
222 3.25262391243086e-07
223 3.24681195706944e-07
224 3.24129104001258e-07
225 3.23700675153304e-07
226 3.23479639519064e-07
227 3.22965973964529e-07
228 3.22756420700898e-07
229 3.224140527891e-07
230 3.22038914646328e-07
231 3.21676736803056e-07
232 3.21625009291893e-07
233 3.21186490737091e-07
234 3.20814194765262e-07
235 3.20394889286035e-07
236 3.19992238928535e-07
237 3.19915869795295e-07
238 3.18923014219763e-07
239 3.18537018983989e-07
240 3.18336304871991e-07
241 3.17815505468388e-07
242 3.17435763008689e-07
243 3.17094816182362e-07
244 3.16713425263515e-07
245 3.16413121481673e-07
246 3.16025847268975e-07
247 3.15750781965107e-07
248 3.15323319455274e-07
249 3.15121837957122e-07
250 3.14669421186409e-07
251 3.14316253025027e-07
252 3.14100674358997e-07
253 3.13778372174056e-07
254 3.13472213520072e-07
255 3.13151019781799e-07
256 3.12835538807121e-07
257 3.12354558218431e-07
258 3.12191218654334e-07
259 3.11905466787721e-07
260 3.11596295432537e-07
261 3.11117531737182e-07
262 3.10971728367804e-07
263 3.10667388703223e-07
264 3.10359752120348e-07
265 3.10056577745854e-07
266 3.09759712990854e-07
267 3.09816385879458e-07
268 3.08969390516722e-07
269 3.08898478351693e-07
270 3.08583423702657e-07
271 3.082863884174e-07
272 3.08490427869401e-07
273 3.08525812897642e-07
274 3.07339632854564e-07
275 3.08012346295072e-07
276 3.07727646031708e-07
277 3.07418787315328e-07
278 3.07119648823573e-07
279 3.0630593528258e-07
280 3.05511747455967e-07
281 3.05926874943907e-07
282 3.0477059453915e-07
283 3.05122199506513e-07
284 3.03957506275765e-07
285 3.03690228520281e-07
286 3.03385547795187e-07
287 3.03890516306637e-07
288 3.03694321246439e-07
289 3.03174772398052e-07
290 3.03016747693619e-07
291 3.0197827527445e-07
292 3.01698179328014e-07
293 3.01339809993806e-07
294 3.01098452837323e-07
295 3.00883158388388e-07
296 3.00593910651514e-07
297 3.00307306133618e-07
298 3.00017973131617e-07
299 2.99715111395926e-07
300 2.99454569585578e-07
301 2.99197637332327e-07
302 2.98889744954067e-07
303 3.00044860068738e-07
304 2.99748279530831e-07
305 2.99429814276664e-07
306 2.99118681823529e-07
307 2.9880280294492e-07
308 2.98532597753365e-07
309 2.98232009754429e-07
310 2.97924827918905e-07
311 2.9765263320769e-07
312 2.9736173701167e-07
313 2.97063564858036e-07
314 2.9677244128834e-07
315 2.96144008871124e-07
316 2.95774611913657e-07
317 2.95426985985614e-07
318 2.95092235091943e-07
319 2.94739180617398e-07
320 2.94398148525943e-07
321 2.94062061811928e-07
322 2.93743937618274e-07
323 2.93196620759772e-07
324 2.92756965336594e-07
325 2.92319725758716e-07
326 2.92098803811314e-07
327 2.91542306740666e-07
328 2.91156851517371e-07
329 2.90976402084198e-07
330 2.90415897552521e-07
331 2.90014213533141e-07
332 2.89642571260629e-07
333 2.89262402475288e-07
334 2.88816380589196e-07
335 2.88668047687679e-07
336 2.88320535446474e-07
337 2.8796992523894e-07
338 2.8751634317814e-07
339 2.86858011122604e-07
340 2.86504132418486e-07
341 2.86220142697857e-07
342 2.85809136357784e-07
343 2.85374937902816e-07
344 2.8496506843112e-07
345 2.84555966345579e-07
346 2.84153315988078e-07
347 2.83780025256419e-07
348 2.83356285990521e-07
349 2.82960485264994e-07
350 2.82562325537583e-07
351 2.82155042441445e-07
352 2.81746480368383e-07
353 2.81370461152619e-07
354 2.8096073378947e-07
355 2.80548249520507e-07
356 2.80156285725752e-07
357 2.79767334632197e-07
358 2.79380941492491e-07
359 2.78997816849369e-07
360 2.78613867976674e-07
361 2.78217925142599e-07
362 2.77826615047161e-07
363 2.7743547548198e-07
364 2.77050901331677e-07
365 2.76872469839873e-07
366 2.76480619731956e-07
367 2.76096358220457e-07
368 2.75717781050844e-07
369 2.75331444754556e-07
370 2.74957614010418e-07
371 2.74579605274994e-07
372 2.74201909178373e-07
373 2.73820120355595e-07
374 2.73534482175819e-07
375 2.7318412776367e-07
376 2.72813053925347e-07
377 2.72542081347638e-07
378 2.7215625664212e-07
379 2.71785467020891e-07
380 2.71359766657042e-07
381 2.70973771421268e-07
382 2.70602498630979e-07
383 2.7022420567846e-07
384 2.69851625489537e-07
385 2.69482626436002e-07
386 2.69114877937682e-07
387 2.68787061941111e-07
388 2.68712000206506e-07
389 2.6835203925657e-07
390 2.67991623559283e-07
391 2.67527155983771e-07
392 2.67115240148996e-07
393 2.66665324488713e-07
394 2.66247212721282e-07
395 2.6582466716718e-07
396 2.65422841039253e-07
397 2.6503229832997e-07
398 2.64646644154709e-07
399 2.64264286897742e-07
400 2.63880053807952e-07
401 2.63507757836123e-07
402 2.63164537273042e-07
403 2.62776893578121e-07
404 2.62401357531417e-07
405 2.6202744152215e-07
406 2.61660261458019e-07
407 2.61274777813014e-07
408 2.60901430237936e-07
409 2.60533795426454e-07
410 2.60170793353609e-07
411 2.59811173464186e-07
412 2.59453116768782e-07
413 2.59096594845687e-07
414 2.5873964659695e-07
415 2.58383550999497e-07
416 2.58050420143263e-07
417 2.57697479355556e-07
418 2.57351331356404e-07
419 2.5700373385007e-07
420 2.5665590897006e-07
421 2.56292935318925e-07
422 2.55941074556176e-07
423 2.55687865546861e-07
424 2.55333048926332e-07
425 2.54963993029378e-07
426 2.5461162067586e-07
427 2.54260697829523e-07
428 2.53908410741133e-07
429 2.53558482654626e-07
430 2.53205570288628e-07
431 2.52859450711185e-07
432 2.52508499443138e-07
433 2.52166984182622e-07
434 2.51776214099664e-07
435 2.51408920348695e-07
436 2.51048476229698e-07
437 2.50721512884411e-07
438 2.50416093194872e-07
439 2.5007540216393e-07
440 2.49754322112494e-07
441 2.49360908810559e-07
442 2.49060860824102e-07
443 2.487106485205e-07
444 2.48367257427162e-07
445 2.48029522254001e-07
446 2.47686301690919e-07
447 2.47346008563909e-07
448 2.46999718456209e-07
449 2.46659396907489e-07
450 2.46315011054321e-07
451 2.45975485313465e-07
452 2.45788982056183e-07
453 2.45424843114961e-07
454 2.45073692894948e-07
455 2.45414724986404e-07
456 2.44458476572618e-07
457 2.43953991230228e-07
458 2.43651271603085e-07
459 2.43314559611463e-07
460 2.42951074369557e-07
461 2.42613879208875e-07
462 2.42259403648859e-07
463 2.41936845668533e-07
464 2.41604084294522e-07
465 2.4130031306413e-07
466 2.40981677279706e-07
467 2.40533495343698e-07
468 2.40200762391396e-07
469 2.39890738384929e-07
470 2.39530010048838e-07
471 2.3923351477606e-07
472 2.38888304693319e-07
473 2.38525700524406e-07
474 2.38172134459091e-07
475 2.37818341020102e-07
476 2.37466693420174e-07
477 2.37115060031101e-07
478 2.36765032468611e-07
479 2.36414862797574e-07
480 2.36062348335508e-07
481 2.35706977491645e-07
482 2.35327874520408e-07
483 2.34997500569989e-07
484 2.34616280181399e-07
485 2.34292599543551e-07
486 2.33939019267382e-07
487 2.33553976158873e-07
488 2.33233791391285e-07
489 2.32873787808785e-07
490 2.32473269079492e-07
491 2.32145666245742e-07
492 2.31743911172089e-07
493 2.31419576834924e-07
494 2.31037873277273e-07
495 2.30647628995939e-07
496 2.30281230528817e-07
497 2.29917191063578e-07
498 2.29630657599955e-07
499 2.29214151659107e-07
500 2.28842921501382e-07
501 2.28597315299339e-07
502 2.28139541036398e-07
503 2.27793393037246e-07
504 2.27538095032287e-07
505 2.27067843638906e-07
506 2.26703022576658e-07
507 2.26321830609777e-07
508 2.26074220677219e-07
509 2.25601013426058e-07
510 2.2523339282543e-07
511 2.24906855805784e-07
512 2.24473907906031e-07
513 2.24226511136294e-07
514 2.23759471396079e-07
515 2.23382428998775e-07
516 2.23056190407078e-07
517 2.22635080149303e-07
518 2.22310191588804e-07
519 2.21884803863759e-07
520 2.21559218971379e-07
521 2.21134058620009e-07
522 2.20856435362293e-07
523 2.20379732240872e-07
524 2.20069665601841e-07
525 2.19587107608277e-07
526 2.1923663950929e-07
527 2.18904474991177e-07
528 2.18413433117348e-07
529 2.1810620864926e-07
530 2.17600089058578e-07
531 2.1728313015501e-07
532 2.1693611529372e-07
533 2.16547348941276e-07
534 2.16156607280027e-07
535 2.1576499875664e-07
536 2.15373006540176e-07
537 2.14949849919321e-07
538 2.14587885238871e-07
539 2.14169133982978e-07
540 2.13802806570129e-07
541 2.13405897397934e-07
542 2.13016832617541e-07
543 2.12620150819021e-07
544 2.12231768159654e-07
545 2.11735610378128e-07
546 2.11423270002342e-07
547 2.11100157798683e-07
548 2.10667536748588e-07
549 2.10238198405932e-07
550 2.0979322812309e-07
551 2.0936110445291e-07
552 2.08910563515019e-07
553 2.08490234854253e-07
554 2.08048234640046e-07
555 2.07625831194491e-07
556 2.0720123927731e-07
557 2.06766074484221e-07
558 2.06351927545256e-07
559 2.05946491860232e-07
560 2.0552816692998e-07
561 2.0511102150067e-07
562 2.04693890282215e-07
563 2.04277426973931e-07
564 2.0385783727761e-07
565 2.03435831735987e-07
566 2.03015360966674e-07
567 2.02543901650643e-07
568 2.02893915002278e-07
569 2.02371040813887e-07
570 2.02040737917741e-07
571 2.01592740722845e-07
572 2.01182416503798e-07
573 2.00729189714366e-07
574 2.00357064272794e-07
575 1.99879437445816e-07
576 1.99517515397929e-07
577 1.99016682245201e-07
578 1.98672154283486e-07
579 1.98103009552142e-07
580 1.97678900804021e-07
581 1.97322847839132e-07
582 1.96817424580331e-07
583 1.96310395494947e-07
584 1.96021545662006e-07
585 1.95528755853047e-07
586 1.95197358721089e-07
587 1.94669823372351e-07
588 1.94337758330221e-07
589 1.93733455944312e-07
590 1.93451327845651e-07
591 1.93023694805561e-07
592 1.92431343748467e-07
593 1.92086730521623e-07
594 1.91576944530425e-07
595 1.91242293112737e-07
596 1.90257750887213e-07
597 1.89836470099181e-07
598 1.89438424058608e-07
599 1.89028142472125e-07
600 1.88602584216824e-07
601 1.88171114245961e-07
602 1.87699839671041e-07
603 1.87283674790706e-07
604 1.86929938195135e-07
605 1.86890005693385e-07
606 1.86550465741675e-07
607 1.85978947797594e-07
608 1.85607987646108e-07
609 1.85122516427327e-07
610 1.84820962090271e-07
611 1.84428841976114e-07
612 1.83374325501973e-07
613 1.82958601158134e-07
614 1.82493963052366e-07
615 1.82089905820249e-07
616 1.81640444907316e-07
617 1.8118443279036e-07
618 1.80745260536241e-07
619 1.80751229095222e-07
620 1.80300162355707e-07
621 1.79748283812842e-07
622 1.78993218469259e-07
623 1.78507463033384e-07
624 1.78114078153158e-07
625 1.77755183017325e-07
626 1.77713388893608e-07
627 1.76815447616718e-07
628 1.76952426045318e-07
629 1.76295230858159e-07
630 1.75488295894866e-07
631 1.75666343693592e-07
632 1.74576868516851e-07
633 1.74262680729953e-07
634 1.74151921328303e-07
635 1.73697458194511e-07
636 1.72928466213307e-07
637 1.72448849866669e-07
638 1.72404327258846e-07
639 1.71954425809417e-07
640 1.71306552942951e-07
641 1.70707309621321e-07
642 1.70675320987357e-07
643 1.70023398027297e-07
644 1.6941747560395e-07
645 1.68995001104122e-07
646 1.6854053797033e-07
647 1.69001438621308e-07
648 1.68198582173318e-07
649 1.67674002682361e-07
650 1.6782125555892e-07
651 1.67298239261982e-07
652 1.66939685186662e-07
653 1.6571554795064e-07
654 1.6608079533853e-07
655 1.65605456459161e-07
656 1.64256704238142e-07
657 1.63983841616755e-07
658 1.64367222055262e-07
659 1.63895535365555e-07
660 1.63452284596133e-07
661 1.63056526503169e-07
662 1.6262612234641e-07
663 1.62183710017416e-07
664 1.61874552873087e-07
665 1.61019457323164e-07
666 1.61011783461618e-07
667 1.60393696546635e-07
668 1.60040286800722e-07
669 1.59672339350436e-07
670 1.59092181206688e-07
671 1.58685622864141e-07
672 1.58264413130382e-07
673 1.57837959591234e-07
674 1.5742224945825e-07
675 1.57228029706857e-07
676 1.56787692162652e-07
677 1.56452685473596e-07
678 1.55228406129027e-07
679 1.55538472768058e-07
680 1.55104899590697e-07
681 1.54727544554589e-07
682 1.53371914279887e-07
683 1.5386052609756e-07
684 1.52568119915486e-07
685 1.53138699943156e-07
686 1.5181645096618e-07
687 1.52076793824563e-07
688 1.51059097674988e-07
689 1.5101109340776e-07
690 1.51102526046998e-07
691 1.50623364447711e-07
692 1.50278168575824e-07
693 1.49899847201596e-07
694 1.49495960499735e-07
695 1.49104323554639e-07
696 1.48700110003119e-07
697 1.48292656376725e-07
698 1.4788300006785e-07
699 1.47527742910825e-07
700 1.47078253576183e-07
701 1.46730783967541e-07
702 1.46353031027502e-07
703 1.46029194070252e-07
704 1.44769387588894e-07
705 1.44588597095208e-07
706 1.44050432027143e-07
707 1.43820969356057e-07
708 1.44242221722379e-07
709 1.43766513360788e-07
710 1.43515762829338e-07
711 1.43185758361142e-07
712 1.42774624123376e-07
713 1.42372030609295e-07
714 1.41090154670565e-07
715 1.40830934469705e-07
716 1.4051205710075e-07
717 1.40240260293467e-07
718 1.40081141353221e-07
719 1.40272220505722e-07
720 1.3996640291225e-07
721 1.39579384494937e-07
722 1.38420531925476e-07
723 1.38144827133146e-07
724 1.38356398338146e-07
725 1.38866326437892e-07
726 1.37810289402296e-07
727 1.38062162591268e-07
728 1.37659554866332e-07
729 1.37244640541212e-07
730 1.36464791467006e-07
731 1.36010612550308e-07
732 1.35967297865136e-07
733 1.35474266471647e-07
734 1.34797645046092e-07
735 1.35186780880758e-07
736 1.34834934328865e-07
737 1.34614481339668e-07
738 1.33937035684539e-07
739 1.33345622543857e-07
740 1.33473847085952e-07
741 1.33441062644124e-07
742 1.33022794557291e-07
743 1.31994170260441e-07
744 1.31716092255374e-07
745 1.32177319756011e-07
746 1.31772139866371e-07
747 1.31138293113509e-07
748 1.30608086124084e-07
749 1.31142570580778e-07
750 1.29989956576537e-07
751 1.29995896713808e-07
752 1.29408533666719e-07
753 1.29944368154611e-07
754 1.29591938957674e-07
755 1.29288153516427e-07
756 1.28980460090133e-07
757 1.28662563270154e-07
758 1.28378360386705e-07
759 1.2807620919375e-07
760 1.27985046560752e-07
761 1.27568554830759e-07
762 1.27382463688264e-07
763 1.27099909263961e-07
764 1.26734605032652e-07
765 1.26486256135649e-07
766 1.26251663346011e-07
767 1.257826767187e-07
768 1.24945572110846e-07
769 1.25402394246521e-07
770 1.24960166658639e-07
771 1.24110542287781e-07
772 1.23818978181589e-07
773 1.23586261224773e-07
774 1.23303919963291e-07
775 1.23757743608621e-07
776 1.23234443094589e-07
777 1.23235452065273e-07
778 1.22850323691637e-07
779 1.21958549925694e-07
780 1.22388982504162e-07
781 1.21285992804587e-07
782 1.21840884048652e-07
783 1.21648881190595e-07
784 1.21296764632461e-07
785 1.21022480925603e-07
786 1.19930092523646e-07
787 1.19732476377976e-07
788 1.19946733434517e-07
789 1.20143354820357e-07
790 1.19083246374885e-07
791 1.19570984224993e-07
792 1.19394172770626e-07
793 1.19018551458794e-07
794 1.1893541795871e-07
795 1.18414938299338e-07
796 1.18315050201545e-07
797 1.18186754605176e-07
798 1.17860736281727e-07
799 1.1763294338607e-07
800 1.1670945809783e-07
801 1.16328166654966e-07
802 1.16019606366535e-07
803 1.1587839310323e-07
804 1.1579269454387e-07
805 1.15807843314997e-07
806 1.15204137784986e-07
807 1.15264313649277e-07
808 1.15011381751629e-07
809 1.14561792941004e-07
810 1.14548925012059e-07
811 1.14296611286591e-07
812 1.14073436918716e-07
813 1.1360529583726e-07
814 1.13643743304692e-07
815 1.13385098643448e-07
816 1.13194367656888e-07
817 1.12997327050834e-07
818 1.12349887615437e-07
819 1.12107159111474e-07
820 1.11998673446578e-07
821 1.12277021457885e-07
822 1.12234438631731e-07
823 1.11369466537781e-07
824 1.1086888207501e-07
825 1.1072768302256e-07
826 1.11343915421003e-07
827 1.10286762833312e-07
828 1.10579094325658e-07
829 1.10096614491795e-07
830 1.09587361407648e-07
831 1.10015179188849e-07
832 1.09404311388062e-07
833 1.09023062577762e-07
834 1.08849526725407e-07
835 1.08551773791987e-07
836 1.08269816223583e-07
837 1.08044211799552e-07
838 1.07820675054882e-07
839 1.07600662602181e-07
840 1.07427730711152e-07
841 1.07221275413849e-07
842 1.07014045624965e-07
843 1.06807547695098e-07
844 1.0662321159316e-07
845 1.0668035343997e-07
846 1.06497786589443e-07
847 1.07000715843242e-07
848 1.06139353306389e-07
849 1.05893221302722e-07
850 1.05481205991964e-07
851 1.05300884456483e-07
852 1.05135562478154e-07
853 1.04966296987641e-07
854 1.05459868393609e-07
855 1.04809238621328e-07
856 1.04610222706469e-07
857 1.04367792630455e-07
858 1.04193311756262e-07
859 1.03889078673092e-07
860 1.03698305053967e-07
861 1.0419589813182e-07
862 1.03555031216729e-07
863 1.03394555139857e-07
864 1.03153809050127e-07
865 1.02993006123597e-07
866 1.03385239924592e-07
867 1.03379534266423e-07
868 1.02675258517593e-07
869 1.02865222117998e-07
870 1.02085259356954e-07
871 1.02736045448637e-07
872 1.0179980591829e-07
873 1.02257217804436e-07
874 1.01941949992579e-07
875 1.01630796223162e-07
876 1.0133751260355e-07
877 1.01321035117508e-07
878 1.01328829771319e-07
879 1.00763486443611e-07
880 1.00549613080148e-07
881 1.00389030421866e-07
882 1.00473648956267e-07
883 1.00340031394808e-07
884 1.00082985454719e-07
885 9.98951108499568e-08
886 9.97877407371561e-08
887 9.96613991333106e-08
888 9.99218485731035e-08
889 9.9491202831814e-08
890 9.93497550894062e-08
891 9.90946773526957e-08
892 9.97949527459241e-08
893 9.86451667017718e-08
894 9.86852768392055e-08
895 9.85640511430574e-08
896 9.84304620033072e-08
897 9.8535608117345e-08
898 9.81372636488231e-08
899 9.81206156325243e-08
900 9.78843672783114e-08
901 9.77494849507821e-08
902 9.81005712219485e-08
903 9.76596581381273e-08
904 9.75426530658297e-08
905 9.72817773003953e-08
906 9.71167821717245e-08
907 9.69823616969734e-08
908 9.6851671571585e-08
909 9.67778888139037e-08
910 9.6600736299024e-08
911 9.66863851203925e-08
912 9.6539459093492e-08
913 9.61340944627409e-08
914 9.61601003268697e-08
915 9.6142798611254e-08
916 9.61116910502824e-08
917 9.57849906058073e-08
918 9.61541175570346e-08
919 9.54655305918095e-08
920 9.53944763182335e-08
921 9.62215267463762e-08
922 9.51886391931112e-08
923 9.51515843894413e-08
924 9.48904670394768e-08
925 9.47517619920291e-08
926 9.47080422974977e-08
927 9.49622176449338e-08
928 9.45833207310898e-08
929 9.42732043540673e-08
930 9.43783078355409e-08
931 9.41301721013588e-08
932 9.40831199613967e-08
933 9.39855837600589e-08
934 9.39743287631245e-08
935 9.3689187963264e-08
936 9.36073121238223e-08
937 9.3626340458286e-08
938 9.33812884795771e-08
939 9.33087278554012e-08
940 9.39026136848042e-08
941 9.30970216472815e-08
942 9.29263421767246e-08
943 9.28285288637198e-08
944 9.3735941675277e-08
945 9.29663528381752e-08
946 9.24193201967682e-08
947 9.23547460729424e-08
948 9.22242406886653e-08
949 9.21787091101578e-08
950 9.21095377748316e-08
951 9.19001479360304e-08
952 9.21820557664432e-08
953 9.17195208671728e-08
954 9.22040896966791e-08
955 9.14302731302996e-08
956 9.18372151659241e-08
957 9.11596913510948e-08
958 9.10334989612238e-08
959 9.12443240963512e-08
960 9.17066387273735e-08
961 9.1311221694923e-08
962 9.05535770812094e-08
963 9.0563325727544e-08
964 9.02676546843395e-08
965 9.01356216331806e-08
966 9.00354422128657e-08
967 8.99354617445169e-08
968 8.9847240758445e-08
969 8.97419738521421e-08
970 8.99959431421848e-08
971 8.98665604154303e-08
972 8.94028886477827e-08
973 8.93375400323748e-08
974 8.92803697638556e-08
975 8.91172717842892e-08
976 8.91583411544161e-08
977 8.90565345912364e-08
978 8.88834463808053e-08
979 8.87408120320288e-08
980 8.87068694055415e-08
981 8.86102640151876e-08
982 8.92604745672543e-08
983 8.84025581626702e-08
984 8.82764581433548e-08
985 8.82644641819752e-08
986 8.78639099255452e-08
987 8.78991457398115e-08
988 8.77500596629943e-08
989 8.77773658203296e-08
990 8.7469771870019e-08
991 8.74669012773666e-08
992 8.74238068604427e-08
993 8.73326868600088e-08
994 8.71838636840039e-08
995 8.71446275141352e-08
996 8.70422667276216e-08
997 8.70264500463236e-08
998 8.76694059570582e-08
999 8.67249667635406e-08
1000 8.67080345301474e-08
1001 8.65172609110232e-08
1002 8.65608882349989e-08
1003 8.63910472048701e-08
1004 8.62311395621873e-08
1005 8.61393090190177e-08
1006 8.60373106092993e-08
1007 8.59125748320366e-08
1008 8.58198845321567e-08
1009 8.57324806702309e-08
1010 8.5640863289882e-08
1011 8.55602380056553e-08
1012 8.54819575124566e-08
1013 8.54056452226359e-08
1014 8.53324948479894e-08
1015 8.52650430260837e-08
1016 8.51460200124166e-08
1017 8.50573655952758e-08
1018 8.49765626753651e-08
1019 8.49027870231112e-08
1020 8.49346335485279e-08
1021 8.4821536461277e-08
1022 8.4680699785622e-08
1023 8.46179801783364e-08
1024 8.45234566781983e-08
1025 8.44495531282519e-08
1026 8.4372778985653e-08
1027 8.43644158976531e-08
1028 8.42191099081901e-08
1029 8.416635211006e-08
1030 8.41966851794496e-08
1031 8.3786027005317e-08
1032 8.38389908608406e-08
1033 8.43291303453952e-08
1034 8.4658147159189e-08
1035 8.4735063410335e-08
1036 8.46624317318856e-08
1037 8.44664000965167e-08
1038 8.44328269522521e-08
1039 8.44256859977577e-08
1040 8.42886791474484e-08
1041 8.47333438969144e-08
1042 8.40293949977422e-08
1043 8.44428242885442e-08
1044 8.43701712938127e-08
1045 8.35638545026995e-08
1046 8.44350438455876e-08
1047 8.34768840718425e-08
1048 8.36322300301617e-08
1049 8.42312175564075e-08
1050 8.34458333542898e-08
1051 8.35388576092555e-08
1052 8.33077393735948e-08
1053 8.3032944075967e-08
1054 8.32777189430089e-08
1055 8.33356352814008e-08
1056 8.27201702691127e-08
1057 8.3096011849193e-08
1058 8.31414652679996e-08
1059 8.27830319849454e-08
1060 8.27671442493738e-08
1061 8.25841652840609e-08
1062 8.2595413175568e-08
1063 8.29085919917816e-08
1064 8.20706063109355e-08
1065 8.21549690499523e-08
1066 8.22591843530063e-08
1067 8.20151058178453e-08
1068 8.24339352334391e-08
1069 8.17202447933596e-08
1070 8.17817067400028e-08
1071 8.22917201048767e-08
1072 8.17378094097876e-08
1073 8.19656307271543e-08
1074 8.18725496287698e-08
1075 8.18366956423233e-08
1076 8.18805290236924e-08
1077 8.12375162695389e-08
1078 8.1076990454676e-08
1079 8.15404987974944e-08
1080 8.12459148846756e-08
1081 8.10728337796718e-08
1082 8.10309401799714e-08
1083 8.11839768743994e-08
1084 8.09633249332364e-08
1085 8.05436641826418e-08
1086 8.07635345267954e-08
1087 8.11516898124864e-08
1088 8.08375517635795e-08
1089 8.07244475709012e-08
1090 8.04183670766179e-08
1091 8.0446206141005e-08
1092 8.10358784519849e-08
1093 8.01741393274824e-08
1094 8.02966013679907e-08
1095 7.99833657083582e-08
1096 8.04523807573787e-08
1097 8.02740132144208e-08
1098 8.01086343926727e-08
1099 8.03199213805783e-08
1100 8.01420370066808e-08
1101 8.00731001504573e-08
1102 7.99811488150226e-08
1103 7.96451260498543e-08
1104 7.99456643107987e-08
1105 7.98959902681418e-08
1106 7.94889203348248e-08
1107 7.98270107793542e-08
1108 7.95513770412981e-08
1109 7.91397880561817e-08
1110 7.94707872842082e-08
1111 7.93933949694292e-08
1112 7.93289984812873e-08
1113 7.92448062725271e-08
1114 7.96551375970012e-08
1115 7.94226693301425e-08
1116 7.88782585914305e-08
1117 7.89709773130198e-08
1118 7.89395429023898e-08
1119 7.84877514092841e-08
1120 7.84601255077177e-08
1121 7.91077852113631e-08
1122 7.89637226716877e-08
1123 7.86450726764087e-08
1124 7.81860052256889e-08
1125 7.89484317920142e-08
1126 7.86582887712939e-08
1127 7.85962086524705e-08
1128 7.83451596930718e-08
1129 7.83343310217788e-08
1130 7.83467299925178e-08
1131 7.81123361548453e-08
1132 7.83557041472704e-08
1133 7.83315030616905e-08
1134 7.82955282829789e-08
1135 7.8061226815862e-08
1136 7.75679964704068e-08
1137 7.81210403033583e-08
1138 7.72879431565343e-08
1139 7.79761677449642e-08
1140 7.78610242946343e-08
1141 7.77516859784555e-08
1142 7.77651649741529e-08
1143 7.76934072632685e-08
1144 7.76284068138011e-08
1145 7.69940911027334e-08
1146 7.73611645854544e-08
1147 7.65473160413421e-08
1148 7.71914656638728e-08
1149 7.64710677003677e-08
1150 7.71632358009811e-08
1151 7.67996866102294e-08
1152 7.7024822076055e-08
1153 7.61904885848708e-08
1154 7.68923413829725e-08
1155 7.67934196233e-08
1156 7.64890231153004e-08
1157 7.67065344575713e-08
1158 7.59268985461858e-08
1159 7.59391696192324e-08
1160 7.60596634563626e-08
1161 7.66046994726821e-08
1162 7.64095986482971e-08
1163 7.63551639693105e-08
1164 7.5677675681618e-08
1165 7.62636460649446e-08
1166 7.54865254748438e-08
1167 7.60724461201789e-08
1168 7.53849036527754e-08
1169 7.54797184754352e-08
1170 7.55255271656097e-08
1171 7.54632694111024e-08
1172 7.59257900995181e-08
1173 7.58286091695481e-08
1174 7.50777218172516e-08
1175 7.51163042878034e-08
1176 7.51579776192557e-08
1177 7.51033368828757e-08
1178 7.56412248392735e-08
1179 7.49178283854235e-08
1180 7.54776223743647e-08
1181 7.47901935937989e-08
1182 7.53738333969523e-08
1183 7.49219637441456e-08
1184 7.50987183550933e-08
1185 7.51734532400405e-08
1186 7.45220489761778e-08
1187 7.49819406564711e-08
1188 7.46366879411653e-08
1189 7.49004200883974e-08
1190 7.47588444482972e-08
1191 7.41483106025953e-08
1192 7.43040757811286e-08
1193 7.47423385405455e-08
1194 7.44372172789554e-08
1195 7.43234309652507e-08
1196 7.39031023044845e-08
1197 7.39391481374696e-08
1198 7.40645020869124e-08
1199 7.38185050863649e-08
1200 7.45438200056014e-08
1201 7.36835090719978e-08
1202 7.42274863796411e-08
1203 7.35658360895286e-08
1204 7.36087528707685e-08
1205 7.43571675343446e-08
1206 7.32932221580995e-08
1207 7.38779988296301e-08
1208 7.40477190674937e-08
1209 7.37454044497099e-08
1210 7.38056229465656e-08
1211 7.3484834217652e-08
1212 7.30867668607971e-08
1213 7.31661913278003e-08
1214 7.38045642378893e-08
1215 7.30199403164988e-08
1216 7.35254701567101e-08
1217 7.32752241106027e-08
1218 7.36130090217557e-08
1219 7.30899074596891e-08
1220 7.34569809424102e-08
1221 7.27997004901226e-08
1222 7.26178157606228e-08
1223 7.32756149091074e-08
1224 7.31297546963106e-08
1225 7.26594393540836e-08
1226 7.24100246429771e-08
1227 7.32174498807581e-08
1228 7.2302448472783e-08
1229 7.27879765349826e-08
1230 7.21816633131311e-08
1231 7.29929183762579e-08
1232 7.2956936492119e-08
1233 7.28839566477291e-08
1234 7.28433775520898e-08
1235 7.27562650126856e-08
1236 7.22705095768106e-08
1237 7.23384232514945e-08
1238 7.2128052863718e-08
1239 7.2072417367508e-08
1240 7.22380661954958e-08
1241 7.13280741138078e-08
1242 7.21702662076495e-08
1243 7.18354655759867e-08
1244 7.16809367418136e-08
1245 7.1818362812337e-08
1246 7.15582260113479e-08
1247 7.18651378406321e-08
1248 7.13534547003292e-08
1249 7.16874595241279e-08
1250 7.0957547393391e-08
1251 7.11874363901188e-08
1252 7.1191195161191e-08
1253 7.11647700768481e-08
1254 7.10612724219573e-08
1255 7.1082368435782e-08
1256 7.10359842059916e-08
1257 7.10024821160005e-08
1258 7.09711045487893e-08
1259 7.09280314481475e-08
1260 7.08935914417452e-08
1261 7.13924137585309e-08
1262 7.04727085576451e-08
1263 7.06900777913688e-08
1264 7.00851501278521e-08
1265 7.03861857687116e-08
1266 7.0714619937462e-08
1267 7.10165295458864e-08
1268 7.03858020756343e-08
1269 7.03992384387675e-08
1270 7.01972027172815e-08
1271 7.00185012192378e-08
1272 7.03205884633462e-08
1273 7.03033862237135e-08
1274 7.03859939221729e-08
1275 6.99013398275383e-08
1276 6.99382383118063e-08
1277 6.98382649488849e-08
1278 6.99582969332369e-08
1279 7.05129536981985e-08
1280 6.96989275184023e-08
1281 6.97601905130796e-08
1282 7.0118588268997e-08
1283 6.94846704618612e-08
1284 6.94331987460828e-08
1285 6.97109499014914e-08
1286 6.89653774088583e-08
1287 6.89480899040973e-08
1288 6.88761545575289e-08
1289 6.95641872994202e-08
1290 6.94954991331542e-08
1291 6.91108539285779e-08
1292 6.94203805551297e-08
1293 6.86579610942317e-08
1294 6.89666777020648e-08
1295 6.91826116394623e-08
1296 6.90174815076716e-08
1297 6.89965418132488e-08
1298 6.92849013717023e-08
1299 6.87673136212652e-08
1300 6.84114453974871e-08
1301 6.87659920117767e-08
1302 6.92899106979894e-08
1303 6.85738399397451e-08
1304 6.85414747181312e-08
1305 6.87746108951615e-08
1306 6.82289567066618e-08
1307 6.8326798441376e-08
1308 6.82805776364148e-08
1309 6.85652068455056e-08
1310 6.80386165186064e-08
1311 6.79315306229e-08
1312 6.81687168935241e-08
1313 6.81360390331065e-08
1314 6.85005048239873e-08
1315 6.78854092939218e-08
1316 6.79246596746452e-08
1317 6.79025262684263e-08
1318 6.79881466680854e-08
1319 6.78182985325293e-08
1320 6.76009506150876e-08
1321 6.8020625576537e-08
1322 6.76104434660374e-08
1323 6.76056899351352e-08
1324 6.77250184821787e-08
1325 6.7429979822009e-08
1326 6.7896102962095e-08
1327 6.78198119885565e-08
1328 6.73264679562635e-08
1329 6.75369591363051e-08
1330 6.76621425554913e-08
1331 6.72892639386191e-08
1332 6.73369768833254e-08
1333 6.76102089869346e-08
1334 6.70909727773505e-08
1335 6.72484148367403e-08
1336 6.72555984237988e-08
1337 6.75432332286618e-08
1338 6.72561242254233e-08
1339 6.72383819733113e-08
1340 6.70167850103098e-08
1341 6.74074911444222e-08
1342 6.68824071681229e-08
1343 6.71814746056043e-08
1344 6.67331576664765e-08
1345 6.68562094574554e-08
1346 6.70063613483762e-08
1347 6.67776234308803e-08
1348 6.66380017833035e-08
1349 6.69519408802444e-08
1350 6.69521540430651e-08
1351 6.69301627453933e-08
1352 6.69432722588681e-08
1353 6.68471074050103e-08
1354 6.67662760633903e-08
1355 6.65633663743392e-08
1356 6.62402825923891e-08
1357 6.63150387936184e-08
1358 6.66274573291048e-08
1359 6.65119301856976e-08
1360 6.64607497924408e-08
1361 6.63997710148578e-08
1362 6.60597549995146e-08
1363 6.63326957806021e-08
1364 6.63009345203136e-08
1365 6.62640289306182e-08
1366 6.65085480022753e-08
1367 6.62384493921309e-08
1368 6.61867147755402e-08
1369 6.60949481812168e-08
1370 6.57339711551685e-08
1371 6.60404566588113e-08
1372 6.60551364717321e-08
1373 6.61111201338827e-08
1374 6.63028885128369e-08
1375 6.60424106513346e-08
1376 6.6035106272011e-08
1377 6.5848389851908e-08
1378 6.5789258485438e-08
1379 6.57076242305266e-08
1380 6.57026291150942e-08
1381 6.56624337125322e-08
1382 6.57434000572721e-08
1383 6.54897576168878e-08
1384 6.55362129009518e-08
1385 6.55621548162344e-08
1386 6.54793197440995e-08
1387 6.55129142046462e-08
1388 6.53807532557948e-08
1389 6.54570584401881e-08
1390 6.52933564992964e-08
1391 6.53348450896374e-08
1392 6.53307878906162e-08
1393 6.49592166723778e-08
1394 6.53397975725056e-08
1395 6.48749960419082e-08
1396 6.51402274343127e-08
1397 6.49964775334411e-08
1398 6.51390550387987e-08
1399 6.50251337219743e-08
1400 6.50043148198165e-08
1401 6.49697042831576e-08
1402 6.50100417942667e-08
1403 6.49163638399841e-08
1404 6.49570708333158e-08
1405 6.48698303962192e-08
1406 6.48797495728104e-08
1407 6.48050288987179e-08
1408 6.50235065791094e-08
1409 6.48304663286581e-08
1410 6.45657465270233e-08
1411 6.47937881126381e-08
1412 6.46385203140198e-08
1413 6.44160067508892e-08
1414 6.45990283487663e-08
1415 6.42647393078732e-08
1416 6.4704252622505e-08
1417 6.46871711751373e-08
1418 6.45108642061132e-08
1419 6.45806679244743e-08
1420 6.42785380478017e-08
1421 6.41119939359669e-08
1422 6.47250928409449e-08
1423 6.46032844997535e-08
1424 6.44284341433377e-08
1425 6.42374260451106e-08
1426 6.42704307551867e-08
1427 6.41826076730467e-08
1428 6.42061621647372e-08
1429 6.38371844274843e-08
1430 6.3871532063331e-08
1431 6.40790460693097e-08
1432 6.39512620637106e-08
1433 6.3930983174032e-08
1434 6.4015587497579e-08
1435 6.35053325481749e-08
1436 6.41570139237047e-08
1437 6.40512780591962e-08
1438 6.39710933114657e-08
1439 6.40241069049807e-08
1440 6.39675050706501e-08
1441 6.36712798041117e-08
1442 6.37953263549207e-08
1443 6.3699083341362e-08
1444 6.34459098591833e-08
1445 6.38301713706824e-08
1446 6.32440944059454e-08
1447 6.35718819808062e-08
1448 6.35575361229712e-08
1449 6.34287431466873e-08
1450 6.34673327226665e-08
1451 6.33773709068919e-08
1452 6.36479171589599e-08
1453 6.35625312384036e-08
1454 6.36011776578016e-08
1455 6.33235970326496e-08
1456 6.33160652796505e-08
1457 6.32365697583737e-08
1458 6.31079615232011e-08
1459 6.29853715850004e-08
1460 6.29512939553933e-08
1461 6.30697627457266e-08
1462 6.27640304173838e-08
1463 6.30565182291321e-08
1464 6.27291285582032e-08
1465 6.2657036892233e-08
1466 6.29095637805221e-08
1467 6.25649150265417e-08
1468 6.28623268994488e-08
1469 6.24394260739791e-08
1470 6.23046148007234e-08
1471 6.27215825943495e-08
1472 6.23260518750612e-08
1473 6.26653857693782e-08
1474 6.2623158214592e-08
1475 6.23233020746738e-08
1476 6.28877572239617e-08
1477 6.27398293318038e-08
1478 6.25980263180281e-08
1479 6.21292031155463e-08
1480 6.24714573405072e-08
1481 6.24196445642156e-08
1482 6.24150615635699e-08
1483 6.20023783426404e-08
1484 6.26350669108433e-08
1485 6.25911908969101e-08
1486 6.27765217586784e-08
1487 6.2636928532811e-08
1488 6.23032647695254e-08
1489 6.2337093709175e-08
1490 6.2025875990912e-08
1491 6.19138518231921e-08
1492 6.22118250248604e-08
1493 6.21201294848106e-08
1494 6.18379658590129e-08
1495 6.17674089653519e-08
1496 6.23994367288105e-08
1497 6.25120790687106e-08
1498 6.22016287366023e-08
1499 6.24173424057517e-08
1500 6.20975484366681e-08
1501 6.19800317736008e-08
1502 6.18309243805015e-08
1503 6.18165003629656e-08
1504 6.15407884652086e-08
1505 6.23302227609202e-08
1506 6.19565767578933e-08
1507 6.17541360270479e-08
1508 6.2277862866722e-08
1509 6.18146103192885e-08
1510 6.18778983607626e-08
1511 6.17712814232618e-08
1512 6.16885387216826e-08
1513 6.16610549286634e-08
1514 6.13257356008035e-08
1515 6.20503186610222e-08
1516 6.16656592455911e-08
1517 6.12823214396485e-08
1518 6.16929156649348e-08
1519 6.1569409126605e-08
1520 6.15156139360806e-08
1521 6.15027246908539e-08
1522 6.14875403925907e-08
1523 6.1510753823768e-08
1524 6.14479205296448e-08
1525 6.14434441104095e-08
1526 6.14018773603675e-08
1527 6.13999660004083e-08
1528 6.14896151773792e-08
1529 6.1479319413138e-08
1530 6.13651351955014e-08
1531 6.1402602113958e-08
1532 6.14763564499299e-08
1533 6.14497608353304e-08
1534 6.14023178968637e-08
1535 6.12544184264152e-08
1536 6.09918444638424e-08
1537 6.12674782018985e-08
1538 6.07688335207968e-08
1539 6.10780617193996e-08
1540 6.07108034955672e-08
1541 6.10784240961948e-08
1542 6.10404597978231e-08
1543 6.08639609822603e-08
1544 6.06490715426844e-08
1545 6.08589445505459e-08
1546 6.08689774139748e-08
1547 6.05157950417379e-08
1548 6.07946830655237e-08
1549 6.07685066711383e-08
1550 6.07526189355667e-08
1551 6.04501622092357e-08
1552 6.06459167329376e-08
1553 6.04359158273837e-08
1554 6.07804793162359e-08
1555 6.04326686470813e-08
1556 6.05656396146514e-08
1557 6.08223729159363e-08
1558 6.04798557901631e-08
1559 6.06603762776103e-08
1560 6.0504412147111e-08
1561 6.03593619530329e-08
1562 6.05644103757186e-08
1563 6.06144610060255e-08
1564 6.02299792262784e-08
1565 6.01647087705715e-08
1566 6.02365730628662e-08
1567 6.05209820037089e-08
1568 6.01217564621948e-08
1569 6.01670819833089e-08
1570 6.0047369743188e-08
1571 6.00166814024305e-08
1572 6.02570366936561e-08
1573 6.0203504403944e-08
1574 6.00843605980117e-08
1575 5.99364256004264e-08
1576 5.99379603727357e-08
1577 5.99029661429995e-08
1578 5.98507341464938e-08
1579 6.01817831125118e-08
1580 5.98141909335936e-08
1581 5.9776326111205e-08
1582 5.98287641651041e-08
1583 5.99492437913796e-08
1584 5.9946927422061e-08
1585 5.98102971594017e-08
1586 5.96918567907778e-08
1587 6.00699365804758e-08
1588 5.96171787492494e-08
1589 5.95912652556763e-08
1590 5.97968394799864e-08
1591 5.94609268489421e-08
1592 5.96039342326549e-08
1593 5.95038791573188e-08
1594 5.98004206153746e-08
1595 5.94734643755146e-08
1596 5.94406159848404e-08
1597 5.93903735079948e-08
1598 5.95376050682717e-08
1599 5.95488245380693e-08
1600 5.9434835719685e-08
1601 5.94344982118855e-08
1602 5.95672844383444e-08
1603 5.95191700369924e-08
1604 5.94971893974616e-08
1605 5.94779976381687e-08
1606 5.96427369714547e-08
1607 5.93340594434721e-08
1608 5.92841367108576e-08
1609 5.92146811584371e-08
1610 5.91889524059752e-08
1611 5.93175677465752e-08
1612 5.92639040064569e-08
1613 5.93693023631658e-08
1614 5.91571058805584e-08
1615 5.91567363983359e-08
1616 5.94120770358586e-08
1617 5.89597917155515e-08
1618 5.90386619592209e-08
1619 5.90258437682678e-08
1620 5.91107358616227e-08
1621 5.90361217689406e-08
1622 5.90407935874282e-08
1623 5.89870801093184e-08
1624 5.8982870143609e-08
1625 5.88955728630935e-08
1626 5.88670623358212e-08
1627 5.88814366153656e-08
1628 5.881597431312e-08
1629 5.88440762783193e-08
1630 5.87011044217434e-08
1631 5.87943169705341e-08
1632 5.87007775720849e-08
1633 5.93515139257761e-08
1634 5.8713396811072e-08
1635 5.86826907067461e-08
1636 5.86824810966391e-08
1637 5.85929029739418e-08
1638 5.86537325375502e-08
1639 5.85529917884742e-08
1640 5.86339687913551e-08
1641 5.85170312206174e-08
1642 5.85737822689225e-08
1643 5.84747930076901e-08
1644 5.84514410206793e-08
1645 5.84273358583687e-08
1646 5.84098209799322e-08
1647 5.83760773054109e-08
1648 5.8447529482919e-08
1649 5.84256092395208e-08
1650 5.8341644404436e-08
1651 5.83170383094966e-08
1652 5.83781485374857e-08
1653 5.82326862286209e-08
1654 5.83077834903634e-08
1655 5.82459591669249e-08
1656 5.82561980877472e-08
1657 5.81893537798805e-08
1658 5.8191240270844e-08
1659 5.81660479781476e-08
1660 5.83858614788824e-08
1661 5.81600332338894e-08
1662 5.81361341289721e-08
1663 5.81397401333561e-08
1664 5.80731587263017e-08
1665 5.80487480306147e-08
1666 5.84395927205605e-08
1667 5.84490926769377e-08
1668 5.81457371140459e-08
1669 5.84126738090163e-08
1670 5.83495385342303e-08
1671 5.80452024223632e-08
1672 5.83121853026114e-08
1673 5.83069699189309e-08
1674 5.81684957978723e-08
1675 5.83517127950017e-08
1676 5.82578287833257e-08
1677 5.7940603426232e-08
1678 5.82655417247224e-08
1679 5.81902206420182e-08
1680 5.78766510272999e-08
1681 5.82002712690155e-08
1682 5.81367807228617e-08
1683 5.78190402222845e-08
1684 5.81482062500527e-08
1685 5.8152604509587e-08
1686 5.80959316209828e-08
1687 5.78059520250918e-08
1688 5.8066131458645e-08
1689 5.80213601608648e-08
1690 5.81028984925069e-08
1691 5.80037138320222e-08
1692 5.78514232074667e-08
1693 5.76901548754449e-08
1694 5.79650816234789e-08
1695 5.79636783015758e-08
1696 5.79534713551766e-08
1697 5.79285845958566e-08
1698 5.79124872501779e-08
1699 5.75571554861654e-08
1700 5.78465346734447e-08
1701 5.78625041214309e-08
1702 5.7837240774461e-08
1703 5.75284069270765e-08
1704 5.7679038434344e-08
1705 5.76839731536438e-08
1706 5.77305492299729e-08
1707 5.73546543591874e-08
1708 5.73327980646354e-08
1709 5.75953329473577e-08
1710 5.76494372239722e-08
1711 5.74254421792375e-08
1712 5.75266270175234e-08
1713 5.7364978545138e-08
1714 5.7552856702614e-08
1715 5.75144589731735e-08
1716 5.72428469070019e-08
1717 5.74683234333406e-08
1718 5.74058987012904e-08
1719 5.74343772541397e-08
1720 5.73990917018818e-08
1721 5.73229392841768e-08
1722 5.73646232737701e-08
1723 5.73274654414035e-08
1724 5.71435805341025e-08
1725 5.73531337977329e-08
1726 5.73641436574235e-08
1727 5.73225662492405e-08
1728 5.69740095102134e-08
1729 5.69407276884704e-08
1730 5.72521798858361e-08
1731 5.7002406350648e-08
1732 5.73750646992721e-08
1733 5.68624507479853e-08
1734 5.72837492995859e-08
1735 5.69228397750976e-08
1736 5.70254030662909e-08
1737 5.71340237343065e-08
1738 5.71070870591939e-08
1739 5.67951659036225e-08
1740 5.680758263793e-08
1741 5.71538301130659e-08
1742 5.71700979890011e-08
1743 5.69596281252416e-08
1744 5.71304923369098e-08
1745 5.71385108116829e-08
1746 5.67938975848392e-08
1747 5.70920128950547e-08
1748 5.6827591521369e-08
1749 5.67490126002212e-08
1750 5.69674227790529e-08
1751 5.70424809609449e-08
1752 5.69930485028181e-08
1753 5.67937199491553e-08
1754 5.65626052662083e-08
1755 5.65702826804682e-08
1756 5.66701459092656e-08
1757 5.68289451052806e-08
1758 5.68298332837003e-08
1759 5.69550842044464e-08
1760 5.67014062369253e-08
1761 5.64673392489112e-08
1762 5.6746991106138e-08
1763 5.65122988405165e-08
1764 5.63985267376665e-08
1765 5.66828539660946e-08
1766 5.64451880791239e-08
1767 5.64001219061083e-08
1768 5.62772868306638e-08
1769 5.66592888162631e-08
1770 5.67761020420221e-08
1771 5.65437616728559e-08
1772 5.66391022971402e-08
1773 5.64965176863552e-08
1774 5.625733123793e-08
1775 5.64259714508353e-08
1776 5.65022801879422e-08
1777 5.62087372202313e-08
1778 5.60955193407153e-08
1779 5.64465025831851e-08
1780 5.65559368226332e-08
1781 5.61040991442496e-08
1782 5.60776634017657e-08
1783 5.59815163114763e-08
1784 5.58538211237192e-08
1785 5.64162760952058e-08
1786 5.60182833453382e-08
1787 5.6359123590255e-08
1788 5.64201272368337e-08
1789 5.59486323936653e-08
1790 5.61951907229741e-08
1791 5.60108155411854e-08
1792 5.58036887809976e-08
1793 5.57083446039996e-08
1794 5.56366082093973e-08
1795 5.56651116312423e-08
1796 5.55997523576934e-08
1797 5.56106769522557e-08
1798 5.55746950681169e-08
1799 5.55678418834304e-08
1800 5.56293251463558e-08
1801 5.61351995997939e-08
1802 5.55313448558081e-08
1803 5.59731851978995e-08
1804 5.56673001028685e-08
1805 5.60389921133719e-08
1806 5.55357360099151e-08
1807 5.53898757971183e-08
1808 5.54527304075236e-08
1809 5.53838859218558e-08
1810 5.53428307625836e-08
1811 5.54523111873095e-08
1812 5.51800596326757e-08
1813 5.51895311673434e-08
1814 5.54057919543993e-08
1815 5.57298562853248e-08
1816 5.52929328989649e-08
1817 5.52279892929164e-08
1818 5.51314514041223e-08
1819 5.51940892989933e-08
1820 5.54486234705109e-08
1821 5.5101889273601e-08
1822 5.49694050278049e-08
1823 5.51018288774685e-08
1824 5.50417809108694e-08
1825 5.48824132806658e-08
1826 5.49005214622866e-08
1827 5.55447989825097e-08
1828 5.51410970217603e-08
1829 5.49086216494743e-08
1830 5.54462289414914e-08
1831 5.50695382628419e-08
1832 5.50413155053775e-08
1833 5.49966152618708e-08
1834 5.50245218278178e-08
1835 5.49675114314141e-08
1836 5.49375798186702e-08
1837 5.4935469506745e-08
1838 5.47384502169734e-08
1839 5.53397896396746e-08
1840 5.46276339719043e-08
1841 5.4556682727025e-08
1842 5.51528493986098e-08
1843 5.45557874431779e-08
1844 5.46108225307762e-08
1845 5.44991749507062e-08
1846 5.45497549353513e-08
1847 5.44156151249808e-08
1848 5.50029533030738e-08
1849 5.46000080703379e-08
1850 5.44190079665441e-08
1851 5.43801128571886e-08
1852 5.43911014005971e-08
1853 5.43162315125301e-08
1854 5.43182352430449e-08
1855 5.42452625040823e-08
1856 5.44753895326267e-08
1857 5.43886962134366e-08
1858 5.42348566057171e-08
1859 5.41821911781426e-08
1860 5.41570948087156e-08
1861 5.42610756326667e-08
1862 5.43223883653354e-08
1863 5.41160929401485e-08
1864 5.4087163192662e-08
1865 5.46998890627037e-08
1866 5.4055629306049e-08
1867 5.46702914050456e-08
1868 5.4083447054154e-08
1869 5.40710303198466e-08
1870 5.46093801290226e-08
1871 5.39698525869881e-08
1872 5.39981819258628e-08
1873 5.44567093641035e-08
1874 5.40003597393479e-08
1875 5.44719540584993e-08
1876 5.44733289586929e-08
1877 5.40005338223182e-08
1878 5.39067137594884e-08
1879 5.44375637900885e-08
1880 5.37881099660353e-08
1881 5.43670211072822e-08
1882 5.38622160206614e-08
1883 5.37606652528666e-08
1884 5.43673763786501e-08
1885 5.37715862947152e-08
1886 5.36915258919635e-08
1887 5.43164624389192e-08
1888 5.37778888087814e-08
1889 5.37863620309054e-08
1890 5.36828572705872e-08
1891 5.42139062531533e-08
1892 5.35799706824491e-08
1893 5.41571054668566e-08
1894 5.36225748248853e-08
1895 5.41493356820411e-08
1896 5.35350004327029e-08
1897 5.34936326346269e-08
1898 5.34545492314464e-08
1899 5.34247845962454e-08
1900 5.34204858126941e-08
1901 5.33912718481133e-08
1902 5.33490975840323e-08
1903 5.34575761435008e-08
1904 5.33407416014597e-08
1905 5.33825179616088e-08
1906 5.33629069821018e-08
1907 5.33492787724299e-08
1908 5.34919095684927e-08
1909 5.34490141035349e-08
1910 5.34671791285746e-08
1911 5.34985389322173e-08
1912 5.34995692191842e-08
1913 5.33210382513971e-08
1914 5.385280843484e-08
1915 5.32966240029964e-08
1916 5.32677475462151e-08
1917 5.38557252127703e-08
1918 5.31570769624068e-08
1919 5.30852943825266e-08
1920 5.35443476223918e-08
1921 5.32481401194218e-08
1922 5.30383736929707e-08
1923 5.30183825731001e-08
1924 5.29649852865077e-08
1925 5.30199493198324e-08
1926 5.30261203834925e-08
1927 5.36757731595117e-08
1928 5.31332240427673e-08
1929 5.3671566746516e-08
1930 5.29316750430553e-08
1931 5.2958345264642e-08
1932 5.29452464093083e-08
1933 5.29173611596434e-08
1934 5.35178834581984e-08
1935 5.28195656102071e-08
1936 5.35424611314284e-08
1937 5.28115151610109e-08
1938 5.32364836658417e-08
1939 5.29581924979539e-08
1940 5.27547214801416e-08
1941 5.33883692810377e-08
1942 5.27133252603562e-08
1943 5.27304244712923e-08
1944 5.34111634920009e-08
1945 5.27968495589448e-08
1946 5.27022656626741e-08
1947 5.26802708122887e-08
1948 5.27864365551522e-08
1949 5.2597311395175e-08
1950 5.25629033631958e-08
1951 5.25387697791757e-08
1952 5.25890087033076e-08
1953 5.25777785753689e-08
1954 5.24892556086343e-08
1955 5.2526079485915e-08
1956 5.24575121119142e-08
1957 5.24656833533754e-08
1958 5.25466496981153e-08
1959 5.26318437721329e-08
1960 5.2545889417388e-08
1961 5.24859480321993e-08
1962 5.24033616500219e-08
1963 5.31916697354973e-08
1964 5.23866390267358e-08
1965 5.23368548499548e-08
1966 5.24527195011615e-08
1967 5.23878576075276e-08
1968 5.30708206269992e-08
1969 5.24696197601315e-08
1970 5.22569507666049e-08
1971 5.23290779597119e-08
1972 5.22772047872877e-08
1973 5.23822407672014e-08
1974 5.22401855107546e-08
1975 5.29432533369345e-08
1976 5.22992351648099e-08
1977 5.24001571022836e-08
1978 5.22866550056733e-08
1979 5.23139291885855e-08
1980 5.23031680188524e-08
1981 5.22655163592844e-08
1982 5.22267740166171e-08
1983 5.21379313056514e-08
1984 5.2879084222468e-08
1985 5.21535454822697e-08
1986 5.27033492403461e-08
1987 5.27541459405256e-08
1988 5.21695966426705e-08
1989 5.25796863826145e-08
1990 5.21400984609954e-08
1991 5.23129308760417e-08
1992 5.20821750171763e-08
1993 5.23470191637898e-08
1994 5.20861860309196e-08
1995 5.20518206315046e-08
1996 5.2565201968946e-08
1997 5.25103374116043e-08
1998 5.20349914268081e-08
1999 5.26126981981179e-08
2000 5.20182297236715e-08
2001 5.19630631856671e-08
2002 5.25720587063461e-08
2003 5.19421732292358e-08
2004 5.19091258865956e-08
2005 5.19395797482503e-08
2006 5.23777572425388e-08
2007 5.24710017657526e-08
2008 5.18886054123868e-08
2009 5.23981995570466e-08
2010 5.19545899635432e-08
2011 5.18510177016651e-08
2012 5.21599652358873e-08
2013 5.18996685627826e-08
2014 5.20804519510421e-08
2015 5.18896072776442e-08
2016 5.18565919094272e-08
2017 5.18726217535459e-08
2018 5.18470208987765e-08
2019 5.17765883500942e-08
2020 5.17927638554738e-08
2021 5.17651166376254e-08
2022 5.18085947476266e-08
2023 5.18085521150624e-08
2024 5.17436440361507e-08
2025 5.24241841048934e-08
2026 5.18086515910454e-08
2027 5.23980929756362e-08
2028 5.17472322769663e-08
2029 5.17116376386184e-08
2030 5.23531369367447e-08
2031 5.17313978320999e-08
2032 5.23057579471242e-08
2033 5.16803631001039e-08
2034 5.18596010579131e-08
2035 5.16538634087738e-08
2036 5.16279214934912e-08
2037 5.22587377815853e-08
2038 5.1632270015034e-08
2039 5.16269338390885e-08
2040 5.15645410814614e-08
2041 5.14976861154537e-08
2042 5.20251681734862e-08
2043 5.17579969994131e-08
2044 5.14912876781182e-08
2045 5.21549630150275e-08
2046 5.15177873694483e-08
2047 5.2049603738169e-08
2048 5.14428926123855e-08
2049 5.14600877465909e-08
2050 5.20930996117386e-08
2051 5.14427718201205e-08
2052 5.20429850325854e-08
2053 5.14173272847529e-08
2054 5.13906748267345e-08
2055 5.13936484480837e-08
2056 5.19602245674378e-08
2057 5.13653723999141e-08
2058 5.13417539593775e-08
2059 5.13581532857188e-08
2060 5.1798270561676e-08
2061 5.13438891402984e-08
2062 5.13600042495455e-08
2063 5.12844877675889e-08
2064 5.13531972501369e-08
2065 5.13428517479042e-08
2066 5.1317524452088e-08
2067 5.13760873843694e-08
2068 5.1408754586646e-08
2069 5.13247897515612e-08
2070 5.11738917907678e-08
2071 5.1578968651711e-08
2072 5.12981834788206e-08
2073 5.12323339307841e-08
2074 5.12439726207958e-08
2075 5.14681701702102e-08
2076 5.12815425679491e-08
2077 5.12421145515418e-08
2078 5.11610274145369e-08
2079 5.19057579140281e-08
2080 5.12542648323233e-08
2081 5.11729574270703e-08
2082 5.11410362946663e-08
2083 5.12444415790014e-08
2084 5.11495059640765e-08
2085 5.10170963252676e-08
2086 5.11984090678652e-08
2087 5.11842159767184e-08
2088 5.12648554717998e-08
2089 5.11705877670465e-08
2090 5.11766060640184e-08
2091 5.11610629416737e-08
2092 5.11764071120524e-08
2093 5.11725808394203e-08
2094 5.11492785904011e-08
2095 5.11091329258306e-08
2096 5.11358599908363e-08
2097 5.10976221335113e-08
2098 5.11107423051271e-08
2099 5.10672180098481e-08
2100 5.10536999343003e-08
2101 5.10309767776107e-08
2102 5.10762951932975e-08
2103 5.099446198642e-08
2104 5.11060598284985e-08
2105 5.10121580532541e-08
2106 5.10266424669226e-08
2107 5.10262800901273e-08
2108 5.09518400804154e-08
2109 5.0941469709187e-08
2110 5.09592297248673e-08
2111 5.09533215620195e-08
2112 5.09026882866692e-08
2113 5.09230346779077e-08
2114 5.08791018205557e-08
2115 5.08498736451202e-08
2116 5.08757196371334e-08
2117 5.08691506695413e-08
2118 5.08328383830303e-08
2119 5.07809403416104e-08
2120 5.08715451985609e-08
2121 5.07807058625076e-08
2122 5.0760913694603e-08
2123 5.07887527589901e-08
2124 5.03967179099618e-08
2125 5.0692591457846e-08
2126 5.07520070414103e-08
2127 5.07198372190487e-08
2128 5.07625337320405e-08
2129 5.0770470494399e-08
2130 5.07763360246827e-08
2131 5.07554140938282e-08
2132 5.07651343184534e-08
2133 5.07890582923665e-08
2134 5.07853492592858e-08
2135 5.07344388722686e-08
2136 5.08078699112957e-08
2137 5.07513249203839e-08
2138 5.07199899857369e-08
2139 5.07286230799764e-08
2140 5.07113533387837e-08
2141 5.07200041965916e-08
2142 5.07115061054719e-08
2143 5.07823045836631e-08
2144 5.07685413708714e-08
2145 5.07609421163124e-08
2146 5.06782242837289e-08
2147 5.06709731951105e-08
2148 5.07534778648733e-08
2149 5.06474400197021e-08
2150 5.05985511267681e-08
2151 5.06428570190565e-08
2152 5.07388655535124e-08
2153 5.06756308027434e-08
2154 5.07283317574547e-08
2155 5.0662350759012e-08
2156 5.06351014450956e-08
2157 5.05590094235231e-08
2158 5.05871184941498e-08
2159 5.06575261738362e-08
2160 5.06849104908724e-08
2161 5.06345365636207e-08
2162 5.05930835004165e-08
2163 5.05079924550955e-08
2164 5.05203097134199e-08
2165 5.05368511483084e-08
2166 5.04372046350454e-08
2167 5.04625106145795e-08
2168 5.05800095140785e-08
2169 5.05885751067581e-08
2170 5.05656529981025e-08
2171 5.04528756550826e-08
2172 5.05160215880096e-08
2173 5.05601889244645e-08
2174 5.04152914970746e-08
2175 5.04766219933117e-08
2176 5.0383221150696e-08
2177 5.03362258541529e-08
2178 5.03109980343197e-08
2179 5.03038570798253e-08
2180 5.0421945729795e-08
2181 5.04428463443674e-08
2182 5.03333943413509e-08
2183 5.03548811536803e-08
2184 5.03560109166301e-08
2185 5.033468752913e-08
2186 5.03311561317332e-08
2187 5.03579933308629e-08
2188 5.34043103073145e-08
2189 5.02223578280336e-08
2190 5.02284223102833e-08
2191 5.02366042098856e-08
2192 5.35681650148945e-08
2193 5.02220984799351e-08
2194 5.0151719221958e-08
2195 5.01295325250339e-08
2196 5.01114634232636e-08
2197 5.35638768894842e-08
2198 5.32373611861203e-08
2199 5.31856514385254e-08
2200 5.30729309389244e-08
2201 5.30136894383304e-08
2202 5.29239549962313e-08
2203 5.28490495810274e-08
2204 5.26440118164828e-08
2205 5.28320924786385e-08
2206 5.27934176375311e-08
2207 5.28148227374459e-08
2208 5.27162704599959e-08
2209 5.26868113581713e-08
2210 5.26712398141171e-08
2211 5.26312255999528e-08
2212 5.2705054542912e-08
2213 5.27163841468337e-08
2214 5.26911350107184e-08
2215 5.25926786565378e-08
2216 5.25749186408575e-08
2217 5.25771781667572e-08
2218 5.26066621375776e-08
2219 5.25316607991044e-08
2220 5.25529983974593e-08
2221 5.2456982757576e-08
2222 5.25432923836888e-08
2223 5.24384020650359e-08
2224 5.25699590525619e-08
2225 5.23875023361597e-08
2226 5.24163645820863e-08
2227 5.23985548284145e-08
2228 5.24056034123532e-08
2229 5.24071950280813e-08
2230 5.24346326358227e-08
2231 5.23114742634334e-08
2232 5.23596490609179e-08
2233 5.23781622518982e-08
2234 5.22782528378229e-08
2235 5.22363237109857e-08
2236 5.221384569154e-08
2237 5.22699430405282e-08
2238 5.22790024604092e-08
2239 5.21737923975252e-08
2240 5.22285859005933e-08
2241 5.22643723854799e-08
2242 5.21728900082508e-08
2243 5.21670990849543e-08
2244 5.21119147833815e-08
2245 5.21390255414644e-08
2246 5.21835232802914e-08
2247 5.20625640376693e-08
2248 5.20734353415264e-08
2249 5.2064049071987e-08
2250 5.2108070747181e-08
2251 5.20079872501356e-08
2252 5.20201233200623e-08
2253 5.20234202383563e-08
2254 5.2037393061255e-08
2255 5.19712095581326e-08
2256 5.19806633292319e-08
2257 5.19435907619936e-08
2258 5.20031626649597e-08
2259 5.18860296949697e-08
2260 5.19577909585678e-08
2261 5.18918135128388e-08
2262 5.19354337313871e-08
2263 5.18281595418557e-08
2264 5.191558116735e-08
2265 5.17212548345469e-08
2266 5.17711420400246e-08
2267 5.18443705743721e-08
2268 5.18378655556262e-08
2269 5.17895166751714e-08
2270 5.18232461388379e-08
2271 5.16186453580758e-08
2272 5.17621216999942e-08
2273 5.1736794404178e-08
2274 5.16866265343197e-08
2275 5.1752909513425e-08
2276 5.17224769680524e-08
2277 5.17447524828185e-08
2278 5.16293745533858e-08
2279 5.16513232184934e-08
2280 5.16917815218676e-08
2281 5.16651681436997e-08
2282 5.16337443912107e-08
2283 5.1650420829219e-08
2284 5.15352098773292e-08
2285 5.11437328043485e-08
2286 5.13116198419539e-08
2287 5.16833580377352e-08
2288 5.14686711028389e-08
2289 5.14731688383563e-08
2290 5.15098861342267e-08
2291 5.15583167270961e-08
2292 5.15325631056385e-08
2293 5.15385885080377e-08
2294 5.15312947868551e-08
2295 5.15361762154498e-08
2296 5.15466531680886e-08
2297 5.15188034455605e-08
2298 5.15180218485511e-08
2299 5.15298737013836e-08
2300 5.15070723849931e-08
2301 5.15013098834061e-08
2302 5.14788425221013e-08
2303 5.14741813617547e-08
2304 5.1454943417184e-08
2305 5.14486373504042e-08
2306 5.14279072660884e-08
2307 5.13807130175792e-08
2308 5.14060474188227e-08
2309 5.1396654043856e-08
2310 5.13725417761179e-08
2311 5.13619689002098e-08
2312 5.13464293305788e-08
2313 5.13327478302017e-08
2314 5.13457116824156e-08
2315 5.13510656219296e-08
2316 5.1290680147531e-08
2317 5.12975653066405e-08
2318 5.12588194112595e-08
2319 5.12207378733365e-08
2320 5.12529716445442e-08
2321 5.12356663762148e-08
2322 5.11906144140539e-08
2323 5.11941209424549e-08
2324 5.1196089145833e-08
2325 5.11718063478384e-08
2326 5.11145650250455e-08
2327 5.1162132308491e-08
2328 5.11531368374563e-08
2329 5.11210380693683e-08
2330 5.10093762784436e-08
2331 5.10176292323195e-08
2332 5.10966948752412e-08
2333 5.11357924892764e-08
2334 5.11020843418919e-08
2335 5.11182811635535e-08
2336 5.10903248596151e-08
2337 5.10655873142696e-08
2338 5.10850135526653e-08
2339 5.1006850299018e-08
2340 5.10256512598062e-08
2341 5.09999331654853e-08
2342 5.09937549963979e-08
2343 5.09453776942337e-08
2344 5.09375475132856e-08
2345 5.09202848775203e-08
2346 5.09285378313962e-08
2347 5.09084507882562e-08
2348 5.09151689698228e-08
2349 5.08655837450078e-08
2350 5.08791409004061e-08
2351 5.08102431240331e-08
2352 5.08460402670607e-08
2353 5.07619439815699e-08
2354 5.08117992126245e-08
2355 5.06801285382608e-08
2356 5.06820967416388e-08
2357 5.07673014737975e-08
2358 5.06571531389e-08
2359 5.0658623962363e-08
2360 5.06560837720826e-08
2361 5.07189277243469e-08
2362 5.06959132451357e-08
2363 5.06281487844262e-08
2364 5.06205353190126e-08
2365 5.06003345890349e-08
2366 5.06200024119607e-08
2367 5.05999828703807e-08
2368 5.05390964633534e-08
2369 5.06358297513998e-08
2370 5.05641146730795e-08
2371 5.04888362229394e-08
2372 5.05447985688079e-08
2373 5.045805906434e-08
2374 5.04883388430244e-08
2375 5.05015620433369e-08
2376 5.04243473642418e-08
2377 5.04280244228994e-08
2378 5.05241111170562e-08
2379 5.04196684403269e-08
2380 5.03755615000046e-08
2381 5.03983095256899e-08
2382 5.03888522018769e-08
2383 5.03729218337412e-08
2384 5.03988388800281e-08
2385 5.04077952712123e-08
2386 5.03014305763827e-08
2387 5.02761174914212e-08
2388 5.03217698621938e-08
2389 5.02717369954553e-08
2390 5.03412138641579e-08
2391 5.02468004981438e-08
2392 5.02410308911294e-08
2393 5.02086123788104e-08
2394 5.02291435111601e-08
2395 5.03106818428023e-08
2396 5.02327104356937e-08
2397 5.01260970509065e-08
2398 5.03093424697454e-08
2399 5.01651555850913e-08
2400 5.01569772382027e-08
2401 5.01372703354264e-08
2402 5.01066104163783e-08
2403 5.02322556883428e-08
2404 5.01320727153143e-08
2405 5.00900618760625e-08
2406 5.00851271567626e-08
2407 5.00710122253167e-08
2408 5.00795280800048e-08
2409 5.01282393372549e-08
2410 5.00612102882769e-08
2411 5.00476993181564e-08
2412 5.00260632918526e-08
2413 5.00166379424627e-08
2414 4.99972401257764e-08
2415 5.01358670135232e-08
2416 4.99707404344463e-08
2417 5.0094303816195e-08
2418 4.99248464791435e-08
2419 4.99112005059033e-08
2420 5.00451697860171e-08
2421 4.98995689213189e-08
2422 4.98882322119698e-08
2423 4.9972822324662e-08
2424 4.99880279392073e-08
2425 4.98534049597765e-08
2426 4.98798868875383e-08
2427 4.9841531790662e-08
2428 4.98699890272292e-08
2429 4.98444840957291e-08
2430 4.98070527044092e-08
2431 4.98156147443751e-08
2432 4.98119483438586e-08
2433 4.97977126201476e-08
2434 4.97833241297485e-08
2435 4.97799135246169e-08
2436 4.97866246007561e-08
2437 4.96894578816409e-08
2438 4.97036332092193e-08
2439 4.97732877136059e-08
2440 4.9697089110623e-08
2441 4.97133569865582e-08
2442 4.97529732967905e-08
2443 4.96904917213215e-08
2444 4.97146572797647e-08
2445 4.96746324074593e-08
2446 4.9658929412999e-08
2447 4.96417698059304e-08
2448 4.95775047681946e-08
2449 4.95717777937443e-08
2450 4.95545293688338e-08
2451 4.95856902205105e-08
2452 4.95724066240655e-08
2453 4.96123639948109e-08
2454 4.95677987544241e-08
2455 4.9579480077e-08
2456 4.96621233025962e-08
2457 4.95820202672803e-08
2458 4.95224128371774e-08
2459 4.95031855507477e-08
2460 4.95153216206745e-08
2461 4.94599419198494e-08
2462 4.94629723846174e-08
2463 4.96275021077963e-08
2464 4.957361809943e-08
2465 4.94356307001453e-08
2466 4.95071361683586e-08
2467 4.93806311396838e-08
2468 4.93648748545183e-08
2469 4.93757639219439e-08
2470 4.95607181960622e-08
2471 4.9440014748825e-08
2472 4.96316516773732e-08
2473 4.95934351363303e-08
2474 4.94880829648991e-08
2475 4.94946341689229e-08
2476 4.94915433080223e-08
2477 4.96422067897129e-08
2478 4.96197252175534e-08
2479 4.94490919322743e-08
2480 4.94327068167877e-08
2481 4.94303762366144e-08
2482 4.95979080028519e-08
2483 4.93859211303516e-08
2484 4.94235941062016e-08
2485 4.94023275621203e-08
2486 4.95416649926028e-08
2487 4.94113692184328e-08
2488 4.94078378210361e-08
2489 4.95369505415511e-08
2490 4.93295537751237e-08
2491 4.93699090498012e-08
2492 4.93305911675179e-08
2493 4.95219545371128e-08
2494 4.93183911487449e-08
2495 4.92980660737885e-08
2496 4.92819260955457e-08
2497 4.94479976964612e-08
2498 4.92871166102304e-08
2499 4.92803593488134e-08
2500 4.9253330303145e-08
2501 4.93997305284211e-08
2502 4.94671148487669e-08
2503 4.91869656116251e-08
2504 4.93792597922038e-08
2505 4.91815583814059e-08
2506 4.92553873243651e-08
2507 4.91843437089301e-08
2508 4.91759344356524e-08
2509 4.93351635100225e-08
2510 4.93200786877424e-08
2511 4.91393805646112e-08
2512 4.90998566249345e-08
2513 4.91334191110582e-08
2514 4.93978298266029e-08
2515 4.92332858925693e-08
2516 4.93422618319528e-08
2517 4.91226970211756e-08
2518 4.93190057682114e-08
2519 4.92579346200728e-08
2520 4.94069176681933e-08
2521 4.89871467834746e-08
2522 4.87802758186717e-08
2523 4.90461253832564e-08
2524 4.89733871233966e-08
2525 4.90084026694149e-08
2526 4.91156413318095e-08
2527 4.91679230663067e-08
2528 4.96730372390175e-08
2529 4.95049619075871e-08
2530 4.95029404135039e-08
2531 4.95142558065709e-08
2532 4.9082419906199e-08
2533 4.88788849395405e-08
2534 4.89735434427985e-08
2535 4.88572950985144e-08
2536 4.89666511782616e-08
2537 4.91732023988334e-08
2538 4.96413221640069e-08
2539 4.90280491760586e-08
2540 4.90122360474743e-08
2541 4.90722982249281e-08
2542 4.89133178405154e-08
2543 4.79134349973265e-08
2544 4.80604462893552e-08
2545 4.79002615350055e-08
2546 4.79322821433925e-08
2547 4.80227022592317e-08
2548 4.80805333324952e-08
2549 4.82104809407247e-08
2550 4.80002420033543e-08
2551 4.8024080712139e-08
2552 4.78095785183541e-08
2553 4.80376698419605e-08
2554 4.84675197753859e-08
2555 4.81326694057316e-08
2556 4.79111577078584e-08
2557 4.79766910643775e-08
2558 4.80051411955174e-08
2559 4.80083137688325e-08
2560 4.79636206307532e-08
2561 4.80261554969275e-08
2562 4.78679318405284e-08
2563 4.79303992051427e-08
2564 4.78301487305544e-08
2565 4.79653330387464e-08
2566 4.82627022790894e-08
2567 4.78654982316584e-08
2568 4.78958916971806e-08
2569 4.77873349780111e-08
2570 4.78643826795633e-08
2571 4.80866937380142e-08
2572 4.81752842063088e-08
2573 4.81550301856259e-08
2574 4.77587711600336e-08
2575 4.80944244429793e-08
2576 4.78245141266598e-08
2577 4.77970765189184e-08
2578 4.77832635681352e-08
2579 4.7659252544463e-08
2580 4.74562185104332e-08
2581 4.77742361226774e-08
2582 4.80172985817262e-08
2583 4.80523887347317e-08
2584 4.77311452584672e-08
2585 4.78374566625916e-08
2586 4.79828372590418e-08
2587 4.76866333087855e-08
2588 4.7650360102125e-08
2589 4.74596006938555e-08
2590 4.75441055414194e-08
2591 4.78919766067065e-08
2592 4.76017127937212e-08
2593 4.76418335892959e-08
2594 4.74951384887845e-08
2595 4.75605013150471e-08
2596 4.78108965751289e-08
2597 4.7871132835553e-08
2598 4.78605173270807e-08
2599 4.78223292077473e-08
2600 4.77178652147359e-08
2601 4.74338506251115e-08
2602 4.76548436267876e-08
2603 4.74137564765442e-08
2604 4.73741934570171e-08
2605 4.72920760330453e-08
2606 4.73058854311148e-08
2607 4.76365187296324e-08
2608 4.73278483070771e-08
2609 4.75413024503268e-08
2610 4.77222634742702e-08
2611 4.76652530778665e-08
2612 4.78156003680397e-08
2613 4.78740176390602e-08
2614 4.78821000626795e-08
2615 4.76070596278078e-08
2616 4.7778797807041e-08
2617 4.73905181763712e-08
2618 4.73809507184342e-08
2619 4.73085997043654e-08
2620 4.74825867513573e-08
2621 4.72377656990375e-08
2622 4.71616701247513e-08
2623 4.72651215943642e-08
2624 4.76004515803652e-08
2625 4.76632138202149e-08
2626 4.76446615493842e-08
2627 4.76501647028726e-08
2628 4.7705011496646e-08
2629 4.76275516803071e-08
2630 4.76149679684568e-08
2631 4.72156358455322e-08
2632 4.73915200416286e-08
2633 4.74310262177369e-08
2634 4.74171173436844e-08
2635 4.72965169251438e-08
2636 4.75412136324849e-08
2637 4.75365489194246e-08
2638 4.75136872069015e-08
2639 4.7530626545722e-08
2640 4.71584513661583e-08
2641 4.70003946873021e-08
2642 4.71202490359701e-08
2643 4.71420982250947e-08
2644 4.721880131342e-08
2645 4.73418708679674e-08
2646 4.72502073023406e-08
2647 4.72770267379019e-08
2648 4.72965453468532e-08
2649 4.73256314137416e-08
2650 4.75286299206346e-08
2651 4.73814871781997e-08
2652 4.73823007496321e-08
2653 4.73441588155765e-08
2654 4.7474117081947e-08
2655 4.78460258079849e-08
2656 4.747596449306e-08
2657 4.73855585880756e-08
2658 4.74034287378799e-08
2659 4.7584592266503e-08
2660 4.77678341326282e-08
2661 4.78311719120939e-08
2662 4.781988849345e-08
2663 4.74679353601459e-08
2664 4.75537724753394e-08
2665 4.7771884226222e-08
2666 4.73071573026118e-08
2667 4.74915218262595e-08
2668 4.77168100587733e-08
2669 4.77585189173624e-08
2670 4.74351722346e-08
2671 4.73236880793593e-08
2672 4.74847148268509e-08
2673 4.77176378410604e-08
2674 4.76977035646087e-08
2675 4.72900758552441e-08
2676 4.72425867314996e-08
2677 4.73500278985739e-08
2678 4.72145416097192e-08
2679 4.73674717227368e-08
2680 4.75996202453643e-08
2681 4.76333781307403e-08
2682 4.76338755106553e-08
2683 4.72899479575517e-08
2684 4.71673899937741e-08
2685 4.72537529105921e-08
2686 4.75395438570558e-08
2687 4.7515396062181e-08
2688 4.71658729850333e-08
2689 4.71817216407544e-08
2690 4.72117989147591e-08
2691 4.7241222489447e-08
2692 4.75519321696538e-08
2693 4.74870169853148e-08
2694 4.7151928583844e-08
2695 4.71389469680616e-08
2696 4.7063121400015e-08
2697 4.70484948777994e-08
2698 4.70243186612151e-08
2699 4.70135717023368e-08
2700 4.71387124889588e-08
2701 4.70222651927088e-08
2702 4.70098342475467e-08
2703 4.69851642037611e-08
2704 4.69635139666025e-08
2705 4.69557406290733e-08
2706 4.69456438167981e-08
2707 4.69840024663881e-08
2708 4.69745771169983e-08
2709 4.68967940037146e-08
2710 4.68965595246118e-08
2711 4.7059216967682e-08
2712 4.69324312746267e-08
2713 4.68752716642484e-08
2714 4.69076297804349e-08
2715 4.68414995680178e-08
2716 4.68257219665702e-08
2717 4.67850505003753e-08
2718 4.68622367577609e-08
2719 4.67935059589308e-08
2720 4.72047005928289e-08
2721 4.72980232757436e-08
2722 4.72877736967803e-08
2723 4.72458694389388e-08
2724 4.72607410983983e-08
2725 4.72333212542253e-08
2726 4.72311398880265e-08
2727 4.7011113224471e-08
2728 4.71499284060428e-08
2729 4.71957015690805e-08
2730 4.67193927988774e-08
2731 4.70264609475635e-08
2732 4.71199399498801e-08
2733 4.71410217528501e-08
2734 4.71346304209419e-08
2735 4.70850842759774e-08
2736 4.71092320708522e-08
2737 4.67321079611338e-08
2738 4.69238941036565e-08
2739 4.69467096309018e-08
2740 4.69243772727168e-08
2741 4.66579805902256e-08
2742 4.68995011715378e-08
2743 4.66381990804621e-08
2744 4.65568525953586e-08
2745 4.65048799469514e-08
2746 4.64835601121649e-08
2747 4.65938008176181e-08
2748 4.64709799530283e-08
2749 4.65314151654184e-08
2750 4.64102711816849e-08
2751 4.6437612866157e-08
2752 4.64830876012456e-08
2753 4.63433238451216e-08
2754 4.63642741976855e-08
2755 4.6315999924218e-08
2756 4.63130227501551e-08
2757 4.63049651955316e-08
2758 4.63111113901959e-08
2759 4.62909888199192e-08
2760 4.62900260345123e-08
2761 4.62996041505903e-08
2762 4.63388651894547e-08
2763 4.63934881622663e-08
2764 4.62622686825398e-08
2765 4.62453364491466e-08
2766 4.62360141284535e-08
2767 4.61948772567666e-08
2768 4.62080045338098e-08
2769 4.6166778844281e-08
2770 4.61707365673192e-08
2771 4.61585294431188e-08
2772 4.61670488505206e-08
2773 4.61556091124748e-08
2774 4.62255407285284e-08
2775 4.61539997331784e-08
2776 4.61841374033156e-08
2777 4.61376288285464e-08
2778 4.61537581486482e-08
2779 4.61339269008931e-08
2780 4.61317313238396e-08
2781 4.62756766239636e-08
2782 4.61429578990646e-08
2783 4.61084965763803e-08
2784 4.61090507997142e-08
2785 4.60872868757178e-08
2786 4.61800127027345e-08
2787 4.6166668710157e-08
2788 4.60837092930433e-08
2789 4.64691574109111e-08
2790 4.6549558874176e-08
2791 4.64968117341868e-08
2792 4.64934934996108e-08
2793 4.65050860043448e-08
2794 4.64825653523349e-08
2795 4.64680596223843e-08
2796 4.64477061257185e-08
2797 4.64659848375959e-08
2798 4.64213307793671e-08
2799 4.64187159820995e-08
2800 4.63544616025047e-08
2801 4.63533602612642e-08
2802 4.60755948950009e-08
2803 4.61905180770827e-08
2804 4.63396823136009e-08
2805 4.63397427097334e-08
2806 4.63207925349707e-08
2807 4.59745557179758e-08
2808 4.59424285281784e-08
2809 4.5934104520029e-08
2810 4.59181350720428e-08
2811 4.58830342608962e-08
2812 4.58689299875914e-08
2813 4.58630786681624e-08
2814 4.58434321615186e-08
2815 4.59017215348467e-08
2816 4.58427571459197e-08
2817 4.58588296226026e-08
2818 4.5851425767296e-08
2819 4.58075923859269e-08
2820 4.58009026260697e-08
2821 4.573053047352e-08
2822 4.5731194830978e-08
2823 4.57000126630192e-08
2824 4.57587070457066e-08
2825 4.57392523856015e-08
2826 4.5907182055771e-08
2827 4.57055300273623e-08
2828 4.57924080876637e-08
2829 4.56819364558214e-08
2830 4.57586430968604e-08
2831 4.5746048726869e-08
2832 4.57519107044391e-08
2833 4.56764723821834e-08
2834 4.57256561503527e-08
2835 4.56577389229551e-08
2836 4.56336302079308e-08
2837 4.57121309693775e-08
2838 4.56933051395936e-08
2839 4.56330617737422e-08
2840 4.55922197772907e-08
2841 4.56078836919005e-08
2842 4.55504149954322e-08
2843 4.56588047370587e-08
2844 4.5576499019262e-08
2845 4.55676421040607e-08
2846 4.55384743247578e-08
2847 4.55129196552662e-08
2848 4.55203164051454e-08
2849 4.54678641403916e-08
2850 4.54910384917184e-08
2851 4.54733211086022e-08
2852 4.54514967884734e-08
2853 4.54269510896665e-08
2854 4.54565984853161e-08
2855 4.54106405811672e-08
2856 4.53857964544113e-08
2857 4.53929693833288e-08
2858 4.53912427644809e-08
2859 4.53891786378335e-08
2860 4.53653896670403e-08
2861 4.53397532851341e-08
2862 4.53218085283424e-08
2863 4.53189521465447e-08
2864 4.53202062544733e-08
2865 4.53456614479819e-08
2866 4.5285180050314e-08
2867 4.52952662044481e-08
2868 4.52956570029528e-08
2869 4.52549535623348e-08
2870 4.5252683378294e-08
2871 4.52190391797558e-08
2872 4.52722375143821e-08
2873 4.52445405585422e-08
2874 4.52149642171662e-08
2875 4.52799788774882e-08
2876 4.52043771304034e-08
2877 4.51990516125989e-08
2878 4.51621815500403e-08
2879 4.51541488644125e-08
2880 4.51378490140542e-08
2881 4.51282460289804e-08
2882 4.51239294818606e-08
2883 4.50832793319478e-08
2884 4.50893331560565e-08
2885 4.50654198402844e-08
2886 4.50723760536675e-08
2887 4.50975683463639e-08
2888 4.51162627257418e-08
2889 4.53065212013826e-08
2890 4.54142181638417e-08
2891 4.54422277584854e-08
2892 4.53933068911283e-08
2893 4.54042812236821e-08
2894 4.53916761955497e-08
2895 4.5471299614519e-08
2896 4.51096582310129e-08
2897 4.50811690200226e-08
2898 4.50334418644616e-08
2899 4.50264856510785e-08
2900 4.49920278811078e-08
2901 4.49942021418792e-08
2902 4.49672690194802e-08
2903 4.49710277905524e-08
2904 4.49685586545456e-08
2905 4.49746906383552e-08
2906 4.49182557815675e-08
2907 4.49364883081671e-08
2908 4.49272725688843e-08
2909 4.49556125659001e-08
2910 4.49259083268316e-08
2911 4.4952759736816e-08
2912 4.51096973108633e-08
2913 4.53194601846008e-08
2914 4.51707116155831e-08
2915 4.51500490328272e-08
2916 4.51447448313047e-08
2917 4.51531967371466e-08
2918 4.51472708107303e-08
2919 4.51881447816049e-08
2920 4.5179561425357e-08
2921 4.52441675236059e-08
2922 4.51511219523582e-08
2923 4.515645457559e-08
2924 4.51484432062443e-08
2925 4.51585648875152e-08
2926 4.4949143074291e-08
2927 4.51127561973408e-08
2928 4.51120634181734e-08
2929 4.51187140981801e-08
2930 4.50941790575143e-08
2931 4.48758576965247e-08
2932 4.5057582553909e-08
2933 4.50737971391391e-08
2934 4.51210695473492e-08
2935 4.50288339948202e-08
2936 4.50161863341236e-08
2937 4.50262085394115e-08
2938 4.49801760282753e-08
2939 4.49861552453967e-08
2940 4.50390658102151e-08
2941 4.49762040943824e-08
2942 4.46570815881842e-08
2943 4.47392167757243e-08
2944 4.46831158740224e-08
2945 4.46800818565407e-08
2946 4.45298056206411e-08
2947 4.47474270970361e-08
2948 4.47962023031323e-08
2949 4.49246826406124e-08
2950 4.47812382731172e-08
2951 4.48035137878833e-08
2952 4.48060895053004e-08
2953 4.47719195051377e-08
2954 4.47737455999686e-08
2955 4.48560015797739e-08
2956 4.4757470618606e-08
2957 4.45517471803214e-08
2958 4.4542460386765e-08
2959 4.46799894859851e-08
2960 4.46644179419309e-08
2961 4.45190906361859e-08
2962 4.44982148906092e-08
2963 4.48253629770079e-08
2964 4.49460735296725e-08
2965 4.4778531105294e-08
2966 4.47911254752853e-08
2967 4.47721113516764e-08
2968 4.4779138619333e-08
2969 4.47511965262493e-08
2970 4.47599894926043e-08
2971 4.47665051694912e-08
2972 4.4747405780754e-08
2973 4.48481287662617e-08
2974 4.4757182848798e-08
2975 4.4689436151657e-08
2976 4.46549464072632e-08
2977 4.46486509986244e-08
2978 4.46273986653978e-08
2979 4.46269581289016e-08
2980 4.46872157056077e-08
2981 4.45559003026119e-08
2982 4.45976802154746e-08
2983 4.45984653651976e-08
2984 4.45775825141936e-08
2985 4.45656453962329e-08
2986 4.46837766787667e-08
2987 4.45309922270098e-08
2988 4.45371526325289e-08
2989 4.45121308700891e-08
2990 4.4628237105826e-08
2991 4.45338308452392e-08
2992 4.45352164035739e-08
2993 4.45206715937729e-08
2994 4.46429488931699e-08
2995 4.45115375669047e-08
2996 4.45175345475946e-08
2997 4.45144294758393e-08
2998 4.44918661912652e-08
2999 4.45972716534015e-08
3000 4.44892123141472e-08
3001 4.44565451118706e-08
3002 4.44763408324889e-08
3003 4.45598757892185e-08
3004 4.44561223389428e-08
3005 4.44274554922686e-08
3006 4.44345396033441e-08
3007 4.43750742817883e-08
3008 4.44805188237751e-08
3009 4.43814407447007e-08
3010 4.43959997653565e-08
3011 4.43998011689928e-08
3012 4.42100969166859e-08
3013 4.42128502697869e-08
3014 4.41921521598942e-08
3015 4.42004761680437e-08
3016 4.40433787218808e-08
3017 4.41342820067803e-08
3018 4.40886438468624e-08
3019 4.40700489434676e-08
3020 4.40582006433488e-08
3021 4.40446115135273e-08
3022 4.40004761514956e-08
3023 4.41735608092131e-08
3024 4.40324363637501e-08
3025 4.40403340462581e-08
3026 4.39301679477921e-08
3027 4.40342837748631e-08
3028 4.39927276829621e-08
3029 4.39228280413317e-08
3030 4.39718448319582e-08
3031 4.39548699660008e-08
3032 4.4218424477549e-08
3033 4.42281660184562e-08
3034 4.41019345487348e-08
3035 4.41068479517526e-08
3036 4.41102621095979e-08
3037 4.40607337282017e-08
3038 4.41098428893838e-08
3039 4.41063612299786e-08
3040 4.41131717821008e-08
3041 4.40967262704817e-08
3042 4.40712355498363e-08
3043 4.40881358088063e-08
3044 4.41621104130263e-08
3045 4.40576073401644e-08
3046 4.40586589434133e-08
3047 4.40376197730075e-08
3048 4.40551310987303e-08
3049 4.413928422764e-08
3050 4.4060310955274e-08
3051 4.40315695016125e-08
3052 4.40319496419761e-08
3053 4.3983316544427e-08
3054 4.41033733977747e-08
3055 4.40025083037199e-08
3056 4.39978826705101e-08
3057 4.39757137371544e-08
3058 4.39804388463472e-08
3059 4.40324576800322e-08
3060 4.3956958961644e-08
3061 4.39623129011579e-08
3062 4.39634142423984e-08
3063 4.39751346448247e-08
3064 4.40045120342347e-08
3065 4.39320650968966e-08
3066 4.39079457237312e-08
3067 4.391393204628e-08
3068 4.37895231186758e-08
3069 4.36051195151776e-08
3070 4.38045759665329e-08
3071 4.37239435768788e-08
3072 4.37197869018746e-08
3073 4.38127365498531e-08
3074 4.36742695342218e-08
3075 4.36671356851548e-08
3076 4.37869474012587e-08
3077 4.36790799085429e-08
3078 4.36664890912652e-08
3079 4.37506741945981e-08
3080 4.3647290226545e-08
3081 4.36512017643054e-08
3082 4.37444072076687e-08
3083 4.36357723287983e-08
3084 4.36346603294169e-08
3085 4.36919478374875e-08
3086 4.36249187885096e-08
3087 4.35953282362789e-08
3088 4.36849560969677e-08
3089 4.35992291158982e-08
3090 4.35428368916746e-08
3091 4.35704663459546e-08
3092 4.36365041878162e-08
3093 4.35950440191846e-08
3094 4.35336424686739e-08
3095 4.36374101298043e-08
3096 4.35394085229746e-08
3097 4.35331557468999e-08
3098 4.36407852077991e-08
3099 4.3489951195852e-08
3100 4.3511132474805e-08
3101 4.35111573438007e-08
3102 4.35678764176828e-08
3103 4.34807674309923e-08
3104 4.34428386597574e-08
3105 4.34738112176092e-08
3106 4.35373159746177e-08
3107 4.34722196018811e-08
3108 4.34402522841992e-08
3109 4.3525965054414e-08
3110 4.34621973965932e-08
3111 4.354100724413e-08
3112 4.34081854905344e-08
3113 4.3482440759135e-08
3114 4.33057714133156e-08
3115 4.33701394797481e-08
3116 4.33609130823243e-08
3117 4.3351171541417e-08
3118 4.33069295979749e-08
3119 4.34515641245525e-08
3120 4.33759979046044e-08
3121 4.3318287623606e-08
3122 4.32999200938866e-08
3123 4.34170104313125e-08
3124 4.3325609766498e-08
3125 4.32883595635758e-08
3126 4.32651177106891e-08
3127 4.32762838897816e-08
3128 4.32719993170849e-08
3129 4.32362092794847e-08
3130 4.32473186151583e-08
3131 4.33401972088632e-08
3132 4.32216893386794e-08
3133 4.32277893480659e-08
3134 4.33191438276026e-08
3135 4.32244782189173e-08
3136 4.32327382782205e-08
3137 4.30833360098859e-08
3138 4.30473576784607e-08
3139 4.30696545095088e-08
3140 4.29880344654521e-08
3141 4.30222968361704e-08
3142 4.30917559413047e-08
3143 4.3031636920432e-08
3144 4.30616644564452e-08
3145 4.30897806324992e-08
3146 4.31664162192646e-08
3147 4.32759499346957e-08
3148 4.31722604332663e-08
3149 4.31497824138205e-08
3150 4.31249702614878e-08
3151 4.31270272827078e-08
3152 4.31685620583266e-08
3153 4.32460645072297e-08
3154 4.31854410010146e-08
3155 4.31621707264185e-08
3156 4.30180229216148e-08
3157 4.3126316739972e-08
3158 4.30961968334032e-08
3159 4.30874536050396e-08
3160 4.3145817585355e-08
3161 4.30886011315579e-08
3162 4.30666311501682e-08
3163 4.30653415151028e-08
3164 4.30172555354602e-08
3165 4.30594973011011e-08
3166 4.29902158316509e-08
3167 4.30253770389299e-08
3168 4.29809645652313e-08
3169 4.29608952856597e-08
3170 4.30331930090233e-08
3171 4.29544861901832e-08
3172 4.30112159222062e-08
3173 4.29989839290101e-08
3174 4.29355928588393e-08
3175 4.29165432080936e-08
3176 4.29149338287971e-08
3177 4.29502122756276e-08
3178 4.29410320634815e-08
3179 4.29666826562425e-08
3180 4.29696989101558e-08
3181 4.29551541003548e-08
3182 4.29241282517978e-08
3183 4.29990869577068e-08
3184 4.30157953701382e-08
3185 4.30361453140904e-08
3186 4.29630624410038e-08
3187 4.30096065429098e-08
3188 4.30774029780423e-08
3189 4.29910755883611e-08
3190 4.30236433146547e-08
3191 4.29948521230017e-08
3192 4.30831903486251e-08
3193 4.30679634177977e-08
3194 4.30349622604353e-08
3195 4.29671516144481e-08
3196 4.30337152579341e-08
3197 4.30775735082989e-08
3198 4.29543476343497e-08
3199 4.2984009240854e-08
3200 4.29953068703526e-08
3201 4.3009205086264e-08
3202 4.29813233893128e-08
3203 4.29923403544308e-08
3204 4.30059934330984e-08
3205 4.29790496525584e-08
3206 4.29702424753486e-08
3207 4.29594706474745e-08
3208 4.29610480523479e-08
3209 4.29166142623671e-08
3210 4.28701945054399e-08
3211 4.28995896584183e-08
3212 4.29032454007938e-08
3213 4.28891340220616e-08
3214 4.28793214268808e-08
3215 4.29203943497214e-08
3216 4.29094271225949e-08
3217 4.29161239878795e-08
3218 4.28868673907346e-08
3219 4.28999129553631e-08
3220 4.28415560804751e-08
3221 4.28654516326787e-08
3222 4.28695976495419e-08
3223 4.28989750389519e-08
3224 4.28983959466223e-08
3225 4.2878689043846e-08
3226 4.28623927462013e-08
3227 4.28582644929065e-08
3228 4.28080291214883e-08
3229 4.28074713454407e-08
3230 4.28226840654133e-08
3231 4.25049222485541e-08
3232 4.28689581610797e-08
3233 4.24892796502263e-08
3234 4.25085708855022e-08
3235 4.24779464935909e-08
3236 4.2451034687474e-08
3237 4.24814672328466e-08
3238 4.25024140326968e-08
3239 4.2523051746457e-08
3240 4.25200887832489e-08
3241 4.24834709633615e-08
3242 4.24956780875618e-08
3243 4.24694057699071e-08
3244 4.2499546992758e-08
3245 4.24799679876742e-08
3246 4.24723225478374e-08
3247 4.24553654454485e-08
3248 4.24598383119701e-08
3249 4.24824548872493e-08
3250 4.24689687861246e-08
3251 4.24224069206502e-08
3252 4.24337933679908e-08
3253 4.24427781808845e-08
3254 4.241568518637e-08
3255 4.24191277659247e-08
3256 4.24006714183633e-08
3257 4.2360447594092e-08
3258 4.23367687574228e-08
3259 4.23802291038555e-08
3260 4.2590997395564e-08
3261 4.25444035556666e-08
3262 4.24597708104102e-08
3263 4.24744222016216e-08
3264 4.24544843724561e-08
3265 4.26131627762061e-08
3266 4.26500932348972e-08
3267 4.23905071045283e-08
3268 4.22803694277718e-08
3269 4.22812398426231e-08
3270 4.22692920665213e-08
3271 4.22983248427045e-08
3272 4.22122425902671e-08
3273 4.22739567795816e-08
3274 4.22402557376245e-08
3275 4.22302584013323e-08
3276 4.22305959091318e-08
3277 4.22566515112521e-08
3278 4.22229824437181e-08
3279 4.2186648840925e-08
3280 4.21480841339417e-08
3281 4.22609254258077e-08
3282 4.22113615172748e-08
3283 4.21840127273754e-08
3284 4.21644408277189e-08
3285 4.21576231701692e-08
3286 4.21853165732955e-08
3287 4.2221369511708e-08
3288 4.21386801008339e-08
3289 4.22088568541312e-08
3290 4.21154027208104e-08
3291 4.21829255969897e-08
3292 4.21693364671683e-08
3293 4.21265298200524e-08
3294 4.21243235848578e-08
3295 4.21392236660267e-08
3296 4.26747810422512e-08
3297 4.26506687745132e-08
3298 4.26193480507209e-08
3299 4.26383301999067e-08
3300 4.26400710296093e-08
3301 4.27133990399398e-08
3302 4.26261905772662e-08
3303 4.27577688810743e-08
3304 4.26609823023227e-08
3305 4.27758877208362e-08
3306 4.27306368067093e-08
3307 4.28026822874017e-08
3308 4.27110435907707e-08
3309 4.27288675552973e-08
3310 4.2786950871232e-08
3311 4.27070112607453e-08
3312 4.27097006650001e-08
3313 4.27125463886568e-08
3314 4.27178719064614e-08
3315 4.2686917112178e-08
3316 4.26984989587709e-08
3317 4.27235846700569e-08
3318 4.26863735469851e-08
3319 4.26866044733742e-08
3320 4.26863984159809e-08
3321 4.26848671963853e-08
3322 4.26623572025164e-08
3323 4.28047997047543e-08
3324 4.28052793211009e-08
3325 4.27871533759117e-08
3326 4.27815933790043e-08
3327 4.27669384350793e-08
3328 4.27759196952593e-08
3329 4.27439630357185e-08
3330 4.27337631947466e-08
3331 4.2764508378923e-08
3332 4.27599253782773e-08
3333 4.27782183010095e-08
3334 4.3357719192727e-08
3335 4.32407603057072e-08
3336 4.32280629070192e-08
3337 4.32147082563006e-08
3338 4.31846700621463e-08
3339 4.31703099934566e-08
3340 4.31423430313771e-08
3341 4.31452988891579e-08
3342 4.31379234555607e-08
3343 4.30620445968088e-08
3344 4.31004067991125e-08
3345 4.30817550522988e-08
3346 4.3102879487833e-08
3347 4.3066389565638e-08
3348 4.30479758506408e-08
3349 4.29998401330067e-08
3350 4.29822968328608e-08
3351 4.29533955070838e-08
3352 4.2994052762424e-08
3353 4.29134203727699e-08
3354 4.29225686104928e-08
3355 4.29111928212933e-08
3356 4.27907984601461e-08
3357 4.2560039048567e-08
3358 4.2799836563745e-08
3359 4.27441371186887e-08
3360 4.26581969747986e-08
3361 4.2622499307754e-08
3362 4.28318074341405e-08
3363 4.28321023093758e-08
3364 4.28432755938957e-08
3365 4.28694129084306e-08
3366 4.27915978207238e-08
3367 4.28577244804274e-08
3368 4.27253254997595e-08
3369 4.26981969781082e-08
3370 4.27842081762719e-08
3371 4.26854036561508e-08
3372 4.27257234036915e-08
3373 4.27087698540163e-08
3374 4.27206430231308e-08
3375 4.26962039057344e-08
3376 4.24778185958985e-08
3377 4.20620729357779e-08
3378 4.17573851052566e-08
3379 4.18702796878279e-08
3380 4.19639327731147e-08
3381 4.19052952338461e-08
3382 4.17851886425069e-08
3383 4.17569729904699e-08
3384 4.17558752019431e-08
3385 4.17251690976173e-08
3386 4.17931396157201e-08
3387 4.18152410475159e-08
3388 4.16866718921938e-08
3389 4.18009484803861e-08
3390 4.16672385483707e-08
3391 4.16770653544063e-08
3392 4.20136707646179e-08
3393 4.20325036998292e-08
3394 4.20341024209847e-08
3395 4.20621475427652e-08
3396 4.20415240398597e-08
3397 4.19751593483397e-08
3398 4.20810728485321e-08
3399 4.20200052531072e-08
3400 4.19430641329654e-08
3401 4.1880365841962e-08
3402 4.19957082442579e-08
3403 4.17308250177939e-08
3404 4.16560546057099e-08
3405 4.16315231177578e-08
3406 4.15647143370279e-08
3407 4.16246415113619e-08
3408 4.14955678706974e-08
3409 4.14613516852569e-08
3410 4.14975218632208e-08
3411 4.16039824813197e-08
3412 4.14620906497021e-08
3413 4.14997742836931e-08
3414 4.15240570816877e-08
3415 4.14984100416405e-08
3416 4.14864302911155e-08
3417 4.14738963172567e-08
3418 4.13981844360478e-08
3419 4.14323473307832e-08
3420 4.14320808772572e-08
3421 4.14797511893994e-08
3422 4.14687697514182e-08
3423 4.13948768596129e-08
3424 4.13375680352601e-08
3425 4.13055900594372e-08
3426 4.13954879263656e-08
3427 4.13306047164497e-08
3428 4.1332270939165e-08
3429 4.13194456427846e-08
3430 4.12966123519709e-08
3431 4.13580245606227e-08
3432 4.1260630467832e-08
3433 4.12516669712204e-08
3434 4.11607636863209e-08
3435 4.11863148030989e-08
3436 4.11473628503245e-08
3437 4.12619058920427e-08
3438 4.05185325291768e-08
3439 3.99661423955422e-08
3440 3.99254176386421e-08
3441 3.98238348964242e-08
3442 3.95387402818415e-08
3443 3.94980688156465e-08
3444 3.95750916482029e-08
3445 3.93842114476683e-08
3446 3.94278529824987e-08
3447 3.93443997381837e-08
3448 3.92935746162948e-08
3449 3.92685315375729e-08
3450 3.93089898409471e-08
3451 3.93488797101327e-08
3452 3.94325319064137e-08
3453 3.93013870336745e-08
3454 3.94236074896526e-08
3455 3.93155765721076e-08
3456 3.92681585026367e-08
3457 3.91796923793208e-08
3458 3.90469843125629e-08
3459 3.89841048331618e-08
3460 3.89389782640137e-08
3461 3.89036109993413e-08
3462 3.92057444287275e-08
3463 3.92746848376646e-08
3464 3.91236767427472e-08
3465 3.91577543723542e-08
3466 3.8842745908596e-08
3467 3.86975287369751e-08
3468 3.87238863197581e-08
3469 3.8591334572402e-08
3470 3.8447279138154e-08
3471 3.84679275100552e-08
3472 3.84724607727094e-08
3473 3.86772605054375e-08
3474 3.81089826362313e-08
3475 3.82497660211811e-08
3476 3.94777401879765e-08
3477 4.0173866011628e-08
3478 4.03058884046459e-08
3479 4.05944646786338e-08
3480 4.04207405324541e-08
3481 4.04329689729366e-08
3482 4.04451050428634e-08
3483 4.05481763721127e-08
3484 4.0687030633535e-08
3485 4.07748288466792e-08
3486 4.09958786917741e-08
3487 4.10761380464919e-08
3488 4.1138044082345e-08
3489 4.11791205578993e-08
3490 4.13443963509508e-08
3491 4.11776888142867e-08
3492 4.11122051957591e-08
3493 4.12813854211436e-08
3494 4.10411615803241e-08
3495 4.10092368952064e-08
3496 4.09880414053987e-08
3497 4.10241831616531e-08
3498 4.11968521518702e-08
3499 4.0989199590058e-08
3500 4.09887910279849e-08
3501 4.10033109687902e-08
3502 4.10136031803177e-08
3503 4.09434797177255e-08
3504 4.08715621347255e-08
3505 4.07284765913118e-08
3506 4.11485387985522e-08
3507 4.11520346688121e-08
3508 4.1058605404487e-08
3509 4.08133757900941e-08
3510 4.0911210419381e-08
3511 4.06734237401452e-08
3512 4.06142497411111e-08
3513 4.05735001152152e-08
3514 4.06668760888351e-08
3515 4.04503097684028e-08
3516 4.04714306512233e-08
3517 4.03216340316703e-08
3518 4.0421660685297e-08
3519 4.02249433761881e-08
3520 4.01140418659907e-08
3521 4.01227282509353e-08
3522 4.00781487996937e-08
3523 4.01005983974301e-08
3524 4.00334343453324e-08
3525 4.0029036085798e-08
3526 4.00155499846733e-08
3527 3.99793300687179e-08
3528 3.99147417340373e-08
3529 3.99868547162896e-08
3530 3.99459842981287e-08
3531 3.97791914963364e-08
3532 3.96843446992534e-08
3533 3.96888282239161e-08
3534 3.96209074438048e-08
3535 3.96191417451064e-08
3536 3.95666788222115e-08
3537 3.95690307186669e-08
3538 3.93969301626385e-08
3539 3.94061885344854e-08
3540 3.93162231659971e-08
3541 3.94003265569154e-08
3542 3.92833321427588e-08
3543 3.92522530034967e-08
3544 3.8939152346984e-08
3545 3.90739707256671e-08
3546 3.88533152317905e-08
3547 3.88629999292789e-08
3548 3.88019465447087e-08
3549 3.87860197292866e-08
3550 3.87404099910782e-08
3551 3.87347967034657e-08
3552 3.86637992733085e-08
3553 3.86899401405572e-08
3554 3.86192056112122e-08
3555 3.86163954146923e-08
3556 3.85668066371636e-08
3557 3.85787686241201e-08
3558 3.86740026669941e-08
3559 3.85188343443588e-08
3560 3.84685527876627e-08
3561 3.84795235675028e-08
3562 3.84219660531926e-08
3563 3.84109348772199e-08
3564 3.83670517578594e-08
3565 3.83671689974108e-08
3566 3.83562195338527e-08
3567 3.84615397308607e-08
3568 3.85640674949173e-08
3569 3.84166582989565e-08
3570 3.81703770813147e-08
3571 3.82509988128277e-08
3572 3.83633249612103e-08
3573 3.79925815252591e-08
3574 3.80622999784919e-08
3575 3.78907216713742e-08
3576 3.77973243814722e-08
3577 3.80139439926097e-08
3578 3.79865667810009e-08
3579 3.81341394017909e-08
3580 3.77943329965547e-08
3581 3.79853055676449e-08
3582 3.80881246542231e-08
3583 3.79077711443188e-08
3584 3.78106399523404e-08
3585 3.77387578964772e-08
3586 3.76355799858175e-08
3587 3.74949884474063e-08
3588 3.7479782832861e-08
3589 3.74367061795056e-08
3590 3.72451829377951e-08
3591 3.71974770985162e-08
3592 3.72380597468691e-08
3593 3.71337591786869e-08
3594 3.70613015832078e-08
3595 3.70567150298484e-08
3596 3.69915795772613e-08
3597 3.69598254224002e-08
3598 3.70360631052336e-08
3599 3.69747539252785e-08
3600 3.68399213357407e-08
3601 3.67004595602793e-08
3602 3.6756098609203e-08
3603 3.67323487182603e-08
3604 3.6753743160034e-08
3605 3.65392338608217e-08
3606 3.64984416023617e-08
3607 3.64781946871062e-08
3608 3.64524410656486e-08
3609 3.64306735889386e-08
3610 3.62279983789904e-08
3611 3.62195642367169e-08
3612 3.61571110829573e-08
3613 3.62701797484988e-08
3614 3.61937573245541e-08
3615 3.62045469159966e-08
3616 3.62169565448767e-08
3617 3.61676129045918e-08
3618 3.60059253523559e-08
3619 3.59219569645575e-08
3620 3.59981768838225e-08
3621 3.59389993320747e-08
3622 3.58806602207551e-08
3623 3.57547129681279e-08
3624 3.58664493660399e-08
3625 3.58072433925827e-08
3626 3.58580649617579e-08
3627 3.57088225655389e-08
3628 3.57594664990302e-08
3629 3.57299292375046e-08
3630 3.55996192524799e-08
3631 3.55794789186348e-08
3632 3.54038220962138e-08
3633 3.55795997108999e-08
3634 3.54784042144729e-08
3635 3.54732456742113e-08
3636 3.53364981720006e-08
3637 3.54086182596802e-08
3638 3.5361502170872e-08
3639 3.54702969218579e-08
3640 3.52004398962436e-08
3641 3.52895561661626e-08
3642 3.51225288852675e-08
3643 3.52132616399103e-08
3644 3.50603777121705e-08
3645 3.50447884045479e-08
3646 3.5268005404987e-08
3647 3.51501903139706e-08
3648 3.48599940025451e-08
3649 3.49297124557779e-08
3650 3.50212765454216e-08
3651 3.48705739838806e-08
3652 3.48270248196059e-08
3653 3.50150131112059e-08
3654 3.48587363419028e-08
3655 3.48659980886623e-08
3656 3.48044686404592e-08
3657 3.47724409266448e-08
3658 3.46991129163143e-08
3659 3.46715580690216e-08
3660 3.46585871113803e-08
3661 3.46607613721517e-08
3662 3.46911903648106e-08
3663 3.45431878656655e-08
3664 3.45503003984504e-08
3665 3.44789441442117e-08
3666 3.45824346936752e-08
3667 3.46335689016541e-08
3668 3.43779404943234e-08
3669 3.46171198373213e-08
3670 3.43312329675882e-08
3671 3.432141326698e-08
3672 3.43286075121796e-08
3673 3.43597399421469e-08
3674 3.4428172313028e-08
3675 3.4419699090904e-08
3676 3.41810455495306e-08
3677 3.42580506185186e-08
3678 3.42935813080203e-08
3679 3.4386882674653e-08
3680 3.40569812351532e-08
3681 3.40451258296071e-08
3682 3.39252181902339e-08
3683 3.42116592833008e-08
3684 3.39618857481128e-08
3685 3.39126451365246e-08
3686 3.37888899082373e-08
3687 3.39704051555145e-08
3688 3.37266641281531e-08
3689 3.39565069396031e-08
3690 3.36458860772382e-08
3691 3.35181766786263e-08
3692 3.34621965691895e-08
3693 3.34573719840137e-08
3694 3.3719629755069e-08
3695 3.36271703815783e-08
3696 3.35562440056947e-08
3697 3.36848273718715e-08
3698 3.37757022350615e-08
3699 3.35848824306595e-08
3700 3.34845324800881e-08
3701 3.34607221930128e-08
3702 3.34573364568769e-08
3703 3.35216228108948e-08
3704 3.35225394110239e-08
3705 3.34417826763911e-08
3706 3.31966134581307e-08
3707 3.31117284702032e-08
3708 3.32690675008962e-08
3709 3.31351373006328e-08
3710 3.31946452547527e-08
3711 3.31664296027157e-08
3712 3.31445022538901e-08
3713 3.31158247490748e-08
3714 3.31146843279839e-08
3715 3.30510161461461e-08
3716 3.30160965233972e-08
3717 3.29426121936649e-08
3718 3.28850191522179e-08
3719 3.30500320444571e-08
3720 3.31337872694348e-08
3721 3.30934497583257e-08
3722 3.31003882081404e-08
3723 3.28457865350629e-08
3724 3.26436300213118e-08
3725 3.25945244128434e-08
3726 3.2457673881936e-08
3727 3.24008162522205e-08
3728 3.23680460212472e-08
3729 3.23321600603776e-08
3730 3.22952473652549e-08
3731 3.22630029359061e-08
3732 3.20248076945973e-08
3733 3.21462998442712e-08
3734 3.22069269031999e-08
3735 3.2085136325577e-08
3736 3.18842339197545e-08
3737 3.17861470477965e-08
3738 3.17698649610065e-08
3739 3.17267812022237e-08
3740 3.17027470941866e-08
3741 3.16609138906188e-08
3742 3.17122399451364e-08
3743 3.15950643425822e-08
3744 3.17017523343566e-08
3745 3.15359187652575e-08
3746 3.14951265067975e-08
3747 3.14766346320994e-08
3748 3.14470760542918e-08
3749 3.14083656860475e-08
3750 3.13891845848957e-08
3751 3.14793950906278e-08
3752 3.15202974832118e-08
3753 3.16309858305885e-08
3754 3.13818837582858e-08
3755 3.12398711344031e-08
3756 3.12064081242625e-08
3757 3.11845305134284e-08
3758 3.11712362588423e-08
3759 3.11310586198488e-08
3760 3.11195869073799e-08
3761 3.10993328866971e-08
3762 3.09971426304401e-08
3763 3.1022107549461e-08
3764 3.10201286879419e-08
3765 3.09671150944268e-08
3766 3.09284295951784e-08
3767 3.09021856992331e-08
3768 3.0944406148592e-08
3769 3.08107672708502e-08
3770 3.08187573239138e-08
3771 3.078426757952e-08
3772 3.0746107881896e-08
3773 3.08078860200567e-08
3774 3.08119574299326e-08
3775 3.07950145383984e-08
3776 3.07973557767127e-08
3777 3.0775296977481e-08
3778 3.07915257735658e-08
3779 3.07758192263918e-08
3780 3.07568122082102e-08
3781 3.07834433499465e-08
3782 3.07923251341435e-08
3783 3.08067384935384e-08
3784 3.08006846694298e-08
3785 3.07349168338078e-08
3786 3.06957694817811e-08
3787 3.05876710626762e-08
3788 3.05517104948194e-08
3789 3.05237328745989e-08
3790 3.0506786430351e-08
3791 3.04834522069086e-08
3792 3.04542204787595e-08
3793 3.04602920664365e-08
3794 3.04117477867294e-08
3795 3.03612281982168e-08
3796 3.03429175119163e-08
3797 3.03612424090716e-08
3798 3.02980183164436e-08
3799 3.02575458022147e-08
3800 3.0227944591843e-08
3801 3.02282820996425e-08
3802 3.02463440959855e-08
3803 3.02008658081832e-08
3804 3.01911136091348e-08
3805 3.00955633747435e-08
3806 3.00804856578907e-08
3807 3.00195672764403e-08
3808 2.99811020454399e-08
3809 2.99376026191567e-08
3810 2.99356983646248e-08
3811 2.99033082740152e-08
3812 2.98376896523678e-08
3813 2.9836481729717e-08
3814 2.98280653510119e-08
3815 2.98259479336593e-08
3816 2.98156344058498e-08
3817 2.97693940609634e-08
3818 2.97381657077267e-08
3819 2.97145881233973e-08
3820 2.97249158620616e-08
3821 2.97123126102861e-08
3822 2.9708187909705e-08
3823 2.96665998433809e-08
3824 2.96369062624535e-08
3825 2.96186151160782e-08
3826 2.958874212311e-08
3827 2.95845534736827e-08
3828 2.94243509557646e-08
3829 2.94282020973924e-08
3830 2.94490014596249e-08
3831 2.94549433732527e-08
3832 2.94498683217626e-08
3833 2.94545206003249e-08
3834 2.94257080923899e-08
3835 2.94362205721654e-08
3836 2.94150463986398e-08
3837 2.93460367117859e-08
3838 2.93049176036675e-08
3839 2.92921953359837e-08
3840 2.92530906165211e-08
3841 2.92831590087417e-08
3842 2.92843704841061e-08
3843 2.92873689744511e-08
3844 2.92787039057885e-08
3845 2.9283796720847e-08
3846 2.91847435107684e-08
3847 2.92064168405659e-08
3848 2.91439175015284e-08
3849 2.9109324728438e-08
3850 2.90865109775496e-08
3851 2.9050370997652e-08
3852 2.90349948528501e-08
3853 2.90886692511094e-08
3854 2.90922823609208e-08
3855 2.91085555659265e-08
3856 2.90956272408494e-08
3857 2.91279924624632e-08
3858 2.90827930626847e-08
3859 2.91172082000912e-08
3860 2.90982775652537e-08
3861 2.90848873873983e-08
3862 2.90565367322415e-08
3863 2.90292625493294e-08
3864 2.90440702599426e-08
3865 2.89674826348119e-08
3866 2.88960695371543e-08
3867 2.87814430066646e-08
3868 2.86993042664108e-08
3869 2.86457328968481e-08
3870 2.86361974133342e-08
3871 2.85871806227078e-08
3872 2.85721224457802e-08
3873 2.85713550596256e-08
3874 2.85681451828168e-08
3875 2.85633152685705e-08
3876 2.85492323115477e-08
3877 2.85536927435714e-08
3878 2.85105077324488e-08
3879 2.84993948440615e-08
3880 2.85102252917113e-08
3881 2.85288397350314e-08
3882 2.85091399376824e-08
3883 2.85222139240204e-08
3884 2.8578396538137e-08
3885 2.86259336235162e-08
3886 2.85050045789603e-08
3887 2.85242656161699e-08
3888 2.85401657862394e-08
3889 2.85865286997478e-08
3890 2.86294241647056e-08
3891 2.86953536487999e-08
3892 2.8813747832146e-08
3893 2.89963306698837e-08
3894 2.90549806436502e-08
3895 2.90506889655262e-08
3896 2.91552382236659e-08
3897 2.9239908272416e-08
3898 2.9286916003457e-08
3899 2.93361956948957e-08
3900 2.92773076893127e-08
3901 2.92442781102409e-08
3902 2.94871220773985e-08
3903 2.9435152981705e-08
3904 2.93663386941034e-08
3905 2.91290529474963e-08
3906 2.90339077224644e-08
3907 2.8981816058149e-08
3908 2.88848038820788e-08
3909 2.88545916049543e-08
3910 2.87872108373222e-08
3911 2.88837718187551e-08
3912 2.88584018903748e-08
3913 2.88937620496199e-08
3914 2.89884756199399e-08
3915 2.90289356996709e-08
3916 2.90131048075182e-08
3917 2.90809847314222e-08
3918 2.89077952686512e-08
3919 2.89188282209807e-08
3920 2.89194623803724e-08
3921 2.88112040891519e-08
3922 2.88690937821912e-08
3923 2.88684702809405e-08
3924 2.88295058936683e-08
3925 2.88918524660176e-08
3926 2.87448020941383e-08
3927 2.87112982277904e-08
3928 2.8874769242293e-08
3929 2.89446191459319e-08
3930 2.88071468901308e-08
3931 2.90040595984919e-08
3932 2.90617823139883e-08
3933 2.90923249934849e-08
3934 2.89066726111287e-08
3935 2.86317050068874e-08
3936 2.86393984083588e-08
3937 2.8807210838977e-08
3938 2.86588832665302e-08
3939 2.87864683201633e-08
3940 2.87941173127138e-08
3941 2.85166610325405e-08
3942 2.86378316616265e-08
3943 2.87129182652279e-08
3944 2.87145844879433e-08
3945 2.88489463429187e-08
3946 2.8841322219364e-08
3947 2.8962006126676e-08
3948 2.90187394114128e-08
3949 2.90340391728705e-08
3950 2.88574444340384e-08
3951 2.90466246610777e-08
3952 2.90322965668111e-08
3953 2.90801160929277e-08
3954 2.909690266506e-08
3955 2.90766415389498e-08
3956 2.91499304694298e-08
3957 2.84712076137339e-08
3958 2.83475394269317e-08
3959 2.82874310642001e-08
3960 2.83724421734632e-08
3961 2.8266184060044e-08
3962 2.82793450878671e-08
3963 2.83432477488077e-08
3964 2.82593060063618e-08
3965 2.83109589105379e-08
3966 2.82266832130063e-08
3967 2.83005707757411e-08
3968 2.82441341425965e-08
3969 2.82391443562346e-08
3970 2.81386487444024e-08
3971 2.81473724328407e-08
3972 2.82434928777775e-08
3973 2.80636491822861e-08
3974 2.79984320172844e-08
3975 2.80459477863815e-08
3976 2.79722041085506e-08
3977 2.79003771197495e-08
3978 2.78153322597063e-08
3979 2.79431855432222e-08
3980 2.77634377710001e-08
3981 2.78237042294904e-08
3982 2.7808113145511e-08
3983 2.77794871550441e-08
3984 2.77490723732399e-08
3985 2.77703389173212e-08
3986 2.77315450603055e-08
3987 2.77380802771177e-08
3988 2.77086478206456e-08
3989 2.77057505826406e-08
3990 2.76589577907771e-08
3991 2.76942326848939e-08
3992 2.76467737592156e-08
3993 2.76482889915997e-08
3994 2.763492368274e-08
3995 2.76489942052649e-08
3996 2.76287899225736e-08
3997 2.76421836531426e-08
3998 2.76973555202176e-08
3999 2.76590395031917e-08
4000 2.75940124083718e-08
4001 2.75715645869923e-08
4002 2.75126179616336e-08
4003 2.75533160731811e-08
4004 2.76181921776697e-08
4005 2.76458127501655e-08
4006 2.76211142846705e-08
4007 2.76592313497304e-08
4008 2.76588245640141e-08
4009 2.76224767503663e-08
4010 2.75427289864183e-08
4011 2.7593157980732e-08
4012 2.75753873069107e-08
4013 2.76177516411735e-08
4014 2.75810734251536e-08
4015 2.78177996193563e-08
4016 2.72415690005801e-08
4017 2.62312447318891e-08
4018 2.60239598759426e-08
4019 2.59427750393115e-08
4020 2.58386840812364e-08
4021 2.57973145068036e-08
4022 2.57414782822707e-08
4023 2.57123868863118e-08
4024 2.56730245951076e-08
4025 2.56604373305436e-08
4026 2.56293919420614e-08
4027 2.56455496838726e-08
4028 2.56290491051914e-08
4029 2.56234091722263e-08
4030 2.55947245619836e-08
4031 2.56089460748399e-08
4032 2.5574424356023e-08
4033 2.55964174300516e-08
4034 2.55724401654334e-08
4035 2.55666350312822e-08
4036 2.55315111274967e-08
4037 2.55316567887576e-08
4038 2.55163623563703e-08
4039 2.55491894307625e-08
4040 2.55590304476527e-08
4041 2.54956287193409e-08
4042 2.54943230970639e-08
4043 2.54953107514666e-08
4044 2.54734882076946e-08
4045 2.54745362582298e-08
4046 2.55178100871944e-08
4047 2.54479637362692e-08
4048 2.54536036692343e-08
4049 2.5443011253401e-08
4050 2.54268126553825e-08
4051 2.54207943584106e-08
4052 2.53963978735783e-08
4053 2.53993466259317e-08
4054 2.54296796953213e-08
4055 2.53276830619598e-08
4056 2.53912535441714e-08
4057 2.53166447805597e-08
4058 2.5379145895954e-08
4059 2.53087577561928e-08
4060 2.53585774601106e-08
4061 2.54090188889222e-08
4062 2.55002277071981e-08
4063 2.53843026598588e-08
4064 2.5431154071498e-08
4065 2.54600536209182e-08
4066 2.53294540897286e-08
4067 2.531195164579e-08
4068 2.52750655960199e-08
4069 2.52464857908308e-08
4070 2.52792062838125e-08
4071 2.52575684811518e-08
4072 2.53928238436174e-08
4073 2.53362095747889e-08
4074 2.52863436855932e-08
4075 2.52683189927438e-08
4076 2.51507845661081e-08
4077 2.51787000138393e-08
4078 2.5371271306085e-08
4079 2.52450114146541e-08
4080 2.51692053865327e-08
4081 2.51448035726298e-08
4082 2.51394052241949e-08
4083 2.51180445332011e-08
4084 2.52630538710719e-08
4085 2.51731009370815e-08
4086 2.50762504094837e-08
4087 2.50688039216129e-08
4088 2.50870453299967e-08
4089 2.50687453018372e-08
4090 2.52961935842677e-08
4091 2.51026861519676e-08
4092 2.51466918399501e-08
4093 2.49992115897157e-08
4094 2.49930707241219e-08
4095 2.50159803982797e-08
4096 2.49807534657975e-08
4097 2.49544225283671e-08
4098 2.49684912745352e-08
4099 2.49525466955447e-08
4100 2.4942245602233e-08
4101 2.51154048669378e-08
4102 2.50793181777453e-08
4103 2.49011620212514e-08
4104 2.50192115913705e-08
4105 2.50104505994386e-08
4106 2.4983480173546e-08
4107 2.50033256321558e-08
4108 2.49727012402445e-08
4109 2.49630129900424e-08
4110 2.49432172694242e-08
4111 2.49245211136895e-08
4112 2.49136657970439e-08
4113 2.49074609826039e-08
4114 2.48837146443748e-08
4115 2.49035672084119e-08
4116 2.48532963098569e-08
4117 2.48747227260537e-08
4118 2.48377158840185e-08
4119 2.48424409932113e-08
4120 2.48245761724775e-08
4121 2.47412224041454e-08
4122 2.47010127907288e-08
4123 2.48198812613509e-08
4124 2.47966553956758e-08
4125 2.48137794756076e-08
4126 2.4880415949724e-08
4127 2.48073011022143e-08
4128 2.47912712580955e-08
4129 2.47888891635739e-08
4130 2.47911504658305e-08
4131 2.47781617446208e-08
4132 2.47550246967876e-08
4133 2.46650557755856e-08
4134 2.48358400511961e-08
4135 2.48331790686507e-08
4136 2.46328415443031e-08
4137 2.46400944092784e-08
4138 2.47732625524577e-08
4139 2.47500953065583e-08
4140 2.46832758676874e-08
4141 2.4644279505992e-08
4142 2.46522144919936e-08
4143 2.47323281854506e-08
4144 2.47000624398197e-08
4145 2.46461571151713e-08
4146 2.46654767721566e-08
4147 2.46279903137747e-08
4148 2.45905393825296e-08
4149 2.45025812972699e-08
4150 2.46298110795351e-08
4151 2.48135467728616e-08
4152 2.46811957538284e-08
4153 2.45924169917089e-08
4154 2.46606575160513e-08
4155 2.45334632609229e-08
4156 2.45435973766917e-08
4157 2.45175044710777e-08
4158 2.44992488518392e-08
4159 2.44875906219022e-08
4160 2.44849918118462e-08
4161 2.4501087381168e-08
4162 2.44501254798024e-08
4163 2.44435796048492e-08
4164 2.44473827848424e-08
4165 2.44553746142628e-08
4166 2.44119107151164e-08
4167 2.44012507977232e-08
4168 2.43820998946376e-08
4169 2.4390478969849e-08
4170 2.43578028857883e-08
4171 2.43725679638374e-08
4172 2.4351280103474e-08
4173 2.43423823320654e-08
4174 2.4324590341962e-08
4175 2.43326780946518e-08
4176 2.42865922928104e-08
4177 2.43327740179211e-08
4178 2.42975257691569e-08
4179 2.42867912447764e-08
4180 2.42682673956551e-08
4181 2.42178739284782e-08
4182 2.42542057549144e-08
4183 2.4145217381033e-08
4184 2.42478552792136e-08
4185 2.42396342997608e-08
4186 2.42119160276388e-08
4187 2.41750850449307e-08
4188 2.42032882624699e-08
4189 2.42105517855862e-08
4190 2.41850521831566e-08
4191 2.42011974904699e-08
4192 2.41643576259776e-08
4193 2.41787390109494e-08
4194 2.41470008432998e-08
4195 2.41745148343853e-08
4196 2.42098874281282e-08
4197 2.41439810366728e-08
4198 2.41182824822772e-08
4199 2.41310296189567e-08
4200 2.41055371219545e-08
4201 2.41206397078031e-08
4202 2.40823769814824e-08
4203 2.40968187625867e-08
4204 2.40733299960993e-08
4205 2.40763124992327e-08
4206 2.4045670343753e-08
4207 2.40616859770171e-08
4208 2.40310509269648e-08
4209 2.40342714619146e-08
4210 2.40203323897958e-08
4211 2.402383714184e-08
4212 2.40194300005214e-08
4213 2.40129498507713e-08
4214 2.40052902000798e-08
4215 2.40021762465403e-08
4216 2.39854038852627e-08
4217 2.3993651510068e-08
4218 2.39586857020413e-08
4219 2.39866153606272e-08
4220 2.3971098883635e-08
4221 2.39733299878253e-08
4222 2.39518005429318e-08
4223 2.39638904275807e-08
4224 2.39347777153398e-08
4225 2.39450468342284e-08
4226 2.39178774563698e-08
4227 2.39260362633331e-08
4228 2.39044872785144e-08
4229 2.39248336697528e-08
4230 2.38928645757142e-08
4231 2.3815950100925e-08
4232 2.38143655906242e-08
4233 2.38457289469807e-08
4234 2.38198705204695e-08
4235 2.38528432561225e-08
4236 2.38978579147897e-08
4237 2.38726745038775e-08
4238 2.38769306548647e-08
4239 2.39309176919278e-08
4240 2.39141275670818e-08
4241 2.3935129433994e-08
4242 2.38991244572162e-08
4243 2.39208812757852e-08
4244 2.38816664221986e-08
4245 2.39081732189561e-08
4246 2.38532340546271e-08
4247 2.39743940255721e-08
4248 2.39614141861466e-08
4249 2.40178792410006e-08
4250 2.3987258401803e-08
4251 2.40918094362996e-08
4252 2.39883917174666e-08
4253 2.40580622090647e-08
4254 2.40652227034843e-08
4255 2.41217179564046e-08
4256 2.41303883541377e-08
4257 2.41310900150893e-08
4258 2.41669990685978e-08
4259 2.41666882061509e-08
4260 2.41878819196018e-08
4261 2.42049367216168e-08
4262 2.41817410540079e-08
4263 2.41504984899166e-08
4264 2.41793145505653e-08
4265 2.41715607529613e-08
4266 2.41463755656923e-08
4267 2.41796858091448e-08
4268 2.41101503206664e-08
4269 2.41782487364617e-08
4270 2.41468693928937e-08
4271 2.41490525354493e-08
4272 2.41237696485541e-08
4273 2.41644215748238e-08
4274 2.41791529020929e-08
4275 2.40963355935264e-08
4276 2.40668889261997e-08
4277 2.41491822094986e-08
4278 2.40346640367761e-08
4279 2.40375470639265e-08
4280 2.4068013360079e-08
4281 2.41117508181787e-08
4282 2.39904007770519e-08
4283 2.39310189442676e-08
4284 2.39570816518153e-08
4285 2.39269120072549e-08
4286 2.39361011011852e-08
4287 2.39529764911595e-08
4288 2.39435991034043e-08
4289 2.39499353682504e-08
4290 2.40325785938467e-08
4291 2.40963196063149e-08
4292 2.4103584905788e-08
4293 2.41050130966869e-08
4294 2.4116404873098e-08
4295 2.4115719199358e-08
4296 2.41129960443232e-08
4297 2.41259545674666e-08
4298 2.41349003005098e-08
4299 2.41429027880713e-08
4300 2.41377815513033e-08
4301 2.41541222578689e-08
4302 2.41410003098963e-08
4303 2.41507454035172e-08
4304 2.4133647968938e-08
4305 2.41734721129205e-08
4306 2.4140085486124e-08
4307 2.41437110304332e-08
4308 2.41332518413628e-08
4309 2.41589965810363e-08
4310 2.41333193429227e-08
4311 2.41467379424876e-08
4312 2.41341986395582e-08
4313 2.41512942977806e-08
4314 2.41319284555175e-08
4315 2.41333513173458e-08
4316 2.41026345548789e-08
4317 2.4112843277635e-08
4318 2.40923423433514e-08
4319 2.40683721841606e-08
4320 2.40837945142403e-08
4321 2.41308342197044e-08
4322 2.41368294240374e-08
4323 2.41229756170469e-08
4324 2.37802471048099e-08
4325 2.3926933323537e-08
4326 2.40734596701486e-08
4327 2.38031407917561e-08
4328 2.38915909278603e-08
4329 2.38521273843162e-08
4330 2.38950139674898e-08
4331 2.38801423080304e-08
4332 2.40424515851601e-08
4333 2.38750814673949e-08
4334 2.38037252131562e-08
4335 2.38735520241562e-08
4336 2.38850947908986e-08
4337 2.39085906628134e-08
4338 2.38942945429699e-08
4339 2.38571882249516e-08
4340 2.38312622968806e-08
4341 2.38180657419207e-08
4342 2.38098820659616e-08
4343 2.3816800975851e-08
4344 2.39174049454505e-08
4345 2.37835315886059e-08
4346 2.37695676474914e-08
4347 2.3755745814924e-08
4348 2.37412738357534e-08
4349 2.37358097621154e-08
4350 2.37219648369091e-08
4351 2.36991226643113e-08
4352 2.37661801349986e-08
4353 2.37621033960522e-08
4354 2.37212436360323e-08
4355 2.37618653642357e-08
4356 2.36915020934703e-08
4357 2.37960886551036e-08
4358 2.37187762763824e-08
4359 2.38999717794286e-08
4360 2.37168826799916e-08
4361 2.38918183015357e-08
4362 2.38837980504059e-08
4363 2.3868610199429e-08
4364 2.3685863936862e-08
4365 2.38529160867529e-08
4366 2.37255335377995e-08
4367 2.38757422721392e-08
4368 2.36770514305817e-08
4369 2.36607622383644e-08
4370 2.37280453063704e-08
4371 2.37760140464616e-08
4372 2.37040129746902e-08
4373 2.37857484819415e-08
4374 2.37003785485967e-08
4375 2.3686240524512e-08
4376 2.38483792713851e-08
4377 2.36045902823889e-08
4378 2.38234889593514e-08
4379 2.35850219354461e-08
4380 2.35936195025488e-08
4381 2.38053114998138e-08
4382 2.38096014015809e-08
4383 2.36210127013692e-08
4384 2.37854411722083e-08
4385 2.36094432892742e-08
4386 2.37748380982339e-08
4387 2.37520723089801e-08
4388 2.37477415510057e-08
4389 2.37316655216091e-08
4390 2.3737252163869e-08
4391 2.37170691974598e-08
4392 2.3716667740814e-08
4393 2.37020607585237e-08
4394 2.37103829903162e-08
4395 2.36872850223335e-08
4396 2.36910331352647e-08
4397 2.36749126969471e-08
4398 2.36732944358664e-08
4399 2.36847377266258e-08
4400 2.36846968704185e-08
4401 2.36668480368962e-08
4402 2.36678445730831e-08
4403 2.36549091425786e-08
4404 2.36447394996731e-08
4405 2.36363746353163e-08
4406 2.36462245339908e-08
4407 2.36265602637786e-08
4408 2.36275603526792e-08
4409 2.3613702992975e-08
4410 2.36187194246895e-08
4411 2.35990746944026e-08
4412 2.36061872271875e-08
4413 2.35884467514325e-08
4414 2.35838761852847e-08
4415 2.35890329491895e-08
4416 2.35683224047989e-08
4417 2.35791137725982e-08
4418 2.35562733763572e-08
4419 2.35587904739987e-08
4420 2.3547416461156e-08
4421 2.35512036539376e-08
4422 2.36131665332096e-08
4423 2.35390711367245e-08
4424 2.35175168228352e-08
4425 2.35268089454621e-08
4426 2.34979573576766e-08
4427 2.35059545161675e-08
4428 2.34957262534863e-08
4429 2.34930492837293e-08
4430 2.34702195456293e-08
4431 2.34787496111721e-08
4432 2.34512729235803e-08
4433 2.34586519098912e-08
4434 2.34405828081208e-08
4435 2.344249416808e-08
4436 2.34211032790199e-08
4437 2.34278516586528e-08
4438 2.34011725552818e-08
4439 2.34050983038969e-08
4440 2.33904717816813e-08
4441 2.33856098930119e-08
4442 2.33786927594792e-08
4443 2.33768968627146e-08
4444 2.33558381523835e-08
4445 2.33650627734505e-08
4446 2.3333976528761e-08
4447 2.33454287013046e-08
4448 2.33115446945931e-08
4449 2.33208670152862e-08
4450 2.32889494355959e-08
4451 2.33060273302499e-08
4452 2.32685000156607e-08
4453 2.32753265549945e-08
4454 2.32562129554026e-08
4455 2.32563390767382e-08
4456 2.30858692162883e-08
4457 2.30717844829087e-08
4458 2.30445174054239e-08
4459 2.30348309315787e-08
4460 2.30105250409451e-08
4461 2.2998868587365e-08
4462 2.29730954259821e-08
4463 2.29621761604903e-08
4464 2.2933134502523e-08
4465 2.29170549204127e-08
4466 2.28911680721922e-08
4467 2.28733494367361e-08
4468 2.28506635835402e-08
4469 2.28442935679141e-08
4470 2.28064056528865e-08
4471 2.27758238935394e-08
4472 2.2744735872493e-08
4473 2.27205934066887e-08
4474 2.26991030416457e-08
4475 2.26808154479841e-08
4476 2.26427481209157e-08
4477 2.26442402606608e-08
4478 2.2614418782041e-08
4479 2.25996057423572e-08
4480 2.25584457780315e-08
4481 2.26275922443619e-08
4482 2.2512537611874e-08
4483 2.24999432418826e-08
4484 2.24763372358439e-08
4485 2.24515446234363e-08
4486 2.24147918004292e-08
4487 2.24465726006429e-08
4488 2.24187566288947e-08
4489 2.23925340492315e-08
4490 2.23331291238082e-08
4491 2.23137881505409e-08
4492 2.22967262430984e-08
4493 2.23163603152443e-08
4494 2.23207532457081e-08
4495 2.22865619292634e-08
4496 2.21632916463932e-08
4497 2.22303047081596e-08
4498 2.21221299057106e-08
4499 2.20937614869854e-08
4500 2.21971330205406e-08
4501 2.21816272016895e-08
4502 2.20877804935071e-08
4503 2.20905818082429e-08
4504 2.20743121559508e-08
4505 2.21547367118546e-08
4506 2.21537650446635e-08
4507 2.21302407510393e-08
4508 2.21233875663529e-08
4509 2.21212044237973e-08
4510 2.20361364711152e-08
4511 2.20836113840051e-08
4512 2.20730971278726e-08
4513 2.20889972979421e-08
4514 2.20122124972022e-08
4515 2.20013109952788e-08
4516 2.20535074646477e-08
4517 2.19961577840877e-08
4518 2.204616045276e-08
4519 2.20524274396894e-08
4520 2.20279563478698e-08
4521 2.20257039273974e-08
4522 2.19978080195915e-08
4523 2.19784226374031e-08
4524 2.20342304402266e-08
4525 2.19474891594018e-08
4526 2.19342020102431e-08
4527 2.19287130676094e-08
4528 2.19204725482314e-08
4529 2.19165094961227e-08
4530 2.19034070880753e-08
4531 2.19253717403944e-08
4532 2.18898730253159e-08
4533 2.19508695664672e-08
4534 2.19730917905281e-08
4535 2.18724025558004e-08
4536 2.18425100229069e-08
4537 2.18558096065635e-08
4538 2.18480984415237e-08
4539 2.18406341900845e-08
4540 2.18296491993897e-08
4541 2.18324984757601e-08
4542 2.18486118086503e-08
4543 2.18259827988732e-08
4544 2.18079101443891e-08
4545 2.17934612578574e-08
4546 2.17815596670334e-08
4547 2.17834390525695e-08
4548 2.17417550629762e-08
4549 2.17700115712205e-08
4550 2.17515623290865e-08
4551 2.17567119875639e-08
4552 2.17464730667416e-08
4553 2.17340616615047e-08
4554 2.17125943891006e-08
4555 2.17327009721657e-08
4556 2.170916069133e-08
4557 2.17053415241253e-08
4558 2.17058140350446e-08
4559 2.17048992112723e-08
4560 2.16842472866574e-08
4561 2.16917062090261e-08
4562 2.17052491535696e-08
4563 2.18042348620884e-08
4564 2.16956319576411e-08
4565 2.1657097448724e-08
4566 2.16559072896416e-08
4567 2.16401669916877e-08
4568 2.16342996850472e-08
4569 2.16330420244049e-08
4570 2.16247695306038e-08
4571 2.16172963973804e-08
4572 2.16142570508282e-08
4573 2.16101270211766e-08
4574 2.15932267622065e-08
4575 2.15882618448404e-08
4576 2.15931628133603e-08
4577 2.16097717498087e-08
4578 2.15606519304856e-08
4579 2.15665032499146e-08
4580 2.15524824653812e-08
4581 2.15337738751487e-08
4582 2.15339976961104e-08
4583 2.16034354849626e-08
4584 2.1531823435339e-08
4585 2.14912656559818e-08
4586 2.14951807464558e-08
4587 2.16084359294655e-08
4588 2.1466178168339e-08
4589 2.14820516930558e-08
4590 2.1637044156364e-08
4591 2.14273505605433e-08
4592 2.14487219096782e-08
4593 2.15911555301318e-08
4594 2.14437960721625e-08
4595 2.15208615372831e-08
4596 2.14108464291485e-08
4597 2.14000053233576e-08
4598 2.14077200411111e-08
4599 2.14059028280644e-08
4600 2.14613535831631e-08
4601 2.13854498554156e-08
4602 2.13836148788005e-08
4603 2.14630198058785e-08
4604 2.13594226750047e-08
4605 2.14492175132364e-08
4606 2.13789377312423e-08
4607 2.13800390724828e-08
4608 2.13840287699441e-08
4609 2.13823767580834e-08
4610 2.13265209936253e-08
4611 2.13656150549468e-08
4612 2.13320063835454e-08
4613 2.13088426903596e-08
4614 2.14708943957476e-08
4615 2.14007886967238e-08
4616 2.13447464147976e-08
4617 2.13731397025185e-08
4618 2.14264979092604e-08
4619 2.12997068871346e-08
4620 2.13398489989913e-08
4621 2.12429966950367e-08
4622 2.12579944758318e-08
4623 2.13830553263961e-08
4624 2.13643680524456e-08
4625 2.13458282161127e-08
4626 2.12083648420958e-08
4627 2.12848956238076e-08
4628 2.12102548857729e-08
4629 2.12317416981023e-08
4630 2.11970885288792e-08
4631 2.12104396268842e-08
4632 2.12078834493923e-08
4633 2.11933315341639e-08
4634 2.11706332464701e-08
4635 2.11855688547757e-08
4636 2.11702406716086e-08
4637 2.1168565567109e-08
4638 2.11466737454202e-08
4639 2.11365414060083e-08
4640 2.11420125850736e-08
4641 2.12450004255516e-08
4642 2.12209769756555e-08
4643 2.11240855918504e-08
4644 2.10945749756775e-08
4645 2.11048476472797e-08
4646 2.1093905289149e-08
4647 2.10784349974347e-08
4648 2.10714929949063e-08
4649 2.10768575925613e-08
4650 2.10745447759564e-08
4651 2.10661372790355e-08
4652 2.10288195745534e-08
4653 2.10592432381418e-08
4654 2.11489066259674e-08
4655 2.11707344988099e-08
4656 2.11622861456817e-08
4657 2.11603534694405e-08
4658 2.11235278158028e-08
4659 2.10228794372824e-08
4660 2.11389590276667e-08
4661 2.10030677294526e-08
4662 2.1113129022865e-08
4663 2.11214032930229e-08
4664 2.11061994548345e-08
4665 2.11086330637045e-08
4666 2.09624868574565e-08
4667 2.09649897442432e-08
4668 2.09166870490662e-08
4669 2.09693791219934e-08
4670 2.09454338317983e-08
4671 2.09461230582519e-08
4672 2.09291712849335e-08
4673 2.09472936774091e-08
4674 2.0913139664458e-08
4675 2.09503827619528e-08
4676 2.10839719017031e-08
4677 2.1069162414733e-08
4678 2.10639541364799e-08
4679 2.08941326462764e-08
4680 2.10522195231988e-08
4681 2.10402451017444e-08
4682 2.08785397859401e-08
4683 2.10405879386144e-08
4684 2.0929299182626e-08
4685 2.09069153100927e-08
4686 2.10458281912906e-08
4687 2.09640571569025e-08
4688 2.10275459266995e-08
4689 2.10386197352364e-08
4690 2.10351274176901e-08
4691 2.10412469670018e-08
4692 2.10167367953318e-08
4693 2.08549195690466e-08
4694 2.10371933206943e-08
4695 2.08680344115919e-08
4696 2.10172128589647e-08
4697 2.0860113636445e-08
4698 2.10104253994814e-08
4699 2.10228527919298e-08
4700 2.10078123785706e-08
4701 2.10105763898127e-08
4702 2.09594457345474e-08
4703 2.09994190925045e-08
4704 2.09567581066494e-08
4705 2.09977546461459e-08
4706 2.09755537383671e-08
4707 2.09878958656873e-08
4708 2.08068264839767e-08
4709 2.09732551326169e-08
4710 2.09740456114105e-08
4711 2.09841903853203e-08
4712 2.09645136806103e-08
4713 2.09585184762773e-08
4714 2.09526138661431e-08
4715 2.07911412530848e-08
4716 2.09427053476929e-08
4717 2.09292778663439e-08
4718 2.09304005238664e-08
4719 2.09229860104188e-08
4720 2.0779625131695e-08
4721 2.09410924156828e-08
4722 2.09138182327706e-08
4723 2.08724539874083e-08
4724 2.0895297936363e-08
4725 2.09237072112956e-08
4726 2.08985948546569e-08
4727 2.08967350090461e-08
4728 2.08882724450632e-08
4729 2.08846593352519e-08
4730 2.08812025448424e-08
4731 2.08782608979163e-08
4732 2.07002681662516e-08
4733 2.06984083206407e-08
4734 2.06927790458167e-08
4735 2.0913946130463e-08
4736 2.08524202349736e-08
4737 2.08666453005435e-08
4738 2.08474553176075e-08
4739 2.06726511464694e-08
4740 2.08436308213322e-08
4741 2.08651034228069e-08
4742 2.08279100633035e-08
4743 2.0810960066342e-08
4744 2.07991242007211e-08
4745 2.06067820585076e-08
4746 2.06075334574507e-08
4747 2.07659240913927e-08
4748 2.05641743633578e-08
4749 2.0795209110247e-08
4750 2.06242791733757e-08
4751 2.07473167534999e-08
4752 2.07656061235184e-08
4753 2.07848085409523e-08
4754 2.07667234519704e-08
4755 2.05743866388275e-08
4756 2.07510968408542e-08
4757 2.07329975410175e-08
4758 2.0724254312654e-08
4759 2.05601224934071e-08
4760 2.07394368345604e-08
4761 2.07034194232847e-08
4762 2.0525536825744e-08
4763 2.07019734688174e-08
4764 2.07120507411673e-08
4765 2.07006891628225e-08
4766 2.05270858089079e-08
4767 2.06828811855075e-08
4768 2.07141042096737e-08
4769 2.06573691485801e-08
4770 2.06632471133616e-08
4771 2.06832169169502e-08
4772 2.03957437605595e-08
4773 2.06494146226532e-08
4774 2.06537187352751e-08
4775 2.07012202935175e-08
4776 2.06474162212089e-08
4777 2.06214707532126e-08
4778 2.06632222443659e-08
4779 2.06386747692022e-08
4780 2.0382012522191e-08
4781 2.06175343464565e-08
4782 2.03733279136031e-08
4783 2.06793178136877e-08
4784 2.06540438085767e-08
4785 2.03591756786636e-08
4786 2.06562109639208e-08
4787 2.05503205563673e-08
4788 2.03168504242512e-08
4789 2.06181578477072e-08
4790 2.05578540857232e-08
4791 2.05312531420532e-08
4792 2.05374064421449e-08
4793 2.05146175602522e-08
4794 2.04890486799059e-08
4795 2.05177403955759e-08
4796 2.05203090075656e-08
4797 2.02529033543897e-08
4798 2.05453609680717e-08
4799 2.05138963593754e-08
4800 2.05250625384679e-08
4801 2.03001437881767e-08
4802 2.05076489123712e-08
4803 2.04967616213025e-08
4804 2.04505195000593e-08
4805 2.04583887608578e-08
4806 2.05069170533534e-08
4807 2.04931147607113e-08
4808 2.04465635533779e-08
4809 2.04388737046202e-08
4810 2.04961470018361e-08
4811 2.047839586794e-08
4812 2.04929744285209e-08
4813 2.04688266336461e-08
4814 2.04831831496222e-08
4815 2.04572039308459e-08
4816 2.04711589901763e-08
4817 2.04397370140441e-08
4818 2.04733616726571e-08
4819 2.04175698570452e-08
4820 2.04870254094658e-08
4821 2.04032986061975e-08
4822 2.04113064228295e-08
4823 2.03741965520976e-08
4824 2.04956602800621e-08
4825 2.03893524286514e-08
4826 2.04339922760255e-08
4827 2.04028385297761e-08
4828 2.04125267799782e-08
4829 2.04352357258131e-08
4830 2.04085406352306e-08
4831 2.03428296430275e-08
4832 2.04046433083249e-08
4833 2.03826413525121e-08
4834 2.04006216364405e-08
4835 2.03298231582494e-08
4836 2.03856984626327e-08
4837 2.03490007066875e-08
4838 2.03835561762844e-08
4839 2.04066346043419e-08
4840 2.03728269809744e-08
4841 2.03245704710753e-08
4842 2.03650536434452e-08
4843 2.03099936868512e-08
4844 2.03560936995473e-08
4845 2.03411243404616e-08
4846 2.0337017403449e-08
4847 2.02974543839218e-08
4848 2.03141699017806e-08
4849 2.03381347319009e-08
4850 2.03272190191228e-08
4851 2.03084642436124e-08
4852 2.03298249346062e-08
4853 2.03473859983205e-08
4854 2.03134273846217e-08
4855 2.02964134388139e-08
4856 2.03076595539642e-08
4857 2.02867269649687e-08
4858 2.03067642701171e-08
4859 2.02535730409181e-08
4860 2.02857695086323e-08
4861 2.02435437302029e-08
4862 2.02877217247988e-08
4863 2.02681356142875e-08
4864 2.02691019524082e-08
4865 2.0239030007474e-08
4866 2.02299119678173e-08
4867 2.02553156469776e-08
4868 2.0311023973818e-08
4869 2.02417957950729e-08
4870 2.02510630487041e-08
4871 2.01840286706556e-08
4872 2.02186285491734e-08
4873 2.01893737283854e-08
4874 2.02265653115319e-08
4875 2.01833234569904e-08
4876 2.02183940700706e-08
4877 2.01852046188833e-08
4878 2.02107131030971e-08
4879 2.01618721717978e-08
4880 2.01824548184959e-08
4881 2.01885050898909e-08
4882 2.01923562315187e-08
4883 2.01718641790194e-08
4884 2.017701206114e-08
4885 2.01282652767532e-08
4886 2.01765999463532e-08
4887 2.01385912390606e-08
4888 2.0175695780722e-08
4889 2.01605665495208e-08
4890 2.01677199385131e-08
4891 2.01607512906321e-08
4892 2.01701801927356e-08
4893 2.01261141086206e-08
4894 2.0138420708804e-08
4895 2.00809111561284e-08
4896 2.01216217021738e-08
4897 2.00867624755574e-08
4898 2.00997405386261e-08
4899 2.00695815522067e-08
4900 2.01269383381941e-08
4901 2.00848440101709e-08
4902 2.00993817145445e-08
4903 2.00380227965979e-08
4904 2.00589340693114e-08
4905 2.00572216613182e-08
4906 2.00574330477821e-08
4907 2.00639469483122e-08
4908 2.00579730602612e-08
4909 2.00469081335086e-08
4910 2.00563547991806e-08
4911 2.00289669294307e-08
4912 2.00378291737024e-08
4913 2.00855243548403e-08
4914 2.00457161980694e-08
4915 2.00699634689272e-08
4916 2.00164702590655e-08
4917 2.00747418688252e-08
4918 2.00942267269966e-08
4919 2.00200389599559e-08
4920 2.00678336170768e-08
4921 2.00013516860054e-08
4922 1.99938714473546e-08
4923 2.00533847305451e-08
4924 2.00184970822193e-08
4925 1.9976845067049e-08
4926 2.00730791988235e-08
4927 2.00099901093154e-08
4928 1.99728731331561e-08
4929 1.99590886040824e-08
4930 1.9992651090206e-08
4931 1.96360510074101e-08
4932 2.00527718874355e-08
4933 2.0032734582287e-08
4934 1.99229823749647e-08
4935 1.9948629415012e-08
4936 1.99523775279431e-08
4937 2.0033686709553e-08
4938 1.99615453055912e-08
4939 1.99382075294352e-08
4940 1.9907316683998e-08
4941 2.00226946134308e-08
4942 1.99269329925755e-08
4943 1.9938939388453e-08
4944 1.9999198741516e-08
4945 1.99402663270121e-08
4946 1.99098231234984e-08
4947 1.99273060275118e-08
4948 1.99396321676204e-08
4949 1.9974057963168e-08
4950 1.98899190451129e-08
4951 1.99829752745018e-08
4952 1.98987688548868e-08
4953 1.98957170738367e-08
4954 1.99524485822167e-08
4955 1.98905283355089e-08
4956 1.98795060413204e-08
4957 1.99714520476846e-08
4958 1.98719085631183e-08
4959 1.98875635959439e-08
4960 1.99454568416968e-08
4961 1.98754133151624e-08
4962 1.99210337115119e-08
4963 1.98786356264691e-08
4964 1.98570546672272e-08
4965 1.99260519195832e-08
4966 1.98248120142352e-08
4967 1.99213179286062e-08
4968 1.98387315464288e-08
4969 1.99143883605757e-08
4970 1.98259186845462e-08
4971 1.98492671188433e-08
4972 1.99098000308595e-08
4973 1.98412895002775e-08
4974 1.98237888326958e-08
4975 1.99002005984994e-08
4976 1.98157437125701e-08
4977 1.98830214515056e-08
4978 1.97990512873503e-08
4979 1.98837177833866e-08
4980 1.98038083709662e-08
4981 1.9879045964899e-08
4982 1.98016802954726e-08
4983 1.9875727730323e-08
4984 1.97754417285978e-08
4985 1.98697538422721e-08
4986 1.97795841927473e-08
4987 1.98675014217997e-08
4988 1.97681053748511e-08
4989 1.98463219192035e-08
4990 1.97551912606286e-08
4991 1.98436254095213e-08
4992 1.97541485391639e-08
4993 1.98364489278902e-08
4994 1.98251672856031e-08
4995 1.97567899817841e-08
4996 1.96877696367892e-08
4997 1.97126102108314e-08
4998 1.95713703021738e-08
4999 1.97650589228715e-08
};
\addlegendentry{Test}

\nextgroupplot[
title={Batch Size 8 $\hy$},
ymin=1.04131095330514e-08, ymax=1e-05,
]
\addplot [semithick, black, dashed]
table {%
0 0.00992921364866197
1 0.00200672075431794
2 0.00126416098116897
3 0.000502407551408396
4 0.00023901743082024
5 0.000216438815314177
6 0.000213986882208701
7 0.000212700884676451
8 0.000211402351040306
9 0.000209579691640101
10 0.000206649657291564
11 0.00020138838245839
12 0.000190890367448446
13 0.00016819439144092
14 0.000118282980005461
15 5.34219913588458e-05
16 2.36267233140097e-05
17 1.74708007177742e-05
18 1.52814658090392e-05
19 1.28881240578949e-05
20 1.07959244194262e-05
21 9.10524824712411e-06
22 7.78487498871527e-06
23 6.78648272477744e-06
24 6.00090184323676e-06
25 5.35711603035338e-06
26 4.83139509694297e-06
27 4.39034949650363e-06
28 4.01557221093185e-06
29 3.686210322428e-06
30 3.40009338026448e-06
31 3.14873896701329e-06
32 2.9219724895313e-06
33 2.71820027893455e-06
34 2.53445844606404e-06
35 2.36520242204108e-06
36 2.20963027382481e-06
37 2.06319378885667e-06
38 1.92788907661168e-06
39 1.80244110994465e-06
40 1.68748155439857e-06
41 1.58085821576037e-06
42 1.48081458138449e-06
43 1.38898999583148e-06
44 1.30267445815946e-06
45 1.22325308054627e-06
46 1.14967516200437e-06
47 1.08226000497069e-06
48 1.01985660982962e-06
49 9.63352078137802e-07
50 9.11907369776088e-07
51 8.65516229815455e-07
52 8.23888238734583e-07
53 7.85104627059496e-07
54 7.50465335620731e-07
55 7.18375335544863e-07
56 6.90192174236159e-07
57 6.65181878037657e-07
58 6.4266843726557e-07
59 6.22291989646584e-07
60 6.03765364701303e-07
61 5.87750111364471e-07
62 5.73313236625239e-07
63 5.60323386402217e-07
64 5.48583425331373e-07
65 5.37490703564458e-07
66 5.27683143268121e-07
67 5.18289648008263e-07
68 5.09672443650189e-07
69 5.01638603761734e-07
70 4.93823304159591e-07
71 4.87081694149793e-07
72 4.80593278043173e-07
73 4.74654873155345e-07
74 4.69209972143503e-07
75 4.64152245800875e-07
76 4.59416280072844e-07
77 4.54937307832637e-07
78 4.50722299964568e-07
79 4.46756310768848e-07
80 4.42911127636947e-07
81 4.39368550898323e-07
82 4.3580246584618e-07
83 4.32537831928315e-07
84 4.29270620273314e-07
85 4.26294999499532e-07
86 4.23264836843629e-07
87 4.20480235025167e-07
88 4.17808279431497e-07
89 4.15256148709631e-07
90 4.12807313178831e-07
91 4.10464862126147e-07
92 4.08175562292001e-07
93 4.05933114317136e-07
94 4.03772737652019e-07
95 4.01782583251631e-07
96 3.99827103544581e-07
97 3.97861613850026e-07
98 3.96038776157681e-07
99 3.94232782863213e-07
100 3.92478577763811e-07
101 3.90764023306289e-07
102 3.89028642922895e-07
103 3.87322530974288e-07
104 3.85694302918793e-07
105 3.84201664688888e-07
106 3.82669817414083e-07
107 3.81219369534946e-07
108 3.79825835368663e-07
109 3.78479776374618e-07
110 3.77162641674644e-07
111 3.75942512947702e-07
112 3.74655052507222e-07
113 3.73505615749892e-07
114 3.72287903998014e-07
115 3.71105319780796e-07
116 3.69652318015667e-07
117 3.68693461300751e-07
118 3.67441280557657e-07
119 3.66364826978938e-07
120 3.65183808375136e-07
121 3.64104361567641e-07
122 3.62924890504601e-07
123 3.61896144907448e-07
124 3.60712123780971e-07
125 3.59640672796147e-07
126 3.58596161801117e-07
127 3.5742738637623e-07
128 3.56457087141493e-07
129 3.55432859292293e-07
130 3.54498676268022e-07
131 3.53562694781928e-07
132 3.52685108152784e-07
133 3.51719392490679e-07
134 3.5092776143486e-07
135 3.50080307839562e-07
136 3.49304965780561e-07
137 3.48480901632087e-07
138 3.47716653401164e-07
139 3.46967606747128e-07
140 3.46112042915792e-07
141 3.45329035271291e-07
142 3.44496941018235e-07
143 3.43824144184879e-07
144 3.43083332678518e-07
145 3.4233321095023e-07
146 3.41679052704436e-07
147 3.40947543747916e-07
148 3.40194928698523e-07
149 3.39521047635571e-07
150 3.38795408056924e-07
151 3.3812119126253e-07
152 3.37482047397586e-07
153 3.36702397470745e-07
154 3.36082607462984e-07
155 3.35380337508795e-07
156 3.34556997048097e-07
157 3.34082712837969e-07
158 3.33378324619105e-07
159 3.32612706875679e-07
160 3.32062574868175e-07
161 3.3132305282102e-07
162 3.30766553322093e-07
163 3.30097321636913e-07
164 3.29414650636295e-07
165 3.28537348163849e-07
166 3.27973180903385e-07
167 3.27137078219408e-07
168 3.26636638044775e-07
169 3.25644847558593e-07
170 3.25163561235797e-07
171 3.24332636919777e-07
172 3.23674372593175e-07
173 3.23095076447899e-07
174 3.22387813003644e-07
175 3.21714049066557e-07
176 3.2105395463411e-07
177 3.20316570972423e-07
178 3.1980331064041e-07
179 3.19049457429443e-07
180 3.18477796302119e-07
181 3.17787826242366e-07
182 3.17100131972836e-07
183 3.16552034925977e-07
184 3.15722945417818e-07
185 3.15131155588233e-07
186 3.14376554619855e-07
187 3.13839009550065e-07
188 3.13176215280819e-07
189 3.12552466571248e-07
190 3.11906353291036e-07
191 3.11253471984685e-07
192 3.10476658679448e-07
193 3.09903623191943e-07
194 3.09106110000457e-07
195 3.08375990968557e-07
196 3.07760972111382e-07
197 3.07047909446467e-07
198 3.06382353867818e-07
199 3.05707962635182e-07
200 3.05187997366119e-07
201 3.04330951889753e-07
202 3.03748435502627e-07
203 3.03039910468428e-07
204 3.02401619199699e-07
205 3.01617496207385e-07
206 3.00954340366033e-07
207 3.00350191194454e-07
208 2.99507105479435e-07
209 2.99040306259712e-07
210 2.98211046018793e-07
211 2.97592594581175e-07
212 2.9682772037809e-07
213 2.96222685824077e-07
214 2.95478178680142e-07
215 2.94832764268449e-07
216 2.94001875284167e-07
217 2.93444783839192e-07
218 2.92554521603705e-07
219 2.91981816447517e-07
220 2.91166709926571e-07
221 2.90588125990254e-07
222 2.89782661505811e-07
223 2.89378933310047e-07
224 2.8829365759897e-07
225 2.87752485601089e-07
226 2.87115306575103e-07
227 2.86335935637538e-07
228 2.85497558003556e-07
229 2.84896734070728e-07
230 2.83985925280561e-07
231 2.8345670237151e-07
232 2.82532933809421e-07
233 2.81932129555074e-07
234 2.81124559474932e-07
235 2.80559271613967e-07
236 2.79614833642583e-07
237 2.79148294065834e-07
238 2.77929935535326e-07
239 2.77727011956941e-07
240 2.76532437833765e-07
241 2.76161216547166e-07
242 2.75075922427703e-07
243 2.74527142540393e-07
244 2.73887049967669e-07
245 2.72647034254447e-07
246 2.72249556365622e-07
247 2.71148433375501e-07
248 2.7058259957613e-07
249 2.70055220751075e-07
250 2.68941878985629e-07
251 2.68189869997215e-07
252 2.67559061356337e-07
253 2.66803659307868e-07
254 2.65958610004446e-07
255 2.65124098994818e-07
256 2.64697630781185e-07
257 2.63409367962986e-07
258 2.63022068216046e-07
259 2.61876886497703e-07
260 2.61363421259375e-07
261 2.60480494876703e-07
262 2.59656966875355e-07
263 2.58743027126584e-07
264 2.58008398548171e-07
265 2.57149901306875e-07
266 2.5630649438213e-07
267 2.55376287976006e-07
268 2.54595903724564e-07
269 2.53602089596683e-07
270 2.52833712810485e-07
271 2.51941589647586e-07
272 2.51059667350617e-07
273 2.50574614073074e-07
274 2.49535521565036e-07
275 2.48394841076305e-07
276 2.47693585624376e-07
277 2.4667336037254e-07
278 2.45769091423043e-07
279 2.44619258934264e-07
280 2.44012597516274e-07
281 2.42730168386274e-07
282 2.42055331984403e-07
283 2.40982313458815e-07
284 2.39931178159125e-07
285 2.3913058593017e-07
286 2.38069788901285e-07
287 2.37142247520694e-07
288 2.35844101290184e-07
289 2.35020797306973e-07
290 2.34060848772089e-07
291 2.32906242009889e-07
292 2.31874630248541e-07
293 2.30836311773785e-07
294 2.2979153503222e-07
295 2.28542839479928e-07
296 2.27529159790407e-07
297 2.26276813455328e-07
298 2.25121207147794e-07
299 2.24057397085531e-07
300 2.22706123464533e-07
301 2.21499047603047e-07
302 2.20050358137769e-07
303 2.19157359666156e-07
304 2.17522500486567e-07
305 2.16525433364367e-07
306 2.14961929874136e-07
307 2.13774665770217e-07
308 2.12247253724485e-07
309 2.11031619619106e-07
310 2.09555445440657e-07
311 2.0825630266863e-07
312 2.06607928188163e-07
313 2.05124780926269e-07
314 2.03731222994108e-07
315 2.02252662269586e-07
316 2.00543955765298e-07
317 1.99132460805274e-07
318 1.97207931892152e-07
319 1.96028824570149e-07
320 1.93877522857377e-07
321 1.922620320709e-07
322 1.90588699020111e-07
323 1.88972460396286e-07
324 1.87196441459037e-07
325 1.85590157910198e-07
326 1.83765746233355e-07
327 1.82188590639676e-07
328 1.80238670103527e-07
329 1.78675458002431e-07
330 1.77092042476623e-07
331 1.75447707614573e-07
332 1.73923921509278e-07
333 1.72553369558237e-07
334 1.71055340431714e-07
335 1.69546539872556e-07
336 1.68209358074378e-07
337 1.66666668827808e-07
338 1.65248274938357e-07
339 1.64065121324164e-07
340 1.62703118817831e-07
341 1.61548255094957e-07
342 1.60328836732226e-07
343 1.59309384464734e-07
344 1.58235727046652e-07
345 1.57089758349827e-07
346 1.56149378742754e-07
347 1.55136367471442e-07
348 1.54267169180322e-07
349 1.53302134526356e-07
350 1.52534413739858e-07
351 1.51720085445284e-07
352 1.50982757982199e-07
353 1.5024580619194e-07
354 1.49613354961886e-07
355 1.49019737149203e-07
356 1.48341590035628e-07
357 1.47822713071122e-07
358 1.47226323369054e-07
359 1.46896607123637e-07
360 1.4627099807818e-07
361 1.45859764717926e-07
362 1.45377279645587e-07
363 1.45073916311134e-07
364 1.44587152755449e-07
365 1.44115452554772e-07
366 1.43958323189253e-07
367 1.43704284532475e-07
368 1.43287681117599e-07
369 1.42678059056323e-07
370 1.42246110966582e-07
371 1.42098128948831e-07
372 1.41863679840704e-07
373 1.41446783704424e-07
374 1.41589847558521e-07
375 1.41007341365906e-07
376 1.40874215525777e-07
377 1.40478163052293e-07
378 1.40558890300291e-07
379 1.40044542031426e-07
380 1.39996844358059e-07
381 1.39725586841166e-07
382 1.3952485137203e-07
383 1.39072659354866e-07
384 1.39123025679311e-07
385 1.38912281296477e-07
386 1.38846833404926e-07
387 1.38485874193073e-07
388 1.38527839115277e-07
389 1.38133886938618e-07
390 1.38127816111222e-07
391 1.38094697967439e-07
392 1.3782527238515e-07
393 1.37559828430334e-07
394 1.37668036550131e-07
395 1.37351703892286e-07
396 1.37323510942089e-07
397 1.37024035455013e-07
398 1.37010230736756e-07
399 1.36805005823248e-07
400 1.36752766215942e-07
401 1.36658234238141e-07
402 1.36398255823345e-07
403 1.36280580456116e-07
404 1.36291368542985e-07
405 1.36148297976924e-07
406 1.35836149919299e-07
407 1.35868911891102e-07
408 1.35506522719986e-07
409 1.35691246809344e-07
410 1.354057372005e-07
411 1.35398193473968e-07
412 1.35180042462935e-07
413 1.35118144183544e-07
414 1.3501529408444e-07
415 1.34790796132656e-07
416 1.34804363578489e-07
417 1.34655041658327e-07
418 1.34386416748455e-07
419 1.34470140811693e-07
420 1.34413915668929e-07
421 1.34160469351663e-07
422 1.34098856398701e-07
423 1.33853180378729e-07
424 1.33922528920749e-07
425 1.33754743404602e-07
426 1.33718301505326e-07
427 1.33468070627529e-07
428 1.33613487438566e-07
429 1.33340412028282e-07
430 1.33129561460166e-07
431 1.331180458779e-07
432 1.3298611439172e-07
433 1.3290237995367e-07
434 1.32681013306879e-07
435 1.32716785639531e-07
436 1.3228231260598e-07
437 1.32397032944098e-07
438 1.3225008732487e-07
439 1.32175409106594e-07
440 1.31989880332029e-07
441 1.32121148130437e-07
442 1.31403901177052e-07
443 1.31855109840373e-07
444 1.31369053636199e-07
445 1.31683956377415e-07
446 1.31280352425378e-07
447 1.31185700146474e-07
448 1.30986063259897e-07
449 1.31032885485993e-07
450 1.30975896351693e-07
451 1.30777439360408e-07
452 1.30625523121708e-07
453 1.30380974598054e-07
454 1.30404333816614e-07
455 1.30561860974154e-07
456 1.30192552500574e-07
457 1.29974513926712e-07
458 1.30074026222005e-07
459 1.29732403561178e-07
460 1.29914991662261e-07
461 1.29510201988481e-07
462 1.29554904837903e-07
463 1.29342622264161e-07
464 1.29447182390052e-07
465 1.29130401937516e-07
466 1.29078021466e-07
467 1.2905404918051e-07
468 1.29112589590363e-07
469 1.28563399002246e-07
470 1.28656974677455e-07
471 1.28463524502465e-07
472 1.28475131308647e-07
473 1.28400459592726e-07
474 1.28441681587432e-07
475 1.28189373365117e-07
476 1.27999679319757e-07
477 1.2795162957957e-07
478 1.27829127912094e-07
479 1.27790509203862e-07
480 1.27514238014825e-07
481 1.27600200574562e-07
482 1.27356179074667e-07
483 1.27263940063571e-07
484 1.2728896299663e-07
485 1.26997854104971e-07
486 1.26975783809158e-07
487 1.26859248855027e-07
488 1.26912741739815e-07
489 1.26519715434981e-07
490 1.26675656715314e-07
491 1.26299687665643e-07
492 1.26432778644769e-07
493 1.26085393705289e-07
494 1.26316919569547e-07
495 1.26116897606998e-07
496 1.26273674043276e-07
497 1.25751559194143e-07
498 1.25628857988502e-07
499 1.25315489196254e-07
500 1.25453702103151e-07
501 1.25125550901828e-07
502 1.25359528042601e-07
503 1.25079495234992e-07
504 1.24851901802892e-07
505 1.24721379810566e-07
506 1.24799847048607e-07
507 1.24672360920286e-07
508 1.25036875566309e-07
509 1.24486701122528e-07
510 1.24399725871527e-07
511 1.24490443867309e-07
512 1.24312774434365e-07
513 1.24101673411481e-07
514 1.24152042808134e-07
515 1.24014100973291e-07
516 1.23881044217455e-07
517 1.23838348779159e-07
518 1.23700184784248e-07
519 1.23544803625641e-07
520 1.23509003769229e-07
521 1.23318160473396e-07
522 1.23166786038631e-07
523 1.23097188374111e-07
524 1.23075328192002e-07
525 1.22795801575037e-07
526 1.22920956532546e-07
527 1.22719954749329e-07
528 1.22615511587654e-07
529 1.22419434997134e-07
530 1.2249641953499e-07
531 1.22195715856144e-07
532 1.22151683097371e-07
533 1.21930512703372e-07
534 1.21953055855606e-07
535 1.21716711187325e-07
536 1.21954333091701e-07
537 1.21610385832582e-07
538 1.21329099072476e-07
539 1.20725858640647e-07
540 1.20942412961789e-07
541 1.20753868710466e-07
542 1.20644963381267e-07
543 1.20525537901095e-07
544 1.20449411774359e-07
545 1.20254195577374e-07
546 1.20137931456377e-07
547 1.20181844176948e-07
548 1.19969868253911e-07
549 1.19878464018619e-07
550 1.19711465918826e-07
551 1.19717940735953e-07
552 1.19390882654891e-07
553 1.19416006300277e-07
554 1.19065409416308e-07
555 1.1920391341036e-07
556 1.19124553316752e-07
557 1.18851538515763e-07
558 1.18815446831455e-07
559 1.18510580794151e-07
560 1.18468417531048e-07
561 1.18577416395205e-07
562 1.18237415130729e-07
563 1.18084561453458e-07
564 1.18283585892165e-07
565 1.17899095330998e-07
566 1.17778002480584e-07
567 1.17744265758724e-07
568 1.17551698332719e-07
569 1.17448735084125e-07
570 1.17574920815322e-07
571 1.17086749381379e-07
572 1.17440545198022e-07
573 1.16962291104805e-07
574 1.1700228750211e-07
575 1.16766782863209e-07
576 1.16646028259559e-07
577 1.16881065177488e-07
578 1.16537355619784e-07
579 1.16428624844289e-07
580 1.16269881194064e-07
581 1.16240599258965e-07
582 1.16141512337187e-07
583 1.16207368294319e-07
584 1.16103182377536e-07
585 1.15947644616199e-07
586 1.15857337320335e-07
587 1.15667909637907e-07
588 1.15927704563923e-07
589 1.15531978823746e-07
590 1.15543204294077e-07
591 1.15292137150114e-07
592 1.15262041508579e-07
593 1.15239616154561e-07
594 1.14997543630757e-07
595 1.14911228040526e-07
596 1.14865359612359e-07
597 1.14816017370956e-07
598 1.14577187169118e-07
599 1.14493252345582e-07
600 1.14365159084251e-07
601 1.14380263292801e-07
602 1.14522381095661e-07
603 1.14058373540615e-07
604 1.14067205493562e-07
605 1.14104901139278e-07
606 1.13698014514441e-07
607 1.1366493384557e-07
608 1.13789974213852e-07
609 1.13452607438447e-07
610 1.13471539264332e-07
611 1.1348867326344e-07
612 1.13111791271159e-07
613 1.13170071247559e-07
614 1.13136127258606e-07
615 1.12829264902636e-07
616 1.12865354131131e-07
617 1.1276192539178e-07
618 1.12722970031953e-07
619 1.12532454583203e-07
620 1.12460691267025e-07
621 1.12278941098332e-07
622 1.12255896143054e-07
623 1.12001460827571e-07
624 1.12163351251304e-07
625 1.11881518908064e-07
626 1.11918165370817e-07
627 1.11592880374545e-07
628 1.11716747395185e-07
629 1.11352572401913e-07
630 1.1134874616836e-07
631 1.11138557483237e-07
632 1.11260068193175e-07
633 1.10919281397948e-07
634 1.10981354802142e-07
635 1.10863942940576e-07
636 1.10801262416693e-07
637 1.10558969159236e-07
638 1.10724243130633e-07
639 1.10348977312213e-07
640 1.1045051743519e-07
641 1.10224538386383e-07
642 1.10288473085696e-07
643 1.10004533510732e-07
644 1.10114767164049e-07
645 1.09837390710688e-07
646 1.09837552175307e-07
647 1.09691784476951e-07
648 1.09912586361283e-07
649 1.09487940306963e-07
650 1.09425556784615e-07
651 1.09327103393397e-07
652 1.09239030370745e-07
653 1.09265873359909e-07
654 1.09008004423039e-07
655 1.08927099962131e-07
656 1.08924782461273e-07
657 1.08795310009846e-07
658 1.08747635980677e-07
659 1.08530513001526e-07
660 1.08618096806623e-07
661 1.08450240848157e-07
662 1.08315205736176e-07
663 1.08414196163586e-07
664 1.08169981978712e-07
665 1.08051510158802e-07
666 1.0791278790645e-07
667 1.08005858932003e-07
668 1.07878166138775e-07
669 1.07689055652394e-07
670 1.07892881868743e-07
671 1.07539050166139e-07
672 1.07496840037413e-07
673 1.07523671808885e-07
674 1.07193277184336e-07
675 1.07297394445816e-07
676 1.0708554678196e-07
677 1.07161906802489e-07
678 1.07117906687826e-07
679 1.06839930356095e-07
680 1.06911912333807e-07
681 1.06696750162172e-07
682 1.06762523054194e-07
683 1.06513298879563e-07
684 1.06623341791234e-07
685 1.06387472997405e-07
686 1.06279337397375e-07
687 1.06241473223356e-07
688 1.06203651604631e-07
689 1.06112566524708e-07
690 1.05914195843049e-07
691 1.05987585365952e-07
692 1.0586157462189e-07
693 1.05796570525563e-07
694 1.05640443638144e-07
695 1.0560942948068e-07
696 1.0552467830216e-07
697 1.05421141482154e-07
698 1.0536793726601e-07
699 1.05170445740654e-07
700 1.05203177099611e-07
701 1.05005269863234e-07
702 1.05053835443236e-07
703 1.05023843158136e-07
704 1.04824690644989e-07
705 1.04726880135964e-07
706 1.04681870645607e-07
707 1.04636874631581e-07
708 1.04504466916566e-07
709 1.04444827637273e-07
710 1.04361506653383e-07
711 1.04258656082656e-07
712 1.04210623618428e-07
713 1.04110616025999e-07
714 1.04005275107788e-07
715 1.04049853670851e-07
716 1.03780368700512e-07
717 1.03907101858702e-07
718 1.03679961773295e-07
719 1.03736410594024e-07
720 1.03518988347595e-07
721 1.03448151581809e-07
722 1.03354885988338e-07
723 1.03372717826566e-07
724 1.03237004646317e-07
725 1.03147705873496e-07
726 1.03038839745828e-07
727 1.03003792749412e-07
728 1.02922307459785e-07
729 1.028185640779e-07
730 1.02748641901407e-07
731 1.0268501551991e-07
732 1.0264917724534e-07
733 1.02539943124924e-07
734 1.02440288250349e-07
735 1.02399548371324e-07
736 1.02322729478743e-07
737 1.02193698007014e-07
738 1.0219464182093e-07
739 1.02006105211494e-07
740 1.02027334409449e-07
741 1.01988515870843e-07
742 1.02811021253757e-07
743 1.014339746499e-07
744 1.0223379355434e-07
745 1.0145927304972e-07
746 1.01517979946841e-07
747 1.0221223485285e-07
748 1.01169380769761e-07
749 1.01262596533758e-07
750 1.01959473902014e-07
751 1.00891126288083e-07
752 1.0121490856374e-07
753 1.01659451659408e-07
754 1.00704914865801e-07
755 1.00905811425633e-07
756 1.01393089427582e-07
757 1.00938773774928e-07
758 1.00368856827515e-07
759 1.01127933790579e-07
760 1.0030037288189e-07
761 1.00952838415935e-07
762 1.00193742011534e-07
763 1.00346110603056e-07
764 1.00819456480217e-07
765 9.99374107246709e-08
766 1.00160732833032e-07
767 1.00477947547972e-07
768 1.00102307870387e-07
769 9.97163229419229e-08
770 1.00291799554064e-07
771 9.98256707447354e-08
772 9.95153247611569e-08
773 1.00091528397961e-07
774 9.96103936250492e-08
775 9.93362153680621e-08
776 9.98928482331962e-08
777 9.90722844909087e-08
778 9.9739392800835e-08
779 9.89679240070984e-08
780 9.96021776575873e-08
781 9.88137820359114e-08
782 9.95558609240277e-08
783 9.85513536884142e-08
784 9.89602590237837e-08
785 9.87828351597386e-08
786 9.88520564861872e-08
787 9.86416073738283e-08
788 9.87046146914139e-08
789 9.84867705806636e-08
790 9.85046546482238e-08
791 9.84393849225995e-08
792 9.84302503157508e-08
793 9.82163410432335e-08
794 9.83739863418265e-08
795 9.80976378706089e-08
796 9.81978526004568e-08
797 9.79254597615054e-08
798 9.80393007932179e-08
799 9.78581151152014e-08
800 9.8592624279803e-08
801 9.74275671818958e-08
802 9.72489105919294e-08
803 9.84910309123066e-08
804 9.71617625520338e-08
805 9.75766307984571e-08
806 9.73502101606627e-08
807 9.81732060427731e-08
808 9.69098024263815e-08
809 9.71954043764711e-08
810 9.78098066823918e-08
811 9.66916605138479e-08
812 9.69459599664546e-08
813 9.7708163761645e-08
814 9.64820346194273e-08
815 9.65686165441682e-08
816 9.74989338171639e-08
817 9.63598134298138e-08
818 9.66107345616862e-08
819 9.65347870423017e-08
820 9.65881227807941e-08
821 9.63969397007602e-08
822 9.64134347700707e-08
823 9.60599173032151e-08
824 9.62245484377178e-08
825 9.62596056455212e-08
826 9.60810136882984e-08
827 9.61121862870229e-08
828 9.59510423594878e-08
829 9.59364160610932e-08
830 9.59976774579374e-08
831 9.57143815787376e-08
832 9.55772441244562e-08
833 9.65233895424689e-08
834 9.53262729979087e-08
835 9.53285597997677e-08
836 9.55185778792611e-08
837 9.5617314039842e-08
838 9.52565698142394e-08
839 9.51738573151672e-08
840 9.59948143304246e-08
841 9.49107437833163e-08
842 9.49602729134469e-08
843 9.50094246716659e-08
844 9.60226167627809e-08
845 9.43043226815377e-08
846 9.55579815338226e-08
847 9.45058484109751e-08
848 9.48965120395329e-08
849 9.45455624004055e-08
850 9.47942502369514e-08
851 9.46238431591695e-08
852 9.46524602847276e-08
853 9.44445758515045e-08
854 9.44442406929369e-08
855 9.43877116670322e-08
856 9.43555026360343e-08
857 9.42692072305462e-08
858 9.39749613708685e-08
859 9.38712820914844e-08
860 9.41206913980253e-08
861 9.40432060385632e-08
862 9.39873546110803e-08
863 9.38680113033996e-08
864 9.46750205086388e-08
865 9.31431298818453e-08
866 9.45066880637668e-08
867 9.37789293331548e-08
868 9.32700528952779e-08
869 9.33686769437614e-08
870 9.34908273704238e-08
871 9.35309067884305e-08
872 9.31133938077977e-08
873 9.41448383295906e-08
874 9.28515620808312e-08
875 9.30728783394841e-08
876 9.31820487757307e-08
877 9.3063945087124e-08
878 9.29865992178591e-08
879 9.30992449008627e-08
880 9.29089245111925e-08
881 9.29120207722178e-08
882 9.27700255610731e-08
883 9.27998438386979e-08
884 9.34596111612507e-08
885 9.29381178274014e-08
886 9.2378845400809e-08
887 9.30733244643989e-08
888 9.26352607972802e-08
889 9.19315548628674e-08
890 9.31159488146704e-08
891 9.20655967533435e-08
892 9.27461955768649e-08
893 9.19789350835032e-08
894 9.28746102113465e-08
895 9.22676532191247e-08
896 9.18383602588335e-08
897 9.2564347734303e-08
898 9.21666677680122e-08
899 9.14046617204178e-08
900 9.17563176443537e-08
901 9.25434725882468e-08
902 9.19264198602221e-08
903 9.12620793309671e-08
904 9.16315926087208e-08
905 9.16355473830066e-08
906 9.21978314023519e-08
907 9.09186351467639e-08
908 9.20797825996189e-08
909 9.16253927876198e-08
910 9.08707156002109e-08
911 9.13018881831818e-08
912 9.1255958625247e-08
913 9.10679085590615e-08
914 9.1222078061115e-08
915 9.10960571403052e-08
916 9.08709917259998e-08
917 9.11676548618345e-08
918 9.08725164547874e-08
919 9.07622649375384e-08
920 9.06163120815151e-08
921 9.13874850221319e-08
922 9.03386978876597e-08
923 9.08811845716784e-08
924 9.02924179424502e-08
925 9.05743732921849e-08
926 9.05516613185497e-08
927 9.03747327809512e-08
928 9.12048730992154e-08
929 8.99823015263479e-08
930 9.06378743490421e-08
931 8.97908594161834e-08
932 9.03441504975788e-08
933 9.07426795615507e-08
934 8.96518661592793e-08
935 9.01816837686198e-08
936 9.00818103435341e-08
937 8.99507629688046e-08
938 9.06435245102699e-08
939 8.95622177843336e-08
940 8.96373378047599e-08
941 8.98199634260521e-08
942 8.95345265936598e-08
943 8.98784231990746e-08
944 8.96560154730608e-08
945 8.94704221963139e-08
946 8.95942801903971e-08
947 8.93604354725852e-08
948 8.93294071380168e-08
949 8.95507709506305e-08
950 8.90859318269932e-08
951 8.93173802785086e-08
952 8.92596041106586e-08
953 8.92664535108167e-08
954 8.88707386677012e-08
955 8.91765430877101e-08
956 8.89775873789134e-08
957 8.99862871266421e-08
958 8.84707484365421e-08
959 8.87018594237787e-08
960 8.90493015308635e-08
961 8.87085552756517e-08
962 8.93550197682558e-08
963 8.82258631342836e-08
964 8.92082657779625e-08
965 8.82544383093986e-08
966 8.90447985231191e-08
967 8.82099455514762e-08
968 8.85320571741843e-08
969 8.84250508468298e-08
970 8.83423099597991e-08
971 8.84771098279558e-08
972 8.82894004625712e-08
973 8.79818003669541e-08
974 8.8261926411981e-08
975 8.8120513877854e-08
976 8.86826449475464e-08
977 8.774217194496e-08
978 8.87183626989696e-08
979 8.76790066977051e-08
980 8.79072576083217e-08
981 8.80860432994623e-08
982 8.78320468382654e-08
983 8.77252277744489e-08
984 8.77723732068958e-08
985 8.7934629156905e-08
986 8.83188352958086e-08
987 8.73201667070944e-08
988 8.75043330506031e-08
989 8.72928016635299e-08
990 8.76041021591334e-08
991 8.73279010873418e-08
992 8.74882470087712e-08
993 8.71622903231284e-08
994 8.74586358046159e-08
995 8.73130701926073e-08
996 8.74654702247568e-08
997 8.71811586016236e-08
998 8.70863416011503e-08
999 8.79202082728625e-08
1000 8.65883544989288e-08
1001 8.69666714198303e-08
1002 8.71845490371115e-08
1003 8.70357206554928e-08
1004 8.67349470032863e-08
1005 8.68304299386224e-08
1006 8.76139849230739e-08
1007 8.64803148070692e-08
1008 8.67046929489135e-08
1009 8.75601630312417e-08
1010 8.6143355759738e-08
1011 8.66784950463995e-08
1012 8.63769116490332e-08
1013 8.6498242773736e-08
1014 8.68369513735701e-08
1015 8.64259720678717e-08
1016 8.69234008487041e-08
1017 8.59307486651417e-08
1018 8.64885832552176e-08
1019 8.61274292516256e-08
1020 8.60205321693286e-08
1021 8.616814576623e-08
1022 8.59726524051752e-08
1023 8.60099813575488e-08
1024 8.67865943181556e-08
1025 8.57542093086749e-08
1026 8.57959661715313e-08
1027 8.59571758802247e-08
1028 8.57515391379238e-08
1029 8.58025710375188e-08
1030 8.5847442393927e-08
1031 8.5636725547289e-08
1032 8.57842188759506e-08
1033 8.57204217279417e-08
1034 8.63580742729297e-08
1035 8.55198876275054e-08
1036 8.61191013141749e-08
1037 8.50749044172261e-08
1038 8.54357005595219e-08
1039 8.52294695228295e-08
1040 8.5291714624347e-08
1041 8.50699286507606e-08
1042 8.59807524813405e-08
1043 8.5004125940813e-08
1044 8.51956005574195e-08
1045 8.5123799378195e-08
1046 8.53687092439159e-08
1047 8.5044914032828e-08
1048 8.50076586464965e-08
1049 8.52451779502772e-08
1050 8.48750091355654e-08
1051 8.56494420693466e-08
1052 8.47313297853702e-08
1053 8.48737466316862e-08
1054 8.48297829376676e-08
1055 8.47870691602637e-08
1056 8.47822186278435e-08
1057 8.48583667503178e-08
1058 8.45606657140507e-08
1059 8.47129197207863e-08
1060 8.4697597535488e-08
1061 8.43913717636369e-08
1062 8.46177675617454e-08
1063 8.45461904779299e-08
1064 8.47051659169651e-08
1065 8.42047847688932e-08
1066 8.44023262747129e-08
1067 8.43362500200229e-08
1068 8.49483326890166e-08
1069 8.3950508028785e-08
1070 8.42248024346404e-08
1071 8.50871447681811e-08
1072 8.3878910861479e-08
1073 8.41959061217423e-08
1074 8.39289905849583e-08
1075 8.43144122217865e-08
1076 8.38659067703418e-08
1077 8.4106835688047e-08
1078 8.37550871208848e-08
1079 8.39844167161985e-08
1080 8.37406287814701e-08
1081 8.37698303985945e-08
1082 8.37709798657826e-08
1083 8.3903732329027e-08
1084 8.35617001362365e-08
1085 8.37050087678648e-08
1086 8.35877431031307e-08
1087 8.37369917299213e-08
1088 8.3357153783048e-08
1089 8.43050460606065e-08
1090 8.30978412782457e-08
1091 8.38647112519908e-08
1092 8.31714864597544e-08
1093 8.32949147993034e-08
1094 8.32593565034401e-08
1095 8.30645639213756e-08
1096 8.33216964579719e-08
1097 8.3148903480712e-08
1098 8.31827725624734e-08
1099 8.29591668178864e-08
1100 8.39684854057054e-08
1101 8.26235190984903e-08
1102 8.28562873245176e-08
1103 8.38977687687148e-08
1104 8.24743131522254e-08
1105 8.27496879898248e-08
1106 8.2805424585608e-08
1107 8.28893177500234e-08
1108 8.27688653455283e-08
1109 8.33410025720127e-08
1110 8.22490398251219e-08
1111 8.27426672502796e-08
1112 8.2388451545512e-08
1113 8.27258766227246e-08
1114 8.24107335892776e-08
1115 8.25264073283094e-08
1116 8.24218042385638e-08
1117 8.31251032531455e-08
1118 8.20978627462665e-08
1119 8.26056992240254e-08
1120 8.22308962584728e-08
1121 8.21160650286146e-08
1122 8.24935175955943e-08
1123 8.20039066189437e-08
1124 8.2165055455441e-08
1125 8.2910598144359e-08
1126 8.16930081155576e-08
1127 8.20894863826638e-08
1128 8.24820000842052e-08
1129 8.17007379119161e-08
1130 8.22695373754456e-08
1131 8.18077433129005e-08
1132 8.18843391616753e-08
1133 8.20153543585889e-08
1134 8.16643794374627e-08
1135 8.15767902748732e-08
1136 8.19928412578719e-08
1137 8.15534767051318e-08
1138 8.17363489744594e-08
1139 8.16068630467015e-08
1140 8.14721741031477e-08
1141 8.16808398349522e-08
1142 8.14376871165479e-08
1143 8.12299829906493e-08
1144 8.17084782127608e-08
1145 8.17646174553488e-08
1146 8.12396369713753e-08
1147 8.1312071272599e-08
1148 8.14062421410355e-08
1149 8.19767902733304e-08
1150 8.10132721316137e-08
1151 8.13543382687243e-08
1152 8.20673160459862e-08
1153 8.08623493782079e-08
1154 8.1252581044744e-08
1155 8.10534367490234e-08
1156 8.09889082278303e-08
1157 8.12016504267987e-08
1158 8.08957114388775e-08
1159 8.08936717051623e-08
1160 8.1157090702888e-08
1161 8.0888745496388e-08
1162 8.07479271953682e-08
1163 8.10392150780714e-08
1164 8.15300589138701e-08
1165 8.02556237875507e-08
1166 8.06888316473575e-08
1167 8.03990957312095e-08
1168 8.14760524132652e-08
1169 8.0213218529579e-08
1170 8.03496166037476e-08
1171 8.06875503123194e-08
1172 8.04822685545048e-08
1173 8.03236791551143e-08
1174 8.05412942055384e-08
1175 8.03581865369551e-08
1176 8.04150945645787e-08
1177 8.02037710263548e-08
1178 8.03552639423444e-08
1179 8.03835786289397e-08
1180 8.01572095454617e-08
1181 8.04134829524017e-08
1182 8.01879969412056e-08
1183 8.01448828902096e-08
1184 8.01971877342211e-08
1185 8.05424965175661e-08
1186 7.97453379055568e-08
1187 7.98986106396171e-08
1188 7.99923515470979e-08
1189 8.00068367992068e-08
1190 7.97398926906112e-08
1191 8.01150767237857e-08
1192 7.98100216661268e-08
1193 7.97565121830601e-08
1194 7.96559080846748e-08
1195 7.9921713426856e-08
1196 7.97004320602923e-08
1197 7.96013532688988e-08
1198 7.9608503820161e-08
1199 7.96963992701905e-08
1200 8.05072426262043e-08
1201 7.91948166130041e-08
1202 7.92320846141337e-08
1203 7.95350457751809e-08
1204 7.93028357071179e-08
1205 7.96830264997439e-08
1206 8.00013984969539e-08
1207 7.9275045696825e-08
1208 7.94015383558388e-08
1209 7.94138813322576e-08
1210 7.99350129989662e-08
1211 7.89530409619488e-08
1212 7.9318068829437e-08
1213 7.98033827766886e-08
1214 7.9069672187515e-08
1215 7.92786345957808e-08
1216 7.90990571051253e-08
1217 7.93818763504817e-08
1218 7.89210506750848e-08
1219 7.89571435984016e-08
1220 7.90651871200865e-08
1221 7.88094255472771e-08
1222 7.89111855636904e-08
1223 7.88273552458918e-08
1224 7.91314978698665e-08
1225 7.87080360353798e-08
1226 7.87190684041761e-08
1227 7.87446169248085e-08
1228 7.85650735073418e-08
1229 7.87202433061296e-08
1230 7.84942132927569e-08
1231 7.87761215397253e-08
1232 7.86545292639218e-08
1233 7.87529791086428e-08
1234 7.82190471362298e-08
1235 7.9562116077625e-08
1236 7.79941225887981e-08
1237 7.82056983332779e-08
1238 7.8345130304136e-08
1239 7.82765514975026e-08
1240 7.83693469319502e-08
1241 7.84311712500596e-08
1242 7.84382878427081e-08
1243 7.79878600392081e-08
1244 7.81572416093113e-08
1245 7.90482322416963e-08
1246 7.78657875128985e-08
1247 7.78081372798667e-08
1248 7.81626624393184e-08
1249 7.78137496650899e-08
1250 7.82687462486109e-08
1251 7.85707323505136e-08
1252 7.77064220729429e-08
1253 7.77963736027232e-08
1254 7.78489187744924e-08
1255 7.77659569912714e-08
1256 7.87390592451587e-08
1257 7.75055060389107e-08
1258 7.77049781968131e-08
1259 7.78392619293911e-08
1260 7.78806858852121e-08
1261 7.758743926356e-08
1262 7.77693472535645e-08
1263 7.84616933726667e-08
1264 7.71723365771137e-08
1265 7.74675424164428e-08
1266 7.7572097954004e-08
1267 7.73169815984076e-08
1268 7.74867966519821e-08
1269 7.74130765774927e-08
1270 7.74753341739043e-08
1271 7.81509773970512e-08
1272 7.70711109865374e-08
1273 7.7096726397663e-08
1274 7.72752909812269e-08
1275 7.7272611174628e-08
1276 7.71895430267122e-08
1277 7.71111806470159e-08
1278 7.78809999903984e-08
1279 7.66866136876487e-08
1280 7.70621339238886e-08
1281 7.70218910677301e-08
1282 7.72043570229641e-08
1283 7.67738111182581e-08
1284 7.78136517265438e-08
1285 7.66646896819267e-08
1286 7.68121333454275e-08
1287 7.67779600288065e-08
1288 7.68152179597692e-08
1289 7.68678377074394e-08
1290 7.67205684502059e-08
1291 7.68197898111112e-08
1292 7.66925853650235e-08
1293 7.66890059358261e-08
1294 7.6604573060024e-08
1295 7.76114026246333e-08
1296 7.61874864325662e-08
1297 7.63776731185217e-08
1298 7.67081750119303e-08
1299 7.6399858377485e-08
1300 7.64984977106309e-08
1301 7.66835569940838e-08
1302 7.64349948827459e-08
1303 7.64207749375245e-08
1304 7.62401597151907e-08
1305 7.63400152292704e-08
1306 7.61760055008054e-08
1307 7.65753015050663e-08
1308 7.62551882722562e-08
1309 7.62379338814512e-08
1310 7.60696973882702e-08
1311 7.63995642936166e-08
1312 7.59360387672103e-08
1313 7.7112949256275e-08
1314 7.56542103301783e-08
1315 7.58363098762871e-08
1316 7.59947845088149e-08
1317 7.59851414908752e-08
1318 7.59771612495186e-08
1319 7.59388758009294e-08
1320 7.59095512208319e-08
1321 7.58388338342186e-08
1322 7.57883586874897e-08
1323 7.5881258813304e-08
1324 7.58456116143336e-08
1325 7.57775113404691e-08
1326 7.56998309610069e-08
1327 7.57570046401312e-08
1328 7.57251141481063e-08
1329 7.55156227452147e-08
1330 7.56458399333582e-08
1331 7.63387105218172e-08
1332 7.53349723856545e-08
1333 7.54451169244774e-08
1334 7.54484256599852e-08
1335 7.54818742914054e-08
1336 7.54510025657851e-08
1337 7.54046271742581e-08
1338 7.53081794897525e-08
1339 7.55385965272026e-08
1340 7.61230360382115e-08
1341 7.49855810839861e-08
1342 7.51563029606217e-08
1343 7.51846957331992e-08
1344 7.53123975263392e-08
1345 7.52695348360533e-08
1346 7.51597703407825e-08
1347 7.51228101263379e-08
1348 7.51514508987583e-08
1349 7.50877861595001e-08
1350 7.50191106453357e-08
1351 7.50536545686131e-08
1352 7.50198703851623e-08
1353 7.48844903188939e-08
1354 7.50965279081583e-08
1355 7.48130132688374e-08
1356 7.49396552031101e-08
1357 7.47683453967696e-08
1358 7.48334526186056e-08
1359 7.48248556394771e-08
1360 7.47823187561281e-08
1361 7.47017727249499e-08
1362 7.47021835039163e-08
1363 7.47863982262231e-08
1364 7.46275046932254e-08
1365 7.45596721296593e-08
1366 7.47178714446406e-08
1367 7.45448883767708e-08
1368 7.45814905727116e-08
1369 7.44911380499147e-08
1370 7.45748276020919e-08
1371 7.43935230635984e-08
1372 7.44159439109993e-08
1373 7.46289477611128e-08
1374 7.43038454311673e-08
1375 7.4331110487158e-08
1376 7.43846922643954e-08
1377 7.42828954907182e-08
1378 7.42632397150444e-08
1379 7.44265800376454e-08
1380 7.42111712686011e-08
1381 7.50527007422619e-08
1382 7.37786453806066e-08
1383 7.40069659910247e-08
1384 7.40912771419744e-08
1385 7.41163925717103e-08
1386 7.40308603104367e-08
1387 7.40665696872256e-08
1388 7.40484681864828e-08
1389 7.39614015499157e-08
1390 7.3978719979273e-08
1391 7.39225828070289e-08
1392 7.4752195030392e-08
1393 7.35775487346046e-08
1394 7.37560083390321e-08
1395 7.39758830912152e-08
1396 7.3681227582334e-08
1397 7.38191027380708e-08
1398 7.36742152209757e-08
1399 7.38082598035206e-08
1400 7.36706234159001e-08
1401 7.36346558740664e-08
1402 7.35913066813865e-08
1403 7.36593110950423e-08
1404 7.35482745986005e-08
1405 7.36083665104914e-08
1406 7.35287906543647e-08
1407 7.35402387181949e-08
1408 7.34356364757716e-08
1409 7.35145872141629e-08
1410 7.34421397856622e-08
1411 7.33560165064873e-08
1412 7.33565729085228e-08
1413 7.33694635055571e-08
1414 7.34083733284763e-08
1415 7.33094151392777e-08
1416 7.31674927436643e-08
1417 7.33672792740947e-08
1418 7.31564961080622e-08
1419 7.33781299135927e-08
1420 7.31535656965932e-08
1421 7.33582848875258e-08
1422 7.30505490746936e-08
1423 7.31938027893619e-08
1424 7.31256369075339e-08
1425 7.30152165102638e-08
1426 7.30318226862536e-08
1427 7.31176609232165e-08
1428 7.30954545078433e-08
1429 7.30930813093167e-08
1430 7.28488759369483e-08
1431 7.29791587676942e-08
1432 7.28573596529003e-08
1433 7.264784463068e-08
1434 7.26638359997978e-08
1435 7.28008401305047e-08
1436 7.25234073000181e-08
1437 7.25680570115017e-08
1438 7.24540708576526e-08
1439 7.2523187952811e-08
1440 7.23662961039651e-08
1441 7.22055408735045e-08
1442 7.23616164632901e-08
1443 7.22188476727226e-08
1444 7.21089308584055e-08
1445 7.22306916607351e-08
1446 7.21392196583182e-08
1447 7.21394103031514e-08
1448 7.20302630865532e-08
1449 7.20093121584497e-08
1450 7.19537828048899e-08
1451 7.20207943940565e-08
1452 7.2026534033931e-08
1453 7.18637583760895e-08
1454 7.20035073715763e-08
1455 7.18677254880617e-08
1456 7.19189192759373e-08
1457 7.17487297219321e-08
1458 7.18390470586527e-08
1459 7.18775323811371e-08
1460 7.17485111616512e-08
1461 7.16854133298028e-08
1462 7.17404637464725e-08
1463 7.17644687462382e-08
1464 7.16685689035046e-08
1465 7.15782904956441e-08
1466 7.1592653108965e-08
1467 7.1507286190986e-08
1468 7.15471559287195e-08
1469 7.15075558597178e-08
1470 7.14703297459707e-08
1471 7.13886486689575e-08
1472 7.14149801765984e-08
1473 7.13544214629991e-08
1474 7.13654131185848e-08
1475 7.13720820098018e-08
1476 7.13411588870727e-08
1477 7.11936620048093e-08
1478 7.12694110669432e-08
1479 7.11708170264558e-08
1480 7.12644830995757e-08
1481 7.11322809374693e-08
1482 7.12611572684096e-08
1483 7.10193736628284e-08
1484 7.10820766602893e-08
1485 7.10444830716028e-08
1486 7.1101500593862e-08
1487 7.09789471944333e-08
1488 7.09954976629845e-08
1489 7.08782766158933e-08
1490 7.09873195798849e-08
1491 7.08079472779843e-08
1492 7.09073315912079e-08
1493 7.08107347620057e-08
1494 7.08628730254901e-08
1495 7.08095653489949e-08
1496 7.07813107778676e-08
1497 7.06560758318986e-08
1498 7.07196366285245e-08
1499 7.07029454121155e-08
1500 7.06731165136532e-08
1501 7.06711249724989e-08
1502 7.05966773297817e-08
1503 7.06881939747106e-08
1504 7.0448110354171e-08
1505 7.05405507366308e-08
1506 7.04995881912751e-08
1507 7.06071240861306e-08
1508 7.03468787790484e-08
1509 7.0472213511863e-08
1510 7.02755005947964e-08
1511 7.03288641643596e-08
1512 7.02311121534294e-08
1513 7.02492599398141e-08
1514 7.01637023592383e-08
1515 7.00250242244849e-08
1516 7.01416177975034e-08
1517 7.00852399937446e-08
1518 7.00804807758359e-08
1519 7.0016982587795e-08
1520 6.99566669970508e-08
1521 7.00993001725436e-08
1522 6.97948531334092e-08
1523 6.98745942262491e-08
1524 6.98554759246406e-08
1525 6.99245045447938e-08
1526 6.98534322109978e-08
1527 6.97516737311332e-08
1528 6.97206436370834e-08
1529 6.98258099127713e-08
1530 6.97172706924931e-08
1531 6.96418353571815e-08
1532 6.97219503136282e-08
1533 6.95428610830717e-08
1534 6.96764629699587e-08
1535 6.95599026077076e-08
1536 6.95224713016529e-08
1537 6.95651401985131e-08
1538 6.95259398026948e-08
1539 6.94967195586926e-08
1540 6.94161921455105e-08
1541 6.94294588052813e-08
1542 6.94684529927159e-08
1543 6.93989178595089e-08
1544 6.9760545080122e-08
1545 6.91024035432974e-08
1546 6.91627678293116e-08
1547 6.91889967292525e-08
1548 6.92227498806375e-08
1549 6.91929246778855e-08
1550 6.91551095561849e-08
1551 6.91124053115999e-08
1552 6.90028514576113e-08
1553 6.91326499584477e-08
1554 6.89497719195131e-08
1555 6.90107826759601e-08
1556 6.89889907841135e-08
1557 6.88996610467285e-08
1558 6.88920608329369e-08
1559 6.89032263867517e-08
1560 6.88675741606914e-08
1561 6.92587599449013e-08
1562 6.85526972485917e-08
1563 6.86351039567157e-08
1564 6.88100230989974e-08
1565 6.86476607842934e-08
1566 6.86978157746765e-08
1567 6.8629363838113e-08
1568 6.8589052163226e-08
1569 6.85649459963855e-08
1570 6.85919597867013e-08
1571 6.85553406976069e-08
1572 6.84994175603748e-08
1573 6.84625530933403e-08
1574 6.84005569029367e-08
1575 6.84266919339649e-08
1576 6.84240950956649e-08
1577 6.82878129580899e-08
1578 6.83836407633365e-08
1579 6.82747997249322e-08
1580 6.83091804809521e-08
1581 6.8191913029203e-08
1582 6.8121731956694e-08
1583 6.83452836724996e-08
1584 6.8194792052445e-08
1585 6.82605432311689e-08
1586 6.81403999136521e-08
1587 6.80700552200264e-08
1588 6.81613530959524e-08
1589 6.80778156540995e-08
1590 6.80074672514408e-08
1591 6.7978940055724e-08
1592 6.80060168507524e-08
1593 6.78606399500126e-08
1594 6.79265654186523e-08
1595 6.79506135883301e-08
1596 6.78349663214561e-08
1597 6.78385662116199e-08
1598 6.77955069923186e-08
1599 6.77865954008539e-08
1600 6.7770792056443e-08
1601 6.76293917498683e-08
1602 6.7706673864798e-08
1603 6.76320894861249e-08
1604 6.76388331699584e-08
1605 6.763050867864e-08
1606 6.75249214951989e-08
1607 6.75641475185174e-08
1608 6.75436126984508e-08
1609 6.74066337378321e-08
1610 6.74670116165288e-08
1611 6.74149815909075e-08
1612 6.74480441587022e-08
1613 6.73365179180152e-08
1614 6.73494056870894e-08
1615 6.72352283039856e-08
1616 6.73804575486159e-08
1617 6.72232134366624e-08
1618 6.72357079762875e-08
1619 6.71899791360175e-08
1620 6.71852148421692e-08
1621 6.71644616838662e-08
1622 6.69852333716747e-08
1623 6.72406888835297e-08
1624 6.69686875722775e-08
1625 6.69899492118375e-08
1626 6.70044856221352e-08
1627 6.69218548621231e-08
1628 6.69450438088859e-08
1629 6.68679447777265e-08
1630 6.69045918773037e-08
1631 6.6800378208498e-08
1632 6.68338731326656e-08
1633 6.6742075225612e-08
1634 6.67118663599808e-08
1635 6.67766119262225e-08
1636 6.66380333891325e-08
1637 6.66501886152915e-08
1638 6.67202218611607e-08
1639 6.65399338597439e-08
1640 6.66009176173432e-08
1641 6.65921920868229e-08
1642 6.649873682818e-08
1643 6.64372789005441e-08
1644 6.65420206091838e-08
1645 6.63867320005451e-08
1646 6.64393945601915e-08
1647 6.63531423166219e-08
1648 6.6430909104298e-08
1649 6.62617994491299e-08
1650 6.63021967151067e-08
1651 6.6250145834168e-08
1652 6.62842754222126e-08
1653 6.61366961294263e-08
1654 6.6207311553157e-08
1655 6.60983172808827e-08
1656 6.62094011190106e-08
1657 6.60480817735731e-08
1658 6.60743685267917e-08
1659 6.59900062940366e-08
1660 6.60955446694089e-08
1661 6.58934151189783e-08
1662 6.60293435430503e-08
1663 6.58872733945159e-08
1664 6.5920634093608e-08
1665 6.58348494653183e-08
1666 6.59045070436903e-08
1667 6.5905970259017e-08
1668 6.57887394437395e-08
1669 6.57956169147766e-08
1670 6.56686834670239e-08
1671 6.58121577545856e-08
1672 6.56209342508873e-08
1673 6.56908763190245e-08
1674 6.55370594948579e-08
1675 6.56949170831922e-08
1676 6.55387060319868e-08
1677 6.55675618581597e-08
1678 6.54437681424369e-08
1679 6.55740097004198e-08
1680 6.54526207961581e-08
1681 6.54140824956073e-08
1682 6.55489819436639e-08
1683 6.52682717188213e-08
1684 6.53196817097523e-08
1685 6.53814242816964e-08
1686 6.52414742763341e-08
1687 6.5411434515994e-08
1688 6.51583409370104e-08
1689 6.5239008354645e-08
1690 6.51325006923997e-08
1691 6.51661370687506e-08
1692 6.50875663366079e-08
1693 6.51190456109063e-08
1694 6.5133018353869e-08
1695 6.5008494101626e-08
1696 6.50202547181067e-08
1697 6.49746860101885e-08
1698 6.49266126497849e-08
1699 6.50105515829225e-08
1700 6.48802880816035e-08
1701 6.48511228806825e-08
1702 6.48538129137677e-08
1703 6.48332329538093e-08
1704 6.48712667503659e-08
1705 6.47110408040774e-08
1706 6.48171622943394e-08
1707 6.47351979905508e-08
1708 6.47797859381072e-08
1709 6.46294351138721e-08
1710 6.47044033694755e-08
1711 6.46849352525791e-08
1712 6.44992343712403e-08
1713 6.4649390079552e-08
1714 6.45244326307193e-08
1715 6.45140475619144e-08
1716 6.45674424077924e-08
1717 6.43960369384189e-08
1718 6.43940577305102e-08
1719 6.4330633026799e-08
1720 6.44459142975506e-08
1721 6.43110130988944e-08
1722 6.43778073197154e-08
1723 6.41911646006577e-08
1724 6.43366173429527e-08
1725 6.42146634701746e-08
1726 6.40920862275962e-08
1727 6.43055092153233e-08
1728 6.40913080145467e-08
1729 6.41845943647468e-08
1730 6.41208115057523e-08
1731 6.40527797424184e-08
1732 6.40128671918205e-08
1733 6.39793726122662e-08
1734 6.40349099700899e-08
1735 6.39761040410747e-08
1736 6.39882384829704e-08
1737 6.38491663487173e-08
1738 6.39527005086293e-08
1739 6.37759483845812e-08
1740 6.39378699016646e-08
1741 6.37427815304292e-08
1742 6.3813321064643e-08
1743 6.37622602113197e-08
1744 6.38299178392643e-08
1745 6.35943528930483e-08
1746 6.3737886429216e-08
1747 6.35936368214018e-08
1748 6.37100415108449e-08
1749 6.35657066494488e-08
1750 6.35920878933049e-08
1751 6.35527683323644e-08
1752 6.35173891980756e-08
1753 6.35801695647586e-08
1754 6.34358769495691e-08
1755 6.34987322252201e-08
1756 6.34144255879932e-08
1757 6.34455510404308e-08
1758 6.33761971062796e-08
1759 6.33563158540795e-08
1760 6.32775735178725e-08
1761 6.33047089033667e-08
1762 6.33456652447961e-08
1763 6.32383548335724e-08
1764 6.32763733365849e-08
1765 6.32533597624274e-08
1766 6.31233252343932e-08
1767 6.31938994883896e-08
1768 6.31131956749797e-08
1769 6.31733941647283e-08
1770 6.31122363259351e-08
1771 6.30468106406212e-08
1772 6.31026056829853e-08
1773 6.30371743719493e-08
1774 6.30347246071139e-08
1775 6.29776218756817e-08
1776 6.29589805800634e-08
1777 6.2933611751248e-08
1778 6.29252203907527e-08
1779 6.28799471602548e-08
1780 6.28239670970032e-08
1781 6.28710495096385e-08
1782 6.27869358051925e-08
1783 6.27721206791776e-08
1784 6.27963443031732e-08
1785 6.27924409526415e-08
1786 6.27216672617337e-08
1787 6.26931416283227e-08
1788 6.28251637149191e-08
1789 6.26032366612961e-08
1790 6.26036779349803e-08
1791 6.25333844821441e-08
1792 6.2548331989376e-08
1793 6.25194036842913e-08
1794 6.24895534597769e-08
1795 6.24835945961522e-08
1796 6.24374780642256e-08
1797 6.24161755427011e-08
1798 6.231820397673e-08
1799 6.23666005683532e-08
1800 6.2373538610494e-08
1801 6.23153591297054e-08
1802 6.23048180035113e-08
1803 6.21733745074238e-08
1804 6.23994213846402e-08
1805 6.20908594344272e-08
1806 6.21966104237259e-08
1807 6.22378884518326e-08
1808 6.27910978838742e-08
1809 6.21740043573737e-08
1810 6.2213847914272e-08
1811 6.23153481047467e-08
1812 6.21221348380274e-08
1813 6.21936168361259e-08
1814 6.21704614811591e-08
1815 6.20685427241696e-08
1816 6.2010124715961e-08
1817 6.19778814510141e-08
1818 6.19705923643465e-08
1819 6.18386550970129e-08
1820 6.18480367213792e-08
1821 6.16000904978264e-08
1822 6.15143383289052e-08
1823 6.14274271839577e-08
1824 6.12069697973894e-08
1825 6.09467506560435e-08
1826 6.04917871429578e-08
1827 6.00436183262332e-08
1828 5.95536764729232e-08
1829 5.89130583463771e-08
1830 5.83866044356895e-08
1831 5.78158924593453e-08
1832 5.72539441447972e-08
1833 5.6627194079617e-08
1834 5.59923545377927e-08
1835 5.53529577027589e-08
1836 5.47810535254101e-08
1837 5.42216816112884e-08
1838 5.36111145317264e-08
1839 5.31092144280265e-08
1840 5.25996017355368e-08
1841 5.2199741417347e-08
1842 5.18038509502716e-08
1843 5.15231307947062e-08
1844 5.13261941499366e-08
1845 5.09403542947595e-08
1846 5.06408547948922e-08
1847 5.0415335828724e-08
1848 5.00610340772312e-08
1849 4.96014174258264e-08
1850 4.96563631369185e-08
1851 4.92838426078279e-08
1852 4.94302213009945e-08
1853 4.89144904873839e-08
1854 4.89146289535114e-08
1855 4.89295439392023e-08
1856 4.88796396682645e-08
1857 4.84082021046106e-08
1858 4.83138832345631e-08
1859 4.8314131284144e-08
1860 4.8141235505561e-08
1861 4.79943633857616e-08
1862 4.80239460900478e-08
1863 4.77722204923481e-08
1864 4.78002216466322e-08
1865 4.76137126774745e-08
1866 4.75367001868676e-08
1867 4.74193127155687e-08
1868 4.75897562974836e-08
1869 4.73180661524353e-08
1870 4.70666924101693e-08
1871 4.72191338136696e-08
1872 4.70293705143732e-08
1873 4.68314798478531e-08
1874 4.67693321768436e-08
1875 4.67364035294793e-08
1876 4.66028650336625e-08
1877 4.67879290635409e-08
1878 4.63618129211696e-08
1879 4.63016202516542e-08
1880 4.65957613355883e-08
1881 4.60409337490475e-08
1882 4.60071324095423e-08
1883 4.60161728783604e-08
1884 4.58478815428265e-08
1885 4.61244592644405e-08
1886 4.55379044304038e-08
1887 4.5587664399882e-08
1888 4.57286901056619e-08
1889 4.53241989850639e-08
1890 4.51882122636249e-08
1891 4.51640279424126e-08
1892 4.50682144199632e-08
1893 4.51756957167149e-08
1894 4.47464069415204e-08
1895 4.47065300672733e-08
1896 4.47826548217733e-08
1897 4.4650522790235e-08
1898 4.45734720644353e-08
1899 4.44097212088934e-08
1900 4.41689168839687e-08
1901 4.42232259780795e-08
1902 4.43053758605316e-08
1903 4.40154964813644e-08
1904 4.38882245736849e-08
1905 4.38060538385798e-08
1906 4.39137290166869e-08
1907 4.35896587314133e-08
1908 4.35467526775923e-08
1909 4.34280380616769e-08
1910 4.34753777307506e-08
1911 4.34482459690244e-08
1912 4.32475223997031e-08
1913 4.30479581741139e-08
1914 4.32371447489643e-08
1915 4.30238199324862e-08
1916 4.30845507004562e-08
1917 4.26851212278478e-08
1918 4.27782621326145e-08
1919 4.24242177405887e-08
1920 4.25007486848372e-08
1921 4.2234751926884e-08
1922 4.23026907796498e-08
1923 4.21695600154592e-08
1924 4.20047532561085e-08
1925 4.20411541535159e-08
1926 4.18446958541097e-08
1927 4.1749051662876e-08
1928 4.19665229438593e-08
1929 4.15849715480476e-08
1930 4.17689248655151e-08
1931 4.13344067853316e-08
1932 4.14164270265971e-08
1933 4.11761727532323e-08
1934 4.13451532974562e-08
1935 4.09707126181047e-08
1936 4.08890355378944e-08
1937 4.11107628046636e-08
1938 4.05845390161197e-08
1939 4.06948759597014e-08
1940 4.03925800211979e-08
1941 4.08616592779687e-08
1942 4.0052873412133e-08
1943 4.0658051188025e-08
1944 3.99692938639618e-08
1945 4.01382260144523e-08
1946 3.98115966024903e-08
1947 3.9753296046996e-08
1948 3.99782886519873e-08
1949 3.93176448039156e-08
1950 3.94851746943203e-08
1951 3.95920657618731e-08
1952 3.92798976811548e-08
1953 3.89937783475958e-08
1954 3.95159988735294e-08
1955 3.8645796308856e-08
1956 3.87569681841171e-08
1957 3.86296527628005e-08
1958 3.84786033040818e-08
1959 3.8459952577341e-08
1960 3.83337239444614e-08
1961 3.82107775589446e-08
1962 3.82042149675854e-08
1963 3.79713909381962e-08
1964 3.80738229104338e-08
1965 3.76926801157751e-08
1966 3.76489846032335e-08
1967 3.75521578264149e-08
1968 3.73910364657171e-08
1969 3.72591263815281e-08
1970 3.71436655779434e-08
1971 3.7031652270425e-08
1972 3.69139630640447e-08
1973 3.6830560497414e-08
1974 3.67635923579357e-08
1975 3.6633674253661e-08
1976 3.65825165231648e-08
1977 3.65729766458323e-08
1978 3.63541472969686e-08
1979 3.63596434231894e-08
1980 3.61751091926266e-08
1981 3.61635906411806e-08
1982 3.59174801198847e-08
1983 3.59450962799102e-08
1984 3.56732720696584e-08
1985 3.57291127826009e-08
1986 3.54707251011277e-08
1987 3.5514036925921e-08
1988 3.55112631376109e-08
1989 3.52649123476212e-08
1990 3.52391370732619e-08
1991 3.5300873990618e-08
1992 3.49902925198542e-08
1993 3.50471570329525e-08
1994 3.50692866182278e-08
1995 3.4878120677817e-08
1996 3.49310336864583e-08
1997 3.47090286525464e-08
1998 3.47295736737863e-08
1999 3.45349312165233e-08
2000 3.49087895061828e-08
2001 3.42982433796379e-08
2002 3.46364176446734e-08
2003 3.43561580558038e-08
2004 3.42172195257895e-08
2005 3.43708388617081e-08
2006 3.42635856629414e-08
2007 3.4092982234224e-08
2008 3.49101367023152e-08
2009 3.38813832727602e-08
2010 3.41003760739866e-08
2011 3.40987360387679e-08
2012 3.38088504432399e-08
2013 3.39202763730562e-08
2014 3.39596713425649e-08
2015 3.38593463276027e-08
2016 3.38368011383317e-08
2017 3.39671758311511e-08
2018 3.35613568047677e-08
2019 3.40953675981304e-08
2020 3.35051689699384e-08
2021 3.35585889619239e-08
2022 3.40004463548205e-08
2023 3.33356738293489e-08
2024 3.37488797610419e-08
2025 3.38277737985671e-08
2026 3.32643338381011e-08
2027 3.35498501606857e-08
2028 3.33210013669927e-08
2029 3.38525896377284e-08
2030 3.30753095418856e-08
2031 3.32735627441849e-08
2032 3.3660717894346e-08
2033 3.3156027345882e-08
2034 3.31537473656773e-08
2035 3.4042997528072e-08
2036 3.28766930168811e-08
2037 3.29272914849632e-08
2038 3.31647163025472e-08
2039 3.32464783214803e-08
2040 3.31538803317599e-08
2041 3.30260262448867e-08
2042 3.28494046990535e-08
2043 3.3063661324384e-08
2044 3.32732993464369e-08
2045 3.26168734474663e-08
2046 3.28665558164687e-08
2047 3.28207568167649e-08
2048 3.27884260800104e-08
2049 3.27552257846087e-08
2050 3.28179868644973e-08
2051 3.26313714751336e-08
2052 3.34856591472921e-08
2053 3.23762131850813e-08
2054 3.25144084483142e-08
2055 3.35217277012134e-08
2056 3.23241541186903e-08
2057 3.32676689023259e-08
2058 3.23610237522587e-08
2059 3.23391214727131e-08
2060 3.25973357000464e-08
2061 3.32359996080278e-08
2062 3.21424428428863e-08
2063 3.23673343949338e-08
2064 3.34668118244785e-08
2065 3.21062770085589e-08
2066 3.21721613643078e-08
2067 3.25791690887201e-08
2068 3.21507931086984e-08
2069 3.31664263604203e-08
2070 3.22859359211947e-08
2071 3.21411882833189e-08
2072 3.23922045404323e-08
2073 3.22206906844258e-08
2074 3.20514918192849e-08
2075 3.23011505782844e-08
2076 3.20864171095003e-08
2077 3.22580639275927e-08
2078 3.2086810222598e-08
2079 3.20634564237032e-08
2080 3.19610862198338e-08
2081 3.22033795350229e-08
2082 3.19048221024687e-08
2083 3.21536776288767e-08
2084 3.19085250968243e-08
2085 3.18496068270058e-08
2086 3.21651061190309e-08
2087 3.20497237766837e-08
2088 3.19282993435088e-08
2089 3.1726073741023e-08
2090 3.20690252393341e-08
2091 3.16860815310349e-08
2092 3.17763218746769e-08
2093 3.17179660331845e-08
2094 3.17456706189212e-08
2095 3.17195176209317e-08
2096 3.16458645475137e-08
2097 3.18919761577163e-08
2098 3.15698311381674e-08
2099 3.15907591219577e-08
2100 3.1669030761794e-08
2101 3.19755549975653e-08
2102 3.13346998548347e-08
2103 3.17951494501756e-08
2104 3.15101795291817e-08
2105 3.17055179843706e-08
2106 3.1322156084812e-08
2107 3.14252994955133e-08
2108 3.1485869949055e-08
2109 3.14892383541654e-08
2110 3.14443914475326e-08
2111 3.13218774938839e-08
2112 3.12466746619222e-08
2113 3.1448173297477e-08
2114 3.12171457386157e-08
2115 3.13089575052494e-08
2116 3.12366155141852e-08
2117 3.23671456348151e-08
2118 3.14858973671228e-08
2119 3.14875744882315e-08
2120 3.1330610241298e-08
2121 3.10090151725007e-08
2122 3.1048084457197e-08
2123 3.11588473178581e-08
2124 3.12176640036022e-08
2125 3.11555114183015e-08
2126 3.12124975088146e-08
2127 3.08158422206439e-08
2128 3.22483652608341e-08
2129 3.11480104238093e-08
2130 3.12107189288646e-08
2131 3.07778952945981e-08
2132 3.06868100334867e-08
2133 3.10001463375009e-08
2134 3.09315214748196e-08
2135 3.11937218486769e-08
2136 3.08992993276469e-08
2137 3.07689874095018e-08
2138 3.14765082145563e-08
2139 3.15280612075419e-08
2140 3.08936005857241e-08
2141 3.06330484693973e-08
2142 3.17320244489494e-08
2143 3.08140472671248e-08
2144 3.11464258091476e-08
2145 3.05451901332354e-08
2146 3.053777231532e-08
2147 3.12437541754029e-08
2148 3.13418208675031e-08
2149 3.09880434543608e-08
2150 3.03890003161555e-08
2151 3.15984894370125e-08
2152 3.05601476209283e-08
2153 3.10489304879979e-08
2154 3.03960495791067e-08
2155 3.04388520895316e-08
2156 3.16897232028879e-08
2157 3.04335962457714e-08
2158 3.03451319969561e-08
2159 3.02736036030815e-08
2160 3.05683438623916e-08
2161 3.070967989105e-08
2162 3.05802997186966e-08
2163 3.03812683473126e-08
2164 3.04588737201072e-08
2165 3.04364030681015e-08
2166 3.04421209378347e-08
2167 3.02101703741897e-08
2168 3.0233863622442e-08
2169 3.0844055145085e-08
2170 2.99996651293988e-08
2171 3.038547538603e-08
2172 3.02665421449966e-08
2173 3.02130777889431e-08
2174 3.03469497806574e-08
2175 3.01694681437148e-08
2176 3.02469104314085e-08
2177 3.00164068827158e-08
2178 3.03679631872278e-08
2179 3.01478056194959e-08
2180 2.99660519047507e-08
2181 3.03030594794862e-08
2182 2.99714969234088e-08
2183 3.01635420960622e-08
2184 3.02705731090036e-08
2185 2.99227252447132e-08
2186 2.98792199924236e-08
2187 3.00964937380854e-08
2188 3.00923457752234e-08
2189 2.97376987261622e-08
2190 2.9952560597124e-08
2191 2.99418946925201e-08
2192 3.00477855077474e-08
2193 2.96745133745446e-08
2194 3.02045017819275e-08
2195 2.98634183182323e-08
2196 2.99278352171584e-08
2197 2.98885648759573e-08
2198 2.97995157603204e-08
2199 2.97329935090573e-08
2200 2.99029071197943e-08
2201 2.97776289612806e-08
2202 3.00259438725803e-08
2203 2.96195465931959e-08
2204 3.0601809459796e-08
2205 2.98453062423931e-08
2206 2.99903779983346e-08
2207 2.93215340003705e-08
2208 2.98462808157041e-08
2209 2.98463362615742e-08
2210 2.93483551310203e-08
2211 3.02216682430689e-08
2212 2.93109718279005e-08
2213 2.95214400125587e-08
2214 2.96405693576141e-08
2215 3.02943054970584e-08
2216 2.9527258015527e-08
2217 2.94079854268325e-08
2218 2.95197093675981e-08
2219 2.96845847955751e-08
2220 2.94470693789073e-08
2221 2.97356445191355e-08
2222 2.92569038582435e-08
2223 2.93853538204658e-08
2224 2.97150017272152e-08
2225 2.95819374795947e-08
2226 2.92663827456963e-08
2227 2.97789413741256e-08
2228 2.91562535665868e-08
2229 2.96140329099082e-08
2230 2.93627850305178e-08
2231 2.92226987155253e-08
2232 3.0362192443345e-08
2233 2.89278793310643e-08
2234 2.93388987873122e-08
2235 2.94254402017913e-08
2236 2.92461176751857e-08
2237 2.94186520797268e-08
2238 2.89696461943656e-08
2239 2.95158199619117e-08
2240 2.88378369943665e-08
2241 2.91451831149203e-08
2242 3.0171881682417e-08
2243 2.88045076355381e-08
2244 2.93283443619785e-08
2245 2.91903769471524e-08
2246 2.91336719233648e-08
2247 2.91333116968495e-08
2248 2.89869600234027e-08
2249 2.94660976725147e-08
2250 2.99135562662478e-08
2251 2.9515639246469e-08
2252 2.91642231127209e-08
2253 2.92598474374017e-08
2254 2.86347130393771e-08
2255 2.95803800343108e-08
2256 2.89639982318235e-08
2257 2.96637286196777e-08
2258 2.89916557476566e-08
2259 2.92879006584812e-08
2260 2.93828886497316e-08
2261 2.87069767559434e-08
2262 2.85596672331412e-08
2263 2.90551520008009e-08
2264 2.88957296183945e-08
2265 2.89381397164945e-08
2266 2.8850575394479e-08
2267 2.89353122946423e-08
2268 2.91747831058231e-08
2269 2.96702101927337e-08
2270 2.84413854374499e-08
2271 2.89164246822793e-08
2272 2.86266880979991e-08
2273 2.92950736313635e-08
2274 2.86309680612717e-08
2275 2.94593600327353e-08
2276 2.85122521850312e-08
2277 2.88053757482309e-08
2278 2.86064950070219e-08
2279 2.87581213491705e-08
2280 2.87175924031935e-08
2281 2.94105206810613e-08
2282 2.8384453887309e-08
2283 2.84969231252319e-08
2284 2.87320607923469e-08
2285 2.95443604989565e-08
2286 2.83199726966998e-08
2287 2.86129106508159e-08
2288 2.87835404475523e-08
2289 2.87139755936749e-08
2290 2.86063444474571e-08
2291 2.87116908044283e-08
2292 2.86271757663492e-08
2293 2.96273228137522e-08
2294 2.839260804266e-08
2295 2.82334894989855e-08
2296 2.8426096617018e-08
2297 2.85548652190215e-08
2298 2.84847166946989e-08
2299 2.93942343225773e-08
2300 2.82073462114507e-08
2301 2.84495034801324e-08
2302 2.8567194678697e-08
2303 2.91618274799177e-08
2304 2.81480966703995e-08
2305 2.90658364066054e-08
2306 2.80593274737395e-08
2307 2.82814031891121e-08
2308 2.85934937926946e-08
2309 2.8309693679951e-08
2310 2.85319160724384e-08
2311 2.92677170499189e-08
2312 2.8237801629194e-08
2313 2.87877627136446e-08
2314 2.85774992612176e-08
2315 2.81039489173907e-08
2316 2.86022866435864e-08
2317 2.89396807398035e-08
2318 2.85446743690798e-08
2319 2.79668999332294e-08
2320 2.88452914150028e-08
2321 2.80128163943072e-08
2322 2.81357957407913e-08
2323 2.83652494839082e-08
2324 2.85164419135953e-08
2325 2.82326733342941e-08
2326 2.80898255984319e-08
2327 2.88944229627219e-08
2328 2.8023831252888e-08
2329 2.81464100342355e-08
2330 2.82313922195243e-08
2331 2.84162951291744e-08
2332 2.8884322253564e-08
2333 2.79383335903827e-08
2334 2.82399930882082e-08
2335 2.83515905439291e-08
2336 2.8459859361174e-08
2337 2.80379589780999e-08
2338 2.79882095277628e-08
2339 2.83421292799346e-08
2340 2.85243345468089e-08
2341 2.78433162983482e-08
2342 2.82432968945479e-08
2343 2.8140773280505e-08
2344 2.85127395671658e-08
2345 2.86980357080413e-08
2346 2.77838367976901e-08
2347 2.77920458984227e-08
2348 2.80137639476763e-08
2349 2.812178900502e-08
2350 2.79573531725141e-08
2351 2.80295726255986e-08
2352 2.80054751660153e-08
2353 2.80607432676661e-08
2354 2.89421704651183e-08
2355 2.79159306368904e-08
2356 2.79063565593596e-08
2357 2.7769078748241e-08
2358 2.7997546563796e-08
2359 2.86269027607311e-08
2360 2.81675379882884e-08
2361 2.79254627297831e-08
2362 2.80110817190149e-08
2363 2.75772251230322e-08
2364 2.8117060738353e-08
2365 2.85897366940624e-08
2366 2.76200756159639e-08
2367 2.7565508176508e-08
2368 2.79965146900985e-08
2369 2.76090659996342e-08
2370 2.76973041453132e-08
2371 2.77334598650647e-08
2372 2.77591952877465e-08
2373 2.83483324512446e-08
2374 2.73940956962981e-08
2375 2.79184756459827e-08
2376 2.76140380783829e-08
2377 2.77866664419868e-08
2378 2.75413185444329e-08
2379 2.77547174625248e-08
2380 2.82193739011838e-08
2381 2.73854679586627e-08
2382 2.76134057917154e-08
2383 2.7638053588408e-08
2384 2.76723898675613e-08
2385 2.75689518334232e-08
2386 2.76374498171528e-08
2387 2.76448404958884e-08
2388 2.74395298234253e-08
2389 2.80906652716517e-08
2390 2.74580597410257e-08
2391 2.76284808657934e-08
2392 2.74474952957071e-08
2393 2.75572117898193e-08
2394 2.78239062243557e-08
2395 2.74689749959478e-08
2396 2.74819488592648e-08
2397 2.80951268156748e-08
2398 2.76901932909635e-08
2399 2.77418967398191e-08
2400 2.73278586808701e-08
2401 2.7956138368701e-08
2402 2.72424639926605e-08
2403 2.7370746066957e-08
2404 2.75714780908487e-08
2405 2.74777452129449e-08
2406 2.73400377293065e-08
2407 2.78795328516601e-08
2408 2.72611426339608e-08
2409 2.80859742289863e-08
2410 2.71295412122008e-08
2411 2.77726943203049e-08
2412 2.72152250055946e-08
2413 2.73494301246124e-08
2414 2.74431836686517e-08
2415 2.82291683313396e-08
2416 2.72650970425392e-08
2417 2.74226920431442e-08
2418 2.73991752157698e-08
2419 2.72763602029435e-08
2420 2.77891263449348e-08
2421 2.71305649510722e-08
2422 2.72596912873624e-08
2423 2.74488888836188e-08
2424 2.7796492474863e-08
2425 2.78572347123252e-08
2426 2.75466775685462e-08
2427 2.70846999215202e-08
2428 2.70691489747676e-08
2429 2.79464907357863e-08
2430 2.70508036872563e-08
2431 2.76148728133307e-08
2432 2.71488011178711e-08
2433 2.71158559668194e-08
2434 2.73382133282318e-08
2435 2.81085815236892e-08
2436 2.69277211413232e-08
2437 2.71164038192495e-08
2438 2.72313469169561e-08
2439 2.71125887492119e-08
2440 2.79957981472734e-08
2441 2.69161363073422e-08
2442 2.70855517423563e-08
2443 2.73352301758401e-08
2444 2.71028631519954e-08
2445 2.74185735373855e-08
2446 2.70138968403444e-08
2447 2.74597789275965e-08
2448 2.71668171647477e-08
2449 2.75255086457094e-08
2450 2.69907239749401e-08
2451 2.70615365143279e-08
2452 2.7180624117662e-08
2453 2.72050831902071e-08
2454 2.72649803534364e-08
2455 2.70611475245985e-08
2456 2.76440596596039e-08
2457 2.6984850806322e-08
2458 2.75109095331239e-08
2459 2.69998464492716e-08
2460 2.68927038087341e-08
2461 2.71217691287795e-08
2462 2.71331103181005e-08
2463 2.71344770226278e-08
2464 2.69946790214526e-08
2465 2.75791873241005e-08
2466 2.69381815540815e-08
2467 2.74802916253591e-08
2468 2.68814349730917e-08
2469 2.69460226518525e-08
2470 2.7120296685279e-08
2471 2.69419382639136e-08
2472 2.70166209177525e-08
2473 2.74592712590227e-08
2474 2.69757358495326e-08
2475 2.68772847209497e-08
2476 2.77444001506311e-08
2477 2.67483118365774e-08
2478 2.69062418642996e-08
2479 2.70837560369941e-08
2480 2.68197845398888e-08
2481 2.71829811371305e-08
2482 2.68176911730755e-08
2483 2.69898854212691e-08
2484 2.71148816142208e-08
2485 2.68047026295015e-08
2486 2.69729184507739e-08
2487 2.68610902756095e-08
2488 2.68922764714574e-08
2489 2.68319689533669e-08
2490 2.69938887331733e-08
2491 2.70064999243758e-08
2492 2.73542912845315e-08
2493 2.66743437449257e-08
2494 2.72082398478091e-08
2495 2.65715452179016e-08
2496 2.68495262978341e-08
2497 2.67509815161659e-08
2498 2.69750776507038e-08
2499 2.73680608025018e-08
2500 2.67126018425046e-08
2501 2.66386877538771e-08
2502 2.67841904579846e-08
2503 2.6811628578649e-08
2504 2.69245251931505e-08
2505 2.69393785909955e-08
2506 2.6803463227143e-08
2507 2.66559693224977e-08
2508 2.74893055016712e-08
2509 2.72002556029882e-08
2510 2.65942741797254e-08
2511 2.67795499153678e-08
2512 2.72180000924216e-08
2513 2.65107394428199e-08
2514 2.68110494987539e-08
2515 2.66347619692908e-08
2516 2.68261398193026e-08
2517 2.66819356378178e-08
2518 2.68935848510843e-08
2519 2.66833987003778e-08
2520 2.66848653995844e-08
2521 2.67534145641513e-08
2522 2.70381284606103e-08
2523 2.66228603811314e-08
2524 2.68068817308453e-08
2525 2.69813272848474e-08
2526 2.67822797450634e-08
2527 2.65162282033771e-08
2528 2.66416379210987e-08
2529 2.73462173598915e-08
2530 2.64353077135127e-08
2531 2.65403951220655e-08
2532 2.66663981234316e-08
2533 2.66322881459224e-08
2534 2.65899221543187e-08
2535 2.71830177407395e-08
2536 2.64120513064903e-08
2537 2.66311890331217e-08
2538 2.65764825635451e-08
2539 2.66191450419839e-08
2540 2.66816705631889e-08
2541 2.68296861283268e-08
2542 2.65224683286824e-08
2543 2.65484441221986e-08
2544 2.68140247090543e-08
2545 2.65602433215939e-08
2546 2.64377562886331e-08
2547 2.70519528471347e-08
2548 2.62550129188632e-08
2549 2.68612688674175e-08
2550 2.64124205675564e-08
2551 2.6543152967351e-08
2552 2.65259421801112e-08
2553 2.64798473024186e-08
2554 2.69719576730942e-08
2555 2.63107587428202e-08
2556 2.67338131285655e-08
2557 2.6375730053374e-08
2558 2.65395585312689e-08
2559 2.63231388366592e-08
2560 2.65165954203006e-08
2561 2.63422312385408e-08
2562 2.65123479104012e-08
2563 2.62917710247557e-08
2564 2.62727835353971e-08
2565 2.63869650631854e-08
2566 2.67973449776804e-08
2567 2.64355888548451e-08
2568 2.61373104195606e-08
2569 2.63517393661594e-08
2570 2.63222409802033e-08
2571 2.62530373058567e-08
2572 2.63263181063955e-08
2573 2.62920801605837e-08
2574 2.6328458601288e-08
2575 2.68630017981053e-08
2576 2.59710810071034e-08
2577 2.66928796781585e-08
2578 2.61673252510697e-08
2579 2.61293057675083e-08
2580 2.61632914750898e-08
2581 2.62022593391364e-08
2582 2.65139292427996e-08
2583 2.61432674983908e-08
2584 2.62803358861419e-08
2585 2.60127245006458e-08
2586 2.65168630035895e-08
2587 2.59819601002853e-08
2588 2.64124952149558e-08
2589 2.59671771978276e-08
2590 2.62145183942408e-08
2591 2.63301630210044e-08
2592 2.59665583355329e-08
2593 2.62601349194647e-08
2594 2.60686458699588e-08
2595 2.61888307679925e-08
2596 2.59845488357335e-08
2597 2.60600048545001e-08
2598 2.60848467759089e-08
2599 2.65126467442478e-08
2600 2.59163848648214e-08
2601 2.6077201578989e-08
2602 2.59442151020473e-08
2603 2.6102383470672e-08
2604 2.63994959257552e-08
2605 2.62266362356378e-08
2606 2.59111613436502e-08
2607 2.62020965937637e-08
2608 2.62737131380142e-08
2609 2.63225085763708e-08
2610 2.56953631128098e-08
2611 2.59793551684595e-08
2612 2.62267141244443e-08
2613 2.59231137760274e-08
2614 2.59436121456957e-08
2615 2.61567934769502e-08
2616 2.57742792819648e-08
2617 2.63621547329507e-08
2618 2.56933753104427e-08
2619 2.63319280593421e-08
2620 2.59610451580627e-08
2621 2.62042962471121e-08
2622 2.59847404660007e-08
2623 2.61074086997937e-08
2624 2.59114710421393e-08
2625 2.60266969736023e-08
2626 2.59858348692354e-08
2627 2.63608730075582e-08
2628 2.5769726069047e-08
2629 2.6212759179689e-08
2630 2.56994845724279e-08
2631 2.597846550767e-08
2632 2.57519854125476e-08
2633 2.57782206833923e-08
2634 2.63061070167048e-08
2635 2.57484254100326e-08
2636 2.60500951450027e-08
2637 2.56437709795776e-08
2638 2.60804181881902e-08
2639 2.60710707613576e-08
2640 2.57459307340113e-08
2641 2.61110713188906e-08
2642 2.56755797054531e-08
2643 2.57505324996465e-08
2644 2.58074598655789e-08
2645 2.59160960491833e-08
2646 2.59189350155786e-08
2647 2.61674104673482e-08
2648 2.58256321750139e-08
2649 2.56397247047069e-08
2650 2.5704526992687e-08
2651 2.57540844676996e-08
2652 2.5790278239235e-08
2653 2.5725139074062e-08
2654 2.63891187679555e-08
2655 2.55966996733914e-08
2656 2.58588222190603e-08
2657 2.57261409024601e-08
2658 2.58565150739187e-08
2659 2.58221412021697e-08
2660 2.57274335537794e-08
2661 2.56125731339374e-08
2662 2.57302693067452e-08
2663 2.58775349895046e-08
2664 2.59623747487225e-08
2665 2.54911931540924e-08
2666 2.57938456105222e-08
2667 2.55949217118356e-08
2668 2.57276274897578e-08
2669 2.56438447854279e-08
2670 2.60183128193425e-08
2671 2.5774565987291e-08
2672 2.56118395878246e-08
2673 2.55683934708983e-08
2674 2.56910991716097e-08
2675 2.62333267597725e-08
2676 2.53748971328349e-08
2677 2.55446684187532e-08
2678 2.55848406194303e-08
2679 2.61647986175007e-08
2680 2.57218008936633e-08
2681 2.54298874462489e-08
2682 2.57381243256027e-08
2683 2.55662888042352e-08
2684 2.57494795321556e-08
2685 2.5899363742532e-08
2686 2.56918855949806e-08
2687 2.53515586683406e-08
2688 2.56698948177814e-08
2689 2.59064015384336e-08
2690 2.61975327795483e-08
2691 2.54118753910681e-08
2692 2.55367396537753e-08
2693 2.56461814842623e-08
2694 2.56093902795129e-08
2695 2.54108621531302e-08
2696 2.57318619887315e-08
2697 2.54518351323618e-08
2698 2.60111497301096e-08
2699 2.55816922181129e-08
2700 2.54028196073897e-08
2701 2.56321213512045e-08
2702 2.57232202689472e-08
2703 2.53515524342163e-08
2704 2.54930813645693e-08
2705 2.55962189164016e-08
2706 2.57576125708425e-08
2707 2.54752007893266e-08
2708 2.5358361400496e-08
2709 2.54749260100162e-08
2710 2.59398750634965e-08
2711 2.52508574649646e-08
2712 2.56499514699193e-08
2713 2.53352064216728e-08
2714 2.56167652077366e-08
2715 2.53982548943377e-08
2716 2.53906982408125e-08
2717 2.58660421736945e-08
2718 2.53654590560703e-08
2719 2.53809806844973e-08
2720 2.55594150373462e-08
2721 2.5483156577355e-08
2722 2.59408824412333e-08
2723 2.51780052047401e-08
2724 2.5469256108579e-08
2725 2.52896152277415e-08
2726 2.54689218595061e-08
2727 2.56872410382414e-08
2728 2.51561739519346e-08
2729 2.54012471629927e-08
2730 2.57297664476575e-08
2731 2.52016131869759e-08
2732 2.54524186624749e-08
2733 2.58531929575589e-08
2734 2.53831230634383e-08
2735 2.51418046022245e-08
2736 2.58280702309932e-08
2737 2.52168841292644e-08
2738 2.51882719659591e-08
2739 2.54185429495735e-08
2740 2.56863434606736e-08
2741 2.52714802750909e-08
2742 2.53446453437611e-08
2743 2.5588741525695e-08
2744 2.51551727989963e-08
2745 2.5252718866664e-08
2746 2.53842429902562e-08
2747 2.53388339626071e-08
2748 2.56165208716297e-08
2749 2.54565775725801e-08
2750 2.54533050787487e-08
2751 2.50507817218804e-08
2752 2.56415599819704e-08
2753 2.53476414777687e-08
2754 2.51847851160392e-08
2755 2.52464485508419e-08
2756 2.52443655921475e-08
2757 2.56448009046117e-08
2758 2.51301368039414e-08
2759 2.52551915500554e-08
2760 2.56800031861459e-08
2761 2.50005131636755e-08
2762 2.53238877174766e-08
2763 2.52673909364454e-08
2764 2.56221634860054e-08
2765 2.50471441236577e-08
2766 2.51141467866312e-08
2767 2.56239736189379e-08
2768 2.49872522468664e-08
2769 2.56452020082065e-08
2770 2.49717143105954e-08
2771 2.53529564298027e-08
2772 2.51145695977506e-08
2773 2.54266915171719e-08
2774 2.51093201284469e-08
2775 2.55013481478272e-08
2776 2.50134727735052e-08
2777 2.51812539975305e-08
2778 2.51966147879479e-08
2779 2.52004513749959e-08
2780 2.51247460427706e-08
2781 2.52237863191951e-08
2782 2.54631965237451e-08
2783 2.49615934211356e-08
2784 2.5390045454099e-08
2785 2.49306906083824e-08
2786 2.51976792879915e-08
2787 2.52767394153253e-08
2788 2.535368678247e-08
2789 2.52261541784371e-08
2790 2.53989489149564e-08
2791 2.51692601680453e-08
2792 2.52187363751766e-08
2793 2.50720977272856e-08
2794 2.49703768662179e-08
2795 2.58219505142598e-08
2796 2.51321344530986e-08
2797 2.50586028118782e-08
2798 2.5070673520311e-08
2799 2.52551264945389e-08
2800 2.49931574844986e-08
2801 2.53992521255242e-08
2802 2.51488978784131e-08
2803 2.51257408003802e-08
2804 2.50015254290581e-08
2805 2.50974961999439e-08
2806 2.50733567352945e-08
2807 2.53515858026354e-08
2808 2.52261495869988e-08
2809 2.48646101699279e-08
2810 2.52707068582048e-08
2811 2.48892635279496e-08
2812 2.49869400303915e-08
2813 2.51155172761308e-08
2814 2.53244562817834e-08
2815 2.50663686918173e-08
2816 2.52156523128377e-08
2817 2.51570168514625e-08
2818 2.48329210350562e-08
2819 2.51814549292462e-08
2820 2.49372681531135e-08
2821 2.49508128167264e-08
2822 2.50137651849336e-08
2823 2.50930225584867e-08
2824 2.51585045054625e-08
2825 2.5232010629761e-08
2826 2.48201437398343e-08
2827 2.50269828119265e-08
2828 2.50610785372807e-08
2829 2.50660338558806e-08
2830 2.52610270008446e-08
2831 2.47971940505742e-08
2832 2.49734130042256e-08
2833 2.49134090477554e-08
2834 2.54640069177192e-08
2835 2.48479197468843e-08
2836 2.48329811278758e-08
2837 2.49284885627254e-08
2838 2.53551246918171e-08
2839 2.52366174824381e-08
2840 2.48029625216084e-08
2841 2.4764732835969e-08
2842 2.52864032346256e-08
2843 2.47670946138534e-08
2844 2.49305679504985e-08
2845 2.51564818356531e-08
2846 2.48578128203558e-08
2847 2.48396595208256e-08
2848 2.49437186976564e-08
2849 2.52644731966178e-08
2850 2.478458545907e-08
2851 2.48559918087921e-08
2852 2.48453726956477e-08
2853 2.48560995608216e-08
2854 2.48802829498906e-08
2855 2.49158207128453e-08
2856 2.48515725527199e-08
2857 2.49767607645168e-08
2858 2.48252889196721e-08
2859 2.49042273687827e-08
2860 2.50049393786256e-08
2861 2.48849512356486e-08
2862 2.50276152407025e-08
2863 2.4757241808615e-08
2864 2.48468566090843e-08
2865 2.47031344646764e-08
2866 2.51061100522421e-08
2867 2.53049034988351e-08
2868 2.46570223569886e-08
2869 2.47118020348935e-08
2870 2.48449240927151e-08
2871 2.49984611664367e-08
2872 2.47781903661704e-08
2873 2.49563644989514e-08
2874 2.48251681016498e-08
2875 2.53020303984108e-08
2876 2.46218793904518e-08
2877 2.50739230049923e-08
2878 2.45618449850049e-08
2879 2.51080400777148e-08
2880 2.48495737866783e-08
2881 2.47143122416027e-08
2882 2.46826921799226e-08
2883 2.47687053294143e-08
2884 2.47544936669009e-08
2885 2.46936875183401e-08
2886 2.47125708963125e-08
2887 2.46368659460039e-08
2888 2.51904265107239e-08
2889 2.44741638719681e-08
2890 2.48743983912725e-08
2891 2.45995718515246e-08
2892 2.46027974508678e-08
2893 2.52113545720078e-08
2894 2.44391479840012e-08
2895 2.49129448266494e-08
2896 2.45670855472113e-08
2897 2.46784022208679e-08
2898 2.47011944884967e-08
2899 2.4627356741469e-08
2900 2.47704530411674e-08
2901 2.45264093394759e-08
2902 2.47790553720151e-08
2903 2.44746041317967e-08
2904 2.45865901065123e-08
2905 2.47359890304111e-08
2906 2.45536587237805e-08
2907 2.45456546996703e-08
2908 2.48637596262924e-08
2909 2.45229532720437e-08
2910 2.48882130486905e-08
2911 2.4415285084789e-08
2912 2.45437345067767e-08
2913 2.46684144975529e-08
2914 2.43778162092312e-08
2915 2.44505710240706e-08
2916 2.49105728116206e-08
2917 2.48633860244762e-08
2918 2.43303481908264e-08
2919 2.44254518002585e-08
2920 2.44942850007313e-08
2921 2.44784139300513e-08
2922 2.48072009121358e-08
2923 2.43321101560667e-08
2924 2.49083655416982e-08
2925 2.44623037208846e-08
2926 2.48746778410691e-08
2927 2.42687582114875e-08
2928 2.43271868236583e-08
2929 2.44749790625498e-08
2930 2.4397933388709e-08
2931 2.48054902516337e-08
2932 2.46046053122839e-08
2933 2.44536479740987e-08
2934 2.42328472190856e-08
2935 2.44783174312424e-08
2936 2.47181530768081e-08
2937 2.42097489846493e-08
2938 2.44877706498947e-08
2939 2.44216730918012e-08
2940 2.43933009627106e-08
2941 2.46856102756876e-08
2942 2.42241387014008e-08
2943 2.4469454810383e-08
2944 2.44306956873608e-08
2945 2.44289435080702e-08
2946 2.43550036573836e-08
2947 2.4356553947058e-08
2948 2.43436874227676e-08
2949 2.46984438043718e-08
2950 2.43288082015791e-08
2951 2.42679767730181e-08
2952 2.47126022743238e-08
2953 2.43818983407529e-08
2954 2.42552586580125e-08
2955 2.42677788930834e-08
2956 2.43836136162301e-08
2957 2.46139979425131e-08
2958 2.4282598866332e-08
2959 2.44697369278235e-08
2960 2.46809650157331e-08
2961 2.43575320912903e-08
2962 2.41687700159865e-08
2963 2.42934710410481e-08
2964 2.42864819464117e-08
2965 2.42692090863805e-08
2966 2.47515097746565e-08
2967 2.41183030795788e-08
2968 2.43837795994573e-08
2969 2.4182988966448e-08
2970 2.42417798506089e-08
2971 2.43105692350198e-08
2972 2.43895854645793e-08
2973 2.42240050734033e-08
2974 2.44938398568095e-08
2975 2.4608235674517e-08
2976 2.40680967409368e-08
2977 2.44415777648221e-08
2978 2.45808772807443e-08
2979 2.41053580394279e-08
2980 2.43436055145096e-08
2981 2.4695615543191e-08
2982 2.41389910349277e-08
2983 2.4167136179809e-08
2984 2.49568564150149e-08
2985 2.40258557941964e-08
2986 2.4150062521322e-08
2987 2.42306845339435e-08
2988 2.42799931702287e-08
2989 2.4202522761918e-08
2990 2.43669843387195e-08
2991 2.41227526012189e-08
2992 2.42623951356435e-08
2993 2.44685517429843e-08
2994 2.43098810672748e-08
2995 2.41337045872037e-08
2996 2.42257184921435e-08
2997 2.42077767498294e-08
2998 2.46117240139121e-08
2999 2.41399218854355e-08
3000 2.43526604895017e-08
3001 2.42162659538359e-08
3002 2.41048410392075e-08
3003 2.45125967937021e-08
3004 2.40437084788603e-08
3005 2.42677388797574e-08
3006 2.43563389212831e-08
3007 2.40617483329153e-08
3008 2.43046177046402e-08
3009 2.4268569213115e-08
3010 2.44446939627885e-08
3011 2.39821124115025e-08
3012 2.4043677394392e-08
3013 2.40876461199413e-08
3014 2.41758821666327e-08
3015 2.41565149985412e-08
3016 2.44657069705667e-08
3017 2.39691081174165e-08
3018 2.43417877525687e-08
3019 2.40802964874831e-08
3020 2.4109452170018e-08
3021 2.43363780376704e-08
3022 2.39757826498987e-08
3023 2.41106614318198e-08
3024 2.40699302027636e-08
3025 2.46867287252428e-08
3026 2.39413598261962e-08
3027 2.40501654600322e-08
3028 2.42279551043723e-08
3029 2.44790534253969e-08
3030 2.39130284098721e-08
3031 2.39902980112561e-08
3032 2.41854850280276e-08
3033 2.40003806006861e-08
3034 2.41277918524574e-08
3035 2.40387643306583e-08
3036 2.44484837814696e-08
3037 2.40300797846871e-08
3038 2.39367012055958e-08
3039 2.39992585120419e-08
3040 2.41683663908532e-08
3041 2.40351649507531e-08
3042 2.40168260865481e-08
3043 2.42091201543282e-08
3044 2.40370626429787e-08
3045 2.4087053524191e-08
3046 2.41167373538076e-08
3047 2.39818482232756e-08
3048 2.41217541083749e-08
3049 2.42073704224133e-08
3050 2.40537641360561e-08
3051 2.39204163956508e-08
3052 2.43834171493873e-08
3053 2.39024520021225e-08
3054 2.40093299055388e-08
3055 2.40492417957761e-08
3056 2.39488807887156e-08
3057 2.40933474602301e-08
3058 2.41233166380361e-08
3059 2.43009380935799e-08
3060 2.38862057684308e-08
3061 2.41338530000412e-08
3062 2.42700358437098e-08
3063 2.38360919269098e-08
3064 2.41130889500063e-08
3065 2.41123122202147e-08
3066 2.44429027151938e-08
3067 2.38771004585914e-08
3068 2.37903582473997e-08
3069 2.40150600796518e-08
3070 2.40680321557107e-08
3071 2.43567680082712e-08
3072 2.37700260199425e-08
3073 2.38207833365145e-08
3074 2.39757982996025e-08
3075 2.43313472574336e-08
3076 2.3817907671031e-08
3077 2.40884345896752e-08
3078 2.3781936578704e-08
3079 2.38553000384556e-08
3080 2.40737603434127e-08
3081 2.40185555209216e-08
3082 2.3969058190243e-08
3083 2.38596027331006e-08
3084 2.38590084125079e-08
3085 2.39882322814466e-08
3086 2.3819880339282e-08
3087 2.40020388888595e-08
3088 2.38539574448637e-08
3089 2.40241638875816e-08
3090 2.37709988244461e-08
3091 2.38943672057346e-08
3092 2.4048350633965e-08
3093 2.37505281752703e-08
3094 2.38071078975111e-08
3095 2.38806270691505e-08
3096 2.38855181282638e-08
3097 2.39529027590279e-08
3098 2.39537974440207e-08
3099 2.37525169453079e-08
3100 2.40054315034932e-08
3101 2.38315973515313e-08
3102 2.37577024320146e-08
3103 2.3858489000439e-08
3104 2.37344242384196e-08
3105 2.38699006525067e-08
3106 2.39131953421179e-08
3107 2.3669164028739e-08
3108 2.39864819397972e-08
3109 2.37346548632722e-08
3110 2.37749152329769e-08
3111 2.37347728582193e-08
3112 2.38065433282308e-08
3113 2.37586032003634e-08
3114 2.39051715453797e-08
3115 2.37598241987769e-08
3116 2.40135098299454e-08
3117 2.39434201891875e-08
3118 2.36665652799672e-08
3119 2.39025728769882e-08
3120 2.37733312065114e-08
3121 2.37023244373802e-08
3122 2.38795563527461e-08
3123 2.36937853377306e-08
3124 2.37257735191676e-08
3125 2.37712044754979e-08
3126 2.40608501238526e-08
3127 2.3626038095248e-08
3128 2.36931362658233e-08
3129 2.39731652857955e-08
3130 2.37717928759373e-08
3131 2.36734276373163e-08
3132 2.38184435117361e-08
3133 2.37109546152858e-08
3134 2.3720684417583e-08
3135 2.36557693034101e-08
3136 2.3858515473929e-08
3137 2.36300389895305e-08
3138 2.37248130563472e-08
3139 2.36497164243232e-08
3140 2.37936469962285e-08
3141 2.38066286795124e-08
3142 2.37598290993013e-08
3143 2.36309031764748e-08
3144 2.38200704711922e-08
3145 2.37082555458734e-08
3146 2.36496334600211e-08
3147 2.36829529942639e-08
3148 2.37590475093974e-08
3149 2.36561528614843e-08
3150 2.38104554890484e-08
3151 2.36814034844102e-08
3152 2.36828115247612e-08
3153 2.37078734715013e-08
3154 2.36648235973913e-08
3155 2.36465125178498e-08
3156 2.37811647485486e-08
3157 2.38054593615367e-08
3158 2.35434329849404e-08
3159 2.36043738457425e-08
3160 2.36422454156227e-08
3161 2.36314708459417e-08
3162 2.37856857228103e-08
3163 2.36443002306075e-08
3164 2.36570594576158e-08
3165 2.35848335559119e-08
3166 2.36636361732323e-08
3167 2.36263875112996e-08
3168 2.36817450369742e-08
3169 2.37022944773457e-08
3170 2.36406617144524e-08
3171 2.36415328078721e-08
3172 2.3637525131015e-08
3173 2.35497147347274e-08
3174 2.36891591240962e-08
3175 2.35443249168021e-08
3176 2.36055478670671e-08
3177 2.39471239940059e-08
3178 2.34375242431639e-08
3179 2.36714718964137e-08
3180 2.37161290153054e-08
3181 2.35126517820206e-08
3182 2.35977380427244e-08
3183 2.37405412306657e-08
3184 2.3688339540584e-08
3185 2.35274490760773e-08
3186 2.35523892491152e-08
3187 2.37599764210117e-08
3188 2.36304461607162e-08
3189 2.36156521631337e-08
3190 2.35025349368811e-08
3191 2.36189844011747e-08
3192 2.36952673731139e-08
3193 2.35078297095015e-08
3194 2.34953362112655e-08
3195 2.35321731287819e-08
3196 2.35747708607903e-08
3197 2.37279464334605e-08
3198 2.34616670486965e-08
3199 2.39208506731536e-08
3200 2.33680067767139e-08
3201 2.35322995094656e-08
3202 2.35694805872377e-08
3203 2.36182021224884e-08
3204 2.34892864354741e-08
3205 2.35631538569869e-08
3206 2.34864312831373e-08
3207 2.38799218963415e-08
3208 2.33773054030273e-08
3209 2.36311885339902e-08
3210 2.34358789183986e-08
3211 2.35977510656404e-08
3212 2.38254423874373e-08
3213 2.3347109022076e-08
3214 2.36155552570949e-08
3215 2.34106608110185e-08
3216 2.36727848297313e-08
3217 2.3384592344744e-08
3218 2.35910017054586e-08
3219 2.35063500433341e-08
3220 2.35173729983273e-08
3221 2.34610663172319e-08
3222 2.36819266659083e-08
3223 2.34341310507702e-08
3224 2.36855267268243e-08
3225 2.33560128628518e-08
3226 2.35632116236673e-08
3227 2.34843248061445e-08
3228 2.34185618444016e-08
3229 2.36080837008323e-08
3230 2.35182289625158e-08
3231 2.34431070644803e-08
3232 2.3449025591038e-08
3233 2.34707512394294e-08
3234 2.33874077144591e-08
3235 2.34788713275869e-08
3236 2.36044021670878e-08
3237 2.35373909163172e-08
3238 2.33400134472816e-08
3239 2.36082478739519e-08
3240 2.34464066852791e-08
3241 2.33984905766427e-08
3242 2.33520644823404e-08
3243 2.39438519247237e-08
3244 2.33180094926233e-08
3245 2.33008589285433e-08
3246 2.35359944040781e-08
3247 2.34911270386995e-08
3248 2.34307660533784e-08
3249 2.33513040281963e-08
3250 2.33384076484544e-08
3251 2.35943273669825e-08
3252 2.3499927534143e-08
3253 2.3441331439944e-08
3254 2.34121360178641e-08
3255 2.36482064948085e-08
3256 2.35184824215473e-08
3257 2.32445300496487e-08
3258 2.33663689683805e-08
3259 2.34195712316421e-08
3260 2.35209411496573e-08
3261 2.33589692140157e-08
3262 2.34028584094048e-08
3263 2.33398499647208e-08
3264 2.34107126622085e-08
3265 2.35069314395986e-08
3266 2.33592751157641e-08
3267 2.33291402316205e-08
3268 2.33492348558073e-08
3269 2.37039649593207e-08
3270 2.33254173713782e-08
3271 2.32979372403186e-08
3272 2.36440198950749e-08
3273 2.33952298662743e-08
3274 2.33490108709766e-08
3275 2.34356935622237e-08
3276 2.3368811495672e-08
3277 2.35302339883781e-08
3278 2.33425220130812e-08
3279 2.32990208322015e-08
3280 2.32198123928384e-08
3281 2.34423462566191e-08
3282 2.33642567804004e-08
3283 2.36168211447119e-08
3284 2.3323066889791e-08
3285 2.32693297839148e-08
3286 2.33144509502736e-08
3287 2.34203551388035e-08
3288 2.34806281835809e-08
3289 2.34013313196169e-08
3290 2.32835667508535e-08
3291 2.33776819622555e-08
3292 2.34180628357894e-08
3293 2.33859093183852e-08
3294 2.32978099434789e-08
3295 2.33840542971375e-08
3296 2.35146654030682e-08
3297 2.32301031817261e-08
3298 2.32546675249523e-08
3299 2.33414660506526e-08
3300 2.32789437482062e-08
3301 2.33406283567383e-08
3302 2.34608169962236e-08
3303 2.32348763513635e-08
3304 2.33585592979146e-08
3305 2.32455096362827e-08
3306 2.32354236477939e-08
3307 2.35410904245725e-08
3308 2.32574776974914e-08
3309 2.32326373499347e-08
3310 2.3305868805279e-08
3311 2.34322862002756e-08
3312 2.31775272636625e-08
3313 2.33516838983316e-08
3314 2.32980773435809e-08
3315 2.32104764976704e-08
3316 2.33254499677482e-08
3317 2.35761140938706e-08
3318 2.31823125200759e-08
3319 2.31957935916682e-08
3320 2.32750074666832e-08
3321 2.3418364258454e-08
3322 2.3212300470643e-08
3323 2.32824551762434e-08
3324 2.32832024691376e-08
3325 2.35094237235334e-08
3326 2.31457973529636e-08
3327 2.3353141466842e-08
3328 2.34375886867255e-08
3329 2.31778239836444e-08
3330 2.32141972875688e-08
3331 2.32530078103643e-08
3332 2.31471119032101e-08
3333 2.3550160704211e-08
3334 2.31770961001132e-08
3335 2.32871507721555e-08
3336 2.32130086135207e-08
3337 2.32848539933883e-08
3338 2.32731836593558e-08
3339 2.31767653544601e-08
3340 2.32620565840946e-08
3341 2.3289651741365e-08
3342 2.32092600955802e-08
3343 2.31916660533571e-08
3344 2.34169415587182e-08
3345 2.31776558430319e-08
3346 2.31653180766678e-08
3347 2.32295603550536e-08
3348 2.3279468242432e-08
3349 2.31947203785943e-08
3350 2.33227593477991e-08
3351 2.31376446491183e-08
3352 2.33015132060643e-08
3353 2.32282545886697e-08
3354 2.30665398253116e-08
3355 2.32909835673478e-08
3356 2.32519126353026e-08
3357 2.31700671766699e-08
3358 2.31733711530957e-08
3359 2.33564564868782e-08
3360 2.31476306100653e-08
3361 2.33004184324592e-08
3362 2.31543814752655e-08
3363 2.31315175289737e-08
3364 2.32605187155954e-08
3365 2.34823936673401e-08
3366 2.31318026955307e-08
3367 2.3312922204477e-08
3368 2.31312841108e-08
3369 2.31070586016457e-08
3370 2.32059629996506e-08
3371 2.31450508851871e-08
3372 2.32912529085638e-08
3373 2.31053131978243e-08
3374 2.32491997331774e-08
3375 2.31422947631899e-08
3376 2.3107608501105e-08
3377 2.31894210687145e-08
3378 2.32729127298548e-08
3379 2.31183210572805e-08
3380 2.33406914951217e-08
3381 2.32819719596655e-08
3382 2.2995664991754e-08
3383 2.31313766809738e-08
3384 2.33016993518298e-08
3385 2.32112905376169e-08
3386 2.33638109388146e-08
3387 2.30609980946817e-08
3388 2.33781276259837e-08
3389 2.31525546539046e-08
3390 2.31377557007306e-08
3391 2.31900846334732e-08
3392 2.31315408651955e-08
3393 2.32573747172005e-08
3394 2.3160034069214e-08
3395 2.31166118824788e-08
3396 2.31796524929884e-08
3397 2.30654046662337e-08
3398 2.3193977092717e-08
3399 2.31509939718766e-08
3400 2.3081674364267e-08
3401 2.30156248943736e-08
3402 2.30607538447281e-08
3403 2.31196495681374e-08
3404 2.30447853972748e-08
3405 2.31295169736967e-08
3406 2.29673716964918e-08
3407 2.29866418313129e-08
3408 2.30527040789852e-08
3409 2.31774263816931e-08
3410 2.29339903548009e-08
3411 2.30594788979133e-08
3412 2.29283069632658e-08
3413 2.31625483144704e-08
3414 2.30078164125125e-08
3415 2.29766394697073e-08
3416 2.30090563611007e-08
3417 2.30763897666186e-08
3418 2.28907823709434e-08
3419 2.30259437392633e-08
3420 2.30402354093329e-08
3421 2.30294365706207e-08
3422 2.29681758070477e-08
3423 2.29923424206468e-08
3424 2.31405762463055e-08
3425 2.30763150645963e-08
3426 2.2965308917211e-08
3427 2.29384818539735e-08
3428 2.30284759390464e-08
3429 2.29651537893005e-08
3430 2.31697708934497e-08
3431 2.31283923115555e-08
3432 2.27981976492764e-08
3433 2.29592823077773e-08
3434 2.30300485921653e-08
3435 2.31138373552042e-08
3436 2.28466751872602e-08
3437 2.29552097170682e-08
3438 2.29596532883569e-08
3439 2.30281578024183e-08
3440 2.29514866338931e-08
3441 2.29579493100651e-08
3442 2.3014936893162e-08
3443 2.28451385324568e-08
3444 2.31818653855242e-08
3445 2.28916997953021e-08
3446 2.28732027873768e-08
3447 2.28804948076977e-08
3448 2.2976215006576e-08
3449 2.30890600421141e-08
3450 2.28177967454002e-08
3451 2.30215510845788e-08
3452 2.28250979272815e-08
3453 2.30269116388016e-08
3454 2.30285137376995e-08
3455 2.28333391638635e-08
3456 2.28009145653019e-08
3457 2.29991992077849e-08
3458 2.29054313813926e-08
3459 2.2812390301663e-08
3460 2.32019232346836e-08
3461 2.2816616294552e-08
3462 2.28515095561654e-08
3463 2.30548556903187e-08
3464 2.29664676649755e-08
3465 2.26805748742009e-08
3466 2.30860922574294e-08
3467 2.28002519233605e-08
3468 2.30304176789264e-08
3469 2.28317865840211e-08
3470 2.27310333009711e-08
3471 2.30202414415182e-08
3472 2.29600448413692e-08
3473 2.27888916128904e-08
3474 2.30319312874983e-08
3475 2.2976385334772e-08
3476 2.27210930101585e-08
3477 2.27052008208162e-08
3478 2.28453643753568e-08
3479 2.27970183543924e-08
3480 2.28090243488133e-08
3481 2.30512649910253e-08
3482 2.27705694344493e-08
3483 2.27424011765009e-08
3484 2.27329494175699e-08
3485 2.27667536805143e-08
3486 2.28761463700877e-08
3487 2.27429633143927e-08
3488 2.27957568785797e-08
3489 2.29970610159391e-08
3490 2.26725895680069e-08
3491 2.28076335435645e-08
3492 2.27252583200332e-08
3493 2.28748383226396e-08
3494 2.28508310526898e-08
3495 2.27449495882048e-08
3496 2.28251648706213e-08
3497 2.30493579378432e-08
3498 2.2740911989505e-08
3499 2.28349068964739e-08
3500 2.27178675831219e-08
3501 2.28324328221952e-08
3502 2.26337093161e-08
3503 2.26626378903028e-08
3504 2.27406186015244e-08
3505 2.28755374942491e-08
3506 2.2732346828036e-08
3507 2.27642675705297e-08
3508 2.2595705246431e-08
3509 2.2795707587342e-08
3510 2.26030903607288e-08
3511 2.28491111231577e-08
3512 2.27786947029074e-08
3513 2.28316518338101e-08
3514 2.27162735777497e-08
3515 2.28076022521506e-08
3516 2.25535487055062e-08
3517 2.2611264882455e-08
3518 2.28160970527913e-08
3519 2.26279516768457e-08
3520 2.3186598260061e-08
3521 2.25495970824774e-08
3522 2.25549882055809e-08
3523 2.2613420135098e-08
3524 2.28865808120648e-08
3525 2.26381884926852e-08
3526 2.26999331101041e-08
3527 2.28741996393111e-08
3528 2.2516737763878e-08
3529 2.26565790002464e-08
3530 2.26420274820338e-08
3531 2.2744501123162e-08
3532 2.25865318985718e-08
3533 2.26714083224611e-08
3534 2.27445378528923e-08
3535 2.25250648491215e-08
3536 2.2687634087859e-08
3537 2.26213327243485e-08
3538 2.26689564155613e-08
3539 2.26246515819817e-08
3540 2.25543794123428e-08
3541 2.27967000991924e-08
3542 2.25818293126956e-08
3543 2.26503310221116e-08
3544 2.25278948919883e-08
3545 2.258407851774e-08
3546 2.27343940883973e-08
3547 2.25139518716944e-08
3548 2.25877662691776e-08
3549 2.26226344159919e-08
3550 2.2825603448684e-08
3551 2.2636480968341e-08
3552 2.25077941482521e-08
3553 2.26614687781623e-08
3554 2.25432831135741e-08
3555 2.25842597032511e-08
3556 2.26054647731733e-08
3557 2.25779542404325e-08
3558 2.26707206532062e-08
3559 2.24289208088368e-08
3560 2.27049983101413e-08
3561 2.24537102724298e-08
3562 2.25290825823787e-08
3563 2.27222487096945e-08
3564 2.27353034401023e-08
3565 2.24529820433972e-08
3566 2.28843825020864e-08
3567 2.2492459394563e-08
3568 2.25928205168646e-08
3569 2.25049306306069e-08
3570 2.27133272059366e-08
3571 2.2514146764685e-08
3572 2.26598071160211e-08
3573 2.25668040176874e-08
3574 2.2669606624337e-08
3575 2.24154775625429e-08
3576 2.24920537719164e-08
3577 2.28808361040223e-08
3578 2.26908430938089e-08
3579 2.24117887999142e-08
3580 2.25368482216215e-08
3581 2.2573202390852e-08
3582 2.24895902884992e-08
3583 2.2529937206528e-08
3584 2.25599897443196e-08
3585 2.24323395787707e-08
3586 2.28228926819618e-08
3587 2.23656785243165e-08
3588 2.24914805142529e-08
3589 2.26616101417498e-08
3590 2.25386441208286e-08
3591 2.24381201345825e-08
3592 2.27061986897148e-08
3593 2.25913807847533e-08
3594 2.25528701431887e-08
3595 2.24223164204673e-08
3596 2.25953716277427e-08
3597 2.24805478414858e-08
3598 2.24917505744493e-08
3599 2.25082465223903e-08
3600 2.24985207224471e-08
3601 2.24676903264331e-08
3602 2.25404676430507e-08
3603 2.23600728097573e-08
3604 2.26344973974779e-08
3605 2.25041748429522e-08
3606 2.2446626677386e-08
3607 2.25573128000978e-08
3608 2.25325850071734e-08
3609 2.24343837391672e-08
3610 2.2680117622631e-08
3611 2.23922082542849e-08
3612 2.25212781437278e-08
3613 2.23820213673953e-08
3614 2.27805290911043e-08
3615 2.2513973653604e-08
3616 2.23521478099897e-08
3617 2.26004396397528e-08
3618 2.23854903413923e-08
3619 2.26035386066137e-08
3620 2.23257969418356e-08
3621 2.25765461432381e-08
3622 2.2432686877405e-08
3623 2.24965111381703e-08
3624 2.24152748109496e-08
3625 2.25089427292602e-08
3626 2.24010484699555e-08
3627 2.24311163026236e-08
3628 2.25663308266455e-08
3629 2.23714418943199e-08
3630 2.24918704148092e-08
3631 2.23608001554965e-08
3632 2.24748066608438e-08
3633 2.24715837073841e-08
3634 2.23384452882058e-08
3635 2.26575447122013e-08
3636 2.23384790278836e-08
3637 2.26010501256368e-08
3638 2.22691405109554e-08
3639 2.27327572268621e-08
3640 2.24387043079588e-08
3641 2.24716293371063e-08
3642 2.22858815577265e-08
3643 2.24953529959215e-08
3644 2.23270646948492e-08
3645 2.24276318014915e-08
3646 2.25136133444881e-08
3647 2.24172751071006e-08
3648 2.24353883959694e-08
3649 2.23705666009266e-08
3650 2.24636448673543e-08
3651 2.2417536007957e-08
3652 2.23725521719675e-08
3653 2.24452487509463e-08
3654 2.2307292261381e-08
3655 2.25086580973866e-08
3656 2.22803886300582e-08
3657 2.24328756188719e-08
3658 2.24307422063141e-08
3659 2.2256584904401e-08
3660 2.25058371181586e-08
3661 2.23722670580351e-08
3662 2.2386171528721e-08
3663 2.23016023277456e-08
3664 2.25770743282894e-08
3665 2.22848286091093e-08
3666 2.25830673319383e-08
3667 2.23214478880518e-08
3668 2.23690538529997e-08
3669 2.23263552978636e-08
3670 2.23499337208555e-08
3671 2.23901080396161e-08
3672 2.25032581522289e-08
3673 2.24274779614397e-08
3674 2.22644855607523e-08
3675 2.24165911408836e-08
3676 2.23027582175739e-08
3677 2.2337452884269e-08
3678 2.25852666682069e-08
3679 2.23281228157735e-08
3680 2.25206689570268e-08
3681 2.24375724684478e-08
3682 2.22208991291772e-08
3683 2.23351954220519e-08
3684 2.23122046076885e-08
3685 2.23981489237968e-08
3686 2.23405959154377e-08
3687 2.23459035724183e-08
3688 2.24084256095214e-08
3689 2.23500930345288e-08
3690 2.23089133353227e-08
3691 2.23911188144132e-08
3692 2.22709635806506e-08
3693 2.23086737114464e-08
3694 2.25743524591149e-08
3695 2.21701642642014e-08
3696 2.24204099126268e-08
3697 2.23096074110085e-08
3698 2.22511818253057e-08
3699 2.24632724254992e-08
3700 2.21883730076033e-08
3701 2.24460991649078e-08
3702 2.23341153728907e-08
3703 2.2390334011968e-08
3704 2.22368700253384e-08
3705 2.23775317544916e-08
3706 2.23407268902243e-08
3707 2.21906637747349e-08
3708 2.25675969462991e-08
3709 2.21787097387516e-08
3710 2.22353435517242e-08
3711 2.24759532523322e-08
3712 2.2227197897573e-08
3713 2.2225924796393e-08
3714 2.2333513600703e-08
3715 2.23401907246679e-08
3716 2.2234182614822e-08
3717 2.22559380804732e-08
3718 2.21915171945142e-08
3719 2.21589610862605e-08
3720 2.24076494950154e-08
3721 2.24987540728971e-08
3722 2.21065498828832e-08
3723 2.22642045848431e-08
3724 2.22082742757923e-08
3725 2.23028388897095e-08
3726 2.22040204362894e-08
3727 2.23174109859059e-08
3728 2.21309623771315e-08
3729 2.22104704326043e-08
3730 2.23292099765793e-08
3731 2.20444573919565e-08
3732 2.25763074257479e-08
3733 2.2169999693844e-08
3734 2.22051823721703e-08
3735 2.2107652345893e-08
3736 2.21145482683927e-08
3737 2.24625571794146e-08
3738 2.20909860220164e-08
3739 2.21209727844229e-08
3740 2.21692454713818e-08
3741 2.21595716685119e-08
3742 2.21962999131797e-08
3743 2.21919804754833e-08
3744 2.21750338955662e-08
3745 2.21973248364371e-08
3746 2.21639357735892e-08
3747 2.21910507909318e-08
3748 2.2286656540027e-08
3749 2.22233942768213e-08
3750 2.20983021279597e-08
3751 2.24584604169298e-08
3752 2.21016890071812e-08
3753 2.20211338242127e-08
3754 2.23810819952597e-08
3755 2.2131645265322e-08
3756 2.22208427005377e-08
3757 2.1982314564184e-08
3758 2.24770603789448e-08
3759 2.20726190209852e-08
3760 2.20398300141422e-08
3761 2.21582529185138e-08
3762 2.20939116404306e-08
3763 2.20289492760539e-08
3764 2.23056096895213e-08
3765 2.19747457164132e-08
3766 2.2302520193751e-08
3767 2.21975433163379e-08
3768 2.19988156364703e-08
3769 2.20497065288594e-08
3770 2.21469046610689e-08
3771 2.20759429474526e-08
3772 2.19862251475966e-08
3773 2.22616856124791e-08
3774 2.20796706571491e-08
3775 2.19327849908524e-08
3776 2.20910852504197e-08
3777 2.21394717978551e-08
3778 2.20255507519251e-08
3779 2.20393042469347e-08
3780 2.2063864300037e-08
3781 2.22596323575797e-08
3782 2.19829765741864e-08
3783 2.20515992783721e-08
3784 2.19877520182266e-08
3785 2.20517793008135e-08
3786 2.23712743467885e-08
3787 2.18877743529688e-08
3788 2.19478323821853e-08
3789 2.20778444490755e-08
3790 2.19829103231817e-08
3791 2.20371464196845e-08
3792 2.21814027256961e-08
3793 2.2225363195183e-08
3794 2.18629894579436e-08
3795 2.20352083939446e-08
3796 2.20531922292544e-08
3797 2.2056295804429e-08
3798 2.19183920564525e-08
3799 2.20761963847238e-08
3800 2.19506100549438e-08
3801 2.20346148296358e-08
3802 2.18041279276271e-08
3803 2.20921015787745e-08
3804 2.19245606589702e-08
3805 2.19799315557267e-08
3806 2.20234565029287e-08
3807 2.18915402114828e-08
3808 2.22323969456539e-08
3809 2.17399340511903e-08
3810 2.20400831274503e-08
3811 2.21042323680365e-08
3812 2.20291402692752e-08
3813 2.1872497305786e-08
3814 2.19715960279032e-08
3815 2.18709622892277e-08
3816 2.20994211717063e-08
3817 2.17874395405104e-08
3818 2.20299748359132e-08
3819 2.20166892495044e-08
3820 2.17741928110193e-08
3821 2.21567818270429e-08
3822 2.18663961146781e-08
3823 2.18522432309243e-08
3824 2.1915894054203e-08
3825 2.19066162308046e-08
3826 2.1941177174245e-08
3827 2.18314289992527e-08
3828 2.19528873421915e-08
3829 2.19825609213409e-08
3830 2.17158118442917e-08
3831 2.21043254966524e-08
3832 2.18840804140363e-08
3833 2.20385996319017e-08
3834 2.17576601238001e-08
3835 2.19174986941795e-08
3836 2.19648226900127e-08
3837 2.17512481337501e-08
3838 2.19195487920487e-08
3839 2.17575000034387e-08
3840 2.19580323590485e-08
3841 2.18614212883494e-08
3842 2.18435730428013e-08
3843 2.18967461331765e-08
3844 2.18887017031655e-08
3845 2.17603535439537e-08
3846 2.1782388207825e-08
3847 2.18898978454618e-08
3848 2.18016015569589e-08
3849 2.18763992436699e-08
3850 2.17774050494945e-08
3851 2.18643044762601e-08
3852 2.17710867276288e-08
3853 2.18200038442085e-08
3854 2.18240467013686e-08
3855 2.17343335768838e-08
3856 2.20036977456317e-08
3857 2.17785489633471e-08
3858 2.176729334602e-08
3859 2.19425135741247e-08
3860 2.16839968891769e-08
3861 2.17965308371149e-08
3862 2.2017282310216e-08
3863 2.16368196181982e-08
3864 2.17781115003923e-08
3865 2.19613329055512e-08
3866 2.17299732838683e-08
3867 2.16885710266901e-08
3868 2.17944807836545e-08
3869 2.17675618614521e-08
3870 2.18321740681482e-08
3871 2.16566725734779e-08
3872 2.18870702335394e-08
3873 2.18087317964688e-08
3874 2.17037830823052e-08
3875 2.17815539613753e-08
3876 2.17161183293513e-08
3877 2.17496452021848e-08
3878 2.1698165604267e-08
3879 2.19202467963697e-08
3880 2.169459875212e-08
3881 2.16038391642748e-08
3882 2.1773406853498e-08
3883 2.18864283647591e-08
3884 2.16502030707666e-08
3885 2.17814071912237e-08
3886 2.16774256784902e-08
3887 2.17489195297738e-08
3888 2.16854026811042e-08
3889 2.17277371334923e-08
3890 2.18362895427759e-08
3891 2.17030984610567e-08
3892 2.17334975789463e-08
3893 2.16377185413563e-08
3894 2.17095363050923e-08
3895 2.17817176428881e-08
3896 2.17962389323922e-08
3897 2.16335342653196e-08
3898 2.18328523713396e-08
3899 2.15894386217386e-08
3900 2.17776753492771e-08
3901 2.1617551967168e-08
3902 2.183479905149e-08
3903 2.16214716766139e-08
3904 2.17055747300243e-08
3905 2.16274966957641e-08
3906 2.17764437024925e-08
3907 2.16464358975266e-08
3908 2.186005442395e-08
3909 2.15790747839506e-08
3910 2.17791212753227e-08
3911 2.1665906428936e-08
3912 2.15975425019721e-08
3913 2.16979608893553e-08
3914 2.16076912966656e-08
3915 2.18175502535445e-08
3916 2.16021971222169e-08
3917 2.17098065751209e-08
3918 2.15884558332213e-08
3919 2.17984093078272e-08
3920 2.15186486225427e-08
3921 2.18234944102669e-08
3922 2.16288545744625e-08
3923 2.17163948796895e-08
3924 2.16417641465405e-08
3925 2.16691205707775e-08
3926 2.16033799591564e-08
3927 2.170101443566e-08
3928 2.17329172604863e-08
3929 2.16277243647589e-08
3930 2.15816566631055e-08
3931 2.17076435338193e-08
3932 2.14913439293696e-08
3933 2.18883194911257e-08
3934 2.15437530766138e-08
3935 2.15919203099268e-08
3936 2.18467206525297e-08
3937 2.14344552995627e-08
3938 2.16398144210483e-08
3939 2.1810016745949e-08
3940 2.16463307518566e-08
3941 2.17440851053574e-08
3942 2.15078771739208e-08
3943 2.18769045070566e-08
3944 2.15028886119129e-08
3945 2.158837054278e-08
3946 2.1769398270477e-08
3947 2.14986533366712e-08
3948 2.17868860552528e-08
3949 2.15853992417969e-08
3950 2.15899818156728e-08
3951 2.16221433939623e-08
3952 2.16383118591956e-08
3953 2.15939625651806e-08
3954 2.15770592619791e-08
3955 2.16261709780063e-08
3956 2.16059532469615e-08
3957 2.16367247087845e-08
3958 2.16388007965307e-08
3959 2.15401498389056e-08
3960 2.17495723462413e-08
3961 2.15600751123368e-08
3962 2.15257444282635e-08
3963 2.16210442709475e-08
3964 2.15741286728743e-08
3965 2.16019041192617e-08
3966 2.16067630618433e-08
3967 2.16205744796305e-08
3968 2.16111578001943e-08
3969 2.15876189373354e-08
3970 2.13769125645413e-08
3971 2.16533818457876e-08
3972 2.17497432681846e-08
3973 2.15284200453247e-08
3974 2.15468751560977e-08
3975 2.15446596185664e-08
3976 2.16399801682421e-08
3977 2.15427750356323e-08
3978 2.16779784198984e-08
3979 2.15487424117811e-08
3980 2.13924482732608e-08
3981 2.15550972102463e-08
3982 2.16769678367257e-08
3983 2.16022542094407e-08
3984 2.15739240205792e-08
3985 2.15782689572119e-08
3986 2.17161357682905e-08
3987 2.14241994256348e-08
3988 2.15703309027759e-08
3989 2.15289537077723e-08
3990 2.15898816002813e-08
3991 2.14851017850126e-08
3992 2.15233502296464e-08
3993 2.18324647698331e-08
3994 2.138924653039e-08
3995 2.16207065655283e-08
3996 2.14782233789457e-08
3997 2.14162489839964e-08
3998 2.18734670529575e-08
3999 2.14575650625548e-08
4000 2.14878860576029e-08
4001 2.15733023263276e-08
4002 2.14616074361018e-08
4003 2.153729158616e-08
4004 2.1547041797465e-08
4005 2.13328376275079e-08
4006 2.16720184400554e-08
4007 2.14491509047399e-08
4008 2.16659852623202e-08
4009 2.13866435854193e-08
4010 2.1646703742384e-08
4011 2.16519047384622e-08
4012 2.15681211852559e-08
4013 2.14905008379951e-08
4014 2.14969303029555e-08
4015 2.16043281988654e-08
4016 2.15950150308508e-08
4017 2.14338183219809e-08
4018 2.15194598394142e-08
4019 2.14827337945422e-08
4020 2.16355822202363e-08
4021 2.14503309505787e-08
4022 2.14667484113029e-08
4023 2.17029546649705e-08
4024 2.13773109338788e-08
4025 2.16580175744774e-08
4026 2.15222487689459e-08
4027 2.12934086190053e-08
4028 2.1772494047001e-08
4029 2.1356254927829e-08
4030 2.14339856197121e-08
4031 2.14106910498835e-08
4032 2.16451962442576e-08
4033 2.14209282187738e-08
4034 2.15179593126003e-08
4035 2.15016141353885e-08
4036 2.14223909122957e-08
4037 2.14927469976978e-08
4038 2.1345921869198e-08
4039 2.17576623464666e-08
4040 2.1416298488397e-08
4041 2.14497937025548e-08
4042 2.1544603912016e-08
4043 2.13958509778234e-08
4044 2.1474590902848e-08
4045 2.16043552385692e-08
4046 2.13855005974928e-08
4047 2.14908766755784e-08
4048 2.15110109009764e-08
4049 2.13965520150516e-08
4050 2.15208448601789e-08
4051 2.14623680068193e-08
4052 2.15701629455722e-08
4053 2.13644955779912e-08
4054 2.15211230538692e-08
4055 2.14733830929958e-08
4056 2.14726840868096e-08
4057 2.14489295964349e-08
4058 2.14514049474701e-08
4059 2.14136654337338e-08
4060 2.14346019613565e-08
4061 2.14430868838988e-08
4062 2.1505491235807e-08
4063 2.13005385523157e-08
4064 2.14978888561923e-08
4065 2.16518458540094e-08
4066 2.13560538395718e-08
4067 2.1486982819896e-08
4068 2.14156319879866e-08
4069 2.14396519324644e-08
4070 2.14461016279088e-08
4071 2.1368759467677e-08
4072 2.15806085817061e-08
4073 2.14017121646926e-08
4074 2.14157044289287e-08
4075 2.14922040926435e-08
4076 2.14236899278575e-08
4077 2.14561318720996e-08
4078 2.13072914023726e-08
4079 2.15843604536303e-08
4080 2.12416766616208e-08
4081 2.15739330755582e-08
4082 2.13429386533015e-08
4083 2.14849834354602e-08
4084 2.13901139627382e-08
4085 2.14861618168527e-08
4086 2.14012401904551e-08
4087 2.15871621858099e-08
4088 2.13403820281677e-08
4089 2.1304491212959e-08
4090 2.15457681957965e-08
4091 2.12909117087712e-08
4092 2.14993969165356e-08
4093 2.14671233149666e-08
4094 2.12849316476582e-08
4095 2.16416177587497e-08
4096 2.12051181849304e-08
4097 2.15050271612505e-08
4098 2.13581918968586e-08
4099 2.14855140105996e-08
4100 2.14197171124475e-08
4101 2.14517644687717e-08
4102 2.11986393823249e-08
4103 2.15614366760875e-08
4104 2.14355865064686e-08
4105 2.12216978603408e-08
4106 2.14609661859377e-08
4107 2.13968892857075e-08
4108 2.13978533567527e-08
4109 2.14738021711014e-08
4110 2.13945056044551e-08
4111 2.12931600804822e-08
4112 2.15873829065849e-08
4113 2.11418341373726e-08
4114 2.15422164999701e-08
4115 2.13570913936145e-08
4116 2.13923512684122e-08
4117 2.13651777412061e-08
4118 2.12343811378801e-08
4119 2.13803133921608e-08
4120 2.14727159510986e-08
4121 2.13652655567387e-08
4122 2.13351932882855e-08
4123 2.1492796515421e-08
4124 2.13340206922652e-08
4125 2.13564299982316e-08
4126 2.13206735675264e-08
4127 2.14477433191362e-08
4128 2.13206926089615e-08
4129 2.1420717798204e-08
4130 2.13497220857839e-08
4131 2.13372410091672e-08
4132 2.13732370908382e-08
4133 2.13610917594309e-08
4134 2.135580200191e-08
4135 2.13427337460992e-08
4136 2.1539774049506e-08
4137 2.13266967559189e-08
4138 2.1338541144722e-08
4139 2.13273278442117e-08
4140 2.14591782108364e-08
4141 2.12977791105295e-08
4142 2.13572001777074e-08
4143 2.12921789175535e-08
4144 2.15716348437311e-08
4145 2.12061189017732e-08
4146 2.13892549365546e-08
4147 2.13253023699789e-08
4148 2.13081042597096e-08
4149 2.1516914562536e-08
4150 2.11817999111741e-08
4151 2.13451261930064e-08
4152 2.13561674320406e-08
4153 2.12837861761628e-08
4154 2.14198775982943e-08
4155 2.12555113869861e-08
4156 2.12054977266618e-08
4157 2.1549469527038e-08
4158 2.11830078400421e-08
4159 2.13061075080567e-08
4160 2.13925691663341e-08
4161 2.14957024580187e-08
4162 2.11939344390011e-08
4163 2.13274050282486e-08
4164 2.14074294913047e-08
4165 2.1381251552155e-08
4166 2.13830597570741e-08
4167 2.13186150728184e-08
4168 2.1362871977626e-08
4169 2.12037755766836e-08
4170 2.12361624967183e-08
4171 2.15483113650272e-08
4172 2.11787359409854e-08
4173 2.16043961875911e-08
4174 2.11280788777746e-08
4175 2.14011933881153e-08
4176 2.12621288364634e-08
4177 2.1225036660244e-08
4178 2.14112388527976e-08
4179 2.13760328762191e-08
4180 2.12408771169681e-08
4181 2.13263091639604e-08
4182 2.12683651237278e-08
4183 2.12698162285196e-08
4184 2.13514172513474e-08
4185 2.11821344779928e-08
4186 2.127402131058e-08
4187 2.13721307442682e-08
4188 2.13372637978271e-08
4189 2.12263494037135e-08
4190 2.12978984832635e-08
4191 2.12557296785931e-08
4192 2.1220624538687e-08
4193 2.13407481255423e-08
4194 2.13846664913753e-08
4195 2.12553524434611e-08
4196 2.14183422189151e-08
4197 2.12172835576396e-08
4198 2.13544295852852e-08
4199 2.12647980324387e-08
4200 2.12952818008372e-08
4201 2.12542187392195e-08
4202 2.12608267895487e-08
4203 2.14497903940902e-08
4204 2.12669961903167e-08
4205 2.13149054633099e-08
4206 2.1271569312642e-08
4207 2.12347798882462e-08
4208 2.1253522862974e-08
4209 2.1157762323698e-08
4210 2.13934636454915e-08
4211 2.13571633449483e-08
4212 2.11960014138146e-08
4213 2.13252734873048e-08
4214 2.12450123082686e-08
4215 2.12015235785401e-08
4216 2.12969834112453e-08
4217 2.11416845619095e-08
4218 2.13386798704196e-08
4219 2.1063930083498e-08
4220 2.14392603474778e-08
4221 2.11845784943154e-08
4222 2.11396753586612e-08
4223 2.14200839843137e-08
4224 2.11985523255187e-08
4225 2.12243809860624e-08
4226 2.13741969179448e-08
4227 2.10802915907582e-08
4228 2.12406267761089e-08
4229 2.11863693437842e-08
4230 2.12080162704797e-08
4231 2.11977768778127e-08
4232 2.12513843262929e-08
4233 2.12130263197441e-08
4234 2.1167127429722e-08
4235 2.11846771114299e-08
4236 2.13218203073406e-08
4237 2.11547510864385e-08
4238 2.11937085563552e-08
4239 2.11806557681715e-08
4240 2.12564781043589e-08
4241 2.11535583736211e-08
4242 2.1254985193897e-08
4243 2.11544343715619e-08
4244 2.11403426659817e-08
4245 2.11707373840575e-08
4246 2.1151070235037e-08
4247 2.1080393530104e-08
4248 2.11674660555161e-08
4249 2.13847656205246e-08
4250 2.10395420274789e-08
4251 2.12974629587581e-08
4252 2.11466770561053e-08
4253 2.11340311475627e-08
4254 2.11836828374334e-08
4255 2.12011392404232e-08
4256 2.1151462354263e-08
4257 2.12220883819558e-08
4258 2.11374118497254e-08
4259 2.12241677648439e-08
4260 2.11025037917256e-08
4261 2.12139966357938e-08
4262 2.10938154163731e-08
4263 2.12437222846251e-08
4264 2.11132826717364e-08
4265 2.10907700399776e-08
4266 2.11663004181517e-08
4267 2.11145610049535e-08
4268 2.10821659836213e-08
4269 2.14068606123607e-08
4270 2.09823131258613e-08
4271 2.12360417584101e-08
4272 2.10098667141612e-08
4273 2.13704471498932e-08
4274 2.10431856049187e-08
4275 2.11382248391789e-08
4276 2.11881684004656e-08
4277 2.10716506010566e-08
4278 2.11183721692265e-08
4279 2.11855765941404e-08
4280 2.1074742994287e-08
4281 2.11690737004222e-08
4282 2.10877605819615e-08
4283 2.10721670161895e-08
4284 2.12449811303195e-08
4285 2.10365596968742e-08
4286 2.12436578896913e-08
4287 2.10445661059655e-08
4288 2.11109443648549e-08
4289 2.11036602033587e-08
4290 2.11584101010853e-08
4291 2.10979741832595e-08
4292 2.11435350767619e-08
4293 2.10309048636059e-08
4294 2.11092080615316e-08
4295 2.11004686665461e-08
4296 2.11238272069814e-08
4297 2.10962758744326e-08
4298 2.10357497978375e-08
4299 2.11087026813495e-08
4300 2.111488153278e-08
4301 2.11085942858347e-08
4302 2.0992530832542e-08
4303 2.10131654272416e-08
4304 2.12807044979169e-08
4305 2.10465745600352e-08
4306 2.11166212713554e-08
4307 2.10355586842681e-08
4308 2.09826337578267e-08
4309 2.11686323121629e-08
4310 2.10542743612763e-08
4311 2.10061667320627e-08
4312 2.10890875655956e-08
4313 2.10508235189977e-08
4314 2.10151690342997e-08
4315 2.10815299652722e-08
4316 2.10521572117095e-08
4317 2.10623416718736e-08
4318 2.09986968857656e-08
4319 2.11110654650959e-08
4320 2.09154564219105e-08
4321 2.11277797173004e-08
4322 2.10587299029896e-08
4323 2.0956035404085e-08
4324 2.11807911854045e-08
4325 2.09200875456172e-08
4326 2.09876796974928e-08
4327 2.10612135458366e-08
4328 2.09874828718259e-08
4329 2.11342094924571e-08
4330 2.1005303682875e-08
4331 2.09981536039017e-08
4332 2.10101800643958e-08
4333 2.10720670366094e-08
4334 2.09850567711722e-08
4335 2.09185448514226e-08
4336 2.11359831083779e-08
4337 2.09508507131861e-08
4338 2.10102513671373e-08
4339 2.0994818151765e-08
4340 2.09671596569372e-08
4341 2.09565560733704e-08
4342 2.10329028613732e-08
4343 2.10195263599111e-08
4344 2.10830118276828e-08
4345 2.09358527789583e-08
4346 2.09746712585002e-08
4347 2.10616870519598e-08
4348 2.09457804787316e-08
4349 2.09804420296944e-08
4350 2.10429964497827e-08
4351 2.09475161734307e-08
4352 2.09914839193193e-08
4353 2.09956621353147e-08
4354 2.09808637441355e-08
4355 2.09820359149404e-08
4356 2.10232776780561e-08
4357 2.09711115650713e-08
4358 2.10403690013017e-08
4359 2.09563126192247e-08
4360 2.10230224109154e-08
4361 2.10442709140946e-08
4362 2.10008575578513e-08
4363 2.10097195370018e-08
4364 2.09265472510545e-08
4365 2.09854195345471e-08
4366 2.10472919444094e-08
4367 2.10857099518513e-08
4368 2.09021064059289e-08
4369 2.09549526122288e-08
4370 2.0942909830346e-08
4371 2.10554717852141e-08
4372 2.09673359066187e-08
4373 2.09070334806771e-08
4374 2.10497682808786e-08
4375 2.09927224246176e-08
4376 2.09002572248984e-08
4377 2.10177316857241e-08
4378 2.09438535669904e-08
4379 2.11131890639482e-08
4380 2.08398362953055e-08
4381 2.10433749623373e-08
4382 2.09003363740301e-08
4383 2.09222884917093e-08
4384 2.09782306384021e-08
4385 2.09888386997203e-08
4386 2.09307537692993e-08
4387 2.11517195982669e-08
4388 2.08101628980106e-08
4389 2.10465479293376e-08
4390 2.08680535906947e-08
4391 2.09731607234698e-08
4392 2.09576001946044e-08
4393 2.09519052243312e-08
4394 2.09010657297171e-08
4395 2.10031360323715e-08
4396 2.09105505715179e-08
4397 2.09994560453897e-08
4398 2.08436263418044e-08
4399 2.10311216184422e-08
4400 2.0875554272104e-08
4401 2.08481261334548e-08
4402 2.11467362913886e-08
4403 2.08256905016579e-08
4404 2.08334512430408e-08
4405 2.10767024002578e-08
4406 2.0805712974914e-08
4407 2.0881536857642e-08
4408 2.10646543239434e-08
4409 2.0891476816054e-08
4410 2.09140044944434e-08
4411 2.09350411175535e-08
4412 2.0915581925518e-08
4413 2.09387407026362e-08
4414 2.09230186469789e-08
4415 2.09329333458186e-08
4416 2.0898626435395e-08
4417 2.0945958883134e-08
4418 2.09381763780492e-08
4419 2.08975639437448e-08
4420 2.09559292168038e-08
4421 2.08757975226348e-08
4422 2.09405477256119e-08
4423 2.08937214458693e-08
4424 2.09464054794495e-08
4425 2.09093607801414e-08
4426 2.08753952326557e-08
4427 2.10828038376132e-08
4428 2.07692432723228e-08
4429 2.09168581297714e-08
4430 2.08911396279987e-08
4431 2.08841915680935e-08
4432 2.08772647589672e-08
4433 2.08780843267142e-08
4434 2.09719377637363e-08
4435 2.09110802651402e-08
4436 2.09200460237202e-08
4437 2.08237541123868e-08
4438 2.09029689131057e-08
4439 2.08169878666009e-08
4440 2.09905390160525e-08
4441 2.08495921509844e-08
4442 2.09255649916695e-08
4443 2.08291237628888e-08
4444 2.08834005952507e-08
4445 2.08705723141467e-08
4446 2.09154775565601e-08
4447 2.08381585422579e-08
4448 2.0909013152437e-08
4449 2.08585681562568e-08
4450 2.09091295872987e-08
4451 2.09315167787327e-08
4452 2.08932774072856e-08
4453 2.08878494083464e-08
4454 2.08565298456254e-08
4455 2.076148429353e-08
4456 2.10034278438354e-08
4457 2.07434719474708e-08
4458 2.09799398076527e-08
4459 2.08101781911107e-08
4460 2.09030235338581e-08
4461 2.09272753439738e-08
4462 2.07275889627656e-08
4463 2.09538222504246e-08
4464 2.08474479452825e-08
4465 2.08395236211967e-08
4466 2.0934646725923e-08
4467 2.08873313207736e-08
4468 2.07774246920867e-08
4469 2.09199528327098e-08
4470 2.07747754759069e-08
4471 2.09089257545703e-08
4472 2.08464444135803e-08
4473 2.08300314037402e-08
4474 2.08987861562981e-08
4475 2.08062828841449e-08
4476 2.08470211056078e-08
4477 2.08428734769228e-08
4478 2.08871575022584e-08
4479 2.07841627657412e-08
4480 2.09055469941877e-08
4481 2.0866239673234e-08
4482 2.07785952206496e-08
4483 2.08225458382394e-08
4484 2.08377533141846e-08
4485 2.08367733394166e-08
4486 2.0858907852972e-08
4487 2.08803925880741e-08
4488 2.08359664284341e-08
4489 2.08568044959279e-08
4490 2.08168888526927e-08
4491 2.08495764990602e-08
4492 2.0757484377576e-08
4493 2.08571756656895e-08
4494 2.08284957219362e-08
4495 2.08599645150631e-08
4496 2.07871206405752e-08
4497 2.07343050173669e-08
4498 2.09740383927404e-08
4499 2.07277063826172e-08
4500 2.08998305102348e-08
4501 2.08469662335009e-08
4502 2.07872685653587e-08
4503 2.08554737630706e-08
4504 2.08035487945235e-08
4505 2.08412512268108e-08
4506 2.07999611037124e-08
4507 2.08712172116154e-08
4508 2.07392644056004e-08
4509 2.08608669760579e-08
4510 2.08365591785054e-08
4511 2.08617589321225e-08
4512 2.0721216744235e-08
4513 2.08055281778474e-08
4514 2.08172897662173e-08
4515 2.07776136487148e-08
4516 2.08945902990809e-08
4517 2.0694630251894e-08
4518 2.09403398909735e-08
4519 2.06878005599709e-08
4520 2.07687846598414e-08
4521 2.08639719896375e-08
4522 2.07599367869626e-08
4523 2.081465560555e-08
4524 2.07675414380937e-08
4525 2.08657128877299e-08
4526 2.07967735665893e-08
4527 2.0786719808763e-08
4528 2.07891682060257e-08
4529 2.08251086286637e-08
4530 2.07482245306867e-08
4531 2.08335631131096e-08
4532 2.07434014032337e-08
4533 2.07988590639197e-08
4534 2.08334565527934e-08
4535 2.06522564818457e-08
4536 2.07469068715937e-08
4537 2.07842955295412e-08
4538 2.07850541942278e-08
4539 2.06836246805508e-08
4540 2.07716984337303e-08
4541 2.07578798927521e-08
4542 2.07536601477543e-08
4543 2.0813416046872e-08
4544 2.07154869280579e-08
4545 2.07065240167559e-08
4546 2.07830980278878e-08
4547 2.07896848332112e-08
4548 2.07305445460992e-08
4549 2.07284929900631e-08
4550 2.07102537084225e-08
4551 2.07113236465606e-08
4552 2.0734667164124e-08
4553 2.08580899918598e-08
4554 2.07091702710827e-08
4555 2.07144639754464e-08
4556 2.07330957431306e-08
4557 2.06801877582485e-08
4558 2.07532102378671e-08
4559 2.07075714908633e-08
4560 2.07901060389482e-08
4561 2.064818665648e-08
4562 2.07322795178211e-08
4563 2.07329204662265e-08
4564 2.06653176502147e-08
4565 2.07620870540381e-08
4566 2.06416729866543e-08
4567 2.07684592652413e-08
4568 2.06991203328677e-08
4569 2.07076956852958e-08
4570 2.07266033578346e-08
4571 2.06749814921636e-08
4572 2.06955809018972e-08
4573 2.07134452772095e-08
4574 2.0700857168876e-08
4575 2.07836560681685e-08
4576 2.06880717832369e-08
4577 2.06560179596416e-08
4578 2.07050969089906e-08
4579 2.06246243603658e-08
4580 2.08212070287317e-08
4581 2.06348238855902e-08
4582 2.06754599212378e-08
4583 2.06559607289769e-08
4584 2.06873637034199e-08
4585 2.07709439676851e-08
4586 2.0667355566939e-08
4587 2.06486316534082e-08
4588 2.08131644332532e-08
4589 2.06093818202469e-08
4590 2.08717516214652e-08
4591 2.06052016482605e-08
4592 2.06958239132859e-08
4593 2.06754464411318e-08
4594 2.0726718685582e-08
4595 2.06135959355258e-08
4596 2.06998030898298e-08
4597 2.0666113260237e-08
4598 2.07055762841968e-08
4599 2.07127778566463e-08
4600 2.05950293508117e-08
4601 2.06822383028715e-08
4602 2.06725594624757e-08
4603 2.06711629200385e-08
4604 2.06894851273454e-08
4605 2.06568811691454e-08
4606 2.06430830731463e-08
4607 2.06465888012985e-08
4608 2.06504892155124e-08
4609 2.07310573654418e-08
4610 2.06297194111915e-08
4611 2.06554875066267e-08
4612 2.06649140148674e-08
4613 2.06644439446624e-08
4614 2.06340524719906e-08
4615 2.07447388969051e-08
4616 2.05881770987126e-08
4617 2.06173431709367e-08
4618 2.06499574562091e-08
4619 2.07215993661691e-08
4620 2.06301702512235e-08
4621 2.07016255933112e-08
4622 2.06885304210314e-08
4623 2.05763562877159e-08
4624 2.06426082485223e-08
4625 2.05860707349625e-08
4626 2.07043521625039e-08
4627 2.05942110411694e-08
4628 2.07105497835869e-08
4629 2.06089959116085e-08
4630 2.06553804522613e-08
4631 2.06374877127491e-08
4632 2.06057364029455e-08
4633 2.0642829430928e-08
4634 2.0597821648316e-08
4635 2.06381300777991e-08
4636 2.06329858554177e-08
4637 2.0634440591305e-08
4638 2.06715976374117e-08
4639 2.0528016556387e-08
4640 2.0654678175358e-08
4641 2.06782235778746e-08
4642 2.06140110408093e-08
4643 2.05342544883358e-08
4644 2.06498574899516e-08
4645 2.0554577167875e-08
4646 2.06232757176128e-08
4647 2.05869167326789e-08
4648 2.05795212444571e-08
4649 2.05690652612667e-08
4650 2.06437538179394e-08
4651 2.05902697554272e-08
4652 2.05287397165854e-08
4653 2.07092471797843e-08
4654 2.07007014192406e-08
4655 2.05308431535656e-08
4656 2.05212203281402e-08
4657 2.06383698349022e-08
4658 2.06047139399423e-08
4659 2.05801031980535e-08
4660 2.05657454421804e-08
4661 2.05285015981715e-08
4662 2.06304251082479e-08
4663 2.05479832744437e-08
4664 2.0670858645655e-08
4665 2.05485701036956e-08
4666 2.05787103042532e-08
4667 2.05439166651722e-08
4668 2.0633716626195e-08
4669 2.0529742170039e-08
4670 2.06523511501189e-08
4671 2.05839572346989e-08
4672 2.04921257362933e-08
4673 2.07266822798147e-08
4674 2.04431272892158e-08
4675 2.06489308620661e-08
4676 2.05677139546445e-08
4677 2.05486415763012e-08
4678 2.05880346999532e-08
4679 2.05420599979078e-08
4680 2.05710557830141e-08
4681 2.05519496176976e-08
4682 2.04919602913023e-08
4683 2.05696202804084e-08
4684 2.0533597056005e-08
4685 2.06036932954845e-08
4686 2.05283126302191e-08
4687 2.05512548974163e-08
4688 2.05289949888332e-08
4689 2.04979643965153e-08
4690 2.06262034341265e-08
4691 2.05597120419565e-08
4692 2.0531094914622e-08
4693 2.05481007293784e-08
4694 2.05671125468321e-08
4695 2.0579976873325e-08
4696 2.0565099869696e-08
4697 2.05433795277266e-08
4698 2.05600619191948e-08
4699 2.05221419875556e-08
4700 2.06257888288874e-08
4701 2.04881738756768e-08
4702 2.04928901594847e-08
4703 2.05344497690163e-08
4704 2.05160064137289e-08
4705 2.05581813679245e-08
4706 2.04708562581146e-08
4707 2.05098596706144e-08
4708 2.06376101796746e-08
4709 2.04632696081397e-08
4710 2.05744713803746e-08
4711 2.04495646256575e-08
4712 2.05088458131719e-08
4713 2.05960871446642e-08
4714 2.05377009476759e-08
4715 2.04472047231619e-08
4716 2.05301003712854e-08
4717 2.0501893503333e-08
4718 2.0483970155194e-08
4719 2.04935940377737e-08
4720 2.05065301095431e-08
4721 2.05049911903288e-08
4722 2.05272245836774e-08
4723 2.04796623841652e-08
4724 2.0591497012834e-08
4725 2.04351097996547e-08
4726 2.04547881890171e-08
4727 2.05550483469708e-08
4728 2.04346726291327e-08
4729 2.05295899280422e-08
4730 2.0468800522977e-08
4731 2.05055880906357e-08
4732 2.05275551699025e-08
4733 2.04868301918459e-08
4734 2.05360606071459e-08
4735 2.04685871718624e-08
4736 2.04628931301798e-08
4737 2.06389611969726e-08
4738 2.04458880639358e-08
4739 2.04551413904852e-08
4740 2.0481203212519e-08
4741 2.048200124527e-08
4742 2.05391200078786e-08
4743 2.04346087167018e-08
4744 2.04149664329023e-08
4745 2.04728753478989e-08
4746 2.05003399362802e-08
4747 2.04571977344692e-08
4748 2.04946293922426e-08
4749 2.04293996479699e-08
4750 2.05208054895323e-08
4751 2.04519667996728e-08
4752 2.04315785672371e-08
4753 2.05045352044131e-08
4754 2.04809451003207e-08
4755 2.04509949637277e-08
4756 2.04320460204244e-08
4757 2.0515720312364e-08
4758 2.04683897160329e-08
4759 2.04164136077267e-08
4760 2.05496432876817e-08
4761 2.038990730302e-08
4762 2.05344062718105e-08
4763 2.04030363288865e-08
4764 2.04532363698995e-08
4765 2.03857950431541e-08
4766 2.04990124075266e-08
4767 2.03692023559299e-08
4768 2.05943154498733e-08
4769 2.04016162665965e-08
4770 2.04719339844672e-08
4771 2.04189616561656e-08
4772 2.04284892029172e-08
4773 2.042176801198e-08
4774 2.04096726377223e-08
4775 2.04909666083886e-08
4776 2.03983700819421e-08
4777 2.04487004311638e-08
4778 2.04080122001571e-08
4779 2.04639499084003e-08
4780 2.04040700082508e-08
4781 2.04665610015198e-08
4782 2.0331005079921e-08
4783 2.05264265180638e-08
4784 2.04982780691587e-08
4785 2.03829079965523e-08
4786 2.04335685012325e-08
4787 2.03866184316226e-08
4788 2.04112551158708e-08
4789 2.04864599355759e-08
4790 2.03778241050223e-08
4791 2.04388915427955e-08
4792 2.04531014955656e-08
4793 2.0365125306121e-08
4794 2.04664981029445e-08
4795 2.04470886850938e-08
4796 2.0316205872728e-08
4797 2.04807365049575e-08
4798 2.04372952881116e-08
4799 2.03896874055864e-08
4800 2.04088572903771e-08
4801 2.04870787277045e-08
4802 2.03738216302263e-08
4803 2.04125467426763e-08
4804 2.038617078437e-08
4805 2.04537628283319e-08
4806 2.03858696377068e-08
4807 2.03865588135343e-08
4808 2.03919816355036e-08
4809 2.04398016365737e-08
4810 2.02973527052563e-08
4811 2.04602003197607e-08
4812 2.03984393389867e-08
4813 2.04425807506237e-08
4814 2.03411392223352e-08
4815 2.0402558022603e-08
4816 2.04179915193059e-08
4817 2.04050691978708e-08
4818 2.0366762744306e-08
4819 2.03304668997539e-08
4820 2.04058449813083e-08
4821 2.03437549872731e-08
4822 2.04896049769143e-08
4823 2.03675731635933e-08
4824 2.03423423745797e-08
4825 2.03435794947637e-08
4826 2.04025167573896e-08
4827 2.03768128828052e-08
4828 2.03918767525124e-08
4829 2.03372167542071e-08
4830 2.03743377387156e-08
4831 2.03731348236147e-08
4832 2.03883340477162e-08
4833 2.03743458473404e-08
4834 2.03781589425134e-08
4835 2.03406581240628e-08
4836 2.03906141695853e-08
4837 2.03541724634881e-08
4838 2.03241137683996e-08
4839 2.04023830510103e-08
4840 2.03409969383728e-08
4841 2.03868959420816e-08
4842 2.04215824042286e-08
4843 2.03365969699831e-08
4844 2.03183189642076e-08
4845 2.03865818226845e-08
4846 2.04147289779577e-08
4847 2.03184662224132e-08
4848 2.0361721922324e-08
4849 2.03733550758756e-08
4850 2.0300578471133e-08
4851 2.03843964663442e-08
4852 2.03126068893944e-08
4853 2.03236638109949e-08
4854 2.04478043617229e-08
4855 2.03188370710983e-08
4856 2.03124540685273e-08
4857 2.03507529961122e-08
4858 2.03060044254855e-08
4859 2.03532234253068e-08
4860 2.03037570254416e-08
4861 2.03254254627083e-08
4862 2.03958343718558e-08
4863 2.03027610665707e-08
4864 2.03188948173505e-08
4865 2.03399952845018e-08
4866 2.02908347057829e-08
4867 2.03318657496787e-08
4868 2.03034376808908e-08
4869 2.03855859668423e-08
4870 2.03526179958224e-08
4871 2.04148060647391e-08
4872 2.02347792810365e-08
4873 2.02727116702661e-08
4874 2.03361004249558e-08
4875 2.03037003569939e-08
4876 2.02838846674602e-08
4877 2.03539809358055e-08
4878 2.03245125875995e-08
4879 2.03280043802323e-08
4880 2.02777560343037e-08
4881 2.03366647779646e-08
4882 2.02988667474813e-08
4883 2.034470141421e-08
4884 2.02644055113943e-08
4885 2.02766786898678e-08
4886 2.03502502036379e-08
4887 2.0222134323955e-08
4888 2.03161592988721e-08
4889 2.02753609759299e-08
4890 2.02963669235778e-08
4891 2.0276884766357e-08
4892 2.02971266820562e-08
4893 2.02825956758801e-08
4894 2.02960430208954e-08
4895 2.02584482269508e-08
4896 2.03169193766506e-08
4897 2.02860097555657e-08
4898 2.02920807348406e-08
4899 2.02305565277783e-08
4900 2.03037009591789e-08
4901 2.03423482676435e-08
4902 2.02081067590676e-08
4903 2.03647542136309e-08
4904 2.01923458247322e-08
4905 2.03371484894799e-08
4906 2.02345741673327e-08
4907 2.02561091824371e-08
4908 2.03148475597104e-08
4909 2.02273128635522e-08
4910 2.02422413075887e-08
4911 2.03012578459116e-08
4912 2.02669533080346e-08
4913 2.02587934632348e-08
4914 2.03025633678244e-08
4915 2.02457610440909e-08
4916 2.02224985286215e-08
4917 2.02824005302027e-08
4918 2.02340486947783e-08
4919 2.02736468759568e-08
4920 2.02399795194985e-08
4921 2.03419537583294e-08
4922 2.02190458598039e-08
4923 2.02442595260699e-08
4924 2.03459508136827e-08
4925 2.00805793659775e-08
4926 2.034907181514e-08
4927 2.01853651176087e-08
4928 2.02402880868924e-08
4929 2.02142350755885e-08
4930 2.03108739889046e-08
4931 2.01912465369602e-08
4932 2.02170391010448e-08
4933 2.02441363943429e-08
4934 2.01759559743664e-08
4935 2.02562371907078e-08
4936 2.01661698135958e-08
4937 2.02499251871213e-08
4938 2.01702522062419e-08
4939 2.02762080254715e-08
4940 2.01865172058113e-08
4941 2.01847540348687e-08
4942 2.01752652522202e-08
4943 2.0316697956213e-08
4944 2.01534882233734e-08
4945 2.02174557197843e-08
4946 2.01713579071061e-08
4947 2.01946578242129e-08
4948 2.02171483953961e-08
4949 2.0163964502995e-08
4950 2.02050516593388e-08
4951 2.0202309662487e-08
4952 2.02017830641665e-08
4953 2.02045131194595e-08
4954 2.01455194948075e-08
4955 2.02298684310875e-08
4956 2.02407901612744e-08
4957 2.01231688561343e-08
4958 2.01881701604734e-08
4959 2.01976388640723e-08
4960 2.01859578403685e-08
4961 2.01397657417779e-08
4962 2.01803662656097e-08
4963 2.01222864113504e-08
4964 2.04488201407393e-08
4965 2.0059063217559e-08
4966 2.01420761616333e-08
4967 2.01756195536973e-08
4968 2.0156069080457e-08
4969 2.0228407581202e-08
4970 2.0154437372355e-08
4971 2.01281161764655e-08
4972 2.02539931282164e-08
4973 2.00831666901102e-08
4974 2.02244769869075e-08
4975 2.01369470660406e-08
4976 2.01014534360056e-08
4977 2.0198765624091e-08
4978 2.01140618436568e-08
4979 2.02388542955845e-08
4980 2.00736591584594e-08
4981 2.01329217359714e-08
4982 2.02038111847269e-08
4983 2.01175326044467e-08
4984 2.01150703111885e-08
4985 2.01768013923242e-08
4986 2.01122756489092e-08
4987 2.02315101129891e-08
4988 2.0064957462429e-08
4989 2.01452728756379e-08
4990 2.00708154944884e-08
4991 2.01374270374366e-08
4992 2.01167336855157e-08
4993 2.01641327284285e-08
4994 2.01318325574462e-08
4995 2.01222238969123e-08
4996 2.01564090298589e-08
4997 2.00926491984443e-08
4998 2.01192292106356e-08
4999 2.02602398757801e-08
};
\addlegendentry{Train}
\addplot [semithick, black]
table {%
0 0.00262331450358033
1 0.00164683803450316
2 0.000853968260344118
3 0.000310535455355421
4 0.00024511877563782
5 0.000240352848777547
6 0.000238824679399841
7 0.000237510757870041
8 0.0002358603378525
9 0.000233356462558731
10 0.000229113196837716
11 0.000221084424993023
12 0.000204312309506349
13 0.000167458536452614
14 9.07256180653349e-05
15 3.44903637596872e-05
16 1.96956298168516e-05
17 1.67535472428426e-05
18 1.44057266879827e-05
19 1.20349704957334e-05
20 1.01429550340981e-05
21 8.5787842181162e-06
22 7.40323685022304e-06
23 6.47741262582713e-06
24 5.71735017729225e-06
25 5.09170604345854e-06
26 4.57565874967258e-06
27 4.15444128520903e-06
28 3.80554138246225e-06
29 3.50655341208039e-06
30 3.25129394695978e-06
31 3.020728172487e-06
32 2.81477423413889e-06
33 2.62768594438967e-06
34 2.45170554080687e-06
35 2.29379566007992e-06
36 2.14892475014494e-06
37 2.01298439606035e-06
38 1.88047170013306e-06
39 1.75954039605131e-06
40 1.64854532158643e-06
41 1.54650183503691e-06
42 1.4531668739437e-06
43 1.36694029606588e-06
44 1.28542410493537e-06
45 1.2102256050639e-06
46 1.14001215933968e-06
47 1.075162572306e-06
48 1.01557895959559e-06
49 9.61341925176384e-07
50 9.11060226371774e-07
51 8.64840330905281e-07
52 8.22878973849583e-07
53 7.85486975019012e-07
54 7.51863524328655e-07
55 7.2020151264951e-07
56 6.92714650085691e-07
57 6.67978554247384e-07
58 6.46560863515333e-07
59 6.27314534540346e-07
60 6.09936193995964e-07
61 5.9524938933464e-07
62 5.81633116780722e-07
63 5.68922416732676e-07
64 5.56714439881034e-07
65 5.46227795439336e-07
66 5.37050368620839e-07
67 5.30403951870539e-07
68 5.22534776337125e-07
69 5.17181717896165e-07
70 5.11263692715147e-07
71 5.0596787559698e-07
72 5.00834914873849e-07
73 4.95913809572812e-07
74 4.911910878036e-07
75 4.86599844862212e-07
76 4.8213757963822e-07
77 4.77642345231288e-07
78 4.73294505809463e-07
79 4.69107931166945e-07
80 4.65046383624212e-07
81 4.61107674709638e-07
82 4.57301666756393e-07
83 4.53658401511348e-07
84 4.50277127583831e-07
85 4.47115951374144e-07
86 4.44002296262624e-07
87 4.41047717458787e-07
88 4.38211969822078e-07
89 4.35475328686152e-07
90 4.32671299677168e-07
91 4.30133042073066e-07
92 4.27176956918629e-07
93 4.25151739591456e-07
94 4.2279984313609e-07
95 4.20451272020728e-07
96 4.18258366607915e-07
97 4.16166528793838e-07
98 4.13992466974378e-07
99 4.11503151553916e-07
100 4.09408926316246e-07
101 4.07356310461182e-07
102 4.05355223165316e-07
103 4.03612176569368e-07
104 4.01918839543214e-07
105 3.99849454879586e-07
106 3.98599667050803e-07
107 3.97194384049726e-07
108 3.95719609969092e-07
109 3.94421590499405e-07
110 3.93118170904927e-07
111 3.91865114579559e-07
112 3.90583750231599e-07
113 3.89479197338005e-07
114 3.87301241744353e-07
115 3.86072343872002e-07
116 3.82861287562264e-07
117 3.81720866471369e-07
118 3.80427650270576e-07
119 3.78876791273797e-07
120 3.77523150518755e-07
121 3.76285157699385e-07
122 3.76365591137073e-07
123 3.75349287651261e-07
124 3.74244905287924e-07
125 3.73081149973586e-07
126 3.72339826526513e-07
127 3.71839888657632e-07
128 3.70770038671253e-07
129 3.69291029755914e-07
130 3.68793962479685e-07
131 3.6716159002026e-07
132 3.66282705499543e-07
133 3.6536073366733e-07
134 3.64134422170537e-07
135 3.6330393982098e-07
136 3.63393866109618e-07
137 3.62457228675339e-07
138 3.61623051503557e-07
139 3.60856120096287e-07
140 3.60298997748032e-07
141 3.59540536010172e-07
142 3.58198747107963e-07
143 3.57251138893844e-07
144 3.57045166765602e-07
145 3.56257800149251e-07
146 3.55535263452111e-07
147 3.54912856437295e-07
148 3.54208509634191e-07
149 3.53602132463493e-07
150 3.5295059319651e-07
151 3.52221633193039e-07
152 3.51712287738337e-07
153 3.50837552787198e-07
154 3.5024663702643e-07
155 3.50152077999155e-07
156 3.49026890944515e-07
157 3.48621398416071e-07
158 3.48172505937328e-07
159 3.47533472222494e-07
160 3.46940964845999e-07
161 3.46258076433514e-07
162 3.45687084291058e-07
163 3.44407766306176e-07
164 3.43611816333578e-07
165 3.42632972660795e-07
166 3.41728679131847e-07
167 3.40937305054467e-07
168 3.40124159947663e-07
169 3.39340715527214e-07
170 3.38695429036306e-07
171 3.37842266162625e-07
172 3.37093382540843e-07
173 3.36343873641454e-07
174 3.35541102458592e-07
175 3.34851193883878e-07
176 3.34084802489087e-07
177 3.33355473003394e-07
178 3.32579304540559e-07
179 3.31795746433272e-07
180 3.31213840354394e-07
181 3.30322791342041e-07
182 3.29666249854199e-07
183 3.29008997823621e-07
184 3.28384260228631e-07
185 3.27763331142705e-07
186 3.2699068697184e-07
187 3.26347930013071e-07
188 3.25728166217232e-07
189 3.25114768884305e-07
190 3.24300998499893e-07
191 3.23684218983544e-07
192 3.22689373888352e-07
193 3.22000147434665e-07
194 3.21139822290206e-07
195 3.20520598506846e-07
196 3.19843081797444e-07
197 3.1910477105157e-07
198 3.18445728453298e-07
199 3.17707502972553e-07
200 3.1691178037363e-07
201 3.16165881031338e-07
202 3.15494418146045e-07
203 3.14963642722432e-07
204 3.14117613697817e-07
205 3.13422475528569e-07
206 3.12820105818901e-07
207 3.1218095841723e-07
208 3.1145142997957e-07
209 3.10501491185278e-07
210 3.09913758655966e-07
211 3.09031236156443e-07
212 3.08538602666886e-07
213 3.0767313319302e-07
214 3.07099924157228e-07
215 3.06526374060923e-07
216 3.05827683177995e-07
217 3.05028663660778e-07
218 3.0446020105046e-07
219 3.03800675283128e-07
220 3.03116223676625e-07
221 3.02604235002946e-07
222 3.01902019828049e-07
223 3.01415525427728e-07
224 3.00729794844301e-07
225 3.00061799407558e-07
226 2.99421401450672e-07
227 2.98626588346451e-07
228 2.98448412650032e-07
229 2.9750245289506e-07
230 2.96908496011383e-07
231 2.96348503070476e-07
232 2.9570659876299e-07
233 2.95098232072633e-07
234 2.94310154913546e-07
235 2.93661003070156e-07
236 2.93034389642344e-07
237 2.92417411174029e-07
238 2.91570671606678e-07
239 2.90947980374767e-07
240 2.90057613483441e-07
241 2.89455442725739e-07
242 2.88764880451708e-07
243 2.87993231040673e-07
244 2.86986079345297e-07
245 2.86300576135545e-07
246 2.8556885922626e-07
247 2.84823471474738e-07
248 2.84090788227331e-07
249 2.83579680626644e-07
250 2.82857314459761e-07
251 2.81797184698007e-07
252 2.81114068911847e-07
253 2.80447153500063e-07
254 2.7956309622823e-07
255 2.78939182862814e-07
256 2.78212581861226e-07
257 2.77550896043977e-07
258 2.76720669489805e-07
259 2.76005039268057e-07
260 2.75231229807105e-07
261 2.74409870826275e-07
262 2.73998864486202e-07
263 2.73085447588528e-07
264 2.72302116854917e-07
265 2.7136883318235e-07
266 2.70793407253223e-07
267 2.69824226961646e-07
268 2.69023502141863e-07
269 2.68309548800971e-07
270 2.67102365114624e-07
271 2.66427349515652e-07
272 2.65536016286205e-07
273 2.64462272525634e-07
274 2.63512617948436e-07
275 2.62353069047094e-07
276 2.61513179111716e-07
277 2.60571226817774e-07
278 2.59748986763952e-07
279 2.58771194694418e-07
280 2.57689947602557e-07
281 2.56960646538573e-07
282 2.55848021879501e-07
283 2.54850647252169e-07
284 2.53866602406561e-07
285 2.52499773978343e-07
286 2.51447602295229e-07
287 2.50377581778594e-07
288 2.4904946371862e-07
289 2.47999082603201e-07
290 2.46839249484765e-07
291 2.45902526785358e-07
292 2.44650692593495e-07
293 2.43668353050452e-07
294 2.42619876189565e-07
295 2.40904427073474e-07
296 2.39490219655636e-07
297 2.38156658838307e-07
298 2.36725412605665e-07
299 2.35303616591409e-07
300 2.34041849012101e-07
301 2.32766737440215e-07
302 2.31364296610082e-07
303 2.29882715530039e-07
304 2.28784372779955e-07
305 2.27401500296764e-07
306 2.26213515475138e-07
307 2.24940833959408e-07
308 2.23382428998775e-07
309 2.21852928916633e-07
310 2.20465310007967e-07
311 2.18583551259144e-07
312 2.17046945749644e-07
313 2.15560760352673e-07
314 2.14240429841084e-07
315 2.12460832926809e-07
316 2.11035043662378e-07
317 2.09389583005759e-07
318 2.07876482249958e-07
319 2.06174703976103e-07
320 2.04697343519911e-07
321 2.03040116275588e-07
322 2.01636282781692e-07
323 1.99706235548547e-07
324 1.98173353282982e-07
325 1.96637500948782e-07
326 1.95460827967509e-07
327 1.94346597481854e-07
328 1.93388700608921e-07
329 1.91749279565556e-07
330 1.90507265074302e-07
331 1.89293785979316e-07
332 1.876781681176e-07
333 1.86446641237126e-07
334 1.85263161256444e-07
335 1.83984667501136e-07
336 1.82562686745769e-07
337 1.81377245667136e-07
338 1.8073534135965e-07
339 1.78974076447957e-07
340 1.77287176938989e-07
341 1.75734612639644e-07
342 1.74778250539021e-07
343 1.73538168724008e-07
344 1.72083630900488e-07
345 1.70916479191874e-07
346 1.70075438177264e-07
347 1.68873057759811e-07
348 1.675570331372e-07
349 1.6655933166021e-07
350 1.65641750982104e-07
351 1.64398286983669e-07
352 1.63382992468541e-07
353 1.62439093287503e-07
354 1.61443537649575e-07
355 1.60732199105951e-07
356 1.60027852302846e-07
357 1.5918787710234e-07
358 1.58020228013811e-07
359 1.57250170218504e-07
360 1.56176739096736e-07
361 1.55686620928464e-07
362 1.54716190081672e-07
363 1.54007651076427e-07
364 1.53710132622109e-07
365 1.53031436411766e-07
366 1.52578195411479e-07
367 1.51988970742423e-07
368 1.51763984490572e-07
369 1.50852997649054e-07
370 1.50604307691538e-07
371 1.50277202237703e-07
372 1.4989913665886e-07
373 1.49567839002884e-07
374 1.49314246300492e-07
375 1.48863037452429e-07
376 1.48641419173146e-07
377 1.48592420146088e-07
378 1.4835156036952e-07
379 1.48195312021926e-07
380 1.47712427178703e-07
381 1.47413032891563e-07
382 1.46899196806771e-07
383 1.46947670032205e-07
384 1.46765671615867e-07
385 1.46532443068281e-07
386 1.4651752167083e-07
387 1.46051149840787e-07
388 1.45861037026407e-07
389 1.45685177699306e-07
390 1.45626486869332e-07
391 1.45421822139724e-07
392 1.45160583997495e-07
393 1.45010488949993e-07
394 1.44713936833796e-07
395 1.4458248642768e-07
396 1.44376230082344e-07
397 1.44345307262483e-07
398 1.44340873475812e-07
399 1.4381429025434e-07
400 1.43828415843927e-07
401 1.43781193173709e-07
402 1.43604211189086e-07
403 1.43534236940468e-07
404 1.43181878797805e-07
405 1.43165564736591e-07
406 1.42723010299051e-07
407 1.42869410524327e-07
408 1.42497555088994e-07
409 1.42664376312496e-07
410 1.42228401500688e-07
411 1.42066014063857e-07
412 1.42199482411343e-07
413 1.42141473702395e-07
414 1.41560022370868e-07
415 1.41450001933663e-07
416 1.41224745675572e-07
417 1.41269154596557e-07
418 1.41030369604778e-07
419 1.4111132884409e-07
420 1.40905569878669e-07
421 1.40576389640046e-07
422 1.40852165486649e-07
423 1.40366296363936e-07
424 1.4013659210832e-07
425 1.40199588827272e-07
426 1.40087252020749e-07
427 1.39898503448421e-07
428 1.39910710572622e-07
429 1.39609497296078e-07
430 1.39299487500466e-07
431 1.3941507859272e-07
432 1.39507221774693e-07
433 1.39077627636652e-07
434 1.38864407972505e-07
435 1.39141903332529e-07
436 1.38883990530303e-07
437 1.38692115569938e-07
438 1.38629971502269e-07
439 1.38432710627967e-07
440 1.37987413495466e-07
441 1.38340041644369e-07
442 1.38062915766568e-07
443 1.38338066335564e-07
444 1.3772452689409e-07
445 1.37548596512715e-07
446 1.37791317911251e-07
447 1.3762220874014e-07
448 1.37504002850619e-07
449 1.36859611643558e-07
450 1.37233769237355e-07
451 1.37126548338529e-07
452 1.36949111606555e-07
453 1.36668319328237e-07
454 1.36288875296486e-07
455 1.36838920639093e-07
456 1.36502634973112e-07
457 1.36335970069013e-07
458 1.36314611154376e-07
459 1.35857149530239e-07
460 1.36206566025976e-07
461 1.35850669380488e-07
462 1.35636895493008e-07
463 1.356849423928e-07
464 1.35610278562126e-07
465 1.35484057750546e-07
466 1.35145626245503e-07
467 1.34852200517344e-07
468 1.35124693656508e-07
469 1.3484414296272e-07
470 1.34637531346016e-07
471 1.34745647528689e-07
472 1.34646398919358e-07
473 1.34106059590522e-07
474 1.3429728085157e-07
475 1.3407844789981e-07
476 1.33928708123676e-07
477 1.33707587224308e-07
478 1.33645798428006e-07
479 1.33450782868749e-07
480 1.33488384790326e-07
481 1.33287755943456e-07
482 1.3312742908056e-07
483 1.32905768168712e-07
484 1.33414516767516e-07
485 1.32754749415653e-07
486 1.32692548504565e-07
487 1.32570164623758e-07
488 1.32382254491858e-07
489 1.32145117959226e-07
490 1.32143213704694e-07
491 1.31971575001444e-07
492 1.31649230183939e-07
493 1.31611471942961e-07
494 1.31892576860082e-07
495 1.30722142444029e-07
496 1.31234074274289e-07
497 1.30753221583291e-07
498 1.30945565501861e-07
499 1.30682238363988e-07
500 1.30786176555375e-07
501 1.30164849565517e-07
502 1.30096381667499e-07
503 1.29940289639308e-07
504 1.29798578996088e-07
505 1.29773084722729e-07
506 1.29536431359156e-07
507 1.28784705566432e-07
508 1.28826869172372e-07
509 1.28499166862639e-07
510 1.28194798776349e-07
511 1.2853952569003e-07
512 1.27854633547031e-07
513 1.27803133409543e-07
514 1.27742083577687e-07
515 1.27474152122886e-07
516 1.27127961491169e-07
517 1.27264954130624e-07
518 1.27104655689436e-07
519 1.26943945133462e-07
520 1.26713160852887e-07
521 1.26648032505727e-07
522 1.26530423472104e-07
523 1.26295674363064e-07
524 1.26046657555889e-07
525 1.25966806763245e-07
526 1.25925851079955e-07
527 1.25696061559211e-07
528 1.25599228795181e-07
529 1.25320340771395e-07
530 1.25466740996671e-07
531 1.25483808233184e-07
532 1.2548454719763e-07
533 1.24941962553748e-07
534 1.24998791761755e-07
535 1.25017209029465e-07
536 1.24934004475108e-07
537 1.24175684845795e-07
538 1.24274180279826e-07
539 1.24017248026576e-07
540 1.23764053228115e-07
541 1.2383115688408e-07
542 1.23842113453065e-07
543 1.23565726539709e-07
544 1.23489272141342e-07
545 1.23513757444016e-07
546 1.23212231528669e-07
547 1.2311434716139e-07
548 1.23161939313832e-07
549 1.23188101497362e-07
550 1.22863582419086e-07
551 1.22532142654563e-07
552 1.22781543154815e-07
553 1.22700157589861e-07
554 1.22382900258344e-07
555 1.22261695878478e-07
556 1.22121477375003e-07
557 1.22268687618998e-07
558 1.22042393968513e-07
559 1.22313366546223e-07
560 1.22121036838507e-07
561 1.22077807418464e-07
562 1.2204979782382e-07
563 1.21804177410922e-07
564 1.21896263749477e-07
565 1.21908414030258e-07
566 1.21582687029331e-07
567 1.21681978271226e-07
568 1.21293552979296e-07
569 1.21244823958477e-07
570 1.21146214837609e-07
571 1.210663214124e-07
572 1.21149497545048e-07
573 1.21023461474579e-07
574 1.20847261086965e-07
575 1.2069332910869e-07
576 1.20566909345143e-07
577 1.20398269132238e-07
578 1.20542068771101e-07
579 1.20418178539694e-07
580 1.20455041496825e-07
581 1.20549799476066e-07
582 1.20821724181042e-07
583 1.20505333711662e-07
584 1.20754748422769e-07
585 1.20527488434163e-07
586 1.20071860010285e-07
587 1.20358109256813e-07
588 1.19924777663982e-07
589 1.20390126312486e-07
590 1.19830502853802e-07
591 1.20326788533021e-07
592 1.19529588005207e-07
593 1.1920262465992e-07
594 1.19684926858099e-07
595 1.19717583402235e-07
596 1.19249705221591e-07
597 1.19451286195726e-07
598 1.19017528277254e-07
599 1.19232950623882e-07
600 1.18697805362444e-07
601 1.18917085956127e-07
602 1.1818693224086e-07
603 1.18778444857526e-07
604 1.1845876457528e-07
605 1.18557849759782e-07
606 1.18568387108553e-07
607 1.18124596326652e-07
608 1.18335812260284e-07
609 1.17723914172529e-07
610 1.17758943929402e-07
611 1.17865255333527e-07
612 1.17401917520965e-07
613 1.17387244813472e-07
614 1.17555700285266e-07
615 1.16944796957341e-07
616 1.17445893010881e-07
617 1.16865749077988e-07
618 1.16685882289858e-07
619 1.16370657110565e-07
620 1.16810305428316e-07
621 1.16766955216008e-07
622 1.17395146048693e-07
623 1.16830548790858e-07
624 1.17068928773278e-07
625 1.16478737766101e-07
626 1.16819968809523e-07
627 1.16206813061126e-07
628 1.16675330730232e-07
629 1.15941034550815e-07
630 1.16362372182266e-07
631 1.15881981344046e-07
632 1.161551992368e-07
633 1.15494792396476e-07
634 1.15964112978872e-07
635 1.15533318023608e-07
636 1.15977933035083e-07
637 1.1535546207142e-07
638 1.15543031142806e-07
639 1.15213019569183e-07
640 1.15534412259422e-07
641 1.15006066891965e-07
642 1.15247253518191e-07
643 1.14863581757163e-07
644 1.1508215180811e-07
645 1.14552491936593e-07
646 1.14982711352241e-07
647 1.14262789452368e-07
648 1.14833426323457e-07
649 1.14237828086061e-07
650 1.14662029204737e-07
651 1.13948928515129e-07
652 1.14525242622676e-07
653 1.13783194422012e-07
654 1.14331896838848e-07
655 1.13520464140038e-07
656 1.14103052339942e-07
657 1.13405022261759e-07
658 1.13840016524591e-07
659 1.13167693882588e-07
660 1.13932095757718e-07
661 1.13181435779097e-07
662 1.12971385135552e-07
663 1.13093697962086e-07
664 1.12708654853577e-07
665 1.13075330432366e-07
666 1.12690123899029e-07
667 1.1256916820912e-07
668 1.12787510886392e-07
669 1.12313422562238e-07
670 1.12553109943292e-07
671 1.12126848250682e-07
672 1.12471170155004e-07
673 1.12043757383162e-07
674 1.11724695273097e-07
675 1.12027805698744e-07
676 1.11755049658768e-07
677 1.11643529976391e-07
678 1.11372123967612e-07
679 1.11448677841963e-07
680 1.1130426003092e-07
681 1.11205487485222e-07
682 1.11095246779769e-07
683 1.1092650709088e-07
684 1.10932056429647e-07
685 1.10867446778684e-07
686 1.10812621301193e-07
687 1.10623908256002e-07
688 1.10603970426837e-07
689 1.10518577400853e-07
690 1.10385705909266e-07
691 1.10246979545536e-07
692 1.10308704392992e-07
693 1.10209441572806e-07
694 1.10151212595611e-07
695 1.10024366506423e-07
696 1.09991781016561e-07
697 1.09797731795425e-07
698 1.09644332724201e-07
699 1.097682726936e-07
700 1.09682900983898e-07
701 1.09594083141928e-07
702 1.09528045300067e-07
703 1.09384870938811e-07
704 1.09271276471645e-07
705 1.0918007831151e-07
706 1.09240545498324e-07
707 1.09049459240396e-07
708 1.09130283476588e-07
709 1.08855715552636e-07
710 1.08880456650695e-07
711 1.08789919295305e-07
712 1.08713635427193e-07
713 1.08506931439933e-07
714 1.08602819182124e-07
715 1.08429716760838e-07
716 1.08425595612971e-07
717 1.08348949368064e-07
718 1.08231624551536e-07
719 1.08026718237397e-07
720 1.08035990820099e-07
721 1.08029105660989e-07
722 1.07893157519356e-07
723 1.07829976059293e-07
724 1.07769196233676e-07
725 1.07773182378423e-07
726 1.07773672652911e-07
727 1.07656894954289e-07
728 1.07427048590125e-07
729 1.07512995839443e-07
730 1.07456259001992e-07
731 1.07252979830719e-07
732 1.07264213511371e-07
733 1.07348618882952e-07
734 1.07131356230639e-07
735 1.06952199985244e-07
736 1.06931040022573e-07
737 1.06933249810481e-07
738 1.06876093752817e-07
739 1.06781129716182e-07
740 1.06704241886746e-07
741 1.06603238236858e-07
742 1.06942309230362e-07
743 1.06917404707474e-07
744 1.06917973141663e-07
745 1.06819449285922e-07
746 1.06965401869275e-07
747 1.06951702605329e-07
748 1.06614550077211e-07
749 1.06674228561587e-07
750 1.06419825840476e-07
751 1.06458557525002e-07
752 1.06445284586698e-07
753 1.06438818647803e-07
754 1.06241230923843e-07
755 1.06271038191608e-07
756 1.06131167854073e-07
757 1.05929252924852e-07
758 1.06040097591631e-07
759 1.05878939393733e-07
760 1.05959315988002e-07
761 1.05943421147003e-07
762 1.05804922156949e-07
763 1.05752576473606e-07
764 1.05728688026829e-07
765 1.05492908630822e-07
766 1.05514807557938e-07
767 1.05349442947045e-07
768 1.05326591892663e-07
769 1.05183886489613e-07
770 1.05060983912608e-07
771 1.0507996250908e-07
772 1.04898710162615e-07
773 1.04920609089731e-07
774 1.04927003974353e-07
775 1.04746973761394e-07
776 1.0473929279442e-07
777 1.04551396873376e-07
778 1.0458698795901e-07
779 1.04508679044102e-07
780 1.0446423459598e-07
781 1.04391084221334e-07
782 1.04350256435737e-07
783 1.03970840825696e-07
784 1.04258511157695e-07
785 1.04346668194921e-07
786 1.04268565337406e-07
787 1.04318345961474e-07
788 1.04110057463913e-07
789 1.04228476516255e-07
790 1.04151439472844e-07
791 1.04078395679608e-07
792 1.04063239803054e-07
793 1.03896482528398e-07
794 1.03909727044993e-07
795 1.03758644343088e-07
796 1.03653036376272e-07
797 1.03775008142293e-07
798 1.0367958225288e-07
799 1.03591503375355e-07
800 1.03366353698675e-07
801 1.04219772367742e-07
802 1.03275375806788e-07
803 1.03233659842772e-07
804 1.03126936323861e-07
805 1.03062937739651e-07
806 1.03077248070349e-07
807 1.02974503590758e-07
808 1.02841191562675e-07
809 1.02837368842756e-07
810 1.02823335623725e-07
811 1.02816414937479e-07
812 1.02841831051137e-07
813 1.02719646122296e-07
814 1.03622504354917e-07
815 1.02645287824998e-07
816 1.02621633857325e-07
817 1.02519130962264e-07
818 1.02109233068859e-07
819 1.02143538072141e-07
820 1.02152789338561e-07
821 1.02073464347541e-07
822 1.02920353128866e-07
823 1.02147048153256e-07
824 1.01799209062392e-07
825 1.0233685543426e-07
826 1.0174118614259e-07
827 1.02252563749516e-07
828 1.01646122629973e-07
829 1.01621751014136e-07
830 1.01571963284641e-07
831 1.02504159826822e-07
832 1.01653135686774e-07
833 1.01433009547236e-07
834 1.01242584094052e-07
835 1.0148779949759e-07
836 1.01277755959472e-07
837 1.01285813514096e-07
838 1.02427073045419e-07
839 1.01511666628085e-07
840 1.0113968329506e-07
841 1.00933064572928e-07
842 1.01591034251669e-07
843 1.01646797645571e-07
844 1.02158161041643e-07
845 1.01150909870285e-07
846 1.00983861273107e-07
847 1.0108227144201e-07
848 1.01266159902025e-07
849 1.01320765111268e-07
850 1.00702145289233e-07
851 1.0094779412384e-07
852 1.00588593454631e-07
853 1.01041273126157e-07
854 1.0041063802646e-07
855 1.00750504827829e-07
856 1.00340173503355e-07
857 1.01589208156838e-07
858 1.00507165257113e-07
859 1.00643106293319e-07
860 1.00478004583238e-07
861 1.00791474721973e-07
862 1.00676182057668e-07
863 1.00475979536441e-07
864 1.00985481310545e-07
865 1.00098624500333e-07
866 1.00220809429175e-07
867 1.00040161044035e-07
868 1.00469058850194e-07
869 1.0001121353298e-07
870 1.00176535511309e-07
871 1.00487305587649e-07
872 1.0009094353336e-07
873 9.98951605879483e-08
874 9.98456002321291e-08
875 1.00047536477632e-07
876 9.98503395521766e-08
877 9.97488811549374e-08
878 9.97551268255847e-08
879 9.97574645111854e-08
880 9.90872663919617e-08
881 9.94938957887825e-08
882 9.94375426444094e-08
883 9.96634383909623e-08
884 9.92575479585867e-08
885 9.9245355045241e-08
886 9.94103075413477e-08
887 9.94510358509615e-08
888 1.00066557706668e-07
889 9.91180115761381e-08
890 9.87857973200335e-08
891 9.89286235153486e-08
892 9.8842377838082e-08
893 9.88578889860037e-08
894 9.87896697779433e-08
895 9.85542314424492e-08
896 9.86697372695744e-08
897 9.87673445251858e-08
898 9.95047102492208e-08
899 9.88786439393152e-08
900 9.88190365092123e-08
901 9.86871242503184e-08
902 9.9335622394392e-08
903 9.88683765967835e-08
904 9.81878542916093e-08
905 9.83477690397194e-08
906 9.91454669474479e-08
907 9.8259718583904e-08
908 9.80278329620887e-08
909 9.89919684002416e-08
910 9.83616814664856e-08
911 9.77558443082671e-08
912 9.88898136711214e-08
913 9.83339134563721e-08
914 9.76932881258108e-08
915 9.88466055673598e-08
916 9.84173524898324e-08
917 9.80354784019255e-08
918 9.73891332023413e-08
919 9.86275736636344e-08
920 9.80877530309954e-08
921 9.85605694836522e-08
922 9.76552172460288e-08
923 9.84359047606631e-08
924 9.7662514519925e-08
925 9.72202798266153e-08
926 9.82348709044345e-08
927 9.75652483248268e-08
928 9.71292948293012e-08
929 9.74326326286246e-08
930 9.81131762500809e-08
931 9.77673124680223e-08
932 9.72395781673185e-08
933 9.80317693688448e-08
934 9.74115081930904e-08
935 9.69360129943198e-08
936 9.65990807344497e-08
937 9.72247917729874e-08
938 9.67529487638785e-08
939 9.65046140777304e-08
940 9.68049249649994e-08
941 9.75318386053914e-08
942 9.69030793385173e-08
943 9.62773682999796e-08
944 9.73746310251045e-08
945 9.67682680652615e-08
946 9.71987503817218e-08
947 9.68900053521793e-08
948 9.65139079767141e-08
949 9.72305542745744e-08
950 9.64154835969566e-08
951 9.65011608400346e-08
952 9.6548816941322e-08
953 9.69457971677912e-08
954 9.66693605164437e-08
955 9.59282786539006e-08
956 9.58141157525461e-08
957 9.57828447667453e-08
958 9.5644487885238e-08
959 9.60700319296848e-08
960 9.59233830144512e-08
961 9.6079077138711e-08
962 9.67978124322144e-08
963 9.61005000021942e-08
964 9.60117816362072e-08
965 9.58011128204816e-08
966 9.5730662508231e-08
967 9.55635997001991e-08
968 9.53929060187875e-08
969 9.54885663873029e-08
970 9.54370733552423e-08
971 9.51354408584848e-08
972 9.61549773137449e-08
973 9.58139096951527e-08
974 9.50450811387782e-08
975 9.55268149027688e-08
976 9.52910568230436e-08
977 9.52518064423202e-08
978 9.5132961064337e-08
979 9.48493195096489e-08
980 9.51428731355008e-08
981 9.50288878698302e-08
982 9.59673869260769e-08
983 9.52609653381842e-08
984 9.51076728483713e-08
985 9.58104635628843e-08
986 9.53083585386594e-08
987 9.46803595525125e-08
988 9.56713677169319e-08
989 9.5232032037984e-08
990 9.58005941242845e-08
991 9.49644345382694e-08
992 9.58294066322196e-08
993 9.51682821437316e-08
994 9.45157765386284e-08
995 9.46314031580187e-08
996 9.44738189900818e-08
997 9.48026581681916e-08
998 9.46131351042823e-08
999 9.55791676915396e-08
1000 9.4717961474089e-08
1001 9.46893976561114e-08
1002 9.44157747539975e-08
1003 9.52011305344058e-08
1004 9.48953413626441e-08
1005 9.41398710097019e-08
1006 9.42719822205618e-08
1007 9.39562099233626e-08
1008 9.41947391197573e-08
1009 9.51358742895536e-08
1010 9.44657401191762e-08
1011 9.50386223053101e-08
1012 9.43278593013019e-08
1013 9.41106037544159e-08
1014 9.39405566668938e-08
1015 9.38184001597619e-08
1016 9.4726765098585e-08
1017 9.42110531809703e-08
1018 9.45862979051526e-08
1019 9.39881985573265e-08
1020 9.37590343141892e-08
1021 9.36497741577114e-08
1022 9.40309305974552e-08
1023 9.38342310519147e-08
1024 9.37599295980363e-08
1025 9.34305077748832e-08
1026 9.35448483119217e-08
1027 9.36489357172832e-08
1028 9.35750605890462e-08
1029 9.33351245180347e-08
1030 9.35234183430111e-08
1031 9.34837274257916e-08
1032 9.32863457592248e-08
1033 9.34546164899075e-08
1034 9.32149148979988e-08
1035 9.31407271309581e-08
1036 9.39902022878414e-08
1037 9.34742558911239e-08
1038 9.39107920316928e-08
1039 9.33475163833464e-08
1040 9.30352541672619e-08
1041 9.31061805431455e-08
1042 9.28710051084636e-08
1043 9.27330319200337e-08
1044 9.30367178852975e-08
1045 9.27572045839042e-08
1046 9.29655712411659e-08
1047 9.27130159311673e-08
1048 9.27482943779978e-08
1049 9.27298415831501e-08
1050 9.27435976905144e-08
1051 9.24048961792323e-08
1052 9.25604339840902e-08
1053 9.24372400845641e-08
1054 9.24762559861847e-08
1055 9.2512365768016e-08
1056 9.23989205148246e-08
1057 9.24808176705483e-08
1058 9.23293157484295e-08
1059 9.23783929351885e-08
1060 9.26019438907133e-08
1061 9.22902430033901e-08
1062 9.22868181874037e-08
1063 9.21446030588413e-08
1064 9.24001000157659e-08
1065 9.21232867767685e-08
1066 9.20483813615647e-08
1067 9.20421499017721e-08
1068 9.19197802318195e-08
1069 9.18237361702268e-08
1070 9.1838046500925e-08
1071 9.191838046263e-08
1072 9.19501488283458e-08
1073 9.17876761263869e-08
1074 9.18363411983592e-08
1075 9.17512181786151e-08
1076 9.17166502745204e-08
1077 9.16953126761655e-08
1078 9.15747477847617e-08
1079 9.16869566935929e-08
1080 9.15667826006938e-08
1081 9.15697881964661e-08
1082 9.15333586704037e-08
1083 9.16017484087206e-08
1084 9.144971357955e-08
1085 9.14587729994309e-08
1086 9.13697633109223e-08
1087 9.1438963067958e-08
1088 9.13054662987633e-08
1089 9.1310823790991e-08
1090 9.12292037469342e-08
1091 9.09454342945537e-08
1092 9.09319055608648e-08
1093 9.11309854245701e-08
1094 9.15308078219823e-08
1095 9.15175561999604e-08
1096 9.1020190495783e-08
1097 9.09887205580162e-08
1098 9.08986947933954e-08
1099 9.0791807849655e-08
1100 9.14493014647633e-08
1101 9.07989488041494e-08
1102 9.08029562651791e-08
1103 9.14013043029627e-08
1104 9.1258030465724e-08
1105 9.09209987298709e-08
1106 9.07270489847178e-08
1107 9.0587249701457e-08
1108 9.06572736880662e-08
1109 9.11701150130284e-08
1110 9.0940829977626e-08
1111 9.12316053813811e-08
1112 9.09673900650887e-08
1113 9.05181849475412e-08
1114 9.06384869381327e-08
1115 9.04238888210784e-08
1116 9.0291806031928e-08
1117 8.99521950259441e-08
1118 9.02403343161495e-08
1119 9.07679549300155e-08
1120 9.04462424955454e-08
1121 9.00032901540726e-08
1122 9.08140762589937e-08
1123 9.04740957707872e-08
1124 9.12674664732549e-08
1125 9.04328203432669e-08
1126 9.03483012848483e-08
1127 9.08363588791872e-08
1128 8.99489478456417e-08
1129 8.98908041335744e-08
1130 9.0617291448325e-08
1131 9.00309728990578e-08
1132 8.99203271842453e-08
1133 8.97477789862933e-08
1134 9.02041819017541e-08
1135 9.03003325447571e-08
1136 8.97382648190614e-08
1137 8.97831000656879e-08
1138 9.02181582773665e-08
1139 9.00362238098751e-08
1140 8.92041782663e-08
1141 8.9782048462439e-08
1142 8.9829200078384e-08
1143 8.99597267789431e-08
1144 8.99143373089828e-08
1145 8.97048835213354e-08
1146 8.92789913109482e-08
1147 8.94809275564512e-08
1148 8.94322411681969e-08
1149 8.9405141068255e-08
1150 8.92525591211779e-08
1151 8.97247858233641e-08
1152 8.92951348419047e-08
1153 8.91332945229806e-08
1154 8.92781670813747e-08
1155 8.91718556772503e-08
1156 8.94473473067592e-08
1157 8.90912730255877e-08
1158 8.89867450837301e-08
1159 8.9188084473335e-08
1160 8.88333744342162e-08
1161 8.96480116807652e-08
1162 8.91845104433742e-08
1163 8.89755753519239e-08
1164 9.05563055653147e-08
1165 8.91594424956565e-08
1166 8.90132625386286e-08
1167 8.91920706180827e-08
1168 8.9242526257749e-08
1169 8.90093403427272e-08
1170 8.93028584414424e-08
1171 8.86146409584398e-08
1172 8.88729516645981e-08
1173 8.90194087332929e-08
1174 8.84126976075095e-08
1175 8.84234339082468e-08
1176 8.87784210590326e-08
1177 8.89928131186934e-08
1178 8.8340044612778e-08
1179 8.82726709505732e-08
1180 8.81920172446371e-08
1181 8.8693070665613e-08
1182 8.85384352500296e-08
1183 8.84500224174189e-08
1184 8.85196485000961e-08
1185 8.83793092043561e-08
1186 8.83124684492032e-08
1187 8.86535644895048e-08
1188 8.80183392837353e-08
1189 8.82997994722245e-08
1190 8.80333459463145e-08
1191 8.82850201833207e-08
1192 8.83325910194799e-08
1193 8.81149091469524e-08
1194 8.83703705767402e-08
1195 8.7801019788003e-08
1196 8.81407657971067e-08
1197 8.81503723348942e-08
1198 8.85788935534038e-08
1199 8.79180817037195e-08
1200 8.84499868902822e-08
1201 8.78176180663104e-08
1202 8.8224481942234e-08
1203 8.78718253716215e-08
1204 8.79904646922114e-08
1205 8.75134560374136e-08
1206 8.78292425454674e-08
1207 8.7892459532668e-08
1208 8.74062351385874e-08
1209 8.77750565564384e-08
1210 8.82107116240149e-08
1211 8.75428440849646e-08
1212 8.75180177217771e-08
1213 8.75159500424161e-08
1214 8.74022276775577e-08
1215 8.74263719197188e-08
1216 8.76960939422133e-08
1217 8.76645458447456e-08
1218 8.68357403760456e-08
1219 8.74743477652373e-08
1220 8.75977406167294e-08
1221 8.78562218531442e-08
1222 8.71258478696291e-08
1223 8.70618137582824e-08
1224 8.71629168841537e-08
1225 8.70290080001723e-08
1226 8.71912817501652e-08
1227 8.69037677375673e-08
1228 8.67578009433601e-08
1229 8.69226539634838e-08
1230 8.70639169647802e-08
1231 8.6896868367603e-08
1232 8.74464731737135e-08
1233 8.66877627458962e-08
1234 8.71032241889225e-08
1235 8.7378111857106e-08
1236 8.63493738734178e-08
1237 8.62963318581933e-08
1238 8.64675797629388e-08
1239 8.63047233679026e-08
1240 8.65734222088577e-08
1241 8.6302684110251e-08
1242 8.61616626934847e-08
1243 8.6176591196363e-08
1244 8.6306577884443e-08
1245 8.70175753675539e-08
1246 8.58867252873097e-08
1247 8.61565041532231e-08
1248 8.60481677023017e-08
1249 8.59245261608521e-08
1250 8.58137880754839e-08
1251 8.65896083723783e-08
1252 8.60516564671343e-08
1253 8.60792113144271e-08
1254 8.56493045375828e-08
1255 8.61681783703716e-08
1256 8.6700296719755e-08
1257 8.58132551684321e-08
1258 8.58588080632217e-08
1259 8.58469775266713e-08
1260 8.58434106021377e-08
1261 8.57457820302443e-08
1262 8.57202735460305e-08
1263 8.63756568492136e-08
1264 8.51185362193974e-08
1265 8.59709032852152e-08
1266 8.55884039197008e-08
1267 8.50374561878198e-08
1268 8.56969535334429e-08
1269 8.55431494528602e-08
1270 8.49267181024516e-08
1271 8.65250413539798e-08
1272 8.53841726211613e-08
1273 8.48549817078492e-08
1274 8.55559108003945e-08
1275 8.46709582447147e-08
1276 8.46115497665778e-08
1277 8.54604991218366e-08
1278 8.48596357627684e-08
1279 8.530552264574e-08
1280 8.46784615760043e-08
1281 8.47437178208565e-08
1282 8.47629380018589e-08
1283 8.53014086032999e-08
1284 8.55061585980366e-08
1285 8.48933225938708e-08
1286 8.42335694528629e-08
1287 8.42856380245394e-08
1288 8.51791099876209e-08
1289 8.43876719613945e-08
1290 8.4746389461543e-08
1291 8.42097662712149e-08
1292 8.48064232172874e-08
1293 8.42769978248725e-08
1294 8.45375822677852e-08
1295 8.4109942122268e-08
1296 8.48401668918086e-08
1297 8.41511109683779e-08
1298 8.42762801767094e-08
1299 8.40530418599883e-08
1300 8.40824370129667e-08
1301 8.40181542116625e-08
1302 8.44290326540431e-08
1303 8.36162641348892e-08
1304 8.43298266772763e-08
1305 8.35422184763956e-08
1306 8.44164489421928e-08
1307 8.39839131572262e-08
1308 8.43447054421631e-08
1309 8.36259061998135e-08
1310 8.37127984709696e-08
1311 8.35820088695982e-08
1312 8.4279271561627e-08
1313 8.4187831816962e-08
1314 8.41168983356511e-08
1315 8.35808933175031e-08
1316 8.38949745229911e-08
1317 8.37779552398388e-08
1318 8.34278068850836e-08
1319 8.31742639206823e-08
1320 8.40205984786735e-08
1321 8.32512156989651e-08
1322 8.3629373648364e-08
1323 8.35885884953314e-08
1324 8.36055136232972e-08
1325 8.3769712944104e-08
1326 8.30126438700063e-08
1327 8.31658155675541e-08
1328 8.2858051086987e-08
1329 8.35284481581766e-08
1330 8.27886239562758e-08
1331 8.42312033455528e-08
1332 8.42745464524342e-08
1333 8.26666806119647e-08
1334 8.36357827438405e-08
1335 8.27134911673966e-08
1336 8.3322127863994e-08
1337 8.24999375481639e-08
1338 8.32814137652349e-08
1339 8.33521269782977e-08
1340 8.35748821259585e-08
1341 8.31664976885804e-08
1342 8.26254265007265e-08
1343 8.31105282372846e-08
1344 8.30416269081979e-08
1345 8.26966441991317e-08
1346 8.22706596181888e-08
1347 8.28819892717547e-08
1348 8.20848740090696e-08
1349 8.29575625971302e-08
1350 8.21912351511855e-08
1351 8.23871530997167e-08
1352 8.19774541582774e-08
1353 8.28165482857912e-08
1354 8.19223799908286e-08
1355 8.27144717163719e-08
1356 8.20450551941576e-08
1357 8.22361982955044e-08
1358 8.17162515431846e-08
1359 8.25952994887302e-08
1360 8.16364860156682e-08
1361 8.24842487645583e-08
1362 8.17049823353955e-08
1363 8.26167720902049e-08
1364 8.18484195974634e-08
1365 8.20907644083491e-08
1366 8.18380812006581e-08
1367 8.13571503499588e-08
1368 8.22300023628486e-08
1369 8.13512457398247e-08
1370 8.23833161689436e-08
1371 8.12395555271905e-08
1372 8.2184449468059e-08
1373 8.14637957091691e-08
1374 8.21399765982278e-08
1375 8.11462541605579e-08
1376 8.19761822867804e-08
1377 8.11130078659517e-08
1378 8.19195591361677e-08
1379 8.12353206924854e-08
1380 8.18706098471012e-08
1381 8.16484941879025e-08
1382 8.16608860532142e-08
1383 8.09229305787085e-08
1384 8.17499810068512e-08
1385 8.08321303225057e-08
1386 8.17082295156979e-08
1387 8.07134838964885e-08
1388 8.14953295957821e-08
1389 8.05896220867908e-08
1390 8.13505494079436e-08
1391 8.05249413815545e-08
1392 8.19137611074439e-08
1393 8.05279540827541e-08
1394 8.11528551025731e-08
1395 8.04935709197707e-08
1396 8.11710094694718e-08
1397 8.04189994596527e-08
1398 8.10909810411431e-08
1399 8.0218818254707e-08
1400 8.11214562190798e-08
1401 8.01733932576099e-08
1402 8.113067906379e-08
1403 8.00907145048768e-08
1404 8.08593298984306e-08
1405 8.01696558028198e-08
1406 8.06552478138656e-08
1407 7.99632218217994e-08
1408 8.07450817319477e-08
1409 7.99261243855653e-08
1410 8.09499454135221e-08
1411 7.98377044475274e-08
1412 8.08534892371426e-08
1413 7.97828434428993e-08
1414 8.09119313771589e-08
1415 7.97638293192904e-08
1416 8.04981326041343e-08
1417 7.96672097180817e-08
1418 8.05415965032807e-08
1419 7.99922190708457e-08
1420 8.04727235959035e-08
1421 8.09658686762305e-08
1422 8.04418789357442e-08
1423 7.9725154478183e-08
1424 8.05107802648308e-08
1425 7.94496912703835e-08
1426 8.05295243822002e-08
1427 7.94451580077293e-08
1428 8.11475402429096e-08
1429 7.96240513523117e-08
1430 8.08127893492383e-08
1431 7.94881671595249e-08
1432 7.99302739551422e-08
1433 7.95883394744124e-08
1434 8.00938693146236e-08
1435 8.10748161939046e-08
1436 8.00558694891151e-08
1437 7.99999000378193e-08
1438 8.03588875442074e-08
1439 8.03396815740598e-08
1440 8.06469415692845e-08
1441 8.06287872023859e-08
1442 8.09104463428412e-08
1443 8.06192872460088e-08
1444 8.07583191431149e-08
1445 8.04167825663171e-08
1446 8.08448916700399e-08
1447 8.08897482329485e-08
1448 8.02831507940027e-08
1449 8.05923789926055e-08
1450 8.08566440468894e-08
1451 8.16430940631108e-08
1452 8.01020121343754e-08
1453 8.07589302098677e-08
1454 8.01087125523736e-08
1455 8.09447939786878e-08
1456 8.00584345483912e-08
1457 8.07896896048987e-08
1458 8.00928816602209e-08
1459 8.08391007467435e-08
1460 8.00715795890028e-08
1461 8.11471778661144e-08
1462 8.00829909053391e-08
1463 8.06546580633949e-08
1464 8.01263766447846e-08
1465 8.06279984999492e-08
1466 7.99784771743361e-08
1467 8.05002997594784e-08
1468 7.99193529132936e-08
1469 8.06137308018151e-08
1470 7.98392107981272e-08
1471 8.03398592097437e-08
1472 7.98190527007137e-08
1473 8.02959831958105e-08
1474 7.97755888015672e-08
1475 8.02486468387542e-08
1476 7.9729566948572e-08
1477 8.02427635449021e-08
1478 7.95381893681224e-08
1479 8.01076964762615e-08
1480 7.94362335909682e-08
1481 8.01148090090464e-08
1482 7.95442076650943e-08
1483 8.02686699330479e-08
1484 7.93837386936502e-08
1485 7.99829749098535e-08
1486 7.92332031096521e-08
1487 7.99783776983531e-08
1488 7.91588234960727e-08
1489 7.993001105433e-08
1490 7.91530823107678e-08
1491 7.98847992200535e-08
1492 7.90516097026739e-08
1493 8.00002410983325e-08
1494 7.91926737520043e-08
1495 7.92325707266173e-08
1496 7.91677834399707e-08
1497 7.91768073327148e-08
1498 7.91321710380544e-08
1499 8.00670107992119e-08
1500 8.00146864321505e-08
1501 7.99306576482195e-08
1502 7.99275170493274e-08
1503 8.00071902062882e-08
1504 7.99578998567085e-08
1505 7.98722226136306e-08
1506 7.98863908357816e-08
1507 7.91559742197023e-08
1508 7.91036924852051e-08
1509 7.98232093757179e-08
1510 7.91516328035868e-08
1511 8.06832147759451e-08
1512 7.9094014893144e-08
1513 8.08781734917829e-08
1514 7.9124454543944e-08
1515 7.92694194728938e-08
1516 7.94375054624652e-08
1517 7.96689931803485e-08
1518 7.92303254115723e-08
1519 7.95122829799766e-08
1520 7.9446998313415e-08
1521 7.92495242762925e-08
1522 7.96258703417152e-08
1523 7.9484870241231e-08
1524 7.94217029920219e-08
1525 7.93610439586701e-08
1526 7.95560666233541e-08
1527 7.92663072957112e-08
1528 7.92891157175291e-08
1529 7.92265026916539e-08
1530 7.93767185314209e-08
1531 7.913051547348e-08
1532 7.90828380559105e-08
1533 7.91544110256837e-08
1534 7.90771750303065e-08
1535 7.90378891224464e-08
1536 7.90732812561146e-08
1537 7.89927128153067e-08
1538 7.90020351359999e-08
1539 7.88953613550802e-08
1540 7.89773935139237e-08
1541 7.88325706935211e-08
1542 7.89171181736492e-08
1543 7.90980365650285e-08
1544 7.9192190582944e-08
1545 7.90516665460927e-08
1546 7.91296486113424e-08
1547 7.91120413623503e-08
1548 7.91878065342644e-08
1549 7.91962619928199e-08
1550 7.91828327351141e-08
1551 7.92085401712939e-08
1552 7.92002552429949e-08
1553 7.91621985740676e-08
1554 7.92115102399293e-08
1555 7.91258756294155e-08
1556 7.91382177567357e-08
1557 7.91468934835393e-08
1558 7.91187986237674e-08
1559 7.91153951240631e-08
1560 7.9068321667819e-08
1561 7.98412855829156e-08
1562 7.90468206446349e-08
1563 7.90655079185854e-08
1564 7.89455611993617e-08
1565 7.90367593594965e-08
1566 7.89844989412813e-08
1567 7.89467335948757e-08
1568 7.89219782859618e-08
1569 7.88969245490989e-08
1570 7.88493821346492e-08
1571 7.89012233326503e-08
1572 7.88512792837537e-08
1573 7.9068520619785e-08
1574 7.87741498697869e-08
1575 7.87556757586572e-08
1576 7.87024703186034e-08
1577 7.8701745565013e-08
1578 7.86013529818774e-08
1579 7.8643587642091e-08
1580 7.86055736057278e-08
1581 7.87751162079076e-08
1582 7.8728888297519e-08
1583 7.85718867746255e-08
1584 7.85103040357171e-08
1585 7.84584273105793e-08
1586 7.84621789762241e-08
1587 7.83801112902438e-08
1588 7.83943363558137e-08
1589 7.82735085635977e-08
1590 7.8302385020379e-08
1591 7.82119755626809e-08
1592 7.81282309958442e-08
1593 7.81609088562618e-08
1594 7.81794540216652e-08
1595 7.80824720436613e-08
1596 7.8093137290125e-08
1597 7.80316185000629e-08
1598 7.79906770276284e-08
1599 7.79073658918605e-08
1600 7.79220030722172e-08
1601 7.78453639327381e-08
1602 7.78821771518778e-08
1603 7.78382442945258e-08
1604 7.77897355419555e-08
1605 7.77372690663469e-08
1606 7.76886110998021e-08
1607 7.76749047304293e-08
1608 7.76267370383721e-08
1609 7.75961481735976e-08
1610 7.75385018414454e-08
1611 7.75513910866721e-08
1612 7.74988890839268e-08
1613 7.7446458135455e-08
1614 7.7279750598791e-08
1615 7.72873534060636e-08
1616 7.73670265630244e-08
1617 7.73547057519863e-08
1618 7.72795587522523e-08
1619 7.72812285276814e-08
1620 7.71941728316961e-08
1621 7.71711796687669e-08
1622 7.71174200053792e-08
1623 7.71505881402845e-08
1624 7.71430137547213e-08
1625 7.71326114090698e-08
1626 7.70720447462736e-08
1627 7.70400561123097e-08
1628 7.70048416143254e-08
1629 7.69953771850851e-08
1630 7.69893233609764e-08
1631 7.70405534922247e-08
1632 7.70211698863932e-08
1633 7.69999175531666e-08
1634 7.69451702353763e-08
1635 7.69319612459185e-08
1636 7.69493482266626e-08
1637 7.68353061175731e-08
1638 7.68585053378956e-08
1639 7.68239161175188e-08
1640 7.67774110954633e-08
1641 7.67156933534352e-08
1642 7.67111956179178e-08
1643 7.669873980376e-08
1644 7.66490586556756e-08
1645 7.66123164908095e-08
1646 7.65648806577701e-08
1647 7.65623724419129e-08
1648 7.6707522111974e-08
1649 7.65346115372267e-08
1650 7.64403011999093e-08
1651 7.64083978310737e-08
1652 7.63666676562025e-08
1653 7.6350026745331e-08
1654 7.62805072440642e-08
1655 7.62817506938518e-08
1656 7.62175318413938e-08
1657 7.62057865699717e-08
1658 7.61700107432262e-08
1659 7.61682343863868e-08
1660 7.6072964816376e-08
1661 7.60939400379357e-08
1662 7.5874446281432e-08
1663 7.59940661509972e-08
1664 7.59593916654921e-08
1665 7.59072946721062e-08
1666 7.58385567678488e-08
1667 7.58054596872171e-08
1668 7.58122951083351e-08
1669 7.57835110221095e-08
1670 7.57362101921899e-08
1671 7.57205640411485e-08
1672 7.56591518324967e-08
1673 7.56337854568301e-08
1674 7.56316822503322e-08
1675 7.55718687628359e-08
1676 7.5526308762619e-08
1677 7.54975033601113e-08
1678 7.54328155494477e-08
1679 7.53808393483268e-08
1680 7.53374962414455e-08
1681 7.53272644260505e-08
1682 7.54845572714657e-08
1683 7.52571835960225e-08
1684 7.52190487673943e-08
1685 7.51687991851213e-08
1686 7.51150821542979e-08
1687 7.5293932866316e-08
1688 7.50652873193758e-08
1689 7.50614503886027e-08
1690 7.50543662775272e-08
1691 7.49701882796217e-08
1692 7.4954279227768e-08
1693 7.49064454907966e-08
1694 7.48255857274671e-08
1695 7.49302628832993e-08
1696 7.48031467878718e-08
1697 7.48135349226686e-08
1698 7.47465378481138e-08
1699 7.4709902264658e-08
1700 7.46423793884787e-08
1701 7.4653968340499e-08
1702 7.45650936551101e-08
1703 7.45771657761907e-08
1704 7.44741228686507e-08
1705 7.45216226505363e-08
1706 7.4513430092793e-08
1707 7.44607291380817e-08
1708 7.43868469044173e-08
1709 7.43789954071872e-08
1710 7.43412513770636e-08
1711 7.43444914519387e-08
1712 7.42975956313785e-08
1713 7.42139292242427e-08
1714 7.42057224556447e-08
1715 7.41188728170528e-08
1716 7.408507940454e-08
1717 7.40996810577599e-08
1718 7.39954089112871e-08
1719 7.40010719368911e-08
1720 7.39274383931843e-08
1721 7.39596686116784e-08
1722 7.38736005700957e-08
1723 7.38518011189626e-08
1724 7.38292840196664e-08
1725 7.37989651611315e-08
1726 7.37203436074196e-08
1727 7.37148582174996e-08
1728 7.36599332640253e-08
1729 7.36170733262043e-08
1730 7.35902290216472e-08
1731 7.3554637936013e-08
1732 7.34782261702094e-08
1733 7.34514244982165e-08
1734 7.34270457769526e-08
1735 7.33966558641441e-08
1736 7.33541796194004e-08
1737 7.32828624450121e-08
1738 7.33389455831457e-08
1739 7.32149274540461e-08
1740 7.32706766370939e-08
1741 7.32111473666919e-08
1742 7.31462463932075e-08
1743 7.31035498802157e-08
1744 7.32735969677378e-08
1745 7.30173326246586e-08
1746 7.3031884539887e-08
1747 7.29224609585799e-08
1748 7.29394145082551e-08
1749 7.28760625179348e-08
1750 7.277687785745e-08
1751 7.28008302530725e-08
1752 7.27645854681214e-08
1753 7.28289677454086e-08
1754 7.26575279941244e-08
1755 7.26094242509134e-08
1756 7.2547997831407e-08
1757 7.2554655616841e-08
1758 7.25197821793699e-08
1759 7.24685236264122e-08
1760 7.24002404695057e-08
1761 7.2440542453478e-08
1762 7.23964888038608e-08
1763 7.23659638879326e-08
1764 7.23832727089757e-08
1765 7.23280138004156e-08
1766 7.23385369383323e-08
1767 7.22375688155807e-08
1768 7.22712556466831e-08
1769 7.2232467118738e-08
1770 7.22018569376814e-08
1771 7.2108619519895e-08
1772 7.20932575859479e-08
1773 7.20190840297619e-08
1774 7.19986843478182e-08
1775 7.19015886829766e-08
1776 7.19153234740588e-08
1777 7.18692163559354e-08
1778 7.18897865681356e-08
1779 7.17827219887113e-08
1780 7.17764052637904e-08
1781 7.17454682330754e-08
1782 7.17185528742448e-08
1783 7.16677774903474e-08
1784 7.15776735660256e-08
1785 7.16210806217532e-08
1786 7.15350694235894e-08
1787 7.15551635721567e-08
1788 7.02283742270993e-08
1789 7.0099652305089e-08
1790 7.00647788676179e-08
1791 6.99940372328456e-08
1792 6.98512891972314e-08
1793 6.98861697401298e-08
1794 6.97739466204439e-08
1795 6.98323177061866e-08
1796 6.97416950856677e-08
1797 6.97432867013958e-08
1798 6.96807873623584e-08
1799 6.96311133197014e-08
1800 6.94511541610154e-08
1801 6.95130779604369e-08
1802 6.94518575983238e-08
1803 6.94163446723906e-08
1804 6.93721347033716e-08
1805 6.93109569738226e-08
1806 6.93207908852855e-08
1807 6.94806487899768e-08
1808 6.97911630709314e-08
1809 6.99794142633436e-08
1810 7.02967710708435e-08
1811 7.02953215636626e-08
1812 7.05195617456411e-08
1813 7.03192384321483e-08
1814 6.94937583034516e-08
1815 6.95344084533644e-08
1816 6.9191607110497e-08
1817 6.89027004341369e-08
1818 6.92125965429113e-08
1819 6.9075966280252e-08
1820 6.94336321771516e-08
1821 6.89831978206712e-08
1822 6.90401620317971e-08
1823 6.83307348481321e-08
1824 6.82077114788626e-08
1825 6.82367584659005e-08
1826 6.78174103541096e-08
1827 6.75918272463605e-08
1828 6.64773764924576e-08
1829 6.59789662904586e-08
1830 6.52732268235923e-08
1831 6.46275140070429e-08
1832 6.4254322751367e-08
1833 6.36326475955684e-08
1834 6.31518233262796e-08
1835 6.24405700477837e-08
1836 6.16768218719699e-08
1837 6.10087909080903e-08
1838 6.024898624446e-08
1839 5.96957008269783e-08
1840 5.91665560989441e-08
1841 5.8972307925842e-08
1842 5.84544288528832e-08
1843 5.81116168518747e-08
1844 5.78658898575668e-08
1845 5.73709755258278e-08
1846 5.70332403526663e-08
1847 5.65609568070613e-08
1848 5.61693838108113e-08
1849 5.5634068019117e-08
1850 5.52505596829178e-08
1851 5.47069376466425e-08
1852 5.44049250095213e-08
1853 5.37344178042076e-08
1854 5.35127036016547e-08
1855 5.33724175966199e-08
1856 5.26456069849246e-08
1857 5.24015604241868e-08
1858 5.22959133775203e-08
1859 5.19914422625334e-08
1860 5.18149931849621e-08
1861 5.17474738614965e-08
1862 5.14896214554028e-08
1863 5.1228706610118e-08
1864 5.14376949922735e-08
1865 5.12009314945772e-08
1866 5.11146858173106e-08
1867 5.10115967244928e-08
1868 5.08261308596047e-08
1869 5.02593131557205e-08
1870 5.00333001696163e-08
1871 4.99852035318327e-08
1872 4.98869603404728e-08
1873 4.96448961939677e-08
1874 4.95232761466013e-08
1875 4.93422795955212e-08
1876 4.94328489253348e-08
1877 4.92732574741694e-08
1878 4.91142735370431e-08
1879 4.89772027378876e-08
1880 4.88352647209922e-08
1881 4.86500901786258e-08
1882 4.86090279139262e-08
1883 4.85422333440511e-08
1884 4.85264557426035e-08
1885 4.84655728882899e-08
1886 4.81902553417513e-08
1887 4.81059814205764e-08
1888 4.80925024248791e-08
1889 4.80851092277135e-08
1890 4.78284079008517e-08
1891 4.78729162978198e-08
1892 4.7627818133833e-08
1893 4.75940957755938e-08
1894 4.74460577493119e-08
1895 4.74029668851017e-08
1896 4.72114223271092e-08
1897 4.71662673362516e-08
1898 4.68103742434778e-08
1899 4.64733922456162e-08
1900 4.64645317777013e-08
1901 4.63973393038941e-08
1902 4.61348719227317e-08
1903 4.61149731734167e-08
1904 4.61058213829801e-08
1905 4.59191014101634e-08
1906 4.5626293854184e-08
1907 4.5641801449392e-08
1908 4.54470949762253e-08
1909 4.52761348412878e-08
1910 4.53795827581871e-08
1911 4.50266774976171e-08
1912 4.49053736417682e-08
1913 4.48558097332352e-08
1914 4.46479155868929e-08
1915 4.45800196757773e-08
1916 4.45981314101118e-08
1917 4.44004975008738e-08
1918 4.42883987261666e-08
1919 4.4065110671454e-08
1920 4.39496723458888e-08
1921 4.38296474669642e-08
1922 4.38231602117867e-08
1923 4.36677893844717e-08
1924 4.36005045401089e-08
1925 4.33912425990002e-08
1926 4.32558664442695e-08
1927 4.32028386398997e-08
1928 4.3075271349835e-08
1929 4.29685123037871e-08
1930 4.28215010117583e-08
1931 4.27392983226582e-08
1932 4.24603854298766e-08
1933 4.23328643250898e-08
1934 4.21018384599847e-08
1935 4.21093488967017e-08
1936 4.21083647950127e-08
1937 4.17872279001585e-08
1938 4.16149035231683e-08
1939 4.15894412242324e-08
1940 4.15438137224555e-08
1941 4.1339671241758e-08
1942 4.12746565814359e-08
1943 4.12182288300755e-08
1944 4.09735179118798e-08
1945 4.08587901290502e-08
1946 4.08046290090169e-08
1947 4.0659045907887e-08
1948 4.06015203679999e-08
1949 4.05228988142881e-08
1950 4.04325177782994e-08
1951 4.01629201007836e-08
1952 4.01789712611844e-08
1953 3.99754185309575e-08
1954 3.97540276253494e-08
1955 3.96626838039538e-08
1956 3.97144290786855e-08
1957 3.96470021257755e-08
1958 3.88024936626152e-08
1959 3.8856793338482e-08
1960 3.87162693016307e-08
1961 3.84662115493484e-08
1962 3.83704552575637e-08
1963 3.839564755026e-08
1964 3.80723612636302e-08
1965 3.81424349882309e-08
1966 3.81332334598028e-08
1967 3.81390989900865e-08
1968 3.79254707638665e-08
1969 3.80774523023319e-08
1970 3.78866218397889e-08
1971 3.79048188392517e-08
1972 3.76228932452705e-08
1973 3.75385162953989e-08
1974 3.74705422245825e-08
1975 3.72987116747936e-08
1976 3.75361146609521e-08
1977 3.73187916125062e-08
1978 3.72063269082901e-08
1979 3.73009214627018e-08
1980 3.68929278238284e-08
1981 3.68235646419635e-08
1982 3.6521964119629e-08
1983 3.65762318210727e-08
1984 3.6381415213782e-08
1985 3.64749475068038e-08
1986 3.61894372247207e-08
1987 3.61729490805374e-08
1988 3.62226622030448e-08
1989 3.5994641933712e-08
1990 3.58846570236437e-08
1991 3.57013263396766e-08
1992 3.56840601511976e-08
1993 3.5650298713108e-08
1994 3.52396511971165e-08
1995 3.53498812444286e-08
1996 3.5377993867769e-08
1997 3.51810029997068e-08
1998 3.520105451571e-08
1999 3.50478117638886e-08
2000 3.50445361618767e-08
2001 3.48507391834119e-08
2002 3.48728868004855e-08
2003 3.4898498313396e-08
2004 3.46200685896747e-08
2005 3.44764394810682e-08
2006 3.46179831467452e-08
2007 3.46166828535388e-08
2008 3.45524142630893e-08
2009 3.41794539338025e-08
2010 3.40868098192004e-08
2011 3.42157200350357e-08
2012 3.43110144740422e-08
2013 3.42564376865084e-08
2014 3.41674528669955e-08
2015 3.42176953438411e-08
2016 3.41506662948632e-08
2017 3.40430190703955e-08
2018 3.38696928281479e-08
2019 3.39663728254891e-08
2020 3.39121477566096e-08
2021 3.3797000753566e-08
2022 3.38640795405354e-08
2023 3.35679253282706e-08
2024 3.37094903102297e-08
2025 3.36198873185367e-08
2026 3.35975407494971e-08
2027 3.36266410272401e-08
2028 3.35961551911623e-08
2029 3.34811218749564e-08
2030 3.33648593198177e-08
2031 3.35496572745342e-08
2032 3.31888756477383e-08
2033 3.33673533248202e-08
2034 3.35329808365259e-08
2035 3.32937162283997e-08
2036 3.32515455170324e-08
2037 3.32928600244031e-08
2038 3.31172032019822e-08
2039 3.29832587908641e-08
2040 3.28887068690165e-08
2041 3.31123573005243e-08
2042 3.29807470222931e-08
2043 3.29965494927364e-08
2044 3.29889324746091e-08
2045 3.30375335977351e-08
2046 3.31699077094072e-08
2047 3.30335829801243e-08
2048 3.29009210986442e-08
2049 3.29244436159115e-08
2050 3.29866267634316e-08
2051 3.2861745324908e-08
2052 3.28835589868959e-08
2053 3.28502842705802e-08
2054 3.27552243106766e-08
2055 3.27283835588332e-08
2056 3.28313198849628e-08
2057 3.24259481487843e-08
2058 3.2302800434536e-08
2059 3.24615569979869e-08
2060 3.27174127789931e-08
2061 3.26685274387728e-08
2062 3.2380182091174e-08
2063 3.23447366668006e-08
2064 3.24422479991426e-08
2065 3.25340856477396e-08
2066 3.24495950110304e-08
2067 3.25025233394172e-08
2068 3.22940465480315e-08
2069 3.25028999270671e-08
2070 3.24137658935797e-08
2071 3.24317426247944e-08
2072 3.21809956460584e-08
2073 3.23218820597049e-08
2074 3.24086713021643e-08
2075 3.24144622254607e-08
2076 3.22502771155087e-08
2077 3.23135012081366e-08
2078 3.21458522023477e-08
2079 3.20061737113519e-08
2080 3.2118528281444e-08
2081 3.20747979287717e-08
2082 3.21043636120066e-08
2083 3.23080442399259e-08
2084 3.19228590228704e-08
2085 3.24687654540412e-08
2086 3.23288062986649e-08
2087 3.22642641492621e-08
2088 3.1854508364404e-08
2089 3.21230935185213e-08
2090 3.2095400115395e-08
2091 3.21237543232655e-08
2092 3.21079802745317e-08
2093 3.2137663197318e-08
2094 3.21925206492324e-08
2095 3.21596331787077e-08
2096 3.19333040010861e-08
2097 3.21659712199107e-08
2098 3.21372120026808e-08
2099 3.20964979039218e-08
2100 3.23714530736652e-08
2101 3.19497210909958e-08
2102 3.20698561040444e-08
2103 3.20171977818973e-08
2104 3.2019844553588e-08
2105 3.19155013528416e-08
2106 3.18968425006005e-08
2107 3.20461523983795e-08
2108 3.20447810508995e-08
2109 3.19960555827947e-08
2110 3.21963469218645e-08
2111 3.19554551708734e-08
2112 3.16842694303432e-08
2113 3.20033066714132e-08
2114 3.20157340638616e-08
2115 3.18435127155681e-08
2116 3.18918722541639e-08
2117 3.14620329788795e-08
2118 3.16693800073153e-08
2119 3.15538777329039e-08
2120 3.12094385890305e-08
2121 3.15201376110963e-08
2122 3.16738599792643e-08
2123 3.15663939431943e-08
2124 3.15540589213015e-08
2125 3.16154604718122e-08
2126 3.19409174664997e-08
2127 3.12618979592116e-08
2128 3.14989740957117e-08
2129 3.151215466346e-08
2130 3.15017203433854e-08
2131 3.13925490047495e-08
2132 3.15282626672797e-08
2133 3.15865946731719e-08
2134 3.16694439561616e-08
2135 3.14152863722938e-08
2136 3.13047010536138e-08
2137 3.15015391549878e-08
2138 3.14113819399608e-08
2139 3.14383292732145e-08
2140 3.15988373245091e-08
2141 3.13306749433195e-08
2142 3.11382741813304e-08
2143 3.11384873441511e-08
2144 3.12325774132205e-08
2145 3.12760732867901e-08
2146 3.14762047537442e-08
2147 3.12750216835411e-08
2148 3.1198847949554e-08
2149 3.12136734237356e-08
2150 3.13390700057425e-08
2151 3.11545065301289e-08
2152 3.08487422273629e-08
2153 3.1409953749062e-08
2154 3.11132204444675e-08
2155 3.11866550362083e-08
2156 3.11744337011532e-08
2157 3.07727248127776e-08
2158 3.11434007471689e-08
2159 3.0905074055454e-08
2160 3.13296908416305e-08
2161 3.10092254096617e-08
2162 3.09890992866713e-08
2163 3.10055554564315e-08
2164 3.09739860426816e-08
2165 3.09449497137848e-08
2166 3.09847365542737e-08
2167 3.11423136167832e-08
2168 3.12025285609252e-08
2169 3.13088328596223e-08
2170 3.12021022352837e-08
2171 3.11703587385637e-08
2172 3.04968068576272e-08
2173 3.08704386497993e-08
2174 3.08612015942344e-08
2175 3.0843587239815e-08
2176 3.07789527198565e-08
2177 3.07658716280912e-08
2178 3.08389864756009e-08
2179 3.04487492996941e-08
2180 3.07551033529307e-08
2181 3.06259089200012e-08
2182 3.09146699351004e-08
2183 3.07592848969307e-08
2184 3.071272658417e-08
2185 3.0560851627115e-08
2186 3.07464596005502e-08
2187 3.0630047831437e-08
2188 3.06808018990523e-08
2189 3.0636307712939e-08
2190 3.07413507982801e-08
2191 3.05503284891984e-08
2192 3.05096712338582e-08
2193 3.05480902795807e-08
2194 3.0393568550835e-08
2195 3.07227274731758e-08
2196 3.04256104755041e-08
2197 3.03760288034027e-08
2198 3.04450722410365e-08
2199 3.05026368607741e-08
2200 3.03845730798002e-08
2201 3.03973699544713e-08
2202 3.07644256736239e-08
2203 3.05320249083252e-08
2204 2.99792013436218e-08
2205 3.0605757928015e-08
2206 3.02784641803555e-08
2207 3.02675715602163e-08
2208 3.02394518314486e-08
2209 3.02734939339189e-08
2210 3.02538225582794e-08
2211 3.01918809952895e-08
2212 3.03311793459216e-08
2213 2.99275129123089e-08
2214 3.03050242678182e-08
2215 3.02735188029146e-08
2216 3.0242411241943e-08
2217 3.02336466972974e-08
2218 3.02567428889233e-08
2219 3.04234717418694e-08
2220 3.01880618280848e-08
2221 3.04113711990794e-08
2222 3.01933518187525e-08
2223 3.01557854243129e-08
2224 3.03337017726335e-08
2225 3.01507121491795e-08
2226 3.01144247316643e-08
2227 3.00277847031793e-08
2228 3.01475004960139e-08
2229 2.99323907881899e-08
2230 3.02106712979366e-08
2231 2.97604358934223e-08
2232 2.99049069951707e-08
2233 3.01979490302529e-08
2234 3.00837541544752e-08
2235 2.99532665337665e-08
2236 2.97530728943229e-08
2237 2.98910833862465e-08
2238 2.97137692228944e-08
2239 2.93894260039451e-08
2240 2.93547373075853e-08
2241 2.94627113817114e-08
2242 2.95071522771195e-08
2243 2.96448821046624e-08
2244 2.93379596172372e-08
2245 2.96052373727207e-08
2246 2.9250205813014e-08
2247 2.951377275906e-08
2248 2.94033810632754e-08
2249 3.00281257636925e-08
2250 2.98369720042047e-08
2251 2.98515452357151e-08
2252 2.98017113209426e-08
2253 2.97723996567356e-08
2254 2.98932683051589e-08
2255 2.93485626912116e-08
2256 2.92231465692794e-08
2257 2.96289695000951e-08
2258 2.9756700214989e-08
2259 2.96403204202988e-08
2260 2.98416971133975e-08
2261 2.94839370695854e-08
2262 2.9138359280978e-08
2263 2.92584623196035e-08
2264 2.91408941421878e-08
2265 2.90278521219989e-08
2266 2.9032086956704e-08
2267 2.91055659573658e-08
2268 2.90795867385896e-08
2269 2.92008284219492e-08
2270 2.9143203406079e-08
2271 2.88323587227524e-08
2272 2.88013026761291e-08
2273 2.88284134342121e-08
2274 2.90540125291727e-08
2275 2.8852470634888e-08
2276 2.90190893537101e-08
2277 2.87314350089218e-08
2278 2.93198834100394e-08
2279 2.95770750113888e-08
2280 2.93294544206901e-08
2281 2.90898647392623e-08
2282 2.89784249929426e-08
2283 2.88694153027791e-08
2284 2.87449477553992e-08
2285 2.87019155109647e-08
2286 2.88471806442203e-08
2287 2.87887900185524e-08
2288 2.92981034988316e-08
2289 2.94113728926959e-08
2290 2.94431767855485e-08
2291 2.91311224032142e-08
2292 2.89425745592098e-08
2293 2.87796151354769e-08
2294 2.86740320376566e-08
2295 2.8666031326452e-08
2296 2.85620451734303e-08
2297 2.84674843697985e-08
2298 2.85101879882177e-08
2299 2.84752612600414e-08
2300 2.91430843901708e-08
2301 2.93490316494172e-08
2302 2.93414164076466e-08
2303 2.91288735354556e-08
2304 2.88502022272041e-08
2305 2.85870402905175e-08
2306 2.83690848590368e-08
2307 2.85431749347254e-08
2308 2.82332379697436e-08
2309 2.85943269062727e-08
2310 2.86186363496199e-08
2311 2.83282037827348e-08
2312 2.82912395732637e-08
2313 2.83951511192981e-08
2314 2.81499588083989e-08
2315 2.8101677429504e-08
2316 2.80519678597102e-08
2317 2.8118710915237e-08
2318 2.81172489735582e-08
2319 2.82795884487541e-08
2320 2.80126677409953e-08
2321 2.81061431905982e-08
2322 2.80492500337459e-08
2323 2.80141474462425e-08
2324 2.79333356445477e-08
2325 2.78678378151653e-08
2326 2.79411196402179e-08
2327 2.80471450508912e-08
2328 2.8221521120031e-08
2329 2.81807910340603e-08
2330 2.7997097973298e-08
2331 2.83606258477676e-08
2332 2.81301399951417e-08
2333 2.80045799883055e-08
2334 2.7993978690688e-08
2335 2.80101968286317e-08
2336 2.78375313911283e-08
2337 2.80294560894845e-08
2338 2.79245693235453e-08
2339 2.80563465793193e-08
2340 2.78760676764023e-08
2341 2.78611356208103e-08
2342 2.8061924339795e-08
2343 2.80394960583408e-08
2344 2.78807235076783e-08
2345 2.7800735935557e-08
2346 2.77720690888827e-08
2347 2.77257719005775e-08
2348 2.7573447525242e-08
2349 2.77270757464976e-08
2350 2.76649014807617e-08
2351 2.76802243348584e-08
2352 2.77416312144396e-08
2353 2.78364762351657e-08
2354 2.75987979136971e-08
2355 2.75608371680391e-08
2356 2.77760641154146e-08
2357 2.7706365202107e-08
2358 2.81997696305325e-08
2359 2.78144813847803e-08
2360 2.76159450862679e-08
2361 2.77657559166755e-08
2362 2.76414002797765e-08
2363 2.75866849364093e-08
2364 2.7953772629985e-08
2365 2.79349219312053e-08
2366 2.76554263933804e-08
2367 2.79739840181037e-08
2368 2.82698913167678e-08
2369 2.77673262161215e-08
2370 2.76269584986721e-08
2371 2.74411799949803e-08
2372 2.73494080715864e-08
2373 2.74081877194021e-08
2374 2.73911595627396e-08
2375 2.71775242310923e-08
2376 2.74045621750929e-08
2377 2.72821285562941e-08
2378 2.73234483927354e-08
2379 2.73658979921265e-08
2380 2.74475748796021e-08
2381 2.74039564374107e-08
2382 2.74347851103585e-08
2383 2.7315278927631e-08
2384 2.73841731512903e-08
2385 2.72548348334567e-08
2386 2.72279585544766e-08
2387 2.72699747227989e-08
2388 2.74223630469805e-08
2389 2.73470135425669e-08
2390 2.73188653920897e-08
2391 2.73339626488678e-08
2392 2.70866813423254e-08
2393 2.72162772319007e-08
2394 2.74339146955072e-08
2395 2.73148774709853e-08
2396 2.72402829182283e-08
2397 2.7474310826392e-08
2398 2.72993005978606e-08
2399 2.71761351200439e-08
2400 2.73959539498492e-08
2401 2.70460240869852e-08
2402 2.70874647156916e-08
2403 2.70803948154708e-08
2404 2.71958722208865e-08
2405 2.70877418273585e-08
2406 2.72513709376199e-08
2407 2.70749751507537e-08
2408 2.69991051737861e-08
2409 2.72125824096747e-08
2410 2.72471538664831e-08
2411 2.73775153658562e-08
2412 2.73703548714366e-08
2413 2.71698787912555e-08
2414 2.75054379272888e-08
2415 2.7678447978019e-08
2416 2.71630096193576e-08
2417 2.75113585246345e-08
2418 2.73927760474635e-08
2419 2.73366360659111e-08
2420 2.70893085740909e-08
2421 2.71124633854924e-08
2422 2.70442583882868e-08
2423 2.69465409985514e-08
2424 2.71170605969928e-08
2425 2.73448819143596e-08
2426 2.71558526776516e-08
2427 2.69817999054567e-08
2428 2.70030007243349e-08
2429 2.70498574650446e-08
2430 2.69733799740379e-08
2431 2.71167035492681e-08
2432 2.69121755991364e-08
2433 2.70290190229616e-08
2434 2.68773590050841e-08
2435 2.7134623437064e-08
2436 2.70692126491667e-08
2437 2.71608282531588e-08
2438 2.70586841821796e-08
2439 2.70563624837905e-08
2440 2.70601994145636e-08
2441 2.7186841222715e-08
2442 2.69864575130896e-08
2443 2.68091842059448e-08
2444 2.66661466241658e-08
2445 2.6687734688835e-08
2446 2.67744439952367e-08
2447 2.68163855565717e-08
2448 2.69029509780694e-08
2449 2.70238267319201e-08
2450 2.68922839552488e-08
2451 2.68345239362588e-08
2452 2.67043755997065e-08
2453 2.66941739823778e-08
2454 2.67580961832437e-08
2455 2.668171816822e-08
2456 2.66931010628468e-08
2457 2.6905963679269e-08
2458 2.6842485567613e-08
2459 2.680471311578e-08
2460 2.66290864914254e-08
2461 2.66503921153571e-08
2462 2.68426383343012e-08
2463 2.70042441741225e-08
2464 2.68113726775709e-08
2465 2.68438782313751e-08
2466 2.6899327210117e-08
2467 2.6965293997705e-08
2468 2.67975721612856e-08
2469 2.67700333012044e-08
2470 2.68477329257166e-08
2471 2.6675644804186e-08
2472 2.66171866769582e-08
2473 2.66803734660925e-08
2474 2.6613218295779e-08
2475 2.66598423337427e-08
2476 2.66547210969748e-08
2477 2.6724279678092e-08
2478 2.68080277976424e-08
2479 2.66160800066473e-08
2480 2.670900478563e-08
2481 2.65778368202518e-08
2482 2.66806843285394e-08
2483 2.65907740271132e-08
2484 2.65343089722592e-08
2485 2.65555168965648e-08
2486 2.65798298926256e-08
2487 2.67104844908772e-08
2488 2.65863455695126e-08
2489 2.66138364679591e-08
2490 2.63418922230585e-08
2491 2.6611083114858e-08
2492 2.65056865345059e-08
2493 2.66864876863337e-08
2494 2.66767692380654e-08
2495 2.64801034433049e-08
2496 2.63845887360503e-08
2497 2.62435797537819e-08
2498 2.63343000739269e-08
2499 2.63238781883501e-08
2500 2.62055497302072e-08
2501 2.61175703286654e-08
2502 2.60734260848494e-08
2503 2.61851500482635e-08
2504 2.63819845969238e-08
2505 2.6298174304884e-08
2506 2.62047734622683e-08
2507 2.618120653608e-08
2508 2.64377568726104e-08
2509 2.65625903494993e-08
2510 2.67850097657174e-08
2511 2.6841302513958e-08
2512 2.67626880656735e-08
2513 2.66155613104502e-08
2514 2.67462283431996e-08
2515 2.66187178965538e-08
2516 2.64897987989343e-08
2517 2.65496904461315e-08
2518 2.64949360229139e-08
2519 2.64870720911858e-08
2520 2.64384212300683e-08
2521 2.64549910866663e-08
2522 2.63657948806895e-08
2523 2.63459121185861e-08
2524 2.63883936924003e-08
2525 2.62055266375683e-08
2526 2.65309534341895e-08
2527 2.62272710216394e-08
2528 2.62521240301794e-08
2529 2.64953747830532e-08
2530 2.64563517760052e-08
2531 2.64419810491745e-08
2532 2.61559129910438e-08
2533 2.63129500410741e-08
2534 2.62453827559739e-08
2535 2.64414499184795e-08
2536 2.61755719321854e-08
2537 2.61390589173516e-08
2538 2.62187516142376e-08
2539 2.62842636544747e-08
2540 2.62344848067642e-08
2541 2.60971635412943e-08
2542 2.60615973246558e-08
2543 2.61824517622244e-08
2544 2.59903902843917e-08
2545 2.61851873517571e-08
2546 2.60632937454375e-08
2547 2.61491717168383e-08
2548 2.62205404055749e-08
2549 2.62272905615646e-08
2550 2.62347175095101e-08
2551 2.61388723998834e-08
2552 2.61160213455014e-08
2553 2.60160568643641e-08
2554 2.61501558185273e-08
2555 2.60030628140839e-08
2556 2.61456030159479e-08
2557 2.58923318341431e-08
2558 2.59136303526475e-08
2559 2.61640931142892e-08
2560 2.58690988630406e-08
2561 2.59859600504342e-08
2562 2.59109249611811e-08
2563 2.59937440461044e-08
2564 2.58134260633369e-08
2565 2.58600678648691e-08
2566 2.59028638538439e-08
2567 2.5951463200613e-08
2568 2.58523211726924e-08
2569 2.57823380422906e-08
2570 2.56302001844233e-08
2571 2.57113903501249e-08
2572 2.57250469815062e-08
2573 2.57002348291735e-08
2574 2.56993573088948e-08
2575 2.58132502040098e-08
2576 2.57897863065182e-08
2577 2.57052743535269e-08
2578 2.55635228540996e-08
2579 2.56260417330623e-08
2580 2.56476937465777e-08
2581 2.56137422383063e-08
2582 2.56308698709518e-08
2583 2.57986378926489e-08
2584 2.57327510411187e-08
2585 2.5612727938551e-08
2586 2.56906211859587e-08
2587 2.55905341361995e-08
2588 2.56786734098569e-08
2589 2.56184797819969e-08
2590 2.5563849703758e-08
2591 2.56638621465299e-08
2592 2.55608956223341e-08
2593 2.5767290523504e-08
2594 2.5658950519869e-08
2595 2.54838266044999e-08
2596 2.55301078055936e-08
2597 2.55077914346202e-08
2598 2.54901344476366e-08
2599 2.53837555419523e-08
2600 2.54938452570741e-08
2601 2.54447236613942e-08
2602 2.54135716915016e-08
2603 2.54790837317387e-08
2604 2.54369432184376e-08
2605 2.55593501918838e-08
2606 2.54982470693221e-08
2607 2.5498879452357e-08
2608 2.55074308341818e-08
2609 2.56012508970116e-08
2610 2.54765080143216e-08
2611 2.54553462752938e-08
2612 2.54325787096832e-08
2613 2.53930370064381e-08
2614 2.54207535022033e-08
2615 2.52993803684376e-08
2616 2.54073526662069e-08
2617 2.54552183776013e-08
2618 2.53912979530924e-08
2619 2.54391370191343e-08
2620 2.54331471438718e-08
2621 2.54517082964867e-08
2622 2.54630059259853e-08
2623 2.53919392179114e-08
2624 2.54912180253086e-08
2625 2.54754795037115e-08
2626 2.54095375851193e-08
2627 2.54858374404421e-08
2628 2.54504062269234e-08
2629 2.55380836478025e-08
2630 2.53128007443593e-08
2631 2.53759591117841e-08
2632 2.5232214539983e-08
2633 2.52327474470349e-08
2634 2.54918148812067e-08
2635 2.53384264681245e-08
2636 2.54923211429059e-08
2637 2.53659493409941e-08
2638 2.51497507264276e-08
2639 2.5273006798443e-08
2640 2.52230574204759e-08
2641 2.52646419340863e-08
2642 2.51584140187333e-08
2643 2.51479299606672e-08
2644 2.5085117982826e-08
2645 2.51954546115485e-08
2646 2.49905625082647e-08
2647 2.48989060480653e-08
2648 2.52317882143416e-08
2649 2.50678144908534e-08
2650 2.51753426994128e-08
2651 2.50723957151422e-08
2652 2.52034659808942e-08
2653 2.50839597981667e-08
2654 2.5456936114665e-08
2655 2.53312286702112e-08
2656 2.53167051766923e-08
2657 2.53182879106362e-08
2658 2.54388456966126e-08
2659 2.52931453559313e-08
2660 2.52654110965977e-08
2661 2.52764351671431e-08
2662 2.51932803507771e-08
2663 2.51596734557324e-08
2664 2.53283811701976e-08
2665 2.5220545651905e-08
2666 2.52265675015906e-08
2667 2.52147600576791e-08
2668 2.51831515640788e-08
2669 2.51023806185913e-08
2670 2.51216274449462e-08
2671 2.50906300180986e-08
2672 2.52736445105484e-08
2673 2.51918912397286e-08
2674 2.51349661084532e-08
2675 2.51756269165071e-08
2676 2.51939216155961e-08
2677 2.51758596192531e-08
2678 2.51612259916101e-08
2679 2.4981268609281e-08
2680 2.52611744855358e-08
2681 2.51531915296255e-08
2682 2.50996770034817e-08
2683 2.515433550343e-08
2684 2.48069778052695e-08
2685 2.51264147266284e-08
2686 2.51663649919465e-08
2687 2.51686014252073e-08
2688 2.51165772624518e-08
2689 2.51222509461968e-08
2690 2.51309248966436e-08
2691 2.54384833198174e-08
2692 2.54183802894659e-08
2693 2.5498104960775e-08
2694 2.53128096261435e-08
2695 2.51249314686675e-08
2696 2.50888483321887e-08
2697 2.50892728814733e-08
2698 2.51652192417851e-08
2699 2.48884362008539e-08
2700 2.51973570897235e-08
2701 2.50935148216058e-08
2702 2.49435849752899e-08
2703 2.48859155504988e-08
2704 2.49108449423829e-08
2705 2.4801696696386e-08
2706 2.49329161761125e-08
2707 2.48154137238998e-08
2708 2.48976448347094e-08
2709 2.50084841724174e-08
2710 2.49262104290437e-08
2711 2.48689318027573e-08
2712 2.50557636860549e-08
2713 2.48989184825632e-08
2714 2.48789966406093e-08
2715 2.49457574597045e-08
2716 2.50250291600196e-08
2717 2.51695482234027e-08
2718 2.50878393615039e-08
2719 2.50563534365256e-08
2720 2.49597746915242e-08
2721 2.49846792144126e-08
2722 2.50794318645831e-08
2723 2.48498750465842e-08
2724 2.48364919741562e-08
2725 2.47891911442366e-08
2726 2.47897347094295e-08
2727 2.49808991270584e-08
2728 2.48357689969225e-08
2729 2.48991280926703e-08
2730 2.49640397242956e-08
2731 2.49381582051456e-08
2732 2.4793068931217e-08
2733 2.45873845727829e-08
2734 2.47953515497557e-08
2735 2.47878393366818e-08
2736 2.48856419915455e-08
2737 2.48192506546729e-08
2738 2.47770728378782e-08
2739 2.48004319303163e-08
2740 2.47681484211171e-08
2741 2.48127189905745e-08
2742 2.48065266106323e-08
2743 2.47468214809032e-08
2744 2.47415883336544e-08
2745 2.47892391058713e-08
2746 2.47346374493418e-08
2747 2.4767064843445e-08
2748 2.46517259938628e-08
2749 2.48660736446027e-08
2750 2.47735041369879e-08
2751 2.48043185990809e-08
2752 2.44671767291038e-08
2753 2.48832421334555e-08
2754 2.47382327955847e-08
2755 2.47142857290328e-08
2756 2.47306193301711e-08
2757 2.46900526690297e-08
2758 2.47298945765806e-08
2759 2.47620697280126e-08
2760 2.47922908869214e-08
2761 2.47488891602643e-08
2762 2.47312552659196e-08
2763 2.46961935346235e-08
2764 2.48089211396518e-08
2765 2.47785649776233e-08
2766 2.47824498700311e-08
2767 2.46398386138935e-08
2768 2.47334206449068e-08
2769 2.47521647622762e-08
2770 2.47137172948442e-08
2771 2.4704450041213e-08
2772 2.47279725584804e-08
2773 2.47152556198671e-08
2774 2.46313867080517e-08
2775 2.46243860857476e-08
2776 2.46627660516197e-08
2777 2.46592097852272e-08
2778 2.46702196449178e-08
2779 2.47133886688289e-08
2780 2.46727331898455e-08
2781 2.47222438076733e-08
2782 2.47629312610798e-08
2783 2.47690792321009e-08
2784 2.48684095538465e-08
2785 2.46864129138658e-08
2786 2.46676492565712e-08
2787 2.46718911967037e-08
2788 2.4729967407211e-08
2789 2.47526994456848e-08
2790 2.47282656573589e-08
2791 2.47419986720843e-08
2792 2.47876688064252e-08
2793 2.47012135190516e-08
2794 2.46303297757322e-08
2795 2.46233327061418e-08
2796 2.46458746744338e-08
2797 2.4630365302869e-08
2798 2.46598919062535e-08
2799 2.47452298651751e-08
2800 2.46473295106853e-08
2801 2.42413626949656e-08
2802 2.45909301810343e-08
2803 2.46333105025087e-08
2804 2.46641249646018e-08
2805 2.45276474686307e-08
2806 2.46133051717834e-08
2807 2.4647244245557e-08
2808 2.47620572935148e-08
2809 2.46112978885549e-08
2810 2.44749465139193e-08
2811 2.46493705446937e-08
2812 2.46125217984172e-08
2813 2.45545610511044e-08
2814 2.45766056394814e-08
2815 2.45608440252454e-08
2816 2.46106850454453e-08
2817 2.45747884264347e-08
2818 2.46138878168267e-08
2819 2.46894575894885e-08
2820 2.45466011961071e-08
2821 2.44818973982319e-08
2822 2.45394176090485e-08
2823 2.45550477728784e-08
2824 2.45700704226692e-08
2825 2.45891875749749e-08
2826 2.45701361478723e-08
2827 2.45352129724097e-08
2828 2.45456988068327e-08
2829 2.45631230910703e-08
2830 2.45054305736403e-08
2831 2.42218032298069e-08
2832 2.45266491560869e-08
2833 2.45038815904763e-08
2834 2.452119574059e-08
2835 2.4502989859343e-08
2836 2.4479639648689e-08
2837 2.44689282169475e-08
2838 2.44713742603153e-08
2839 2.4593024505748e-08
2840 2.44999185383676e-08
2841 2.45192293135688e-08
2842 2.44711912955609e-08
2843 2.45254749842161e-08
2844 2.44390108150583e-08
2845 2.45965594558584e-08
2846 2.44091964418658e-08
2847 2.43373836639194e-08
2848 2.44411957339707e-08
2849 2.4386851649183e-08
2850 2.43037199254559e-08
2851 2.40677309193416e-08
2852 2.43321185422474e-08
2853 2.43058622118042e-08
2854 2.42854998333542e-08
2855 2.42280595585953e-08
2856 2.39350139707994e-08
2857 2.44010269767614e-08
2858 2.44228051116124e-08
2859 2.41274911161327e-08
2860 2.40956090635791e-08
2861 2.45775328977516e-08
2862 2.40649473681742e-08
2863 2.44141045158131e-08
2864 2.43826807633241e-08
2865 2.43998510285337e-08
2866 2.45831834888577e-08
2867 2.43376820918684e-08
2868 2.42978135389649e-08
2869 2.42658000360052e-08
2870 2.42399877947719e-08
2871 2.39752946384897e-08
2872 2.46052866970103e-08
2873 2.43424853607621e-08
2874 2.40738255996575e-08
2875 2.40883188951102e-08
2876 2.40601245593552e-08
2877 2.39927171463705e-08
2878 2.39602577778442e-08
2879 2.38370336802518e-08
2880 2.38965487397991e-08
2881 2.36855370872036e-08
2882 2.393018583291e-08
2883 2.39759572195908e-08
2884 2.39773871868465e-08
2885 2.38719994882786e-08
2886 2.38538984120851e-08
2887 2.39210642405396e-08
2888 2.37276225334426e-08
2889 2.36451338508914e-08
2890 2.38999273705076e-08
2891 2.38346160585934e-08
2892 2.3847428920476e-08
2893 2.39081074937531e-08
2894 2.38170478894517e-08
2895 2.37932713531563e-08
2896 2.37340422870602e-08
2897 2.37353692256193e-08
2898 2.38960247145314e-08
2899 2.36939108333445e-08
2900 2.38081305781179e-08
2901 2.37537065572724e-08
2902 2.39033237647845e-08
2903 2.39252440081827e-08
2904 2.38987425404957e-08
2905 2.40228370529394e-08
2906 2.42102125014299e-08
2907 2.40881572466378e-08
2908 2.38760655690839e-08
2909 2.39497452980686e-08
2910 2.40442634691362e-08
2911 2.40169146792368e-08
2912 2.41159110458966e-08
2913 2.43694788792936e-08
2914 2.40032047571503e-08
2915 2.39998954043585e-08
2916 2.41170585724149e-08
2917 2.38829276355546e-08
2918 2.39309141392141e-08
2919 2.39534809765019e-08
2920 2.39364439380552e-08
2921 2.38445725386782e-08
2922 2.39603217266904e-08
2923 2.39222934794725e-08
2924 2.38500525995278e-08
2925 2.3862947173825e-08
2926 2.37936710334452e-08
2927 2.3831955076048e-08
2928 2.38688038223245e-08
2929 2.38599415780527e-08
2930 2.38475017511064e-08
2931 2.39211956909458e-08
2932 2.39348576513976e-08
2933 2.38350210679528e-08
2934 2.38282691356062e-08
2935 2.37279458303874e-08
2936 2.40789290728571e-08
2937 2.40534827611327e-08
2938 2.36857875535179e-08
2939 2.39208759467147e-08
2940 2.39832065318524e-08
2941 2.41537474465758e-08
2942 2.37904558275659e-08
2943 2.37806254688167e-08
2944 2.39373711963253e-08
2945 2.37449881979046e-08
2946 2.37425457072504e-08
2947 2.38744437552896e-08
2948 2.39299708937324e-08
2949 2.39965896042804e-08
2950 2.40290933817278e-08
2951 2.39610322694261e-08
2952 2.43216522477496e-08
2953 2.3798595094604e-08
2954 2.37392683288817e-08
2955 2.39500899112954e-08
2956 2.39808777280359e-08
2957 2.41634587894168e-08
2958 2.39569697413344e-08
2959 2.38977726496614e-08
2960 2.38936035401593e-08
2961 2.39909745403111e-08
2962 2.3730018838819e-08
2963 2.38959891873947e-08
2964 2.3895012191133e-08
2965 2.38499868743247e-08
2966 2.39569164506293e-08
2967 2.38030608556983e-08
2968 2.3634520118776e-08
2969 2.39156712211752e-08
2970 2.38077362268996e-08
2971 2.37599504515629e-08
2972 2.3566739670855e-08
2973 2.36338468795338e-08
2974 2.3670002846643e-08
2975 2.37701200944684e-08
2976 2.37195152408276e-08
2977 2.36973196621193e-08
2978 2.3748324196049e-08
2979 2.36641373163593e-08
2980 2.38832207344331e-08
2981 2.36428672195643e-08
2982 2.38001121033449e-08
2983 2.38101822702674e-08
2984 2.37419843784892e-08
2985 2.37344384146354e-08
2986 2.37579431683344e-08
2987 2.3811470128976e-08
2988 2.38824906517721e-08
2989 2.36716317658647e-08
2990 2.36934685204915e-08
2991 2.36708359580007e-08
2992 2.36874999615111e-08
2993 2.38349269210403e-08
2994 2.38269528551882e-08
2995 2.3703185192403e-08
2996 2.3655955416757e-08
2997 2.36898927141738e-08
2998 2.39424782222386e-08
2999 2.36429666955473e-08
3000 2.35654731284285e-08
3001 2.37026558380649e-08
3002 2.36077681847746e-08
3003 2.37151702719984e-08
3004 2.36696813260551e-08
3005 2.36978792145237e-08
3006 2.38321327117319e-08
3007 2.36726034330559e-08
3008 2.38262671814482e-08
3009 2.36523209906636e-08
3010 2.3621833378229e-08
3011 2.35969341844111e-08
3012 2.3720009068029e-08
3013 2.36892780947073e-08
3014 2.36797124131272e-08
3015 2.36328663305585e-08
3016 2.36651604978988e-08
3017 2.35826131955719e-08
3018 2.3706503426979e-08
3019 2.3685277739105e-08
3020 2.36501982442405e-08
3021 2.35784316515719e-08
3022 2.36166695088968e-08
3023 2.36648514118087e-08
3024 2.36187940316768e-08
3025 2.36994566193971e-08
3026 2.36424959609849e-08
3027 2.35424089112257e-08
3028 2.35702515283265e-08
3029 2.3603284660112e-08
3030 2.36170176748374e-08
3031 2.36424231303545e-08
3032 2.36347243998125e-08
3033 2.36373889350716e-08
3034 2.36071269199556e-08
3035 2.36087114302563e-08
3036 2.36666064523661e-08
3037 2.33964634333006e-08
3038 2.35868302667086e-08
3039 2.36128006037006e-08
3040 2.35681945071065e-08
3041 2.35361188316574e-08
3042 2.35652031221889e-08
3043 2.35575310369995e-08
3044 2.35286652383593e-08
3045 2.3513161195865e-08
3046 2.35434463036199e-08
3047 2.35271464532616e-08
3048 2.34038708413209e-08
3049 2.34131771748025e-08
3050 2.35588348829197e-08
3051 2.35211050636508e-08
3052 2.3518769154407e-08
3053 2.34957209244158e-08
3054 2.34843628987846e-08
3055 2.34955681577276e-08
3056 2.35346764299038e-08
3057 2.35387958014144e-08
3058 2.32866650407004e-08
3059 2.36263382191737e-08
3060 2.35049295582712e-08
3061 2.35197603615234e-08
3062 2.36354065208388e-08
3063 2.34716956981629e-08
3064 2.3697632300923e-08
3065 2.32584227433108e-08
3066 2.33611068267692e-08
3067 2.32972166003265e-08
3068 2.33695356399721e-08
3069 2.34377974805966e-08
3070 2.33563142160165e-08
3071 2.34913084540267e-08
3072 2.32902248598066e-08
3073 2.33042989350452e-08
3074 2.34246506636282e-08
3075 2.34203287874379e-08
3076 2.33811370264903e-08
3077 2.3353399214443e-08
3078 2.32987229509263e-08
3079 2.33488570700047e-08
3080 2.33119870074461e-08
3081 2.3326068188112e-08
3082 2.33796750848114e-08
3083 2.33508803404447e-08
3084 2.34745289873217e-08
3085 2.31731469568786e-08
3086 2.33248869108138e-08
3087 2.31135928174808e-08
3088 2.30968932868336e-08
3089 2.3056296427626e-08
3090 2.31170158571103e-08
3091 2.30954828595031e-08
3092 2.31788437332625e-08
3093 2.3059515186219e-08
3094 2.31503474168449e-08
3095 2.32169075076172e-08
3096 2.33166979057842e-08
3097 2.31352785817762e-08
3098 2.32583801107467e-08
3099 2.30470238449243e-08
3100 2.32445227510425e-08
3101 2.3311390151548e-08
3102 2.3211921273969e-08
3103 2.32259846910665e-08
3104 2.32017161039266e-08
3105 2.32166765812281e-08
3106 2.33490808909664e-08
3107 2.31591119614905e-08
3108 2.33792398773858e-08
3109 2.32254784293673e-08
3110 2.30727277283904e-08
3111 2.32114967246844e-08
3112 2.3198603926744e-08
3113 2.32027783653166e-08
3114 2.32137260525178e-08
3115 2.31623502600087e-08
3116 2.32652173082215e-08
3117 2.31416397156181e-08
3118 2.33220571743686e-08
3119 2.31905801229004e-08
3120 2.30431087544503e-08
3121 2.31681109852389e-08
3122 2.32037535852214e-08
3123 2.31738308542617e-08
3124 2.30297203529517e-08
3125 2.31490933089162e-08
3126 2.32182557624583e-08
3127 2.32295889190937e-08
3128 2.31998793509547e-08
3129 2.33609043220895e-08
3130 2.3086116129889e-08
3131 2.30369785469975e-08
3132 2.28608563190846e-08
3133 2.29581065269713e-08
3134 2.30816645796494e-08
3135 2.30443681914494e-08
3136 2.30797141398398e-08
3137 2.30555308178282e-08
3138 2.30517596122581e-08
3139 2.30358718766865e-08
3140 2.30672529966114e-08
3141 2.31085088842065e-08
3142 2.30609042972674e-08
3143 2.31310419707143e-08
3144 2.31132837313908e-08
3145 2.30704468862086e-08
3146 2.30129781897404e-08
3147 2.29987122679631e-08
3148 2.30978631776679e-08
3149 2.29992256350897e-08
3150 2.31142998075029e-08
3151 2.30611334472997e-08
3152 2.29311449828629e-08
3153 2.30496528530466e-08
3154 2.30831656011787e-08
3155 2.30202736872798e-08
3156 2.28002896562884e-08
3157 2.30649614962886e-08
3158 2.31350565371713e-08
3159 2.3024551154549e-08
3160 2.30424870295565e-08
3161 2.30373231602243e-08
3162 2.30976127113536e-08
3163 2.29936869544645e-08
3164 2.29998189382741e-08
3165 2.31252421656336e-08
3166 2.30514469734544e-08
3167 2.28616325870235e-08
3168 2.30404104684112e-08
3169 2.30539196621748e-08
3170 2.29633148052244e-08
3171 2.29836736309608e-08
3172 2.29936816253939e-08
3173 2.29997993983488e-08
3174 2.29143282126643e-08
3175 2.2987796555185e-08
3176 2.29899654868859e-08
3177 2.2993349446665e-08
3178 2.2992242776354e-08
3179 2.30358399022634e-08
3180 2.29765078074706e-08
3181 2.30862298167267e-08
3182 2.30962236003052e-08
3183 2.30866419315134e-08
3184 2.32100827446402e-08
3185 2.31354526647465e-08
3186 2.31457502053445e-08
3187 2.29838601484289e-08
3188 2.30867591710648e-08
3189 2.30024905789605e-08
3190 2.29394405693029e-08
3191 2.30202683582093e-08
3192 2.31228245439752e-08
3193 2.29380745508934e-08
3194 2.29512373550733e-08
3195 2.30964083414165e-08
3196 2.30203660578354e-08
3197 2.316215841347e-08
3198 2.29156196240865e-08
3199 2.3142392890918e-08
3200 2.30423111702294e-08
3201 2.30288552671709e-08
3202 2.30324328498455e-08
3203 2.30017480618017e-08
3204 2.30150245528193e-08
3205 2.30262902078948e-08
3206 2.29754615332922e-08
3207 2.28305072624835e-08
3208 2.29777779026108e-08
3209 2.31751648982481e-08
3210 2.2988958292558e-08
3211 2.322300396429e-08
3212 2.31400942851678e-08
3213 2.30980781168455e-08
3214 2.31401315886615e-08
3215 2.29573871024513e-08
3216 2.30626042707627e-08
3217 2.29572965082525e-08
3218 2.29609504742712e-08
3219 2.30528431899302e-08
3220 2.30750760721321e-08
3221 2.30638850240439e-08
3222 2.30636594267253e-08
3223 2.30415864166389e-08
3224 2.32223484886163e-08
3225 2.30964989356153e-08
3226 2.30788970156937e-08
3227 2.30603465212198e-08
3228 2.31027090791258e-08
3229 2.30791332711533e-08
3230 2.31616752444097e-08
3231 2.29115464378538e-08
3232 2.28703846971712e-08
3233 2.31109229531512e-08
3234 2.30947279078464e-08
3235 2.30945254031667e-08
3236 2.31746870582583e-08
3237 2.31555272733885e-08
3238 2.29971597320855e-08
3239 2.30591261640711e-08
3240 2.31238868053651e-08
3241 2.30952377222593e-08
3242 2.30265566614207e-08
3243 2.31264163375045e-08
3244 2.30383534471912e-08
3245 2.31149961393839e-08
3246 2.29434267140505e-08
3247 2.30125252187463e-08
3248 2.30184671323741e-08
3249 2.3054802511524e-08
3250 2.3098614576611e-08
3251 2.30258105915482e-08
3252 2.3067201482263e-08
3253 2.30335626127953e-08
3254 2.30183587746069e-08
3255 2.30459935579574e-08
3256 2.30387691146916e-08
3257 2.30281056445847e-08
3258 2.31034231745753e-08
3259 2.30391119515616e-08
3260 2.30325873928905e-08
3261 2.30365699849244e-08
3262 2.30072139117965e-08
3263 2.29493792858193e-08
3264 2.31082228907553e-08
3265 2.31595667088413e-08
3266 2.30297256820222e-08
3267 2.30579679794118e-08
3268 2.3063430276693e-08
3269 2.31162058383916e-08
3270 2.30959393832109e-08
3271 2.30926477939875e-08
3272 2.30558399039182e-08
3273 2.30691998837074e-08
3274 2.31597372390979e-08
3275 2.2946794686618e-08
3276 2.30640466725163e-08
3277 2.28927117262856e-08
3278 2.29563319464887e-08
3279 2.32172006064957e-08
3280 2.31375416603896e-08
3281 2.29649099736662e-08
3282 2.31284911222929e-08
3283 2.30727525973862e-08
3284 2.30322214633816e-08
3285 2.30651657773251e-08
3286 2.31326744426497e-08
3287 2.31578134446409e-08
3288 2.30062084938254e-08
3289 2.31296866104458e-08
3290 2.31437358166886e-08
3291 2.29049703648343e-08
3292 2.30403873757723e-08
3293 2.31935644023906e-08
3294 2.3001843985071e-08
3295 2.30596022277041e-08
3296 2.31733707778403e-08
3297 2.30805561329817e-08
3298 2.30578311999352e-08
3299 2.3056506037733e-08
3300 2.30188916816587e-08
3301 2.30526602251757e-08
3302 2.31409735818033e-08
3303 2.30338557116738e-08
3304 2.30828351988066e-08
3305 2.29708465582235e-08
3306 2.30891306074454e-08
3307 2.31050982790748e-08
3308 2.30347954044419e-08
3309 2.30592096528426e-08
3310 2.29647962868285e-08
3311 2.29819221431171e-08
3312 2.30498162778758e-08
3313 2.31456596111457e-08
3314 2.31257395455486e-08
3315 2.30592771544025e-08
3316 2.30196661732407e-08
3317 2.30416379309872e-08
3318 2.3085197753403e-08
3319 2.30653487420795e-08
3320 2.30561987279998e-08
3321 2.30559553671128e-08
3322 2.30085301922145e-08
3323 2.29944241425528e-08
3324 2.28515233402504e-08
3325 2.30519177080168e-08
3326 2.29763692516372e-08
3327 2.29913350580091e-08
3328 2.3173562624379e-08
3329 2.29399788054252e-08
3330 2.28901591015074e-08
3331 2.29650201077902e-08
3332 2.2950754186013e-08
3333 2.30610019968935e-08
3334 2.29670380491598e-08
3335 2.30380212684622e-08
3336 2.33575452313062e-08
3337 2.30963266290019e-08
3338 2.30679777502019e-08
3339 2.31850734166983e-08
3340 2.32018919632537e-08
3341 2.31665957528548e-08
3342 2.306226853932e-08
3343 2.34057253578612e-08
3344 2.31003998152346e-08
3345 2.30630963216072e-08
3346 2.32055423765587e-08
3347 2.29786305538937e-08
3348 2.30205774442993e-08
3349 2.29790426686804e-08
3350 2.30691359348612e-08
3351 2.29562928666383e-08
3352 2.30781047605433e-08
3353 2.29396999174014e-08
3354 2.30878551832348e-08
3355 2.30915890853112e-08
3356 2.32430981128573e-08
3357 2.30188845762314e-08
3358 2.30472352313882e-08
3359 2.3089191003578e-08
3360 2.30584689120406e-08
3361 2.29477645774523e-08
3362 2.30538788059675e-08
3363 2.30374208598505e-08
3364 2.30650183397074e-08
3365 2.30482370966456e-08
3366 2.30889867225414e-08
3367 2.32075372252893e-08
3368 2.32918040410368e-08
3369 2.30215917440546e-08
3370 2.314076752441e-08
3371 2.29419008235254e-08
3372 2.30150440927446e-08
3373 2.3072887600506e-08
3374 2.31289014607228e-08
3375 2.30090613229095e-08
3376 2.30415029278674e-08
3377 2.29451604383257e-08
3378 2.30041852233853e-08
3379 2.3087741496397e-08
3380 2.29922747507771e-08
3381 2.30693100178314e-08
3382 2.30255494670928e-08
3383 2.30913990151294e-08
3384 2.3029185669543e-08
3385 2.30328716099848e-08
3386 2.30930048417122e-08
3387 2.30023449176997e-08
3388 2.30605969875342e-08
3389 2.30847820859026e-08
3390 2.3188505338112e-08
3391 2.32358434715252e-08
3392 2.30005081647278e-08
3393 2.30137047196877e-08
3394 2.27941523434083e-08
3395 2.29997727529963e-08
3396 2.31287238250388e-08
3397 2.30444587856482e-08
3398 2.31915020521001e-08
3399 2.31422845331508e-08
3400 2.32319656845448e-08
3401 2.33017036777028e-08
3402 2.33841461749762e-08
3403 2.33275105898656e-08
3404 2.33499459767472e-08
3405 2.34609505156413e-08
3406 2.35006094584378e-08
3407 2.33523191894847e-08
3408 2.34448958025268e-08
3409 2.34107471186462e-08
3410 2.34559820455615e-08
3411 2.3418923689178e-08
3412 2.34073826987924e-08
3413 2.38207036318272e-08
3414 2.33764190227248e-08
3415 2.34716441838145e-08
3416 2.34982966418329e-08
3417 2.34563266587884e-08
3418 2.34766694973132e-08
3419 2.34176873448178e-08
3420 2.34091945827686e-08
3421 2.35157759931326e-08
3422 2.35487522814992e-08
3423 2.33520545123156e-08
3424 2.34548789279643e-08
3425 2.3389457481926e-08
3426 2.35778827573085e-08
3427 2.33485142331347e-08
3428 2.35773178758336e-08
3429 2.33731185517172e-08
3430 2.37934365543424e-08
3431 2.34425243661462e-08
3432 2.3392388470711e-08
3433 2.33916246372701e-08
3434 2.36955148835705e-08
3435 2.33572272634319e-08
3436 2.34588650727119e-08
3437 2.34478640948055e-08
3438 2.33829577922506e-08
3439 2.35454411523506e-08
3440 2.35192629816083e-08
3441 2.34743797733472e-08
3442 2.35396324654857e-08
3443 2.34452350866832e-08
3444 2.38283615061619e-08
3445 2.34845600743938e-08
3446 2.33741026534062e-08
3447 2.33828192364172e-08
3448 2.33709140928795e-08
3449 2.36278214771346e-08
3450 2.33612844624531e-08
3451 2.3531024240242e-08
3452 2.33996626519684e-08
3453 2.37041000161753e-08
3454 2.35881341126287e-08
3455 2.33076615785421e-08
3456 2.34829773404499e-08
3457 2.33926282788843e-08
3458 2.33582397868304e-08
3459 2.33507737590344e-08
3460 2.36133690378892e-08
3461 2.34408137345099e-08
3462 2.34521468911453e-08
3463 2.38336852476095e-08
3464 2.33863985954486e-08
3465 2.35351897970304e-08
3466 2.35576766982604e-08
3467 2.34304629032067e-08
3468 2.36517792018276e-08
3469 2.35082300292788e-08
3470 2.34422330436246e-08
3471 2.351917771648e-08
3472 2.33973018737288e-08
3473 2.34334667226221e-08
3474 2.34262973464183e-08
3475 2.36229009686895e-08
3476 2.33538539617939e-08
3477 2.34200108195637e-08
3478 2.33852599507145e-08
3479 2.33973604935045e-08
3480 2.34276473776163e-08
3481 2.33576216146503e-08
3482 2.35881447707698e-08
3483 2.34816717181729e-08
3484 2.33970318674892e-08
3485 2.34051000802538e-08
3486 2.3435902107849e-08
3487 2.33561561202578e-08
3488 2.36499992922745e-08
3489 2.33363302015732e-08
3490 2.35290436023661e-08
3491 2.34043096014602e-08
3492 2.34327313108906e-08
3493 2.34271659849128e-08
3494 2.3400916759897e-08
3495 2.33869421606414e-08
3496 2.35702444228991e-08
3497 2.35208723609048e-08
3498 2.36522144092532e-08
3499 2.34791475151042e-08
3500 2.34740298310498e-08
3501 2.37468036345945e-08
3502 2.34308643598524e-08
3503 2.34729089498842e-08
3504 2.34424213374496e-08
3505 2.36744774895215e-08
3506 2.35763337741446e-08
3507 2.33762413870409e-08
3508 2.33643220326485e-08
3509 2.33660326642848e-08
3510 2.33735502064292e-08
3511 2.3387411118847e-08
3512 2.34995329861931e-08
3513 2.35012347360453e-08
3514 2.33798616022796e-08
3515 2.34298234147445e-08
3516 2.35083703614691e-08
3517 2.34667343335104e-08
3518 2.35471340204185e-08
3519 2.34870629611805e-08
3520 2.36659758456881e-08
3521 2.35430093198374e-08
3522 2.35643540236197e-08
3523 2.35024177897003e-08
3524 2.36405135467521e-08
3525 2.34273667132356e-08
3526 2.34482584460238e-08
3527 2.35410944071646e-08
3528 2.35506050216827e-08
3529 2.3561508299963e-08
3530 2.35500277057099e-08
3531 2.35404460369182e-08
3532 2.35403980752835e-08
3533 2.34781367680625e-08
3534 2.35699673112322e-08
3535 2.35492283451322e-08
3536 2.34679458088749e-08
3537 2.34435351131879e-08
3538 2.34341737126442e-08
3539 2.3592079401169e-08
3540 2.35200552367587e-08
3541 2.35834693995685e-08
3542 2.35713581986374e-08
3543 2.36557102795132e-08
3544 2.36013004695224e-08
3545 2.35648673907463e-08
3546 2.36595525393568e-08
3547 2.35572397144779e-08
3548 2.36478854276356e-08
3549 2.35859118902226e-08
3550 2.35616237631575e-08
3551 2.36032313694068e-08
3552 2.35806929538285e-08
3553 2.36102160044993e-08
3554 2.35678729865185e-08
3555 2.36324133595645e-08
3556 2.37400783476005e-08
3557 2.35684165517114e-08
3558 2.35446826479802e-08
3559 2.39485107300652e-08
3560 2.35285426697374e-08
3561 2.35876917997757e-08
3562 2.36863311187108e-08
3563 2.35905801559966e-08
3564 2.35920740720985e-08
3565 2.36705233191969e-08
3566 2.36205988102256e-08
3567 2.37059705199272e-08
3568 2.35865371678301e-08
3569 2.35765096334717e-08
3570 2.36679813525598e-08
3571 2.3536786741829e-08
3572 2.37578436923513e-08
3573 2.36043184997925e-08
3574 2.36701165334807e-08
3575 2.36569768219397e-08
3576 2.37151791537826e-08
3577 2.35480808186139e-08
3578 2.37470878516888e-08
3579 2.36259278807438e-08
3580 2.37140280745507e-08
3581 2.36669315256677e-08
3582 2.37378472434102e-08
3583 2.36469954728591e-08
3584 2.36463790770358e-08
3585 2.36241373130497e-08
3586 2.36808173070813e-08
3587 2.36281003651584e-08
3588 2.3712399155329e-08
3589 2.36543691300994e-08
3590 2.37067929731438e-08
3591 2.36774511108706e-08
3592 2.36862351954414e-08
3593 2.37298127814256e-08
3594 2.35778028212508e-08
3595 2.36198722802783e-08
3596 2.37054678109416e-08
3597 2.36306725298618e-08
3598 2.37258248603212e-08
3599 2.36577974987995e-08
3600 2.36543815645973e-08
3601 2.36641710671393e-08
3602 2.39434942983507e-08
3603 2.3738458310163e-08
3604 2.3821634442811e-08
3605 2.36408119747011e-08
3606 2.36787656149318e-08
3607 2.37532589153489e-08
3608 2.36331985092875e-08
3609 2.40048212418742e-08
3610 2.3571882223905e-08
3611 2.36436719092126e-08
3612 2.36983730417251e-08
3613 2.36319106505789e-08
3614 2.37325643581698e-08
3615 2.36528556740723e-08
3616 2.36399113617836e-08
3617 2.36890116411814e-08
3618 2.36531878528012e-08
3619 2.36930688402026e-08
3620 2.39679334157472e-08
3621 2.36552057941708e-08
3622 2.37390818114136e-08
3623 2.36414976484411e-08
3624 2.36724648772224e-08
3625 2.36633272976405e-08
3626 2.3658422776407e-08
3627 2.36780515194823e-08
3628 2.37003945358083e-08
3629 2.39509407862215e-08
3630 2.36297008626707e-08
3631 2.36586483737256e-08
3632 2.36430945932398e-08
3633 2.38424711085372e-08
3634 2.36292212463241e-08
3635 2.3662835246796e-08
3636 2.36158097521866e-08
3637 2.3627805489923e-08
3638 2.37829240745668e-08
3639 2.36039880974204e-08
3640 2.42299300623472e-08
3641 2.35535342341109e-08
3642 2.36657644592242e-08
3643 2.39552555569844e-08
3644 2.3826933315263e-08
3645 2.38492550153069e-08
3646 2.36125998753778e-08
3647 2.36977175660513e-08
3648 2.36962041100242e-08
3649 2.3920717850956e-08
3650 2.39217072817155e-08
3651 2.37614443676648e-08
3652 2.36091501903957e-08
3653 2.36932997665917e-08
3654 2.36987123258814e-08
3655 2.36933725972222e-08
3656 2.36879884596419e-08
3657 2.4206240567537e-08
3658 2.357625206173e-08
3659 2.42191386945478e-08
3660 2.37806165870325e-08
3661 2.36621904292633e-08
3662 2.37235511235667e-08
3663 2.37845672046433e-08
3664 2.37081874132627e-08
3665 2.38978454802918e-08
3666 2.36585790958088e-08
3667 2.39542554680838e-08
3668 2.36750388182827e-08
3669 2.39281519043288e-08
3670 2.36683295185003e-08
3671 2.37500970001747e-08
3672 2.40491093705941e-08
3673 2.35905766032829e-08
3674 2.38402328989196e-08
3675 2.37974919770068e-08
3676 2.36921966489945e-08
3677 2.43939677346816e-08
3678 2.34779893304449e-08
3679 2.37559323323921e-08
3680 2.36199078074151e-08
3681 2.37316282181155e-08
3682 2.37084876175686e-08
3683 2.3741455024151e-08
3684 2.3745116095597e-08
3685 2.37199131447596e-08
3686 2.37037358630232e-08
3687 2.37622455045994e-08
3688 2.37243380496466e-08
3689 2.37309958350806e-08
3690 2.37902693100978e-08
3691 2.36910064899121e-08
3692 2.37555344284601e-08
3693 2.37409985004433e-08
3694 2.37750175102747e-08
3695 2.39260167234079e-08
3696 2.37383996903873e-08
3697 2.39108999267046e-08
3698 2.38556001619372e-08
3699 2.39213058250698e-08
3700 2.38803821162037e-08
3701 2.40283064556479e-08
3702 2.38289032949979e-08
3703 2.39373996180348e-08
3704 2.38280488673581e-08
3705 2.43059030680115e-08
3706 2.38007711317323e-08
3707 2.38991475498551e-08
3708 2.39449970962369e-08
3709 2.3932088311085e-08
3710 2.39801760670844e-08
3711 2.40440076737514e-08
3712 2.40210376034611e-08
3713 2.39620856490319e-08
3714 2.39702320214974e-08
3715 2.38933903773386e-08
3716 2.45366962303706e-08
3717 2.37766979438447e-08
3718 2.40367601378466e-08
3719 2.44846560804035e-08
3720 2.40391173633725e-08
3721 2.37934507651971e-08
3722 2.4741957815877e-08
3723 2.4240362606065e-08
3724 2.44079583211487e-08
3725 2.40797479733601e-08
3726 2.43767068752732e-08
3727 2.44320936815257e-08
3728 2.46330511544102e-08
3729 2.43925093457165e-08
3730 2.38161206311815e-08
3731 2.44875959509727e-08
3732 2.38308164313139e-08
3733 2.43751738793208e-08
3734 2.41526230126965e-08
3735 2.44086937328802e-08
3736 2.45878446492043e-08
3737 2.38142288111476e-08
3738 2.40459385736358e-08
3739 2.40446080823631e-08
3740 2.40753035285479e-08
3741 2.45668463350057e-08
3742 2.463383808049e-08
3743 2.44595259601965e-08
3744 2.40224338199369e-08
3745 2.39397159873533e-08
3746 2.42291768870473e-08
3747 2.43996876037045e-08
3748 2.47307738732161e-08
3749 2.39401583002063e-08
3750 2.44015208039627e-08
3751 2.41971118697393e-08
3752 2.38244659556131e-08
3753 2.48930369650679e-08
3754 2.35853647723161e-08
3755 2.44418227879351e-08
3756 2.37417214776769e-08
3757 2.46642049006596e-08
3758 2.38484769710112e-08
3759 2.40333868362086e-08
3760 2.39684094793802e-08
3761 2.41839046566383e-08
3762 2.39300401716491e-08
3763 2.47354119409238e-08
3764 2.37875426023493e-08
3765 2.41803963518805e-08
3766 2.39901556398081e-08
3767 2.40725626099447e-08
3768 2.44216025180322e-08
3769 2.45092337536335e-08
3770 2.40501094594947e-08
3771 2.44425439888118e-08
3772 2.49115945649692e-08
3773 2.5009727622205e-08
3774 2.3847340102634e-08
3775 2.44610038890869e-08
3776 2.4946780641244e-08
3777 2.3742074972688e-08
3778 2.45324205394581e-08
3779 2.49689939835207e-08
3780 2.42596485122704e-08
3781 2.4038481427624e-08
3782 2.50860505701667e-08
3783 2.43573889946447e-08
3784 2.45458480208072e-08
3785 2.42277611306463e-08
3786 2.42703919184351e-08
3787 2.4235916384896e-08
3788 2.42900792812861e-08
3789 2.43873810035211e-08
3790 2.43216451423223e-08
3791 2.42851108112063e-08
3792 2.42918236637024e-08
3793 2.41750122143003e-08
3794 2.47264750896647e-08
3795 2.47435547606756e-08
3796 2.47916442930318e-08
3797 2.42874182987407e-08
3798 2.4280513599706e-08
3799 2.42978135389649e-08
3800 2.48089211396518e-08
3801 2.4052384972606e-08
3802 2.47038940415223e-08
3803 2.43011708533913e-08
3804 2.43016771150906e-08
3805 2.43291538026824e-08
3806 2.42376430037439e-08
3807 2.52923353372125e-08
3808 2.37513013701118e-08
3809 2.44596858323121e-08
3810 2.41990925076152e-08
3811 2.45081928085256e-08
3812 2.42584743403995e-08
3813 2.51444269849799e-08
3814 2.41259350275413e-08
3815 2.42893847257619e-08
3816 2.43128681631788e-08
3817 2.48253630985573e-08
3818 2.49382949846222e-08
3819 2.40037714149821e-08
3820 2.48982452433211e-08
3821 2.41883864049441e-08
3822 2.42495303837131e-08
3823 2.43070061856088e-08
3824 2.4393775888143e-08
3825 2.43827784629502e-08
3826 2.44951756656064e-08
3827 2.4407954768435e-08
3828 2.52196539207716e-08
3829 2.39653203948365e-08
3830 2.45291680300852e-08
3831 2.43493971652242e-08
3832 2.52263472333425e-08
3833 2.41081252738695e-08
3834 2.45815083843581e-08
3835 2.49271394636708e-08
3836 2.41530795364042e-08
3837 2.45376483576365e-08
3838 2.43166748958856e-08
3839 2.51611442791955e-08
3840 2.45165079348908e-08
3841 2.44247093661443e-08
3842 2.44610589561489e-08
3843 2.48158187332592e-08
3844 2.41847537552076e-08
3845 2.45184832436962e-08
3846 2.44851428021775e-08
3847 2.44242226443703e-08
3848 2.4377726504099e-08
3849 2.49114400219241e-08
3850 2.45624498518282e-08
3851 2.446103586351e-08
3852 2.44398812299096e-08
3853 2.45153479738747e-08
3854 2.44546178862493e-08
3855 2.4827944145045e-08
3856 2.47593590074757e-08
3857 2.45619133920627e-08
3858 2.54043843739282e-08
3859 2.41908910680877e-08
3860 2.48475604536225e-08
3861 2.53626044610655e-08
3862 2.40803146311919e-08
3863 2.48439349093132e-08
3864 2.43781155262468e-08
3865 2.47564173605497e-08
3866 2.45038265234143e-08
3867 2.47210181214541e-08
3868 2.46708307116705e-08
3869 2.45954030475559e-08
3870 2.43532198851426e-08
3871 2.48823237569695e-08
3872 2.42624942359271e-08
3873 2.45503279927561e-08
3874 2.45034090795571e-08
3875 2.45710971569224e-08
3876 2.45443416702074e-08
3877 2.48193483542991e-08
3878 2.54626986162521e-08
3879 2.43162290303189e-08
3880 2.46147671134622e-08
3881 2.53442031805662e-08
3882 2.52204763739883e-08
3883 2.39677859781295e-08
3884 2.47842226741568e-08
3885 2.45633327011774e-08
3886 2.47266402908508e-08
3887 2.45844216095747e-08
3888 2.4610788074142e-08
3889 2.45977478385839e-08
3890 2.46714080276433e-08
3891 2.47774227801756e-08
3892 2.44517792680199e-08
3893 2.46665354808329e-08
3894 2.44356339607066e-08
3895 2.47516158680128e-08
3896 2.45683668964602e-08
3897 2.55181227259982e-08
3898 2.42241746661875e-08
3899 2.46689975114123e-08
3900 2.45150033606478e-08
3901 2.52559306801459e-08
3902 2.45546303290212e-08
3903 2.50145877345176e-08
3904 2.4421151323395e-08
3905 2.45874929305501e-08
3906 2.4498211459445e-08
3907 2.45504043761002e-08
3908 2.46346818499887e-08
3909 2.49894220871738e-08
3910 2.45562734590976e-08
3911 2.46365985390185e-08
3912 2.46438851547737e-08
3913 2.45248070740445e-08
3914 2.50208049834555e-08
3915 2.4236424422952e-08
3916 2.46636346901141e-08
3917 2.47924809571032e-08
3918 2.47936426944761e-08
3919 2.44393554282851e-08
3920 2.4621980898587e-08
3921 2.4366078932303e-08
3922 2.52137812850606e-08
3923 2.45982043622917e-08
3924 2.45039366575384e-08
3925 2.44995170817219e-08
3926 2.47143141507422e-08
3927 2.47713707324237e-08
3928 2.46098910139381e-08
3929 2.4577063939546e-08
3930 2.48031764016332e-08
3931 2.51273792883921e-08
3932 2.55003591576042e-08
3933 2.3945172955564e-08
3934 2.48280151993185e-08
3935 2.50405953750032e-08
3936 2.43007569622478e-08
3937 2.47933495955976e-08
3938 2.49417375641769e-08
3939 2.44285871531247e-08
3940 2.46765985423281e-08
3941 2.48305518368852e-08
3942 2.53801655247798e-08
3943 2.42125519633873e-08
3944 2.46860576424979e-08
3945 2.5442037809853e-08
3946 2.43465425597833e-08
3947 2.46474129994567e-08
3948 2.49933389540047e-08
3949 2.43278659439738e-08
3950 2.46820182070451e-08
3951 2.45084894601177e-08
3952 2.45567282064485e-08
3953 2.4550374178034e-08
3954 2.45778739582647e-08
3955 2.45104416762842e-08
3956 2.47745148840295e-08
3957 2.44903084478665e-08
3958 2.45850486635391e-08
3959 2.45997426873146e-08
3960 2.451320924024e-08
3961 2.45596130099557e-08
3962 2.4627766492813e-08
3963 2.44352733602682e-08
3964 2.44958755502012e-08
3965 2.4472152304611e-08
3966 2.45028264345137e-08
3967 2.4791974695404e-08
3968 2.47124329888493e-08
3969 2.46617588572917e-08
3970 2.53983394316037e-08
3971 2.50322305106465e-08
3972 2.44636311208524e-08
3973 2.47260221186707e-08
3974 2.49403449004149e-08
3975 2.49111025141247e-08
3976 2.4529313691346e-08
3977 2.56751775395969e-08
3978 2.42230075997441e-08
3979 2.47592222279991e-08
3980 2.53206184908095e-08
3981 2.517202268848e-08
3982 2.47870879377388e-08
3983 2.46058231567758e-08
3984 2.47045601753371e-08
3985 2.48154474746798e-08
3986 2.45300650902891e-08
3987 2.47114968487949e-08
3988 2.47054749991094e-08
3989 2.47399842834284e-08
3990 2.46761029387699e-08
3991 2.50058711515067e-08
3992 2.56990926317258e-08
3993 2.40883775148859e-08
3994 2.48748541764598e-08
3995 2.46444660234602e-08
3996 2.45231674966817e-08
3997 2.55571084295525e-08
3998 2.41872921691311e-08
3999 2.47563978206244e-08
4000 2.46464644249045e-08
4001 2.46369431522453e-08
4002 2.47329108304939e-08
4003 2.45251072783503e-08
4004 2.47430573807605e-08
4005 2.5623423383081e-08
4006 2.43793216725408e-08
4007 2.5284233373668e-08
4008 2.43526674381656e-08
4009 2.55152201589226e-08
4010 2.52540726108919e-08
4011 2.44844873265038e-08
4012 2.45223290562535e-08
4013 2.46225546618462e-08
4014 2.54661234322384e-08
4015 2.45368720896977e-08
4016 2.4525840913725e-08
4017 2.45807569854151e-08
4018 2.47069031900082e-08
4019 2.45848337243615e-08
4020 2.49486422632117e-08
4021 2.50943710256024e-08
4022 2.50941560864248e-08
4023 2.44196609600067e-08
4024 2.46429685546445e-08
4025 2.45083420225001e-08
4026 2.46603715226001e-08
4027 2.5693603689092e-08
4028 2.41251729704572e-08
4029 2.48153995130451e-08
4030 2.56708023727015e-08
4031 2.52721896742969e-08
4032 2.4349532168344e-08
4033 2.50477825147755e-08
4034 2.45150832967056e-08
4035 2.46131417469542e-08
4036 2.44215350164723e-08
4037 2.45027269585307e-08
4038 2.54264556076578e-08
4039 2.41662103661611e-08
4040 2.47330262936885e-08
4041 2.4602204717894e-08
4042 2.50703866555568e-08
4043 2.46057627606433e-08
4044 2.51670400075454e-08
4045 2.45196307702145e-08
4046 2.45819755662069e-08
4047 2.46357867439428e-08
4048 2.5090637123526e-08
4049 2.53670204841683e-08
4050 2.475069571517e-08
4051 2.46146747429066e-08
4052 2.46050770869033e-08
4053 2.45407161258981e-08
4054 2.45390765485354e-08
4055 2.45701006207355e-08
4056 2.45656259778571e-08
4057 2.45142697252732e-08
4058 2.4533505893487e-08
4059 2.4603719950278e-08
4060 2.45101805518289e-08
4061 2.48088980470129e-08
4062 2.44176767694171e-08
4063 2.52394194433236e-08
4064 2.52771510389493e-08
4065 2.44351934242104e-08
4066 2.53553444906629e-08
4067 2.47184637203191e-08
4068 2.48336373687152e-08
4069 2.45179698765696e-08
4070 2.46543407911304e-08
4071 2.56463241754545e-08
4072 2.43514932662947e-08
4073 2.46377531709641e-08
4074 2.46324791675079e-08
4075 2.45893172490241e-08
4076 2.5149059723617e-08
4077 2.45651676777925e-08
4078 2.54348169193008e-08
4079 2.40904221016081e-08
4080 2.52593075344976e-08
4081 2.48677860525959e-08
4082 2.47502871530969e-08
4083 2.47276314979672e-08
4084 2.48062637098201e-08
4085 2.4482371685508e-08
4086 2.56934260534081e-08
4087 2.40264625972486e-08
4088 2.45614835137076e-08
4089 2.50934046874818e-08
4090 2.45018192401858e-08
4091 2.49690597087238e-08
4092 2.44158009365947e-08
4093 2.47076918924449e-08
4094 2.55712109265005e-08
4095 2.42702284936058e-08
4096 2.45965097178669e-08
4097 2.45820217514847e-08
4098 2.54285037470936e-08
4099 2.44873969990067e-08
4100 2.46692373195856e-08
4101 2.46207676468657e-08
4102 2.5689654847838e-08
4103 2.5046814400298e-08
4104 2.43324418391921e-08
4105 2.53311345232987e-08
4106 2.48538114533403e-08
4107 2.45070275184389e-08
4108 2.50309231120127e-08
4109 2.44743922905855e-08
4110 2.47619382776065e-08
4111 2.55593715081659e-08
4112 2.40879423074603e-08
4113 2.53014427187281e-08
4114 2.45168578771882e-08
4115 2.47445779422151e-08
4116 2.42543514161753e-08
4117 2.46345432941553e-08
4118 2.54158880608202e-08
4119 2.50366216647535e-08
4120 2.43500046792633e-08
4121 2.45928770681303e-08
4122 2.4855573599325e-08
4123 2.41316193694274e-08
4124 2.45254518915772e-08
4125 2.46226612432565e-08
4126 2.45455442637876e-08
4127 2.45472193682872e-08
4128 2.47177514012265e-08
4129 2.44644127178617e-08
4130 2.48293172688818e-08
4131 2.4566842782292e-08
4132 2.50192080386569e-08
4133 2.47164706479452e-08
4134 2.46821194593849e-08
4135 2.46241018686533e-08
4136 2.49028460075351e-08
4137 2.4781975582755e-08
4138 2.46942626347391e-08
4139 2.5059854635856e-08
4140 2.43533548882624e-08
4141 2.45992453073995e-08
4142 2.43534525878886e-08
4143 2.53667771232813e-08
4144 2.4014966015784e-08
4145 2.46777300816348e-08
4146 2.46069333798005e-08
4147 2.45307063551081e-08
4148 2.5129615721653e-08
4149 2.43322269000146e-08
4150 2.45791138553386e-08
4151 2.43552094048027e-08
4152 2.46403715209453e-08
4153 2.47462583757851e-08
4154 2.44167761564995e-08
4155 2.45889264505195e-08
4156 2.54419649792226e-08
4157 2.40048514399405e-08
4158 2.48602756158789e-08
4159 2.43758258022808e-08
4160 2.54517082964867e-08
4161 2.40009914165284e-08
4162 2.48916709466585e-08
4163 2.42225031144017e-08
4164 2.45902960216426e-08
4165 2.48541844882766e-08
4166 2.42115749671257e-08
4167 2.44812738969813e-08
4168 2.45396094555872e-08
4169 2.47587603752208e-08
4170 2.53135539196592e-08
4171 2.42909923287016e-08
4172 2.5291798877447e-08
4173 2.4111814767025e-08
4174 2.48791955925753e-08
4175 2.43589717285886e-08
4176 2.45964653089459e-08
4177 2.55605190346841e-08
4178 2.46442493079257e-08
4179 2.43606663730134e-08
4180 2.44806734883696e-08
4181 2.43148541301252e-08
4182 2.44497790902187e-08
4183 2.5190859176405e-08
4184 2.41823787661133e-08
4185 2.47346640946944e-08
4186 2.47527474073195e-08
4187 2.42670328276517e-08
4188 2.44963356266226e-08
4189 2.47254963170462e-08
4190 2.43460185345157e-08
4191 2.48354528054051e-08
4192 2.5076987597572e-08
4193 2.48336657904247e-08
4194 2.49166376420362e-08
4195 2.4631448880541e-08
4196 2.45138824794822e-08
4197 2.4777008889032e-08
4198 2.41770088393878e-08
4199 2.44981652741671e-08
4200 2.43655442488944e-08
4201 2.43758098150693e-08
4202 2.4476012328023e-08
4203 2.43958524492882e-08
4204 2.44909870161791e-08
4205 2.43868853999629e-08
4206 2.43146054401677e-08
4207 2.43215225737003e-08
4208 2.43659261656148e-08
4209 2.53902481262003e-08
4210 2.45660896069921e-08
4211 2.42784707893406e-08
4212 2.43707649616454e-08
4213 2.4713653345998e-08
4214 2.43713671466139e-08
4215 2.43412756617545e-08
4216 2.43826310253326e-08
4217 2.51249225868833e-08
4218 2.42199238442709e-08
4219 2.51388581062884e-08
4220 2.39888837683111e-08
4221 2.4555959043937e-08
4222 2.51080880531163e-08
4223 2.47357618832211e-08
4224 2.495078454956e-08
4225 2.47993590107853e-08
4226 2.44363356216581e-08
4227 2.47017339916056e-08
4228 2.45909248519638e-08
4229 2.46088944777512e-08
4230 2.47207623260692e-08
4231 2.46254998614859e-08
4232 2.4651580332602e-08
4233 2.46758293798166e-08
4234 2.46870612841121e-08
4235 2.47104630091144e-08
4236 2.46582132490403e-08
4237 2.470369864227e-08
4238 2.46994389385691e-08
4239 2.4660074871008e-08
4240 2.46807001502702e-08
4241 2.46627056554871e-08
4242 2.46057147990086e-08
4243 2.46720652796739e-08
4244 2.45948079680147e-08
4245 2.46105464896118e-08
4246 2.46220910327111e-08
4247 2.5076390741674e-08
4248 2.52728753480369e-08
4249 2.43534525878886e-08
4250 2.47038141054645e-08
4251 2.48499318900031e-08
4252 2.4587725633296e-08
4253 2.46811584503348e-08
4254 2.45658711151009e-08
4255 2.47920244333955e-08
4256 2.48981120165581e-08
4257 2.44219684475411e-08
4258 2.48182487894155e-08
4259 2.45221833949927e-08
4260 2.4702625722739e-08
4261 2.46850380136721e-08
4262 2.53271377204101e-08
4263 2.4667894393815e-08
4264 2.45675249033184e-08
4265 2.47209683834626e-08
4266 2.45232953943741e-08
4267 2.46606042253461e-08
4268 2.50769431886511e-08
4269 2.4327034608973e-08
4270 2.48757636711616e-08
4271 2.48604745678449e-08
4272 2.52781369169952e-08
4273 2.44169182650467e-08
4274 2.47817215637269e-08
4275 2.46339890708214e-08
4276 2.46952787108512e-08
4277 2.46573161888364e-08
4278 2.49384584094514e-08
4279 2.49894469561696e-08
4280 2.51824285868452e-08
4281 2.47032900801969e-08
4282 2.4719758684455e-08
4283 2.46907312373423e-08
4284 2.46907188028445e-08
4285 2.49747280633983e-08
4286 2.46576767892748e-08
4287 2.47457805357953e-08
4288 2.47164813060863e-08
4289 2.49944331898178e-08
4290 2.46469831211016e-08
4291 2.4939890153064e-08
4292 2.46289424410406e-08
4293 2.47120617302699e-08
4294 2.46794424896279e-08
4295 2.48120883838965e-08
4296 2.46429063821552e-08
4297 2.45718307922971e-08
4298 2.50551703828705e-08
4299 2.45221709604948e-08
4300 2.4771688700298e-08
4301 2.45423361633357e-08
4302 2.45211460025985e-08
4303 2.50989042882566e-08
4304 2.44162574603024e-08
4305 2.4651104268969e-08
4306 2.46066385045651e-08
4307 2.44052600351097e-08
4308 2.49431071353001e-08
4309 2.42937687744416e-08
4310 2.46258267111443e-08
4311 2.45591547098911e-08
4312 2.45258888753597e-08
4313 2.45440876511793e-08
4314 2.4582837099274e-08
4315 2.4728750602776e-08
4316 2.45857965097684e-08
4317 2.467013970886e-08
4318 2.45326230441378e-08
4319 2.44344864341883e-08
4320 2.4871832593476e-08
4321 2.40451445421286e-08
4322 2.4608498350176e-08
4323 2.50808422919135e-08
4324 2.41317970051114e-08
4325 2.45199807125118e-08
4326 2.43313902359432e-08
4327 2.46027411776595e-08
4328 2.50636098542145e-08
4329 2.41093065511677e-08
4330 2.46256810498835e-08
4331 2.42716993170689e-08
4332 2.47395721686416e-08
4333 2.4382703855963e-08
4334 2.4456898728431e-08
4335 2.4862021774652e-08
4336 2.40904078907533e-08
4337 2.45970070977819e-08
4338 2.43830982071813e-08
4339 2.42655602278319e-08
4340 2.43783464526359e-08
4341 2.46550513338661e-08
4342 2.45320119773851e-08
4343 2.45219844430267e-08
4344 2.41920457000333e-08
4345 2.44200855092913e-08
4346 2.41766446862357e-08
4347 2.43772468877523e-08
4348 2.43510260844459e-08
4349 2.44071358679321e-08
4350 2.4060780035029e-08
4351 2.45878712945569e-08
4352 2.44803732840637e-08
4353 2.44143105732064e-08
4354 2.42225599578205e-08
4355 2.414583732957e-08
4356 2.45581475155632e-08
4357 2.47834499589317e-08
4358 2.4297801104467e-08
4359 2.42901343483481e-08
4360 2.43884024087038e-08
4361 2.47606646297527e-08
4362 2.46263844871919e-08
4363 2.44034108476399e-08
4364 2.44152698058997e-08
4365 2.44960123296778e-08
4366 2.43135307442799e-08
4367 2.44096227675072e-08
4368 2.43517739306753e-08
4369 2.47294469346571e-08
4370 2.46748879106917e-08
4371 2.45430413770009e-08
4372 2.43985400771862e-08
4373 2.44183624431571e-08
4374 2.46661286951166e-08
4375 2.42322890642299e-08
4376 2.44693350026637e-08
4377 2.45136906329435e-08
4378 2.49285054820803e-08
4379 2.38742536851078e-08
4380 2.44647768710138e-08
4381 2.45215652228126e-08
4382 2.45165825418781e-08
4383 2.46738807163638e-08
4384 2.45803839504788e-08
4385 2.44489424261474e-08
4386 2.53679068862311e-08
4387 2.39852617767156e-08
4388 2.47496583227758e-08
4389 2.42431550390165e-08
4390 2.46737439368871e-08
4391 2.45110651775349e-08
4392 2.46125324565583e-08
4393 2.4687206945373e-08
4394 2.47531666275336e-08
4395 2.43825777346274e-08
4396 2.45070044258e-08
4397 2.45157796285866e-08
4398 2.47344402737326e-08
4399 2.45573730239812e-08
4400 2.44763747048182e-08
4401 2.50828868786357e-08
4402 2.4257772679448e-08
4403 2.47493261440468e-08
4404 2.52057006377981e-08
4405 2.42740831879473e-08
4406 2.46762397182465e-08
4407 2.48539517855306e-08
4408 2.44011779670927e-08
4409 2.46535662995484e-08
4410 2.45798830178501e-08
4411 2.43854252346409e-08
4412 2.44035067709092e-08
4413 2.48303297922803e-08
4414 2.46572611217744e-08
4415 2.44874591714961e-08
4416 2.43040272351891e-08
4417 2.42711024611708e-08
4418 2.43035280789172e-08
4419 2.4806231735397e-08
4420 2.4150471844564e-08
4421 2.46693616645643e-08
4422 2.45443700919168e-08
4423 2.46198599285208e-08
4424 2.46484770372035e-08
4425 2.43471962591002e-08
4426 2.51339482559843e-08
4427 2.36613058035573e-08
4428 2.46701699069263e-08
4429 2.443025337584e-08
4430 2.45532856268937e-08
4431 2.44759750245294e-08
4432 2.4419259503361e-08
4433 2.488342509821e-08
4434 2.41733957295764e-08
4435 2.48188900542345e-08
4436 2.42088660229456e-08
4437 2.45235813878253e-08
4438 2.42933033689496e-08
4439 2.48466829333438e-08
4440 2.41407906997892e-08
4441 2.48075959774496e-08
4442 2.41225937003264e-08
4443 2.44996858356217e-08
4444 2.44046773900664e-08
4445 2.47808014108841e-08
4446 2.42175559606039e-08
4447 2.45321984948532e-08
4448 2.42691218232949e-08
4449 2.44810589578037e-08
4450 2.44865709930764e-08
4451 2.43338469374521e-08
4452 2.4327908576538e-08
4453 2.46946605386711e-08
4454 2.41172433135262e-08
4455 2.47685623122607e-08
4456 2.40567157305804e-08
4457 2.47093527860898e-08
4458 2.41484965357586e-08
4459 2.45541738053134e-08
4460 2.49171350219513e-08
4461 2.40930422279462e-08
4462 2.45445122004639e-08
4463 2.43281998990597e-08
4464 2.4697282441366e-08
4465 2.42424107455008e-08
4466 2.47799576413854e-08
4467 2.40776003579413e-08
4468 2.45720457314746e-08
4469 2.43806379529588e-08
4470 2.45939659748728e-08
4471 2.40534152595728e-08
4472 2.44983375807806e-08
4473 2.47095197636327e-08
4474 2.41356694630213e-08
4475 2.42426612118152e-08
4476 2.44831035445259e-08
4477 2.45073543680974e-08
4478 2.39215474096e-08
4479 2.40563835518515e-08
4480 2.44846862784698e-08
4481 2.40348736468832e-08
4482 2.42569715425134e-08
4483 2.4070889281802e-08
4484 2.40829152176048e-08
4485 2.42154367668945e-08
4486 2.41512907450669e-08
4487 2.42279245554755e-08
4488 2.45301698953426e-08
4489 2.38863169244041e-08
4490 2.45905393825296e-08
4491 2.39888588993153e-08
4492 2.42840290098911e-08
4493 2.41314594973119e-08
4494 2.42572983921718e-08
4495 2.39540813851136e-08
4496 2.41150726054684e-08
4497 2.46073330600893e-08
4498 2.39129143153605e-08
4499 2.42648976467308e-08
4500 2.4250505603618e-08
4501 2.37999859820093e-08
4502 2.4176475932336e-08
4503 2.41968862724207e-08
4504 2.41993625138548e-08
4505 2.43209434813707e-08
4506 2.46434748163438e-08
4507 2.39072619478975e-08
4508 2.4167992052071e-08
4509 2.38812241093456e-08
4510 2.45939055787403e-08
4511 2.38487185555414e-08
4512 2.41910900200537e-08
4513 2.40057431710738e-08
4514 2.42003821426806e-08
4515 2.39223147957546e-08
4516 2.40834499010134e-08
4517 2.44729800868981e-08
4518 2.37389308210822e-08
4519 2.41141542289824e-08
4520 2.42225262070406e-08
4521 2.36701591660449e-08
4522 2.41627553521084e-08
4523 2.384548203338e-08
4524 2.38851729505996e-08
4525 2.39366499954485e-08
4526 2.39468622709182e-08
4527 2.40579165478039e-08
4528 2.39469137852666e-08
4529 2.39035102822527e-08
4530 2.39501734000669e-08
4531 2.385034925112e-08
4532 2.37683686066248e-08
4533 2.42809825579116e-08
4534 2.35471553367006e-08
4535 2.38839046318162e-08
4536 2.37550672466114e-08
4537 2.37018404902756e-08
4538 2.3874582311123e-08
4539 2.37662192148491e-08
4540 2.38972184263275e-08
4541 2.38848745226505e-08
4542 2.37963817539821e-08
4543 2.37843398309678e-08
4544 2.37102959488311e-08
4545 2.38715500699982e-08
4546 2.35803501169585e-08
4547 2.38933210994219e-08
4548 2.38224124871067e-08
4549 2.38414852304913e-08
4550 2.38300152943793e-08
4551 2.39008350888525e-08
4552 2.38522375184402e-08
4553 2.35688553118507e-08
4554 2.38867663426845e-08
4555 2.40370283677294e-08
4556 2.39417055070135e-08
4557 2.39237536447945e-08
4558 2.37944579595251e-08
4559 2.38171011801569e-08
4560 2.37957813453704e-08
4561 2.38617214876058e-08
4562 2.37561490479266e-08
4563 2.3926501668825e-08
4564 2.39994442097213e-08
4565 2.37266064573305e-08
4566 2.39026896053929e-08
4567 2.41011655077727e-08
4568 2.36914310391967e-08
4569 2.38536852492643e-08
4570 2.38335129409961e-08
4571 2.37811050851633e-08
4572 2.38522321893697e-08
4573 2.38509425543043e-08
4574 2.38438264688057e-08
4575 2.40846098620295e-08
4576 2.37116299928175e-08
4577 2.38367618976554e-08
4578 2.38924275919317e-08
4579 2.40172646215342e-08
4580 2.36430626188167e-08
4581 2.38446649092339e-08
4582 2.3714688879295e-08
4583 2.3727087850034e-08
4584 2.38866988411246e-08
4585 2.3917255731476e-08
4586 2.38096635740703e-08
4587 2.37931043756134e-08
4588 2.35188988284563e-08
4589 2.37696280436239e-08
4590 2.36998847213954e-08
4591 2.397437270929e-08
4592 2.36527668562303e-08
4593 2.38702000388002e-08
4594 2.38672708263721e-08
4595 2.39469812868265e-08
4596 2.37591990526198e-08
4597 2.38016948372888e-08
4598 2.40564368425566e-08
4599 2.36753638915843e-08
4600 2.3897809953155e-08
4601 2.37930404267672e-08
4602 2.36487096572091e-08
4603 2.39109692046213e-08
4604 2.38996555879112e-08
4605 2.35687469540835e-08
4606 2.36759518656982e-08
4607 2.37545059178501e-08
4608 2.37531398994406e-08
4609 2.39857822492695e-08
4610 2.35780941437724e-08
4611 2.38776571848121e-08
4612 2.3629686651816e-08
4613 2.36499246852873e-08
4614 2.36689672306056e-08
4615 2.35480381860498e-08
4616 2.36259385388848e-08
4617 2.40107684845725e-08
4618 2.35996253650228e-08
4619 2.38929303009172e-08
4620 2.36512054385685e-08
4621 2.3559417527963e-08
4622 2.36536621400774e-08
4623 2.37633255295577e-08
4624 2.37119017754139e-08
4625 2.38947155395408e-08
4626 2.36204087400438e-08
4627 2.39735857832102e-08
4628 2.35139410165175e-08
4629 2.3890288858297e-08
4630 2.3769660018047e-08
4631 2.36558754806993e-08
4632 2.37112960377317e-08
4633 2.37406840852827e-08
4634 2.36641461981435e-08
4635 2.40876403267976e-08
4636 2.36327668545755e-08
4637 2.35671961945627e-08
4638 2.37093882304862e-08
4639 2.37055353125015e-08
4640 2.3560733808381e-08
4641 2.37747581621761e-08
4642 2.3588739850311e-08
4643 2.37293082960832e-08
4644 2.35809540782839e-08
4645 2.38095925197968e-08
4646 2.36947776954821e-08
4647 2.36655424146193e-08
4648 2.37117028234479e-08
4649 2.36888677562774e-08
4650 2.36469954728591e-08
4651 2.36591919389184e-08
4652 2.36699513322947e-08
4653 2.34087007555672e-08
4654 2.36405490738889e-08
4655 2.3685034378218e-08
4656 2.36873489711797e-08
4657 2.3697754869545e-08
4658 2.35322605846022e-08
4659 2.37121469126578e-08
4660 2.37186270624079e-08
4661 2.35266828241265e-08
4662 2.37105854949959e-08
4663 2.38170940747295e-08
4664 2.35214194788114e-08
4665 2.35879156207375e-08
4666 2.37237980371674e-08
4667 2.36628263650118e-08
4668 2.34045032243557e-08
4669 2.34789023778603e-08
4670 2.37198012342787e-08
4671 2.35472938925341e-08
4672 2.35311485852208e-08
4673 2.35614390220462e-08
4674 2.37304043082531e-08
4675 2.33211814304468e-08
4676 2.36927473196147e-08
4677 2.37865318553077e-08
4678 2.36072974502122e-08
4679 2.36779200690762e-08
4680 2.34745893834543e-08
4681 2.36496582317614e-08
4682 2.37105428624318e-08
4683 2.37386217349922e-08
4684 2.36147048582325e-08
4685 2.35878445664639e-08
4686 2.34859598435833e-08
4687 2.362739692785e-08
4688 2.37222383958624e-08
4689 2.36390889085669e-08
4690 2.3709334939781e-08
4691 2.35190178443645e-08
4692 2.32009433887015e-08
4693 2.38291040233207e-08
4694 2.3521600667209e-08
4695 2.34941737176086e-08
4696 2.34017516476115e-08
4697 2.36599699832141e-08
4698 2.36681021448248e-08
4699 2.36373320916528e-08
4700 2.38192097157253e-08
4701 2.36702000222522e-08
4702 2.36813200160668e-08
4703 2.34310011393291e-08
4704 2.36093633532164e-08
4705 2.36327224456545e-08
4706 2.36169697132027e-08
4707 2.34873880344821e-08
4708 2.341308302789e-08
4709 2.35533956782774e-08
4710 2.36001955755683e-08
4711 2.36007906551094e-08
4712 2.34165575818679e-08
4713 2.35589094899069e-08
4714 2.34356942740988e-08
4715 2.36087025484721e-08
4716 2.33484911404958e-08
4717 2.37030270966443e-08
4718 2.35650574609281e-08
4719 2.35585240204728e-08
4720 2.35300667839056e-08
4721 2.34134933663199e-08
4722 2.35984867202887e-08
4723 2.37471589059624e-08
4724 2.34039632118765e-08
4725 2.3642517277267e-08
4726 2.32051835524771e-08
4727 2.3483927691359e-08
4728 2.3521257830339e-08
4729 2.35009469662373e-08
4730 2.34881198935e-08
4731 2.33153976125777e-08
4732 2.35694042061141e-08
4733 2.35522676916844e-08
4734 2.35437607187805e-08
4735 2.32625456675351e-08
4736 2.33328965038027e-08
4737 2.34484645034172e-08
4738 2.34890400463428e-08
4739 2.34974901758278e-08
4740 2.35868409248496e-08
4741 2.34625261441579e-08
4742 2.36341310966282e-08
4743 2.34078871841348e-08
4744 2.35868000686423e-08
4745 2.34670221033184e-08
4746 2.35557848782264e-08
4747 2.33901573665207e-08
4748 2.33710828467792e-08
4749 2.3487245925935e-08
4750 2.35313706298257e-08
4751 2.34334489590537e-08
4752 2.3610915889094e-08
4753 2.35317223484799e-08
4754 2.34980941371532e-08
4755 2.34890613626249e-08
4756 2.34368648932559e-08
4757 2.32477326278513e-08
4758 2.34480772576262e-08
4759 2.34511201568921e-08
4760 2.33720065523357e-08
4761 2.32285142232058e-08
4762 2.34389698761106e-08
4763 2.34429702317129e-08
4764 2.35107791013434e-08
4765 2.34350885364165e-08
4766 2.34429951007087e-08
4767 2.33889139167331e-08
4768 2.34876971205722e-08
4769 2.34924026898398e-08
4770 2.32450183546007e-08
4771 2.35894681566151e-08
4772 2.34187780279171e-08
4773 2.3532450654784e-08
4774 2.33658710158124e-08
4775 2.33933210580517e-08
4776 2.34741328597465e-08
4777 2.32683703416114e-08
4778 2.34965842338397e-08
4779 2.33794388293518e-08
4780 2.34581332136941e-08
4781 2.32194956595322e-08
4782 2.36985240320564e-08
4783 2.3172932017701e-08
4784 2.34848975821933e-08
4785 2.33761472401284e-08
4786 2.34054429171238e-08
4787 2.33971206853312e-08
4788 2.34225208117778e-08
4789 2.34373001006816e-08
4790 2.34183836766988e-08
4791 2.34470647342278e-08
4792 2.32386376808336e-08
4793 2.3397166870609e-08
4794 2.3804799909044e-08
4795 2.30143797352866e-08
4796 2.33716637154657e-08
4797 2.32814301170947e-08
4798 2.35637234169417e-08
4799 2.33220500689413e-08
4800 2.32132464361712e-08
4801 2.33823715944936e-08
4802 2.34670096688205e-08
4803 2.32958008439255e-08
4804 2.32692585200311e-08
4805 2.31991439392232e-08
4806 2.32513297504511e-08
4807 2.33020127637928e-08
4808 2.36474893000604e-08
4809 2.30298038417232e-08
4810 2.34170798307787e-08
4811 2.30704362280676e-08
4812 2.31869865530143e-08
4813 2.33120349690807e-08
4814 2.34964581125041e-08
4815 2.31773356063059e-08
4816 2.32953798473545e-08
4817 2.33112213976483e-08
4818 2.32839347802383e-08
4819 2.33586039399825e-08
4820 2.32400036992431e-08
4821 2.33726442644411e-08
4822 2.33778401081963e-08
4823 2.32387975529491e-08
4824 2.37183641615957e-08
4825 2.33619950051889e-08
4826 2.32375914066552e-08
4827 2.33135288851827e-08
4828 2.32359411711514e-08
4829 2.32613839301621e-08
4830 2.32939871835924e-08
4831 2.31054446686585e-08
4832 2.31164811737017e-08
4833 2.32722072723845e-08
4834 2.33287646977942e-08
4835 2.3253653225197e-08
4836 2.31389059024423e-08
4837 2.33303367735971e-08
4838 2.32140902056699e-08
4839 2.32773427200073e-08
4840 2.3269846494145e-08
4841 2.31681624995872e-08
4842 2.33113777170502e-08
4843 2.32251871068456e-08
4844 2.31134826833568e-08
4845 2.32068320116241e-08
4846 2.33886527922778e-08
4847 2.31228352021162e-08
4848 2.34135946186598e-08
4849 2.3065521048693e-08
4850 2.31601315903163e-08
4851 2.33210339928291e-08
4852 2.30375061249788e-08
4853 2.32822774393071e-08
4854 2.35475532406326e-08
4855 2.29577246102508e-08
4856 2.310062718891e-08
4857 2.31107435411104e-08
4858 2.31517933713121e-08
4859 2.31596519739696e-08
4860 2.32179786507913e-08
4861 2.31953602991553e-08
4862 2.33945307570593e-08
4863 2.33849881681181e-08
4864 2.30893135721999e-08
4865 2.30698216086012e-08
4866 2.31610091105949e-08
4867 2.31005543582796e-08
4868 2.32232988395253e-08
4869 2.30166978809621e-08
4870 2.29833378995181e-08
4871 2.32266490485245e-08
4872 2.3316301778209e-08
4873 2.31274395190439e-08
4874 2.31200854017288e-08
4875 2.30799894751499e-08
4876 2.31500596470369e-08
4877 2.31107737391767e-08
4878 2.31671464234751e-08
4879 2.31844765608002e-08
4880 2.30418351065964e-08
4881 2.32629613350355e-08
4882 2.30247927390792e-08
4883 2.34467272264283e-08
4884 2.29755503511342e-08
4885 2.31908643399947e-08
4886 2.31338894707278e-08
4887 2.30951595625584e-08
4888 2.29756036418394e-08
4889 2.31033538966585e-08
4890 2.30848282711804e-08
4891 2.32051604598382e-08
4892 2.30572663184603e-08
4893 2.31259651428672e-08
4894 2.29792949113516e-08
4895 2.31140511175454e-08
4896 2.31288463936608e-08
4897 2.31084946733517e-08
4898 2.29677414864682e-08
4899 2.31598864530724e-08
4900 2.30208296869705e-08
4901 2.31546177786868e-08
4902 2.32569359326362e-08
4903 2.28859367013001e-08
4904 2.30736887374405e-08
4905 2.30014070012885e-08
4906 2.31645280734938e-08
4907 2.29947740848502e-08
4908 2.30756178609681e-08
4909 2.2997662441071e-08
4910 2.31142411877272e-08
4911 2.28746284136605e-08
4912 2.30281944624267e-08
4913 2.29818422070593e-08
4914 2.31146444207297e-08
4915 2.29692744824206e-08
4916 2.3040456653689e-08
4917 2.29084431424553e-08
4918 2.31212364809608e-08
4919 2.29884431490746e-08
4920 2.30286527624912e-08
4921 2.29645760185804e-08
4922 2.27161418564492e-08
4923 2.33904717816813e-08
4924 2.27750138748206e-08
4925 2.32886048223691e-08
4926 2.28205383479008e-08
4927 2.30670877954253e-08
4928 2.29195702416973e-08
4929 2.31618138002432e-08
4930 2.27614034287171e-08
4931 2.2952457712222e-08
4932 2.28145324854268e-08
4933 2.29722019184919e-08
4934 2.29713847943458e-08
4935 2.30398562450773e-08
4936 2.29070327151248e-08
4937 2.28703296301092e-08
4938 2.30274590506951e-08
4939 2.28782646161108e-08
4940 2.29382237648679e-08
4941 2.29377459248781e-08
4942 2.30755645702629e-08
4943 2.30521415289786e-08
4944 2.29082743885556e-08
4945 2.2777266295293e-08
4946 2.28842278460206e-08
4947 2.28987833139627e-08
4948 2.28504841714994e-08
4949 2.29205294743906e-08
4950 2.28520207201655e-08
4951 2.28489813736132e-08
4952 2.28034462423921e-08
4953 2.2853939185552e-08
4954 2.28641070521007e-08
4955 2.29046275279643e-08
4956 2.28892194087393e-08
4957 2.28498162613278e-08
4958 2.2861504689331e-08
4959 2.26981509143798e-08
4960 2.28048833150751e-08
4961 2.28661054535451e-08
4962 2.27309531197761e-08
4963 2.33941133132021e-08
4964 2.2525268761342e-08
4965 2.28805792090725e-08
4966 2.28263736801182e-08
4967 2.29270202822818e-08
4968 2.28097878363087e-08
4969 2.27825545096039e-08
4970 2.28485781406107e-08
4971 2.28621637177184e-08
4972 2.27901288951671e-08
4973 2.27592753532235e-08
4974 2.28385861333891e-08
4975 2.26196092967257e-08
4976 2.29239436322359e-08
4977 2.2687297374091e-08
4978 2.29237677729088e-08
4979 2.30104699738831e-08
4980 2.27923493412163e-08
4981 2.2875939365008e-08
4982 2.29351755365315e-08
4983 2.26047589535483e-08
4984 2.28180461192551e-08
4985 2.28420002912344e-08
4986 2.29487060465772e-08
4987 2.28426753068334e-08
4988 2.29665495510289e-08
4989 2.26314131879235e-08
4990 2.28458105766549e-08
4991 2.27827179344331e-08
4992 2.29142322893949e-08
4993 2.27169500988111e-08
4994 2.27541274711029e-08
4995 2.27180727563336e-08
4996 2.27663541352285e-08
4997 2.27109886452581e-08
4998 2.30543211188206e-08
4999 2.28131593615899e-08
};
\addlegendentry{Test}

\nextgroupplot[
title={Batch Size 4 $\hy$},
ymin=8.04913970636907e-09, ymax=1e-05,
]
\addplot [semithick, black, dashed]
table {%
0 0.00559679022408091
1 0.000650229895545635
2 0.00022808890736269
3 0.000198959646657386
4 0.000171832855196044
5 0.000105786866648486
6 3.8532815532335e-05
7 2.02961691232986e-05
8 1.75136847517479e-05
9 1.72895896983221e-05
10 1.72592997587628e-05
11 1.72361989505703e-05
12 1.72130318063068e-05
13 1.71918167245622e-05
14 1.71715817538427e-05
15 1.71538061258332e-05
16 1.71358313915846e-05
17 1.71181113863952e-05
18 1.71003660451845e-05
19 1.70817074534853e-05
20 1.70614798353768e-05
21 1.70387688371818e-05
22 1.70123199201839e-05
23 1.69804432084675e-05
24 1.6943170201273e-05
25 1.68918474809345e-05
26 1.68180439846708e-05
27 1.67100975685912e-05
28 1.65551271759625e-05
29 1.63274087362595e-05
30 1.59815374233006e-05
31 1.52596990158997e-05
32 1.42521224295109e-05
33 1.2944315643395e-05
34 1.11161244223581e-05
35 8.76415773501549e-06
36 6.17208484857201e-06
37 3.87005008831665e-06
38 2.37853139041988e-06
39 1.66197402278101e-06
40 1.33014761965633e-06
41 1.13131407185207e-06
42 9.80837486956432e-07
43 8.60497913759417e-07
44 7.74547995231245e-07
45 7.17578383891748e-07
46 6.80854326091662e-07
47 6.56917669710211e-07
48 6.40063947953351e-07
49 6.27874525576999e-07
50 6.17305873061191e-07
51 6.08705471756288e-07
52 5.99146671201645e-07
53 5.884408456307e-07
54 5.77623485912504e-07
55 5.66430369564053e-07
56 5.5333911475941e-07
57 5.40790307265837e-07
58 5.30481150327944e-07
59 5.20389879584471e-07
60 5.10952432643208e-07
61 5.03094480477806e-07
62 4.98124654300547e-07
63 4.9444734322357e-07
64 4.90634941710866e-07
65 4.87732961593323e-07
66 4.84098780171749e-07
67 4.812135990786e-07
68 4.79436245765186e-07
69 4.77795843750073e-07
70 4.76231751225242e-07
71 4.74265091112258e-07
72 4.72610436894172e-07
73 4.70659639232096e-07
74 4.68898033871845e-07
75 4.67320104490909e-07
76 4.66468047606305e-07
77 4.65256960544558e-07
78 4.63622722908674e-07
79 4.62372012838053e-07
80 4.6148927869627e-07
81 4.6042160249371e-07
82 4.59728487383515e-07
83 4.59122625557029e-07
84 4.58281142615924e-07
85 4.59081047058163e-07
86 4.58610283135208e-07
87 4.57105142857017e-07
88 4.56634078556561e-07
89 4.55665628875224e-07
90 4.55135134666662e-07
91 4.54372307812534e-07
92 4.54291537413276e-07
93 4.53450837290426e-07
94 4.52989366218404e-07
95 4.52057548272933e-07
96 4.51730267237771e-07
97 4.51180828769715e-07
98 4.50607211563536e-07
99 4.50157004237894e-07
100 4.49340749271165e-07
101 4.48797108050236e-07
102 4.48454544638821e-07
103 4.47662613131783e-07
104 4.47396111975529e-07
105 4.46758818931414e-07
106 4.46047449354836e-07
107 4.45654567869269e-07
108 4.45169298206594e-07
109 4.44437815917631e-07
110 4.43916837070901e-07
111 4.43363399562635e-07
112 4.42705018522282e-07
113 4.42152956651398e-07
114 4.41781581011824e-07
115 4.41023435366006e-07
116 4.40329138498896e-07
117 4.39712273583304e-07
118 4.39068628346639e-07
119 4.38816359000782e-07
120 4.37820470383876e-07
121 4.37267052563861e-07
122 4.36763951123353e-07
123 4.3606052577605e-07
124 4.35591049768647e-07
125 4.35094550391923e-07
126 4.3481689523972e-07
127 4.34209437384148e-07
128 4.33537395207573e-07
129 4.33057221740363e-07
130 4.3248709403354e-07
131 4.32051967412583e-07
132 4.31272475651667e-07
133 4.30781255699841e-07
134 4.30178314687168e-07
135 4.29778016453497e-07
136 4.29021419632747e-07
137 4.28491167234313e-07
138 4.27833241452014e-07
139 4.27218963934273e-07
140 4.26612634090162e-07
141 4.25994118730344e-07
142 4.25311039462528e-07
143 4.24783086170777e-07
144 4.2366307566688e-07
145 4.23714197415492e-07
146 4.23068518008662e-07
147 4.22409593811679e-07
148 4.21736104569526e-07
149 4.21005406294306e-07
150 4.20434856412299e-07
151 4.19709783873401e-07
152 4.19077305680027e-07
153 4.18374291209389e-07
154 4.17766423758792e-07
155 4.16966582806211e-07
156 4.16339270199906e-07
157 4.15630936739575e-07
158 4.14883340948968e-07
159 4.14269270806678e-07
160 4.13500222324892e-07
161 4.12834297640607e-07
162 4.11994347222944e-07
163 4.11390230475206e-07
164 4.10201951845934e-07
165 4.09845223167338e-07
166 4.09065456963731e-07
167 4.08140571778404e-07
168 4.07440803430248e-07
169 4.06449339527271e-07
170 4.05886309055248e-07
171 4.04602727828163e-07
172 4.03921648455707e-07
173 4.03150075203307e-07
174 4.02373743473738e-07
175 4.01748700187277e-07
176 4.0093339191305e-07
177 4.00197581207351e-07
178 3.99477407200699e-07
179 3.98817781757543e-07
180 3.98069259356149e-07
181 3.97315670434217e-07
182 3.9653764962555e-07
183 3.95869676526495e-07
184 3.95285978456172e-07
185 3.94405617575977e-07
186 3.93638960282772e-07
187 3.92447187257616e-07
188 3.89958435068394e-07
189 3.88443209180522e-07
190 3.87202330841774e-07
191 3.85949914578632e-07
192 3.8484248946169e-07
193 3.83430756988545e-07
194 3.82287035693096e-07
195 3.81119747945746e-07
196 3.79561695690533e-07
197 3.7831185887427e-07
198 3.77086781950808e-07
199 3.75848411843904e-07
200 3.74601995524415e-07
201 3.73352982029118e-07
202 3.72085744908546e-07
203 3.71019857780652e-07
204 3.69706229728095e-07
205 3.68491959232387e-07
206 3.67096292327673e-07
207 3.65789885718115e-07
208 3.64192743921521e-07
209 3.62603918693694e-07
210 3.60898078584526e-07
211 3.59503568256869e-07
212 3.57776730426806e-07
213 3.56326932067219e-07
214 3.5473863792479e-07
215 3.53278613636654e-07
216 3.51775589834347e-07
217 3.5034518800714e-07
218 3.48805684968312e-07
219 3.47371593447221e-07
220 3.45916424437931e-07
221 3.4516325392886e-07
222 3.43494031717739e-07
223 3.41965334452965e-07
224 3.40358679951081e-07
225 3.38899196059117e-07
226 3.37373102860639e-07
227 3.35958377149836e-07
228 3.34161323454651e-07
229 3.32541685494903e-07
230 3.30463636700529e-07
231 3.28358795616879e-07
232 3.26395371645916e-07
233 3.24619051994546e-07
234 3.22842014636393e-07
235 3.21152230613464e-07
236 3.19553879139178e-07
237 3.17740120301835e-07
238 3.1557038383756e-07
239 3.13968921267005e-07
240 3.11854757415908e-07
241 3.10068355150683e-07
242 3.08247088566027e-07
243 3.06494076293795e-07
244 3.04229089413965e-07
245 3.02182450686139e-07
246 2.99869795647112e-07
247 2.97931116883809e-07
248 2.95347551760194e-07
249 2.9338107704735e-07
250 2.91671808531291e-07
251 2.89726868358819e-07
252 2.87721102631977e-07
253 2.8566917967332e-07
254 2.83292869603713e-07
255 2.81603690290844e-07
256 2.79713743500665e-07
257 2.77705699816444e-07
258 2.76110084116432e-07
259 2.74407457635739e-07
260 2.72692113147954e-07
261 2.71083824324947e-07
262 2.69431324505298e-07
263 2.6789902100699e-07
264 2.65888011587556e-07
265 2.64243395460717e-07
266 2.62376686563037e-07
267 2.60698919778868e-07
268 2.5849622707419e-07
269 2.56909966728003e-07
270 2.55157448081356e-07
271 2.53463792078357e-07
272 2.51854621924785e-07
273 2.502732354559e-07
274 2.4873732981856e-07
275 2.47195514003984e-07
276 2.45477830456586e-07
277 2.44133535085567e-07
278 2.42267136749241e-07
279 2.40760273463536e-07
280 2.39097631743235e-07
281 2.37912612269575e-07
282 2.35989415426907e-07
283 2.3457954172823e-07
284 2.3328364878239e-07
285 2.31658512711341e-07
286 2.30210311523038e-07
287 2.28926698680354e-07
288 2.27363944406811e-07
289 2.2550899469298e-07
290 2.24070869666804e-07
291 2.22989297398613e-07
292 2.21488576742424e-07
293 2.20654359334382e-07
294 2.19626671229101e-07
295 2.18291985356167e-07
296 2.17341388698777e-07
297 2.16220590183447e-07
298 2.15278259906526e-07
299 2.14295841342249e-07
300 2.13248371418118e-07
301 2.12387105121614e-07
302 2.11241901561188e-07
303 2.10424578478197e-07
304 2.09279741389956e-07
305 2.08391964490851e-07
306 2.07561679105517e-07
307 2.06355041490802e-07
308 2.05660997961488e-07
309 2.03559605020942e-07
310 2.00839540843489e-07
311 1.99000485748613e-07
312 1.96857315030385e-07
313 1.94953900345851e-07
314 1.93307805989917e-07
315 1.91826169518627e-07
316 1.90052627475623e-07
317 1.89225944198235e-07
318 1.87790508533681e-07
319 1.86392842758565e-07
320 1.8536443296302e-07
321 1.84249016871085e-07
322 1.83361070305033e-07
323 1.82481331915785e-07
324 1.81127779130819e-07
325 1.80076686278952e-07
326 1.79337995389606e-07
327 1.78085164750375e-07
328 1.77387791722161e-07
329 1.76548674612587e-07
330 1.75859899866015e-07
331 1.75281949886585e-07
332 1.74390577765138e-07
333 1.73811509779132e-07
334 1.73045633730773e-07
335 1.72575666893771e-07
336 1.72271889227638e-07
337 1.71609930565086e-07
338 1.70947867687232e-07
339 1.70258006745527e-07
340 1.69703592441728e-07
341 1.693896447863e-07
342 1.68926965434579e-07
343 1.68460895939582e-07
344 1.67702709480189e-07
345 1.66966248511091e-07
346 1.65969406645594e-07
347 1.65039101744036e-07
348 1.64378518905384e-07
349 1.63761609597302e-07
350 1.63335836451672e-07
351 1.62822603737567e-07
352 1.62165696404593e-07
353 1.60824545470284e-07
354 1.59895336634186e-07
355 1.59290404415202e-07
356 1.58436855207889e-07
357 1.57093827890975e-07
358 1.55511351459303e-07
359 1.54342790770201e-07
360 1.53133496241686e-07
361 1.52210367351913e-07
362 1.51068207926208e-07
363 1.49659381697198e-07
364 1.48536680569578e-07
365 1.47741405180035e-07
366 1.46722633637353e-07
367 1.45986463984116e-07
368 1.45218134121272e-07
369 1.44498953349448e-07
370 1.44107058836163e-07
371 1.42836462084794e-07
372 1.41731179476245e-07
373 1.40349586278177e-07
374 1.39184194809516e-07
375 1.37710327114959e-07
376 1.36465775750771e-07
377 1.35583795453975e-07
378 1.3483018430982e-07
379 1.33817019009719e-07
380 1.33071867165491e-07
381 1.31744151242508e-07
382 1.30865041031214e-07
383 1.2984276367245e-07
384 1.28985905648804e-07
385 1.28526717955602e-07
386 1.27756412641844e-07
387 1.27038029579474e-07
388 1.25597402726996e-07
389 1.2489029839724e-07
390 1.24074737845703e-07
391 1.23293769975774e-07
392 1.22498451504249e-07
393 1.21924711412369e-07
394 1.21215965595667e-07
395 1.20460255252475e-07
396 1.19762620319097e-07
397 1.19082195497278e-07
398 1.18654493253878e-07
399 1.17642384770633e-07
400 1.16953168754375e-07
401 1.16530827686923e-07
402 1.1561032544094e-07
403 1.15210414978861e-07
404 1.1483768556797e-07
405 1.14273961353994e-07
406 1.14089245351057e-07
407 1.13495536617325e-07
408 1.1335229106324e-07
409 1.1309990809405e-07
410 1.12561666108402e-07
411 1.12393458590265e-07
412 1.11988542602859e-07
413 1.11819032238447e-07
414 1.1162124857611e-07
415 1.11120225001926e-07
416 1.11321021905031e-07
417 1.10557313645288e-07
418 1.10682789114414e-07
419 1.10364844443822e-07
420 1.10302313633959e-07
421 1.09874972787249e-07
422 1.09834111172713e-07
423 1.09472408971634e-07
424 1.08999939870458e-07
425 1.09486131994352e-07
426 1.09036677886198e-07
427 1.09187654118603e-07
428 1.08662391301806e-07
429 1.08853631505035e-07
430 1.08448001395089e-07
431 1.08681620325957e-07
432 1.0807346309738e-07
433 1.07981647779809e-07
434 1.07954196331583e-07
435 1.07859034720992e-07
436 1.0756920401489e-07
437 1.07675178560029e-07
438 1.07134179085477e-07
439 1.07421971810684e-07
440 1.07143619304306e-07
441 1.0710420692206e-07
442 1.07032891234926e-07
443 1.06952521691461e-07
444 1.06730727189586e-07
445 1.06755134840153e-07
446 1.06469430565248e-07
447 1.06401053057148e-07
448 1.0637754384879e-07
449 1.06390662591327e-07
450 1.06073922954142e-07
451 1.06205954537142e-07
452 1.06184668756892e-07
453 1.06024735889942e-07
454 1.06218097711608e-07
455 1.05746077766256e-07
456 1.05833259282928e-07
457 1.05579934205569e-07
458 1.05758399565836e-07
459 1.0556819988361e-07
460 1.05563702247125e-07
461 1.05530675696297e-07
462 1.0544973869564e-07
463 1.05337505231784e-07
464 1.05173917907209e-07
465 1.05471277975777e-07
466 1.05178932122918e-07
467 1.0534417782404e-07
468 1.04966547632035e-07
469 1.05195173840666e-07
470 1.05111500726451e-07
471 1.05090745526226e-07
472 1.04966924979077e-07
473 1.04938281929368e-07
474 1.04908661585856e-07
475 1.05008037369014e-07
476 1.04611206418692e-07
477 1.04801478031824e-07
478 1.04782514241286e-07
479 1.04689883161591e-07
480 1.04544049411182e-07
481 1.04615130902541e-07
482 1.04617409967478e-07
483 1.04376127829209e-07
484 1.04549977154811e-07
485 1.04120880144443e-07
486 1.04571014701627e-07
487 1.04395247915168e-07
488 1.04385515356409e-07
489 1.03989649764102e-07
490 1.04226893304471e-07
491 1.03957686757639e-07
492 1.04061532394617e-07
493 1.03828463665412e-07
494 1.03990379021646e-07
495 1.03934915557602e-07
496 1.03852461214693e-07
497 1.03918123759428e-07
498 1.03901493268665e-07
499 1.03839756969037e-07
500 1.03639028375735e-07
501 1.03686271829329e-07
502 1.03679622830644e-07
503 1.03933912699805e-07
504 1.03487217063325e-07
505 1.03547756728606e-07
506 1.03771384022178e-07
507 1.03381495972332e-07
508 1.03533845402737e-07
509 1.03321952743762e-07
510 1.03661168436187e-07
511 1.0328716839858e-07
512 1.03445341974151e-07
513 1.03461927035475e-07
514 1.0333407885188e-07
515 1.03236449739974e-07
516 1.03273602328979e-07
517 1.03059259253691e-07
518 1.03285237972095e-07
519 1.03226117496469e-07
520 1.03341810183899e-07
521 1.03167999611742e-07
522 1.03070642993863e-07
523 1.03172266922691e-07
524 1.02946019620909e-07
525 1.02731450506433e-07
526 1.02958741353909e-07
527 1.03052523128699e-07
528 1.02945764335782e-07
529 1.03041097239132e-07
530 1.02734958418171e-07
531 1.02783368790771e-07
532 1.02744121779796e-07
533 1.02883643090301e-07
534 1.02641141247695e-07
535 1.02609824496724e-07
536 1.02829026379148e-07
537 1.02506855535367e-07
538 1.02777261290932e-07
539 1.02751935956213e-07
540 1.02406487365592e-07
541 1.02558882086434e-07
542 1.02455440054605e-07
543 1.02709343133611e-07
544 1.0215489846388e-07
545 1.02493679095872e-07
546 1.02412463927948e-07
547 1.02357481193582e-07
548 1.02337649247275e-07
549 1.02396486023615e-07
550 1.02414892726443e-07
551 1.02175902793711e-07
552 1.02410442323286e-07
553 1.02102513240165e-07
554 1.02253519769668e-07
555 1.02130720262661e-07
556 1.02059160980605e-07
557 1.0201568281154e-07
558 1.02144103613533e-07
559 1.02000102132571e-07
560 1.01980689796832e-07
561 1.01964962323731e-07
562 1.01909250863841e-07
563 1.01929628403497e-07
564 1.01816166493407e-07
565 1.0187910058912e-07
566 1.01885792462397e-07
567 1.01749776620252e-07
568 1.01818655780406e-07
569 1.01665981999233e-07
570 1.01705785944706e-07
571 1.01846664570804e-07
572 1.01652676470998e-07
573 1.01617355155348e-07
574 1.01743166411872e-07
575 1.01682773727241e-07
576 1.01348977712412e-07
577 1.01468721488196e-07
578 1.01442688515974e-07
579 1.01369227332171e-07
580 1.01494299143745e-07
581 1.0125527259941e-07
582 1.01451802165808e-07
583 1.01311211978761e-07
584 1.01186979984291e-07
585 1.01347010985187e-07
586 1.01019958755444e-07
587 1.01050336205333e-07
588 1.01327751104385e-07
589 1.01104485420667e-07
590 1.01184879030214e-07
591 1.0090931366058e-07
592 1.01190592459943e-07
593 1.01093189710966e-07
594 1.0103422012131e-07
595 1.01009416976439e-07
596 1.00698530709487e-07
597 1.00804672953814e-07
598 1.00961439881253e-07
599 1.00821274743978e-07
600 1.00832992846911e-07
601 1.00874150780417e-07
602 1.00766831080179e-07
603 1.00550136774125e-07
604 1.00720323538805e-07
605 1.00751735896409e-07
606 1.00557602409346e-07
607 1.00615484175837e-07
608 1.0057287616938e-07
609 1.00452901873993e-07
610 1.00580621698043e-07
611 1.00498553729178e-07
612 1.00500693141825e-07
613 1.00334577074257e-07
614 1.00407404794556e-07
615 1.00379269988959e-07
616 1.00306171988862e-07
617 1.00332562989358e-07
618 1.00227322362034e-07
619 1.00116608809486e-07
620 1.00163279189047e-07
621 1.00051969305515e-07
622 1.00135484472297e-07
623 9.99322467980512e-08
624 1.00114912056082e-07
625 9.98429123386657e-08
626 9.99906579610688e-08
627 1.00054608680455e-07
628 9.99336048006505e-08
629 9.97654704182338e-08
630 9.99599826680964e-08
631 9.99049529757379e-08
632 9.97736205401445e-08
633 9.97911180782296e-08
634 9.98803697616069e-08
635 9.96715461862863e-08
636 9.95274905544186e-08
637 9.96842048968105e-08
638 9.97615648499206e-08
639 9.95483681189846e-08
640 9.95148424944858e-08
641 9.9575888556025e-08
642 9.93219666036715e-08
643 9.94130672231641e-08
644 9.93420585873039e-08
645 9.95547692967946e-08
646 9.91731225283843e-08
647 9.93322202984537e-08
648 9.91791554678656e-08
649 9.9304789309862e-08
650 9.90977335257881e-08
651 9.91226195168338e-08
652 9.91641745367033e-08
653 9.89785059419823e-08
654 9.90157732179142e-08
655 9.89207283317306e-08
656 9.89457665525606e-08
657 9.87852590292881e-08
658 9.8844160734135e-08
659 9.87027239141014e-08
660 9.87830791476796e-08
661 9.87847441518142e-08
662 9.85997736639987e-08
663 9.86459616942881e-08
664 9.86462297372093e-08
665 9.85318634629806e-08
666 9.83986557985261e-08
667 9.85125348451632e-08
668 9.84652117481666e-08
669 9.83746351286285e-08
670 9.83152550184663e-08
671 9.83271491836746e-08
672 9.8248927829836e-08
673 9.82212706617247e-08
674 9.81270516562382e-08
675 9.81488775204653e-08
676 9.80663000840209e-08
677 9.8132314666266e-08
678 9.80175909011116e-08
679 9.80088212627628e-08
680 9.79779565541961e-08
681 9.79756184209535e-08
682 9.7826382334798e-08
683 9.78836278617834e-08
684 9.7765154097651e-08
685 9.77094007010848e-08
686 9.75534890499397e-08
687 9.76094807385586e-08
688 9.74676856984047e-08
689 9.76013239579743e-08
690 9.73772505070691e-08
691 9.73875111336397e-08
692 9.72077609646682e-08
693 9.72961031804331e-08
694 9.71740471809035e-08
695 9.72328557442559e-08
696 9.70984587800316e-08
697 9.690481597735e-08
698 9.69669066139822e-08
699 9.69441908482693e-08
700 9.67938457385742e-08
701 9.69285792100649e-08
702 9.65514269934964e-08
703 9.67548305887966e-08
704 9.64859792365047e-08
705 9.64537112673192e-08
706 9.6437159867957e-08
707 9.63900912673843e-08
708 9.6272135049702e-08
709 9.62560565747062e-08
710 9.62852958932281e-08
711 9.64164515262489e-08
712 9.60605812019288e-08
713 9.60866711334773e-08
714 9.60728155541268e-08
715 9.58700806035218e-08
716 9.59815840730904e-08
717 9.57617122168131e-08
718 9.57345231493711e-08
719 9.5775518027974e-08
720 9.56565253154018e-08
721 9.56432961096709e-08
722 9.55138101690345e-08
723 9.55301852423318e-08
724 9.52824753857051e-08
725 9.54044825589584e-08
726 9.51711435766001e-08
727 9.51372525750394e-08
728 9.51276127656442e-08
729 9.50263027994502e-08
730 9.49252096837938e-08
731 9.49314954135083e-08
732 9.49153535931835e-08
733 9.48482629756775e-08
734 9.47867798379853e-08
735 9.46108220456488e-08
736 9.45604131605826e-08
737 9.44490692096345e-08
738 9.43757193638817e-08
739 9.43271172064719e-08
740 9.42066928053542e-08
741 9.41585718066129e-08
742 9.40199542713849e-08
743 9.3981339527538e-08
744 9.38916760264696e-08
745 9.37538080076905e-08
746 9.37185572689181e-08
747 9.36147807304444e-08
748 9.35166526558895e-08
749 9.31593483368687e-08
750 9.35671546775652e-08
751 9.32847503487544e-08
752 9.30042420193011e-08
753 9.29921048307136e-08
754 9.30973443349359e-08
755 9.25917363132633e-08
756 9.28462451272871e-08
757 9.26026026037974e-08
758 9.21787057377443e-08
759 9.22146148312208e-08
760 9.20627424036979e-08
761 9.1961575864552e-08
762 9.17721035800234e-08
763 9.15517966224577e-08
764 9.16264338894912e-08
765 9.13489405856005e-08
766 9.08411612914328e-08
767 9.10179498823105e-08
768 9.06421809494518e-08
769 9.05145692335907e-08
770 8.99126918088022e-08
771 9.00038264326497e-08
772 8.98702010574404e-08
773 8.96328593045226e-08
774 8.93812117350734e-08
775 8.92069329991862e-08
776 8.8804320904412e-08
777 8.83875980917104e-08
778 8.8375448020539e-08
779 8.81019366048008e-08
780 8.77850295779581e-08
781 8.76398108946574e-08
782 8.75005917668759e-08
783 8.70841559357238e-08
784 8.67763794616394e-08
785 8.66034032047658e-08
786 8.62687572769083e-08
787 8.58799902889729e-08
788 8.58188829155893e-08
789 8.52291705850661e-08
790 8.49910403495358e-08
791 8.45416936647858e-08
792 8.41581769588373e-08
793 8.37171891912014e-08
794 8.33197949692277e-08
795 8.2944155126663e-08
796 8.25567884443768e-08
797 8.19809015966122e-08
798 8.12234382190358e-08
799 8.04031614431988e-08
800 7.97350346115611e-08
801 7.89137599306322e-08
802 7.80892658567289e-08
803 7.72916544704572e-08
804 7.63346349073046e-08
805 7.55520105459873e-08
806 7.4689558623664e-08
807 7.4173502822461e-08
808 7.36029421770645e-08
809 7.30266977302385e-08
810 7.2588171754262e-08
811 7.22556269190378e-08
812 7.17934256924124e-08
813 7.15452110267023e-08
814 7.10695392978078e-08
815 7.07184888086854e-08
816 7.05645992435944e-08
817 7.00826036872382e-08
818 6.99255557086165e-08
819 6.95175117022551e-08
820 6.93049411288893e-08
821 6.88251511287152e-08
822 6.87052447623238e-08
823 6.84503688850313e-08
824 6.81543191161094e-08
825 6.7928633614045e-08
826 6.76947226148084e-08
827 6.74902625359408e-08
828 6.70755343468166e-08
829 6.69796152499291e-08
830 6.66251215664104e-08
831 6.64979501334706e-08
832 6.63535027813911e-08
833 6.59369306297286e-08
834 6.5788098968067e-08
835 6.56387975570549e-08
836 6.53906959946493e-08
837 6.5190968737916e-08
838 6.50161696029716e-08
839 6.48399224889573e-08
840 6.47227546330953e-08
841 6.44666515239312e-08
842 6.43110796427759e-08
843 6.40867570806947e-08
844 6.40983984379062e-08
845 6.37689743128167e-08
846 6.37532160179255e-08
847 6.33519622474932e-08
848 6.35671196285159e-08
849 6.3205126254573e-08
850 6.31521667315837e-08
851 6.28550770160885e-08
852 6.27445566823148e-08
853 6.25156053679632e-08
854 6.23160626211927e-08
855 6.22881579244172e-08
856 6.22006890229621e-08
857 6.20154327803935e-08
858 6.18193913517473e-08
859 6.17659573038942e-08
860 6.16613141317668e-08
861 6.14692148539486e-08
862 6.13628421282098e-08
863 6.13818855601522e-08
864 6.14461349104367e-08
865 6.08528031511568e-08
866 6.1166336993157e-08
867 6.08369414325516e-08
868 6.07980841633715e-08
869 6.06549455395289e-08
870 6.07076925036587e-08
871 6.04566816040375e-08
872 6.03741859255269e-08
873 6.03648819597069e-08
874 6.02301488048518e-08
875 6.00835605646388e-08
876 6.01702823557204e-08
877 6.00442270712875e-08
878 6.00029741057995e-08
879 5.97344756161355e-08
880 5.96431504749084e-08
881 5.97513095230706e-08
882 5.94397389810375e-08
883 5.95347092704479e-08
884 5.94242988887217e-08
885 5.94178875332752e-08
886 5.92739183007396e-08
887 5.92366169289704e-08
888 5.9119048743117e-08
889 5.9029322156956e-08
890 5.92399282193234e-08
891 5.90934439199664e-08
892 5.89878091825646e-08
893 5.88081621790693e-08
894 5.8840669791671e-08
895 5.88053282535306e-08
896 5.87757789007881e-08
897 5.85919337532381e-08
898 5.87675394099207e-08
899 5.85187444066548e-08
900 5.84898288766844e-08
901 5.84199637394534e-08
902 5.8372596384082e-08
903 5.82888739226739e-08
904 5.84251662378854e-08
905 5.82020057167121e-08
906 5.82229085255292e-08
907 5.8244366706095e-08
908 5.81941562431965e-08
909 5.80404622003705e-08
910 5.80556073690452e-08
911 5.80162970997833e-08
912 5.80077179459515e-08
913 5.80019632216988e-08
914 5.7941614444168e-08
915 5.79719961417702e-08
916 5.80082798125048e-08
917 5.78986540702431e-08
918 5.77559026040397e-08
919 5.7826159339136e-08
920 5.78580153569597e-08
921 5.7780167315169e-08
922 5.76714666427058e-08
923 5.76270723842143e-08
924 5.76024306990774e-08
925 5.76991754641654e-08
926 5.75840275804929e-08
927 5.75938217517447e-08
928 5.74317866801621e-08
929 5.74847258651978e-08
930 5.75122109407467e-08
931 5.75674925675784e-08
932 5.7390020265391e-08
933 5.7354783452368e-08
934 5.74961810406727e-08
935 5.71245422766875e-08
936 5.72327069403578e-08
937 5.71852442208254e-08
938 5.74255468173135e-08
939 5.71274159342217e-08
940 5.7227907056312e-08
941 5.71213897053724e-08
942 5.70318194967889e-08
943 5.69772816203518e-08
944 5.68237470504052e-08
945 5.72522818580445e-08
946 5.68181944036183e-08
947 5.67899044057185e-08
948 5.69804102950044e-08
949 5.66891130802105e-08
950 5.69222188149254e-08
951 5.71218893643532e-08
952 5.69442166153955e-08
953 5.66943123438968e-08
954 5.70953054679357e-08
955 5.66788111933114e-08
956 5.66425767489775e-08
957 5.66935884669384e-08
958 5.66963546968502e-08
959 5.67263071364899e-08
960 5.65831956875762e-08
961 5.66952338401094e-08
962 5.65940314625202e-08
963 5.65363217583936e-08
964 5.65168565138663e-08
965 5.66126339069939e-08
966 5.65228904072512e-08
967 5.64340508546479e-08
968 5.65711318696316e-08
969 5.64891870162842e-08
970 5.6423487398316e-08
971 5.64622793772784e-08
972 5.62160892725316e-08
973 5.63034568217269e-08
974 5.63921122447297e-08
975 5.63239247035607e-08
976 5.63493510619129e-08
977 5.61269716399249e-08
978 5.62248983797531e-08
979 5.62462478770342e-08
980 5.62215525032883e-08
981 5.60987805222624e-08
982 5.5969840880632e-08
983 5.62168480615632e-08
984 5.62152851193431e-08
985 5.60947495997777e-08
986 5.61095442455084e-08
987 5.61788008770492e-08
988 5.60554453474804e-08
989 5.60390925978815e-08
990 5.61301587422847e-08
991 5.60145730190698e-08
992 5.60894685113222e-08
993 5.58959232870304e-08
994 5.59498448220452e-08
995 5.58756801036786e-08
996 5.59918753966215e-08
997 5.58700783264143e-08
998 5.61189025509723e-08
999 5.5856718734093e-08
1000 5.59469393670575e-08
1001 5.59063025993289e-08
1002 5.59690071693097e-08
1003 5.57895281481091e-08
1004 5.59247126341589e-08
1005 5.57899793356409e-08
1006 5.5718087802914e-08
1007 5.58276732318674e-08
1008 5.56318056601501e-08
1009 5.57015923132731e-08
1010 5.57853723093249e-08
1011 5.58337893437066e-08
1012 5.57789748585336e-08
1013 5.5626412637455e-08
1014 5.57019082902954e-08
1015 5.56563842928348e-08
1016 5.56627271173227e-08
1017 5.5624215594241e-08
1018 5.55576689138526e-08
1019 5.55817890564025e-08
1020 5.5650340260005e-08
1021 5.55732049976054e-08
1022 5.57293209211274e-08
1023 5.54551306102269e-08
1024 5.54888865842429e-08
1025 5.54955580143179e-08
1026 5.54484654926579e-08
1027 5.54977678834945e-08
1028 5.54273840382979e-08
1029 5.54206003329227e-08
1030 5.54382016211274e-08
1031 5.54021608905053e-08
1032 5.54418381908395e-08
1033 5.54575569162719e-08
1034 5.52068352348556e-08
1035 5.55685932837058e-08
1036 5.53054225693739e-08
1037 5.52229790486969e-08
1038 5.52784943075313e-08
1039 5.5336379761739e-08
1040 5.52294718896285e-08
1041 5.52676929512774e-08
1042 5.52080465403559e-08
1043 5.52592687961262e-08
1044 5.52390036805406e-08
1045 5.52746436146556e-08
1046 5.5175930229856e-08
1047 5.51783924396254e-08
1048 5.51750301736043e-08
1049 5.51549455409805e-08
1050 5.51015371659247e-08
1051 5.50977913373885e-08
1052 5.5191426515222e-08
1053 5.50388273266034e-08
1054 5.50797248541901e-08
1055 5.50605794793491e-08
1056 5.49912650555928e-08
1057 5.49759548604811e-08
1058 5.5080808565533e-08
1059 5.48425930615437e-08
1060 5.49559350333517e-08
1061 5.49274091956597e-08
1062 5.50798873177882e-08
1063 5.48410807861277e-08
1064 5.46148567817095e-08
1065 5.49103405218521e-08
1066 5.48610321791099e-08
1067 5.47943861815003e-08
1068 5.48306121734843e-08
1069 5.47222380820056e-08
1070 5.45823962294634e-08
1071 5.49460074457109e-08
1072 5.46046513081322e-08
1073 5.4785600347218e-08
1074 5.46743346301337e-08
1075 5.47590835455747e-08
1076 5.47979758460659e-08
1077 5.49543735934854e-08
1078 5.46774183698417e-08
1079 5.46731719048843e-08
1080 5.47375540596029e-08
1081 5.46613475276292e-08
1082 5.48774356843307e-08
1083 5.46370958516285e-08
1084 5.4658590694201e-08
1085 5.46234988978433e-08
1086 5.4709202164327e-08
1087 5.45970000238594e-08
1088 5.45820081023773e-08
1089 5.46176107871776e-08
1090 5.44879702870649e-08
1091 5.44302397493812e-08
1092 5.46610198619568e-08
1093 5.45011841899257e-08
1094 5.44789229153242e-08
1095 5.46408906665352e-08
1096 5.43814926976527e-08
1097 5.44707699245972e-08
1098 5.44938177908794e-08
1099 5.44501974655276e-08
1100 5.44339546517669e-08
1101 5.43999943991569e-08
1102 5.43909739820769e-08
1103 5.45058098111451e-08
1104 5.42931769675903e-08
1105 5.43534602830409e-08
1106 5.43762416918003e-08
1107 5.42779197560961e-08
1108 5.44107650266579e-08
1109 5.43113730500089e-08
1110 5.43286214316208e-08
1111 5.42196085673297e-08
1112 5.42144799742239e-08
1113 5.42560167888784e-08
1114 5.42748547440741e-08
1115 5.41110505365427e-08
1116 5.42247096764203e-08
1117 5.41681023704665e-08
1118 5.4207848618093e-08
1119 5.43049994379263e-08
1120 5.41289522550947e-08
1121 5.4087908428091e-08
1122 5.40195105049879e-08
1123 5.4141835932775e-08
1124 5.4116607820287e-08
1125 5.3975242005011e-08
1126 5.40592472655366e-08
1127 5.40964356285922e-08
1128 5.40088986085152e-08
1129 5.39679259494719e-08
1130 5.3960600441938e-08
1131 5.39436789985803e-08
1132 5.39632046956395e-08
1133 5.40467165790481e-08
1134 5.38884137486484e-08
1135 5.39740444445158e-08
1136 5.3901255080202e-08
1137 5.3920074478464e-08
1138 5.38374408398656e-08
1139 5.3942087989256e-08
1140 5.38764780035894e-08
1141 5.3824884595377e-08
1142 5.39331422995115e-08
1143 5.37319521760615e-08
1144 5.37567958758256e-08
1145 5.37722629732063e-08
1146 5.38548024635599e-08
1147 5.38110091077293e-08
1148 5.36235945813868e-08
1149 5.37827859430351e-08
1150 5.37967322136179e-08
1151 5.36964044692834e-08
1152 5.3808951333334e-08
1153 5.36471437857511e-08
1154 5.36579723606767e-08
1155 5.35978178954899e-08
1156 5.36915642099789e-08
1157 5.37382605345638e-08
1158 5.36988100909852e-08
1159 5.3547324957659e-08
1160 5.37724043594423e-08
1161 5.35668747856377e-08
1162 5.35596952864203e-08
1163 5.35214321519817e-08
1164 5.36334246430403e-08
1165 5.35928592049206e-08
1166 5.3540765090343e-08
1167 5.34698577352444e-08
1168 5.35537341561643e-08
1169 5.35036357085517e-08
1170 5.35510873147516e-08
1171 5.33283802905959e-08
1172 5.34923048769453e-08
1173 5.34367608460151e-08
1174 5.33288333777193e-08
1175 5.34817078128302e-08
1176 5.33697044198256e-08
1177 5.33707124565908e-08
1178 5.33397013675607e-08
1179 5.34820723816498e-08
1180 5.34089830235285e-08
1181 5.33494322081385e-08
1182 5.33594465947917e-08
1183 5.33378537344031e-08
1184 5.32442493286211e-08
1185 5.33172547934058e-08
1186 5.32908565189505e-08
1187 5.32555614709551e-08
1188 5.32457527062657e-08
1189 5.33375667699509e-08
1190 5.32274566735769e-08
1191 5.32818182255035e-08
1192 5.32155870625051e-08
1193 5.31651073178985e-08
1194 5.32965277475483e-08
1195 5.32395655561402e-08
1196 5.30587415099593e-08
1197 5.32692713066663e-08
1198 5.29914874882742e-08
1199 5.31366961675772e-08
1200 5.31781692199029e-08
1201 5.30765557629209e-08
1202 5.31081655439358e-08
1203 5.31358261897097e-08
1204 5.30198970836171e-08
1205 5.30244554097781e-08
1206 5.29939024465076e-08
1207 5.30046809923412e-08
1208 5.30137791290297e-08
1209 5.30005657299881e-08
1210 5.29717028698862e-08
1211 5.2961980773647e-08
1212 5.30472652979785e-08
1213 5.28871871405823e-08
1214 5.29021066619784e-08
1215 5.29730447258459e-08
1216 5.29241749915865e-08
1217 5.28950319893529e-08
1218 5.2865996770457e-08
1219 5.28306666494061e-08
1220 5.29622695484289e-08
1221 5.27944078738685e-08
1222 5.28865354136876e-08
1223 5.27530616620808e-08
1224 5.28044289818119e-08
1225 5.28425667321031e-08
1226 5.27429724372919e-08
1227 5.26620050504967e-08
1228 5.2858322357352e-08
1229 5.26948867327626e-08
1230 5.26135566703001e-08
1231 5.27734352533393e-08
1232 5.26695825648904e-08
1233 5.2526649305662e-08
1234 5.27208395724177e-08
1235 5.2674932346175e-08
1236 5.2453460438695e-08
1237 5.26115899568413e-08
1238 5.2672406459342e-08
1239 5.25541921923001e-08
1240 5.25819306105291e-08
1241 5.26850987636607e-08
1242 5.24033023638904e-08
1243 5.25644333817521e-08
1244 5.26522350239311e-08
1245 5.25092595484722e-08
1246 5.24427837227392e-08
1247 5.24800539409842e-08
1248 5.26158036455726e-08
1249 5.24414549400998e-08
1250 5.23907603133811e-08
1251 5.25041915975244e-08
1252 5.23444679729757e-08
1253 5.24364656431242e-08
1254 5.2541199248024e-08
1255 5.23143371931045e-08
1256 5.24666602028656e-08
1257 5.23239909502315e-08
1258 5.25475487800353e-08
1259 5.23368533922319e-08
1260 5.23667704641628e-08
1261 5.23606596847248e-08
1262 5.23689044371611e-08
1263 5.22773162325851e-08
1264 5.23672579115786e-08
1265 5.23081480401366e-08
1266 5.2317039239913e-08
1267 5.23360871036438e-08
1268 5.21969125115707e-08
1269 5.22085211287493e-08
1270 5.21577246150873e-08
1271 5.21616402962e-08
1272 5.21223013056904e-08
1273 5.20662667748972e-08
1274 5.21698321849229e-08
1275 5.19739483624271e-08
1276 5.17945131317532e-08
1277 5.22218586567202e-08
1278 5.20629885758517e-08
1279 5.19565730194849e-08
1280 5.20464899889106e-08
1281 5.17541620692619e-08
1282 5.22902196191666e-08
1283 5.1781549895491e-08
1284 5.20579992331349e-08
1285 5.1853499591914e-08
1286 5.18762688050956e-08
1287 5.17707142719814e-08
1288 5.18845504169363e-08
1289 5.20126740941595e-08
1290 5.18173147947554e-08
1291 5.18392150024027e-08
1292 5.18289305659891e-08
1293 5.19438259383165e-08
1294 5.18562159737002e-08
1295 5.18552214994195e-08
1296 5.18063132952662e-08
1297 5.15365433093606e-08
1298 5.14809287510687e-08
1299 5.18516962380033e-08
1300 5.18191315364014e-08
1301 5.16740708449426e-08
1302 5.14328890681703e-08
1303 5.16714466476387e-08
1304 5.17727641211607e-08
1305 5.13522889709073e-08
1306 5.16006372142108e-08
1307 5.16042869618261e-08
1308 5.17058695104211e-08
1309 5.16771628311652e-08
1310 5.15676696006828e-08
1311 5.15088894079696e-08
1312 5.16572576976238e-08
1313 5.12908941558976e-08
1314 5.14953611032709e-08
1315 5.16518338011807e-08
1316 5.1235144595374e-08
1317 5.17779559241482e-08
1318 5.13804241553117e-08
1319 5.13995729853889e-08
1320 5.14777740248107e-08
1321 5.13129894272879e-08
1322 5.14693538682387e-08
1323 5.13786983682429e-08
1324 5.13691347525658e-08
1325 5.15406455230405e-08
1326 5.14866103578093e-08
1327 5.10135570990222e-08
1328 5.11011405626149e-08
1329 5.1321497262613e-08
1330 5.1327574210891e-08
1331 5.09624432454281e-08
1332 5.11112084815579e-08
1333 5.12394221328094e-08
1334 5.11787409780418e-08
1335 5.13059400373272e-08
1336 5.10586234558907e-08
1337 5.10651457559241e-08
1338 5.12160758519009e-08
1339 5.11679195405179e-08
1340 5.10184425523974e-08
1341 5.09810129578625e-08
1342 5.11689901006029e-08
1343 5.08187902590329e-08
1344 5.10224844521012e-08
1345 5.09003158182253e-08
1346 5.08712496629649e-08
1347 5.09150018475069e-08
1348 5.09758039979324e-08
1349 5.0847425813183e-08
1350 5.0984536264842e-08
1351 5.07108279381718e-08
1352 5.08419749429834e-08
1353 5.080359431342e-08
1354 5.09154263239608e-08
1355 5.06301215135174e-08
1356 5.05360578788583e-08
1357 5.04963529590619e-08
1358 5.08045670122304e-08
1359 5.08820833173829e-08
1360 5.04010123840715e-08
1361 5.05803111550129e-08
1362 5.05387778388933e-08
1363 5.05110824371435e-08
1364 5.06542682456867e-08
1365 5.02667206365715e-08
1366 5.06715337995622e-08
1367 5.03926760280216e-08
1368 5.04030152739254e-08
1369 5.05058511803824e-08
1370 5.01763518323539e-08
1371 5.03571353371868e-08
1372 5.03813890793126e-08
1373 5.00564019567662e-08
1374 5.0287894396428e-08
1375 5.01283703018274e-08
1376 5.03292024944812e-08
1377 5.01483646733192e-08
1378 5.01793439409148e-08
1379 5.01754333908089e-08
1380 5.01272759834137e-08
1381 5.00331121910946e-08
1382 5.01435823707652e-08
1383 5.00926870252716e-08
1384 4.99893013363728e-08
1385 5.00342840252799e-08
1386 4.99456582336855e-08
1387 4.99196334584617e-08
1388 4.99523944612612e-08
1389 4.99766429689075e-08
1390 4.98379437892105e-08
1391 4.99041905961395e-08
1392 4.98132484529901e-08
1393 4.98786969649334e-08
1394 4.95050627165039e-08
1395 4.96953361648345e-08
1396 4.97669438623838e-08
1397 4.97335562967827e-08
1398 4.9601187225301e-08
1399 4.97466120097378e-08
1400 4.95720703701519e-08
1401 4.96844062121138e-08
1402 4.95708199359512e-08
1403 4.95675797442807e-08
1404 4.93948280149148e-08
1405 4.94111011637433e-08
1406 4.9404332855163e-08
1407 4.92001040997003e-08
1408 4.9513742074847e-08
1409 4.92098624842452e-08
1410 4.92840830734753e-08
1411 4.91248994554105e-08
1412 4.91976859566812e-08
1413 4.89940597865335e-08
1414 4.93246275172776e-08
1415 4.89294613572611e-08
1416 4.91998910381319e-08
1417 4.88956396991203e-08
1418 4.90265857011618e-08
1419 4.88550583912506e-08
1420 4.89569656547673e-08
1421 4.89096353935015e-08
1422 4.88001555156714e-08
1423 4.86349586026336e-08
1424 4.87148874515064e-08
1425 4.86660959198026e-08
1426 4.85662074918292e-08
1427 4.85084134433933e-08
1428 4.86927304510498e-08
1429 4.83121153214139e-08
1430 4.84874828639015e-08
1431 4.83686944301454e-08
1432 4.83148538592548e-08
1433 4.82903695711023e-08
1434 4.82129831578249e-08
1435 4.81062845716362e-08
1436 4.82523671121449e-08
1437 4.82276616071697e-08
1438 4.80148075505671e-08
1439 4.7994673324947e-08
1440 4.8054595421565e-08
1441 4.78974293869339e-08
1442 4.79631219354459e-08
1443 4.78352216657374e-08
1444 4.77952215767186e-08
1445 4.77718391929116e-08
1446 4.77694918854521e-08
1447 4.77102562939091e-08
1448 4.75346567043733e-08
1449 4.76812829091067e-08
1450 4.75071119068193e-08
1451 4.74314178056101e-08
1452 4.75894045197656e-08
1453 4.72621173837062e-08
1454 4.73463947174846e-08
1455 4.72873592010092e-08
1456 4.71856293313699e-08
1457 4.70082222598922e-08
1458 4.70095031783746e-08
1459 4.68910669391143e-08
1460 4.68804158049174e-08
1461 4.70214880379238e-08
1462 4.68831279651916e-08
1463 4.68463315486112e-08
1464 4.66050335685697e-08
1465 4.67543339026655e-08
1466 4.64847452512629e-08
1467 4.66957640530286e-08
1468 4.63975537881023e-08
1469 4.6425570566111e-08
1470 4.64406185587407e-08
1471 4.62731193113797e-08
1472 4.64232869539227e-08
1473 4.62204918478815e-08
1474 4.61393014012934e-08
1475 4.60188834521258e-08
1476 4.60776093804682e-08
1477 4.60226239218375e-08
1478 4.61251517789307e-08
1479 4.59024918930684e-08
1480 4.56791527005951e-08
1481 4.58176279174438e-08
1482 4.57196149679095e-08
1483 4.54913652196964e-08
1484 4.55927940916645e-08
1485 4.54639755746999e-08
1486 4.54809509400356e-08
1487 4.5190838259046e-08
1488 4.5194124380199e-08
1489 4.51272091788191e-08
1490 4.48751546620052e-08
1491 4.49820738588347e-08
1492 4.46000519815915e-08
1493 4.48316162167117e-08
1494 4.46692632831702e-08
1495 4.4570401988242e-08
1496 4.45403572773007e-08
1497 4.44337638654257e-08
1498 4.44346210737301e-08
1499 4.44036081510557e-08
1500 4.43378510732728e-08
1501 4.41500417689422e-08
1502 4.41065746721314e-08
1503 4.37687224554839e-08
1504 4.39243929672983e-08
1505 4.41237352026835e-08
1506 4.35818569124446e-08
1507 4.35032563361748e-08
1508 4.37459083815206e-08
1509 4.32860018801939e-08
1510 4.31918199279568e-08
1511 4.33927774992071e-08
1512 4.33139185354037e-08
1513 4.33856957953171e-08
1514 4.31346792080234e-08
1515 4.29929267651286e-08
1516 4.29489446598375e-08
1517 4.29643557635639e-08
1518 4.2646787887346e-08
1519 4.27075978675084e-08
1520 4.28223910795555e-08
1521 4.24433759524945e-08
1522 4.24753292000979e-08
1523 4.24917369949984e-08
1524 4.23237456601555e-08
1525 4.22225211020866e-08
1526 4.2153564890679e-08
1527 4.22977280036818e-08
1528 4.17977601183672e-08
1529 4.1929576873212e-08
1530 4.18273011417725e-08
1531 4.1848594548366e-08
1532 4.16575200776759e-08
1533 4.17089356310463e-08
1534 4.15995547491388e-08
1535 4.1841262842901e-08
1536 4.13187739007892e-08
1537 4.13578307989493e-08
1538 4.1275051125389e-08
1539 4.12440914698831e-08
1540 4.10540450204167e-08
1541 4.09785586126255e-08
1542 4.11806393392222e-08
1543 4.07550056209516e-08
1544 4.07583067516537e-08
1545 4.05738792499388e-08
1546 4.04985871400587e-08
1547 4.069136332574e-08
1548 4.04438336396584e-08
1549 4.05198397381845e-08
1550 4.02059164070767e-08
1551 4.05203975479829e-08
1552 4.01250251655938e-08
1553 4.01975548596223e-08
1554 4.00447287280947e-08
1555 3.99434507425411e-08
1556 3.98278435593813e-08
1557 3.98243838344303e-08
1558 3.96970725127233e-08
1559 3.95746356123272e-08
1560 3.95441347247338e-08
1561 3.95049254227064e-08
1562 3.92418934673788e-08
1563 3.92097029562333e-08
1564 3.91416694336399e-08
1565 3.91137595598945e-08
1566 3.90561460834604e-08
1567 3.92768473660432e-08
1568 3.89146719457223e-08
1569 3.88712150110049e-08
1570 3.89829728115743e-08
1571 3.86714591733561e-08
1572 3.85483669804731e-08
1573 3.84953191230331e-08
1574 3.84972808202821e-08
1575 3.85276303425108e-08
1576 3.84002197728606e-08
1577 3.84873486136694e-08
1578 3.82010382047326e-08
1579 3.80970169742145e-08
1580 3.80678417732927e-08
1581 3.80586060193533e-08
1582 3.82549467337068e-08
1583 3.80699879749535e-08
1584 3.8100013092901e-08
1585 3.79841144777515e-08
1586 3.80213797110951e-08
1587 3.77313382338151e-08
1588 3.78084818866142e-08
1589 3.76512921629324e-08
1590 3.77529376638375e-08
1591 3.77942267282272e-08
1592 3.77663475941148e-08
1593 3.76212202042314e-08
1594 3.75406787143095e-08
1595 3.74741595972061e-08
1596 3.74752594916039e-08
1597 3.74817614041678e-08
1598 3.70991694191858e-08
1599 3.73738989556394e-08
1600 3.6876595526536e-08
1601 3.74955205215688e-08
1602 3.71857607561754e-08
1603 3.71678241721796e-08
1604 3.69448407600004e-08
1605 3.70446703725769e-08
1606 3.68542212638712e-08
1607 3.67091444979373e-08
1608 3.70468074615271e-08
1609 3.6872965612389e-08
1610 3.68773773038455e-08
1611 3.67294960665898e-08
1612 3.67718985458954e-08
1613 3.68477635073194e-08
1614 3.67127960270164e-08
1615 3.65457940303404e-08
1616 3.66974750949378e-08
1617 3.65521988958672e-08
1618 3.64635065019936e-08
1619 3.64606751714902e-08
1620 3.62962272015821e-08
1621 3.63988045515296e-08
1622 3.63712167215091e-08
1623 3.62943253786607e-08
1624 3.62187403737391e-08
1625 3.61851363526178e-08
1626 3.62201414720875e-08
1627 3.61112072737591e-08
1628 3.6183123990563e-08
1629 3.60104018128915e-08
1630 3.5862459132785e-08
1631 3.60455589010211e-08
1632 3.59387287736102e-08
1633 3.59771402695941e-08
1634 3.60870228617305e-08
1635 3.57826095152003e-08
1636 3.57856576749249e-08
1637 3.59786630006464e-08
1638 3.59200286383921e-08
1639 3.5717789315548e-08
1640 3.57561143506935e-08
1641 3.59713496804748e-08
1642 3.57785515880948e-08
1643 3.57233208549967e-08
1644 3.56841646580452e-08
1645 3.56834440680132e-08
1646 3.57238967814144e-08
1647 3.55699151297628e-08
1648 3.55313755417974e-08
1649 3.57595925255527e-08
1650 3.54862265075084e-08
1651 3.55607543389702e-08
1652 3.54712654395684e-08
1653 3.54674702411906e-08
1654 3.55352034315803e-08
1655 3.54332533401003e-08
1656 3.56103380501072e-08
1657 3.56338591127603e-08
1658 3.5160385403632e-08
1659 3.52363546813894e-08
1660 3.54794067205511e-08
1661 3.5431849241041e-08
1662 3.51940813854146e-08
1663 3.52079912460113e-08
1664 3.51275357637348e-08
1665 3.51160690568175e-08
1666 3.52377171075613e-08
1667 3.49464114248743e-08
1668 3.53010240496943e-08
1669 3.51782423391178e-08
1670 3.52463426152028e-08
1671 3.51127888391645e-08
1672 3.50597533937957e-08
1673 3.49992945287703e-08
1674 3.50740572279129e-08
1675 3.48677280201937e-08
1676 3.49707819427003e-08
1677 3.53194266478063e-08
1678 3.48341384390771e-08
1679 3.48170465764852e-08
1680 3.51957675079895e-08
1681 3.48384399140311e-08
1682 3.52707471309444e-08
1683 3.46512172515467e-08
1684 3.50979739038859e-08
1685 3.49174730329693e-08
1686 3.48277309232259e-08
1687 3.44769947528967e-08
1688 3.51023022784336e-08
1689 3.48558936946741e-08
1690 3.51318046916127e-08
1691 3.46746027055644e-08
1692 3.50836512013064e-08
1693 3.4742588495007e-08
1694 3.49838132549518e-08
1695 3.49015753140502e-08
1696 3.44707926749344e-08
1697 3.43440156134189e-08
1698 3.48747551393025e-08
1699 3.49516863571431e-08
1700 3.45256469509447e-08
1701 3.44649668995167e-08
1702 3.47640615898648e-08
1703 3.45251422735338e-08
1704 3.49669878725312e-08
1705 3.44525049349453e-08
1706 3.4777124157781e-08
1707 3.46900329932787e-08
1708 3.43309902195443e-08
1709 3.44558142646445e-08
1710 3.4736249139522e-08
1711 3.45738210041358e-08
1712 3.443391216984e-08
1713 3.45664027954218e-08
1714 3.44711528490471e-08
1715 3.43193184850765e-08
1716 3.43777140874302e-08
1717 3.41512736377059e-08
1718 3.41768649725349e-08
1719 3.41214815657853e-08
1720 3.46329146665436e-08
1721 3.40930031248465e-08
1722 3.4230946622138e-08
1723 3.42536389879022e-08
1724 3.39206672130832e-08
1725 3.41256568756609e-08
1726 3.41303722466435e-08
1727 3.3901115441326e-08
1728 3.379644736623e-08
1729 3.35945342959665e-08
1730 3.36617494729463e-08
1731 3.3941827971784e-08
1732 3.3800585084176e-08
1733 3.36154068643069e-08
1734 3.40904461655356e-08
1735 3.36788436501312e-08
1736 3.35994076827717e-08
1737 3.35437944751771e-08
1738 3.34718312993143e-08
1739 3.40162828786905e-08
1740 3.34722844268498e-08
1741 3.3574532573688e-08
1742 3.34812244005001e-08
1743 3.37085974984053e-08
1744 3.33761462016913e-08
1745 3.33755203789643e-08
1746 3.34580818248664e-08
1747 3.35512264766358e-08
1748 3.40579853683742e-08
1749 3.34660004233101e-08
1750 3.34090687093003e-08
1751 3.33485146062351e-08
1752 3.36934245714904e-08
1753 3.36378064798026e-08
1754 3.33886416590268e-08
1755 3.36110341161433e-08
1756 3.36076728140178e-08
1757 3.35735682739369e-08
1758 3.35215969895497e-08
1759 3.34261122874047e-08
1760 3.32134393257188e-08
1761 3.34399441626054e-08
1762 3.33088759922884e-08
1763 3.34600880966818e-08
1764 3.31812814002852e-08
1765 3.33242031682612e-08
1766 3.337506122425e-08
1767 3.35078712723202e-08
1768 3.32889573371187e-08
1769 3.32000849669889e-08
1770 3.32059096481707e-08
1771 3.33756336701185e-08
1772 3.3275314368808e-08
1773 3.31812113869567e-08
1774 3.32671812455221e-08
1775 3.34418166842987e-08
1776 3.30332344378181e-08
1777 3.30014686997782e-08
1778 3.31134501081465e-08
1779 3.3086883968636e-08
1780 3.31744442734205e-08
1781 3.30846049316769e-08
1782 3.31385209746404e-08
1783 3.32810588064003e-08
1784 3.30944860691407e-08
1785 3.30445074789854e-08
1786 3.30924436209301e-08
1787 3.3257922360308e-08
1788 3.292427301127e-08
1789 3.32142907548683e-08
1790 3.32764591450818e-08
1791 3.31435071641017e-08
1792 3.30144345668248e-08
1793 3.28494998140805e-08
1794 3.33048215663823e-08
1795 3.28922237948781e-08
1796 3.29881614160588e-08
1797 3.29481509400154e-08
1798 3.32278539791897e-08
1799 3.33459040751105e-08
1800 3.28180925959209e-08
1801 3.286345170328e-08
1802 3.31302118639076e-08
1803 3.30545583304698e-08
1804 3.29896534603247e-08
1805 3.30136250890067e-08
1806 3.29708649191573e-08
1807 3.29375728849168e-08
1808 3.29584732507993e-08
1809 3.30759406683701e-08
1810 3.26727723212183e-08
1811 3.29126615155051e-08
1812 3.27097393977294e-08
1813 3.31365848653675e-08
1814 3.27542318201424e-08
1815 3.2688300757755e-08
1816 3.2757951207163e-08
1817 3.25938695682204e-08
1818 3.25255430486671e-08
1819 3.26416299745169e-08
1820 3.2567819272078e-08
1821 3.28019866888685e-08
1822 3.264519897761e-08
1823 3.24765142420969e-08
1824 3.23567726092655e-08
1825 3.26195809782215e-08
1826 3.2368462442478e-08
1827 3.26535393454508e-08
1828 3.22922166864359e-08
1829 3.28254064148048e-08
1830 3.25970093091321e-08
1831 3.28429771216943e-08
1832 3.22367063556595e-08
1833 3.25654543835707e-08
1834 3.24002746698859e-08
1835 3.2145519800908e-08
1836 3.24086740900453e-08
1837 3.22401046555232e-08
1838 3.2375356793013e-08
1839 3.21329867825071e-08
1840 3.24970150727966e-08
1841 3.22697721955034e-08
1842 3.24375935307764e-08
1843 3.23349596473843e-08
1844 3.22530021765743e-08
1845 3.23972422455743e-08
1846 3.2403083657595e-08
1847 3.2391616504035e-08
1848 3.22603515532371e-08
1849 3.23840268994235e-08
1850 3.23131341983807e-08
1851 3.24335115935437e-08
1852 3.2195868266971e-08
1853 3.20678411864783e-08
1854 3.22093465685569e-08
1855 3.23499340313393e-08
1856 3.22403795824933e-08
1857 3.22982276641515e-08
1858 3.20484389864095e-08
1859 3.23778803299479e-08
1860 3.21407798531403e-08
1861 3.21378105178072e-08
1862 3.2269863146972e-08
1863 3.20877019310828e-08
1864 3.21393591102836e-08
1865 3.20610241240082e-08
1866 3.22750349319723e-08
1867 3.20759868555065e-08
1868 3.19455806309099e-08
1869 3.18752563761571e-08
1870 3.20365043058324e-08
1871 3.1987450458848e-08
1872 3.190478142856e-08
1873 3.20084311614677e-08
1874 3.20397758721835e-08
1875 3.21618689440584e-08
1876 3.21484486736079e-08
1877 3.21601086706869e-08
1878 3.21426507832179e-08
1879 3.20736332380944e-08
1880 3.19680327097727e-08
1881 3.17354398056135e-08
1882 3.22545682087672e-08
1883 3.20111203475637e-08
1884 3.17579959610015e-08
1885 3.2051055420812e-08
1886 3.18628771671658e-08
1887 3.19253713141343e-08
1888 3.16806506397427e-08
1889 3.17817672524923e-08
1890 3.18705913298078e-08
1891 3.16520344243454e-08
1892 3.19755399245114e-08
1893 3.17410123054085e-08
1894 3.17748752415303e-08
1895 3.15764807627961e-08
1896 3.17395728870462e-08
1897 3.17761909442993e-08
1898 3.16948498895187e-08
1899 3.16914683828884e-08
1900 3.1725972318819e-08
1901 3.17040115762612e-08
1902 3.15797447274502e-08
1903 3.17914948744225e-08
1904 3.1556721885595e-08
1905 3.15991149022477e-08
1906 3.14761764115268e-08
1907 3.1890572127713e-08
1908 3.17374879850174e-08
1909 3.17017944698739e-08
1910 3.16788406136137e-08
1911 3.18696204233415e-08
1912 3.16862355682623e-08
1913 3.16801947449763e-08
1914 3.19155541520511e-08
1915 3.17196889152438e-08
1916 3.15880888166475e-08
1917 3.16818348933268e-08
1918 3.17066987188808e-08
1919 3.18166739046166e-08
1920 3.16737962622327e-08
1921 3.16276628081757e-08
1922 3.16401680604539e-08
1923 3.15894348807966e-08
1924 3.15784317074241e-08
1925 3.17113685384118e-08
1926 3.15141062032787e-08
1927 3.1658210717711e-08
1928 3.16009527309147e-08
1929 3.16011468696198e-08
1930 3.15881001584639e-08
1931 3.17910540561517e-08
1932 3.1896889689742e-08
1933 3.1758501656709e-08
1934 3.16772888877548e-08
1935 3.1337921064245e-08
1936 3.16707616346834e-08
1937 3.17411212900076e-08
1938 3.15537064423665e-08
1939 3.17707709058812e-08
1940 3.14743735070344e-08
1941 3.15281545950619e-08
1942 3.14337457821834e-08
1943 3.14495387474523e-08
1944 3.15912997389534e-08
1945 3.13756825685019e-08
1946 3.15630503476827e-08
1947 3.14893775001934e-08
1948 3.15733848208488e-08
1949 3.15867462676866e-08
1950 3.16332006670139e-08
1951 3.12915624240961e-08
1952 3.17217354433019e-08
1953 3.13682624678568e-08
1954 3.13793869699541e-08
1955 3.12607052568303e-08
1956 3.15945102754567e-08
1957 3.13981664425977e-08
1958 3.14886380484713e-08
1959 3.15370424787131e-08
1960 3.1333441017134e-08
1961 3.12896869942847e-08
1962 3.13015277328965e-08
1963 3.12341310557684e-08
1964 3.15357306969188e-08
1965 3.13460720317016e-08
1966 3.1344920814691e-08
1967 3.14290160757791e-08
1968 3.11443971334757e-08
1969 3.14083876886695e-08
1970 3.12433503726339e-08
1971 3.13177147430732e-08
1972 3.1389176233465e-08
1973 3.14480519518767e-08
1974 3.11831187130851e-08
1975 3.14193256538697e-08
1976 3.12550333423367e-08
1977 3.12127615282876e-08
1978 3.14050642433728e-08
1979 3.12700158652257e-08
1980 3.13140126722011e-08
1981 3.12501760129091e-08
1982 3.1413858266327e-08
1983 3.1176238075914e-08
1984 3.13241454108493e-08
1985 3.12557544821512e-08
1986 3.12570098319753e-08
1987 3.126779659679e-08
1988 3.13864372989414e-08
1989 3.0963313260024e-08
1990 3.13036487800122e-08
1991 3.12473687580361e-08
1992 3.14401644535556e-08
1993 3.09719248758888e-08
1994 3.17309949269262e-08
1995 3.12084479188179e-08
1996 3.13300513519188e-08
1997 3.13830005300719e-08
1998 3.12320737769767e-08
1999 3.10204771938327e-08
2000 3.13866487916537e-08
2001 3.11669106134449e-08
2002 3.10365715670002e-08
2003 3.12886546303126e-08
2004 3.11756460862345e-08
2005 3.12019688007981e-08
2006 3.11479906132117e-08
2007 3.11215200088366e-08
2008 3.12833431963533e-08
2009 3.11302408264602e-08
2010 3.12215302925578e-08
2011 3.1335691489609e-08
2012 3.13096151460801e-08
2013 3.1495933910386e-08
2014 3.1299168284038e-08
2015 3.10789179432946e-08
2016 3.13420242298301e-08
2017 3.11986472787407e-08
2018 3.11741675922406e-08
2019 3.13263519152729e-08
2020 3.1041790113373e-08
2021 3.08692434097813e-08
2022 3.10781057911713e-08
2023 3.1075846172679e-08
2024 3.0877343000113e-08
2025 3.11024461417952e-08
2026 3.08338607130088e-08
2027 3.10596256014595e-08
2028 3.13491740233651e-08
2029 3.08366633634538e-08
2030 3.07805692917329e-08
2031 3.08477971555643e-08
2032 3.10142757631304e-08
2033 3.08051460784498e-08
2034 3.08347869315551e-08
2035 3.10144103566889e-08
2036 3.08977156658896e-08
2037 3.08665392416696e-08
2038 3.12070932643183e-08
2039 3.09461479837125e-08
2040 3.06734264030606e-08
2041 3.14647470206486e-08
2042 3.07961665537704e-08
2043 3.1174583214999e-08
2044 3.10792188229492e-08
2045 3.10642638263747e-08
2046 3.10332487276632e-08
2047 3.09795625847986e-08
2048 3.07294781166645e-08
2049 3.12073470873919e-08
2050 3.0716066179326e-08
2051 3.11156088357301e-08
2052 3.06718905294101e-08
2053 3.0933877358974e-08
2054 3.08780717426238e-08
2055 3.07069155336404e-08
2056 3.13913090613793e-08
2057 3.05582729480669e-08
2058 3.10691138539765e-08
2059 3.063635101197e-08
2060 3.11148900088432e-08
2061 3.0841071518406e-08
2062 3.08832320111696e-08
2063 3.06090417524496e-08
2064 3.10595884239762e-08
2065 3.07513889817868e-08
2066 3.08360748574321e-08
2067 3.07587906387408e-08
2068 3.0854508721867e-08
2069 3.10573415496229e-08
2070 3.0544984608194e-08
2071 3.09554084808594e-08
2072 3.06719842376735e-08
2073 3.09046185215101e-08
2074 3.10153290534743e-08
2075 3.06345342712122e-08
2076 3.10759447652575e-08
2077 3.05411092734742e-08
2078 3.05765281382042e-08
2079 3.08768581923635e-08
2080 3.09159058674568e-08
2081 3.08055070755708e-08
2082 3.06505936858992e-08
2083 3.10528235386576e-08
2084 3.06993720521342e-08
2085 3.0777143303018e-08
2086 3.07939553330083e-08
2087 3.09366288125945e-08
2088 3.07179707070837e-08
2089 3.08734834801871e-08
2090 3.06300392239889e-08
2091 3.04250169985698e-08
2092 3.10692510995247e-08
2093 3.06002114990322e-08
2094 3.07929585623423e-08
2095 3.07848064136085e-08
2096 3.08471376930797e-08
2097 3.06030844894334e-08
2098 3.07605082993101e-08
2099 3.0460339140892e-08
2100 3.08857137113483e-08
2101 3.03815031437171e-08
2102 3.08145387576397e-08
2103 3.04511622176706e-08
2104 3.09113535776895e-08
2105 3.047389419919e-08
2106 3.10055711930657e-08
2107 3.04439085808683e-08
2108 3.05293772570048e-08
2109 3.04229522911648e-08
2110 3.0893542182997e-08
2111 3.03828479845114e-08
2112 3.0585250758941e-08
2113 3.06665533955641e-08
2114 3.06575269627407e-08
2115 3.06126904067172e-08
2116 3.0564735490235e-08
2117 3.07105342438607e-08
2118 3.07347670883695e-08
2119 3.04122914909222e-08
2120 3.07145014463162e-08
2121 3.09969188769799e-08
2122 3.0285731024815e-08
2123 3.07204860564569e-08
2124 3.04740476899656e-08
2125 3.075518173834e-08
2126 3.02707890651455e-08
2127 3.11874672241919e-08
2128 3.06102781615358e-08
2129 3.05344551656539e-08
2130 3.03559467387249e-08
2131 3.06085562793434e-08
2132 3.03037813668139e-08
2133 3.07540372864734e-08
2134 3.04376691587782e-08
2135 3.07392891625291e-08
2136 3.04157028567786e-08
2137 3.06495742631308e-08
2138 3.04195236033866e-08
2139 3.06398631632065e-08
2140 3.0129897061526e-08
2141 3.05744950500619e-08
2142 3.03027893291397e-08
2143 3.06003800286669e-08
2144 3.06173228630247e-08
2145 3.05327516726406e-08
2146 3.02142557797591e-08
2147 3.05308549564121e-08
2148 3.0685624099247e-08
2149 3.01502466866221e-08
2150 3.04980425384072e-08
2151 3.03957446385894e-08
2152 3.02532885986251e-08
2153 3.11220568098847e-08
2154 3.00107630419877e-08
2155 3.0781626386811e-08
2156 3.0322722075149e-08
2157 3.02381349867042e-08
2158 3.04582505173956e-08
2159 3.01540122387145e-08
2160 3.06704995055584e-08
2161 3.02779206247106e-08
2162 3.0249870121013e-08
2163 3.02147419952714e-08
2164 3.08217382215536e-08
2165 3.02612559565096e-08
2166 3.04049990850208e-08
2167 3.02366305167112e-08
2168 3.04013264075254e-08
2169 3.05284235307957e-08
2170 3.03710477588259e-08
2171 3.03698405117458e-08
2172 3.06631708874017e-08
2173 3.04518916084406e-08
2174 3.0250717969027e-08
2175 3.05538271090366e-08
2176 3.05889016347649e-08
2177 3.0220209546572e-08
2178 3.08721310173787e-08
2179 3.01722992932518e-08
2180 3.05440442651683e-08
2181 3.02868807747769e-08
2182 3.04897354629352e-08
2183 3.02111079679701e-08
2184 3.09814145894594e-08
2185 2.98233715612106e-08
2186 3.04169479101724e-08
2187 3.05657341910237e-08
2188 3.0358909214101e-08
2189 3.01604787942278e-08
2190 3.03752394622547e-08
2191 3.03802513212936e-08
2192 3.08188362133643e-08
2193 2.98784167223065e-08
2194 3.04261456209876e-08
2195 3.04676496395428e-08
2196 3.02828552514178e-08
2197 3.02509306707721e-08
2198 3.05268325484498e-08
2199 3.00125618833968e-08
2200 3.10070152275133e-08
2201 2.97640617223927e-08
2202 3.04707150990957e-08
2203 2.98923823267572e-08
2204 3.04513381393923e-08
2205 3.0172615393731e-08
2206 3.03171897508836e-08
2207 3.01721402297117e-08
2208 3.04941942272929e-08
2209 2.9903953325916e-08
2210 3.03429373138542e-08
2211 3.03634185387924e-08
2212 2.97425601515355e-08
2213 2.99556811672597e-08
2214 3.0275326028395e-08
2215 2.99315567123815e-08
2216 2.97544889735768e-08
2217 3.02826758750152e-08
2218 3.00994481486905e-08
2219 2.97988267602411e-08
2220 3.03690160802228e-08
2221 3.02326522103602e-08
2222 2.9876018748487e-08
2223 3.00050987203093e-08
2224 2.99282133032763e-08
2225 3.01561400195549e-08
2226 3.01406641953772e-08
2227 2.98678376080552e-08
2228 3.06666243701237e-08
2229 3.03155072108874e-08
2230 2.95334916102785e-08
2231 3.01139974956399e-08
2232 3.00505288801656e-08
2233 2.97213111112304e-08
2234 3.00531011132588e-08
2235 3.02848937391298e-08
2236 3.08922193918981e-08
2237 2.94116795903632e-08
2238 2.98273489180989e-08
2239 2.98187282353757e-08
2240 3.00760743929818e-08
2241 2.9747982626116e-08
2242 2.99089785744666e-08
2243 3.04171914278228e-08
2244 2.9569544663377e-08
2245 3.04042390526504e-08
2246 3.03768514127167e-08
2247 2.94560204543437e-08
2248 3.03424599876756e-08
2249 2.96948970888167e-08
2250 2.97333395278265e-08
2251 2.99897985072128e-08
2252 2.99557007614748e-08
2253 2.99110626271748e-08
2254 2.9834661319228e-08
2255 3.02227323196735e-08
2256 2.95475689230384e-08
2257 3.02040558897154e-08
2258 3.02228408315397e-08
2259 2.95271886876503e-08
2260 3.02729675618618e-08
2261 2.9890361300744e-08
2262 2.95705936407264e-08
2263 3.06097191902222e-08
2264 2.91894327352216e-08
2265 2.99505679433043e-08
2266 3.05699087155276e-08
2267 2.93050693443986e-08
2268 3.09662185290494e-08
2269 2.91768470535025e-08
2270 3.06263548436325e-08
2271 2.94136841214598e-08
2272 3.07603772433662e-08
2273 2.93494543633921e-08
2274 2.99070578407834e-08
2275 3.0622996608165e-08
2276 2.93591309176167e-08
2277 3.06217155876531e-08
2278 2.97760988310358e-08
2279 2.97992648563561e-08
2280 2.94139059049403e-08
2281 2.98957140962841e-08
2282 2.97721318667232e-08
2283 3.06889469874339e-08
2284 3.01081150393401e-08
2285 2.94271328902251e-08
2286 2.97922859062716e-08
2287 3.03896043745144e-08
2288 2.97602673249298e-08
2289 2.93806861386692e-08
2290 3.07173939956273e-08
2291 2.90212440451354e-08
2292 2.99000961614393e-08
2293 2.96484043212919e-08
2294 2.975014107931e-08
2295 2.99547251958465e-08
2296 3.00360338957306e-08
2297 2.92701941396745e-08
2298 3.02490154586721e-08
2299 2.97703548127748e-08
2300 3.01846086983026e-08
2301 2.97013639826149e-08
2302 2.98718815361365e-08
2303 2.97744867133742e-08
2304 3.05502880124653e-08
2305 2.89710731813386e-08
2306 3.042728129099e-08
2307 2.93363585363027e-08
2308 3.04395676502534e-08
2309 2.92000883450605e-08
2310 2.97441971210954e-08
2311 2.99568349160229e-08
2312 3.04154686978642e-08
2313 2.92412169661249e-08
2314 3.01671311898621e-08
2315 2.98941951442089e-08
2316 2.98397616687041e-08
2317 2.92874388443387e-08
2318 2.97383373231153e-08
2319 2.94353261289793e-08
2320 3.05385505123823e-08
2321 2.93338496889506e-08
2322 2.93648129168211e-08
2323 3.03710279947467e-08
2324 2.9533808314941e-08
2325 2.95260375601236e-08
2326 2.9302703373979e-08
2327 2.94889455623171e-08
2328 3.07821921363693e-08
2329 2.90491629808542e-08
2330 2.9775071914484e-08
2331 3.02324490033534e-08
2332 2.90101973845491e-08
2333 2.94474897735153e-08
2334 3.0512086895973e-08
2335 2.90937988570716e-08
2336 3.00642946284846e-08
2337 2.93832095917823e-08
2338 3.00295259756389e-08
2339 2.92324881251416e-08
2340 3.00672078952235e-08
2341 2.98411706562973e-08
2342 2.96734502702733e-08
2343 3.01239723143887e-08
2344 2.91235318838901e-08
2345 2.94947413164159e-08
2346 2.96759819662373e-08
2347 2.96830984733987e-08
2348 3.01810753086107e-08
2349 2.94066042534347e-08
2350 2.95907422727115e-08
2351 2.95599581977557e-08
2352 2.95247419236366e-08
2353 2.96643320860657e-08
2354 2.94082322804812e-08
2355 3.01323524305452e-08
2356 2.97312604604638e-08
2357 3.00321828730077e-08
2358 2.94392018662659e-08
2359 2.96166353941363e-08
2360 2.94867837108415e-08
2361 2.97435207736729e-08
2362 3.00884913658761e-08
2363 2.91452966347805e-08
2364 3.0324718255037e-08
2365 2.90985033277735e-08
2366 2.98004780090455e-08
2367 2.96012489855446e-08
2368 2.92611176587743e-08
2369 2.99039306319582e-08
2370 2.94178431254899e-08
2371 2.92096704046507e-08
2372 2.94758605436929e-08
2373 2.94820463795364e-08
2374 2.94090509223022e-08
2375 2.9809428323424e-08
2376 2.96039291505235e-08
2377 2.94427199986069e-08
2378 2.96902910483787e-08
2379 2.97745611930189e-08
2380 2.93633762394885e-08
2381 2.97806223035213e-08
2382 2.92948470450582e-08
2383 2.95823781295557e-08
2384 2.96783724760274e-08
2385 2.97070316810366e-08
2386 2.88886537535005e-08
2387 2.95929514739779e-08
2388 2.93819075475321e-08
2389 2.96361339887641e-08
2390 2.96993371017296e-08
2391 2.91663757250316e-08
2392 2.92190238249113e-08
2393 3.05163065134062e-08
2394 2.88426034316114e-08
2395 2.96760019791176e-08
2396 2.96531132636391e-08
2397 3.03518628921307e-08
2398 2.89987893417054e-08
2399 3.05551540028537e-08
2400 2.8446345491262e-08
2401 2.95164668903114e-08
2402 2.98806903293869e-08
2403 2.96243795286921e-08
2404 2.94033380729974e-08
2405 2.91018539273491e-08
2406 2.98373022298293e-08
2407 2.92671290426094e-08
2408 2.97560581331568e-08
2409 2.94869425110367e-08
2410 2.93049797864819e-08
2411 3.01471638752826e-08
2412 2.90197480704579e-08
2413 2.92625820268455e-08
2414 2.9505805632879e-08
2415 2.96556618089028e-08
2416 2.93979993148952e-08
2417 2.95725031499439e-08
2418 2.92233427995336e-08
2419 2.93992075721672e-08
2420 2.96679460244365e-08
2421 2.93073308114122e-08
2422 2.97189875322346e-08
2423 3.02913834857588e-08
2424 2.86712339274686e-08
2425 2.97590895077526e-08
2426 2.90782042966509e-08
2427 2.93777987127042e-08
2428 2.91761828155046e-08
2429 2.93774138564462e-08
2430 2.99367906583248e-08
2431 2.91483037826445e-08
2432 2.95520981989927e-08
2433 2.92296172224038e-08
2434 2.90627924979248e-08
2435 2.9477099779962e-08
2436 2.90331117527298e-08
2437 2.94265043406794e-08
2438 2.97683959215034e-08
2439 2.94016497960348e-08
2440 2.94534988490858e-08
2441 2.97891489009494e-08
2442 2.91483416388072e-08
2443 2.94071501136806e-08
2444 2.95422320764072e-08
2445 2.97798357493662e-08
2446 2.87567224028784e-08
2447 2.9765348987687e-08
2448 2.94075054042553e-08
2449 2.92116983732216e-08
2450 2.94092605326313e-08
2451 2.91900765163611e-08
2452 2.94084152545615e-08
2453 2.96476204241225e-08
2454 2.94020676164797e-08
2455 2.91896250371737e-08
2456 2.95758691024606e-08
2457 2.90378047234086e-08
2458 2.96332233375995e-08
2459 2.89555009489462e-08
2460 3.02219568752982e-08
2461 2.85789160027194e-08
2462 2.97070410389955e-08
2463 2.96087320859062e-08
2464 2.92476068841641e-08
2465 3.00842624293418e-08
2466 2.86249110250658e-08
2467 2.95020601009943e-08
2468 2.89757507143662e-08
2469 3.03053679220344e-08
2470 2.85543855292891e-08
2471 2.9476207702217e-08
2472 2.91501779201564e-08
2473 2.91710773572262e-08
2474 2.99704805988288e-08
2475 2.91141007964235e-08
2476 2.91909008671709e-08
2477 3.0136496931954e-08
2478 2.88333809307373e-08
2479 2.89910655966175e-08
2480 2.98018728570604e-08
2481 2.86439270900107e-08
2482 2.95102365577282e-08
2483 2.9344848162749e-08
2484 2.94958841262627e-08
2485 2.87770026957679e-08
2486 2.98831858029924e-08
2487 2.88803769620261e-08
2488 2.91911596564942e-08
2489 2.92011914315715e-08
2490 2.93099389765406e-08
2491 2.94275095169549e-08
2492 2.90304562542421e-08
2493 2.92442855540642e-08
2494 2.98031357803818e-08
2495 2.88171529789771e-08
2496 2.96570630038406e-08
2497 2.88751488003447e-08
2498 2.95277399180449e-08
2499 2.90497714845461e-08
2500 2.93610117803045e-08
2501 2.99737144118239e-08
2502 2.83481893645909e-08
2503 2.95796554451444e-08
2504 2.8981138991524e-08
2505 2.95250432090777e-08
2506 2.91148320099577e-08
2507 2.95071061200414e-08
2508 2.87168071558819e-08
2509 2.96792209928398e-08
2510 2.8558336714557e-08
2511 2.96109520485643e-08
2512 2.90088585397363e-08
2513 2.89088062241927e-08
2514 2.95291311773749e-08
2515 2.9096772066417e-08
2516 2.93760811537203e-08
2517 2.91710088367036e-08
2518 2.92550894743782e-08
2519 2.90080049636376e-08
2520 3.03524851450465e-08
2521 2.82524354489055e-08
2522 2.93634623788064e-08
2523 2.93425111345114e-08
2524 2.89845940789624e-08
2525 2.95355403471254e-08
2526 2.91874271977077e-08
2527 2.90694051767737e-08
2528 2.92883608488115e-08
2529 2.93661540253787e-08
2530 2.90914872923542e-08
2531 2.89353043294582e-08
2532 2.95072477864977e-08
2533 2.90573133113758e-08
2534 2.92474864496128e-08
2535 2.93479420989673e-08
2536 2.90438185051034e-08
2537 2.8942473043525e-08
2538 2.94118537222943e-08
2539 2.8989593207629e-08
2540 2.93055280805588e-08
2541 2.90685388795175e-08
2542 2.97183287754077e-08
2543 2.8761914614206e-08
2544 2.89713888955712e-08
2545 2.92280984525162e-08
2546 2.88409373289111e-08
2547 2.93033669609422e-08
2548 2.92154036594106e-08
2549 2.89304318793482e-08
2550 2.90346399383079e-08
2551 2.89512803124392e-08
2552 2.9327612569352e-08
2553 2.89178605066276e-08
2554 2.89350521486265e-08
2555 2.9567547811804e-08
2556 2.90225874842731e-08
2557 2.87376551788565e-08
2558 2.94303438301835e-08
2559 2.93241973912117e-08
2560 2.8419500489929e-08
2561 2.95345057752527e-08
2562 2.89618138373804e-08
2563 2.89628176164403e-08
2564 2.90808605741821e-08
2565 2.90085793794859e-08
2566 2.94219799400475e-08
2567 2.8752838328816e-08
2568 2.92456847612588e-08
2569 2.89504067293445e-08
2570 2.89740358574431e-08
2571 2.89333305185657e-08
2572 2.91548230781924e-08
2573 2.89687184552578e-08
2574 2.90249588765779e-08
2575 2.87988448908205e-08
2576 2.95803976130271e-08
2577 2.90101704867318e-08
2578 2.88224963045369e-08
2579 2.90058302943041e-08
2580 2.87493800251726e-08
2581 2.92536496799833e-08
2582 2.94828864215724e-08
2583 2.83998022370113e-08
2584 2.91293505153556e-08
2585 2.89117738462297e-08
2586 2.86073398264586e-08
2587 2.94442734799505e-08
2588 2.88064851936554e-08
2589 2.88228818585701e-08
2590 2.90562393376881e-08
2591 2.90499225531482e-08
2592 2.88316280404599e-08
2593 2.88127358524237e-08
2594 2.94542488767924e-08
2595 2.87098107450978e-08
2596 2.87337372276708e-08
2597 2.92478309422695e-08
2598 2.89900987107128e-08
2599 2.9475395028733e-08
2600 2.85901274899025e-08
2601 2.91677190313866e-08
2602 2.89157102195681e-08
2603 2.88904392282685e-08
2604 2.91138894407128e-08
2605 2.86386446978204e-08
2606 2.91344425984441e-08
2607 2.89048290411653e-08
2608 2.89487257574272e-08
2609 2.89616200224163e-08
2610 2.890575056258e-08
2611 2.89035483189704e-08
2612 2.88936174394072e-08
2613 2.97543129208488e-08
2614 2.80770325494784e-08
2615 2.93763200853725e-08
2616 2.88527366796298e-08
2617 2.92216607766793e-08
2618 2.8785151172217e-08
2619 2.91022536345054e-08
2620 2.8960060857175e-08
2621 2.86491077747808e-08
2622 2.91189867358543e-08
2623 2.86182097426479e-08
2624 2.87418293603015e-08
2625 2.93511708971961e-08
2626 2.86144644325859e-08
2627 2.89315233508169e-08
2628 2.88744798287999e-08
2629 2.88524149767433e-08
2630 2.89036601626158e-08
2631 2.87992470507925e-08
2632 2.90577971906458e-08
2633 2.94033483774214e-08
2634 2.80727245012269e-08
2635 2.92945257474031e-08
2636 2.8943403441839e-08
2637 2.8993800473831e-08
2638 2.89037577483331e-08
2639 2.88968998628514e-08
2640 2.86610964650436e-08
2641 2.88057360648963e-08
2642 2.91256781119742e-08
2643 2.88621869973804e-08
2644 2.85393486534335e-08
2645 2.85607162150425e-08
2646 2.87800735699895e-08
2647 2.88654821907208e-08
2648 2.88893562043757e-08
2649 2.95167208735903e-08
2650 2.78552102760399e-08
2651 2.99494551040391e-08
2652 2.78221993947891e-08
2653 2.91567559038697e-08
2654 2.88610512567677e-08
2655 2.82780638034552e-08
2656 2.89442908778526e-08
2657 2.86864566834133e-08
2658 2.89582424273238e-08
2659 2.88562080693611e-08
2660 2.90086992796867e-08
2661 2.85461694420341e-08
2662 2.896008420461e-08
2663 2.86066213932568e-08
2664 2.8896861445693e-08
2665 2.96469528258125e-08
2666 2.82770423565282e-08
2667 2.88280657537721e-08
2668 2.86769818498289e-08
2669 2.89949596321559e-08
2670 2.885488664639e-08
2671 2.86452711960594e-08
2672 2.9126635659682e-08
2673 2.88870629351345e-08
2674 2.88877027971868e-08
2675 2.91223377834049e-08
2676 2.8491901051142e-08
2677 2.87265059305541e-08
2678 2.86967222259404e-08
2679 2.9057823594858e-08
2680 2.89638898656186e-08
2681 2.89873532677287e-08
2682 2.85232326571183e-08
2683 2.87868258789237e-08
2684 2.89051587254452e-08
2685 2.86483070253141e-08
2686 2.8893219997661e-08
2687 2.91578343226684e-08
2688 2.80908882622777e-08
2689 2.86275743478503e-08
2690 2.89960903584463e-08
2691 2.86079943035977e-08
2692 2.87798809994744e-08
2693 2.86731442935562e-08
2694 2.95909955615459e-08
2695 2.83427868925656e-08
2696 2.88777850979693e-08
2697 2.86975519592225e-08
2698 2.88388717009092e-08
2699 2.88948398544653e-08
2700 2.91035345427737e-08
2701 2.86055671714358e-08
2702 2.8652346825031e-08
2703 2.86394855829597e-08
2704 2.88412715589992e-08
2705 2.8859951043847e-08
2706 2.85123831221812e-08
2707 2.86570441710987e-08
2708 2.91009851848267e-08
2709 2.86402339003677e-08
2710 2.8676396647942e-08
2711 2.84736364271065e-08
2712 2.91778053446157e-08
2713 2.84199467066593e-08
2714 2.94946188694745e-08
2715 2.81431516194752e-08
2716 2.93217650139121e-08
2717 2.83594388118535e-08
2718 2.87414469603009e-08
2719 2.87639301173037e-08
2720 2.85349998283557e-08
2721 2.90342546286348e-08
2722 2.8870388074087e-08
2723 2.87414357962312e-08
2724 2.86418310073699e-08
2725 2.87557609932598e-08
2726 2.85301034532726e-08
2727 2.8726375664534e-08
2728 2.93554184720435e-08
2729 2.79592132197415e-08
2730 2.91668633887188e-08
2731 2.85447821311013e-08
2732 2.90475944352275e-08
2733 2.85916568538713e-08
2734 2.90951629713376e-08
2735 2.82249664581569e-08
2736 2.8750047088022e-08
2737 2.88546447407834e-08
2738 2.83133875902397e-08
2739 2.8914681793335e-08
2740 2.89805220952122e-08
2741 2.85283959524651e-08
2742 2.92453952149829e-08
2743 2.81444097642858e-08
2744 2.884523234159e-08
2745 2.85979015627325e-08
2746 2.87500807166774e-08
2747 2.88701982241735e-08
2748 2.95651443316114e-08
2749 2.78217534022129e-08
2750 2.88465314753905e-08
2751 2.90049195759146e-08
2752 2.8969853720251e-08
2753 2.80174128495236e-08
2754 2.86621334034676e-08
2755 2.86334036286862e-08
2756 2.93620204765421e-08
2757 2.79956426453287e-08
2758 2.87804667206126e-08
2759 2.86587617144285e-08
2760 2.86707497936245e-08
2761 2.8349644561998e-08
2762 2.94719194152693e-08
2763 2.79536468815733e-08
2764 2.87476128572051e-08
2765 2.83823352685841e-08
2766 2.9046867660032e-08
2767 2.83741571404095e-08
2768 2.86127745385834e-08
2769 2.84097319902532e-08
2770 2.8950820968876e-08
2771 2.85706163739574e-08
2772 2.84831077568271e-08
2773 2.86874044364005e-08
2774 2.85190856891271e-08
2775 2.85915549812499e-08
2776 2.86476926649737e-08
2777 2.83911106051526e-08
2778 2.8763956467448e-08
2779 2.89560570344571e-08
2780 2.80553924104643e-08
2781 2.86258397156347e-08
2782 2.86940498629429e-08
2783 2.90747988633822e-08
2784 2.80850562303092e-08
2785 2.8670527930208e-08
2786 2.85400981439032e-08
2787 2.86425029797366e-08
2788 2.81599489548867e-08
2789 2.87414126385332e-08
2790 2.85455419412051e-08
2791 2.88066140872179e-08
2792 2.85066101781695e-08
2793 2.82825563236999e-08
2794 2.90721196258836e-08
2795 2.79984276913003e-08
2796 2.8596202046316e-08
2797 2.84322214718635e-08
2798 2.86890750418323e-08
2799 2.85837655387144e-08
2800 2.83836377753532e-08
2801 2.88100122587398e-08
2802 2.83783179431119e-08
2803 2.84862537922592e-08
2804 2.85944094797763e-08
2805 2.85899573834181e-08
2806 2.86930665206464e-08
2807 2.85238319038816e-08
2808 2.81540354489795e-08
2809 2.94100408168019e-08
2810 2.79276479001611e-08
2811 2.85979077857545e-08
2812 2.85546881335641e-08
2813 2.85856095063597e-08
2814 2.83752169656371e-08
2815 2.90946490355548e-08
2816 2.7735220615277e-08
2817 2.86537087965666e-08
2818 2.84097717669907e-08
2819 2.85976141484179e-08
2820 2.84083651301836e-08
2821 2.83823193297783e-08
2822 2.83347098419817e-08
2823 2.87782148424842e-08
2824 2.84326240735933e-08
2825 2.8411303279352e-08
2826 2.85194184547199e-08
2827 2.97172384683408e-08
2828 2.75930576575378e-08
2829 2.85075400549006e-08
2830 2.89290695040245e-08
2831 2.80529374704352e-08
2832 2.86050977158503e-08
2833 2.82639000561691e-08
2834 2.88847915791424e-08
2835 2.82687776023138e-08
2836 2.82526925516891e-08
2837 2.83731995802672e-08
2838 2.86073533470876e-08
2839 2.83169491699375e-08
2840 2.83325687118197e-08
2841 2.83774051650365e-08
2842 2.86117084744575e-08
2843 2.83134076421998e-08
2844 2.80535206841348e-08
2845 2.8559956799179e-08
2846 2.90166623770904e-08
2847 2.78716578998583e-08
2848 2.88523020606224e-08
2849 2.77786349786036e-08
2850 2.85989839614587e-08
2851 2.83404239220797e-08
2852 2.89866962595031e-08
2853 2.80792777636041e-08
2854 2.85012278549157e-08
2855 2.82806928293411e-08
2856 2.81697042323614e-08
2857 2.83376187173445e-08
2858 2.83680161158317e-08
2859 2.86134583711295e-08
2860 2.83303538716195e-08
2861 2.82068457736484e-08
2862 2.85627960087131e-08
2863 2.8386771767086e-08
2864 2.8431141650298e-08
2865 2.84216588941621e-08
2866 2.83713823308052e-08
2867 2.82172252245028e-08
2868 2.83083117298411e-08
2869 2.81007825067014e-08
2870 2.90718448121563e-08
2871 2.78446450046044e-08
2872 2.84185669156001e-08
2873 2.82877916302215e-08
2874 2.84560814690638e-08
2875 2.84731634641044e-08
2876 2.84095249567518e-08
2877 2.84163475769983e-08
2878 2.82850195585382e-08
2879 2.82649835248172e-08
2880 2.83527075310985e-08
2881 2.82508198988785e-08
2882 2.81548445170143e-08
2883 2.83681794999158e-08
2884 2.83962266929283e-08
2885 2.82499229449229e-08
2886 2.85268819857354e-08
2887 2.81174507295034e-08
2888 2.83996574266254e-08
2889 2.82712126599138e-08
2890 2.82130006072911e-08
2891 2.83281600872431e-08
2892 2.81340804906272e-08
2893 2.84489638030516e-08
2894 2.81354023048408e-08
2895 2.83116045376453e-08
2896 2.81982578587137e-08
2897 2.8118047060488e-08
2898 2.85208489595457e-08
2899 2.78947529663842e-08
2900 2.85355309161961e-08
2901 2.81678612776837e-08
2902 2.83957389075606e-08
2903 2.81588537888178e-08
2904 2.80500982020593e-08
2905 2.82646933612574e-08
2906 2.82334258714378e-08
2907 2.83976593853374e-08
2908 2.82937302742425e-08
2909 2.83765647750567e-08
2910 2.77987882280062e-08
2911 2.85269583746306e-08
2912 2.83577735460394e-08
2913 2.81982733044694e-08
2914 2.83104854187366e-08
2915 2.86199759012007e-08
2916 2.80212551946324e-08
2917 2.83902935516167e-08
2918 2.81241435315938e-08
2919 2.82733346324004e-08
2920 2.83641809512014e-08
2921 2.82841858434413e-08
2922 2.85251515659279e-08
2923 2.80728458237345e-08
2924 2.81676307243295e-08
2925 2.82416357182402e-08
2926 2.84035605991884e-08
2927 2.83863157306552e-08
2928 2.81927711149876e-08
2929 2.82463207450512e-08
2930 2.83581972480018e-08
2931 2.80791124555035e-08
2932 2.8381891526652e-08
2933 2.82222253775721e-08
2934 2.81983419159193e-08
2935 2.83148695735536e-08
2936 2.78467907054436e-08
2937 2.82910142264114e-08
2938 2.83898382722469e-08
2939 2.82872072989715e-08
2940 2.84329279720552e-08
2941 2.82127936377385e-08
2942 2.83537506836629e-08
2943 2.82198639199871e-08
2944 2.81466178581047e-08
2945 2.87068259952061e-08
2946 2.80936947552046e-08
2947 2.82579290219465e-08
2948 2.84824074296974e-08
2949 2.81979475034166e-08
2950 2.81696485685545e-08
2951 2.83842467021511e-08
2952 2.80958759237837e-08
2953 2.79256651717352e-08
2954 2.82538396856324e-08
2955 2.82642671387556e-08
2956 2.8278059008624e-08
2957 2.82930575041807e-08
2958 2.80076368951443e-08
2959 2.81913895903152e-08
2960 2.78253146550611e-08
2961 2.82454930556897e-08
2962 2.81805331964202e-08
2963 2.81135265844945e-08
2964 2.83333956498932e-08
2965 2.82534181579308e-08
2966 2.82063516949815e-08
2967 2.83269296792454e-08
2968 2.79209698333949e-08
2969 2.82396759910819e-08
2970 2.81661355914231e-08
2971 2.81527829903983e-08
2972 2.80172681664803e-08
2973 2.82573512276896e-08
2974 2.81350629602883e-08
2975 2.84578959969251e-08
2976 2.80058662265326e-08
2977 2.81972478658465e-08
2978 2.81002942715824e-08
2979 2.83443139219575e-08
2980 2.79673694880689e-08
2981 2.83762513125785e-08
2982 2.81720147458087e-08
2983 2.82798376860516e-08
2984 2.8320491458067e-08
2985 2.80200739929404e-08
2986 2.80250258282155e-08
2987 2.82528663901882e-08
2988 2.84744600704823e-08
2989 2.78163033761158e-08
2990 2.80425233798454e-08
2991 2.79856252106692e-08
2992 2.84480237608964e-08
2993 2.80238439888114e-08
2994 2.79858017362411e-08
2995 2.83757469484724e-08
2996 2.80435749681063e-08
2997 2.82983341284293e-08
2998 2.82698186515606e-08
2999 2.8199431490239e-08
3000 2.79091114618879e-08
3001 2.83055894922946e-08
3002 2.79298435759134e-08
3003 2.81626771870824e-08
3004 2.81493968414814e-08
3005 2.80206475711253e-08
3006 2.80815410320656e-08
3007 2.80952622641051e-08
3008 2.80271784957042e-08
3009 2.8235720414882e-08
3010 2.795950542267e-08
3011 2.77871548410857e-08
3012 2.82513461487e-08
3013 2.82935446046517e-08
3014 2.76727798734777e-08
3015 2.82596458657247e-08
3016 2.81011000822229e-08
3017 2.79621727553581e-08
3018 2.76745936821365e-08
3019 2.83982442910169e-08
3020 2.79102976760148e-08
3021 2.80427712975317e-08
3022 2.80634418313719e-08
3023 2.8192872375099e-08
3024 2.7960027652818e-08
3025 2.79712581416902e-08
3026 2.81854522783398e-08
3027 2.79375560896522e-08
3028 2.808093783635e-08
3029 2.78317703604758e-08
3030 2.80189360943872e-08
3031 2.81492678246842e-08
3032 2.81460490049179e-08
3033 2.78709664287558e-08
3034 2.79107633406328e-08
3035 2.79678644515835e-08
3036 2.78809628710963e-08
3037 2.79836345568052e-08
3038 2.79947002315151e-08
3039 2.8117260939986e-08
3040 2.78411516010379e-08
3041 2.7966441608962e-08
3042 2.77456996856662e-08
3043 2.81840890948848e-08
3044 2.7863352474955e-08
3045 2.79436265058219e-08
3046 2.81165151310159e-08
3047 2.77483319854888e-08
3048 2.8069222722138e-08
3049 2.78737463591527e-08
3050 2.79356282917309e-08
3051 2.78617407964976e-08
3052 2.7935579124061e-08
3053 2.80236768510633e-08
3054 2.78544102667588e-08
3055 2.78889985583231e-08
3056 2.80171046446176e-08
3057 2.80413317912398e-08
3058 2.7690788237722e-08
3059 2.80023163667931e-08
3060 2.76263318524927e-08
3061 2.80384687021407e-08
3062 2.80195101539649e-08
3063 2.78813628309393e-08
3064 2.76387834953207e-08
3065 2.78788636691729e-08
3066 2.77957652689009e-08
3067 2.77782616324718e-08
3068 2.77054197079796e-08
3069 2.79212522411587e-08
3070 2.78154891166782e-08
3071 2.80668877475909e-08
3072 2.75467163519671e-08
3073 2.77690045441537e-08
3074 2.77794684397925e-08
3075 2.778427329031e-08
3076 2.78553184965835e-08
3077 2.7779591066146e-08
3078 2.78453358508735e-08
3079 2.77492309133098e-08
3080 2.78373710536073e-08
3081 2.78422435481263e-08
3082 2.78921056811043e-08
3083 2.76621400935317e-08
3084 2.78926978622973e-08
3085 2.76104009843126e-08
3086 2.79870068472521e-08
3087 2.80112244332997e-08
3088 2.75314212770938e-08
3089 2.77554793818302e-08
3090 2.75598573672431e-08
3091 2.7783453274699e-08
3092 2.77891953087694e-08
3093 2.78972306193559e-08
3094 2.75785681637108e-08
3095 2.75578075392691e-08
3096 2.78254986207926e-08
3097 2.76985386863382e-08
3098 2.76641067515904e-08
3099 2.75344376735598e-08
3100 2.78293600179946e-08
3101 2.75611608813175e-08
3102 2.75746115881992e-08
3103 2.75887629543892e-08
3104 2.76321048162664e-08
3105 2.7446046206081e-08
3106 2.76459600561108e-08
3107 2.7567236129844e-08
3108 2.75590360548916e-08
3109 2.75658043481508e-08
3110 2.74795975464537e-08
3111 2.76100277540881e-08
3112 2.7538884219247e-08
3113 2.74964252265919e-08
3114 2.74920240281862e-08
3115 2.74738173520195e-08
3116 2.75929861723867e-08
3117 2.76905557566876e-08
3118 2.7592008135402e-08
3119 2.72800261640338e-08
3120 2.76826469786107e-08
3121 2.75223683647852e-08
3122 2.74218763001155e-08
3123 2.75229544806077e-08
3124 2.77055701420892e-08
3125 2.7533849395911e-08
3126 2.74562799968958e-08
3127 2.7641547263868e-08
3128 2.73432013916386e-08
3129 2.75130208585717e-08
3130 2.74304387838376e-08
3131 2.76711635285309e-08
3132 2.72780012937623e-08
3133 2.77249775620936e-08
3134 2.71146841819281e-08
3135 2.74942579004955e-08
3136 2.7433166561619e-08
3137 2.73241959698467e-08
3138 2.78105984573562e-08
3139 2.73570337067097e-08
3140 2.73705430947624e-08
3141 2.7349809064825e-08
3142 2.73527357333991e-08
3143 2.75773410667313e-08
3144 2.74888749294266e-08
3145 2.72510686593064e-08
3146 2.73142910460766e-08
3147 2.71815379683238e-08
3148 2.73207265210074e-08
3149 2.73605206141392e-08
3150 2.72063679226386e-08
3151 2.73371626652308e-08
3152 2.7245666361253e-08
3153 2.72159411578432e-08
3154 2.71609758980551e-08
3155 2.75866332541952e-08
3156 2.69058729057692e-08
3157 2.74389790785312e-08
3158 2.74429869955295e-08
3159 2.7067948081716e-08
3160 2.71760569935386e-08
3161 2.7257598278263e-08
3162 2.70215362450088e-08
3163 2.72413410380423e-08
3164 2.71686692963069e-08
3165 2.72808731902607e-08
3166 2.68342822181689e-08
3167 2.71794309028017e-08
3168 2.71724197842271e-08
3169 2.69884155441602e-08
3170 2.71172351539173e-08
3171 2.71544485644926e-08
3172 2.69768486024224e-08
3173 2.71209147028051e-08
3174 2.70248692471053e-08
3175 2.70219993201426e-08
3176 2.69864458248836e-08
3177 2.70829218127489e-08
3178 2.68198978687906e-08
3179 2.70420039116814e-08
3180 2.69352479116414e-08
3181 2.70500435398668e-08
3182 2.68783779927562e-08
3183 2.69980539552295e-08
3184 2.70073065707921e-08
3185 2.67736913084349e-08
3186 2.68310359857749e-08
3187 2.6828798302847e-08
3188 2.67435424922136e-08
3189 2.67935684025167e-08
3190 2.67125504821442e-08
3191 2.67968044155298e-08
3192 2.66279777652034e-08
3193 2.66731707653234e-08
3194 2.66862853003369e-08
3195 2.65747471291933e-08
3196 2.65646306256695e-08
3197 2.66970578132186e-08
3198 2.65286635356965e-08
3199 2.65000179009434e-08
3200 2.66396445861572e-08
3201 2.63169318284184e-08
3202 2.66643296861213e-08
3203 2.62023226496044e-08
3204 2.6534131509881e-08
3205 2.62582025558622e-08
3206 2.63978947931953e-08
3207 2.61657953012362e-08
3208 2.62115586976419e-08
3209 2.61652698657633e-08
3210 2.61396239490397e-08
3211 2.60952647571644e-08
3212 2.61129713615693e-08
3213 2.59728101540402e-08
3214 2.61637379793678e-08
3215 2.61099466271064e-08
3216 2.60147822199741e-08
3217 2.60451803246742e-08
3218 2.58620108545271e-08
3219 2.58816574110199e-08
3220 2.57308607921614e-08
3221 2.59106916662422e-08
3222 2.55465615981443e-08
3223 2.57637011725764e-08
3224 2.55745806446717e-08
3225 2.58174272969036e-08
3226 2.55421040380455e-08
3227 2.55320722412078e-08
3228 2.56440008442604e-08
3229 2.55824956335582e-08
3230 2.55045796695441e-08
3231 2.55155588158029e-08
3232 2.54135027706326e-08
3233 2.55356180016708e-08
3234 2.5315493921152e-08
3235 2.53959738724108e-08
3236 2.53071498000912e-08
3237 2.5348794585045e-08
3238 2.5395621145563e-08
3239 2.51948092977461e-08
3240 2.50281380965722e-08
3241 2.51241555769832e-08
3242 2.51803180477683e-08
3243 2.50487469175553e-08
3244 2.50027343184911e-08
3245 2.50172455327213e-08
3246 2.50034757349749e-08
3247 2.4802950796543e-08
3248 2.47983386888473e-08
3249 2.48387833662322e-08
3250 2.48319105866646e-08
3251 2.47321568982439e-08
3252 2.47088250158178e-08
3253 2.44659451085205e-08
3254 2.4690066613875e-08
3255 2.44801841973175e-08
3256 2.45084866066225e-08
3257 2.44491681123948e-08
3258 2.44119536589871e-08
3259 2.4386582128777e-08
3260 2.42280119142624e-08
3261 2.42861124393201e-08
3262 2.42433896637806e-08
3263 2.41555658339054e-08
3264 2.41266607357016e-08
3265 2.40801400634982e-08
3266 2.38926721207733e-08
3267 2.39838350898358e-08
3268 2.40506120056105e-08
3269 2.3916483878117e-08
3270 2.37057272631791e-08
3271 2.39021048126231e-08
3272 2.37873047519432e-08
3273 2.37448052139433e-08
3274 2.36883957996925e-08
3275 2.37279451729133e-08
3276 2.35399021545346e-08
3277 2.35209500220046e-08
3278 2.34263225770137e-08
3279 2.35776355063111e-08
3280 2.33919672301131e-08
3281 2.34329803233679e-08
3282 2.33126912406734e-08
3283 2.35004734441269e-08
3284 2.31742621744635e-08
3285 2.31931167944399e-08
3286 2.34330146651196e-08
3287 2.30433868618762e-08
3288 2.30964814371681e-08
3289 2.29994626480545e-08
3290 2.31289862495654e-08
3291 2.29159458738915e-08
3292 2.29901219590545e-08
3293 2.30029163238443e-08
3294 2.28312812222686e-08
3295 2.29063712794408e-08
3296 2.28959773932447e-08
3297 2.27306364943836e-08
3298 2.27655924831538e-08
3299 2.26784687136217e-08
3300 2.26461730532534e-08
3301 2.2606170950068e-08
3302 2.250042299623e-08
3303 2.25686510207934e-08
3304 2.25845469965513e-08
3305 2.24978937750642e-08
3306 2.26201085872235e-08
3307 2.24112469240367e-08
3308 2.25353949415696e-08
3309 2.23580190832351e-08
3310 2.2297263171156e-08
3311 2.23790232520837e-08
3312 2.24466665393885e-08
3313 2.23067037571356e-08
3314 2.23208244515272e-08
3315 2.22678170412882e-08
3316 2.24187223273331e-08
3317 2.20547565594753e-08
3318 2.19542988268984e-08
3319 2.22034416796912e-08
3320 2.18869837789182e-08
3321 2.23114572620586e-08
3322 2.19153354918955e-08
3323 2.20372019642534e-08
3324 2.18876831032944e-08
3325 2.19587715867275e-08
3326 2.18222054039208e-08
3327 2.17799231747584e-08
3328 2.19262232447059e-08
3329 2.17003897614587e-08
3330 2.18653845295291e-08
3331 2.20210654383601e-08
3332 2.16570453549503e-08
3333 2.18818665207454e-08
3334 2.17300078979576e-08
3335 2.19063769106853e-08
3336 2.14952040836769e-08
3337 2.16748563930391e-08
3338 2.15727356435158e-08
3339 2.15737895983281e-08
3340 2.16211743219175e-08
3341 2.14782548310088e-08
3342 2.1572836051198e-08
3343 2.16143502199673e-08
3344 2.16258473714204e-08
3345 2.16003263364684e-08
3346 2.16210139494244e-08
3347 2.15460362633646e-08
3348 2.13036852214632e-08
3349 2.15049846075122e-08
3350 2.13416783302334e-08
3351 2.12637118572889e-08
3352 2.16902532874652e-08
3353 2.12694339138952e-08
3354 2.14261858635378e-08
3355 2.12369422081249e-08
3356 2.13582918046074e-08
3357 2.12815205899997e-08
3358 2.14256679310632e-08
3359 2.10742473176762e-08
3360 2.1226377969974e-08
3361 2.13088348247625e-08
3362 2.10928710274727e-08
3363 2.1086249603175e-08
3364 2.12522901992251e-08
3365 2.11466176675001e-08
3366 2.12056394339744e-08
3367 2.10208478640395e-08
3368 2.11184369736106e-08
3369 2.11114070808316e-08
3370 2.10466769855477e-08
3371 2.11174465594155e-08
3372 2.09319472896929e-08
3373 2.11001993211113e-08
3374 2.11496547556367e-08
3375 2.09091995156951e-08
3376 2.08423462020324e-08
3377 2.08113332457183e-08
3378 2.09913278942375e-08
3379 2.07633296707144e-08
3380 2.08169918171075e-08
3381 2.08840307112101e-08
3382 2.11106708863928e-08
3383 2.0748616208377e-08
3384 2.10230551900281e-08
3385 2.1030296608493e-08
3386 2.07626094909097e-08
3387 2.09592193520791e-08
3388 2.07804836095393e-08
3389 2.07938029402888e-08
3390 2.08890248210025e-08
3391 2.08260918885816e-08
3392 2.06996619841471e-08
3393 2.08077140638752e-08
3394 2.0651007292205e-08
3395 2.08290535850253e-08
3396 2.07875670261704e-08
3397 2.06521226284728e-08
3398 2.08220181593388e-08
3399 2.05428421374831e-08
3400 2.06260095988453e-08
3401 2.06088334322452e-08
3402 2.054931689488e-08
3403 2.08016685867562e-08
3404 2.0661535245492e-08
3405 2.06920555571033e-08
3406 2.02342798437671e-08
3407 2.07037053663317e-08
3408 2.06358862396838e-08
3409 2.05914988679057e-08
3410 2.03042139450549e-08
3411 2.04995299794009e-08
3412 2.05692742294428e-08
3413 2.07628378954272e-08
3414 2.04154577216897e-08
3415 2.04392907611206e-08
3416 2.05519064319093e-08
3417 2.05288405481507e-08
3418 2.04131706327271e-08
3419 2.05944367069888e-08
3420 2.04313066544115e-08
3421 2.0525549609407e-08
3422 2.03326906680346e-08
3423 2.03926289948919e-08
3424 2.04649986342842e-08
3425 2.05701290373339e-08
3426 2.05334375967814e-08
3427 2.05836521574021e-08
3428 2.02607399655275e-08
3429 2.0288102376087e-08
3430 2.05155836032755e-08
3431 2.04091768267745e-08
3432 2.06022678473916e-08
3433 2.02149246720795e-08
3434 2.02372371850279e-08
3435 2.06067556529632e-08
3436 2.04647079823372e-08
3437 2.02712737537025e-08
3438 2.03115415376987e-08
3439 2.03790528048309e-08
3440 2.02208359421086e-08
3441 2.01446425234231e-08
3442 2.0144239427089e-08
3443 2.0194728801548e-08
3444 2.03472436051122e-08
3445 2.00871820106863e-08
3446 2.01554177798879e-08
3447 2.02507993577461e-08
3448 2.02269489643081e-08
3449 2.03056738815599e-08
3450 2.02552716678239e-08
3451 2.01746287793458e-08
3452 2.00254406783618e-08
3453 2.02492774203922e-08
3454 2.04233538162457e-08
3455 1.98678674194763e-08
3456 2.08340536229645e-08
3457 1.99150352416799e-08
3458 2.0088927602524e-08
3459 1.99543420581483e-08
3460 1.99859954440607e-08
3461 1.99950265119542e-08
3462 2.00830198858748e-08
3463 2.0302433976771e-08
3464 2.01071026636468e-08
3465 1.99955682436137e-08
3466 2.04297008412624e-08
3467 1.97105903335659e-08
3468 2.01992976099907e-08
3469 1.99106921218162e-08
3470 2.02184139486139e-08
3471 1.98602676667603e-08
3472 2.00949374680182e-08
3473 2.03604155989412e-08
3474 1.97432559466693e-08
3475 2.01167006801395e-08
3476 1.99338691754258e-08
3477 2.01398931839458e-08
3478 1.9843330370084e-08
3479 1.99221019376861e-08
3480 2.0074491318578e-08
3481 1.97414845081179e-08
3482 2.01965751633892e-08
3483 1.98351384923523e-08
3484 1.99440456919442e-08
3485 1.99886535905414e-08
3486 1.99083337206751e-08
3487 1.99318102591661e-08
3488 1.98541701432742e-08
3489 1.98469671135459e-08
3490 1.98829971330694e-08
3491 1.98145497295466e-08
3492 1.99893887107283e-08
3493 1.98098771061472e-08
3494 1.99491361169146e-08
3495 1.98092463585819e-08
3496 1.97894414438737e-08
3497 1.99392445080493e-08
3498 1.98606795582812e-08
3499 2.04737338653782e-08
3500 1.94500959165289e-08
3501 2.00015775844165e-08
3502 1.97740072873098e-08
3503 1.97318487216158e-08
3504 1.98427481989949e-08
3505 2.0020485375416e-08
3506 1.96137231172067e-08
3507 2.0273106138724e-08
3508 1.97381692259579e-08
3509 1.97420545173799e-08
3510 2.0189011973315e-08
3511 1.96803210443797e-08
3512 1.97295501885297e-08
3513 1.97554418031043e-08
3514 1.95984642300528e-08
3515 1.9905108216034e-08
3516 1.95825600302069e-08
3517 1.97158007412268e-08
3518 2.00738268899547e-08
3519 1.95265160355396e-08
3520 1.97931136487473e-08
3521 2.00545691371179e-08
3522 1.95283524628831e-08
3523 1.98421429071782e-08
3524 1.95968357952214e-08
3525 1.97028747664918e-08
3526 1.96775321277265e-08
3527 1.99269168309479e-08
3528 1.94447768420147e-08
3529 1.97860284033569e-08
3530 1.98933422053127e-08
3531 1.95760755414831e-08
3532 1.97671138253419e-08
3533 1.96574089929014e-08
3534 1.96586220211903e-08
3535 1.96589536163305e-08
3536 1.96083746922815e-08
3537 1.94270151208098e-08
3538 1.98854414222849e-08
3539 1.96722842464858e-08
3540 1.95178513436867e-08
3541 1.97794519233296e-08
3542 1.97161488323383e-08
3543 1.93748241499314e-08
3544 1.97948093791922e-08
3545 1.98135051507897e-08
3546 1.96795152547224e-08
3547 1.94891969489408e-08
3548 2.01518177852522e-08
3549 1.9382425125003e-08
3550 1.9890252528465e-08
3551 1.94439208551778e-08
3552 1.98800220051476e-08
3553 1.94739927028564e-08
3554 1.96770772700194e-08
3555 1.95973567769236e-08
3556 1.95144419145588e-08
3557 1.9370209166758e-08
3558 1.95760165211389e-08
3559 1.96507015237657e-08
3560 1.9346489371741e-08
3561 1.95399649081107e-08
3562 1.9467744677093e-08
3563 1.93864727194848e-08
3564 1.96351701592379e-08
3565 1.93582798303504e-08
3566 1.94476394420606e-08
3567 1.995010598832e-08
3568 1.93561940188269e-08
3569 1.9427631128055e-08
3570 1.94622925053789e-08
3571 2.00819316207301e-08
3572 1.95531294374662e-08
3573 1.92042797497738e-08
3574 1.98316077600902e-08
3575 1.94983776506952e-08
3576 1.93554313913102e-08
3577 1.95373353387707e-08
3578 1.93071585845583e-08
3579 1.93157417887058e-08
3580 1.95001357721214e-08
3581 1.94048716919415e-08
3582 1.96097919515914e-08
3583 1.94538471600669e-08
3584 1.93928954499434e-08
3585 1.94015148347049e-08
3586 1.95237705248319e-08
3587 1.98156417847706e-08
3588 1.94564287002152e-08
3589 1.95556149797937e-08
3590 1.94601944112915e-08
3591 1.93986332526208e-08
3592 1.93916439301667e-08
3593 1.9241454913943e-08
3594 1.94311731230856e-08
3595 1.94163889387644e-08
3596 1.93713783538385e-08
3597 1.92933984327626e-08
3598 1.93731371082029e-08
3599 1.94631309045068e-08
3600 1.92185033621728e-08
3601 1.92979121470538e-08
3602 1.94132234596633e-08
3603 1.92357453284719e-08
3604 1.95465936018158e-08
3605 1.94580885858331e-08
3606 1.9329651522515e-08
3607 1.92161421109782e-08
3608 1.91721530734545e-08
3609 1.92976955660784e-08
3610 1.93376808781398e-08
3611 1.93477287179222e-08
3612 1.92370867245772e-08
3613 1.94077803805648e-08
3614 1.92829351456369e-08
3615 1.94385736747016e-08
3616 1.91271535296966e-08
3617 1.93188536889988e-08
3618 1.94569932203681e-08
3619 1.91392434273352e-08
3620 1.92265867456909e-08
3621 1.93448527131546e-08
3622 1.92596772750075e-08
3623 1.92860557803876e-08
3624 1.93377357023961e-08
3625 1.92188200694998e-08
3626 1.93496095836077e-08
3627 1.94194053509955e-08
3628 1.91551871911999e-08
3629 1.91416044414971e-08
3630 1.94123930682411e-08
3631 1.91771119417705e-08
3632 1.91052352809251e-08
3633 1.9285150060222e-08
3634 1.91765616903705e-08
3635 1.91668869669037e-08
3636 1.93751134314191e-08
3637 1.9011724302076e-08
3638 1.93779719269704e-08
3639 1.9181259613088e-08
3640 1.92687653305379e-08
3641 1.90869587056719e-08
3642 1.91625062240242e-08
3643 1.92241475531763e-08
3644 1.9277164853615e-08
3645 1.90299798774607e-08
3646 1.92236918584721e-08
3647 1.91456657203659e-08
3648 1.93872644058679e-08
3649 1.91710485017982e-08
3650 1.89834676308509e-08
3651 1.92315756285533e-08
3652 1.91476609828767e-08
3653 1.9239069113941e-08
3654 1.93332295580495e-08
3655 1.89901691984851e-08
3656 1.90166875435605e-08
3657 1.94154823216497e-08
3658 1.92573718218458e-08
3659 1.93905425340812e-08
3660 1.91295660494362e-08
3661 1.88404853020963e-08
3662 1.91782462884982e-08
3663 1.91857683353724e-08
3664 1.89041054664552e-08
3665 1.94165642112276e-08
3666 1.93713986905708e-08
3667 1.91469240068409e-08
3668 1.90218115593277e-08
3669 1.93541289029708e-08
3670 1.89911731858228e-08
3671 1.92155910349046e-08
3672 1.91974466157152e-08
3673 1.90404540618694e-08
3674 1.93617334971474e-08
3675 1.90520799988603e-08
3676 1.89742912231061e-08
3677 1.90932648871378e-08
3678 1.91064340141489e-08
3679 1.92610978712038e-08
3680 1.90337636103433e-08
3681 1.90506621176034e-08
3682 1.92715854070435e-08
3683 1.90289143291444e-08
3684 1.89615405876697e-08
3685 1.94670514475082e-08
3686 1.920629496166e-08
3687 1.88003180705554e-08
3688 1.9029280281524e-08
3689 1.89357676843027e-08
3690 1.93651630618907e-08
3691 1.91696073534864e-08
3692 1.88684581632348e-08
3693 1.89973981438163e-08
3694 1.90069930983139e-08
3695 1.89995533612652e-08
3696 1.94116632323826e-08
3697 1.89517408876183e-08
3698 1.90220145644959e-08
3699 1.91609872810528e-08
3700 1.9139696994408e-08
3701 1.90576272183218e-08
3702 1.8869744696004e-08
3703 1.90411292422255e-08
3704 1.91045564036374e-08
3705 1.94223655702874e-08
3706 1.89673364072718e-08
3707 1.91063297177996e-08
3708 1.89178027274473e-08
3709 1.91902042183667e-08
3710 1.9103045425628e-08
3711 1.88865028811769e-08
3712 1.89706263540312e-08
3713 1.91670225168084e-08
3714 1.91678321270761e-08
3715 1.88549846424912e-08
3716 1.92952909235933e-08
3717 1.88379875073474e-08
3718 1.90356831267779e-08
3719 1.88638450685508e-08
3720 1.89739086483565e-08
3721 1.91322736243782e-08
3722 1.89829354541038e-08
3723 1.89566071462366e-08
3724 1.9008269390608e-08
3725 1.88833747695183e-08
3726 1.88695192051558e-08
3727 1.91722223743529e-08
3728 1.90448063014381e-08
3729 1.91930123175643e-08
3730 1.89360845745945e-08
3731 1.88622994220511e-08
3732 1.89511800262609e-08
3733 1.89990445170762e-08
3734 1.88203837536705e-08
3735 1.89545221378484e-08
3736 1.89407989961143e-08
3737 1.88812224598545e-08
3738 1.92579936043602e-08
3739 1.88095337164684e-08
3740 1.91264719842099e-08
3741 1.88689659460506e-08
3742 1.87580022200651e-08
3743 1.89789163435039e-08
3744 1.88618365725146e-08
3745 1.910417867379e-08
3746 1.89216035529904e-08
3747 1.87789413828021e-08
3748 1.91892249233883e-08
3749 1.88075505604957e-08
3750 1.88453737963723e-08
3751 1.89359005504652e-08
3752 1.8943391278059e-08
3753 1.88844860229409e-08
3754 1.88668736960107e-08
3755 1.89651927466894e-08
3756 1.89466653279791e-08
3757 1.90315935872931e-08
3758 1.88195179940953e-08
3759 1.88531753670951e-08
3760 1.89479245378266e-08
3761 1.86893563658419e-08
3762 1.88947021735331e-08
3763 1.91648341945072e-08
3764 1.88384566806032e-08
3765 1.90978672676145e-08
3766 1.87321867666235e-08
3767 1.90069559312667e-08
3768 1.88570157247936e-08
3769 1.8963783514514e-08
3770 1.88737557670349e-08
3771 1.88715299600517e-08
3772 1.89664147102198e-08
3773 1.89981390684713e-08
3774 1.87633205387394e-08
3775 1.85609815379628e-08
3776 1.89318983901954e-08
3777 1.8984863151883e-08
3778 1.91220557932414e-08
3779 1.85169175798139e-08
3780 1.87339930706187e-08
3781 1.86591365668454e-08
3782 1.87593908536066e-08
3783 1.92081936690736e-08
3784 1.88455629125395e-08
3785 1.92994762743259e-08
3786 1.86310612622087e-08
3787 1.87066970153138e-08
3788 1.89153779807105e-08
3789 1.88203566343637e-08
3790 1.894324860785e-08
3791 1.87322577296367e-08
3792 1.8734251690411e-08
3793 1.88853661517996e-08
3794 1.8583672932504e-08
3795 1.90287244721254e-08
3796 1.86269771357273e-08
3797 1.87372534786734e-08
3798 1.85120146586337e-08
3799 1.89040177708266e-08
3800 1.89885206554052e-08
3801 1.87531746743685e-08
3802 1.89284947005319e-08
3803 1.87692136939255e-08
3804 1.89363275125975e-08
3805 1.856850493831e-08
3806 1.91352258871458e-08
3807 1.88146941310086e-08
3808 1.86800021116929e-08
3809 1.90525881976766e-08
3810 1.87641638454972e-08
3811 1.88329063083303e-08
3812 1.88594862713387e-08
3813 1.86209481186639e-08
3814 1.90001626604319e-08
3815 1.87408806739953e-08
3816 1.89002604604749e-08
3817 1.88250154734559e-08
3818 1.85054755714731e-08
3819 1.87639380391236e-08
3820 1.88598880697111e-08
3821 1.87933379587335e-08
3822 1.8775800903259e-08
3823 1.8881062862075e-08
3824 1.88647653351914e-08
3825 1.89076898807761e-08
3826 1.85905896317173e-08
3827 1.86646233671928e-08
3828 1.88749318683623e-08
3829 1.87061744255645e-08
3830 1.85979168660921e-08
3831 1.85943793589161e-08
3832 1.88792440787289e-08
3833 1.8732076007888e-08
3834 1.90384355974738e-08
3835 1.88024904351769e-08
3836 1.86206032526393e-08
3837 1.8524852045676e-08
3838 1.87477323192464e-08
3839 1.88373445699774e-08
3840 1.87057135164759e-08
3841 1.85562752526014e-08
3842 1.90067275472883e-08
3843 1.85399173436851e-08
3844 1.85967446800772e-08
3845 1.88503128073503e-08
3846 1.86756053022208e-08
3847 1.8750540020096e-08
3848 1.88985965936528e-08
3849 1.8889713446657e-08
3850 1.86329569338284e-08
3851 1.89210242340643e-08
3852 1.87988617522672e-08
3853 1.8590867279511e-08
3854 1.86695465365094e-08
3855 1.87214220843668e-08
3856 1.89313352407794e-08
3857 1.85596157806778e-08
3858 1.86125494390232e-08
3859 1.86516145800342e-08
3860 1.88281226115583e-08
3861 1.86347137791643e-08
3862 1.88205656171947e-08
3863 1.8276289187269e-08
3864 1.89837334261256e-08
3865 1.85933126177762e-08
3866 1.85213250635252e-08
3867 1.85716440824768e-08
3868 1.8819136600845e-08
3869 1.87099917301481e-08
3870 1.87247360949883e-08
3871 1.86710728469208e-08
3872 1.876577496851e-08
3873 1.85100895350176e-08
3874 1.85281703523055e-08
3875 1.86349043527212e-08
3876 1.86791453810065e-08
3877 1.86301742541861e-08
3878 1.87749145424965e-08
3879 1.86702941720185e-08
3880 1.8603751319124e-08
3881 1.86855807050579e-08
3882 1.83424035807844e-08
3883 1.88392016070571e-08
3884 1.87201418927474e-08
3885 1.85095759598353e-08
3886 1.84086047455567e-08
3887 1.88222527648385e-08
3888 1.8523195278064e-08
3889 1.85757435247069e-08
3890 1.87422742384813e-08
3891 1.84633584874305e-08
3892 1.86043784372592e-08
3893 1.85996195174454e-08
3894 1.87718849538054e-08
3895 1.85942197150624e-08
3896 1.8775036347618e-08
3897 1.87095835093576e-08
3898 1.84787136612297e-08
3899 1.82509262909303e-08
3900 1.87818383484695e-08
3901 1.88258969927579e-08
3902 1.85888771125908e-08
3903 1.84707138616291e-08
3904 1.86056817796931e-08
3905 1.84808865619779e-08
3906 1.86448812469209e-08
3907 1.85265273240365e-08
3908 1.86638600390143e-08
3909 1.87866353467125e-08
3910 1.86254637321026e-08
3911 1.84568186745171e-08
3912 1.87421547606093e-08
3913 1.84363482125249e-08
3914 1.83822095323194e-08
3915 1.85663463732055e-08
3916 1.86467606371199e-08
3917 1.8547431269389e-08
3918 1.86782413602593e-08
3919 1.86152682051244e-08
3920 1.84129949974965e-08
3921 1.86318062230795e-08
3922 1.87728860427949e-08
3923 1.87283420125528e-08
3924 1.83082339797469e-08
3925 1.84637523822362e-08
3926 1.83805048615815e-08
3927 1.83669191742286e-08
3928 1.83835412660427e-08
3929 1.85290415481987e-08
3930 1.85926789656454e-08
3931 1.84257754851647e-08
3932 1.82641077144385e-08
3933 1.8747519471618e-08
3934 1.85458104579039e-08
3935 1.84895279045083e-08
3936 1.83260086716874e-08
3937 1.85873748446141e-08
3938 1.86836807887225e-08
3939 1.83990642903531e-08
3940 1.86582412728953e-08
3941 1.84172776901415e-08
3942 1.84749328018263e-08
3943 1.84720867277832e-08
3944 1.84797200230014e-08
3945 1.85308849146582e-08
3946 1.83916440580356e-08
3947 1.84193788774767e-08
3948 1.87682199637162e-08
3949 1.85843069137048e-08
3950 1.88958514101278e-08
3951 1.80888887767461e-08
3952 1.86800413324306e-08
3953 1.80682781988351e-08
3954 1.86649962249374e-08
3955 1.82627310980621e-08
3956 1.85250423493377e-08
3957 1.85099398551936e-08
3958 1.8380441877297e-08
3959 1.83144395129453e-08
3960 1.85630838009931e-08
3961 1.82794598569958e-08
3962 1.85978395675912e-08
3963 1.82150609826515e-08
3964 1.83378688628499e-08
3965 1.87122788447569e-08
3966 1.82224039476653e-08
3967 1.84177097000138e-08
3968 1.84725631721117e-08
3969 1.8404017488427e-08
3970 1.85611537034669e-08
3971 1.85948235236211e-08
3972 1.83189031095488e-08
3973 1.84271280773141e-08
3974 1.82665599950393e-08
3975 1.81853255397657e-08
3976 1.88674012275847e-08
3977 1.83471525626588e-08
3978 1.82929939559484e-08
3979 1.84234745704837e-08
3980 1.82578896483765e-08
3981 1.82978962932623e-08
3982 1.84811756999137e-08
3983 1.8304778142797e-08
3984 1.85518058662959e-08
3985 1.85010573839905e-08
3986 1.84859083398603e-08
3987 1.85467719112653e-08
3988 1.82895245834924e-08
3989 1.85842080034915e-08
3990 1.84882994900271e-08
3991 1.82896776688146e-08
3992 1.83432231006808e-08
3993 1.82996054648443e-08
3994 1.84500277723432e-08
3995 1.84605122001136e-08
3996 1.82331776130695e-08
3997 1.83027311010386e-08
3998 1.84008320551765e-08
3999 1.81986064576867e-08
4000 1.83837227608619e-08
4001 1.84052685177072e-08
4002 1.83741971730988e-08
4003 1.80532008534628e-08
4004 1.84680083105127e-08
4005 1.82912108781519e-08
4006 1.85477868781536e-08
4007 1.84694689102649e-08
4008 1.85242367928273e-08
4009 1.82660877515728e-08
4010 1.83681998717766e-08
4011 1.83198410682595e-08
4012 1.84316662643624e-08
4013 1.81921840095578e-08
4014 1.82874422900436e-08
4015 1.83750694667806e-08
4016 1.83280414448594e-08
4017 1.822759505965e-08
4018 1.82253338130156e-08
4019 1.84561276901363e-08
4020 1.842580180067e-08
4021 1.81850219930224e-08
4022 1.81107677810211e-08
4023 1.82087900713057e-08
4024 1.82190313035013e-08
4025 1.83224452195985e-08
4026 1.84177788299378e-08
4027 1.8551432300451e-08
4028 1.82062717473119e-08
4029 1.82329917115531e-08
4030 1.81914827600549e-08
4031 1.82279130470642e-08
4032 1.8258862745979e-08
4033 1.82515464590693e-08
4034 1.82573802479658e-08
4035 1.84074186734273e-08
4036 1.81109899383136e-08
4037 1.82799790676702e-08
4038 1.8173130241661e-08
4039 1.83458975957507e-08
4040 1.81962363826615e-08
4041 1.82392200872572e-08
4042 1.84579289993492e-08
4043 1.84015924968861e-08
4044 1.83211985882448e-08
4045 1.83582158505002e-08
4046 1.83415605473636e-08
4047 1.83601502270481e-08
4048 1.81710481267361e-08
4049 1.8131239545971e-08
4050 1.86094955878513e-08
4051 1.81560000231951e-08
4052 1.8421562899773e-08
4053 1.81565797156003e-08
4054 1.8378938199004e-08
4055 1.8139933952388e-08
4056 1.83510350850735e-08
4057 1.83802894668794e-08
4058 1.80897672398261e-08
4059 1.81987418612639e-08
4060 1.8311162743645e-08
4061 1.82014525766938e-08
4062 1.80477807346646e-08
4063 1.83430284793618e-08
4064 1.85168586949169e-08
4065 1.80615050420219e-08
4066 1.82232149126271e-08
4067 1.81896753155275e-08
4068 1.80956139818056e-08
4069 1.841148162951e-08
4070 1.83158833950703e-08
4071 1.82540754397609e-08
4072 1.8281622931271e-08
4073 1.81085125415814e-08
4074 1.85146044622275e-08
4075 1.81331585258349e-08
4076 1.82952026667182e-08
4077 1.8225461108079e-08
4078 1.81748349568078e-08
4079 1.81797418962182e-08
4080 1.85911685193219e-08
4081 1.80562621957536e-08
4082 1.85596663914156e-08
4083 1.81517973482093e-08
4084 1.81430519968773e-08
4085 1.84302125924019e-08
4086 1.80813273626956e-08
4087 1.8322247146707e-08
4088 1.82319609861592e-08
4089 1.82695578946346e-08
4090 1.81780626085093e-08
4091 1.80816778749815e-08
4092 1.85135802127645e-08
4093 1.81409313912972e-08
4094 1.81519024803345e-08
4095 1.83949880188106e-08
4096 1.81288415782577e-08
4097 1.81723589439686e-08
4098 1.81643976883139e-08
4099 1.80891736893951e-08
4100 1.85205310253567e-08
4101 1.81096440071693e-08
4102 1.81727913096674e-08
4103 1.82620188359239e-08
4104 1.83081954110431e-08
4105 1.84234215225843e-08
4106 1.78916497147741e-08
4107 1.82090075553365e-08
4108 1.83148837519242e-08
4109 1.82577430366537e-08
4110 1.81358519619756e-08
4111 1.81890436201648e-08
4112 1.79867056204763e-08
4113 1.82416135292574e-08
4114 1.82870802036827e-08
4115 1.80375411863443e-08
4116 1.82383318634294e-08
4117 1.83487029579155e-08
4118 1.7946151404824e-08
4119 1.8115050802181e-08
4120 1.81835224202231e-08
4121 1.83758628844455e-08
4122 1.7960683311613e-08
4123 1.82513445543542e-08
4124 1.79292193585034e-08
4125 1.82856401205189e-08
4126 1.80131646283499e-08
4127 1.81945800736827e-08
4128 1.79605504144753e-08
4129 1.82102909155324e-08
4130 1.82601934003435e-08
4131 1.79265018828145e-08
4132 1.79076482641483e-08
4133 1.83650686799153e-08
4134 1.82467722626978e-08
4135 1.81015787504846e-08
4136 1.80624612456937e-08
4137 1.80957005595506e-08
4138 1.7883939573915e-08
4139 1.80480310550957e-08
4140 1.80464048422646e-08
4141 1.81209059412213e-08
4142 1.80348914130546e-08
4143 1.80360526323975e-08
4144 1.7869232599943e-08
4145 1.81402218426552e-08
4146 1.80019847016988e-08
4147 1.79471774633955e-08
4148 1.82517574719432e-08
4149 1.78659752646526e-08
4150 1.81380172435963e-08
4151 1.80177743621668e-08
4152 1.79691764178314e-08
4153 1.8012126625e-08
4154 1.82062754328083e-08
4155 1.79191629778863e-08
4156 1.80565963987522e-08
4157 1.79223230878378e-08
4158 1.79596428607764e-08
4159 1.79967734050823e-08
4160 1.78243166960579e-08
4161 1.80531826801111e-08
4162 1.817699624862e-08
4163 1.7772206491129e-08
4164 1.79058063949356e-08
4165 1.81801299939943e-08
4166 1.79129346501439e-08
4167 1.80610365162481e-08
4168 1.79480389523867e-08
4169 1.79220309314276e-08
4170 1.79139121456728e-08
4171 1.79538653737321e-08
4172 1.80829263191074e-08
4173 1.77564990774259e-08
4174 1.79797798036674e-08
4175 1.82060177603693e-08
4176 1.76672716946102e-08
4177 1.80313643707297e-08
4178 1.78449749125287e-08
4179 1.8081366402134e-08
4180 1.79810881585363e-08
4181 1.80448754723006e-08
4182 1.79173445169489e-08
4183 1.78684515630412e-08
4184 1.78747775024801e-08
4185 1.81069623401697e-08
4186 1.7747195261042e-08
4187 1.78015871671322e-08
4188 1.80063963197696e-08
4189 1.78546642255473e-08
4190 1.79537527886797e-08
4191 1.79708072960416e-08
4192 1.80307167997329e-08
4193 1.77970189070287e-08
4194 1.79153526280729e-08
4195 1.7957552534309e-08
4196 1.81393205559432e-08
4197 1.76878760107524e-08
4198 1.79241030197064e-08
4199 1.78168702993364e-08
4200 1.79517877127999e-08
4201 1.78391428374836e-08
4202 1.77729335121279e-08
4203 1.78592716059134e-08
4204 1.78805034054541e-08
4205 1.78523820762999e-08
4206 1.79324315038309e-08
4207 1.77599042761045e-08
4208 1.79042652089789e-08
4209 1.77345558599606e-08
4210 1.77767198936696e-08
4211 1.77778000113316e-08
4212 1.78657643004065e-08
4213 1.7763741377963e-08
4214 1.79275398386469e-08
4215 1.78074127238359e-08
4216 1.79244224048913e-08
4217 1.76652254182397e-08
4218 1.78499357412765e-08
4219 1.77681482432801e-08
4220 1.78952317031467e-08
4221 1.79638654561609e-08
4222 1.77035633118772e-08
4223 1.78132190282732e-08
4224 1.78726879929147e-08
4225 1.78122151626159e-08
4226 1.78262549375141e-08
4227 1.77813586702547e-08
4228 1.80336138656534e-08
4229 1.76184156526205e-08
4230 1.79521084588963e-08
4231 1.77099457753238e-08
4232 1.79043129858725e-08
4233 1.77786323239948e-08
4234 1.7461107991501e-08
4235 1.79806291458195e-08
4236 1.78562783074154e-08
4237 1.77799521491329e-08
4238 1.77531658155372e-08
4239 1.76170620058702e-08
4240 1.77410902610919e-08
4241 1.77799346087193e-08
4242 1.76583922676388e-08
4243 1.782482175261e-08
4244 1.78267395242182e-08
4245 1.77405554823151e-08
4246 1.77597317823075e-08
4247 1.78430022138576e-08
4248 1.78945735255232e-08
4249 1.75376478405154e-08
4250 1.77260860388939e-08
4251 1.77718186931131e-08
4252 1.78455228709851e-08
4253 1.74170104603189e-08
4254 1.77756545303165e-08
4255 1.80978686219468e-08
4256 1.77151317530821e-08
4257 1.76004201101199e-08
4258 1.75985868374751e-08
4259 1.77985106067924e-08
4260 1.77953228178707e-08
4261 1.76389404222821e-08
4262 1.79826712473075e-08
4263 1.77215782293239e-08
4264 1.78481967663124e-08
4265 1.74955374067087e-08
4266 1.77513505257298e-08
4267 1.79085557241443e-08
4268 1.75559471614628e-08
4269 1.779282625336e-08
4270 1.7831150285863e-08
4271 1.76627901193882e-08
4272 1.7827434824369e-08
4273 1.76933544755453e-08
4274 1.77202073359251e-08
4275 1.77624564337009e-08
4276 1.74584022077928e-08
4277 1.77878023405187e-08
4278 1.79256596364308e-08
4279 1.75678504433785e-08
4280 1.77543252162238e-08
4281 1.7643135307277e-08
4282 1.76694785519738e-08
4283 1.75165939196864e-08
4284 1.76919796830433e-08
4285 1.80021885851644e-08
4286 1.76813968842948e-08
4287 1.76782346797966e-08
4288 1.77376412171526e-08
4289 1.76042123711806e-08
4290 1.77017829863235e-08
4291 1.77291192051676e-08
4292 1.77619924113248e-08
4293 1.77300962259652e-08
4294 1.75298655996636e-08
4295 1.76978249541992e-08
4296 1.75846112060452e-08
4297 1.78834828434837e-08
4298 1.74023256519495e-08
4299 1.7945406233677e-08
4300 1.76112639584947e-08
4301 1.77437690195292e-08
4302 1.76389970575341e-08
4303 1.76453858896641e-08
4304 1.75219669810156e-08
4305 1.77883945393642e-08
4306 1.76668483842235e-08
4307 1.79118687265722e-08
4308 1.74153330123605e-08
4309 1.78961363195285e-08
4310 1.73505626475112e-08
4311 1.76474169264473e-08
4312 1.77416821285359e-08
4313 1.75976139784595e-08
4314 1.74612009833375e-08
4315 1.77584825251653e-08
4316 1.73219331230934e-08
4317 1.75192653841805e-08
4318 1.75052675811527e-08
4319 1.76222908109258e-08
4320 1.75427479096602e-08
4321 1.77592416942263e-08
4322 1.76290674506197e-08
4323 1.76703764125374e-08
4324 1.76016069979301e-08
4325 1.74655987442707e-08
4326 1.74409693661781e-08
4327 1.77147181175119e-08
4328 1.75659638003145e-08
4329 1.75334653627068e-08
4330 1.76801544604643e-08
4331 1.7468676387411e-08
4332 1.7664022354702e-08
4333 1.75252285439242e-08
4334 1.74239234974616e-08
4335 1.75277695481091e-08
4336 1.77532575931227e-08
4337 1.77268173527922e-08
4338 1.74172692155583e-08
4339 1.75902973850173e-08
4340 1.75267153244008e-08
4341 1.73928345827967e-08
4342 1.76651195307187e-08
4343 1.75042966101824e-08
4344 1.75872378687103e-08
4345 1.74171656367461e-08
4346 1.75738843611883e-08
4347 1.74988289002309e-08
4348 1.74463874830222e-08
4349 1.77053700737284e-08
4350 1.75324340512262e-08
4351 1.76286197921538e-08
4352 1.74541028946651e-08
4353 1.77017263794932e-08
4354 1.75377198152749e-08
4355 1.73905281751763e-08
4356 1.75992696713756e-08
4357 1.74218417271499e-08
4358 1.74460512230024e-08
4359 1.76282102322123e-08
4360 1.74129160301906e-08
4361 1.74533636826402e-08
4362 1.73426375736918e-08
4363 1.75112426138435e-08
4364 1.761613933704e-08
4365 1.75805361678494e-08
4366 1.75454233224404e-08
4367 1.75314256403158e-08
4368 1.76164771505949e-08
4369 1.75346354486239e-08
4370 1.72724316384043e-08
4371 1.76360104050532e-08
4372 1.7380963926672e-08
4373 1.75504285802042e-08
4374 1.73619996262087e-08
4375 1.73252581563421e-08
4376 1.73897930575428e-08
4377 1.74654022392362e-08
4378 1.73730641157555e-08
4379 1.74223895027525e-08
4380 1.74179105418837e-08
4381 1.74306007445502e-08
4382 1.73656564816937e-08
4383 1.74293945410797e-08
4384 1.73760254075228e-08
4385 1.73510544054789e-08
4386 1.73395000916399e-08
4387 1.73251861518287e-08
4388 1.75230273004035e-08
4389 1.76854617207622e-08
4390 1.71980147468664e-08
4391 1.73548418525016e-08
4392 1.74393251364258e-08
4393 1.74376531252296e-08
4394 1.742642887681e-08
4395 1.72402090283796e-08
4396 1.7333183894408e-08
4397 1.73557299673055e-08
4398 1.76600624776646e-08
4399 1.72831698118658e-08
4400 1.72291462106067e-08
4401 1.754358327688e-08
4402 1.75029087592371e-08
4403 1.72065046938341e-08
4404 1.73739750570778e-08
4405 1.72944838041644e-08
4406 1.74948562349231e-08
4407 1.72516245043886e-08
4408 1.72374736463476e-08
4409 1.76028426663866e-08
4410 1.73990551067815e-08
4411 1.74537780068817e-08
4412 1.72884936736617e-08
4413 1.74079328173482e-08
4414 1.72823029720437e-08
4415 1.73950533317591e-08
4416 1.72718317463794e-08
4417 1.75236757405939e-08
4418 1.71708232814538e-08
4419 1.72684149964963e-08
4420 1.71775650289474e-08
4421 1.74633976637306e-08
4422 1.74837383720705e-08
4423 1.74006501965085e-08
4424 1.71756573407178e-08
4425 1.73300498530482e-08
4426 1.74336866597402e-08
4427 1.73711167554824e-08
4428 1.71766399654771e-08
4429 1.7339762240276e-08
4430 1.73454122578409e-08
4431 1.72193233064499e-08
4432 1.76631241891601e-08
4433 1.73204301653351e-08
4434 1.7351365134699e-08
4435 1.72191447638248e-08
4436 1.72329903431745e-08
4437 1.72264939698463e-08
4438 1.74654042766065e-08
4439 1.71754120302792e-08
4440 1.71862331829198e-08
4441 1.73544152205496e-08
4442 1.74102917688268e-08
4443 1.72401020538393e-08
4444 1.73993388561389e-08
4445 1.71492993406019e-08
4446 1.74831670577413e-08
4447 1.70551979107314e-08
4448 1.72593371652319e-08
4449 1.73742824949308e-08
4450 1.70678180121397e-08
4451 1.72508361002688e-08
4452 1.74497366545179e-08
4453 1.71737309238029e-08
4454 1.71507128734483e-08
4455 1.73317417788699e-08
4456 1.72929754210793e-08
4457 1.74794613092555e-08
4458 1.72510743827736e-08
4459 1.72315717100746e-08
4460 1.72352610720017e-08
4461 1.73555136886439e-08
4462 1.71253708038943e-08
4463 1.70465476860837e-08
4464 1.72876721142856e-08
4465 1.70830039505354e-08
4466 1.74177370610984e-08
4467 1.71593791356006e-08
4468 1.71299115976353e-08
4469 1.71115667846333e-08
4470 1.72531169206902e-08
4471 1.75921864676853e-08
4472 1.72244679063072e-08
4473 1.72803301664581e-08
4474 1.73442822237746e-08
4475 1.69933462735861e-08
4476 1.71362651969487e-08
4477 1.71780941184974e-08
4478 1.71329544346177e-08
4479 1.73090231929685e-08
4480 1.71825683659144e-08
4481 1.71646466733399e-08
4482 1.72939177484066e-08
4483 1.70962073162695e-08
4484 1.7387698881155e-08
4485 1.71327903804031e-08
4486 1.71343471614405e-08
4487 1.71645700620671e-08
4488 1.71615234489941e-08
4489 1.73251234562022e-08
4490 1.70570614669296e-08
4491 1.70739892070282e-08
4492 1.73851343070464e-08
4493 1.71321607125297e-08
4494 1.71284027002949e-08
4495 1.70832976205126e-08
4496 1.72941008949046e-08
4497 1.7015336258086e-08
4498 1.71665968805579e-08
4499 1.7355649141404e-08
4500 1.69834826262427e-08
4501 1.71734976865956e-08
4502 1.70827220313807e-08
4503 1.72704838777848e-08
4504 1.71713832669562e-08
4505 1.72589683850033e-08
4506 1.73787014527971e-08
4507 1.70663339252863e-08
4508 1.70201987346186e-08
4509 1.73289492432227e-08
4510 1.70697111075979e-08
4511 1.73484789411926e-08
4512 1.70418730957023e-08
4513 1.70557732324106e-08
4514 1.72583140454208e-08
4515 1.7123083238868e-08
4516 1.71795524066543e-08
4517 1.71782000221166e-08
4518 1.68970187781614e-08
4519 1.74087959772251e-08
4520 1.73569729897682e-08
4521 1.68881569913015e-08
4522 1.70927127846054e-08
4523 1.71420687615775e-08
4524 1.70638383806265e-08
4525 1.73739585742627e-08
4526 1.69847088623554e-08
4527 1.71482857956873e-08
4528 1.70534039973802e-08
4529 1.71772218297006e-08
4530 1.70929754040872e-08
4531 1.69865959006588e-08
4532 1.69513671545163e-08
4533 1.71081794551053e-08
4534 1.70387364140101e-08
4535 1.69740533541018e-08
4536 1.7195048820029e-08
4537 1.70556184694304e-08
4538 1.73019516607198e-08
4539 1.70148308079598e-08
4540 1.71395535292218e-08
4541 1.70882516159487e-08
4542 1.70710775184713e-08
4543 1.70924869580258e-08
4544 1.72202254327125e-08
4545 1.70365569114317e-08
4546 1.7153353457755e-08
4547 1.69771986191503e-08
4548 1.71121681136199e-08
4549 1.71555498023057e-08
4550 1.68776435072182e-08
4551 1.71448351221626e-08
4552 1.69835106437199e-08
4553 1.7229951796538e-08
4554 1.68948278096437e-08
4555 1.72925416003222e-08
4556 1.69726709637885e-08
4557 1.70533621188795e-08
4558 1.70698757957499e-08
4559 1.73264424816466e-08
4560 1.69555920964681e-08
4561 1.69923616185619e-08
4562 1.69897217288106e-08
4563 1.71924424261505e-08
4564 1.70163526858058e-08
4565 1.69457716558963e-08
4566 1.71027094544307e-08
4567 1.71448634317395e-08
4568 1.68593108254012e-08
4569 1.71122860705974e-08
4570 1.70424882283138e-08
4571 1.70133256439664e-08
4572 1.7173464455067e-08
4573 1.68453172157967e-08
4574 1.70857612414865e-08
4575 1.7027269543668e-08
4576 1.71864967692947e-08
4577 1.70909423797827e-08
4578 1.70004278849944e-08
4579 1.70026233132781e-08
4580 1.68522648062996e-08
4581 1.72309698105444e-08
4582 1.68740604126194e-08
4583 1.69434683873204e-08
4584 1.71720309500856e-08
4585 1.69229647292157e-08
4586 1.71389590635229e-08
4587 1.67524404749431e-08
4588 1.70601790817626e-08
4589 1.69451190232817e-08
4590 1.69814513003574e-08
4591 1.71668182112894e-08
4592 1.68332297454388e-08
4593 1.70067405352858e-08
4594 1.7012860490051e-08
4595 1.6975529076424e-08
4596 1.70940628967386e-08
4597 1.69920034334137e-08
4598 1.69123165981722e-08
4599 1.71377307933707e-08
4600 1.69717185619644e-08
4601 1.7061917959027e-08
4602 1.68383663341487e-08
4603 1.7189330991263e-08
4604 1.68608024444517e-08
4605 1.69899833835085e-08
4606 1.70590032108064e-08
4607 1.68550347927621e-08
4608 1.67970200566359e-08
4609 1.72109363230089e-08
4610 1.71048535228535e-08
4611 1.67819714488315e-08
4612 1.68431490251697e-08
4613 1.68723900466627e-08
4614 1.72966678106956e-08
4615 1.68470402148735e-08
4616 1.70564830257458e-08
4617 1.67529469605743e-08
4618 1.69179672485642e-08
4619 1.68404241496223e-08
4620 1.69904164257773e-08
4621 1.69417362871904e-08
4622 1.69026880614265e-08
4623 1.69910243208449e-08
4624 1.69560545253411e-08
4625 1.70611738172788e-08
4626 1.68911272773187e-08
4627 1.66435610379212e-08
4628 1.71333409824115e-08
4629 1.70862996459187e-08
4630 1.70544243621729e-08
4631 1.68577871640929e-08
4632 1.6890514906942e-08
4633 1.7066390314624e-08
4634 1.68853240496425e-08
4635 1.68107111787608e-08
4636 1.70800555534534e-08
4637 1.70502827248065e-08
4638 1.67913590788382e-08
4639 1.72110074503351e-08
4640 1.69049379734054e-08
4641 1.69862048194913e-08
4642 1.67612423181973e-08
4643 1.7135457602957e-08
4644 1.68546336780651e-08
4645 1.70049711131215e-08
4646 1.70990734723597e-08
4647 1.67801898682818e-08
4648 1.68042886194986e-08
4649 1.71518684760619e-08
4650 1.66370963665674e-08
4651 1.72104085403024e-08
4652 1.66594424932942e-08
4653 1.70135763531976e-08
4654 1.7041377676108e-08
4655 1.69131862947092e-08
4656 1.69079864852373e-08
4657 1.70868404861801e-08
4658 1.67742391946613e-08
4659 1.67599980012145e-08
4660 1.71410102212111e-08
4661 1.69608023066203e-08
4662 1.66379218647927e-08
4663 1.70886267489845e-08
4664 1.69307215944769e-08
4665 1.68679052834353e-08
4666 1.6809716977928e-08
4667 1.68883174135326e-08
4668 1.7045683053496e-08
4669 1.70278061819018e-08
4670 1.68395166456059e-08
4671 1.67636265604454e-08
4672 1.71235725086039e-08
4673 1.67466802798444e-08
4674 1.69384089117042e-08
4675 1.68450802149889e-08
4676 1.6798403217777e-08
4677 1.69232421231014e-08
4678 1.68087401407613e-08
4679 1.68573270641348e-08
4680 1.68862284938842e-08
4681 1.69001301275062e-08
4682 1.70586934589156e-08
4683 1.67661522250673e-08
4684 1.6742208785947e-08
4685 1.69790023661909e-08
4686 1.67673233986698e-08
4687 1.688862480137e-08
4688 1.66958896876679e-08
4689 1.69211787621748e-08
4690 1.67031267032103e-08
4691 1.70215193664447e-08
4692 1.68302268915843e-08
4693 1.68801045903333e-08
4694 1.67807145018961e-08
4695 1.67813555252971e-08
4696 1.69632207938086e-08
4697 1.6792416286382e-08
4698 1.69489564896264e-08
4699 1.67713205665443e-08
4700 1.67085313598769e-08
4701 1.67624012740175e-08
4702 1.68920419012508e-08
4703 1.68306103662808e-08
4704 1.68516659125317e-08
4705 1.67026265550652e-08
4706 1.69196929696858e-08
4707 1.67229452896489e-08
4708 1.68164352016831e-08
4709 1.6857510585e-08
4710 1.69914028751439e-08
4711 1.67560440608172e-08
4712 1.66694637411258e-08
4713 1.70660322923455e-08
4714 1.68094539432784e-08
4715 1.67541583544484e-08
4716 1.6722459262708e-08
4717 1.68775221437434e-08
4718 1.69493378912033e-08
4719 1.69392838749172e-08
4720 1.67480506925166e-08
4721 1.66204337987974e-08
4722 1.6913416139408e-08
4723 1.67944560754973e-08
4724 1.66913649399802e-08
4725 1.67979549630104e-08
4726 1.67201536346306e-08
4727 1.6829120109696e-08
4728 1.69292612818284e-08
4729 1.66340598919401e-08
4730 1.67537342089519e-08
4731 1.68274266528767e-08
4732 1.66936269215823e-08
4733 1.69194550753149e-08
4734 1.67920389866905e-08
4735 1.66097788975028e-08
4736 1.69374512192233e-08
4737 1.66176399337692e-08
4738 1.69474884768039e-08
4739 1.65364893843778e-08
4740 1.68816748170597e-08
4741 1.6741099188422e-08
4742 1.67893346019743e-08
4743 1.67624429666668e-08
4744 1.67769036426568e-08
4745 1.67157620115099e-08
4746 1.69051743543758e-08
4747 1.68569169544108e-08
4748 1.66614705864321e-08
4749 1.67788277196657e-08
4750 1.66945917237094e-08
4751 1.68494488280713e-08
4752 1.64794155403936e-08
4753 1.69091695277901e-08
4754 1.66556504131443e-08
4755 1.67606637868079e-08
4756 1.66451817809055e-08
4757 1.68554407788957e-08
4758 1.65527905261476e-08
4759 1.67466051510523e-08
4760 1.67776998061253e-08
4761 1.66101309704292e-08
4762 1.67878307882896e-08
4763 1.65984370189554e-08
4764 1.68441182898382e-08
4765 1.66991599228061e-08
4766 1.66433165443292e-08
4767 1.66803406785876e-08
4768 1.68617088249778e-08
4769 1.67207232318889e-08
4770 1.64245255980244e-08
4771 1.70788779567665e-08
4772 1.67018614686265e-08
4773 1.68086699836034e-08
4774 1.67505836440318e-08
4775 1.65186482500301e-08
4776 1.65735739607875e-08
4777 1.70176624426643e-08
4778 1.66966072405739e-08
4779 1.66467258578828e-08
4780 1.68414976347009e-08
4781 1.64608594285798e-08
4782 1.70673365323948e-08
4783 1.66479530295249e-08
4784 1.660597502251e-08
4785 1.65856338049108e-08
4786 1.68258985426828e-08
4787 1.65255464268932e-08
4788 1.67948721876421e-08
4789 1.66170202716698e-08
4790 1.65006094282605e-08
4791 1.69266163907444e-08
4792 1.65211919133656e-08
4793 1.68462618838072e-08
4794 1.64820335197824e-08
4795 1.69394120601574e-08
4796 1.66414763251854e-08
4797 1.67287926300941e-08
4798 1.67016814026089e-08
4799 1.66911520841362e-08
4800 1.66915927028999e-08
4801 1.65356628512026e-08
4802 1.64749475254355e-08
4803 1.67100426099465e-08
4804 1.65056391022711e-08
4805 1.62340124357785e-08
4806 1.65261518334514e-08
4807 1.64565529758853e-08
4808 1.66737028708397e-08
4809 1.65415391142343e-08
4810 1.68830464319925e-08
4811 1.67582798712429e-08
4812 1.64309788808215e-08
4813 1.65594694172544e-08
4814 1.6541206896703e-08
4815 1.65137422719619e-08
4816 1.65408241612486e-08
4817 1.65454695525424e-08
4818 1.65441774322428e-08
4819 1.63370473986735e-08
4820 1.62631914923939e-08
4821 1.68833098460608e-08
4822 1.64126974976364e-08
4823 1.68533258964043e-08
4824 1.6336250414084e-08
4825 1.65527251920761e-08
4826 1.68664710165078e-08
4827 1.63358024835025e-08
4828 1.65339321883673e-08
4829 1.66053318014781e-08
4830 1.64059378269243e-08
4831 1.65856597538205e-08
4832 1.67914740490938e-08
4833 1.64762233718641e-08
4834 1.65977407623474e-08
4835 1.67707993218302e-08
4836 1.66871399442936e-08
4837 1.60926608414047e-08
4838 1.69686439711603e-08
4839 1.65014769377692e-08
4840 1.64320048807731e-08
4841 1.63952850371629e-08
4842 1.66892347741587e-08
4843 1.64231793387537e-08
4844 1.65606880639935e-08
4845 1.66090266174868e-08
4846 1.63354562104923e-08
4847 1.63725579653828e-08
4848 1.63537502479438e-08
4849 1.64452260130177e-08
4850 1.64834706244399e-08
4851 1.61836487547196e-08
4852 1.67541920720327e-08
4853 1.65754852121114e-08
4854 1.62627669532123e-08
4855 1.65849662776552e-08
4856 1.65584922457995e-08
4857 1.66693606370449e-08
4858 1.62289063301846e-08
4859 1.66014309944118e-08
4860 1.62116953889568e-08
4861 1.69304890752509e-08
4862 1.63327189276474e-08
4863 1.65336785544756e-08
4864 1.64373351432268e-08
4865 1.63479396859678e-08
4866 1.66019332240586e-08
4867 1.64318664960805e-08
4868 1.6751832268791e-08
4869 1.61973099379598e-08
4870 1.64140694784987e-08
4871 1.66021080019174e-08
4872 1.64257232679876e-08
4873 1.63719404046025e-08
4874 1.60263303000119e-08
4875 1.64473427595735e-08
4876 1.67352104857299e-08
4877 1.67575023380939e-08
4878 1.62852443859052e-08
4879 1.63134581239732e-08
4880 1.6393747249821e-08
4881 1.65156275760969e-08
4882 1.63711567315872e-08
4883 1.63545676614141e-08
4884 1.62982177894788e-08
4885 1.60884449323384e-08
4886 1.63404373385023e-08
4887 1.68562201673383e-08
4888 1.64189797140502e-08
4889 1.69162163920489e-08
4890 1.62671058377972e-08
4891 1.63768731495373e-08
4892 1.6332956691123e-08
4893 1.63424134935197e-08
4894 1.61765934061919e-08
4895 1.64101326735611e-08
4896 1.63322084945072e-08
4897 1.636395746063e-08
4898 1.6518867272719e-08
4899 1.62344049356444e-08
4900 1.63673822806132e-08
4901 1.60467764156458e-08
4902 1.66872931594564e-08
4903 1.60230713731058e-08
4904 1.68442670147595e-08
4905 1.65987504873177e-08
4906 1.63891213676992e-08
4907 1.63194316571413e-08
4908 1.67317466713279e-08
4909 1.61987875541003e-08
4910 1.643052147976e-08
4911 1.62130802403571e-08
4912 1.65303760435442e-08
4913 1.64140244783839e-08
4914 1.67046202480536e-08
4915 1.65279916224392e-08
4916 1.62197804521869e-08
4917 1.61923881063508e-08
4918 1.63901130360022e-08
4919 1.68174319019054e-08
4920 1.63753056060534e-08
4921 1.63392103309512e-08
4922 1.6402091670753e-08
4923 1.60639428489207e-08
4924 1.63973309512677e-08
4925 1.64078318152239e-08
4926 1.62428404504311e-08
4927 1.63551606429113e-08
4928 1.62478777889219e-08
4929 1.61325203700269e-08
4930 1.65618068648232e-08
4931 1.63005153487328e-08
4932 1.61977232737143e-08
4933 1.63703753324196e-08
4934 1.6218677475921e-08
4935 1.67095221180191e-08
4936 1.62123631652356e-08
4937 1.63145507052764e-08
4938 1.64393882024494e-08
4939 1.61406524052388e-08
4940 1.6299001466713e-08
4941 1.63392756327707e-08
4942 1.6252764015956e-08
4943 1.64103039889119e-08
4944 1.63212519641576e-08
4945 1.64755184198784e-08
4946 1.64000755950022e-08
4947 1.66285865638716e-08
4948 1.62855368809334e-08
4949 1.61825194672227e-08
4950 1.624459599886e-08
4951 1.67196666550629e-08
4952 1.61663339377505e-08
4953 1.6204591819724e-08
4954 1.6251834608183e-08
4955 1.63570100748522e-08
4956 1.64059062073285e-08
4957 1.61825646550762e-08
4958 1.65174844208305e-08
4959 1.61669264191477e-08
4960 1.61918164292008e-08
4961 1.60721679576814e-08
4962 1.63635260928552e-08
4963 1.61783971341922e-08
4964 1.62525320991369e-08
4965 1.60503752171248e-08
4966 1.60735954215507e-08
4967 1.6401382430864e-08
4968 1.63799394879982e-08
4969 1.60675117207854e-08
4970 1.63730541326013e-08
4971 1.61584292005523e-08
4972 1.62566010112331e-08
4973 1.62902645448604e-08
4974 1.64673515615776e-08
4975 1.63018730271469e-08
4976 1.62586987157987e-08
4977 1.60891444646039e-08
4978 1.61153717327944e-08
4979 1.66274126373045e-08
4980 1.62542838194613e-08
4981 1.60781556476919e-08
4982 1.66697129819759e-08
4983 1.6042320583054e-08
4984 1.6517851259279e-08
4985 1.6018824705144e-08
4986 1.66443634482816e-08
4987 1.58567684445665e-08
4988 1.626763877699e-08
4989 1.64232827256083e-08
4990 1.61982060921106e-08
4991 1.63939429731474e-08
4992 1.60743620686299e-08
4993 1.62636618625256e-08
4994 1.63899525738587e-08
4995 1.59045007155911e-08
4996 1.61617269744396e-08
4997 1.66838010265402e-08
4998 1.59008261564897e-08
4999 1.62453878328472e-08
};
\addlegendentry{Train}
\addplot [semithick, black]
table {%
0 0.00133256160188466
1 0.00030136777786538
2 0.00023241296003107
3 0.00021211568673607
4 0.000164820507052355
5 6.91222085151821e-05
6 2.57609481195686e-05
7 1.83596021088306e-05
8 1.7615773685975e-05
9 1.75720106199151e-05
10 1.75389359355904e-05
11 1.75092318386305e-05
12 1.74845699802972e-05
13 1.74632623384241e-05
14 1.74414381035604e-05
15 1.74238284671446e-05
16 1.74076358234743e-05
17 1.73911266756477e-05
18 1.73736552824266e-05
19 1.73542248376179e-05
20 1.73340085893869e-05
21 1.73070729942992e-05
22 1.72801137523493e-05
23 1.72488653333858e-05
24 1.7208672943525e-05
25 1.71454230439849e-05
26 1.70485818671295e-05
27 1.69107788678957e-05
28 1.67160360433627e-05
29 1.64347075042315e-05
30 1.60132603923557e-05
31 1.50223122545867e-05
32 1.40243882924551e-05
33 1.25001006381353e-05
34 1.03704860521248e-05
35 7.76213710196316e-06
36 5.08496623297106e-06
37 3.03674073620641e-06
38 1.92546531252447e-06
39 1.46145055168745e-06
40 1.22899655252695e-06
41 1.07195489817968e-06
42 9.34778370265121e-07
43 8.3726325783573e-07
44 7.671799835407e-07
45 7.22460640645295e-07
46 6.94718323757115e-07
47 6.73032445774879e-07
48 6.57462919662066e-07
49 6.44565261609387e-07
50 6.33110118997138e-07
51 6.21760875674227e-07
52 6.10679762758082e-07
53 6.11104837844323e-07
54 5.98870826706843e-07
55 5.8825418136621e-07
56 5.77939829327079e-07
57 5.68450559512712e-07
58 5.59075374439999e-07
59 5.50745994587487e-07
60 5.37258813437802e-07
61 5.32155013388547e-07
62 5.24854613104253e-07
63 5.17395903898432e-07
64 5.12967858412594e-07
65 5.06992591908784e-07
66 5.02281864100951e-07
67 4.98972383411456e-07
68 4.95939218581043e-07
69 4.93815775826079e-07
70 4.90533068386867e-07
71 4.84239308207179e-07
72 4.8122842599696e-07
73 4.79342531889415e-07
74 4.76582187047825e-07
75 4.75187306392399e-07
76 4.73690192848153e-07
77 4.68853784241219e-07
78 4.66373819563159e-07
79 4.64142516420907e-07
80 4.62263955114395e-07
81 4.60606798924346e-07
82 4.58996311181181e-07
83 4.58342981346505e-07
84 4.58120467783374e-07
85 4.65825365836281e-07
86 4.64217606577222e-07
87 4.63523349480965e-07
88 4.62698665160133e-07
89 4.62585319382924e-07
90 4.61883843172473e-07
91 4.61301937093594e-07
92 4.60412678648936e-07
93 4.59764123661444e-07
94 4.59283398868138e-07
95 4.58567910754937e-07
96 4.57741862192051e-07
97 4.56680680827048e-07
98 4.56049178865214e-07
99 4.55497854545683e-07
100 4.55271305099814e-07
101 4.54979044661741e-07
102 4.54112807801721e-07
103 4.54055793852604e-07
104 4.5315630359255e-07
105 4.52874104439616e-07
106 4.52033020792442e-07
107 4.51483373353767e-07
108 4.50871624479987e-07
109 4.49900227295075e-07
110 4.49771277999389e-07
111 4.48846662948199e-07
112 4.48276495035316e-07
113 4.4790053266297e-07
114 4.47697203753705e-07
115 4.46863197112179e-07
116 4.46151432242914e-07
117 4.45746280774983e-07
118 4.44973721869246e-07
119 4.43883436673786e-07
120 4.43398192828681e-07
121 4.43018961959751e-07
122 4.42561258751084e-07
123 4.42092158436935e-07
124 4.41637439507758e-07
125 4.40986411831545e-07
126 4.40023768533138e-07
127 4.39356455217421e-07
128 4.3879816757908e-07
129 4.38172548911098e-07
130 4.37924882135121e-07
131 4.37317567048012e-07
132 4.36760188904373e-07
133 4.36461306208002e-07
134 4.35788194863562e-07
135 4.35335351767208e-07
136 4.34434952012452e-07
137 4.33886839346087e-07
138 4.3369109903324e-07
139 4.33006874800412e-07
140 4.32787004456259e-07
141 4.31688050639423e-07
142 4.31136726319892e-07
143 4.30691187602861e-07
144 4.30451052579883e-07
145 4.29387711164964e-07
146 4.28825842391234e-07
147 4.285840304874e-07
148 4.27375255185325e-07
149 4.26793434371575e-07
150 4.26156702815206e-07
151 4.25948201154824e-07
152 4.24875707949468e-07
153 4.2457872950763e-07
154 4.23529513682297e-07
155 4.23294920892658e-07
156 4.2210777451146e-07
157 4.21908936232285e-07
158 4.21416274321018e-07
159 4.20253996935571e-07
160 4.19572046439498e-07
161 4.18897883491809e-07
162 4.18668975044056e-07
163 4.1775766135288e-07
164 4.17067866465004e-07
165 4.16604308384194e-07
166 4.15805516240653e-07
167 4.14924329561472e-07
168 4.14315110219832e-07
169 4.12629219681548e-07
170 4.10739914968872e-07
171 4.09381073041004e-07
172 4.0888329522204e-07
173 4.08155244713271e-07
174 4.07432054316814e-07
175 4.07222984222244e-07
176 4.05907911726899e-07
177 4.05264472647104e-07
178 4.04436576673106e-07
179 4.03843472440713e-07
180 4.03131309667515e-07
181 4.02146099531819e-07
182 4.01481656808755e-07
183 4.00547662593453e-07
184 3.99897061242882e-07
185 3.98835254600272e-07
186 3.98252183231307e-07
187 3.92586684938578e-07
188 3.88281335972351e-07
189 3.85565101623797e-07
190 3.8366934518308e-07
191 3.82144008881369e-07
192 3.80797985144454e-07
193 3.79203385136861e-07
194 3.77977301013743e-07
195 3.76635853172047e-07
196 3.75073057057307e-07
197 3.73542945908412e-07
198 3.72313991192641e-07
199 3.71016255940049e-07
200 3.69900391206102e-07
201 3.68861691413258e-07
202 3.67654223509817e-07
203 3.66103591886713e-07
204 3.64607046776655e-07
205 3.63511190926147e-07
206 3.62145470944597e-07
207 3.61067236553936e-07
208 3.58844260972546e-07
209 3.56920168087527e-07
210 3.55555954456577e-07
211 3.53996369995002e-07
212 3.5195714076508e-07
213 3.50319709241376e-07
214 3.48693049545545e-07
215 3.47305757486538e-07
216 3.45902748222215e-07
217 3.4393303849356e-07
218 3.42525453334019e-07
219 3.41245311119565e-07
220 3.39938878823887e-07
221 3.38037409619574e-07
222 3.36553853230726e-07
223 3.35054664901691e-07
224 3.33779865968609e-07
225 3.323551993617e-07
226 3.3093061801992e-07
227 3.29342270788402e-07
228 3.27603004279808e-07
229 3.25684624158384e-07
230 3.24334507695312e-07
231 3.23412450597971e-07
232 3.22019076293145e-07
233 3.20650201501849e-07
234 3.19064298537342e-07
235 3.1752850304656e-07
236 3.15827833219373e-07
237 3.14271460410964e-07
238 3.12441585492707e-07
239 3.10554440829947e-07
240 3.09255653974105e-07
241 3.07844373992339e-07
242 3.0641822945654e-07
243 3.05013401202814e-07
244 3.03063785622726e-07
245 3.01655660450706e-07
246 3.00239292982951e-07
247 2.98695908895752e-07
248 2.97306286256571e-07
249 2.96099756269541e-07
250 2.94674691758701e-07
251 2.93047321520135e-07
252 2.91117430606391e-07
253 2.90058295604467e-07
254 2.8864155865449e-07
255 2.86760723611224e-07
256 2.85584945913797e-07
257 2.83873703210702e-07
258 2.82062728729215e-07
259 2.80392669083085e-07
260 2.78851274515546e-07
261 2.77488027222716e-07
262 2.7625037546386e-07
263 2.74254801979623e-07
264 2.73395045269353e-07
265 2.71682267793949e-07
266 2.71689600594982e-07
267 2.69225978399845e-07
268 2.68532403424615e-07
269 2.67146816668173e-07
270 2.66054854591857e-07
271 2.64342020273034e-07
272 2.63327876837138e-07
273 2.62136808260038e-07
274 2.62368530457024e-07
275 2.59910819977449e-07
276 2.58532480756912e-07
277 2.55601293019936e-07
278 2.528272489144e-07
279 2.51589170829902e-07
280 2.50331993356667e-07
281 2.48700786187328e-07
282 2.43313820647018e-07
283 2.41492585928427e-07
284 2.37914093759173e-07
285 2.35681056892645e-07
286 2.34424518907872e-07
287 2.33797720738949e-07
288 2.32217971074533e-07
289 2.30271609780175e-07
290 2.28722925044167e-07
291 2.27628333959728e-07
292 2.26936748504158e-07
293 2.24680306359915e-07
294 2.2345456329731e-07
295 2.22211426148533e-07
296 2.21164029312604e-07
297 2.20318014498844e-07
298 2.19247183963489e-07
299 2.18740822788277e-07
300 2.18081979141971e-07
301 2.16916092199426e-07
302 2.17054790141447e-07
303 2.16265874541932e-07
304 2.15546108961462e-07
305 2.14519715768802e-07
306 2.13572448615196e-07
307 2.13210242350215e-07
308 2.12560024692721e-07
309 2.1349964640649e-07
310 2.13435129126083e-07
311 2.13624801403967e-07
312 2.14021468991632e-07
313 2.1531219829285e-07
314 2.14572054346718e-07
315 2.15685801663312e-07
316 2.14823273836373e-07
317 2.15429992067584e-07
318 2.14802184927976e-07
319 2.14671274534339e-07
320 2.14006021792557e-07
321 2.1332023436571e-07
322 2.12667558230351e-07
323 2.06336196129087e-07
324 2.05991057100618e-07
325 2.05480574777539e-07
326 2.04613755272476e-07
327 2.04632129907623e-07
328 2.03235742901597e-07
329 2.02566695861606e-07
330 2.02145002958787e-07
331 2.0140309686667e-07
332 2.00291310648026e-07
333 1.9937800743719e-07
334 1.981687347552e-07
335 1.97802350498932e-07
336 1.96745929770259e-07
337 1.95367277910918e-07
338 1.94328677594058e-07
339 1.93927562008867e-07
340 1.93379307233954e-07
341 1.92255143360853e-07
342 1.9216896873786e-07
343 1.90896329854695e-07
344 1.90227055441028e-07
345 1.89184746091087e-07
346 1.88040075954632e-07
347 1.87118246230966e-07
348 1.86157990356151e-07
349 1.85253583140366e-07
350 1.84485287491043e-07
351 1.83624067062738e-07
352 1.82673076665196e-07
353 1.81840476898287e-07
354 1.79887180706828e-07
355 1.78287962171453e-07
356 1.77397112111066e-07
357 1.7631241178151e-07
358 1.74875268044161e-07
359 1.74280756937151e-07
360 1.73089020449879e-07
361 1.72054441804903e-07
362 1.69684881257126e-07
363 1.68149952628482e-07
364 1.66405996537833e-07
365 1.64755022069585e-07
366 1.63405346143009e-07
367 1.62186609031778e-07
368 1.61832701905951e-07
369 1.60871934440365e-07
370 1.58993259447016e-07
371 1.5611315973274e-07
372 1.57032175707172e-07
373 1.55663229861602e-07
374 1.54390107809377e-07
375 1.54064835555801e-07
376 1.54516243355829e-07
377 1.53049001028194e-07
378 1.51615836330166e-07
379 1.50627940342929e-07
380 1.49505055446753e-07
381 1.47692290397572e-07
382 1.4637056722222e-07
383 1.45199010148644e-07
384 1.43968719612531e-07
385 1.43027790500128e-07
386 1.42289280802288e-07
387 1.4106868206909e-07
388 1.40886342592239e-07
389 1.36631086888883e-07
390 1.36087692226283e-07
391 1.34802988327465e-07
392 1.34111189709074e-07
393 1.32823743115296e-07
394 1.3202912896304e-07
395 1.3092018491534e-07
396 1.28017731526597e-07
397 1.2755454292801e-07
398 1.26355203633466e-07
399 1.25210448231883e-07
400 1.23780580452149e-07
401 1.22692910053956e-07
402 1.21896718496828e-07
403 1.21793661378433e-07
404 1.2063965471043e-07
405 1.19967111800179e-07
406 1.19404901965936e-07
407 1.18979244234652e-07
408 1.18416338068528e-07
409 1.1855817660944e-07
410 1.19478315241395e-07
411 1.19035213685947e-07
412 1.1862814375263e-07
413 1.18088578915376e-07
414 1.18110790481296e-07
415 1.17823233836134e-07
416 1.17553440759366e-07
417 1.17862214210618e-07
418 1.17376615094145e-07
419 1.17857240411468e-07
420 1.1569814972745e-07
421 1.16128603622201e-07
422 1.16353014334436e-07
423 1.16850436882032e-07
424 1.15633554287342e-07
425 1.15513408616152e-07
426 1.16073742617573e-07
427 1.1565784063805e-07
428 1.16187813148372e-07
429 1.15219066287864e-07
430 1.1680249656365e-07
431 1.15123441446485e-07
432 1.14270541473616e-07
433 1.17124450582651e-07
434 1.16420132201256e-07
435 1.15331644678918e-07
436 1.16465486144079e-07
437 1.15002045220081e-07
438 1.17490330353576e-07
439 1.17108456265669e-07
440 1.16508793723824e-07
441 1.14486070401654e-07
442 1.17195597226782e-07
443 1.15692216695606e-07
444 1.18330639509168e-07
445 1.15444343862237e-07
446 1.16693428253711e-07
447 1.16144725836875e-07
448 1.17824917822418e-07
449 1.15265429201372e-07
450 1.16046116716007e-07
451 1.18328834730619e-07
452 1.15766660258032e-07
453 1.19420718647234e-07
454 1.1556829804249e-07
455 1.18606479304617e-07
456 1.19212828053605e-07
457 1.14455254163204e-07
458 1.14363082559521e-07
459 1.15443413051253e-07
460 1.16601761135371e-07
461 1.16426825513827e-07
462 1.14678755380737e-07
463 1.17161555124312e-07
464 1.15675668155291e-07
465 1.14669489903463e-07
466 1.14201775147649e-07
467 1.15594893657089e-07
468 1.15421642021829e-07
469 1.16520304516143e-07
470 1.16176316566907e-07
471 1.15526837873858e-07
472 1.15126454147685e-07
473 1.16589184528948e-07
474 1.14825986941014e-07
475 1.15569719127961e-07
476 1.17693161882926e-07
477 1.16161956498217e-07
478 1.16201242406078e-07
479 1.1628846863232e-07
480 1.17696892232289e-07
481 1.16236073210985e-07
482 1.16789415471885e-07
483 1.16791461834964e-07
484 1.16851175846477e-07
485 1.17407857658236e-07
486 1.15858696858595e-07
487 1.16195273847097e-07
488 1.15871841899207e-07
489 1.15281025614422e-07
490 1.16132945038316e-07
491 1.15568134617661e-07
492 1.14421254693298e-07
493 1.14643235349376e-07
494 1.16378615189205e-07
495 1.15966045655114e-07
496 1.16624306656377e-07
497 1.17111561337424e-07
498 1.19772678885965e-07
499 1.15619386065191e-07
500 1.15932287769738e-07
501 1.14945876816819e-07
502 1.15820881774198e-07
503 1.16116147808043e-07
504 1.20383774060429e-07
505 1.16049363896309e-07
506 1.15723274518587e-07
507 1.15171602033115e-07
508 1.14075767498889e-07
509 1.15192719363222e-07
510 1.1435028568485e-07
511 1.16386772219812e-07
512 1.15894273733375e-07
513 1.1573496294659e-07
514 1.14600993583736e-07
515 1.14214877555696e-07
516 1.15388210986112e-07
517 1.16057009336146e-07
518 1.15755561580499e-07
519 1.16076073197746e-07
520 1.15442212234029e-07
521 1.183965139262e-07
522 1.15259140898161e-07
523 1.1569139957146e-07
524 1.1869190075231e-07
525 1.19157270717096e-07
526 1.18104324542401e-07
527 1.15054461957698e-07
528 1.15179609849747e-07
529 1.15180057491671e-07
530 1.18548157956866e-07
531 1.14593710520694e-07
532 1.14986562493868e-07
533 1.15218036000897e-07
534 1.15270559319924e-07
535 1.1825017764977e-07
536 1.15034382019985e-07
537 1.1796038279499e-07
538 1.14653943228404e-07
539 1.14965700959146e-07
540 1.14444020482551e-07
541 1.18312158292611e-07
542 1.14468832634884e-07
543 1.14916197446746e-07
544 1.14509184356848e-07
545 1.15160823099814e-07
546 1.15235295083949e-07
547 1.15061837391295e-07
548 1.1515263054207e-07
549 1.14724144850697e-07
550 1.14818696772545e-07
551 1.14859808775236e-07
552 1.14882858781584e-07
553 1.15148644397323e-07
554 1.14893623504031e-07
555 1.14973424558684e-07
556 1.15342388085082e-07
557 1.15081967066999e-07
558 1.13926304834422e-07
559 1.15080482032681e-07
560 1.14956272057043e-07
561 1.1505644437193e-07
562 1.14914847415548e-07
563 1.15030829306306e-07
564 1.14886503865819e-07
565 1.14828274888623e-07
566 1.14402688211612e-07
567 1.14673298412526e-07
568 1.1467264471321e-07
569 1.14930394090607e-07
570 1.14809417084416e-07
571 1.1437661129321e-07
572 1.14500565473463e-07
573 1.1460662818763e-07
574 1.15179034310131e-07
575 1.14170383369583e-07
576 1.14446933707768e-07
577 1.14641473203392e-07
578 1.14736423029171e-07
579 1.1443001568523e-07
580 1.14381862204027e-07
581 1.14611268031695e-07
582 1.15347575047053e-07
583 1.15348754547995e-07
584 1.13197764051165e-07
585 1.14688575081345e-07
586 1.13685587166401e-07
587 1.14579876253629e-07
588 1.1462335436363e-07
589 1.15091893349017e-07
590 1.14860995381605e-07
591 1.14579080445765e-07
592 1.15045736492903e-07
593 1.14697201070157e-07
594 1.14780242199686e-07
595 1.14690699604125e-07
596 1.14978696785784e-07
597 1.14042521204283e-07
598 1.14231141878918e-07
599 1.13827894665519e-07
600 1.13707471882663e-07
601 1.13897002051999e-07
602 1.13900028964053e-07
603 1.12443451882882e-07
604 1.13522077072048e-07
605 1.1370998009852e-07
606 1.13797433698437e-07
607 1.12186313572238e-07
608 1.1388984688665e-07
609 1.14046464716466e-07
610 1.14230161329942e-07
611 1.13672165014123e-07
612 1.13812234303623e-07
613 1.13485270958336e-07
614 1.13531633871844e-07
615 1.13818465763416e-07
616 1.13474534657598e-07
617 1.13525587153163e-07
618 1.11757508136634e-07
619 1.11981421468954e-07
620 1.13475771001958e-07
621 1.13677153024128e-07
622 1.12791404660584e-07
623 1.13478954233415e-07
624 1.13163899584379e-07
625 1.13334920115449e-07
626 1.13029862802705e-07
627 1.11084247578219e-07
628 1.12637287941197e-07
629 1.11577783457051e-07
630 1.12706075583446e-07
631 1.12935389040558e-07
632 1.11629823607018e-07
633 1.13129331680284e-07
634 1.1254760323709e-07
635 1.13101741305854e-07
636 1.12813580699367e-07
637 1.12984260169924e-07
638 1.1250794074158e-07
639 1.1109666786524e-07
640 1.12643327554451e-07
641 1.12563910192875e-07
642 1.11340163755358e-07
643 1.12683153474791e-07
644 1.12495634141396e-07
645 1.11995383633712e-07
646 1.10804684538834e-07
647 1.11095978638787e-07
648 1.11980000383483e-07
649 1.12296248744315e-07
650 1.10578710632581e-07
651 1.12284304520927e-07
652 1.12019364451044e-07
653 1.10822774956887e-07
654 1.12199330715157e-07
655 1.10617840221039e-07
656 1.11633731592065e-07
657 1.10701812161551e-07
658 1.10273006725947e-07
659 1.10398758579322e-07
660 1.11525260138023e-07
661 1.11805590563563e-07
662 1.10340373282725e-07
663 1.11409349301539e-07
664 1.11568262184392e-07
665 1.10194761759885e-07
666 1.10216923587814e-07
667 1.11463194230055e-07
668 1.10279408715996e-07
669 1.11457488571887e-07
670 1.10189091628854e-07
671 1.11524151691356e-07
672 1.09997749575541e-07
673 1.11312161266142e-07
674 1.10060113911459e-07
675 1.11411210923507e-07
676 1.10077550630194e-07
677 1.11270821889775e-07
678 1.0984574316808e-07
679 1.09700359018916e-07
680 1.11318719575593e-07
681 1.0958991225607e-07
682 1.100536834997e-07
683 1.11490358278843e-07
684 1.10141108677908e-07
685 1.11287619120048e-07
686 1.09694816785577e-07
687 1.09401071313187e-07
688 1.09369132417214e-07
689 1.10581012791044e-07
690 1.09210226639789e-07
691 1.10574603695568e-07
692 1.09228849964893e-07
693 1.10800442598702e-07
694 1.09350203558733e-07
695 1.08575385127097e-07
696 1.10406610076552e-07
697 1.08834655065948e-07
698 1.08558090516908e-07
699 1.10205341741221e-07
700 1.08560435307936e-07
701 1.10036694422888e-07
702 1.08609192750464e-07
703 1.09453488050804e-07
704 1.08358470640724e-07
705 1.08614095495341e-07
706 1.08265425069476e-07
707 1.08184735836403e-07
708 1.07847867525379e-07
709 1.07440129681891e-07
710 1.08944348653495e-07
711 1.07619975153739e-07
712 1.06947830147419e-07
713 1.06963447876751e-07
714 1.06877436678587e-07
715 1.0662022731367e-07
716 1.084051888256e-07
717 1.07111766567414e-07
718 1.06665567045638e-07
719 1.06217825646127e-07
720 1.06564606028314e-07
721 1.08466146286901e-07
722 1.06458514892438e-07
723 1.08425155076475e-07
724 1.06708732516836e-07
725 1.08157131251119e-07
726 1.06384447917662e-07
727 1.06635610563899e-07
728 1.06286371703845e-07
729 1.06213690287404e-07
730 1.06158140056323e-07
731 1.05803195538101e-07
732 1.05820312512606e-07
733 1.05817143492004e-07
734 1.05797127503138e-07
735 1.05591517751691e-07
736 1.05510629566652e-07
737 1.05573313646801e-07
738 1.05147506701542e-07
739 1.04932112776623e-07
740 1.04805451428547e-07
741 1.04660948352375e-07
742 1.04362989361562e-07
743 1.05679205830711e-07
744 1.04279010315622e-07
745 1.03595830580616e-07
746 1.03584184785177e-07
747 1.02994327733086e-07
748 1.02856482442348e-07
749 1.02614606589668e-07
750 1.02556150238797e-07
751 1.02135814472604e-07
752 1.02942188107136e-07
753 1.01837947852346e-07
754 1.02346199071235e-07
755 1.02205547136691e-07
756 1.01516818062919e-07
757 1.01506188343592e-07
758 1.01206715896751e-07
759 1.01212947356544e-07
760 1.01753641956748e-07
761 1.00955112714018e-07
762 1.00103029865295e-07
763 1.00101516409268e-07
764 1.00325131313639e-07
765 1.00530421320855e-07
766 9.99280942437508e-08
767 9.98178890654344e-08
768 9.95918867374712e-08
769 9.93424720263647e-08
770 9.89801378636912e-08
771 9.85332277991802e-08
772 9.84717374308275e-08
773 9.8128538184028e-08
774 9.78318936972755e-08
775 9.70960982726865e-08
776 9.66009139347079e-08
777 9.65365813954122e-08
778 9.61730179938058e-08
779 9.83793810860334e-08
780 9.83386243547102e-08
781 9.63059534342392e-08
782 9.71609495081793e-08
783 9.45296250165484e-08
784 9.45978868571729e-08
785 9.42193096875599e-08
786 9.47142098084441e-08
787 9.3487251717761e-08
788 9.24947443081692e-08
789 9.26609828866276e-08
790 9.18719393894207e-08
791 9.2772161508492e-08
792 9.24137353308652e-08
793 9.23344032344176e-08
794 9.15165330184209e-08
795 9.13055302476096e-08
796 9.15647007104781e-08
797 9.34274098085552e-08
798 9.40401818638747e-08
799 9.16422351338042e-08
800 9.00559768979292e-08
801 8.84599273831554e-08
802 8.78991031072474e-08
803 8.72866934287231e-08
804 8.57807549436984e-08
805 8.35903364304613e-08
806 8.25482260324861e-08
807 8.08226303661286e-08
808 7.87003671121056e-08
809 7.70427490692782e-08
810 7.66438645882772e-08
811 7.53548832221895e-08
812 7.53626636651461e-08
813 7.34056868623156e-08
814 7.23070456842834e-08
815 7.25697901771127e-08
816 7.12211800646401e-08
817 7.12108203515527e-08
818 7.09927121533838e-08
819 6.98637450113893e-08
820 6.8546377463008e-08
821 6.89194550318462e-08
822 6.79205882647693e-08
823 6.8045721945964e-08
824 6.72005171509227e-08
825 6.69040574052815e-08
826 6.64496795366176e-08
827 6.60379626538088e-08
828 6.62031069964542e-08
829 6.5851807562467e-08
830 6.48860449814492e-08
831 6.48514486556451e-08
832 6.48090505706023e-08
833 6.3535019023675e-08
834 6.33374384051422e-08
835 6.3279912865255e-08
836 6.26652720825405e-08
837 6.27700771360651e-08
838 6.2652645738126e-08
839 6.23933900101292e-08
840 6.22489437773766e-08
841 6.16988486967784e-08
842 6.17640623090665e-08
843 6.17018827142601e-08
844 6.18171185351457e-08
845 6.11473538469909e-08
846 6.07771895033693e-08
847 6.04150045546703e-08
848 6.00189906663218e-08
849 5.91894178114671e-08
850 5.95558056204482e-08
851 5.87596247214606e-08
852 5.80307641939726e-08
853 5.74882861315018e-08
854 5.70424170120987e-08
855 5.65763329518632e-08
856 5.62824027383613e-08
857 5.60607666955093e-08
858 5.56282984121026e-08
859 5.51444934160372e-08
860 5.52952492682834e-08
861 5.52446408619289e-08
862 5.50781216190899e-08
863 5.45987965949735e-08
864 5.65867033230916e-08
865 5.41163096556829e-08
866 5.34630792969892e-08
867 5.35892574760055e-08
868 5.39726592307943e-08
869 5.3831051616271e-08
870 5.38442996855792e-08
871 5.38441717878868e-08
872 5.36468718337346e-08
873 5.30069392823407e-08
874 5.26315488968976e-08
875 5.2772811898194e-08
876 5.28679393596576e-08
877 5.20445837537409e-08
878 5.19246761143677e-08
879 5.20747285293055e-08
880 5.24656691425207e-08
881 5.16060971733623e-08
882 5.18555403061782e-08
883 5.11895876798008e-08
884 5.14217255442873e-08
885 5.14134477214157e-08
886 5.17792848597765e-08
887 5.13169737814678e-08
888 5.03520283245962e-08
889 5.06718009773977e-08
890 5.03711490296155e-08
891 4.99462267100625e-08
892 5.01873458347291e-08
893 5.04112094290576e-08
894 5.06157320501188e-08
895 4.98535754900331e-08
896 4.96334351396399e-08
897 4.99661219066638e-08
898 5.0032340936923e-08
899 5.00638854816771e-08
900 4.93181389060737e-08
901 4.9461771567394e-08
902 4.92869425272602e-08
903 4.98486443234469e-08
904 4.93673688595209e-08
905 4.94451057875267e-08
906 4.86883706685148e-08
907 4.91756608766991e-08
908 4.86045692582593e-08
909 4.9409774049991e-08
910 4.96286318707462e-08
911 4.89394622604777e-08
912 4.9211056563081e-08
913 4.91420060200198e-08
914 4.91940959079784e-08
915 4.85720050846794e-08
916 4.8815401498814e-08
917 4.91870117969029e-08
918 4.89153642035944e-08
919 4.83472071266533e-08
920 4.83670170581263e-08
921 4.82533941692509e-08
922 4.77640895724107e-08
923 4.79427875177407e-08
924 4.81122164330827e-08
925 4.85745133005366e-08
926 4.84363553709954e-08
927 4.81694151233114e-08
928 4.82591389072695e-08
929 4.78475961074309e-08
930 4.78854573771059e-08
931 4.72474148693891e-08
932 4.76839154828212e-08
933 4.84280775481238e-08
934 4.74222829893733e-08
935 4.73700048075898e-08
936 4.76523034365073e-08
937 4.76015564743193e-08
938 4.67605687504147e-08
939 4.74184176368908e-08
940 4.86426010581908e-08
941 4.7132228786495e-08
942 4.78500545852967e-08
943 4.82992135175664e-08
944 4.69733087982149e-08
945 4.74384229676161e-08
946 4.84449742543802e-08
947 4.74615546863788e-08
948 4.70484842196583e-08
949 4.68743195369825e-08
950 4.79747335191405e-08
951 4.7086061272239e-08
952 4.6963950950385e-08
953 4.78102535339531e-08
954 4.67376572999001e-08
955 4.69503369515678e-08
956 4.70946588393417e-08
957 4.72253027794522e-08
958 4.71287435743761e-08
959 4.7143743131528e-08
960 4.68164031985907e-08
961 4.7151154092262e-08
962 4.75378776343405e-08
963 4.70546410724637e-08
964 4.70494576632063e-08
965 4.69356642440744e-08
966 4.68053400481949e-08
967 4.74097276992325e-08
968 4.67044856122811e-08
969 4.69166074879013e-08
970 4.66300171808598e-08
971 4.67641498858029e-08
972 4.6575916456959e-08
973 4.66699070500454e-08
974 4.64876244166135e-08
975 4.62599807349307e-08
976 4.69512677625517e-08
977 4.64455993665069e-08
978 4.67340406373751e-08
979 4.71752699127137e-08
980 4.649060514339e-08
981 4.67681786631147e-08
982 4.66812899446722e-08
983 4.65078002775954e-08
984 4.61831994869044e-08
985 4.65363605428593e-08
986 4.64826435120358e-08
987 4.65099816437942e-08
988 4.64998883842327e-08
989 4.65191440923718e-08
990 4.64205172079346e-08
991 4.64787888176943e-08
992 4.58770017530696e-08
993 4.61699514175962e-08
994 4.62497027342579e-08
995 4.6286920962757e-08
996 4.63383145188345e-08
997 4.639592532385e-08
998 4.56291644468365e-08
999 4.80763908683457e-08
1000 4.58446862694473e-08
1001 4.63452849430723e-08
1002 4.57340796344852e-08
1003 4.59652902407015e-08
1004 4.58019435711776e-08
1005 4.62354563524059e-08
1006 4.61506708404613e-08
1007 4.59772060423802e-08
1008 4.60850877459507e-08
1009 4.59780551409494e-08
1010 4.6084707605587e-08
1011 4.63370106729144e-08
1012 4.5989253294465e-08
1013 4.60558702286562e-08
1014 4.61853097988296e-08
1015 4.60303901661518e-08
1016 4.59626576798655e-08
1017 4.58337972020217e-08
1018 4.57224977878923e-08
1019 4.58329800778756e-08
1020 4.59034019684168e-08
1021 4.58208013753847e-08
1022 4.55192470383281e-08
1023 4.55769644247539e-08
1024 4.64126088672856e-08
1025 4.56128717019055e-08
1026 4.59767477423156e-08
1027 4.56520368175006e-08
1028 4.59202169622586e-08
1029 4.56384299241108e-08
1030 4.59763356275289e-08
1031 4.59815296949273e-08
1032 4.58914719558834e-08
1033 4.56562894157742e-08
1034 4.89080704824119e-08
1035 4.56799682524434e-08
1036 4.56717899055548e-08
1037 4.56342590382519e-08
1038 4.55386093278776e-08
1039 4.56914186486301e-08
1040 4.5657330360882e-08
1041 4.55274467014988e-08
1042 4.56478872479238e-08
1043 4.5869409603938e-08
1044 4.55626789630514e-08
1045 4.56633237888582e-08
1046 4.58758826482608e-08
1047 4.55730599924209e-08
1048 4.5822556415942e-08
1049 4.56035280649303e-08
1050 4.54679458528062e-08
1051 4.59671305463871e-08
1052 4.53819417600698e-08
1053 4.55898643281216e-08
1054 4.5570050843935e-08
1055 4.5383952596012e-08
1056 4.5442597240708e-08
1057 4.53895871999066e-08
1058 4.53898536534325e-08
1059 4.57490294536456e-08
1060 4.81129163176774e-08
1061 4.8099206395591e-08
1062 4.57237447903935e-08
1063 4.51993678041163e-08
1064 4.55508448737874e-08
1065 4.54722446363576e-08
1066 4.5265746706491e-08
1067 4.52965736030819e-08
1068 4.54173942898706e-08
1069 4.58047288987018e-08
1070 4.51183552740986e-08
1071 4.55256348175226e-08
1072 4.51654500466248e-08
1073 4.53042794390512e-08
1074 4.53243913511869e-08
1075 4.5094679990143e-08
1076 4.52382025173392e-08
1077 4.51228707731843e-08
1078 4.52776980353065e-08
1079 4.52317472365849e-08
1080 4.51426274139521e-08
1081 4.50177566335697e-08
1082 4.53555948354278e-08
1083 4.53045245762951e-08
1084 4.54191351195732e-08
1085 4.49693793314054e-08
1086 4.47992682950371e-08
1087 4.50598669488045e-08
1088 4.52668693640135e-08
1089 4.5097589662646e-08
1090 4.48833077371091e-08
1091 4.71441801153105e-08
1092 4.50542749774741e-08
1093 4.51959181191341e-08
1094 4.53949944301257e-08
1095 4.52167761011424e-08
1096 4.49320012307908e-08
1097 4.4837307200396e-08
1098 4.4684767885883e-08
1099 4.47638299760911e-08
1100 4.48950814302407e-08
1101 4.47187566976481e-08
1102 4.45455015096741e-08
1103 4.47382006996122e-08
1104 4.4709157265288e-08
1105 4.45763035372693e-08
1106 4.45273791171985e-08
1107 4.44220695783315e-08
1108 4.46625527672495e-08
1109 4.47960992744356e-08
1110 4.47598296204887e-08
1111 4.44464767213049e-08
1112 4.47577122031362e-08
1113 4.45182735120397e-08
1114 4.48586270351825e-08
1115 4.44706635960301e-08
1116 4.44176144753783e-08
1117 4.45373444790675e-08
1118 4.44994263659737e-08
1119 4.44258567711131e-08
1120 4.45171899343677e-08
1121 4.42297860558938e-08
1122 4.41043397358953e-08
1123 4.45709567031827e-08
1124 4.42045404724922e-08
1125 4.42471055350779e-08
1126 4.43610126410476e-08
1127 4.42938414835226e-08
1128 4.43644943004529e-08
1129 4.4378406727219e-08
1130 4.40771259491157e-08
1131 4.43355716583937e-08
1132 4.45240679880499e-08
1133 4.43603198618803e-08
1134 4.45510472957267e-08
1135 4.44245600306203e-08
1136 4.45735111043177e-08
1137 4.42719212401244e-08
1138 4.42514789256165e-08
1139 4.4282067790391e-08
1140 4.45093704115607e-08
1141 4.43898962032563e-08
1142 4.41814336227253e-08
1143 4.42467644745648e-08
1144 4.42055814176001e-08
1145 4.4221255990351e-08
1146 4.43234391411806e-08
1147 4.41822010088799e-08
1148 4.42094574282237e-08
1149 4.42243823783883e-08
1150 4.42392042998563e-08
1151 4.41707257436974e-08
1152 4.41422898234123e-08
1153 4.4250047182004e-08
1154 4.40803162859993e-08
1155 4.39982841271558e-08
1156 4.40846648075421e-08
1157 4.38748237741038e-08
1158 4.43499530433655e-08
1159 4.40778364918515e-08
1160 4.41730350075886e-08
1161 4.41844463239249e-08
1162 4.40701626303053e-08
1163 4.3745775002435e-08
1164 4.39940386343096e-08
1165 4.41889547175833e-08
1166 4.40716263483409e-08
1167 4.39621885561792e-08
1168 4.39537046759142e-08
1169 4.41291803099375e-08
1170 4.37732019520354e-08
1171 4.38909957267697e-08
1172 4.39379377326077e-08
1173 4.37244551676486e-08
1174 4.39271090613147e-08
1175 4.42287060309354e-08
1176 4.38794387491726e-08
1177 4.37600533587101e-08
1178 4.42018297519553e-08
1179 4.44272743038709e-08
1180 4.38076206421556e-08
1181 4.36674305603901e-08
1182 4.4296513124209e-08
1183 4.39241070182561e-08
1184 4.38782841172269e-08
1185 4.38691927229229e-08
1186 4.38686100778796e-08
1187 4.38856417872557e-08
1188 4.39224123738313e-08
1189 4.42242189535591e-08
1190 4.37864287050616e-08
1191 4.38459544227499e-08
1192 4.38364864407959e-08
1193 4.38734133467733e-08
1194 4.36660805291922e-08
1195 4.37437144285013e-08
1196 4.39523724082846e-08
1197 4.36833573758122e-08
1198 4.33770104280029e-08
1199 4.3842668162597e-08
1200 4.38566338800683e-08
1201 4.38730829444012e-08
1202 4.34521929548737e-08
1203 4.37689280374798e-08
1204 4.3669334814922e-08
1205 4.36617924037819e-08
1206 4.36858051955369e-08
1207 4.37732516900269e-08
1208 4.36800817738003e-08
1209 4.34038049945684e-08
1210 4.38239666777918e-08
1211 4.38090133059177e-08
1212 4.36200942033338e-08
1213 4.33957332290902e-08
1214 4.38345431064135e-08
1215 4.37372840167427e-08
1216 4.38011511505465e-08
1217 4.36064517828072e-08
1218 4.38994192109021e-08
1219 4.36449241192349e-08
1220 4.35797957720752e-08
1221 4.37550156107136e-08
1222 4.36105374035378e-08
1223 4.37068266023743e-08
1224 4.38617568931932e-08
1225 4.37160672106529e-08
1226 4.36521325752892e-08
1227 4.34218598854841e-08
1228 4.36312959095631e-08
1229 4.34967653006879e-08
1230 4.35341895865804e-08
1231 4.3597108145832e-08
1232 4.35391953601538e-08
1233 4.36841496309626e-08
1234 4.36882245935521e-08
1235 4.35199964954336e-08
1236 4.38701732718982e-08
1237 4.35627249828485e-08
1238 4.29544755320421e-08
1239 4.29865210094249e-08
1240 4.30913580373726e-08
1241 4.29946034330442e-08
1242 4.34363620627209e-08
1243 4.3053194787035e-08
1244 4.31471605111255e-08
1245 4.29341007190942e-08
1246 4.29157296366611e-08
1247 4.27252722090543e-08
1248 4.29878888041912e-08
1249 4.25940456239005e-08
1250 4.26678354870091e-08
1251 4.25825383842948e-08
1252 4.26391579821939e-08
1253 4.26275796883147e-08
1254 4.26554542798385e-08
1255 4.27079633880112e-08
1256 4.28534470131581e-08
1257 4.27053947760214e-08
1258 4.27003676861659e-08
1259 4.28362234572432e-08
1260 4.26535606834477e-08
1261 4.25548911664464e-08
1262 4.27225010923848e-08
1263 4.27349071685512e-08
1264 4.23956691975036e-08
1265 4.25962483063813e-08
1266 4.24686810163166e-08
1267 4.22715480397073e-08
1268 4.25143760196534e-08
1269 4.25474446785756e-08
1270 4.28881605785136e-08
1271 4.23484927125628e-08
1272 4.25173212192931e-08
1273 4.26752571058842e-08
1274 4.26945554465874e-08
1275 4.24970885148923e-08
1276 4.27992965512658e-08
1277 4.23239896463201e-08
1278 4.24386001895982e-08
1279 4.23173602825955e-08
1280 4.2339365791122e-08
1281 4.26044159951289e-08
1282 4.20511376830746e-08
1283 4.22370298736041e-08
1284 4.22098835883844e-08
1285 4.23530401860717e-08
1286 4.23983337327627e-08
1287 4.28112656436497e-08
1288 4.28134541152758e-08
1289 4.23402859439648e-08
1290 4.20803090150912e-08
1291 4.21690131702235e-08
1292 4.25935020587076e-08
1293 4.20916883570044e-08
1294 4.20112762355984e-08
1295 4.20092156616647e-08
1296 4.22945234390681e-08
1297 4.25735642295422e-08
1298 4.25644550716697e-08
1299 4.22768593466571e-08
1300 4.22120152165917e-08
1301 4.2490157170505e-08
1302 4.25329886866166e-08
1303 4.23020836137766e-08
1304 4.22517061338112e-08
1305 4.2597999794225e-08
1306 4.25238546597484e-08
1307 4.26368309547343e-08
1308 4.24690540512529e-08
1309 4.23737702703875e-08
1310 4.25423465344466e-08
1311 4.29427231551927e-08
1312 4.24932586895466e-08
1313 4.26664179542513e-08
1314 4.29654036793181e-08
1315 4.26085087212869e-08
1316 4.25888337929337e-08
1317 4.25970405615317e-08
1318 4.24126262998925e-08
1319 4.24298036705295e-08
1320 4.27037711858702e-08
1321 4.32276792139419e-08
1322 4.25710702245397e-08
1323 4.24224779749238e-08
1324 4.24180619518211e-08
1325 4.21136903128172e-08
1326 4.19070786961129e-08
1327 4.26384545448855e-08
1328 4.23230943624731e-08
1329 4.20404759893245e-08
1330 4.2346112394398e-08
1331 4.2330206895258e-08
1332 4.21922088378324e-08
1333 4.21958006313616e-08
1334 4.27233750599498e-08
1335 4.19991650346674e-08
1336 4.21422825525042e-08
1337 4.21655776960961e-08
1338 4.20705248416198e-08
1339 4.23479313838016e-08
1340 4.23663486515125e-08
1341 4.18833430160248e-08
1342 4.16364507316302e-08
1343 4.23404316052256e-08
1344 4.18364329846099e-08
1345 4.21500132574693e-08
1346 4.22947508127436e-08
1347 4.16260768076882e-08
1348 4.16655687729417e-08
1349 4.23500097213036e-08
1350 4.21272048356514e-08
1351 4.17330632274115e-08
1352 4.1966362829271e-08
1353 4.13284588773877e-08
1354 4.18732248874676e-08
1355 4.18714982686197e-08
1356 4.17604368863067e-08
1357 4.24974722079696e-08
1358 4.16429557503761e-08
1359 4.09291693870273e-08
1360 4.18951415781521e-08
1361 4.14082172994767e-08
1362 4.14513579016784e-08
1363 4.14427852035715e-08
1364 4.13961060985457e-08
1365 4.13616767502845e-08
1366 4.13839522650505e-08
1367 4.16750900456009e-08
1368 4.15652010588019e-08
1369 4.16205274689219e-08
1370 4.12718286213476e-08
1371 4.19339585278067e-08
1372 4.15382288565525e-08
1373 4.14764969036696e-08
1374 4.1585529686472e-08
1375 4.16594225782774e-08
1376 4.09880307472577e-08
1377 4.1444980780625e-08
1378 4.11917788767369e-08
1379 4.12960297069276e-08
1380 4.14320844299709e-08
1381 4.10753280277731e-08
1382 4.14583851693351e-08
1383 4.15362784167428e-08
1384 4.11797849153572e-08
1385 4.13409679822507e-08
1386 4.07817744019212e-08
1387 4.0794983391379e-08
1388 4.11015754764321e-08
1389 4.10826430652378e-08
1390 4.11637230968154e-08
1391 4.10887928126158e-08
1392 4.10138341067068e-08
1393 4.04882172233556e-08
1394 4.11631404517721e-08
1395 4.09868725625984e-08
1396 4.10306313369802e-08
1397 4.12353848844305e-08
1398 4.07452276363074e-08
1399 4.08097768911375e-08
1400 4.05649842605271e-08
1401 4.10345144530311e-08
1402 4.06091196225589e-08
1403 4.06339886183105e-08
1404 4.06986409018373e-08
1405 4.0315487837006e-08
1406 4.08942248952826e-08
1407 4.05194739983017e-08
1408 4.06343829695288e-08
1409 4.06558982035676e-08
1410 4.05968840766491e-08
1411 4.09049079053148e-08
1412 4.03727291597988e-08
1413 4.05497644351271e-08
1414 4.06506899253145e-08
1415 4.03728215303545e-08
1416 4.02156246082086e-08
1417 4.08427247577947e-08
1418 4.04370936735177e-08
1419 4.06118481066642e-08
1420 4.0353359764822e-08
1421 4.08094713577611e-08
1422 3.99299793230057e-08
1423 4.036125744733e-08
1424 3.98902280096536e-08
1425 3.96603283547847e-08
1426 4.00951449819331e-08
1427 4.0294395375895e-08
1428 3.98774417931236e-08
1429 4.01739441713289e-08
1430 4.0104293219656e-08
1431 4.00514430509702e-08
1432 3.98881425667241e-08
1433 4.00021278323948e-08
1434 4.04275972698542e-08
1435 3.95582056000876e-08
1436 3.96327237695004e-08
1437 4.00220443452781e-08
1438 3.99653785621013e-08
1439 4.0217003061116e-08
1440 3.99052453303739e-08
1441 3.976472839895e-08
1442 3.99740187617681e-08
1443 3.96777437572382e-08
1444 3.98635968679173e-08
1445 3.95086914295462e-08
1446 3.98341128970969e-08
1447 4.01825843709958e-08
1448 3.95378130235713e-08
1449 3.96038011274413e-08
1450 3.92249788205845e-08
1451 3.97780546279591e-08
1452 3.97508976845984e-08
1453 3.95300219224737e-08
1454 3.96920967205006e-08
1455 3.9853269129253e-08
1456 3.96347843434341e-08
1457 3.98903168274956e-08
1458 3.97354327219546e-08
1459 3.97965678189394e-08
1460 3.99332265033081e-08
1461 3.98137203205806e-08
1462 3.94548891335944e-08
1463 3.98518196220721e-08
1464 3.97075758939991e-08
1465 3.95062080826847e-08
1466 3.96754309406333e-08
1467 3.96907324784479e-08
1468 4.0317434724102e-08
1469 3.94905050882244e-08
1470 3.95534094366212e-08
1471 3.93241172957914e-08
1472 3.95966850419427e-08
1473 3.97307893251764e-08
1474 4.02005433386421e-08
1475 3.97676060970298e-08
1476 3.98362622888726e-08
1477 3.94220371902065e-08
1478 3.93266610387855e-08
1479 3.92599410758976e-08
1480 3.9469693291494e-08
1481 3.89633960651281e-08
1482 3.95467338876188e-08
1483 3.89736456440914e-08
1484 3.94690928828823e-08
1485 3.8743088737192e-08
1486 3.97222699177746e-08
1487 3.9023291265039e-08
1488 3.92253731718029e-08
1489 3.90431225127941e-08
1490 3.84961857946564e-08
1491 3.85612750619657e-08
1492 3.83776175283401e-08
1493 3.9189888667579e-08
1494 3.89654957189123e-08
1495 3.94392216662709e-08
1496 3.87225718156969e-08
1497 3.82619518290994e-08
1498 3.96244033140647e-08
1499 3.89238472564557e-08
1500 3.81312901254205e-08
1501 3.8294121651461e-08
1502 3.83300928774588e-08
1503 3.88895422531732e-08
1504 3.83545248894279e-08
1505 3.78979549964242e-08
1506 3.8499873511455e-08
1507 3.82277036692358e-08
1508 3.86633232096756e-08
1509 3.76708904070711e-08
1510 3.85758411880488e-08
1511 3.81040443642178e-08
1512 3.78772000431127e-08
1513 3.77796673944886e-08
1514 3.77310627186489e-08
1515 3.75774895644554e-08
1516 3.72905155643366e-08
1517 3.74677959769087e-08
1518 3.81045168751371e-08
1519 3.73171467060729e-08
1520 3.82089240247296e-08
1521 3.75607456248872e-08
1522 3.76010085290091e-08
1523 3.75270339247891e-08
1524 3.96414954195734e-08
1525 3.72593440545188e-08
1526 3.76640620913804e-08
1527 3.76848063865509e-08
1528 3.83178537788353e-08
1529 3.70275934358233e-08
1530 3.85357203924741e-08
1531 3.76175499638975e-08
1532 3.91301782087794e-08
1533 3.7841878963718e-08
1534 3.94676291648466e-08
1535 3.8596688511916e-08
1536 3.91420194034708e-08
1537 3.83904144030112e-08
1538 3.88996781452988e-08
1539 3.76346847019704e-08
1540 3.8801744040029e-08
1541 3.81669167381915e-08
1542 3.68624633040326e-08
1543 3.80868989680039e-08
1544 3.65452130779431e-08
1545 3.73539101872211e-08
1546 3.67789461108714e-08
1547 3.67915085064396e-08
1548 3.53269520303456e-08
1549 3.78959370550547e-08
1550 3.71807047372386e-08
1551 3.64370258409963e-08
1552 3.67311088211864e-08
1553 3.68319916788096e-08
1554 3.6820395621362e-08
1555 3.80256643950361e-08
1556 3.70810013805567e-08
1557 3.68303183506669e-08
1558 3.75612998482211e-08
1559 3.66963739395487e-08
1560 3.63977150641404e-08
1561 3.65929437862178e-08
1562 3.68681938311965e-08
1563 3.67861261452163e-08
1564 3.5781866358775e-08
1565 3.66699310916374e-08
1566 3.7440539557565e-08
1567 3.53558675669774e-08
1568 3.65129650958806e-08
1569 3.58496805574759e-08
1570 3.54621185749693e-08
1571 3.50803688320411e-08
1572 3.40487709138415e-08
1573 3.64410652764491e-08
1574 3.46640121051678e-08
1575 3.54903093580106e-08
1576 3.58323433147234e-08
1577 3.48868098853927e-08
1578 3.4720198271998e-08
1579 3.61398022619142e-08
1580 3.41645645107747e-08
1581 3.48144482131829e-08
1582 3.51246995933252e-08
1583 3.41950006088609e-08
1584 3.6033089401144e-08
1585 3.5193732372818e-08
1586 3.53848683687374e-08
1587 3.37577148457058e-08
1588 3.53748887960137e-08
1589 3.59246996595175e-08
1590 3.45920767585994e-08
1591 3.42100321404359e-08
1592 3.50699274065391e-08
1593 3.35998713296703e-08
1594 3.42556703003538e-08
1595 3.42465398261993e-08
1596 3.48499291646931e-08
1597 3.42826638188853e-08
1598 3.46156596719993e-08
1599 3.41877637310972e-08
1600 3.35535546014398e-08
1601 3.42917161333389e-08
1602 3.52936133651838e-08
1603 3.41528298974936e-08
1604 3.27679643419287e-08
1605 3.49791875464689e-08
1606 3.3735823024017e-08
1607 3.36859251603983e-08
1608 3.33362955018401e-08
1609 3.4247339186777e-08
1610 3.36834347081094e-08
1611 3.36675789469609e-08
1612 3.40841417312276e-08
1613 3.43674564362573e-08
1614 3.39740964250268e-08
1615 3.5311916946057e-08
1616 3.41372796697215e-08
1617 3.48746986844617e-08
1618 3.34879821650702e-08
1619 3.33049143819153e-08
1620 3.48724995546945e-08
1621 3.40376509200269e-08
1622 3.28701617036131e-08
1623 3.29248095454204e-08
1624 3.35674350537829e-08
1625 3.386827529539e-08
1626 3.45794113343345e-08
1627 3.59008893724422e-08
1628 3.31638716488669e-08
1629 3.38178374192921e-08
1630 3.35871241929908e-08
1631 3.34902416909699e-08
1632 3.40278951682649e-08
1633 3.29082077143994e-08
1634 3.2602791577574e-08
1635 3.09341423587739e-08
1636 3.4073796228995e-08
1637 3.27190257110033e-08
1638 3.22662252472128e-08
1639 3.27891491735954e-08
1640 3.27960094637092e-08
1641 3.32310570172467e-08
1642 3.30007559057321e-08
1643 3.40575887491923e-08
1644 3.18992974257526e-08
1645 3.35720926614158e-08
1646 3.27556826107411e-08
1647 3.2767651703125e-08
1648 3.36052181637569e-08
1649 3.3519341968713e-08
1650 3.2720521403462e-08
1651 3.40058647907426e-08
1652 3.23799440593575e-08
1653 3.25977609350048e-08
1654 3.37018626339614e-08
1655 3.47656197163815e-08
1656 3.31128831021488e-08
1657 3.14928279010473e-08
1658 3.23492557186e-08
1659 3.34824576952997e-08
1660 3.25910107790151e-08
1661 3.20266835274197e-08
1662 3.30558762584587e-08
1663 3.31680567455805e-08
1664 3.29351834693625e-08
1665 3.42390507057644e-08
1666 3.18884083583271e-08
1667 3.36894068198035e-08
1668 3.30054668040702e-08
1669 3.25096571884842e-08
1670 3.4501571377632e-08
1671 3.20404112130745e-08
1672 3.36265770783939e-08
1673 3.22049338308261e-08
1674 3.2955320250494e-08
1675 3.17030455221357e-08
1676 3.51331976844449e-08
1677 3.36173542336837e-08
1678 3.25771161158173e-08
1679 3.31106377871038e-08
1680 3.2529513305235e-08
1681 3.27183222736949e-08
1682 3.0398069839066e-08
1683 3.04091258840344e-08
1684 3.17843280583929e-08
1685 3.12078825004392e-08
1686 3.17812229866377e-08
1687 3.04864151701167e-08
1688 3.18079464989296e-08
1689 3.16794626087358e-08
1690 3.01370484123709e-08
1691 3.02602778390337e-08
1692 3.11587342594066e-08
1693 3.16495558649876e-08
1694 3.09634948791881e-08
1695 3.04715790377941e-08
1696 3.16782333698029e-08
1697 2.97771478585673e-08
1698 3.15115542548483e-08
1699 2.98107849516782e-08
1700 3.03904954535028e-08
1701 3.18003365862296e-08
1702 3.16476125306053e-08
1703 3.01644433875481e-08
1704 3.0950090490478e-08
1705 2.94178921222965e-08
1706 3.36217453877907e-08
1707 2.90828658933151e-08
1708 2.89154122867785e-08
1709 3.36296643865808e-08
1710 2.96940818600433e-08
1711 3.19493516087732e-08
1712 3.1715142512212e-08
1713 3.11460937041375e-08
1714 3.00153146781668e-08
1715 3.13733394818883e-08
1716 2.94939912492964e-08
1717 2.99039264461953e-08
1718 3.01306677386037e-08
1719 3.1735847727532e-08
1720 3.03587626149238e-08
1721 3.01209297504101e-08
1722 3.09274419407757e-08
1723 3.02893603532084e-08
1724 3.26420597218657e-08
1725 3.21841220340957e-08
1726 2.95741937605953e-08
1727 2.97639921598147e-08
1728 3.00747444725857e-08
1729 3.11836103605856e-08
1730 3.10746202103473e-08
1731 3.33994343293398e-08
1732 3.10063938968597e-08
1733 3.26414451023993e-08
1734 3.04109377680106e-08
1735 3.17958779305627e-08
1736 2.90973396488425e-08
1737 2.95182758236479e-08
1738 3.09210328452991e-08
1739 2.97532398718658e-08
1740 3.20047632840215e-08
1741 2.88534121040129e-08
1742 3.00426741262072e-08
1743 2.89255623897589e-08
1744 2.92335453622172e-08
1745 2.95702982100465e-08
1746 2.90741262176653e-08
1747 3.07069356608736e-08
1748 2.85631962526622e-08
1749 2.86657932946355e-08
1750 2.85803256616646e-08
1751 3.01248412881705e-08
1752 2.96956788048419e-08
1753 3.01579241579475e-08
1754 3.03361424869308e-08
1755 2.79302732053566e-08
1756 2.95203737010752e-08
1757 2.90194730467874e-08
1758 2.92120514444605e-08
1759 2.90975723515885e-08
1760 2.90270598668485e-08
1761 2.79225176313957e-08
1762 2.82278787011592e-08
1763 3.01894367282785e-08
1764 3.00061380187344e-08
1765 2.80959238097012e-08
1766 2.97713373953457e-08
1767 2.86348473821363e-08
1768 2.87755668182399e-08
1769 2.81562577697514e-08
1770 2.92515807132077e-08
1771 2.91958883735788e-08
1772 2.96695290558091e-08
1773 2.92094775034002e-08
1774 3.03884419849965e-08
1775 2.8581382593984e-08
1776 2.87383770114502e-08
1777 2.83178120952243e-08
1778 2.81648695477088e-08
1779 2.92133037760323e-08
1780 2.79061911356848e-08
1781 2.7791267953603e-08
1782 2.93770803239113e-08
1783 2.8090362036437e-08
1784 2.8873795798745e-08
1785 2.78581726576022e-08
1786 2.88487598254505e-08
1787 2.81636811649832e-08
1788 2.92137833923789e-08
1789 2.95263511418398e-08
1790 2.82646031024569e-08
1791 2.85941048616678e-08
1792 2.86108896574433e-08
1793 2.94736679506968e-08
1794 2.75133249516557e-08
1795 2.78762666283683e-08
1796 2.78624430194441e-08
1797 2.80551741838053e-08
1798 2.94355846364169e-08
1799 2.75858980103294e-08
1800 2.78806222553385e-08
1801 2.79796967816992e-08
1802 2.8679989938496e-08
1803 2.80112679718059e-08
1804 2.73333160549782e-08
1805 2.71837059528934e-08
1806 2.85599650595714e-08
1807 2.76788227893121e-08
1808 2.84285075480284e-08
1809 2.74028764124523e-08
1810 2.76586948899649e-08
1811 2.71475730784232e-08
1812 2.82591141598232e-08
1813 2.80009384567848e-08
1814 2.77610574528353e-08
1815 2.74334226446626e-08
1816 2.76940248511437e-08
1817 2.7320455231461e-08
1818 2.79391532131967e-08
1819 2.77710494600569e-08
1820 2.71592188738623e-08
1821 2.72503157816573e-08
1822 2.82713035204551e-08
1823 2.67985154067674e-08
1824 2.7960641801883e-08
1825 2.69100475236428e-08
1826 2.68824997817774e-08
1827 2.71248108418831e-08
1828 2.79816489978657e-08
1829 2.81499321630463e-08
1830 2.87675199217574e-08
1831 2.72715681148838e-08
1832 2.80643952521586e-08
1833 2.71391442652202e-08
1834 2.77089959865862e-08
1835 2.82630558956498e-08
1836 2.69240505446078e-08
1837 2.69816577969095e-08
1838 2.66260986592215e-08
1839 2.66941295734568e-08
1840 2.73902767133904e-08
1841 2.70828763859754e-08
1842 2.72502980180889e-08
1843 2.75910831959436e-08
1844 2.70862354767587e-08
1845 2.80564709242981e-08
1846 2.7329237539675e-08
1847 2.71129412254822e-08
1848 2.71575437693627e-08
1849 2.64285908713191e-08
1850 2.72597660000429e-08
1851 2.73838320907771e-08
1852 2.78916481022407e-08
1853 2.67967621425669e-08
1854 2.67222866057182e-08
1855 2.7134634095205e-08
1856 2.75959397555425e-08
1857 2.70833027116169e-08
1858 2.70618549791379e-08
1859 2.6634101146783e-08
1860 2.72498663633769e-08
1861 2.66844946139599e-08
1862 2.70839191074401e-08
1863 2.72098610309968e-08
1864 2.70556839154779e-08
1865 2.67304312018268e-08
1866 2.69101541050532e-08
1867 2.66207322852097e-08
1868 2.65628443685273e-08
1869 2.68410929038509e-08
1870 2.69258322305177e-08
1871 2.72690137137488e-08
1872 2.72835745107614e-08
1873 2.73783840043507e-08
1874 2.77386718039452e-08
1875 2.71515947503076e-08
1876 2.74471432248902e-08
1877 2.7440950844948e-08
1878 2.75054912179939e-08
1879 2.65441642000042e-08
1880 2.80247043349391e-08
1881 2.68617750265321e-08
1882 2.70755506903697e-08
1883 2.69800022323352e-08
1884 2.7387073942009e-08
1885 2.76120015740844e-08
1886 2.71712803368018e-08
1887 2.67774495910089e-08
1888 2.64186414966616e-08
1889 2.64048267695216e-08
1890 2.69907811656367e-08
1891 2.73238462966674e-08
1892 2.67659245878349e-08
1893 2.59106762712236e-08
1894 2.61372878895827e-08
1895 2.72297473458138e-08
1896 2.67431428113696e-08
1897 2.64131454486005e-08
1898 2.62870862854925e-08
1899 2.69399968999551e-08
1900 2.67273900789178e-08
1901 2.59546304448577e-08
1902 2.64260915372461e-08
1903 2.63999648808522e-08
1904 2.67957922517326e-08
1905 2.75827396478689e-08
1906 2.76548846045443e-08
1907 2.67043063217898e-08
1908 2.62830894826038e-08
1909 2.67004196530252e-08
1910 2.67498929673593e-08
1911 2.64404658167905e-08
1912 2.62005901419116e-08
1913 2.65500670337815e-08
1914 2.66492214962e-08
1915 2.66000537152422e-08
1916 2.67541029330687e-08
1917 2.65054875825399e-08
1918 2.65990252046322e-08
1919 2.67327369130044e-08
1920 2.58639776262726e-08
1921 2.61830415126951e-08
1922 2.60912322858076e-08
1923 2.64947992434372e-08
1924 2.63931010380247e-08
1925 2.66331738885128e-08
1926 2.6314323164911e-08
1927 2.61371866372428e-08
1928 2.63502553110584e-08
1929 2.69916533568448e-08
1930 2.69233915162204e-08
1931 2.61120156608285e-08
1932 2.62502624082117e-08
1933 2.65556217016183e-08
1934 2.72215316954316e-08
1935 2.67502748840798e-08
1936 2.64026720486754e-08
1937 2.6078120995976e-08
1938 2.66252033753744e-08
1939 2.64219490730966e-08
1940 2.59353267750839e-08
1941 2.60481520797384e-08
1942 2.59080081832508e-08
1943 2.60140691210609e-08
1944 2.58259120755611e-08
1945 2.61541703849844e-08
1946 2.6128457619734e-08
1947 2.66571777984836e-08
1948 2.6392987351187e-08
1949 2.64422137519205e-08
1950 2.5926798485898e-08
1951 2.68514437351541e-08
1952 2.54178775804803e-08
1953 2.63662283117583e-08
1954 2.6060700264452e-08
1955 2.6072822123524e-08
1956 2.62383448301762e-08
1957 2.54861429738185e-08
1958 2.59018353432339e-08
1959 2.57235797107569e-08
1960 2.63828692226298e-08
1961 2.65765027762654e-08
1962 2.61087720332398e-08
1963 2.62746517876167e-08
1964 2.56854093549919e-08
1965 2.60450843114768e-08
1966 2.61146535507351e-08
1967 2.57748151710757e-08
1968 2.58701291500074e-08
1969 2.65582098535333e-08
1970 2.60753427738791e-08
1971 2.57071359754946e-08
1972 2.57477825726937e-08
1973 2.55561705131413e-08
1974 2.59254182566337e-08
1975 2.57074113108047e-08
1976 2.58033097111365e-08
1977 2.59317545214799e-08
1978 2.53449385922977e-08
1979 2.54608707450643e-08
1980 2.62579540333263e-08
1981 2.54962220225252e-08
1982 2.58376289252737e-08
1983 2.55333212351161e-08
1984 2.62713317766838e-08
1985 2.59980126315895e-08
1986 2.54650061037864e-08
1987 2.60651749073304e-08
1988 2.58904346850386e-08
1989 2.58531365204817e-08
1990 2.55048426822668e-08
1991 2.52717011761661e-08
1992 2.54404302069133e-08
1993 2.61630201947582e-08
1994 2.60121240103217e-08
1995 2.59812544811666e-08
1996 2.55778278557273e-08
1997 2.53850434006608e-08
1998 2.53515075598898e-08
1999 2.58163446176241e-08
2000 2.5380893831084e-08
2001 2.56873065040963e-08
2002 2.59114045775277e-08
2003 2.56915395624446e-08
2004 2.50423521919174e-08
2005 2.58768952932087e-08
2006 2.58049368540014e-08
2007 2.61945487523008e-08
2008 2.52344243278912e-08
2009 2.61361954301265e-08
2010 2.55612384592041e-08
2011 2.54860008652713e-08
2012 2.57952805782224e-08
2013 2.56106709173309e-08
2014 2.56431835765625e-08
2015 2.56481182958623e-08
2016 2.54911967090266e-08
2017 2.53469512045967e-08
2018 2.50877381091641e-08
2019 2.49333762525339e-08
2020 2.48615208420233e-08
2021 2.50832226100783e-08
2022 2.50291343206754e-08
2023 2.50626914777285e-08
2024 2.49436453714225e-08
2025 2.46389628699717e-08
2026 2.58650381113057e-08
2027 2.57128647263016e-08
2028 2.53485445966817e-08
2029 2.53489300661158e-08
2030 2.46601246089995e-08
2031 2.50225440368013e-08
2032 2.56416949895311e-08
2033 2.55344545507796e-08
2034 2.51297791464822e-08
2035 2.51762877212514e-08
2036 2.5336241549212e-08
2037 2.49809506414067e-08
2038 2.48312126416295e-08
2039 2.54696850277014e-08
2040 2.52780321119417e-08
2041 2.51453720068184e-08
2042 2.47883491510947e-08
2043 2.51073064561069e-08
2044 2.48437341809904e-08
2045 2.5580574103401e-08
2046 2.51731240297204e-08
2047 2.46545504012374e-08
2048 2.48732643370886e-08
2049 2.48145468617622e-08
2050 2.49958311826504e-08
2051 2.46025191330546e-08
2052 2.49496832083196e-08
2053 2.5143529924776e-08
2054 2.50062797135797e-08
2055 2.55960550532564e-08
2056 2.49941418672961e-08
2057 2.46777851486968e-08
2058 2.51522873639942e-08
2059 2.52185952120954e-08
2060 2.46297116035521e-08
2061 2.48088340981667e-08
2062 2.52894682972737e-08
2063 2.49175222677422e-08
2064 2.48298057670127e-08
2065 2.49312321898287e-08
2066 2.52084273455466e-08
2067 2.45913440721779e-08
2068 2.47109124273948e-08
2069 2.46355451594127e-08
2070 2.51796112848979e-08
2071 2.50382257149795e-08
2072 2.48032581140478e-08
2073 2.50660132650182e-08
2074 2.46442404261416e-08
2075 2.5268848347082e-08
2076 2.47833735755876e-08
2077 2.47915572515467e-08
2078 2.47726976709828e-08
2079 2.47969929034753e-08
2080 2.46738487419407e-08
2081 2.48238158917502e-08
2082 2.49734934953949e-08
2083 2.48401050839675e-08
2084 2.47397355934709e-08
2085 2.49679743546949e-08
2086 2.49029845633686e-08
2087 2.45481786009805e-08
2088 2.47309284162611e-08
2089 2.44667308635371e-08
2090 2.45441782453781e-08
2091 2.46326958830423e-08
2092 2.45265940890249e-08
2093 2.48673117653198e-08
2094 2.48332909791316e-08
2095 2.46174298723645e-08
2096 2.46225955180535e-08
2097 2.44558027162611e-08
2098 2.46626115085746e-08
2099 2.47297542443903e-08
2100 2.46634606071439e-08
2101 2.4663314945883e-08
2102 2.46120759328505e-08
2103 2.44320137454679e-08
2104 2.44395472748238e-08
2105 2.49229845650234e-08
2106 2.47525449026398e-08
2107 2.45006290811034e-08
2108 2.45354208061599e-08
2109 2.42156410479311e-08
2110 2.43604354466243e-08
2111 2.4500501183411e-08
2112 2.43078233097549e-08
2113 2.46701343797895e-08
2114 2.45102054208246e-08
2115 2.44935964843762e-08
2116 2.5250553647993e-08
2117 2.43360354090782e-08
2118 2.45916726981932e-08
2119 2.52205367701208e-08
2120 2.46017908267504e-08
2121 2.49338931723742e-08
2122 2.4722412561573e-08
2123 2.45541240673219e-08
2124 2.42495818980615e-08
2125 2.44727313969406e-08
2126 2.42391084981364e-08
2127 2.45311024826833e-08
2128 2.45872939785841e-08
2129 2.49308502731083e-08
2130 2.42330049360362e-08
2131 2.41882389673265e-08
2132 2.44356783696276e-08
2133 2.41989202010018e-08
2134 2.44141897809413e-08
2135 2.42982061138264e-08
2136 2.43319142612108e-08
2137 2.4413552068836e-08
2138 2.42509656800394e-08
2139 2.44228122170398e-08
2140 2.44276350258588e-08
2141 2.47531275476831e-08
2142 2.44304150243124e-08
2143 2.41006059553683e-08
2144 2.48142892900205e-08
2145 2.43173978731193e-08
2146 2.41904825060146e-08
2147 2.43600588589743e-08
2148 2.44908608948435e-08
2149 2.43334685734453e-08
2150 2.44408902005944e-08
2151 2.41829578584429e-08
2152 2.42306192888009e-08
2153 2.43509710173839e-08
2154 2.41689619429053e-08
2155 2.43032882707439e-08
2156 2.42288056284679e-08
2157 2.43248639009153e-08
2158 2.42618352075397e-08
2159 2.42574600406442e-08
2160 2.43164031132892e-08
2161 2.40620305902439e-08
2162 2.46967246653185e-08
2163 2.49095464255333e-08
2164 2.41171580483979e-08
2165 2.41693474123394e-08
2166 2.42884627965623e-08
2167 2.39106672239586e-08
2168 2.40366357928679e-08
2169 2.41718041138483e-08
2170 2.45638283047356e-08
2171 2.43519782117119e-08
2172 2.4183963276414e-08
2173 2.42098856517714e-08
2174 2.38472548375057e-08
2175 2.43999380700188e-08
2176 2.38924151574338e-08
2177 2.40398403406061e-08
2178 2.41757867058823e-08
2179 2.40138859908257e-08
2180 2.4156488365179e-08
2181 2.43734099569792e-08
2182 2.42771189817859e-08
2183 2.41311717275039e-08
2184 2.43184086201609e-08
2185 2.38746942216039e-08
2186 2.41969786429763e-08
2187 2.37447022044535e-08
2188 2.38194761692512e-08
2189 2.40068533940985e-08
2190 2.38143247344169e-08
2191 2.39881714492185e-08
2192 2.39644357691304e-08
2193 2.40875710488808e-08
2194 2.38030146704205e-08
2195 2.3837772644697e-08
2196 2.38353177195449e-08
2197 2.41768773889817e-08
2198 2.38538309105252e-08
2199 2.39366517718054e-08
2200 2.40178668065028e-08
2201 2.38426931531421e-08
2202 2.40547812779823e-08
2203 2.42726407861937e-08
2204 2.38016877318614e-08
2205 2.41120954314056e-08
2206 2.38498483184912e-08
2207 2.38132056296081e-08
2208 2.39693029868704e-08
2209 2.38302479971253e-08
2210 2.34782788766097e-08
2211 2.33619505962679e-08
2212 2.35724719743757e-08
2213 2.36570585343543e-08
2214 2.37570745298399e-08
2215 2.40705517740025e-08
2216 2.38830466514628e-08
2217 2.37174884176738e-08
2218 2.38956925358025e-08
2219 2.37297168581563e-08
2220 2.35395862802079e-08
2221 2.35062511677597e-08
2222 2.40299211640149e-08
2223 2.36241586293318e-08
2224 2.37801653923952e-08
2225 2.36760619998222e-08
2226 2.35886101762617e-08
2227 2.3766375534251e-08
2228 2.35494876932307e-08
2229 2.36910810968993e-08
2230 2.34576180702106e-08
2231 2.3341154786749e-08
2232 2.39343833641215e-08
2233 2.38883171022053e-08
2234 2.36215846882715e-08
2235 2.35517347846326e-08
2236 2.36435369060928e-08
2237 2.39273454383238e-08
2238 2.35532642278713e-08
2239 2.37870025898701e-08
2240 2.36436132894369e-08
2241 2.34432508960936e-08
2242 2.35640484902433e-08
2243 2.35211423671444e-08
2244 2.37230981525727e-08
2245 2.34647732355597e-08
2246 2.33867201160365e-08
2247 2.36558754806993e-08
2248 2.35093313705192e-08
2249 2.37317188123143e-08
2250 2.38012294317969e-08
2251 2.33927330839379e-08
2252 2.39427979664697e-08
2253 2.3490430933748e-08
2254 2.37120403312474e-08
2255 2.36760016036897e-08
2256 2.38030288812752e-08
2257 2.33780692582286e-08
2258 2.35178276852821e-08
2259 2.39687487635365e-08
2260 2.32408705613807e-08
2261 2.33075407862771e-08
2262 2.37331807539931e-08
2263 2.36095747396803e-08
2264 2.36421424659738e-08
2265 2.35144810289967e-08
2266 2.35111219382134e-08
2267 2.35968862227764e-08
2268 2.3764030743223e-08
2269 2.36020269994697e-08
2270 2.35479618027057e-08
2271 2.36164225952962e-08
2272 2.37149233583978e-08
2273 2.37648904999332e-08
2274 2.33042207753442e-08
2275 2.36482655679993e-08
2276 2.34358275008617e-08
2277 2.31851782217518e-08
2278 2.37616060161372e-08
2279 2.36182025048493e-08
2280 2.37643043021762e-08
2281 2.37172130823637e-08
2282 2.39379698285802e-08
2283 2.31010570672652e-08
2284 2.35087060929118e-08
2285 2.34289068146154e-08
2286 2.36632455852259e-08
2287 2.34061090509385e-08
2288 2.35241017776389e-08
2289 2.37056312357709e-08
2290 2.3676307137066e-08
2291 2.35416113270048e-08
2292 2.39698891846274e-08
2293 2.34970443102611e-08
2294 2.34215598027276e-08
2295 2.33430981211313e-08
2296 2.34626291728546e-08
2297 2.33885977252157e-08
2298 2.32908661246256e-08
2299 2.35711929974514e-08
2300 2.33020696072117e-08
2301 2.36674857490016e-08
2302 2.35814585636263e-08
2303 2.31000676365056e-08
2304 2.33190764475921e-08
2305 2.37205739495039e-08
2306 2.35120136693467e-08
2307 2.3456452780124e-08
2308 2.35395347658596e-08
2309 2.34792434383735e-08
2310 2.33883792333245e-08
2311 2.34201387172561e-08
2312 2.32477539441334e-08
2313 2.33964332352343e-08
2314 2.32023928958824e-08
2315 2.37034143424353e-08
2316 2.34434445189891e-08
2317 2.35012294069747e-08
2318 2.33582184705483e-08
2319 2.36004016329616e-08
2320 2.35522481517592e-08
2321 2.3447814356814e-08
2322 2.34345893801446e-08
2323 2.30037979775943e-08
2324 2.33952359707246e-08
2325 2.35984334295836e-08
2326 2.36547634813178e-08
2327 2.35815207361156e-08
2328 2.38801174390346e-08
2329 2.34333086268634e-08
2330 2.34054251535554e-08
2331 2.34522730124809e-08
2332 2.34486421391011e-08
2333 2.33418919748374e-08
2334 2.34101023011135e-08
2335 2.34802399745604e-08
2336 2.34556924993967e-08
2337 2.33792256665311e-08
2338 2.33443113728526e-08
2339 2.35381314439564e-08
2340 2.32914576514531e-08
2341 2.33230288415598e-08
2342 2.32082548734525e-08
2343 2.3330260390253e-08
2344 2.361701056941e-08
2345 2.34765078488408e-08
2346 2.32887753526256e-08
2347 2.35973818263346e-08
2348 2.29575540799942e-08
2349 2.34074857274891e-08
2350 2.33318626641221e-08
2351 2.3287212158607e-08
2352 2.33039809671709e-08
2353 2.31523618055007e-08
2354 2.37739250508184e-08
2355 2.33195578402956e-08
2356 2.35222845645922e-08
2357 2.30072458862196e-08
2358 2.32972929836706e-08
2359 2.33051320464028e-08
2360 2.31722854238114e-08
2361 2.31818848561716e-08
2362 2.35847110729992e-08
2363 2.33482850831024e-08
2364 2.30942003298651e-08
2365 2.30532126721528e-08
2366 2.33154491269261e-08
2367 2.31867289812726e-08
2368 2.32781616205102e-08
2369 2.34806449839198e-08
2370 2.31649828208447e-08
2371 2.3463874398999e-08
2372 2.346734717662e-08
2373 2.33266703730806e-08
2374 2.3325526399276e-08
2375 2.30972769799109e-08
2376 2.32506298658564e-08
2377 2.30209202811693e-08
2378 2.32665140487143e-08
2379 2.28656062972732e-08
2380 2.29850023458766e-08
2381 2.31034498199278e-08
2382 2.30472654294545e-08
2383 2.33588703935084e-08
2384 2.33477770450463e-08
2385 2.29949446151068e-08
2386 2.35541754989299e-08
2387 2.35511752322282e-08
2388 2.31615970847088e-08
2389 2.32949659562109e-08
2390 2.2868240634466e-08
2391 2.32290116031209e-08
2392 2.34452048886169e-08
2393 2.33416539430209e-08
2394 2.35339641108112e-08
2395 2.32886314677216e-08
2396 2.31837304909277e-08
2397 2.32464998362047e-08
2398 2.31224870361757e-08
2399 2.35722801278371e-08
2400 2.33870274257697e-08
2401 2.32779147069095e-08
2402 2.30293029090944e-08
2403 2.33426842299878e-08
2404 2.34267218957029e-08
2405 2.33046044684215e-08
2406 2.34447821156891e-08
2407 2.32102443931126e-08
2408 2.28367973420518e-08
2409 2.31955361584824e-08
2410 2.30528591771417e-08
2411 2.31635066683111e-08
2412 2.3098371215724e-08
2413 2.32978614178592e-08
2414 2.31342927037304e-08
2415 2.31906280845351e-08
2416 2.30644374710209e-08
2417 2.30491608022021e-08
2418 2.3009951277686e-08
2419 2.32290773283239e-08
2420 2.33344259470414e-08
2421 2.32097683294796e-08
2422 2.3128379211812e-08
2423 2.36245902840437e-08
2424 2.32650307907534e-08
2425 2.30487557928427e-08
2426 2.31708501274852e-08
2427 2.30635457398876e-08
2428 2.32854464599086e-08
2429 2.33177228636805e-08
2430 2.32510402042863e-08
2431 2.29670149565209e-08
2432 2.32839649783045e-08
2433 2.33968719953737e-08
2434 2.35387638269913e-08
2435 2.32149108825297e-08
2436 2.33176891129006e-08
2437 2.32484307360892e-08
2438 2.3142780136709e-08
2439 2.32485160012175e-08
2440 2.30514949350891e-08
2441 2.32971615332644e-08
2442 2.31526513516656e-08
2443 2.29972432208569e-08
2444 2.29623520198174e-08
2445 2.30744863216614e-08
2446 2.32309282921506e-08
2447 2.31422809804371e-08
2448 2.31485657309349e-08
2449 2.28489334119786e-08
2450 2.32277610479059e-08
2451 2.31025829577902e-08
2452 2.32483401418904e-08
2453 2.29016059449805e-08
2454 2.27986074463615e-08
2455 2.29936638618256e-08
2456 2.31677237394479e-08
2457 2.28664802648382e-08
2458 2.28613696862112e-08
2459 2.32114860665433e-08
2460 2.30549233037891e-08
2461 2.31798296113084e-08
2462 2.31650023607699e-08
2463 2.30549961344195e-08
2464 2.31426113828093e-08
2465 2.31953034557364e-08
2466 2.28947545366509e-08
2467 2.3138433391523e-08
2468 2.3115459768519e-08
2469 2.3133589266422e-08
2470 2.29742092017204e-08
2471 2.30806840306741e-08
2472 2.3403858406823e-08
2473 2.3133853943591e-08
2474 2.31444534648517e-08
2475 2.29210748159403e-08
2476 2.29807746165989e-08
2477 2.31895622704315e-08
2478 2.29960388509198e-08
2479 2.31015579998939e-08
2480 2.317943526009e-08
2481 2.29587939770681e-08
2482 2.34276011923384e-08
2483 2.34725447967321e-08
2484 2.29609593560554e-08
2485 2.3276305327613e-08
2486 2.29480843216834e-08
2487 2.27422383147768e-08
2488 2.28888836772967e-08
2489 2.32950547740529e-08
2490 2.31760743929499e-08
2491 2.31947083761952e-08
2492 2.30557954949973e-08
2493 2.31060610644818e-08
2494 2.30374901377672e-08
2495 2.30107985998984e-08
2496 2.31509122983198e-08
2497 2.30974954718022e-08
2498 2.30766890041423e-08
2499 2.28267840185481e-08
2500 2.30828387515203e-08
2501 2.31003021156084e-08
2502 2.32518626575029e-08
2503 2.30823786750989e-08
2504 2.29494361292382e-08
2505 2.30128591738321e-08
2506 2.26958327687043e-08
2507 2.30254446620393e-08
2508 2.33751826783646e-08
2509 2.29226291281748e-08
2510 2.30528698352828e-08
2511 2.27036363042998e-08
2512 2.25146763455086e-08
2513 2.31377601522809e-08
2514 2.29855352529285e-08
2515 2.30857803984463e-08
2516 2.30555361468987e-08
2517 2.3079911315449e-08
2518 2.30348078389397e-08
2519 2.32628831753345e-08
2520 2.34566908119405e-08
2521 2.29286474251467e-08
2522 2.2869059534969e-08
2523 2.29286811759266e-08
2524 2.31415331342077e-08
2525 2.27794352269939e-08
2526 2.32336088146212e-08
2527 2.29546266439229e-08
2528 2.25595915281929e-08
2529 2.28123333556596e-08
2530 2.27746621561664e-08
2531 2.31145360629625e-08
2532 2.27273648789605e-08
2533 2.30000871681568e-08
2534 2.29928271977542e-08
2535 2.31476633416605e-08
2536 2.29587993061386e-08
2537 2.28320367057222e-08
2538 2.30120011934787e-08
2539 2.28803340718287e-08
2540 2.27969625399282e-08
2541 2.2929240728331e-08
2542 2.28167671423307e-08
2543 2.32453309934044e-08
2544 2.26710152873011e-08
2545 2.30595915695631e-08
2546 2.29331913459419e-08
2547 2.2576314151479e-08
2548 2.27654073370331e-08
2549 2.2951429201612e-08
2550 2.29664856021827e-08
2551 2.27589094237146e-08
2552 2.29411369900845e-08
2553 2.30157279901277e-08
2554 2.27851622014441e-08
2555 2.25738130410491e-08
2556 2.29013537023093e-08
2557 2.30396679512523e-08
2558 2.25403748999042e-08
2559 2.30456702610127e-08
2560 2.30695764713573e-08
2561 2.2927581611043e-08
2562 2.32561649937679e-08
2563 2.29048708888513e-08
2564 2.30058976313785e-08
2565 2.28780372424353e-08
2566 2.30643522058926e-08
2567 2.30796359801388e-08
2568 2.29042615984554e-08
2569 2.3267954674111e-08
2570 2.30812808865721e-08
2571 2.26891270216356e-08
2572 2.28266401336441e-08
2573 2.29634284920621e-08
2574 2.32184298454285e-08
2575 2.31182593068979e-08
2576 2.29919585592597e-08
2577 2.32982788617164e-08
2578 2.29544081520316e-08
2579 2.27588454748684e-08
2580 2.30585737170941e-08
2581 2.30215402297063e-08
2582 2.33327543952555e-08
2583 2.31981758247457e-08
2584 2.32162413738024e-08
2585 2.29650130023629e-08
2586 2.30982664106705e-08
2587 2.31429879704592e-08
2588 2.29563568154845e-08
2589 2.29760974690407e-08
2590 2.31458958666053e-08
2591 2.27557581666815e-08
2592 2.30871197715032e-08
2593 2.31102355030544e-08
2594 2.29602417078922e-08
2595 2.30074643781109e-08
2596 2.26458620744552e-08
2597 2.31012826645838e-08
2598 2.32099761632298e-08
2599 2.30029453263114e-08
2600 2.31819292650925e-08
2601 2.25231708839146e-08
2602 2.30737544626436e-08
2603 2.26989769203101e-08
2604 2.29241301497041e-08
2605 2.30022880742808e-08
2606 2.25241656437447e-08
2607 2.3003369875596e-08
2608 2.29429186759944e-08
2609 2.30444712201461e-08
2610 2.27169234534585e-08
2611 2.30338308426781e-08
2612 2.28831478210623e-08
2613 2.29083969571775e-08
2614 2.31656915872236e-08
2615 2.3111002889209e-08
2616 2.27278835751576e-08
2617 2.22897877932837e-08
2618 2.29129888396074e-08
2619 2.26978027484392e-08
2620 2.30305499115957e-08
2621 2.29983232458153e-08
2622 2.29749179680994e-08
2623 2.28790977274684e-08
2624 2.25312888346707e-08
2625 2.29809256069302e-08
2626 2.31553318741362e-08
2627 2.30630163855494e-08
2628 2.25929746022757e-08
2629 2.25952181409639e-08
2630 2.28338450369847e-08
2631 2.30424177516397e-08
2632 2.30920260690937e-08
2633 2.2753518180707e-08
2634 2.31189432042811e-08
2635 2.29258283468425e-08
2636 2.29609700141964e-08
2637 2.29889263181349e-08
2638 2.29336052370854e-08
2639 2.295271173125e-08
2640 2.26237997225098e-08
2641 2.27384475692816e-08
2642 2.27927916540693e-08
2643 2.32000125777176e-08
2644 2.28537793134365e-08
2645 2.26868923647316e-08
2646 2.2661238219257e-08
2647 2.2948166034098e-08
2648 2.28804832858032e-08
2649 2.2816598388431e-08
2650 2.26867804542508e-08
2651 2.28774279520394e-08
2652 2.26852048257342e-08
2653 2.30464092254579e-08
2654 2.28419789749523e-08
2655 2.26472121056531e-08
2656 2.31149375196082e-08
2657 2.30208385687547e-08
2658 2.29741239365922e-08
2659 2.32791563803403e-08
2660 2.30648300458824e-08
2661 2.31452528254295e-08
2662 2.29922854089182e-08
2663 2.25142251508714e-08
2664 2.28155379033979e-08
2665 2.27326903967651e-08
2666 2.30418777391606e-08
2667 2.2616188033453e-08
2668 2.26381313694901e-08
2669 2.29594210310324e-08
2670 2.27742766867323e-08
2671 2.30627179576004e-08
2672 2.28808296753868e-08
2673 2.2515424191738e-08
2674 2.31609558198898e-08
2675 2.24256471170747e-08
2676 2.31278214357644e-08
2677 2.31029755326517e-08
2678 2.31616557044845e-08
2679 2.29507470805856e-08
2680 2.29372041360421e-08
2681 2.24746230514938e-08
2682 2.26511929213302e-08
2683 2.26116583235125e-08
2684 2.30709993331857e-08
2685 2.29326975187405e-08
2686 2.30168275550113e-08
2687 2.29724950173704e-08
2688 2.30239525222942e-08
2689 2.26514771384245e-08
2690 2.29717347366432e-08
2691 2.29426682096801e-08
2692 2.30285905900018e-08
2693 2.29911076843337e-08
2694 2.30059100658764e-08
2695 2.30284715740936e-08
2696 2.30096013353887e-08
2697 2.29683561059346e-08
2698 2.30370869047647e-08
2699 2.30881465057564e-08
2700 2.29053522815548e-08
2701 2.32112320475153e-08
2702 2.31714523124538e-08
2703 2.28657999201687e-08
2704 2.30008687651662e-08
2705 2.2823057221899e-08
2706 2.30546977064705e-08
2707 2.31449206467005e-08
2708 2.27316352408025e-08
2709 2.30475425411214e-08
2710 2.29690293451768e-08
2711 2.30485213137399e-08
2712 2.28756231734906e-08
2713 2.29024124109856e-08
2714 2.28927117262856e-08
2715 2.30983587812261e-08
2716 2.29480914271107e-08
2717 2.32582060277764e-08
2718 2.25028120581783e-08
2719 2.29547527652585e-08
2720 2.28964882609262e-08
2721 2.28994085915701e-08
2722 2.25496830097427e-08
2723 2.27187904044968e-08
2724 2.30590337935155e-08
2725 2.29540475515932e-08
2726 2.25060787784059e-08
2727 2.24706049323231e-08
2728 2.27315979373088e-08
2729 2.29600143342168e-08
2730 2.30470309503517e-08
2731 2.31931540639607e-08
2732 2.29013927821597e-08
2733 2.29203305224246e-08
2734 2.24221423650306e-08
2735 2.30752057461814e-08
2736 2.26452634422003e-08
2737 2.29516654570716e-08
2738 2.25636167527909e-08
2739 2.30725270000676e-08
2740 2.21664890887041e-08
2741 2.27476064651455e-08
2742 2.27639826988479e-08
2743 2.268869181421e-08
2744 2.28951275715872e-08
2745 2.31862387067849e-08
2746 2.27856968848528e-08
2747 2.24924097125268e-08
2748 2.22215472689413e-08
2749 2.25336975745449e-08
2750 2.2510834085665e-08
2751 2.24456933040074e-08
2752 2.28206058494607e-08
2753 2.2684609746193e-08
2754 2.28026557635985e-08
2755 2.28572218929912e-08
2756 2.21244231823903e-08
2757 2.28788454847972e-08
2758 2.25217640092978e-08
2759 2.29917542782232e-08
2760 2.26668674940811e-08
2761 2.31475958401006e-08
2762 2.22987708298206e-08
2763 2.31147492257833e-08
2764 2.24325589215368e-08
2765 2.30636487685842e-08
2766 2.29630003900638e-08
2767 2.31386483307006e-08
2768 2.31075478751563e-08
2769 2.33126105086967e-08
2770 2.28761241061193e-08
2771 2.30369092690808e-08
2772 2.3372169977165e-08
2773 2.32590693372003e-08
2774 2.32363035479466e-08
2775 2.26454321961e-08
2776 2.28603109775349e-08
2777 2.25509086959619e-08
2778 2.28816379177488e-08
2779 2.2448407577258e-08
2780 2.30349339602753e-08
2781 2.29417604913351e-08
2782 2.27790550866303e-08
2783 2.24685088312526e-08
2784 2.27386891538117e-08
2785 2.31041408227384e-08
2786 2.30342038776143e-08
2787 2.27273471153921e-08
2788 2.25127951836157e-08
2789 2.29778471805275e-08
2790 2.29955912089963e-08
2791 2.28794760914752e-08
2792 2.25766054740006e-08
2793 2.24052438824174e-08
2794 2.29453558375781e-08
2795 2.25658016717034e-08
2796 2.28626788612019e-08
2797 2.30211600893426e-08
2798 2.29235279647355e-08
2799 2.31564101227377e-08
2800 2.25870753212121e-08
2801 2.29803021056796e-08
2802 2.30609540352589e-08
2803 2.30555450286829e-08
2804 2.20112585935794e-08
2805 2.28567333948604e-08
2806 2.25041230095258e-08
2807 2.2938410282336e-08
2808 2.31638495051811e-08
2809 2.25176002288663e-08
2810 2.23298304291575e-08
2811 2.28134098279043e-08
2812 2.29072352198045e-08
2813 2.29976375720753e-08
2814 2.2848851699564e-08
2815 2.25922160979053e-08
2816 2.32247856501999e-08
2817 2.29439152121813e-08
2818 2.24519123293021e-08
2819 2.25413145926723e-08
2820 2.22128875293492e-08
2821 2.29591865519296e-08
2822 2.262959419852e-08
2823 2.26127081504046e-08
2824 2.28991616779695e-08
2825 2.28598473483999e-08
2826 2.29433396725653e-08
2827 2.26935146230289e-08
2828 2.2741119209968e-08
2829 2.29674181895234e-08
2830 2.28707719429622e-08
2831 2.27906671312894e-08
2832 2.29129124562633e-08
2833 2.25087894989429e-08
2834 2.27437109145967e-08
2835 2.27280843034805e-08
2836 2.33775843128115e-08
2837 2.31857235633015e-08
2838 2.27907719363429e-08
2839 2.28226504361828e-08
2840 2.27282459519529e-08
2841 2.28522196721315e-08
2842 2.24022436157156e-08
2843 2.25212097859639e-08
2844 2.29258763084772e-08
2845 2.27997247748135e-08
2846 2.27725625023822e-08
2847 2.29647270089117e-08
2848 2.26597816066487e-08
2849 2.28318963735319e-08
2850 2.24995240216685e-08
2851 2.2912406194564e-08
2852 2.29341541313488e-08
2853 2.29652563632499e-08
2854 2.30268337730877e-08
2855 2.29585346289696e-08
2856 2.28997762974359e-08
2857 2.28640306687566e-08
2858 2.28691057202468e-08
2859 2.26995169327893e-08
2860 2.24325429343253e-08
2861 2.25097025463583e-08
2862 2.25362182249e-08
2863 2.28135146329578e-08
2864 2.2834814927819e-08
2865 2.28206626928795e-08
2866 2.29885070979208e-08
2867 2.27666063778997e-08
2868 2.29302781207252e-08
2869 2.28196643803358e-08
2870 2.26977370232362e-08
2871 2.29234107251841e-08
2872 2.28832028881243e-08
2873 2.2909043551067e-08
2874 2.30257288791336e-08
2875 2.29209646818163e-08
2876 2.28051177941779e-08
2877 2.2591942538952e-08
2878 2.28506937816064e-08
2879 2.23099547724814e-08
2880 2.24283294159022e-08
2881 2.24155645156543e-08
2882 2.1990922860482e-08
2883 2.19606750562207e-08
2884 2.22393765625384e-08
2885 2.18010303143501e-08
2886 2.23031761947823e-08
2887 2.24983054408767e-08
2888 2.2089253093327e-08
2889 2.23994476300504e-08
2890 2.20236007208996e-08
2891 2.22454037412945e-08
2892 2.19972022819093e-08
2893 2.18961933029505e-08
2894 2.18191882339624e-08
2895 2.26368470634952e-08
2896 2.16958255805366e-08
2897 2.28022347670276e-08
2898 2.19770903697736e-08
2899 2.2149954759243e-08
2900 2.1919426274053e-08
2901 2.15996074359737e-08
2902 2.21515232823322e-08
2903 2.20617124568889e-08
2904 2.21179092818602e-08
2905 2.18836682108758e-08
2906 2.21738449823761e-08
2907 2.1761975332879e-08
2908 2.15666062786113e-08
2909 2.20463096667345e-08
2910 2.18784244054859e-08
2911 2.22286971052199e-08
2912 2.1553159257337e-08
2913 2.21163212188458e-08
2914 2.14323154779095e-08
2915 2.1840772745918e-08
2916 2.20550511187412e-08
2917 2.24484928423863e-08
2918 2.17346673991869e-08
2919 2.20787743643314e-08
2920 2.20687557117571e-08
2921 2.18226929860066e-08
2922 2.19469153961427e-08
2923 2.21333547045788e-08
2924 2.19686864255664e-08
2925 2.21841229830488e-08
2926 2.18610018976051e-08
2927 2.20943690010245e-08
2928 2.1725398369199e-08
2929 2.16516848894344e-08
2930 2.20022577934742e-08
2931 2.18293330078723e-08
2932 2.14612132509728e-08
2933 2.18142357510942e-08
2934 2.14840429890728e-08
2935 2.20378915116726e-08
2936 2.20902300895887e-08
2937 2.17434390492599e-08
2938 2.20504947634481e-08
2939 2.16333688740633e-08
2940 2.15651887458534e-08
2941 2.17533990820584e-08
2942 2.15917683732414e-08
2943 2.18504609961201e-08
2944 2.2111651176715e-08
2945 2.1558999918625e-08
2946 2.20290132801892e-08
2947 2.15572484307813e-08
2948 2.192621728625e-08
2949 2.16153974719191e-08
2950 2.21224105700912e-08
2951 2.17895230747445e-08
2952 2.18558398046298e-08
2953 2.16054640844732e-08
2954 2.17917044409432e-08
2955 2.18258993101017e-08
2956 2.16091944338359e-08
2957 2.16332303182298e-08
2958 2.15746993603716e-08
2959 2.24228529077664e-08
2960 2.15958024796237e-08
2961 2.19497149345216e-08
2962 2.18190834289089e-08
2963 2.23800995513557e-08
2964 2.16016307064137e-08
2965 2.15227391464623e-08
2966 2.15595026276105e-08
2967 2.19103952758815e-08
2968 2.16520739115822e-08
2969 2.13773745372237e-08
2970 2.22795257798225e-08
2971 2.20052829291717e-08
2972 2.15506759104755e-08
2973 2.17437818861299e-08
2974 2.155257305958e-08
2975 2.19397708889346e-08
2976 2.20068248069083e-08
2977 2.20141700424392e-08
2978 2.12160955470608e-08
2979 2.17693134629826e-08
2980 2.17648672418136e-08
2981 2.16903952576786e-08
2982 2.13077306909781e-08
2983 2.14521005403867e-08
2984 2.14344559879009e-08
2985 2.21123013233182e-08
2986 2.15285140825472e-08
2987 2.14047943813966e-08
2988 2.1378752990131e-08
2989 2.19757687602851e-08
2990 2.15543245474237e-08
2991 2.15290008043212e-08
2992 2.19768629960981e-08
2993 2.18086491088343e-08
2994 2.16488711402008e-08
2995 2.23469829307987e-08
2996 2.17149960235474e-08
2997 2.13676223381754e-08
2998 2.14579074508947e-08
2999 2.13624424816317e-08
3000 2.18679545582745e-08
3001 2.13675246385492e-08
3002 2.22780638381437e-08
3003 2.17551452408316e-08
3004 2.135313614815e-08
3005 2.15357225386015e-08
3006 2.13640163337914e-08
3007 2.16353068793751e-08
3008 2.14392432695831e-08
3009 2.1529379168328e-08
3010 2.17370441646381e-08
3011 2.12111537223336e-08
3012 2.12803339394441e-08
3013 2.15550244320184e-08
3014 2.15222151211947e-08
3015 2.12582058622957e-08
3016 2.1895081303569e-08
3017 2.23734346604942e-08
3018 2.12921342779282e-08
3019 2.15409574622072e-08
3020 2.14605595516559e-08
3021 2.11337898292641e-08
3022 2.12430535384556e-08
3023 2.13433288820397e-08
3024 2.11189998822192e-08
3025 2.11113544423824e-08
3026 2.12582325076482e-08
3027 2.15924416124835e-08
3028 2.13543600580124e-08
3029 2.10003694434135e-08
3030 2.1203982569773e-08
3031 2.1650848225363e-08
3032 2.1196870036988e-08
3033 2.1140659001162e-08
3034 2.17712212702281e-08
3035 2.24430642958851e-08
3036 2.12148805189827e-08
3037 2.10633341879429e-08
3038 2.16067892466754e-08
3039 2.15541611225945e-08
3040 2.12988311432127e-08
3041 2.16359410387668e-08
3042 2.09709494214394e-08
3043 2.12218740358594e-08
3044 2.11976587394247e-08
3045 2.18877502788928e-08
3046 2.12582165204367e-08
3047 2.11924859883084e-08
3048 2.13532729276267e-08
3049 2.08732657824839e-08
3050 2.1157338991884e-08
3051 2.10962340929655e-08
3052 2.14739710457934e-08
3053 2.14284909816342e-08
3054 2.13050874720011e-08
3055 2.08983834681931e-08
3056 2.16293383203947e-08
3057 2.12707949032165e-08
3058 2.13013855443478e-08
3059 2.12455990578064e-08
3060 2.15152855531642e-08
3061 2.10732444827499e-08
3062 2.09949551077671e-08
3063 2.15501216871417e-08
3064 2.1225860180607e-08
3065 2.12303561397675e-08
3066 2.10327151251022e-08
3067 2.14186304248187e-08
3068 2.09978079368511e-08
3069 2.14673931964171e-08
3070 2.12869561977413e-08
3071 2.11073842848464e-08
3072 2.14350475147285e-08
3073 2.10111075205077e-08
3074 2.12721946724059e-08
3075 2.17389946044477e-08
3076 2.10267394606944e-08
3077 2.1168476749267e-08
3078 2.10932320499069e-08
3079 2.11530011284822e-08
3080 2.07661141615745e-08
3081 2.09810444573577e-08
3082 2.08491588438164e-08
3083 2.10924078203334e-08
3084 2.1223415913596e-08
3085 2.10914841147769e-08
3086 2.06652135403829e-08
3087 2.11510862158093e-08
3088 2.09522141858542e-08
3089 2.18667892681879e-08
3090 2.11787956061471e-08
3091 2.08579340466031e-08
3092 2.07627603998617e-08
3093 2.07624051284938e-08
3094 2.12392929910266e-08
3095 2.0966684388668e-08
3096 2.08619983510516e-08
3097 2.10434958347605e-08
3098 2.09015134089441e-08
3099 2.06829930959884e-08
3100 2.09212700497119e-08
3101 2.06146939518703e-08
3102 2.19421085745353e-08
3103 2.06753583142927e-08
3104 2.06348929054911e-08
3105 2.1335704758485e-08
3106 2.1009180173337e-08
3107 2.08535944068444e-08
3108 2.05892725091417e-08
3109 2.09999395650584e-08
3110 2.15552429239096e-08
3111 2.09790211869176e-08
3112 2.09815294027749e-08
3113 2.14980673263199e-08
3114 2.10771400332987e-08
3115 2.09919264193559e-08
3116 2.10232116160114e-08
3117 2.07934665041876e-08
3118 2.05381329720922e-08
3119 2.11642543490598e-08
3120 2.18240181482088e-08
3121 2.08660662082139e-08
3122 2.12848192404635e-08
3123 2.10112531817686e-08
3124 2.17455760065377e-08
3125 2.1875470324062e-08
3126 2.19315250404861e-08
3127 2.14176569812707e-08
3128 2.18389786255102e-08
3129 2.08918589095219e-08
3130 2.17480344844034e-08
3131 2.11263166960407e-08
3132 2.18354365699724e-08
3133 2.09932036199234e-08
3134 2.14208650817227e-08
3135 2.10259827326809e-08
3136 2.19197975326324e-08
3137 2.17727738061058e-08
3138 2.18961240250337e-08
3139 2.2054058135268e-08
3140 2.19344808982669e-08
3141 2.20577618392781e-08
3142 2.1745616862745e-08
3143 2.07027230914036e-08
3144 2.14603215198395e-08
3145 2.19159819181414e-08
3146 2.18638636084734e-08
3147 2.17998081808446e-08
3148 2.14816857635469e-08
3149 2.18730100698394e-08
3150 2.13532445059172e-08
3151 2.16983515599622e-08
3152 2.19154063785254e-08
3153 2.18850004785054e-08
3154 2.14730331293822e-08
3155 2.17530882196115e-08
3156 2.1403872452197e-08
3157 2.07148804776125e-08
3158 2.09907025094935e-08
3159 2.09507717841007e-08
3160 2.06154364690292e-08
3161 2.10351309704038e-08
3162 2.14312834145858e-08
3163 2.17279652048319e-08
3164 2.04563832539861e-08
3165 2.15299600370145e-08
3166 2.08409414170774e-08
3167 2.10396660094148e-08
3168 2.04156940242228e-08
3169 2.1373764980126e-08
3170 2.12943636057616e-08
3171 2.07387955697413e-08
3172 2.07175201438758e-08
3173 2.06560724080873e-08
3174 2.04321963792609e-08
3175 2.04074197540649e-08
3176 2.13334754306516e-08
3177 2.14112017005164e-08
3178 2.03034229429022e-08
3179 2.06900701016366e-08
3180 2.07728962919873e-08
3181 2.15586180019045e-08
3182 2.05551824450367e-08
3183 2.05819485898928e-08
3184 1.99291143587743e-08
3185 2.02922851855192e-08
3186 2.03973709034244e-08
3187 2.01998489046673e-08
3188 2.01903471719334e-08
3189 2.06727399643114e-08
3190 2.03792716035878e-08
3191 2.0221023078193e-08
3192 2.02911518698556e-08
3193 2.06423411697187e-08
3194 2.02515550995486e-08
3195 2.03588168545821e-08
3196 2.01108605324407e-08
3197 2.05231973637865e-08
3198 2.01106686859021e-08
3199 2.05448422718746e-08
3200 2.01409289246612e-08
3201 2.03187102698621e-08
3202 2.03866541426123e-08
3203 2.04596286579317e-08
3204 2.0245245480055e-08
3205 2.03698693468368e-08
3206 2.06508907751868e-08
3207 1.99094287722801e-08
3208 2.01308623104524e-08
3209 2.02505852087143e-08
3210 2.00097929337062e-08
3211 2.02047143460504e-08
3212 2.01117984488519e-08
3213 2.05204866432496e-08
3214 2.02477750121943e-08
3215 1.990600040358e-08
3216 1.99538128242693e-08
3217 1.96850304945428e-08
3218 1.97789749023514e-08
3219 2.01832914825673e-08
3220 1.95849665374226e-08
3221 1.96379641437261e-08
3222 1.99137257794746e-08
3223 2.00387120230516e-08
3224 1.9632910408518e-08
3225 1.98615719426698e-08
3226 1.96104661398522e-08
3227 1.96349869696633e-08
3228 1.95819129800157e-08
3229 1.99605612039022e-08
3230 1.94968681199725e-08
3231 1.9724470945448e-08
3232 1.9471633194712e-08
3233 1.95978380190809e-08
3234 1.95092386690021e-08
3235 1.95758520504796e-08
3236 1.97046308159088e-08
3237 1.97584473227153e-08
3238 1.96935161511647e-08
3239 2.05315142665086e-08
3240 1.96478335823258e-08
3241 2.0251450294495e-08
3242 2.00201171196568e-08
3243 1.95255829282814e-08
3244 2.01252401410557e-08
3245 2.04526919844739e-08
3246 2.05002859132719e-08
3247 2.00030356722891e-08
3248 2.10982005199867e-08
3249 2.0329329331048e-08
3250 2.00752801049475e-08
3251 1.99945695555925e-08
3252 2.03852614788502e-08
3253 1.99769161213226e-08
3254 1.97748111219198e-08
3255 1.96093914439643e-08
3256 1.99111145349207e-08
3257 1.94998861502427e-08
3258 1.96004066310707e-08
3259 1.95042328954287e-08
3260 1.94835205746813e-08
3261 1.91413853656286e-08
3262 1.90195166283047e-08
3263 2.04120187419221e-08
3264 2.08960759806587e-08
3265 1.86558057890807e-08
3266 2.01582004422107e-08
3267 1.99619858420874e-08
3268 1.98022753750138e-08
3269 2.01947969458161e-08
3270 2.01270875521686e-08
3271 1.97090184173021e-08
3272 1.96704146304683e-08
3273 1.97790441802681e-08
3274 1.95336440356186e-08
3275 1.94540117348652e-08
3276 1.9612071966435e-08
3277 1.91287998774214e-08
3278 1.90877393890787e-08
3279 1.90675955025199e-08
3280 1.93185059060852e-08
3281 1.90882403217074e-08
3282 1.92770510665241e-08
3283 1.9122312622244e-08
3284 1.89928499594316e-08
3285 1.89917557236186e-08
3286 1.88845543647176e-08
3287 1.90442044356587e-08
3288 1.82178041541192e-08
3289 1.84867090524676e-08
3290 1.87024884468201e-08
3291 1.89046591714259e-08
3292 1.87127664474929e-08
3293 1.86066717589028e-08
3294 1.78612307166759e-08
3295 1.85178983258538e-08
3296 1.82523525182887e-08
3297 1.84347186404921e-08
3298 1.85005628594581e-08
3299 1.84243784673299e-08
3300 1.82102493084813e-08
3301 1.82155694972153e-08
3302 1.87352160452292e-08
3303 1.79142283229794e-08
3304 1.73665046787619e-08
3305 1.78184453858421e-08
3306 1.80136314753554e-08
3307 1.81053696479694e-08
3308 1.75346173136859e-08
3309 1.77865686623591e-08
3310 1.84503630151767e-08
3311 1.79964541047184e-08
3312 1.67762905789459e-08
3313 1.76918142358318e-08
3314 1.75943384306265e-08
3315 1.77493948427809e-08
3316 1.78114269999696e-08
3317 1.76898566905948e-08
3318 1.82923596270257e-08
3319 1.69888281220665e-08
3320 1.82420620831181e-08
3321 1.73102137068781e-08
3322 1.75458048090604e-08
3323 1.69638632030455e-08
3324 1.75229057930437e-08
3325 1.73092136179775e-08
3326 1.71889631417343e-08
3327 1.80042381003886e-08
3328 1.67345515222905e-08
3329 1.73235079614642e-08
3330 1.63390510010686e-08
3331 1.67239093684657e-08
3332 1.79401684619052e-08
3333 1.69090217383427e-08
3334 1.64303912697505e-08
3335 1.64297535576452e-08
3336 1.6749854836462e-08
3337 1.66701319415097e-08
3338 1.69974345709534e-08
3339 1.70716631942014e-08
3340 1.74987544454552e-08
3341 1.6949391223875e-08
3342 1.65735425383673e-08
3343 1.65801878893035e-08
3344 1.73490466437443e-08
3345 1.62228541711329e-08
3346 1.62748392540379e-08
3347 1.64113114209385e-08
3348 1.68881300055546e-08
3349 1.63193494273628e-08
3350 1.688177597714e-08
3351 1.76001080376409e-08
3352 1.61598787684625e-08
3353 1.73411347503816e-08
3354 1.60825006645382e-08
3355 1.72573102474871e-08
3356 1.7311199584924e-08
3357 1.72561094302637e-08
3358 1.63197579894359e-08
3359 1.66092419817687e-08
3360 1.6216205267483e-08
3361 1.60229340906426e-08
3362 1.61625859362857e-08
3363 1.73345959808557e-08
3364 1.6054320539638e-08
3365 1.69423124418699e-08
3366 1.61188697944681e-08
3367 1.63244191497824e-08
3368 1.62093893862902e-08
3369 1.63145035259049e-08
3370 1.62583368989999e-08
3371 1.61449342783726e-08
3372 1.62165516570667e-08
3373 1.63090927429721e-08
3374 1.60380242419933e-08
3375 1.66635985010544e-08
3376 1.62381574853043e-08
3377 1.64042770478545e-08
3378 1.60553312866796e-08
3379 1.61122830633076e-08
3380 1.67756155633469e-08
3381 1.62413744675405e-08
3382 1.65115530137427e-08
3383 1.62632769473703e-08
3384 1.73457248564546e-08
3385 1.6063689045609e-08
3386 1.61018114397393e-08
3387 1.583279463091e-08
3388 1.60255826386901e-08
3389 1.61505422369146e-08
3390 1.58448383302812e-08
3391 1.60461492981767e-08
3392 1.68201204076013e-08
3393 1.60076343291848e-08
3394 1.59195163718096e-08
3395 1.6097800425996e-08
3396 1.65287179498819e-08
3397 1.56658614969274e-08
3398 1.59945532374195e-08
3399 1.72339476023353e-08
3400 1.61313327140533e-08
3401 1.60283022410113e-08
3402 1.62037885331756e-08
3403 1.63450959433931e-08
3404 1.65644227223538e-08
3405 1.6588499462955e-08
3406 1.69408878036847e-08
3407 1.59912030284204e-08
3408 1.63521836071823e-08
3409 1.59725992432413e-08
3410 1.6922523826679e-08
3411 1.65273519314724e-08
3412 1.63601665548185e-08
3413 1.63242628303806e-08
3414 1.64555196136007e-08
3415 1.63382711804161e-08
3416 1.61249396057883e-08
3417 1.60646322910907e-08
3418 1.70673519761522e-08
3419 1.59172817149056e-08
3420 1.60196140797098e-08
3421 1.70095635354528e-08
3422 1.62814668414057e-08
3423 1.6586243489769e-08
3424 1.60453037523212e-08
3425 1.67316809296381e-08
3426 1.60489523892693e-08
3427 1.66740647955521e-08
3428 1.66363722797769e-08
3429 1.59881601291545e-08
3430 1.66754752228826e-08
3431 1.60754094480353e-08
3432 1.63362852134696e-08
3433 1.65348836844714e-08
3434 1.6024666038561e-08
3435 1.6283289383523e-08
3436 1.62792925806343e-08
3437 1.60881334920759e-08
3438 1.64159175142231e-08
3439 1.64170952388076e-08
3440 1.65701194987378e-08
3441 1.64921534206997e-08
3442 1.62453552832176e-08
3443 1.70731890847264e-08
3444 1.65806319785133e-08
3445 1.67286415830858e-08
3446 1.69964948781853e-08
3447 1.6645097744572e-08
3448 1.61675455245813e-08
3449 1.60284958639068e-08
3450 1.63726152635491e-08
3451 1.70385945352791e-08
3452 1.59566919677445e-08
3453 1.61412287980056e-08
3454 1.64649094358538e-08
3455 1.63394933139216e-08
3456 1.62172977269393e-08
3457 1.60979425345431e-08
3458 1.64839519811721e-08
3459 1.63992517343559e-08
3460 1.67335247880374e-08
3461 1.66715619087654e-08
3462 1.61278794763575e-08
3463 1.68759051177858e-08
3464 1.66014899605216e-08
3465 1.65198272839007e-08
3466 1.61372035734075e-08
3467 1.64978271044447e-08
3468 1.6212478470834e-08
3469 1.63886255677426e-08
3470 1.60955480055236e-08
3471 1.62371858181132e-08
3472 1.61541073850913e-08
3473 1.6129433788592e-08
3474 1.62270641368423e-08
3475 1.61907234286218e-08
3476 1.61207989179957e-08
3477 1.59804969257493e-08
3478 1.62585163110407e-08
3479 1.65944662455786e-08
3480 1.6483477693896e-08
3481 1.68664069377655e-08
3482 1.57904942454934e-08
3483 1.67776441628575e-08
3484 1.65009890196188e-08
3485 1.61342175175605e-08
3486 1.63963260746414e-08
3487 1.60779318747473e-08
3488 1.61760596029126e-08
3489 1.63816178400111e-08
3490 1.60047584074618e-08
3491 1.62118496405128e-08
3492 1.60836624019112e-08
3493 1.61714126534207e-08
3494 1.65825024822652e-08
3495 1.63569726652213e-08
3496 1.59637068009033e-08
3497 1.60118496239647e-08
3498 1.60545461369566e-08
3499 1.60158357687123e-08
3500 1.63756421756034e-08
3501 1.62179798479656e-08
3502 1.60647051217211e-08
3503 1.63050053458846e-08
3504 1.62749866916556e-08
3505 1.62679221205053e-08
3506 1.66218061536938e-08
3507 1.61657744968124e-08
3508 1.61033177903391e-08
3509 1.64936402313742e-08
3510 1.5906625350226e-08
3511 1.62538800196899e-08
3512 1.61545159471643e-08
3513 1.61874957882446e-08
3514 1.62642166401383e-08
3515 1.62482365340111e-08
3516 1.61816249288904e-08
3517 1.62284319316086e-08
3518 1.65902562798692e-08
3519 1.61968909395682e-08
3520 1.61722883973425e-08
3521 1.62330451303205e-08
3522 1.62159548011687e-08
3523 1.60131499171712e-08
3524 1.63282560805555e-08
3525 1.62785269708365e-08
3526 1.61597117909196e-08
3527 1.61166813228419e-08
3528 1.66952514035756e-08
3529 1.61454511982129e-08
3530 1.60901336698771e-08
3531 1.63965001576116e-08
3532 1.5887987814267e-08
3533 1.62429607541981e-08
3534 1.61462860859274e-08
3535 1.63696132204905e-08
3536 1.68301177438934e-08
3537 1.65693414544421e-08
3538 1.60184097097726e-08
3539 1.59916631048418e-08
3540 1.6217997611534e-08
3541 1.61639199802721e-08
3542 1.59768873686517e-08
3543 1.62152673510718e-08
3544 1.60600972520797e-08
3545 1.5830384114679e-08
3546 1.60755835310056e-08
3547 1.65213887015625e-08
3548 1.58210244904922e-08
3549 1.62435878081624e-08
3550 1.6273002501066e-08
3551 1.61872186765777e-08
3552 1.5862006819134e-08
3553 1.61306896728775e-08
3554 1.60068758248144e-08
3555 1.61297677436778e-08
3556 1.60512989566541e-08
3557 1.61975410861714e-08
3558 1.64022058157798e-08
3559 1.61073714366466e-08
3560 1.62325637376171e-08
3561 1.61537077048024e-08
3562 1.61848685564792e-08
3563 1.65783671235431e-08
3564 1.67889453450698e-08
3565 1.61796336328734e-08
3566 1.62963118555126e-08
3567 1.60553792483142e-08
3568 1.63529794150463e-08
3569 1.63443907297278e-08
3570 1.71043499364032e-08
3571 1.59136277488869e-08
3572 1.60509454616431e-08
3573 1.60406798954682e-08
3574 1.59671884603085e-08
3575 1.60742548160897e-08
3576 1.62101123635239e-08
3577 1.62506896828063e-08
3578 1.61671529497198e-08
3579 1.61114783736593e-08
3580 1.62377382650902e-08
3581 1.63034261646544e-08
3582 1.6255382817576e-08
3583 1.65307447730356e-08
3584 1.64920361811483e-08
3585 1.67532956396599e-08
3586 1.6224843690793e-08
3587 1.61265418796575e-08
3588 1.62285775928694e-08
3589 1.61978466195478e-08
3590 1.6052339901762e-08
3591 1.626896484197e-08
3592 1.6070689667913e-08
3593 1.60725601716649e-08
3594 1.6301704874877e-08
3595 1.61134980913857e-08
3596 1.59872239891001e-08
3597 1.59875490624017e-08
3598 1.61097766238072e-08
3599 1.63674300779348e-08
3600 1.64735318719522e-08
3601 1.65466165213957e-08
3602 1.61514233099069e-08
3603 1.63547948517362e-08
3604 1.61853890290331e-08
3605 1.59450568304464e-08
3606 1.60574398222479e-08
3607 1.6114745093887e-08
3608 1.61335993453804e-08
3609 1.62797153535621e-08
3610 1.69408362893364e-08
3611 1.69246394676748e-08
3612 1.63847815315421e-08
3613 1.60209960853308e-08
3614 1.62476023746194e-08
3615 1.61926845265725e-08
3616 1.64631046573049e-08
3617 1.61313273849828e-08
3618 1.60916631131158e-08
3619 1.63747859716068e-08
3620 1.63700750732687e-08
3621 1.61955906463618e-08
3622 1.61317483815537e-08
3623 1.68493006214021e-08
3624 1.60977737806434e-08
3625 1.60856750142102e-08
3626 1.61852966584775e-08
3627 1.61690003608328e-08
3628 1.62557345362302e-08
3629 1.64458704432491e-08
3630 1.65929492368377e-08
3631 1.65413105435164e-08
3632 1.62952105142722e-08
3633 1.60438382579287e-08
3634 1.60641580038146e-08
3635 1.62785873669691e-08
3636 1.62700430905716e-08
3637 1.6183886231147e-08
3638 1.62314837126587e-08
3639 1.62205147091754e-08
3640 1.61350328653498e-08
3641 1.62559548044783e-08
3642 1.61429269951441e-08
3643 1.61621311889348e-08
3644 1.60585003072811e-08
3645 1.65039999444616e-08
3646 1.6692769833071e-08
3647 1.64000457658631e-08
3648 1.63067799263672e-08
3649 1.62004702985996e-08
3650 1.61885154170704e-08
3651 1.62689968163932e-08
3652 1.57563562197538e-08
3653 1.65010476393945e-08
3654 1.59753987816202e-08
3655 1.62806514936165e-08
3656 1.64479434516807e-08
3657 1.61092561512532e-08
3658 1.60472897192676e-08
3659 1.60730753151483e-08
3660 1.6026602267516e-08
3661 1.65227564963288e-08
3662 1.63985784951137e-08
3663 1.62694639982419e-08
3664 1.64987987716358e-08
3665 1.60674300531127e-08
3666 1.60748605537719e-08
3667 1.6113716583277e-08
3668 1.61392428310592e-08
3669 1.62082223198468e-08
3670 1.61834048384435e-08
3671 1.59577933089849e-08
3672 1.61734874382091e-08
3673 1.61247513119633e-08
3674 1.59731978754962e-08
3675 1.60742601451602e-08
3676 1.59097908181138e-08
3677 1.6285822468376e-08
3678 1.6456910501006e-08
3679 1.6168323568877e-08
3680 1.62130593395204e-08
3681 1.61092792438922e-08
3682 1.5995476942976e-08
3683 1.62533719816338e-08
3684 1.62729563157882e-08
3685 1.6020518245341e-08
3686 1.59211435146744e-08
3687 1.62747131327023e-08
3688 1.66788520772343e-08
3689 1.62383262392041e-08
3690 1.5852460677479e-08
3691 1.62329900632585e-08
3692 1.62979141293818e-08
3693 1.62276911908066e-08
3694 1.6469950736564e-08
3695 1.62201381215255e-08
3696 1.60051119024729e-08
3697 1.62516933244206e-08
3698 1.56280091090366e-08
3699 1.60835345042187e-08
3700 1.60640727386863e-08
3701 1.60342121802159e-08
3702 1.66297713377617e-08
3703 1.6662861312966e-08
3704 1.65163385190681e-08
3705 1.61694391209721e-08
3706 1.58874922107088e-08
3707 1.6054883644756e-08
3708 1.61242894591851e-08
3709 1.56265258510757e-08
3710 1.63229980643109e-08
3711 1.63160649435667e-08
3712 1.64644955447102e-08
3713 1.60738320431619e-08
3714 1.60119100200973e-08
3715 1.61318371993957e-08
3716 1.62519402380212e-08
3717 1.59728816839788e-08
3718 1.57584274518285e-08
3719 1.62884372656436e-08
3720 1.64257549783997e-08
3721 1.56982249421844e-08
3722 1.62096984723803e-08
3723 1.59344679673268e-08
3724 1.57688440083348e-08
3725 1.64164344340634e-08
3726 1.57222075358732e-08
3727 1.61203370652174e-08
3728 1.603326005295e-08
3729 1.57819641799506e-08
3730 1.6211128439636e-08
3731 1.60859610076614e-08
3732 1.56520147953643e-08
3733 1.61997419922955e-08
3734 1.64277569325577e-08
3735 1.62336366571481e-08
3736 1.61501549911236e-08
3737 1.62129900616037e-08
3738 1.5960536003945e-08
3739 1.6104413802509e-08
3740 1.5863721003484e-08
3741 1.62031064121493e-08
3742 1.56896167169407e-08
3743 1.60877551280691e-08
3744 1.61607189852475e-08
3745 1.61565374412476e-08
3746 1.6076430853218e-08
3747 1.62881459431219e-08
3748 1.5820816656742e-08
3749 1.61597188963469e-08
3750 1.64608611186168e-08
3751 1.62266324821303e-08
3752 1.62160151973012e-08
3753 1.61327715630932e-08
3754 1.60771342905264e-08
3755 1.64323470386307e-08
3756 1.60177293651032e-08
3757 1.59963740031799e-08
3758 1.60612962929463e-08
3759 1.61951980715003e-08
3760 1.61667355058626e-08
3761 1.64613869202412e-08
3762 1.61730326908582e-08
3763 1.5744065606782e-08
3764 1.64381201983588e-08
3765 1.64706115413082e-08
3766 1.59090660645234e-08
3767 1.60367328305711e-08
3768 1.6114446665938e-08
3769 1.59461386317616e-08
3770 1.60926738601574e-08
3771 1.64202127450608e-08
3772 1.58303077313349e-08
3773 1.61443907131797e-08
3774 1.59586228676289e-08
3775 1.66700200310288e-08
3776 1.59714197423e-08
3777 1.61368749473922e-08
3778 1.62043534146505e-08
3779 1.6194537266756e-08
3780 1.65339422153465e-08
3781 1.58713522324661e-08
3782 1.61887996341648e-08
3783 1.59510111785721e-08
3784 1.61426498834771e-08
3785 1.60431401496908e-08
3786 1.60886362010615e-08
3787 1.62688706950576e-08
3788 1.56746580159961e-08
3789 1.62595359398665e-08
3790 1.57281903057083e-08
3791 1.59189319504094e-08
3792 1.63890785387366e-08
3793 1.60345390298744e-08
3794 1.64627547150076e-08
3795 1.62529172342829e-08
3796 1.63011044662653e-08
3797 1.6119821921734e-08
3798 1.63872009295574e-08
3799 1.6014622516991e-08
3800 1.5624996407837e-08
3801 1.61818434207817e-08
3802 1.62701567774093e-08
3803 1.59560933354896e-08
3804 1.54969974630603e-08
3805 1.6171343375504e-08
3806 1.60546260730143e-08
3807 1.59050532744232e-08
3808 1.59452486769851e-08
3809 1.57447530568788e-08
3810 1.61650213215125e-08
3811 1.59923096987313e-08
3812 1.61779922791538e-08
3813 1.58860906651626e-08
3814 1.56085224745084e-08
3815 1.61239057661078e-08
3816 1.58457709176218e-08
3817 1.57508761589042e-08
3818 1.61630708817029e-08
3819 1.59045026038029e-08
3820 1.5982513090762e-08
3821 1.57878243811638e-08
3822 1.57469663975007e-08
3823 1.58616693113345e-08
3824 1.57826729463295e-08
3825 1.63015041465542e-08
3826 1.58143720341286e-08
3827 1.5914125128802e-08
3828 1.630111867712e-08
3829 1.60654778369462e-08
3830 1.64963882554048e-08
3831 1.64020850235147e-08
3832 1.58943063155448e-08
3833 1.55438311111311e-08
3834 1.57928372601646e-08
3835 1.62530700009711e-08
3836 1.61371200846361e-08
3837 1.58482382772718e-08
3838 1.57761075314511e-08
3839 1.55535175849764e-08
3840 1.55382551270122e-08
3841 1.5733911951088e-08
3842 1.5716114631914e-08
3843 1.56205999246595e-08
3844 1.58675597106139e-08
3845 1.57611044215855e-08
3846 1.54797206164403e-08
3847 1.59434545565773e-08
3848 1.57109472098682e-08
3849 1.57222945773583e-08
3850 1.57653481380748e-08
3851 1.58115636139655e-08
3852 1.56454760258384e-08
3853 1.56956563301947e-08
3854 1.55221382414084e-08
3855 1.58219357615508e-08
3856 1.56771768899944e-08
3857 1.57834829650483e-08
3858 1.57231188069318e-08
3859 1.5735471592393e-08
3860 1.56682720131585e-08
3861 1.57283430723965e-08
3862 1.56245683058387e-08
3863 1.58618664869437e-08
3864 1.57299613334771e-08
3865 1.56040549370573e-08
3866 1.60607704913218e-08
3867 1.58400528249558e-08
3868 1.56866697409441e-08
3869 1.58907500491523e-08
3870 1.56719597299571e-08
3871 1.58988289200579e-08
3872 1.56755035618517e-08
3873 1.56329846845438e-08
3874 1.58862878407717e-08
3875 1.5631400174243e-08
3876 1.56811879037377e-08
3877 1.56444617260831e-08
3878 1.57285082735825e-08
3879 1.58482738044086e-08
3880 1.57338639894533e-08
3881 1.58069521916104e-08
3882 1.60482329647493e-08
3883 1.58939510441769e-08
3884 1.56074815294005e-08
3885 1.58178909970275e-08
3886 1.59224811113745e-08
3887 1.58047139819928e-08
3888 1.57683714974155e-08
3889 1.5610318371273e-08
3890 1.58691193519189e-08
3891 1.56097730297233e-08
3892 1.58269841676884e-08
3893 1.58681370265867e-08
3894 1.55340043050956e-08
3895 1.58460515820025e-08
3896 1.58811559458627e-08
3897 1.57943613743328e-08
3898 1.58272612793553e-08
3899 1.59060817850332e-08
3900 1.59596105220317e-08
3901 1.61285438338155e-08
3902 1.55618931074741e-08
3903 1.56051811472935e-08
3904 1.54788004635975e-08
3905 1.58462789556779e-08
3906 1.53963171101168e-08
3907 1.56745194601626e-08
3908 1.56425574715513e-08
3909 1.5831954414125e-08
3910 1.53049928286464e-08
3911 1.57516666376978e-08
3912 1.57052557625548e-08
3913 1.58720183662808e-08
3914 1.57527733080087e-08
3915 1.57968660374763e-08
3916 1.58072808176257e-08
3917 1.56219428504301e-08
3918 1.56190296252134e-08
3919 1.53479327025252e-08
3920 1.5904236150277e-08
3921 1.59421897905077e-08
3922 1.58739812405884e-08
3923 1.57754858065573e-08
3924 1.58818043161091e-08
3925 1.5971673761328e-08
3926 1.59105262298453e-08
3927 1.59855346737459e-08
3928 1.62263429359655e-08
3929 1.58221720170104e-08
3930 1.59421666978687e-08
3931 1.57534927325287e-08
3932 1.59748125838632e-08
3933 1.58942690120512e-08
3934 1.57234900655112e-08
3935 1.55996904283029e-08
3936 1.58489452672939e-08
3937 1.56224722047682e-08
3938 1.59268207511332e-08
3939 1.58098618641134e-08
3940 1.57059041328012e-08
3941 1.57956048241203e-08
3942 1.5804976882805e-08
3943 1.56102899495636e-08
3944 1.56885899826875e-08
3945 1.58433639541045e-08
3946 1.56659627492672e-08
3947 1.58525441662505e-08
3948 1.56285917540799e-08
3949 1.58749831058458e-08
3950 1.57294053337864e-08
3951 1.56886805768863e-08
3952 1.5599921354692e-08
3953 1.57714268311793e-08
3954 1.56703823250837e-08
3955 1.56150345986816e-08
3956 1.52738657277496e-08
3957 1.58505688574451e-08
3958 1.58000492689325e-08
3959 1.57395554367668e-08
3960 1.57177257875674e-08
3961 1.57191077931884e-08
3962 1.58859947418932e-08
3963 1.58810262718134e-08
3964 1.56772603787658e-08
3965 1.56455062239047e-08
3966 1.59576671876493e-08
3967 1.57002499889813e-08
3968 1.54871120372491e-08
3969 1.56951571739228e-08
3970 1.56004862361669e-08
3971 1.56471884338316e-08
3972 1.5650137186185e-08
3973 1.5666518748958e-08
3974 1.56551376306879e-08
3975 1.56099844161872e-08
3976 1.57866697492182e-08
3977 1.58124837668083e-08
3978 1.56093218350861e-08
3979 1.55529988887793e-08
3980 1.55398023338194e-08
3981 1.56928088301811e-08
3982 1.5586980595117e-08
3983 1.57674193701496e-08
3984 1.56469699419404e-08
3985 1.56861315048218e-08
3986 1.55389781042459e-08
3987 1.57441153447735e-08
3988 1.57543880163757e-08
3989 1.56405928208869e-08
3990 1.58469042332854e-08
3991 1.59415822764686e-08
3992 1.583762809787e-08
3993 1.5812885223454e-08
3994 1.55501691523341e-08
3995 1.56534465389768e-08
3996 1.58010031725553e-08
3997 1.53947716796665e-08
3998 1.56804009776579e-08
3999 1.58935780092406e-08
4000 1.54277248753942e-08
4001 1.57799942002157e-08
4002 1.58599231525614e-08
4003 1.59220405748783e-08
4004 1.56269397422193e-08
4005 1.58131268079842e-08
4006 1.57445665394107e-08
4007 1.5473663239618e-08
4008 1.55010884128615e-08
4009 1.55551607150528e-08
4010 1.55756971764731e-08
4011 1.55052806150024e-08
4012 1.57452664240054e-08
4013 1.58015875939554e-08
4014 1.55753259178937e-08
4015 1.56800812334268e-08
4016 1.57505937181668e-08
4017 1.56389337035989e-08
4018 1.56934483186433e-08
4019 1.56489363689616e-08
4020 1.57075508155913e-08
4021 1.56953117169678e-08
4022 1.5791560059597e-08
4023 1.57589834515193e-08
4024 1.56889399249849e-08
4025 1.5534762809466e-08
4026 1.54856483192134e-08
4027 1.56250230531896e-08
4028 1.55704089621622e-08
4029 1.55075525754e-08
4030 1.58607065259275e-08
4031 1.60150861461261e-08
4032 1.57290163116386e-08
4033 1.57624491237129e-08
4034 1.57359956176606e-08
4035 1.54122350437547e-08
4036 1.56033301834668e-08
4037 1.57045150217527e-08
4038 1.58457460486261e-08
4039 1.56240815840647e-08
4040 1.54151127418345e-08
4041 1.5832396726978e-08
4042 1.58149582318856e-08
4043 1.55944928081908e-08
4044 1.56670996176445e-08
4045 1.5493384353249e-08
4046 1.56407775619982e-08
4047 1.55841632931697e-08
4048 1.57217545648791e-08
4049 1.58447228670866e-08
4050 1.56878918744496e-08
4051 1.54495243265274e-08
4052 1.570455587796e-08
4053 1.57192960870134e-08
4054 1.57868029759811e-08
4055 1.57413033718967e-08
4056 1.56914250482032e-08
4057 1.5680964082776e-08
4058 1.54161590160129e-08
4059 1.56851083232823e-08
4060 1.56577684151671e-08
4061 1.56895989533723e-08
4062 1.56108992399595e-08
4063 1.59102686581036e-08
4064 1.58560311547262e-08
4065 1.55821240355181e-08
4066 1.57456092608754e-08
4067 1.58683839401874e-08
4068 1.56862718370121e-08
4069 1.5534718400545e-08
4070 1.57208166484679e-08
4071 1.57311763615553e-08
4072 1.58339368283578e-08
4073 1.60029074436352e-08
4074 1.5937319020054e-08
4075 1.60200244181397e-08
4076 1.558742290797e-08
4077 1.57029944602982e-08
4078 1.53650940859507e-08
4079 1.59493698248525e-08
4080 1.56945976215184e-08
4081 1.53865133967201e-08
4082 1.56343329393849e-08
4083 1.60966973083987e-08
4084 1.5842649858655e-08
4085 1.60184967512578e-08
4086 1.61848863200476e-08
4087 1.58472097666618e-08
4088 1.55165054138706e-08
4089 1.60347166655583e-08
4090 1.56711390530972e-08
4091 1.58584310128163e-08
4092 1.62303752659909e-08
4093 1.5775677653096e-08
4094 1.57560844371574e-08
4095 1.59042112812813e-08
4096 1.5853622414852e-08
4097 1.56394044381614e-08
4098 1.60447761743399e-08
4099 1.5868510061523e-08
4100 1.55628558928811e-08
4101 1.57758410779252e-08
4102 1.56653978677923e-08
4103 1.57864938898911e-08
4104 1.57806034906116e-08
4105 1.58218806944888e-08
4106 1.57004169665242e-08
4107 1.59729633963934e-08
4108 1.55738479890033e-08
4109 1.56838702025652e-08
4110 1.5698439881362e-08
4111 1.58792445859035e-08
4112 1.56694550668135e-08
4113 1.59676893929372e-08
4114 1.56377701898691e-08
4115 1.56952477681216e-08
4116 1.56951003305039e-08
4117 1.5975578193661e-08
4118 1.59137858446456e-08
4119 1.589778797495e-08
4120 1.5734320513161e-08
4121 1.57789283861121e-08
4122 1.59430193491517e-08
4123 1.59070498995106e-08
4124 1.59922262099599e-08
4125 1.56679362817158e-08
4126 1.56816071239518e-08
4127 1.59569140123494e-08
4128 1.57973634173914e-08
4129 1.58935797855975e-08
4130 1.56058153066851e-08
4131 1.59132493848801e-08
4132 1.5769161976209e-08
4133 1.57485384733036e-08
4134 1.58178554698907e-08
4135 1.5951355791799e-08
4136 1.57582480397878e-08
4137 1.6324644747101e-08
4138 1.61601096948516e-08
4139 1.58152246854115e-08
4140 1.59760809026466e-08
4141 1.5883840021047e-08
4142 1.58529704918919e-08
4143 1.57441331083419e-08
4144 1.59048259007477e-08
4145 1.59135762345386e-08
4146 1.60682223082631e-08
4147 1.57697943592439e-08
4148 1.60961466377785e-08
4149 1.61293254308248e-08
4150 1.61405377951951e-08
4151 1.57658153199236e-08
4152 1.59583848358125e-08
4153 1.6237807543007e-08
4154 1.57603441408583e-08
4155 1.56109969395857e-08
4156 1.59575463953843e-08
4157 1.58168926844837e-08
4158 1.57801434141902e-08
4159 1.64333684438134e-08
4160 1.5905895267565e-08
4161 1.58809019268347e-08
4162 1.6081408205082e-08
4163 1.64027529336863e-08
4164 1.6176828765424e-08
4165 1.60855755382272e-08
4166 1.58733506339104e-08
4167 1.59323896298247e-08
4168 1.64281370729213e-08
4169 1.63641065142883e-08
4170 1.62221347466129e-08
4171 1.64113931333532e-08
4172 1.61490518735263e-08
4173 1.6044088724243e-08
4174 1.62051474461578e-08
4175 1.60470925436584e-08
4176 1.62481850196627e-08
4177 1.61108122398446e-08
4178 1.60669095805588e-08
4179 1.65137592489373e-08
4180 1.63812732267843e-08
4181 1.63999107627433e-08
4182 1.57391362165527e-08
4183 1.58442254871716e-08
4184 1.59535300525704e-08
4185 1.59076947170433e-08
4186 1.57267479039547e-08
4187 1.58338835376526e-08
4188 1.60872044574489e-08
4189 1.6522058388091e-08
4190 1.60890571976324e-08
4191 1.59574966573928e-08
4192 1.59963704504662e-08
4193 1.67048970212136e-08
4194 1.64577738104299e-08
4195 1.6288531412556e-08
4196 1.60478084154647e-08
4197 1.6132077007569e-08
4198 1.62655648949794e-08
4199 1.59656039500078e-08
4200 1.59616817541064e-08
4201 1.58647921466581e-08
4202 1.58498245639294e-08
4203 1.59501087892977e-08
4204 1.61018753885855e-08
4205 1.59234172514289e-08
4206 1.60161643947276e-08
4207 1.58700537156165e-08
4208 1.61598059378321e-08
4209 1.6123681945146e-08
4210 1.61230602202522e-08
4211 1.62130397995952e-08
4212 1.58250319515219e-08
4213 1.59551554190784e-08
4214 1.60072648469622e-08
4215 1.57991877358654e-08
4216 1.59201363203465e-08
4217 1.59837778568317e-08
4218 1.5979050971282e-08
4219 1.58929136517827e-08
4220 1.62205733289511e-08
4221 1.58220334611769e-08
4222 1.6171309624724e-08
4223 1.59907500574263e-08
4224 1.58815556261516e-08
4225 1.59494355500556e-08
4226 1.61051953995184e-08
4227 1.56725636912824e-08
4228 1.55285970748764e-08
4229 1.55925761191611e-08
4230 1.58615875989199e-08
4231 1.62268616321626e-08
4232 1.59142832245607e-08
4233 1.56988413380077e-08
4234 1.58738604483233e-08
4235 1.58351056711581e-08
4236 1.57422164193122e-08
4237 1.58469184441401e-08
4238 1.58003654604499e-08
4239 1.57606354633799e-08
4240 1.55382444688712e-08
4241 1.59160773449685e-08
4242 1.55082311437127e-08
4243 1.57161093028435e-08
4244 1.59253588094543e-08
4245 1.55350257102782e-08
4246 1.58378288261929e-08
4247 1.55453712125109e-08
4248 1.58024526797362e-08
4249 1.55117305666863e-08
4250 1.5851123080779e-08
4251 1.56461634759353e-08
4252 1.58951198869772e-08
4253 1.58456288090747e-08
4254 1.57555035684709e-08
4255 1.55817279079429e-08
4256 1.57655435373272e-08
4257 1.5867337666009e-08
4258 1.57191326621842e-08
4259 1.59943880362334e-08
4260 1.60619499922632e-08
4261 1.58012358753012e-08
4262 1.59317217196531e-08
4263 1.59621791340214e-08
4264 1.63433799826862e-08
4265 1.65585642974975e-08
4266 1.65559193021636e-08
4267 1.64515920886288e-08
4268 1.6206682218467e-08
4269 1.63451225887457e-08
4270 1.63936775265938e-08
4271 1.61366617845715e-08
4272 1.64108833189403e-08
4273 1.63806816999568e-08
4274 1.65930700291028e-08
4275 1.65143525521216e-08
4276 1.62050195484653e-08
4277 1.60694479944823e-08
4278 1.6114114487209e-08
4279 1.61750541849415e-08
4280 1.64302083049961e-08
4281 1.61086646244257e-08
4282 1.64759494936106e-08
4283 1.64402695901344e-08
4284 1.63460285307337e-08
4285 1.64444706740596e-08
4286 1.63091566918183e-08
4287 1.64010671710457e-08
4288 1.63410227571603e-08
4289 1.61134625642489e-08
4290 1.60761235434848e-08
4291 1.60451012476415e-08
4292 1.6138812952704e-08
4293 1.61597721870521e-08
4294 1.66645506283203e-08
4295 1.63680464737581e-08
4296 1.63806870290273e-08
4297 1.63531446162324e-08
4298 1.62657034508129e-08
4299 1.61385305119666e-08
4300 1.61681050769857e-08
4301 1.62202962172842e-08
4302 1.63769318106688e-08
4303 1.62081654764279e-08
4304 1.65982676492149e-08
4305 1.63998397084697e-08
4306 1.60467124032948e-08
4307 1.64450515427461e-08
4308 1.64022910809081e-08
4309 1.6747907949366e-08
4310 1.62509170564817e-08
4311 1.63376405737381e-08
4312 1.61761679606798e-08
4313 1.61023336886501e-08
4314 1.63411613129938e-08
4315 1.61035984547198e-08
4316 1.65287783460144e-08
4317 1.64136597646802e-08
4318 1.64371769528771e-08
4319 1.63464299873795e-08
4320 1.62013744642309e-08
4321 1.62377880030817e-08
4322 1.64087214926667e-08
4323 1.63101852024283e-08
4324 1.61659947650605e-08
4325 1.62950364313019e-08
4326 1.64078723940975e-08
4327 1.6472881725349e-08
4328 1.6283310699805e-08
4329 1.62738711395605e-08
4330 1.64551607895191e-08
4331 1.64141642500226e-08
4332 1.62764504096913e-08
4333 1.62701567774093e-08
4334 1.61967665945895e-08
4335 1.59533772858822e-08
4336 1.62118212188034e-08
4337 1.66037263937824e-08
4338 1.62484905530391e-08
4339 1.67825859875848e-08
4340 1.66844351667805e-08
4341 1.63590918589307e-08
4342 1.66802305301417e-08
4343 1.62861191199681e-08
4344 1.6376906941673e-08
4345 1.64034741345631e-08
4346 1.65168678734062e-08
4347 1.64933275925705e-08
4348 1.63640798689357e-08
4349 1.65123807960299e-08
4350 1.64796816193302e-08
4351 1.62631454969642e-08
4352 1.6594693619254e-08
4353 1.64126063850745e-08
4354 1.67007012663589e-08
4355 1.63003512909654e-08
4356 1.64516968936823e-08
4357 1.70781540020926e-08
4358 1.71389142877842e-08
4359 1.67319669230892e-08
4360 1.63023177179866e-08
4361 1.65596549805969e-08
4362 1.6137619240908e-08
4363 1.64394045043537e-08
4364 1.62790456670336e-08
4365 1.62615378940245e-08
4366 1.70574612212704e-08
4367 1.63289666232913e-08
4368 1.65844458166475e-08
4369 1.65568199150812e-08
4370 1.67586424737465e-08
4371 1.64940789915136e-08
4372 1.65234244065005e-08
4373 1.68998681715493e-08
4374 1.67375286963534e-08
4375 1.66854174921127e-08
4376 1.64488387355277e-08
4377 1.67952212137834e-08
4378 1.62815894100277e-08
4379 1.63085758231318e-08
4380 1.63339937131468e-08
4381 1.67158802355516e-08
4382 1.62263855685296e-08
4383 1.64906186483904e-08
4384 1.62931765856911e-08
4385 1.62617652677e-08
4386 1.62413442694742e-08
4387 1.65160063403391e-08
4388 1.61748214821955e-08
4389 1.65749316494157e-08
4390 1.68569567193799e-08
4391 1.64564717408666e-08
4392 1.64488991316603e-08
4393 1.6328449703451e-08
4394 1.63066928848821e-08
4395 1.6342442066275e-08
4396 1.62028257477687e-08
4397 1.65592410894533e-08
4398 1.62769566713905e-08
4399 1.62830833261296e-08
4400 1.63432627431348e-08
4401 1.69202216682152e-08
4402 1.63945284015199e-08
4403 1.61845541413186e-08
4404 1.65622786596487e-08
4405 1.68837210878792e-08
4406 1.65521285566683e-08
4407 1.62904250089468e-08
4408 1.63584825685348e-08
4409 1.62541784476389e-08
4410 1.63227529270671e-08
4411 1.6505710576098e-08
4412 1.62729012487262e-08
4413 1.62156688077175e-08
4414 1.62360329625244e-08
4415 1.63691105115049e-08
4416 1.65322582290628e-08
4417 1.63296878241681e-08
4418 1.62372515433162e-08
4419 1.63906932471036e-08
4420 1.63483289128408e-08
4421 1.63095208449704e-08
4422 1.63184719070841e-08
4423 1.62331694752993e-08
4424 1.62533897452022e-08
4425 1.67307163678743e-08
4426 1.61565125722518e-08
4427 1.65866467227715e-08
4428 1.63963491672803e-08
4429 1.67007154772136e-08
4430 1.6112123191192e-08
4431 1.65065152657462e-08
4432 1.68272418221704e-08
4433 1.63394737739964e-08
4434 1.66424172221014e-08
4435 1.63265045927119e-08
4436 1.63486664206403e-08
4437 1.64270606006767e-08
4438 1.64950044734269e-08
4439 1.65869593615753e-08
4440 1.65680908992272e-08
4441 1.6295876648087e-08
4442 1.62866324870947e-08
4443 1.62168021233811e-08
4444 1.64193707519189e-08
4445 1.64157256676845e-08
4446 1.62044404561357e-08
4447 1.62347131293927e-08
4448 1.62922084712136e-08
4449 1.62727840091748e-08
4450 1.6443692629764e-08
4451 1.61317252889148e-08
4452 1.62222537625212e-08
4453 1.62076467802308e-08
4454 1.60775641688815e-08
4455 1.64824669468544e-08
4456 1.60318638364743e-08
4457 1.63127520380613e-08
4458 1.62389230951021e-08
4459 1.61086433081437e-08
4460 1.66494622533264e-08
4461 1.63539404240964e-08
4462 1.68238738496029e-08
4463 1.64228435295399e-08
4464 1.66430247361404e-08
4465 1.64232112354057e-08
4466 1.64877338448832e-08
4467 1.65544982166921e-08
4468 1.63840923050884e-08
4469 1.63946918263491e-08
4470 1.66368074872025e-08
4471 1.59448205749868e-08
4472 1.59599391480469e-08
4473 1.59549209399756e-08
4474 1.59911213160058e-08
4475 1.59754929285327e-08
4476 1.61663500364284e-08
4477 1.59458277693147e-08
4478 1.61123878683611e-08
4479 1.6284605663941e-08
4480 1.60683288896735e-08
4481 1.61550133270794e-08
4482 1.68682365853101e-08
4483 1.65199249835268e-08
4484 1.61600404169349e-08
4485 1.61837512280272e-08
4486 1.59647566277954e-08
4487 1.61341038307228e-08
4488 1.62110787016445e-08
4489 1.63735407454624e-08
4490 1.63596016733436e-08
4491 1.62088955590889e-08
4492 1.61060818015812e-08
4493 1.67823319685567e-08
4494 1.6331647145762e-08
4495 1.63551749920998e-08
4496 1.63551270304652e-08
4497 1.63335833747169e-08
4498 1.63488955706725e-08
4499 1.64510733924317e-08
4500 1.63852416079635e-08
4501 1.62379443224836e-08
4502 1.63550097909138e-08
4503 1.69096630031618e-08
4504 1.6445694583922e-08
4505 1.62241597934099e-08
4506 1.61392836872665e-08
4507 1.63142992448684e-08
4508 1.61952709021307e-08
4509 1.68990439419758e-08
4510 1.63320859059013e-08
4511 1.65227369564036e-08
4512 1.63674211961506e-08
4513 1.63790510043782e-08
4514 1.68891709506624e-08
4515 1.61265845122216e-08
4516 1.61308317814246e-08
4517 1.61408770793514e-08
4518 1.66240763377346e-08
4519 1.59724873327605e-08
4520 1.62839572936946e-08
4521 1.65263305262897e-08
4522 1.72047975866008e-08
4523 1.6853704210007e-08
4524 1.63258011554035e-08
4525 1.62011026816344e-08
4526 1.65328142287535e-08
4527 1.6227089005838e-08
4528 1.63085775994887e-08
4529 1.61119082520145e-08
4530 1.60996105336153e-08
4531 1.66146634228426e-08
4532 1.60383546443654e-08
4533 1.62758819755027e-08
4534 1.60500466250824e-08
4535 1.65819997732797e-08
4536 1.63825397692108e-08
4537 1.62956865779051e-08
4538 1.60924340519841e-08
4539 1.60291317996553e-08
4540 1.59761057716423e-08
4541 1.60843089958007e-08
4542 1.60372586321955e-08
4543 1.65509455030133e-08
4544 1.60297801699016e-08
4545 1.60926312275933e-08
4546 1.62012536719658e-08
4547 1.60849040753419e-08
4548 1.68978537828934e-08
4549 1.71042682239886e-08
4550 1.66572462489967e-08
4551 1.72594827319017e-08
4552 1.64869131680234e-08
4553 1.63416178367015e-08
4554 1.65093734239008e-08
4555 1.64428328730537e-08
4556 1.67026499298117e-08
4557 1.61194257941588e-08
4558 1.64406017688634e-08
4559 1.5908925732333e-08
4560 1.64811151392996e-08
4561 1.65942299901189e-08
4562 1.60939315207997e-08
4563 1.61183368874163e-08
4564 1.70146279288019e-08
4565 1.60779052293947e-08
4566 1.60669930693302e-08
4567 1.65805982277334e-08
4568 1.64158340254517e-08
4569 1.58682187390013e-08
4570 1.68715637016703e-08
4571 1.66257105860268e-08
4572 1.60281352634684e-08
4573 1.64667088853321e-08
4574 1.66568163706415e-08
4575 1.62029465400337e-08
4576 1.59383155562409e-08
4577 1.60172781704659e-08
4578 1.69458473919804e-08
4579 1.73602963116082e-08
4580 1.60119757453003e-08
4581 1.61088209438276e-08
4582 1.59502331342765e-08
4583 1.67057816469196e-08
4584 1.60196549359171e-08
4585 1.66778697519021e-08
4586 1.63078688331098e-08
4587 1.67090430380767e-08
4588 1.62293964933724e-08
4589 1.63702242872432e-08
4590 1.70394578447031e-08
4591 1.61941482446082e-08
4592 1.6257816426446e-08
4593 1.61862541148139e-08
4594 1.62043694018621e-08
4595 1.66184772609768e-08
4596 1.61246198615572e-08
4597 1.68582872106526e-08
4598 1.63415307952164e-08
4599 1.6322156071169e-08
4600 1.62477000742456e-08
4601 1.60401683046985e-08
4602 1.67152922614378e-08
4603 1.62417617133315e-08
4604 1.65670268614804e-08
4605 1.61875099990993e-08
4606 1.68172729075877e-08
4607 1.67981877297052e-08
4608 1.6175169648136e-08
4609 1.60206923283113e-08
4610 1.68030780400841e-08
4611 1.69593299403914e-08
4612 1.65483253766752e-08
4613 1.67839697695626e-08
4614 1.6046820761062e-08
4615 1.63834172894894e-08
4616 1.68003619904766e-08
4617 1.65704587828941e-08
4618 1.61722404357079e-08
4619 1.63993618684799e-08
4620 1.70409926170123e-08
4621 1.70318017467253e-08
4622 1.60126063519783e-08
4623 1.60797011261593e-08
4624 1.64585820527918e-08
4625 1.62967008776604e-08
4626 1.6380900191848e-08
4627 1.65667124463198e-08
4628 1.64001736635555e-08
4629 1.67786140536919e-08
4630 1.66366085352365e-08
4631 1.68032734393364e-08
4632 1.62515938484376e-08
4633 1.58323985033348e-08
4634 1.59296824620014e-08
4635 1.62518709601045e-08
4636 1.60556226092012e-08
4637 1.68655525101258e-08
4638 1.62819322468977e-08
4639 1.59813087208249e-08
4640 1.69459877241707e-08
4641 1.61001985077291e-08
4642 1.63621383109103e-08
4643 1.60682311900473e-08
4644 1.67548517282512e-08
4645 1.67674283346742e-08
4646 1.60965534234947e-08
4647 1.62590705343746e-08
4648 1.61554289945798e-08
4649 1.60718904851365e-08
4650 1.69366902724732e-08
4651 1.61435984580294e-08
4652 1.69771432467769e-08
4653 1.67359779368326e-08
4654 1.61468456383318e-08
4655 1.64437850003196e-08
4656 1.71226037792849e-08
4657 1.73583458717985e-08
4658 1.66820512959021e-08
4659 1.62804170145137e-08
4660 1.58132014149714e-08
4661 1.66266254097991e-08
4662 1.61908637608121e-08
4663 1.66486096020435e-08
4664 1.60897961620776e-08
4665 1.59060160598301e-08
4666 1.6867971908141e-08
4667 1.66202678286709e-08
4668 1.65520237516148e-08
4669 1.62674744785818e-08
4670 1.65993210288207e-08
4671 1.66404365842254e-08
4672 1.67054086119833e-08
4673 1.61717217395108e-08
4674 1.64875029184941e-08
4675 1.6833839211472e-08
4676 1.60184292496979e-08
4677 1.64557896198403e-08
4678 1.6561589433195e-08
4679 1.67008753493292e-08
4680 1.63907536432362e-08
4681 1.67469877965232e-08
4682 1.63067941372219e-08
4683 1.6584152717769e-08
4684 1.63962035060194e-08
4685 1.63260285290789e-08
4686 1.6493730825573e-08
4687 1.64381788181345e-08
4688 1.65518159178646e-08
4689 1.62289346405942e-08
4690 1.66913753929521e-08
4691 1.70144183186949e-08
4692 1.66341216356614e-08
4693 1.65956208775242e-08
4694 1.62082987031908e-08
4695 1.65273945640365e-08
4696 1.60848863117735e-08
4697 1.63713824719025e-08
4698 1.65897802162362e-08
4699 1.60105706470404e-08
4700 1.6923994650142e-08
4701 1.66413105517904e-08
4702 1.66431526338329e-08
4703 1.6502115229855e-08
4704 1.67398823691656e-08
4705 1.68828613311689e-08
4706 1.6876798625276e-08
4707 1.64134625890711e-08
4708 1.63186140156313e-08
4709 1.66642859511512e-08
4710 1.61060356163034e-08
4711 1.62260036518092e-08
4712 1.69120575321813e-08
4713 1.67141607221311e-08
4714 1.62240780809952e-08
4715 1.60498174750501e-08
4716 1.65539386642877e-08
4717 1.65543863062112e-08
4718 1.71763634426725e-08
4719 1.62034332618077e-08
4720 1.60192161757777e-08
4721 1.64973954497327e-08
4722 1.69100040636749e-08
4723 1.64110236511306e-08
4724 1.61528976860836e-08
4725 1.63075668524471e-08
4726 1.6485225629026e-08
4727 1.6032474903227e-08
4728 1.6276111125535e-08
4729 1.6175700778831e-08
4730 1.65547895392137e-08
4731 1.64536189117825e-08
4732 1.69888139112118e-08
4733 1.64576299255259e-08
4734 1.64879807584839e-08
4735 1.61292881273312e-08
4736 1.58575552688944e-08
4737 1.61389106523302e-08
4738 1.71066467657965e-08
4739 1.66892100139648e-08
4740 1.58742139433343e-08
4741 1.65329243628776e-08
4742 1.67303184639422e-08
4743 1.70609482097461e-08
4744 1.65425007025988e-08
4745 1.63245363893338e-08
4746 1.67552816066063e-08
4747 1.6265492064349e-08
4748 1.66067906093303e-08
4749 1.64823621418009e-08
4750 1.62816551352307e-08
4751 1.66645239829677e-08
4752 1.6616207076936e-08
4753 1.57623123442363e-08
4754 1.64850337824873e-08
4755 1.66659539502234e-08
4756 1.66572995397019e-08
4757 1.65385216632785e-08
4758 1.70264460308545e-08
4759 1.69518941106617e-08
4760 1.59741802008284e-08
4761 1.63803424158004e-08
4762 1.62819784321755e-08
4763 1.6333361330112e-08
4764 1.6114357848096e-08
4765 1.59690731749151e-08
4766 1.70723879477919e-08
4767 1.67551181817771e-08
4768 1.63601026059723e-08
4769 1.64867568486216e-08
4770 1.67694640396121e-08
4771 1.64744378139403e-08
4772 1.64641917876907e-08
4773 1.64433551219645e-08
4774 1.75120611345392e-08
4775 1.66604152695982e-08
4776 1.6846303907414e-08
4777 1.62577808993092e-08
4778 1.65332831869591e-08
4779 1.68849041415342e-08
4780 1.66363722797769e-08
4781 1.66462363893061e-08
4782 1.63234101790977e-08
4783 1.66278812940845e-08
4784 1.67095528524897e-08
4785 1.62511373247298e-08
4786 1.67267746320476e-08
4787 1.64875011421373e-08
4788 1.66138640622648e-08
4789 1.6300045757589e-08
4790 1.60982693842016e-08
4791 1.60337059185167e-08
4792 1.61143081101045e-08
4793 1.64646163369753e-08
4794 1.70111018604757e-08
4795 1.63072080283655e-08
4796 1.64423106241429e-08
4797 1.60227067169671e-08
4798 1.67488725111298e-08
4799 1.61154165567723e-08
4800 1.58651207726734e-08
4801 1.64871654106946e-08
4802 1.62944751025407e-08
4803 1.64172284655706e-08
4804 1.67295191033645e-08
4805 1.66453038019654e-08
4806 1.73579142170865e-08
4807 1.6627025090088e-08
4808 1.64507447664164e-08
4809 1.62661280000975e-08
4810 1.63887285964393e-08
4811 1.67007527807073e-08
4812 1.72507022000445e-08
4813 1.60923523395695e-08
4814 1.61070126125651e-08
4815 1.62716169427313e-08
4816 1.7679688824046e-08
4817 1.63552993370786e-08
4818 1.65596762968789e-08
4819 1.6353757459342e-08
4820 1.64139155600651e-08
4821 1.62273074977293e-08
4822 1.69565854690745e-08
4823 1.66690625746924e-08
4824 1.69848437536757e-08
4825 1.65069611313129e-08
4826 1.61566529044421e-08
4827 1.63744857673009e-08
4828 1.58851030107598e-08
4829 1.60352868761038e-08
4830 1.65985998279439e-08
4831 1.6495622645607e-08
4832 1.71996710207623e-08
4833 1.64976512451176e-08
4834 1.71812253313419e-08
4835 1.60821702621661e-08
4836 1.64485491893629e-08
4837 1.64397206958711e-08
4838 1.69162355234675e-08
4839 1.67662825845127e-08
4840 1.61841366974613e-08
4841 1.59694959478429e-08
4842 1.63384648033116e-08
4843 1.6347707187947e-08
4844 1.65567488608076e-08
4845 1.66333222750836e-08
4846 1.65410263264221e-08
4847 1.69853446863044e-08
4848 1.66821454428145e-08
4849 1.61989976987797e-08
4850 1.64441935623927e-08
4851 1.61234812168232e-08
4852 1.65953455422141e-08
4853 1.63660445196001e-08
4854 1.62469557807299e-08
4855 1.66584346317222e-08
4856 1.63898938865259e-08
4857 1.63863180802082e-08
4858 1.7501433191569e-08
4859 1.61697908396263e-08
4860 1.64809375036157e-08
4861 1.62287889793333e-08
4862 1.64486078091386e-08
4863 1.58322777110698e-08
4864 1.65740523527802e-08
4865 1.65322511236354e-08
4866 1.60629873846574e-08
4867 1.63130984276449e-08
4868 1.64939031321865e-08
4869 1.69830283169858e-08
4870 1.67101177339646e-08
4871 1.69051439513623e-08
4872 1.6727179641407e-08
4873 1.69491851664816e-08
4874 1.66874958296148e-08
4875 1.58509330105971e-08
4876 1.65703912813342e-08
4877 1.61715245639016e-08
4878 1.63738054226314e-08
4879 1.64609961217366e-08
4880 1.5915595952265e-08
4881 1.73300556127742e-08
4882 1.67986531351971e-08
4883 1.67383777949226e-08
4884 1.70220140205402e-08
4885 1.70110237007748e-08
4886 1.63573137257345e-08
4887 1.63737148284326e-08
4888 1.66431384229782e-08
4889 1.65039750754659e-08
4890 1.65056839307454e-08
4891 1.63581290735237e-08
4892 1.63557345445042e-08
4893 1.61796727127239e-08
4894 1.61277338150967e-08
4895 1.64229216892409e-08
4896 1.65954165964877e-08
4897 1.81166175394765e-08
4898 1.61845843393849e-08
4899 1.65593920797846e-08
4900 1.68976637127116e-08
4901 1.69698264329554e-08
4902 1.61689559519118e-08
4903 1.67713132270819e-08
4904 1.65549440822588e-08
4905 1.70779124175624e-08
4906 1.63331996816396e-08
4907 1.62620956700721e-08
4908 1.62953028848278e-08
4909 1.60440940533135e-08
4910 1.66086024933065e-08
4911 1.66201612472605e-08
4912 1.62120326052673e-08
4913 1.66757665454043e-08
4914 1.62877391574057e-08
4915 1.61800404185897e-08
4916 1.68662008803722e-08
4917 1.63243232265131e-08
4918 1.69199676491871e-08
4919 1.68141465195504e-08
4920 1.62118816149359e-08
4921 1.63567204225501e-08
4922 1.67794222960538e-08
4923 1.59359618834287e-08
4924 1.60775499580268e-08
4925 1.62758748700753e-08
4926 1.71587331010414e-08
4927 1.61921072105997e-08
4928 1.69668403771084e-08
4929 1.66189852990328e-08
4930 1.63449449530617e-08
4931 1.67053997301991e-08
4932 1.6164175775657e-08
4933 1.81868795579021e-08
4934 1.65802944707139e-08
4935 1.67771876391498e-08
4936 1.6727501161995e-08
4937 1.65042237654234e-08
4938 1.62731428332563e-08
4939 1.61858793035208e-08
4940 1.64592250939677e-08
4941 1.6817457648699e-08
4942 1.66909597254516e-08
4943 1.67759512947896e-08
4944 1.65512599181739e-08
4945 1.67638116721491e-08
4946 1.66273128598959e-08
4947 1.68193885485834e-08
4948 1.69672009775468e-08
4949 1.67487694824331e-08
4950 1.76070944490903e-08
4951 1.63675100139926e-08
4952 1.627451773345e-08
4953 1.63720006440826e-08
4954 1.72449805546648e-08
4955 1.65946048014121e-08
4956 1.66327858153181e-08
4957 1.69149707573979e-08
4958 1.65578324384796e-08
4959 1.67612199675204e-08
4960 1.65230051862864e-08
4961 1.65268385643458e-08
4962 1.63726987523205e-08
4963 1.64004916314298e-08
4964 1.66195270878688e-08
4965 1.67721498911533e-08
4966 1.6824879267574e-08
4967 1.60783724112434e-08
4968 1.6649920553391e-08
4969 1.67227351965948e-08
4970 1.65377223027008e-08
4971 1.62663340574909e-08
4972 1.62856359509078e-08
4973 1.64569016192218e-08
4974 1.69278830952635e-08
4975 1.68503166975142e-08
4976 1.67239484483162e-08
4977 1.64024847038036e-08
4978 1.64279345682417e-08
4979 1.6598741936491e-08
4980 1.67492917313439e-08
4981 1.63485971427235e-08
4982 1.62315423324344e-08
4983 1.67199853962074e-08
4984 1.64352922382704e-08
4985 1.70734555382523e-08
4986 1.58966972918506e-08
4987 1.66751572550083e-08
4988 1.68119544952106e-08
4989 1.63686042498057e-08
4990 1.61316116020771e-08
4991 1.64939706337464e-08
4992 1.64414508674326e-08
4993 1.6116006307243e-08
4994 1.65523061923523e-08
4995 1.6534567492954e-08
4996 1.65672844332221e-08
4997 1.62344786502899e-08
4998 1.63299205269141e-08
4999 1.65501798932155e-08
};
\addlegendentry{Test}

\nextgroupplot[
title={Batch Size 2 $\hy$},
ymin=6.19347993920736e-09, ymax=1e-05,
]
\addplot [semithick, black, dashed]
table {%
0 0.00306741265612072
1 0.000265674245993068
2 0.000125299643250173
3 3.38868312203573e-05
4 1.492250543685e-05
5 1.15433040430801e-05
6 8.21631598362416e-06
7 5.38262238156761e-06
8 3.53436111006644e-06
9 2.53232460225661e-06
10 2.06409569288812e-06
11 1.73679744801269e-06
12 1.46765533981164e-06
13 1.22196761374571e-06
14 1.0470694309106e-06
15 9.33562057841009e-07
16 8.59054600717446e-07
17 8.19124568118834e-07
18 7.90674564870475e-07
19 7.7162313418544e-07
20 7.56275273131823e-07
21 7.39744081638349e-07
22 7.24924018603801e-07
23 7.12342167080937e-07
24 7.00742809842225e-07
25 6.85856914625305e-07
26 6.73236351108386e-07
27 6.6047132256708e-07
28 6.50258060604259e-07
29 6.42031441826951e-07
30 6.34580290698405e-07
31 6.28046531220505e-07
32 6.22305416801971e-07
33 6.17123799450336e-07
34 6.12420645909051e-07
35 6.08182137327606e-07
36 6.04290542874253e-07
37 5.99914742586805e-07
38 5.92247310043703e-07
39 5.83204968338524e-07
40 5.73931377432668e-07
41 5.62571699263437e-07
42 5.54365182828365e-07
43 5.50345088566662e-07
44 5.49173952883564e-07
45 5.46772859338462e-07
46 5.450192586125e-07
47 5.43522436547983e-07
48 5.42139762093052e-07
49 5.40908892075809e-07
50 5.39752543872174e-07
51 5.38640021041203e-07
52 5.37835667577147e-07
53 5.36530756534814e-07
54 5.35721194413252e-07
55 5.34924981403284e-07
56 5.34009887551923e-07
57 5.33192001415994e-07
58 5.32335491298719e-07
59 5.32201855981906e-07
60 5.31341320955781e-07
61 5.30468216554114e-07
62 5.29737567503519e-07
63 5.29071212640675e-07
64 5.28232694819719e-07
65 5.2750664344936e-07
66 5.26625893549459e-07
67 5.25929413118353e-07
68 5.25000883261795e-07
69 5.24638943876221e-07
70 5.23679289974321e-07
71 5.22931354894007e-07
72 5.22515048865468e-07
73 5.21412128572374e-07
74 5.21069384042061e-07
75 5.19876543445097e-07
76 5.19572606148211e-07
77 5.18771037664179e-07
78 5.18043747333108e-07
79 5.16917752067192e-07
80 5.16781307783099e-07
81 5.15951250259272e-07
82 5.1520221855883e-07
83 5.1441635990046e-07
84 5.13674896208238e-07
85 5.13745835452939e-07
86 5.13401695074966e-07
87 5.12931297699692e-07
88 5.12120954463624e-07
89 5.11215562450706e-07
90 5.10590166556746e-07
91 5.09565476347396e-07
92 5.08348262574954e-07
93 5.07235604870981e-07
94 5.06071636632077e-07
95 5.04817230764454e-07
96 5.03451791567189e-07
97 5.02101875782257e-07
98 5.00719251695925e-07
99 4.99430619129937e-07
100 4.97970350068266e-07
101 4.96458518587151e-07
102 4.94814463219084e-07
103 4.93358688801404e-07
104 4.91760010977727e-07
105 4.90182819387952e-07
106 4.88534152842002e-07
107 4.86832348059574e-07
108 4.85113314903085e-07
109 4.83916914896243e-07
110 4.82456481527338e-07
111 4.80236249518429e-07
112 4.78044777784126e-07
113 4.75873101998703e-07
114 4.73467966723318e-07
115 4.71654388369158e-07
116 4.68517658200662e-07
117 4.65623492069467e-07
118 4.62344689772687e-07
119 4.58739415325926e-07
120 4.55262914632826e-07
121 4.51475668928003e-07
122 4.47903539116057e-07
123 4.42651070045708e-07
124 4.36648704221998e-07
125 4.2895460388892e-07
126 4.20702152826413e-07
127 4.12556538603059e-07
128 4.02554077442385e-07
129 3.9235525562642e-07
130 3.8148185357656e-07
131 3.69948853331525e-07
132 3.58242733928371e-07
133 3.45941700698127e-07
134 3.33327418200025e-07
135 3.21185148519199e-07
136 3.09766489023477e-07
137 2.97795769825804e-07
138 2.8693194036622e-07
139 2.77139715851416e-07
140 2.68001394519324e-07
141 2.59416898884979e-07
142 2.51360330377981e-07
143 2.44530719600622e-07
144 2.38254591661269e-07
145 2.3253222150621e-07
146 2.27537018587132e-07
147 2.23117744077017e-07
148 2.1876264106524e-07
149 2.14791166693651e-07
150 2.11000089370472e-07
151 2.07510934952637e-07
152 2.04681670929796e-07
153 2.02007863562059e-07
154 1.99716280692819e-07
155 1.97379543103837e-07
156 1.95081052229318e-07
157 1.928601050043e-07
158 1.91035827848962e-07
159 1.89122343758674e-07
160 1.87362332023211e-07
161 1.85813777573429e-07
162 1.8432800461432e-07
163 1.82915838351949e-07
164 1.81612477310189e-07
165 1.80287330154849e-07
166 1.79067777817066e-07
167 1.7843645779303e-07
168 1.77189567621472e-07
169 1.76167037854835e-07
170 1.75476514234774e-07
171 1.74415554623941e-07
172 1.73325837471427e-07
173 1.72437495304933e-07
174 1.71515766207331e-07
175 1.70724038906611e-07
176 1.69951548684932e-07
177 1.69151426653413e-07
178 1.68472941009412e-07
179 1.67710027633161e-07
180 1.67050817463377e-07
181 1.6622893598095e-07
182 1.65564303310495e-07
183 1.64907508534373e-07
184 1.64232229207806e-07
185 1.63657641954917e-07
186 1.62804945846973e-07
187 1.62673482272502e-07
188 1.61965698953215e-07
189 1.6131525172236e-07
190 1.60774144983611e-07
191 1.60075956011196e-07
192 1.59579302214663e-07
193 1.59023682625548e-07
194 1.58449894712587e-07
195 1.58041162171552e-07
196 1.57555063040382e-07
197 1.57007285632282e-07
198 1.5650955608093e-07
199 1.56047117815028e-07
200 1.55566717586098e-07
201 1.55096615913064e-07
202 1.54614603805059e-07
203 1.54163878699376e-07
204 1.53684609289773e-07
205 1.53309156006531e-07
206 1.52818182016734e-07
207 1.52416530871169e-07
208 1.5197907235398e-07
209 1.51548656603162e-07
210 1.51121735547322e-07
211 1.50756044237421e-07
212 1.50309933928616e-07
213 1.4987995835325e-07
214 1.49666520836478e-07
215 1.49227317655654e-07
216 1.48807712974874e-07
217 1.48483563840007e-07
218 1.48068378336585e-07
219 1.47770001959424e-07
220 1.47398065041493e-07
221 1.47122620158591e-07
222 1.46601257685841e-07
223 1.46517019687376e-07
224 1.45929976864467e-07
225 1.45436858238868e-07
226 1.45555609354253e-07
227 1.451150202465e-07
228 1.44501960547938e-07
229 1.44140408439353e-07
230 1.43771233124035e-07
231 1.43379917112996e-07
232 1.43061548615653e-07
233 1.42677398472513e-07
234 1.42359261385061e-07
235 1.4212522951107e-07
236 1.41667987290317e-07
237 1.41740636669052e-07
238 1.41406615015049e-07
239 1.41074028015553e-07
240 1.40735942082082e-07
241 1.40283597674151e-07
242 1.39546979508243e-07
243 1.39393218104522e-07
244 1.39011018596147e-07
245 1.3865747674735e-07
246 1.38319080729365e-07
247 1.3836745033724e-07
248 1.37994613498726e-07
249 1.37639781943255e-07
250 1.37321556679559e-07
251 1.37021351090061e-07
252 1.36637636949821e-07
253 1.36448617264917e-07
254 1.35998164497941e-07
255 1.35772365226439e-07
256 1.35393158373853e-07
257 1.35134999624809e-07
258 1.34827004358407e-07
259 1.34426657765863e-07
260 1.34256318004455e-07
261 1.33986115562479e-07
262 1.3359408421576e-07
263 1.33412788987153e-07
264 1.32988049580618e-07
265 1.32812065404497e-07
266 1.32553299815319e-07
267 1.32207662040695e-07
268 1.31988359704405e-07
269 1.31739405759612e-07
270 1.31411598505382e-07
271 1.31142318841704e-07
272 1.30933265684696e-07
273 1.3059646544944e-07
274 1.30352719334814e-07
275 1.30028671764926e-07
276 1.29869443436181e-07
277 1.29727673522595e-07
278 1.29330524489601e-07
279 1.29012916838089e-07
280 1.28849016431509e-07
281 1.28583473425259e-07
282 1.28156405005742e-07
283 1.27994401100784e-07
284 1.27878645783897e-07
285 1.27399157883845e-07
286 1.2727746522323e-07
287 1.27093813520052e-07
288 1.2674337731533e-07
289 1.26533215268232e-07
290 1.26217393814132e-07
291 1.26036751923686e-07
292 1.25768541275217e-07
293 1.25434035487704e-07
294 1.25316867423786e-07
295 1.24971127020812e-07
296 1.2481641549722e-07
297 1.24543415904954e-07
298 1.24247162872404e-07
299 1.24068768242669e-07
300 1.23818662524755e-07
301 1.23539009616702e-07
302 1.23308325579941e-07
303 1.23167522036338e-07
304 1.22755260151486e-07
305 1.2261334526642e-07
306 1.2222478899615e-07
307 1.21994532175651e-07
308 1.21789199407973e-07
309 1.21570957311024e-07
310 1.21319993342306e-07
311 1.21059980292326e-07
312 1.20933775707655e-07
313 1.20590968956003e-07
314 1.20321317519712e-07
315 1.20052696539585e-07
316 1.19743351220225e-07
317 1.19650892515755e-07
318 1.1921819875349e-07
319 1.19143649057518e-07
320 1.18796560188095e-07
321 1.18701410397271e-07
322 1.18169253666145e-07
323 1.18150939143913e-07
324 1.17778584999839e-07
325 1.17548551012536e-07
326 1.17241185090933e-07
327 1.17058538379511e-07
328 1.1680952818971e-07
329 1.16541787862245e-07
330 1.16272201908574e-07
331 1.16040971132048e-07
332 1.15892756700764e-07
333 1.15688012522264e-07
334 1.15371529618158e-07
335 1.15224057518626e-07
336 1.14934525180299e-07
337 1.14717826691924e-07
338 1.14542823524966e-07
339 1.14360510098943e-07
340 1.14047950307228e-07
341 1.13884854991753e-07
342 1.1363092209915e-07
343 1.13427604325533e-07
344 1.13301459870985e-07
345 1.1298413348515e-07
346 1.12807649773949e-07
347 1.12593988695897e-07
348 1.1244121852072e-07
349 1.12211577953047e-07
350 1.11974465146769e-07
351 1.11832951364965e-07
352 1.11639281988318e-07
353 1.11470174017247e-07
354 1.11223472677335e-07
355 1.10973720599872e-07
356 1.1114702914039e-07
357 1.10952908787487e-07
358 1.1078093409167e-07
359 1.10562667719138e-07
360 1.10344840546173e-07
361 1.1024338904253e-07
362 1.09999896422464e-07
363 1.09886889553268e-07
364 1.09635297385946e-07
365 1.09562031573418e-07
366 1.09303229770319e-07
367 1.09228010626738e-07
368 1.09015342022123e-07
369 1.08889883162711e-07
370 1.08650894392381e-07
371 1.08575416302603e-07
372 1.08380525691731e-07
373 1.08223819775155e-07
374 1.08006261299032e-07
375 1.07884743817843e-07
376 1.07738390934919e-07
377 1.07600758840531e-07
378 1.07449023576622e-07
379 1.07303083465515e-07
380 1.06996801886972e-07
381 1.06937142038688e-07
382 1.06728224942509e-07
383 1.0661676623247e-07
384 1.06431977255461e-07
385 1.06291999772079e-07
386 1.06120482154859e-07
387 1.0584334710162e-07
388 1.05753763900429e-07
389 1.0574025044563e-07
390 1.05479485042137e-07
391 1.05314029236192e-07
392 1.05192325825287e-07
393 1.05098223087685e-07
394 1.04906101092705e-07
395 1.0474937975391e-07
396 1.04774729837942e-07
397 1.04490206949803e-07
398 1.04323630228942e-07
399 1.04302639450049e-07
400 1.04065464728409e-07
401 1.03910763385118e-07
402 1.03909218980958e-07
403 1.0362491693261e-07
404 1.03526212211547e-07
405 1.03512512910964e-07
406 1.03277046822114e-07
407 1.03165277032025e-07
408 1.03004509306448e-07
409 1.02855338362629e-07
410 1.02656471732798e-07
411 1.02603458033146e-07
412 1.02538084947978e-07
413 1.02447490842428e-07
414 1.02153471597699e-07
415 1.02119328443839e-07
416 1.01980842861282e-07
417 1.01801879716978e-07
418 1.01707440426368e-07
419 1.01568936151653e-07
420 1.01545653223956e-07
421 1.01465030748704e-07
422 1.01269779952506e-07
423 1.01158352042718e-07
424 1.00954768286421e-07
425 1.00854979087628e-07
426 1.00862418631831e-07
427 1.00656625467543e-07
428 1.0054948046756e-07
429 1.00577645247846e-07
430 1.00230382672595e-07
431 1.0021866715193e-07
432 1.00263921147592e-07
433 1.00170122439236e-07
434 9.99135936068374e-08
435 9.97377645068909e-08
436 9.96326191131658e-08
437 9.95097011464718e-08
438 9.95391910710852e-08
439 9.94335132957769e-08
440 9.92378193275245e-08
441 9.91329916717465e-08
442 9.90905630871008e-08
443 9.88923953624266e-08
444 9.89967125042401e-08
445 9.86707303383128e-08
446 9.86958525899251e-08
447 9.86552574357979e-08
448 9.85680406089262e-08
449 9.834825553523e-08
450 9.81714271861556e-08
451 9.80366192372406e-08
452 9.80381835876676e-08
453 9.80040747128319e-08
454 9.79117646466854e-08
455 9.7649625805718e-08
456 9.77059925804102e-08
457 9.75824092960265e-08
458 9.74108154856435e-08
459 9.7359272335007e-08
460 9.7253023671251e-08
461 9.71680386643303e-08
462 9.69555502429165e-08
463 9.69578147961281e-08
464 9.66265457580384e-08
465 9.66247500784334e-08
466 9.651211238626e-08
467 9.63909760101078e-08
468 9.63696629224797e-08
469 9.64362124276086e-08
470 9.61292290909688e-08
471 9.60709402010318e-08
472 9.59476476690702e-08
473 9.59065928727299e-08
474 9.57852590287578e-08
475 9.57153325604754e-08
476 9.56412270439744e-08
477 9.54865132475335e-08
478 9.54352015765103e-08
479 9.52058073944162e-08
480 9.5206508429535e-08
481 9.50620755071352e-08
482 9.50991108567001e-08
483 9.48219602406875e-08
484 9.47654664216202e-08
485 9.46235153801434e-08
486 9.45948978088307e-08
487 9.45191575769666e-08
488 9.4391575502395e-08
489 9.43160478119598e-08
490 9.42821972048646e-08
491 9.40874888119003e-08
492 9.41322512888476e-08
493 9.37289695789323e-08
494 9.36088168660421e-08
495 9.35543124827021e-08
496 9.35314301305778e-08
497 9.32931144272953e-08
498 9.32669726391167e-08
499 9.30721589735839e-08
500 9.30309696645093e-08
501 9.29977573372875e-08
502 9.25859461234912e-08
503 9.2615084446579e-08
504 9.25229503013769e-08
505 9.22628864090669e-08
506 9.25583905942995e-08
507 9.22428990747237e-08
508 9.20094165776098e-08
509 9.19211464331937e-08
510 9.20481786045313e-08
511 9.1530572090992e-08
512 9.13439922851644e-08
513 9.14117610120169e-08
514 9.11697103994635e-08
515 9.08400841850288e-08
516 9.06979108524775e-08
517 9.05483892389425e-08
518 9.04109625394778e-08
519 9.02954561483593e-08
520 9.01209986201579e-08
521 9.03764568286336e-08
522 8.98920193026509e-08
523 9.02074459815383e-08
524 8.97659428408248e-08
525 8.965605914657e-08
526 8.94485356378594e-08
527 8.93733531954766e-08
528 8.90835338639429e-08
529 8.91512617406942e-08
530 8.88776343257458e-08
531 8.87256957029514e-08
532 8.86619102947739e-08
533 8.85405211241697e-08
534 8.83568465226414e-08
535 8.82831207672519e-08
536 8.80177280669914e-08
537 8.79907853987838e-08
538 8.78978967571165e-08
539 8.7689247418421e-08
540 8.76242440600583e-08
541 8.74498181003824e-08
542 8.74552271514784e-08
543 8.72743120950981e-08
544 8.70612748662403e-08
545 8.70378788322412e-08
546 8.68600122077012e-08
547 8.66210667006539e-08
548 8.66559963176305e-08
549 8.62958748970577e-08
550 8.63804655301115e-08
551 8.6182624746467e-08
552 8.59430052345678e-08
553 8.58041641198515e-08
554 8.58253255466845e-08
555 8.55211456876059e-08
556 8.56302357982175e-08
557 8.52435867351176e-08
558 8.53124372760527e-08
559 8.51116057435375e-08
560 8.50099190892184e-08
561 8.47701996015138e-08
562 8.46672116857627e-08
563 8.43529215296401e-08
564 8.4261487110493e-08
565 8.41136966825973e-08
566 8.39507801762007e-08
567 8.40644052139128e-08
568 8.36533107380078e-08
569 8.36516451181435e-08
570 8.35909256499123e-08
571 8.35025898520136e-08
572 8.33121509247192e-08
573 8.32596252093865e-08
574 8.30480322131511e-08
575 8.30179635117334e-08
576 8.28964494223872e-08
577 8.2565429915471e-08
578 8.25836318201212e-08
579 8.2472876894446e-08
580 8.21982308704117e-08
581 8.21931952409427e-08
582 8.18718890797099e-08
583 8.19325687374528e-08
584 8.16627477474574e-08
585 8.15718781465025e-08
586 8.13227232471236e-08
587 8.12443410436181e-08
588 8.10000993749504e-08
589 8.09066510123468e-08
590 8.07106466637064e-08
591 8.05636762891693e-08
592 8.09365523939398e-08
593 8.01965561574303e-08
594 8.03885502865809e-08
595 8.00746956772791e-08
596 8.01113555719546e-08
597 7.97027806642703e-08
598 7.96567901402145e-08
599 7.96043301739546e-08
600 7.94477736547616e-08
601 7.93306346946743e-08
602 7.92507571483192e-08
603 7.89611197408169e-08
604 7.86317848828766e-08
605 7.84276330018097e-08
606 7.85616977158066e-08
607 7.81841739869726e-08
608 7.81198933411664e-08
609 7.78116617881075e-08
610 7.78058564784301e-08
611 7.75767251771642e-08
612 7.76207030630527e-08
613 7.7226832374655e-08
614 7.70228895065061e-08
615 7.69546900726281e-08
616 7.69505653328562e-08
617 7.64382393998453e-08
618 7.6704942049699e-08
619 7.64071159532476e-08
620 7.62479964800944e-08
621 7.59253143525207e-08
622 7.60353703270766e-08
623 7.59156913178183e-08
624 7.545417894117e-08
625 7.55600338459006e-08
626 7.55454969452174e-08
627 7.52707234745742e-08
628 7.50383889003281e-08
629 7.51719401479445e-08
630 7.48773085263998e-08
631 7.47501040616605e-08
632 7.44542932524794e-08
633 7.41664930630215e-08
634 7.41005779802961e-08
635 7.39756929504232e-08
636 7.36763772687388e-08
637 7.35612952774689e-08
638 7.34394317847276e-08
639 7.31431165564844e-08
640 7.32798016146452e-08
641 7.30825932537815e-08
642 7.29585219824136e-08
643 7.26401978294877e-08
644 7.25219842027114e-08
645 7.23493223760974e-08
646 7.22593529147719e-08
647 7.20460255435595e-08
648 7.18968045660962e-08
649 7.15276212329696e-08
650 7.15568331285388e-08
651 7.13156414648974e-08
652 7.11498424329493e-08
653 7.10890518599294e-08
654 7.10073586079885e-08
655 7.07730809645923e-08
656 7.04866815914862e-08
657 7.05557814283386e-08
658 7.03524198435446e-08
659 7.01162037815006e-08
660 6.99810506342713e-08
661 6.99224218196903e-08
662 6.94651437532778e-08
663 6.9530984473265e-08
664 6.93053189426696e-08
665 6.90407779735391e-08
666 6.88480138956171e-08
667 6.8582191737887e-08
668 6.84467945063494e-08
669 6.82718962696782e-08
670 6.80341752595792e-08
671 6.78828539403264e-08
672 6.77208140138497e-08
673 6.76242986338282e-08
674 6.7386692047755e-08
675 6.73066241554388e-08
676 6.71914398454598e-08
677 6.69543023604779e-08
678 6.7064076818335e-08
679 6.65473798679983e-08
680 6.66374284821192e-08
681 6.63615867073641e-08
682 6.62389947699849e-08
683 6.61719003264283e-08
684 6.60291982490513e-08
685 6.57408365530854e-08
686 6.59239165741177e-08
687 6.54500899280874e-08
688 6.5428143957047e-08
689 6.51721163460461e-08
690 6.5140377194739e-08
691 6.4967269257532e-08
692 6.51810155373189e-08
693 6.51256574532022e-08
694 6.43957526277328e-08
695 6.42903876941769e-08
696 6.42616789727768e-08
697 6.41012382327499e-08
698 6.37745528587752e-08
699 6.37106172977076e-08
700 6.36259915511861e-08
701 6.31745437116304e-08
702 6.2983369589209e-08
703 6.28772586632742e-08
704 6.24339381558592e-08
705 6.25024825418041e-08
706 6.19965121692045e-08
707 6.1879161776246e-08
708 6.17481751437055e-08
709 6.15764399395369e-08
710 6.14676604779874e-08
711 6.12920367450975e-08
712 6.10521108466155e-08
713 6.08520522429323e-08
714 6.06596333331266e-08
715 6.05929553630924e-08
716 6.03738942350773e-08
717 6.00589574365484e-08
718 5.99962987486435e-08
719 5.98978085512014e-08
720 5.97794570991184e-08
721 5.95206624158795e-08
722 5.95323287794214e-08
723 5.90307601692208e-08
724 5.90409689706917e-08
725 5.88154682507058e-08
726 5.87140569194844e-08
727 5.84512538280846e-08
728 5.83432357781355e-08
729 5.825818050742e-08
730 5.79848779944658e-08
731 5.8045107534932e-08
732 5.78461136950814e-08
733 5.77380837332164e-08
734 5.7514333987041e-08
735 5.74051895585459e-08
736 5.7310055201043e-08
737 5.7182728948002e-08
738 5.6981602528372e-08
739 5.68558987984469e-08
740 5.67192003109174e-08
741 5.66558019543528e-08
742 5.69056805798107e-08
743 5.64915012340461e-08
744 5.63157721956142e-08
745 5.61189192815892e-08
746 5.60711149584225e-08
747 5.58798132230809e-08
748 5.55553268880016e-08
749 5.56175554761174e-08
750 5.55461281451231e-08
751 5.5139609520749e-08
752 5.50449699123212e-08
753 5.48997653507799e-08
754 5.47678269492979e-08
755 5.44755842753486e-08
756 5.45971252224886e-08
757 5.43908190882569e-08
758 5.41715743301308e-08
759 5.3922778538773e-08
760 5.39990367852683e-08
761 5.36951486611037e-08
762 5.38336117438809e-08
763 5.36852364874663e-08
764 5.32976519712069e-08
765 5.31818822989139e-08
766 5.31956040235593e-08
767 5.31610149329675e-08
768 5.29971575496946e-08
769 5.28984865318383e-08
770 5.26774844472055e-08
771 5.25724531360017e-08
772 5.23799425810045e-08
773 5.23108510258607e-08
774 5.21311004664238e-08
775 5.20626904423893e-08
776 5.22610680830926e-08
777 5.20248332623319e-08
778 5.16983238604274e-08
779 5.15767590351657e-08
780 5.14704047127057e-08
781 5.12529280972673e-08
782 5.10376587047889e-08
783 5.09008378715148e-08
784 5.08547375295265e-08
785 5.07224447564569e-08
786 5.05875744671314e-08
787 5.03973711082351e-08
788 5.03831155934664e-08
789 5.02286737819047e-08
790 5.01374685584754e-08
791 5.01281209256965e-08
792 4.99458375957662e-08
793 4.99318560078232e-08
794 4.99257999713176e-08
795 4.9939899219531e-08
796 4.97111985757748e-08
797 4.95077811699107e-08
798 4.94449857453838e-08
799 4.91151402815526e-08
800 4.92120070035296e-08
801 4.93138329861753e-08
802 4.90465021102393e-08
803 4.93697099269719e-08
804 4.89049392942698e-08
805 4.89493875543423e-08
806 4.91536222790856e-08
807 4.87713006360191e-08
808 4.86183735310397e-08
809 4.84042899051573e-08
810 4.8573817879094e-08
811 4.85080990330622e-08
812 4.83260303126287e-08
813 4.82373730481545e-08
814 4.81222262893599e-08
815 4.81885642473756e-08
816 4.81692429421532e-08
817 4.8017518253729e-08
818 4.77465199413829e-08
819 4.77471017165665e-08
820 4.76389284508882e-08
821 4.770065317794e-08
822 4.7597782838249e-08
823 4.74767562095302e-08
824 4.72787077394066e-08
825 4.71621813094569e-08
826 4.72879522611658e-08
827 4.71842946521694e-08
828 4.69818451943049e-08
829 4.68450605148885e-08
830 4.67135629327875e-08
831 4.68911635013725e-08
832 4.67314163835608e-08
833 4.64964929118294e-08
834 4.64953962842629e-08
835 4.61920341903732e-08
836 4.65707029425055e-08
837 4.64963107069116e-08
838 4.62222491912079e-08
839 4.60719237990737e-08
840 4.5970896745573e-08
841 4.58720779625832e-08
842 4.58993783972184e-08
843 4.57407792142095e-08
844 4.57403081224328e-08
845 4.55848450592322e-08
846 4.56220068692503e-08
847 4.52969827015592e-08
848 4.53362454415074e-08
849 4.52713056764975e-08
850 4.53151045082256e-08
851 4.50618464189478e-08
852 4.50117597622923e-08
853 4.48441608054129e-08
854 4.48051456433673e-08
855 4.45870395437975e-08
856 4.46878490383829e-08
857 4.45116273110058e-08
858 4.44413126114429e-08
859 4.45541892156176e-08
860 4.4345201820839e-08
861 4.44150699081547e-08
862 4.41560872543878e-08
863 4.41571751961245e-08
864 4.4013115911512e-08
865 4.40552572353314e-08
866 4.38993787784114e-08
867 4.39341473359445e-08
868 4.3427686712727e-08
869 4.34445921117521e-08
870 4.34019301368549e-08
871 4.32507974206797e-08
872 4.32477741511006e-08
873 4.30297705911187e-08
874 4.29691245804054e-08
875 4.28922119811159e-08
876 4.28605460819909e-08
877 4.29315272127973e-08
878 4.2611313348262e-08
879 4.25932218920955e-08
880 4.25147778056467e-08
881 4.24834442423472e-08
882 4.24085899455306e-08
883 4.23433829566111e-08
884 4.24646284318264e-08
885 4.22362609960869e-08
886 4.20039546396023e-08
887 4.21454146772304e-08
888 4.19191694611132e-08
889 4.17308967803343e-08
890 4.18082731102887e-08
891 4.16142392125063e-08
892 4.16794200172044e-08
893 4.1448585167414e-08
894 4.13782849030264e-08
895 4.13604698392689e-08
896 4.15024302536371e-08
897 4.12025377092284e-08
898 4.11557500289428e-08
899 4.11185960788596e-08
900 4.11491850210721e-08
901 4.09240660667054e-08
902 4.11158992482674e-08
903 4.08288245627997e-08
904 4.08298975963506e-08
905 4.0678412431272e-08
906 4.0594256854265e-08
907 4.02817207205697e-08
908 4.00301116202373e-08
909 4.04066312337803e-08
910 4.01790716487715e-08
911 3.99108652279767e-08
912 4.01199298653565e-08
913 3.98774671544477e-08
914 3.98928832915368e-08
915 3.98208671466005e-08
916 3.9611264293582e-08
917 3.97211665680297e-08
918 3.96226320063531e-08
919 3.92441238176722e-08
920 3.93905363831881e-08
921 3.90620920390217e-08
922 3.91074512091638e-08
923 3.94633986888215e-08
924 3.89573068269033e-08
925 3.88815601053971e-08
926 3.86664655587232e-08
927 3.86582599644636e-08
928 3.84013594389443e-08
929 3.83579269455647e-08
930 3.82271598144968e-08
931 3.79562154094737e-08
932 3.76373382701134e-08
933 3.76939190064873e-08
934 3.7514790357851e-08
935 3.73312320945951e-08
936 3.74319881659146e-08
937 3.74440865502645e-08
938 3.70531547866926e-08
939 3.69561198420021e-08
940 3.70382063645813e-08
941 3.66458968341088e-08
942 3.65840580799914e-08
943 3.67274926228456e-08
944 3.67472347588271e-08
945 3.58182675179175e-08
946 3.57980585682371e-08
947 3.56246858380094e-08
948 3.56956810734776e-08
949 3.52888255332195e-08
950 3.52709104017301e-08
951 3.50560956689505e-08
952 3.48873525198856e-08
953 3.49303549037061e-08
954 3.47922759399988e-08
955 3.46391465236295e-08
956 3.45630556820575e-08
957 3.44709076649519e-08
958 3.45706618315733e-08
959 3.40775471446286e-08
960 3.45298198463073e-08
961 3.39115107389487e-08
962 3.38780564629881e-08
963 3.37272811965494e-08
964 3.36009010908911e-08
965 3.36453391518998e-08
966 3.3465642234054e-08
967 3.37993323790975e-08
968 3.34670345817356e-08
969 3.32178836434105e-08
970 3.33054011066591e-08
971 3.35085671365476e-08
972 3.31654848998164e-08
973 3.30521118950822e-08
974 3.28728810946033e-08
975 3.27687213851902e-08
976 3.27917246662479e-08
977 3.27136559077623e-08
978 3.25911787914801e-08
979 3.25957754811201e-08
980 3.25793882108016e-08
981 3.24149784645678e-08
982 3.28143832783734e-08
983 3.23524663017327e-08
984 3.27007039085947e-08
985 3.18106616616354e-08
986 3.13396159400958e-08
987 3.16052015809642e-08
988 3.13869878620365e-08
989 3.1338116637164e-08
990 3.12829597451381e-08
991 3.11959312661592e-08
992 3.15697013792415e-08
993 3.11132560725513e-08
994 3.10159608524485e-08
995 3.09282212183348e-08
996 3.0700589540994e-08
997 3.05448706532652e-08
998 3.06292452211809e-08
999 3.03772127640822e-08
1000 3.03404357134551e-08
1001 3.01429844545287e-08
1002 3.00970518508359e-08
1003 3.00535915059852e-08
1004 3.0315382809204e-08
1005 3.01059454184127e-08
1006 3.00047103228007e-08
1007 3.01501612311739e-08
1008 2.99717358676344e-08
1009 2.98497170079237e-08
1010 2.97341266947693e-08
1011 2.96859480807821e-08
1012 2.95512835677414e-08
1013 2.94334922798511e-08
1014 2.94827021626309e-08
1015 2.94677459456438e-08
1016 2.93640036518328e-08
1017 2.93636311400336e-08
1018 2.92956224196561e-08
1019 2.92629967308944e-08
1020 2.92430699352364e-08
1021 2.90869081014389e-08
1022 2.90246588981746e-08
1023 2.90371275502577e-08
1024 2.90548178574213e-08
1025 2.90286503580317e-08
1026 2.89554002132486e-08
1027 2.88906269826916e-08
1028 2.90222820745201e-08
1029 2.89748512701249e-08
1030 2.86981681175691e-08
1031 2.87005700587706e-08
1032 2.86837088484537e-08
1033 2.85193363128711e-08
1034 2.84464543745533e-08
1035 2.8548776101367e-08
1036 2.84554390296843e-08
1037 2.84482274937603e-08
1038 2.83654316273196e-08
1039 2.84013983482545e-08
1040 2.82663853195531e-08
1041 2.82806659627211e-08
1042 2.8205769954659e-08
1043 2.81657453835016e-08
1044 2.81951215721499e-08
1045 2.80523647904762e-08
1046 2.81990824065881e-08
1047 2.81297992028584e-08
1048 2.80573478196877e-08
1049 2.82200504225782e-08
1050 2.80387469872578e-08
1051 2.84896597390616e-08
1052 2.80325885796406e-08
1053 2.80230724085118e-08
1054 2.8054887978024e-08
1055 2.79877611403245e-08
1056 2.78117660158506e-08
1057 2.78603977135194e-08
1058 2.79645356323632e-08
1059 2.78133377452638e-08
1060 2.77433113430314e-08
1061 2.78214822660994e-08
1062 2.77463107207221e-08
1063 2.77378879401935e-08
1064 2.75888625955734e-08
1065 2.76251288711671e-08
1066 2.76463700320528e-08
1067 2.75196810263778e-08
1068 2.74495258098528e-08
1069 2.74206748074324e-08
1070 2.76035365460725e-08
1071 2.74622620801135e-08
1072 2.76549107219859e-08
1073 2.73318041611459e-08
1074 2.73230314246642e-08
1075 2.72682154569548e-08
1076 2.72603326641474e-08
1077 2.72593684462197e-08
1078 2.72557354735192e-08
1079 2.7167911715531e-08
1080 2.73309542602185e-08
1081 2.72884519846861e-08
1082 2.72700480877797e-08
1083 2.70969281155331e-08
1084 2.71795298418254e-08
1085 2.71913304979443e-08
1086 2.72472436389504e-08
1087 2.72217029466115e-08
1088 2.71323154711345e-08
1089 2.70723734657197e-08
1090 2.70497277374826e-08
1091 2.72485932766298e-08
1092 2.69246238473464e-08
1093 2.7039346689739e-08
1094 2.68595349719458e-08
1095 2.68404578994685e-08
1096 2.68328779850502e-08
1097 2.68147954822773e-08
1098 2.68054033761289e-08
1099 2.70453040286112e-08
1100 2.67214803977844e-08
1101 2.68230361629707e-08
1102 2.68121698352719e-08
1103 2.67292634736349e-08
1104 2.67364943587756e-08
1105 2.66151566862827e-08
1106 2.67030677069402e-08
1107 2.6669573224003e-08
1108 2.65978784330378e-08
1109 2.65566182642285e-08
1110 2.66606438313055e-08
1111 2.67264584256355e-08
1112 2.65731244862288e-08
1113 2.65865874259641e-08
1114 2.65375709454785e-08
1115 2.65202695795275e-08
1116 2.64801884802546e-08
1117 2.65070363190678e-08
1118 2.63667136446433e-08
1119 2.65245470571773e-08
1120 2.65950877740284e-08
1121 2.63880594505439e-08
1122 2.63705405300907e-08
1123 2.63006169626823e-08
1124 2.62602964596348e-08
1125 2.64761795460422e-08
1126 2.61909668977933e-08
1127 2.63528627637843e-08
1128 2.6155696328245e-08
1129 2.62100663100395e-08
1130 2.62613176600646e-08
1131 2.62372813014666e-08
1132 2.60949013284884e-08
1133 2.62433103817294e-08
1134 2.62090552082261e-08
1135 2.60337244435971e-08
1136 2.61291853319578e-08
1137 2.60930714351404e-08
1138 2.61061540569874e-08
1139 2.60149745108518e-08
1140 2.61299909994905e-08
1141 2.56264310699583e-08
1142 2.57217699607959e-08
1143 2.50648269901343e-08
1144 2.53500022899533e-08
1145 2.55062641409953e-08
1146 2.50807165248224e-08
1147 2.55759655551635e-08
1148 2.49003677064985e-08
1149 2.50072273710877e-08
1150 2.54768939481342e-08
1151 2.48697166314593e-08
1152 2.55119624959499e-08
1153 2.54420942808475e-08
1154 2.54593143163562e-08
1155 2.50086457439791e-08
1156 2.49501794051254e-08
1157 2.48752862328783e-08
1158 2.48176208041129e-08
1159 2.52242706980377e-08
1160 2.51924385978319e-08
1161 2.48816096705129e-08
1162 2.50112441412542e-08
1163 2.50854124836386e-08
1164 2.47154663217675e-08
1165 2.46021244405814e-08
1166 2.4673301504019e-08
1167 2.45890170385843e-08
1168 2.49608011546054e-08
1169 2.44415283476018e-08
1170 2.47014364313236e-08
1171 2.46231828580634e-08
1172 2.44060256993361e-08
1173 2.47838693948343e-08
1174 2.44337023760643e-08
1175 2.45797811289916e-08
1176 2.48549346989202e-08
1177 2.45850966477668e-08
1178 2.44622395638716e-08
1179 2.44884400261991e-08
1180 2.47969102342427e-08
1181 2.4881336891075e-08
1182 2.49299477659559e-08
1183 2.51385400474036e-08
1184 2.50250900529769e-08
1185 2.49666905084156e-08
1186 2.5006053816945e-08
1187 2.51177447368467e-08
1188 2.49790565089725e-08
1189 2.49063803646477e-08
1190 2.50782126468008e-08
1191 2.49672681888746e-08
1192 2.49476203104948e-08
1193 2.48088653797551e-08
1194 2.49200330271204e-08
1195 2.4839017681938e-08
1196 2.48773374730005e-08
1197 2.48597363322889e-08
1198 2.48401449668145e-08
1199 2.48882575253351e-08
1200 2.4793455414257e-08
1201 2.49008748177959e-08
1202 2.49302128079443e-08
1203 2.48598319072257e-08
1204 2.47648221177732e-08
1205 2.49196235632132e-08
1206 2.48130790436452e-08
1207 2.48767926997462e-08
1208 2.46309686955382e-08
1209 2.49178240870618e-08
1210 2.46210239559785e-08
1211 2.47923548760687e-08
1212 2.48102059201283e-08
1213 2.46724411373522e-08
1214 2.46980960339205e-08
1215 2.46111671879101e-08
1216 2.46135100758216e-08
1217 2.45968024669974e-08
1218 2.47303142367206e-08
1219 2.46779919888807e-08
1220 2.45938245330979e-08
1221 2.46484527737789e-08
1222 2.45194752968025e-08
1223 2.45558327257311e-08
1224 2.47138142689673e-08
1225 2.45999685963005e-08
1226 2.4533136938315e-08
1227 2.44687231410745e-08
1228 2.44780293413849e-08
1229 2.44787720921624e-08
1230 2.44146814482593e-08
1231 2.45379507482768e-08
1232 2.45715218290532e-08
1233 2.43640193517691e-08
1234 2.45459432365036e-08
1235 2.45070031431038e-08
1236 2.44188560154235e-08
1237 2.44874072519996e-08
1238 2.44379819299978e-08
1239 2.4374482481021e-08
1240 2.44974896363159e-08
1241 2.45102784955098e-08
1242 2.43199136793493e-08
1243 2.45605573913066e-08
1244 2.42620351063871e-08
1245 2.4363525784582e-08
1246 2.43883911193454e-08
1247 2.41977286823958e-08
1248 2.43890788272638e-08
1249 2.422368943511e-08
1250 2.43319635999384e-08
1251 2.43426262961921e-08
1252 2.43132499165022e-08
1253 2.42569725439346e-08
1254 2.41398585177355e-08
1255 2.41634204675156e-08
1256 2.42218740564015e-08
1257 2.40888550487595e-08
1258 2.41580078369774e-08
1259 2.42422327252911e-08
1260 2.40484883421987e-08
1261 2.41578002918852e-08
1262 2.41428553929834e-08
1263 2.41044942130841e-08
1264 2.41060218404698e-08
1265 2.4187681370802e-08
1266 2.42517966043132e-08
1267 2.40671834782213e-08
1268 2.41087165782083e-08
1269 2.39705718996508e-08
1270 2.40682563576733e-08
1271 2.40282750016974e-08
1272 2.39482698921323e-08
1273 2.40710295697499e-08
1274 2.39062979558724e-08
1275 2.39759116141269e-08
1276 2.39799388018769e-08
1277 2.40100978687596e-08
1278 2.39160248624304e-08
1279 2.38947036383663e-08
1280 2.39618748618109e-08
1281 2.38598430404025e-08
1282 2.39126078803997e-08
1283 2.38971957355338e-08
1284 2.38634571276819e-08
1285 2.39061049153166e-08
1286 2.38883898827091e-08
1287 2.3894890822912e-08
1288 2.38157500636049e-08
1289 2.39122788620039e-08
1290 2.37436195848484e-08
1291 2.38808925865641e-08
1292 2.3808966678307e-08
1293 2.37931422864823e-08
1294 2.39069779964241e-08
1295 2.37997956144265e-08
1296 2.38142161922972e-08
1297 2.39235790932546e-08
1298 2.37803995968566e-08
1299 2.38881137069324e-08
1300 2.36573063225887e-08
1301 2.37071961530222e-08
1302 2.37993492752664e-08
1303 2.36581737053654e-08
1304 2.37971133421611e-08
1305 2.36432040169043e-08
1306 2.37641359001894e-08
1307 2.37819265639372e-08
1308 2.37994506411821e-08
1309 2.37373291379628e-08
1310 2.32301826701975e-08
1311 2.35628297735602e-08
1312 2.37714321615068e-08
1313 2.35145336927589e-08
1314 2.34808951392329e-08
1315 2.35864978717082e-08
1316 2.35630637858975e-08
1317 2.36115173107843e-08
1318 2.35408392223191e-08
1319 2.35872655866831e-08
1320 2.34638668739073e-08
1321 2.35019266352499e-08
1322 2.34587822889321e-08
1323 2.36395835846448e-08
1324 2.34182173300401e-08
1325 2.35031052728263e-08
1326 2.33686115068688e-08
1327 2.35588237717521e-08
1328 2.34474421555997e-08
1329 2.33681738344427e-08
1330 2.35111306569058e-08
1331 2.34673526680607e-08
1332 2.35931032256109e-08
1333 2.34243434903136e-08
1334 2.33732751684923e-08
1335 2.34740673394906e-08
1336 2.32258131506735e-08
1337 2.32826910816497e-08
1338 2.32420943461487e-08
1339 2.32803773397017e-08
1340 2.34037732327053e-08
1341 2.32419290492059e-08
1342 2.33164220988513e-08
1343 2.31981257389213e-08
1344 2.32294208419859e-08
1345 2.31379325843772e-08
1346 2.3367214685599e-08
1347 2.3141204893995e-08
1348 2.32905623816504e-08
1349 2.33508517994419e-08
1350 2.30935229443563e-08
1351 2.33256395837378e-08
1352 2.30883832663953e-08
1353 2.29866942995371e-08
1354 2.3204563178697e-08
1355 2.3142152985961e-08
1356 2.31186917570014e-08
1357 2.31887730811897e-08
1358 2.32120257382396e-08
1359 2.30343275562928e-08
1360 2.30456302469928e-08
1361 2.31936443494118e-08
1362 2.31301414762297e-08
1363 2.30044599763835e-08
1364 2.31375780198051e-08
1365 2.30704745868282e-08
1366 2.30777030858775e-08
1367 2.30791348765358e-08
1368 2.31016879653212e-08
1369 2.28983432646324e-08
1370 2.30448799834471e-08
1371 2.30509570185167e-08
1372 2.29865822116704e-08
1373 2.30397081490141e-08
1374 2.30524229671569e-08
1375 2.30572452594724e-08
1376 2.28223552253271e-08
1377 2.2871405571695e-08
1378 2.27332947856373e-08
1379 2.27796034031613e-08
1380 2.29461638145689e-08
1381 2.29773700202718e-08
1382 2.29596983293834e-08
1383 2.27613795674708e-08
1384 2.28502646492934e-08
1385 2.28805909563423e-08
1386 2.29414867579103e-08
1387 2.27074942757977e-08
1388 2.25753815161489e-08
1389 2.29310711725439e-08
1390 2.26808421276425e-08
1391 2.26279947052599e-08
1392 2.27283991741378e-08
1393 2.26060458916599e-08
1394 2.27983463508274e-08
1395 2.25396843407666e-08
1396 2.27252014355916e-08
1397 2.27918396346893e-08
1398 2.25302829553464e-08
1399 2.26612635048917e-08
1400 2.25813159309418e-08
1401 2.25905599813003e-08
1402 2.26748374222152e-08
1403 2.25498945695124e-08
1404 2.27402037797642e-08
1405 2.24752232795278e-08
1406 2.25891450567339e-08
1407 2.26373519762735e-08
1408 2.24762205071338e-08
1409 2.23858685605427e-08
1410 2.26485813743515e-08
1411 2.24384510328357e-08
1412 2.26161439588757e-08
1413 2.24280596801685e-08
1414 2.25554249739524e-08
1415 2.2390683969753e-08
1416 2.234397921691e-08
1417 2.23950081304491e-08
1418 2.24602976747756e-08
1419 2.25760247935525e-08
1420 2.23444690804764e-08
1421 2.23930679525419e-08
1422 2.2341832889794e-08
1423 2.25516577161966e-08
1424 2.23574106426039e-08
1425 2.25982192287311e-08
1426 2.23583331019905e-08
1427 2.23891319204028e-08
1428 2.24031955656789e-08
1429 2.25157341567084e-08
1430 2.24958275256149e-08
1431 2.22570661799482e-08
1432 2.23158140953972e-08
1433 2.23563048838693e-08
1434 2.24427357493895e-08
1435 2.22182223793832e-08
1436 2.2213215816913e-08
1437 2.23175801805364e-08
1438 2.22651225642578e-08
1439 2.22566202149888e-08
1440 2.24373737436057e-08
1441 2.24380027730176e-08
1442 2.21492510463217e-08
1443 2.21675983790859e-08
1444 2.22059090078386e-08
1445 2.23499171358721e-08
1446 2.24059688766487e-08
1447 2.23217548293853e-08
1448 2.22274128844069e-08
1449 2.20602442569939e-08
1450 2.20568415675027e-08
1451 2.19546345544275e-08
1452 2.19704955820299e-08
1453 2.21553452343093e-08
1454 2.21314972317643e-08
1455 2.21153343653024e-08
1456 2.20658533197082e-08
1457 2.18458593123083e-08
1458 2.21869252488294e-08
1459 2.18377026729721e-08
1460 2.21809622269753e-08
1461 2.20480463175021e-08
1462 2.17850938044328e-08
1463 2.19658442033588e-08
1464 2.19413879654073e-08
1465 2.17750560699836e-08
1466 2.20452150099526e-08
1467 2.23115310852007e-08
1468 2.18992396062156e-08
1469 2.1967560558861e-08
1470 2.19608136869387e-08
1471 2.20315787926584e-08
1472 2.19173254580873e-08
1473 2.1902231492732e-08
1474 2.20064815022181e-08
1475 2.20534625090807e-08
1476 2.20305229042539e-08
1477 2.19365018712947e-08
1478 2.19455514590128e-08
1479 2.16775116799461e-08
1480 2.2116683624146e-08
1481 2.21985585480144e-08
1482 2.17298344636352e-08
1483 2.20138761840849e-08
1484 2.19180922754469e-08
1485 2.16294716098298e-08
1486 2.17939938229883e-08
1487 2.20923234175718e-08
1488 2.18145028068328e-08
1489 2.18523100751211e-08
1490 2.16021246764175e-08
1491 2.2195382344381e-08
1492 2.19446128812695e-08
1493 2.16651854539063e-08
1494 2.17478094033663e-08
1495 2.18723493382311e-08
1496 2.20674949820976e-08
1497 2.1634100418666e-08
1498 2.19534887785089e-08
1499 2.18568329046487e-08
1500 2.17588902212962e-08
1501 2.18153812131527e-08
1502 2.17638364118644e-08
1503 2.18633377818411e-08
1504 2.18330283525969e-08
1505 2.19442601063768e-08
1506 2.17123475118042e-08
1507 2.17155698977733e-08
1508 2.16315048128246e-08
1509 2.19430536635701e-08
1510 2.1875963392648e-08
1511 2.16839639999578e-08
1512 2.18459783358216e-08
1513 2.1619693912428e-08
1514 2.1604695256916e-08
1515 2.16941721096764e-08
1516 2.17093602394025e-08
1517 2.15191328931108e-08
1518 2.18076480996698e-08
1519 2.17839623200267e-08
1520 2.14044261316604e-08
1521 2.1763885537332e-08
1522 2.14586050353849e-08
1523 2.15792708916918e-08
1524 2.15512557652509e-08
1525 2.15542904039578e-08
1526 2.17800064638007e-08
1527 2.15785088987508e-08
1528 2.15770320149133e-08
1529 2.15676620900496e-08
1530 2.16807773190386e-08
1531 2.15361292188743e-08
1532 2.1632858251075e-08
1533 2.1514078347773e-08
1534 2.17472498599269e-08
1535 2.19432286434618e-08
1536 2.1854052182857e-08
1537 2.15956059780864e-08
1538 2.13137516643924e-08
1539 2.15144808036749e-08
1540 2.1010893077017e-08
1541 2.18220968833116e-08
1542 2.14093134457138e-08
1543 2.17246232045443e-08
1544 2.15655568575335e-08
1545 2.17225187733316e-08
1546 2.15804941748055e-08
1547 2.14689658573075e-08
1548 2.1453424060125e-08
1549 2.1595096824284e-08
1550 2.17010759518133e-08
1551 2.14066381115652e-08
1552 2.15858224054677e-08
1553 2.14660792088595e-08
1554 2.14435637098664e-08
1555 2.14820213606248e-08
1556 2.12560290237529e-08
1557 2.13816004495937e-08
1558 2.12643492230047e-08
1559 2.15275386958036e-08
1560 2.14894381600494e-08
1561 2.15575967620063e-08
1562 2.15610979402425e-08
1563 2.10960668971805e-08
1564 2.12826811839262e-08
1565 2.15568268839561e-08
1566 2.13324893452438e-08
1567 2.14362246216265e-08
1568 2.13942897713537e-08
1569 2.10436940292491e-08
1570 2.1674325201726e-08
1571 2.11509561108825e-08
1572 2.13347523839724e-08
1573 2.12413111164989e-08
1574 2.130485384777e-08
1575 2.159389468262e-08
1576 2.14185590635685e-08
1577 2.12981211731855e-08
1578 2.11091723049117e-08
1579 2.11446224897827e-08
1580 2.12408123602126e-08
1581 2.15832629739121e-08
1582 2.12961506800002e-08
1583 2.12010743157742e-08
1584 2.11180149192469e-08
1585 2.12746201324332e-08
1586 2.1191863310871e-08
1587 2.12651064293301e-08
1588 2.12504446200912e-08
1589 2.09695097046214e-08
1590 2.16752988130009e-08
1591 2.13369720667711e-08
1592 2.11990286240749e-08
1593 2.14251407884003e-08
1594 2.13815723758837e-08
1595 2.15107287235561e-08
1596 2.11782314186448e-08
1597 2.12477787494558e-08
1598 2.1054579825458e-08
1599 2.08009021972766e-08
1600 2.09402850503138e-08
1601 2.12735845291079e-08
1602 2.13053134992813e-08
1603 2.11113799180873e-08
1604 2.07088874157324e-08
1605 2.13271530877523e-08
1606 2.13773497913516e-08
1607 2.12559840120363e-08
1608 2.11177596167456e-08
1609 2.06869521516551e-08
1610 2.08284238401857e-08
1611 2.1179984274311e-08
1612 2.09221216142808e-08
1613 2.08891321901994e-08
1614 2.12808381198637e-08
1615 2.15898037166651e-08
1616 2.10177449458671e-08
1617 2.05839649883577e-08
1618 2.12892448042401e-08
1619 2.09843657933417e-08
1620 2.06616991576258e-08
1621 2.07671416883881e-08
1622 2.1084484291195e-08
1623 2.10632622610896e-08
1624 2.11142641262374e-08
1625 2.11159610299938e-08
1626 2.05505632860925e-08
1627 2.09699418970144e-08
1628 2.08639915635078e-08
1629 2.11103937524382e-08
1630 2.06929038352455e-08
1631 2.10893949427748e-08
1632 2.05822990280435e-08
1633 2.10978145205876e-08
1634 2.10802599422688e-08
1635 2.13291546617145e-08
1636 2.08311776732639e-08
1637 2.11834660396593e-08
1638 2.11263702549758e-08
1639 2.05849119040979e-08
1640 2.0877008631931e-08
1641 2.1113550700308e-08
1642 2.11955532762875e-08
1643 2.07908993010153e-08
1644 2.08997627134111e-08
1645 2.08325599463377e-08
1646 2.08180089751042e-08
1647 2.08628834984237e-08
1648 2.10024294360622e-08
1649 2.05743568147398e-08
1650 2.0884930657189e-08
1651 2.09968160804486e-08
1652 2.04654416302852e-08
1653 2.09268698152798e-08
1654 2.07288363766045e-08
1655 2.12069696792772e-08
1656 2.09099855978945e-08
1657 2.05648199398667e-08
1658 2.08994720967137e-08
1659 2.06858995041026e-08
1660 2.08554451086862e-08
1661 2.09228216969393e-08
1662 2.07240321788149e-08
1663 2.04641679863726e-08
1664 2.08649423995566e-08
1665 2.06889850923597e-08
1666 2.05915213059849e-08
1667 2.08316939623587e-08
1668 2.07227751870265e-08
1669 2.05898613005184e-08
1670 2.08954456092647e-08
1671 2.07255912976489e-08
1672 2.06559103771986e-08
1673 2.0762275452807e-08
1674 2.04820639114756e-08
1675 2.11363346349913e-08
1676 2.05924834355553e-08
1677 2.03062320444092e-08
1678 2.09389010481509e-08
1679 2.07191078569757e-08
1680 2.06928298436271e-08
1681 2.05644544337691e-08
1682 2.05856897420309e-08
1683 2.06487629442198e-08
1684 2.07799675122911e-08
1685 2.08799896044554e-08
1686 2.05033568333735e-08
1687 2.07864206485386e-08
1688 2.05442319682059e-08
1689 2.05542292589789e-08
1690 2.06268362596684e-08
1691 2.05504129069944e-08
1692 2.06815796978954e-08
1693 2.06824139591527e-08
1694 2.07798020702421e-08
1695 2.06426836490969e-08
1696 2.06933217532235e-08
1697 2.04094608676486e-08
1698 2.06503047653628e-08
1699 2.10149856248454e-08
1700 2.05766771797788e-08
1701 2.05565499081517e-08
1702 2.06370847509879e-08
1703 2.05673910104176e-08
1704 2.06276611721956e-08
1705 2.03240175977171e-08
1706 2.04173433851623e-08
1707 2.05948747470652e-08
1708 2.06203560838003e-08
1709 2.05331069569881e-08
1710 2.00941817981859e-08
1711 2.09104845699248e-08
1712 2.01359624875663e-08
1713 2.04447542733466e-08
1714 2.02060434584261e-08
1715 2.04433028410944e-08
1716 2.04404358859489e-08
1717 2.05154025604559e-08
1718 2.00628208861797e-08
1719 2.03925354453904e-08
1720 2.06654586138999e-08
1721 2.05580346461787e-08
1722 2.03056144413016e-08
1723 2.04813294957251e-08
1724 2.02789414962545e-08
1725 2.06540468022653e-08
1726 2.00078098557543e-08
1727 2.05613757610912e-08
1728 1.98826677573205e-08
1729 2.03105072993648e-08
1730 2.05168724987104e-08
1731 2.0607967131131e-08
1732 2.05350168863971e-08
1733 2.00999105212374e-08
1734 2.06033309239906e-08
1735 2.03217713504178e-08
1736 2.01185197862275e-08
1737 2.04004799305435e-08
1738 2.04204468276237e-08
1739 2.0372937081653e-08
1740 2.04392249232299e-08
1741 2.03744672365586e-08
1742 2.05762911360402e-08
1743 2.01933256962078e-08
1744 2.02344745924565e-08
1745 2.03884946658739e-08
1746 2.06356511583683e-08
1747 2.02926470990322e-08
1748 2.03309683271535e-08
1749 2.06356617380998e-08
1750 1.99791632735924e-08
1751 2.03207506370984e-08
1752 2.02805551996754e-08
1753 2.05711006853182e-08
1754 2.00591988404053e-08
1755 2.02903076483574e-08
1756 2.03523719666254e-08
1757 2.03945092286106e-08
1758 2.02654201899299e-08
1759 2.03404908011295e-08
1760 2.00261232891907e-08
1761 2.04519979363771e-08
1762 2.01618393988745e-08
1763 2.05263291205571e-08
1764 2.03661774132169e-08
1765 2.03761915843204e-08
1766 1.98223823780486e-08
1767 2.03125573706997e-08
1768 2.03017542269868e-08
1769 2.01382567932562e-08
1770 2.0437263476919e-08
1771 1.98729806656639e-08
1772 2.05124185053696e-08
1773 2.02185616375999e-08
1774 2.0618078168777e-08
1775 2.04869251224926e-08
1776 2.03365603143879e-08
1777 2.02483284200694e-08
1778 2.02009458984409e-08
1779 2.03359455409335e-08
1780 2.01519287872654e-08
1781 2.02931824062336e-08
1782 1.99060148738384e-08
1783 2.0222356980501e-08
1784 2.01623016400365e-08
1785 2.00364656746249e-08
1786 2.00789627700232e-08
1787 2.03510798552631e-08
1788 2.00238263062119e-08
1789 2.01975323299852e-08
1790 2.01605255351622e-08
1791 2.03341814576097e-08
1792 2.01218253525481e-08
1793 1.99454512297415e-08
1794 2.00606496913169e-08
1795 2.02616337739853e-08
1796 2.00605314569224e-08
1797 2.02454070596103e-08
1798 2.00180937037497e-08
1799 2.0262806098445e-08
1800 2.00482491780896e-08
1801 2.04899482600662e-08
1802 2.00942948824501e-08
1803 2.01494838422828e-08
1804 2.02136417312565e-08
1805 2.02999213904376e-08
1806 2.02010621960236e-08
1807 1.98850193324218e-08
1808 2.03066975355826e-08
1809 2.03209297834628e-08
1810 1.97541084593078e-08
1811 2.00361994467657e-08
1812 1.99555611172658e-08
1813 2.00050446837463e-08
1814 2.02161354040942e-08
1815 2.00473846596327e-08
1816 1.98766767908892e-08
1817 1.99389833813457e-08
1818 2.0023910848932e-08
1819 1.98387563648955e-08
1820 2.00856314560988e-08
1821 1.95942871887567e-08
1822 1.97758822751692e-08
1823 2.02253719410184e-08
1824 1.95719200443423e-08
1825 2.02013035701665e-08
1826 1.97513118702364e-08
1827 1.97373259512579e-08
1828 2.02380019999338e-08
1829 2.00869210537635e-08
1830 1.99911085281002e-08
1831 1.96181095835912e-08
1832 1.98250683212386e-08
1833 2.00520951300076e-08
1834 1.98615729903595e-08
1835 1.99315610746043e-08
1836 1.97289158375275e-08
1837 2.00021623030788e-08
1838 1.98363328686585e-08
1839 2.03499582376726e-08
1840 1.96733146393263e-08
1841 1.9765084282572e-08
1842 2.00229430229876e-08
1843 2.00558941454709e-08
1844 1.95396757116661e-08
1845 2.02303131351667e-08
1846 1.95683447328232e-08
1847 1.98432514416369e-08
1848 1.96519741100798e-08
1849 1.99339962215356e-08
1850 2.01798387811508e-08
1851 1.98449714308713e-08
1852 1.99336030733549e-08
1853 1.98994413307257e-08
1854 1.98802682372656e-08
1855 1.97569266916497e-08
1856 2.00298759282097e-08
1857 1.98324837600017e-08
1858 1.96810571221895e-08
1859 1.98788089331575e-08
1860 1.98298827587123e-08
1861 1.97842545126697e-08
1862 1.99701629149535e-08
1863 1.96244701254067e-08
1864 1.9634184488429e-08
1865 1.96781413627084e-08
1866 1.98626553588876e-08
1867 1.94902458288959e-08
1868 2.00477638816199e-08
1869 1.96815742366241e-08
1870 1.99208859417432e-08
1871 1.97909554045944e-08
1872 1.96132286674339e-08
1873 1.95849275451965e-08
1874 2.022796929739e-08
1875 1.94558967947511e-08
1876 1.99972128580861e-08
1877 1.99072603370015e-08
1878 1.95484324466283e-08
1879 1.97254750509274e-08
1880 1.98149109023604e-08
1881 1.96220406484537e-08
1882 1.9604191703701e-08
1883 1.97088456949102e-08
1884 1.93849111722461e-08
1885 1.96615342436535e-08
1886 1.96219642950857e-08
1887 1.92782443043105e-08
1888 1.99535336221357e-08
1889 1.93009876887917e-08
1890 1.97123463689663e-08
1891 1.95538009867685e-08
1892 1.98720201110003e-08
1893 1.95670386833602e-08
1894 1.95022807448764e-08
1895 1.96044935331374e-08
1896 1.97432839596223e-08
1897 1.98664956388428e-08
1898 1.92883682990919e-08
1899 1.98250831747598e-08
1900 1.96418445924107e-08
1901 1.95021195741746e-08
1902 1.95380842446385e-08
1903 1.95250956781939e-08
1904 1.92465768590333e-08
1905 1.97758402860626e-08
1906 1.96032989128181e-08
1907 1.92600520028535e-08
1908 1.98875903361928e-08
1909 1.93579331004556e-08
1910 1.99504998459887e-08
1911 1.92930379204481e-08
1912 1.97679080110869e-08
1913 1.94152450516127e-08
1914 1.9951594106421e-08
1915 1.95767660975954e-08
1916 1.91605301879116e-08
1917 1.96986829582579e-08
1918 1.93312549982283e-08
1919 1.93949549022743e-08
1920 1.96671571384005e-08
1921 1.94502721710954e-08
1922 1.95824106766496e-08
1923 1.98598708809478e-08
1924 1.93831080103346e-08
1925 1.99786795775786e-08
1926 1.9240849446267e-08
1927 1.96717041909128e-08
1928 1.9498536947618e-08
1929 1.94655612861527e-08
1930 1.96799187964225e-08
1931 1.92652540269844e-08
1932 1.93867243986345e-08
1933 1.97296174595935e-08
1934 1.95504755817755e-08
1935 1.93475427211071e-08
1936 1.93255428594941e-08
1937 1.97144761731188e-08
1938 1.9813541291297e-08
1939 1.98540182436424e-08
1940 1.94708698346147e-08
1941 1.921447977693e-08
1942 1.92056466897517e-08
1943 1.92718005500581e-08
1944 1.9392477817487e-08
1945 1.96998231156292e-08
1946 1.90204510822634e-08
1947 1.93177467068817e-08
1948 1.94123624301379e-08
1949 1.96359767290905e-08
1950 1.96770460262341e-08
1951 1.9152041010051e-08
1952 1.96051102953621e-08
1953 1.91375670186011e-08
1954 1.9459682860129e-08
1955 1.93411022529899e-08
1956 1.92139490317322e-08
1957 1.9125655962432e-08
1958 1.95648559590256e-08
1959 1.95740048487825e-08
1960 1.95016391204561e-08
1961 1.93977425455155e-08
1962 1.94403299419998e-08
1963 1.95314536558522e-08
1964 1.94733612915721e-08
1965 1.93514317582688e-08
1966 1.91150949500063e-08
1967 1.93931412364734e-08
1968 1.92780238957441e-08
1969 1.96170210099433e-08
1970 1.91943090885188e-08
1971 1.94171391951492e-08
1972 1.93448941312874e-08
1973 1.9528755982906e-08
1974 1.93095868653015e-08
1975 1.92122389515714e-08
1976 1.93221856078507e-08
1977 1.92192917146949e-08
1978 1.96234646774041e-08
1979 1.93294758921436e-08
1980 1.9142667389227e-08
1981 1.92397936413374e-08
1982 1.92317584076507e-08
1983 1.94039790116507e-08
1984 1.90533443519825e-08
1985 1.96008755265209e-08
1986 1.96685334789004e-08
1987 1.90960910154969e-08
1988 1.93231098040703e-08
1989 1.92483560931267e-08
1990 1.91980247099682e-08
1991 1.95973273281191e-08
1992 1.90946657087632e-08
1993 1.95828070608667e-08
1994 1.90403451837406e-08
1995 1.91817503685798e-08
1996 1.93569613365085e-08
1997 1.89749003951523e-08
1998 1.90341474100991e-08
1999 1.92806212607338e-08
2000 1.95001595256622e-08
2001 1.9569759081739e-08
2002 1.88603637261697e-08
2003 1.87912502787463e-08
2004 1.95540734619082e-08
2005 1.89157689771402e-08
2006 1.92269206519546e-08
2007 1.89036638031659e-08
2008 1.9484714279025e-08
2009 1.925844141136e-08
2010 1.91850605117672e-08
2011 1.9361514174171e-08
2012 1.93027042797722e-08
2013 1.94606914629969e-08
2014 1.90078376242353e-08
2015 1.93305302437913e-08
2016 1.92045914218508e-08
2017 1.93583187935442e-08
2018 1.90000165055521e-08
2019 1.8921446251069e-08
2020 1.91843068779174e-08
2021 1.92367905300084e-08
2022 1.91720965206643e-08
2023 1.91385274544431e-08
2024 1.92930998821061e-08
2025 1.891102461965e-08
2026 1.90771696745029e-08
2027 1.90552360682383e-08
2028 1.90593086863838e-08
2029 1.92674425026074e-08
2030 1.91789513772356e-08
2031 1.94284758639474e-08
2032 1.92757090947682e-08
2033 1.91472942194237e-08
2034 1.88333998266538e-08
2035 1.89567890717668e-08
2036 1.90370879514096e-08
2037 1.89824016719187e-08
2038 1.91196438678876e-08
2039 1.87808825544283e-08
2040 1.93522414169006e-08
2041 1.92327113290869e-08
2042 1.89548787113414e-08
2043 1.92131244823313e-08
2044 1.91828084690759e-08
2045 1.91378335362424e-08
2046 1.88358792889698e-08
2047 1.95248990873254e-08
2048 1.92520936423968e-08
2049 1.90041264316043e-08
2050 1.91280492776863e-08
2051 1.87503265791367e-08
2052 1.88944447047645e-08
2053 1.92115503826196e-08
2054 1.90574377864905e-08
2055 1.91492080088562e-08
2056 1.90956524638575e-08
2057 1.88683441333382e-08
2058 1.91411564210708e-08
2059 1.85601504310018e-08
2060 1.91096277473851e-08
2061 1.92999464793675e-08
2062 1.8968121296925e-08
2063 1.89419762968435e-08
2064 1.90636848862169e-08
2065 1.91550455616307e-08
2066 1.90967455544894e-08
2067 1.88184356942622e-08
2068 1.90937190533147e-08
2069 1.93329740882653e-08
2070 1.88976632992943e-08
2071 1.91242242309209e-08
2072 1.91262933974812e-08
2073 1.89767341468305e-08
2074 1.90403417924812e-08
2075 1.91790767277711e-08
2076 1.90093438317551e-08
2077 1.91345447775049e-08
2078 1.88675213495249e-08
2079 1.90808443747276e-08
2080 1.87279708147303e-08
2081 1.91384518671567e-08
2082 1.89650598757807e-08
2083 1.93437475885377e-08
2084 1.88106883749345e-08
2085 1.92372454672074e-08
2086 1.8863133637595e-08
2087 1.90604794587101e-08
2088 1.8993061241146e-08
2089 1.86104034888557e-08
2090 1.86063758816635e-08
2091 1.91001678627734e-08
2092 1.89048179538992e-08
2093 1.91220613575682e-08
2094 1.88259980897776e-08
2095 1.89023411279188e-08
2096 1.89472294824106e-08
2097 1.86135440457258e-08
2098 1.91114191243485e-08
2099 1.89193545760968e-08
2100 1.86628482648499e-08
2101 1.87507535235332e-08
2102 1.91733591309723e-08
2103 1.85707093256493e-08
2104 1.89540190369403e-08
2105 1.88177945192325e-08
2106 1.89925054374651e-08
2107 1.88249654144435e-08
2108 1.88393897069539e-08
2109 1.89395022088257e-08
2110 1.87976924513056e-08
2111 1.89708682654943e-08
2112 1.86180294589472e-08
2113 1.90277981850229e-08
2114 1.9027049316972e-08
2115 1.88506279131806e-08
2116 1.91808057209419e-08
2117 1.86660070349287e-08
2118 1.94000344337086e-08
2119 1.94208977705168e-08
2120 1.92774887038394e-08
2121 1.88788457442246e-08
2122 1.92631538716986e-08
2123 1.84459654685859e-08
2124 1.89418886425707e-08
2125 1.90495376294342e-08
2126 1.87158021612843e-08
2127 1.91404307448295e-08
2128 1.8934649617941e-08
2129 1.90203854695814e-08
2130 1.86501743237866e-08
2131 1.89385727067115e-08
2132 1.89589469834495e-08
2133 1.93966507340271e-08
2134 1.89824154051832e-08
2135 1.87078741447078e-08
2136 1.87615089425608e-08
2137 1.88051857528543e-08
2138 1.8844835933729e-08
2139 1.9026375290665e-08
2140 1.89266908988822e-08
2141 1.89728001731837e-08
2142 1.8866981265131e-08
2143 1.9486140506661e-08
2144 1.89351491989703e-08
2145 1.90059419043348e-08
2146 1.86560559028159e-08
2147 1.89933845838874e-08
2148 1.84518796434441e-08
2149 1.88306046330633e-08
2150 1.89731278298633e-08
2151 1.87476524746155e-08
2152 1.91011083121029e-08
2153 1.94746069805962e-08
2154 1.81934744497569e-08
2155 1.87172694961213e-08
2156 1.94781728002358e-08
2157 1.84548235784288e-08
2158 1.88919975460389e-08
2159 1.89922177548374e-08
2160 1.85839602671045e-08
2161 1.89286042109604e-08
2162 1.85973281805751e-08
2163 1.931245323325e-08
2164 1.88719140332971e-08
2165 1.93951362408157e-08
2166 1.87522025904552e-08
2167 1.86698944324715e-08
2168 1.91348889043153e-08
2169 1.87677650145546e-08
2170 1.86063355855082e-08
2171 1.85260397825571e-08
2172 1.86327194567071e-08
2173 1.88676867052262e-08
2174 1.8841158090005e-08
2175 1.85665197741658e-08
2176 1.9162205440279e-08
2177 1.9190192388982e-08
2178 1.87985090882303e-08
2179 1.91552345329971e-08
2180 1.89351585739433e-08
2181 1.9033168813673e-08
2182 1.87506583086694e-08
2183 1.8840535848455e-08
2184 1.87860368426052e-08
2185 1.99765156233267e-08
2186 1.82167588836102e-08
2187 1.87664803147913e-08
2188 1.8781115902422e-08
2189 1.93676063444809e-08
2190 1.8470532733017e-08
2191 1.91366909621793e-08
2192 1.83820534923607e-08
2193 1.90772923145399e-08
2194 1.86463110390667e-08
2195 1.86635081315251e-08
2196 1.89194107945434e-08
2197 1.88636716700052e-08
2198 1.86439346893386e-08
2199 1.86744434176045e-08
2200 1.88409444573795e-08
2201 1.92180641265527e-08
2202 1.88660013397957e-08
2203 1.91005360905688e-08
2204 1.93355881311919e-08
2205 1.84239869719244e-08
2206 1.92580975900414e-08
2207 1.84103464130614e-08
2208 1.91880655870669e-08
2209 1.84776684238192e-08
2210 1.8513448318852e-08
2211 1.88318099417495e-08
2212 1.84894088281484e-08
2213 1.84629860615071e-08
2214 1.88097259914977e-08
2215 1.86876706118055e-08
2216 1.89756157566306e-08
2217 1.88409728167638e-08
2218 1.8754817169328e-08
2219 1.89808007645675e-08
2220 1.88624028399365e-08
2221 1.86197423644452e-08
2222 1.91252509827711e-08
2223 1.87835095737526e-08
2224 1.85246842604181e-08
2225 1.84526197127832e-08
2226 1.86915445358204e-08
2227 1.89574546526627e-08
2228 1.91640318370906e-08
2229 1.90858174442948e-08
2230 1.88197124259826e-08
2231 1.89731417229305e-08
2232 1.89255229283286e-08
2233 1.84794370135105e-08
2234 1.89344663566043e-08
2235 1.89235136590915e-08
2236 1.8873900303959e-08
2237 1.87261100583341e-08
2238 1.90076188605615e-08
2239 1.88009556511548e-08
2240 1.88823938808147e-08
2241 1.86711092492187e-08
2242 1.88546316745586e-08
2243 1.84609769574989e-08
2244 1.90896185384981e-08
2245 1.85433328911966e-08
2246 1.88048099805382e-08
2247 1.88290966061794e-08
2248 1.87878878483427e-08
2249 1.87354467092032e-08
2250 1.86453777260426e-08
2251 1.90038003312631e-08
2252 1.84784975016117e-08
2253 1.89303209123942e-08
2254 1.86376529710541e-08
2255 1.86902292695634e-08
2256 1.87183379694034e-08
2257 1.88069406354963e-08
2258 1.84055163503644e-08
2259 1.88231037150516e-08
2260 1.84994496162383e-08
2261 1.91086899020559e-08
2262 1.85082043600432e-08
2263 1.87154699068381e-08
2264 1.83621747905371e-08
2265 1.85996251562265e-08
2266 1.8696869255666e-08
2267 1.91117184068362e-08
2268 1.87646965525146e-08
2269 1.82044435691892e-08
2270 1.86948533236614e-08
2271 1.86350191471452e-08
2272 1.87174158608333e-08
2273 1.86362415660335e-08
2274 1.89371005670513e-08
2275 1.8451521329052e-08
2276 1.8872768433556e-08
2277 1.84142437862633e-08
2278 1.8433364075171e-08
2279 1.85116093451149e-08
2280 1.88186532612961e-08
2281 1.84134522705354e-08
2282 1.87993488809934e-08
2283 1.85485390186146e-08
2284 1.867150330645e-08
2285 1.87882179187027e-08
2286 1.83378308113991e-08
2287 1.88767017111485e-08
2288 1.8971289990205e-08
2289 1.82495902130292e-08
2290 1.88846983301821e-08
2291 1.85394189426841e-08
2292 1.86833538094039e-08
2293 1.87696317546365e-08
2294 1.89583264947613e-08
2295 1.8674239692193e-08
2296 1.88838107260114e-08
2297 1.8463712641803e-08
2298 1.87675392872427e-08
2299 1.90398472586922e-08
2300 1.83586050258727e-08
2301 1.82793572538198e-08
2302 1.88955687974446e-08
2303 1.86164771073943e-08
2304 1.90000423515857e-08
2305 1.8281505243356e-08
2306 1.83157359409936e-08
2307 1.86299246369703e-08
2308 1.88215388175761e-08
2309 1.87073663154846e-08
2310 1.82252477786693e-08
2311 1.83646082943661e-08
2312 1.891579173878e-08
2313 1.80588916852686e-08
2314 1.88098506395118e-08
2315 1.80622664817526e-08
2316 1.87007771592362e-08
2317 1.86409204110255e-08
2318 1.85775097105018e-08
2319 1.87717032972268e-08
2320 1.84322098160938e-08
2321 1.84324424497423e-08
2322 1.85756119353842e-08
2323 1.85320734128469e-08
2324 1.85287736019363e-08
2325 1.84113665688962e-08
2326 1.85906449744061e-08
2327 1.85682518910324e-08
2328 1.84876493526942e-08
2329 1.84658939495763e-08
2330 1.8880209176661e-08
2331 1.91473723460955e-08
2332 1.78005869957421e-08
2333 1.88266468980625e-08
2334 1.81810499484125e-08
2335 1.83849767315392e-08
2336 1.90198365382782e-08
2337 1.83126811818396e-08
2338 1.86059546848638e-08
2339 1.82000189213743e-08
2340 1.85939712758765e-08
2341 1.88828877526331e-08
2342 1.81244461752578e-08
2343 1.84367477127217e-08
2344 1.86770422745952e-08
2345 1.84968620084158e-08
2346 1.81817259641837e-08
2347 1.90352540400673e-08
2348 1.78997872496561e-08
2349 1.87784719094808e-08
2350 1.8168088350215e-08
2351 1.82445242491031e-08
2352 1.85188582624285e-08
2353 1.84773810161243e-08
2354 1.86132011536083e-08
2355 1.88097143117988e-08
2356 1.88256449847046e-08
2357 1.82021713416303e-08
2358 1.8531075719766e-08
2359 1.85395176737213e-08
2360 1.85645329865286e-08
2361 1.85361042806254e-08
2362 1.85147168489802e-08
2363 1.87940595206271e-08
2364 1.82955706000965e-08
2365 1.84460902627748e-08
2366 1.85464875806368e-08
2367 1.84096386627447e-08
2368 1.84893082878917e-08
2369 1.805750982109e-08
2370 1.84070460527708e-08
2371 1.85124442157597e-08
2372 1.84858494154255e-08
2373 1.83349707987418e-08
2374 1.83310311609591e-08
2375 1.84857455227255e-08
2376 1.83224677120231e-08
2377 1.87144829314756e-08
2378 1.83273826261099e-08
2379 1.83516016131896e-08
2380 1.84933911245855e-08
2381 1.83038465391416e-08
2382 1.81658218179059e-08
2383 1.83787802123803e-08
2384 1.87757409662453e-08
2385 1.8034291750138e-08
2386 1.89230163893384e-08
2387 1.79491980294572e-08
2388 1.83043707023545e-08
2389 1.81809479403022e-08
2390 1.80874024150873e-08
2391 1.8605386500864e-08
2392 1.86299367996162e-08
2393 1.83361189584896e-08
2394 1.84889412741945e-08
2395 1.85917251818574e-08
2396 1.84652592023632e-08
2397 1.79371279086377e-08
2398 1.85922448538017e-08
2399 1.83188446622601e-08
2400 1.80787221887302e-08
2401 1.85018903415685e-08
2402 1.84897644303628e-08
2403 1.79231706851357e-08
2404 1.80433468444652e-08
2405 1.81641867974813e-08
2406 1.86601875438036e-08
2407 1.81833110048241e-08
2408 1.86527672542403e-08
2409 1.78494407100244e-08
2410 1.83863381427313e-08
2411 1.8101994192804e-08
2412 1.82141930059332e-08
2413 1.85833196136853e-08
2414 1.83209025404196e-08
2415 1.84849057528191e-08
2416 1.79977031356249e-08
2417 1.81813961421806e-08
2418 1.85920001351259e-08
2419 1.8172378455486e-08
2420 1.87254752851973e-08
2421 1.89675778273918e-08
2422 1.78180553452589e-08
2423 1.78904109210259e-08
2424 1.80939618621284e-08
2425 1.82637533808344e-08
2426 1.80730670769486e-08
2427 1.83765488776938e-08
2428 1.82999361951763e-08
2429 1.81999520027926e-08
2430 1.82459327369433e-08
2431 1.83952562591433e-08
2432 1.82909563210959e-08
2433 1.80149152464865e-08
2434 1.82042157412243e-08
2435 1.82153137300023e-08
2436 1.8108438882905e-08
2437 1.81048570564168e-08
2438 1.85737927092394e-08
2439 1.8493692991034e-08
2440 1.81837525409095e-08
2441 1.81703725460614e-08
2442 1.84537128132267e-08
2443 1.86962576659427e-08
2444 1.79652529952862e-08
2445 1.82618886608726e-08
2446 1.84738199288242e-08
2447 1.77570923618336e-08
2448 1.84381255092514e-08
2449 1.8195315384481e-08
2450 1.83029720803862e-08
2451 1.78729349001594e-08
2452 1.81451560801738e-08
2453 1.81193309947147e-08
2454 1.81609580584829e-08
2455 1.82049760888287e-08
2456 1.86439783394776e-08
2457 1.78538091745728e-08
2458 1.8470805007359e-08
2459 1.84461179998097e-08
2460 1.79764295556994e-08
2461 1.85060645056523e-08
2462 1.81457316006106e-08
2463 1.81913988472338e-08
2464 1.8199104086819e-08
2465 1.85152291584945e-08
2466 1.82904474063106e-08
2467 1.81230272799232e-08
2468 1.83672623691455e-08
2469 1.80348147017784e-08
2470 1.81855643023865e-08
2471 1.80653294549193e-08
2472 1.79879741070421e-08
2473 1.82082947174089e-08
2474 1.79656412531981e-08
2475 1.77342744358822e-08
2476 1.84390154429337e-08
2477 1.77255485095118e-08
2478 1.843087814829e-08
2479 1.7753957169242e-08
2480 1.84184377618191e-08
2481 1.80678203193507e-08
2482 1.80926022815553e-08
2483 1.81639441897141e-08
2484 1.86637732457196e-08
2485 1.79457288595752e-08
2486 1.75549831731847e-08
2487 1.82371350133936e-08
2488 1.85040033482409e-08
2489 1.79328153105673e-08
2490 1.86816351491759e-08
2491 1.74118541605706e-08
2492 1.85012929618811e-08
2493 1.83819051767231e-08
2494 1.7907228498204e-08
2495 1.83495160615277e-08
2496 1.77208685739338e-08
2497 1.83414534004056e-08
2498 1.80550075945252e-08
2499 1.85899030032688e-08
2500 1.76593504317424e-08
2501 1.80057706605508e-08
2502 1.80031091353561e-08
2503 1.81896432326911e-08
2504 1.79173761430396e-08
2505 1.80967233054941e-08
2506 1.83959504850145e-08
2507 1.7853899353576e-08
2508 1.82318170663109e-08
2509 1.82988495223135e-08
2510 1.77597735263324e-08
2511 1.82046620340881e-08
2512 1.82645510502788e-08
2513 1.79930430081465e-08
2514 1.88530638475515e-08
2515 1.80126906976219e-08
2516 1.82546491857005e-08
2517 1.79335244368251e-08
2518 1.80800185771435e-08
2519 1.83567323967238e-08
2520 1.73608267472758e-08
2521 1.79932303889241e-08
2522 1.80309721648647e-08
2523 1.79210039933098e-08
2524 1.82228238145843e-08
2525 1.81892806413453e-08
2526 1.8010689271708e-08
2527 1.81015056955613e-08
2528 1.79057862377263e-08
2529 1.79953517742137e-08
2530 1.80299044336696e-08
2531 1.82235495015948e-08
2532 1.76188380384412e-08
2533 1.83193906151402e-08
2534 1.79713422366057e-08
2535 1.81252249866759e-08
2536 1.79864345209724e-08
2537 1.80522451183884e-08
2538 1.8287308905296e-08
2539 1.79130672378902e-08
2540 1.83049400532542e-08
2541 1.82039700665493e-08
2542 1.75057222765451e-08
2543 1.85395567816882e-08
2544 1.75502093625046e-08
2545 1.79760519839201e-08
2546 1.8155560255112e-08
2547 1.77821018365609e-08
2548 1.79223668665962e-08
2549 1.79615302385305e-08
2550 1.82054911726237e-08
2551 1.81527261515491e-08
2552 1.78814108402342e-08
2553 1.81042073053583e-08
2554 1.83824190135573e-08
2555 1.80073308537199e-08
2556 1.79987702424167e-08
2557 1.81087224015997e-08
2558 1.77320765627764e-08
2559 1.8041565116228e-08
2560 1.79411403138652e-08
2561 1.80532438228409e-08
2562 1.84937479042746e-08
2563 1.79098295100544e-08
2564 1.81655069755038e-08
2565 1.79038155536382e-08
2566 1.81531284703662e-08
2567 1.82253345195615e-08
2568 1.79353070843824e-08
2569 1.81433533847919e-08
2570 1.8595877310984e-08
2571 1.8032677771701e-08
2572 1.7544287985255e-08
2573 1.80593592206124e-08
2574 1.8453359101861e-08
2575 1.76520875201647e-08
2576 1.82058045337663e-08
2577 1.78347876560181e-08
2578 1.79973023719249e-08
2579 1.79948605793234e-08
2580 1.85391112149935e-08
2581 1.780326732298e-08
2582 1.76771999869929e-08
2583 1.87342383523026e-08
2584 1.76426993825085e-08
2585 1.79082615946874e-08
2586 1.76333902427855e-08
2587 1.79061855440921e-08
2588 1.80852416586524e-08
2589 1.82583104458567e-08
2590 1.83701764097655e-08
2591 1.8502152087152e-08
2592 1.78941998181248e-08
2593 1.78968492050846e-08
2594 1.79414889593388e-08
2595 1.80990026006911e-08
2596 1.80111202886524e-08
2597 1.79653477218322e-08
2598 1.81290567923542e-08
2599 1.82714605079237e-08
2600 1.8121889142908e-08
2601 1.8502070078974e-08
2602 1.85066838619685e-08
2603 1.80632762201427e-08
2604 1.76601826871459e-08
2605 1.78886832692682e-08
2606 1.8503146739679e-08
2607 1.82350015938415e-08
2608 1.75348737162517e-08
2609 1.75314359673329e-08
2610 1.79054467649153e-08
2611 1.77683245411731e-08
2612 1.80294826394589e-08
2613 1.82634933690584e-08
2614 1.77394466004399e-08
2615 1.81623034904155e-08
2616 1.7421069262169e-08
2617 1.76315070010979e-08
2618 1.76145746261791e-08
2619 1.75270826615614e-08
2620 1.7571628816937e-08
2621 1.74472224372113e-08
2622 1.7487744811473e-08
2623 1.7544088836724e-08
2624 1.8298683218787e-08
2625 1.72901694837635e-08
2626 1.77753481955256e-08
2627 1.77030948558532e-08
2628 1.80799011064436e-08
2629 1.80907226062788e-08
2630 1.78630179168693e-08
2631 1.72581397624161e-08
2632 1.75462147345429e-08
2633 1.79532144534145e-08
2634 1.75339805508767e-08
2635 1.7321240662449e-08
2636 1.79595335847715e-08
2637 1.74211955435988e-08
2638 1.73298889643569e-08
2639 1.76481593112676e-08
2640 1.70192630555288e-08
2641 1.80449486058554e-08
2642 1.7001722117771e-08
2643 1.77021051084303e-08
2644 1.81319330019303e-08
2645 1.76943418527531e-08
2646 1.70977209288103e-08
2647 1.77366639395049e-08
2648 1.73386854826207e-08
2649 1.78039611178904e-08
2650 1.7445135048838e-08
2651 1.78937758224529e-08
2652 1.76859027832266e-08
2653 1.71654002135424e-08
2654 1.78480262235936e-08
2655 1.7726499200621e-08
2656 1.76262604440203e-08
2657 1.77601285099027e-08
2658 1.73207618620053e-08
2659 1.81325795316767e-08
2660 1.74515748800341e-08
2661 1.81741832820936e-08
2662 1.76038093032682e-08
2663 1.75640456515358e-08
2664 1.78130613368577e-08
2665 1.72762133091586e-08
2666 1.83733822635979e-08
2667 1.76087157399418e-08
2668 1.80565517586273e-08
2669 1.7283346097019e-08
2670 1.69892157665663e-08
2671 1.77706405546374e-08
2672 1.7543151379279e-08
2673 1.77963941892789e-08
2674 1.73669141926569e-08
2675 1.75378800761627e-08
2676 1.77073467600986e-08
2677 1.7626857171521e-08
2678 1.83307702401325e-08
2679 1.74446278089213e-08
2680 1.7595940412285e-08
2681 1.78536084550485e-08
2682 1.74453141062736e-08
2683 1.76580996990028e-08
2684 1.82555321223965e-08
2685 1.68856148952679e-08
2686 1.81726639091884e-08
2687 1.77319030238432e-08
2688 1.73891948525295e-08
2689 1.78651649421213e-08
2690 1.82525879597328e-08
2691 1.75409348489031e-08
2692 1.75937458076758e-08
2693 1.79130691726204e-08
2694 1.78156825340892e-08
2695 1.82553722533618e-08
2696 1.71482165550463e-08
2697 1.75733780160559e-08
2698 1.77747539010231e-08
2699 1.80828063665317e-08
2700 1.78902176035944e-08
2701 1.77510316959895e-08
2702 1.78667177537783e-08
2703 1.72748123619948e-08
2704 1.79254951010732e-08
2705 1.75880905495529e-08
2706 1.79450806484238e-08
2707 1.76595312210148e-08
2708 1.75385167963948e-08
2709 1.82677457966351e-08
2710 1.77574974483674e-08
2711 1.74078098278685e-08
2712 1.74886722309475e-08
2713 1.75569275166765e-08
2714 1.78957170880734e-08
2715 1.80754818299156e-08
2716 1.72799174945337e-08
2717 1.78677984400966e-08
2718 1.70602074000825e-08
2719 1.7738307065418e-08
2720 1.76388848486808e-08
2721 1.77118576683566e-08
2722 1.73019468851787e-08
2723 1.71267668492847e-08
2724 1.76784750804126e-08
2725 1.75819516393116e-08
2726 1.73195764519851e-08
2727 1.77916340347029e-08
2728 1.6943710380829e-08
2729 1.83503947742136e-08
2730 1.79254375298477e-08
2731 1.70072251661491e-08
2732 1.84991147602898e-08
2733 1.73942128913751e-08
2734 1.78796773747936e-08
2735 1.74043984366434e-08
2736 1.74652886349991e-08
2737 1.75969894501693e-08
2738 1.74259987274494e-08
2739 1.78120251488612e-08
2740 1.74065996417228e-08
2741 1.80081224271655e-08
2742 1.74783688234037e-08
2743 1.7633275398915e-08
2744 1.76030150772499e-08
2745 1.85733848707403e-08
2746 1.72592050374787e-08
2747 1.75044228995502e-08
2748 1.75319506270366e-08
2749 1.7574851493618e-08
2750 1.81906679924404e-08
2751 1.73313689353638e-08
2752 1.71538874439159e-08
2753 1.76959462408754e-08
2754 1.81680005049423e-08
2755 1.7182055987025e-08
2756 1.78442334765605e-08
2757 1.75668287154818e-08
2758 1.70162075188718e-08
2759 1.78008715004951e-08
2760 1.77410289670943e-08
2761 1.72489068950288e-08
2762 1.88479845791012e-08
2763 1.69621130245157e-08
2764 1.81360672499298e-08
2765 1.70900312717603e-08
2766 1.74982045831051e-08
2767 1.82576762790543e-08
2768 1.79297468368145e-08
2769 1.76249912627047e-08
2770 1.75628970599528e-08
2771 1.71880341199859e-08
2772 1.7877661438348e-08
2773 1.79957950680043e-08
2774 1.71127974622209e-08
2775 1.77867793608177e-08
2776 1.75026379581889e-08
2777 1.85738010108905e-08
2778 1.76131180775196e-08
2779 1.77844408377203e-08
2780 1.75703174689623e-08
2781 1.70834054502023e-08
2782 1.76772246167911e-08
2783 1.74578280286719e-08
2784 1.73819348855131e-08
2785 1.78742647709285e-08
2786 1.79827726271842e-08
2787 1.69421898796596e-08
2788 1.72661168417187e-08
2789 1.79770273491253e-08
2790 1.75468931063183e-08
2791 1.73116308183341e-08
2792 1.76902112675181e-08
2793 1.74892734112059e-08
2794 1.69508230828119e-08
2795 1.74087030125614e-08
2796 1.82875315507536e-08
2797 1.68136647992756e-08
2798 1.77700656909252e-08
2799 1.70494900106621e-08
2800 1.73987151786681e-08
2801 1.74188731936198e-08
2802 1.74238463006293e-08
2803 1.79344236060641e-08
2804 1.76661582957371e-08
2805 1.78741515320935e-08
2806 1.7748211971591e-08
2807 1.78106411517498e-08
2808 1.75495193004693e-08
2809 1.74048379745079e-08
2810 1.70185226889274e-08
2811 1.81065949538262e-08
2812 1.70022029910566e-08
2813 1.79828073357258e-08
2814 1.78514023366771e-08
2815 1.78171091870794e-08
2816 1.71788073584089e-08
2817 1.77527756213824e-08
2818 1.73206269106774e-08
2819 1.73200164000631e-08
2820 1.68792305385301e-08
2821 1.72066834123186e-08
2822 1.81972243611106e-08
2823 1.68887167143195e-08
2824 1.70201483152554e-08
2825 1.69723249736076e-08
2826 1.80566745364708e-08
2827 1.74417504653634e-08
2828 1.75182805003671e-08
2829 1.75624046794831e-08
2830 1.70946764127544e-08
2831 1.73913807480774e-08
2832 1.72860259450291e-08
2833 1.74636054014188e-08
2834 1.78319704494945e-08
2835 1.72864091291525e-08
2836 1.77813133702687e-08
2837 1.80364034785274e-08
2838 1.69746332140119e-08
2839 1.74514368790346e-08
2840 1.84902601503278e-08
2841 1.73402689082891e-08
2842 1.76299927438228e-08
2843 1.71424223734951e-08
2844 1.76060923300636e-08
2845 1.70310214557268e-08
2846 1.74076377338628e-08
2847 1.72614318576791e-08
2848 1.76912550107211e-08
2849 1.80264459947177e-08
2850 1.75121466055583e-08
2851 1.7120515309943e-08
2852 1.80220255736885e-08
2853 1.75267024066783e-08
2854 1.75683000564197e-08
2855 1.76796225396791e-08
2856 1.71640844167653e-08
2857 1.8042237106719e-08
2858 1.68327717843808e-08
2859 1.72909346559391e-08
2860 1.75612102700229e-08
2861 1.72460440610589e-08
2862 1.81338980801971e-08
2863 1.74350964583791e-08
2864 1.75720857400197e-08
2865 1.76954502782434e-08
2866 1.76302250410598e-08
2867 1.73643283797598e-08
2868 1.72522101413441e-08
2869 1.7197380388384e-08
2870 1.73892450415214e-08
2871 1.72429074034586e-08
2872 1.66831010283697e-08
2873 1.76923419466235e-08
2874 1.73317149202712e-08
2875 1.75170678430869e-08
2876 1.76581387159314e-08
2877 1.72849197869751e-08
2878 1.75767375606428e-08
2879 1.73502408457038e-08
2880 1.75844117688284e-08
2881 1.72650747281289e-08
2882 1.74294268007191e-08
2883 1.73586332611886e-08
2884 1.88088484587012e-08
2885 1.71694101148978e-08
2886 1.74571839010318e-08
2887 1.8247148764583e-08
2888 1.70227053390981e-08
2889 1.73459581974955e-08
2890 1.7326880739521e-08
2891 1.70170496142374e-08
2892 1.75110134629786e-08
2893 1.75064266735025e-08
2894 1.71719219905914e-08
2895 1.76470124208516e-08
2896 1.68427554235462e-08
2897 1.72841031723936e-08
2898 1.72295917066634e-08
2899 1.78316656442579e-08
2900 1.8078290858814e-08
2901 1.75597646997483e-08
2902 1.74252549237919e-08
2903 1.74723013179334e-08
2904 1.74355378293189e-08
2905 1.74972458922562e-08
2906 1.71242879109246e-08
2907 1.73145497034954e-08
2908 1.77377993116068e-08
2909 1.77110222798482e-08
2910 1.70750397155972e-08
2911 1.77110536678515e-08
2912 1.76321152523251e-08
2913 1.75360195590335e-08
2914 1.73072085248305e-08
2915 1.71675461321519e-08
2916 1.76491897243836e-08
2917 1.78067591032061e-08
2918 1.68675593540946e-08
2919 1.7672517796774e-08
2920 1.68110008455791e-08
2921 1.7693910956823e-08
2922 1.76668818714643e-08
2923 1.75723047289023e-08
2924 1.72774131592945e-08
2925 1.77348586715143e-08
2926 1.74488701944475e-08
2927 1.79512746497634e-08
2928 1.72633919656995e-08
2929 1.66424239609192e-08
2930 1.72997699898481e-08
2931 1.76856821039739e-08
2932 1.72353171292994e-08
2933 1.73133156864125e-08
2934 1.75623923301255e-08
2935 1.78612251637289e-08
2936 1.71656462125969e-08
2937 1.71293979873421e-08
2938 1.74166518904922e-08
2939 1.78897970618541e-08
2940 1.7198295387294e-08
2941 1.76762172990619e-08
2942 1.68838435579133e-08
2943 1.78014764951884e-08
2944 1.71649174432742e-08
2945 1.74359869680174e-08
2946 1.71551547931414e-08
2947 1.8021682949565e-08
2948 1.73569395766626e-08
2949 1.72896989099058e-08
2950 1.74415548490903e-08
2951 1.73626884282241e-08
2952 1.75305115248103e-08
2953 1.77696476869327e-08
2954 1.72177194392442e-08
2955 1.71220675645734e-08
2956 1.73829408771364e-08
2957 1.7036334790943e-08
2958 1.74463993789786e-08
2959 1.76279860849005e-08
2960 1.75300877495732e-08
2961 1.69349214423076e-08
2962 1.76435647192552e-08
2963 1.73313164234523e-08
2964 1.7468383937097e-08
2965 1.76264024729367e-08
2966 1.63362870424788e-08
2967 1.75837932596956e-08
2968 1.71336047333215e-08
2969 1.72079998851105e-08
2970 1.73819313992463e-08
2971 1.82894757037866e-08
2972 1.64146530280129e-08
2973 1.78366049344569e-08
2974 1.69133282567624e-08
2975 1.72071186320955e-08
2976 1.71264725290776e-08
2977 1.75893415469142e-08
2978 1.75565540562472e-08
2979 1.72476714462577e-08
2980 1.73829306384654e-08
2981 1.692582528251e-08
2982 1.77570219823397e-08
2983 1.72757253031897e-08
2984 1.70912965822834e-08
2985 1.77215275107001e-08
2986 1.70779845988234e-08
2987 1.73738766298392e-08
2988 1.75523033668634e-08
2989 1.68365966669581e-08
2990 1.78440330720619e-08
2991 1.72633376710196e-08
2992 1.76163595028839e-08
2993 1.74245230902259e-08
2994 1.73711779988261e-08
2995 1.74227714412889e-08
2996 1.71070381568827e-08
2997 1.74341203219297e-08
2998 1.74507785898614e-08
2999 1.69353017674956e-08
3000 1.76881399246431e-08
3001 1.75408895083939e-08
3002 1.72358825953067e-08
3003 1.7446000780047e-08
3004 1.74180877960672e-08
3005 1.7130566077328e-08
3006 1.73579701812365e-08
3007 1.71613898887757e-08
3008 1.72891647341744e-08
3009 1.74069260061027e-08
3010 1.75545192024895e-08
3011 1.70643838853235e-08
3012 1.69598656356296e-08
3013 1.79397050977248e-08
3014 1.69955379026032e-08
3015 1.76492504380954e-08
3016 1.72485054593385e-08
3017 1.71924549899616e-08
3018 1.75155651813419e-08
3019 1.68872154960309e-08
3020 1.83413643954078e-08
3021 1.6848214081655e-08
3022 1.73482868478236e-08
3023 1.71181117008812e-08
3024 1.67641767374327e-08
3025 1.79803271636025e-08
3026 1.70898248610107e-08
3027 1.68449662644454e-08
3028 1.75260863759796e-08
3029 1.74406767998736e-08
3030 1.69075041153155e-08
3031 1.77647682821913e-08
3032 1.69406916380177e-08
3033 1.76708891545252e-08
3034 1.702908247872e-08
3035 1.70578495092277e-08
3036 1.74481795269243e-08
3037 1.6818263259083e-08
3038 1.77820922736827e-08
3039 1.71686571339147e-08
3040 1.73096723133925e-08
3041 1.72525775102295e-08
3042 1.73569456228262e-08
3043 1.71468958436172e-08
3044 1.72060723184486e-08
3045 1.69782648714034e-08
3046 1.7332094090966e-08
3047 1.68981816333624e-08
3048 1.68898142577367e-08
3049 1.77948707196784e-08
3050 1.6998012881797e-08
3051 1.73414909998559e-08
3052 1.74988208946347e-08
3053 1.73558484525549e-08
3054 1.74970622302262e-08
3055 1.71464272640531e-08
3056 1.69703173748814e-08
3057 1.76155165461656e-08
3058 1.76236037513489e-08
3059 1.67192638662328e-08
3060 1.73614646765963e-08
3061 1.71855119526221e-08
3062 1.75750587184664e-08
3063 1.62554558515438e-08
3064 1.7466872969496e-08
3065 1.7021220973662e-08
3066 1.7452162997561e-08
3067 1.67115565525844e-08
3068 1.69664300722072e-08
3069 1.71132351843295e-08
3070 1.72477212207478e-08
3071 1.70472051830017e-08
3072 1.7350787160586e-08
3073 1.73056923796766e-08
3074 1.71016093132581e-08
3075 1.73506171980281e-08
3076 1.69861044624353e-08
3077 1.65645471873754e-08
3078 1.67206617100457e-08
3079 1.698353674498e-08
3080 1.71611019616075e-08
3081 1.69625239357929e-08
3082 1.79351404319683e-08
3083 1.68692719456909e-08
3084 1.67284652541067e-08
3085 1.76299857207185e-08
3086 1.65874575368818e-08
3087 1.72895764729009e-08
3088 1.69334282385802e-08
3089 1.71403912218038e-08
3090 1.6961843736174e-08
3091 1.72812892570151e-08
3092 1.71144413307345e-08
3093 1.73071342682851e-08
3094 1.71883793436689e-08
3095 1.70543630167441e-08
3096 1.74384635849845e-08
3097 1.70790431356094e-08
3098 1.72019543654145e-08
3099 1.66890201288017e-08
3100 1.7347253174177e-08
3101 1.69392402020463e-08
3102 1.73757771534944e-08
3103 1.68475484990105e-08
3104 1.72199666357287e-08
3105 1.68812364003124e-08
3106 1.70076576572475e-08
3107 1.73434459898314e-08
3108 1.70411399580961e-08
3109 1.64919996148982e-08
3110 1.65869777407346e-08
3111 1.68260997642777e-08
3112 1.74552141569417e-08
3113 1.66277112580993e-08
3114 1.71371955313693e-08
3115 1.72475661929239e-08
3116 1.7203893388662e-08
3117 1.63856477305852e-08
3118 1.7473041525079e-08
3119 1.70749857815455e-08
3120 1.72588191667822e-08
3121 1.68551230843306e-08
3122 1.71787239308385e-08
3123 1.69344267411525e-08
3124 1.71247384665174e-08
3125 1.70446557136439e-08
3126 1.69783580422078e-08
3127 1.73467273939243e-08
3128 1.72139424177709e-08
3129 1.72508084094469e-08
3130 1.68579139676328e-08
3131 1.71032957340111e-08
3132 1.68089006272465e-08
3133 1.72853744120349e-08
3134 1.70058514655791e-08
3135 1.69059704347718e-08
3136 1.6776794044604e-08
3137 1.76001427269201e-08
3138 1.66162508226375e-08
3139 1.74922622629969e-08
3140 1.65660030811005e-08
3141 1.72940378133091e-08
3142 1.71893799870981e-08
3143 1.71126603371874e-08
3144 1.69189897815947e-08
3145 1.75592332118113e-08
3146 1.67235881625427e-08
3147 1.71318199068127e-08
3148 1.75720068066598e-08
3149 1.68161686399104e-08
3150 1.66127179638958e-08
3151 1.70386756684071e-08
3152 1.73096624634106e-08
3153 1.69180662191182e-08
3154 1.73892636421702e-08
3155 1.70950699142358e-08
3156 1.72936018596748e-08
3157 1.7030916868721e-08
3158 1.64253889088917e-08
3159 1.75206027990815e-08
3160 1.69389646072216e-08
3161 1.70844152852512e-08
3162 1.6862099674525e-08
3163 1.72616805933223e-08
3164 1.69168332434522e-08
3165 1.70508649658396e-08
3166 1.70438997671996e-08
3167 1.70939541919712e-08
3168 1.67412752504481e-08
3169 1.65895872106248e-08
3170 1.69374288652158e-08
3171 1.70669967231862e-08
3172 1.71983354299021e-08
3173 1.70305601049914e-08
3174 1.68618652138519e-08
3175 1.71454736415944e-08
3176 1.65347370799651e-08
3177 1.69602399927582e-08
3178 1.73337240339244e-08
3179 1.68211847260125e-08
3180 1.69865184295737e-08
3181 1.68120008425809e-08
3182 1.68553211041533e-08
3183 1.73493417407744e-08
3184 1.68626363828139e-08
3185 1.68981187768646e-08
3186 1.76034527854807e-08
3187 1.7009494725051e-08
3188 1.66499933274833e-08
3189 1.71513538632373e-08
3190 1.6800622579527e-08
3191 1.70990427551532e-08
3192 1.67688039770242e-08
3193 1.69954734827726e-08
3194 1.71861825413733e-08
3195 1.71146877042394e-08
3196 1.69279523788424e-08
3197 1.72522510026585e-08
3198 1.68881408072197e-08
3199 1.69985231983499e-08
3200 1.7140858868947e-08
3201 1.6975252557283e-08
3202 1.70798980881071e-08
3203 1.67125704589877e-08
3204 1.69105550333337e-08
3205 1.73598712442657e-08
3206 1.70998964925673e-08
3207 1.69619916322583e-08
3208 1.69194657491101e-08
3209 1.70687066467223e-08
3210 1.663289297274e-08
3211 1.70461653843657e-08
3212 1.71185181093159e-08
3213 1.67800675152513e-08
3214 1.69297118274847e-08
3215 1.71001543443905e-08
3216 1.67659332892256e-08
3217 1.67616717244579e-08
3218 1.71315559620022e-08
3219 1.70778470599264e-08
3220 1.68284616158532e-08
3221 1.66895847584225e-08
3222 1.72366030833648e-08
3223 1.69463905862199e-08
3224 1.70492674697598e-08
3225 1.68321626310142e-08
3226 1.68673372685502e-08
3227 1.73283311789008e-08
3228 1.69076044976846e-08
3229 1.69292439030078e-08
3230 1.69759412234627e-08
3231 1.6696047973469e-08
3232 1.63052425772858e-08
3233 1.65239550092289e-08
3234 1.61699318325242e-08
3235 1.6150756355221e-08
3236 1.61271058137236e-08
3237 1.61692276971215e-08
3238 1.6453896316132e-08
3239 1.57205479523193e-08
3240 1.65361257350727e-08
3241 1.59485450975938e-08
3242 1.63634535925161e-08
3243 1.62853350867964e-08
3244 1.6183235171302e-08
3245 1.63923499851282e-08
3246 1.63110983160547e-08
3247 1.61678463975745e-08
3248 1.61347804627371e-08
3249 1.60376294212894e-08
3250 1.64135059596837e-08
3251 1.65515286238982e-08
3252 1.58450752927863e-08
3253 1.63557898444078e-08
3254 1.60015761047905e-08
3255 1.62852296650973e-08
3256 1.61135573864268e-08
3257 1.609490483237e-08
3258 1.60806482261966e-08
3259 1.62318977993264e-08
3260 1.62848059363507e-08
3261 1.61169927259119e-08
3262 1.62593298072278e-08
3263 1.61729600826332e-08
3264 1.6218396220069e-08
3265 1.65528208512789e-08
3266 1.60001118283126e-08
3267 1.63526867895947e-08
3268 1.60894103103937e-08
3269 1.58299713266252e-08
3270 1.60540073584903e-08
3271 1.59012109085949e-08
3272 1.60891499306343e-08
3273 1.6245663563369e-08
3274 1.57465973087134e-08
3275 1.60946620298974e-08
3276 1.61609926357942e-08
3277 1.61710613773153e-08
3278 1.61307090234819e-08
3279 1.65811054435167e-08
3280 1.58562313124222e-08
3281 1.56134626591864e-08
3282 1.59949973467244e-08
3283 1.5879592642154e-08
3284 1.59217004261636e-08
3285 1.57787724183872e-08
3286 1.56252879526531e-08
3287 1.56357457153089e-08
3288 1.59749183987756e-08
3289 1.59166860196902e-08
3290 1.5725198151656e-08
3291 1.58155637907664e-08
3292 1.59955289112113e-08
3293 1.60372468054615e-08
3294 1.5662689475171e-08
3295 1.58748717981327e-08
3296 1.56369688971703e-08
3297 1.60609434237263e-08
3298 1.56579708239096e-08
3299 1.57366124950431e-08
3300 1.57869214591488e-08
3301 1.58336851992236e-08
3302 1.58864821099558e-08
3303 1.58297179031763e-08
3304 1.548633775178e-08
3305 1.57548612424174e-08
3306 1.54488333540675e-08
3307 1.59456507542732e-08
3308 1.54528883238303e-08
3309 1.55389631978642e-08
3310 1.56903749321813e-08
3311 1.5568015597045e-08
3312 1.55838170375067e-08
3313 1.54982457445663e-08
3314 1.58972326266965e-08
3315 1.57639662922149e-08
3316 1.57956894590028e-08
3317 1.57492110785806e-08
3318 1.58207470305483e-08
3319 1.55358933881383e-08
3320 1.55551600461989e-08
3321 1.554735044762e-08
3322 1.5487197952524e-08
3323 1.5448736419954e-08
3324 1.57250223436212e-08
3325 1.54123419389418e-08
3326 1.56155947207137e-08
3327 1.57702471003662e-08
3328 1.54487525276248e-08
3329 1.55073858874449e-08
3330 1.55873667890827e-08
3331 1.549921796698e-08
3332 1.53475410019643e-08
3333 1.56339746445744e-08
3334 1.5589732714455e-08
3335 1.52065720400302e-08
3336 1.57024130988392e-08
3337 1.5235359922483e-08
3338 1.56309198241777e-08
3339 1.5417198378137e-08
3340 1.56316523140521e-08
3341 1.54505584765152e-08
3342 1.53439951719736e-08
3343 1.54624640708034e-08
3344 1.53959782918933e-08
3345 1.5540918208129e-08
3346 1.54038311339366e-08
3347 1.53394493220271e-08
3348 1.53397224234575e-08
3349 1.53710763581838e-08
3350 1.52738594796087e-08
3351 1.52530891275871e-08
3352 1.54607730147305e-08
3353 1.53124848964736e-08
3354 1.533519571012e-08
3355 1.55962195326487e-08
3356 1.5176515433174e-08
3357 1.52055061909961e-08
3358 1.52648840018588e-08
3359 1.53022527370483e-08
3360 1.52469552630874e-08
3361 1.53174499252784e-08
3362 1.51545241111051e-08
3363 1.53654357502864e-08
3364 1.525627268259e-08
3365 1.54331268937824e-08
3366 1.51467240364911e-08
3367 1.54847047667828e-08
3368 1.53891206339796e-08
3369 1.52244474709018e-08
3370 1.50576007896253e-08
3371 1.55255508603458e-08
3372 1.5091656682148e-08
3373 1.52603707955501e-08
3374 1.51749828587178e-08
3375 1.52617133181432e-08
3376 1.53564045378141e-08
3377 1.52675841720851e-08
3378 1.53137410302817e-08
3379 1.52912185499843e-08
3380 1.57003538548417e-08
3381 1.52832466147612e-08
3382 1.5385910108523e-08
3383 1.51408132695596e-08
3384 1.52792678086711e-08
3385 1.51279502480783e-08
3386 1.53131045225741e-08
3387 1.53364197293671e-08
3388 1.53063665843944e-08
3389 1.52103265226855e-08
3390 1.5058139017976e-08
3391 1.53851837297603e-08
3392 1.52356609706139e-08
3393 1.53814326581958e-08
3394 1.55190682192552e-08
3395 1.53397466346183e-08
3396 1.52826754933888e-08
3397 1.51982497930414e-08
3398 1.50243436509434e-08
3399 1.54251687550233e-08
3400 1.52348696527971e-08
3401 1.51414837865904e-08
3402 1.51392987569332e-08
3403 1.5118808065262e-08
3404 1.51144421081384e-08
3405 1.52033400270257e-08
3406 1.51333912397633e-08
3407 1.53694754620737e-08
3408 1.50118790189502e-08
3409 1.51310488756273e-08
3410 1.56581386143162e-08
3411 1.50375043945217e-08
3412 1.54792157275319e-08
3413 1.52635498104348e-08
3414 1.5363915051414e-08
3415 1.49531926712798e-08
3416 1.52198624929767e-08
3417 1.53135666843829e-08
3418 1.53363221823133e-08
3419 1.50735388359802e-08
3420 1.52189456523732e-08
3421 1.51692707764106e-08
3422 1.52629982928254e-08
3423 1.51531287800599e-08
3424 1.53722529702971e-08
3425 1.50927925079147e-08
3426 1.55908936626814e-08
3427 1.53745417305895e-08
3428 1.51808433943024e-08
3429 1.5296708774315e-08
3430 1.52334947260413e-08
3431 1.54607584448851e-08
3432 1.51918539180729e-08
3433 1.54471059235495e-08
3434 1.5201022124739e-08
3435 1.54630976959835e-08
3436 1.5320592850282e-08
3437 1.49023913421087e-08
3438 1.52488178947752e-08
3439 1.51150519195065e-08
3440 1.48861166511804e-08
3441 1.52625365436587e-08
3442 1.53698144968462e-08
3443 1.5106636940071e-08
3444 1.53082562741047e-08
3445 1.54741345469167e-08
3446 1.50425963795775e-08
3447 1.52788616806232e-08
3448 1.48704830029434e-08
3449 1.54528714307323e-08
3450 1.49195880630171e-08
3451 1.51191192189759e-08
3452 1.50830473743258e-08
3453 1.50295361734942e-08
3454 1.51632075039432e-08
3455 1.54479282157005e-08
3456 1.51431651596867e-08
3457 1.4992060637653e-08
3458 1.51005814871263e-08
3459 1.50357673699419e-08
3460 1.54727647006525e-08
3461 1.51625457230908e-08
3462 1.55197222181935e-08
3463 1.48238129593081e-08
3464 1.54779854715514e-08
3465 1.50964481749061e-08
3466 1.5112473603085e-08
3467 1.52820346979721e-08
3468 1.48031708292196e-08
3469 1.52929529094459e-08
3470 1.55399282142921e-08
3471 1.49801515050285e-08
3472 1.53496285662524e-08
3473 1.48659243161264e-08
3474 1.55520807283149e-08
3475 1.48897732944464e-08
3476 1.50858313096025e-08
3477 1.50138203972028e-08
3478 1.49481702262644e-08
3479 1.53230137697746e-08
3480 1.51088316911185e-08
3481 1.50406556544136e-08
3482 1.53760364310918e-08
3483 1.50781750609674e-08
3484 1.49278905364758e-08
3485 1.52507597493134e-08
3486 1.45974550564698e-08
3487 1.52716788635598e-08
3488 1.52217520879294e-08
3489 1.52925910054391e-08
3490 1.54452650759918e-08
3491 1.49726264955241e-08
3492 1.53828600243378e-08
3493 1.50841070294394e-08
3494 1.45570382757976e-08
3495 1.5710424249693e-08
3496 1.50282378923727e-08
3497 1.51162831290297e-08
3498 1.47919288570608e-08
3499 1.51816110175174e-08
3500 1.53150282781178e-08
3501 1.49264927285508e-08
3502 1.50732828776423e-08
3503 1.47261898951756e-08
3504 1.48964056052026e-08
3505 1.47555164405899e-08
3506 1.50463085105412e-08
3507 1.51354006266424e-08
3508 1.46191506436022e-08
3509 1.47954040174147e-08
3510 1.4958268564097e-08
3511 1.51974524097154e-08
3512 1.50884598885126e-08
3513 1.49400392951693e-08
3514 1.49837250728868e-08
3515 1.50394690643652e-08
3516 1.51635199975852e-08
3517 1.48976094822006e-08
3518 1.49872659929995e-08
3519 1.50616157513217e-08
3520 1.50601681622453e-08
3521 1.5010981670921e-08
3522 1.47845248507361e-08
3523 1.50103344248587e-08
3524 1.48768351792838e-08
3525 1.4964012409413e-08
3526 1.48639092679426e-08
3527 1.48704274303968e-08
3528 1.47614155092124e-08
3529 1.4946863217763e-08
3530 1.47117458513535e-08
3531 1.49484921328491e-08
3532 1.49623545402933e-08
3533 1.49545730627754e-08
3534 1.49738662781895e-08
3535 1.46079792152365e-08
3536 1.48009503140867e-08
3537 1.477040052561e-08
3538 1.46506482177899e-08
3539 1.49754769680488e-08
3540 1.48172421333692e-08
3541 1.48955539683027e-08
3542 1.48570588624497e-08
3543 1.49729453409464e-08
3544 1.49269248973793e-08
3545 1.50506500677383e-08
3546 1.46921051189519e-08
3547 1.44959896654639e-08
3548 1.47416267485823e-08
3549 1.47324143741634e-08
3550 1.47299341305973e-08
3551 1.46201809134716e-08
3552 1.4612243645129e-08
3553 1.49102040095384e-08
3554 1.46135792880309e-08
3555 1.50104285597252e-08
3556 1.46272671494307e-08
3557 1.46743887669831e-08
3558 1.51730171712161e-08
3559 1.4756597349258e-08
3560 1.48353375770405e-08
3561 1.49605394190178e-08
3562 1.48362646417155e-08
3563 1.46826577246684e-08
3564 1.48854598463477e-08
3565 1.47192087028014e-08
3566 1.46234137141099e-08
3567 1.48952992338747e-08
3568 1.45555060359348e-08
3569 1.4853889529487e-08
3570 1.49623873900995e-08
3571 1.46050429450373e-08
3572 1.4554563505409e-08
3573 1.49375645304151e-08
3574 1.46532178689485e-08
3575 1.48575217314151e-08
3576 1.50192563414808e-08
3577 1.47945903754554e-08
3578 1.47900187416616e-08
3579 1.49478607801429e-08
3580 1.48919069522246e-08
3581 1.47386936822369e-08
3582 1.45675488821134e-08
3583 1.47671037468294e-08
3584 1.52246800281808e-08
3585 1.43024299092132e-08
3586 1.44691389762919e-08
3587 1.47460668392385e-08
3588 1.45062585947819e-08
3589 1.49664339211264e-08
3590 1.44769652395405e-08
3591 1.47168736431835e-08
3592 1.49307112687069e-08
3593 1.47229923529169e-08
3594 1.47937215367044e-08
3595 1.46285490520148e-08
3596 1.47996766366176e-08
3597 1.42786178404242e-08
3598 1.49262829684804e-08
3599 1.46781377499405e-08
3600 1.50196310813866e-08
3601 1.4587656601478e-08
3602 1.44273125619876e-08
3603 1.49391214130656e-08
3604 1.4792274884573e-08
3605 1.45608770429351e-08
3606 1.47519587702372e-08
3607 1.44652868809547e-08
3608 1.48435990574292e-08
3609 1.46159056519102e-08
3610 1.47458588326543e-08
3611 1.46851567661976e-08
3612 1.46107060367273e-08
3613 1.4574671140577e-08
3614 1.51989012607823e-08
3615 1.42846110657568e-08
3616 1.4667582134309e-08
3617 1.46682287053002e-08
3618 1.52803398284218e-08
3619 1.4325902539053e-08
3620 1.44829597944762e-08
3621 1.45068663833792e-08
3622 1.46931909230652e-08
3623 1.47194856168487e-08
3624 1.4640262102808e-08
3625 1.45524779558981e-08
3626 1.46118110666282e-08
3627 1.47300318055904e-08
3628 1.45524317655132e-08
3629 1.46154536058984e-08
3630 1.44829600747243e-08
3631 1.45289289011225e-08
3632 1.42307299458022e-08
3633 1.47193909595666e-08
3634 1.44071836151172e-08
3635 1.46729620465191e-08
3636 1.46983944941115e-08
3637 1.483032133201e-08
3638 1.42981624761673e-08
3639 1.44836225408063e-08
3640 1.49675825048268e-08
3641 1.42723818583879e-08
3642 1.47205077832141e-08
3643 1.5096709643031e-08
3644 1.4100294689956e-08
3645 1.42847753243081e-08
3646 1.48337213087824e-08
3647 1.454802660035e-08
3648 1.42891945188184e-08
3649 1.45912893146916e-08
3650 1.46600759685922e-08
3651 1.47075119934642e-08
3652 1.47633636610744e-08
3653 1.45078670389653e-08
3654 1.44223176147262e-08
3655 1.42213353136467e-08
3656 1.43145201887129e-08
3657 1.45930318379284e-08
3658 1.45661939914277e-08
3659 1.41399774732098e-08
3660 1.4686952368087e-08
3661 1.4523387062676e-08
3662 1.4141562022868e-08
3663 1.43055856715313e-08
3664 1.46274492276166e-08
3665 1.46430657224916e-08
3666 1.43076770343642e-08
3667 1.41121750666107e-08
3668 1.42979776678598e-08
3669 1.45776258992369e-08
3670 1.44169786123888e-08
3671 1.44567410665097e-08
3672 1.44567426203779e-08
3673 1.44161172754842e-08
3674 1.46437586817649e-08
3675 1.45505453121864e-08
3676 1.43268862925761e-08
3677 1.43997318597655e-08
3678 1.44681077893227e-08
3679 1.45807958378957e-08
3680 1.43687193461528e-08
3681 1.44556587861472e-08
3682 1.45668112429276e-08
3683 1.4352791406852e-08
3684 1.44209900527908e-08
3685 1.42693713661574e-08
3686 1.45494189714712e-08
3687 1.45514003990904e-08
3688 1.4296151023982e-08
3689 1.45515748899705e-08
3690 1.42129680909403e-08
3691 1.44861007293356e-08
3692 1.43408262169437e-08
3693 1.44354731446417e-08
3694 1.44541184464997e-08
3695 1.43145555933644e-08
3696 1.41732814193107e-08
3697 1.42389098191364e-08
3698 1.43692026826625e-08
3699 1.42680042219956e-08
3700 1.459970319731e-08
3701 1.41385792139931e-08
3702 1.40285720946764e-08
3703 1.4431134841647e-08
3704 1.44063780154191e-08
3705 1.44114063970746e-08
3706 1.42839427452884e-08
3707 1.43363642842037e-08
3708 1.42763642769772e-08
3709 1.42082187874759e-08
3710 1.41607551745748e-08
3711 1.46037849209912e-08
3712 1.39198321291722e-08
3713 1.42296261344405e-08
3714 1.40052980457939e-08
3715 1.42880965363912e-08
3716 1.42357977737789e-08
3717 1.42392440716371e-08
3718 1.42211410604359e-08
3719 1.42096027378469e-08
3720 1.39652385769962e-08
3721 1.47776790892229e-08
3722 1.35905818767906e-08
3723 1.46331051160825e-08
3724 1.4004596613798e-08
3725 1.40851251178992e-08
3726 1.40143845222124e-08
3727 1.4037646665202e-08
3728 1.41802772895827e-08
3729 1.39966344981785e-08
3730 1.40953203966593e-08
3731 1.42502838919728e-08
3732 1.41173359175378e-08
3733 1.40681907580958e-08
3734 1.42271057819165e-08
3735 1.41773702292541e-08
3736 1.3931211398005e-08
3737 1.42755278752793e-08
3738 1.38327304754909e-08
3739 1.39507135359235e-08
3740 1.40239712888984e-08
3741 1.39744822659354e-08
3742 1.41665418574438e-08
3743 1.38166485975644e-08
3744 1.42191764064825e-08
3745 1.38708948684702e-08
3746 1.40576692685968e-08
3747 1.45007587725454e-08
3748 1.39136662224842e-08
3749 1.36996588719074e-08
3750 1.4169462229402e-08
3751 1.43352683830561e-08
3752 1.41109282744967e-08
3753 1.41541105960763e-08
3754 1.4044293834975e-08
3755 1.40274701312393e-08
3756 1.41374473149047e-08
3757 1.38335723305488e-08
3758 1.4087867997753e-08
3759 1.39878170190288e-08
3760 1.42100759585456e-08
3761 1.42196469021943e-08
3762 1.40077440265451e-08
3763 1.42246846346061e-08
3764 1.50100282276799e-08
3765 1.39593436020435e-08
3766 1.41080232266694e-08
3767 1.39917398183642e-08
3768 1.39133825315246e-08
3769 1.36757070435534e-08
3770 1.51272451300727e-08
3771 1.29430557311572e-08
3772 1.4077617191216e-08
3773 1.37786953145896e-08
3774 1.36330638927651e-08
3775 1.3782148179084e-08
3776 1.36789183467767e-08
3777 1.38694300877984e-08
3778 1.4102488342646e-08
3779 1.38329600559983e-08
3780 1.40080620262134e-08
3781 1.3950766495352e-08
3782 1.4125871946491e-08
3783 1.38007835023962e-08
3784 1.38622892121576e-08
3785 1.40119049705512e-08
3786 1.37912633863951e-08
3787 1.37973123396917e-08
3788 1.39443436256437e-08
3789 1.3912887685083e-08
3790 1.35023964303316e-08
3791 1.44321905179712e-08
3792 1.3752308592116e-08
3793 1.39592979466335e-08
3794 1.39219728695072e-08
3795 1.383209241064e-08
3796 1.42972538628716e-08
3797 1.35215925948368e-08
3798 1.40713558858391e-08
3799 1.37722347967051e-08
3800 1.39034616302575e-08
3801 1.44699940125836e-08
3802 1.32750569054041e-08
3803 1.39661099435445e-08
3804 1.3664450640577e-08
3805 1.35030089871702e-08
3806 1.4314821746117e-08
3807 1.40874115680117e-08
3808 1.397972323057e-08
3809 1.40898362082365e-08
3810 1.41617531985855e-08
3811 1.35725792685809e-08
3812 1.43431853807524e-08
3813 1.40138315076338e-08
3814 1.39326058310602e-08
3815 1.39209089937697e-08
3816 1.40248853691405e-08
3817 1.37615404672276e-08
3818 1.40451680153697e-08
3819 1.34239130015479e-08
3820 1.38806289156773e-08
3821 1.38824198016307e-08
3822 1.38316817389766e-08
3823 1.39209451057859e-08
3824 1.39105500636949e-08
3825 1.39953077472471e-08
3826 1.36338694750604e-08
3827 1.38106919026076e-08
3828 1.38035100013395e-08
3829 1.38256063986764e-08
3830 1.37258131105988e-08
3831 1.38247642009481e-08
3832 1.40687269124529e-08
3833 1.40188619961595e-08
3834 1.36841420152467e-08
3835 1.36854091858374e-08
3836 1.38947246214605e-08
3837 1.37848178606986e-08
3838 1.38053475264993e-08
3839 1.39640104656891e-08
3840 1.38354210071356e-08
3841 1.35499593733879e-08
3842 1.39407176617812e-08
3843 1.40143309952379e-08
3844 1.36098547557917e-08
3845 1.36396761260676e-08
3846 1.39689884973426e-08
3847 1.36492362916851e-08
3848 1.41522285435902e-08
3849 1.38803232992146e-08
3850 1.3689918090462e-08
3851 1.39289868852643e-08
3852 1.37401536803444e-08
3853 1.36021558851535e-08
3854 1.38632108624415e-08
3855 1.39012174297098e-08
3856 1.37540576383016e-08
3857 1.37028431861086e-08
3858 1.38616227784438e-08
3859 1.40291520865227e-08
3860 1.38607445585304e-08
3861 1.35372867216682e-08
3862 1.37939128280334e-08
3863 1.39188502680243e-08
3864 1.3681356125464e-08
3865 1.40286303265402e-08
3866 1.34317922861893e-08
3867 1.46634473162238e-08
3868 1.35728796054474e-08
3869 1.39265719693027e-08
3870 1.42815411615649e-08
3871 1.33945177048789e-08
3872 1.38430161429892e-08
3873 1.38372525873148e-08
3874 1.38310508998457e-08
3875 1.41524844560204e-08
3876 1.39460788486923e-08
3877 1.38139765001877e-08
3878 1.38591916715308e-08
3879 1.40783615758117e-08
3880 1.34289185325098e-08
3881 1.39862770999893e-08
3882 1.36348228672173e-08
3883 1.40113456118113e-08
3884 1.38514157933811e-08
3885 1.37660552799457e-08
3886 1.41760153646031e-08
3887 1.36343560051128e-08
3888 1.3889496958977e-08
3889 1.37755873256817e-08
3890 1.41579214479465e-08
3891 1.32143891897196e-08
3892 1.37026561064929e-08
3893 1.43996094624926e-08
3894 1.3570783493802e-08
3895 1.39223656607512e-08
3896 1.37706111578845e-08
3897 1.34315660642731e-08
3898 1.36538102563782e-08
3899 1.39344725728352e-08
3900 1.38701287867027e-08
3901 1.37208291080548e-08
3902 1.37317304560458e-08
3903 1.39969955116892e-08
3904 1.35513396549852e-08
3905 1.39924633859034e-08
3906 1.36991576741435e-08
3907 1.37496290480155e-08
3908 1.37329689813559e-08
3909 1.38890951760368e-08
3910 1.38049405800228e-08
3911 1.35792190446571e-08
3912 1.38441998921157e-08
3913 1.39080662404784e-08
3914 1.3436040972023e-08
3915 1.36420576705709e-08
3916 1.38115093559071e-08
3917 1.40428050787317e-08
3918 1.36275282113241e-08
3919 1.39590078024471e-08
3920 1.37030597254367e-08
3921 1.34226310904428e-08
3922 1.39183874852539e-08
3923 1.36948143765747e-08
3924 1.37665469493453e-08
3925 1.39174952436399e-08
3926 1.40806587076991e-08
3927 1.39132307318129e-08
3928 1.40876881626201e-08
3929 1.36738307826428e-08
3930 1.39265633196523e-08
3931 1.3783748092383e-08
3932 1.37755193485994e-08
3933 1.38673815422569e-08
3934 1.35356733867997e-08
3935 1.3844600734475e-08
3936 1.36613890097115e-08
3937 1.35004547509587e-08
3938 1.38939283198247e-08
3939 1.35221999973817e-08
3940 1.39656142918859e-08
3941 1.34059910310569e-08
3942 1.37349353595967e-08
3943 1.36336034598367e-08
3944 1.3919394940512e-08
3945 1.37668220139137e-08
3946 1.40246803173733e-08
3947 1.35803413657543e-08
3948 1.3637149429313e-08
3949 1.35127757965037e-08
3950 1.35663123450452e-08
3951 1.38956620540642e-08
3952 1.37873596268573e-08
3953 1.389500478928e-08
3954 1.41921779171356e-08
3955 1.31993182012763e-08
3956 1.35656532124218e-08
3957 1.36282364083112e-08
3958 1.35701460890758e-08
3959 1.34878009735578e-08
3960 1.37990726468695e-08
3961 1.37681858143751e-08
3962 1.36420324621661e-08
3963 1.36053237718425e-08
3964 1.38063610611167e-08
3965 1.38407061601592e-08
3966 1.40161243985665e-08
3967 1.3533568064189e-08
3968 1.40392057074307e-08
3969 1.35268286005769e-08
3970 1.40897960881903e-08
3971 1.34499837557839e-08
3972 1.35653758576715e-08
3973 1.4020159731315e-08
3974 1.37645487304494e-08
3975 1.37371891483218e-08
3976 1.39118706521668e-08
3977 1.38792272484506e-08
3978 1.35885746550546e-08
3979 1.39029441903749e-08
3980 1.38559428358165e-08
3981 1.35535984634311e-08
3982 1.3462829135058e-08
3983 1.35551647392762e-08
3984 1.3747024903199e-08
3985 1.36768024728423e-08
3986 1.37431502682528e-08
3987 1.381484174659e-08
3988 1.35455357931635e-08
3989 1.36133058910026e-08
3990 1.36882003002325e-08
3991 1.37803477152643e-08
3992 1.37190487103228e-08
3993 1.38437307397321e-08
3994 1.35144704440787e-08
3995 1.3794105976031e-08
3996 1.37043577684848e-08
3997 1.3840643533769e-08
3998 1.34506144003899e-08
3999 1.35861270589704e-08
4000 1.39005346494886e-08
4001 1.38183272626047e-08
4002 1.36303394027704e-08
4003 1.36245634381194e-08
4004 1.3680840263125e-08
4005 1.35781910667321e-08
4006 1.37303025162377e-08
4007 1.36343220547286e-08
4008 1.36435178021144e-08
4009 1.34497295611047e-08
4010 1.36708040521238e-08
4011 1.38620091414271e-08
4012 1.37161142409747e-08
4013 1.38177451935034e-08
4014 1.35276180291272e-08
4015 1.34387059442242e-08
4016 1.37084362236556e-08
4017 1.38113068635509e-08
4018 1.37476760893734e-08
4019 1.35783341890378e-08
4020 1.35729475955609e-08
4021 1.36380185672275e-08
4022 1.39614020112644e-08
4023 1.37467739228514e-08
4024 1.36158021585564e-08
4025 1.36477029388932e-08
4026 1.3704889667232e-08
4027 1.40872893482913e-08
4028 1.35973619416196e-08
4029 1.37195930681316e-08
4030 1.34498809742645e-08
4031 1.35371200695178e-08
4032 1.36212882550363e-08
4033 1.35013479600071e-08
4034 1.38860713415762e-08
4035 1.35460875339416e-08
4036 1.3514212863136e-08
4037 1.35410332759989e-08
4038 1.37715818749329e-08
4039 1.3804704706008e-08
4040 1.35676737846729e-08
4041 1.33751017850903e-08
4042 1.3582552846933e-08
4043 1.38557688547092e-08
4044 1.34322057869507e-08
4045 1.38245421331118e-08
4046 1.33117865525612e-08
4047 1.33419675560492e-08
4048 1.35621679040465e-08
4049 1.38144619372116e-08
4050 1.38133937959972e-08
4051 1.36260255404613e-08
4052 1.3596134188526e-08
4053 1.33833608849943e-08
4054 1.33834472949429e-08
4055 1.34710414223665e-08
4056 1.378915325706e-08
4057 1.36682503219521e-08
4058 1.34128664806693e-08
4059 1.36673763113937e-08
4060 1.37671952491897e-08
4061 1.32428241360616e-08
4062 1.36788222597217e-08
4063 1.37333078139845e-08
4064 1.35646130436234e-08
4065 1.36569821799215e-08
4066 1.36328390664131e-08
4067 1.38478220098742e-08
4068 1.34861755490545e-08
4069 1.35063194958984e-08
4070 1.38552865567854e-08
4071 1.37325548410505e-08
4072 1.42426036916221e-08
4073 1.31436410029107e-08
4074 1.37212752766014e-08
4075 1.35917654662809e-08
4076 1.39589447859245e-08
4077 1.3553013394188e-08
4078 1.3523943470678e-08
4079 1.35761654224409e-08
4080 1.3556829263156e-08
4081 1.36514528562698e-08
4082 1.34924813353088e-08
4083 1.35941451101379e-08
4084 1.36555259608317e-08
4085 1.38476003617438e-08
4086 1.34840413565512e-08
4087 1.34236806101362e-08
4088 1.36383139582447e-08
4089 1.35043131805068e-08
4090 1.32874107080794e-08
4091 1.34503462620311e-08
4092 1.37661861772942e-08
4093 1.38237019917109e-08
4094 1.33597941669866e-08
4095 1.3689184557214e-08
4096 1.33212191474885e-08
4097 1.35349681434499e-08
4098 1.3553091652907e-08
4099 1.35637130259381e-08
4100 1.3587759710762e-08
4101 1.35325834363653e-08
4102 1.37372322838308e-08
4103 1.36225042511456e-08
4104 1.3521915970871e-08
4105 1.37262974879426e-08
4106 1.35304324442032e-08
4107 1.35607897480267e-08
4108 1.3464393315038e-08
4109 1.36963132517376e-08
4110 1.35122679077171e-08
4111 1.36241901251e-08
4112 1.36506630121075e-08
4113 1.36419126389592e-08
4114 1.3489391028812e-08
4115 1.33235165034745e-08
4116 1.36801308039253e-08
4117 1.33351804215698e-08
4118 1.32495605363603e-08
4119 1.39164070417364e-08
4120 1.32191820441385e-08
4121 1.32982026497946e-08
4122 1.35835920575256e-08
4123 1.33596119497731e-08
4124 1.3857246768445e-08
4125 1.36163268377176e-08
4126 1.35285388934842e-08
4127 1.34778374773215e-08
4128 1.37295287884059e-08
4129 1.3616267646066e-08
4130 1.3628912802488e-08
4131 1.36125368644102e-08
4132 1.35091172899227e-08
4133 1.34115446070726e-08
4134 1.33896119727805e-08
4135 1.3707995324172e-08
4136 1.34594574842384e-08
4137 1.33991135369271e-08
4138 1.34732973071927e-08
4139 1.34763759311041e-08
4140 1.34860008683679e-08
4141 1.35824347832214e-08
4142 1.33206735322966e-08
4143 1.32530129715869e-08
4144 1.42065448020812e-08
4145 1.33614013278771e-08
4146 1.34177173213845e-08
4147 1.35178055088447e-08
4148 1.32866011416793e-08
4149 1.34990402854418e-08
4150 1.37433578079327e-08
4151 1.34238649024121e-08
4152 1.36251036577661e-08
4153 1.3764550899048e-08
4154 1.34000528387024e-08
4155 1.32887938172344e-08
4156 1.35504170967055e-08
4157 1.34137170361426e-08
4158 1.33148765947555e-08
4159 1.34516905340165e-08
4160 1.34162005106242e-08
4161 1.36333518361564e-08
4162 1.35290906257968e-08
4163 1.33306515047449e-08
4164 1.31293947410738e-08
4165 1.3191180613617e-08
4166 1.35374480642464e-08
4167 1.36954866072764e-08
4168 1.36763480285546e-08
4169 1.35953404069855e-08
4170 1.33925926698586e-08
4171 1.35456742801493e-08
4172 1.34660883157334e-08
4173 1.36889292137871e-08
4174 1.34016290174316e-08
4175 1.33046199504994e-08
4176 1.37332544778573e-08
4177 1.34762914541509e-08
4178 1.35544807915167e-08
4179 1.33686031348196e-08
4180 1.34142808644044e-08
4181 1.33358717092913e-08
4182 1.33436833815725e-08
4183 1.34381543113044e-08
4184 1.34251407494512e-08
4185 1.32760682758221e-08
4186 1.2856335802558e-08
4187 1.35922150872325e-08
4188 1.35345152018906e-08
4189 1.37314359527907e-08
4190 1.34138611187912e-08
4191 1.34814561790486e-08
4192 1.33244847846936e-08
4193 1.35233423649572e-08
4194 1.37962896832183e-08
4195 1.31740589642826e-08
4196 1.35623467731466e-08
4197 1.34438309010998e-08
4198 1.33925725484957e-08
4199 1.33686282881157e-08
4200 1.33232783130488e-08
4201 1.3373875752995e-08
4202 1.36174799593852e-08
4203 1.35109861745392e-08
4204 1.35194571390523e-08
4205 1.3524326529693e-08
4206 1.36029002401616e-08
4207 1.33344639852429e-08
4208 1.32985087879511e-08
4209 1.32609526249661e-08
4210 1.32496700992746e-08
4211 1.34546248140438e-08
4212 1.33924608344826e-08
4213 1.33603590112136e-08
4214 1.32487561997213e-08
4215 1.35176492170597e-08
4216 1.36303298627627e-08
4217 1.30596568519992e-08
4218 1.33541963465711e-08
4219 1.3369071479391e-08
4220 1.36120810988044e-08
4221 1.33614702286788e-08
4222 1.31910921223055e-08
4223 1.33883728377915e-08
4224 1.34309058316406e-08
4225 1.32667014772492e-08
4226 1.32426737858155e-08
4227 1.31499371731347e-08
4228 1.32792941656135e-08
4229 1.32453117229064e-08
4230 1.37239328216326e-08
4231 1.30120693228181e-08
4232 1.32785107272299e-08
4233 1.31124704153135e-08
4234 1.29614212886764e-08
4235 1.36182369093324e-08
4236 1.33089313168327e-08
4237 1.32352743586889e-08
4238 1.32790974135544e-08
4239 1.33232260560029e-08
4240 1.32003594009861e-08
4241 1.31628871869321e-08
4242 1.32783225201966e-08
4243 1.32234295663614e-08
4244 1.34800865450174e-08
4245 1.34446571173064e-08
4246 1.29475257186901e-08
4247 1.34771524551852e-08
4248 1.34420432449656e-08
4249 1.31907885228544e-08
4250 1.35776940189819e-08
4251 1.30979214206794e-08
4252 1.32995400590752e-08
4253 1.32738054005044e-08
4254 1.32454790176539e-08
4255 1.32018774925968e-08
4256 1.30777397421844e-08
4257 1.32976877899732e-08
4258 1.31517608132625e-08
4259 1.3294601732855e-08
4260 1.31223376067524e-08
4261 1.33041520701399e-08
4262 1.33227426199617e-08
4263 1.36571092010096e-08
4264 1.31942867594576e-08
4265 1.33588675613749e-08
4266 1.31919120683888e-08
4267 1.35213847201748e-08
4268 1.3184960669238e-08
4269 1.34068764901379e-08
4270 1.40607365032575e-08
4271 1.32064735945681e-08
4272 1.30170380427397e-08
4273 1.39103889195968e-08
4274 1.32872802639517e-08
4275 1.33107627028617e-08
4276 1.33091509931377e-08
4277 1.32216164285826e-08
4278 1.29869817364764e-08
4279 1.33960413427892e-08
4280 1.31286848841228e-08
4281 1.29500837605379e-08
4282 1.33054596912346e-08
4283 1.32388140041545e-08
4284 1.31102778615499e-08
4285 1.32306761168044e-08
4286 1.33905225244979e-08
4287 1.33709987401032e-08
4288 1.35887468570572e-08
4289 1.3092257781136e-08
4290 1.30964530649369e-08
4291 1.30306221443888e-08
4292 1.3265505452581e-08
4293 1.31513285114987e-08
4294 1.30719460787188e-08
4295 1.31757019835588e-08
4296 1.31645008268899e-08
4297 1.33213163410612e-08
4298 1.30179427967753e-08
4299 1.35187722423713e-08
4300 1.30096873803831e-08
4301 1.32440070697015e-08
4302 1.32954042486172e-08
4303 1.36549159108906e-08
4304 1.31512597748018e-08
4305 1.31655698972916e-08
4306 1.33834541262146e-08
4307 1.31096814254827e-08
4308 1.30340767384718e-08
4309 1.37275832674116e-08
4310 1.28716474368379e-08
4311 1.32454504136875e-08
4312 1.35050602648734e-08
4313 1.32234820565402e-08
4314 1.36554738924544e-08
4315 1.3253965558499e-08
4316 1.33276319772696e-08
4317 1.36549989379064e-08
4318 1.30505531295705e-08
4319 1.33402567069202e-08
4320 1.31512399160899e-08
4321 1.3326595049129e-08
4322 1.3267517895002e-08
4323 1.31517145939841e-08
4324 1.32865264483445e-08
4325 1.32732543961789e-08
4326 1.37445696873628e-08
4327 1.31899273896202e-08
4328 1.32244538567222e-08
4329 1.33679648810642e-08
4330 1.31431576975011e-08
4331 1.31020058382753e-08
4332 1.343812007365e-08
4333 1.31205532297174e-08
4334 1.35644238650617e-08
4335 1.29432128024853e-08
4336 1.34027172060819e-08
4337 1.34854155569436e-08
4338 1.30665313601563e-08
4339 1.32606839298999e-08
4340 1.30631276785836e-08
4341 1.34028939204334e-08
4342 1.31788486859852e-08
4343 1.32197965462405e-08
4344 1.34261605857361e-08
4345 1.28350922784098e-08
4346 1.33700318540736e-08
4347 1.32188238463199e-08
4348 1.36485014491372e-08
4349 1.36383239262017e-08
4350 1.297920755855e-08
4351 1.32150673587128e-08
4352 1.27043033828211e-08
4353 1.31530684711612e-08
4354 1.31339647868739e-08
4355 1.2998820252394e-08
4356 1.30652630609684e-08
4357 1.36098854733729e-08
4358 1.35226051468651e-08
4359 1.2965150365099e-08
4360 1.34344481206028e-08
4361 1.30101588171755e-08
4362 1.31695634378454e-08
4363 1.28182635182189e-08
4364 1.31879790489925e-08
4365 1.32528433919865e-08
4366 1.3148410928171e-08
4367 1.30988715567326e-08
4368 1.27995171463335e-08
4369 1.34156412045644e-08
4370 1.31014481928315e-08
4371 1.31832682415539e-08
4372 1.28145493120402e-08
4373 1.33290179228363e-08
4374 1.31091478553086e-08
4375 1.31077763938642e-08
4376 1.30598084032985e-08
4377 1.33094952258772e-08
4378 1.31075031922778e-08
4379 1.32612991205067e-08
4380 1.31614358742232e-08
4381 1.32408065381534e-08
4382 1.25542365331521e-08
4383 1.34862032444144e-08
4384 1.31072405427368e-08
4385 1.30523697055851e-08
4386 1.31546324476095e-08
4387 1.30318239339733e-08
4388 1.32829402173795e-08
4389 1.35036964464125e-08
4390 1.26598078185047e-08
4391 1.32028139858548e-08
4392 1.2946399443034e-08
4393 1.33827740011194e-08
4394 1.30785292603297e-08
4395 1.33364334020669e-08
4396 1.31507063190206e-08
4397 1.31349178502105e-08
4398 1.3197638798923e-08
4399 1.30108248116018e-08
4400 1.30538324159962e-08
4401 1.34347644293459e-08
4402 1.33762354308509e-08
4403 1.29954318914827e-08
4404 1.33791857282323e-08
4405 1.42752958026265e-08
4406 1.27724579558758e-08
4407 1.30673234288448e-08
4408 1.30350210919977e-08
4409 1.34522298797651e-08
4410 1.31798444452796e-08
4411 1.30222537948615e-08
4412 1.30407352372103e-08
4413 1.29744076581284e-08
4414 1.29885877248603e-08
4415 1.32277465715314e-08
4416 1.29183259770932e-08
4417 1.33733459904001e-08
4418 1.30145540810922e-08
4419 1.29440829187838e-08
4420 1.28230143008196e-08
4421 1.29098838006769e-08
4422 1.32113965129743e-08
4423 1.29314521113977e-08
4424 1.31423010607395e-08
4425 1.28597741869269e-08
4426 1.31556113327214e-08
4427 1.31282919978992e-08
4428 1.31488176746963e-08
4429 1.34952134836219e-08
4430 1.28293438321769e-08
4431 1.31151443243105e-08
4432 1.35533072843225e-08
4433 1.32314829173197e-08
4434 1.30597291116419e-08
4435 1.31205687851632e-08
4436 1.31293758537793e-08
4437 1.32806504639371e-08
4438 1.29569300392346e-08
4439 1.28938359158504e-08
4440 1.29930981373205e-08
4441 1.32908136079718e-08
4442 1.28307898223379e-08
4443 1.31225982940714e-08
4444 1.28808001697012e-08
4445 1.29653922673612e-08
4446 1.27550072312188e-08
4447 1.28989775417532e-08
4448 1.29408463955816e-08
4449 1.30107915528033e-08
4450 1.29361655168136e-08
4451 1.31361236910682e-08
4452 1.29970901930204e-08
4453 1.27937087475299e-08
4454 1.30928539290032e-08
4455 1.29919078396962e-08
4456 1.29132231660445e-08
4457 1.33016700994415e-08
4458 1.32996754879239e-08
4459 1.28177393275142e-08
4460 1.3620635214262e-08
4461 1.27701639586891e-08
4462 1.30807826042162e-08
4463 1.3141799510008e-08
4464 1.32467738914593e-08
4465 1.29736012574738e-08
4466 1.33348010961792e-08
4467 1.31236330704887e-08
4468 1.29514744776349e-08
4469 1.33377582627187e-08
4470 1.31631762754075e-08
4471 1.31343094941644e-08
4472 1.28853416501012e-08
4473 1.31901999436967e-08
4474 1.33062741709267e-08
4475 1.29804329566802e-08
4476 1.32901931675367e-08
4477 1.32568898210395e-08
4478 1.30875613065101e-08
4479 1.32022748910721e-08
4480 1.30522572098224e-08
4481 1.31136227753403e-08
4482 1.27189534008393e-08
4483 1.28261946238106e-08
4484 1.33588692326769e-08
4485 1.3280904051241e-08
4486 1.29794560066454e-08
4487 1.30065058336071e-08
4488 1.29818798054598e-08
4489 1.29847048669363e-08
4490 1.28919983229669e-08
4491 1.2959387713507e-08
4492 1.29883109098589e-08
4493 1.29113036975714e-08
4494 1.29871511140023e-08
4495 1.28308110897285e-08
4496 1.3531306463449e-08
4497 1.31587937963612e-08
4498 1.30779227058841e-08
4499 1.30481866243842e-08
4500 1.32490716647921e-08
4501 1.31401299318101e-08
4502 1.29657300584757e-08
4503 1.30456294652331e-08
4504 1.32916757270424e-08
4505 1.27087701221329e-08
4506 1.33495318098281e-08
4507 1.30590718114693e-08
4508 1.31306049711427e-08
4509 1.26709733637431e-08
4510 1.29888297639991e-08
4511 1.2968087579196e-08
4512 1.31738963367878e-08
4513 1.28331551585564e-08
4514 1.35817497475821e-08
4515 1.30173207789214e-08
4516 1.32125228477359e-08
4517 1.31840131202959e-08
4518 1.30858182795374e-08
4519 1.32125186949605e-08
4520 1.31345343532402e-08
4521 1.28874707527593e-08
4522 1.28919198068428e-08
4523 1.31350556774185e-08
4524 1.26958646042286e-08
4525 1.27897964027901e-08
4526 1.29655585223432e-08
4527 1.32831012570334e-08
4528 1.32443363592555e-08
4529 1.32426527784668e-08
4530 1.26825770271066e-08
4531 1.3408183538996e-08
4532 1.28250887314468e-08
4533 1.30743876678302e-08
4534 1.29457030919461e-08
4535 1.27335047122729e-08
4536 1.29840570689405e-08
4537 1.29914578252815e-08
4538 1.3134796287298e-08
4539 1.27481108717076e-08
4540 1.31849083478686e-08
4541 1.35716039575684e-08
4542 1.2438255635891e-08
4543 1.33296196716121e-08
4544 1.29952222527457e-08
4545 1.32208519619537e-08
4546 1.27831660207689e-08
4547 1.31030078364958e-08
4548 1.26833881679972e-08
4549 1.28227900536571e-08
4550 1.26985583612516e-08
4551 1.27208555701086e-08
4552 1.35245897673469e-08
4553 1.27001056119264e-08
4554 1.32826847250023e-08
4555 1.28047183243496e-08
4556 1.29626070777961e-08
4557 1.34490973272006e-08
4558 1.27441567289027e-08
4559 1.31310686175556e-08
4560 1.31175620958351e-08
4561 1.29092461568781e-08
4562 1.29684897937632e-08
4563 1.31256624166659e-08
4564 1.31330326431367e-08
4565 1.30170881624536e-08
4566 1.35705993056651e-08
4567 1.27376327232198e-08
4568 1.29552371863628e-08
4569 1.29998013271154e-08
4570 1.30915978988899e-08
4571 1.35690808170524e-08
4572 1.25848515971361e-08
4573 1.33156566690246e-08
4574 1.3057063496219e-08
4575 1.29032710743382e-08
4576 1.31382066241026e-08
4577 1.25567229948276e-08
4578 1.27449805975566e-08
4579 1.31498571533106e-08
4580 1.2783330698693e-08
4581 1.3109462971686e-08
4582 1.27921291969979e-08
4583 1.29699330091992e-08
4584 1.29396619799466e-08
4585 1.29891844878743e-08
4586 1.28758522315586e-08
4587 1.27247756417093e-08
4588 1.29921470492927e-08
4589 1.27843422095264e-08
4590 1.28263867734402e-08
4591 1.31477752242232e-08
4592 1.29764418576178e-08
4593 1.28343637017525e-08
4594 1.3060026463757e-08
4595 1.27319574634577e-08
4596 1.30331497610325e-08
4597 1.27840675637897e-08
4598 1.2814637035008e-08
4599 1.31008383563863e-08
4600 1.30164291405332e-08
4601 1.27632119900078e-08
4602 1.29809321270796e-08
4603 1.28884139886792e-08
4604 1.2944085374042e-08
4605 1.36238810732603e-08
4606 1.25069450522264e-08
4607 1.30939418198917e-08
4608 1.28394741788634e-08
4609 1.28778909369298e-08
4610 1.28305015857366e-08
4611 1.31027643682086e-08
4612 1.31929830689131e-08
4613 1.30026672822692e-08
4614 1.26861520227672e-08
4615 1.3242859036422e-08
4616 1.30208608979665e-08
4617 1.28957770320487e-08
4618 1.28794465018567e-08
4619 1.24246127465033e-08
4620 1.31919068565439e-08
4621 1.30275659550733e-08
4622 1.31209785403652e-08
4623 1.29041731259244e-08
4624 1.33596328098784e-08
4625 1.30197117211711e-08
4626 1.27660186609868e-08
4627 1.29404371091296e-08
4628 1.27384432064281e-08
4629 1.31944278105728e-08
4630 1.28891949229548e-08
4631 1.25416641167664e-08
4632 1.27297231364148e-08
4633 1.28086266833888e-08
4634 1.30094835501943e-08
4635 1.31425114198863e-08
4636 1.29950267204054e-08
4637 1.28586141091569e-08
4638 1.306638336529e-08
4639 1.28725148222514e-08
4640 1.30725426246148e-08
4641 1.29414071762812e-08
4642 1.31229471630051e-08
4643 1.32762600693614e-08
4644 1.2923829041514e-08
4645 1.27576847009092e-08
4646 1.32461431544689e-08
4647 1.28108995623155e-08
4648 1.30768505664652e-08
4649 1.32034354177607e-08
4650 1.28937872310303e-08
4651 1.29381591923716e-08
4652 1.28896623408786e-08
4653 1.29311663726039e-08
4654 1.30769854580493e-08
4655 1.28306733761518e-08
4656 1.27302731188078e-08
4657 1.33015369260109e-08
4658 1.30062215800142e-08
4659 1.30585125563548e-08
4660 1.26863521793391e-08
4661 1.32467825182253e-08
4662 1.33687929152471e-08
4663 1.25363319839339e-08
4664 1.33670395457558e-08
4665 1.29880846100605e-08
4666 1.29676656977773e-08
4667 1.30441263544168e-08
4668 1.27514700188347e-08
4669 1.31199516508473e-08
4670 1.30292971317336e-08
4671 1.32567677928325e-08
4672 1.26027202630996e-08
4673 1.31931854092382e-08
4674 1.28650237609523e-08
4675 1.31355232942804e-08
4676 1.30253596985624e-08
4677 1.31819473492556e-08
4678 1.28273034007975e-08
4679 1.28467171315594e-08
4680 1.29138783038774e-08
4681 1.29996891310435e-08
4682 1.30267933839828e-08
4683 1.27700785111651e-08
4684 1.29607890275857e-08
4685 1.28487232561036e-08
4686 1.27292183069588e-08
4687 1.29346319608925e-08
4688 1.29066432871389e-08
4689 1.30710846255044e-08
4690 1.28710976227409e-08
4691 1.27005516307871e-08
4692 1.32084616542433e-08
4693 1.31725873005473e-08
4694 1.27466180761676e-08
4695 1.30544798218613e-08
4696 1.28575850557172e-08
4697 1.31216506464166e-08
4698 1.28142181437252e-08
4699 1.30457833538294e-08
4700 1.29999486896681e-08
4701 1.27748725398946e-08
4702 1.3204233277872e-08
4703 1.34817647323149e-08
4704 1.3287303667342e-08
4705 1.2796475909238e-08
4706 1.33787124405732e-08
4707 1.26704807378863e-08
4708 1.27689249022067e-08
4709 1.28582212104156e-08
4710 1.2626181203193e-08
4711 1.27403014400507e-08
4712 1.2723715856533e-08
4713 1.28717251347277e-08
4714 1.3051013816337e-08
4715 1.28292723423906e-08
4716 1.3060904442988e-08
4717 1.28971714923043e-08
4718 1.28280368354161e-08
4719 1.2821115781339e-08
4720 1.27973929112957e-08
4721 1.29689401134891e-08
4722 1.2759322923897e-08
4723 1.30892280081618e-08
4724 1.30585503204383e-08
4725 1.26272815301198e-08
4726 1.32113139268419e-08
4727 1.2827026082296e-08
4728 1.27167551506185e-08
4729 1.29048067355614e-08
4730 1.30034485604502e-08
4731 1.33208029595572e-08
4732 1.29308991272115e-08
4733 1.28214745424154e-08
4734 1.28235412211464e-08
4735 1.31784481162134e-08
4736 1.28335332626739e-08
4737 1.30068523860327e-08
4738 1.28293430960574e-08
4739 1.29144646446033e-08
4740 1.28995160498108e-08
4741 1.26221520700548e-08
4742 1.28324658137668e-08
4743 1.28938531389344e-08
4744 1.28153310181983e-08
4745 1.27366965280706e-08
4746 1.31569396573106e-08
4747 1.26725232064695e-08
4748 1.30075352263864e-08
4749 1.31165838557906e-08
4750 1.25363050171301e-08
4751 1.30746677737825e-08
4752 1.25849912804843e-08
4753 1.27432226156021e-08
4754 1.29595728164678e-08
4755 1.26176800851613e-08
4756 1.27891017451254e-08
4757 1.25602945409142e-08
4758 1.30656941822876e-08
4759 1.30146424450195e-08
4760 1.29602834881976e-08
4761 1.30427066766386e-08
4762 1.27664745600276e-08
4763 1.2542790015474e-08
4764 1.29029625746818e-08
4765 1.27837049243712e-08
4766 1.26967182678128e-08
4767 1.29418051145469e-08
4768 1.27158228037799e-08
4769 1.3026307325012e-08
4770 1.27943353940146e-08
4771 1.26518436955741e-08
4772 1.28219076869079e-08
4773 1.3102739601728e-08
4774 1.31818383132809e-08
4775 1.28029806621827e-08
4776 1.30172341849116e-08
4777 1.2797678823312e-08
4778 1.31010731085346e-08
4779 1.28029753587861e-08
4780 1.25875221247407e-08
4781 1.29827863985937e-08
4782 1.31517324661878e-08
4783 1.263552092369e-08
4784 1.29296666635248e-08
4785 1.28005525406594e-08
4786 1.2819232782485e-08
4787 1.32764636893434e-08
4788 1.25692889632462e-08
4789 1.3169001660554e-08
4790 1.30311471544797e-08
4791 1.26980088425016e-08
4792 1.31012888846815e-08
4793 1.26708424853239e-08
4794 1.32590295565671e-08
4795 1.27040036335124e-08
4796 1.25655711764316e-08
4797 1.29091977557616e-08
4798 1.31561416982229e-08
4799 1.28413184984355e-08
4800 1.32341369764588e-08
4801 1.25057044357557e-08
4802 1.29729344253149e-08
4803 1.29965734556453e-08
4804 1.26676506869289e-08
4805 1.28965118407487e-08
4806 1.24481274571403e-08
4807 1.28583638573931e-08
4808 1.27118488269101e-08
4809 1.28618767835098e-08
4810 1.30749005772812e-08
4811 1.29259413087085e-08
4812 1.28660851501233e-08
4813 1.3157728813662e-08
4814 1.28052654473076e-08
4815 1.28957416539038e-08
4816 1.30596810861e-08
4817 1.27894117685073e-08
4818 1.29917634598409e-08
4819 1.2660966955666e-08
4820 1.28289811217042e-08
4821 1.29353873954963e-08
4822 1.28886957151964e-08
4823 1.28525509733579e-08
4824 1.29771452234695e-08
4825 1.30019138364496e-08
4826 1.3053947674424e-08
4827 1.27629517327338e-08
4828 1.31279596530531e-08
4829 1.27337296678132e-08
4830 1.29649054313452e-08
4831 1.28092651611039e-08
4832 1.32326628091012e-08
4833 1.23954886785821e-08
4834 1.25491447101472e-08
4835 1.30181708002386e-08
4836 1.25668973780524e-08
4837 1.30203560617381e-08
4838 1.27915738466489e-08
4839 1.31857611227609e-08
4840 1.25859062923978e-08
4841 1.31939431467498e-08
4842 1.31819816273082e-08
4843 1.28876489312313e-08
4844 1.28847288848877e-08
4845 1.2799917316661e-08
4846 1.27609476957419e-08
4847 1.31473546676059e-08
4848 1.28979067590268e-08
4849 1.29845594820516e-08
4850 1.29293298091054e-08
4851 1.2836872496591e-08
4852 1.30367553834165e-08
4853 1.2981211732363e-08
4854 1.26179806529403e-08
4855 1.28601690325969e-08
4856 1.24575215208894e-08
4857 1.29217060892417e-08
4858 1.25899479927888e-08
4859 1.27897948877659e-08
4860 1.26689516535966e-08
4861 1.28272483134184e-08
4862 1.25822769906636e-08
4863 1.29321904134422e-08
4864 1.28647396461651e-08
4865 1.2975281647204e-08
4866 1.29125726134743e-08
4867 1.27894101994985e-08
4868 1.29869823015938e-08
4869 1.2978565796587e-08
4870 1.30112406720451e-08
4871 1.30455620000747e-08
4872 1.26035083864023e-08
4873 1.32011977984764e-08
4874 1.27461139083213e-08
4875 1.27420802815553e-08
4876 1.31638091034336e-08
4877 1.27697459132436e-08
4878 1.27151296766131e-08
4879 1.27737779429954e-08
4880 1.33632969397252e-08
4881 1.27775827548499e-08
4882 1.28018742564517e-08
4883 1.28676062160493e-08
4884 1.2960959160771e-08
4885 1.27081919968075e-08
4886 1.27552495314681e-08
4887 1.32248002268076e-08
4888 1.24803717528599e-08
4889 1.33952773442858e-08
4890 1.24862761489986e-08
4891 1.32010253992287e-08
4892 1.26129715345286e-08
4893 1.29475138578306e-08
4894 1.28509033333474e-08
4895 1.30185403859617e-08
4896 1.30548180060086e-08
4897 1.28235093066476e-08
4898 1.2677211963158e-08
4899 1.30374408407047e-08
4900 1.28092941057145e-08
4901 1.30149558212889e-08
4902 1.30474528834967e-08
4903 1.27689435033412e-08
4904 1.29200623552594e-08
4905 1.33033985629666e-08
4906 1.24666108310412e-08
4907 1.30093926228175e-08
4908 1.26476945502829e-08
4909 1.26458434634019e-08
4910 1.28899857715914e-08
4911 1.36582854863287e-08
4912 1.26344846391874e-08
4913 1.28040542664853e-08
4914 1.31058173021975e-08
4915 1.26777445045717e-08
4916 1.2881079511029e-08
4917 1.28152687251942e-08
4918 1.29206278645239e-08
4919 1.28681102588707e-08
4920 1.2737347856881e-08
4921 1.28443715230742e-08
4922 1.27466114287905e-08
4923 1.30853325694541e-08
4924 1.2863812705835e-08
4925 1.25749652210433e-08
4926 1.30580940068958e-08
4927 1.3258789432119e-08
4928 1.31978872879857e-08
4929 1.23673667958568e-08
4930 1.32896604827099e-08
4931 1.31045289004511e-08
4932 1.22539286562529e-08
4933 1.27376357518111e-08
4934 1.34718055633531e-08
4935 1.28591894625191e-08
4936 1.31895504891372e-08
4937 1.26753659425299e-08
4938 1.27053464364368e-08
4939 1.29720949690881e-08
4940 1.2717938560447e-08
4941 1.30315189761504e-08
4942 1.26254753887861e-08
4943 1.30364400789396e-08
4944 1.27455752264305e-08
4945 1.30780049645424e-08
4946 1.26293813037803e-08
4947 1.33473489832281e-08
4948 1.25315001415632e-08
4949 1.30687320783057e-08
4950 1.27528909927843e-08
4951 1.33203006656146e-08
4952 1.25560515719658e-08
4953 1.27870195036489e-08
4954 1.27599832532577e-08
4955 1.30526254576158e-08
4956 1.32259908464383e-08
4957 1.29287848874837e-08
4958 1.27137060618454e-08
4959 1.27926347702201e-08
4960 1.29190337478796e-08
4961 1.37793756380122e-08
4962 1.20645201804326e-08
4963 1.33781664074656e-08
4964 1.30610722673397e-08
4965 1.25353230823466e-08
4966 1.30891645263587e-08
4967 1.24017882477956e-08
4968 1.30328088890075e-08
4969 1.27558922027871e-08
4970 1.33169791794591e-08
4971 1.27554890066855e-08
4972 1.30940079350744e-08
4973 1.292705601319e-08
4974 1.26724673171208e-08
4975 1.3161207742321e-08
4976 1.25609173506902e-08
4977 1.29994745429601e-08
4978 1.26616697759818e-08
4979 1.30209682356469e-08
4980 1.29769598319202e-08
4981 1.30410783559382e-08
4982 1.23555575762546e-08
4983 1.35210629061411e-08
4984 1.25184134622675e-08
4985 1.31403591329543e-08
4986 1.27320362731387e-08
4987 1.24431980601525e-08
4988 1.30023610245694e-08
4989 1.30563304766851e-08
4990 1.25156422165901e-08
4991 1.29361751022294e-08
4992 1.30920197602968e-08
4993 1.27058908088451e-08
4994 1.30621166848643e-08
4995 1.25923898224578e-08
4996 1.33749360499147e-08
4997 1.24326246972312e-08
4998 1.29659620470846e-08
4999 1.294472577551e-08
};
\addlegendentry{Train}
\addplot [semithick, black]
table {%
0 0.000777359528001398
1 0.000187346624443308
2 8.65513939061202e-05
3 1.83830889000092e-05
4 1.34670754050603e-05
5 1.01761179394089e-05
6 6.73514796289965e-06
7 4.25538701165351e-06
8 2.84654197457712e-06
9 2.18189506995259e-06
10 1.83603970071999e-06
11 1.54532961005316e-06
12 1.29179306895821e-06
13 1.05860533494706e-06
14 9.39080791795277e-07
15 8.61741114022152e-07
16 8.09938967449852e-07
17 7.85533188718546e-07
18 7.72480120758701e-07
19 7.57008706386841e-07
20 7.37843549813988e-07
21 7.29057603621186e-07
22 7.15740895884664e-07
23 7.03616933606099e-07
24 6.88216005073627e-07
25 6.74261343647231e-07
26 6.62238676341076e-07
27 6.52030053061026e-07
28 6.43392297661194e-07
29 6.34511025054962e-07
30 6.26706764705887e-07
31 6.19682509750419e-07
32 6.12294627444498e-07
33 6.07001140906505e-07
34 6.02098566560016e-07
35 5.98068879753555e-07
36 5.94000823639362e-07
37 5.83144242227718e-07
38 5.85827478971623e-07
39 5.72973135604116e-07
40 5.55597750917514e-07
41 5.45317050182348e-07
42 5.38703829988663e-07
43 5.36432196440728e-07
44 5.38036772468331e-07
45 5.3760618357046e-07
46 5.37702419478592e-07
47 5.37043661097414e-07
48 5.35787194166915e-07
49 5.34447735844878e-07
50 5.32961223598249e-07
51 5.33149545844935e-07
52 5.32697129074222e-07
53 5.29584781361336e-07
54 5.28703026247968e-07
55 5.27298084307404e-07
56 5.28481280070991e-07
57 5.27101235547889e-07
58 5.3791455911778e-07
59 5.36206300694175e-07
60 5.35509059318429e-07
61 5.34878950020357e-07
62 5.3413577916217e-07
63 5.33274260305916e-07
64 5.32769774963526e-07
65 5.32295246102876e-07
66 5.31823957317101e-07
67 5.31233922629326e-07
68 5.30216084371204e-07
69 5.29460407960869e-07
70 5.28835187196819e-07
71 5.28820237377658e-07
72 5.2834064945273e-07
73 5.27815473105875e-07
74 5.27101633451821e-07
75 5.26659675870178e-07
76 5.2648744031103e-07
77 5.26130520484003e-07
78 5.25434188602958e-07
79 5.25198970535712e-07
80 5.24480299191055e-07
81 5.2383882120921e-07
82 5.23312849054491e-07
83 5.22808988989709e-07
84 5.22292225468846e-07
85 5.24915890309785e-07
86 5.25109783211519e-07
87 5.25396103512321e-07
88 5.24752749697655e-07
89 5.24684537595022e-07
90 5.22752202414267e-07
91 5.19840455126541e-07
92 5.20170772233541e-07
93 5.19807429100183e-07
94 5.18687841122301e-07
95 5.17908858910232e-07
96 5.16508123382664e-07
97 5.1506475529095e-07
98 5.13530267198803e-07
99 5.12421138409991e-07
100 5.11966334215685e-07
101 5.10309178025636e-07
102 5.08496100337652e-07
103 5.06749188389222e-07
104 5.05221919411269e-07
105 5.03618878155976e-07
106 5.02419936765364e-07
107 5.00481121434859e-07
108 4.98291115036409e-07
109 4.95753567975044e-07
110 4.93978291160602e-07
111 4.91493892695871e-07
112 4.88971920731274e-07
113 4.86799422105832e-07
114 4.84047404825105e-07
115 4.77726700864878e-07
116 4.69826318294508e-07
117 4.6575755163758e-07
118 4.61183304878432e-07
119 4.57334209613691e-07
120 4.52497943115304e-07
121 4.47717525275948e-07
122 4.43374943870367e-07
123 4.36790514868335e-07
124 4.28024122811621e-07
125 4.19477572677351e-07
126 4.11483910056631e-07
127 4.02599283688687e-07
128 3.93052431491014e-07
129 3.83601815201473e-07
130 3.73567388578522e-07
131 3.62550963473041e-07
132 3.50950870142697e-07
133 3.39161914553188e-07
134 3.28222739653938e-07
135 3.17283110007338e-07
136 3.10856677288029e-07
137 3.03670674384193e-07
138 2.93439939014206e-07
139 2.80101943417321e-07
140 2.62831150621423e-07
141 2.57546020066002e-07
142 2.50567467219298e-07
143 2.45113710661826e-07
144 2.40527953110359e-07
145 2.3659191583647e-07
146 2.34661726494778e-07
147 2.30480736718164e-07
148 2.27243546646605e-07
149 2.20341874523911e-07
150 2.15194333463842e-07
151 2.10820829238401e-07
152 2.06476215680595e-07
153 2.01864651216965e-07
154 1.97809768565094e-07
155 1.93859477803926e-07
156 1.90396718835473e-07
157 1.87455142963699e-07
158 1.85121763252027e-07
159 1.82732492248761e-07
160 1.80646637204518e-07
161 1.77969923242927e-07
162 1.76131564444404e-07
163 1.74486274318042e-07
164 1.72949697230251e-07
165 1.71402021464928e-07
166 1.70101259300282e-07
167 1.67175969068012e-07
168 1.657974024738e-07
169 1.64478834108195e-07
170 1.63152478194206e-07
171 1.62308182893867e-07
172 1.61354023475724e-07
173 1.59227951712637e-07
174 1.5858236679378e-07
175 1.57711923520765e-07
176 1.5669226627324e-07
177 1.56149951635598e-07
178 1.55781108901465e-07
179 1.55022377157366e-07
180 1.54382618688942e-07
181 1.53938870539605e-07
182 1.53378223899381e-07
183 1.53023378857142e-07
184 1.5193595004348e-07
185 1.52895083260773e-07
186 1.55070082996644e-07
187 1.5518881468779e-07
188 1.54808773800141e-07
189 1.53124418034167e-07
190 1.53141172631877e-07
191 1.53622124798858e-07
192 1.53419208004379e-07
193 1.53281007442274e-07
194 1.52916783235924e-07
195 1.51894795408225e-07
196 1.48734059735034e-07
197 1.47840410136268e-07
198 1.48023275414744e-07
199 1.46965533076582e-07
200 1.45964307307622e-07
201 1.45697583775473e-07
202 1.4520246338634e-07
203 1.43728016155364e-07
204 1.4335148534883e-07
205 1.43084122328219e-07
206 1.42727628826833e-07
207 1.43111634542947e-07
208 1.4319482488645e-07
209 1.42944685421753e-07
210 1.4270908366143e-07
211 1.42267666092266e-07
212 1.41815192478134e-07
213 1.43327369528379e-07
214 1.40363127343335e-07
215 1.42094194188758e-07
216 1.39744003035958e-07
217 1.41339441483979e-07
218 1.38784258751912e-07
219 1.40305743911995e-07
220 1.37711452907752e-07
221 1.39203663707121e-07
222 1.38730896992456e-07
223 1.37619110773812e-07
224 1.37259647203791e-07
225 1.36470845291115e-07
226 1.35781050403239e-07
227 1.36284555196653e-07
228 1.35210598273261e-07
229 1.34834237996984e-07
230 1.3455149883157e-07
231 1.34182030819829e-07
232 1.33906652877158e-07
233 1.33486153686135e-07
234 1.3322920722203e-07
235 1.3252713415568e-07
236 1.32118785245439e-07
237 1.31591704644052e-07
238 1.31288501847848e-07
239 1.30750649418587e-07
240 1.30325943814569e-07
241 1.26580374626428e-07
242 1.26107138953557e-07
243 1.2464097665088e-07
244 1.24805595191901e-07
245 1.23907909710397e-07
246 1.23419866326913e-07
247 1.23333393275971e-07
248 1.22963228932349e-07
249 1.23205836644047e-07
250 1.22857159112755e-07
251 1.21833394928217e-07
252 1.21694966992436e-07
253 1.2090788459318e-07
254 1.21152922361034e-07
255 1.20481743692835e-07
256 1.19927051400737e-07
257 1.18985219899059e-07
258 1.18395476533806e-07
259 1.18400137694152e-07
260 1.18017425165817e-07
261 1.17792190224009e-07
262 1.17284159273368e-07
263 1.16871220257053e-07
264 1.17445509317804e-07
265 1.17097769702923e-07
266 1.16662903337783e-07
267 1.16198272337442e-07
268 1.15801896072298e-07
269 1.15846304993283e-07
270 1.14524311811692e-07
271 1.14929633809879e-07
272 1.14892948488432e-07
273 1.14556321761938e-07
274 1.14178469345916e-07
275 1.14596119260568e-07
276 1.13696309256284e-07
277 1.12857954093215e-07
278 1.12506100435894e-07
279 1.12747130742719e-07
280 1.11450539463931e-07
281 1.11759256071764e-07
282 1.12033468724348e-07
283 1.11683199577328e-07
284 1.10532347719072e-07
285 1.10940959530126e-07
286 1.1007589506562e-07
287 1.09623272237513e-07
288 1.09278687432379e-07
289 1.08825517486366e-07
290 1.08437831158881e-07
291 1.08261843934088e-07
292 1.07873020738225e-07
293 1.07580788721862e-07
294 1.07441238128558e-07
295 1.07035205587636e-07
296 1.06707972236109e-07
297 1.06524957743659e-07
298 1.0631932667593e-07
299 1.05661108307231e-07
300 1.0664004435057e-07
301 1.05597180777295e-07
302 1.05124783544852e-07
303 1.04606471040825e-07
304 1.04209938456279e-07
305 1.04301811632013e-07
306 1.04022319646901e-07
307 1.02712355953827e-07
308 1.03009753615879e-07
309 1.0266254690805e-07
310 1.02187165396117e-07
311 1.02836459348055e-07
312 1.01949851227801e-07
313 1.02012407410257e-07
314 1.01794164208968e-07
315 1.01469701974111e-07
316 1.01550540421158e-07
317 1.00315773465809e-07
318 9.96960523025336e-08
319 1.00564683691573e-07
320 9.89142776575136e-08
321 9.88256161349454e-08
322 9.8214435695354e-08
323 9.8053767771944e-08
324 9.76315845946374e-08
325 9.78619496549982e-08
326 9.69887707924499e-08
327 9.73915632584976e-08
328 9.69260298688823e-08
329 9.63865645076112e-08
330 9.60307033892605e-08
331 9.56481471803272e-08
332 9.59278594336865e-08
333 9.5534765875982e-08
334 9.49589988863409e-08
335 9.6131294924362e-08
336 9.44246423273398e-08
337 9.52402956500009e-08
338 9.40729734111301e-08
339 9.35659230094643e-08
340 9.35213435582227e-08
341 9.31104509049874e-08
342 9.30594552528419e-08
343 9.28070491568178e-08
344 9.26096817011057e-08
345 9.24633170029665e-08
346 9.22616578691304e-08
347 9.15201638918006e-08
348 9.16448641419265e-08
349 9.10687489863449e-08
350 9.15740301365986e-08
351 9.07804249550281e-08
352 9.02917420830818e-08
353 9.02443559880339e-08
354 8.98234588930791e-08
355 9.14441287136469e-08
356 9.18641589464642e-08
357 9.36905024673251e-08
358 9.28346253203927e-08
359 9.28888823636953e-08
360 9.25948953067746e-08
361 9.23495235838345e-08
362 9.12594444457682e-08
363 9.20386042935206e-08
364 9.17183626825135e-08
365 9.13732876028917e-08
366 9.12726250135165e-08
367 9.11804605152611e-08
368 9.0826937082511e-08
369 9.09200608134597e-08
370 9.06134118849877e-08
371 9.03908841110024e-08
372 9.02687986581441e-08
373 9.00117527180555e-08
374 8.97215315376343e-08
375 8.96432723607177e-08
376 8.97199257110515e-08
377 8.90884876980635e-08
378 8.91837643735016e-08
379 8.93043576866148e-08
380 8.85018209828559e-08
381 8.87308928554376e-08
382 8.82181652173131e-08
383 8.83190978129278e-08
384 8.79617232385499e-08
385 8.75695747026839e-08
386 8.74287309216015e-08
387 8.71456649065294e-08
388 8.72467822432554e-08
389 8.70098801897257e-08
390 8.70440928224525e-08
391 8.71754934905766e-08
392 8.66344720407142e-08
393 8.65728537746691e-08
394 8.678824059416e-08
395 8.66506368879527e-08
396 8.60840003724661e-08
397 8.6210746985671e-08
398 8.59985220813542e-08
399 8.6295173673534e-08
400 8.58356941080274e-08
401 8.57985256175198e-08
402 8.5891201706545e-08
403 8.57672901588558e-08
404 8.5368569102684e-08
405 8.54384722970281e-08
406 8.53525818911294e-08
407 8.510642857118e-08
408 8.50278425446049e-08
409 8.51251016342758e-08
410 8.4956333523678e-08
411 8.490733449662e-08
412 8.47031103035079e-08
413 8.43584828658095e-08
414 8.45128127480166e-08
415 8.45306900032483e-08
416 8.4080951978649e-08
417 8.43866914124192e-08
418 8.39348572867493e-08
419 8.42327310124347e-08
420 8.36534255199695e-08
421 8.4162557811851e-08
422 8.36996107977939e-08
423 8.36068849707772e-08
424 8.34208080391363e-08
425 8.35114661867919e-08
426 8.33220639151477e-08
427 8.32289259733443e-08
428 8.33392448384984e-08
429 8.29204438446141e-08
430 8.29663946433357e-08
431 8.31548021551498e-08
432 8.28244992590044e-08
433 8.28996604695931e-08
434 8.26602430947787e-08
435 8.26115780228065e-08
436 8.25105459512088e-08
437 8.23151680151568e-08
438 8.22561929680887e-08
439 8.26073005555372e-08
440 8.23738304234212e-08
441 8.22460606286768e-08
442 8.22777721509738e-08
443 8.20875456497561e-08
444 8.20832255499226e-08
445 8.18027530158361e-08
446 8.19844530042246e-08
447 8.17955410070681e-08
448 8.19935763729518e-08
449 8.15913736573748e-08
450 8.16315264273726e-08
451 8.1537770313389e-08
452 8.13485740991382e-08
453 8.10621330060712e-08
454 8.11126952271479e-08
455 8.04393422981775e-08
456 8.01011879048019e-08
457 7.99959423147811e-08
458 7.98836197191122e-08
459 7.97341144220809e-08
460 7.95465027181308e-08
461 7.96396335545069e-08
462 7.95335353132032e-08
463 7.9400642505334e-08
464 7.94816372717833e-08
465 7.92081209510798e-08
466 7.9175777045748e-08
467 7.92295153928535e-08
468 7.92500074453528e-08
469 7.90302507880369e-08
470 7.91063996530283e-08
471 7.88745637692045e-08
472 7.91024277191354e-08
473 7.88437191090452e-08
474 7.88946081797803e-08
475 7.85952707360593e-08
476 7.83242981583498e-08
477 7.85057139296441e-08
478 7.83416354011024e-08
479 7.845414984331e-08
480 7.83177611651809e-08
481 7.81970115326658e-08
482 7.81096645141588e-08
483 7.79860442889913e-08
484 7.79399798034319e-08
485 7.79226922986709e-08
486 7.80308084813441e-08
487 7.78958408886865e-08
488 7.76646587041796e-08
489 7.77208839508603e-08
490 7.75507018602184e-08
491 7.77192212808586e-08
492 7.74591200070063e-08
493 7.7384186170093e-08
494 7.75996369384302e-08
495 7.70551267237352e-08
496 7.7188587965793e-08
497 7.73685329136242e-08
498 7.75969155597522e-08
499 7.69876180584106e-08
500 7.68885470847636e-08
501 7.75640600636507e-08
502 7.73106805240786e-08
503 7.74820207993798e-08
504 7.75725155222062e-08
505 7.68931585071186e-08
506 7.64812781994806e-08
507 7.6503837931341e-08
508 7.71570540791799e-08
509 7.67294565662269e-08
510 7.67017596103869e-08
511 7.68499930359212e-08
512 7.7622622995932e-08
513 7.7964500633243e-08
514 7.80154749691064e-08
515 7.84418290322719e-08
516 7.74486110799444e-08
517 7.73548549659608e-08
518 7.75997435198406e-08
519 7.73894086592009e-08
520 7.74426567318187e-08
521 7.71193384707658e-08
522 7.7726959091251e-08
523 7.6874428600604e-08
524 7.75839126276878e-08
525 7.70470407474022e-08
526 7.85978571116175e-08
527 7.69736345773708e-08
528 7.74011255089135e-08
529 7.67925314448803e-08
530 7.65272716307663e-08
531 7.73847546042816e-08
532 7.64042695777789e-08
533 7.63393401825851e-08
534 7.64848451240141e-08
535 7.69205001915907e-08
536 7.66882521929801e-08
537 7.58246230248005e-08
538 7.63581127216639e-08
539 7.5445399261298e-08
540 7.516899103166e-08
541 7.58490301677739e-08
542 7.58527534117093e-08
543 7.48063513356101e-08
544 7.55561657683756e-08
545 7.53945883502638e-08
546 7.57625286951225e-08
547 7.54723075147012e-08
548 7.52009512439145e-08
549 7.55284403908263e-08
550 7.51345154981209e-08
551 7.4644013636771e-08
552 7.37775991410672e-08
553 7.38115346621271e-08
554 7.45875397001328e-08
555 7.45797805734583e-08
556 7.39211358791181e-08
557 7.43720391938041e-08
558 7.43171213457572e-08
559 7.37893302016346e-08
560 7.41593808584184e-08
561 7.392715417609e-08
562 7.35372580606963e-08
563 7.37887688728733e-08
564 7.41726395858677e-08
565 7.40608427918232e-08
566 7.42232160177991e-08
567 7.36541068135921e-08
568 7.45677510849418e-08
569 7.49505986163967e-08
570 7.51068256477083e-08
571 7.52095559164445e-08
572 7.6000837623269e-08
573 7.55211715386395e-08
574 7.55119415885019e-08
575 7.53614202153585e-08
576 7.56074882701796e-08
577 7.56152189751447e-08
578 7.48917230453117e-08
579 7.52414450744254e-08
580 7.53841149503387e-08
581 7.46483337366044e-08
582 7.49524886600739e-08
583 7.46131547657569e-08
584 7.48161923525004e-08
585 7.41872980825065e-08
586 7.45768318211049e-08
587 7.41100620871293e-08
588 7.42567252132176e-08
589 7.45523109912938e-08
590 7.41565742146122e-08
591 7.36443652726848e-08
592 7.39831946816594e-08
593 7.36948706503426e-08
594 7.363886567191e-08
595 7.32186151708447e-08
596 7.2934781769618e-08
597 7.30091826994794e-08
598 7.17824306661896e-08
599 7.1518549304983e-08
600 7.28058395793596e-08
601 7.14883796604227e-08
602 7.23028321658603e-08
603 7.04391567296625e-08
604 7.0884944136651e-08
605 7.02809401786908e-08
606 7.07899516783073e-08
607 7.09736056592192e-08
608 7.1211779584246e-08
609 7.10408158965947e-08
610 7.06084435364573e-08
611 7.00483155924303e-08
612 6.9623773413241e-08
613 7.12449903517154e-08
614 7.07012048906108e-08
615 6.96599187222091e-08
616 7.02947033914825e-08
617 7.01616826859208e-08
618 6.96015547418938e-08
619 7.03716906969021e-08
620 6.93677009167004e-08
621 7.05791123323252e-08
622 6.96006523526194e-08
623 6.84134420225746e-08
624 6.87775170149507e-08
625 6.86860204268669e-08
626 6.84100953662892e-08
627 6.83917846799886e-08
628 6.80243630313271e-08
629 6.72757636266397e-08
630 6.86808334648958e-08
631 6.84847805132449e-08
632 6.87280916622512e-08
633 6.82852459021888e-08
634 6.95506017223124e-08
635 6.85467469452306e-08
636 6.84895695712839e-08
637 6.77893510214744e-08
638 6.78091041095286e-08
639 6.84093848235534e-08
640 6.97313353725804e-08
641 6.81189717965935e-08
642 6.68948061388619e-08
643 6.74046916060433e-08
644 6.73398403705505e-08
645 6.6976909351979e-08
646 6.70166784288995e-08
647 6.85603183114836e-08
648 6.69160513666611e-08
649 6.73604603207423e-08
650 6.56007657084956e-08
651 6.57005045923142e-08
652 6.62665371464755e-08
653 6.54162519708734e-08
654 6.48416573767463e-08
655 6.58804637510002e-08
656 6.50899849574671e-08
657 6.54692584589611e-08
658 6.378400740914e-08
659 6.38012451759096e-08
660 6.56304735002777e-08
661 6.48819522552913e-08
662 6.5257047765499e-08
663 6.44589306375565e-08
664 6.26205718390338e-08
665 6.33370618174922e-08
666 6.36622701222223e-08
667 6.41772643916738e-08
668 6.167744004415e-08
669 6.3690102081182e-08
670 6.32045953352645e-08
671 6.35717398722591e-08
672 6.29652490147237e-08
673 6.28162837301716e-08
674 6.26323668484474e-08
675 6.2186501281758e-08
676 6.15321198438323e-08
677 6.20347506696817e-08
678 6.01144947154353e-08
679 6.03185981162824e-08
680 6.17799429392107e-08
681 5.97077658426315e-08
682 6.10256307709278e-08
683 6.03926224584939e-08
684 5.8857064999529e-08
685 6.03591985282037e-08
686 6.03067107363131e-08
687 6.04365624212733e-08
688 6.00909473291722e-08
689 5.98408931296035e-08
690 6.02089684775819e-08
691 5.94972711098762e-08
692 5.84641810519315e-08
693 5.95081317555923e-08
694 5.74940308695204e-08
695 5.79678847145715e-08
696 5.71096236967605e-08
697 5.72755460837016e-08
698 5.71980756092216e-08
699 5.7388295005012e-08
700 5.57480106522235e-08
701 5.67019000641267e-08
702 5.60081048206484e-08
703 5.65777895644715e-08
704 5.54536185859433e-08
705 5.33666835167423e-08
706 5.50024168433083e-08
707 5.57188890581983e-08
708 5.45479039715246e-08
709 5.41099218764884e-08
710 5.42709592821211e-08
711 5.25448591304212e-08
712 5.37267652589435e-08
713 5.24760821463133e-08
714 5.39910374186547e-08
715 5.20090814859486e-08
716 5.18225995449484e-08
717 5.27500212399445e-08
718 5.15775333553847e-08
719 5.14554514552401e-08
720 5.21243848083941e-08
721 5.3108252018319e-08
722 5.11688291737755e-08
723 5.55073853547583e-08
724 5.16207201428642e-08
725 5.38152598039687e-08
726 5.27027879115849e-08
727 5.2964473695738e-08
728 5.21562668609477e-08
729 5.21909697681622e-08
730 5.00891204069376e-08
731 5.26829602165435e-08
732 5.11523730040153e-08
733 5.12476674430218e-08
734 5.31984447604827e-08
735 5.34539843499715e-08
736 5.31669606118612e-08
737 5.11842905837057e-08
738 5.11411357706493e-08
739 5.22068397401654e-08
740 4.83251270111396e-08
741 4.67640042245421e-08
742 4.93823826275275e-08
743 4.73367229858468e-08
744 4.97890155770619e-08
745 4.9567301374509e-08
746 4.74815919915272e-08
747 4.81816186947981e-08
748 4.66782630326179e-08
749 4.77224517680952e-08
750 4.77896584527571e-08
751 4.7446786055616e-08
752 4.69659937607503e-08
753 4.70120902207327e-08
754 4.80689337223339e-08
755 4.78384514224217e-08
756 4.79026240896019e-08
757 4.75016044276799e-08
758 4.67338061582723e-08
759 4.69184442408732e-08
760 4.59523761264791e-08
761 4.67185508057355e-08
762 4.67192400321892e-08
763 4.60217002284935e-08
764 4.55086528461379e-08
765 4.52131772021858e-08
766 4.58771651778989e-08
767 4.65796006210439e-08
768 4.56589646091743e-08
769 4.54121718007627e-08
770 4.54297826024685e-08
771 4.5842465823398e-08
772 4.56100934798087e-08
773 4.53261179700348e-08
774 4.83178013155339e-08
775 4.91096123766965e-08
776 4.82180375627195e-08
777 4.82486086639256e-08
778 4.86725291182211e-08
779 4.68682443965918e-08
780 4.73087169439168e-08
781 4.74569787911605e-08
782 4.79232049599432e-08
783 4.86701488000563e-08
784 4.72127794637345e-08
785 4.64578349124167e-08
786 4.733288250236e-08
787 4.52708057707696e-08
788 4.53963977520289e-08
789 4.49679227187971e-08
790 4.46094716721745e-08
791 4.45689138928174e-08
792 4.42702585701227e-08
793 4.3774743829772e-08
794 4.31617834806275e-08
795 4.39955165632e-08
796 4.24547259569863e-08
797 4.24052188918722e-08
798 4.19857535405299e-08
799 4.16447747397797e-08
800 4.15800975872571e-08
801 4.1151604790457e-08
802 4.05756779287003e-08
803 4.07228206711352e-08
804 4.08316083166937e-08
805 4.0079605412302e-08
806 4.01016215789696e-08
807 3.9774295856887e-08
808 4.00646342768596e-08
809 3.97221242565138e-08
810 3.97243908878409e-08
811 3.96394419510671e-08
812 3.93683059485284e-08
813 3.90353420698375e-08
814 3.89008256718171e-08
815 3.90986407694527e-08
816 4.07050997353053e-08
817 4.03217157440849e-08
818 3.80610565287043e-08
819 3.79608806611031e-08
820 3.81308389307833e-08
821 3.79206639422591e-08
822 3.78539759537944e-08
823 3.78105156073616e-08
824 3.77108015925387e-08
825 3.78513753673815e-08
826 3.77079771851641e-08
827 3.76910911370487e-08
828 3.79073625822457e-08
829 3.79530895600055e-08
830 3.7789938289734e-08
831 3.78770153020014e-08
832 3.71864175008341e-08
833 3.72514712410066e-08
834 3.73852557800092e-08
835 3.7701717303662e-08
836 3.68976564857348e-08
837 3.67698689274221e-08
838 3.70594435139537e-08
839 3.6861983687686e-08
840 3.69220067852893e-08
841 3.68814738749279e-08
842 3.68159227548404e-08
843 3.66093715342686e-08
844 3.66639802962254e-08
845 3.66534820273046e-08
846 3.60280338895791e-08
847 3.64628043314497e-08
848 3.64853676160237e-08
849 3.64481707038067e-08
850 3.60630210138879e-08
851 3.59624756640642e-08
852 3.60987186809325e-08
853 3.59745442324311e-08
854 3.59038025976588e-08
855 3.62015164512286e-08
856 3.57075577994692e-08
857 3.57522544902622e-08
858 3.67987311733486e-08
859 3.55948017727314e-08
860 3.57143754570188e-08
861 3.57493270541909e-08
862 3.59598288923735e-08
863 3.58623459817409e-08
864 3.55755140901692e-08
865 3.55030707055448e-08
866 3.53984574985589e-08
867 3.556979777386e-08
868 3.52855522578466e-08
869 3.53713929257538e-08
870 3.5430311129403e-08
871 3.47701423208946e-08
872 3.5092039496476e-08
873 3.48913218317648e-08
874 3.54704035032682e-08
875 3.55789708805787e-08
876 3.54809372993259e-08
877 3.51327038572435e-08
878 3.46339419365904e-08
879 3.54091014287405e-08
880 3.59953133965973e-08
881 3.58730822824782e-08
882 3.542191606698e-08
883 3.58295224600624e-08
884 3.56859892747252e-08
885 3.48793349758125e-08
886 3.54321265660928e-08
887 3.48212836343009e-08
888 3.52721585272775e-08
889 3.38913608288749e-08
890 3.52598021891026e-08
891 3.45600668083534e-08
892 3.4662694048393e-08
893 3.48369013636329e-08
894 3.39710517494041e-08
895 3.33844312194742e-08
896 3.37448575749022e-08
897 3.32728831153872e-08
898 3.34471721430418e-08
899 3.31281704291086e-08
900 3.34090231035589e-08
901 3.33330412161104e-08
902 3.56920040189834e-08
903 3.62679237753127e-08
904 3.6088902533038e-08
905 3.60289647005629e-08
906 3.51955655730762e-08
907 3.31549152576827e-08
908 3.41702097728103e-08
909 3.41205321774396e-08
910 3.45315740446495e-08
911 3.36580896487249e-08
912 3.60479255334667e-08
913 3.36707941528402e-08
914 3.36831220693057e-08
915 3.53597471303146e-08
916 3.33356204862412e-08
917 3.37723804477719e-08
918 3.40380879038094e-08
919 3.31622551641431e-08
920 3.37385763771181e-08
921 3.39504389046397e-08
922 3.20840669587596e-08
923 3.23297513205034e-08
924 3.20916377916092e-08
925 3.30656817482122e-08
926 3.37836958408388e-08
927 3.34928351719554e-08
928 3.23870210650057e-08
929 3.30886997801372e-08
930 3.35387397853992e-08
931 3.27221627571816e-08
932 3.21014965720678e-08
933 3.21640563072378e-08
934 3.22754765136324e-08
935 3.21533484282099e-08
936 3.07062855142703e-08
937 3.1216622176089e-08
938 3.1914943576794e-08
939 3.06387129000996e-08
940 3.17146024997328e-08
941 3.05691401081276e-08
942 3.01217362164152e-08
943 3.04362508529721e-08
944 3.01106801714468e-08
945 2.97374604940615e-08
946 2.98065430115457e-08
947 2.93495467929006e-08
948 3.00302716027545e-08
949 3.00000557729163e-08
950 2.96453102066607e-08
951 2.98326163772344e-08
952 2.9885580232758e-08
953 3.00667295505264e-08
954 2.94356556906905e-08
955 2.98251521257953e-08
956 2.92049211481071e-08
957 2.90130017788215e-08
958 2.88170625140083e-08
959 2.8809013841169e-08
960 2.83357977082233e-08
961 2.83120549227078e-08
962 2.83387553423609e-08
963 2.83619776553223e-08
964 2.84074381795563e-08
965 2.76062905868457e-08
966 2.77339058385451e-08
967 2.78265943620681e-08
968 2.79553624693563e-08
969 2.70956377335096e-08
970 2.79014482629236e-08
971 2.78990484048336e-08
972 2.8271072594066e-08
973 2.74198281857707e-08
974 2.66569433193808e-08
975 2.72173927839958e-08
976 2.70988884665258e-08
977 2.72460738415248e-08
978 2.73114881821357e-08
979 2.71245195193615e-08
980 2.70023825521548e-08
981 2.65423025780365e-08
982 2.70408690994373e-08
983 2.94884898721648e-08
984 2.58462584667996e-08
985 2.84474790390732e-08
986 2.82739947010668e-08
987 2.88814003823745e-08
988 2.82682162122683e-08
989 2.9451943106551e-08
990 2.85924066645293e-08
991 2.87588299840991e-08
992 2.85528596322138e-08
993 2.81110992261802e-08
994 2.79459442253938e-08
995 2.84139094475222e-08
996 2.82552772290501e-08
997 2.82283618702195e-08
998 2.81606364893605e-08
999 2.78682570353794e-08
1000 2.78437912726304e-08
1001 2.80990608558795e-08
1002 2.80554228737628e-08
1003 2.78655800656225e-08
1004 2.79276299863795e-08
1005 2.7951626790923e-08
1006 2.76863545423112e-08
1007 2.73763482994127e-08
1008 2.7522546020009e-08
1009 2.7564555082904e-08
1010 2.73942504236402e-08
1011 2.7375515188055e-08
1012 2.74623861429291e-08
1013 2.72446634141943e-08
1014 2.73281930418534e-08
1015 2.71343836288906e-08
1016 2.72477347351696e-08
1017 2.71458571177163e-08
1018 2.71995688194693e-08
1019 2.71765507875443e-08
1020 2.69330637792109e-08
1021 2.70168882821054e-08
1022 2.69372826267045e-08
1023 2.64871946598078e-08
1024 2.66880544330661e-08
1025 2.62800359251969e-08
1026 2.70235887001036e-08
1027 2.63419703827594e-08
1028 2.62714880960857e-08
1029 2.62249812976734e-08
1030 2.62680508456015e-08
1031 2.65473136806804e-08
1032 2.61116994693111e-08
1033 2.63549395640439e-08
1034 2.65657629228144e-08
1035 2.59619472586792e-08
1036 2.55130636617196e-08
1037 2.63027484237455e-08
1038 2.59260541923823e-08
1039 2.62276547147167e-08
1040 2.60638586269124e-08
1041 2.59989025863661e-08
1042 2.56622634253745e-08
1043 2.55719996289372e-08
1044 2.58255905549731e-08
1045 2.59504222555051e-08
1046 2.53391814197812e-08
1047 2.55033487661649e-08
1048 2.54260736909373e-08
1049 2.56142520527192e-08
1050 2.56805972043139e-08
1051 2.50774387922093e-08
1052 2.52971599223883e-08
1053 2.54604852756302e-08
1054 2.49834037902019e-08
1055 2.58715804335452e-08
1056 2.5405691772562e-08
1057 2.53682568285285e-08
1058 2.50629899056776e-08
1059 2.51168614795461e-08
1060 2.47882887549622e-08
1061 2.48020981530317e-08
1062 2.52804142064633e-08
1063 2.48858462725821e-08
1064 2.51479352897377e-08
1065 2.59557371151686e-08
1066 2.5333156017382e-08
1067 2.52705909531414e-08
1068 2.47756215543404e-08
1069 2.47809204267924e-08
1070 2.49812472929989e-08
1071 2.45734153025978e-08
1072 2.47828619848178e-08
1073 2.47967140154515e-08
1074 2.52994993843458e-08
1075 2.47529943209202e-08
1076 2.48248106515803e-08
1077 2.47056792801459e-08
1078 2.45966873535508e-08
1079 2.45970550594166e-08
1080 2.51364014047795e-08
1081 2.50512073307618e-08
1082 2.49434854993069e-08
1083 2.48875533515047e-08
1084 2.44134117366457e-08
1085 2.44388278503038e-08
1086 2.48483384979181e-08
1087 2.47562716992888e-08
1088 2.45774849361169e-08
1089 2.45996876202526e-08
1090 2.43430591240212e-08
1091 2.45845974689018e-08
1092 2.46658551361634e-08
1093 2.44873312738036e-08
1094 2.48163090077469e-08
1095 2.46135840598072e-08
1096 2.47823610521891e-08
1097 2.48868392560553e-08
1098 2.45086884120838e-08
1099 2.43316922166059e-08
1100 2.45557849609668e-08
1101 2.46525022618016e-08
1102 2.40275479512775e-08
1103 2.42633007019322e-08
1104 2.42830804353389e-08
1105 2.44073010691181e-08
1106 2.41142608103928e-08
1107 2.43819808787293e-08
1108 2.44760034462388e-08
1109 2.45636417872674e-08
1110 2.44878233246482e-08
1111 2.56009418109215e-08
1112 2.43049171899656e-08
1113 2.39677522273496e-08
1114 2.44272015947899e-08
1115 2.42328646038459e-08
1116 2.41889832608422e-08
1117 2.46406433035418e-08
1118 2.48042937300852e-08
1119 2.45046472002741e-08
1120 2.34547634647697e-08
1121 2.48077434150673e-08
1122 2.46185543062438e-08
1123 2.46614266785627e-08
1124 2.5633001499159e-08
1125 2.46269884485173e-08
1126 2.54522589671069e-08
1127 2.43058497773063e-08
1128 2.56079015770183e-08
1129 2.44763178613994e-08
1130 2.51779184168299e-08
1131 2.47918716667073e-08
1132 2.5122027125235e-08
1133 2.48043061645831e-08
1134 2.47177691647948e-08
1135 2.48990872364629e-08
1136 2.52211584950146e-08
1137 2.47006628484314e-08
1138 2.49390463835653e-08
1139 2.59627146448338e-08
1140 2.55105643276465e-08
1141 2.86302341834244e-08
1142 2.68724669183484e-08
1143 2.80885181780377e-08
1144 2.76209721761234e-08
1145 2.70261253376702e-08
1146 3.04083300761704e-08
1147 2.68605511166697e-08
1148 2.80488610115981e-08
1149 2.70841908900366e-08
1150 2.74804889954794e-08
1151 2.79007377201879e-08
1152 2.99984286300514e-08
1153 2.67843258683342e-08
1154 2.63943054079618e-08
1155 2.67378048590672e-08
1156 2.79684133630553e-08
1157 2.8291314180251e-08
1158 2.81904597443372e-08
1159 2.82808798601764e-08
1160 2.65460258219719e-08
1161 2.73081468549208e-08
1162 2.65061785853504e-08
1163 2.54747494210505e-08
1164 2.67880491122696e-08
1165 2.70831090887214e-08
1166 2.83329892880602e-08
1167 2.60581956013084e-08
1168 2.68406630254958e-08
1169 2.82908043658381e-08
1170 2.80269389918431e-08
1171 2.67332751491267e-08
1172 2.83939005640832e-08
1173 2.69765187965731e-08
1174 2.78488485605521e-08
1175 2.87969115220221e-08
1176 2.76225158302168e-08
1177 2.7062705854064e-08
1178 2.81966698878477e-08
1179 2.72570446213649e-08
1180 2.68381601387091e-08
1181 2.64231019286854e-08
1182 2.69906248462348e-08
1183 2.52342911011283e-08
1184 2.55687844230579e-08
1185 2.61576200699665e-08
1186 2.62244395088373e-08
1187 2.47305322886859e-08
1188 2.48878855302337e-08
1189 2.63972275149627e-08
1190 2.58743888537083e-08
1191 2.64216399870065e-08
1192 2.55910421742556e-08
1193 2.61726782468941e-08
1194 2.55314240860116e-08
1195 2.59815102765515e-08
1196 2.64408566152952e-08
1197 2.58879708781024e-08
1198 2.65040682734252e-08
1199 2.56416132771164e-08
1200 2.52087470897777e-08
1201 2.69464361934979e-08
1202 2.60893848746946e-08
1203 2.59316212947169e-08
1204 2.50796237111217e-08
1205 2.56239438556349e-08
1206 2.52385294885471e-08
1207 2.56678660548459e-08
1208 2.60357939652067e-08
1209 2.60599772872183e-08
1210 2.73405582618125e-08
1211 2.62458748068184e-08
1212 2.6193241353667e-08
1213 2.55660719261641e-08
1214 2.56104701890081e-08
1215 2.6049384871385e-08
1216 2.69844893097115e-08
1217 2.62334012290921e-08
1218 2.62996309174923e-08
1219 2.53616843082227e-08
1220 2.64017661066873e-08
1221 2.63215298446084e-08
1222 2.71449973610061e-08
1223 2.67277950882772e-08
1224 2.60159378484559e-08
1225 2.61910315657587e-08
1226 2.62345594137514e-08
1227 2.66691486672244e-08
1228 2.55348364675001e-08
1229 2.57974139827866e-08
1230 2.64997694898739e-08
1231 2.70452318318348e-08
1232 2.64620414469618e-08
1233 2.68194106922692e-08
1234 2.60373376193002e-08
1235 2.62700812214689e-08
1236 2.68147779536321e-08
1237 2.64040380670849e-08
1238 2.50900829001921e-08
1239 2.68112767543016e-08
1240 2.56954812982713e-08
1241 2.48937386260195e-08
1242 2.50906868615175e-08
1243 2.63203681072355e-08
1244 2.65794017906273e-08
1245 2.52103902198542e-08
1246 2.50967335801988e-08
1247 2.43974671576552e-08
1248 2.48745717357224e-08
1249 2.5886613741477e-08
1250 2.53301557506802e-08
1251 2.55439172036631e-08
1252 2.57122998448267e-08
1253 2.54232705998447e-08
1254 2.39769182286409e-08
1255 2.59063650531743e-08
1256 2.5534040659636e-08
1257 2.57286050242556e-08
1258 2.54223326834335e-08
1259 2.58254253537871e-08
1260 2.59679531211532e-08
1261 2.54207481731328e-08
1262 2.63791850585449e-08
1263 2.57567887018695e-08
1264 2.6250203788436e-08
1265 2.62332608969018e-08
1266 2.48083384946085e-08
1267 2.53601690758387e-08
1268 2.52438958625589e-08
1269 2.52717100579503e-08
1270 2.44595845799722e-08
1271 2.57703263173426e-08
1272 2.57502481559868e-08
1273 2.60257237982842e-08
1274 2.63047503779035e-08
1275 2.56712304746998e-08
1276 2.72069708984191e-08
1277 2.60858943335052e-08
1278 2.59189718576636e-08
1279 2.65425015300025e-08
1280 2.60297490228822e-08
1281 2.67365880546322e-08
1282 2.57364902722657e-08
1283 2.91312858280435e-08
1284 2.68878554976482e-08
1285 2.63804285083324e-08
1286 2.55396450654644e-08
1287 2.62487791502508e-08
1288 2.74283991075208e-08
1289 2.70418283321305e-08
1290 2.7158293391949e-08
1291 2.78848002466248e-08
1292 2.94666939737454e-08
1293 2.75998619514439e-08
1294 2.80199916602442e-08
1295 2.84683761009319e-08
1296 2.97599171972251e-08
1297 3.03821394709303e-08
1298 2.67078021920497e-08
1299 2.71603930457331e-08
1300 2.76235905261046e-08
1301 2.86539947325082e-08
1302 3.18909911811716e-08
1303 2.86479870936773e-08
1304 2.70062106011437e-08
1305 2.77384213376308e-08
1306 2.87143748778362e-08
1307 2.84814802853361e-08
1308 2.9694970038463e-08
1309 2.49132341423319e-08
1310 3.30942668824719e-08
1311 2.94829689551079e-08
1312 2.91791923956453e-08
1313 2.94590840610454e-08
1314 3.07688203804446e-08
1315 3.2636393143548e-08
1316 3.24845146337793e-08
1317 3.13252641603867e-08
1318 2.7056069384912e-08
1319 2.86687971140509e-08
1320 2.8901094850653e-08
1321 2.69445017408998e-08
1322 2.96514723885366e-08
1323 2.93325470579475e-08
1324 2.82393681771964e-08
1325 3.1155234836433e-08
1326 2.86472126020954e-08
1327 3.00115274853852e-08
1328 2.72008477963936e-08
1329 2.94279001167297e-08
1330 2.88461432518261e-08
1331 2.31464927225034e-08
1332 2.90558368476468e-08
1333 2.95688167284425e-08
1334 2.89087278559919e-08
1335 2.92276869373609e-08
1336 3.12832817428443e-08
1337 2.88931314429419e-08
1338 3.01035285588114e-08
1339 2.85755366036256e-08
1340 3.09896073247273e-08
1341 2.8717197508854e-08
1342 2.96287101519965e-08
1343 3.08379135560699e-08
1344 3.09117247354607e-08
1345 2.95655464555011e-08
1346 3.04302645304233e-08
1347 2.93935435990988e-08
1348 3.10435339656578e-08
1349 2.94737922956756e-08
1350 2.93021322761433e-08
1351 3.06528633586822e-08
1352 3.00800415686808e-08
1353 2.9768790099638e-08
1354 3.18649462371923e-08
1355 2.5748354559596e-08
1356 3.15537604933525e-08
1357 3.15963788466433e-08
1358 3.18029833579203e-08
1359 3.07979100000466e-08
1360 3.22068274272169e-08
1361 3.09049177360521e-08
1362 3.09111705121268e-08
1363 3.241562041012e-08
1364 3.0583890975322e-08
1365 3.23189865980567e-08
1366 3.13874153334837e-08
1367 3.17199706501015e-08
1368 3.16833528302141e-08
1369 3.16153325741197e-08
1370 3.2764432944532e-08
1371 3.13499661785954e-08
1372 3.04275609153137e-08
1373 3.2178633091462e-08
1374 2.61568189330319e-08
1375 3.04152223407073e-08
1376 2.9831614511977e-08
1377 3.00903408856357e-08
1378 2.95224253932247e-08
1379 3.19789101865808e-08
1380 2.9939656087663e-08
1381 3.04920142468745e-08
1382 3.14917301125206e-08
1383 2.96782420861064e-08
1384 3.33537819585672e-08
1385 3.21938316005799e-08
1386 2.93218551661312e-08
1387 3.18835162715914e-08
1388 3.09933945175089e-08
1389 2.93138260332171e-08
1390 3.2214646950024e-08
1391 3.14618446850545e-08
1392 3.04143377150012e-08
1393 3.0550719287703e-08
1394 3.1553931023609e-08
1395 3.119106040117e-08
1396 3.08517904556993e-08
1397 3.11800008034879e-08
1398 3.056272035451e-08
1399 3.08653014258198e-08
1400 3.10613970100349e-08
1401 3.1029241398528e-08
1402 3.1200119821051e-08
1403 3.05569116676452e-08
1404 3.1590509763646e-08
1405 2.98712272694956e-08
1406 3.11978887168607e-08
1407 3.12107673039463e-08
1408 3.06905967306648e-08
1409 3.26011218021449e-08
1410 3.10090300104093e-08
1411 2.98084117389408e-08
1412 3.16509911613139e-08
1413 3.16384607401687e-08
1414 3.14976915660736e-08
1415 3.06630596469404e-08
1416 3.25362279340879e-08
1417 3.13383594630068e-08
1418 2.98592439662571e-08
1419 3.13517460881485e-08
1420 3.00921030316204e-08
1421 3.09622230076911e-08
1422 3.14716928073722e-08
1423 3.07265821675173e-08
1424 3.04703533515749e-08
1425 2.97928188786045e-08
1426 3.10714298734638e-08
1427 3.09181515945056e-08
1428 3.20132471642864e-08
1429 3.09099874584717e-08
1430 3.12862518114798e-08
1431 3.09557570687957e-08
1432 2.50355220998699e-08
1433 3.08924832381763e-08
1434 3.08933358894592e-08
1435 3.08879855026589e-08
1436 3.0501233538871e-08
1437 3.00845925949034e-08
1438 3.10019920846116e-08
1439 2.87035959445348e-08
1440 3.11767820448949e-08
1441 3.07515755082477e-08
1442 3.13252890293825e-08
1443 3.11691685794813e-08
1444 3.01542719682857e-08
1445 3.08390042391693e-08
1446 3.14283390423498e-08
1447 2.93052124789028e-08
1448 2.4682051957825e-08
1449 2.30143673007888e-08
1450 2.3280884775545e-08
1451 2.52713423520845e-08
1452 2.80236385208354e-08
1453 3.04488949609549e-08
1454 2.40098323445181e-08
1455 2.61412527180482e-08
1456 2.48444287365146e-08
1457 2.89237007677912e-08
1458 2.50621177144694e-08
1459 2.7820645343013e-08
1460 2.44977922392309e-08
1461 3.53615341452951e-08
1462 3.10669534542285e-08
1463 3.00932612162796e-08
1464 2.32686350187805e-08
1465 2.9717964977749e-08
1466 2.88342043575085e-08
1467 2.45603786197535e-08
1468 3.18712913838226e-08
1469 2.87602546222843e-08
1470 2.86339627564303e-08
1471 2.45937794574047e-08
1472 2.36530404151836e-08
1473 2.55630805412466e-08
1474 2.25493543837274e-08
1475 2.8519464123633e-08
1476 2.91007715702563e-08
1477 2.91021748921594e-08
1478 3.12793169143788e-08
1479 2.74467186756056e-08
1480 3.01772153932234e-08
1481 3.48442945607985e-08
1482 2.86442141117504e-08
1483 2.7664190938026e-08
1484 2.45084290639852e-08
1485 2.77720353381028e-08
1486 3.18393240661408e-08
1487 2.62222599189954e-08
1488 3.01023028725922e-08
1489 2.44247093661443e-08
1490 2.63965329594384e-08
1491 2.86751973277433e-08
1492 2.58024677179947e-08
1493 2.33825439011071e-08
1494 2.77637433043765e-08
1495 2.76345240024511e-08
1496 3.46579476229181e-08
1497 3.11587129431246e-08
1498 3.2156588503085e-08
1499 3.29333822435274e-08
1500 2.49710190303176e-08
1501 2.53956766727015e-08
1502 3.0313444199237e-08
1503 3.08012673144731e-08
1504 2.98970874723636e-08
1505 2.88208337195783e-08
1506 2.95539663852651e-08
1507 2.53647307602023e-08
1508 2.99320497276767e-08
1509 2.98471505288944e-08
1510 2.83363341679888e-08
1511 3.39094228252179e-08
1512 2.72303690707076e-08
1513 3.09464454062436e-08
1514 2.51548630814113e-08
1515 2.89717014823054e-08
1516 2.88409758297803e-08
1517 2.88131474235342e-08
1518 2.99303621886793e-08
1519 2.64623860601887e-08
1520 2.39775790333852e-08
1521 2.49841853872113e-08
1522 2.52440859327407e-08
1523 2.49282496866954e-08
1524 2.57209311627093e-08
1525 2.37156978499797e-08
1526 2.65185420289527e-08
1527 2.41956872315541e-08
1528 2.34490489248174e-08
1529 3.10576204753943e-08
1530 2.39976323257451e-08
1531 2.57374583867431e-08
1532 2.6586340240442e-08
1533 2.4139410470525e-08
1534 3.07194589765913e-08
1535 2.98589775127311e-08
1536 3.18228217111027e-08
1537 2.66675801441352e-08
1538 2.68566129335568e-08
1539 2.75371991875772e-08
1540 2.82719732069836e-08
1541 2.46644162871235e-08
1542 2.46761562294751e-08
1543 2.21025757696225e-08
1544 2.96798479126892e-08
1545 3.11193950608413e-08
1546 2.87361174855505e-08
1547 2.54033789559571e-08
1548 2.2383170872331e-08
1549 3.03869960305292e-08
1550 2.35312320739922e-08
1551 3.01662090862465e-08
1552 2.66202153653694e-08
1553 2.92166273396788e-08
1554 2.82174141830183e-08
1555 2.52276421974784e-08
1556 2.43387354714741e-08
1557 2.17863149742925e-08
1558 2.27181153888978e-08
1559 2.54624197282283e-08
1560 3.20702362444081e-08
1561 3.24113003102866e-08
1562 3.17732720134245e-08
1563 2.48417713066829e-08
1564 2.68014783699755e-08
1565 3.22352349257926e-08
1566 2.53552521201073e-08
1567 3.02209599567504e-08
1568 2.97392723780376e-08
1569 3.09780681106986e-08
1570 2.45251516872713e-08
1571 2.60381742833715e-08
1572 2.47101592520949e-08
1573 2.41788296051482e-08
1574 2.401519871853e-08
1575 3.09570893364253e-08
1576 2.63231640929007e-08
1577 2.72641091925152e-08
1578 2.56706567114406e-08
1579 2.67743054394032e-08
1580 2.4612591076334e-08
1581 2.56833114775645e-08
1582 2.60627253112489e-08
1583 2.45890987571329e-08
1584 2.55581191765941e-08
1585 2.54023095891398e-08
1586 2.35001174075933e-08
1587 2.53258267690626e-08
1588 2.49434712884522e-08
1589 2.0218987373255e-08
1590 2.72470028761518e-08
1591 2.49641356475649e-08
1592 2.51091147873694e-08
1593 2.0316576865298e-08
1594 2.20510436577115e-08
1595 2.97781070912606e-08
1596 2.43563995638851e-08
1597 2.6058465607548e-08
1598 2.22237659386337e-08
1599 2.1906009450845e-08
1600 2.54408032418496e-08
1601 2.32293277946383e-08
1602 2.25841851886344e-08
1603 2.2773162910994e-08
1604 2.30627179576004e-08
1605 2.45982612057105e-08
1606 2.14578612656169e-08
1607 2.82413488150723e-08
1608 2.63630628438705e-08
1609 3.03035534443552e-08
1610 2.63517794252266e-08
1611 2.50448213279242e-08
1612 2.67036117662656e-08
1613 2.52548826296106e-08
1614 2.12981881020369e-08
1615 2.5361957867176e-08
1616 2.38533086616144e-08
1617 2.16515001483231e-08
1618 2.61645567434243e-08
1619 2.37743513764599e-08
1620 2.42376998471627e-08
1621 2.56824019828628e-08
1622 2.50544331947822e-08
1623 2.2024115864383e-08
1624 2.58674894837441e-08
1625 2.5607263864913e-08
1626 2.22086811163535e-08
1627 2.65213451200452e-08
1628 2.9212444019322e-08
1629 2.47018689947254e-08
1630 2.85377677045062e-08
1631 2.34859314218738e-08
1632 2.1066584920959e-08
1633 2.83417360691374e-08
1634 2.27871055358264e-08
1635 2.12253947751151e-08
1636 2.44956712691646e-08
1637 2.65758028916707e-08
1638 2.28070469177055e-08
1639 2.24163176909542e-08
1640 1.96893523707331e-08
1641 2.1128979454943e-08
1642 2.25276544085773e-08
1643 2.5475209497472e-08
1644 2.57279690885071e-08
1645 2.38128734508791e-08
1646 2.43966162827292e-08
1647 2.36343709048015e-08
1648 2.07351131820133e-08
1649 2.03581524971241e-08
1650 2.18002842444776e-08
1651 2.80294365495593e-08
1652 2.35878392373934e-08
1653 2.27862120283362e-08
1654 2.96767002083698e-08
1655 2.3709732843713e-08
1656 2.43882816164387e-08
1657 2.21331344363307e-08
1658 2.23436362745133e-08
1659 2.00481888867898e-08
1660 1.97718925676327e-08
1661 2.43117757037226e-08
1662 2.2908213992423e-08
1663 2.88849317797713e-08
1664 2.19557172442819e-08
1665 1.99130631983735e-08
1666 2.50617890884541e-08
1667 2.32246435416528e-08
1668 2.51826914876574e-08
1669 2.41385276211759e-08
1670 2.05538821518303e-08
1671 3.05845908599167e-08
1672 2.04277359472371e-08
1673 2.38228494708892e-08
1674 2.27092638027671e-08
1675 2.09014991980894e-08
1676 2.4390667263674e-08
1677 2.31541896766885e-08
1678 2.60333230528431e-08
1679 2.49491201032015e-08
1680 2.48845317685209e-08
1681 2.40011708285692e-08
1682 2.36668942221741e-08
1683 2.42655833204708e-08
1684 2.96595477067285e-08
1685 2.31462742306121e-08
1686 2.21477236550527e-08
1687 2.96433970703447e-08
1688 3.03723446393178e-08
1689 2.27510561501276e-08
1690 2.80913088346324e-08
1691 1.90166957736437e-08
1692 2.38183073264508e-08
1693 2.82170908860735e-08
1694 2.62656509875114e-08
1695 2.88278361182392e-08
1696 2.51208174262274e-08
1697 2.19243840859917e-08
1698 2.28525145473668e-08
1699 2.2738099403341e-08
1700 2.4535840026374e-08
1701 2.8962945819444e-08
1702 2.2241007258117e-08
1703 2.24713438967683e-08
1704 2.91959949549891e-08
1705 1.85741590996713e-08
1706 2.12042152725189e-08
1707 2.10919246512731e-08
1708 2.08411403690434e-08
1709 2.52654235310956e-08
1710 2.58834909061534e-08
1711 2.20746780854597e-08
1712 2.01649648090552e-08
1713 2.2748883665713e-08
1714 2.3956122419122e-08
1715 2.19557207969956e-08
1716 1.92938323095859e-08
1717 2.05104146999702e-08
1718 2.13353370526193e-08
1719 2.2862511883659e-08
1720 2.55401904070141e-08
1721 2.09874517764774e-08
1722 2.50140317348269e-08
1723 2.3160554363244e-08
1724 2.95069693123651e-08
1725 2.24346052846158e-08
1726 1.96804457175404e-08
1727 2.44329658727338e-08
1728 2.11803694583068e-08
1729 2.36710384626804e-08
1730 2.69671840413821e-08
1731 2.22715321740452e-08
1732 2.34170389745714e-08
1733 2.2418852552164e-08
1734 2.8852493727527e-08
1735 2.9372980492326e-08
1736 2.96190201254376e-08
1737 2.98225870665192e-08
1738 3.01393860979715e-08
1739 3.38531158661226e-08
1740 2.34409913701938e-08
1741 3.50109274904753e-08
1742 2.80186913670377e-08
1743 2.5152333549272e-08
1744 2.22051657061684e-08
1745 2.55727030662456e-08
1746 2.40515607430325e-08
1747 2.14860467195876e-08
1748 2.41640858433811e-08
1749 2.89231998351624e-08
1750 2.24476064403234e-08
1751 2.61577159932358e-08
1752 2.15565503225434e-08
1753 2.92838322479838e-08
1754 2.32101786679095e-08
1755 2.9019455283219e-08
1756 2.92173290006303e-08
1757 2.73177356291399e-08
1758 2.26968417393891e-08
1759 2.78506409046031e-08
1760 2.51091538672199e-08
1761 2.7202613495092e-08
1762 2.53260363791696e-08
1763 2.31927295146761e-08
1764 2.30949144253145e-08
1765 2.34371899665575e-08
1766 2.22278799810738e-08
1767 2.43806539401703e-08
1768 2.8260775053468e-08
1769 2.43240911856901e-08
1770 2.96798479126892e-08
1771 2.34436985380171e-08
1772 2.15330295816329e-08
1773 2.72767675113528e-08
1774 2.36857129465307e-08
1775 2.19114983934787e-08
1776 2.92986950256591e-08
1777 2.35626700373359e-08
1778 2.33155184048428e-08
1779 2.80789969053785e-08
1780 2.57984478224671e-08
1781 2.45690916500507e-08
1782 2.2438518598733e-08
1783 2.35895427636024e-08
1784 2.53135699068707e-08
1785 2.3191281783852e-08
1786 2.73588938171088e-08
1787 2.13781614633035e-08
1788 2.68274860104611e-08
1789 2.62818744545257e-08
1790 2.66722963715438e-08
1791 2.52763125985211e-08
1792 2.1044995079933e-08
1793 2.29146106534017e-08
1794 2.43286688572653e-08
1795 2.31376304782316e-08
1796 2.09044088705923e-08
1797 2.28734755580717e-08
1798 2.43422277890204e-08
1799 2.40295587872197e-08
1800 2.62196238054457e-08
1801 2.20027001063272e-08
1802 2.3859573872187e-08
1803 2.42675159967121e-08
1804 2.3180751540508e-08
1805 2.56769379092248e-08
1806 2.57055230434844e-08
1807 2.67123141384218e-08
1808 2.11208810441121e-08
1809 1.9786865479432e-08
1810 2.61077470753435e-08
1811 2.45351401417793e-08
1812 2.23424159173646e-08
1813 2.50684664138134e-08
1814 2.42301965158731e-08
1815 2.51484486568643e-08
1816 2.40829205466753e-08
1817 2.40751028002251e-08
1818 2.32971792968328e-08
1819 2.56820378297107e-08
1820 2.44820981265548e-08
1821 2.68134705549983e-08
1822 2.27507896966017e-08
1823 2.26886864851394e-08
1824 2.45530422660067e-08
1825 2.22726210807878e-08
1826 2.0230158881418e-08
1827 2.25511076479279e-08
1828 2.70949929159769e-08
1829 2.29213092950431e-08
1830 2.26416094761817e-08
1831 2.61673331891643e-08
1832 2.62032013864655e-08
1833 2.2749713224357e-08
1834 2.39823609859968e-08
1835 2.47737688141569e-08
1836 2.34558452660849e-08
1837 2.42766553526508e-08
1838 2.42859830024145e-08
1839 2.46508538026546e-08
1840 2.44655069536748e-08
1841 2.36899850847294e-08
1842 2.84804926309334e-08
1843 2.12147917011407e-08
1844 2.69097366611959e-08
1845 2.2986171188677e-08
1846 2.06811279213071e-08
1847 2.24596359288398e-08
1848 2.70512146016699e-08
1849 2.41038264903182e-08
1850 2.49218405912188e-08
1851 2.26370620026728e-08
1852 2.19445102089821e-08
1853 2.37682247217208e-08
1854 2.43518183395963e-08
1855 2.47216007664974e-08
1856 2.42071003242472e-08
1857 2.43908786501379e-08
1858 2.03792964725835e-08
1859 2.08565875681188e-08
1860 2.25726690672445e-08
1861 2.39036985760777e-08
1862 2.54190997139858e-08
1863 2.10415791457308e-08
1864 2.10349764273587e-08
1865 2.26829275362661e-08
1866 2.44929339032751e-08
1867 2.37007586889604e-08
1868 2.12831210433251e-08
1869 2.08832187098551e-08
1870 2.12773496599539e-08
1871 2.17356799225854e-08
1872 2.21342322248574e-08
1873 2.35889103805675e-08
1874 2.21672848965682e-08
1875 2.50683527269757e-08
1876 2.27304504107906e-08
1877 2.29643344340502e-08
1878 2.26829257599093e-08
1879 2.20682618845558e-08
1880 3.10560785976577e-08
1881 2.20814122542379e-08
1882 2.25371667994523e-08
1883 2.27917578143888e-08
1884 2.2865497939506e-08
1885 2.18663362971938e-08
1886 2.42805988648342e-08
1887 2.32571650826685e-08
1888 2.22612612787998e-08
1889 2.24661693692951e-08
1890 2.18792894912667e-08
1891 2.31134347217221e-08
1892 2.28887664377453e-08
1893 2.27273275754669e-08
1894 2.08787795941134e-08
1895 1.93891569466587e-08
1896 3.03158955716754e-08
1897 2.25484999560877e-08
1898 2.12494235540817e-08
1899 2.25179590529478e-08
1900 2.06512282829863e-08
1901 2.30354757491114e-08
1902 2.32392824983663e-08
1903 2.19413323065965e-08
1904 2.29981775845545e-08
1905 2.00656380400233e-08
1906 2.31395542726887e-08
1907 2.27192309409929e-08
1908 2.26726744045891e-08
1909 2.32996963944743e-08
1910 2.27228920124389e-08
1911 2.45133282561483e-08
1912 2.00768752733893e-08
1913 2.38379307404557e-08
1914 2.05456931468007e-08
1915 2.1709592346042e-08
1916 2.19324345351879e-08
1917 2.00054550703044e-08
1918 2.14951594301738e-08
1919 2.39600961293718e-08
1920 2.29326069245417e-08
1921 2.90421677817676e-08
1922 2.3136520255207e-08
1923 2.08600052786778e-08
1924 2.7610235875386e-08
1925 2.06921484391387e-08
1926 2.70129962842702e-08
1927 2.92708026705668e-08
1928 2.08047481464746e-08
1929 2.26363248145844e-08
1930 2.07608934488235e-08
1931 2.07666559504105e-08
1932 2.15670894476716e-08
1933 2.28932801604742e-08
1934 2.1318660614611e-08
1935 2.09265014206039e-08
1936 1.98813658869312e-08
1937 2.37890773746585e-08
1938 2.27846115308239e-08
1939 2.30364918252235e-08
1940 2.23891021278178e-08
1941 2.29208918511858e-08
1942 2.11173283304333e-08
1943 2.02059933229748e-08
1944 2.0803117450896e-08
1945 2.08154542491457e-08
1946 2.1344044753846e-08
1947 2.01695407042735e-08
1948 2.1522266635543e-08
1949 2.14352020577735e-08
1950 2.16994315849206e-08
1951 2.2829384604961e-08
1952 2.17843574290555e-08
1953 2.93372668380698e-08
1954 2.58859191859528e-08
1955 2.25605472081725e-08
1956 2.09190300637374e-08
1957 1.98597351896979e-08
1958 2.47912073092493e-08
1959 2.2683979139515e-08
1960 3.13892769554514e-08
1961 2.40826416586515e-08
1962 2.31433183728313e-08
1963 2.39950832536806e-08
1964 2.05278496423489e-08
1965 1.98145198027078e-08
1966 2.28335750307451e-08
1967 2.23204974503233e-08
1968 2.21781650822095e-08
1969 2.45625457750975e-08
1970 2.03294270306742e-08
1971 2.06917079026425e-08
1972 1.96583176403919e-08
1973 1.98471283852086e-08
1974 2.03806571619225e-08
1975 2.13839062013221e-08
1976 2.63403094891146e-08
1977 2.30171970372339e-08
1978 2.11493276225383e-08
1979 2.06445829320501e-08
1980 2.04597547792673e-08
1981 2.22387210868646e-08
1982 2.0147092882894e-08
1983 1.95857374762909e-08
1984 2.12324060555602e-08
1985 2.09182626775828e-08
1986 2.21774012487685e-08
1987 2.67921524965686e-08
1988 1.95025346982902e-08
1989 2.03900043516114e-08
1990 2.28118164358193e-08
1991 1.96319938083889e-08
1992 1.99822771662639e-08
1993 1.77443979509917e-08
1994 1.97325960016315e-08
1995 2.08963868431056e-08
1996 1.90666575861087e-08
1997 2.01348129280632e-08
1998 2.39367068388674e-08
1999 1.79162924496268e-08
2000 2.16502407113239e-08
2001 1.87960313979829e-08
2002 1.89675102291176e-08
2003 2.47350158133486e-08
2004 2.01473522309925e-08
2005 2.18595275214284e-08
2006 2.33230057489209e-08
2007 2.17013713665892e-08
2008 1.84706916428468e-08
2009 1.85111801442872e-08
2010 1.94859453017671e-08
2011 2.66198458831468e-08
2012 2.21610232387093e-08
2013 2.02882510791369e-08
2014 2.10995416694004e-08
2015 1.97594403061885e-08
2016 2.40755060332276e-08
2017 2.54149075118448e-08
2018 1.94328908520447e-08
2019 1.94836822231537e-08
2020 2.54437289015641e-08
2021 1.91478228828146e-08
2022 1.6931483770577e-08
2023 2.6014564724619e-08
2024 1.79905104147338e-08
2025 2.05377190809486e-08
2026 1.90533437915974e-08
2027 1.68639768816092e-08
2028 1.70975642532767e-08
2029 1.8513063082537e-08
2030 2.71644520211112e-08
2031 2.46398830228145e-08
2032 1.94116136498224e-08
2033 1.7928909912257e-08
2034 2.00927789961725e-08
2035 2.1723804977114e-08
2036 1.96473806113318e-08
2037 1.84835773353598e-08
2038 1.99591045912939e-08
2039 2.03067305193372e-08
2040 2.1841458419658e-08
2041 1.98760545799814e-08
2042 1.96627976123409e-08
2043 2.37102621980512e-08
2044 1.94338145576012e-08
2045 2.01285477174906e-08
2046 2.1543879569208e-08
2047 2.12635420382412e-08
2048 2.48748541764598e-08
2049 2.02936867310655e-08
2050 2.00696526064803e-08
2051 1.96903879867705e-08
2052 1.7186765788324e-08
2053 1.99142622392401e-08
2054 2.13098125811939e-08
2055 2.06423678150713e-08
2056 2.14188524694237e-08
2057 2.10920259036129e-08
2058 1.70881904182352e-08
2059 1.90632416519065e-08
2060 1.95900593524811e-08
2061 2.06608774533379e-08
2062 2.18279954111722e-08
2063 2.42497844027412e-08
2064 2.07352677250583e-08
2065 2.18444000665841e-08
2066 1.93483575827713e-08
2067 2.48784726153417e-08
2068 2.0108169351829e-08
2069 1.7473118063549e-08
2070 1.69013247841576e-08
2071 2.16746549597246e-08
2072 2.08199644191609e-08
2073 1.98888638891503e-08
2074 2.1243534931159e-08
2075 1.99680876278308e-08
2076 2.03789518593567e-08
2077 2.01216625583811e-08
2078 1.74213106163279e-08
2079 2.07445598476852e-08
2080 2.13816964134139e-08
2081 1.7947536790075e-08
2082 2.22122675808123e-08
2083 1.94571043721226e-08
2084 1.98517309257795e-08
2085 2.10799502298187e-08
2086 2.11497042101882e-08
2087 1.983243613779e-08
2088 2.1748116196818e-08
2089 2.0742980666455e-08
2090 2.07286809938978e-08
2091 1.70989142844746e-08
2092 2.0422582736046e-08
2093 1.91440072683235e-08
2094 2.15732693931159e-08
2095 1.98718925759067e-08
2096 1.91492368628587e-08
2097 2.02871301979712e-08
2098 1.92450926306265e-08
2099 2.34322197201209e-08
2100 2.24358007727687e-08
2101 1.98105105653212e-08
2102 1.96239842154e-08
2103 1.7397717044787e-08
2104 2.26338254805114e-08
2105 2.1227752000641e-08
2106 2.07929176099242e-08
2107 2.10279580414863e-08
2108 1.95552107840058e-08
2109 2.13912425550689e-08
2110 2.01922514264652e-08
2111 1.96088478787715e-08
2112 2.27693881527102e-08
2113 1.99910683562621e-08
2114 1.90244957565255e-08
2115 2.15738786835118e-08
2116 1.90464497507037e-08
2117 1.81015256117689e-08
2118 2.33024728402143e-08
2119 1.95504501476762e-08
2120 1.75685439529616e-08
2121 2.26172787165524e-08
2122 2.16461266688839e-08
2123 1.75913168476427e-08
2124 2.00791880899942e-08
2125 1.61610547166902e-08
2126 2.06479935371817e-08
2127 2.0964792568634e-08
2128 2.36122374985825e-08
2129 2.50157938808115e-08
2130 2.25344027882102e-08
2131 2.60101487015163e-08
2132 2.44190232479014e-08
2133 2.33558292705993e-08
2134 2.0772059627916e-08
2135 1.95895459853546e-08
2136 2.13319566455539e-08
2137 2.29036416499184e-08
2138 2.265410437019e-08
2139 2.40261197603786e-08
2140 2.49117189099479e-08
2141 2.0195018990421e-08
2142 1.86664941281833e-08
2143 1.82604278364806e-08
2144 1.81430426238194e-08
2145 1.93720612884363e-08
2146 2.40482709301659e-08
2147 1.94609093284726e-08
2148 2.30535892598027e-08
2149 2.11881623357613e-08
2150 2.18004743146594e-08
2151 2.67147761690012e-08
2152 2.17877094144114e-08
2153 1.75440462157894e-08
2154 2.30203660578354e-08
2155 2.17672173619121e-08
2156 1.94752232118844e-08
2157 2.04670129733131e-08
2158 2.2109350794608e-08
2159 2.04896739575133e-08
2160 2.11664925586774e-08
2161 2.29712160404461e-08
2162 1.97800069656751e-08
2163 2.23569252000289e-08
2164 2.19666276279895e-08
2165 1.92403053489443e-08
2166 1.71139955540411e-08
2167 2.27569572075481e-08
2168 1.81571611079789e-08
2169 2.00641192549256e-08
2170 2.08934789469595e-08
2171 2.30967174275065e-08
2172 2.05198080749369e-08
2173 1.72043339574657e-08
2174 2.38265407404015e-08
2175 2.22799396709661e-08
2176 2.33613057787352e-08
2177 2.52317242654954e-08
2178 2.06298675919925e-08
2179 2.87841341872763e-08
2180 2.57794887659202e-08
2181 1.75756049713982e-08
2182 1.9933132477945e-08
2183 1.93062561493207e-08
2184 2.70690563297649e-08
2185 1.85463431279231e-08
2186 2.24832632511607e-08
2187 2.27155592114059e-08
2188 2.32696297786106e-08
2189 1.90550970557979e-08
2190 2.35477664034534e-08
2191 2.1187148036006e-08
2192 2.02992698206117e-08
2193 2.02913046365438e-08
2194 2.20683418206136e-08
2195 1.89623587942833e-08
2196 1.9073269186265e-08
2197 1.86029271986854e-08
2198 1.98574188203793e-08
2199 2.35376393931119e-08
2200 2.26792078450444e-08
2201 2.06457411167094e-08
2202 1.98100700288251e-08
2203 2.98363076467467e-08
2204 1.76779266780613e-08
2205 2.0309519399575e-08
2206 2.14645812235403e-08
2207 2.11215436252132e-08
2208 1.87331146150882e-08
2209 2.35209629551036e-08
2210 2.29121486228223e-08
2211 1.74976033662233e-08
2212 1.99001366496532e-08
2213 2.12913793262715e-08
2214 2.31892052227067e-08
2215 2.57888217447544e-08
2216 2.02772483248737e-08
2217 2.56570054091299e-08
2218 2.04866594799569e-08
2219 2.11533546234932e-08
2220 1.93521589864076e-08
2221 1.82857622377242e-08
2222 1.74652026174726e-08
2223 2.06908428168617e-08
2224 1.79492438689977e-08
2225 1.7832858745237e-08
2226 2.32402115329933e-08
2227 2.59439936201034e-08
2228 2.04579500007185e-08
2229 1.80108656877564e-08
2230 1.94774756323568e-08
2231 1.91314750708216e-08
2232 1.7920493533552e-08
2233 1.9249355887041e-08
2234 1.72148055810339e-08
2235 1.71917147184786e-08
2236 1.758614764924e-08
2237 1.65150382258616e-08
2238 2.02165146845346e-08
2239 1.69739635680344e-08
2240 2.28058034679179e-08
2241 1.79983441483955e-08
2242 1.9835809439428e-08
2243 2.30728911532196e-08
2244 2.01439238622925e-08
2245 1.90860394155834e-08
2246 2.0618060148081e-08
2247 2.46620501798134e-08
2248 2.48275835446066e-08
2249 1.91757418832594e-08
2250 1.78881762735728e-08
2251 2.14799964481927e-08
2252 2.70488733633556e-08
2253 2.17713260752816e-08
2254 2.00301712993678e-08
2255 2.46637270606698e-08
2256 2.04084944499527e-08
2257 1.88644584397935e-08
2258 2.28943068947274e-08
2259 2.10139496914508e-08
2260 2.59514365552604e-08
2261 1.76676397956044e-08
2262 1.93349602994886e-08
2263 2.27491394610979e-08
2264 2.3787988467916e-08
2265 2.4701563461349e-08
2266 2.56937013887182e-08
2267 2.40727455746992e-08
2268 1.9442776277856e-08
2269 2.54370302599227e-08
2270 2.05067376413126e-08
2271 2.01823375789445e-08
2272 1.87855686561988e-08
2273 1.9879632162656e-08
2274 1.74296648225436e-08
2275 2.24064482523545e-08
2276 1.82926189751242e-08
2277 1.8124884704207e-08
2278 2.22745359934606e-08
2279 1.65593370127226e-08
2280 2.28033520954796e-08
2281 1.58321658005889e-08
2282 1.68418381463198e-08
2283 2.14329372028033e-08
2284 1.83202342185496e-08
2285 1.82667374559742e-08
2286 1.93817815130615e-08
2287 2.14504201068166e-08
2288 2.38706263644417e-08
2289 1.89816091733519e-08
2290 2.10517683285616e-08
2291 2.21322284943426e-08
2292 1.67477409718231e-08
2293 1.65084355074896e-08
2294 1.58688848728161e-08
2295 2.24528786674227e-08
2296 1.82224333400427e-08
2297 2.3177035402e-08
2298 2.16410231956843e-08
2299 1.65011897479417e-08
2300 2.42301041453175e-08
2301 2.49417055897538e-08
2302 2.12801527510464e-08
2303 2.24576819363165e-08
2304 1.7750290126628e-08
2305 2.63595776317516e-08
2306 1.84757169563454e-08
2307 2.59012562509042e-08
2308 2.73412776863324e-08
2309 2.57480863297133e-08
2310 2.44016646888667e-08
2311 2.17077662512111e-08
2312 1.86325017637046e-08
2313 1.9105002024844e-08
2314 1.6391862089904e-08
2315 1.63743312242559e-08
2316 1.5772370076661e-08
2317 1.6593743268345e-08
2318 1.54904675753187e-08
2319 1.82148109928448e-08
2320 1.66373705923206e-08
2321 1.58755177892544e-08
2322 2.02032985896494e-08
2323 2.33202754884587e-08
2324 2.31006289652669e-08
2325 2.18481801539383e-08
2326 2.69027129462529e-08
2327 2.4254287467329e-08
2328 2.43470577032667e-08
2329 2.22087042089925e-08
2330 2.34301928969671e-08
2331 1.63090430049806e-08
2332 2.18343600977278e-08
2333 1.88381843457819e-08
2334 2.32096972752061e-08
2335 2.45817943778093e-08
2336 2.18840270349574e-08
2337 2.41561526337364e-08
2338 1.79818648859964e-08
2339 2.33301307162037e-08
2340 1.77622041519498e-08
2341 2.11261319549294e-08
2342 2.34735129112096e-08
2343 2.24144045546382e-08
2344 2.61773944743027e-08
2345 2.3079302025053e-08
2346 2.14771862516727e-08
2347 1.58926791726799e-08
2348 1.95531377755742e-08
2349 1.5977878575768e-08
2350 1.63492384075425e-08
2351 2.38427819709841e-08
2352 2.52962220059771e-08
2353 1.81056467596363e-08
2354 2.55101806345692e-08
2355 1.74972409894281e-08
2356 1.5713073509005e-08
2357 1.85182997824995e-08
2358 2.04001651127328e-08
2359 2.80017857789971e-08
2360 2.49081093528503e-08
2361 2.23402114585269e-08
2362 1.9820943109039e-08
2363 1.67316862587086e-08
2364 2.33659473991565e-08
2365 1.76335337442879e-08
2366 1.56307873311334e-08
2367 1.74805556696356e-08
2368 1.59838258184664e-08
2369 1.67904783410222e-08
2370 1.77823480385086e-08
2371 1.69640319569453e-08
2372 1.74033765176773e-08
2373 1.66937734746853e-08
2374 1.68408753609128e-08
2375 1.69412874839736e-08
2376 1.71739102938773e-08
2377 2.00500043234797e-08
2378 2.34412702582176e-08
2379 2.10539710110424e-08
2380 1.93474658516379e-08
2381 2.08169979032391e-08
2382 2.24969145534715e-08
2383 2.14080380089854e-08
2384 2.10554507162897e-08
2385 2.28109477973248e-08
2386 2.05528607466476e-08
2387 1.91746281075211e-08
2388 2.11387316539913e-08
2389 2.28919390110605e-08
2390 2.22593925514047e-08
2391 2.3161328854826e-08
2392 2.04803711767454e-08
2393 2.30795684785789e-08
2394 1.90599234173305e-08
2395 1.60428044182481e-08
2396 2.30515855292879e-08
2397 2.17155626813792e-08
2398 2.76984906122379e-08
2399 1.9484110325152e-08
2400 2.20906208880933e-08
2401 2.38239277194907e-08
2402 1.9851537302884e-08
2403 2.42469369027276e-08
2404 2.04854941898702e-08
2405 2.02808472238303e-08
2406 2.16406323971796e-08
2407 1.74243446338096e-08
2408 1.66720841576762e-08
2409 2.24348930544238e-08
2410 1.82213142352339e-08
2411 1.89330062738691e-08
2412 1.75314109895908e-08
2413 2.23553691114375e-08
2414 2.18097628845726e-08
2415 2.13273327887009e-08
2416 1.70048313208326e-08
2417 2.15704627493096e-08
2418 2.20947313778197e-08
2419 2.00704288744191e-08
2420 2.07369179605621e-08
2421 1.92319671299401e-08
2422 2.28470113938783e-08
2423 1.69894942558813e-08
2424 1.77064300999064e-08
2425 1.63944928743831e-08
2426 1.62634474776269e-08
2427 1.68219642660006e-08
2428 1.65683147201889e-08
2429 1.59906257124476e-08
2430 1.65532796359003e-08
2431 1.67064815315143e-08
2432 1.63277356080016e-08
2433 1.68436109504455e-08
2434 1.64501194888089e-08
2435 1.68884337625741e-08
2436 1.72578769053189e-08
2437 1.74235346150908e-08
2438 1.73345586773621e-08
2439 1.62923043944829e-08
2440 1.59112989450705e-08
2441 1.63867870384138e-08
2442 1.65143934083289e-08
2443 1.57907731335172e-08
2444 1.69612199840685e-08
2445 1.666231952413e-08
2446 1.58719242193683e-08
2447 1.98820586660986e-08
2448 1.69990634901751e-08
2449 1.63094568961242e-08
2450 1.6379738454475e-08
2451 1.79751200590772e-08
2452 1.68887481777347e-08
2453 1.68495049024386e-08
2454 1.63321143276107e-08
2455 1.65383156058851e-08
2456 1.6199226848812e-08
2457 1.88068511874917e-08
2458 2.03034282719727e-08
2459 1.7646584637987e-08
2460 1.96341662928035e-08
2461 1.68691531854392e-08
2462 1.99726599703354e-08
2463 1.71477498867034e-08
2464 1.73150347393403e-08
2465 1.62963207372968e-08
2466 1.69962834917214e-08
2467 1.9608355827927e-08
2468 1.72492828909299e-08
2469 1.67076734669536e-08
2470 1.6054540807886e-08
2471 1.61234545714706e-08
2472 1.77267178713691e-08
2473 1.7346760472492e-08
2474 1.823560324965e-08
2475 1.92395308573623e-08
2476 1.60255080317029e-08
2477 1.70714944403016e-08
2478 1.68009535173042e-08
2479 1.8307929394723e-08
2480 1.71550063043924e-08
2481 1.73558305505139e-08
2482 1.69141003425466e-08
2483 1.57023531954792e-08
2484 1.62182196561389e-08
2485 1.64239342126393e-08
2486 1.8959774195082e-08
2487 1.86375022082075e-08
2488 1.56365711490025e-08
2489 1.64256359624915e-08
2490 1.6208330677614e-08
2491 1.73266379022152e-08
2492 1.62003601644756e-08
2493 1.69135336847148e-08
2494 1.98307112952989e-08
2495 1.57359529850964e-08
2496 1.74965535393312e-08
2497 1.6953029202682e-08
2498 1.7080338921005e-08
2499 1.56290429487171e-08
2500 1.72948979582088e-08
2501 1.60916151514812e-08
2502 2.08966213222084e-08
2503 1.80868937604828e-08
2504 1.89233304581649e-08
2505 1.70979568281382e-08
2506 1.61689523991981e-08
2507 1.84174631101541e-08
2508 1.74962551113822e-08
2509 1.71752709832163e-08
2510 1.70707856739227e-08
2511 1.79076344863915e-08
2512 1.77153669511654e-08
2513 1.91857374431947e-08
2514 1.59332440574644e-08
2515 1.60937823068252e-08
2516 1.70437548518976e-08
2517 1.80800387994395e-08
2518 1.77781434018698e-08
2519 1.64834172977635e-08
2520 1.7042500743969e-08
2521 1.96449434497481e-08
2522 1.57777542142412e-08
2523 1.59446837955102e-08
2524 1.76059877787793e-08
2525 1.73499401512345e-08
2526 1.74320913259862e-08
2527 1.87622877234617e-08
2528 1.89640676495628e-08
2529 1.89469844258383e-08
2530 1.63766689098566e-08
2531 1.59159210255666e-08
2532 1.61800866038675e-08
2533 1.68905209818604e-08
2534 1.68466751659935e-08
2535 1.58779318581992e-08
2536 1.69682632389367e-08
2537 1.73936385294837e-08
2538 1.56431507747357e-08
2539 1.55974042570506e-08
2540 1.77535106615778e-08
2541 1.83158004318784e-08
2542 1.60787365643955e-08
2543 1.76850321054189e-08
2544 1.46683030166628e-08
2545 1.82881745303121e-08
2546 1.67892402203051e-08
2547 1.72250089747195e-08
2548 1.91648332759087e-08
2549 1.68389782118084e-08
2550 1.55451242989102e-08
2551 1.6886222198309e-08
2552 1.92741200777391e-08
2553 1.55403334645143e-08
2554 1.64527325097197e-08
2555 1.56834687459195e-08
2556 1.52737289482729e-08
2557 1.56887409730189e-08
2558 1.89795965610529e-08
2559 1.88691480218495e-08
2560 1.71825842443241e-08
2561 1.59550967993027e-08
2562 1.51240016066367e-08
2563 1.59725193071836e-08
2564 1.51630707989625e-08
2565 1.84063697616921e-08
2566 1.60903663726231e-08
2567 1.67548606100354e-08
2568 1.64974292005127e-08
2569 1.75147611969351e-08
2570 1.54444528277509e-08
2571 1.65885154501666e-08
2572 1.72681904331284e-08
2573 2.03389447506197e-08
2574 1.51605092924001e-08
2575 1.84750277298917e-08
2576 1.87709563448379e-08
2577 2.11142090478234e-08
2578 1.81917734209947e-08
2579 1.79262720223505e-08
2580 1.47779521952884e-08
2581 1.62566244910067e-08
2582 1.93023996786224e-08
2583 1.66497340359228e-08
2584 2.01463699056603e-08
2585 1.60350754896399e-08
2586 2.04154257943401e-08
2587 1.80487713663524e-08
2588 1.79409216372051e-08
2589 1.42838922911892e-08
2590 1.56373793913644e-08
2591 1.52090855465303e-08
2592 1.49804648685858e-08
2593 1.69747966793921e-08
2594 1.76436678600567e-08
2595 1.77335177653504e-08
2596 1.58311586062609e-08
2597 2.04680592474915e-08
2598 1.79463146565695e-08
2599 1.63345710291196e-08
2600 1.71393441661394e-08
2601 1.58746384926189e-08
2602 1.58581112685852e-08
2603 1.62955462457148e-08
2604 2.32319621318311e-08
2605 2.06000478897295e-08
2606 1.55308708116308e-08
2607 1.44001441881869e-08
2608 1.53987489426299e-08
2609 1.63306523859319e-08
2610 1.58194826127556e-08
2611 1.58822892615262e-08
2612 1.66439111382033e-08
2613 1.56171449106068e-08
2614 1.51045060903243e-08
2615 1.56244226445779e-08
2616 1.60776405522256e-08
2617 1.68553508927971e-08
2618 1.44377869659706e-08
2619 1.50742103244283e-08
2620 1.6186275431096e-08
2621 1.70295972878876e-08
2622 1.60684283656565e-08
2623 1.62096274181067e-08
2624 1.49831755891228e-08
2625 1.55555053282797e-08
2626 1.53363473032186e-08
2627 1.58073536482561e-08
2628 1.5975468059537e-08
2629 1.50625059092135e-08
2630 1.50235841545054e-08
2631 1.65763207604641e-08
2632 1.50178411928437e-08
2633 1.57234474329471e-08
2634 1.53272274872052e-08
2635 1.67423497288155e-08
2636 1.55662238654486e-08
2637 1.48806744704189e-08
2638 1.43217269155116e-08
2639 1.59351110085026e-08
2640 1.6236755939758e-08
2641 1.55912349697473e-08
2642 1.69176779252211e-08
2643 1.59233586316532e-08
2644 1.51671546433363e-08
2645 1.59442290481593e-08
2646 1.59165747248835e-08
2647 1.77384205102271e-08
2648 1.61819926347562e-08
2649 1.66303042448135e-08
2650 1.55500448073553e-08
2651 1.53621222409583e-08
2652 1.61001434406671e-08
2653 1.5992347002225e-08
2654 1.66304872095679e-08
2655 1.57719597382311e-08
2656 1.50457761805001e-08
2657 1.5530813968212e-08
2658 1.52064405511965e-08
2659 1.5581969492473e-08
2660 1.51079841970159e-08
2661 1.50602073034634e-08
2662 1.57440513959273e-08
2663 1.59702597812839e-08
2664 1.56988662070034e-08
2665 1.62073501286386e-08
2666 1.48089291940323e-08
2667 1.71184773023469e-08
2668 1.87423481179394e-08
2669 1.57714712401003e-08
2670 1.77139796164738e-08
2671 1.5062312286318e-08
2672 1.64643481070925e-08
2673 1.5359626459599e-08
2674 1.66223035336088e-08
2675 1.52785748497308e-08
2676 1.66962301761941e-08
2677 1.69329137378327e-08
2678 1.44357050757549e-08
2679 1.55902135645647e-08
2680 1.55313770733301e-08
2681 1.53288119975059e-08
2682 1.58020814211568e-08
2683 1.60084265843352e-08
2684 1.50427261758068e-08
2685 1.73612093590236e-08
2686 1.58557895701961e-08
2687 1.59995590109929e-08
2688 1.52016710330827e-08
2689 1.69529137394875e-08
2690 1.51537129511325e-08
2691 1.54514818717644e-08
2692 1.53719170725708e-08
2693 1.4963783101507e-08
2694 1.60435664753322e-08
2695 1.47893022273138e-08
2696 1.68208149631255e-08
2697 1.68759459739931e-08
2698 1.61969442302734e-08
2699 1.5552249266193e-08
2700 1.54611807801075e-08
2701 1.56131108042246e-08
2702 1.49627243928308e-08
2703 1.60478741406678e-08
2704 1.59642716823782e-08
2705 1.62186566399214e-08
2706 1.52954999776966e-08
2707 1.50010173172177e-08
2708 1.59661368570596e-08
2709 1.46139278456303e-08
2710 1.51276324800165e-08
2711 1.49332972654292e-08
2712 1.59492454798738e-08
2713 1.56132351492033e-08
2714 1.56645558746504e-08
2715 1.52913059991988e-08
2716 1.73160543681661e-08
2717 1.49194097076588e-08
2718 1.55491246545125e-08
2719 1.63540399000794e-08
2720 1.55989710037829e-08
2721 1.57708299752812e-08
2722 1.62732547437372e-08
2723 1.5466406821929e-08
2724 1.65488209802334e-08
2725 1.54519366191153e-08
2726 1.53031951555249e-08
2727 1.46322873817439e-08
2728 1.74426979526743e-08
2729 1.46760941177604e-08
2730 1.49801291371432e-08
2731 1.63719917622984e-08
2732 1.50165515577783e-08
2733 1.55275934332622e-08
2734 1.5039129053207e-08
2735 1.49735850385468e-08
2736 1.72275722576387e-08
2737 1.51376511325907e-08
2738 1.5628945249091e-08
2739 1.70517164832518e-08
2740 1.56942459028642e-08
2741 1.56617030455664e-08
2742 1.65829714404708e-08
2743 1.60264619353256e-08
2744 1.60505795321342e-08
2745 1.44488474518312e-08
2746 1.58004986872129e-08
2747 1.50986423363975e-08
2748 1.7727117551658e-08
2749 1.6361090260375e-08
2750 1.61545443688738e-08
2751 1.73818737181364e-08
2752 1.77240142562596e-08
2753 1.59395163734644e-08
2754 1.53256962676096e-08
2755 1.80050019338296e-08
2756 1.58294444219109e-08
2757 1.72839040857298e-08
2758 1.68240426035027e-08
2759 1.65710467570079e-08
2760 1.50122883013637e-08
2761 1.59672879362915e-08
2762 1.53514143619304e-08
2763 1.6493576282528e-08
2764 1.51466554854096e-08
2765 1.65819784569976e-08
2766 1.62977986661872e-08
2767 1.60482489519609e-08
2768 1.63094959759746e-08
2769 1.68186105042878e-08
2770 1.72662435460325e-08
2771 1.91588771514262e-08
2772 1.50190047065735e-08
2773 1.63799285246569e-08
2774 1.85406161534729e-08
2775 1.54343879898988e-08
2776 1.7662014073494e-08
2777 1.52685828425092e-08
2778 1.60246447222789e-08
2779 1.6563934224223e-08
2780 1.66203726337244e-08
2781 1.88920079580157e-08
2782 1.70757790129983e-08
2783 1.80347559108895e-08
2784 1.69906151370469e-08
2785 1.81987065417388e-08
2786 1.58393476112906e-08
2787 1.54176653666127e-08
2788 2.06360422083662e-08
2789 1.78502101988443e-08
2790 1.81662045406483e-08
2791 1.88090858443957e-08
2792 1.48772141272957e-08
2793 1.73712759732325e-08
2794 1.81988255576471e-08
2795 1.86845952043768e-08
2796 1.54789585593562e-08
2797 1.57407011869282e-08
2798 1.6748980868897e-08
2799 1.87180315691649e-08
2800 1.90673787869855e-08
2801 1.82285528893544e-08
2802 1.79728427696091e-08
2803 1.55792747591477e-08
2804 1.65745781544047e-08
2805 1.61070214943493e-08
2806 1.58061510546759e-08
2807 1.64827405058077e-08
2808 1.49991059572585e-08
2809 1.81601649273944e-08
2810 1.68996923122222e-08
2811 1.52703893974149e-08
2812 1.78787882276765e-08
2813 1.57177257875674e-08
2814 1.4614462529039e-08
2815 1.47710617071084e-08
2816 1.59489648154931e-08
2817 1.68586620219457e-08
2818 1.52899684024987e-08
2819 1.56749884183682e-08
2820 1.76561947284881e-08
2821 1.52568837563649e-08
2822 1.56862878242237e-08
2823 1.67187792499135e-08
2824 1.66169176196718e-08
2825 1.78756884849918e-08
2826 1.75747203456922e-08
2827 1.76579408872612e-08
2828 1.86153279457812e-08
2829 1.78156440711064e-08
2830 1.81294659284958e-08
2831 1.85307680311553e-08
2832 1.80698407348245e-08
2833 1.79083645690525e-08
2834 1.66744911211936e-08
2835 1.9143493901197e-08
2836 1.76883041547171e-08
2837 1.64114197787057e-08
2838 2.18333635615409e-08
2839 1.82999695397257e-08
2840 1.6109956035848e-08
2841 1.75617049791299e-08
2842 1.78406605044756e-08
2843 1.80950223693799e-08
2844 1.84411472758939e-08
2845 2.08098160925374e-08
2846 1.97505709564894e-08
2847 1.96384171147201e-08
2848 1.49293786222415e-08
2849 1.75356600351506e-08
2850 1.69084604095815e-08
2851 1.77148145041883e-08
2852 1.61966333678265e-08
2853 1.50887249361631e-08
2854 1.64813940273234e-08
2855 1.79414243461906e-08
2856 1.75082721654007e-08
2857 1.63570117450718e-08
2858 2.03438510482101e-08
2859 1.828709095264e-08
2860 1.86408133373561e-08
2861 1.79993691062919e-08
2862 1.68181575332937e-08
2863 1.68640728048786e-08
2864 1.75718124495461e-08
2865 1.71221206102246e-08
2866 1.67366991377094e-08
2867 1.64398716862024e-08
2868 1.46684469015668e-08
2869 2.04141290538473e-08
2870 1.91994740106338e-08
2871 1.63431650435086e-08
2872 2.17792432977149e-08
2873 1.87432451781433e-08
2874 1.85198221203109e-08
2875 1.59771804675302e-08
2876 1.74197491986661e-08
2877 2.07657002704309e-08
2878 1.84369408628982e-08
2879 1.88654762922624e-08
2880 1.78605699119316e-08
2881 1.9664444295131e-08
2882 1.81532531229323e-08
2883 1.88084694485724e-08
2884 1.49267318505508e-08
2885 1.61609658988482e-08
2886 1.83473645165577e-08
2887 1.66208646845689e-08
2888 1.82321695518795e-08
2889 1.77480981022882e-08
2890 1.88544788670697e-08
2891 1.54613655212188e-08
2892 1.86016340109063e-08
2893 1.54927910500646e-08
2894 1.94796658803398e-08
2895 1.7425579201813e-08
2896 1.92225577677618e-08
2897 1.86514856892472e-08
2898 1.90643731912132e-08
2899 1.73328835728626e-08
2900 1.60765765144788e-08
2901 1.63225291061053e-08
2902 1.77368324472127e-08
2903 1.96825276077561e-08
2904 1.69681850792358e-08
2905 1.52420351895444e-08
2906 1.98151717256678e-08
2907 1.59858402071222e-08
2908 1.76518497596589e-08
2909 1.58993724852508e-08
2910 1.89936049110884e-08
2911 1.77359389397225e-08
2912 1.7474246050142e-08
2913 1.79017423107553e-08
2914 2.12008668398767e-08
2915 2.16725553059405e-08
2916 2.09198027789625e-08
2917 1.724781029111e-08
2918 1.69360490076542e-08
2919 1.47793883797931e-08
2920 2.02196162035762e-08
2921 1.79571966185677e-08
2922 1.64868581009614e-08
2923 1.82957045069543e-08
2924 1.90697821977892e-08
2925 1.70092420148649e-08
2926 1.76524110884202e-08
2927 1.57135193745717e-08
2928 1.7569302457332e-08
2929 2.14726512126617e-08
2930 2.14566480138956e-08
2931 1.78715335863444e-08
2932 1.94735978453764e-08
2933 1.87134077123119e-08
2934 1.95359444177257e-08
2935 1.78386923010976e-08
2936 1.75636749588648e-08
2937 2.18922391326259e-08
2938 1.97934415524514e-08
2939 1.72026339839704e-08
2940 1.88455739902338e-08
2941 1.77166032955256e-08
2942 2.31632899527767e-08
2943 1.94070395309609e-08
2944 1.98430054609844e-08
2945 1.86551183389838e-08
2946 2.04499066569497e-08
2947 1.61372017970507e-08
2948 1.85820017151173e-08
2949 1.54594097523386e-08
2950 1.71999410270018e-08
2951 1.79899632968272e-08
2952 1.86349691233545e-08
2953 1.71409801907885e-08
2954 1.86200139751236e-08
2955 2.09607833312475e-08
2956 1.78353261048869e-08
2957 2.08299670845236e-08
2958 2.16779714179438e-08
2959 1.87880395685625e-08
2960 1.93604385856361e-08
2961 2.26876775144547e-08
2962 1.78414598650534e-08
2963 2.06722141626869e-08
2964 1.64711106975801e-08
2965 1.52324020064043e-08
2966 2.07145909314477e-08
2967 1.99107876852622e-08
2968 1.97429006476568e-08
2969 2.19271427681633e-08
2970 2.10392787636238e-08
2971 1.48641987607334e-08
2972 2.22956053619328e-08
2973 1.71669487514237e-08
2974 2.14417301691583e-08
2975 2.24806981918846e-08
2976 2.21732623373327e-08
2977 2.03967314149622e-08
2978 1.95718676820889e-08
2979 2.12597335291775e-08
2980 1.57251083265919e-08
2981 2.25124665576004e-08
2982 1.67724465427455e-08
2983 1.9612658164192e-08
2984 1.66291247438721e-08
2985 1.85174826583534e-08
2986 1.71122138681312e-08
2987 2.18504325744107e-08
2988 1.92313880376105e-08
2989 2.21931379940088e-08
2990 1.94534042208261e-08
2991 1.94389784269333e-08
2992 1.93253448799169e-08
2993 1.91489082368435e-08
2994 1.91558910955791e-08
2995 1.66431632919739e-08
2996 1.67380385107663e-08
2997 2.14386570718261e-08
2998 2.06965431459594e-08
2999 2.34331576365321e-08
3000 1.66085403208172e-08
3001 1.69330753863051e-08
3002 1.85194366508767e-08
3003 1.97752125785655e-08
3004 1.71926899383834e-08
3005 1.9679520235627e-08
3006 1.83754060856245e-08
3007 2.12777209185333e-08
3008 1.97520915179439e-08
3009 1.89361841762548e-08
3010 1.86906135013487e-08
3011 2.29776642157731e-08
3012 2.27688214948785e-08
3013 1.86082758091288e-08
3014 2.38805899499539e-08
3015 1.91399234239498e-08
3016 1.93914768686909e-08
3017 2.14792486019633e-08
3018 1.87953261843177e-08
3019 2.2805416222127e-08
3020 1.67744254042645e-08
3021 2.01626271234545e-08
3022 1.68900751162937e-08
3023 1.6369419597595e-08
3024 2.50762095532764e-08
3025 1.57281405677168e-08
3026 1.96961078557933e-08
3027 2.30832970515849e-08
3028 1.61880979732132e-08
3029 2.12662794041307e-08
3030 1.76392074280329e-08
3031 1.71965481854386e-08
3032 2.15156941152372e-08
3033 1.76832255505133e-08
3034 2.05109813578019e-08
3035 1.87173903043458e-08
3036 2.02583514408161e-08
3037 2.31902248515325e-08
3038 1.86586905925878e-08
3039 1.97899225895526e-08
3040 1.93596445541289e-08
3041 2.03455901015559e-08
3042 1.75922139078466e-08
3043 2.10265760358652e-08
3044 2.04665280278959e-08
3045 1.6688566972789e-08
3046 1.62294533367913e-08
3047 2.04912833368098e-08
3048 1.77259931177787e-08
3049 1.94353244609147e-08
3050 2.15097664124642e-08
3051 1.90562552404572e-08
3052 1.73128640312825e-08
3053 1.80530221882691e-08
3054 1.75116134926157e-08
3055 1.66624811726024e-08
3056 2.32173924530343e-08
3057 2.07567332211056e-08
3058 1.76666450357743e-08
3059 2.18665281437325e-08
3060 2.00143261963603e-08
3061 2.28063790075339e-08
3062 1.51221062338891e-08
3063 2.38773250060831e-08
3064 1.58757202939341e-08
3065 2.62029615782922e-08
3066 2.16331486058152e-08
3067 2.28942429458812e-08
3068 2.37155752813578e-08
3069 2.35470203335808e-08
3070 2.33793997495013e-08
3071 2.40148754215852e-08
3072 2.21295231028762e-08
3073 2.03674730414605e-08
3074 1.84451742768488e-08
3075 2.04597423447694e-08
3076 1.63179549872439e-08
3077 1.85556707776868e-08
3078 1.78411401208223e-08
3079 1.75680110459098e-08
3080 1.97941396606893e-08
3081 2.25477432280741e-08
3082 1.77024244152335e-08
3083 2.01868743943123e-08
3084 2.33294379370363e-08
3085 1.93744078558211e-08
3086 2.320778413889e-08
3087 2.15933049219075e-08
3088 2.26561223115596e-08
3089 2.23684608613439e-08
3090 2.39207533780927e-08
3091 2.23015028666396e-08
3092 1.67388751748376e-08
3093 1.95371505640196e-08
3094 2.14574349399754e-08
3095 2.38353798920343e-08
3096 2.01745979921952e-08
3097 2.3163618578792e-08
3098 1.95489118226533e-08
3099 1.73852949814091e-08
3100 2.04036698647769e-08
3101 1.87626589820411e-08
3102 1.75718604111808e-08
3103 2.07091108705981e-08
3104 1.67131091188821e-08
3105 2.40780213545122e-08
3106 2.37204602626662e-08
3107 1.95413285553059e-08
3108 1.72826606359422e-08
3109 1.65343543301333e-08
3110 2.64746446987374e-08
3111 2.30775309972842e-08
3112 2.02931840220799e-08
3113 2.15115480983741e-08
3114 2.36133796960303e-08
3115 2.09689261509993e-08
3116 1.60254742809229e-08
3117 2.37356179155768e-08
3118 2.27763976567985e-08
3119 2.26762360000521e-08
3120 1.79909580566573e-08
3121 1.87547026797574e-08
3122 2.09293400388333e-08
3123 1.97749034924755e-08
3124 1.92966460588195e-08
3125 2.32075940687082e-08
3126 2.43991706838642e-08
3127 2.14046238511401e-08
3128 2.0845856596452e-08
3129 2.05431316402382e-08
3130 2.53899425928239e-08
3131 2.18378453098467e-08
3132 2.23318430414565e-08
3133 2.08489918662735e-08
3134 1.77345018670394e-08
3135 2.30025491987362e-08
3136 2.57065480013807e-08
3137 1.625412160422e-08
3138 2.23890275208305e-08
3139 1.7475993985272e-08
3140 2.42975293218706e-08
3141 1.73792145119478e-08
3142 2.23768754636922e-08
3143 2.09051940203153e-08
3144 2.19745182050701e-08
3145 1.69727343291015e-08
3146 2.19838689474727e-08
3147 2.28811938285389e-08
3148 1.63592819291125e-08
3149 2.00985432741163e-08
3150 2.37951507386924e-08
3151 2.75897669155256e-08
3152 2.11083754919628e-08
3153 2.4581410684732e-08
3154 2.17771685129264e-08
3155 2.25072884774136e-08
3156 1.84613888620788e-08
3157 2.18746283309201e-08
3158 1.86155286741041e-08
3159 1.97954790337462e-08
3160 2.13067110621523e-08
3161 2.22881553213483e-08
3162 1.81793691211851e-08
3163 1.88076256790737e-08
3164 2.00585965615119e-08
3165 2.23418314959645e-08
3166 2.38809185759692e-08
3167 2.33035333252474e-08
3168 1.78750898527369e-08
3169 2.88353412258857e-08
3170 2.39195490081556e-08
3171 2.45798865705638e-08
3172 2.07385433270701e-08
3173 2.1737520228271e-08
3174 2.45650983998758e-08
3175 1.71254654901531e-08
3176 1.92036022639286e-08
3177 2.52427163616176e-08
3178 1.79487358309416e-08
3179 2.21054872184823e-08
3180 1.70783849284817e-08
3181 2.73850062626479e-08
3182 2.42303261899224e-08
3183 2.12651602993219e-08
3184 2.49645673022769e-08
3185 2.52289478197554e-08
3186 2.1792946114374e-08
3187 1.57199515626871e-08
3188 2.35335555487381e-08
3189 2.5020964855571e-08
3190 2.53375880276963e-08
3191 1.74535763619588e-08
3192 2.5274268011799e-08
3193 2.74913158904155e-08
3194 2.49639366955989e-08
3195 2.3387256575802e-08
3196 2.02133065840826e-08
3197 2.03507930507385e-08
3198 1.73688690097151e-08
3199 2.24287362016184e-08
3200 2.13207886901046e-08
3201 2.31417978113768e-08
3202 2.04516847901459e-08
3203 2.02813321692474e-08
3204 2.42285391749419e-08
3205 2.5282275828431e-08
3206 2.30179288962518e-08
3207 2.4085268890417e-08
3208 2.48920901668725e-08
3209 2.55999879072988e-08
3210 2.6000252617564e-08
3211 2.29642687088472e-08
3212 1.72900822548172e-08
3213 2.48122660195804e-08
3214 2.09154009667145e-08
3215 1.91845295205439e-08
3216 2.29942020979479e-08
3217 2.03195007486556e-08
3218 2.38958275389223e-08
3219 2.19302744852712e-08
3220 1.94854052892879e-08
3221 2.73009757023601e-08
3222 2.35498873735196e-08
3223 2.84083334634033e-08
3224 1.81978911939495e-08
3225 2.25208065529614e-08
3226 2.45970106504956e-08
3227 2.11297486174544e-08
3228 2.17685549586122e-08
3229 2.07291179776803e-08
3230 2.38824267029258e-08
3231 2.71275908403368e-08
3232 3.20066284587028e-08
3233 1.67569762510311e-08
3234 1.98367207104866e-08
3235 1.81148624989191e-08
3236 1.79687784651605e-08
3237 1.88030142567186e-08
3238 1.74255134766099e-08
3239 1.99877945306071e-08
3240 1.97560279247e-08
3241 2.83584604687803e-08
3242 2.80568155375249e-08
3243 2.52358525187901e-08
3244 2.56606806914306e-08
3245 2.42151525498002e-08
3246 2.42994033783361e-08
3247 2.64746997657994e-08
3248 2.57184140650679e-08
3249 2.5751573318189e-08
3250 2.34932677756206e-08
3251 2.3206988331026e-08
3252 2.76952576427902e-08
3253 2.74402278677144e-08
3254 2.8108811278571e-08
3255 2.4587825109279e-08
3256 2.54160283930105e-08
3257 2.83534173917133e-08
3258 2.95135151873183e-08
3259 2.44036844065931e-08
3260 2.26500045386047e-08
3261 2.86688344175445e-08
3262 2.3282732186658e-08
3263 2.78165721567802e-08
3264 2.3410484217834e-08
3265 1.71621348243889e-08
3266 2.26493614974288e-08
3267 2.17480806696813e-08
3268 2.41486652896583e-08
3269 2.16571400812882e-08
3270 2.61422723468741e-08
3271 2.7508422206779e-08
3272 2.49674450003567e-08
3273 2.36552004651003e-08
3274 2.40032900222786e-08
3275 3.00220790450112e-08
3276 3.11564107846607e-08
3277 2.19399751699711e-08
3278 1.83830675126728e-08
3279 2.64958472939725e-08
3280 2.60820982589394e-08
3281 3.08054737274688e-08
3282 2.65895447881803e-08
3283 2.6289789900602e-08
3284 3.00585014656463e-08
3285 2.59224091081478e-08
3286 3.53828291110858e-08
3287 3.45022748149404e-08
3288 3.38899361906897e-08
3289 3.16393773402979e-08
3290 3.37005765516096e-08
3291 3.42185693114061e-08
3292 2.6645183837104e-08
3293 3.47938957645511e-08
3294 2.95156219465298e-08
3295 3.02387697104223e-08
3296 2.46842120077417e-08
3297 3.2561423779498e-08
3298 2.57562877692408e-08
3299 3.25217399677058e-08
3300 2.43799647137166e-08
3301 2.68162345662404e-08
3302 3.14806634094111e-08
3303 2.80748686520838e-08
3304 2.21679474776693e-08
3305 2.7067960317595e-08
3306 3.10636529832209e-08
3307 2.77867169273804e-08
3308 3.49097284413347e-08
3309 3.24418962804884e-08
3310 3.2968028307323e-08
3311 2.66342414789733e-08
3312 3.16122275023645e-08
3313 3.21517354961998e-08
3314 3.64566652422127e-08
3315 3.05797343003178e-08
3316 3.07237968399932e-08
3317 2.42027038410697e-08
3318 3.17693462648094e-08
3319 3.73972568468162e-08
3320 2.63879389450494e-08
3321 3.29532987564107e-08
3322 3.16270920563966e-08
3323 2.25831708888791e-08
3324 2.82856316147218e-08
3325 2.07202894841885e-08
3326 2.6784439555172e-08
3327 2.41884166030104e-08
3328 2.84618586476881e-08
3329 2.6816758591508e-08
3330 2.61471146956183e-08
3331 3.96684285419724e-08
3332 2.21603126959735e-08
3333 2.51053595690109e-08
3334 3.22630384630429e-08
3335 2.94914350718045e-08
3336 3.36528458433349e-08
3337 2.84283885321202e-08
3338 3.67639891862837e-08
3339 2.94987589910534e-08
3340 2.51544314266994e-08
3341 3.01154159387806e-08
3342 3.82787490593728e-08
3343 3.31405090037151e-08
3344 3.32120606572062e-08
3345 3.64229322258325e-08
3346 3.69784061149403e-08
3347 3.12812851177569e-08
3348 3.56787204225384e-08
3349 3.52434490480391e-08
3350 3.29326113046591e-08
3351 3.26858433652433e-08
3352 3.75576902911234e-08
3353 2.87516357388995e-08
3354 2.97335329690895e-08
3355 3.77779514337817e-08
3356 3.89435967917962e-08
3357 3.67590935468343e-08
3358 3.52833922079299e-08
3359 3.34069483187704e-08
3360 3.44066535262755e-08
3361 3.50105615609664e-08
3362 4.41148273466752e-08
3363 2.7791676515676e-08
3364 3.04980183329917e-08
3365 3.93442434187818e-08
3366 3.21627631194588e-08
3367 3.96187225248923e-08
3368 4.08796587691995e-08
3369 3.8154755799269e-08
3370 3.62017402721904e-08
3371 4.02387065889798e-08
3372 3.31226495120518e-08
3373 4.1693681396282e-08
3374 3.72826640671065e-08
3375 3.34318706052272e-08
3376 4.25038351181684e-08
3377 3.03453440153589e-08
3378 2.90009438685956e-08
3379 3.15323731570061e-08
3380 3.42090018534691e-08
3381 3.45151107694619e-08
3382 3.89874017514558e-08
3383 3.44872859159295e-08
3384 3.26423155172506e-08
3385 3.6291712746106e-08
3386 3.40106680596364e-08
3387 3.14203276730041e-08
3388 2.93417414809483e-08
3389 3.70566546337159e-08
3390 3.61123113634676e-08
3391 3.13244434835269e-08
3392 2.82729875067389e-08
3393 3.15962793706603e-08
3394 3.65293395532262e-08
3395 3.56234046705595e-08
3396 4.0626151331935e-08
3397 3.98386390543237e-08
3398 3.3511543762188e-08
3399 3.50551125904985e-08
3400 3.21364304056715e-08
3401 4.07530897916786e-08
3402 3.95598185320978e-08
3403 4.00449593485064e-08
3404 3.66190029410518e-08
3405 3.35277334784223e-08
3406 4.07603266694423e-08
3407 3.69687995771528e-08
3408 3.79437317121756e-08
3409 3.60312135683216e-08
3410 3.36927463706616e-08
3411 3.33514975636717e-08
3412 3.33203900027002e-08
3413 3.97201134205716e-08
3414 3.74799320468355e-08
3415 3.3088049633534e-08
3416 3.77585571698091e-08
3417 3.45548123448225e-08
3418 3.76949635949586e-08
3419 4.1136960504673e-08
3420 3.93451173863468e-08
3421 3.63407721692965e-08
3422 4.21881516388112e-08
3423 4.25708464035779e-08
3424 3.46868880285456e-08
3425 2.83283743129914e-08
3426 4.36754845623e-08
3427 3.92818542138684e-08
3428 3.73956083876692e-08
3429 3.08736041176871e-08
3430 3.06191267895883e-08
3431 3.89862577776512e-08
3432 2.9006281820898e-08
3433 3.77816569141487e-08
3434 2.80823755360871e-08
3435 3.05024450142355e-08
3436 4.10298781616802e-08
3437 3.33207204050723e-08
3438 4.054718871771e-08
3439 3.873645226804e-08
3440 4.22094963425934e-08
3441 3.78216675755993e-08
3442 2.41457591698691e-08
3443 2.48525715562664e-08
3444 2.85161334545592e-08
3445 3.31197789193993e-08
3446 3.53183047252514e-08
3447 3.78778324261475e-08
3448 2.86116463854569e-08
3449 4.5896431544179e-08
3450 4.43718768394774e-08
3451 3.33250511630467e-08
3452 4.17784598027993e-08
3453 3.96194153040597e-08
3454 3.75689630516263e-08
3455 3.90260161964306e-08
3456 4.09078531049545e-08
3457 3.54596920715267e-08
3458 3.41397630165829e-08
3459 2.75947744654559e-08
3460 3.21217399346096e-08
3461 2.95409527950596e-08
3462 4.0856470207018e-08
3463 2.96776221375694e-08
3464 3.69802144462028e-08
3465 3.75345905467839e-08
3466 3.13414858510441e-08
3467 4.60766074183994e-08
3468 3.54349722897496e-08
3469 3.15573203124586e-08
3470 4.06355944448933e-08
3471 3.19046833396897e-08
3472 4.1793505545229e-08
3473 3.38787842224519e-08
3474 3.7914471562317e-08
3475 3.55693074993724e-08
3476 4.32276614503735e-08
3477 2.49588616441088e-08
3478 4.07627460674576e-08
3479 3.27451239456877e-08
3480 3.57428042718766e-08
3481 2.57606362907836e-08
3482 3.6322759910945e-08
3483 3.56303822002246e-08
3484 4.2169549629989e-08
3485 2.65658712805816e-08
3486 3.80436766533876e-08
3487 2.86289196793632e-08
3488 2.80260703533486e-08
3489 2.78633720540711e-08
3490 3.43869785979223e-08
3491 3.48133042393783e-08
3492 2.76252478670358e-08
3493 3.59252929627019e-08
3494 3.32361338450937e-08
3495 4.34213660582827e-08
3496 3.80885865070013e-08
3497 3.55480054281543e-08
3498 2.34687398403821e-08
3499 2.88974462137048e-08
3500 3.97547381680852e-08
3501 3.4699404238836e-08
3502 3.95870785041552e-08
3503 3.73404276388101e-08
3504 3.45303270421482e-08
3505 3.65470071983509e-08
3506 3.88761414171768e-08
3507 3.53351339299479e-08
3508 4.23039807628811e-08
3509 4.50606840729506e-08
3510 4.43230305791076e-08
3511 4.37652509788222e-08
3512 4.24389305919703e-08
3513 4.12873220057008e-08
3514 3.35486483038494e-08
3515 3.21455857488218e-08
3516 4.26164667999274e-08
3517 4.12715941422448e-08
3518 4.07181701689296e-08
3519 4.8881943826018e-08
3520 4.87740621224475e-08
3521 3.86716081379745e-08
3522 5.08296444934331e-08
3523 3.60723291237264e-08
3524 2.88797519232276e-08
3525 4.16963601423959e-08
3526 3.18699058254879e-08
3527 4.49770141131012e-08
3528 3.81161520124351e-08
3529 2.80563785537424e-08
3530 4.32119513504858e-08
3531 3.92717254271702e-08
3532 3.6580509288342e-08
3533 4.1977191500564e-08
3534 3.74659698820778e-08
3535 3.74498831945402e-08
3536 5.01500529992427e-08
3537 3.55372549165622e-08
3538 3.92411401151094e-08
3539 4.02217494865909e-08
3540 4.5471011844711e-08
3541 3.52836444506011e-08
3542 4.3717459874415e-08
3543 3.79192393040739e-08
3544 3.60293412882129e-08
3545 3.25638005449491e-08
3546 3.94643038248432e-08
3547 3.376773349828e-08
3548 3.67946739743275e-08
3549 3.40521744135458e-08
3550 3.45789530342699e-08
3551 2.76231801876747e-08
3552 3.18984803016065e-08
3553 2.90069639419244e-08
3554 3.69262167509987e-08
3555 3.49857707249157e-08
3556 4.17839416400057e-08
3557 4.13466132442863e-08
3558 3.59851135556255e-08
3559 3.50334126153484e-08
3560 4.04659417085895e-08
3561 4.53687682977488e-08
3562 3.64771217675752e-08
3563 3.73695527855489e-08
3564 3.37595977839555e-08
3565 3.81558109552316e-08
3566 3.88560650321779e-08
3567 5.3216659523514e-08
3568 3.1234584696449e-08
3569 3.42246835316473e-08
3570 3.46104904735967e-08
3571 3.58783438514365e-08
3572 4.19262029538459e-08
3573 3.67939172463139e-08
3574 3.60990988212961e-08
3575 3.16211803408351e-08
3576 3.5533574305191e-08
3577 3.29664260334539e-08
3578 3.88317538124738e-08
3579 3.84571201550443e-08
3580 3.51389211061814e-08
3581 3.85664158386589e-08
3582 3.56410474466884e-08
3583 4.26249044949145e-08
3584 3.48262965133017e-08
3585 2.28084395814676e-08
3586 3.9381671257388e-08
3587 3.21855893048451e-08
3588 3.65762851117779e-08
3589 2.97051698794348e-08
3590 3.9458377898427e-08
3591 3.15221448943248e-08
3592 3.02965617038353e-08
3593 3.41700072681306e-08
3594 3.65382426537053e-08
3595 3.83780189849858e-08
3596 3.62038505841156e-08
3597 3.50812285887514e-08
3598 4.33385345388615e-08
3599 3.26151443630351e-08
3600 3.50072468791041e-08
3601 3.22276498820884e-08
3602 2.97238322843896e-08
3603 3.4559462847028e-08
3604 4.74558881080611e-08
3605 4.27817639092609e-08
3606 3.64446570699783e-08
3607 4.13868512794124e-08
3608 3.54518085998734e-08
3609 3.37923538040741e-08
3610 4.16601260155858e-08
3611 3.31134231146279e-08
3612 3.72272239701488e-08
3613 3.84678919829184e-08
3614 2.53765968238895e-08
3615 3.54071651997856e-08
3616 3.47341533313283e-08
3617 3.52829196970106e-08
3618 3.44842590038752e-08
3619 4.04113258412053e-08
3620 3.52838931405586e-08
3621 3.53289912879973e-08
3622 3.20935136244316e-08
3623 3.30539080550807e-08
3624 3.6671462311233e-08
3625 2.70752096298565e-08
3626 3.18879251892668e-08
3627 3.23975726246317e-08
3628 3.30293516981328e-08
3629 3.27740501404605e-08
3630 3.76833817483657e-08
3631 5.32806723185786e-08
3632 4.20014210078534e-08
3633 3.65124144252604e-08
3634 3.05596721261736e-08
3635 3.02536840024459e-08
3636 3.12031218641096e-08
3637 3.32639338296303e-08
3638 3.5725545188825e-08
3639 3.46550663721246e-08
3640 3.18339417049174e-08
3641 3.35535474960125e-08
3642 3.5105809814695e-08
3643 2.67077258087056e-08
3644 3.48711530762102e-08
3645 3.47465984873452e-08
3646 3.37487477963805e-08
3647 4.76457451270562e-08
3648 4.36254765645572e-08
3649 5.23840171240408e-08
3650 3.98429591541571e-08
3651 4.88135754039831e-08
3652 4.10717611032396e-08
3653 4.66163179169143e-08
3654 4.08049558586754e-08
3655 3.47215554086233e-08
3656 3.06592937704409e-08
3657 3.8504850863319e-08
3658 2.92458217643343e-08
3659 4.23157580087263e-08
3660 3.67424490832491e-08
3661 3.18290140910449e-08
3662 3.85803069491431e-08
3663 3.78083200303081e-08
3664 3.62750363080977e-08
3665 4.81008370911695e-08
3666 4.6336296577465e-08
3667 5.04211072893668e-08
3668 4.72915786531303e-08
3669 4.18611847408101e-08
3670 3.9277423979911e-08
3671 4.10958094221314e-08
3672 3.00940072861522e-08
3673 3.5196972447693e-08
3674 3.09425374211969e-08
3675 3.25703375381181e-08
3676 4.47205685816243e-08
3677 3.15673460704602e-08
3678 4.39319762790547e-08
3679 4.04797972919368e-08
3680 3.38597629934156e-08
3681 3.1512353615426e-08
3682 3.96792998458295e-08
3683 3.60709435653916e-08
3684 4.42080470008932e-08
3685 4.43640040259652e-08
3686 4.62003519885457e-08
3687 2.82689374131451e-08
3688 3.71327644188568e-08
3689 4.23023216455931e-08
3690 3.39409922389677e-08
3691 3.60435485902144e-08
3692 3.68987151944111e-08
3693 4.08462774714735e-08
3694 3.53913058859234e-08
3695 3.67386583377538e-08
3696 4.07887235098769e-08
3697 4.48370300887291e-08
3698 3.64386565365749e-08
3699 3.69829749047312e-08
3700 2.75118168246991e-08
3701 3.67515404775531e-08
3702 3.67960915070853e-08
3703 2.96507227659504e-08
3704 2.61556856173684e-08
3705 3.34998269124753e-08
3706 5.14150606534258e-08
3707 4.21547596829441e-08
3708 3.74075170839205e-08
3709 2.92197359641477e-08
3710 3.42704957745354e-08
3711 2.92421660219588e-08
3712 3.57077425405805e-08
3713 3.6394965263753e-08
3714 3.5453638247418e-08
3715 3.90787810999882e-08
3716 3.42908350603466e-08
3717 3.47542155054725e-08
3718 3.48505260205911e-08
3719 3.24276427932091e-08
3720 3.18954569422658e-08
3721 4.28709618915946e-08
3722 3.72351536270799e-08
3723 3.81078457678541e-08
3724 4.43553247464479e-08
3725 3.8051204853673e-08
3726 3.80339741923308e-08
3727 4.03785485048047e-08
3728 3.68385784099701e-08
3729 4.01974844521646e-08
3730 3.18237418639455e-08
3731 3.76894959686069e-08
3732 3.0694465635861e-08
3733 3.34236247567787e-08
3734 3.26635962721866e-08
3735 3.47253568122596e-08
3736 3.18256141440543e-08
3737 3.16281258960771e-08
3738 3.53351161663795e-08
3739 3.76266626744837e-08
3740 3.44087567327733e-08
3741 3.40838752777017e-08
3742 3.81351767941851e-08
3743 4.13318979042288e-08
3744 3.27762208485183e-08
3745 4.01987101383838e-08
3746 2.9223537367784e-08
3747 3.66034811349891e-08
3748 3.64151411247349e-08
3749 4.35598757064781e-08
3750 4.73489123464788e-08
3751 3.73118105301273e-08
3752 3.13271328877818e-08
3753 3.82498370754547e-08
3754 3.87607741458851e-08
3755 3.62709009493756e-08
3756 3.16068735628505e-08
3757 3.46789654770419e-08
3758 4.51187069927528e-08
3759 2.14169428858213e-08
3760 3.30501350731538e-08
3761 3.55000437934905e-08
3762 2.43277735734182e-08
3763 3.8652135714301e-08
3764 2.69158917376444e-08
3765 4.46714238933055e-08
3766 4.334297543096e-08
3767 2.28901804177895e-08
3768 3.14458930006367e-08
3769 4.01137931760331e-08
3770 1.90973956648577e-08
3771 3.63152246052323e-08
3772 2.98168671974963e-08
3773 3.95285617571517e-08
3774 4.42251035792651e-08
3775 3.66132759666016e-08
3776 2.83899854736092e-08
3777 3.12750003672591e-08
3778 3.38484049677845e-08
3779 3.58235539010821e-08
3780 3.97845845157008e-08
3781 3.19586277441886e-08
3782 3.24743503199443e-08
3783 2.8615737335258e-08
3784 4.32637321523544e-08
3785 3.13063885926113e-08
3786 2.6727420276984e-08
3787 4.09724343342077e-08
3788 3.75691193710281e-08
3789 2.3800552639841e-08
3790 3.06729255328264e-08
3791 2.71618247893457e-08
3792 3.94819998916773e-08
3793 3.18002477683876e-08
3794 2.998933013032e-08
3795 2.70888609321673e-08
3796 3.22966116073076e-08
3797 3.84131304542734e-08
3798 3.44500463711483e-08
3799 3.70294017670858e-08
3800 2.96154460954767e-08
3801 1.92201703441697e-08
3802 3.24993543188157e-08
3803 3.55531390994202e-08
3804 3.85208061004505e-08
3805 3.71269415211373e-08
3806 2.77750533683729e-08
3807 3.91793157916709e-08
3808 4.32125659699523e-08
3809 2.07175752109379e-08
3810 3.63976653261489e-08
3811 2.21846985226648e-08
3812 3.03999101447516e-08
3813 4.3616243061706e-08
3814 3.15317478793986e-08
3815 2.90146608961095e-08
3816 3.65535264279515e-08
3817 2.92816135782914e-08
3818 2.72248943389286e-08
3819 3.7840109712306e-08
3820 3.36922418853192e-08
3821 3.01124316592905e-08
3822 4.31205791073808e-08
3823 3.89064318540022e-08
3824 3.23344444552731e-08
3825 2.7820004078194e-08
3826 3.24874136481412e-08
3827 3.59631471269495e-08
3828 3.06387342163816e-08
3829 3.47758479790627e-08
3830 3.60854883751927e-08
3831 3.16196846483763e-08
3832 2.34053114667176e-08
3833 2.34511592367426e-08
3834 3.22880389092006e-08
3835 2.99947906512443e-08
3836 3.12024894810747e-08
3837 3.56250282607107e-08
3838 3.5212668336726e-08
3839 2.93311224197623e-08
3840 2.66019082317825e-08
3841 2.97956734840454e-08
3842 3.56294904690913e-08
3843 2.77981762053514e-08
3844 1.72016108024309e-08
3845 2.68410449422163e-08
3846 2.75097296054128e-08
3847 2.74661200450055e-08
3848 2.95220861090684e-08
3849 3.12354906384371e-08
3850 3.7057183988054e-08
3851 2.90380111067634e-08
3852 3.77448259314406e-08
3853 3.59428398155615e-08
3854 2.8167013610414e-08
3855 2.84335364142407e-08
3856 3.5021074040742e-08
3857 2.73803504313719e-08
3858 3.80437477076612e-08
3859 2.5785132251599e-08
3860 2.90722130813492e-08
3861 1.97395717549398e-08
3862 2.75955436279673e-08
3863 3.83454690222607e-08
3864 3.35146239649475e-08
3865 3.24120748018686e-08
3866 2.63579256198909e-08
3867 3.30460387942821e-08
3868 2.65995669934682e-08
3869 3.08862020403922e-08
3870 1.66458438144446e-08
3871 3.7648813844271e-08
3872 3.19405408788498e-08
3873 2.80947229924777e-08
3874 2.66311523944296e-08
3875 2.70795030843374e-08
3876 3.01346432252103e-08
3877 3.88993157685036e-08
3878 3.02103053684277e-08
3879 3.12343466646325e-08
3880 2.9642365007021e-08
3881 3.20428625855129e-08
3882 3.57083997926111e-08
3883 2.67221498262415e-08
3884 2.89097226158219e-08
3885 3.354238131692e-08
3886 2.4218660854558e-08
3887 2.45024018852291e-08
3888 2.5104119671937e-08
3889 3.34598979634393e-08
3890 2.70212190400798e-08
3891 1.99467731221148e-08
3892 2.98545153043506e-08
3893 3.90926473414765e-08
3894 2.59859458395795e-08
3895 3.64332528590694e-08
3896 3.027275141676e-08
3897 3.38485719453274e-08
3898 3.49735103100102e-08
3899 1.86976532035033e-08
3900 2.46310811746753e-08
3901 3.73394115626979e-08
3902 3.73599569059024e-08
3903 2.87992278913407e-08
3904 2.88095343137229e-08
3905 3.37548868856175e-08
3906 3.04574250264977e-08
3907 3.58656144783254e-08
3908 2.69977498135177e-08
3909 3.46504087644917e-08
3910 3.35642518223267e-08
3911 2.88277437476836e-08
3912 3.26368727598947e-08
3913 2.42619133672406e-08
3914 2.97203133214907e-08
3915 2.67193769332152e-08
3916 3.65151606729341e-08
3917 3.10421235383274e-08
3918 3.54394984469764e-08
3919 1.89269382389057e-08
3920 2.48471465624789e-08
3921 2.88582864271802e-08
3922 2.94702307002126e-08
3923 2.84983023846053e-08
3924 3.39370060942201e-08
3925 2.42649331738676e-08
3926 2.68492534871712e-08
3927 2.4066432402492e-08
3928 3.24946398677639e-08
3929 2.44066811205812e-08
3930 3.09525631791985e-08
3931 2.71434910104063e-08
3932 3.47872166628349e-08
3933 2.71433151510792e-08
3934 3.33094405391421e-08
3935 2.82558918485165e-08
3936 2.80434946375863e-08
3937 3.033568773958e-08
3938 2.40407285190258e-08
3939 3.12601535767953e-08
3940 3.17290904661149e-08
3941 3.1047491688696e-08
3942 3.30528457936907e-08
3943 3.65893413345475e-08
3944 3.06320799836612e-08
3945 2.71511577665251e-08
3946 2.63320014681767e-08
3947 2.79334138042486e-08
3948 2.57162255934418e-08
3949 3.56163170067703e-08
3950 2.84693228991273e-08
3951 3.31276410747705e-08
3952 2.53604834909993e-08
3953 2.17945927971641e-08
3954 2.39855548755941e-08
3955 2.58461714253144e-08
3956 2.62330299705127e-08
3957 3.1645082998466e-08
3958 2.80488325898887e-08
3959 2.89297545918998e-08
3960 2.58723957813345e-08
3961 2.47313405310479e-08
3962 2.65995776516093e-08
3963 2.93988975386128e-08
3964 3.22407629482768e-08
3965 2.73494293878684e-08
3966 2.78494880490143e-08
3967 3.11005905473394e-08
3968 2.81895040643576e-08
3969 3.00785494289357e-08
3970 2.9154772818174e-08
3971 2.3735770682265e-08
3972 2.6923274276669e-08
3973 2.56034606849198e-08
3974 2.97515256875158e-08
3975 2.50812561830571e-08
3976 3.0582331334017e-08
3977 2.86961725493029e-08
3978 2.43490578810679e-08
3979 3.39695418460906e-08
3980 3.41311547913392e-08
3981 2.50923370970213e-08
3982 2.799746212645e-08
3983 3.65562335957748e-08
3984 2.35700987616383e-08
3985 3.27771694230705e-08
3986 2.65749378058899e-08
3987 3.31758691629602e-08
3988 3.01181266593176e-08
3989 2.52803680211855e-08
3990 2.57298520267568e-08
3991 2.49004745711545e-08
3992 3.08035055240907e-08
3993 1.98626342040598e-08
3994 3.28106075642154e-08
3995 3.58313094750429e-08
3996 2.9231921772066e-08
3997 2.58763694915842e-08
3998 2.94235817932531e-08
3999 2.58350674187113e-08
4000 2.60636223714528e-08
4001 3.29798339748777e-08
4002 2.73166449460405e-08
4003 2.77358651601389e-08
4004 3.38674936983807e-08
4005 2.11103010627767e-08
4006 2.56883954108389e-08
4007 1.97946476987454e-08
4008 3.72649111568535e-08
4009 2.84503727243646e-08
4010 2.73266902439673e-08
4011 2.97832745133064e-08
4012 1.77832522041399e-08
4013 2.65815103261957e-08
4014 3.09527763420192e-08
4015 2.77973573048484e-08
4016 2.01819378986556e-08
4017 2.981481372899e-08
4018 2.89333907943501e-08
4019 2.72134386136713e-08
4020 3.1028367430963e-08
4021 2.69860596091576e-08
4022 2.66387978342664e-08
4023 2.42751241330552e-08
4024 2.5621575971968e-08
4025 2.69681041942249e-08
4026 3.05619778373512e-08
4027 2.68981477091756e-08
4028 3.20716537771659e-08
4029 2.98926678965472e-08
4030 3.23967768167677e-08
4031 3.29922933417492e-08
4032 2.83601000461431e-08
4033 3.2894043044962e-08
4034 2.94141280221538e-08
4035 2.94479161055961e-08
4036 2.85043704195687e-08
4037 3.04685165986029e-08
4038 2.6342240388999e-08
4039 2.17536832991527e-08
4040 3.22584163825468e-08
4041 2.88187198549394e-08
4042 2.83641856668737e-08
4043 3.63467798081274e-08
4044 3.40302257484382e-08
4045 2.37926833790425e-08
4046 2.94596986805118e-08
4047 2.32990728932236e-08
4048 2.55540602012161e-08
4049 2.94091879737834e-08
4050 2.52867042860316e-08
4051 3.00954106080553e-08
4052 3.16594217508737e-08
4053 2.67936766107368e-08
4054 2.79415353077184e-08
4055 2.68390927260498e-08
4056 2.25312177803971e-08
4057 2.23159446477439e-08
4058 3.15703658770872e-08
4059 2.57931347391605e-08
4060 2.35730670539169e-08
4061 2.7346768405323e-08
4062 3.19472022169975e-08
4063 2.78428213817961e-08
4064 2.64280810569062e-08
4065 2.90411854564354e-08
4066 2.74558669133285e-08
4067 2.56297045808651e-08
4068 2.76375313745802e-08
4069 3.58930627442078e-08
4070 2.40189752531705e-08
4071 2.32009291778468e-08
4072 2.16018616328029e-08
4073 2.82673831009106e-08
4074 2.94335436024085e-08
4075 2.63327439853356e-08
4076 2.87905432827529e-08
4077 3.49409035038661e-08
4078 2.57069157072465e-08
4079 3.50605091625766e-08
4080 3.1686472112824e-08
4081 2.57258108149472e-08
4082 3.11857988322117e-08
4083 3.252833025158e-08
4084 1.97812894953131e-08
4085 2.57502303924184e-08
4086 2.38795809792691e-08
4087 2.77373839452366e-08
4088 3.01675164848803e-08
4089 2.28041638905552e-08
4090 2.25625864658241e-08
4091 2.85003434186137e-08
4092 2.51881182578018e-08
4093 3.21315418716495e-08
4094 2.75985598818806e-08
4095 2.81273280222649e-08
4096 3.25326858785502e-08
4097 2.88787926905343e-08
4098 3.49163649104867e-08
4099 2.81430043713726e-08
4100 3.12714973915718e-08
4101 3.36439818227063e-08
4102 2.03882439819836e-08
4103 2.5327802077868e-08
4104 3.62312775337159e-08
4105 2.82359078340733e-08
4106 3.06450118614521e-08
4107 3.29892131389897e-08
4108 3.53462539237626e-08
4109 2.38323405454821e-08
4110 3.00312201773067e-08
4111 2.13247464131427e-08
4112 3.07049390357861e-08
4113 2.47402187625312e-08
4114 3.83245364332652e-08
4115 2.87405601540058e-08
4116 2.3743456978309e-08
4117 3.16945545364433e-08
4118 2.62017412211435e-08
4119 2.1751423773253e-08
4120 2.82757088854169e-08
4121 2.86613239808275e-08
4122 3.63311194462312e-08
4123 2.57602348341379e-08
4124 2.05518730922449e-08
4125 3.13891845848957e-08
4126 3.2337332811494e-08
4127 2.56759058459011e-08
4128 2.08638475385214e-08
4129 2.66573021434624e-08
4130 3.06035303765384e-08
4131 4.0965584702235e-08
4132 3.42427100008535e-08
4133 2.62651056459617e-08
4134 2.50575595828195e-08
4135 2.36244392937124e-08
4136 2.87024484180165e-08
4137 2.76355702766296e-08
4138 3.59797702742526e-08
4139 1.57333932548909e-08
4140 2.65742219340837e-08
4141 2.17073505837106e-08
4142 1.5255301022421e-08
4143 2.18681908137341e-08
4144 2.45490543449023e-08
4145 2.92070225782481e-08
4146 2.54774299435212e-08
4147 2.53516674320053e-08
4148 2.33159873630484e-08
4149 2.83061236672211e-08
4150 2.60259938045238e-08
4151 2.53516247994412e-08
4152 2.23222542672374e-08
4153 2.63085748741787e-08
4154 2.08357402442516e-08
4155 1.87972535314884e-08
4156 2.49457663414887e-08
4157 2.64398600791083e-08
4158 1.95217744192178e-08
4159 2.85850649817121e-08
4160 2.83711010240495e-08
4161 2.35435848594534e-08
4162 1.81311943237006e-08
4163 2.07830606058224e-08
4164 2.86044308239752e-08
4165 2.473163718264e-08
4166 2.01091179263813e-08
4167 2.42320172816335e-08
4168 2.34687842493031e-08
4169 2.39058195461439e-08
4170 2.04379553281342e-08
4171 2.33629844359484e-08
4172 2.07510186811533e-08
4173 3.05961229685181e-08
4174 2.90031287875081e-08
4175 3.08759844358519e-08
4176 2.94810469370077e-08
4177 1.87790174521751e-08
4178 2.24738645471234e-08
4179 2.25482494897733e-08
4180 2.27777263717144e-08
4181 2.66482267363699e-08
4182 2.83721597327258e-08
4183 2.22293330409684e-08
4184 2.24323155606498e-08
4185 2.2302891977688e-08
4186 1.82259949355057e-08
4187 2.05016039700467e-08
4188 2.22618439238431e-08
4189 2.29163585885317e-08
4190 2.47295748323495e-08
4191 1.99215364204974e-08
4192 2.51315857013878e-08
4193 3.10534957748132e-08
4194 2.44636719770597e-08
4195 2.26843983597291e-08
4196 2.59619739040318e-08
4197 2.29752163960484e-08
4198 2.71203717261415e-08
4199 3.08141423488451e-08
4200 3.10396650604616e-08
4201 1.69677534245238e-08
4202 2.23651497321953e-08
4203 1.79152213064526e-08
4204 1.94212308457509e-08
4205 2.77463474418482e-08
4206 2.97157942696913e-08
4207 1.71124163728109e-08
4208 2.69182294232451e-08
4209 2.76880101068855e-08
4210 2.29624976810783e-08
4211 2.20749143409193e-08
4212 2.53015617346364e-08
4213 3.03342027052622e-08
4214 2.40532109785363e-08
4215 2.45393714237707e-08
4216 2.34663826148562e-08
4217 2.46699798367445e-08
4218 2.28304379845667e-08
4219 2.50844944815753e-08
4220 2.42133673111766e-08
4221 2.87058448122934e-08
4222 2.29956960140498e-08
4223 2.33836630059159e-08
4224 2.1018072615675e-08
4225 2.30408296886253e-08
4226 2.69138293873539e-08
4227 2.7965640470029e-08
4228 2.60956856124039e-08
4229 1.92993088177218e-08
4230 2.47848923606853e-08
4231 2.50115927968864e-08
4232 2.46946605386711e-08
4233 2.58002579300864e-08
4234 2.06694359405901e-08
4235 1.86492137288496e-08
4236 2.55762948597749e-08
4237 2.39841977389688e-08
4238 2.67909694429136e-08
4239 1.91303879404359e-08
4240 2.7559146076328e-08
4241 2.63412047729616e-08
4242 2.24857963360137e-08
4243 2.65318380598956e-08
4244 2.80919607575925e-08
4245 2.45633238193932e-08
4246 3.41356489741429e-08
4247 2.50201761531343e-08
4248 2.08231529796876e-08
4249 2.31471588563181e-08
4250 1.51013370697228e-08
4251 2.15852864471344e-08
4252 2.15450093321579e-08
4253 2.66419064587353e-08
4254 2.89709287670803e-08
4255 2.20658566973952e-08
4256 2.31580283838184e-08
4257 2.57963534977534e-08
4258 2.38726443058113e-08
4259 2.67760462691058e-08
4260 2.01211296513293e-08
4261 2.44899389656439e-08
4262 2.12425224077606e-08
4263 2.63216879403672e-08
4264 2.05634904659746e-08
4265 3.24928421946424e-08
4266 2.04831316352738e-08
4267 3.35991678923619e-08
4268 3.06519574166941e-08
4269 2.41370869957791e-08
4270 2.19039684168365e-08
4271 2.14695781153296e-08
4272 1.81108177343958e-08
4273 1.79959567248034e-08
4274 1.93101730161516e-08
4275 2.00125569449483e-08
4276 2.32777903619308e-08
4277 2.2396886123488e-08
4278 2.14883080218442e-08
4279 1.91713169783725e-08
4280 2.02295762363747e-08
4281 2.29830323661417e-08
4282 2.3663167425525e-08
4283 2.68400714986683e-08
4284 2.30358878638981e-08
4285 2.86889854095307e-08
4286 2.36645334439345e-08
4287 1.91659204062944e-08
4288 1.83896844418996e-08
4289 1.78250036952932e-08
4290 2.75795244419896e-08
4291 2.9061022033261e-08
4292 2.00028331676094e-08
4293 1.85842505828759e-08
4294 2.7311495287563e-08
4295 2.41220092789263e-08
4296 2.35711201668209e-08
4297 2.27448531120444e-08
4298 2.27013163822676e-08
4299 2.23556426703908e-08
4300 2.64383164250148e-08
4301 2.76797056386613e-08
4302 1.97035650018051e-08
4303 1.51901726752612e-08
4304 1.92179090419131e-08
4305 2.59274983704927e-08
4306 1.61938196185929e-08
4307 1.9029036124607e-08
4308 2.49574529931351e-08
4309 2.93974444787182e-08
4310 2.80881451431014e-08
4311 2.09147152929745e-08
4312 2.49778633332198e-08
4313 2.62901398428994e-08
4314 2.08883577101915e-08
4315 2.21922977772238e-08
4316 2.230431839223e-08
4317 2.28594618789657e-08
4318 2.5210745491222e-08
4319 2.12847055536258e-08
4320 1.55940060864168e-08
4321 1.39927838205267e-08
4322 1.83802288944435e-08
4323 2.86576948838047e-08
4324 1.70990173131713e-08
4325 1.92795805986634e-08
4326 2.70922253520212e-08
4327 2.64438284602875e-08
4328 1.91534574867092e-08
4329 1.89577491482851e-08
4330 2.3531301351909e-08
4331 2.42292106378272e-08
4332 2.76576930247074e-08
4333 1.86787456613047e-08
4334 2.28662262458101e-08
4335 1.94863574165538e-08
4336 2.35540511539511e-08
4337 1.95312601647402e-08
4338 2.44023823370298e-08
4339 2.53872620703532e-08
4340 1.74762302407316e-08
4341 2.0340689133036e-08
4342 2.46111611090782e-08
4343 2.1207743117202e-08
4344 1.70711427216474e-08
4345 2.96155633350281e-08
4346 2.72398104073091e-08
4347 1.45774912141405e-08
4348 2.07235508753456e-08
4349 1.87214119762302e-08
4350 2.25929852604168e-08
4351 1.56773669601762e-08
4352 2.26210694620477e-08
4353 1.81446750957548e-08
4354 2.60052832601332e-08
4355 1.67342228962752e-08
4356 1.92033464685437e-08
4357 1.93499101186489e-08
4358 1.42309781736571e-08
4359 1.69376210834571e-08
4360 2.0436297987203e-08
4361 1.87544024754516e-08
4362 1.52304426848104e-08
4363 2.14579944923798e-08
4364 2.28206626928795e-08
4365 2.30193677452917e-08
4366 2.3230802170815e-08
4367 2.71944013974235e-08
4368 1.47357468449627e-08
4369 2.02354701883678e-08
4370 1.75212253594736e-08
4371 1.75027405902028e-08
4372 2.14505142537291e-08
4373 1.66182179128782e-08
4374 1.84037443062834e-08
4375 2.16145785714161e-08
4376 2.31710508558081e-08
4377 2.23781189134797e-08
4378 2.57939056780288e-08
4379 2.892098116547e-08
4380 3.32847704953565e-08
4381 2.27583658585218e-08
4382 1.73691958593736e-08
4383 2.16805045027968e-08
4384 2.47409719378311e-08
4385 3.7575613731633e-08
4386 2.15752091747845e-08
4387 1.63372124717398e-08
4388 1.67224261105048e-08
4389 1.61145052857137e-08
4390 2.58564742949829e-08
4391 2.1265668337378e-08
4392 1.9433274545122e-08
4393 2.06718571149622e-08
4394 3.25781179810747e-08
4395 2.31487256030505e-08
4396 2.17746229935756e-08
4397 2.70769824339823e-08
4398 1.9964589981214e-08
4399 2.22821334716627e-08
4400 2.38446435929518e-08
4401 1.88237780918143e-08
4402 2.38794886087135e-08
4403 2.5571027961746e-08
4404 2.19489670882922e-08
4405 2.2719882863953e-08
4406 1.75010885783422e-08
4407 2.75015175077442e-08
4408 1.85581061629136e-08
4409 1.89585787069291e-08
4410 2.27859455748103e-08
4411 2.20648175286442e-08
4412 1.88901871922553e-08
4413 2.5583837270915e-08
4414 1.92870217574637e-08
4415 2.21830642743726e-08
4416 2.41654731780727e-08
4417 2.65350461603475e-08
4418 3.80018789769565e-08
4419 2.33882921918394e-08
4420 1.79200814187652e-08
4421 2.66227999645707e-08
4422 2.40470061640963e-08
4423 2.45995366299212e-08
4424 2.05707433309499e-08
4425 1.81908639262929e-08
4426 1.916536085389e-08
4427 1.72063590042626e-08
4428 2.64452690856842e-08
4429 2.13494342204967e-08
4430 1.59756510242914e-08
4431 1.84063342345553e-08
4432 1.63755196069815e-08
4433 2.26199681208072e-08
4434 2.01294856339018e-08
4435 2.12696029677772e-08
4436 2.19015916513854e-08
4437 2.03759373818002e-08
4438 2.4178433477573e-08
4439 2.15162625494258e-08
4440 1.57356403462927e-08
4441 3.24607078994177e-08
4442 1.53317696316435e-08
4443 1.78026091646188e-08
4444 1.56722492761219e-08
4445 2.23052456505002e-08
4446 2.62320405397531e-08
4447 2.32933068389229e-08
4448 3.39334889076781e-08
4449 2.18654623296288e-08
4450 2.70547833025603e-08
4451 2.87941883669873e-08
4452 1.59117057307867e-08
4453 1.91396463122828e-08
4454 2.15341060538776e-08
4455 2.53457663745849e-08
4456 2.01979037939282e-08
4457 2.18021192210927e-08
4458 1.3222464900764e-08
4459 1.66298583792468e-08
4460 1.7985176015145e-08
4461 3.57550611340685e-08
4462 1.91299385221555e-08
4463 1.79727166482735e-08
4464 1.56353543445675e-08
4465 1.4690322736044e-08
4466 1.91394544657442e-08
4467 1.5637185768469e-08
4468 1.6859178941786e-08
4469 1.82550206062615e-08
4470 1.79769319430534e-08
4471 2.04285441895991e-08
4472 1.48827608015267e-08
4473 2.75342859623606e-08
4474 3.17039408059827e-08
4475 2.67202882042739e-08
4476 3.64231560467942e-08
4477 1.77095618170142e-08
4478 2.27176073508417e-08
4479 1.59100910224197e-08
4480 1.82749033683649e-08
4481 1.44890757169946e-08
4482 2.24542162641228e-08
4483 3.06384109194369e-08
4484 1.63739599656765e-08
4485 2.90426740434668e-08
4486 1.65396905060788e-08
4487 2.39577460092733e-08
4488 2.50444891491952e-08
4489 1.95189517882e-08
4490 2.27831815635682e-08
4491 2.28687646597336e-08
4492 2.43326923055065e-08
4493 1.62373581247266e-08
4494 2.68544049220054e-08
4495 2.17076987496512e-08
4496 1.88208595375272e-08
4497 1.86256858825118e-08
4498 1.545709338302e-08
4499 1.54678971853173e-08
4500 2.32507275654825e-08
4501 1.64239040145731e-08
4502 1.52575623246776e-08
4503 1.85179374057043e-08
4504 2.3127162407377e-08
4505 2.08005879187567e-08
4506 2.71894009529206e-08
4507 2.5825370286725e-08
4508 1.69777418790318e-08
4509 2.17427515991631e-08
4510 2.46561207006835e-08
4511 1.72511462892544e-08
4512 1.6270380598371e-08
4513 1.38663436288766e-08
4514 1.42073854902947e-08
4515 2.30389396449482e-08
4516 1.91799074400478e-08
4517 2.24922231950586e-08
4518 1.60689364037125e-08
4519 2.09918447069413e-08
4520 1.50446144431271e-08
4521 2.11489350476768e-08
4522 2.29007497409839e-08
4523 1.71167080509349e-08
4524 1.74928249663253e-08
4525 2.50308911375896e-08
4526 1.80883485967342e-08
4527 1.56295119069227e-08
4528 1.76360188675062e-08
4529 1.82974506657274e-08
4530 2.03107397567237e-08
4531 1.73564895789013e-08
4532 2.4230562445382e-08
4533 2.21570637393143e-08
4534 2.04123331570827e-08
4535 2.5094109901147e-08
4536 2.1540992989344e-08
4537 2.2464533344646e-08
4538 1.52678349962798e-08
4539 3.08118508485222e-08
4540 1.77056431738265e-08
4541 2.5105435952355e-08
4542 1.70731571103033e-08
4543 2.72299462977799e-08
4544 1.55503485643749e-08
4545 2.91149646614031e-08
4546 2.3261153003773e-08
4547 2.24462635145528e-08
4548 2.2710658242886e-08
4549 1.96955554088163e-08
4550 2.74461680049853e-08
4551 1.87321891331749e-08
4552 1.86382713707189e-08
4553 1.48160523849583e-08
4554 1.85295316867951e-08
4555 1.67912936888115e-08
4556 1.81804420407161e-08
4557 1.7313787736839e-08
4558 2.47121647589665e-08
4559 2.14986179969401e-08
4560 2.15537951930855e-08
4561 1.74986993783932e-08
4562 2.04170333972797e-08
4563 2.09441708420854e-08
4564 2.11304378439081e-08
4565 2.08447055172201e-08
4566 1.46327741035179e-08
4567 2.51721186117493e-08
4568 1.69840568275959e-08
4569 1.60726436604364e-08
4570 1.65079008240809e-08
4571 1.83712600687613e-08
4572 2.0271825107443e-08
4573 2.43903741647955e-08
4574 1.6148822723494e-08
4575 1.66585785166262e-08
4576 1.53930201918229e-08
4577 1.68817368972896e-08
4578 1.81709296498411e-08
4579 1.9189661415453e-08
4580 1.93567615269785e-08
4581 2.90815584946813e-08
4582 1.90272970712613e-08
4583 2.13031228213367e-08
4584 1.82983761476407e-08
4585 2.49981102484753e-08
4586 1.5788087281976e-08
4587 2.38932837959283e-08
4588 1.60980491159535e-08
4589 2.47897737892799e-08
4590 2.03790797570491e-08
4591 2.52663081568016e-08
4592 2.04561345640286e-08
4593 1.66164788595324e-08
4594 1.99927114863385e-08
4595 1.67094764691456e-08
4596 2.26743246400929e-08
4597 1.72446483759359e-08
4598 2.06319974438429e-08
4599 1.45337786250366e-08
4600 2.1388155246882e-08
4601 1.41278677645573e-08
4602 1.65779905358932e-08
4603 1.88760367336727e-08
4604 2.34222028439035e-08
4605 1.55956101366428e-08
4606 2.30949463997376e-08
4607 1.57351163210251e-08
4608 2.04942498527316e-08
4609 1.64395483892577e-08
4610 1.65661493412017e-08
4611 2.027138279459e-08
4612 1.54725441348091e-08
4613 2.29475496382747e-08
4614 1.92846147939463e-08
4615 1.95805416325356e-08
4616 2.19572822146574e-08
4617 2.38626913784401e-08
4618 3.03705576243374e-08
4619 2.32864838523028e-08
4620 2.62100030568035e-08
4621 2.03696099987383e-08
4622 2.4102467577336e-08
4623 2.23867999693539e-08
4624 2.1043247144803e-08
4625 2.42647715253952e-08
4626 1.65289364417731e-08
4627 1.58147237527828e-08
4628 1.69164966479229e-08
4629 2.46979716678197e-08
4630 1.95610141418001e-08
4631 2.49013947239973e-08
4632 1.89158324559457e-08
4633 1.87356850034348e-08
4634 1.78006143158882e-08
4635 2.76262177578701e-08
4636 1.5213380777368e-08
4637 1.9578690668709e-08
4638 1.57205377604441e-08
4639 1.86406072799628e-08
4640 1.82102439794107e-08
4641 2.5437458361921e-08
4642 1.51014099003532e-08
4643 1.70513452246723e-08
4644 2.37802915137308e-08
4645 1.89471016653897e-08
4646 2.00610958955849e-08
4647 1.62433124728523e-08
4648 1.84582003015521e-08
4649 1.90949496214898e-08
4650 1.8591306272242e-08
4651 2.0743831541381e-08
4652 2.40281377017482e-08
4653 1.83695529898387e-08
4654 1.66522671207758e-08
4655 1.92072011628852e-08
4656 2.32475851902336e-08
4657 1.88839202053259e-08
4658 1.84705690742248e-08
4659 2.1007686257235e-08
4660 2.13369553136999e-08
4661 2.64176254205495e-08
4662 2.02323793274672e-08
4663 1.51540238135794e-08
4664 1.46605954043366e-08
4665 1.49567451757093e-08
4666 1.94412574927583e-08
4667 1.54951216302379e-08
4668 1.70096381424401e-08
4669 2.40228406056531e-08
4670 1.85894357684901e-08
4671 2.59298982285827e-08
4672 1.81125905385215e-08
4673 2.25947278664762e-08
4674 2.20686384722057e-08
4675 2.26438228168035e-08
4676 2.28745538066732e-08
4677 1.85892208293126e-08
4678 2.15309086115667e-08
4679 1.79484196394242e-08
4680 1.66251741262613e-08
4681 1.56977328913399e-08
4682 2.33690649054097e-08
4683 1.96589766687794e-08
4684 2.23040359514926e-08
4685 1.93955624894215e-08
4686 1.78606551770599e-08
4687 2.0740536399444e-08
4688 2.06002948033301e-08
4689 2.76678324695467e-08
4690 1.58034527686368e-08
4691 1.88948590107429e-08
4692 2.0110746845603e-08
4693 2.05718801993271e-08
4694 1.60128443837948e-08
4695 1.56431827491588e-08
4696 1.49855523545739e-08
4697 2.02929939518981e-08
4698 2.32291661461659e-08
4699 1.65613851521584e-08
4700 1.51578127827179e-08
4701 1.51763632771917e-08
4702 1.69432006202896e-08
4703 1.49523717851707e-08
4704 1.48996956994552e-08
4705 1.45252094796433e-08
4706 1.48623007234505e-08
4707 1.76211916169677e-08
4708 2.17272493330256e-08
4709 2.0937903855156e-08
4710 1.65218487779839e-08
4711 1.81188060111026e-08
4712 1.55613584240655e-08
4713 1.79481016715499e-08
4714 1.43667371332867e-08
4715 1.48087222484605e-08
4716 1.65141660346535e-08
4717 1.84778734535485e-08
4718 1.55692401193619e-08
4719 2.10407016254521e-08
4720 2.19064144602044e-08
4721 1.51006549486965e-08
4722 2.30026451220056e-08
4723 2.73169629139147e-08
4724 1.2501694790501e-08
4725 1.85880644210101e-08
4726 1.86798025936241e-08
4727 1.76836039145201e-08
4728 1.93225044853307e-08
4729 1.41932723352056e-08
4730 1.37909035302641e-08
4731 1.55253019329393e-08
4732 1.2897150902802e-08
4733 1.88220408148254e-08
4734 1.55917092570235e-08
4735 1.50724126513069e-08
4736 2.11299759911299e-08
4737 1.85272561736838e-08
4738 1.68041651704698e-08
4739 1.77013728119846e-08
4740 2.61129819989492e-08
4741 1.74315797352165e-08
4742 1.5408581077736e-08
4743 1.97897112030887e-08
4744 2.27255227969181e-08
4745 2.33566588292433e-08
4746 2.03736547632616e-08
4747 1.59502295815628e-08
4748 1.55483501629305e-08
4749 1.51718957397406e-08
4750 2.22456453258246e-08
4751 1.86315531891523e-08
4752 2.14880309101773e-08
4753 1.34713298294287e-08
4754 1.5907190231701e-08
4755 2.15952553617171e-08
4756 2.18080931091436e-08
4757 1.33039481653441e-08
4758 1.6968039417975e-08
4759 1.82582517993524e-08
4760 2.00798115912448e-08
4761 1.84133845948509e-08
4762 1.77654122524018e-08
4763 2.07411172681304e-08
4764 2.32230732422067e-08
4765 1.44374281418891e-08
4766 1.80443020525445e-08
4767 1.73377614487435e-08
4768 1.34034081611389e-08
4769 1.28206130156627e-08
4770 2.13003019666758e-08
4771 1.7125611151414e-08
4772 1.59077924166695e-08
4773 1.28927402087697e-08
4774 1.24878916096804e-08
4775 1.50882382143891e-08
4776 1.47372505310273e-08
4777 1.89476594414373e-08
4778 1.94144682552633e-08
4779 1.87392021899768e-08
4780 1.84989890072984e-08
4781 1.43450389344935e-08
4782 1.46248657628689e-08
4783 1.69013194550871e-08
4784 1.82300023965354e-08
4785 2.03833465661774e-08
4786 1.59198734195343e-08
4787 2.16118980489455e-08
4788 1.41306513157247e-08
4789 1.78644903314762e-08
4790 1.4732082220803e-08
4791 1.29953701133445e-08
4792 3.54932510049366e-08
4793 1.7442657096467e-08
4794 1.50158268041878e-08
4795 1.34240734084301e-08
4796 1.87279063368351e-08
4797 1.41923255370102e-08
4798 1.33069937291452e-08
4799 1.4002957016146e-08
4800 1.32745254788347e-08
4801 1.5783884421694e-08
4802 1.57154307345309e-08
4803 1.52573260692179e-08
4804 1.63990598878172e-08
4805 2.43206610406332e-08
4806 1.60840993856937e-08
4807 1.81064780946372e-08
4808 1.56180011146034e-08
4809 1.47955150353596e-08
4810 1.68480180917641e-08
4811 1.34108555371881e-08
4812 1.50217704941724e-08
4813 1.56366599668445e-08
4814 1.86300272986273e-08
4815 1.38018867446021e-08
4816 1.3668219445151e-08
4817 1.50031524981387e-08
4818 1.83207760073856e-08
4819 1.37722251380978e-08
4820 1.83764985450807e-08
4821 1.90677518219218e-08
4822 1.37432420999062e-08
4823 1.74369958472198e-08
4824 1.61079150018395e-08
4825 1.54674992813852e-08
4826 1.44188971873405e-08
4827 1.73010761272963e-08
4828 1.25846977283572e-08
4829 2.38420998499578e-08
4830 2.3625217338008e-08
4831 1.58371999958717e-08
4832 1.77653358690577e-08
4833 1.89533615468918e-08
4834 1.95350171594555e-08
4835 1.38847990882596e-08
4836 1.47112491077905e-08
4837 1.42750602449837e-08
4838 1.6184859674695e-08
4839 1.44472371843563e-08
4840 1.49137768801211e-08
4841 1.52222305871419e-08
4842 1.49784948888509e-08
4843 1.95192519925058e-08
4844 2.26211831488854e-08
4845 1.77915389087957e-08
4846 1.53936525748577e-08
4847 1.91432096841027e-08
4848 1.4332861120181e-08
4849 1.45800518325245e-08
4850 1.71893219658159e-08
4851 1.35768658537927e-08
4852 1.51806123227516e-08
4853 1.54361501358835e-08
4854 1.78268582118335e-08
4855 2.1281573836518e-08
4856 2.2043671776828e-08
4857 2.25829097644237e-08
4858 1.96019538378778e-08
4859 2.04305621309686e-08
4860 1.89015700868822e-08
4861 1.66785945054926e-08
4862 1.81095920481766e-08
4863 1.48586636328218e-08
4864 1.48067620386882e-08
4865 1.28267094723356e-08
4866 1.44076635066881e-08
4867 1.28145938305124e-08
4868 1.77895085329283e-08
4869 1.77489773989237e-08
4870 2.54717456016351e-08
4871 2.42860167531944e-08
4872 1.83799553354902e-08
4873 1.36325457589237e-08
4874 1.82402057902209e-08
4875 1.50775925078506e-08
4876 1.27646426761885e-08
4877 2.06901642485491e-08
4878 1.76161929488217e-08
4879 1.83440125312018e-08
4880 1.60714357377856e-08
4881 1.37764422092346e-08
4882 1.48847973946431e-08
4883 1.42877487618875e-08
4884 1.50094034978565e-08
4885 1.53240531375332e-08
4886 1.69435487862302e-08
4887 1.58958091134309e-08
4888 1.26783161746857e-08
4889 1.20101093514791e-08
4890 1.39519826802825e-08
4891 1.80096524360351e-08
4892 1.66668616685683e-08
4893 1.29677317772803e-08
4894 1.37832945057426e-08
4895 1.55819197544815e-08
4896 1.24047110361403e-08
4897 1.68794063171163e-08
4898 1.38220297429825e-08
4899 1.81793389231188e-08
4900 1.479663325199e-08
4901 1.81760206885428e-08
4902 1.39917597508088e-08
4903 1.48380996378705e-08
4904 1.41153595478727e-08
4905 1.24174190929693e-08
4906 1.65964557652387e-08
4907 1.64582498740629e-08
4908 1.877550204199e-08
4909 1.70819713929404e-08
4910 1.48087897500204e-08
4911 1.20496901345746e-08
4912 1.37731239746586e-08
4913 1.9478180846022e-08
4914 1.84822113169503e-08
4915 1.38088669388026e-08
4916 2.07706900567928e-08
4917 1.31930546487524e-08
4918 1.70387828291041e-08
4919 1.69924856407988e-08
4920 2.1259197069412e-08
4921 2.40869351131323e-08
4922 1.42337688302518e-08
4923 1.50296877166056e-08
4924 1.40923512859104e-08
4925 1.70987153325086e-08
4926 1.43869876012559e-08
4927 1.605091348722e-08
4928 1.39049038949679e-08
4929 1.50599994697131e-08
4930 1.58677746497915e-08
4931 1.27955894768661e-08
4932 1.93751308330548e-08
4933 1.54090269433027e-08
4934 1.33371971244856e-08
4935 1.90824440693405e-08
4936 1.40383651370257e-08
4937 1.76671832718966e-08
4938 1.61285029776082e-08
4939 1.79930399468731e-08
4940 1.49436232277367e-08
4941 1.50497267981109e-08
4942 1.60375588365014e-08
4943 1.55315031946657e-08
4944 1.58806923167276e-08
4945 1.70112173236703e-08
4946 1.47223282453979e-08
4947 1.17013447820113e-08
4948 1.47780454540225e-08
4949 1.25776482562401e-08
4950 1.5238367012671e-08
4951 1.20805383474476e-08
4952 1.65846127941904e-08
4953 1.3733864712151e-08
4954 1.50894816641767e-08
4955 1.69870997268617e-08
4956 1.15642500020385e-08
4957 1.99809111478544e-08
4958 1.65279683272956e-08
4959 2.08821209213284e-08
4960 1.93847835561201e-08
4961 1.17951071132438e-08
4962 1.67087037539204e-08
4963 1.30084067961889e-08
4964 2.46315181584578e-08
4965 1.36282167773061e-08
4966 1.383921244269e-08
4967 1.59497730578551e-08
4968 1.56885828772602e-08
4969 1.51978305495959e-08
4970 1.44748177888232e-08
4971 1.93247000623842e-08
4972 1.45727030442799e-08
4973 1.50204115811903e-08
4974 1.69282117212788e-08
4975 1.71792038372587e-08
4976 1.58665436345018e-08
4977 2.02528784853939e-08
4978 1.55731303408402e-08
4979 1.52538675024516e-08
4980 1.94709670608972e-08
4981 1.77006320711826e-08
4982 1.42185614393497e-08
4983 1.20210339460414e-08
4984 1.45421443775717e-08
4985 1.40645557422658e-08
4986 1.29973747320378e-08
4987 1.87181470323594e-08
4988 1.38872513488764e-08
4989 1.24865424666609e-08
4990 1.9330951062102e-08
4991 1.67953597696169e-08
4992 1.81857942038732e-08
4993 1.64034830163473e-08
4994 1.44356890885433e-08
4995 1.71977863061556e-08
4996 1.24544499158219e-08
4997 1.3932941911321e-08
4998 1.33531230517292e-08
4999 1.29822925742928e-08
};
\addlegendentry{Test}
\end{groupplot}

\end{tikzpicture}

		% This file was created by tikzplotlib v0.9.6.
\begin{tikzpicture}

\begin{groupplot}[group style={group size=1 by 5},
legend cell align={left},
legend style={fill opacity=1, draw opacity=1, text opacity=1, draw=white},
log basis y={10},
tick align=outside,
tick pos=left,
title style={at={(0.3,0.85)},anchor=north},
x grid style={white!69.0196078431373!black},
xlabel={Epoch},
x label style={yshift=13pt},
xmin=-99.95, xmax=5098.95,
xtick style={color=black},
xtick = {0,1000,4000,5000},
y grid style={white!69.0196078431373!black},
ylabel={MSE Loss},
ymode=log,
ytick style={color=black},
width=0.45\textwidth,
height=0.25\textwidth
]
\nextgroupplot[
title={Batch Size 32 $\rare$},
ymin=1.05374705124308e-08, ymax=1e-05,
]
\addplot [semithick, black, dashed]
table {%
0 0.0185213608369231
1 0.0100256001874805
2 0.00511598332412541
3 0.00266714832093567
4 0.00179120110906661
5 0.00144799546990544
6 0.00114979036618024
7 0.000814766860101372
8 0.000510174112394452
9 0.000329686338198371
10 0.000244710172759369
11 0.000205625825270545
12 0.00018631382289459
13 0.00017530717747286
14 0.000167745494400151
15 0.000161573188117472
16 0.00015586399770109
17 0.000150204380508512
18 0.000144401938305236
19 0.000138238270650618
20 0.000130775678553618
21 0.000122665890172357
22 0.000114659473838401
23 0.000106363622428034
24 9.76855874760076e-05
25 8.86985533143161e-05
26 7.95836760516977e-05
27 7.0616739205434e-05
28 6.21391900349408e-05
29 5.44985102314968e-05
30 4.79656076204265e-05
31 4.26814938036841e-05
32 3.8612608499534e-05
33 3.55862038995838e-05
34 3.33759931600071e-05
35 3.17640865978319e-05
36 3.05771953862859e-05
37 2.96883334303857e-05
38 2.90059693143121e-05
39 2.84673240530537e-05
40 2.80267686393927e-05
41 2.76490723626921e-05
42 2.73102718856535e-05
43 2.69937950652093e-05
44 2.66956577979727e-05
45 2.64040889669559e-05
46 2.61062633799156e-05
47 2.57937593196402e-05
48 2.54627911053831e-05
49 2.51038547576172e-05
50 2.47119145278702e-05
51 2.42804027948296e-05
52 2.38050044499687e-05
53 2.32816320567508e-05
54 2.27040185272926e-05
55 2.20680549537065e-05
56 2.13706233989797e-05
57 2.06065907259472e-05
58 1.97811869220459e-05
59 1.88985617751314e-05
60 1.79618474503513e-05
61 1.69827113422798e-05
62 1.59747925827105e-05
63 1.49555455536756e-05
64 1.38275694916956e-05
65 1.26876792637631e-05
66 1.12698398916109e-05
67 9.53350929376029e-06
68 8.1114827280544e-06
69 7.06149710822501e-06
70 6.28023362514796e-06
71 5.682561220965e-06
72 5.20727496405016e-06
73 4.81536020743079e-06
74 4.48226160369813e-06
75 4.1903823175744e-06
76 3.92923815888935e-06
77 3.69496772054845e-06
78 3.48329872758768e-06
79 3.2906890992308e-06
80 3.11474296267988e-06
81 2.9551299094237e-06
82 2.80990815917903e-06
83 2.67591957708646e-06
84 2.55399901561759e-06
85 2.44380857793658e-06
86 2.34454597421063e-06
87 2.25512032693587e-06
88 2.17415749739303e-06
89 2.10023164117956e-06
90 2.03179113850638e-06
91 1.96943895070945e-06
92 1.91336066791337e-06
93 1.86234339207658e-06
94 1.8137314445994e-06
95 1.76814178485074e-06
96 1.72594649347957e-06
97 1.68744496795625e-06
98 1.65267439206218e-06
99 1.61943572766177e-06
100 1.58699234452797e-06
101 1.55465797615761e-06
102 1.52490673599459e-06
103 1.49749505953878e-06
104 1.47220658982405e-06
105 1.44839148697429e-06
106 1.42518704569738e-06
107 1.40295422079362e-06
108 1.38139719206265e-06
109 1.3604042615043e-06
110 1.3386478888151e-06
111 1.31467914638961e-06
112 1.29182604973721e-06
113 1.27053595815596e-06
114 1.25014606237528e-06
115 1.23000724124722e-06
116 1.21053629754897e-06
117 1.19191057501666e-06
118 1.17316421210489e-06
119 1.15501116124506e-06
120 1.13750963123493e-06
121 1.11990525147121e-06
122 1.10245057840075e-06
123 1.0850101084543e-06
124 1.06801857259597e-06
125 1.0507991025861e-06
126 1.0334132978187e-06
127 1.01728472532159e-06
128 1.00184256598368e-06
129 9.86175931302569e-07
130 9.71090967141208e-07
131 9.56330864255506e-07
132 9.41725935490467e-07
133 9.27916013552021e-07
134 9.14784782253264e-07
135 9.01841580116525e-07
136 8.89269085291744e-07
137 8.76967326348677e-07
138 8.64968807945843e-07
139 8.53698949640602e-07
140 8.42906854245484e-07
141 8.32554830026311e-07
142 8.22901245783214e-07
143 8.13463982467511e-07
144 8.04595268732555e-07
145 7.962967694084e-07
146 7.88325216944941e-07
147 7.80183357505848e-07
148 7.72475732674138e-07
149 7.65302009767765e-07
150 7.58297480047077e-07
151 7.51497822307101e-07
152 7.45269478329646e-07
153 7.39257797022219e-07
154 7.33223259203442e-07
155 7.27614483025718e-07
156 7.22271873542013e-07
157 7.17282795108076e-07
158 7.12139038910209e-07
159 7.07288850321675e-07
160 7.02631720855607e-07
161 6.97950020594362e-07
162 6.9341542928214e-07
163 6.88575328808838e-07
164 6.84098517467646e-07
165 6.79956147109806e-07
166 6.7576959975213e-07
167 6.71695563141839e-07
168 6.67876880356744e-07
169 6.63949073555159e-07
170 6.60169360230611e-07
171 6.56619765891264e-07
172 6.53260171930015e-07
173 6.4957598908677e-07
174 6.4607227579927e-07
175 6.42691795746941e-07
176 6.39468844383373e-07
177 6.3644487761394e-07
178 6.32937741329442e-07
179 6.29768981639245e-07
180 6.26748726972437e-07
181 6.23714342737003e-07
182 6.20784524357987e-07
183 6.17936972389543e-07
184 6.15162336544017e-07
185 6.1250354792719e-07
186 6.09875560485307e-07
187 6.07228826424944e-07
188 6.04710761308525e-07
189 6.02254500108756e-07
190 5.99862777107774e-07
191 5.97323293845875e-07
192 5.94940298924485e-07
193 5.92553270735152e-07
194 5.90278918821241e-07
195 5.87862195970956e-07
196 5.85565089295415e-07
197 5.83287933409338e-07
198 5.81123970619046e-07
199 5.7901030334051e-07
200 5.76958834130892e-07
201 5.74851887904515e-07
202 5.72869793018072e-07
203 5.70961808080028e-07
204 5.69018330452309e-07
205 5.67025770010332e-07
206 5.6522778049839e-07
207 5.63374698117514e-07
208 5.6155098252475e-07
209 5.59765112939203e-07
210 5.58035291533088e-07
211 5.56295385763406e-07
212 5.54577313891968e-07
213 5.52870644128234e-07
214 5.51226922539172e-07
215 5.4956966505415e-07
216 5.47952494571291e-07
217 5.46373169186154e-07
218 5.44931209105925e-07
219 5.43313873890838e-07
220 5.4181577536383e-07
221 5.4034543313719e-07
222 5.38908848966457e-07
223 5.37434068860421e-07
224 5.36108092092036e-07
225 5.34586550315908e-07
226 5.3321597795275e-07
227 5.31795008100744e-07
228 5.30406377151849e-07
229 5.28978781972e-07
230 5.27560962609641e-07
231 5.26288467426639e-07
232 5.24752812452789e-07
233 5.23475015484109e-07
234 5.2215384982901e-07
235 5.21034239568508e-07
236 5.19823385729978e-07
237 5.18555743155957e-07
238 5.17368580744915e-07
239 5.16210805130868e-07
240 5.15153198762164e-07
241 5.13930031502241e-07
242 5.12899209525131e-07
243 5.11791230678682e-07
244 5.10723405113822e-07
245 5.09490656327216e-07
246 5.08409917983954e-07
247 5.0734909473249e-07
248 5.06350299133373e-07
249 5.05293355104186e-07
250 5.04277433037714e-07
251 5.03256441447775e-07
252 5.02261668316351e-07
253 5.01283572816646e-07
254 5.00328452176291e-07
255 4.99379939583378e-07
256 4.98446030633204e-07
257 4.97528622531718e-07
258 4.96615409474543e-07
259 4.95723476319654e-07
260 4.94844554964402e-07
261 4.94011196337851e-07
262 4.93106419639844e-07
263 4.92229671294808e-07
264 4.91303239073204e-07
265 4.90439448412872e-07
266 4.89604005906585e-07
267 4.88780170257996e-07
268 4.87970775907343e-07
269 4.87166112407067e-07
270 4.86385979115767e-07
271 4.85587184130054e-07
272 4.84800064214141e-07
273 4.84025108221431e-07
274 4.83258642702822e-07
275 4.82502226532233e-07
276 4.81754507745791e-07
277 4.81013788430573e-07
278 4.80280547435541e-07
279 4.79572060612554e-07
280 4.78862895079146e-07
281 4.78152566870449e-07
282 4.77446612990207e-07
283 4.76746079471013e-07
284 4.76050876159206e-07
285 4.75361166536459e-07
286 4.74711966489849e-07
287 4.74005847763692e-07
288 4.73316154398162e-07
289 4.72505169341275e-07
290 4.7175724284898e-07
291 4.70991727524961e-07
292 4.70303329052513e-07
293 4.69632239514795e-07
294 4.68987740418925e-07
295 4.68347638616251e-07
296 4.67706041945348e-07
297 4.67066472197075e-07
298 4.66436172700924e-07
299 4.65819643181931e-07
300 4.6519587988314e-07
301 4.6458026486107e-07
302 4.63975161324015e-07
303 4.63367187876429e-07
304 4.62763844893743e-07
305 4.62163034285368e-07
306 4.61567032516541e-07
307 4.60977072975766e-07
308 4.60386529880452e-07
309 4.59811661130516e-07
310 4.59230181604653e-07
311 4.58641048680875e-07
312 4.58067391150507e-07
313 4.57471872664428e-07
314 4.56890578675484e-07
315 4.56316948998392e-07
316 4.5574583839425e-07
317 4.55179300502095e-07
318 4.54645132208498e-07
319 4.54078644793299e-07
320 4.53513786624171e-07
321 4.52951653983291e-07
322 4.52389646170559e-07
323 4.5183226654899e-07
324 4.51286291763608e-07
325 4.50746562023596e-07
326 4.50201638614089e-07
327 4.49617204708375e-07
328 4.49071586444916e-07
329 4.48541439027395e-07
330 4.48002679945603e-07
331 4.4747507934062e-07
332 4.46942362600566e-07
333 4.46416720819798e-07
334 4.45883008865167e-07
335 4.45353649183744e-07
336 4.44804065296012e-07
337 4.44134348526859e-07
338 4.43515091546942e-07
339 4.42925153379292e-07
340 4.42338806010412e-07
341 4.41806229218855e-07
342 4.41315731677605e-07
343 4.40803275751023e-07
344 4.40289753782963e-07
345 4.39783605429511e-07
346 4.39238003536957e-07
347 4.3873517472548e-07
348 4.38237527305318e-07
349 4.37745029216785e-07
350 4.37252352298856e-07
351 4.36761686955833e-07
352 4.3626986473555e-07
353 4.35827755836726e-07
354 4.35319205962514e-07
355 4.34815810649525e-07
356 4.3431166341179e-07
357 4.33812067626604e-07
358 4.33318690511442e-07
359 4.32825826123917e-07
360 4.32334809602253e-07
361 4.31852442034142e-07
362 4.313736782251e-07
363 4.30905241842083e-07
364 4.30419585995878e-07
365 4.29928817993641e-07
366 4.29442891004328e-07
367 4.28962666433108e-07
368 4.28483186738049e-07
369 4.28003588581305e-07
370 4.27527776309944e-07
371 4.27073076934903e-07
372 4.2659085897867e-07
373 4.26108542342263e-07
374 4.25632207566196e-07
375 4.25156665414761e-07
376 4.24683829919559e-07
377 4.24200477709746e-07
378 4.23732120452769e-07
379 4.23265454855937e-07
380 4.22798863610296e-07
381 4.22334155814497e-07
382 4.21870258719537e-07
383 4.21407641169935e-07
384 4.20946088411256e-07
385 4.20485005747651e-07
386 4.20026333245005e-07
387 4.1956940469845e-07
388 4.19112675444921e-07
389 4.18659692286383e-07
390 4.18207336338128e-07
391 4.17756915339851e-07
392 4.17323406850301e-07
393 4.16873820995534e-07
394 4.16422651255743e-07
395 4.15972308474011e-07
396 4.15522133266677e-07
397 4.15084301380375e-07
398 4.14641929637583e-07
399 4.14197380450787e-07
400 4.13755675594984e-07
401 4.13315258242619e-07
402 4.1287598139661e-07
403 4.12409738942188e-07
404 4.11963612350519e-07
405 4.11529805319333e-07
406 4.11095109598136e-07
407 4.10662509921167e-07
408 4.10231388173088e-07
409 4.09801475484528e-07
410 4.09372446483758e-07
411 4.08946104698771e-07
412 4.0852090558019e-07
413 4.08187630569046e-07
414 4.07759484915005e-07
415 4.07332115287318e-07
416 4.06905044656014e-07
417 4.06474459509809e-07
418 4.06050205583597e-07
419 4.05616316811574e-07
420 4.05190133619726e-07
421 4.04756621492197e-07
422 4.04271327170136e-07
423 4.03848127348283e-07
424 4.03422405724996e-07
425 4.02979115165181e-07
426 4.02551475531254e-07
427 4.02127053689583e-07
428 4.01698157247665e-07
429 4.01263217440828e-07
430 4.00836544145022e-07
431 4.00410393922357e-07
432 3.99995897510053e-07
433 3.99568458533395e-07
434 3.99168272679162e-07
435 3.98736000533972e-07
436 3.98306118427172e-07
437 3.97879050751726e-07
438 3.97453581626905e-07
439 3.97028946395039e-07
440 3.96590492982796e-07
441 3.96160542777579e-07
442 3.95696088276054e-07
443 3.95270011722459e-07
444 3.94841801153234e-07
445 3.94415229038714e-07
446 3.93989487974977e-07
447 3.93564666410384e-07
448 3.93140786513868e-07
449 3.92719500382555e-07
450 3.92298088854659e-07
451 3.91881866335098e-07
452 3.91460772220853e-07
453 3.9104061602302e-07
454 3.90616273648448e-07
455 3.90210867408314e-07
456 3.89778205999392e-07
457 3.89298790651083e-07
458 3.88835131161613e-07
459 3.88384121947638e-07
460 3.87940273867571e-07
461 3.87511125836681e-07
462 3.87079769438969e-07
463 3.86645365551885e-07
464 3.86214937293516e-07
465 3.85783523825012e-07
466 3.8535608462098e-07
467 3.84926936703778e-07
468 3.84500410177679e-07
469 3.84072783049305e-07
470 3.83657095426315e-07
471 3.83230365400777e-07
472 3.82805546109921e-07
473 3.82380842779639e-07
474 3.81945783487936e-07
475 3.81532733172207e-07
476 3.81126588649749e-07
477 3.80694610271348e-07
478 3.80268383310067e-07
479 3.79843750238251e-07
480 3.79418695047207e-07
481 3.78994493303253e-07
482 3.78573906004931e-07
483 3.78133138042358e-07
484 3.77648691141985e-07
485 3.7722791125816e-07
486 3.76804605707548e-07
487 3.763683781699e-07
488 3.75936130012633e-07
489 3.75502033762132e-07
490 3.75070941117883e-07
491 3.7463960597961e-07
492 3.74207424556516e-07
493 3.73759208173396e-07
494 3.73334394566882e-07
495 3.7290415264124e-07
496 3.72472547269354e-07
497 3.72039375974964e-07
498 3.71606638623234e-07
499 3.71173888765952e-07
500 3.70742516679456e-07
501 3.7031062333881e-07
502 3.69883992448194e-07
503 3.6946489217371e-07
504 3.69041602198195e-07
505 3.68620761605598e-07
506 3.68207438896206e-07
507 3.67786844662987e-07
508 3.67350584951964e-07
509 3.66932141787402e-07
510 3.66519052818148e-07
511 3.66103566079801e-07
512 3.65683795621408e-07
513 3.65263786193282e-07
514 3.64843802799442e-07
515 3.6441569204726e-07
516 3.6400681676696e-07
517 3.6358559054861e-07
518 3.63164501322899e-07
519 3.62746938435521e-07
520 3.6232749539522e-07
521 3.61904533406232e-07
522 3.6149247114281e-07
523 3.61073087901786e-07
524 3.60654856308429e-07
525 3.60233938749843e-07
526 3.59805903144661e-07
527 3.59397946226636e-07
528 3.58979113229907e-07
529 3.58570702360339e-07
530 3.58152379476451e-07
531 3.57733576265673e-07
532 3.57283286234633e-07
533 3.56865689241204e-07
534 3.56481844619339e-07
535 3.560531470157e-07
536 3.5562543416745e-07
537 3.55202099513008e-07
538 3.54767151975466e-07
539 3.54350218003674e-07
540 3.53868238562427e-07
541 3.53445645487227e-07
542 3.53023489537918e-07
543 3.52600569215156e-07
544 3.52181922892214e-07
545 3.51785656221182e-07
546 3.51369444388183e-07
547 3.50940250882559e-07
548 3.50511328520042e-07
549 3.50086427886254e-07
550 3.4966140447068e-07
551 3.49248210113728e-07
552 3.48822502417079e-07
553 3.48412709456625e-07
554 3.47991883700161e-07
555 3.47554684822171e-07
556 3.47131746480045e-07
557 3.46698975590698e-07
558 3.46266814119645e-07
559 3.45854583656546e-07
560 3.45416659229159e-07
561 3.44981989314874e-07
562 3.44532436656664e-07
563 3.44083417530783e-07
564 3.43583697201666e-07
565 3.43132300884008e-07
566 3.42692654442089e-07
567 3.42256293151877e-07
568 3.41824444376471e-07
569 3.41390641779071e-07
570 3.40952211388412e-07
571 3.40516456901696e-07
572 3.40081166200434e-07
573 3.3964617955462e-07
574 3.39211207858625e-07
575 3.38775180978246e-07
576 3.38345385387129e-07
577 3.37907625691969e-07
578 3.37475298749723e-07
579 3.37009300665159e-07
580 3.36573081881397e-07
581 3.36206419092377e-07
582 3.35764889769052e-07
583 3.35318149552677e-07
584 3.34874418626896e-07
585 3.34421750153524e-07
586 3.33988169757049e-07
587 3.33541937493465e-07
588 3.33096089946139e-07
589 3.32616425396282e-07
590 3.32171234106227e-07
591 3.31726922240705e-07
592 3.31282373906561e-07
593 3.30856286836934e-07
594 3.30414409745572e-07
595 3.2996329883872e-07
596 3.29514442512391e-07
597 3.2906655366105e-07
598 3.28620184347983e-07
599 3.28172313288633e-07
600 3.27754828333582e-07
601 3.27299891694111e-07
602 3.26791949646577e-07
603 3.26343840242771e-07
604 3.25866860180213e-07
605 3.25405265243717e-07
606 3.24946678460947e-07
607 3.24471016085681e-07
608 3.24006525602272e-07
609 3.2355661079464e-07
610 3.23067498300134e-07
611 3.22547068776657e-07
612 3.22036515171931e-07
613 3.215269095449e-07
614 3.21033148054539e-07
615 3.20562532010626e-07
616 3.20066749623038e-07
617 3.1958168347046e-07
618 3.19108507255805e-07
619 3.18618406140558e-07
620 3.18146146923937e-07
621 3.17659693337191e-07
622 3.17181050263571e-07
623 3.16679004583875e-07
624 3.16186531563289e-07
625 3.15694480377715e-07
626 3.15204478454234e-07
627 3.1471761570856e-07
628 3.14214815205105e-07
629 3.1374244105109e-07
630 3.13252027069666e-07
631 3.12825649984916e-07
632 3.12330473661859e-07
633 3.11826514632685e-07
634 3.11322462607677e-07
635 3.10854835333885e-07
636 3.10345695481828e-07
637 3.09930875232567e-07
638 3.09420084818157e-07
639 3.08922876001816e-07
640 3.08413442326128e-07
641 3.0791088335036e-07
642 3.07412373729221e-07
643 3.06920373247976e-07
644 3.06423529252697e-07
645 3.05929234343694e-07
646 3.05435574148305e-07
647 3.04946460801148e-07
648 3.04456118783492e-07
649 3.03963243084127e-07
650 3.03473363203466e-07
651 3.02987180361924e-07
652 3.02522311471876e-07
653 3.02038389065729e-07
654 3.01502559693745e-07
655 3.00952283851075e-07
656 3.00445974289687e-07
657 2.99952013904203e-07
658 2.99459923269296e-07
659 2.9896857085987e-07
660 2.98477365674898e-07
661 2.97985838358272e-07
662 2.97495124812031e-07
663 2.9700242060926e-07
664 2.9651086396143e-07
665 2.9601517366018e-07
666 2.95518898781211e-07
667 2.95013197558092e-07
668 2.94539917092607e-07
669 2.94100331927893e-07
670 2.93593171647899e-07
671 2.93133579702953e-07
672 2.92588175113906e-07
673 2.92073646960489e-07
674 2.91576083611744e-07
675 2.91068527246807e-07
676 2.90541520257648e-07
677 2.90048884949101e-07
678 2.89539593723021e-07
679 2.89038977484779e-07
680 2.88519004925547e-07
681 2.87999056524768e-07
682 2.87411364070067e-07
683 2.86895903968798e-07
684 2.86383701848081e-07
685 2.85874086500826e-07
686 2.85370644462546e-07
687 2.84865456762873e-07
688 2.84372104204067e-07
689 2.83867105338231e-07
690 2.83315091905934e-07
691 2.82804742880671e-07
692 2.82301618312886e-07
693 2.81788576046438e-07
694 2.8127672823075e-07
695 2.80772214523495e-07
696 2.8024275570715e-07
697 2.79723454298164e-07
698 2.79204135210875e-07
699 2.78686133867723e-07
700 2.7816904946576e-07
701 2.77635648217256e-07
702 2.77106846112929e-07
703 2.76564707121452e-07
704 2.76043478322663e-07
705 2.7552292931432e-07
706 2.74989282956994e-07
707 2.74451538984977e-07
708 2.73919823143842e-07
709 2.73383016121898e-07
710 2.7285828133472e-07
711 2.72334689384479e-07
712 2.71813517997543e-07
713 2.71297532890458e-07
714 2.70724480685658e-07
715 2.702090679918e-07
716 2.69609163979112e-07
717 2.69066292332809e-07
718 2.68534173642365e-07
719 2.6803095380501e-07
720 2.6750520493124e-07
721 2.66954143228304e-07
722 2.66426574967227e-07
723 2.65932182173856e-07
724 2.65378602591682e-07
725 2.64837616555269e-07
726 2.64279638940934e-07
727 2.63742836125402e-07
728 2.63196838830027e-07
729 2.62663483056258e-07
730 2.62165100934908e-07
731 2.6164052741251e-07
732 2.6108835766081e-07
733 2.60547652430887e-07
734 2.5985914373905e-07
735 2.59363373231736e-07
736 2.58823922308693e-07
737 2.5826781603655e-07
738 2.57730944269952e-07
739 2.57195841413704e-07
740 2.5667771035387e-07
741 2.56121474137672e-07
742 2.55567423778302e-07
743 2.54995225645871e-07
744 2.54450984641608e-07
745 2.53922574643184e-07
746 2.53376336218025e-07
747 2.52830342560628e-07
748 2.52296219173331e-07
749 2.51764227641615e-07
750 2.51235566565811e-07
751 2.50710121235898e-07
752 2.50186598123037e-07
753 2.49664393493276e-07
754 2.4915178619267e-07
755 2.48576101739673e-07
756 2.48021704862822e-07
757 2.47490895560531e-07
758 2.4698219971242e-07
759 2.46460666971871e-07
760 2.45941125115223e-07
761 2.45425631248963e-07
762 2.44873848430416e-07
763 2.44391138096489e-07
764 2.43867456845237e-07
765 2.433545328131e-07
766 2.42844925708141e-07
767 2.42337454267272e-07
768 2.41832610015535e-07
769 2.41330177800592e-07
770 2.40827523271037e-07
771 2.40326087464382e-07
772 2.39832782597205e-07
773 2.39340287123468e-07
774 2.38822613766843e-07
775 2.38318634671941e-07
776 2.37761325024621e-07
777 2.37361407641856e-07
778 2.36908274132475e-07
779 2.36416801953965e-07
780 2.35894517459201e-07
781 2.35417795465764e-07
782 2.34940998097954e-07
783 2.34431632520682e-07
784 2.33937869950296e-07
785 2.33457018566696e-07
786 2.32956179672783e-07
787 2.32458378945921e-07
788 2.31979477348432e-07
789 2.31492987097681e-07
790 2.31008329649285e-07
791 2.30512317102693e-07
792 2.30025402402134e-07
793 2.29548284664816e-07
794 2.29058705428997e-07
795 2.28590926042216e-07
796 2.28118926315801e-07
797 2.27657759864996e-07
798 2.27187208565738e-07
799 2.26727645724623e-07
800 2.26269467816564e-07
801 2.25824073538661e-07
802 2.25381281381942e-07
803 2.24889024195818e-07
804 2.24432982975031e-07
805 2.23897555315489e-07
806 2.23425031151692e-07
807 2.2295173977227e-07
808 2.22504936573387e-07
809 2.22031358589447e-07
810 2.21586511543137e-07
811 2.21129962255873e-07
812 2.20660490867886e-07
813 2.20224475640407e-07
814 2.19789697155193e-07
815 2.19333101028951e-07
816 2.18906489408255e-07
817 2.18447235852182e-07
818 2.1802253485248e-07
819 2.17587594136148e-07
820 2.171213056954e-07
821 2.16699075224369e-07
822 2.16260562240222e-07
823 2.15830427805486e-07
824 2.15395490897663e-07
825 2.14986567698361e-07
826 2.14559953121807e-07
827 2.14100065988987e-07
828 2.13742608934808e-07
829 2.13357985956009e-07
830 2.12944119368785e-07
831 2.12532104740149e-07
832 2.12122578318485e-07
833 2.11714633735482e-07
834 2.11308834479951e-07
835 2.10849997927198e-07
836 2.10445568654904e-07
837 2.10049146232905e-07
838 2.09655660114549e-07
839 2.09260511041975e-07
840 2.08855573191613e-07
841 2.08466095273252e-07
842 2.08078252569521e-07
843 2.07700746102546e-07
844 2.07313690111732e-07
845 2.06920507594077e-07
846 2.06544750881221e-07
847 2.06162186827896e-07
848 2.05777530311479e-07
849 2.05416958294791e-07
850 2.05037628063565e-07
851 2.04666853164781e-07
852 2.04303049542887e-07
853 2.03929313386197e-07
854 2.03553203959927e-07
855 2.03187631257151e-07
856 2.02841443922352e-07
857 2.02474036314015e-07
858 2.02113415184613e-07
859 2.01754178931424e-07
860 2.01373242305181e-07
861 2.01010403770852e-07
862 2.00656504603103e-07
863 2.00288501162049e-07
864 1.9992708035943e-07
865 1.99573282657184e-07
866 1.99245734393116e-07
867 1.98867363678801e-07
868 1.98505497905899e-07
869 1.98158252942449e-07
870 1.97831863943065e-07
871 1.97476405389807e-07
872 1.97114097829854e-07
873 1.96777017492877e-07
874 1.96421795067181e-07
875 1.96073604513458e-07
876 1.95731925714426e-07
877 1.95410672688467e-07
878 1.9507339879965e-07
879 1.94744070427078e-07
880 1.94403248713115e-07
881 1.9405795410421e-07
882 1.93761960872507e-07
883 1.93391343259464e-07
884 1.93013741949244e-07
885 1.92732023151621e-07
886 1.92379488197503e-07
887 1.92062848697105e-07
888 1.91666795728906e-07
889 1.91327077430969e-07
890 1.90992436387205e-07
891 1.90675125026019e-07
892 1.90355343164583e-07
893 1.90039261468655e-07
894 1.89738615972601e-07
895 1.89489817643107e-07
896 1.89161859964315e-07
897 1.88857485596827e-07
898 1.88550196980941e-07
899 1.88281318457939e-07
900 1.87998503605513e-07
901 1.87705945393191e-07
902 1.87365529825456e-07
903 1.87090701018633e-07
904 1.8682072081333e-07
905 1.86533131596889e-07
906 1.8617735247517e-07
907 1.85849786277004e-07
908 1.85555433830586e-07
909 1.85290518203374e-07
910 1.85007594780018e-07
911 1.84740976209241e-07
912 1.84413795096816e-07
913 1.84188078776515e-07
914 1.83867984986819e-07
915 1.83598712567345e-07
916 1.83321817303295e-07
917 1.83044159058454e-07
918 1.82777629504471e-07
919 1.82503174556814e-07
920 1.82209366499819e-07
921 1.82021597879611e-07
922 1.81733254407845e-07
923 1.81411800667775e-07
924 1.8115524892437e-07
925 1.80859339337758e-07
926 1.80594035271042e-07
927 1.8029719359447e-07
928 1.80034858402678e-07
929 1.79789717037693e-07
930 1.79509198858341e-07
931 1.79225189015142e-07
932 1.78970508443399e-07
933 1.7871346764764e-07
934 1.78445978065156e-07
935 1.78181604468364e-07
936 1.77948226280478e-07
937 1.77688481159066e-07
938 1.77433303264252e-07
939 1.77141203664632e-07
940 1.76903954496765e-07
941 1.76639763878939e-07
942 1.76401640089807e-07
943 1.76149772926237e-07
944 1.75987998716209e-07
945 1.75748525435893e-07
946 1.75482385913028e-07
947 1.75235935529372e-07
948 1.74978851646301e-07
949 1.7472245940553e-07
950 1.74479472093481e-07
951 1.74206433655399e-07
952 1.73980947380414e-07
953 1.7370674370909e-07
954 1.73465508026993e-07
955 1.73237816028404e-07
956 1.72998334960539e-07
957 1.72744316898843e-07
958 1.72513198037905e-07
959 1.72277799947551e-07
960 1.72013212107913e-07
961 1.71797714727973e-07
962 1.71535733926476e-07
963 1.71309136078435e-07
964 1.71067951129089e-07
965 1.70814558458687e-07
966 1.7058751026866e-07
967 1.70329375663414e-07
968 1.70064358030686e-07
969 1.69841669986681e-07
970 1.69645710855093e-07
971 1.69402572822719e-07
972 1.69113098081652e-07
973 1.68882366153866e-07
974 1.68665797360745e-07
975 1.6844378566816e-07
976 1.68193679371598e-07
977 1.679658043372e-07
978 1.67721343188987e-07
979 1.67482980003797e-07
980 1.67276143656636e-07
981 1.67045190551107e-07
982 1.66820757868891e-07
983 1.66584006336734e-07
984 1.66362110149976e-07
985 1.6610529868899e-07
986 1.65910131357805e-07
987 1.65689847761996e-07
988 1.65440177568144e-07
989 1.65233987956981e-07
990 1.65027243696159e-07
991 1.64787183962289e-07
992 1.64550278469733e-07
993 1.64329672372787e-07
994 1.641152552736e-07
995 1.63866262198553e-07
996 1.63681186563736e-07
997 1.63456090092495e-07
998 1.63221955062909e-07
999 1.63006642424079e-07
1000 1.62814612366446e-07
1001 1.62547091662191e-07
1002 1.6236365999589e-07
1003 1.62137027615472e-07
1004 1.61914140363706e-07
1005 1.61691737190495e-07
1006 1.61510571047074e-07
1007 1.61248653995472e-07
1008 1.6104978601561e-07
1009 1.60817439819994e-07
1010 1.60609903360864e-07
1011 1.60398300806719e-07
1012 1.60187364826925e-07
1013 1.59997554277425e-07
1014 1.59782047916224e-07
1015 1.59536282126282e-07
1016 1.59365163369785e-07
1017 1.59153467194528e-07
1018 1.5890577961386e-07
1019 1.58727719423268e-07
1020 1.58539938752256e-07
1021 1.58312121328663e-07
1022 1.58082366368717e-07
1023 1.5789587882864e-07
1024 1.57671714788421e-07
1025 1.57459437872376e-07
1026 1.57293676068093e-07
1027 1.57060246635865e-07
1028 1.56861209717363e-07
1029 1.56664727143152e-07
1030 1.56458571524354e-07
1031 1.56244335585143e-07
1032 1.56054935729344e-07
1033 1.55841487725183e-07
1034 1.55671341644847e-07
1035 1.55414295591072e-07
1036 1.55219929496297e-07
1037 1.5502700952652e-07
1038 1.54778249651599e-07
1039 1.54609648319592e-07
1040 1.54416214684261e-07
1041 1.54183319125423e-07
1042 1.54000696682033e-07
1043 1.5378528854626e-07
1044 1.53589135948096e-07
1045 1.5335699526986e-07
1046 1.53167287663791e-07
1047 1.52971405356084e-07
1048 1.52732748915696e-07
1049 1.52534419157746e-07
1050 1.52338159836063e-07
1051 1.52186203195015e-07
1052 1.51968354543897e-07
1053 1.51778432524452e-07
1054 1.51582191392663e-07
1055 1.51362850829173e-07
1056 1.51173687186201e-07
1057 1.51031862088757e-07
1058 1.50862339921787e-07
1059 1.50641980155797e-07
1060 1.50485465383099e-07
1061 1.50308139154731e-07
1062 1.50094176035509e-07
1063 1.49907080469802e-07
1064 1.49740256318864e-07
1065 1.49536210102497e-07
1066 1.49349157055667e-07
1067 1.49136879514344e-07
1068 1.48908395800618e-07
1069 1.48712377381344e-07
1070 1.48493782148762e-07
1071 1.4830978798841e-07
1072 1.48146137405547e-07
1073 1.47950765637006e-07
1074 1.47774900824515e-07
1075 1.47583466315382e-07
1076 1.4737144300625e-07
1077 1.47156897128298e-07
1078 1.46975856750942e-07
1079 1.4680339833717e-07
1080 1.46632637409994e-07
1081 1.46427892815382e-07
1082 1.46234485896457e-07
1083 1.46021229681992e-07
1084 1.45853940580309e-07
1085 1.45659838665324e-07
1086 1.45499808667182e-07
1087 1.45302000902348e-07
1088 1.45118163345614e-07
1089 1.44915564533221e-07
1090 1.44756839233651e-07
1091 1.44625436035994e-07
1092 1.44455908525742e-07
1093 1.4425951536623e-07
1094 1.44062650861088e-07
1095 1.43971858193481e-07
1096 1.43767995723465e-07
1097 1.43548971578866e-07
1098 1.4336696054329e-07
1099 1.43227281000691e-07
1100 1.42972960190946e-07
1101 1.42832374251611e-07
1102 1.42647749527214e-07
1103 1.42420790638198e-07
1104 1.42293767822821e-07
1105 1.42080506805087e-07
1106 1.41926133068182e-07
1107 1.41727618796494e-07
1108 1.41548562424987e-07
1109 1.4137656182811e-07
1110 1.41162390974614e-07
1111 1.41009764803357e-07
1112 1.40850409763971e-07
1113 1.40619346012727e-07
1114 1.40469351919137e-07
1115 1.40236711899888e-07
1116 1.40055121704563e-07
1117 1.39835729783044e-07
1118 1.39705407320889e-07
1119 1.39515229420795e-07
1120 1.39346126985629e-07
1121 1.3915980667889e-07
1122 1.3894486130539e-07
1123 1.38818959328546e-07
1124 1.38635964475498e-07
1125 1.38440284246144e-07
1126 1.38380431536689e-07
1127 1.38207009939606e-07
1128 1.3803130016754e-07
1129 1.37817676886698e-07
1130 1.37681640268283e-07
1131 1.3742020453833e-07
1132 1.37303870701544e-07
1133 1.37097208693149e-07
1134 1.3695340081199e-07
1135 1.36768997549552e-07
1136 1.36616483757734e-07
1137 1.36413310798389e-07
1138 1.36295977767986e-07
1139 1.36083534300724e-07
1140 1.35963834509312e-07
1141 1.357222713807e-07
1142 1.35606256634446e-07
1143 1.35436983526915e-07
1144 1.35217161329138e-07
1145 1.35080818665756e-07
1146 1.34906291776815e-07
1147 1.34764612568006e-07
1148 1.34538124029859e-07
1149 1.34437458484626e-07
1150 1.34265883701801e-07
1151 1.3410143142778e-07
1152 1.33920875498461e-07
1153 1.33777525007872e-07
1154 1.33588935767648e-07
1155 1.33419826369163e-07
1156 1.33264174451142e-07
1157 1.3310219384266e-07
1158 1.32939824624145e-07
1159 1.32826000481145e-07
1160 1.32648920214251e-07
1161 1.32496974060814e-07
1162 1.32340923499896e-07
1163 1.32131768481258e-07
1164 1.31964101143467e-07
1165 1.31850519750287e-07
1166 1.31695612139993e-07
1167 1.31505942960075e-07
1168 1.31346545458655e-07
1169 1.31156192026083e-07
1170 1.31067693899922e-07
1171 1.30887381260436e-07
1172 1.30729647679573e-07
1173 1.30576428858831e-07
1174 1.30391574771238e-07
1175 1.30241802793307e-07
1176 1.30092601807519e-07
1177 1.29940641812709e-07
1178 1.2978442197209e-07
1179 1.29702830406586e-07
1180 1.29564918552205e-07
1181 1.29329984446258e-07
1182 1.29192453158566e-07
1183 1.29019271810193e-07
1184 1.2888260789623e-07
1185 1.28742152895711e-07
1186 1.2855812948942e-07
1187 1.28389245219296e-07
1188 1.28260248743572e-07
1189 1.28133878774861e-07
1190 1.27946985173821e-07
1191 1.27757678200169e-07
1192 1.27662547413365e-07
1193 1.27471893478059e-07
1194 1.273243918547e-07
1195 1.27171785749169e-07
1196 1.27086931712483e-07
1197 1.26907419939926e-07
1198 1.26737666363397e-07
1199 1.26660317988581e-07
1200 1.26535082586088e-07
1201 1.26340583960882e-07
1202 1.2621424323811e-07
1203 1.2601926437128e-07
1204 1.25920169324445e-07
1205 1.25740624866921e-07
1206 1.2557388711798e-07
1207 1.25481804900573e-07
1208 1.25368563232087e-07
1209 1.25208813869904e-07
1210 1.25018988057946e-07
1211 1.24933488478973e-07
1212 1.24746758558558e-07
1213 1.24599515942236e-07
1214 1.24469573506758e-07
1215 1.24318066127671e-07
1216 1.24179299945126e-07
1217 1.24018012570559e-07
1218 1.23949695364445e-07
1219 1.23787564859867e-07
1220 1.23621690903519e-07
1221 1.23461992103557e-07
1222 1.23359578964255e-07
1223 1.23199914270344e-07
1224 1.23059030244121e-07
1225 1.22928636955066e-07
1226 1.22774252503177e-07
1227 1.22605700397571e-07
1228 1.2247761691242e-07
1229 1.22310282620219e-07
1230 1.22203671708121e-07
1231 1.22031712521675e-07
1232 1.21910517151491e-07
1233 1.21736889269641e-07
1234 1.21613467399584e-07
1235 1.214787870083e-07
1236 1.21305380901049e-07
1237 1.21206207609248e-07
1238 1.21023092845007e-07
1239 1.20945347987345e-07
1240 1.20883079858913e-07
1241 1.20674643767416e-07
1242 1.20576853333887e-07
1243 1.20447198781903e-07
1244 1.20319473182917e-07
1245 1.20158708739382e-07
1246 1.20045059986751e-07
1247 1.19856119056294e-07
1248 1.19752005417695e-07
1249 1.19630266766535e-07
1250 1.19465553751752e-07
1251 1.19370260136975e-07
1252 1.19239946116068e-07
1253 1.19060969154816e-07
1254 1.18954238587321e-07
1255 1.18861689969663e-07
1256 1.18681055795378e-07
1257 1.1854536361966e-07
1258 1.18406778511826e-07
1259 1.18261358295513e-07
1260 1.1812020741786e-07
1261 1.17996310450508e-07
1262 1.17843934617667e-07
1263 1.17783204274247e-07
1264 1.17580303879095e-07
1265 1.17537257636968e-07
1266 1.1735062625462e-07
1267 1.17211613513746e-07
1268 1.17074364794689e-07
1269 1.16957124220107e-07
1270 1.16827153647137e-07
1271 1.1669826719185e-07
1272 1.16565958165893e-07
1273 1.16455746024258e-07
1274 1.16320386638336e-07
1275 1.16225041551843e-07
1276 1.16092117281141e-07
1277 1.15950706572221e-07
1278 1.15833334689341e-07
1279 1.15700019279075e-07
1280 1.15587255180571e-07
1281 1.15437792004514e-07
1282 1.15330216146958e-07
1283 1.15189609118715e-07
1284 1.15066688010756e-07
1285 1.14926542011062e-07
1286 1.14846013190117e-07
1287 1.14713855282389e-07
1288 1.14599199065424e-07
1289 1.14470935528743e-07
1290 1.14359950686094e-07
1291 1.14234656109602e-07
1292 1.14117928518453e-07
1293 1.1398997003198e-07
1294 1.13820058118108e-07
1295 1.13818541620958e-07
1296 1.13339524574485e-07
1297 1.12663470900998e-07
1298 1.12361569904351e-07
1299 1.12089587332775e-07
1300 1.11964709418544e-07
1301 1.11820746525382e-07
1302 1.11636760721012e-07
1303 1.11483826003678e-07
1304 1.11383499302065e-07
1305 1.11267894283174e-07
1306 1.11115151270269e-07
1307 1.11017267045099e-07
1308 1.10896492941492e-07
1309 1.10744365827031e-07
1310 1.10644878901667e-07
1311 1.10550607274718e-07
1312 1.10358803596e-07
1313 1.10191754401967e-07
1314 1.10072820973528e-07
1315 1.09909369911065e-07
1316 1.09817545961732e-07
1317 1.09704210359496e-07
1318 1.0948728268545e-07
1319 1.09406620794061e-07
1320 1.09284726192982e-07
1321 1.09091099972147e-07
1322 1.09042478015908e-07
1323 1.08925033288187e-07
1324 1.08841563360329e-07
1325 1.08699135381585e-07
1326 1.08578928035286e-07
1327 1.08461759822376e-07
1328 1.08338431630273e-07
1329 1.08334709835844e-07
1330 1.08194711401666e-07
1331 1.08026963999919e-07
1332 1.07970406304503e-07
1333 1.07841701264988e-07
1334 1.07688680998308e-07
1335 1.07582052095267e-07
1336 1.07469422914619e-07
1337 1.07296660473821e-07
1338 1.07229844473977e-07
1339 1.07034789039062e-07
1340 1.06912480532628e-07
1341 1.06783147487022e-07
1342 1.066593461303e-07
1343 1.06509381225806e-07
1344 1.06420976919708e-07
1345 1.06206791770092e-07
1346 1.0609659514671e-07
1347 1.05980542883799e-07
1348 1.05862056386741e-07
1349 1.05813596292137e-07
1350 1.05752091627664e-07
1351 1.05628402849334e-07
1352 1.05501596806334e-07
1353 1.05356113238031e-07
1354 1.05215304216699e-07
1355 1.05107244252167e-07
1356 1.04985850612138e-07
1357 1.04835606094866e-07
1358 1.04752508718775e-07
1359 1.046250343677e-07
1360 1.04495485800271e-07
1361 1.04372938295683e-07
1362 1.04263034870655e-07
1363 1.04146862298649e-07
1364 1.04034263813446e-07
1365 1.03983576508426e-07
1366 1.0383213731302e-07
1367 1.03719245117873e-07
1368 1.03590368638606e-07
1369 1.03480423746305e-07
1370 1.03351360991155e-07
1371 1.03220049027186e-07
1372 1.0311302881405e-07
1373 1.02996904814745e-07
1374 1.02872726614578e-07
1375 1.02754415451045e-07
1376 1.02634036494464e-07
1377 1.02518056451117e-07
1378 1.02415923862509e-07
1379 1.02323679698202e-07
1380 1.02211516264106e-07
1381 1.02091620306055e-07
1382 1.01971102623111e-07
1383 1.01870679060312e-07
1384 1.0179658139009e-07
1385 1.01681100517226e-07
1386 1.01586428115752e-07
1387 1.01476204747541e-07
1388 1.01367837061161e-07
1389 1.01290550361455e-07
1390 1.01194307617902e-07
1391 1.01084147658526e-07
1392 1.00980223763258e-07
1393 1.0086355020178e-07
1394 1.00736980030547e-07
1395 1.00650129539304e-07
1396 1.00546723885486e-07
1397 1.00419138519214e-07
1398 1.00316875574435e-07
1399 1.0026016775555e-07
1400 1.00106740532624e-07
1401 9.9985386981416e-08
1402 9.98807984728955e-08
1403 9.97536856601755e-08
1404 9.9668220926219e-08
1405 9.95527654055195e-08
1406 9.94410838615067e-08
1407 9.93242042284237e-08
1408 9.9216212021247e-08
1409 9.91328911368328e-08
1410 9.90092147503674e-08
1411 9.89142205725102e-08
1412 9.87894828625713e-08
1413 9.87242921013376e-08
1414 9.86143573555864e-08
1415 9.84818851037517e-08
1416 9.83712935749281e-08
1417 9.82762960575201e-08
1418 9.81511430211412e-08
1419 9.80457570420867e-08
1420 9.78985418953471e-08
1421 9.7802210106579e-08
1422 9.75800401334936e-08
1423 9.73497928384859e-08
1424 9.72207337639475e-08
1425 9.71265016147527e-08
1426 9.70150914696433e-08
1427 9.68607581484093e-08
1428 9.68013227975462e-08
1429 9.67239543996357e-08
1430 9.66186112520973e-08
1431 9.65188904160641e-08
1432 9.6407868468873e-08
1433 9.62835954538832e-08
1434 9.61846360496566e-08
1435 9.60902012820952e-08
1436 9.59982494208589e-08
1437 9.59091606773654e-08
1438 9.57719880290142e-08
1439 9.56798667033354e-08
1440 9.55663077206736e-08
1441 9.54916833819652e-08
1442 9.53860347010505e-08
1443 9.52897061807789e-08
1444 9.51862622713406e-08
1445 9.50990866783741e-08
1446 9.49752749761501e-08
1447 9.48816515773387e-08
1448 9.47771617632043e-08
1449 9.4695990242144e-08
1450 9.46143253059972e-08
1451 9.45202227882191e-08
1452 9.44320654383546e-08
1453 9.43229522505362e-08
1454 9.42529038212569e-08
1455 9.41463178492086e-08
1456 9.4068416373716e-08
1457 9.39887819413343e-08
1458 9.38982673375222e-08
1459 9.37753829362009e-08
1460 9.37255089894506e-08
1461 9.36219205698308e-08
1462 9.35364635097358e-08
1463 9.34252245201606e-08
1464 9.33777980662853e-08
1465 9.32803416020533e-08
1466 9.31650765920722e-08
1467 9.30805050671779e-08
1468 9.29610439328599e-08
1469 9.287470193442e-08
1470 9.27757243687211e-08
1471 9.26921939878866e-08
1472 9.2576508720299e-08
1473 9.24887030322452e-08
1474 9.23843173552541e-08
1475 9.22476585287768e-08
1476 9.21249195897644e-08
1477 9.19396049852139e-08
1478 9.16524064678015e-08
1479 9.12750976311827e-08
1480 9.06833057285894e-08
1481 9.02523407830813e-08
1482 8.98370776809543e-08
1483 8.959618487836e-08
1484 8.94454999809113e-08
1485 8.92552902627131e-08
1486 8.91261259141629e-08
1487 8.89889859507775e-08
1488 8.88643282337398e-08
1489 8.87856272271392e-08
1490 8.86530855837009e-08
1491 8.85686529272789e-08
1492 8.84279600654736e-08
1493 8.83821760027104e-08
1494 8.82774851334034e-08
1495 8.8184456544127e-08
1496 8.80874322035652e-08
1497 8.79965566582541e-08
1498 8.79217438125579e-08
1499 8.78179770751331e-08
1500 8.77400387508942e-08
1501 8.76169939800775e-08
1502 8.75252724483744e-08
1503 8.74732476034978e-08
1504 8.73397060559e-08
1505 8.73137317967121e-08
1506 8.7211646601304e-08
1507 8.71337833956431e-08
1508 8.705234297679e-08
1509 8.69138714136852e-08
1510 8.68927885022686e-08
1511 8.67575567440326e-08
1512 8.6724243189451e-08
1513 8.66463673645512e-08
1514 8.65392750739602e-08
1515 8.64611098165824e-08
1516 8.63612686288207e-08
1517 8.63293309549817e-08
1518 8.62158176317962e-08
1519 8.61772485336587e-08
1520 8.60930073400823e-08
1521 8.60225565162409e-08
1522 8.59654042244529e-08
1523 8.5892122129394e-08
1524 8.58183232708143e-08
1525 8.57418458650727e-08
1526 8.56366906134554e-08
1527 8.55414217824091e-08
1528 8.54949409614392e-08
1529 8.54261305249793e-08
1530 8.53500751389902e-08
1531 8.52584533674872e-08
1532 8.51586839871743e-08
1533 8.51175963134665e-08
1534 8.50235133498245e-08
1535 8.49865759988688e-08
1536 8.48735124066025e-08
1537 8.48198776992604e-08
1538 8.47225467737189e-08
1539 8.46476027902554e-08
1540 8.45698268108208e-08
1541 8.44968632662813e-08
1542 8.44229106746752e-08
1543 8.43584369363271e-08
1544 8.43106598580334e-08
1545 8.42087358563504e-08
1546 8.41415686068103e-08
1547 8.40622797397828e-08
1548 8.39969525969764e-08
1549 8.39222875583801e-08
1550 8.38631996913364e-08
1551 8.37817928101003e-08
1552 8.37054291906725e-08
1553 8.36330925011453e-08
1554 8.35762737807499e-08
1555 8.35003280030833e-08
1556 8.34323542164839e-08
1557 8.33614796817983e-08
1558 8.32791213554174e-08
1559 8.32229494420744e-08
1560 8.31509793499663e-08
1561 8.30974982761745e-08
1562 8.30279069106155e-08
1563 8.29462467066833e-08
1564 8.28767359735139e-08
1565 8.28320653170067e-08
1566 8.27367432520987e-08
1567 8.26758389678162e-08
1568 8.26105227105245e-08
1569 8.25380142117638e-08
1570 8.24846440394822e-08
1571 8.24061097119966e-08
1572 8.23438409867094e-08
1573 8.22637429394035e-08
1574 8.22266813429451e-08
1575 8.2150413405202e-08
1576 8.20791995579384e-08
1577 8.20127622063183e-08
1578 8.19548766486378e-08
1579 8.18789175127677e-08
1580 8.18459514420056e-08
1581 8.17594470277072e-08
1582 8.16947884061392e-08
1583 8.16395185267993e-08
1584 8.15806382092887e-08
1585 8.14948606802091e-08
1586 8.14358932217374e-08
1587 8.13648371007503e-08
1588 8.13039597034049e-08
1589 8.12378415417925e-08
1590 8.11729484695434e-08
1591 8.11095378452364e-08
1592 8.10414111072078e-08
1593 8.09895950482087e-08
1594 8.09623106761137e-08
1595 8.09027162063103e-08
1596 8.08412271027237e-08
1597 8.07740352968267e-08
1598 8.0709158424952e-08
1599 8.06479694119844e-08
1600 8.05820829583581e-08
1601 8.05096885159173e-08
1602 8.04581248843306e-08
1603 8.0393222276598e-08
1604 8.03214767728377e-08
1605 8.02655540468322e-08
1606 8.02068767313813e-08
1607 8.01321103836017e-08
1608 8.00844711363879e-08
1609 8.00213825726814e-08
1610 7.99599952472363e-08
1611 7.98976112292849e-08
1612 7.9834806555823e-08
1613 7.97754659345173e-08
1614 7.97135347028188e-08
1615 7.96568865126801e-08
1616 7.95917305822513e-08
1617 7.95318744053475e-08
1618 7.94548566460662e-08
1619 7.94151298464385e-08
1620 7.93352987784601e-08
1621 7.92899142680881e-08
1622 7.92236218103426e-08
1623 7.91863368334589e-08
1624 7.91302377791681e-08
1625 7.90388985478785e-08
1626 7.90103603804937e-08
1627 7.89311625197797e-08
1628 7.88984124397985e-08
1629 7.88078053091112e-08
1630 7.87757863065508e-08
1631 7.86895295732393e-08
1632 7.86431169075286e-08
1633 7.85704401096154e-08
1634 7.8529926369697e-08
1635 7.84526377231032e-08
1636 7.84164920872854e-08
1637 7.83758315776595e-08
1638 7.82719817067346e-08
1639 7.82363932074759e-08
1640 7.81609860780463e-08
1641 7.81229463626687e-08
1642 7.80462926712744e-08
1643 7.80029670579552e-08
1644 7.79492562514861e-08
1645 7.7874848827264e-08
1646 7.78748980962973e-08
1647 7.78464856381333e-08
1648 7.77667935381032e-08
1649 7.7674670393435e-08
1650 7.76239029249837e-08
1651 7.75937119357195e-08
1652 7.74994474710411e-08
1653 7.74321212873019e-08
1654 7.74145367472556e-08
1655 7.73238617028937e-08
1656 7.7310983030543e-08
1657 7.73176963946298e-08
1658 7.7305709197617e-08
1659 7.72266385524745e-08
1660 7.71846794407338e-08
1661 7.71044600185178e-08
1662 7.70360870632203e-08
1663 7.6990116184561e-08
1664 7.68984523915606e-08
1665 7.68559426518323e-08
1666 7.67996079531485e-08
1667 7.67342406646776e-08
1668 7.67052079027053e-08
1669 7.66515564833981e-08
1670 7.65959734536636e-08
1671 7.65839691467818e-08
1672 7.65009739609468e-08
1673 7.63829222734103e-08
1674 7.63229548823574e-08
1675 7.6267942404229e-08
1676 7.61775889372984e-08
1677 7.61232539332468e-08
1678 7.60569423050583e-08
1679 7.60101576702255e-08
1680 7.59661880636031e-08
1681 7.58766135788846e-08
1682 7.58261601134791e-08
1683 7.57681564920176e-08
1684 7.57341910855303e-08
1685 7.56938010937347e-08
1686 7.55829074847725e-08
1687 7.55643320360377e-08
1688 7.54766571446908e-08
1689 7.54520423669192e-08
1690 7.53472286874057e-08
1691 7.53446218482168e-08
1692 7.52449624599194e-08
1693 7.5231898293282e-08
1694 7.51617080965161e-08
1695 7.5136856168001e-08
1696 7.50744397777225e-08
1697 7.49899303684742e-08
1698 7.49304649900751e-08
1699 7.48902372862403e-08
1700 7.47820001407717e-08
1701 7.47753556851194e-08
1702 7.46935666313675e-08
1703 7.46611515154427e-08
1704 7.45570163900311e-08
1705 7.45577602998537e-08
1706 7.4471433549661e-08
1707 7.44485144821283e-08
1708 7.43752279674936e-08
1709 7.43456927949637e-08
1710 7.42526671899668e-08
1711 7.42398117239418e-08
1712 7.41700458632977e-08
1713 7.41169396292207e-08
1714 7.40263327116963e-08
1715 7.39864196930284e-08
1716 7.39412575399001e-08
1717 7.38989007089685e-08
1718 7.38174102252742e-08
1719 7.37935158383607e-08
1720 7.37559998640336e-08
1721 7.36524492737089e-08
1722 7.36184399414697e-08
1723 7.35392518009803e-08
1724 7.35269673270977e-08
1725 7.34544109519675e-08
1726 7.34032248317362e-08
1727 7.33428511523471e-08
1728 7.32681853321537e-08
1729 7.32143738702007e-08
1730 7.31719510724815e-08
1731 7.31501556998637e-08
1732 7.30892445091058e-08
1733 7.30219066298332e-08
1734 7.2979406468221e-08
1735 7.29247632023089e-08
1736 7.28633535658219e-08
1737 7.28210725498002e-08
1738 7.27666082553924e-08
1739 7.27055496980711e-08
1740 7.26660435219628e-08
1741 7.26065471923221e-08
1742 7.25587729561994e-08
1743 7.25217219610386e-08
1744 7.24717678650677e-08
1745 7.24192664733891e-08
1746 7.23900978130132e-08
1747 7.23444256749417e-08
1748 7.22931122396631e-08
1749 7.22318920054477e-08
1750 7.21730701371826e-08
1751 7.21197674948826e-08
1752 7.2068223502697e-08
1753 7.20293178915199e-08
1754 7.19787946508177e-08
1755 7.1902128382817e-08
1756 7.186078650534e-08
1757 7.17925612576664e-08
1758 7.17526729800966e-08
1759 7.16827797475617e-08
1760 7.16302113090705e-08
1761 7.15883441131382e-08
1762 7.15355734399736e-08
1763 7.14979385350034e-08
1764 7.14462546795858e-08
1765 7.14090052440497e-08
1766 7.13444932216589e-08
1767 7.1314508303999e-08
1768 7.12945840461998e-08
1769 7.12300264069654e-08
1770 7.11618054367591e-08
1771 7.1129282360971e-08
1772 7.10771613370298e-08
1773 7.10353654227447e-08
1774 7.09667575193862e-08
1775 7.09453047704756e-08
1776 7.08791320533919e-08
1777 7.08689498480908e-08
1778 7.07894625264771e-08
1779 7.07654877487585e-08
1780 7.07449104453417e-08
1781 7.06691416922922e-08
1782 7.06064589621747e-08
1783 7.06071714091649e-08
1784 7.05371180771408e-08
1785 7.04618763336384e-08
1786 7.04391132160254e-08
1787 7.0436156150322e-08
1788 7.03822607022175e-08
1789 7.03128608421366e-08
1790 7.02480001280037e-08
1791 7.01905164817163e-08
1792 7.01403554188573e-08
1793 7.00816554370931e-08
1794 7.00428220454796e-08
1795 6.99959764602909e-08
1796 6.99281957992071e-08
1797 6.98799748306556e-08
1798 6.98345102421172e-08
1799 6.9812578843198e-08
1800 6.97628552615015e-08
1801 6.97126851321173e-08
1802 6.96650610905181e-08
1803 6.9620020497041e-08
1804 6.95732604896193e-08
1805 6.95269130375209e-08
1806 6.9450036392027e-08
1807 6.94258809090798e-08
1808 6.93804265097242e-08
1809 6.93312598656348e-08
1810 6.92662360961549e-08
1811 6.92461122326904e-08
1812 6.91778652424091e-08
1813 6.91592371993011e-08
1814 6.91084712798329e-08
1815 6.90440532196135e-08
1816 6.90230407087711e-08
1817 6.89588040501121e-08
1818 6.89371421600526e-08
1819 6.8864143017322e-08
1820 6.8837407681599e-08
1821 6.87807199710733e-08
1822 6.87425491889826e-08
1823 6.87316239549318e-08
1824 6.87130460335084e-08
1825 6.86086011967291e-08
1826 6.85816692680419e-08
1827 6.85487725604617e-08
1828 6.84762398179828e-08
1829 6.84496302625348e-08
1830 6.84361277762946e-08
1831 6.83638677543286e-08
1832 6.83591063932454e-08
1833 6.8299204102118e-08
1834 6.82733310384265e-08
1835 6.81935777322451e-08
1836 6.81807968021531e-08
1837 6.81156993778131e-08
1838 6.80841945381872e-08
1839 6.80452653654129e-08
1840 6.80089791416094e-08
1841 6.79751051819721e-08
1842 6.7968244835015e-08
1843 6.78748466498291e-08
1844 6.78418427497718e-08
1845 6.78073349718034e-08
1846 6.77318409998406e-08
1847 6.77308798628928e-08
1848 6.76905800105487e-08
1849 6.76790822211615e-08
1850 6.76080521344602e-08
1851 6.75728001056086e-08
1852 6.75344841027936e-08
1853 6.74922161749691e-08
1854 6.74847693602487e-08
1855 6.74052995464081e-08
1856 6.73843340308622e-08
1857 6.73418571182083e-08
1858 6.73255632506198e-08
1859 6.7246962132117e-08
1860 6.72148490679092e-08
1861 6.71952462028003e-08
1862 6.71176834146081e-08
1863 6.71036098367495e-08
1864 6.70586717745891e-08
1865 6.70325235319069e-08
1866 6.69660769716529e-08
1867 6.69481376718295e-08
1868 6.69079221182756e-08
1869 6.68658473159667e-08
1870 6.6817939952557e-08
1871 6.67913105019124e-08
1872 6.67606790472064e-08
1873 6.670499389827e-08
1874 6.66827096864608e-08
1875 6.66532561552913e-08
1876 6.65725077482193e-08
1877 6.65576889815611e-08
1878 6.65029085098468e-08
1879 6.64830166670072e-08
1880 6.6452839078579e-08
1881 6.64153293854497e-08
1882 6.63944830705532e-08
1883 6.63362933721601e-08
1884 6.63005359768931e-08
1885 6.62983337349487e-08
1886 6.62430377360579e-08
1887 6.61870239753171e-08
1888 6.61763318134945e-08
1889 6.61309613718686e-08
1890 6.60921866142417e-08
1891 6.60340205058674e-08
1892 6.60028475039098e-08
1893 6.59876501316603e-08
1894 6.5944907305493e-08
1895 6.58683348007116e-08
1896 6.58535005158001e-08
1897 6.58302000999811e-08
1898 6.57803167030124e-08
1899 6.57766341163324e-08
1900 6.57306995037743e-08
1901 6.56779970285015e-08
1902 6.56210140590474e-08
1903 6.56104001848234e-08
1904 6.55520660046705e-08
1905 6.55097281168082e-08
1906 6.54655234058055e-08
1907 6.54436386042789e-08
1908 6.5365522857519e-08
1909 6.53579310778696e-08
1910 6.53442751712419e-08
1911 6.5264507441043e-08
1912 6.5268909935412e-08
1913 6.51961480429009e-08
1914 6.51695983862055e-08
1915 6.50831466373347e-08
1916 6.50933851318314e-08
1917 6.50553775898288e-08
1918 6.50183242356661e-08
1919 6.49697049084352e-08
1920 6.48905779314646e-08
1921 6.48916117000908e-08
1922 6.48621363268376e-08
1923 6.48038537605089e-08
1924 6.47869810279644e-08
1925 6.47372211943775e-08
1926 6.47247042309118e-08
1927 6.46281489622424e-08
1928 6.46134867281489e-08
1929 6.46225764455721e-08
1930 6.45606906743978e-08
1931 6.45003795938237e-08
1932 6.44912326350777e-08
1933 6.44101307756273e-08
1934 6.43985000010616e-08
1935 6.44029893379638e-08
1936 6.43383797722663e-08
1937 6.42712544873802e-08
1938 6.42418095111452e-08
1939 6.42460130677591e-08
1940 6.41611594431879e-08
1941 6.41290721290488e-08
1942 6.41379059089786e-08
1943 6.40654033361443e-08
1944 6.40728545704405e-08
1945 6.39778109956524e-08
1946 6.39457395692489e-08
1947 6.39477572690339e-08
1948 6.39519293486046e-08
1949 6.39255639782732e-08
1950 6.38890929280933e-08
1951 6.38307860896248e-08
1952 6.37995794789958e-08
1953 6.37578965694274e-08
1954 6.37294341032657e-08
1955 6.36969363227991e-08
1956 6.36185869211658e-08
1957 6.36073982178686e-08
1958 6.36167952308142e-08
1959 6.35515753657501e-08
1960 6.35560674453472e-08
1961 6.34721468344424e-08
1962 6.34954503340168e-08
1963 6.34581862470895e-08
1964 6.3440140834814e-08
1965 6.33436748245231e-08
1966 6.33241977823218e-08
1967 6.3330045563248e-08
1968 6.32534731295209e-08
1969 6.32515762788444e-08
1970 6.31989108228481e-08
1971 6.31844320082564e-08
1972 6.30921162780851e-08
1973 6.30801136480841e-08
1974 6.30812256190438e-08
1975 6.29971758741021e-08
1976 6.2974930784776e-08
1977 6.29929236311e-08
1978 6.28992476805479e-08
1979 6.28779257283441e-08
1980 6.29304949057996e-08
1981 6.28546442129618e-08
1982 6.27929165233354e-08
1983 6.27738021989899e-08
1984 6.278494024059e-08
1985 6.27150396041998e-08
1986 6.26567076835727e-08
1987 6.26154294280923e-08
1988 6.26368192939708e-08
1989 6.26394451330725e-08
1990 6.25982872151099e-08
1991 6.25437579628851e-08
1992 6.25170877270875e-08
1993 6.24817048731074e-08
1994 6.24461865896819e-08
1995 6.24120992824828e-08
1996 6.23773715915377e-08
1997 6.23425795254207e-08
1998 6.23105942736402e-08
1999 6.2277998139848e-08
2000 6.22461903390104e-08
2001 6.22127373617332e-08
2002 6.21821952364598e-08
2003 6.2149363145636e-08
2004 6.21173484773863e-08
2005 6.20852950845574e-08
2006 6.20531317707673e-08
2007 6.20216508195881e-08
2008 6.19906964658412e-08
2009 6.1958614253399e-08
2010 6.19268154906649e-08
2011 6.18964157013124e-08
2012 6.18609249443125e-08
2013 6.18275104500299e-08
2014 6.18173643402997e-08
2015 6.17683831478644e-08
2016 6.17357134586882e-08
2017 6.17279402916893e-08
2018 6.16703573541599e-08
2019 6.16638588013529e-08
2020 6.16074237740349e-08
2021 6.16050630100062e-08
2022 6.1546931718226e-08
2023 6.1543627765559e-08
2024 6.14859119707489e-08
2025 6.14836635293159e-08
2026 6.14235252953677e-08
2027 6.14251977424374e-08
2028 6.13633359023424e-08
2029 6.13598864021014e-08
2030 6.13047321564864e-08
2031 6.1298523164055e-08
2032 6.124742975544e-08
2033 6.12394919130565e-08
2034 6.11869464819392e-08
2035 6.1178607197121e-08
2036 6.11275622617313e-08
2037 6.11171328301907e-08
2038 6.10669761869076e-08
2039 6.10576244923777e-08
2040 6.1005148452864e-08
2041 6.09954447554628e-08
2042 6.09439424010816e-08
2043 6.09357319234505e-08
2044 6.08808908566516e-08
2045 6.08685983962687e-08
2046 6.08216330419964e-08
2047 6.08090171709819e-08
2048 6.07619251553615e-08
2049 6.075195592814e-08
2050 6.07010631910043e-08
2051 6.06881004188153e-08
2052 6.06408254526514e-08
2053 6.06266131626398e-08
2054 6.05796112580492e-08
2055 6.0590814669581e-08
2056 6.05654850858173e-08
2057 6.05426475175364e-08
2058 6.04808317206107e-08
2059 6.04651365989639e-08
2060 6.04225408125103e-08
2061 6.03897549922294e-08
2062 6.03572617023929e-08
2063 6.03172002087149e-08
2064 6.03263562766188e-08
2065 6.02959871969233e-08
2066 6.02613738607261e-08
2067 6.02281989756648e-08
2068 6.02043586326317e-08
2069 6.01709518264215e-08
2070 6.01381010056912e-08
2071 6.01057412978889e-08
2072 6.00736400144797e-08
2073 6.00422240637499e-08
2074 6.00106283172863e-08
2075 6.00027021420146e-08
2076 5.99716185405441e-08
2077 5.99382256751824e-08
2078 5.9930828058441e-08
2079 5.99099485043553e-08
2080 5.98730123613223e-08
2081 5.98383643648503e-08
2082 5.98075041153834e-08
2083 5.97959699604189e-08
2084 5.97496516832052e-08
2085 5.97421506540741e-08
2086 5.96924957108058e-08
2087 5.96646969768244e-08
2088 5.96312347624917e-08
2089 5.95999002541703e-08
2090 5.9569816770022e-08
2091 5.95344718732349e-08
2092 5.95126554117087e-08
2093 5.94837991201302e-08
2094 5.94516339731399e-08
2095 5.9421772689916e-08
2096 5.93990174735382e-08
2097 5.93778387383281e-08
2098 5.93458914437406e-08
2099 5.93124394754341e-08
2100 5.92817038835847e-08
2101 5.92539787191981e-08
2102 5.92198925915e-08
2103 5.91904924931441e-08
2104 5.91592163772248e-08
2105 5.9132834024922e-08
2106 5.90999805325509e-08
2107 5.90700926039744e-08
2108 5.90375228881612e-08
2109 5.90114405838449e-08
2110 5.8975464270361e-08
2111 5.89421195797968e-08
2112 5.89139560673857e-08
2113 5.88773572900436e-08
2114 5.88281711912941e-08
2115 5.88023901428869e-08
2116 5.87553580970734e-08
2117 5.87240764815533e-08
2118 5.8700576559545e-08
2119 5.86723750046758e-08
2120 5.86520583567562e-08
2121 5.86210052944125e-08
2122 5.85963523036526e-08
2123 5.85663352978827e-08
2124 5.85366293961442e-08
2125 5.85030698658784e-08
2126 5.8482686199568e-08
2127 5.845026454665e-08
2128 5.84260956344451e-08
2129 5.83999301113636e-08
2130 5.83638958318033e-08
2131 5.83349919338616e-08
2132 5.83079842897405e-08
2133 5.83255651775971e-08
2134 5.82979676835294e-08
2135 5.82581641310753e-08
2136 5.82281556091857e-08
2137 5.81862352930784e-08
2138 5.81692256247379e-08
2139 5.81248189774897e-08
2140 5.81089617810449e-08
2141 5.80547872743864e-08
2142 5.80393352578312e-08
2143 5.79973278433954e-08
2144 5.79827605804439e-08
2145 5.79418344699434e-08
2146 5.79210172730882e-08
2147 5.78791718055527e-08
2148 5.78561989073023e-08
2149 5.78237141155569e-08
2150 5.77968961437136e-08
2151 5.77617138333153e-08
2152 5.77346312695681e-08
2153 5.77033215165557e-08
2154 5.7675280714875e-08
2155 5.76474446489783e-08
2156 5.7611468477603e-08
2157 5.75883159967816e-08
2158 5.75629800891875e-08
2159 5.75305862469122e-08
2160 5.74885371378286e-08
2161 5.74628412408629e-08
2162 5.74148827325871e-08
2163 5.73684654341378e-08
2164 5.73362525813081e-08
2165 5.73060245017132e-08
2166 5.72897094031077e-08
2167 5.72569111767507e-08
2168 5.72249473407283e-08
2169 5.71929760582179e-08
2170 5.71620881828494e-08
2171 5.71314519390853e-08
2172 5.70993541657572e-08
2173 5.70694985242426e-08
2174 5.70399189570026e-08
2175 5.70097678860293e-08
2176 5.69844155222654e-08
2177 5.69534986993858e-08
2178 5.69238088132806e-08
2179 5.69004774888526e-08
2180 5.68702275103305e-08
2181 5.68380102237143e-08
2182 5.68067609094669e-08
2183 5.67785446321523e-08
2184 5.67521175867114e-08
2185 5.67200614653984e-08
2186 5.66879774481777e-08
2187 5.66613684185313e-08
2188 5.66398348951225e-08
2189 5.66087342832589e-08
2190 5.65760084896283e-08
2191 5.65476991454261e-08
2192 5.65186091847636e-08
2193 5.64934620399526e-08
2194 5.64675011247573e-08
2195 5.644228554047e-08
2196 5.64070904687242e-08
2197 5.6381275669537e-08
2198 5.63543352853912e-08
2199 5.63281861474252e-08
2200 5.62995095378938e-08
2201 5.6274239909726e-08
2202 5.62470820000272e-08
2203 5.62188356383331e-08
2204 5.6193502857127e-08
2205 5.61675762469349e-08
2206 5.61421054783295e-08
2207 5.61182778113789e-08
2208 5.60900593455926e-08
2209 5.60619199916346e-08
2210 5.60255562476186e-08
2211 5.59999451610338e-08
2212 5.59675503666313e-08
2213 5.59393063923608e-08
2214 5.59170821787802e-08
2215 5.58857626771214e-08
2216 5.58612730259256e-08
2217 5.58395199732331e-08
2218 5.58073723340158e-08
2219 5.57874459445884e-08
2220 5.57769465956426e-08
2221 5.57228337498827e-08
2222 5.56932525626053e-08
2223 5.56712606680776e-08
2224 5.56417176937884e-08
2225 5.56043607105039e-08
2226 5.55671280011438e-08
2227 5.55463118274702e-08
2228 5.55174466398967e-08
2229 5.54880128618152e-08
2230 5.54602056297426e-08
2231 5.54390986593489e-08
2232 5.54112100275006e-08
2233 5.53895747685829e-08
2234 5.53604678401598e-08
2235 5.53377834933144e-08
2236 5.53138184642421e-08
2237 5.52895072871706e-08
2238 5.5309207240839e-08
2239 5.52338280641607e-08
2240 5.52076225517339e-08
2241 5.51849741015076e-08
2242 5.5158525213983e-08
2243 5.51328407993879e-08
2244 5.51089436129359e-08
2245 5.50848163527462e-08
2246 5.50599461632828e-08
2247 5.50344777536793e-08
2248 5.50097749396627e-08
2249 5.49839805188412e-08
2250 5.49585503932803e-08
2251 5.49341108637691e-08
2252 5.49094793598215e-08
2253 5.48849489092618e-08
2254 5.48610987181064e-08
2255 5.48355443754644e-08
2256 5.48114287681756e-08
2257 5.47870539122641e-08
2258 5.47622884568e-08
2259 5.4737745898592e-08
2260 5.47136313144847e-08
2261 5.46879182365956e-08
2262 5.46594948218626e-08
2263 5.46381439221477e-08
2264 5.46611908873729e-08
2265 5.45826493976165e-08
2266 5.4557064487426e-08
2267 5.45368490492137e-08
2268 5.45103498978961e-08
2269 5.44887604974065e-08
2270 5.44630529191181e-08
2271 5.44407142655245e-08
2272 5.44164786902002e-08
2273 5.43926400951023e-08
2274 5.43692594874301e-08
2275 5.43442478573297e-08
2276 5.43211097436824e-08
2277 5.42975720634331e-08
2278 5.42668261829249e-08
2279 5.42432729417897e-08
2280 5.42201152171629e-08
2281 5.4192620339677e-08
2282 5.41733479479944e-08
2283 5.41478344260327e-08
2284 5.41192693717107e-08
2285 5.40948535672214e-08
2286 5.40710082219675e-08
2287 5.40479412975969e-08
2288 5.40284910712785e-08
2289 5.40066533289973e-08
2290 5.3981616609633e-08
2291 5.3958577993285e-08
2292 5.39472287002241e-08
2293 5.39702509954054e-08
2294 5.38833141874306e-08
2295 5.38469747581871e-08
2296 5.38259265070451e-08
2297 5.38046774849477e-08
2298 5.37837739500446e-08
2299 5.3758324973785e-08
2300 5.37310326720331e-08
2301 5.37103558926333e-08
2302 5.36839241789266e-08
2303 5.3659915316473e-08
2304 5.363479341014e-08
2305 5.36096340795211e-08
2306 5.35862739781123e-08
2307 5.35634372624827e-08
2308 5.35402283574626e-08
2309 5.35125736291775e-08
2310 5.34900339985711e-08
2311 5.34524641651046e-08
2312 5.34272319825391e-08
2313 5.34011343162888e-08
2314 5.33784600307285e-08
2315 5.33529870736515e-08
2316 5.33292425259901e-08
2317 5.33024389142156e-08
2318 5.32797142227537e-08
2319 5.32673468498501e-08
2320 5.3241777756341e-08
2321 5.32169946438898e-08
2322 5.32001309281327e-08
2323 5.31747396834703e-08
2324 5.31492992124072e-08
2325 5.31191143480214e-08
2326 5.30887879435227e-08
2327 5.30652246837349e-08
2328 5.30427696858737e-08
2329 5.30209634277412e-08
2330 5.2994308511245e-08
2331 5.29940913054361e-08
2332 5.30017497339941e-08
2333 5.29160146882646e-08
2334 5.2900385760779e-08
2335 5.29354027065665e-08
2336 5.28621932787132e-08
2337 5.28393319925158e-08
2338 5.28329576710007e-08
2339 5.2803487676556e-08
2340 5.27797820382148e-08
2341 5.27581060154603e-08
2342 5.27893031474491e-08
2343 5.2751232033188e-08
2344 5.26682941455192e-08
2345 5.2651605720655e-08
2346 5.26790588111226e-08
2347 5.26532650155787e-08
2348 5.26227773960386e-08
2349 5.25388971439611e-08
2350 5.25217370395126e-08
2351 5.24930901448783e-08
2352 5.24712954685924e-08
2353 5.24370233776494e-08
2354 5.24492368256801e-08
2355 5.24341462337929e-08
2356 5.24879223604557e-08
2357 5.24429051012021e-08
2358 5.23589332601659e-08
2359 5.23884398475616e-08
2360 5.24156020134114e-08
2361 5.23645213448276e-08
2362 5.22784227285911e-08
2363 5.22559460875982e-08
2364 5.22322851708168e-08
2365 5.22135751950259e-08
2366 5.22226000825299e-08
2367 5.21576795904366e-08
2368 5.2108217523994e-08
2369 5.20880798475787e-08
2370 5.20893634359254e-08
2371 5.20753084529701e-08
2372 5.20496626492672e-08
2373 5.20896342308674e-08
2374 5.20460975437231e-08
2375 5.20061106783487e-08
2376 5.19895802710835e-08
2377 5.19595369965486e-08
2378 5.18726792151369e-08
2379 5.18506605189373e-08
2380 5.18312427360001e-08
2381 5.18078889015783e-08
2382 5.1786883425109e-08
2383 5.17648865354659e-08
2384 5.17431741045016e-08
2385 5.17201288587898e-08
2386 5.17072507406624e-08
2387 5.16847418552402e-08
2388 5.16658670264292e-08
2389 5.16796673011299e-08
2390 5.16747518091165e-08
2391 5.16523813587355e-08
2392 5.16070973333171e-08
2393 5.15772757054833e-08
2394 5.15547084916079e-08
2395 5.14739723271873e-08
2396 5.14575380634597e-08
2397 5.14430300952995e-08
2398 5.14238462798744e-08
2399 5.14012053045576e-08
2400 5.13812144546932e-08
2401 5.13595555631241e-08
2402 5.13368483936461e-08
2403 5.13152900296632e-08
2404 5.12904956408988e-08
2405 5.12859767738405e-08
2406 5.12864031279037e-08
2407 5.12741133746886e-08
2408 5.12484853985029e-08
2409 5.11881224483091e-08
2410 5.11605333528564e-08
2411 5.11412373640496e-08
2412 5.11242890155472e-08
2413 5.10580763446455e-08
2414 5.10422022585999e-08
2415 5.1021262613915e-08
2416 5.10017917250138e-08
2417 5.09783831716959e-08
2418 5.09495943816773e-08
2419 5.0926697234388e-08
2420 5.09021244852192e-08
2421 5.08941419141706e-08
2422 5.08636651872507e-08
2423 5.08361664799395e-08
2424 5.0826551529326e-08
2425 5.08552678155638e-08
2426 5.07902592303822e-08
2427 5.07894599337533e-08
2428 5.07542686705165e-08
2429 5.06899608154754e-08
2430 5.07001850706956e-08
2431 5.06617058917413e-08
2432 5.06358982903521e-08
2433 5.05561707200286e-08
2434 5.05353068405157e-08
2435 5.05127491692292e-08
2436 5.0488929289827e-08
2437 5.04646149579457e-08
2438 5.04382225017253e-08
2439 5.04155458287414e-08
2440 5.03915942005051e-08
2441 5.03688402488933e-08
2442 5.03601074441917e-08
2443 5.03938501168477e-08
2444 5.03578180328645e-08
2445 5.03318347071513e-08
2446 5.03011151806731e-08
2447 5.02561212059049e-08
2448 5.02269110782549e-08
2449 5.01940734736195e-08
2450 5.01237616177264e-08
2451 5.01292328323188e-08
2452 5.00894308501643e-08
2453 4.99955365640403e-08
2454 4.99798094324433e-08
2455 4.99626190872959e-08
2456 4.99442521828541e-08
2457 4.99258378994227e-08
2458 4.99061377610133e-08
2459 4.98842099716512e-08
2460 4.98649811220275e-08
2461 4.98419346399714e-08
2462 4.98222857814312e-08
2463 4.98002944055997e-08
2464 4.97799984842118e-08
2465 4.97568928210512e-08
2466 4.97341661755968e-08
2467 4.97147862361658e-08
2468 4.96899604627288e-08
2469 4.96690425109136e-08
2470 4.96193078589613e-08
2471 4.95958824018317e-08
2472 4.95998455534163e-08
2473 4.96598748114252e-08
2474 4.96232796152185e-08
2475 4.95919710843395e-08
2476 4.95675161928943e-08
2477 4.95410535279461e-08
2478 4.95178578461264e-08
2479 4.94908368224856e-08
2480 4.94677099709406e-08
2481 4.94488908699964e-08
2482 4.94397475065966e-08
2483 4.9392762328182e-08
2484 4.93809350601282e-08
2485 4.9356890109209e-08
2486 4.93344661833817e-08
2487 4.93133994083905e-08
2488 4.9296750383121e-08
2489 4.92729934862268e-08
2490 4.92536026825974e-08
2491 4.92309012329883e-08
2492 4.92131448339705e-08
2493 4.91880869972761e-08
2494 4.91617636839692e-08
2495 4.91408934379933e-08
2496 4.91217928981769e-08
2497 4.90975854106068e-08
2498 4.90792836203013e-08
2499 4.9053655367004e-08
2500 4.90377634392303e-08
2501 4.90137660946743e-08
2502 4.89972134545269e-08
2503 4.89727917667437e-08
2504 4.89556048322015e-08
2505 4.89381764623431e-08
2506 4.89127083014296e-08
2507 4.88878782647362e-08
2508 4.887129821185e-08
2509 4.88470637520777e-08
2510 4.88311834772048e-08
2511 4.88072383433291e-08
2512 4.87895519114545e-08
2513 4.87660514281174e-08
2514 4.87490308600513e-08
2515 4.87253658150166e-08
2516 4.87085530878062e-08
2517 4.86853617331917e-08
2518 4.86620033726126e-08
2519 4.864128955262e-08
2520 4.8624142458209e-08
2521 4.86132947230544e-08
2522 4.85991770204919e-08
2523 4.85690377800552e-08
2524 4.85519708419702e-08
2525 4.8532649991273e-08
2526 4.85047468714583e-08
2527 4.84788994796759e-08
2528 4.84612886637592e-08
2529 4.84380039083021e-08
2530 4.84229414610127e-08
2531 4.83976012404241e-08
2532 4.83802874740036e-08
2533 4.83574045588853e-08
2534 4.83403152955475e-08
2535 4.83166452198702e-08
2536 4.83026442523737e-08
2537 4.82752392443331e-08
2538 4.82583331091746e-08
2539 4.82338360612289e-08
2540 4.82174579161665e-08
2541 4.81932135159013e-08
2542 4.81776931664513e-08
2543 4.81455163452438e-08
2544 4.81365974351888e-08
2545 4.81145725501619e-08
2546 4.80900606660839e-08
2547 4.80728455087842e-08
2548 4.80513025067353e-08
2549 4.80342722610771e-08
2550 4.80124693069683e-08
2551 4.79954934391458e-08
2552 4.79824469792334e-08
2553 4.7949488468646e-08
2554 4.794014432008e-08
2555 4.79116774911859e-08
2556 4.79018537333786e-08
2557 4.78733109261498e-08
2558 4.78632446174743e-08
2559 4.78343959784411e-08
2560 4.78257118672332e-08
2561 4.78055568109426e-08
2562 4.77898454604997e-08
2563 4.7758338581616e-08
2564 4.77403356740069e-08
2565 4.77261551026231e-08
2566 4.77064287451867e-08
2567 4.77001610761363e-08
2568 4.76858774831612e-08
2569 4.76603335997083e-08
2570 4.76431574227831e-08
2571 4.76180651816094e-08
2572 4.76021762949586e-08
2573 4.75776562822716e-08
2574 4.75620541138255e-08
2575 4.75384704330395e-08
2576 4.75233046799417e-08
2577 4.74996820045703e-08
2578 4.74747835497169e-08
2579 4.74592439871913e-08
2580 4.7446440575527e-08
2581 4.74139427169007e-08
2582 4.73858270026994e-08
2583 4.73869730868159e-08
2584 4.73541909542519e-08
2585 4.73293489022808e-08
2586 4.73185471321358e-08
2587 4.72976877787801e-08
2588 4.72850313357753e-08
2589 4.72638957660365e-08
2590 4.72601782846027e-08
2591 4.72380057061628e-08
2592 4.72235527269049e-08
2593 4.7200782844925e-08
2594 4.71868160403233e-08
2595 4.71642748465229e-08
2596 4.71423125674164e-08
2597 4.71042593588322e-08
2598 4.71006750686342e-08
2599 4.70799386036447e-08
2600 4.70663620575351e-08
2601 4.70355079045248e-08
2602 4.70304024347001e-08
2603 4.70071159952568e-08
2604 4.69616275111662e-08
2605 4.69415164587872e-08
2606 4.69373472711254e-08
2607 4.69074515265788e-08
2608 4.69016400188593e-08
2609 4.68710124721383e-08
2610 4.68710357850455e-08
2611 4.687585286689e-08
2612 4.68618322315706e-08
2613 4.68215372038117e-08
2614 4.67998634547939e-08
2615 4.67717358176856e-08
2616 4.67596120117264e-08
2617 4.67337389409295e-08
2618 4.67397930918878e-08
2619 4.67028056476693e-08
2620 4.66888253711772e-08
2621 4.66705996942096e-08
2622 4.66488213248795e-08
2623 4.66254449378312e-08
2624 4.66182270244531e-08
2625 4.65869506882655e-08
2626 4.65746689002344e-08
2627 4.65510848215445e-08
2628 4.65375115439315e-08
2629 4.65138104601692e-08
2630 4.64987848189935e-08
2631 4.64786855971511e-08
2632 4.64636254591255e-08
2633 4.64404174564947e-08
2634 4.64328485421106e-08
2635 4.64085118068169e-08
2636 4.63954805027811e-08
2637 4.63717111855999e-08
2638 4.63522522338167e-08
2639 4.63255237406202e-08
2640 4.63125699710076e-08
2641 4.62891272761112e-08
2642 4.62767599600511e-08
2643 4.62532915577185e-08
2644 4.62578429605287e-08
2645 4.62335701456595e-08
2646 4.62203011508677e-08
2647 4.61955336348296e-08
2648 4.61817033610146e-08
2649 4.61574102246232e-08
2650 4.61443693922092e-08
2651 4.61202641659497e-08
2652 4.61076916664638e-08
2653 4.60832899946695e-08
2654 4.60708873646354e-08
2655 4.60483360740227e-08
2656 4.60347446207265e-08
2657 4.60038463430124e-08
2658 4.59956762739466e-08
2659 4.59717139733584e-08
2660 4.59589704675523e-08
2661 4.59328179900353e-08
2662 4.5919823904228e-08
2663 4.58982463911184e-08
2664 4.58839099835018e-08
2665 4.58621428123251e-08
2666 4.58491076926748e-08
2667 4.58196193093841e-08
2668 4.5801640560228e-08
2669 4.57808543146143e-08
2670 4.57682126082659e-08
2671 4.57470084143097e-08
2672 4.57337647858935e-08
2673 4.57124123371955e-08
2674 4.56992296449243e-08
2675 4.56735129503727e-08
2676 4.56556823067444e-08
2677 4.56338792105271e-08
2678 4.56228265761638e-08
2679 4.56019131291896e-08
2680 4.55890048129959e-08
2681 4.55677483444106e-08
2682 4.55554961718008e-08
2683 4.55335153475289e-08
2684 4.55219631376735e-08
2685 4.55166307276045e-08
2686 4.55020441734177e-08
2687 4.54821450404097e-08
2688 4.54644332776866e-08
2689 4.54418680817525e-08
2690 4.54313781688143e-08
2691 4.5411819840524e-08
2692 4.53971976597245e-08
2693 4.53782479254983e-08
2694 4.5362922101333e-08
2695 4.5341641829566e-08
2696 4.53276667187197e-08
2697 4.53050192916749e-08
2698 4.52947493840838e-08
2699 4.52743609997697e-08
2700 4.52590555966026e-08
2701 4.5239104778716e-08
2702 4.52241358672723e-08
2703 4.52062379636686e-08
2704 4.51905150242737e-08
2705 4.52159435226918e-08
2706 4.5196377172374e-08
2707 4.51752978989362e-08
2708 4.51588237595502e-08
2709 4.51266193408628e-08
2710 4.51022526135603e-08
2711 4.5084997609024e-08
2712 4.50681127901476e-08
2713 4.50546086128156e-08
2714 4.50407367580397e-08
2715 4.50213217888518e-08
2716 4.49973773370971e-08
2717 4.49780991047533e-08
2718 4.49689434987022e-08
2719 4.49535173743243e-08
2720 4.49305922956e-08
2721 4.49089128764513e-08
2722 4.48956475267437e-08
2723 4.48706800639798e-08
2724 4.48739265408449e-08
2725 4.48165095647823e-08
2726 4.47993773065036e-08
2727 4.47823677092174e-08
2728 4.47671926906423e-08
2729 4.47444438549383e-08
2730 4.47293455465569e-08
2731 4.47125652556224e-08
2732 4.47062771300466e-08
2733 4.46878061097777e-08
2734 4.46701727909726e-08
2735 4.46424202991125e-08
2736 4.46270202658638e-08
2737 4.45942176625636e-08
2738 4.45856198609818e-08
2739 4.45373968389617e-08
2740 4.45249869187592e-08
2741 4.45066017462636e-08
2742 4.44969815163176e-08
2743 4.44737041362941e-08
2744 4.44615917629676e-08
2745 4.44384191737868e-08
2746 4.4426124787833e-08
2747 4.44028485233616e-08
2748 4.43907933984633e-08
2749 4.43729020744854e-08
2750 4.43586442884225e-08
2751 4.4331720268076e-08
2752 4.43174091699916e-08
2753 4.42953355559439e-08
2754 4.42815412853292e-08
2755 4.42543480332347e-08
2756 4.42403402303171e-08
2757 4.42216706062482e-08
2758 4.42064307364376e-08
2759 4.41817823499946e-08
2760 4.4170972273605e-08
2761 4.41410298606115e-08
2762 4.41284960928101e-08
2763 4.41099405250611e-08
2764 4.40921843818387e-08
2765 4.40695949421865e-08
2766 4.40557260148466e-08
2767 4.40344740866294e-08
2768 4.40200800753132e-08
2769 4.39979046547023e-08
2770 4.39841996282553e-08
2771 4.39614013032497e-08
2772 4.39471834070559e-08
2773 4.39258115534358e-08
2774 4.39114889729808e-08
2775 4.38922767429517e-08
2776 4.38652625120994e-08
2777 4.3847781370232e-08
2778 4.38374935924912e-08
2779 4.38180753334905e-08
2780 4.38039917938227e-08
2781 4.3782932408476e-08
2782 4.37671083020064e-08
2783 4.37494838863017e-08
2784 4.37311115391026e-08
2785 4.37203485859072e-08
2786 4.37029379725118e-08
2787 4.36785390434125e-08
2788 4.36603118671997e-08
2789 4.36426450960425e-08
2790 4.36267958576764e-08
2791 4.36074457965674e-08
2792 4.35910614129398e-08
2793 4.35729325189982e-08
2794 4.35575719563985e-08
2795 4.35407245973352e-08
2796 4.3522179851152e-08
2797 4.35045051006e-08
2798 4.34886420919156e-08
2799 4.34696198823303e-08
2800 4.34574589647241e-08
2801 4.34398077118203e-08
2802 4.34243197844353e-08
2803 4.34025379405512e-08
2804 4.33904131895702e-08
2805 4.33680900258082e-08
2806 4.33422248349302e-08
2807 4.33290512589224e-08
2808 4.33123796597101e-08
2809 4.32959559333312e-08
2810 4.32819207460966e-08
2811 4.32535382373089e-08
2812 4.32386168043308e-08
2813 4.32179117026976e-08
2814 4.32032744797084e-08
2815 4.31824882412002e-08
2816 4.31707620123234e-08
2817 4.31521513064581e-08
2818 4.31365217181678e-08
2819 4.31176904811537e-08
2820 4.31047752300628e-08
2821 4.3087796015584e-08
2822 4.30729518612338e-08
2823 4.30549221590581e-08
2824 4.30430366336054e-08
2825 4.30236979127585e-08
2826 4.30086014588937e-08
2827 4.29914814503718e-08
2828 4.29761729563438e-08
2829 4.29589451371726e-08
2830 4.29443767160365e-08
2831 4.29259492946699e-08
2832 4.2911345097707e-08
2833 4.28906516489747e-08
2834 4.28779408068181e-08
2835 4.28613560430335e-08
2836 4.2845481260656e-08
2837 4.28282266824453e-08
2838 4.2814149018966e-08
2839 4.27967468183965e-08
2840 4.27841328161094e-08
2841 4.27665098072794e-08
2842 4.27548866852589e-08
2843 4.27409075101082e-08
2844 4.27238805116303e-08
2845 4.27056477647625e-08
2846 4.26933390897943e-08
2847 4.26735692400371e-08
2848 4.26647361493337e-08
2849 4.26475516945857e-08
2850 4.26316335193633e-08
2851 4.26175247483229e-08
2852 4.26032420932643e-08
2853 4.25844038431933e-08
2854 4.25742655991712e-08
2855 4.25582884062692e-08
2856 4.25406372670523e-08
2857 4.25243402233377e-08
2858 4.25115039917046e-08
2859 4.24905639775375e-08
2860 4.24798570506368e-08
2861 4.24610634084388e-08
2862 4.24490893564666e-08
2863 4.24308456246081e-08
2864 4.24207427158763e-08
2865 4.24017342623984e-08
2866 4.23918691510039e-08
2867 4.23744082311828e-08
2868 4.23609847501893e-08
2869 4.23456238252129e-08
2870 4.23323595981628e-08
2871 4.2318178493872e-08
2872 4.23014924137988e-08
2873 4.22864041169646e-08
2874 4.22735440892552e-08
2875 4.22578179453126e-08
2876 4.22444613548123e-08
2877 4.22290945607529e-08
2878 4.22173619440969e-08
2879 4.21980116840359e-08
2880 4.21890508377487e-08
2881 4.2171351601894e-08
2882 4.21584626622007e-08
2883 4.21424769427858e-08
2884 4.21281442299914e-08
2885 4.21130238024148e-08
2886 4.20994069116887e-08
2887 4.20851290883206e-08
2888 4.20726973828778e-08
2889 4.20562085707843e-08
2890 4.204445245648e-08
2891 4.20430907652758e-08
2892 4.20335134947436e-08
2893 4.20098776743316e-08
2894 4.1988902104606e-08
2895 4.19831162119522e-08
2896 4.19707570600281e-08
2897 4.19548197712061e-08
2898 4.19410543912591e-08
2899 4.19227782941789e-08
2900 4.19155923196968e-08
2901 4.18967207664878e-08
2902 4.18871959837475e-08
2903 4.18697274966462e-08
2904 4.18580986618622e-08
2905 4.18279861236215e-08
2906 4.18218179518703e-08
2907 4.18103405337433e-08
2908 4.17935734020602e-08
2909 4.17779905959037e-08
2910 4.17732937663118e-08
2911 4.17558014760289e-08
2912 4.17394089851086e-08
2913 4.17176305020916e-08
2914 4.17057988357783e-08
2915 4.1688475164392e-08
2916 4.16751988296937e-08
2917 4.16470806285929e-08
2918 4.16311340174502e-08
2919 4.16142032833022e-08
2920 4.15998318530342e-08
2921 4.15851204991213e-08
2922 4.15688157673344e-08
2923 4.15543170504407e-08
2924 4.15397563671149e-08
2925 4.15239846489612e-08
2926 4.15137445841651e-08
2927 4.14951228719929e-08
2928 4.14820092302648e-08
2929 4.14673525241938e-08
2930 4.14531900290172e-08
2931 4.14383706583976e-08
2932 4.1424275167401e-08
2933 4.14098686007947e-08
2934 4.13963362504433e-08
2935 4.13826761018754e-08
2936 4.13704054338382e-08
2937 4.13559172116607e-08
2938 4.13415708706566e-08
2939 4.13266338696872e-08
2940 4.13125160179106e-08
2941 4.1296852423045e-08
2942 4.12831569533978e-08
2943 4.12677493883962e-08
2944 4.12606522246506e-08
2945 4.12442331665375e-08
2946 4.12302000114551e-08
2947 4.12140783652148e-08
2948 4.12017463773395e-08
2949 4.1187607990878e-08
2950 4.11787100063066e-08
2951 4.11644223987651e-08
2952 4.11313876682584e-08
2953 4.11228321439694e-08
2954 4.11097234191971e-08
2955 4.10767606169316e-08
2956 4.10641093964159e-08
2957 4.10475510221886e-08
2958 4.10360093212603e-08
2959 4.10201875524763e-08
2960 4.10063494058477e-08
2961 4.09879233842503e-08
2962 4.09741548352827e-08
2963 4.0958383166867e-08
2964 4.09454055798619e-08
2965 4.09293945295985e-08
2966 4.09143654209743e-08
2967 4.09019723548454e-08
2968 4.08860789846699e-08
2969 4.08711291086661e-08
2970 4.08603659209916e-08
2971 4.08442417096921e-08
2972 4.08289844457954e-08
2973 4.08159299070121e-08
2974 4.07975592509047e-08
2975 4.07803956434805e-08
2976 4.0768644929301e-08
2977 4.07503394512787e-08
2978 4.07323937636761e-08
2979 4.07188326363439e-08
2980 4.07077332909012e-08
2981 4.06953093659013e-08
2982 4.06884420556253e-08
2983 4.06678218922707e-08
2984 4.06570048880894e-08
2985 4.06467258997623e-08
2986 4.06307259837035e-08
2987 4.06189892956377e-08
2988 4.06083526769407e-08
2989 4.05932729137248e-08
2990 4.05821767230918e-08
2991 4.05657117923397e-08
2992 4.05558655884875e-08
2993 4.05389514597232e-08
2994 4.05286307909591e-08
2995 4.0515213449055e-08
2996 4.05014611430943e-08
2997 4.04881957720704e-08
2998 4.04919090257749e-08
2999 4.04771566593354e-08
3000 4.046176100303e-08
3001 4.04479992255347e-08
3002 4.04334260082351e-08
3003 4.04201921853087e-08
3004 4.04052076419248e-08
3005 4.03921915008709e-08
3006 4.03798288033386e-08
3007 4.03705484970374e-08
3008 4.0354571630985e-08
3009 4.03400449684455e-08
3010 4.03298340856395e-08
3011 4.03173966958548e-08
3012 4.03040917831277e-08
3013 4.02889352812963e-08
3014 4.02761895372805e-08
3015 4.02620533321851e-08
3016 4.02506762711141e-08
3017 4.02397802758969e-08
3018 4.02229375566776e-08
3019 4.02091450055764e-08
3020 4.01957830646893e-08
3021 4.01833250052164e-08
3022 4.01702993571007e-08
3023 4.01569427239679e-08
3024 4.01469599893289e-08
3025 4.01341909253006e-08
3026 4.01261119193919e-08
3027 4.01072963427396e-08
3028 4.00939477671614e-08
3029 4.00808263805175e-08
3030 4.00759715830645e-08
3031 4.00584877908727e-08
3032 4.00473919341948e-08
3033 4.00354138179182e-08
3034 4.00218731613222e-08
3035 4.00098893109657e-08
3036 3.9995878296395e-08
3037 3.99821971228675e-08
3038 3.99702219411324e-08
3039 3.99575938274666e-08
3040 3.99452826727043e-08
3041 3.99354739215596e-08
3042 3.99227829177562e-08
3043 3.99116366338603e-08
3044 3.98986479552832e-08
3045 3.98852733027866e-08
3046 3.98732751278885e-08
3047 3.98626612962971e-08
3048 3.98523909908022e-08
3049 3.98386416264884e-08
3050 3.98254571578605e-08
3051 3.98136581196695e-08
3052 3.98038344755491e-08
3053 3.97879686815372e-08
3054 3.97789093042888e-08
3055 3.97659436046638e-08
3056 3.9754807048098e-08
3057 3.97414143264996e-08
3058 3.97304668595666e-08
3059 3.97172155146563e-08
3060 3.97086329826379e-08
3061 3.96913380882324e-08
3062 3.96835653759808e-08
3063 3.96691759050327e-08
3064 3.96597965988121e-08
3065 3.96444469572543e-08
3066 3.96353289033868e-08
3067 3.9620772291471e-08
3068 3.96110539071515e-08
3069 3.95950820077928e-08
3070 3.95870274445542e-08
3071 3.95812294300413e-08
3072 3.9563223893424e-08
3073 3.95425863928267e-08
3074 3.95348274935259e-08
3075 3.95146205036667e-08
3076 3.95126518881739e-08
3077 3.94890157124905e-08
3078 3.94861648231881e-08
3079 3.94651645194699e-08
3080 3.94566089454429e-08
3081 3.94390235740616e-08
3082 3.94295230137232e-08
3083 3.94082742545265e-08
3084 3.93867384360647e-08
3085 3.9363637476697e-08
3086 3.93610967890368e-08
3087 3.93382747674309e-08
3088 3.93367597126826e-08
3089 3.93139847929547e-08
3090 3.93131090703491e-08
3091 3.92901309282934e-08
3092 3.92891952714081e-08
3093 3.92652831564533e-08
3094 3.92657417123132e-08
3095 3.9242823433483e-08
3096 3.92400080784228e-08
3097 3.92171715688505e-08
3098 3.92189900679796e-08
3099 3.91928416547671e-08
3100 3.9198681250241e-08
3101 3.91769238774486e-08
3102 3.91778507520257e-08
3103 3.91524141889477e-08
3104 3.91534893893208e-08
3105 3.91283961036493e-08
3106 3.91290557715251e-08
3107 3.91046331316147e-08
3108 3.91055654986872e-08
3109 3.90808604606718e-08
3110 3.90734568824769e-08
3111 3.904657974374e-08
3112 3.90475629430398e-08
3113 3.90247265968924e-08
3114 3.90154835940848e-08
3115 3.8994325329611e-08
3116 3.89934052975605e-08
3117 3.89704202703456e-08
3118 3.89892652350454e-08
3119 3.89628834653877e-08
3120 3.89596161198824e-08
3121 3.89375558000893e-08
3122 3.89352122809328e-08
3123 3.89129509414943e-08
3124 3.89106101792436e-08
3125 3.88902558299264e-08
3126 3.8886522759185e-08
3127 3.88662221340041e-08
3128 3.88622090525814e-08
3129 3.88425939519266e-08
3130 3.88242137105976e-08
3131 3.8805395384145e-08
3132 3.8801755096074e-08
3133 3.87814247773122e-08
3134 3.87778140336081e-08
3135 3.87611530285881e-08
3136 3.8756328855527e-08
3137 3.87363043543587e-08
3138 3.87313330065808e-08
3139 3.87168013276096e-08
3140 3.87097271499215e-08
3141 3.86900851623295e-08
3142 3.86860399501643e-08
3143 3.86724925647286e-08
3144 3.86630044246772e-08
3145 3.864517064045e-08
3146 3.86427964329528e-08
3147 3.8624116470487e-08
3148 3.86179430478251e-08
3149 3.86017193818589e-08
3150 3.85928856019291e-08
3151 3.85803526157247e-08
3152 3.85752468261558e-08
3153 3.85547069043923e-08
3154 3.85516356118387e-08
3155 3.85340500486109e-08
3156 3.85289099114061e-08
3157 3.85102986797392e-08
3158 3.84967854500928e-08
3159 3.84789900991223e-08
3160 3.84740432437525e-08
3161 3.8457080528076e-08
3162 3.84515223856852e-08
3163 3.84473909775807e-08
3164 3.84321618867034e-08
3165 3.84203100338709e-08
3166 3.84074992538785e-08
3167 3.8398100677739e-08
3168 3.8386929219314e-08
3169 3.83752461416975e-08
3170 3.83697237538172e-08
3171 3.83564175692186e-08
3172 3.83462775772614e-08
3173 3.83339616476519e-08
3174 3.83245094468521e-08
3175 3.83117415125867e-08
3176 3.83018532090773e-08
3177 3.82873992847976e-08
3178 3.82689804823144e-08
3179 3.8259270944252e-08
3180 3.82491751054204e-08
3181 3.82389944491024e-08
3182 3.82321212342163e-08
3183 3.82117534272197e-08
3184 3.82035198853714e-08
3185 3.82079280356606e-08
3186 3.82116207902072e-08
3187 3.81986275854729e-08
3188 3.81865571625895e-08
3189 3.81726239879754e-08
3190 3.81609378976577e-08
3191 3.81505390265602e-08
3192 3.81360402599284e-08
3193 3.81252356262962e-08
3194 3.81101850663867e-08
3195 3.80950387608436e-08
3196 3.80883108803687e-08
3197 3.80906600341291e-08
3198 3.80786708973346e-08
3199 3.80687033612048e-08
3200 3.80546417986238e-08
3201 3.8080723236078e-08
3202 3.8067338302028e-08
3203 3.80631663787767e-08
3204 3.80479003467826e-08
3205 3.80382149671732e-08
3206 3.80283326819608e-08
3207 3.80275854041656e-08
3208 3.80176091709927e-08
3209 3.80071860988096e-08
3210 3.79927562192961e-08
3211 3.7975492240605e-08
3212 3.79609438851958e-08
3213 3.795092587211e-08
3214 3.79396348719752e-08
3215 3.79256221165747e-08
3216 3.79158809664659e-08
3217 3.79039208127097e-08
3218 3.78890104357765e-08
3219 3.78798831306426e-08
3220 3.78645086129836e-08
3221 3.78567310761468e-08
3222 3.78411606050122e-08
3223 3.78337010076279e-08
3224 3.78182806741734e-08
3225 3.781084270571e-08
3226 3.77951070760218e-08
3227 3.77873039809629e-08
3228 3.77728340765771e-08
3229 3.77665485160605e-08
3230 3.77545880709818e-08
3231 3.77410317256022e-08
3232 3.77306293941615e-08
3233 3.77150158854533e-08
3234 3.77073309039133e-08
3235 3.7691418150132e-08
3236 3.7684202034427e-08
3237 3.76711888563364e-08
3238 3.76618295803155e-08
3239 3.7651661834559e-08
3240 3.7638533228801e-08
3241 3.76323413888713e-08
3242 3.76131536583557e-08
3243 3.76107091000222e-08
3244 3.75924668816197e-08
3245 3.75803058716429e-08
3246 3.75705547170924e-08
3247 3.75560000520636e-08
3248 3.75542074095847e-08
3249 3.75383384181305e-08
3250 3.75266151237952e-08
3251 3.75147246813867e-08
3252 3.75037313276039e-08
3253 3.748947926141e-08
3254 3.74790994612795e-08
3255 3.74702097403201e-08
3256 3.74567193333064e-08
3257 3.74512890743972e-08
3258 3.74335862431963e-08
3259 3.74297425835834e-08
3260 3.74129326559114e-08
3261 3.7399272365235e-08
3262 3.73952782126707e-08
3263 3.73769501109678e-08
3264 3.73745569319794e-08
3265 3.7360158195554e-08
3266 3.73485039801835e-08
3267 3.7337447402308e-08
3268 3.73256673427136e-08
3269 3.73150760353269e-08
3270 3.72984142984478e-08
3271 3.7292514647902e-08
3272 3.72774681807186e-08
3273 3.72643097534819e-08
3274 3.72552490048861e-08
3275 3.72178171730297e-08
3276 3.72163666426673e-08
3277 3.72049072367986e-08
3278 3.719403964908e-08
3279 3.71819295352793e-08
3280 3.71715450881993e-08
3281 3.71593040853213e-08
3282 3.71513867705175e-08
3283 3.71381066059939e-08
3284 3.71348101282365e-08
3285 3.71112768533521e-08
3286 3.71011989557246e-08
3287 3.71036003059544e-08
3288 3.71029040238113e-08
3289 3.70808633789466e-08
3290 3.70771267910186e-08
3291 3.70593195171409e-08
3292 3.70530485938048e-08
3293 3.70400719376107e-08
3294 3.70248974803644e-08
3295 3.7014163389415e-08
3296 3.70011813188853e-08
3297 3.69900357739539e-08
3298 3.69795556665053e-08
3299 3.69681904572872e-08
3300 3.69590690567634e-08
3301 3.69503588615316e-08
3302 3.69369965937949e-08
3303 3.69244115034917e-08
3304 3.69162392672706e-08
3305 3.68983712561999e-08
3306 3.68959891048348e-08
3307 3.68757898741023e-08
3308 3.68739116396455e-08
3309 3.68517671915924e-08
3310 3.68464923994338e-08
3311 3.6831081189348e-08
3312 3.68271626598471e-08
3313 3.68114171251932e-08
3314 3.68017709106994e-08
3315 3.67869782920138e-08
3316 3.67832153571612e-08
3317 3.67643679268781e-08
3318 3.67584074894012e-08
3319 3.67406888699406e-08
3320 3.67388760906806e-08
3321 3.67182506764152e-08
3322 3.67108351042589e-08
3323 3.66945575862587e-08
3324 3.66822997719396e-08
3325 3.66636205342274e-08
3326 3.66588140892077e-08
3327 3.66347983060678e-08
3328 3.66272575575977e-08
3329 3.66106290954349e-08
3330 3.66032302778763e-08
3331 3.65933701260701e-08
3332 3.65771773331858e-08
3333 3.65696961779349e-08
3334 3.65541307587591e-08
3335 3.65486321527442e-08
3336 3.65325340538902e-08
3337 3.65244410929222e-08
3338 3.65089543663544e-08
3339 3.65024305324368e-08
3340 3.64861289838814e-08
3341 3.64768484217848e-08
3342 3.64688676981473e-08
3343 3.64565065709144e-08
3344 3.64442536309184e-08
3345 3.64329204671776e-08
3346 3.64249823476825e-08
3347 3.64059989834686e-08
3348 3.64043202623066e-08
3349 3.63875363476041e-08
3350 3.63793168318693e-08
3351 3.63668588789778e-08
3352 3.63585467368921e-08
3353 3.6343113343662e-08
3354 3.63346291720745e-08
3355 3.63223594845863e-08
3356 3.63138196988189e-08
3357 3.63010715389578e-08
3358 3.62879774158387e-08
3359 3.6281143529493e-08
3360 3.62697089641983e-08
3361 3.62603576675724e-08
3362 3.62458867684268e-08
3363 3.62351615947887e-08
3364 3.62289287139106e-08
3365 3.6211692027166e-08
3366 3.62060187129032e-08
3367 3.61911529509484e-08
3368 3.61854417221252e-08
3369 3.61682155300969e-08
3370 3.61642567554554e-08
3371 3.61480296717787e-08
3372 3.61442279910307e-08
3373 3.61258299861333e-08
3374 3.61222605960165e-08
3375 3.61027600845887e-08
3376 3.61022549597578e-08
3377 3.60813769972879e-08
3378 3.60782625961065e-08
3379 3.60571125384013e-08
3380 3.60577015925401e-08
3381 3.60359040030289e-08
3382 3.60331194357855e-08
3383 3.60135832622177e-08
3384 3.60155224328196e-08
3385 3.59924039656789e-08
3386 3.59922819583858e-08
3387 3.5969543951353e-08
3388 3.59730649819312e-08
3389 3.59478515008505e-08
3390 3.59505691918116e-08
3391 3.59257402919866e-08
3392 3.59286133928549e-08
3393 3.5902589004877e-08
3394 3.59159080360882e-08
3395 3.58785872762724e-08
3396 3.58801675659493e-08
3397 3.58627305843129e-08
3398 3.58546506902258e-08
3399 3.58529851283151e-08
3400 3.58261291850681e-08
3401 3.58246107268201e-08
3402 3.58105788222929e-08
3403 3.5798880674065e-08
3404 3.57974921101345e-08
3405 3.57729636775161e-08
3406 3.5778935902897e-08
3407 3.57535355064442e-08
3408 3.57575081224582e-08
3409 3.5730992436811e-08
3410 3.57347490975712e-08
3411 3.57138966933235e-08
3412 3.57104252657336e-08
3413 3.56815236131069e-08
3414 3.56927309610455e-08
3415 3.56724801164887e-08
3416 3.566526734744e-08
3417 3.56540629411484e-08
3418 3.56417861482328e-08
3419 3.56334529953983e-08
3420 3.56202497755476e-08
3421 3.56115462594175e-08
3422 3.55980006503387e-08
3423 3.5589767051647e-08
3424 3.55638259463831e-08
3425 3.55695605378514e-08
3426 3.55556278535119e-08
3427 3.55390269746181e-08
3428 3.55313720774575e-08
3429 3.55209111546628e-08
3430 3.55114427748049e-08
3431 3.55000733591737e-08
3432 3.54916292906182e-08
3433 3.54764353502901e-08
3434 3.5470765972434e-08
3435 3.54551312113927e-08
3436 3.54496592436249e-08
3437 3.54337121066806e-08
3438 3.54285677204302e-08
3439 3.54118671523906e-08
3440 3.54071523176458e-08
3441 3.53907921919472e-08
3442 3.53855409969128e-08
3443 3.537057043701e-08
3444 3.53646606612301e-08
3445 3.53483285451262e-08
3446 3.534286874185e-08
3447 3.53269296837766e-08
3448 3.53209661128062e-08
3449 3.53051326271725e-08
3450 3.52992794105944e-08
3451 3.52831921404118e-08
3452 3.52773814285001e-08
3453 3.5261503576578e-08
3454 3.52547049189411e-08
3455 3.52392497831033e-08
3456 3.52322250805059e-08
3457 3.52161399064244e-08
3458 3.52092312425611e-08
3459 3.51922550549943e-08
3460 3.51850080164695e-08
3461 3.51693211015913e-08
3462 3.51610170596928e-08
3463 3.51439666843589e-08
3464 3.51379992480361e-08
3465 3.51219849505924e-08
3466 3.51160157450181e-08
3467 3.51095473121177e-08
3468 3.51001254230709e-08
3469 3.50876322201543e-08
3470 3.50818412115927e-08
3471 3.50720928281589e-08
3472 3.50784229112833e-08
3473 3.50662273689295e-08
3474 3.50579362091708e-08
3475 3.50462921758776e-08
3476 3.5043142538882e-08
3477 3.50264445145854e-08
3478 3.50191422242574e-08
3479 3.50074448292048e-08
3480 3.49994862247627e-08
3481 3.49885635415603e-08
3482 3.49821480369883e-08
3483 3.49683260623124e-08
3484 3.49619907353826e-08
3485 3.4945432389577e-08
3486 3.49414438645113e-08
3487 3.49311672422914e-08
3488 3.49188645145659e-08
3489 3.49052984489617e-08
3490 3.48996241470445e-08
3491 3.48857442347139e-08
3492 3.48804790135659e-08
3493 3.48677208350523e-08
3494 3.48608398752503e-08
3495 3.48504902589752e-08
3496 3.48454537828502e-08
3497 3.48305080990485e-08
3498 3.48282907509656e-08
3499 3.48091660242744e-08
3500 3.48004820907022e-08
3501 3.47840456100812e-08
3502 3.47763331802753e-08
3503 3.47620976555163e-08
3504 3.47539269753838e-08
3505 3.47411040095835e-08
3506 3.47296038185618e-08
3507 3.47237625248908e-08
3508 3.47093004151589e-08
3509 3.46918069027424e-08
3510 3.46858215110046e-08
3511 3.46708874090496e-08
3512 3.46581057399931e-08
3513 3.46444854599781e-08
3514 3.46382737888007e-08
3515 3.46228795464754e-08
3516 3.46190336131258e-08
3517 3.46069910648339e-08
3518 3.46028858047021e-08
3519 3.45863054462825e-08
3520 3.45811295971998e-08
3521 3.45677723245785e-08
3522 3.45594709756369e-08
3523 3.45392037246484e-08
3524 3.45377303005989e-08
3525 3.45167774398192e-08
3526 3.45083487545139e-08
3527 3.44985208684534e-08
3528 3.44911492078381e-08
3529 3.44823403963801e-08
3530 3.44693229408222e-08
3531 3.44547607227241e-08
3532 3.44506289522428e-08
3533 3.44367005808977e-08
3534 3.44325829431114e-08
3535 3.44146877822027e-08
3536 3.44063281545459e-08
3537 3.4392648885273e-08
3538 3.43841341390316e-08
3539 3.43757319853921e-08
3540 3.4363752646982e-08
3541 3.43503312052462e-08
3542 3.43413350023525e-08
3543 3.43289497308774e-08
3544 3.43203152652904e-08
3545 3.4308064428501e-08
3546 3.42992609532189e-08
3547 3.42901128078665e-08
3548 3.42780916824381e-08
3549 3.42644275903581e-08
3550 3.42575321070626e-08
3551 3.4242902991366e-08
3552 3.4236774176577e-08
3553 3.42220339177857e-08
3554 3.42156371999636e-08
3555 3.42030228921431e-08
3556 3.41976543651867e-08
3557 3.41827594496635e-08
3558 3.41744523666421e-08
3559 3.41596787052367e-08
3560 3.41524770846036e-08
3561 3.41392130280838e-08
3562 3.41311505422937e-08
3563 3.41133040535624e-08
3564 3.41032863602209e-08
3565 3.40906806073349e-08
3566 3.40844792745543e-08
3567 3.40691712281682e-08
3568 3.40619438503609e-08
3569 3.40492743475806e-08
3570 3.40418569066969e-08
3571 3.40298686509755e-08
3572 3.40146594766111e-08
3573 3.4001974697162e-08
3574 3.39990728832618e-08
3575 3.39858077182953e-08
3576 3.39735657988172e-08
3577 3.39603348464834e-08
3578 3.39525780006511e-08
3579 3.39374403779402e-08
3580 3.39301001019976e-08
3581 3.39224295373697e-08
3582 3.39097494901353e-08
3583 3.38972523437064e-08
3584 3.38932488617161e-08
3585 3.38749824848605e-08
3586 3.38685158496332e-08
3587 3.38549553404732e-08
3588 3.38474401289091e-08
3589 3.3835446210162e-08
3590 3.38262332206796e-08
3591 3.38119987759455e-08
3592 3.38049184804845e-08
3593 3.37906771576968e-08
3594 3.3783869540116e-08
3595 3.37694574099601e-08
3596 3.3761980141378e-08
3597 3.37479884464642e-08
3598 3.37392171019246e-08
3599 3.37262810106154e-08
3600 3.37184948193681e-08
3601 3.37049281853297e-08
3602 3.3696383823667e-08
3603 3.36820556796624e-08
3604 3.36747257279058e-08
3605 3.36574259733879e-08
3606 3.36531031663867e-08
3607 3.36392065989344e-08
3608 3.36302865235893e-08
3609 3.36159836606953e-08
3610 3.36077356664077e-08
3611 3.35885932258861e-08
3612 3.35842195369196e-08
3613 3.35665717301481e-08
3614 3.35639001747268e-08
3615 3.3545768701515e-08
3616 3.35447348618345e-08
3617 3.35111887821427e-08
3618 3.35263324942048e-08
3619 3.35082618150295e-08
3620 3.35033607328228e-08
3621 3.34912495887352e-08
3622 3.34748900314708e-08
3623 3.34737051446155e-08
3624 3.343919658505e-08
3625 3.34586860333275e-08
3626 3.34290940102733e-08
3627 3.34345064914032e-08
3628 3.34155637844447e-08
3629 3.34004349298311e-08
3630 3.33896532822564e-08
3631 3.33933335738834e-08
3632 3.33662698679404e-08
3633 3.33747231309189e-08
3634 3.33455896921464e-08
3635 3.33535102114979e-08
3636 3.33148166618003e-08
3637 3.33343818894605e-08
3638 3.3305808329942e-08
3639 3.33129916114672e-08
3640 3.3288036547674e-08
3641 3.32764985060408e-08
3642 3.32681293002679e-08
3643 3.32727537681876e-08
3644 3.32434693746109e-08
3645 3.32553603072938e-08
3646 3.32138806555804e-08
3647 3.3232087361057e-08
3648 3.31827644330929e-08
3649 3.321329107564e-08
3650 3.3183311977325e-08
3651 3.31661651031823e-08
3652 3.31565396081146e-08
3653 3.31672256095317e-08
3654 3.31320671094204e-08
3655 3.3148453837839e-08
3656 3.31102233062097e-08
3657 3.31194370772891e-08
3658 3.30888106105931e-08
3659 3.31071833414853e-08
3660 3.30669686547935e-08
3661 3.30763284850377e-08
3662 3.30438890046025e-08
3663 3.30535417205624e-08
3664 3.30221753799265e-08
3665 3.30354896007634e-08
3666 3.30014431781933e-08
3667 3.3013780708302e-08
3668 3.29791264519486e-08
3669 3.29901544589006e-08
3670 3.29574707080837e-08
3671 3.29666317497868e-08
3672 3.29374887684253e-08
3673 3.29481989069791e-08
3674 3.29198227504435e-08
3675 3.29281573456797e-08
3676 3.2901224145121e-08
3677 3.29136804140262e-08
3678 3.28827518529806e-08
3679 3.28943584690933e-08
3680 3.28612080977564e-08
3681 3.28719777300535e-08
3682 3.28323992988544e-08
3683 3.28547566965653e-08
3684 3.28252676240481e-08
3685 3.28284076758223e-08
3686 3.28064611920809e-08
3687 3.28062017302955e-08
3688 3.27859282620579e-08
3689 3.27699544442339e-08
3690 3.27656607339577e-08
3691 3.27677405493887e-08
3692 3.27286356522905e-08
3693 3.27437685712084e-08
3694 3.2723216911279e-08
3695 3.27161879667415e-08
3696 3.27020507810971e-08
3697 3.269537931061e-08
3698 3.26809388013771e-08
3699 3.26740854745822e-08
3700 3.26620078610063e-08
3701 3.2653511290448e-08
3702 3.26398585457355e-08
3703 3.26278020281734e-08
3704 3.26153986307531e-08
3705 3.26065990989832e-08
3706 3.25965519891724e-08
3707 3.25718671305708e-08
3708 3.25771448856926e-08
3709 3.25674590797576e-08
3710 3.25355279571227e-08
3711 3.25495691058109e-08
3712 3.25355876569233e-08
3713 3.25191141357095e-08
3714 3.25147477227006e-08
3715 3.24877089781239e-08
3716 3.2493933545652e-08
3717 3.24846668888767e-08
3718 3.24588131874748e-08
3719 3.24615737312683e-08
3720 3.2455382644514e-08
3721 3.24241282143589e-08
3722 3.24328633638515e-08
3723 3.24157408400083e-08
3724 3.2398979008974e-08
3725 3.23929918550903e-08
3726 3.23923256217995e-08
3727 3.2357578106712e-08
3728 3.23718217885016e-08
3729 3.23495816090258e-08
3730 3.23358037022103e-08
3731 3.2327814302846e-08
3732 3.23295104749377e-08
3733 3.22927330458356e-08
3734 3.23103367279032e-08
3735 3.22816646090018e-08
3736 3.22684475904111e-08
3737 3.22629676290376e-08
3738 3.22666013801154e-08
3739 3.22397947059017e-08
3740 3.22294000625334e-08
3741 3.22179437546311e-08
3742 3.22284999043632e-08
3743 3.21891249868145e-08
3744 3.22104894152631e-08
3745 3.21525418698343e-08
3746 3.21904711100274e-08
3747 3.21484092467017e-08
3748 3.21511697123356e-08
3749 3.21285390043613e-08
3750 3.21470031678928e-08
3751 3.20983691750598e-08
3752 3.21211779521491e-08
3753 3.20647226033088e-08
3754 3.21011035211427e-08
3755 3.20590492535189e-08
3756 3.20598713727804e-08
3757 3.20363486494557e-08
3758 3.20585757265235e-08
3759 3.1998179828463e-08
3760 3.20428550679708e-08
3761 3.19963541883794e-08
3762 3.20163749378821e-08
3763 3.19726626543115e-08
3764 3.1978184395598e-08
3765 3.19588953558991e-08
3766 3.19707042990558e-08
3767 3.19143940075151e-08
3768 3.19574582761106e-08
3769 3.19097125753842e-08
3770 3.19252073666121e-08
3771 3.18904940286302e-08
3772 3.18904840597156e-08
3773 3.18759204418484e-08
3774 3.18864447379497e-08
3775 3.18300039765518e-08
3776 3.18707082129777e-08
3777 3.18294844845468e-08
3778 3.18321164129998e-08
3779 3.18098368836672e-08
3780 3.18289397185367e-08
3781 3.17726708658483e-08
3782 3.18076564767011e-08
3783 3.17665239961684e-08
3784 3.17701939280823e-08
3785 3.17458852379104e-08
3786 3.17659805730841e-08
3787 3.17085997352251e-08
3788 3.17460722314422e-08
3789 3.17018875861663e-08
3790 3.17076838385333e-08
3791 3.1680313270499e-08
3792 3.1700428280601e-08
3793 3.16432924947208e-08
3794 3.16761357694872e-08
3795 3.16522947159115e-08
3796 3.1658028895265e-08
3797 3.16338625054868e-08
3798 3.16524417272035e-08
3799 3.15989477996936e-08
3800 3.16319964355216e-08
3801 3.15950993723391e-08
3802 3.15969225539447e-08
3803 3.15739777789759e-08
3804 3.15915449036197e-08
3805 3.15394955734405e-08
3806 3.15731230315919e-08
3807 3.15341055738827e-08
3808 3.15337239129576e-08
3809 3.15115175908431e-08
3810 3.15263181249748e-08
3811 3.14745695888519e-08
3812 3.15065361533584e-08
3813 3.14691477925066e-08
3814 3.14697901444561e-08
3815 3.14479817689062e-08
3816 3.14669035930137e-08
3817 3.14120356037506e-08
3818 3.14742907008281e-08
3819 3.14099808775836e-08
3820 3.1425754386305e-08
3821 3.13824884017322e-08
3822 3.14030713610691e-08
3823 3.13441183976693e-08
3824 3.1380986584395e-08
3825 3.13409841652401e-08
3826 3.13552790700555e-08
3827 3.13193064158668e-08
3828 3.13202646111677e-08
3829 3.12971757310265e-08
3830 3.13163689256157e-08
3831 3.12711675150013e-08
3832 3.12933262307524e-08
3833 3.12524907499778e-08
3834 3.12745856163588e-08
3835 3.12182766890601e-08
3836 3.12579568131355e-08
3837 3.1214683168912e-08
3838 3.12318223123498e-08
3839 3.11908217227597e-08
3840 3.11923704714445e-08
3841 3.11701293398414e-08
3842 3.11902918213036e-08
3843 3.1145616219419e-08
3844 3.11673932245071e-08
3845 3.11040846199262e-08
3846 3.11454231720631e-08
3847 3.1103573270741e-08
3848 3.11081342516673e-08
3849 3.10786887389725e-08
3850 3.10812042982889e-08
3851 3.10575770541277e-08
3852 3.107483357212e-08
3853 3.10300314403378e-08
3854 3.10520032194006e-08
3855 3.10115061665783e-08
3856 3.10147317321707e-08
3857 3.09891756131719e-08
3858 3.10110550749698e-08
3859 3.09501417810054e-08
3860 3.09877336270858e-08
3861 3.09438629209069e-08
3862 3.09488021699167e-08
3863 3.09222465872949e-08
3864 3.09430615104134e-08
3865 3.08839670815075e-08
3866 3.09211290634437e-08
3867 3.08771894417248e-08
3868 3.08848295169639e-08
3869 3.0857929573358e-08
3870 3.08289015187313e-08
3871 3.07415755820273e-08
3872 3.07495059423957e-08
3873 3.07177191380958e-08
3874 3.07300431146018e-08
3875 3.06760078956358e-08
3876 3.06983175377695e-08
3877 3.06592280878704e-08
3878 3.06665301579301e-08
3879 3.06410557193715e-08
3880 3.06516955177472e-08
3881 3.06046959934747e-08
3882 3.06366294857696e-08
3883 3.05940993072795e-08
3884 3.06074431044578e-08
3885 3.05688539796733e-08
3886 3.05853715580895e-08
3887 3.05410074226131e-08
3888 3.05622825926832e-08
3889 3.05190872573746e-08
3890 3.05432754785784e-08
3891 3.04958634664843e-08
3892 3.05202351000844e-08
3893 3.0472980252938e-08
3894 3.04981447882824e-08
3895 3.04499234076161e-08
3896 3.0476184047501e-08
3897 3.0427016021406e-08
3898 3.04540591464786e-08
3899 3.04049657238181e-08
3900 3.04321181019418e-08
3901 3.03821827465356e-08
3902 3.03977419449097e-08
3903 3.03591450290242e-08
3904 3.03884029513313e-08
3905 3.03362794618067e-08
3906 3.03531751413289e-08
3907 3.03153963550073e-08
3908 3.03451509999775e-08
3909 3.02921841317527e-08
3910 3.0323609692573e-08
3911 3.0270049656167e-08
3912 3.0301191433324e-08
3913 3.0244935405932e-08
3914 3.02681837105467e-08
3915 3.02197754997735e-08
3916 3.02490430144076e-08
3917 3.01979701688992e-08
3918 3.02252670216774e-08
3919 3.0179908485195e-08
3920 3.02021447105005e-08
3921 3.01528764552472e-08
3922 3.01792062806783e-08
3923 3.01295504137045e-08
3924 3.01551038575099e-08
3925 3.01073822868148e-08
3926 3.01321152349487e-08
3927 3.00720209658323e-08
3928 3.01103582849294e-08
3929 3.00620465303325e-08
3930 3.00755841138312e-08
3931 3.00417549929932e-08
3932 3.00663593009176e-08
3933 3.00172178349101e-08
3934 3.00382749891526e-08
3935 2.99935888214975e-08
3936 3.00167766447146e-08
3937 2.99604559224065e-08
3938 2.99951612525717e-08
3939 2.99510963266414e-08
3940 2.99735896653885e-08
3941 2.99281354010361e-08
3942 2.9952415601997e-08
3943 2.99051477981038e-08
3944 2.99285104325975e-08
3945 2.98847716351247e-08
3946 2.99086246222657e-08
3947 2.98646954171034e-08
3948 2.9889387484161e-08
3949 2.98404595326929e-08
3950 2.98703756342888e-08
3951 2.9820180497353e-08
3952 2.98488861290025e-08
3953 2.97967073521477e-08
3954 2.9824962098246e-08
3955 2.97649698950408e-08
3956 2.98007402967926e-08
3957 2.97441878664984e-08
3958 2.97796519674876e-08
3959 2.97219902130053e-08
3960 2.97585184156901e-08
3961 2.9713992077518e-08
3962 2.97375442350756e-08
3963 2.96953419685053e-08
3964 2.97167022900169e-08
3965 2.96590025889998e-08
3966 2.96975427787061e-08
3967 2.96484380442052e-08
3968 2.96750383697031e-08
3969 2.96297782682586e-08
3970 2.96542956732537e-08
3971 2.96066774012616e-08
3972 2.9634286203617e-08
3973 2.9585124703857e-08
3974 2.9613978320242e-08
3975 2.95667216256845e-08
3976 2.95939414272084e-08
3977 2.95287625675655e-08
3978 2.95689415779066e-08
3979 2.95094521760575e-08
3980 2.95367214633302e-08
3981 2.94874316502103e-08
3982 2.95068394216003e-08
3983 2.94675790328824e-08
3984 2.94908823939011e-08
3985 2.94502856483803e-08
3986 2.94829572133892e-08
3987 2.94307915105208e-08
3988 2.94600783199428e-08
3989 2.94078895457517e-08
3990 2.94274692507202e-08
3991 2.93863024651841e-08
3992 2.94166264147577e-08
3993 2.93661521908461e-08
3994 2.93955193271245e-08
3995 2.93351474063286e-08
3996 2.93731820342202e-08
3997 2.93170975105284e-08
3998 2.93502579964411e-08
3999 2.93015023622445e-08
4000 2.93151621768573e-08
4001 2.92810163031731e-08
4002 2.93093000358624e-08
4003 2.92604181133527e-08
4004 2.92924340676848e-08
4005 2.92246628994519e-08
4006 2.9268730298071e-08
4007 2.92158835009104e-08
4008 2.92483042301228e-08
4009 2.91921070818546e-08
4010 2.92184153103392e-08
4011 2.91723908603103e-08
4012 2.92077401944368e-08
4013 2.91524189179881e-08
4014 2.91802719374346e-08
4015 2.91429870387105e-08
4016 2.91728885777331e-08
4017 2.91131549055024e-08
4018 2.91364616451517e-08
4019 2.90926982806639e-08
4020 2.91280514268522e-08
4021 2.90794011483797e-08
4022 2.90977076744525e-08
4023 2.90508764777542e-08
4024 2.90860595910658e-08
4025 2.90325733089958e-08
4026 2.90547221091231e-08
4027 2.9010017751574e-08
4028 2.90456076150747e-08
4029 2.899013394142e-08
4030 2.90137706819849e-08
4031 2.89682564407201e-08
4032 2.90055709264436e-08
4033 2.89479929200809e-08
4034 2.89710188212666e-08
4035 2.89243279567586e-08
4036 2.89590222024572e-08
4037 2.89017855550355e-08
4038 2.89254875980305e-08
4039 2.88777186518985e-08
4040 2.89113215679038e-08
4041 2.88525922371718e-08
4042 2.88766471179258e-08
4043 2.88304726225874e-08
4044 2.88667062058323e-08
4045 2.88139288890932e-08
4046 2.88392755898315e-08
4047 2.87982209705717e-08
4048 2.88354083330944e-08
4049 2.87819560078617e-08
4050 2.88126819789625e-08
4051 2.87644502954265e-08
4052 2.87955221907055e-08
4053 2.87341328579771e-08
4054 2.87743456226508e-08
4055 2.87242684038347e-08
4056 2.87545503994124e-08
4057 2.87018756104374e-08
4058 2.87352801819907e-08
4059 2.86837225509373e-08
4060 2.87129527301033e-08
4061 2.86644924578638e-08
4062 2.86966121940679e-08
4063 2.86413249703799e-08
4064 2.86776526223775e-08
4065 2.86237368491982e-08
4066 2.86552720645261e-08
4067 2.86046134014839e-08
4068 2.86378976426249e-08
4069 2.85826525612265e-08
4070 2.86239337761174e-08
4071 2.85682396672371e-08
4072 2.85937200530384e-08
4073 2.85469523326753e-08
4074 2.85835176079274e-08
4075 2.85280046803393e-08
4076 2.85621585156548e-08
4077 2.85082385147462e-08
4078 2.85447293997265e-08
4079 2.84859772676782e-08
4080 2.85251638310058e-08
4081 2.84688927472132e-08
4082 2.85077248882715e-08
4083 2.84515571031818e-08
4084 2.84854465029127e-08
4085 2.84313302536532e-08
4086 2.84684185807294e-08
4087 2.84100143801425e-08
4088 2.84510591548326e-08
4089 2.83914185850165e-08
4090 2.84271625297094e-08
4091 2.8372355146189e-08
4092 2.84099650826874e-08
4093 2.83544772585742e-08
4094 2.83909188603104e-08
4095 2.83308205162314e-08
4096 2.83713102611216e-08
4097 2.83137924661503e-08
4098 2.83534547449449e-08
4099 2.83024086620287e-08
4100 2.83318379814546e-08
4101 2.82823212494065e-08
4102 2.83126193458827e-08
4103 2.82626743519643e-08
4104 2.82936153972457e-08
4105 2.82441843815207e-08
4106 2.82769411086292e-08
4107 2.82249039109672e-08
4108 2.82561829259009e-08
4109 2.82082778930715e-08
4110 2.82394411534881e-08
4111 2.81888202522396e-08
4112 2.82200713428438e-08
4113 2.81692900081509e-08
4114 2.82101681499114e-08
4115 2.81399804649141e-08
4116 2.8193283480249e-08
4117 2.81314926375842e-08
4118 2.81703096511876e-08
4119 2.8110237185075e-08
4120 2.81490497258119e-08
4121 2.80840509603308e-08
4122 2.81246080398034e-08
4123 2.8063639977205e-08
4124 2.81057820181729e-08
4125 2.80460404091798e-08
4126 2.80856506797988e-08
4127 2.80288817080532e-08
4128 2.80665782028677e-08
4129 2.80056005976803e-08
4130 2.80581663183455e-08
4131 2.79875815714092e-08
4132 2.80292396332982e-08
4133 2.79673645842138e-08
4134 2.80094358195981e-08
4135 2.79556402027481e-08
4136 2.79994018228535e-08
4137 2.79366991016161e-08
4138 2.7979078957685e-08
4139 2.79199357819948e-08
4140 2.79627934745008e-08
4141 2.79000850404998e-08
4142 2.79438950840927e-08
4143 2.78804878846017e-08
4144 2.7926775828746e-08
4145 2.78614467852378e-08
4146 2.7906398585742e-08
4147 2.78428152498122e-08
4148 2.78872093737448e-08
4149 2.78245771916374e-08
4150 2.78686630075242e-08
4151 2.78055072833183e-08
4152 2.78532726198932e-08
4153 2.77856069814675e-08
4154 2.78264222437485e-08
4155 2.77621673596684e-08
4156 2.78077338791149e-08
4157 2.77439579789984e-08
4158 2.77896116465115e-08
4159 2.77258898222499e-08
4160 2.77720386101521e-08
4161 2.77068424345828e-08
4162 2.77552731695607e-08
4163 2.76752662529134e-08
4164 2.77180569661084e-08
4165 2.76487061761088e-08
4166 2.76968376446973e-08
4167 2.76355484274404e-08
4168 2.76845334035158e-08
4169 2.76141824677723e-08
4170 2.76567369361658e-08
4171 2.75951611854452e-08
4172 2.76440988074e-08
4173 2.7574941416475e-08
4174 2.76254995306147e-08
4175 2.75553251398719e-08
4176 2.75992268718994e-08
4177 2.75369327198405e-08
4178 2.75865644425721e-08
4179 2.75167723913228e-08
4180 2.75642670182208e-08
4181 2.75001379961282e-08
4182 2.75488974104121e-08
4183 2.74871590413284e-08
4184 2.75271526994914e-08
4185 2.74633939376656e-08
4186 2.75095964639149e-08
4187 2.74462349594273e-08
4188 2.74895188105972e-08
4189 2.74260241894808e-08
4190 2.74711758301294e-08
4191 2.74031664950769e-08
4192 2.74547903558187e-08
4193 2.7390249123016e-08
4194 2.74339708035143e-08
4195 2.737205082326e-08
4196 2.7416038363981e-08
4197 2.73519168274561e-08
4198 2.7406621466497e-08
4199 2.73296758095398e-08
4200 2.73783335664746e-08
4201 2.73188979456052e-08
4202 2.73686548801777e-08
4203 2.72923209365672e-08
4204 2.73412369296011e-08
4205 2.72751461629639e-08
4206 2.73221051969585e-08
4207 2.72552169349183e-08
4208 2.73057196942261e-08
4209 2.72359540041123e-08
4210 2.72884191154787e-08
4211 2.72237226077721e-08
4212 2.72762546771332e-08
4213 2.72068437183748e-08
4214 2.72548390718441e-08
4215 2.71846344048754e-08
4216 2.72476544473932e-08
4217 2.71819866028977e-08
4218 2.72223254320636e-08
4219 2.71526674318068e-08
4220 2.72031130137407e-08
4221 2.71459044896005e-08
4222 2.71816599628494e-08
4223 2.71236022335586e-08
4224 2.71665149256251e-08
4225 2.71029605407591e-08
4226 2.71328178271801e-08
4227 2.70708211900228e-08
4228 2.71288861313224e-08
4229 2.70568232565438e-08
4230 2.71078253071266e-08
4231 2.70351370978972e-08
4232 2.71066636550188e-08
4233 2.70143728648975e-08
4234 2.70852806636412e-08
4235 2.69939432691046e-08
4236 2.7054122373471e-08
4237 2.69837813284823e-08
4238 2.70472272738687e-08
4239 2.69601959352883e-08
4240 2.70162143500841e-08
4241 2.69439151203699e-08
4242 2.69948308968537e-08
4243 2.69241841088785e-08
4244 2.69789294904399e-08
4245 2.6906694518658e-08
4246 2.69560943308989e-08
4247 2.68850557318956e-08
4248 2.69387966547185e-08
4249 2.68671128793585e-08
4250 2.69171781397404e-08
4251 2.68469487210155e-08
4252 2.69027746142569e-08
4253 2.68298867176497e-08
4254 2.68817863116055e-08
4255 2.68100733329391e-08
4256 2.68643187766315e-08
4257 2.678988579774e-08
4258 2.68376951346738e-08
4259 2.67619497513749e-08
4260 2.68262422586929e-08
4261 2.67632403492257e-08
4262 2.68137901713317e-08
4263 2.6740148449278e-08
4264 2.67961610482814e-08
4265 2.67218582159501e-08
4266 2.67610562509901e-08
4267 2.66933822494764e-08
4268 2.67409711263156e-08
4269 2.66764690941557e-08
4270 2.67249765748545e-08
4271 2.66572861065129e-08
4272 2.6712959073194e-08
4273 2.6640212229978e-08
4274 2.66891102391753e-08
4275 2.66214238671125e-08
4276 2.66757440883225e-08
4277 2.66048813202246e-08
4278 2.66524324636919e-08
4279 2.65847619367321e-08
4280 2.66378772728615e-08
4281 2.65662733589522e-08
4282 2.66130376900264e-08
4283 2.65485915562635e-08
4284 2.65829011993901e-08
4285 2.65061262609834e-08
4286 2.65535064620792e-08
4287 2.64877961200227e-08
4288 2.6543090587694e-08
4289 2.64616693534947e-08
4290 2.65168710882335e-08
4291 2.64416792212785e-08
4292 2.64999961636647e-08
4293 2.64282514628178e-08
4294 2.64800653759778e-08
4295 2.64067103294963e-08
4296 2.64631986368613e-08
4297 2.63934639797014e-08
4298 2.64397706040143e-08
4299 2.63673868765579e-08
4300 2.64144269408462e-08
4301 2.63528539470315e-08
4302 2.63912160995972e-08
4303 2.63391862844742e-08
4304 2.63644401812257e-08
4305 2.63184512689918e-08
4306 2.63358227066135e-08
4307 2.63028907916407e-08
4308 2.63212374989052e-08
4309 2.62867802263145e-08
4310 2.6298808894154e-08
4311 2.62683326361923e-08
4312 2.62757846307693e-08
4313 2.62466078559953e-08
4314 2.62502551890975e-08
4315 2.62208332912905e-08
4316 2.62314696115595e-08
4317 2.62053846533661e-08
4318 2.62123594261254e-08
4319 2.61857271901533e-08
4320 2.62001668218659e-08
4321 2.61571699731178e-08
4322 2.61759744155654e-08
4323 2.61406981785228e-08
4324 2.61569150801222e-08
4325 2.61208620990772e-08
4326 2.61224650159875e-08
4327 2.61036626838518e-08
4328 2.61069333724606e-08
4329 2.60752178142809e-08
4330 2.60819372641663e-08
4331 2.60574927324342e-08
4332 2.60555315669819e-08
4333 2.60396879205871e-08
4334 2.60496667898735e-08
4335 2.60307142703198e-08
4336 2.60161297980233e-08
4337 2.60034560071176e-08
4338 2.60027205492008e-08
4339 2.59839279976859e-08
4340 2.59869933394441e-08
4341 2.5966459642035e-08
4342 2.59567750937606e-08
4343 2.59393584620682e-08
4344 2.59292016799861e-08
4345 2.59276087639648e-08
4346 2.5915905229823e-08
4347 2.59009904439722e-08
4348 2.58905406944621e-08
4349 2.5889677448987e-08
4350 2.58751478661168e-08
4351 2.5865372432321e-08
4352 2.58462303932561e-08
4353 2.58373664330236e-08
4354 2.58350855482092e-08
4355 2.58097752627862e-08
4356 2.58096619525361e-08
4357 2.57891614268146e-08
4358 2.57875776412675e-08
4359 2.57775972194452e-08
4360 2.57775398040394e-08
4361 2.57497278290941e-08
4362 2.5765773902009e-08
4363 2.57501636475865e-08
4364 2.57521938671346e-08
4365 2.57252104383099e-08
4366 2.57310787574738e-08
4367 2.57048157727979e-08
4368 2.57112629817868e-08
4369 2.56814138630546e-08
4370 2.56910549794043e-08
4371 2.56618720193558e-08
4372 2.56705756562781e-08
4373 2.5639806086275e-08
4374 2.56513612448828e-08
4375 2.5620470580634e-08
4376 2.56310008381888e-08
4377 2.56006514263163e-08
4378 2.56081957381582e-08
4379 2.56000563254588e-08
4380 2.55949008547418e-08
4381 2.55841652219146e-08
4382 2.56138789609395e-08
4383 2.5588008874422e-08
4384 2.56107807814487e-08
4385 2.55589217204033e-08
4386 2.55895601881662e-08
4387 2.55387992851297e-08
4388 2.55587991055961e-08
4389 2.55351680813476e-08
4390 2.55340630239687e-08
4391 2.54933062748819e-08
4392 2.55178383348209e-08
4393 2.54882165435788e-08
4394 2.55111156128862e-08
4395 2.54772414010063e-08
4396 2.54926945508771e-08
4397 2.54517810027721e-08
4398 2.54713849194843e-08
4399 2.54416391989309e-08
4400 2.54521286144893e-08
4401 2.54354106985488e-08
4402 2.54197309850213e-08
4403 2.54091258753419e-08
4404 2.54163644299865e-08
4405 2.53769004778803e-08
4406 2.5394715283511e-08
4407 2.5370251591994e-08
4408 2.5382200842472e-08
4409 2.5352361063824e-08
4410 2.53566603376498e-08
4411 2.5330693969039e-08
4412 2.53423567286859e-08
4413 2.53148441515805e-08
4414 2.5306673272496e-08
4415 2.52967829261763e-08
4416 2.53022563008187e-08
4417 2.52796856585746e-08
4418 2.52632357593541e-08
4419 2.52516259386937e-08
4420 2.52484785683293e-08
4421 2.5235526148748e-08
4422 2.52361725934236e-08
4423 2.52194511993764e-08
4424 2.52139239229621e-08
4425 2.5197091549245e-08
4426 2.5199968401779e-08
4427 2.51751784468013e-08
4428 2.5178024170458e-08
4429 2.51599163618721e-08
4430 2.51584673378602e-08
4431 2.5134243411884e-08
4432 2.51304084137871e-08
4433 2.51166895459676e-08
4434 2.51142127041248e-08
4435 2.50976612257148e-08
4436 2.50813559787844e-08
4437 2.50767367511173e-08
4438 2.50758681197283e-08
4439 2.50611198886475e-08
4440 2.50539448067855e-08
4441 2.5034031683191e-08
4442 2.5028459646137e-08
4443 2.50164238551065e-08
4444 2.50095935605543e-08
4445 2.49973347798971e-08
4446 2.49992508649655e-08
4447 2.49828877443292e-08
4448 2.49705379502529e-08
4449 2.49737943747164e-08
4450 2.49644761076695e-08
4451 2.49568029104807e-08
4452 2.49471216129393e-08
4453 2.49467064676878e-08
4454 2.49277061143971e-08
4455 2.49219696684122e-08
4456 2.49075670204491e-08
4457 2.49125583273724e-08
4458 2.48921782137757e-08
4459 2.48998930558741e-08
4460 2.48716369846136e-08
4461 2.48695182740732e-08
4462 2.48555511568327e-08
4463 2.48541563756532e-08
4464 2.48359712919921e-08
4465 2.48462548775308e-08
4466 2.48153609518909e-08
4467 2.48324964964297e-08
4468 2.47966014264023e-08
4469 2.48113138461292e-08
4470 2.47826055641553e-08
4471 2.47949919049972e-08
4472 2.47671719542097e-08
4473 2.47652831610878e-08
4474 2.47681856819781e-08
4475 2.4772011336438e-08
4476 2.47450731727383e-08
4477 2.47682723752973e-08
4478 2.47186924795528e-08
4479 2.47321178576954e-08
4480 2.47104380761698e-08
4481 2.47146841232393e-08
4482 2.46805886092716e-08
4483 2.46971943056451e-08
4484 2.46630898352862e-08
4485 2.4675874875868e-08
4486 2.4644378349592e-08
4487 2.46599544944104e-08
4488 2.46261192451414e-08
4489 2.46446963814151e-08
4490 2.46152364979935e-08
4491 2.46284984761758e-08
4492 2.45936859855078e-08
4493 2.46099703105074e-08
4494 2.45766120166024e-08
4495 2.4596246539943e-08
4496 2.45668620912909e-08
4497 2.45658594693055e-08
4498 2.45473505380289e-08
4499 2.45752608698524e-08
4500 2.45342342175547e-08
4501 2.45538887249097e-08
4502 2.4518495617798e-08
4503 2.45341445257452e-08
4504 2.44901423762656e-08
4505 2.45097340894063e-08
4506 2.44675160203656e-08
4507 2.44887705349583e-08
4508 2.44607212387393e-08
4509 2.44836364053924e-08
4510 2.44351343816618e-08
4511 2.44505745534696e-08
4512 2.44103238991045e-08
4513 2.44449860602458e-08
4514 2.43987839034787e-08
4515 2.44286559478724e-08
4516 2.43935756927272e-08
4517 2.4433798220258e-08
4518 2.43937556163587e-08
4519 2.44182105078039e-08
4520 2.43943457363116e-08
4521 2.44006744694047e-08
4522 2.43673550635037e-08
4523 2.43930054395491e-08
4524 2.43492919906885e-08
4525 2.43731881717224e-08
4526 2.43320855446427e-08
4527 2.43533099890669e-08
4528 2.43124870493716e-08
4529 2.4343245993208e-08
4530 2.42905397946913e-08
4531 2.43153704566623e-08
4532 2.42832867414222e-08
4533 2.43004112014944e-08
4534 2.42537133416931e-08
4535 2.42793855029788e-08
4536 2.42534410226369e-08
4537 2.42551119455925e-08
4538 2.42086687016752e-08
4539 2.42389515534569e-08
4540 2.41875350575071e-08
4541 2.42215437715743e-08
4542 2.41802252780587e-08
4543 2.41987790587928e-08
4544 2.41597527086412e-08
4545 2.41933334876876e-08
4546 2.4149329153289e-08
4547 2.41768954403199e-08
4548 2.41323769714086e-08
4549 2.4145179970958e-08
4550 2.41022837244032e-08
4551 2.41356745753762e-08
4552 2.40892966978379e-08
4553 2.41114870682679e-08
4554 2.40715519090884e-08
4555 2.41016318724974e-08
4556 2.40546498879723e-08
4557 2.40758384073558e-08
4558 2.40364317498631e-08
4559 2.406515683262e-08
4560 2.40184398840881e-08
4561 2.4044896424158e-08
4562 2.39904797929569e-08
4563 2.401524693596e-08
4564 2.39674789135336e-08
4565 2.39867031055496e-08
4566 2.39427040433782e-08
4567 2.39676761637497e-08
4568 2.39233407377526e-08
4569 2.39527625787161e-08
4570 2.39069484280208e-08
4571 2.39325595821072e-08
4572 2.3888968868846e-08
4573 2.39140309084007e-08
4574 2.38937857979238e-08
4575 2.39331376477026e-08
4576 2.38593463386394e-08
4577 2.38818224289616e-08
4578 2.38326963888369e-08
4579 2.38463622181939e-08
4580 2.3827790482045e-08
4581 2.3862812657427e-08
4582 2.38127920759723e-08
4583 2.3831651894568e-08
4584 2.37947379631009e-08
4585 2.37998170931064e-08
4586 2.3761450805182e-08
4587 2.3775201022147e-08
4588 2.37622212821975e-08
4589 2.37582382318635e-08
4590 2.37338287156774e-08
4591 2.37475931967879e-08
4592 2.37193298531224e-08
4593 2.37299077703312e-08
4594 2.36940402125185e-08
4595 2.3718793602967e-08
4596 2.36703561036222e-08
4597 2.36837332998618e-08
4598 2.36586250181858e-08
4599 2.36814752447856e-08
4600 2.36314799941795e-08
4601 2.36616242261789e-08
4602 2.36077235520327e-08
4603 2.36398599717802e-08
4604 2.35877100784876e-08
4605 2.36190034890171e-08
4606 2.35715052596674e-08
4607 2.35917815452069e-08
4608 2.35680650924053e-08
4609 2.35680236819746e-08
4610 2.3539718686294e-08
4611 2.35475145764497e-08
4612 2.35167687669957e-08
4613 2.35258666130278e-08
4614 2.34958231253302e-08
4615 2.35020998005098e-08
4616 2.34636023357382e-08
4617 2.34923263597864e-08
4618 2.34508079728357e-08
4619 2.34548360644737e-08
4620 2.34394321338982e-08
4621 2.34520562258922e-08
4622 2.34208516189938e-08
4623 2.34183476912619e-08
4624 2.33962601114968e-08
4625 2.34016008597848e-08
4626 2.33837018761562e-08
4627 2.33836451606351e-08
4628 2.33602719958981e-08
4629 2.33643823754903e-08
4630 2.33411252850146e-08
4631 2.33435999170695e-08
4632 2.33132215221588e-08
4633 2.33154147366577e-08
4634 2.33015515576085e-08
4635 2.33003139733512e-08
4636 2.32868246108353e-08
4637 2.32883931410299e-08
4638 2.32543445513045e-08
4639 2.32481804580686e-08
4640 2.32389033385516e-08
4641 2.32312770371834e-08
4642 2.32156578725551e-08
4643 2.32152068271319e-08
4644 2.31939401196257e-08
4645 2.32004917144479e-08
4646 2.31789832021434e-08
4647 2.31799589229809e-08
4648 2.31733882110063e-08
4649 2.31758048876429e-08
4650 2.31585231169618e-08
4651 2.31602065987602e-08
4652 2.31388582783154e-08
4653 2.31396367489367e-08
4654 2.31278689994952e-08
4655 2.31192958217719e-08
4656 2.31084755668576e-08
4657 2.31026239809751e-08
4658 2.30863401888826e-08
4659 2.30868126429584e-08
4660 2.30772278726477e-08
4661 2.30681956168155e-08
4662 2.30535469043502e-08
4663 2.30549900450683e-08
4664 2.30436474168982e-08
4665 2.30390406983361e-08
4666 2.3027818876642e-08
4667 2.30232307067979e-08
4668 2.30071417313127e-08
4669 2.29997759539913e-08
4670 2.29949083845327e-08
4671 2.29842807222269e-08
4672 2.29788771264339e-08
4673 2.29701268636973e-08
4674 2.29582453208366e-08
4675 2.29535599238773e-08
4676 2.2945172954536e-08
4677 2.29388652073226e-08
4678 2.29254003691892e-08
4679 2.2920201608656e-08
4680 2.29091228085565e-08
4681 2.29006327892023e-08
4682 2.28904783341477e-08
4683 2.2885039197007e-08
4684 2.28700931259596e-08
4685 2.28693268411462e-08
4686 2.28559447776888e-08
4687 2.28555027277366e-08
4688 2.28360740557321e-08
4689 2.28364735228581e-08
4690 2.28263567656484e-08
4691 2.28207356158805e-08
4692 2.2809025985282e-08
4693 2.28062496496761e-08
4694 2.27921997542069e-08
4695 2.27925123965633e-08
4696 2.27778895691699e-08
4697 2.27770640712777e-08
4698 2.27587344667768e-08
4699 2.27616229864225e-08
4700 2.27485184893794e-08
4701 2.27337210247924e-08
4702 2.27309432929701e-08
4703 2.27252855431459e-08
4704 2.27092334093015e-08
4705 2.27239572048177e-08
4706 2.26955947120189e-08
4707 2.26983244822065e-08
4708 2.26870563437842e-08
4709 2.26834642695906e-08
4710 2.26723367084958e-08
4711 2.26693148448476e-08
4712 2.26430552530132e-08
4713 2.26940478320614e-08
4714 2.26683158075502e-08
4715 2.2671745991687e-08
4716 2.2651083099845e-08
4717 2.26607762492392e-08
4718 2.26452092775276e-08
4719 2.26545303725345e-08
4720 2.26278589785522e-08
4721 2.26360405939374e-08
4722 2.26200293305112e-08
4723 2.26247202732566e-08
4724 2.25980471739717e-08
4725 2.26070618403185e-08
4726 2.25907151047977e-08
4727 2.25913928169064e-08
4728 2.25690476156615e-08
4729 2.25759256444746e-08
4730 2.25629770334024e-08
4731 2.25622765874789e-08
4732 2.25381418310633e-08
4733 2.2545848072042e-08
4734 2.25350368374677e-08
4735 2.25335571641949e-08
4736 2.25101687512108e-08
4737 2.25169976246775e-08
4738 2.25028950886497e-08
4739 2.25022869919655e-08
4740 2.24789175327089e-08
4741 2.24833880224651e-08
4742 2.24673302042788e-08
4743 2.24673843725043e-08
4744 2.24427576434039e-08
4745 2.2450084127712e-08
4746 2.2432510416337e-08
4747 2.24339558450026e-08
4748 2.24101192820569e-08
4749 2.24178397942865e-08
4750 2.240238763207e-08
4751 2.24031767999122e-08
4752 2.2379224798641e-08
4753 2.23875362763692e-08
4754 2.23717392877631e-08
4755 2.23747494416671e-08
4756 2.23525970746152e-08
4757 2.23668714056657e-08
4758 2.23450614917908e-08
4759 2.23470761966382e-08
4760 2.23321931898113e-08
4761 2.23380126200823e-08
4762 2.23258055385145e-08
4763 2.23296198242906e-08
4764 2.23067086118078e-08
4765 2.23900824067869e-08
4766 2.23235813550104e-08
4767 2.22528540234634e-08
4768 2.22795100377482e-08
4769 2.22898287205453e-08
4770 2.2273114748117e-08
4771 2.22761082042666e-08
4772 2.22603765642759e-08
4773 2.22651275585406e-08
4774 2.22451829401393e-08
4775 2.22830543172847e-08
4776 2.22362534323395e-08
4777 2.22364324073965e-08
4778 2.22181442701697e-08
4779 2.22263430380565e-08
4780 2.22347156189073e-08
4781 2.22410927399608e-08
4782 2.21862944194129e-08
4783 2.22266164904283e-08
4784 2.21788414975777e-08
4785 2.22148633959307e-08
4786 2.21673703357794e-08
4787 2.21720039519369e-08
4788 2.21548712424635e-08
4789 2.21953057213398e-08
4790 2.21395889710152e-08
4791 2.21427929858464e-08
4792 2.21217378921779e-08
4793 2.21297385678554e-08
4794 2.21078003761477e-08
4795 2.21159737527898e-08
4796 2.20940446382656e-08
4797 2.21017684367553e-08
4798 2.20800908010688e-08
4799 2.20908837675893e-08
4800 2.20726802311333e-08
4801 2.20628428095893e-08
4802 2.20437112510297e-08
4803 2.2049278406655e-08
4804 2.20299774120747e-08
4805 2.20342227308379e-08
4806 2.20139340783021e-08
4807 2.20186306982839e-08
4808 2.20009649218866e-08
4809 2.20099036738475e-08
4810 2.19831670662529e-08
4811 2.19909911720606e-08
4812 2.19724123944331e-08
4813 2.19823759941562e-08
4814 2.1956229314668e-08
4815 2.19653011690468e-08
4816 2.19403683132668e-08
4817 2.19279959203789e-08
4818 2.19221679067516e-08
4819 2.19395168912229e-08
4820 2.1919435045703e-08
4821 2.1924777904303e-08
4822 2.19071210771915e-08
4823 2.19166101302903e-08
4824 2.18910029694541e-08
4825 2.19010057342928e-08
4826 2.18778423324295e-08
4827 2.18865585850381e-08
4828 2.18652802637109e-08
4829 2.18731313132992e-08
4830 2.185165797286e-08
4831 2.18603912678361e-08
4832 2.18381148862079e-08
4833 2.18486104088811e-08
4834 2.18262014897164e-08
4835 2.18355836523187e-08
4836 2.18130890452528e-08
4837 2.182329176037e-08
4838 2.17990999651363e-08
4839 2.18096540685053e-08
4840 2.17859602393844e-08
4841 2.17971104170545e-08
4842 2.17732359821809e-08
4843 2.178385160434e-08
4844 2.17604597487764e-08
4845 2.17706905942805e-08
4846 2.17479303934454e-08
4847 2.1758709312536e-08
4848 2.17334759256005e-08
4849 2.17460579641227e-08
4850 2.1723074137725e-08
4851 2.17332786753843e-08
4852 2.17070705446076e-08
4853 2.17221779834631e-08
4854 2.16954363381205e-08
4855 2.17033981080306e-08
4856 2.16699466442094e-08
4857 2.16886405155492e-08
4858 2.16705833331332e-08
4859 2.16791213389911e-08
4860 2.1655914903107e-08
4861 2.16669641019962e-08
4862 2.16411050715237e-08
4863 2.16421406697975e-08
4864 2.16193200515136e-08
4865 2.16316233085934e-08
4866 2.16204169518619e-08
4867 2.16287922931713e-08
4868 2.15938830585571e-08
4869 2.16073707157705e-08
4870 2.15856502165934e-08
4871 2.15955970013226e-08
4872 2.15842094064556e-08
4873 2.15923230122428e-08
4874 2.15577342359552e-08
4875 2.15706531214721e-08
4876 2.15496752780098e-08
4877 2.15618348029523e-08
4878 2.1546705919917e-08
4879 2.15556100577885e-08
4880 2.15220909254299e-08
4881 2.15347324505899e-08
4882 2.15120377333733e-08
4883 2.15253708901741e-08
4884 2.15103272473982e-08
4885 2.15187356928936e-08
4886 2.14861490661633e-08
4887 2.14992962384031e-08
4888 2.14766423169976e-08
4889 2.14919464802676e-08
4890 2.14752588476586e-08
4891 2.14852314570635e-08
4892 2.14533196967182e-08
4893 2.14749957940796e-08
4894 2.14495972841178e-08
4895 2.14627410244361e-08
4896 2.14379934249109e-08
4897 2.14508605083097e-08
4898 2.14237689846186e-08
4899 2.14366332151883e-08
4900 2.141131527722e-08
4901 2.14242625276029e-08
4902 2.13991245203715e-08
4903 2.14129911952909e-08
4904 2.13869943053169e-08
4905 2.14015668511536e-08
4906 2.13763538923217e-08
4907 2.13903950516681e-08
4908 2.13642800197533e-08
4909 2.13674394373697e-08
4910 2.13515608287196e-08
4911 2.13677065055151e-08
4912 2.13428085196199e-08
4913 2.1356463268063e-08
4914 2.13325593385605e-08
4915 2.13458681876944e-08
4916 2.13208834658474e-08
4917 2.13355092597567e-08
4918 2.13088437988063e-08
4919 2.13247567906194e-08
4920 2.12980774207949e-08
4921 2.13134384559055e-08
4922 2.1287726077901e-08
4923 2.13020435069211e-08
4924 2.12772627179447e-08
4925 2.12908456411753e-08
4926 2.12662815677334e-08
4927 2.1279048109335e-08
4928 2.12568812223424e-08
4929 2.12680385089925e-08
4930 2.12368172505251e-08
4931 2.12626249478376e-08
4932 2.12292723773544e-08
4933 2.12407254984726e-08
4934 2.12267303609792e-08
4935 2.12412140463414e-08
4936 2.12122703224793e-08
4937 2.12183724705994e-08
4938 2.12052848880262e-08
4939 2.12134739072667e-08
4940 2.11832625112152e-08
4941 2.12106415204971e-08
4942 2.11790733217754e-08
4943 2.11875427176267e-08
4944 2.11640137877112e-08
4945 2.11773393594683e-08
4946 2.11556152471815e-08
4947 2.11789987183408e-08
4948 2.11461970316407e-08
4949 2.11558049798555e-08
4950 2.11359734372252e-08
4951 2.11529213771655e-08
4952 2.11181562157492e-08
4953 2.11413677391192e-08
4954 2.1113497854941e-08
4955 2.11343203950776e-08
4956 2.10979749049045e-08
4957 2.11176140538782e-08
4958 2.10976556118681e-08
4959 2.11097182329922e-08
4960 2.1083833626534e-08
4961 2.10946750378582e-08
4962 2.10726945262252e-08
4963 2.10846661659048e-08
4964 2.10615025437733e-08
4965 2.10908257827214e-08
4966 2.1068025208848e-08
4967 2.10812935392823e-08
4968 2.10556641349058e-08
4969 2.1060426004027e-08
4970 2.10497417540978e-08
4971 2.10603736192638e-08
4972 2.10309360220151e-08
4973 2.10466159806799e-08
4974 2.10278455377022e-08
4975 2.10373763209759e-08
4976 2.10177296473546e-08
4977 2.10257287669435e-08
4978 2.10065904475698e-08
4979 2.10164650908951e-08
4980 2.09950496206091e-08
4981 2.10027198050966e-08
4982 2.09839334246453e-08
4983 2.09995022437681e-08
4984 2.09756990834364e-08
4985 2.09814381371132e-08
4986 2.09639576418397e-08
4987 2.09792033345479e-08
4988 2.09470085934527e-08
4989 2.09686972354461e-08
4990 2.09387493441682e-08
4991 2.09592232351952e-08
4992 2.09285868493225e-08
4993 2.09510220159359e-08
4994 2.09203127887747e-08
4995 2.09400728259368e-08
4996 2.09077166530847e-08
4997 2.093160034633e-08
4998 2.0898460846297e-08
4999 2.09240002284616e-08
};
\addlegendentry{Train}
\addplot [semithick, black]
table {%
0 0.0136999320238829
1 0.00725397793576121
2 0.00361626571975648
3 0.00206828350201249
4 0.00157912413123995
5 0.00128151394892484
6 0.000972299894783646
7 0.000636491109617054
8 0.000398999283788726
9 0.000281849381281063
10 0.000228688048082404
11 0.000203205068828538
12 0.00018946903583128
13 0.000180647373781539
14 0.000173849082784727
15 0.000167821664945222
16 0.000161973352078348
17 0.000156042820890434
18 0.000149883097037673
19 0.000142970005981624
20 0.000134596877614968
21 0.000126220344100147
22 0.000117783180030528
23 0.000108916196040809
24 9.96621893136762e-05
25 9.01717357919551e-05
26 8.06669631856494e-05
27 7.14705893187784e-05
28 6.29579881206155e-05
29 5.54864818695933e-05
30 4.92828039568849e-05
31 4.4397238525562e-05
32 4.07058359996881e-05
33 3.79873927158769e-05
34 3.60024278052151e-05
35 3.45432054018602e-05
36 3.34547003149055e-05
37 3.26255249092355e-05
38 3.19766695611179e-05
39 3.14542521664407e-05
40 3.10164323309436e-05
41 3.06306865240913e-05
42 3.02754906442715e-05
43 2.99387629638659e-05
44 2.96185771730961e-05
45 2.92943950626068e-05
46 2.89551899186336e-05
47 2.85950336547103e-05
48 2.82102282653796e-05
49 2.77904600807233e-05
50 2.73296791419853e-05
51 2.68197509285528e-05
52 2.62580924754729e-05
53 2.56422408710932e-05
54 2.49586955760606e-05
55 2.42093028646195e-05
56 2.33883456530748e-05
57 2.24988525587833e-05
58 2.15411182580283e-05
59 2.05223386728903e-05
60 1.94432177522685e-05
61 1.83263164217351e-05
62 1.71822593983961e-05
63 1.60300078277942e-05
64 1.46560860230238e-05
65 1.33445164465229e-05
66 1.1433436156949e-05
67 9.59420412982581e-06
68 8.20235982246231e-06
69 7.17446164344437e-06
70 6.39771906207898e-06
71 5.79806555833784e-06
72 5.31561909156153e-06
73 4.91008495373535e-06
74 4.56213274446782e-06
75 4.25944244852872e-06
76 3.99191958422307e-06
77 3.75083482140326e-06
78 3.52777169609908e-06
79 3.326395244585e-06
80 3.14299791170924e-06
81 2.98137297249923e-06
82 2.83103190668044e-06
83 2.69107135864033e-06
84 2.56436715062591e-06
85 2.44709576691093e-06
86 2.34214621741557e-06
87 2.24576319851622e-06
88 2.15836030292849e-06
89 2.07907874028024e-06
90 2.00650879378372e-06
91 1.94133122022322e-06
92 1.88152557711874e-06
93 1.82628082256997e-06
94 1.77829792846751e-06
95 1.73221019394987e-06
96 1.68884673712455e-06
97 1.65142682817532e-06
98 1.61827324518526e-06
99 1.58704131081322e-06
100 1.55518955580192e-06
101 1.52485245052958e-06
102 1.49564016282966e-06
103 1.46852676152776e-06
104 1.44331215778948e-06
105 1.41924272156757e-06
106 1.39638973450928e-06
107 1.3743779163633e-06
108 1.35404286538687e-06
109 1.333376758339e-06
110 1.31110220991104e-06
111 1.28931105791708e-06
112 1.26907173125801e-06
113 1.25000349271431e-06
114 1.23105371585552e-06
115 1.21204652714368e-06
116 1.19384867502959e-06
117 1.17566878543585e-06
118 1.15776265374734e-06
119 1.14107285753562e-06
120 1.12401858132216e-06
121 1.10735413727525e-06
122 1.09089683064667e-06
123 1.07533605842036e-06
124 1.06028278423764e-06
125 1.04501464193163e-06
126 1.02934347978589e-06
127 1.01361172255565e-06
128 9.97942606772995e-07
129 9.83234826890111e-07
130 9.6894359558064e-07
131 9.55674522629124e-07
132 9.40452594022645e-07
133 9.26970812997752e-07
134 9.1364813670225e-07
135 9.0098552618656e-07
136 8.88481451966072e-07
137 8.76631020219065e-07
138 8.65373635861033e-07
139 8.53925087085372e-07
140 8.4495252394845e-07
141 8.34589400255936e-07
142 8.24859796466626e-07
143 8.15363875972253e-07
144 8.06127843588911e-07
145 7.96959398030594e-07
146 7.88590284628299e-07
147 7.80813195433439e-07
148 7.73515694163507e-07
149 7.66706762078684e-07
150 7.60510374675505e-07
151 7.53956044263759e-07
152 7.48300976738392e-07
153 7.42004374387761e-07
154 7.36133245027304e-07
155 7.30246654256916e-07
156 7.248455062836e-07
157 7.20061507308856e-07
158 7.15414671503822e-07
159 7.1070792273531e-07
160 7.05474406004214e-07
161 7.01037549788452e-07
162 6.97043219588522e-07
163 6.9296515903261e-07
164 6.88878913024382e-07
165 6.85164366132085e-07
166 6.80529240071337e-07
167 6.76270985877636e-07
168 6.72101407417358e-07
169 6.68121913349751e-07
170 6.64602737288078e-07
171 6.61020408188051e-07
172 6.57418524951936e-07
173 6.53901224723086e-07
174 6.50329980089737e-07
175 6.46550859073614e-07
176 6.43137070710509e-07
177 6.39780921574129e-07
178 6.36254810615355e-07
179 6.32793842214596e-07
180 6.2934520883573e-07
181 6.26152370841737e-07
182 6.2297721115101e-07
183 6.19956608716166e-07
184 6.1699745401711e-07
185 6.1400902495734e-07
186 6.11127632055286e-07
187 6.08268521773425e-07
188 6.0543908375621e-07
189 6.02686725414969e-07
190 5.99919530941406e-07
191 5.96682809828053e-07
192 5.93255322201003e-07
193 5.90439128700382e-07
194 5.87768511195463e-07
195 5.85024338306539e-07
196 5.82502025281428e-07
197 5.79913603360183e-07
198 5.77436708226742e-07
199 5.75052467866044e-07
200 5.7275286735603e-07
201 5.70496581531188e-07
202 5.68337327422341e-07
203 5.66237133625691e-07
204 5.64223853416479e-07
205 5.62028560580075e-07
206 5.59938428068563e-07
207 5.57887972263416e-07
208 5.55913459265867e-07
209 5.54006533093343e-07
210 5.52149856503092e-07
211 5.50345532701613e-07
212 5.48618857010297e-07
213 5.4611729183307e-07
214 5.44310296390904e-07
215 5.42583848073264e-07
216 5.40924361303041e-07
217 5.39299492174905e-07
218 5.37683774837205e-07
219 5.36095910774748e-07
220 5.34532091478468e-07
221 5.33023921889253e-07
222 5.31552700522298e-07
223 5.30132354015223e-07
224 5.28429154655896e-07
225 5.27266081462585e-07
226 5.25736595591297e-07
227 5.24394465628575e-07
228 5.22948084835662e-07
229 5.21767958616692e-07
230 5.20385583513416e-07
231 5.19047205216339e-07
232 5.17512205533421e-07
233 5.16259206051473e-07
234 5.15048213856062e-07
235 5.1384478183536e-07
236 5.12602582602995e-07
237 5.11148925852467e-07
238 5.09869948928099e-07
239 5.08647531205497e-07
240 5.07175684560934e-07
241 5.06587014115212e-07
242 5.0532219120214e-07
243 5.04224374253681e-07
244 5.03153785302857e-07
245 5.0192414846606e-07
246 5.00800524605438e-07
247 4.99750342441985e-07
248 4.98737961152074e-07
249 4.97733253723709e-07
250 4.96737413868686e-07
251 4.95772269459849e-07
252 4.94820824314957e-07
253 4.93892173381028e-07
254 4.92990977818408e-07
255 4.92097797177848e-07
256 4.91213711484306e-07
257 4.90331558467005e-07
258 4.89474530240841e-07
259 4.88628359107679e-07
260 4.87804300064454e-07
261 4.8693425469537e-07
262 4.86044939407293e-07
263 4.85144596495957e-07
264 4.84215433971258e-07
265 4.83383473692811e-07
266 4.82561574699503e-07
267 4.81736321944481e-07
268 4.80921812595625e-07
269 4.80115943446435e-07
270 4.79328036817606e-07
271 4.78519609714567e-07
272 4.77723176572908e-07
273 4.76933507798094e-07
274 4.76130310289591e-07
275 4.75331944471691e-07
276 4.74529457505923e-07
277 4.73712134407833e-07
278 4.72893106007177e-07
279 4.72124412453923e-07
280 4.7131820224422e-07
281 4.70505284511091e-07
282 4.69689666715567e-07
283 4.68881012238853e-07
284 4.68081566395995e-07
285 4.67294825057252e-07
286 4.66657496644984e-07
287 4.65847449504508e-07
288 4.63988641286051e-07
289 4.62856405647472e-07
290 4.61560290432317e-07
291 4.60871632412818e-07
292 4.60214408803949e-07
293 4.59551841913708e-07
294 4.58901126876299e-07
295 4.58239185263665e-07
296 4.57593500868825e-07
297 4.56957849337414e-07
298 4.5633282752533e-07
299 4.55726024028991e-07
300 4.55122972198296e-07
301 4.54516452919052e-07
302 4.53929004606834e-07
303 4.53344057405047e-07
304 4.52765874570105e-07
305 4.52188061217385e-07
306 4.51616301688773e-07
307 4.51018877356546e-07
308 4.50467069867955e-07
309 4.49930524837328e-07
310 4.49382071110449e-07
311 4.48833333166476e-07
312 4.48296646027302e-07
313 4.47759560984196e-07
314 4.4721946323989e-07
315 4.46682292931655e-07
316 4.46144525767522e-07
317 4.45628614897942e-07
318 4.45098436330227e-07
319 4.44549357325741e-07
320 4.44002779431685e-07
321 4.43459924781564e-07
322 4.42922328147688e-07
323 4.42383083054665e-07
324 4.41856286670372e-07
325 4.41355297198243e-07
326 4.40625029796138e-07
327 4.40120032862978e-07
328 4.39604121993398e-07
329 4.39103899907423e-07
330 4.38599755625546e-07
331 4.38102262023676e-07
332 4.37597947211543e-07
333 4.37103722106258e-07
334 4.36608530662852e-07
335 4.3611262867671e-07
336 4.35665697295917e-07
337 4.35149701161208e-07
338 4.34606306498608e-07
339 4.34163752061067e-07
340 4.33633260854549e-07
341 4.33133692467891e-07
342 4.32630372415588e-07
343 4.32127421845507e-07
344 4.31620748031492e-07
345 4.31123936550648e-07
346 4.30589210509424e-07
347 4.30101351867052e-07
348 4.295909548091e-07
349 4.29088316877824e-07
350 4.28564675303278e-07
351 4.28053766654557e-07
352 4.27563747962267e-07
353 4.27134040137389e-07
354 4.26629185312777e-07
355 4.26141241405276e-07
356 4.25655258595725e-07
357 4.25145174176578e-07
358 4.24662175646517e-07
359 4.24180967684151e-07
360 4.23694388018703e-07
361 4.2322895410507e-07
362 4.22755562112798e-07
363 4.22318976234237e-07
364 4.21849051690515e-07
365 4.21384868332098e-07
366 4.20914346932477e-07
367 4.20448230897819e-07
368 4.19977993715293e-07
369 4.19505767013106e-07
370 4.19036439325282e-07
371 4.18576775018664e-07
372 4.18105656763146e-07
373 4.17643633454645e-07
374 4.17175215261523e-07
375 4.1670580230857e-07
376 4.16236503042455e-07
377 4.15777577700283e-07
378 4.1530765315656e-07
379 4.14831902162405e-07
380 4.14359448086543e-07
381 4.13888272987606e-07
382 4.13416501032771e-07
383 4.12948253369905e-07
384 4.12482108913537e-07
385 4.12019147688625e-07
386 4.11561387636539e-07
387 4.11107890840867e-07
388 4.10665251138198e-07
389 4.1022008190339e-07
390 4.09780341215082e-07
391 4.09342987950367e-07
392 4.08914104355063e-07
393 4.0848161120266e-07
394 4.08048464350941e-07
395 4.07616113307085e-07
396 4.07178561090404e-07
397 4.06732056035253e-07
398 4.06300245003877e-07
399 4.05858344265653e-07
400 4.05419342541791e-07
401 4.04981477686306e-07
402 4.04548814003647e-07
403 4.04096056172421e-07
404 4.03651881697442e-07
405 4.03218308520081e-07
406 4.02787406983407e-07
407 4.02359034978872e-07
408 4.01929469262541e-07
409 4.01501978330998e-07
410 4.01078693812451e-07
411 4.00648730192188e-07
412 4.00219136054147e-07
413 3.99836437736667e-07
414 3.99405934103925e-07
415 3.98971309323315e-07
416 3.98534240275694e-07
417 3.98103708221242e-07
418 3.97671186647131e-07
419 3.97248697936448e-07
420 3.96822059656188e-07
421 3.96017838966145e-07
422 3.95622294036002e-07
423 3.9519676420241e-07
424 3.94764327893427e-07
425 3.94337916986842e-07
426 3.93905622786406e-07
427 3.93472703308362e-07
428 3.93040124890831e-07
429 3.92615447708522e-07
430 3.92186080944157e-07
431 3.91759215290222e-07
432 3.91332321214577e-07
433 3.90901931268672e-07
434 3.90478760436963e-07
435 3.9005729490782e-07
436 3.89634777775427e-07
437 3.89215188079106e-07
438 3.88795172057144e-07
439 3.88374559179283e-07
440 3.8787572975707e-07
441 3.87462307571695e-07
442 3.87115619560063e-07
443 3.86708592259311e-07
444 3.86293436349661e-07
445 3.85878763609071e-07
446 3.85464090868481e-07
447 3.85050043405499e-07
448 3.84633295880121e-07
449 3.84221010563124e-07
450 3.83809975801341e-07
451 3.83395786229812e-07
452 3.8298125559777e-07
453 3.82571528234621e-07
454 3.8215790709728e-07
455 3.81776857238947e-07
456 3.81472062827015e-07
457 3.81118155701188e-07
458 3.80733013116696e-07
459 3.80324053139702e-07
460 3.799030139362e-07
461 3.79485271650992e-07
462 3.79048941567817e-07
463 3.78612696749769e-07
464 3.78177304583005e-07
465 3.77741173451795e-07
466 3.77303223331182e-07
467 3.76863823703388e-07
468 3.76431415816114e-07
469 3.75997956325591e-07
470 3.75564837895581e-07
471 3.75130582597194e-07
472 3.74696014660003e-07
473 3.7426156040965e-07
474 3.73803402453632e-07
475 3.73414451360077e-07
476 3.72946459492596e-07
477 3.72490887912136e-07
478 3.72047736618697e-07
479 3.71610155980306e-07
480 3.71177208080553e-07
481 3.70746505495845e-07
482 3.70140639915917e-07
483 3.69684272527593e-07
484 3.69251864640319e-07
485 3.688126923862e-07
486 3.68372070624901e-07
487 3.67927299294024e-07
488 3.67492248187773e-07
489 3.67075472240685e-07
490 3.66642836979736e-07
491 3.66207387969553e-07
492 3.65771313681762e-07
493 3.65339246855001e-07
494 3.64903002036954e-07
495 3.64462465540782e-07
496 3.64022781695894e-07
497 3.63587474794258e-07
498 3.63156516414165e-07
499 3.62731896075275e-07
500 3.62310345281003e-07
501 3.61890556632716e-07
502 3.61473155408021e-07
503 3.61064280696155e-07
504 3.60657281817112e-07
505 3.60251647180121e-07
506 3.59848627340398e-07
507 3.59447710707173e-07
508 3.59117450443591e-07
509 3.58761468532975e-07
510 3.58364786734455e-07
511 3.57963813257811e-07
512 3.57564573505442e-07
513 3.57159933628282e-07
514 3.5675483900377e-07
515 3.56353751840288e-07
516 3.55936407459012e-07
517 3.55529579110225e-07
518 3.55126360318536e-07
519 3.54720725681545e-07
520 3.54310543571046e-07
521 3.53899508809263e-07
522 3.53488786686285e-07
523 3.53075819248261e-07
524 3.52661317037928e-07
525 3.5225352235102e-07
526 3.5194565839447e-07
527 3.51538858467393e-07
528 3.51138595533484e-07
529 3.50717641595111e-07
530 3.50298250850756e-07
531 3.49874312632892e-07
532 3.49461686255381e-07
533 3.49040988112392e-07
534 3.48582886999793e-07
535 3.48132800809253e-07
536 3.47729695704402e-07
537 3.47321048366211e-07
538 3.46912997883919e-07
539 3.46499263059741e-07
540 3.46135720974416e-07
541 3.45728551565117e-07
542 3.45314873584357e-07
543 3.44899234505647e-07
544 3.44501756899263e-07
545 3.44096150683981e-07
546 3.43687730719466e-07
547 3.4329187315052e-07
548 3.4288319739062e-07
549 3.4250197700203e-07
550 3.42129283126269e-07
551 3.4170560070379e-07
552 3.41341092280345e-07
553 3.410489171074e-07
554 3.40647858365628e-07
555 3.40288863753813e-07
556 3.39869529852876e-07
557 3.39457443487845e-07
558 3.39048710884526e-07
559 3.38624204232474e-07
560 3.38207541972224e-07
561 3.37795171390098e-07
562 3.37307056952341e-07
563 3.35871760626105e-07
564 3.35305372800576e-07
565 3.34828229142659e-07
566 3.34377205035707e-07
567 3.33942665520226e-07
568 3.33524610596214e-07
569 3.33112865291696e-07
570 3.32690319737594e-07
571 3.3226987738999e-07
572 3.31851964574525e-07
573 3.31434364397865e-07
574 3.31017645294196e-07
575 3.30600130382663e-07
576 3.30182928109934e-07
577 3.29769420659431e-07
578 3.29345510863277e-07
579 3.28932088677902e-07
580 3.28496440715753e-07
581 3.28093506141158e-07
582 3.27643761011132e-07
583 3.27192537952214e-07
584 3.26747056078602e-07
585 3.26363135627616e-07
586 3.25930500366667e-07
587 3.2550073569837e-07
588 3.25069805739986e-07
589 3.24717092325955e-07
590 3.24315209354609e-07
591 3.23901645060687e-07
592 3.23483732245222e-07
593 3.23079518693703e-07
594 3.22662344842684e-07
595 3.22249690043463e-07
596 3.21830469829365e-07
597 3.21409515890991e-07
598 3.20987652457916e-07
599 3.2056729537544e-07
600 3.20120818741998e-07
601 3.19355535793875e-07
602 3.18915766683858e-07
603 3.18405483312745e-07
604 3.17974979680002e-07
605 3.17531345217503e-07
606 3.16757393648004e-07
607 3.16266124400499e-07
608 3.15930549277255e-07
609 3.1546176160191e-07
610 3.14991950745025e-07
611 3.14307015969462e-07
612 3.13698109266625e-07
613 3.13221136138964e-07
614 3.12771902599707e-07
615 3.12330314500286e-07
616 3.11872668135038e-07
617 3.11430625288267e-07
618 3.10979089590546e-07
619 3.10551598659004e-07
620 3.10125670921479e-07
621 3.096890281995e-07
622 3.09233314510493e-07
623 3.08817675431783e-07
624 3.0840766385154e-07
625 3.07983981429061e-07
626 3.07546883959731e-07
627 3.07107939079287e-07
628 3.06669903693546e-07
629 3.06229992474982e-07
630 3.05791473920181e-07
631 3.05400476463547e-07
632 3.04927510796915e-07
633 3.04449855548228e-07
634 3.04251329907856e-07
635 3.03574637428028e-07
636 3.03350418562331e-07
637 3.02914003214028e-07
638 3.02486313330519e-07
639 3.0201022127585e-07
640 3.01541234648539e-07
641 3.01088846299535e-07
642 3.00650782492085e-07
643 3.00183870649562e-07
644 2.99727844321751e-07
645 2.9927082323411e-07
646 2.98815507449035e-07
647 2.98360646411311e-07
648 2.9790859912282e-07
649 2.97444245234146e-07
650 2.96992283210784e-07
651 2.96539013788788e-07
652 2.96100893137918e-07
653 2.95643843628568e-07
654 2.94806824285843e-07
655 2.94305721126875e-07
656 2.93836592391017e-07
657 2.93384232463723e-07
658 2.92930650402923e-07
659 2.92481473707085e-07
660 2.92031785420477e-07
661 2.91584228762076e-07
662 2.9113596156094e-07
663 2.9068689855194e-07
664 2.90238432398837e-07
665 2.89798350650017e-07
666 2.893570751894e-07
667 2.88953373228651e-07
668 2.88519032665135e-07
669 2.88037995233026e-07
670 2.87567672785372e-07
671 2.87024334966191e-07
672 2.86532895188429e-07
673 2.86071156097023e-07
674 2.85620131990072e-07
675 2.85176810166377e-07
676 2.84837824438e-07
677 2.84366677760772e-07
678 2.83896582686793e-07
679 2.83453715610449e-07
680 2.82981233112878e-07
681 2.82549478924921e-07
682 2.82066366708023e-07
683 2.81602069662767e-07
684 2.81146441238889e-07
685 2.80704000488186e-07
686 2.8025667120346e-07
687 2.79805760783347e-07
688 2.79344391174163e-07
689 2.78891548077809e-07
690 2.78388199603796e-07
691 2.77927341585382e-07
692 2.77461026598758e-07
693 2.77000026471796e-07
694 2.76544540156465e-07
695 2.76120033504412e-07
696 2.75684129746878e-07
697 2.75239727898224e-07
698 2.74792682830594e-07
699 2.74346461992536e-07
700 2.73901889613626e-07
701 2.73459534128051e-07
702 2.73021129260087e-07
703 2.72547111990207e-07
704 2.72133718226542e-07
705 2.71726264600147e-07
706 2.71317674105376e-07
707 2.70908174115903e-07
708 2.70481137931711e-07
709 2.7006402092411e-07
710 2.69617004278189e-07
711 2.69197556690415e-07
712 2.68766257249808e-07
713 2.68332627229029e-07
714 2.67882683147036e-07
715 2.67461956582338e-07
716 2.6706317157732e-07
717 2.66739874632549e-07
718 2.6644841000234e-07
719 2.6607267500367e-07
720 2.65660446530092e-07
721 2.65185605030638e-07
722 2.6480125825401e-07
723 2.64315303866169e-07
724 2.63845947756636e-07
725 2.63403279632257e-07
726 2.62955353491634e-07
727 2.62522632965556e-07
728 2.62114582483264e-07
729 2.61710169979779e-07
730 2.61248146671278e-07
731 2.60766171322757e-07
732 2.60304659605026e-07
733 2.5983294449361e-07
734 2.59324309581643e-07
735 2.58819426335322e-07
736 2.58342083725438e-07
737 2.57875058196078e-07
738 2.57420538218867e-07
739 2.56970622558583e-07
740 2.56524657515911e-07
741 2.56103731999247e-07
742 2.55660069115038e-07
743 2.55151888950422e-07
744 2.54694384693721e-07
745 2.54229007623508e-07
746 2.5377133283655e-07
747 2.53339038636113e-07
748 2.52907767617216e-07
749 2.52476212381225e-07
750 2.520288830965e-07
751 2.51588545552295e-07
752 2.51149970154074e-07
753 2.50713640070899e-07
754 2.50017706093786e-07
755 2.49189838541497e-07
756 2.48698427185445e-07
757 2.48362198362884e-07
758 2.47920752372011e-07
759 2.47475895776006e-07
760 2.47037547751461e-07
761 2.46585926788612e-07
762 2.46127171976696e-07
763 2.45671714083073e-07
764 2.45213584548765e-07
765 2.44767420554126e-07
766 2.44319693365469e-07
767 2.43875518890491e-07
768 2.4343444238184e-07
769 2.4299683332174e-07
770 2.42562066432583e-07
771 2.4211263394136e-07
772 2.41682073465199e-07
773 2.41238637954666e-07
774 2.40684329355645e-07
775 2.40226910364072e-07
776 2.3977042928891e-07
777 2.39492862874613e-07
778 2.3906341084512e-07
779 2.38627023918525e-07
780 2.38216728121188e-07
781 2.37769725686121e-07
782 2.37340472608594e-07
783 2.36697545119569e-07
784 2.36251040064417e-07
785 2.3578134289437e-07
786 2.35315610552789e-07
787 2.34866917026011e-07
788 2.34403714216569e-07
789 2.3393114645387e-07
790 2.334488726774e-07
791 2.33007568795074e-07
792 2.32548643452901e-07
793 2.32100319408346e-07
794 2.31671108963383e-07
795 2.31230217195844e-07
796 2.3079476818566e-07
797 2.30351446361965e-07
798 2.29907769266902e-07
799 2.2946997546569e-07
800 2.28970591820143e-07
801 2.28467953888867e-07
802 2.28056336482041e-07
803 2.27596146373799e-07
804 2.27110973582967e-07
805 2.26637482114711e-07
806 2.26199489361534e-07
807 2.25748962634498e-07
808 2.25324868097232e-07
809 2.24904454171337e-07
810 2.24480373844926e-07
811 2.24063995801771e-07
812 2.23682008027026e-07
813 2.2328811155603e-07
814 2.22892410306486e-07
815 2.22512213099435e-07
816 2.22115119186128e-07
817 2.21735504624121e-07
818 2.21337430161839e-07
819 2.20927617533562e-07
820 2.20526274574695e-07
821 2.2013958300704e-07
822 2.19791289168825e-07
823 2.19430290826494e-07
824 2.19076369489812e-07
825 2.18688199993267e-07
826 2.18312166566648e-07
827 2.17950102410214e-07
828 2.18083371805733e-07
829 2.17702194049707e-07
830 2.17312987160767e-07
831 2.1693961116398e-07
832 2.16563009303172e-07
833 2.16186464285784e-07
834 2.15812548276517e-07
835 2.15385966839676e-07
836 2.14999772651936e-07
837 2.14615269555907e-07
838 2.14231477002613e-07
839 2.138414458841e-07
840 2.13486757161263e-07
841 2.13115754377213e-07
842 2.12743984207009e-07
843 2.1236994030005e-07
844 2.11988734122315e-07
845 2.11669672012249e-07
846 2.11361395940912e-07
847 2.11021387030996e-07
848 2.10717004733851e-07
849 2.10363793939905e-07
850 2.100307483488e-07
851 2.09734537293116e-07
852 2.09423163255451e-07
853 2.09118596217195e-07
854 2.08790311262419e-07
855 2.08480983587833e-07
856 2.08138942525693e-07
857 2.07811766017585e-07
858 2.07485427949905e-07
859 2.07161292564706e-07
860 2.06840681471476e-07
861 2.06546417302889e-07
862 2.06227966259576e-07
863 2.05909287842587e-07
864 2.0551243551381e-07
865 2.0524326771465e-07
866 2.04882667276252e-07
867 2.0452100102375e-07
868 2.04216320298656e-07
869 2.03813797838848e-07
870 2.03439796564453e-07
871 2.03034502987975e-07
872 2.0276631573779e-07
873 2.02360610046526e-07
874 2.02021581685585e-07
875 2.01737591964957e-07
876 2.01330777827025e-07
877 2.01004553446182e-07
878 2.00609846956468e-07
879 2.00209228751191e-07
880 1.9985853327853e-07
881 1.99771349684852e-07
882 1.99030537828548e-07
883 1.9862602584908e-07
884 1.98399575879193e-07
885 1.98317607669196e-07
886 1.97415303659909e-07
887 1.97188256834124e-07
888 1.96541478203471e-07
889 1.9623767855137e-07
890 1.9605953127666e-07
891 1.95635863065036e-07
892 1.95074974840281e-07
893 1.94765647165696e-07
894 1.94856554003309e-07
895 1.9441337428816e-07
896 1.94086780425096e-07
897 1.93510743429215e-07
898 1.93428377315286e-07
899 1.93146490801155e-07
900 1.92879824112424e-07
901 1.9220392744046e-07
902 1.91888105405269e-07
903 1.91710753938423e-07
904 1.91520314274385e-07
905 1.9116053806556e-07
906 1.90498042229592e-07
907 1.90156399071384e-07
908 1.89862419119891e-07
909 1.89680150697313e-07
910 1.89253654525601e-07
911 1.88958338753764e-07
912 1.88709307735735e-07
913 1.88433403991439e-07
914 1.88058209005249e-07
915 1.87838537613061e-07
916 1.87512810612134e-07
917 1.87152366493137e-07
918 1.87012332730774e-07
919 1.86654062872549e-07
920 1.86439322646947e-07
921 1.86166289495304e-07
922 1.85780862693719e-07
923 1.85579267508729e-07
924 1.85287703402537e-07
925 1.85035005983991e-07
926 1.84697398708522e-07
927 1.84371387490501e-07
928 1.8426638348501e-07
929 1.83932982622537e-07
930 1.83677457243903e-07
931 1.83278316967517e-07
932 1.83023615818456e-07
933 1.82883411525836e-07
934 1.82465385023534e-07
935 1.8239072119286e-07
936 1.82121951297631e-07
937 1.81875407179177e-07
938 1.81716742986282e-07
939 1.81256837095134e-07
940 1.81016147848823e-07
941 1.80780460823371e-07
942 1.80519450054817e-07
943 1.80259505100366e-07
944 1.79946624712102e-07
945 1.79421618895503e-07
946 1.79220350560172e-07
947 1.78946905293742e-07
948 1.78525070282376e-07
949 1.78455792365639e-07
950 1.78068020773026e-07
951 1.77781487309403e-07
952 1.77681215518533e-07
953 1.7716216405006e-07
954 1.77131383338747e-07
955 1.76778826244117e-07
956 1.76552049424572e-07
957 1.76270560814373e-07
958 1.76057696421594e-07
959 1.75854509620876e-07
960 1.75674315983088e-07
961 1.75529905277472e-07
962 1.75192525375678e-07
963 1.75061572349478e-07
964 1.74716092260496e-07
965 1.74368494754162e-07
966 1.74183909962267e-07
967 1.73922714452601e-07
968 1.73596774288853e-07
969 1.73473523545908e-07
970 1.73333020825339e-07
971 1.72834575096203e-07
972 1.72641449580624e-07
973 1.72359094108288e-07
974 1.72161037426122e-07
975 1.71862978959325e-07
976 1.71758983924519e-07
977 1.71397545045693e-07
978 1.71088217371107e-07
979 1.70994368886568e-07
980 1.70711430769188e-07
981 1.70473427374418e-07
982 1.70241122532389e-07
983 1.69979543329646e-07
984 1.69640571812124e-07
985 1.6937157454322e-07
986 1.69151604723083e-07
987 1.69082298384637e-07
988 1.68748485407377e-07
989 1.68652093179844e-07
990 1.68389547638981e-07
991 1.68078258866444e-07
992 1.68057198379756e-07
993 1.67608050105628e-07
994 1.67414626162099e-07
995 1.67150844276875e-07
996 1.67088131775017e-07
997 1.66716461080796e-07
998 1.66416370461775e-07
999 1.66427909675804e-07
1000 1.6599673813289e-07
1001 1.65842351407264e-07
1002 1.65848533129065e-07
1003 1.65273803531818e-07
1004 1.65423770681628e-07
1005 1.65170803256842e-07
1006 1.64670140634371e-07
1007 1.64676265512753e-07
1008 1.64533446422865e-07
1009 1.63983386869404e-07
1010 1.63757604809689e-07
1011 1.63542836162378e-07
1012 1.63323065294207e-07
1013 1.63395071695049e-07
1014 1.62888298405051e-07
1015 1.62661763170036e-07
1016 1.62421045502015e-07
1017 1.62243168233545e-07
1018 1.6199339825107e-07
1019 1.62040194595647e-07
1020 1.61607218274185e-07
1021 1.61376320306772e-07
1022 1.61149969812868e-07
1023 1.60925509362642e-07
1024 1.60947848826254e-07
1025 1.60476943733556e-07
1026 1.60401313564762e-07
1027 1.60038908347815e-07
1028 1.59849932401812e-07
1029 1.59692291390456e-07
1030 1.59467333560315e-07
1031 1.59285093559447e-07
1032 1.59147603540077e-07
1033 1.58805576688792e-07
1034 1.58634236413491e-07
1035 1.58719046794431e-07
1036 1.58309617859231e-07
1037 1.57896423047532e-07
1038 1.57652792154295e-07
1039 1.5748710779917e-07
1040 1.57352047835957e-07
1041 1.57114953935888e-07
1042 1.56859485400673e-07
1043 1.57068328121568e-07
1044 1.56560687969431e-07
1045 1.56143769913797e-07
1046 1.56012660568194e-07
1047 1.5582352830279e-07
1048 1.5558401855742e-07
1049 1.55402730683818e-07
1050 1.55484997321764e-07
1051 1.55004883595211e-07
1052 1.54806897967319e-07
1053 1.54957660924993e-07
1054 1.54344874658818e-07
1055 1.54176746036683e-07
1056 1.53926123402925e-07
1057 1.54101414295837e-07
1058 1.53541407144075e-07
1059 1.53732031549225e-07
1060 1.53101410660383e-07
1061 1.52932670971495e-07
1062 1.52869986891346e-07
1063 1.5268292941073e-07
1064 1.52322286339768e-07
1065 1.52506416384313e-07
1066 1.52320552615492e-07
1067 1.51690883853917e-07
1068 1.51517269841861e-07
1069 1.51293832573174e-07
1070 1.51137896864384e-07
1071 1.50915809626895e-07
1072 1.50834821965873e-07
1073 1.50667091247669e-07
1074 1.50455690572926e-07
1075 1.50276719068643e-07
1076 1.50092390072132e-07
1077 1.49834448848196e-07
1078 1.49693264006601e-07
1079 1.49829716633576e-07
1080 1.49734134424762e-07
1081 1.49551382833124e-07
1082 1.49364311141653e-07
1083 1.48963110291334e-07
1084 1.48596100757459e-07
1085 1.48837841607019e-07
1086 1.48319472259573e-07
1087 1.48128492583055e-07
1088 1.4789523561376e-07
1089 1.47755883972422e-07
1090 1.47550238693839e-07
1091 1.47898859381712e-07
1092 1.47215644119569e-07
1093 1.47491704183267e-07
1094 1.47164101349517e-07
1095 1.47121156146568e-07
1096 1.46421271551844e-07
1097 1.46258472000227e-07
1098 1.46295690228726e-07
1099 1.4593371133742e-07
1100 1.45920097338603e-07
1101 1.45872036227956e-07
1102 1.45372595738991e-07
1103 1.45602612633411e-07
1104 1.45103967952309e-07
1105 1.4549661386809e-07
1106 1.45027755138472e-07
1107 1.44947776448134e-07
1108 1.4441606310811e-07
1109 1.44299107773804e-07
1110 1.44510920563334e-07
1111 1.43914533623501e-07
1112 1.43793570828166e-07
1113 1.43936901508823e-07
1114 1.43383630302196e-07
1115 1.43122747431335e-07
1116 1.42935618896445e-07
1117 1.43056070101011e-07
1118 1.43096798410625e-07
1119 1.42714341677674e-07
1120 1.42831765970186e-07
1121 1.42096141075854e-07
1122 1.42263274938159e-07
1123 1.42396430646841e-07
1124 1.41628078154099e-07
1125 1.42179473527904e-07
1126 1.41667655384481e-07
1127 1.41746639314988e-07
1128 1.4101081546869e-07
1129 1.41462322744701e-07
1130 1.4066250741962e-07
1131 1.40843596341256e-07
1132 1.40239379220475e-07
1133 1.40734371711915e-07
1134 1.40440292284438e-07
1135 1.40225694167384e-07
1136 1.39597148063331e-07
1137 1.40191374953247e-07
1138 1.39467715598585e-07
1139 1.39499192641779e-07
1140 1.39119578079772e-07
1141 1.3958421618554e-07
1142 1.3900637441111e-07
1143 1.38642846536641e-07
1144 1.38985612352371e-07
1145 1.38849017616849e-07
1146 1.38722015208259e-07
1147 1.37933739097207e-07
1148 1.38154518936062e-07
1149 1.38267822080707e-07
1150 1.37570651759233e-07
1151 1.37449461590222e-07
1152 1.37373675102026e-07
1153 1.37060666816069e-07
1154 1.36922537308237e-07
1155 1.36819053864201e-07
1156 1.3666965514858e-07
1157 1.36513179427311e-07
1158 1.36703050657161e-07
1159 1.36880245804605e-07
1160 1.36455824417681e-07
1161 1.36606843170739e-07
1162 1.3583114366611e-07
1163 1.35637463927196e-07
1164 1.35910141807472e-07
1165 1.35958572400341e-07
1166 1.35197240069829e-07
1167 1.3497039219601e-07
1168 1.34833783249633e-07
1169 1.35255518785016e-07
1170 1.34495294901171e-07
1171 1.34310425892181e-07
1172 1.34174953814181e-07
1173 1.34106201699069e-07
1174 1.33927883894103e-07
1175 1.33775984068052e-07
1176 1.33631459675598e-07
1177 1.33486466324939e-07
1178 1.33397961121773e-07
1179 1.33627281684312e-07
1180 1.32101192207301e-07
1181 1.32018044496363e-07
1182 1.31810438119828e-07
1183 1.31888114651701e-07
1184 1.31933234115422e-07
1185 1.31261899127821e-07
1186 1.31070024167457e-07
1187 1.30948052401436e-07
1188 1.31101842271164e-07
1189 1.30724600921894e-07
1190 1.30495152461663e-07
1191 1.30889674210266e-07
1192 1.30180055180062e-07
1193 1.30034791823164e-07
1194 1.29880490362666e-07
1195 1.3017194078202e-07
1196 1.29569841078592e-07
1197 1.29400987702866e-07
1198 1.2990024345072e-07
1199 1.29262630821358e-07
1200 1.29043414176522e-07
1201 1.28912105878953e-07
1202 1.28653596220829e-07
1203 1.28771446838982e-07
1204 1.28359260997968e-07
1205 1.28171137703248e-07
1206 1.28515125652484e-07
1207 1.28418818690079e-07
1208 1.27674496752661e-07
1209 1.27444153008582e-07
1210 1.27805904526213e-07
1211 1.27095532320709e-07
1212 1.26971656300157e-07
1213 1.26784883036635e-07
1214 1.26633210584259e-07
1215 1.26478965967181e-07
1216 1.26310226278292e-07
1217 1.2675741345447e-07
1218 1.25980704979156e-07
1219 1.25779138215876e-07
1220 1.25619621371698e-07
1221 1.25926987948333e-07
1222 1.25359690628102e-07
1223 1.2521508097052e-07
1224 1.25122099348118e-07
1225 1.24866886608288e-07
1226 1.24702040693592e-07
1227 1.24519274891099e-07
1228 1.24253531907925e-07
1229 1.24582555827146e-07
1230 1.23899738468936e-07
1231 1.23687954101115e-07
1232 1.23549810382428e-07
1233 1.23410941910151e-07
1234 1.23204330293447e-07
1235 1.23030702070537e-07
1236 1.22987671602459e-07
1237 1.22732188856389e-07
1238 1.22945792213613e-07
1239 1.23098729432058e-07
1240 1.22205079833293e-07
1241 1.22078660069747e-07
1242 1.22175634942323e-07
1243 1.22356098586351e-07
1244 1.21523314078331e-07
1245 1.21469156511012e-07
1246 1.21186502610726e-07
1247 1.21080262260875e-07
1248 1.20897794886332e-07
1249 1.20755615284907e-07
1250 1.20590314622859e-07
1251 1.20439054285271e-07
1252 1.20257752200814e-07
1253 1.2010852401545e-07
1254 1.20586619800633e-07
1255 1.19775563689473e-07
1256 1.19610959359306e-07
1257 1.19518873020752e-07
1258 1.19183653168875e-07
1259 1.19028825906753e-07
1260 1.18823486161546e-07
1261 1.18714332586478e-07
1262 1.19127449238476e-07
1263 1.18366855872409e-07
1264 1.18828182849029e-07
1265 1.18047140063027e-07
1266 1.17897499762876e-07
1267 1.17747745775887e-07
1268 1.17650714059891e-07
1269 1.17511127939451e-07
1270 1.1736874938606e-07
1271 1.17225113172026e-07
1272 1.17135463995055e-07
1273 1.16969872010486e-07
1274 1.16836396557574e-07
1275 1.16693556151404e-07
1276 1.16530713967222e-07
1277 1.16428743979213e-07
1278 1.16229990965167e-07
1279 1.1610384120786e-07
1280 1.15955252510958e-07
1281 1.158355047437e-07
1282 1.1566983459943e-07
1283 1.15547194923238e-07
1284 1.15362972508137e-07
1285 1.15478755446929e-07
1286 1.15110587728395e-07
1287 1.15169058290121e-07
1288 1.14819385998999e-07
1289 1.14889893154668e-07
1290 1.14541627738163e-07
1291 1.14624299385468e-07
1292 1.14266740069979e-07
1293 1.14174994791938e-07
1294 1.14257105110482e-07
1295 1.14571228948535e-07
1296 1.13204450258308e-07
1297 1.1269493427335e-07
1298 1.12570170074378e-07
1299 1.12565167853518e-07
1300 1.12540348595758e-07
1301 1.12301698607098e-07
1302 1.12253957240682e-07
1303 1.12465571078246e-07
1304 1.124548845155e-07
1305 1.12125960072262e-07
1306 1.12121519180164e-07
1307 1.12106519623012e-07
1308 1.11932763502409e-07
1309 1.11869766783457e-07
1310 1.11672740388258e-07
1311 1.11656952128669e-07
1312 1.11297708826896e-07
1313 1.11368720467908e-07
1314 1.11203810604366e-07
1315 1.10965906685578e-07
1316 1.10825332910736e-07
1317 1.10522073271113e-07
1318 1.10417403220708e-07
1319 1.103200304442e-07
1320 1.10050514479099e-07
1321 1.09980767604156e-07
1322 1.09950342164211e-07
1323 1.09927384528419e-07
1324 1.09756072674827e-07
1325 1.0969666419669e-07
1326 1.0966999042239e-07
1327 1.09563899286513e-07
1328 1.09457317876149e-07
1329 1.09309304718863e-07
1330 1.08957166844448e-07
1331 1.08917333818681e-07
1332 1.08652322694525e-07
1333 1.086190977162e-07
1334 1.08462650416641e-07
1335 1.08630622719375e-07
1336 1.08423755307285e-07
1337 1.0843014308648e-07
1338 1.08201945181463e-07
1339 1.07892766720852e-07
1340 1.07850546271493e-07
1341 1.08030469903042e-07
1342 1.07728702403165e-07
1343 1.07643714386541e-07
1344 1.07205458732551e-07
1345 1.07307144503466e-07
1346 1.06898760066088e-07
1347 1.06776788300067e-07
1348 1.06655569709346e-07
1349 1.07237354995959e-07
1350 1.06985147851901e-07
1351 1.07227528189924e-07
1352 1.06756232298721e-07
1353 1.06972329660948e-07
1354 1.06643220476599e-07
1355 1.06499932428505e-07
1356 1.06227560081606e-07
1357 1.06231780705457e-07
1358 1.06232590724176e-07
1359 1.06035059843634e-07
1360 1.0575396203194e-07
1361 1.05822969942437e-07
1362 1.05749613510397e-07
1363 1.05630206803653e-07
1364 1.0549813111993e-07
1365 1.05376038561644e-07
1366 1.05047242016099e-07
1367 1.04876008322208e-07
1368 1.04917191379172e-07
1369 1.04631524777687e-07
1370 1.04495661901183e-07
1371 1.04397599898221e-07
1372 1.04279067159041e-07
1373 1.04170105430512e-07
1374 1.04037631842857e-07
1375 1.03935320794335e-07
1376 1.0380548332023e-07
1377 1.04068540451863e-07
1378 1.03842417331634e-07
1379 1.03823055042085e-07
1380 1.03729604461478e-07
1381 1.03677770368904e-07
1382 1.03477326263146e-07
1383 1.03367192139103e-07
1384 1.03284371277823e-07
1385 1.03314178545588e-07
1386 1.03049934807586e-07
1387 1.02812322211321e-07
1388 1.03066305712218e-07
1389 1.02685525860124e-07
1390 1.02853533689995e-07
1391 1.02671009472033e-07
1392 1.02332819551521e-07
1393 1.02471830132345e-07
1394 1.02138081103931e-07
1395 1.02034022120279e-07
1396 1.01941019181595e-07
1397 1.01837095201063e-07
1398 1.01777892780319e-07
1399 1.01549517239619e-07
1400 1.01664078044905e-07
1401 1.01337356284148e-07
1402 1.01516675954372e-07
1403 1.01389190376722e-07
1404 1.01247792372305e-07
1405 1.01067435309687e-07
1406 1.00609206299396e-07
1407 1.0052927734705e-07
1408 1.00701676331028e-07
1409 1.00447266504489e-07
1410 1.003051224302e-07
1411 1.00213853215791e-07
1412 1.00231424937647e-07
1413 1.00118576540353e-07
1414 1.00084129428524e-07
1415 1.00303090277976e-07
1416 1.00130975511092e-07
1417 1.00073620501462e-07
1418 9.98764946302799e-08
1419 9.99694975689636e-08
1420 9.99685667579797e-08
1421 1.00121802404374e-07
1422 1.00034021954798e-07
1423 9.98229339188583e-08
1424 9.97327873619724e-08
1425 9.97860851725818e-08
1426 9.84984538376921e-08
1427 9.86016601700612e-08
1428 9.84332118036946e-08
1429 9.8403951653836e-08
1430 9.83692700629035e-08
1431 9.82619781098037e-08
1432 9.81628005547464e-08
1433 9.83460353154442e-08
1434 9.82377983405058e-08
1435 9.81258310162048e-08
1436 9.80497247837775e-08
1437 9.78932490625084e-08
1438 9.78206813329052e-08
1439 9.77165868221164e-08
1440 9.75819389736898e-08
1441 9.74093481431737e-08
1442 9.70791802501481e-08
1443 9.71947002881279e-08
1444 9.71153184536888e-08
1445 9.69798747973982e-08
1446 9.68008748714055e-08
1447 9.66790594247868e-08
1448 9.65915276651685e-08
1449 9.64737623121437e-08
1450 9.64045057116891e-08
1451 9.63079926918908e-08
1452 9.60867083676931e-08
1453 9.61494208695512e-08
1454 9.5933650356983e-08
1455 9.58599244427205e-08
1456 9.57672483536953e-08
1457 9.54315382273307e-08
1458 9.5483521533879e-08
1459 9.51986152131212e-08
1460 9.50279002154275e-08
1461 9.4980421749824e-08
1462 9.49601854927096e-08
1463 9.47128029338273e-08
1464 9.49326306454168e-08
1465 9.46832940940112e-08
1466 9.44456033380447e-08
1467 9.45710283417611e-08
1468 9.41997697623265e-08
1469 9.40364159873752e-08
1470 9.39500210961342e-08
1471 9.4089742219694e-08
1472 9.3706752579692e-08
1473 9.30883246041958e-08
1474 9.30171353275e-08
1475 9.27784071791393e-08
1476 9.26262799794131e-08
1477 9.24496319498758e-08
1478 9.21000733455912e-08
1479 9.16431872610701e-08
1480 9.13207216513001e-08
1481 9.0671854025004e-08
1482 9.02743693131924e-08
1483 9.0183476686434e-08
1484 8.98801673088201e-08
1485 8.99463401538014e-08
1486 8.98468144328035e-08
1487 8.98518166536633e-08
1488 8.9882767895233e-08
1489 8.95892071639537e-08
1490 8.96687026852305e-08
1491 8.9417540038994e-08
1492 8.95433132086509e-08
1493 8.92772860083824e-08
1494 8.91700437932741e-08
1495 8.90668232500502e-08
1496 8.91539642111638e-08
1497 8.90316087520659e-08
1498 8.88539020138523e-08
1499 8.88874325255529e-08
1500 8.85380941895164e-08
1501 8.84143389612291e-08
1502 8.85630768721057e-08
1503 8.82108963651262e-08
1504 8.8328420133621e-08
1505 8.79698589528743e-08
1506 8.79032739931063e-08
1507 8.78433397133449e-08
1508 8.76944596939211e-08
1509 8.78204247101166e-08
1510 8.74923387073068e-08
1511 8.76305605856942e-08
1512 8.75227499363973e-08
1513 8.72021956865865e-08
1514 8.70947829412216e-08
1515 8.69828511440573e-08
1516 8.70983640766099e-08
1517 8.67746621224796e-08
1518 8.68741949489049e-08
1519 8.66633271812134e-08
1520 8.6471040106062e-08
1521 8.65492566504145e-08
1522 8.640739679322e-08
1523 8.6307373692307e-08
1524 8.62751150521035e-08
1525 8.59987849821664e-08
1526 8.58692530414373e-08
1527 8.59567066413547e-08
1528 8.58494004774002e-08
1529 8.57056576819559e-08
1530 8.55525144061176e-08
1531 8.53365662578653e-08
1532 8.54463877431044e-08
1533 8.51341042107379e-08
1534 8.52813073493053e-08
1535 8.49819130621654e-08
1536 8.49199892627439e-08
1537 8.47242631607514e-08
1538 8.46704395485176e-08
1539 8.45200887056308e-08
1540 8.44118588361198e-08
1541 8.43098248992646e-08
1542 8.42075493778793e-08
1543 8.41609377744135e-08
1544 8.4024463831156e-08
1545 8.39343599068343e-08
1546 8.38286808857447e-08
1547 8.37282385646176e-08
1548 8.36835880591025e-08
1549 8.3699653430358e-08
1550 8.35012059496876e-08
1551 8.34477162925396e-08
1552 8.33213036344205e-08
1553 8.32198807643181e-08
1554 8.31188557981477e-08
1555 8.30198061407827e-08
1556 8.29203585794858e-08
1557 8.28257000762278e-08
1558 8.27314465823292e-08
1559 8.27158501692793e-08
1560 8.26458261826701e-08
1561 8.25194348408331e-08
1562 8.2422687341932e-08
1563 8.23278227812807e-08
1564 8.22291497115657e-08
1565 8.21557790686711e-08
1566 8.20555854375016e-08
1567 8.1963101195015e-08
1568 8.1903053228416e-08
1569 8.17935870145448e-08
1570 8.16628116240281e-08
1571 8.15249023844444e-08
1572 8.14648615232727e-08
1573 8.14014526895335e-08
1574 8.12175926512282e-08
1575 8.11847371551266e-08
1576 8.0999399187931e-08
1577 8.09860623007808e-08
1578 8.0823930659335e-08
1579 8.08247477834811e-08
1580 8.06812394671397e-08
1581 8.06238276140903e-08
1582 8.05660675951003e-08
1583 8.03850852548749e-08
1584 8.03343098709774e-08
1585 8.0281964187634e-08
1586 8.00410902002113e-08
1587 8.0075061248408e-08
1588 7.99771839865571e-08
1589 7.98990527073329e-08
1590 7.97891388515382e-08
1591 7.95606993619913e-08
1592 7.96205199549149e-08
1593 7.9511337958138e-08
1594 7.94555674588082e-08
1595 7.93401255805293e-08
1596 7.92838363850024e-08
1597 7.919900468778e-08
1598 7.91249803455685e-08
1599 7.89789496025151e-08
1600 7.89130041312092e-08
1601 7.88963774311924e-08
1602 7.87550575864771e-08
1603 7.8688771054658e-08
1604 7.86753773240889e-08
1605 7.85314924200975e-08
1606 7.84644882401153e-08
1607 7.84553293442514e-08
1608 7.83707605478412e-08
1609 7.82923450515227e-08
1610 7.8215535381787e-08
1611 7.81381714887175e-08
1612 7.80632802843684e-08
1613 7.797962098266e-08
1614 7.79022428787357e-08
1615 7.78295117243033e-08
1616 7.77545707819627e-08
1617 7.75904993588483e-08
1618 7.76230208998641e-08
1619 7.74428983163489e-08
1620 7.74891901755836e-08
1621 7.73859483160777e-08
1622 7.73574342360916e-08
1623 7.72746986399397e-08
1624 7.70747874412336e-08
1625 7.71278010347487e-08
1626 7.69369279396415e-08
1627 7.70137660310866e-08
1628 7.68021450880951e-08
1629 7.68702008713262e-08
1630 7.66613368341496e-08
1631 7.66555316999984e-08
1632 7.65712542261099e-08
1633 7.6516158742379e-08
1634 7.64275895903666e-08
1635 7.63812124660035e-08
1636 7.64106147244092e-08
1637 7.61552314543223e-08
1638 7.6165001416939e-08
1639 7.60713732006479e-08
1640 7.60298988211616e-08
1641 7.59325260446531e-08
1642 7.58737357386963e-08
1643 7.57892877345512e-08
1644 7.56844329430351e-08
1645 7.57694778030782e-08
1646 7.56745848207174e-08
1647 7.55075291181129e-08
1648 7.53922861917999e-08
1649 7.53232853867303e-08
1650 7.53241877760047e-08
1651 7.50706092844666e-08
1652 7.49999458093953e-08
1653 7.51033724100125e-08
1654 7.48474278111644e-08
1655 7.4970792240947e-08
1656 7.47130428635501e-08
1657 7.49169188907217e-08
1658 7.4677593886463e-08
1659 7.46766701809065e-08
1660 7.45293817772108e-08
1661 7.44620436421428e-08
1662 7.43874934983069e-08
1663 7.43230756938829e-08
1664 7.42754195925954e-08
1665 7.42555954502677e-08
1666 7.41939771842226e-08
1667 7.41948298355055e-08
1668 7.41178283192312e-08
1669 7.40204129101585e-08
1670 7.39737231469917e-08
1671 7.38460812499397e-08
1672 7.35497991399825e-08
1673 7.35203045110211e-08
1674 7.35412015728798e-08
1675 7.33253031626191e-08
1676 7.32565581529343e-08
1677 7.32384393131724e-08
1678 7.31819937982436e-08
1679 7.31000469045284e-08
1680 7.29960873968594e-08
1681 7.29115825492954e-08
1682 7.29073406091629e-08
1683 7.28654683257446e-08
1684 7.28990059428725e-08
1685 7.26663884620393e-08
1686 7.26533002648466e-08
1687 7.25377589105847e-08
1688 7.25203577189859e-08
1689 7.24158724096924e-08
1690 7.23911455224879e-08
1691 7.22796400509651e-08
1692 7.22763076055344e-08
1693 7.21293389460698e-08
1694 7.20543980037291e-08
1695 7.20181461133507e-08
1696 7.19398514092973e-08
1697 7.19390698122879e-08
1698 7.18894668239045e-08
1699 7.17691932550224e-08
1700 7.17841359687554e-08
1701 7.16459069849407e-08
1702 7.16442798420758e-08
1703 7.15135328732686e-08
1704 7.15332575396133e-08
1705 7.13939556362675e-08
1706 7.14542238711147e-08
1707 7.1292909353815e-08
1708 7.13018408760036e-08
1709 7.11674488229619e-08
1710 7.1197455042693e-08
1711 7.11221517235572e-08
1712 7.10753127464159e-08
1713 7.09320389091772e-08
1714 7.08788192582688e-08
1715 7.08981318098267e-08
1716 7.08569842799989e-08
1717 7.07006009292854e-08
1718 7.09038232571402e-08
1719 7.0951685415821e-08
1720 7.07234590890948e-08
1721 7.07688130319184e-08
1722 7.05506906228948e-08
1723 7.05979132931134e-08
1724 7.05532130496067e-08
1725 7.04784639538047e-08
1726 7.04198512835319e-08
1727 7.03614588815071e-08
1728 7.03006506341808e-08
1729 7.02484541648118e-08
1730 7.0473561208928e-08
1731 7.03661200418537e-08
1732 7.03065339280329e-08
1733 7.02887348325021e-08
1734 7.0277586416978e-08
1735 7.02629563420487e-08
1736 7.02271520935938e-08
1737 7.0182032629873e-08
1738 7.01347246945261e-08
1739 7.00865214753321e-08
1740 7.00359237271186e-08
1741 6.99892197530971e-08
1742 6.99342663779134e-08
1743 6.98686335454113e-08
1744 6.98018141065404e-08
1745 6.97383768510917e-08
1746 6.97646527214602e-08
1747 6.9688468329332e-08
1748 6.96011070999702e-08
1749 6.95438586717501e-08
1750 6.95003166129027e-08
1751 6.9447011696866e-08
1752 6.94011035307085e-08
1753 6.93556430064746e-08
1754 6.93082782277088e-08
1755 6.92322430495551e-08
1756 6.91313744027866e-08
1757 6.91539696617838e-08
1758 6.89748205218166e-08
1759 6.88774690615901e-08
1760 6.88458356989941e-08
1761 6.87942218746684e-08
1762 6.87411159105977e-08
1763 6.87033150370553e-08
1764 6.87353534090107e-08
1765 6.86519356918325e-08
1766 6.8723913670965e-08
1767 6.88017536276675e-08
1768 6.86749785927532e-08
1769 6.85536818423316e-08
1770 6.84355896396482e-08
1771 6.84380410120866e-08
1772 6.83899017417389e-08
1773 6.83270044987694e-08
1774 6.82115057770716e-08
1775 6.82557086406632e-08
1776 6.82902694393306e-08
1777 6.81637715160832e-08
1778 6.80815759324105e-08
1779 6.82034340115933e-08
1780 6.80433700495087e-08
1781 6.79741987141824e-08
1782 6.80144367493085e-08
1783 6.78805704978913e-08
1784 6.79229898992162e-08
1785 6.77901326184838e-08
1786 6.81149430192818e-08
1787 6.80927811913534e-08
1788 6.79589149399362e-08
1789 6.78686546962126e-08
1790 6.78021905287096e-08
1791 6.77365505907801e-08
1792 6.76787550446534e-08
1793 6.76029827673119e-08
1794 6.75883953249468e-08
1795 6.74696849500833e-08
1796 6.73105731152646e-08
1797 6.73267450679305e-08
1798 6.7468604925125e-08
1799 6.74463294103589e-08
1800 6.74091538144239e-08
1801 6.73704079190429e-08
1802 6.73282443131029e-08
1803 6.72837927595538e-08
1804 6.72506672572126e-08
1805 6.71498483484356e-08
1806 6.69825652721556e-08
1807 6.71383020289795e-08
1808 6.71335556035046e-08
1809 6.69184743173901e-08
1810 6.70592754659083e-08
1811 6.68374724455134e-08
1812 6.70063400320942e-08
1813 6.69992203938818e-08
1814 6.68757422772615e-08
1815 6.69175577172609e-08
1816 6.68593145292107e-08
1817 6.68689636995623e-08
1818 6.67954367372658e-08
1819 6.6786796537599e-08
1820 6.68064501496701e-08
1821 6.655559303681e-08
1822 6.67592203740242e-08
1823 6.65502923880013e-08
1824 6.64068409150786e-08
1825 6.656046025455e-08
1826 6.65776909158922e-08
1827 6.62802506212756e-08
1828 6.62796750816597e-08
1829 6.65306600922122e-08
1830 6.6207455517997e-08
1831 6.64692620944152e-08
1832 6.6274417065415e-08
1833 6.63858230609549e-08
1834 6.61965344761484e-08
1835 6.62771810766571e-08
1836 6.599049839906e-08
1837 6.62389396666185e-08
1838 6.59837269267882e-08
1839 6.6143037713573e-08
1840 6.59848211626013e-08
1841 6.5998008835777e-08
1842 6.60215064840486e-08
1843 6.57846328522282e-08
1844 6.58951506693484e-08
1845 6.56922480857247e-08
1846 6.56810712484912e-08
1847 6.56242136187757e-08
1848 6.58497114613965e-08
1849 6.56183161140689e-08
1850 6.56772272122907e-08
1851 6.55983782849034e-08
1852 6.54205365435701e-08
1853 6.56453451597372e-08
1854 6.53798579719478e-08
1855 6.55481855460494e-08
1856 6.52626468422568e-08
1857 6.54719300996476e-08
1858 6.52981242410533e-08
1859 6.52195453199056e-08
1860 6.5348011446531e-08
1861 6.50361471343786e-08
1862 6.51189751010861e-08
1863 6.52600888884081e-08
1864 6.50993499107244e-08
1865 6.48990479135136e-08
1866 6.49866649382602e-08
1867 6.51264500106663e-08
1868 6.49754312576079e-08
1869 6.47176747747835e-08
1870 6.47996216684987e-08
1871 6.49760849569248e-08
1872 6.46650377689184e-08
1873 6.47397158104468e-08
1874 6.48835580818741e-08
1875 6.45053717107658e-08
1876 6.45392432829794e-08
1877 6.44404138938626e-08
1878 6.44515765202414e-08
1879 6.46687965399906e-08
1880 6.43495070562494e-08
1881 6.46476934207385e-08
1882 6.43371933506387e-08
1883 6.42329069933112e-08
1884 6.45527507003862e-08
1885 6.45255795461708e-08
1886 6.40812984897821e-08
1887 6.4293267598714e-08
1888 6.42583586341061e-08
1889 6.42406945416951e-08
1890 6.41486508357048e-08
1891 6.39999839791017e-08
1892 6.40074517832545e-08
1893 6.39942854263609e-08
1894 6.37673238657044e-08
1895 6.38196979707573e-08
1896 6.40647641603209e-08
1897 6.37114681012463e-08
1898 6.396476237569e-08
1899 6.39113153511062e-08
1900 6.38642134731526e-08
1901 6.35065759979625e-08
1902 6.38199395552874e-08
1903 6.34232506513399e-08
1904 6.37328767538747e-08
1905 6.34595096471458e-08
1906 6.35978096852341e-08
1907 6.32693257784922e-08
1908 6.33303400832119e-08
1909 6.36247960983383e-08
1910 6.31404191153706e-08
1911 6.34869863347376e-08
1912 6.34224193163391e-08
1913 6.33049523912632e-08
1914 6.30434655590761e-08
1915 6.32398098332487e-08
1916 6.32234744557536e-08
1917 6.31938021911083e-08
1918 6.31619911928283e-08
1919 6.28471426011856e-08
1920 6.31035774745214e-08
1921 6.32578149861729e-08
1922 6.28749106112991e-08
1923 6.31443555221267e-08
1924 6.27878407044591e-08
1925 6.30876968443772e-08
1926 6.27248795126434e-08
1927 6.26596659003553e-08
1928 6.29600833690347e-08
1929 6.2918680043822e-08
1930 6.25894571726349e-08
1931 6.28264942292844e-08
1932 6.2494343922026e-08
1933 6.24515692493333e-08
1934 6.27500398309166e-08
1935 6.2698603642275e-08
1936 6.23398719312718e-08
1937 6.23711784442094e-08
1938 6.26138714210356e-08
1939 6.2224295049873e-08
1940 6.22344060730029e-08
1941 6.24964968665154e-08
1942 6.22403319994191e-08
1943 6.25663858500047e-08
1944 6.2172176740205e-08
1945 6.21960225544171e-08
1946 6.21385751742309e-08
1947 6.24088656309141e-08
1948 6.23688762857455e-08
1949 6.23606553062928e-08
1950 6.23058440396562e-08
1951 6.22207068090574e-08
1952 6.21858191607316e-08
1953 6.21531839328782e-08
1954 6.21159017555328e-08
1955 6.18522975059932e-08
1956 6.18508408933849e-08
1957 6.20700504327942e-08
1958 6.16888584659137e-08
1959 6.19952515990008e-08
1960 6.15966726513761e-08
1961 6.18524751416771e-08
1962 6.18061619661603e-08
1963 6.21262898903296e-08
1964 6.1743328672037e-08
1965 6.17325710550176e-08
1966 6.1999607225971e-08
1967 6.16104074424584e-08
1968 6.19474676000209e-08
1969 6.16228135186248e-08
1970 6.18573352539897e-08
1971 6.14893167494301e-08
1972 6.14966353396085e-08
1973 6.17713311612533e-08
1974 6.13856698805648e-08
1975 6.13782233926941e-08
1976 6.17606588093622e-08
1977 6.13421491379995e-08
1978 6.13404651517158e-08
1979 6.16661068875146e-08
1980 6.13663644344342e-08
1981 6.12038206782017e-08
1982 6.11875066169887e-08
1983 6.15370723267006e-08
1984 6.12420834045224e-08
1985 6.10663235534048e-08
1986 6.10697270531091e-08
1987 6.12011206158058e-08
1988 6.14295245782159e-08
1989 6.13719208786279e-08
1990 6.13247976843923e-08
1991 6.12832877777691e-08
1992 6.124744089675e-08
1993 6.1207600765556e-08
1994 6.11730754940254e-08
1995 6.11352035662094e-08
1996 6.10989872029677e-08
1997 6.10656840649426e-08
1998 6.10305548320866e-08
1999 6.09959442954278e-08
2000 6.09621579883424e-08
2001 6.0929167489121e-08
2002 6.08971930660118e-08
2003 6.08657089173903e-08
2004 6.0834288717615e-08
2005 6.08019092851464e-08
2006 6.07715477940474e-08
2007 6.07427850241038e-08
2008 6.0714356209246e-08
2009 6.06817920356661e-08
2010 6.06521126655934e-08
2011 6.06275705195003e-08
2012 6.05540790843406e-08
2013 6.03579763946982e-08
2014 6.05300058964531e-08
2015 6.04548162641549e-08
2016 6.02775926950017e-08
2017 6.04481940058577e-08
2018 6.0264405021826e-08
2019 6.03826251222017e-08
2020 6.02092455892489e-08
2021 6.03374985530536e-08
2022 6.01569638547517e-08
2023 6.02872205490712e-08
2024 6.01057763560675e-08
2025 6.02430390017616e-08
2026 6.00541696371693e-08
2027 6.01867427008074e-08
2028 5.99991309968573e-08
2029 6.0132968826565e-08
2030 5.99491585262513e-08
2031 6.00788609972369e-08
2032 5.98963438847022e-08
2033 6.00328178279597e-08
2034 5.98496825432449e-08
2035 5.99795129119229e-08
2036 5.97938836222056e-08
2037 5.99276219759304e-08
2038 5.97640905652952e-08
2039 5.98933667106394e-08
2040 5.97097695731463e-08
2041 5.98385625494302e-08
2042 5.9661758200491e-08
2043 5.97863589746339e-08
2044 5.95649645163121e-08
2045 5.97362515009081e-08
2046 5.95161928629295e-08
2047 5.96866129853879e-08
2048 5.94657869612547e-08
2049 5.96542619746288e-08
2050 5.94356563965448e-08
2051 5.96081051185138e-08
2052 5.93867603981835e-08
2053 5.95624847221643e-08
2054 5.93849414087799e-08
2055 5.96645577388699e-08
2056 5.95288405236261e-08
2057 5.95629607857973e-08
2058 5.94221276628559e-08
2059 5.94851634616589e-08
2060 5.94606888171256e-08
2061 5.94339191195559e-08
2062 5.94127449460302e-08
2063 5.93307767360329e-08
2064 5.93638098678184e-08
2065 5.93167754914248e-08
2066 5.92754005879215e-08
2067 5.92359370443774e-08
2068 5.92139528521329e-08
2069 5.91718354314708e-08
2070 5.91295794549751e-08
2071 5.90876148010011e-08
2072 5.90465383254468e-08
2073 5.90030957425824e-08
2074 5.89592694666408e-08
2075 5.89150204177713e-08
2076 5.88513664467882e-08
2077 5.87914499305953e-08
2078 5.89758286650977e-08
2079 5.89406532469638e-08
2080 5.88445239202429e-08
2081 5.87625521575319e-08
2082 5.87065791535224e-08
2083 5.85978732203785e-08
2084 5.86914588041054e-08
2085 5.87736508350645e-08
2086 5.87259130213624e-08
2087 5.86397739255062e-08
2088 5.85715973500101e-08
2089 5.85094603877678e-08
2090 5.84547912296784e-08
2091 5.83828736466785e-08
2092 5.83279273769222e-08
2093 5.82830104178811e-08
2094 5.8238857292281e-08
2095 5.81928958354183e-08
2096 5.81529846499507e-08
2097 5.80924108817271e-08
2098 5.80664121230257e-08
2099 5.8016734527655e-08
2100 5.79997845306934e-08
2101 5.79568784075946e-08
2102 5.79112864329545e-08
2103 5.78930432482139e-08
2104 5.78613850166221e-08
2105 5.78138426021724e-08
2106 5.77941001722593e-08
2107 5.77419996261597e-08
2108 5.7725802804498e-08
2109 5.76916114880532e-08
2110 5.766058919221e-08
2111 5.76368286431261e-08
2112 5.7581548418284e-08
2113 5.75732848062671e-08
2114 5.74283660625952e-08
2115 5.74048613088962e-08
2116 5.74015359688929e-08
2117 5.73721159469187e-08
2118 5.7316466239854e-08
2119 5.72808254162283e-08
2120 5.72412339749917e-08
2121 5.72240850260641e-08
2122 5.71435379015384e-08
2123 5.7290694854828e-08
2124 5.72726044367755e-08
2125 5.71633940182892e-08
2126 5.71069520560741e-08
2127 5.70893519125093e-08
2128 5.70158427137812e-08
2129 5.71652130076927e-08
2130 5.71330431853312e-08
2131 5.71136595794997e-08
2132 5.7092883309906e-08
2133 5.7070828063388e-08
2134 5.69779885495336e-08
2135 5.68539917367161e-08
2136 5.69970701747025e-08
2137 5.68136826473165e-08
2138 5.69450335774491e-08
2139 5.67644029558778e-08
2140 5.68955833557538e-08
2141 5.67127855788385e-08
2142 5.68439340042914e-08
2143 5.6662464942292e-08
2144 5.67955176222767e-08
2145 5.66147626557267e-08
2146 5.67424471853428e-08
2147 5.65484690184803e-08
2148 5.6672906367794e-08
2149 5.66189655160088e-08
2150 5.65297781918161e-08
2151 5.64317623741317e-08
2152 5.65615678738141e-08
2153 5.6509616541689e-08
2154 5.63612267967528e-08
2155 5.64902506994258e-08
2156 5.63919790863565e-08
2157 5.62872735088149e-08
2158 5.64284761139788e-08
2159 5.62012978377879e-08
2160 5.62576047968832e-08
2161 5.61601538606737e-08
2162 5.60514443748161e-08
2163 5.60387647396965e-08
2164 5.6019601402113e-08
2165 5.60020509965398e-08
2166 5.60068649235745e-08
2167 5.59812143308136e-08
2168 5.59661543775292e-08
2169 5.59484796269771e-08
2170 5.59300126212747e-08
2171 5.59118902287992e-08
2172 5.58776775960723e-08
2173 5.5860692071974e-08
2174 5.5838359713789e-08
2175 5.58231789682395e-08
2176 5.59183455095535e-08
2177 5.57929347166919e-08
2178 5.58388819626998e-08
2179 5.59279129674906e-08
2180 5.57975781134701e-08
2181 5.57712418469691e-08
2182 5.57312063165227e-08
2183 5.5712842339517e-08
2184 5.58056463262346e-08
2185 5.56648416250027e-08
2186 5.56386225980532e-08
2187 5.56167982779243e-08
2188 5.57053247973727e-08
2189 5.55439250149448e-08
2190 5.55342971608752e-08
2191 5.55184307415857e-08
2192 5.54829711063576e-08
2193 5.5474693283486e-08
2194 5.54394006258008e-08
2195 5.54106520667119e-08
2196 5.53842696149331e-08
2197 5.53648185075417e-08
2198 5.53185266483069e-08
2199 5.53151870974489e-08
2200 5.52800543118792e-08
2201 5.52649943585948e-08
2202 5.5216467842456e-08
2203 5.52777308371333e-08
2204 5.52125172248452e-08
2205 5.51986154562201e-08
2206 5.51895489309118e-08
2207 5.51726131448049e-08
2208 5.51296466255735e-08
2209 5.50990293390896e-08
2210 5.50262839738025e-08
2211 5.5017938649371e-08
2212 5.49757608325763e-08
2213 5.49370007263406e-08
2214 5.49290959384052e-08
2215 5.48993384086316e-08
2216 5.48882503892401e-08
2217 5.48736274197381e-08
2218 5.481883391667e-08
2219 5.4740954880117e-08
2220 5.47343717016702e-08
2221 5.47496945557668e-08
2222 5.470026209764e-08
2223 5.46940697176979e-08
2224 5.46499308029524e-08
2225 5.45023688403035e-08
2226 5.44603011576328e-08
2227 5.44608589336804e-08
2228 5.44199210139595e-08
2229 5.44626637122292e-08
2230 5.44186171680394e-08
2231 5.44163896165628e-08
2232 5.43707976419228e-08
2233 5.43658984497597e-08
2234 5.43238982686489e-08
2235 5.43023865873238e-08
2236 5.42799440950148e-08
2237 5.42543290293906e-08
2238 5.42094475974864e-08
2239 5.4229460033639e-08
2240 5.42055040853029e-08
2241 5.41819922261766e-08
2242 5.41394520325866e-08
2243 5.41167572976065e-08
2244 5.41154072664085e-08
2245 5.4089309031724e-08
2246 5.40678328775357e-08
2247 5.4042633479412e-08
2248 5.40193667575295e-08
2249 5.39908739938255e-08
2250 5.39647082575812e-08
2251 5.39460387471991e-08
2252 5.39206759242461e-08
2253 5.39005284849736e-08
2254 5.38779083569807e-08
2255 5.38512914260991e-08
2256 5.38304085750951e-08
2257 5.38066728950071e-08
2258 5.37827133939572e-08
2259 5.37617275142566e-08
2260 5.37422835122925e-08
2261 5.37185407267771e-08
2262 5.36547020146827e-08
2263 5.36150714935957e-08
2264 5.35661044409608e-08
2265 5.35654862687807e-08
2266 5.35512754140655e-08
2267 5.35370716647776e-08
2268 5.35169277782188e-08
2269 5.34750910219373e-08
2270 5.34501616300531e-08
2271 5.34251398676133e-08
2272 5.34056283640894e-08
2273 5.33807131830599e-08
2274 5.3357396723186e-08
2275 5.33321014017929e-08
2276 5.33077049169606e-08
2277 5.32756416760094e-08
2278 5.32589794488558e-08
2279 5.32350661330838e-08
2280 5.32051522839083e-08
2281 5.31794697167243e-08
2282 5.31629602562589e-08
2283 5.31339452436441e-08
2284 5.31077155585535e-08
2285 5.30784056707034e-08
2286 5.30541974796961e-08
2287 5.3033371472111e-08
2288 5.29947605798498e-08
2289 5.29644132996054e-08
2290 5.29532755422224e-08
2291 5.2927486393628e-08
2292 5.29105115276707e-08
2293 5.28638146590765e-08
2294 5.28481969297445e-08
2295 5.28122434673151e-08
2296 5.27763219793087e-08
2297 5.27397538974128e-08
2298 5.2715805054504e-08
2299 5.2700563912822e-08
2300 5.26646317666746e-08
2301 5.26368140185696e-08
2302 5.26251326959937e-08
2303 5.258707957978e-08
2304 5.25609742396682e-08
2305 5.25490797542716e-08
2306 5.25107850535278e-08
2307 5.24861754058747e-08
2308 5.24471523988268e-08
2309 5.24320498129782e-08
2310 5.24017806924348e-08
2311 5.24066656737432e-08
2312 5.23655359074837e-08
2313 5.234894118189e-08
2314 5.23118011130919e-08
2315 5.22907548372586e-08
2316 5.22590255513933e-08
2317 5.22373504452389e-08
2318 5.22090140009368e-08
2319 5.22037666428332e-08
2320 5.21771852390884e-08
2321 5.2152060447952e-08
2322 5.21236955819404e-08
2323 5.20995975250571e-08
2324 5.20769312117864e-08
2325 5.20232497080997e-08
2326 5.19986684821561e-08
2327 5.19704492774054e-08
2328 5.19502378892867e-08
2329 5.19217984162879e-08
2330 5.18899554435848e-08
2331 5.18971816632074e-08
2332 5.18345224520544e-08
2333 5.18121900938695e-08
2334 5.17416580692043e-08
2335 5.17748937056695e-08
2336 5.17431715252314e-08
2337 5.17346023798382e-08
2338 5.17087954676754e-08
2339 5.16859977039985e-08
2340 5.16644327319682e-08
2341 5.15873068707151e-08
2342 5.16240667991497e-08
2343 5.15811287016277e-08
2344 5.1567255354712e-08
2345 5.15223348429572e-08
2346 5.1480991913877e-08
2347 5.14555473785094e-08
2348 5.14616651514643e-08
2349 5.1452786919981e-08
2350 5.14343767576975e-08
2351 5.1414239976566e-08
2352 5.13934494961177e-08
2353 5.13588140904631e-08
2354 5.1493490360599e-08
2355 5.14380005256498e-08
2356 5.14564995057754e-08
2357 5.1385846688845e-08
2358 5.1364654751751e-08
2359 5.14103817295108e-08
2360 5.14259248518556e-08
2361 5.12719360301617e-08
2362 5.12335134317254e-08
2363 5.12008853092993e-08
2364 5.11622921806065e-08
2365 5.11329361074786e-08
2366 5.09574675788826e-08
2367 5.10706392731208e-08
2368 5.10393149966148e-08
2369 5.10041147094853e-08
2370 5.09645410318171e-08
2371 5.08999598025639e-08
2372 5.08697652890078e-08
2373 5.0970648146631e-08
2374 5.08208977123559e-08
2375 5.07325914611556e-08
2376 5.08644362184896e-08
2377 5.06445090309171e-08
2378 5.05887385315873e-08
2379 5.05355757240977e-08
2380 5.04968866721356e-08
2381 5.0467626522277e-08
2382 5.04388673050471e-08
2383 5.04170181159225e-08
2384 5.03925221551071e-08
2385 5.03682251462578e-08
2386 5.04024200154163e-08
2387 5.03740267276953e-08
2388 5.03567250120796e-08
2389 5.03988033528913e-08
2390 5.0393651918057e-08
2391 5.03420523045861e-08
2392 5.03164265808209e-08
2393 5.02150108161459e-08
2394 5.01580004197422e-08
2395 5.01339378899957e-08
2396 5.01064825186859e-08
2397 5.00677934667237e-08
2398 5.00491772470468e-08
2399 5.00325860741668e-08
2400 5.0018680752828e-08
2401 5.00049139873227e-08
2402 4.99786985130868e-08
2403 4.99491363825655e-08
2404 4.9910827470967e-08
2405 4.99265979669872e-08
2406 4.98986132413393e-08
2407 4.98640169155351e-08
2408 4.97944618871315e-08
2409 4.97057541792856e-08
2410 4.96429315433033e-08
2411 4.9611124097737e-08
2412 4.96564318552828e-08
2413 4.95824963309133e-08
2414 4.95313123849428e-08
2415 4.94977427933918e-08
2416 4.94755454383267e-08
2417 4.94530141281757e-08
2418 4.94280811835779e-08
2419 4.93942735602104e-08
2420 4.93676139967647e-08
2421 4.93264948886463e-08
2422 4.9301345228514e-08
2423 4.92819509645415e-08
2424 4.93181815386379e-08
2425 4.90704437083878e-08
2426 4.9282498082448e-08
2427 4.92118630290861e-08
2428 4.91610485653382e-08
2429 4.91987002249061e-08
2430 4.90893121707359e-08
2431 4.90524243446089e-08
2432 4.90049778534285e-08
2433 4.90387144225224e-08
2434 4.90049139045823e-08
2435 4.89748117615818e-08
2436 4.89372666834242e-08
2437 4.89007518922335e-08
2438 4.88606275439452e-08
2439 4.88270579523942e-08
2440 4.87935203352663e-08
2441 4.87577267449524e-08
2442 4.87377711522186e-08
2443 4.86974904845283e-08
2444 4.86466973370625e-08
2445 4.86996754034408e-08
2446 4.86108504560434e-08
2447 4.85286086870929e-08
2448 4.84706852432737e-08
2449 4.83950159946289e-08
2450 4.84139448531096e-08
2451 4.82637361187699e-08
2452 4.82266138135401e-08
2453 4.81942521446399e-08
2454 4.81657949080727e-08
2455 4.81325805878896e-08
2456 4.81106354754957e-08
2457 4.80898236787652e-08
2458 4.80724153817391e-08
2459 4.80481787690223e-08
2460 4.8027189336608e-08
2461 4.80048640838504e-08
2462 4.79850150725269e-08
2463 4.79621213855808e-08
2464 4.79423611920993e-08
2465 4.79141242237802e-08
2466 4.78897135280931e-08
2467 4.78704187401036e-08
2468 4.78449777574497e-08
2469 4.78229935652053e-08
2470 4.78159485339802e-08
2471 4.77850115032652e-08
2472 4.78068429288214e-08
2473 4.77875445881182e-08
2474 4.77677595256409e-08
2475 4.77471786552996e-08
2476 4.77271449028649e-08
2477 4.77102020113307e-08
2478 4.76932235926597e-08
2479 4.76750621203337e-08
2480 4.76462922449628e-08
2481 4.7627619181867e-08
2482 4.76608796873279e-08
2483 4.76132484550362e-08
2484 4.76268304794303e-08
2485 4.7595122509847e-08
2486 4.75711701142245e-08
2487 4.75474450922775e-08
2488 4.75338950423065e-08
2489 4.75088910434351e-08
2490 4.74881218792689e-08
2491 4.74655728055495e-08
2492 4.74418619944572e-08
2493 4.73869228301282e-08
2494 4.73566927894353e-08
2495 4.73303281012249e-08
2496 4.73040202564334e-08
2497 4.72779930760225e-08
2498 4.72540975238189e-08
2499 4.72290970776612e-08
2500 4.72045869059912e-08
2501 4.71862691142633e-08
2502 4.71644732158438e-08
2503 4.71396610635111e-08
2504 4.71177195038308e-08
2505 4.69937830871459e-08
2506 4.69626257881828e-08
2507 4.69373482303581e-08
2508 4.69338488073845e-08
2509 4.69070080555412e-08
2510 4.68861074409688e-08
2511 4.68619987259444e-08
2512 4.68407463927178e-08
2513 4.68168686040826e-08
2514 4.67944651916241e-08
2515 4.677433551592e-08
2516 4.67593430641955e-08
2517 4.6735877390347e-08
2518 4.67081271438019e-08
2519 4.66825724743103e-08
2520 4.66602578796937e-08
2521 4.66807072996289e-08
2522 4.66923246733586e-08
2523 4.66682195110479e-08
2524 4.66370977392216e-08
2525 4.66583180980251e-08
2526 4.66283509581444e-08
2527 4.660088137598e-08
2528 4.65749252498426e-08
2529 4.655210261717e-08
2530 4.65292586682153e-08
2531 4.65032492513728e-08
2532 4.64778366904284e-08
2533 4.64556357826496e-08
2534 4.6432546696451e-08
2535 4.64137350775218e-08
2536 4.63822011909087e-08
2537 4.6360220551378e-08
2538 4.63308928999595e-08
2539 4.63082301394024e-08
2540 4.62847324911309e-08
2541 4.62611282614489e-08
2542 4.62359039943294e-08
2543 4.61887736946665e-08
2544 4.61859777090012e-08
2545 4.61622633451952e-08
2546 4.61200109214133e-08
2547 4.61161420162171e-08
2548 4.60727882511947e-08
2549 4.60700135818115e-08
2550 4.60261340151646e-08
2551 4.60238709365512e-08
2552 4.60044233818735e-08
2553 4.59601920965724e-08
2554 4.59588989087933e-08
2555 4.59179787526409e-08
2556 4.59135378605424e-08
2557 4.58708200312685e-08
2558 4.58694913163527e-08
2559 4.5828109307422e-08
2560 4.58257289892572e-08
2561 4.58068960540459e-08
2562 4.57855868774004e-08
2563 4.57450290980432e-08
2564 4.5719481533979e-08
2565 4.57333904080315e-08
2566 4.56268303139495e-08
2567 4.56434570139663e-08
2568 4.56223823164237e-08
2569 4.56002204884953e-08
2570 4.55762396711634e-08
2571 4.55566215862291e-08
2572 4.55307009872286e-08
2573 4.55076651917352e-08
2574 4.54802311367075e-08
2575 4.54561366325379e-08
2576 4.54312178987948e-08
2577 4.54055921750296e-08
2578 4.53571118441687e-08
2579 4.53491821872376e-08
2580 4.53233433006517e-08
2581 4.52848247789461e-08
2582 4.52633912573219e-08
2583 4.52634481007408e-08
2584 4.52219204305493e-08
2585 4.52027677511069e-08
2586 4.51828618963646e-08
2587 4.51619435182238e-08
2588 4.51414194913013e-08
2589 4.51239934307068e-08
2590 4.51216344288241e-08
2591 4.50935431217658e-08
2592 4.50666846063541e-08
2593 4.50454180622728e-08
2594 4.50267592100317e-08
2595 4.50058763590278e-08
2596 4.49687611592253e-08
2597 4.49464181428993e-08
2598 4.49339871977372e-08
2599 4.49032526717019e-08
2600 4.48802737196274e-08
2601 4.48440964362362e-08
2602 4.48355308435566e-08
2603 4.48092940530387e-08
2604 4.47567991557207e-08
2605 4.47265975367372e-08
2606 4.47134738124078e-08
2607 4.46716477142672e-08
2608 4.46628654060532e-08
2609 4.46218955119093e-08
2610 4.46775203499783e-08
2611 4.47048869034461e-08
2612 4.46829737654753e-08
2613 4.46542607335232e-08
2614 4.46216645855202e-08
2615 4.45919603464517e-08
2616 4.45746941579728e-08
2617 4.45471144416842e-08
2618 4.45458816500377e-08
2619 4.45022294570663e-08
2620 4.44874928007266e-08
2621 4.44695302803666e-08
2622 4.44376659913814e-08
2623 4.44138876787292e-08
2624 4.44044339076299e-08
2625 4.43672298899855e-08
2626 4.43526424476204e-08
2627 4.43217587076106e-08
2628 4.42980372383772e-08
2629 4.42614123130625e-08
2630 4.42360672536779e-08
2631 4.42066898642679e-08
2632 4.41833449826845e-08
2633 4.41720615640406e-08
2634 4.41475229706612e-08
2635 4.41234320192052e-08
2636 4.41006520190967e-08
2637 4.40788276989679e-08
2638 4.40507257337686e-08
2639 4.40191030293136e-08
2640 4.39939391583266e-08
2641 4.39688143671901e-08
2642 4.39461445012057e-08
2643 4.39203482471839e-08
2644 4.38976783811995e-08
2645 4.38735341390384e-08
2646 4.38585594508822e-08
2647 4.38370371114161e-08
2648 4.38148397563509e-08
2649 4.37920810725245e-08
2650 4.37706120237635e-08
2651 4.37447411627545e-08
2652 4.37207106074311e-08
2653 4.36953371263371e-08
2654 4.36765716926857e-08
2655 4.36541505166588e-08
2656 4.36391971447847e-08
2657 4.36631673039756e-08
2658 4.36381419888221e-08
2659 4.36099867329176e-08
2660 4.35894449424268e-08
2661 4.35735501014278e-08
2662 4.3557101037095e-08
2663 4.3535642646475e-08
2664 4.35173994617344e-08
2665 4.34973337348765e-08
2666 4.3480284261932e-08
2667 4.34187477083015e-08
2668 4.33950724243459e-08
2669 4.33712976644074e-08
2670 4.33492459706031e-08
2671 4.33240856523298e-08
2672 4.3304517305387e-08
2673 4.32828173302369e-08
2674 4.32638316283374e-08
2675 4.32225739643854e-08
2676 4.31994315874817e-08
2677 4.31775859510708e-08
2678 4.31540208012393e-08
2679 4.31335251960263e-08
2680 4.31114095533758e-08
2681 4.30935287454304e-08
2682 4.30717754795751e-08
2683 4.30543174445575e-08
2684 4.3096154200839e-08
2685 4.3078966172061e-08
2686 4.30574758070179e-08
2687 4.30458548805746e-08
2688 4.30228759285001e-08
2689 4.30043733956609e-08
2690 4.29817426095269e-08
2691 4.29655102607285e-08
2692 4.29428581583124e-08
2693 4.29252366984656e-08
2694 4.29018776060275e-08
2695 4.28816733233361e-08
2696 4.28585735789966e-08
2697 4.28389341777802e-08
2698 4.28178772438059e-08
2699 4.28006394770364e-08
2700 4.27618722653733e-08
2701 4.27441904093939e-08
2702 4.27236237499073e-08
2703 4.27070538933094e-08
2704 4.2687592127777e-08
2705 4.26412789522601e-08
2706 4.26171169465306e-08
2707 4.26012078946769e-08
2708 4.25800870118564e-08
2709 4.25640394041693e-08
2710 4.25485282562477e-08
2711 4.25380406454678e-08
2712 4.25174171425624e-08
2713 4.25063468867393e-08
2714 4.24895993944574e-08
2715 4.24800603582298e-08
2716 4.2462261262699e-08
2717 4.24478265870221e-08
2718 4.24596038328673e-08
2719 4.25518038582595e-08
2720 4.25281037053082e-08
2721 4.25181454488666e-08
2722 4.25055475261615e-08
2723 4.25324415687101e-08
2724 4.25185504582259e-08
2725 4.24982360414106e-08
2726 4.24816199995348e-08
2727 4.24648582963982e-08
2728 4.24468140636236e-08
2729 4.24327097903188e-08
2730 4.2414434631155e-08
2731 4.23259542969845e-08
2732 4.22980406256102e-08
2733 4.22745216610565e-08
2734 4.22413961587154e-08
2735 4.22229113894446e-08
2736 4.21963690655502e-08
2737 4.21788008964086e-08
2738 4.21586996424139e-08
2739 4.21411741058364e-08
2740 4.21148342866218e-08
2741 4.20911234755295e-08
2742 4.2068442951404e-08
2743 4.20539230105987e-08
2744 4.20312495919006e-08
2745 4.20112762355984e-08
2746 4.19898817938247e-08
2747 4.19713757082718e-08
2748 4.19485850500223e-08
2749 4.19278514129928e-08
2750 4.190722080466e-08
2751 4.18887324826756e-08
2752 4.18636787458126e-08
2753 4.18428456328002e-08
2754 4.18121715028974e-08
2755 4.17936512064898e-08
2756 4.17827088483591e-08
2757 4.17669099306295e-08
2758 4.17412131525907e-08
2759 4.1724607768856e-08
2760 4.16947827375225e-08
2761 4.16797547586611e-08
2762 4.16511518608331e-08
2763 4.16401171321468e-08
2764 4.1618513080266e-08
2765 4.15998684388796e-08
2766 4.15784597862512e-08
2767 4.15598719882837e-08
2768 4.15374969975346e-08
2769 4.15171257373004e-08
2770 4.14971026430067e-08
2771 4.14821315075642e-08
2772 4.1464403466307e-08
2773 4.14516918567642e-08
2774 4.14329832665317e-08
2775 4.14144842864062e-08
2776 4.13815435251763e-08
2777 4.13613321370576e-08
2778 4.13446450409083e-08
2779 4.13324912074131e-08
2780 4.13139282784414e-08
2781 4.13027372303532e-08
2782 4.12835206020645e-08
2783 4.12733740517979e-08
2784 4.1201886347153e-08
2785 4.1190151733872e-08
2786 4.11699829783174e-08
2787 4.11577296688392e-08
2788 4.11409288858522e-08
2789 4.1131208661227e-08
2790 4.11152853985186e-08
2791 4.11055864901755e-08
2792 4.10886435986413e-08
2793 4.10787066584817e-08
2794 4.1061614552973e-08
2795 4.10518232740742e-08
2796 4.10336618017482e-08
2797 4.10219804791723e-08
2798 4.10024192376568e-08
2799 4.09804492562671e-08
2800 4.09725906536096e-08
2801 4.09608347240464e-08
2802 4.09435934045632e-08
2803 4.09193212647097e-08
2804 4.0914240884149e-08
2805 4.08906437598944e-08
2806 4.08849096800168e-08
2807 4.08759071035547e-08
2808 4.0854423843939e-08
2809 4.08410478769383e-08
2810 4.08250642180974e-08
2811 4.08401099605271e-08
2812 4.08166336285376e-08
2813 4.08060785161979e-08
2814 4.07923792522524e-08
2815 4.07812841274335e-08
2816 4.07676772340437e-08
2817 4.07614102471143e-08
2818 4.07460980511587e-08
2819 4.07382820810653e-08
2820 4.07269737934257e-08
2821 4.0722479610622e-08
2822 4.07049505213308e-08
2823 4.06975608768789e-08
2824 4.06803017938273e-08
2825 4.06734059765768e-08
2826 4.06542639552754e-08
2827 4.06454176982152e-08
2828 4.06263218621916e-08
2829 4.06169782252164e-08
2830 4.0597360140282e-08
2831 4.0587384120272e-08
2832 4.0569929637968e-08
2833 4.05594349217608e-08
2834 4.05453342011697e-08
2835 4.05367188704986e-08
2836 4.05222984056763e-08
2837 4.05157898342168e-08
2838 4.05009465964667e-08
2839 4.04943776288746e-08
2840 4.04790334584959e-08
2841 4.04718001334459e-08
2842 4.04549815868904e-08
2843 4.0447144300515e-08
2844 4.04297537670573e-08
2845 4.04222717520497e-08
2846 4.04062490133583e-08
2847 4.0397381440016e-08
2848 4.03810709315167e-08
2849 4.0371709530973e-08
2850 4.0355406127901e-08
2851 4.03490361122749e-08
2852 4.03358306755308e-08
2853 4.03265474346881e-08
2854 4.03146636074325e-08
2855 4.03079702948617e-08
2856 4.02919546615976e-08
2857 4.02882172068075e-08
2858 4.0274020562947e-08
2859 4.02677322597356e-08
2860 4.0256821876028e-08
2861 4.02500219820467e-08
2862 4.02390334386382e-08
2863 4.02299953350393e-08
2864 4.02198168103496e-08
2865 4.02123383480557e-08
2866 4.02014741496259e-08
2867 4.01946138595122e-08
2868 4.01812307870841e-08
2869 4.01778521563756e-08
2870 4.01634956403996e-08
2871 4.01593851506732e-08
2872 4.01442648012562e-08
2873 4.01393798199479e-08
2874 4.01249380388435e-08
2875 4.0119402910932e-08
2876 4.01039876862797e-08
2877 4.00988859894369e-08
2878 4.00850161952349e-08
2879 4.0076365337427e-08
2880 4.00651138932062e-08
2881 4.00582038651009e-08
2882 4.00419999380119e-08
2883 4.00375981257639e-08
2884 4.00229787089756e-08
2885 4.00192625704676e-08
2886 4.00059256833174e-08
2887 4.00007600376284e-08
2888 3.9985302180412e-08
2889 3.99826944885717e-08
2890 3.99677304585566e-08
2891 3.99547026574965e-08
2892 3.9939820339896e-08
2893 3.99257054084501e-08
2894 3.99150046348495e-08
2895 3.99081336865947e-08
2896 3.98980404270333e-08
2897 3.98896240483282e-08
2898 3.98784010258169e-08
2899 3.98694233183505e-08
2900 3.98609891760771e-08
2901 3.98523454236965e-08
2902 3.98426500680671e-08
2903 3.98374702115234e-08
2904 3.98268653611922e-08
2905 3.98171060567165e-08
2906 3.99048403210145e-08
2907 3.99033091014189e-08
2908 3.98977810789347e-08
2909 3.98997812567359e-08
2910 3.98848989391354e-08
2911 3.98722548311525e-08
2912 3.98384045752209e-08
2913 3.98176887017598e-08
2914 3.97942372387661e-08
2915 3.97910007166047e-08
2916 3.97749424507765e-08
2917 3.97749744251996e-08
2918 3.97660606665795e-08
2919 3.97639361437996e-08
2920 3.97568875598608e-08
2921 3.9750016611606e-08
2922 3.97379622540939e-08
2923 3.973086037945e-08
2924 3.97184969358477e-08
2925 3.97103541160959e-08
2926 3.97034938259822e-08
2927 3.96967436699924e-08
2928 3.96873822694488e-08
2929 3.96832788851498e-08
2930 3.96716863804158e-08
2931 3.96636110622239e-08
2932 3.96510806410788e-08
2933 3.96424013615615e-08
2934 3.96308550421054e-08
2935 3.96239165922907e-08
2936 3.96140613645457e-08
2937 3.96043979833394e-08
2938 3.95921162521518e-08
2939 3.95829253818647e-08
2940 3.95701675870441e-08
2941 3.95610797454538e-08
2942 3.95496115856986e-08
2943 3.96129102853138e-08
2944 3.96009127712205e-08
2945 3.95967134636521e-08
2946 3.95954025123046e-08
2947 3.95944645958934e-08
2948 3.9591991907173e-08
2949 3.95912991280056e-08
2950 3.95852097767602e-08
2951 3.95824706345138e-08
2952 3.95968307032035e-08
2953 3.95959673937796e-08
2954 3.95876966763353e-08
2955 3.95885919601824e-08
2956 3.95797741248316e-08
2957 3.95653323437273e-08
2958 3.95573245270953e-08
2959 3.95472277148201e-08
2960 3.95351911208763e-08
2961 3.95215522530634e-08
2962 3.95049681856108e-08
2963 3.94903594269636e-08
2964 3.94738677300666e-08
2965 3.94559158678476e-08
2966 3.94398966818699e-08
2967 3.94230887934555e-08
2968 3.9402742402217e-08
2969 3.93835009049326e-08
2970 3.93635453121988e-08
2971 3.93405485965559e-08
2972 3.93178574142894e-08
2973 3.92887642419737e-08
2974 3.92553189954015e-08
2975 3.92195964593611e-08
2976 3.91889116713173e-08
2977 3.91572534397255e-08
2978 3.91327468207692e-08
2979 3.91196763871449e-08
2980 3.91089969298264e-08
2981 3.9096669013361e-08
2982 3.90850161124945e-08
2983 3.90691212714955e-08
2984 3.90594792065713e-08
2985 3.90471370792511e-08
2986 3.90304393249608e-08
2987 3.90210601608487e-08
2988 3.9007883145814e-08
2989 3.8991082362827e-08
2990 3.89785519416819e-08
2991 3.89638543651927e-08
2992 3.8954294012683e-08
2993 3.89353509433477e-08
2994 3.89277730050708e-08
2995 3.89125567323845e-08
2996 3.88988290467296e-08
2997 3.88836767228895e-08
2998 3.88672880546892e-08
2999 3.88454317601372e-08
3000 3.88273306839437e-08
3001 3.88094001380068e-08
3002 3.87930221279476e-08
3003 3.8776946098551e-08
3004 3.8763008802789e-08
3005 3.87492988807026e-08
3006 3.87328107365192e-08
3007 3.87248242361693e-08
3008 3.87051883876666e-08
3009 3.86887748504705e-08
3010 3.86750826919524e-08
3011 3.86603744573222e-08
3012 3.86432965626682e-08
3013 3.86269185526089e-08
3014 3.8610259878169e-08
3015 3.85931748780877e-08
3016 3.85806551150836e-08
3017 3.85590617213438e-08
3018 3.85415326320526e-08
3019 3.85249130374632e-08
3020 3.8514119893307e-08
3021 3.85072560504796e-08
3022 3.84915104234551e-08
3023 3.84741660752752e-08
3024 3.84559086796799e-08
3025 3.84433427313979e-08
3026 3.84282010656989e-08
3027 3.84181575441289e-08
3028 3.84056519919795e-08
3029 3.83959637417775e-08
3030 3.83837601702908e-08
3031 3.83778697710113e-08
3032 3.83698264272425e-08
3033 3.83600280429164e-08
3034 3.83486131738664e-08
3035 3.83023248673453e-08
3036 3.82893254879946e-08
3037 3.82816125465979e-08
3038 3.82736295989616e-08
3039 3.82629892214936e-08
3040 3.82521889719101e-08
3041 3.82461315950877e-08
3042 3.82346847516146e-08
3043 3.82241154284202e-08
3044 3.82148499511459e-08
3045 3.82033142898308e-08
3046 3.81929403658887e-08
3047 3.81841971375252e-08
3048 3.81724127862526e-08
3049 3.81629448042986e-08
3050 3.81513416414236e-08
3051 3.81440408148137e-08
3052 3.81319402720237e-08
3053 3.81211044953034e-08
3054 3.81109401814683e-08
3055 3.81021827422501e-08
3056 3.80916524989061e-08
3057 3.80811577826989e-08
3058 3.80704463509574e-08
3059 3.80625486684494e-08
3060 3.80488494045039e-08
3061 3.80393352372721e-08
3062 3.80308833314302e-08
3063 3.80213265316343e-08
3064 3.8013155290173e-08
3065 3.80040141578775e-08
3066 3.79976157205419e-08
3067 3.79889719681614e-08
3068 3.79845950249091e-08
3069 3.79771023517605e-08
3070 3.79717590703876e-08
3071 3.7962298193861e-08
3072 3.79248774606822e-08
3073 3.79022750962577e-08
3074 3.78933400213555e-08
3075 3.78727698091552e-08
3076 3.78650675258996e-08
3077 3.78452682525676e-08
3078 3.7842632139018e-08
3079 3.78223603547667e-08
3080 3.78263926847922e-08
3081 3.77993778499786e-08
3082 3.77957825037356e-08
3083 3.77768252235455e-08
3084 3.7779486206091e-08
3085 3.77630726688949e-08
3086 3.77613709190427e-08
3087 3.77438347243242e-08
3088 3.77494764336461e-08
3089 3.77276947460814e-08
3090 3.77336668577755e-08
3091 3.7714681155876e-08
3092 3.77178572819048e-08
3093 3.76984345962228e-08
3094 3.77070605850349e-08
3095 3.76834243809299e-08
3096 3.76871049923011e-08
3097 3.76727307127567e-08
3098 3.76751501107719e-08
3099 3.76877871133274e-08
3100 3.76985198613511e-08
3101 3.76743187757711e-08
3102 3.76833497739426e-08
3103 3.76616497987925e-08
3104 3.76687516734364e-08
3105 3.76443622940315e-08
3106 3.76572906191086e-08
3107 3.76313629146807e-08
3108 3.76405751012499e-08
3109 3.7617500225906e-08
3110 3.76265845147827e-08
3111 3.76057514017702e-08
3112 3.76254014611277e-08
3113 3.76002482482818e-08
3114 3.76476485541843e-08
3115 3.76194826401388e-08
3116 3.76106399357923e-08
3117 3.76066466856173e-08
3118 3.75899666948953e-08
3119 3.75678581576722e-08
3120 3.75832129861919e-08
3121 3.75622235537776e-08
3122 3.75731445956262e-08
3123 3.75513806716299e-08
3124 3.75642521532882e-08
3125 3.75414828113207e-08
3126 3.75513877770572e-08
3127 3.75297410926123e-08
3128 3.7542154274206e-08
3129 3.75205502223253e-08
3130 3.75279292086361e-08
3131 3.75094053595149e-08
3132 3.75280428954738e-08
3133 3.75030246857477e-08
3134 3.75148836440076e-08
3135 3.75004596264716e-08
3136 3.75109401318241e-08
3137 3.74882915821217e-08
3138 3.74960009708047e-08
3139 3.74781770062782e-08
3140 3.74904693956069e-08
3141 3.746386667558e-08
3142 3.74865685159875e-08
3143 3.74458224428054e-08
3144 3.74629145483141e-08
3145 3.74557842519607e-08
3146 3.74371857958522e-08
3147 3.74299702343706e-08
3148 3.74449840023772e-08
3149 3.74095350252901e-08
3150 3.74179620621362e-08
3151 3.74135353808924e-08
3152 3.74067994357574e-08
3153 3.73851278823167e-08
3154 3.73818878074417e-08
3155 3.73752726545717e-08
3156 3.73690305366381e-08
3157 3.73571573675235e-08
3158 3.73460018465721e-08
3159 3.73394257735526e-08
3160 3.73383386431669e-08
3161 3.7329190405444e-08
3162 3.73273181253353e-08
3163 3.73198965064603e-08
3164 3.7334832114766e-08
3165 3.73234811945622e-08
3166 3.73358624017328e-08
3167 3.73166102463074e-08
3168 3.73218256299879e-08
3169 3.73078066218113e-08
3170 3.73017634558437e-08
3171 3.72910378132474e-08
3172 3.72882951182874e-08
3173 3.72733346409859e-08
3174 3.72737396503453e-08
3175 3.72603778941993e-08
3176 3.72551340888094e-08
3177 3.72462061193346e-08
3178 3.72514392665835e-08
3179 3.72601398623829e-08
3180 3.72825894601192e-08
3181 3.72678208293564e-08
3182 3.7255016849258e-08
3183 3.72914357171794e-08
3184 3.73113380192081e-08
3185 3.72341659726771e-08
3186 3.72215609445448e-08
3187 3.72151731653503e-08
3188 3.72029340667268e-08
3189 3.719391727941e-08
3190 3.7185234447179e-08
3191 3.71769459661664e-08
3192 3.71714889979557e-08
3193 3.71695243472914e-08
3194 3.7181031586897e-08
3195 3.71999959725144e-08
3196 3.72527040326531e-08
3197 3.72066644160896e-08
3198 3.71848898339522e-08
3199 3.71552317801616e-08
3200 3.71397810283725e-08
3201 3.7145394315985e-08
3202 3.71389070608075e-08
3203 3.71026480650016e-08
3204 3.70591060061543e-08
3205 3.70324890752727e-08
3206 3.69912669384576e-08
3207 3.69551607093399e-08
3208 3.69251331733267e-08
3209 3.689253347261e-08
3210 3.68703219066902e-08
3211 3.6858168073195e-08
3212 3.68362620406515e-08
3213 3.68230175240569e-08
3214 3.68125832039823e-08
3215 3.6794745028601e-08
3216 3.67866164197039e-08
3217 3.67766155306981e-08
3218 3.67596406647408e-08
3219 3.67554982005913e-08
3220 3.67405341705762e-08
3221 3.67363099940121e-08
3222 3.67252859234668e-08
3223 3.67197792172647e-08
3224 3.67100412290711e-08
3225 3.67063961448366e-08
3226 3.66993546663252e-08
3227 3.66956811603814e-08
3228 3.668863968187e-08
3229 3.66856482969524e-08
3230 3.6680265935729e-08
3231 3.66700696474709e-08
3232 3.66664707485143e-08
3233 3.6676460979379e-08
3234 3.66745851465566e-08
3235 3.66672878726604e-08
3236 3.66631169868015e-08
3237 3.66631134340878e-08
3238 3.66446926136632e-08
3239 3.66305954457857e-08
3240 3.66089309977724e-08
3241 3.66294763409769e-08
3242 3.6595540819917e-08
3243 3.65962939952169e-08
3244 3.65775321142792e-08
3245 3.65585464123797e-08
3246 3.65842929284099e-08
3247 3.65369530186399e-08
3248 3.65292684989527e-08
3249 3.65288279624565e-08
3250 3.6505550582433e-08
3251 3.64920289541715e-08
3252 3.64921284301545e-08
3253 3.64281262932309e-08
3254 3.64976102673609e-08
3255 3.65004027003124e-08
3256 3.64875027969447e-08
3257 3.65048933304024e-08
3258 3.64793457663382e-08
3259 3.64828380838844e-08
3260 3.64903876004519e-08
3261 3.64670675878642e-08
3262 3.64740628810978e-08
3263 3.64495775784235e-08
3264 3.64242183081842e-08
3265 3.64385606133055e-08
3266 3.64160541721503e-08
3267 3.64154786325344e-08
3268 3.64126329088776e-08
3269 3.64180223755284e-08
3270 3.63280072690486e-08
3271 3.63520911150772e-08
3272 3.6352613363988e-08
3273 3.64242929151715e-08
3274 3.64149279619141e-08
3275 3.64910128780593e-08
3276 3.64669539010265e-08
3277 3.64624490600818e-08
3278 3.64614649583928e-08
3279 3.64548355946681e-08
3280 3.64631915772407e-08
3281 3.64268686325886e-08
3282 3.64705599054105e-08
3283 3.63670409342376e-08
3284 3.63787684420913e-08
3285 3.63755923160625e-08
3286 3.63578323003821e-08
3287 3.63467549391316e-08
3288 3.63349030862992e-08
3289 3.63530183733474e-08
3290 3.63272079084709e-08
3291 3.64252947804289e-08
3292 3.63674068637465e-08
3293 3.63567131955733e-08
3294 3.63604151232266e-08
3295 3.63431489347477e-08
3296 3.63503573908019e-08
3297 3.63456429397502e-08
3298 3.6359548261089e-08
3299 3.63396672753424e-08
3300 3.63533132485827e-08
3301 3.63262024904998e-08
3302 3.6351142540525e-08
3303 3.62698386879856e-08
3304 3.63006691372902e-08
3305 3.62775267603865e-08
3306 3.63127874436486e-08
3307 3.62439820378313e-08
3308 3.62898155970015e-08
3309 3.61544145732751e-08
3310 3.62885188565087e-08
3311 3.62310927926046e-08
3312 3.62601362269288e-08
3313 3.62365462081016e-08
3314 3.62698884259771e-08
3315 3.61963721218217e-08
3316 3.62378180795986e-08
3317 3.61810883475755e-08
3318 3.62266590059335e-08
3319 3.61620990929623e-08
3320 3.61936791648532e-08
3321 3.6183870122386e-08
3322 3.61929544112627e-08
3323 3.61767753531694e-08
3324 3.62666057185379e-08
3325 3.6111266865646e-08
3326 3.61106167190428e-08
3327 3.6155167748575e-08
3328 3.61234491208506e-08
3329 3.61235841239704e-08
3330 3.61230263479229e-08
3331 3.61082079791686e-08
3332 3.61307819218837e-08
3333 3.61056571307472e-08
3334 3.61318299724189e-08
3335 3.61023069217481e-08
3336 3.61266216941658e-08
3337 3.61043284158313e-08
3338 3.61083145605789e-08
3339 3.60911833752198e-08
3340 3.61033265505739e-08
3341 3.60926755149649e-08
3342 3.61009107052723e-08
3343 3.60748373395836e-08
3344 3.6101301503777e-08
3345 3.6049382146075e-08
3346 3.60362477636045e-08
3347 3.60653018560697e-08
3348 3.60579868186051e-08
3349 3.60541356769772e-08
3350 3.60483305428261e-08
3351 3.60499079476995e-08
3352 3.60262042420345e-08
3353 3.60495775453273e-08
3354 3.6032304251421e-08
3355 3.60360026263606e-08
3356 3.60180898439921e-08
3357 3.60457086401311e-08
3358 3.60151943823439e-08
3359 3.60620653339083e-08
3360 3.59728993259978e-08
3361 3.5972753664737e-08
3362 3.60152370149081e-08
3363 3.59420084805606e-08
3364 3.59220315715447e-08
3365 3.59480800682377e-08
3366 3.59433052210534e-08
3367 3.59370204705556e-08
3368 3.59224863188956e-08
3369 3.59265364124894e-08
3370 3.59093590418524e-08
3371 3.59088261348006e-08
3372 3.58981679937642e-08
3373 3.58879823636471e-08
3374 3.58752672013907e-08
3375 3.58768588171188e-08
3376 3.58751570672666e-08
3377 3.58724925320075e-08
3378 3.58618841289626e-08
3379 3.5855752145153e-08
3380 3.58463978500367e-08
3381 3.58499541164292e-08
3382 3.58356366803037e-08
3383 3.58391254451362e-08
3384 3.58277389977957e-08
3385 3.5824914590421e-08
3386 3.58132332678451e-08
3387 3.5812814047631e-08
3388 3.58112366427576e-08
3389 3.57927660843416e-08
3390 3.57876892564946e-08
3391 3.58202321137924e-08
3392 3.58370648712025e-08
3393 3.57814613494156e-08
3394 3.57435503417491e-08
3395 3.57798768391149e-08
3396 3.57860514554886e-08
3397 3.58888776474942e-08
3398 3.57710199239136e-08
3399 3.57693394903436e-08
3400 3.58119862653439e-08
3401 3.57893981117741e-08
3402 3.58786849119497e-08
3403 3.57920235671827e-08
3404 3.57985108223602e-08
3405 3.58082061779896e-08
3406 3.57986742471894e-08
3407 3.57430387509794e-08
3408 3.57411877871527e-08
3409 3.57838203512983e-08
3410 3.5758628058602e-08
3411 3.57456926280975e-08
3412 3.57618894497591e-08
3413 3.56990419447811e-08
3414 3.5741461346106e-08
3415 3.57517571103472e-08
3416 3.57484424284849e-08
3417 3.57445735232886e-08
3418 3.57342884171885e-08
3419 3.57264475780994e-08
3420 3.57042253540385e-08
3421 3.56924658717617e-08
3422 3.56880214269495e-08
3423 3.56801237444415e-08
3424 3.56593723438436e-08
3425 3.56613618635038e-08
3426 3.56751428398638e-08
3427 3.55720999323239e-08
3428 3.56746774343719e-08
3429 3.5645808083018e-08
3430 3.56398324186102e-08
3431 3.56350575714259e-08
3432 3.56307552351609e-08
3433 3.56170097859376e-08
3434 3.56127927148009e-08
3435 3.55990827927144e-08
3436 3.55974627552769e-08
3437 3.55796068163272e-08
3438 3.55784735006637e-08
3439 3.55629481418873e-08
3440 3.55590614731227e-08
3441 3.55435965104789e-08
3442 3.55449785160999e-08
3443 3.55265576956754e-08
3444 3.55206353219728e-08
3445 3.55061082757402e-08
3446 3.55005447261192e-08
3447 3.54896592114073e-08
3448 3.54859928108908e-08
3449 3.54690463666429e-08
3450 3.54650353528996e-08
3451 3.54468845387146e-08
3452 3.54437759142456e-08
3453 3.54265274893351e-08
3454 3.54229321430921e-08
3455 3.53998892421714e-08
3456 3.53920448503686e-08
3457 3.5371478190882e-08
3458 3.53581697254413e-08
3459 3.53416531595485e-08
3460 3.53341320646905e-08
3461 3.53114195661419e-08
3462 3.53097320271445e-08
3463 3.52984770302101e-08
3464 3.52936524450342e-08
3465 3.52826248217752e-08
3466 3.52814311099792e-08
3467 3.52559368366201e-08
3468 3.51666571418718e-08
3469 3.51768356665616e-08
3470 3.51806228593432e-08
3471 3.51703377532431e-08
3472 3.51662414743714e-08
3473 3.51452378311023e-08
3474 3.51403031118025e-08
3475 3.52092399680259e-08
3476 3.52055593566547e-08
3477 3.51701636702728e-08
3478 3.51663373976407e-08
3479 3.5151479949036e-08
3480 3.51188766956056e-08
3481 3.50844331364897e-08
3482 3.50792035419545e-08
3483 3.50660300796335e-08
3484 3.5056036296055e-08
3485 3.49227917695316e-08
3486 3.50227189471752e-08
3487 3.50089592870972e-08
3488 3.49013760114758e-08
3489 3.48871189714828e-08
3490 3.48852893239382e-08
3491 3.48713165010395e-08
3492 3.49621949169432e-08
3493 3.49504567509484e-08
3494 3.49292861301365e-08
3495 3.48540254435648e-08
3496 3.48673943051381e-08
3497 3.48473463418486e-08
3498 3.48069875144574e-08
3499 3.48006459205408e-08
3500 3.47773472242352e-08
3501 3.4788907754546e-08
3502 3.47399726763342e-08
3503 3.47327251404295e-08
3504 3.47542226108999e-08
3505 3.47247315346522e-08
3506 3.47172459669309e-08
3507 3.46980861820612e-08
3508 3.46879360790808e-08
3509 3.46532118555842e-08
3510 3.47064990080526e-08
3511 3.46789050809093e-08
3512 3.45790915901034e-08
3513 3.47014434964876e-08
3514 3.47267530287354e-08
3515 3.46154074293281e-08
3516 3.46329755984698e-08
3517 3.47114301746387e-08
3518 3.47190720617618e-08
3519 3.46875097534394e-08
3520 3.47203226169768e-08
3521 3.4693751871373e-08
3522 3.46806707796077e-08
3523 3.4671387538765e-08
3524 3.46828876729433e-08
3525 3.45570363435854e-08
3526 3.45767787734985e-08
3527 3.46504442916284e-08
3528 3.46762547565049e-08
3529 3.46337785117612e-08
3530 3.4649232816264e-08
3531 3.46000135209579e-08
3532 3.46310109478054e-08
3533 3.46029906950207e-08
3534 3.46139579221472e-08
3535 3.45822144254271e-08
3536 3.46043158572229e-08
3537 3.45678081714595e-08
3538 3.45851738359215e-08
3539 3.45565780435209e-08
3540 3.45719364247543e-08
3541 3.45498101239627e-08
3542 3.45634489917757e-08
3543 3.45373400989502e-08
3544 3.45529400647138e-08
3545 3.45296520265492e-08
3546 3.45400614776281e-08
3547 3.45086412778528e-08
3548 3.45248238886597e-08
3549 3.44915811467672e-08
3550 3.45159669734585e-08
3551 3.44856729839194e-08
3552 3.45059341100296e-08
3553 3.44821522446637e-08
3554 3.45031878623558e-08
3555 3.44729755852313e-08
3556 3.44773205540605e-08
3557 3.44652271166979e-08
3558 3.44846782240893e-08
3559 3.4452483532732e-08
3560 3.44545227903836e-08
3561 3.44383082051536e-08
3562 3.44533503948696e-08
3563 3.44088277870469e-08
3564 3.44443442656939e-08
3565 3.44162849330587e-08
3566 3.4431391071621e-08
3567 3.44002586416536e-08
3568 3.43997719198796e-08
3569 3.43761321630609e-08
3570 3.43906521038662e-08
3571 3.43661596957645e-08
3572 3.43591040063984e-08
3573 3.43541657343849e-08
3574 3.43496004973076e-08
3575 3.43204540342867e-08
3576 3.4304733276258e-08
3577 3.42883836879082e-08
3578 3.43108297329309e-08
3579 3.42929276087034e-08
3580 3.42870265512829e-08
3581 3.41618751065198e-08
3582 3.41785373336734e-08
3583 3.41660282288103e-08
3584 3.41758799038416e-08
3585 3.41618076049599e-08
3586 3.41719008645214e-08
3587 3.41560451033729e-08
3588 3.41684476268256e-08
3589 3.41437527140442e-08
3590 3.41541479542684e-08
3591 3.4125825720821e-08
3592 3.41513946011673e-08
3593 3.41171038087396e-08
3594 3.41389423397231e-08
3595 3.40993011604951e-08
3596 3.41110997226224e-08
3597 3.40809158672073e-08
3598 3.40985906177593e-08
3599 3.40694406020248e-08
3600 3.40814985122506e-08
3601 3.40505295071125e-08
3602 3.40701369339058e-08
3603 3.40396475451143e-08
3604 3.40529311415594e-08
3605 3.4041697460907e-08
3606 3.40530270648287e-08
3607 3.40313199842512e-08
3608 3.40564056955373e-08
3609 3.40147217059439e-08
3610 3.41154340333105e-08
3611 3.40862165160161e-08
3612 3.4125626768855e-08
3613 3.40814985122506e-08
3614 3.41022463601348e-08
3615 3.40775336837851e-08
3616 3.41215411481244e-08
3617 3.40437900092638e-08
3618 3.40432997347762e-08
3619 3.4074968624509e-08
3620 3.40048238456347e-08
3621 3.4028126094654e-08
3622 3.39982193509059e-08
3623 3.40174501900492e-08
3624 3.3964791867902e-08
3625 3.39675558791441e-08
3626 3.40163168743857e-08
3627 3.39879839827972e-08
3628 3.39898313939102e-08
3629 3.3924862918866e-08
3630 3.39798162940497e-08
3631 3.39614416589029e-08
3632 3.39623227318953e-08
3633 3.3942150423627e-08
3634 3.39403101179414e-08
3635 3.39121442038959e-08
3636 3.39091528189783e-08
3637 3.3901645934975e-08
3638 3.39054118114746e-08
3639 3.38748655792642e-08
3640 3.38824328594001e-08
3641 3.38628964868803e-08
3642 3.38638592722873e-08
3643 3.38434098523521e-08
3644 3.3844749225409e-08
3645 3.38289716239615e-08
3646 3.38250849551969e-08
3647 3.38049055415013e-08
3648 3.38060068827417e-08
3649 3.37933236949084e-08
3650 3.37965033736509e-08
3651 3.37770416081185e-08
3652 3.37837562369714e-08
3653 3.37673853323395e-08
3654 3.37625785107321e-08
3655 3.37455929866337e-08
3656 3.37463852417841e-08
3657 3.37316343745897e-08
3658 3.37346683920714e-08
3659 3.37198535760308e-08
3660 3.37196333077827e-08
3661 3.37097603164693e-08
3662 3.3716002434403e-08
3663 3.3712794333951e-08
3664 3.37237295866544e-08
3665 3.37126522254039e-08
3666 3.37098171598882e-08
3667 3.36978125403675e-08
3668 3.36981642590217e-08
3669 3.36787415733397e-08
3670 3.36802656875079e-08
3671 3.36592691496662e-08
3672 3.36685914703594e-08
3673 3.36389014421457e-08
3674 3.36492540498057e-08
3675 3.36122099042768e-08
3676 3.36506360554267e-08
3677 3.36247900634135e-08
3678 3.36240013609768e-08
3679 3.35966916509278e-08
3680 3.36036158898878e-08
3681 3.3570110247183e-08
3682 3.35762528891337e-08
3683 3.35529755091102e-08
3684 3.35634950943131e-08
3685 3.35277086094266e-08
3686 3.35364624959311e-08
3687 3.35066587808797e-08
3688 3.35196368439483e-08
3689 3.3488515072122e-08
3690 3.34980008176444e-08
3691 3.34856053996191e-08
3692 3.34717746852675e-08
3693 3.3454348624673e-08
3694 3.34662537682107e-08
3695 3.34420953151948e-08
3696 3.34457439521429e-08
3697 3.34309540050981e-08
3698 3.34364393950182e-08
3699 3.34148957392699e-08
3700 3.34208110075451e-08
3701 3.34033209981044e-08
3702 3.34075487273822e-08
3703 3.33755707515593e-08
3704 3.33852590017614e-08
3705 3.33720642231583e-08
3706 3.33752367964735e-08
3707 3.3364269569347e-08
3708 3.33716236866621e-08
3709 3.33648308981083e-08
3710 3.33606458013946e-08
3711 3.33453478162937e-08
3712 3.33516680939283e-08
3713 3.33336522828631e-08
3714 3.33391589890653e-08
3715 3.33217400338981e-08
3716 3.33293570520254e-08
3717 3.33021219489638e-08
3718 3.3308623414996e-08
3719 3.32933716151729e-08
3720 3.32879679376674e-08
3721 3.32754659382317e-08
3722 3.32737641883796e-08
3723 3.32554748183611e-08
3724 3.32566401084478e-08
3725 3.32446354889271e-08
3726 3.32444223261064e-08
3727 3.32278169423716e-08
3728 3.32305134520539e-08
3729 3.32141496528493e-08
3730 3.3217880002212e-08
3731 3.32018146309565e-08
3732 3.31989689072998e-08
3733 3.31806617737129e-08
3734 3.31422711496998e-08
3735 3.31351408533465e-08
3736 3.31168301670459e-08
3737 3.31199885295064e-08
3738 3.31181375656797e-08
3739 3.31089147209696e-08
3740 3.31004770259824e-08
3741 3.31018341626077e-08
3742 3.30943841220233e-08
3743 3.30921388069783e-08
3744 3.30738920695239e-08
3745 3.3088547013449e-08
3746 3.30614611243618e-08
3747 3.30800844494661e-08
3748 3.30341336507445e-08
3749 3.30627116795768e-08
3750 3.30289253724914e-08
3751 3.30370006906833e-08
3752 3.30115241808926e-08
3753 3.30285097049909e-08
3754 3.29943325994009e-08
3755 3.30176348484201e-08
3756 3.29848433011648e-08
3757 3.30150378147209e-08
3758 3.29711653535014e-08
3759 3.29893197204001e-08
3760 3.29583933478261e-08
3761 3.29750804439755e-08
3762 3.29310871904909e-08
3763 3.29589298075916e-08
3764 3.29187628267391e-08
3765 3.29391554032554e-08
3766 3.29002034504811e-08
3767 3.29272999977093e-08
3768 3.28926148540631e-08
3769 3.29139346888496e-08
3770 3.28753699818662e-08
3771 3.28967857399221e-08
3772 3.28615499256557e-08
3773 3.28823510642451e-08
3774 3.28504654589779e-08
3775 3.28712594921399e-08
3776 3.28389617720859e-08
3777 3.28523910297918e-08
3778 3.28194218468525e-08
3779 3.28400524551853e-08
3780 3.28015019590566e-08
3781 3.28148033190701e-08
3782 3.27838876046371e-08
3783 3.28021911855103e-08
3784 3.27618678852559e-08
3785 3.27918137088545e-08
3786 3.274812954146e-08
3787 3.27674349875906e-08
3788 3.27257829724203e-08
3789 3.27415889955773e-08
3790 3.26976490327979e-08
3791 3.27133520272582e-08
3792 3.26699733932401e-08
3793 3.26843938580623e-08
3794 3.26452322951809e-08
3795 3.26867812816545e-08
3796 3.26070512812748e-08
3797 3.26417151086389e-08
3798 3.25844844439871e-08
3799 3.26071152301211e-08
3800 3.25550217894488e-08
3801 3.25772333553687e-08
3802 3.25294884362393e-08
3803 3.25649232024716e-08
3804 3.25108118204298e-08
3805 3.2533218785602e-08
3806 3.24890621072882e-08
3807 3.25085380836754e-08
3808 3.24603739443319e-08
3809 3.24844471322194e-08
3810 3.24390505568317e-08
3811 3.24589635170014e-08
3812 3.24265165829729e-08
3813 3.24438822474349e-08
3814 3.24009548080539e-08
3815 3.24397397832854e-08
3816 3.23884421504772e-08
3817 3.23956399483905e-08
3818 3.23335100915756e-08
3819 3.23687601166966e-08
3820 3.23278577241126e-08
3821 3.23564215420902e-08
3822 3.23175370908757e-08
3823 3.23503783761225e-08
3824 3.23219424558374e-08
3825 3.23374855781822e-08
3826 3.23049142991749e-08
3827 3.23477671315686e-08
3828 3.23095221688163e-08
3829 3.23449249606256e-08
3830 3.23061613016762e-08
3831 3.23260884727006e-08
3832 3.22858824119976e-08
3833 3.23064988094757e-08
3834 3.22693871623869e-08
3835 3.22973434663254e-08
3836 3.22653512796478e-08
3837 3.22833919597088e-08
3838 3.22962776522218e-08
3839 3.22511937156378e-08
3840 3.22265805152711e-08
3841 3.22795408180809e-08
3842 3.2215929479662e-08
3843 3.22543165509614e-08
3844 3.21568194294741e-08
3845 3.22364464011571e-08
3846 3.2188498977348e-08
3847 3.21825552873634e-08
3848 3.21503357270103e-08
3849 3.2174838793253e-08
3850 3.21406190550988e-08
3851 3.21719397788911e-08
3852 3.2123793403116e-08
3853 3.21470174924343e-08
3854 3.21065307673507e-08
3855 3.21261452995714e-08
3856 3.20846460510893e-08
3857 3.21186952589869e-08
3858 3.20731672331931e-08
3859 3.20941566656074e-08
3860 3.20622994820496e-08
3861 3.20789439456348e-08
3862 3.20400559417067e-08
3863 3.20710178414174e-08
3864 3.2026978402655e-08
3865 3.20399955455741e-08
3866 3.20106430251599e-08
3867 3.20301118961197e-08
3868 3.19880513188764e-08
3869 3.20081134930206e-08
3870 3.15790593674592e-08
3871 3.16345953876862e-08
3872 3.1638247577348e-08
3873 3.17485024936559e-08
3874 3.16409867195944e-08
3875 3.16581143522399e-08
3876 3.16907069475292e-08
3877 3.16297565916557e-08
3878 3.15816066631669e-08
3879 3.16186508086957e-08
3880 3.15468184908241e-08
3881 3.1556762536411e-08
3882 3.15936077299739e-08
3883 3.15302983722177e-08
3884 3.1496636410111e-08
3885 3.14941779322453e-08
3886 3.14649746258056e-08
3887 3.14687937930103e-08
3888 3.14249142263634e-08
3889 3.14852641736252e-08
3890 3.14496659825636e-08
3891 3.14183949967628e-08
3892 3.138185888929e-08
3893 3.13786365779833e-08
3894 3.13401002927094e-08
3895 3.13375601024291e-08
3896 3.12990557915782e-08
3897 3.12991055295697e-08
3898 3.12551833303587e-08
3899 3.12741654795445e-08
3900 3.12354835330098e-08
3901 3.11908294747809e-08
3902 3.11656194185161e-08
3903 3.12177874661757e-08
3904 3.12050936202013e-08
3905 3.11547090348085e-08
3906 3.11300816235871e-08
3907 3.11823633580843e-08
3908 3.11608800984686e-08
3909 3.11506873629241e-08
3910 3.1141574652338e-08
3911 3.1087896701365e-08
3912 3.10806029801824e-08
3913 3.10584127305447e-08
3914 3.10273762238467e-08
3915 3.11022354537727e-08
3916 3.10803827119344e-08
3917 3.10824610494365e-08
3918 3.10970733607974e-08
3919 3.11010630582587e-08
3920 3.10542844772499e-08
3921 3.10768442091103e-08
3922 3.10371035538992e-08
3923 3.10775902789828e-08
3924 3.10312131546198e-08
3925 3.11061896240972e-08
3926 3.09898631201122e-08
3927 3.10324459462663e-08
3928 3.10466710118362e-08
3929 3.10162420191773e-08
3930 3.09860368474801e-08
3931 3.10353343024872e-08
3932 3.10438288408932e-08
3933 3.10334691278058e-08
3934 3.0969104614087e-08
3935 3.10226511146539e-08
3936 3.09139309706552e-08
3937 3.09517851349028e-08
3938 3.09510959084491e-08
3939 3.09905523465659e-08
3940 3.09261842801334e-08
3941 3.09634984319018e-08
3942 3.08991481290377e-08
3943 3.10324281826979e-08
3944 3.08398746540206e-08
3945 3.08669712012488e-08
3946 3.08285876826631e-08
3947 3.09096144235355e-08
3948 3.08354763944862e-08
3949 3.09104812856731e-08
3950 3.0812362439292e-08
3951 3.08883727484499e-08
3952 3.08093746070881e-08
3953 3.08314831443113e-08
3954 3.07490317652537e-08
3955 3.07801002463748e-08
3956 3.07484739892061e-08
3957 3.07715133374131e-08
3958 3.07461611726012e-08
3959 3.07649052899706e-08
3960 3.07300140889311e-08
3961 3.073798993114e-08
3962 3.07768850404955e-08
3963 3.07605567684277e-08
3964 3.06746379408196e-08
3965 3.06967642416112e-08
3966 3.0717206556119e-08
3967 3.06785459258663e-08
3968 3.06324956511617e-08
3969 3.07087724138455e-08
3970 3.06133856042834e-08
3971 3.06257952331634e-08
3972 3.06561638296898e-08
3973 3.06051290976939e-08
3974 3.05664471511591e-08
3975 3.06454630560893e-08
3976 3.05538314648857e-08
3977 3.05708454106934e-08
3978 3.05441645309656e-08
3979 3.05478344841958e-08
3980 3.05995691007865e-08
3981 3.0532547157236e-08
3982 3.05144034484783e-08
3983 3.05044309811819e-08
3984 3.04820453322918e-08
3985 3.05082963336645e-08
3986 3.04595637601324e-08
3987 3.05185210436321e-08
3988 3.05075893436424e-08
3989 3.04530018979676e-08
3990 3.04254932359527e-08
3991 3.0453804811259e-08
3992 3.04250207250334e-08
3993 3.04894420821711e-08
3994 3.03892306874332e-08
3995 3.04106784199121e-08
3996 3.03852090155488e-08
3997 3.03285538905129e-08
3998 3.04347231860902e-08
3999 3.03814289281945e-08
4000 3.02769613824694e-08
4001 3.03812974777884e-08
4002 3.02612122027313e-08
4003 3.04174179177608e-08
4004 3.03156753034273e-08
4005 3.02457685563695e-08
4006 3.03074401131198e-08
4007 3.0233309189498e-08
4008 3.0344281753969e-08
4009 3.02086995418449e-08
4010 3.02597875645461e-08
4011 3.01975013883293e-08
4012 3.02513250005632e-08
4013 3.01840081817772e-08
4014 3.01489393450538e-08
4015 3.02414093766856e-08
4016 3.01306037897575e-08
4017 3.01509963662738e-08
4018 3.0125367089795e-08
4019 3.01373574984609e-08
4020 3.00930125263221e-08
4021 3.01811589054068e-08
4022 3.00773663752807e-08
4023 3.00925506735439e-08
4024 3.00534246377993e-08
4025 3.00591658231042e-08
4026 3.00403577568886e-08
4027 3.00527602803413e-08
4028 3.00142772857726e-08
4029 3.00151690169059e-08
4030 2.99922184865409e-08
4031 3.00053031310199e-08
4032 2.99662481495488e-08
4033 2.99608977627486e-08
4034 2.99379294688151e-08
4035 2.9947266000363e-08
4036 2.99025622041427e-08
4037 2.99001108317043e-08
4038 2.98806099863214e-08
4039 2.98909483831267e-08
4040 2.98515203667193e-08
4041 2.98467774939581e-08
4042 2.98335010029405e-08
4043 2.98636173567957e-08
4044 2.98362756723236e-08
4045 2.98456903635724e-08
4046 2.98238234108794e-08
4047 2.98265092624206e-08
4048 2.97812796645758e-08
4049 2.97924351855272e-08
4050 2.97445392760665e-08
4051 2.97663138582038e-08
4052 2.97459727960359e-08
4053 2.97479800792644e-08
4054 2.97237043866971e-08
4055 2.97252888969979e-08
4056 2.97052213937832e-08
4057 2.97012316963219e-08
4058 2.96759381512857e-08
4059 2.96848181591258e-08
4060 2.96564834911806e-08
4061 2.96622708617633e-08
4062 2.96380822106812e-08
4063 2.96388869003295e-08
4064 2.96236084551538e-08
4065 2.96288682477552e-08
4066 2.95966557928296e-08
4067 2.95964621699341e-08
4068 2.95891560142536e-08
4069 2.95786986015401e-08
4070 2.95367517111345e-08
4071 2.9540096591063e-08
4072 2.95089979118757e-08
4073 2.95246849191244e-08
4074 2.94985760262989e-08
4075 2.94991302496328e-08
4076 2.94720479132593e-08
4077 2.94784161525286e-08
4078 2.94537993994481e-08
4079 2.94461983685324e-08
4080 2.94326945038392e-08
4081 2.94380431142827e-08
4082 2.94156468072515e-08
4083 2.94183024607264e-08
4084 2.93922823857429e-08
4085 2.93957391761523e-08
4086 2.93853510413555e-08
4087 2.93749149449241e-08
4088 2.93603985568325e-08
4089 2.93633810599658e-08
4090 2.93331599010571e-08
4091 2.93385600258489e-08
4092 2.93279320828788e-08
4093 2.93191426692374e-08
4094 2.93044575272461e-08
4095 2.92918116429064e-08
4096 2.92839530402489e-08
4097 2.92819031244562e-08
4098 2.92597270856731e-08
4099 2.93389774697062e-08
4100 2.9232154474812e-08
4101 2.93175386190114e-08
4102 2.92104402888071e-08
4103 2.92721811234742e-08
4104 2.91851343092731e-08
4105 2.92572917004463e-08
4106 2.91660207096811e-08
4107 2.92357533737686e-08
4108 2.91460011681011e-08
4109 2.92134760826457e-08
4110 2.91231927462832e-08
4111 2.91889037384863e-08
4112 2.9105441612387e-08
4113 2.91684010278459e-08
4114 2.92449033878484e-08
4115 2.90652284462567e-08
4116 2.91291915033298e-08
4117 2.91268253960197e-08
4118 2.91482926684239e-08
4119 2.91746875547005e-08
4120 2.89809065634472e-08
4121 2.89867916336561e-08
4122 2.89844326317734e-08
4123 2.89796684427301e-08
4124 2.89767960737208e-08
4125 2.89744743753317e-08
4126 2.8966054443913e-08
4127 2.89603150349649e-08
4128 2.8943036411988e-08
4129 2.89493229388427e-08
4130 2.89322485969024e-08
4131 2.89284152188429e-08
4132 2.89153234689365e-08
4133 2.89087278559919e-08
4134 2.88976913509487e-08
4135 2.88864114850185e-08
4136 2.89485626581154e-08
4137 2.89447559254086e-08
4138 2.89134636233257e-08
4139 2.88700903183781e-08
4140 2.88888006849675e-08
4141 2.8854207911877e-08
4142 2.88574319995405e-08
4143 2.88237433920813e-08
4144 2.88333446007982e-08
4145 2.88030150841223e-08
4146 2.88071895226949e-08
4147 2.87765189455058e-08
4148 2.87878680893527e-08
4149 2.87513159946684e-08
4150 2.87656138908687e-08
4151 2.87276709087791e-08
4152 2.87350712113721e-08
4153 2.86985031294762e-08
4154 2.87164851897614e-08
4155 2.86786878689327e-08
4156 2.86987020814422e-08
4157 2.86638464075395e-08
4158 2.86809385130482e-08
4159 2.86410788419289e-08
4160 2.86638179858301e-08
4161 2.86230061874448e-08
4162 2.86467987109518e-08
4163 2.85620913587081e-08
4164 2.8571273347211e-08
4165 2.85700156865687e-08
4166 2.85744210515304e-08
4167 2.85636758690089e-08
4168 2.86538917038115e-08
4169 2.855223790732e-08
4170 2.85519305975868e-08
4171 2.85347887540865e-08
4172 2.86144814509726e-08
4173 2.85583485748475e-08
4174 2.86051555775657e-08
4175 2.85019901014039e-08
4176 2.84973058484184e-08
4177 2.84825230068009e-08
4178 2.85615371353742e-08
4179 2.85859371729202e-08
4180 2.85300973956737e-08
4181 2.85566787994185e-08
4182 2.85281682721461e-08
4183 2.85053864956808e-08
4184 2.84967089925203e-08
4185 2.84808105988077e-08
4186 2.84384285009764e-08
4187 2.84613896894825e-08
4188 2.84555170537715e-08
4189 2.84410841544513e-08
4190 2.83362240338647e-08
4191 2.83049725879891e-08
4192 2.83005210377496e-08
4193 2.84158794272571e-08
4194 2.82910672666503e-08
4195 2.82699694764688e-08
4196 2.82780625582291e-08
4197 2.82668750628545e-08
4198 2.825980160992e-08
4199 2.82516676719524e-08
4200 2.82518506367069e-08
4201 2.82340213431098e-08
4202 2.82365242298965e-08
4203 2.82252532457505e-08
4204 2.82268590723334e-08
4205 2.82144014818186e-08
4206 2.82176486621211e-08
4207 2.82658199068919e-08
4208 2.8271061935925e-08
4209 2.82583023647476e-08
4210 2.82727565803498e-08
4211 2.8252280515062e-08
4212 2.8257069573101e-08
4213 2.82427947695396e-08
4214 2.82431091847002e-08
4215 2.82145009578016e-08
4216 2.82234875470522e-08
4217 2.81971193061281e-08
4218 2.82036385357287e-08
4219 2.81682250857784e-08
4220 2.81936234358682e-08
4221 2.81621304054624e-08
4222 2.81773928634266e-08
4223 2.81332397378264e-08
4224 2.81521082001746e-08
4225 2.81918897115929e-08
4226 2.81359930909275e-08
4227 2.80991159229416e-08
4228 2.80966290233664e-08
4229 2.80879639547038e-08
4230 2.81135079660544e-08
4231 2.80527601148606e-08
4232 2.8253696271463e-08
4233 2.80527743257153e-08
4234 2.82175740551338e-08
4235 2.79904508460049e-08
4236 2.80197234303614e-08
4237 2.79794143409617e-08
4238 2.80076264402851e-08
4239 2.79563625582568e-08
4240 2.79767693456279e-08
4241 2.79509198009009e-08
4242 2.79652390133833e-08
4243 2.79226171073788e-08
4244 2.79523728607955e-08
4245 2.79145346837595e-08
4246 2.7944990321771e-08
4247 2.78962541955252e-08
4248 2.79171086248198e-08
4249 2.78896319372279e-08
4250 2.78987037916067e-08
4251 2.78499499017926e-08
4252 2.78824661137378e-08
4253 2.78357781269278e-08
4254 2.78617786619861e-08
4255 2.78422511712506e-08
4256 2.78976877154946e-08
4257 2.77590963548846e-08
4258 2.78673262243956e-08
4259 2.79261946900533e-08
4260 2.77774248047535e-08
4261 2.77345222343683e-08
4262 2.77564158324139e-08
4263 2.77000093973356e-08
4264 2.77369522905246e-08
4265 2.76953091571386e-08
4266 2.77252070191025e-08
4267 2.76967906387426e-08
4268 2.78319252089432e-08
4269 2.76628213669028e-08
4270 2.76776006558066e-08
4271 2.76175313729254e-08
4272 2.76454112935198e-08
4273 2.75891789414118e-08
4274 2.7609583952426e-08
4275 2.75548135419967e-08
4276 2.7593003437687e-08
4277 2.75316551778815e-08
4278 2.75617288991725e-08
4279 2.74959681689779e-08
4280 2.75364069324269e-08
4281 2.74748845896511e-08
4282 2.75021321272106e-08
4283 2.74677347533725e-08
4284 2.75279585792987e-08
4285 2.74582898640574e-08
4286 2.74289568835684e-08
4287 2.74390625776277e-08
4288 2.74088929330674e-08
4289 2.74114100307088e-08
4290 2.74059619442824e-08
4291 2.74118878706986e-08
4292 2.74449405424093e-08
4293 2.73702536190967e-08
4294 2.74093316932067e-08
4295 2.73264628702918e-08
4296 2.73873457246054e-08
4297 2.73436082665057e-08
4298 2.72783040600189e-08
4299 2.72654698818542e-08
4300 2.73302180886503e-08
4301 2.72552309610319e-08
4302 2.74533658028986e-08
4303 2.72174531801284e-08
4304 2.72331561745887e-08
4305 2.72259903510985e-08
4306 2.72251732269524e-08
4307 2.71982081301303e-08
4308 2.7284894343893e-08
4309 2.71787570227389e-08
4310 2.72292197678325e-08
4311 2.71484221769924e-08
4312 2.71906337445671e-08
4313 2.72136713164173e-08
4314 2.71908788818109e-08
4315 2.70820894598955e-08
4316 2.71922591110751e-08
4317 2.71085429659479e-08
4318 2.72268394496678e-08
4319 2.70858393491835e-08
4320 2.71042193134008e-08
4321 2.70849334071954e-08
4322 2.70835585070017e-08
4323 2.70830451398751e-08
4324 2.71455160572032e-08
4325 2.69443258815727e-08
4326 2.70806399527146e-08
4327 2.71048374855809e-08
4328 2.70190589901631e-08
4329 2.69401727592822e-08
4330 2.7013529191322e-08
4331 2.68927085045334e-08
4332 2.68769309030858e-08
4333 2.69520032958326e-08
4334 2.69918078998899e-08
4335 2.69142663711364e-08
4336 2.69058446633608e-08
4337 2.70941633573329e-08
4338 2.69529678575964e-08
4339 2.69164459609783e-08
4340 2.68134670022846e-08
4341 2.68413060666717e-08
4342 2.69675624053889e-08
4343 2.68085056376322e-08
4344 2.67359094863195e-08
4345 2.68114686008403e-08
4346 2.69315307832585e-08
4347 2.69036828370872e-08
4348 2.68143320880654e-08
4349 2.68172186679294e-08
4350 2.67948596643919e-08
4351 2.6853513190872e-08
4352 2.68195083918954e-08
4353 2.6781441064827e-08
4354 2.6725208712719e-08
4355 2.6750935688824e-08
4356 2.67779878271313e-08
4357 2.66915982649607e-08
4358 2.67281858867818e-08
4359 2.66884683242097e-08
4360 2.68061857155999e-08
4361 2.66871396092938e-08
4362 2.66858002362369e-08
4363 2.66610076238294e-08
4364 2.66675215243595e-08
4365 2.66508433099943e-08
4366 2.66526960501778e-08
4367 2.66332680354253e-08
4368 2.66434234674762e-08
4369 2.66215387512148e-08
4370 2.66317066177635e-08
4371 2.65970960811046e-08
4372 2.66163358020322e-08
4373 2.65891308970367e-08
4374 2.66027821993475e-08
4375 2.65737138960276e-08
4376 2.65732769122451e-08
4377 2.65503992125105e-08
4378 2.64279567119274e-08
4379 2.65146677946859e-08
4380 2.66194799536379e-08
4381 2.64925912318859e-08
4382 2.65019366452179e-08
4383 2.64614588019185e-08
4384 2.64796611304519e-08
4385 2.64798600824179e-08
4386 2.64995350107711e-08
4387 2.65236135277291e-08
4388 2.65269175514504e-08
4389 2.63951900336679e-08
4390 2.65851376468618e-08
4391 2.64676636163585e-08
4392 2.66003645776891e-08
4393 2.64812207717569e-08
4394 2.64899249202699e-08
4395 2.65301256519024e-08
4396 2.66413167082646e-08
4397 2.66539874616001e-08
4398 2.65761794793207e-08
4399 2.65853703496077e-08
4400 2.64800537053134e-08
4401 2.65521151732173e-08
4402 2.65335344806772e-08
4403 2.64616808465234e-08
4404 2.64954564954678e-08
4405 2.65287116718582e-08
4406 2.64047379516796e-08
4407 2.65732769122451e-08
4408 2.64092285817696e-08
4409 2.63979380576984e-08
4410 2.65393840237493e-08
4411 2.65115218667233e-08
4412 2.63445834036702e-08
4413 2.65538453447789e-08
4414 2.64660506843484e-08
4415 2.63839243785924e-08
4416 2.64537671768039e-08
4417 2.64705839470025e-08
4418 2.64706425667782e-08
4419 2.6578387490872e-08
4420 2.6459838764481e-08
4421 2.64593698062754e-08
4422 2.64335096034074e-08
4423 2.65655692999189e-08
4424 2.6638581118732e-08
4425 2.66204587262564e-08
4426 2.64634749669312e-08
4427 2.63497348385044e-08
4428 2.64390784820989e-08
4429 2.64406612160428e-08
4430 2.6427839472376e-08
4431 2.62634536341011e-08
4432 2.62881769685919e-08
4433 2.62367390035934e-08
4434 2.63702428782153e-08
4435 2.63847290682406e-08
4436 2.62104666859386e-08
4437 2.62355452917973e-08
4438 2.64108326319956e-08
4439 2.63913708664631e-08
4440 2.63540833600473e-08
4441 2.63112553966494e-08
4442 2.63145008005949e-08
4443 2.6294358690393e-08
4444 2.62824482177848e-08
4445 2.6169205469273e-08
4446 2.61486761132801e-08
4447 2.62610644341521e-08
4448 2.61485340047329e-08
4449 2.61535237910948e-08
4450 2.6319444401679e-08
4451 2.62794657146515e-08
4452 2.61801460510469e-08
4453 2.62835815334483e-08
4454 2.61956731861801e-08
4455 2.61049706296035e-08
4456 2.6197930935723e-08
4457 2.62155879227066e-08
4458 2.5998836861163e-08
4459 2.60622652348275e-08
4460 2.61370693976914e-08
4461 2.61018762159893e-08
4462 2.59366572663566e-08
4463 2.60173269595043e-08
4464 2.60475729874088e-08
4465 2.59840398086908e-08
4466 2.59328061247288e-08
4467 2.5940911640987e-08
4468 2.58934687025203e-08
4469 2.59271200064859e-08
4470 2.58155985477515e-08
4471 2.58945309639103e-08
4472 2.58382062412466e-08
4473 2.58659778040737e-08
4474 2.58111700901509e-08
4475 2.588259917502e-08
4476 2.5787631585672e-08
4477 2.58264858388202e-08
4478 2.57224730404459e-08
4479 2.5834518524448e-08
4480 2.5686579974149e-08
4481 2.5774800960221e-08
4482 2.56542289633899e-08
4483 2.57153818239431e-08
4484 2.56299923506731e-08
4485 2.57056065322558e-08
4486 2.55976981833328e-08
4487 2.56759093986147e-08
4488 2.55831587026023e-08
4489 2.5654802726649e-08
4490 2.55487755396189e-08
4491 2.56330920933578e-08
4492 2.5529429237281e-08
4493 2.5612152398935e-08
4494 2.55029153350961e-08
4495 2.56343923865643e-08
4496 2.55158330020322e-08
4497 2.55498822099298e-08
4498 2.54948187006221e-08
4499 2.5600748188026e-08
4500 2.54658640841399e-08
4501 2.55518166625279e-08
4502 2.54354510786925e-08
4503 2.55400998128152e-08
4504 2.53847218800729e-08
4505 2.54630965201841e-08
4506 2.53520866522194e-08
4507 2.54480809758206e-08
4508 2.53793857041273e-08
4509 2.55959573536302e-08
4510 2.54586094428078e-08
4511 2.53697312047052e-08
4512 2.53240060033022e-08
4513 2.5374976786452e-08
4514 2.52782932363971e-08
4515 2.5438122719379e-08
4516 2.52529854805061e-08
4517 2.53547618456196e-08
4518 2.54181617975746e-08
4519 2.5413942950081e-08
4520 2.52804639444548e-08
4521 2.54208067929085e-08
4522 2.52409009249277e-08
4523 2.53701060159983e-08
4524 2.52949572399075e-08
4525 2.54200482885381e-08
4526 2.52515430787525e-08
4527 2.53775969127901e-08
4528 2.52812935030988e-08
4529 2.53275391770558e-08
4530 2.52297063241258e-08
4531 2.53989913545638e-08
4532 2.5192584018896e-08
4533 2.53086263057867e-08
4534 2.52718024285059e-08
4535 2.53383714010624e-08
4536 2.51915786009249e-08
4537 2.52997498506602e-08
4538 2.51564742370647e-08
4539 2.5217525845278e-08
4540 2.50868517071012e-08
4541 2.52680383283632e-08
4542 2.50545824087567e-08
4543 2.51687097829745e-08
4544 2.50770373355635e-08
4545 2.52137812850606e-08
4546 2.50529854639581e-08
4547 2.51766625325445e-08
4548 2.49800251594934e-08
4549 2.51080702895479e-08
4550 2.49338434343827e-08
4551 2.50624996311899e-08
4552 2.49238247818084e-08
4553 2.50438265680941e-08
4554 2.48825813287112e-08
4555 2.50275817847978e-08
4556 2.48676581549034e-08
4557 2.49913192362783e-08
4558 2.4829116540559e-08
4559 2.49844926969445e-08
4560 2.47993252600054e-08
4561 2.49299034749129e-08
4562 2.47869866853989e-08
4563 2.49625795589736e-08
4564 2.47837910194448e-08
4565 2.4911352980439e-08
4566 2.47458888935626e-08
4567 2.48800109403646e-08
4568 2.47088483007474e-08
4569 2.48447697970278e-08
4570 2.46627340771965e-08
4571 2.48044553785576e-08
4572 2.46193554431784e-08
4573 2.47624480920194e-08
4574 2.45914293373062e-08
4575 2.4611340521119e-08
4576 2.44775968383237e-08
4577 2.46158275984953e-08
4578 2.44115145875412e-08
4579 2.45065088222418e-08
4580 2.43561384394297e-08
4581 2.4479076543571e-08
4582 2.430962453559e-08
4583 2.44397018178688e-08
4584 2.43843860658899e-08
4585 2.44096423074325e-08
4586 2.42089193136508e-08
4587 2.438188495546e-08
4588 2.42506086323147e-08
4589 2.43812454669978e-08
4590 2.42810340722599e-08
4591 2.44055957665523e-08
4592 2.42650806114852e-08
4593 2.44376465730056e-08
4594 2.4271155751876e-08
4595 2.44321984865792e-08
4596 2.43445121839159e-08
4597 2.44035263108344e-08
4598 2.42682283158047e-08
4599 2.44849918118462e-08
4600 2.42710971321003e-08
4601 2.43806859145934e-08
4602 2.4255349728719e-08
4603 2.43791422605e-08
4604 2.42531008609603e-08
4605 2.43989592974003e-08
4606 2.42625866064827e-08
4607 2.43945219580155e-08
4608 2.42628939162159e-08
4609 2.43630200458256e-08
4610 2.42546462914106e-08
4611 2.4349819938152e-08
4612 2.42252991000669e-08
4613 2.43268072352976e-08
4614 2.4209381166429e-08
4615 2.43318059034436e-08
4616 2.41751276774949e-08
4617 2.43471305338971e-08
4618 2.41739765982629e-08
4619 2.42054571941708e-08
4620 2.41794673172535e-08
4621 2.43015936263191e-08
4622 2.42112871973177e-08
4623 2.42340991718493e-08
4624 2.40893243130813e-08
4625 2.41604194428646e-08
4626 2.40722179967179e-08
4627 2.41359465746882e-08
4628 2.40609985269202e-08
4629 2.41209594520342e-08
4630 2.40453150723852e-08
4631 2.41117028565441e-08
4632 2.4009635168909e-08
4633 2.41015953861279e-08
4634 2.40654607353008e-08
4635 2.40757600522556e-08
4636 2.4049608526866e-08
4637 2.40001174489635e-08
4638 2.40102924209395e-08
4639 2.3966519435703e-08
4640 2.39155681924785e-08
4641 2.40208901658434e-08
4642 2.38936319618688e-08
4643 2.39339090768453e-08
4644 2.38851765033132e-08
4645 2.39333477480841e-08
4646 2.3861840503514e-08
4647 2.39029471771346e-08
4648 2.39536461776879e-08
4649 2.39599504681109e-08
4650 2.38391475448907e-08
4651 2.38571633559559e-08
4652 2.38474093805507e-08
4653 2.3912045676866e-08
4654 2.38420376774684e-08
4655 2.3837097629098e-08
4656 2.38521806750214e-08
4657 2.38245565498119e-08
4658 2.38320048140395e-08
4659 2.39051800576817e-08
4660 2.38217161552257e-08
4661 2.37831603300265e-08
4662 2.38883366421305e-08
4663 2.37794068880248e-08
4664 2.37990764873075e-08
4665 2.38129977958579e-08
4666 2.37410695547169e-08
4667 2.38620536663348e-08
4668 2.37093153998558e-08
4669 2.37078907616706e-08
4670 2.36873276548977e-08
4671 2.37072264042126e-08
4672 2.36734560843388e-08
4673 2.36706920730967e-08
4674 2.366647500196e-08
4675 2.36781954043863e-08
4676 2.36644268625241e-08
4677 2.36204762416037e-08
4678 2.36443895573757e-08
4679 2.36633574957068e-08
4680 2.36310437884413e-08
4681 2.3585641883983e-08
4682 2.36032100531247e-08
4683 2.35586057328874e-08
4684 2.35965185169107e-08
4685 2.35591901542875e-08
4686 2.35859918262804e-08
4687 2.3502730428504e-08
4688 2.35493704536793e-08
4689 2.35182593399941e-08
4690 2.35216770505531e-08
4691 2.34850521252383e-08
4692 2.34933192899689e-08
4693 2.34613164451503e-08
4694 2.35000552351039e-08
4695 2.34612294036651e-08
4696 2.34972201695882e-08
4697 2.34160797418781e-08
4698 2.34778507746114e-08
4699 2.34389183617623e-08
4700 2.34137758070574e-08
4701 2.33598740351226e-08
4702 2.33944916772089e-08
4703 2.34162982337693e-08
4704 2.33909922542352e-08
4705 2.34266863685662e-08
4706 2.33697701190749e-08
4707 2.32381882625532e-08
4708 2.34535182386253e-08
4709 2.33702710517036e-08
4710 2.33931647386498e-08
4711 2.33508679059469e-08
4712 2.30436842940662e-08
4713 2.28928129786254e-08
4714 2.29552128416799e-08
4715 2.2903236640559e-08
4716 2.30213021978898e-08
4717 2.30117613853054e-08
4718 2.31390053784253e-08
4719 2.298901335962e-08
4720 2.30904220188677e-08
4721 2.31156249697051e-08
4722 2.31777175230263e-08
4723 2.30733547823547e-08
4724 2.31504735381804e-08
4725 2.31803660710739e-08
4726 2.3198847287631e-08
4727 2.30618102392555e-08
4728 2.31561898544896e-08
4729 2.32087042917328e-08
4730 2.32476278227978e-08
4731 2.30737651207846e-08
4732 2.31725429955532e-08
4733 2.32260131127759e-08
4734 2.32539321132208e-08
4735 2.30730492489784e-08
4736 2.31903403147271e-08
4737 2.32201013972144e-08
4738 2.32224675045245e-08
4739 2.30639347620354e-08
4740 2.31575452147581e-08
4741 2.31474714951219e-08
4742 2.31587797827615e-08
4743 2.30100933862332e-08
4744 2.31212204937492e-08
4745 2.31345236301195e-08
4746 2.3160920292753e-08
4747 2.30048424754159e-08
4748 2.31075194534469e-08
4749 2.31220642632479e-08
4750 2.31407053519206e-08
4751 2.29839738352666e-08
4752 2.30932375444581e-08
4753 2.30855015104225e-08
4754 2.31068835176984e-08
4755 2.29774190785292e-08
4756 2.30785328625416e-08
4757 2.30318111249517e-08
4758 2.30624390695766e-08
4759 2.29452243871719e-08
4760 2.30181971261345e-08
4761 2.30009078450166e-08
4762 2.30356924646458e-08
4763 2.28666383605969e-08
4764 2.29896510717253e-08
4765 2.32976180569722e-08
4766 2.34805437315799e-08
4767 2.26072778275466e-08
4768 2.28283010272889e-08
4769 2.28637464516623e-08
4770 2.28461036755334e-08
4771 2.2813225086793e-08
4772 2.28903225263366e-08
4773 2.27269243424644e-08
4774 2.28508181265852e-08
4775 2.28966854365353e-08
4776 2.27258283302945e-08
4777 2.26628795729766e-08
4778 2.2754635509159e-08
4779 2.27245511297269e-08
4780 2.27685870157757e-08
4781 2.25407070786332e-08
4782 2.27296030885782e-08
4783 2.26471996711552e-08
4784 2.27109531181213e-08
4785 2.2659369491862e-08
4786 2.26896368360485e-08
4787 2.26459828667203e-08
4788 2.27237340055808e-08
4789 2.28331948903815e-08
4790 2.27280363418458e-08
4791 2.26513794387984e-08
4792 2.27391367957352e-08
4793 2.26600569419588e-08
4794 2.27382628281703e-08
4795 2.26558434235358e-08
4796 2.27424052923197e-08
4797 2.26557812510464e-08
4798 2.27431193877692e-08
4799 2.26740830555627e-08
4800 2.27585452705625e-08
4801 2.26267822256432e-08
4802 2.27777938732743e-08
4803 2.26153407112406e-08
4804 2.27654481932404e-08
4805 2.25999858827208e-08
4806 2.27572307665014e-08
4807 2.26033751715704e-08
4808 2.27502745531183e-08
4809 2.25637570849813e-08
4810 2.27408296638032e-08
4811 2.25843823642435e-08
4812 2.27418848197658e-08
4813 2.25408598453214e-08
4814 2.26841745387674e-08
4815 2.25887646365663e-08
4816 2.26844321105091e-08
4817 2.25871783499088e-08
4818 2.26834639960316e-08
4819 2.25016023591706e-08
4820 2.26802949754301e-08
4821 2.25133476305928e-08
4822 2.26837642003375e-08
4823 2.24828529127308e-08
4824 2.26368808142752e-08
4825 2.24984439967102e-08
4826 2.26214886822618e-08
4827 2.24894520783891e-08
4828 2.26185736806883e-08
4829 2.24851035568463e-08
4830 2.26096652511387e-08
4831 2.24837144457979e-08
4832 2.26035368200428e-08
4833 2.24729816977742e-08
4834 2.25979004397914e-08
4835 2.24655707370403e-08
4836 2.25919194463131e-08
4837 2.24465814824271e-08
4838 2.25865388614466e-08
4839 2.243739061214e-08
4840 2.25698730815793e-08
4841 2.24339142818053e-08
4842 2.25665228725802e-08
4843 2.24310170438002e-08
4844 2.256011022439e-08
4845 2.24265903625565e-08
4846 2.25452119195779e-08
4847 2.24057288278345e-08
4848 2.25494112271463e-08
4849 2.23918892316988e-08
4850 2.25557528210629e-08
4851 2.2334580407346e-08
4852 2.25300595957378e-08
4853 2.23404992283349e-08
4854 2.25033822687237e-08
4855 2.23265761434277e-08
4856 2.24696634631982e-08
4857 2.23626237527697e-08
4858 2.24649241431507e-08
4859 2.23250768982552e-08
4860 2.24689671313172e-08
4861 2.23399734267105e-08
4862 2.24497416212444e-08
4863 2.22903437929745e-08
4864 2.2453162884517e-08
4865 2.22983143061128e-08
4866 2.24843557106169e-08
4867 2.22865796928318e-08
4868 2.24315197527858e-08
4869 2.23050218295384e-08
4870 2.24328484677017e-08
4871 2.22669029881217e-08
4872 2.2480318051521e-08
4873 2.22664642279824e-08
4874 2.24067324694488e-08
4875 2.22824230178276e-08
4876 2.24102709722729e-08
4877 2.22768168356424e-08
4878 2.24333813747535e-08
4879 2.22309459729786e-08
4880 2.23865068704754e-08
4881 2.22531397753301e-08
4882 2.23796661202869e-08
4883 2.2236692487354e-08
4884 2.23706351221153e-08
4885 2.22152767292982e-08
4886 2.2354910811373e-08
4887 2.22115765780018e-08
4888 2.23486349426594e-08
4889 2.22166516294919e-08
4890 2.2370981511699e-08
4891 2.21976392822398e-08
4892 2.23563940693339e-08
4893 2.21794085319971e-08
4894 2.23584013525624e-08
4895 2.21733298388926e-08
4896 2.23403926469246e-08
4897 2.21528289046091e-08
4898 2.23089244855146e-08
4899 2.21631761831986e-08
4900 2.22957652340483e-08
4901 2.21168896530344e-08
4902 2.22858318466024e-08
4903 2.2115798969935e-08
4904 2.22656701964752e-08
4905 2.21159890401168e-08
4906 2.22545981642952e-08
4907 2.20864038169566e-08
4908 2.22520473158738e-08
4909 2.20690257179967e-08
4910 2.2221495754593e-08
4911 2.21059632821152e-08
4912 2.22112852554801e-08
4913 2.20873115353015e-08
4914 2.22061231625048e-08
4915 2.2076466876797e-08
4916 2.21996216964726e-08
4917 2.20505675940785e-08
4918 2.21900560148924e-08
4919 2.20536673367633e-08
4920 2.21738307715214e-08
4921 2.20580353982314e-08
4922 2.21593001725751e-08
4923 2.20398916894737e-08
4924 2.21445741743764e-08
4925 2.20170939257969e-08
4926 2.21603890793176e-08
4927 2.2017225376203e-08
4928 2.21459917071343e-08
4929 2.19877396290258e-08
4930 2.21543547951342e-08
4931 2.19405862367239e-08
4932 2.21175646686333e-08
4933 2.20380158566513e-08
4934 2.20874554202055e-08
4935 2.20248637106124e-08
4936 2.21052918192299e-08
4937 2.19677129820184e-08
4938 2.20924238902853e-08
4939 2.19576303805979e-08
4940 2.2102979002625e-08
4941 2.19348308405642e-08
4942 2.20812648166202e-08
4943 2.1954775775157e-08
4944 2.20537152983979e-08
4945 2.19261213629807e-08
4946 2.20989573307406e-08
4947 2.18837286070084e-08
4948 2.2056800830228e-08
4949 2.19375753118811e-08
4950 2.20774118986355e-08
4951 2.18628883885685e-08
4952 2.20656541927156e-08
4953 2.188922998414e-08
4954 2.20515126159171e-08
4955 2.18863753786991e-08
4956 2.20213447477136e-08
4957 2.19276117263689e-08
4958 2.20122124972022e-08
4959 2.18759357295539e-08
4960 2.20008082862932e-08
4961 2.18805134011291e-08
4962 2.19680860169547e-08
4963 2.18270876928273e-08
4964 2.19827533953776e-08
4965 2.18935962692512e-08
4966 2.20082174706704e-08
4967 2.18745110913687e-08
4968 2.19961346914488e-08
4969 2.19163194259409e-08
4970 2.19744062945892e-08
4971 2.18839133481197e-08
4972 2.19938609546944e-08
4973 2.1850638631804e-08
4974 2.19945448520775e-08
4975 2.1853747256273e-08
4976 2.19764206832451e-08
4977 2.18445741495543e-08
4978 2.19723599315103e-08
4979 2.18213997982275e-08
4980 2.1965881558117e-08
4981 2.17986801942516e-08
4982 2.19694911152146e-08
4983 2.18407496532791e-08
4984 2.18994369305392e-08
4985 2.17910116617759e-08
4986 2.1929977833679e-08
4987 2.18265956419827e-08
4988 2.18678017915863e-08
4989 2.18155449260848e-08
4990 2.18905924498358e-08
4991 2.17545839120703e-08
4992 2.18855671363372e-08
4993 2.18220268521918e-08
4994 2.18592042244836e-08
4995 2.17687503578645e-08
4996 2.18688125386279e-08
4997 2.17845474992373e-08
4998 2.18518501071685e-08
4999 2.17676863201177e-08
};
\addlegendentry{Test}

\nextgroupplot[
title={Batch Size 16 $\rare$},
ymin=1.04995733040024e-08, ymax=1e-05,
]
\addplot [semithick, black, dashed]
table {%
0 0.0157170567642897
1 0.00494477174337953
2 0.00178249403554946
3 0.00104628965561278
4 0.000540450631990097
5 0.000267249523603823
6 0.000195556676466367
7 0.000179966989569948
8 0.000174315613148792
9 0.000170275954966201
10 0.000166128802236926
11 0.000161243435519282
12 0.000155176517626387
13 0.000147447619914601
14 0.000137505138925917
15 0.000124870715357247
16 0.000109628831036389
17 9.26289391609316e-05
18 7.54085500957444e-05
19 5.99092843294784e-05
20 4.77161089384026e-05
21 3.93279772288224e-05
22 3.40980909077189e-05
23 3.09422128148071e-05
24 2.89322912085481e-05
25 2.72737210743799e-05
26 2.54317353028455e-05
27 2.28156418916114e-05
28 2.01819365893243e-05
29 1.77557636425263e-05
30 1.5502294282669e-05
31 1.34128514700933e-05
32 1.15395934462867e-05
33 9.93898418664685e-06
34 8.62693892850075e-06
35 7.58853412253302e-06
36 6.78498193133237e-06
37 6.16869731970837e-06
38 5.69058382416188e-06
39 5.31324215398854e-06
40 5.00480225616684e-06
41 4.7444518992279e-06
42 4.51447482453204e-06
43 4.30798976503866e-06
44 4.11608894853543e-06
45 3.93454810227922e-06
46 3.75880521266936e-06
47 3.58901737558881e-06
48 3.42562672744862e-06
49 3.26669840717386e-06
50 3.11160224288187e-06
51 2.96258664974403e-06
52 2.82159659042236e-06
53 2.68886794032142e-06
54 2.56387212152731e-06
55 2.44723567112715e-06
56 2.34057172980329e-06
57 2.24493696725858e-06
58 2.15844092952011e-06
59 2.0788237059719e-06
60 1.99667843895668e-06
61 1.90497609787599e-06
62 1.79736609072734e-06
63 1.68848505768437e-06
64 1.59369150190969e-06
65 1.51194072418548e-06
66 1.44278207682191e-06
67 1.38194286273574e-06
68 1.32816223009513e-06
69 1.2809325050398e-06
70 1.23942618449746e-06
71 1.20316316116487e-06
72 1.17089864494346e-06
73 1.14246875506296e-06
74 1.11685223265567e-06
75 1.09369868118847e-06
76 1.07318007189861e-06
77 1.05437576991108e-06
78 1.03672772496566e-06
79 1.02062385838053e-06
80 1.00592634129271e-06
81 9.92105790942332e-07
82 9.7894169545043e-07
83 9.66602892503943e-07
84 9.54589052753363e-07
85 9.42918769112566e-07
86 9.31959767683566e-07
87 9.2148744329279e-07
88 9.11429445267231e-07
89 9.01841054997021e-07
90 8.92400122751269e-07
91 8.83255546455075e-07
92 8.74700299107189e-07
93 8.65906616240863e-07
94 8.57358865317792e-07
95 8.48961237267076e-07
96 8.40659787854747e-07
97 8.3273339777179e-07
98 8.24562840250565e-07
99 8.16930169150964e-07
100 8.08981725157309e-07
101 8.01335254408286e-07
102 7.938409220003e-07
103 7.86272286035228e-07
104 7.79077216805035e-07
105 7.71825237762869e-07
106 7.64540684656367e-07
107 7.57455605054247e-07
108 7.50291343621257e-07
109 7.43385836699417e-07
110 7.36310198249157e-07
111 7.2935094047466e-07
112 7.22335523079209e-07
113 7.15656543093246e-07
114 7.08781524224378e-07
115 7.02204949249108e-07
116 6.95669085445161e-07
117 6.89374548812793e-07
118 6.83004592076486e-07
119 6.76743621340847e-07
120 6.70653783501507e-07
121 6.64517023594158e-07
122 6.58444838109062e-07
123 6.52467115031641e-07
124 6.46645066922247e-07
125 6.40550082820823e-07
126 6.34422589044448e-07
127 6.28629438992334e-07
128 6.22746939754393e-07
129 6.16957024178077e-07
130 6.11050226723364e-07
131 6.0553354312276e-07
132 6.0002472520182e-07
133 5.94472997008211e-07
134 5.89109220683781e-07
135 5.83457079159189e-07
136 5.78418984446216e-07
137 5.7332289055978e-07
138 5.68441488383087e-07
139 5.63714549315364e-07
140 5.59090271252671e-07
141 5.54642511374936e-07
142 5.50334273199837e-07
143 5.46195153745543e-07
144 5.42178440980479e-07
145 5.38200247120812e-07
146 5.34317242667726e-07
147 5.30567891871669e-07
148 5.26957831226582e-07
149 5.23221014177011e-07
150 5.19686565098709e-07
151 5.16319246941066e-07
152 5.1299718701614e-07
153 5.09589032390068e-07
154 5.06361061226812e-07
155 5.03223137172881e-07
156 5.00183228808737e-07
157 4.9724213354807e-07
158 4.9438490189857e-07
159 4.9162455306373e-07
160 4.89330421473255e-07
161 4.86669611902357e-07
162 4.83983281668543e-07
163 4.81407683992074e-07
164 4.78924291158478e-07
165 4.7652420248312e-07
166 4.74232413509412e-07
167 4.71954440158129e-07
168 4.69751064102297e-07
169 4.67584087743944e-07
170 4.65374450342892e-07
171 4.63257154720509e-07
172 4.6126252958345e-07
173 4.59285346551042e-07
174 4.57338758977244e-07
175 4.55391532227623e-07
176 4.53448316278582e-07
177 4.5134003435976e-07
178 4.49530939363285e-07
179 4.47626082561214e-07
180 4.45694395651231e-07
181 4.43928685967876e-07
182 4.42185508745752e-07
183 4.40439555106309e-07
184 4.3882785571725e-07
185 4.37217972901749e-07
186 4.35614033278853e-07
187 4.33923493403654e-07
188 4.32394612360554e-07
189 4.30824806841201e-07
190 4.29448743943794e-07
191 4.27996334764202e-07
192 4.2654474542303e-07
193 4.25223923485873e-07
194 4.23847797321741e-07
195 4.22281450582318e-07
196 4.20877642255846e-07
197 4.19539155572579e-07
198 4.18226270070932e-07
199 4.16828583993834e-07
200 4.15641147853307e-07
201 4.14374378181037e-07
202 4.12955820522143e-07
203 4.11872127500601e-07
204 4.10614599175574e-07
205 4.09486309507656e-07
206 4.08297302655569e-07
207 4.07291623872652e-07
208 4.06151649784192e-07
209 4.04880044627021e-07
210 4.03888402871644e-07
211 4.02805585792976e-07
212 4.01743319073944e-07
213 4.00739752507206e-07
214 3.99736535186435e-07
215 3.98833434005041e-07
216 3.97754167352105e-07
217 3.96727826768029e-07
218 3.95718042653925e-07
219 3.94693230930443e-07
220 3.9383311586505e-07
221 3.9275161007879e-07
222 3.9172802642895e-07
223 3.90782656552346e-07
224 3.89721222845196e-07
225 3.88612679515177e-07
226 3.87859798280488e-07
227 3.8695441405423e-07
228 3.8589167816383e-07
229 3.84909645688936e-07
230 3.83900921491431e-07
231 3.82917928007487e-07
232 3.82064893713618e-07
233 3.81161208053982e-07
234 3.80179631534361e-07
235 3.79220531740998e-07
236 3.78215918999558e-07
237 3.7710426870774e-07
238 3.76155795223099e-07
239 3.74884881878756e-07
240 3.73761657613159e-07
241 3.72517982626164e-07
242 3.71497203701665e-07
243 3.70091190802668e-07
244 3.6877978438099e-07
245 3.67199387056871e-07
246 3.65626293969967e-07
247 3.64062025596468e-07
248 3.62244684353641e-07
249 3.60450857769479e-07
250 3.58647998822903e-07
251 3.56561039907888e-07
252 3.54525802322314e-07
253 3.52287566386167e-07
254 3.50287391583493e-07
255 3.4827622160094e-07
256 3.46361628018599e-07
257 3.442452336202e-07
258 3.42522692250213e-07
259 3.4098111838432e-07
260 3.39507249535131e-07
261 3.3804830252393e-07
262 3.36976891986751e-07
263 3.35771423678466e-07
264 3.34539207671014e-07
265 3.33171887447747e-07
266 3.32210861387239e-07
267 3.30899532073659e-07
268 3.29677021355224e-07
269 3.28394180627356e-07
270 3.27252484524365e-07
271 3.26162355179349e-07
272 3.25076022861026e-07
273 3.24031567018324e-07
274 3.23017687350102e-07
275 3.22020703876547e-07
276 3.21058469410218e-07
277 3.20131131445578e-07
278 3.19224168208621e-07
279 3.18332019588752e-07
280 3.17414626763934e-07
281 3.16470977736572e-07
282 3.15361310200046e-07
283 3.14367141285743e-07
284 3.13330990451277e-07
285 3.12373081598594e-07
286 3.11041730753914e-07
287 3.09975267860807e-07
288 3.08969859794672e-07
289 3.07846929878508e-07
290 3.06794977639413e-07
291 3.05737578266019e-07
292 3.0466266837692e-07
293 3.03954595523237e-07
294 3.03101196678313e-07
295 3.02118985189281e-07
296 3.01218168196726e-07
297 3.00310456772479e-07
298 2.99479211918197e-07
299 2.98664209118726e-07
300 2.97876342834513e-07
301 2.97069236964376e-07
302 2.96287139768481e-07
303 2.95540525151239e-07
304 2.94592372938496e-07
305 2.93832868003108e-07
306 2.93041105045688e-07
307 2.92138023390009e-07
308 2.91344262571158e-07
309 2.90479741146044e-07
310 2.89638650002644e-07
311 2.88783956314376e-07
312 2.87966960804908e-07
313 2.87156888049367e-07
314 2.86230183249359e-07
315 2.85199473267994e-07
316 2.84329060711741e-07
317 2.83470531364571e-07
318 2.82711612797471e-07
319 2.8189036959958e-07
320 2.8108939352478e-07
321 2.80323620067691e-07
322 2.79618150941019e-07
323 2.78579928540523e-07
324 2.77787970787813e-07
325 2.76974661190366e-07
326 2.76175905796094e-07
327 2.75381109268835e-07
328 2.74673050796537e-07
329 2.73941170341629e-07
330 2.73343178925245e-07
331 2.72956742115582e-07
332 2.72267732029263e-07
333 2.71421207877154e-07
334 2.70669167790061e-07
335 2.69858946552404e-07
336 2.69127144292725e-07
337 2.68486125065692e-07
338 2.67699487956463e-07
339 2.67018376263195e-07
340 2.66254324294835e-07
341 2.65609372050335e-07
342 2.64946595834203e-07
343 2.64183357693071e-07
344 2.63509790364935e-07
345 2.62888811377593e-07
346 2.62296822654662e-07
347 2.61795223821082e-07
348 2.60765411624675e-07
349 2.60187987706217e-07
350 2.59608676287826e-07
351 2.58937510935198e-07
352 2.58401657028173e-07
353 2.57939879027447e-07
354 2.57488526514749e-07
355 2.56912723124003e-07
356 2.5638282590279e-07
357 2.55946380399052e-07
358 2.55382050632136e-07
359 2.54851634466036e-07
360 2.5443849939677e-07
361 2.53912852173244e-07
362 2.53413875391573e-07
363 2.52993562064319e-07
364 2.52566530065224e-07
365 2.52173733429117e-07
366 2.51742222630469e-07
367 2.51256598509997e-07
368 2.50831636513738e-07
369 2.50515218077396e-07
370 2.5000812983933e-07
371 2.49434692122463e-07
372 2.48966004456008e-07
373 2.48557739986666e-07
374 2.48152726946671e-07
375 2.47777446951147e-07
376 2.4732192088095e-07
377 2.46895147625992e-07
378 2.46492991010427e-07
379 2.46027582377906e-07
380 2.45625955962225e-07
381 2.45214678422201e-07
382 2.44813897467111e-07
383 2.44392247530811e-07
384 2.439667089007e-07
385 2.43561708266782e-07
386 2.43146217513868e-07
387 2.42745924310839e-07
388 2.42299840301996e-07
389 2.41924007092109e-07
390 2.41548250386359e-07
391 2.4110301239233e-07
392 2.40753827760898e-07
393 2.40407377113172e-07
394 2.40073736314628e-07
395 2.39752746800548e-07
396 2.39428752649928e-07
397 2.39145928070172e-07
398 2.38728963807944e-07
399 2.38479965169347e-07
400 2.38172701713779e-07
401 2.37863011172124e-07
402 2.37620931294202e-07
403 2.3731942192029e-07
404 2.36979224602862e-07
405 2.36684770619888e-07
406 2.36476611952696e-07
407 2.36186071909117e-07
408 2.35960636956634e-07
409 2.35469004792321e-07
410 2.34993137034678e-07
411 2.34665876455153e-07
412 2.34368971142374e-07
413 2.34073551396818e-07
414 2.33405668247144e-07
415 2.3353564838402e-07
416 2.33036555627564e-07
417 2.32731610225301e-07
418 2.32511658502688e-07
419 2.32205579486333e-07
420 2.31897210944965e-07
421 2.31614303118022e-07
422 2.31387427064078e-07
423 2.31009555875517e-07
424 2.30927051632079e-07
425 2.30570250145945e-07
426 2.30109878870621e-07
427 2.29755619159278e-07
428 2.29419798976949e-07
429 2.2913787184109e-07
430 2.28831625989301e-07
431 2.28494739516805e-07
432 2.28182287109746e-07
433 2.27858054557828e-07
434 2.27546540337187e-07
435 2.27196947875541e-07
436 2.26895820269135e-07
437 2.26625020268045e-07
438 2.26283218495382e-07
439 2.25993931898927e-07
440 2.25655906049838e-07
441 2.2537209050455e-07
442 2.2510194525438e-07
443 2.24828690761569e-07
444 2.24556286852362e-07
445 2.24300207833039e-07
446 2.24011245563815e-07
447 2.23872584399487e-07
448 2.23551614816131e-07
449 2.23257345162153e-07
450 2.22980819252427e-07
451 2.2271834546217e-07
452 2.22460312336636e-07
453 2.22206780733813e-07
454 2.21958781267517e-07
455 2.21711269247749e-07
456 2.21467916212248e-07
457 2.21235154199917e-07
458 2.20999668862021e-07
459 2.20767076868356e-07
460 2.20535877438977e-07
461 2.20308807939773e-07
462 2.20059808896167e-07
463 2.19824142021707e-07
464 2.19612100529787e-07
465 2.1938929498333e-07
466 2.19157623597255e-07
467 2.1896537287347e-07
468 2.18743536777311e-07
469 2.18522104340479e-07
470 2.18272721745905e-07
471 2.18055716807442e-07
472 2.1783752742266e-07
473 2.1761818953081e-07
474 2.17430983695976e-07
475 2.17206068221287e-07
476 2.16991443927839e-07
477 2.16773518530999e-07
478 2.16515748135748e-07
479 2.16300841934469e-07
480 2.16058449872492e-07
481 2.15869073706187e-07
482 2.1564869785351e-07
483 2.15435289980803e-07
484 2.15200336675991e-07
485 2.14946434098806e-07
486 2.1470916516364e-07
487 2.14493426646811e-07
488 2.14261333354671e-07
489 2.14043833416611e-07
490 2.13684532845093e-07
491 2.13446484544022e-07
492 2.13225306985976e-07
493 2.13031618301329e-07
494 2.12794633135616e-07
495 2.1251043921211e-07
496 2.12278367698104e-07
497 2.12043049543809e-07
498 2.11812802142219e-07
499 2.11570569483399e-07
500 2.11340852310116e-07
501 2.11117200315414e-07
502 2.10896458291643e-07
503 2.1067865063884e-07
504 2.10462695349634e-07
505 2.10248212908937e-07
506 2.10038392978618e-07
507 2.09967990151938e-07
508 2.09629130488054e-07
509 2.09356091531276e-07
510 2.09101517214094e-07
511 2.0885782751634e-07
512 2.08675676212522e-07
513 2.08429224507256e-07
514 2.08143801842198e-07
515 2.07879654823273e-07
516 2.07599692124916e-07
517 2.07531877720157e-07
518 2.07242607004332e-07
519 2.06976028913175e-07
520 2.06722016230287e-07
521 2.06477896398383e-07
522 2.06244004296252e-07
523 2.05996478037207e-07
524 2.05893390926803e-07
525 2.05615459904607e-07
526 2.05380138218914e-07
527 2.05141496991246e-07
528 2.04895898015423e-07
529 2.04667326379138e-07
530 2.04441963525426e-07
531 2.04216897543574e-07
532 2.0399652140668e-07
533 2.03779658797032e-07
534 2.03562826108339e-07
535 2.0334601751415e-07
536 2.03135240347763e-07
537 2.02906057346297e-07
538 2.02689399074529e-07
539 2.02473491917488e-07
540 2.02258676225142e-07
541 2.02040906152945e-07
542 2.01827895203621e-07
543 2.01646046050996e-07
544 2.0143258397809e-07
545 2.01214925588999e-07
546 2.00992143142287e-07
547 2.00781135099248e-07
548 2.00562146595473e-07
549 2.00357410903962e-07
550 2.00143986297974e-07
551 1.99943427894311e-07
552 1.99730849530511e-07
553 1.99533505224281e-07
554 1.99330869968151e-07
555 1.9912389529253e-07
556 1.98918559668471e-07
557 1.98708590382068e-07
558 1.98505848487684e-07
559 1.98292419888446e-07
560 1.98096410962023e-07
561 1.97904820154804e-07
562 1.9770706975919e-07
563 1.9748347308024e-07
564 1.97292339883859e-07
565 1.97073662711489e-07
566 1.96870710993835e-07
567 1.96664619174669e-07
568 1.96475939667096e-07
569 1.96348054068096e-07
570 1.96120539378342e-07
571 1.95933779174595e-07
572 1.95736368809207e-07
573 1.95534025024813e-07
574 1.95333484313664e-07
575 1.95124152348569e-07
576 1.95009315866912e-07
577 1.94775902379263e-07
578 1.94541645868185e-07
579 1.94320748363452e-07
580 1.9411251212631e-07
581 1.93909075257181e-07
582 1.93718050923053e-07
583 1.93516266911331e-07
584 1.93320917162509e-07
585 1.93129007548976e-07
586 1.92928064961961e-07
587 1.92837274340718e-07
588 1.92624708525102e-07
589 1.92408956387169e-07
590 1.92225819553471e-07
591 1.92020735894971e-07
592 1.91878943944346e-07
593 1.91675015734916e-07
594 1.91448435110431e-07
595 1.91271641604374e-07
596 1.91048381573466e-07
597 1.90840330120068e-07
598 1.90653138474772e-07
599 1.90466430986191e-07
600 1.90273618727588e-07
601 1.90071319252638e-07
602 1.89884589708811e-07
603 1.89694812256391e-07
604 1.89509724471293e-07
605 1.89317692722568e-07
606 1.89151745139782e-07
607 1.88957836265047e-07
608 1.88763875129894e-07
609 1.8860838834911e-07
610 1.88492758660175e-07
611 1.88309009992338e-07
612 1.88081675830176e-07
613 1.87860870838108e-07
614 1.87664490702844e-07
615 1.87469595530843e-07
616 1.87259174673216e-07
617 1.87060099065661e-07
618 1.86873242554952e-07
619 1.86672734876936e-07
620 1.86483752926847e-07
621 1.86293344697219e-07
622 1.86105415899362e-07
623 1.85915503529088e-07
624 1.8575178152247e-07
625 1.85554022223755e-07
626 1.85353178373759e-07
627 1.8521270283145e-07
628 1.85098913320303e-07
629 1.84897171813248e-07
630 1.84691331647002e-07
631 1.84487640524367e-07
632 1.84295574889859e-07
633 1.84111957629796e-07
634 1.83914716345157e-07
635 1.83755724989965e-07
636 1.83571663399107e-07
637 1.8338817892527e-07
638 1.8321129060439e-07
639 1.83026746398696e-07
640 1.82798685713692e-07
641 1.82628072302293e-07
642 1.8240755436949e-07
643 1.82209592708205e-07
644 1.81998398467442e-07
645 1.81880033061077e-07
646 1.81668815613989e-07
647 1.81459775753012e-07
648 1.81254756874694e-07
649 1.81092687050466e-07
650 1.80890330682359e-07
651 1.80684028741496e-07
652 1.80452216731908e-07
653 1.80272235375867e-07
654 1.80111763242508e-07
655 1.79915874568337e-07
656 1.79712098308471e-07
657 1.79537765347959e-07
658 1.79391292441267e-07
659 1.7920577938213e-07
660 1.79012431054559e-07
661 1.78831277722225e-07
662 1.78644232065039e-07
663 1.78457527518106e-07
664 1.78267877934957e-07
665 1.78093018554648e-07
666 1.77905254446387e-07
667 1.77723857326839e-07
668 1.7753547782462e-07
669 1.77379771038488e-07
670 1.77215719631363e-07
671 1.77056875465098e-07
672 1.76856059582065e-07
673 1.76671049281651e-07
674 1.76487213458643e-07
675 1.76302384872429e-07
676 1.76117043793056e-07
677 1.75936534887455e-07
678 1.75756247521974e-07
679 1.75606320475197e-07
680 1.75418388813853e-07
681 1.75196992749704e-07
682 1.75045590090406e-07
683 1.74854614705566e-07
684 1.74652316815127e-07
685 1.74424186383249e-07
686 1.74264706537031e-07
687 1.74055298188591e-07
688 1.73926764027499e-07
689 1.73701496798628e-07
690 1.73572706358982e-07
691 1.73321090102263e-07
692 1.73147434431087e-07
693 1.72996947163995e-07
694 1.72786218151089e-07
695 1.72599083192893e-07
696 1.72568876031676e-07
697 1.72387436194299e-07
698 1.72159440069208e-07
699 1.71958998173238e-07
700 1.71749518663944e-07
701 1.7157847770477e-07
702 1.71360882688987e-07
703 1.71189904619951e-07
704 1.70959397792103e-07
705 1.7079805885345e-07
706 1.70577748185963e-07
707 1.70391250101432e-07
708 1.70294054527176e-07
709 1.70153339674073e-07
710 1.69961250442441e-07
711 1.69790923798985e-07
712 1.69513855112768e-07
713 1.69349986016698e-07
714 1.69170517544615e-07
715 1.68975520637105e-07
716 1.68794829697561e-07
717 1.68674960491444e-07
718 1.68560368251747e-07
719 1.68348613023284e-07
720 1.68180192382295e-07
721 1.68044255872246e-07
722 1.67784312189667e-07
723 1.67622006166823e-07
724 1.67433583676768e-07
725 1.67142194790415e-07
726 1.66952169308843e-07
727 1.66777577675248e-07
728 1.66608952660852e-07
729 1.66446427947164e-07
730 1.66270591492435e-07
731 1.66101942618013e-07
732 1.65938628249762e-07
733 1.65765831525277e-07
734 1.65649616683083e-07
735 1.65385300405774e-07
736 1.65217369541892e-07
737 1.65014219057014e-07
738 1.6483034158199e-07
739 1.64667229618942e-07
740 1.64397729420784e-07
741 1.64213927241974e-07
742 1.64072907040236e-07
743 1.6379700686997e-07
744 1.63645981643867e-07
745 1.63382232301501e-07
746 1.6327906817537e-07
747 1.63015408730871e-07
748 1.62801310864324e-07
749 1.62617077627658e-07
750 1.62424714076792e-07
751 1.62247915383773e-07
752 1.62051023785637e-07
753 1.61872072872882e-07
754 1.6165106788435e-07
755 1.61600835717479e-07
756 1.61240600313306e-07
757 1.61194861057368e-07
758 1.60852658879662e-07
759 1.60787009278351e-07
760 1.60590858342857e-07
761 1.60234822551786e-07
762 1.60201742232857e-07
763 1.59901149771713e-07
764 1.59803023151994e-07
765 1.59525831236351e-07
766 1.59374475643403e-07
767 1.59135792628717e-07
768 1.58920532690843e-07
769 1.58900042478649e-07
770 1.58551003600849e-07
771 1.58445956166986e-07
772 1.58137577130901e-07
773 1.58083954872268e-07
774 1.57764831811846e-07
775 1.57709431590547e-07
776 1.57401405722624e-07
777 1.57347303854749e-07
778 1.57050745258402e-07
779 1.56966653740653e-07
780 1.56687031115155e-07
781 1.56785914370516e-07
782 1.56612737448825e-07
783 1.56451782203249e-07
784 1.56252137685442e-07
785 1.56076418790008e-07
786 1.55898435949098e-07
787 1.55699541132037e-07
788 1.55520961151012e-07
789 1.55345307412347e-07
790 1.55173079157578e-07
791 1.54974727074375e-07
792 1.54768641245084e-07
793 1.5459558374431e-07
794 1.5444234325912e-07
795 1.54301182689665e-07
796 1.54076760843225e-07
797 1.53900922057915e-07
798 1.53745949539541e-07
799 1.53580229117267e-07
800 1.53406590548855e-07
801 1.53240547525968e-07
802 1.53047686090702e-07
803 1.52824890882641e-07
804 1.52664273521452e-07
805 1.52475196550483e-07
806 1.52327802844354e-07
807 1.52147398885916e-07
808 1.51979647590395e-07
809 1.51795929646426e-07
810 1.51624505434711e-07
811 1.51454105321136e-07
812 1.51283279407721e-07
813 1.51115457406092e-07
814 1.50930618310952e-07
815 1.50760781984616e-07
816 1.50782627734714e-07
817 1.50500275672982e-07
818 1.50235237292407e-07
819 1.50018968902543e-07
820 1.49857105313345e-07
821 1.49713744534097e-07
822 1.49508386748209e-07
823 1.49243716315084e-07
824 1.49054966783524e-07
825 1.48848405203239e-07
826 1.48593713156231e-07
827 1.48452548614841e-07
828 1.48339782398921e-07
829 1.48084122237435e-07
830 1.47866147088394e-07
831 1.47622342979048e-07
832 1.4737123991182e-07
833 1.47169714985296e-07
834 1.47018789085962e-07
835 1.46823949506825e-07
836 1.46512769518381e-07
837 1.46378438842021e-07
838 1.46098743840639e-07
839 1.45954007734872e-07
840 1.45737554120728e-07
841 1.45556221561094e-07
842 1.45273102816645e-07
843 1.45146473812474e-07
844 1.44893944799662e-07
845 1.44685848667336e-07
846 1.44500907168776e-07
847 1.4436663649775e-07
848 1.44028160448784e-07
849 1.43881057880435e-07
850 1.43606745851343e-07
851 1.43507942325982e-07
852 1.43128769224177e-07
853 1.4301058635624e-07
854 1.42788225716117e-07
855 1.42625498355642e-07
856 1.42412673760361e-07
857 1.42159079061344e-07
858 1.4199432023787e-07
859 1.41772513316596e-07
860 1.41599934678993e-07
861 1.41388858359903e-07
862 1.41182926185479e-07
863 1.40962782744225e-07
864 1.40806364584023e-07
865 1.40581685961649e-07
866 1.40403925101396e-07
867 1.4016057497912e-07
868 1.40044896177471e-07
869 1.3980589160667e-07
870 1.3957648116758e-07
871 1.39445378636083e-07
872 1.39138843742614e-07
873 1.38897649193837e-07
874 1.38654308422304e-07
875 1.38455802293436e-07
876 1.38271871641393e-07
877 1.38072660327282e-07
878 1.3789356398064e-07
879 1.37655154013316e-07
880 1.37478279633285e-07
881 1.37314475196604e-07
882 1.37122865879746e-07
883 1.36933145142848e-07
884 1.36802316887952e-07
885 1.36559793908475e-07
886 1.3642202107178e-07
887 1.36200022627264e-07
888 1.36101916211828e-07
889 1.35843518506817e-07
890 1.35694482299442e-07
891 1.35512572025931e-07
892 1.35300152894757e-07
893 1.35296917736838e-07
894 1.35076997914041e-07
895 1.34893043981066e-07
896 1.34717172727505e-07
897 1.34527160362552e-07
898 1.34308240479442e-07
899 1.34143823828481e-07
900 1.33862042787314e-07
901 1.336907333922e-07
902 1.33526360556857e-07
903 1.33354756073345e-07
904 1.33178832584235e-07
905 1.32975719669304e-07
906 1.328086220731e-07
907 1.32602404118387e-07
908 1.3244473960583e-07
909 1.322507284236e-07
910 1.3208645066598e-07
911 1.31908025320371e-07
912 1.31727454142805e-07
913 1.31562949462705e-07
914 1.31399596458692e-07
915 1.3120474050865e-07
916 1.31009608221433e-07
917 1.30844098823246e-07
918 1.30663171997725e-07
919 1.30506657299634e-07
920 1.30347919839124e-07
921 1.30182991870953e-07
922 1.30027269566568e-07
923 1.29857494112429e-07
924 1.29701414859795e-07
925 1.29552905523411e-07
926 1.2940177445131e-07
927 1.29231482656422e-07
928 1.29098281369977e-07
929 1.28939781010473e-07
930 1.28751404755434e-07
931 1.28589202464724e-07
932 1.2841671961894e-07
933 1.28243070928846e-07
934 1.2803982670917e-07
935 1.27876534680382e-07
936 1.27715794622674e-07
937 1.27549439078223e-07
938 1.27395809307984e-07
939 1.27251330805933e-07
940 1.27115895857344e-07
941 1.26910032076211e-07
942 1.26811345971589e-07
943 1.26729663211478e-07
944 1.26444568536499e-07
945 1.26253609963101e-07
946 1.26082584579024e-07
947 1.25873118491882e-07
948 1.25708876002051e-07
949 1.25526543680365e-07
950 1.25347965983735e-07
951 1.25173203823437e-07
952 1.24986479754341e-07
953 1.24805979790921e-07
954 1.2464467223694e-07
955 1.24486626290832e-07
956 1.24325891309951e-07
957 1.24059086388684e-07
958 1.23900421499457e-07
959 1.23752723684589e-07
960 1.23606460093129e-07
961 1.23458894080386e-07
962 1.23334903548766e-07
963 1.23164554814537e-07
964 1.23017373574896e-07
965 1.22884383220168e-07
966 1.2272103016997e-07
967 1.22591781686054e-07
968 1.22412395523241e-07
969 1.22283938569723e-07
970 1.2213839731956e-07
971 1.21992413387062e-07
972 1.21762341468212e-07
973 1.21653056602611e-07
974 1.21516884128425e-07
975 1.21363378561057e-07
976 1.21249401036749e-07
977 1.21074073003768e-07
978 1.20967150103013e-07
979 1.20810225507029e-07
980 1.20688818554981e-07
981 1.20541718686695e-07
982 1.20421648816205e-07
983 1.2028323980573e-07
984 1.20130155050191e-07
985 1.20018588045667e-07
986 1.19883109864105e-07
987 1.19717597538482e-07
988 1.19591703700905e-07
989 1.19459623334706e-07
990 1.19327731148644e-07
991 1.19182412216645e-07
992 1.190378214595e-07
993 1.18928106189742e-07
994 1.18777606907372e-07
995 1.18669771886459e-07
996 1.18526345442405e-07
997 1.18412091222098e-07
998 1.182686517609e-07
999 1.18146764471305e-07
1000 1.18045295611324e-07
1001 1.17885191471601e-07
1002 1.1773723693409e-07
1003 1.17636993980597e-07
1004 1.17508308122183e-07
1005 1.17379135396334e-07
1006 1.17220094796977e-07
1007 1.17123960905019e-07
1008 1.16971413607558e-07
1009 1.16875604081912e-07
1010 1.16721507264828e-07
1011 1.16625413646432e-07
1012 1.16471765217341e-07
1013 1.16374956864007e-07
1014 1.16226475117998e-07
1015 1.1612951336204e-07
1016 1.15985186340595e-07
1017 1.15878762532162e-07
1018 1.15722544556718e-07
1019 1.1563705219686e-07
1020 1.15482216056506e-07
1021 1.15382735749847e-07
1022 1.15234289868482e-07
1023 1.15194369406879e-07
1024 1.1509215372385e-07
1025 1.14971480446258e-07
1026 1.14853542346083e-07
1027 1.14729510986677e-07
1028 1.14465596745106e-07
1029 1.14398147832873e-07
1030 1.14321383524185e-07
1031 1.14196341097994e-07
1032 1.14066242570487e-07
1033 1.13935783559782e-07
1034 1.13807661961118e-07
1035 1.13695412860437e-07
1036 1.13558107294409e-07
1037 1.13417669808769e-07
1038 1.13302818903094e-07
1039 1.13176249655567e-07
1040 1.13044207392221e-07
1041 1.12916077910086e-07
1042 1.12789789074696e-07
1043 1.12666461593136e-07
1044 1.12543457490233e-07
1045 1.12422214286312e-07
1046 1.12315864466694e-07
1047 1.12182835920294e-07
1048 1.12061487143933e-07
1049 1.11943023419769e-07
1050 1.11824035883501e-07
1051 1.11708382185327e-07
1052 1.11589450629168e-07
1053 1.11475569585906e-07
1054 1.11358585638044e-07
1055 1.11240735918727e-07
1056 1.11126858243438e-07
1057 1.11011553304508e-07
1058 1.10896632492086e-07
1059 1.10790323411436e-07
1060 1.10676093481032e-07
1061 1.10561356954975e-07
1062 1.10458826345194e-07
1063 1.10332153219872e-07
1064 1.10221367368268e-07
1065 1.10119643316864e-07
1066 1.10016076984465e-07
1067 1.09895937949744e-07
1068 1.09784279111125e-07
1069 1.09660139557377e-07
1070 1.09532256768574e-07
1071 1.09413658780255e-07
1072 1.09296013629745e-07
1073 1.09178800776988e-07
1074 1.09062563716122e-07
1075 1.08961397675245e-07
1076 1.0882023883596e-07
1077 1.08682747345767e-07
1078 1.08570803114105e-07
1079 1.08453086824056e-07
1080 1.08348394981306e-07
1081 1.08229575708663e-07
1082 1.08175652648868e-07
1083 1.08044552664666e-07
1084 1.07924311190999e-07
1085 1.07804279075197e-07
1086 1.07687016416946e-07
1087 1.07583657079857e-07
1088 1.07472328480185e-07
1089 1.07360298155612e-07
1090 1.07231535281471e-07
1091 1.07122651012759e-07
1092 1.07014493000435e-07
1093 1.0690706458405e-07
1094 1.06793603364963e-07
1095 1.06669464503995e-07
1096 1.06556396463731e-07
1097 1.06448062449971e-07
1098 1.0633262673565e-07
1099 1.06225804859861e-07
1100 1.06111840107559e-07
1101 1.06007312659528e-07
1102 1.05889961115935e-07
1103 1.05787279665037e-07
1104 1.05671067569091e-07
1105 1.05566635443921e-07
1106 1.0545252139238e-07
1107 1.05352690265903e-07
1108 1.05238320241341e-07
1109 1.05129935327852e-07
1110 1.05013805679732e-07
1111 1.04924952804453e-07
1112 1.04802457894237e-07
1113 1.0472310326648e-07
1114 1.04609571572212e-07
1115 1.0450942516016e-07
1116 1.04393734819297e-07
1117 1.04297197307091e-07
1118 1.04180416496291e-07
1119 1.0408402435047e-07
1120 1.03966197073646e-07
1121 1.0387017652036e-07
1122 1.03754794164246e-07
1123 1.0365835006354e-07
1124 1.03542971995552e-07
1125 1.03447081940544e-07
1126 1.03333663226124e-07
1127 1.03239306419312e-07
1128 1.03124784988751e-07
1129 1.03028840477748e-07
1130 1.02916153586818e-07
1131 1.02787292341588e-07
1132 1.02657459869704e-07
1133 1.02566233646684e-07
1134 1.02459099572627e-07
1135 1.0236741592351e-07
1136 1.02258474193206e-07
1137 1.02178690926991e-07
1138 1.02057533133859e-07
1139 1.0195955497494e-07
1140 1.01866740447321e-07
1141 1.0174640036098e-07
1142 1.01654885199309e-07
1143 1.01564086460115e-07
1144 1.01452568284088e-07
1145 1.01358281408892e-07
1146 1.01252124068196e-07
1147 1.01132042544805e-07
1148 1.01040741810721e-07
1149 1.0093680472778e-07
1150 1.00835201639171e-07
1151 1.00737907963833e-07
1152 1.00636074826355e-07
1153 1.00535662330259e-07
1154 1.00435070386595e-07
1155 1.00334794463919e-07
1156 1.00233066344657e-07
1157 1.00139313264691e-07
1158 1.00035746370963e-07
1159 9.99342205645348e-08
1160 9.98369444076275e-08
1161 9.97294167923712e-08
1162 9.96306041329831e-08
1163 9.95292283505478e-08
1164 9.94305740533719e-08
1165 9.93309828345446e-08
1166 9.92319580497281e-08
1167 9.91331406723361e-08
1168 9.90346987315149e-08
1169 9.89363423364864e-08
1170 9.88384333915349e-08
1171 9.87406676955516e-08
1172 9.86426246996075e-08
1173 9.85450470523119e-08
1174 9.84475341887503e-08
1175 9.83505173692833e-08
1176 9.82529710178426e-08
1177 9.81562854889262e-08
1178 9.80592281862869e-08
1179 9.79621718073531e-08
1180 9.78654439443005e-08
1181 9.77698567261598e-08
1182 9.76725255235067e-08
1183 9.75756527665794e-08
1184 9.74796394181965e-08
1185 9.73854815420339e-08
1186 9.72912396584036e-08
1187 9.71942784673274e-08
1188 9.70993046678359e-08
1189 9.7005777128345e-08
1190 9.69105955412886e-08
1191 9.68398183474051e-08
1192 9.67422502249349e-08
1193 9.6656792269556e-08
1194 9.65493723619204e-08
1195 9.64579490911888e-08
1196 9.63569642422613e-08
1197 9.62606689931533e-08
1198 9.61624434623332e-08
1199 9.60775235725464e-08
1200 9.5981298237291e-08
1201 9.58870457914429e-08
1202 9.57886685490905e-08
1203 9.56941159380165e-08
1204 9.55973716543213e-08
1205 9.550779267542e-08
1206 9.54129343995192e-08
1207 9.53207871106088e-08
1208 9.52259731121785e-08
1209 9.51386579863822e-08
1210 9.50452932002577e-08
1211 9.49372938769955e-08
1212 9.48370358955231e-08
1213 9.47485660738323e-08
1214 9.46488406015078e-08
1215 9.45476629787834e-08
1216 9.44520911225766e-08
1217 9.43511249644757e-08
1218 9.42532653489536e-08
1219 9.41596363510655e-08
1220 9.40521965198116e-08
1221 9.3958234778313e-08
1222 9.38527993739058e-08
1223 9.37583681590581e-08
1224 9.3655063739817e-08
1225 9.35592185555834e-08
1226 9.34639041965113e-08
1227 9.33635799533761e-08
1228 9.32726998605915e-08
1229 9.31820874114919e-08
1230 9.3086685076571e-08
1231 9.2996467174089e-08
1232 9.28943459186371e-08
1233 9.28112276987747e-08
1234 9.27086237609842e-08
1235 9.26266937000264e-08
1236 9.25256117518813e-08
1237 9.24348963096122e-08
1238 9.23422541383445e-08
1239 9.22336319888473e-08
1240 9.21328738066052e-08
1241 9.20548228862117e-08
1242 9.19620167856294e-08
1243 9.1858883305207e-08
1244 9.17831347067022e-08
1245 9.16764033469519e-08
1246 9.16021301513581e-08
1247 9.15102322807115e-08
1248 9.14096642823381e-08
1249 9.13258232415615e-08
1250 9.12370573509236e-08
1251 9.11314231082372e-08
1252 9.10548857149251e-08
1253 9.09575262930673e-08
1254 9.0878455086596e-08
1255 9.07892068404692e-08
1256 9.06909464717387e-08
1257 9.06085933252143e-08
1258 9.0521073303762e-08
1259 9.04282962750358e-08
1260 9.03382387598128e-08
1261 9.02488962388759e-08
1262 9.01657012271073e-08
1263 9.00837959534329e-08
1264 8.99963811491489e-08
1265 8.9906916667104e-08
1266 8.98200719099407e-08
1267 8.97321847581622e-08
1268 8.96452091616595e-08
1269 8.95563996046178e-08
1270 8.94744537696113e-08
1271 8.93858809583037e-08
1272 8.93017317054046e-08
1273 8.92115444202091e-08
1274 8.91289444773236e-08
1275 8.90431513695944e-08
1276 8.89543760465017e-08
1277 8.88647354990724e-08
1278 8.87773948683446e-08
1279 8.86847180474604e-08
1280 8.8604710200002e-08
1281 8.85189234693939e-08
1282 8.84320530047944e-08
1283 8.83455266311728e-08
1284 8.82566362889747e-08
1285 8.81749002772381e-08
1286 8.8091375637589e-08
1287 8.8006731434831e-08
1288 8.79208263704356e-08
1289 8.78380143163327e-08
1290 8.7753197313134e-08
1291 8.76683468540307e-08
1292 8.7583503496802e-08
1293 8.74997554873858e-08
1294 8.74166315618652e-08
1295 8.7332899298076e-08
1296 8.72496221155927e-08
1297 8.71674850380089e-08
1298 8.70835239226153e-08
1299 8.69979467985615e-08
1300 8.69180498774824e-08
1301 8.68290546023331e-08
1302 8.67538605575646e-08
1303 8.66866929456478e-08
1304 8.66013013194333e-08
1305 8.65141546739778e-08
1306 8.6432395224989e-08
1307 8.63492025722223e-08
1308 8.62639010534849e-08
1309 8.61873499360399e-08
1310 8.60964398192721e-08
1311 8.60305623895386e-08
1312 8.5945265599463e-08
1313 8.58654531619152e-08
1314 8.57828558054052e-08
1315 8.57025365945674e-08
1316 8.56227119179209e-08
1317 8.55251783349331e-08
1318 8.5447654544879e-08
1319 8.53730991074997e-08
1320 8.53113611576362e-08
1321 8.52530063752965e-08
1322 8.5145811016929e-08
1323 8.50639250415952e-08
1324 8.49800504987286e-08
1325 8.49213991926945e-08
1326 8.48373153417015e-08
1327 8.47663028693546e-08
1328 8.4683932776386e-08
1329 8.46030600811787e-08
1330 8.45243451159661e-08
1331 8.44448748686943e-08
1332 8.43658713378659e-08
1333 8.42873192290483e-08
1334 8.42077087881421e-08
1335 8.413040495725e-08
1336 8.40533396413434e-08
1337 8.39737880227176e-08
1338 8.39027628565248e-08
1339 8.38281100463689e-08
1340 8.37473770438635e-08
1341 8.36699169042276e-08
1342 8.35880798319977e-08
1343 8.35109091923414e-08
1344 8.34303905108413e-08
1345 8.33509666051668e-08
1346 8.32730190971631e-08
1347 8.31975916035788e-08
1348 8.31306992026271e-08
1349 8.30542468079898e-08
1350 8.29774157509178e-08
1351 8.2896661286469e-08
1352 8.28168008588648e-08
1353 8.27384261050668e-08
1354 8.26573808154762e-08
1355 8.25796255305988e-08
1356 8.25036329494821e-08
1357 8.24257792118033e-08
1358 8.23485480161423e-08
1359 8.22789238732469e-08
1360 8.22006595520008e-08
1361 8.21227055496365e-08
1362 8.2044704448947e-08
1363 8.19672813463512e-08
1364 8.18913912894459e-08
1365 8.18145910308488e-08
1366 8.17310067056098e-08
1367 8.16556159399795e-08
1368 8.15891315895101e-08
1369 8.15134725122846e-08
1370 8.14405360145543e-08
1371 8.13719105217103e-08
1372 8.12908016314395e-08
1373 8.12063895203607e-08
1374 8.11270667853137e-08
1375 8.10392181485042e-08
1376 8.09575813001118e-08
1377 8.08839921795368e-08
1378 8.08079852667731e-08
1379 8.07220590779423e-08
1380 8.06590918109862e-08
1381 8.05818785387658e-08
1382 8.05058287483007e-08
1383 8.04375114924483e-08
1384 8.03611219843958e-08
1385 8.02734742073596e-08
1386 8.02103674963917e-08
1387 8.01333753592814e-08
1388 8.00467322505938e-08
1389 7.99712219219373e-08
1390 7.99084653912985e-08
1391 7.98324605639777e-08
1392 7.97513225130331e-08
1393 7.96915628207273e-08
1394 7.96128401034935e-08
1395 7.95309664347599e-08
1396 7.94698840849151e-08
1397 7.9392765581332e-08
1398 7.9295083487807e-08
1399 7.92480824038933e-08
1400 7.91767527488219e-08
1401 7.90798937977399e-08
1402 7.90350093922143e-08
1403 7.89427710436996e-08
1404 7.88719372764035e-08
1405 7.88001505966918e-08
1406 7.87278963976235e-08
1407 7.86439569182562e-08
1408 7.85879019211677e-08
1409 7.85154309888014e-08
1410 7.84318538968876e-08
1411 7.83922933358383e-08
1412 7.83064902059039e-08
1413 7.82436948192355e-08
1414 7.81700940777341e-08
1415 7.81046548645747e-08
1416 7.80235332982215e-08
1417 7.79860344124472e-08
1418 7.78859922760944e-08
1419 7.78468724114134e-08
1420 7.77458072285242e-08
1421 7.77082536167484e-08
1422 7.76031458613602e-08
1423 7.75538769453021e-08
1424 7.74638346889844e-08
1425 7.7412013162359e-08
1426 7.73240489770899e-08
1427 7.72703984992518e-08
1428 7.71842282603075e-08
1429 7.71299326345343e-08
1430 7.70427543983487e-08
1431 7.69885805844694e-08
1432 7.68999306934859e-08
1433 7.68471794536651e-08
1434 7.67474068190665e-08
1435 7.669313940184e-08
1436 7.6607799300632e-08
1437 7.65571879561833e-08
1438 7.64666975019423e-08
1439 7.64233642662759e-08
1440 7.63317421643706e-08
1441 7.62877545348317e-08
1442 7.6203435483535e-08
1443 7.61529052084597e-08
1444 7.60647247997071e-08
1445 7.60194817654991e-08
1446 7.59333198274703e-08
1447 7.58907218383342e-08
1448 7.58090455246219e-08
1449 7.57582843178284e-08
1450 7.57179583370515e-08
1451 7.56195299302931e-08
1452 7.55561748508882e-08
1453 7.54916978635833e-08
1454 7.5432396256403e-08
1455 7.53674571498664e-08
1456 7.52965677950357e-08
1457 7.5238625441898e-08
1458 7.5167802796372e-08
1459 7.51119294637448e-08
1460 7.50374735734027e-08
1461 7.49760933977939e-08
1462 7.48975998057233e-08
1463 7.48535242411918e-08
1464 7.47783008758773e-08
1465 7.47050625609091e-08
1466 7.46483551417043e-08
1467 7.45740001022455e-08
1468 7.45270812885224e-08
1469 7.4443754344955e-08
1470 7.4380113401773e-08
1471 7.43380596901488e-08
1472 7.42537142617294e-08
1473 7.42098589494589e-08
1474 7.41303028082285e-08
1475 7.40660056592901e-08
1476 7.40370165512871e-08
1477 7.39397521467566e-08
1478 7.38789098857495e-08
1479 7.38492542762259e-08
1480 7.37513224109421e-08
1481 7.37224941023129e-08
1482 7.36224275552644e-08
1483 7.35667011948493e-08
1484 7.35384280314832e-08
1485 7.34439351628424e-08
1486 7.34135366222688e-08
1487 7.33215193999826e-08
1488 7.3262118958084e-08
1489 7.32310826201399e-08
1490 7.31369283712979e-08
1491 7.30799931130122e-08
1492 7.30507533877045e-08
1493 7.29692056040676e-08
1494 7.29461080162253e-08
1495 7.28552913358271e-08
1496 7.28172455186637e-08
1497 7.2759770612052e-08
1498 7.26808831839065e-08
1499 7.26333379219568e-08
1500 7.25761171178618e-08
1501 7.25002446042566e-08
1502 7.24530508353638e-08
1503 7.23712498071904e-08
1504 7.23255573245041e-08
1505 7.22771065948535e-08
1506 7.22012546212625e-08
1507 7.21588228156378e-08
1508 7.20717930615677e-08
1509 7.20151864914698e-08
1510 7.19812252931717e-08
1511 7.18593284645408e-08
1512 7.18661725915837e-08
1513 7.18080895314443e-08
1514 7.17542879211663e-08
1515 7.16964362812433e-08
1516 7.16366252806466e-08
1517 7.15322650410855e-08
1518 7.15308601897391e-08
1519 7.14554064966677e-08
1520 7.13549446889061e-08
1521 7.13500851201587e-08
1522 7.1280213971292e-08
1523 7.12213592048983e-08
1524 7.1125808473127e-08
1525 7.11177791856699e-08
1526 7.10524085931752e-08
1527 7.09491985855237e-08
1528 7.09230811022366e-08
1529 7.08539147549203e-08
1530 7.07737182583656e-08
1531 7.07214989734695e-08
1532 7.06564270451793e-08
1533 7.05835089735984e-08
1534 7.05460777865596e-08
1535 7.04877491930489e-08
1536 7.04216205349439e-08
1537 7.03814124722868e-08
1538 7.03216431912068e-08
1539 7.02704037696122e-08
1540 7.01973612020623e-08
1541 7.0164162307762e-08
1542 7.01039659212199e-08
1543 7.0039195259497e-08
1544 6.99994840633877e-08
1545 6.99282411940061e-08
1546 6.98895919128972e-08
1547 6.98095471864946e-08
1548 6.97712376460657e-08
1549 6.97109298943843e-08
1550 6.9632473101322e-08
1551 6.95940093482506e-08
1552 6.95389268248192e-08
1553 6.94699652452613e-08
1554 6.94294041974075e-08
1555 6.9355845011998e-08
1556 6.93180881405908e-08
1557 6.92588990727216e-08
1558 6.91887736632424e-08
1559 6.91512637267522e-08
1560 6.9092847537533e-08
1561 6.90220523242857e-08
1562 6.89855159823338e-08
1563 6.89106090323577e-08
1564 6.88816554959004e-08
1565 6.88365903069865e-08
1566 6.87812293040935e-08
1567 6.87185002323787e-08
1568 6.8642084881887e-08
1569 6.86118347612563e-08
1570 6.85325593572372e-08
1571 6.85040618790822e-08
1572 6.84324654880442e-08
1573 6.839076007914e-08
1574 6.83189021835062e-08
1575 6.82729641408031e-08
1576 6.81955693746517e-08
1577 6.81605108194816e-08
1578 6.81111089928521e-08
1579 6.80341699759168e-08
1580 6.80005206437073e-08
1581 6.79343854468328e-08
1582 6.78743013686756e-08
1583 6.78180592128541e-08
1584 6.77644835676006e-08
1585 6.77205531331992e-08
1586 6.76518835902584e-08
1587 6.75858284839848e-08
1588 6.7559896509195e-08
1589 6.74892750094358e-08
1590 6.74610934225939e-08
1591 6.74030582157315e-08
1592 6.73369531245527e-08
1593 6.72861732180508e-08
1594 6.72273089517006e-08
1595 6.7194645049895e-08
1596 6.7137755896951e-08
1597 6.70708017658228e-08
1598 6.70592294174099e-08
1599 6.69806256290428e-08
1600 6.69565935051963e-08
1601 6.6876471620958e-08
1602 6.68527393017371e-08
1603 6.67749209544155e-08
1604 6.67497319373211e-08
1605 6.66836660396797e-08
1606 6.66453354263297e-08
1607 6.65926854352961e-08
1608 6.65282488423458e-08
1609 6.6488449387947e-08
1610 6.64343398302236e-08
1611 6.63779134839615e-08
1612 6.63125629074557e-08
1613 6.62765239027863e-08
1614 6.62034557681324e-08
1615 6.61748533499207e-08
1616 6.61297554014339e-08
1617 6.60409584458677e-08
1618 6.6028880171487e-08
1619 6.59719671087799e-08
1620 6.59495228756413e-08
1621 6.58748443882473e-08
1622 6.58512198743466e-08
1623 6.57803125960754e-08
1624 6.57593430002379e-08
1625 6.57009276565645e-08
1626 6.56188756789078e-08
1627 6.56008984094569e-08
1628 6.5510794748036e-08
1629 6.5506174689034e-08
1630 6.54147165271013e-08
1631 6.54004263029861e-08
1632 6.5330469803726e-08
1633 6.53194631468068e-08
1634 6.52330189456762e-08
1635 6.52263919054263e-08
1636 6.51325602660791e-08
1637 6.51270379510294e-08
1638 6.50832606332585e-08
1639 6.50223449909504e-08
1640 6.49905436045373e-08
1641 6.49078730869945e-08
1642 6.48822075746125e-08
1643 6.48035599741803e-08
1644 6.47770159538652e-08
1645 6.4707539873865e-08
1646 6.46819224456863e-08
1647 6.46097836529691e-08
1648 6.45885289962678e-08
1649 6.45144189377334e-08
1650 6.44925645705285e-08
1651 6.44240081335568e-08
1652 6.43979384005178e-08
1653 6.43428221973608e-08
1654 6.428219044885e-08
1655 6.42558258956427e-08
1656 6.41921231387954e-08
1657 6.41689319547112e-08
1658 6.41028819785561e-08
1659 6.40726781870882e-08
1660 6.40072181106177e-08
1661 6.39795522463515e-08
1662 6.39297982321096e-08
1663 6.3871073178845e-08
1664 6.38408367290566e-08
1665 6.37792557007799e-08
1666 6.37521571729138e-08
1667 6.36896845236379e-08
1668 6.36601125965086e-08
1669 6.36008435410673e-08
1670 6.35746379664681e-08
1671 6.35181804433671e-08
1672 6.34676094435349e-08
1673 6.34354211879184e-08
1674 6.33740493203305e-08
1675 6.33314038243071e-08
1676 6.32749130726751e-08
1677 6.32347840738845e-08
1678 6.31732688756159e-08
1679 6.31338394310177e-08
1680 6.30652151922817e-08
1681 6.30463459234676e-08
1682 6.29724075018601e-08
1683 6.2956256465796e-08
1684 6.28833726636913e-08
1685 6.28631577619387e-08
1686 6.2810514361189e-08
1687 6.27526249825649e-08
1688 6.27297131359228e-08
1689 6.26645676735649e-08
1690 6.26481085017616e-08
1691 6.25765496540254e-08
1692 6.25491643475584e-08
1693 6.24812895697602e-08
1694 6.2463616799846e-08
1695 6.23948352806991e-08
1696 6.23737416489689e-08
1697 6.23239637480566e-08
1698 6.22703101509359e-08
1699 6.22446263580656e-08
1700 6.21824421340733e-08
1701 6.21636741069409e-08
1702 6.20918038123364e-08
1703 6.20736190857229e-08
1704 6.20149772938561e-08
1705 6.19964937431661e-08
1706 6.19347309811502e-08
1707 6.1910406349952e-08
1708 6.18521268460626e-08
1709 6.18265581540101e-08
1710 6.17684524790718e-08
1711 6.17448781028429e-08
1712 6.16845532288579e-08
1713 6.16598977511984e-08
1714 6.16143801561719e-08
1715 6.15619190327266e-08
1716 6.15333821283315e-08
1717 6.14823349280869e-08
1718 6.14551080015957e-08
1719 6.14102441911513e-08
1720 6.13515261029818e-08
1721 6.13220992118357e-08
1722 6.12588356148081e-08
1723 6.12405779989444e-08
1724 6.11997082504701e-08
1725 6.11438640394368e-08
1726 6.11246135520105e-08
1727 6.1070817375608e-08
1728 6.10205517528328e-08
1729 6.09990552220552e-08
1730 6.09413689733884e-08
1731 6.08886698465483e-08
1732 6.08616994828282e-08
1733 6.08214279971264e-08
1734 6.07714548870319e-08
1735 6.07395625191742e-08
1736 6.06988344191706e-08
1737 6.064735849165e-08
1738 6.06205199655818e-08
1739 6.05749239124265e-08
1740 6.05129370594426e-08
1741 6.05012419647721e-08
1742 6.04553985272815e-08
1743 6.03960760319211e-08
1744 6.03828635998838e-08
1745 6.03404926344808e-08
1746 6.02797830424606e-08
1747 6.02628985220122e-08
1748 6.02163528338195e-08
1749 6.01650723552893e-08
1750 6.01461151159555e-08
1751 6.0099468852215e-08
1752 6.00380151585256e-08
1753 6.00097526533006e-08
1754 5.99578201949669e-08
1755 5.99032240202746e-08
1756 5.98908391200581e-08
1757 5.98447522186518e-08
1758 5.97969285838218e-08
1759 5.9783278864245e-08
1760 5.97381680282894e-08
1761 5.96831014316734e-08
1762 5.96634693312836e-08
1763 5.9620811219574e-08
1764 5.95710671564831e-08
1765 5.95497044262316e-08
1766 5.94933353976756e-08
1767 5.94397501458843e-08
1768 5.94266319691172e-08
1769 5.93856979289598e-08
1770 5.93300072733172e-08
1771 5.93130644528372e-08
1772 5.92704806923194e-08
1773 5.92268136809793e-08
1774 5.91797808926486e-08
1775 5.91507497311738e-08
1776 5.91200140238612e-08
1777 5.90610537880565e-08
1778 5.90448508877017e-08
1779 5.90022764068721e-08
1780 5.89430245447886e-08
1781 5.89263584167554e-08
1782 5.88846336242455e-08
1783 5.88285095606267e-08
1784 5.88103355436687e-08
1785 5.87698267064951e-08
1786 5.87322994558548e-08
1787 5.86814049867712e-08
1788 5.86593097935406e-08
1789 5.86253984682372e-08
1790 5.85660904164342e-08
1791 5.85476309495903e-08
1792 5.85049212116218e-08
1793 5.84561298868635e-08
1794 5.84374685121958e-08
1795 5.83932191009495e-08
1796 5.83553954029981e-08
1797 5.83059133898445e-08
1798 5.82895867626831e-08
1799 5.82480555006981e-08
1800 5.8200157324606e-08
1801 5.81786484090685e-08
1802 5.81429857593463e-08
1803 5.80878021807507e-08
1804 5.80690354983204e-08
1805 5.80282149797995e-08
1806 5.79956561210793e-08
1807 5.79425478619555e-08
1808 5.79221452419176e-08
1809 5.78838757476774e-08
1810 5.78314735779628e-08
1811 5.78163176037094e-08
1812 5.77766614888731e-08
1813 5.77403952473077e-08
1814 5.76883562946051e-08
1815 5.76788046178223e-08
1816 5.76407622698838e-08
1817 5.75972874177211e-08
1818 5.75486887459675e-08
1819 5.75290882096624e-08
1820 5.74849039995939e-08
1821 5.74410002496251e-08
1822 5.74063457214891e-08
1823 5.73605157381252e-08
1824 5.73157828593907e-08
1825 5.7272223271454e-08
1826 5.72637396345499e-08
1827 5.72324278689251e-08
1828 5.71935515232269e-08
1829 5.71610200132966e-08
1830 5.71233054333931e-08
1831 5.7078984424308e-08
1832 5.70541767270782e-08
1833 5.70163162052495e-08
1834 5.6980763867287e-08
1835 5.6939591541294e-08
1836 5.69168955646404e-08
1837 5.68663307110029e-08
1838 5.68295884537662e-08
1839 5.68088100383335e-08
1840 5.67678091236701e-08
1841 5.67343938175924e-08
1842 5.67020331203594e-08
1843 5.66667278611987e-08
1844 5.66293540043006e-08
1845 5.6587106973538e-08
1846 5.65633186884185e-08
1847 5.65253377242669e-08
1848 5.6488802282928e-08
1849 5.64697279958892e-08
1850 5.64184602804829e-08
1851 5.6386358695093e-08
1852 5.63529641777194e-08
1853 5.63152014105839e-08
1854 5.63030739169079e-08
1855 5.62536226720312e-08
1856 5.621887777707e-08
1857 5.61840797956847e-08
1858 5.61519734709748e-08
1859 5.6119060975135e-08
1860 5.60813896104406e-08
1861 5.60604641695051e-08
1862 5.60212938207627e-08
1863 5.59822433103818e-08
1864 5.59557859460824e-08
1865 5.59342552808317e-08
1866 5.58837556017267e-08
1867 5.58555222394119e-08
1868 5.58171534166263e-08
1869 5.57975660431254e-08
1870 5.57658234239256e-08
1871 5.57404777250525e-08
1872 5.56964996949461e-08
1873 5.56634625237251e-08
1874 5.56313199027159e-08
1875 5.55997774540629e-08
1876 5.55613583443915e-08
1877 5.55269757676058e-08
1878 5.5502511211003e-08
1879 5.54778267520817e-08
1880 5.5444562415019e-08
1881 5.54123262350714e-08
1882 5.53820424755713e-08
1883 5.53495265638304e-08
1884 5.53182410154562e-08
1885 5.5270342071978e-08
1886 5.52461083671574e-08
1887 5.52285828891996e-08
1888 5.51930989818317e-08
1889 5.51618536857035e-08
1890 5.51298880360207e-08
1891 5.50989660688117e-08
1892 5.50679799324882e-08
1893 5.50219489507953e-08
1894 5.49859079370663e-08
1895 5.49978789443628e-08
1896 5.49897674844146e-08
1897 5.49477678379873e-08
1898 5.489871420572e-08
1899 5.48642930553456e-08
1900 5.48544498872872e-08
1901 5.48280247141264e-08
1902 5.4770532685211e-08
1903 5.4748959600559e-08
1904 5.47057154562935e-08
1905 5.4679610769881e-08
1906 5.4649364141568e-08
1907 5.46201876634456e-08
1908 5.45880473250548e-08
1909 5.45829187341695e-08
1910 5.45216613279109e-08
1911 5.44992172866188e-08
1912 5.44700705287227e-08
1913 5.44351320002079e-08
1914 5.44117753751294e-08
1915 5.43976634208576e-08
1916 5.43506271544203e-08
1917 5.43152112477685e-08
1918 5.42734870077055e-08
1919 5.42208746470152e-08
1920 5.42188376080333e-08
1921 5.42318459899604e-08
1922 5.41535841005469e-08
1923 5.40810903277844e-08
1924 5.40626227607532e-08
1925 5.40615722570692e-08
1926 5.40531078154771e-08
1927 5.40417447325581e-08
1928 5.40066272378681e-08
1929 5.39541558524093e-08
1930 5.39220726949452e-08
1931 5.38704136285872e-08
1932 5.38764866675479e-08
1933 5.3832834163714e-08
1934 5.38132092149368e-08
1935 5.37755291372122e-08
1936 5.37120134787017e-08
1937 5.37266989422136e-08
1938 5.36942823483599e-08
1939 5.36637576598054e-08
1940 5.36314578081232e-08
1941 5.35836024884162e-08
1942 5.35198249060898e-08
1943 5.34807587548869e-08
1944 5.35186063448379e-08
1945 5.34852590980961e-08
1946 5.34530285616341e-08
1947 5.34172343797934e-08
1948 5.33864700216213e-08
1949 5.33384438252682e-08
1950 5.3282433940538e-08
1951 5.32578811167639e-08
1952 5.32278484239868e-08
1953 5.32070314616107e-08
1954 5.32098277403747e-08
1955 5.31718705811812e-08
1956 5.3149979528655e-08
1957 5.31092216693452e-08
1958 5.30581348936465e-08
1959 5.30666372675626e-08
1960 5.30339547051284e-08
1961 5.30024818807817e-08
1962 5.29706637326655e-08
1963 5.29289977535541e-08
1964 5.28774677004407e-08
1965 5.28940590527327e-08
1966 5.28638908967594e-08
1967 5.28358995826039e-08
1968 5.28036098508267e-08
1969 5.27743610430065e-08
1970 5.272617962504e-08
1971 5.26712675981145e-08
1972 5.26441776536046e-08
1973 5.26639329834211e-08
1974 5.26350127412201e-08
1975 5.26090606758345e-08
1976 5.2560195625162e-08
1977 5.25175980872206e-08
1978 5.25127587494012e-08
1979 5.24721666028682e-08
1980 5.25028947304662e-08
1981 5.24535227519607e-08
1982 5.24334885607658e-08
1983 5.23928427504927e-08
1984 5.23647968595498e-08
1985 5.23303921706741e-08
1986 5.22664249142935e-08
1987 5.22732280447968e-08
1988 5.22278388448427e-08
1989 5.22206757729293e-08
1990 5.21576788745648e-08
1991 5.21696426716289e-08
1992 5.21383016103272e-08
1993 5.21176466321549e-08
1994 5.20795234777438e-08
1995 5.20577898903696e-08
1996 5.20088351390058e-08
1997 5.19967646770425e-08
1998 5.19834512235917e-08
1999 5.19351928325307e-08
2000 5.18789852410606e-08
2001 5.18892107059798e-08
2002 5.18384501262403e-08
2003 5.18320842655129e-08
2004 5.17724597930425e-08
2005 5.17933313215479e-08
2006 5.17651106353156e-08
2007 5.17365716401486e-08
2008 5.16944155677379e-08
2009 5.16442815303719e-08
2010 5.16417980787054e-08
2011 5.15751542451426e-08
2012 5.16057571360307e-08
2013 5.15795475681813e-08
2014 5.15523841428944e-08
2015 5.14852089530393e-08
2016 5.14251952807854e-08
2017 5.14224534473584e-08
2018 5.13930180208177e-08
2019 5.13651756417488e-08
2020 5.13398442567592e-08
2021 5.1278221276263e-08
2022 5.12808904371553e-08
2023 5.12275395170292e-08
2024 5.12120996791765e-08
2025 5.11728608216799e-08
2026 5.11418466579983e-08
2027 5.1111997334985e-08
2028 5.10851505044485e-08
2029 5.10588654680788e-08
2030 5.10363068499942e-08
2031 5.10086910185947e-08
2032 5.09770729344439e-08
2033 5.09473898517854e-08
2034 5.09033281179683e-08
2035 5.08676520567519e-08
2036 5.08378530117426e-08
2037 5.07910789444566e-08
2038 5.07609317761393e-08
2039 5.07433153380532e-08
2040 5.07163217573492e-08
2041 5.0683928199291e-08
2042 5.06598294993665e-08
2043 5.06497230698955e-08
2044 5.06451459418855e-08
2045 5.05945661295471e-08
2046 5.0563461080344e-08
2047 5.05418403413671e-08
2048 5.05143432345534e-08
2049 5.04725191596833e-08
2050 5.04620838448488e-08
2051 5.04257198841174e-08
2052 5.03963417504139e-08
2053 5.03694018423317e-08
2054 5.03383870249507e-08
2055 5.03116513481672e-08
2056 5.02853407446935e-08
2057 5.02635672177121e-08
2058 5.02253421217347e-08
2059 5.01998919251179e-08
2060 5.01762324986998e-08
2061 5.01381022903757e-08
2062 5.01165475785825e-08
2063 5.00791420972035e-08
2064 5.00571208412737e-08
2065 5.00191744006173e-08
2066 4.99969106808607e-08
2067 4.99652818515273e-08
2068 4.99406647023193e-08
2069 4.99400747422385e-08
2070 4.98836416173987e-08
2071 4.98563354618398e-08
2072 4.98195677050006e-08
2073 4.9797368346205e-08
2074 4.97628820195217e-08
2075 4.97410512299012e-08
2076 4.97102558565388e-08
2077 4.96750526295386e-08
2078 4.96484829373145e-08
2079 4.96207821676364e-08
2080 4.95880412643146e-08
2081 4.95595028215945e-08
2082 4.95292679438819e-08
2083 4.95411185248429e-08
2084 4.94985872645515e-08
2085 4.94496979275283e-08
2086 4.94542133306908e-08
2087 4.94274170268483e-08
2088 4.93998827000297e-08
2089 4.9341949491577e-08
2090 4.93363587459328e-08
2091 4.92780220415767e-08
2092 4.92824592868146e-08
2093 4.92250463803856e-08
2094 4.92274303507401e-08
2095 4.9199288582713e-08
2096 4.92094329818116e-08
2097 4.91981392123364e-08
2098 4.9152802006347e-08
2099 4.9114675860551e-08
2100 4.9033944433674e-08
2101 4.90316926367029e-08
2102 4.89976539537196e-08
2103 4.89319942857946e-08
2104 4.89405658434805e-08
2105 4.8911156577347e-08
2106 4.88543132597385e-08
2107 4.88466899071227e-08
2108 4.8827074209612e-08
2109 4.87882665449746e-08
2110 4.87584398154439e-08
2111 4.87325002040961e-08
2112 4.86804870529767e-08
2113 4.86855391237384e-08
2114 4.86544531472788e-08
2115 4.8617656753791e-08
2116 4.85787829731521e-08
2117 4.85733001180932e-08
2118 4.85324971588597e-08
2119 4.84764087342882e-08
2120 4.84760197316803e-08
2121 4.84426904971969e-08
2122 4.83886872384431e-08
2123 4.83911269064663e-08
2124 4.83602271259542e-08
2125 4.83256255741082e-08
2126 4.82837732302954e-08
2127 4.8239692453933e-08
2128 4.82207547154445e-08
2129 4.81778962928558e-08
2130 4.81067450763817e-08
2131 4.81156722784704e-08
2132 4.80754237326408e-08
2133 4.80259648902859e-08
2134 4.80284511787943e-08
2135 4.79973507125919e-08
2136 4.79372854584881e-08
2137 4.79401473363339e-08
2138 4.79183141877115e-08
2139 4.78493483910825e-08
2140 4.78522535107828e-08
2141 4.78160090278124e-08
2142 4.77685009858675e-08
2143 4.77758861539002e-08
2144 4.77813948318584e-08
2145 4.76947985745824e-08
2146 4.77116186328175e-08
2147 4.76938963718254e-08
2148 4.76333050194455e-08
2149 4.76372162339089e-08
2150 4.76071146007229e-08
2151 4.75721667516638e-08
2152 4.75435503730637e-08
2153 4.75133242439085e-08
2154 4.74922096334041e-08
2155 4.74584957572688e-08
2156 4.74221293362831e-08
2157 4.73951286199537e-08
2158 4.73822662385714e-08
2159 4.73562189053922e-08
2160 4.73235726463628e-08
2161 4.72915564326826e-08
2162 4.72337256347544e-08
2163 4.72509053803805e-08
2164 4.7218481450173e-08
2165 4.71451687893421e-08
2166 4.71442502174568e-08
2167 4.71200138481009e-08
2168 4.70387778790382e-08
2169 4.70340720397644e-08
2170 4.70194344508457e-08
2171 4.69458315883742e-08
2172 4.69377323550901e-08
2173 4.69049265703347e-08
2174 4.68407022253814e-08
2175 4.68512344546923e-08
2176 4.68242696403109e-08
2177 4.67862859920842e-08
2178 4.675672202481e-08
2179 4.67329635984726e-08
2180 4.67290277104127e-08
2181 4.67369049772515e-08
2182 4.66904141855906e-08
2183 4.66776925449608e-08
2184 4.66068987403645e-08
2185 4.66382937318599e-08
2186 4.66308690576511e-08
2187 4.6599705859407e-08
2188 4.65626946262177e-08
2189 4.65253339285709e-08
2190 4.65358002479377e-08
2191 4.64807787796673e-08
2192 4.64176081464984e-08
2193 4.63983410501356e-08
2194 4.64273724798403e-08
2195 4.640886666607e-08
2196 4.63435281936597e-08
2197 4.63560296068977e-08
2198 4.63218608501847e-08
2199 4.62366903271061e-08
2200 4.61693773647909e-08
2201 4.62482984460166e-08
2202 4.62294063954261e-08
2203 4.61995459630771e-08
2204 4.61792493666735e-08
2205 4.61512429854594e-08
2206 4.61249547125675e-08
2207 4.60375801054624e-08
2208 4.60698271300686e-08
2209 4.59933516996358e-08
2210 4.60264295139012e-08
2211 4.60035768874434e-08
2212 4.59734902644726e-08
2213 4.59353135298102e-08
2214 4.58914672414323e-08
2215 4.58740757185439e-08
2216 4.58688842339683e-08
2217 4.58386290240753e-08
2218 4.57829815037059e-08
2219 4.56978217844295e-08
2220 4.57745255282305e-08
2221 4.57488838971898e-08
2222 4.5727252913963e-08
2223 4.57010261865065e-08
2224 4.56773095969254e-08
2225 4.56523360714556e-08
2226 4.56079954567201e-08
2227 4.56050133692543e-08
2228 4.55653490529784e-08
2229 4.55445668929855e-08
2230 4.55087976121149e-08
2231 4.54029153669211e-08
2232 4.54461209677959e-08
2233 4.54225250106077e-08
2234 4.54019309152898e-08
2235 4.53404661069357e-08
2236 4.53332942402795e-08
2237 4.53523821555279e-08
2238 4.53081998852412e-08
2239 4.5266097503216e-08
2240 4.52505924535274e-08
2241 4.52123030143525e-08
2242 4.51701882777655e-08
2243 4.50957478328462e-08
2244 4.51356051591745e-08
2245 4.51142677349026e-08
2246 4.50897702144459e-08
2247 4.50595261867193e-08
2248 4.50494245658462e-08
2249 4.50183158289263e-08
2250 4.50067496764461e-08
2251 4.49808961811016e-08
2252 4.49704534073447e-08
2253 4.4942033234463e-08
2254 4.49186483884034e-08
2255 4.48899370297795e-08
2256 4.48860294177678e-08
2257 4.48582368086647e-08
2258 4.48423634527018e-08
2259 4.48272912656478e-08
2260 4.47910778156313e-08
2261 4.47529692735316e-08
2262 4.47352508654575e-08
2263 4.47558882399335e-08
2264 4.47259823328494e-08
2265 4.47116440920325e-08
2266 4.46645435889792e-08
2267 4.45904586143797e-08
2268 4.46212184943562e-08
2269 4.45933494130912e-08
2270 4.4556232083437e-08
2271 4.44549301743535e-08
2272 4.45132543731575e-08
2273 4.4487074548627e-08
2274 4.44562111763247e-08
2275 4.44253362701375e-08
2276 4.43661219300395e-08
2277 4.43817699178339e-08
2278 4.43931922635699e-08
2279 4.43681545636565e-08
2280 4.43440845323551e-08
2281 4.43181080367339e-08
2282 4.42871757613261e-08
2283 4.42132345153112e-08
2284 4.42363947890101e-08
2285 4.42302015866147e-08
2286 4.42060766481234e-08
2287 4.41694749007127e-08
2288 4.40859274100092e-08
2289 4.4124187398964e-08
2290 4.40992242758398e-08
2291 4.40730994952787e-08
2292 4.40390695288784e-08
2293 4.39597891244148e-08
2294 4.40097331111389e-08
2295 4.39897901678421e-08
2296 4.3957182674248e-08
2297 4.39196966919297e-08
2298 4.38521444845463e-08
2299 4.38917233474001e-08
2300 4.38693765580922e-08
2301 4.38479594233598e-08
2302 4.38210303030928e-08
2303 4.37870435021637e-08
2304 4.37428878932167e-08
2305 4.37000997894899e-08
2306 4.37085213143007e-08
2307 4.3717084924566e-08
2308 4.36649006712031e-08
2309 4.36409786672698e-08
2310 4.34777031923517e-08
2311 4.36167849890978e-08
2312 4.35950487265302e-08
2313 4.35607050199849e-08
2314 4.35596397885263e-08
2315 4.35196548487227e-08
2316 4.34872846142298e-08
2317 4.3431660145643e-08
2318 4.34496065366829e-08
2319 4.3421044448877e-08
2320 4.33884562980325e-08
2321 4.33578444276605e-08
2322 4.33293603858687e-08
2323 4.33035741327359e-08
2324 4.33122432195177e-08
2325 4.32818126085976e-08
2326 4.32411842474778e-08
2327 4.31457829890292e-08
2328 4.32293729311084e-08
2329 4.31999107632919e-08
2330 4.31917391914283e-08
2331 4.31691175624138e-08
2332 4.31291352445129e-08
2333 4.31195264631867e-08
2334 4.30985505737169e-08
2335 4.29237812014094e-08
2336 4.30645557596421e-08
2337 4.30232016519483e-08
2338 4.30212624458193e-08
2339 4.29797165875101e-08
2340 4.29744402090648e-08
2341 4.29535872150666e-08
2342 4.29391151524072e-08
2343 4.27658471480186e-08
2344 4.28850892895127e-08
2345 4.28908921232818e-08
2346 4.28589313798966e-08
2347 4.28305705089116e-08
2348 4.27907999327459e-08
2349 4.27842802377398e-08
2350 4.27525152275621e-08
2351 4.27468373160878e-08
2352 4.27208658919653e-08
2353 4.27022871232197e-08
2354 4.25323109904951e-08
2355 4.26755401274903e-08
2356 4.2645332879232e-08
2357 4.26318120414493e-08
2358 4.26082215181367e-08
2359 4.25841586757514e-08
2360 4.25584666157164e-08
2361 4.25372690084913e-08
2362 4.23709302985742e-08
2363 4.25131132040235e-08
2364 4.23913715348334e-08
2365 4.23720828202079e-08
2366 4.2352994340078e-08
2367 4.23302519880764e-08
2368 4.23044159525432e-08
2369 4.2250244712605e-08
2370 4.22742721681857e-08
2371 4.22946492975029e-08
2372 4.22739869438971e-08
2373 4.22563522484154e-08
2374 4.2234660261542e-08
2375 4.22083518429872e-08
2376 4.21490197464181e-08
2377 4.21866258086823e-08
2378 4.21650758184455e-08
2379 4.21400457426557e-08
2380 4.21161352512911e-08
2381 4.21020009646611e-08
2382 4.20884409741973e-08
2383 4.19733471215267e-08
2384 4.20905425766449e-08
2385 4.20430005583228e-08
2386 4.20222229386979e-08
2387 4.19904432593654e-08
2388 4.19794556147934e-08
2389 4.19649085223739e-08
2390 4.18466850504728e-08
2391 4.19342578208415e-08
2392 4.1882618136313e-08
2393 4.18892802205306e-08
2394 4.1895994508323e-08
2395 4.18643812913899e-08
2396 4.18539959916586e-08
2397 4.18279299267965e-08
2398 4.17947687090248e-08
2399 4.18051694737187e-08
2400 4.17965911676532e-08
2401 4.17732240958202e-08
2402 4.17605825830947e-08
2403 4.17218980555134e-08
2404 4.17179504790255e-08
2405 4.16725791012595e-08
2406 4.16248882331161e-08
2407 4.16615631451123e-08
2408 4.16556923710232e-08
2409 4.16239897962356e-08
2410 4.15970008198485e-08
2411 4.15729522504904e-08
2412 4.15627426324505e-08
2413 4.15561389246477e-08
2414 4.14949939298026e-08
2415 4.15068270882557e-08
2416 4.14900548388886e-08
2417 4.14713520573429e-08
2418 4.14528661298874e-08
2419 4.1435806293677e-08
2420 4.14104725727782e-08
2421 4.13649932333726e-08
2422 4.13387416529076e-08
2423 4.1359001530239e-08
2424 4.13445906701782e-08
2425 4.1311115928977e-08
2426 4.12899158401814e-08
2427 4.12732021999318e-08
2428 4.12611721092304e-08
2429 4.12355817314136e-08
2430 4.12125230599969e-08
2431 4.1200550587206e-08
2432 4.11687094317159e-08
2433 4.11638790023261e-08
2434 4.11351490576806e-08
2435 4.11333790975021e-08
2436 4.11096864532112e-08
2437 4.10678282172228e-08
2438 4.10770999508259e-08
2439 4.10713886722647e-08
2440 4.10290819772285e-08
2441 4.09995307908417e-08
2442 4.09955186526645e-08
2443 4.09814307928968e-08
2444 4.09585854601602e-08
2445 4.09618614636287e-08
2446 4.09241405527894e-08
2447 4.09119637332367e-08
2448 4.08728439058592e-08
2449 4.08834688148119e-08
2450 4.08647507761373e-08
2451 4.08233834221505e-08
2452 4.08249627934509e-08
2453 4.07964790305471e-08
2454 4.08009151140476e-08
2455 4.0729824826613e-08
2456 4.07461509510654e-08
2457 4.07305479601661e-08
2458 4.0708263957967e-08
2459 4.07079554900491e-08
2460 4.06662952840975e-08
2461 4.06503679801773e-08
2462 4.06051581549605e-08
2463 4.06041341616259e-08
2464 4.05801480312107e-08
2465 4.0528524275274e-08
2466 4.057029738469e-08
2467 4.05316035507752e-08
2468 4.05247992691926e-08
2469 4.04905550723811e-08
2470 4.04988158564379e-08
2471 4.0452744801911e-08
2472 4.04322017182324e-08
2473 4.04285536017568e-08
2474 4.04032839274038e-08
2475 4.04101021374004e-08
2476 4.03787823071156e-08
2477 4.03408877769351e-08
2478 4.03375875492884e-08
2479 4.03150791168372e-08
2480 4.03052542381488e-08
2481 4.0297185352145e-08
2482 4.02454644667927e-08
2483 4.02437306039616e-08
2484 4.02249133522048e-08
2485 4.02310821723262e-08
2486 4.0186476757853e-08
2487 4.01852896878552e-08
2488 4.01486818244479e-08
2489 4.01480245599828e-08
2490 4.01276958648111e-08
2491 4.01227758022316e-08
2492 4.00835044800374e-08
2493 4.00761183385612e-08
2494 4.00452910511717e-08
2495 4.0046273008798e-08
2496 4.00169855208077e-08
2497 4.00113447369677e-08
2498 3.99825379275853e-08
2499 3.99732115248241e-08
2500 3.99439878435715e-08
2501 3.99430942366052e-08
2502 3.99196359399667e-08
2503 3.98822080160954e-08
2504 3.9889929384529e-08
2505 3.98999155901691e-08
2506 3.98632284870359e-08
2507 3.98313786131865e-08
2508 3.98015456877232e-08
2509 3.98418905174225e-08
2510 3.98002116561713e-08
2511 3.97732814434448e-08
2512 3.97323871776933e-08
2513 3.97520846817656e-08
2514 3.97345723150977e-08
2515 3.96853234896355e-08
2516 3.96939292990339e-08
2517 3.96833799349849e-08
2518 3.96488600298284e-08
2519 3.96014515047227e-08
2520 3.96305648298068e-08
2521 3.96192375475835e-08
2522 3.96118811334389e-08
2523 3.95793836709402e-08
2524 3.95282199612268e-08
2525 3.95468313829639e-08
2526 3.95483649935358e-08
2527 3.95259738752429e-08
2528 3.94511645094298e-08
2529 3.94848817428795e-08
2530 3.94658198583642e-08
2531 3.94601141060491e-08
2532 3.94597062314261e-08
2533 3.94148429734287e-08
2534 3.93701627956489e-08
2535 3.93957675086654e-08
2536 3.9372652512526e-08
2537 3.93802923763786e-08
2538 3.93214715241896e-08
2539 3.93113239844922e-08
2540 3.92591092399641e-08
2541 3.93089573247352e-08
2542 3.93019358764235e-08
2543 3.92702769502762e-08
2544 3.92534881772377e-08
2545 3.92334921741622e-08
2546 3.91851903209783e-08
2547 3.91910531725159e-08
2548 3.91866449103162e-08
2549 3.91798933190302e-08
2550 3.91505315207041e-08
2551 3.91154232310953e-08
2552 3.91411963445165e-08
2553 3.91113729723003e-08
2554 3.90869754678391e-08
2555 3.90730157313612e-08
2556 3.90500811509042e-08
2557 3.90366445284229e-08
2558 3.90071481142229e-08
2559 3.9006419271459e-08
2560 3.8968755111668e-08
2561 3.89910452405218e-08
2562 3.89834474265882e-08
2563 3.89505073048468e-08
2564 3.8943785124701e-08
2565 3.8922639314265e-08
2566 3.89024450750952e-08
2567 3.89105179294802e-08
2568 3.88764581860102e-08
2569 3.88603908767493e-08
2570 3.88348846644959e-08
2571 3.88413576875024e-08
2572 3.88328551252926e-08
2573 3.87999664894778e-08
2574 3.87720016323811e-08
2575 3.87866802356029e-08
2576 3.87507390975372e-08
2577 3.87514867696836e-08
2578 3.87138331419123e-08
2579 3.87078911732175e-08
2580 3.86815368909055e-08
2581 3.86847731963513e-08
2582 3.86536464684895e-08
2583 3.86725805885391e-08
2584 3.86340200240198e-08
2585 3.86118314903428e-08
2586 3.86030046310992e-08
2587 3.86050445673192e-08
2588 3.85744072790573e-08
2589 3.8550422253536e-08
2590 3.85523964165913e-08
2591 3.85327702634442e-08
2592 3.85300959138135e-08
2593 3.85020067774633e-08
2594 3.84280085246047e-08
2595 3.84675528213307e-08
2596 3.8465537885557e-08
2597 3.84522505534335e-08
2598 3.84184443831259e-08
2599 3.84043870127471e-08
2600 3.83860050341411e-08
2601 3.8379754769835e-08
2602 3.83639172341077e-08
2603 3.83380048400994e-08
2604 3.83257116975955e-08
2605 3.83306565474584e-08
2606 3.82549308195479e-08
2607 3.83024122463382e-08
2608 3.82637356199922e-08
2609 3.82580971631796e-08
2610 3.82399417855339e-08
2611 3.82485485097561e-08
2612 3.82105855418757e-08
2613 3.81983997819191e-08
2614 3.81853628113049e-08
2615 3.8165821440117e-08
2616 3.81544839322956e-08
2617 3.81454330096176e-08
2618 3.8146543017703e-08
2619 3.8112728198314e-08
2620 3.80845848617639e-08
2621 3.80943796791655e-08
2622 3.80820519580993e-08
2623 3.80511252373594e-08
2624 3.80414085014991e-08
2625 3.8024895159694e-08
2626 3.80398642683133e-08
2627 3.79925073641374e-08
2628 3.79806511334735e-08
2629 3.79813928716999e-08
2630 3.79603886511148e-08
2631 3.79710934588218e-08
2632 3.79297719508287e-08
2633 3.7916140959382e-08
2634 3.7905934382465e-08
2635 3.78893997687868e-08
2636 3.78788426900201e-08
2637 3.78771935558575e-08
2638 3.78313198590163e-08
2639 3.78423559395102e-08
2640 3.78250560828519e-08
2641 3.77896914525167e-08
2642 3.77779407338963e-08
2643 3.77647311005092e-08
2644 3.77487602944981e-08
2645 3.77482479674285e-08
2646 3.7721190391693e-08
2647 3.76954762915105e-08
2648 3.77130418227622e-08
2649 3.76971419138172e-08
2650 3.7659381816546e-08
2651 3.76341910648392e-08
2652 3.76425724111229e-08
2653 3.76045361578647e-08
2654 3.76260388739169e-08
2655 3.75920251469708e-08
2656 3.75935030696439e-08
2657 3.7587102260872e-08
2658 3.75575901641056e-08
2659 3.75534213556961e-08
2660 3.75173137525664e-08
2661 3.74976227908164e-08
2662 3.75115347654997e-08
2663 3.74805105334275e-08
2664 3.74661058897274e-08
2665 3.74398156148814e-08
2666 3.7445197374808e-08
2667 3.74174023445306e-08
2668 3.74378029581734e-08
2669 3.74151325663874e-08
2670 3.73740181291993e-08
2671 3.7371464413738e-08
2672 3.73682982068857e-08
2673 3.73575995311626e-08
2674 3.73395527155651e-08
2675 3.73212800797162e-08
2676 3.73028246674068e-08
2677 3.72894001152702e-08
2678 3.72873531127027e-08
2679 3.7269188326583e-08
2680 3.72623258844129e-08
2681 3.72557005547947e-08
2682 3.72475442285136e-08
2683 3.7211453760122e-08
2684 3.72067180594016e-08
2685 3.71848209734793e-08
2686 3.71837137773667e-08
2687 3.71622289234708e-08
2688 3.71379570447417e-08
2689 3.71243856358561e-08
2690 3.71490210744341e-08
2691 3.71421973026642e-08
2692 3.71273158563668e-08
2693 3.71171329414111e-08
2694 3.71096861861986e-08
2695 3.70848700299575e-08
2696 3.70601374450885e-08
2697 3.70216858360806e-08
2698 3.70182125335461e-08
2699 3.70256939774549e-08
2700 3.70052595819459e-08
2701 3.6996633171249e-08
2702 3.6982613762504e-08
2703 3.69419850461128e-08
2704 3.6945403619093e-08
2705 3.69222981060346e-08
2706 3.69169170015837e-08
2707 3.69114695146777e-08
2708 3.68895974274253e-08
2709 3.68587028640732e-08
2710 3.68505588745904e-08
2711 3.68381309616694e-08
2712 3.68324303581247e-08
2713 3.68148309251026e-08
2714 3.67982019673363e-08
2715 3.67824967888453e-08
2716 3.67947296524562e-08
2717 3.67626030799428e-08
2718 3.67050601113306e-08
2719 3.67340034523878e-08
2720 3.67466272113148e-08
2721 3.67088876513932e-08
2722 3.66999255563272e-08
2723 3.66875307040715e-08
2724 3.66663744291174e-08
2725 3.66674989447091e-08
2726 3.66432242602244e-08
2727 3.66120439982964e-08
2728 3.6591898177285e-08
2729 3.66049471498542e-08
2730 3.65913431883413e-08
2731 3.65817399679003e-08
2732 3.65641131310213e-08
2733 3.65229880792128e-08
2734 3.65337926657716e-08
2735 3.65244138942344e-08
2736 3.65164683957531e-08
2737 3.64805107224697e-08
2738 3.64782965904809e-08
2739 3.64732983051397e-08
2740 3.64379345763055e-08
2741 3.643934704467e-08
2742 3.64287587109047e-08
2743 3.6413844652472e-08
2744 3.63983738154161e-08
2745 3.63832682888088e-08
2746 3.63731179238158e-08
2747 3.63630475748167e-08
2748 3.63427284204576e-08
2749 3.63085527181894e-08
2750 3.63072569768974e-08
2751 3.62954209691679e-08
2752 3.62854342093044e-08
2753 3.6261050288644e-08
2754 3.62600296828219e-08
2755 3.62485685982961e-08
2756 3.622081621657e-08
2757 3.62108120892657e-08
2758 3.62113313308043e-08
2759 3.61933094072953e-08
2760 3.61696258295296e-08
2761 3.61588501451848e-08
2762 3.6141711703408e-08
2763 3.61311573975343e-08
2764 3.61212795798593e-08
2765 3.612757666005e-08
2766 3.6089704030573e-08
2767 3.60850355285436e-08
2768 3.60815522064684e-08
2769 3.6053272669534e-08
2770 3.60299104080752e-08
2771 3.60151649569929e-08
2772 3.60264565717472e-08
2773 3.59930005551234e-08
2774 3.59797762943259e-08
2775 3.59715989723952e-08
2776 3.596097021763e-08
2777 3.59424708769041e-08
2778 3.59223157193611e-08
2779 3.59249871699774e-08
2780 3.59222075783094e-08
2781 3.58991567601663e-08
2782 3.58934539317346e-08
2783 3.58483505671359e-08
2784 3.57931873686823e-08
2785 3.58323707434494e-08
2786 3.58400749895793e-08
2787 3.58267225664122e-08
2788 3.58107481499559e-08
2789 3.57935264201359e-08
2790 3.57612293893084e-08
2791 3.57207612520227e-08
2792 3.57600267211211e-08
2793 3.56976520183849e-08
2794 3.57281426612843e-08
2795 3.572241723937e-08
2796 3.57028620765476e-08
2797 3.56410245316852e-08
2798 3.56804971506364e-08
2799 3.56675034414167e-08
2800 3.56559541554446e-08
2801 3.5640810729376e-08
2802 3.55740307167451e-08
2803 3.56331063393611e-08
2804 3.56113960631177e-08
2805 3.55708678636546e-08
2806 3.5515884167836e-08
2807 3.55491469594682e-08
2808 3.55034729295767e-08
2809 3.55670982443712e-08
2810 3.5483534123415e-08
2811 3.55162239618068e-08
2812 3.54999613936258e-08
2813 3.54870322922807e-08
2814 3.54180654760228e-08
2815 3.54583764909933e-08
2816 3.5453208534264e-08
2817 3.53818354010826e-08
2818 3.54304230629765e-08
2819 3.5413887268021e-08
2820 3.5332078157424e-08
2821 3.53933670371731e-08
2822 3.53679826989861e-08
2823 3.53254223561805e-08
2824 3.53583047267847e-08
2825 3.53362263272317e-08
2826 3.52855342491409e-08
2827 3.53195658853167e-08
2828 3.5293208727083e-08
2829 3.52370237379773e-08
2830 3.52878328957473e-08
2831 3.5256458643218e-08
2832 3.51931423931973e-08
2833 3.52453957663812e-08
2834 3.52301080042139e-08
2835 3.51596725316483e-08
2836 3.52107794174827e-08
2837 3.51865573779975e-08
2838 3.5115083230508e-08
2839 3.51889354881507e-08
2840 3.51183755018525e-08
2841 3.51337076569536e-08
2842 3.50906864321487e-08
2843 3.50812442455606e-08
2844 3.51010255794648e-08
2845 3.50519028824436e-08
2846 3.50436249174635e-08
2847 3.5071131227582e-08
2848 3.50157383524063e-08
2849 3.50093028407272e-08
2850 3.50466128029581e-08
2851 3.4992784928356e-08
2852 3.5013205932799e-08
2853 3.49604573424145e-08
2854 3.49471525575851e-08
2855 3.49763429294825e-08
2856 3.49319283463601e-08
2857 3.48963831111604e-08
2858 3.49528450040992e-08
2859 3.48960244105356e-08
2860 3.49257197589381e-08
2861 3.49169837612351e-08
2862 3.48477945530945e-08
2863 3.48671259464695e-08
2864 3.48349230376854e-08
2865 3.48274674752957e-08
2866 3.48478227021332e-08
2867 3.47965070792711e-08
2868 3.47771791426865e-08
2869 3.48148009798876e-08
2870 3.47976449299736e-08
2871 3.47306411656589e-08
2872 3.47751444671474e-08
2873 3.47281974075742e-08
2874 3.47188109159902e-08
2875 3.47554732993416e-08
2876 3.46848429400026e-08
2877 3.47051756754979e-08
2878 3.46873941543535e-08
2879 3.46612394777068e-08
2880 3.46489446068077e-08
2881 3.46731915747966e-08
2882 3.46189911759609e-08
2883 3.46298074465068e-08
2884 3.46266480164559e-08
2885 3.4583620668549e-08
2886 3.45736194820034e-08
2887 3.45726523001133e-08
2888 3.46041050676149e-08
2889 3.45347349055558e-08
2890 3.45253473224005e-08
2891 3.45592167452224e-08
2892 3.45033674218342e-08
2893 3.44828832066213e-08
2894 3.45126397895967e-08
2895 3.45049588279522e-08
2896 3.44627479709203e-08
2897 3.4456447741249e-08
2898 3.44434392793858e-08
2899 3.44550878459415e-08
2900 3.44327372179265e-08
2901 3.43962602240566e-08
2902 3.43959430750829e-08
2903 3.44090698760624e-08
2904 3.43922568681876e-08
2905 3.43469687891229e-08
2906 3.43461781682208e-08
2907 3.43688260802111e-08
2908 3.43201310446517e-08
2909 3.43133899143311e-08
2910 3.4302094537253e-08
2911 3.43219062681754e-08
2912 3.42809786637588e-08
2913 3.42736876213223e-08
2914 3.42652996501158e-08
2915 3.426597947076e-08
2916 3.42391126491037e-08
2917 3.42273181797026e-08
2918 3.42128124977137e-08
2919 3.42335618324086e-08
2920 3.41815797106904e-08
2921 3.41784687432067e-08
2922 3.41722865524474e-08
2923 3.41539279329339e-08
2924 3.41716430121153e-08
2925 3.41354028039476e-08
2926 3.41006305406566e-08
2927 3.41110815202938e-08
2928 3.41301376121095e-08
2929 3.40873233426464e-08
2930 3.40744038824781e-08
2931 3.40664148348324e-08
2932 3.40624125092504e-08
2933 3.40576035808837e-08
2934 3.40345196025993e-08
2935 3.40238345044597e-08
2936 3.40087893331287e-08
2937 3.39991784308324e-08
2938 3.40093287123366e-08
2939 3.39697551590135e-08
2940 3.39723045241769e-08
2941 3.39586630309086e-08
2942 3.39468321524095e-08
2943 3.3956623310516e-08
2944 3.39173221437505e-08
2945 3.39032341960532e-08
2946 3.39015448096447e-08
2947 3.39311414201404e-08
2948 3.38749385058179e-08
2949 3.38971272402233e-08
2950 3.38519091851452e-08
2951 3.383097439702e-08
2952 3.3875461234345e-08
2953 3.38097902492507e-08
2954 3.38039199014872e-08
2955 3.37882950010027e-08
2956 3.38409256688266e-08
2957 3.3753443384299e-08
2958 3.3757267889456e-08
2959 3.37340020966082e-08
2960 3.37965984602562e-08
2961 3.37113758686058e-08
2962 3.37166249710918e-08
2963 3.36868613963759e-08
2964 3.37058772199583e-08
2965 3.37071112817e-08
2966 3.36725606935317e-08
2967 3.36538970309164e-08
2968 3.36446257485079e-08
2969 3.36759991093061e-08
2970 3.36194108143673e-08
2971 3.3605576875928e-08
2972 3.36171867427737e-08
2973 3.35867481719987e-08
2974 3.36139828664983e-08
2975 3.35659101899921e-08
2976 3.35577716334967e-08
2977 3.35596111504799e-08
2978 3.35536629645361e-08
2979 3.35498784398425e-08
2980 3.35312849486513e-08
2981 3.35104112849649e-08
2982 3.35183736837052e-08
2983 3.34913795736469e-08
2984 3.3498254023101e-08
2985 3.35009802441277e-08
2986 3.34581236405285e-08
2987 3.34612830918957e-08
2988 3.34433859716654e-08
2989 3.34397935262132e-08
2990 3.34546299463057e-08
2991 3.33986035787603e-08
2992 3.33964256480357e-08
2993 3.33910725220932e-08
2994 3.33921079498367e-08
2995 3.33957425251441e-08
2996 3.33518347055417e-08
2997 3.33447586697844e-08
2998 3.33790950506341e-08
2999 3.33246440256119e-08
3000 3.33226458888447e-08
3001 3.33261658518325e-08
3002 3.33058218675575e-08
3003 3.32757160492747e-08
3004 3.32736772357123e-08
3005 3.32764549355602e-08
3006 3.32648458858387e-08
3007 3.3275189977644e-08
3008 3.32380122589626e-08
3009 3.323918306819e-08
3010 3.321297577763e-08
3011 3.3193527693598e-08
3012 3.3195481286441e-08
3013 3.32087202288278e-08
3014 3.31601590968234e-08
3015 3.31417172869664e-08
3016 3.3175327432744e-08
3017 3.31273028866264e-08
3018 3.31238603479278e-08
3019 3.31124241217395e-08
3020 3.31338131029924e-08
3021 3.30732189475214e-08
3022 3.30805242487742e-08
3023 3.30620634052536e-08
3024 3.30503184731157e-08
3025 3.30454387604817e-08
3026 3.30581174932121e-08
3027 3.30102115331243e-08
3028 3.30171609039809e-08
3029 3.29913578909213e-08
3030 3.29877789955191e-08
3031 3.29902162672369e-08
3032 3.29759938342278e-08
3033 3.29942981789344e-08
3034 3.29358447750394e-08
3035 3.29296326597728e-08
3036 3.2932736118596e-08
3037 3.29147308804067e-08
3038 3.29148777407084e-08
3039 3.28885892066921e-08
3040 3.29327022470238e-08
3041 3.28661929902552e-08
3042 3.28549725150396e-08
3043 3.28448576318863e-08
3044 3.2831045029269e-08
3045 3.28191639358266e-08
3046 3.28593494351992e-08
3047 3.27826811172827e-08
3048 3.27934241521888e-08
3049 3.27997882223485e-08
3050 3.27407038138716e-08
3051 3.27394402361847e-08
3052 3.27573368252843e-08
3053 3.27206876313824e-08
3054 3.27034582969787e-08
3055 3.2755227065806e-08
3056 3.26787495499303e-08
3057 3.26819023204195e-08
3058 3.26726872454941e-08
3059 3.26454832215717e-08
3060 3.26526280556294e-08
3061 3.26519700024619e-08
3062 3.26212117407465e-08
3063 3.26532961807402e-08
3064 3.26191587447511e-08
3065 3.2596290889586e-08
3066 3.26020497745105e-08
3067 3.25715931381865e-08
3068 3.25626066413065e-08
3069 3.25576006421358e-08
3070 3.25377844436758e-08
3071 3.25359949293613e-08
3072 3.25645296630483e-08
3073 3.25298557921627e-08
3074 3.2492625349434e-08
3075 3.24766670711796e-08
3076 3.24891637362157e-08
3077 3.24687187021056e-08
3078 3.24445253685468e-08
3079 3.24570922707323e-08
3080 3.24381635010695e-08
3081 3.24478097173397e-08
3082 3.24148258492585e-08
3083 3.24444184975903e-08
3084 3.24116177221612e-08
3085 3.23846534975303e-08
3086 3.2383421757487e-08
3087 3.23849222478856e-08
3088 3.23682744607368e-08
3089 3.23356247236006e-08
3090 3.23556642491951e-08
3091 3.23403282251888e-08
3092 3.23284786318823e-08
3093 3.22946433861659e-08
3094 3.2301182599781e-08
3095 3.23345774173589e-08
3096 3.22846054885417e-08
3097 3.22702678090536e-08
3098 3.22615816301663e-08
3099 3.22461515747108e-08
3100 3.22470324309876e-08
3101 3.22431559727221e-08
3102 3.22226959124094e-08
3103 3.22232005842693e-08
3104 3.21962778997431e-08
3105 3.2190317018177e-08
3106 3.21803027159007e-08
3107 3.21738950930239e-08
3108 3.21987416853631e-08
3109 3.21542074654957e-08
3110 3.21394319691137e-08
3111 3.21179362483548e-08
3112 3.21252830968177e-08
3113 3.21156675191503e-08
3114 3.21263296214624e-08
3115 3.20942541609526e-08
3116 3.2088067742464e-08
3117 3.20694870090676e-08
3118 3.20709211081294e-08
3119 3.20529243982293e-08
3120 3.20344427855446e-08
3121 3.20563697524534e-08
3122 3.2031343586425e-08
3123 3.20204357038278e-08
3124 3.20161546021325e-08
3125 3.19903786962783e-08
3126 3.19792866427804e-08
3127 3.19709238780774e-08
3128 3.19444532443924e-08
3129 3.19430888460204e-08
3130 3.19561198818263e-08
3131 3.19294523141167e-08
3132 3.19149268062091e-08
3133 3.18999331252456e-08
3134 3.18925400737413e-08
3135 3.18963544874151e-08
3136 3.18629420377192e-08
3137 3.18690245766362e-08
3138 3.18562362906505e-08
3139 3.18477000185169e-08
3140 3.18054936698786e-08
3141 3.18323724073366e-08
3142 3.18140324555571e-08
3143 3.18000488501724e-08
3144 3.18064157465159e-08
3145 3.17529248192727e-08
3146 3.17728682546203e-08
3147 3.17576308237477e-08
3148 3.17629466639602e-08
3149 3.17268123666281e-08
3150 3.17009766526155e-08
3151 3.17173004837912e-08
3152 3.17090565324918e-08
3153 3.1687891393517e-08
3154 3.16879303436934e-08
3155 3.16490807570347e-08
3156 3.1663408810445e-08
3157 3.16577604184687e-08
3158 3.16368841932757e-08
3159 3.16188921267724e-08
3160 3.16290986113188e-08
3161 3.16094933499045e-08
3162 3.15948663978105e-08
3163 3.15854255710235e-08
3164 3.15780778716857e-08
3165 3.15741093146471e-08
3166 3.15577985361415e-08
3167 3.15488278150866e-08
3168 3.15676848803292e-08
3169 3.15240520425419e-08
3170 3.15230348455486e-08
3171 3.15138087483291e-08
3172 3.14894466075799e-08
3173 3.14958266578458e-08
3174 3.149612039266e-08
3175 3.14787527706528e-08
3176 3.14922364275105e-08
3177 3.14502925728988e-08
3178 3.14545552697609e-08
3179 3.14412834097055e-08
3180 3.1434627249638e-08
3181 3.14297572945321e-08
3182 3.14107391332641e-08
3183 3.14010909345797e-08
3184 3.13908607463276e-08
3185 3.13742971158604e-08
3186 3.13790958781368e-08
3187 3.13564836567082e-08
3188 3.13293032370154e-08
3189 3.13329292556119e-08
3190 3.13217340188743e-08
3191 3.13037690347073e-08
3192 3.13081407874449e-08
3193 3.12980313665889e-08
3194 3.12455954798452e-08
3195 3.13018276809629e-08
3196 3.12615314879139e-08
3197 3.12500283765615e-08
3198 3.12458849069941e-08
3199 3.12334217955623e-08
3200 3.12527116754779e-08
3201 3.1209481099026e-08
3202 3.11993381814091e-08
3203 3.11977669085195e-08
3204 3.11698805788296e-08
3205 3.11717048386839e-08
3206 3.11628351159499e-08
3207 3.11474883680773e-08
3208 3.11317612737838e-08
3209 3.1131566519349e-08
3210 3.11250501958682e-08
3211 3.11101239525158e-08
3212 3.1070482366502e-08
3213 3.11025106487506e-08
3214 3.1087290551568e-08
3215 3.10874076046019e-08
3216 3.10661931752776e-08
3217 3.10607972195953e-08
3218 3.10088888486604e-08
3219 3.10497015885147e-08
3220 3.10386649609029e-08
3221 3.10324884402746e-08
3222 3.10231695141994e-08
3223 3.10046817340037e-08
3224 3.0998177711794e-08
3225 3.09833966749551e-08
3226 3.09775187012917e-08
3227 3.09292222517854e-08
3228 3.09629137404954e-08
3229 3.09683497814461e-08
3230 3.09487840510769e-08
3231 3.09367305106889e-08
3232 3.0933183257531e-08
3233 3.09186857485599e-08
3234 3.08963153727859e-08
3235 3.09035411554248e-08
3236 3.08961565398391e-08
3237 3.08825570414228e-08
3238 3.08738404299902e-08
3239 3.08674600777437e-08
3240 3.08177137728904e-08
3241 3.08554063224165e-08
3242 3.08403053725925e-08
3243 3.08473986443403e-08
3244 3.08223508174166e-08
3245 3.08128239403516e-08
3246 3.07953079321521e-08
3247 3.07585378571673e-08
3248 3.07846144647073e-08
3249 3.07845835259002e-08
3250 3.07666694698838e-08
3251 3.0756750778238e-08
3252 3.07497899783016e-08
3253 3.07199204261366e-08
3254 3.07341069074596e-08
3255 3.07197006996773e-08
3256 3.07119001305978e-08
3257 3.07032242261585e-08
3258 3.06978727735441e-08
3259 3.06907280247515e-08
3260 3.06436479693417e-08
3261 3.06827634091178e-08
3262 3.06653848021199e-08
3263 3.06575156763245e-08
3264 3.06561952942985e-08
3265 3.06304495794052e-08
3266 3.06200165240966e-08
3267 3.05806554177224e-08
3268 3.06183212614997e-08
3269 3.06241753129655e-08
3270 3.05945953051889e-08
3271 3.059702481778e-08
3272 3.05847435360107e-08
3273 3.05905642896676e-08
3274 3.0529358625131e-08
3275 3.05585155508936e-08
3276 3.05533962663418e-08
3277 3.0546981006907e-08
3278 3.05219828984349e-08
3279 3.05325772931297e-08
3280 3.05147988619581e-08
3281 3.04981958230144e-08
3282 3.05035789125441e-08
3283 3.04969247526543e-08
3284 3.04844692475825e-08
3285 3.04412345553118e-08
3286 3.04673283597623e-08
3287 3.04786368250376e-08
3288 3.04484539590533e-08
3289 3.04429651549754e-08
3290 3.04306414200539e-08
3291 3.04334017187102e-08
3292 3.03852491772005e-08
3293 3.04165218505403e-08
3294 3.04094226351026e-08
3295 3.04038744705082e-08
3296 3.03848307741106e-08
3297 3.03781186392627e-08
3298 3.03767770475361e-08
3299 3.03246655661837e-08
3300 3.03618594035271e-08
3301 3.0352035656378e-08
3302 3.03458556949465e-08
3303 3.03315679470728e-08
3304 3.03321631989206e-08
3305 3.0319675566659e-08
3306 3.02742999149785e-08
3307 3.03101878209588e-08
3308 3.03070294052077e-08
3309 3.02875477675713e-08
3310 3.02780334777708e-08
3311 3.02835558372294e-08
3312 3.02667381681943e-08
3313 3.02528447644335e-08
3314 3.02482807512661e-08
3315 3.02522361046442e-08
3316 3.01976981074148e-08
3317 3.02393796847156e-08
3318 3.02264512370698e-08
3319 3.0214168837972e-08
3320 3.02136211711712e-08
3321 3.02056938465967e-08
3322 3.0191739442742e-08
3323 3.01883279529847e-08
3324 3.01362650727555e-08
3325 3.01884495961247e-08
3326 3.01583961661578e-08
3327 3.01543728440379e-08
3328 3.01471925787666e-08
3329 3.01593281903934e-08
3330 3.01307843688647e-08
3331 3.01141655381088e-08
3332 3.01010578560579e-08
3333 3.01103685149684e-08
3334 3.01219981366785e-08
3335 3.0061185901431e-08
3336 3.00841548828146e-08
3337 3.00865231004366e-08
3338 3.00759385662985e-08
3339 3.00575560530092e-08
3340 3.00526854921657e-08
3341 3.00584563603934e-08
3342 3.00423610237743e-08
3343 3.0031049735868e-08
3344 3.00368119372507e-08
3345 3.00368259633643e-08
3346 2.99969403751987e-08
3347 3.00251379954375e-08
3348 3.0002362903403e-08
3349 3.00074251438076e-08
3350 3.00016303658168e-08
3351 2.99903835294657e-08
3352 2.99805532684161e-08
3353 2.99745802916362e-08
3354 2.99705945145945e-08
3355 2.99332354991577e-08
3356 2.99590261132465e-08
3357 2.99595489554605e-08
3358 2.99436511603801e-08
3359 2.99393638005796e-08
3360 2.99566965598075e-08
3361 2.99268269738917e-08
3362 2.99191758923456e-08
3363 2.99071994618316e-08
3364 2.98836899972343e-08
3365 2.99201858986464e-08
3366 2.98984826940796e-08
3367 2.98911041216599e-08
3368 2.98804803655628e-08
3369 2.98893392205457e-08
3370 2.98673842689112e-08
3371 2.98590652771935e-08
3372 2.9845521003935e-08
3373 2.98437528609696e-08
3374 2.982681233199e-08
3375 2.98380682419719e-08
3376 2.98183987386125e-08
3377 2.98062721491021e-08
3378 2.98226640005339e-08
3379 2.98027508556231e-08
3380 2.97858168529785e-08
3381 2.98056735523744e-08
3382 2.97739556636145e-08
3383 2.97647932523404e-08
3384 2.97711074743745e-08
3385 2.97593303493215e-08
3386 2.97475350343745e-08
3387 2.9746108905826e-08
3388 2.97374208741985e-08
3389 2.97381274343422e-08
3390 2.96983938525841e-08
3391 2.97292966422447e-08
3392 2.97140044143163e-08
3393 2.97208665838156e-08
3394 2.96965916142256e-08
3395 2.96947497133715e-08
3396 2.96874901302147e-08
3397 2.96789193630076e-08
3398 2.96891326492243e-08
3399 2.96488377333759e-08
3400 2.96578379632706e-08
3401 2.96581761656256e-08
3402 2.96430169548501e-08
3403 2.96273827409266e-08
3404 2.9636173549008e-08
3405 2.9634537952461e-08
3406 2.96047538377309e-08
3407 2.96262109991119e-08
3408 2.96041990903717e-08
3409 2.95906878964303e-08
3410 2.95959361551468e-08
3411 2.96037632718793e-08
3412 2.95627976765189e-08
3413 2.95910818941536e-08
3414 2.95582902989366e-08
3415 2.95498301472463e-08
3416 2.95568795021239e-08
3417 2.96028556689976e-08
3418 2.95472083760018e-08
3419 2.95579444493654e-08
3420 2.95430449348544e-08
3421 2.95486031127723e-08
3422 2.9515453336515e-08
3423 2.95147808024865e-08
3424 2.95111736150488e-08
3425 2.94837864345254e-08
3426 2.94932959103988e-08
3427 2.94880167892586e-08
3428 2.94793182593622e-08
3429 2.94689392923431e-08
3430 2.94586316567091e-08
3431 2.94575297452582e-08
3432 2.94573583712321e-08
3433 2.9444882414964e-08
3434 2.94358231212044e-08
3435 2.94407327174895e-08
3436 2.94241843974419e-08
3437 2.93771793398179e-08
3438 2.94239479945446e-08
3439 2.94059500802746e-08
3440 2.94033828893703e-08
3441 2.93922603411545e-08
3442 2.94080431668675e-08
3443 2.93812358496126e-08
3444 2.9363902982027e-08
3445 2.93716641301955e-08
3446 2.9352542398442e-08
3447 2.9323485101429e-08
3448 2.93485045652631e-08
3449 2.93573673140202e-08
3450 2.9331110177111e-08
3451 2.93276159073486e-08
3452 2.93234412733767e-08
3453 2.93033425933231e-08
3454 2.93031124325438e-08
3455 2.92906887811029e-08
3456 2.92897768652267e-08
3457 2.92763853426692e-08
3458 2.92591270358855e-08
3459 2.92743911654014e-08
3460 2.92694937620297e-08
3461 2.92593001756103e-08
3462 2.92394049417055e-08
3463 2.92526000098547e-08
3464 2.92306848219681e-08
3465 2.92303294671115e-08
3466 2.92070744514206e-08
3467 2.92165801827338e-08
3468 2.92098096501547e-08
3469 2.92016568241849e-08
3470 2.91938745569098e-08
3471 2.91876248184053e-08
3472 2.91383946660062e-08
3473 2.91860052925585e-08
3474 2.91616663439243e-08
3475 2.91653742703346e-08
3476 2.91580272708813e-08
3477 2.91521434476749e-08
3478 2.91409051929037e-08
3479 2.91282261155601e-08
3480 2.91352057590899e-08
3481 2.9130669993549e-08
3482 2.91087607209306e-08
3483 2.91030048149565e-08
3484 2.91072219127386e-08
3485 2.90902239381552e-08
3486 2.90817061667781e-08
3487 2.90858963012397e-08
3488 2.90717359536785e-08
3489 2.90743037787422e-08
3490 2.90542463527998e-08
3491 2.90430776495043e-08
3492 2.90503239988027e-08
3493 2.90481068585535e-08
3494 2.90419312545254e-08
3495 2.90233203088519e-08
3496 2.90238575839652e-08
3497 2.90164932419401e-08
3498 2.89796522565666e-08
3499 2.90175056765207e-08
3500 2.89943303197759e-08
3501 2.89981651953042e-08
3502 2.89771236410274e-08
3503 2.89858673951926e-08
3504 2.89770372585707e-08
3505 2.8962363952445e-08
3506 2.8966058097879e-08
3507 2.89557560755327e-08
3508 2.89441889620434e-08
3509 2.89268171407286e-08
3510 2.89434127562771e-08
3511 2.89283843990518e-08
3512 2.89252245977423e-08
3513 2.89236518558056e-08
3514 2.89033845586317e-08
3515 2.89061767269061e-08
3516 2.88905125636063e-08
3517 2.88831436296988e-08
3518 2.88560542767158e-08
3519 2.88774539747294e-08
3520 2.88672562831493e-08
3521 2.88574887541415e-08
3522 2.88353518129725e-08
3523 2.88327692512524e-08
3524 2.88102519263589e-08
3525 2.8814491972895e-08
3526 2.88033451880665e-08
3527 2.87933878304614e-08
3528 2.87748030434187e-08
3529 2.87854977685242e-08
3530 2.87795501421328e-08
3531 2.87698944223536e-08
3532 2.87683161186436e-08
3533 2.8754158512001e-08
3534 2.87447741822433e-08
3535 2.87496519018049e-08
3536 2.87298404533232e-08
3537 2.87171174413459e-08
3538 2.87282357316343e-08
3539 2.87262300791014e-08
3540 2.87199414881201e-08
3541 2.86984017776604e-08
3542 2.87044352482724e-08
3543 2.87062420909479e-08
3544 2.86905590627384e-08
3545 2.86801403177606e-08
3546 2.86761320520412e-08
3547 2.86546235788165e-08
3548 2.86686802084546e-08
3549 2.86599204759597e-08
3550 2.86663994444325e-08
3551 2.86563869380529e-08
3552 2.86394563850934e-08
3553 2.86303629586371e-08
3554 2.86215290632441e-08
3555 2.86130422804121e-08
3556 2.86346822928607e-08
3557 2.86087313075001e-08
3558 2.86085644312095e-08
3559 2.86028457097132e-08
3560 2.85928994561146e-08
3561 2.85865783311579e-08
3562 2.85818177214736e-08
3563 2.85807938595894e-08
3564 2.85975265299498e-08
3565 2.85408836262491e-08
3566 2.85564303901253e-08
3567 2.8546471149582e-08
3568 2.85453319701645e-08
3569 2.85458333326716e-08
3570 2.85346541666343e-08
3571 2.85198625462613e-08
3572 2.85127151329334e-08
3573 2.85259895722589e-08
3574 2.85064312670613e-08
3575 2.85065644440863e-08
3576 2.8511733468406e-08
3577 2.8492953456194e-08
3578 2.84888863344435e-08
3579 2.84793328635402e-08
3580 2.84748212404651e-08
3581 2.84731083119993e-08
3582 2.8455909479419e-08
3583 2.84400008396801e-08
3584 2.84572819939655e-08
3585 2.84424645755621e-08
3586 2.84425115832931e-08
3587 2.84393040654862e-08
3588 2.84243705941378e-08
3589 2.84614100980463e-08
3590 2.84327507173998e-08
3591 2.84211793459832e-08
3592 2.84119993949616e-08
3593 2.83958976901033e-08
3594 2.83959893536689e-08
3595 2.83867914117053e-08
3596 2.83818313242534e-08
3597 2.83824364775143e-08
3598 2.83714037347949e-08
3599 2.83709514530273e-08
3600 2.83577618809261e-08
3601 2.83770171787268e-08
3602 2.83892165029442e-08
3603 2.83807764169808e-08
3604 2.8364803347003e-08
3605 2.83289252092089e-08
3606 2.83295445306919e-08
3607 2.83270674401592e-08
3608 2.83158148466356e-08
3609 2.83098015287919e-08
3610 2.83096540218963e-08
3611 2.82998656313538e-08
3612 2.82903563508796e-08
3613 2.82835175919871e-08
3614 2.82809397944561e-08
3615 2.82719383442043e-08
3616 2.82697573741331e-08
3617 2.82566463880585e-08
3618 2.82617692928255e-08
3619 2.82600117014198e-08
3620 2.82571929197672e-08
3621 2.82443945476274e-08
3622 2.82570042582364e-08
3623 2.82229269661372e-08
3624 2.82244652485275e-08
3625 2.82130810820291e-08
3626 2.82085318179526e-08
3627 2.82072075012962e-08
3628 2.82174675323432e-08
3629 2.82115261835969e-08
3630 2.81890591722345e-08
3631 2.8182806238064e-08
3632 2.82027010474195e-08
3633 2.81970808959642e-08
3634 2.81634714163204e-08
3635 2.81713540388751e-08
3636 2.81777203703371e-08
3637 2.81497894185634e-08
3638 2.81411058598024e-08
3639 2.81332882003937e-08
3640 2.81353381552663e-08
3641 2.8126256795602e-08
3642 2.81371170807176e-08
3643 2.81403343915798e-08
3644 2.81040223857332e-08
3645 2.81168049802716e-08
3646 2.81023712140893e-08
3647 2.80882286851636e-08
3648 2.80854555629873e-08
3649 2.81138815161341e-08
3650 2.81193557860604e-08
3651 2.8069522509e-08
3652 2.80638600713701e-08
3653 2.80598782307351e-08
3654 2.8103934811341e-08
3655 2.80409075834598e-08
3656 2.803171907928e-08
3657 2.80474489748883e-08
3658 2.80392403517737e-08
3659 2.8028339878361e-08
3660 2.80333434918845e-08
3661 2.80625149979841e-08
3662 2.80079953949297e-08
3663 2.80066135491808e-08
3664 2.80435344137686e-08
3665 2.79748910720912e-08
3666 2.79971063381623e-08
3667 2.80357045951973e-08
3668 2.79860644010199e-08
3669 2.79852697300242e-08
3670 2.79703339707282e-08
3671 2.80141919226651e-08
3672 2.79431114229567e-08
3673 2.79558549109993e-08
3674 2.79619327212544e-08
3675 2.79889395180533e-08
3676 2.79551812276679e-08
3677 2.79747244995576e-08
3678 2.79432234382426e-08
3679 2.79265982729981e-08
3680 2.79234288420582e-08
3681 2.79701146421729e-08
3682 2.79233585853689e-08
3683 2.79134397036529e-08
3684 2.79544529941944e-08
3685 2.79013928459193e-08
3686 2.78811392515621e-08
3687 2.78913587372642e-08
3688 2.79284348483344e-08
3689 2.78777187663337e-08
3690 2.7891109008138e-08
3691 2.79307694466269e-08
3692 2.78571407719141e-08
3693 2.78487151970097e-08
3694 2.78852399873131e-08
3695 2.78406989959734e-08
3696 2.78307847398906e-08
3697 2.78376000650837e-08
3698 2.78509133178062e-08
3699 2.78630931678236e-08
3700 2.78177801256163e-08
3701 2.78127150821206e-08
3702 2.78257029204809e-08
3703 2.78084223328534e-08
3704 2.7808694230913e-08
3705 2.78301350604693e-08
3706 2.77778621793345e-08
3707 2.77909962864697e-08
3708 2.78382621221596e-08
3709 2.77894463067696e-08
3710 2.77836892284e-08
3711 2.77523865026552e-08
3712 2.77670563217924e-08
3713 2.7783848853602e-08
3714 2.77600114824139e-08
3715 2.77686537071276e-08
3716 2.77371583425889e-08
3717 2.7758188160476e-08
3718 2.77242780644116e-08
3719 2.77626187408231e-08
3720 2.77250068787538e-08
3721 2.77207272496582e-08
3722 2.77076880603744e-08
3723 2.77331410085679e-08
3724 2.76998177159982e-08
3725 2.77209881094365e-08
3726 2.76864221984141e-08
3727 2.77036353093507e-08
3728 2.7681281581593e-08
3729 2.76946618686935e-08
3730 2.76816145774461e-08
3731 2.76852219620594e-08
3732 2.76546092887742e-08
3733 2.76862914727616e-08
3734 2.76727591383974e-08
3735 2.76164823596048e-08
3736 2.76308009432569e-08
3737 2.76441028628227e-08
3738 2.7653687601159e-08
3739 2.76090838564613e-08
3740 2.76355687738317e-08
3741 2.76170963413591e-08
3742 2.76363444005057e-08
3743 2.7576114344896e-08
3744 2.76296647712115e-08
3745 2.75635197244384e-08
3746 2.76011163364842e-08
3747 2.76054259664704e-08
3748 2.75629440693592e-08
3749 2.75815966457316e-08
3750 2.75651092209728e-08
3751 2.75815371431065e-08
3752 2.75482953888684e-08
3753 2.75333639887521e-08
3754 2.75644690450605e-08
3755 2.75459935092925e-08
3756 2.7558958837659e-08
3757 2.74959824935195e-08
3758 2.75107486853443e-08
3759 2.74947570471085e-08
3760 2.74971848206462e-08
3761 2.751030957171e-08
3762 2.74964879452e-08
3763 2.748798446639e-08
3764 2.74774031847613e-08
3765 2.74718010384589e-08
3766 2.74621194584768e-08
3767 2.74559114235018e-08
3768 2.74539536384566e-08
3769 2.74447386185983e-08
3770 2.74444520762529e-08
3771 2.74264385708989e-08
3772 2.74442770482608e-08
3773 2.74174087380175e-08
3774 2.74182120492128e-08
3775 2.74196304861363e-08
3776 2.74114250160551e-08
3777 2.74058766329688e-08
3778 2.74174400978211e-08
3779 2.74013560943587e-08
3780 2.7384415410836e-08
3781 2.73811285786962e-08
3782 2.73857213173301e-08
3783 2.73692665828662e-08
3784 2.7376701865478e-08
3785 2.73595425603901e-08
3786 2.73600836706578e-08
3787 2.73484104198474e-08
3788 2.73393730836347e-08
3789 2.73512563495615e-08
3790 2.73416266303172e-08
3791 2.73289954222378e-08
3792 2.73324786430607e-08
3793 2.73179644558752e-08
3794 2.73047499099732e-08
3795 2.72910815368732e-08
3796 2.73011270479628e-08
3797 2.72804265666338e-08
3798 2.72831040515342e-08
3799 2.72715220770436e-08
3800 2.72751158654216e-08
3801 2.72588055985068e-08
3802 2.72673922765421e-08
3803 2.72470041409179e-08
3804 2.72617983192447e-08
3805 2.72181665543769e-08
3806 2.7243751612005e-08
3807 2.72321206651327e-08
3808 2.72044324152176e-08
3809 2.7259422870074e-08
3810 2.72438397939112e-08
3811 2.72204671141196e-08
3812 2.72047638709694e-08
3813 2.7183858039237e-08
3814 2.71881989526435e-08
3815 2.71698902807316e-08
3816 2.71723847440342e-08
3817 2.71585407247699e-08
3818 2.71599456400651e-08
3819 2.71371580158331e-08
3820 2.71476270050641e-08
3821 2.71063391110715e-08
3822 2.71270522773648e-08
3823 2.71152997530777e-08
3824 2.71076138336213e-08
3825 2.71018345419094e-08
3826 2.70930511483414e-08
3827 2.70480223623082e-08
3828 2.71018834894221e-08
3829 2.70801244379726e-08
3830 2.70749477966348e-08
3831 2.70551747867387e-08
3832 2.70571711347145e-08
3833 2.70459468083573e-08
3834 2.70426679804814e-08
3835 2.70286666825825e-08
3836 2.70244467888148e-08
3837 2.69671459758314e-08
3838 2.70191259943431e-08
3839 2.69841398967685e-08
3840 2.69981746168213e-08
3841 2.6973683246112e-08
3842 2.69957210488059e-08
3843 2.69588080215044e-08
3844 2.69641523011899e-08
3845 2.69526936804709e-08
3846 2.69614385466355e-08
3847 2.69412010318604e-08
3848 2.69519243261129e-08
3849 2.69208332497328e-08
3850 2.69370041916517e-08
3851 2.69398374577179e-08
3852 2.6920632372196e-08
3853 2.69115162918609e-08
3854 2.69215331005768e-08
3855 2.69107542241187e-08
3856 2.69038697862101e-08
3857 2.69102666621279e-08
3858 2.68846006559187e-08
3859 2.68881270386601e-08
3860 2.68732587276332e-08
3861 2.68712477904387e-08
3862 2.68632593982687e-08
3863 2.68558909315431e-08
3864 2.68512111158969e-08
3865 2.68364261817311e-08
3866 2.68401865053391e-08
3867 2.6839332504025e-08
3868 2.68264424025944e-08
3869 2.68333113346841e-08
3870 2.68121013338174e-08
3871 2.680938669819e-08
3872 2.67999532965746e-08
3873 2.6799958087409e-08
3874 2.67927686632419e-08
3875 2.67905651334388e-08
3876 2.67813979739628e-08
3877 2.67771469513178e-08
3878 2.67650260283858e-08
3879 2.67674259983863e-08
3880 2.67634797683769e-08
3881 2.67539005065487e-08
3882 2.67436045220393e-08
3883 2.67462789835804e-08
3884 2.67299504859153e-08
3885 2.673323272262e-08
3886 2.67228823673804e-08
3887 2.67255658119581e-08
3888 2.67168635836867e-08
3889 2.6706554878686e-08
3890 2.66974550822141e-08
3891 2.67016835255873e-08
3892 2.66875617640494e-08
3893 2.6688668183894e-08
3894 2.66622142444106e-08
3895 2.67180297157665e-08
3896 2.66522845056016e-08
3897 2.66544689200288e-08
3898 2.66521788958585e-08
3899 2.66531839265838e-08
3900 2.66480117350198e-08
3901 2.66324280069341e-08
3902 2.66370121764226e-08
3903 2.66305288540991e-08
3904 2.66256784957619e-08
3905 2.65909270993347e-08
3906 2.66232726691129e-08
3907 2.66033547937639e-08
3908 2.66065536003168e-08
3909 2.66008101892368e-08
3910 2.65944667692253e-08
3911 2.65897219300371e-08
3912 2.65735650248899e-08
3913 2.65856419261468e-08
3914 2.65799110330533e-08
3915 2.65491782958094e-08
3916 2.65624003326081e-08
3917 2.65481537162771e-08
3918 2.65593259616281e-08
3919 2.65296036197071e-08
3920 2.65359278550648e-08
3921 2.65205129537094e-08
3922 2.65259265308515e-08
3923 2.6510725813722e-08
3924 2.6516258349929e-08
3925 2.64927528288439e-08
3926 2.65016266727258e-08
3927 2.64993007341729e-08
3928 2.6497980551099e-08
3929 2.64913372536313e-08
3930 2.64785234342924e-08
3931 2.6461081215956e-08
3932 2.64721096190357e-08
3933 2.64545128736415e-08
3934 2.64509312124517e-08
3935 2.64535586058656e-08
3936 2.64449047815418e-08
3937 2.64347427343381e-08
3938 2.64363115682897e-08
3939 2.6432000462151e-08
3940 2.64210551410571e-08
3941 2.64371955882581e-08
3942 2.64103496849799e-08
3943 2.64016930859867e-08
3944 2.6413660473068e-08
3945 2.63750526343642e-08
3946 2.63749705737837e-08
3947 2.63802159476967e-08
3948 2.63592483555897e-08
3949 2.63530002708734e-08
3950 2.6376160498387e-08
3951 2.63448647395137e-08
3952 2.6358290828199e-08
3953 2.63309282111379e-08
3954 2.63296884774888e-08
3955 2.63436769145642e-08
3956 2.63142629481905e-08
3957 2.63086489695752e-08
3958 2.63387747274635e-08
3959 2.62902817169675e-08
3960 2.62915436533007e-08
3961 2.63005313563269e-08
3962 2.62848410486072e-08
3963 2.6268476140956e-08
3964 2.63097721919792e-08
3965 2.62657921865639e-08
3966 2.6259941733997e-08
3967 2.62927592622475e-08
3968 2.62442016882858e-08
3969 2.62467871046113e-08
3970 2.62477776402648e-08
3971 2.62468843033048e-08
3972 2.62264079555763e-08
3973 2.62186958668309e-08
3974 2.62253418821246e-08
3975 2.62023047046256e-08
3976 2.62303393956387e-08
3977 2.62063540414914e-08
3978 2.61946488979703e-08
3979 2.6200828676437e-08
3980 2.61805904067103e-08
3981 2.61687687537204e-08
3982 2.62062486537928e-08
3983 2.61647406887278e-08
3984 2.61806744816795e-08
3985 2.61523151969811e-08
3986 2.61517284840807e-08
3987 2.62108637656411e-08
3988 2.6140648468953e-08
3989 2.61761519890769e-08
3990 2.61611716894095e-08
3991 2.61531011460647e-08
3992 2.61443141873485e-08
3993 2.6143780569754e-08
3994 2.61397210383763e-08
3995 2.61191269554928e-08
3996 2.61280858175894e-08
3997 2.60710432051781e-08
3998 2.61200252946736e-08
3999 2.61209382266259e-08
4000 2.60963915810208e-08
4001 2.60938795424437e-08
4002 2.60912948082392e-08
4003 2.60477789062463e-08
4004 2.60362395305691e-08
4005 2.60331254029467e-08
4006 2.60533123643825e-08
4007 2.60840048706967e-08
4008 2.60871243629168e-08
4009 2.60443169342039e-08
4010 2.60581244546643e-08
4011 2.60319748939253e-08
4012 2.60120861454993e-08
4013 2.60125964253177e-08
4014 2.60051192668698e-08
4015 2.60492563644021e-08
4016 2.60479829599092e-08
4017 2.60322216316666e-08
4018 2.5984807541235e-08
4019 2.59998945804085e-08
4020 2.6005131529061e-08
4021 2.59771586392787e-08
4022 2.59558629203127e-08
4023 2.5970592744784e-08
4024 2.59566327382998e-08
4025 2.59925421648433e-08
4026 2.59439160945618e-08
4027 2.5927900173528e-08
4028 2.59423821411531e-08
4029 2.59275221630162e-08
4030 2.59237465822792e-08
4031 2.59250477867567e-08
4032 2.59352458087392e-08
4033 2.59154629880243e-08
4034 2.59030975549024e-08
4035 2.58899492902032e-08
4036 2.59458464846318e-08
4037 2.58865152336085e-08
4038 2.58977520601888e-08
4039 2.59045796422441e-08
4040 2.58944448585652e-08
4041 2.58619083943756e-08
4042 2.58835575319694e-08
4043 2.58695772927808e-08
4044 2.58594870921058e-08
4045 2.58706071853965e-08
4046 2.58595276498852e-08
4047 2.58403480444258e-08
4048 2.58680399909395e-08
4049 2.5834097801436e-08
4050 2.58308066580781e-08
4051 2.58710581331201e-08
4052 2.58419553276212e-08
4053 2.57996955621564e-08
4054 2.58411076519138e-08
4055 2.58046328376338e-08
4056 2.58177608039034e-08
4057 2.5822129076758e-08
4058 2.58144150890871e-08
4059 2.57679942166078e-08
4060 2.5803904508237e-08
4061 2.57596279045202e-08
4062 2.57507435321713e-08
4063 2.57996992676368e-08
4064 2.57633430980064e-08
4065 2.57508773859882e-08
4066 2.57485749823871e-08
4067 2.57338102738203e-08
4068 2.57570893769099e-08
4069 2.57312843405799e-08
4070 2.57575366617857e-08
4071 2.57340061668998e-08
4072 2.57210405347763e-08
4073 2.57219370158879e-08
4074 2.57486886390268e-08
4075 2.56897825750002e-08
4076 2.5675190139296e-08
4077 2.57204419735757e-08
4078 2.57021945060387e-08
4079 2.56806490828154e-08
4080 2.5624152057091e-08
4081 2.56598714010181e-08
4082 2.56471842199346e-08
4083 2.56106366194331e-08
4084 2.56822088822162e-08
4085 2.56418346911147e-08
4086 2.56068226196504e-08
4087 2.5630410698696e-08
4088 2.56064539385648e-08
4089 2.55960173021208e-08
4090 2.56196927956864e-08
4091 2.56006300212164e-08
4092 2.5548005817555e-08
4093 2.5606735427175e-08
4094 2.55789563503583e-08
4095 2.55812647935727e-08
4096 2.55691778061617e-08
4097 2.55889999785097e-08
4098 2.55648142140075e-08
4099 2.55497552146267e-08
4100 2.55654589675913e-08
4101 2.55407575018296e-08
4102 2.55412392267118e-08
4103 2.55481346229658e-08
4104 2.55337595582716e-08
4105 2.55202382390962e-08
4106 2.55208846695609e-08
4107 2.55115830896102e-08
4108 2.54848694929422e-08
4109 2.54864908342256e-08
4110 2.5492029301688e-08
4111 2.54790770881641e-08
4112 2.54420145964218e-08
4113 2.54365995733252e-08
4114 2.54142320059714e-08
4115 2.53921112278732e-08
4116 2.54004922233264e-08
4117 2.53696250549496e-08
4118 2.53719028400212e-08
4119 2.53617266405826e-08
4120 2.53826659566414e-08
4121 2.5396280694423e-08
4122 2.5382326363399e-08
4123 2.53384274806479e-08
4124 2.53360880471121e-08
4125 2.5356686169431e-08
4126 2.53464360326916e-08
4127 2.53247703358994e-08
4128 2.53386084274609e-08
4129 2.53401314829205e-08
4130 2.53180693459143e-08
4131 2.53183755702935e-08
4132 2.53032174573065e-08
4133 2.5281421130785e-08
4134 2.53005586188237e-08
4135 2.52883560509787e-08
4136 2.52787017966938e-08
4137 2.52745435762591e-08
4138 2.52731637022663e-08
4139 2.52608470834303e-08
4140 2.5250335953686e-08
4141 2.52393119524186e-08
4142 2.52404360150393e-08
4143 2.52245481320301e-08
4144 2.52366231805468e-08
4145 2.52218185803343e-08
4146 2.52196899772628e-08
4147 2.52151285486946e-08
4148 2.52187482683297e-08
4149 2.52046216271395e-08
4150 2.51892440221724e-08
4151 2.51970920146505e-08
4152 2.51769648684785e-08
4153 2.51787404117465e-08
4154 2.51763838363672e-08
4155 2.51881412545174e-08
4156 2.51680231340146e-08
4157 2.51569313576283e-08
4158 2.51320979245406e-08
4159 2.51559708992488e-08
4160 2.5133737699079e-08
4161 2.5137312018586e-08
4162 2.51290175139474e-08
4163 2.51202007284235e-08
4164 2.51124972887595e-08
4165 2.51076404200745e-08
4166 2.51091170060391e-08
4167 2.50964866879144e-08
4168 2.5081242851499e-08
4169 2.50742771239487e-08
4170 2.50574200020282e-08
4171 2.50689528940029e-08
4172 2.50580115128685e-08
4173 2.50485766901676e-08
4174 2.50264540770928e-08
4175 2.50455182548848e-08
4176 2.50145814728597e-08
4177 2.50188757746628e-08
4178 2.49935735379125e-08
4179 2.49944142822756e-08
4180 2.49633307820574e-08
4181 2.49777545402452e-08
4182 2.49628666164625e-08
4183 2.4973963322239e-08
4184 2.49503546534413e-08
4185 2.49677036983087e-08
4186 2.49258438405064e-08
4187 2.49455196961179e-08
4188 2.49293815173246e-08
4189 2.49252634176855e-08
4190 2.49267687966181e-08
4191 2.49234909475149e-08
4192 2.48907189952519e-08
4193 2.48943058167583e-08
4194 2.48961270781223e-08
4195 2.48575121215566e-08
4196 2.48064869783349e-08
4197 2.48053547053928e-08
4198 2.480419910178e-08
4199 2.4822380296996e-08
4200 2.48154456325977e-08
4201 2.48024292623938e-08
4202 2.48089678258623e-08
4203 2.47990931345754e-08
4204 2.47840760767559e-08
4205 2.47682039677954e-08
4206 2.47403939095392e-08
4207 2.47522125125244e-08
4208 2.47711686416352e-08
4209 2.47463273286286e-08
4210 2.47497526260076e-08
4211 2.47362902516812e-08
4212 2.47569570070993e-08
4213 2.47391397536489e-08
4214 2.47241885737992e-08
4215 2.47077437460774e-08
4216 2.47146066811865e-08
4217 2.47099166124087e-08
4218 2.46941977071202e-08
4219 2.46983278682222e-08
4220 2.46720793803945e-08
4221 2.46671298569368e-08
4222 2.46744232814677e-08
4223 2.46391512987998e-08
4224 2.46422688157111e-08
4225 2.46454973140686e-08
4226 2.46131532186666e-08
4227 2.46194648738651e-08
4228 2.46204436908926e-08
4229 2.46012394740092e-08
4230 2.45887556982183e-08
4231 2.46019490024452e-08
4232 2.45597762127403e-08
4233 2.4571362857273e-08
4234 2.45777367577915e-08
4235 2.4554000995991e-08
4236 2.45508573417652e-08
4237 2.45530410651895e-08
4238 2.45322395198144e-08
4239 2.45309693251983e-08
4240 2.45294846106248e-08
4241 2.45035579879982e-08
4242 2.44999501664012e-08
4243 2.45135249876682e-08
4244 2.44855499342833e-08
4245 2.44844249515097e-08
4246 2.44921074390447e-08
4247 2.44671804914276e-08
4248 2.44727459044469e-08
4249 2.44759163248176e-08
4250 2.4446664388833e-08
4251 2.44524884660535e-08
4252 2.44637629283062e-08
4253 2.4441498563732e-08
4254 2.44245680320887e-08
4255 2.44348066633648e-08
4256 2.44139196396986e-08
4257 2.44154065054403e-08
4258 2.44176019119635e-08
4259 2.43846254619484e-08
4260 2.43967393913636e-08
4261 2.43967100050924e-08
4262 2.43677791544883e-08
4263 2.43717944208299e-08
4264 2.43863550579704e-08
4265 2.43543107192323e-08
4266 2.43610458543486e-08
4267 2.43573545386511e-08
4268 2.43548784339964e-08
4269 2.43287421213267e-08
4270 2.43318828001549e-08
4271 2.43323210682433e-08
4272 2.43085825495371e-08
4273 2.43187897321917e-08
4274 2.43171880836002e-08
4275 2.4309752570062e-08
4276 2.42827206093921e-08
4277 2.42946243673714e-08
4278 2.4295215167669e-08
4279 2.42618475283507e-08
4280 2.42689596454682e-08
4281 2.42706229371947e-08
4282 2.42487093249366e-08
4283 2.42447853580074e-08
4284 2.42517316344504e-08
4285 2.42478594003614e-08
4286 2.42076772352107e-08
4287 2.42254405122821e-08
4288 2.42253078788224e-08
4289 2.41963947278379e-08
4290 2.42006584230126e-08
4291 2.42052014751692e-08
4292 2.42027442070025e-08
4293 2.41705207919551e-08
4294 2.41768975950407e-08
4295 2.41882926452774e-08
4296 2.41498684765418e-08
4297 2.41650608483468e-08
4298 2.41599436812123e-08
4299 2.41308265405138e-08
4300 2.41338466775431e-08
4301 2.41286170208355e-08
4302 2.41451931586312e-08
4303 2.41437534054256e-08
4304 2.41534369980201e-08
4305 2.41502322708698e-08
4306 2.41624646797334e-08
4307 2.41355657273346e-08
4308 2.41332003838579e-08
4309 2.41314762376987e-08
4310 2.41130983464899e-08
4311 2.40959417379116e-08
4312 2.40986857544812e-08
4313 2.41148403379299e-08
4314 2.40779522400203e-08
4315 2.40740979933207e-08
4316 2.40778277138531e-08
4317 2.40657372145847e-08
4318 2.40482468605308e-08
4319 2.40640065616304e-08
4320 2.40891328520121e-08
4321 2.4024561007252e-08
4322 2.40361443886172e-08
4323 2.40319103941289e-08
4324 2.40185932831594e-08
4325 2.40158133735235e-08
4326 2.40117241059323e-08
4327 2.4007778712587e-08
4328 2.39886345454465e-08
4329 2.39980802803075e-08
4330 2.3995488238171e-08
4331 2.39920826707873e-08
4332 2.3967051168583e-08
4333 2.3974960594586e-08
4334 2.39800211261354e-08
4335 2.39482541477543e-08
4336 2.39550138587674e-08
4337 2.3954802479409e-08
4338 2.39442060934181e-08
4339 2.39285290115632e-08
4340 2.39408976661082e-08
4341 2.39318114090281e-08
4342 2.39266408836869e-08
4343 2.39086380595666e-08
4344 2.39043672136674e-08
4345 2.39074661232408e-08
4346 2.38574179931561e-08
4347 2.38640388143807e-08
4348 2.39089505047474e-08
4349 2.38974481892029e-08
4350 2.38939486774115e-08
4351 2.38878225067296e-08
4352 2.38873813218632e-08
4353 2.38758039214915e-08
4354 2.3873969117183e-08
4355 2.38574777950973e-08
4356 2.38484168182396e-08
4357 2.38097344569255e-08
4358 2.38541165860084e-08
4359 2.38132778944689e-08
4360 2.38573847459733e-08
4361 2.3805500705798e-08
4362 2.384285813406e-08
4363 2.38302804120849e-08
4364 2.37918590899611e-08
4365 2.3824617934487e-08
4366 2.37955472499607e-08
4367 2.38003716521717e-08
4368 2.37713933737282e-08
4369 2.38123540992063e-08
4370 2.37953373911637e-08
4371 2.37896991501785e-08
4372 2.37832671095006e-08
4373 2.37747756921536e-08
4374 2.37744017974606e-08
4375 2.37753974845489e-08
4376 2.3745857259172e-08
4377 2.36974863785377e-08
4378 2.37503568998321e-08
4379 2.37242189715658e-08
4380 2.36947711806934e-08
4381 2.37527982704933e-08
4382 2.37347800178256e-08
4383 2.37100647195732e-08
4384 2.37261999860294e-08
4385 2.37200520576408e-08
4386 2.37164755745312e-08
4387 2.37197554522339e-08
4388 2.36934161454982e-08
4389 2.36980509784601e-08
4390 2.36671128632793e-08
4391 2.36701415259333e-08
4392 2.37107016491933e-08
4393 2.3655244225651e-08
4394 2.36955259929061e-08
4395 2.3664805157253e-08
4396 2.36656618630704e-08
4397 2.3663374098426e-08
4398 2.36629776404484e-08
4399 2.36523784309384e-08
4400 2.3655010840784e-08
4401 2.36445987535916e-08
4402 2.36245869329466e-08
4403 2.3632513465266e-08
4404 2.36307854102336e-08
4405 2.35934711936281e-08
4406 2.35961988410693e-08
4407 2.35899917369053e-08
4408 2.3625703566843e-08
4409 2.36114157594614e-08
4410 2.35865442537175e-08
4411 2.35979709275469e-08
4412 2.35906647612083e-08
4413 2.35936357855238e-08
4414 2.35854934649282e-08
4415 2.3585304308682e-08
4416 2.35744661019055e-08
4417 2.35768911949208e-08
4418 2.35602333553686e-08
4419 2.35461501363332e-08
4420 2.35500427665514e-08
4421 2.35467163287595e-08
4422 2.3547840016569e-08
4423 2.3543639366963e-08
4424 2.35378396427066e-08
4425 2.35338134801921e-08
4426 2.3530439695385e-08
4427 2.35259604544069e-08
4428 2.35232315448641e-08
4429 2.35122051579495e-08
4430 2.35141921223203e-08
4431 2.35074547383363e-08
4432 2.34849720532893e-08
4433 2.34958976736976e-08
4434 2.34849202493947e-08
4435 2.34920714943243e-08
4436 2.34832516543548e-08
4437 2.34765360387357e-08
4438 2.34914277177367e-08
4439 2.34670770016265e-08
4440 2.34666982610321e-08
4441 2.3465105988052e-08
4442 2.34652486019726e-08
4443 2.3437085151734e-08
4444 2.34508431162794e-08
4445 2.34489239874236e-08
4446 2.34492767452465e-08
4447 2.34498074886957e-08
4448 2.34344440768197e-08
4449 2.34335825739507e-08
4450 2.34395679710175e-08
4451 2.3426080282718e-08
4452 2.34222603907597e-08
4453 2.33947821266511e-08
4454 2.34044958604684e-08
4455 2.34100311891083e-08
4456 2.34040674227387e-08
4457 2.34002135766076e-08
4458 2.33996613667742e-08
4459 2.33906207194323e-08
4460 2.33779740668183e-08
4461 2.33589766098774e-08
4462 2.3370521557986e-08
4463 2.33692673372587e-08
4464 2.33682190824425e-08
4465 2.33643559779395e-08
4466 2.33637610458359e-08
4467 2.33527765098884e-08
4468 2.33691378310752e-08
4469 2.33387672157193e-08
4470 2.33417816088988e-08
4471 2.33359514991704e-08
4472 2.3332100011153e-08
4473 2.33041152766233e-08
4474 2.33391380444203e-08
4475 2.33186232465599e-08
4476 2.33397838718119e-08
4477 2.33047861364355e-08
4478 2.33027969676058e-08
4479 2.33041198081096e-08
4480 2.33036753956029e-08
4481 2.32960639525714e-08
4482 2.32853903776586e-08
4483 2.3289393809911e-08
4484 2.32774230104482e-08
4485 2.32808068831858e-08
4486 2.32882278972113e-08
4487 2.32698092270667e-08
4488 2.32691759221026e-08
4489 2.32546186182958e-08
4490 2.32639009150049e-08
4491 2.32435837963507e-08
4492 2.32483299198449e-08
4493 2.32358267808763e-08
4494 2.32402723341352e-08
4495 2.32314026638036e-08
4496 2.32299203739572e-08
4497 2.3222362913522e-08
4498 2.32215507915967e-08
4499 2.32269631039728e-08
4500 2.32044861085967e-08
4501 2.32091386331845e-08
4502 2.320150971169e-08
4503 2.32027531872347e-08
4504 2.31977178453135e-08
4505 2.31840494526736e-08
4506 2.31799164795987e-08
4507 2.31876794360986e-08
4508 2.3174407649762e-08
4509 2.31714579177478e-08
4510 2.31702437689663e-08
4511 2.31674074493426e-08
4512 2.31574104274657e-08
4513 2.31416326927913e-08
4514 2.31477013814541e-08
4515 2.31492304179071e-08
4516 2.3137155805486e-08
4517 2.31484072275023e-08
4518 2.31294407129212e-08
4519 2.31331964322123e-08
4520 2.31288382321893e-08
4521 2.31368169965052e-08
4522 2.31743762357794e-08
4523 2.31673651436282e-08
4524 2.3165136290082e-08
4525 2.31589823496137e-08
4526 2.31437590034744e-08
4527 2.31134153221291e-08
4528 2.31213869037461e-08
4529 2.31151375640337e-08
4530 2.31051961785411e-08
4531 2.31019758087925e-08
4532 2.30939060488211e-08
4533 2.30832461127761e-08
4534 2.30810818599991e-08
4535 2.30914099557111e-08
4536 2.30937537812892e-08
4537 2.30696411200881e-08
4538 2.30661363298523e-08
4539 2.30816114177301e-08
4540 2.30666409963831e-08
4541 2.30615431986436e-08
4542 2.3065953450363e-08
4543 2.30883442888796e-08
4544 2.3015575190577e-08
4545 2.30450567295648e-08
4546 2.30357424912953e-08
4547 2.30380841204081e-08
4548 2.30313977365171e-08
4549 2.30275576775796e-08
4550 2.3016911816498e-08
4551 2.30773636671344e-08
4552 2.29626961756324e-08
4553 2.30033470431934e-08
4554 2.30007704900004e-08
4555 2.30302587338471e-08
4556 2.29581993798078e-08
4557 2.29750609976875e-08
4558 2.29702880121252e-08
4559 2.29812332559476e-08
4560 2.29650359839795e-08
4561 2.29728793179618e-08
4562 2.29805209750467e-08
4563 2.29398945581494e-08
4564 2.29576057346748e-08
4565 2.29485030898147e-08
4566 2.2942474156018e-08
4567 2.2949896581359e-08
4568 2.29260150019783e-08
4569 2.29549307748655e-08
4570 2.29333683234856e-08
4571 2.29173673913508e-08
4572 2.2917316567117e-08
4573 2.29411791776712e-08
4574 2.29102610882492e-08
4575 2.29162105114256e-08
4576 2.28971144453638e-08
4577 2.28740601473376e-08
4578 2.28998866154129e-08
4579 2.28993046533787e-08
4580 2.28783258906518e-08
4581 2.28966213189352e-08
4582 2.28815778529068e-08
4583 2.28622224796027e-08
4584 2.28733879055198e-08
4585 2.28498726260185e-08
4586 2.28471713414891e-08
4587 2.28572258276216e-08
4588 2.2852818193364e-08
4589 2.28345791297713e-08
4590 2.28437036975393e-08
4591 2.28459093687405e-08
4592 2.28262248027633e-08
4593 2.28425660289133e-08
4594 2.28758871978485e-08
4595 2.28590419455443e-08
4596 2.27824149572342e-08
4597 2.28488726774501e-08
4598 2.28497586709508e-08
4599 2.28741956931344e-08
4600 2.28434169251557e-08
4601 2.28528183994214e-08
4602 2.28453234578652e-08
4603 2.28397785679846e-08
4604 2.28446444250352e-08
4605 2.28408507911837e-08
4606 2.28367473225077e-08
4607 2.28351771953683e-08
4608 2.28329125206983e-08
4609 2.28289041679375e-08
4610 2.28278602465437e-08
4611 2.28284053793715e-08
4612 2.2801772016301e-08
4613 2.28284988912364e-08
4614 2.28167227076526e-08
4615 2.2827853806362e-08
4616 2.28009209513047e-08
4617 2.2791081739193e-08
4618 2.27925237252791e-08
4619 2.27833062087512e-08
4620 2.27730172879248e-08
4621 2.27652674853474e-08
4622 2.27758698674307e-08
4623 2.27611597409805e-08
4624 2.27376988402028e-08
4625 2.27397947094588e-08
4626 2.27507052974474e-08
4627 2.27443968441321e-08
4628 2.27397736178858e-08
4629 2.27366662342021e-08
4630 2.2732190960717e-08
4631 2.27133270502833e-08
4632 2.27172235556239e-08
4633 2.27003045942809e-08
4634 2.2704261874118e-08
4635 2.26903084525887e-08
4636 2.26938537757349e-08
4637 2.26798408018425e-08
4638 2.26599164916408e-08
4639 2.26822670947868e-08
4640 2.26729771313217e-08
4641 2.26638975986404e-08
4642 2.26635577185164e-08
4643 2.26539179326579e-08
4644 2.26579659132753e-08
4645 2.26502136806417e-08
4646 2.26529059590419e-08
4647 2.26265500034017e-08
4648 2.26299991794576e-08
4649 2.26250057133726e-08
4650 2.26120803352714e-08
4651 2.26133606320289e-08
4652 2.26177804556471e-08
4653 2.26105178500191e-08
4654 2.26072802407273e-08
4655 2.25950432097832e-08
4656 2.25928337718173e-08
4657 2.25854100293077e-08
4658 2.25866818235332e-08
4659 2.25838714289495e-08
4660 2.25752235065713e-08
4661 2.2576378925443e-08
4662 2.25377966787121e-08
4663 2.25640704458741e-08
4664 2.25337496537747e-08
4665 2.252574659245e-08
4666 2.25038724321891e-08
4667 2.25217491420793e-08
4668 2.24803589299327e-08
4669 2.25041201700193e-08
4670 2.25075274267184e-08
4671 2.24716865444563e-08
4672 2.25378248428498e-08
4673 2.24469832676988e-08
4674 2.24733317280013e-08
4675 2.24406940612099e-08
4676 2.2437253133667e-08
4677 2.24878017860419e-08
4678 2.24978732124015e-08
4679 2.24261745236376e-08
4680 2.24085394480156e-08
4681 2.24594741800033e-08
4682 2.24496237875016e-08
4683 2.24370517951655e-08
4684 2.24031158611027e-08
4685 2.23986531766585e-08
4686 2.23951051880888e-08
4687 2.24401820139164e-08
4688 2.24305286566917e-08
4689 2.23831309424938e-08
4690 2.24160492043879e-08
4691 2.23824636247372e-08
4692 2.23885403034529e-08
4693 2.24233665813145e-08
4694 2.23537327537215e-08
4695 2.23533161500811e-08
4696 2.23559205094759e-08
4697 2.23163444319496e-08
4698 2.232898277299e-08
4699 2.23753944581517e-08
4700 2.23952731390753e-08
4701 2.23620581136785e-08
4702 2.23147635445287e-08
4703 2.22950471266969e-08
4704 2.23666343917017e-08
4705 2.23439948570103e-08
4706 2.23304228113008e-08
4707 2.22947603862877e-08
4708 2.23408532580294e-08
4709 2.23233536607026e-08
4710 2.23034489881258e-08
4711 2.22614402058952e-08
4712 2.22397950055964e-08
4713 2.2237650951773e-08
4714 2.22473138817847e-08
4715 2.22454824951868e-08
4716 2.22460114978062e-08
4717 2.22460742982378e-08
4718 2.22440506876254e-08
4719 2.22350406913208e-08
4720 2.22381744139355e-08
4721 2.22278521118113e-08
4722 2.22368266404871e-08
4723 2.22347887728347e-08
4724 2.22190669729727e-08
4725 2.22130461375869e-08
4726 2.22088099288698e-08
4727 2.22054643490566e-08
4728 2.22007848851291e-08
4729 2.21933065143176e-08
4730 2.21879446771212e-08
4731 2.21847764541039e-08
4732 2.21748334423566e-08
4733 2.21695406521505e-08
4734 2.21683331069755e-08
4735 2.21668652748974e-08
4736 2.21847107537698e-08
4737 2.21505204569539e-08
4738 2.21464295471208e-08
4739 2.21427672766339e-08
4740 2.21423396302711e-08
4741 2.21348383906417e-08
4742 2.2131145670734e-08
4743 2.21206885679948e-08
4744 2.21233171062707e-08
4745 2.21202395893627e-08
4746 2.2116922939297e-08
4747 2.21039111805155e-08
4748 2.21056082816418e-08
4749 2.20964606141294e-08
4750 2.20942888464748e-08
4751 2.20922840306059e-08
4752 2.20853592702852e-08
4753 2.20749346970806e-08
4754 2.20659161795922e-08
4755 2.20687985859058e-08
4756 2.20639679957557e-08
4757 2.20727213848804e-08
4758 2.20521973020027e-08
4759 2.20409980284941e-08
4760 2.20472624521406e-08
4761 2.20403494530785e-08
4762 2.20324749147238e-08
4763 2.20285903864692e-08
4764 2.20205568934873e-08
4765 2.20195148434854e-08
4766 2.20160136032987e-08
4767 2.20064181659652e-08
4768 2.20033353164339e-08
4769 2.20027399278067e-08
4770 2.19901671698608e-08
4771 2.19894386841446e-08
4772 2.19796662968008e-08
4773 2.19815847746219e-08
4774 2.19801950729348e-08
4775 2.1967856874916e-08
4776 2.19553542839535e-08
4777 2.19650781261294e-08
4778 2.19634437472749e-08
4779 2.19442785054369e-08
4780 2.19439196778026e-08
4781 2.19426985879068e-08
4782 2.19129962601983e-08
4783 2.19566495482937e-08
4784 2.19300841655112e-08
4785 2.19279814785978e-08
4786 2.19200662323615e-08
4787 2.19173990059218e-08
4788 2.19126714053885e-08
4789 2.18957401001418e-08
4790 2.19039383893005e-08
4791 2.18937145683995e-08
4792 2.18674894716386e-08
4793 2.18912076830335e-08
4794 2.18842889010418e-08
4795 2.18783981038584e-08
4796 2.18715049422613e-08
4797 2.18719289284408e-08
4798 2.1864034144059e-08
4799 2.18478170506131e-08
4800 2.18534800122683e-08
4801 2.18672363798689e-08
4802 2.18441503649913e-08
4803 2.18373330236332e-08
4804 2.18303847727697e-08
4805 2.18254133423912e-08
4806 2.18217139025256e-08
4807 2.18206677358168e-08
4808 2.18126239346361e-08
4809 2.18170610510882e-08
4810 2.18037554438055e-08
4811 2.17983194481519e-08
4812 2.17963092321583e-08
4813 2.17918741967082e-08
4814 2.180400183871e-08
4815 2.17837595535286e-08
4816 2.17809754241571e-08
4817 2.17778197679763e-08
4818 2.17730163427632e-08
4819 2.17621517064615e-08
4820 2.17618406423981e-08
4821 2.17534122999297e-08
4822 2.17548265188938e-08
4823 2.17460319582585e-08
4824 2.17543730682834e-08
4825 2.17340083938922e-08
4826 2.17341403327964e-08
4827 2.17251437870658e-08
4828 2.17255138412753e-08
4829 2.17193374041358e-08
4830 2.17171161747132e-08
4831 2.17082931026624e-08
4832 2.17049647988077e-08
4833 2.17529655079929e-08
4834 2.17333511383089e-08
4835 2.1706491935447e-08
4836 2.17101759307781e-08
4837 2.16934300327765e-08
4838 2.17059411822262e-08
4839 2.16760215758782e-08
4840 2.16978998182071e-08
4841 2.16628350591108e-08
4842 2.1669399496993e-08
4843 2.16634110046243e-08
4844 2.16231421399371e-08
4845 2.1639193898082e-08
4846 2.1644919208974e-08
4847 2.16851750902691e-08
4848 2.16440213050006e-08
4849 2.16588235746329e-08
4850 2.16467229421369e-08
4851 2.16441820608537e-08
4852 2.16165101827315e-08
4853 2.16152491239185e-08
4854 2.1633073396643e-08
4855 2.16249544546798e-08
4856 2.15665665814768e-08
4857 2.15896318236375e-08
4858 2.15626397714885e-08
4859 2.15588113574583e-08
4860 2.15622962489448e-08
4861 2.16077079864263e-08
4862 2.15733996533629e-08
4863 2.15439775024251e-08
4864 2.15557434639635e-08
4865 2.15936805361139e-08
4866 2.15907255061154e-08
4867 2.16205478764664e-08
4868 2.16229249208055e-08
4869 2.15865107229263e-08
4870 2.16573033666734e-08
4871 2.1645691560046e-08
4872 2.16394985859125e-08
4873 2.15402574568202e-08
4874 2.15885958638751e-08
4875 2.15434973451778e-08
4876 2.15766408366846e-08
4877 2.15541770787198e-08
4878 2.15677728290231e-08
4879 2.15403929670899e-08
4880 2.15596638764026e-08
4881 2.15246100223609e-08
4882 2.15430004262274e-08
4883 2.15182544129888e-08
4884 2.15445953610782e-08
4885 2.15043809141946e-08
4886 2.15434477119913e-08
4887 2.14943892098418e-08
4888 2.15227428510545e-08
4889 2.1484310335218e-08
4890 2.15091929325339e-08
4891 2.14339722770518e-08
4892 2.14973007013342e-08
4893 2.14802841780326e-08
4894 2.14885291667244e-08
4895 2.14572247623224e-08
4896 2.14742138915369e-08
4897 2.14348154008448e-08
4898 2.14051595568421e-08
4899 2.14802857101404e-08
4900 2.14309553765446e-08
4901 2.13865736622409e-08
4902 2.13892166796015e-08
4903 2.13926850793911e-08
4904 2.13945008109562e-08
4905 2.13903122681103e-08
4906 2.13863933264946e-08
4907 2.13925554906069e-08
4908 2.14035682137848e-08
4909 2.14291242919273e-08
4910 2.13779254281121e-08
4911 2.13393475512191e-08
4912 2.13410824816407e-08
4913 2.13458619091611e-08
4914 2.13343062327453e-08
4915 2.13291332640253e-08
4916 2.13357616933862e-08
4917 2.13306686092096e-08
4918 2.13276843235022e-08
4919 2.13217121673992e-08
4920 2.13142615530515e-08
4921 2.13096324390705e-08
4922 2.13042754442228e-08
4923 2.13033292544296e-08
4924 2.12891414896887e-08
4925 2.12944102226942e-08
4926 2.12811972231108e-08
4927 2.1277742775716e-08
4928 2.12844291924696e-08
4929 2.12952974187885e-08
4930 2.12587544989873e-08
4931 2.12546832445426e-08
4932 2.12566543087433e-08
4933 2.12502684160398e-08
4934 2.12428399191111e-08
4935 2.1231166515534e-08
4936 2.12354011175364e-08
4937 2.12320125427823e-08
4938 2.12503024563659e-08
4939 2.12157342005526e-08
4940 2.12162028967455e-08
4941 2.12080345391996e-08
4942 2.12008552908927e-08
4943 2.11997791002005e-08
4944 2.11926738620249e-08
4945 2.11908802798533e-08
4946 2.12000602077822e-08
4947 2.11727087400249e-08
4948 2.11792500568464e-08
4949 2.11968552878972e-08
4950 2.11619455487977e-08
4951 2.11787798614083e-08
4952 2.11510870098408e-08
4953 2.11448550127002e-08
4954 2.11453269125528e-08
4955 2.114527110475e-08
4956 2.11619277941111e-08
4957 2.11306034403336e-08
4958 2.11339507130148e-08
4959 2.11221945676243e-08
4960 2.11111836101452e-08
4961 2.11149804671962e-08
4962 2.1103642602327e-08
4963 2.11021738181216e-08
4964 2.10969550700213e-08
4965 2.10953447616902e-08
4966 2.10899621597704e-08
4967 2.11006666352986e-08
4968 2.10770435478125e-08
4969 2.10763439971728e-08
4970 2.10690782846967e-08
4971 2.10649875365121e-08
4972 2.1059825543901e-08
4973 2.10583682900278e-08
4974 2.10485568716834e-08
4975 2.10521048744639e-08
4976 2.10423759234857e-08
4977 2.10349421854161e-08
4978 2.10493278265389e-08
4979 2.10203984778445e-08
4980 2.10173188301965e-08
4981 2.10190275522493e-08
4982 2.10126252335741e-08
4983 2.10091935937129e-08
4984 2.10104440263592e-08
4985 2.10007125627243e-08
4986 2.09899150567239e-08
4987 2.1006961993919e-08
4988 2.09835371007827e-08
4989 2.0985018951869e-08
4990 2.0979330408899e-08
4991 2.09681232465897e-08
4992 2.09713806231804e-08
4993 2.09628946929996e-08
4994 2.09560033264111e-08
4995 2.0983173427247e-08
4996 2.0980758495881e-08
4997 2.09824553554228e-08
4998 2.09686583110269e-08
4999 2.09778803776928e-08
};
\addlegendentry{Train}
\addplot [semithick, black]
table {%
0 0.00879009626805782
1 0.00259806332178414
2 0.00132246734574437
3 0.000764447031542659
4 0.000361335638444871
5 0.000223734779865481
6 0.0001946582342498
7 0.000186475284863263
8 0.00018190588161815
9 0.000177760535734706
10 0.000173088075825945
11 0.000167389312991872
12 0.00016019721806515
13 0.000150978463352658
14 0.000139155017677695
15 0.000124497004435398
16 0.000107436324469745
17 8.93068790901452e-05
18 7.19462550478056e-05
19 5.73328252357896e-05
20 4.66494093416259e-05
21 3.97209623770323e-05
22 3.55129195668269e-05
23 3.29227150359657e-05
24 3.10922514472622e-05
25 2.92677486868342e-05
26 2.67961841018405e-05
27 2.34794588322984e-05
28 2.0802894141525e-05
29 1.82010626303963e-05
30 1.57492049766006e-05
31 1.3507353287423e-05
32 1.15488801384345e-05
33 9.91529395832913e-06
34 8.59739247971447e-06
35 7.56222607378731e-06
36 6.77195566822775e-06
37 6.16964643995743e-06
38 5.70556494494667e-06
39 5.33995171281276e-06
40 5.0391258810123e-06
41 4.78195715913898e-06
42 4.55251074527041e-06
43 4.34287949246936e-06
44 4.14018040828523e-06
45 3.93901746065239e-06
46 3.75004356101272e-06
47 3.56886175723048e-06
48 3.39298776452779e-06
49 3.22296705235203e-06
50 3.06307128994376e-06
51 2.91069977720326e-06
52 2.76485275207961e-06
53 2.62891330748971e-06
54 2.49938784691039e-06
55 2.3833008526708e-06
56 2.27962414101057e-06
57 2.18545915231516e-06
58 2.10013399737363e-06
59 2.01840612135129e-06
60 1.92549282473919e-06
61 1.81850714398024e-06
62 1.70594387327583e-06
63 1.59490082296543e-06
64 1.50384903463419e-06
65 1.42871556363389e-06
66 1.36508560899529e-06
67 1.3088628065816e-06
68 1.25837732412037e-06
69 1.21599475733092e-06
70 1.18001958071545e-06
71 1.14895135538973e-06
72 1.12208601876773e-06
73 1.09918153157196e-06
74 1.07761036360898e-06
75 1.05867843558372e-06
76 1.0417608109492e-06
77 1.02640399290976e-06
78 1.01286536846601e-06
79 1.00001409464312e-06
80 9.87824250842095e-07
81 9.76483420345176e-07
82 9.65413505582546e-07
83 9.55938958213665e-07
84 9.43363602345926e-07
85 9.32325519897859e-07
86 9.22968808936275e-07
87 9.12366317606939e-07
88 9.03169620869448e-07
89 8.92521200057672e-07
90 8.83719565081265e-07
91 8.75202317729418e-07
92 8.68671236275986e-07
93 8.60114312217775e-07
94 8.51822562708549e-07
95 8.44168482672103e-07
96 8.36109734336787e-07
97 8.28085603643558e-07
98 8.20074660623504e-07
99 8.13941596788936e-07
100 8.06331229341595e-07
101 7.99028782694222e-07
102 7.91461332028121e-07
103 7.84079645654856e-07
104 7.77742513946578e-07
105 7.70678752815002e-07
106 7.63725950037042e-07
107 7.56707720483973e-07
108 7.49879745853832e-07
109 7.42823601740383e-07
110 7.35903483928269e-07
111 7.2827850772228e-07
112 7.20960088074207e-07
113 7.13537474439363e-07
114 7.06506455117051e-07
115 6.99633631029428e-07
116 6.9237836441971e-07
117 6.85465749938885e-07
118 6.78620665439666e-07
119 6.72096291509661e-07
120 6.66016205741471e-07
121 6.59910938338726e-07
122 6.53199549560668e-07
123 6.46446551400004e-07
124 6.39643417343905e-07
125 6.32503088127123e-07
126 6.2512134491044e-07
127 6.18554906850477e-07
128 6.12034057212441e-07
129 6.05344553150644e-07
130 5.98897543113708e-07
131 5.92767776197434e-07
132 5.86994985951605e-07
133 5.811731398353e-07
134 5.75714409478678e-07
135 5.70401141430921e-07
136 5.65225661830482e-07
137 5.60172168206918e-07
138 5.55251574496651e-07
139 5.50631796158996e-07
140 5.46020942238101e-07
141 5.41608187631937e-07
142 5.37377388809546e-07
143 5.33303079919278e-07
144 5.2955766705054e-07
145 5.25821064911725e-07
146 5.22127095337055e-07
147 5.18448473485478e-07
148 5.15096360231837e-07
149 5.11862538132846e-07
150 5.08678169808263e-07
151 5.05662455907441e-07
152 5.0285723318666e-07
153 5.00124144764413e-07
154 4.97448013447865e-07
155 4.94909670578636e-07
156 4.92438232413406e-07
157 4.89856120111654e-07
158 4.87594661535695e-07
159 4.8536446684011e-07
160 4.8455615342391e-07
161 4.8178930001086e-07
162 4.7901295374686e-07
163 4.76548194683346e-07
164 4.74265533512153e-07
165 4.7208609998961e-07
166 4.69997814889211e-07
167 4.6801244479866e-07
168 4.66129591814024e-07
169 4.64328735461095e-07
170 4.62470836737339e-07
171 4.60718041495056e-07
172 4.59034765754041e-07
173 4.57426409639083e-07
174 4.55459854720175e-07
175 4.53826743296304e-07
176 4.52174163001473e-07
177 4.50747904778837e-07
178 4.4924581743544e-07
179 4.4779775976167e-07
180 4.4644900754065e-07
181 4.4507680740935e-07
182 4.43740958644412e-07
183 4.4244120545045e-07
184 4.41182464783196e-07
185 4.39794291651197e-07
186 4.38452786966081e-07
187 4.37126686847478e-07
188 4.35825029398984e-07
189 4.34578424801657e-07
190 4.33207731020957e-07
191 4.31901554520664e-07
192 4.30647475013757e-07
193 4.29393821832491e-07
194 4.27673768399472e-07
195 4.26629185312777e-07
196 4.25440305207303e-07
197 4.24143422605994e-07
198 4.22832840740739e-07
199 4.21010440732061e-07
200 4.20581443449919e-07
201 4.19173943555506e-07
202 4.18214455066845e-07
203 4.17754762338518e-07
204 4.16233234545871e-07
205 4.15462807268341e-07
206 4.14477284493842e-07
207 4.14119796232626e-07
208 4.12501123037146e-07
209 4.11678144018879e-07
210 4.10887253110559e-07
211 4.1013072404894e-07
212 4.09275259016795e-07
213 4.08535186124936e-07
214 4.07263456736473e-07
215 4.0585899796497e-07
216 4.05018482751984e-07
217 4.04266103259943e-07
218 4.03571846163686e-07
219 4.02827595280542e-07
220 4.01913979430901e-07
221 4.01246296632962e-07
222 4.0045401306088e-07
223 3.9966212739273e-07
224 3.98867456397056e-07
225 3.98258748646185e-07
226 3.97463537638032e-07
227 3.96584596273897e-07
228 3.95832330468693e-07
229 3.94987523577583e-07
230 3.9425816567018e-07
231 3.93509793639168e-07
232 3.92842480323452e-07
233 3.91923322240473e-07
234 3.91236483210378e-07
235 3.90444029108039e-07
236 3.89474337225693e-07
237 3.88786162375254e-07
238 3.85923101475782e-07
239 3.84994933710914e-07
240 3.83083744281976e-07
241 3.81821166683949e-07
242 3.80439303171443e-07
243 3.7917800455034e-07
244 3.76943006585861e-07
245 3.75412128050812e-07
246 3.74009516690421e-07
247 3.72236343082477e-07
248 3.70694436924168e-07
249 3.68981744713892e-07
250 3.67365117881491e-07
251 3.65552466519148e-07
252 3.6155006455374e-07
253 3.61999525466672e-07
254 3.57748916712808e-07
255 3.55843582156012e-07
256 3.50868475607058e-07
257 3.49305707914027e-07
258 3.47636927244821e-07
259 3.45204909990571e-07
260 3.43455724305386e-07
261 3.41964351946444e-07
262 3.40570494472558e-07
263 3.39326135190277e-07
264 3.38158741897132e-07
265 3.36830652258868e-07
266 3.35541272988849e-07
267 3.33178803657574e-07
268 3.32106310452218e-07
269 3.31135566966623e-07
270 3.29852127833874e-07
271 3.28768720692096e-07
272 3.28030921536993e-07
273 3.27019904489134e-07
274 3.25894234265434e-07
275 3.24874321222524e-07
276 3.23904146171117e-07
277 3.22899921911812e-07
278 3.21991251439613e-07
279 3.2109022640725e-07
280 3.20501669648365e-07
281 3.19411611826581e-07
282 3.18673613719511e-07
283 3.1748109563523e-07
284 3.16403742317561e-07
285 3.15358590796677e-07
286 3.14244971377775e-07
287 3.13197773493812e-07
288 3.12159613713447e-07
289 3.11754746462611e-07
290 3.10908859546544e-07
291 3.09760821437521e-07
292 3.11762676119542e-07
293 3.1038709380482e-07
294 3.08874092525002e-07
295 3.07867452420396e-07
296 3.06692783169638e-07
297 3.05824812585342e-07
298 3.04805240602946e-07
299 3.03775806287376e-07
300 3.02794461504163e-07
301 3.0181163879206e-07
302 3.00850899748184e-07
303 2.98723193736805e-07
304 2.98752723892903e-07
305 2.97751000744029e-07
306 2.96769997021329e-07
307 2.95718564302661e-07
308 2.94715249538058e-07
309 2.93442951715406e-07
310 2.92452085659534e-07
311 2.91460565904345e-07
312 2.90712932837778e-07
313 2.90040730988039e-07
314 2.88954367988481e-07
315 2.87689914557632e-07
316 2.86605995825084e-07
317 2.85751752926444e-07
318 2.85076708905763e-07
319 2.84223659718918e-07
320 2.83383030819095e-07
321 2.82578326959992e-07
322 2.8151663400422e-07
323 2.80690471754497e-07
324 2.80041092537431e-07
325 2.79274871672897e-07
326 2.7826575887957e-07
327 2.77471656318085e-07
328 2.76703985946369e-07
329 2.76018568001746e-07
330 2.75526730320053e-07
331 2.74849639936292e-07
332 2.74066991323707e-07
333 2.7265824087408e-07
334 2.7183142492504e-07
335 2.70850279093793e-07
336 2.70094119514397e-07
337 2.69346912773472e-07
338 2.68641770162503e-07
339 2.67735742909281e-07
340 2.6703122557592e-07
341 2.66505566060005e-07
342 2.65956117573296e-07
343 2.65591552306432e-07
344 2.6504878292144e-07
345 2.64589942844395e-07
346 2.64151168494209e-07
347 2.6377037443126e-07
348 2.63325375726708e-07
349 2.628368065416e-07
350 2.62330530631516e-07
351 2.61600064277445e-07
352 2.61177859783857e-07
353 2.60772736737636e-07
354 2.60400923934867e-07
355 2.59820609471717e-07
356 2.59275907410483e-07
357 2.58738992897634e-07
358 2.58381334106161e-07
359 2.57959470673086e-07
360 2.57583678830997e-07
361 2.57138339065932e-07
362 2.56828968758782e-07
363 2.56390279673724e-07
364 2.56036884138666e-07
365 2.5564500560904e-07
366 2.5533910275044e-07
367 2.54932302823363e-07
368 2.54496768548051e-07
369 2.53773379199629e-07
370 2.53254768267652e-07
371 2.52856125371181e-07
372 2.52463166816597e-07
373 2.52156070246201e-07
374 2.51847950494266e-07
375 2.51548840424221e-07
376 2.5142063009298e-07
377 2.51013375418552e-07
378 2.50503859433593e-07
379 2.5003646442201e-07
380 2.4954746891126e-07
381 2.49176650868321e-07
382 2.48806543368119e-07
383 2.48453147833061e-07
384 2.47779126993919e-07
385 2.471763025369e-07
386 2.4665635578458e-07
387 2.46169747697422e-07
388 2.45669440346319e-07
389 2.45211253968591e-07
390 2.44780068214823e-07
391 2.44338536958821e-07
392 2.43919259901304e-07
393 2.43522464415946e-07
394 2.43072662442501e-07
395 2.42782078885284e-07
396 2.42419446294662e-07
397 2.42013783235961e-07
398 2.42625873170255e-07
399 2.42235017822168e-07
400 2.41869457795474e-07
401 2.4150591571015e-07
402 2.41111536070093e-07
403 2.40748477153829e-07
404 2.40424782305126e-07
405 2.40097193682232e-07
406 2.39724641915018e-07
407 2.40461616840548e-07
408 2.4007280785554e-07
409 2.40142867369286e-07
410 2.39705599369699e-07
411 2.39272281987724e-07
412 2.38892454262896e-07
413 2.38529622720307e-07
414 2.37530599633828e-07
415 2.3588363262661e-07
416 2.35445824614544e-07
417 2.35063978948347e-07
418 2.3474171939597e-07
419 2.34356747341735e-07
420 2.34571629675884e-07
421 2.34214766692276e-07
422 2.34056656722714e-07
423 2.33716747288781e-07
424 2.33313102171451e-07
425 2.32963913049389e-07
426 2.32645405162657e-07
427 2.32144643064203e-07
428 2.31839052844407e-07
429 2.31512530035616e-07
430 2.3120045966607e-07
431 2.30902458042692e-07
432 2.30625204267199e-07
433 2.30376116405751e-07
434 2.30094173048201e-07
435 2.29823143627073e-07
436 2.29547566732435e-07
437 2.29273993568313e-07
438 2.29099626380957e-07
439 2.28803997970317e-07
440 2.28569277282986e-07
441 2.28296045179377e-07
442 2.28025058390813e-07
443 2.27761063342768e-07
444 2.27491526061385e-07
445 2.27057654456075e-07
446 2.27393201157611e-07
447 2.27168797550803e-07
448 2.26928776214663e-07
449 2.2670167254546e-07
450 2.26468728214968e-07
451 2.2623787288012e-07
452 2.26001063197145e-07
453 2.25738816084231e-07
454 2.25478927973199e-07
455 2.25234217055004e-07
456 2.25000931663999e-07
457 2.24764008294187e-07
458 2.24531177650533e-07
459 2.2430130286466e-07
460 2.24068571696989e-07
461 2.2384500653061e-07
462 2.23610456373535e-07
463 2.23356437345501e-07
464 2.23090964368566e-07
465 2.22832170493348e-07
466 2.22995538479154e-07
467 2.22743182121121e-07
468 2.22485951439921e-07
469 2.22231577140519e-07
470 2.21983881942833e-07
471 2.21737607830619e-07
472 2.21494744323536e-07
473 2.21251895027308e-07
474 2.20967720565568e-07
475 2.20705828724022e-07
476 2.20454452914964e-07
477 2.20201897604966e-07
478 2.19953122382321e-07
479 2.19716852711827e-07
480 2.19485698949029e-07
481 2.19250949839989e-07
482 2.19028422066003e-07
483 2.18796174067393e-07
484 2.18573688925972e-07
485 2.18335287627269e-07
486 2.18089127201893e-07
487 2.1782453529795e-07
488 2.17581586525739e-07
489 2.17329898077878e-07
490 2.17059692886323e-07
491 2.16801666397259e-07
492 2.16371958572381e-07
493 2.16094008465006e-07
494 2.15808299230957e-07
495 2.15511860801598e-07
496 2.15218818766516e-07
497 2.14945387710941e-07
498 2.14671672438271e-07
499 2.14420850852548e-07
500 2.14168480283661e-07
501 2.13911960145197e-07
502 2.13665117598794e-07
503 2.13418715588887e-07
504 2.13174843111119e-07
505 2.12928355836084e-07
506 2.12679140076943e-07
507 2.12713189284841e-07
508 2.12616214412265e-07
509 2.1245924131108e-07
510 2.1226692581422e-07
511 2.12056818327255e-07
512 2.11788020010317e-07
513 2.11557704687948e-07
514 2.11118916126907e-07
515 2.10869004035885e-07
516 2.10623213092731e-07
517 2.10350435736473e-07
518 2.10072442996534e-07
519 2.09803900474981e-07
520 2.09540118589757e-07
521 2.09279178875477e-07
522 2.09021507657781e-07
523 2.08792926059687e-07
524 2.08509746357777e-07
525 2.08237338483741e-07
526 2.0797220656732e-07
527 2.07706861488077e-07
528 2.07453169309701e-07
529 2.07190183232342e-07
530 2.0694201907645e-07
531 2.06669568569851e-07
532 2.06416601145065e-07
533 2.06179478823287e-07
534 2.05920827056616e-07
535 2.0568934644416e-07
536 2.05439548039976e-07
537 2.05304047540267e-07
538 2.04966383421379e-07
539 2.0466421801757e-07
540 2.04386552127289e-07
541 2.04117526436676e-07
542 2.03911767471254e-07
543 2.03635224238496e-07
544 2.0339527395663e-07
545 2.03113160068824e-07
546 2.02879391508759e-07
547 2.026240935038e-07
548 2.02400116222634e-07
549 2.02155518991276e-07
550 2.01932877530453e-07
551 2.01697019974745e-07
552 2.0148667090325e-07
553 2.01238762542744e-07
554 2.01014088929696e-07
555 2.00781187231769e-07
556 2.00534259420238e-07
557 2.0030159930684e-07
558 2.00074651957038e-07
559 1.9986202914879e-07
560 1.99628686914366e-07
561 1.9940893025705e-07
562 1.99190083094436e-07
563 1.98969956954897e-07
564 1.98755799374339e-07
565 1.98519700234101e-07
566 1.9831119857372e-07
567 1.98099996850942e-07
568 1.97885256625341e-07
569 1.9766277148392e-07
570 1.97394143697238e-07
571 1.97132806079026e-07
572 1.9689282737545e-07
573 1.96665823182229e-07
574 1.96441945377046e-07
575 1.96179840372679e-07
576 1.95856557638763e-07
577 1.95594523688669e-07
578 1.95298298422131e-07
579 1.95004545844313e-07
580 1.94709528500425e-07
581 1.94423066091076e-07
582 1.94085757243556e-07
583 1.93821222183033e-07
584 1.9359406167041e-07
585 1.93323103303555e-07
586 1.93060344599871e-07
587 1.92718061953201e-07
588 1.92528773368394e-07
589 1.92008840826929e-07
590 1.91871166066448e-07
591 1.91524691217637e-07
592 1.91385083780915e-07
593 1.91044833286469e-07
594 1.90811462630336e-07
595 1.9068353651619e-07
596 1.90399020993937e-07
597 1.90169188840628e-07
598 1.89980113418642e-07
599 1.89877681577855e-07
600 1.89611981227245e-07
601 1.89398306815747e-07
602 1.89215796808639e-07
603 1.89024078167677e-07
604 1.88854599514343e-07
605 1.88655377542091e-07
606 1.88593972438866e-07
607 1.88266653822211e-07
608 1.8808485435784e-07
609 1.87890478287045e-07
610 1.87222539693721e-07
611 1.87052904720986e-07
612 1.86934897783431e-07
613 1.86798686740985e-07
614 1.86624859566109e-07
615 1.86467786988942e-07
616 1.86323333650762e-07
617 1.86153314984949e-07
618 1.86005166824543e-07
619 1.85836071864287e-07
620 1.85672746511045e-07
621 1.85516000783537e-07
622 1.85354338100296e-07
623 1.85193286483809e-07
624 1.84995769814122e-07
625 1.84802260605466e-07
626 1.84618471621434e-07
627 1.84413792680971e-07
628 1.84202548325629e-07
629 1.84013799753302e-07
630 1.83825292765505e-07
631 1.83629865091461e-07
632 1.83485411753281e-07
633 1.83299150080529e-07
634 1.83157183641924e-07
635 1.82986568120214e-07
636 1.82835279360916e-07
637 1.82672579285281e-07
638 1.82496009415445e-07
639 1.82321500119542e-07
640 1.82173323537427e-07
641 1.82024521677704e-07
642 1.81955513767207e-07
643 1.81783164521221e-07
644 1.81627456186106e-07
645 1.81437329160872e-07
646 1.81271218480106e-07
647 1.81106457830538e-07
648 1.80955737505428e-07
649 1.80798949145355e-07
650 1.80645344016739e-07
651 1.80500222768387e-07
652 1.80390230752892e-07
653 1.80243915792744e-07
654 1.80093735480114e-07
655 1.79944009914834e-07
656 1.79798931299047e-07
657 1.79649191522913e-07
658 1.79444910486382e-07
659 1.79314540105224e-07
660 1.79160565494385e-07
661 1.79000963385079e-07
662 1.78850356746807e-07
663 1.78697916908277e-07
664 1.78553747787191e-07
665 1.78401180050969e-07
666 1.78255774585523e-07
667 1.78113396032131e-07
668 1.77922700572708e-07
669 1.77996597017227e-07
670 1.77661462430478e-07
671 1.77541750190358e-07
672 1.77422535330152e-07
673 1.77293912884124e-07
674 1.77161112446811e-07
675 1.76964363163279e-07
676 1.76879851210288e-07
677 1.76739447965701e-07
678 1.76597680479063e-07
679 1.76395559492448e-07
680 1.7626099690915e-07
681 1.76238941662632e-07
682 1.76125340090039e-07
683 1.75919367961797e-07
684 1.75760305864969e-07
685 1.75553580561427e-07
686 1.75429761384294e-07
687 1.75246611888724e-07
688 1.75264162294297e-07
689 1.75105142830034e-07
690 1.7497285398349e-07
691 1.74797776253399e-07
692 1.7458117440583e-07
693 1.74536083363819e-07
694 1.74308468103845e-07
695 1.74153285570355e-07
696 1.74200351921172e-07
697 1.74050967416406e-07
698 1.73852640728001e-07
699 1.73696207639296e-07
700 1.73523716284762e-07
701 1.73364441025115e-07
702 1.73204654174697e-07
703 1.73049016893856e-07
704 1.72891759575577e-07
705 1.72739291315338e-07
706 1.72628460859414e-07
707 1.72467537140619e-07
708 1.72434184264603e-07
709 1.72290611999415e-07
710 1.72093024275455e-07
711 1.71980573782093e-07
712 1.71832724049636e-07
713 1.71661540093737e-07
714 1.7151820941308e-07
715 1.71342165344868e-07
716 1.71181540054022e-07
717 1.71029995499339e-07
718 1.70865064319514e-07
719 1.70708830182775e-07
720 1.70495908946577e-07
721 1.70355988871052e-07
722 1.70202838489786e-07
723 1.70048622294416e-07
724 1.69910066460943e-07
725 1.68744435313783e-07
726 1.6858619744653e-07
727 1.68426453228676e-07
728 1.68288877944178e-07
729 1.68140829259755e-07
730 1.67953757568284e-07
731 1.67808124729163e-07
732 1.67651265314817e-07
733 1.6748566622482e-07
734 1.67388307659166e-07
735 1.6723569729038e-07
736 1.6707777206193e-07
737 1.66928245448617e-07
738 1.66727559758328e-07
739 1.66400582202186e-07
740 1.66096413067862e-07
741 1.65823706765877e-07
742 1.65448213351738e-07
743 1.65359296033785e-07
744 1.65169154797695e-07
745 1.65031082133282e-07
746 1.64903582344778e-07
747 1.64683470416094e-07
748 1.6441796901745e-07
749 1.64292131898947e-07
750 1.641063391844e-07
751 1.63939176900385e-07
752 1.63758443250117e-07
753 1.63613151471509e-07
754 1.63490923910103e-07
755 1.63775709438596e-07
756 1.62799707936756e-07
757 1.63553323773158e-07
758 1.62538015047176e-07
759 1.63102754413558e-07
760 1.6301132177432e-07
761 1.62433380523908e-07
762 1.62714954399235e-07
763 1.62080070253978e-07
764 1.62381141421974e-07
765 1.61808912935157e-07
766 1.62074854870298e-07
767 1.61647463414738e-07
768 1.60981770136459e-07
769 1.61582875080057e-07
770 1.60687335437615e-07
771 1.61288895128564e-07
772 1.60364407975067e-07
773 1.60970969886876e-07
774 1.60047662234319e-07
775 1.60688756523086e-07
776 1.59715767722446e-07
777 1.60390499104324e-07
778 1.5946484666074e-07
779 1.60066875309894e-07
780 1.5917392204301e-07
781 1.59789252052178e-07
782 1.59644216068955e-07
783 1.59507806074544e-07
784 1.59380050490654e-07
785 1.59232953933497e-07
786 1.59097837126865e-07
787 1.5899752270343e-07
788 1.58815197437434e-07
789 1.58733456601112e-07
790 1.58583574716431e-07
791 1.58506082925669e-07
792 1.58377048364855e-07
793 1.58237043024201e-07
794 1.58039270559129e-07
795 1.57862515948182e-07
796 1.57784825205454e-07
797 1.57641167675138e-07
798 1.57472328510266e-07
799 1.57323611915672e-07
800 1.57186974547585e-07
801 1.57035998427091e-07
802 1.56863947609054e-07
803 1.56706136067442e-07
804 1.5654082119454e-07
805 1.5635266947811e-07
806 1.56228480818754e-07
807 1.56075032009539e-07
808 1.55921483724342e-07
809 1.55750711883229e-07
810 1.55586505456995e-07
811 1.55426278070081e-07
812 1.55270200252744e-07
813 1.55113653477201e-07
814 1.54952033426525e-07
815 1.54794065565511e-07
816 1.54831084842044e-07
817 1.54509237404454e-07
818 1.5423550792093e-07
819 1.53985041606575e-07
820 1.53773086708497e-07
821 1.53541492409204e-07
822 1.53345951048323e-07
823 1.53100430111408e-07
824 1.53131153979302e-07
825 1.53050976336999e-07
826 1.52843441014738e-07
827 1.52297928934786e-07
828 1.51770322531775e-07
829 1.51634026224201e-07
830 1.51486815980206e-07
831 1.51187379060502e-07
832 1.51076307020048e-07
833 1.5096172489848e-07
834 1.50784174479668e-07
835 1.50702135215397e-07
836 1.50499928963654e-07
837 1.50383669961229e-07
838 1.5032125588732e-07
839 1.50146078681246e-07
840 1.49928837345215e-07
841 1.49811853589199e-07
842 1.49678740513082e-07
843 1.4970251527302e-07
844 1.49411178540504e-07
845 1.4928934888303e-07
846 1.49185652276174e-07
847 1.49226565326899e-07
848 1.48952736367391e-07
849 1.48834942592657e-07
850 1.4871105236125e-07
851 1.48669897725995e-07
852 1.48619534456884e-07
853 1.48482101280933e-07
854 1.47908991721124e-07
855 1.47663030247713e-07
856 1.47573402387025e-07
857 1.47134173289487e-07
858 1.47028941910321e-07
859 1.46825357205671e-07
860 1.46547776580519e-07
861 1.46204968132224e-07
862 1.46116448718203e-07
863 1.45984500932173e-07
864 1.46011657875533e-07
865 1.45878999546767e-07
866 1.4571641315797e-07
867 1.457470943933e-07
868 1.45600253631528e-07
869 1.45359479120089e-07
870 1.45541136475913e-07
871 1.4604549392061e-07
872 1.45811526408579e-07
873 1.45873315204881e-07
874 1.4590480645893e-07
875 1.45816898111661e-07
876 1.45817040220209e-07
877 1.4565929973287e-07
878 1.45529540418465e-07
879 1.45183079780509e-07
880 1.45066948675776e-07
881 1.45054471545336e-07
882 1.45092897696486e-07
883 1.44988888450825e-07
884 1.44814919167402e-07
885 1.4471120834969e-07
886 1.44699470183696e-07
887 1.44893419928849e-07
888 1.44795407663878e-07
889 1.44715855299182e-07
890 1.44653398592709e-07
891 1.44510039490342e-07
892 1.44429549209235e-07
893 1.4429083705636e-07
894 1.44146042657667e-07
895 1.44157368708875e-07
896 1.44095068321803e-07
897 1.43961116805258e-07
898 1.43894951065704e-07
899 1.43769582905406e-07
900 1.43338439784202e-07
901 1.43298876764675e-07
902 1.43345829428654e-07
903 1.43257111062667e-07
904 1.43154750276153e-07
905 1.4303192585885e-07
906 1.42956196214072e-07
907 1.42808062264521e-07
908 1.42685536275167e-07
909 1.42592156748833e-07
910 1.42508937983621e-07
911 1.42312657658294e-07
912 1.42170534900288e-07
913 1.42054730645214e-07
914 1.41957002597337e-07
915 1.41868028435965e-07
916 1.41809564979667e-07
917 1.41699743494428e-07
918 1.41595194236288e-07
919 1.41449078228106e-07
920 1.41122129093674e-07
921 1.41005997988941e-07
922 1.40858489316997e-07
923 1.4081395249832e-07
924 1.4066398534851e-07
925 1.40580397101076e-07
926 1.40473261467378e-07
927 1.40291049888219e-07
928 1.40371028578556e-07
929 1.40218702426864e-07
930 1.40046367391733e-07
931 1.3988460523251e-07
932 1.39765759854527e-07
933 1.39603514526243e-07
934 1.39679613653243e-07
935 1.39577679192371e-07
936 1.39503484319903e-07
937 1.39382152042344e-07
938 1.3929277997704e-07
939 1.39175384106238e-07
940 1.39102482421549e-07
941 1.38927063630945e-07
942 1.38659714821188e-07
943 1.34510827365375e-07
944 1.34744993829372e-07
945 1.34905704385346e-07
946 1.34919233119035e-07
947 1.34869026169326e-07
948 1.34759773118276e-07
949 1.34608697521799e-07
950 1.3442301849409e-07
951 1.34220101699611e-07
952 1.33996067575026e-07
953 1.33768239152232e-07
954 1.33525460910278e-07
955 1.33265913859759e-07
956 1.32998692947695e-07
957 1.32664723651033e-07
958 1.32385622464426e-07
959 1.32121741103219e-07
960 1.31878238107674e-07
961 1.31652001300608e-07
962 1.31438255834837e-07
963 1.31211280063326e-07
964 1.30989619151478e-07
965 1.30783988083749e-07
966 1.30555648070185e-07
967 1.30348766447241e-07
968 1.30128000819241e-07
969 1.29967460793523e-07
970 1.2976977359358e-07
971 1.29577244933898e-07
972 1.294464482271e-07
973 1.29269196236237e-07
974 1.2909389113247e-07
975 1.28912816421689e-07
976 1.28754223283067e-07
977 1.28562788859199e-07
978 1.28396905552108e-07
979 1.28206778526874e-07
980 1.28037470403797e-07
981 1.2786439640422e-07
982 1.27674084637874e-07
983 1.27496306845387e-07
984 1.2730430398733e-07
985 1.27136388528015e-07
986 1.26963044522199e-07
987 1.26757313978487e-07
988 1.26565922187183e-07
989 1.26377557307933e-07
990 1.26193029359456e-07
991 1.26001879152682e-07
992 1.25809464179838e-07
993 1.25635978065475e-07
994 1.25443563092631e-07
995 1.25276415019471e-07
996 1.25096974556982e-07
997 1.24931005984763e-07
998 1.24749405472357e-07
999 1.24581646332445e-07
1000 1.24411840829453e-07
1001 1.24230595588415e-07
1002 1.23985628874834e-07
1003 1.23809741126024e-07
1004 1.23616530345316e-07
1005 1.23448643307711e-07
1006 1.23256214124012e-07
1007 1.23093172987865e-07
1008 1.22900431165363e-07
1009 1.22741013797167e-07
1010 1.2255175363407e-07
1011 1.22393629453654e-07
1012 1.22203971386625e-07
1013 1.22047907780143e-07
1014 1.21863010349443e-07
1015 1.21707131484072e-07
1016 1.21516734452598e-07
1017 1.213357592178e-07
1018 1.21140445230594e-07
1019 1.20996404007201e-07
1020 1.20792066127251e-07
1021 1.2063395615769e-07
1022 1.20454842544859e-07
1023 1.20311767659587e-07
1024 1.20193348607245e-07
1025 1.20073735843107e-07
1026 1.19958301070255e-07
1027 1.19836997214406e-07
1028 1.19614469440421e-07
1029 1.20054195917874e-07
1030 1.19933261544247e-07
1031 1.19820313670971e-07
1032 1.19709909540688e-07
1033 1.19592613145869e-07
1034 1.19467216563862e-07
1035 1.19364884199058e-07
1036 1.1920958797873e-07
1037 1.19183305002935e-07
1038 1.19124592856679e-07
1039 1.1904916163985e-07
1040 1.18956833716766e-07
1041 1.18854650565936e-07
1042 1.18745326460612e-07
1043 1.18629614576093e-07
1044 1.18506207513747e-07
1045 1.18381002778278e-07
1046 1.18279317007364e-07
1047 1.18124155790156e-07
1048 1.1799184562733e-07
1049 1.17854582981636e-07
1050 1.17716282943547e-07
1051 1.17582949599182e-07
1052 1.17445111413872e-07
1053 1.17308935898564e-07
1054 1.17175595448771e-07
1055 1.17037238567264e-07
1056 1.16901844648964e-07
1057 1.16766628366349e-07
1058 1.16633856350745e-07
1059 1.1649654396706e-07
1060 1.16363487734361e-07
1061 1.16231667846023e-07
1062 1.16128603622201e-07
1063 1.15977023540381e-07
1064 1.15881022111353e-07
1065 1.15793568511435e-07
1066 1.15599306127478e-07
1067 1.15521473276203e-07
1068 1.15403793188307e-07
1069 1.15227713592958e-07
1070 1.15095879493765e-07
1071 1.14963185637862e-07
1072 1.14832864994696e-07
1073 1.14701698805675e-07
1074 1.14570887888021e-07
1075 1.14477188617457e-07
1076 1.14311056620409e-07
1077 1.14188267730242e-07
1078 1.14059645284215e-07
1079 1.13912705046459e-07
1080 1.1378680397911e-07
1081 1.13672392387798e-07
1082 1.1357450802052e-07
1083 1.13517572231103e-07
1084 1.13361302567228e-07
1085 1.13213502572762e-07
1086 1.13070370844071e-07
1087 1.1294302026954e-07
1088 1.12814369401804e-07
1089 1.12700007548483e-07
1090 1.12534202401093e-07
1091 1.12410823760456e-07
1092 1.12275763797243e-07
1093 1.1217547069009e-07
1094 1.12041554700681e-07
1095 1.11879934650005e-07
1096 1.11728375884468e-07
1097 1.11622156850899e-07
1098 1.11465325858262e-07
1099 1.11338906094716e-07
1100 1.11212848707964e-07
1101 1.11099375033064e-07
1102 1.10971406286353e-07
1103 1.10848233703109e-07
1104 1.10723021862213e-07
1105 1.10598229241532e-07
1106 1.10473621361962e-07
1107 1.10355038884791e-07
1108 1.10230210736972e-07
1109 1.10107940543003e-07
1110 1.09964119587858e-07
1111 1.09874832787682e-07
1112 1.09722918750776e-07
1113 1.09619321619903e-07
1114 1.09489228350412e-07
1115 1.09385389635008e-07
1116 1.092538113312e-07
1117 1.09151379490413e-07
1118 1.09022188610197e-07
1119 1.08909354423758e-07
1120 1.08775353169221e-07
1121 1.08665304310307e-07
1122 1.08533150466883e-07
1123 1.08418952038392e-07
1124 1.08289462730227e-07
1125 1.08180600477681e-07
1126 1.08052823577509e-07
1127 1.07964154949514e-07
1128 1.07841174212808e-07
1129 1.07728801879148e-07
1130 1.07606631161161e-07
1131 1.07188483866594e-07
1132 1.07022763984332e-07
1133 1.06884215256287e-07
1134 1.06746405492686e-07
1135 1.06626671936283e-07
1136 1.06500230856454e-07
1137 1.06406751854138e-07
1138 1.06269723687547e-07
1139 1.061585237494e-07
1140 1.0605945988118e-07
1141 1.05944096162602e-07
1142 1.05836711838947e-07
1143 1.05728162225205e-07
1144 1.05610745038121e-07
1145 1.0550361650985e-07
1146 1.05389055704563e-07
1147 1.05269748473802e-07
1148 1.05160886221256e-07
1149 1.05038168385363e-07
1150 1.04919713805884e-07
1151 1.04801614497774e-07
1152 1.04684829693724e-07
1153 1.04570794690062e-07
1154 1.04459360272813e-07
1155 1.04347932960991e-07
1156 1.04235553521903e-07
1157 1.04136454126547e-07
1158 1.04027321867761e-07
1159 1.03919795435559e-07
1160 1.03809547624678e-07
1161 1.03700109832516e-07
1162 1.03589989919328e-07
1163 1.03481639257552e-07
1164 1.03371071702441e-07
1165 1.03261342587757e-07
1166 1.03150966879184e-07
1167 1.03041784882407e-07
1168 1.02935381107727e-07
1169 1.02827790726678e-07
1170 1.02720058237082e-07
1171 1.02614023944625e-07
1172 1.0250787596533e-07
1173 1.02402701429583e-07
1174 1.02296901616228e-07
1175 1.02192423412362e-07
1176 1.02086637809862e-07
1177 1.0198131406014e-07
1178 1.01874718438921e-07
1179 1.01769273896934e-07
1180 1.01662330109775e-07
1181 1.01555201581505e-07
1182 1.01449089129346e-07
1183 1.0134547068219e-07
1184 1.01245390737859e-07
1185 1.0114938930883e-07
1186 1.01055746881684e-07
1187 1.00960527049665e-07
1188 1.00866195396065e-07
1189 1.00771266886568e-07
1190 1.0067596178942e-07
1191 1.00566673211233e-07
1192 1.00441411632346e-07
1193 1.00328854557574e-07
1194 1.00232824706836e-07
1195 1.00126079871643e-07
1196 1.0002705153056e-07
1197 9.99302613990949e-08
1198 9.98347360336993e-08
1199 9.97326097262885e-08
1200 9.96346329884545e-08
1201 9.95379778601091e-08
1202 9.94416211597127e-08
1203 9.93442341723494e-08
1204 9.92488367046462e-08
1205 9.91605659805828e-08
1206 9.9052300583935e-08
1207 9.8944845206006e-08
1208 9.88227384368656e-08
1209 9.87127606322247e-08
1210 9.85913999329568e-08
1211 9.84842500884042e-08
1212 9.83796368814183e-08
1213 9.82668169058343e-08
1214 9.81627152896181e-08
1215 9.80577539166916e-08
1216 9.79488916641458e-08
1217 9.78471277335302e-08
1218 9.77414913450048e-08
1219 9.7640537433108e-08
1220 9.75143592540917e-08
1221 9.7417718336601e-08
1222 9.73065894527281e-08
1223 9.72051026337795e-08
1224 9.70929789900765e-08
1225 9.69919469184788e-08
1226 9.6882708078283e-08
1227 9.67830686704474e-08
1228 9.66857740536398e-08
1229 9.65989741530393e-08
1230 9.64963788874229e-08
1231 9.64079944765217e-08
1232 9.63066213444108e-08
1233 9.62332649123709e-08
1234 9.61296677814971e-08
1235 9.60598214305719e-08
1236 9.59576382797422e-08
1237 9.58727923716651e-08
1238 9.57889838559822e-08
1239 9.56930463757999e-08
1240 9.55922203615955e-08
1241 9.55063299556969e-08
1242 9.54228696059545e-08
1243 9.5326988969191e-08
1244 9.52336733917036e-08
1245 9.51518330793988e-08
1246 9.50633349816599e-08
1247 9.49702467778479e-08
1248 9.48903817743485e-08
1249 9.48184180060707e-08
1250 9.47368477000055e-08
1251 9.46515683608595e-08
1252 9.45757605563813e-08
1253 9.44892732945846e-08
1254 9.44112983347623e-08
1255 9.43264382158304e-08
1256 9.42426297001475e-08
1257 9.41627860129302e-08
1258 9.40823596806695e-08
1259 9.39960003165652e-08
1260 9.39177979830674e-08
1261 9.38370163794389e-08
1262 9.37522628419174e-08
1263 9.3668802492175e-08
1264 9.35843402771752e-08
1265 9.34991177814481e-08
1266 9.34166024535443e-08
1267 9.33351813614536e-08
1268 9.32536181608157e-08
1269 9.3173390780521e-08
1270 9.30736163695656e-08
1271 9.29951298189735e-08
1272 9.29104331248709e-08
1273 9.284224233852e-08
1274 9.27547674223206e-08
1275 9.26763732422842e-08
1276 9.26005014889597e-08
1277 9.25271024243557e-08
1278 9.24516143641085e-08
1279 9.23705414379583e-08
1280 9.22909322298437e-08
1281 9.22128435831837e-08
1282 9.21348757287888e-08
1283 9.20564957596071e-08
1284 9.19833951229521e-08
1285 9.19071254656956e-08
1286 9.18343161515622e-08
1287 9.17589844107169e-08
1288 9.1686281677994e-08
1289 9.16147655516397e-08
1290 9.15413806978904e-08
1291 9.14601159252015e-08
1292 9.1390560896798e-08
1293 9.13196416263418e-08
1294 9.12508539840928e-08
1295 9.11810289494497e-08
1296 9.11114668156188e-08
1297 9.10430557610198e-08
1298 9.09739412691124e-08
1299 9.09054449493851e-08
1300 9.08388884113265e-08
1301 9.07685233642042e-08
1302 9.06970569758414e-08
1303 9.06176538251202e-08
1304 9.05394372807677e-08
1305 9.04631747289386e-08
1306 9.03863366374935e-08
1307 9.03121843975896e-08
1308 9.02371724009754e-08
1309 9.01582097867504e-08
1310 9.00862104913358e-08
1311 9.00106513768151e-08
1312 8.9938225755759e-08
1313 8.98636685064957e-08
1314 8.97915910513802e-08
1315 8.97165008950651e-08
1316 8.96429099839224e-08
1317 8.95740512873999e-08
1318 8.95029756975418e-08
1319 8.94285392405436e-08
1320 8.9375056688823e-08
1321 8.93066953722155e-08
1322 8.92343123837236e-08
1323 8.91604798880508e-08
1324 8.90804727760042e-08
1325 8.90050984025947e-08
1326 8.89286226879449e-08
1327 8.88430093937131e-08
1328 8.8764679162523e-08
1329 8.86874360617185e-08
1330 8.86110171904875e-08
1331 8.85355504465224e-08
1332 8.84599913320017e-08
1333 8.83855904021402e-08
1334 8.83115660599287e-08
1335 8.82368951238277e-08
1336 8.81665869201242e-08
1337 8.80933725966315e-08
1338 8.80195969443776e-08
1339 8.79468373682357e-08
1340 8.78787247415858e-08
1341 8.78061499065552e-08
1342 8.77396644227701e-08
1343 8.76707062502646e-08
1344 8.76065371357981e-08
1345 8.7538658988251e-08
1346 8.74708945275415e-08
1347 8.74012684448644e-08
1348 8.73330350259494e-08
1349 8.72643397542561e-08
1350 8.71945502467497e-08
1351 8.70833076760391e-08
1352 8.7005872728696e-08
1353 8.69267040570776e-08
1354 8.68498304384957e-08
1355 8.6776438479319e-08
1356 8.67033733698008e-08
1357 8.66307345859241e-08
1358 8.65578968500813e-08
1359 8.64903810793294e-08
1360 8.64203570927202e-08
1361 8.63503615278205e-08
1362 8.62823839042903e-08
1363 8.62134825752037e-08
1364 8.61451141531688e-08
1365 8.6075971239552e-08
1366 8.6005393029609e-08
1367 8.59386162233022e-08
1368 8.5872763122552e-08
1369 8.58026467653872e-08
1370 8.57369926166029e-08
1371 8.56298427720503e-08
1372 8.55540989164183e-08
1373 8.54821138318584e-08
1374 8.54160049357233e-08
1375 8.533832840385e-08
1376 8.52693062824983e-08
1377 8.52016128760624e-08
1378 8.51347721209095e-08
1379 8.50625383463921e-08
1380 8.49971186767107e-08
1381 8.49616696996236e-08
1382 8.48898764616024e-08
1383 8.48227657002099e-08
1384 8.47554488814239e-08
1385 8.4687059143107e-08
1386 8.46209644578266e-08
1387 8.45563619122913e-08
1388 8.45396144200095e-08
1389 8.4473100514515e-08
1390 8.44090450868862e-08
1391 8.43347933709993e-08
1392 8.42595468952823e-08
1393 8.42041885107392e-08
1394 8.41295317854929e-08
1395 8.40542639934938e-08
1396 8.39971363575387e-08
1397 8.39229983284895e-08
1398 8.3858054722441e-08
1399 8.37807831999271e-08
1400 8.37194207292669e-08
1401 8.36489775224436e-08
1402 8.35875795246466e-08
1403 8.35257125686439e-08
1404 8.34510558433976e-08
1405 8.3389089411412e-08
1406 8.3324273703056e-08
1407 8.32566016129022e-08
1408 8.31955020430541e-08
1409 8.31322637395715e-08
1410 8.30597670642419e-08
1411 8.30007351737549e-08
1412 8.29412556413445e-08
1413 8.28734485480709e-08
1414 8.28162782795516e-08
1415 8.28798434326927e-08
1416 8.280800045668e-08
1417 8.27456716478991e-08
1418 8.26811330512101e-08
1419 8.26154860078532e-08
1420 8.25520629632592e-08
1421 8.24866290827231e-08
1422 8.24028916213138e-08
1423 8.23348145218006e-08
1424 8.22633978714293e-08
1425 8.22009198486739e-08
1426 8.21275065732152e-08
1427 8.20657177769135e-08
1428 8.19930718876094e-08
1429 8.19365908455438e-08
1430 8.1872812529582e-08
1431 8.18226268961553e-08
1432 8.17415823917145e-08
1433 8.16599978747945e-08
1434 8.15478600202368e-08
1435 8.14719030017841e-08
1436 8.13905671748216e-08
1437 8.13276841427069e-08
1438 8.12620157830679e-08
1439 8.1194023948683e-08
1440 8.11202767181385e-08
1441 8.10534430684129e-08
1442 8.09682347835405e-08
1443 8.09098494869431e-08
1444 8.08490767667536e-08
1445 8.07897180266082e-08
1446 8.07177542583304e-08
1447 8.06580615630992e-08
1448 8.05827937711001e-08
1449 8.05284727789513e-08
1450 8.04408131216405e-08
1451 8.03743418487102e-08
1452 8.02998840754299e-08
1453 8.02488884232844e-08
1454 8.01638364578139e-08
1455 8.0054356033088e-08
1456 8.00165409486908e-08
1457 7.99373935933545e-08
1458 7.98998129880601e-08
1459 7.98081813968565e-08
1460 7.97529224882965e-08
1461 7.97012731368341e-08
1462 7.96241153011579e-08
1463 7.95726293745247e-08
1464 7.95056607216793e-08
1465 7.9426911270275e-08
1466 7.93787648944999e-08
1467 7.9299447008907e-08
1468 7.92406567029502e-08
1469 7.9165147326421e-08
1470 7.91031240510165e-08
1471 7.90544376627622e-08
1472 7.89857779182057e-08
1473 7.89336951356745e-08
1474 7.8859471841497e-08
1475 7.87981377925462e-08
1476 7.87357663512012e-08
1477 7.86735299129759e-08
1478 7.86097515970141e-08
1479 7.85462006547277e-08
1480 7.84837084211176e-08
1481 7.84262397246493e-08
1482 7.83542972726536e-08
1483 7.828998604964e-08
1484 7.82222073780758e-08
1485 7.81589477583111e-08
1486 7.80989353188488e-08
1487 7.80301050440357e-08
1488 7.79623121616169e-08
1489 7.7898313577407e-08
1490 7.78329152240076e-08
1491 7.77693713871486e-08
1492 7.77099344873022e-08
1493 7.76358533016719e-08
1494 7.75712436507092e-08
1495 7.75090782667576e-08
1496 7.74368942302317e-08
1497 7.73880373117208e-08
1498 7.73159953837421e-08
1499 7.72331532061798e-08
1500 7.7207026549786e-08
1501 7.71387647091615e-08
1502 7.71028254575867e-08
1503 7.70385781834193e-08
1504 7.69779973097684e-08
1505 7.6931613079978e-08
1506 7.6868765575e-08
1507 7.68238024306811e-08
1508 7.67606067597626e-08
1509 7.6702654894234e-08
1510 7.66567609389313e-08
1511 7.65682415249103e-08
1512 7.65389174262054e-08
1513 7.6520329628238e-08
1514 7.64725172075487e-08
1515 7.64324141755424e-08
1516 7.63930287916992e-08
1517 7.6343290800196e-08
1518 7.62990168823308e-08
1519 7.62513465701886e-08
1520 7.61924496828215e-08
1521 7.61514584723955e-08
1522 7.60996954340953e-08
1523 7.60459286652804e-08
1524 7.59926450655257e-08
1525 7.5945578714709e-08
1526 7.58870797312738e-08
1527 7.58334621764334e-08
1528 7.57805267426193e-08
1529 7.57232143655528e-08
1530 7.56636495680141e-08
1531 7.56129665546723e-08
1532 7.5568074464627e-08
1533 7.55139382135894e-08
1534 7.54690603343988e-08
1535 7.54206936903756e-08
1536 7.53722773083609e-08
1537 7.53288915689154e-08
1538 7.52827702399372e-08
1539 7.52342756982216e-08
1540 7.5182761349879e-08
1541 7.51435749180018e-08
1542 7.5069841898312e-08
1543 7.50141779803926e-08
1544 7.49726538629147e-08
1545 7.49183115544838e-08
1546 7.48717283727274e-08
1547 7.48014272744513e-08
1548 7.47573665194068e-08
1549 7.46930766126752e-08
1550 7.46268327134203e-08
1551 7.45685611036606e-08
1552 7.45196757634403e-08
1553 7.44569348398727e-08
1554 7.44133004104697e-08
1555 7.43539345648969e-08
1556 7.43122470225899e-08
1557 7.42596526492889e-08
1558 7.42048626989344e-08
1559 7.41612424803861e-08
1560 7.41093657552483e-08
1561 7.40522025921564e-08
1562 7.40071968152733e-08
1563 7.39478025479912e-08
1564 7.39170431529601e-08
1565 7.38166008318331e-08
1566 7.3718503301734e-08
1567 7.36544549795326e-08
1568 7.35979028831935e-08
1569 7.35568903564854e-08
1570 7.34977376737334e-08
1571 7.34357428200383e-08
1572 7.33764977667306e-08
1573 7.33080227632854e-08
1574 7.32133500491727e-08
1575 7.31509572915456e-08
1576 7.30612654820106e-08
1577 7.3017936585984e-08
1578 7.29798728116293e-08
1579 7.29273352817472e-08
1580 7.28900886315387e-08
1581 7.281006020321e-08
1582 7.27625391050424e-08
1583 7.2715877763585e-08
1584 7.26124227412583e-08
1585 7.2554932728508e-08
1586 7.24988495903744e-08
1587 7.24044681987834e-08
1588 7.23679178804559e-08
1589 7.23028961147065e-08
1590 7.22562134569671e-08
1591 7.21897137623273e-08
1592 7.21411907989022e-08
1593 7.21099127076741e-08
1594 7.20591586400587e-08
1595 7.20146431376634e-08
1596 7.19863848530622e-08
1597 7.19271469051819e-08
1598 7.18948669486963e-08
1599 7.18389827625288e-08
1600 7.18068946525818e-08
1601 7.17511454695341e-08
1602 7.17178139097996e-08
1603 7.16588246518768e-08
1604 7.16086745455868e-08
1605 7.15588868160921e-08
1606 7.15143997354062e-08
1607 7.14697279136089e-08
1608 7.14342434093851e-08
1609 7.13906089799821e-08
1610 7.1378686072876e-08
1611 7.13124279627664e-08
1612 7.12724173013157e-08
1613 7.12257133272942e-08
1614 7.11925380869616e-08
1615 7.1150303426748e-08
1616 7.11138810061129e-08
1617 7.11019083610154e-08
1618 7.10375758217197e-08
1619 7.10144050231065e-08
1620 7.09712537627638e-08
1621 7.09444236690615e-08
1622 7.08988281417078e-08
1623 7.08616383349181e-08
1624 7.08294365381335e-08
1625 7.0791436712625e-08
1626 7.07982081848968e-08
1627 7.07315592762825e-08
1628 7.07352114659443e-08
1629 7.06666867245076e-08
1630 7.06632619085212e-08
1631 7.06228391322838e-08
1632 7.05526872479822e-08
1633 7.05170251080744e-08
1634 7.04602385326325e-08
1635 7.04346234670084e-08
1636 7.03797127243888e-08
1637 7.03662763612556e-08
1638 7.0304530197518e-08
1639 7.02873066416032e-08
1640 7.02644413763664e-08
1641 7.02164371091385e-08
1642 7.01853224427396e-08
1643 7.01125557611704e-08
1644 7.0099950733038e-08
1645 7.00387801089164e-08
1646 7.00040772017019e-08
1647 6.9949258829638e-08
1648 6.99281628158133e-08
1649 6.98738986670833e-08
1650 6.98529589726604e-08
1651 6.98033986168412e-08
1652 6.97787143622008e-08
1653 6.9748494979649e-08
1654 6.96821089718469e-08
1655 6.96667825650366e-08
1656 6.96114810239123e-08
1657 6.95843880293978e-08
1658 6.9531459701011e-08
1659 6.94990944793972e-08
1660 6.94509552090494e-08
1661 6.94222634933794e-08
1662 6.93946375918131e-08
1663 6.93338577661962e-08
1664 6.93070845159127e-08
1665 6.92578012717604e-08
1666 6.92276600489095e-08
1667 6.91920263307111e-08
1668 6.91633559313232e-08
1669 6.9110697609176e-08
1670 6.9079462150512e-08
1671 6.9051765194672e-08
1672 6.89484451754652e-08
1673 6.89520760488449e-08
1674 6.88674219873064e-08
1675 6.88904506773724e-08
1676 6.88338062104776e-08
1677 6.88034447193786e-08
1678 6.87358223672163e-08
1679 6.87091770146253e-08
1680 6.86352805701063e-08
1681 6.85908503328392e-08
1682 6.85730512373084e-08
1683 6.85537813183146e-08
1684 6.84924614802185e-08
1685 6.8459151236766e-08
1686 6.83832368508774e-08
1687 6.83856455907517e-08
1688 6.83340957152723e-08
1689 6.8314022882987e-08
1690 6.82694789588822e-08
1691 6.82179148725481e-08
1692 6.81997462947947e-08
1693 6.81821035186658e-08
1694 6.81380072364846e-08
1695 6.81234837429656e-08
1696 6.80709604239382e-08
1697 6.80450966683566e-08
1698 6.79749092569182e-08
1699 6.79206166864788e-08
1700 6.78750495808345e-08
1701 6.78276776966413e-08
1702 6.77502995927171e-08
1703 6.7741218856554e-08
1704 6.76403644206403e-08
1705 6.76113458553118e-08
1706 6.75592133347891e-08
1707 6.75327953558735e-08
1708 6.7488237220914e-08
1709 6.74542945944268e-08
1710 6.74224196473006e-08
1711 6.73832474262781e-08
1712 6.73399256356788e-08
1713 6.73060966960293e-08
1714 6.72678837077001e-08
1715 6.72348932084788e-08
1716 6.71956001951912e-08
1717 6.70948239189784e-08
1718 6.70681572501053e-08
1719 6.70066455654705e-08
1720 6.69944455466975e-08
1721 6.69418156462598e-08
1722 6.69140831632831e-08
1723 6.68663489022947e-08
1724 6.68396324954301e-08
1725 6.6818529376178e-08
1726 6.67688482280937e-08
1727 6.67662973796723e-08
1728 6.67131629938922e-08
1729 6.66839099494609e-08
1730 6.66779911284721e-08
1731 6.66279831307293e-08
1732 6.66299939666715e-08
1733 6.66224124756809e-08
1734 6.65566091129222e-08
1735 6.65409203293166e-08
1736 6.65288268919539e-08
1737 6.64589236976099e-08
1738 6.64651409465478e-08
1739 6.64288108964683e-08
1740 6.63774315512455e-08
1741 6.63561436908822e-08
1742 6.63305357306854e-08
1743 6.62979573462508e-08
1744 6.62571366660814e-08
1745 6.62506138837671e-08
1746 6.61922925360159e-08
1747 6.61557422176884e-08
1748 6.61251462474866e-08
1749 6.60967884869024e-08
1750 6.60539924979275e-08
1751 6.60249455108897e-08
1752 6.59650325474104e-08
1753 6.59201262465103e-08
1754 6.58724985669323e-08
1755 6.58460876934441e-08
1756 6.58210268511539e-08
1757 6.57818191029946e-08
1758 6.58436789535699e-08
1759 6.58145538068311e-08
1760 6.57702727835385e-08
1761 6.5742106869493e-08
1762 6.56977192647901e-08
1763 6.56712941804471e-08
1764 6.55665317594867e-08
1765 6.55283756145764e-08
1766 6.551771747354e-08
1767 6.54904255270594e-08
1768 6.54206502304078e-08
1769 6.53959588703401e-08
1770 6.53642615588979e-08
1771 6.53357972169033e-08
1772 6.52947704793405e-08
1773 6.52571898740462e-08
1774 6.51914717764157e-08
1775 6.51549072472335e-08
1776 6.51354028491369e-08
1777 6.50764064857867e-08
1778 6.50365237220285e-08
1779 6.50157403470075e-08
1780 6.49614975145596e-08
1781 6.49180478262679e-08
1782 6.48888303089734e-08
1783 6.48034088612803e-08
1784 6.4769842822443e-08
1785 6.47290434585557e-08
1786 6.47047926349842e-08
1787 6.4677983857564e-08
1788 6.46378595092756e-08
1789 6.45980193780815e-08
1790 6.45360316298138e-08
1791 6.44965325591329e-08
1792 6.44593001197791e-08
1793 6.44567137442209e-08
1794 6.44063575805376e-08
1795 6.43593693894218e-08
1796 6.43255830823364e-08
1797 6.42824531382757e-08
1798 6.42467412603764e-08
1799 6.42174313725263e-08
1800 6.41754169805608e-08
1801 6.41381276977881e-08
1802 6.41039719084802e-08
1803 6.40510080529566e-08
1804 6.40166106791185e-08
1805 6.39702335547554e-08
1806 6.39377049083123e-08
1807 6.38900061744607e-08
1808 6.38557722254518e-08
1809 6.38252970475151e-08
1810 6.37793036162293e-08
1811 6.37447570284166e-08
1812 6.37078017007298e-08
1813 6.36757917504838e-08
1814 6.36299333223178e-08
1815 6.35944630289487e-08
1816 6.35586303587843e-08
1817 6.35330152931601e-08
1818 6.3473841294126e-08
1819 6.34322034898105e-08
1820 6.33965413499027e-08
1821 6.33389021231778e-08
1822 6.32086170071489e-08
1823 6.3120815241291e-08
1824 6.30653289590555e-08
1825 6.29816128139282e-08
1826 6.29323579914853e-08
1827 6.28835934435301e-08
1828 6.28391134682715e-08
1829 6.28091925136687e-08
1830 6.27735445846156e-08
1831 6.27344647341488e-08
1832 6.26779481649464e-08
1833 6.26048972662829e-08
1834 6.25537595055903e-08
1835 6.24794864734213e-08
1836 6.24444567165483e-08
1837 6.23790370468669e-08
1838 6.23636324803556e-08
1839 6.23133260546638e-08
1840 6.22847409204041e-08
1841 6.22367792857403e-08
1842 6.21999447503185e-08
1843 6.21586835336529e-08
1844 6.21269293787918e-08
1845 6.20631084302659e-08
1846 6.20300824039077e-08
1847 6.19973263837892e-08
1848 6.19825684111674e-08
1849 6.19381168576183e-08
1850 6.19293700765411e-08
1851 6.18971327526197e-08
1852 6.18584437006575e-08
1853 6.18260003193427e-08
1854 6.17847675243866e-08
1855 6.17510664824295e-08
1856 6.17125408552965e-08
1857 6.16760971183794e-08
1858 6.16403710296254e-08
1859 6.1604239931512e-08
1860 6.15919120150465e-08
1861 6.15315016716522e-08
1862 6.15064550402167e-08
1863 6.14907520457564e-08
1864 6.14451352021206e-08
1865 6.1415398988629e-08
1866 6.13915105418528e-08
1867 6.13422130868457e-08
1868 6.13142745464756e-08
1869 6.12678903166852e-08
1870 6.12291017887401e-08
1871 6.12202768479619e-08
1872 6.11831723063005e-08
1873 6.11496560054547e-08
1874 6.11119403970406e-08
1875 6.10772090681166e-08
1876 6.10464638839403e-08
1877 6.09820816066531e-08
1878 6.09871335655043e-08
1879 6.09309083188236e-08
1880 6.08997652307153e-08
1881 6.0867769491324e-08
1882 6.08424102210847e-08
1883 6.08065917617751e-08
1884 6.07726491352878e-08
1885 6.07789019113625e-08
1886 6.07021064524815e-08
1887 6.06725478746739e-08
1888 6.0631101916897e-08
1889 6.06034049610571e-08
1890 6.0562967973965e-08
1891 6.05443659651428e-08
1892 6.04883041432913e-08
1893 6.0457381323431e-08
1894 6.03694658707354e-08
1895 6.07111232397983e-08
1896 6.06768182365158e-08
1897 6.06512955414473e-08
1898 6.02079097689057e-08
1899 6.05719421287176e-08
1900 6.05429946176628e-08
1901 6.05499721473279e-08
1902 6.04996017727899e-08
1903 6.00796994376651e-08
1904 6.04119421154792e-08
1905 6.0016510872174e-08
1906 6.03290573053528e-08
1907 5.99449876403924e-08
1908 6.02518639425398e-08
1909 6.0254073730448e-08
1910 6.02130967308767e-08
1911 5.98302705157039e-08
1912 6.01487286644442e-08
1913 5.97591309769996e-08
1914 6.00768075287306e-08
1915 6.00650764681632e-08
1916 5.96634421867748e-08
1917 5.99641651888305e-08
1918 5.9184607437146e-08
1919 5.9889167403071e-08
1920 5.98384204408831e-08
1921 5.94984044255398e-08
1922 5.90189550564446e-08
1923 5.89204631751272e-08
1924 5.95533364844414e-08
1925 5.92995661463647e-08
1926 5.92745337257838e-08
1927 5.96279292608415e-08
1928 5.92491744555446e-08
1929 5.92226641060734e-08
1930 5.91906434976863e-08
1931 5.91336970501288e-08
1932 5.90379869436219e-08
1933 5.90585358395401e-08
1934 5.90219286777938e-08
1935 5.87599231494096e-08
1936 5.89553188490299e-08
1937 5.89011541762829e-08
1938 5.88640745036173e-08
1939 5.8828252491594e-08
1940 5.88037316617829e-08
1941 5.85204666947448e-08
1942 5.84275916537536e-08
1943 5.86750701359051e-08
1944 5.8659232138325e-08
1945 5.86329385043882e-08
1946 5.85990207468967e-08
1947 5.85645878459218e-08
1948 5.85270036879137e-08
1949 5.81989780812364e-08
1950 5.81269752331082e-08
1951 5.80587560250478e-08
1952 5.79652521537355e-08
1953 5.8274991943108e-08
1954 5.82189407793976e-08
1955 5.82492383216504e-08
1956 5.82467691856436e-08
1957 5.79366385977664e-08
1958 5.81785002395918e-08
1959 5.81214152362008e-08
1960 5.80778163339346e-08
1961 5.8032153305021e-08
1962 5.79979335668668e-08
1963 5.76819516595606e-08
1964 5.79381058685158e-08
1965 5.79032430891857e-08
1966 5.78792871408496e-08
1967 5.78552210583894e-08
1968 5.78294567787907e-08
1969 5.78036569720553e-08
1970 5.75100997934896e-08
1971 5.73729153074964e-08
1972 5.76930787588026e-08
1973 5.76807153152004e-08
1974 5.76594842982558e-08
1975 5.7629861771602e-08
1976 5.73157485916909e-08
1977 5.7554814247851e-08
1978 5.72647493868317e-08
1979 5.77607615070974e-08
1980 5.75064085239774e-08
1981 5.74538816522363e-08
1982 5.74210829995536e-08
1983 5.73908245371513e-08
1984 5.73619765020794e-08
1985 5.70873339711397e-08
1986 5.72779477181484e-08
1987 5.70115901155077e-08
1988 5.72057707870499e-08
1989 5.69341587208783e-08
1990 5.71861527021156e-08
1991 5.71656499914752e-08
1992 5.71422340556182e-08
1993 5.70956792955712e-08
1994 5.70369920183111e-08
1995 5.6934947423315e-08
1996 5.69591982468864e-08
1997 5.69246871862106e-08
1998 5.68935298872475e-08
1999 5.66868258999875e-08
2000 5.68104958631466e-08
2001 5.65986582046207e-08
2002 5.67396050143998e-08
2003 5.64618787279869e-08
2004 5.66743700858297e-08
2005 5.66524853695682e-08
2006 5.66311371130723e-08
2007 5.66084423780921e-08
2008 5.64200561825601e-08
2009 5.65402515917413e-08
2010 5.62776136803222e-08
2011 5.64876678765813e-08
2012 5.64515616474637e-08
2013 5.64184610141183e-08
2014 5.6398754111342e-08
2015 5.62058524167242e-08
2016 5.63198696568179e-08
2017 5.62833832873366e-08
2018 5.62477069365741e-08
2019 5.62227491229805e-08
2020 5.61563382461827e-08
2021 5.61302719859214e-08
2022 5.59389725651727e-08
2023 5.5892336092711e-08
2024 5.58333113076515e-08
2025 5.57732349193429e-08
2026 5.57341905960129e-08
2027 5.56988233313405e-08
2028 5.56813546381818e-08
2029 5.56454864408806e-08
2030 5.56285471020601e-08
2031 5.55672947655239e-08
2032 5.55426922232982e-08
2033 5.5482889393943e-08
2034 5.54493944093792e-08
2035 5.54038521727307e-08
2036 5.53481278586787e-08
2037 5.52980452539487e-08
2038 5.53012178272638e-08
2039 5.52828005595529e-08
2040 5.52811911802564e-08
2041 5.52450529767157e-08
2042 5.51845928953298e-08
2043 5.53745316267396e-08
2044 5.51820527050495e-08
2045 5.51619265820591e-08
2046 5.51232552936654e-08
2047 5.51023013883878e-08
2048 5.50842926827499e-08
2049 5.50530856457954e-08
2050 5.50459056114505e-08
2051 5.50073764316039e-08
2052 5.50020686773678e-08
2053 5.49698810914379e-08
2054 5.49567467089673e-08
2055 5.49313625697323e-08
2056 5.487446941288e-08
2057 5.48779546249989e-08
2058 5.48348779716434e-08
2059 5.48040155479157e-08
2060 5.47785425908387e-08
2061 5.47498508751687e-08
2062 5.47143734763722e-08
2063 5.46860263739291e-08
2064 5.46529612677205e-08
2065 5.46192957529001e-08
2066 5.45885363578691e-08
2067 5.45183560518581e-08
2068 5.46484990593399e-08
2069 5.44502078980713e-08
2070 5.44078417874516e-08
2071 5.43680762632448e-08
2072 5.43341549530396e-08
2073 5.43017470988616e-08
2074 5.42689200244695e-08
2075 5.42373115308692e-08
2076 5.42057883023972e-08
2077 5.4176158670316e-08
2078 5.41491829153529e-08
2079 5.41194786762844e-08
2080 5.40925952918769e-08
2081 5.40173274998779e-08
2082 5.416492143695e-08
2083 5.41505116302687e-08
2084 5.39482982730988e-08
2085 5.40963824846585e-08
2086 5.4070959265573e-08
2087 5.40416991157144e-08
2088 5.39651097142269e-08
2089 5.38940696515056e-08
2090 5.38483320156047e-08
2091 5.38192352905753e-08
2092 5.37947748568968e-08
2093 5.37530802091624e-08
2094 5.37333022521125e-08
2095 5.40187485853494e-08
2096 5.44266214319578e-08
2097 5.44374074706866e-08
2098 5.44719398476445e-08
2099 5.44253211387513e-08
2100 5.44893907772348e-08
2101 5.45306626520414e-08
2102 5.45718030764419e-08
2103 5.46031309056616e-08
2104 5.45977023591604e-08
2105 5.46083889219062e-08
2106 5.44677405400762e-08
2107 5.45434666321398e-08
2108 5.44645288869106e-08
2109 5.44144604930352e-08
2110 5.44647846822954e-08
2111 5.44199210139595e-08
2112 5.44243476952033e-08
2113 5.43999476576573e-08
2114 5.43546754272484e-08
2115 5.42833689110012e-08
2116 5.43088312099371e-08
2117 5.42684119864134e-08
2118 5.42711582340871e-08
2119 5.41492646277675e-08
2120 5.41139080212361e-08
2121 5.41211235827177e-08
2122 5.40198570320172e-08
2123 5.39932045739988e-08
2124 5.39447277958516e-08
2125 5.38862394705575e-08
2126 5.37585798099371e-08
2127 5.37001092482114e-08
2128 5.36359188174629e-08
2129 5.3664951593646e-08
2130 5.34439656973973e-08
2131 5.34346895619819e-08
2132 5.34634096993614e-08
2133 5.33098152288858e-08
2134 5.32050243862159e-08
2135 5.32970254596421e-08
2136 5.30917425578536e-08
2137 5.30840189583159e-08
2138 5.31281294513519e-08
2139 5.2974151287799e-08
2140 5.293443194887e-08
2141 5.30358796879682e-08
2142 5.2874394640412e-08
2143 5.28753858475284e-08
2144 5.27870298583366e-08
2145 5.27804573380308e-08
2146 5.27918793125082e-08
2147 5.29293906481598e-08
2148 5.27027772534439e-08
2149 5.2671285999395e-08
2150 5.26220453878068e-08
2151 5.25845678112091e-08
2152 5.25470049694832e-08
2153 5.25015444452492e-08
2154 5.25191303779593e-08
2155 5.24429815129679e-08
2156 5.24017309544433e-08
2157 5.24634273801894e-08
2158 5.24277865565637e-08
2159 5.23707655020189e-08
2160 5.2337892242349e-08
2161 5.24386578604208e-08
2162 5.22775636113693e-08
2163 5.22513232681376e-08
2164 5.24053014316905e-08
2165 5.21384073692843e-08
2166 5.20705150108824e-08
2167 5.19989526992504e-08
2168 5.1986933868875e-08
2169 5.19796614639745e-08
2170 5.19140783694638e-08
2171 5.18796561266299e-08
2172 5.18252001313613e-08
2173 5.20350837973638e-08
2174 5.17408409450582e-08
2175 5.17134282063125e-08
2176 5.16802103334157e-08
2177 5.16597822297626e-08
2178 5.1600746786562e-08
2179 5.16432585584425e-08
2180 5.19299554468944e-08
2181 5.19345810801042e-08
2182 5.19589313796587e-08
2183 5.18880440836256e-08
2184 5.16302058883866e-08
2185 5.17627611884564e-08
2186 5.18374534408395e-08
2187 5.17479392669884e-08
2188 5.17468912164532e-08
2189 5.17122558107985e-08
2190 5.17275182687627e-08
2191 5.15912610410396e-08
2192 5.18411766847748e-08
2193 5.16471452272071e-08
2194 5.16276301709695e-08
2195 5.17090938956244e-08
2196 5.16348741541606e-08
2197 5.16309164311224e-08
2198 5.16276017492601e-08
2199 5.14226883296942e-08
2200 5.15758564745283e-08
2201 5.15871612094543e-08
2202 5.16519271798188e-08
2203 5.15608533646628e-08
2204 5.15410611967582e-08
2205 5.15120213151476e-08
2206 5.1406022549827e-08
2207 5.14754887603885e-08
2208 5.15854345906064e-08
2209 5.14699713960454e-08
2210 5.1457774929986e-08
2211 5.14552702668425e-08
2212 5.14232816328786e-08
2213 5.12711970657165e-08
2214 5.13962277182145e-08
2215 5.13201143803599e-08
2216 5.13648643618581e-08
2217 5.13524405221233e-08
2218 5.11252267187956e-08
2219 5.12690263576587e-08
2220 5.12742417413392e-08
2221 5.12363342863864e-08
2222 5.12171247635251e-08
2223 5.11980715600657e-08
2224 5.11957765070292e-08
2225 5.11185724860752e-08
2226 5.10109927631675e-08
2227 5.09700015527415e-08
2228 5.09203132992297e-08
2229 5.08952879840763e-08
2230 5.09558475414451e-08
2231 5.08739255167256e-08
2232 5.08741067051233e-08
2233 5.08544140132017e-08
2234 5.08437913993021e-08
2235 5.0727521738736e-08
2236 5.09729538578085e-08
2237 5.08682767019764e-08
2238 5.08392368203658e-08
2239 5.09478788046636e-08
2240 5.08323232395469e-08
2241 5.07960820073095e-08
2242 5.10033579814717e-08
2243 5.07937834015593e-08
2244 5.07754869261134e-08
2245 5.07492501355955e-08
2246 5.07237416513817e-08
2247 5.06507866759875e-08
2248 5.07068307342706e-08
2249 5.07202742028312e-08
2250 5.07038144803573e-08
2251 5.06545134726366e-08
2252 5.06465731575645e-08
2253 5.06431980795696e-08
2254 5.06334352223803e-08
2255 5.05794517380309e-08
2256 5.05872286282738e-08
2257 5.05833135377998e-08
2258 5.05352986124308e-08
2259 5.0403230034135e-08
2260 5.05562383068536e-08
2261 5.05737176581533e-08
2262 5.05585440180312e-08
2263 5.05566148945036e-08
2264 5.05562951502725e-08
2265 5.05612369749997e-08
2266 5.04657435840272e-08
2267 5.05320514321284e-08
2268 5.04997537120744e-08
2269 5.04777659671163e-08
2270 5.04280279756131e-08
2271 5.04580128790622e-08
2272 5.04560553338251e-08
2273 5.04441395321464e-08
2274 5.04331403305969e-08
2275 5.0270280382847e-08
2276 5.04720070182429e-08
2277 5.05174426734811e-08
2278 5.05302040210154e-08
2279 5.05268715755847e-08
2280 5.05282535812057e-08
2281 5.05203558986977e-08
2282 5.069064457075e-08
2283 5.05467525613312e-08
2284 5.05158581631804e-08
2285 5.05135417938618e-08
2286 5.049916040889e-08
2287 5.07053243836708e-08
2288 5.05193504807266e-08
2289 5.05148385343546e-08
2290 5.05108310733249e-08
2291 5.05083832536002e-08
2292 5.04457879912934e-08
2293 5.04852692984059e-08
2294 5.0483929925349e-08
2295 5.04773147724791e-08
2296 5.0457195754916e-08
2297 5.02637149679686e-08
2298 5.04416348690029e-08
2299 5.04569293013901e-08
2300 5.04313604210438e-08
2301 5.04255233124695e-08
2302 5.04069603834978e-08
2303 5.03891257608302e-08
2304 5.04597785777605e-08
2305 5.03725559042323e-08
2306 5.04213275576149e-08
2307 5.03826811382169e-08
2308 5.03691381936733e-08
2309 5.0070887880338e-08
2310 5.04306640891627e-08
2311 5.04162898096183e-08
2312 5.03765384962662e-08
2313 5.04165278414348e-08
2314 5.03677526353385e-08
2315 5.03720762878856e-08
2316 5.03977410915013e-08
2317 5.03487918024348e-08
2318 5.03376256233423e-08
2319 5.0336179668875e-08
2320 5.01858394841292e-08
2321 5.03069834678627e-08
2322 5.03639547844159e-08
2323 5.03964159292991e-08
2324 5.03607537893913e-08
2325 5.03706303334184e-08
2326 5.00808354786386e-08
2327 5.04565846881633e-08
2328 5.04128507827772e-08
2329 5.03646830907201e-08
2330 5.04295911696317e-08
2331 5.03654007388832e-08
2332 5.03570021237465e-08
2333 5.03390964468053e-08
2334 4.99843118006993e-08
2335 5.04036101744987e-08
2336 5.03538579721408e-08
2337 5.03095911597029e-08
2338 5.03452319833286e-08
2339 5.03030364029655e-08
2340 5.03636350401848e-08
2341 5.02890351583574e-08
2342 4.98473902155183e-08
2343 5.02895645126955e-08
2344 5.05220079105584e-08
2345 5.03066601709179e-08
2346 5.02636154919855e-08
2347 5.02150143688596e-08
2348 5.03928490047656e-08
2349 5.01277135356304e-08
2350 5.02070847119285e-08
2351 5.02204393626471e-08
2352 5.04252390953752e-08
2353 4.97493601869792e-08
2354 5.02305432803496e-08
2355 5.02101613619743e-08
2356 5.0413184737863e-08
2357 5.01506143280039e-08
2358 5.01415762244051e-08
2359 5.01578085732035e-08
2360 5.04828356895359e-08
2361 4.97474950122978e-08
2362 5.02426047432891e-08
2363 5.02314918549018e-08
2364 5.0512831251126e-08
2365 5.03195138890078e-08
2366 5.03508381655138e-08
2367 5.03748687208372e-08
2368 5.0749189739463e-08
2369 5.03631554238382e-08
2370 5.04081363317255e-08
2371 5.03847559230053e-08
2372 5.04816917157314e-08
2373 5.03658590389477e-08
2374 5.03416863750772e-08
2375 5.0340361212875e-08
2376 5.02976291727464e-08
2377 5.03216952552066e-08
2378 5.02863564122435e-08
2379 5.05649531135077e-08
2380 5.02751866804374e-08
2381 5.02850205919003e-08
2382 5.0238757154375e-08
2383 5.05809012452119e-08
2384 5.01949379838607e-08
2385 5.0133170503841e-08
2386 5.00806365266726e-08
2387 5.01452817047721e-08
2388 5.00616899046236e-08
2389 4.96590892851145e-08
2390 5.03568280407762e-08
2391 4.98189045572417e-08
2392 4.9739441010388e-08
2393 4.9668219759269e-08
2394 4.98610965848911e-08
2395 4.95880101425428e-08
2396 4.951692034183e-08
2397 4.92130816098779e-08
2398 4.98555330352701e-08
2399 4.95011533985235e-08
2400 4.94794001326682e-08
2401 4.94779932580514e-08
2402 4.96099161750863e-08
2403 4.94318399546501e-08
2404 4.944363141135e-08
2405 4.93820984104332e-08
2406 4.93780945021172e-08
2407 4.94021570318637e-08
2408 4.93962453163022e-08
2409 4.96585634834901e-08
2410 4.93232157339207e-08
2411 4.93746732388445e-08
2412 4.92974763233178e-08
2413 4.92816951691566e-08
2414 4.92588547729156e-08
2415 4.9313833017095e-08
2416 4.92508078764331e-08
2417 4.96710264030753e-08
2418 4.92289622400222e-08
2419 4.92243010796756e-08
2420 4.92572169719097e-08
2421 4.91381904055288e-08
2422 4.92332468127188e-08
2423 4.92461182943771e-08
2424 4.93573821813698e-08
2425 4.92349876424214e-08
2426 4.93258376366157e-08
2427 4.93654788158437e-08
2428 4.93264487033684e-08
2429 4.89510121326475e-08
2430 4.93021907743696e-08
2431 4.91830789428604e-08
2432 4.94110743431975e-08
2433 4.89736713404909e-08
2434 4.93766378895089e-08
2435 4.92431269094595e-08
2436 4.89511755574767e-08
2437 4.92969611798344e-08
2438 4.92906906401913e-08
2439 4.92128755524845e-08
2440 4.89121951829929e-08
2441 4.92320033629312e-08
2442 4.93166609771833e-08
2443 4.90930283092439e-08
2444 4.92763057025059e-08
2445 4.91238587585485e-08
2446 4.92384621963993e-08
2447 4.87882623190217e-08
2448 4.92338791957536e-08
2449 4.91694081006244e-08
2450 4.89197446995604e-08
2451 4.91042548844689e-08
2452 4.91794125423439e-08
2453 4.9071203989115e-08
2454 4.87896656409248e-08
2455 4.91395830692909e-08
2456 4.91349148035169e-08
2457 4.89695501926235e-08
2458 4.91672764724171e-08
2459 4.90055533930445e-08
2460 4.9137160118562e-08
2461 4.86599915916486e-08
2462 4.9151726244645e-08
2463 4.90430700494926e-08
2464 4.87089444334288e-08
2465 4.93130016820942e-08
2466 4.89160676409028e-08
2467 4.88980020918461e-08
2468 4.8863615376149e-08
2469 4.91632867749558e-08
2470 4.86100439900383e-08
2471 4.87677453975266e-08
2472 4.89913070111925e-08
2473 4.85544653372472e-08
2474 4.89129128311561e-08
2475 4.88053224501073e-08
2476 4.85008655459751e-08
2477 4.87867701792766e-08
2478 4.87689391093227e-08
2479 4.891415272823e-08
2480 4.87677880300907e-08
2481 4.87489018041742e-08
2482 4.88373537166353e-08
2483 4.85512998693594e-08
2484 4.8808864505645e-08
2485 4.84885269713686e-08
2486 4.88011266952526e-08
2487 4.84989080007381e-08
2488 4.87763109902062e-08
2489 4.84868678540806e-08
2490 4.8533717489363e-08
2491 4.82196966800075e-08
2492 4.85450151188616e-08
2493 4.81846385014251e-08
2494 4.84391513566607e-08
2495 4.82007926905226e-08
2496 4.84223647845283e-08
2497 4.81789363959706e-08
2498 4.82348099239971e-08
2499 4.83185971233979e-08
2500 4.82819046965233e-08
2501 4.82076529806363e-08
2502 4.80243080858145e-08
2503 4.82252353606327e-08
2504 4.82154760561571e-08
2505 4.84765259045616e-08
2506 4.81250985728821e-08
2507 4.79454449475725e-08
2508 4.82813966584672e-08
2509 4.80937210056709e-08
2510 4.80821213955096e-08
2511 4.81114987849196e-08
2512 4.81283670694665e-08
2513 4.80531809898821e-08
2514 4.7928239155226e-08
2515 4.8003961694576e-08
2516 4.80611639375184e-08
2517 4.80543995706739e-08
2518 4.7741025355208e-08
2519 4.81215032266391e-08
2520 4.81259583295923e-08
2521 4.84350337615069e-08
2522 4.81089514892119e-08
2523 4.79511932383048e-08
2524 4.82164814741282e-08
2525 4.80510777833842e-08
2526 4.80860933294025e-08
2527 4.78422208516349e-08
2528 4.79972825928598e-08
2529 4.80044235473542e-08
2530 4.81471147395496e-08
2531 4.7866144825548e-08
2532 4.81130690843656e-08
2533 4.75613290973342e-08
2534 4.79399737685071e-08
2535 4.79208566162015e-08
2536 4.79614072901313e-08
2537 4.75121026966008e-08
2538 4.78958384064754e-08
2539 4.74901682423479e-08
2540 4.77953534527842e-08
2541 4.80067292585318e-08
2542 4.76558987827502e-08
2543 4.76313317676613e-08
2544 4.78178456830847e-08
2545 4.74807713146674e-08
2546 4.77401798093524e-08
2547 4.80549005033026e-08
2548 4.76484096623153e-08
2549 4.75385704135078e-08
2550 4.75761297025201e-08
2551 4.76081254419114e-08
2552 4.75256562992854e-08
2553 4.79459671964833e-08
2554 4.74625565516362e-08
2555 4.72630432568621e-08
2556 4.76190606946147e-08
2557 4.7596106611536e-08
2558 4.74831338692638e-08
2559 4.74354138191302e-08
2560 4.74762558155817e-08
2561 4.75040522474046e-08
2562 4.75023789192619e-08
2563 4.73681716073315e-08
2564 4.74918699922e-08
2565 4.71424996817404e-08
2566 4.74950212492331e-08
2567 4.72867718315229e-08
2568 4.74300705377573e-08
2569 4.70609045066794e-08
2570 4.73857610927553e-08
2571 4.72282870589424e-08
2572 4.73309462734051e-08
2573 4.7038611228345e-08
2574 4.73246153376294e-08
2575 4.70320777878896e-08
2576 4.72762380354652e-08
2577 4.69829153360024e-08
2578 4.72290011543919e-08
2579 4.69666261437851e-08
2580 4.72198387058143e-08
2581 4.69330885266572e-08
2582 4.71909444854646e-08
2583 4.70162468957369e-08
2584 4.69845495842947e-08
2585 4.68338434700399e-08
2586 4.69966252580889e-08
2587 4.72463952405633e-08
2588 4.68215901605618e-08
2589 4.69204017861102e-08
2590 4.69882408538069e-08
2591 4.68271821318922e-08
2592 4.68768597272629e-08
2593 4.73555203939213e-08
2594 4.68507010964458e-08
2595 4.69036969263925e-08
2596 4.70942218555592e-08
2597 4.68904310935159e-08
2598 4.69388332646758e-08
2599 4.70514009975886e-08
2600 4.69333336639011e-08
2601 4.69569627625788e-08
2602 4.69279122228272e-08
2603 4.68196894587436e-08
2604 4.69071501640883e-08
2605 4.73100740805421e-08
2606 4.68084095928134e-08
2607 4.73474059958789e-08
2608 4.69663561375455e-08
2609 4.67854057717432e-08
2610 4.67684380112132e-08
2611 4.71393484247074e-08
2612 4.67870826525996e-08
2613 4.67680187909991e-08
2614 4.6766903238904e-08
2615 4.68147476340164e-08
2616 4.71029260040723e-08
2617 4.66910350382932e-08
2618 4.68024161648373e-08
2619 4.65416292172449e-08
2620 4.66663792053623e-08
2621 4.74897952074116e-08
2622 4.64744438488651e-08
2623 4.66117278108413e-08
2624 4.66005545263215e-08
2625 4.68741987447174e-08
2626 4.64359573015827e-08
2627 4.64727669680087e-08
2628 4.72731116474279e-08
2629 4.63523441851521e-08
2630 4.69464112029527e-08
2631 4.64529641419631e-08
2632 4.67467486942041e-08
2633 4.63345308787666e-08
2634 4.67107845736336e-08
2635 4.67197480702453e-08
2636 4.64131666433332e-08
2637 4.62616149832229e-08
2638 4.68191423408371e-08
2639 4.69496512778278e-08
2640 4.61694895648179e-08
2641 4.62885054730577e-08
2642 4.64475640171713e-08
2643 4.61111184790752e-08
2644 4.62701628123341e-08
2645 4.65858462916913e-08
2646 4.61144971097838e-08
2647 4.62100828713119e-08
2648 4.66110066099645e-08
2649 4.66265355214546e-08
2650 4.59975701971871e-08
2651 4.61377567262389e-08
2652 4.59302142985507e-08
2653 4.64384335430168e-08
2654 4.64560976354278e-08
2655 4.60553337688907e-08
2656 4.61809328555773e-08
2657 4.59906317473724e-08
2658 4.60252316258902e-08
2659 4.64309550807229e-08
2660 4.58093758481937e-08
2661 4.58753000032175e-08
2662 4.64741383154887e-08
2663 4.63451073073884e-08
2664 4.57062476755254e-08
2665 4.58142075387968e-08
2666 4.58159910010636e-08
2667 4.61180817978857e-08
2668 4.6171965806252e-08
2669 4.55934383580825e-08
2670 4.57579325541246e-08
2671 4.62913334331461e-08
2672 4.57318840574317e-08
2673 4.56720599117943e-08
2674 4.58766749034112e-08
2675 4.53521771248688e-08
2676 4.56105588853006e-08
2677 4.6094740469016e-08
2678 4.56261055603591e-08
2679 4.55273791999389e-08
2680 4.5951281890666e-08
2681 4.63773197623141e-08
2682 4.5511018953448e-08
2683 4.55334685511843e-08
2684 4.54922179926598e-08
2685 4.5597680298215e-08
2686 4.54805189065155e-08
2687 4.54591813081606e-08
2688 4.52422703745015e-08
2689 4.58316300466777e-08
2690 4.57863471581277e-08
2691 4.60196538654145e-08
2692 4.57116087204668e-08
2693 4.56852582431111e-08
2694 4.58445370554728e-08
2695 4.5613795407462e-08
2696 4.52626061075989e-08
2697 4.53640254249876e-08
2698 4.56109070512412e-08
2699 4.55551258937703e-08
2700 4.584461166246e-08
2701 4.55291022660731e-08
2702 4.55843469637784e-08
2703 4.54245494552197e-08
2704 4.5513939284092e-08
2705 4.53692017288176e-08
2706 4.53530724087159e-08
2707 4.56032900331138e-08
2708 4.5138968118863e-08
2709 4.5264783921084e-08
2710 4.56303297369232e-08
2711 4.53060344796086e-08
2712 4.52340884748992e-08
2713 4.5187103836497e-08
2714 4.54541648764462e-08
2715 4.52044481846769e-08
2716 4.5152749095223e-08
2717 4.5278930826953e-08
2718 4.51072494911386e-08
2719 4.51226327413679e-08
2720 4.54151809492487e-08
2721 4.51047696969908e-08
2722 4.50673027785342e-08
2723 4.50572983368147e-08
2724 4.51335431250754e-08
2725 4.50352608538651e-08
2726 4.50064341350753e-08
2727 4.47905463829557e-08
2728 4.49983268424603e-08
2729 4.49864998586236e-08
2730 4.52667592298894e-08
2731 4.49695214399526e-08
2732 4.46271570808676e-08
2733 4.48654695617279e-08
2734 4.51363071363176e-08
2735 4.48422809995463e-08
2736 4.48241692652118e-08
2737 4.48111805440021e-08
2738 4.49576909034022e-08
2739 4.46482282256966e-08
2740 4.47554775462322e-08
2741 4.49632615584505e-08
2742 4.47252439528256e-08
2743 4.46788241958984e-08
2744 4.47874697329098e-08
2745 4.48962680366094e-08
2746 4.46457697478309e-08
2747 4.47485746235543e-08
2748 4.46115038243988e-08
2749 4.46845547230623e-08
2750 4.45417214223198e-08
2751 4.45181385089199e-08
2752 4.48073791403658e-08
2753 4.46154047040181e-08
2754 4.45932215598077e-08
2755 4.45236914003999e-08
2756 4.45857644137959e-08
2757 4.44210499495057e-08
2758 4.45743850718827e-08
2759 4.46709549350999e-08
2760 4.43690240103933e-08
2761 4.43508447744989e-08
2762 4.43459242660538e-08
2763 4.44773355923189e-08
2764 4.42982468484843e-08
2765 4.44166943225355e-08
2766 4.45381260760769e-08
2767 4.43038778996652e-08
2768 4.42198029304564e-08
2769 4.43048229215037e-08
2770 4.43428938012858e-08
2771 4.41904788317515e-08
2772 4.41502621129075e-08
2773 4.44937775512244e-08
2774 4.41206715606768e-08
2775 4.41475727086527e-08
2776 4.41246754689928e-08
2777 4.39992788869858e-08
2778 4.40858691774793e-08
2779 4.40663932010921e-08
2780 4.43312124787099e-08
2781 4.40154792613612e-08
2782 4.42428671476591e-08
2783 4.39720864164883e-08
2784 4.3874212707351e-08
2785 4.39687362074892e-08
2786 4.39483294201182e-08
2787 4.41968310838092e-08
2788 4.39059704149258e-08
2789 4.36630820388473e-08
2790 4.38207372610577e-08
2791 4.40046648009229e-08
2792 4.38404903491119e-08
2793 4.376704509923e-08
2794 4.41402079331965e-08
2795 4.37923084462e-08
2796 4.37614531278996e-08
2797 4.37663096874985e-08
2798 4.39084004710821e-08
2799 4.37132960939834e-08
2800 4.37288996124607e-08
2801 4.36745857257392e-08
2802 4.38826219806288e-08
2803 4.36724114649678e-08
2804 4.40593090900165e-08
2805 4.38864802276839e-08
2806 4.38120117962626e-08
2807 4.35760867389945e-08
2808 4.34702442930757e-08
2809 4.40288907554987e-08
2810 4.34915357061527e-08
2811 4.35218971972517e-08
2812 4.34197140464221e-08
2813 4.36577778373248e-08
2814 4.36201972320305e-08
2815 4.34471267851677e-08
2816 4.35082156968747e-08
2817 4.3383977299527e-08
2818 4.33478959394051e-08
2819 4.31408437862046e-08
2820 4.3657632176064e-08
2821 4.32600373301284e-08
2822 4.32062954303092e-08
2823 4.32266062944109e-08
2824 4.3429238871795e-08
2825 4.31588524918425e-08
2826 4.3174431141324e-08
2827 4.33317168813119e-08
2828 4.31134559164548e-08
2829 4.31300293257664e-08
2830 4.30786961658214e-08
2831 4.31596056671424e-08
2832 4.30435704856791e-08
2833 4.30219131430931e-08
2834 4.30359534675517e-08
2835 4.31055191540963e-08
2836 4.29379802824315e-08
2837 4.27618722653733e-08
2838 4.32372111447421e-08
2839 4.28752962022827e-08
2840 4.27014796855474e-08
2841 4.27984367945555e-08
2842 4.28042241651383e-08
2843 4.26683719467746e-08
2844 4.27250306245242e-08
2845 4.26151167687294e-08
2846 4.27004103187301e-08
2847 4.26625632599098e-08
2848 4.24837516277421e-08
2849 4.26220552185441e-08
2850 4.26316510981906e-08
2851 4.26016555366004e-08
2852 4.25383781532673e-08
2853 4.24221404671243e-08
2854 4.25081623234291e-08
2855 4.24891410943928e-08
2856 4.23052703979465e-08
2857 4.26058157643183e-08
2858 4.23919388481409e-08
2859 4.23777599678488e-08
2860 4.23560138074208e-08
2861 4.23391277593055e-08
2862 4.216750326691e-08
2863 4.23760262435735e-08
2864 4.22850163772637e-08
2865 4.22991917048421e-08
2866 4.22841992531175e-08
2867 4.21151113982887e-08
2868 4.22183852322178e-08
2869 4.21723065358037e-08
2870 4.19987458144533e-08
2871 4.23048192033093e-08
2872 4.21012558149414e-08
2873 4.2103870612209e-08
2874 4.20970813763688e-08
2875 4.20173336124208e-08
2876 4.18703400839604e-08
2877 4.23649701986051e-08
2878 4.20144488089136e-08
2879 4.20269117284988e-08
2880 4.20252419530698e-08
2881 4.19549230912253e-08
2882 4.21753583168538e-08
2883 4.18237142696398e-08
2884 4.1875896528154e-08
2885 4.18642258637192e-08
2886 4.18689687364804e-08
2887 4.1844760545473e-08
2888 4.17810248620754e-08
2889 4.16655190349502e-08
2890 4.20864623151829e-08
2891 4.17370031868813e-08
2892 4.15605967418742e-08
2893 4.16974579309226e-08
2894 4.16629291066783e-08
2895 4.18388701461936e-08
2896 4.16322194496388e-08
2897 4.16122709623323e-08
2898 4.16166017203068e-08
2899 4.16928465085675e-08
2900 4.146069798594e-08
2901 4.15408614173884e-08
2902 4.15442897860885e-08
2903 4.18207939389958e-08
2904 4.13454301906313e-08
2905 4.14839895768182e-08
2906 4.14548750882204e-08
2907 4.14091871903111e-08
2908 4.14922958213992e-08
2909 4.14016874117351e-08
2910 4.13695175893736e-08
2911 4.13551433098291e-08
2912 4.15076719662011e-08
2913 4.13247427388796e-08
2914 4.11968130720197e-08
2915 4.12895531098911e-08
2916 4.15851921786725e-08
2917 4.11597795846319e-08
2918 4.122050256683e-08
2919 4.1186535071347e-08
2920 4.10691889385362e-08
2921 4.13211758143461e-08
2922 4.1148098262056e-08
2923 4.10521536764463e-08
2924 4.11199607697199e-08
2925 4.14305176832386e-08
2926 4.10828633334859e-08
2927 4.1084863511287e-08
2928 4.10745926160416e-08
2929 4.10569924724768e-08
2930 4.10076168577689e-08
2931 4.13641281227228e-08
2932 4.08751894553916e-08
2933 4.09755074315399e-08
2934 4.0922458310888e-08
2935 4.09432630021911e-08
2936 4.10426466146419e-08
2937 4.07698443893878e-08
2938 4.08738678459031e-08
2939 4.0834304826376e-08
2940 4.11280858259033e-08
2941 4.08340667945595e-08
2942 4.07999642959567e-08
2943 4.07839486626926e-08
2944 4.066644621048e-08
2945 4.07420124304281e-08
2946 4.10591241006841e-08
2947 4.06044762257807e-08
2948 4.06979232536742e-08
2949 4.06398150687437e-08
2950 4.0635995901539e-08
2951 4.06843128075707e-08
2952 4.05159816807554e-08
2953 4.06047497847339e-08
2954 4.04976390200318e-08
2955 4.07360118970246e-08
2956 4.03324307285402e-08
2957 4.04296542910743e-08
2958 4.03106419355481e-08
2959 4.0430663261759e-08
2960 4.03519031522137e-08
2961 4.06356619464532e-08
2962 4.02376549857308e-08
2963 4.03344202482003e-08
2964 4.0390446542915e-08
2965 4.02592270631885e-08
2966 4.05349958043644e-08
2967 4.01183974929609e-08
2968 4.02720452541416e-08
2969 4.01773405656058e-08
2970 4.00350295137741e-08
2971 4.04793247810176e-08
2972 4.01674213890146e-08
2973 4.00284250190452e-08
2974 4.00948962919756e-08
2975 4.01715816167325e-08
2976 4.03753901423443e-08
2977 4.00945019407573e-08
2978 4.01219288903576e-08
2979 3.99949975360414e-08
2980 4.01050286313875e-08
2981 4.02326740811532e-08
2982 4.00065900407753e-08
2983 3.99320860822172e-08
2984 3.9859209266524e-08
2985 3.98414350399889e-08
2986 4.01861335319609e-08
2987 3.99670270212482e-08
2988 3.99207849000049e-08
2989 3.98670430001857e-08
2990 3.97455295342297e-08
2991 3.98457835615318e-08
2992 3.98423267711223e-08
2993 3.98190884709493e-08
2994 3.97772801363772e-08
2995 3.96559514115324e-08
2996 3.97320718548144e-08
2997 3.9760148951018e-08
2998 3.97355144343692e-08
2999 3.9716425703773e-08
3000 3.94810264481293e-08
3001 3.98009483149053e-08
3002 3.96598736074338e-08
3003 3.94284356275421e-08
3004 3.96335337882192e-08
3005 3.96083308373818e-08
3006 3.98367099307961e-08
3007 3.94626162858458e-08
3008 3.95299473154864e-08
3009 3.95053696422565e-08
3010 3.94783405965882e-08
3011 3.9434464582655e-08
3012 3.94545978110727e-08
3013 3.93052914660075e-08
3014 3.92547150340761e-08
3015 3.95548838127979e-08
3016 3.93625398942277e-08
3017 3.91977650338049e-08
3018 3.93091177386395e-08
3019 3.93091958983405e-08
3020 3.93085421990236e-08
3021 3.92480963284925e-08
3022 3.92239769553271e-08
3023 3.90697749708124e-08
3024 3.92146120020698e-08
3025 3.91514802799975e-08
3026 3.91482082306993e-08
3027 3.91365624352602e-08
3028 3.90772960656705e-08
3029 3.89721748206284e-08
3030 3.90637246994174e-08
3031 3.90910912528852e-08
3032 3.90273804384833e-08
3033 3.89026020286565e-08
3034 3.89627992092301e-08
3035 3.89932779398805e-08
3036 3.89346688223213e-08
3037 3.89032450698323e-08
3038 3.88917946736456e-08
3039 3.87825629388772e-08
3040 3.89362710961905e-08
3041 3.86599943169585e-08
3042 3.86754379633203e-08
3043 3.86253766748723e-08
3044 3.86263465657066e-08
3045 3.87955729763689e-08
3046 3.85502403332794e-08
3047 3.8535787894034e-08
3048 3.85600102958961e-08
3049 3.83005662740743e-08
3050 3.84078155946099e-08
3051 3.84805183273329e-08
3052 3.84735798775182e-08
3053 3.82573972501632e-08
3054 3.83244902479873e-08
3055 3.81468723276157e-08
3056 3.82341411864218e-08
3057 3.81700608897972e-08
3058 3.82568821066798e-08
3059 3.81369069657467e-08
3060 3.8177418559826e-08
3061 3.80994933379952e-08
3062 3.81160489837384e-08
3063 3.81996372311733e-08
3064 3.81917288905242e-08
3065 3.80961289181414e-08
3066 3.81075437871914e-08
3067 3.79859770305302e-08
3068 3.78873963313708e-08
3069 3.79460800559173e-08
3070 3.80673945699073e-08
3071 3.780360913197e-08
3072 3.80492295448676e-08
3073 3.78607971640577e-08
3074 3.78066999928706e-08
3075 3.78784719146097e-08
3076 3.78938942446894e-08
3077 3.76832112181091e-08
3078 3.78977667025993e-08
3079 3.77377809002155e-08
3080 3.7789650519926e-08
3081 3.77158713149583e-08
3082 3.77840727594503e-08
3083 3.78306808102025e-08
3084 3.76309898797444e-08
3085 3.77037601140273e-08
3086 3.77697233489016e-08
3087 3.76712421257253e-08
3088 3.75922653006455e-08
3089 3.76185980144328e-08
3090 3.77121160965999e-08
3091 3.75664015450639e-08
3092 3.75128976770611e-08
3093 3.75204791680517e-08
3094 3.76296505066875e-08
3095 3.7478987024997e-08
3096 3.75026374399567e-08
3097 3.73689346133688e-08
3098 3.75080730918853e-08
3099 3.74710751316343e-08
3100 3.74369655276041e-08
3101 3.7556379339776e-08
3102 3.74484123710772e-08
3103 3.73489150717887e-08
3104 3.73303983280948e-08
3105 3.74852042739349e-08
3106 3.73971928979699e-08
3107 3.728965936034e-08
3108 3.72945940796399e-08
3109 3.72494604050644e-08
3110 3.72403228254825e-08
3111 3.72077906263257e-08
3112 3.71576334146084e-08
3113 3.71433301893376e-08
3114 3.72203672327487e-08
3115 3.70988146869422e-08
3116 3.70611630273743e-08
3117 3.70661190629562e-08
3118 3.70554325002104e-08
3119 3.71792907571944e-08
3120 3.705775952767e-08
3121 3.69994630489145e-08
3122 3.69771768760074e-08
3123 3.68653481075398e-08
3124 3.69809072253702e-08
3125 3.68780739279373e-08
3126 3.68712775866697e-08
3127 3.67828931757685e-08
3128 3.68154680074895e-08
3129 3.69941410838237e-08
3130 3.67681813884246e-08
3131 3.67340753371082e-08
3132 3.67331693951201e-08
3133 3.67065418060974e-08
3134 3.68464938560464e-08
3135 3.6664836500222e-08
3136 3.66833141640655e-08
3137 3.66468917434304e-08
3138 3.6636244260535e-08
3139 3.67686041613524e-08
3140 3.66097125947817e-08
3141 3.65916896782892e-08
3142 3.65749741604304e-08
3143 3.65594523543677e-08
3144 3.6787135115901e-08
3145 3.65301211502356e-08
3146 3.65199817053963e-08
3147 3.65118530964992e-08
3148 3.65042964745044e-08
3149 3.65822074854805e-08
3150 3.6528092550725e-08
3151 3.65684869052529e-08
3152 3.64837013933084e-08
3153 3.65740469021603e-08
3154 3.63545851200797e-08
3155 3.6477334930396e-08
3156 3.64896663995751e-08
3157 3.64606407288193e-08
3158 3.63942973535814e-08
3159 3.66610777291498e-08
3160 3.63968375438617e-08
3161 3.6397697300572e-08
3162 3.63586494245283e-08
3163 3.63676875281271e-08
3164 3.65355674603052e-08
3165 3.62904124528995e-08
3166 3.63333327868531e-08
3167 3.62931409370049e-08
3168 3.62790508745547e-08
3169 3.62730609992923e-08
3170 3.62528247421778e-08
3171 3.64818646403364e-08
3172 3.6237548073359e-08
3173 3.62207259740899e-08
3174 3.62119578767306e-08
3175 3.621030586487e-08
3176 3.6246927237471e-08
3177 3.61708636376079e-08
3178 3.61518850411358e-08
3179 3.61579601815265e-08
3180 3.61498315726294e-08
3181 3.60870764382071e-08
3182 3.60762442142004e-08
3183 3.60779672803346e-08
3184 3.60306948721245e-08
3185 3.61019303340981e-08
3186 3.62471794801422e-08
3187 3.59231400182125e-08
3188 3.59983083342286e-08
3189 3.59578073982902e-08
3190 3.59634491076122e-08
3191 3.59432483776345e-08
3192 3.61964289652406e-08
3193 3.59089931123435e-08
3194 3.59072913624914e-08
3195 3.58976919301313e-08
3196 3.59328673482651e-08
3197 3.58638168052039e-08
3198 3.58453569049288e-08
3199 3.58229712560387e-08
3200 3.60252165876318e-08
3201 3.57657832239511e-08
3202 3.57637013337353e-08
3203 3.5697826916703e-08
3204 3.57991609689634e-08
3205 3.57157361463578e-08
3206 3.56466003381684e-08
3207 3.56036728987874e-08
3208 3.57424383423677e-08
3209 3.56327802819578e-08
3210 3.55951073061078e-08
3211 3.55893590153755e-08
3212 3.5728486835751e-08
3213 3.55642093552433e-08
3214 3.55570151100437e-08
3215 3.55457991929597e-08
3216 3.56457121597487e-08
3217 3.55231755122531e-08
3218 3.55162477205795e-08
3219 3.55759972592296e-08
3220 3.56143310398238e-08
3221 3.54754554621195e-08
3222 3.55169049726101e-08
3223 3.54149456427422e-08
3224 3.5486742433477e-08
3225 3.54791147572087e-08
3226 3.53788038864877e-08
3227 3.5460871572468e-08
3228 3.55009852626154e-08
3229 3.54208076203122e-08
3230 3.53183011725378e-08
3231 3.53886164816686e-08
3232 3.54155815784907e-08
3233 3.52123592506359e-08
3234 3.53659430629705e-08
3235 3.53344908887721e-08
3236 3.53395499530507e-08
3237 3.52912756795831e-08
3238 3.5288689304025e-08
3239 3.51977682555571e-08
3240 3.53079379067367e-08
3241 3.52350859600392e-08
3242 3.51374360718637e-08
3243 3.52992941543562e-08
3244 3.52454136987035e-08
3245 3.51812126098139e-08
3246 3.5087559524527e-08
3247 3.5177738055836e-08
3248 3.52165017147854e-08
3249 3.50549846928061e-08
3250 3.51463427250565e-08
3251 3.5030254252888e-08
3252 3.51604114712245e-08
3253 3.50890374534174e-08
3254 3.50861313336281e-08
3255 3.50844011620666e-08
3256 3.50876874222195e-08
3257 3.49461828363928e-08
3258 3.50271562865601e-08
3259 3.50092150824821e-08
3260 3.50658666548043e-08
3261 3.49979316638382e-08
3262 3.48971518349117e-08
3263 3.49637936380987e-08
3264 3.50062023812825e-08
3265 3.48874280575728e-08
3266 3.48571376207474e-08
3267 3.48748194767268e-08
3268 3.49409567945713e-08
3269 3.48944659833705e-08
3270 3.48552902096344e-08
3271 3.48800739402577e-08
3272 3.49871669413915e-08
3273 3.48295543517452e-08
3274 3.48334232569414e-08
3275 3.48284032725132e-08
3276 3.49490285600496e-08
3277 3.4705355034248e-08
3278 3.47978925674397e-08
3279 3.47734214756201e-08
3280 3.47275168621763e-08
3281 3.47464670369391e-08
3282 3.47444384374285e-08
3283 3.47197790517839e-08
3284 3.47663693389677e-08
3285 3.47265114442052e-08
3286 3.47167983250074e-08
3287 3.46704069897896e-08
3288 3.47205038053744e-08
3289 3.46488704394687e-08
3290 3.46362298841996e-08
3291 3.46323290045802e-08
3292 3.47615092266551e-08
3293 3.46110553550716e-08
3294 3.45911708166113e-08
3295 3.45790702738213e-08
3296 3.46011432839077e-08
3297 3.45589086236942e-08
3298 3.44595818546622e-08
3299 3.45706467896889e-08
3300 3.45930040168696e-08
3301 3.45058168704782e-08
3302 3.44366029025878e-08
3303 3.45120625411255e-08
3304 3.45222090913921e-08
3305 3.44625874504345e-08
3306 3.44814949926331e-08
3307 3.44666730711651e-08
3308 3.44876056601606e-08
3309 3.44279555974936e-08
3310 3.44163950671827e-08
3311 3.44035342436655e-08
3312 3.45128654544169e-08
3313 3.43762600607533e-08
3314 3.43759687382317e-08
3315 3.43803883140481e-08
3316 3.44777184579925e-08
3317 3.43448895989695e-08
3318 3.43373116606926e-08
3319 3.43158212956496e-08
3320 3.44292416798453e-08
3321 3.42794628238607e-08
3322 3.42796830921088e-08
3323 3.42507391337676e-08
3324 3.43750379272478e-08
3325 3.42342900694348e-08
3326 3.42103945172312e-08
3327 3.41798234160251e-08
3328 3.42755832605235e-08
3329 3.4148754934904e-08
3330 3.41478312293475e-08
3331 3.41252075486409e-08
3332 3.41753434440761e-08
3333 3.41544037496533e-08
3334 3.41078205678969e-08
3335 3.41149046789724e-08
3336 3.4204866494747e-08
3337 3.40655219588371e-08
3338 3.40327197534407e-08
3339 3.40116912411759e-08
3340 3.40423262912282e-08
3341 3.39966987894513e-08
3342 3.40186403491316e-08
3343 3.39749810507328e-08
3344 3.41082859733888e-08
3345 3.39419976569388e-08
3346 3.39750663158611e-08
3347 3.39067831589546e-08
3348 3.40280337240983e-08
3349 3.39046231090379e-08
3350 3.39273356075864e-08
3351 3.38627366147648e-08
3352 3.39908048374582e-08
3353 3.38517054387921e-08
3354 3.38370682584355e-08
3355 3.38017720480366e-08
3356 3.38649677189551e-08
3357 3.3734153248588e-08
3358 3.37231718106068e-08
3359 3.36996883731899e-08
3360 3.37791945526078e-08
3361 3.3640667140844e-08
3362 3.36659020661045e-08
3363 3.36337109274609e-08
3364 3.37419372442582e-08
3365 3.36286944957465e-08
3366 3.36222392149921e-08
3367 3.36012888624282e-08
3368 3.35968266540476e-08
3369 3.36761516450679e-08
3370 3.35656444860888e-08
3371 3.35534870998799e-08
3372 3.35681207275229e-08
3373 3.35627419190132e-08
3374 3.36381091869953e-08
3375 3.35024807895934e-08
3376 3.3506744046008e-08
3377 3.35317231758836e-08
3378 3.34860104089785e-08
3379 3.3471128091378e-08
3380 3.35482717161995e-08
3381 3.3457766335232e-08
3382 3.34488134967614e-08
3383 3.34295791049044e-08
3384 3.34099468091154e-08
3385 3.34887069186607e-08
3386 3.339480869613e-08
3387 3.33815854958175e-08
3388 3.33815002306892e-08
3389 3.33711582811702e-08
3390 3.33794716311786e-08
3391 3.33535545848918e-08
3392 3.34269394386411e-08
3393 3.33719043510428e-08
3394 3.33125562690384e-08
3395 3.32983063344727e-08
3396 3.33006795472102e-08
3397 3.33312257794205e-08
3398 3.32974430250488e-08
3399 3.32523626411785e-08
3400 3.32674865433091e-08
3401 3.32373311096035e-08
3402 3.32406493441795e-08
3403 3.32294440852365e-08
3404 3.3220512563048e-08
3405 3.32036407257874e-08
3406 3.32002976222157e-08
3407 3.31862750613254e-08
3408 3.31770202421922e-08
3409 3.31843494905115e-08
3410 3.3167996349448e-08
3411 3.31417986387805e-08
3412 3.31428253730337e-08
3413 3.31364446992666e-08
3414 3.31038627621183e-08
3415 3.31211502668793e-08
3416 3.30913110246911e-08
3417 3.25085878216669e-08
3418 3.2506275005062e-08
3419 3.23218927178459e-08
3420 3.2394140703218e-08
3421 3.2377172942688e-08
3422 3.22651700912502e-08
3423 3.22394129170789e-08
3424 3.23483106967615e-08
3425 3.21172386463786e-08
3426 3.22827382603919e-08
3427 3.21273958547863e-08
3428 3.21928936841687e-08
3429 3.19496074041581e-08
3430 3.20306519085989e-08
3431 3.21451558704666e-08
3432 3.20128350494997e-08
3433 3.21327888741507e-08
3434 3.21330659858177e-08
3435 3.21050315221783e-08
3436 3.19687956107373e-08
3437 3.1947241296848e-08
3438 3.19709876350771e-08
3439 3.20316217994332e-08
3440 3.19436814777418e-08
3441 3.19167483553429e-08
3442 3.20005106857479e-08
3443 3.18794803888522e-08
3444 3.1966475688705e-08
3445 3.18563131429528e-08
3446 3.1937087641154e-08
3447 3.18308579494442e-08
3448 3.19422177597062e-08
3449 3.18260617859778e-08
3450 3.19301705076214e-08
3451 3.19024948680635e-08
3452 3.18201678339847e-08
3453 3.18889057382421e-08
3454 3.18492041628815e-08
3455 3.19066728593498e-08
3456 3.18163344559252e-08
3457 3.17253032733333e-08
3458 3.18851505198836e-08
3459 3.18614183925092e-08
3460 3.18267332488631e-08
3461 3.1785123866257e-08
3462 3.18690176470682e-08
3463 3.17956505568873e-08
3464 3.18488240225179e-08
3465 3.17552029116541e-08
3466 3.17918527059646e-08
3467 3.17652677495062e-08
3468 3.18207469263143e-08
3469 3.18099004914529e-08
3470 3.1762969143756e-08
3471 3.1733314642679e-08
3472 3.17765866952868e-08
3473 3.17727746335095e-08
3474 3.16999617666625e-08
3475 3.17617470102505e-08
3476 3.17462642840383e-08
3477 3.17333288535337e-08
3478 3.16178727644001e-08
3479 3.169503415279e-08
3480 3.17468540345089e-08
3481 3.16847774683993e-08
3482 3.15778905246589e-08
3483 3.16065218441963e-08
3484 3.16713979486849e-08
3485 3.15372581383144e-08
3486 3.16132258149082e-08
3487 3.15706714104635e-08
3488 3.15686605745213e-08
3489 3.16236992148333e-08
3490 3.14044505955735e-08
3491 3.15258859018286e-08
3492 3.16090478236219e-08
3493 3.15129575767514e-08
3494 3.15417594265455e-08
3495 3.13328776258004e-08
3496 3.1500519526162e-08
3497 3.14978798598986e-08
3498 3.1530593247453e-08
3499 3.1488539775637e-08
3500 3.15168122710929e-08
3501 3.14582919713757e-08
3502 3.14472643481167e-08
3503 3.14853672023219e-08
3504 3.1391127919278e-08
3505 3.14505506082696e-08
3506 3.14445323112977e-08
3507 3.15324832911301e-08
3508 3.11931138696764e-08
3509 3.14388017841338e-08
3510 3.14713375360043e-08
3511 3.14530268497037e-08
3512 3.14738599627162e-08
3513 3.14090726760696e-08
3514 3.14563770587029e-08
3515 3.12314512029843e-08
3516 3.13911456828464e-08
3517 3.11856247492415e-08
3518 3.12167642846362e-08
3519 3.13499519677407e-08
3520 3.12050794093466e-08
3521 3.12289358816997e-08
3522 3.10734300512649e-08
3523 3.10732986008588e-08
3524 3.0986878840622e-08
3525 3.1048543291945e-08
3526 3.10472216824564e-08
3527 3.08747871713422e-08
3528 3.1015094492659e-08
3529 3.10455021690359e-08
3530 3.10176382356531e-08
3531 3.09982901569583e-08
3532 3.09982119972574e-08
3533 3.0908790193962e-08
3534 3.09681063015432e-08
3535 3.09766541306544e-08
3536 3.07780894104326e-08
3537 3.09275449694724e-08
3538 3.0964326214189e-08
3539 3.09587058211491e-08
3540 3.09274099663526e-08
3541 3.08594181319677e-08
3542 3.09394785347195e-08
3543 3.09108507678957e-08
3544 3.09077989868456e-08
3545 3.09072269999433e-08
3546 3.08184162634006e-08
3547 3.0845356491227e-08
3548 3.0891911251274e-08
3549 3.08850189867371e-08
3550 3.0989323107633e-08
3551 3.08470724519339e-08
3552 3.08622425393423e-08
3553 3.08052108266565e-08
3554 3.07665644072586e-08
3555 3.0830999975251e-08
3556 3.08559364725625e-08
3557 3.08281400407395e-08
3558 3.08177661167974e-08
3559 3.07703231783307e-08
3560 3.08208498722706e-08
3561 3.08054630693277e-08
3562 3.07999385995572e-08
3563 3.07824734591122e-08
3564 3.07662197940317e-08
3565 3.07596224047302e-08
3566 3.07728100779059e-08
3567 3.07568797097701e-08
3568 3.07579668401559e-08
3569 3.07328100745963e-08
3570 3.07341885275036e-08
3571 3.06116376691534e-08
3572 3.07033012347802e-08
3573 3.07124849996399e-08
3574 3.07014396128125e-08
3575 3.06953999995585e-08
3576 3.06753733525511e-08
3577 3.06734371235962e-08
3578 3.06523695314809e-08
3579 3.06507921266075e-08
3580 3.06957623763537e-08
3581 3.06314120734896e-08
3582 3.05246672382964e-08
3583 3.05723517612932e-08
3584 3.06238945313453e-08
3585 3.06457259569015e-08
3586 3.06104404046437e-08
3587 3.05945953016362e-08
3588 3.05812868361954e-08
3589 3.05571674630301e-08
3590 3.04809191220556e-08
3591 3.03190610395632e-08
3592 3.02574534316591e-08
3593 3.03640348420231e-08
3594 3.02663529794245e-08
3595 3.0276765983217e-08
3596 3.02737390711627e-08
3597 3.02591338652292e-08
3598 3.02991018941157e-08
3599 3.02330001034079e-08
3600 3.01817948411554e-08
3601 3.03871487972174e-08
3602 3.00654470208883e-08
3603 3.03654807964904e-08
3604 2.99943430093208e-08
3605 2.9995778305647e-08
3606 3.0101087844514e-08
3607 2.99635978251445e-08
3608 2.99636973011275e-08
3609 3.00068769831796e-08
3610 3.00261611130281e-08
3611 2.99445410689714e-08
3612 2.99273317239113e-08
3613 2.99245677126692e-08
3614 2.9899329234695e-08
3615 2.99076319265623e-08
3616 2.99059479402786e-08
3617 2.9875913298838e-08
3618 2.98805851173256e-08
3619 2.99913551771169e-08
3620 2.9858259864568e-08
3621 3.0026793496063e-08
3622 2.98527211839428e-08
3623 2.98377997864918e-08
3624 2.98348510341384e-08
3625 2.98127069697784e-08
3626 2.98113569385805e-08
3627 2.98352951233483e-08
3628 2.99686959692735e-08
3629 2.98337816673211e-08
3630 2.9793245204246e-08
3631 2.98151512367895e-08
3632 2.99118170232759e-08
3633 2.97472784183128e-08
3634 2.97469409105133e-08
3635 2.99202014275579e-08
3636 2.9747132757052e-08
3637 2.97205495769504e-08
3638 2.97373681235058e-08
3639 2.97150872796692e-08
3640 2.97410842620138e-08
3641 2.97098772250592e-08
3642 2.99051450269872e-08
3643 2.96804589794419e-08
3644 2.96847950664869e-08
3645 2.98080387040045e-08
3646 2.96476034833404e-08
3647 2.96574231839486e-08
3648 2.96404021327135e-08
3649 3.10177803442002e-08
3650 2.96899251850391e-08
3651 2.96861522031122e-08
3652 2.96121420717554e-08
3653 3.09828678268786e-08
3654 2.96510940245298e-08
3655 2.95581585874061e-08
3656 2.95695681273855e-08
3657 3.09435002066039e-08
3658 2.96238127361903e-08
3659 2.9526624700793e-08
3660 3.09382031105088e-08
3661 2.9595423001183e-08
3662 2.95112929649122e-08
3663 3.09028429512637e-08
3664 2.94987572146965e-08
3665 2.94856583593628e-08
3666 3.08600540677162e-08
3667 2.97149096439853e-08
3668 2.96747604267011e-08
3669 2.94517601417965e-08
3670 3.08307015473019e-08
3671 2.96901845331377e-08
3672 2.97008551086719e-08
3673 2.94301027992105e-08
3674 3.08231058454567e-08
3675 2.96634041774269e-08
3676 3.07547765032723e-08
3677 2.96268343191741e-08
3678 2.96599669269426e-08
3679 2.93709589982427e-08
3680 3.07696836898685e-08
3681 2.96061877236298e-08
3682 2.96110354014445e-08
3683 3.07097707263893e-08
3684 2.95758102453192e-08
3685 2.95849442721874e-08
3686 2.9599947382053e-08
3687 3.0660032734886e-08
3688 2.95485094170544e-08
3689 2.95174871212112e-08
3690 3.06831999807855e-08
3691 2.96032460767037e-08
3692 2.93552240293593e-08
3693 3.06362508695202e-08
3694 2.9489317654452e-08
3695 2.93338384693698e-08
3696 2.93534530015904e-08
3697 2.95006366002326e-08
3698 3.06132506011636e-08
3699 2.94293442948401e-08
3700 2.93193060940666e-08
3701 3.05737763994784e-08
3702 2.94061255345923e-08
3703 2.92789756883849e-08
3704 3.05974587888613e-08
3705 2.9317654082206e-08
3706 2.92675750301896e-08
3707 3.0574721421317e-08
3708 2.94927211541562e-08
3709 3.05299607816778e-08
3710 2.93499340386916e-08
3711 2.92253528044739e-08
3712 3.05262339850287e-08
3713 2.93119644112494e-08
3714 3.05041005788098e-08
3715 2.92229085374629e-08
3716 3.04624663272079e-08
3717 2.91982331646068e-08
3718 3.04371390313918e-08
3719 2.9377526189478e-08
3720 3.04309715204454e-08
3721 2.91800255070029e-08
3722 3.04397538286594e-08
3723 2.92357977826896e-08
3724 3.04217167013121e-08
3725 2.92225763587339e-08
3726 3.03794571721028e-08
3727 2.92090387432609e-08
3728 3.03496427989103e-08
3729 2.93012902830014e-08
3730 3.03469036566639e-08
3731 2.91742470182044e-08
3732 3.03354035224856e-08
3733 3.02708365040871e-08
3734 2.91387145523458e-08
3735 2.90012653891836e-08
3736 3.03056530981394e-08
3737 3.02155811482407e-08
3738 2.92132913415344e-08
3739 3.02472713542556e-08
3740 3.01864488960746e-08
3741 3.01748990239048e-08
3742 2.9056431927188e-08
3743 3.01954834469598e-08
3744 2.90514634571082e-08
3745 3.01911526889853e-08
3746 3.01915257239216e-08
3747 2.91403523533518e-08
3748 3.01769489396975e-08
3749 3.01456104523368e-08
3750 3.0037256237847e-08
3751 2.90124351209897e-08
3752 2.90165544925003e-08
3753 3.01585565409823e-08
3754 3.01034930316746e-08
3755 3.00635534244975e-08
3756 2.90787056655972e-08
3757 3.00211127068906e-08
3758 2.98919076158199e-08
3759 2.99519768987011e-08
3760 2.98397040410237e-08
3761 3.00198408353936e-08
3762 3.00135312159e-08
3763 2.99742239917578e-08
3764 2.99772437983847e-08
3765 2.99464737452126e-08
3766 2.99433402517479e-08
3767 2.98845108659407e-08
3768 2.98913036544945e-08
3769 2.98819955446561e-08
3770 2.98536271259309e-08
3771 2.97815265781765e-08
3772 2.9893499231548e-08
3773 2.97560625028837e-08
3774 2.97813933514135e-08
3775 2.98586435576453e-08
3776 2.98265732112668e-08
3777 2.96940942945412e-08
3778 2.9947841539979e-08
3779 2.97971247675832e-08
3780 2.97538633731165e-08
3781 2.97370537083452e-08
3782 2.97022886286413e-08
3783 2.96955651180042e-08
3784 2.96765492180384e-08
3785 2.96760021001319e-08
3786 2.96443278813285e-08
3787 2.96489695017499e-08
3788 2.96646174291482e-08
3789 2.96397750787492e-08
3790 2.96225159956975e-08
3791 2.96098843222126e-08
3792 2.95934032834566e-08
3793 2.95957338636299e-08
3794 2.95357907020843e-08
3795 2.95180022646946e-08
3796 2.9508552046309e-08
3797 2.9498453457677e-08
3798 2.9496163733711e-08
3799 2.9474993112899e-08
3800 2.94778459419831e-08
3801 2.94615052354175e-08
3802 2.94465909433939e-08
3803 2.94541688816707e-08
3804 2.94238677867042e-08
3805 2.94493407437812e-08
3806 2.94390058996896e-08
3807 2.94136306422388e-08
3808 2.93393416228582e-08
3809 2.90951085446522e-08
3810 2.91254824702492e-08
3811 2.91750748004915e-08
3812 2.90292874183251e-08
3813 2.90074115838479e-08
3814 2.89959505295201e-08
3815 2.898563167264e-08
3816 2.89641608475222e-08
3817 2.89672090758586e-08
3818 2.89487154248036e-08
3819 2.89431678623941e-08
3820 2.8925351003295e-08
3821 2.89488131244298e-08
3822 2.89353785376534e-08
3823 2.8926136153018e-08
3824 2.89017609844677e-08
3825 2.88649335544733e-08
3826 2.88661787806177e-08
3827 2.88413950499944e-08
3828 2.890339523276e-08
3829 2.87637327289758e-08
3830 2.87398904674774e-08
3831 2.87297545753518e-08
3832 2.87035906154642e-08
3833 2.86874470845078e-08
3834 2.8690408271359e-08
3835 2.86670740479167e-08
3836 2.86793486736769e-08
3837 2.87477490701349e-08
3838 2.8696536702455e-08
3839 2.87359167572276e-08
3840 2.86944388250276e-08
3841 2.87077757121779e-08
3842 2.86599437515633e-08
3843 2.87000965215611e-08
3844 2.86869550336633e-08
3845 2.86899677348629e-08
3846 2.86527530590774e-08
3847 2.86688806028224e-08
3848 2.86616188560629e-08
3849 2.8652735295509e-08
3850 2.86539449945167e-08
3851 2.86691079764978e-08
3852 2.86613470734665e-08
3853 2.86823951256565e-08
3854 2.86594357135073e-08
3855 2.86466708132593e-08
3856 2.86395902548975e-08
3857 2.85901400332023e-08
3858 2.85534742516802e-08
3859 2.85480030726148e-08
3860 2.8524684836384e-08
3861 2.85283991985352e-08
3862 2.85127104149296e-08
3863 2.84878787226717e-08
3864 2.85110210995754e-08
3865 2.84837877728705e-08
3866 2.85202510497129e-08
3867 2.84683938645003e-08
3868 2.84480101697682e-08
3869 2.84598513644596e-08
3870 2.84591354926533e-08
3871 2.8433213117296e-08
3872 2.84386718618634e-08
3873 2.84173591325043e-08
3874 2.8429285592324e-08
3875 2.84172152476003e-08
3876 2.83768315512134e-08
3877 2.83820558166781e-08
3878 2.83942327428122e-08
3879 2.83855108307307e-08
3880 2.83792171984487e-08
3881 2.8345146674269e-08
3882 2.83704402193052e-08
3883 2.83113408272584e-08
3884 2.83266796685666e-08
3885 2.82990466615729e-08
3886 2.83434662406989e-08
3887 2.83101702081012e-08
3888 2.83242318488419e-08
3889 2.82675944873745e-08
3890 2.83216010643628e-08
3891 2.82853953592621e-08
3892 2.83161067926585e-08
3893 2.8251793793288e-08
3894 2.83154335534164e-08
3895 2.83338525974841e-08
3896 2.82852123945077e-08
3897 2.82325451905763e-08
3898 2.82461449785387e-08
3899 2.81739236385192e-08
3900 2.81789169775948e-08
3901 2.81690351044972e-08
3902 2.81670224921982e-08
3903 2.81598211415712e-08
3904 2.81818923753008e-08
3905 2.81235070787034e-08
3906 2.8186878608949e-08
3907 2.81611889363376e-08
3908 2.80838321486954e-08
3909 2.81514189737209e-08
3910 2.81087881859321e-08
3911 2.81111915967358e-08
3912 2.80226455373622e-08
3913 2.81791638911955e-08
3914 2.80512502115471e-08
3915 2.80314438327878e-08
3916 2.797363762852e-08
3917 2.79584853046799e-08
3918 2.80198975133317e-08
3919 2.79344440912155e-08
3920 2.80120548978857e-08
3921 2.80023595422563e-08
3922 2.79005991643544e-08
3923 2.79577783146578e-08
3924 2.79708203265727e-08
3925 2.78770588835187e-08
3926 2.78929661590155e-08
3927 2.79514527079527e-08
3928 2.80439564903645e-08
3929 2.79345169218459e-08
3930 2.78505911666116e-08
3931 2.790069153491e-08
3932 2.78314082891029e-08
3933 2.79152239102132e-08
3934 2.78218053040291e-08
3935 2.78906036044191e-08
3936 2.78260881003689e-08
3937 2.78609597614832e-08
3938 2.78583325297177e-08
3939 2.78106480067208e-08
3940 2.77863598796557e-08
3941 2.79572436312492e-08
3942 2.77688183558666e-08
3943 2.78468270664689e-08
3944 2.77358598310684e-08
3945 2.77424323513742e-08
3946 2.77426916994727e-08
3947 2.77306480001016e-08
3948 2.77422085304124e-08
3949 2.77270189030787e-08
3950 2.77115947966422e-08
3951 2.77998015718595e-08
3952 2.77211267274424e-08
3953 2.77092535583279e-08
3954 2.77012581761937e-08
3955 2.77116818381273e-08
3956 2.76715486080548e-08
3957 2.76659211095875e-08
3958 2.77748597454774e-08
3959 2.76814713373597e-08
3960 2.76509108942946e-08
3961 2.76585261360651e-08
3962 2.76358349537986e-08
3963 2.7634724730774e-08
3964 2.7729969431789e-08
3965 2.76420664135912e-08
3966 2.76052567471652e-08
3967 2.7597184981687e-08
3968 2.75883635936225e-08
3969 2.75794480586455e-08
3970 2.7680947312092e-08
3971 2.76032210422272e-08
3972 2.76031499879537e-08
3973 2.75829332707644e-08
3974 2.75532467952644e-08
3975 2.75536251592712e-08
3976 2.76547300614993e-08
3977 2.75489426826425e-08
3978 2.75719784781359e-08
3979 2.75468536869994e-08
3980 2.75398619464795e-08
3981 2.75510458891404e-08
3982 2.7613056730047e-08
3983 2.75314757658407e-08
3984 2.75260649829079e-08
3985 2.75116729397951e-08
3986 2.75046403430679e-08
3987 2.77607128396085e-08
3988 2.7487097042922e-08
3989 2.77081859678674e-08
3990 2.76381797448266e-08
3991 2.76816400912594e-08
3992 2.76058305104243e-08
3993 2.76973022295124e-08
3994 2.76122609221829e-08
3995 2.75887934719776e-08
3996 2.75920157832843e-08
3997 2.74037557090878e-08
3998 2.76910210317283e-08
3999 2.75809455274612e-08
4000 2.75891736123413e-08
4001 2.76145186717258e-08
4002 2.76541332056013e-08
4003 2.73808193895775e-08
4004 2.73863580702027e-08
4005 2.73978937315178e-08
4006 2.73896407776419e-08
4007 2.76483600458732e-08
4008 2.76514189323507e-08
4009 2.75123657189624e-08
4010 2.75393734483487e-08
4011 2.71977924626299e-08
4012 2.73823115293226e-08
4013 2.70883067088334e-08
4014 2.74016382917353e-08
4015 2.7544336589358e-08
4016 2.76079550332042e-08
4017 2.75297935559138e-08
4018 2.71377604832423e-08
4019 2.75401532690012e-08
4020 2.7236138677722e-08
4021 2.70724118678345e-08
4022 2.67518824870194e-08
4023 2.71416613628617e-08
4024 2.73912608150795e-08
4025 2.74207252459746e-08
4026 2.73727547295266e-08
4027 2.73510281090239e-08
4028 2.67749147297991e-08
4029 2.70982205563541e-08
4030 2.70870135210544e-08
4031 2.74013931544914e-08
4032 2.69732431945613e-08
4033 2.70896709508861e-08
4034 2.7068889352222e-08
4035 2.73918754345459e-08
4036 2.73780411674807e-08
4037 2.70855835537986e-08
4038 2.70593307760691e-08
4039 2.73891362922996e-08
4040 2.70644360256256e-08
4041 2.70701043803001e-08
4042 2.70967390747501e-08
4043 2.70596949292212e-08
4044 2.70923123935063e-08
4045 2.7064642083019e-08
4046 2.70542539482221e-08
4047 2.70261377721681e-08
4048 2.70555009507234e-08
4049 2.70268305513355e-08
4050 2.70249085332352e-08
4051 2.75051235121282e-08
4052 2.74291629409618e-08
4053 2.69218531911974e-08
4054 2.74179612347325e-08
4055 2.69819722120701e-08
4056 2.73811977535843e-08
4057 2.74028728597386e-08
4058 2.73529856542609e-08
4059 2.69518452000739e-08
4060 2.74153997281701e-08
4061 2.68682782689211e-08
4062 2.69523958706941e-08
4063 2.74285909540595e-08
4064 2.69400501906603e-08
4065 2.69469282443424e-08
4066 2.68842192951979e-08
4067 2.68677773362924e-08
4068 2.69559734533686e-08
4069 2.69528221963355e-08
4070 2.73204712186725e-08
4071 2.6914779738263e-08
4072 2.69164779354014e-08
4073 2.72541491597167e-08
4074 2.7366766630621e-08
4075 2.68027608996135e-08
4076 2.682267563614e-08
4077 2.74200964156535e-08
4078 2.68508095757625e-08
4079 2.68880189224774e-08
4080 2.64117812065479e-08
4081 2.68244857437594e-08
4082 2.67315165558557e-08
4083 2.65527067000448e-08
4084 2.72466831319207e-08
4085 2.66814978999719e-08
4086 2.66816986282947e-08
4087 2.67941100418057e-08
4088 2.66671147386432e-08
4089 2.66416400052094e-08
4090 2.67776734119707e-08
4091 2.66381103841695e-08
4092 2.6277136910835e-08
4093 2.68143516279906e-08
4094 2.64880259948086e-08
4095 2.66410786764482e-08
4096 2.66388848757515e-08
4097 2.67613913251807e-08
4098 2.66134190241019e-08
4099 2.66184265740321e-08
4100 2.67003379406106e-08
4101 2.66106905399965e-08
4102 2.65779380725917e-08
4103 2.67011888155366e-08
4104 2.65610182736964e-08
4105 2.65584745307024e-08
4106 2.65576467484152e-08
4107 2.63903423558531e-08
4108 2.61977035620475e-08
4109 2.61807446833018e-08
4110 2.63040540460224e-08
4111 2.61657415734362e-08
4112 2.61441659432649e-08
4113 2.61998156503296e-08
4114 2.61361776665581e-08
4115 2.62115076310465e-08
4116 2.60521737516228e-08
4117 2.60928274542493e-08
4118 2.60893617820557e-08
4119 2.60957673248186e-08
4120 2.61987000982344e-08
4121 2.61708521520632e-08
4122 2.61832102665949e-08
4123 2.6014896903348e-08
4124 2.61934243184214e-08
4125 2.62217092483752e-08
4126 2.61629100606342e-08
4127 2.62055017685725e-08
4128 2.61792738598388e-08
4129 2.6130205554864e-08
4130 2.60868553425553e-08
4131 2.63045745185764e-08
4132 2.61666688317064e-08
4133 2.61988724048479e-08
4134 2.62997819078237e-08
4135 2.60440007338048e-08
4136 2.6274991071773e-08
4137 2.60776324978451e-08
4138 2.62895660796403e-08
4139 2.60380996763843e-08
4140 2.62243595727796e-08
4141 2.61484398578204e-08
4142 2.61934598455582e-08
4143 2.61419170755062e-08
4144 2.62303210263326e-08
4145 2.61228123576984e-08
4146 2.62666883799056e-08
4147 2.61775756627003e-08
4148 2.62994106492442e-08
4149 2.6086139470749e-08
4150 2.6286361531902e-08
4151 2.60854022826607e-08
4152 2.62865356148723e-08
4153 2.60945185459605e-08
4154 2.6289248111766e-08
4155 2.62940957895808e-08
4156 2.61723194228125e-08
4157 2.62002011197637e-08
4158 2.61278110258445e-08
4159 2.62947548179682e-08
4160 2.61613806173955e-08
4161 2.62726214117492e-08
4162 2.61607073781533e-08
4163 2.63046775472731e-08
4164 2.61485357810898e-08
4165 2.62805954776013e-08
4166 2.61217749653042e-08
4167 2.62595580835523e-08
4168 2.61164316839313e-08
4169 2.6211900205908e-08
4170 2.60895482995238e-08
4171 2.61652992605832e-08
4172 2.61821320179934e-08
4173 2.62016595087289e-08
4174 2.61175792104495e-08
4175 2.61275658886007e-08
4176 2.61061856576816e-08
4177 2.61621462271933e-08
4178 2.61053898498176e-08
4179 2.61681893931609e-08
4180 2.60111274741348e-08
4181 2.6152394028145e-08
4182 2.60646384475649e-08
4183 2.61273704893483e-08
4184 2.60973429533351e-08
4185 2.61819401714547e-08
4186 2.60010040165071e-08
4187 2.61459867090252e-08
4188 2.61323300776439e-08
4189 2.60784016603566e-08
4190 2.60376964433817e-08
4191 2.6169681532906e-08
4192 2.58789025764372e-08
4193 2.61238550791631e-08
4194 2.60566768162107e-08
4195 2.58606842606923e-08
4196 2.57334722419955e-08
4197 2.58149466247914e-08
4198 2.57973908901477e-08
4199 2.58315573375967e-08
4200 2.59240167110875e-08
4201 2.5834419048465e-08
4202 2.59626045107098e-08
4203 2.58349430737326e-08
4204 2.59322874285317e-08
4205 2.57113850210544e-08
4206 2.58019419163702e-08
4207 2.57821977101003e-08
4208 2.59253134515802e-08
4209 2.57715431217775e-08
4210 2.58922536744421e-08
4211 2.57802366121496e-08
4212 2.59217642906151e-08
4213 2.5883705845331e-08
4214 2.57546233228823e-08
4215 2.59107704181361e-08
4216 2.58120742557821e-08
4217 2.57198564668215e-08
4218 2.58438603850664e-08
4219 2.58073189485231e-08
4220 2.56894754357972e-08
4221 2.575854196607e-08
4222 2.58252992324515e-08
4223 2.56745948945536e-08
4224 2.58054893009785e-08
4225 2.5718192020463e-08
4226 2.56730103842528e-08
4227 2.57753924870485e-08
4228 2.57910137690942e-08
4229 2.56818228905331e-08
4230 2.58007002429395e-08
4231 2.57248053969761e-08
4232 2.56441712309652e-08
4233 2.57251038249251e-08
4234 2.5770553691018e-08
4235 2.56130956444167e-08
4236 2.57100030154334e-08
4237 2.57420698090982e-08
4238 2.56058871883624e-08
4239 2.57257379843168e-08
4240 2.56408352328208e-08
4241 2.55359111633879e-08
4242 2.56994638903052e-08
4243 2.56670382725588e-08
4244 2.54785295084048e-08
4245 2.56741845561237e-08
4246 2.55773606738785e-08
4247 2.55258196801833e-08
4248 2.5685729099223e-08
4249 2.55452103914422e-08
4250 2.54675125432868e-08
4251 2.5599332431625e-08
4252 2.56785455121644e-08
4253 2.54653702569385e-08
4254 2.55427980988543e-08
4255 2.56273704479781e-08
4256 2.54734899840514e-08
4257 2.56131720277608e-08
4258 2.55252068370737e-08
4259 2.54370569052753e-08
4260 2.55675196569882e-08
4261 2.55285872441391e-08
4262 2.54265870580639e-08
4263 2.54821603817845e-08
4264 2.55587444542016e-08
4265 2.54014800304958e-08
4266 2.55573642249374e-08
4267 2.54882728256689e-08
4268 2.55232173174136e-08
4269 2.53614036438421e-08
4270 2.54693883761092e-08
4271 2.54558294443541e-08
4272 2.5368704470452e-08
4273 2.54924117371047e-08
4274 2.54422598544579e-08
4275 2.54366376850612e-08
4276 2.53340282085901e-08
4277 2.54650363018527e-08
4278 2.54515821751511e-08
4279 2.53242422587618e-08
4280 2.53836738295377e-08
4281 2.54236702801336e-08
4282 2.53036080977154e-08
4283 2.53459404575551e-08
4284 2.54238479158175e-08
4285 2.53837395547407e-08
4286 2.5228487743334e-08
4287 2.53986307541254e-08
4288 2.54138559085959e-08
4289 2.51386644833929e-08
4290 2.53648799741768e-08
4291 2.53539411687598e-08
4292 2.54007854749716e-08
4293 2.52178224968702e-08
4294 2.53310368236725e-08
4295 2.53320795451373e-08
4296 2.51927652072936e-08
4297 2.53627447932558e-08
4298 2.51951135510353e-08
4299 2.53750034318045e-08
4300 2.52178491422228e-08
4301 2.53650274117945e-08
4302 2.53142236061876e-08
4303 2.5338575682099e-08
4304 2.51330742884193e-08
4305 2.52886991347623e-08
4306 2.53069654121418e-08
4307 2.51279317353692e-08
4308 2.51919693994296e-08
4309 2.51042848731231e-08
4310 2.50605864948739e-08
4311 2.49658462792013e-08
4312 2.51534189033009e-08
4313 2.51419507435457e-08
4314 2.49176679290031e-08
4315 2.50650984412459e-08
4316 2.50048568517514e-08
4317 2.50310367988504e-08
4318 2.49416540754055e-08
4319 2.50762735021226e-08
4320 2.49538860686016e-08
4321 2.48673668323818e-08
4322 2.50023486358941e-08
4323 2.49527385420834e-08
4324 2.49788332240541e-08
4325 2.48436116123685e-08
4326 2.49925857787048e-08
4327 2.49264928697812e-08
4328 2.48940494884664e-08
4329 2.49815066410974e-08
4330 2.48889193699142e-08
4331 2.49863294499164e-08
4332 2.48024001336944e-08
4333 2.497049500505e-08
4334 2.49687150954969e-08
4335 2.47590747903814e-08
4336 2.49390659234905e-08
4337 2.49460789802924e-08
4338 2.48448657202971e-08
4339 2.48262637114749e-08
4340 2.49425617937504e-08
4341 2.49224196835485e-08
4342 2.49244394012749e-08
4343 2.48022313797946e-08
4344 2.49518290473816e-08
4345 2.48114062628702e-08
4346 2.48552769477328e-08
4347 2.48177212114342e-08
4348 2.49468428137334e-08
4349 2.47465621328047e-08
4350 2.48976110839294e-08
4351 2.47978881873223e-08
4352 2.4889452276966e-08
4353 2.48053542151183e-08
4354 2.48267202351826e-08
4355 2.47853790824593e-08
4356 2.46801885595005e-08
4357 2.47716744894433e-08
4358 2.4826251276977e-08
4359 2.48112055345473e-08
4360 2.48422686865979e-08
4361 2.48001406077947e-08
4362 2.47444837953026e-08
4363 2.47884468507209e-08
4364 2.47089282368051e-08
4365 2.47886262627617e-08
4366 2.4597515135838e-08
4367 2.47901965622077e-08
4368 2.47417979437614e-08
4369 2.4750983484978e-08
4370 2.46814924054206e-08
4371 2.46889015897978e-08
4372 2.46777780432694e-08
4373 2.46581688401193e-08
4374 2.47153373322817e-08
4375 2.4640176121693e-08
4376 2.45679743215987e-08
4377 2.45852724845008e-08
4378 2.46509905821313e-08
4379 2.45842350921066e-08
4380 2.46988101082479e-08
4381 2.46749500831811e-08
4382 2.46787408286764e-08
4383 2.45722908687185e-08
4384 2.46975915274561e-08
4385 2.46380942314772e-08
4386 2.46311575580194e-08
4387 2.47122802221611e-08
4388 2.4517788688172e-08
4389 2.46607214648975e-08
4390 2.46543763182672e-08
4391 2.46716087559662e-08
4392 2.4634235984422e-08
4393 2.46561473460361e-08
4394 2.46323139663218e-08
4395 2.44816522609881e-08
4396 2.45690756628392e-08
4397 2.45557814082531e-08
4398 2.45599913739625e-08
4399 2.45695215284059e-08
4400 2.4561710887383e-08
4401 2.45629863115937e-08
4402 2.4448105762076e-08
4403 2.45768791984347e-08
4404 2.45354616623672e-08
4405 2.45975169121948e-08
4406 2.45443647628463e-08
4407 2.46395277514466e-08
4408 2.45644891094798e-08
4409 2.45484201855106e-08
4410 2.44586431108473e-08
4411 2.45646578633796e-08
4412 2.4511683349715e-08
4413 2.45013875854738e-08
4414 2.44917419678359e-08
4415 2.44971012364203e-08
4416 2.44770230750646e-08
4417 2.44994069475979e-08
4418 2.4480790727921e-08
4419 2.43713742520413e-08
4420 2.44903279877917e-08
4421 2.44705056218208e-08
4422 2.44691555906229e-08
4423 2.45067557358425e-08
4424 2.44612010646961e-08
4425 2.44648248326484e-08
4426 2.44647839764411e-08
4427 2.44533389093249e-08
4428 2.44751614530969e-08
4429 2.44369111612741e-08
4430 2.44352378331314e-08
4431 2.44337083898927e-08
4432 2.43295712465397e-08
4433 2.44341862298825e-08
4434 2.44307720720371e-08
4435 2.44190889731044e-08
4436 2.4419104960316e-08
4437 2.44200517585114e-08
4438 2.44045779140833e-08
4439 2.43914382025423e-08
4440 2.44028601770196e-08
4441 2.43704967317626e-08
4442 2.440677171478e-08
4443 2.42951117002121e-08
4444 2.43843434333257e-08
4445 2.43808031541448e-08
4446 2.43805295951915e-08
4447 2.43503190944239e-08
4448 2.43792168674872e-08
4449 2.43624551643506e-08
4450 2.43644162623013e-08
4451 2.43447999537239e-08
4452 2.43519711062845e-08
4453 2.42909941050584e-08
4454 2.43347262340876e-08
4455 2.43252138432126e-08
4456 2.43377975550629e-08
4457 2.43265390054148e-08
4458 2.43446471870357e-08
4459 2.43160638291329e-08
4460 2.43335094296526e-08
4461 2.42102089487162e-08
4462 2.43320439352601e-08
4463 2.43286475409832e-08
4464 2.43269742128405e-08
4465 2.43344242534249e-08
4466 2.43277344935677e-08
4467 2.43064199878518e-08
4468 2.42998456911891e-08
4469 2.43567512825393e-08
4470 2.43693563106717e-08
4471 2.43515749787093e-08
4472 2.43788313980531e-08
4473 2.43643523134551e-08
4474 2.42827074004026e-08
4475 2.42639650593901e-08
4476 2.4252551966697e-08
4477 2.42485285184557e-08
4478 2.42430768793156e-08
4479 2.42543158890385e-08
4480 2.42403856987039e-08
4481 2.4323771441459e-08
4482 2.43088784657175e-08
4483 2.42921629478587e-08
4484 2.42203750389081e-08
4485 2.4235772499992e-08
4486 2.42421975826801e-08
4487 2.42106175107892e-08
4488 2.42427002916656e-08
4489 2.42151028118087e-08
4490 2.42022544227893e-08
4491 2.4246844532172e-08
4492 2.4238106632879e-08
4493 2.42596769339798e-08
4494 2.42597266719713e-08
4495 2.42364226465952e-08
4496 2.42531008609603e-08
4497 2.42056632515641e-08
4498 2.42510775905203e-08
4499 2.41473454565266e-08
4500 2.41496618258452e-08
4501 2.41652582388951e-08
4502 2.41458213423584e-08
4503 2.41435511583177e-08
4504 2.41577815529581e-08
4505 2.41327402505931e-08
4506 2.41329303207749e-08
4507 2.4133980147667e-08
4508 2.41219986207852e-08
4509 2.41265318834394e-08
4510 2.41181545845848e-08
4511 2.41198687689348e-08
4512 2.41335200712456e-08
4513 2.41675852663548e-08
4514 2.41081856700021e-08
4515 2.41293918179508e-08
4516 2.41701894054813e-08
4517 2.41350583962685e-08
4518 2.41132127598576e-08
4519 2.41230893038846e-08
4520 2.41151081326052e-08
4521 2.41164848091557e-08
4522 2.39857627093443e-08
4523 2.40192825629038e-08
4524 2.39929072165523e-08
4525 2.3890278200156e-08
4526 2.38604425106814e-08
4527 2.38727260182259e-08
4528 2.3865103671028e-08
4529 2.38727828616447e-08
4530 2.38624640047647e-08
4531 2.38123334383999e-08
4532 2.38590995849108e-08
4533 2.38002684227467e-08
4534 2.38046418132853e-08
4535 2.38355060133699e-08
4536 2.38292656717931e-08
4537 2.36985755464048e-08
4538 2.37919568490952e-08
4539 2.38408333075313e-08
4540 2.37573463124363e-08
4541 2.38230768445646e-08
4542 2.38164510335537e-08
4543 2.43539606259446e-08
4544 2.36825545840702e-08
4545 2.37594584007184e-08
4546 2.36894894811712e-08
4547 2.37732198371532e-08
4548 2.37528841040557e-08
4549 2.36365806927097e-08
4550 2.37773463140911e-08
4551 2.42853896992301e-08
4552 2.36159465316632e-08
4553 2.36567725409031e-08
4554 2.36759181149182e-08
4555 2.42493065627514e-08
4556 2.41003235146309e-08
4557 2.35471109277796e-08
4558 2.40993802691492e-08
4559 2.35867005926593e-08
4560 2.35637891421447e-08
4561 2.41939801526314e-08
4562 2.40753124103321e-08
4563 2.34826842415714e-08
4564 2.4106018514658e-08
4565 2.35470913878544e-08
4566 2.3559030282172e-08
4567 2.40984157073854e-08
4568 2.35700969852815e-08
4569 2.36226878058687e-08
4570 2.41146729251795e-08
4571 2.3552448880082e-08
4572 2.34766250883922e-08
4573 2.41287274604929e-08
4574 2.35508519352834e-08
4575 2.41012863000378e-08
4576 2.40230964010379e-08
4577 2.34090578032919e-08
4578 2.35113635227435e-08
4579 2.40865603018392e-08
4580 2.34854411473862e-08
4581 2.40767974446499e-08
4582 2.39701272164439e-08
4583 2.34223698214464e-08
4584 2.40360904513182e-08
4585 2.39601245510812e-08
4586 2.35688819572033e-08
4587 2.34423218614666e-08
4588 2.39875177499016e-08
4589 2.35791723923739e-08
4590 2.3446155239526e-08
4591 2.3995896825113e-08
4592 2.36036985512555e-08
4593 2.34524790698742e-08
4594 2.37040129746902e-08
4595 2.35719461727513e-08
4596 2.29987549005273e-08
4597 2.31292958119411e-08
4598 2.31482371049196e-08
4599 2.36447856849509e-08
4600 2.35363017964119e-08
4601 2.35070451992669e-08
4602 2.35450627883438e-08
4603 2.35128698733433e-08
4604 2.35042083573944e-08
4605 2.3501060653075e-08
4606 2.34939250276511e-08
4607 2.3488041733799e-08
4608 2.34912924668151e-08
4609 2.34747830063498e-08
4610 2.34719479408341e-08
4611 2.34967263423869e-08
4612 2.29845529275963e-08
4613 2.35198900355726e-08
4614 2.35019719241336e-08
4615 2.34817356670192e-08
4616 2.29743104540603e-08
4617 2.30123635702739e-08
4618 2.30221210983927e-08
4619 2.30321912653153e-08
4620 2.30131540490675e-08
4621 2.3036541563215e-08
4622 2.35323867059378e-08
4623 2.3456120601395e-08
4624 2.29016094976942e-08
4625 2.29091465797637e-08
4626 2.34706600821255e-08
4627 2.34132730980718e-08
4628 2.33892922807399e-08
4629 2.33828227891308e-08
4630 2.3382433766983e-08
4631 2.28909993182924e-08
4632 2.28906351651403e-08
4633 2.29232171022886e-08
4634 2.29261747364262e-08
4635 2.2932171717116e-08
4636 2.3423867290262e-08
4637 2.34220447481448e-08
4638 2.2821099676662e-08
4639 2.33707897479007e-08
4640 2.33596626486587e-08
4641 2.33444676922545e-08
4642 2.32851995463079e-08
4643 2.33265922133796e-08
4644 2.33194068499643e-08
4645 2.32697878743693e-08
4646 2.32724630677694e-08
4647 2.27781278283601e-08
4648 2.27978524947048e-08
4649 2.28379271050017e-08
4650 2.28009859881695e-08
4651 2.28399965607196e-08
4652 2.34113333164032e-08
4653 2.33354899847882e-08
4654 2.32567778368775e-08
4655 2.32249472986723e-08
4656 2.3198747811648e-08
4657 2.31931043259692e-08
4658 2.31737988798386e-08
4659 2.32215153772586e-08
4660 2.31746088985574e-08
4661 2.31669936567869e-08
4662 2.31589876165117e-08
4663 2.27359358007106e-08
4664 2.27293046606292e-08
4665 2.28125678347624e-08
4666 2.28527525791833e-08
4667 2.28047998263037e-08
4668 2.28700383075875e-08
4669 2.28657981438118e-08
4670 2.29615739755218e-08
4671 2.33983108444136e-08
4672 2.33558647977361e-08
4673 2.27720846623924e-08
4674 2.32120473953046e-08
4675 2.28715109074074e-08
4676 2.28518235445563e-08
4677 2.28579288830133e-08
4678 2.33851285003084e-08
4679 2.2887354234058e-08
4680 2.28075016650564e-08
4681 2.29490861869408e-08
4682 2.28596626072886e-08
4683 2.33064874066713e-08
4684 2.2865464188726e-08
4685 2.29882637370338e-08
4686 2.27930367913132e-08
4687 2.28961116732762e-08
4688 2.27593854873476e-08
4689 2.29924523864611e-08
4690 2.30111627530505e-08
4691 2.29951702124254e-08
4692 2.29757013414655e-08
4693 2.30013803559359e-08
4694 2.29991279354635e-08
4695 2.29322072442528e-08
4696 2.29539196539008e-08
4697 2.29452936650887e-08
4698 2.27643983663484e-08
4699 2.28435901306057e-08
4700 2.28157013282271e-08
4701 2.28543655111935e-08
4702 2.2956450962397e-08
4703 2.27112479933567e-08
4704 2.28030714310989e-08
4705 2.282032518508e-08
4706 2.29040075794273e-08
4707 2.26731362573673e-08
4708 2.27750494019574e-08
4709 2.28001439950276e-08
4710 2.28007142055731e-08
4711 2.24514202784576e-08
4712 2.25367973172297e-08
4713 2.25126175479318e-08
4714 2.25420766497564e-08
4715 2.25759428928995e-08
4716 2.25559553257426e-08
4717 2.25589307234486e-08
4718 2.24938059290025e-08
4719 2.25440839329849e-08
4720 2.25787779584152e-08
4721 2.2535621369002e-08
4722 2.24426308648162e-08
4723 2.25406822096375e-08
4724 2.25150298405197e-08
4725 2.25111200791162e-08
4726 2.24824230343756e-08
4727 2.24777370050333e-08
4728 2.24898109024707e-08
4729 2.24777654267427e-08
4730 2.2469654581414e-08
4731 2.24888587752048e-08
4732 2.24543530435994e-08
4733 2.24647287438984e-08
4734 2.24469403065086e-08
4735 2.24509335566836e-08
4736 2.24324061548486e-08
4737 2.24433005513447e-08
4738 2.2413809475097e-08
4739 2.24274092630594e-08
4740 2.23997265180742e-08
4741 2.2419870404633e-08
4742 2.23896794437906e-08
4743 2.23899085938228e-08
4744 2.23524168063705e-08
4745 2.23861835735306e-08
4746 2.24269101067875e-08
4747 2.23979874647284e-08
4748 2.24289475880823e-08
4749 2.23911875707472e-08
4750 2.23891518658093e-08
4751 2.22967297958121e-08
4752 2.23599077031622e-08
4753 2.23620286732285e-08
4754 2.2377696140552e-08
4755 2.22935749860653e-08
4756 2.2374765151767e-08
4757 2.23565059798148e-08
4758 2.23404992283349e-08
4759 2.23313385561141e-08
4760 2.23427463197368e-08
4761 2.23399378995737e-08
4762 2.23282992095619e-08
4763 2.23232667906359e-08
4764 2.2298555890643e-08
4765 2.2351537509735e-08
4766 2.23055813819428e-08
4767 2.23449632130723e-08
4768 2.23040856894841e-08
4769 2.22899334545446e-08
4770 2.23095604212631e-08
4771 2.2282351963554e-08
4772 2.22787210901743e-08
4773 2.22810765393433e-08
4774 2.22686669104633e-08
4775 2.22805986993535e-08
4776 2.22524612070174e-08
4777 2.22687379647368e-08
4778 2.22531610916121e-08
4779 2.22412772643565e-08
4780 2.22741647348812e-08
4781 2.2247366615602e-08
4782 2.22018812223723e-08
4783 2.21827782809214e-08
4784 2.21998455174344e-08
4785 2.22369322955274e-08
4786 2.22127400917316e-08
4787 2.22143690109533e-08
4788 2.22304343822088e-08
4789 2.21894183027871e-08
4790 2.2224238449553e-08
4791 2.21896812035993e-08
4792 2.22473595101746e-08
4793 2.21726388360821e-08
4794 2.22050005049823e-08
4795 2.21560618740568e-08
4796 2.21836469194159e-08
4797 2.2158980428344e-08
4798 2.21602718397662e-08
4799 2.21504237174486e-08
4800 2.21323066540435e-08
4801 2.21288534163477e-08
4802 2.21174261127999e-08
4803 2.20838085596142e-08
4804 2.20941647199879e-08
4805 2.20903846326337e-08
4806 2.20887397262004e-08
4807 2.20928981775614e-08
4808 2.20759872604503e-08
4809 2.20567901720869e-08
4810 2.20639169157266e-08
4811 2.20821956276041e-08
4812 2.20553637575449e-08
4813 2.20423519436963e-08
4814 2.1999992938504e-08
4815 2.20582432319816e-08
4816 2.19206572893427e-08
4817 2.20296847430745e-08
4818 2.19938964818311e-08
4819 2.20352642799071e-08
4820 2.20106901593908e-08
4821 2.19896882924786e-08
4822 2.20121911809201e-08
4823 2.19914131349697e-08
4824 2.19803073520097e-08
4825 2.19320881456042e-08
4826 2.19472831020084e-08
4827 2.19395506206865e-08
4828 2.19700915238263e-08
4829 2.19460698502871e-08
4830 2.19409148627392e-08
4831 2.19408988755276e-08
4832 2.19426450343008e-08
4833 2.17954738701565e-08
4834 2.17635225396862e-08
4835 2.17632667443013e-08
4836 2.17518838496744e-08
4837 2.17692548432069e-08
4838 2.17382307710068e-08
4839 2.17322586593127e-08
4840 2.15982201012821e-08
4841 2.16811280040474e-08
4842 2.17157261062084e-08
4843 2.16666915520136e-08
4844 2.16747242376414e-08
4845 2.16617337400749e-08
4846 2.16558024845881e-08
4847 2.16352127324626e-08
4848 2.17020321713335e-08
4849 2.16769606709022e-08
4850 2.1671432648418e-08
4851 2.16542055397895e-08
4852 2.16372235684048e-08
4853 2.16627036309092e-08
4854 2.16526547802687e-08
4855 2.16302495914533e-08
4856 2.16034763411699e-08
4857 2.15998454677901e-08
4858 2.16528022178863e-08
4859 2.16187903134824e-08
4860 2.16082476356405e-08
4861 2.15084394739051e-08
4862 2.15698801042663e-08
4863 2.15725481922391e-08
4864 2.15540083559063e-08
4865 2.15842330675287e-08
4866 2.15351008137077e-08
4867 2.15261692915192e-08
4868 2.15756301713554e-08
4869 2.1471564082276e-08
4870 2.13989768127476e-08
4871 2.13414956817815e-08
4872 2.13391153636167e-08
4873 2.12499298157809e-08
4874 2.12636539487221e-08
4875 2.12074695582487e-08
4876 2.12213873140854e-08
4877 2.12053894443898e-08
4878 2.12000319521621e-08
4879 2.12288924217319e-08
4880 2.12022577272819e-08
4881 2.11824211504563e-08
4882 2.11705977193333e-08
4883 2.11549640027897e-08
4884 2.11647570580453e-08
4885 2.11257003002174e-08
4886 2.11359907353881e-08
4887 2.11342161549055e-08
4888 2.11232578095633e-08
4889 2.11103916569755e-08
4890 2.10897557195722e-08
4891 2.1179726417131e-08
4892 2.1120650117723e-08
4893 2.11145216866271e-08
4894 2.10956354607106e-08
4895 2.11095390056926e-08
4896 2.10736619266072e-08
4897 2.10915143128432e-08
4898 2.11078710066204e-08
4899 2.10728448024611e-08
4900 2.10534931710527e-08
4901 2.11107078484929e-08
4902 2.11103685643366e-08
4903 2.11321022902666e-08
4904 2.11552322326725e-08
4905 2.11328519128529e-08
4906 2.11373833991502e-08
4907 2.11436859132164e-08
4908 2.11173976083501e-08
4909 2.10364525798923e-08
4910 2.10198898287217e-08
4911 2.10556745372514e-08
4912 2.1061984156745e-08
4913 2.10842650005816e-08
4914 2.10624424568095e-08
4915 2.10787440835247e-08
4916 2.10740171979751e-08
4917 2.10896171637387e-08
4918 2.1078188083834e-08
4919 2.10859472105085e-08
4920 2.10775219500192e-08
4921 2.10736228467567e-08
4922 2.1054805898757e-08
4923 2.10626023289251e-08
4924 2.10456150284699e-08
4925 2.10370387776493e-08
4926 2.10461479355217e-08
4927 2.10298765068728e-08
4928 2.10353210405856e-08
4929 2.10114183829546e-08
4930 2.10045687509819e-08
4931 2.1011292261619e-08
4932 2.10065511652147e-08
4933 2.10008863632538e-08
4934 2.0992112936824e-08
4935 2.10087627294797e-08
4936 2.09903081582752e-08
4937 2.09983905818945e-08
4938 2.09685087071421e-08
4939 2.09613215673699e-08
4940 2.09625827807258e-08
4941 2.09735429024249e-08
4942 2.09556265673427e-08
4943 2.09634745118592e-08
4944 2.09566035636044e-08
4945 2.09614494650623e-08
4946 2.09408295148705e-08
4947 2.09426485042741e-08
4948 2.09464960931882e-08
4949 2.09135855300246e-08
4950 2.09282848828707e-08
4951 2.08993853334505e-08
4952 2.09088231173382e-08
4953 2.09242259074927e-08
4954 2.09216839408555e-08
4955 2.09324557687296e-08
4956 2.08751202990243e-08
4957 2.08777173327235e-08
4958 2.08820409852706e-08
4959 2.08798844880675e-08
4960 2.08797530376614e-08
4961 2.08853787597718e-08
4962 2.08779500354694e-08
4963 2.08826023140318e-08
4964 2.08710275728663e-08
4965 2.08631174558604e-08
4966 2.0854152182892e-08
4967 2.08155075398508e-08
4968 2.08308836846527e-08
4969 2.08337045393137e-08
4970 2.0836985470396e-08
4971 2.08241228705219e-08
4972 2.08558024183958e-08
4973 2.08119388389605e-08
4974 2.08190318318202e-08
4975 2.08205914731252e-08
4976 2.08095478626547e-08
4977 2.08104502519291e-08
4978 2.07965680232292e-08
4979 2.07907255855844e-08
4980 2.07930472839735e-08
4981 2.07898160908826e-08
4982 2.07806891694418e-08
4983 2.07900523463422e-08
4984 2.0771937059294e-08
4985 2.07768771076644e-08
4986 2.07651531525244e-08
4987 2.07307149224789e-08
4988 2.07272243812895e-08
4989 2.07408472618908e-08
4990 2.07267554230839e-08
4991 2.07297574661425e-08
4992 2.07258334938842e-08
4993 2.07248689321204e-08
4994 2.07211119374051e-08
4995 2.07126689133474e-08
4996 2.07162820231588e-08
4997 2.07043253652728e-08
4998 2.06825934156996e-08
4999 2.06646841860447e-08
};
\addlegendentry{Test}

\nextgroupplot[
title={Batch Size 8 $\rare$},
ymin=5.12526233709927e-09, ymax=1e-05,
]
\addplot [semithick, black, dashed]
table {%
0 0.0140922155501321
1 0.00164667243149597
2 0.00065037911516265
3 0.000295232196316647
4 0.000193644413906441
5 0.00017554680505782
6 0.000166322737537485
7 0.000155888844841684
8 0.000142262006907913
9 0.00012419570547172
10 0.000100196558816606
11 7.53355795714015e-05
12 5.41193190329068e-05
13 3.99818541809509e-05
14 3.23710513916922e-05
15 2.83831747833574e-05
16 2.58379168235479e-05
17 2.3768776641873e-05
18 2.16936214364978e-05
19 1.93698575890267e-05
20 1.67099984373635e-05
21 1.38117074528736e-05
22 1.10197760235451e-05
23 8.81342460627366e-06
24 7.32618501214688e-06
25 6.44809396973756e-06
26 5.96682320880859e-06
27 5.69452180099006e-06
28 5.51694411510084e-06
29 5.37627059711099e-06
30 5.25268244874155e-06
31 5.13194294012465e-06
32 5.00552247908104e-06
33 4.87103920659138e-06
34 4.72795897735523e-06
35 4.57529531857404e-06
36 4.41450717971747e-06
37 4.24678919441135e-06
38 4.07240037165479e-06
39 3.89275960455393e-06
40 3.70977897651414e-06
41 3.52646615996832e-06
42 3.3453291746639e-06
43 3.1669885086103e-06
44 2.99429524324069e-06
45 2.82882308749777e-06
46 2.67223944220518e-06
47 2.52574185277865e-06
48 2.39030650546113e-06
49 2.26648519112871e-06
50 2.15448263186602e-06
51 2.05455952405487e-06
52 1.96556939043546e-06
53 1.88664972873198e-06
54 1.81654708246981e-06
55 1.75456746703162e-06
56 1.69953624238417e-06
57 1.65011574182472e-06
58 1.60480939666741e-06
59 1.56276212375417e-06
60 1.52249059640042e-06
61 1.4837170463835e-06
62 1.44544674019187e-06
63 1.40767455231128e-06
64 1.37012475846632e-06
65 1.3339168483526e-06
66 1.29931519345661e-06
67 1.2659092755527e-06
68 1.23280311538565e-06
69 1.20021240133639e-06
70 1.16748595181093e-06
71 1.13551588817273e-06
72 1.10446207130366e-06
73 1.07352417389706e-06
74 1.04239164762987e-06
75 1.01698211097556e-06
76 9.92916835798496e-07
77 9.70485951363287e-07
78 9.49992109070763e-07
79 9.31409722205956e-07
80 9.14265601551279e-07
81 8.97762910810229e-07
82 8.82654424465557e-07
83 8.68559461117968e-07
84 8.55493358947967e-07
85 8.43273811710787e-07
86 8.31858977321076e-07
87 8.20936465963484e-07
88 8.10674030482517e-07
89 8.00620192784152e-07
90 7.91515300498702e-07
91 7.82651789464239e-07
92 7.7411152496154e-07
93 7.66956414445019e-07
94 7.5921274802937e-07
95 7.51383507328285e-07
96 7.44354396381652e-07
97 7.36274504674839e-07
98 7.30296529290797e-07
99 7.22306401783612e-07
100 7.16731141153559e-07
101 7.0944152068364e-07
102 7.04137545035621e-07
103 6.97249927974042e-07
104 6.922896165662e-07
105 6.85678604298801e-07
106 6.79918118848377e-07
107 6.74806719885623e-07
108 6.68366211726834e-07
109 6.63462581393048e-07
110 6.57123162881135e-07
111 6.52642981187057e-07
112 6.46561416587588e-07
113 6.42274032088608e-07
114 6.36438868973244e-07
115 6.32267128040098e-07
116 6.26555323151479e-07
117 6.21985995039154e-07
118 6.16880253367924e-07
119 6.11863263415557e-07
120 6.07826088987906e-07
121 6.02515582706076e-07
122 5.97315892207462e-07
123 5.93690181311501e-07
124 5.87228766487868e-07
125 5.84643050316913e-07
126 5.79739358123277e-07
127 5.74914281592953e-07
128 5.70551833760646e-07
129 5.66140960202688e-07
130 5.61847894573475e-07
131 5.57626214323648e-07
132 5.53413915262979e-07
133 5.49454119457948e-07
134 5.45167176518646e-07
135 5.41143863905802e-07
136 5.37298371497741e-07
137 5.32898270137139e-07
138 5.29843257034201e-07
139 5.25404720370659e-07
140 5.21684867152317e-07
141 5.18092104179857e-07
142 5.13483512598611e-07
143 5.08875732489145e-07
144 5.05149497538326e-07
145 5.01876758473685e-07
146 4.97770456185265e-07
147 4.94343421156174e-07
148 4.90650616956856e-07
149 4.8737844441149e-07
150 4.83746504770011e-07
151 4.80410617424099e-07
152 4.77105002900657e-07
153 4.73949040991073e-07
154 4.70770607652327e-07
155 4.67602424610192e-07
156 4.64573913710353e-07
157 4.61537018406233e-07
158 4.58560601526159e-07
159 4.55594075592813e-07
160 4.52629424799511e-07
161 4.49810896430591e-07
162 4.46874828853083e-07
163 4.44053090966179e-07
164 4.41227867604255e-07
165 4.38426361519362e-07
166 4.35670840431612e-07
167 4.33036380659502e-07
168 4.30322021191643e-07
169 4.27488975983437e-07
170 4.24709639364806e-07
171 4.21982422551537e-07
172 4.19321082635093e-07
173 4.16750839320912e-07
174 4.14194871584783e-07
175 4.1157767810418e-07
176 4.09005205327162e-07
177 4.06454590493155e-07
178 4.03974517769967e-07
179 4.01356386795726e-07
180 3.98910379288964e-07
181 3.96400903827754e-07
182 3.93962698311157e-07
183 3.91538277543901e-07
184 3.89058316706326e-07
185 3.86672758100559e-07
186 3.84376024534561e-07
187 3.82156077666451e-07
188 3.79795015795992e-07
189 3.77472126810829e-07
190 3.75254422486648e-07
191 3.72999132707719e-07
192 3.70814309334833e-07
193 3.68606052909115e-07
194 3.66414772166479e-07
195 3.64246124291867e-07
196 3.62064812147622e-07
197 3.60114379777343e-07
198 3.57652658479424e-07
199 3.55473573875997e-07
200 3.53420759978462e-07
201 3.51302201732295e-07
202 3.49260392450645e-07
203 3.47137964237021e-07
204 3.45095554795449e-07
205 3.43017129246448e-07
206 3.40932977728414e-07
207 3.38848146498094e-07
208 3.36741637614324e-07
209 3.34673315357747e-07
210 3.32618128226159e-07
211 3.30624918607469e-07
212 3.28681106719131e-07
213 3.2678425436572e-07
214 3.24867698765274e-07
215 3.23009222441328e-07
216 3.21105375114428e-07
217 3.19259059818222e-07
218 3.17433803207479e-07
219 3.15584080713194e-07
220 3.13753979209963e-07
221 3.11972391951798e-07
222 3.10204712469897e-07
223 3.08459960738361e-07
224 3.06718977086007e-07
225 3.04951894634087e-07
226 3.03379400019566e-07
227 3.01692094268446e-07
228 3.00041570543641e-07
229 2.98378097685514e-07
230 2.96704027594075e-07
231 2.95068247101682e-07
232 2.93436909618805e-07
233 2.91839041182129e-07
234 2.90256448501225e-07
235 2.88692379278643e-07
236 2.87134213266427e-07
237 2.85589026884026e-07
238 2.84047657121533e-07
239 2.82538714426295e-07
240 2.81039958053952e-07
241 2.79510103869285e-07
242 2.78032811682039e-07
243 2.76594212650139e-07
244 2.7475024603163e-07
245 2.72926708213106e-07
246 2.71423920384706e-07
247 2.7000172076086e-07
248 2.68524305077733e-07
249 2.67011570478815e-07
250 2.65667921034662e-07
251 2.64317999302932e-07
252 2.62892964080663e-07
253 2.6150448949025e-07
254 2.60125897110441e-07
255 2.58924373436997e-07
256 2.57517830519305e-07
257 2.56230467748253e-07
258 2.54854917034919e-07
259 2.53686509193329e-07
260 2.52175349391592e-07
261 2.51046833938062e-07
262 2.49770576434827e-07
263 2.48423187827029e-07
264 2.47144563388702e-07
265 2.45901549853045e-07
266 2.4462504500633e-07
267 2.43350985510205e-07
268 2.42261470663863e-07
269 2.41032267901176e-07
270 2.39729021991764e-07
271 2.38334006315455e-07
272 2.37114604335176e-07
273 2.35892772845858e-07
274 2.34670065896836e-07
275 2.3353716209229e-07
276 2.3238534521397e-07
277 2.31254778810808e-07
278 2.30117605983793e-07
279 2.29054269301088e-07
280 2.27954691752075e-07
281 2.26809562416719e-07
282 2.25817690040486e-07
283 2.24762463858497e-07
284 2.2381367495683e-07
285 2.22726548656738e-07
286 2.21723776196114e-07
287 2.20762035670674e-07
288 2.19673128100339e-07
289 2.18611050652129e-07
290 2.1762178107565e-07
291 2.16632967322994e-07
292 2.15608462355021e-07
293 2.14716398861725e-07
294 2.13740754045233e-07
295 2.12850587857361e-07
296 2.11970101840819e-07
297 2.11049697046661e-07
298 2.1004613132547e-07
299 2.09148392926295e-07
300 2.08267972048759e-07
301 2.0738952559185e-07
302 2.0650375899578e-07
303 2.0562344525743e-07
304 2.04789969075136e-07
305 2.03896006780724e-07
306 2.03049032204916e-07
307 2.02233469991597e-07
308 2.0148574800416e-07
309 2.00686525744231e-07
310 1.99914377798649e-07
311 1.99131872182789e-07
312 1.98344035812426e-07
313 1.97613068664282e-07
314 1.9680639949371e-07
315 1.96114129776603e-07
316 1.95406764222028e-07
317 1.94725376895022e-07
318 1.93973928373836e-07
319 1.93272308163017e-07
320 1.9257287543617e-07
321 1.91904341296123e-07
322 1.91237644701303e-07
323 1.90597502623291e-07
324 1.89894711086325e-07
325 1.8928915396188e-07
326 1.88604807627613e-07
327 1.8794622811491e-07
328 1.87344297193803e-07
329 1.86699944878299e-07
330 1.86075537186525e-07
331 1.85402446655658e-07
332 1.84775635814205e-07
333 1.84167428516346e-07
334 1.83570768633956e-07
335 1.82982711312007e-07
336 1.82418746863533e-07
337 1.81827815703528e-07
338 1.81145456025433e-07
339 1.80619840630669e-07
340 1.80006042857173e-07
341 1.79483078060372e-07
342 1.78932678077004e-07
343 1.7839446073431e-07
344 1.7783793585302e-07
345 1.77330097105965e-07
346 1.76734144670121e-07
347 1.76271198734312e-07
348 1.75766690267309e-07
349 1.75184670666795e-07
350 1.74708623170261e-07
351 1.74188886322924e-07
352 1.73653229948556e-07
353 1.73181383031462e-07
354 1.7267201918969e-07
355 1.72169975215297e-07
356 1.716809444261e-07
357 1.71193413185833e-07
358 1.7070678104858e-07
359 1.70231708297663e-07
360 1.69763145491331e-07
361 1.69323822717615e-07
362 1.68841542647513e-07
363 1.68392174817455e-07
364 1.67920915593811e-07
365 1.67448701848372e-07
366 1.67067471872784e-07
367 1.66522447294426e-07
368 1.66187803337436e-07
369 1.65655882883087e-07
370 1.65297342158866e-07
371 1.64844675270004e-07
372 1.64449132043387e-07
373 1.63934461077986e-07
374 1.63596360001961e-07
375 1.63166515610769e-07
376 1.62640494570709e-07
377 1.62330137088773e-07
378 1.61810680310737e-07
379 1.61525042159383e-07
380 1.61029843134486e-07
381 1.60723079604352e-07
382 1.60256959116367e-07
383 1.59806124516493e-07
384 1.59371342730807e-07
385 1.59003193211049e-07
386 1.58613461659129e-07
387 1.58135951810578e-07
388 1.5777919826121e-07
389 1.57490906419255e-07
390 1.57049220966599e-07
391 1.56678830967039e-07
392 1.56399091556736e-07
393 1.5595287873893e-07
394 1.55577467211288e-07
395 1.55218681813452e-07
396 1.54868281256881e-07
397 1.54552626023374e-07
398 1.54198211124168e-07
399 1.53841266364196e-07
400 1.53452866729964e-07
401 1.53133977093489e-07
402 1.52802964166199e-07
403 1.52480551506073e-07
404 1.52156185608376e-07
405 1.51788253479168e-07
406 1.51498513695003e-07
407 1.51127850402943e-07
408 1.50798184616718e-07
409 1.50510982436458e-07
410 1.50194480964316e-07
411 1.4992832301175e-07
412 1.49619644533416e-07
413 1.49345439441007e-07
414 1.49000281972178e-07
415 1.48698896465405e-07
416 1.48378767759638e-07
417 1.48073516024638e-07
418 1.47793847567357e-07
419 1.47497233387028e-07
420 1.47152927594263e-07
421 1.46810806187503e-07
422 1.46530012960611e-07
423 1.46194294019963e-07
424 1.45904546680953e-07
425 1.45637120098741e-07
426 1.45390437303661e-07
427 1.45085242406751e-07
428 1.44843584227061e-07
429 1.44539950433398e-07
430 1.44255259536763e-07
431 1.43986164975018e-07
432 1.43712213986902e-07
433 1.43442524212389e-07
434 1.43141190728002e-07
435 1.42893512014908e-07
436 1.42647763382797e-07
437 1.42393991835021e-07
438 1.42116199961961e-07
439 1.41860749801381e-07
440 1.41584198152245e-07
441 1.41324433158729e-07
442 1.41084530103086e-07
443 1.40788790234936e-07
444 1.40560038717652e-07
445 1.40289742095234e-07
446 1.40070903469791e-07
447 1.39810986377498e-07
448 1.3955653855291e-07
449 1.39309414347366e-07
450 1.39040228608778e-07
451 1.38806914293355e-07
452 1.38617263488072e-07
453 1.38398338824786e-07
454 1.38211991345116e-07
455 1.37910985833045e-07
456 1.37707504212159e-07
457 1.37452863837595e-07
458 1.37235590139895e-07
459 1.37014003817271e-07
460 1.36758330285147e-07
461 1.36494116027563e-07
462 1.36306082161752e-07
463 1.36068334931849e-07
464 1.35855103517102e-07
465 1.35629422052475e-07
466 1.35411948866349e-07
467 1.35155687953414e-07
468 1.34934801762299e-07
469 1.34734451673779e-07
470 1.34524718356843e-07
471 1.34328436132591e-07
472 1.34129347609147e-07
473 1.33910911843671e-07
474 1.3371256061312e-07
475 1.33461909047838e-07
476 1.33277299719126e-07
477 1.33085340515038e-07
478 1.32889105676881e-07
479 1.32672969037628e-07
480 1.32476193449804e-07
481 1.32274759009121e-07
482 1.32073942387123e-07
483 1.31855791474678e-07
484 1.31636651781619e-07
485 1.31416223496217e-07
486 1.31259498012071e-07
487 1.31016023310337e-07
488 1.30887004155866e-07
489 1.30695356258315e-07
490 1.30501306800923e-07
491 1.30296649251349e-07
492 1.30041917019597e-07
493 1.29918106384963e-07
494 1.29697845192567e-07
495 1.29461063567149e-07
496 1.29204722520981e-07
497 1.29046236825303e-07
498 1.28783909607577e-07
499 1.28579553615893e-07
500 1.28412652287579e-07
501 1.28222582777227e-07
502 1.28080913981421e-07
503 1.27826674720666e-07
504 1.2768455570189e-07
505 1.27511456184948e-07
506 1.27337297175956e-07
507 1.2712039346674e-07
508 1.26940336869552e-07
509 1.26781960378963e-07
510 1.26628617971747e-07
511 1.26438190140021e-07
512 1.26261214711931e-07
513 1.26080714901278e-07
514 1.25882205830763e-07
515 1.25689170451082e-07
516 1.25458669902656e-07
517 1.25292414027811e-07
518 1.25118541685509e-07
519 1.24933668057992e-07
520 1.24744541571076e-07
521 1.24592485745367e-07
522 1.2438628746736e-07
523 1.24239199989162e-07
524 1.24081532147713e-07
525 1.23898961273738e-07
526 1.23711435064067e-07
527 1.23683571343847e-07
528 1.23402232516057e-07
529 1.23180608545326e-07
530 1.22983389335474e-07
531 1.2279557410011e-07
532 1.22695858955524e-07
533 1.2242326393519e-07
534 1.22268119062241e-07
535 1.22114195065137e-07
536 1.2186837597028e-07
537 1.21758532072747e-07
538 1.21501212188235e-07
539 1.21415872920139e-07
540 1.21146897740232e-07
541 1.21149452825264e-07
542 1.20954131578088e-07
543 1.20601052417513e-07
544 1.2042635933085e-07
545 1.20382659668294e-07
546 1.20114501360291e-07
547 1.19902081415546e-07
548 1.19723047657061e-07
549 1.19519860986017e-07
550 1.19464099865851e-07
551 1.19455365275911e-07
552 1.18856987027627e-07
553 1.18855618111979e-07
554 1.18722306197583e-07
555 1.18556314758322e-07
556 1.18349971197418e-07
557 1.18197913097973e-07
558 1.18190822574249e-07
559 1.17993449931575e-07
560 1.17722103356144e-07
561 1.17562712910413e-07
562 1.1739236306596e-07
563 1.17146097792897e-07
564 1.17130330764326e-07
565 1.16876142195466e-07
566 1.16746778818566e-07
567 1.16807582864809e-07
568 1.16375001658398e-07
569 1.16238238121014e-07
570 1.16071960784225e-07
571 1.15917950825661e-07
572 1.15715357653201e-07
573 1.15682018952512e-07
574 1.15512718215527e-07
575 1.15298490829474e-07
576 1.15141919192041e-07
577 1.15017629779501e-07
578 1.14785245981963e-07
579 1.14825111019456e-07
580 1.14467956363384e-07
581 1.14309470399476e-07
582 1.14250206504352e-07
583 1.14062090345257e-07
584 1.14072716723257e-07
585 1.13818667545118e-07
586 1.13543024539808e-07
587 1.13510344416312e-07
588 1.13261899898021e-07
589 1.13188994674829e-07
590 1.12949545007623e-07
591 1.12861762920247e-07
592 1.12966328723374e-07
593 1.12624403906025e-07
594 1.12488853648784e-07
595 1.12194247433806e-07
596 1.12081747023041e-07
597 1.11879273879012e-07
598 1.11686604855166e-07
599 1.1151889847838e-07
600 1.11518363773655e-07
601 1.11229455521666e-07
602 1.11068491328581e-07
603 1.10914127503747e-07
604 1.10845691843053e-07
605 1.10635652848856e-07
606 1.10445679641913e-07
607 1.10333368983362e-07
608 1.10191389911307e-07
609 1.10141222277704e-07
610 1.09968580785491e-07
611 1.10079902308158e-07
612 1.09896487208161e-07
613 1.09495213120425e-07
614 1.09437622697328e-07
615 1.09409666039895e-07
616 1.09359268291698e-07
617 1.0894876301748e-07
618 1.08862762553841e-07
619 1.0888215664373e-07
620 1.08902593192184e-07
621 1.08681523236953e-07
622 1.08473326777769e-07
623 1.08357849692098e-07
624 1.08215046504156e-07
625 1.08083012595017e-07
626 1.07983994128702e-07
627 1.0788382092386e-07
628 1.07778989340446e-07
629 1.07629510504026e-07
630 1.07486674595592e-07
631 1.07359776917448e-07
632 1.07198026283228e-07
633 1.07110649613773e-07
634 1.06957342852709e-07
635 1.06551703673574e-07
636 1.06703632233973e-07
637 1.06261747919945e-07
638 1.06245268353788e-07
639 1.05981521867804e-07
640 1.0613362208467e-07
641 1.05918131012572e-07
642 1.05605284026922e-07
643 1.05518851803765e-07
644 1.05296745056549e-07
645 1.0529636806389e-07
646 1.05115613724394e-07
647 1.04897444288099e-07
648 1.04692954625563e-07
649 1.04653310055269e-07
650 1.04388389308596e-07
651 1.04356357626401e-07
652 1.04121267867185e-07
653 1.03941493522441e-07
654 1.03825485911813e-07
655 1.03654870686753e-07
656 1.03468656945438e-07
657 1.03318351046156e-07
658 1.03309234573246e-07
659 1.03166787249975e-07
660 1.02888646404509e-07
661 1.02764241914244e-07
662 1.02620244419427e-07
663 1.02643449945816e-07
664 1.02388692905464e-07
665 1.02254439710237e-07
666 1.02264427352949e-07
667 1.01974317667342e-07
668 1.01824479767032e-07
669 1.01781015130697e-07
670 1.01579884494996e-07
671 1.01662138689562e-07
672 1.01463956596604e-07
673 1.01146636220406e-07
674 1.01043345342333e-07
675 1.01026062447218e-07
676 1.00765059274366e-07
677 1.00705422818592e-07
678 1.00596716810841e-07
679 1.00494772908633e-07
680 1.00339713535291e-07
681 1.00227720652768e-07
682 1.00199328135986e-07
683 9.99779721606586e-08
684 9.99164768895611e-08
685 9.97651532674837e-08
686 9.96916367359546e-08
687 9.95852735758973e-08
688 9.94225873363064e-08
689 9.93057265326058e-08
690 9.91738692572852e-08
691 9.90683583950158e-08
692 9.89918791169941e-08
693 9.89294559996523e-08
694 9.87324563563874e-08
695 9.8580230650569e-08
696 9.84844026170606e-08
697 9.83782085786089e-08
698 9.82969909344433e-08
699 9.80663861103182e-08
700 9.79155029394718e-08
701 9.78743212769473e-08
702 9.7774189509181e-08
703 9.74998872109722e-08
704 9.7537862576047e-08
705 9.7373689259328e-08
706 9.72140168737923e-08
707 9.71121613257964e-08
708 9.69666632180122e-08
709 9.69298370527838e-08
710 9.67698093869984e-08
711 9.6669013320394e-08
712 9.6617836987889e-08
713 9.65235347489823e-08
714 9.643995225872e-08
715 9.6384693021534e-08
716 9.61863593840206e-08
717 9.60114623218544e-08
718 9.60902926934182e-08
719 9.57746484626654e-08
720 9.57750240715427e-08
721 9.55805155538059e-08
722 9.55347426128128e-08
723 9.54729408171318e-08
724 9.52798347118033e-08
725 9.52589754721345e-08
726 9.51665310111594e-08
727 9.49889552952499e-08
728 9.48892586905004e-08
729 9.48318259261782e-08
730 9.46984865208833e-08
731 9.46578152714039e-08
732 9.44619410319092e-08
733 9.44395689952415e-08
734 9.43228704386456e-08
735 9.42138219972577e-08
736 9.41150636339927e-08
737 9.40356634000494e-08
738 9.39251921092676e-08
739 9.37723432130611e-08
740 9.36953663508433e-08
741 9.36475301536177e-08
742 9.34891315171882e-08
743 9.34929558358277e-08
744 9.32899375643004e-08
745 9.31277580900058e-08
746 9.29872254360475e-08
747 9.29021212048298e-08
748 9.27547188318556e-08
749 9.26676536128212e-08
750 9.25360728327718e-08
751 9.24768817487376e-08
752 9.23224709517001e-08
753 9.2233401355557e-08
754 9.21099720416763e-08
755 9.20273056550513e-08
756 9.18983567661513e-08
757 9.17982077250912e-08
758 9.16838067226422e-08
759 9.1583944425544e-08
760 9.14591718057523e-08
761 9.13134026792051e-08
762 9.12507480110847e-08
763 9.11265217853341e-08
764 9.10621974288262e-08
765 9.08905435199614e-08
766 9.0788907545658e-08
767 9.0717570659038e-08
768 9.05781980975462e-08
769 9.05061573064359e-08
770 9.03604748181408e-08
771 9.02703232927848e-08
772 9.02022289084314e-08
773 9.01490784093184e-08
774 8.99447075433102e-08
775 8.98894601180089e-08
776 8.97487045516954e-08
777 8.9711841648743e-08
778 8.95438815806671e-08
779 8.95039619379645e-08
780 8.93075877943517e-08
781 8.92706220980699e-08
782 8.91444243862338e-08
783 8.90533289634732e-08
784 8.89315970269422e-08
785 8.88538352157298e-08
786 8.87460719418698e-08
787 8.86892839258024e-08
788 8.85925947713417e-08
789 8.84801011071801e-08
790 8.83455844515879e-08
791 8.83479853879265e-08
792 8.81836671791092e-08
793 8.8178019002072e-08
794 8.80678381420808e-08
795 8.79570468814705e-08
796 8.78213846133846e-08
797 8.77077261600689e-08
798 8.7557868305943e-08
799 8.75062553475914e-08
800 8.73836898467317e-08
801 8.73753748438233e-08
802 8.71596677542996e-08
803 8.71341073862553e-08
804 8.69886648713347e-08
805 8.69691683735851e-08
806 8.68006289946877e-08
807 8.67195310281232e-08
808 8.66247496071892e-08
809 8.66069136327141e-08
810 8.63993041644307e-08
811 8.63324029767298e-08
812 8.62302148672001e-08
813 8.61057656562636e-08
814 8.61300473475879e-08
815 8.58610517182612e-08
816 8.5876022253295e-08
817 8.58245899744148e-08
818 8.56204245254233e-08
819 8.56291569748535e-08
820 8.54643526437826e-08
821 8.53652859822418e-08
822 8.5391497251841e-08
823 8.51209773058414e-08
824 8.51588658763447e-08
825 8.50582824751811e-08
826 8.49786741063952e-08
827 8.48149538770215e-08
828 8.48154887371777e-08
829 8.46310325455235e-08
830 8.46116082833248e-08
831 8.44586885744292e-08
832 8.43162639991846e-08
833 8.43594839841089e-08
834 8.41931290338494e-08
835 8.40042691381271e-08
836 8.39640144789655e-08
837 8.39935478929021e-08
838 8.38417056181484e-08
839 8.3761031231866e-08
840 8.36453677903748e-08
841 8.35098131304335e-08
842 8.3516487031865e-08
843 8.3398541787183e-08
844 8.32627191780233e-08
845 8.32743370144939e-08
846 8.31479871719054e-08
847 8.30611524476055e-08
848 8.29425600832323e-08
849 8.28054075387996e-08
850 8.28294606325386e-08
851 8.27534693756959e-08
852 8.26178241633002e-08
853 8.25921608678115e-08
854 8.24915612467336e-08
855 8.23746438047834e-08
856 8.23552042801268e-08
857 8.22746717608069e-08
858 8.21175156664466e-08
859 8.20935714083149e-08
860 8.20074700209616e-08
861 8.18721728084171e-08
862 8.18467062071448e-08
863 8.17524560883243e-08
864 8.16378868124801e-08
865 8.16203907341162e-08
866 8.15071311217608e-08
867 8.14008423501988e-08
868 8.13587263230886e-08
869 8.12950596609241e-08
870 8.11408385228418e-08
871 8.11435524941118e-08
872 8.10090526517371e-08
873 8.09792461664571e-08
874 8.08584962870285e-08
875 8.07094791053231e-08
876 8.07686938379959e-08
877 8.06270915392204e-08
878 8.05262200671564e-08
879 8.04814699018053e-08
880 8.03900555492731e-08
881 8.03408116238913e-08
882 8.02471276806216e-08
883 8.01388805129477e-08
884 8.00124187625428e-08
885 8.00657544806072e-08
886 7.98942147488546e-08
887 7.99531881936488e-08
888 7.97822659910352e-08
889 7.97795756035668e-08
890 7.9595778335495e-08
891 7.94955623648619e-08
892 7.94511712500778e-08
893 7.94300374051815e-08
894 7.93113748613905e-08
895 7.92388654584642e-08
896 7.91362802941009e-08
897 7.90537630148691e-08
898 7.89241159324661e-08
899 7.89327640422499e-08
900 7.87739521186381e-08
901 7.87634696282069e-08
902 7.86302511679438e-08
903 7.85142067414313e-08
904 7.84941220688395e-08
905 7.836267482908e-08
906 7.83394425170059e-08
907 7.82585197347529e-08
908 7.82273210271356e-08
909 7.80942995328715e-08
910 7.80388083398975e-08
911 7.79501971406305e-08
912 7.78230592546336e-08
913 7.77868578483165e-08
914 7.76701064806318e-08
915 7.76812620495448e-08
916 7.75076760506366e-08
917 7.75319989765322e-08
918 7.74967118450931e-08
919 7.73487059220201e-08
920 7.73361544830209e-08
921 7.71912398640495e-08
922 7.71142926412338e-08
923 7.70105475949023e-08
924 7.69942951457381e-08
925 7.68621367575051e-08
926 7.66920159733786e-08
927 7.6753732895618e-08
928 7.66649599208691e-08
929 7.65232764639023e-08
930 7.64789896630091e-08
931 7.63227431415103e-08
932 7.62898755857222e-08
933 7.62348100540322e-08
934 7.61062690850522e-08
935 7.59985500060623e-08
936 7.60535598010037e-08
937 7.59387081998852e-08
938 7.58480387572646e-08
939 7.57766501973123e-08
940 7.56443777145677e-08
941 7.55885656298361e-08
942 7.55352047487889e-08
943 7.55366707450023e-08
944 7.54625999297431e-08
945 7.53840733613842e-08
946 7.52436891762187e-08
947 7.520958560292e-08
948 7.51185533998111e-08
949 7.50705502872151e-08
950 7.49230383103594e-08
951 7.48538313226632e-08
952 7.47696083367444e-08
953 7.4698309575183e-08
954 7.46352352933855e-08
955 7.45532024719608e-08
956 7.45396909715979e-08
957 7.43793192778952e-08
958 7.43306780917052e-08
959 7.42391270653897e-08
960 7.4170969384113e-08
961 7.41744250110088e-08
962 7.39807714547069e-08
963 7.3907636275905e-08
964 7.3864839967186e-08
965 7.38722046689233e-08
966 7.37782955395616e-08
967 7.35998169503205e-08
968 7.36751591370322e-08
969 7.35319244977717e-08
970 7.34493053959895e-08
971 7.33622083135543e-08
972 7.33287411573968e-08
973 7.32267693424049e-08
974 7.3227407262344e-08
975 7.30911046540328e-08
976 7.3059925388641e-08
977 7.29211026548882e-08
978 7.29121208236094e-08
979 7.2779587866556e-08
980 7.27771263386856e-08
981 7.26923735738794e-08
982 7.26410809015476e-08
983 7.25908489931371e-08
984 7.24250146495464e-08
985 7.2414066065285e-08
986 7.23511959135337e-08
987 7.23106430315923e-08
988 7.22220240687577e-08
989 7.21276343647048e-08
990 7.20563575073996e-08
991 7.19845479606462e-08
992 7.19308399528273e-08
993 7.18280783758019e-08
994 7.17602781392657e-08
995 7.16623081844503e-08
996 7.1633209976163e-08
997 7.1613316125152e-08
998 7.14358405220494e-08
999 7.13820533686516e-08
1000 7.13005416006496e-08
1001 7.13117426327514e-08
1002 7.11042069294621e-08
1003 7.11583764241297e-08
1004 7.11103543000746e-08
1005 7.09215685787967e-08
1006 7.08726079992061e-08
1007 7.08819332659871e-08
1008 7.06523474676146e-08
1009 7.06114099751076e-08
1010 7.05629681077013e-08
1011 7.04900955836862e-08
1012 7.03779293322881e-08
1013 7.03613790653535e-08
1014 7.02909734737744e-08
1015 7.02346028056411e-08
1016 7.00847629646617e-08
1017 7.00646524416371e-08
1018 7.00006208278481e-08
1019 6.99058087274551e-08
1020 6.97008041994351e-08
1021 6.96666002131252e-08
1022 6.95315661491946e-08
1023 6.93929467434629e-08
1024 6.93491352254938e-08
1025 6.92687724743735e-08
1026 6.9181248853134e-08
1027 6.90323947898364e-08
1028 6.89361719263815e-08
1029 6.88460079372177e-08
1030 6.88259702794625e-08
1031 6.87216120782708e-08
1032 6.85840478462652e-08
1033 6.85226009862205e-08
1034 6.83794042464214e-08
1035 6.83534466769942e-08
1036 6.82413636488022e-08
1037 6.81742337604874e-08
1038 6.8081035704104e-08
1039 6.80005548439055e-08
1040 6.79186499672468e-08
1041 6.78197289332161e-08
1042 6.77157139037376e-08
1043 6.77006809528535e-08
1044 6.75445062991997e-08
1045 6.74544157597268e-08
1046 6.73693972919054e-08
1047 6.72946742232838e-08
1048 6.71474435236519e-08
1049 6.70465003258514e-08
1050 6.70302539118595e-08
1051 6.69601158707067e-08
1052 6.68708544226959e-08
1053 6.67221369274884e-08
1054 6.66394672963477e-08
1055 6.65038577301047e-08
1056 6.64386373996351e-08
1057 6.6429855464456e-08
1058 6.63477771514209e-08
1059 6.62112182396868e-08
1060 6.61053233539377e-08
1061 6.60354965731358e-08
1062 6.58835158864335e-08
1063 6.58057612259455e-08
1064 6.57003935407019e-08
1065 6.56495188930961e-08
1066 6.55546697672094e-08
1067 6.54752256359359e-08
1068 6.53355711985881e-08
1069 6.52995786598609e-08
1070 6.52046299594034e-08
1071 6.50510368949142e-08
1072 6.50725278443787e-08
1073 6.49379873642886e-08
1074 6.47580108266155e-08
1075 6.46967440784962e-08
1076 6.47110855958033e-08
1077 6.46408487385841e-08
1078 6.45357608251018e-08
1079 6.44439404986485e-08
1080 6.4284627761424e-08
1081 6.42473115846087e-08
1082 6.42346976027497e-08
1083 6.4099857985056e-08
1084 6.39759074045898e-08
1085 6.39553085513e-08
1086 6.39084181939253e-08
1087 6.37432019150452e-08
1088 6.36723728701938e-08
1089 6.35793202752311e-08
1090 6.34713747382776e-08
1091 6.35341950525614e-08
1092 6.33713871458497e-08
1093 6.3277285804908e-08
1094 6.31845582761414e-08
1095 6.31149904926076e-08
1096 6.29787609565113e-08
1097 6.30093429236922e-08
1098 6.28861612330667e-08
1099 6.28055558475893e-08
1100 6.26232452294317e-08
1101 6.25952487345316e-08
1102 6.25822700701661e-08
1103 6.24765654775317e-08
1104 6.23883908117406e-08
1105 6.22942513581748e-08
1106 6.21327299015206e-08
1107 6.20863260349935e-08
1108 6.20415422387666e-08
1109 6.21000680087747e-08
1110 6.18504896578997e-08
1111 6.18067457205385e-08
1112 6.1644829599139e-08
1113 6.15710698008698e-08
1114 6.15897150124667e-08
1115 6.14760737152054e-08
1116 6.14502748659262e-08
1117 6.13122701871305e-08
1118 6.12455619730667e-08
1119 6.11872074065545e-08
1120 6.09162053128998e-08
1121 6.08576885987588e-08
1122 6.08179615930737e-08
1123 6.07086520618694e-08
1124 6.06223983101728e-08
1125 6.05575171297445e-08
1126 6.04104159798169e-08
1127 6.0348267340693e-08
1128 6.02285505344469e-08
1129 6.01458489990492e-08
1130 6.01179870587387e-08
1131 6.0027065746926e-08
1132 5.97797516208587e-08
1133 5.9795340693114e-08
1134 5.96402379002825e-08
1135 5.94761561911739e-08
1136 5.95539313339444e-08
1137 5.94249178975659e-08
1138 5.92505395076159e-08
1139 5.91597576615754e-08
1140 5.90666865125655e-08
1141 5.88915692842917e-08
1142 5.90203142101231e-08
1143 5.87497151824934e-08
1144 5.87532072353625e-08
1145 5.85907652910222e-08
1146 5.8551039190391e-08
1147 5.84408526567159e-08
1148 5.82552799874847e-08
1149 5.81301490427677e-08
1150 5.8109338837653e-08
1151 5.80481490803919e-08
1152 5.79101373157087e-08
1153 5.78085702569453e-08
1154 5.76846728073122e-08
1155 5.76280860729028e-08
1156 5.75032301490808e-08
1157 5.73179610068308e-08
1158 5.72661070288305e-08
1159 5.71943030065469e-08
1160 5.69885338492782e-08
1161 5.69631700626516e-08
1162 5.69194471786716e-08
1163 5.67838021701128e-08
1164 5.66640349766168e-08
1165 5.65589359924346e-08
1166 5.6512820266974e-08
1167 5.63850263919363e-08
1168 5.62500756329243e-08
1169 5.61824805593858e-08
1170 5.60970445784292e-08
1171 5.59599677218969e-08
1172 5.58494863724945e-08
1173 5.57552466800004e-08
1174 5.56996282252697e-08
1175 5.56156806448271e-08
1176 5.5491953369291e-08
1177 5.53300107037913e-08
1178 5.52866972709509e-08
1179 5.5107895852835e-08
1180 5.50283653844019e-08
1181 5.51412584899325e-08
1182 5.48718789001867e-08
1183 5.4910465046909e-08
1184 5.4751843118872e-08
1185 5.4634272609988e-08
1186 5.45681121217889e-08
1187 5.44585529360653e-08
1188 5.43135494264213e-08
1189 5.42500546694136e-08
1190 5.41413124164336e-08
1191 5.40952266248063e-08
1192 5.39357546243124e-08
1193 5.39131514880609e-08
1194 5.39165687554188e-08
1195 5.3634612921627e-08
1196 5.35923273101702e-08
1197 5.35096682368064e-08
1198 5.34264127121098e-08
1199 5.33043504136188e-08
1200 5.31984917513384e-08
1201 5.31392249949469e-08
1202 5.30054221310472e-08
1203 5.30110809604523e-08
1204 5.28841194253893e-08
1205 5.28213592305704e-08
1206 5.27674378001386e-08
1207 5.25633076429166e-08
1208 5.25129558854864e-08
1209 5.2446357926339e-08
1210 5.22644627349855e-08
1211 5.21429107900317e-08
1212 5.21280838032823e-08
1213 5.20980213041255e-08
1214 5.18598821455107e-08
1215 5.19323746939193e-08
1216 5.1801599847856e-08
1217 5.16898352844741e-08
1218 5.14871654297977e-08
1219 5.15548961050882e-08
1220 5.14475486421695e-08
1221 5.13119661880168e-08
1222 5.12700813648515e-08
1223 5.12630925495472e-08
1224 5.11038289716659e-08
1225 5.10397683979313e-08
1226 5.09275734623671e-08
1227 5.09132049710814e-08
1228 5.0759712610926e-08
1229 5.06871001304532e-08
1230 5.06944659219855e-08
1231 5.05630977118976e-08
1232 5.05891275865977e-08
1233 5.04078954683962e-08
1234 5.0369708529896e-08
1235 5.03202734072339e-08
1236 5.03486166767431e-08
1237 5.01460485304861e-08
1238 5.00217914414236e-08
1239 5.00083646550742e-08
1240 4.99302473584429e-08
1241 4.987592799921e-08
1242 4.97896254327834e-08
1243 4.97413813849157e-08
1244 4.96402420528952e-08
1245 4.9622389326931e-08
1246 4.95539652263233e-08
1247 4.94904343217861e-08
1248 4.94052279922386e-08
1249 4.92951875696868e-08
1250 4.93147989524267e-08
1251 4.91164378098041e-08
1252 4.91569000953262e-08
1253 4.91227655383675e-08
1254 4.90244803059703e-08
1255 4.89576740712039e-08
1256 4.8822162876494e-08
1257 4.87467866965297e-08
1258 4.87341117052509e-08
1259 4.87276126515113e-08
1260 4.87346524646881e-08
1261 4.85851170131113e-08
1262 4.84871747534754e-08
1263 4.85831895207234e-08
1264 4.82988585561728e-08
1265 4.83467545695504e-08
1266 4.83161799191834e-08
1267 4.81926020485801e-08
1268 4.81261100500063e-08
1269 4.80794487556224e-08
1270 4.79064465315204e-08
1271 4.78844559088643e-08
1272 4.80212023230564e-08
1273 4.79060066149728e-08
1274 4.7783847305638e-08
1275 4.76988225441843e-08
1276 4.77218500298804e-08
1277 4.76159679054788e-08
1278 4.76466013443755e-08
1279 4.74691612195599e-08
1280 4.75158579309465e-08
1281 4.73825057967225e-08
1282 4.7485986818252e-08
1283 4.72961260644666e-08
1284 4.73102585516472e-08
1285 4.72351163418594e-08
1286 4.72344711912598e-08
1287 4.71515645523724e-08
1288 4.71153546151015e-08
1289 4.68685075079023e-08
1290 4.70706135908827e-08
1291 4.68408979266144e-08
1292 4.68855339947893e-08
1293 4.69027191636329e-08
1294 4.68532288326884e-08
1295 4.66869751258869e-08
1296 4.67236158723239e-08
1297 4.66455751588768e-08
1298 4.6581428891912e-08
1299 4.66710647764046e-08
1300 4.65150191413244e-08
1301 4.6453202255492e-08
1302 4.64446993628798e-08
1303 4.63894488405003e-08
1304 4.62309609279288e-08
1305 4.62655005257773e-08
1306 4.63196391784493e-08
1307 4.61808473293246e-08
1308 4.61542245702162e-08
1309 4.62172598085786e-08
1310 4.61607299291344e-08
1311 4.60153494588056e-08
1312 4.59913277452983e-08
1313 4.59276329918268e-08
1314 4.58297194576573e-08
1315 4.59273223443191e-08
1316 4.58042492530453e-08
1317 4.57084866845037e-08
1318 4.56549604122003e-08
1319 4.56686972283293e-08
1320 4.56331827685119e-08
1321 4.56046227030882e-08
1322 4.5550667001848e-08
1323 4.55369211307399e-08
1324 4.5590441208887e-08
1325 4.54679211445708e-08
1326 4.54479248590545e-08
1327 4.52664653485257e-08
1328 4.5337347214236e-08
1329 4.53551621877324e-08
1330 4.51916858850154e-08
1331 4.51722750653971e-08
1332 4.51394618616874e-08
1333 4.51769472675778e-08
1334 4.51334482649557e-08
1335 4.51266249861249e-08
1336 4.50874914861288e-08
1337 4.49287061963233e-08
1338 4.48967479993456e-08
1339 4.48264219876648e-08
1340 4.4867042350738e-08
1341 4.48034820754728e-08
1342 4.48350610398052e-08
1343 4.48033875457554e-08
1344 4.45712150360933e-08
1345 4.46036571952746e-08
1346 4.46692990454523e-08
1347 4.45814032055125e-08
1348 4.45665362684977e-08
1349 4.45728043025895e-08
1350 4.45147311864957e-08
1351 4.44823261869232e-08
1352 4.43828213363417e-08
1353 4.42768181132536e-08
1354 4.43993598802095e-08
1355 4.44070804617169e-08
1356 4.42842968162438e-08
1357 4.41155342043587e-08
1358 4.41278631075903e-08
1359 4.40644807078172e-08
1360 4.42094289478945e-08
1361 4.40818237334994e-08
1362 4.39918047412391e-08
1363 4.41803186754797e-08
1364 4.40159865835454e-08
1365 4.39419652753514e-08
1366 4.39800256470946e-08
1367 4.38194544480908e-08
1368 4.38346918683052e-08
1369 4.38826430064765e-08
1370 4.38439412224767e-08
1371 4.38131168918332e-08
1372 4.37225367591054e-08
1373 4.36888630588328e-08
1374 4.37150416381371e-08
1375 4.36178424365608e-08
1376 4.36608899700985e-08
1377 4.35089207826422e-08
1378 4.35441604231812e-08
1379 4.35481492102596e-08
1380 4.34936193034474e-08
1381 4.33650798727925e-08
1382 4.34177889596654e-08
1383 4.34463673570917e-08
1384 4.33263139223428e-08
1385 4.33727313220444e-08
1386 4.33343354258042e-08
1387 4.3228653130889e-08
1388 4.31082869178923e-08
1389 4.324872136241e-08
1390 4.30600019534211e-08
1391 4.32475265323973e-08
1392 4.30882091100315e-08
1393 4.30473245973673e-08
1394 4.29888738828765e-08
1395 4.30272572833346e-08
1396 4.29292623698174e-08
1397 4.2927458932418e-08
1398 4.30014096650666e-08
1399 4.30005459408633e-08
1400 4.28852556577652e-08
1401 4.28427411769405e-08
1402 4.29186751995658e-08
1403 4.26496567271784e-08
1404 4.28333187523222e-08
1405 4.27500164761341e-08
1406 4.27967534255558e-08
1407 4.26256254471014e-08
1408 4.26668547057751e-08
1409 4.25627499347492e-08
1410 4.27053181430992e-08
1411 4.24960702827271e-08
1412 4.25527307035267e-08
1413 4.25026517083538e-08
1414 4.24830301199997e-08
1415 4.24793107010046e-08
1416 4.24730674390972e-08
1417 4.23903328332642e-08
1418 4.25100638867804e-08
1419 4.24156434370992e-08
1420 4.23989374351841e-08
1421 4.23957842898837e-08
1422 4.23277789733945e-08
1423 4.23183708635477e-08
1424 4.22206468280173e-08
1425 4.22159647799347e-08
1426 4.22320841448887e-08
1427 4.21671459114314e-08
1428 4.22204840990759e-08
1429 4.21945885649144e-08
1430 4.21728435395785e-08
1431 4.21243299655316e-08
1432 4.20114667738503e-08
1433 4.20510037324462e-08
1434 4.2026753572344e-08
1435 4.19436382799176e-08
1436 4.19389113277546e-08
1437 4.19057039202642e-08
1438 4.18952429424024e-08
1439 4.19827620414814e-08
1440 4.19652518064417e-08
1441 4.18845844243343e-08
1442 4.19072387116692e-08
1443 4.17922589410757e-08
1444 4.16854369809094e-08
1445 4.1889259613459e-08
1446 4.17623865209826e-08
1447 4.17697262604655e-08
1448 4.17888626045304e-08
1449 4.16201110771119e-08
1450 4.17640830239208e-08
1451 4.1783336582224e-08
1452 4.153928639683e-08
1453 4.16737873538686e-08
1454 4.1592476381247e-08
1455 4.15322517932637e-08
1456 4.1644548808506e-08
1457 4.15027544096169e-08
1458 4.148360849765e-08
1459 4.13861753427724e-08
1460 4.14018035996833e-08
1461 4.13481794314663e-08
1462 4.14854741146442e-08
1463 4.13563179764154e-08
1464 4.13556541101201e-08
1465 4.13902445641767e-08
1466 4.1323798298798e-08
1467 4.13480203884653e-08
1468 4.1203276207824e-08
1469 4.1235213055657e-08
1470 4.11417126029434e-08
1471 4.12246168219887e-08
1472 4.11880018207356e-08
1473 4.10834203830035e-08
1474 4.12011794010958e-08
1475 4.10904470413698e-08
1476 4.10302819586761e-08
1477 4.10799662917682e-08
1478 4.10220235567138e-08
1479 4.10007051749872e-08
1480 4.09940534513709e-08
1481 4.09689597491436e-08
1482 4.10433389328446e-08
1483 4.10684464080546e-08
1484 4.0948897821913e-08
1485 4.09499040134875e-08
1486 4.08794801574075e-08
1487 4.09105480914107e-08
1488 4.08452793374536e-08
1489 4.08942971645843e-08
1490 4.07402955646674e-08
1491 4.08870578434417e-08
1492 4.07994149629332e-08
1493 4.08190274594489e-08
1494 4.08339842810079e-08
1495 4.06289100567392e-08
1496 4.07976245524466e-08
1497 4.0624878185902e-08
1498 4.07296249225197e-08
1499 4.06837669739701e-08
1500 4.06156222441112e-08
1501 4.07313091956851e-08
1502 4.05451847482752e-08
1503 4.06221682425212e-08
1504 4.05266821212891e-08
1505 4.04707060539522e-08
1506 4.06462516782113e-08
1507 4.04694767723868e-08
1508 4.04878702155997e-08
1509 4.05097304012614e-08
1510 4.04145102264053e-08
1511 4.0458490955686e-08
1512 4.04152247091005e-08
1513 4.04283759909418e-08
1514 4.04303081200652e-08
1515 4.03291169508435e-08
1516 4.03538379005752e-08
1517 4.04357018073398e-08
1518 4.01908035403409e-08
1519 4.0378684301956e-08
1520 4.01831596095192e-08
1521 4.03242693085559e-08
1522 4.01736299417976e-08
1523 4.02171476885371e-08
1524 4.03139829083798e-08
1525 4.0050756290988e-08
1526 4.01579128519458e-08
1527 4.01703933876618e-08
1528 4.01928873969837e-08
1529 4.01609750353415e-08
1530 4.00341430619733e-08
1531 4.01794336895023e-08
1532 4.00297975486907e-08
1533 4.00986562185679e-08
1534 4.01208160880628e-08
1535 3.99242789086429e-08
1536 4.00895154815117e-08
1537 3.99392733978488e-08
1538 4.00255155073026e-08
1539 3.99330273790355e-08
1540 4.00368776256599e-08
1541 3.98720245327056e-08
1542 3.99692181636269e-08
1543 3.98561598462521e-08
1544 3.98875270821719e-08
1545 3.9898592707921e-08
1546 3.99073250436643e-08
1547 3.99202588603487e-08
1548 3.9863083950209e-08
1549 3.98795083480508e-08
1550 3.98830174299647e-08
1551 3.97511638707826e-08
1552 3.98775771195403e-08
1553 3.97159798799507e-08
1554 3.98367431264646e-08
1555 3.97162449825572e-08
1556 3.97871075605849e-08
1557 3.96644445856964e-08
1558 3.9699416047867e-08
1559 3.97870561856806e-08
1560 3.96381743366092e-08
1561 3.97042046440532e-08
1562 3.96156072683951e-08
1563 3.96154665756043e-08
1564 3.96605191550492e-08
1565 3.958909697932e-08
1566 3.96173939121169e-08
1567 3.949263310421e-08
1568 3.96089839078684e-08
1569 3.95455036246162e-08
1570 3.9636709490587e-08
1571 3.94831752732472e-08
1572 3.94724301058247e-08
1573 3.95125908312366e-08
1574 3.94526429312592e-08
1575 3.95341919947612e-08
1576 3.93538525056414e-08
1577 3.94563540533355e-08
1578 3.93826284383891e-08
1579 3.94670636287842e-08
1580 3.93370668776427e-08
1581 3.93753766632088e-08
1582 3.93889478891296e-08
1583 3.93959847713177e-08
1584 3.93097157589395e-08
1585 3.93532651656869e-08
1586 3.91723971162605e-08
1587 3.9439265921537e-08
1588 3.92176067336436e-08
1589 3.92966383158111e-08
1590 3.93851996420835e-08
1591 3.91521733389411e-08
1592 3.91947608742171e-08
1593 3.92991181339397e-08
1594 3.91132583290599e-08
1595 3.93028898422187e-08
1596 3.90764033886271e-08
1597 3.92399381112796e-08
1598 3.92021937143383e-08
1599 3.90802838641235e-08
1600 3.91499000782503e-08
1601 3.91313884398059e-08
1602 3.90865781119132e-08
1603 3.91318111354622e-08
1604 3.90454653018679e-08
1605 3.90933970058072e-08
1606 3.91323979753722e-08
1607 3.91136423418814e-08
1608 3.90225514497189e-08
1609 3.89170263863647e-08
1610 3.90865480301983e-08
1611 3.8980921593712e-08
1612 3.89813223291569e-08
1613 3.90147075677305e-08
1614 3.89207366335853e-08
1615 3.89198063510676e-08
1616 3.90334313173923e-08
1617 3.88514370586179e-08
1618 3.89586336879688e-08
1619 3.88994614555216e-08
1620 3.89837869310128e-08
1621 3.88157792121646e-08
1622 3.8903180284322e-08
1623 3.88374568531802e-08
1624 3.89662983231176e-08
1625 3.87433496351619e-08
1626 3.88509129178871e-08
1627 3.87580687961631e-08
1628 3.89098066246873e-08
1629 3.86810810075744e-08
1630 3.8847601955716e-08
1631 3.8731702770356e-08
1632 3.88384859340007e-08
1633 3.86796453755167e-08
1634 3.86607207518708e-08
1635 3.86814175348249e-08
1636 3.86742364932857e-08
1637 3.87164597510647e-08
1638 3.86513170154146e-08
1639 3.88603046239666e-08
1640 3.85722263551713e-08
1641 3.86626669151013e-08
1642 3.85764308683534e-08
1643 3.85580688213594e-08
1644 3.86169037662754e-08
1645 3.85148538581959e-08
1646 3.86165236889724e-08
1647 3.84727122453299e-08
1648 3.88424589639058e-08
1649 3.84465692357949e-08
1650 3.85114503771433e-08
1651 3.84911400512777e-08
1652 3.84304911449362e-08
1653 3.85630037955664e-08
1654 3.84011487293279e-08
1655 3.84661550612009e-08
1656 3.85328484400205e-08
1657 3.84369141146479e-08
1658 3.84122835210832e-08
1659 3.8412436121682e-08
1660 3.83894825235487e-08
1661 3.84446199950261e-08
1662 3.8335402538614e-08
1663 3.83735077038594e-08
1664 3.83685282532298e-08
1665 3.83480501557898e-08
1666 3.82956904871889e-08
1667 3.84011462086775e-08
1668 3.83502299774463e-08
1669 3.83458634498624e-08
1670 3.8245394529568e-08
1671 3.83146489473241e-08
1672 3.82573301918043e-08
1673 3.82244613819083e-08
1674 3.8246418380794e-08
1675 3.82150598499109e-08
1676 3.82407916803551e-08
1677 3.82599868506972e-08
1678 3.8215464568836e-08
1679 3.81728400284942e-08
1680 3.83210382013388e-08
1681 3.80955480121514e-08
1682 3.81806894989012e-08
1683 3.82224091151073e-08
1684 3.81114148124695e-08
1685 3.82258913358413e-08
1686 3.82052000684752e-08
1687 3.81070638280079e-08
1688 3.8081548687785e-08
1689 3.81398353761497e-08
1690 3.81179427018097e-08
1691 3.80720296826453e-08
1692 3.80211410555553e-08
1693 3.81585196489453e-08
1694 3.80863659596997e-08
1695 3.7976557523578e-08
1696 3.80630843066498e-08
1697 3.81245449689871e-08
1698 3.80392725380929e-08
1699 3.801984486973e-08
1700 3.79566180743751e-08
1701 3.79708953985869e-08
1702 3.80453341737308e-08
1703 3.79378156063481e-08
1704 3.79759017397063e-08
1705 3.79490948132499e-08
1706 3.80014099050641e-08
1707 3.78876995812405e-08
1708 3.79235057170746e-08
1709 3.79901633928981e-08
1710 3.79401265870882e-08
1711 3.78765139581461e-08
1712 3.78788724297863e-08
1713 3.79089320423631e-08
1714 3.79844686686504e-08
1715 3.79382417801111e-08
1716 3.79571741651041e-08
1717 3.79882111483099e-08
1718 3.78223503592068e-08
1719 3.7959839022772e-08
1720 3.76949858327258e-08
1721 3.80500853780497e-08
1722 3.77684928363209e-08
1723 3.79319000765044e-08
1724 3.78127555658025e-08
1725 3.76808580231369e-08
1726 3.78913456544616e-08
1727 3.76658090703863e-08
1728 3.7851494917529e-08
1729 3.77664108097697e-08
1730 3.79108147838814e-08
1731 3.76634401306752e-08
1732 3.78151026674267e-08
1733 3.78816969464069e-08
1734 3.77709782481972e-08
1735 3.77660620167752e-08
1736 3.76888900031069e-08
1737 3.77818670016516e-08
1738 3.78022734786043e-08
1739 3.77456426572387e-08
1740 3.76173546103864e-08
1741 3.76604630600852e-08
1742 3.77668573965373e-08
1743 3.77400727162858e-08
1744 3.77201424601736e-08
1745 3.77242986160375e-08
1746 3.75634211788878e-08
1747 3.74809592673664e-08
1748 3.7722135395768e-08
1749 3.76457967710131e-08
1750 3.76064268419185e-08
1751 3.76065779379431e-08
1752 3.74286121975764e-08
1753 3.77629903707266e-08
1754 3.76661396659372e-08
1755 3.74908666405105e-08
1756 3.76121619192205e-08
1757 3.75940156729371e-08
1758 3.76416019105541e-08
1759 3.74089387640275e-08
1760 3.75451150600448e-08
1761 3.74251382719848e-08
1762 3.73434886444812e-08
1763 3.76087765929789e-08
1764 3.75553967626452e-08
1765 3.73977909142731e-08
1766 3.74640324758424e-08
1767 3.73550987626814e-08
1768 3.74835799159534e-08
1769 3.73618815268095e-08
1770 3.75936487801987e-08
1771 3.72713335470287e-08
1772 3.75196970185954e-08
1773 3.72949849896109e-08
1774 3.7603680296705e-08
1775 3.72440201465984e-08
1776 3.75571327109192e-08
1777 3.72496371237041e-08
1778 3.74038666710597e-08
1779 3.74239376252916e-08
1780 3.74428224656498e-08
1781 3.7323855651028e-08
1782 3.74017605850874e-08
1783 3.72666274963684e-08
1784 3.73440678775872e-08
1785 3.73906923156753e-08
1786 3.73588830671068e-08
1787 3.71107454029129e-08
1788 3.72834189805715e-08
1789 3.73839526646158e-08
1790 3.74207804099136e-08
1791 3.72704231348386e-08
1792 3.72604873839499e-08
1793 3.72039002338731e-08
1794 3.72388135945201e-08
1795 3.73753077895778e-08
1796 3.72486858588594e-08
1797 3.72918126916311e-08
1798 3.72434037134717e-08
1799 3.72466739793076e-08
1800 3.72663990639843e-08
1801 3.71341596454577e-08
1802 3.7279309918592e-08
1803 3.72007997335722e-08
1804 3.72522711966639e-08
1805 3.71977966189263e-08
1806 3.71243192054393e-08
1807 3.71166673645007e-08
1808 3.69281813226152e-08
1809 3.7265723722868e-08
1810 3.72295604456063e-08
1811 3.7045100355293e-08
1812 3.71322109171679e-08
1813 3.71508310981206e-08
1814 3.71456299328443e-08
1815 3.71071213347562e-08
1816 3.71199989213089e-08
1817 3.70692861344502e-08
1818 3.71239575356341e-08
1819 3.70205668143164e-08
1820 3.70877458459873e-08
1821 3.69541319535926e-08
1822 3.70777116702747e-08
1823 3.67921550732397e-08
1824 3.7050849953868e-08
1825 3.70229679376166e-08
1826 3.69968310884872e-08
1827 3.71215581407291e-08
1828 3.67214813636885e-08
1829 3.70434229832739e-08
1830 3.67917332182444e-08
1831 3.71148817803757e-08
1832 3.69631161016848e-08
1833 3.67094972708593e-08
1834 3.69223256053708e-08
1835 3.67426318259589e-08
1836 3.70516558798606e-08
1837 3.67933732445813e-08
1838 3.69378140621102e-08
1839 3.68091435007933e-08
1840 3.68699657187221e-08
1841 3.67680227189027e-08
1842 3.67932825673378e-08
1843 3.66279344761189e-08
1844 3.70051625235845e-08
1845 3.66218960352604e-08
1846 3.71567434198639e-08
1847 3.67378085286418e-08
1848 3.67072405182967e-08
1849 3.68038352065447e-08
1850 3.67439507895639e-08
1851 3.66124083939212e-08
1852 3.67425429224077e-08
1853 3.67981550968288e-08
1854 3.66166572089988e-08
1855 3.68664959489173e-08
1856 3.65924359608805e-08
1857 3.66692661883938e-08
1858 3.66733336858438e-08
1859 3.67367757094783e-08
1860 3.65539251201419e-08
1861 3.67730002568401e-08
1862 3.65517606497612e-08
1863 3.67414380395559e-08
1864 3.65365003003326e-08
1865 3.66813095511453e-08
1866 3.65326868894122e-08
1867 3.68337799656615e-08
1868 3.66286341191291e-08
1869 3.65044348935584e-08
1870 3.66127535955663e-08
1871 3.65746113235588e-08
1872 3.65417026371162e-08
1873 3.67053717078569e-08
1874 3.6658561624936e-08
1875 3.65626657981011e-08
1876 3.65201574044072e-08
1877 3.67783549450884e-08
1878 3.66494288788211e-08
1879 3.66888207783411e-08
1880 3.65948743836775e-08
1881 3.65293158486324e-08
1882 3.6500014441998e-08
1883 3.64699401549373e-08
1884 3.64751547068387e-08
1885 3.65656375094225e-08
1886 3.65746387260835e-08
1887 3.66372703726192e-08
1888 3.64976410645035e-08
1889 3.65722141060232e-08
1890 3.64866452176038e-08
1891 3.65375467361595e-08
1892 3.65167638012309e-08
1893 3.63294874006215e-08
1894 3.63945058015069e-08
1895 3.6536670052989e-08
1896 3.6430282676303e-08
1897 3.62453244746597e-08
1898 3.66222929399918e-08
1899 3.64198722717646e-08
1900 3.63831767380418e-08
1901 3.63298386836242e-08
1902 3.61740805980837e-08
1903 3.64556771561553e-08
1904 3.63519154173986e-08
1905 3.62696372944171e-08
1906 3.61171992451226e-08
1907 3.64603584808165e-08
1908 3.61317012207429e-08
1909 3.62745675124287e-08
1910 3.64189258177383e-08
1911 3.6266710974342e-08
1912 3.62556905648681e-08
1913 3.6238342407291e-08
1914 3.62670215623417e-08
1915 3.60366312679439e-08
1916 3.63011385027256e-08
1917 3.61772445773845e-08
1918 3.61867287574924e-08
1919 3.61102679971026e-08
1920 3.63929243683003e-08
1921 3.60180869836135e-08
1922 3.61357699847353e-08
1923 3.61040660279421e-08
1924 3.6139368550181e-08
1925 3.60739116982423e-08
1926 3.6196284967982e-08
1927 3.59936592344567e-08
1928 3.60563996575358e-08
1929 3.60782758486167e-08
1930 3.60398286818331e-08
1931 3.60291056034079e-08
1932 3.60345146028784e-08
1933 3.59823884230615e-08
1934 3.60840954800601e-08
1935 3.59893449890514e-08
1936 3.59963851610523e-08
1937 3.62121882626632e-08
1938 3.59255595241414e-08
1939 3.60697188535042e-08
1940 3.59639030174108e-08
1941 3.61143853102597e-08
1942 3.58569910274831e-08
1943 3.6043464469504e-08
1944 3.58307471461927e-08
1945 3.60832720742721e-08
1946 3.58909315378853e-08
1947 3.59922691823833e-08
1948 3.588628298834e-08
1949 3.57991387196499e-08
1950 3.59611625162515e-08
1951 3.59351795946594e-08
1952 3.58898386458861e-08
1953 3.57527423555659e-08
1954 3.5908256137418e-08
1955 3.57621854756296e-08
1956 3.58349212685738e-08
1957 3.58016397492555e-08
1958 3.59061841495034e-08
1959 3.58764264776212e-08
1960 3.59239804565981e-08
1961 3.56813113904231e-08
1962 3.58514508875807e-08
1963 3.57986901380336e-08
1964 3.57795846923636e-08
1965 3.58297134064323e-08
1966 3.58044209156638e-08
1967 3.57522844649516e-08
1968 3.57979316274459e-08
1969 3.57875962246901e-08
1970 3.58005632263847e-08
1971 3.57308950551527e-08
1972 3.57868481506429e-08
1973 3.57633905760935e-08
1974 3.59752080516529e-08
1975 3.54793153327648e-08
1976 3.56424363725516e-08
1977 3.56962309773223e-08
1978 3.57075252499506e-08
1979 3.5845165480719e-08
1980 3.56351077415162e-08
1981 3.56275620205793e-08
1982 3.56691139065113e-08
1983 3.5589426081728e-08
1984 3.56025159495843e-08
1985 3.58654780168166e-08
1986 3.56388935309759e-08
1987 3.5567677426851e-08
1988 3.56262408662822e-08
1989 3.56126999379036e-08
1990 3.57467316645099e-08
1991 3.55701531344899e-08
1992 3.57601672029695e-08
1993 3.55033272736449e-08
1994 3.57265346040414e-08
1995 3.5396074536731e-08
1996 3.54844572565405e-08
1997 3.57656564653475e-08
1998 3.53924049982801e-08
1999 3.55584950182397e-08
2000 3.56697449870325e-08
2001 3.56787138220405e-08
2002 3.55943911167778e-08
2003 3.54479785218409e-08
2004 3.56183040364222e-08
2005 3.53309536897939e-08
2006 3.5582972430781e-08
2007 3.55872772117571e-08
2008 3.56063758508718e-08
2009 3.55415195936182e-08
2010 3.55386896822019e-08
2011 3.55435800902804e-08
2012 3.54341543582493e-08
2013 3.5326350536824e-08
2014 3.54429674618295e-08
2015 3.56998978769951e-08
2016 3.54859376350269e-08
2017 3.53400733024145e-08
2018 3.55125124622546e-08
2019 3.53422700452022e-08
2020 3.55198382075983e-08
2021 3.53294625350387e-08
2022 3.53245762756416e-08
2023 3.54871031174042e-08
2024 3.55245614671595e-08
2025 3.52888825934095e-08
2026 3.55585673830205e-08
2027 3.54134896176639e-08
2028 3.53721047030575e-08
2029 3.54860238958032e-08
2030 3.54437497889215e-08
2031 3.55012276966882e-08
2032 3.53924766249847e-08
2033 3.53858738901813e-08
2034 3.53863012767519e-08
2035 3.54097912538265e-08
2036 3.52552421158947e-08
2037 3.54500118371348e-08
2038 3.53148007081749e-08
2039 3.51771727915562e-08
2040 3.52485998993402e-08
2041 3.52664494878141e-08
2042 3.53212110244527e-08
2043 3.52696402448061e-08
2044 3.53278196350004e-08
2045 3.50798110102524e-08
2046 3.53038244638171e-08
2047 3.50380686655605e-08
2048 3.53732404003715e-08
2049 3.52515104800055e-08
2050 3.52898717146388e-08
2051 3.51912950766753e-08
2052 3.52262799583336e-08
2053 3.52031993360313e-08
2054 3.51986800883886e-08
2055 3.49191987663033e-08
2056 3.50503156130166e-08
2057 3.50460932994068e-08
2058 3.51862639345057e-08
2059 3.54123910732973e-08
2060 3.5012984096916e-08
2061 3.5237672851629e-08
2062 3.50989451058936e-08
2063 3.51451456923613e-08
2064 3.50956153982729e-08
2065 3.51490209666849e-08
2066 3.51532928246634e-08
2067 3.50932326336206e-08
2068 3.50858328599379e-08
2069 3.50466975369557e-08
2070 3.5128578741439e-08
2071 3.50624365870189e-08
2072 3.51087634351543e-08
2073 3.50725950184483e-08
2074 3.48720382215006e-08
2075 3.50937104562909e-08
2076 3.50986688211208e-08
2077 3.49998700812648e-08
2078 3.50636079833322e-08
2079 3.50096038435055e-08
2080 3.4841453009804e-08
2081 3.50405389926145e-08
2082 3.50292538255914e-08
2083 3.49853204753003e-08
2084 3.50112577467421e-08
2085 3.49354300377414e-08
2086 3.49946228852538e-08
2087 3.47063039427553e-08
2088 3.50340738730637e-08
2089 3.49463078346268e-08
2090 3.48853253449022e-08
2091 3.50643476894064e-08
2092 3.48874224593843e-08
2093 3.4948104152388e-08
2094 3.49539611530503e-08
2095 3.47071131532317e-08
2096 3.48301070332013e-08
2097 3.48174625681708e-08
2098 3.48316413565364e-08
2099 3.46571736122847e-08
2100 3.48714099800418e-08
2101 3.47187161162665e-08
2102 3.47542465775064e-08
2103 3.48218386880816e-08
2104 3.46967600517445e-08
2105 3.48042466460363e-08
2106 3.46572571419124e-08
2107 3.46827273878247e-08
2108 3.47622488425792e-08
2109 3.47469773482878e-08
2110 3.47767646564634e-08
2111 3.47355067442656e-08
2112 3.4546701973337e-08
2113 3.47556978592678e-08
2114 3.46318142816493e-08
2115 3.46471681780258e-08
2116 3.47924431811641e-08
2117 3.46481810264976e-08
2118 3.44733216182114e-08
2119 3.46518337575041e-08
2120 3.46098472623346e-08
2121 3.45996842279206e-08
2122 3.47283432957646e-08
2123 3.46137393982815e-08
2124 3.45159064116807e-08
2125 3.46012808627449e-08
2126 3.44771602875937e-08
2127 3.45803697943126e-08
2128 3.44515754568953e-08
2129 3.45153512402163e-08
2130 3.45798061651692e-08
2131 3.45268843418012e-08
2132 3.44025088856448e-08
2133 3.45208335361669e-08
2134 3.4445654149895e-08
2135 3.45518178868076e-08
2136 3.45473493346127e-08
2137 3.4456031099861e-08
2138 3.4364484506888e-08
2139 3.44972221721918e-08
2140 3.43248367657978e-08
2141 3.4492944448683e-08
2142 3.43385404715235e-08
2143 3.45781598660722e-08
2144 3.4290496093714e-08
2145 3.43060144554208e-08
2146 3.44058043646456e-08
2147 3.43255302404088e-08
2148 3.41100668381777e-08
2149 3.41952895204223e-08
2150 3.43601639203328e-08
2151 3.4153532772585e-08
2152 3.41090135576039e-08
2153 3.4155436126504e-08
2154 3.42936208577171e-08
2155 3.3856299904933e-08
2156 3.42150716563516e-08
2157 3.3988354551262e-08
2158 3.405076265528e-08
2159 3.38644607005278e-08
2160 3.41631483760096e-08
2161 3.39845888581713e-08
2162 3.38284178185155e-08
2163 3.38537514537052e-08
2164 3.40268143110833e-08
2165 3.39710133356874e-08
2166 3.3770219998619e-08
2167 3.40399171161465e-08
2168 3.39991786018068e-08
2169 3.38950475993194e-08
2170 3.36694171254592e-08
2171 3.35623848246591e-08
2172 3.358883385296e-08
2173 3.35593926847899e-08
2174 3.36249531995847e-08
2175 3.35927736521136e-08
2176 3.36960097553352e-08
2177 3.3357819268609e-08
2178 3.34103945465181e-08
2179 3.33792154614265e-08
2180 3.33711916660206e-08
2181 3.33325034604925e-08
2182 3.34902264049752e-08
2183 3.34050771559902e-08
2184 3.35643477549219e-08
2185 3.31968241833458e-08
2186 3.31900424805909e-08
2187 3.30820723366543e-08
2188 3.33634762741397e-08
2189 3.29869779860381e-08
2190 3.29270568819595e-08
2191 3.29535492369359e-08
2192 3.30423635066524e-08
2193 3.2968385207166e-08
2194 3.29141850103909e-08
2195 3.27679847420548e-08
2196 3.26790389784115e-08
2197 3.27125180756838e-08
2198 3.28457829645856e-08
2199 3.26832230150842e-08
2200 3.25508643124195e-08
2201 3.25069563849034e-08
2202 3.25951758823884e-08
2203 3.23611900134857e-08
2204 3.24298381007004e-08
2205 3.23351022797347e-08
2206 3.21446096176459e-08
2207 3.22599062272388e-08
2208 3.19862765207901e-08
2209 3.20416892187758e-08
2210 3.19288464889489e-08
2211 3.17090809489606e-08
2212 3.17792905293324e-08
2213 3.15104079864348e-08
2214 3.1462118692982e-08
2215 3.13828922715587e-08
2216 3.13321562357416e-08
2217 3.12738587888717e-08
2218 3.12520617335998e-08
2219 3.12137117948197e-08
2220 3.10057306038836e-08
2221 3.12358310861072e-08
2222 3.10054063796805e-08
2223 3.09421507194152e-08
2224 3.08758929641328e-08
2225 3.08620648987734e-08
2226 3.08012100713739e-08
2227 3.07649796962295e-08
2228 3.07027676993421e-08
2229 3.06398884948322e-08
2230 3.0567268998638e-08
2231 3.05280521204487e-08
2232 3.04718943495708e-08
2233 3.04078486101922e-08
2234 3.03745679617329e-08
2235 3.03228287630297e-08
2236 3.02672447705099e-08
2237 3.02545674801813e-08
2238 3.01988235289485e-08
2239 3.01402142439677e-08
2240 3.01196001135651e-08
2241 3.00748270705142e-08
2242 3.00020169574644e-08
2243 2.99772841376722e-08
2244 2.99949804882793e-08
2245 2.99182164966716e-08
2246 2.98174154473863e-08
2247 2.98337497213197e-08
2248 2.97877957007309e-08
2249 2.97282025942813e-08
2250 2.96885503319189e-08
2251 2.96915600075387e-08
2252 2.96268828261503e-08
2253 2.96020686509912e-08
2254 2.95862130044178e-08
2255 2.95726172665489e-08
2256 2.95009567898852e-08
2257 2.94234437800966e-08
2258 2.94252965509223e-08
2259 2.94418085973192e-08
2260 2.93423016528571e-08
2261 2.93364664556428e-08
2262 2.9278265921473e-08
2263 2.92813284907822e-08
2264 2.92852222609774e-08
2265 2.92450523597942e-08
2266 2.92166284019402e-08
2267 2.91476965621484e-08
2268 2.91206469871241e-08
2269 2.9084005692237e-08
2270 2.90973084280388e-08
2271 2.9066154888735e-08
2272 2.90592396314793e-08
2273 2.903219871353e-08
2274 2.8993103499797e-08
2275 2.90072672499697e-08
2276 2.89885272022339e-08
2277 2.89520249459052e-08
2278 2.89304525491474e-08
2279 2.8873197399637e-08
2280 2.88416605038755e-08
2281 2.87946236485759e-08
2282 2.87163995920103e-08
2283 2.8751229569135e-08
2284 2.86675678098369e-08
2285 2.87072197124871e-08
2286 2.86591628351207e-08
2287 2.86360354149195e-08
2288 2.86377652902736e-08
2289 2.86282400225879e-08
2290 2.86060725163129e-08
2291 2.85601986473871e-08
2292 2.8566027550081e-08
2293 2.85328380456029e-08
2294 2.84778560919463e-08
2295 2.8503181678019e-08
2296 2.84794163121216e-08
2297 2.84508193038047e-08
2298 2.84375073329457e-08
2299 2.84022412997409e-08
2300 2.84504239880246e-08
2301 2.83795681350618e-08
2302 2.80858092560621e-08
2303 2.81308356857579e-08
2304 2.84017445939533e-08
2305 2.83484147467483e-08
2306 2.82965770237453e-08
2307 2.81869080178687e-08
2308 2.82838599325874e-08
2309 2.82717122983556e-08
2310 2.82287362001199e-08
2311 2.81732639646481e-08
2312 2.82028487006425e-08
2313 2.81964688206848e-08
2314 2.82919290341965e-08
2315 2.81617670132572e-08
2316 2.80558309988521e-08
2317 2.80766361031581e-08
2318 2.80740395859347e-08
2319 2.80186498251567e-08
2320 2.78623242437881e-08
2321 2.78319890574252e-08
2322 2.7838097744759e-08
2323 2.81434602773523e-08
2324 2.79450416238447e-08
2325 2.78161572744295e-08
2326 2.79697963896375e-08
2327 2.79070327633413e-08
2328 2.78919659102428e-08
2329 2.78619289257875e-08
2330 2.78659629824318e-08
2331 2.78614516817655e-08
2332 2.77982614309558e-08
2333 2.75627960046876e-08
2334 2.7683113924315e-08
2335 2.76638133267504e-08
2336 2.7976017778375e-08
2337 2.77359681546407e-08
2338 2.75629913719655e-08
2339 2.77539640132218e-08
2340 2.77165115951661e-08
2341 2.76678804733699e-08
2342 2.76632074442951e-08
2343 2.76779803503047e-08
2344 2.7455451997227e-08
2345 2.74844656082962e-08
2346 2.74150879100432e-08
2347 2.73005283002448e-08
2348 2.73293975054933e-08
2349 2.73166359772148e-08
2350 2.73018226644162e-08
2351 2.72489444093083e-08
2352 2.73889708166664e-08
2353 2.72320520533498e-08
2354 2.72790375470677e-08
2355 2.72348291288083e-08
2356 2.73146983129635e-08
2357 2.73088766502561e-08
2358 2.72420718752109e-08
2359 2.71807621974318e-08
2360 2.71759283165807e-08
2361 2.71592088800787e-08
2362 2.716162148797e-08
2363 2.71283624564411e-08
2364 2.71220685803542e-08
2365 2.71014957067273e-08
2366 2.70860176589949e-08
2367 2.70522221228475e-08
2368 2.70808371154452e-08
2369 2.70027544879703e-08
2370 2.70345289230534e-08
2371 2.6964751616898e-08
2372 2.696691184223e-08
2373 2.69479568371089e-08
2374 2.69498847638161e-08
2375 2.69362674423235e-08
2376 2.69187825274741e-08
2377 2.68844778372745e-08
2378 2.68902544915406e-08
2379 2.68711434427971e-08
2380 2.68514626293737e-08
2381 2.68734419184291e-08
2382 2.68483514633822e-08
2383 2.68327053034589e-08
2384 2.68972461920924e-08
2385 2.68946966976991e-08
2386 2.68688440390186e-08
2387 2.68711721576054e-08
2388 2.68394169249397e-08
2389 2.68374860716847e-08
2390 2.68135587213614e-08
2391 2.6805691601961e-08
2392 2.67745706401534e-08
2393 2.6809520248694e-08
2394 2.67572526135851e-08
2395 2.6740364799771e-08
2396 2.67548818264629e-08
2397 2.67323597937796e-08
2398 2.67107425182544e-08
2399 2.66933362711441e-08
2400 2.66796495886901e-08
2401 2.66811331712802e-08
2402 2.66719013644057e-08
2403 2.66482157882386e-08
2404 2.66498887371291e-08
2405 2.65529375331752e-08
2406 2.64868419983522e-08
2407 2.64362802493423e-08
2408 2.63785517100601e-08
2409 2.64098169204807e-08
2410 2.62780924766837e-08
2411 2.62741602781169e-08
2412 2.6275515581986e-08
2413 2.63046898743013e-08
2414 2.62378383566464e-08
2415 2.62339208467743e-08
2416 2.6197952087248e-08
2417 2.61643609240636e-08
2418 2.61683168703009e-08
2419 2.61335110480765e-08
2420 2.61862989838768e-08
2421 2.6123462132599e-08
2422 2.61582386786863e-08
2423 2.60744872910834e-08
2424 2.61248545112558e-08
2425 2.60602128983045e-08
2426 2.5996013666596e-08
2427 2.59825018495974e-08
2428 2.61028798762553e-08
2429 2.59763141152725e-08
2430 2.59699663778257e-08
2431 2.5952968202958e-08
2432 2.58987610708417e-08
2433 2.58901513450205e-08
2434 2.59073284434308e-08
2435 2.58808899733509e-08
2436 2.58690743470957e-08
2437 2.59155831829894e-08
2438 2.58984107750493e-08
2439 2.58327015063564e-08
2440 2.58910232133758e-08
2441 2.5851744575256e-08
2442 2.58734603728605e-08
2443 2.58497839209504e-08
2444 2.57852705596306e-08
2445 2.5749621920923e-08
2446 2.56977506314371e-08
2447 2.57797519069847e-08
2448 2.56253563124531e-08
2449 2.568607414144e-08
2450 2.56544600301112e-08
2451 2.57075833505205e-08
2452 2.55754417626264e-08
2453 2.554663458465e-08
2454 2.55213013602429e-08
2455 2.55292079387459e-08
2456 2.55761564451618e-08
2457 2.56380117495958e-08
2458 2.5466044155209e-08
2459 2.55924793446383e-08
2460 2.5411626344507e-08
2461 2.5483353592648e-08
2462 2.54037278160091e-08
2463 2.53894689294931e-08
2464 2.5385459480276e-08
2465 2.54166745379258e-08
2466 2.53300338006746e-08
2467 2.52633231934141e-08
2468 2.52905496420652e-08
2469 2.52612174573841e-08
2470 2.52215043445858e-08
2471 2.52222532011182e-08
2472 2.51468582752601e-08
2473 2.51776597233189e-08
2474 2.51615491562163e-08
2475 2.51395283474842e-08
2476 2.52183839921649e-08
2477 2.51443643182192e-08
2478 2.51232011265756e-08
2479 2.51217024658246e-08
2480 2.50529616079298e-08
2481 2.51389312446726e-08
2482 2.50886437704878e-08
2483 2.50070869021179e-08
2484 2.50177555844999e-08
2485 2.49742417359755e-08
2486 2.49877498199602e-08
2487 2.49677537893511e-08
2488 2.50068211697929e-08
2489 2.49653871566835e-08
2490 2.4933863711496e-08
2491 2.49424296749901e-08
2492 2.4951844725507e-08
2493 2.4892882751093e-08
2494 2.4931655667082e-08
2495 2.48718469033626e-08
2496 2.49019510607518e-08
2497 2.48731343424069e-08
2498 2.48883164082336e-08
2499 2.49033845207691e-08
2500 2.49035872372794e-08
2501 2.48862970964048e-08
2502 2.48035862342455e-08
2503 2.48739398363007e-08
2504 2.47867319709272e-08
2505 2.48239095244074e-08
2506 2.48499960635584e-08
2507 2.48351267178037e-08
2508 2.48323786125049e-08
2509 2.47996488513813e-08
2510 2.48657050256895e-08
2511 2.48231005408606e-08
2512 2.47950744980407e-08
2513 2.47893814209199e-08
2514 2.47993815580827e-08
2515 2.47897829086519e-08
2516 2.47634090388971e-08
2517 2.47896581013762e-08
2518 2.47696634980699e-08
2519 2.47453253470198e-08
2520 2.47223928200313e-08
2521 2.47334208407501e-08
2522 2.47096111714029e-08
2523 2.4687650667321e-08
2524 2.46956074168025e-08
2525 2.46711931128907e-08
2526 2.46819320248726e-08
2527 2.46199760338683e-08
2528 2.45927725193162e-08
2529 2.46706242497119e-08
2530 2.46061201409908e-08
2531 2.46551861158295e-08
2532 2.4620513753959e-08
2533 2.46363766653879e-08
2534 2.4628943702254e-08
2535 2.46116972482113e-08
2536 2.46342331093885e-08
2537 2.46114901591987e-08
2538 2.46096278395669e-08
2539 2.45866363863811e-08
2540 2.45964233274165e-08
2541 2.45708998334315e-08
2542 2.45835269470085e-08
2543 2.45638348301824e-08
2544 2.45521365713763e-08
2545 2.45779312835204e-08
2546 2.45374288643241e-08
2547 2.45665522609073e-08
2548 2.45263032567777e-08
2549 2.45214198875132e-08
2550 2.45059835601857e-08
2551 2.45173493294004e-08
2552 2.44947074992119e-08
2553 2.45010167594373e-08
2554 2.45626643931018e-08
2555 2.44792750354605e-08
2556 2.44580011252715e-08
2557 2.45601274948548e-08
2558 2.44549451076104e-08
2559 2.44704517249339e-08
2560 2.44637011022064e-08
2561 2.44879491315686e-08
2562 2.4345519065605e-08
2563 2.44300616065729e-08
2564 2.43928618703926e-08
2565 2.43841448002158e-08
2566 2.43780653921277e-08
2567 2.43811462623533e-08
2568 2.43616524753243e-08
2569 2.43661697965081e-08
2570 2.43490519786782e-08
2571 2.43574382601253e-08
2572 2.43382521336599e-08
2573 2.43445802485809e-08
2574 2.44352526195257e-08
2575 2.42636506588845e-08
2576 2.4382859165506e-08
2577 2.43264350885397e-08
2578 2.43318004828907e-08
2579 2.4318235475107e-08
2580 2.43151912489026e-08
2581 2.43053389064052e-08
2582 2.42865047273e-08
2583 2.42982581002416e-08
2584 2.43043741323667e-08
2585 2.43125716861137e-08
2586 2.42699668686619e-08
2587 2.42551985039086e-08
2588 2.42744805540873e-08
2589 2.42553433427162e-08
2590 2.42339634275446e-08
2591 2.42310187354988e-08
2592 2.42103561043372e-08
2593 2.41927280453247e-08
2594 2.42068397344752e-08
2595 2.41830105895957e-08
2596 2.42476137515268e-08
2597 2.42324421826368e-08
2598 2.41915985257357e-08
2599 2.41748273044351e-08
2600 2.4186519977043e-08
2601 2.4179407354552e-08
2602 2.41663565696548e-08
2603 2.41800421174609e-08
2604 2.41570701149385e-08
2605 2.41318774478749e-08
2606 2.41287477287244e-08
2607 2.41291152303091e-08
2608 2.41133490512802e-08
2609 2.41305678247983e-08
2610 2.40912506952462e-08
2611 2.40801481559139e-08
2612 2.41406477159423e-08
2613 2.41297449727895e-08
2614 2.4095290498849e-08
2615 2.41204857425181e-08
2616 2.40823391250977e-08
2617 2.41169306987032e-08
2618 2.40800200885793e-08
2619 2.40420970376576e-08
2620 2.40115907583771e-08
2621 2.40670964557488e-08
2622 2.40035303660235e-08
2623 2.40694404207709e-08
2624 2.39871997282926e-08
2625 2.40326550962067e-08
2626 2.3954168761442e-08
2627 2.40268140512256e-08
2628 2.39998666065055e-08
2629 2.3985291254025e-08
2630 2.39567574289445e-08
2631 2.39846544216604e-08
2632 2.39369897876429e-08
2633 2.38818613977898e-08
2634 2.39493084817077e-08
2635 2.38887064112347e-08
2636 2.38993242347441e-08
2637 2.38915883681301e-08
2638 2.39090372167183e-08
2639 2.39165006532538e-08
2640 2.3909196885441e-08
2641 2.38644657239995e-08
2642 2.38849699494281e-08
2643 2.38983762548273e-08
2644 2.38486406014538e-08
2645 2.38773196845621e-08
2646 2.37812354573208e-08
2647 2.38795882485654e-08
2648 2.38290460594648e-08
2649 2.38504818601548e-08
2650 2.37533120950317e-08
2651 2.38623648134428e-08
2652 2.38244231485218e-08
2653 2.38374669074837e-08
2654 2.3715871300567e-08
2655 2.38580457896376e-08
2656 2.37788429551244e-08
2657 2.38263465703881e-08
2658 2.37195858003858e-08
2659 2.37419101249969e-08
2660 2.37934447815391e-08
2661 2.37060152064039e-08
2662 2.36939189370844e-08
2663 2.37918087830913e-08
2664 2.36732907157311e-08
2665 2.37708965618033e-08
2666 2.36870809127154e-08
2667 2.37533571190163e-08
2668 2.37491758019459e-08
2669 2.3639161234712e-08
2670 2.37320687501708e-08
2671 2.37475380013841e-08
2672 2.37412641430623e-08
2673 2.36923716476767e-08
2674 2.36276849037154e-08
2675 2.36815473018126e-08
2676 2.37157332745319e-08
2677 2.36273267089082e-08
2678 2.36026643971599e-08
2679 2.36067487993097e-08
2680 2.36209799626685e-08
2681 2.36060413589811e-08
2682 2.36104203490406e-08
2683 2.35962008585666e-08
2684 2.35850015508632e-08
2685 2.35924942075805e-08
2686 2.35782205866286e-08
2687 2.35772843812931e-08
2688 2.35597968245571e-08
2689 2.35777446437879e-08
2690 2.35565437538554e-08
2691 2.36214133733093e-08
2692 2.35524341407611e-08
2693 2.35394026484315e-08
2694 2.35335459319863e-08
2695 2.35269057853316e-08
2696 2.35304583857676e-08
2697 2.35221716367029e-08
2698 2.3512224325728e-08
2699 2.35274898261473e-08
2700 2.35030608459752e-08
2701 2.35141017910223e-08
2702 2.34844594322325e-08
2703 2.36064162817406e-08
2704 2.34715549627396e-08
2705 2.34835558088342e-08
2706 2.34769890599118e-08
2707 2.3475392880723e-08
2708 2.35423416667935e-08
2709 2.34751648244824e-08
2710 2.34421265172813e-08
2711 2.35501992356113e-08
2712 2.34397110476792e-08
2713 2.34394784577319e-08
2714 2.34332887245614e-08
2715 2.34243712990967e-08
2716 2.34388668758356e-08
2717 2.34305880923991e-08
2718 2.34182796852167e-08
2719 2.34181999414496e-08
2720 2.34149909239534e-08
2721 2.35061410664983e-08
2722 2.33975469403624e-08
2723 2.33830739988505e-08
2724 2.34114976724875e-08
2725 2.34548312159077e-08
2726 2.34424559604207e-08
2727 2.33738831645347e-08
2728 2.3432231981424e-08
2729 2.33567075063057e-08
2730 2.34046533869048e-08
2731 2.33488871375087e-08
2732 2.3352429380008e-08
2733 2.33481457661e-08
2734 2.33407855181333e-08
2735 2.33648137433207e-08
2736 2.33318322822029e-08
2737 2.33070653319345e-08
2738 2.3346116288625e-08
2739 2.33266172862123e-08
2740 2.33008381118616e-08
2741 2.33355860723705e-08
2742 2.32842534395594e-08
2743 2.33014976913637e-08
2744 2.32890997100554e-08
2745 2.32929399976989e-08
2746 2.3302574435391e-08
2747 2.33146102432968e-08
2748 2.32785370557487e-08
2749 2.32942535798308e-08
2750 2.32616284869813e-08
2751 2.32929211234634e-08
2752 2.32491365919074e-08
2753 2.32627183809342e-08
2754 2.32306790786119e-08
2755 2.3276313755094e-08
2756 2.32429791906519e-08
2757 2.32749975759283e-08
2758 2.32213192310482e-08
2759 2.32215358355603e-08
2760 2.32546402201272e-08
2761 2.32203828449684e-08
2762 2.33159703797448e-08
2763 2.31969889625816e-08
2764 2.314681362936e-08
2765 2.31738245357604e-08
2766 2.3170235046166e-08
2767 2.31266902575022e-08
2768 2.3123720971796e-08
2769 2.31477672127944e-08
2770 2.30991476701803e-08
2771 2.31053931303293e-08
2772 2.31188731598664e-08
2773 2.3145674900249e-08
2774 2.30820694473444e-08
2775 2.31041685183619e-08
2776 2.30864719110713e-08
2777 2.30861167858087e-08
2778 2.30348852974238e-08
2779 2.30820793194475e-08
2780 2.30819818543004e-08
2781 2.30532784768478e-08
2782 2.30654754611592e-08
2783 2.30319875065277e-08
2784 2.30562718042115e-08
2785 2.3061908677402e-08
2786 2.3006683151916e-08
2787 2.30508197067714e-08
2788 2.30231742781584e-08
2789 2.30193855776939e-08
2790 2.30073178633106e-08
2791 2.29989750257786e-08
2792 2.30151464095663e-08
2793 2.30415139754747e-08
2794 2.2970916178533e-08
2795 2.3006379019197e-08
2796 2.30143007993178e-08
2797 2.29839285088573e-08
2798 2.29939372036192e-08
2799 2.29693911610873e-08
2800 2.30020177247603e-08
2801 2.29397145639076e-08
2802 2.29599945593684e-08
2803 2.29954475106098e-08
2804 2.29464646674948e-08
2805 2.29727204277275e-08
2806 2.2946284010672e-08
2807 2.29688330946054e-08
2808 2.28977551794962e-08
2809 2.29595473451027e-08
2810 2.29503427240374e-08
2811 2.29251608412362e-08
2812 2.29668454219123e-08
2813 2.29262388535822e-08
2814 2.29372032620745e-08
2815 2.29429834188721e-08
2816 2.29121144466049e-08
2817 2.29283576671513e-08
2818 2.29099798922938e-08
2819 2.29299884892953e-08
2820 2.28775421917682e-08
2821 2.28866969096408e-08
2822 2.29147223489434e-08
2823 2.29009073966502e-08
2824 2.28908831978458e-08
2825 2.29076539066675e-08
2826 2.28706648472965e-08
2827 2.29095146546676e-08
2828 2.28287743642142e-08
2829 2.28744850696572e-08
2830 2.28264675494749e-08
2831 2.28482780260109e-08
2832 2.28631989376282e-08
2833 2.28285269741058e-08
2834 2.28815637091095e-08
2835 2.28192816194017e-08
2836 2.28238579511597e-08
2837 2.28280531362479e-08
2838 2.28150668024618e-08
2839 2.28126944015195e-08
2840 2.28211046033877e-08
2841 2.27821876359613e-08
2842 2.28276182587805e-08
2843 2.27965611223624e-08
2844 2.27861874484425e-08
2845 2.27895648836629e-08
2846 2.28024336195176e-08
2847 2.27751800885301e-08
2848 2.27899082791971e-08
2849 2.27718561216506e-08
2850 2.27767926768152e-08
2851 2.27349871697591e-08
2852 2.27359381210768e-08
2853 2.27825065404197e-08
2854 2.27540307560226e-08
2855 2.27609142315899e-08
2856 2.27369288285928e-08
2857 2.27386377598116e-08
2858 2.27283334819361e-08
2859 2.27572066249238e-08
2860 2.27120586546725e-08
2861 2.27790808859929e-08
2862 2.26962135125852e-08
2863 2.27254233706731e-08
2864 2.27075532377441e-08
2865 2.27113100152998e-08
2866 2.27105394650096e-08
2867 2.26963902347777e-08
2868 2.27075190419868e-08
2869 2.26801028460066e-08
2870 2.26805808400954e-08
2871 2.27163207671133e-08
2872 2.26599603716515e-08
2873 2.27128980405666e-08
2874 2.26495166697482e-08
2875 2.26838920949213e-08
2876 2.2633144937334e-08
2877 2.27002717037017e-08
2878 2.26501465849793e-08
2879 2.26438821955277e-08
2880 2.26743442386379e-08
2881 2.26622551937616e-08
2882 2.26392203792614e-08
2883 2.26614374052581e-08
2884 2.26081309948611e-08
2885 2.26377313041759e-08
2886 2.26622117187603e-08
2887 2.26091042265786e-08
2888 2.26329164516592e-08
2889 2.26028324581407e-08
2890 2.26154009084212e-08
2891 2.26111500527537e-08
2892 2.25871219257101e-08
2893 2.25940610474318e-08
2894 2.26068414987957e-08
2895 2.25933509092613e-08
2896 2.26463827064372e-08
2897 2.25531392392675e-08
2898 2.2575582794282e-08
2899 2.25953645252019e-08
2900 2.25612005455567e-08
2901 2.25841375893765e-08
2902 2.26227083439667e-08
2903 2.25383530056966e-08
2904 2.25521555567987e-08
2905 2.25481722559984e-08
2906 2.25636249209238e-08
2907 2.25257892529918e-08
2908 2.25722115554383e-08
2909 2.24744241084096e-08
2910 2.25321640257015e-08
2911 2.26051607103983e-08
2912 2.24659051037968e-08
2913 2.25132282407614e-08
2914 2.25238143851669e-08
2915 2.24942209117174e-08
2916 2.2489711597462e-08
2917 2.25633682031656e-08
2918 2.24660171594948e-08
2919 2.24490727580573e-08
2920 2.24820107517232e-08
2921 2.25042746353488e-08
2922 2.24400970441074e-08
2923 2.25074355393318e-08
2924 2.24257760801372e-08
2925 2.24687806360535e-08
2926 2.24606469632072e-08
2927 2.24621305324746e-08
2928 2.24441385925367e-08
2929 2.24451444834628e-08
2930 2.25064913146333e-08
2931 2.24001227961956e-08
2932 2.24331048723769e-08
2933 2.24461386779673e-08
2934 2.24210934156588e-08
2935 2.2414025148354e-08
2936 2.24331432985281e-08
2937 2.23916673203206e-08
2938 2.24072612478032e-08
2939 2.24330685445473e-08
2940 2.23980179203664e-08
2941 2.23874248690414e-08
2942 2.24090919109798e-08
2943 2.24122577434649e-08
2944 2.23516667210433e-08
2945 2.23846981870501e-08
2946 2.23750139611845e-08
2947 2.23675531274559e-08
2948 2.23690972482871e-08
2949 2.23966252086427e-08
2950 2.23590922039385e-08
2951 2.23575589957115e-08
2952 2.23446854898945e-08
2953 2.23787074973281e-08
2954 2.23484926151762e-08
2955 2.23410897159937e-08
2956 2.23185536469828e-08
2957 2.23394644485175e-08
2958 2.23433849302346e-08
2959 2.23287105440839e-08
2960 2.23443335674034e-08
2961 2.23401842065485e-08
2962 2.22854145679463e-08
2963 2.23340432303765e-08
2964 2.23200163498305e-08
2965 2.23124990528234e-08
2966 2.23158065382201e-08
2967 2.2295199315181e-08
2968 2.23108840757824e-08
2969 2.22952372217478e-08
2970 2.2297369233204e-08
2971 2.23066160116581e-08
2972 2.22738460897709e-08
2973 2.22975133237213e-08
2974 2.22771114106735e-08
2975 2.22468722848035e-08
2976 2.22956516267025e-08
2977 2.22850922777518e-08
2978 2.22585025828614e-08
2979 2.22740814486144e-08
2980 2.22606010553683e-08
2981 2.22279482837706e-08
2982 2.2239388069778e-08
2983 2.22621345362661e-08
2984 2.22273305126031e-08
2985 2.22571749564082e-08
2986 2.22320354779093e-08
2987 2.21837540239633e-08
2988 2.22493222183928e-08
2989 2.22160562559637e-08
2990 2.22250939407864e-08
2991 2.22165201968494e-08
2992 2.2228362487553e-08
2993 2.21891013674203e-08
2994 2.21847167445333e-08
2995 2.21994414095761e-08
2996 2.22162727019359e-08
2997 2.21921287617555e-08
2998 2.21996001563696e-08
2999 2.22098814934846e-08
3000 2.21687695001194e-08
3001 2.21704665168687e-08
3002 2.21909749278382e-08
3003 2.21792000441035e-08
3004 2.21533350694969e-08
3005 2.21854726114579e-08
3006 2.21406713594163e-08
3007 2.21303854832655e-08
3008 2.21595867402335e-08
3009 2.21411071437139e-08
3010 2.21570214553601e-08
3011 2.21308956120936e-08
3012 2.2145175264221e-08
3013 2.20978097704716e-08
3014 2.21337124810539e-08
3015 2.213698945841e-08
3016 2.21160757813976e-08
3017 2.21377153253322e-08
3018 2.21067992236534e-08
3019 2.20749179997703e-08
3020 2.21201113475011e-08
3021 2.21214007685155e-08
3022 2.20958036827312e-08
3023 2.2100835159744e-08
3024 2.2089924967883e-08
3025 2.20724450814558e-08
3026 2.21520109171536e-08
3027 2.20784932372098e-08
3028 2.20787505909037e-08
3029 2.210819119286e-08
3030 2.205058977367e-08
3031 2.20721831047932e-08
3032 2.21007145100316e-08
3033 2.20594649165129e-08
3034 2.20779247355196e-08
3035 2.2065045580888e-08
3036 2.20671636110836e-08
3037 2.20351267556929e-08
3038 2.20663921060016e-08
3039 2.20749036450307e-08
3040 2.2036442459239e-08
3041 2.20661018555113e-08
3042 2.20112964162134e-08
3043 2.20287236962768e-08
3044 2.2036696433414e-08
3045 2.2033487441675e-08
3046 2.20482124375643e-08
3047 2.19986733447364e-08
3048 2.20261848933312e-08
3049 2.20180432979333e-08
3050 2.19916903834161e-08
3051 2.20354305335846e-08
3052 2.19936760732509e-08
3053 2.2012358825485e-08
3054 2.19823301530475e-08
3055 2.19895461279762e-08
3056 2.20009061440152e-08
3057 2.1990407818695e-08
3058 2.19933704497244e-08
3059 2.19855577583417e-08
3060 2.19631290372391e-08
3061 2.19632550515492e-08
3062 2.19995851642452e-08
3063 2.19225890072394e-08
3064 2.19883454324332e-08
3065 2.19422637068867e-08
3066 2.19546776714097e-08
3067 2.19750038099598e-08
3068 2.1963026813232e-08
3069 2.19410699955347e-08
3070 2.19848328590899e-08
3071 2.19490459274496e-08
3072 2.19564844661235e-08
3073 2.18946246381968e-08
3074 2.19621685024762e-08
3075 2.19197135522542e-08
3076 2.19659115079374e-08
3077 2.19565917758402e-08
3078 2.19178880258575e-08
3079 2.18915289300625e-08
3080 2.1928348689304e-08
3081 2.1930553886218e-08
3082 2.19125431870637e-08
3083 2.18721921578791e-08
3084 2.18938633156363e-08
3085 2.19392245131012e-08
3086 2.18755562690909e-08
3087 2.18965646321401e-08
3088 2.18930631565861e-08
3089 2.19043130695873e-08
3090 2.1827368372751e-08
3091 2.19189626342597e-08
3092 2.18870753552203e-08
3093 2.18487186969263e-08
3094 2.1907880333405e-08
3095 2.18638142679417e-08
3096 2.18362129582594e-08
3097 2.18813988852595e-08
3098 2.18046438562602e-08
3099 2.18679803349886e-08
3100 2.18485226421983e-08
3101 2.18373796214699e-08
3102 2.18420358018001e-08
3103 2.18338422790509e-08
3104 2.182555513075e-08
3105 2.18232174042932e-08
3106 2.18269937160009e-08
3107 2.18810129353209e-08
3108 2.18176193689246e-08
3109 2.18121707757923e-08
3110 2.18176314947804e-08
3111 2.18521586723419e-08
3112 2.1822844110897e-08
3113 2.18226264543375e-08
3114 2.18146837682731e-08
3115 2.18266402898237e-08
3116 2.1797249545763e-08
3117 2.18337363104837e-08
3118 2.17943581559688e-08
3119 2.17801985038513e-08
3120 2.18150160078423e-08
3121 2.17451462840579e-08
3122 2.17997612432796e-08
3123 2.18112886614108e-08
3124 2.1781716691649e-08
3125 2.17831019662107e-08
3126 2.18203982123022e-08
3127 2.17311511483409e-08
3128 2.17963758100126e-08
3129 2.17604646408631e-08
3130 2.17680344984927e-08
3131 2.1771633042178e-08
3132 2.17222026774877e-08
3133 2.17787614333886e-08
3134 2.17612338828665e-08
3135 2.17384585425862e-08
3136 2.17687563277558e-08
3137 2.17257975196894e-08
3138 2.17392041044206e-08
3139 2.17420430881354e-08
3140 2.1727512941716e-08
3141 2.17326930904704e-08
3142 2.16855810033501e-08
3143 2.17356646525779e-08
3144 2.17583970401058e-08
3145 2.17018689556703e-08
3146 2.17293022601872e-08
3147 2.16785215441107e-08
3148 2.16881248085166e-08
3149 2.17184150912431e-08
3150 2.16626902629358e-08
3151 2.17215274846971e-08
3152 2.16902152536669e-08
3153 2.17020599158069e-08
3154 2.16771852601383e-08
3155 2.16862417885544e-08
3156 2.16977439424504e-08
3157 2.16527767911145e-08
3158 2.17032874689771e-08
3159 2.16466983880004e-08
3160 2.1656359545652e-08
3161 2.16904747447622e-08
3162 2.1650364535386e-08
3163 2.16550410483407e-08
3164 2.16512570880845e-08
3165 2.16322670891778e-08
3166 2.16247854747387e-08
3167 2.1653922899656e-08
3168 2.16527886065521e-08
3169 2.16168469227007e-08
3170 2.16333068907559e-08
3171 2.16363271952069e-08
3172 2.16265557799744e-08
3173 2.16125990699645e-08
3174 2.16807606303604e-08
3175 2.15695254204284e-08
3176 2.16161103168133e-08
3177 2.15907795411141e-08
3178 2.16195452593659e-08
3179 2.15418148714619e-08
3180 2.16571494613405e-08
3181 2.15841697666086e-08
3182 2.16082622572777e-08
3183 2.15719454699226e-08
3184 2.15863817372153e-08
3185 2.15548015503053e-08
3186 2.16029170014842e-08
3187 2.16010583233839e-08
3188 2.15397611591506e-08
3189 2.15976560427045e-08
3190 2.15834718968466e-08
3191 2.15778589369719e-08
3192 2.15669044276723e-08
3193 2.15485774703872e-08
3194 2.15636065985514e-08
3195 2.15482165630831e-08
3196 2.15464711939006e-08
3197 2.1536500831143e-08
3198 2.15400105458841e-08
3199 2.15707673403287e-08
3200 2.15383237551769e-08
3201 2.15658678439645e-08
3202 2.15369579068536e-08
3203 2.15072301466002e-08
3204 2.15104451006809e-08
3205 2.15575860580408e-08
3206 2.15099622189463e-08
3207 2.15302659127836e-08
3208 2.15311675084706e-08
3209 2.1508721563368e-08
3210 2.15256488766968e-08
3211 2.15072695053387e-08
3212 2.15262534362104e-08
3213 2.15188523973175e-08
3214 2.14726626999173e-08
3215 2.15109641308331e-08
3216 2.15401788108416e-08
3217 2.14823432211908e-08
3218 2.15061169450692e-08
3219 2.14719783424577e-08
3220 2.14971135048536e-08
3221 2.14702153606972e-08
3222 2.14634970046035e-08
3223 2.14831036755569e-08
3224 2.15222001087589e-08
3225 2.14445914306083e-08
3226 2.14907608842019e-08
3227 2.14861768452756e-08
3228 2.14297729819179e-08
3229 2.15039103803605e-08
3230 2.14107461538049e-08
3231 2.14637411510843e-08
3232 2.14337900539263e-08
3233 2.15165886991997e-08
3234 2.14175441932696e-08
3235 2.1489381258899e-08
3236 2.13838108975573e-08
3237 2.14659023125563e-08
3238 2.13735652545566e-08
3239 2.14298049536765e-08
3240 2.14161557074988e-08
3241 2.14231783526309e-08
3242 2.14548777219825e-08
3243 2.13800572197442e-08
3244 2.14337972539447e-08
3245 2.13684538743486e-08
3246 2.14378032428186e-08
3247 2.14492479688744e-08
3248 2.13907811441594e-08
3249 2.13822059342839e-08
3250 2.13836046771831e-08
3251 2.13773905599623e-08
3252 2.14169026491184e-08
3253 2.13631505259215e-08
3254 2.14166999583654e-08
3255 2.13933367798624e-08
3256 2.13934772950175e-08
3257 2.13402327302603e-08
3258 2.14618821909873e-08
3259 2.13323398168264e-08
3260 2.13386501113355e-08
3261 2.13784530149752e-08
3262 2.13981007917141e-08
3263 2.13628058585158e-08
3264 2.12900106246749e-08
3265 2.13758048279722e-08
3266 2.13815525347272e-08
3267 2.13491450202774e-08
3268 2.1313029055392e-08
3269 2.14342074245089e-08
3270 2.12998099060613e-08
3271 2.12737043416844e-08
3272 2.13637889245888e-08
3273 2.13264798589741e-08
3274 2.13802371118454e-08
3275 2.13604836059034e-08
3276 2.12547488183112e-08
3277 2.13074798707247e-08
3278 2.12655721729682e-08
3279 2.13014111469789e-08
3280 2.12857192583016e-08
3281 2.12891251791802e-08
3282 2.12593904400649e-08
3283 2.13600534046954e-08
3284 2.12829583055019e-08
3285 2.12671847181767e-08
3286 2.12568764652588e-08
3287 2.13579354264581e-08
3288 2.1305539830152e-08
3289 2.12667168804082e-08
3290 2.12316987631134e-08
3291 2.12311735507953e-08
3292 2.12432494723913e-08
3293 2.1269267307833e-08
3294 2.12497827223324e-08
3295 2.13222853320261e-08
3296 2.11838706336565e-08
3297 2.12922240421243e-08
3298 2.12232332388318e-08
3299 2.12145826270493e-08
3300 2.1237372449967e-08
3301 2.12513145658733e-08
3302 2.11992673571082e-08
3303 2.1269172591154e-08
3304 2.1190366187529e-08
3305 2.11898637676455e-08
3306 2.12319360928248e-08
3307 2.11493863595535e-08
3308 2.12441200870828e-08
3309 2.12210222803044e-08
3310 2.12274181570216e-08
3311 2.11681438182509e-08
3312 2.11717982203652e-08
3313 2.11169987101023e-08
3314 2.11492795774149e-08
3315 2.11264695146873e-08
3316 2.11101241167633e-08
3317 2.10845823880312e-08
3318 2.11219206680546e-08
3319 2.10501196602486e-08
3320 2.11789324460199e-08
3321 2.10530289299626e-08
3322 2.10851792261657e-08
3323 2.11030166608062e-08
3324 2.11210016884955e-08
3325 2.10998387597527e-08
3326 2.10869489314369e-08
3327 2.10636434010425e-08
3328 2.1115969884633e-08
3329 2.11141026436046e-08
3330 2.10760582222136e-08
3331 2.10384522101315e-08
3332 2.11177203475188e-08
3333 2.10778646478893e-08
3334 2.10776436428972e-08
3335 2.1030505461539e-08
3336 2.11174304012296e-08
3337 2.10499662376407e-08
3338 2.10524282491242e-08
3339 2.10865452099362e-08
3340 2.10406638676552e-08
3341 2.1019053820126e-08
3342 2.11006779227141e-08
3343 2.10438363188459e-08
3344 2.09946124432037e-08
3345 2.10195176584271e-08
3346 2.10518691559081e-08
3347 2.10005129446245e-08
3348 2.10220320941978e-08
3349 2.10140251306612e-08
3350 2.10159006583943e-08
3351 2.10147737162991e-08
3352 2.10216470470925e-08
3353 2.10009598968774e-08
3354 2.10259721442618e-08
3355 2.10197305565707e-08
3356 2.09858724962153e-08
3357 2.10033416467859e-08
3358 2.10114468899292e-08
3359 2.09497396190805e-08
3360 2.0960480739074e-08
3361 2.09384684071168e-08
3362 2.10092809158624e-08
3363 2.09653530802711e-08
3364 2.09935286696883e-08
3365 2.09592592783636e-08
3366 2.09118432370481e-08
3367 2.1029495349989e-08
3368 2.09495598939569e-08
3369 2.09747856785292e-08
3370 2.09958426009571e-08
3371 2.09972287081861e-08
3372 2.09262047530245e-08
3373 2.0972526100671e-08
3374 2.08793751808045e-08
3375 2.0994788243911e-08
3376 2.0891433716308e-08
3377 2.10213736080433e-08
3378 2.09212698427663e-08
3379 2.08983914280481e-08
3380 2.0993441756989e-08
3381 2.08942705048898e-08
3382 2.09779551960665e-08
3383 2.08364713714104e-08
3384 2.10196041647848e-08
3385 2.08414303068949e-08
3386 2.0976809575135e-08
3387 2.09010961049749e-08
3388 2.08596438038278e-08
3389 2.09597354041691e-08
3390 2.08907457861507e-08
3391 2.09876697390143e-08
3392 2.08082000030529e-08
3393 2.09027252480176e-08
3394 2.09651828222412e-08
3395 2.07656573074644e-08
3396 2.09557648918057e-08
3397 2.09330058034141e-08
3398 2.07636040050474e-08
3399 2.09524205727618e-08
3400 2.07978295669875e-08
3401 2.09210333790288e-08
3402 2.08938805981163e-08
3403 2.08135556420963e-08
3404 2.08289395402517e-08
3405 2.08893075477334e-08
3406 2.08531551746383e-08
3407 2.08582454224171e-08
3408 2.08028621133671e-08
3409 2.08702040969122e-08
3410 2.08287841352295e-08
3411 2.08013496436621e-08
3412 2.08745286252032e-08
3413 2.08389676772391e-08
3414 2.08218070922861e-08
3415 2.0757803374849e-08
3416 2.08667715710931e-08
3417 2.08270570669633e-08
3418 2.0813982486878e-08
3419 2.07634117850297e-08
3420 2.08405904724707e-08
3421 2.07979208808329e-08
3422 2.08398381968955e-08
3423 2.0823387108404e-08
3424 2.07629737154491e-08
3425 2.07646393319827e-08
3426 2.08282434361884e-08
3427 2.07434608130441e-08
3428 2.08526106173501e-08
3429 2.07556768558881e-08
3430 2.08387247373487e-08
3431 2.07434618135771e-08
3432 2.0719680957626e-08
3433 2.08375195640542e-08
3434 2.079629743168e-08
3435 2.06686173922499e-08
3436 2.07852894558158e-08
3437 2.07277968251596e-08
3438 2.07762395354472e-08
3439 2.07250036128315e-08
3440 2.07017250577479e-08
3441 2.07384890176243e-08
3442 2.07537893470722e-08
3443 2.07450182556634e-08
3444 2.07303692409955e-08
3445 2.0771849525314e-08
3446 2.07385902188939e-08
3447 2.066503382947e-08
3448 2.07226407380468e-08
3449 2.0671032428865e-08
3450 2.07290574554264e-08
3451 2.0745542629097e-08
3452 2.07119559036961e-08
3453 2.06725957467846e-08
3454 2.06849666817277e-08
3455 2.07184878999733e-08
3456 2.06509278464218e-08
3457 2.069651387826e-08
3458 2.07245401142586e-08
3459 2.06460405132169e-08
3460 2.07335424891042e-08
3461 2.07140642802806e-08
3462 2.06869744676652e-08
3463 2.07177524189639e-08
3464 2.06755050897733e-08
3465 2.05660129060092e-08
3466 2.07089341031086e-08
3467 2.06825005060196e-08
3468 2.06822374324567e-08
3469 2.063414087905e-08
3470 2.06455534024208e-08
3471 2.06210929190043e-08
3472 2.05986965502447e-08
3473 2.0682724793275e-08
3474 2.06190250875427e-08
3475 2.06720903062063e-08
3476 2.05748666783911e-08
3477 2.06011158625508e-08
3478 2.06822828854314e-08
3479 2.06599068159896e-08
3480 2.05715683931906e-08
3481 2.06481677382797e-08
3482 2.0610334013238e-08
3483 2.0614188172452e-08
3484 2.06714848274281e-08
3485 2.05375389765727e-08
3486 2.06349045281939e-08
3487 2.06157280060459e-08
3488 2.05796189507446e-08
3489 2.06475533746087e-08
3490 2.05879050891866e-08
3491 2.0590204846016e-08
3492 2.0612727927638e-08
3493 2.05807906978883e-08
3494 2.05794081677979e-08
3495 2.04815985531681e-08
3496 2.06398099216187e-08
3497 2.06040445385192e-08
3498 2.0592928041907e-08
3499 2.05276768197038e-08
3500 2.05728933555527e-08
3501 2.05378468725037e-08
3502 2.06336309407362e-08
3503 2.05606469245723e-08
3504 2.05507069823696e-08
3505 2.05094354552848e-08
3506 2.05593701898543e-08
3507 2.05953686078786e-08
3508 2.05419367405035e-08
3509 2.05356032063619e-08
3510 2.05503351069503e-08
3511 2.05335763561187e-08
3512 2.05722680060028e-08
3513 2.05409844995508e-08
3514 2.0495632158557e-08
3515 2.05539169626512e-08
3516 2.05119017966382e-08
3517 2.05933653245616e-08
3518 2.0506182802027e-08
3519 2.05209241999071e-08
3520 2.05332984593021e-08
3521 2.05240996811185e-08
3522 2.05099056742597e-08
3523 2.04953897657845e-08
3524 2.04637710710109e-08
3525 2.04863717487846e-08
3526 2.04985522049839e-08
3527 2.051269569181e-08
3528 2.05433490161333e-08
3529 2.05028087485459e-08
3530 2.04672480190737e-08
3531 2.05467523080038e-08
3532 2.04346922996201e-08
3533 2.04819874127793e-08
3534 2.04723056222988e-08
3535 2.04981540803395e-08
3536 2.04167132946687e-08
3537 2.04799826502011e-08
3538 2.04344570455817e-08
3539 2.04882182601729e-08
3540 2.04466549948989e-08
3541 2.04885600449956e-08
3542 2.04413395836767e-08
3543 2.04973981614565e-08
3544 2.04646140447018e-08
3545 2.04508008825321e-08
3546 2.04356187194499e-08
3547 2.04280594320316e-08
3548 2.0439408563e-08
3549 2.05040978058513e-08
3550 2.04376580676957e-08
3551 2.04884789396509e-08
3552 2.04144338162848e-08
3553 2.0479609188051e-08
3554 2.04398712879694e-08
3555 2.03761112094192e-08
3556 2.04481570547088e-08
3557 2.04281734181855e-08
3558 2.03980532011983e-08
3559 2.045244934612e-08
3560 2.04005255515227e-08
3561 2.04332068141078e-08
3562 2.04148300406715e-08
3563 2.04111516515226e-08
3564 2.03773893212578e-08
3565 2.0427105165588e-08
3566 2.04125530087751e-08
3567 2.04345912480086e-08
3568 2.03935337692585e-08
3569 2.03446882447444e-08
3570 2.03791932915642e-08
3571 2.03746600568877e-08
3572 2.04116476170135e-08
3573 2.03983174071887e-08
3574 2.03639442659664e-08
3575 2.04289542073965e-08
3576 2.03709256201279e-08
3577 2.0358893047856e-08
3578 2.03158887788213e-08
3579 2.03659595712224e-08
3580 2.03235311135863e-08
3581 2.03824262885455e-08
3582 2.03141220036507e-08
3583 2.03624374535138e-08
3584 2.03305308232871e-08
3585 2.03819636328539e-08
3586 2.03543432775177e-08
3587 2.02794939547779e-08
3588 2.03603227868498e-08
3589 2.03787868535699e-08
3590 2.03391895883254e-08
3591 2.03281282433743e-08
3592 2.03062476238358e-08
3593 2.03480689342506e-08
3594 2.02830903068829e-08
3595 2.03571382302314e-08
3596 2.03458910643661e-08
3597 2.03012091786192e-08
3598 2.03286807471947e-08
3599 2.02973783038907e-08
3600 2.03271296119745e-08
3601 2.02569222627069e-08
3602 2.03319638005794e-08
3603 2.02548718490903e-08
3604 2.02893453762343e-08
3605 2.02706250020945e-08
3606 2.0288892164988e-08
3607 2.0282166280694e-08
3608 2.02456816289498e-08
3609 2.02348110351913e-08
3610 2.0281172138148e-08
3611 2.03464379850971e-08
3612 2.02534298776591e-08
3613 2.02549198626834e-08
3614 2.02666557376219e-08
3615 2.02566666289705e-08
3616 2.02241171414208e-08
3617 2.02632904091615e-08
3618 2.01777417609961e-08
3619 2.02483087821115e-08
3620 2.02601047214479e-08
3621 2.02526061494623e-08
3622 2.02299007949769e-08
3623 2.01914864406127e-08
3624 2.02772057402711e-08
3625 2.02383889198465e-08
3626 2.01687548133656e-08
3627 2.02595614129386e-08
3628 2.01360055669397e-08
3629 2.02465502749938e-08
3630 2.02044093713383e-08
3631 2.01629140073045e-08
3632 2.02202669887797e-08
3633 2.01891147808553e-08
3634 2.01579994403289e-08
3635 2.02324167077883e-08
3636 2.01422724623868e-08
3637 2.02228304626573e-08
3638 2.02145098011641e-08
3639 2.01890524409443e-08
3640 2.02090383432108e-08
3641 2.01754183177805e-08
3642 2.01762795373206e-08
3643 2.01835414359586e-08
3644 2.01047331542803e-08
3645 2.02174081764817e-08
3646 2.01716729297807e-08
3647 2.0119675114838e-08
3648 2.01824303913689e-08
3649 2.01630396259311e-08
3650 2.01133522521602e-08
3651 2.01520632323593e-08
3652 2.0082043795e-08
3653 2.0138014491522e-08
3654 2.00867012640771e-08
3655 2.00911172743012e-08
3656 2.01259748759952e-08
3657 2.01010629843346e-08
3658 2.01575386418185e-08
3659 2.01030101565358e-08
3660 2.00618446219991e-08
3661 2.01138792665922e-08
3662 2.00816097226664e-08
3663 2.00510294581235e-08
3664 2.01364412624194e-08
3665 2.00781157748686e-08
3666 2.00458224766109e-08
3667 2.00731819788302e-08
3668 2.00927574360854e-08
3669 2.0013734914448e-08
3670 2.00582387930304e-08
3671 2.00671751180437e-08
3672 2.00303676716196e-08
3673 2.0030308967911e-08
3674 2.00368375700144e-08
3675 2.00192400856558e-08
3676 2.00585460286007e-08
3677 2.00732257109593e-08
3678 2.00309258309161e-08
3679 2.00281595774676e-08
3680 2.00614621777007e-08
3681 2.00030092716297e-08
3682 1.99708637800278e-08
3683 1.99940316907288e-08
3684 1.99619670961937e-08
3685 1.99965589682272e-08
3686 1.9965106347275e-08
3687 1.99819993338401e-08
3688 1.99273325343086e-08
3689 2.00345855430051e-08
3690 1.99571074945837e-08
3691 2.00519632893581e-08
3692 1.99555886366554e-08
3693 2.0022110188922e-08
3694 2.00030216177538e-08
3695 1.99618207368246e-08
3696 1.99229992290384e-08
3697 2.0033356158855e-08
3698 1.99808605212404e-08
3699 1.99382002250559e-08
3700 1.99547271879652e-08
3701 1.99487726142422e-08
3702 1.99549711843439e-08
3703 1.99969478305029e-08
3704 1.99339184203673e-08
3705 1.99584217548399e-08
3706 1.99434761936068e-08
3707 1.99232801199045e-08
3708 1.99455399982895e-08
3709 1.99708385268949e-08
3710 1.99151931754571e-08
3711 1.99497308095431e-08
3712 1.99127188116321e-08
3713 1.99639109901284e-08
3714 1.98992685556121e-08
3715 1.99640790303768e-08
3716 1.98799956891982e-08
3717 1.99667244795698e-08
3718 1.98897157712707e-08
3719 1.99098325519564e-08
3720 1.99124090782821e-08
3721 1.98937341133743e-08
3722 1.99287334354814e-08
3723 1.99030217156171e-08
3724 1.98556124555438e-08
3725 1.9931343443913e-08
3726 1.98550719052726e-08
3727 1.98683320165038e-08
3728 1.99138918639541e-08
3729 1.98941694848909e-08
3730 1.98507598834219e-08
3731 1.98637503170396e-08
3732 1.98178590924414e-08
3733 1.98281836056857e-08
3734 1.98850259882644e-08
3735 1.98636367301219e-08
3736 1.98630307970404e-08
3737 1.99044310273955e-08
3738 1.98300163041232e-08
3739 1.98250609280137e-08
3740 1.97838486770685e-08
3741 1.98666139086257e-08
3742 1.98292250046528e-08
3743 1.98700883191627e-08
3744 1.98955787880095e-08
3745 1.98241038127378e-08
3746 1.98105431636897e-08
3747 1.97475362959665e-08
3748 1.98834222966404e-08
3749 1.97409792010994e-08
3750 1.9855159600457e-08
3751 1.97969295439027e-08
3752 1.97930043133177e-08
3753 1.97980504101913e-08
3754 1.97907346430881e-08
3755 1.97976423690349e-08
3756 1.97205545777734e-08
3757 1.97932780032772e-08
3758 1.97805810469021e-08
3759 1.97612010173209e-08
3760 1.97238893253626e-08
3761 1.97445834952958e-08
3762 1.97024362997844e-08
3763 1.97681738436373e-08
3764 1.97181804182378e-08
3765 1.96899783357907e-08
3766 1.97194439914838e-08
3767 1.96802305598709e-08
3768 1.97551555436437e-08
3769 1.97092741753835e-08
3770 1.97042424248117e-08
3771 1.96622460930662e-08
3772 1.96847367943676e-08
3773 1.96673907830736e-08
3774 1.96899599123057e-08
3775 1.96429288239486e-08
3776 1.96586160154943e-08
3777 1.97008144979804e-08
3778 1.96887177792426e-08
3779 1.96613369336518e-08
3780 1.96038454061132e-08
3781 1.96639633736062e-08
3782 1.96580057783002e-08
3783 1.96646669259337e-08
3784 1.96148778011107e-08
3785 1.96489125774413e-08
3786 1.96304341102405e-08
3787 1.96096511122512e-08
3788 1.96285346123481e-08
3789 1.96238515886016e-08
3790 1.96014222986207e-08
3791 1.95979864399121e-08
3792 1.95810869740853e-08
3793 1.96238801075665e-08
3794 1.96235495533159e-08
3795 1.95909694089558e-08
3796 1.95778199114649e-08
3797 1.96187040604201e-08
3798 1.955298718892e-08
3799 1.95812147332219e-08
3800 1.96017223599299e-08
3801 1.95464425072345e-08
3802 1.95396896747191e-08
3803 1.95669884504035e-08
3804 1.95146975281446e-08
3805 1.95167134457996e-08
3806 1.9551960784181e-08
3807 1.95421899622517e-08
3808 1.95437600036819e-08
3809 1.95127280533391e-08
3810 1.95346844740207e-08
3811 1.95389539849877e-08
3812 1.95601625909703e-08
3813 1.94872443000094e-08
3814 1.9513686233541e-08
3815 1.9507801685581e-08
3816 1.94918104217123e-08
3817 1.95218319758439e-08
3818 1.95151250537151e-08
3819 1.94707554079798e-08
3820 1.94418362924331e-08
3821 1.95205785653574e-08
3822 1.95169000027917e-08
3823 1.94202297292279e-08
3824 1.95476831501562e-08
3825 1.9516631299954e-08
3826 1.95164150422755e-08
3827 1.94616475535092e-08
3828 1.9544131131255e-08
3829 1.94203897359024e-08
3830 1.94633209291695e-08
3831 1.94247603579889e-08
3832 1.94717146388967e-08
3833 1.94924050238576e-08
3834 1.94089443925627e-08
3835 1.94516369536046e-08
3836 1.94412978906655e-08
3837 1.94358488525559e-08
3838 1.93863460933308e-08
3839 1.94253615073414e-08
3840 1.93920022519833e-08
3841 1.94213735134063e-08
3842 1.93850294563092e-08
3843 1.94103063684281e-08
3844 1.93566751631735e-08
3845 1.934558078176e-08
3846 1.94315614567842e-08
3847 1.93678126296781e-08
3848 1.93779284569651e-08
3849 1.93964812726932e-08
3850 1.93594626312077e-08
3851 1.93605784333251e-08
3852 1.93419655900584e-08
3853 1.93773909353823e-08
3854 1.93665810805932e-08
3855 1.93246324005081e-08
3856 1.9311073464312e-08
3857 1.93518309763441e-08
3858 1.93487793955782e-08
3859 1.93654567506307e-08
3860 1.92740208544429e-08
3861 1.92795362385922e-08
3862 1.92782846779593e-08
3863 1.93328627489109e-08
3864 1.93163050061784e-08
3865 1.92607598705763e-08
3866 1.93087871958042e-08
3867 1.93058723709782e-08
3868 1.92340318667661e-08
3869 1.92605801307977e-08
3870 1.92022509488687e-08
3871 1.92527710458634e-08
3872 1.92541538219793e-08
3873 1.93061368305436e-08
3874 1.9247504000397e-08
3875 1.92667731351293e-08
3876 1.92697490755123e-08
3877 1.93072429381935e-08
3878 1.91905443540641e-08
3879 1.92227623689867e-08
3880 1.92576757407359e-08
3881 1.92601662458713e-08
3882 1.92448000988499e-08
3883 1.92112803567568e-08
3884 1.91152082016366e-08
3885 1.91325349669924e-08
3886 1.92018148279516e-08
3887 1.92237880476398e-08
3888 1.92207105551567e-08
3889 1.92262932046106e-08
3890 1.91597656487019e-08
3891 1.91606666071209e-08
3892 1.91849875452732e-08
3893 1.91800834610234e-08
3894 1.91818122958765e-08
3895 1.90816836092012e-08
3896 1.91362051977784e-08
3897 1.9114062689507e-08
3898 1.9142968704422e-08
3899 1.9214118526012e-08
3900 1.91132159512719e-08
3901 1.91088693912711e-08
3902 1.91186917049002e-08
3903 1.91122530530663e-08
3904 1.91526600223924e-08
3905 1.90808684354948e-08
3906 1.90699092952329e-08
3907 1.90858942104022e-08
3908 1.90981958287928e-08
3909 1.91197522945163e-08
3910 1.90473123624635e-08
3911 1.90376816608939e-08
3912 1.90176392651509e-08
3913 1.91055428320208e-08
3914 1.90636218717621e-08
3915 1.91318005393626e-08
3916 1.90725665261127e-08
3917 1.89831931609596e-08
3918 1.90201759990849e-08
3919 1.90479613673133e-08
3920 1.903398902936e-08
3921 1.9023744464608e-08
3922 1.90023900721314e-08
3923 1.90084773894483e-08
3924 1.90250939331449e-08
3925 1.89853768550741e-08
3926 1.90428136424003e-08
3927 1.89514077710839e-08
3928 1.89865061113181e-08
3929 1.89543856454755e-08
3930 1.90331772982333e-08
3931 1.89178973262205e-08
3932 1.89721728314218e-08
3933 1.89306206768158e-08
3934 1.89345701802068e-08
3935 1.8944118140185e-08
3936 1.89173068818604e-08
3937 1.89361234741447e-08
3938 1.89566417430065e-08
3939 1.89054652564025e-08
3940 1.8962945841583e-08
3941 1.89297574548775e-08
3942 1.88987443934252e-08
3943 1.89177232083892e-08
3944 1.88936027045727e-08
3945 1.88592796628306e-08
3946 1.88900279813886e-08
3947 1.88843535653405e-08
3948 1.89304122311107e-08
3949 1.87863476177608e-08
3950 1.87839806087275e-08
3951 1.87350766731598e-08
3952 1.87192794145474e-08
3953 1.87791979993079e-08
3954 1.8736686409504e-08
3955 1.87377223763718e-08
3956 1.8728333859297e-08
3957 1.87476113140406e-08
3958 1.87238404298906e-08
3959 1.87262575228608e-08
3960 1.86999515323638e-08
3961 1.87108388423063e-08
3962 1.87132657925027e-08
3963 1.86611942150083e-08
3964 1.87019604347416e-08
3965 1.8665302699894e-08
3966 1.86725034938551e-08
3967 1.86681885643303e-08
3968 1.86853636909845e-08
3969 1.8652986235157e-08
3970 1.86788455676101e-08
3971 1.85860550763195e-08
3972 1.86162446023097e-08
3973 1.8629903498546e-08
3974 1.86356632196905e-08
3975 1.85995468418021e-08
3976 1.85956609599636e-08
3977 1.85780489312393e-08
3978 1.86183647548077e-08
3979 1.85619644210711e-08
3980 1.85718788192624e-08
3981 1.85913064751908e-08
3982 1.86028210693578e-08
3983 1.85901188012316e-08
3984 1.85736612881016e-08
3985 1.85689402500966e-08
3986 1.85051923606849e-08
3987 1.85325862491759e-08
3988 1.85521941311606e-08
3989 1.85237128282889e-08
3990 1.85357825583932e-08
3991 1.8532812043226e-08
3992 1.85117146487279e-08
3993 1.84937022202902e-08
3994 1.84711347974709e-08
3995 1.85589386489937e-08
3996 1.84165980967599e-08
3997 1.84447537550092e-08
3998 1.84170808237294e-08
3999 1.84625862402754e-08
4000 1.84433423258135e-08
4001 1.84296774374815e-08
4002 1.84341587849968e-08
4003 1.84660294428873e-08
4004 1.84172263253402e-08
4005 1.83735700569621e-08
4006 1.83748987865329e-08
4007 1.83634037327352e-08
4008 1.83864797143052e-08
4009 1.83884945172963e-08
4010 1.83475968511537e-08
4011 1.83749462290272e-08
4012 1.83080331552787e-08
4013 1.83442993022531e-08
4014 1.82934336563356e-08
4015 1.82726166411129e-08
4016 1.83389002654799e-08
4017 1.83216380067464e-08
4018 1.82790459537507e-08
4019 1.83196961001109e-08
4020 1.82884773543002e-08
4021 1.8242726808726e-08
4022 1.82640243764354e-08
4023 1.8270071036941e-08
4024 1.82935635142378e-08
4025 1.82623265021498e-08
4026 1.82013389280478e-08
4027 1.8211938556334e-08
4028 1.81878416469772e-08
4029 1.82107918949015e-08
4030 1.81794613407504e-08
4031 1.82141702302907e-08
4032 1.81906947225308e-08
4033 1.81875226465955e-08
4034 1.81064341155945e-08
4035 1.80853141249493e-08
4036 1.81820153497547e-08
4037 1.81114032198337e-08
4038 1.81497107987205e-08
4039 1.81584348908359e-08
4040 1.8098474205086e-08
4041 1.81119556192932e-08
4042 1.80993403433582e-08
4043 1.80733677819234e-08
4044 1.8074482599939e-08
4045 1.80476170843491e-08
4046 1.79811623812753e-08
4047 1.80529679676411e-08
4048 1.80652147552252e-08
4049 1.80684037625056e-08
4050 1.80245578054183e-08
4051 1.79677972216297e-08
4052 1.80358708994355e-08
4053 1.80053402001334e-08
4054 1.80099193345384e-08
4055 1.79910364530578e-08
4056 1.8000663349671e-08
4057 1.79934312036778e-08
4058 1.79614351698554e-08
4059 1.79427050386316e-08
4060 1.79516946170466e-08
4061 1.79644900235587e-08
4062 1.79252868912627e-08
4063 1.78980179494914e-08
4064 1.7951932947291e-08
4065 1.79234123720562e-08
4066 1.78691763292882e-08
4067 1.7854855649535e-08
4068 1.78943298090317e-08
4069 1.7893384477663e-08
4070 1.78894076197089e-08
4071 1.78348647588145e-08
4072 1.78164592772312e-08
4073 1.78193622799583e-08
4074 1.78302525832841e-08
4075 1.78320855641623e-08
4076 1.78651025675158e-08
4077 1.77947688926317e-08
4078 1.77988848957256e-08
4079 1.77270112673433e-08
4080 1.7808047955814e-08
4081 1.77722954965986e-08
4082 1.77817439372951e-08
4083 1.77493969446552e-08
4084 1.77520991258007e-08
4085 1.76928279169708e-08
4086 1.77105882035455e-08
4087 1.77610819398488e-08
4088 1.76536075744416e-08
4089 1.77300107710998e-08
4090 1.76288822379966e-08
4091 1.7721419375949e-08
4092 1.7640029853716e-08
4093 1.76809816236911e-08
4094 1.76372497948663e-08
4095 1.76550840444989e-08
4096 1.76017929662819e-08
4097 1.75763462260115e-08
4098 1.76239444340531e-08
4099 1.75590607911502e-08
4100 1.76254081760696e-08
4101 1.75940621804926e-08
4102 1.75639892741053e-08
4103 1.76139188408086e-08
4104 1.7589630143533e-08
4105 1.75747281581096e-08
4106 1.75380906748401e-08
4107 1.75724410831357e-08
4108 1.75097798718049e-08
4109 1.7569789751537e-08
4110 1.75269935001054e-08
4111 1.75206621961799e-08
4112 1.74217314872216e-08
4113 1.75492463383442e-08
4114 1.74264468815188e-08
4115 1.74470649314529e-08
4116 1.75070981991432e-08
4117 1.74486216324432e-08
4118 1.73790015898234e-08
4119 1.73792681543716e-08
4120 1.74379757247323e-08
4121 1.73989009448761e-08
4122 1.73464465893503e-08
4123 1.74187846120333e-08
4124 1.73915794099422e-08
4125 1.74130553070029e-08
4126 1.73628533692849e-08
4127 1.73537704890592e-08
4128 1.73195852948282e-08
4129 1.73279869937204e-08
4130 1.73592121597288e-08
4131 1.72894878329721e-08
4132 1.73036143751304e-08
4133 1.72864322820487e-08
4134 1.72635585560243e-08
4135 1.73220881785063e-08
4136 1.72762161891882e-08
4137 1.72306307422154e-08
4138 1.72288428892386e-08
4139 1.72131283284749e-08
4140 1.73073210194552e-08
4141 1.72403595075643e-08
4142 1.71949024148077e-08
4143 1.72538348790496e-08
4144 1.72030505374288e-08
4145 1.72214184610553e-08
4146 1.72209199327256e-08
4147 1.71527492152102e-08
4148 1.71356254106136e-08
4149 1.71685173042668e-08
4150 1.71748380868308e-08
4151 1.71128654482544e-08
4152 1.71530815382681e-08
4153 1.70755736861317e-08
4154 1.711336991983e-08
4155 1.70917868338449e-08
4156 1.70989764729512e-08
4157 1.70076894492333e-08
4158 1.70277751108117e-08
4159 1.70367533871563e-08
4160 1.70701696049491e-08
4161 1.70797660077326e-08
4162 1.71038898266751e-08
4163 1.70437834712267e-08
4164 1.70586450605192e-08
4165 1.70150566844995e-08
4166 1.69039057658082e-08
4167 1.69369159879196e-08
4168 1.69068380180271e-08
4169 1.69276041264155e-08
4170 1.69561718688094e-08
4171 1.68152893529339e-08
4172 1.68547888570458e-08
4173 1.68419920414387e-08
4174 1.68406085321315e-08
4175 1.68674026927995e-08
4176 1.68372294697683e-08
4177 1.67743738712645e-08
4178 1.67718440513553e-08
4179 1.67964175359359e-08
4180 1.67936710888661e-08
4181 1.67525321321804e-08
4182 1.67054348740869e-08
4183 1.66864038235737e-08
4184 1.6708261608045e-08
4185 1.66986793934676e-08
4186 1.6675510559061e-08
4187 1.67214662187831e-08
4188 1.67054717925552e-08
4189 1.66639542809044e-08
4190 1.66616377286211e-08
4191 1.66860512034184e-08
4192 1.67356022706677e-08
4193 1.66833023262924e-08
4194 1.66431083723495e-08
4195 1.66468080280424e-08
4196 1.66366081053582e-08
4197 1.66440733435635e-08
4198 1.65469473487612e-08
4199 1.65940537506515e-08
4200 1.66196949584752e-08
4201 1.6517697873919e-08
4202 1.65682152468705e-08
4203 1.66172059041791e-08
4204 1.66038298150539e-08
4205 1.65350962877397e-08
4206 1.65549801516285e-08
4207 1.65270338747803e-08
4208 1.65280045418825e-08
4209 1.64591143629877e-08
4210 1.65311280921898e-08
4211 1.64701670160028e-08
4212 1.64602046406337e-08
4213 1.6431448391252e-08
4214 1.64960786839252e-08
4215 1.6450433030446e-08
4216 1.6460286938802e-08
4217 1.64439105105885e-08
4218 1.65084258316739e-08
4219 1.63567687461175e-08
4220 1.63822907222055e-08
4221 1.6307327986631e-08
4222 1.63906271297698e-08
4223 1.63551449960941e-08
4224 1.63699985944454e-08
4225 1.63658603336003e-08
4226 1.63193460811506e-08
4227 1.62861572094997e-08
4228 1.6270843395727e-08
4229 1.63546586193775e-08
4230 1.6312426284415e-08
4231 1.62714706504197e-08
4232 1.62197781161666e-08
4233 1.62582434857228e-08
4234 1.62284459346296e-08
4235 1.62331398501081e-08
4236 1.62164343082694e-08
4237 1.62137703920706e-08
4238 1.61736326389494e-08
4239 1.62282102844635e-08
4240 1.62132150212102e-08
4241 1.61729203469463e-08
4242 1.62147819602332e-08
4243 1.61924985611073e-08
4244 1.6176748036667e-08
4245 1.60978907097764e-08
4246 1.61591436667052e-08
4247 1.61509549072569e-08
4248 1.61213469964672e-08
4249 1.61001017482398e-08
4250 1.60822882753209e-08
4251 1.60708766596684e-08
4252 1.61268482590238e-08
4253 1.60725594704481e-08
4254 1.61105158698049e-08
4255 1.60678224316868e-08
4256 1.60943858875662e-08
4257 1.60939242066505e-08
4258 1.60104791322446e-08
4259 1.60374234292604e-08
4260 1.60265376800695e-08
4261 1.60551357355487e-08
4262 1.60318449831109e-08
4263 1.60359723149206e-08
4264 1.60011481904476e-08
4265 1.59222284961125e-08
4266 1.5950274290244e-08
4267 1.59653227855827e-08
4268 1.59558127243997e-08
4269 1.59086928031016e-08
4270 1.59503458192489e-08
4271 1.59242568784634e-08
4272 1.58907414427034e-08
4273 1.59502850562987e-08
4274 1.59062705034074e-08
4275 1.58746383811525e-08
4276 1.58882244583047e-08
4277 1.58982163274146e-08
4278 1.58214878589469e-08
4279 1.58724510579411e-08
4280 1.58082924857084e-08
4281 1.58119876618734e-08
4282 1.58233133453756e-08
4283 1.58007579393882e-08
4284 1.58591029291166e-08
4285 1.58103922855979e-08
4286 1.5783045872908e-08
4287 1.57775722646747e-08
4288 1.57691552429284e-08
4289 1.5746860735355e-08
4290 1.57228032882095e-08
4291 1.57424015019281e-08
4292 1.57325915277617e-08
4293 1.57418649826546e-08
4294 1.57107647580368e-08
4295 1.56869900336254e-08
4296 1.57189753462461e-08
4297 1.56657150109929e-08
4298 1.57102736011439e-08
4299 1.57622093635013e-08
4300 1.5582832413763e-08
4301 1.56297787872184e-08
4302 1.56594536981913e-08
4303 1.56782057190163e-08
4304 1.55904941845364e-08
4305 1.56407341020959e-08
4306 1.55963595007691e-08
4307 1.56230002192892e-08
4308 1.55998134845348e-08
4309 1.55920006505994e-08
4310 1.55478651415741e-08
4311 1.55887191644055e-08
4312 1.55735182008065e-08
4313 1.55835472797072e-08
4314 1.55020087375668e-08
4315 1.54976343442748e-08
4316 1.55153302379141e-08
4317 1.54858783734113e-08
4318 1.54991650713754e-08
4319 1.55310156455535e-08
4320 1.55474266998468e-08
4321 1.55023181833691e-08
4322 1.5432950933647e-08
4323 1.54693057727862e-08
4324 1.54829713752136e-08
4325 1.54524219091456e-08
4326 1.54632812643385e-08
4327 1.54366633906555e-08
4328 1.54430562879782e-08
4329 1.54454741823074e-08
4330 1.54364852646971e-08
4331 1.540744912365e-08
4332 1.54678239230321e-08
4333 1.54072595317523e-08
4334 1.54478421259263e-08
4335 1.540560033364e-08
4336 1.53809769702917e-08
4337 1.54060947288315e-08
4338 1.53413554668802e-08
4339 1.53861392009347e-08
4340 1.53123072315076e-08
4341 1.54044646394347e-08
4342 1.53578823947065e-08
4343 1.54024611149772e-08
4344 1.5282936181471e-08
4345 1.53142729248934e-08
4346 1.53167793599529e-08
4347 1.53330203049862e-08
4348 1.53072364614815e-08
4349 1.52692829722412e-08
4350 1.52847418197766e-08
4351 1.52857745416846e-08
4352 1.52555085426442e-08
4353 1.52526374312956e-08
4354 1.52756527795894e-08
4355 1.52279553398138e-08
4356 1.52077148118934e-08
4357 1.52385996048388e-08
4358 1.52268404050027e-08
4359 1.51924830693595e-08
4360 1.51948904494326e-08
4361 1.51699156769602e-08
4362 1.51575653131175e-08
4363 1.52382010458751e-08
4364 1.51612607273144e-08
4365 1.51735580145029e-08
4366 1.51676864925676e-08
4367 1.5177365291219e-08
4368 1.51453437648996e-08
4369 1.51121525933107e-08
4370 1.51522423643691e-08
4371 1.50796475466564e-08
4372 1.51576175424495e-08
4373 1.50662450280059e-08
4374 1.51042981082483e-08
4375 1.50696959630991e-08
4376 1.50893355033155e-08
4377 1.49915346936425e-08
4378 1.5041567391183e-08
4379 1.50447932072417e-08
4380 1.50879961982042e-08
4381 1.50258683775384e-08
4382 1.50710774731166e-08
4383 1.49463769911762e-08
4384 1.50368909257459e-08
4385 1.49921904535333e-08
4386 1.49818691070891e-08
4387 1.50274410057882e-08
4388 1.49590715041725e-08
4389 1.50069932538521e-08
4390 1.49317910107527e-08
4391 1.49527161132923e-08
4392 1.49833451352777e-08
4393 1.48495214373767e-08
4394 1.495527979678e-08
4395 1.48767521954696e-08
4396 1.48733805898082e-08
4397 1.4910085432529e-08
4398 1.48944457611933e-08
4399 1.48406889257657e-08
4400 1.48924719063359e-08
4401 1.48282961141e-08
4402 1.48763867016122e-08
4403 1.48035289759818e-08
4404 1.48365906533776e-08
4405 1.48350061364155e-08
4406 1.48907971917467e-08
4407 1.47500201062023e-08
4408 1.48321435444743e-08
4409 1.47762668833984e-08
4410 1.47717705956119e-08
4411 1.48028139452805e-08
4412 1.47846030169596e-08
4413 1.47661330802684e-08
4414 1.47433217292381e-08
4415 1.47323930281829e-08
4416 1.47165007082783e-08
4417 1.48170399323355e-08
4418 1.47706704245465e-08
4419 1.46695420912835e-08
4420 1.47822084644034e-08
4421 1.46790826813792e-08
4422 1.4710769506987e-08
4423 1.46511678065231e-08
4424 1.46839875680982e-08
4425 1.45905277859626e-08
4426 1.46631569926114e-08
4427 1.46912259766374e-08
4428 1.46307875490415e-08
4429 1.45763874863647e-08
4430 1.46281942949855e-08
4431 1.46372861702382e-08
4432 1.46151425162167e-08
4433 1.46293408045395e-08
4434 1.45863835649962e-08
4435 1.45849204571391e-08
4436 1.46284056765644e-08
4437 1.45517419714025e-08
4438 1.45612305639808e-08
4439 1.45888071250155e-08
4440 1.45940993729887e-08
4441 1.45228740833225e-08
4442 1.45760011940332e-08
4443 1.45157468649515e-08
4444 1.4603206011099e-08
4445 1.44922088889388e-08
4446 1.45256418635498e-08
4447 1.45128816217976e-08
4448 1.45193821938783e-08
4449 1.44610234080744e-08
4450 1.45046944624028e-08
4451 1.44334591349882e-08
4452 1.4497151684445e-08
4453 1.45022336934808e-08
4454 1.4416962613506e-08
4455 1.44045891916633e-08
4456 1.4537929716063e-08
4457 1.43799349836549e-08
4458 1.4501441152337e-08
4459 1.43867464545977e-08
4460 1.43988850513921e-08
4461 1.43847746216785e-08
4462 1.44416498311095e-08
4463 1.4366935912502e-08
4464 1.43587365042386e-08
4465 1.43745554717079e-08
4466 1.44060701594562e-08
4467 1.43352769796934e-08
4468 1.43643563181861e-08
4469 1.43596516402056e-08
4470 1.43585492331511e-08
4471 1.43281410913687e-08
4472 1.43196151745073e-08
4473 1.43660444953397e-08
4474 1.42622667755887e-08
4475 1.42659440127701e-08
4476 1.42052862197595e-08
4477 1.43191373962459e-08
4478 1.42116120187552e-08
4479 1.42440490455975e-08
4480 1.42488731698087e-08
4481 1.4248530981309e-08
4482 1.42125052091657e-08
4483 1.42228110342479e-08
4484 1.42387515023046e-08
4485 1.41416586534326e-08
4486 1.42563104015281e-08
4487 1.41765676890948e-08
4488 1.42022496976146e-08
4489 1.41493062542075e-08
4490 1.41751881068686e-08
4491 1.41951099257298e-08
4492 1.41459325018189e-08
4493 1.41691254893495e-08
4494 1.41387407546212e-08
4495 1.41042805408276e-08
4496 1.41888513613964e-08
4497 1.41004546940771e-08
4498 1.4119386814393e-08
4499 1.41178388655128e-08
4500 1.41353796010435e-08
4501 1.40769214636016e-08
4502 1.40778122625917e-08
4503 1.4149834111965e-08
4504 1.40822416629938e-08
4505 1.40519698179631e-08
4506 1.40402969339704e-08
4507 1.4096649065376e-08
4508 1.40355950555282e-08
4509 1.40170553546426e-08
4510 1.41240190973946e-08
4511 1.39861387564721e-08
4512 1.39966801184777e-08
4513 1.41090176675185e-08
4514 1.3981208017988e-08
4515 1.40194567199714e-08
4516 1.39780333339168e-08
4517 1.40236984185194e-08
4518 1.3935516692154e-08
4519 1.39812675783446e-08
4520 1.40271823032556e-08
4521 1.39815774331531e-08
4522 1.39221402619683e-08
4523 1.39681247035561e-08
4524 1.39183527201325e-08
4525 1.39406827721622e-08
4526 1.39336536197909e-08
4527 1.39033685440104e-08
4528 1.3971698409776e-08
4529 1.39138153052443e-08
4530 1.38614626963296e-08
4531 1.38962051181579e-08
4532 1.39149639681868e-08
4533 1.38983961761596e-08
4534 1.37945832321407e-08
4535 1.39266980756503e-08
4536 1.38913142042618e-08
4537 1.38925685124747e-08
4538 1.38935045712607e-08
4539 1.38295424512158e-08
4540 1.39207267122465e-08
4541 1.39158993239796e-08
4542 1.38346192706251e-08
4543 1.37995498019627e-08
4544 1.38915207816837e-08
4545 1.38000396434634e-08
4546 1.37643516135988e-08
4547 1.37544390494959e-08
4548 1.38335443886639e-08
4549 1.37678486757942e-08
4550 1.38078348150827e-08
4551 1.37715614076939e-08
4552 1.37929655301683e-08
4553 1.37115757841144e-08
4554 1.38045839683798e-08
4555 1.37513429754321e-08
4556 1.37900984467088e-08
4557 1.37872682537399e-08
4558 1.37590351787509e-08
4559 1.37710094820775e-08
4560 1.37154572898979e-08
4561 1.37124729442384e-08
4562 1.37876242245483e-08
4563 1.37644778179791e-08
4564 1.36822918817003e-08
4565 1.36664313927781e-08
4566 1.37266312916218e-08
4567 1.36917301638562e-08
4568 1.36958899563666e-08
4569 1.36390584546398e-08
4570 1.36936171530877e-08
4571 1.36147212290716e-08
4572 1.37296443254442e-08
4573 1.36519967903759e-08
4574 1.36761240940864e-08
4575 1.36327548467818e-08
4576 1.36503979368818e-08
4577 1.36275914961459e-08
4578 1.36338372542788e-08
4579 1.37243567910339e-08
4580 1.36029408239757e-08
4581 1.35935331448955e-08
4582 1.36243445516548e-08
4583 1.36126607115372e-08
4584 1.35220003039116e-08
4585 1.36723072756695e-08
4586 1.3557148236476e-08
4587 1.36112903112995e-08
4588 1.35378039609613e-08
4589 1.35751134022755e-08
4590 1.35818014852518e-08
4591 1.35350943053503e-08
4592 1.35814354247366e-08
4593 1.35087775694487e-08
4594 1.35785522106247e-08
4595 1.35317337153218e-08
4596 1.35657332513972e-08
4597 1.35002122143568e-08
4598 1.36091403173388e-08
4599 1.34938741482848e-08
4600 1.35378515100371e-08
4601 1.35244136725277e-08
4602 1.35007577264368e-08
4603 1.35109208976303e-08
4604 1.35066804038964e-08
4605 1.35117400765772e-08
4606 1.35894733475084e-08
4607 1.35048950804517e-08
4608 1.33866543494676e-08
4609 1.34597702263761e-08
4610 1.34751184956983e-08
4611 1.34902459807407e-08
4612 1.34512133422859e-08
4613 1.34667661351173e-08
4614 1.35142385708775e-08
4615 1.34254932944167e-08
4616 1.34688738322453e-08
4617 1.34154798199226e-08
4618 1.34175918771184e-08
4619 1.34035260623833e-08
4620 1.34579290951287e-08
4621 1.35160863918848e-08
4622 1.3355622180633e-08
4623 1.33795438848061e-08
4624 1.34039695178778e-08
4625 1.34118037524722e-08
4626 1.3407230337048e-08
4627 1.34123350643556e-08
4628 1.34068503880869e-08
4629 1.33815846492524e-08
4630 1.33921556173533e-08
4631 1.33977782841299e-08
4632 1.34307625323515e-08
4633 1.33674641986659e-08
4634 1.34216696592304e-08
4635 1.33432817470691e-08
4636 1.33956341832331e-08
4637 1.33340151826644e-08
4638 1.33868387091063e-08
4639 1.33552164220951e-08
4640 1.33231611156859e-08
4641 1.33836408364729e-08
4642 1.33281464265167e-08
4643 1.33721645383389e-08
4644 1.33461716793626e-08
4645 1.33229227521348e-08
4646 1.33246282838506e-08
4647 1.33370233315055e-08
4648 1.32963008132592e-08
4649 1.33193576221657e-08
4650 1.32654683975275e-08
4651 1.33608244623495e-08
4652 1.32804582997359e-08
4653 1.32979344340534e-08
4654 1.32623174429192e-08
4655 1.32721030587923e-08
4656 1.32451530316935e-08
4657 1.32752904797862e-08
4658 1.33166751536962e-08
4659 1.32367192566818e-08
4660 1.32444676457233e-08
4661 1.3254211304492e-08
4662 1.32409760431607e-08
4663 1.3233709367011e-08
4664 1.32116556086359e-08
4665 1.32078894856669e-08
4666 1.32039032365583e-08
4667 1.32191600608067e-08
4668 1.32357782178794e-08
4669 1.32024606576131e-08
4670 1.3183118301896e-08
4671 1.3213982399396e-08
4672 1.32347773331709e-08
4673 1.31554376410214e-08
4674 1.3195374318542e-08
4675 1.32164485688868e-08
4676 1.3141307185105e-08
4677 1.31886955303528e-08
4678 1.31932414140223e-08
4679 1.31529342493053e-08
4680 1.32079995114331e-08
4681 1.31649088026542e-08
4682 1.31159350469368e-08
4683 1.31268333229961e-08
4684 1.31244813368347e-08
4685 1.31218054364446e-08
4686 1.31543436299175e-08
4687 1.31396392899497e-08
4688 1.31069689972563e-08
4689 1.31215828527154e-08
4690 1.30861055063214e-08
4691 1.31056517678196e-08
4692 1.31388402198063e-08
4693 1.30840606407112e-08
4694 1.30962281978952e-08
4695 1.310902506102e-08
4696 1.30734314547354e-08
4697 1.3087421073088e-08
4698 1.30925590076103e-08
4699 1.31138889205573e-08
4700 1.30520919841004e-08
4701 1.31141086572306e-08
4702 1.3036550515988e-08
4703 1.30408307734697e-08
4704 1.30979523538244e-08
4705 1.30449120208098e-08
4706 1.30275470704433e-08
4707 1.31118711230727e-08
4708 1.30201823513865e-08
4709 1.30500420083557e-08
4710 1.30550556609599e-08
4711 1.30312785961983e-08
4712 1.30217303353497e-08
4713 1.30871490564566e-08
4714 1.30067358306007e-08
4715 1.30171802998902e-08
4716 1.30601568200106e-08
4717 1.30149657389111e-08
4718 1.29864215026032e-08
4719 1.30533825521972e-08
4720 1.29660037546664e-08
4721 1.30182411002266e-08
4722 1.30274308696166e-08
4723 1.29737988436851e-08
4724 1.29939155453229e-08
4725 1.30071766477613e-08
4726 1.29990788688694e-08
4727 1.29756121975966e-08
4728 1.29904866548181e-08
4729 1.29476558754504e-08
4730 1.2997163162165e-08
4731 1.29968980773221e-08
4732 1.2954328323378e-08
4733 1.29352602962207e-08
4734 1.29989563366628e-08
4735 1.30164572000879e-08
4736 1.28993207946237e-08
4737 1.29560809005724e-08
4738 1.29684587428791e-08
4739 1.29172222917973e-08
4740 1.29605482963591e-08
4741 1.29344935846376e-08
4742 1.29353451288061e-08
4743 1.29500121213155e-08
4744 1.29204601670985e-08
4745 1.29283013428072e-08
4746 1.29468468830218e-08
4747 1.29135161257565e-08
4748 1.28784008404104e-08
4749 1.28944584476542e-08
4750 1.29321187247733e-08
4751 1.29353873581373e-08
4752 1.29083490971738e-08
4753 1.29285391432532e-08
4754 1.28841305153671e-08
4755 1.29308937437678e-08
4756 1.2877291649005e-08
4757 1.289704641394e-08
4758 1.29147838587862e-08
4759 1.28645258765658e-08
4760 1.28861609911546e-08
4761 1.28822058953482e-08
4762 1.28914934087732e-08
4763 1.28547141931001e-08
4764 1.28941869208532e-08
4765 1.28291282335269e-08
4766 1.28974710755791e-08
4767 1.28004008228544e-08
4768 1.28321281893662e-08
4769 1.28371710421682e-08
4770 1.28780097403691e-08
4771 1.28829309340439e-08
4772 1.28224293964863e-08
4773 1.28372876226912e-08
4774 1.28161237529234e-08
4775 1.28229979927674e-08
4776 1.28442454365718e-08
4777 1.28389936340234e-08
4778 1.27931216220567e-08
4779 1.2860870918896e-08
4780 1.27821152493546e-08
4781 1.28233084089047e-08
4782 1.27965600582591e-08
4783 1.28118758708773e-08
4784 1.28231267009227e-08
4785 1.28009283733022e-08
4786 1.28005216737392e-08
4787 1.28156988137285e-08
4788 1.27840016435066e-08
4789 1.27693137854656e-08
4790 1.2774123016257e-08
4791 1.27696231753127e-08
4792 1.28133781749362e-08
4793 1.27535789302691e-08
4794 1.27936344576085e-08
4795 1.27372253007074e-08
4796 1.27736340336249e-08
4797 1.2732245298519e-08
4798 1.28052089252151e-08
4799 1.27541174110846e-08
4800 1.27587679488173e-08
4801 1.27400950105105e-08
4802 1.27526809610146e-08
4803 1.27571696015849e-08
4804 1.27313641136162e-08
4805 1.27930670998921e-08
4806 1.26937298725593e-08
4807 1.27560710168062e-08
4808 1.2751249695242e-08
4809 1.26937667550564e-08
4810 1.27037884065651e-08
4811 1.27578812452178e-08
4812 1.27417984683298e-08
4813 1.26558338320315e-08
4814 1.27447073170472e-08
4815 1.27170528649856e-08
4816 1.26925496961583e-08
4817 1.27651127583839e-08
4818 1.2718575193027e-08
4819 1.26803516766749e-08
4820 1.27247187506896e-08
4821 1.27095746518258e-08
4822 1.27086079557692e-08
4823 1.27106502794128e-08
4824 1.26944092917469e-08
4825 1.27097256301667e-08
4826 1.26814855496704e-08
4827 1.26408325673566e-08
4828 1.26544231138226e-08
4829 1.27092215245206e-08
4830 1.26563473368257e-08
4831 1.26405643272598e-08
4832 1.27208022653313e-08
4833 1.26854819253452e-08
4834 1.2682508213846e-08
4835 1.26748470856697e-08
4836 1.26674696252493e-08
4837 1.26995244134065e-08
4838 1.26843467240789e-08
4839 1.26381161731359e-08
4840 1.26628761756287e-08
4841 1.27010090698043e-08
4842 1.26281496943825e-08
4843 1.26550584478302e-08
4844 1.26565171125748e-08
4845 1.26324591298577e-08
4846 1.26058759915004e-08
4847 1.26568881606559e-08
4848 1.26236713060557e-08
4849 1.2623362067643e-08
4850 1.26308835448619e-08
4851 1.26261141959461e-08
4852 1.26227248458122e-08
4853 1.26518438774426e-08
4854 1.26755996201489e-08
4855 1.26005724445655e-08
4856 1.26184479904978e-08
4857 1.26288369748373e-08
4858 1.26087886336279e-08
4859 1.26114042080516e-08
4860 1.25755991806642e-08
4861 1.26273747849126e-08
4862 1.26233819890409e-08
4863 1.26255582215684e-08
4864 1.26024531454938e-08
4865 1.25996096755898e-08
4866 1.26020511030944e-08
4867 1.26117982319762e-08
4868 1.25791548337695e-08
4869 1.25528973908828e-08
4870 1.2606577542762e-08
4871 1.25667832455711e-08
4872 1.25919419917686e-08
4873 1.2545075549486e-08
4874 1.26128899009537e-08
4875 1.26030805427391e-08
4876 1.25629923024562e-08
4877 1.25769474377257e-08
4878 1.25713969461039e-08
4879 1.25627332625555e-08
4880 1.25627240215032e-08
4881 1.25677987723449e-08
4882 1.25716699330702e-08
4883 1.25571185494167e-08
4884 1.25623670883535e-08
4885 1.25426960830843e-08
4886 1.25143658942228e-08
4887 1.26353292353443e-08
4888 1.25811011453258e-08
4889 1.24977333624265e-08
4890 1.24957274327819e-08
4891 1.26635733779246e-08
4892 1.24523953961564e-08
4893 1.25376226192131e-08
4894 1.25251265687965e-08
4895 1.24906993796969e-08
4896 1.2498473553002e-08
4897 1.25061393005943e-08
4898 1.25102364889607e-08
4899 1.25457491497727e-08
4900 1.24737584283885e-08
4901 1.25354769506814e-08
4902 1.24650873893906e-08
4903 1.25138356033005e-08
4904 1.24738586588791e-08
4905 1.25288393904022e-08
4906 1.2456492127999e-08
4907 1.25169431166583e-08
4908 1.24552243758735e-08
4909 1.24727059880314e-08
4910 1.24735921507302e-08
4911 1.24820302498385e-08
4912 1.24734973017127e-08
4913 1.2468504888119e-08
4914 1.24611617517978e-08
4915 1.24577201106035e-08
4916 1.24552010518642e-08
4917 1.24559578038586e-08
4918 1.24612584397887e-08
4919 1.24606279197081e-08
4920 1.2455757301133e-08
4921 1.24812089787874e-08
4922 1.24596049144721e-08
4923 1.24609104728002e-08
4924 1.23527107720456e-08
4925 1.25790504710288e-08
4926 1.23432885019703e-08
4927 1.24272037389517e-08
4928 1.24208065646059e-08
4929 1.24460164925289e-08
4930 1.23491598755798e-08
4931 1.25030751294553e-08
4932 1.24218818209343e-08
4933 1.24090279975952e-08
4934 1.23913747662741e-08
4935 1.23834825189739e-08
4936 1.25274381779228e-08
4937 1.23338185908928e-08
4938 1.24848967444358e-08
4939 1.23215589979964e-08
4940 1.24754465637977e-08
4941 1.24553645504122e-08
4942 1.23636374782521e-08
4943 1.2381325161126e-08
4944 1.23716382045558e-08
4945 1.2399631608595e-08
4946 1.23002176577458e-08
4947 1.250120953733e-08
4948 1.22770575390341e-08
4949 1.24957210272392e-08
4950 1.2338916660859e-08
4951 1.25069504570696e-08
4952 1.2290783447888e-08
4953 1.24222442106081e-08
4954 1.23643923144456e-08
4955 1.22991122135652e-08
4956 1.24614365719644e-08
4957 1.22744726054336e-08
4958 1.24876588984968e-08
4959 1.22495333818584e-08
4960 1.24064542661451e-08
4961 1.23895572161636e-08
4962 1.23226516657304e-08
4963 1.23872960084981e-08
4964 1.25616524759842e-08
4965 1.23268742080462e-08
4966 1.23596656740332e-08
4967 1.23922596872994e-08
4968 1.23263861180334e-08
4969 1.25889751849684e-08
4970 1.22876214838641e-08
4971 1.23104543692243e-08
4972 1.23515386736273e-08
4973 1.2525284604159e-08
4974 1.22498012866679e-08
4975 1.24422861746964e-08
4976 1.23115978643007e-08
4977 1.23211408662449e-08
4978 1.23445018895829e-08
4979 1.23643393474815e-08
4980 1.23511689698041e-08
4981 1.22488134741694e-08
4982 1.23301429852951e-08
4983 1.23378549843345e-08
4984 1.22839793799123e-08
4985 1.24443025466547e-08
4986 1.22205992121316e-08
4987 1.24033281023728e-08
4988 1.22371920450171e-08
4989 1.24159665015888e-08
4990 1.22234792931941e-08
4991 1.23748956846192e-08
4992 1.22458326892172e-08
4993 1.23812230183873e-08
4994 1.22081892635073e-08
4995 1.23786632770795e-08
4996 1.22532220849791e-08
4997 1.23634774413794e-08
4998 1.22278668301945e-08
4999 1.24522625069012e-08
};
\addlegendentry{Train}
\addplot [semithick, black]
table {%
0 0.00325515912845731
1 0.000995962880551815
2 0.000406711827963591
3 0.000229110664804466
4 0.00019099612836726
5 0.000180200979229994
6 0.000170479150256142
7 0.000158077789819799
8 0.000141750555485487
9 0.000119582837214693
10 9.31990216486156e-05
11 6.83226244291291e-05
12 4.95343447255436e-05
13 3.85890489269514e-05
14 3.29497415805236e-05
15 2.96805537800537e-05
16 2.72613324341364e-05
17 2.50375014729798e-05
18 2.2613923647441e-05
19 1.98212419491028e-05
20 1.6647667507641e-05
21 1.33240828290582e-05
22 1.04137798189186e-05
23 8.30350836622529e-06
24 6.98973508406198e-06
25 6.26542350801174e-06
26 5.88433340453776e-06
27 5.66067501495127e-06
28 5.51697758055525e-06
29 5.39687334821792e-06
30 5.27402062289184e-06
31 5.15247529619955e-06
32 5.01721706314129e-06
33 4.87310626340332e-06
34 4.70917575512431e-06
35 4.53457232651999e-06
36 4.35217270933208e-06
37 4.15744125348283e-06
38 3.9596661736141e-06
39 3.76154116565885e-06
40 3.56626696884632e-06
41 3.37277333528618e-06
42 3.18269712806796e-06
43 2.99632893074886e-06
44 2.81752363662235e-06
45 2.647349219842e-06
46 2.48700393967738e-06
47 2.33781292990898e-06
48 2.20122001337586e-06
49 2.07764992410375e-06
50 1.96794553630752e-06
51 1.87079683655611e-06
52 1.78505229087023e-06
53 1.71029603279749e-06
54 1.64511391176347e-06
55 1.58889224621817e-06
56 1.5399557469209e-06
57 1.49715890529478e-06
58 1.45889327995974e-06
59 1.42421117743652e-06
60 1.39182179736963e-06
61 1.35834773118404e-06
62 1.32547324938059e-06
63 1.29291527173336e-06
64 1.26101417663449e-06
65 1.26090117191779e-06
66 1.20157824312628e-06
67 1.17061972559895e-06
68 1.14009912977053e-06
69 1.1094434739789e-06
70 1.12134296159638e-06
71 1.0930331200143e-06
72 1.06329821392137e-06
73 9.93865114651271e-07
74 9.64382138590736e-07
75 9.39907522479189e-07
76 9.15891405384173e-07
77 8.96064307198685e-07
78 8.77666764154128e-07
79 8.59116255469417e-07
80 8.43339705625112e-07
81 8.28355496196309e-07
82 8.14282543615263e-07
83 8.01091232460749e-07
84 7.88376951277314e-07
85 7.7647200669162e-07
86 7.65237018640619e-07
87 7.54798179514182e-07
88 7.44265435059788e-07
89 7.34764853405068e-07
90 7.25669224266312e-07
91 7.17075465672679e-07
92 7.30558895156719e-07
93 7.2124089456338e-07
94 7.0975085009195e-07
95 7.0684029651602e-07
96 6.79147262871993e-07
97 6.93132562901155e-07
98 6.66063613152801e-07
99 6.80307721268036e-07
100 6.53859956400993e-07
101 6.68063478315162e-07
102 6.42690167751425e-07
103 6.56318036362791e-07
104 6.31469731615653e-07
105 6.31490593150374e-07
106 6.40550922526018e-07
107 6.20105538473581e-07
108 6.29641647265089e-07
109 6.07102322192077e-07
110 6.19580703187239e-07
111 5.97919267875113e-07
112 6.09498670200992e-07
113 5.88735474593705e-07
114 6.00782186666038e-07
115 5.79865798044921e-07
116 5.77701086967863e-07
117 5.71275052152487e-07
118 5.66986102512601e-07
119 5.79477671180939e-07
120 5.58563726826833e-07
121 5.54663188268023e-07
122 5.67245820093376e-07
123 5.47385980098625e-07
124 5.58833562536165e-07
125 5.42344480436441e-07
126 5.36032359832461e-07
127 5.32334581748728e-07
128 5.28945861333341e-07
129 5.25704592746479e-07
130 5.22396078395104e-07
131 5.19135880949761e-07
132 5.15668830303184e-07
133 5.12456949763873e-07
134 5.09458288888709e-07
135 5.07158006257669e-07
136 5.0391457762089e-07
137 5.17422847678972e-07
138 4.97831138090987e-07
139 4.94980213261442e-07
140 4.9240259158978e-07
141 4.89592252961302e-07
142 5.03185844991094e-07
143 4.98332610732177e-07
144 4.94812127271871e-07
145 4.77688502087403e-07
146 4.74924348736749e-07
147 4.72534765094679e-07
148 4.83530016026634e-07
149 4.79727873425873e-07
150 4.77133596632484e-07
151 4.74554155971418e-07
152 4.71783209832211e-07
153 4.69199846975243e-07
154 4.66512716457146e-07
155 4.64051566950729e-07
156 4.61589394262774e-07
157 4.59211747738664e-07
158 4.56908907153775e-07
159 4.54393813242859e-07
160 4.52160463737528e-07
161 4.49718413619848e-07
162 4.47402925374263e-07
163 4.45136635107701e-07
164 4.42860255134292e-07
165 4.405540892094e-07
166 4.38296353877377e-07
167 4.35965603173827e-07
168 4.3364266844037e-07
169 4.31464258099368e-07
170 4.29192681394852e-07
171 4.27019642756932e-07
172 4.24821166689071e-07
173 4.22576420078258e-07
174 4.20254337996084e-07
175 4.18182480643736e-07
176 4.16100164102318e-07
177 4.13916893649002e-07
178 4.11302266911662e-07
179 4.08876701385452e-07
180 4.06410237019372e-07
181 4.04191013103627e-07
182 4.02019594503145e-07
183 3.99790110350295e-07
184 3.97648904026937e-07
185 3.95618172888135e-07
186 3.93339803395065e-07
187 3.91363244034437e-07
188 3.89242160281356e-07
189 3.87133979984355e-07
190 3.85096626587256e-07
191 3.83163268224962e-07
192 3.81098516299971e-07
193 3.79056132260303e-07
194 3.77116180061421e-07
195 3.7503500038838e-07
196 3.7302760347302e-07
197 3.71664214071643e-07
198 3.69897009022679e-07
199 3.68101296999157e-07
200 3.66393408057775e-07
201 3.64674264119458e-07
202 3.62664195563411e-07
203 3.60748117600451e-07
204 3.59031247398889e-07
205 3.58061214456029e-07
206 3.56198256667994e-07
207 3.54236874500202e-07
208 3.52459807118066e-07
209 3.50565841245043e-07
210 3.4876418908425e-07
211 3.46701455100629e-07
212 3.44962870713061e-07
213 3.43279964454268e-07
214 3.42136814879268e-07
215 3.40594311865061e-07
216 3.38810053790439e-07
217 3.37079114842709e-07
218 3.35430883069421e-07
219 3.33934110585687e-07
220 3.32268541569647e-07
221 3.30765573153258e-07
222 3.29128198472972e-07
223 3.27673831179709e-07
224 3.26104327541543e-07
225 3.24336554058391e-07
226 3.22795955298716e-07
227 3.21325018148855e-07
228 3.19841660711973e-07
229 3.18513855290803e-07
230 3.1735686434331e-07
231 3.15926570237934e-07
232 3.14630256070814e-07
233 3.13249671535232e-07
234 3.11865420599133e-07
235 3.10580617224332e-07
236 3.09279442944899e-07
237 3.07975227542556e-07
238 3.06591800836031e-07
239 3.05287585433689e-07
240 3.0397421824091e-07
241 3.02622538583819e-07
242 3.01242749856101e-07
243 3.00529393371107e-07
244 2.9580291993625e-07
245 2.94141756285171e-07
246 2.92826086933928e-07
247 2.9154114145058e-07
248 2.90628150878547e-07
249 2.88748594812205e-07
250 2.87390946596133e-07
251 2.85662537180542e-07
252 2.84210386780614e-07
253 2.83201103457031e-07
254 2.81831347592743e-07
255 2.80602421298681e-07
256 2.78733153891153e-07
257 2.7715802275452e-07
258 2.77723898989279e-07
259 2.74382102816162e-07
260 2.74733366723012e-07
261 2.73389531457724e-07
262 2.71948380259346e-07
263 2.70437425342607e-07
264 2.69090662641247e-07
265 2.6773179229167e-07
266 2.66411575466918e-07
267 2.64959197693315e-07
268 2.64163332985845e-07
269 2.62518881299911e-07
270 2.59881034025966e-07
271 2.58482799608828e-07
272 2.57256516533744e-07
273 2.55979585972455e-07
274 2.54688245604484e-07
275 2.53403356964554e-07
276 2.51760639002896e-07
277 2.50482742103486e-07
278 2.49339478841648e-07
279 2.48687683779281e-07
280 2.47592282676123e-07
281 2.46283832439076e-07
282 2.4518038799215e-07
283 2.44108349534145e-07
284 2.42454319732133e-07
285 2.41822704083461e-07
286 2.399910101758e-07
287 2.39254148937107e-07
288 2.38050148482216e-07
289 2.36141644904819e-07
290 2.3479211108679e-07
291 2.341878655443e-07
292 2.32349933071418e-07
293 2.31181999765795e-07
294 2.29937910489753e-07
295 2.2888296768997e-07
296 2.27647220185645e-07
297 2.26523340529639e-07
298 2.25297668521307e-07
299 2.24231385459461e-07
300 2.23026773937818e-07
301 2.21973792235985e-07
302 2.2098159035977e-07
303 2.20005560436221e-07
304 2.18952351360713e-07
305 2.18890491510138e-07
306 2.17885357756131e-07
307 2.16135433106501e-07
308 2.15391793290109e-07
309 2.14249638474939e-07
310 2.13329741427515e-07
311 2.1244154879696e-07
312 2.11474926459232e-07
313 2.10653183785325e-07
314 2.09703912901205e-07
315 2.08832588555197e-07
316 2.08044227179016e-07
317 2.08246433430759e-07
318 2.07498999316158e-07
319 2.0691233260095e-07
320 2.06205683639382e-07
321 2.05530398034171e-07
322 2.04806056558482e-07
323 2.04196112463251e-07
324 2.03144153942958e-07
325 2.02600460852409e-07
326 2.01788822096205e-07
327 2.01127264176648e-07
328 2.00359465907241e-07
329 1.99738721562426e-07
330 1.99057097916011e-07
331 1.98419300545538e-07
332 1.97557611159027e-07
333 1.96948363395677e-07
334 1.96172749156176e-07
335 1.95519916701414e-07
336 1.94738362324642e-07
337 1.9404092199693e-07
338 1.93127306147289e-07
339 1.92423414091536e-07
340 1.91707968610899e-07
341 1.91035240959536e-07
342 1.90347236639354e-07
343 1.89711315101704e-07
344 1.88996423844401e-07
345 1.88488840535683e-07
346 1.87894869441152e-07
347 1.87084310709906e-07
348 1.86629648624148e-07
349 1.86065591378792e-07
350 1.8532996648446e-07
351 1.84660891022759e-07
352 1.83950376708708e-07
353 1.83382866225656e-07
354 1.82761141331866e-07
355 1.82118654379337e-07
356 1.81452605829691e-07
357 1.80879567324155e-07
358 1.802782918503e-07
359 1.79683709689016e-07
360 1.79092509711154e-07
361 1.78543302808976e-07
362 1.77915225663128e-07
363 1.77378055354893e-07
364 1.76805812657221e-07
365 1.76270319229843e-07
366 1.7581542977041e-07
367 1.75117520484491e-07
368 1.74709327893652e-07
369 1.74012484421837e-07
370 1.73562142435912e-07
371 1.73014939264249e-07
372 1.72355896665977e-07
373 1.71798575365756e-07
374 1.71292029449432e-07
375 1.70834368873329e-07
376 1.7016569131556e-07
377 1.69876997802021e-07
378 1.69078973044634e-07
379 1.68786883136818e-07
380 1.6802580660169e-07
381 1.67611617030161e-07
382 1.67097851999642e-07
383 1.66505728316224e-07
384 1.66013208513505e-07
385 1.65483768910235e-07
386 1.65071284641272e-07
387 1.64365005161926e-07
388 1.64044593020662e-07
389 1.63798659968961e-07
390 1.63231135275055e-07
391 1.62779144829983e-07
392 1.62417421734062e-07
393 1.61883662030959e-07
394 1.61546608978824e-07
395 1.61130387255071e-07
396 1.60726159492697e-07
397 1.60387330083722e-07
398 1.59909092189991e-07
399 1.59435472824043e-07
400 1.58964951424423e-07
401 1.58633739033576e-07
402 1.58083082624216e-07
403 1.57654127974638e-07
404 1.57314801185748e-07
405 1.56700338038718e-07
406 1.56477995005844e-07
407 1.56022608166495e-07
408 1.55659677147924e-07
409 1.55098206278126e-07
410 1.54723721834671e-07
411 1.54224863990748e-07
412 1.54039028643638e-07
413 1.53605384412003e-07
414 1.53102860167564e-07
415 1.52831191257974e-07
416 1.52381460338802e-07
417 1.52022408883568e-07
418 1.51685085825193e-07
419 1.5123188745747e-07
420 1.50820085309533e-07
421 1.50204542137544e-07
422 1.49971583596198e-07
423 1.4962894567816e-07
424 1.49240591440503e-07
425 1.50567501577825e-07
426 1.50161781675706e-07
427 1.49902291468607e-07
428 1.49565821061515e-07
429 1.49085295220175e-07
430 1.48842360658819e-07
431 1.48493640494962e-07
432 1.48153773693593e-07
433 1.47751208601221e-07
434 1.47423776297728e-07
435 1.46983836657455e-07
436 1.46731778727371e-07
437 1.46496034858501e-07
438 1.46093100283906e-07
439 1.45869961443168e-07
440 1.45431727105461e-07
441 1.45066735512955e-07
442 1.447097304208e-07
443 1.44496794973747e-07
444 1.44032071602851e-07
445 1.43755201520435e-07
446 1.43609241831655e-07
447 1.43137569352803e-07
448 1.43017302889348e-07
449 1.42648062251283e-07
450 1.42329142249764e-07
451 1.41959517918622e-07
452 1.41709421086489e-07
453 1.41498631478498e-07
454 1.41238828632595e-07
455 1.40940329629302e-07
456 1.40605919796144e-07
457 1.40367419021459e-07
458 1.39999457360318e-07
459 1.39920175001862e-07
460 1.39584528824344e-07
461 1.39248982122808e-07
462 1.39030035484211e-07
463 1.38736595545197e-07
464 1.3867882842078e-07
465 1.38207539635005e-07
466 1.38007692385145e-07
467 1.37566971147862e-07
468 1.37413522338647e-07
469 1.37236412456332e-07
470 1.37142023959314e-07
471 1.36724096932994e-07
472 1.36621878255028e-07
473 1.3645259855366e-07
474 1.36290012164864e-07
475 1.36099984615612e-07
476 1.35765276354505e-07
477 1.35529901967857e-07
478 1.35273225509991e-07
479 1.3514076613319e-07
480 1.34809894802856e-07
481 1.34606011670257e-07
482 1.34568409748681e-07
483 1.34086334924177e-07
484 1.33836081772643e-07
485 1.33647972688777e-07
486 1.33292786586026e-07
487 1.33051031525611e-07
488 1.32906137650934e-07
489 1.32754706783089e-07
490 1.32266237073964e-07
491 1.32031246380393e-07
492 1.32055674839648e-07
493 1.31570999428732e-07
494 1.31285275983828e-07
495 1.31008519588249e-07
496 1.3065870518858e-07
497 1.29938584336742e-07
498 1.29713356500361e-07
499 1.29536601889413e-07
500 1.29180406815976e-07
501 1.28960579104387e-07
502 1.28806590282693e-07
503 1.28583252489989e-07
504 1.28396678178433e-07
505 1.28178555769409e-07
506 1.2781077884938e-07
507 1.27595967569505e-07
508 1.27433466445837e-07
509 1.27244220493594e-07
510 1.26916816611811e-07
511 1.26679793766016e-07
512 1.26509391407126e-07
513 1.2635150881124e-07
514 1.26009368273117e-07
515 1.25780104553996e-07
516 1.25531428807335e-07
517 1.25229988157116e-07
518 1.24907927556706e-07
519 1.24690060943067e-07
520 1.24460086681211e-07
521 1.24193050510257e-07
522 1.23864012380182e-07
523 1.23718251643368e-07
524 1.23439235721889e-07
525 1.23746858093909e-07
526 1.23660598205788e-07
527 1.23297667187217e-07
528 1.22784442169177e-07
529 1.22458587270557e-07
530 1.22152414405718e-07
531 1.21876496450568e-07
532 1.21588556112329e-07
533 1.21215222748106e-07
534 1.21032826427836e-07
535 1.20650980761638e-07
536 1.2046503172769e-07
537 1.20308783380096e-07
538 1.20049179486159e-07
539 1.19791607744446e-07
540 1.19640532147969e-07
541 1.19720894531383e-07
542 1.18710147489764e-07
543 1.18878219268481e-07
544 1.18478119759402e-07
545 1.1829096990823e-07
546 1.18101283419492e-07
547 1.18075760724423e-07
548 1.17678517597142e-07
549 1.17661208776099e-07
550 1.17280592348834e-07
551 1.16703013475217e-07
552 1.17188278636604e-07
553 1.17082770145771e-07
554 1.1661816756714e-07
555 1.16619325751799e-07
556 1.16508687142414e-07
557 1.16044077458355e-07
558 1.16095748126099e-07
559 1.15560865765474e-07
560 1.15122539057211e-07
561 1.15290077928876e-07
562 1.14801402162357e-07
563 1.14846685050907e-07
564 1.14406105922171e-07
565 1.14318119415202e-07
566 1.14443786003449e-07
567 1.1353220941146e-07
568 1.13742522955818e-07
569 1.1369589003607e-07
570 1.13551216429641e-07
571 1.13351340758072e-07
572 1.13540806978563e-07
573 1.12962119658278e-07
574 1.12701066257159e-07
575 1.12637501104018e-07
576 1.12531779450364e-07
577 1.12194257440024e-07
578 1.11704217431452e-07
579 1.11659076651449e-07
580 1.11235202382431e-07
581 1.11222128396093e-07
582 1.11240630928933e-07
583 1.10995053148599e-07
584 1.10607281555986e-07
585 1.10649267526242e-07
586 1.1022360268953e-07
587 1.10400762309837e-07
588 1.0988180321192e-07
589 1.10062472913341e-07
590 1.09856934216168e-07
591 1.09496802735976e-07
592 1.09253996072312e-07
593 1.09058085229208e-07
594 1.08897850736867e-07
595 1.08917241448125e-07
596 1.08862842296276e-07
597 1.08454493386034e-07
598 1.08448645619319e-07
599 1.08176614332933e-07
600 1.08104920570895e-07
601 1.07912910607411e-07
602 1.07750643962845e-07
603 1.07561398010603e-07
604 1.07673081117809e-07
605 1.07145318395396e-07
606 1.07148636629972e-07
607 1.07030849960665e-07
608 1.0679834616667e-07
609 1.06911599573323e-07
610 1.06392093357499e-07
611 1.06420991130562e-07
612 1.06336798921802e-07
613 1.06006396549674e-07
614 1.05868956268296e-07
615 1.05971899699853e-07
616 1.056622949136e-07
617 1.05374311942796e-07
618 1.05326989796595e-07
619 1.05197706545823e-07
620 1.05010002471317e-07
621 1.04758527186277e-07
622 1.04706408876609e-07
623 1.04450016635838e-07
624 1.04416201907043e-07
625 1.04180337245907e-07
626 1.0402950323396e-07
627 1.03923035510434e-07
628 1.03722776145787e-07
629 1.03809952634037e-07
630 1.0341764777877e-07
631 1.03298773979077e-07
632 1.03149801589097e-07
633 1.03096020609428e-07
634 1.02938635393457e-07
635 1.02706877669334e-07
636 1.02609988061886e-07
637 1.0233421932071e-07
638 1.0210379741693e-07
639 1.02081649799857e-07
640 1.01841500566024e-07
641 1.02356182196672e-07
642 1.01960218046315e-07
643 1.01815558650742e-07
644 1.01576389965885e-07
645 1.0138201389509e-07
646 1.01221196757706e-07
647 1.00789179668936e-07
648 1.00728470897593e-07
649 1.00280509229833e-07
650 1.00169266659123e-07
651 1.00089579291307e-07
652 9.98675631080914e-08
653 9.97505438249391e-08
654 1.00072519160221e-07
655 9.94860300806977e-08
656 9.91798074778671e-08
657 9.9414585008617e-08
658 9.94244757634988e-08
659 9.96484246229556e-08
660 9.86716202078242e-08
661 9.85887353976977e-08
662 9.84767822842514e-08
663 9.85173329581812e-08
664 9.89549349128538e-08
665 9.80941337047625e-08
666 9.78090071157567e-08
667 9.76723342205332e-08
668 9.77138157054469e-08
669 9.78103358306726e-08
670 9.71266729266063e-08
671 9.72675167076886e-08
672 9.73546221416655e-08
673 9.67241433613708e-08
674 9.66344728681179e-08
675 9.68128688327852e-08
676 9.6673559824012e-08
677 9.63781729979019e-08
678 9.66187769790849e-08
679 9.60811874506362e-08
680 9.63350927918327e-08
681 9.58204040557575e-08
682 9.59869481675923e-08
683 9.61155208756281e-08
684 9.57554533442817e-08
685 9.55404146907313e-08
686 9.54742205294679e-08
687 9.57573718096683e-08
688 9.52489571659498e-08
689 9.4999457189715e-08
690 9.55509023015111e-08
691 9.46490601450023e-08
692 9.49867455801723e-08
693 9.49114067338996e-08
694 9.42921190016932e-08
695 9.40685822570231e-08
696 9.45489802006705e-08
697 9.4773120906666e-08
698 9.36919448690787e-08
699 9.42156290761886e-08
700 9.35543269520167e-08
701 9.38238500225452e-08
702 9.36575048626764e-08
703 9.38900868163728e-08
704 9.3297934711245e-08
705 9.29984338426948e-08
706 9.3361279596138e-08
707 9.32405725961871e-08
708 9.25241181448655e-08
709 9.24629546261713e-08
710 9.26400005596406e-08
711 9.22589435958798e-08
712 9.25024323805701e-08
713 9.19040203939403e-08
714 9.25223488934535e-08
715 9.20170464269177e-08
716 9.1523091327872e-08
717 9.19099178986471e-08
718 9.140305934352e-08
719 9.13095092869298e-08
720 9.13198547891625e-08
721 9.15440878657137e-08
722 9.10716266844247e-08
723 9.10710156176719e-08
724 9.09608246502103e-08
725 9.05052246480409e-08
726 9.02238284083978e-08
727 9.0296872201634e-08
728 9.00695766858917e-08
729 8.97503511509967e-08
730 8.96251250992464e-08
731 8.97621532658377e-08
732 8.95215066520905e-08
733 8.95581990789651e-08
734 8.94605278745075e-08
735 8.91691769311365e-08
736 8.89729605546563e-08
737 8.89455762376201e-08
738 8.87982949393518e-08
739 8.8949747123479e-08
740 8.86299460489681e-08
741 8.85276492113007e-08
742 8.78378259017154e-08
743 8.80267307934446e-08
744 8.74380461368673e-08
745 8.75272831990515e-08
746 8.74467716016625e-08
747 8.72199308332711e-08
748 8.71552359171801e-08
749 8.70826326604401e-08
750 8.68211103011163e-08
751 8.69276135517794e-08
752 8.66144915789846e-08
753 8.69143477189027e-08
754 8.64350440110684e-08
755 8.64879154960363e-08
756 8.62631921449974e-08
757 8.61608100422018e-08
758 8.62945910284907e-08
759 8.59794511143264e-08
760 8.58662545510924e-08
761 8.53574277925873e-08
762 8.57251549746252e-08
763 8.58176321116844e-08
764 8.5467540600348e-08
765 8.53540385037377e-08
766 8.53080948104434e-08
767 8.51700292514579e-08
768 8.51305159699223e-08
769 8.52152837182985e-08
770 8.49135446401306e-08
771 8.47753938160167e-08
772 8.46905621187943e-08
773 8.46557313138874e-08
774 8.47086241151374e-08
775 8.43750171952706e-08
776 8.45196126419978e-08
777 8.42190246430619e-08
778 8.43343030965116e-08
779 8.40549247982381e-08
780 8.4113160880861e-08
781 8.39146352404896e-08
782 8.3737425882191e-08
783 8.35636413398788e-08
784 8.35921554198649e-08
785 8.33923508025691e-08
786 8.33985538406523e-08
787 8.34407956062932e-08
788 8.34346494116289e-08
789 8.29608879371335e-08
790 8.2932224643173e-08
791 8.29718089789822e-08
792 8.30005149055069e-08
793 8.28089454785186e-08
794 8.24493255890957e-08
795 8.25481336619305e-08
796 8.2537354728629e-08
797 8.23332939603461e-08
798 8.23294001861541e-08
799 8.21751200419385e-08
800 8.21103824932834e-08
801 8.19145142827438e-08
802 8.19121837025705e-08
803 8.17393726038063e-08
804 8.17176371015194e-08
805 8.14952016980897e-08
806 8.15019589595067e-08
807 8.1298360044002e-08
808 8.12099898439556e-08
809 8.11066627193213e-08
810 8.09591895745143e-08
811 8.08467248702982e-08
812 8.07021720561352e-08
813 8.0611343378223e-08
814 8.03869255605605e-08
815 8.04480393412632e-08
816 8.01949013862213e-08
817 8.02028381485798e-08
818 8.01238613235e-08
819 8.00201647166432e-08
820 7.98418682279589e-08
821 7.99264014972323e-08
822 7.97571502175742e-08
823 7.963997461502e-08
824 7.95731551761492e-08
825 7.93959458178506e-08
826 7.93807615195874e-08
827 7.93917394048549e-08
828 7.92196175325444e-08
829 7.90879752798901e-08
830 7.90543595030613e-08
831 7.89107659215915e-08
832 7.88487639624691e-08
833 7.86752778481059e-08
834 7.87221452469566e-08
835 7.86007561259794e-08
836 7.84543843224128e-08
837 7.83270124316005e-08
838 7.83110394309006e-08
839 7.81446019004761e-08
840 7.81114266601435e-08
841 7.79710873644035e-08
842 7.7982839741253e-08
843 7.77976083554677e-08
844 7.77218289726989e-08
845 7.76150415049415e-08
846 7.75443638190154e-08
847 7.75127872998382e-08
848 7.73999744296816e-08
849 7.72791963754571e-08
850 7.72464403553386e-08
851 7.71877850525016e-08
852 7.70500605540292e-08
853 7.70412640349605e-08
854 7.67911245702635e-08
855 7.6767371126607e-08
856 7.67418555369659e-08
857 7.66036762911426e-08
858 7.65088046250639e-08
859 7.65049392725814e-08
860 7.63026548611379e-08
861 7.62622107686184e-08
862 7.62279768196095e-08
863 7.60525935561418e-08
864 7.60430509672005e-08
865 7.59298117714025e-08
866 7.58126859068398e-08
867 7.57397344841593e-08
868 7.55951461428594e-08
869 7.55391909024183e-08
870 7.55384732542552e-08
871 7.5416032530029e-08
872 7.52531050807193e-08
873 7.53949009890675e-08
874 7.50351105693881e-08
875 7.52568993789282e-08
876 7.49295168134267e-08
877 7.47766790709647e-08
878 7.47702841863429e-08
879 7.47054045291407e-08
880 7.48964907870686e-08
881 7.44853636547305e-08
882 7.44738315461291e-08
883 7.43404697800543e-08
884 7.45446087080381e-08
885 7.4426857565868e-08
886 7.43827825999688e-08
887 7.43406900483023e-08
888 7.41981480700815e-08
889 7.38533074695624e-08
890 7.4032485031239e-08
891 7.39254275572421e-08
892 7.38673051614569e-08
893 7.34991445483502e-08
894 7.3804677924727e-08
895 7.34079819153521e-08
896 7.35694740683357e-08
897 7.32791534119315e-08
898 7.34463299068011e-08
899 7.31357303607183e-08
900 7.32437541728359e-08
901 7.31690121824613e-08
902 7.29662446019574e-08
903 7.28129521121446e-08
904 7.30627647271831e-08
905 7.29189011394737e-08
906 7.2729349653855e-08
907 7.2521849858731e-08
908 7.25434219361887e-08
909 7.23455002571427e-08
910 7.23498843058223e-08
911 7.24999651424696e-08
912 7.21425266192455e-08
913 7.24430790910446e-08
914 7.22647470752236e-08
915 7.22693016541598e-08
916 7.2072516843491e-08
917 7.20859176794875e-08
918 7.17207058187341e-08
919 7.19162116524785e-08
920 7.15814536533799e-08
921 7.16920212084915e-08
922 7.13973662413991e-08
923 7.16231198794048e-08
924 7.12597625351918e-08
925 7.14249068778372e-08
926 7.11057808189253e-08
927 7.10602279241357e-08
928 7.09582934632635e-08
929 7.0873120705528e-08
930 7.10935736947249e-08
931 7.10731526964992e-08
932 7.0714420985496e-08
933 7.0551251951656e-08
934 7.04061946521506e-08
935 7.03609259744553e-08
936 7.06000307104659e-08
937 7.04504969917252e-08
938 7.01859761420565e-08
939 7.00636846318048e-08
940 7.00064077818752e-08
941 6.99727635833369e-08
942 7.00547033716248e-08
943 6.9931417101543e-08
944 6.98034909873968e-08
945 7.0104341887145e-08
946 6.96129944799395e-08
947 6.94950443858033e-08
948 6.94046136118232e-08
949 6.93834749654343e-08
950 6.93329127443576e-08
951 6.92485642161955e-08
952 6.91911807848555e-08
953 6.91064414581888e-08
954 6.90000163672266e-08
955 6.89287631416846e-08
956 6.88137973270386e-08
957 6.87472621052621e-08
958 6.86805918803657e-08
959 6.8616756720985e-08
960 6.85244359033277e-08
961 6.88911470092535e-08
962 6.88129659920378e-08
963 6.87607268901047e-08
964 6.82461447354399e-08
965 6.81441889582857e-08
966 6.80525360508e-08
967 6.79863845221007e-08
968 6.83756269381774e-08
969 6.82837040244522e-08
970 6.82138434626722e-08
971 6.76982807590321e-08
972 6.80641178973929e-08
973 6.75125306770497e-08
974 6.78770604167767e-08
975 6.73166979936468e-08
976 6.76694469348149e-08
977 6.76162841273253e-08
978 6.71165167887011e-08
979 6.69894575366925e-08
980 6.6937090537067e-08
981 6.67930137865369e-08
982 6.68178472551517e-08
983 6.71299176246976e-08
984 6.70166855343268e-08
985 6.69725039870173e-08
986 6.68290809358041e-08
987 6.67748025762194e-08
988 6.65591741721983e-08
989 6.65157173784792e-08
990 6.64176909026537e-08
991 6.59115926282539e-08
992 6.62324168843043e-08
993 6.61424550685297e-08
994 6.60855761225321e-08
995 6.56153886779975e-08
996 6.55497061075039e-08
997 6.58482832704976e-08
998 6.56706120594208e-08
999 6.57166268069886e-08
1000 6.52948628498962e-08
1001 6.55794565318502e-08
1002 6.55087859513515e-08
1003 6.50929337098205e-08
1004 6.53591243349183e-08
1005 6.51976250765074e-08
1006 6.48884110887593e-08
1007 6.50298304094576e-08
1008 6.50516227551634e-08
1009 6.49347100534214e-08
1010 6.49216005399467e-08
1011 6.47730971081728e-08
1012 6.477149128159e-08
1013 6.45821884859288e-08
1014 6.4250031073243e-08
1015 6.4242122732594e-08
1016 6.43735660332823e-08
1017 6.43217390461359e-08
1018 6.42476081225141e-08
1019 6.38431600918921e-08
1020 6.40329034240494e-08
1021 6.39231174659471e-08
1022 6.343771730144e-08
1023 6.33109848990898e-08
1024 6.32279792966983e-08
1025 6.3585964937829e-08
1026 6.30923793210059e-08
1027 6.35481995914233e-08
1028 6.29233198878865e-08
1029 6.34452987924305e-08
1030 6.28542622393979e-08
1031 6.28028971050298e-08
1032 6.33427532648057e-08
1033 6.25963991751632e-08
1034 6.32249737009261e-08
1035 6.25113898422569e-08
1036 6.27003799991144e-08
1037 6.26982128437703e-08
1038 6.2893803942643e-08
1039 6.25594438474764e-08
1040 6.27802521080412e-08
1041 6.27198915026383e-08
1042 6.25557987632419e-08
1043 6.23992093551351e-08
1044 6.23657854248449e-08
1045 6.22760651936005e-08
1046 6.20254212435611e-08
1047 6.15790298752472e-08
1048 6.14456823200271e-08
1049 6.18361823967462e-08
1050 6.12977260061598e-08
1051 6.06647105882985e-08
1052 6.04313754593022e-08
1053 6.02781682346176e-08
1054 6.02238259261867e-08
1055 6.06156120852575e-08
1056 6.05715939627771e-08
1057 5.98377525307114e-08
1058 5.97212803654656e-08
1059 5.95074354237113e-08
1060 5.93538445059494e-08
1061 5.91968998264747e-08
1062 5.95865401464835e-08
1063 5.89911408610533e-08
1064 5.90381361575965e-08
1065 5.88567026227338e-08
1066 5.87384647587896e-08
1067 5.8578034867196e-08
1068 5.89264637085307e-08
1069 5.84319970187153e-08
1070 5.83082382377142e-08
1071 5.86176902572788e-08
1072 5.80402641503497e-08
1073 5.78617544988447e-08
1074 5.79442698267485e-08
1075 5.80927270732445e-08
1076 5.79573580239412e-08
1077 5.74208556258782e-08
1078 5.7274789355688e-08
1079 5.71655220937828e-08
1080 5.73095277900393e-08
1081 5.70234242047718e-08
1082 5.68715883275672e-08
1083 5.67902667114595e-08
1084 5.67097728776389e-08
1085 5.67741196277893e-08
1086 5.64518281009896e-08
1087 5.64300037808607e-08
1088 5.63336080006138e-08
1089 5.63983704182647e-08
1090 5.66004239033191e-08
1091 5.62567308293183e-08
1092 5.59793669197006e-08
1093 5.59125012955519e-08
1094 5.58014079388158e-08
1095 5.57473924800433e-08
1096 5.58025128327699e-08
1097 5.5565955392467e-08
1098 5.54812054076592e-08
1099 5.54003669606118e-08
1100 5.54742776159856e-08
1101 5.52717764890076e-08
1102 5.49984022768513e-08
1103 5.49059748777836e-08
1104 5.47882663681776e-08
1105 5.47266445494188e-08
1106 5.49530838611645e-08
1107 5.46023386505112e-08
1108 5.46977680926375e-08
1109 5.43834346444783e-08
1110 5.42821823046324e-08
1111 5.42360005795217e-08
1112 5.44188374362875e-08
1113 5.41187503699803e-08
1114 5.40442002261443e-08
1115 5.39444258151889e-08
1116 5.38302700192617e-08
1117 5.37732631755716e-08
1118 5.39154854095614e-08
1119 5.38101261327029e-08
1120 5.37833493297057e-08
1121 5.30634309825473e-08
1122 5.28608303795863e-08
1123 5.29011039418492e-08
1124 5.28263726096156e-08
1125 5.30287174171917e-08
1126 5.26620773655395e-08
1127 5.24801535561892e-08
1128 5.25775298854114e-08
1129 5.27378318793126e-08
1130 5.24850740646343e-08
1131 5.23471115343455e-08
1132 5.23697991638983e-08
1133 5.24037382376719e-08
1134 5.22165599647906e-08
1135 5.22367500366272e-08
1136 5.20656122660057e-08
1137 5.20560128336456e-08
1138 5.22056531337967e-08
1139 5.19875733573372e-08
1140 5.19178300351086e-08
1141 5.19154603750849e-08
1142 5.17898079976931e-08
1143 5.20028713424381e-08
1144 5.14840188259313e-08
1145 5.14215905411675e-08
1146 5.15809794876532e-08
1147 5.11323818841447e-08
1148 5.12141831165991e-08
1149 5.13394944334777e-08
1150 5.12295486032599e-08
1151 5.09733339981722e-08
1152 5.06374497888373e-08
1153 5.10902324890594e-08
1154 5.05281896323595e-08
1155 5.04367854148313e-08
1156 5.05960215946288e-08
1157 5.03655108730072e-08
1158 5.02118275846897e-08
1159 5.05225372648965e-08
1160 5.03420771735819e-08
1161 4.99101133755175e-08
1162 4.97891790018912e-08
1163 4.9768342336165e-08
1164 4.96981336084446e-08
1165 4.98014500749377e-08
1166 4.95814944656559e-08
1167 4.95046492687834e-08
1168 4.94668448425273e-08
1169 4.92030913790131e-08
1170 4.91606506614062e-08
1171 4.90939306985183e-08
1172 4.91156235682411e-08
1173 4.90032512345806e-08
1174 4.92659744111279e-08
1175 4.89362221856027e-08
1176 4.84053543914342e-08
1177 4.85740621058994e-08
1178 4.87864326714771e-08
1179 4.88608407067659e-08
1180 4.87312767916137e-08
1181 4.86784621500647e-08
1182 4.85885323087132e-08
1183 4.84924598254111e-08
1184 4.8398465679611e-08
1185 4.84726889737885e-08
1186 4.81045780986733e-08
1187 4.83827840014328e-08
1188 4.82935753609581e-08
1189 4.81475588287594e-08
1190 4.8234102933975e-08
1191 4.80771866762097e-08
1192 4.793876939857e-08
1193 4.7849862738758e-08
1194 4.78730832753627e-08
1195 4.73926178301554e-08
1196 4.77776218588133e-08
1197 4.69367478217464e-08
1198 4.71016115000111e-08
1199 4.75050008219569e-08
1200 4.7408327930043e-08
1201 4.71592223050266e-08
1202 4.67485676836077e-08
1203 4.70187266898847e-08
1204 4.66456633319012e-08
1205 4.68194762959229e-08
1206 4.6326153579912e-08
1207 4.65027234497484e-08
1208 4.61296103537734e-08
1209 4.64072655859127e-08
1210 4.60691111925371e-08
1211 4.63708325071366e-08
1212 4.61097506843089e-08
1213 4.57225262096017e-08
1214 4.60606379704132e-08
1215 4.58967512884101e-08
1216 4.57707933776419e-08
1217 4.57120847840997e-08
1218 4.56549997807087e-08
1219 4.55256063958132e-08
1220 4.53301467473466e-08
1221 4.51482407015646e-08
1222 4.51111397126169e-08
1223 4.51487949248985e-08
1224 4.52257431504677e-08
1225 4.50449526567809e-08
1226 4.48524808405182e-08
1227 4.48033361521993e-08
1228 4.49507808752969e-08
1229 4.46340031601267e-08
1230 4.44225669582465e-08
1231 4.42646843623606e-08
1232 4.44799397314455e-08
1233 4.42768559594242e-08
1234 4.38005400837937e-08
1235 4.4183991576574e-08
1236 4.4172765001349e-08
1237 4.38139018399397e-08
1238 4.37268212749586e-08
1239 4.3724892151431e-08
1240 4.35672582455027e-08
1241 4.35550866484391e-08
1242 4.33725730886181e-08
1243 4.31244728815727e-08
1244 4.31879385587308e-08
1245 4.33165681101855e-08
1246 4.30392788075551e-08
1247 4.31664446409741e-08
1248 4.29334363616363e-08
1249 4.30893614122851e-08
1250 4.28496420568081e-08
1251 4.29322355444128e-08
1252 4.28386783823953e-08
1253 4.26195434499732e-08
1254 4.26717932100473e-08
1255 4.26471444825438e-08
1256 4.2202245253975e-08
1257 4.24153761002799e-08
1258 4.28580619882268e-08
1259 4.24918127350793e-08
1260 4.2388752063971e-08
1261 4.24105870422409e-08
1262 4.20530206213243e-08
1263 4.22037658154295e-08
1264 4.21356851632027e-08
1265 4.2275146938664e-08
1266 4.2036035097226e-08
1267 4.21609414047452e-08
1268 4.19729531131452e-08
1269 4.1898537972429e-08
1270 4.17636485394723e-08
1271 4.17922656481551e-08
1272 4.18798116186281e-08
1273 4.23832062779184e-08
1274 4.1948290174787e-08
1275 4.22935713118022e-08
1276 4.19202024204424e-08
1277 4.19246575233956e-08
1278 4.18098586862925e-08
1279 4.17699972388164e-08
1280 4.13655989461859e-08
1281 4.14294554218486e-08
1282 4.14234762047272e-08
1283 4.15719512147916e-08
1284 4.16455492313617e-08
1285 4.150354726562e-08
1286 4.1175020726314e-08
1287 4.16421173099479e-08
1288 4.14607761456409e-08
1289 4.11966674107589e-08
1290 4.12913259140169e-08
1291 4.15214529425612e-08
1292 4.1449084164924e-08
1293 4.09297093995065e-08
1294 4.09474125717679e-08
1295 4.06012361509056e-08
1296 4.07110896105678e-08
1297 4.068166603588e-08
1298 4.06912050721076e-08
1299 4.04615043692047e-08
1300 4.07632079202358e-08
1301 4.05254070301453e-08
1302 4.09602840534262e-08
1303 4.05340188081027e-08
1304 4.06740632286073e-08
1305 4.04702902301324e-08
1306 4.0942754964135e-08
1307 4.0388290045712e-08
1308 4.04107005635979e-08
1309 4.03314572849922e-08
1310 4.05556050964151e-08
1311 4.02518196551682e-08
1312 4.03434157192351e-08
1313 4.01258972715368e-08
1314 4.02322832826485e-08
1315 4.00739992301169e-08
1316 4.03699758066978e-08
1317 4.00307733627869e-08
1318 4.00748980666776e-08
1319 4.02040676306115e-08
1320 3.98376158727842e-08
1321 3.9881285829324e-08
1322 3.96931305601811e-08
1323 3.97631261250808e-08
1324 3.958671612736e-08
1325 3.98529103051715e-08
1326 3.95863324342827e-08
1327 4.00001916034398e-08
1328 3.95690769039447e-08
1329 3.94990244956261e-08
1330 3.9452437761156e-08
1331 3.95061867664026e-08
1332 3.92504446722342e-08
1333 3.95191470659029e-08
1334 3.90043410902763e-08
1335 3.92543917371313e-08
1336 3.92146830563433e-08
1337 3.9048476452308e-08
1338 3.90654868454021e-08
1339 3.92114820613187e-08
1340 3.89044316762011e-08
1341 3.91764771734415e-08
1342 3.90772321168242e-08
1343 3.89328498329178e-08
1344 3.90961645280186e-08
1345 3.90719314680155e-08
1346 3.87141803059876e-08
1347 3.88669398887487e-08
1348 3.8858530615471e-08
1349 3.89128977928976e-08
1350 3.87291123615796e-08
1351 3.86946688024636e-08
1352 3.87663696699292e-08
1353 3.89190013549978e-08
1354 3.88329972622614e-08
1355 3.86215468495266e-08
1356 3.86746883407341e-08
1357 3.87796212919511e-08
1358 3.84750791226907e-08
1359 3.86991843015494e-08
1360 3.85319687268293e-08
1361 3.85711658168475e-08
1362 3.84015734766763e-08
1363 3.81288245421274e-08
1364 3.83609339849045e-08
1365 3.84543206166654e-08
1366 3.80084088646981e-08
1367 3.83377383172956e-08
1368 3.83105280832297e-08
1369 3.81067160049042e-08
1370 3.81055258458218e-08
1371 3.82876308435698e-08
1372 3.80491300688846e-08
1373 3.83868865583281e-08
1374 3.79084603707724e-08
1375 3.81274922744979e-08
1376 3.81903397794758e-08
1377 3.81111817659985e-08
1378 3.78618345564519e-08
1379 3.78529527722549e-08
1380 3.81679896577225e-08
1381 3.78243214527174e-08
1382 3.790764679934e-08
1383 3.7658193008383e-08
1384 3.78456093130808e-08
1385 3.79578679599035e-08
1386 3.75964646082139e-08
1387 3.78252416055602e-08
1388 3.76901496679238e-08
1389 3.77089754977078e-08
1390 3.79234705860654e-08
1391 3.77307181054221e-08
1392 3.78393565370061e-08
1393 3.76629962772768e-08
1394 3.75055257961776e-08
1395 3.75192286128367e-08
1396 3.76749689223743e-08
1397 3.76306701355134e-08
1398 3.74995003937784e-08
1399 3.75671653785048e-08
1400 3.74120112667242e-08
1401 3.72558517369725e-08
1402 3.73410351528491e-08
1403 3.74213477982721e-08
1404 3.73190296443227e-08
1405 3.73504249751022e-08
1406 3.68795731731097e-08
1407 3.74186797102993e-08
1408 3.71649626629278e-08
1409 3.71946313748595e-08
1410 3.69160062518858e-08
1411 3.7414103815081e-08
1412 3.69661847798852e-08
1413 3.70952299988403e-08
1414 3.73134234621375e-08
1415 3.69613033512906e-08
1416 3.69579566950051e-08
1417 3.67928016942187e-08
1418 3.72748907295772e-08
1419 3.72670534432018e-08
1420 3.6964252103644e-08
1421 3.67560062386474e-08
1422 3.69938817357252e-08
1423 3.65746011254942e-08
1424 3.71101229745818e-08
1425 3.70013175654549e-08
1426 3.68869912392711e-08
1427 3.69594239657545e-08
1428 3.66851011790459e-08
1429 3.71938142507133e-08
1430 3.69399160149442e-08
1431 3.71644190977349e-08
1432 3.79072417899806e-08
1433 3.68960364482973e-08
1434 3.71082045091953e-08
1435 3.68736543521209e-08
1436 3.68407810924509e-08
1437 3.69355674934013e-08
1438 3.70467709842615e-08
1439 3.64312739975503e-08
1440 3.69142689748969e-08
1441 3.63073269227243e-08
1442 3.62535317321999e-08
1443 3.63665577651773e-08
1444 3.6196169617142e-08
1445 3.63374113021564e-08
1446 3.68407775397372e-08
1447 3.64106718109269e-08
1448 3.62164875866711e-08
1449 3.6207893572282e-08
1450 3.67032484405172e-08
1451 3.60769227825131e-08
1452 3.5988282576227e-08
1453 3.63134411429655e-08
1454 3.59596192822664e-08
1455 3.63708316797329e-08
1456 3.63522758561885e-08
1457 3.63052450325085e-08
1458 3.60091014783848e-08
1459 3.61752974242791e-08
1460 3.60750078698402e-08
1461 3.59987453180111e-08
1462 3.60358320961041e-08
1463 3.62526115793571e-08
1464 3.60359742046512e-08
1465 3.59154626039526e-08
1466 3.61843710550147e-08
1467 3.55876927926602e-08
1468 3.61992462671878e-08
1469 3.60296965595808e-08
1470 3.59613707701101e-08
1471 3.59331195909363e-08
1472 3.61843959240105e-08
1473 3.6078759535485e-08
1474 3.56908209653284e-08
1475 3.56079752350524e-08
1476 3.60723824144316e-08
1477 3.56475737817163e-08
1478 3.61434189244392e-08
1479 3.5379088103582e-08
1480 3.56661651323975e-08
1481 3.59236196345591e-08
1482 3.57188483235404e-08
1483 3.53768285776823e-08
1484 3.55865701351377e-08
1485 3.56243461396843e-08
1486 3.56132865420022e-08
1487 3.57028611119858e-08
1488 3.55456464262716e-08
1489 3.52667797187678e-08
1490 3.53726044011182e-08
1491 3.51994593472682e-08
1492 3.54556384252191e-08
1493 3.53005162878617e-08
1494 3.53040832123952e-08
1495 3.54013636183481e-08
1496 3.52450904017587e-08
1497 3.50900997148074e-08
1498 3.52230067335313e-08
1499 3.50164661711005e-08
1500 3.51833726597306e-08
1501 3.50219622191617e-08
1502 3.50814453042858e-08
1503 3.46155673014437e-08
1504 3.48473712108444e-08
1505 3.53341640391136e-08
1506 3.49796529519608e-08
1507 3.50425750639261e-08
1508 3.49404558619426e-08
1509 3.47499096164938e-08
1510 3.46768196379799e-08
1511 3.49269448918221e-08
1512 3.5398002751208e-08
1513 3.48107782599527e-08
1514 3.49913698016735e-08
1515 3.4856920905213e-08
1516 3.48204807210095e-08
1517 3.43215447173861e-08
1518 3.44268826779626e-08
1519 3.46642110571338e-08
1520 3.46852786492491e-08
1521 3.46746844570589e-08
1522 3.50683286853837e-08
1523 3.46772885961855e-08
1524 3.42402692865562e-08
1525 3.4411005600532e-08
1526 3.47984929760514e-08
1527 3.45530111189873e-08
1528 3.50800100079596e-08
1529 3.42733699199016e-08
1530 3.41970185502305e-08
1531 3.41712436124908e-08
1532 3.47679964818326e-08
1533 3.50081776900879e-08
1534 3.40295400746982e-08
1535 3.4390410519336e-08
1536 3.44664350393487e-08
1537 3.44782087324802e-08
1538 3.41405197445965e-08
1539 3.42290320531902e-08
1540 3.41816814852791e-08
1541 3.43681350045699e-08
1542 3.42050192614352e-08
1543 3.46058932620963e-08
1544 3.43757271537015e-08
1545 3.48226123492168e-08
1546 3.46764252867615e-08
1547 3.4404767035312e-08
1548 3.43438024685838e-08
1549 3.41886163823801e-08
1550 3.39881225386307e-08
1551 3.4608351739962e-08
1552 3.40520251995713e-08
1553 3.41696591021901e-08
1554 3.40402017684482e-08
1555 3.42363648542232e-08
1556 3.42505117600922e-08
1557 3.42780559492439e-08
1558 3.4144175486972e-08
1559 3.406736936995e-08
1560 3.43142190217804e-08
1561 3.39191181808474e-08
1562 3.43468933294844e-08
1563 3.39144357042187e-08
1564 3.43839516858679e-08
1565 3.38612657913018e-08
1566 3.38767307539456e-08
1567 3.439021156737e-08
1568 3.38786882991826e-08
1569 3.38087602358428e-08
1570 3.39761143663964e-08
1571 3.38303429714415e-08
1572 3.40810935028912e-08
1573 3.39399619520009e-08
1574 3.40505721396767e-08
1575 3.41899522027234e-08
1576 3.39039978314304e-08
1577 3.39065380217107e-08
1578 3.40734160886313e-08
1579 3.38656995779729e-08
1580 3.41494619249261e-08
1581 3.37756453916427e-08
1582 3.36657137722796e-08
1583 3.37285364082618e-08
1584 3.34876517626981e-08
1585 3.38725065773815e-08
1586 3.36193579641986e-08
1587 3.35889218661123e-08
1588 3.37977290598701e-08
1589 3.37669909811211e-08
1590 3.35403704809778e-08
1591 3.3906800922523e-08
1592 3.38541354949484e-08
1593 3.35714354093852e-08
1594 3.3657734377357e-08
1595 3.35945777862889e-08
1596 3.38991625881135e-08
1597 3.35969794207358e-08
1598 3.36591092775507e-08
1599 3.38024079837851e-08
1600 3.34826708581204e-08
1601 3.33083711723248e-08
1602 3.35855823152542e-08
1603 3.36247261145672e-08
1604 3.3785685360499e-08
1605 3.349846977585e-08
1606 3.35875931511964e-08
1607 3.33417524700508e-08
1608 3.34005747504307e-08
1609 3.33876819524903e-08
1610 3.32681224790576e-08
1611 3.33526237739079e-08
1612 3.31019727184412e-08
1613 3.30929985636885e-08
1614 3.33821930098566e-08
1615 3.34982992455934e-08
1616 3.32658771640126e-08
1617 3.36836585290712e-08
1618 3.34653158517995e-08
1619 3.33414575948154e-08
1620 3.29728173653621e-08
1621 3.35736451972934e-08
1622 3.34142029601026e-08
1623 3.33687282250139e-08
1624 3.30684706284501e-08
1625 3.33568088706215e-08
1626 3.31008145337819e-08
1627 3.32757714716081e-08
1628 3.28508882319056e-08
1629 3.34787095823685e-08
1630 3.34315508609961e-08
1631 3.30758034294831e-08
1632 3.29031628609755e-08
1633 3.31345582083031e-08
1634 3.3648664299335e-08
1635 3.29807434695795e-08
1636 3.28709788277592e-08
1637 3.28499680790628e-08
1638 3.34434986370979e-08
1639 3.28935847448975e-08
1640 3.28655644921128e-08
1641 3.29682414701438e-08
1642 3.30589315922225e-08
1643 3.32039391537364e-08
1644 3.28414913042252e-08
1645 3.29809708432549e-08
1646 3.27642943886985e-08
1647 3.35847083476892e-08
1648 3.26786775417531e-08
1649 3.26694795660387e-08
1650 3.31684546495126e-08
1651 3.29258362796736e-08
1652 3.30600933295955e-08
1653 3.2760240742391e-08
1654 3.30735616671518e-08
1655 3.29216298666779e-08
1656 3.31612746151677e-08
1657 3.29724691994215e-08
1658 3.29011058397555e-08
1659 3.29610116978074e-08
1660 3.27084883622319e-08
1661 3.27434754865408e-08
1662 3.2988314302429e-08
1663 3.29444240776411e-08
1664 3.28856621933937e-08
1665 3.29530216447438e-08
1666 3.27906057862037e-08
1667 3.25587947713757e-08
1668 3.29727889436526e-08
1669 3.25730020733772e-08
1670 3.27079092699023e-08
1671 3.34967857895663e-08
1672 3.28986438091761e-08
1673 3.3150829636952e-08
1674 3.30646550139591e-08
1675 3.31597469482858e-08
1676 3.23462892026782e-08
1677 3.31638112527344e-08
1678 3.2488333800984e-08
1679 3.25584998961403e-08
1680 3.20875166437418e-08
1681 3.28668292581824e-08
1682 3.25279252422206e-08
1683 3.27543716593937e-08
1684 3.2606916278155e-08
1685 3.25320996807932e-08
1686 3.22489874804432e-08
1687 3.28021450002325e-08
1688 3.29850244895624e-08
1689 3.27307745351391e-08
1690 3.30350609090146e-08
1691 3.28624913947806e-08
1692 3.29756346673094e-08
1693 3.24000382079248e-08
1694 3.27730838023399e-08
1695 3.24460742717747e-08
1696 3.24452074096371e-08
1697 3.26430118491317e-08
1698 3.29104850038675e-08
1699 3.19098987233701e-08
1700 3.26107816306376e-08
1701 3.26433138297944e-08
1702 3.23387538969655e-08
1703 3.26004858663964e-08
1704 3.23639710586576e-08
1705 3.24020064113029e-08
1706 3.20278168430832e-08
1707 3.24998801204401e-08
1708 3.23305755500769e-08
1709 3.23580628958098e-08
1710 3.24068594181881e-08
1711 3.25548192847691e-08
1712 3.21411981474284e-08
1713 3.19543467242056e-08
1714 3.22238236094563e-08
1715 3.15469286249481e-08
1716 3.16227932728452e-08
1717 3.18198303261852e-08
1718 3.19367607914955e-08
1719 3.14291064285044e-08
1720 3.13491170800262e-08
1721 3.2028040664045e-08
1722 3.17638750857441e-08
1723 3.18783364150477e-08
1724 3.10592795926823e-08
1725 3.172091567194e-08
1726 3.10731245178886e-08
1727 3.16370218911288e-08
1728 3.14663459732856e-08
1729 3.14371177978501e-08
1730 3.14202388551621e-08
1731 3.17230366420063e-08
1732 3.14329362538501e-08
1733 3.14983665816726e-08
1734 3.14366630504992e-08
1735 3.15184998100904e-08
1736 3.18458930337329e-08
1737 3.13317087830001e-08
1738 3.12860350959454e-08
1739 3.14474704055101e-08
1740 3.14423758140947e-08
1741 3.13861576728414e-08
1742 3.16553006030063e-08
1743 3.13089323356053e-08
1744 3.13865946566239e-08
1745 3.15042747445204e-08
1746 3.13943822050078e-08
1747 3.13065342538721e-08
1748 3.0979467879888e-08
1749 3.14103232312846e-08
1750 3.11848324940911e-08
1751 3.11238288475124e-08
1752 3.11841290567827e-08
1753 3.12127994561706e-08
1754 3.14781054555624e-08
1755 3.13837453802535e-08
1756 3.14041272986287e-08
1757 3.1177577852759e-08
1758 3.13248946781641e-08
1759 3.12217629527822e-08
1760 3.09518206620396e-08
1761 3.07637399998839e-08
1762 3.1430161584467e-08
1763 3.12964409943106e-08
1764 3.1351099494259e-08
1765 3.13699715093207e-08
1766 3.16729327209941e-08
1767 3.14017292168955e-08
1768 3.15644399506709e-08
1769 3.14960182379309e-08
1770 3.12401517987837e-08
1771 3.15586916599386e-08
1772 3.08051362196693e-08
1773 3.11757979432059e-08
1774 3.11400008001783e-08
1775 3.13403347718122e-08
1776 3.12816759162615e-08
1777 3.12333874319393e-08
1778 3.10982706253071e-08
1779 3.11152241749824e-08
1780 3.10527745739364e-08
1781 3.11676657815951e-08
1782 3.10460812613655e-08
1783 3.12434060845135e-08
1784 3.13303196719517e-08
1785 3.10428234229221e-08
1786 3.10651095958292e-08
1787 3.13420223108096e-08
1788 3.11336698644027e-08
1789 3.10109200540865e-08
1790 3.1095606090048e-08
1791 3.09892591587868e-08
1792 3.10046459617297e-08
1793 3.11478878245453e-08
1794 3.11949150955115e-08
1795 3.0912993054244e-08
1796 3.07598533311193e-08
1797 3.09388497043983e-08
1798 3.08450012198591e-08
1799 3.09571817069809e-08
1800 3.08582990271589e-08
1801 3.1043764892047e-08
1802 3.07383416497942e-08
1803 3.08060705833668e-08
1804 3.08970697915356e-08
1805 3.09403880294212e-08
1806 3.07319716341681e-08
1807 3.07417842293489e-08
1808 3.09037524459654e-08
1809 3.0931452954519e-08
1810 3.07944247879277e-08
1811 3.08500816004198e-08
1812 3.08041059327024e-08
1813 3.07593488457769e-08
1814 3.07310941138894e-08
1815 3.07886054429218e-08
1816 3.0739869316676e-08
1817 3.07249585773661e-08
1818 3.06658911597424e-08
1819 3.08541459048683e-08
1820 3.04369009995753e-08
1821 3.10157091121255e-08
1822 3.03560980796647e-08
1823 3.07330623172675e-08
1824 3.08788088432266e-08
1825 3.05312148896064e-08
1826 3.08142453775417e-08
1827 3.04873246648185e-08
1828 3.09021253031005e-08
1829 3.06955314499646e-08
1830 3.09595726832868e-08
1831 3.07497671769852e-08
1832 3.08901171308662e-08
1833 3.09168797230086e-08
1834 3.03774250198785e-08
1835 3.07790415376985e-08
1836 3.09088790118039e-08
1837 3.09904635287239e-08
1838 3.07997147785954e-08
1839 3.04707548082206e-08
1840 3.02922416040019e-08
1841 3.03432123871517e-08
1842 3.05748422135821e-08
1843 3.07539345101304e-08
1844 3.08298702123011e-08
1845 3.09370271622811e-08
1846 3.06150624851398e-08
1847 3.06525080873143e-08
1848 3.05878415929328e-08
1849 3.06688114903864e-08
1850 3.05039726811174e-08
1851 3.0373460191413e-08
1852 3.05300709158018e-08
1853 3.08010861260755e-08
1854 3.06440739450409e-08
1855 3.05859693128241e-08
1856 3.07248768649515e-08
1857 3.05004981271395e-08
1858 3.0581635002136e-08
1859 3.0778881665583e-08
1860 3.08711456398214e-08
1861 3.01564746507665e-08
1862 3.06612726319599e-08
1863 3.02653084816029e-08
1864 3.06037541975002e-08
1865 3.01403595415195e-08
1866 3.07669303367675e-08
1867 3.03334850570991e-08
1868 3.06872500743793e-08
1869 3.01310585371084e-08
1870 3.04328615641225e-08
1871 3.01443492389808e-08
1872 3.03097529297247e-08
1873 3.00162987798558e-08
1874 3.04459994993067e-08
1875 3.04187999233818e-08
1876 3.04102982795484e-08
1877 3.02601783630507e-08
1878 3.03728135975234e-08
1879 3.00111651085899e-08
1880 3.03395317757804e-08
1881 3.01031413130204e-08
1882 3.0107315751593e-08
1883 3.04056868571934e-08
1884 3.04909448800572e-08
1885 2.99191817987321e-08
1886 3.03986418259683e-08
1887 3.01666140956058e-08
1888 3.02632159332461e-08
1889 3.00774907202594e-08
1890 3.03704048576492e-08
1891 3.0059489120049e-08
1892 3.02631590898272e-08
1893 3.00270208697384e-08
1894 2.99657614277749e-08
1895 3.02579614697152e-08
1896 3.00221749682805e-08
1897 3.0267862882738e-08
1898 3.01892129073167e-08
1899 3.01476710262705e-08
1900 2.99866336206378e-08
1901 3.01708311667426e-08
1902 3.03634628551208e-08
1903 3.00090476912374e-08
1904 3.0010301799166e-08
1905 2.98344957627705e-08
1906 3.02858893519442e-08
1907 3.0005931961341e-08
1908 3.02591267598018e-08
1909 3.01538882752084e-08
1910 2.9940331103262e-08
1911 2.99463280839518e-08
1912 2.98536129150762e-08
1913 3.00082305670912e-08
1914 2.98783859875584e-08
1915 3.01210079101111e-08
1916 2.99733109443423e-08
1917 3.00492892790771e-08
1918 2.97265962956317e-08
1919 3.00159435084879e-08
1920 2.99480440446587e-08
1921 3.00156806076757e-08
1922 2.99607094689236e-08
1923 3.02330711576815e-08
1924 2.9965416814548e-08
1925 3.01073761477255e-08
1926 3.00345526227375e-08
1927 3.01714706552048e-08
1928 3.00213791604165e-08
1929 3.01342630848467e-08
1930 2.98860420855362e-08
1931 3.00810505393656e-08
1932 2.97251983027991e-08
1933 2.98880529214784e-08
1934 2.97520497127834e-08
1935 2.97511419944385e-08
1936 2.96964799417765e-08
1937 2.95487172508047e-08
1938 2.98517264241127e-08
1939 2.9761300979203e-08
1940 2.9538039569843e-08
1941 2.96992439530186e-08
1942 2.97274915794787e-08
1943 2.98114350982814e-08
1944 2.96225834972574e-08
1945 2.96629796281422e-08
1946 2.95788353810167e-08
1947 2.95538598038547e-08
1948 2.97431803630843e-08
1949 2.97761033607458e-08
1950 2.97380555736027e-08
1951 2.96271398525505e-08
1952 2.94663031752407e-08
1953 2.9668656864601e-08
1954 2.9785491406642e-08
1955 2.97487492417758e-08
1956 2.96708986269323e-08
1957 2.97082340949828e-08
1958 2.95481896728234e-08
1959 2.97417557248991e-08
1960 2.95153181895103e-08
1961 2.94880475593118e-08
1962 2.93167392584337e-08
1963 2.94677686696332e-08
1964 2.93137034645952e-08
1965 2.92559985126672e-08
1966 2.93909803161796e-08
1967 2.94474791218136e-08
1968 2.94180466653415e-08
1969 2.92183344186014e-08
1970 2.93383006777503e-08
1971 2.92661415102202e-08
1972 2.92240063259896e-08
1973 2.92570465632025e-08
1974 2.91628126092291e-08
1975 2.92971567006362e-08
1976 2.93262552020224e-08
1977 2.93786595051415e-08
1978 2.91907742422381e-08
1979 2.90964798921323e-08
1980 2.90852852913304e-08
1981 2.90969150995579e-08
1982 2.90682269366016e-08
1983 2.93589970112862e-08
1984 2.91467774360399e-08
1985 2.90678983105863e-08
1986 2.90410717695977e-08
1987 2.89813257836613e-08
1988 2.90445267836503e-08
1989 2.91457062928657e-08
1990 2.90069408492855e-08
1991 2.88578760887503e-08
1992 2.89867134739552e-08
1993 2.88750925392378e-08
1994 2.88209065502087e-08
1995 2.91131669882816e-08
1996 2.90205122155385e-08
1997 2.88047505847544e-08
1998 2.89707138279027e-08
1999 2.8894126202772e-08
2000 2.87861769976416e-08
2001 2.87738188831099e-08
2002 2.89751707072128e-08
2003 2.87645445240514e-08
2004 2.89178156975822e-08
2005 2.87831412038031e-08
2006 2.87312715840926e-08
2007 2.88527459701982e-08
2008 2.87053314451668e-08
2009 2.86139716365597e-08
2010 2.87487402772513e-08
2011 2.87679906563199e-08
2012 2.9005873258825e-08
2013 2.88149095695189e-08
2014 2.86544050709381e-08
2015 2.86268164728654e-08
2016 2.86220736001042e-08
2017 2.86168493346395e-08
2018 2.85523498178009e-08
2019 2.87146839639263e-08
2020 2.87106516339009e-08
2021 2.85136074751335e-08
2022 2.85781052156153e-08
2023 2.86922521297583e-08
2024 2.85950463307927e-08
2025 2.85074470696145e-08
2026 2.85965437996083e-08
2027 2.86862213982886e-08
2028 2.85636652108678e-08
2029 2.86061876408894e-08
2030 2.83877064077842e-08
2031 2.8598394763435e-08
2032 2.85713213088457e-08
2033 2.85466654759148e-08
2034 2.85507493202886e-08
2035 2.86774408664314e-08
2036 2.85552808065859e-08
2037 2.85793522181166e-08
2038 2.8465864332361e-08
2039 2.83967800385199e-08
2040 2.84547532203305e-08
2041 2.86264807414227e-08
2042 2.83652568100479e-08
2043 2.84438019804156e-08
2044 2.85996843985004e-08
2045 2.83123746669389e-08
2046 2.85517050002682e-08
2047 2.85792154386399e-08
2048 2.83500725117847e-08
2049 2.83972596548665e-08
2050 2.82679728513813e-08
2051 2.83742593865099e-08
2052 2.82865748602035e-08
2053 2.83840009274172e-08
2054 2.83969132652828e-08
2055 2.84361902913588e-08
2056 2.81728667061998e-08
2057 2.85864061311258e-08
2058 2.85216259499066e-08
2059 2.82889232039452e-08
2060 2.85232069074937e-08
2061 2.83326642147586e-08
2062 2.8401014873225e-08
2063 2.83949841417552e-08
2064 2.84252941185059e-08
2065 2.82419225783315e-08
2066 2.83908825338131e-08
2067 2.84072783074407e-08
2068 2.83037433490563e-08
2069 2.83300032322131e-08
2070 2.83187606697766e-08
2071 2.83389276489743e-08
2072 2.82705983067899e-08
2073 2.83499339559512e-08
2074 2.78583094370788e-08
2075 2.83310548354621e-08
2076 2.82580998600679e-08
2077 2.82291594544404e-08
2078 2.81928631551409e-08
2079 2.81933196788486e-08
2080 2.83497172404168e-08
2081 2.81480048158755e-08
2082 2.82268786122586e-08
2083 2.82859087263887e-08
2084 2.81756840081471e-08
2085 2.8317472811068e-08
2086 2.80492429283186e-08
2087 2.78227698657929e-08
2088 2.82368333159866e-08
2089 2.81997678541757e-08
2090 2.81852354788725e-08
2091 2.79176379791579e-08
2092 2.80119323292638e-08
2093 2.78343907922363e-08
2094 2.80962897392101e-08
2095 2.81729803930375e-08
2096 2.82125416362078e-08
2097 2.7886335018934e-08
2098 2.78845515566672e-08
2099 2.79500831368296e-08
2100 2.80635745752988e-08
2101 2.79567995420393e-08
2102 2.82786860594797e-08
2103 2.78154068666936e-08
2104 2.78989258362117e-08
2105 2.79318257412342e-08
2106 2.76855391945219e-08
2107 2.82307208721022e-08
2108 2.80134333507931e-08
2109 2.82192580414176e-08
2110 2.8215248804031e-08
2111 2.78881149284871e-08
2112 2.79890510768155e-08
2113 2.7688358272826e-08
2114 2.78845817547335e-08
2115 2.81833969495437e-08
2116 2.78142895382416e-08
2117 2.78468910153151e-08
2118 2.79073333331326e-08
2119 2.79522502921736e-08
2120 2.80476033509558e-08
2121 2.79870420172301e-08
2122 2.78031428990744e-08
2123 2.81916978650543e-08
2124 2.78015370724916e-08
2125 2.79174958706108e-08
2126 2.76756093597896e-08
2127 2.76355471839906e-08
2128 2.77412759430717e-08
2129 2.79748206821751e-08
2130 2.76481024741315e-08
2131 2.75442229025202e-08
2132 2.74143161504981e-08
2133 2.75629119528276e-08
2134 2.76483067551681e-08
2135 2.75661626858437e-08
2136 2.74499463159827e-08
2137 2.74764744290223e-08
2138 2.75053508858036e-08
2139 2.75153499984526e-08
2140 2.75571210295311e-08
2141 2.74122857746306e-08
2142 2.74529234900456e-08
2143 2.73510281090239e-08
2144 2.71890954195442e-08
2145 2.73220930324669e-08
2146 2.71529323470077e-08
2147 2.72416222912852e-08
2148 2.71624500669532e-08
2149 2.70681006497853e-08
2150 2.72873545981156e-08
2151 2.73664806371698e-08
2152 2.70299711502275e-08
2153 2.71486619851657e-08
2154 2.69651820872241e-08
2155 2.69162772070786e-08
2156 2.71134261708994e-08
2157 2.67167425960224e-08
2158 2.69269158081897e-08
2159 2.68451305629469e-08
2160 2.7226482401943e-08
2161 2.68874007502973e-08
2162 2.66792739012089e-08
2163 2.66654893721352e-08
2164 2.66208513011179e-08
2165 2.63951527301742e-08
2166 2.63537369704636e-08
2167 2.66152344607917e-08
2168 2.6238545558499e-08
2169 2.64888235790295e-08
2170 2.60746890745622e-08
2171 2.62985704324592e-08
2172 2.61163943804377e-08
2173 2.60846189092945e-08
2174 2.60639545501817e-08
2175 2.59596202312196e-08
2176 2.57905039546813e-08
2177 2.5990431140599e-08
2178 2.57456793661959e-08
2179 2.57610270892883e-08
2180 2.58141419351432e-08
2181 2.63938932931751e-08
2182 2.61322909977935e-08
2183 2.60837325072316e-08
2184 2.6055689161808e-08
2185 2.59108201561276e-08
2186 2.58806736042061e-08
2187 2.59914649802795e-08
2188 2.57451269192188e-08
2189 2.57637182699e-08
2190 2.57221017818665e-08
2191 2.56596610626048e-08
2192 2.55693439754623e-08
2193 2.55185419462123e-08
2194 2.55119019243466e-08
2195 2.53452991927361e-08
2196 2.50779788046884e-08
2197 2.49308715893903e-08
2198 2.50503742194041e-08
2199 2.47305305123291e-08
2200 2.48820146708795e-08
2201 2.49517952966016e-08
2202 2.49063472068656e-08
2203 2.48198457342141e-08
2204 2.48133318336841e-08
2205 2.46357245714535e-08
2206 2.47359750460419e-08
2207 2.44688660444581e-08
2208 2.43737030558577e-08
2209 2.42661073457384e-08
2210 2.42184050591732e-08
2211 2.42198243682878e-08
2212 2.40997461986581e-08
2213 2.40594371092584e-08
2214 2.4074848781197e-08
2215 2.40074324864281e-08
2216 2.40276367691195e-08
2217 2.39776838384387e-08
2218 2.37429009786183e-08
2219 2.36668373787552e-08
2220 2.37208066522498e-08
2221 2.35903030443296e-08
2222 2.35896600031538e-08
2223 2.34928343445517e-08
2224 2.33782948555472e-08
2225 2.33436647789631e-08
2226 2.33162591456448e-08
2227 2.32513670539447e-08
2228 2.31821815077637e-08
2229 2.30854109162237e-08
2230 2.31059562594282e-08
2231 2.30209487028787e-08
2232 2.29721077715794e-08
2233 2.2870718652257e-08
2234 2.28331842322405e-08
2235 2.26899867783459e-08
2236 2.26336442921138e-08
2237 2.24784049152049e-08
2238 2.24090719314063e-08
2239 2.23827125722664e-08
2240 2.23210587790845e-08
2241 2.22549143558126e-08
2242 2.2178159753139e-08
2243 2.21448956949644e-08
2244 2.19791225219979e-08
2245 2.1925686155555e-08
2246 2.19637055209887e-08
2247 2.18642899341148e-08
2248 2.17808047153767e-08
2249 2.16972875222154e-08
2250 2.16686792953169e-08
2251 2.15753370724769e-08
2252 2.14403268472552e-08
2253 2.14503472761862e-08
2254 2.13877200394563e-08
2255 2.13952127126049e-08
2256 2.13958042394324e-08
2257 2.1373820047188e-08
2258 2.12393338472339e-08
2259 2.11377937375801e-08
2260 2.10741539774517e-08
2261 2.10330899363953e-08
2262 2.10054729166131e-08
2263 2.09555448549281e-08
2264 2.09019752617223e-08
2265 2.08535464452098e-08
2266 2.08544079782769e-08
2267 2.08113366539919e-08
2268 2.07730526113892e-08
2269 2.07521644313147e-08
2270 2.06718286932528e-08
2271 2.06926813461905e-08
2272 2.07720560752023e-08
2273 2.06448866890696e-08
2274 2.0589999039089e-08
2275 2.05071497560994e-08
2276 2.0484744567284e-08
2277 2.05551415888294e-08
2278 2.03525711839347e-08
2279 2.03623482519788e-08
2280 2.03283043731517e-08
2281 2.03612753324478e-08
2282 2.03146903743345e-08
2283 2.03051211400407e-08
2284 2.04900150180265e-08
2285 2.03707770651818e-08
2286 2.04133954184726e-08
2287 2.02307539609592e-08
2288 2.03737240411783e-08
2289 2.03446841595678e-08
2290 2.0195892957986e-08
2291 2.03049488334273e-08
2292 2.02647161273717e-08
2293 2.01650127706898e-08
2294 2.03668957254877e-08
2295 2.02474623733906e-08
2296 2.02372074653567e-08
2297 2.02040570940198e-08
2298 2.01697964996583e-08
2299 2.01849807979215e-08
2300 2.01445438108294e-08
2301 2.07594705869951e-08
2302 2.06542480896132e-08
2303 2.0216200269374e-08
2304 2.03956354027923e-08
2305 2.03458228043019e-08
2306 2.03518197849917e-08
2307 2.04886188015507e-08
2308 2.04244976487189e-08
2309 2.02780494618082e-08
2310 2.02692866935195e-08
2311 2.04028189898509e-08
2312 2.04182182272916e-08
2313 2.02654799608126e-08
2314 2.02165129081777e-08
2315 2.04111980650623e-08
2316 2.03669561216202e-08
2317 2.04019929839205e-08
2318 2.03543759624836e-08
2319 2.05059471625191e-08
2320 2.04978665152566e-08
2321 2.04458689978537e-08
2322 2.08303916338082e-08
2323 2.01405505606544e-08
2324 2.04121821667513e-08
2325 2.07061869872405e-08
2326 2.03390264630343e-08
2327 2.02920507064164e-08
2328 2.04708285878041e-08
2329 2.05339816261585e-08
2330 2.03929140241144e-08
2331 2.01624210660611e-08
2332 2.02285672656899e-08
2333 2.04021013416877e-08
2334 2.03169623347321e-08
2335 2.02488354972274e-08
2336 2.01014103140551e-08
2337 2.00893026658377e-08
2338 1.99321110727624e-08
2339 2.00293985841427e-08
2340 2.05033838795998e-08
2341 2.03881800331374e-08
2342 2.04602237374729e-08
2343 2.01701073621052e-08
2344 2.03895265116216e-08
2345 1.96668938912126e-08
2346 1.96650908890206e-08
2347 1.97385663369687e-08
2348 2.00907006586704e-08
2349 1.95906650901634e-08
2350 1.97024458969963e-08
2351 1.96868708002285e-08
2352 1.99758840579989e-08
2353 1.98174952004138e-08
2354 1.99614582641061e-08
2355 1.97517380229328e-08
2356 1.98422842601076e-08
2357 1.9732258493832e-08
2358 1.95400495783815e-08
2359 1.96007299280154e-08
2360 1.95794029878016e-08
2361 1.95745961661942e-08
2362 1.95579481498953e-08
2363 1.95679668024695e-08
2364 1.95148359694031e-08
2365 1.96085423453951e-08
2366 1.95003480030209e-08
2367 1.95598168772904e-08
2368 1.95464622265717e-08
2369 1.95369302957715e-08
2370 1.95162517258041e-08
2371 1.95044140838263e-08
2372 1.9558870079095e-08
2373 1.94702032274563e-08
2374 1.94665261687987e-08
2375 1.94459204294617e-08
2376 1.94203835235385e-08
2377 1.94218010562963e-08
2378 1.94232701034025e-08
2379 1.94065901126805e-08
2380 1.94348945825595e-08
2381 1.94145002296864e-08
2382 1.9349986501993e-08
2383 1.93751183985569e-08
2384 1.93475138132726e-08
2385 1.93430764738878e-08
2386 1.93924218905295e-08
2387 1.93285210059457e-08
2388 1.93749816190802e-08
2389 1.92928943931747e-08
2390 1.93496259015546e-08
2391 1.92522193742661e-08
2392 1.93388185465437e-08
2393 1.92244726804347e-08
2394 1.92816465016676e-08
2395 1.928424175901e-08
2396 1.92860802883388e-08
2397 1.9216958691004e-08
2398 1.92629752149287e-08
2399 1.91909457214479e-08
2400 1.92483415872857e-08
2401 1.92441831359247e-08
2402 1.91465847620975e-08
2403 1.92264479892401e-08
2404 1.91201827703935e-08
2405 1.92102351803669e-08
2406 1.92334361770463e-08
2407 1.92989784153497e-08
2408 1.93152196459323e-08
2409 1.93025293526716e-08
2410 1.92841280721723e-08
2411 1.91377402813941e-08
2412 1.91253963777172e-08
2413 1.91835738405643e-08
2414 1.91073343813741e-08
2415 1.90724449566915e-08
2416 1.90308018233054e-08
2417 1.9088853164817e-08
2418 1.90354842999341e-08
2419 1.90732869498333e-08
2420 1.90998541427234e-08
2421 1.90567899238658e-08
2422 1.90763618235223e-08
2423 1.90780955477976e-08
2424 1.90826323631654e-08
2425 1.90359852325628e-08
2426 1.90480822226391e-08
2427 1.90633677732421e-08
2428 1.90645117470467e-08
2429 1.90219662243862e-08
2430 1.90563316238013e-08
2431 1.91479117006566e-08
2432 1.90624973583908e-08
2433 1.90852293968646e-08
2434 1.90010958078801e-08
2435 1.90129352262147e-08
2436 1.89815203555099e-08
2437 1.89853075482915e-08
2438 1.89554096863276e-08
2439 1.9074120061191e-08
2440 1.89312903131622e-08
2441 1.89172748576993e-08
2442 1.88028668191009e-08
2443 1.88511748433484e-08
2444 1.88835382886055e-08
2445 1.8847794436283e-08
2446 1.87772890569704e-08
2447 1.88006694656906e-08
2448 1.87649789040734e-08
2449 1.87404314289097e-08
2450 1.87522388728212e-08
2451 1.87063484702321e-08
2452 1.87258564210424e-08
2453 1.86755908515579e-08
2454 1.86479898189873e-08
2455 1.86280377789672e-08
2456 1.87061353074114e-08
2457 1.8539653368066e-08
2458 1.86214723640887e-08
2459 1.86022965920074e-08
2460 1.85537736285823e-08
2461 1.85640391947572e-08
2462 1.85159407806168e-08
2463 1.85371771266318e-08
2464 1.85357897919403e-08
2465 1.86279862646188e-08
2466 1.85832860211121e-08
2467 1.86636412990993e-08
2468 1.85794171159159e-08
2469 1.86488424702702e-08
2470 1.87033055709662e-08
2471 1.87383140115571e-08
2472 1.86942443747284e-08
2473 1.86781878852571e-08
2474 1.85918480610781e-08
2475 1.85779267525277e-08
2476 1.85321891166268e-08
2477 1.85614013048507e-08
2478 1.85459807511279e-08
2479 1.85824635678955e-08
2480 1.84781043799376e-08
2481 1.85435737876105e-08
2482 1.8501470577803e-08
2483 1.8461154382976e-08
2484 1.8556608694098e-08
2485 1.8500873721905e-08
2486 1.83982376000813e-08
2487 1.8428691461736e-08
2488 1.83638029227495e-08
2489 1.83355624017167e-08
2490 1.85151520781801e-08
2491 1.84047177498314e-08
2492 1.83291426480992e-08
2493 1.84167472383479e-08
2494 1.83750614723976e-08
2495 1.82597457154543e-08
2496 1.8331588691467e-08
2497 1.83431154709979e-08
2498 1.8799768852773e-08
2499 1.83580866064403e-08
2500 1.82964097206195e-08
2501 1.83757791205608e-08
2502 1.8284293190618e-08
2503 1.83662844932542e-08
2504 1.85949406983354e-08
2505 1.83411383858356e-08
2506 1.82269719317674e-08
2507 1.82901267464786e-08
2508 1.82250694535924e-08
2509 1.83049806423696e-08
2510 1.82697394990328e-08
2511 1.81389552267319e-08
2512 1.82184951569297e-08
2513 1.81532708865006e-08
2514 1.80949033534716e-08
2515 1.81493504669561e-08
2516 1.80063945975917e-08
2517 1.80215700140707e-08
2518 1.79653039111827e-08
2519 1.79242167774873e-08
2520 1.79760970553389e-08
2521 1.79889667606403e-08
2522 1.79010868350815e-08
2523 1.79275261302791e-08
2524 1.79170385194993e-08
2525 1.7924154604998e-08
2526 1.78779586690325e-08
2527 1.80303185715047e-08
2528 1.79027441760127e-08
2529 1.78856431887198e-08
2530 1.78795627192585e-08
2531 1.79008807776881e-08
2532 1.78021277719154e-08
2533 1.78758234881116e-08
2534 1.78571433195884e-08
2535 1.77970207460021e-08
2536 1.78574897091721e-08
2537 1.7775176885948e-08
2538 1.77482011309849e-08
2539 1.77476184859415e-08
2540 1.77852612637253e-08
2541 1.77020105240899e-08
2542 1.77580741222982e-08
2543 1.77730239414586e-08
2544 1.76906560511725e-08
2545 1.77920682631338e-08
2546 1.76850249999916e-08
2547 1.77978307647209e-08
2548 1.76422432218715e-08
2549 1.77133347989411e-08
2550 1.76511090188569e-08
2551 1.77262453604499e-08
2552 1.77346102248066e-08
2553 1.76485226432987e-08
2554 1.76927947848071e-08
2555 1.76071850432891e-08
2556 1.76453731626225e-08
2557 1.75886682995952e-08
2558 1.75774257371586e-08
2559 1.75673608993065e-08
2560 1.76611099078627e-08
2561 1.75790422218824e-08
2562 1.7625170656288e-08
2563 1.75391683399084e-08
2564 1.76627032999477e-08
2565 1.75771805999148e-08
2566 1.75877890029597e-08
2567 1.75374985644794e-08
2568 1.75755552334067e-08
2569 1.75247265588041e-08
2570 1.75883894115714e-08
2571 1.75214029951576e-08
2572 1.76031473841931e-08
2573 1.74905867567077e-08
2574 1.76959105147034e-08
2575 1.74309455758248e-08
2576 1.73504446365769e-08
2577 1.73881851139868e-08
2578 1.72969745193541e-08
2579 1.73223089205976e-08
2580 1.73126952773828e-08
2581 1.72683360943893e-08
2582 1.7371380778286e-08
2583 1.72737770753884e-08
2584 1.73349956611446e-08
2585 1.71877267973741e-08
2586 1.73292153959892e-08
2587 1.71709082508187e-08
2588 1.73124146130021e-08
2589 1.71657763559097e-08
2590 1.73808913928042e-08
2591 1.72540612908278e-08
2592 1.74398344654492e-08
2593 1.72205147919158e-08
2594 1.74292349441885e-08
2595 1.72055969471785e-08
2596 1.72795910913237e-08
2597 1.7194475177007e-08
2598 1.72365997030965e-08
2599 1.72470375758849e-08
2600 1.71420104777553e-08
2601 1.72463465730743e-08
2602 1.72122849306788e-08
2603 1.72462453207345e-08
2604 1.72273555421043e-08
2605 1.72631118289246e-08
2606 1.71601346465877e-08
2607 1.72610601367751e-08
2608 1.71507164026252e-08
2609 1.72386425134619e-08
2610 1.715861586149e-08
2611 1.70314429226437e-08
2612 1.71631864276378e-08
2613 1.71767169376835e-08
2614 1.70717804337528e-08
2615 1.71917928781795e-08
2616 1.70714962166585e-08
2617 1.71679062077601e-08
2618 1.71719012342919e-08
2619 1.70233924734475e-08
2620 1.72260836706073e-08
2621 1.72020460098565e-08
2622 1.72210867788181e-08
2623 1.72146421562047e-08
2624 1.72629501804522e-08
2625 1.72579426305219e-08
2626 1.7312327571517e-08
2627 1.72456466884796e-08
2628 1.74297234423193e-08
2629 1.72697642852881e-08
2630 1.73479755005701e-08
2631 1.73478262865956e-08
2632 1.73311445195168e-08
2633 1.72911160944977e-08
2634 1.73571930162097e-08
2635 1.73119527602239e-08
2636 1.72705458822975e-08
2637 1.73323417840265e-08
2638 1.73882970244676e-08
2639 1.73476131237749e-08
2640 1.73489862476117e-08
2641 1.74113132800358e-08
2642 1.73023142480133e-08
2643 1.739229738007e-08
2644 1.74087979587512e-08
2645 1.73461103258887e-08
2646 1.73674408188162e-08
2647 1.735424781657e-08
2648 1.73908158984659e-08
2649 1.72426553035621e-08
2650 1.73477907594588e-08
2651 1.73176903928152e-08
2652 1.73270588987862e-08
2653 1.73259504521184e-08
2654 1.73170011663615e-08
2655 1.72905441075955e-08
2656 1.72471228410132e-08
2657 1.72931109432284e-08
2658 1.73161609495764e-08
2659 1.73373813083799e-08
2660 1.73185039642476e-08
2661 1.73412253445804e-08
2662 1.71807386095679e-08
2663 1.73223764221575e-08
2664 1.73478476028777e-08
2665 1.73008274373387e-08
2666 1.72366068085239e-08
2667 1.72892757888121e-08
2668 1.73032876915613e-08
2669 1.72180687485479e-08
2670 1.72953580346302e-08
2671 1.72855507685199e-08
2672 1.72816072563364e-08
2673 1.72861653879863e-08
2674 1.71850800256834e-08
2675 1.72592180547326e-08
2676 1.72505885132068e-08
2677 1.72963314781782e-08
2678 1.72732228520545e-08
2679 1.72606213766358e-08
2680 1.7324284229403e-08
2681 1.72119349883815e-08
2682 1.72792038455327e-08
2683 1.72819483168496e-08
2684 1.72374612361637e-08
2685 1.72761556171963e-08
2686 1.72101852768947e-08
2687 1.71832965634167e-08
2688 1.72770793227528e-08
2689 1.72461174230421e-08
2690 1.72335905546106e-08
2691 1.72493077599256e-08
2692 1.72488299199358e-08
2693 1.71672684956548e-08
2694 1.71794631853572e-08
2695 1.71636305168477e-08
2696 1.72356866556811e-08
2697 1.72382517149572e-08
2698 1.72206959803134e-08
2699 1.72575145285236e-08
2700 1.71872134302475e-08
2701 1.72497287564966e-08
2702 1.71543597105028e-08
2703 1.71627814182784e-08
2704 1.72580616464302e-08
2705 1.71675598181764e-08
2706 1.72079381854928e-08
2707 1.7166280841252e-08
2708 1.7176468247726e-08
2709 1.71913772106791e-08
2710 1.71068244014805e-08
2711 1.71469878296193e-08
2712 1.72052772029474e-08
2713 1.7139083041684e-08
2714 1.71732441600625e-08
2715 1.71191665288006e-08
2716 1.71497216427952e-08
2717 1.71272951376977e-08
2718 1.71609855215138e-08
2719 1.71009624239105e-08
2720 1.71093965661839e-08
2721 1.71196923304251e-08
2722 1.71720060393454e-08
2723 1.70575606972534e-08
2724 1.70944733923761e-08
2725 1.71327165787716e-08
2726 1.70566192281285e-08
2727 1.7145630692994e-08
2728 1.71299667783842e-08
2729 1.7142102848311e-08
2730 1.71526579606507e-08
2731 1.71505618595802e-08
2732 1.7036560606698e-08
2733 1.71122973569027e-08
2734 1.70494924844888e-08
2735 1.71101746104796e-08
2736 1.70711178526517e-08
2737 1.70092562257196e-08
2738 1.71035789975349e-08
2739 1.71567524631655e-08
2740 1.69830762786205e-08
2741 1.70846909952616e-08
2742 1.70807687993602e-08
2743 1.69648934900124e-08
2744 1.70710059421708e-08
2745 1.70581895275745e-08
2746 1.70175713520848e-08
2747 1.70610405803018e-08
2748 1.70601168747453e-08
2749 1.70264655707797e-08
2750 1.7061601909063e-08
2751 1.70793157394655e-08
2752 1.70594702808557e-08
2753 1.69922973469738e-08
2754 1.69687268680718e-08
2755 1.7059633705685e-08
2756 1.70322849157856e-08
2757 1.70449609981915e-08
2758 1.69692579987668e-08
2759 1.70268137367202e-08
2760 1.70236571506166e-08
2761 1.69737717214957e-08
2762 1.72104961393416e-08
2763 1.70102278929107e-08
2764 1.69329954502473e-08
2765 1.68738036876448e-08
2766 1.72253962205104e-08
2767 1.68571396841344e-08
2768 1.68420282165016e-08
2769 1.67862417299602e-08
2770 1.68628417895889e-08
2771 1.6869520891305e-08
2772 1.68626836938301e-08
2773 1.72197083259107e-08
2774 1.68942726475052e-08
2775 1.68317413340446e-08
2776 1.68520859489263e-08
2777 1.68352176643793e-08
2778 1.69000546890175e-08
2779 1.67868350331446e-08
2780 1.67769709236154e-08
2781 1.68378786469248e-08
2782 1.67918372540043e-08
2783 1.67889915303476e-08
2784 1.67967293407401e-08
2785 1.68208860173991e-08
2786 1.68618843332524e-08
2787 1.67798877015457e-08
2788 1.67872080680809e-08
2789 1.67555551655596e-08
2790 1.68611578033051e-08
2791 1.6866401608695e-08
2792 1.6827257809382e-08
2793 1.70997154214092e-08
2794 1.68799108024587e-08
2795 1.68737113170891e-08
2796 1.68216711671221e-08
2797 1.67952176610697e-08
2798 1.68031473180008e-08
2799 1.67745657364549e-08
2800 1.67787579385958e-08
2801 1.6693887161523e-08
2802 1.67521694294237e-08
2803 1.67028986197693e-08
2804 1.67434439646286e-08
2805 1.67080393964625e-08
2806 1.67818061669323e-08
2807 1.67306577480986e-08
2808 1.66977240922961e-08
2809 1.6996155594029e-08
2810 1.70447691516529e-08
2811 1.67984577359448e-08
2812 1.68102580744289e-08
2813 1.67971077047468e-08
2814 1.67494835778825e-08
2815 1.67302562914529e-08
2816 1.67267675266203e-08
2817 1.67098797021481e-08
2818 1.67116684934854e-08
2819 1.6944126102203e-08
2820 1.70543845712245e-08
2821 1.70079523797995e-08
2822 1.67583529275817e-08
2823 1.67698139819095e-08
2824 1.67583653620795e-08
2825 1.67183955568362e-08
2826 1.66900555598204e-08
2827 1.69371219271852e-08
2828 1.69906382296858e-08
2829 1.71239502577691e-08
2830 1.68062346261877e-08
2831 1.67946101470307e-08
2832 1.67287677044214e-08
2833 1.67806124551362e-08
2834 1.69159743990122e-08
2835 1.69876095412747e-08
2836 1.70401825982935e-08
2837 1.70646963226773e-08
2838 1.67638436465722e-08
2839 1.67032876419171e-08
2840 1.67471387868545e-08
2841 1.66473714813264e-08
2842 1.6659333468283e-08
2843 1.66171947313387e-08
2844 1.66492242215099e-08
2845 1.66039413329599e-08
2846 1.66344396035356e-08
2847 1.66086433495138e-08
2848 1.66581131111343e-08
2849 1.65810938312916e-08
2850 1.65820654984827e-08
2851 1.65975002630603e-08
2852 1.66573919102575e-08
2853 1.66924429834125e-08
2854 1.68582197090927e-08
2855 1.69626392931832e-08
2856 1.66960028025187e-08
2857 1.6708476380245e-08
2858 1.67148375140869e-08
2859 1.66882561103421e-08
2860 1.66774469789743e-08
2861 1.68294054248008e-08
2862 1.68803335753864e-08
2863 1.66892295538901e-08
2864 1.67120433047785e-08
2865 1.66531197720587e-08
2866 1.66807190282725e-08
2867 1.66426215031379e-08
2868 1.66618612240654e-08
2869 1.65833515808345e-08
2870 1.66316009853062e-08
2871 1.6649666534363e-08
2872 1.66762177400415e-08
2873 1.66159814796174e-08
2874 1.66067408713388e-08
2875 1.66390758948864e-08
2876 1.66303131265977e-08
2877 1.67797011840776e-08
2878 1.66162603676412e-08
2879 1.66047193772556e-08
2880 1.66033800041987e-08
2881 1.65702811472102e-08
2882 1.65706683930011e-08
2883 1.64966973414948e-08
2884 1.65193192458446e-08
2885 1.65645577254736e-08
2886 1.64892792753335e-08
2887 1.65063713808422e-08
2888 1.65262203921657e-08
2889 1.64918496636801e-08
2890 1.65964380016703e-08
2891 1.64909703670446e-08
2892 1.65471831792274e-08
2893 1.66186424621628e-08
2894 1.66029465731299e-08
2895 1.65624101100548e-08
2896 1.67995306554758e-08
2897 1.65133080543001e-08
2898 1.65797420237368e-08
2899 1.65902616089397e-08
2900 1.65549582931135e-08
2901 1.65489009162911e-08
2902 1.66694249514876e-08
2903 1.65851510303128e-08
2904 1.66258011802256e-08
2905 1.65626872217217e-08
2906 1.65754308056876e-08
2907 1.66303042448135e-08
2908 1.65458242662453e-08
2909 1.65780100758184e-08
2910 1.66413656188524e-08
2911 1.67196443356943e-08
2912 1.66034439530449e-08
2913 1.66745568463966e-08
2914 1.65869362689364e-08
2915 1.66033196080662e-08
2916 1.66878688645511e-08
2917 1.67341731582837e-08
2918 1.65535336549283e-08
2919 1.66653180144749e-08
2920 1.65847442445965e-08
2921 1.6618010079128e-08
2922 1.66573688176186e-08
2923 1.6597967444909e-08
2924 1.66737414986073e-08
2925 1.66262381640081e-08
2926 1.6674652769666e-08
2927 1.66069344942343e-08
2928 1.66482294616799e-08
2929 1.66866946926802e-08
2930 1.67299472053628e-08
2931 1.6588201035006e-08
2932 1.67134359685406e-08
2933 1.65975766464044e-08
2934 1.66009677116108e-08
2935 1.66773777010576e-08
2936 1.65866698154105e-08
2937 1.66354894304277e-08
2938 1.67068865408737e-08
2939 1.65900857496126e-08
2940 1.66069682450143e-08
2941 1.66684657187943e-08
2942 1.6578225014996e-08
2943 1.66636233700501e-08
2944 1.66374309884532e-08
2945 1.66078510943635e-08
2946 1.66079239249939e-08
2947 1.66189959571739e-08
2948 1.66431437520487e-08
2949 1.67792055805194e-08
2950 1.662156101645e-08
2951 1.6616240827716e-08
2952 1.67015343777166e-08
2953 1.65535709584219e-08
2954 1.66245595067949e-08
2955 1.66206692853166e-08
2956 1.66746261243134e-08
2957 1.65701852239408e-08
2958 1.66162763548527e-08
2959 1.667058135979e-08
2960 1.64947504543989e-08
2961 1.66080269536906e-08
2962 1.66236500120931e-08
2963 1.65551092834448e-08
2964 1.66298281811805e-08
2965 1.65289879561215e-08
2966 1.65723417211439e-08
2967 1.65593316836521e-08
2968 1.66221187924975e-08
2969 1.65440869892564e-08
2970 1.66211329144517e-08
2971 1.65424012266158e-08
2972 1.65588289746665e-08
2973 1.66200813112027e-08
2974 1.65314499867009e-08
2975 1.65996283385539e-08
2976 1.65870890356246e-08
2977 1.65447708866395e-08
2978 1.65691957931813e-08
2979 1.65050249023579e-08
2980 1.64954094827863e-08
2981 1.65677125352204e-08
2982 1.64785500800235e-08
2983 1.65327005419158e-08
2984 1.66053162331536e-08
2985 1.64765108223719e-08
2986 1.65074123259501e-08
2987 1.65737201740512e-08
2988 1.65258615680841e-08
2989 1.6549400072563e-08
2990 1.65950471142651e-08
2991 1.65097837623307e-08
2992 1.65702083165797e-08
2993 1.64947184799757e-08
2994 1.65669664653478e-08
2995 1.6511329192781e-08
2996 1.66073501617348e-08
2997 1.6505278921386e-08
2998 1.65534839169368e-08
2999 1.65742477520325e-08
3000 1.67170313147835e-08
3001 1.64747380182462e-08
3002 1.65136704310953e-08
3003 1.65748410552169e-08
3004 1.65943827568071e-08
3005 1.65071210034284e-08
3006 1.64974611749358e-08
3007 1.66000990731163e-08
3008 1.65404756558019e-08
3009 1.6583367568046e-08
3010 1.65512901162401e-08
3011 1.66194666917363e-08
3012 1.65734856949484e-08
3013 1.65506452987074e-08
3014 1.6591021889667e-08
3015 1.65345479530288e-08
3016 1.65239857352617e-08
3017 1.66298459447489e-08
3018 1.65297979748402e-08
3019 1.65248952299635e-08
3020 1.65923204065166e-08
3021 1.66899738474058e-08
3022 1.65842344301836e-08
3023 1.66388218758584e-08
3024 1.65867213297588e-08
3025 1.66422644554132e-08
3026 1.65288014386533e-08
3027 1.6568092675584e-08
3028 1.66481424201947e-08
3029 1.65280802377765e-08
3030 1.65763172077504e-08
3031 1.66517537536492e-08
3032 1.65436073729097e-08
3033 1.65732849666256e-08
3034 1.66419589220368e-08
3035 1.65173652533213e-08
3036 1.65405840135691e-08
3037 1.66375286880793e-08
3038 1.65336917490322e-08
3039 1.65859859180273e-08
3040 1.65906097748802e-08
3041 1.6547829773117e-08
3042 1.65838294208243e-08
3043 1.6553338255676e-08
3044 1.66476255003545e-08
3045 1.65451989886378e-08
3046 1.65688707198797e-08
3047 1.65730842383027e-08
3048 1.64962052906503e-08
3049 1.65356244252735e-08
3050 1.66055915684638e-08
3051 1.65078404279484e-08
3052 1.65336420110407e-08
3053 1.66343117058432e-08
3054 1.65007598695865e-08
3055 1.65767115589688e-08
3056 1.66103308885113e-08
3057 1.65125690898549e-08
3058 1.65004880869901e-08
3059 1.66195093243005e-08
3060 1.64628062293559e-08
3061 1.6668469271508e-08
3062 1.64759441645401e-08
3063 1.64428168858421e-08
3064 1.64854618844856e-08
3065 1.66327041029035e-08
3066 1.6484539955286e-08
3067 1.65728977208346e-08
3068 1.64821862824738e-08
3069 1.64723044093762e-08
3070 1.67092100156196e-08
3071 1.66125886380541e-08
3072 1.65582321187685e-08
3073 1.64975766381303e-08
3074 1.64514339928701e-08
3075 1.65458544643116e-08
3076 1.6624015941602e-08
3077 1.67307838694342e-08
3078 1.65583031730421e-08
3079 1.64996531992756e-08
3080 1.64848845685128e-08
3081 1.67427671726728e-08
3082 1.64718372275274e-08
3083 1.64960685111737e-08
3084 1.66058935491264e-08
3085 1.64641154043466e-08
3086 1.65231064386262e-08
3087 1.65603104562706e-08
3088 1.64624118781376e-08
3089 1.66317644101355e-08
3090 1.64057354368197e-08
3091 1.65617510816674e-08
3092 1.64406586122823e-08
3093 1.64620885811928e-08
3094 1.66452558403307e-08
3095 1.66022680048172e-08
3096 1.65510076755027e-08
3097 1.6714681194685e-08
3098 1.64345284048295e-08
3099 1.6414071879467e-08
3100 1.65437192833906e-08
3101 1.64472826469364e-08
3102 1.65071512014947e-08
3103 1.65438311938715e-08
3104 1.66327502881813e-08
3105 1.65523790229827e-08
3106 1.66879505769657e-08
3107 1.65908300431283e-08
3108 1.66783511446056e-08
3109 1.65269895546771e-08
3110 1.66139599855342e-08
3111 1.65913967009601e-08
3112 1.66371183496494e-08
3113 1.65695315246239e-08
3114 1.66458455908014e-08
3115 1.64871103436326e-08
3116 1.6624259302489e-08
3117 1.65276308194962e-08
3118 1.65495706028196e-08
3119 1.66858313832563e-08
3120 1.6488554521743e-08
3121 1.65506683913463e-08
3122 1.66762088582573e-08
3123 1.65670623886172e-08
3124 1.65750826397471e-08
3125 1.6656512613622e-08
3126 1.65656288686478e-08
3127 1.6607867081575e-08
3128 1.65372231464289e-08
3129 1.66547771129899e-08
3130 1.65291016429592e-08
3131 1.65543756480702e-08
3132 1.65888387471114e-08
3133 1.65569851162672e-08
3134 1.65752460645763e-08
3135 1.66374451993079e-08
3136 1.65328657431019e-08
3137 1.66314855221117e-08
3138 1.65398255091986e-08
3139 1.65826730125218e-08
3140 1.65350666492259e-08
3141 1.65807776397742e-08
3142 1.64444546868481e-08
3143 1.64428257676263e-08
3144 1.65242504124308e-08
3145 1.65663198714583e-08
3146 1.64250693046597e-08
3147 1.64358180398949e-08
3148 1.65625859693819e-08
3149 1.63933719932174e-08
3150 1.64106008782028e-08
3151 1.65370153126787e-08
3152 1.64330984375738e-08
3153 1.65305760191359e-08
3154 1.64766298382801e-08
3155 1.64951075021236e-08
3156 1.64684248460389e-08
3157 1.65372675553499e-08
3158 1.64156386261993e-08
3159 1.64181841455502e-08
3160 1.65108833272143e-08
3161 1.64077320619072e-08
3162 1.63948197240416e-08
3163 1.65413158725869e-08
3164 1.64120059764628e-08
3165 1.6405433456157e-08
3166 1.64857336670821e-08
3167 1.63912172723713e-08
3168 1.63866360480824e-08
3169 1.64529811996772e-08
3170 1.6420917958726e-08
3171 1.64480553621615e-08
3172 1.64140594449691e-08
3173 1.65469611346225e-08
3174 1.6422511350811e-08
3175 1.64229572163777e-08
3176 1.63821916032703e-08
3177 1.64573137340085e-08
3178 1.63977720291086e-08
3179 1.63791806784275e-08
3180 1.65262772355845e-08
3181 1.64822093751127e-08
3182 1.640237989875e-08
3183 1.63972568856252e-08
3184 1.65336970781027e-08
3185 1.63852416079635e-08
3186 1.65051954326145e-08
3187 1.64998734675237e-08
3188 1.63704001465703e-08
3189 1.63999374080959e-08
3190 1.65094196091786e-08
3191 1.63854014800791e-08
3192 1.64936047042374e-08
3193 1.64064566376965e-08
3194 1.64749902609174e-08
3195 1.63795803587163e-08
3196 1.64676396963159e-08
3197 1.63112758855277e-08
3198 1.64559637028105e-08
3199 1.64793920731654e-08
3200 1.63816178400111e-08
3201 1.6502015753872e-08
3202 1.63871103353586e-08
3203 1.63996887181383e-08
3204 1.63758926419177e-08
3205 1.64090696586072e-08
3206 1.63212323656126e-08
3207 1.63064850511319e-08
3208 1.64156404025562e-08
3209 1.64822253623242e-08
3210 1.63665383468015e-08
3211 1.64078972630932e-08
3212 1.63348179427203e-08
3213 1.64848437123055e-08
3214 1.627002710336e-08
3215 1.65110023431225e-08
3216 1.63551501231041e-08
3217 1.63542495101865e-08
3218 1.62984807872135e-08
3219 1.62763864608451e-08
3220 1.64195448348892e-08
3221 1.63696274313452e-08
3222 1.63714091172551e-08
3223 1.64675011404825e-08
3224 1.63570934574864e-08
3225 1.62733932995707e-08
3226 1.64321836138015e-08
3227 1.63397242403107e-08
3228 1.64787223866369e-08
3229 1.65668492257964e-08
3230 1.6587083706554e-08
3231 1.6419491544184e-08
3232 1.64535087776585e-08
3233 1.63141429254665e-08
3234 1.61914250895734e-08
3235 1.65752869207836e-08
3236 1.62820068538849e-08
3237 1.63122333418642e-08
3238 1.62540896297969e-08
3239 1.62232360878534e-08
3240 1.64856928108748e-08
3241 1.6278908887557e-08
3242 1.62918212254226e-08
3243 1.63047229051472e-08
3244 1.6551789272512e-08
3245 1.62375624057631e-08
3246 1.63416586929088e-08
3247 1.62154574212536e-08
3248 1.64410867142806e-08
3249 1.64454814211012e-08
3250 1.64107927247414e-08
3251 1.63740470071616e-08
3252 1.63411826292759e-08
3253 1.62256004188066e-08
3254 1.63521374219044e-08
3255 1.63089826088481e-08
3256 1.63939599673313e-08
3257 1.62171094331143e-08
3258 1.65713043287496e-08
3259 1.62563740246924e-08
3260 1.61841278156771e-08
3261 1.61661635189603e-08
3262 1.63004134634548e-08
3263 1.63219961990535e-08
3264 1.6316924700277e-08
3265 1.60927928760657e-08
3266 1.64622395715242e-08
3267 1.62935602787684e-08
3268 1.62477391540961e-08
3269 1.61466093828722e-08
3270 1.61475330884286e-08
3271 1.62058046981883e-08
3272 1.61516062746614e-08
3273 1.60485651434783e-08
3274 1.62038045203872e-08
3275 1.61147859500943e-08
3276 1.62649484991562e-08
3277 1.62387117086382e-08
3278 1.62683768678562e-08
3279 1.61933311204621e-08
3280 1.61380295793379e-08
3281 1.64027138538358e-08
3282 1.63047566559271e-08
3283 1.61803512810366e-08
3284 1.62751305765596e-08
3285 1.62993192276417e-08
3286 1.62868438735586e-08
3287 1.62123825475646e-08
3288 1.63002980002602e-08
3289 1.61668261000614e-08
3290 1.62837991979359e-08
3291 1.63156155252864e-08
3292 1.63301336897348e-08
3293 1.62416977644853e-08
3294 1.63429749733268e-08
3295 1.61885438387799e-08
3296 1.64408522351778e-08
3297 1.61202926562964e-08
3298 1.59894000262284e-08
3299 1.62974167494667e-08
3300 1.6262605484485e-08
3301 1.6245428113848e-08
3302 1.6182193363079e-08
3303 1.62943454284914e-08
3304 1.61187436731325e-08
3305 1.62135496140081e-08
3306 1.62549138593704e-08
3307 1.62539102177561e-08
3308 1.61443516333293e-08
3309 1.62583884133483e-08
3310 1.62574789186465e-08
3311 1.62469575570867e-08
3312 1.61147859500943e-08
3313 1.6182381656904e-08
3314 1.6499427601957e-08
3315 1.63571503009052e-08
3316 1.6392547763644e-08
3317 1.62588591479107e-08
3318 1.64536277935667e-08
3319 1.61732547354632e-08
3320 1.61475863791338e-08
3321 1.63900981675624e-08
3322 1.64129456692308e-08
3323 1.6325994778299e-08
3324 1.61743773929857e-08
3325 1.63455204926777e-08
3326 1.61785713714835e-08
3327 1.62711710771646e-08
3328 1.62960454019867e-08
3329 1.63105902117877e-08
3330 1.62252362656545e-08
3331 1.6295777172104e-08
3332 1.60538018434409e-08
3333 1.62901514499936e-08
3334 1.62095190603395e-08
3335 1.60177116015348e-08
3336 1.62150310956122e-08
3337 1.62219802035679e-08
3338 1.633882540375e-08
3339 1.62170774586912e-08
3340 1.62790740887431e-08
3341 1.6012654313613e-08
3342 1.62284159443971e-08
3343 1.57996282723616e-08
3344 1.62464743880264e-08
3345 1.62889470800565e-08
3346 1.58650568238272e-08
3347 1.62492632682643e-08
3348 1.62847957341228e-08
3349 1.63373865547101e-08
3350 1.63446287615443e-08
3351 1.63261013597094e-08
3352 1.60086219835875e-08
3353 1.62969975292526e-08
3354 1.63560258670259e-08
3355 1.63109952211471e-08
3356 1.63270748032573e-08
3357 1.63637849937004e-08
3358 1.59012465417163e-08
3359 1.59336792648901e-08
3360 1.63083981874479e-08
3361 1.6358800536409e-08
3362 1.63159477040153e-08
3363 1.63599906954914e-08
3364 1.6292911908522e-08
3365 1.60083839517711e-08
3366 1.62626729860449e-08
3367 1.632326274148e-08
3368 1.61567417222841e-08
3369 1.62248898760708e-08
3370 1.62602109554655e-08
3371 1.63396833841034e-08
3372 1.61848117130603e-08
3373 1.62774149714551e-08
3374 1.62036819517652e-08
3375 1.62693876148978e-08
3376 1.63220228444061e-08
3377 1.64007687430967e-08
3378 1.6151156856381e-08
3379 1.62304161221982e-08
3380 1.61487943017846e-08
3381 1.62286504234999e-08
3382 1.62880251508568e-08
3383 1.6401395797061e-08
3384 1.63723719026621e-08
3385 1.63546687304006e-08
3386 1.64031561666889e-08
3387 1.63291975496804e-08
3388 1.63185216450756e-08
3389 1.63488742543905e-08
3390 1.6327277307937e-08
3391 1.64203886043879e-08
3392 1.62286877269935e-08
3393 1.64065490082521e-08
3394 1.60971502793927e-08
3395 1.61937752096719e-08
3396 1.61187188041367e-08
3397 1.63272453335139e-08
3398 1.623996404021e-08
3399 1.63896132221453e-08
3400 1.60176938379664e-08
3401 1.63987774470797e-08
3402 1.63427191779419e-08
3403 1.64033515659412e-08
3404 1.60500981394307e-08
3405 1.62952602522637e-08
3406 1.63858313584342e-08
3407 1.63939049002693e-08
3408 1.59321000836599e-08
3409 1.59415503020455e-08
3410 1.6350295339862e-08
3411 1.62661599745206e-08
3412 1.6372858624436e-08
3413 1.60868012244464e-08
3414 1.63745568215745e-08
3415 1.62362283617767e-08
3416 1.63497695382375e-08
3417 1.60527697801172e-08
3418 1.61382232022333e-08
3419 1.62228435129919e-08
3420 1.61482436311644e-08
3421 1.56469539547288e-08
3422 1.64657532053525e-08
3423 1.64455133955244e-08
3424 1.64431632754258e-08
3425 1.55095136733507e-08
3426 1.6231435751024e-08
3427 1.61126170183934e-08
3428 1.61452966551678e-08
3429 1.53816088754866e-08
3430 1.61558553202212e-08
3431 1.61753206384674e-08
3432 1.61128337339278e-08
3433 1.57619428620137e-08
3434 1.57375659171066e-08
3435 1.59109241337774e-08
3436 1.61742903515005e-08
3437 1.60244937319476e-08
3438 1.61413886701212e-08
3439 1.60107447300106e-08
3440 1.60885296196511e-08
3441 1.61147362121028e-08
3442 1.61586708458117e-08
3443 1.61723434644045e-08
3444 1.53115635725953e-08
3445 1.60971005414012e-08
3446 1.61075011106959e-08
3447 1.61047886138022e-08
3448 1.60467568122158e-08
3449 1.5965712307775e-08
3450 1.60745052824041e-08
3451 1.58284407802967e-08
3452 1.60802589022069e-08
3453 1.606993826897e-08
3454 1.60419375561105e-08
3455 1.604882271522e-08
3456 1.61162230227774e-08
3457 1.61340647508723e-08
3458 1.57255364285902e-08
3459 1.53930397317481e-08
3460 1.60282596084471e-08
3461 1.60027955331543e-08
3462 1.57066928352378e-08
3463 1.60346438349279e-08
3464 1.59815112255046e-08
3465 1.53061172625257e-08
3466 1.60683644168103e-08
3467 1.61407367471611e-08
3468 1.61489595029707e-08
3469 1.62940434478287e-08
3470 1.60009268057593e-08
3471 1.60392907844198e-08
3472 1.58820014917183e-08
3473 1.60746580490922e-08
3474 1.52844474854419e-08
3475 1.61305599988282e-08
3476 1.58554076534756e-08
3477 1.60321302900002e-08
3478 1.56977151277715e-08
3479 1.60032591622894e-08
3480 1.62114037749461e-08
3481 1.60066822019189e-08
3482 1.59614863548541e-08
3483 1.56156971797827e-08
3484 1.59392747889342e-08
3485 1.60431188334087e-08
3486 1.57855932769735e-08
3487 1.59930593213176e-08
3488 1.59965640733617e-08
3489 1.60996034281879e-08
3490 1.60283448735754e-08
3491 1.56714161647642e-08
3492 1.60711621788323e-08
3493 1.61817901300765e-08
3494 1.6084936049765e-08
3495 1.58670996341925e-08
3496 1.59936366372904e-08
3497 1.60435291718386e-08
3498 1.59550310740997e-08
3499 1.59517696829425e-08
3500 1.581418018759e-08
3501 1.56789319305517e-08
3502 1.59706079472244e-08
3503 1.59470836536002e-08
3504 1.59452575587693e-08
3505 1.58961714902262e-08
3506 1.59659414578073e-08
3507 1.59637369989696e-08
3508 1.60445612351623e-08
3509 1.54113664052602e-08
3510 1.59824953271936e-08
3511 1.58512012404799e-08
3512 1.60904782831039e-08
3513 1.59945852118426e-08
3514 1.58542672323847e-08
3515 1.58960240526085e-08
3516 1.55235007071042e-08
3517 1.58677515571526e-08
3518 1.59771023078292e-08
3519 1.55858419503829e-08
3520 1.59426765122817e-08
3521 1.58808557415568e-08
3522 1.58543720374382e-08
3523 1.59098565433169e-08
3524 1.59465631810463e-08
3525 1.59205821859132e-08
3526 1.59463180438024e-08
3527 1.54518495776301e-08
3528 1.60131126136775e-08
3529 1.58489292800823e-08
3530 1.59389763609852e-08
3531 1.59357167461849e-08
3532 1.58736241928636e-08
3533 1.58921906745491e-08
3534 1.58827440088771e-08
3535 1.58254866988727e-08
3536 1.57847388493337e-08
3537 1.58858579624166e-08
3538 1.57701158798318e-08
3539 1.55605679452719e-08
3540 1.59172017788478e-08
3541 1.58954627238472e-08
3542 1.57868278449769e-08
3543 1.54241615035744e-08
3544 1.58728710175637e-08
3545 1.57283999158153e-08
3546 1.59497925977803e-08
3547 1.57469610684302e-08
3548 1.55662309708759e-08
3549 1.59295705515206e-08
3550 1.58324127141896e-08
3551 1.55741304297408e-08
3552 1.59062540916466e-08
3553 1.58248791848337e-08
3554 1.58235291536357e-08
3555 1.59408983790854e-08
3556 1.56980899390646e-08
3557 1.59162318880135e-08
3558 1.57701922631759e-08
3559 1.5957748900064e-08
3560 1.57282826762639e-08
3561 1.5850293522135e-08
3562 1.56531871908783e-08
3563 1.59732973514792e-08
3564 1.56974380161046e-08
3565 1.5878320880347e-08
3566 1.55885206964967e-08
3567 1.58045398990225e-08
3568 1.58728816757048e-08
3569 1.56946722285056e-08
3570 1.58860657961668e-08
3571 1.59024438062261e-08
3572 1.53743613395818e-08
3573 1.58390260907026e-08
3574 1.58082009704685e-08
3575 1.57799622257926e-08
3576 1.58084638712808e-08
3577 1.57921959953455e-08
3578 1.583313746778e-08
3579 1.55775907728639e-08
3580 1.57941002498774e-08
3581 1.57625787977622e-08
3582 1.56568908948884e-08
3583 1.58851491960377e-08
3584 1.57799160405148e-08
3585 1.55538497637053e-08
3586 1.58498991709166e-08
3587 1.56715209698177e-08
3588 1.57424580038423e-08
3589 1.57286166313497e-08
3590 1.50032608559059e-08
3591 1.58319473086976e-08
3592 1.55790633726838e-08
3593 1.57480410933886e-08
3594 1.56046748855942e-08
3595 1.55766155529591e-08
3596 1.5839708211729e-08
3597 1.55906008103557e-08
3598 1.57970525549445e-08
3599 1.56006816354193e-08
3600 1.58161732599638e-08
3601 1.56401220863245e-08
3602 1.54656500939154e-08
3603 1.57246109466769e-08
3604 1.52208805559439e-08
3605 1.58277675410545e-08
3606 1.58437618580365e-08
3607 1.5773929717966e-08
3608 1.5843756528966e-08
3609 1.55855968131391e-08
3610 1.54954005182617e-08
3611 1.54153134701573e-08
3612 1.5831721711379e-08
3613 1.57475756878966e-08
3614 1.57283199797575e-08
3615 1.57355408703097e-08
3616 1.57596531380477e-08
3617 1.57566475422755e-08
3618 1.5763255589718e-08
3619 1.57710271508904e-08
3620 1.55558215197971e-08
3621 1.56956865282609e-08
3622 1.57990793780982e-08
3623 1.56208255219781e-08
3624 1.55549546576594e-08
3625 1.57609427731131e-08
3626 1.55285455605281e-08
3627 1.57087445273874e-08
3628 1.56237920378999e-08
3629 1.55402641865976e-08
3630 1.57339030693038e-08
3631 1.55767114762284e-08
3632 1.57202695305614e-08
3633 1.57811133050245e-08
3634 1.55462061002254e-08
3635 1.57715813742243e-08
3636 1.5582887868959e-08
3637 1.54657477935416e-08
3638 1.57412163304116e-08
3639 1.56677391061066e-08
3640 1.53989514473096e-08
3641 1.56551571706132e-08
3642 1.568971264021e-08
3643 1.5721656865253e-08
3644 1.55720911720891e-08
3645 1.54141019947929e-08
3646 1.57794506350228e-08
3647 1.55245114541458e-08
3648 1.57947077639164e-08
3649 1.56773705128899e-08
3650 1.57595483329942e-08
3651 1.53143027148417e-08
3652 1.55140131852249e-08
3653 1.54375570105003e-08
3654 1.54181467593162e-08
3655 1.54116328587861e-08
3656 1.57353952090489e-08
3657 1.54957220388496e-08
3658 1.57544448597946e-08
3659 1.55599000351003e-08
3660 1.56464032841086e-08
3661 1.56382871097094e-08
3662 1.56557309338723e-08
3663 1.54649395511797e-08
3664 1.56899719883086e-08
3665 1.56109312143826e-08
3666 1.553384443298e-08
3667 1.57265240829929e-08
3668 1.56518318306098e-08
3669 1.54544039787652e-08
3670 1.56579602617057e-08
3671 1.59072950367545e-08
3672 1.5748035764318e-08
3673 1.56618575886114e-08
3674 1.56627315561764e-08
3675 1.58304480635252e-08
3676 1.57341251139087e-08
3677 1.53467230035176e-08
3678 1.56067958556605e-08
3679 1.56534323281221e-08
3680 1.53145460757287e-08
3681 1.57022768121351e-08
3682 1.56104995596706e-08
3683 1.5564880939678e-08
3684 1.55955675040786e-08
3685 1.54964396870128e-08
3686 1.54836037324912e-08
3687 1.54899062465574e-08
3688 1.53261812130268e-08
3689 1.52594061830769e-08
3690 1.50572976309604e-08
3691 1.53870267638467e-08
3692 1.5337084491307e-08
3693 1.53543897596364e-08
3694 1.53258543633683e-08
3695 1.54345496383712e-08
3696 1.52568784272944e-08
3697 1.52992001289931e-08
3698 1.53783563661136e-08
3699 1.53628487709057e-08
3700 1.53735175700831e-08
3701 1.53080534914807e-08
3702 1.5295949395977e-08
3703 1.51817989291203e-08
3704 1.51889096855484e-08
3705 1.53564858607069e-08
3706 1.53523664891964e-08
3707 1.53098245192496e-08
3708 1.53146366699275e-08
3709 1.52752495097275e-08
3710 1.52340717818333e-08
3711 1.51946579762807e-08
3712 1.52273482711962e-08
3713 1.50831542811147e-08
3714 1.52334642677943e-08
3715 1.51676697868197e-08
3716 1.53409622782874e-08
3717 1.50437955426241e-08
3718 1.52786636675728e-08
3719 1.51884957944048e-08
3720 1.48427004020846e-08
3721 1.52520698293301e-08
3722 1.52234278516516e-08
3723 1.48129206678504e-08
3724 1.51117305335902e-08
3725 1.48233034735767e-08
3726 1.50740824267359e-08
3727 1.5148774679119e-08
3728 1.50767984763434e-08
3729 1.48375844943871e-08
3730 1.49406016447529e-08
3731 1.51411523319211e-08
3732 1.49801895332757e-08
3733 1.49315937392203e-08
3734 1.47655310200889e-08
3735 1.49938514937276e-08
3736 1.49964733964225e-08
3737 1.4925822355849e-08
3738 1.49532652926609e-08
3739 1.49961323359094e-08
3740 1.48432217628169e-08
3741 1.48284318157721e-08
3742 1.50261794118478e-08
3743 1.49615839717399e-08
3744 1.49153525086376e-08
3745 1.51136099191262e-08
3746 1.49795216231041e-08
3747 1.49431294005353e-08
3748 1.47692702512359e-08
3749 1.49139136595977e-08
3750 1.50035930346348e-08
3751 1.46893190944297e-08
3752 1.49378180935855e-08
3753 1.50839394308377e-08
3754 1.49916878910972e-08
3755 1.49316399244981e-08
3756 1.50943204602072e-08
3757 1.51161181349835e-08
3758 1.48621692730444e-08
3759 1.50737626825048e-08
3760 1.49731000931297e-08
3761 1.50841970025795e-08
3762 1.52079984161446e-08
3763 1.49685934758281e-08
3764 1.48430645552367e-08
3765 1.47867629252119e-08
3766 1.46798440070484e-08
3767 1.48301220193048e-08
3768 1.43359963900025e-08
3769 1.48526719812025e-08
3770 1.4799962144707e-08
3771 1.47700944808093e-08
3772 1.50774646101581e-08
3773 1.47000100980677e-08
3774 1.49729189047321e-08
3775 1.47321053134419e-08
3776 1.51261581038398e-08
3777 1.50766350515141e-08
3778 1.47293528485193e-08
3779 1.48247307762972e-08
3780 1.47392986704631e-08
3781 1.473340471847e-08
3782 1.50871581894307e-08
3783 1.49787506842358e-08
3784 1.49997170240113e-08
3785 1.50083643291055e-08
3786 1.50386743058561e-08
3787 1.50206709292888e-08
3788 1.49896592915866e-08
3789 1.51417278715371e-08
3790 1.50221559636066e-08
3791 1.50181964642115e-08
3792 1.50136987286942e-08
3793 1.46793048827476e-08
3794 1.49659715731332e-08
3795 1.47027012786793e-08
3796 1.47452476895182e-08
3797 1.44690366354894e-08
3798 1.46862806360559e-08
3799 1.47301211228523e-08
3800 1.46770560149889e-08
3801 1.47571652675538e-08
3802 1.48220564710755e-08
3803 1.47757175383845e-08
3804 1.47301157937818e-08
3805 1.49149030903573e-08
3806 1.48438044078603e-08
3807 1.5044548717924e-08
3808 1.44204062024755e-08
3809 1.46857841443193e-08
3810 1.46637475495481e-08
3811 1.4670026082797e-08
3812 1.50709329460597e-08
3813 1.51022891969887e-08
3814 1.47306646880452e-08
3815 1.48806993394146e-08
3816 1.4673283033062e-08
3817 1.47101406611228e-08
3818 1.49055434661705e-08
3819 1.47209009426774e-08
3820 1.47340575296084e-08
3821 1.4257519609373e-08
3822 1.4482014698558e-08
3823 1.46177727700092e-08
3824 1.42178704365392e-08
3825 1.47097214409087e-08
3826 1.46871999007203e-08
3827 1.53316008777438e-08
3828 1.51410119997308e-08
3829 1.49575960506354e-08
3830 1.50954555522276e-08
3831 1.5178439838337e-08
3832 1.49248347014463e-08
3833 1.51125938430141e-08
3834 1.46834553405029e-08
3835 1.51313894747318e-08
3836 1.4750764165683e-08
3837 1.4939228520916e-08
3838 1.49393954984589e-08
3839 1.47343985901216e-08
3840 1.47223691016052e-08
3841 1.51018149097126e-08
3842 1.51098529244109e-08
3843 1.50956100952726e-08
3844 1.46259884203914e-08
3845 1.46710137371997e-08
3846 1.50485277572443e-08
3847 1.50664796194633e-08
3848 1.4706177609014e-08
3849 1.51045949081663e-08
3850 1.49558925244264e-08
3851 1.45701379850038e-08
3852 1.46095224806686e-08
3853 1.49365462220885e-08
3854 1.45180161226222e-08
3855 1.5260898322822e-08
3856 1.49405856575413e-08
3857 1.49662433557296e-08
3858 1.48421026580081e-08
3859 1.50446357594092e-08
3860 1.48667025356986e-08
3861 1.49507357605216e-08
3862 1.52981360912463e-08
3863 1.48791654552838e-08
3864 1.50070729176832e-08
3865 1.48828860346839e-08
3866 1.48211531936226e-08
3867 1.49752583666896e-08
3868 1.45738869861134e-08
3869 1.46222216557135e-08
3870 1.45522065508885e-08
3871 1.4613554810694e-08
3872 1.46173402271188e-08
3873 1.49317855857589e-08
3874 1.47392142935132e-08
3875 1.45946223994997e-08
3876 1.48009267064708e-08
3877 1.49637671142955e-08
3878 1.48947085776285e-08
3879 1.46999266092962e-08
3880 1.48350878248493e-08
3881 1.50359831252445e-08
3882 1.47717527099189e-08
3883 1.49453160958046e-08
3884 1.3960478106867e-08
3885 1.44930130119292e-08
3886 1.46886716123618e-08
3887 1.45402418993967e-08
3888 1.49687782169394e-08
3889 1.45200145240665e-08
3890 1.49934020754472e-08
3891 1.48572532054914e-08
3892 1.45633975989767e-08
3893 1.47671990191611e-08
3894 1.48269014843549e-08
3895 1.45231320303196e-08
3896 1.45738505707982e-08
3897 1.48645993292007e-08
3898 1.50021541855949e-08
3899 1.44984122485425e-08
3900 1.50182231095641e-08
3901 1.50168801837935e-08
3902 1.51470427312006e-08
3903 1.51234882395102e-08
3904 1.46419987245849e-08
3905 1.47567771335844e-08
3906 1.50095669226857e-08
3907 1.48522820708763e-08
3908 1.47283207851956e-08
3909 1.45901140058413e-08
3910 1.45764538217463e-08
3911 1.43486751369437e-08
3912 1.43963339027664e-08
3913 1.40261500192196e-08
3914 1.50444261493021e-08
3915 1.51059662556463e-08
3916 1.43957068488021e-08
3917 1.46015066704308e-08
3918 1.49445060770859e-08
3919 1.52340629000491e-08
3920 1.50698582501718e-08
3921 1.44850549332887e-08
3922 1.48793537491088e-08
3923 1.48041312542091e-08
3924 1.48888945616932e-08
3925 1.49106380575859e-08
3926 1.41548879284414e-08
3927 1.45002898577218e-08
3928 1.44480267749714e-08
3929 1.49888563782952e-08
3930 1.44971652460413e-08
3931 1.41234348660646e-08
3932 1.42806992897704e-08
3933 1.45586041000456e-08
3934 1.48175232084213e-08
3935 1.42902258915001e-08
3936 1.4444999862917e-08
3937 1.47403849126704e-08
3938 1.46539242962263e-08
3939 1.47640673020533e-08
3940 1.47310883491514e-08
3941 1.45394576378521e-08
3942 1.46013512392074e-08
3943 1.44246161681849e-08
3944 1.46174787829523e-08
3945 1.45908831683528e-08
3946 1.46028771297324e-08
3947 1.46898040398469e-08
3948 1.45524658989871e-08
3949 1.43124827545194e-08
3950 1.44758880438189e-08
3951 1.43953853282142e-08
3952 1.45506451332267e-08
3953 1.43863294610469e-08
3954 1.45093066450386e-08
3955 1.43445717526447e-08
3956 1.45235308224301e-08
3957 1.43429650378835e-08
3958 1.44223522013931e-08
3959 1.43077780734302e-08
3960 1.44441054672484e-08
3961 1.41335307901613e-08
3962 1.44105563038011e-08
3963 1.42589833274087e-08
3964 1.42915013157108e-08
3965 1.44304923566096e-08
3966 1.42014933146584e-08
3967 1.43953897691063e-08
3968 1.41130156450231e-08
3969 1.42863028074203e-08
3970 1.39229578977051e-08
3971 1.41236826678437e-08
3972 1.43240050931581e-08
3973 1.44137697333235e-08
3974 1.42479539277929e-08
3975 1.42542173620086e-08
3976 1.43237279814912e-08
3977 1.39891858097485e-08
3978 1.40439526674641e-08
3979 1.39913414187731e-08
3980 1.4219260435766e-08
3981 1.44826763914807e-08
3982 1.41870568626246e-08
3983 1.43258258589185e-08
3984 1.40905127565816e-08
3985 1.39681910482636e-08
3986 1.40797631331679e-08
3987 1.41904390460468e-08
3988 1.42116203249998e-08
3989 1.41355380733899e-08
3990 1.42160008209657e-08
3991 1.39780933494649e-08
3992 1.40409301963018e-08
3993 1.3948405097608e-08
3994 1.41333442726932e-08
3995 1.37263951316413e-08
3996 1.39230689200076e-08
3997 1.38658267090364e-08
3998 1.40729783382199e-08
3999 1.37993456661434e-08
4000 1.40878944066003e-08
4001 1.39126301590409e-08
4002 1.40176847907014e-08
4003 1.35566748937777e-08
4004 1.35109026189184e-08
4005 1.38973730301473e-08
4006 1.38861047105365e-08
4007 1.3894284833782e-08
4008 1.39820830469262e-08
4009 1.3713225222034e-08
4010 1.39848772562345e-08
4011 1.37199389627085e-08
4012 1.39336728821604e-08
4013 1.40059652764535e-08
4014 1.38059395027312e-08
4015 1.36081297341661e-08
4016 1.40875533460871e-08
4017 1.38653026837687e-08
4018 1.40418725536051e-08
4019 1.38351952116977e-08
4020 1.40084690514186e-08
4021 1.38809834737685e-08
4022 1.3862827330513e-08
4023 1.40032136997092e-08
4024 1.37442093262052e-08
4025 1.38577149755292e-08
4026 1.3448739899502e-08
4027 1.38844269415017e-08
4028 1.36692159813379e-08
4029 1.39893199246899e-08
4030 1.38701459206914e-08
4031 1.39401521437321e-08
4032 1.40973970275127e-08
4033 1.37925670884442e-08
4034 1.34696520603939e-08
4035 1.39324249914807e-08
4036 1.39498812501415e-08
4037 1.38618343470398e-08
4038 1.39994176251435e-08
4039 1.35309052851085e-08
4040 1.38179681030692e-08
4041 1.3772350371255e-08
4042 1.35322908434432e-08
4043 1.35416682311984e-08
4044 1.35318103389181e-08
4045 1.34897186754301e-08
4046 1.33964626058969e-08
4047 1.34862201406349e-08
4048 1.37644411424276e-08
4049 1.38188349652069e-08
4050 1.33108555289141e-08
4051 1.36817721596572e-08
4052 1.35118058963712e-08
4053 1.39091955730919e-08
4054 1.39029410206604e-08
4055 1.37501290353725e-08
4056 1.36903031133784e-08
4057 1.38722171527661e-08
4058 1.34050015532239e-08
4059 1.36851801002535e-08
4060 1.37787203868811e-08
4061 1.37974431879684e-08
4062 1.35608813067734e-08
4063 1.374420666167e-08
4064 1.40020484096226e-08
4065 1.3511731289384e-08
4066 1.31577779782788e-08
4067 1.36975826237062e-08
4068 1.36491999924715e-08
4069 1.38546702999065e-08
4070 1.35746969220918e-08
4071 1.31958213245298e-08
4072 1.34926505523936e-08
4073 1.33590472017886e-08
4074 1.36374342929457e-08
4075 1.36197915168168e-08
4076 1.35284876634501e-08
4077 1.35795907851843e-08
4078 1.32108697314948e-08
4079 1.34405215845845e-08
4080 1.35728059902362e-08
4081 1.36765994085408e-08
4082 1.35267175238596e-08
4083 1.34737847545807e-08
4084 1.34413262742328e-08
4085 1.3474688920212e-08
4086 1.35426416747464e-08
4087 1.34528592710126e-08
4088 1.33846240757407e-08
4089 1.34794841954999e-08
4090 1.31710171658028e-08
4091 1.32519382134433e-08
4092 1.37556774859604e-08
4093 1.33468684992977e-08
4094 1.33189255180355e-08
4095 1.35886013552522e-08
4096 1.29684076810577e-08
4097 1.33838948812581e-08
4098 1.3159611178537e-08
4099 1.34410038654664e-08
4100 1.35146533963848e-08
4101 1.31511850298693e-08
4102 1.33458559758992e-08
4103 1.34070994306512e-08
4104 1.33774733512837e-08
4105 1.31964315031041e-08
4106 1.34050113231865e-08
4107 1.32384307960365e-08
4108 1.3278481425516e-08
4109 1.35888562624586e-08
4110 1.32963045018641e-08
4111 1.30024178091048e-08
4112 1.32475701519752e-08
4113 1.32849340417351e-08
4114 1.31719097851146e-08
4115 1.31218476084882e-08
4116 1.30934765252277e-08
4117 1.28506201235723e-08
4118 1.30280080057332e-08
4119 1.30567912037804e-08
4120 1.36084707946793e-08
4121 1.30455717339828e-08
4122 1.30930066788437e-08
4123 1.40305660423223e-08
4124 1.43258462870222e-08
4125 1.39261917553313e-08
4126 1.39891271899728e-08
4127 1.40356473110614e-08
4128 1.39613405281125e-08
4129 1.41387399565929e-08
4130 1.39115785557919e-08
4131 1.39692470924047e-08
4132 1.38012623551731e-08
4133 1.39805829135753e-08
4134 1.4099439837878e-08
4135 1.36397737549032e-08
4136 1.38161828644456e-08
4137 1.37057627469517e-08
4138 1.3866748638236e-08
4139 1.37340032679845e-08
4140 1.40328273445789e-08
4141 1.34265425444369e-08
4142 1.39059821435694e-08
4143 1.31204114239836e-08
4144 1.38615598999081e-08
4145 1.33567983340299e-08
4146 1.3708076451735e-08
4147 1.33053648099235e-08
4148 1.38607489930109e-08
4149 1.34113902205968e-08
4150 1.38764306711892e-08
4151 1.3143849564301e-08
4152 1.37205509176397e-08
4153 1.33832989135385e-08
4154 1.383612691086e-08
4155 1.36837021713632e-08
4156 1.3714281266175e-08
4157 1.36684752405358e-08
4158 1.31178392592801e-08
4159 1.40338203280521e-08
4160 1.3839777324165e-08
4161 1.36187132682153e-08
4162 1.36857076782348e-08
4163 1.36779085835315e-08
4164 1.36780249349044e-08
4165 1.36261633087997e-08
4166 1.28739934268651e-08
4167 1.38209310662774e-08
4168 1.30529613784347e-08
4169 1.43193163992805e-08
4170 1.32266162466976e-08
4171 1.36730111677252e-08
4172 1.30761064198737e-08
4173 1.38280427108839e-08
4174 1.30186448288327e-08
4175 1.36386901772312e-08
4176 1.36319133758889e-08
4177 1.34422784014987e-08
4178 1.35705597870128e-08
4179 1.36115572146878e-08
4180 1.35847812998691e-08
4181 1.34348958624741e-08
4182 1.36257520821914e-08
4183 1.34240369931149e-08
4184 1.35943558632334e-08
4185 1.34196902479289e-08
4186 1.28006210076137e-08
4187 1.28524453302248e-08
4188 1.38366385016297e-08
4189 1.31519861668039e-08
4190 1.28825758949347e-08
4191 1.41129747888158e-08
4192 1.34634303705639e-08
4193 1.34486004554901e-08
4194 1.34249971139866e-08
4195 1.34197533085967e-08
4196 1.34494611003788e-08
4197 1.33828939041791e-08
4198 1.32449198275708e-08
4199 1.31303137251848e-08
4200 1.3326181047546e-08
4201 1.28230475127111e-08
4202 1.36405633455183e-08
4203 1.34602888834934e-08
4204 1.33507471744565e-08
4205 1.39698475010164e-08
4206 1.3366818762961e-08
4207 1.3371491469627e-08
4208 1.33163249316226e-08
4209 1.28334516347195e-08
4210 1.38821718564941e-08
4211 1.28281731903712e-08
4212 1.27668364768851e-08
4213 1.33302480165298e-08
4214 1.33776616451087e-08
4215 1.34377513560935e-08
4216 1.34454012368224e-08
4217 1.32440289846159e-08
4218 1.32286555043493e-08
4219 1.26970762792666e-08
4220 1.27112160797083e-08
4221 1.29769013312853e-08
4222 1.28065709148473e-08
4223 1.26602435202017e-08
4224 1.32193820334692e-08
4225 1.32532145258324e-08
4226 1.283693595866e-08
4227 1.29912187674108e-08
4228 1.31559581006968e-08
4229 1.30825217325992e-08
4230 1.3167409385062e-08
4231 1.35225937114569e-08
4232 1.30096946548974e-08
4233 1.2975023722106e-08
4234 1.27985479991821e-08
4235 1.27432109309211e-08
4236 1.28708439461889e-08
4237 1.26997612426294e-08
4238 1.27676607064586e-08
4239 1.27669474991876e-08
4240 1.26676305001183e-08
4241 1.31055104546363e-08
4242 1.31242368084372e-08
4243 1.30119204300172e-08
4244 1.26380061971076e-08
4245 1.26890746798836e-08
4246 1.270007565779e-08
4247 1.27511663450264e-08
4248 1.26622170526502e-08
4249 1.26494787977549e-08
4250 1.26337500461204e-08
4251 1.2566040652473e-08
4252 1.27417925099849e-08
4253 1.28008785793554e-08
4254 1.30641133466725e-08
4255 1.30152200128464e-08
4256 1.29819541783149e-08
4257 1.30087522975941e-08
4258 1.2944069816001e-08
4259 1.29874413445918e-08
4260 1.29889503597269e-08
4261 1.3004335386313e-08
4262 1.29757662392649e-08
4263 1.29356214628729e-08
4264 1.25861623345713e-08
4265 1.26685621992806e-08
4266 1.2635994472987e-08
4267 1.29626380740433e-08
4268 1.27796226934151e-08
4269 1.2589202569302e-08
4270 1.25198997835696e-08
4271 1.25717800614211e-08
4272 1.25344543633332e-08
4273 1.26296173519336e-08
4274 1.24745058727171e-08
4275 1.25412515927792e-08
4276 1.27685870765504e-08
4277 1.2512091807082e-08
4278 1.27859518528339e-08
4279 1.24599699447003e-08
4280 1.24028511905294e-08
4281 1.2424789197496e-08
4282 1.25569412645632e-08
4283 1.24794459210875e-08
4284 1.25161054853606e-08
4285 1.2518889036528e-08
4286 1.2476039756848e-08
4287 1.2473091892673e-08
4288 1.24406067669725e-08
4289 1.241869451718e-08
4290 1.25442261023068e-08
4291 1.26272778899761e-08
4292 1.22963186299785e-08
4293 1.22908172528469e-08
4294 1.23690755415851e-08
4295 1.23808918672808e-08
4296 1.26057093652321e-08
4297 1.24218093588979e-08
4298 1.23913705962764e-08
4299 1.22163239524298e-08
4300 1.23241497007598e-08
4301 1.25701902220499e-08
4302 1.23043299993242e-08
4303 1.22717027650765e-08
4304 1.22887264808469e-08
4305 1.23088170767005e-08
4306 1.25432029207673e-08
4307 1.25073524870345e-08
4308 1.24566312820207e-08
4309 1.24683019464555e-08
4310 1.24653833921684e-08
4311 1.23823706843496e-08
4312 1.24051826588811e-08
4313 1.22494041576715e-08
4314 1.23239924931795e-08
4315 1.20340919451678e-08
4316 1.19745280358075e-08
4317 1.23776491278704e-08
4318 1.2204304233876e-08
4319 1.224834278446e-08
4320 1.24220056463287e-08
4321 1.2198353438464e-08
4322 1.2006824867683e-08
4323 1.23186261191677e-08
4324 1.21771037697727e-08
4325 1.22676553360179e-08
4326 1.22411210057294e-08
4327 1.23291199471964e-08
4328 1.22241372579879e-08
4329 1.22253700496344e-08
4330 1.20982068807507e-08
4331 1.20440226680785e-08
4332 1.21919150330996e-08
4333 1.279581773872e-08
4334 1.26567121228049e-08
4335 1.24190355776932e-08
4336 1.235788271714e-08
4337 1.22930723378545e-08
4338 1.24136354529014e-08
4339 1.21239756012415e-08
4340 1.2238508872997e-08
4341 1.22511760736188e-08
4342 1.22027881133135e-08
4343 1.21024097410327e-08
4344 1.28094947982049e-08
4345 1.24320145289403e-08
4346 1.2104414359726e-08
4347 1.2085146217089e-08
4348 1.21070256042799e-08
4349 1.20437917416893e-08
4350 1.20572511974615e-08
4351 1.22003065428089e-08
4352 1.21080194759315e-08
4353 1.20503713674225e-08
4354 1.22080061615293e-08
4355 1.26087025265065e-08
4356 1.23741390467558e-08
4357 1.20270255976607e-08
4358 1.23643912885996e-08
4359 1.21816503551031e-08
4360 1.23338512736382e-08
4361 1.23072609881092e-08
4362 1.19431859957331e-08
4363 1.20501075784318e-08
4364 1.2408071015102e-08
4365 1.22370495958535e-08
4366 1.19601102355205e-08
4367 1.20062990660585e-08
4368 1.19294467637587e-08
4369 1.198735777308e-08
4370 1.19857457292483e-08
4371 1.22702301652566e-08
4372 1.19593126512996e-08
4373 1.20085505983525e-08
4374 1.19618430716173e-08
4375 1.19161081002517e-08
4376 1.21805232566885e-08
4377 1.20982743823106e-08
4378 1.20288117244627e-08
4379 1.200532029344e-08
4380 1.22602337171429e-08
4381 1.19757626038108e-08
4382 1.1683701117704e-08
4383 1.19281446941955e-08
4384 1.20761747268716e-08
4385 1.20171987916251e-08
4386 1.19413767762921e-08
4387 1.19271899023943e-08
4388 1.19524097286217e-08
4389 1.2002046467785e-08
4390 1.19829843825414e-08
4391 1.2059861553837e-08
4392 1.18909291302316e-08
4393 1.17801137733409e-08
4394 1.19213767746373e-08
4395 1.19600453984958e-08
4396 1.20731389330331e-08
4397 1.17893748097231e-08
4398 1.17487060080634e-08
4399 1.18090062173337e-08
4400 1.18377210256426e-08
4401 1.17686811407225e-08
4402 1.16745031419896e-08
4403 1.17650911235501e-08
4404 1.19044365476384e-08
4405 1.21411947162642e-08
4406 1.18283676187048e-08
4407 1.16969829377922e-08
4408 1.21858221291404e-08
4409 1.18218146383242e-08
4410 1.19732321834931e-08
4411 1.19297665079898e-08
4412 1.19001963838627e-08
4413 1.1690149293031e-08
4414 1.17002656452314e-08
4415 1.17673284449893e-08
4416 1.19231771122941e-08
4417 1.20327481312188e-08
4418 1.19194440983961e-08
4419 1.18671419357952e-08
4420 1.19961658384682e-08
4421 1.19074670124064e-08
4422 1.18761960266056e-08
4423 1.1895105345161e-08
4424 1.19086598360241e-08
4425 1.19229754957928e-08
4426 1.1960403334399e-08
4427 1.205475097521e-08
4428 1.17128537979738e-08
4429 1.17189724591071e-08
4430 1.18900951306955e-08
4431 1.174889341371e-08
4432 1.20657608349006e-08
4433 1.18537153426246e-08
4434 1.17246292674622e-08
4435 1.17705543090096e-08
4436 1.18504095425465e-08
4437 1.17696368207021e-08
4438 1.18881073873922e-08
4439 1.18965450823794e-08
4440 1.18807861326786e-08
4441 1.18824434736098e-08
4442 1.16952030282391e-08
4443 1.18874829979632e-08
4444 1.18008962601834e-08
4445 1.17463825333175e-08
4446 1.1728651827525e-08
4447 1.18699885476303e-08
4448 1.17683232048194e-08
4449 1.17944249922175e-08
4450 1.1679354372518e-08
4451 1.18596350517919e-08
4452 1.22016077241938e-08
4453 1.16122444993039e-08
4454 1.17171792268778e-08
4455 1.18113394620423e-08
4456 1.18589964515081e-08
4457 1.19044516466715e-08
4458 1.17732517068703e-08
4459 1.17557821255332e-08
4460 1.17596909987583e-08
4461 1.1783457765091e-08
4462 1.17590444048687e-08
4463 1.17557652501432e-08
4464 1.17808056643298e-08
4465 1.17643343955365e-08
4466 1.17688614409417e-08
4467 1.17795666554343e-08
4468 1.17096421448082e-08
4469 1.18870486787159e-08
4470 1.17108891473094e-08
4471 1.18127427839454e-08
4472 1.1831201796042e-08
4473 1.16899414592808e-08
4474 1.15100284858727e-08
4475 1.13775238119729e-08
4476 1.18495933065788e-08
4477 1.15185097016024e-08
4478 1.15720499849203e-08
4479 1.14064535594594e-08
4480 1.15561045177515e-08
4481 1.14785008165086e-08
4482 1.15480265350243e-08
4483 1.14714682197814e-08
4484 1.14728928579666e-08
4485 1.14769065362452e-08
4486 1.13255413936031e-08
4487 1.12819122932706e-08
4488 1.13709877069823e-08
4489 1.11631539567725e-08
4490 1.1183367121248e-08
4491 1.12169775690063e-08
4492 1.12836096022306e-08
4493 1.126572612975e-08
4494 1.1457309767593e-08
4495 1.15098881536824e-08
4496 1.13474065699393e-08
4497 1.13141238600178e-08
4498 1.12267857232951e-08
4499 1.12112976680123e-08
4500 1.13184732697391e-08
4501 1.15276197476533e-08
4502 1.15202958284044e-08
4503 1.12349631820052e-08
4504 1.13683453761837e-08
4505 1.13476996688178e-08
4506 1.1215432138556e-08
4507 1.13085336650443e-08
4508 1.13954241598435e-08
4509 1.15770122377512e-08
4510 1.1428241464273e-08
4511 1.1432937263578e-08
4512 1.14175700005603e-08
4513 1.14415499297138e-08
4514 1.16016325435453e-08
4515 1.13190639083882e-08
4516 1.13356684039445e-08
4517 1.13956923897263e-08
4518 1.14632490166855e-08
4519 1.14656133476387e-08
4520 1.13959846004263e-08
4521 1.12913047800589e-08
4522 1.13459099893021e-08
4523 1.12817737374371e-08
4524 1.11221858389854e-08
4525 1.15403429035155e-08
4526 1.11306066585826e-08
4527 1.13985700878061e-08
4528 1.12557110298894e-08
4529 1.15756773055864e-08
4530 1.1137531785721e-08
4531 1.12198366153393e-08
4532 1.1218441287042e-08
4533 1.11238067646013e-08
4534 1.15803855393892e-08
4535 1.14778861970422e-08
4536 1.13046292327112e-08
4537 1.12527205331503e-08
4538 1.1442343961221e-08
4539 1.13006466406773e-08
4540 1.16805569660983e-08
4541 1.1547050426941e-08
4542 1.15017471102874e-08
4543 1.16016476425784e-08
4544 1.13404459156641e-08
4545 1.12558451448308e-08
4546 1.1241048980537e-08
4547 1.12853566491822e-08
4548 1.12993321366162e-08
4549 1.12895639503563e-08
4550 1.11659952395371e-08
4551 1.12206937075143e-08
4552 1.12286846487564e-08
4553 1.12548681485691e-08
4554 1.1733248150847e-08
4555 1.12844267263768e-08
4556 1.15891625185327e-08
4557 1.14582636712157e-08
4558 1.12722302603174e-08
4559 1.11684208548013e-08
4560 1.10884226245389e-08
4561 1.1506209318668e-08
4562 1.11189031315462e-08
4563 1.15070202255652e-08
4564 1.1327854210208e-08
4565 1.12185727374481e-08
4566 1.15609131157157e-08
4567 1.13204778884324e-08
4568 1.16710126008002e-08
4569 1.1279109202178e-08
4570 1.16316511977743e-08
4571 1.14361276004615e-08
4572 1.15513740794881e-08
4573 1.13041274119041e-08
4574 1.12486917558385e-08
4575 1.12261568929739e-08
4576 1.13602123263945e-08
4577 1.12518696582242e-08
4578 1.15038467640716e-08
4579 1.13642792953783e-08
4580 1.1165349533826e-08
4581 1.1943623867694e-08
4582 1.14081828428425e-08
4583 1.11470814800896e-08
4584 1.14342251222865e-08
4585 1.12322364742568e-08
4586 1.18897789391781e-08
4587 1.11846514272429e-08
4588 1.13123554967842e-08
4589 1.13151354952379e-08
4590 1.11299023330957e-08
4591 1.13837081983093e-08
4592 1.12273221830606e-08
4593 1.12712168487406e-08
4594 1.12838609567234e-08
4595 1.14831015807226e-08
4596 1.12172866550964e-08
4597 1.12692788434288e-08
4598 1.13939195856005e-08
4599 1.15423093305367e-08
4600 1.14896447911406e-08
4601 1.12355547088328e-08
4602 1.14407008311446e-08
4603 1.13601839046851e-08
4604 1.13826672532014e-08
4605 1.15086500329653e-08
4606 1.13739959672898e-08
4607 1.10767173211457e-08
4608 1.11598437158023e-08
4609 1.12553850684094e-08
4610 1.12551887809786e-08
4611 1.14418456931276e-08
4612 1.14318901012211e-08
4613 1.20623599997316e-08
4614 1.13740696860987e-08
4615 1.12619060743668e-08
4616 1.14558700303746e-08
4617 1.1286321210946e-08
4618 1.12962297293961e-08
4619 1.13771108090077e-08
4620 1.17727401161005e-08
4621 1.11326929896904e-08
4622 1.17226770512957e-08
4623 1.120073278571e-08
4624 1.13152349712209e-08
4625 1.1293023405301e-08
4626 1.15513660858824e-08
4627 1.12948690400572e-08
4628 1.12254658901634e-08
4629 1.12407860797248e-08
4630 1.13534301959817e-08
4631 1.13500728815552e-08
4632 1.14861702371627e-08
4633 1.15735563355202e-08
4634 1.11913704969879e-08
4635 1.12378861771845e-08
4636 1.12437446020408e-08
4637 1.12917124539536e-08
4638 1.14721396826667e-08
4639 1.12771099125553e-08
4640 1.13335500984135e-08
4641 1.12489280112982e-08
4642 1.13882938634902e-08
4643 1.12209370684013e-08
4644 1.12624478632029e-08
4645 1.14160023656495e-08
4646 1.12562448251197e-08
4647 1.12550830877467e-08
4648 1.12656426409785e-08
4649 1.12879616764872e-08
4650 1.147791550693e-08
4651 1.13928573242106e-08
4652 1.15298304237399e-08
4653 1.14155058739129e-08
4654 1.1314265968565e-08
4655 1.13642792953783e-08
4656 1.12552633879659e-08
4657 1.16043565867585e-08
4658 1.14162448383581e-08
4659 1.1431106727855e-08
4660 1.14447447074895e-08
4661 1.14401865758396e-08
4662 1.13536220425203e-08
4663 1.14632756620381e-08
4664 1.12856115563886e-08
4665 1.12134506125017e-08
4666 1.12456151057927e-08
4667 1.12780185190786e-08
4668 1.13437863547006e-08
4669 1.16528333649057e-08
4670 1.14209051105263e-08
4671 1.16048433085325e-08
4672 1.11740083852396e-08
4673 1.11587565854165e-08
4674 1.13200542273262e-08
4675 1.14234666170887e-08
4676 1.13727720574275e-08
4677 1.13804778933968e-08
4678 1.1316638293124e-08
4679 1.1272047295563e-08
4680 1.12252109829569e-08
4681 1.15655414134608e-08
4682 1.12593196988087e-08
4683 1.13495879361381e-08
4684 1.14668381456795e-08
4685 1.12089990622621e-08
4686 1.12851354927557e-08
4687 1.11578994932415e-08
4688 1.11329301333285e-08
4689 1.10606572789607e-08
4690 1.11393800850124e-08
4691 1.11473159591924e-08
4692 1.10840421285729e-08
4693 1.1120498299988e-08
4694 1.10417293086584e-08
4695 1.11037881111997e-08
4696 1.1099897001543e-08
4697 1.11499005583937e-08
4698 1.11879554509642e-08
4699 1.11118376722175e-08
4700 1.1105881547735e-08
4701 1.12140341457234e-08
4702 1.08950217736492e-08
4703 1.10089466431873e-08
4704 1.10472315739685e-08
4705 1.10377635920145e-08
4706 1.09758424571282e-08
4707 1.13532268031236e-08
4708 1.09921689528392e-08
4709 1.07586526354453e-08
4710 1.09985967000625e-08
4711 1.10118802965076e-08
4712 1.1037121439017e-08
4713 1.0943984385392e-08
4714 1.09595061914547e-08
4715 1.10707416567379e-08
4716 1.1055714566055e-08
4717 1.10172244660589e-08
4718 1.09906901357704e-08
4719 1.1026514812329e-08
4720 1.11082050224809e-08
4721 1.11940163805002e-08
4722 1.09113811319617e-08
4723 1.10919122775499e-08
4724 1.0975253594836e-08
4725 1.11121147838844e-08
4726 1.10393090224647e-08
4727 1.10931699381922e-08
4728 1.08528857012402e-08
4729 1.10498428185224e-08
4730 1.10462181623916e-08
4731 1.09864046748953e-08
4732 1.11495390697769e-08
4733 1.10476676695725e-08
4734 1.13740359353187e-08
4735 1.09052562535794e-08
4736 1.10049338530871e-08
4737 1.10523270535623e-08
4738 1.10321494162235e-08
4739 1.09712638973747e-08
4740 1.09705560191742e-08
4741 1.10975211242703e-08
4742 1.08516342578469e-08
4743 1.09247215718256e-08
4744 1.14082281399419e-08
4745 1.10709539313802e-08
4746 1.11524727230972e-08
4747 1.09589448626934e-08
4748 1.09971542983089e-08
4749 1.10636149130983e-08
4750 1.1079921868884e-08
4751 1.11798854618428e-08
4752 1.10171578526774e-08
4753 1.11844054018206e-08
4754 1.10770281835926e-08
4755 1.10013038678858e-08
4756 1.10464259961418e-08
4757 1.09696003391946e-08
4758 1.1017533552149e-08
4759 1.09996376451704e-08
4760 1.09947411175426e-08
4761 1.10092095439995e-08
4762 1.11029230254189e-08
4763 1.10168807410105e-08
4764 1.09528297542738e-08
4765 1.09190887442878e-08
4766 1.11121130075276e-08
4767 1.07282547290311e-08
4768 1.11148619197365e-08
4769 1.10019398036343e-08
4770 1.12783373751313e-08
4771 1.10897584448821e-08
4772 1.11622977527759e-08
4773 1.09773585776907e-08
4774 1.11624114396136e-08
4775 1.10525171237441e-08
4776 1.10795488339477e-08
4777 1.10514823958852e-08
4778 1.11081632780952e-08
4779 1.11061853047545e-08
4780 1.10506537254196e-08
4781 1.10325117930188e-08
4782 1.10428768351767e-08
4783 1.09633999656467e-08
4784 1.08833200229697e-08
4785 1.11270583857959e-08
4786 1.10675122400039e-08
4787 1.10656701579614e-08
4788 1.10062883251771e-08
4789 1.10154010357633e-08
4790 1.09756612687306e-08
4791 1.0953363549504e-08
4792 1.10145892406877e-08
4793 1.10606892533838e-08
4794 1.11502629351889e-08
4795 1.10277786902202e-08
4796 1.07422835071702e-08
4797 1.11571303307301e-08
4798 1.09832081207628e-08
4799 1.09524105340597e-08
4800 1.11824123294468e-08
4801 1.09441478102212e-08
4802 1.09516564705814e-08
4803 1.10013891330141e-08
4804 1.11800586566346e-08
4805 1.10241886730478e-08
4806 1.0977879938423e-08
4807 1.11589475437768e-08
4808 1.08973683410341e-08
4809 1.0969446684328e-08
4810 1.09205862131034e-08
4811 1.10070246250871e-08
4812 1.08583657620898e-08
4813 1.07348929745399e-08
4814 1.12515969874494e-08
4815 1.10571480860244e-08
4816 1.09986846297261e-08
4817 1.10402895714401e-08
4818 1.14149738550395e-08
4819 1.10906634986918e-08
4820 1.1097867513854e-08
4821 1.09458779817828e-08
4822 1.1107297304136e-08
4823 1.09529461056468e-08
4824 1.12782299055425e-08
4825 1.09647517732014e-08
4826 1.07820241623813e-08
4827 1.10276010545363e-08
4828 1.07492628131922e-08
4829 1.08232436346611e-08
4830 1.10070166314813e-08
4831 1.08049436065016e-08
4832 1.08898561279602e-08
4833 1.08974855805855e-08
4834 1.09912283718927e-08
4835 1.08451683189514e-08
4836 1.09388453850556e-08
4837 1.10764490912629e-08
4838 1.09101083722862e-08
4839 1.09399040937319e-08
4840 1.10080184967387e-08
4841 1.09668700787324e-08
4842 1.09497761968669e-08
4843 1.09220810173838e-08
4844 1.09742215315123e-08
4845 1.11900355648231e-08
4846 1.08779767415967e-08
4847 1.0929213090094e-08
4848 1.09698730099694e-08
4849 1.08282991462261e-08
4850 1.09249640445341e-08
4851 1.0896926028181e-08
4852 1.09713971241376e-08
4853 1.08821662792025e-08
4854 1.10415951937171e-08
4855 1.10847748757692e-08
4856 1.11326610152673e-08
4857 1.11058611196313e-08
4858 1.1134340560659e-08
4859 1.11590958695729e-08
4860 1.11205302744111e-08
4861 1.11571409888711e-08
4862 1.10430855571053e-08
4863 1.0996794586049e-08
4864 1.11165521232692e-08
4865 1.11459357299282e-08
4866 1.11082387732608e-08
4867 1.11478373199247e-08
4868 1.10556950261298e-08
4869 1.10500968375504e-08
4870 1.11606777153384e-08
4871 1.10263211894335e-08
4872 1.10838875855279e-08
4873 1.10158042687658e-08
4874 1.10968736422024e-08
4875 1.11078799491793e-08
4876 1.10702282896114e-08
4877 1.10623155080702e-08
4878 1.10049080959129e-08
4879 1.10718945123267e-08
4880 1.10328421953909e-08
4881 1.10459721369693e-08
4882 1.10672324638017e-08
4883 1.10643743056471e-08
4884 1.0981215048389e-08
4885 1.09617976917775e-08
4886 1.1045372616536e-08
4887 1.06937942945251e-08
4888 1.07527959869458e-08
4889 1.09805595727153e-08
4890 1.0997203148122e-08
4891 1.07648761016321e-08
4892 1.09174029816472e-08
4893 1.08520943342683e-08
4894 1.09843600881732e-08
4895 1.0938194350274e-08
4896 1.09676596693475e-08
4897 1.09342686016589e-08
4898 1.0969806396588e-08
4899 1.09968096850821e-08
4900 1.09461568698066e-08
4901 1.10015410115238e-08
4902 1.08364703876873e-08
4903 1.10151150423121e-08
4904 1.09745865728428e-08
4905 1.09852038576719e-08
4906 1.09373896606257e-08
4907 1.09915267998417e-08
4908 1.09368301082213e-08
4909 1.0938647321268e-08
4910 1.0912605041824e-08
4911 1.09238973422521e-08
4912 1.09288729177592e-08
4913 1.09247508817134e-08
4914 1.09175299911612e-08
4915 1.09245537061042e-08
4916 1.09801305825385e-08
4917 1.10175264467216e-08
4918 1.09989226615426e-08
4919 1.10180975454455e-08
4920 1.09353539556878e-08
4921 1.17048557513044e-08
4922 1.12425695419915e-08
4923 1.11865512408826e-08
4924 1.09570885697963e-08
4925 1.07892867973192e-08
4926 1.08663078535187e-08
4927 1.08470397108817e-08
4928 1.08693356537515e-08
4929 1.07661000114945e-08
4930 1.0778781422971e-08
4931 1.08136664067615e-08
4932 1.06405355637662e-08
4933 1.06275379607723e-08
4934 1.07634274826296e-08
4935 1.0790532911642e-08
4936 1.08499751405589e-08
4937 1.08563869005707e-08
4938 1.07631894508131e-08
4939 1.07877484722962e-08
4940 1.07985647090914e-08
4941 1.1151191969816e-08
4942 1.0697783103808e-08
4943 1.07158664164331e-08
4944 1.06733581972662e-08
4945 1.07704662966057e-08
4946 1.07003979010756e-08
4947 1.06414015377254e-08
4948 1.07394209081235e-08
4949 1.08602957737958e-08
4950 1.0791804783139e-08
4951 1.08337028237315e-08
4952 1.08174722512899e-08
4953 1.0678254724894e-08
4954 1.05391784188669e-08
4955 1.06533875054993e-08
4956 1.0715845100151e-08
4957 1.07628483903e-08
4958 1.0685826445922e-08
4959 1.07662669890374e-08
4960 1.03835935405527e-08
4961 1.05080424361859e-08
4962 1.06465467553107e-08
4963 1.07154010109412e-08
4964 1.04707007508864e-08
4965 1.06936699495463e-08
4966 1.07670414806194e-08
4967 1.08183595415312e-08
4968 1.07529585235966e-08
4969 1.04642428055968e-08
4970 1.06995399207221e-08
4971 1.08123030528873e-08
4972 1.06533839527856e-08
4973 1.0510474268699e-08
4974 1.07048414577093e-08
4975 1.04585069493623e-08
4976 1.11092770538335e-08
4977 1.08660112019265e-08
4978 1.07058513165725e-08
4979 1.06613127215383e-08
4980 1.06496029772529e-08
4981 1.07264437332333e-08
4982 1.07308437691245e-08
4983 1.06085709106196e-08
4984 1.07550963690528e-08
4985 1.07848627806106e-08
4986 1.07169979557398e-08
4987 1.09492637179187e-08
4988 1.08215223448838e-08
4989 1.08738635873351e-08
4990 1.07408322236324e-08
4991 1.07288897766011e-08
4992 1.081778133738e-08
4993 1.1047308845491e-08
4994 1.07598214782456e-08
4995 1.0765656810463e-08
4996 1.07223998568884e-08
4997 1.09964970462784e-08
4998 1.0715673681716e-08
4999 1.07647810665412e-08
};
\addlegendentry{Test}

\nextgroupplot[
title={Batch Size 4 $\rare$},
ymin=3.2014696107911e-09, ymax=1e-05,
]
\addplot [semithick, black, dashed]
table {%
0 0.00486928264575545
1 0.000350302332377396
2 0.000112899104227381
3 4.92373441177278e-05
4 2.95915884202032e-05
5 2.58721892537324e-05
6 2.30623341833507e-05
7 1.81885760139266e-05
8 1.31848863890411e-05
9 9.19836386411532e-06
10 6.61884140328084e-06
11 5.23575631805784e-06
12 4.47661068722738e-06
13 3.97060816662531e-06
14 3.56443101529891e-06
15 3.21384756439613e-06
16 2.79766197827769e-06
17 2.3323946371363e-06
18 2.07286921630256e-06
19 1.90984389664095e-06
20 1.79467490693597e-06
21 1.70759462266901e-06
22 1.63993165348586e-06
23 1.58436301196829e-06
24 1.53856174380351e-06
25 1.49583589962177e-06
26 1.45612416064722e-06
27 1.41963088946895e-06
28 1.38662386232014e-06
29 1.35637906121744e-06
30 1.32893618783214e-06
31 1.30357349408428e-06
32 1.27861436564913e-06
33 1.25542081098828e-06
34 1.23332833967993e-06
35 1.21277894584182e-06
36 1.19314392136971e-06
37 1.1731752795221e-06
38 1.15417504557769e-06
39 1.1354889641666e-06
40 1.11642244548626e-06
41 1.09781236666429e-06
42 1.0788499333767e-06
43 1.06024237514646e-06
44 1.04151938735164e-06
45 1.02357043085632e-06
46 1.00544299148631e-06
47 9.87574210883579e-07
48 9.69728331083886e-07
49 9.52232605847314e-07
50 9.34880836394925e-07
51 9.16788711256444e-07
52 8.99033485064038e-07
53 8.8160149435268e-07
54 8.64742913972094e-07
55 8.47911547339208e-07
56 8.31613815488907e-07
57 8.15690236123601e-07
58 8.00118873030442e-07
59 7.84902033000634e-07
60 7.70082775126468e-07
61 7.55871025372557e-07
62 7.42266747433717e-07
63 7.28959693812925e-07
64 7.16031757388436e-07
65 7.03374120714528e-07
66 6.91292855044878e-07
67 6.7983844368058e-07
68 6.68761314248201e-07
69 6.57842065084679e-07
70 6.46740577916916e-07
71 6.36769715487873e-07
72 6.27409500946641e-07
73 6.18641659475116e-07
74 6.10596213657999e-07
75 6.029399081946e-07
76 5.9537499446094e-07
77 5.88241557522906e-07
78 5.82021935649379e-07
79 5.76350524028157e-07
80 5.69763454308259e-07
81 5.64415524120676e-07
82 5.5969643297793e-07
83 5.55332608264791e-07
84 5.51518709066556e-07
85 5.47950638958739e-07
86 5.44411114470122e-07
87 5.41230120036929e-07
88 5.37922137466396e-07
89 5.35070778074242e-07
90 5.32327759815487e-07
91 5.2978967778472e-07
92 5.27304940034057e-07
93 5.24934913306296e-07
94 5.22744452414514e-07
95 5.20594670506469e-07
96 5.18565543405458e-07
97 5.16566021071085e-07
98 5.14629408788991e-07
99 5.12800865163143e-07
100 5.11070603058883e-07
101 5.0934343240705e-07
102 5.07618297840295e-07
103 5.06021174880189e-07
104 5.04427595014789e-07
105 5.02869565478292e-07
106 5.01358246795647e-07
107 4.99770406509725e-07
108 4.98429138756862e-07
109 4.96931511051457e-07
110 4.95490524810904e-07
111 4.94084306984277e-07
112 4.92625669824776e-07
113 4.91211894482646e-07
114 4.89838969066625e-07
115 4.88490026301136e-07
116 4.87132417218206e-07
117 4.85764320957927e-07
118 4.8435934881752e-07
119 4.83021097066327e-07
120 4.8175874966816e-07
121 4.80493870044896e-07
122 4.79151559524382e-07
123 4.77874232963416e-07
124 4.76768139426653e-07
125 4.7550159669818e-07
126 4.74172523981409e-07
127 4.72866406425965e-07
128 4.71550716447666e-07
129 4.70242109258834e-07
130 4.68982970250664e-07
131 4.67684803917123e-07
132 4.66406712517298e-07
133 4.65131236857985e-07
134 4.638463640525e-07
135 4.62560511847343e-07
136 4.61228673302116e-07
137 4.59907744533083e-07
138 4.58593424760068e-07
139 4.5722934129877e-07
140 4.55862428140108e-07
141 4.54423386585745e-07
142 4.52989150540262e-07
143 4.51597771466083e-07
144 4.50033753775614e-07
145 4.48605304500305e-07
146 4.47119185071898e-07
147 4.45555279226184e-07
148 4.43979357715563e-07
149 4.4239211766417e-07
150 4.40777705071227e-07
151 4.39064387149912e-07
152 4.37335975533415e-07
153 4.35509456826466e-07
154 4.33698422368067e-07
155 4.31787157339158e-07
156 4.29792286903208e-07
157 4.27787356384712e-07
158 4.25667300023136e-07
159 4.2353073810375e-07
160 4.21265811754168e-07
161 4.19002800425616e-07
162 4.16412755489759e-07
163 4.13966501287888e-07
164 4.11382122999626e-07
165 4.08500570463488e-07
166 4.05658512791618e-07
167 4.02629143756883e-07
168 3.99500009926612e-07
169 3.9612873879058e-07
170 3.92639631625613e-07
171 3.8905503575215e-07
172 3.85305524728707e-07
173 3.8139450649588e-07
174 3.7737101094848e-07
175 3.73166467166897e-07
176 3.68655906825666e-07
177 3.64077984702504e-07
178 3.59319700207195e-07
179 3.54376131893375e-07
180 3.49533657169943e-07
181 3.44436365956824e-07
182 3.39390576680643e-07
183 3.34318113797139e-07
184 3.29168043180061e-07
185 3.23892373520707e-07
186 3.1888533908031e-07
187 3.13877956541475e-07
188 3.08834816150849e-07
189 3.03891082303664e-07
190 2.99234509711255e-07
191 2.94752690111366e-07
192 2.90483530850594e-07
193 2.86449031943192e-07
194 2.82658962721172e-07
195 2.7910958165478e-07
196 2.75874005133403e-07
197 2.72804671453741e-07
198 2.70059167181813e-07
199 2.67527181998517e-07
200 2.65266616741755e-07
201 2.63246644149007e-07
202 2.61403034064855e-07
203 2.59843644178304e-07
204 2.5837366574244e-07
205 2.56949747200252e-07
206 2.55352170935197e-07
207 2.54063532608484e-07
208 2.528322887283e-07
209 2.51759960091746e-07
210 2.50631134777635e-07
211 2.49548728278981e-07
212 2.4857018661173e-07
213 2.47594738663537e-07
214 2.46718316256e-07
215 2.45915112794926e-07
216 2.45147778325183e-07
217 2.44317462531285e-07
218 2.43523754584807e-07
219 2.42738031666079e-07
220 2.41942099230563e-07
221 2.41171761917514e-07
222 2.40411356026371e-07
223 2.39698510036135e-07
224 2.39098284896322e-07
225 2.38508116078506e-07
226 2.3769243240146e-07
227 2.36974880409413e-07
228 2.36030367180895e-07
229 2.35508053394895e-07
230 2.34728520570826e-07
231 2.34155605117969e-07
232 2.33429535727581e-07
233 2.32832722351972e-07
234 2.32095596702564e-07
235 2.31324497016239e-07
236 2.30749924878815e-07
237 2.3007873926506e-07
238 2.29329782985843e-07
239 2.28685494506387e-07
240 2.27963940177212e-07
241 2.27253166844577e-07
242 2.26606089911208e-07
243 2.25963472038515e-07
244 2.25330820036618e-07
245 2.24754372485592e-07
246 2.2408015439801e-07
247 2.2343552334636e-07
248 2.22729544575806e-07
249 2.22115576911541e-07
250 2.21482560844422e-07
251 2.2083844536791e-07
252 2.20248329079631e-07
253 2.19653981724299e-07
254 2.1906902759472e-07
255 2.18499403785266e-07
256 2.17884342641916e-07
257 2.17316084996533e-07
258 2.16660151613368e-07
259 2.16062183331367e-07
260 2.15492559727082e-07
261 2.14895637125956e-07
262 2.143174627669e-07
263 2.13804952088559e-07
264 2.13242638746003e-07
265 2.12740001657608e-07
266 2.1213771204831e-07
267 2.11678450832586e-07
268 2.11128579218567e-07
269 2.10641309391413e-07
270 2.10007184509919e-07
271 2.09431549118122e-07
272 2.08879432545217e-07
273 2.08302704360896e-07
274 2.0779540653848e-07
275 2.07243015838188e-07
276 2.06741730146121e-07
277 2.06207968734162e-07
278 2.05653077589574e-07
279 2.05135657436628e-07
280 2.04632724810416e-07
281 2.04141893866527e-07
282 2.03645577040668e-07
283 2.03152510704463e-07
284 2.02594836878234e-07
285 2.02119696123937e-07
286 2.01518749075369e-07
287 2.01000610143609e-07
288 2.00514204596658e-07
289 2.00036968714734e-07
290 1.99529038900081e-07
291 1.9908152033743e-07
292 1.986493722681e-07
293 1.98117685925148e-07
294 1.97629430966906e-07
295 1.97163178368598e-07
296 1.96694588593616e-07
297 1.96203631370295e-07
298 1.95721600197096e-07
299 1.9529497334414e-07
300 1.94898602043914e-07
301 1.94448849307527e-07
302 1.9399486413807e-07
303 1.93469036807947e-07
304 1.93095240454966e-07
305 1.92599265446347e-07
306 1.92196664113631e-07
307 1.91659119226806e-07
308 1.9136045603485e-07
309 1.90893927825009e-07
310 1.90409718820561e-07
311 1.89968442025012e-07
312 1.89524329029744e-07
313 1.89119531466275e-07
314 1.88665820913592e-07
315 1.88229502883885e-07
316 1.87805592467782e-07
317 1.87564482550862e-07
318 1.8703287246602e-07
319 1.8665139510432e-07
320 1.86203741683322e-07
321 1.85743123540583e-07
322 1.85427741623556e-07
323 1.8508588946986e-07
324 1.84623793020755e-07
325 1.84258024505723e-07
326 1.83860369309841e-07
327 1.83419213565017e-07
328 1.83029493861397e-07
329 1.82652506374836e-07
330 1.82224966268407e-07
331 1.81837925927297e-07
332 1.81298180754297e-07
333 1.81030893134704e-07
334 1.8059832751316e-07
335 1.80174815233691e-07
336 1.79694117651508e-07
337 1.79412464642148e-07
338 1.78895239167254e-07
339 1.78628311384266e-07
340 1.78206150543758e-07
341 1.77847103284279e-07
342 1.7747329776796e-07
343 1.77064353409584e-07
344 1.76514895852797e-07
345 1.76271852151189e-07
346 1.75809587209308e-07
347 1.7549909772363e-07
348 1.74998525804604e-07
349 1.74723031046931e-07
350 1.7432631072456e-07
351 1.73968862506868e-07
352 1.7354035948447e-07
353 1.73174524765507e-07
354 1.72809457806622e-07
355 1.72568194109601e-07
356 1.72170920730963e-07
357 1.71711824959786e-07
358 1.71242221139067e-07
359 1.70855023834626e-07
360 1.70666092653704e-07
361 1.70249898513219e-07
362 1.69748669210712e-07
363 1.69596770411395e-07
364 1.69192290127462e-07
365 1.68752635068437e-07
366 1.68365989406283e-07
367 1.67985177403018e-07
368 1.6757252225208e-07
369 1.67191745984674e-07
370 1.668165722295e-07
371 1.66415648296336e-07
372 1.66030894126479e-07
373 1.6565979347849e-07
374 1.65275587570513e-07
375 1.64925893437307e-07
376 1.64493543477917e-07
377 1.64146919967578e-07
378 1.637826227876e-07
379 1.6329899836709e-07
380 1.62950787519378e-07
381 1.62607421534844e-07
382 1.62213210790796e-07
383 1.61887200520461e-07
384 1.6143898988652e-07
385 1.61107500267654e-07
386 1.60760886903866e-07
387 1.60298536920678e-07
388 1.59974030317755e-07
389 1.59452789503689e-07
390 1.59283187120884e-07
391 1.5889263915625e-07
392 1.58496816630382e-07
393 1.58138081167714e-07
394 1.57705274535402e-07
395 1.57551289961866e-07
396 1.57007128707143e-07
397 1.56642293985598e-07
398 1.56217521269042e-07
399 1.55807506895478e-07
400 1.55365375743344e-07
401 1.55054678817379e-07
402 1.54597957171987e-07
403 1.54253002598637e-07
404 1.53701983655274e-07
405 1.53264653822482e-07
406 1.52800066148373e-07
407 1.52506072921277e-07
408 1.52143444928754e-07
409 1.51678748011364e-07
410 1.51246484665357e-07
411 1.50847529249987e-07
412 1.50443685671853e-07
413 1.50031012156049e-07
414 1.49641716396864e-07
415 1.49217341846253e-07
416 1.48725271698957e-07
417 1.48353625932351e-07
418 1.47933224883268e-07
419 1.47333547807271e-07
420 1.47017269469529e-07
421 1.46574195043492e-07
422 1.46119966465186e-07
423 1.45854417000102e-07
424 1.45375496158451e-07
425 1.44967504621008e-07
426 1.44559285010004e-07
427 1.44035478843563e-07
428 1.4371563400406e-07
429 1.43252088013135e-07
430 1.4273886734717e-07
431 1.42331582940969e-07
432 1.41908849740524e-07
433 1.41470563384871e-07
434 1.41038971086083e-07
435 1.40580652771227e-07
436 1.40184319595171e-07
437 1.39712898155508e-07
438 1.39289071428905e-07
439 1.38843821553714e-07
440 1.38436886131466e-07
441 1.3801453646245e-07
442 1.37519894346738e-07
443 1.3691409079275e-07
444 1.36554112370568e-07
445 1.36286821131648e-07
446 1.35691009514716e-07
447 1.35178841614625e-07
448 1.34904146215753e-07
449 1.34402608498618e-07
450 1.3400619583237e-07
451 1.33663448342425e-07
452 1.33060426262155e-07
453 1.32757016465312e-07
454 1.32205418640119e-07
455 1.3174278048389e-07
456 1.31337010612853e-07
457 1.30896798030911e-07
458 1.30524573922486e-07
459 1.2983478482731e-07
460 1.29502884893995e-07
461 1.29041521899076e-07
462 1.28550170552799e-07
463 1.28224798724652e-07
464 1.27775214070702e-07
465 1.27316997038385e-07
466 1.26899270812331e-07
467 1.26418344668089e-07
468 1.25859960069974e-07
469 1.25441739755594e-07
470 1.25025031499248e-07
471 1.24579951041426e-07
472 1.240866164256e-07
473 1.23737525024481e-07
474 1.23217988348578e-07
475 1.22779755565183e-07
476 1.22247736693559e-07
477 1.21895132672911e-07
478 1.21550133198589e-07
479 1.21066930938163e-07
480 1.20570156934896e-07
481 1.20083683862404e-07
482 1.19639320788067e-07
483 1.19258683893619e-07
484 1.18866439752452e-07
485 1.18405519808462e-07
486 1.17940430818031e-07
487 1.17543426755518e-07
488 1.17079891924554e-07
489 1.16669969232319e-07
490 1.16171664836351e-07
491 1.15784260311003e-07
492 1.15373368853255e-07
493 1.14941578658101e-07
494 1.14503417810941e-07
495 1.1405303417078e-07
496 1.13634485306324e-07
497 1.13196805790583e-07
498 1.12758465984353e-07
499 1.12349958923286e-07
500 1.11901007949111e-07
501 1.11436191712944e-07
502 1.11015331699438e-07
503 1.10625269084785e-07
504 1.10171516162882e-07
505 1.09769248083591e-07
506 1.09347897257805e-07
507 1.08948607173254e-07
508 1.08505179081675e-07
509 1.08185940739691e-07
510 1.07726001954411e-07
511 1.07325022284055e-07
512 1.0693701997555e-07
513 1.06581990888532e-07
514 1.06053476929713e-07
515 1.05540240943558e-07
516 1.05232578594627e-07
517 1.04774660637741e-07
518 1.04303369579561e-07
519 1.03938072337773e-07
520 1.03547188116782e-07
521 1.03078788660227e-07
522 1.02633600863822e-07
523 1.02340933975764e-07
524 1.02002931096745e-07
525 1.01597081988203e-07
526 1.01260212386212e-07
527 1.00859086867189e-07
528 1.00504079348696e-07
529 1.00142996699137e-07
530 9.97067500914106e-08
531 9.93925517440708e-08
532 9.8809195760996e-08
533 9.84943954591699e-08
534 9.8241455833481e-08
535 9.78641130733315e-08
536 9.75123699702429e-08
537 9.71348680391948e-08
538 9.67754558702794e-08
539 9.63093263952608e-08
540 9.59425862401275e-08
541 9.5643732767936e-08
542 9.54148839746871e-08
543 9.51305131677849e-08
544 9.46572312128247e-08
545 9.42274927671072e-08
546 9.41419347726402e-08
547 9.3747136037603e-08
548 9.33775007401216e-08
549 9.30384080755076e-08
550 9.26622015278156e-08
551 9.23233684768654e-08
552 9.21031602780786e-08
553 9.16107693886836e-08
554 9.13629893855017e-08
555 9.09935877011492e-08
556 9.07569359540616e-08
557 9.04642636756492e-08
558 9.01058276898681e-08
559 8.98100773132349e-08
560 8.95268519496284e-08
561 8.91128803282015e-08
562 8.88231804618833e-08
563 8.84896423110071e-08
564 8.82651483684604e-08
565 8.79097643879945e-08
566 8.76364422990328e-08
567 8.7275960429789e-08
568 8.69913222953933e-08
569 8.66906349767938e-08
570 8.6472576619645e-08
571 8.60509626274819e-08
572 8.5711851539827e-08
573 8.55016508265471e-08
574 8.52492806937555e-08
575 8.48945221814112e-08
576 8.46147522861607e-08
577 8.42961679796694e-08
578 8.40456449036076e-08
579 8.37152144028686e-08
580 8.34370295126341e-08
581 8.31648471102397e-08
582 8.28156173811045e-08
583 8.25937627872975e-08
584 8.2348127370846e-08
585 8.20489235282551e-08
586 8.18392544239543e-08
587 8.15956032469423e-08
588 8.12944979409203e-08
589 8.09497048890329e-08
590 8.06980258825085e-08
591 8.05358484599061e-08
592 8.01999269457454e-08
593 7.99880833328714e-08
594 7.97004497234965e-08
595 7.9623193768974e-08
596 7.91518006648673e-08
597 7.9014864248439e-08
598 7.87192581284124e-08
599 7.84505882838182e-08
600 7.81809798819921e-08
601 7.80375351947527e-08
602 7.77022386468928e-08
603 7.75376682913276e-08
604 7.72058366438344e-08
605 7.70324738685879e-08
606 7.67674897472759e-08
607 7.66424876550431e-08
608 7.63108704844306e-08
609 7.61668927147241e-08
610 7.59401792413961e-08
611 7.57523447747488e-08
612 7.54287875071036e-08
613 7.52654689013532e-08
614 7.5128164240823e-08
615 7.4835329387124e-08
616 7.46445140840102e-08
617 7.4396397296006e-08
618 7.42470429928588e-08
619 7.42373911282179e-08
620 7.37251354445689e-08
621 7.37510510173678e-08
622 7.32852199156753e-08
623 7.33992837038677e-08
624 7.304790291629e-08
625 7.28642194260942e-08
626 7.26349420969719e-08
627 7.26347480970446e-08
628 7.22298741893113e-08
629 7.21364442508587e-08
630 7.19327082909782e-08
631 7.17430949936926e-08
632 7.1472644795012e-08
633 7.13277439010618e-08
634 7.10014252955915e-08
635 7.10757075539359e-08
636 7.079794177689e-08
637 7.06474505214327e-08
638 7.04265655828706e-08
639 7.01858057845506e-08
640 6.99883778965127e-08
641 6.99730740973958e-08
642 6.96544864333681e-08
643 6.95446782339459e-08
644 6.94121435229622e-08
645 6.92176125951516e-08
646 6.9106990585377e-08
647 6.89104590736456e-08
648 6.87518380986241e-08
649 6.87161599335351e-08
650 6.83989474490332e-08
651 6.82471215391889e-08
652 6.80566649799985e-08
653 6.80284873832981e-08
654 6.7573323845771e-08
655 6.76426803951635e-08
656 6.74395424598018e-08
657 6.7196328862984e-08
658 6.70377822298285e-08
659 6.69812432518846e-08
660 6.67405032799273e-08
661 6.66030201523782e-08
662 6.63621907821543e-08
663 6.63919383616651e-08
664 6.61702763418148e-08
665 6.60650535442286e-08
666 6.57866410569419e-08
667 6.5733904003995e-08
668 6.56131581138197e-08
669 6.53032756567917e-08
670 6.51916280771658e-08
671 6.508007550865e-08
672 6.48974965353766e-08
673 6.46660154921541e-08
674 6.46251129903241e-08
675 6.44519284311862e-08
676 6.42735072506895e-08
677 6.41280079478612e-08
678 6.39421985422217e-08
679 6.4037499849956e-08
680 6.37210684906186e-08
681 6.3605817879786e-08
682 6.3332328223531e-08
683 6.32715077539814e-08
684 6.31199333098742e-08
685 6.29391041484517e-08
686 6.27877328702731e-08
687 6.26351011194792e-08
688 6.24855446997863e-08
689 6.23685253486883e-08
690 6.21369429358864e-08
691 6.20785595915052e-08
692 6.18195813237854e-08
693 6.16463076923424e-08
694 6.15133427119829e-08
695 6.12913308732921e-08
696 6.11983667497817e-08
697 6.09597369050086e-08
698 6.08525416430083e-08
699 6.06952620327839e-08
700 6.06292700715549e-08
701 6.0467461503233e-08
702 6.02232797279889e-08
703 6.00774583250363e-08
704 5.99802192309262e-08
705 5.99138987555925e-08
706 5.96529776579757e-08
707 5.94439599814756e-08
708 5.93488789295549e-08
709 5.91191249483813e-08
710 5.90509369944314e-08
711 5.8764360014063e-08
712 5.87102516429461e-08
713 5.84830820549165e-08
714 5.82001496360895e-08
715 5.80379002044218e-08
716 5.78686199750322e-08
717 5.76828642224747e-08
718 5.7519342471668e-08
719 5.73357202044988e-08
720 5.71690256290225e-08
721 5.7051607858849e-08
722 5.68496983737177e-08
723 5.66367383028066e-08
724 5.6469285955707e-08
725 5.63005654159454e-08
726 5.58945011581713e-08
727 5.58659372664749e-08
728 5.57107605767548e-08
729 5.54813185784653e-08
730 5.53107776513073e-08
731 5.51034107010295e-08
732 5.49890983365664e-08
733 5.47299373030441e-08
734 5.47062981999247e-08
735 5.43087927971087e-08
736 5.42005763226605e-08
737 5.40573729033156e-08
738 5.38716019318208e-08
739 5.37200165950935e-08
740 5.35698398493878e-08
741 5.33542734593162e-08
742 5.33181842818919e-08
743 5.29934684685429e-08
744 5.28895062608115e-08
745 5.2659328385829e-08
746 5.26506201001808e-08
747 5.23340035916142e-08
748 5.23407040828872e-08
749 5.20224998945551e-08
750 5.198102876669e-08
751 5.18328061573925e-08
752 5.1706441485333e-08
753 5.15110802226104e-08
754 5.13741665399969e-08
755 5.11474913991172e-08
756 5.10646442761775e-08
757 5.09990725854337e-08
758 5.07952090034181e-08
759 5.06957665926677e-08
760 5.04987859129002e-08
761 5.02071280825689e-08
762 5.02014579684129e-08
763 5.00300075980675e-08
764 4.9891479202202e-08
765 4.97476606282632e-08
766 4.96430679621795e-08
767 4.95959192741147e-08
768 4.93552453524337e-08
769 4.92283292756746e-08
770 4.90314534395964e-08
771 4.88806763461191e-08
772 4.8893050674792e-08
773 4.86641427905887e-08
774 4.8503306516956e-08
775 4.84056896397078e-08
776 4.82029189154964e-08
777 4.80327825136406e-08
778 4.81864530952336e-08
779 4.80931294339904e-08
780 4.80780515670354e-08
781 4.79351775575232e-08
782 4.77244248640041e-08
783 4.77772638634022e-08
784 4.75343879910994e-08
785 4.74205797496019e-08
786 4.71146356066932e-08
787 4.6978397190145e-08
788 4.69604634192322e-08
789 4.67145638141098e-08
790 4.67702726418828e-08
791 4.65112772567444e-08
792 4.65137779435132e-08
793 4.6340095665176e-08
794 4.63601666136348e-08
795 4.62391020206176e-08
796 4.61007684635639e-08
797 4.59572329223334e-08
798 4.57476748798591e-08
799 4.56002378967924e-08
800 4.56212873056749e-08
801 4.55703488575487e-08
802 4.54591962908424e-08
803 4.53404875258023e-08
804 4.52173996445815e-08
805 4.51095457112416e-08
806 4.5040188180856e-08
807 4.49133125250967e-08
808 4.47971788939405e-08
809 4.4689219532934e-08
810 4.45204215107076e-08
811 4.44015256415575e-08
812 4.4273178080978e-08
813 4.42030602290089e-08
814 4.42646110512257e-08
815 4.40590288222076e-08
816 4.40811206448011e-08
817 4.38526203057066e-08
818 4.38724336953022e-08
819 4.37269205191271e-08
820 4.36528602545394e-08
821 4.35661914628405e-08
822 4.34768874786329e-08
823 4.33669310875295e-08
824 4.32832922840909e-08
825 4.31412113730545e-08
826 4.295238615315e-08
827 4.2959445216928e-08
828 4.2836005751834e-08
829 4.28940601193695e-08
830 4.27452369047288e-08
831 4.2728019705951e-08
832 4.25904807364041e-08
833 4.24807084011825e-08
834 4.24233180007505e-08
835 4.24186836003315e-08
836 4.22110019475674e-08
837 4.21635790881503e-08
838 4.2145717749964e-08
839 4.18376807251875e-08
840 4.19747164908113e-08
841 4.18098988010929e-08
842 4.18925289054073e-08
843 4.17786855577695e-08
844 4.15438888676789e-08
845 4.16874891153718e-08
846 4.16398012617059e-08
847 4.14498297125476e-08
848 4.14658361633791e-08
849 4.1408419811928e-08
850 4.12969678316166e-08
851 4.11286127846022e-08
852 4.12115970582771e-08
853 4.09247455082085e-08
854 4.09231375835262e-08
855 4.09103592584614e-08
856 4.07641736832609e-08
857 4.08518073657937e-08
858 4.07248035094732e-08
859 4.0687652782756e-08
860 4.06042426002173e-08
861 4.04954579267258e-08
862 4.04663198487754e-08
863 4.03850417254326e-08
864 4.0374153827516e-08
865 4.02985160150582e-08
866 4.02291788599296e-08
867 4.01799022116123e-08
868 4.00456513383052e-08
869 4.0069836553025e-08
870 3.98978265023775e-08
871 3.98939461789816e-08
872 3.98716514335984e-08
873 3.98007784103704e-08
874 3.96832114282208e-08
875 3.97224607959767e-08
876 3.94621682004992e-08
877 3.94710190312342e-08
878 3.94724599268592e-08
879 3.93471788810995e-08
880 3.93096657393954e-08
881 3.91851050520753e-08
882 3.91218806794491e-08
883 3.90816342161759e-08
884 3.9110660739361e-08
885 3.89789307420063e-08
886 3.89097073156819e-08
887 3.88760793714749e-08
888 3.88416146410719e-08
889 3.88294734594563e-08
890 3.85374385931669e-08
891 3.85907936684138e-08
892 3.848380212923e-08
893 3.84728094415809e-08
894 3.84110760125456e-08
895 3.84017305750106e-08
896 3.8414860605851e-08
897 3.82664554892109e-08
898 3.82565173382421e-08
899 3.83362052367442e-08
900 3.803603682484e-08
901 3.82114997190541e-08
902 3.80546493632394e-08
903 3.80228942138405e-08
904 3.79180901055598e-08
905 3.79547694804305e-08
906 3.79687372700221e-08
907 3.77856651034847e-08
908 3.7838493908593e-08
909 3.77898371060059e-08
910 3.76842202813865e-08
911 3.77307766754598e-08
912 3.75638375929022e-08
913 3.76010396883064e-08
914 3.7545538183581e-08
915 3.75129799246032e-08
916 3.75557299230334e-08
917 3.75208549985295e-08
918 3.73482513444934e-08
919 3.7332622816022e-08
920 3.73691007933274e-08
921 3.72714881107239e-08
922 3.72040941492013e-08
923 3.71390937214944e-08
924 3.71484648620246e-08
925 3.70715555613188e-08
926 3.69701617175533e-08
927 3.69799789319281e-08
928 3.69921059022449e-08
929 3.69386272387473e-08
930 3.68318831953651e-08
931 3.68130612795614e-08
932 3.67938662615419e-08
933 3.67245253005688e-08
934 3.66858015556915e-08
935 3.66837242720131e-08
936 3.65930792021185e-08
937 3.65943939681923e-08
938 3.65737538803312e-08
939 3.65744999328843e-08
940 3.6483820296862e-08
941 3.6374838057851e-08
942 3.63558936853359e-08
943 3.64409315538605e-08
944 3.62755855443098e-08
945 3.6154443470604e-08
946 3.62699794222987e-08
947 3.61620877968871e-08
948 3.60739759892592e-08
949 3.61024951520683e-08
950 3.60922962765464e-08
951 3.59835895853422e-08
952 3.59452737725974e-08
953 3.60188997752253e-08
954 3.58096597956603e-08
955 3.58370294282206e-08
956 3.57051049730828e-08
957 3.5781208129082e-08
958 3.57690093322205e-08
959 3.56582549343454e-08
960 3.55929825863743e-08
961 3.563327067857e-08
962 3.5519742493717e-08
963 3.55721068667769e-08
964 3.52910360090775e-08
965 3.54369383699371e-08
966 3.54298008848897e-08
967 3.53048037260439e-08
968 3.53303382540915e-08
969 3.53736509202118e-08
970 3.52646468904094e-08
971 3.51453236964172e-08
972 3.51399570799327e-08
973 3.52133483423334e-08
974 3.50978684560133e-08
975 3.50919909444247e-08
976 3.50662311416894e-08
977 3.50469447483182e-08
978 3.49577872131857e-08
979 3.49179059448979e-08
980 3.49559843053626e-08
981 3.47747632354345e-08
982 3.48906772462687e-08
983 3.47507076996489e-08
984 3.48491926653871e-08
985 3.46549600067636e-08
986 3.48741567552935e-08
987 3.46601675520475e-08
988 3.44383614763188e-08
989 3.47041420418748e-08
990 3.44843077744184e-08
991 3.46088835299074e-08
992 3.44429281071701e-08
993 3.44559599769756e-08
994 3.44425943774596e-08
995 3.43475023683038e-08
996 3.4365315195739e-08
997 3.427031893799e-08
998 3.42936569068808e-08
999 3.42346833976936e-08
1000 3.41510599668471e-08
1001 3.4206762521638e-08
1002 3.40839003591942e-08
1003 3.40237495475293e-08
1004 3.41261593108655e-08
1005 3.39552331927173e-08
1006 3.39974889722594e-08
1007 3.39643058016037e-08
1008 3.37831585150994e-08
1009 3.38690364685146e-08
1010 3.38402029029794e-08
1011 3.3835349895428e-08
1012 3.37991853793795e-08
1013 3.38365567873478e-08
1014 3.35355296545714e-08
1015 3.36893533747773e-08
1016 3.3619091527104e-08
1017 3.36406058953909e-08
1018 3.36031955008309e-08
1019 3.35925188004182e-08
1020 3.35568345770643e-08
1021 3.35863308509321e-08
1022 3.3575291110699e-08
1023 3.35246441640624e-08
1024 3.34339765315495e-08
1025 3.33473399927176e-08
1026 3.33672245236283e-08
1027 3.33040763897952e-08
1028 3.32595972507566e-08
1029 3.32690312370154e-08
1030 3.32473990867221e-08
1031 3.3034711617308e-08
1032 3.33090121655832e-08
1033 3.30398067240889e-08
1034 3.31549429994915e-08
1035 3.30423049319517e-08
1036 3.3075959241069e-08
1037 3.28627315795416e-08
1038 3.29539954904146e-08
1039 3.27996382536444e-08
1040 3.28009125400985e-08
1041 3.28646044127634e-08
1042 3.26890079458408e-08
1043 3.27011909093677e-08
1044 3.27765900471899e-08
1045 3.27460171103411e-08
1046 3.26075229515332e-08
1047 3.25443273192505e-08
1048 3.25981282227605e-08
1049 3.24898638695004e-08
1050 3.25412653550128e-08
1051 3.24630834769302e-08
1052 3.24339710451316e-08
1053 3.24093587751317e-08
1054 3.23810939542035e-08
1055 3.22858885792865e-08
1056 3.22612496199692e-08
1057 3.21882965481635e-08
1058 3.22229741680502e-08
1059 3.20858763722676e-08
1060 3.21172712267614e-08
1061 3.21363750641623e-08
1062 3.19717914081252e-08
1063 3.20278643013427e-08
1064 3.205671236306e-08
1065 3.1898752517856e-08
1066 3.19291525201493e-08
1067 3.18389496132276e-08
1068 3.18015853382203e-08
1069 3.17593459266963e-08
1070 3.1768674417787e-08
1071 3.16501642263489e-08
1072 3.16521665399971e-08
1073 3.17010441091004e-08
1074 3.16980698329417e-08
1075 3.16614309292529e-08
1076 3.14834660068986e-08
1077 3.15264426782313e-08
1078 3.14563054391037e-08
1079 3.1464544944626e-08
1080 3.15132817274577e-08
1081 3.13296150364906e-08
1082 3.13629828778694e-08
1083 3.14305842732399e-08
1084 3.128294842214e-08
1085 3.12108803759426e-08
1086 3.13196139907213e-08
1087 3.11416546907584e-08
1088 3.10921847981316e-08
1089 3.12172487351159e-08
1090 3.11078798453934e-08
1091 3.10827853102769e-08
1092 3.099276820695e-08
1093 3.10448014910758e-08
1094 3.10201908746421e-08
1095 3.09677646848083e-08
1096 3.10071516780308e-08
1097 3.09203788515511e-08
1098 3.0897133686647e-08
1099 3.08326222870914e-08
1100 3.08929372645483e-08
1101 3.07741319753863e-08
1102 3.07743112621939e-08
1103 3.0658292102359e-08
1104 3.06419309050732e-08
1105 3.07155669452275e-08
1106 3.05904924644551e-08
1107 3.05361916057567e-08
1108 3.05410542282836e-08
1109 3.05086558103351e-08
1110 3.04696778827829e-08
1111 3.04987204091667e-08
1112 3.03478893701747e-08
1113 3.04059520941369e-08
1114 3.03686496048172e-08
1115 3.02852553577537e-08
1116 3.03155511360842e-08
1117 3.03149888523091e-08
1118 3.03096363851729e-08
1119 3.02719871818624e-08
1120 3.02518269450491e-08
1121 3.01708523446909e-08
1122 3.00577273495417e-08
1123 3.01499492547652e-08
1124 3.00446851662084e-08
1125 3.02068794582055e-08
1126 3.00043192251742e-08
1127 2.99348810117728e-08
1128 2.99712109244066e-08
1129 2.99780068400146e-08
1130 2.98569245642533e-08
1131 2.9978415413412e-08
1132 2.98284436734964e-08
1133 2.97259776551595e-08
1134 2.98227201391832e-08
1135 2.97707906089517e-08
1136 2.97512858149496e-08
1137 2.96904672587583e-08
1138 2.97316578905527e-08
1139 2.9733319817371e-08
1140 2.9537605520602e-08
1141 2.96742595271571e-08
1142 2.95638020224587e-08
1143 2.95894863180379e-08
1144 2.95487075285816e-08
1145 2.9509946586348e-08
1146 2.94668183311586e-08
1147 2.94905266102763e-08
1148 2.93638329151857e-08
1149 2.93173780590017e-08
1150 2.93008286125218e-08
1151 2.93044312069668e-08
1152 2.92637827328246e-08
1153 2.92381577955414e-08
1154 2.91342719305199e-08
1155 2.92136726824932e-08
1156 2.90941411493772e-08
1157 2.92054115702545e-08
1158 2.89858638879981e-08
1159 2.91545504105262e-08
1160 2.9164449862451e-08
1161 2.89519257485882e-08
1162 2.89733068750131e-08
1163 2.89410535800894e-08
1164 2.89594609896859e-08
1165 2.88730730955145e-08
1166 2.88980695204444e-08
1167 2.88377979864585e-08
1168 2.87895822275175e-08
1169 2.88259339678021e-08
1170 2.87770308553537e-08
1171 2.88801563943508e-08
1172 2.88047763974397e-08
1173 2.86855771534089e-08
1174 2.87029059806709e-08
1175 2.86951151122761e-08
1176 2.86126594679637e-08
1177 2.8678156759776e-08
1178 2.86142081478147e-08
1179 2.86252237027274e-08
1180 2.85023615564928e-08
1181 2.85801745698588e-08
1182 2.84517991251398e-08
1183 2.84790658553469e-08
1184 2.84641571071109e-08
1185 2.84210409566832e-08
1186 2.84092041571427e-08
1187 2.8431483710678e-08
1188 2.83597863945939e-08
1189 2.83290374947232e-08
1190 2.8262411261526e-08
1191 2.82316532582705e-08
1192 2.82902013932773e-08
1193 2.81266377759604e-08
1194 2.81803386814605e-08
1195 2.82268250024753e-08
1196 2.81036395999301e-08
1197 2.799171723189e-08
1198 2.8100160887945e-08
1199 2.79566470053894e-08
1200 2.80439731363824e-08
1201 2.80584257206229e-08
1202 2.79591926009015e-08
1203 2.79188154745924e-08
1204 2.78984723061093e-08
1205 2.78538394893335e-08
1206 2.78309333912041e-08
1207 2.7790968072372e-08
1208 2.78413277801093e-08
1209 2.77928154874818e-08
1210 2.7684885306245e-08
1211 2.78246245528635e-08
1212 2.76733528390416e-08
1213 2.76244779899848e-08
1214 2.76030318147047e-08
1215 2.76272296220181e-08
1216 2.77426890085142e-08
1217 2.7746438752585e-08
1218 2.75692316329401e-08
1219 2.76093195186178e-08
1220 2.75533542783979e-08
1221 2.76059149821872e-08
1222 2.75580918693841e-08
1223 2.75265549802839e-08
1224 2.74039909198276e-08
1225 2.75041485477967e-08
1226 2.74158204478514e-08
1227 2.74204466537142e-08
1228 2.74304719216634e-08
1229 2.73923238354179e-08
1230 2.72761359900731e-08
1231 2.7314310945048e-08
1232 2.72814025668033e-08
1233 2.74127333534935e-08
1234 2.71858379523593e-08
1235 2.728386080042e-08
1236 2.72426670051562e-08
1237 2.71489408498748e-08
1238 2.70743958183939e-08
1239 2.70213966981903e-08
1240 2.7106135567001e-08
1241 2.70803380491014e-08
1242 2.69852984671193e-08
1243 2.69867751734321e-08
1244 2.69525619445155e-08
1245 2.69878323604367e-08
1246 2.7008909714743e-08
1247 2.69117866551483e-08
1248 2.67988697231214e-08
1249 2.68888890473384e-08
1250 2.67938691937974e-08
1251 2.68196137669374e-08
1252 2.67166621183979e-08
1253 2.68597509784918e-08
1254 2.67949456487226e-08
1255 2.67409885139625e-08
1256 2.65965457661999e-08
1257 2.67482816171949e-08
1258 2.66277082421329e-08
1259 2.66190206190675e-08
1260 2.65257870109004e-08
1261 2.66999173463844e-08
1262 2.66855951491785e-08
1263 2.63767610879651e-08
1264 2.65026532233481e-08
1265 2.65595534101326e-08
1266 2.64789049568837e-08
1267 2.65791981610697e-08
1268 2.63272813334492e-08
1269 2.63875124517643e-08
1270 2.63965429585511e-08
1271 2.61694157965842e-08
1272 2.64390822721783e-08
1273 2.63079476945371e-08
1274 2.62992858401923e-08
1275 2.63518429144405e-08
1276 2.62376432640377e-08
1277 2.61193032518037e-08
1278 2.61473376446109e-08
1279 2.61737059665812e-08
1280 2.60395028448546e-08
1281 2.6205738545837e-08
1282 2.60912234129052e-08
1283 2.60750156830802e-08
1284 2.59780574698443e-08
1285 2.59517656127084e-08
1286 2.59566535441014e-08
1287 2.59144413199408e-08
1288 2.60654470609634e-08
1289 2.58282520138042e-08
1290 2.59249676495266e-08
1291 2.58729737836472e-08
1292 2.57684588553797e-08
1293 2.57625621917779e-08
1294 2.59584278912151e-08
1295 2.56563196538995e-08
1296 2.57427778220798e-08
1297 2.56141928760556e-08
1298 2.56624729324528e-08
1299 2.56994277705314e-08
1300 2.57970220385317e-08
1301 2.55198419958358e-08
1302 2.57622968307114e-08
1303 2.55421692794133e-08
1304 2.554613891248e-08
1305 2.5631566099138e-08
1306 2.56164250354018e-08
1307 2.55406750049314e-08
1308 2.5571789679546e-08
1309 2.5518351186804e-08
1310 2.55339635852891e-08
1311 2.54609531078476e-08
1312 2.54247606450431e-08
1313 2.53672977699182e-08
1314 2.52846483936864e-08
1315 2.53057642909393e-08
1316 2.55092355776476e-08
1317 2.51617287410077e-08
1318 2.54298866626534e-08
1319 2.51244750828494e-08
1320 2.54002517798835e-08
1321 2.50574497013822e-08
1322 2.52803273601554e-08
1323 2.50831306836119e-08
1324 2.51809996001384e-08
1325 2.52061338503751e-08
1326 2.51360724041749e-08
1327 2.50026984491836e-08
1328 2.50740132463623e-08
1329 2.5063092250921e-08
1330 2.47948256841823e-08
1331 2.49944485648523e-08
1332 2.50297154691381e-08
1333 2.50926668727836e-08
1334 2.47683031402435e-08
1335 2.48054384577312e-08
1336 2.49934615372815e-08
1337 2.47909264563528e-08
1338 2.48494603811711e-08
1339 2.51004152345136e-08
1340 2.46119608791062e-08
1341 2.48639560518349e-08
1342 2.46492825477507e-08
1343 2.4631140760123e-08
1344 2.46967776480478e-08
1345 2.4722648032327e-08
1346 2.46976723290437e-08
1347 2.47001103839128e-08
1348 2.43590155288853e-08
1349 2.46534739776205e-08
1350 2.46942299915176e-08
1351 2.45596331720499e-08
1352 2.43730873934478e-08
1353 2.45287131872551e-08
1354 2.45881895291067e-08
1355 2.44361531442916e-08
1356 2.43767800873762e-08
1357 2.43523992227157e-08
1358 2.42704107176195e-08
1359 2.4439515162733e-08
1360 2.42620995702936e-08
1361 2.43977456009237e-08
1362 2.39512724731217e-08
1363 2.45043313722437e-08
1364 2.42216050663213e-08
1365 2.43118173972601e-08
1366 2.40723863478287e-08
1367 2.44414207151156e-08
1368 2.40090610299504e-08
1369 2.43350970452472e-08
1370 2.40306719883066e-08
1371 2.39969974786813e-08
1372 2.40551341363915e-08
1373 2.41248664609728e-08
1374 2.3913141375953e-08
1375 2.39796402019543e-08
1376 2.40638536135318e-08
1377 2.39892189637381e-08
1378 2.40778462052837e-08
1379 2.38657652085195e-08
1380 2.38616633976285e-08
1381 2.40231590611373e-08
1382 2.3770683490465e-08
1383 2.3945140265047e-08
1384 2.38241180434695e-08
1385 2.37183517211026e-08
1386 2.40310206176542e-08
1387 2.35833720811929e-08
1388 2.39524699789939e-08
1389 2.36142681127038e-08
1390 2.38287212004362e-08
1391 2.37229796737903e-08
1392 2.35258076177747e-08
1393 2.37591768703638e-08
1394 2.36864405804837e-08
1395 2.35686081351272e-08
1396 2.35779605977093e-08
1397 2.37688138629011e-08
1398 2.35455635773096e-08
1399 2.35537930168839e-08
1400 2.35627855496023e-08
1401 2.35525360869904e-08
1402 2.35209678021153e-08
1403 2.35402313129018e-08
1404 2.3367300583832e-08
1405 2.35075675170116e-08
1406 2.35329972921861e-08
1407 2.33367242634674e-08
1408 2.34683380897494e-08
1409 2.34637789258141e-08
1410 2.34559535114975e-08
1411 2.32725956905711e-08
1412 2.34463609187774e-08
1413 2.314003252879e-08
1414 2.33155453153167e-08
1415 2.33124041733035e-08
1416 2.32891585612016e-08
1417 2.35266043824289e-08
1418 2.34266025151975e-08
1419 2.32630839207548e-08
1420 2.33229027397641e-08
1421 2.32605360150906e-08
1422 2.32289312054323e-08
1423 2.34092411905973e-08
1424 2.33689047457464e-08
1425 2.34704362283011e-08
1426 2.31109815276298e-08
1427 2.34579097901921e-08
1428 2.33307424233331e-08
1429 2.32334165637393e-08
1430 2.31948279416638e-08
1431 2.33085266122535e-08
1432 2.30428383360959e-08
1433 2.31099726790696e-08
1434 2.31166487274503e-08
1435 2.31082839483587e-08
1436 2.32555529866563e-08
1437 2.31180274776754e-08
1438 2.29872870376457e-08
1439 2.30145203321541e-08
1440 2.32585559321041e-08
1441 2.28887259285937e-08
1442 2.31566849531184e-08
1443 2.3111103361062e-08
1444 2.29433707481519e-08
1445 2.30670104743869e-08
1446 2.28272363147486e-08
1447 2.30598605275301e-08
1448 2.32135172080206e-08
1449 2.27794038223372e-08
1450 2.28833533493233e-08
1451 2.28995265130205e-08
1452 2.28192016367146e-08
1453 2.30710485041863e-08
1454 2.2748563100361e-08
1455 2.29869738765931e-08
1456 2.27074571930164e-08
1457 2.25083388645242e-08
1458 2.3001914926768e-08
1459 2.28829666422126e-08
1460 2.2766488424697e-08
1461 2.25939828499833e-08
1462 2.2542097077638e-08
1463 2.27347583570126e-08
1464 2.26255675983555e-08
1465 2.29011440400217e-08
1466 2.25792347801068e-08
1467 2.28410804448131e-08
1468 2.25181362631943e-08
1469 2.23774865917292e-08
1470 2.25785788641186e-08
1471 2.26249670522982e-08
1472 2.25763085075492e-08
1473 2.25697206979181e-08
1474 2.25264820248317e-08
1475 2.25240643800806e-08
1476 2.25694339364146e-08
1477 2.25494381862479e-08
1478 2.24421797987429e-08
1479 2.25888529219453e-08
1480 2.24594848734494e-08
1481 2.22099804776388e-08
1482 2.2362406801868e-08
1483 2.23588498595717e-08
1484 2.2409279135438e-08
1485 2.24500938390548e-08
1486 2.2372508492019e-08
1487 2.21429693956221e-08
1488 2.2146498314779e-08
1489 2.28482915638484e-08
1490 2.21502911088578e-08
1491 2.21853336148659e-08
1492 2.22070967657029e-08
1493 2.21082032012543e-08
1494 2.21738842856034e-08
1495 2.2181952095135e-08
1496 2.21563974183159e-08
1497 2.23237816656674e-08
1498 2.20864689222111e-08
1499 2.22139736663074e-08
1500 2.20498001894942e-08
1501 2.2132071230363e-08
1502 2.20175502267939e-08
1503 2.22189715317622e-08
1504 2.23150985692033e-08
1505 2.18299870771155e-08
1506 2.20270861908123e-08
1507 2.20283282250655e-08
1508 2.21954670915903e-08
1509 2.18551380581911e-08
1510 2.18124281294862e-08
1511 2.19907783809425e-08
1512 2.20003774227262e-08
1513 2.19312017379902e-08
1514 2.18035843044806e-08
1515 2.20080713893012e-08
1516 2.1954385542311e-08
1517 2.20349817436905e-08
1518 2.17708784082671e-08
1519 2.18396470077487e-08
1520 2.17719952251283e-08
1521 2.19831031031958e-08
1522 2.17679380087876e-08
1523 2.18099334816646e-08
1524 2.17107812894302e-08
1525 2.16657814677834e-08
1526 2.19636309675142e-08
1527 2.17266010336115e-08
1528 2.15621040773328e-08
1529 2.16860464830049e-08
1530 2.16955572909239e-08
1531 2.17921856519032e-08
1532 2.17010569727449e-08
1533 2.1520933499275e-08
1534 2.1634944025628e-08
1535 2.1728842493518e-08
1536 2.15351329388991e-08
1537 2.15976227841974e-08
1538 2.16721124863817e-08
1539 2.15027708927451e-08
1540 2.14518821886056e-08
1541 2.15063420965222e-08
1542 2.14314254276537e-08
1543 2.15533333065565e-08
1544 2.13981188825763e-08
1545 2.13790400409586e-08
1546 2.16741716361213e-08
1547 2.13967878714971e-08
1548 2.13191234139654e-08
1549 2.14992052953722e-08
1550 2.12914447370594e-08
1551 2.12289078322936e-08
1552 2.09739416534571e-08
1553 2.11059788568502e-08
1554 2.10318915709884e-08
1555 2.1109242660744e-08
1556 2.08832856862795e-08
1557 2.09598620313223e-08
1558 2.09374835826726e-08
1559 2.08800725123304e-08
1560 2.09180076886639e-08
1561 2.09966523461169e-08
1562 2.07402581930971e-08
1563 2.07462493533317e-08
1564 2.09478761366011e-08
1565 2.09131386643691e-08
1566 2.07543737347216e-08
1567 2.09014710297328e-08
1568 2.07261037057371e-08
1569 2.07071762154953e-08
1570 2.05325146263036e-08
1571 2.08968214938654e-08
1572 2.05360499581086e-08
1573 2.06129800772725e-08
1574 2.06375154412353e-08
1575 2.06796023864975e-08
1576 2.0575869606354e-08
1577 2.05687933689802e-08
1578 2.05636346548577e-08
1579 2.07421351015036e-08
1580 2.07122184199271e-08
1581 2.05106068369432e-08
1582 2.04764053977069e-08
1583 2.06629790684421e-08
1584 2.0516532208692e-08
1585 2.05237371622147e-08
1586 2.02423784250172e-08
1587 2.04624736801495e-08
1588 2.05934661030582e-08
1589 2.03817209138979e-08
1590 2.03259710165327e-08
1591 2.05399316255406e-08
1592 2.03996931928963e-08
1593 2.02125051846913e-08
1594 2.04356841930764e-08
1595 2.01577630922767e-08
1596 2.03819550688156e-08
1597 2.02591025690868e-08
1598 2.02197509995283e-08
1599 2.01969681672409e-08
1600 2.0153282305202e-08
1601 2.03301373997711e-08
1602 2.00481428802579e-08
1603 2.01878981451742e-08
1604 2.0073333473869e-08
1605 2.01429696171651e-08
1606 2.03956966147167e-08
1607 2.00959328306993e-08
1608 2.00108011241262e-08
1609 1.98728171560969e-08
1610 2.02501339952033e-08
1611 1.97730569322374e-08
1612 1.9866981094907e-08
1613 1.95980760206993e-08
1614 1.99777181491001e-08
1615 2.00505464196254e-08
1616 2.00318149221612e-08
1617 1.95215669418491e-08
1618 1.9846997783679e-08
1619 1.9859504508446e-08
1620 1.97874537037901e-08
1621 1.97849804706163e-08
1622 1.97743420973229e-08
1623 1.96260620557442e-08
1624 1.96240892689215e-08
1625 1.95783767087399e-08
1626 1.97484445618734e-08
1627 1.97204118792538e-08
1628 1.92942759236203e-08
1629 1.96337561439996e-08
1630 1.98016882211327e-08
1631 1.96582247935506e-08
1632 1.95601447792182e-08
1633 1.96353725490095e-08
1634 1.98187375370917e-08
1635 1.95464950676349e-08
1636 1.94576272478653e-08
1637 1.98140204781261e-08
1638 1.93352056772067e-08
1639 1.94459766371668e-08
1640 1.94588532447249e-08
1641 1.93408856827837e-08
1642 1.95828800906295e-08
1643 1.92815268726942e-08
1644 1.95883449825018e-08
1645 1.91224191243844e-08
1646 1.9398677184812e-08
1647 1.94228901053695e-08
1648 1.93167228346169e-08
1649 1.96204960145607e-08
1650 1.91453508393558e-08
1651 1.91323550022826e-08
1652 1.95756127889801e-08
1653 1.90362325998006e-08
1654 1.92601476964871e-08
1655 1.92035814514657e-08
1656 1.93381501407686e-08
1657 1.90217073929855e-08
1658 1.9085082954895e-08
1659 1.90623900984122e-08
1660 1.91121958814655e-08
1661 1.92156169296354e-08
1662 1.89122012397291e-08
1663 1.92701745111989e-08
1664 1.90726237634387e-08
1665 1.92071681650585e-08
1666 1.89938573857962e-08
1667 1.88003436578654e-08
1668 1.89069728471214e-08
1669 1.92107008262221e-08
1670 1.90517660743073e-08
1671 1.87120528174489e-08
1672 1.89185896073418e-08
1673 1.89339196179361e-08
1674 1.90240059299018e-08
1675 1.88000865994908e-08
1676 1.91559399289609e-08
1677 1.91956304012031e-08
1678 1.90439697422828e-08
1679 1.88161664478326e-08
1680 1.89718277976425e-08
1681 1.86360430913801e-08
1682 1.87779760389972e-08
1683 1.89505470307161e-08
1684 1.88850722706579e-08
1685 1.86282049645659e-08
1686 1.87469064338863e-08
1687 1.89342417467042e-08
1688 1.88762447133062e-08
1689 1.90958584247447e-08
1690 1.89225392071002e-08
1691 1.88198876216461e-08
1692 1.8784403255534e-08
1693 1.86560577022377e-08
1694 1.87834276834664e-08
1695 1.90486503883758e-08
1696 1.87146432542029e-08
1697 1.86933367318787e-08
1698 1.87629763335195e-08
1699 1.86194823441621e-08
1700 1.84701603862525e-08
1701 1.89714468152857e-08
1702 1.90290671069349e-08
1703 1.83247069263093e-08
1704 1.86472369314572e-08
1705 1.90126289485448e-08
1706 1.8576032494888e-08
1707 1.83848881327719e-08
1708 1.88015298634436e-08
1709 1.84289164102402e-08
1710 1.8505331277785e-08
1711 1.86311799432737e-08
1712 1.83885691920072e-08
1713 1.84877514106674e-08
1714 1.85744019132184e-08
1715 1.85256653730814e-08
1716 1.84056185348336e-08
1717 1.83005048799423e-08
1718 1.85593150467955e-08
1719 1.85125962486321e-08
1720 1.81246684265446e-08
1721 1.82754453232903e-08
1722 1.83058625702337e-08
1723 1.8239788641683e-08
1724 1.81059325732313e-08
1725 1.85089281188322e-08
1726 1.82710534466501e-08
1727 1.84176992945595e-08
1728 1.84155593132562e-08
1729 1.81961120129248e-08
1730 1.81947675670369e-08
1731 1.8061558775484e-08
1732 1.81625632968974e-08
1733 1.81391559033184e-08
1734 1.80435099301768e-08
1735 1.80040817976401e-08
1736 1.80456990115374e-08
1737 1.80498566062504e-08
1738 1.79958704659144e-08
1739 1.80429441600793e-08
1740 1.79716562795917e-08
1741 1.7952118816611e-08
1742 1.79860533545684e-08
1743 1.794989952697e-08
1744 1.78997817402715e-08
1745 1.80734665811144e-08
1746 1.78515400041102e-08
1747 1.80240360121431e-08
1748 1.78833272062029e-08
1749 1.79726687310477e-08
1750 1.79923807479554e-08
1751 1.78742973790946e-08
1752 1.74691757355294e-08
1753 1.79098686483581e-08
1754 1.75743002372997e-08
1755 1.79224025245173e-08
1756 1.7519384436171e-08
1757 1.77504444341947e-08
1758 1.78569715371113e-08
1759 1.78387024869497e-08
1760 1.7719283757156e-08
1761 1.77636979077356e-08
1762 1.76260271496087e-08
1763 1.74763495299768e-08
1764 1.79717323564033e-08
1765 1.75075691133397e-08
1766 1.75535755160272e-08
1767 1.77693958152147e-08
1768 1.75692361374935e-08
1769 1.7723048172158e-08
1770 1.78269089022853e-08
1771 1.785061982984e-08
1772 1.75119589722605e-08
1773 1.78411728093408e-08
1774 1.7542610500576e-08
1775 1.76452945763739e-08
1776 1.75141132993106e-08
1777 1.74566870112525e-08
1778 1.73860756147448e-08
1779 1.76379575266283e-08
1780 1.74793753757729e-08
1781 1.76328300030004e-08
1782 1.75238546885614e-08
1783 1.7185187922486e-08
1784 1.73849327305131e-08
1785 1.74402415917818e-08
1786 1.76598234390424e-08
1787 1.7698651126663e-08
1788 1.74401670186564e-08
1789 1.73624861370403e-08
1790 1.73655190220945e-08
1791 1.74669215404766e-08
1792 1.72777070803765e-08
1793 1.72751163014517e-08
1794 1.72720618414335e-08
1795 1.72975204715264e-08
1796 1.72167005920709e-08
1797 1.71248912256283e-08
1798 1.70126727748743e-08
1799 1.70047745187141e-08
1800 1.68892786767127e-08
1801 1.7117972704872e-08
1802 1.67646589812609e-08
1803 1.71467767304812e-08
1804 1.68312750621347e-08
1805 1.71148771701546e-08
1806 1.67103957633419e-08
1807 1.70076672014741e-08
1808 1.70193240593974e-08
1809 1.67929818419843e-08
1810 1.70541171746752e-08
1811 1.6749487707246e-08
1812 1.69988971154833e-08
1813 1.67916585207539e-08
1814 1.72264907900566e-08
1815 1.66355806281437e-08
1816 1.70269622397079e-08
1817 1.68028464195835e-08
1818 1.69797355973333e-08
1819 1.68744866819726e-08
1820 1.6763924274743e-08
1821 1.68808459348746e-08
1822 1.68363571493302e-08
1823 1.67041592438189e-08
1824 1.68886356575193e-08
1825 1.68842352756826e-08
1826 1.67788964151594e-08
1827 1.6826572553752e-08
1828 1.66753780017626e-08
1829 1.66759248181325e-08
1830 1.66505147980445e-08
1831 1.66487173727248e-08
1832 1.66877887899375e-08
1833 1.66715579981158e-08
1834 1.66495091280527e-08
1835 1.66339911178426e-08
1836 1.66314569927106e-08
1837 1.66290177521233e-08
1838 1.66732888360333e-08
1839 1.65975304462496e-08
1840 1.65719136400178e-08
1841 1.67237756867333e-08
1842 1.63733969900104e-08
1843 1.67900233305485e-08
1844 1.65103791822663e-08
1845 1.66747769458908e-08
1846 1.65853289824103e-08
1847 1.66529135559035e-08
1848 1.66922795550306e-08
1849 1.6562266385467e-08
1850 1.66309818099286e-08
1851 1.66236001637454e-08
1852 1.65681517403371e-08
1853 1.66810143045204e-08
1854 1.64797571295949e-08
1855 1.6575064405e-08
1856 1.65312198456835e-08
1857 1.65729814081139e-08
1858 1.64355150154005e-08
1859 1.64524007486566e-08
1860 1.6710320545954e-08
1861 1.63966108852609e-08
1862 1.65218598839889e-08
1863 1.65987639604293e-08
1864 1.64567630018819e-08
1865 1.64100447921367e-08
1866 1.64889989129335e-08
1867 1.66069723630535e-08
1868 1.62963632950319e-08
1869 1.65504222324753e-08
1870 1.63664706500644e-08
1871 1.66762356699213e-08
1872 1.63512625519502e-08
1873 1.64588527951093e-08
1874 1.64982688957149e-08
1875 1.62941990360377e-08
1876 1.64322773634762e-08
1877 1.66005172814732e-08
1878 1.62840189465996e-08
1879 1.64847669017476e-08
1880 1.62682435491668e-08
1881 1.6371106726476e-08
1882 1.62636336005217e-08
1883 1.63566070214838e-08
1884 1.63133977579832e-08
1885 1.63759882556569e-08
1886 1.6322216807696e-08
1887 1.63267807551382e-08
1888 1.65416808830532e-08
1889 1.6234482757671e-08
1890 1.6314436024345e-08
1891 1.62356918798068e-08
1892 1.63133635959767e-08
1893 1.62697696997061e-08
1894 1.63943454527526e-08
1895 1.62101606531184e-08
1896 1.61840385255463e-08
1897 1.62592760688529e-08
1898 1.65063763395423e-08
1899 1.61389891857322e-08
1900 1.6181392328285e-08
1901 1.63975334481758e-08
1902 1.61010211279233e-08
1903 1.62739347668861e-08
1904 1.62051769301286e-08
1905 1.61994661862508e-08
1906 1.61154116729012e-08
1907 1.61510326304182e-08
1908 1.60794572146017e-08
1909 1.61460676910696e-08
1910 1.61365605571007e-08
1911 1.62084894175241e-08
1912 1.64909864854845e-08
1913 1.59863427069418e-08
1914 1.63179146868142e-08
1915 1.59333045723908e-08
1916 1.63431863073882e-08
1917 1.60875680907946e-08
1918 1.60509664710595e-08
1919 1.60278259107027e-08
1920 1.60695954032342e-08
1921 1.61797109419215e-08
1922 1.59902968457359e-08
1923 1.60813088423417e-08
1924 1.60519439378337e-08
1925 1.6029875481216e-08
1926 1.60933969062249e-08
1927 1.59801892558598e-08
1928 1.59728857680452e-08
1929 1.62057112262914e-08
1930 1.58781219348203e-08
1931 1.60390002847954e-08
1932 1.59950126532582e-08
1933 1.60569135758681e-08
1934 1.62145023119287e-08
1935 1.61138999361654e-08
1936 1.58320431749015e-08
1937 1.60660667394286e-08
1938 1.59667817707376e-08
1939 1.60593205902337e-08
1940 1.58487189529932e-08
1941 1.59265843493461e-08
1942 1.60715753843022e-08
1943 1.6069725732315e-08
1944 1.5951082299237e-08
1945 1.58490217911922e-08
1946 1.59033797233477e-08
1947 1.58616787109267e-08
1948 1.59154992267485e-08
1949 1.62385433488677e-08
1950 1.58089039421494e-08
1951 1.57499867818878e-08
1952 1.60706593881343e-08
1953 1.5833401183718e-08
1954 1.59705842459612e-08
1955 1.57125053890095e-08
1956 1.59794954057713e-08
1957 1.59069799432476e-08
1958 1.5872191806876e-08
1959 1.57125300010996e-08
1960 1.58338434155247e-08
1961 1.58657048443533e-08
1962 1.58194291572933e-08
1963 1.59290328605177e-08
1964 1.57018798763175e-08
1965 1.57371850022514e-08
1966 1.576235846934e-08
1967 1.58354606294431e-08
1968 1.59819982814557e-08
1969 1.57487126235534e-08
1970 1.58266590470912e-08
1971 1.59313947050155e-08
1972 1.56762076526373e-08
1973 1.60293097062336e-08
1974 1.56725053916951e-08
1975 1.57122786645925e-08
1976 1.56302734626301e-08
1977 1.58158211325254e-08
1978 1.57121492125878e-08
1979 1.56523790786345e-08
1980 1.58191951555864e-08
1981 1.58234649900724e-08
1982 1.5542420571224e-08
1983 1.57663330138114e-08
1984 1.58238202365713e-08
1985 1.56597953193671e-08
1986 1.57737778536671e-08
1987 1.55751427890483e-08
1988 1.58644156800225e-08
1989 1.55691516010581e-08
1990 1.56052808288898e-08
1991 1.56805624649259e-08
1992 1.57117552699315e-08
1993 1.54707966077972e-08
1994 1.56720159325552e-08
1995 1.56320991206993e-08
1996 1.58213014906927e-08
1997 1.54760501513973e-08
1998 1.57563015814599e-08
1999 1.5748248131775e-08
2000 1.55527633352026e-08
2001 1.55465119082709e-08
2002 1.56092949228359e-08
2003 1.57444284389907e-08
2004 1.5457369290095e-08
2005 1.55578888938468e-08
2006 1.5764521248629e-08
2007 1.55564756190163e-08
2008 1.5706088762002e-08
2009 1.54642634857538e-08
2010 1.54752887149279e-08
2011 1.56173848668839e-08
2012 1.56258020138633e-08
2013 1.53910138636082e-08
2014 1.56826518720177e-08
2015 1.5390212843025e-08
2016 1.58232341891384e-08
2017 1.54167165695718e-08
2018 1.56302338862879e-08
2019 1.54843433100016e-08
2020 1.56197887450915e-08
2021 1.53096893815707e-08
2022 1.56705657674561e-08
2023 1.54394779339206e-08
2024 1.5538242929658e-08
2025 1.53446692006121e-08
2026 1.5597708775239e-08
2027 1.54367705642589e-08
2028 1.53360655417156e-08
2029 1.53693335029903e-08
2030 1.53896348111804e-08
2031 1.55492527900147e-08
2032 1.54025620580089e-08
2033 1.5484794022802e-08
2034 1.53689262705203e-08
2035 1.57723142477639e-08
2036 1.55089048721191e-08
2037 1.52314945993659e-08
2038 1.55140806175069e-08
2039 1.52828274566641e-08
2040 1.55423737786542e-08
2041 1.53364815205226e-08
2042 1.53298195340046e-08
2043 1.54721175262829e-08
2044 1.54305250679165e-08
2045 1.53594591216777e-08
2046 1.54225187196655e-08
2047 1.53095994672725e-08
2048 1.55333004889791e-08
2049 1.53210138180704e-08
2050 1.54293463958677e-08
2051 1.53455407307934e-08
2052 1.52542682350099e-08
2053 1.53023604194669e-08
2054 1.54005255998957e-08
2055 1.53952223564957e-08
2056 1.51315312957312e-08
2057 1.54996779571093e-08
2058 1.50544311214862e-08
2059 1.5416044043759e-08
2060 1.51638924557052e-08
2061 1.54089244566169e-08
2062 1.50420671736207e-08
2063 1.53209964766088e-08
2064 1.53660921542453e-08
2065 1.51456958319418e-08
2066 1.53702447849291e-08
2067 1.50588643175187e-08
2068 1.53567877942962e-08
2069 1.52584628583252e-08
2070 1.51635643219628e-08
2071 1.50965366536582e-08
2072 1.55209059700123e-08
2073 1.51434820940821e-08
2074 1.52506580770595e-08
2075 1.51504517151846e-08
2076 1.53218586633752e-08
2077 1.51649623250094e-08
2078 1.51587939782871e-08
2079 1.51248319102404e-08
2080 1.53678355572229e-08
2081 1.5085052275321e-08
2082 1.52122803331878e-08
2083 1.52277622178509e-08
2084 1.53737454706704e-08
2085 1.53995079372748e-08
2086 1.50434689241141e-08
2087 1.53753615541596e-08
2088 1.50174397892666e-08
2089 1.51495351650155e-08
2090 1.52529382868583e-08
2091 1.49773978344037e-08
2092 1.53448986535132e-08
2093 1.49876064561294e-08
2094 1.52754104367769e-08
2095 1.52092268030923e-08
2096 1.503631707922e-08
2097 1.51909723813404e-08
2098 1.51260214737992e-08
2099 1.51001645494198e-08
2100 1.52292980633018e-08
2101 1.5047456742634e-08
2102 1.50512778867018e-08
2103 1.52652616314253e-08
2104 1.50679739694404e-08
2105 1.53788999821547e-08
2106 1.49026939026964e-08
2107 1.53270790426152e-08
2108 1.49525200630052e-08
2109 1.52480254520526e-08
2110 1.52206865167148e-08
2111 1.4986640472392e-08
2112 1.50648994030611e-08
2113 1.54148442081947e-08
2114 1.48870626370812e-08
2115 1.52250879930094e-08
2116 1.49446314183788e-08
2117 1.54842651893805e-08
2118 1.50140191006454e-08
2119 1.49414152874616e-08
2120 1.50644192147276e-08
2121 1.50306053989802e-08
2122 1.50073412663598e-08
2123 1.49161393951935e-08
2124 1.49697733509147e-08
2125 1.48944649809302e-08
2126 1.52184995132476e-08
2127 1.4843857840674e-08
2128 1.4968432819229e-08
2129 1.51189548651676e-08
2130 1.48963407531344e-08
2131 1.49162403533243e-08
2132 1.50707395731864e-08
2133 1.5064233650941e-08
2134 1.50446119770997e-08
2135 1.48969215787442e-08
2136 1.502559877542e-08
2137 1.50096540665334e-08
2138 1.49169673859806e-08
2139 1.51563433916646e-08
2140 1.47916077612997e-08
2141 1.51391866936024e-08
2142 1.49991422129236e-08
2143 1.48767700989261e-08
2144 1.49415175059175e-08
2145 1.48268472759394e-08
2146 1.51109321542275e-08
2147 1.48803611759174e-08
2148 1.48502350856283e-08
2149 1.49049608537677e-08
2150 1.48771213093202e-08
2151 1.50330604415938e-08
2152 1.48694402186678e-08
2153 1.50325948164998e-08
2154 1.47947191782016e-08
2155 1.49915012250812e-08
2156 1.49105401627825e-08
2157 1.48151376762051e-08
2158 1.49127165607332e-08
2159 1.50419665529977e-08
2160 1.4776611993339e-08
2161 1.49078058198082e-08
2162 1.49839274423691e-08
2163 1.47843200106745e-08
2164 1.50897594464183e-08
2165 1.47585041936349e-08
2166 1.49435001219889e-08
2167 1.48391924983393e-08
2168 1.50543399586311e-08
2169 1.48809884459311e-08
2170 1.47667841869925e-08
2171 1.49418180628302e-08
2172 1.48474399279674e-08
2173 1.50997395405028e-08
2174 1.47525164291284e-08
2175 1.49486901899198e-08
2176 1.46469772350777e-08
2177 1.4887233337646e-08
2178 1.48939853088503e-08
2179 1.4957492644907e-08
2180 1.47110418791119e-08
2181 1.48099920083133e-08
2182 1.46865371610794e-08
2183 1.51106781618449e-08
2184 1.46668205243117e-08
2185 1.49261745592266e-08
2186 1.45941646725323e-08
2187 1.48459323325856e-08
2188 1.46405912995196e-08
2189 1.48702136393286e-08
2190 1.45347712290356e-08
2191 1.48093870935284e-08
2192 1.44951902867341e-08
2193 1.49154769837345e-08
2194 1.45880158255451e-08
2195 1.47864484805194e-08
2196 1.45986616746363e-08
2197 1.49511275124858e-08
2198 1.45870690722028e-08
2199 1.49062493923768e-08
2200 1.44439453166889e-08
2201 1.4988889311951e-08
2202 1.45477436115371e-08
2203 1.45467613750228e-08
2204 1.49101553961195e-08
2205 1.43552067826125e-08
2206 1.48435075373321e-08
2207 1.45914717033513e-08
2208 1.48035466498442e-08
2209 1.44143877203184e-08
2210 1.46228250723635e-08
2211 1.48737793490561e-08
2212 1.4538361773786e-08
2213 1.449192470937e-08
2214 1.4736262744286e-08
2215 1.46536052971769e-08
2216 1.44319446095942e-08
2217 1.46177590207852e-08
2218 1.45499799859561e-08
2219 1.46991494907045e-08
2220 1.44607092484872e-08
2221 1.46208735936071e-08
2222 1.44175600096386e-08
2223 1.45018722919055e-08
2224 1.43679764017524e-08
2225 1.4671391247445e-08
2226 1.45909641698916e-08
2227 1.46060929470115e-08
2228 1.45248533727216e-08
2229 1.43470345512764e-08
2230 1.45830420539284e-08
2231 1.4644664250163e-08
2232 1.43655178415081e-08
2233 1.45116092213904e-08
2234 1.43535520014115e-08
2235 1.45176361152632e-08
2236 1.44699258080028e-08
2237 1.46677142305318e-08
2238 1.4507503829142e-08
2239 1.4408715378833e-08
2240 1.48653502503038e-08
2241 1.46189233909411e-08
2242 1.42742275184293e-08
2243 1.45541696743301e-08
2244 1.42866778324313e-08
2245 1.44132543169473e-08
2246 1.46367729185748e-08
2247 1.42299466412421e-08
2248 1.4597219224477e-08
2249 1.42884581180169e-08
2250 1.44050718391409e-08
2251 1.44220002586959e-08
2252 1.43639719303312e-08
2253 1.44567580928623e-08
2254 1.44919069997584e-08
2255 1.42607161879305e-08
2256 1.4336776056556e-08
2257 1.43317893721573e-08
2258 1.4318787715295e-08
2259 1.44509181383423e-08
2260 1.45348891491537e-08
2261 1.41817797303379e-08
2262 1.43558007352773e-08
2263 1.45361034498137e-08
2264 1.42982039121886e-08
2265 1.41625402658718e-08
2266 1.43858149366238e-08
2267 1.42497472399583e-08
2268 1.44142319988827e-08
2269 1.42658671973273e-08
2270 1.45817579861873e-08
2271 1.42708052073282e-08
2272 1.42051821456768e-08
2273 1.43822051261733e-08
2274 1.42402229577066e-08
2275 1.42178208830668e-08
2276 1.45770469341944e-08
2277 1.41684960313082e-08
2278 1.42279522974409e-08
2279 1.44119161158418e-08
2280 1.41793614103491e-08
2281 1.43539773658219e-08
2282 1.41978286445354e-08
2283 1.43959410170424e-08
2284 1.43022215555177e-08
2285 1.41906396433633e-08
2286 1.42923123127581e-08
2287 1.42261521196563e-08
2288 1.42735479888856e-08
2289 1.43709216982035e-08
2290 1.42871719446269e-08
2291 1.41773387734023e-08
2292 1.43403587780089e-08
2293 1.41950157968029e-08
2294 1.42481061149446e-08
2295 1.41096525954065e-08
2296 1.42202431698824e-08
2297 1.41356032399287e-08
2298 1.41866324265827e-08
2299 1.4255966529042e-08
2300 1.43194529822477e-08
2301 1.41303449781027e-08
2302 1.42029390399756e-08
2303 1.42377434630969e-08
2304 1.41549614935954e-08
2305 1.42152571849241e-08
2306 1.41799824622257e-08
2307 1.40424886743151e-08
2308 1.43054304773127e-08
2309 1.40322029564821e-08
2310 1.42606018296298e-08
2311 1.42182590041617e-08
2312 1.41580845993694e-08
2313 1.41209680737386e-08
2314 1.434679157164e-08
2315 1.41953751864321e-08
2316 1.40374366406348e-08
2317 1.43269029990734e-08
2318 1.39360503264019e-08
2319 1.43620389134735e-08
2320 1.41286138612973e-08
2321 1.42673274290406e-08
2322 1.39790161761688e-08
2323 1.42110507439508e-08
2324 1.40861556590099e-08
2325 1.4041913044549e-08
2326 1.4193149335151e-08
2327 1.41044563712889e-08
2328 1.39682213300407e-08
2329 1.40472541640957e-08
2330 1.40596069915233e-08
2331 1.39896445698895e-08
2332 1.41753284235158e-08
2333 1.40586957080302e-08
2334 1.40888340494083e-08
2335 1.40730582056658e-08
2336 1.42153796147682e-08
2337 1.39944531862835e-08
2338 1.40267758477197e-08
2339 1.41264951498687e-08
2340 1.39711268953846e-08
2341 1.4170873825492e-08
2342 1.41264913011696e-08
2343 1.39304572255305e-08
2344 1.41708390504203e-08
2345 1.3933182499315e-08
2346 1.41338540984304e-08
2347 1.41122925301218e-08
2348 1.39124291778092e-08
2349 1.41465071596958e-08
2350 1.40166964297528e-08
2351 1.41130809001577e-08
2352 1.4047265430861e-08
2353 1.38833670695337e-08
2354 1.41745045454389e-08
2355 1.39893301320804e-08
2356 1.38334140882268e-08
2357 1.40654650022665e-08
2358 1.37720685333687e-08
2359 1.40494478240161e-08
2360 1.4015743839968e-08
2361 1.41097177912552e-08
2362 1.41370609696434e-08
2363 1.40141055824383e-08
2364 1.4079108995757e-08
2365 1.39445871165655e-08
2366 1.39255084823375e-08
2367 1.40892172773022e-08
2368 1.38032928942433e-08
2369 1.41174964087831e-08
2370 1.39935513112643e-08
2371 1.39415199853499e-08
2372 1.38634140482985e-08
2373 1.40171235660791e-08
2374 1.40096343073104e-08
2375 1.37921941800734e-08
2376 1.39486343226913e-08
2377 1.38717214579476e-08
2378 1.40190484303471e-08
2379 1.39277333854881e-08
2380 1.39720399408017e-08
2381 1.3866220449632e-08
2382 1.40251566989047e-08
2383 1.38753601544028e-08
2384 1.37740771821715e-08
2385 1.404509557279e-08
2386 1.39639110325884e-08
2387 1.37538638937951e-08
2388 1.40050249380907e-08
2389 1.38671847966787e-08
2390 1.38141625338761e-08
2391 1.38801213516171e-08
2392 1.39001210803791e-08
2393 1.400047530975e-08
2394 1.41016700456742e-08
2395 1.37881740718271e-08
2396 1.39741969999996e-08
2397 1.38791476504974e-08
2398 1.37890552532882e-08
2399 1.37510563087417e-08
2400 1.40283488705428e-08
2401 1.38920017747068e-08
2402 1.39024279389721e-08
2403 1.38303372936299e-08
2404 1.39305556972058e-08
2405 1.3767366617623e-08
2406 1.39737633171322e-08
2407 1.38411650076886e-08
2408 1.40019896917032e-08
2409 1.38083470961892e-08
2410 1.36461000426191e-08
2411 1.3965413260264e-08
2412 1.3812502136723e-08
2413 1.38602603863003e-08
2414 1.37415846481748e-08
2415 1.38213063154424e-08
2416 1.3714666057707e-08
2417 1.38172420693916e-08
2418 1.39105306057319e-08
2419 1.37631511507585e-08
2420 1.36803244314976e-08
2421 1.39455869216931e-08
2422 1.36916768149753e-08
2423 1.38661974367071e-08
2424 1.38433235348234e-08
2425 1.37198172514008e-08
2426 1.38355597532058e-08
2427 1.38060186594124e-08
2428 1.36678518434241e-08
2429 1.39975644048995e-08
2430 1.35297852079663e-08
2431 1.38105012110756e-08
2432 1.37073296369028e-08
2433 1.37392478576359e-08
2434 1.38243442753261e-08
2435 1.375203388565e-08
2436 1.39418836696548e-08
2437 1.35428592666909e-08
2438 1.38340865090125e-08
2439 1.37432495779244e-08
2440 1.38399673033085e-08
2441 1.37608220238672e-08
2442 1.35892773397428e-08
2443 1.38813414090055e-08
2444 1.37053065352166e-08
2445 1.36753972768133e-08
2446 1.38476288049638e-08
2447 1.364882708077e-08
2448 1.37304545639871e-08
2449 1.3752143596113e-08
2450 1.370826896685e-08
2451 1.37210694561851e-08
2452 1.35850768676615e-08
2453 1.36554495009378e-08
2454 1.37473818715428e-08
2455 1.35429369059192e-08
2456 1.38872157713354e-08
2457 1.38306991090964e-08
2458 1.36778022277184e-08
2459 1.36373059116401e-08
2460 1.36652871480791e-08
2461 1.37161399311747e-08
2462 1.35473098425987e-08
2463 1.36878215148961e-08
2464 1.36233974530331e-08
2465 1.36723616985801e-08
2466 1.3725985149371e-08
2467 1.38408580399041e-08
2468 1.37650904257214e-08
2469 1.37494516567749e-08
2470 1.36912278039247e-08
2471 1.36133404691208e-08
2472 1.36215395556416e-08
2473 1.35836955506008e-08
2474 1.35509382540588e-08
2475 1.38705918495408e-08
2476 1.36895569928797e-08
2477 1.37148010048715e-08
2478 1.35716730798041e-08
2479 1.3909969577508e-08
2480 1.36585178438509e-08
2481 1.36806949124768e-08
2482 1.37105445030539e-08
2483 1.35515608887804e-08
2484 1.37336489161033e-08
2485 1.376298264677e-08
2486 1.35039431577999e-08
2487 1.38477287117134e-08
2488 1.34672162601657e-08
2489 1.37278245448957e-08
2490 1.35734883859318e-08
2491 1.35854170622007e-08
2492 1.37237693531578e-08
2493 1.35103786074176e-08
2494 1.3568633325356e-08
2495 1.35705022117349e-08
2496 1.35283215990656e-08
2497 1.37982020280703e-08
2498 1.3682044835317e-08
2499 1.36406846373838e-08
2500 1.35953621702711e-08
2501 1.34256677342126e-08
2502 1.35108625052283e-08
2503 1.37002816265053e-08
2504 1.36345143106897e-08
2505 1.36181393217694e-08
2506 1.34724567319999e-08
2507 1.35965666410165e-08
2508 1.35763364836894e-08
2509 1.34909989795151e-08
2510 1.36406634538844e-08
2511 1.35081688966698e-08
2512 1.3638746218847e-08
2513 1.3752693190705e-08
2514 1.3340044360155e-08
2515 1.3772601298756e-08
2516 1.34202316230958e-08
2517 1.36460381642278e-08
2518 1.35352909038655e-08
2519 1.35986593412518e-08
2520 1.34299068355048e-08
2521 1.34223995940097e-08
2522 1.36369369723166e-08
2523 1.35071825740907e-08
2524 1.36999209083832e-08
2525 1.34544996526209e-08
2526 1.34096982888909e-08
2527 1.36828752990148e-08
2528 1.35948496677862e-08
2529 1.34196183421142e-08
2530 1.36951123439477e-08
2531 1.33354384540541e-08
2532 1.36193287720854e-08
2533 1.33821837629e-08
2534 1.36437988362825e-08
2535 1.35588052356805e-08
2536 1.33648931877062e-08
2537 1.3603747350488e-08
2538 1.34346554729881e-08
2539 1.36784648294697e-08
2540 1.33371273514094e-08
2541 1.34051086919662e-08
2542 1.35944426018497e-08
2543 1.35151843297932e-08
2544 1.34519735183192e-08
2545 1.37123280525842e-08
2546 1.34673013585385e-08
2547 1.33724502019428e-08
2548 1.36526545543303e-08
2549 1.34637314963548e-08
2550 1.36403395324391e-08
2551 1.34908148153867e-08
2552 1.34084806852064e-08
2553 1.35526276858755e-08
2554 1.35033407340224e-08
2555 1.35222426224102e-08
2556 1.33713409318226e-08
2557 1.33650788805006e-08
2558 1.36520787374916e-08
2559 1.33554703165562e-08
2560 1.35396974711988e-08
2561 1.34487225598168e-08
2562 1.33152270321846e-08
2563 1.34364969340828e-08
2564 1.35619053974745e-08
2565 1.33863058322525e-08
2566 1.36050281067668e-08
2567 1.33659348657833e-08
2568 1.37255042317319e-08
2569 1.34244166412056e-08
2570 1.3519588680122e-08
2571 1.32766994732636e-08
2572 1.35028802028536e-08
2573 1.35831575912571e-08
2574 1.35768645407319e-08
2575 1.32348685356609e-08
2576 1.35520635804465e-08
2577 1.34222149472674e-08
2578 1.33839490529253e-08
2579 1.34082565182991e-08
2580 1.35213558137837e-08
2581 1.33390427724533e-08
2582 1.35850905917945e-08
2583 1.3422868959112e-08
2584 1.32464868863869e-08
2585 1.38154081271713e-08
2586 1.31752892771342e-08
2587 1.33158316373283e-08
2588 1.33797019630233e-08
2589 1.33396303825251e-08
2590 1.33530084598377e-08
2591 1.33678609043342e-08
2592 1.33515470744916e-08
2593 1.33812431815095e-08
2594 1.33434626209405e-08
2595 1.34477306710235e-08
2596 1.34127029473019e-08
2597 1.34654325052441e-08
2598 1.33639408929076e-08
2599 1.33319922863695e-08
2600 1.34768838446764e-08
2601 1.31791077137011e-08
2602 1.33533299144784e-08
2603 1.33335717759087e-08
2604 1.34681602890208e-08
2605 1.32414936400149e-08
2606 1.32864522507381e-08
2607 1.34735510083361e-08
2608 1.32059107220694e-08
2609 1.3334054516867e-08
2610 1.34215289667727e-08
2611 1.32657797027314e-08
2612 1.35989561345085e-08
2613 1.31205984885696e-08
2614 1.34574489334405e-08
2615 1.33029648045069e-08
2616 1.33842032632447e-08
2617 1.3357692433158e-08
2618 1.32575467388651e-08
2619 1.35273623923515e-08
2620 1.3277165931469e-08
2621 1.335528048374e-08
2622 1.31872967777946e-08
2623 1.33831317866706e-08
2624 1.33388513036126e-08
2625 1.32144205600149e-08
2626 1.32852943173223e-08
2627 1.31935045304399e-08
2628 1.33542605561532e-08
2629 1.3354061429105e-08
2630 1.32371038795709e-08
2631 1.33018290884301e-08
2632 1.34557489450682e-08
2633 1.32379623968282e-08
2634 1.32958427130347e-08
2635 1.31507755313232e-08
2636 1.32096375565771e-08
2637 1.34073030578774e-08
2638 1.33222986702375e-08
2639 1.3401855897488e-08
2640 1.31645463549157e-08
2641 1.33627694371974e-08
2642 1.32331624366211e-08
2643 1.31408740116079e-08
2644 1.33200566758696e-08
2645 1.31626890380598e-08
2646 1.32798392529221e-08
2647 1.32784131157138e-08
2648 1.33079137436543e-08
2649 1.3286313587435e-08
2650 1.31924705708553e-08
2651 1.33048195907204e-08
2652 1.30137468740132e-08
2653 1.34281471921849e-08
2654 1.3057351264445e-08
2655 1.32196091675585e-08
2656 1.32159875364524e-08
2657 1.32543261541773e-08
2658 1.31364035212966e-08
2659 1.3204991437199e-08
2660 1.32414342139953e-08
2661 1.32986774195043e-08
2662 1.3041312209694e-08
2663 1.31474040477864e-08
2664 1.30802353920378e-08
2665 1.32568124702193e-08
2666 1.29940446317311e-08
2667 1.32352883573938e-08
2668 1.31357354056227e-08
2669 1.32277744692155e-08
2670 1.33734919461892e-08
2671 1.32428376378169e-08
2672 1.32623260472586e-08
2673 1.32571301487694e-08
2674 1.31517546434479e-08
2675 1.31302757278018e-08
2676 1.31688670726993e-08
2677 1.33119552264693e-08
2678 1.3104957399368e-08
2679 1.31819562724234e-08
2680 1.31432037391299e-08
2681 1.32742551740561e-08
2682 1.29295538519036e-08
2683 1.32204741050135e-08
2684 1.31958548137412e-08
2685 1.33010567293645e-08
2686 1.31019791972387e-08
2687 1.31020519272829e-08
2688 1.30707277548048e-08
2689 1.32388215745571e-08
2690 1.33160208809491e-08
2691 1.31285953072213e-08
2692 1.30915295030176e-08
2693 1.32061807573969e-08
2694 1.31916891197292e-08
2695 1.32221508465369e-08
2696 1.30800646940266e-08
2697 1.29486734545825e-08
2698 1.32070453564559e-08
2699 1.3080578836866e-08
2700 1.3303537313436e-08
2701 1.28718722278709e-08
2702 1.31577675236416e-08
2703 1.30238560679397e-08
2704 1.30839731938837e-08
2705 1.32252224842588e-08
2706 1.31557619740263e-08
2707 1.32034301623118e-08
2708 1.29130689573431e-08
2709 1.31619552578011e-08
2710 1.31710308566291e-08
2711 1.30400202996706e-08
2712 1.31301207346723e-08
2713 1.3017391601422e-08
2714 1.3164336552407e-08
2715 1.30099783269833e-08
2716 1.30728472078623e-08
2717 1.29553605768828e-08
2718 1.302479999854e-08
2719 1.29843404087504e-08
2720 1.31267940748359e-08
2721 1.31959914988489e-08
2722 1.28730747417372e-08
2723 1.31897120920765e-08
2724 1.29261496428779e-08
2725 1.3157492564364e-08
2726 1.28798928281659e-08
2727 1.30678707526144e-08
2728 1.28864734409984e-08
2729 1.30451557331934e-08
2730 1.31218652167364e-08
2731 1.28550515375858e-08
2732 1.31797102462805e-08
2733 1.29813788312116e-08
2734 1.28739894749152e-08
2735 1.29199545153558e-08
2736 1.29891977667107e-08
2737 1.31881727003513e-08
2738 1.29794092971203e-08
2739 1.29697906667836e-08
2740 1.28169413220469e-08
2741 1.30101680566597e-08
2742 1.3001858667705e-08
2743 1.27702882438685e-08
2744 1.31385045826216e-08
2745 1.29998826778888e-08
2746 1.2962397603844e-08
2747 1.29797338302984e-08
2748 1.30283088849437e-08
2749 1.31080680207951e-08
2750 1.3031614337522e-08
2751 1.28903602788499e-08
2752 1.31225722368455e-08
2753 1.32075114296359e-08
2754 1.29988942841885e-08
2755 1.30375464666432e-08
2756 1.31089299169052e-08
2757 1.29446168279923e-08
2758 1.3020685737053e-08
2759 1.32783974520212e-08
2760 1.29848785803688e-08
2761 1.30084427093502e-08
2762 1.31500941968898e-08
2763 1.29836102878977e-08
2764 1.31115839295814e-08
2765 1.29538080844149e-08
2766 1.29995829281082e-08
2767 1.29302793039354e-08
2768 1.28580117608745e-08
2769 1.31190529739644e-08
2770 1.29641982362649e-08
2771 1.31361205150116e-08
2772 1.30407074664385e-08
2773 1.28694385059802e-08
2774 1.30992503906135e-08
2775 1.29015260232901e-08
2776 1.30024812032836e-08
2777 1.2885147859576e-08
2778 1.31026255408839e-08
2779 1.2951651289228e-08
2780 1.29980507450611e-08
2781 1.29561991140115e-08
2782 1.2988132938041e-08
2783 1.31504733772436e-08
2784 1.29048229442486e-08
2785 1.2944282392402e-08
2786 1.31725279619577e-08
2787 1.28808262329638e-08
2788 1.31504360870727e-08
2789 1.28385636604111e-08
2790 1.284708941518e-08
2791 1.29636830525914e-08
2792 1.29449775334578e-08
2793 1.31626443535815e-08
2794 1.29462369538036e-08
2795 1.31072495517248e-08
2796 1.29005571761764e-08
2797 1.28414072578797e-08
2798 1.29530307113601e-08
2799 1.30208934938647e-08
2800 1.29905153812837e-08
2801 1.28483177693761e-08
2802 1.29786480203009e-08
2803 1.29855664268153e-08
2804 1.28054845308645e-08
2805 1.29539135413914e-08
2806 1.28665635185099e-08
2807 1.29152078950368e-08
2808 1.29058740212518e-08
2809 1.28908363739022e-08
2810 1.2921545773148e-08
2811 1.29304910168049e-08
2812 1.27782128476195e-08
2813 1.29774449716402e-08
2814 1.29612136388069e-08
2815 1.29126925950684e-08
2816 1.28917863571054e-08
2817 1.2875600503226e-08
2818 1.29427548414274e-08
2819 1.29617389124093e-08
2820 1.29227674415811e-08
2821 1.29728709926669e-08
2822 1.27519686191624e-08
2823 1.28984655433095e-08
2824 1.29685633021293e-08
2825 1.27998455705658e-08
2826 1.28593525329235e-08
2827 1.28526657603434e-08
2828 1.29714934526959e-08
2829 1.28717928703503e-08
2830 1.28904537906038e-08
2831 1.29237767704238e-08
2832 1.29320912570785e-08
2833 1.27779278918938e-08
2834 1.28684619191688e-08
2835 1.28633674778555e-08
2836 1.2865347506108e-08
2837 1.32162628966315e-08
2838 1.28213776287023e-08
2839 1.27756023843295e-08
2840 1.28309748881339e-08
2841 1.27862703132031e-08
2842 1.28640088730148e-08
2843 1.30217109789443e-08
2844 1.26958545757505e-08
2845 1.30367317437985e-08
2846 1.29173305016828e-08
2847 1.27095155968426e-08
2848 1.28978311798589e-08
2849 1.27279873946007e-08
2850 1.2868218245643e-08
2851 1.27287758900918e-08
2852 1.28639928815844e-08
2853 1.27030911900583e-08
2854 1.28284177168814e-08
2855 1.2927207599378e-08
2856 1.27366578507315e-08
2857 1.27587414143759e-08
2858 1.28787940716357e-08
2859 1.29568711501049e-08
2860 1.30056360345687e-08
2861 1.27367155499103e-08
2862 1.26641222850665e-08
2863 1.27975356226662e-08
2864 1.29459480575633e-08
2865 1.27520865185193e-08
2866 1.26843475078964e-08
2867 1.27797181959099e-08
2868 1.27041951146767e-08
2869 1.29419129299979e-08
2870 1.28504834717713e-08
2871 1.27504527530631e-08
2872 1.27788067906254e-08
2873 1.27660894297943e-08
2874 1.27549396932158e-08
2875 1.28400117693772e-08
2876 1.27173464878894e-08
2877 1.28136967831871e-08
2878 1.26572399338221e-08
2879 1.28094380984489e-08
2880 1.27066031329059e-08
2881 1.27737193396094e-08
2882 1.27841473560597e-08
2883 1.27638147036091e-08
2884 1.27545328958423e-08
2885 1.27409032517622e-08
2886 1.27130146592158e-08
2887 1.25730234397103e-08
2888 1.28801449946758e-08
2889 1.27212882523597e-08
2890 1.2727330540252e-08
2891 1.26529798218922e-08
2892 1.28023001624289e-08
2893 1.27013094695094e-08
2894 1.2836636874014e-08
2895 1.25894901774615e-08
2896 1.28501557294936e-08
2897 1.27658795829877e-08
2898 1.27611768965252e-08
2899 1.25339836433147e-08
2900 1.29507780697313e-08
2901 1.26953562243903e-08
2902 1.27380390992915e-08
2903 1.26650729995736e-08
2904 1.27495271409206e-08
2905 1.26221697327455e-08
2906 1.28285369701553e-08
2907 1.26907206959848e-08
2908 1.27939680620859e-08
2909 1.2582503063463e-08
2910 1.27220199463984e-08
2911 1.26326543680166e-08
2912 1.27689875558668e-08
2913 1.26551395989161e-08
2914 1.27011788300102e-08
2915 1.26727060891785e-08
2916 1.26509812337083e-08
2917 1.27208448601479e-08
2918 1.27198795476513e-08
2919 1.27581997717563e-08
2920 1.25767286569545e-08
2921 1.28270613077852e-08
2922 1.26836638176808e-08
2923 1.27192956252964e-08
2924 1.26669364463039e-08
2925 1.27766982761512e-08
2926 1.27112218956116e-08
2927 1.27417836429666e-08
2928 1.28625708450958e-08
2929 1.27768200507417e-08
2930 1.26412186929326e-08
2931 1.28145099680399e-08
2932 1.27348042601172e-08
2933 1.27624384977931e-08
2934 1.27461877947876e-08
2935 1.25583175614175e-08
2936 1.2798607938791e-08
2937 1.26090474448226e-08
2938 1.25311180850796e-08
2939 1.27364157710419e-08
2940 1.27218140713037e-08
2941 1.26061884315742e-08
2942 1.2565958056987e-08
2943 1.26511467738455e-08
2944 1.25448423411445e-08
2945 1.26631490873219e-08
2946 1.26975736634005e-08
2947 1.26435551422999e-08
2948 1.26659404422469e-08
2949 1.26662933693789e-08
2950 1.26209524472509e-08
2951 1.26658115905398e-08
2952 1.27243844223468e-08
2953 1.26229271585343e-08
2954 1.27194141053e-08
2955 1.26684454904158e-08
2956 1.25313995383847e-08
2957 1.29395350114692e-08
2958 1.25526705354595e-08
2959 1.25787466112026e-08
2960 1.25561079427072e-08
2961 1.26790272412247e-08
2962 1.26185195330475e-08
2963 1.2585826579814e-08
2964 1.26786975781501e-08
2965 1.25989877042265e-08
2966 1.26459115117328e-08
2967 1.244493678354e-08
2968 1.26320767652732e-08
2969 1.26147080377059e-08
2970 1.26063095446938e-08
2971 1.26300362193188e-08
2972 1.2479303094004e-08
2973 1.2612908664611e-08
2974 1.25588364849882e-08
2975 1.2583749208428e-08
2976 1.24199685107573e-08
2977 1.26457500677635e-08
2978 1.26720407525038e-08
2979 1.24517379171962e-08
2980 1.25845378344813e-08
2981 1.25156524511949e-08
2982 1.26454206730298e-08
2983 1.24522749678224e-08
2984 1.25524758279871e-08
2985 1.25343531763855e-08
2986 1.24884376135892e-08
2987 1.2691015436328e-08
2988 1.2492628525651e-08
2989 1.25745085266526e-08
2990 1.25912885520219e-08
2991 1.25712481874318e-08
2992 1.25756403224209e-08
2993 1.24959370508826e-08
2994 1.2543754300931e-08
2995 1.24573120547922e-08
2996 1.24928301967753e-08
2997 1.25988466871396e-08
2998 1.2601220312658e-08
2999 1.25069590488636e-08
3000 1.24827408019001e-08
3001 1.25361182525818e-08
3002 1.25905353212108e-08
3003 1.24206886662481e-08
3004 1.24132642753683e-08
3005 1.24748642609251e-08
3006 1.26619439100306e-08
3007 1.25173795406663e-08
3008 1.25236954018337e-08
3009 1.25783748350372e-08
3010 1.24754689281303e-08
3011 1.23531089624151e-08
3012 1.24100798434279e-08
3013 1.2492930736796e-08
3014 1.23824709507003e-08
3015 1.24713766755935e-08
3016 1.24711918570997e-08
3017 1.26096053020275e-08
3018 1.23924971414668e-08
3019 1.23219691486876e-08
3020 1.24702957604583e-08
3021 1.24699168091436e-08
3022 1.24955696857443e-08
3023 1.24546930058145e-08
3024 1.24405578912912e-08
3025 1.24743166455277e-08
3026 1.25360221884252e-08
3027 1.24916599761882e-08
3028 1.24500235547664e-08
3029 1.2478918919645e-08
3030 1.24869328483879e-08
3031 1.24602419936393e-08
3032 1.25455092663262e-08
3033 1.23763265702514e-08
3034 1.2495644907351e-08
3035 1.23490911003721e-08
3036 1.23677045593684e-08
3037 1.24534547262245e-08
3038 1.24302798703013e-08
3039 1.24129740288748e-08
3040 1.24239170373741e-08
3041 1.25385056799487e-08
3042 1.24598660120601e-08
3043 1.24864651059875e-08
3044 1.23495830074427e-08
3045 1.25019950100169e-08
3046 1.2381913360171e-08
3047 1.24107256169736e-08
3048 1.22973514702363e-08
3049 1.23705242283112e-08
3050 1.24912603520766e-08
3051 1.24421974776467e-08
3052 1.23907293668735e-08
3053 1.24274013342252e-08
3054 1.22608731608631e-08
3055 1.24273369511707e-08
3056 1.24091042723595e-08
3057 1.24298240818943e-08
3058 1.24189522621165e-08
3059 1.23780490033365e-08
3060 1.2218805521047e-08
3061 1.23499759677737e-08
3062 1.24127540624963e-08
3063 1.24126001533886e-08
3064 1.24087955015728e-08
3065 1.23124641762118e-08
3066 1.23474887434583e-08
3067 1.23024445362052e-08
3068 1.22454262895255e-08
3069 1.24441190598734e-08
3070 1.23445799133926e-08
3071 1.23637994741133e-08
3072 1.23178367081866e-08
3073 1.22499472302584e-08
3074 1.23600529389201e-08
3075 1.23091195246561e-08
3076 1.23154479925169e-08
3077 1.24539648853617e-08
3078 1.23930148905327e-08
3079 1.24463001709429e-08
3080 1.22570391863919e-08
3081 1.23195528020092e-08
3082 1.2287365025565e-08
3083 1.22927613702695e-08
3084 1.23412279429136e-08
3085 1.21692546883878e-08
3086 1.24556919586238e-08
3087 1.24146133756442e-08
3088 1.22403589128961e-08
3089 1.22697023237084e-08
3090 1.21464158473472e-08
3091 1.24155451212138e-08
3092 1.22327571332459e-08
3093 1.23553317755709e-08
3094 1.22262967056086e-08
3095 1.23398293248034e-08
3096 1.23832230024545e-08
3097 1.22180334660715e-08
3098 1.23262975452176e-08
3099 1.22021943794737e-08
3100 1.23616622784706e-08
3101 1.2221561740744e-08
3102 1.23347015758002e-08
3103 1.23631759850751e-08
3104 1.22300479048487e-08
3105 1.22020365632691e-08
3106 1.23488363763524e-08
3107 1.23171231009023e-08
3108 1.23048198036324e-08
3109 1.22826019579581e-08
3110 1.23707787968996e-08
3111 1.22293526371164e-08
3112 1.22919292040446e-08
3113 1.23884422248421e-08
3114 1.21827114558659e-08
3115 1.22469047383333e-08
3116 1.23279879371552e-08
3117 1.22886665470112e-08
3118 1.22081997029344e-08
3119 1.22600391062599e-08
3120 1.22551829253581e-08
3121 1.21513975200882e-08
3122 1.22909846441699e-08
3123 1.22629457339762e-08
3124 1.22669598129344e-08
3125 1.22110587688073e-08
3126 1.21765399876406e-08
3127 1.22154028070476e-08
3128 1.23236420868089e-08
3129 1.22640595607848e-08
3130 1.22744318146184e-08
3131 1.23616035526686e-08
3132 1.21928273577598e-08
3133 1.23580472617402e-08
3134 1.21665787120584e-08
3135 1.21490908761013e-08
3136 1.22604584849029e-08
3137 1.21425842164458e-08
3138 1.21882966067632e-08
3139 1.21690841199396e-08
3140 1.23088224095458e-08
3141 1.22407523152068e-08
3142 1.2201373489229e-08
3143 1.22338057561011e-08
3144 1.2269440024526e-08
3145 1.21453471857436e-08
3146 1.21665416792371e-08
3147 1.22031038156667e-08
3148 1.23535330034397e-08
3149 1.20790477423638e-08
3150 1.22330635604628e-08
3151 1.22182638032653e-08
3152 1.22683826826453e-08
3153 1.22664455273203e-08
3154 1.21033765133616e-08
3155 1.20065042932271e-08
3156 1.19805545032436e-08
3157 1.21893945480567e-08
3158 1.21012253206931e-08
3159 1.20572098587512e-08
3160 1.22418666649304e-08
3161 1.21597204472135e-08
3162 1.21420031153896e-08
3163 1.19866848391492e-08
3164 1.20832606376187e-08
3165 1.21122499141535e-08
3166 1.20180061664588e-08
3167 1.21569859860005e-08
3168 1.20225054637269e-08
3169 1.21333479853325e-08
3170 1.21546785006865e-08
3171 1.20704107506864e-08
3172 1.21050232363418e-08
3173 1.20097799930496e-08
3174 1.2279239204771e-08
3175 1.19673032324963e-08
3176 1.21465957146905e-08
3177 1.20022387718466e-08
3178 1.21609852758997e-08
3179 1.19798918323255e-08
3180 1.21661550569474e-08
3181 1.20989683484174e-08
3182 1.2202835625641e-08
3183 1.20139791308649e-08
3184 1.19383170875675e-08
3185 1.22750231212887e-08
3186 1.21123770467912e-08
3187 1.20226011756097e-08
3188 1.21788859273053e-08
3189 1.20428523838756e-08
3190 1.2030838963728e-08
3191 1.20955890047236e-08
3192 1.21110425919113e-08
3193 1.20412093064237e-08
3194 1.21228476495094e-08
3195 1.20455553275223e-08
3196 1.20178184697117e-08
3197 1.21071264981287e-08
3198 1.22744237577299e-08
3199 1.18565174934693e-08
3200 1.20218359787039e-08
3201 1.21616001322877e-08
3202 1.19539210577946e-08
3203 1.21947089832819e-08
3204 1.19192906981036e-08
3205 1.21685767026092e-08
3206 1.20364045845811e-08
3207 1.2058401393622e-08
3208 1.21970317448206e-08
3209 1.19015011449397e-08
3210 1.22034944042282e-08
3211 1.19711576951342e-08
3212 1.19261809677917e-08
3213 1.201746030699e-08
3214 1.2095558953984e-08
3215 1.18764128336224e-08
3216 1.20847981341932e-08
3217 1.20346177032715e-08
3218 1.21082430635244e-08
3219 1.18841043644791e-08
3220 1.21618978170535e-08
3221 1.19772637163784e-08
3222 1.19348417766396e-08
3223 1.21272629334257e-08
3224 1.21333791520684e-08
3225 1.20331801540807e-08
3226 1.19787194224896e-08
3227 1.21797619329067e-08
3228 1.199807992458e-08
3229 1.19835496070753e-08
3230 1.19656150598946e-08
3231 1.21770040353297e-08
3232 1.19530995823514e-08
3233 1.20372374293742e-08
3234 1.20072512416192e-08
3235 1.21927483328621e-08
3236 1.17398581148676e-08
3237 1.21068398194479e-08
3238 1.20971648559509e-08
3239 1.19099769095854e-08
3240 1.19712968459362e-08
3241 1.20240873983102e-08
3242 1.19557595227304e-08
3243 1.19604335007129e-08
3244 1.19485781346906e-08
3245 1.19126404551917e-08
3246 1.20156140565042e-08
3247 1.18723316306912e-08
3248 1.19383256275141e-08
3249 1.21041150882295e-08
3250 1.1831603861423e-08
3251 1.20942475184682e-08
3252 1.19279491719304e-08
3253 1.17869465114939e-08
3254 1.21646093799166e-08
3255 1.19352093826963e-08
3256 1.1917382253368e-08
3257 1.20691251297433e-08
3258 1.17996257963382e-08
3259 1.21671530470824e-08
3260 1.19047278144269e-08
3261 1.20948725259584e-08
3262 1.19227281482059e-08
3263 1.20291821515961e-08
3264 1.18442795342677e-08
3265 1.2037174346391e-08
3266 1.19648114162985e-08
3267 1.1924443348077e-08
3268 1.19189285979759e-08
3269 1.19316589797247e-08
3270 1.18112469508214e-08
3271 1.19891326210153e-08
3272 1.18897858664146e-08
3273 1.18930797624595e-08
3274 1.18475735289003e-08
3275 1.20714702070979e-08
3276 1.16261695213282e-08
3277 1.20717354089583e-08
3278 1.20265051567481e-08
3279 1.18194004508076e-08
3280 1.20962635954402e-08
3281 1.179955589381e-08
3282 1.20689207485647e-08
3283 1.18965635643731e-08
3284 1.19260693683954e-08
3285 1.16441409372392e-08
3286 1.21104374021241e-08
3287 1.18036055846149e-08
3288 1.20134509437042e-08
3289 1.19493719836772e-08
3290 1.19975623150692e-08
3291 1.18336913008665e-08
3292 1.18525895310695e-08
3293 1.19608704568508e-08
3294 1.18522623936412e-08
3295 1.18743140122834e-08
3296 1.19055007665736e-08
3297 1.18920359560848e-08
3298 1.1919617607159e-08
3299 1.17696050183635e-08
3300 1.1920979585911e-08
3301 1.19140202082813e-08
3302 1.18875625383374e-08
3303 1.19173530659378e-08
3304 1.18585712032271e-08
3305 1.17698651097564e-08
3306 1.19141688277313e-08
3307 1.18901792399706e-08
3308 1.19107881573211e-08
3309 1.18232848596023e-08
3310 1.18678801446226e-08
3311 1.18430009111714e-08
3312 1.17999902785604e-08
3313 1.18243790626638e-08
3314 1.19541133128953e-08
3315 1.18772256171296e-08
3316 1.21337015570599e-08
3317 1.18263152344422e-08
3318 1.18587807167447e-08
3319 1.18224585098359e-08
3320 1.18068788990122e-08
3321 1.20390671750625e-08
3322 1.17679786700853e-08
3323 1.18663252175466e-08
3324 1.18566682008048e-08
3325 1.19563194566075e-08
3326 1.18549221055364e-08
3327 1.18599108289708e-08
3328 1.18045061594518e-08
3329 1.18659244965347e-08
3330 1.18471208412352e-08
3331 1.18093742338399e-08
3332 1.19554801352173e-08
3333 1.18451090492799e-08
3334 1.18801382319456e-08
3335 1.18201911213367e-08
3336 1.18142059761794e-08
3337 1.16756097243709e-08
3338 1.19491870772537e-08
3339 1.18746605430875e-08
3340 1.1679284222188e-08
3341 1.18830476316667e-08
3342 1.17050216325021e-08
3343 1.17035651351349e-08
3344 1.18880924503406e-08
3345 1.17451599362939e-08
3346 1.17961983198134e-08
3347 1.17283087978093e-08
3348 1.1647312328722e-08
3349 1.17963609441718e-08
3350 1.17776159139771e-08
3351 1.1741538077481e-08
3352 1.1635588667902e-08
3353 1.18773601370803e-08
3354 1.17518256657068e-08
3355 1.16728841952352e-08
3356 1.17781339303846e-08
3357 1.16865779591357e-08
3358 1.16740936230153e-08
3359 1.16961090871337e-08
3360 1.17630970311033e-08
3361 1.17772707379782e-08
3362 1.17135179759087e-08
3363 1.17125534202511e-08
3364 1.15475401756271e-08
3365 1.16739037492319e-08
3366 1.19871747631395e-08
3367 1.14944913279791e-08
3368 1.17140122952719e-08
3369 1.18451152133492e-08
3370 1.16727181581622e-08
3371 1.18134656461599e-08
3372 1.16800266196648e-08
3373 1.17781283438534e-08
3374 1.15551202372055e-08
3375 1.16391273145e-08
3376 1.17036688188632e-08
3377 1.15852520188309e-08
3378 1.18450441369822e-08
3379 1.16307025770368e-08
3380 1.15886218724492e-08
3381 1.17250487267073e-08
3382 1.17926541509661e-08
3383 1.17887550900031e-08
3384 1.16148132989213e-08
3385 1.17333160287725e-08
3386 1.15300117611294e-08
3387 1.17308465822363e-08
3388 1.17143098178341e-08
3389 1.16411068048494e-08
3390 1.17138005936157e-08
3391 1.16668510110607e-08
3392 1.1716263887962e-08
3393 1.15918866363529e-08
3394 1.16459187228246e-08
3395 1.16756567553056e-08
3396 1.15771008786236e-08
3397 1.14604446039834e-08
3398 1.17598079347747e-08
3399 1.17045232475022e-08
3400 1.16282462323536e-08
3401 1.15781526013325e-08
3402 1.16013079228816e-08
3403 1.15777289245722e-08
3404 1.16921247137469e-08
3405 1.15869787765677e-08
3406 1.16868900909006e-08
3407 1.16780403207617e-08
3408 1.16384350163923e-08
3409 1.16542223367322e-08
3410 1.1714863638268e-08
3411 1.16071121962769e-08
3412 1.16216457409779e-08
3413 1.15729205126813e-08
3414 1.15826353362936e-08
3415 1.16966123988593e-08
3416 1.16022560868734e-08
3417 1.16074119951293e-08
3418 1.16403504156759e-08
3419 1.15806943044028e-08
3420 1.16778067896783e-08
3421 1.16497168748397e-08
3422 1.16510462413455e-08
3423 1.15712716928229e-08
3424 1.15379166240048e-08
3425 1.15624306962214e-08
3426 1.1631782367183e-08
3427 1.15591818165006e-08
3428 1.16052897722874e-08
3429 1.16089783421813e-08
3430 1.15491841566895e-08
3431 1.15991464765264e-08
3432 1.15787202296103e-08
3433 1.16756626170611e-08
3434 1.15453121535269e-08
3435 1.16057201595687e-08
3436 1.16285898584811e-08
3437 1.14992614246301e-08
3438 1.15538398899329e-08
3439 1.16749788141579e-08
3440 1.14467703133947e-08
3441 1.16060629322678e-08
3442 1.14599823746175e-08
3443 1.1500587814961e-08
3444 1.16727993011434e-08
3445 1.15301269126844e-08
3446 1.14118573297795e-08
3447 1.15578581239006e-08
3448 1.15453476408067e-08
3449 1.13085325024187e-08
3450 1.16178723646998e-08
3451 1.15456656490931e-08
3452 1.15263954908462e-08
3453 1.15313389835725e-08
3454 1.13123329141818e-08
3455 1.17534853371071e-08
3456 1.14425797514972e-08
3457 1.14171010594522e-08
3458 1.15472187851573e-08
3459 1.14724863494731e-08
3460 1.15678232979199e-08
3461 1.1456993234904e-08
3462 1.1374440495926e-08
3463 1.14860447132381e-08
3464 1.13386492669454e-08
3465 1.14994115725242e-08
3466 1.1531178444435e-08
3467 1.15628821951708e-08
3468 1.14007412738149e-08
3469 1.14142820336616e-08
3470 1.1616755057009e-08
3471 1.13220017829985e-08
3472 1.15555358626285e-08
3473 1.13571001770785e-08
3474 1.13187439937379e-08
3475 1.14160510894523e-08
3476 1.15210650757369e-08
3477 1.15000994811432e-08
3478 1.13730303652471e-08
3479 1.15356391230392e-08
3480 1.15318592038838e-08
3481 1.13606639946529e-08
3482 1.15110280018982e-08
3483 1.12820390455459e-08
3484 1.14559026325178e-08
3485 1.1500695156208e-08
3486 1.14388078543692e-08
3487 1.13163693512552e-08
3488 1.1417295585181e-08
3489 1.14587170153557e-08
3490 1.14756259192994e-08
3491 1.14855673952752e-08
3492 1.14743690822205e-08
3493 1.1518184987791e-08
3494 1.14891360670777e-08
3495 1.1500933534081e-08
3496 1.11923943947323e-08
3497 1.16260063559581e-08
3498 1.13600809249492e-08
3499 1.14645093608479e-08
3500 1.15043888481159e-08
3501 1.14414826982712e-08
3502 1.14966460378341e-08
3503 1.13043758523945e-08
3504 1.14041341366944e-08
3505 1.14581738775987e-08
3506 1.13487078240437e-08
3507 1.13254841540567e-08
3508 1.1523862663454e-08
3509 1.13553434937241e-08
3510 1.1386845117034e-08
3511 1.14212314447082e-08
3512 1.13532928551763e-08
3513 1.13146536031561e-08
3514 1.14213090585125e-08
3515 1.14586641116743e-08
3516 1.13672053139169e-08
3517 1.12380934724765e-08
3518 1.13277275124446e-08
3519 1.13810661639402e-08
3520 1.14142512056548e-08
3521 1.14098708093868e-08
3522 1.13476815533087e-08
3523 1.14483209738836e-08
3524 1.13382380902971e-08
3525 1.14034751745873e-08
3526 1.1367052643485e-08
3527 1.13186997237058e-08
3528 1.14124609948973e-08
3529 1.12919896196884e-08
3530 1.13381804152102e-08
3531 1.13997073383221e-08
3532 1.12375090719485e-08
3533 1.14527475296722e-08
3534 1.1240949424618e-08
3535 1.13786500774982e-08
3536 1.13449880245753e-08
3537 1.12690242047853e-08
3538 1.13410149473747e-08
3539 1.12942434038477e-08
3540 1.12565459231551e-08
3541 1.12567011363307e-08
3542 1.13292356759276e-08
3543 1.12749140849244e-08
3544 1.14025525396189e-08
3545 1.13821397105252e-08
3546 1.13348179904271e-08
3547 1.13561017995867e-08
3548 1.1350831696344e-08
3549 1.14455885472653e-08
3550 1.12000564007619e-08
3551 1.1296409570094e-08
3552 1.11943729098618e-08
3553 1.14064156360172e-08
3554 1.11494709790216e-08
3555 1.12130198550719e-08
3556 1.12556735406555e-08
3557 1.11659621194748e-08
3558 1.12883001623931e-08
3559 1.1437866712205e-08
3560 1.13715104579359e-08
3561 1.12426579318425e-08
3562 1.12894665125207e-08
3563 1.11848778020507e-08
3564 1.1249472149033e-08
3565 1.12485956881292e-08
3566 1.14340501186083e-08
3567 1.1337988854665e-08
3568 1.13308049273231e-08
3569 1.12650662290603e-08
3570 1.125084578113e-08
3571 1.12495631648946e-08
3572 1.13250239689444e-08
3573 1.12160220320234e-08
3574 1.12324548555698e-08
3575 1.12101002274212e-08
3576 1.11704795233702e-08
3577 1.14110854756433e-08
3578 1.13200259689306e-08
3579 1.13873427669553e-08
3580 1.11012453466452e-08
3581 1.12853127038903e-08
3582 1.12442979022243e-08
3583 1.11240085047815e-08
3584 1.12356080977927e-08
3585 1.13213141115232e-08
3586 1.12688868612043e-08
3587 1.12058233918688e-08
3588 1.12852916765993e-08
3589 1.12023264260408e-08
3590 1.13156630987543e-08
3591 1.12122271407333e-08
3592 1.12773198782667e-08
3593 1.14075036375949e-08
3594 1.10806595969848e-08
3595 1.10955691644543e-08
3596 1.11606412656062e-08
3597 1.13199104222472e-08
3598 1.11434142199274e-08
3599 1.13359353192122e-08
3600 1.11590868654421e-08
3601 1.11277741636773e-08
3602 1.12190129902823e-08
3603 1.11387492103265e-08
3604 1.12902166711226e-08
3605 1.12506559672987e-08
3606 1.13137897345128e-08
3607 1.11063434535819e-08
3608 1.1236132868464e-08
3609 1.1312379917805e-08
3610 1.11519425577411e-08
3611 1.11484318897626e-08
3612 1.11723257182339e-08
3613 1.12637568554508e-08
3614 1.11565736994335e-08
3615 1.11928757725588e-08
3616 1.12558891622871e-08
3617 1.10638572369259e-08
3618 1.12057430291523e-08
3619 1.11764174460793e-08
3620 1.12380682746327e-08
3621 1.10983551562249e-08
3622 1.11329374526958e-08
3623 1.12641046434714e-08
3624 1.11132982654194e-08
3625 1.1106776179215e-08
3626 1.12444722636385e-08
3627 1.09870580731197e-08
3628 1.11870515479007e-08
3629 1.11244775220509e-08
3630 1.10766631943315e-08
3631 1.11832848115334e-08
3632 1.09995312033151e-08
3633 1.11248998970748e-08
3634 1.11907743438611e-08
3635 1.10974128814112e-08
3636 1.12871339136289e-08
3637 1.11303588559153e-08
3638 1.1059693947546e-08
3639 1.10744796696371e-08
3640 1.10992376988195e-08
3641 1.11145530665713e-08
3642 1.10901890704174e-08
3643 1.11637755076943e-08
3644 1.11233646171716e-08
3645 1.09358569160278e-08
3646 1.13097052459876e-08
3647 1.11522604958614e-08
3648 1.10052510556846e-08
3649 1.12150446214265e-08
3650 1.10272163638125e-08
3651 1.11788222130205e-08
3652 1.11234158177709e-08
3653 1.11475927389026e-08
3654 1.104735662949e-08
3655 1.09908905069345e-08
3656 1.10896092604396e-08
3657 1.10805031663386e-08
3658 1.12711148648748e-08
3659 1.10934338185542e-08
3660 1.10537595875426e-08
3661 1.1141215163879e-08
3662 1.10835910320795e-08
3663 1.12249724805169e-08
3664 1.11560492399576e-08
3665 1.09352895972803e-08
3666 1.10792386132097e-08
3667 1.10857880011306e-08
3668 1.11223034714447e-08
3669 1.10669331289115e-08
3670 1.11947880349073e-08
3671 1.09623065996933e-08
3672 1.10035263062302e-08
3673 1.092437383865e-08
3674 1.11366366711829e-08
3675 1.10623491951234e-08
3676 1.14359212113335e-08
3677 1.09610931564585e-08
3678 1.10227751891623e-08
3679 1.11065337929928e-08
3680 1.09973735595847e-08
3681 1.10639168363624e-08
3682 1.10846906454798e-08
3683 1.09221177507779e-08
3684 1.1140073666982e-08
3685 1.09281217344215e-08
3686 1.08707264692143e-08
3687 1.1213734112836e-08
3688 1.09178519761555e-08
3689 1.11070250818912e-08
3690 1.11130392496106e-08
3691 1.10197683682589e-08
3692 1.10965319439771e-08
3693 1.09763450755196e-08
3694 1.1128469448396e-08
3695 1.09765416026475e-08
3696 1.11326862950456e-08
3697 1.0940846743579e-08
3698 1.10710942490266e-08
3699 1.11177670193419e-08
3700 1.10422895119866e-08
3701 1.10320337617376e-08
3702 1.09692051271093e-08
3703 1.11084014846607e-08
3704 1.09106208937559e-08
3705 1.09675180162139e-08
3706 1.10971802341764e-08
3707 1.09930399881941e-08
3708 1.09540430874855e-08
3709 1.10402721604785e-08
3710 1.09608701357455e-08
3711 1.11777522545653e-08
3712 1.08526465936176e-08
3713 1.10583560472e-08
3714 1.0950658328901e-08
3715 1.08996732536282e-08
3716 1.11340053338127e-08
3717 1.09768914465791e-08
3718 1.08930704492138e-08
3719 1.09644200287917e-08
3720 1.10686481827882e-08
3721 1.08825986543382e-08
3722 1.08920607677643e-08
3723 1.10502843729821e-08
3724 1.09033852726537e-08
3725 1.08011608977998e-08
3726 1.09985006432334e-08
3727 1.08696884858483e-08
3728 1.09376907498904e-08
3729 1.08881612355116e-08
3730 1.10587662016659e-08
3731 1.10244627558043e-08
3732 1.10486545918942e-08
3733 1.09034771386129e-08
3734 1.0920275646753e-08
3735 1.08579959419153e-08
3736 1.10136727878851e-08
3737 1.09928440825691e-08
3738 1.09400618537592e-08
3739 1.08564756970964e-08
3740 1.10160673775228e-08
3741 1.08078498834985e-08
3742 1.08474952007498e-08
3743 1.08673404735082e-08
3744 1.08583503772852e-08
3745 1.09025990496781e-08
3746 1.0843015249784e-08
3747 1.11491370612393e-08
3748 1.08594800845019e-08
3749 1.0854328402421e-08
3750 1.09674125434722e-08
3751 1.09495455619113e-08
3752 1.07879479771533e-08
3753 1.0795940355135e-08
3754 1.10595239932731e-08
3755 1.0970226502649e-08
3756 1.09473568096208e-08
3757 1.08707489108184e-08
3758 1.09032670448928e-08
3759 1.08082071514914e-08
3760 1.09943345731889e-08
3761 1.07771279913571e-08
3762 1.09808414699986e-08
3763 1.10086036892998e-08
3764 1.09080072664414e-08
3765 1.08618981766728e-08
3766 1.0939159620138e-08
3767 1.08247682522045e-08
3768 1.08426656409977e-08
3769 1.08428788408999e-08
3770 1.08664731930386e-08
3771 1.09354033440701e-08
3772 1.08423632717569e-08
3773 1.07416601348165e-08
3774 1.08741286740655e-08
3775 1.08188052480029e-08
3776 1.07572827547919e-08
3777 1.09671009885792e-08
3778 1.08877276496777e-08
3779 1.08417535547023e-08
3780 1.08188208836069e-08
3781 1.08835078252989e-08
3782 1.07463201581837e-08
3783 1.08710297255232e-08
3784 1.07908934889878e-08
3785 1.08636543144636e-08
3786 1.09079918700905e-08
3787 1.08805675305135e-08
3788 1.08422985530821e-08
3789 1.09595906159221e-08
3790 1.08302748237366e-08
3791 1.074407268431e-08
3792 1.08847786275401e-08
3793 1.0957197388306e-08
3794 1.08682180883779e-08
3795 1.09360256710378e-08
3796 1.08438371585473e-08
3797 1.08062466426251e-08
3798 1.08089259402977e-08
3799 1.07852600421676e-08
3800 1.09298326926854e-08
3801 1.08779953319482e-08
3802 1.09727603645471e-08
3803 1.08535898732942e-08
3804 1.06551754625039e-08
3805 1.09215675688779e-08
3806 1.07351288809454e-08
3807 1.08972292175435e-08
3808 1.08099961884101e-08
3809 1.07851001494019e-08
3810 1.10072718059184e-08
3811 1.07522467448495e-08
3812 1.07260150277177e-08
3813 1.08281430485357e-08
3814 1.0779037328823e-08
3815 1.07957369249734e-08
3816 1.08182051433703e-08
3817 1.09125942685528e-08
3818 1.07868145295953e-08
3819 1.0747845791248e-08
3820 1.06857124000381e-08
3821 1.08333250308235e-08
3822 1.08050436490315e-08
3823 1.08400901732697e-08
3824 1.08442985923274e-08
3825 1.07450582806923e-08
3826 1.07164520357639e-08
3827 1.08707679815634e-08
3828 1.07300134930544e-08
3829 1.07309483968754e-08
3830 1.08314681811494e-08
3831 1.07978808493447e-08
3832 1.08616539976625e-08
3833 1.07699007003692e-08
3834 1.07937115713108e-08
3835 1.066489327195e-08
3836 1.08749325461455e-08
3837 1.06550005450901e-08
3838 1.07583081107032e-08
3839 1.06689405799942e-08
3840 1.08368041036311e-08
3841 1.06259561529765e-08
3842 1.07703825618088e-08
3843 1.07532418688328e-08
3844 1.0687957395783e-08
3845 1.0693758934921e-08
3846 1.07521308755354e-08
3847 1.07404054868754e-08
3848 1.0706181421738e-08
3849 1.06882111071682e-08
3850 1.08903723413656e-08
3851 1.08122772186636e-08
3852 1.07100657821269e-08
3853 1.06539830637686e-08
3854 1.07255326495803e-08
3855 1.06345872816771e-08
3856 1.0670064209628e-08
3857 1.06834370409148e-08
3858 1.062020638809e-08
3859 1.0751135038678e-08
3860 1.07361704663189e-08
3861 1.06709888919587e-08
3862 1.07370957644903e-08
3863 1.08456199305884e-08
3864 1.05328838078123e-08
3865 1.07803439378662e-08
3866 1.04257315668432e-08
3867 1.07812738884272e-08
3868 1.06055087891743e-08
3869 1.06215173439894e-08
3870 1.07579872345998e-08
3871 1.07545678392773e-08
3872 1.06813205451584e-08
3873 1.06148907098591e-08
3874 1.07748673172647e-08
3875 1.06855317693055e-08
3876 1.06962037573766e-08
3877 1.05074663985238e-08
3878 1.06697579099757e-08
3879 1.06539424760133e-08
3880 1.05210131114974e-08
3881 1.06947266768076e-08
3882 1.07632675167002e-08
3883 1.07455515887533e-08
3884 1.06941729609567e-08
3885 1.0694117636767e-08
3886 1.05127730851695e-08
3887 1.06380141147167e-08
3888 1.0513768066267e-08
3889 1.0675129199389e-08
3890 1.07080167036644e-08
3891 1.07235292664543e-08
3892 1.06614473568412e-08
3893 1.07664677591046e-08
3894 1.06185825771155e-08
3895 1.05964082407484e-08
3896 1.06311196286235e-08
3897 1.05031823451895e-08
3898 1.07000736269125e-08
3899 1.05006050783452e-08
3900 1.06400850508059e-08
3901 1.05436554754812e-08
3902 1.056071560257e-08
3903 1.06526954460895e-08
3904 1.0631245291659e-08
3905 1.05320502976625e-08
3906 1.06615443585811e-08
3907 1.06270451549761e-08
3908 1.04466145278703e-08
3909 1.06531909325192e-08
3910 1.07244830327424e-08
3911 1.05426733845171e-08
3912 1.07175501179446e-08
3913 1.05907129959171e-08
3914 1.07405469763622e-08
3915 1.05138343253763e-08
3916 1.05616021807142e-08
3917 1.05044153945189e-08
3918 1.05555747641484e-08
3919 1.05447358631494e-08
3920 1.03711858460143e-08
3921 1.05873773611487e-08
3922 1.06085076688744e-08
3923 1.04708496082573e-08
3924 1.10005706743799e-08
3925 1.06987948176318e-08
3926 1.05816768218858e-08
3927 1.0428493256831e-08
3928 1.05680495955385e-08
3929 1.05746540104423e-08
3930 1.03786356321356e-08
3931 1.0409160270175e-08
3932 1.06581347739665e-08
3933 1.05147694483554e-08
3934 1.05815891870975e-08
3935 1.05312051288387e-08
3936 1.05784928381425e-08
3937 1.05050153457187e-08
3938 1.05530321375946e-08
3939 1.05701593959839e-08
3940 1.058644866192e-08
3941 1.06626376951136e-08
3942 1.04121902346765e-08
3943 1.06748618163843e-08
3944 1.05605704462386e-08
3945 1.0524763645714e-08
3946 1.04952799401259e-08
3947 1.06820577175926e-08
3948 1.04440126991179e-08
3949 1.05620675588947e-08
3950 1.05531527060387e-08
3951 1.04307558043137e-08
3952 1.06481012098758e-08
3953 1.038060701819e-08
3954 1.06221125877326e-08
3955 1.04530573915573e-08
3956 1.07036377112735e-08
3957 1.04436420620413e-08
3958 1.06919258842098e-08
3959 1.04388231269015e-08
3960 1.04635600366443e-08
3961 1.05559398171362e-08
3962 1.04899711742545e-08
3963 1.03654421050869e-08
3964 1.03851385192533e-08
3965 1.0727255508991e-08
3966 1.04816073946967e-08
3967 1.05864603717754e-08
3968 1.03652009777466e-08
3969 1.05901338505188e-08
3970 1.05647818201549e-08
3971 1.03724903727231e-08
3972 1.05189202666556e-08
3973 1.05853479215323e-08
3974 1.05241426193725e-08
3975 1.05650826544013e-08
3976 1.06316332417755e-08
3977 1.04926797678262e-08
3978 1.05583357475902e-08
3979 1.05804066903303e-08
3980 1.05431056038885e-08
3981 1.07093656218638e-08
3982 1.04751819829385e-08
3983 1.04849324527034e-08
3984 1.05713746242353e-08
3985 1.05584778054002e-08
3986 1.05296445342962e-08
3987 1.061413325254e-08
3988 1.04226050553491e-08
3989 1.043630179276e-08
3990 1.0465477950139e-08
3991 1.04035999516361e-08
3992 1.04140313432755e-08
3993 1.04104082095935e-08
3994 1.05619237812382e-08
3995 1.04259065443202e-08
3996 1.05118369042589e-08
3997 1.04523092896436e-08
3998 1.05340644515062e-08
3999 1.04478229123739e-08
4000 1.05345138416979e-08
4001 1.05322071574099e-08
4002 1.04228600263934e-08
4003 1.02879746012308e-08
4004 1.05028007946206e-08
4005 1.05935411341962e-08
4006 1.05378882151452e-08
4007 1.0274399517618e-08
4008 1.0368184944709e-08
4009 1.049217650817e-08
4010 1.04512730929596e-08
4011 1.03469080602325e-08
4012 1.03504376403052e-08
4013 1.03993618085774e-08
4014 1.03750475846098e-08
4015 1.06110367198431e-08
4016 1.0264443947805e-08
4017 1.04514598626704e-08
4018 1.04891535421814e-08
4019 1.04503479619877e-08
4020 1.03846667823815e-08
4021 1.05046302648071e-08
4022 1.03517109157902e-08
4023 1.02720210740093e-08
4024 1.04024233968936e-08
4025 1.04865723097314e-08
4026 1.06157818079433e-08
4027 1.04546697823338e-08
4028 1.0331873688818e-08
4029 1.04735675815482e-08
4030 1.0571988518282e-08
4031 1.04967566626479e-08
4032 1.02958204221126e-08
4033 1.02147829661448e-08
4034 1.04025411417075e-08
4035 1.05363968584404e-08
4036 1.02170117783906e-08
4037 1.04167346798856e-08
4038 1.0639966524062e-08
4039 1.0500831911564e-08
4040 1.04759080272743e-08
4041 1.01832587938722e-08
4042 1.033268952505e-08
4043 1.03508394088125e-08
4044 1.04093326745991e-08
4045 1.02807544301586e-08
4046 1.0353337179192e-08
4047 1.04398108907722e-08
4048 1.03306165597505e-08
4049 1.03351106339744e-08
4050 1.03191973427341e-08
4051 1.04559523691483e-08
4052 1.03354314787141e-08
4053 1.0261214707985e-08
4054 1.03727380198482e-08
4055 1.04526641633296e-08
4056 1.03324596009147e-08
4057 1.02954500157404e-08
4058 1.04249189389893e-08
4059 1.04108682147386e-08
4060 1.02843339155889e-08
4061 1.02358576488104e-08
4062 1.04873259614835e-08
4063 1.02232585650341e-08
4064 1.03018758531581e-08
4065 1.03393762693194e-08
4066 1.03044683398279e-08
4067 1.02525863372449e-08
4068 1.03823078696497e-08
4069 1.03026474288503e-08
4070 1.02847272046569e-08
4071 1.02947173113987e-08
4072 1.05002417173372e-08
4073 1.0331049994261e-08
4074 1.04655853803703e-08
4075 1.02876100452898e-08
4076 1.02962308586863e-08
4077 1.03940821783999e-08
4078 1.02640754627292e-08
4079 1.03724080662282e-08
4080 1.0293413462481e-08
4081 1.01888033032793e-08
4082 1.02630857937713e-08
4083 1.03053743362724e-08
4084 1.03316170078083e-08
4085 1.03460008727962e-08
4086 1.03326892731959e-08
4087 1.04660896193542e-08
4088 1.02330560493025e-08
4089 1.03720165000043e-08
4090 1.01971753992958e-08
4091 1.03235422297399e-08
4092 1.04422288041972e-08
4093 1.05136939803074e-08
4094 1.02634807223057e-08
4095 1.02878426996855e-08
4096 1.02510779248499e-08
4097 1.02269906083752e-08
4098 1.02160602540868e-08
4099 1.0461197928191e-08
4100 1.02568497367672e-08
4101 1.0244973545348e-08
4102 1.03091471908012e-08
4103 1.04334456131894e-08
4104 1.03644006151926e-08
4105 1.04181655282698e-08
4106 1.03001338629394e-08
4107 1.03733417808893e-08
4108 1.0254335622073e-08
4109 1.05113713856353e-08
4110 1.0318097804829e-08
4111 1.01048659720826e-08
4112 1.03570749161452e-08
4113 1.02058777491365e-08
4114 1.05661234599541e-08
4115 1.02554611155492e-08
4116 1.0328385972258e-08
4117 1.02475934370405e-08
4118 1.02490854867465e-08
4119 1.01883802366975e-08
4120 1.02696392710877e-08
4121 1.04198907157627e-08
4122 1.01074227804032e-08
4123 1.03515762877593e-08
4124 1.03399003082982e-08
4125 1.03206573560666e-08
4126 1.03277293332926e-08
4127 1.03149691421667e-08
4128 1.02804048102145e-08
4129 1.01788064756358e-08
4130 1.0401334295862e-08
4131 1.02045510427806e-08
4132 1.02895212132914e-08
4133 1.02948943186965e-08
4134 1.03056636023835e-08
4135 1.01040914899375e-08
4136 1.03395489553515e-08
4137 1.03483959249528e-08
4138 1.01508139982442e-08
4139 1.02611279849674e-08
4140 1.04336788885329e-08
4141 1.01922082499373e-08
4142 1.04017291578939e-08
4143 1.03336370618767e-08
4144 1.02406946652644e-08
4145 1.02414787914729e-08
4146 1.0214895423688e-08
4147 1.01558859767481e-08
4148 1.0106952489819e-08
4149 1.02660620279194e-08
4150 1.0149515587754e-08
4151 1.04644894900474e-08
4152 1.02514085790206e-08
4153 1.02520754194924e-08
4154 1.02257553855623e-08
4155 1.02183231458497e-08
4156 1.03822028641454e-08
4157 1.01854297198667e-08
4158 1.02505331724956e-08
4159 1.02611531608843e-08
4160 1.03520562014792e-08
4161 1.02985477228312e-08
4162 1.02256466748551e-08
4163 9.9901643677569e-09
4164 1.03116485552501e-08
4165 1.0245550247201e-08
4166 1.02494211893234e-08
4167 1.02178089580462e-08
4168 1.02563409442036e-08
4169 1.0206710092775e-08
4170 1.03693935289417e-08
4171 1.02433656909984e-08
4172 1.02961453961603e-08
4173 1.02008576519097e-08
4174 1.0141921266249e-08
4175 1.02238851575898e-08
4176 1.0259876449259e-08
4177 1.02628290677975e-08
4178 1.01901917986535e-08
4179 1.0173172744099e-08
4180 1.02552436457293e-08
4181 1.02345317577468e-08
4182 1.02955662006354e-08
4183 1.03000694742228e-08
4184 1.02548575175998e-08
4185 1.01922659465625e-08
4186 1.03667995591805e-08
4187 1.03034991628115e-08
4188 1.00309787228747e-08
4189 1.02389891526444e-08
4190 1.02912933658272e-08
4191 1.02917859939877e-08
4192 1.03071177408043e-08
4193 1.0186932508871e-08
4194 1.04370429162559e-08
4195 9.91608802342059e-09
4196 1.01923078063004e-08
4197 1.02235337331447e-08
4198 1.05051355029362e-08
4199 1.02759278665099e-08
4200 1.01557185518963e-08
4201 1.010560072634e-08
4202 1.0378080280149e-08
4203 1.01431045073097e-08
4204 1.01926166538036e-08
4205 1.02538940182084e-08
4206 1.01731362927016e-08
4207 1.02172597626904e-08
4208 1.03666333794994e-08
4209 1.03582790665913e-08
4210 1.02748220670579e-08
4211 1.01923511185431e-08
4212 1.01606868891935e-08
4213 1.00367118903155e-08
4214 1.01526236458982e-08
4215 1.00875122783783e-08
4216 1.01512156049921e-08
4217 1.02490714876113e-08
4218 1.01096776483622e-08
4219 1.01099803601068e-08
4220 1.01486460492017e-08
4221 1.01372758207763e-08
4222 1.0122144662672e-08
4223 1.00512413571119e-08
4224 1.00414662023152e-08
4225 1.00816728642039e-08
4226 1.01340311760012e-08
4227 1.01150931738347e-08
4228 1.00955081297482e-08
4229 1.01364327341757e-08
4230 1.00130171928337e-08
4231 9.9308107345264e-09
4232 1.01438307644197e-08
4233 1.02353824522616e-08
4234 1.00284545042628e-08
4235 1.00301070312669e-08
4236 1.01895259385087e-08
4237 1.01475679651353e-08
4238 1.01082569917699e-08
4239 1.00794010569616e-08
4240 1.00368574845744e-08
4241 1.00774181976959e-08
4242 1.01829857282465e-08
4243 1.00017405723074e-08
4244 1.00605234859552e-08
4245 1.01651189665652e-08
4246 9.96236638994841e-09
4247 1.01471954315757e-08
4248 1.00079117935636e-08
4249 1.01224013115409e-08
4250 1.00377736831359e-08
4251 1.00557044571126e-08
4252 1.00916276366547e-08
4253 1.01140592143611e-08
4254 1.0048538696128e-08
4255 1.00905711182819e-08
4256 1.02334883458344e-08
4257 9.90102863079878e-09
4258 1.02138553396691e-08
4259 1.017939307002e-08
4260 1.0048467527668e-08
4261 1.02416874552258e-08
4262 1.01836352488505e-08
4263 1.0204119285262e-08
4264 1.00475635423924e-08
4265 1.00569602879963e-08
4266 1.00699731274134e-08
4267 1.00711511330509e-08
4268 9.97773070066099e-09
4269 1.01174357252098e-08
4270 1.02316801179336e-08
4271 1.00893130178248e-08
4272 1.01182773193931e-08
4273 9.99419312835981e-09
4274 1.01217771144579e-08
4275 1.00605921138364e-08
4276 9.9923640123345e-09
4277 9.93522338493813e-09
4278 1.01188718705791e-08
4279 9.94200981180304e-09
4280 1.02416567410168e-08
4281 1.00685577365534e-08
4282 1.00530603757143e-08
4283 1.01934349684973e-08
4284 9.93651786385108e-09
4285 1.02716100850997e-08
4286 9.88423356051982e-09
4287 1.03612936471764e-08
4288 9.99847130478404e-09
4289 1.01558974375804e-08
4290 9.95599884767717e-09
4291 1.02139248318034e-08
4292 1.00772919159331e-08
4293 1.01149257836775e-08
4294 9.93861131182161e-09
4295 1.01072502268762e-08
4296 1.00862530883483e-08
4297 1.01275372415e-08
4298 9.99482546359953e-09
4299 9.93953367961487e-09
4300 1.02064637201305e-08
4301 9.996466035056e-09
4302 9.99532823059424e-09
4303 1.000994296807e-08
4304 1.00185487437177e-08
4305 1.00966451679896e-08
4306 9.95302468753678e-09
4307 1.00783409664462e-08
4308 1.01338561652176e-08
4309 9.83479304955681e-09
4310 1.0305187404136e-08
4311 9.86353442666843e-09
4312 1.00430632374304e-08
4313 9.98076436103945e-09
4314 1.01004762504964e-08
4315 1.007606366632e-08
4316 1.0486364848683e-08
4317 9.70791626564438e-09
4318 1.02291578891744e-08
4319 9.96055174551369e-09
4320 1.0139774740181e-08
4321 1.01384518846892e-08
4322 1.00232093115937e-08
4323 1.01296753060032e-08
4324 9.9254125490722e-09
4325 1.0303652451582e-08
4326 9.78204950308204e-09
4327 9.99693260733681e-09
4328 9.98826594650337e-09
4329 1.01122726599012e-08
4330 9.84464320552547e-09
4331 1.01882126593011e-08
4332 9.96689381793647e-09
4333 1.00661590108908e-08
4334 1.03383666139045e-08
4335 9.71268972060901e-09
4336 1.00864391362476e-08
4337 9.86682167269937e-09
4338 9.97895456367193e-09
4339 1.00233946097039e-08
4340 9.98934464196966e-09
4341 1.01246046180781e-08
4342 1.0040653789567e-08
4343 1.00742596910175e-08
4344 1.01637648535768e-08
4345 9.98378580036041e-09
4346 1.00331546220578e-08
4347 9.96594350277658e-09
4348 1.00161595232295e-08
4349 1.01278227333523e-08
4350 9.86259156832148e-09
4351 1.02295279723608e-08
4352 9.84014023031055e-09
4353 1.00538903496927e-08
4354 9.82255425574507e-09
4355 1.02103081331406e-08
4356 9.9090940686164e-09
4357 1.00898850146081e-08
4358 1.00644717012099e-08
4359 9.9912432992566e-09
4360 1.0102455811456e-08
4361 9.99360967357532e-09
4362 1.0164436004545e-08
4363 1.01562495127272e-08
4364 9.78600156542608e-09
4365 1.00306503074687e-08
4366 1.0005557702697e-08
4367 1.00867103374203e-08
4368 1.00457697937428e-08
4369 9.93200895149649e-09
4370 1.01314583440537e-08
4371 9.88944367841249e-09
4372 9.92286585294044e-09
4373 9.95825285532437e-09
4374 1.00076993244103e-08
4375 1.01595238410912e-08
4376 1.00441294754727e-08
4377 9.93935279369751e-09
4378 9.8466376599049e-09
4379 1.0236482041126e-08
4380 9.77309064498222e-09
4381 9.98622145087502e-09
4382 1.00022764315533e-08
4383 1.015632064294e-08
4384 9.73910300650083e-09
4385 9.90678091417063e-09
4386 1.00351038148649e-08
4387 9.92259916604565e-09
4388 1.00229760294779e-08
4389 9.9544555015596e-09
4390 1.00218224902138e-08
4391 9.9186558276676e-09
4392 1.00684098298665e-08
4393 9.9240438519721e-09
4394 9.94777569524041e-09
4395 9.93444971025825e-09
4396 1.00153510903533e-08
4397 9.96288869137274e-09
4398 9.85227717203996e-09
4399 9.84442500545635e-09
4400 1.00453939309575e-08
4401 9.85350031562637e-09
4402 1.00734750632103e-08
4403 9.85924679336447e-09
4404 9.82793731235665e-09
4405 9.94689114952596e-09
4406 9.92747220829315e-09
4407 9.95942390746851e-09
4408 9.91936546573813e-09
4409 9.82640210911434e-09
4410 9.96609929959646e-09
4411 9.86227109478488e-09
4412 1.02002749438057e-08
4413 9.9072238963771e-09
4414 9.96528632618565e-09
4415 9.93141708804934e-09
4416 9.79412783191913e-09
4417 9.88612932073529e-09
4418 9.84266540449141e-09
4419 9.86087783094725e-09
4420 9.9247438315464e-09
4421 9.98011917485719e-09
4422 9.91965499380631e-09
4423 9.83767151852e-09
4424 9.89112595811692e-09
4425 9.95954956883871e-09
4426 9.92580452408021e-09
4427 1.02052967208555e-08
4428 9.73488022948388e-09
4429 9.73967688006061e-09
4430 9.94049925739926e-09
4431 1.00573959219141e-08
4432 9.85646598683498e-09
4433 9.99460009670772e-09
4434 9.78407853724494e-09
4435 1.01542158768009e-08
4436 9.95888030208558e-09
4437 9.72806725257902e-09
4438 9.88149804465044e-09
4439 9.85214862625483e-09
4440 9.82894147755209e-09
4441 9.87483834885561e-09
4442 9.8423803326364e-09
4443 1.00325817952163e-08
4444 9.87666006291299e-09
4445 9.91623405838249e-09
4446 9.82504794011518e-09
4447 9.98875042745873e-09
4448 9.85601572889294e-09
4449 9.83267159349044e-09
4450 9.96004807019002e-09
4451 9.88051812317314e-09
4452 9.88297453874409e-09
4453 9.86084388532316e-09
4454 1.02925916403152e-08
4455 9.76169855954057e-09
4456 9.9264703254387e-09
4457 1.00679976202667e-08
4458 9.89934991513231e-09
4459 9.92820778672465e-09
4460 9.91290646967791e-09
4461 9.85953065507639e-09
4462 9.90595564603902e-09
4463 9.96183417928398e-09
4464 9.82337563670832e-09
4465 1.00314381631939e-08
4466 9.699270473007e-09
4467 9.83092116518813e-09
4468 9.93293804413264e-09
4469 1.00400836622327e-08
4470 9.84881107712887e-09
4471 9.79171623061159e-09
4472 9.88147254976646e-09
4473 9.81768349966883e-09
4474 9.87254016315831e-09
4475 9.9210378990966e-09
4476 9.81827764195975e-09
4477 9.88973262688875e-09
4478 9.74109107343413e-09
4479 1.00055967914292e-08
4480 9.83333857829116e-09
4481 9.90957614888099e-09
4482 9.71079486355997e-09
4483 9.77007094044024e-09
4484 9.93045896308109e-09
4485 9.7645670218971e-09
4486 9.84984966723168e-09
4487 9.85716973023187e-09
4488 9.87844093003343e-09
4489 9.80652769305701e-09
4490 9.77496448134607e-09
4491 9.87702645555144e-09
4492 9.78371219118035e-09
4493 9.75642251899167e-09
4494 9.8768189586429e-09
4495 9.91213254014633e-09
4496 9.88088052839009e-09
4497 9.72014290162582e-09
4498 9.79501085895951e-09
4499 9.75549136894571e-09
4500 9.90787894028511e-09
4501 9.99190189149291e-09
4502 9.79824372843163e-09
4503 9.8414611908737e-09
4504 9.9012137657617e-09
4505 9.81230727553628e-09
4506 9.79234895898129e-09
4507 9.92639839036968e-09
4508 9.86833848654056e-09
4509 9.66475204161599e-09
4510 9.79321367983177e-09
4511 9.83179257391109e-09
4512 9.73851866103992e-09
4513 9.86969771415991e-09
4514 9.80520877991165e-09
4515 9.87234920923896e-09
4516 9.96684783977075e-09
4517 9.65544869768742e-09
4518 9.94459812370119e-09
4519 9.82511679570308e-09
4520 9.71593714593766e-09
4521 9.87910649663526e-09
4522 9.79919568250276e-09
4523 9.62482401872799e-09
4524 9.78041961507969e-09
4525 9.80654576432372e-09
4526 9.79158949732106e-09
4527 9.72538558713376e-09
4528 9.8001764863298e-09
4529 9.80659275162665e-09
4530 9.78145428015553e-09
4531 9.75462031449492e-09
4532 9.9510400545233e-09
4533 9.7550267181834e-09
4534 9.80811634049283e-09
4535 9.82880948174891e-09
4536 9.80643977493978e-09
4537 9.72555711509226e-09
4538 9.91386257009896e-09
4539 9.78862712941453e-09
4540 9.77038582350431e-09
4541 9.79249594867948e-09
4542 9.69411709406121e-09
4543 9.8477527926133e-09
4544 9.73321904829127e-09
4545 9.68606428997099e-09
4546 9.84257627056895e-09
4547 9.57576670046478e-09
4548 9.94586122832208e-09
4549 9.74869910408405e-09
4550 9.81343752481933e-09
4551 9.8172424033427e-09
4552 9.67894314984896e-09
4553 9.83076062199828e-09
4554 9.72825449724324e-09
4555 9.78468866641435e-09
4556 9.94480406896203e-09
4557 9.79473264278718e-09
4558 9.79199677697551e-09
4559 9.63425794836548e-09
4560 9.79831089625849e-09
4561 9.75542789277695e-09
4562 9.75330906549665e-09
4563 9.7359889082993e-09
4564 1.00067804180703e-08
4565 9.54415213239024e-09
4566 9.81092201174372e-09
4567 9.76814915132351e-09
4568 9.78110300070556e-09
4569 9.67463911427657e-09
4570 9.79780457166513e-09
4571 9.72272732047141e-09
4572 9.94177240859484e-09
4573 9.96419546683835e-09
4574 9.59070455153199e-09
4575 9.87069510938188e-09
4576 9.72988334968505e-09
4577 9.77404851615349e-09
4578 9.7053528312907e-09
4579 9.83096036066833e-09
4580 9.78664867945511e-09
4581 9.71191849807873e-09
4582 9.77881328789731e-09
4583 9.77417135683556e-09
4584 9.79125413513859e-09
4585 9.60690013435128e-09
4586 9.78639749449339e-09
4587 9.50985413161742e-09
4588 9.92894405571487e-09
4589 9.67361136072764e-09
4590 9.76159008164723e-09
4591 9.73054432346077e-09
4592 9.81479203854274e-09
4593 9.70525287713464e-09
4594 9.7141244038701e-09
4595 9.69599474676119e-09
4596 9.77962978471192e-09
4597 9.74274858578372e-09
4598 9.54215035486872e-09
4599 9.71293234908188e-09
4600 9.84198307552031e-09
4601 9.59866835281709e-09
4602 9.64291228711112e-09
4603 9.75643018069627e-09
4604 9.7661699964835e-09
4605 9.76242880634803e-09
4606 9.75135667291571e-09
4607 9.70975830050236e-09
4608 9.71745425604187e-09
4609 9.86024334836966e-09
4610 9.64909356737653e-09
4611 9.79500791231658e-09
4612 9.71795485948945e-09
4613 9.81213564454908e-09
4614 9.77735064777718e-09
4615 9.586495672953e-09
4616 9.75021563476286e-09
4617 9.79786490418189e-09
4618 9.59302297487419e-09
4619 9.79652424148991e-09
4620 9.76727226009722e-09
4621 9.72076524830001e-09
4622 9.76557347442952e-09
4623 9.71447541675552e-09
4624 9.79076172774596e-09
4625 9.75975917616445e-09
4626 9.78489688341355e-09
4627 9.60886842760189e-09
4628 9.86766388777482e-09
4629 9.62966077755478e-09
4630 9.8333110868376e-09
4631 9.77758809816987e-09
4632 9.75033939365488e-09
4633 9.73056017211649e-09
4634 9.62337515431067e-09
4635 9.83279747857058e-09
4636 9.58842570786267e-09
4637 9.69842186809844e-09
4638 9.64833534089493e-09
4639 9.72822585554312e-09
4640 9.91500720448624e-09
4641 9.67566478693449e-09
4642 9.60135863570244e-09
4643 9.69763080410635e-09
4644 9.69669051964495e-09
4645 9.77181294875562e-09
4646 9.57460282113853e-09
4647 9.7868809766366e-09
4648 9.72921488018308e-09
4649 9.81264556576544e-09
4650 9.70102713693244e-09
4651 9.52609466153831e-09
4652 9.80033168673788e-09
4653 9.74583152646424e-09
4654 9.66952729009041e-09
4655 9.76996456336643e-09
4656 9.62088958778029e-09
4657 9.66107859545762e-09
4658 9.78307606924744e-09
4659 9.85861645419472e-09
4660 9.61261701848404e-09
4661 9.69008450679709e-09
4662 9.74111144602663e-09
4663 9.54460938951129e-09
4664 9.73031255691037e-09
4665 9.94309567226015e-09
4666 9.59792700017115e-09
4667 9.75466874408859e-09
4668 9.60713442355043e-09
4669 9.53113183255194e-09
4670 9.88714326766171e-09
4671 9.5392069254796e-09
4672 9.61314203518082e-09
4673 9.72451821279474e-09
4674 9.64675419096173e-09
4675 9.61104660468504e-09
4676 9.80964812391516e-09
4677 9.56193516943316e-09
4678 9.78603800272371e-09
4679 9.57174663362981e-09
4680 9.61534827548238e-09
4681 9.66995105111668e-09
4682 9.7736669577575e-09
4683 9.58378310905728e-09
4684 1.05183915561913e-08
4685 9.44382858647508e-09
4686 9.64635585976037e-09
4687 9.69279271789691e-09
4688 9.63775345053097e-09
4689 9.83160093281921e-09
4690 9.76544210018471e-09
4691 9.64781795742553e-09
4692 9.7171619513059e-09
4693 9.95686139793062e-09
4694 9.67924446493296e-09
4695 9.53463925024156e-09
4696 9.75949840853474e-09
4697 9.54809193121076e-09
4698 9.82528025450513e-09
4699 9.66822902292996e-09
4700 9.6479702351715e-09
4701 9.77862042350619e-09
4702 9.7424737924845e-09
4703 9.60666443006186e-09
4704 9.58579084453515e-09
4705 9.73785786251291e-09
4706 9.66173960670336e-09
4707 9.62244883606633e-09
4708 9.70326064086979e-09
4709 9.71026131257879e-09
4710 9.62683605998382e-09
4711 9.82467498022999e-09
4712 9.53867261133112e-09
4713 9.75503105182796e-09
4714 9.62715978680695e-09
4715 9.71542279770698e-09
4716 9.57309921867378e-09
4717 9.66523657786045e-09
4718 9.68201501660992e-09
4719 9.64113263990374e-09
4720 9.69935011407852e-09
4721 9.64348772414247e-09
4722 9.68455000610602e-09
4723 9.89855757405067e-09
4724 9.59489737067942e-09
4725 9.70003171085754e-09
4726 9.63150134491997e-09
4727 9.75759394161724e-09
4728 9.65745401759754e-09
4729 9.572733847385e-09
4730 9.7353470736028e-09
4731 9.75861330587691e-09
4732 9.63070423865009e-09
4733 9.6336719528356e-09
4734 9.58618204516259e-09
4735 9.91612725342872e-09
4736 9.6008067264064e-09
4737 9.77190747164514e-09
4738 9.59138140443905e-09
4739 9.67182945288414e-09
4740 9.95144704607487e-09
4741 9.58548510776591e-09
4742 9.61628043305574e-09
4743 9.62686350769459e-09
4744 9.69908888814874e-09
4745 9.83357642447835e-09
4746 9.61367295038151e-09
4747 9.57685460867275e-09
4748 9.65400571484309e-09
4749 9.68580061067037e-09
4750 9.68396961076312e-09
4751 9.69001702644245e-09
4752 9.67586727240732e-09
4753 9.63448528401578e-09
4754 9.77644999478056e-09
4755 9.71278232986261e-09
4756 9.56054592060607e-09
4757 9.58986312415977e-09
4758 9.75363583421984e-09
4759 9.585034945081e-09
4760 9.73242676827191e-09
4761 9.70225326002439e-09
4762 9.65783101214424e-09
4763 9.55261914592498e-09
4764 9.7665169904726e-09
4765 9.59117536003529e-09
4766 9.65782747563981e-09
4767 9.70667635569189e-09
4768 9.67199299095611e-09
4769 9.73653903135752e-09
4770 9.65418510978644e-09
4771 9.65306764177942e-09
4772 9.68807466117383e-09
4773 9.60262151550317e-09
4774 9.64816460335971e-09
4775 9.7477737176499e-09
4776 9.58156352526274e-09
4777 9.81876767447432e-09
4778 9.58006272544498e-09
4779 9.70155031998488e-09
4780 9.70575257330797e-09
4781 9.58045870647073e-09
4782 9.67900490700391e-09
4783 9.62847059637895e-09
4784 9.82403469595994e-09
4785 9.60212851863762e-09
4786 9.63816397048234e-09
4787 9.6624589062122e-09
4788 9.59708665593251e-09
4789 9.67632215986836e-09
4790 9.62288017181123e-09
4791 9.75619093790403e-09
4792 9.597841626241e-09
4793 9.63048234747799e-09
4794 9.68624007280372e-09
4795 9.59137187495074e-09
4796 9.69961222341276e-09
4797 9.67068272972327e-09
4798 9.59572866865432e-09
4799 9.66903625310156e-09
4800 9.52675394361169e-09
4801 9.68977817650263e-09
4802 9.56114898320593e-09
4803 9.67007988106339e-09
4804 9.73422836780813e-09
4805 9.68983801685752e-09
4806 9.65307449396491e-09
4807 9.59377245657311e-09
4808 9.62940474680263e-09
4809 9.59440422365621e-09
4810 9.81364277968799e-09
4811 1.0002564986844e-08
4812 9.48601130368321e-09
4813 9.62190450293221e-09
4814 9.69144059392857e-09
4815 9.51382006164625e-09
4816 9.69785334070661e-09
4817 9.53808322712302e-09
4818 9.56231523130402e-09
4819 9.57803533252477e-09
4820 9.65766547911251e-09
4821 9.58980436349677e-09
4822 9.66426098358841e-09
4823 9.61858955000894e-09
4824 9.62255937719814e-09
4825 9.69220932145465e-09
4826 9.5218866480451e-09
4827 9.64380901447015e-09
4828 9.69910745873825e-09
4829 9.53188683239237e-09
4830 9.60019091056363e-09
4831 9.50860291976108e-09
4832 9.69567752295841e-09
4833 9.60619259771089e-09
4834 9.70554818968683e-09
4835 9.51711318908366e-09
4836 9.64627180144451e-09
4837 9.66616048819757e-09
4838 9.53208797149774e-09
4839 9.60860324311641e-09
4840 9.69560235508649e-09
4841 9.58280059004668e-09
4842 9.59145865742084e-09
4843 9.72360849049725e-09
4844 9.64360691080302e-09
4845 9.55209373820764e-09
4846 9.81820593037863e-09
4847 9.58385905491799e-09
4848 9.97159471505249e-09
4849 9.38315640419685e-09
4850 9.53948748139144e-09
4851 9.58041415299871e-09
4852 9.68579576360318e-09
4853 9.74892696925522e-09
4854 9.3885563695606e-09
4855 9.80295980423929e-09
4856 9.64742240072702e-09
4857 9.83974660073716e-09
4858 9.45249686423732e-09
4859 9.85227621075335e-09
4860 9.39901997354387e-09
4861 9.99677526758447e-09
4862 9.30000438170708e-09
4863 9.82862109444316e-09
4864 9.3577256347066e-09
4865 9.84824956484776e-09
4866 9.48052046523262e-09
4867 9.72447062769222e-09
4868 9.66731087903572e-09
4869 9.51081906330753e-09
4870 9.85650629270474e-09
4871 9.43730659552777e-09
4872 9.58902498415792e-09
4873 9.76434755473044e-09
4874 9.56995263801108e-09
4875 9.55581092287883e-09
4876 1.00144320339668e-08
4877 9.28067219441164e-09
4878 9.84000607834323e-09
4879 9.32691754240711e-09
4880 9.61291840029244e-09
4881 9.60888585910258e-09
4882 9.58190084898725e-09
4883 9.58881809376422e-09
4884 9.53012087923088e-09
4885 9.64399848601971e-09
4886 9.56440646088241e-09
4887 9.60832361696085e-09
4888 9.56695523235407e-09
4889 9.5509762526591e-09
4890 9.6597949648114e-09
4891 9.64025602390173e-09
4892 9.58178030618928e-09
4893 9.76357740212208e-09
4894 9.46397111745156e-09
4895 9.62188742370529e-09
4896 9.6248662284637e-09
4897 9.59601533689902e-09
4898 9.3845100166412e-09
4899 9.74853933666253e-09
4900 9.29235590407806e-09
4901 9.65391492979695e-09
4902 9.88055174300229e-09
4903 9.50581614056478e-09
4904 9.3893028055847e-09
4905 9.72134755528042e-09
4906 9.50780869324319e-09
4907 9.46443938010244e-09
4908 9.58683311569342e-09
4909 9.7220985729729e-09
4910 9.53829005145845e-09
4911 9.62988218622396e-09
4912 9.42456096975075e-09
4913 9.66656607004346e-09
4914 9.44826391569009e-09
4915 9.7734434303387e-09
4916 9.52103949292127e-09
4917 9.60100389113538e-09
4918 9.40105390545565e-09
4919 9.6108021125918e-09
4920 9.45892297177231e-09
4921 9.60625522017366e-09
4922 9.59036940539892e-09
4923 9.50620483763931e-09
4924 9.55336447949762e-09
4925 9.5591361101155e-09
4926 9.55005181368929e-09
4927 9.61109955155415e-09
4928 9.46306369853378e-09
4929 9.73902069734134e-09
4930 9.74095851619117e-09
4931 9.51907438218047e-09
4932 9.49501193869828e-09
4933 9.60946998118661e-09
4934 9.42066990639034e-09
4935 9.57957855585168e-09
4936 9.60690820128729e-09
4937 9.4632985844223e-09
4938 9.54168199274186e-09
4939 9.46934691137269e-09
4940 9.63728103986483e-09
4941 9.39307943415013e-09
4942 9.51153306005814e-09
4943 9.51602618970426e-09
4944 9.50461699278371e-09
4945 9.49027587271445e-09
4946 9.53563321104411e-09
4947 9.62417414018812e-09
4948 9.53430245520526e-09
4949 9.4864614812451e-09
4950 9.63786034213765e-09
4951 9.44121519874663e-09
4952 9.70591654503306e-09
4953 9.30220595918296e-09
4954 9.8066651951223e-09
4955 9.44892883669723e-09
4956 9.44512810696629e-09
4957 9.59171397429959e-09
4958 9.71943614563875e-09
4959 9.30641663587917e-09
4960 9.626870070778e-09
4961 9.91363122559985e-09
4962 9.3296968913803e-09
4963 9.70738718636399e-09
4964 9.35723746819939e-09
4965 9.53260892694319e-09
4966 9.80220823598721e-09
4967 9.42916344637279e-09
4968 9.49533733612951e-09
4969 9.70209742207073e-09
4970 9.6665366441373e-09
4971 9.60744833033988e-09
4972 9.48881687101721e-09
4973 9.52938755871546e-09
4974 9.62558030065397e-09
4975 9.4370811462019e-09
4976 9.43868654790236e-09
4977 9.58103922671372e-09
4978 9.61664243492866e-09
4979 9.4553853349777e-09
4980 9.56988334654962e-09
4981 9.5500209201238e-09
4982 9.50675908667486e-09
4983 9.42991707214569e-09
4984 9.44360731075511e-09
4985 9.48039370202158e-09
4986 9.64477359510774e-09
4987 9.52399384523961e-09
4988 9.48740315276364e-09
4989 9.46469778329018e-09
4990 9.58956531627031e-09
4991 9.78096671877537e-09
4992 9.3168803442234e-09
4993 9.60643911795955e-09
4994 9.47015987395883e-09
4995 9.60025760649019e-09
4996 9.52531536807522e-09
4997 9.34926822587778e-09
4998 9.51997705334051e-09
4999 9.63257412206486e-09
};
\addlegendentry{Train}
\addplot [semithick, black]
table {%
0 0.00127039779908955
1 0.000149965009768493
2 8.17546533653513e-05
3 3.73395378119312e-05
4 3.01134696201188e-05
5 2.73718633252429e-05
6 2.30066034418996e-05
7 1.70099756360287e-05
8 1.17571007649531e-05
9 8.04488900030265e-06
10 5.93155709793791e-06
11 4.8917727326625e-06
12 4.24400832343963e-06
13 3.78305344383989e-06
14 3.38963150170457e-06
15 3.02348962577526e-06
16 2.47724301516428e-06
17 2.12976010516286e-06
18 1.91501817425888e-06
19 1.77405206613912e-06
20 1.67278881235688e-06
21 1.59747378347674e-06
22 1.54330939494685e-06
23 1.49572508689744e-06
24 1.45093815717701e-06
25 1.41425664423878e-06
26 1.38140887884219e-06
27 1.3493603319148e-06
28 1.31558192606462e-06
29 1.28497549667372e-06
30 1.25687063246005e-06
31 1.22943117730756e-06
32 1.20259824143432e-06
33 1.18032448881422e-06
34 1.15747934614774e-06
35 1.13645569399523e-06
36 1.11478527742292e-06
37 1.09417817384383e-06
38 1.07582764030667e-06
39 1.05782510217978e-06
40 1.04104594811361e-06
41 1.02015849279269e-06
42 1.00191948604333e-06
43 9.85044835033477e-07
44 9.6779945124581e-07
45 9.50062769788929e-07
46 9.31559384298453e-07
47 9.13231588128838e-07
48 8.9642531975187e-07
49 8.80041000073106e-07
50 8.64174126036232e-07
51 8.48156105348608e-07
52 8.32996363442362e-07
53 8.17361865301791e-07
54 8.03285217898519e-07
55 7.8890906252127e-07
56 7.76084448261827e-07
57 7.60517878006794e-07
58 7.4799021376748e-07
59 7.32621685983759e-07
60 7.20634091067041e-07
61 7.07904803221027e-07
62 6.96537199473823e-07
63 6.8358235694177e-07
64 6.72980092986109e-07
65 6.63917603560549e-07
66 6.54096083962941e-07
67 6.44372960323381e-07
68 6.35092533229908e-07
69 6.256652227421e-07
70 6.18590320300427e-07
71 6.11007749284909e-07
72 6.03203488935833e-07
73 5.96531890550978e-07
74 5.90364095387486e-07
75 5.8505423794486e-07
76 5.79990398819064e-07
77 5.75535750613199e-07
78 5.71535963445058e-07
79 5.67881727420172e-07
80 5.65467360047478e-07
81 5.62572438411735e-07
82 5.59569855340669e-07
83 5.56648160454642e-07
84 5.53847769424465e-07
85 5.50681647837337e-07
86 5.47908086900861e-07
87 5.45074726687744e-07
88 5.42706004580396e-07
89 5.39917607511597e-07
90 5.38202300504054e-07
91 5.356901056075e-07
92 5.335963919606e-07
93 5.31310377027694e-07
94 5.28847408531874e-07
95 5.27149950357853e-07
96 5.24995755313284e-07
97 5.22720881690475e-07
98 5.20846924700891e-07
99 5.19334719228937e-07
100 5.17561602464411e-07
101 5.15577482929075e-07
102 5.14021849085111e-07
103 5.12327630985965e-07
104 5.1048664317932e-07
105 5.08641562646517e-07
106 5.0874621138064e-07
107 5.06849744397186e-07
108 5.047663194091e-07
109 5.02975581184728e-07
110 5.00965711580648e-07
111 4.99656948704796e-07
112 4.97892585826776e-07
113 4.96109635150788e-07
114 4.94698781494662e-07
115 4.92814365316008e-07
116 4.91384980705334e-07
117 4.90032675770635e-07
118 4.88347325244831e-07
119 4.86777537389571e-07
120 4.85418638618285e-07
121 4.8399130037069e-07
122 4.82408381685673e-07
123 4.81021515952307e-07
124 4.78850722629431e-07
125 4.77190781111858e-07
126 4.75605560268377e-07
127 4.73970857228778e-07
128 4.72537038831433e-07
129 4.71215003017278e-07
130 4.69837402761186e-07
131 4.68486575755378e-07
132 4.67095645717563e-07
133 4.65636901481048e-07
134 4.64271948885653e-07
135 4.62577844473344e-07
136 4.6104369744171e-07
137 4.59437558220088e-07
138 4.58044667084323e-07
139 4.56496337619683e-07
140 4.54897929103026e-07
141 4.53255296406496e-07
142 4.51859733630045e-07
143 4.50265645213221e-07
144 4.48699751132153e-07
145 4.46946188503716e-07
146 4.45382283942308e-07
147 4.4377847530086e-07
148 4.41963976527404e-07
149 4.40553748148886e-07
150 4.38755677123481e-07
151 4.37035367895078e-07
152 4.35296129808194e-07
153 4.33364704122141e-07
154 4.31529116440288e-07
155 4.29915502309086e-07
156 4.28279918196495e-07
157 4.26498104388884e-07
158 4.24515377517309e-07
159 4.22790805032491e-07
160 4.20756293806335e-07
161 4.18555856640523e-07
162 4.16458647123363e-07
163 4.14274808235859e-07
164 4.1171850284627e-07
165 4.09166773351899e-07
166 4.07026846005465e-07
167 4.04671965270609e-07
168 4.02098748963908e-07
169 3.99266923523101e-07
170 3.96448541550853e-07
171 3.93317577618291e-07
172 3.90314880860387e-07
173 3.87046384275891e-07
174 3.82958091904584e-07
175 3.79208671574816e-07
176 3.74245274770146e-07
177 3.69710051018046e-07
178 3.65055797146852e-07
179 3.60219331696499e-07
180 3.55451192035616e-07
181 3.50745949617703e-07
182 3.46091582059671e-07
183 3.40886145977493e-07
184 3.35951540364476e-07
185 3.31085487914606e-07
186 3.25988111171682e-07
187 3.21086929488956e-07
188 3.15775224635217e-07
189 3.11413913323122e-07
190 3.06682323980567e-07
191 3.02466503399046e-07
192 2.98877694149269e-07
193 2.95210298872917e-07
194 2.9150072577977e-07
195 2.87995646885975e-07
196 2.84799085648046e-07
197 2.81817079894608e-07
198 2.79259808166898e-07
199 2.77028590289774e-07
200 2.74707105063499e-07
201 2.72762605391108e-07
202 2.71058127054857e-07
203 2.6988189461008e-07
204 2.69547768994016e-07
205 2.68394359181912e-07
206 2.65320011294534e-07
207 2.64071530864385e-07
208 2.62314500787397e-07
209 2.60875907542868e-07
210 2.59635640986744e-07
211 2.58335745684235e-07
212 2.57220705179861e-07
213 2.56210057614226e-07
214 2.55189235076614e-07
215 2.54354063145001e-07
216 2.52980129289426e-07
217 2.51939781037436e-07
218 2.50960596304139e-07
219 2.5014273319357e-07
220 2.4939029685811e-07
221 2.48770021471501e-07
222 2.47971570388472e-07
223 2.47217030846514e-07
224 2.45079263549997e-07
225 2.44594815512755e-07
226 2.44083423694974e-07
227 2.43687196643805e-07
228 2.42857282728437e-07
229 2.42217168988645e-07
230 2.41509013676477e-07
231 2.40862334521807e-07
232 2.40125785921919e-07
233 2.39425418158135e-07
234 2.38837287724891e-07
235 2.38178714084825e-07
236 2.37463254393333e-07
237 2.36920868701418e-07
238 2.36350189197765e-07
239 2.35873756082583e-07
240 2.35209412835502e-07
241 2.34224856399123e-07
242 2.33640918168021e-07
243 2.33047714459644e-07
244 2.32449821169212e-07
245 2.31767728564591e-07
246 2.31316846566187e-07
247 2.30790234923006e-07
248 2.30370872600361e-07
249 2.30384571864306e-07
250 2.29394032658092e-07
251 2.29412805197171e-07
252 2.28400566015807e-07
253 2.28493092890858e-07
254 2.2758759143926e-07
255 2.27690264864577e-07
256 2.26777274292544e-07
257 2.27178659883975e-07
258 2.26123844981885e-07
259 2.25164242806386e-07
260 2.24791719460882e-07
261 2.24733128106891e-07
262 2.24416766059221e-07
263 2.24016872607535e-07
264 2.23443024083281e-07
265 2.22819366513249e-07
266 2.22431523866362e-07
267 2.21920458898239e-07
268 2.21401592170878e-07
269 2.21018282786645e-07
270 2.20458645117105e-07
271 2.20197591715987e-07
272 2.19832116954422e-07
273 2.1939531791304e-07
274 2.19231239384499e-07
275 2.18515211258818e-07
276 2.18240884919396e-07
277 2.17393477441874e-07
278 2.16996113522328e-07
279 2.16499444150031e-07
280 2.15999392594313e-07
281 2.1548781603542e-07
282 2.14943398191281e-07
283 2.14361250527872e-07
284 2.13842284324528e-07
285 2.13370242363453e-07
286 2.12873871419106e-07
287 2.12307327274175e-07
288 2.11807275718456e-07
289 2.11278589290487e-07
290 2.10671743161583e-07
291 2.10146367862762e-07
292 2.09560724329094e-07
293 2.09076461032964e-07
294 2.08684127755987e-07
295 2.08166071047344e-07
296 2.07744605518201e-07
297 2.07578381150597e-07
298 2.07193465939781e-07
299 2.06663784751981e-07
300 2.06326916440958e-07
301 2.05771826244927e-07
302 2.05338281489276e-07
303 2.05314719892158e-07
304 2.04509206014336e-07
305 2.04359679401023e-07
306 2.03838496304343e-07
307 2.0345491691387e-07
308 2.02982903374505e-07
309 2.02577155050676e-07
310 2.02128688897574e-07
311 2.01792900611508e-07
312 2.01446312075859e-07
313 2.0128561573074e-07
314 2.00771978597913e-07
315 2.00268758021593e-07
316 1.99881824869408e-07
317 1.99185564042637e-07
318 1.98776803017608e-07
319 1.9794143213403e-07
320 1.97717639593975e-07
321 1.97233958942888e-07
322 1.96716541722708e-07
323 1.96131239249553e-07
324 1.95836250327375e-07
325 1.95425229776447e-07
326 1.95087281440465e-07
327 1.94632079342227e-07
328 1.94028956457259e-07
329 1.93775377965721e-07
330 1.93406336279622e-07
331 1.92871240756176e-07
332 1.92518925246077e-07
333 1.91864756970972e-07
334 1.91525558079775e-07
335 1.90719404713491e-07
336 1.9069372569902e-07
337 1.901006925209e-07
338 1.89855839494157e-07
339 1.8940910706533e-07
340 1.88681312351946e-07
341 1.88717322657794e-07
342 1.88262802680583e-07
343 1.87654464411935e-07
344 1.87227925607658e-07
345 1.87049863598077e-07
346 1.86613902997124e-07
347 1.86250360911799e-07
348 1.85781701134147e-07
349 1.85374162242624e-07
350 1.84984898510265e-07
351 1.84651241852407e-07
352 1.84260770197398e-07
353 1.83895707550619e-07
354 1.84990469165314e-07
355 1.84096776933984e-07
356 1.83771803108357e-07
357 1.83433797928956e-07
358 1.83341612114418e-07
359 1.81653732056475e-07
360 1.82345630150849e-07
361 1.82038775164983e-07
362 1.80920338266333e-07
363 1.80142507133496e-07
364 1.79791868504253e-07
365 1.7963336063076e-07
366 1.79298410785123e-07
367 1.7895226278597e-07
368 1.78646658355319e-07
369 1.78058741084897e-07
370 1.7765763971056e-07
371 1.77302567294646e-07
372 1.77062432271669e-07
373 1.76619622038743e-07
374 1.76320597233826e-07
375 1.75889724118861e-07
376 1.75698389170975e-07
377 1.75228990428877e-07
378 1.74908706185306e-07
379 1.74853369117045e-07
380 1.7507616689727e-07
381 1.74984833734015e-07
382 1.74621717974333e-07
383 1.7425236364943e-07
384 1.73807933379067e-07
385 1.736493544513e-07
386 1.73151050830711e-07
387 1.72740371340296e-07
388 1.72672017129116e-07
389 1.71124796111144e-07
390 1.71316642649799e-07
391 1.70939799204461e-07
392 1.70726664805443e-07
393 1.70471793126126e-07
394 1.70259909282322e-07
395 1.69519154269437e-07
396 1.6963775806289e-07
397 1.69240649938729e-07
398 1.68989046755996e-07
399 1.6875084440926e-07
400 1.68617432905194e-07
401 1.68195413152716e-07
402 1.6786290757409e-07
403 1.67484500934734e-07
404 1.67468954259675e-07
405 1.673766547583e-07
406 1.67023003427857e-07
407 1.66609638085902e-07
408 1.66244092270063e-07
409 1.65996198120411e-07
410 1.65894235237829e-07
411 1.65537457519349e-07
412 1.6530964330741e-07
413 1.65125385365172e-07
414 1.65302154186975e-07
415 1.64512101719083e-07
416 1.64388481493916e-07
417 1.63969502864347e-07
418 1.63731257885047e-07
419 1.64656853485212e-07
420 1.6365552824027e-07
421 1.63828758559248e-07
422 1.62944118642372e-07
423 1.6236988642504e-07
424 1.62085640909027e-07
425 1.6182323747671e-07
426 1.60941311833085e-07
427 1.61045932145498e-07
428 1.60883985245164e-07
429 1.60739745069804e-07
430 1.60472083621244e-07
431 1.6024570470563e-07
432 1.5997876801066e-07
433 1.59756012862999e-07
434 1.59469138338864e-07
435 1.5931512109546e-07
436 1.5893067484285e-07
437 1.58756236601221e-07
438 1.58375300429725e-07
439 1.58043619080672e-07
440 1.57699233227504e-07
441 1.57320641847036e-07
442 1.57105148446135e-07
443 1.57527878741348e-07
444 1.56659766048506e-07
445 1.56129516426518e-07
446 1.56413605623129e-07
447 1.55829781078864e-07
448 1.55337957608026e-07
449 1.5503312056353e-07
450 1.55594861439567e-07
451 1.54361103454903e-07
452 1.54955685616187e-07
453 1.5476220482924e-07
454 1.53509304823274e-07
455 1.54251154071972e-07
456 1.53840673533523e-07
457 1.53536788616293e-07
458 1.52386505192226e-07
459 1.52322385815751e-07
460 1.51394075942335e-07
461 1.51095505884768e-07
462 1.52648098605823e-07
463 1.51396974956697e-07
464 1.50229340079022e-07
465 1.50731395365256e-07
466 1.50509237073493e-07
467 1.49449789432765e-07
468 1.48777843378411e-07
469 1.49091732737361e-07
470 1.49176884178814e-07
471 1.47612041700995e-07
472 1.48521394294221e-07
473 1.46913293974649e-07
474 1.46471009543347e-07
475 1.4647415014224e-07
476 1.45697356401797e-07
477 1.46405582768239e-07
478 1.45819655017476e-07
479 1.44368428323105e-07
480 1.43938464702842e-07
481 1.43516828643442e-07
482 1.4313894780571e-07
483 1.43921553785731e-07
484 1.42060443408809e-07
485 1.42924207580108e-07
486 1.41175647172531e-07
487 1.42048307338882e-07
488 1.40218560318317e-07
489 1.4110234758391e-07
490 1.38914657554778e-07
491 1.40869858000769e-07
492 1.38571223828876e-07
493 1.40259814429555e-07
494 1.37274852818337e-07
495 1.38738798227678e-07
496 1.36394007199669e-07
497 1.37783757736543e-07
498 1.35164626158257e-07
499 1.36889738655555e-07
500 1.34245283334167e-07
501 1.3372643081766e-07
502 1.33607386487711e-07
503 1.3304705248629e-07
504 1.32057067503411e-07
505 1.3220203243236e-07
506 1.31253386825847e-07
507 1.30971713474537e-07
508 1.30290771949149e-07
509 1.29528814341029e-07
510 1.28850501823763e-07
511 1.28562135159882e-07
512 1.28430585277783e-07
513 1.2744776256568e-07
514 1.26980935988286e-07
515 1.2587122455443e-07
516 1.25175645848685e-07
517 1.24623866781803e-07
518 1.24194883710516e-07
519 1.23697589060612e-07
520 1.22076514230685e-07
521 1.21832869126592e-07
522 1.21396269037177e-07
523 1.20691964866637e-07
524 1.20567705153007e-07
525 1.20304264328297e-07
526 1.19778363227852e-07
527 1.19065198589396e-07
528 1.1856307224889e-07
529 1.18135027094013e-07
530 1.17468324845049e-07
531 1.16142288675292e-07
532 1.15674858136572e-07
533 1.15224715102613e-07
534 1.14387482597067e-07
535 1.14064384604262e-07
536 1.13456508188392e-07
537 1.1294600454903e-07
538 1.12512132943721e-07
539 1.11665940494277e-07
540 1.11162123062059e-07
541 1.10356623395091e-07
542 1.10219787075039e-07
543 1.10038314460326e-07
544 1.09400545511562e-07
545 1.08384298869169e-07
546 1.08234438300769e-07
547 1.07691711548341e-07
548 1.07534184223823e-07
549 1.06898902174635e-07
550 1.06741651961784e-07
551 1.06078807959875e-07
552 1.05620252099925e-07
553 1.05154150276121e-07
554 1.04629805264267e-07
555 1.0410320783194e-07
556 1.03414443231031e-07
557 1.03581612620474e-07
558 1.02465804729945e-07
559 1.02294514192636e-07
560 1.0178644060943e-07
561 1.00847529438397e-07
562 1.00427349991605e-07
563 9.99918086108664e-08
564 9.97137448166541e-08
565 9.93751925193465e-08
566 9.90645929732636e-08
567 9.85544161835605e-08
568 9.78692398234671e-08
569 9.74814682308534e-08
570 9.72244720287563e-08
571 9.65363824434462e-08
572 9.60343555789223e-08
573 9.56421288833553e-08
574 9.52264187503715e-08
575 9.70450528825495e-08
576 9.414283397291e-08
577 9.37274649004394e-08
578 9.31365704559539e-08
579 9.28175509784523e-08
580 9.22051270890734e-08
581 9.18197855526159e-08
582 9.14476032676248e-08
583 9.10477595539305e-08
584 9.06320991589382e-08
585 9.00575543028026e-08
586 8.97470968652669e-08
587 8.93791778366904e-08
588 8.90014604237876e-08
589 8.84439330661735e-08
590 8.81150441500722e-08
591 8.77384991326835e-08
592 8.73159748948638e-08
593 8.69918750368015e-08
594 8.64337721395714e-08
595 8.61417319697466e-08
596 8.55997299709088e-08
597 8.51697308235089e-08
598 8.48898054073288e-08
599 8.43328962218948e-08
600 8.39473699443261e-08
601 8.37389251273635e-08
602 8.34367526181268e-08
603 8.29979711625128e-08
604 8.24746209104887e-08
605 8.22346137852037e-08
606 8.18162533278155e-08
607 8.15010352539502e-08
608 8.10751075164262e-08
609 8.07716773465472e-08
610 8.04118016617394e-08
611 8.00373669562759e-08
612 7.96781733924945e-08
613 7.94281262983532e-08
614 7.913812538618e-08
615 7.87387364198366e-08
616 7.83988554076132e-08
617 7.81573845642924e-08
618 7.78472397655605e-08
619 7.75433477429033e-08
620 7.72700587958752e-08
621 7.68629675462762e-08
622 7.64833174571322e-08
623 7.6189877518118e-08
624 7.59370735181619e-08
625 7.57813083396286e-08
626 7.53913553808161e-08
627 7.50531370385943e-08
628 7.48012496387673e-08
629 7.45240100741285e-08
630 7.42499395300911e-08
631 7.39492236334627e-08
632 7.37040792841981e-08
633 7.35497991399825e-08
634 7.33390947971202e-08
635 7.29626634665692e-08
636 7.27126803212741e-08
637 7.25942683743597e-08
638 7.23808852853836e-08
639 7.21710406992315e-08
640 7.18697492629872e-08
641 7.16415371471157e-08
642 7.14784675892588e-08
643 7.13153767151198e-08
644 7.10196630393511e-08
645 7.08917724523417e-08
646 7.06668998873283e-08
647 7.04120779460027e-08
648 7.09559415668082e-08
649 7.05519909161012e-08
650 7.05020610780593e-08
651 7.06262142102787e-08
652 7.03219598108262e-08
653 6.98633400020299e-08
654 6.99134972137472e-08
655 6.94697916969744e-08
656 6.93237964810578e-08
657 6.95369450909311e-08
658 6.95272817097248e-08
659 6.90575703288232e-08
660 6.86363605950646e-08
661 6.86194709942356e-08
662 6.85067007566431e-08
663 6.83495429143477e-08
664 6.82535556961739e-08
665 6.80365275229633e-08
666 6.78120315455999e-08
667 6.75610749567568e-08
668 6.73582150056973e-08
669 6.71954154540799e-08
670 6.70807409619556e-08
671 6.69364368377501e-08
672 6.68249313662272e-08
673 6.68640112166941e-08
674 6.66180355324286e-08
675 6.6325839043202e-08
676 6.63309762671815e-08
677 6.63962751445979e-08
678 6.65871269234231e-08
679 6.59970069705196e-08
680 6.56805383414394e-08
681 6.43344506556787e-08
682 6.61189929473949e-08
683 6.57764402944849e-08
684 6.55737011356905e-08
685 6.52565432801566e-08
686 6.50928058121281e-08
687 6.50034621685336e-08
688 6.49419575893262e-08
689 6.47736513315067e-08
690 6.47299742695395e-08
691 6.46600852860502e-08
692 6.47331646064231e-08
693 6.47592131031161e-08
694 6.47255191665863e-08
695 6.46571081119873e-08
696 6.42042721210601e-08
697 6.43709512360147e-08
698 6.41939124079727e-08
699 6.42055084654203e-08
700 6.40360013903774e-08
701 6.38549479958783e-08
702 6.37699812955361e-08
703 6.36092991612713e-08
704 6.33092298585325e-08
705 6.2922296706347e-08
706 6.24425595674438e-08
707 6.24315390496122e-08
708 6.2140642853592e-08
709 6.17686879422763e-08
710 6.1822071018014e-08
711 6.16963617972033e-08
712 6.15478157328653e-08
713 6.13503416957428e-08
714 6.10039379012051e-08
715 6.10390387123516e-08
716 6.09631172210356e-08
717 6.07591985612999e-08
718 6.05736545367108e-08
719 6.04212502253176e-08
720 6.01973866309891e-08
721 5.99806995182917e-08
722 5.99019500668874e-08
723 5.97548392988756e-08
724 5.98440124122135e-08
725 6.05723258217949e-08
726 5.94407296716781e-08
727 5.92771129959146e-08
728 5.907108047154e-08
729 5.87979194222044e-08
730 5.85588644241852e-08
731 5.84315777985012e-08
732 5.83012607080491e-08
733 5.93183138164477e-08
734 5.78173882104238e-08
735 5.77054919403963e-08
736 5.74420404575449e-08
737 5.73721514740555e-08
738 5.71276537186804e-08
739 5.68611930873431e-08
740 5.66392941436789e-08
741 5.64902968847036e-08
742 5.61004043220237e-08
743 5.59795658716666e-08
744 5.58086412638659e-08
745 5.55688330905468e-08
746 5.53505898892581e-08
747 5.53228041155762e-08
748 5.50586847225532e-08
749 5.48481366990927e-08
750 5.45339489121943e-08
751 5.44999672058566e-08
752 5.43339275793642e-08
753 5.40667990378552e-08
754 5.39902913487822e-08
755 5.36846869181318e-08
756 5.34523962869571e-08
757 5.31858361796367e-08
758 5.28832799773227e-08
759 5.27511261338987e-08
760 5.35343076535355e-08
761 5.22139949055145e-08
762 5.20052871877397e-08
763 5.18651823711025e-08
764 5.15502556197589e-08
765 5.13524689438327e-08
766 5.12149220810443e-08
767 5.14439193466387e-08
768 5.11093922739292e-08
769 5.09720017305426e-08
770 5.06248412079913e-08
771 5.06027149071997e-08
772 5.03951333996611e-08
773 4.99739378767572e-08
774 4.94825940222654e-08
775 4.93944867230312e-08
776 5.04009456392396e-08
777 4.90213380999194e-08
778 4.90438480937883e-08
779 4.87403397642083e-08
780 4.86046545233876e-08
781 4.84506514908389e-08
782 4.82919553235206e-08
783 4.84308770865027e-08
784 4.81672230989716e-08
785 4.77495767370328e-08
786 4.81345878711181e-08
787 4.76570782836916e-08
788 4.76998600618117e-08
789 4.72684362762266e-08
790 4.71639438615057e-08
791 4.86732218973884e-08
792 4.6857856261795e-08
793 4.65118219494798e-08
794 4.65179041952979e-08
795 4.65407481442526e-08
796 4.6263703978866e-08
797 4.6074166704102e-08
798 4.62775275877902e-08
799 4.55073276839357e-08
800 4.50618529157509e-08
801 4.510219042686e-08
802 4.50732322576641e-08
803 4.48862884638856e-08
804 4.47497541244957e-08
805 4.45563514972491e-08
806 4.43997265620055e-08
807 4.40707275117802e-08
808 4.40554224212519e-08
809 4.45462191578372e-08
810 4.37166924882604e-08
811 4.30366213777233e-08
812 4.26985664603308e-08
813 4.26593125268937e-08
814 4.45163550466532e-08
815 4.23679402672406e-08
816 4.45674643856364e-08
817 4.22745394246249e-08
818 4.22846184733316e-08
819 4.20500754216846e-08
820 4.17371417427148e-08
821 4.19201917623013e-08
822 4.15942089659893e-08
823 4.15599750169804e-08
824 4.12802378946253e-08
825 4.13561132006635e-08
826 4.11343457074054e-08
827 4.08600904222567e-08
828 4.06440072708847e-08
829 4.06630640270578e-08
830 4.05275635273483e-08
831 4.06766140770287e-08
832 4.0252718491729e-08
833 4.02026287815715e-08
834 4.01666291338643e-08
835 4.0060061934355e-08
836 3.98790547251338e-08
837 3.96073929209706e-08
838 3.97037354105123e-08
839 3.90766494717809e-08
840 3.91704695346107e-08
841 3.95012627052438e-08
842 3.9636290694034e-08
843 3.90896062185675e-08
844 3.85312119988157e-08
845 3.85250302770146e-08
846 3.8875143104633e-08
847 3.81010814010097e-08
848 3.78663855826744e-08
849 3.77588449396171e-08
850 3.80702473989913e-08
851 3.74437192363075e-08
852 3.73381752183377e-08
853 3.7259795249156e-08
854 3.71920130248782e-08
855 3.75236481886532e-08
856 3.66813495134011e-08
857 3.69020369817008e-08
858 3.67438559578659e-08
859 3.68036943143579e-08
860 3.65231862531346e-08
861 3.64532510843674e-08
862 3.64733701019304e-08
863 3.63492596022752e-08
864 3.62987826463268e-08
865 3.60635965535039e-08
866 3.59804595007063e-08
867 3.64464902702366e-08
868 3.57421718888418e-08
869 3.59156935303417e-08
870 3.55287035347374e-08
871 3.51074191939915e-08
872 3.54977061078898e-08
873 3.54004434655053e-08
874 3.54179761075102e-08
875 3.52850584306452e-08
876 3.47041648751656e-08
877 3.52468028097519e-08
878 3.50442697083508e-08
879 3.48130804184166e-08
880 3.48510198477925e-08
881 3.42732491276365e-08
882 3.47798518873788e-08
883 3.47452200344378e-08
884 3.4567438689237e-08
885 3.45095045872768e-08
886 3.4228580858553e-08
887 3.45074546714841e-08
888 3.4173741170207e-08
889 3.39572672203303e-08
890 3.42758887938999e-08
891 3.38773133989889e-08
892 3.38880283834442e-08
893 3.37731158595034e-08
894 3.39683587924355e-08
895 3.35417915664493e-08
896 3.40963026701502e-08
897 3.37436709685335e-08
898 3.35789174243928e-08
899 3.36240297826862e-08
900 3.35142296137292e-08
901 3.38187398085665e-08
902 3.34632161980153e-08
903 3.35091066006044e-08
904 3.35454473088248e-08
905 3.35009282537158e-08
906 3.33351763970313e-08
907 3.29984750635504e-08
908 3.31669482989128e-08
909 3.33712790734353e-08
910 3.29057989745252e-08
911 3.26665805516768e-08
912 3.32532401614571e-08
913 3.32648539824731e-08
914 3.27885807394068e-08
915 3.32381908663137e-08
916 3.30521388036686e-08
917 3.30039853224662e-08
918 3.275056315033e-08
919 3.29543006216682e-08
920 3.34537268997792e-08
921 3.26755085211516e-08
922 3.25896678532445e-08
923 3.27091029816984e-08
924 3.27239533248758e-08
925 3.27378479880736e-08
926 3.22172191147274e-08
927 3.27117781750985e-08
928 3.26172937548108e-08
929 3.26688649465723e-08
930 3.26093143598882e-08
931 3.24268114582082e-08
932 3.25635554077053e-08
933 3.25544071699824e-08
934 3.24962741160562e-08
935 3.24834914522398e-08
936 3.24934070761174e-08
937 3.25233457942886e-08
938 3.24234932236322e-08
939 3.22747553127556e-08
940 3.22394875240661e-08
941 3.17372652602899e-08
942 3.20960147348615e-08
943 3.18272661559149e-08
944 3.17794572879393e-08
945 3.16564445768108e-08
946 3.16611377115805e-08
947 3.15230366254582e-08
948 3.17490709278445e-08
949 3.15916182103138e-08
950 3.15818233787013e-08
951 3.16526502786019e-08
952 3.14598693762491e-08
953 3.15501509362548e-08
954 3.14717034655132e-08
955 3.14937764755996e-08
956 3.15790913418823e-08
957 3.13643440108535e-08
958 3.13248769145957e-08
959 3.13044346000879e-08
960 3.14073744789312e-08
961 3.10236210054882e-08
962 3.13437027443797e-08
963 3.08956984440556e-08
964 3.10049088625419e-08
965 3.08613188337858e-08
966 3.0823411378833e-08
967 3.06760483681501e-08
968 3.05077314521895e-08
969 3.06524228221861e-08
970 3.06839709196538e-08
971 3.08061522957814e-08
972 3.04964586916867e-08
973 3.04333731548923e-08
974 3.08688186123618e-08
975 3.06182741383054e-08
976 3.05668486078048e-08
977 3.08454168873595e-08
978 3.07848395664223e-08
979 3.06111154202426e-08
980 3.05876035611163e-08
981 3.06384961845652e-08
982 3.06311740416731e-08
983 3.05929894750534e-08
984 3.04431715392184e-08
985 3.05948901768716e-08
986 3.05701703950945e-08
987 3.05803879996347e-08
988 3.02347729075336e-08
989 3.02765705839647e-08
990 3.05515754916996e-08
991 3.02081986092162e-08
992 3.0169985620887e-08
993 3.04452036914427e-08
994 3.04127674155552e-08
995 3.03427825087965e-08
996 3.01154443604901e-08
997 3.0152619956425e-08
998 3.01759683907221e-08
999 3.01537887992254e-08
1000 3.02071647695357e-08
1001 3.03076497232269e-08
1002 3.01539273550588e-08
1003 3.00166540512237e-08
1004 2.96632762797344e-08
1005 2.96638820174167e-08
1006 2.99340108256274e-08
1007 2.98694402545152e-08
1008 2.95886817269775e-08
1009 2.93840951570701e-08
1010 3.02386240491614e-08
1011 2.96837807667316e-08
1012 3.02992155809534e-08
1013 3.01933376078978e-08
1014 3.00138971454089e-08
1015 2.97416740124845e-08
1016 2.98196987102983e-08
1017 3.02339806523833e-08
1018 2.97427913409365e-08
1019 2.98900779682754e-08
1020 2.95920639103997e-08
1021 2.97954443340132e-08
1022 2.97993878461966e-08
1023 2.98511011465052e-08
1024 2.96055375770266e-08
1025 2.92165047710569e-08
1026 2.89847612577887e-08
1027 2.94808639722532e-08
1028 2.90322716978153e-08
1029 2.93809314655391e-08
1030 3.03865697048877e-08
1031 2.956282685318e-08
1032 3.00303035771776e-08
1033 3.03988976213532e-08
1034 3.02540961172326e-08
1035 3.02988958367223e-08
1036 2.99833899930491e-08
1037 3.01047791140263e-08
1038 2.96874169691819e-08
1039 2.94802831035668e-08
1040 2.9352550612316e-08
1041 2.93185120625594e-08
1042 2.92430684112333e-08
1043 2.92041235638862e-08
1044 2.95619670964697e-08
1045 2.99793576630236e-08
1046 2.94434858716386e-08
1047 2.92503923304821e-08
1048 2.89075732240462e-08
1049 2.90214998699412e-08
1050 2.90829245130908e-08
1051 2.91665482876624e-08
1052 2.90542470082755e-08
1053 2.89634503047864e-08
1054 2.87893460182431e-08
1055 2.88687100891138e-08
1056 2.90255623980329e-08
1057 2.88529307113095e-08
1058 2.89021144794788e-08
1059 2.85716765802135e-08
1060 2.83701435677131e-08
1061 2.8750656966281e-08
1062 2.86881149946794e-08
1063 2.87060668568984e-08
1064 2.83113852361794e-08
1065 2.84317103194098e-08
1066 2.8311267996628e-08
1067 2.82084346991951e-08
1068 2.81285004177789e-08
1069 2.81512235744685e-08
1070 2.82580590038606e-08
1071 2.82978671606315e-08
1072 2.82356715786136e-08
1073 2.8364127047098e-08
1074 2.79388210344678e-08
1075 2.83069674367198e-08
1076 2.82474488244588e-08
1077 2.7791223544682e-08
1078 2.75733036403381e-08
1079 2.7717492301349e-08
1080 2.76653171482621e-08
1081 2.75759877155224e-08
1082 2.78784195728576e-08
1083 2.75014624406822e-08
1084 2.79188707708045e-08
1085 2.75814819872267e-08
1086 2.76854450476094e-08
1087 2.79886620546677e-08
1088 2.72914100207799e-08
1089 2.76990732572813e-08
1090 2.73540639028624e-08
1091 2.72854432381564e-08
1092 2.72330478168215e-08
1093 2.72404712120533e-08
1094 2.72988049943024e-08
1095 2.75906870683684e-08
1096 2.70890474496355e-08
1097 2.72515094934533e-08
1098 2.72431837089471e-08
1099 2.74582081516428e-08
1100 2.72233737774741e-08
1101 2.71575029131554e-08
1102 2.71424855924352e-08
1103 2.75190430443217e-08
1104 2.69765116911458e-08
1105 2.70271858227034e-08
1106 2.69330548974267e-08
1107 2.73140230433455e-08
1108 2.6848292478121e-08
1109 2.72002935730598e-08
1110 2.66103654666949e-08
1111 2.6749464865361e-08
1112 2.71272728724625e-08
1113 2.65681432409792e-08
1114 2.67024908850999e-08
1115 2.71793645367779e-08
1116 2.65030433155289e-08
1117 2.68048783169661e-08
1118 2.65850701453019e-08
1119 2.65677240207651e-08
1120 2.6425439614286e-08
1121 2.66671200677138e-08
1122 2.63843578096612e-08
1123 2.67427697764333e-08
1124 2.62178776466726e-08
1125 2.6360291727201e-08
1126 2.64238657621263e-08
1127 2.62137156425979e-08
1128 2.61325219241826e-08
1129 2.65427111401095e-08
1130 2.60462940104844e-08
1131 2.61773838161616e-08
1132 2.5964084215957e-08
1133 2.60229153781211e-08
1134 2.64664841154172e-08
1135 2.59976395966532e-08
1136 2.59152539427987e-08
1137 2.58421017917954e-08
1138 2.61115307154114e-08
1139 2.57875676368258e-08
1140 2.61607837614974e-08
1141 2.5789876900717e-08
1142 2.58837431488246e-08
1143 2.57498946609758e-08
1144 2.57417109850167e-08
1145 2.57280685644901e-08
1146 2.55959822226259e-08
1147 2.56397001408004e-08
1148 2.55741330335013e-08
1149 2.54938310462194e-08
1150 2.58513157547213e-08
1151 2.57238355061418e-08
1152 2.53476422074073e-08
1153 2.53541561079373e-08
1154 2.5424517602346e-08
1155 2.56971723899824e-08
1156 2.51804479489692e-08
1157 2.53502605573885e-08
1158 2.5630734867832e-08
1159 2.52610679041254e-08
1160 2.51088234648478e-08
1161 2.50616665198322e-08
1162 2.5118167101823e-08
1163 2.50556499992172e-08
1164 2.50529907930286e-08
1165 2.53746303968683e-08
1166 2.49008351715929e-08
1167 2.52936285249916e-08
1168 2.48157938642635e-08
1169 2.51781582250032e-08
1170 2.47178313372842e-08
1171 2.51148346563923e-08
1172 2.47870950431661e-08
1173 2.51471199419484e-08
1174 2.47302089917412e-08
1175 2.4845077106761e-08
1176 2.47295623978516e-08
1177 2.49294416221346e-08
1178 2.44677504923629e-08
1179 2.47502356387486e-08
1180 2.43647875208808e-08
1181 2.46782683177571e-08
1182 2.48201921237978e-08
1183 2.44249473979608e-08
1184 2.4576577217772e-08
1185 2.4415365729169e-08
1186 2.45058622283523e-08
1187 2.42663951155464e-08
1188 2.44926425807535e-08
1189 2.4129633402481e-08
1190 2.44200606402956e-08
1191 2.41433770753474e-08
1192 2.43512552344782e-08
1193 2.40798421202726e-08
1194 2.4173521850912e-08
1195 2.39757387276995e-08
1196 2.41631230579742e-08
1197 2.38956481268815e-08
1198 2.42272157890966e-08
1199 2.3747240618377e-08
1200 2.40281003982545e-08
1201 2.38381172579238e-08
1202 2.39828992221192e-08
1203 2.4254797281742e-08
1204 2.43321292003884e-08
1205 2.41330333494716e-08
1206 2.39725803652391e-08
1207 2.36791493080091e-08
1208 2.38847999156633e-08
1209 2.35007480142713e-08
1210 2.36627286653857e-08
1211 2.36888020310744e-08
1212 2.36008528275988e-08
1213 2.36441106693519e-08
1214 2.35443682328196e-08
1215 2.30753105512349e-08
1216 2.2990517933863e-08
1217 2.31698695785099e-08
1218 2.28127881030105e-08
1219 2.29737135981622e-08
1220 2.29552732378124e-08
1221 2.30081198537846e-08
1222 2.27245191553038e-08
1223 2.28678409541772e-08
1224 2.27392540352866e-08
1225 2.27473258007649e-08
1226 2.27538166086561e-08
1227 2.2868096749562e-08
1228 2.29636967219449e-08
1229 2.26983019047111e-08
1230 2.24906848700357e-08
1231 2.27737295688257e-08
1232 2.24309282259583e-08
1233 2.26466116970414e-08
1234 2.27686989262565e-08
1235 2.29268675155936e-08
1236 2.28358860709932e-08
1237 2.28563354909284e-08
1238 2.2720325176806e-08
1239 2.2760010764955e-08
1240 2.2496278617723e-08
1241 2.2601996718663e-08
1242 2.25057856795274e-08
1243 2.25628937755573e-08
1244 2.23605436389107e-08
1245 2.24674678861447e-08
1246 2.24374758772683e-08
1247 2.2332999449759e-08
1248 2.2281467337848e-08
1249 2.22446701059198e-08
1250 2.22658087523087e-08
1251 2.24750653643468e-08
1252 2.20852776067204e-08
1253 2.20561346964132e-08
1254 2.23067768700957e-08
1255 2.20182911903066e-08
1256 2.20875566725454e-08
1257 2.19943920853893e-08
1258 2.19082565422468e-08
1259 2.16471907066307e-08
1260 2.14588684599448e-08
1261 2.16889208815019e-08
1262 2.1957143658824e-08
1263 2.16351097037659e-08
1264 2.18229168069684e-08
1265 2.19384812538692e-08
1266 2.14589395142184e-08
1267 2.15331183994749e-08
1268 2.14067057413558e-08
1269 2.17384297229728e-08
1270 2.16244959716505e-08
1271 2.15056381591694e-08
1272 2.16752091830585e-08
1273 2.13392059578155e-08
1274 2.11403428096446e-08
1275 2.12489226214529e-08
1276 2.17521964884781e-08
1277 2.13172128837869e-08
1278 2.1147538831201e-08
1279 2.10995541038983e-08
1280 2.12163566715162e-08
1281 2.09429682485052e-08
1282 2.09960759889327e-08
1283 2.10646327047925e-08
1284 2.13446984531629e-08
1285 2.08485921859847e-08
1286 2.08691872671807e-08
1287 2.09479846802196e-08
1288 2.10729496075146e-08
1289 2.08478105889753e-08
1290 2.09834674080867e-08
1291 2.10176569481746e-08
1292 2.06926902279747e-08
1293 2.0952471757596e-08
1294 2.13584243624609e-08
1295 2.06252579459942e-08
1296 2.05994048485536e-08
1297 2.04782057977582e-08
1298 2.13185522568438e-08
1299 2.10656718735436e-08
1300 2.04848635831922e-08
1301 2.016038713748e-08
1302 2.06757349019426e-08
1303 2.05304111489113e-08
1304 2.05293115840277e-08
1305 2.05988879287133e-08
1306 2.07087627046576e-08
1307 2.04981898122014e-08
1308 2.03090824157925e-08
1309 2.01077394734739e-08
1310 2.05694625776687e-08
1311 2.03062473502769e-08
1312 2.01483896233867e-08
1313 2.06155981175016e-08
1314 2.01459275928073e-08
1315 2.01908196828526e-08
1316 2.05306616152257e-08
1317 2.00814156414708e-08
1318 2.01410710332084e-08
1319 1.99962482128058e-08
1320 2.02288248374316e-08
1321 1.97970670967607e-08
1322 2.00209129275208e-08
1323 1.98070786439075e-08
1324 2.01338732352951e-08
1325 2.02019290185262e-08
1326 1.98771505921513e-08
1327 1.97263165802042e-08
1328 1.98588629984897e-08
1329 2.00298337915683e-08
1330 1.99518996879533e-08
1331 1.99559977431818e-08
1332 1.963197071575e-08
1333 2.0170286774146e-08
1334 1.97432452608837e-08
1335 1.96401988006301e-08
1336 1.99818099844151e-08
1337 1.99809395695638e-08
1338 1.96584757361506e-08
1339 1.97370795262941e-08
1340 1.96067322377758e-08
1341 2.02992698206117e-08
1342 2.02039167618295e-08
1343 1.99939496070556e-08
1344 2.00064089739271e-08
1345 1.97807015211993e-08
1346 1.99032044179148e-08
1347 1.98372838156047e-08
1348 1.98637835069349e-08
1349 1.98078993207673e-08
1350 1.99068423967219e-08
1351 1.98369463078052e-08
1352 1.97315053185321e-08
1353 1.97389411482618e-08
1354 2.00692191754115e-08
1355 1.96327096801951e-08
1356 1.99075742557397e-08
1357 2.0195018990421e-08
1358 1.9679708529452e-08
1359 1.96526546147879e-08
1360 1.96219751558147e-08
1361 2.00742409361965e-08
1362 2.01959089451975e-08
1363 1.9501387171772e-08
1364 1.93829823302849e-08
1365 1.98149976426976e-08
1366 1.95654390466871e-08
1367 1.92022646672285e-08
1368 1.92599678427996e-08
1369 1.96804990082455e-08
1370 1.96857605772038e-08
1371 1.92763707218546e-08
1372 1.92778948360228e-08
1373 2.09907629056261e-08
1374 1.88379480903222e-08
1375 1.96083309589312e-08
1376 2.03139745025283e-08
1377 1.94150739929455e-08
1378 1.92134042009684e-08
1379 1.93121625358117e-08
1380 1.90291125079511e-08
1381 1.94234424100159e-08
1382 1.9212224700027e-08
1383 1.91348821232395e-08
1384 1.91743811939205e-08
1385 1.86176389860293e-08
1386 1.87497182224661e-08
1387 1.85620390169561e-08
1388 1.97144416347328e-08
1389 1.87431403730898e-08
1390 1.98887395441716e-08
1391 1.90283717671491e-08
1392 1.90214421991186e-08
1393 1.91251530168302e-08
1394 1.89339495193508e-08
1395 1.88337772044633e-08
1396 1.90662277077536e-08
1397 1.90641848973883e-08
1398 1.88190139027711e-08
1399 1.88460855810035e-08
1400 1.87508515381296e-08
1401 1.86831989879011e-08
1402 1.86178450434227e-08
1403 1.82806711990224e-08
1404 1.86312085759255e-08
1405 1.87223978542761e-08
1406 1.90624600548972e-08
1407 1.8633315335137e-08
1408 1.83137753850815e-08
1409 1.84920683210521e-08
1410 1.85078103953629e-08
1411 1.87016429009645e-08
1412 1.84786568269146e-08
1413 1.84610478015657e-08
1414 1.83145303367382e-08
1415 1.84494535204749e-08
1416 1.84299722150172e-08
1417 1.90371451935789e-08
1418 1.82817991856155e-08
1419 1.85197137625437e-08
1420 1.83889365956702e-08
1421 1.88652720112259e-08
1422 1.92988487413004e-08
1423 1.9122577299413e-08
1424 1.9388677330312e-08
1425 1.96901019933193e-08
1426 1.89338678069362e-08
1427 2.03412593435814e-08
1428 2.0206190498584e-08
1429 1.87492297243352e-08
1430 1.87341040458477e-08
1431 1.94738447589771e-08
1432 1.9037077692019e-08
1433 1.89410531703516e-08
1434 1.87831030729058e-08
1435 1.86013249248163e-08
1436 1.96583851419518e-08
1437 1.9681477780864e-08
1438 1.84949691117708e-08
1439 1.85523774121066e-08
1440 1.90667908128717e-08
1441 1.83139050591308e-08
1442 1.93503097989378e-08
1443 1.93875813181421e-08
1444 1.85219093395972e-08
1445 1.84457640273195e-08
1446 1.84062329822154e-08
1447 1.85433037813709e-08
1448 1.87360562620142e-08
1449 1.87381896665784e-08
1450 1.86514856892472e-08
1451 1.82725568009801e-08
1452 1.86049700090507e-08
1453 1.80071246802527e-08
1454 1.79525123655822e-08
1455 1.87292812370288e-08
1456 1.7991027334574e-08
1457 1.82514146018775e-08
1458 1.84365802624598e-08
1459 1.81928800913056e-08
1460 1.7851462530416e-08
1461 1.78982855203458e-08
1462 1.78567791664364e-08
1463 1.82163901740751e-08
1464 1.80282953010646e-08
1465 1.7896148563068e-08
1466 1.81242860719522e-08
1467 1.80848402919764e-08
1468 1.82673201010175e-08
1469 1.76352372704969e-08
1470 1.76008629892976e-08
1471 1.83790636043568e-08
1472 1.8198914375489e-08
1473 1.84052986185179e-08
1474 1.74966583443847e-08
1475 1.85122637219592e-08
1476 1.80686132722485e-08
1477 1.82495298872709e-08
1478 1.79821810775138e-08
1479 1.83810122678096e-08
1480 1.79161929736438e-08
1481 1.76143721830613e-08
1482 1.73787189083896e-08
1483 1.76923578010246e-08
1484 1.79273857980888e-08
1485 1.82167489981566e-08
1486 1.7873329483109e-08
1487 1.73941039349756e-08
1488 1.75857906015153e-08
1489 1.78386923010976e-08
1490 1.71978751239976e-08
1491 1.72911125417841e-08
1492 1.69354628098972e-08
1493 1.68740132977518e-08
1494 1.69596336974109e-08
1495 1.79734573890755e-08
1496 1.74494232396682e-08
1497 1.78370935799421e-08
1498 1.74565002453164e-08
1499 1.78911783166313e-08
1500 1.77695991254723e-08
1501 1.66976192872426e-08
1502 1.63432627431348e-08
1503 1.71659895187304e-08
1504 1.79019146173687e-08
1505 1.70740399596525e-08
1506 1.75620122888631e-08
1507 1.81333899007541e-08
1508 1.79693753210586e-08
1509 1.78282721918777e-08
1510 1.70245559871773e-08
1511 1.69790581594498e-08
1512 1.78343384504842e-08
1513 1.773995883525e-08
1514 1.68928355748221e-08
1515 1.70262808296684e-08
1516 1.6536747082796e-08
1517 1.79349566309384e-08
1518 1.68584559645524e-08
1519 1.62421791571887e-08
1520 1.68867551053609e-08
1521 1.76251493400059e-08
1522 1.65330913404205e-08
1523 1.67997669109354e-08
1524 1.66264122469784e-08
1525 1.71032041862418e-08
1526 1.75643197763975e-08
1527 1.65126916584768e-08
1528 1.66422307046332e-08
1529 1.65247762140552e-08
1530 1.68438720749009e-08
1531 1.70060285853424e-08
1532 1.76068724044853e-08
1533 1.81248260844313e-08
1534 1.64664371027357e-08
1535 1.7404508056984e-08
1536 1.79358981000632e-08
1537 1.65213567271394e-08
1538 1.77559904557256e-08
1539 1.77056200811876e-08
1540 1.74898158178394e-08
1541 1.77538499457341e-08
1542 1.71449094921172e-08
1543 1.73814456161381e-08
1544 1.78518604343481e-08
1545 1.63669184871651e-08
1546 1.72279470689318e-08
1547 1.72418257449181e-08
1548 1.76775909466187e-08
1549 1.68715814652387e-08
1550 1.64281424019919e-08
1551 1.81220993766829e-08
1552 1.67883431601012e-08
1553 1.83924182550754e-08
1554 1.7397296048216e-08
1555 1.75207084396334e-08
1556 1.84016677451382e-08
1557 1.80646502201398e-08
1558 1.74172711808751e-08
1559 1.75523702239389e-08
1560 1.66726579209353e-08
1561 1.82132406933988e-08
1562 1.68205129824628e-08
1563 1.67138818341073e-08
1564 1.66620086616831e-08
1565 1.77013781410551e-08
1566 1.6403054914349e-08
1567 1.82309154439508e-08
1568 1.81054584658114e-08
1569 1.66771769727347e-08
1570 1.64837565819198e-08
1571 1.75357968146272e-08
1572 1.65330984458478e-08
1573 1.64184417172919e-08
1574 1.64206905850506e-08
1575 1.64371964928023e-08
1576 1.6368124633459e-08
1577 1.65371645266532e-08
1578 1.62000102221782e-08
1579 1.69096310287387e-08
1580 1.76486185665681e-08
1581 1.75972107996358e-08
1582 1.59866253568453e-08
1583 1.72256786612479e-08
1584 1.76199783652464e-08
1585 1.76900165627103e-08
1586 1.60608504273796e-08
1587 1.58480553125173e-08
1588 1.72709899715073e-08
1589 1.68799765276617e-08
1590 1.74008043529739e-08
1591 1.73131233793811e-08
1592 1.73682259685393e-08
1593 1.7820459774498e-08
1594 1.58794843940768e-08
1595 1.5808449660426e-08
1596 1.71515051050619e-08
1597 1.70060232562719e-08
1598 1.71370757584555e-08
1599 1.70636624829967e-08
1600 1.61907358631197e-08
1601 1.70032361523909e-08
1602 1.55631703080417e-08
1603 1.5554185495148e-08
1604 1.58243178560724e-08
1605 1.61654689634361e-08
1606 1.72100147466381e-08
1607 1.56829909059297e-08
1608 1.5455954738286e-08
1609 1.51969743455993e-08
1610 1.6937997671107e-08
1611 1.52710306622339e-08
1612 1.65614917335688e-08
1613 1.4822620464372e-08
1614 1.64306133143555e-08
1615 1.7569014687524e-08
1616 1.50680019572746e-08
1617 1.51984771434854e-08
1618 1.53228700838781e-08
1619 1.51504373491207e-08
1620 1.52479362469649e-08
1621 1.68890483820405e-08
1622 1.48253107568053e-08
1623 1.50947112587119e-08
1624 1.50497108108993e-08
1625 1.65959157527595e-08
1626 1.7508789085241e-08
1627 1.49699186380303e-08
1628 1.5831313149306e-08
1629 1.63180633450111e-08
1630 1.66197562379011e-08
1631 1.48944110378579e-08
1632 1.72342318194296e-08
1633 1.5613958126437e-08
1634 1.68407350287225e-08
1635 1.48395802312962e-08
1636 1.72465419723267e-08
1637 1.65388680528622e-08
1638 1.46815857249294e-08
1639 1.642867175633e-08
1640 1.61958002564688e-08
1641 1.67683396057328e-08
1642 1.4315008733945e-08
1643 1.70526242015967e-08
1644 1.59722777226534e-08
1645 1.57302189052189e-08
1646 1.5964628730103e-08
1647 1.45369529747086e-08
1648 1.61350932614823e-08
1649 1.62547912907485e-08
1650 1.57843036419081e-08
1651 1.65235931604002e-08
1652 1.53341126463147e-08
1653 1.6463701513203e-08
1654 1.62784807855587e-08
1655 1.61214277483168e-08
1656 1.51201025033743e-08
1657 1.56934767403527e-08
1658 1.47491574509218e-08
1659 1.44804213064731e-08
1660 1.55085082553796e-08
1661 1.5048541968099e-08
1662 1.41863871760961e-08
1663 1.64108922007244e-08
1664 1.49270320548567e-08
1665 1.49921177694523e-08
1666 1.44998253404083e-08
1667 1.46664396183382e-08
1668 1.41766083316952e-08
1669 1.56697232966962e-08
1670 1.46742094031538e-08
1671 1.48415510992095e-08
1672 1.45604364121255e-08
1673 1.45955940666909e-08
1674 1.62491353705718e-08
1675 1.42311931128347e-08
1676 1.43496814430932e-08
1677 1.7945103181205e-08
1678 1.4995981345578e-08
1679 1.46713690085676e-08
1680 1.45742324875187e-08
1681 1.44886342923201e-08
1682 1.45887071312245e-08
1683 1.43184317735745e-08
1684 1.47940886208175e-08
1685 1.44421861136834e-08
1686 1.53811381409241e-08
1687 1.48368979324687e-08
1688 1.47889194224149e-08
1689 1.65431544019157e-08
1690 1.46217731256115e-08
1691 1.48500083341219e-08
1692 1.4390121982899e-08
1693 1.43503147143065e-08
1694 1.44372860333419e-08
1695 1.71450658115191e-08
1696 1.45118148608958e-08
1697 1.45507472737449e-08
1698 1.4641251766534e-08
1699 1.42700713468003e-08
1700 1.37487736751041e-08
1701 1.65664957307854e-08
1702 1.6905424615743e-08
1703 1.45705163490106e-08
1704 1.45218415070758e-08
1705 1.76530541295961e-08
1706 1.7257953288663e-08
1707 1.42518050694207e-08
1708 1.69405822703084e-08
1709 1.41676999021456e-08
1710 1.69501515046022e-08
1711 1.45489904568308e-08
1712 1.42498501887189e-08
1713 1.41556313337787e-08
1714 1.62666129455147e-08
1715 1.46514373966511e-08
1716 1.43854164136314e-08
1717 1.43868073010367e-08
1718 1.65364006932123e-08
1719 1.40211042776173e-08
1720 1.40431177797495e-08
1721 1.38602826993406e-08
1722 1.44234562071688e-08
1723 1.40819418348315e-08
1724 1.38067504096284e-08
1725 1.66046554284094e-08
1726 1.64344164943486e-08
1727 1.75395431512015e-08
1728 1.43826035525763e-08
1729 1.40954599103793e-08
1730 1.39865496961988e-08
1731 1.36518334414859e-08
1732 1.35411557522502e-08
1733 1.41537270792469e-08
1734 1.36468765177256e-08
1735 1.35256845723575e-08
1736 1.36598510280805e-08
1737 1.39111007158021e-08
1738 1.35305748827363e-08
1739 1.35455078265068e-08
1740 1.36095890113097e-08
1741 1.33265887214407e-08
1742 1.40720759489454e-08
1743 1.32611823744355e-08
1744 1.36467654954231e-08
1745 1.38119835568773e-08
1746 1.40464564424292e-08
1747 1.34431115128564e-08
1748 1.34099922277642e-08
1749 1.36052875632231e-08
1750 1.35760886976755e-08
1751 1.2829436180084e-08
1752 1.35649038668362e-08
1753 1.29386688030309e-08
1754 1.31594690699899e-08
1755 1.27549899531232e-08
1756 1.30351311966592e-08
1757 1.34121318495772e-08
1758 1.36077424883752e-08
1759 1.36893785196435e-08
1760 1.4150866256557e-08
1761 1.3301620249706e-08
1762 1.39885569794274e-08
1763 1.35225226571833e-08
1764 1.35290303404645e-08
1765 1.26251133991673e-08
1766 1.3219024097566e-08
1767 1.35028725978259e-08
1768 1.35285498359394e-08
1769 1.38759403967015e-08
1770 1.567178387063e-08
1771 1.29896138290064e-08
1772 1.37007880596229e-08
1773 1.31013644377731e-08
1774 1.32172122135898e-08
1775 1.293026130611e-08
1776 1.28571722157744e-08
1777 1.28683694811116e-08
1778 1.33225963594441e-08
1779 1.3914016605554e-08
1780 1.35211237761723e-08
1781 1.35679201207495e-08
1782 1.28510340147159e-08
1783 1.28401698162861e-08
1784 1.36980178311319e-08
1785 1.3361714401583e-08
1786 1.34762858650106e-08
1787 1.35167832482352e-08
1788 1.31153949922691e-08
1789 1.27415784589857e-08
1790 1.32984823153492e-08
1791 1.23483605563024e-08
1792 1.25770043268858e-08
1793 1.27621024859081e-08
1794 1.26493606700251e-08
1795 1.46999949990345e-08
1796 1.22900365440159e-08
1797 1.22633654342508e-08
1798 1.24350254537831e-08
1799 1.24413759294839e-08
1800 1.26401227262818e-08
1801 1.22110295208699e-08
1802 1.25631771652479e-08
1803 1.25991208577148e-08
1804 1.26549837276002e-08
1805 1.24446009053258e-08
1806 1.24959456115903e-08
1807 1.26297132752029e-08
1808 1.22208465569429e-08
1809 1.25615340351715e-08
1810 1.25657262373124e-08
1811 1.25425154706704e-08
1812 1.25216947921558e-08
1813 1.28819150901904e-08
1814 1.23369563453934e-08
1815 1.25865353695076e-08
1816 1.24593571015907e-08
1817 1.25476171675132e-08
1818 1.28471020488519e-08
1819 1.26433388203395e-08
1820 1.26209069861716e-08
1821 1.2707370267151e-08
1822 1.22623555753876e-08
1823 1.2496645496185e-08
1824 1.27927464177446e-08
1825 1.20691359128955e-08
1826 1.24439472060089e-08
1827 1.19098739759238e-08
1828 1.24520402877693e-08
1829 1.21425483001758e-08
1830 1.21575656208961e-08
1831 1.20537180237079e-08
1832 1.21985150869364e-08
1833 1.21007595055289e-08
1834 1.21011245468594e-08
1835 1.19869092429781e-08
1836 1.20957901472707e-08
1837 1.21757866011762e-08
1838 1.20717427165573e-08
1839 1.20013270432651e-08
1840 1.19934711051428e-08
1841 1.1777513186928e-08
1842 1.2188436926408e-08
1843 1.21203038716544e-08
1844 1.20113616830508e-08
1845 1.23322774214785e-08
1846 1.2155846995654e-08
1847 1.22322498796734e-08
1848 1.20818226534425e-08
1849 1.20088730071188e-08
1850 1.23797265771941e-08
1851 1.22835031035606e-08
1852 1.2323301490369e-08
1853 1.24552537172917e-08
1854 1.2305775065613e-08
1855 1.22405934277481e-08
1856 1.21846870371201e-08
1857 1.21131789043716e-08
1858 1.21496563920687e-08
1859 1.21800685093376e-08
1860 1.20753362864434e-08
1861 1.21742456116181e-08
1862 1.26313723924909e-08
1863 1.21585781442946e-08
1864 1.21291146015778e-08
1865 1.27494352852864e-08
1866 1.21760743709842e-08
1867 1.2022774775744e-08
1868 1.21836052358049e-08
1869 1.22399068658297e-08
1870 1.2266713866893e-08
1871 1.20142127357781e-08
1872 1.22998145002384e-08
1873 1.2596624188177e-08
1874 1.17823510947801e-08
1875 1.21446408485326e-08
1876 1.26218546725454e-08
1877 1.29266890525059e-08
1878 1.21052394774779e-08
1879 1.18162741813421e-08
1880 1.20096110833856e-08
1881 1.20490364352577e-08
1882 1.25313039944785e-08
1883 1.19416885269175e-08
1884 1.20191270269743e-08
1885 1.21751151382909e-08
1886 1.1945615163711e-08
1887 1.21435732580721e-08
1888 1.23039747279563e-08
1889 1.21993464219372e-08
1890 1.18405925064735e-08
1891 1.2373898350404e-08
1892 1.19058194414379e-08
1893 1.2161869733518e-08
1894 1.19647127760913e-08
1895 1.19418466226762e-08
1896 1.17860512460766e-08
1897 1.21111529693962e-08
1898 1.21410375086839e-08
1899 1.2017211226123e-08
1900 1.18996972275909e-08
1901 1.23456365130892e-08
1902 1.21747136816452e-08
1903 1.19121574826409e-08
1904 1.19107514962025e-08
1905 1.20244569856709e-08
1906 1.18776073421145e-08
1907 1.17239125074775e-08
1908 1.17370060337407e-08
1909 1.17963869783466e-08
1910 1.19591332392588e-08
1911 1.20096048661367e-08
1912 1.2155127571134e-08
1913 1.20789760416073e-08
1914 1.17921921116704e-08
1915 1.23711485500166e-08
1916 1.19666010434116e-08
1917 1.19386953656431e-08
1918 1.18092406964365e-08
1919 1.1806537081327e-08
1920 1.18081029398809e-08
1921 1.2185711106838e-08
1922 1.23007497521144e-08
1923 1.21111654038941e-08
1924 1.20303056405646e-08
1925 1.18548681982134e-08
1926 1.22214238729157e-08
1927 1.18733911591562e-08
1928 1.20998562280761e-08
1929 1.17075904526587e-08
1930 1.17639133989655e-08
1931 1.18825012052071e-08
1932 1.16771721181408e-08
1933 1.20003775805344e-08
1934 1.17294822743474e-08
1935 1.21401599884052e-08
1936 1.19336149850824e-08
1937 1.1768857888228e-08
1938 1.18725402842301e-08
1939 1.19203829029857e-08
1940 1.17109957287198e-08
1941 1.21485062010152e-08
1942 1.17582024117269e-08
1943 1.26843726633297e-08
1944 1.23464554135921e-08
1945 1.16800906724279e-08
1946 1.17696146162416e-08
1947 1.20311378637439e-08
1948 1.1923183329543e-08
1949 1.19913856622134e-08
1950 1.20909433576344e-08
1951 1.22476118136206e-08
1952 1.19080256766324e-08
1953 1.20609700005048e-08
1954 1.22708110339431e-08
1955 1.18768346268894e-08
1956 1.1912244524126e-08
1957 1.24516894572935e-08
1958 1.21119620999366e-08
1959 1.18465273146739e-08
1960 1.17922756004418e-08
1961 1.19768523987318e-08
1962 1.18972449669741e-08
1963 1.19113634511336e-08
1964 1.18236389567983e-08
1965 1.18634675416729e-08
1966 1.20972387662732e-08
1967 1.19412577603839e-08
1968 1.1753500395173e-08
1969 1.20223040411815e-08
1970 1.20852341467526e-08
1971 1.16380531878235e-08
1972 1.23801573437277e-08
1973 1.18425473871753e-08
1974 1.13621085873206e-08
1975 1.12497353654817e-08
1976 1.15398011146794e-08
1977 1.20305969630863e-08
1978 1.195969456802e-08
1979 1.1912111297363e-08
1980 1.18453442610189e-08
1981 1.24402941281687e-08
1982 1.15108740317282e-08
1983 1.18253460357209e-08
1984 1.18768417323167e-08
1985 1.19368079865012e-08
1986 1.16691447615835e-08
1987 1.15996412475283e-08
1988 1.19276339916041e-08
1989 1.18164349416361e-08
1990 1.13027196491089e-08
1991 1.14674918449964e-08
1992 1.15752767371191e-08
1993 1.15288329993746e-08
1994 1.16595071375514e-08
1995 1.16590221921342e-08
1996 1.20207799270133e-08
1997 1.16800942251416e-08
1998 1.18323537634524e-08
1999 1.15560947477888e-08
2000 1.20186944840839e-08
2001 1.16571783337349e-08
2002 1.27923600601321e-08
2003 1.18307976748611e-08
2004 1.15046185911183e-08
2005 1.18796750214756e-08
2006 1.16856471166216e-08
2007 1.14821192553904e-08
2008 1.17362111140551e-08
2009 1.13139115853755e-08
2010 1.1381072972938e-08
2011 1.1622999451788e-08
2012 1.19424141686864e-08
2013 1.17108580610648e-08
2014 1.19453833491434e-08
2015 1.15889031704342e-08
2016 1.16257954374532e-08
2017 1.1596840820971e-08
2018 1.19024656797251e-08
2019 1.15867502259448e-08
2020 1.1394517329677e-08
2021 1.21712018241737e-08
2022 1.16088614277032e-08
2023 1.20405951875568e-08
2024 1.19356915462276e-08
2025 1.1528480392542e-08
2026 1.17212879402473e-08
2027 1.14526530481385e-08
2028 1.1188399540174e-08
2029 1.10563354027704e-08
2030 1.1487078843686e-08
2031 1.15775247166994e-08
2032 1.1739182070869e-08
2033 1.19370602291724e-08
2034 1.1813931166671e-08
2035 1.21698260358016e-08
2036 1.1702084634635e-08
2037 1.15757226026858e-08
2038 1.22367111998756e-08
2039 1.1616307915574e-08
2040 1.18243175251109e-08
2041 1.16855378706759e-08
2042 1.18617542455013e-08
2043 1.18741461108129e-08
2044 1.13778844124113e-08
2045 1.1495905560821e-08
2046 1.1556884338404e-08
2047 1.16930918281355e-08
2048 1.17855742942652e-08
2049 1.17398988308537e-08
2050 1.18646701352532e-08
2051 1.08848992041999e-08
2052 1.16095986157916e-08
2053 1.15389173771518e-08
2054 1.08916937691106e-08
2055 1.174196917475e-08
2056 1.14968230491286e-08
2057 1.13140110613585e-08
2058 1.14101252890464e-08
2059 1.17845155855889e-08
2060 1.16889653511976e-08
2061 1.18539800197937e-08
2062 1.10256834773281e-08
2063 1.10617319748485e-08
2064 1.1360710594488e-08
2065 1.16266480887361e-08
2066 1.27687709294833e-08
2067 1.13261551248911e-08
2068 1.12857820866452e-08
2069 1.18024710005216e-08
2070 1.14535572137697e-08
2071 1.21621219761892e-08
2072 1.15459846128374e-08
2073 1.14688294416965e-08
2074 1.10390176999431e-08
2075 1.07466808785261e-08
2076 1.14314824273265e-08
2077 1.16549259132626e-08
2078 1.1442559788577e-08
2079 1.11751230491564e-08
2080 1.09160716021961e-08
2081 1.15021379087921e-08
2082 1.15633360664447e-08
2083 1.127299320558e-08
2084 1.1817812506365e-08
2085 1.13900062714833e-08
2086 1.13960574310568e-08
2087 1.22892860332513e-08
2088 1.13803828583059e-08
2089 1.1434922342346e-08
2090 1.18318652653215e-08
2091 1.13844569327171e-08
2092 1.1259028376287e-08
2093 1.08396411846456e-08
2094 1.15213563134375e-08
2095 1.12961648923715e-08
2096 1.11026103866152e-08
2097 1.1229380092459e-08
2098 1.23099939131066e-08
2099 1.13051568106926e-08
2100 1.11786038203832e-08
2101 1.16657821180866e-08
2102 1.14086349256581e-08
2103 1.16721983189905e-08
2104 1.18925020942129e-08
2105 1.15964002844748e-08
2106 1.11365645594219e-08
2107 1.11869935537356e-08
2108 1.15926912513942e-08
2109 1.22528760471141e-08
2110 1.15965157476694e-08
2111 1.17780034614157e-08
2112 1.19295977540901e-08
2113 1.14458353905889e-08
2114 1.14988081278966e-08
2115 1.1687423473461e-08
2116 1.14527614059057e-08
2117 1.10697220279121e-08
2118 1.21583321188723e-08
2119 1.15243778964214e-08
2120 1.13027756043493e-08
2121 1.08495221695648e-08
2122 1.12353832903977e-08
2123 1.14761995462231e-08
2124 1.06560955615009e-08
2125 1.11571347716222e-08
2126 1.13620579611506e-08
2127 1.16298588537234e-08
2128 1.20022383143237e-08
2129 1.16610188172217e-08
2130 1.14012719265588e-08
2131 1.05802273608901e-08
2132 1.10551390264391e-08
2133 1.14804077355757e-08
2134 1.14621485636235e-08
2135 1.11346016851144e-08
2136 1.07930793191713e-08
2137 1.17559535439682e-08
2138 1.11267945968052e-08
2139 1.17724985315704e-08
2140 1.11226290400168e-08
2141 1.09503428546986e-08
2142 1.15933476152463e-08
2143 1.14467937351037e-08
2144 1.08047384372867e-08
2145 1.14087317371059e-08
2146 1.11944968850253e-08
2147 1.17891403306203e-08
2148 1.11737739061368e-08
2149 1.04594475303088e-08
2150 1.16363381152951e-08
2151 1.17344240990747e-08
2152 1.12617355441103e-08
2153 1.08812550081439e-08
2154 1.12984759326196e-08
2155 1.08521094333014e-08
2156 1.1494551088731e-08
2157 1.13729239359373e-08
2158 1.13767510967477e-08
2159 1.14252918237412e-08
2160 1.06330126925513e-08
2161 1.12889191328236e-08
2162 1.07445448094268e-08
2163 1.17595151394312e-08
2164 1.1129058563597e-08
2165 1.08862732162152e-08
2166 1.08396438491809e-08
2167 1.07957802697456e-08
2168 1.08883995153519e-08
2169 1.12834310783683e-08
2170 1.12822640119248e-08
2171 1.067365396068e-08
2172 1.09881570509174e-08
2173 1.14978213616723e-08
2174 1.14820455365816e-08
2175 1.10339151149219e-08
2176 1.14148619445587e-08
2177 1.14266631712212e-08
2178 1.18679066574146e-08
2179 1.09408011539358e-08
2180 1.12088578418934e-08
2181 1.08064392989604e-08
2182 1.1957870249546e-08
2183 1.12474003444163e-08
2184 1.07874864596624e-08
2185 1.07982227603998e-08
2186 1.04602877470938e-08
2187 1.11042304240527e-08
2188 1.06851993919577e-08
2189 1.07273017135867e-08
2190 1.09806777004451e-08
2191 1.06086686102458e-08
2192 1.04140323031743e-08
2193 1.09309787887923e-08
2194 1.04131236966509e-08
2195 1.09191713448809e-08
2196 1.08411786214901e-08
2197 1.10874420755636e-08
2198 1.1118020282197e-08
2199 1.07955404615723e-08
2200 1.06018553935883e-08
2201 1.08630935358178e-08
2202 1.07916431346666e-08
2203 1.06029709456834e-08
2204 1.11200781915954e-08
2205 1.08664934828084e-08
2206 1.06310649172769e-08
2207 1.07932009996148e-08
2208 1.05817159479216e-08
2209 1.04015738244811e-08
2210 1.10294262611887e-08
2211 1.17476623984203e-08
2212 1.08698561263054e-08
2213 1.02775521426679e-08
2214 1.13656062339373e-08
2215 1.04649391374778e-08
2216 1.10896403171523e-08
2217 1.05171160669215e-08
2218 1.06781543607326e-08
2219 1.17100444896323e-08
2220 1.14332161516018e-08
2221 1.0599967126268e-08
2222 1.03077102409088e-08
2223 1.05933439797923e-08
2224 1.03267749906877e-08
2225 1.19656782260336e-08
2226 1.12110516425901e-08
2227 1.08764579564991e-08
2228 1.07629594126024e-08
2229 1.03112558491603e-08
2230 1.06876063554751e-08
2231 1.114504666333e-08
2232 1.03403960949322e-08
2233 1.044590192123e-08
2234 1.14691998120975e-08
2235 1.06154738332975e-08
2236 1.08768203332943e-08
2237 1.09486251176349e-08
2238 1.17012133316052e-08
2239 1.07488977718617e-08
2240 1.13537206303249e-08
2241 1.07443591801371e-08
2242 1.05671693617637e-08
2243 1.12099138860344e-08
2244 1.04844621873212e-08
2245 1.13117675226704e-08
2246 1.06809947553188e-08
2247 1.11557447723953e-08
2248 1.05131823247007e-08
2249 1.01874153557446e-08
2250 1.02011696867521e-08
2251 1.0535515571064e-08
2252 1.02635597798439e-08
2253 1.02580965943844e-08
2254 1.0602593469855e-08
2255 1.03520747529728e-08
2256 1.032662755307e-08
2257 1.06319735238003e-08
2258 1.02685930869484e-08
2259 1.05258122218288e-08
2260 1.06868887073119e-08
2261 1.15793081789661e-08
2262 1.11704041572125e-08
2263 1.15153966362413e-08
2264 1.03092077097244e-08
2265 1.07983542108059e-08
2266 1.0433352848338e-08
2267 1.08365654227782e-08
2268 1.04188915273085e-08
2269 1.04949435808521e-08
2270 1.10477644810203e-08
2271 1.03252952854405e-08
2272 1.08498676709701e-08
2273 1.02786588129788e-08
2274 1.020805573404e-08
2275 1.04178017323875e-08
2276 1.03420836339296e-08
2277 1.02085735420587e-08
2278 1.01060324553259e-08
2279 1.02248529643134e-08
2280 1.06915480913017e-08
2281 1.02250981015573e-08
2282 1.05566195784945e-08
2283 1.0307160458467e-08
2284 1.00604538033622e-08
2285 1.13399218903965e-08
2286 1.03387627348184e-08
2287 1.03331281309238e-08
2288 1.05445145948124e-08
2289 1.0766414426655e-08
2290 1.02628003872951e-08
2291 1.04360982078333e-08
2292 1.03909716386852e-08
2293 1.0596289179432e-08
2294 1.02014610092738e-08
2295 1.01868780078007e-08
2296 1.04152704238913e-08
2297 1.05402229166884e-08
2298 1.01890016424022e-08
2299 1.11311502237754e-08
2300 1.02961550396685e-08
2301 1.06376703001843e-08
2302 1.02471240381874e-08
2303 1.07906723556539e-08
2304 1.04611341811278e-08
2305 1.00719308449015e-08
2306 1.03240109794456e-08
2307 1.01324957313409e-08
2308 1.05162296648587e-08
2309 1.08321298597502e-08
2310 1.029726703905e-08
2311 1.05160937735604e-08
2312 1.02882493635548e-08
2313 1.04020587698983e-08
2314 1.0317330989551e-08
2315 1.04049862059696e-08
2316 1.09497859668295e-08
2317 1.04465938122189e-08
2318 1.02326653816931e-08
2319 1.02716510852474e-08
2320 1.05767599123396e-08
2321 1.1526091192593e-08
2322 1.0610419209911e-08
2323 1.03535029438717e-08
2324 1.01804893404278e-08
2325 1.06393329701859e-08
2326 1.01503747629295e-08
2327 9.91845183762052e-09
2328 1.01290513754293e-08
2329 1.01469614932626e-08
2330 1.01909929384192e-08
2331 1.00973922556591e-08
2332 1.00727728380434e-08
2333 1.02427346604372e-08
2334 1.03189083944244e-08
2335 1.06649657993785e-08
2336 1.00876116349014e-08
2337 1.04225552632897e-08
2338 1.01191064416639e-08
2339 1.02850217231776e-08
2340 1.01315764666765e-08
2341 1.01120187778747e-08
2342 1.05194137844933e-08
2343 1.02054755757308e-08
2344 1.01551034248359e-08
2345 1.0699137575898e-08
2346 1.03105968207728e-08
2347 1.036051244796e-08
2348 1.01331760760104e-08
2349 1.00719672602168e-08
2350 1.01062687107856e-08
2351 1.0270881922736e-08
2352 1.02620560937794e-08
2353 1.01184918221975e-08
2354 1.01052277656777e-08
2355 1.00380450618331e-08
2356 9.83669856680081e-09
2357 1.01302637389722e-08
2358 1.00112824696907e-08
2359 1.03035180387678e-08
2360 1.01764250359793e-08
2361 1.12472013924503e-08
2362 1.07538689064768e-08
2363 1.06486650608417e-08
2364 9.96820048726477e-09
2365 1.06093152041353e-08
2366 1.01857207113198e-08
2367 9.96506077655113e-09
2368 1.05052224697033e-08
2369 9.84122117131392e-09
2370 9.95979387852231e-09
2371 1.00574295558431e-08
2372 1.03564463671546e-08
2373 9.92108795117019e-09
2374 9.80725989307984e-09
2375 1.01419814768633e-08
2376 9.98239091387632e-09
2377 1.08967501688539e-08
2378 1.00000976388515e-08
2379 1.03276498464311e-08
2380 1.04384154653303e-08
2381 1.01446087086288e-08
2382 9.93167503793302e-09
2383 1.00713908324224e-08
2384 9.93510695934674e-09
2385 9.89069093293438e-09
2386 1.02203578933313e-08
2387 9.91956827789409e-09
2388 1.00743138276016e-08
2389 9.8870085452063e-09
2390 9.84319203922723e-09
2391 1.00735375596628e-08
2392 1.05430233432457e-08
2393 1.02383301836539e-08
2394 1.02752704123077e-08
2395 1.0456409071935e-08
2396 1.04543929069223e-08
2397 1.00405124214831e-08
2398 9.98741533919656e-09
2399 1.04402806400117e-08
2400 1.01097512583692e-08
2401 1.12252021011727e-08
2402 9.98171323374208e-09
2403 1.00991828233532e-08
2404 1.00523243062867e-08
2405 9.90366455511094e-09
2406 1.02797397261156e-08
2407 1.01246602213223e-08
2408 9.85749082360599e-09
2409 1.03772679338476e-08
2410 1.00822443727111e-08
2411 1.00004307057588e-08
2412 9.80660797011979e-09
2413 9.89302240128609e-09
2414 9.76437686261988e-09
2415 1.02811927860103e-08
2416 9.85614967419224e-09
2417 9.9123509400556e-09
2418 9.91105508774126e-09
2419 9.86924408863388e-09
2420 1.01193906587582e-08
2421 9.97443017070054e-09
2422 9.8223837952105e-09
2423 1.00355075360881e-08
2424 1.00271213554493e-08
2425 9.76483782721971e-09
2426 9.95734072972709e-09
2427 1.01154977727447e-08
2428 1.02155714998275e-08
2429 1.02018624659195e-08
2430 9.6371479685331e-09
2431 9.77008429714488e-09
2432 9.91947146644634e-09
2433 9.91595694443959e-09
2434 1.01925134998737e-08
2435 1.02245882871443e-08
2436 9.87802373231261e-09
2437 9.34434130073214e-09
2438 1.04307957826677e-08
2439 1.01569206378826e-08
2440 1.08310622692898e-08
2441 9.96689575316623e-09
2442 1.03568504883356e-08
2443 9.77337499818987e-09
2444 9.97020332960119e-09
2445 1.03096535752911e-08
2446 1.00403294567286e-08
2447 9.39917033093707e-09
2448 9.70391589305564e-09
2449 1.01071897518068e-08
2450 1.03823643016199e-08
2451 1.00904582467365e-08
2452 1.0198737854239e-08
2453 9.84686199245743e-09
2454 9.85773684902824e-09
2455 1.00293000571128e-08
2456 1.01890238468627e-08
2457 1.01088968307295e-08
2458 9.77019531944734e-09
2459 1.09563842443094e-08
2460 1.00042294448599e-08
2461 9.80517889104249e-09
2462 1.03365795922628e-08
2463 9.80001058081825e-09
2464 9.69593116906253e-09
2465 1.05794812910176e-08
2466 1.00984092199496e-08
2467 1.08235234108633e-08
2468 1.04336361772539e-08
2469 9.7733314774473e-09
2470 1.10606341863217e-08
2471 9.89471882206772e-09
2472 1.00980717121502e-08
2473 9.65485824622192e-09
2474 1.03923749605883e-08
2475 9.88018555858616e-09
2476 9.79986669591426e-09
2477 1.02017772007912e-08
2478 9.80237135905782e-09
2479 9.74702984990472e-09
2480 1.05134265737661e-08
2481 9.85941017717096e-09
2482 1.00222123933236e-08
2483 9.71069269439795e-09
2484 1.00679669046144e-08
2485 9.96034277278568e-09
2486 1.00417958392995e-08
2487 9.85547288223643e-09
2488 9.60731405541537e-09
2489 1.00627266519382e-08
2490 9.70207292283476e-09
2491 9.55691703552475e-09
2492 1.0006163897458e-08
2493 1.01239514549434e-08
2494 1.0133130778911e-08
2495 9.77305347760193e-09
2496 1.07057855913695e-08
2497 9.97560256621455e-09
2498 9.75072467213067e-09
2499 1.0141094186622e-08
2500 9.80841718956071e-09
2501 1.01225774429281e-08
2502 9.67245661342986e-09
2503 1.09624540556297e-08
2504 9.95677584825216e-09
2505 9.55843049155192e-09
2506 1.01720845080422e-08
2507 9.75196190466932e-09
2508 9.97980365013973e-09
2509 9.63357837946432e-09
2510 9.73140767968061e-09
2511 1.01336912194938e-08
2512 1.05929345295408e-08
2513 9.71816049855079e-09
2514 9.75971214955962e-09
2515 9.93563009643594e-09
2516 1.07108961699964e-08
2517 9.78943770491014e-09
2518 1.05228394886581e-08
2519 9.78066161394509e-09
2520 1.0176919751359e-08
2521 9.42029565464964e-09
2522 9.84268666570642e-09
2523 9.85006476383887e-09
2524 9.84824932714901e-09
2525 9.92738069527377e-09
2526 1.02854977868105e-08
2527 1.03072554935579e-08
2528 9.73009761651156e-09
2529 1.07290780704261e-08
2530 1.04325623695445e-08
2531 9.55694279269892e-09
2532 9.89522597194537e-09
2533 9.93346027655662e-09
2534 9.99831506476312e-09
2535 1.1472899963394e-08
2536 9.85505632655759e-09
2537 1.06809139310826e-08
2538 1.02308224114722e-08
2539 9.9926156238439e-09
2540 9.62395763082213e-09
2541 1.21166339184242e-08
2542 1.06026263324566e-08
2543 1.04393444999573e-08
2544 1.00003427760953e-08
2545 1.0347643630837e-08
2546 1.04987174509574e-08
2547 9.37346378293569e-09
2548 9.54605461345182e-09
2549 1.03015445063193e-08
2550 1.03386224026281e-08
2551 1.03330233258703e-08
2552 1.01940829111413e-08
2553 1.01314805434072e-08
2554 1.09867288600185e-08
2555 1.0365627467479e-08
2556 1.01755528447711e-08
2557 1.08325632908191e-08
2558 1.04154818103552e-08
2559 1.03085566749428e-08
2560 1.0001933503645e-08
2561 1.07555653272584e-08
2562 1.01017372244883e-08
2563 1.01057855417253e-08
2564 1.03957917829689e-08
2565 1.0042996656523e-08
2566 1.0368719216558e-08
2567 1.05906243774712e-08
2568 1.11022169235753e-08
2569 1.01435775334835e-08
2570 1.04180886140171e-08
2571 1.06577848768552e-08
2572 1.02071258112346e-08
2573 1.0166909980569e-08
2574 1.05268291861194e-08
2575 1.00501242883411e-08
2576 1.01835029298059e-08
2577 1.00973940320159e-08
2578 9.93976456697965e-09
2579 1.03185726629818e-08
2580 1.01476231861852e-08
2581 1.00630952459824e-08
2582 1.00628740895559e-08
2583 1.04750368379314e-08
2584 1.05564135211011e-08
2585 1.02953086056345e-08
2586 1.06159925294946e-08
2587 9.97162263871587e-09
2588 9.9823900256979e-09
2589 9.80211911638662e-09
2590 9.84113146529353e-09
2591 9.97007543190875e-09
2592 1.03031219111926e-08
2593 1.01966275423138e-08
2594 1.03501971437936e-08
2595 9.72949187882932e-09
2596 1.02528314727124e-08
2597 1.04718811400062e-08
2598 1.0406298933674e-08
2599 9.80437775410792e-09
2600 1.00014512227631e-08
2601 1.02432444748501e-08
2602 1.00288417570482e-08
2603 9.90923787469455e-09
2604 9.80574910158794e-09
2605 1.02934025747459e-08
2606 1.06516546694024e-08
2607 1.05113047155214e-08
2608 1.00683585912975e-08
2609 1.00988426510185e-08
2610 1.02935748813593e-08
2611 1.00769073085871e-08
2612 1.10660876018187e-08
2613 9.91937820771227e-09
2614 1.02078949737461e-08
2615 1.03089456970906e-08
2616 1.04961301872208e-08
2617 1.0083987866949e-08
2618 1.00466071017991e-08
2619 1.05448538789688e-08
2620 1.00118615620204e-08
2621 9.93282878170021e-09
2622 1.00185157947408e-08
2623 9.89641613102776e-09
2624 1.01962447374149e-08
2625 9.89582638055708e-09
2626 9.83640369156547e-09
2627 1.0163697439225e-08
2628 1.04645891951805e-08
2629 9.88670567636518e-09
2630 1.02228563392259e-08
2631 9.75369296440931e-09
2632 1.04171178350043e-08
2633 1.00323918061918e-08
2634 9.7785930464056e-09
2635 9.81579795222842e-09
2636 9.86699344451836e-09
2637 9.84020687155862e-09
2638 1.02182804440076e-08
2639 1.0132555239295e-08
2640 9.91161641650251e-09
2641 9.81357661800075e-09
2642 9.98674565266811e-09
2643 9.96937821184929e-09
2644 1.00362509414254e-08
2645 1.00021724236399e-08
2646 9.93041915364756e-09
2647 1.06311226488742e-08
2648 1.0532601457669e-08
2649 9.85873427339357e-09
2650 1.08268274345846e-08
2651 1.00231858368716e-08
2652 9.94081528205015e-09
2653 1.1330613780558e-08
2654 1.01815960107388e-08
2655 1.01577555255972e-08
2656 9.82304104724108e-09
2657 9.89108261961746e-09
2658 9.89244153259961e-09
2659 9.95357396504915e-09
2660 9.87849535505347e-09
2661 1.00896828669761e-08
2662 9.6692032158785e-09
2663 1.00534496283444e-08
2664 9.58724921673593e-09
2665 1.05849498055477e-08
2666 1.00285131310329e-08
2667 9.47328793188262e-09
2668 9.86559811622101e-09
2669 9.92752458017776e-09
2670 1.00343084952215e-08
2671 1.01648582884195e-08
2672 9.92505277963573e-09
2673 9.59178070303324e-09
2674 1.00434434102681e-08
2675 1.01515871264724e-08
2676 1.0223806690135e-08
2677 9.67588764666516e-09
2678 1.00883239539939e-08
2679 9.72653957376224e-09
2680 9.91500748170893e-09
2681 1.02249888556116e-08
2682 1.10773452632884e-08
2683 9.68569313641865e-09
2684 9.99441684967906e-09
2685 1.00528056989901e-08
2686 9.77602887530793e-09
2687 9.92883020245472e-09
2688 9.54726875335155e-09
2689 9.95412730020462e-09
2690 9.5612158190761e-09
2691 9.65770574623548e-09
2692 9.74948211052151e-09
2693 9.91326398747105e-09
2694 9.861251371035e-09
2695 1.03934700845798e-08
2696 1.01934505281065e-08
2697 9.81701653302025e-09
2698 9.72613278804602e-09
2699 9.34416544140504e-09
2700 9.74708846968042e-09
2701 9.68698277148405e-09
2702 9.88156756420722e-09
2703 9.73889147104501e-09
2704 9.43012867793414e-09
2705 1.05681250417433e-08
2706 9.95203119913413e-09
2707 9.91161108743199e-09
2708 1.02017096992313e-08
2709 9.85386083840467e-09
2710 9.54246903717149e-09
2711 9.57023260639289e-09
2712 9.97103910549413e-09
2713 9.5785734899323e-09
2714 9.67285185282662e-09
2715 1.1158956425561e-08
2716 9.76599956459268e-09
2717 9.81521619536352e-09
2718 9.7040135926818e-09
2719 9.64836921468759e-09
2720 9.64807167491699e-09
2721 9.92525439613701e-09
2722 9.7334087456602e-09
2723 9.97545335224004e-09
2724 9.78781500293735e-09
2725 9.56714529820601e-09
2726 1.02944710533848e-08
2727 1.10928644048158e-08
2728 9.86112702605624e-09
2729 9.89862325440072e-09
2730 1.00256549728783e-08
2731 9.99404115020752e-09
2732 1.03784119076522e-08
2733 1.03607131762828e-08
2734 1.02323660655657e-08
2735 9.75971836680856e-09
2736 9.83637526985603e-09
2737 1.07719104747162e-08
2738 1.0055893007177e-08
2739 1.03231840853368e-08
2740 9.89154536057413e-09
2741 9.92014204115321e-09
2742 9.93787008241043e-09
2743 9.67571267551648e-09
2744 9.75427116856054e-09
2745 9.65080371173599e-09
2746 9.3878105289491e-09
2747 9.41392652720197e-09
2748 9.43420808141582e-09
2749 9.90322313043634e-09
2750 9.9165795575118e-09
2751 9.64614166321098e-09
2752 9.71095737156702e-09
2753 1.00794999013942e-08
2754 9.75674829817308e-09
2755 9.92739401795006e-09
2756 1.00750909837188e-08
2757 9.69606528400391e-09
2758 9.6255341475171e-09
2759 1.00275681091944e-08
2760 9.92126558685413e-09
2761 9.9883097348652e-09
2762 1.04152935165303e-08
2763 9.93473570076731e-09
2764 9.89892257052816e-09
2765 9.45692857357017e-09
2766 9.61192192505678e-09
2767 9.5279659717562e-09
2768 9.95946525250702e-09
2769 9.94901139250715e-09
2770 9.93757165446141e-09
2771 9.9691801480617e-09
2772 9.49631928648387e-09
2773 9.79233316655836e-09
2774 9.77892877784825e-09
2775 9.53567180772552e-09
2776 1.10130669028763e-08
2777 9.7068628690522e-09
2778 1.01230446247769e-08
2779 9.77986758243787e-09
2780 1.02424264625256e-08
2781 9.94744819848847e-09
2782 9.55024859194964e-09
2783 9.91422322016433e-09
2784 9.60817114759038e-09
2785 9.62901669510074e-09
2786 1.00151389403891e-08
2787 1.0208935918854e-08
2788 9.90483783880336e-09
2789 9.22676957060276e-09
2790 9.68935953693517e-09
2791 9.55513801415009e-09
2792 9.61554036393863e-09
2793 1.02331831897118e-08
2794 9.64625890276238e-09
2795 9.52922452057692e-09
2796 9.5450891635096e-09
2797 9.42339539733439e-09
2798 9.59045998172314e-09
2799 9.7072865301584e-09
2800 9.90117055010842e-09
2801 9.74119540586571e-09
2802 9.05268215944943e-09
2803 9.33365029709421e-09
2804 9.37777588916333e-09
2805 9.52395406983442e-09
2806 9.85823245258644e-09
2807 9.9376658013739e-09
2808 9.93032855944875e-09
2809 9.59050083793045e-09
2810 9.91889415047353e-09
2811 9.64359614386012e-09
2812 9.51834966400611e-09
2813 9.8053147823407e-09
2814 9.54887546811278e-09
2815 9.92190241078106e-09
2816 9.83446923896736e-09
2817 1.00183514817331e-08
2818 9.81752901196842e-09
2819 9.5348502426873e-09
2820 9.62319024466751e-09
2821 9.54345136250367e-09
2822 9.72919700359398e-09
2823 9.95583793184096e-09
2824 9.7222185857504e-09
2825 9.59605728212409e-09
2826 9.52985779179016e-09
2827 9.45325506762629e-09
2828 9.88281723124373e-09
2829 9.87365478266611e-09
2830 9.93175675034763e-09
2831 9.90659021482543e-09
2832 1.01211306002824e-08
2833 9.65611768322105e-09
2834 9.00945895665473e-09
2835 9.78325243039535e-09
2836 9.86925563495333e-09
2837 9.7497565576532e-09
2838 9.57861079342592e-09
2839 1.00197379282463e-08
2840 9.48856548887989e-09
2841 9.39124245036282e-09
2842 9.21870046965978e-09
2843 9.7272003785065e-09
2844 9.32730426228545e-09
2845 9.8337586962316e-09
2846 1.00538919411974e-08
2847 9.77643033195363e-09
2848 9.68616298280267e-09
2849 9.78056569067576e-09
2850 1.02264925416762e-08
2851 1.00350563414509e-08
2852 9.79446124205197e-09
2853 9.75578284823087e-09
2854 9.93642501612158e-09
2855 9.77700320703434e-09
2856 9.60969881447227e-09
2857 9.45719236256082e-09
2858 9.49112788362072e-09
2859 9.56778212213294e-09
2860 1.04965192093687e-08
2861 9.62492396894277e-09
2862 9.25874310553354e-09
2863 9.2347018920691e-09
2864 9.51116163605548e-09
2865 9.66750413056161e-09
2866 9.51249834457712e-09
2867 1.01904822358279e-08
2868 9.67940128049349e-09
2869 9.76857617018823e-09
2870 1.01011394804118e-08
2871 9.67525970452243e-09
2872 9.91267690153563e-09
2873 9.5423571266906e-09
2874 9.70168212433009e-09
2875 9.58048396171307e-09
2876 9.75964109528604e-09
2877 9.5631049745748e-09
2878 9.82789494230474e-09
2879 9.93790383319038e-09
2880 9.70431379698766e-09
2881 9.93705029372904e-09
2882 9.14081788039312e-09
2883 9.34402777374999e-09
2884 9.22293352800807e-09
2885 8.96126906013706e-09
2886 9.28180909909315e-09
2887 9.50489997819659e-09
2888 9.57412193969276e-09
2889 9.22128684521795e-09
2890 9.57138901469534e-09
2891 9.53685042048846e-09
2892 9.5774828068329e-09
2893 9.46833722537122e-09
2894 9.54958245813486e-09
2895 9.41247968455627e-09
2896 1.01652730677415e-08
2897 9.68825553115948e-09
2898 9.54361301097606e-09
2899 9.75269820457925e-09
2900 9.82376757718839e-09
2901 9.66986402062275e-09
2902 9.51608569721429e-09
2903 9.13174691419272e-09
2904 9.38608035738753e-09
2905 9.5293497537341e-09
2906 9.66849178496432e-09
2907 9.636900166754e-09
2908 9.60869339650117e-09
2909 9.47980804966164e-09
2910 9.48895539920613e-09
2911 9.28384480403111e-09
2912 9.54947232401082e-09
2913 9.55665235835568e-09
2914 9.63102664286453e-09
2915 9.30199295368084e-09
2916 9.8252153080125e-09
2917 9.44749700693137e-09
2918 9.39770750107982e-09
2919 9.5858592175091e-09
2920 9.58899892822274e-09
2921 9.06452601867613e-09
2922 9.49731937538445e-09
2923 9.71733893351256e-09
2924 9.44375777578443e-09
2925 9.42339806186965e-09
2926 9.55906820365726e-09
2927 9.51195300302743e-09
2928 1.06227933116543e-08
2929 9.73355351874261e-09
2930 9.40914279823346e-09
2931 9.59642765252511e-09
2932 9.18206044531189e-09
2933 9.41756805872274e-09
2934 9.34568777921641e-09
2935 9.47706979559371e-09
2936 9.26235355080962e-09
2937 9.62608659449415e-09
2938 9.49856282517203e-09
2939 9.75589387053333e-09
2940 9.31313781649123e-09
2941 9.53722434360316e-09
2942 9.87400916585557e-09
2943 9.52382173124988e-09
2944 9.44082056975049e-09
2945 9.47720391053508e-09
2946 9.53366363631858e-09
2947 9.53910994638818e-09
2948 9.65111102146921e-09
2949 9.75031166916551e-09
2950 9.89739668000311e-09
2951 9.49182421550177e-09
2952 9.49760092794349e-09
2953 9.49683531814571e-09
2954 9.21982667989596e-09
2955 9.51813206029328e-09
2956 9.30892340988976e-09
2957 9.34150534703804e-09
2958 9.31573662654728e-09
2959 9.66398516766276e-09
2960 9.34638677563271e-09
2961 9.43143252385426e-09
2962 9.66106750155404e-09
2963 9.58555634866798e-09
2964 9.61278967537282e-09
2965 9.31116339586424e-09
2966 9.19738951665749e-09
2967 9.2475014312754e-09
2968 9.57147339164521e-09
2969 9.95200544195995e-09
2970 9.51329948151169e-09
2971 9.77468950225102e-09
2972 9.33287314097697e-09
2973 9.86737802577409e-09
2974 9.3237133569346e-09
2975 9.86049908391351e-09
2976 9.31623311828389e-09
2977 9.42308364670907e-09
2978 9.61965618273553e-09
2979 9.67529611983764e-09
2980 9.33114385759382e-09
2981 9.07115893511445e-09
2982 9.85507675466124e-09
2983 9.56907975080412e-09
2984 9.62984358920949e-09
2985 9.52837453382926e-09
2986 9.37410771228997e-09
2987 9.64613278142679e-09
2988 9.33428268012904e-09
2989 9.57630685860522e-09
2990 9.64513802159672e-09
2991 9.36043242916185e-09
2992 9.65430135835277e-09
2993 9.87661064044687e-09
2994 9.26889409669229e-09
2995 9.43149913723573e-09
2996 9.3376533172318e-09
2997 9.28413168566067e-09
2998 9.39682998080116e-09
2999 9.30531296461368e-09
3000 9.5324290683152e-09
3001 9.90149295887477e-09
3002 8.87579609809563e-09
3003 9.74331104686144e-09
3004 9.61768886753589e-09
3005 9.260928024446e-09
3006 9.58393098215993e-09
3007 9.65224788984642e-09
3008 8.88075302185598e-09
3009 9.20104881174666e-09
3010 9.23797927043779e-09
3011 9.26010557122936e-09
3012 9.48295841851632e-09
3013 9.67911795157761e-09
3014 9.43394695696043e-09
3015 9.37204092110733e-09
3016 9.30376664598498e-09
3017 9.84415393645577e-09
3018 9.39250632825406e-09
3019 9.50298950641582e-09
3020 9.48932399325031e-09
3021 9.48972633807443e-09
3022 9.41437949819601e-09
3023 9.37517263821519e-09
3024 9.4392884619765e-09
3025 9.78702896503592e-09
3026 8.88273188337507e-09
3027 9.17838605118959e-09
3028 9.06000074962776e-09
3029 9.17983555837054e-09
3030 9.17341935746663e-09
3031 9.41992972514072e-09
3032 9.3762011488252e-09
3033 9.49410328132672e-09
3034 9.63286694855015e-09
3035 9.25190679623711e-09
3036 9.38698008212668e-09
3037 9.14449671540751e-09
3038 9.46189349093629e-09
3039 9.23173448796888e-09
3040 9.24791976331107e-09
3041 9.36676691765115e-09
3042 9.16636899717105e-09
3043 9.72609370819555e-09
3044 9.33882748910264e-09
3045 9.3518774946233e-09
3046 9.3800558431667e-09
3047 9.50537693000797e-09
3048 9.11571707007397e-09
3049 9.04724473116403e-09
3050 9.65980984091175e-09
3051 9.48710177084422e-09
3052 9.57140677826374e-09
3053 9.47844647214424e-09
3054 9.52176115731618e-09
3055 9.11624198352001e-09
3056 9.25067134005531e-09
3057 9.30259247411414e-09
3058 9.16230114000882e-09
3059 9.10750674876226e-09
3060 8.92684770548158e-09
3061 9.92644100250573e-09
3062 9.28757426521543e-09
3063 9.92604487493054e-09
3064 9.51764622669771e-09
3065 9.0731404611688e-09
3066 9.28390164744997e-09
3067 9.48659906185867e-09
3068 9.0900922344872e-09
3069 8.99247343255638e-09
3070 9.19609099980789e-09
3071 9.07325770072021e-09
3072 9.41539823884341e-09
3073 9.22723764062994e-09
3074 9.3220089425472e-09
3075 9.01995989011084e-09
3076 8.96339180656014e-09
3077 9.05066421807987e-09
3078 8.87444073782717e-09
3079 9.21135789866412e-09
3080 8.70053362689305e-09
3081 9.08353037232246e-09
3082 9.22151777160707e-09
3083 8.88997586656615e-09
3084 9.50145206957131e-09
3085 9.09016950600972e-09
3086 9.55040757588677e-09
3087 9.27854770793601e-09
3088 9.17466458361105e-09
3089 9.62111723623593e-09
3090 9.04530317313856e-09
3091 8.96850504972235e-09
3092 9.61832213874914e-09
3093 8.83710971066876e-09
3094 9.12649156248335e-09
3095 9.19398690513162e-09
3096 8.94518148442103e-09
3097 9.50005141220345e-09
3098 9.17196896210726e-09
3099 9.33110833045703e-09
3100 9.33385813084442e-09
3101 9.32031962719293e-09
3102 9.13485731501851e-09
3103 9.14900599724433e-09
3104 9.3628740316376e-09
3105 9.34974409005918e-09
3106 8.77834605006456e-09
3107 9.36938793216768e-09
3108 9.60748636202879e-09
3109 9.58673762596618e-09
3110 9.35856991901574e-09
3111 9.42878308762829e-09
3112 9.21299481149163e-09
3113 9.53937728809251e-09
3114 9.20149823002703e-09
3115 9.44098665911497e-09
3116 9.27990750909657e-09
3117 9.1272509550322e-09
3118 9.220594954229e-09
3119 8.98663987669579e-09
3120 9.12611319847656e-09
3121 9.28701471281101e-09
3122 8.74042083154336e-09
3123 8.98451180120219e-09
3124 9.32376575946137e-09
3125 9.08088093609649e-09
3126 9.33135080316561e-09
3127 9.27566290442883e-09
3128 8.94331098066914e-09
3129 9.34165189647729e-09
3130 9.48004696965654e-09
3131 8.91458373786236e-09
3132 9.20863296727248e-09
3133 9.21750054061476e-09
3134 8.99862229175596e-09
3135 9.2631999848436e-09
3136 8.74367511727314e-09
3137 9.20593201669817e-09
3138 9.54238998929213e-09
3139 9.39010647016403e-09
3140 9.17642939413099e-09
3141 9.86629444810205e-09
3142 9.31036048257283e-09
3143 9.70851221637758e-09
3144 9.64228252797739e-09
3145 8.85873419065319e-09
3146 8.78307115925736e-09
3147 9.26932663958269e-09
3148 9.00860896990707e-09
3149 8.79715678081538e-09
3150 8.88111539865122e-09
3151 8.59780424633527e-09
3152 8.95752716445486e-09
3153 9.59020685087353e-09
3154 8.9960403570899e-09
3155 8.86650486364715e-09
3156 8.96014462625772e-09
3157 8.99345042881805e-09
3158 9.19713638580788e-09
3159 8.74891981084147e-09
3160 8.53961701352546e-09
3161 9.36386257421873e-09
3162 9.25214838076727e-09
3163 8.80968986649577e-09
3164 8.89444784490934e-09
3165 9.10196984449385e-09
3166 8.72182326361326e-09
3167 8.88540352406153e-09
3168 9.0822247500455e-09
3169 8.88319373615332e-09
3170 8.6745535199384e-09
3171 9.34629706961232e-09
3172 9.24884702158124e-09
3173 9.15634590370473e-09
3174 9.08150088463344e-09
3175 9.17437148473255e-09
3176 8.74815331286527e-09
3177 9.17967923896867e-09
3178 9.27081433843568e-09
3179 8.93324347828184e-09
3180 8.84774209453099e-09
3181 8.47802184011925e-09
3182 8.82121842238348e-09
3183 9.27822796370492e-09
3184 8.82888251396707e-09
3185 8.71979732863792e-09
3186 8.98335894561342e-09
3187 8.99304897217235e-09
3188 8.68164917733338e-09
3189 9.02362540244894e-09
3190 8.67985683328243e-09
3191 8.48655012930521e-09
3192 8.84602080475361e-09
3193 8.76383676740033e-09
3194 8.66699423340833e-09
3195 8.80635653288664e-09
3196 9.19572862301266e-09
3197 9.26294507763714e-09
3198 9.640290343782e-09
3199 9.07350905521298e-09
3200 9.45472855562457e-09
3201 9.38647737314113e-09
3202 9.2248466643241e-09
3203 8.89068640930191e-09
3204 8.81101058780587e-09
3205 8.34314928255253e-09
3206 9.55624113174736e-09
3207 8.73702177273117e-09
3208 7.97427990306687e-09
3209 8.40307112781602e-09
3210 8.83356587877415e-09
3211 8.69722338592283e-09
3212 8.69235972089655e-09
3213 8.7950882132759e-09
3214 8.9586560392263e-09
3215 8.28511481643091e-09
3216 7.79971109921007e-09
3217 8.44523562193444e-09
3218 8.95740548401136e-09
3219 8.93828477899206e-09
3220 8.21633605596617e-09
3221 8.00714783366629e-09
3222 8.40795522094595e-09
3223 8.66331451021551e-09
3224 8.25674906224094e-09
3225 7.85486431453819e-09
3226 8.4840339198422e-09
3227 8.84237483234074e-09
3228 7.82610776184356e-09
3229 8.41521696770542e-09
3230 8.26951129795361e-09
3231 8.36306845997115e-09
3232 8.41617264768502e-09
3233 8.83248763017264e-09
3234 9.10516817498319e-09
3235 8.89831230921345e-09
3236 8.63235083414793e-09
3237 8.57712478818939e-09
3238 8.67579164065546e-09
3239 8.41830694042756e-09
3240 7.7778823381891e-09
3241 8.89698092976232e-09
3242 8.28356494508853e-09
3243 8.24976353897e-09
3244 8.47153369676334e-09
3245 9.11730335673155e-09
3246 7.85115261692226e-09
3247 8.87845086339212e-09
3248 8.81481643233428e-09
3249 8.75388472820759e-09
3250 8.62478710672576e-09
3251 8.7451157426699e-09
3252 7.70292185592325e-09
3253 8.22999801641799e-09
3254 8.24967116841435e-09
3255 9.2381373661965e-09
3256 7.9506747852065e-09
3257 9.0338261315992e-09
3258 8.65619309564636e-09
3259 8.80094219724015e-09
3260 8.35849167657443e-09
3261 9.20023524031421e-09
3262 9.06359165497861e-09
3263 8.33790103627052e-09
3264 8.78828743111626e-09
3265 7.91952281531394e-09
3266 7.82691689238391e-09
3267 8.15089329364582e-09
3268 8.24187296188938e-09
3269 8.14961254036461e-09
3270 7.8358119992572e-09
3271 9.01938612685171e-09
3272 7.98038080063179e-09
3273 8.59507753858679e-09
3274 8.34956193074277e-09
3275 8.35057445414122e-09
3276 8.42258351951841e-09
3277 8.60605098296219e-09
3278 8.57828741374078e-09
3279 8.09245292998639e-09
3280 8.72985239652735e-09
3281 8.13943490385327e-09
3282 8.01499222546909e-09
3283 8.19295831178124e-09
3284 8.36058777764492e-09
3285 7.66958407893981e-09
3286 8.50735570878669e-09
3287 8.56202220234081e-09
3288 8.45689029915775e-09
3289 8.24280910194375e-09
3290 7.73211450422195e-09
3291 7.69007613143913e-09
3292 8.4033180414167e-09
3293 8.30145374663971e-09
3294 8.15926348707308e-09
3295 7.88282950026087e-09
3296 7.78322650774044e-09
3297 8.40317504469112e-09
3298 8.32369728698268e-09
3299 8.02879629446807e-09
3300 8.27953883231203e-09
3301 7.89120591093706e-09
3302 7.74602337827446e-09
3303 8.42475955664668e-09
3304 8.22810797274087e-09
3305 7.70577912589943e-09
3306 8.95152663105137e-09
3307 8.03220334688604e-09
3308 8.69364935596195e-09
3309 8.1591835510153e-09
3310 8.77944206223447e-09
3311 7.70042163367179e-09
3312 8.36694091788104e-09
3313 7.97439536626143e-09
3314 8.76593286847083e-09
3315 8.61749782643528e-09
3316 8.6693416889716e-09
3317 8.06979105760774e-09
3318 8.7426856865136e-09
3319 8.25381896163435e-09
3320 8.61273363739201e-09
3321 8.81431017063505e-09
3322 8.86926887488926e-09
3323 9.12002651176635e-09
3324 8.28138446706816e-09
3325 8.43099989822349e-09
3326 9.25870313750465e-09
3327 7.79325048938517e-09
3328 8.88977336188645e-09
3329 8.38662383984001e-09
3330 8.36992075647913e-09
3331 8.2551423474797e-09
3332 9.16811249140892e-09
3333 8.81704043109721e-09
3334 7.73610953075377e-09
3335 9.04488928199498e-09
3336 8.29957080838994e-09
3337 8.23268830885127e-09
3338 8.64029203739847e-09
3339 8.68598615255678e-09
3340 7.70672325955957e-09
3341 8.75745609363321e-09
3342 8.147628349775e-09
3343 7.75490960336356e-09
3344 8.48665138164506e-09
3345 8.36916314028713e-09
3346 8.79491057759196e-09
3347 7.86802800689657e-09
3348 7.97467691882048e-09
3349 8.32671265271756e-09
3350 8.48189696256441e-09
3351 7.92446375186273e-09
3352 8.07138622604953e-09
3353 8.97302410152179e-09
3354 7.6827078032693e-09
3355 7.97502774929626e-09
3356 7.98489008246861e-09
3357 7.62465379722244e-09
3358 7.84024578592835e-09
3359 8.30383939387502e-09
3360 8.01640975822693e-09
3361 7.97066590507711e-09
3362 7.91691157076002e-09
3363 8.18005574387826e-09
3364 8.24406765076446e-09
3365 7.9621580439948e-09
3366 9.08881414574125e-09
3367 8.1941466945068e-09
3368 8.78091732658959e-09
3369 9.29173538111172e-09
3370 8.59113846729542e-09
3371 8.75945893596963e-09
3372 8.14733880361018e-09
3373 7.94509347201711e-09
3374 8.13972267366125e-09
3375 8.17508549744161e-09
3376 8.31421598235238e-09
3377 8.0696178628159e-09
3378 8.22727219684793e-09
3379 7.93611310001552e-09
3380 7.89560594682825e-09
3381 8.79905215356303e-09
3382 8.81620820791795e-09
3383 8.5144300498996e-09
3384 7.97200083724192e-09
3385 8.81172645961215e-09
3386 8.50180459366356e-09
3387 8.4219111684547e-09
3388 7.91972354363679e-09
3389 7.66548247099763e-09
3390 8.6788682907013e-09
3391 7.79934428152274e-09
3392 7.97465471435999e-09
3393 7.86896059423725e-09
3394 7.72939579007925e-09
3395 7.70337216238204e-09
3396 7.63128671366076e-09
3397 8.55857873460764e-09
3398 8.65278693140681e-09
3399 8.17032486111202e-09
3400 8.35956370792701e-09
3401 8.62156035452699e-09
3402 8.68137650655854e-09
3403 8.51332515594549e-09
3404 7.92001753069371e-09
3405 8.53204262796226e-09
3406 8.33054869531225e-09
3407 8.50609627178756e-09
3408 7.80809639167046e-09
3409 8.15185074998226e-09
3410 8.90119800089906e-09
3411 8.53810533385513e-09
3412 8.27179569284908e-09
3413 7.92806709171145e-09
3414 8.32332780476008e-09
3415 8.33092794749746e-09
3416 8.42484393359655e-09
3417 7.97538657337782e-09
3418 8.53769321906839e-09
3419 8.49377990164157e-09
3420 8.83099904314122e-09
3421 7.89731480210776e-09
3422 8.16857870233889e-09
3423 7.86745779635112e-09
3424 8.32141022755195e-09
3425 8.29137380975453e-09
3426 7.71026620327575e-09
3427 8.62492210984556e-09
3428 8.18623568932253e-09
3429 7.62943930254778e-09
3430 8.59166071620621e-09
3431 8.45415915051717e-09
3432 8.36869418208153e-09
3433 7.99953259189579e-09
3434 7.720367456443e-09
3435 8.79972450462674e-09
3436 7.69425589908224e-09
3437 8.51163672876964e-09
3438 8.26363333317204e-09
3439 7.94348675725587e-09
3440 8.42479597196188e-09
3441 8.03492827827768e-09
3442 8.56136495031024e-09
3443 8.50205150726424e-09
3444 8.3553972629602e-09
3445 8.6465243853695e-09
3446 8.5839824137679e-09
3447 7.78228592679397e-09
3448 7.79171571707593e-09
3449 8.8145331034184e-09
3450 8.44491054863283e-09
3451 8.01264032901372e-09
3452 8.69689120719386e-09
3453 7.86417508891191e-09
3454 8.24893930939652e-09
3455 7.63666996306256e-09
3456 7.52447526508604e-09
3457 8.59007975861914e-09
3458 7.53634044059481e-09
3459 8.05269095849326e-09
3460 7.65378338485334e-09
3461 8.37759550620376e-09
3462 8.4675280120905e-09
3463 8.49611492270697e-09
3464 8.01930788441041e-09
3465 8.35861246883951e-09
3466 8.39402947150347e-09
3467 8.50325942991503e-09
3468 7.92186760634195e-09
3469 8.86892870255451e-09
3470 7.54681828141202e-09
3471 8.69858407526181e-09
3472 8.28554558296446e-09
3473 8.81372841377015e-09
3474 8.28573298861102e-09
3475 8.22148571444359e-09
3476 8.14296807760684e-09
3477 7.53885220916573e-09
3478 8.37333047343236e-09
3479 8.29770563370857e-09
3480 8.06089595073445e-09
3481 7.59750395928904e-09
3482 8.47597814157552e-09
3483 8.24250889763789e-09
3484 8.48509440487533e-09
3485 8.34401703286858e-09
3486 8.12577027886618e-09
3487 8.75383499021609e-09
3488 8.21339973811064e-09
3489 8.41286951214215e-09
3490 8.06227706817708e-09
3491 8.43612113499148e-09
3492 8.21214385382518e-09
3493 8.42233749409615e-09
3494 8.49915782197286e-09
3495 9.08430930479653e-09
3496 7.80931141974861e-09
3497 8.86274964528866e-09
3498 8.41209324420333e-09
3499 9.50573308955427e-09
3500 7.76095987475856e-09
3501 8.41448333233075e-09
3502 7.72543451432739e-09
3503 8.32314661636246e-09
3504 8.02542921007898e-09
3505 8.67765059808789e-09
3506 8.01258615013012e-09
3507 8.3645721460357e-09
3508 8.57974846724119e-09
3509 8.03855204623005e-09
3510 8.36589642005947e-09
3511 7.6090662659567e-09
3512 8.25196977416454e-09
3513 8.23445933662015e-09
3514 8.16862222308146e-09
3515 7.53300710698568e-09
3516 7.8450135276853e-09
3517 8.23592838372633e-09
3518 7.697726012168e-09
3519 8.19559264897407e-09
3520 7.53052287194578e-09
3521 8.6804821108899e-09
3522 8.13212430728072e-09
3523 8.11628897423589e-09
3524 8.71342198394132e-09
3525 8.07167044314383e-09
3526 8.6440987701053e-09
3527 7.94334020781662e-09
3528 8.17556333743141e-09
3529 7.38228544960862e-09
3530 8.18459700013818e-09
3531 7.9961290921915e-09
3532 8.03873856369819e-09
3533 8.61442472910312e-09
3534 8.11411027257236e-09
3535 8.51865333828528e-09
3536 7.5771957597226e-09
3537 8.82747741570711e-09
3538 8.15164913348099e-09
3539 8.62707860704859e-09
3540 8.0856228379389e-09
3541 8.27644619505463e-09
3542 7.97356403126059e-09
3543 7.65232766042345e-09
3544 8.35995006553958e-09
3545 8.00375143938936e-09
3546 7.75724551260737e-09
3547 8.04088884365228e-09
3548 8.07452860129843e-09
3549 8.28385449125335e-09
3550 8.22090662211394e-09
3551 8.14673484228479e-09
3552 7.80277442657962e-09
3553 7.51535456089414e-09
3554 8.28653501372401e-09
3555 7.57571516629696e-09
3556 7.74241737389048e-09
3557 7.72710073704275e-09
3558 8.24982571145938e-09
3559 8.13557932133335e-09
3560 1.04954045454519e-08
3561 9.19945719601856e-09
3562 7.73431274581071e-09
3563 7.87059217799424e-09
3564 8.1875644042384e-09
3565 8.57963478040347e-09
3566 8.96192720034605e-09
3567 7.64369190164871e-09
3568 8.29053004025582e-09
3569 8.73743299933949e-09
3570 7.95119348140361e-09
3571 8.58168647255297e-09
3572 8.17955303489271e-09
3573 8.19049628120183e-09
3574 8.13537148758314e-09
3575 8.19372214522218e-09
3576 8.61698623566554e-09
3577 8.83710438159824e-09
3578 7.74284014681825e-09
3579 8.14606426757791e-09
3580 8.4492102203626e-09
3581 8.17247691742296e-09
3582 7.87231346777162e-09
3583 8.16135248271621e-09
3584 8.12496470103952e-09
3585 7.38036431968681e-09
3586 8.54633430691365e-09
3587 8.30181523525653e-09
3588 8.23748891320975e-09
3589 8.13707679014897e-09
3590 8.3730391509107e-09
3591 8.28214563597385e-09
3592 8.5936848748247e-09
3593 8.76950423389644e-09
3594 7.53698969901961e-09
3595 8.31058688532949e-09
3596 7.57880336266226e-09
3597 8.20753065511326e-09
3598 7.37465155609129e-09
3599 8.45349124034556e-09
3600 8.43560066243754e-09
3601 8.3031439501724e-09
3602 8.1978122068449e-09
3603 7.75091990590226e-09
3604 8.23635470936779e-09
3605 7.93737964244201e-09
3606 8.15711942436792e-09
3607 7.83041720353594e-09
3608 8.13498868268425e-09
3609 8.6737239612944e-09
3610 7.77627651160628e-09
3611 8.04396815823338e-09
3612 8.09767719545107e-09
3613 7.87039500238507e-09
3614 7.80899700458804e-09
3615 7.96758392596075e-09
3616 8.17535017461068e-09
3617 8.04802358089773e-09
3618 7.9275324083028e-09
3619 7.88110465776981e-09
3620 7.64409335829441e-09
3621 8.57457571612485e-09
3622 8.35223445960764e-09
3623 7.81273357119971e-09
3624 7.64298668798347e-09
3625 8.13577472058569e-09
3626 8.05819944105224e-09
3627 8.1624431658156e-09
3628 8.02722954773571e-09
3629 7.91729526383733e-09
3630 7.86060194712945e-09
3631 8.08497269133568e-09
3632 7.92456056331048e-09
3633 8.04450017710678e-09
3634 7.93976084878523e-09
3635 7.90033993780526e-09
3636 9.06239971953937e-09
3637 7.81762743429226e-09
3638 8.35914093499923e-09
3639 7.81972264718434e-09
3640 8.53786641386023e-09
3641 7.62851293245603e-09
3642 8.61795435014301e-09
3643 7.45382333633415e-09
3644 8.34738234090082e-09
3645 8.53767456732157e-09
3646 7.93293519762983e-09
3647 8.83296458198402e-09
3648 7.66034702337492e-09
3649 8.48128411945481e-09
3650 8.12410849704293e-09
3651 8.32242985637777e-09
3652 8.21199019895857e-09
3653 7.99793387074033e-09
3654 8.64776161790815e-09
3655 8.03776423197178e-09
3656 7.85098652755778e-09
3657 8.48566639177761e-09
3658 7.80631115304686e-09
3659 7.68100072434663e-09
3660 7.97601806823423e-09
3661 8.63016591523547e-09
3662 8.2638749177022e-09
3663 8.05138622439472e-09
3664 8.4511038167534e-09
3665 8.03358801704235e-09
3666 7.71234187624259e-09
3667 8.22919865584026e-09
3668 8.1719377931222e-09
3669 7.84737430592486e-09
3670 9.25388032868568e-09
3671 7.85520803958661e-09
3672 8.24026447077131e-09
3673 8.66084715056559e-09
3674 8.65967031415948e-09
3675 7.8903283906584e-09
3676 8.03289434969656e-09
3677 7.60232055085908e-09
3678 8.19024137399538e-09
3679 8.08574185384714e-09
3680 8.09331535123192e-09
3681 8.29914448274849e-09
3682 8.12256839566317e-09
3683 8.57892157313245e-09
3684 8.09200262352761e-09
3685 8.12310307907183e-09
3686 7.75709985134654e-09
3687 8.0597306606478e-09
3688 7.43408179459948e-09
3689 8.11362355079837e-09
3690 7.81425768536792e-09
3691 8.31999447115095e-09
3692 8.09189515393882e-09
3693 7.42183425828102e-09
3694 8.05037370099626e-09
3695 8.07861155749379e-09
3696 8.45677217142793e-09
3697 8.34796320958731e-09
3698 8.88885942629258e-09
3699 8.62549409674784e-09
3700 8.23924573012391e-09
3701 9.72786118325075e-09
3702 7.60347607098311e-09
3703 8.75750849615997e-09
3704 8.0150739378837e-09
3705 7.88833443010617e-09
3706 7.61564145079774e-09
3707 8.1434397003477e-09
3708 7.90033372055632e-09
3709 8.84851125704245e-09
3710 7.66259500295519e-09
3711 7.89471776840855e-09
3712 7.29743598881782e-09
3713 8.69588667740118e-09
3714 8.1151352304687e-09
3715 8.16539280634743e-09
3716 8.6378140196075e-09
3717 8.00886024165948e-09
3718 7.66734942203584e-09
3719 8.19608914071068e-09
3720 7.92464938115245e-09
3721 7.72563879536392e-09
3722 7.72003883042771e-09
3723 9.2307343990683e-09
3724 8.31397262146538e-09
3725 7.82674014487839e-09
3726 8.07079381104359e-09
3727 8.22752355134071e-09
3728 7.75670283559293e-09
3729 8.53503134834455e-09
3730 8.19799872431304e-09
3731 7.90119614180185e-09
3732 7.66849339584041e-09
3733 9.06234642883419e-09
3734 9.11792596980376e-09
3735 7.9907191974371e-09
3736 8.24664869725211e-09
3737 7.28320648235581e-09
3738 8.48865333580306e-09
3739 7.67128938150563e-09
3740 7.88629428427612e-09
3741 7.82033016122341e-09
3742 8.20927947842165e-09
3743 7.87561393877922e-09
3744 7.78200526241335e-09
3745 8.02830069090987e-09
3746 7.71463781745751e-09
3747 1.08707389756546e-08
3748 7.76756436948745e-09
3749 7.67484564789811e-09
3750 8.42195824191094e-09
3751 8.62726068362463e-09
3752 7.54822959692092e-09
3753 8.00745691975635e-09
3754 8.32507307535479e-09
3755 8.73477645768617e-09
3756 9.06680508450108e-09
3757 7.6889747901987e-09
3758 7.18572756852609e-09
3759 7.10946146398328e-09
3760 7.98513521971245e-09
3761 7.89014720226078e-09
3762 7.77619391101325e-09
3763 8.81221051685088e-09
3764 8.10648526083924e-09
3765 8.78800232584354e-09
3766 9.23451981549306e-09
3767 7.7905735196282e-09
3768 8.07163402782862e-09
3769 7.80803155464582e-09
3770 8.05021560523755e-09
3771 7.82992959358353e-09
3772 7.65945529224155e-09
3773 7.6271948756812e-09
3774 7.44044026390611e-09
3775 7.86617881942675e-09
3776 8.90279938658978e-09
3777 8.56348769673332e-09
3778 7.85114551149491e-09
3779 7.76330555396498e-09
3780 7.80531905775206e-09
3781 8.0072561914335e-09
3782 8.23012680228885e-09
3783 8.24390866682734e-09
3784 7.71574271141162e-09
3785 7.33635907579355e-09
3786 8.4247471221488e-09
3787 9.30304366875134e-09
3788 8.31591506766927e-09
3789 7.85473996955943e-09
3790 7.91715226711176e-09
3791 7.85455256391288e-09
3792 8.40013036906839e-09
3793 8.09465916518093e-09
3794 7.95420351806797e-09
3795 7.79298936492978e-09
3796 7.79491404756527e-09
3797 8.04228239559279e-09
3798 8.40271763280498e-09
3799 7.67549845903659e-09
3800 8.2001250234498e-09
3801 9.42807965031989e-09
3802 8.26482793314653e-09
3803 8.61983640021435e-09
3804 7.85280374060449e-09
3805 7.61584040276375e-09
3806 8.60422488813128e-09
3807 8.04856892244743e-09
3808 7.30872073972932e-09
3809 8.26778911999781e-09
3810 8.10974487563954e-09
3811 7.44789874218554e-09
3812 8.15742762227956e-09
3813 7.94398502534932e-09
3814 8.31078406093866e-09
3815 7.35606597856986e-09
3816 8.13443001845826e-09
3817 8.0903115318165e-09
3818 8.00067923023562e-09
3819 7.58340679141156e-09
3820 8.30313240385294e-09
3821 7.89949083923602e-09
3822 8.5146689698945e-09
3823 8.19599677015503e-09
3824 9.07138986150358e-09
3825 8.33002022915252e-09
3826 7.25527238287782e-09
3827 8.19268297647113e-09
3828 8.07248934364679e-09
3829 8.17313594581037e-09
3830 7.77942332774728e-09
3831 8.06708744249818e-09
3832 8.13056999504624e-09
3833 7.86174148004193e-09
3834 8.25164558904135e-09
3835 8.24875634464206e-09
3836 7.52532258729843e-09
3837 8.47998027353469e-09
3838 7.94793209024647e-09
3839 8.09005307189636e-09
3840 7.50908402125106e-09
3841 8.29705548710535e-09
3842 7.37688310437079e-09
3843 7.8259834168648e-09
3844 7.64669305652887e-09
3845 7.35287875031077e-09
3846 7.54405338199149e-09
3847 7.80923237186926e-09
3848 7.67829799741548e-09
3849 8.12588751841758e-09
3850 8.15553935495927e-09
3851 7.6789028469193e-09
3852 8.21631473968409e-09
3853 7.81398501459307e-09
3854 7.41550731930829e-09
3855 7.53824203059139e-09
3856 8.31808755208385e-09
3857 8.00630317598916e-09
3858 8.21601364719982e-09
3859 8.35105051777418e-09
3860 8.71132321833556e-09
3861 7.8202351261325e-09
3862 8.04900768258676e-09
3863 8.30566548870593e-09
3864 7.61826868256321e-09
3865 8.13306400004876e-09
3866 7.14687153902105e-09
3867 8.35247071506728e-09
3868 8.24818346956135e-09
3869 7.62694085665316e-09
3870 8.14806178084382e-09
3871 7.60606333471969e-09
3872 8.24222112782991e-09
3873 7.85287745941332e-09
3874 7.66466534685151e-09
3875 7.81908671143583e-09
3876 8.4738536187956e-09
3877 7.96950860859624e-09
3878 8.14654921299507e-09
3879 7.44383799045067e-09
3880 7.71802177723657e-09
3881 7.75498509852923e-09
3882 7.85825626792303e-09
3883 7.33371141592443e-09
3884 7.63214114130051e-09
3885 8.67318927788574e-09
3886 7.58082308038865e-09
3887 7.79583242405124e-09
3888 8.24195289794716e-09
3889 7.06058278510113e-09
3890 8.29699953186491e-09
3891 7.92037990748895e-09
3892 8.23973866914685e-09
3893 7.84455433944231e-09
3894 8.05832733874468e-09
3895 7.76078401543145e-09
3896 7.63063567887912e-09
3897 8.8534415354502e-09
3898 7.91442822389854e-09
3899 7.78367414966397e-09
3900 8.04816036037437e-09
3901 8.43047764931271e-09
3902 8.08976530208838e-09
3903 7.61277263450211e-09
3904 7.86020226684059e-09
3905 7.97330024226994e-09
3906 8.62743032570279e-09
3907 8.08747557812239e-09
3908 7.66085239689573e-09
3909 7.74532349367973e-09
3910 7.92851650999182e-09
3911 9.34233224114678e-09
3912 8.25745694044144e-09
3913 7.99566457487799e-09
3914 8.70582361756078e-09
3915 7.731981277459e-09
3916 8.07126543378445e-09
3917 7.86528708829337e-09
3918 8.33251512233346e-09
3919 8.090189851373e-09
3920 7.19951565031351e-09
3921 7.58148743784659e-09
3922 8.35536706489393e-09
3923 8.38784774970236e-09
3924 8.80578721051961e-09
3925 7.96506860467616e-09
3926 7.86309861666723e-09
3927 8.35323987757874e-09
3928 7.76637865129715e-09
3929 7.02612945602255e-09
3930 7.91069609817896e-09
3931 7.92230370194602e-09
3932 7.60425411527876e-09
3933 7.77348407865475e-09
3934 7.91314302972523e-09
3935 7.97376209504819e-09
3936 8.03976529795136e-09
3937 8.17905476679925e-09
3938 8.27123258773099e-09
3939 7.89019605207386e-09
3940 7.6530151105203e-09
3941 7.86119702667065e-09
3942 7.75611130876541e-09
3943 8.99281271671271e-09
3944 7.28516491577125e-09
3945 7.59700657937401e-09
3946 7.73923591879111e-09
3947 8.89698537065442e-09
3948 7.6293140693906e-09
3949 8.18081602460552e-09
3950 7.68792673966345e-09
3951 7.49499928787145e-09
3952 8.20220957820084e-09
3953 7.8928126256983e-09
3954 8.14988876385314e-09
3955 1.01218482484455e-08
3956 8.35835312074096e-09
3957 7.28864923971173e-09
3958 7.15091896807962e-09
3959 8.19958589914904e-09
3960 8.22236501107909e-09
3961 8.27047230700373e-09
3962 8.71869509921908e-09
3963 7.3657293597762e-09
3964 7.24895432568928e-09
3965 7.79782993731715e-09
3966 8.50468140356497e-09
3967 9.45304901023292e-09
3968 7.48099093783594e-09
3969 8.29701019000595e-09
3970 7.84719489388408e-09
3971 7.74953257121069e-09
3972 7.53299733702306e-09
3973 8.23605539324035e-09
3974 7.84276110721294e-09
3975 8.07089595156185e-09
3976 7.63551977200905e-09
3977 7.60513874098478e-09
3978 7.84595055591808e-09
3979 8.28326651713951e-09
3980 8.31606694617903e-09
3981 7.71185337811175e-09
3982 8.02644617436954e-09
3983 7.61142793237468e-09
3984 8.61627746928662e-09
3985 7.81434028596095e-09
3986 8.41324254707843e-09
3987 7.66549046460341e-09
3988 8.55096260465871e-09
3989 7.74120412216917e-09
3990 7.86827580867566e-09
3991 7.45718597983114e-09
3992 7.94687338157019e-09
3993 8.40267677659767e-09
3994 7.52640794132731e-09
3995 8.61733706614132e-09
3996 8.4522939758358e-09
3997 8.049956257139e-09
3998 8.107051918671e-09
3999 8.01238630998569e-09
4000 8.02989763570849e-09
4001 8.68058869230026e-09
4002 7.961237891152e-09
4003 7.73220065752867e-09
4004 7.51197060111508e-09
4005 9.36998656442256e-09
4006 7.7209536542e-09
4007 7.7325044145482e-09
4008 7.92622145695532e-09
4009 8.19512901983899e-09
4010 7.60257190535185e-09
4011 8.49016679183023e-09
4012 8.49228509736122e-09
4013 8.19185252964871e-09
4014 7.75259056950972e-09
4015 8.06692312949053e-09
4016 7.6690742645269e-09
4017 7.55897744397771e-09
4018 9.85359083216508e-09
4019 7.82387044040433e-09
4020 7.94990473451662e-09
4021 7.22956716714407e-09
4022 7.99531285622379e-09
4023 7.33758609428037e-09
4024 7.56603757423591e-09
4025 7.87571607929749e-09
4026 7.7679862542368e-09
4027 8.27057533570041e-09
4028 7.68016406027527e-09
4029 7.84526044128597e-09
4030 7.43347383647119e-09
4031 7.32382465784553e-09
4032 7.80336950612082e-09
4033 7.63071472675847e-09
4034 8.09847300331512e-09
4035 7.67521690647754e-09
4036 7.68313324073233e-09
4037 7.38610150818886e-09
4038 8.23905121905e-09
4039 9.11875996933986e-09
4040 8.36296631945288e-09
4041 7.1482038066506e-09
4042 7.1411885294026e-09
4043 8.12371681035984e-09
4044 7.95828647426333e-09
4045 7.19234982682337e-09
4046 7.74148833926347e-09
4047 7.99212074298339e-09
4048 7.59485896395518e-09
4049 7.45140393831889e-09
4050 8.02194133342482e-09
4051 8.39304359345761e-09
4052 7.12029146754389e-09
4053 7.37101535364104e-09
4054 9.25936660678417e-09
4055 7.25255677735959e-09
4056 8.62100346665784e-09
4057 7.98420796144228e-09
4058 7.34586214079513e-09
4059 7.77684761033015e-09
4060 7.24013293762482e-09
4061 7.32173344175635e-09
4062 8.03858757336684e-09
4063 7.16263404143547e-09
4064 7.14481984687154e-09
4065 7.36226990483146e-09
4066 7.35224370274068e-09
4067 7.41024441808236e-09
4068 7.24488602443785e-09
4069 7.31762561656524e-09
4070 7.34931893120461e-09
4071 8.33413960066309e-09
4072 8.58000426262606e-09
4073 7.2853114652105e-09
4074 7.93058596570972e-09
4075 7.32926652702304e-09
4076 7.58783880172587e-09
4077 7.26328464040193e-09
4078 8.31000868117826e-09
4079 7.27547044832022e-09
4080 7.91034082681108e-09
4081 7.26594917566104e-09
4082 7.18001391675216e-09
4083 7.85678899717368e-09
4084 7.25193016748449e-09
4085 8.19760970216521e-09
4086 7.39769756563646e-09
4087 7.60083551654134e-09
4088 7.19301951335183e-09
4089 7.91319010318148e-09
4090 8.13436518143362e-09
4091 9.14370179572188e-09
4092 7.51570183865624e-09
4093 7.88556064890145e-09
4094 7.07435621194463e-09
4095 7.8368893596803e-09
4096 7.06920966209168e-09
4097 7.07542646694037e-09
4098 7.04697766806817e-09
4099 7.56784679367684e-09
4100 7.07644565167698e-09
4101 7.1450703131859e-09
4102 7.48489181745526e-09
4103 8.54871018418635e-09
4104 7.09652603347877e-09
4105 7.92504017965712e-09
4106 7.49785034059869e-09
4107 7.39342187472403e-09
4108 7.79044651011418e-09
4109 8.05663624703357e-09
4110 7.92804932814306e-09
4111 7.6312884900176e-09
4112 7.49980078040835e-09
4113 7.2570838227648e-09
4114 8.00011434876069e-09
4115 7.35918881389352e-09
4116 8.37186764357511e-09
4117 8.03561395201768e-09
4118 8.2068245532696e-09
4119 8.0962179183075e-09
4120 7.77568054388666e-09
4121 7.8894180077782e-09
4122 7.86084797255171e-09
4123 9.20239706658776e-09
4124 7.14300441018167e-09
4125 7.60831397883521e-09
4126 7.82877762617318e-09
4127 8.02722244230836e-09
4128 7.17759585100453e-09
4129 7.08085723388763e-09
4130 7.31956451005544e-09
4131 7.26495708036623e-09
4132 8.40059666273874e-09
4133 7.18074577576999e-09
4134 7.43936290348302e-09
4135 7.31124183417364e-09
4136 8.4803817301804e-09
4137 8.7081284405599e-09
4138 8.04044120172875e-09
4139 8.15520184715979e-09
4140 7.86449838585668e-09
4141 7.19807058402466e-09
4142 7.07113390063796e-09
4143 8.2168289949891e-09
4144 8.1279667440981e-09
4145 7.46444328569851e-09
4146 7.45201500507164e-09
4147 8.25176638130642e-09
4148 7.32724680929664e-09
4149 7.55307727473564e-09
4150 7.20993442726581e-09
4151 7.79324960120675e-09
4152 8.21207102319477e-09
4153 8.77979555724551e-09
4154 8.11683431578558e-09
4155 8.01792854332462e-09
4156 7.56696216797081e-09
4157 7.06136216166442e-09
4158 8.26419199739803e-09
4159 7.31146654331383e-09
4160 7.84028753031407e-09
4161 8.63151772279025e-09
4162 8.32012947427074e-09
4163 7.10158376548975e-09
4164 7.74208785969677e-09
4165 7.46006900698148e-09
4166 7.47091988273496e-09
4167 8.2665350120692e-09
4168 8.30884250291319e-09
4169 8.15945977450383e-09
4170 8.06621347493319e-09
4171 8.81712303169024e-09
4172 8.61788418404785e-09
4173 7.6610717769654e-09
4174 7.70474706257573e-09
4175 8.20731393957885e-09
4176 8.66581117975329e-09
4177 7.39502992175289e-09
4178 8.69064731290337e-09
4179 9.13721720507965e-09
4180 8.64883986650966e-09
4181 9.1197671636678e-09
4182 8.39629965554423e-09
4183 8.44449932202451e-09
4184 9.13572595351297e-09
4185 8.77761952011724e-09
4186 8.30061619438993e-09
4187 8.37877145443144e-09
4188 8.76146710737657e-09
4189 9.22685838844473e-09
4190 8.58221671506953e-09
4191 8.84786288679607e-09
4192 9.01914276596472e-09
4193 9.6184828990431e-09
4194 9.09989417152701e-09
4195 8.99797836240168e-09
4196 8.39192804136246e-09
4197 8.52870751799628e-09
4198 9.58711954268665e-09
4199 8.45847747399375e-09
4200 8.75600303373858e-09
4201 7.70676766848055e-09
4202 8.51441583904489e-09
4203 9.60873691724373e-09
4204 9.21254450503284e-09
4205 8.72409700036769e-09
4206 9.08803698962402e-09
4207 9.57027879167072e-09
4208 1.02354800191051e-08
4209 9.54493462046457e-09
4210 8.69646310519556e-09
4211 8.49822789916743e-09
4212 7.86501175298326e-09
4213 7.49146078504737e-09
4214 7.59608465017436e-09
4215 7.35510008453844e-09
4216 7.65861329909967e-09
4217 7.41658379155297e-09
4218 8.38415203929799e-09
4219 7.01666058589012e-09
4220 7.32025995375807e-09
4221 7.15181336374826e-09
4222 7.01458047203118e-09
4223 7.29631555174137e-09
4224 7.77599584722566e-09
4225 7.41206251930748e-09
4226 7.00811186860051e-09
4227 7.15188752664631e-09
4228 7.58655716026624e-09
4229 7.21398141223517e-09
4230 7.62658824982054e-09
4231 7.09456271508202e-09
4232 7.22144832820959e-09
4233 7.32762339694659e-09
4234 7.11835701494579e-09
4235 6.83062451045657e-09
4236 7.28091320567614e-09
4237 7.29865501369886e-09
4238 7.02649360917462e-09
4239 7.14667214296583e-09
4240 7.60860885407055e-09
4241 7.53413864629238e-09
4242 8.03297073304066e-09
4243 7.650413635929e-09
4244 6.93556723163624e-09
4245 6.80346445847135e-09
4246 6.94794355382555e-09
4247 7.209636443406e-09
4248 6.90660240110219e-09
4249 7.29252391806767e-09
4250 6.70622402054732e-09
4251 6.55172538444049e-09
4252 7.44552330900206e-09
4253 7.65254970502838e-09
4254 7.46840456145037e-09
4255 7.00856306323772e-09
4256 7.18044956826702e-09
4257 6.5936096582675e-09
4258 7.39042915753885e-09
4259 6.97044821862391e-09
4260 6.53030518549258e-09
4261 6.62492460890007e-09
4262 7.31667837428063e-09
4263 6.76032163582363e-09
4264 6.72554856251395e-09
4265 6.44961151152756e-09
4266 6.86184842280113e-09
4267 6.57953558302893e-09
4268 6.59381527157166e-09
4269 6.65528920862357e-09
4270 7.90456411436935e-09
4271 7.54436690897364e-09
4272 7.02166547128513e-09
4273 6.57831122907737e-09
4274 6.85006051881487e-09
4275 7.29508631280851e-09
4276 7.19427006856677e-09
4277 6.51257625605695e-09
4278 7.41773087398201e-09
4279 6.51802256612655e-09
4280 6.825675580302e-09
4281 7.25482163232982e-09
4282 6.45774500540597e-09
4283 6.61393517731312e-09
4284 6.56932153120238e-09
4285 6.52392806443913e-09
4286 6.77726808007151e-09
4287 7.27605087291749e-09
4288 6.41214548124935e-09
4289 6.45667608267786e-09
4290 6.77344624833154e-09
4291 7.64043939227577e-09
4292 7.35778682425803e-09
4293 6.43268949218623e-09
4294 6.5078538113994e-09
4295 7.2523409500036e-09
4296 6.50406306590412e-09
4297 7.19263004711479e-09
4298 7.06088476576383e-09
4299 6.54519372034201e-09
4300 6.72773792231851e-09
4301 6.61240839860966e-09
4302 6.88693368999793e-09
4303 7.18098691621094e-09
4304 6.82775791460699e-09
4305 6.59842402939148e-09
4306 6.67458754932682e-09
4307 6.56214771410646e-09
4308 7.18788140119386e-09
4309 7.12887038289978e-09
4310 6.5041074748251e-09
4311 6.93670276774583e-09
4312 6.71167343924139e-09
4313 6.99837432449613e-09
4314 7.05594560557188e-09
4315 6.73711442189528e-09
4316 6.90020440785588e-09
4317 6.47933484643204e-09
4318 6.62828947284311e-09
4319 6.4414753531139e-09
4320 7.13037584532117e-09
4321 7.05249236787608e-09
4322 6.69424515820083e-09
4323 6.38662012164559e-09
4324 6.63816202006728e-09
4325 6.78553035982077e-09
4326 6.58052901059136e-09
4327 6.74588784832508e-09
4328 7.017709968693e-09
4329 7.15587233912629e-09
4330 6.97737112531627e-09
4331 6.45835562806951e-09
4332 6.74826372559778e-09
4333 6.61690879866228e-09
4334 6.57400889281234e-09
4335 6.62240484672338e-09
4336 6.61457510986452e-09
4337 6.70922384315986e-09
4338 6.67856747682549e-09
4339 7.3149548640572e-09
4340 7.75396813423868e-09
4341 6.59077370457339e-09
4342 6.59949472847643e-09
4343 6.8326588831269e-09
4344 6.61894938858154e-09
4345 6.57569598772056e-09
4346 6.56751231176145e-09
4347 7.0596839485404e-09
4348 6.59404797431762e-09
4349 6.76892453199684e-09
4350 6.73802391659706e-09
4351 6.44134479088621e-09
4352 6.59110011014263e-09
4353 7.24724058542847e-09
4354 6.50736176055489e-09
4355 6.82276679597749e-09
4356 6.85619383489211e-09
4357 6.58679999432366e-09
4358 7.18169790303591e-09
4359 6.40594954859353e-09
4360 6.4627778684212e-09
4361 7.00920166352148e-09
4362 6.70973099303751e-09
4363 6.54579679348899e-09
4364 6.58614096593624e-09
4365 7.45452233275046e-09
4366 6.72328948070344e-09
4367 6.99141011750726e-09
4368 6.46712150498274e-09
4369 6.5585519237743e-09
4370 6.99229074641039e-09
4371 6.41428599124083e-09
4372 6.65380950337635e-09
4373 6.58905463524206e-09
4374 6.58744525594557e-09
4375 6.42721253996115e-09
4376 7.04038249921268e-09
4377 7.09801106779651e-09
4378 6.48248654755434e-09
4379 6.43578035308678e-09
4380 7.11586478630011e-09
4381 7.81902542712487e-09
4382 6.90223433963411e-09
4383 6.89825530031385e-09
4384 6.44361586310538e-09
4385 6.63224009045393e-09
4386 6.52553477920037e-09
4387 6.56061605042169e-09
4388 6.58822063570597e-09
4389 6.51419052033475e-09
4390 6.96934643329428e-09
4391 6.61542465252296e-09
4392 7.16352177576596e-09
4393 6.77210598709621e-09
4394 6.53745724221722e-09
4395 6.46158948569564e-09
4396 6.8298504629638e-09
4397 6.55555920658912e-09
4398 6.51445697386066e-09
4399 6.5777081559304e-09
4400 6.89892765137756e-09
4401 6.48657749735548e-09
4402 6.51367093595923e-09
4403 6.49512221784221e-09
4404 6.49368203653466e-09
4405 6.63759047725421e-09
4406 6.90379131640384e-09
4407 6.56124532483204e-09
4408 6.69489841342852e-09
4409 6.53630172209319e-09
4410 6.61131238643975e-09
4411 6.55844933916683e-09
4412 6.67828015110672e-09
4413 6.49996190205115e-09
4414 7.12112679934762e-09
4415 6.46510267543476e-09
4416 7.10467684683636e-09
4417 6.61266463808374e-09
4418 6.50921583300601e-09
4419 7.04036295928745e-09
4420 6.58828414046297e-09
4421 6.41407327250931e-09
4422 6.71423272535776e-09
4423 6.6802181564185e-09
4424 6.42355635349645e-09
4425 6.80469325331501e-09
4426 6.58039667200683e-09
4427 6.48455067420173e-09
4428 7.30957250283382e-09
4429 6.85290402202554e-09
4430 6.68349064980589e-09
4431 7.16994552618644e-09
4432 6.90563961569524e-09
4433 7.21420434501852e-09
4434 6.57361587386163e-09
4435 7.58621609975307e-09
4436 7.26017423957614e-09
4437 6.59429710836434e-09
4438 6.60476473512972e-09
4439 6.45770947826918e-09
4440 6.80399647734475e-09
4441 6.60718546541261e-09
4442 6.8242687056852e-09
4443 6.86850709641362e-09
4444 6.5953229544391e-09
4445 6.91751278480979e-09
4446 6.53124132554694e-09
4447 7.50352935341425e-09
4448 6.62842580823053e-09
4449 6.54716147963086e-09
4450 6.61738353002761e-09
4451 6.43825392998565e-09
4452 6.5213701105904e-09
4453 6.57681153981571e-09
4454 6.41024255898515e-09
4455 6.74606814854428e-09
4456 6.99327618036705e-09
4457 6.44497033519542e-09
4458 6.86169254748847e-09
4459 7.50451700781696e-09
4460 6.6377570107079e-09
4461 6.6670664544688e-09
4462 6.74690880941853e-09
4463 6.81295908577795e-09
4464 6.61284493830294e-09
4465 6.8619776527612e-09
4466 7.21590298624619e-09
4467 7.17767401070546e-09
4468 7.25063609152699e-09
4469 6.8563781319142e-09
4470 7.0280301578407e-09
4471 6.64587140875028e-09
4472 6.53969189912118e-09
4473 6.66319577291574e-09
4474 6.38456176815794e-09
4475 7.18038339897475e-09
4476 6.48884768139624e-09
4477 6.69080435500291e-09
4478 6.53913101444914e-09
4479 6.89778278939457e-09
4480 7.64488738980162e-09
4481 6.80909018058173e-09
4482 7.06721570153945e-09
4483 6.74401645639477e-09
4484 6.74100641973041e-09
4485 7.03912794719486e-09
4486 6.47004494425119e-09
4487 6.52098508524546e-09
4488 6.6184333569197e-09
4489 6.49648557171645e-09
4490 6.68493260747027e-09
4491 6.8652825646609e-09
4492 6.89169521450594e-09
4493 6.70601218999423e-09
4494 6.74602018690962e-09
4495 6.65432198232452e-09
4496 6.82988554601138e-09
4497 6.56988508040968e-09
4498 6.84473855372403e-09
4499 7.90722776145003e-09
4500 6.6899787931618e-09
4501 6.42980824139272e-09
4502 7.32370164513441e-09
4503 6.42389119676068e-09
4504 6.61150956204892e-09
4505 6.97650293091101e-09
4506 6.91888368820059e-09
4507 6.84339740431028e-09
4508 6.43328235128138e-09
4509 6.56212062466466e-09
4510 6.5141114724554e-09
4511 7.08128089499382e-09
4512 6.79218414845195e-09
4513 6.53089937685536e-09
4514 7.31260829667235e-09
4515 6.33942187633352e-09
4516 6.41689767988396e-09
4517 6.44087938539428e-09
4518 6.7780971946263e-09
4519 6.71356747972141e-09
4520 6.40301900389773e-09
4521 7.04879044022277e-09
4522 7.17307679920509e-09
4523 6.65035937430503e-09
4524 7.42310524159961e-09
4525 6.51372289439678e-09
4526 6.69254074381342e-09
4527 6.56854215463909e-09
4528 6.53690968022147e-09
4529 6.36576080736972e-09
4530 7.59080354129082e-09
4531 7.14066628049181e-09
4532 6.46323083941525e-09
4533 6.54904663832667e-09
4534 6.30573682158797e-09
4535 6.96371360575654e-09
4536 7.17194659216602e-09
4537 6.53335474609662e-09
4538 7.17094073010571e-09
4539 6.7370518053167e-09
4540 7.11312475587533e-09
4541 6.81471279406765e-09
4542 6.70908528732639e-09
4543 7.70578356679152e-09
4544 6.5658283254777e-09
4545 6.55131149329691e-09
4546 6.45262998588692e-09
4547 6.962444842884e-09
4548 7.11858127999676e-09
4549 6.9087353615771e-09
4550 7.287813907908e-09
4551 6.93841384347138e-09
4552 7.26353377444866e-09
4553 6.65509425346045e-09
4554 7.26289606234332e-09
4555 6.84980427934079e-09
4556 6.59738264019438e-09
4557 7.05171832038332e-09
4558 6.75571154573618e-09
4559 6.68264599212876e-09
4560 7.04268776630101e-09
4561 6.68754296384577e-09
4562 7.48684225726493e-09
4563 6.54308962566574e-09
4564 6.36523189712079e-09
4565 6.52920117971689e-09
4566 7.53973328215807e-09
4567 6.76594069659586e-09
4568 6.61662680201403e-09
4569 7.22328508118153e-09
4570 6.45382991493193e-09
4571 6.62638566240048e-09
4572 7.60710605618442e-09
4573 7.46399742013182e-09
4574 6.60166987742627e-09
4575 6.75717970466394e-09
4576 6.75720146503522e-09
4577 6.93986601518759e-09
4578 7.13553083286911e-09
4579 7.23796400592391e-09
4580 7.36727656658331e-09
4581 6.60435883759192e-09
4582 6.74878508633014e-09
4583 6.95621604762664e-09
4584 7.01013691539742e-09
4585 6.83745104979039e-09
4586 7.65199281715923e-09
4587 6.65296973068052e-09
4588 6.81421408188498e-09
4589 7.21787740687319e-09
4590 6.76105438301988e-09
4591 7.26771309800256e-09
4592 6.77946943028473e-09
4593 7.10809988646588e-09
4594 7.40883443484108e-09
4595 6.80375222827934e-09
4596 6.82950940245064e-09
4597 7.45010808600455e-09
4598 6.84078482748873e-09
4599 7.31712379575811e-09
4600 6.80965195343219e-09
4601 6.77520572978096e-09
4602 6.91480206427286e-09
4603 7.20050108427017e-09
4604 7.24380377903344e-09
4605 8.64907701014772e-09
4606 7.45230810395014e-09
4607 8.14033374041401e-09
4608 7.94253995906047e-09
4609 7.6752959543569e-09
4610 7.92660870274631e-09
4611 7.65521601664432e-09
4612 8.52084802716035e-09
4613 8.71693472959123e-09
4614 7.57054241518063e-09
4615 8.25179746755111e-09
4616 8.21289525276825e-09
4617 7.69981589598956e-09
4618 8.37844549295141e-09
4619 7.55061257962097e-09
4620 7.92175924857474e-09
4621 8.92919604922326e-09
4622 7.84681919441255e-09
4623 8.13582445857719e-09
4624 8.85788864479764e-09
4625 8.07800226709787e-09
4626 7.58305951364946e-09
4627 8.88133033782879e-09
4628 7.98042432137436e-09
4629 8.00946597934171e-09
4630 7.35834415621639e-09
4631 9.86378356770956e-09
4632 8.27117929702581e-09
4633 8.18753687070739e-09
4634 8.87949092032159e-09
4635 7.57524087902084e-09
4636 7.89481191532104e-09
4637 7.75660247143151e-09
4638 7.46563078024565e-09
4639 8.49006820402565e-09
4640 8.52870751799628e-09
4641 7.791737033358e-09
4642 8.16612999443578e-09
4643 8.24643375807455e-09
4644 7.79377540283122e-09
4645 8.08750399983182e-09
4646 8.4033002778483e-09
4647 7.72727126729933e-09
4648 8.01271582417939e-09
4649 7.54672413449953e-09
4650 7.40070360549794e-09
4651 7.82933806675601e-09
4652 8.34752267309113e-09
4653 7.97525601115012e-09
4654 7.37424654673191e-09
4655 8.50837000854199e-09
4656 7.79631470493314e-09
4657 7.89633602948925e-09
4658 8.23529866522676e-09
4659 7.3635160191543e-09
4660 8.17627121563191e-09
4661 7.45250616773774e-09
4662 7.33742400171877e-09
4663 8.86279050149597e-09
4664 8.50448689249106e-09
4665 8.36554114869159e-09
4666 7.71369101926211e-09
4667 8.09052913552932e-09
4668 7.50417328276853e-09
4669 8.34216784539876e-09
4670 6.9523866663701e-09
4671 7.32822647009357e-09
4672 8.04389710395981e-09
4673 7.62144392041364e-09
4674 7.91043852643725e-09
4675 7.32129290526018e-09
4676 7.59275664563575e-09
4677 8.59806981168276e-09
4678 7.92497889534616e-09
4679 8.2451601102207e-09
4680 8.27904234057542e-09
4681 7.96347787712648e-09
4682 7.48554196405848e-09
4683 8.21525514282939e-09
4684 8.15804845899493e-09
4685 8.00186938931802e-09
4686 8.20785928112855e-09
4687 8.29528357115805e-09
4688 7.94365639933403e-09
4689 7.81361819690574e-09
4690 8.35598168436036e-09
4691 7.49947748346358e-09
4692 8.50647374761593e-09
4693 9.10632014239354e-09
4694 6.92958179726588e-09
4695 8.72093330883672e-09
4696 7.90967646935314e-09
4697 8.32260127481277e-09
4698 8.05932121039632e-09
4699 7.45080264152875e-09
4700 9.04825281367039e-09
4701 8.8209759496749e-09
4702 7.81875009181476e-09
4703 7.79803333017526e-09
4704 7.84046694235485e-09
4705 8.14567346907324e-09
4706 8.03880428890125e-09
4707 8.75620376206143e-09
4708 8.03798538839828e-09
4709 8.13196532334359e-09
4710 8.41378788862812e-09
4711 8.5771612035046e-09
4712 8.02850763648166e-09
4713 8.81810002795191e-09
4714 7.57620544078463e-09
4715 8.25770829493422e-09
4716 7.77430742004981e-09
4717 7.76003261648839e-09
4718 7.86317144729765e-09
4719 7.90036125408733e-09
4720 8.36852098728968e-09
4721 8.40024494408453e-09
4722 8.15555889488451e-09
4723 8.10733702394373e-09
4724 7.42355821259366e-09
4725 7.7261654851668e-09
4726 7.81831843710279e-09
4727 8.34447710928998e-09
4728 7.92991805553811e-09
4729 8.6112725838916e-09
4730 8.2774995746604e-09
4731 7.89110909948931e-09
4732 8.2055757744115e-09
4733 7.94969867712325e-09
4734 7.68380203908237e-09
4735 7.52862483466288e-09
4736 7.8746165144139e-09
4737 7.50319006925793e-09
4738 7.73201680459579e-09
4739 8.87035067620445e-09
4740 7.84222553562586e-09
4741 8.71733973895061e-09
4742 7.94114285440628e-09
4743 8.14802625370703e-09
4744 7.54232232225149e-09
4745 7.92526666515414e-09
4746 8.12468492483731e-09
4747 7.73251418451082e-09
4748 8.82115003264516e-09
4749 8.39551805853489e-09
4750 8.44585468229297e-09
4751 8.16210654619454e-09
4752 8.54966586416595e-09
4753 8.52489101532683e-09
4754 7.87142795388718e-09
4755 8.62800852985401e-09
4756 8.45417602590715e-09
4757 8.18812928571333e-09
4758 7.76489628151467e-09
4759 8.44810177369482e-09
4760 8.37353208993363e-09
4761 8.15691070243929e-09
4762 8.23155854590141e-09
4763 8.57159498934834e-09
4764 7.97838062283063e-09
4765 7.77916220329189e-09
4766 8.63513438531527e-09
4767 8.92473117630743e-09
4768 7.50172635122226e-09
4769 8.47430658978965e-09
4770 7.39372918445724e-09
4771 8.14122813608265e-09
4772 7.17123826987631e-09
4773 8.17824297172365e-09
4774 9.10878927840031e-09
4775 7.85921194790262e-09
4776 8.65402860483755e-09
4777 7.93236498708438e-09
4778 8.00471777751e-09
4779 7.75782282858017e-09
4780 7.80493714103159e-09
4781 7.98458454909223e-09
4782 8.17477463499472e-09
4783 8.45983638697589e-09
4784 9.54080103809929e-09
4785 8.43744007994474e-09
4786 7.49760342699801e-09
4787 9.28455179405319e-09
4788 7.56803419932339e-09
4789 9.70888969220596e-09
4790 8.76605188437907e-09
4791 7.90377985282475e-09
4792 8.66027871637698e-09
4793 8.16672240944172e-09
4794 8.06338551484487e-09
4795 7.59392992932817e-09
4796 7.55447260303299e-09
4797 8.6594500459114e-09
4798 7.85823672799779e-09
4799 7.86816123365952e-09
4800 7.63041541063103e-09
4801 8.79814709975335e-09
4802 7.72640262880486e-09
4803 8.26645685236826e-09
4804 8.57601367698635e-09
4805 7.78735298467836e-09
4806 8.57448601010447e-09
4807 8.18382339673462e-09
4808 8.70166960709184e-09
4809 7.64991003876503e-09
4810 9.17850151438415e-09
4811 7.31546645482695e-09
4812 7.46776063209609e-09
4813 7.5141128874634e-09
4814 7.23075066488832e-09
4815 7.3000228084652e-09
4816 7.32075111642416e-09
4817 7.57896945202674e-09
4818 7.56921725297843e-09
4819 7.72608732546587e-09
4820 8.06776867534609e-09
4821 8.9009022374853e-09
4822 7.9352329152016e-09
4823 7.90465026767606e-09
4824 7.73574715395853e-09
4825 7.75844277711712e-09
4826 7.81686892992184e-09
4827 7.89256127120552e-09
4828 7.74781128143331e-09
4829 7.44360173499103e-09
4830 7.66002195007331e-09
4831 7.5407626809465e-09
4832 7.59259766169862e-09
4833 7.47772688214354e-09
4834 7.89860887806526e-09
4835 7.78738407092305e-09
4836 8.02440425218265e-09
4837 8.12627298785173e-09
4838 7.88831666653778e-09
4839 8.37661140451473e-09
4840 7.60589902171205e-09
4841 7.9334299130096e-09
4842 7.92080179223831e-09
4843 8.17072809411457e-09
4844 8.0844326788565e-09
4845 7.59452944976147e-09
4846 7.51167661405816e-09
4847 7.36252614430555e-09
4848 7.4751147494112e-09
4849 7.31098870332403e-09
4850 7.367622956167e-09
4851 7.55434026444846e-09
4852 7.48915240933457e-09
4853 7.699285653473e-09
4854 8.09579336902289e-09
4855 8.27746049480993e-09
4856 7.71742580951695e-09
4857 7.81524978066273e-09
4858 7.75545583309167e-09
4859 7.56305684745939e-09
4860 8.19415646446942e-09
4861 8.01953614626427e-09
4862 8.68044036650417e-09
4863 6.99969548989543e-09
4864 8.02873767469237e-09
4865 8.02700661495237e-09
4866 7.75151320908662e-09
4867 7.59178675480143e-09
4868 8.04145994237615e-09
4869 7.49711048797508e-09
4870 7.19614634547838e-09
4871 7.7773050222163e-09
4872 7.92808130256617e-09
4873 8.14386247327548e-09
4874 8.34953084449808e-09
4875 7.45519024292207e-09
4876 7.47369099940443e-09
4877 7.57389884142867e-09
4878 7.56725082595722e-09
4879 7.75295738719706e-09
4880 8.04971644896568e-09
4881 7.46768158421673e-09
4882 7.46710870913603e-09
4883 8.01017385754221e-09
4884 7.46146522345725e-09
4885 7.84167841771932e-09
4886 8.07409605840803e-09
4887 7.38155137014473e-09
4888 7.67557928327278e-09
4889 8.60925819523572e-09
4890 7.39149896844538e-09
4891 7.72795871739618e-09
4892 7.93371679463917e-09
4893 8.41549052665869e-09
4894 7.95358445770944e-09
4895 7.78438558057815e-09
4896 8.66786109554596e-09
4897 7.39654915093979e-09
4898 7.47317940863468e-09
4899 7.5343695726815e-09
4900 7.9616437886898e-09
4901 7.50662110249323e-09
4902 7.60775531460922e-09
4903 7.46266604068069e-09
4904 7.80224773677674e-09
4905 7.75384911833044e-09
4906 7.70824737372777e-09
4907 7.36649274912793e-09
4908 7.42437400447216e-09
4909 7.37109839832328e-09
4910 7.21481896448495e-09
4911 7.94714072327452e-09
4912 7.49461914750782e-09
4913 7.38900096663997e-09
4914 7.81615128175872e-09
4915 7.70290320417644e-09
4916 7.57196438883057e-09
4917 7.450362993211e-09
4918 8.0733686402823e-09
4919 7.65924834666976e-09
4920 8.11652522969553e-09
4921 7.622866782242e-09
4922 7.73001929132988e-09
4923 7.92963739115748e-09
4924 7.4499300062314e-09
4925 8.18459966467344e-09
4926 8.2201969675566e-09
4927 7.52002016213282e-09
4928 7.65520713486012e-09
4929 7.38476790829168e-09
4930 7.78860176353646e-09
4931 7.29007609834298e-09
4932 7.94941001913685e-09
4933 7.48053707866347e-09
4934 7.47146433610624e-09
4935 7.69099273156826e-09
4936 7.59085594381759e-09
4937 7.54635731681219e-09
4938 7.72165442697315e-09
4939 7.7845321300174e-09
4940 7.47055750593972e-09
4941 7.69703945024958e-09
4942 7.74227348898648e-09
4943 7.71015340461645e-09
4944 7.67867014417334e-09
4945 7.95281351884114e-09
4946 7.58605622763753e-09
4947 7.63586616159273e-09
4948 7.78367681419923e-09
4949 8.00381005916506e-09
4950 7.94299648276819e-09
4951 7.92514676106748e-09
4952 7.71500285878801e-09
4953 7.68664953909592e-09
4954 7.44087991222386e-09
4955 7.54641948930157e-09
4956 7.53078666093643e-09
4957 7.61426299789036e-09
4958 8.04176814028779e-09
4959 7.77898545578637e-09
4960 8.73118732869216e-09
4961 6.96989976844975e-09
4962 7.51784323682614e-09
4963 7.66541408125931e-09
4964 8.02127342325321e-09
4965 7.86706344513277e-09
4966 7.43402717162667e-09
4967 7.85444598250251e-09
4968 8.02670996336019e-09
4969 8.31850588411953e-09
4970 7.90090837199386e-09
4971 8.12558820229015e-09
4972 7.57647100613212e-09
4973 8.05378608248475e-09
4974 7.87324783146914e-09
4975 8.43953973372891e-09
4976 7.84293163746952e-09
4977 7.73440778090162e-09
4978 7.76624631271261e-09
4979 8.20346279795103e-09
4980 7.73656960717517e-09
4981 7.64279040055271e-09
4982 7.90927145999376e-09
4983 7.97421328968539e-09
4984 7.58067564277098e-09
4985 7.79753506208181e-09
4986 7.84263409769892e-09
4987 7.89610332674329e-09
4988 7.63482965737694e-09
4989 7.85256037971749e-09
4990 7.57392815131652e-09
4991 7.37152960894605e-09
4992 8.08064903878858e-09
4993 8.78405259641113e-09
4994 7.82553932765495e-09
4995 7.7316339996969e-09
4996 7.78823849856281e-09
4997 7.91733612004464e-09
4998 7.97396282337104e-09
4999 7.82844544744421e-09
};
\addlegendentry{Test}

\nextgroupplot[
title={Batch Size 2 $\rare$},
ymin=4.90143384155618e-09, ymax=1e-05,
]
\addplot [semithick, black, dashed]
table {%
0 0.00261398348659895
1 0.000146684363119675
2 7.82634508095725e-05
3 3.37719028447054e-05
4 2.5150746102085e-05
5 2.14879413085001e-05
6 1.73131360368757e-05
7 1.25710438198219e-05
8 8.384994563432e-06
9 6.01536302192685e-06
10 5.04094846438363e-06
11 4.50247262365266e-06
12 4.07342948584244e-06
13 3.68863333453007e-06
14 3.33568109506999e-06
15 3.02494765912797e-06
16 2.7562514826851e-06
17 2.52888349787028e-06
18 2.34137056487427e-06
19 2.18880018897138e-06
20 2.06381847633885e-06
21 1.96098151437774e-06
22 1.87465510770934e-06
23 1.80052594150659e-06
24 1.73351173886616e-06
25 1.67263015383323e-06
26 1.61614738892268e-06
27 1.56198234465066e-06
28 1.50867286552803e-06
29 1.46046563780544e-06
30 1.41356673157844e-06
31 1.37106688897681e-06
32 1.33059727563989e-06
33 1.28919083652868e-06
34 1.25175892679241e-06
35 1.21776896658865e-06
36 1.18564458267567e-06
37 1.15591732029952e-06
38 1.12700339668681e-06
39 1.10075354224648e-06
40 1.07587387876951e-06
41 1.05260377831939e-06
42 1.0313208120829e-06
43 1.01144230139827e-06
44 9.92442267119031e-07
45 9.7406527741839e-07
46 9.57213383753874e-07
47 9.40670659906218e-07
48 9.25038841827153e-07
49 9.10250043492766e-07
50 8.96198767826206e-07
51 8.83881172112133e-07
52 8.71137165300517e-07
53 8.58803400506147e-07
54 8.4801414614688e-07
55 8.36522416101104e-07
56 8.25851773152308e-07
57 8.15359453404163e-07
58 8.05294669189216e-07
59 7.95505481549164e-07
60 7.86104835461821e-07
61 7.76836224471111e-07
62 7.67232828448705e-07
63 7.40208569903089e-07
64 7.24697941538288e-07
65 7.12473284208315e-07
66 6.98334301669767e-07
67 6.8646508113801e-07
68 6.76835167410239e-07
69 6.67913647898644e-07
70 6.59476819643956e-07
71 6.51745536447201e-07
72 6.44344590145174e-07
73 6.37315424147467e-07
74 6.29807009134176e-07
75 6.23752871640626e-07
76 6.17110982172875e-07
77 6.10710473006648e-07
78 6.04971023602019e-07
79 5.99273899850594e-07
80 5.93674280890077e-07
81 5.88465108357639e-07
82 5.83262429605824e-07
83 5.78088403622168e-07
84 5.73076810137074e-07
85 5.68146300770955e-07
86 5.63457285606361e-07
87 5.59085755743816e-07
88 5.54764195544344e-07
89 5.50326957575287e-07
90 5.46110211153916e-07
91 5.41905436579171e-07
92 5.38032570275071e-07
93 5.33861362991495e-07
94 5.29962581785171e-07
95 5.25934999759059e-07
96 5.21953731732694e-07
97 5.17996030810153e-07
98 5.13999557568923e-07
99 5.10104440494707e-07
100 5.06136233847876e-07
101 5.02081006532151e-07
102 4.98216791648076e-07
103 4.94461015678738e-07
104 4.90604048617449e-07
105 4.86462815026734e-07
106 4.82587882216556e-07
107 4.78588768475063e-07
108 4.74869900535246e-07
109 4.7077929747541e-07
110 4.66663813166512e-07
111 4.62867599476446e-07
112 4.58659658813865e-07
113 4.54709814113663e-07
114 4.50870674558068e-07
115 4.46929221819659e-07
116 4.42388200826649e-07
117 4.38506839680386e-07
118 4.34678859113191e-07
119 4.3089226689208e-07
120 4.27085763439194e-07
121 4.23271578807061e-07
122 4.19777154811563e-07
123 4.15960641148239e-07
124 4.12074457444511e-07
125 4.08489014579239e-07
126 4.04658556217186e-07
127 4.00951681431394e-07
128 3.9733187562252e-07
129 3.93702260550022e-07
130 3.90062106880862e-07
131 3.86685338110482e-07
132 3.83489258807046e-07
133 3.79915736270497e-07
134 3.76464143924338e-07
135 3.73031685953151e-07
136 3.69614877314461e-07
137 3.66566005281399e-07
138 3.63254276162017e-07
139 3.59711858066269e-07
140 3.56280412384891e-07
141 3.52569088435395e-07
142 3.49126307100711e-07
143 3.45766744868303e-07
144 3.42371599904911e-07
145 3.39262808538354e-07
146 3.36348832967737e-07
147 3.33401700036262e-07
148 3.30368856395147e-07
149 3.2688863233421e-07
150 3.24198848596247e-07
151 3.21372436439926e-07
152 3.17807970312778e-07
153 3.15233077704757e-07
154 3.12444794476452e-07
155 3.10027523902079e-07
156 3.07321188873999e-07
157 3.04717508505359e-07
158 3.02148876103026e-07
159 2.99573802886144e-07
160 2.96987011524585e-07
161 2.94648659848296e-07
162 2.92419047800419e-07
163 2.90140671955541e-07
164 2.88099955823906e-07
165 2.8558504167453e-07
166 2.83444139733025e-07
167 2.8141647362645e-07
168 2.79245677041295e-07
169 2.77200937237154e-07
170 2.75588299680773e-07
171 2.73590638562027e-07
172 2.71654279449418e-07
173 2.69519297966259e-07
174 2.67458664513498e-07
175 2.65497534967629e-07
176 2.63673217523852e-07
177 2.61948017285185e-07
178 2.60005301479183e-07
179 2.58152771367648e-07
180 2.56357327418133e-07
181 2.54637377330891e-07
182 2.5292912211583e-07
183 2.51359738450141e-07
184 2.49551562201411e-07
185 2.48045789688955e-07
186 2.46444842606497e-07
187 2.4465752219438e-07
188 2.43001887576533e-07
189 2.41387644265956e-07
190 2.39743852080032e-07
191 2.38591042211134e-07
192 2.36896491812466e-07
193 2.35331076664291e-07
194 2.33869138022147e-07
195 2.32390313131603e-07
196 2.30942950676827e-07
197 2.29478455254561e-07
198 2.28119508614011e-07
199 2.26882294488195e-07
200 2.25635528665613e-07
201 2.24384542468314e-07
202 2.23268744437011e-07
203 2.21910245763457e-07
204 2.20867471478314e-07
205 2.19511795347405e-07
206 2.18217545115529e-07
207 2.17078903803181e-07
208 2.15910442586509e-07
209 2.14819290250023e-07
210 2.1386440760951e-07
211 2.12621667512458e-07
212 2.11595954674237e-07
213 2.10598725390199e-07
214 2.09534008213996e-07
215 2.08534801116933e-07
216 2.07571893242164e-07
217 2.06601709264564e-07
218 2.05631415472141e-07
219 2.04732785008765e-07
220 2.03928731128844e-07
221 2.02986788394366e-07
222 2.02125836123557e-07
223 2.01693005556169e-07
224 2.00247696379385e-07
225 1.99351367456746e-07
226 1.9847183270949e-07
227 1.97567987980651e-07
228 1.96726100124334e-07
229 1.95874223668291e-07
230 1.95032362352077e-07
231 1.9429453191222e-07
232 1.93577463155847e-07
233 1.92833159978534e-07
234 1.91930465522017e-07
235 1.91151176648585e-07
236 1.90727425620985e-07
237 1.89940357579133e-07
238 1.89282947241409e-07
239 1.88559956265433e-07
240 1.87832262899867e-07
241 1.87312221969105e-07
242 1.8638003640481e-07
243 1.85629820950606e-07
244 1.84861143206927e-07
245 1.84173773354779e-07
246 1.8346385132828e-07
247 1.82691622820474e-07
248 1.81981703223588e-07
249 1.81241538350196e-07
250 1.80819522359599e-07
251 1.80233316547929e-07
252 1.79373761339141e-07
253 1.78795408649846e-07
254 1.78261502097099e-07
255 1.77732386884122e-07
256 1.77024827346939e-07
257 1.76212229890949e-07
258 1.75559959624039e-07
259 1.75028014774092e-07
260 1.74455442444987e-07
261 1.73865569054188e-07
262 1.73293729991419e-07
263 1.72699000502652e-07
264 1.72175157529209e-07
265 1.71600795157545e-07
266 1.71060159445346e-07
267 1.70516953865052e-07
268 1.69965971575703e-07
269 1.69306111652556e-07
270 1.68603019955915e-07
271 1.68096036264442e-07
272 1.67533045300372e-07
273 1.67098511944852e-07
274 1.66452356481939e-07
275 1.6594090199118e-07
276 1.65430003029599e-07
277 1.6504094741765e-07
278 1.64535466716353e-07
279 1.64067817844948e-07
280 1.63518295869691e-07
281 1.63008348729843e-07
282 1.62516949801628e-07
283 1.62122773014195e-07
284 1.61727112577337e-07
285 1.61419068025026e-07
286 1.60981843060126e-07
287 1.60612840249152e-07
288 1.60113057938505e-07
289 1.59996616506453e-07
290 1.59500434703475e-07
291 1.59041253554371e-07
292 1.58633176553291e-07
293 1.58230017347494e-07
294 1.57726207859366e-07
295 1.57391601654178e-07
296 1.56932869004311e-07
297 1.56524124702795e-07
298 1.56090241693274e-07
299 1.55728700207858e-07
300 1.55397476291563e-07
301 1.54913462214301e-07
302 1.54460410595414e-07
303 1.54037259622264e-07
304 1.53807496733549e-07
305 1.53307295429617e-07
306 1.52889717186522e-07
307 1.52815606427659e-07
308 1.5255875534792e-07
309 1.51992299653791e-07
310 1.51631543023223e-07
311 1.51434879765988e-07
312 1.50798544854336e-07
313 1.50500502398065e-07
314 1.50309462715548e-07
315 1.5012875835807e-07
316 1.49606689905601e-07
317 1.49262755533508e-07
318 1.489406507591e-07
319 1.48563328749907e-07
320 1.48307872039899e-07
321 1.47996224146629e-07
322 1.47477975280585e-07
323 1.46864158302451e-07
324 1.46399182789247e-07
325 1.4600326030334e-07
326 1.45447961132694e-07
327 1.45286996641181e-07
328 1.45124027577159e-07
329 1.4457256800382e-07
330 1.44243595714633e-07
331 1.43893380800231e-07
332 1.43401146519118e-07
333 1.43232647254354e-07
334 1.42721284159375e-07
335 1.42463075213817e-07
336 1.42195513040511e-07
337 1.41774924736993e-07
338 1.41613115455463e-07
339 1.41290982773601e-07
340 1.40814341321294e-07
341 1.40588078612947e-07
342 1.40190531043416e-07
343 1.39982248199333e-07
344 1.39655671995254e-07
345 1.39315308924859e-07
346 1.39008234261606e-07
347 1.38621841989428e-07
348 1.38385781822015e-07
349 1.38106749037359e-07
350 1.37768745258837e-07
351 1.37302368972358e-07
352 1.37273293095319e-07
353 1.36685038603535e-07
354 1.36805641671289e-07
355 1.36218938038724e-07
356 1.36174170560555e-07
357 1.35144350333238e-07
358 1.35377791940172e-07
359 1.3519789696903e-07
360 1.34321391878522e-07
361 1.34628653254643e-07
362 1.34382502061214e-07
363 1.33540979393221e-07
364 1.33748807011447e-07
365 1.33484170640186e-07
366 1.32627974005484e-07
367 1.3297263567158e-07
368 1.32693508702708e-07
369 1.32514474537659e-07
370 1.32011076459548e-07
371 1.31861206790651e-07
372 1.31521757749109e-07
373 1.31331652518663e-07
374 1.31312317972254e-07
375 1.30603197236123e-07
376 1.30561156867826e-07
377 1.29773564399915e-07
378 1.30057639510683e-07
379 1.29861310177226e-07
380 1.29112238212326e-07
381 1.29198381696227e-07
382 1.28842748736258e-07
383 1.2872677043152e-07
384 1.2815145879741e-07
385 1.28243387647053e-07
386 1.28023895214824e-07
387 1.27487541142557e-07
388 1.27568656234978e-07
389 1.27031404781208e-07
390 1.27159839126145e-07
391 1.26598800443301e-07
392 1.26791784661684e-07
393 1.26477510927092e-07
394 1.26444590979657e-07
395 1.26158001562349e-07
396 1.25962107867528e-07
397 1.25775401044637e-07
398 1.25593913733013e-07
399 1.24813759142306e-07
400 1.25107088592591e-07
401 1.24789898830358e-07
402 1.24521163066271e-07
403 1.24255962218545e-07
404 1.24277898207348e-07
405 1.23580057124872e-07
406 1.23692217324445e-07
407 1.23466668605632e-07
408 1.23292362624872e-07
409 1.23125864866624e-07
410 1.22844255137977e-07
411 1.22260624106474e-07
412 1.22632293348746e-07
413 1.22149892797907e-07
414 1.21888570189244e-07
415 1.2183269884436e-07
416 1.21501794058565e-07
417 1.21598056354921e-07
418 1.21115364892788e-07
419 1.20845609222542e-07
420 1.20702182457633e-07
421 1.20670180585147e-07
422 1.20313101976244e-07
423 1.20242447746888e-07
424 1.19796707906294e-07
425 1.19679437805553e-07
426 1.19502443590935e-07
427 1.19306255657259e-07
428 1.19001857039169e-07
429 1.18975247541897e-07
430 1.18751168056708e-07
431 1.18668970422453e-07
432 1.18519750465174e-07
433 1.18310549850564e-07
434 1.18205436012264e-07
435 1.18052298470728e-07
436 1.17879357376394e-07
437 1.17379495784053e-07
438 1.17300958525135e-07
439 1.17437472770154e-07
440 1.17254779579357e-07
441 1.16698817170358e-07
442 1.1665886642831e-07
443 1.16422076178457e-07
444 1.16437312810413e-07
445 1.16211605543537e-07
446 1.15907504869339e-07
447 1.15631951317585e-07
448 1.15533178603799e-07
449 1.1553922251073e-07
450 1.15184505566468e-07
451 1.15018143974188e-07
452 1.15032055581388e-07
453 1.14700197587414e-07
454 1.14470164584324e-07
455 1.14392085056592e-07
456 1.14392627027948e-07
457 1.13851511679419e-07
458 1.13809522381825e-07
459 1.13363633469143e-07
460 1.13507051060058e-07
461 1.12955144015325e-07
462 1.12687809270984e-07
463 1.12958656940165e-07
464 1.12561146591084e-07
465 1.1211510408593e-07
466 1.11795333277431e-07
467 1.11874699544767e-07
468 1.11499599053211e-07
469 1.11183911756685e-07
470 1.1097315462405e-07
471 1.10746568335651e-07
472 1.10528108287378e-07
473 1.10256356484095e-07
474 1.10099632397498e-07
475 1.09766951710499e-07
476 1.09138375835371e-07
477 1.08834864133156e-07
478 1.0881889858938e-07
479 1.08393336007095e-07
480 1.08144232992657e-07
481 1.0793406884968e-07
482 1.07731355728502e-07
483 1.07404242227993e-07
484 1.07224436205255e-07
485 1.06623043470755e-07
486 1.06600979004057e-07
487 1.06223428439911e-07
488 1.06042661533579e-07
489 1.05791827745172e-07
490 1.05572618482874e-07
491 1.05332027782401e-07
492 1.05210153051427e-07
493 1.04844322206032e-07
494 1.04628191569089e-07
495 1.04377984629922e-07
496 1.04216347538877e-07
497 1.03955194479699e-07
498 1.03688604859542e-07
499 1.03510769820403e-07
500 1.03342628648528e-07
501 1.03084363703987e-07
502 1.0284133121341e-07
503 1.02668651758897e-07
504 1.02572349458985e-07
505 1.02124076692522e-07
506 1.01995372278507e-07
507 1.01731550677941e-07
508 1.01490412855565e-07
509 1.01381675744916e-07
510 1.01240593376462e-07
511 1.00909068223354e-07
512 1.00669337769643e-07
513 1.00404250107822e-07
514 1.00423404139294e-07
515 1.00214220461581e-07
516 9.99527497607122e-08
517 9.97062718719466e-08
518 9.94624701242675e-08
519 9.93852768302883e-08
520 9.90874772937023e-08
521 9.88455424544288e-08
522 9.86786483039293e-08
523 9.85174300247582e-08
524 9.83541504726571e-08
525 9.82658134411896e-08
526 9.78364765480411e-08
527 9.77660970995498e-08
528 9.7614606940466e-08
529 9.7288654843819e-08
530 9.69792118370449e-08
531 9.69268959594149e-08
532 9.67892044398955e-08
533 9.64910404850361e-08
534 9.62170009874974e-08
535 9.62202412035929e-08
536 9.58558749346583e-08
537 9.56776447187391e-08
538 9.5498134587757e-08
539 9.53894120798715e-08
540 9.50934551380289e-08
541 9.48801197169224e-08
542 9.46125329708281e-08
543 9.44436887060363e-08
544 9.4335708752169e-08
545 9.40767704953327e-08
546 9.38061104351906e-08
547 9.36843962597855e-08
548 9.34865146009489e-08
549 9.32645453668446e-08
550 9.30323500869523e-08
551 9.2910364995813e-08
552 9.24915208440069e-08
553 9.26237489684567e-08
554 9.20383658806756e-08
555 9.20737594465315e-08
556 9.1773045729715e-08
557 9.15990957992552e-08
558 9.16372675546784e-08
559 9.13390870608266e-08
560 9.12179028298432e-08
561 9.10503901381254e-08
562 9.07964728142918e-08
563 9.05843692435848e-08
564 9.03890795960205e-08
565 9.00719094163449e-08
566 8.99008989017069e-08
567 8.98419131223349e-08
568 8.94903660499935e-08
569 8.92771059846087e-08
570 8.92133772965043e-08
571 8.92505096738994e-08
572 8.88596866948088e-08
573 8.88809712964456e-08
574 8.85358828333072e-08
575 8.83346679561026e-08
576 8.84105503005106e-08
577 8.8131889763865e-08
578 8.80912545679902e-08
579 8.76220811723005e-08
580 8.76374261120638e-08
581 8.73306817090747e-08
582 8.69455808030217e-08
583 8.70990354120416e-08
584 8.65921273616177e-08
585 8.64360111918483e-08
586 8.64471381185616e-08
587 8.61044444553372e-08
588 8.59900289835735e-08
589 8.59733905260729e-08
590 8.56495371353017e-08
591 8.5448452592729e-08
592 8.52556495878343e-08
593 8.50720472700406e-08
594 8.48990268551564e-08
595 8.46750809213592e-08
596 8.45080046549818e-08
597 8.42445121276292e-08
598 8.40143372828894e-08
599 8.37218631160042e-08
600 8.35349402392716e-08
601 8.31654908488577e-08
602 8.30242949245719e-08
603 8.2982439593593e-08
604 8.2921621885168e-08
605 8.26500533532837e-08
606 8.25823081787025e-08
607 8.2389301115926e-08
608 8.22421491531999e-08
609 8.19840703343289e-08
610 8.1801761944611e-08
611 8.18592255097395e-08
612 8.16676565214003e-08
613 8.16171640491969e-08
614 8.13247282057672e-08
615 8.10373738009407e-08
616 8.08642084723088e-08
617 8.06555648807938e-08
618 8.039214622102e-08
619 8.02460349925704e-08
620 8.00810798282647e-08
621 7.99409526959227e-08
622 7.97110490066144e-08
623 7.94116076875406e-08
624 7.92543457824868e-08
625 7.90482818721072e-08
626 7.87830216444352e-08
627 7.86491988641336e-08
628 7.84446622161816e-08
629 7.827243654146e-08
630 7.81483883874889e-08
631 7.79432979882699e-08
632 7.77564276402964e-08
633 7.75270853509147e-08
634 7.74125279346949e-08
635 7.70996542881486e-08
636 7.70233436728773e-08
637 7.66381009270622e-08
638 7.64740521639329e-08
639 7.66816242462332e-08
640 7.64246416755654e-08
641 7.61894967465926e-08
642 7.59976832137577e-08
643 7.56666841013054e-08
644 7.54723867576468e-08
645 7.54306852314146e-08
646 7.53268538478125e-08
647 7.51730338366396e-08
648 7.47495809915177e-08
649 7.48541830168925e-08
650 7.43791588843079e-08
651 7.42978026984087e-08
652 7.42460074155682e-08
653 7.40438198457705e-08
654 7.3634554738522e-08
655 7.37113431152903e-08
656 7.34467651670734e-08
657 7.33724932918678e-08
658 7.31816905603644e-08
659 7.31439444945359e-08
660 7.28667297711372e-08
661 7.26887656533615e-08
662 7.25385389976907e-08
663 7.24498825926956e-08
664 7.22832119443018e-08
665 7.21822083353807e-08
666 7.19495448092688e-08
667 7.18707898982318e-08
668 7.16480042597389e-08
669 7.14073806823423e-08
670 7.13205865152666e-08
671 7.10196900129967e-08
672 7.09722795270151e-08
673 7.07187652627672e-08
674 7.06024032132158e-08
675 7.05165593821722e-08
676 7.03380461603009e-08
677 7.03425385415457e-08
678 7.00737257445239e-08
679 6.99320477606236e-08
680 6.97025158427067e-08
681 6.95539946653501e-08
682 6.93178849876519e-08
683 6.94211070342288e-08
684 6.90671745279259e-08
685 6.8847339870004e-08
686 6.89086112218851e-08
687 6.87327516597502e-08
688 6.84967959185823e-08
689 6.84914813144921e-08
690 6.84173265950161e-08
691 6.81482263986677e-08
692 6.79657498626751e-08
693 6.76911481836129e-08
694 6.78754415739391e-08
695 6.76873341790563e-08
696 6.7420077696112e-08
697 6.73318192355721e-08
698 6.71849346519648e-08
699 6.70631731659599e-08
700 6.6936109860527e-08
701 6.67347293343834e-08
702 6.66230708357141e-08
703 6.67056503975694e-08
704 6.66208884274599e-08
705 6.6402336841076e-08
706 6.637572022683e-08
707 6.60298795550629e-08
708 6.58703416722695e-08
709 6.59932807797192e-08
710 6.57574264092409e-08
711 6.5755870203299e-08
712 6.55362282209193e-08
713 6.54728074114264e-08
714 6.5351697348004e-08
715 6.53087951680842e-08
716 6.51708659051842e-08
717 6.50125667877033e-08
718 6.49798564733572e-08
719 6.47724273014072e-08
720 6.47707329970437e-08
721 6.45535312721046e-08
722 6.44558372296933e-08
723 6.43093136186712e-08
724 6.406546154758e-08
725 6.40719362525743e-08
726 6.45583160179264e-08
727 6.38616268305858e-08
728 6.39997100977396e-08
729 6.38944717001877e-08
730 6.37970664736365e-08
731 6.36738414186988e-08
732 6.35405101436781e-08
733 6.3468621691154e-08
734 6.31237134999241e-08
735 6.32146194542438e-08
736 6.31069264531714e-08
737 6.30518270504643e-08
738 6.29606186937082e-08
739 6.26848402600633e-08
740 6.28600875853813e-08
741 6.26476349310234e-08
742 6.24892591627457e-08
743 6.24455489549591e-08
744 6.23294498194316e-08
745 6.2253770162779e-08
746 6.21718271596183e-08
747 6.20670583866279e-08
748 6.18798210956228e-08
749 6.188538615437e-08
750 6.18037680910621e-08
751 6.16810066836893e-08
752 6.15384937397989e-08
753 6.14532277937174e-08
754 6.1445070853372e-08
755 6.13365998902715e-08
756 6.12798440569051e-08
757 6.13104954930721e-08
758 6.12681508648238e-08
759 6.10669092498961e-08
760 6.10586326349472e-08
761 6.09786342657959e-08
762 6.08765920058207e-08
763 6.07757833117617e-08
764 6.07843563827926e-08
765 6.05380540213973e-08
766 6.04225618350274e-08
767 6.03992417815835e-08
768 6.03464287243227e-08
769 6.01462231776262e-08
770 6.00604040950081e-08
771 6.00231084527669e-08
772 5.99093010943408e-08
773 5.98377934263317e-08
774 5.96081898500689e-08
775 5.95742457141224e-08
776 5.96780586127332e-08
777 5.93245081615956e-08
778 5.91501711765252e-08
779 5.92791223117395e-08
780 5.91412333191821e-08
781 5.90747720404794e-08
782 5.89476858179339e-08
783 5.88208731741036e-08
784 5.88255402458326e-08
785 5.8786889870488e-08
786 5.86065183543205e-08
787 5.86318991023793e-08
788 5.8584062483602e-08
789 5.84684124078638e-08
790 5.85145645294327e-08
791 5.82407579362565e-08
792 5.82223076095456e-08
793 5.79912457159271e-08
794 5.8108856008543e-08
795 5.79709602686052e-08
796 5.78531142740868e-08
797 5.78370420644125e-08
798 5.76913316553407e-08
799 5.76839951780261e-08
800 5.75831435745133e-08
801 5.75909274685982e-08
802 5.74083612785437e-08
803 5.74391697699683e-08
804 5.72027329296398e-08
805 5.73407223503075e-08
806 5.72023227312091e-08
807 5.70735137997991e-08
808 5.70729130350278e-08
809 5.69626252675537e-08
810 5.6882644279832e-08
811 5.68058462963039e-08
812 5.66683241300936e-08
813 5.66744659303842e-08
814 5.65901798542656e-08
815 5.6436309426422e-08
816 5.65831056307253e-08
817 5.63878848396371e-08
818 5.62771926944095e-08
819 5.62657206842898e-08
820 5.62157826032861e-08
821 5.61009023639647e-08
822 5.60516779550824e-08
823 5.59690822986569e-08
824 5.61079724488156e-08
825 5.58611277436949e-08
826 5.57958161638838e-08
827 5.57157521758889e-08
828 5.56878466763111e-08
829 5.57414290146552e-08
830 5.54114399603511e-08
831 5.54050776053749e-08
832 5.54322031087739e-08
833 5.54243826426104e-08
834 5.53643694637396e-08
835 5.51954801198962e-08
836 5.53016565781883e-08
837 5.52197657462949e-08
838 5.50475004045259e-08
839 5.50929451459403e-08
840 5.49398851192873e-08
841 5.50032194527317e-08
842 5.49059920349482e-08
843 5.48543149434533e-08
844 5.47220845609209e-08
845 5.45919308593268e-08
846 5.46621458888952e-08
847 5.45036415513511e-08
848 5.44871118437484e-08
849 5.44909918462899e-08
850 5.44362895656958e-08
851 5.42284673225035e-08
852 5.43319143780918e-08
853 5.42738582536284e-08
854 5.39490024304978e-08
855 5.41623808413272e-08
856 5.38711755492249e-08
857 5.38292376828231e-08
858 5.38144721166089e-08
859 5.40801493253973e-08
860 5.35556771573686e-08
861 5.34009007191472e-08
862 5.36913170899878e-08
863 5.35750088621612e-08
864 5.34702983183699e-08
865 5.31324628701979e-08
866 5.32663798854527e-08
867 5.34713005498899e-08
868 5.3268240315818e-08
869 5.3106281430626e-08
870 5.30999553991496e-08
871 5.30073745623749e-08
872 5.3024831385029e-08
873 5.29475644116539e-08
874 5.29035295295799e-08
875 5.27633898719237e-08
876 5.27644872644339e-08
877 5.26363374481198e-08
878 5.21344579949012e-08
879 5.26222953115552e-08
880 5.25046283263997e-08
881 5.25887310142137e-08
882 5.23786171260365e-08
883 5.22556745392588e-08
884 5.22890715103364e-08
885 5.22051752521735e-08
886 5.22562960395545e-08
887 5.21403642420593e-08
888 5.21259177299616e-08
889 5.23303570957312e-08
890 5.21940686630806e-08
891 5.20369648042696e-08
892 5.18643744533698e-08
893 5.1959078596342e-08
894 5.1800517703815e-08
895 5.17839124822839e-08
896 5.1749980221838e-08
897 5.20235737372365e-08
898 5.14930678036096e-08
899 5.13947631269884e-08
900 5.14576784906851e-08
901 5.15045462358144e-08
902 5.14380473438658e-08
903 5.14084751558341e-08
904 5.13051172055246e-08
905 5.13162940889433e-08
906 5.12426185311776e-08
907 5.11894807858626e-08
908 5.12040236309019e-08
909 5.11771593126875e-08
910 5.1391161834613e-08
911 5.09132827229974e-08
912 5.07956156252654e-08
913 5.07086769411247e-08
914 5.07240060290126e-08
915 5.05156226283665e-08
916 5.09034994145008e-08
917 5.10812179872477e-08
918 5.09299017443787e-08
919 5.06071179470213e-08
920 5.06153786566932e-08
921 5.08228125277732e-08
922 5.07701664046456e-08
923 5.04631669587807e-08
924 5.04625080791632e-08
925 5.03999359618978e-08
926 5.03120141140956e-08
927 5.02950299255955e-08
928 5.02036412520779e-08
929 5.05533185670704e-08
930 5.03805240215094e-08
931 5.03134604930011e-08
932 5.03399029133655e-08
933 5.02495765224431e-08
934 5.02485439819456e-08
935 5.01097002670869e-08
936 4.98450323295208e-08
937 5.00081839931443e-08
938 5.02201354417586e-08
939 5.00173292057315e-08
940 4.99454150872936e-08
941 4.98917330522541e-08
942 4.97869760859304e-08
943 4.96477238839388e-08
944 4.96061485512067e-08
945 4.96272340385628e-08
946 4.96270244700892e-08
947 4.96611786039436e-08
948 4.93043708666985e-08
949 4.93675331766363e-08
950 4.94701070694603e-08
951 4.95901512781449e-08
952 4.95660200040549e-08
953 4.93224323655506e-08
954 4.91889253581013e-08
955 4.92657079411152e-08
956 4.93090457755474e-08
957 4.90304114769136e-08
958 4.86113878614414e-08
959 4.87386970814407e-08
960 4.86080396257527e-08
961 4.91473815086851e-08
962 4.82496446461145e-08
963 4.8349681513904e-08
964 4.83994517888053e-08
965 4.84674849848266e-08
966 4.89019669674406e-08
967 4.87339815461452e-08
968 4.8887384868801e-08
969 4.86750621041532e-08
970 4.87458362598003e-08
971 4.87206623641101e-08
972 4.85998083747385e-08
973 4.86616961933861e-08
974 4.85624361134529e-08
975 4.85805682470253e-08
976 4.85125720613433e-08
977 4.82887160206391e-08
978 4.84719224294605e-08
979 4.83484899962972e-08
980 4.82735046126725e-08
981 4.84206677583976e-08
982 4.82095535099814e-08
983 4.81738937898046e-08
984 4.81529513432499e-08
985 4.80463511752238e-08
986 4.8008118166909e-08
987 4.78388747605085e-08
988 4.79545476203547e-08
989 4.77904276480756e-08
990 4.76947455610444e-08
991 4.7953770904885e-08
992 4.75322841403392e-08
993 4.7785734386907e-08
994 4.75348239130091e-08
995 4.7667171884147e-08
996 4.74739370615596e-08
997 4.75618149212709e-08
998 4.75811124144854e-08
999 4.74248252290144e-08
1000 4.76309443844802e-08
1001 4.72376396284391e-08
1002 4.71671802128038e-08
1003 4.74440110288521e-08
1004 4.69942068811458e-08
1005 4.71157604259309e-08
1006 4.69819341240574e-08
1007 4.69522687752133e-08
1008 4.68813971180038e-08
1009 4.66572676727783e-08
1010 4.68712826183215e-08
1011 4.65230975445485e-08
1012 4.6986280031025e-08
1013 4.65285257113535e-08
1014 4.65172853401641e-08
1015 4.69000739997116e-08
1016 4.59802069521786e-08
1017 4.61438803179282e-08
1018 4.61487103370906e-08
1019 4.60385496252047e-08
1020 4.60279651077755e-08
1021 4.5991702721071e-08
1022 4.6049907213741e-08
1023 4.58232247232404e-08
1024 4.575472711843e-08
1025 4.57867423995784e-08
1026 4.57194202890809e-08
1027 4.5735625373744e-08
1028 4.56614160241897e-08
1029 4.57159697072051e-08
1030 4.55817242379086e-08
1031 4.55687318516862e-08
1032 4.55431716767651e-08
1033 4.54868766998073e-08
1034 4.54809728112071e-08
1035 4.53985480052266e-08
1036 4.5422408025686e-08
1037 4.52892244754421e-08
1038 4.52831030046119e-08
1039 4.53999871000699e-08
1040 4.54181504337958e-08
1041 4.52862972734613e-08
1042 4.51580050345179e-08
1043 4.51480138310423e-08
1044 4.51643003261393e-08
1045 4.51240597721392e-08
1046 4.49955824772252e-08
1047 4.50499606158838e-08
1048 4.50189186058103e-08
1049 4.50657565729817e-08
1050 4.4959013388024e-08
1051 4.48627814274016e-08
1052 4.48606086906889e-08
1053 4.4802549736378e-08
1054 4.48229560718882e-08
1055 4.48897614062638e-08
1056 4.47645272669828e-08
1057 4.46646057877254e-08
1058 4.47493429872603e-08
1059 4.46290876262023e-08
1060 4.47393384138683e-08
1061 4.44872028639853e-08
1062 4.46542057774835e-08
1063 4.44682002784802e-08
1064 4.45064564659758e-08
1065 4.45753047110253e-08
1066 4.4414909836088e-08
1067 4.44585654978957e-08
1068 4.43124520668192e-08
1069 4.43833488869561e-08
1070 4.4300507340167e-08
1071 4.42940364882016e-08
1072 4.42912890031288e-08
1073 4.43040135522099e-08
1074 4.42048048830967e-08
1075 4.416254044598e-08
1076 4.41335653407759e-08
1077 4.40365813249022e-08
1078 4.41081502742802e-08
1079 4.40969820470483e-08
1080 4.3948786994652e-08
1081 4.41720446373584e-08
1082 4.39464986777271e-08
1083 4.40942140758627e-08
1084 4.37837767735538e-08
1085 4.39278263835718e-08
1086 4.37930967392974e-08
1087 4.37182112345003e-08
1088 4.37573598288665e-08
1089 4.37867421974048e-08
1090 4.36447713969557e-08
1091 4.38515382742977e-08
1092 4.37362916652084e-08
1093 4.36174015362445e-08
1094 4.3778218633217e-08
1095 4.35680120209891e-08
1096 4.35805192225414e-08
1097 4.36738350335086e-08
1098 4.36676859579821e-08
1099 4.35005779383379e-08
1100 4.35071349083049e-08
1101 4.35172271734396e-08
1102 4.34009853089168e-08
1103 4.33678863978049e-08
1104 4.33504221901138e-08
1105 4.32441569210851e-08
1106 4.3280685350644e-08
1107 4.34523939446541e-08
1108 4.32948200780325e-08
1109 4.31160244203088e-08
1110 4.31606878995017e-08
1111 4.31132841002824e-08
1112 4.31126667642112e-08
1113 4.30161841746823e-08
1114 4.31351706885463e-08
1115 4.30137107217399e-08
1116 4.30419452301378e-08
1117 4.28521687167449e-08
1118 4.29494802055563e-08
1119 4.2983084970083e-08
1120 4.28279311986968e-08
1121 4.29285861436868e-08
1122 4.28456561525348e-08
1123 4.29562623107116e-08
1124 4.28213921451759e-08
1125 4.27720357785155e-08
1126 4.27197061031448e-08
1127 4.27676665794574e-08
1128 4.27508851308933e-08
1129 4.2682300908925e-08
1130 4.26003897978089e-08
1131 4.24758110147416e-08
1132 4.25463953803162e-08
1133 4.25152052525024e-08
1134 4.25598208884104e-08
1135 4.239308792503e-08
1136 4.2449155177704e-08
1137 4.24381763784454e-08
1138 4.24335556179489e-08
1139 4.23089017043132e-08
1140 4.23211221574071e-08
1141 4.22933782630586e-08
1142 4.23067300873714e-08
1143 4.23677056163863e-08
1144 4.23486314717891e-08
1145 4.19986635529779e-08
1146 4.20469467001805e-08
1147 4.22094996923028e-08
1148 4.20193057089624e-08
1149 4.21166019328179e-08
1150 4.21501204259656e-08
1151 4.19883999723814e-08
1152 4.20073089106854e-08
1153 4.18943582077835e-08
1154 4.19936113004726e-08
1155 4.20366788216886e-08
1156 4.20818990342631e-08
1157 4.18583599697264e-08
1158 4.17444217319929e-08
1159 4.17618667362674e-08
1160 4.17829855202667e-08
1161 4.17799846383904e-08
1162 4.16705735562517e-08
1163 4.18948547789011e-08
1164 4.16620335019213e-08
1165 4.17125577941713e-08
1166 4.15728996195353e-08
1167 4.16248975123046e-08
1168 4.17766256601282e-08
1169 4.14573240685168e-08
1170 4.15731547034337e-08
1171 4.14832258551212e-08
1172 4.14894388031661e-08
1173 4.17455580651316e-08
1174 4.15013680700738e-08
1175 4.15968621256257e-08
1176 4.13717257054524e-08
1177 4.15543938007135e-08
1178 4.12514978142542e-08
1179 4.14077720277683e-08
1180 4.15308862883323e-08
1181 4.13152643243264e-08
1182 4.13804941682416e-08
1183 4.13969383722401e-08
1184 4.13074685233217e-08
1185 4.14006716444315e-08
1186 4.12031826499959e-08
1187 4.13484403469777e-08
1188 4.11279918449137e-08
1189 4.1116992403778e-08
1190 4.10200866945987e-08
1191 4.10916682780371e-08
1192 4.10232857664949e-08
1193 4.11754650986307e-08
1194 4.08613160786109e-08
1195 4.10294067033634e-08
1196 4.11262635822141e-08
1197 4.09699648660222e-08
1198 4.09178326050697e-08
1199 4.08179365538608e-08
1200 4.08433911486261e-08
1201 4.07925646619955e-08
1202 4.0843865591389e-08
1203 4.08273227012201e-08
1204 4.08248592192462e-08
1205 4.09464814088434e-08
1206 4.07076831734909e-08
1207 4.06216350464228e-08
1208 4.05717194499888e-08
1209 4.07536141571185e-08
1210 4.06037834261297e-08
1211 4.05800763356723e-08
1212 4.05475472338157e-08
1213 4.05940800142779e-08
1214 4.07310513145798e-08
1215 4.0481364380951e-08
1216 4.05854570876274e-08
1217 4.0419865971697e-08
1218 4.04724098932996e-08
1219 4.03760262344677e-08
1220 4.05545208236879e-08
1221 4.04062817561668e-08
1222 4.04085032262302e-08
1223 4.0206302078627e-08
1224 4.02853182185359e-08
1225 4.02691618917039e-08
1226 4.02556384009323e-08
1227 3.99654077228306e-08
1228 4.0084840536081e-08
1229 3.99519415278937e-08
1230 3.99425685834176e-08
1231 3.99511088683413e-08
1232 3.99962168105006e-08
1233 3.98757909992886e-08
1234 3.99903259502565e-08
1235 3.97727424241157e-08
1236 3.97362323014128e-08
1237 3.98615986861861e-08
1238 3.97920895919279e-08
1239 3.96503417985916e-08
1240 3.98500998187723e-08
1241 3.94888271154081e-08
1242 3.97363479529567e-08
1243 3.9670114947099e-08
1244 3.95152142287913e-08
1245 3.94588912658311e-08
1246 3.96170589693212e-08
1247 3.95713690846677e-08
1248 3.9653338033907e-08
1249 3.96299430795444e-08
1250 3.93851973877202e-08
1251 3.95682592178592e-08
1252 3.94245985052555e-08
1253 3.94186836629173e-08
1254 3.93995191990681e-08
1255 3.94417245035417e-08
1256 3.93175576240967e-08
1257 3.93010308639696e-08
1258 3.94500662774244e-08
1259 3.92660184653781e-08
1260 3.91611785369173e-08
1261 3.93277218297405e-08
1262 3.90468253679832e-08
1263 3.92391634491673e-08
1264 3.90210088450083e-08
1265 3.90619693654837e-08
1266 3.91635332250839e-08
1267 3.91998140386596e-08
1268 3.88992044481062e-08
1269 3.92167338549299e-08
1270 3.88048741853941e-08
1271 3.90122393857384e-08
1272 3.90509703188657e-08
1273 3.89146043306421e-08
1274 3.89563392785841e-08
1275 3.89260859701701e-08
1276 3.90563198613969e-08
1277 3.903581282938e-08
1278 3.87682184300187e-08
1279 3.90418795276903e-08
1280 3.88307571365099e-08
1281 3.88432197571675e-08
1282 3.88089232857825e-08
1283 3.88497545305011e-08
1284 3.8808237176613e-08
1285 3.86636885413849e-08
1286 3.8951220149408e-08
1287 3.87320291024285e-08
1288 3.86920816649039e-08
1289 3.8919329096998e-08
1290 3.86722198195133e-08
1291 3.86278853801714e-08
1292 3.88914554788622e-08
1293 3.8599475284451e-08
1294 3.8693181822369e-08
1295 3.85323008162963e-08
1296 3.85450245753982e-08
1297 3.8618123346601e-08
1298 3.86272235451401e-08
1299 3.85501028578594e-08
1300 3.85360730471018e-08
1301 3.84817750264665e-08
1302 3.83798020365811e-08
1303 3.84028968161143e-08
1304 3.83411669929168e-08
1305 3.86072583850039e-08
1306 3.83613074982914e-08
1307 3.8552181814977e-08
1308 3.82130065704755e-08
1309 3.80664171260592e-08
1310 3.84765729979919e-08
1311 3.81774920919509e-08
1312 3.82775483957487e-08
1313 3.83419538739216e-08
1314 3.82416022891574e-08
1315 3.81589909581037e-08
1316 3.84554827882466e-08
1317 3.81900211631758e-08
1318 3.80497460764073e-08
1319 3.80913908638036e-08
1320 3.79748055899243e-08
1321 3.82118173543611e-08
1322 3.8044942906712e-08
1323 3.7897786287433e-08
1324 3.80364340222261e-08
1325 3.80164513292258e-08
1326 3.80070932131105e-08
1327 3.77907139890721e-08
1328 3.79741725815008e-08
1329 3.7962045192963e-08
1330 3.78899051253767e-08
1331 3.78060559529936e-08
1332 3.78670913306345e-08
1333 3.78315917831662e-08
1334 3.80222686703902e-08
1335 3.77478427376343e-08
1336 3.76959864369364e-08
1337 3.76758144395861e-08
1338 3.78092513653727e-08
1339 3.77412872711314e-08
1340 3.76240642890879e-08
1341 3.77037665962865e-08
1342 3.75619689473861e-08
1343 3.77010441950931e-08
1344 3.77110146798088e-08
1345 3.75804105830491e-08
1346 3.76246193385943e-08
1347 3.75692676248818e-08
1348 3.74983338812807e-08
1349 3.74260324288445e-08
1350 3.73596034612955e-08
1351 3.74343606266425e-08
1352 3.74937736646319e-08
1353 3.75229895516749e-08
1354 3.71330517868751e-08
1355 3.74137304480215e-08
1356 3.7181566376332e-08
1357 3.72471152274567e-08
1358 3.74778136069676e-08
1359 3.72647145150395e-08
1360 3.72154928966473e-08
1361 3.7239597339922e-08
1362 3.72700734292408e-08
1363 3.74581570247168e-08
1364 3.7001714643492e-08
1365 3.71595026116833e-08
1366 3.72408672513203e-08
1367 3.7080263740441e-08
1368 3.72318790692883e-08
1369 3.70415177685657e-08
1370 3.72782474327149e-08
1371 3.70204627198611e-08
1372 3.70034276288567e-08
1373 3.71566659543299e-08
1374 3.69266780143596e-08
1375 3.71390608163713e-08
1376 3.69570218351489e-08
1377 3.6941323282691e-08
1378 3.6822000582748e-08
1379 3.6959429911998e-08
1380 3.68920858652144e-08
1381 3.701040906956e-08
1382 3.681259792776e-08
1383 3.66875012499102e-08
1384 3.6890900990294e-08
1385 3.68982291191755e-08
1386 3.69947587329777e-08
1387 3.70211979827673e-08
1388 3.68647417595125e-08
1389 3.6786965299962e-08
1390 3.69163455046229e-08
1391 3.6661925532866e-08
1392 3.68543216258121e-08
1393 3.68127328680412e-08
1394 3.68141683004253e-08
1395 3.68085077421254e-08
1396 3.67071528235008e-08
1397 3.68935591985031e-08
1398 3.66122590514939e-08
1399 3.66694211784147e-08
1400 3.65310263724661e-08
1401 3.66279319389262e-08
1402 3.64981576203016e-08
1403 3.64699415433267e-08
1404 3.67245235168845e-08
1405 3.65029932370975e-08
1406 3.66564263549751e-08
1407 3.66455188268722e-08
1408 3.64214779541294e-08
1409 3.65147219616446e-08
1410 3.67536794762535e-08
1411 3.66980817745333e-08
1412 3.63030791232788e-08
1413 3.66292846185612e-08
1414 3.64065608632891e-08
1415 3.63810815168786e-08
1416 3.65551502848893e-08
1417 3.63827365366665e-08
1418 3.68315672601427e-08
1419 3.6253791672769e-08
1420 3.65551811753462e-08
1421 3.63489426132846e-08
1422 3.64301096009711e-08
1423 3.63845785113504e-08
1424 3.63241670645609e-08
1425 3.62743841785806e-08
1426 3.64269679675733e-08
1427 3.62643482531011e-08
1428 3.64093605202953e-08
1429 3.65368469820715e-08
1430 3.64458330048834e-08
1431 3.61357457412947e-08
1432 3.62946508546957e-08
1433 3.61692249251089e-08
1434 3.63927227604033e-08
1435 3.62667733102562e-08
1436 3.6191466375235e-08
1437 3.60597453187284e-08
1438 3.61780122727362e-08
1439 3.63371264393564e-08
1440 3.60288973001444e-08
1441 3.61311922566498e-08
1442 3.6101228826857e-08
1443 3.61737874227663e-08
1444 3.60708859069026e-08
1445 3.60423173657032e-08
1446 3.61806412192967e-08
1447 3.61474959655883e-08
1448 3.61653036710097e-08
1449 3.61737482946212e-08
1450 3.60483866692074e-08
1451 3.60628494541215e-08
1452 3.59418912091458e-08
1453 3.59477794594221e-08
1454 3.60916861414928e-08
1455 3.61096439289721e-08
1456 3.60066770979661e-08
1457 3.60769590888044e-08
1458 3.60448918488854e-08
1459 3.59734010469959e-08
1460 3.5930709048837e-08
1461 3.59758459184678e-08
1462 3.58558317283264e-08
1463 3.60298738923404e-08
1464 3.59392365429922e-08
1465 3.59062993798842e-08
1466 3.60486273298655e-08
1467 3.56902185318364e-08
1468 3.57968897950478e-08
1469 3.59835234380324e-08
1470 3.56674581156735e-08
1471 3.59051779348363e-08
1472 3.57930280444063e-08
1473 3.61311931985631e-08
1474 3.55918575497549e-08
1475 3.56933443580454e-08
1476 3.56426915007479e-08
1477 3.5738645992045e-08
1478 3.57809168338719e-08
1479 3.57045625825059e-08
1480 3.57558388912049e-08
1481 3.56408534146202e-08
1482 3.59447247259559e-08
1483 3.5626910365294e-08
1484 3.57225171889186e-08
1485 3.55112521048251e-08
1486 3.57546030416156e-08
1487 3.56030605532798e-08
1488 3.55800063335243e-08
1489 3.57159678091579e-08
1490 3.5622840438454e-08
1491 3.56112106800199e-08
1492 3.55408875938323e-08
1493 3.55446255211334e-08
1494 3.56145687260834e-08
1495 3.54259949193469e-08
1496 3.54903710649279e-08
1497 3.53642074025218e-08
1498 3.54873146918844e-08
1499 3.54727140838285e-08
1500 3.55334545208419e-08
1501 3.5409026780342e-08
1502 3.54803372437096e-08
1503 3.55096652833176e-08
1504 3.54438842816163e-08
1505 3.54438994277229e-08
1506 3.52586356386353e-08
1507 3.54853364768481e-08
1508 3.52613148048575e-08
1509 3.56022012296675e-08
1510 3.509981280575e-08
1511 3.56337536586149e-08
1512 3.52175760410245e-08
1513 3.53154326817595e-08
1514 3.534399718369e-08
1515 3.51297950942908e-08
1516 3.53575105652149e-08
1517 3.51708020545627e-08
1518 3.53134347643169e-08
1519 3.52938552420912e-08
1520 3.53772712797795e-08
1521 3.54995465590147e-08
1522 3.50098286754363e-08
1523 3.51518915935878e-08
1524 3.50881709478834e-08
1525 3.50589046707039e-08
1526 3.50215923819452e-08
1527 3.53548170237139e-08
1528 3.5026552623274e-08
1529 3.52368347221188e-08
1530 3.50323764222171e-08
1531 3.51654090468001e-08
1532 3.51068960543488e-08
1533 3.51634139128532e-08
1534 3.51571836484377e-08
1535 3.49424351043237e-08
1536 3.50187879449848e-08
1537 3.53500963033437e-08
1538 3.50289750721822e-08
1539 3.48855184874042e-08
1540 3.50088064540177e-08
1541 3.49903093593285e-08
1542 3.50582522964382e-08
1543 3.50921210990895e-08
1544 3.49305118901855e-08
1545 3.48348799145137e-08
1546 3.50040681207364e-08
1547 3.50324450644757e-08
1548 3.47821500198964e-08
1549 3.48002036693606e-08
1550 3.50278476385357e-08
1551 3.48182352611914e-08
1552 3.4843449333799e-08
1553 3.47927909744028e-08
1554 3.46968657146118e-08
1555 3.48270575143417e-08
1556 3.48542658725748e-08
1557 3.47570996516722e-08
1558 3.47461852138742e-08
1559 3.48612541893889e-08
1560 3.47328968683946e-08
1561 3.4757167578614e-08
1562 3.46516574474265e-08
1563 3.49862555948932e-08
1564 3.46321199928834e-08
1565 3.4780982513416e-08
1566 3.46651921784291e-08
1567 3.45752515271136e-08
1568 3.47753969680009e-08
1569 3.48198636946351e-08
1570 3.47757455861908e-08
1571 3.4507271549189e-08
1572 3.47078772566789e-08
1573 3.49576113828354e-08
1574 3.47395628798042e-08
1575 3.45694570675903e-08
1576 3.47136032560202e-08
1577 3.44184528642821e-08
1578 3.45566974758738e-08
1579 3.45180234527787e-08
1580 3.4580202604495e-08
1581 3.45497845699594e-08
1582 3.42231714617336e-08
1583 3.46553255673454e-08
1584 3.43230252612958e-08
1585 3.44667043732372e-08
1586 3.42018570680391e-08
1587 3.42796177371651e-08
1588 3.45346177799133e-08
1589 3.42886514995699e-08
1590 3.41987356230478e-08
1591 3.42419555141582e-08
1592 3.42777873509892e-08
1593 3.43240454155902e-08
1594 3.44707572889069e-08
1595 3.41868560851388e-08
1596 3.42473797330656e-08
1597 3.41525397445919e-08
1598 3.42604636796029e-08
1599 3.42073004447885e-08
1600 3.42863649285263e-08
1601 3.42130756987857e-08
1602 3.4182281366868e-08
1603 3.41949527709051e-08
1604 3.42302469623634e-08
1605 3.43229221849706e-08
1606 3.40175767419293e-08
1607 3.41228139911776e-08
1608 3.39518507423975e-08
1609 3.41163186207138e-08
1610 3.41803030444177e-08
1611 3.39146702968973e-08
1612 3.39262604691637e-08
1613 3.41989028934675e-08
1614 3.40345293181055e-08
1615 3.3939421488216e-08
1616 3.42013139937869e-08
1617 3.40021184344064e-08
1618 3.40370750067098e-08
1619 3.39722234156126e-08
1620 3.39850721161605e-08
1621 3.39282047469025e-08
1622 3.38887960671941e-08
1623 3.38494148963142e-08
1624 3.39835219518303e-08
1625 3.40568993758006e-08
1626 3.37798887363183e-08
1627 3.40222135623014e-08
1628 3.40237355868633e-08
1629 3.3838818874754e-08
1630 3.40310748536687e-08
1631 3.38542083794802e-08
1632 3.38119319162056e-08
1633 3.37274185027159e-08
1634 3.38834484395756e-08
1635 3.37319519347901e-08
1636 3.38475110375214e-08
1637 3.3742468025566e-08
1638 3.38427758715953e-08
1639 3.36246807055574e-08
1640 3.36959911075851e-08
1641 3.37534677092854e-08
1642 3.37967688742169e-08
1643 3.36699845108202e-08
1644 3.37426600273694e-08
1645 3.37344562306718e-08
1646 3.38293949403434e-08
1647 3.37148696983869e-08
1648 3.36567350142092e-08
1649 3.3601624474966e-08
1650 3.37595379983346e-08
1651 3.35222202904051e-08
1652 3.36508149685333e-08
1653 3.36155920830361e-08
1654 3.36637392334138e-08
1655 3.37210855487768e-08
1656 3.36275210927051e-08
1657 3.35611687754533e-08
1658 3.35721064157468e-08
1659 3.34675595425882e-08
1660 3.37159029577538e-08
1661 3.36373670382084e-08
1662 3.35888019271646e-08
1663 3.36828748648821e-08
1664 3.34646427895269e-08
1665 3.35515610040416e-08
1666 3.35947361035371e-08
1667 3.34564912090052e-08
1668 3.34181075747342e-08
1669 3.35241747271286e-08
1670 3.34368204799596e-08
1671 3.34092864074931e-08
1672 3.34194796703935e-08
1673 3.3333014183845e-08
1674 3.35243826823373e-08
1675 3.3302939108637e-08
1676 3.34453444166272e-08
1677 3.33856948302458e-08
1678 3.34887581139309e-08
1679 3.33952367745916e-08
1680 3.35330803781231e-08
1681 3.33581926370563e-08
1682 3.32969732232957e-08
1683 3.32953395532076e-08
1684 3.34571433814324e-08
1685 3.33550652995562e-08
1686 3.32937111034992e-08
1687 3.32506361424567e-08
1688 3.33360527959847e-08
1689 3.32355419143671e-08
1690 3.33320998507913e-08
1691 3.32850410757479e-08
1692 3.32285877008287e-08
1693 3.36223233193267e-08
1694 3.30572607050161e-08
1695 3.33324064248908e-08
1696 3.32067269875891e-08
1697 3.3382993633535e-08
1698 3.29964895396939e-08
1699 3.3353039733508e-08
1700 3.32527547346473e-08
1701 3.32842838860481e-08
1702 3.3211624715368e-08
1703 3.30948041158408e-08
1704 3.32192775527318e-08
1705 3.32964941330838e-08
1706 3.31867500892313e-08
1707 3.33129591025827e-08
1708 3.3169680643208e-08
1709 3.30380156989829e-08
1710 3.32138991709918e-08
1711 3.31168547595961e-08
1712 3.29955022294604e-08
1713 3.31696833573147e-08
1714 3.31071103822911e-08
1715 3.29911585382758e-08
1716 3.30640230706836e-08
1717 3.31011232377332e-08
1718 3.30119848857002e-08
1719 3.32400608088479e-08
1720 3.29399066828495e-08
1721 3.30989357427702e-08
1722 3.31842004653859e-08
1723 3.29844918106059e-08
1724 3.30439799620663e-08
1725 3.29636689391233e-08
1726 3.29760576840421e-08
1727 3.31259523160821e-08
1728 3.29099122855503e-08
1729 3.31252851235586e-08
1730 3.29565339626248e-08
1731 3.2933619026787e-08
1732 3.29731152904911e-08
1733 3.2967441671361e-08
1734 3.29583114501153e-08
1735 3.29571830928188e-08
1736 3.294992565267e-08
1737 3.28714499581162e-08
1738 3.29391727102446e-08
1739 3.30046044072496e-08
1740 3.28930657988447e-08
1741 3.27286026597373e-08
1742 3.29633379357874e-08
1743 3.30884039795865e-08
1744 3.29163909054686e-08
1745 3.28890617712352e-08
1746 3.28505852125738e-08
1747 3.29173914941916e-08
1748 3.27714658130418e-08
1749 3.28818424779609e-08
1750 3.29466758246522e-08
1751 3.28881627214006e-08
1752 3.29665107388077e-08
1753 3.27789897298092e-08
1754 3.28292318407808e-08
1755 3.29639002245585e-08
1756 3.27062794595045e-08
1757 3.28777669305058e-08
1758 3.28838672842835e-08
1759 3.27090928905482e-08
1760 3.27078591643715e-08
1761 3.27427905351119e-08
1762 3.2742562721122e-08
1763 3.26971729969205e-08
1764 3.28467459483894e-08
1765 3.27235642829105e-08
1766 3.27576554469133e-08
1767 3.28746474431152e-08
1768 3.27093816700486e-08
1769 3.26207805004808e-08
1770 3.27084321883886e-08
1771 3.26689043277373e-08
1772 3.25940506983313e-08
1773 3.25382356093362e-08
1774 3.27274942943379e-08
1775 3.2649802828022e-08
1776 3.25513969394176e-08
1777 3.26714601275313e-08
1778 3.26417605264195e-08
1779 3.24679085978996e-08
1780 3.26437137586066e-08
1781 3.26677886172688e-08
1782 3.26053935369996e-08
1783 3.25618422603036e-08
1784 3.25413010623943e-08
1785 3.26730452121504e-08
1786 3.23643237255533e-08
1787 3.24920763569714e-08
1788 3.25881814949858e-08
1789 3.26288977586797e-08
1790 3.26260023874037e-08
1791 3.24209111122187e-08
1792 3.24541659781352e-08
1793 3.26103605466366e-08
1794 3.25891861961969e-08
1795 3.24869675866757e-08
1796 3.24133952375738e-08
1797 3.26027567005838e-08
1798 3.24327672510116e-08
1799 3.24082111111124e-08
1800 3.24791570279204e-08
1801 3.24894469376225e-08
1802 3.24914168143842e-08
1803 3.2498659108815e-08
1804 3.23716170825272e-08
1805 3.25123986862352e-08
1806 3.25158265228609e-08
1807 3.24504894727573e-08
1808 3.22844443543802e-08
1809 3.25390335000342e-08
1810 3.24773551020896e-08
1811 3.24574931713228e-08
1812 3.26283847812414e-08
1813 3.25419206180233e-08
1814 3.22390805437278e-08
1815 3.23777570677142e-08
1816 3.23154311702711e-08
1817 3.23766837164174e-08
1818 3.23822722189115e-08
1819 3.24803485940439e-08
1820 3.23273785041156e-08
1821 3.24866097815013e-08
1822 3.23613387902544e-08
1823 3.23865933614664e-08
1824 3.23740548756613e-08
1825 3.22023771703872e-08
1826 3.22628976631711e-08
1827 3.24329326433226e-08
1828 3.22848744794268e-08
1829 3.2333048819666e-08
1830 3.21447192484503e-08
1831 3.22492441485189e-08
1832 3.23439771298673e-08
1833 3.22016925594082e-08
1834 3.22413180942616e-08
1835 3.22508212050043e-08
1836 3.23331717683728e-08
1837 3.22526863833494e-08
1838 3.22049099903965e-08
1839 3.22260313751488e-08
1840 3.22660938885666e-08
1841 3.22769188986771e-08
1842 3.21834319502168e-08
1843 3.21989121608302e-08
1844 3.21279421234277e-08
1845 3.22666971560026e-08
1846 3.20395286486641e-08
1847 3.22519133347798e-08
1848 3.21164050468559e-08
1849 3.20826890799486e-08
1850 3.22139288879697e-08
1851 3.20971465407882e-08
1852 3.22144581577644e-08
1853 3.19576047081993e-08
1854 3.22492194267943e-08
1855 3.2001513273372e-08
1856 3.20688977735184e-08
1857 3.21258851190276e-08
1858 3.19568626853672e-08
1859 3.21142625784865e-08
1860 3.21689443721906e-08
1861 3.22674206943985e-08
1862 3.20655194468999e-08
1863 3.19210747238285e-08
1864 3.225900338516e-08
1865 3.20700552524644e-08
1866 3.19831241844537e-08
1867 3.19343602522282e-08
1868 3.21790412932121e-08
1869 3.1888733194374e-08
1870 3.20019269745009e-08
1871 3.19844209887288e-08
1872 3.1916823753253e-08
1873 3.19012913799766e-08
1874 3.18707387482187e-08
1875 3.19547724227376e-08
1876 3.19718961688809e-08
1877 3.18135722976454e-08
1878 3.20486251085827e-08
1879 3.19182358901604e-08
1880 3.1802250702706e-08
1881 3.1912665152678e-08
1882 3.20600555783201e-08
1883 3.1876967291955e-08
1884 3.19136879050053e-08
1885 3.17703411333303e-08
1886 3.18145031775741e-08
1887 3.19690884297286e-08
1888 3.17968731653462e-08
1889 3.17749977128967e-08
1890 3.18037000409666e-08
1891 3.19548389934865e-08
1892 3.21812974408142e-08
1893 3.19703307075092e-08
1894 3.18700482752066e-08
1895 3.1767971117258e-08
1896 3.17770810412954e-08
1897 3.18876867773099e-08
1898 3.17295194213196e-08
1899 3.17485751455404e-08
1900 3.19865055798396e-08
1901 3.18115280090736e-08
1902 3.16908150180817e-08
1903 3.16812669847732e-08
1904 3.18243754039438e-08
1905 3.15680841989074e-08
1906 3.17794767200064e-08
1907 3.17023454823318e-08
1908 3.18693976453233e-08
1909 3.18415298098951e-08
1910 3.16212344529943e-08
1911 3.15152333014157e-08
1912 3.18137065316027e-08
1913 3.16121963152005e-08
1914 3.18365857246428e-08
1915 3.15445727220864e-08
1916 3.19133697208041e-08
1917 3.16703839357557e-08
1918 3.1650899292468e-08
1919 3.1760027935257e-08
1920 3.14671925175092e-08
1921 3.1500259185524e-08
1922 3.18302700934581e-08
1923 3.1607784766241e-08
1924 3.16657346015048e-08
1925 3.17786208388626e-08
1926 3.17326468079471e-08
1927 3.15705208227546e-08
1928 3.14717073418569e-08
1929 3.16227015642045e-08
1930 3.17144774631961e-08
1931 3.14909882969672e-08
1932 3.15732886108644e-08
1933 3.15512149867692e-08
1934 3.13949121090173e-08
1935 3.16296184100784e-08
1936 3.14198480395045e-08
1937 3.15226660778101e-08
1938 3.17126843175641e-08
1939 3.14657064188761e-08
1940 3.15932768773508e-08
1941 3.1443393735342e-08
1942 3.16141726211527e-08
1943 3.15122500015308e-08
1944 3.14652990107134e-08
1945 3.13788911216473e-08
1946 3.15152149659159e-08
1947 3.14730961917253e-08
1948 3.15698774646656e-08
1949 3.1550271654357e-08
1950 3.13441771770395e-08
1951 3.16197460578649e-08
1952 3.14996658703492e-08
1953 3.13541585869603e-08
1954 3.14610052112307e-08
1955 3.14312879078349e-08
1956 3.14704378882036e-08
1957 3.13241971531819e-08
1958 3.14248889605184e-08
1959 3.14207973276526e-08
1960 3.14065819677078e-08
1961 3.15407988204508e-08
1962 3.13036407677547e-08
1963 3.15315154483797e-08
1964 3.13164938139932e-08
1965 3.14168198956022e-08
1966 3.13689630251912e-08
1967 3.1467224094861e-08
1968 3.1286993568469e-08
1969 3.14485542282639e-08
1970 3.14030466217474e-08
1971 3.14314477946054e-08
1972 3.13443898842802e-08
1973 3.13606572237557e-08
1974 3.11702090874388e-08
1975 3.12693217240922e-08
1976 3.14604242830918e-08
1977 3.1121771425946e-08
1978 3.1322542874801e-08
1979 3.12161918926135e-08
1980 3.11495583838473e-08
1981 3.11583331053522e-08
1982 3.13254005013808e-08
1983 3.12729520904886e-08
1984 3.11461744881281e-08
1985 3.13323024260792e-08
1986 3.13907108118183e-08
1987 3.13655652386946e-08
1988 3.11655180309511e-08
1989 3.1232479092036e-08
1990 3.12728004174256e-08
1991 3.12459387827313e-08
1992 3.1174769391018e-08
1993 3.11907300250547e-08
1994 3.12891360435552e-08
1995 3.12464769737231e-08
1996 3.10742082554327e-08
1997 3.12473689971227e-08
1998 3.11899032888752e-08
1999 3.11680944086179e-08
2000 3.1141825201586e-08
2001 3.12114750503634e-08
2002 3.11833326198663e-08
2003 3.12268478230848e-08
2004 3.1243899399569e-08
2005 3.12468263996002e-08
2006 3.10603575045532e-08
2007 3.10653063969601e-08
2008 3.1099255507594e-08
2009 3.10287037288415e-08
2010 3.10913486818998e-08
2011 3.12361349462109e-08
2012 3.11044983772324e-08
2013 3.11007808574404e-08
2014 3.09466392803825e-08
2015 3.10259497358079e-08
2016 3.08938892823463e-08
2017 3.10815462466474e-08
2018 3.0910124273642e-08
2019 3.10034297373862e-08
2020 3.09633540489518e-08
2021 3.07705914774914e-08
2022 3.08049865301863e-08
2023 3.08481227213053e-08
2024 3.08194594452749e-08
2025 3.0798884199601e-08
2026 3.08229509665137e-08
2027 3.07917427186943e-08
2028 3.07887233217397e-08
2029 3.06908348912116e-08
2030 3.08258532973338e-08
2031 3.08507094164967e-08
2032 3.05987606622482e-08
2033 3.06995983832548e-08
2034 3.06284207655105e-08
2035 3.11544920770235e-08
2036 3.08567936838089e-08
2037 3.08853927087349e-08
2038 3.09116344232585e-08
2039 3.07694246246548e-08
2040 3.0770290553539e-08
2041 3.06329194651456e-08
2042 3.0735486348743e-08
2043 3.06941076504419e-08
2044 3.07786228317952e-08
2045 3.05916976109932e-08
2046 3.06041222019049e-08
2047 3.03975880185381e-08
2048 3.06741377441022e-08
2049 3.04118579695922e-08
2050 3.06461325317864e-08
2051 3.02404866541761e-08
2052 3.03093538310262e-08
2053 3.04301779117111e-08
2054 3.02446610165319e-08
2055 3.02679929634619e-08
2056 3.02352103317416e-08
2057 3.02331172425951e-08
2058 3.0128775501681e-08
2059 3.02011595891827e-08
2060 3.0007549780664e-08
2061 3.00848353176897e-08
2062 2.98747668335819e-08
2063 3.00316643550569e-08
2064 3.0117707158106e-08
2065 2.99687146976257e-08
2066 2.98156149807771e-08
2067 2.98984092579335e-08
2068 2.9785586031672e-08
2069 2.97338539848035e-08
2070 2.97417253360965e-08
2071 2.98349979198087e-08
2072 2.97037669018119e-08
2073 2.97324748397276e-08
2074 2.98842761720652e-08
2075 2.95087293873952e-08
2076 2.9793110981724e-08
2077 2.94817493307065e-08
2078 2.9623437378945e-08
2079 2.94992458614307e-08
2080 2.93395228469495e-08
2081 2.95644886328383e-08
2082 2.95037340061755e-08
2083 2.93686518565428e-08
2084 2.93700670015995e-08
2085 2.93561102789885e-08
2086 2.93984716253637e-08
2087 2.93142320599293e-08
2088 2.94468056700747e-08
2089 2.93131497905996e-08
2090 2.94773077634813e-08
2091 2.937108485912e-08
2092 2.92484716676866e-08
2093 2.89886704768483e-08
2094 2.92568424740125e-08
2095 2.92341266436846e-08
2096 2.91217434948421e-08
2097 2.92166745153866e-08
2098 2.89551600186622e-08
2099 2.90605456158555e-08
2100 2.91152608135059e-08
2101 2.88841942701068e-08
2102 2.91059553458872e-08
2103 2.91119298919673e-08
2104 2.90696666399581e-08
2105 2.90093629484733e-08
2106 2.90381459653322e-08
2107 2.90745290363326e-08
2108 2.89264128013889e-08
2109 2.88948745336137e-08
2110 2.90397824622701e-08
2111 2.90415643817155e-08
2112 2.88822297421221e-08
2113 2.89976527343416e-08
2114 2.86938413039395e-08
2115 2.89633103322529e-08
2116 2.87821006728084e-08
2117 2.88976436754185e-08
2118 2.88603208842275e-08
2119 2.88596850510636e-08
2120 2.88746302158915e-08
2121 2.88181110194019e-08
2122 2.88853125290922e-08
2123 2.85808219244732e-08
2124 2.90243188461603e-08
2125 2.88172930941188e-08
2126 2.87400682771333e-08
2127 2.8880185860114e-08
2128 2.87882850022458e-08
2129 2.87523249073995e-08
2130 2.87639309308751e-08
2131 2.87207743725482e-08
2132 2.85770675890751e-08
2133 2.87604952308174e-08
2134 2.87696307069707e-08
2135 2.86701969300918e-08
2136 2.87404659693458e-08
2137 2.8683869157331e-08
2138 2.8443150920654e-08
2139 2.87520656896412e-08
2140 2.87211186856262e-08
2141 2.86254891624371e-08
2142 2.84823430793946e-08
2143 2.86302661858251e-08
2144 2.85750332014723e-08
2145 2.84325688150733e-08
2146 2.86569700660988e-08
2147 2.87193292580201e-08
2148 2.84673748816022e-08
2149 2.84479925056758e-08
2150 2.8520896063311e-08
2151 2.87009661593118e-08
2152 2.83243046084736e-08
2153 2.87693030022185e-08
2154 2.83070180850387e-08
2155 2.84467006402833e-08
2156 2.85311689540157e-08
2157 2.85636943747614e-08
2158 2.83576655114026e-08
2159 2.84039773241207e-08
2160 2.85006532977361e-08
2161 2.82637110415873e-08
2162 2.84873636483551e-08
2163 2.84144913008655e-08
2164 2.85237408383932e-08
2165 2.83759064276801e-08
2166 2.85662780042384e-08
2167 2.83450627757142e-08
2168 2.84098236272845e-08
2169 2.84011668292283e-08
2170 2.82374178718348e-08
2171 2.83200894874991e-08
2172 2.85092011100829e-08
2173 2.83604326167253e-08
2174 2.83592399264454e-08
2175 2.82656590931407e-08
2176 2.82783226583372e-08
2177 2.82179511477132e-08
2178 2.82270159823739e-08
2179 2.81188971091306e-08
2180 2.82934509213684e-08
2181 2.83082078013641e-08
2182 2.82810304326198e-08
2183 2.83714839244831e-08
2184 2.81412104484735e-08
2185 2.82804683485738e-08
2186 2.85097599458939e-08
2187 2.83637902004363e-08
2188 2.82174344556352e-08
2189 2.81769104051866e-08
2190 2.82165920606481e-08
2191 2.82312550822228e-08
2192 2.83485769805303e-08
2193 2.81632272965404e-08
2194 2.81590176106072e-08
2195 2.81641059212134e-08
2196 2.82577300941833e-08
2197 2.81721537987445e-08
2198 2.83347862771177e-08
2199 2.81332217080821e-08
2200 2.80960597620061e-08
2201 2.81816518465927e-08
2202 2.81848878621038e-08
2203 2.80455250479816e-08
2204 2.81257446292926e-08
2205 2.79430988607277e-08
2206 2.80939963687721e-08
2207 2.81725907032571e-08
2208 2.80281667569549e-08
2209 2.81091520826227e-08
2210 2.81403308689532e-08
2211 2.81926626897189e-08
2212 2.80349455075712e-08
2213 2.80470274648947e-08
2214 2.79965266455906e-08
2215 2.82325374742487e-08
2216 2.81503143659823e-08
2217 2.80674220280441e-08
2218 2.79080505989904e-08
2219 2.8217803777375e-08
2220 2.79117819860786e-08
2221 2.80878702083598e-08
2222 2.79001914723076e-08
2223 2.7753500663863e-08
2224 2.80517956558479e-08
2225 2.80929400464647e-08
2226 2.78888162720814e-08
2227 2.79714189495572e-08
2228 2.79450240299739e-08
2229 2.80078720314436e-08
2230 2.78237194925035e-08
2231 2.81363657145262e-08
2232 2.79624077814677e-08
2233 2.75638344032214e-08
2234 2.78783446319708e-08
2235 2.77442884835666e-08
2236 2.7791697375712e-08
2237 2.77764855464713e-08
2238 2.76901487797909e-08
2239 2.77148007098993e-08
2240 2.78679769604717e-08
2241 2.78787188556229e-08
2242 2.76911629885634e-08
2243 2.77044135057469e-08
2244 2.76687854712798e-08
2245 2.76150799576325e-08
2246 2.8107940509503e-08
2247 2.784664981903e-08
2248 2.78624093764668e-08
2249 2.77034292535672e-08
2250 2.77434805624477e-08
2251 2.76187254488747e-08
2252 2.80042804580161e-08
2253 2.77298626837896e-08
2254 2.77404206531218e-08
2255 2.77065382016106e-08
2256 2.75256017164827e-08
2257 2.74629385354497e-08
2258 2.75618577095238e-08
2259 2.75727999160535e-08
2260 2.77150455156305e-08
2261 2.74181509739013e-08
2262 2.77797533775881e-08
2263 2.74279192256088e-08
2264 2.74156893949051e-08
2265 2.76811647131336e-08
2266 2.76534110356108e-08
2267 2.74904401915133e-08
2268 2.79117870366497e-08
2269 2.74123948122984e-08
2270 2.74753062155519e-08
2271 2.76965761695225e-08
2272 2.75416573278231e-08
2273 2.74593727190853e-08
2274 2.75487950876507e-08
2275 2.75405033715592e-08
2276 2.74526957492194e-08
2277 2.74007147967326e-08
2278 2.75566595775989e-08
2279 2.76396843518767e-08
2280 2.7284466198374e-08
2281 2.75787044954345e-08
2282 2.73148999526129e-08
2283 2.73982418769747e-08
2284 2.76707067479176e-08
2285 2.75655295928767e-08
2286 2.73729985399429e-08
2287 2.73162462773868e-08
2288 2.72726202391049e-08
2289 2.75478820131458e-08
2290 2.72448585816321e-08
2291 2.74667414407181e-08
2292 2.73681332507714e-08
2293 2.74075114678474e-08
2294 2.72055017765949e-08
2295 2.73214537540034e-08
2296 2.71916536462857e-08
2297 2.74837481770707e-08
2298 2.71613782623636e-08
2299 2.71493945752654e-08
2300 2.70657676154085e-08
2301 2.75041724322467e-08
2302 2.72390461553695e-08
2303 2.73672820426674e-08
2304 2.72036402202414e-08
2305 2.73739726358579e-08
2306 2.70225030896132e-08
2307 2.73021468470414e-08
2308 2.74597508000962e-08
2309 2.71504954206248e-08
2310 2.69813136619113e-08
2311 2.74732334399341e-08
2312 2.70260445958126e-08
2313 2.72452419976532e-08
2314 2.72712560276944e-08
2315 2.71563999844626e-08
2316 2.69382242565897e-08
2317 2.73337592224254e-08
2318 2.71649655779749e-08
2319 2.72537714895993e-08
2320 2.67999947234365e-08
2321 2.72218455678042e-08
2322 2.70087710201872e-08
2323 2.68581968150272e-08
2324 2.69072167775608e-08
2325 2.69125762906719e-08
2326 2.69429441958069e-08
2327 2.71218184903499e-08
2328 2.68156149664245e-08
2329 2.67376088771698e-08
2330 2.68203926537813e-08
2331 2.69718827458632e-08
2332 2.6891238692317e-08
2333 2.67623734263589e-08
2334 2.69493190674375e-08
2335 2.70750335312764e-08
2336 2.71823019047934e-08
2337 2.68549484279124e-08
2338 2.64484143829291e-08
2339 2.68895768810173e-08
2340 2.71908919151076e-08
2341 2.66353979994083e-08
2342 2.67876414504764e-08
2343 2.67778926467122e-08
2344 2.68572851197524e-08
2345 2.66753379839502e-08
2346 2.66638636640115e-08
2347 2.67707806984041e-08
2348 2.67593289431378e-08
2349 2.65844222191447e-08
2350 2.68055574759729e-08
2351 2.68939725039941e-08
2352 2.66254969395474e-08
2353 2.67511989283098e-08
2354 2.64753723443478e-08
2355 2.69067399348843e-08
2356 2.65509187585278e-08
2357 2.64628765242469e-08
2358 2.671681020322e-08
2359 2.65206275546492e-08
2360 2.63849329499299e-08
2361 2.67899979213837e-08
2362 2.64669895163605e-08
2363 2.64410616060973e-08
2364 2.63288421390451e-08
2365 2.6835559842231e-08
2366 2.64452923502967e-08
2367 2.63383568431808e-08
2368 2.62932330459265e-08
2369 2.65512091473519e-08
2370 2.60378024751762e-08
2371 2.65010926117082e-08
2372 2.62417164202944e-08
2373 2.63448761931295e-08
2374 2.67367775445004e-08
2375 2.62594408427241e-08
2376 2.64982488256682e-08
2377 2.63495153218218e-08
2378 2.63498554136654e-08
2379 2.62722632677903e-08
2380 2.61911816658023e-08
2381 2.60767960098551e-08
2382 2.63970491051202e-08
2383 2.58503263143584e-08
2384 2.65093357366686e-08
2385 2.59839602409495e-08
2386 2.63933272312e-08
2387 2.58901651171151e-08
2388 2.63699719618704e-08
2389 2.60247775151767e-08
2390 2.60361099292949e-08
2391 2.6124992739851e-08
2392 2.59119032025579e-08
2393 2.61000070272965e-08
2394 2.58950009518477e-08
2395 2.61446871605009e-08
2396 2.6388116967202e-08
2397 2.59870085877467e-08
2398 2.60919525510439e-08
2399 2.56847685944361e-08
2400 2.59955212293383e-08
2401 2.62403433153868e-08
2402 2.58295529177444e-08
2403 2.61232096630537e-08
2404 2.57108753542457e-08
2405 2.62916766876065e-08
2406 2.58992670283398e-08
2407 2.56961081577245e-08
2408 2.59262695966322e-08
2409 2.57733248840153e-08
2410 2.61730104297864e-08
2411 2.62498780017051e-08
2412 2.57292127138764e-08
2413 2.58732551315921e-08
2414 2.59459525693839e-08
2415 2.59116994496544e-08
2416 2.57704587377505e-08
2417 2.5810997233866e-08
2418 2.60466090178935e-08
2419 2.59323644831722e-08
2420 2.56050902190386e-08
2421 2.59200030226503e-08
2422 2.58222657447682e-08
2423 2.57422415032571e-08
2424 2.58569474143044e-08
2425 2.59948852117109e-08
2426 2.55991675110478e-08
2427 2.60799623752472e-08
2428 2.57543883603328e-08
2429 2.60146298837194e-08
2430 2.58217541256323e-08
2431 2.5824920944606e-08
2432 2.59697545961224e-08
2433 2.56513986417461e-08
2434 2.61871964443716e-08
2435 2.56115083968611e-08
2436 2.58546923010416e-08
2437 2.59762558231236e-08
2438 2.57588050454061e-08
2439 2.59961585876711e-08
2440 2.55404731401843e-08
2441 2.62563394666793e-08
2442 2.58312579287656e-08
2443 2.58760886555631e-08
2444 2.58239903698221e-08
2445 2.55024165568551e-08
2446 2.58646765881054e-08
2447 2.56382502770158e-08
2448 2.5959289819466e-08
2449 2.56434067728573e-08
2450 2.57972115480509e-08
2451 2.57494277798864e-08
2452 2.56344585527479e-08
2453 2.58640764358997e-08
2454 2.57094829358429e-08
2455 2.57415757824453e-08
2456 2.56383989131748e-08
2457 2.56081052064094e-08
2458 2.57148362056703e-08
2459 2.56645881738238e-08
2460 2.58453002466852e-08
2461 2.5681047932824e-08
2462 2.54455356013539e-08
2463 2.56472550492215e-08
2464 2.56583760824269e-08
2465 2.57860806567312e-08
2466 2.568821718818e-08
2467 2.57879121329241e-08
2468 2.54519757413307e-08
2469 2.58021475035286e-08
2470 2.56313535881292e-08
2471 2.57153351024275e-08
2472 2.5651701927587e-08
2473 2.55432614715279e-08
2474 2.55526575046461e-08
2475 2.5877087356907e-08
2476 2.56142908365287e-08
2477 2.56008580630795e-08
2478 2.5699534749013e-08
2479 2.56968930597012e-08
2480 2.54677525396985e-08
2481 2.56950539696543e-08
2482 2.57176781590096e-08
2483 2.55371142806049e-08
2484 2.58579683481419e-08
2485 2.54377771901071e-08
2486 2.57100947660405e-08
2487 2.5386835335206e-08
2488 2.57077416881968e-08
2489 2.56053195833994e-08
2490 2.5543725387378e-08
2491 2.58108828410375e-08
2492 2.55590901825364e-08
2493 2.56749662443378e-08
2494 2.54444623219996e-08
2495 2.5308176014649e-08
2496 2.57641427531263e-08
2497 2.52910653913418e-08
2498 2.55799756492259e-08
2499 2.54964276820147e-08
2500 2.55938530180355e-08
2501 2.53986463427114e-08
2502 2.57035843662656e-08
2503 2.5485912755141e-08
2504 2.52461609301924e-08
2505 2.55609956749114e-08
2506 2.5203930547002e-08
2507 2.60198541561507e-08
2508 2.51563590796255e-08
2509 2.56666948582618e-08
2510 2.53878843449185e-08
2511 2.5524186992143e-08
2512 2.54317125102932e-08
2513 2.53063409375864e-08
2514 2.56665926434696e-08
2515 2.52823295699978e-08
2516 2.56281197646147e-08
2517 2.5335225195211e-08
2518 2.54053805253118e-08
2519 2.55028791674716e-08
2520 2.52696215010673e-08
2521 2.54318929323039e-08
2522 2.53925204548311e-08
2523 2.54898152967087e-08
2524 2.51996063052595e-08
2525 2.54825902193945e-08
2526 2.53172368783472e-08
2527 2.53348651256813e-08
2528 2.54369468705162e-08
2529 2.53326566811984e-08
2530 2.5193202645879e-08
2531 2.53412297172573e-08
2532 2.54461310169041e-08
2533 2.52883710182283e-08
2534 2.54385213213637e-08
2535 2.51784613093109e-08
2536 2.54970660465981e-08
2537 2.54193201558173e-08
2538 2.51257069131539e-08
2539 2.56703921365231e-08
2540 2.52282948193794e-08
2541 2.5277340455182e-08
2542 2.52956146673533e-08
2543 2.54068277316666e-08
2544 2.54142813951863e-08
2545 2.53757503270169e-08
2546 2.53271774368069e-08
2547 2.51724801613173e-08
2548 2.52173648683796e-08
2549 2.53503789414133e-08
2550 2.56233863776822e-08
2551 2.52027122863985e-08
2552 2.67651193383844e-08
2553 2.53071639082503e-08
2554 2.52250666391562e-08
2555 2.52287176726318e-08
2556 2.51615501534186e-08
2557 2.51754363691237e-08
2558 2.52650080577732e-08
2559 2.52047834537472e-08
2560 2.50551118720077e-08
2561 2.53908913839807e-08
2562 2.51072887965664e-08
2563 2.5180725370777e-08
2564 2.50764410337778e-08
2565 2.55751141157168e-08
2566 2.51823331782752e-08
2567 2.50993306487035e-08
2568 2.52178780812407e-08
2569 2.52570891777215e-08
2570 2.52047310875803e-08
2571 2.52153075693262e-08
2572 2.4975612876621e-08
2573 2.52732610724338e-08
2574 2.50366580593075e-08
2575 2.52758612046544e-08
2576 2.49990857856819e-08
2577 2.51710591102072e-08
2578 2.50809658411399e-08
2579 2.52312801273247e-08
2580 2.50930512400993e-08
2581 2.51404365772534e-08
2582 2.49710873584941e-08
2583 2.51308541416306e-08
2584 2.5120587877181e-08
2585 2.51324970780287e-08
2586 2.49474458782761e-08
2587 2.5164461941285e-08
2588 2.49202684803884e-08
2589 2.53785438781851e-08
2590 2.50576682644388e-08
2591 2.5040686949751e-08
2592 2.49910094622385e-08
2593 2.50912760518251e-08
2594 2.49732595201113e-08
2595 2.49726893593871e-08
2596 2.48765664391803e-08
2597 2.66024552114108e-08
2598 2.4885685754289e-08
2599 2.47855819498488e-08
2600 2.49050387198246e-08
2601 2.50764980774809e-08
2602 2.48733177842519e-08
2603 2.51143863544412e-08
2604 2.47311451954113e-08
2605 2.64271752157819e-08
2606 2.48654185729391e-08
2607 2.49349789724351e-08
2608 2.49027868028384e-08
2609 2.47576566705932e-08
2610 2.47900945419544e-08
2611 2.53085713809165e-08
2612 2.48587338153028e-08
2613 2.50715089846754e-08
2614 2.48899176791273e-08
2615 2.51381627885405e-08
2616 2.46404866871552e-08
2617 2.50890648625379e-08
2618 2.48524800026684e-08
2619 2.49707291020695e-08
2620 2.47590877648363e-08
2621 2.5003308205318e-08
2622 2.46363700309726e-08
2623 2.50666462729976e-08
2624 2.49075541654542e-08
2625 2.47905617206667e-08
2626 2.46594732969951e-08
2627 2.48809981557852e-08
2628 2.46415427364588e-08
2629 2.66178159518959e-08
2630 2.4662049568358e-08
2631 2.48226545515595e-08
2632 2.45699177099934e-08
2633 2.48499178657213e-08
2634 2.48336574992236e-08
2635 2.47933798980804e-08
2636 2.46871136053706e-08
2637 2.47282066413446e-08
2638 2.47496504456879e-08
2639 2.48065303353195e-08
2640 2.47677027416793e-08
2641 2.49868708769307e-08
2642 2.6469345701885e-08
2643 2.46643451393691e-08
2644 2.47495532477715e-08
2645 2.4734172838714e-08
2646 2.47781883313536e-08
2647 2.46257175644526e-08
2648 2.48755915527865e-08
2649 2.46769337509667e-08
2650 2.6450117796839e-08
2651 2.46906257666546e-08
2652 2.45795968303875e-08
2653 2.46632111112755e-08
2654 2.46417420212697e-08
2655 2.4676113525246e-08
2656 2.46951156688824e-08
2657 2.47193443617122e-08
2658 2.48528569778417e-08
2659 2.45064780983739e-08
2660 2.47335475564991e-08
2661 2.4633206694713e-08
2662 2.46570264145762e-08
2663 2.49158657505411e-08
2664 2.45791144584118e-08
2665 2.47635576788863e-08
2666 2.45469968966905e-08
2667 2.47070416706796e-08
2668 2.49504090155672e-08
2669 2.45037627530364e-08
2670 2.48665354896649e-08
2671 2.63715149815269e-08
2672 2.4570424254855e-08
2673 2.47016415894086e-08
2674 2.47112375723058e-08
2675 2.64119623519798e-08
2676 2.45776119711105e-08
2677 2.45163648259772e-08
2678 2.46132134592569e-08
2679 2.46154079708294e-08
2680 2.45302852884821e-08
2681 2.64456265926527e-08
2682 2.44834474151245e-08
2683 2.42845494750066e-08
2684 2.45706002527379e-08
2685 2.46466175719551e-08
2686 2.63721000177686e-08
2687 2.44402859903015e-08
2688 2.43836821821608e-08
2689 2.45833905411774e-08
2690 2.47048173062092e-08
2691 2.65448324179296e-08
2692 2.43192588306185e-08
2693 2.44654997759164e-08
2694 2.61797277858467e-08
2695 2.4457144344181e-08
2696 2.4371425287939e-08
2697 2.62568189638435e-08
2698 2.43205571386351e-08
2699 2.42632422892131e-08
2700 2.44074811006079e-08
2701 2.74268386168952e-08
2702 2.38105807319755e-08
2703 2.45020811940333e-08
2704 2.72241303735932e-08
2705 2.37963307826439e-08
2706 2.42512086809921e-08
2707 2.45086419500273e-08
2708 2.72703695279874e-08
2709 2.49234850730473e-08
2710 2.41447734624645e-08
2711 2.44792128306637e-08
2712 2.61169972569086e-08
2713 2.42263703494339e-08
2714 2.4208431539996e-08
2715 2.44357604546885e-08
2716 2.63538456185541e-08
2717 2.42326673886573e-08
2718 2.43660343728913e-08
2719 2.426724232929e-08
2720 2.59506631047346e-08
2721 2.41312842764185e-08
2722 2.42115054375769e-08
2723 2.61002105904629e-08
2724 2.41580052820489e-08
2725 2.43706112557085e-08
2726 2.61765535122249e-08
2727 2.42228172858683e-08
2728 2.43515386793569e-08
2729 2.43316412300576e-08
2730 2.6323091096403e-08
2731 2.40153707683488e-08
2732 2.43325664938676e-08
2733 2.62181364356073e-08
2734 2.40712695672163e-08
2735 2.42797893698077e-08
2736 2.5997656709853e-08
2737 2.41303063030496e-08
2738 2.40899568952346e-08
2739 2.59472848927467e-08
2740 2.40154869309839e-08
2741 2.43858734345626e-08
2742 2.60914976769611e-08
2743 2.40940599745554e-08
2744 2.41831417785443e-08
2745 2.4381769519688e-08
2746 2.69759275102199e-08
2747 2.36414424222842e-08
2748 2.4235993123678e-08
2749 2.62410082416675e-08
2750 2.40293373679479e-08
2751 2.40339116221455e-08
2752 2.6093894667123e-08
2753 2.4064785852207e-08
2754 2.429957019634e-08
2755 2.62071100470829e-08
2756 2.39368909459303e-08
2757 2.41430302949097e-08
2758 2.61358483515384e-08
2759 2.38512074776098e-08
2760 2.4181793992939e-08
2761 2.59842877083916e-08
2762 2.39701214681531e-08
2763 2.41938700876188e-08
2764 2.62449710676238e-08
2765 2.38497354608791e-08
2766 2.5735916095726e-08
2767 2.41777149576627e-08
2768 2.42085684415971e-08
2769 2.604829607461e-08
2770 2.37686271294391e-08
2771 2.40964224129669e-08
2772 2.59209175210717e-08
2773 2.37172568175459e-08
2774 2.5657989424277e-08
2775 2.4135051021279e-08
2776 2.41200615044845e-08
2777 2.60902589227507e-08
2778 2.38883555125358e-08
2779 2.551460375394e-08
2780 2.39125368530213e-08
2781 2.56403937143324e-08
2782 2.39248128093283e-08
2783 2.56156840138244e-08
2784 2.3887733001382e-08
2785 2.55950873226385e-08
2786 2.41195813266981e-08
2787 2.5613936536828e-08
2788 2.38278519177904e-08
2789 2.3919953026752e-08
2790 2.58915047156583e-08
2791 2.37893740785977e-08
2792 2.54479991106393e-08
2793 2.38857482591226e-08
2794 2.57332568658342e-08
2795 2.38718583520092e-08
2796 2.5623416297027e-08
2797 2.40508752151203e-08
2798 2.5686652043666e-08
2799 2.38760822461881e-08
2800 2.56088148887779e-08
2801 2.38490847401796e-08
2802 2.56024735714733e-08
2803 2.37204599314311e-08
2804 2.53691261775657e-08
2805 2.37832656500014e-08
2806 2.58016057918531e-08
2807 2.38599232011971e-08
2808 2.5600989826402e-08
2809 2.38743840473288e-08
2810 2.5544194499183e-08
2811 2.36689930753764e-08
2812 2.54669413963238e-08
2813 2.36934302850766e-08
2814 2.58015422940905e-08
2815 2.37128471869963e-08
2816 2.56565276577359e-08
2817 2.37042255589737e-08
2818 2.56677915840831e-08
2819 2.3753516123437e-08
2820 2.55503305800486e-08
2821 2.46404599884009e-08
2822 2.46461380636887e-08
2823 2.4748964603527e-08
2824 2.46714502822321e-08
2825 2.46378420366011e-08
2826 2.46449872051691e-08
2827 2.46555496336009e-08
2828 2.46688712455256e-08
2829 2.46995193537436e-08
2830 2.45315985392958e-08
2831 2.47080241843611e-08
2832 2.44373875825099e-08
2833 2.47081171723673e-08
2834 2.44355231014959e-08
2835 2.44228082624809e-08
2836 2.44444995679549e-08
2837 2.46563790944898e-08
2838 2.4365031343121e-08
2839 2.43340649875301e-08
2840 2.43471628570968e-08
2841 2.46297233797987e-08
2842 2.40055375629322e-08
2843 2.42107641142408e-08
2844 2.4172061804828e-08
2845 2.43192182767804e-08
2846 2.43572115820601e-08
2847 2.42937494105422e-08
2848 2.43463991991821e-08
2849 2.38937973862097e-08
2850 2.39898205738331e-08
2851 2.51033243336973e-08
2852 2.38704615524443e-08
2853 2.50318310195952e-08
2854 2.37526580733616e-08
2855 2.3691395141745e-08
2856 2.41513570903296e-08
2857 2.41927939456654e-08
2858 2.37844167955137e-08
2859 2.43841456398219e-08
2860 2.35823895845533e-08
2861 2.36422921604529e-08
2862 2.45945214200627e-08
2863 2.41613890721304e-08
2864 2.38477758087718e-08
2865 2.35040572936773e-08
2866 2.37340254122254e-08
2867 2.48680740206897e-08
2868 2.34569567516552e-08
2869 2.5043967687044e-08
2870 2.41365749355793e-08
2871 2.36085657913665e-08
2872 2.34582171867448e-08
2873 2.61779525760897e-08
2874 2.31622592476444e-08
2875 2.34557731569884e-08
2876 2.32525235319692e-08
2877 2.69565022407159e-08
2878 2.39129899834989e-08
2879 2.42190921949659e-08
2880 2.40645742880519e-08
2881 2.40446461997057e-08
2882 2.41488511743548e-08
2883 2.37563051573297e-08
2884 2.36540528150697e-08
2885 2.46436061271949e-08
2886 2.39790100653448e-08
2887 2.38894659708722e-08
2888 2.34057229610007e-08
2889 2.49620469867096e-08
2890 2.47925620787681e-08
2891 2.39875860922889e-08
2892 2.3580203720841e-08
2893 2.41850108396169e-08
2894 2.42605554711672e-08
2895 2.34626299315255e-08
2896 2.31435182080908e-08
2897 2.56834570239173e-08
2898 2.43961729848308e-08
2899 2.38399561514613e-08
2900 2.37302679929052e-08
2901 2.37295441289365e-08
2902 2.58972209779551e-08
2903 2.3548561310871e-08
2904 2.67464092623126e-08
2905 2.36825439163812e-08
2906 2.34308066440203e-08
2907 2.33040404595819e-08
2908 2.84293201870955e-08
2909 2.31585015665892e-08
2910 2.38955031898191e-08
2911 2.45492368541322e-08
2912 2.42468271374197e-08
2913 2.40514538337733e-08
2914 2.42090927596861e-08
2915 2.38563597010355e-08
2916 2.40642262879232e-08
2917 2.37849415861136e-08
2918 2.39921967743473e-08
2919 2.38468294908034e-08
2920 2.50340597015564e-08
2921 2.32700212364767e-08
2922 2.69027215572537e-08
2923 2.32338750171257e-08
2924 2.30915544995325e-08
2925 2.81891133313006e-08
2926 2.32313699162212e-08
2927 2.68947228955208e-08
2928 2.37027486850727e-08
2929 2.34055874757666e-08
2930 2.33106829480301e-08
2931 2.72705714307597e-08
2932 2.37149025278383e-08
2933 2.45570163298625e-08
2934 2.44958927809735e-08
2935 2.43267534958402e-08
2936 2.43067754318593e-08
2937 2.37242557752371e-08
2938 2.41121418422252e-08
2939 2.45162638340402e-08
2940 2.44035339761473e-08
2941 2.36925143797806e-08
2942 2.43467477753501e-08
2943 2.41730038733623e-08
2944 2.33187836613524e-08
2945 2.79162451592985e-08
2946 2.39545829103238e-08
2947 2.35055517215366e-08
2948 2.31643885644228e-08
2949 2.53348260893516e-08
2950 2.38979312988663e-08
2951 2.36345107542113e-08
2952 2.29892626696371e-08
2953 2.9736692277571e-08
2954 2.27411050022774e-08
2955 2.33911882562277e-08
2956 2.39239903781496e-08
2957 2.45105987469185e-08
2958 2.36667767985477e-08
2959 2.371451986527e-08
2960 2.36686464538671e-08
2961 2.2938227633329e-08
2962 2.82781198859827e-08
2963 2.59264831597439e-08
2964 2.2375732104174e-08
2965 2.63544810605865e-08
2966 2.35694690126076e-08
2967 2.41318125325241e-08
2968 2.33676716591691e-08
2969 2.42966702185821e-08
2970 2.75007296280938e-08
2971 2.29055353707097e-08
2972 2.65256856931684e-08
2973 2.45281723658675e-08
2974 2.29588234351152e-08
2975 2.65382731570729e-08
2976 2.37756620095619e-08
2977 2.31513439558073e-08
2978 2.49262255322957e-08
2979 2.40664785237099e-08
2980 2.34798701630456e-08
2981 2.38366344145757e-08
2982 2.43632129626747e-08
2983 2.41197926618697e-08
2984 2.34169928653438e-08
2985 2.82337545353673e-08
2986 2.40694630933014e-08
2987 2.30936657037439e-08
2988 2.53315035873003e-08
2989 2.6399554424239e-08
2990 2.26491095291492e-08
2991 2.7884623823915e-08
2992 2.43570243846913e-08
2993 2.297094557141e-08
2994 2.46933960348361e-08
2995 2.75385349691737e-08
2996 2.22289005873955e-08
2997 2.30402065597435e-08
2998 2.95257646894531e-08
2999 2.5462631186024e-08
3000 2.44555080626818e-08
3001 2.25789339172167e-08
3002 2.65096335698156e-08
3003 2.23578820588988e-08
3004 2.53132240732867e-08
3005 2.69851803044174e-08
3006 2.23119400817873e-08
3007 2.47448824798324e-08
3008 2.47280706041075e-08
3009 2.25222803202363e-08
3010 2.83051665060929e-08
3011 2.58510716570903e-08
3012 2.67155577990952e-08
3013 2.29470702067247e-08
3014 2.25230910664842e-08
3015 2.29374203183363e-08
3016 3.04783926677876e-08
3017 2.18979640786432e-08
3018 2.60729072168453e-08
3019 2.24467696064989e-08
3020 2.83353435590605e-08
3021 2.55780278868301e-08
3022 2.60081213721008e-08
3023 2.22037235939054e-08
3024 2.79371019589192e-08
3025 2.41066564089509e-08
3026 2.27599498724973e-08
3027 2.64476878171771e-08
3028 2.26157516165504e-08
3029 2.3269291142991e-08
3030 2.45603451560772e-08
3031 2.19832821060084e-08
3032 2.66555481838737e-08
3033 2.34551834458196e-08
3034 2.35227511339309e-08
3035 2.23879979832087e-08
3036 2.68823712484267e-08
3037 2.40751494902103e-08
3038 2.32120748758224e-08
3039 2.30747272280674e-08
3040 2.37301255620603e-08
3041 2.21943889640031e-08
3042 2.53070579350756e-08
3043 2.19971109801653e-08
3044 2.565704112234e-08
3045 2.29017131210263e-08
3046 2.35610261595798e-08
3047 2.37477831039334e-08
3048 2.21784326019381e-08
3049 2.63586663752369e-08
3050 2.20312610759166e-08
3051 2.81692951281665e-08
3052 2.34058944625914e-08
3053 2.27125802853556e-08
3054 2.44331406523135e-08
3055 2.62138313626981e-08
3056 2.18565079869726e-08
3057 2.44779641664916e-08
3058 2.59078325473561e-08
3059 2.16023048559277e-08
3060 2.81522491746289e-08
3061 2.14092306563551e-08
3062 2.8039085690601e-08
3063 2.53361943066688e-08
3064 2.42696727882352e-08
3065 2.38834500874141e-08
3066 2.34804321173621e-08
3067 2.43865383612873e-08
3068 2.19570238185196e-08
3069 2.91627608517975e-08
3070 2.50335541451263e-08
3071 2.59026397816386e-08
3072 2.15677469860287e-08
3073 2.70351845508277e-08
3074 2.54633000946192e-08
3075 2.52282848258178e-08
3076 2.26641143512563e-08
3077 2.21238370166077e-08
3078 2.97954613449058e-08
3079 2.12124316945339e-08
3080 2.72240155398951e-08
3081 2.12981612630891e-08
3082 2.44766731626878e-08
3083 2.15426443996858e-08
3084 2.85669674394695e-08
3085 2.48574127263423e-08
3086 2.53835889953979e-08
3087 2.13202899824183e-08
3088 2.51661601313735e-08
3089 2.1129158089439e-08
3090 2.24484321891816e-08
3091 2.99440480769353e-08
3092 2.11504458047795e-08
3093 2.20893060127181e-08
3094 2.6079030282844e-08
3095 2.13618032490204e-08
3096 2.41508933485068e-08
3097 2.26887158122913e-08
3098 2.57324443028728e-08
3099 2.09812675576737e-08
3100 2.49736071492035e-08
3101 2.14281108166237e-08
3102 2.44891282706328e-08
3103 2.15416881779729e-08
3104 2.9738659833134e-08
3105 2.09033212648369e-08
3106 2.37453212932892e-08
3107 2.32872055805911e-08
3108 2.5603636168714e-08
3109 2.11556206426433e-08
3110 2.41908074029573e-08
3111 2.1307543685678e-08
3112 2.46003925098992e-08
3113 2.1170323545705e-08
3114 2.38748967142932e-08
3115 2.49246554863136e-08
3116 2.10264407878857e-08
3117 2.62916213036868e-08
3118 2.09109269543872e-08
3119 2.93479950398412e-08
3120 2.07545486716199e-08
3121 2.37779945156946e-08
3122 2.13741319012306e-08
3123 2.66118393634773e-08
3124 2.10378538479383e-08
3125 2.20234706806988e-08
3126 2.93609506391013e-08
3127 2.03117122249408e-08
3128 2.73027544674953e-08
3129 2.06100552005428e-08
3130 2.52196707153374e-08
3131 2.11472395049706e-08
3132 2.5785020980218e-08
3133 2.05909858876918e-08
3134 2.81397473869682e-08
3135 2.03427122479338e-08
3136 2.61164045383122e-08
3137 2.08741024062542e-08
3138 2.48360669436454e-08
3139 2.22842755941977e-08
3140 2.52663977375334e-08
3141 2.10940566215401e-08
3142 2.6157678838068e-08
3143 2.08332539179401e-08
3144 2.33363585746549e-08
3145 2.09129854679135e-08
3146 2.64374769398668e-08
3147 2.09669658379208e-08
3148 2.15011790096198e-08
3149 2.37418803453182e-08
3150 2.10959529367005e-08
3151 2.34264985850552e-08
3152 2.10447054767027e-08
3153 2.84023079594187e-08
3154 2.033577580135e-08
3155 2.6040544737288e-08
3156 2.04709686922322e-08
3157 2.6615514595707e-08
3158 2.03140260616741e-08
3159 2.34653639523685e-08
3160 2.11298832880069e-08
3161 2.55774337706016e-08
3162 2.05504061912554e-08
3163 2.75612600501574e-08
3164 2.02583753978192e-08
3165 2.64530397040552e-08
3166 2.05076246251601e-08
3167 2.58361997899637e-08
3168 2.07360312490246e-08
3169 2.11776772203187e-08
3170 2.66603001092269e-08
3171 2.04679715864464e-08
3172 2.21262384160203e-08
3173 2.2922889934246e-08
3174 2.29370746278879e-08
3175 2.25164922081356e-08
3176 2.51311695249612e-08
3177 2.0656106809791e-08
3178 2.24491321528797e-08
3179 2.47575007634448e-08
3180 2.05877720753367e-08
3181 2.4891344144351e-08
3182 2.10620340485856e-08
3183 2.2755233849231e-08
3184 2.13857519098803e-08
3185 2.54083346616085e-08
3186 2.14479932320877e-08
3187 2.10207274771446e-08
3188 2.64955326952643e-08
3189 2.05464955762014e-08
3190 2.16895067582101e-08
3191 2.35033014724939e-08
3192 2.23781934524936e-08
3193 2.09061605336569e-08
3194 2.60303238428794e-08
3195 2.0785325277467e-08
3196 2.27633204374356e-08
3197 2.10316505297736e-08
3198 2.12581376997367e-08
3199 2.19904063912257e-08
3200 2.54530163907585e-08
3201 2.09206785447835e-08
3202 2.1449018010461e-08
3203 2.80403926264383e-08
3204 2.07196543022814e-08
3205 2.17724605186265e-08
3206 2.61241221947706e-08
3207 2.12282302033684e-08
3208 2.19837548816604e-08
3209 2.47821587645303e-08
3210 2.05818947449921e-08
3211 2.26947248191112e-08
3212 2.11058697476008e-08
3213 2.19546977291152e-08
3214 2.58063941322717e-08
3215 2.1868488492699e-08
3216 2.27754883908859e-08
3217 2.23891525298336e-08
3218 2.06507509527276e-08
3219 2.58675942414188e-08
3220 2.04823722667058e-08
3221 2.27027695928195e-08
3222 2.41518571323929e-08
3223 2.03935487749773e-08
3224 2.71863811213413e-08
3225 1.98216914080729e-08
3226 2.42000629805672e-08
3227 2.05212732317106e-08
3228 2.10653156766383e-08
3229 2.56883462171342e-08
3230 2.02318759215669e-08
3231 2.4403450058913e-08
3232 2.01410305081806e-08
3233 2.34769927491274e-08
3234 2.03674524825725e-08
3235 2.434987267938e-08
3236 2.04283521111071e-08
3237 2.09426650029376e-08
3238 2.29519747851903e-08
3239 2.23020306710509e-08
3240 2.13080973885671e-08
3241 2.63697090728265e-08
3242 1.99866536023663e-08
3243 2.13840300471446e-08
3244 2.7745241106103e-08
3245 1.9790842224976e-08
3246 2.56215712388097e-08
3247 1.99721752322801e-08
3248 2.28471273172881e-08
3249 2.23734526056396e-08
3250 2.5021827903815e-08
3251 2.05790671198303e-08
3252 2.08180974639316e-08
3253 2.58823646108464e-08
3254 2.08570314524648e-08
3255 2.09829669368111e-08
3256 2.72024773891877e-08
3257 2.01634772789006e-08
3258 2.45044650996062e-08
3259 2.00427029627137e-08
3260 2.53642724540037e-08
3261 2.03419147455641e-08
3262 2.49507348927536e-08
3263 2.01174543677518e-08
3264 2.34879528727694e-08
3265 2.02251436416945e-08
3266 2.40752767237673e-08
3267 2.04901349617481e-08
3268 2.30927331955622e-08
3269 2.26712894051961e-08
3270 2.2843950917395e-08
3271 2.06580282405278e-08
3272 2.5790707024409e-08
3273 2.11124540655527e-08
3274 2.10117196303428e-08
3275 2.40857963296481e-08
3276 2.16210581741594e-08
3277 2.22317444184827e-08
3278 2.17475867503314e-08
3279 2.03555861549298e-08
3280 2.49716260958266e-08
3281 1.992516817656e-08
3282 2.40155373149586e-08
3283 2.02026336805772e-08
3284 2.11567026117965e-08
3285 2.63236588042837e-08
3286 1.96596775009772e-08
3287 2.60389747207268e-08
3288 1.98441885216616e-08
3289 2.61674097293829e-08
3290 1.97103012196109e-08
3291 2.49230256955113e-08
3292 2.02240325875569e-08
3293 2.08684297180461e-08
3294 2.39774224846667e-08
3295 2.04039478887696e-08
3296 2.28462034673471e-08
3297 2.12575176435359e-08
3298 2.0298001481095e-08
3299 2.48836159304489e-08
3300 2.00829175693029e-08
3301 2.55105254778065e-08
3302 2.00453380405918e-08
3303 2.38545116935107e-08
3304 1.99416888802539e-08
3305 2.56649865551517e-08
3306 1.95571315599607e-08
3307 2.57475369091209e-08
3308 1.98501850191546e-08
3309 2.24575032535812e-08
3310 2.53879832327608e-08
3311 2.03096311729434e-08
3312 2.28124548058428e-08
3313 2.23129761440788e-08
3314 2.42157013255195e-08
3315 1.99774442056211e-08
3316 2.65233576119961e-08
3317 1.96969883471954e-08
3318 2.44561049085879e-08
3319 1.96571493724651e-08
3320 2.27878792598002e-08
3321 2.19908116109169e-08
3322 2.54255286163962e-08
3323 2.08384969622166e-08
3324 2.09113039197628e-08
3325 2.62060430898936e-08
3326 1.97354217617851e-08
3327 2.38763813272813e-08
3328 1.98579529880893e-08
3329 2.28856242023845e-08
3330 2.19836681387142e-08
3331 2.21188747931422e-08
3332 2.45538442655291e-08
3333 2.07216030874702e-08
3334 2.23308487738438e-08
3335 2.35631468744058e-08
3336 2.02559752570663e-08
3337 2.46614071338358e-08
3338 2.00184480083632e-08
3339 2.05110721009927e-08
3340 2.32841747150325e-08
3341 2.15659061765239e-08
3342 2.03360294753208e-08
3343 2.47565350573187e-08
3344 1.98555443315929e-08
3345 2.26117555452987e-08
3346 2.25138872873565e-08
3347 2.11464443455878e-08
3348 2.18033425155584e-08
3349 2.44196093490689e-08
3350 2.06679944579902e-08
3351 2.24225640734765e-08
3352 2.34539509699205e-08
3353 2.12089698334306e-08
3354 2.13310988435578e-08
3355 2.1745458397282e-08
3356 2.10325682371004e-08
3357 2.71586738637142e-08
3358 1.90461439445733e-08
3359 2.10884706551839e-08
3360 2.69142138402123e-08
3361 1.92441120231424e-08
3362 2.5519365049409e-08
3363 1.94715381988675e-08
3364 2.26945307197912e-08
3365 2.16909905090823e-08
3366 2.47908064759939e-08
3367 2.00773120551667e-08
3368 2.21093120314486e-08
3369 2.04181843966866e-08
3370 2.69335401581472e-08
3371 1.93939568151891e-08
3372 2.03138348691678e-08
3373 2.23205245177271e-08
3374 2.26426922327883e-08
3375 2.1320299382066e-08
3376 2.20751051068779e-08
3377 1.99087214391436e-08
3378 2.07431014608295e-08
3379 2.66491462545537e-08
3380 1.91906784721696e-08
3381 2.0086216991233e-08
3382 2.35510668019634e-08
3383 1.9744779760078e-08
3384 2.33942522089581e-08
3385 2.15950162750222e-08
3386 2.02217463889376e-08
3387 2.32628942288815e-08
3388 1.96147510830302e-08
3389 2.27277452670704e-08
3390 2.14696232114786e-08
3391 2.03726541549987e-08
3392 2.32532399690455e-08
3393 1.99714752285307e-08
3394 2.37129356139842e-08
3395 1.98155044704995e-08
3396 2.57917527373697e-08
3397 2.19509171561216e-08
3398 2.23896483035335e-08
3399 2.14352946323326e-08
3400 2.06198573659e-08
3401 2.41778102968437e-08
3402 1.96979709626288e-08
3403 2.4257385976445e-08
3404 2.03514401462823e-08
3405 2.20808542218953e-08
3406 2.45763631464557e-08
3407 1.98222975871798e-08
3408 2.01798786827467e-08
3409 2.68902765071521e-08
3410 1.93851844156323e-08
3411 2.35667740337009e-08
3412 1.94738728019339e-08
3413 2.2978855689304e-08
3414 1.98535717773618e-08
3415 2.68140434117603e-08
3416 1.89516622225161e-08
3417 2.08512708166908e-08
3418 2.6263712986585e-08
3419 1.94743487607341e-08
3420 2.00303558616222e-08
3421 2.67450065279951e-08
3422 1.88452239020809e-08
3423 2.50085122405208e-08
3424 1.93879770755134e-08
3425 2.04392037988776e-08
3426 2.60207598006268e-08
3427 1.90628158739103e-08
3428 2.35289967045194e-08
3429 2.13584506268683e-08
3430 2.19680383795051e-08
3431 2.17665519120502e-08
3432 2.22280782863349e-08
3433 2.1639026820508e-08
3434 2.16521749915599e-08
3435 2.17780308683357e-08
3436 2.1992962182027e-08
3437 2.17096378870574e-08
3438 2.01857737310851e-08
3439 2.27603451707081e-08
3440 2.44445704541962e-08
3441 2.00491373101852e-08
3442 2.38468834498629e-08
3443 1.96333342942223e-08
3444 2.44103616559022e-08
3445 1.94347727695554e-08
3446 2.010400368746e-08
3447 2.31382120227663e-08
3448 2.14384329411677e-08
3449 2.19431780493773e-08
3450 2.18535129632991e-08
3451 2.1689051244278e-08
3452 2.14312246805681e-08
3453 2.1956750959895e-08
3454 2.05813771918528e-08
3455 2.70501330863482e-08
3456 1.9951837733595e-08
3457 2.08365614924877e-08
3458 2.22027887886722e-08
3459 2.35032386803891e-08
3460 1.92967955549017e-08
3461 2.30941711735211e-08
3462 1.93614898282846e-08
3463 2.34787942902659e-08
3464 2.00701774547563e-08
3465 2.20433175351364e-08
3466 2.13083108567269e-08
3467 2.25935513626374e-08
3468 2.02663028479044e-08
3469 2.66679417160187e-08
3470 1.94536166080728e-08
3471 2.20408283629903e-08
3472 2.12959434334192e-08
3473 1.95049864358809e-08
3474 2.55964789651064e-08
3475 1.90562972237629e-08
3476 2.40484171196154e-08
3477 1.93315095416424e-08
3478 2.17789953124992e-08
3479 2.01796127721754e-08
3480 2.46600214142134e-08
3481 1.92499393052714e-08
3482 2.23418491501204e-08
3483 2.01794213427475e-08
3484 2.58871746519618e-08
3485 2.01243157411102e-08
3486 2.00186113129552e-08
3487 2.65466370233947e-08
3488 1.89289117681346e-08
3489 2.04587679199508e-08
3490 2.49583987282087e-08
3491 1.95455067125705e-08
3492 2.3132504449408e-08
3493 1.9549597960189e-08
3494 2.48942070499858e-08
3495 1.88722087959559e-08
3496 2.53213819388365e-08
3497 1.87669049348804e-08
3498 2.15979169758129e-08
3499 2.23027126488029e-08
3500 2.24561513437749e-08
3501 2.09832219280615e-08
3502 2.20161556764853e-08
3503 2.17972591888271e-08
3504 2.15451568065239e-08
3505 2.13597674357247e-08
3506 2.02607331616922e-08
3507 2.64680014908092e-08
3508 1.94183142694371e-08
3509 2.22737421344821e-08
3510 2.05224232119883e-08
3511 2.02814060125678e-08
3512 2.65924225755221e-08
3513 1.86188925068997e-08
3514 2.42836424232951e-08
3515 1.92555224085011e-08
3516 2.1263856220699e-08
3517 2.18985854829601e-08
3518 1.99429776995275e-08
3519 2.64090669083727e-08
3520 1.86695164761663e-08
3521 2.0693674358252e-08
3522 2.26873136241479e-08
3523 2.17837121421205e-08
3524 2.08072940974802e-08
3525 2.22244673604433e-08
3526 2.12970941911861e-08
3527 2.1410548735723e-08
3528 2.15281088343233e-08
3529 2.1530327861341e-08
3530 1.98862761966456e-08
3531 2.67984395395104e-08
3532 1.93994854461588e-08
3533 2.27808222226233e-08
3534 2.15306845874896e-08
3535 2.00873893826947e-08
3536 2.15477087088245e-08
3537 1.93912089105908e-08
3538 2.63071367128664e-08
3539 1.83494197585898e-08
3540 2.47915036702406e-08
3541 1.86564323805816e-08
3542 2.31556667901756e-08
3543 2.09973938744001e-08
3544 2.12709128627475e-08
3545 2.16838835783162e-08
3546 2.15009068950112e-08
3547 2.09765749070723e-08
3548 2.15130656242568e-08
3549 2.11447518960184e-08
3550 1.9301384447834e-08
3551 2.48357131412691e-08
3552 2.02144886622957e-08
3553 2.39199204426499e-08
3554 2.03918015194149e-08
3555 2.24677080183922e-08
3556 2.0293878489841e-08
3557 1.92482063733346e-08
3558 2.31686911447149e-08
3559 1.90701201957522e-08
3560 2.1737947386552e-08
3561 1.9479344611939e-08
3562 2.66174212184556e-08
3563 1.86621553091015e-08
3564 2.45820789175899e-08
3565 1.87807057653866e-08
3566 2.30207109138125e-08
3567 2.09347828974138e-08
3568 2.00866709537117e-08
3569 2.59488270085706e-08
3570 1.85504658220537e-08
3571 2.27245832589151e-08
3572 2.05847795748393e-08
3573 1.96713568063611e-08
3574 2.60372131207776e-08
3575 1.86727379496987e-08
3576 2.31762690622306e-08
3577 1.96422924497175e-08
3578 2.55723047496437e-08
3579 1.86376072737193e-08
3580 2.45637379365227e-08
3581 1.8820411873427e-08
3582 2.41177313689556e-08
3583 1.85763511016401e-08
3584 2.19301147399431e-08
3585 2.06449309178569e-08
3586 2.1607570200477e-08
3587 2.07010781949069e-08
3588 2.11101606475417e-08
3589 2.09698829569116e-08
3590 2.09147493219208e-08
3591 2.12800151309644e-08
3592 2.08734680302025e-08
3593 1.98674733516424e-08
3594 2.62364430656525e-08
3595 1.98223506983064e-08
3596 2.10730631765021e-08
3597 2.14035446290395e-08
3598 1.99005169499444e-08
3599 2.18966822823075e-08
3600 2.15201325574621e-08
3601 1.87508037762796e-08
3602 2.09781107116114e-08
3603 2.09979753514888e-08
3604 2.15054666916625e-08
3605 2.07216178982339e-08
3606 2.06109302659518e-08
3607 2.15705113468245e-08
3608 2.25833306142231e-08
3609 1.94202868118443e-08
3610 2.03056331428697e-08
3611 2.11216449636509e-08
3612 2.18673377454515e-08
3613 1.9672700166562e-08
3614 2.18369078733294e-08
3615 2.03222276657367e-08
3616 2.16974915424606e-08
3617 2.04071904785774e-08
3618 2.09651867646987e-08
3619 2.15113993067839e-08
3620 1.9855092942056e-08
3621 2.64778680247324e-08
3622 1.925523707208e-08
3623 2.09496564539391e-08
3624 2.15983774993211e-08
3625 2.00115834616543e-08
3626 1.93329892308469e-08
3627 2.18415311880604e-08
3628 1.85224471712653e-08
3629 1.95400991905292e-08
3630 2.69213365006182e-08
3631 1.79270997282266e-08
3632 2.46556927224195e-08
3633 1.80566158539397e-08
3634 2.31526686693129e-08
3635 2.08386813792538e-08
3636 2.03331886977631e-08
3637 2.15346340225464e-08
3638 2.14960183441693e-08
3639 2.08001849794637e-08
3640 1.95945404304343e-08
3641 2.63718601403728e-08
3642 1.83453207531492e-08
3643 2.48385731786982e-08
3644 1.83058318834473e-08
3645 2.34161800595767e-08
3646 1.97083039001344e-08
3647 1.9697723456169e-08
3648 2.61577853543638e-08
3649 1.76219439833036e-08
3650 2.42701722528715e-08
3651 1.81607599877842e-08
3652 2.21638405984437e-08
3653 2.06940090625163e-08
3654 2.0768378239e-08
3655 2.08438876278527e-08
3656 2.13413642486948e-08
3657 2.05274347161999e-08
3658 2.03675628012712e-08
3659 2.10733069755387e-08
3660 1.95385731762143e-08
3661 2.62296945403584e-08
3662 1.85014781808879e-08
3663 2.33013920511449e-08
3664 1.84384211099342e-08
3665 2.49245210155458e-08
3666 1.80290120662396e-08
3667 2.13049376860397e-08
3668 2.1640038027293e-08
3669 1.94158759799778e-08
3670 2.54227906906213e-08
3671 1.77902655089657e-08
3672 2.44926649055621e-08
3673 1.81564238279086e-08
3674 2.36129973359978e-08
3675 1.84951379128551e-08
3676 2.41313815212418e-08
3677 1.84193420259549e-08
3678 2.341038019682e-08
3679 1.99725008595553e-08
3680 1.91198025880079e-08
3681 2.4996825935153e-08
3682 1.79415859788701e-08
3683 2.09201233051792e-08
3684 2.15041502187874e-08
3685 1.98583099355609e-08
3686 2.05108684677158e-08
3687 2.19176443929103e-08
3688 2.02868336366957e-08
3689 1.94321838575817e-08
3690 2.51401910260074e-08
3691 1.79351464386912e-08
3692 2.34942926982673e-08
3693 1.83696059163352e-08
3694 1.92925509810149e-08
3695 2.46258965634905e-08
3696 1.79303059822389e-08
3697 2.10712812955816e-08
3698 2.18747623795257e-08
3699 1.85923975759006e-08
3700 2.43666251077412e-08
3701 1.81590552753297e-08
3702 2.34035025335477e-08
3703 1.82428398726731e-08
3704 2.21342720101436e-08
3705 1.98105280043159e-08
3706 2.15293262401106e-08
3707 2.0853720949815e-08
3708 2.02220775513129e-08
3709 2.05369232024188e-08
3710 2.0615818564329e-08
3711 2.08603765251558e-08
3712 2.0637623045161e-08
3713 1.88833788382636e-08
3714 2.52110828008467e-08
3715 1.80906772440648e-08
3716 2.18384009922912e-08
3717 2.10340486638816e-08
3718 1.87237038065113e-08
3719 2.2607159223087e-08
3720 1.84373208922117e-08
3721 2.39744602779646e-08
3722 1.82689866700114e-08
3723 2.29166452189178e-08
3724 1.85580230222526e-08
3725 1.98599699026136e-08
3726 2.44375736990765e-08
3727 1.82516179518255e-08
3728 2.14959878345189e-08
3729 2.03924685300838e-08
3730 2.12734234707468e-08
3731 1.87630399836602e-08
3732 2.49779588331056e-08
3733 1.80873367371381e-08
3734 2.02241936723413e-08
3735 2.18247974329566e-08
3736 1.97244641731431e-08
3737 2.03405512640975e-08
3738 2.16967050197248e-08
3739 1.87837700797444e-08
3740 2.43577734968525e-08
3741 1.80328302865351e-08
3742 2.37547461318655e-08
3743 1.80811484882526e-08
3744 2.25138677407144e-08
3745 1.85580428547216e-08
3746 2.34642254815509e-08
3747 1.83920807706461e-08
3748 1.9308697897874e-08
3749 2.388804615433e-08
3750 1.80144637842161e-08
3751 2.28260967325977e-08
3752 1.8303984625434e-08
3753 2.19136218222449e-08
3754 2.05040722249805e-08
3755 1.89197872496882e-08
3756 2.41940635789528e-08
3757 1.79991684019221e-08
3758 2.35717275918756e-08
3759 1.8020501624777e-08
3760 2.35645826432429e-08
3761 1.87720266464075e-08
3762 2.31562156503551e-08
3763 1.82127415846167e-08
3764 2.21478903288941e-08
3765 1.80403051179001e-08
3766 2.40302988367691e-08
3767 1.80711364649033e-08
3768 2.06237971065448e-08
3769 2.14229394780441e-08
3770 1.92852544972921e-08
3771 2.35334826378941e-08
3772 1.80733708692871e-08
3773 1.92058209032842e-08
3774 2.39727514241006e-08
3775 1.78664506551018e-08
3776 1.9175431606816e-08
3777 2.3578123205803e-08
3778 1.7856258917287e-08
3779 1.93107498291933e-08
3780 2.34222750811197e-08
3781 1.8022067170359e-08
3782 2.24498239994042e-08
3783 1.83462641508703e-08
3784 1.9198134277143e-08
3785 2.32132194346035e-08
3786 1.81161954887488e-08
3787 2.28605612999089e-08
3788 1.81927896131806e-08
3789 2.06516184651229e-08
3790 2.08982991486417e-08
3791 1.87260519272647e-08
3792 1.90028220341254e-08
3793 1.92821348051486e-08
3794 2.30189179752927e-08
3795 1.84829515708362e-08
3796 1.871216179955e-08
3797 2.2452924363453e-08
3798 1.84047736375703e-08
3799 1.85858474893996e-08
3800 2.12778606513975e-08
3801 1.85191507522942e-08
3802 2.32772978778772e-08
3803 1.8078898141205e-08
3804 1.88284403214167e-08
3805 2.09363407673857e-08
3806 2.07917152267312e-08
3807 2.06941188723442e-08
3808 1.87642244849351e-08
3809 2.29158324954459e-08
3810 1.8162580322556e-08
3811 2.24840255937386e-08
3812 1.81949047289853e-08
3813 1.89291685019077e-08
3814 1.89318589871312e-08
3815 2.13130317407717e-08
3816 2.00723655959245e-08
3817 1.86725361417672e-08
3818 1.94174871242514e-08
3819 2.28705159193432e-08
3820 1.83540636871649e-08
3821 2.2743374575751e-08
3822 1.81251388738368e-08
3823 1.88443702971997e-08
3824 2.31391735388276e-08
3825 1.81079144774832e-08
3826 1.90563799737087e-08
3827 2.30150338761392e-08
3828 1.80071538739834e-08
3829 1.90968304918382e-08
3830 2.2930983447994e-08
3831 1.80733848095249e-08
3832 1.86259319302218e-08
3833 1.88852107255166e-08
3834 2.10676115574571e-08
3835 1.82739070370419e-08
3836 1.88138785836567e-08
3837 1.93738487272166e-08
3838 2.07938264629726e-08
3839 1.93843906595714e-08
3840 2.05302120995232e-08
3841 1.84363960583078e-08
3842 2.0889674018143e-08
3843 1.84595398317611e-08
3844 2.2284923333199e-08
3845 1.8886640181015e-08
3846 1.85322877641059e-08
3847 2.17816771510004e-08
3848 1.91901596048316e-08
3849 1.84585130172665e-08
3850 2.28988756520998e-08
3851 1.86515833678902e-08
3852 2.05594219886418e-08
3853 1.86246085803754e-08
3854 2.09293922331655e-08
3855 1.83904272107738e-08
3856 2.28899928929605e-08
3857 1.80662986589697e-08
3858 2.24751940209322e-08
3859 1.80580092371541e-08
3860 1.86522905069875e-08
3861 1.9966421370976e-08
3862 1.89082864166523e-08
3863 2.25350424655213e-08
3864 1.81246620286069e-08
3865 2.24785138420169e-08
3866 1.80992850036255e-08
3867 2.10553173010108e-08
3868 2.01489465359861e-08
3869 1.85088101688491e-08
3870 1.87663879576416e-08
3871 2.26197130171468e-08
3872 1.79942965582991e-08
3873 2.18086996653932e-08
3874 1.8359253227046e-08
3875 2.22221349062968e-08
3876 1.82818911378391e-08
3877 2.26715591055693e-08
3878 1.80669309118309e-08
3879 2.19283945884774e-08
3880 1.81519469273084e-08
3881 2.26202590586089e-08
3882 1.82695113709053e-08
3883 2.1886418513084e-08
3884 1.80804561311643e-08
3885 2.12750461253131e-08
3886 1.82030393265364e-08
3887 2.29176878970283e-08
3888 1.85609242013274e-08
3889 1.83557537759005e-08
3890 2.20581433422462e-08
3891 1.85779182502177e-08
3892 1.88830092756609e-08
3893 2.19946287677852e-08
3894 1.82893277640983e-08
3895 2.19604325569223e-08
3896 1.81335723369869e-08
3897 2.2313013978037e-08
3898 1.80044540977753e-08
3899 2.19596213901774e-08
3900 1.80502590708398e-08
3901 1.87680355224484e-08
3902 2.25673945115079e-08
3903 1.7928493970365e-08
3904 2.08584916918109e-08
3905 2.0089212547203e-08
3906 1.85721327373434e-08
3907 1.85685788387235e-08
3908 2.23382543259321e-08
3909 1.83164021325677e-08
3910 1.88632173620251e-08
3911 2.25128730544089e-08
3912 1.85309274317036e-08
3913 1.82532370627819e-08
3914 2.21389140507844e-08
3915 1.81584096135556e-08
3916 1.87760436612672e-08
3917 1.88716549068268e-08
3918 1.88992004616917e-08
3919 2.21487863883152e-08
3920 1.90417845895019e-08
3921 1.82925626103225e-08
3922 2.05411718397219e-08
3923 1.8184478036698e-08
3924 2.10459637094984e-08
3925 1.82732260827212e-08
3926 2.20197016863322e-08
3927 1.81486552295052e-08
3928 1.88661842821791e-08
3929 2.25303752172112e-08
3930 1.79689666086613e-08
3931 2.03342122228056e-08
3932 1.83869740902898e-08
3933 1.88269673005725e-08
3934 1.8523023558481e-08
3935 2.0409313203934e-08
3936 1.86877329284019e-08
3937 1.86536791641212e-08
3938 1.8615600708427e-08
3939 1.9836547992258e-08
3940 1.87717711173097e-08
3941 1.86373850731836e-08
3942 1.85786257693443e-08
3943 2.31972426184557e-08
3944 1.86817805182815e-08
3945 1.89952480562361e-08
3946 1.85956842773116e-08
3947 1.85751566518511e-08
3948 2.04918841249235e-08
3949 2.08251441383145e-08
3950 1.91935301798174e-08
3951 2.0099966341891e-08
3952 1.96798077009563e-08
3953 1.88126579959424e-08
3954 1.84943801666837e-08
3955 1.90772081350432e-08
3956 2.07132162047818e-08
3957 1.92898573135869e-08
3958 1.84524808803654e-08
3959 2.01226396630561e-08
3960 2.01164820455846e-08
3961 1.8872034534656e-08
3962 1.86385344229933e-08
3963 1.83293392412021e-08
3964 2.15319185081497e-08
3965 1.86129161892645e-08
3966 1.86664932673997e-08
3967 1.86027540768074e-08
3968 1.86606598600036e-08
3969 2.18689294249341e-08
3970 1.85839865640136e-08
3971 1.85291322038206e-08
3972 2.18946676830978e-08
3973 1.84522167748502e-08
3974 2.04821882431316e-08
3975 1.82267763339239e-08
3976 2.18811405411357e-08
3977 1.8340317379073e-08
3978 1.8827615373973e-08
3979 2.11609269249458e-08
3980 1.81056951250314e-08
3981 2.13979014193066e-08
3982 1.80302389720666e-08
3983 2.06660691233207e-08
3984 1.83036274201132e-08
3985 1.856157858382e-08
3986 1.85733618462136e-08
3987 2.06723862179459e-08
3988 1.79218726824026e-08
3989 2.16611363654806e-08
3990 1.86364595429755e-08
3991 1.83217571117777e-08
3992 2.32148580517344e-08
3993 1.8222091296205e-08
3994 2.08995524043909e-08
3995 1.79214896702251e-08
3996 2.20252729182246e-08
3997 1.76893298789171e-08
3998 2.12734340664378e-08
3999 1.78950164356484e-08
4000 2.11394428352163e-08
4001 1.80288437842124e-08
4002 2.14580352013161e-08
4003 1.78085848764054e-08
4004 2.22426354171412e-08
4005 1.82692045808952e-08
4006 1.81986510425225e-08
4007 1.92939101216483e-08
4008 1.84000105787618e-08
4009 1.88435372883711e-08
4010 2.25374404402012e-08
4011 1.79751690414787e-08
4012 2.16146494734504e-08
4013 1.76719789823365e-08
4014 2.18965563300866e-08
4015 1.75690084189828e-08
4016 1.99448902914456e-08
4017 1.80834291902687e-08
4018 1.85627427728985e-08
4019 2.02324745560978e-08
4020 1.82150994717523e-08
4021 1.83104193195727e-08
4022 2.17649740308079e-08
4023 1.85576212404781e-08
4024 1.79482441640122e-08
4025 2.18905507862188e-08
4026 1.79760354222513e-08
4027 2.11980532734135e-08
4028 1.7580208773682e-08
4029 2.07909913795268e-08
4030 1.76723773160081e-08
4031 2.15206737352591e-08
4032 1.79585269201865e-08
4033 1.87147732749748e-08
4034 2.2332770301392e-08
4035 1.82524126558781e-08
4036 1.77383431738676e-08
4037 2.27314958334279e-08
4038 1.8251398313518e-08
4039 1.76768294957441e-08
4040 2.200884280118e-08
4041 1.72807178454304e-08
4042 2.0207313874171e-08
4043 1.77921346934673e-08
4044 1.83274700215064e-08
4045 1.80403527121342e-08
4046 2.11418188825752e-08
4047 1.81764327954748e-08
4048 1.78239150734616e-08
4049 2.15217652435873e-08
4050 1.85695720878731e-08
4051 1.77596050665307e-08
4052 2.02837178496229e-08
4053 1.76522904608567e-08
4054 2.17210110070309e-08
4055 1.84726391791457e-08
4056 1.78065804258831e-08
4057 2.16229946590762e-08
4058 1.84145680587333e-08
4059 1.76934013810248e-08
4060 2.11422369840453e-08
4061 1.77213662215792e-08
4062 1.96006668532756e-08
4063 1.77956594442363e-08
4064 1.79503334011566e-08
4065 1.7754999448838e-08
4066 1.96329674684859e-08
4067 1.93805886928022e-08
4068 1.90238766145101e-08
4069 1.75113108981384e-08
4070 2.10830558911301e-08
4071 1.76767879423445e-08
4072 2.10901788629814e-08
4073 1.71397457968681e-08
4074 2.17210021754843e-08
4075 1.69851585167247e-08
4076 2.08653987224805e-08
4077 1.71398802410183e-08
4078 2.14473425815265e-08
4079 1.78198697556864e-08
4080 1.76268577719019e-08
4081 2.03899529265805e-08
4082 1.75902675842166e-08
4083 1.77046243763912e-08
4084 1.77844792721704e-08
4085 2.2401019066226e-08
4086 1.69958505936152e-08
4087 1.82695407962563e-08
4088 2.17320458656689e-08
4089 1.69588775181251e-08
4090 2.13891166492008e-08
4091 1.68535752092802e-08
4092 1.94092367128973e-08
4093 1.90935578939511e-08
4094 1.77832988079163e-08
4095 1.78367107578348e-08
4096 2.18655128941814e-08
4097 1.69570728288937e-08
4098 2.13442728228819e-08
4099 1.72296409614581e-08
4100 2.09940178488854e-08
4101 1.73276718409832e-08
4102 2.10599120486488e-08
4103 1.76192125796726e-08
4104 2.0072119795489e-08
4105 1.75307863884144e-08
4106 1.93876011235494e-08
4107 1.7704836329846e-08
4108 2.066949200491e-08
4109 1.88739333361609e-08
4110 1.78784695995804e-08
4111 1.97664461544234e-08
4112 1.78987767468497e-08
4113 2.13829316192404e-08
4114 1.74841101729661e-08
4115 2.10916573764597e-08
4116 1.70395958020708e-08
4117 2.01963836384267e-08
4118 1.74428724244724e-08
4119 2.05464366940522e-08
4120 1.73066686458001e-08
4121 2.18859612624023e-08
4122 1.7048762967764e-08
4123 2.13151338911144e-08
4124 1.72011310265174e-08
4125 2.1338882063654e-08
4126 1.72622648782206e-08
4127 1.96496880668917e-08
4128 1.90745488220512e-08
4129 1.79976479116883e-08
4130 1.77579938355488e-08
4131 2.21981063324439e-08
4132 1.72774658725727e-08
4133 2.04613863623027e-08
4134 1.74354333832838e-08
4135 2.12236886564243e-08
4136 1.75270033283548e-08
4137 1.9967890596051e-08
4138 1.76156873533673e-08
4139 1.94030698716929e-08
4140 1.75924824014351e-08
4141 2.13214969602415e-08
4142 1.78319341320454e-08
4143 1.77812479994766e-08
4144 2.03631573734431e-08
4145 1.74315776871326e-08
4146 2.25014366255216e-08
4147 1.81168701570977e-08
4148 1.78575819513582e-08
4149 1.90594396893007e-08
4150 1.75461024018997e-08
4151 1.75876989255697e-08
4152 2.01207921157465e-08
4153 1.74796392004017e-08
4154 2.07360272112822e-08
4155 1.71476520308134e-08
4156 2.10887988053576e-08
4157 1.70701810125462e-08
4158 2.01510992243747e-08
4159 1.76016469400353e-08
4160 1.94448574806216e-08
4161 1.74361617849561e-08
4162 2.09537702421658e-08
4163 1.70575969071496e-08
4164 2.13611602770392e-08
4165 1.69583870228418e-08
4166 1.99100022263698e-08
4167 1.77703266509011e-08
4168 1.78151617090949e-08
4169 1.76089610287034e-08
4170 2.15646584416085e-08
4171 1.76426624873272e-08
4172 1.8486108095267e-08
4173 1.75629282174461e-08
4174 1.7601697865216e-08
4175 2.17946521524071e-08
4176 1.72218361675081e-08
4177 1.98989624230483e-08
4178 1.71440912447307e-08
4179 1.75231052404745e-08
4180 2.18455267662898e-08
4181 1.67964306209412e-08
4182 2.06824723446852e-08
4183 1.71524074320806e-08
4184 2.08384951367879e-08
4185 1.72124055606016e-08
4186 1.95885137891372e-08
4187 1.7523465654895e-08
4188 1.78678852369163e-08
4189 1.75596179851079e-08
4190 1.79721917391573e-08
4191 1.75797889641061e-08
4192 2.15662226309699e-08
4193 1.73209242545214e-08
4194 1.85604773865478e-08
4195 1.74610829177524e-08
4196 2.09705370415314e-08
4197 1.72804747005095e-08
4198 1.8945242923285e-08
4199 1.77986172087141e-08
4200 1.74549385566769e-08
4201 2.05499268511877e-08
4202 1.77031875453759e-08
4203 1.75569899149297e-08
4204 2.04141266886282e-08
4205 1.81483525284187e-08
4206 1.74660312969033e-08
4207 2.15687328317948e-08
4208 1.70859839392357e-08
4209 2.01908501137604e-08
4210 1.71296987933611e-08
4211 2.07866668479051e-08
4212 1.69999048913738e-08
4213 2.06787901047001e-08
4214 1.69925975125051e-08
4215 1.79410434620886e-08
4216 1.99394203974634e-08
4217 1.7264949797674e-08
4218 1.95904046162154e-08
4219 1.72478321394109e-08
4220 2.04136305814451e-08
4221 1.70483050327963e-08
4222 2.06118119119025e-08
4223 1.69452247550383e-08
4224 2.02453779141465e-08
4225 1.7224633092644e-08
4226 2.13752635456754e-08
4227 1.73428112535678e-08
4228 1.74452426890404e-08
4229 1.76530822803556e-08
4230 1.77386114092182e-08
4231 1.80300840222358e-08
4232 2.11677874413507e-08
4233 1.67599638832006e-08
4234 2.04685585302178e-08
4235 1.6950992297815e-08
4236 1.77742966250283e-08
4237 2.07720870241401e-08
4238 1.69361659495271e-08
4239 1.90651724795432e-08
4240 1.87660568257975e-08
4241 1.76097857543522e-08
4242 1.74019412277626e-08
4243 2.10545196389911e-08
4244 1.68815471238148e-08
4245 2.06553492718697e-08
4246 1.69506218927751e-08
4247 1.90849669951254e-08
4248 1.90801686940079e-08
4249 1.74937423941812e-08
4250 1.73556438431977e-08
4251 1.78618833415745e-08
4252 1.90371867726524e-08
4253 1.75726168775703e-08
4254 2.12702102010964e-08
4255 1.66978598419021e-08
4256 2.05260913362648e-08
4257 1.69338107452455e-08
4258 1.96470507494162e-08
4259 1.72980177562121e-08
4260 1.74582249185262e-08
4261 1.77343889557202e-08
4262 2.11774916240581e-08
4263 1.68737408257069e-08
4264 1.74100511319553e-08
4265 1.80833018853244e-08
4266 1.90173500993207e-08
4267 1.78118472075051e-08
4268 1.75003274643093e-08
4269 1.94478442505364e-08
4270 1.87577600223587e-08
4271 1.72611248678289e-08
4272 2.09620264907395e-08
4273 1.68878287573337e-08
4274 1.7770800129796e-08
4275 1.7343647671586e-08
4276 2.07984847893916e-08
4277 1.70369816245575e-08
4278 2.01551302860814e-08
4279 1.69182649952249e-08
4280 2.05727704552239e-08
4281 1.6871736951668e-08
4282 2.01808328975095e-08
4283 1.71828486526249e-08
4284 2.07372787691673e-08
4285 1.71447097695177e-08
4286 1.85119616378515e-08
4287 1.92978286791268e-08
4288 1.71991954759088e-08
4289 2.00511927627989e-08
4290 1.7369468770817e-08
4291 1.74188324384428e-08
4292 1.80366830728196e-08
4293 1.91895225361283e-08
4294 1.72314859563094e-08
4295 2.10484626054774e-08
4296 1.68412052300726e-08
4297 1.78021436819109e-08
4298 1.81676585863932e-08
4299 1.78502965801353e-08
4300 1.73599070675545e-08
4301 2.10550750943672e-08
4302 1.68575011435801e-08
4303 1.96904939300246e-08
4304 1.70635833916799e-08
4305 1.9556755988609e-08
4306 1.70991431810708e-08
4307 1.9777231490048e-08
4308 1.75661619529754e-08
4309 2.06038321737823e-08
4310 1.68651155319222e-08
4311 2.02865191865909e-08
4312 1.69372131499923e-08
4313 1.73522701679696e-08
4314 2.09943676302005e-08
4315 1.68008099928896e-08
4316 1.86858030378489e-08
4317 1.90261718335527e-08
4318 1.73280825044864e-08
4319 1.7373541086968e-08
4320 2.0851510584563e-08
4321 1.67375968966044e-08
4322 2.01952019331098e-08
4323 1.71359433138729e-08
4324 2.04392730173975e-08
4325 1.69952613734425e-08
4326 1.93322432293086e-08
4327 1.74612587545142e-08
4328 1.74863085288524e-08
4329 1.76139610705284e-08
4330 1.77071432677645e-08
4331 2.00602078861412e-08
4332 1.81341364557386e-08
4333 1.73624029818631e-08
4334 1.72768543015289e-08
4335 2.11717926652533e-08
4336 1.67631566056758e-08
4337 1.77149005951005e-08
4338 1.73138875576018e-08
4339 2.10081832105746e-08
4340 1.67256429148277e-08
4341 1.90306911582172e-08
4342 1.72878812962629e-08
4343 1.9232368442812e-08
4344 1.74097271769258e-08
4345 1.7295757711866e-08
4346 1.99717081746098e-08
4347 1.79891910552232e-08
4348 1.74840384741803e-08
4349 1.76059526329209e-08
4350 1.72059363324484e-08
4351 2.11044606848176e-08
4352 1.70245901601473e-08
4353 1.86980193853104e-08
4354 1.70129430960009e-08
4355 2.10400505044339e-08
4356 1.65764041861194e-08
4357 1.90701596135268e-08
4358 1.87890719626216e-08
4359 1.72529725532e-08
4360 1.87961786508595e-08
4361 1.70712186887689e-08
4362 2.10294619805984e-08
4363 1.66794536106407e-08
4364 1.7292091368637e-08
4365 1.73158227432801e-08
4366 1.77218547665337e-08
4367 1.89305110272486e-08
4368 1.68995713481501e-08
4369 1.94803302784041e-08
4370 1.69728295569826e-08
4371 2.0697233095418e-08
4372 1.66281413091773e-08
4373 1.96217066861515e-08
4374 1.69739872328267e-08
4375 2.03760195638947e-08
4376 1.72757884729091e-08
4377 1.70812352938698e-08
4378 1.75260858435444e-08
4379 2.10670705023674e-08
4380 1.65664823076894e-08
4381 1.73172459389526e-08
4382 1.77929826907786e-08
4383 1.80745368967561e-08
4384 1.89365574632672e-08
4385 1.6831723763211e-08
4386 1.74573338279638e-08
4387 1.74660823017148e-08
4388 1.93341610190412e-08
4389 1.69253934930447e-08
4390 2.04995097422833e-08
4391 1.65493010287343e-08
4392 2.06995195229653e-08
4393 1.65516124687182e-08
4394 2.0422317387081e-08
4395 1.66291511875249e-08
4396 2.10418559040138e-08
4397 1.69221562459354e-08
4398 1.7259572123951e-08
4399 2.00555798290092e-08
4400 1.66527148057205e-08
4401 1.99160050160796e-08
4402 1.67419564029159e-08
4403 2.05799172086074e-08
4404 1.646044464923e-08
4405 2.03061257535164e-08
4406 1.67199978051147e-08
4407 1.90246978141739e-08
4408 1.71124093663322e-08
4409 1.70775983823479e-08
4410 1.86853401011877e-08
4411 1.73472432276056e-08
4412 1.70899775346622e-08
4413 1.9924744873584e-08
4414 1.76819434189179e-08
4415 1.71037567674726e-08
4416 2.10983493629524e-08
4417 1.65085790743424e-08
4418 1.98940442081574e-08
4419 1.66843999780952e-08
4420 1.73888307384573e-08
4421 1.82882352603719e-08
4422 1.89360445382591e-08
4423 1.68516457870194e-08
4424 2.02920414859309e-08
4425 1.731371835903e-08
4426 1.70933198924417e-08
4427 1.8755855431013e-08
4428 1.71227546258146e-08
4429 1.70863748796823e-08
4430 1.99435662093517e-08
4431 1.82965705663174e-08
4432 1.72409813414542e-08
4433 1.72650761267323e-08
4434 1.93093449488702e-08
4435 1.72161350967537e-08
4436 1.70578098424023e-08
4437 1.97004050240812e-08
4438 1.73379706154553e-08
4439 1.70844381065793e-08
4440 1.96276551140129e-08
4441 1.72631338692941e-08
4442 1.70050643262531e-08
4443 2.0236228980286e-08
4444 1.80596206840189e-08
4445 1.74504605255643e-08
4446 1.70189952526656e-08
4447 1.94619201044632e-08
4448 1.74400356943438e-08
4449 1.69628336791905e-08
4450 2.12769883604913e-08
4451 1.65457799027879e-08
4452 1.92319931824336e-08
4453 1.66281920083167e-08
4454 1.95937088983134e-08
4455 1.6639601470303e-08
4456 1.81139929723573e-08
4457 1.87217140061868e-08
4458 1.66935162829729e-08
4459 1.75418153249007e-08
4460 2.07678376447917e-08
4461 1.78149871034594e-08
4462 1.68637794328574e-08
4463 2.00759560012864e-08
4464 1.76993792560531e-08
4465 1.67728812495993e-08
4466 1.86647911655236e-08
4467 1.72230418861441e-08
4468 1.75083014017641e-08
4469 1.71070582671018e-08
4470 1.95042662776423e-08
4471 1.73980979496224e-08
4472 1.69146142978138e-08
4473 2.11701511320661e-08
4474 1.63860623122458e-08
4475 1.96136420808068e-08
4476 1.64289785102856e-08
4477 2.05561638303708e-08
4478 1.65000580369945e-08
4479 1.94952357297495e-08
4480 1.66514992304967e-08
4481 1.88079765202909e-08
4482 1.66600948601769e-08
4483 1.73662747267878e-08
4484 2.05900680733673e-08
4485 1.64362273854801e-08
4486 2.00418519616108e-08
4487 1.63181289183356e-08
4488 2.06083386327105e-08
4489 1.61915787194877e-08
4490 1.99796710596833e-08
4491 1.64881497393687e-08
4492 1.91470332935872e-08
4493 1.65199144980532e-08
4494 1.95407064672026e-08
4495 1.65825500906824e-08
4496 2.02881401923649e-08
4497 1.61856653707049e-08
4498 2.01440629614968e-08
4499 1.64053888723481e-08
4500 2.03162139576218e-08
4501 1.63910203209994e-08
4502 1.87503803855404e-08
4503 1.66764433758293e-08
4504 1.72727371443582e-08
4505 1.7383552048611e-08
4506 2.06200365665266e-08
4507 1.61572906929586e-08
4508 1.7138687067153e-08
4509 2.05431660378064e-08
4510 1.59659500680309e-08
4511 2.01374163938672e-08
4512 1.61748815543372e-08
4513 1.93912076240088e-08
4514 1.64111783918508e-08
4515 1.87081234981623e-08
4516 1.66243035696345e-08
4517 1.82222251238762e-08
4518 1.65570549808813e-08
4519 1.82783223601768e-08
4520 1.69722753177171e-08
4521 1.80943550765755e-08
4522 1.69083001703707e-08
4523 1.68722235259211e-08
4524 2.08663341810522e-08
4525 1.64487306657246e-08
4526 1.85276778510157e-08
4527 1.64573281295211e-08
4528 2.02538066862434e-08
4529 1.66024668490361e-08
4530 1.78194055238112e-08
4531 1.66308656227576e-08
4532 1.83275995227772e-08
4533 1.65615312989476e-08
4534 1.85316926668289e-08
4535 1.70657863604595e-08
4536 1.70371743820619e-08
4537 2.04976862573092e-08
4538 1.64681365129593e-08
4539 1.88982885895028e-08
4540 1.6452421492813e-08
4541 1.85677343251534e-08
4542 1.67197554745424e-08
4543 1.69428128258542e-08
4544 2.04197335610945e-08
4545 1.62839207553944e-08
4546 1.9835230218368e-08
4547 1.61415520621155e-08
4548 1.95643842945403e-08
4549 1.6315354999602e-08
4550 1.97274103722056e-08
4551 1.6387393581313e-08
4552 1.94753349124777e-08
4553 1.60899176692153e-08
4554 1.90498222034374e-08
4555 1.63765221054268e-08
4556 2.02510016074353e-08
4557 1.59509667476965e-08
4558 1.72071517818284e-08
4559 2.04577120516691e-08
4560 1.5903306957582e-08
4561 1.84760080141066e-08
4562 1.79429520432983e-08
4563 1.77910482787869e-08
4564 1.71600558779295e-08
4565 2.01412774855914e-08
4566 1.59562872660834e-08
4567 1.91688091405462e-08
4568 1.65227329655682e-08
4569 1.68553630897628e-08
4570 2.04100475419133e-08
4571 1.58356005942883e-08
4572 1.96092107189461e-08
4573 1.62196336530029e-08
4574 1.71433589744585e-08
4575 1.98213121714585e-08
4576 1.59477976610367e-08
4577 1.85880308001263e-08
4578 1.67877456512533e-08
4579 1.97727893873345e-08
4580 1.59867221063148e-08
4581 1.7448180684998e-08
4582 1.90327497555931e-08
4583 1.62063948271896e-08
4584 2.01805212426964e-08
4585 1.59449501176689e-08
4586 1.83615162460682e-08
4587 1.80189802196362e-08
4588 1.659004516949e-08
4589 1.72358128499861e-08
4590 1.65748891103601e-08
4591 1.82509787667595e-08
4592 1.66403728913966e-08
4593 1.65852639673336e-08
4594 1.96243950968678e-08
4595 1.64909211735897e-08
4596 1.7721641331625e-08
4597 1.80994749842955e-08
4598 1.6477854152902e-08
4599 1.73565044480073e-08
4600 1.79939668079887e-08
4601 1.64233595572327e-08
4602 1.68087787626447e-08
4603 1.86280807965566e-08
4604 1.63743340975686e-08
4605 1.96044370784632e-08
4606 1.69979080426041e-08
4607 1.63484662473456e-08
4608 2.04982288055933e-08
4609 1.58661026322671e-08
4610 1.92747247730052e-08
4611 1.61324106280891e-08
4612 1.96301539553134e-08
4613 1.65781608339743e-08
4614 1.66394927555158e-08
4615 1.8024085895324e-08
4616 1.82256015298365e-08
4617 1.6141513146023e-08
4618 2.08656326110279e-08
4619 1.57381919927424e-08
4620 1.92259919208349e-08
4621 1.60505202289329e-08
4622 1.78699990956432e-08
4623 1.79276173955245e-08
4624 1.64898403341163e-08
4625 2.03976769342362e-08
4626 1.56961959732194e-08
4627 1.98050044785958e-08
4628 1.58045516005512e-08
4629 1.93478473918818e-08
4630 1.61912065955727e-08
4631 1.6527434925806e-08
4632 1.99974699532646e-08
4633 1.56291058996672e-08
4634 1.69133595293025e-08
4635 2.02590315713513e-08
4636 1.55815584430841e-08
4637 1.82119770459066e-08
4638 1.76857844242928e-08
4639 1.81372987997919e-08
4640 1.76118715500828e-08
4641 1.74571598576534e-08
4642 1.67765414889609e-08
4643 2.03860954011148e-08
4644 1.56579326552608e-08
4645 1.96014280016144e-08
4646 1.60412538822707e-08
4647 1.77943397776659e-08
4648 1.70084607991061e-08
4649 2.01866067529588e-08
4650 1.57971703623472e-08
4651 1.95173252981185e-08
4652 1.58536808411436e-08
4653 1.66608480566266e-08
4654 1.67775996214592e-08
4655 1.69639652628784e-08
4656 1.82299889927018e-08
4657 1.69284195359609e-08
4658 1.65395964771831e-08
4659 1.69995838490977e-08
4660 1.83568431848247e-08
4661 1.76663120023401e-08
4662 1.66201915648145e-08
4663 1.81409240861408e-08
4664 1.63919953594382e-08
4665 1.86004055452438e-08
4666 1.61142676262938e-08
4667 1.99094553883228e-08
4668 1.71108500524597e-08
4669 1.68569111522743e-08
4670 1.66133618600961e-08
4671 2.04566319166044e-08
4672 1.57903309739582e-08
4673 1.68676974630078e-08
4674 1.98782965828148e-08
4675 1.56787048704587e-08
4676 1.95176507066797e-08
4677 1.56813083659058e-08
4678 1.91224009246926e-08
4679 1.59457974275301e-08
4680 1.88092786711502e-08
4681 1.59942835408489e-08
4682 1.81385056462202e-08
4683 1.66171066253162e-08
4684 1.97476174892963e-08
4685 1.57344866760778e-08
4686 1.95655145481255e-08
4687 1.56226083787847e-08
4688 1.70503182330139e-08
4689 1.95846761705809e-08
4690 1.56654953214486e-08
4691 1.66934399498664e-08
4692 2.00907855950061e-08
4693 1.58349635258626e-08
4694 1.86479979052689e-08
4695 1.64821329809162e-08
4696 1.68269860515158e-08
4697 1.78891349219557e-08
4698 1.6563021972088e-08
4699 1.93367198958461e-08
4700 1.62285847934152e-08
4701 1.86764576649012e-08
4702 1.63229432391387e-08
4703 1.96588481166382e-08
4704 1.67171268301014e-08
4705 1.66569260555971e-08
4706 1.71239879863394e-08
4707 1.82282449739946e-08
4708 1.6686095048235e-08
4709 1.68763921839232e-08
4710 1.80393906543119e-08
4711 1.73225157014123e-08
4712 1.73266105276404e-08
4713 1.65612437124096e-08
4714 1.93987675685681e-08
4715 1.75228059583199e-08
4716 1.66486873910965e-08
4717 1.80407058743837e-08
4718 1.67247105389845e-08
4719 1.8063274186686e-08
4720 1.63129651658545e-08
4721 1.94437756181054e-08
4722 1.62679078654637e-08
4723 1.93928178436054e-08
4724 1.59615957888437e-08
4725 1.97944535855477e-08
4726 1.56505277613406e-08
4727 1.81162103407573e-08
4728 1.66026308005274e-08
4729 1.91371176429533e-08
4730 1.65869472129321e-08
4731 1.85358239737088e-08
4732 1.67401312289039e-08
4733 1.66710888601929e-08
4734 2.04179432510831e-08
4735 1.57579322465906e-08
4736 1.68028829050093e-08
4737 1.8617135426946e-08
4738 1.77465784985675e-08
4739 1.63375445445257e-08
4740 1.92123077765172e-08
4741 1.64347515110852e-08
4742 1.67826869547638e-08
4743 1.94305977435361e-08
4744 1.57378447833978e-08
4745 1.95687131833988e-08
4746 1.58874307275847e-08
4747 2.04909889932048e-08
4748 1.58453677170933e-08
4749 1.74357046728024e-08
4750 1.77151687672239e-08
4751 1.66746190399802e-08
4752 2.00016173719508e-08
4753 1.56635749634093e-08
4754 1.83801737276001e-08
4755 1.62880519629094e-08
4756 1.84718071579992e-08
4757 1.65091475068935e-08
4758 1.99640424822789e-08
4759 1.57192663242656e-08
4760 1.86316913177997e-08
4761 1.62692829415445e-08
4762 1.89696693657371e-08
4763 1.60905419595014e-08
4764 1.82692539008877e-08
4765 1.64140281603831e-08
4766 1.91705729368497e-08
4767 1.67788770621957e-08
4768 1.66736289056724e-08
4769 1.90496850870636e-08
4770 1.77471734418844e-08
4771 1.6318733718379e-08
4772 1.93530727437552e-08
4773 1.60690578249467e-08
4774 1.83249326140278e-08
4775 1.66449085859388e-08
4776 1.88442908582997e-08
4777 1.71835940538934e-08
4778 1.63672961976669e-08
4779 1.99149920367125e-08
4780 1.57273246366163e-08
4781 1.90568879916697e-08
4782 1.63825056626599e-08
4783 2.02410184426405e-08
4784 1.63354275909966e-08
4785 1.65196898822551e-08
4786 1.73275419013408e-08
4787 1.96667774824688e-08
4788 1.55250464279888e-08
4789 1.89845196062743e-08
4790 1.63727368349409e-08
4791 1.79596361375989e-08
4792 1.59994811632924e-08
4793 1.97884261169123e-08
4794 1.61719360654011e-08
4795 1.92091381803761e-08
4796 1.56807381753443e-08
4797 1.85691765508256e-08
4798 1.6856434395307e-08
4799 1.91194492699109e-08
4800 1.57871598633341e-08
4801 1.81270843739589e-08
4802 1.59293034674157e-08
4803 1.97333598062621e-08
4804 1.57183040007669e-08
4805 1.69088052919819e-08
4806 1.96804980887311e-08
4807 1.55662045783489e-08
4808 1.90941793552779e-08
4809 1.58314524503211e-08
4810 1.70092234122177e-08
4811 1.898977091222e-08
4812 1.57907362737797e-08
4813 1.97567958619138e-08
4814 1.57139414190821e-08
4815 1.97685515486157e-08
4816 1.55160912839736e-08
4817 1.96424029858255e-08
4818 1.60016817870035e-08
4819 1.67696110482396e-08
4820 1.93632402500088e-08
4821 1.60547948494794e-08
4822 1.95281267956482e-08
4823 1.55625195895348e-08
4824 1.82804093167566e-08
4825 1.74771649644467e-08
4826 1.79034641018538e-08
4827 1.79969975262206e-08
4828 1.65650615325807e-08
4829 1.96506786519779e-08
4830 1.78880184023289e-08
4831 1.61064272932165e-08
4832 1.90087696037844e-08
4833 1.84009224735437e-08
4834 1.55224823406852e-08
4835 2.04588556161622e-08
4836 1.58732713437248e-08
4837 1.73396639829293e-08
4838 1.84841296460569e-08
4839 1.79670666586029e-08
4840 1.59753461954348e-08
4841 1.94646596520243e-08
4842 1.59096227589361e-08
4843 1.77503763579556e-08
4844 1.65864847505304e-08
4845 1.9499634765191e-08
4846 1.5880149300096e-08
4847 1.77407942667795e-08
4848 1.64551847436079e-08
4849 1.9755145281547e-08
4850 1.54790944395244e-08
4851 1.91434378244704e-08
4852 1.67707467416123e-08
4853 1.60462137859529e-08
4854 1.97262337183479e-08
4855 1.52489355091101e-08
4856 1.8097529167177e-08
4857 1.77916414754592e-08
4858 1.62447747153738e-08
4859 1.92029008847794e-08
4860 1.54838750841546e-08
4861 1.91866437931065e-08
4862 1.7295349423796e-08
4863 1.64001729832525e-08
4864 2.00075725376586e-08
4865 1.66528145743583e-08
4866 1.71021652245334e-08
4867 1.87569283292555e-08
4868 1.57074734137008e-08
4869 1.99584204850223e-08
4870 1.67817390455682e-08
4871 1.60737461975952e-08
4872 2.02012885420488e-08
4873 1.53193162244214e-08
4874 1.79104013448839e-08
4875 1.83653784066873e-08
4876 1.59684626470924e-08
4877 2.00176356508208e-08
4878 1.49440579596538e-08
4879 1.92589184456748e-08
4880 1.55660893209064e-08
4881 1.99286823999306e-08
4882 1.52809842093871e-08
4883 1.95122850329549e-08
4884 1.53300857554384e-08
4885 1.99233962383666e-08
4886 1.51722333556481e-08
4887 1.75984014934433e-08
4888 1.83774365042633e-08
4889 1.57387388818181e-08
4890 1.82941053853969e-08
4891 1.57690715948122e-08
4892 1.88682973672472e-08
4893 1.58399196108494e-08
4894 1.86916834732098e-08
4895 1.5792227191419e-08
4896 1.92374112550353e-08
4897 1.53342947562252e-08
4898 2.00419274149211e-08
4899 1.53256490983966e-08
4900 1.94175080652781e-08
4901 1.55866999240195e-08
4902 2.0002320103879e-08
4903 1.55952063710107e-08
4904 1.94355533617019e-08
4905 1.53654521510005e-08
4906 1.74100460514914e-08
4907 1.88187024163511e-08
4908 1.5469330915785e-08
4909 1.90128899230368e-08
4910 1.57738052226908e-08
4911 1.97672393723591e-08
4912 1.52232821827858e-08
4913 1.87379149201861e-08
4914 1.57070501291545e-08
4915 1.99632346983558e-08
4916 1.58437047504678e-08
4917 1.63938164055855e-08
4918 1.89684835728565e-08
4919 1.52985115136339e-08
4920 1.99540336829063e-08
4921 1.53924151184703e-08
4922 1.776857708502e-08
4923 1.76469854955008e-08
4924 1.60580458551662e-08
4925 2.04709999229447e-08
4926 1.51729117106869e-08
4927 1.93965757104286e-08
4928 1.5209395224175e-08
4929 2.00627282961607e-08
4930 1.52716112376827e-08
4931 1.97159725343621e-08
4932 1.52958341874954e-08
4933 1.84265845855625e-08
4934 1.60669420121162e-08
4935 1.92622171991674e-08
4936 1.57347450179535e-08
4937 1.78809806896385e-08
4938 1.87154895557318e-08
4939 1.53458601965806e-08
4940 1.78076799255689e-08
4941 1.65305109732228e-08
4942 1.98683262528643e-08
4943 1.50249345359232e-08
4944 1.8554340998761e-08
4945 1.59358624915451e-08
4946 1.84005105920149e-08
4947 1.7266246325004e-08
4948 1.74840155779121e-08
4949 1.73875440915849e-08
4950 1.75401157955501e-08
4951 1.71630156658997e-08
4952 1.76049238117537e-08
4953 1.68046519222198e-08
4954 1.74723180723368e-08
4955 1.71409285373569e-08
4956 1.71856549830152e-08
4957 1.62409659302809e-08
4958 2.04382906551503e-08
4959 1.49381045181363e-08
4960 1.96492022394856e-08
4961 1.57608125126996e-08
4962 1.76765980498539e-08
4963 1.59262474846433e-08
4964 1.61048695790644e-08
4965 2.00647359544504e-08
4966 1.62633671123569e-08
4967 1.70153131267559e-08
4968 1.73377989216572e-08
4969 1.78307296370139e-08
4970 1.61140302287754e-08
4971 2.02701087448354e-08
4972 1.48414166364352e-08
4973 1.8911982888753e-08
4974 1.54742600732005e-08
4975 1.88077765085337e-08
4976 1.58885798261788e-08
4977 1.92581477018172e-08
4978 1.56269844167889e-08
4979 1.83827741516429e-08
4980 1.66953241162171e-08
4981 1.80676422885107e-08
4982 1.58168564981243e-08
4983 1.93508372982321e-08
4984 1.52264352592801e-08
4985 1.74026341080158e-08
4986 1.80195431300767e-08
4987 1.60993272914356e-08
4988 1.92850391783628e-08
4989 1.48703462496957e-08
4990 1.76929826222472e-08
4991 1.80719763670512e-08
4992 1.6036856031687e-08
4993 1.94080066795721e-08
4994 1.51599391199908e-08
4995 1.90338312084093e-08
4996 1.5493081115564e-08
4997 1.92603721155538e-08
4998 1.53256365367505e-08
4999 1.8959963760723e-08
};
\addlegendentry{Train}
\addplot [semithick, black]
table {%
0 0.00019669190805871
1 0.000121803539514076
2 5.06477190356236e-05
3 3.08722737827338e-05
4 2.61585664702579e-05
5 2.17200340557611e-05
6 1.63250897458056e-05
7 1.07491596281761e-05
8 6.86548946760013e-06
9 5.32681042386685e-06
10 4.70896475235349e-06
11 4.25185999120004e-06
12 3.81059498977265e-06
13 3.42357270710636e-06
14 3.0652172426926e-06
15 2.76353875960922e-06
16 2.50644848165393e-06
17 2.2880246888235e-06
18 2.11424685403472e-06
19 1.97893518816272e-06
20 1.86458703410608e-06
21 1.7763461528375e-06
22 1.70534144672274e-06
23 1.63794641139248e-06
24 1.58007946993166e-06
25 1.51537722103967e-06
26 1.4685064115838e-06
27 1.42226144816959e-06
28 1.37800543598132e-06
29 1.33905484744901e-06
30 1.30291061850585e-06
31 1.27347300349356e-06
32 1.23451479794312e-06
33 1.20161973882205e-06
34 1.1709975069607e-06
35 1.14199951894989e-06
36 1.11520557766198e-06
37 1.08497602013813e-06
38 1.05993512988789e-06
39 1.03600996226305e-06
40 1.01176874522935e-06
41 9.88412352853629e-07
42 9.66812194747035e-07
43 9.4633730896021e-07
44 9.2763804104834e-07
45 9.10723372271605e-07
46 8.93399374035653e-07
47 8.79878882642515e-07
48 8.58876319398405e-07
49 8.44101862185198e-07
50 8.3004601947323e-07
51 8.17074919723382e-07
52 8.04196019998926e-07
53 7.90984643117554e-07
54 7.79875733769586e-07
55 7.68315089771932e-07
56 7.57242901272548e-07
57 7.46835098652809e-07
58 7.39182269171579e-07
59 7.29650878383836e-07
60 7.19781212410453e-07
61 7.10009942395118e-07
62 6.83565986037138e-07
63 6.70554300086224e-07
64 6.60068110391876e-07
65 6.50958838832594e-07
66 6.45066677407158e-07
67 6.45306784008426e-07
68 6.66406776872464e-07
69 6.46018747829658e-07
70 6.3238297798307e-07
71 6.22494951585395e-07
72 6.1007909835098e-07
73 6.00780367676634e-07
74 5.90571517022909e-07
75 5.78754963953543e-07
76 5.70562406210229e-07
77 5.63569301448297e-07
78 5.57112400656479e-07
79 5.50790502984455e-07
80 5.45036584753689e-07
81 5.3822941481485e-07
82 5.31745740772749e-07
83 5.26991868810001e-07
84 5.2073067990932e-07
85 5.15669739797886e-07
86 5.12023177634546e-07
87 5.08000994159374e-07
88 5.03669241425087e-07
89 5.00305759487674e-07
90 4.96625034429599e-07
91 4.93036907300848e-07
92 4.88532805320574e-07
93 4.85360544644209e-07
94 4.79266532238398e-07
95 4.75067821525954e-07
96 4.71261444090487e-07
97 4.67321740416082e-07
98 4.62274499568593e-07
99 4.57727622915627e-07
100 4.53780074849419e-07
101 4.42873442807468e-07
102 4.38599613516999e-07
103 4.34502823054572e-07
104 4.30399040851626e-07
105 4.25994869601709e-07
106 4.21961374286184e-07
107 4.17575989786201e-07
108 4.1323639266011e-07
109 4.08979644817009e-07
110 4.07620973419398e-07
111 4.03287288008869e-07
112 3.99004449036511e-07
113 3.94794540170551e-07
114 3.91021615087084e-07
115 3.87619849107068e-07
116 3.83799232395177e-07
117 3.80299979951815e-07
118 3.76193355577925e-07
119 3.72495748024448e-07
120 3.689648906402e-07
121 3.65434686955268e-07
122 3.61203404963817e-07
123 3.57727657274154e-07
124 3.54482011744039e-07
125 3.51345249782753e-07
126 3.48507626313221e-07
127 3.45741113960685e-07
128 3.42637406447466e-07
129 3.3960088785534e-07
130 3.36271085643602e-07
131 3.33206344294013e-07
132 3.30194694697639e-07
133 3.27111052911278e-07
134 3.2414149586657e-07
135 3.21167874517414e-07
136 3.18960985623562e-07
137 3.16335416528091e-07
138 3.13998469891885e-07
139 3.1232963237926e-07
140 3.09894375050135e-07
141 3.0810528528491e-07
142 3.04922650684603e-07
143 3.01909238942244e-07
144 2.98930132203168e-07
145 2.96136704491801e-07
146 2.93448636057292e-07
147 2.9079862429171e-07
148 2.88481174948174e-07
149 2.85843668734742e-07
150 2.83400538592105e-07
151 2.81155138281974e-07
152 2.76395326181955e-07
153 2.77354814670616e-07
154 2.75216365253073e-07
155 2.72825161573564e-07
156 2.7032947969019e-07
157 2.68040906803435e-07
158 2.66045844909968e-07
159 2.64306436292827e-07
160 2.62890040403363e-07
161 2.60755683711977e-07
162 2.58639033745567e-07
163 2.56108194207627e-07
164 2.54142804578805e-07
165 2.51803413675589e-07
166 2.49918599593002e-07
167 2.48296345262133e-07
168 2.46638137468835e-07
169 2.45305329826806e-07
170 2.42480496126518e-07
171 2.4088265604405e-07
172 2.39626189113551e-07
173 2.38264917129527e-07
174 2.36696678257431e-07
175 2.3481483424348e-07
176 2.33771942248495e-07
177 2.32439674618945e-07
178 2.30499310305277e-07
179 2.29630614967391e-07
180 2.2762228013562e-07
181 2.26402889325072e-07
182 2.25073307547063e-07
183 2.23918050323846e-07
184 2.2230665308598e-07
185 2.21458819282816e-07
186 2.2059690252263e-07
187 2.19742673834844e-07
188 2.18581135413842e-07
189 2.17038476080234e-07
190 2.15228041611226e-07
191 2.12976317470748e-07
192 2.11479147083082e-07
193 2.09751490842791e-07
194 2.08269483437107e-07
195 2.06940626412688e-07
196 2.05457169499823e-07
197 2.04376689794117e-07
198 2.03110360530445e-07
199 2.01918098241549e-07
200 2.00769235902953e-07
201 1.99489861074653e-07
202 1.98388860894738e-07
203 1.96571505739485e-07
204 1.95589393570117e-07
205 1.94606116110663e-07
206 1.93789986724369e-07
207 1.9295636377592e-07
208 1.92159873790843e-07
209 1.91284215134147e-07
210 1.90278939271593e-07
211 1.89454254950761e-07
212 1.88519038601953e-07
213 1.87697736464543e-07
214 1.86803561064153e-07
215 1.85648246997516e-07
216 1.85101015404143e-07
217 1.84363997846049e-07
218 1.827632019058e-07
219 1.81896126605352e-07
220 1.81109541586011e-07
221 1.80346233946693e-07
222 1.79656240106851e-07
223 1.7973691512907e-07
224 1.77911232412953e-07
225 1.77087045472035e-07
226 1.75970015448002e-07
227 1.75655358702898e-07
228 1.74756806359255e-07
229 1.74382705608878e-07
230 1.73979003648128e-07
231 1.72955921584617e-07
232 1.72172974544083e-07
233 1.73826521177034e-07
234 1.71299916473799e-07
235 1.70263803056514e-07
236 1.68922227317125e-07
237 1.68769247466116e-07
238 1.67985405141735e-07
239 1.67888416058304e-07
240 1.67816878615668e-07
241 1.67876564205471e-07
242 1.67335272749369e-07
243 1.6723244300465e-07
244 1.66905607557055e-07
245 1.66602120543757e-07
246 1.66129140666271e-07
247 1.65006056818129e-07
248 1.64809392799725e-07
249 1.64609147645933e-07
250 1.64477185649048e-07
251 1.66384054978153e-07
252 1.63881409775968e-07
253 1.64056743301444e-07
254 1.64028563176544e-07
255 1.64274609915083e-07
256 1.63178697221156e-07
257 1.63094568961242e-07
258 1.60793192094388e-07
259 1.61189518621541e-07
260 1.6100693756016e-07
261 1.59796215370989e-07
262 1.60338785804015e-07
263 1.59404905275551e-07
264 1.5919935947295e-07
265 1.5897431637768e-07
266 1.58632431634942e-07
267 1.58233476099667e-07
268 1.58127349436654e-07
269 1.57114158128024e-07
270 1.55895449438503e-07
271 1.56855207933404e-07
272 1.55480748276204e-07
273 1.5545029441455e-07
274 1.55485579966808e-07
275 1.55580138994083e-07
276 1.54475600311343e-07
277 1.55880869101566e-07
278 1.54039753397228e-07
279 1.55499364495881e-07
280 1.53484123188719e-07
281 1.53273276737309e-07
282 1.5356863514171e-07
283 1.53155056636933e-07
284 1.54313497091607e-07
285 1.56969505837878e-07
286 1.57508395659534e-07
287 1.57477003881468e-07
288 1.58701723762533e-07
289 1.57025979774517e-07
290 1.5810063302979e-07
291 1.5810883269296e-07
292 1.57604048922622e-07
293 1.59198904725599e-07
294 1.58174628950292e-07
295 1.58159082275233e-07
296 1.59308029878957e-07
297 1.59285661993636e-07
298 1.59835678914533e-07
299 1.59350378226009e-07
300 1.59618579687049e-07
301 1.58873092459544e-07
302 1.58682212259009e-07
303 1.5949176201957e-07
304 1.58176419517986e-07
305 1.57744040052421e-07
306 1.58311920017695e-07
307 1.58002535499691e-07
308 1.56944523155289e-07
309 1.57018206436987e-07
310 1.56316971811066e-07
311 1.54929466589238e-07
312 1.56356108504951e-07
313 1.56148047381066e-07
314 1.56440691512216e-07
315 1.56187397237773e-07
316 1.54396914808785e-07
317 1.552307367092e-07
318 1.55043707650293e-07
319 1.5552313925582e-07
320 1.55131289147903e-07
321 1.64170870675662e-07
322 1.65912368288446e-07
323 1.65344829383685e-07
324 1.64534867508337e-07
325 1.64666417390436e-07
326 1.64511149591817e-07
327 1.63691439070135e-07
328 1.63771360917053e-07
329 1.63716293855032e-07
330 1.60638961688164e-07
331 1.63758954840887e-07
332 1.61213591809428e-07
333 1.62655510393961e-07
334 1.60419261874267e-07
335 1.61471504611654e-07
336 1.627782921787e-07
337 1.62372700174274e-07
338 1.61033952394973e-07
339 1.60556382411414e-07
340 1.61893964900628e-07
341 1.61337084136903e-07
342 1.61027330136676e-07
343 1.62245555657137e-07
344 1.60726656872612e-07
345 1.59944136157719e-07
346 1.59120176590477e-07
347 1.60415510208622e-07
348 1.60562677820053e-07
349 1.60350950295651e-07
350 1.60027099127547e-07
351 1.59688312351136e-07
352 1.59594733872837e-07
353 1.60349671318727e-07
354 1.5576861756017e-07
355 1.60410991156823e-07
356 1.55689377834278e-07
357 1.58727502252987e-07
358 1.58217929424609e-07
359 1.5271737652256e-07
360 1.58485377710349e-07
361 1.53208759456902e-07
362 1.52522417806722e-07
363 1.56895254121991e-07
364 1.55536312718141e-07
365 1.52248233575847e-07
366 1.57855751581337e-07
367 1.55896657361154e-07
368 1.52427418242951e-07
369 1.51962211702994e-07
370 1.5145593579291e-07
371 1.51548789517619e-07
372 1.50277173815994e-07
373 1.53253296275579e-07
374 1.5023096011646e-07
375 1.49019442119425e-07
376 1.49045192188169e-07
377 1.55809843249699e-07
378 1.545381564938e-07
379 1.47621932455877e-07
380 1.54116463590981e-07
381 1.53003014702335e-07
382 1.53477188291617e-07
383 1.47219466839488e-07
384 1.53182938333885e-07
385 1.51732109543445e-07
386 1.46853054161511e-07
387 1.51185460595116e-07
388 1.49366911728066e-07
389 1.49118989156705e-07
390 1.46750096519099e-07
391 1.47071816058997e-07
392 1.44067257679126e-07
393 1.45287017971896e-07
394 1.43601084801048e-07
395 1.39848111757601e-07
396 1.39040395197298e-07
397 1.38584482556325e-07
398 1.39905935725437e-07
399 1.42649113854532e-07
400 1.4285450333773e-07
401 1.44156032888532e-07
402 1.44606701724115e-07
403 1.43812101782714e-07
404 1.38578869268713e-07
405 1.43415661568724e-07
406 1.44092481946245e-07
407 1.43929028695311e-07
408 1.44027154647119e-07
409 1.37784951448339e-07
410 1.36879336309903e-07
411 1.41683429433215e-07
412 1.40040214091641e-07
413 1.4022839422978e-07
414 1.40114465807528e-07
415 1.39737196036549e-07
416 1.4058240083159e-07
417 1.3548759625337e-07
418 1.35173564785873e-07
419 1.34930061790328e-07
420 1.33642998889627e-07
421 1.34433889797947e-07
422 1.34397055262525e-07
423 1.33582574335378e-07
424 1.3340958560093e-07
425 1.35054563088488e-07
426 1.33669786350765e-07
427 1.33790720724392e-07
428 1.32111324546713e-07
429 1.32133763486308e-07
430 1.37565649538374e-07
431 1.36660773364383e-07
432 1.36595289745856e-07
433 1.37402111022311e-07
434 1.3603370518922e-07
435 1.31133617742307e-07
436 1.32453592982529e-07
437 1.29878571897279e-07
438 1.33479730379804e-07
439 1.30704421508199e-07
440 1.29360500977782e-07
441 1.29353679767519e-07
442 1.30067078885077e-07
443 1.30277541643409e-07
444 1.27477036926393e-07
445 1.29690789663073e-07
446 1.29186574326923e-07
447 1.26784470921848e-07
448 1.27285233020302e-07
449 1.264889704089e-07
450 1.26768910035935e-07
451 1.26757072393957e-07
452 1.23596876733245e-07
453 1.24597633543999e-07
454 1.2295271289986e-07
455 1.23581628486136e-07
456 1.16260451932249e-07
457 1.16696512009185e-07
458 1.20045953622139e-07
459 1.17997565496353e-07
460 1.15753117313488e-07
461 1.18615886890439e-07
462 1.17812319899713e-07
463 1.15955302248949e-07
464 1.11581648809533e-07
465 1.15311721060607e-07
466 1.17699435975283e-07
467 1.16452888221374e-07
468 1.16872662658807e-07
469 1.1588815596042e-07
470 1.13118602484974e-07
471 1.1357975182591e-07
472 1.15960588686903e-07
473 1.16925065185569e-07
474 1.15630783170673e-07
475 1.12145428943222e-07
476 1.1432344848572e-07
477 1.1247863085373e-07
478 1.15618362883652e-07
479 1.14956193897342e-07
480 1.14458238442694e-07
481 1.13554648351055e-07
482 1.14467539447105e-07
483 1.12907493132752e-07
484 1.11160332494364e-07
485 1.12962645459902e-07
486 1.12826420206602e-07
487 1.11745549702391e-07
488 1.11823908355291e-07
489 1.11425208615401e-07
490 1.11639636202199e-07
491 1.1043054826132e-07
492 1.1195005100717e-07
493 1.0976042119637e-07
494 1.1087525564335e-07
495 1.10233145278471e-07
496 1.09433457851082e-07
497 1.07389105608036e-07
498 1.08089189154725e-07
499 1.05844655706733e-07
500 1.05469240452294e-07
501 1.05777665737605e-07
502 1.05796075899889e-07
503 1.07389418246839e-07
504 1.04531650890749e-07
505 1.05590046928228e-07
506 1.04316704607754e-07
507 1.04775793374756e-07
508 1.04784767529509e-07
509 1.0544273720825e-07
510 1.03060969536273e-07
511 1.03715258603643e-07
512 1.02903918275388e-07
513 1.03144721208537e-07
514 1.01888090853208e-07
515 1.01390448037364e-07
516 1.02121617828743e-07
517 1.01539200159095e-07
518 1.01296592447397e-07
519 1.00926897061981e-07
520 1.00227367738626e-07
521 1.00359130783545e-07
522 9.91911193182204e-08
523 9.91673516637093e-08
524 9.87359456416925e-08
525 9.67326485579179e-08
526 9.75968603711408e-08
527 9.70598179605986e-08
528 9.53447454321577e-08
529 9.44358475862828e-08
530 9.33933606006576e-08
531 9.313502147279e-08
532 9.45859710554942e-08
533 9.31871895204495e-08
534 9.48459515370814e-08
535 9.25781051819285e-08
536 9.1436739069195e-08
537 9.13280331360511e-08
538 9.10804800469123e-08
539 9.06196646610624e-08
540 9.18134048788488e-08
541 9.12348951942477e-08
542 8.94485197022732e-08
543 8.96600909072731e-08
544 8.98293492923585e-08
545 9.00244430113162e-08
546 9.10730904024604e-08
547 8.94139589036058e-08
548 9.07762398583145e-08
549 8.89151081651107e-08
550 8.78887576050147e-08
551 8.74887433610638e-08
552 8.75427943469731e-08
553 8.66258318410473e-08
554 8.58183923924116e-08
555 8.54549213613609e-08
556 8.57916120367008e-08
557 8.56516209069014e-08
558 8.53217301255427e-08
559 8.50712922328967e-08
560 8.48316261681248e-08
561 8.49834407290473e-08
562 8.47094980827023e-08
563 8.45124645820761e-08
564 8.3823302077235e-08
565 8.37856077851029e-08
566 8.36789411096106e-08
567 8.44105656483407e-08
568 8.29580741878999e-08
569 8.33072419936798e-08
570 8.26986621405013e-08
571 8.21973884512772e-08
572 8.22096168917597e-08
573 8.22872507910688e-08
574 8.17825878129952e-08
575 8.17090963778355e-08
576 8.16349299270769e-08
577 8.1596105871995e-08
578 8.11855542792728e-08
579 8.23009145278775e-08
580 8.07521729484506e-08
581 8.20417369595816e-08
582 8.17844991729544e-08
583 8.12101603742121e-08
584 7.99757202685214e-08
585 7.98741339735898e-08
586 8.05481761290139e-08
587 8.05181841201374e-08
588 8.14270535443029e-08
589 7.99253072614192e-08
590 7.94520786939756e-08
591 7.95845025436392e-08
592 7.88703573562088e-08
593 7.89920875376993e-08
594 7.77390098960495e-08
595 7.89399621226039e-08
596 7.84006388698799e-08
597 7.83401077342205e-08
598 7.96866146401953e-08
599 7.76551303260931e-08
600 7.74711423900953e-08
601 7.82157485446078e-08
602 7.73826300815017e-08
603 7.75201556280081e-08
604 7.70851826814578e-08
605 7.71291510659466e-08
606 7.89588128213836e-08
607 7.90207508316598e-08
608 7.89330769634944e-08
609 7.85971607797364e-08
610 7.83942653015401e-08
611 7.91923469023459e-08
612 7.90888066148909e-08
613 7.89564111869367e-08
614 7.87414222713778e-08
615 7.86059430879504e-08
616 7.84488136673644e-08
617 7.77104602889267e-08
618 7.87705687343987e-08
619 7.80077940021329e-08
620 7.76216992903755e-08
621 7.63005942872041e-08
622 7.61379652658434e-08
623 7.66182992606446e-08
624 7.56719273908857e-08
625 7.66797114692963e-08
626 7.60399743171547e-08
627 7.56274758373365e-08
628 7.57532419015661e-08
629 7.50807203075965e-08
630 7.55981588440591e-08
631 7.52766524669823e-08
632 7.53159952182614e-08
633 7.47568407177823e-08
634 7.43488755006183e-08
635 7.415144409606e-08
636 7.40541139521156e-08
637 7.34859071371829e-08
638 7.3776362796707e-08
639 7.33048040046924e-08
640 7.31293923195153e-08
641 7.29748919070516e-08
642 7.30577625063233e-08
643 7.27743199036013e-08
644 7.31134477405249e-08
645 7.31739433490475e-08
646 7.19472339483218e-08
647 7.15155508146381e-08
648 7.21647879231568e-08
649 7.11140160092327e-08
650 7.12125540758279e-08
651 7.05310583271057e-08
652 7.07382241671439e-08
653 7.09116747543703e-08
654 7.17959167673143e-08
655 7.16319377147556e-08
656 7.09188938685656e-08
657 7.04949982832659e-08
658 7.11299747990779e-08
659 7.00557194477369e-08
660 7.12181957851499e-08
661 6.96435193958678e-08
662 7.1098583021012e-08
663 6.95579700504823e-08
664 7.09130603127051e-08
665 6.93009241103937e-08
666 7.06525895566301e-08
667 6.85041143810849e-08
668 7.03132343460311e-08
669 6.9509582090177e-08
670 6.87413717059826e-08
671 6.92913246780336e-08
672 6.89806967102413e-08
673 6.88105785684456e-08
674 6.85379362153071e-08
675 6.85348879869707e-08
676 6.84376644244367e-08
677 6.81198173424491e-08
678 6.73791191729833e-08
679 6.80118219520409e-08
680 6.73822171393113e-08
681 6.68817108362418e-08
682 6.80069618397283e-08
683 6.74814728540696e-08
684 6.65911699115895e-08
685 6.82589060829741e-08
686 6.72195170636769e-08
687 6.60986501088701e-08
688 6.71926869699746e-08
689 6.68400232939348e-08
690 6.63665460365337e-08
691 6.64148487317107e-08
692 6.56216272432175e-08
693 6.41766106923569e-08
694 6.62997621247996e-08
695 6.57178276242121e-08
696 6.35334558296563e-08
697 6.35878265597967e-08
698 6.23241689368115e-08
699 6.33474996902805e-08
700 6.35890202715927e-08
701 6.40770565496496e-08
702 6.22903684188714e-08
703 6.1781079807588e-08
704 6.13746493627332e-08
705 6.12101800356868e-08
706 6.06558216986741e-08
707 6.17138695702124e-08
708 6.1795446981705e-08
709 5.97658811329893e-08
710 6.03946546107181e-08
711 6.03825327516461e-08
712 6.1194626255201e-08
713 6.12255348642066e-08
714 6.06221917109906e-08
715 6.03526757458894e-08
716 6.0434331317083e-08
717 6.02187881781902e-08
718 6.01987935056059e-08
719 5.98955764985476e-08
720 5.94624935956745e-08
721 5.98899276837983e-08
722 6.01751466433598e-08
723 5.89207083123711e-08
724 5.98222698044992e-08
725 6.18892244119706e-08
726 5.57958053093444e-08
727 5.65757893866703e-08
728 5.673497938119e-08
729 5.62492736833065e-08
730 5.61317108349613e-08
731 5.53824861526664e-08
732 5.60137571881114e-08
733 5.60696982176978e-08
734 5.63601290082261e-08
735 5.58617578860776e-08
736 5.55386492351317e-08
737 5.48256195997965e-08
738 5.51514318658519e-08
739 5.50776135810338e-08
740 5.46715881455384e-08
741 5.4368229029933e-08
742 5.42229052769017e-08
743 5.4367056634419e-08
744 5.38720357212696e-08
745 5.39496980422882e-08
746 5.41793063746354e-08
747 5.41012710186806e-08
748 5.41932330122563e-08
749 5.38091669000096e-08
750 5.36809103834912e-08
751 5.37141460199564e-08
752 5.34765796089687e-08
753 5.33217132669961e-08
754 5.35904653986563e-08
755 5.32401536190719e-08
756 5.45378640026684e-08
757 5.4428067386425e-08
758 5.37402726763503e-08
759 5.43587255208422e-08
760 5.40337055099371e-08
761 5.46911316234855e-08
762 5.36040651866188e-08
763 5.4210403277466e-08
764 5.31372776890748e-08
765 5.37766524644212e-08
766 5.5133590137757e-08
767 5.46628449171749e-08
768 5.41475735360564e-08
769 5.35163913184533e-08
770 5.42244897872024e-08
771 5.33177626493853e-08
772 5.37127000654891e-08
773 5.33642072753082e-08
774 5.37723785498656e-08
775 5.36319042510058e-08
776 5.35574571358666e-08
777 5.30812727106422e-08
778 5.4025367290933e-08
779 5.23967891297161e-08
780 5.40073337162994e-08
781 5.26210079954126e-08
782 5.35621609287773e-08
783 5.29881276634114e-08
784 5.25213614821496e-08
785 5.20271683512874e-08
786 5.19443084101567e-08
787 5.21171870104808e-08
788 5.22616048215241e-08
789 5.18582261577194e-08
790 5.15503479903145e-08
791 5.14641484983258e-08
792 5.15712237358912e-08
793 5.39588107528743e-08
794 5.20135863268933e-08
795 5.24810062074721e-08
796 5.33353663456637e-08
797 5.30822177324808e-08
798 5.21797609565056e-08
799 5.30362100903403e-08
800 5.32090922433781e-08
801 5.24365972864871e-08
802 5.24962011638763e-08
803 5.31670742986989e-08
804 5.16089997404379e-08
805 5.28368815366775e-08
806 5.2325127342101e-08
807 5.33813881986589e-08
808 5.26162686753651e-08
809 5.24161514192656e-08
810 5.27998125221529e-08
811 5.14007858498644e-08
812 5.16740570333241e-08
813 5.20630472067296e-08
814 5.21871470482438e-08
815 5.22828322857549e-08
816 5.05310495668709e-08
817 5.22845873263122e-08
818 5.15994251770735e-08
819 5.14812619201166e-08
820 5.08949611344178e-08
821 5.23282039921469e-08
822 5.10642479412127e-08
823 5.25827914543697e-08
824 5.02381887201864e-08
825 5.19711562674274e-08
826 5.07953963335694e-08
827 5.09289002081914e-08
828 5.0579096466663e-08
829 5.18918206182661e-08
830 5.19740090965115e-08
831 5.10580484558432e-08
832 5.11861841800965e-08
833 5.15232301268043e-08
834 5.00458092744793e-08
835 5.03742718649391e-08
836 5.04828356895359e-08
837 5.02336732211006e-08
838 4.98825656336521e-08
839 4.96811480843462e-08
840 4.98793291114907e-08
841 4.99604340120641e-08
842 4.97166254831427e-08
843 5.02231678467524e-08
844 4.98690368999632e-08
845 4.94663545680396e-08
846 4.94662053540651e-08
847 5.00097314670711e-08
848 4.93102980669846e-08
849 4.95146643686439e-08
850 4.97380376884848e-08
851 4.9809820268365e-08
852 4.95953216272937e-08
853 4.91125256019131e-08
854 4.97276566591154e-08
855 4.8948528785786e-08
856 4.95219651952539e-08
857 4.95482197493402e-08
858 4.86741242866628e-08
859 4.85094844293599e-08
860 4.92635265914032e-08
861 4.92055747258746e-08
862 4.62446294591246e-08
863 4.84602828976222e-08
864 4.75592365489774e-08
865 4.94128258310411e-08
866 4.81340869384894e-08
867 4.77432884338214e-08
868 4.73252868005147e-08
869 4.93496976616825e-08
870 4.86669122778949e-08
871 4.93431819847956e-08
872 4.84930922084459e-08
873 4.68187053570546e-08
874 4.8802405672177e-08
875 4.87754583389233e-08
876 4.90029030686401e-08
877 4.58862707830576e-08
878 5.14677971352739e-08
879 4.96233667490742e-08
880 4.91238694166896e-08
881 4.8450075951223e-08
882 4.54006396921613e-08
883 5.20500371692378e-08
884 4.60598563734038e-08
885 4.84005759915362e-08
886 4.54230324464788e-08
887 4.83245763405193e-08
888 4.79925610363807e-08
889 4.55113493558201e-08
890 4.73179646576227e-08
891 4.71074237395896e-08
892 4.70587977474679e-08
893 4.50896315840055e-08
894 4.68666456754363e-08
895 4.66891698636118e-08
896 4.72485375269116e-08
897 4.45107843916048e-08
898 4.2890999196743e-08
899 4.99896728456406e-08
900 4.75660080212492e-08
901 4.43309993158891e-08
902 4.44371721641801e-08
903 4.74015351414891e-08
904 4.45136301152615e-08
905 4.43308074693505e-08
906 4.62040823379084e-08
907 4.6804277786805e-08
908 4.60730618101479e-08
909 4.45057466436083e-08
910 4.50121717676666e-08
911 4.37543867803925e-08
912 4.76668375881673e-08
913 4.86098130636492e-08
914 5.0993886446804e-08
915 4.66399718845878e-08
916 4.37822933463394e-08
917 4.4593893022693e-08
918 4.50330688295253e-08
919 4.31471818274076e-08
920 4.34400391213785e-08
921 4.52258674954464e-08
922 4.45406733717846e-08
923 4.47024639527172e-08
924 4.47210375398299e-08
925 4.51137225354614e-08
926 4.45672796445251e-08
927 4.44361241136448e-08
928 4.27303561423287e-08
929 4.28103419380932e-08
930 4.28673132546464e-08
931 4.23770778468224e-08
932 4.35207851978703e-08
933 4.30094253545121e-08
934 4.2914624742707e-08
935 4.30220019609351e-08
936 4.34331575149827e-08
937 4.3466336308029e-08
938 4.21225259117364e-08
939 4.24474855265089e-08
940 4.26999733349476e-08
941 4.25858246444477e-08
942 4.13143368405144e-08
943 4.28824797893412e-08
944 4.08420426367684e-08
945 4.27256345858495e-08
946 4.32409095196817e-08
947 4.23633430557402e-08
948 4.08381168881533e-08
949 4.28772430893787e-08
950 4.29416395775206e-08
951 4.02950597333529e-08
952 4.14405612048085e-08
953 4.1746265111442e-08
954 4.18992165407417e-08
955 4.15038279300006e-08
956 4.10631315617138e-08
957 4.08657321315786e-08
958 4.84420219493131e-08
959 4.54400570504276e-08
960 4.29033732984863e-08
961 4.01356992085766e-08
962 5.07838926466775e-08
963 4.83982844912134e-08
964 4.77417749777942e-08
965 4.19658867656381e-08
966 4.07374720623466e-08
967 4.14611314170088e-08
968 4.08921927430583e-08
969 4.12002449934334e-08
970 4.01591613297114e-08
971 4.09652010091577e-08
972 4.08535782980834e-08
973 4.07164577609365e-08
974 3.91267107602289e-08
975 4.03641458035509e-08
976 4.01841759867239e-08
977 4.11004172917728e-08
978 4.04556601552031e-08
979 4.03790032521556e-08
980 4.05605398157149e-08
981 3.99209483248342e-08
982 4.02082456218977e-08
983 4.00647017784195e-08
984 4.01698052598931e-08
985 4.01653998949314e-08
986 3.96041883732323e-08
987 4.0160216485674e-08
988 4.00049948723336e-08
989 4.01389179671696e-08
990 4.06335125546775e-08
991 3.9781241412129e-08
992 4.10754239510425e-08
993 4.00517343734919e-08
994 3.99608062195966e-08
995 3.96073289721244e-08
996 3.95894197424695e-08
997 3.92864230036594e-08
998 3.84018470356295e-08
999 3.89779906129206e-08
1000 3.81257372339405e-08
1001 3.83909863899135e-08
1002 3.99168449405352e-08
1003 3.9311313315693e-08
1004 3.98103274790174e-08
1005 3.79446269960226e-08
1006 3.92622894196393e-08
1007 3.8194286844373e-08
1008 3.79494657920532e-08
1009 3.87037353277719e-08
1010 3.88611915980164e-08
1011 3.91020371637296e-08
1012 3.72590136521467e-08
1013 3.91837318147736e-08
1014 3.88863234945802e-08
1015 3.79777063130859e-08
1016 4.25951611759956e-08
1017 4.24660093756302e-08
1018 4.29910187449423e-08
1019 4.26240411854906e-08
1020 4.35557545586107e-08
1021 4.62950247026583e-08
1022 4.18826360260027e-08
1023 4.193591252033e-08
1024 4.11853484649782e-08
1025 4.30460538325406e-08
1026 4.29942943469541e-08
1027 4.31121520705346e-08
1028 4.10063982769771e-08
1029 4.41592149513781e-08
1030 4.31953175450417e-08
1031 4.13421439304784e-08
1032 4.28532942464699e-08
1033 4.20504164821978e-08
1034 4.33228493079696e-08
1035 4.16575041128908e-08
1036 4.17469436797546e-08
1037 4.15149905563794e-08
1038 4.04351787608448e-08
1039 4.15429539657453e-08
1040 4.40336620499693e-08
1041 4.23379908909283e-08
1042 4.11507699027425e-08
1043 4.24538306731392e-08
1044 4.17818810660719e-08
1045 4.10360350144856e-08
1046 4.18008845315399e-08
1047 4.29067732454769e-08
1048 4.12931306925657e-08
1049 4.12268690297424e-08
1050 4.08841849264263e-08
1051 4.13403107302202e-08
1052 4.22144843525984e-08
1053 4.03145286043127e-08
1054 4.27160244953484e-08
1055 4.11821119428168e-08
1056 3.97273289820532e-08
1057 4.0213596008698e-08
1058 3.99674071616118e-08
1059 4.09316669447435e-08
1060 4.00758679575119e-08
1061 4.10555358598685e-08
1062 3.84103664430313e-08
1063 4.00664319499811e-08
1064 3.96005255254295e-08
1065 3.96466113272709e-08
1066 3.95612893555608e-08
1067 3.85283023263128e-08
1068 4.04902955608577e-08
1069 3.95804562458579e-08
1070 3.83813869575533e-08
1071 4.02106223873488e-08
1072 3.71916932806471e-08
1073 4.0479164908902e-08
1074 3.80929172649758e-08
1075 4.06795166441043e-08
1076 3.86615752745456e-08
1077 3.82653837505131e-08
1078 4.0199847006761e-08
1079 3.78229003672459e-08
1080 3.87487197883729e-08
1081 3.79858704491198e-08
1082 3.81807012672652e-08
1083 3.85742033870429e-08
1084 3.74295296978744e-08
1085 3.76196069851176e-08
1086 3.72271671267299e-08
1087 3.62735157466432e-08
1088 3.64833141475174e-08
1089 3.80413354150733e-08
1090 3.69041295300576e-08
1091 3.81287676987085e-08
1092 3.59548728567916e-08
1093 3.76512900857051e-08
1094 3.83759299893427e-08
1095 3.78124234146071e-08
1096 3.56375977617063e-08
1097 3.66446073485349e-08
1098 3.64454137979919e-08
1099 3.77501088166809e-08
1100 3.62994398983574e-08
1101 3.69075792150397e-08
1102 3.62806495957102e-08
1103 3.80216711448611e-08
1104 3.60493466189382e-08
1105 3.60424259326919e-08
1106 3.56186014016657e-08
1107 3.66874743917833e-08
1108 3.74076734033224e-08
1109 3.61073766441677e-08
1110 3.67718833160779e-08
1111 3.64852859036091e-08
1112 3.72569992634908e-08
1113 3.51461650893725e-08
1114 3.76881885699731e-08
1115 3.73404596132332e-08
1116 3.63671048830838e-08
1117 3.51763453920739e-08
1118 3.63901477840045e-08
1119 3.58938976319223e-08
1120 3.61575018814619e-08
1121 3.58391254451362e-08
1122 3.60601077886713e-08
1123 3.69742991779276e-08
1124 3.56881422192146e-08
1125 3.55252325334732e-08
1126 3.6637395339767e-08
1127 3.55645610738975e-08
1128 3.62357681638059e-08
1129 3.52051792162911e-08
1130 3.48711566289239e-08
1131 3.63913841283647e-08
1132 3.58449661064242e-08
1133 3.53179707701656e-08
1134 3.48144162387598e-08
1135 3.56758995678774e-08
1136 3.5743518367326e-08
1137 3.36385852506282e-08
1138 3.55048044298201e-08
1139 3.55088225489908e-08
1140 3.50262681081404e-08
1141 3.47694033564494e-08
1142 3.50882594091217e-08
1143 3.40927606146124e-08
1144 3.46173365528557e-08
1145 3.51521833863444e-08
1146 3.49402569099766e-08
1147 3.46592550215519e-08
1148 3.53277798126328e-08
1149 3.48784254811108e-08
1150 3.44385533423974e-08
1151 3.55759617320928e-08
1152 3.49671118726746e-08
1153 3.43936541469247e-08
1154 3.46344037893687e-08
1155 3.42730324121021e-08
1156 3.38892505169497e-08
1157 3.45690764902429e-08
1158 3.40433814471908e-08
1159 3.39072556698738e-08
1160 3.41155264038662e-08
1161 3.42410082510014e-08
1162 3.4292124695412e-08
1163 3.39034578189512e-08
1164 3.49303839186632e-08
1165 3.42199584224545e-08
1166 3.45149011593548e-08
1167 3.35440688559174e-08
1168 3.27638112196382e-08
1169 3.43197044117005e-08
1170 3.44914070637969e-08
1171 3.417627425506e-08
1172 3.40024612910383e-08
1173 3.17911315050878e-08
1174 3.42746240278302e-08
1175 3.34987539929443e-08
1176 3.44126611651063e-08
1177 3.34098118059956e-08
1178 3.39086234646402e-08
1179 3.34356400344404e-08
1180 3.40063230908072e-08
1181 3.32075877906846e-08
1182 3.31490177529759e-08
1183 3.22969704313891e-08
1184 3.29162048728904e-08
1185 3.36809762302437e-08
1186 3.21835678107618e-08
1187 3.23277582481296e-08
1188 3.34395444667734e-08
1189 3.33632144133844e-08
1190 3.2758968870894e-08
1191 3.29517639841015e-08
1192 3.30665130832131e-08
1193 3.28684670591883e-08
1194 3.33701599686265e-08
1195 3.23566169413425e-08
1196 3.36274759149546e-08
1197 3.2154133577933e-08
1198 3.30946825499723e-08
1199 3.26251488047546e-08
1200 3.25100053544247e-08
1201 3.27662590393629e-08
1202 3.31269696118852e-08
1203 3.2161455720825e-08
1204 3.17547801387263e-08
1205 3.14721653182914e-08
1206 3.2480702572002e-08
1207 3.25885345375809e-08
1208 3.20435873391034e-08
1209 3.26366631497876e-08
1210 3.25651576815744e-08
1211 3.25707567583322e-08
1212 3.25811875256932e-08
1213 3.2220132339944e-08
1214 3.11911136918752e-08
1215 3.15310479948039e-08
1216 3.13441681498716e-08
1217 3.22714868161711e-08
1218 3.23063638063559e-08
1219 3.27831202184825e-08
1220 3.16236423714145e-08
1221 3.22724069690139e-08
1222 3.21371125266978e-08
1223 3.27113731657391e-08
1224 3.23801288004688e-08
1225 3.25055857786083e-08
1226 3.20567892231338e-08
1227 3.29899130235844e-08
1228 3.1636012920444e-08
1229 3.24245732485906e-08
1230 3.24408304663848e-08
1231 3.2400489402562e-08
1232 3.16846815451299e-08
1233 3.26539257855529e-08
1234 3.20415551868791e-08
1235 3.19503854484537e-08
1236 3.25145244062242e-08
1237 3.25575868487249e-08
1238 3.26891473889646e-08
1239 3.23334816698662e-08
1240 3.06322966991956e-08
1241 3.26239302239628e-08
1242 3.21180984030889e-08
1243 3.20332311787297e-08
1244 3.2822015327838e-08
1245 3.28670530791442e-08
1246 3.2426648033379e-08
1247 3.24745990099018e-08
1248 3.2921473547276e-08
1249 3.18707087387793e-08
1250 3.26102522762994e-08
1251 3.08150980288247e-08
1252 3.20216173577137e-08
1253 3.21642268374944e-08
1254 3.25055715677536e-08
1255 3.15327746136518e-08
1256 3.22099751315363e-08
1257 3.14351424890447e-08
1258 3.21371729228304e-08
1259 3.18406208066335e-08
1260 3.23692823656074e-08
1261 3.16900745644944e-08
1262 3.20673052556231e-08
1263 3.21024913318979e-08
1264 3.19579065433118e-08
1265 3.24113464955644e-08
1266 3.21555972959686e-08
1267 3.18674437949085e-08
1268 3.21731263852598e-08
1269 3.20495487926564e-08
1270 3.16434096703233e-08
1271 3.22647473183224e-08
1272 3.24343965019125e-08
1273 3.18650883457394e-08
1274 3.2066122201968e-08
1275 3.22562385690617e-08
1276 3.18936947962811e-08
1277 3.15446833099031e-08
1278 3.125087744138e-08
1279 3.16195851723933e-08
1280 3.19198676379528e-08
1281 3.25552065305601e-08
1282 3.22188888901564e-08
1283 3.19524637859558e-08
1284 3.23971498517039e-08
1285 3.19072093191153e-08
1286 3.16281543177865e-08
1287 3.16870654160084e-08
1288 3.12594536922006e-08
1289 3.08764569467712e-08
1290 3.16780983666831e-08
1291 3.18066426530095e-08
1292 3.20755368932168e-08
1293 3.0881459167631e-08
1294 3.1669671329837e-08
1295 3.16342756434551e-08
1296 3.17318118447929e-08
1297 3.1689879165242e-08
1298 3.15572741271808e-08
1299 3.13611181468332e-08
1300 3.1686848700474e-08
1301 3.20249995411359e-08
1302 3.17641486446973e-08
1303 3.07779650654538e-08
1304 3.08742400534356e-08
1305 2.95310833564599e-08
1306 3.00168885303265e-08
1307 3.00270244224521e-08
1308 3.14927781630558e-08
1309 3.17589083920211e-08
1310 2.96469622185214e-08
1311 3.20609032655739e-08
1312 3.20527213659716e-08
1313 3.12738102081767e-08
1314 3.00459674917875e-08
1315 2.97933606674405e-08
1316 2.93522610661512e-08
1317 2.98433207035487e-08
1318 3.2266687099991e-08
1319 3.19222337452629e-08
1320 3.20224486927145e-08
1321 3.03678469038005e-08
1322 3.04024574404593e-08
1323 3.3063084714513e-08
1324 3.07674987709561e-08
1325 3.0998194233689e-08
1326 3.06388585613604e-08
1327 3.18049195868753e-08
1328 3.07284295786303e-08
1329 2.96681186284786e-08
1330 2.99652498370051e-08
1331 3.15268415818082e-08
1332 3.17071879862851e-08
1333 3.11341707970314e-08
1334 2.95124404914304e-08
1335 2.98300584233857e-08
1336 3.27242695163932e-08
1337 3.26142917117522e-08
1338 3.17533448424001e-08
1339 2.961500022991e-08
1340 3.08838963292146e-08
1341 3.18161177403908e-08
1342 3.20317035118478e-08
1343 3.14368051590463e-08
1344 3.06049052767321e-08
1345 3.13458627942964e-08
1346 3.07915435371342e-08
1347 2.92627166942339e-08
1348 2.99247489010668e-08
1349 3.00875271364021e-08
1350 2.98272659904342e-08
1351 3.02918152783604e-08
1352 3.05175689163661e-08
1353 2.91135311414337e-08
1354 3.15634558489819e-08
1355 3.05051024440672e-08
1356 3.20141211318514e-08
1357 3.04700016329207e-08
1358 3.0368603631814e-08
1359 3.06154070983666e-08
1360 3.02124156803529e-08
1361 3.05260741129132e-08
1362 2.97010380734264e-08
1363 2.92252533284909e-08
1364 3.12172154792734e-08
1365 3.03296125991892e-08
1366 2.98169311463425e-08
1367 3.18653086139875e-08
1368 3.12791961221137e-08
1369 3.08398533377385e-08
1370 3.07586098813317e-08
1371 3.08812566629513e-08
1372 3.13316341760128e-08
1373 3.11187626778064e-08
1374 3.07193275261852e-08
1375 3.08855909736394e-08
1376 3.10187040497567e-08
1377 3.05751548523858e-08
1378 3.08105576607431e-08
1379 3.14896233533091e-08
1380 3.05090459562507e-08
1381 3.13395780437986e-08
1382 3.08246406177659e-08
1383 3.09999741432421e-08
1384 3.17412869321743e-08
1385 3.08952046168542e-08
1386 3.053159502997e-08
1387 2.96277189448801e-08
1388 3.06032177377347e-08
1389 3.05682164025711e-08
1390 2.97628002243755e-08
1391 3.01300886462741e-08
1392 3.18185193748377e-08
1393 3.00082412252323e-08
1394 3.05789171761717e-08
1395 3.06217060597191e-08
1396 2.97663262927017e-08
1397 2.89462125380169e-08
1398 3.05563752078797e-08
1399 2.99322309160743e-08
1400 3.14643529009118e-08
1401 3.17312327524633e-08
1402 3.08985903529901e-08
1403 3.07491561102324e-08
1404 2.87982491187222e-08
1405 2.94087243446484e-08
1406 2.94576381065781e-08
1407 2.97958191453063e-08
1408 3.15825445795781e-08
1409 3.02956699727019e-08
1410 2.99276656789971e-08
1411 2.89174408862891e-08
1412 2.95694508878341e-08
1413 3.07395282561629e-08
1414 2.9883135965747e-08
1415 3.03614946517428e-08
1416 3.01772047350823e-08
1417 3.00712770240352e-08
1418 2.94357889174535e-08
1419 3.0293922037572e-08
1420 3.060145559175e-08
1421 2.96859514747894e-08
1422 3.10516377055592e-08
1423 3.05017415769271e-08
1424 2.96188549242515e-08
1425 2.98589846181585e-08
1426 3.06856691167923e-08
1427 2.96296160939846e-08
1428 3.02480671621197e-08
1429 2.91005033403735e-08
1430 3.06599510224714e-08
1431 3.07010523670215e-08
1432 2.9748990826306e-08
1433 2.94162632030748e-08
1434 2.89514296980542e-08
1435 2.96826794254912e-08
1436 3.08550660577112e-08
1437 2.97322344522399e-08
1438 3.03089073838692e-08
1439 2.95111579617924e-08
1440 2.94407147549691e-08
1441 3.02948279795601e-08
1442 3.00356823856873e-08
1443 2.91587234357849e-08
1444 2.92818267411121e-08
1445 3.04414875529346e-08
1446 2.89762454031006e-08
1447 2.85784107489917e-08
1448 2.92496586951074e-08
1449 2.95255482285484e-08
1450 2.95010647022309e-08
1451 2.95714261966395e-08
1452 2.92513160360386e-08
1453 2.99914830748094e-08
1454 2.96466335925061e-08
1455 2.86518613279441e-08
1456 2.85651502451856e-08
1457 2.86378423197675e-08
1458 2.91141262209749e-08
1459 2.89783894658058e-08
1460 3.13337764623611e-08
1461 2.96097653063043e-08
1462 2.97342861443894e-08
1463 2.88335986198263e-08
1464 2.97224662659801e-08
1465 2.91884934000564e-08
1466 2.89597217317805e-08
1467 2.9168388593348e-08
1468 3.01702769434087e-08
1469 2.84581282983254e-08
1470 3.01152169868146e-08
1471 2.91993043077809e-08
1472 2.96890760864699e-08
1473 2.88596240238803e-08
1474 3.02734797230642e-08
1475 2.93633739545385e-08
1476 2.90414536863182e-08
1477 2.935303733409e-08
1478 2.90846120520882e-08
1479 3.04508098736278e-08
1480 2.8912863214714e-08
1481 2.9648260735371e-08
1482 2.8405516161456e-08
1483 3.05574729964064e-08
1484 2.90632300448124e-08
1485 2.98420665956201e-08
1486 2.94090103380995e-08
1487 2.9531292966567e-08
1488 2.87417343258767e-08
1489 2.8454305578407e-08
1490 2.94550464019494e-08
1491 2.92168422788563e-08
1492 2.88045942653525e-08
1493 2.87418071565071e-08
1494 2.81637504429e-08
1495 3.00660225605043e-08
1496 2.97373734525763e-08
1497 2.86073298383371e-08
1498 2.97069693289131e-08
1499 2.94714883608549e-08
1500 3.04234966108652e-08
1501 2.95759470247958e-08
1502 3.01342346631372e-08
1503 2.94477562334805e-08
1504 2.85197092608769e-08
1505 2.81076104613476e-08
1506 3.09220347105565e-08
1507 3.05356522289912e-08
1508 2.9964354553158e-08
1509 2.85845818126518e-08
1510 2.96949327349694e-08
1511 2.80237930638805e-08
1512 3.00827913690682e-08
1513 3.08146503869011e-08
1514 2.90329040808501e-08
1515 2.96764159912755e-08
1516 2.99082714150245e-08
1517 2.86998727005994e-08
1518 2.90123232105088e-08
1519 2.96912130437477e-08
1520 2.83885235319303e-08
1521 2.77198051179539e-08
1522 3.04654221849887e-08
1523 2.94384427945715e-08
1524 2.94880067031045e-08
1525 2.91037380861781e-08
1526 3.02184126610427e-08
1527 2.94102182607503e-08
1528 2.93142292662196e-08
1529 3.00615177195596e-08
1530 2.89474453296634e-08
1531 3.03233136378367e-08
1532 2.89770909489562e-08
1533 2.92194801687629e-08
1534 2.90308062034228e-08
1535 2.9617279295735e-08
1536 2.88090546973763e-08
1537 2.85570536107116e-08
1538 2.93068751489045e-08
1539 2.98649602825662e-08
1540 2.97890174749682e-08
1541 2.94436812708909e-08
1542 2.86954033867914e-08
1543 2.93592918865215e-08
1544 2.92786115352328e-08
1545 3.03102716259218e-08
1546 2.86451964370826e-08
1547 2.95616207068861e-08
1548 2.94073725370936e-08
1549 2.97943891780506e-08
1550 3.0178192389485e-08
1551 3.08247152247532e-08
1552 2.90025621296763e-08
1553 2.98609279525408e-08
1554 2.95570377062404e-08
1555 2.92178334859727e-08
1556 2.93608177770466e-08
1557 2.95129272132044e-08
1558 3.02814520125594e-08
1559 2.87846901869671e-08
1560 2.97680919914001e-08
1561 2.99634734801657e-08
1562 2.94095858777155e-08
1563 2.88190040720337e-08
1564 2.94581976589825e-08
1565 2.91951280928515e-08
1566 2.90894970333966e-08
1567 3.05104812525769e-08
1568 2.9347384966627e-08
1569 2.87730017589638e-08
1570 2.96525559662086e-08
1571 3.01284295289861e-08
1572 2.90876034370058e-08
1573 2.83883370144622e-08
1574 2.92375830213132e-08
1575 2.90781230205539e-08
1576 2.88510086932092e-08
1577 3.10062375774578e-08
1578 2.90582526929484e-08
1579 3.01973130945044e-08
1580 2.90641111178047e-08
1581 2.90409332137642e-08
1582 2.97156290685052e-08
1583 2.89803949726775e-08
1584 2.9565349279892e-08
1585 2.96743092320639e-08
1586 2.97257436443488e-08
1587 2.96710584990478e-08
1588 2.87136270316068e-08
1589 2.93478912283263e-08
1590 2.96680333633503e-08
1591 2.95973432429264e-08
1592 2.91176451838737e-08
1593 3.13760217807157e-08
1594 2.94852888771402e-08
1595 3.05110248177698e-08
1596 3.10110728207746e-08
1597 2.96285556089515e-08
1598 2.89620860627338e-08
1599 2.99465057196358e-08
1600 3.00457152491163e-08
1601 2.89121668828329e-08
1602 2.88008994431266e-08
1603 2.92730195639024e-08
1604 3.01250508982776e-08
1605 3.00297955391216e-08
1606 2.9913110211055e-08
1607 2.9596993300629e-08
1608 2.93589064170874e-08
1609 2.95432247554572e-08
1610 3.01574480943145e-08
1611 3.07386720521663e-08
1612 3.10323713392791e-08
1613 2.93296693598677e-08
1614 2.87453261194059e-08
1615 2.94693442981497e-08
1616 2.89895236704751e-08
1617 2.96267721466847e-08
1618 2.88431198924854e-08
1619 3.07950287492531e-08
1620 3.0707063558566e-08
1621 2.99430347183716e-08
1622 3.05157072943985e-08
1623 3.03515328425874e-08
1624 3.01017770709677e-08
1625 2.97227682466428e-08
1626 2.91512360917068e-08
1627 3.00358173888071e-08
1628 2.93549220486966e-08
1629 3.06759488921671e-08
1630 2.99427220795678e-08
1631 2.93190538513954e-08
1632 3.00287332777316e-08
1633 3.03607770035796e-08
1634 2.92892963216218e-08
1635 3.0454479826858e-08
1636 2.98474560622708e-08
1637 2.96038429326018e-08
1638 2.99584428375965e-08
1639 3.05173344372633e-08
1640 3.08181888897252e-08
1641 3.00330746938471e-08
1642 2.94997093419624e-08
1643 2.94010362722474e-08
1644 3.00901099592465e-08
1645 2.99625355637545e-08
1646 2.92485982100743e-08
1647 2.9270855961272e-08
1648 3.12420205261787e-08
1649 3.00794233965007e-08
1650 2.95702839991918e-08
1651 3.08008303306906e-08
1652 3.04513463333933e-08
1653 3.05726715055243e-08
1654 3.12553858350384e-08
1655 2.97638607094086e-08
1656 2.93516055904774e-08
1657 2.84350019086332e-08
1658 2.97614608513186e-08
1659 3.01097102806125e-08
1660 2.92493087528101e-08
1661 3.01735063601427e-08
1662 3.05431022695757e-08
1663 2.88494526046179e-08
1664 3.13021857323292e-08
1665 3.04677243434526e-08
1666 2.99696019112616e-08
1667 2.93365030046289e-08
1668 2.98593612058085e-08
1669 3.12095664867229e-08
1670 3.06742862221654e-08
1671 2.97376505642433e-08
1672 3.04545224594222e-08
1673 3.10202139530702e-08
1674 3.08133820681178e-08
1675 2.92648252298022e-08
1676 3.1746200335192e-08
1677 2.99628020172804e-08
1678 3.11039300981975e-08
1679 2.87547941013599e-08
1680 2.89795547558924e-08
1681 3.27591074267275e-08
1682 2.96973734492667e-08
1683 3.10549950199857e-08
1684 2.93205388857132e-08
1685 2.92632691412109e-08
1686 2.944425503415e-08
1687 3.07817273892397e-08
1688 3.00303533151691e-08
1689 2.97557587458641e-08
1690 3.18720374536952e-08
1691 3.02580822619802e-08
1692 3.02867491086545e-08
1693 3.07640810603971e-08
1694 2.8327471923717e-08
1695 2.90679640357894e-08
1696 3.0356194002934e-08
1697 2.92555899505942e-08
1698 3.02119147477242e-08
1699 2.92580608629578e-08
1700 2.85761974083698e-08
1701 2.84587837739991e-08
1702 2.97629352274953e-08
1703 2.90952684167678e-08
1704 3.20113535678956e-08
1705 2.85198016314325e-08
1706 2.86085644063405e-08
1707 2.86976522545501e-08
1708 2.88719661512005e-08
1709 2.79827325755377e-08
1710 3.10044576679047e-08
1711 2.91200219493248e-08
1712 2.96080369110996e-08
1713 3.02567642052054e-08
1714 2.99644824508505e-08
1715 2.98381372942913e-08
1716 2.93638038328936e-08
1717 3.00389295659897e-08
1718 2.93110122839835e-08
1719 2.97319715514277e-08
1720 2.98108915330886e-08
1721 2.86654060488445e-08
1722 2.80879817182722e-08
1723 2.96478130934474e-08
1724 2.91408621677647e-08
1725 2.93621909008834e-08
1726 3.00998017621623e-08
1727 2.88322929975493e-08
1728 2.94702129366442e-08
1729 2.93399704531794e-08
1730 2.91649335792954e-08
1731 2.88554140581709e-08
1732 2.92603186125007e-08
1733 2.89580892598451e-08
1734 2.90615993492338e-08
1735 2.98137159404632e-08
1736 2.99362099553946e-08
1737 3.00567997157941e-08
1738 3.05917282616974e-08
1739 2.9342009710831e-08
1740 2.89449246793083e-08
1741 2.89427255495411e-08
1742 2.85212031769788e-08
1743 3.00842053491124e-08
1744 2.92962312187228e-08
1745 2.82673884299811e-08
1746 2.97214555189385e-08
1747 3.05607059658541e-08
1748 3.14895274300397e-08
1749 2.96971940372259e-08
1750 2.97497990686679e-08
1751 3.03694740466653e-08
1752 2.79815495218827e-08
1753 2.99054576657909e-08
1754 3.00902947003578e-08
1755 2.85955366052804e-08
1756 3.04068201728569e-08
1757 2.86534937998795e-08
1758 2.93544850649141e-08
1759 3.09299039713551e-08
1760 3.0844759635329e-08
1761 2.84397323468966e-08
1762 2.98299838163985e-08
1763 3.0241238846429e-08
1764 2.88549770743884e-08
1765 3.1094135266585e-08
1766 2.98977589352489e-08
1767 2.74282996315378e-08
1768 3.01292431004185e-08
1769 2.88103567669395e-08
1770 2.87648429520004e-08
1771 2.83783911925184e-08
1772 3.03101437282294e-08
1773 2.97439743945915e-08
1774 2.87265535803272e-08
1775 3.07412939548612e-08
1776 2.97506410618098e-08
1777 2.86444823416332e-08
1778 2.88915966706327e-08
1779 2.99345508381066e-08
1780 2.87207608806739e-08
1781 2.97360092105237e-08
1782 2.90693638049788e-08
1783 2.88509163226536e-08
1784 3.0068370904246e-08
1785 2.9633572040666e-08
1786 3.06258485238686e-08
1787 2.91730444246241e-08
1788 2.92956485736795e-08
1789 2.84826313645681e-08
1790 2.80463616775251e-08
1791 3.07819654210562e-08
1792 2.95016526763447e-08
1793 2.91745010372324e-08
1794 2.9956058966718e-08
1795 2.86083245981672e-08
1796 3.06172260877702e-08
1797 2.98539859500124e-08
1798 2.91654238537831e-08
1799 2.9107864563116e-08
1800 2.96080582273817e-08
1801 2.87098007589748e-08
1802 3.10610737130901e-08
1803 2.95043793840932e-08
1804 2.96454327752826e-08
1805 2.89467365632845e-08
1806 2.8585088074351e-08
1807 3.1128379873735e-08
1808 3.11263121943739e-08
1809 2.82247718530471e-08
1810 2.82620220559693e-08
1811 2.82407146556807e-08
1812 2.80193503954251e-08
1813 2.77021641181818e-08
1814 3.09995975555921e-08
1815 2.99253493096785e-08
1816 3.1982779091777e-08
1817 2.80331100555031e-08
1818 2.93553359398402e-08
1819 2.93854505173385e-08
1820 2.87050934133504e-08
1821 2.8487542991229e-08
1822 2.80846670364099e-08
1823 2.88566308626059e-08
1824 2.85444166081561e-08
1825 3.04620684232759e-08
1826 2.74682658840675e-08
1827 2.90863653162887e-08
1828 2.85160410840035e-08
1829 3.03605922624683e-08
1830 2.95223259172417e-08
1831 2.96845428238157e-08
1832 3.16846069381427e-08
1833 2.9758348674136e-08
1834 2.74741847050564e-08
1835 3.04818073004753e-08
1836 2.91903052840325e-08
1837 2.83687420221668e-08
1838 3.08954568595254e-08
1839 2.84480883294691e-08
1840 3.02039744326521e-08
1841 2.87347674543525e-08
1842 2.92886319641639e-08
1843 2.98672127030386e-08
1844 2.8986704592171e-08
1845 2.97991871178738e-08
1846 2.97841289409462e-08
1847 2.92339432661493e-08
1848 3.04160714392765e-08
1849 2.96549913514355e-08
1850 2.94197768369031e-08
1851 2.83095378250664e-08
1852 2.79605298914021e-08
1853 2.8223423598206e-08
1854 2.88518862134879e-08
1855 2.79869372121766e-08
1856 2.82067951218323e-08
1857 2.88628942968217e-08
1858 3.02447240585479e-08
1859 2.84431500574556e-08
1860 2.90676993586203e-08
1861 2.90025301552532e-08
1862 2.99690050553636e-08
1863 2.92182740224689e-08
1864 2.9313948601839e-08
1865 2.92379116473285e-08
1866 3.014519833755e-08
1867 2.94100157560706e-08
1868 2.85287509171894e-08
1869 3.03103036003449e-08
1870 2.88286585714559e-08
1871 2.75674221228428e-08
1872 2.79215068843541e-08
1873 2.98387128339073e-08
1874 2.95020896601272e-08
1875 2.79975580497194e-08
1876 2.91739681301806e-08
1877 2.85989205650594e-08
1878 2.98669391440853e-08
1879 2.91484703041078e-08
1880 2.92678432600724e-08
1881 3.09981729174069e-08
1882 2.78101293105237e-08
1883 2.81292447112946e-08
1884 2.87972170553985e-08
1885 2.93441679843909e-08
1886 2.97319520115025e-08
1887 2.78969878308999e-08
1888 2.98773770168737e-08
1889 2.94157462832345e-08
1890 2.90722503848428e-08
1891 2.89204464820614e-08
1892 2.70994711115691e-08
1893 2.877131777268e-08
1894 2.87722325964523e-08
1895 2.81102625621088e-08
1896 2.95998567878542e-08
1897 3.01538491953579e-08
1898 2.78429741484842e-08
1899 2.83336838435844e-08
1900 2.79706746653119e-08
1901 2.83268164480432e-08
1902 3.06017149398485e-08
1903 3.04273015672152e-08
1904 2.83761618646849e-08
1905 2.76798921561294e-08
1906 2.75329288257353e-08
1907 3.00072997561074e-08
1908 2.87193540060571e-08
1909 2.7184993811602e-08
1910 2.78688787602732e-08
1911 2.9866544792867e-08
1912 2.75599560950468e-08
1913 2.85613275252672e-08
1914 2.83682695112475e-08
1915 2.89865909053333e-08
1916 2.67253170704862e-08
1917 2.76316889369355e-08
1918 3.06494634116916e-08
1919 2.92361939102648e-08
1920 2.95018978135886e-08
1921 2.8255112027864e-08
1922 2.76712111002553e-08
1923 2.88970039008518e-08
1924 2.81886212150084e-08
1925 2.80981549138914e-08
1926 2.81587109185466e-08
1927 2.88639032675064e-08
1928 2.84616259449422e-08
1929 3.1551476098457e-08
1930 2.8023277920397e-08
1931 2.98193718606399e-08
1932 2.90825923343618e-08
1933 2.77737655096644e-08
1934 2.93533855000305e-08
1935 2.76140728061591e-08
1936 2.96954443257391e-08
1937 2.85688201984158e-08
1938 2.81429493043106e-08
1939 2.72842211046509e-08
1940 2.72550444435637e-08
1941 2.77940817028366e-08
1942 2.73461342459314e-08
1943 2.77094045486592e-08
1944 2.74917990594759e-08
1945 2.82342753621379e-08
1946 2.70285731573949e-08
1947 2.72024731629017e-08
1948 2.58114685180999e-08
1949 2.6879085623932e-08
1950 2.69391584595269e-08
1951 2.84516374904342e-08
1952 2.85640702202272e-08
1953 2.95104332082019e-08
1954 2.69053064272384e-08
1955 2.78098255535042e-08
1956 2.67707989110022e-08
1957 2.85627326235272e-08
1958 2.85404215816243e-08
1959 2.8142524755026e-08
1960 2.72797802125524e-08
1961 2.7881446484912e-08
1962 3.29086162764725e-08
1963 2.7276112035679e-08
1964 2.79438889805306e-08
1965 2.82348402436128e-08
1966 2.69549094156218e-08
1967 2.82768937154287e-08
1968 2.9357538622321e-08
1969 2.80964833621056e-08
1970 2.93056690026106e-08
1971 2.78103797768381e-08
1972 2.72637343812221e-08
1973 2.80229297544565e-08
1974 2.8641752081171e-08
1975 3.04753058344431e-08
1976 3.01694456084078e-08
1977 2.86353554201924e-08
1978 2.73995990340836e-08
1979 2.89510424522632e-08
1980 2.98396578557458e-08
1981 2.7829726079176e-08
1982 2.6575301959042e-08
1983 2.97171602881008e-08
1984 2.74305520520102e-08
1985 2.84362844382713e-08
1986 2.73780411674807e-08
1987 2.7559533322119e-08
1988 2.81305325700032e-08
1989 2.73015459129056e-08
1990 2.63657273791296e-08
1991 2.84826757734891e-08
1992 2.93122237593479e-08
1993 2.84628871582981e-08
1994 2.87522485820091e-08
1995 2.7560362880763e-08
1996 2.94200592776406e-08
1997 2.92500406118279e-08
1998 2.84702927899616e-08
1999 2.86594588061462e-08
2000 2.8709035149177e-08
2001 2.83350285457118e-08
2002 2.7979886851881e-08
2003 2.80106018379911e-08
2004 2.71412829988549e-08
2005 2.69363269467249e-08
2006 3.15686996543718e-08
2007 2.87663972642349e-08
2008 2.88182633312317e-08
2009 3.0488923385974e-08
2010 2.79607874631438e-08
2011 2.70085926956654e-08
2012 2.81570091686945e-08
2013 2.79399721136997e-08
2014 2.74130123045779e-08
2015 2.71553268760272e-08
2016 2.81295342574595e-08
2017 2.71600555379337e-08
2018 2.87674382093428e-08
2019 2.73484008772584e-08
2020 2.7344700725962e-08
2021 2.84009686879472e-08
2022 2.66006221494308e-08
2023 2.7366777288762e-08
2024 2.70883706576797e-08
2025 2.71906692717039e-08
2026 2.59541117486606e-08
2027 2.65100066343393e-08
2028 2.59492924925553e-08
2029 2.52154919166969e-08
2030 2.62461874456221e-08
2031 2.57470436082485e-08
2032 2.7323276086122e-08
2033 2.69452709034113e-08
2034 2.69987925349824e-08
2035 2.56230645589994e-08
2036 2.55319143604993e-08
2037 2.45414106814223e-08
2038 2.50464555762164e-08
2039 2.6257659158091e-08
2040 2.51482461521846e-08
2041 2.6002387798485e-08
2042 2.55625387524105e-08
2043 2.50317473415862e-08
2044 2.5953056592698e-08
2045 2.52443772552624e-08
2046 2.60967514265076e-08
2047 2.517510822031e-08
2048 2.63229491537231e-08
2049 2.56423238198522e-08
2050 2.62345078994031e-08
2051 2.64281077022588e-08
2052 2.50366234411104e-08
2053 2.5730248154332e-08
2054 2.60369592552934e-08
2055 2.57294008321196e-08
2056 2.60965684617531e-08
2057 2.60205723634499e-08
2058 2.61386023936439e-08
2059 2.5249480728462e-08
2060 2.63164547931183e-08
2061 2.53582026488175e-08
2062 2.51142147078554e-08
2063 2.58019561272249e-08
2064 2.50305287607944e-08
2065 2.53738559052863e-08
2066 2.45757529881985e-08
2067 2.48793927681845e-08
2068 2.52640184328357e-08
2069 2.5100337808226e-08
2070 2.59598760266044e-08
2071 2.49738718594017e-08
2072 2.48987319650951e-08
2073 2.66747033350612e-08
2074 2.55569947427148e-08
2075 2.58641925654501e-08
2076 2.60919605921117e-08
2077 2.54612899652784e-08
2078 2.61186077210596e-08
2079 2.47186093815799e-08
2080 2.59136712088548e-08
2081 2.56637875395427e-08
2082 2.55652832237274e-08
2083 2.43372397790154e-08
2084 2.4328693726261e-08
2085 2.47799221142486e-08
2086 2.55685108641046e-08
2087 2.67052850944083e-08
2088 2.60073669267058e-08
2089 2.58349288628779e-08
2090 2.50734313311796e-08
2091 2.55528949111294e-08
2092 2.58395278507351e-08
2093 2.5672754588868e-08
2094 2.51038283494154e-08
2095 2.5236207790158e-08
2096 2.57993839625215e-08
2097 2.60690633524518e-08
2098 2.65564903401128e-08
2099 2.65758242079528e-08
2100 2.56232919326749e-08
2101 2.74506160025112e-08
2102 2.59536943048033e-08
2103 2.68355559995825e-08
2104 2.69831144095178e-08
2105 2.55806469340314e-08
2106 2.72599116613037e-08
2107 2.51450398280895e-08
2108 2.5210745491222e-08
2109 2.61401869039446e-08
2110 2.5923093005531e-08
2111 2.63393857835581e-08
2112 2.56910137608202e-08
2113 2.72275393342625e-08
2114 2.47950122655993e-08
2115 2.57837040607001e-08
2116 2.65952611044895e-08
2117 2.55409364768866e-08
2118 2.6478431891519e-08
2119 2.57972505579573e-08
2120 2.60288715026036e-08
2121 2.55337830878943e-08
2122 2.55086156641937e-08
2123 2.51128220440933e-08
2124 2.59763570653604e-08
2125 2.48842582095676e-08
2126 2.52276173284827e-08
2127 2.58215298032383e-08
2128 2.46900704325981e-08
2129 2.52755079088729e-08
2130 2.72753410968107e-08
2131 2.65790358611184e-08
2132 2.67418762689431e-08
2133 2.53664786953323e-08
2134 2.55690437711564e-08
2135 2.58951953213682e-08
2136 2.66888733335691e-08
2137 2.53495766600054e-08
2138 2.63115342846731e-08
2139 2.63644803766283e-08
2140 2.6546421949547e-08
2141 2.62470223333366e-08
2142 2.70801479018701e-08
2143 2.73210556400727e-08
2144 2.64536623717504e-08
2145 2.48913032407927e-08
2146 2.51680436491597e-08
2147 2.603637661025e-08
2148 2.50896174947002e-08
2149 2.61746357921311e-08
2150 2.53235334923829e-08
2151 2.78637912742852e-08
2152 2.60903902926657e-08
2153 2.87775421270453e-08
2154 2.56749483895646e-08
2155 2.58702907984798e-08
2156 2.63190766958132e-08
2157 2.47968419131439e-08
2158 2.74561369195681e-08
2159 2.63754529328253e-08
2160 2.47420928189968e-08
2161 2.8847670918708e-08
2162 2.59132981739185e-08
2163 2.61741153195771e-08
2164 2.45697275857992e-08
2165 2.74746252415525e-08
2166 2.58057752944296e-08
2167 2.6578575784697e-08
2168 2.58586165813313e-08
2169 2.66967870032886e-08
2170 2.50564440307244e-08
2171 2.62544581630664e-08
2172 2.56605829918044e-08
2173 2.75763039070398e-08
2174 2.53876528688579e-08
2175 2.55382417435612e-08
2176 2.48063773966578e-08
2177 2.51583767152397e-08
2178 2.6295785104935e-08
2179 2.65732538196062e-08
2180 2.64340105360361e-08
2181 2.56644714369259e-08
2182 2.51139145035495e-08
2183 2.37050272744455e-08
2184 2.52128113942263e-08
2185 2.58591352775284e-08
2186 2.43321398585294e-08
2187 2.48690366078108e-08
2188 2.47777638406887e-08
2189 2.57172398931971e-08
2190 2.55128842496788e-08
2191 2.53587408849398e-08
2192 2.51897009917457e-08
2193 2.46321363306379e-08
2194 2.49185223566428e-08
2195 2.46828282257638e-08
2196 2.46831355354971e-08
2197 2.5424849781075e-08
2198 2.51331453426928e-08
2199 2.5124460734105e-08
2200 2.45374369711726e-08
2201 2.45012472532835e-08
2202 2.46339446619004e-08
2203 2.43737563465629e-08
2204 2.29317151934083e-08
2205 2.48259848234511e-08
2206 2.67729536318484e-08
2207 2.47644962314553e-08
2208 2.48714435713282e-08
2209 2.42871927014221e-08
2210 2.56944456822339e-08
2211 2.51812402041196e-08
2212 2.45698679179895e-08
2213 2.6117364271272e-08
2214 2.47590392632446e-08
2215 2.36351347382424e-08
2216 2.58964245603011e-08
2217 2.61989825389719e-08
2218 2.44079494393645e-08
2219 2.57420023075383e-08
2220 2.31572414577386e-08
2221 2.42399718075603e-08
2222 2.51422278552127e-08
2223 2.49335432300768e-08
2224 2.55132981408224e-08
2225 2.5084000654374e-08
2226 2.51027554298844e-08
2227 2.39729267548228e-08
2228 2.68212509979548e-08
2229 2.36737900394246e-08
2230 2.41637589937227e-08
2231 2.41463595784808e-08
2232 2.26072529585508e-08
2233 2.48518041701118e-08
2234 2.42571651654089e-08
2235 2.5870409814388e-08
2236 2.45932376685687e-08
2237 2.4128139486379e-08
2238 2.57381103097032e-08
2239 2.38354509463079e-08
2240 2.42494255786596e-08
2241 2.46975861983856e-08
2242 2.41393252053967e-08
2243 2.48918237133466e-08
2244 2.49939233754048e-08
2245 2.52840539616273e-08
2246 2.41360744723806e-08
2247 2.56281360577759e-08
2248 2.39897275378098e-08
2249 2.53058125565531e-08
2250 2.30140191348482e-08
2251 2.57875782949668e-08
2252 2.43224214102611e-08
2253 2.40430786391244e-08
2254 2.48244944600629e-08
2255 2.44810642868742e-08
2256 2.36125732300252e-08
2257 2.49019826981112e-08
2258 2.40168311904654e-08
2259 2.54863934401328e-08
2260 2.45652405084229e-08
2261 2.46773659284827e-08
2262 2.54697383184066e-08
2263 2.46095464007112e-08
2264 2.49042955147161e-08
2265 2.61193537909321e-08
2266 2.33582149178346e-08
2267 2.41445938797824e-08
2268 2.41552857715988e-08
2269 2.39653292766207e-08
2270 2.57716070706238e-08
2271 2.55953249705954e-08
2272 2.49233700344575e-08
2273 2.5133342518302e-08
2274 2.62572488196611e-08
2275 2.40078374957875e-08
2276 2.64787178849701e-08
2277 2.54980427882856e-08
2278 2.48879139519431e-08
2279 2.37199344610417e-08
2280 2.39001156643326e-08
2281 2.41413680157621e-08
2282 2.57115413404563e-08
2283 2.52512464271604e-08
2284 2.45764919526437e-08
2285 2.45552609356992e-08
2286 2.4821849464729e-08
2287 2.42675373129941e-08
2288 2.55736143373042e-08
2289 2.60310919486528e-08
2290 2.53411833739392e-08
2291 2.45015279176641e-08
2292 2.44391333836802e-08
2293 2.39383393108028e-08
2294 2.53841108133201e-08
2295 2.39263187040706e-08
2296 2.50555363123794e-08
2297 2.42222100155232e-08
2298 2.42468090050352e-08
2299 2.31771632996924e-08
2300 2.34532251397468e-08
2301 2.48316016637773e-08
2302 2.35079529176119e-08
2303 2.40805295703694e-08
2304 2.29093153336635e-08
2305 2.34603092508223e-08
2306 2.36304291689748e-08
2307 2.37833930327724e-08
2308 2.38169572952529e-08
2309 2.31016805685158e-08
2310 2.3214310473918e-08
2311 2.37339712327866e-08
2312 2.35366801604187e-08
2313 2.33333441457262e-08
2314 2.39131221491107e-08
2315 2.30250414290367e-08
2316 2.35667325654276e-08
2317 2.45973836854319e-08
2318 2.34140919985748e-08
2319 2.37496244892554e-08
2320 2.30507115617229e-08
2321 2.35356605315928e-08
2322 2.41572912784704e-08
2323 2.41858746363732e-08
2324 2.43206574879196e-08
2325 2.30250289945388e-08
2326 2.43181439429918e-08
2327 2.45916176311312e-08
2328 2.31661267946492e-08
2329 2.34198207493819e-08
2330 2.51019933728003e-08
2331 2.40404229856495e-08
2332 2.41336746142906e-08
2333 2.39274484670204e-08
2334 2.38620412318369e-08
2335 2.2934251830975e-08
2336 2.40605650958514e-08
2337 2.4044497948239e-08
2338 2.39141204616544e-08
2339 2.28941736679644e-08
2340 2.41503634867968e-08
2341 2.2607261840335e-08
2342 2.5036571926762e-08
2343 2.38820501152759e-08
2344 2.4290445210795e-08
2345 2.34421477784963e-08
2346 2.46561242533971e-08
2347 2.35689618932611e-08
2348 2.3869871412785e-08
2349 2.34490720174563e-08
2350 2.35291643946312e-08
2351 2.3724222586452e-08
2352 2.4325077063736e-08
2353 2.32712960013259e-08
2354 2.49907685656581e-08
2355 2.40652635596916e-08
2356 2.46264466596813e-08
2357 2.3929223047503e-08
2358 2.31169519082641e-08
2359 2.42655424642635e-08
2360 2.45650486618842e-08
2361 2.31663754846068e-08
2362 2.30885355279042e-08
2363 2.36816433130116e-08
2364 2.32601689020839e-08
2365 2.43155575674336e-08
2366 2.48330955798792e-08
2367 2.46539588744099e-08
2368 2.40444233412518e-08
2369 2.38423005782806e-08
2370 2.38251587347804e-08
2371 2.34078747496369e-08
2372 2.43472584315896e-08
2373 2.49404905616757e-08
2374 2.40152449038078e-08
2375 2.40954154406836e-08
2376 2.27835190713677e-08
2377 2.3681483440896e-08
2378 2.38097435101281e-08
2379 2.47875817649401e-08
2380 2.43286226719874e-08
2381 2.40076314383941e-08
2382 2.3973090179652e-08
2383 2.28772947252764e-08
2384 2.33569661389765e-08
2385 2.39684574410148e-08
2386 2.37619719456461e-08
2387 2.32836914193513e-08
2388 2.26808989367555e-08
2389 2.51337475276614e-08
2390 2.36828583410897e-08
2391 2.50531826395672e-08
2392 2.27252314743964e-08
2393 2.55167034168835e-08
2394 2.28968968229992e-08
2395 2.4354752881095e-08
2396 2.38391670848159e-08
2397 2.42513671366851e-08
2398 2.30701378001186e-08
2399 2.2927814313789e-08
2400 2.46211815380093e-08
2401 2.44392523995884e-08
2402 2.2522536724523e-08
2403 2.47426381605464e-08
2404 2.32891128604251e-08
2405 2.4188210545617e-08
2406 2.42865514366031e-08
2407 2.3125592107931e-08
2408 2.36332002856443e-08
2409 2.48110936240664e-08
2410 2.49585134781682e-08
2411 2.44567814888796e-08
2412 2.35155379613161e-08
2413 2.31283525664594e-08
2414 2.42369964098543e-08
2415 2.43785596154567e-08
2416 2.31741097422855e-08
2417 2.32638903696625e-08
2418 2.42231799063575e-08
2419 2.39833450876858e-08
2420 2.30778720577973e-08
2421 2.4049619185007e-08
2422 2.4116305397115e-08
2423 2.30973480341845e-08
2424 2.45128504161585e-08
2425 2.40815172247721e-08
2426 2.35073009946518e-08
2427 2.3924135561515e-08
2428 2.2997806325975e-08
2429 2.47442102363493e-08
2430 2.30585897043056e-08
2431 2.46623770294718e-08
2432 2.48060079144352e-08
2433 2.29451213584753e-08
2434 2.46052866970103e-08
2435 2.24560761097337e-08
2436 2.41450148763533e-08
2437 2.48247626899456e-08
2438 2.25510188300859e-08
2439 2.40195596745707e-08
2440 2.30777708054575e-08
2441 2.33427464024771e-08
2442 2.45880347193861e-08
2443 2.41256383759492e-08
2444 2.38804656049751e-08
2445 2.22780460745753e-08
2446 2.43077256101287e-08
2447 2.28109708899638e-08
2448 2.32429808733059e-08
2449 2.26642260514609e-08
2450 2.42874342859523e-08
2451 2.45713067670295e-08
2452 2.2259024845539e-08
2453 2.46443043749878e-08
2454 2.33096955071233e-08
2455 2.37719355311583e-08
2456 2.36389698926587e-08
2457 2.23090168560702e-08
2458 2.28087717601966e-08
2459 2.29734755663458e-08
2460 2.42935502825503e-08
2461 2.36429436029084e-08
2462 2.33023929041565e-08
2463 2.25091039141034e-08
2464 2.22595204490972e-08
2465 2.31001884287707e-08
2466 2.35295001260738e-08
2467 2.35429453709912e-08
2468 2.33090879930842e-08
2469 2.37909905109746e-08
2470 2.3402940030337e-08
2471 2.51559697517223e-08
2472 2.34010784083694e-08
2473 2.26829488525482e-08
2474 2.36881554371848e-08
2475 2.32327277416289e-08
2476 2.40137278950669e-08
2477 2.31959429441986e-08
2478 2.37782948886434e-08
2479 2.32337793448778e-08
2480 2.73102429559913e-08
2481 2.28455068196354e-08
2482 2.3165222629018e-08
2483 2.23524843079304e-08
2484 2.44838815888215e-08
2485 2.45041906765664e-08
2486 2.42697915098233e-08
2487 2.2028009638575e-08
2488 2.45253257702416e-08
2489 2.38113884165614e-08
2490 2.21680878098596e-08
2491 2.2555132872526e-08
2492 2.252449426976e-08
2493 2.34020607337015e-08
2494 2.3503151425075e-08
2495 2.32580106285241e-08
2496 2.2955720879736e-08
2497 2.32179573345093e-08
2498 2.34077948135791e-08
2499 2.25893916905306e-08
2500 2.31627463875839e-08
2501 2.36556179089575e-08
2502 2.25007550369583e-08
2503 2.40445121590938e-08
2504 2.340478744145e-08
2505 2.36749233550881e-08
2506 2.24270468862642e-08
2507 2.30545182944297e-08
2508 2.27534115992967e-08
2509 2.54609684446905e-08
2510 2.4019589872637e-08
2511 2.41321949090434e-08
2512 2.29546444074913e-08
2513 2.24568328377472e-08
2514 2.41402524636669e-08
2515 2.23657359299523e-08
2516 2.37686474946486e-08
2517 2.40226736281102e-08
2518 2.49105163163676e-08
2519 2.31789289983908e-08
2520 2.29863346135062e-08
2521 2.3046927921655e-08
2522 2.28082317477174e-08
2523 2.36241319839792e-08
2524 2.34657306918962e-08
2525 2.2598579008104e-08
2526 2.35681696381107e-08
2527 2.31465726585611e-08
2528 2.2163570534417e-08
2529 2.36501893624563e-08
2530 2.24105995982882e-08
2531 2.37865709351581e-08
2532 2.45845743762629e-08
2533 2.23218226125255e-08
2534 2.31436310116351e-08
2535 2.3705297280685e-08
2536 2.40123760875122e-08
2537 2.43854447745662e-08
2538 2.25257466013318e-08
2539 2.40806254936388e-08
2540 2.21384617304921e-08
2541 2.34297541368278e-08
2542 2.21137455014286e-08
2543 2.39620163711152e-08
2544 2.35193962083713e-08
2545 2.36693704636082e-08
2546 2.39383179945207e-08
2547 2.34922179487285e-08
2548 2.33410517580523e-08
2549 2.38451143275142e-08
2550 2.37262707258878e-08
2551 2.36299353417735e-08
2552 2.13952038308207e-08
2553 2.44967779394756e-08
2554 2.19389395539338e-08
2555 2.3834642703946e-08
2556 2.37528183788527e-08
2557 2.38888997472486e-08
2558 2.3483694988613e-08
2559 2.35736816733834e-08
2560 2.37466721841884e-08
2561 2.42413200624014e-08
2562 2.31004939621471e-08
2563 2.44411957339707e-08
2564 2.26923244639465e-08
2565 2.37675958913997e-08
2566 2.26162129024488e-08
2567 2.43558559986923e-08
2568 2.43333033722593e-08
2569 2.43535005495232e-08
2570 2.37304114136805e-08
2571 2.45649616203991e-08
2572 2.39177548877478e-08
2573 2.47130387265315e-08
2574 2.47279015042068e-08
2575 2.43023183799096e-08
2576 2.42059474686585e-08
2577 2.47364724259569e-08
2578 2.55572647489544e-08
2579 2.38471571378795e-08
2580 2.45298128476179e-08
2581 2.44088997902736e-08
2582 2.41621780361356e-08
2583 2.48744118636068e-08
2584 2.40262032491501e-08
2585 2.46007658688541e-08
2586 2.40122837169565e-08
2587 2.47005438325232e-08
2588 2.38810500263753e-08
2589 2.46725466723774e-08
2590 2.41036239856385e-08
2591 2.5193502395382e-08
2592 2.50710758820105e-08
2593 2.33489796386266e-08
2594 2.40819186814178e-08
2595 2.49574281241394e-08
2596 2.45358577899424e-08
2597 2.1008128570088e-08
2598 2.44540547811312e-08
2599 2.3324840725536e-08
2600 2.35798296444045e-08
2601 2.55435850249341e-08
2602 2.3304886909159e-08
2603 2.40400375162153e-08
2604 2.39123583156697e-08
2605 2.11836024277545e-08
2606 2.57686885163366e-08
2607 2.45535680676312e-08
2608 2.37448372075733e-08
2609 2.49333123036877e-08
2610 2.42292745866735e-08
2611 2.35932464676125e-08
2612 2.35426398376148e-08
2613 2.42547280038252e-08
2614 2.36134223285944e-08
2615 2.46216398380739e-08
2616 2.4224467765066e-08
2617 2.38815633935019e-08
2618 2.38911805894304e-08
2619 2.44053062203875e-08
2620 2.41206610240852e-08
2621 2.4624990047073e-08
2622 2.4296738843077e-08
2623 2.41883686413757e-08
2624 2.51248319926844e-08
2625 2.56406149645727e-08
2626 2.44284965589259e-08
2627 2.46490223787532e-08
2628 2.48085516574292e-08
2629 2.07800177065565e-08
2630 2.73568190323203e-08
2631 2.32130297206368e-08
2632 2.39174724470104e-08
2633 2.46705802453562e-08
2634 2.43585187575945e-08
2635 2.464148174397e-08
2636 2.41008226709027e-08
2637 2.41630129238501e-08
2638 2.5257532954015e-08
2639 2.24253149383458e-08
2640 2.40487043612347e-08
2641 2.36350743421099e-08
2642 2.06582981832071e-08
2643 2.35133335024784e-08
2644 2.35230395162489e-08
2645 2.47296707556188e-08
2646 2.45554527822378e-08
2647 2.41323121485948e-08
2648 2.4323750125177e-08
2649 2.32822667811661e-08
2650 2.06544736869319e-08
2651 2.36330350844582e-08
2652 2.3442881413871e-08
2653 2.61470560758426e-08
2654 2.39864128559475e-08
2655 2.40724542521775e-08
2656 2.47377194284581e-08
2657 2.61457397954246e-08
2658 2.37719852691498e-08
2659 2.442785351775e-08
2660 2.45148328303912e-08
2661 2.46123832425837e-08
2662 2.60035122323643e-08
2663 2.53411283068772e-08
2664 2.50501042131646e-08
2665 2.46784050972337e-08
2666 2.61996149220067e-08
2667 2.5181932983287e-08
2668 2.42487487867038e-08
2669 2.48584228756954e-08
2670 2.37063240149382e-08
2671 2.02284624606364e-08
2672 2.14655653252294e-08
2673 2.28382415201622e-08
2674 2.41758435493011e-08
2675 2.02157259820979e-08
2676 2.13292263850917e-08
2677 2.42073721068436e-08
2678 2.52964209579432e-08
2679 2.32953709655703e-08
2680 2.42863240629276e-08
2681 2.07415098429919e-08
2682 2.24619842725815e-08
2683 2.54977781111165e-08
2684 2.54084824291567e-08
2685 2.52030876168874e-08
2686 1.96926368545292e-08
2687 2.29569909748761e-08
2688 2.36259651842374e-08
2689 2.50541454249742e-08
2690 2.35756818511845e-08
2691 2.02160865825363e-08
2692 2.4328748793323e-08
2693 2.36031638678469e-08
2694 2.05089207838682e-08
2695 2.3151109473929e-08
2696 2.46268214709744e-08
2697 2.05496224481294e-08
2698 2.29467023160623e-08
2699 2.4501495943241e-08
2700 2.42689175422584e-08
2701 1.98683309804437e-08
2702 2.44220750289514e-08
2703 2.42330866484508e-08
2704 1.94241085438307e-08
2705 2.46664964009824e-08
2706 2.46365026157491e-08
2707 2.53168241926005e-08
2708 2.01444283476349e-08
2709 2.16713349487918e-08
2710 2.36661126251647e-08
2711 2.49551561637418e-08
2712 2.02950865002549e-08
2713 2.33756001222218e-08
2714 2.49487595027631e-08
2715 2.38071660163541e-08
2716 1.97896845577361e-08
2717 2.2804940158494e-08
2718 2.48108467104657e-08
2719 2.54440486457952e-08
2720 2.03467873660657e-08
2721 2.36255779384464e-08
2722 2.55229384293898e-08
2723 1.97566283333117e-08
2724 2.25922676122536e-08
2725 2.48159590654495e-08
2726 2.02711554209145e-08
2727 2.22313882858316e-08
2728 2.43956623791064e-08
2729 2.42411228867923e-08
2730 2.01182217551832e-08
2731 2.3114221647802e-08
2732 2.47217339932604e-08
2733 2.04458743269242e-08
2734 2.29983072586037e-08
2735 2.48157636661972e-08
2736 2.03081729210908e-08
2737 2.33477095434864e-08
2738 2.44205544674969e-08
2739 1.95528819801893e-08
2740 2.41626523234117e-08
2741 2.2857699732981e-08
2742 1.99073557638485e-08
2743 2.41296973513272e-08
2744 2.41449455984366e-08
2745 2.4320055302951e-08
2746 1.9251491067962e-08
2747 2.37153496840392e-08
2748 2.44433078222528e-08
2749 1.99098550979215e-08
2750 2.19826521430377e-08
2751 2.50953853253577e-08
2752 1.93275049298336e-08
2753 2.24707346063724e-08
2754 2.3759639589116e-08
2755 1.96702902854895e-08
2756 2.38360531312765e-08
2757 2.45196893899902e-08
2758 2.06951504821973e-08
2759 2.33391830306573e-08
2760 2.43898092833206e-08
2761 2.03455350344939e-08
2762 2.02606376120684e-08
2763 2.11437303221373e-08
2764 2.00330187993814e-08
2765 2.42103901371138e-08
2766 2.03736814086142e-08
2767 2.32077699280353e-08
2768 2.42571953634751e-08
2769 2.00225347413152e-08
2770 2.38721984402446e-08
2771 2.46280809079735e-08
2772 2.03431849143954e-08
2773 2.35233468259821e-08
2774 2.10383372944989e-08
2775 2.36639827733143e-08
2776 2.31867272049158e-08
2777 2.02132532933774e-08
2778 2.33484378497906e-08
2779 2.05976622424942e-08
2780 2.40115873850755e-08
2781 2.08464268069974e-08
2782 2.39684094793802e-08
2783 2.00076168965779e-08
2784 2.39220732112244e-08
2785 2.04529762015682e-08
2786 2.37376820422242e-08
2787 2.08352108899135e-08
2788 2.43823770063045e-08
2789 2.57667025493902e-08
2790 2.02944754335022e-08
2791 2.34815011879164e-08
2792 2.09620214519646e-08
2793 2.41069333384303e-08
2794 2.04429770889192e-08
2795 2.40850663857373e-08
2796 2.05177919099242e-08
2797 2.42363231706122e-08
2798 2.0669592259992e-08
2799 2.32423733592668e-08
2800 2.07153441067476e-08
2801 2.39599753371067e-08
2802 2.03731129744256e-08
2803 2.40497719516952e-08
2804 2.13374136137645e-08
2805 2.3932342330113e-08
2806 2.03582342095388e-08
2807 2.44325484288765e-08
2808 2.05887076276667e-08
2809 2.33818564510102e-08
2810 2.05084909055131e-08
2811 2.49170675203914e-08
2812 2.11569908259435e-08
2813 2.42230342450966e-08
2814 2.0061614591782e-08
2815 2.41795969913028e-08
2816 2.06415116110747e-08
2817 2.37662458602017e-08
2818 2.03914218843693e-08
2819 2.5462842501156e-08
2820 2.08816341995544e-08
2821 2.22687450701642e-08
2822 2.25073328863346e-08
2823 2.19328359918336e-08
2824 2.19844640270139e-08
2825 2.23459384329772e-08
2826 2.25904965844848e-08
2827 2.21540403799736e-08
2828 2.21720934945324e-08
2829 2.21811777834091e-08
2830 2.28520882217254e-08
2831 2.26054801544251e-08
2832 2.28203926866399e-08
2833 2.26626255539486e-08
2834 2.31030288233569e-08
2835 2.13754898226171e-08
2836 2.36985666646206e-08
2837 2.13967474849142e-08
2838 2.34384760489093e-08
2839 2.3040069407898e-08
2840 2.35588828445543e-08
2841 2.29961578668281e-08
2842 2.35688126792866e-08
2843 2.48618938769596e-08
2844 2.30174812543282e-08
2845 2.37767956434709e-08
2846 2.43928290899476e-08
2847 2.42988935639232e-08
2848 2.34965789047692e-08
2849 2.41513991028341e-08
2850 2.60107171357049e-08
2851 2.13473274612852e-08
2852 2.60898236348339e-08
2853 2.21610569894892e-08
2854 2.48296050386898e-08
2855 2.48793838864003e-08
2856 2.38891644244177e-08
2857 2.33013466299781e-08
2858 2.51510385851361e-08
2859 2.46120119840043e-08
2860 2.47038940415223e-08
2861 2.56602916692827e-08
2862 2.34291679390708e-08
2863 2.30847145843427e-08
2864 2.28873897611948e-08
2865 2.52337404305081e-08
2866 2.30532251066506e-08
2867 2.19264428835686e-08
2868 2.3937463566881e-08
2869 2.1450183851357e-08
2870 2.23905320950735e-08
2871 2.39292763382082e-08
2872 2.50526408507312e-08
2873 1.9605003842571e-08
2874 2.55439314145178e-08
2875 2.51478073920453e-08
2876 2.35921113755921e-08
2877 1.90104749719922e-08
2878 2.13803446058591e-08
2879 2.3791848491328e-08
2880 2.27297221044864e-08
2881 2.26986056617307e-08
2882 2.25219114469155e-08
2883 2.40328787981525e-08
2884 2.441475821513e-08
2885 2.14057660485878e-08
2886 2.34942092447454e-08
2887 2.3724499698119e-08
2888 2.51422278552127e-08
2889 2.26909744327486e-08
2890 2.15261763969465e-08
2891 2.28088534726112e-08
2892 2.51047573840424e-08
2893 2.27535892349806e-08
2894 2.26417764537246e-08
2895 2.49010909669778e-08
2896 2.62092907377109e-08
2897 1.9223351799269e-08
2898 2.08864587847302e-08
2899 2.38407018571252e-08
2900 2.43721860471169e-08
2901 2.45453932734563e-08
2902 1.90969569047184e-08
2903 3.20718491764183e-08
2904 1.82908621582101e-08
2905 2.21693206015061e-08
2906 2.23756657646845e-08
2907 2.56133976250794e-08
2908 1.76748020663808e-08
2909 2.301770507529e-08
2910 2.37735928720895e-08
2911 2.08002681745256e-08
2912 2.23680860500508e-08
2913 2.33504451330191e-08
2914 2.19621014707627e-08
2915 2.1643492331691e-08
2916 2.31749677226389e-08
2917 2.37061019703333e-08
2918 2.25958185495756e-08
2919 2.56735805947983e-08
2920 1.99802041578323e-08
2921 3.09220702376933e-08
2922 1.81413621902493e-08
2923 2.22161240515106e-08
2924 2.47198901348611e-08
2925 1.76682029007225e-08
2926 2.13121111869441e-08
2927 1.80191257470597e-08
2928 2.13711057739374e-08
2929 2.51510474669203e-08
2930 2.53335592503845e-08
2931 1.73088956501033e-08
2932 2.10104555975477e-08
2933 1.98800549355838e-08
2934 2.07817620889728e-08
2935 2.05756229831877e-08
2936 2.10988009285984e-08
2937 2.32755930085204e-08
2938 2.20646505511013e-08
2939 2.07221813042224e-08
2940 1.98404475071357e-08
2941 2.53251926096709e-08
2942 1.92404598919893e-08
2943 2.10266239974999e-08
2944 3.01988833939504e-08
2945 1.77943437762451e-08
2946 1.96526190876511e-08
2947 2.30355077235345e-08
2948 2.54249545861285e-08
2949 1.90517042142346e-08
2950 2.16014477416593e-08
2951 2.16076294634604e-08
2952 2.66387036873539e-08
2953 1.70469878213453e-08
2954 2.16288942311849e-08
2955 2.58975152434004e-08
2956 2.14862154734874e-08
2957 1.95273877068303e-08
2958 2.23082494699156e-08
2959 2.25842153867006e-08
2960 2.21288924961982e-08
2961 2.38998030255289e-08
2962 2.02244390123951e-08
2963 2.12040180969097e-08
2964 2.6921824769488e-08
2965 1.95077483056139e-08
2966 2.1177879006018e-08
2967 2.05985148937771e-08
2968 3.12864010254543e-08
2969 2.05858974311468e-08
2970 1.71962639683443e-08
2971 2.12820001621594e-08
2972 1.75531909007987e-08
2973 1.805556770762e-08
2974 2.28304415372804e-08
2975 1.77910290943828e-08
2976 2.04134931180988e-08
2977 2.43104043562425e-08
2978 1.8405032164992e-08
2979 1.96865794777068e-08
2980 2.13266933002387e-08
2981 2.11225898993916e-08
2982 1.93742142329256e-08
2983 1.98864835709855e-08
2984 2.58168579847506e-08
2985 1.70132405941104e-08
2986 1.8590068151525e-08
2987 2.20784652782413e-08
2988 1.89857374266467e-08
2989 1.73735994479784e-08
2990 2.61494133013684e-08
2991 1.74531482599605e-08
2992 1.81797208398393e-08
2993 2.27527134910588e-08
2994 1.93696632067031e-08
2995 1.72244423168877e-08
2996 2.27311467426716e-08
2997 2.63231711983281e-08
2998 1.91827638218456e-08
2999 1.85159354515463e-08
3000 1.94147453669302e-08
3001 2.76541634036676e-08
3002 1.73747096710031e-08
3003 2.30126619982229e-08
3004 1.94727363123093e-08
3005 1.69130949245755e-08
3006 2.3655513103904e-08
3007 1.89766939939773e-08
3008 1.81062862480985e-08
3009 2.62923940397286e-08
3010 1.96315816936021e-08
3011 2.27676686392897e-08
3012 1.74424226173642e-08
3013 2.07005772523416e-08
3014 2.60157388964899e-08
3015 2.38322996892748e-08
3016 2.06220409637581e-08
3017 1.89086826196672e-08
3018 1.69760490109638e-08
3019 2.5022503180594e-08
3020 1.94316580603981e-08
3021 1.99759924157661e-08
3022 1.70417742140216e-08
3023 2.28809113878015e-08
3024 1.92622575667656e-08
3025 1.86424369275073e-08
3026 2.58157086818755e-08
3027 1.72400476117218e-08
3028 2.20609130963112e-08
3029 2.24541185644966e-08
3030 1.8641124199803e-08
3031 2.55271448423855e-08
3032 1.72103309381555e-08
3033 1.97695282366794e-08
3034 1.93285760730078e-08
3035 2.55458747489001e-08
3036 1.70463234638873e-08
3037 1.82660517822342e-08
3038 2.18010498542753e-08
3039 2.52455478744196e-08
3040 1.93548768123719e-08
3041 2.53149288198529e-08
3042 1.8685909708438e-08
3043 2.2169432511987e-08
3044 1.77776549037389e-08
3045 2.21515481513279e-08
3046 2.18880433777713e-08
3047 1.85143491648887e-08
3048 2.55867096399243e-08
3049 1.71529705994544e-08
3050 2.28199237284343e-08
3051 1.66539972923374e-08
3052 1.99112815124636e-08
3053 2.56480170435225e-08
3054 2.02834371521021e-08
3055 1.70760543483084e-08
3056 2.35406396598137e-08
3057 1.98030303266705e-08
3058 1.69221010537512e-08
3059 2.59430752436174e-08
3060 1.67164646569518e-08
3061 2.60332129187191e-08
3062 1.93313578478183e-08
3063 1.78507484349666e-08
3064 1.85905744132242e-08
3065 1.97808081026096e-08
3066 2.02665244586342e-08
3067 1.9169574372313e-08
3068 2.43081927919775e-08
3069 1.92354239203496e-08
3070 2.17134878965908e-08
3071 1.98020497776952e-08
3072 2.45625138006744e-08
3073 2.01721199744043e-08
3074 1.73513807766312e-08
3075 1.72133365339278e-08
3076 2.12334541060955e-08
3077 2.38403341512594e-08
3078 1.97552747494001e-08
3079 2.31983516840728e-08
3080 1.65526561346496e-08
3081 2.46404976422809e-08
3082 1.90768041363754e-08
3083 2.35618458077624e-08
3084 1.92861531189692e-08
3085 1.88053164151825e-08
3086 1.69210778722118e-08
3087 2.04111803014939e-08
3088 1.7441683652919e-08
3089 2.44652973435677e-08
3090 2.28965841841955e-08
3091 1.77707839554841e-08
3092 2.19809539458993e-08
3093 2.25175025292401e-08
3094 1.64587952156126e-08
3095 2.25458940406043e-08
3096 1.91919777847716e-08
3097 2.09452757360395e-08
3098 1.6843550554313e-08
3099 2.44167068785828e-08
3100 1.71559157990941e-08
3101 2.42636470915158e-08
3102 1.70975535951357e-08
3103 2.39262494261538e-08
3104 1.9369299053551e-08
3105 2.16635225314121e-08
3106 1.92620195349491e-08
3107 2.49584601874631e-08
3108 1.65902438453713e-08
3109 2.04116634705542e-08
3110 1.86835542592689e-08
3111 2.39836808191285e-08
3112 1.74760295124088e-08
3113 2.17473807850865e-08
3114 1.91152640383052e-08
3115 1.66776796817203e-08
3116 2.42598492405932e-08
3117 1.65739493240835e-08
3118 2.47037625911162e-08
3119 1.64595004292778e-08
3120 2.21139639933199e-08
3121 1.9343383783621e-08
3122 2.38506530081395e-08
3123 1.61451740865459e-08
3124 2.2162508273027e-08
3125 2.19748361729444e-08
3126 2.22078071487886e-08
3127 2.31292016650286e-08
3128 1.83023889377409e-08
3129 2.41758737473674e-08
3130 1.60212945132798e-08
3131 2.21153104718042e-08
3132 1.81740738014469e-08
3133 2.19910845089544e-08
3134 2.18659490514028e-08
3135 2.42354332158357e-08
3136 1.65278670749558e-08
3137 2.32149055534592e-08
3138 1.83252524266209e-08
3139 2.2223497708751e-08
3140 1.69197065247317e-08
3141 2.34254962094838e-08
3142 1.62275615167573e-08
3143 2.28411618508062e-08
3144 1.89072206779883e-08
3145 2.4463878034453e-08
3146 2.04366017442226e-08
3147 2.22681109107725e-08
3148 2.3811359994852e-08
3149 1.71994205544479e-08
3150 2.45656028852181e-08
3151 1.76938872442634e-08
3152 2.26137526482262e-08
3153 2.21298801506009e-08
3154 1.98210976520841e-08
3155 1.73381824453145e-08
3156 2.48865248408947e-08
3157 1.78723045252127e-08
3158 2.43001654354202e-08
3159 1.92741005378139e-08
3160 2.30205152718099e-08
3161 1.6226879395731e-08
3162 2.26982930229269e-08
3163 2.4978923818253e-08
3164 2.42074076339804e-08
3165 1.89828970320605e-08
3166 2.40105233473287e-08
3167 1.80445116626515e-08
3168 2.3461430131988e-08
3169 2.2826275980492e-08
3170 1.66273874668832e-08
3171 2.20490665725492e-08
3172 1.87024902231769e-08
3173 1.84871993269553e-08
3174 1.95266682823103e-08
3175 1.91690148199086e-08
3176 1.70070855176618e-08
3177 2.43154634205212e-08
3178 2.00483434298349e-08
3179 1.63491620241984e-08
3180 2.31249810411782e-08
3181 1.67736828871057e-08
3182 2.46631977063316e-08
3183 1.78854726584632e-08
3184 2.54157725976256e-08
3185 1.62009374804484e-08
3186 1.91332922838683e-08
3187 2.40392985517701e-08
3188 1.78196319922108e-08
3189 2.0463076566557e-08
3190 2.70697952942101e-08
3191 1.80137451621931e-08
3192 1.93897093936357e-08
3193 2.37283099835395e-08
3194 1.61209037230492e-08
3195 2.01281356027039e-08
3196 1.79457995130861e-08
3197 2.35032651119127e-08
3198 2.10316688509238e-08
3199 1.93293310246645e-08
3200 1.63880180537035e-08
3201 2.00759462387623e-08
3202 2.42654287774258e-08
3203 1.62020423744025e-08
3204 1.69520877335572e-08
3205 2.23430767221089e-08
3206 1.73388148283493e-08
3207 1.85408186581526e-08
3208 1.97190388462332e-08
3209 1.61338586934789e-08
3210 2.07643768845855e-08
3211 1.76777810168005e-08
3212 2.41893669539195e-08
3213 1.90881923600728e-08
3214 1.71381451252728e-08
3215 1.82221597810894e-08
3216 1.76633871973308e-08
3217 1.78226464697673e-08
3218 2.41349304985761e-08
3219 1.72572747203503e-08
3220 2.09328199218817e-08
3221 1.92405984478228e-08
3222 1.6637576649714e-08
3223 2.14243485174848e-08
3224 2.70648552458397e-08
3225 2.42054607468845e-08
3226 1.68112208598359e-08
3227 2.49820484299335e-08
3228 2.35875976528632e-08
3229 1.60404347582244e-08
3230 2.34091537265613e-08
3231 1.6866977148311e-08
3232 2.4201771253729e-08
3233 1.75707377536582e-08
3234 2.39598971774058e-08
3235 1.64518976220052e-08
3236 2.05303187783556e-08
3237 2.20338218781535e-08
3238 1.82099615386733e-08
3239 1.8202564788794e-08
3240 2.196042814262e-08
3241 1.58568678187976e-08
3242 2.2672985267036e-08
3243 2.18513083183325e-08
3244 1.617394218556e-08
3245 2.08102441945357e-08
3246 1.78144858864471e-08
3247 2.46914044765845e-08
3248 1.85638526772891e-08
3249 2.05647499029737e-08
3250 1.73093273048153e-08
3251 2.19060627415502e-08
3252 2.57790464530672e-08
3253 1.62091726707558e-08
3254 2.0933347499863e-08
3255 2.4258314468284e-08
3256 1.65555924525052e-08
3257 2.10901678343589e-08
3258 1.61722475411352e-08
3259 2.37593713592332e-08
3260 1.5891350457764e-08
3261 2.4356113570434e-08
3262 1.80895725065966e-08
3263 2.58314880596799e-08
3264 1.71538960813677e-08
3265 2.40351134550565e-08
3266 1.83521269292442e-08
3267 2.52148133483843e-08
3268 1.88144806401169e-08
3269 1.82159993755704e-08
3270 1.70628826623442e-08
3271 1.91779179203877e-08
3272 1.75470695751301e-08
3273 2.08682333635579e-08
3274 2.48782754397325e-08
3275 1.62471991416169e-08
3276 1.81921429032172e-08
3277 1.81927397591153e-08
3278 1.96999696555622e-08
3279 2.47177602830106e-08
3280 1.89336031297671e-08
3281 2.56188634750742e-08
3282 1.67392855132675e-08
3283 2.28138681279688e-08
3284 2.39821105196825e-08
3285 1.57836286263091e-08
3286 2.37557067350735e-08
3287 1.70364700124992e-08
3288 2.42633166891437e-08
3289 1.66142459789853e-08
3290 2.5294752958871e-08
3291 1.67701639242068e-08
3292 2.48734686181251e-08
3293 2.29633503323612e-08
3294 1.62836908401687e-08
3295 2.50110794297598e-08
3296 1.87330932988061e-08
3297 2.05393639873819e-08
3298 2.5191942754077e-08
3299 1.74684675613435e-08
3300 2.34368240370486e-08
3301 1.62977027429179e-08
3302 2.29375558546963e-08
3303 1.63767115424207e-08
3304 2.52516194620966e-08
3305 1.67870197742559e-08
3306 2.54940495381106e-08
3307 1.87929245498708e-08
3308 2.52579521742291e-08
3309 1.93706064521848e-08
3310 1.85812556452447e-08
3311 2.20619416069212e-08
3312 1.92376994334609e-08
3313 1.96132816654426e-08
3314 1.69340843569898e-08
3315 2.40009878638148e-08
3316 1.68703664371606e-08
3317 2.40165753950805e-08
3318 1.66689613223525e-08
3319 2.43055175985774e-08
3320 1.580139397106e-08
3321 1.86752018294101e-08
3322 1.45755345570819e-08
3323 1.75599659257841e-08
3324 2.45008866528451e-08
3325 1.64929971901984e-08
3326 2.0117006727105e-08
3327 1.60511532953933e-08
3328 2.14978363999307e-08
3329 1.67778768656035e-08
3330 1.763629775553e-08
3331 1.72452292446224e-08
3332 1.55495243348014e-08
3333 1.78185466381819e-08
3334 1.72426961597694e-08
3335 1.5923315999089e-08
3336 1.89978699438598e-08
3337 1.55342281260573e-08
3338 1.94865474867356e-08
3339 2.15180424589789e-08
3340 1.59087516493628e-08
3341 1.65129119267249e-08
3342 1.96255847129123e-08
3343 1.54132990815015e-08
3344 2.01288727907922e-08
3345 1.63588254054048e-08
3346 1.59319561987559e-08
3347 1.69770544289349e-08
3348 1.7031682730817e-08
3349 1.53220263143794e-08
3350 1.75049237327585e-08
3351 1.62880944287735e-08
3352 1.54794275175618e-08
3353 1.66407669865976e-08
3354 1.69982197206764e-08
3355 1.686658990252e-08
3356 2.16968238930804e-08
3357 1.54243497973994e-08
3358 2.08814565638704e-08
3359 2.27403749164523e-08
3360 1.61361768391544e-08
3361 2.15329603037162e-08
3362 1.82162818163079e-08
3363 2.41987230253926e-08
3364 1.6716573014719e-08
3365 1.70616125672041e-08
3366 1.52557131372077e-08
3367 1.84867001706834e-08
3368 1.68615432727393e-08
3369 2.37280151083041e-08
3370 1.62403654968557e-08
3371 1.68717093629311e-08
3372 2.52214817919594e-08
3373 1.68003122524851e-08
3374 1.5669803232754e-08
3375 1.63181450574257e-08
3376 1.76091496939534e-08
3377 2.05309209633242e-08
3378 2.37364581323618e-08
3379 1.66589444461351e-08
3380 2.07248316286268e-08
3381 2.5996193642186e-08
3382 1.66791949141043e-08
3383 2.53194709642912e-08
3384 1.56987862709457e-08
3385 1.63177027445727e-08
3386 1.87777420279645e-08
3387 1.53815520320677e-08
3388 2.44838673779668e-08
3389 1.61781752439083e-08
3390 1.68277534129402e-08
3391 1.85470998559367e-08
3392 1.53909116562545e-08
3393 2.00332621602684e-08
3394 1.52500110317533e-08
3395 2.11896402646516e-08
3396 1.542603556004e-08
3397 1.50750913974207e-08
3398 1.55944608337677e-08
3399 1.60106754520939e-08
3400 1.80094978929901e-08
3401 1.60227688894565e-08
3402 2.08926813627386e-08
3403 1.54582746603182e-08
3404 1.74584044998483e-08
3405 1.63730078384106e-08
3406 1.67662790317991e-08
3407 1.81545640742797e-08
3408 2.56592596059591e-08
3409 1.66231348686097e-08
3410 1.86287056891388e-08
3411 1.59122190979133e-08
3412 2.21283595891464e-08
3413 1.58954041040715e-08
3414 2.5292871796978e-08
3415 1.57801434141902e-08
3416 2.18168221266524e-08
3417 1.8343202512483e-08
3418 1.55048116567968e-08
3419 1.87042275001659e-08
3420 2.17739994923249e-08
3421 1.55748836050407e-08
3422 2.23489298178947e-08
3423 1.75862577833641e-08
3424 2.50327616413415e-08
3425 2.05504573358439e-08
3426 1.65135141116934e-08
3427 2.50503227050558e-08
3428 1.71421703498709e-08
3429 1.87102084936441e-08
3430 1.76455614564475e-08
3431 1.79313293102723e-08
3432 1.75481353892337e-08
3433 1.79881656237058e-08
3434 1.75404757385422e-08
3435 1.75432735005643e-08
3436 1.67118869853766e-08
3437 1.72717395940936e-08
3438 2.62208796897312e-08
3439 1.59263127130771e-08
3440 1.581760855629e-08
3441 1.66843978632869e-08
3442 1.63347060322394e-08
3443 1.88864479611084e-08
3444 1.66814277946514e-08
3445 2.13204476295914e-08
3446 2.11448671905146e-08
3447 1.5657072083286e-08
3448 1.7695406029361e-08
3449 1.77625683051019e-08
3450 1.71872489573843e-08
3451 1.77255383704278e-08
3452 1.76993779632539e-08
3453 1.77066414863702e-08
3454 2.47653257900993e-08
3455 1.44706708837816e-08
3456 1.62240780809952e-08
3457 1.74392287277669e-08
3458 1.57786264054494e-08
3459 1.65708993193903e-08
3460 2.16233253524933e-08
3461 1.63680162756918e-08
3462 2.55872958376813e-08
3463 1.59227226959047e-08
3464 1.98031813170019e-08
3465 1.71641421076174e-08
3466 1.79509456188498e-08
3467 1.71862506448406e-08
3468 2.49124649798205e-08
3469 1.47632395197661e-08
3470 1.73431331518259e-08
3471 1.59857957982013e-08
3472 1.62583351226431e-08
3473 2.47412472731412e-08
3474 1.76524554973412e-08
3475 2.56937973119875e-08
3476 1.62490128019499e-08
3477 2.52193803618184e-08
3478 1.65617759506631e-08
3479 1.93232967404811e-08
3480 1.68270108957813e-08
3481 2.00010603634837e-08
3482 1.6356798582251e-08
3483 1.79983334902545e-08
3484 1.48765471053025e-08
3485 1.66503451026756e-08
3486 2.56270560328176e-08
3487 1.53857904194865e-08
3488 1.94100788775131e-08
3489 1.86768129850634e-08
3490 1.59376085662188e-08
3491 1.90769586794204e-08
3492 1.64215965270387e-08
3493 1.95398914826228e-08
3494 1.74575465194948e-08
3495 2.29363319448339e-08
3496 1.6028858240702e-08
3497 2.37909691946925e-08
3498 1.66950400171118e-08
3499 1.67877534096306e-08
3500 1.64494338150689e-08
3501 1.78796781824531e-08
3502 1.66577631688369e-08
3503 1.69088032464515e-08
3504 1.71813159255407e-08
3505 1.71360419187749e-08
3506 2.33311308051043e-08
3507 1.48920555886889e-08
3508 1.71915743862883e-08
3509 1.50893111339201e-08
3510 1.72810299403636e-08
3511 2.30671659551263e-08
3512 1.48182968118249e-08
3513 2.01591365822651e-08
3514 1.77383974175882e-08
3515 2.49754155134951e-08
3516 1.73454175467214e-08
3517 1.64849360828612e-08
3518 2.49814320341102e-08
3519 1.45040264243335e-08
3520 1.99823464441806e-08
3521 1.79572960945507e-08
3522 1.75555179282583e-08
3523 1.73177703288729e-08
3524 1.83486044136316e-08
3525 1.73503167388844e-08
3526 1.74635097494047e-08
3527 1.7074532010497e-08
3528 1.66461013861863e-08
3529 1.71526846060033e-08
3530 2.41461641792284e-08
3531 1.43259253349015e-08
3532 1.61998183756396e-08
3533 1.495558699105e-08
3534 1.49763685897142e-08
3535 1.77478209906212e-08
3536 1.58444510844902e-08
3537 2.50803751100648e-08
3538 1.55525068379347e-08
3539 2.19625988506778e-08
3540 1.71024421291577e-08
3541 2.51989060728874e-08
3542 1.80076717981592e-08
3543 1.7594874890392e-08
3544 1.79071726336133e-08
3545 1.72920717744773e-08
3546 1.71654068736871e-08
3547 1.72885350480101e-08
3548 1.716554010045e-08
3549 1.68445772885661e-08
3550 2.44516868974642e-08
3551 1.65308318145208e-08
3552 1.61459219327753e-08
3553 1.53588164408802e-08
3554 1.59124269316635e-08
3555 1.52102934691811e-08
3556 1.6855041806707e-08
3557 2.50978331450824e-08
3558 1.69202447608541e-08
3559 2.54806824528941e-08
3560 1.62025912686659e-08
3561 2.24367031620432e-08
3562 1.33862441131782e-08
3563 1.8168851312339e-08
3564 1.49453711628666e-08
3565 2.43178703840385e-08
3566 1.90781346276481e-08
3567 1.83354167404559e-08
3568 2.3403117666021e-08
3569 1.41438851741782e-08
3570 1.90415470058269e-08
3571 1.77333134843138e-08
3572 1.79046786286108e-08
3573 2.43864732851762e-08
3574 1.35754500973917e-08
3575 1.84266806257938e-08
3576 2.08360297904164e-08
3577 2.40612862967282e-08
3578 1.53575641093084e-08
3579 1.89457747268307e-08
3580 1.43638674288127e-08
3581 2.00400531724654e-08
3582 1.70381451169987e-08
3583 2.52094469743724e-08
3584 1.72717484758778e-08
3585 1.77974097681499e-08
3586 1.73621224064391e-08
3587 1.80972214991471e-08
3588 1.81606711890936e-08
3589 1.76380829941536e-08
3590 1.7605350066674e-08
3591 1.71267906523553e-08
3592 1.7738056357075e-08
3593 2.36659758456881e-08
3594 1.33193838181001e-08
3595 1.58027795293947e-08
3596 1.5815993847923e-08
3597 1.52886006077324e-08
3598 1.75654264467084e-08
3599 1.60038577945443e-08
3600 1.49129224524813e-08
3601 2.50428406900483e-08
3602 1.66548819180434e-08
3603 1.76967756004842e-08
3604 1.78528551941781e-08
3605 1.80983583675243e-08
3606 1.75824013126658e-08
3607 1.7784888228789e-08
3608 1.49949617167522e-08
3609 1.71630176737381e-08
3610 1.71343668142754e-08
3611 1.72170633305768e-08
3612 1.69434546393177e-08
3613 1.74722316614861e-08
3614 1.68731446592574e-08
3615 1.72207474946617e-08
3616 1.82616570754135e-08
3617 1.84427264571241e-08
3618 1.78830799058005e-08
3619 1.64291229509672e-08
3620 2.22550049500114e-08
3621 1.37389788434916e-08
3622 1.65952744879405e-08
3623 1.57781165910365e-08
3624 1.51479966348234e-08
3625 1.69206462174998e-08
3626 1.96940526109302e-08
3627 1.61742530480069e-08
3628 2.53774157243924e-08
3629 2.51199914202971e-08
3630 1.39215758920841e-08
3631 2.20703704201242e-08
3632 1.56593298328289e-08
3633 2.69847983958016e-08
3634 1.8665335943524e-08
3635 1.80998469545557e-08
3636 1.94607068237929e-08
3637 1.82984578600554e-08
3638 1.82668493664551e-08
3639 1.8034020499158e-08
3640 2.44439117835782e-08
3641 1.36025306574084e-08
3642 1.84812716241822e-08
3643 1.37804683220111e-08
3644 2.0610738005189e-08
3645 1.69187668319637e-08
3646 1.95507858791188e-08
3647 2.36015598176209e-08
3648 1.38407330041446e-08
3649 2.78192064939731e-08
3650 1.59114250664061e-08
3651 2.6870624836306e-08
3652 1.82802644133062e-08
3653 1.8755315522867e-08
3654 1.83274782017406e-08
3655 1.90086826279412e-08
3656 1.77095440534458e-08
3657 1.8783520516763e-08
3658 1.98042826582423e-08
3659 1.70920362307925e-08
3660 2.42223148205767e-08
3661 1.30493624794781e-08
3662 1.85329120938604e-08
3663 1.6025628823968e-08
3664 2.3327816123242e-08
3665 1.54596975221466e-08
3666 2.6833726352038e-08
3667 1.79745001105402e-08
3668 1.72543916932e-08
3669 2.37070576503129e-08
3670 1.44448151218057e-08
3671 2.45313973579186e-08
3672 1.56109152271711e-08
3673 2.62079904445045e-08
3674 1.69470535382743e-08
3675 2.68560391702977e-08
3676 1.60446180785812e-08
3677 2.58775010308909e-08
3678 1.69852540921056e-08
3679 1.94850553469905e-08
3680 2.64265658245222e-08
3681 1.52725476709747e-08
3682 2.87457044834127e-08
3683 1.97186764694379e-08
3684 1.7771338178818e-08
3685 1.85324271484433e-08
3686 1.93765146150326e-08
3687 1.82017743100005e-08
3688 2.12022133183609e-08
3689 2.47984992540751e-08
3690 1.40887941313395e-08
3691 2.35357440203643e-08
3692 1.58222235313588e-08
3693 2.83438534864899e-08
3694 2.74075055983758e-08
3695 1.54333381630067e-08
3696 2.7483014974905e-08
3697 1.80672241612001e-08
3698 1.82719688268662e-08
3699 2.60336943114226e-08
3700 1.55293005121848e-08
3701 2.63928274790715e-08
3702 1.60192623610556e-08
3703 2.93679374152589e-08
3704 1.68904374930889e-08
3705 1.87419377795095e-08
3706 2.01108001363082e-08
3707 2.17205027297496e-08
3708 1.96334610791382e-08
3709 1.9660618022499e-08
3710 1.88835116432529e-08
3711 2.13220214817511e-08
3712 1.91418507711205e-08
3713 2.65913229213766e-08
3714 1.44846294958256e-08
3715 2.29733654322217e-08
3716 1.76404544305342e-08
3717 1.6651988232752e-08
3718 2.74762772534132e-08
3719 1.70089897721937e-08
3720 2.59665426938227e-08
3721 1.60763438117328e-08
3722 2.76067684268355e-08
3723 1.71882224009323e-08
3724 2.82880545654507e-08
3725 2.18911164751034e-08
3726 1.47033540898178e-08
3727 2.25438157031022e-08
3728 1.93359355193934e-08
3729 2.06398649282846e-08
3730 1.89981452791699e-08
3731 2.59217802778267e-08
3732 1.42972247374473e-08
3733 2.44316264996769e-08
3734 1.94928286845197e-08
3735 1.64308389116741e-08
3736 1.98244549665105e-08
3737 2.0019484736622e-08
3738 1.67139315720988e-08
3739 2.67584177038316e-08
3740 1.47380792014928e-08
3741 2.40963089481738e-08
3742 1.50539758436707e-08
3743 2.65775277341618e-08
3744 1.76002359353333e-08
3745 2.61384069943915e-08
3746 1.56478687785011e-08
3747 2.76230540663391e-08
3748 2.66411159799418e-08
3749 1.52239270079235e-08
3750 2.94234254738512e-08
3751 1.6742244923762e-08
3752 3.01427895976758e-08
3753 2.10014601265129e-08
3754 2.07219308379081e-08
3755 2.70509943334218e-08
3756 1.44374485699927e-08
3757 2.52064396022433e-08
3758 1.63425148969054e-08
3759 2.93729787159691e-08
3760 1.67332103728768e-08
3761 2.3264979276405e-08
3762 1.51185073349325e-08
3763 2.82755561187287e-08
3764 1.80518497927551e-08
3765 2.76366236562353e-08
3766 1.51927270763963e-08
3767 2.95374391612313e-08
3768 2.02061958276545e-08
3769 2.14302726675442e-08
3770 2.65475357252853e-08
3771 1.53409267511506e-08
3772 2.75374496538916e-08
3773 2.80181442491312e-08
3774 1.47472078992905e-08
3775 2.83222050256882e-08
3776 2.79451768392391e-08
3777 1.51977310736129e-08
3778 2.91189614642917e-08
3779 2.71003042229268e-08
3780 1.52364894034918e-08
3781 2.99241662560235e-08
3782 1.67015148377914e-08
3783 2.82754175628952e-08
3784 2.72956750535513e-08
3785 1.54761501391931e-08
3786 2.63534580824398e-08
3787 1.49139118832409e-08
3788 2.82835337372944e-08
3789 2.12785877806709e-08
3790 2.03320933422901e-08
3791 3.012480220832e-08
3792 2.85755969997581e-08
3793 2.37024213589621e-08
3794 1.50006300714267e-08
3795 2.35510846380294e-08
3796 2.40346604840624e-08
3797 1.68975784475833e-08
3798 2.95726518828587e-08
3799 3.03131173495785e-08
3800 1.66933489254006e-08
3801 2.71710103305622e-08
3802 1.41663640818024e-08
3803 2.79202634345666e-08
3804 2.34515784569567e-08
3805 2.05926564689207e-08
3806 2.06686348036556e-08
3807 2.18164295517909e-08
3808 2.67999133996e-08
3809 1.52390366991995e-08
3810 2.73337867895407e-08
3811 1.56257069505727e-08
3812 2.74424856172573e-08
3813 2.44579290153979e-08
3814 2.44170266228139e-08
3815 1.89088371627122e-08
3816 1.97624601128155e-08
3817 2.69364370808489e-08
3818 2.34923369646367e-08
3819 1.50350363270491e-08
3820 2.51692320318853e-08
3821 1.44653986566823e-08
3822 2.84669603445309e-08
3823 2.71441606969347e-08
3824 1.51081867016956e-08
3825 2.87609900340158e-08
3826 2.59617145559332e-08
3827 1.51832999506496e-08
3828 2.85091221741141e-08
3829 2.62106887305436e-08
3830 1.45528531447781e-08
3831 2.8514421046566e-08
3832 2.83574550508092e-08
3833 2.98013986821388e-08
3834 1.88672117928945e-08
3835 2.86589383335922e-08
3836 2.80145542319588e-08
3837 2.31331718225647e-08
3838 1.8109842514491e-08
3839 1.98333012235707e-08
3840 2.04263113090519e-08
3841 2.92059123552235e-08
3842 2.0120845434235e-08
3843 3.05582936732662e-08
3844 1.56747486101949e-08
3845 2.22466685073641e-08
3846 2.94375315235129e-08
3847 1.72780563190145e-08
3848 2.07734363044665e-08
3849 2.84796151106548e-08
3850 1.60184931985441e-08
3851 2.06260555302151e-08
3852 1.68011755619091e-08
3853 2.24391225600584e-08
3854 1.71277445559781e-08
3855 2.77698415374061e-08
3856 1.51910874990335e-08
3857 2.61491273079173e-08
3858 1.50393972830898e-08
3859 2.78493068606167e-08
3860 2.54688981016216e-08
3861 2.01539620547919e-08
3862 2.46862192909703e-08
3863 1.53117873935571e-08
3864 2.56175418655857e-08
3865 1.47978145292882e-08
3866 2.77876299747959e-08
3867 1.77052683625334e-08
3868 2.0782511711559e-08
3869 2.67343995830061e-08
3870 2.71038036459004e-08
3871 1.5146463638871e-08
3872 2.61439332405189e-08
3873 1.61973421342054e-08
3874 3.19527373449091e-08
3875 1.80947825612066e-08
3876 2.69655853202266e-08
3877 1.46693208691318e-08
3878 2.74319127413492e-08
3879 1.6344156250625e-08
3880 2.85444752279318e-08
3881 1.66370579535169e-08
3882 2.36934898367736e-08
3883 1.58487747370373e-08
3884 2.88296426731449e-08
3885 2.0435015457565e-08
3886 3.04881453416783e-08
3887 1.48547556477752e-08
3888 2.22138023531215e-08
3889 2.92990645078817e-08
3890 1.6918519918363e-08
3891 2.20873097589447e-08
3892 2.37456063700847e-08
3893 1.62521462954146e-08
3894 2.43270310562593e-08
3895 1.59819730782829e-08
3896 2.71393361117589e-08
3897 1.55241188792843e-08
3898 2.75017644213449e-08
3899 1.60832254181287e-08
3900 2.82494365677621e-08
3901 2.41916406906739e-08
3902 1.52537484865434e-08
3903 2.75462284093919e-08
3904 1.82083503830199e-08
3905 2.16616449222329e-08
3906 2.9779307908484e-08
3907 3.01606277730571e-08
3908 1.59877924232887e-08
3909 2.23310916425135e-08
3910 2.47836631217524e-08
3911 1.58205519795729e-08
3912 2.29461871725789e-08
3913 2.73387961158278e-08
3914 1.58882187406562e-08
3915 2.56986947277937e-08
3916 2.55737973020587e-08
3917 2.64072870237442e-08
3918 2.34762165263191e-08
3919 1.54417385545003e-08
3920 2.16029452104749e-08
3921 2.83379648635673e-08
3922 1.92284410616139e-08
3923 2.69317066425856e-08
3924 1.80401791283202e-08
3925 3.07326004644892e-08
3926 1.6126834978536e-08
3927 2.64653454706831e-08
3928 2.62068482470568e-08
3929 1.5011654141972e-08
3930 2.55845389318665e-08
3931 2.13198063647724e-08
3932 2.86908932167762e-08
3933 2.73042051190941e-08
3934 3.03131812984248e-08
3935 2.10081392282291e-08
3936 2.33407657646012e-08
3937 2.64990109855034e-08
3938 2.93286674946103e-08
3939 2.15432844896668e-08
3940 2.32377797004801e-08
3941 2.7039279260066e-08
3942 2.78812866127964e-08
3943 1.43879086422771e-08
3944 2.13806305993103e-08
3945 2.33815811157001e-08
3946 2.93033917131424e-08
3947 2.80750960257592e-08
3948 2.17596856089131e-08
3949 1.64201932051355e-08
3950 1.98124610051309e-08
3951 1.81184667269463e-08
3952 2.00743226486111e-08
3953 2.38575896815973e-08
3954 2.78717298130005e-08
3955 2.40720083866108e-08
3956 1.83924839802785e-08
3957 2.0517052945479e-08
3958 2.9687027947034e-08
3959 2.17266205027045e-08
3960 1.87262756412565e-08
3961 2.28909673438693e-08
3962 2.62650559079702e-08
3963 2.64612669553799e-08
3964 1.62184381480301e-08
3965 2.24959766370603e-08
3966 2.38677859698555e-08
3967 2.48049012441243e-08
3968 2.42496209779119e-08
3969 1.6227078347697e-08
3970 2.10660022759157e-08
3971 2.44998545895214e-08
3972 1.44970009330336e-08
3973 2.37256614354919e-08
3974 1.74164842547953e-08
3975 2.49865976797992e-08
3976 1.59546900135865e-08
3977 2.26382184109752e-08
3978 2.29901448989267e-08
3979 1.63200173375344e-08
3980 2.60232173587838e-08
3981 1.68161786717747e-08
3982 2.96624023121694e-08
3983 2.13466861964662e-08
3984 2.68844750905828e-08
3985 2.66156217065827e-08
3986 2.38964688037413e-08
3987 1.81501711438159e-08
3988 2.52756109375696e-08
3989 1.59427617774099e-08
3990 2.16097539862403e-08
3991 2.72897491271351e-08
3992 1.42466625163706e-08
3993 2.13038582330682e-08
3994 1.63842130973535e-08
3995 2.82463883394257e-08
3996 1.52279611143058e-08
3997 2.74152434087682e-08
3998 1.68427032321006e-08
3999 2.94880280193865e-08
4000 1.77806924739343e-08
4001 2.63113832943418e-08
4002 1.58086290724668e-08
4003 2.61437289594824e-08
4004 1.4518377611239e-08
4005 2.25299991996053e-08
4006 2.71873759061236e-08
4007 2.163655743459e-08
4008 2.67076760707141e-08
4009 2.39937048007732e-08
4010 1.40432918627198e-08
4011 2.26295533423126e-08
4012 1.46361998076827e-08
4013 2.53120475690594e-08
4014 1.50779566610026e-08
4015 2.86451413700206e-08
4016 1.80948678263348e-08
4017 2.70743694130715e-08
4018 2.66536730464395e-08
4019 1.91741396093903e-08
4020 2.83721384164437e-08
4021 2.80825336318458e-08
4022 1.50479593230557e-08
4023 2.14644337859227e-08
4024 2.66959325756488e-08
4025 1.42082674514654e-08
4026 2.07700079357664e-08
4027 1.47159280317055e-08
4028 2.24542873183964e-08
4029 1.53580970163603e-08
4030 2.68324846786072e-08
4031 1.49838523810786e-08
4032 2.04894146094148e-08
4033 1.89146032170129e-08
4034 1.33245157130091e-08
4035 1.87156015130086e-08
4036 2.64130743943269e-08
4037 1.26550414591975e-08
4038 1.81647834551768e-08
4039 2.45114115671186e-08
4040 1.3821067845754e-08
4041 2.32793642140905e-08
4042 1.60357238598863e-08
4043 2.64746322642395e-08
4044 2.20789377891606e-08
4045 2.62408406115355e-08
4046 1.48629375473774e-08
4047 1.9782230964438e-08
4048 2.28107701616409e-08
4049 1.43955718456823e-08
4050 1.76303931453958e-08
4051 2.27021796916915e-08
4052 1.68800919908563e-08
4053 2.31759198499049e-08
4054 1.32090152149544e-08
4055 1.66281566293947e-08
4056 2.1973800556907e-08
4057 1.29507808921403e-08
4058 1.7115015182867e-08
4059 2.24671339310589e-08
4060 1.42970861816138e-08
4061 2.03763583783712e-08
4062 1.67653784188815e-08
4063 2.34806680765587e-08
4064 2.15723527929867e-08
4065 2.14326885128457e-08
4066 1.92846822955062e-08
4067 1.66522156064275e-08
4068 1.69134715122254e-08
4069 2.4936637643691e-08
4070 1.41497418226777e-08
4071 1.99649097254451e-08
4072 1.45620155933557e-08
4073 2.36671073849948e-08
4074 1.30306645473866e-08
4075 2.47907703254668e-08
4076 1.45382355043466e-08
4077 2.7420687942481e-08
4078 1.416239747698e-08
4079 1.94400016084728e-08
4080 2.49807534657975e-08
4081 1.77908638931967e-08
4082 2.26329941455106e-08
4083 2.45251889907649e-08
4084 2.46269422632395e-08
4085 1.37948941159038e-08
4086 2.34847217228662e-08
4087 2.20040607956662e-08
4088 1.24841372795004e-08
4089 2.41668498546233e-08
4090 1.36663782512869e-08
4091 2.63450754545147e-08
4092 1.85291089138673e-08
4093 2.02433483309505e-08
4094 2.36641941597782e-08
4095 2.23895035844635e-08
4096 1.2527123338657e-08
4097 2.42450450826937e-08
4098 1.3320175185072e-08
4099 2.5344602860855e-08
4100 1.46314089732869e-08
4101 2.79267275971051e-08
4102 1.42355141008466e-08
4103 2.10657304933193e-08
4104 1.58001025596377e-08
4105 2.72116480459772e-08
4106 1.84138979619775e-08
4107 2.72883138308089e-08
4108 1.63001470099289e-08
4109 1.74152283705098e-08
4110 2.73928577598781e-08
4111 1.82180670549315e-08
4112 2.28481695785376e-08
4113 1.27955734896545e-08
4114 2.12545128164265e-08
4115 1.40557228078819e-08
4116 2.47793856544831e-08
4117 1.55995714123947e-08
4118 2.62516834936832e-08
4119 1.69400120597629e-08
4120 2.39736746010522e-08
4121 1.27006591910117e-08
4122 2.41807285306095e-08
4123 1.36856934673801e-08
4124 2.67994355596102e-08
4125 1.43876768277096e-08
4126 2.1945639971932e-08
4127 1.83201684933465e-08
4128 1.9568492604094e-08
4129 2.53563232632814e-08
4130 2.37411299508494e-08
4131 1.31109567647059e-08
4132 1.98889136271418e-08
4133 1.43694149912221e-08
4134 2.5733422504004e-08
4135 1.36912898796027e-08
4136 1.94180795887178e-08
4137 1.58436606056966e-08
4138 2.59760390974861e-08
4139 1.90703453029073e-08
4140 2.53829739449429e-08
4141 1.38023530382725e-08
4142 1.78304553344333e-08
4143 2.43037021618875e-08
4144 1.51236374534847e-08
4145 2.12422701650894e-08
4146 1.26513679532536e-08
4147 1.57419961510641e-08
4148 1.86439130800409e-08
4149 1.60323061493273e-08
4150 2.38767530191808e-08
4151 2.3857593234311e-08
4152 1.48682515188625e-08
4153 2.16071391889727e-08
4154 1.34521371819574e-08
4155 2.22392451121323e-08
4156 1.28978223656873e-08
4157 2.24449241414959e-08
4158 1.57642823239712e-08
4159 2.41173534476502e-08
4160 1.66397793321948e-08
4161 2.25670593323457e-08
4162 1.409428485033e-08
4163 2.27539231900664e-08
4164 1.34078970148721e-08
4165 2.30339214368769e-08
4166 1.46091378994129e-08
4167 2.42521220883418e-08
4168 2.27091554449999e-08
4169 2.35224870692718e-08
4170 1.29305757212705e-08
4171 1.73819412196963e-08
4172 1.74609002812076e-08
4173 2.40117756789004e-08
4174 2.22741203259602e-08
4175 1.25373347259483e-08
4176 1.85094304328004e-08
4177 1.39759972483944e-08
4178 2.38837198907049e-08
4179 2.32828991642009e-08
4180 1.23555459197178e-08
4181 2.07815773478615e-08
4182 1.33391040435527e-08
4183 2.34202701676622e-08
4184 1.40058364905826e-08
4185 2.00994776378138e-08
4186 1.66067213314136e-08
4187 2.36843327172664e-08
4188 2.42651569948293e-08
4189 2.20322782240601e-08
4190 2.26496226218842e-08
4191 2.08086916586581e-08
4192 1.26644783549068e-08
4193 1.72968892542258e-08
4194 1.69725478116334e-08
4195 2.22655458514964e-08
4196 1.30771171669153e-08
4197 1.94775147122073e-08
4198 1.83015718135948e-08
4199 2.35495996037116e-08
4200 2.25994121194617e-08
4201 1.44959599879257e-08
4202 1.80217014644768e-08
4203 2.04018899552239e-08
4204 1.43183225276289e-08
4205 1.72013532306892e-08
4206 2.11980815123525e-08
4207 1.27485311196551e-08
4208 1.81517680886145e-08
4209 1.41613298865195e-08
4210 2.0947107159941e-08
4211 1.36065851918943e-08
4212 2.02452650199803e-08
4213 1.36548736762165e-08
4214 2.13751771838133e-08
4215 1.86048954020634e-08
4216 1.59094177831776e-08
4217 2.33550512263037e-08
4218 1.59758446471869e-08
4219 2.20867057976193e-08
4220 1.40287923500182e-08
4221 2.01649061892795e-08
4222 1.30139543585983e-08
4223 2.00941414618683e-08
4224 1.41223877037078e-08
4225 2.1511903369742e-08
4226 1.33142101788053e-08
4227 1.6239223299408e-08
4228 2.21532676647485e-08
4229 1.78436803111026e-08
4230 1.8517409827723e-08
4231 2.13555573225221e-08
4232 1.29549162508624e-08
4233 1.88996782668482e-08
4234 1.35825946045998e-08
4235 1.99807459466683e-08
4236 1.70859735248996e-08
4237 1.40155202998926e-08
4238 1.80883770184437e-08
4239 1.61450532942808e-08
4240 1.60920059499858e-08
4241 2.14133688558604e-08
4242 1.87534823226088e-08
4243 1.31546746828803e-08
4244 1.81973316415451e-08
4245 1.36731808098034e-08
4246 2.03347294558398e-08
4247 1.64388129775261e-08
4248 1.58667141647584e-08
4249 2.1217653412009e-08
4250 1.99991774252339e-08
4251 1.69829608154259e-08
4252 1.64038453931425e-08
4253 1.89345712442446e-08
4254 1.28620269990165e-08
4255 1.84621740118018e-08
4256 1.3530025988473e-08
4257 2.11498250024533e-08
4258 1.54445380928792e-08
4259 2.15189768226764e-08
4260 2.19142837210029e-08
4261 1.86300326276978e-08
4262 1.31660264912625e-08
4263 1.74151981724435e-08
4264 2.22451053133454e-08
4265 1.71581984176328e-08
4266 1.64118656442724e-08
4267 2.20680469453782e-08
4268 2.15630837629988e-08
4269 1.51486041488624e-08
4270 1.45546827923226e-08
4271 1.88565696390697e-08
4272 1.31014115112293e-08
4273 1.73550880333551e-08
4274 2.15280717696942e-08
4275 1.8805277335332e-08
4276 1.33549775682695e-08
4277 1.72292455857814e-08
4278 1.37657840681982e-08
4279 1.9320996358374e-08
4280 1.35221949193465e-08
4281 1.99773477760345e-08
4282 1.41695712940759e-08
4283 2.17770601551592e-08
4284 1.43903369220766e-08
4285 1.71688459005281e-08
4286 1.69905423064165e-08
4287 1.61572089041329e-08
4288 2.17946620750809e-08
4289 1.53613797237995e-08
4290 1.75690431092335e-08
4291 2.26108358702959e-08
4292 1.69055205390123e-08
4293 1.63117004348123e-08
4294 2.04645012047422e-08
4295 1.2900208901101e-08
4296 1.77047798644026e-08
4297 2.3034019136503e-08
4298 1.69687055517898e-08
4299 2.23195915083352e-08
4300 1.91989553144367e-08
4301 1.28025270385024e-08
4302 1.74291105992097e-08
4303 1.39503955054465e-08
4304 2.03673362619838e-08
4305 1.52408219378231e-08
4306 2.14316564495221e-08
4307 1.67356617453152e-08
4308 1.90337772210114e-08
4309 1.30909958429015e-08
4310 1.90660518484265e-08
4311 1.34264244167071e-08
4312 2.03058423409175e-08
4313 2.13220250344648e-08
4314 1.34495108383703e-08
4315 1.960636453191e-08
4316 1.67439999643193e-08
4317 1.62056679187117e-08
4318 2.21207958617242e-08
4319 1.96710665534283e-08
4320 1.29109398727678e-08
4321 2.0014130797108e-08
4322 1.44343079711007e-08
4323 2.13434958595826e-08
4324 1.46053000804613e-08
4325 1.86177402383692e-08
4326 1.49709471486403e-08
4327 2.23280185451813e-08
4328 2.2424128331977e-08
4329 2.20198295153295e-08
4330 2.21216236440114e-08
4331 1.46745966489448e-08
4332 1.54889505665778e-08
4333 1.7582221900625e-08
4334 2.10414281553994e-08
4335 1.28522614772919e-08
4336 1.86770847676598e-08
4337 2.27882512859878e-08
4338 1.93040285978441e-08
4339 1.32349509129881e-08
4340 1.83166424250203e-08
4341 1.6245472522769e-08
4342 2.20433342690285e-08
4343 1.63982640799532e-08
4344 1.7730203083488e-08
4345 2.31094272606924e-08
4346 1.52805341713247e-08
4347 1.5990201163163e-08
4348 1.83810318077349e-08
4349 2.2699254031977e-08
4350 1.95568716776506e-08
4351 1.31239801248739e-08
4352 1.73327432406722e-08
4353 1.55648205435455e-08
4354 1.96946867703218e-08
4355 1.29331274578703e-08
4356 1.86536990298691e-08
4357 1.63939937181112e-08
4358 1.55132013901493e-08
4359 2.23946763355798e-08
4360 1.61211310967246e-08
4361 1.98014049601625e-08
4362 1.30454251845435e-08
4363 1.86516562195038e-08
4364 2.25033769396532e-08
4365 2.31213093115912e-08
4366 1.59478101835475e-08
4367 1.52618540028016e-08
4368 1.87665438744489e-08
4369 1.58265667238311e-08
4370 1.8977349469651e-08
4371 1.29099380075104e-08
4372 1.76906898019524e-08
4373 1.45821115182798e-08
4374 2.2559103030062e-08
4375 1.37010385259373e-08
4376 1.69927076854037e-08
4377 2.17136655322747e-08
4378 2.2888825057521e-08
4379 1.31283623971967e-08
4380 1.81655650521861e-08
4381 2.30057572991882e-08
4382 1.62761821798085e-08
4383 1.62312083773486e-08
4384 1.5012805221204e-08
4385 2.12864463833284e-08
4386 1.81117911779438e-08
4387 1.83010211429746e-08
4388 1.47631045166463e-08
4389 1.80708124020157e-08
4390 1.31865798280728e-08
4391 1.92767188877951e-08
4392 1.29860566744355e-08
4393 1.94099385453228e-08
4394 1.37856481785548e-08
4395 2.18835261023287e-08
4396 1.38001867711068e-08
4397 1.66693308045751e-08
4398 1.83398114472766e-08
4399 1.40936391446189e-08
4400 2.16415969589434e-08
4401 1.48907766117645e-08
4402 2.07510186811533e-08
4403 1.34299176224317e-08
4404 1.94793443597518e-08
4405 1.40194780229308e-08
4406 2.12540172128683e-08
4407 1.50781911401054e-08
4408 2.16632116689652e-08
4409 2.02086187783834e-08
4410 1.58862665244897e-08
4411 2.20457998523216e-08
4412 2.13024051731736e-08
4413 1.39531746157218e-08
4414 1.58410742301385e-08
4415 2.02960634965166e-08
4416 1.2570023244507e-08
4417 1.75769834243056e-08
4418 1.37160878210807e-08
4419 2.07349764025366e-08
4420 2.21403340106008e-08
4421 1.51384398350274e-08
4422 1.49444439045965e-08
4423 2.16713829104265e-08
4424 1.3415807131878e-08
4425 1.64293396665016e-08
4426 1.8872128748626e-08
4427 1.51309329510241e-08
4428 2.17887130560257e-08
4429 2.1795036886374e-08
4430 1.49594843179557e-08
4431 1.45044456445476e-08
4432 1.6859655005419e-08
4433 1.75862790996462e-08
4434 1.46477061591099e-08
4435 1.5981285628186e-08
4436 2.20028208985923e-08
4437 1.56846624577156e-08
4438 1.62285420657327e-08
4439 2.20162323927298e-08
4440 1.49383936332015e-08
4441 1.61963544798027e-08
4442 2.21476774697749e-08
4443 1.37736675398514e-08
4444 1.39890374839524e-08
4445 1.57604098660613e-08
4446 2.17830180559986e-08
4447 1.53339634323402e-08
4448 1.61348285843133e-08
4449 2.0276093692928e-08
4450 1.21571757105698e-08
4451 1.62491353705718e-08
4452 1.35102169451784e-08
4453 2.09147525964681e-08
4454 1.38035245456081e-08
4455 2.09495567560225e-08
4456 1.54130077589798e-08
4457 1.43922562756416e-08
4458 2.02511873936828e-08
4459 1.73632237476795e-08
4460 1.26886687823458e-08
4461 1.42616114473526e-08
4462 1.7098777504998e-08
4463 1.3500399020927e-08
4464 1.48055399051827e-08
4465 2.05866044211689e-08
4466 1.47943275408124e-08
4467 2.11777919645328e-08
4468 1.61666999787258e-08
4469 1.73221277322e-08
4470 1.41407081599709e-08
4471 1.58785571358067e-08
4472 1.91252791381658e-08
4473 1.21216281456782e-08
4474 1.59242681263549e-08
4475 1.35948532431485e-08
4476 2.09569979148228e-08
4477 1.34259963147088e-08
4478 1.64212359266003e-08
4479 1.38517544101546e-08
4480 2.07954879982708e-08
4481 1.48440273406436e-08
4482 2.084854422435e-08
4483 1.99219645224957e-08
4484 1.27299291108329e-08
4485 1.64330469232254e-08
4486 1.31395010427582e-08
4487 1.81245081165571e-08
4488 1.2798245130341e-08
4489 1.8749320318534e-08
4490 1.43415412878767e-08
4491 2.07291606102444e-08
4492 1.41561180555527e-08
4493 2.07268300300711e-08
4494 1.43011078534983e-08
4495 1.64126419122113e-08
4496 1.29612542920654e-08
4497 1.87212521041147e-08
4498 1.36335360778617e-08
4499 2.05966959043735e-08
4500 1.32487469883813e-08
4501 1.67580491705621e-08
4502 1.47868846056554e-08
4503 2.12057589266124e-08
4504 1.56703183762374e-08
4505 1.97685583458451e-08
4506 1.26329213756549e-08
4507 1.63198983216262e-08
4508 1.97822771497158e-08
4509 1.30565771527813e-08
4510 1.76362693338206e-08
4511 1.2717983111088e-08
4512 2.00686525175797e-08
4513 1.40595854958292e-08
4514 2.06227070975729e-08
4515 1.42537306402346e-08
4516 2.0996294480824e-08
4517 1.46557539437708e-08
4518 2.131309706499e-08
4519 1.50924019948206e-08
4520 1.6369419597595e-08
4521 1.52886876492175e-08
4522 1.97044389693701e-08
4523 1.81045027858318e-08
4524 1.23677832419844e-08
4525 1.5226852667638e-08
4526 1.3598043580032e-08
4527 2.14711040058546e-08
4528 1.232295243625e-08
4529 1.52070604997334e-08
4530 1.49271155436281e-08
4531 2.15640412193352e-08
4532 1.44069645102718e-08
4533 2.13664073100972e-08
4534 1.41689975308168e-08
4535 1.54822661357912e-08
4536 1.98942888829379e-08
4537 1.26326717975189e-08
4538 1.55186228312232e-08
4539 1.31089796795436e-08
4540 1.62073465759249e-08
4541 1.40875711096555e-08
4542 1.97302387761056e-08
4543 1.56145247842687e-08
4544 1.20888694610244e-08
4545 1.56067070378185e-08
4546 1.26986421378206e-08
4547 1.68638116804232e-08
4548 1.33933495405358e-08
4549 1.91443820796167e-08
4550 1.39845992563892e-08
4551 1.5899580319001e-08
4552 1.26903332287043e-08
4553 2.00705443376137e-08
4554 1.43301788213535e-08
4555 1.72570278067496e-08
4556 1.29271375826079e-08
4557 1.9279010388118e-08
4558 1.97953387015559e-08
4559 1.2350604983169e-08
4560 1.94215008519905e-08
4561 1.39197107174027e-08
4562 1.39640308205458e-08
4563 1.42035299077747e-08
4564 1.90060074345411e-08
4565 1.28406387744917e-08
4566 1.80263963756033e-08
4567 1.3113560015654e-08
4568 1.93522797786727e-08
4569 1.67672649098449e-08
4570 1.20891279209445e-08
4571 1.75083290088196e-08
4572 1.24507453236333e-08
4573 1.894758483445e-08
4574 1.90806748179284e-08
4575 1.3091930206599e-08
4576 1.91289899476033e-08
4577 1.38176279307345e-08
4578 1.91404154747943e-08
4579 1.3020755140758e-08
4580 1.87260287276558e-08
4581 1.5098514438705e-08
4582 1.35690010338863e-08
4583 1.81060588744231e-08
4584 1.20657253077638e-08
4585 1.88589162064545e-08
4586 1.38331817112203e-08
4587 1.37797426802422e-08
4588 1.89369977476872e-08
4589 1.49571217633593e-08
4590 1.94465101799324e-08
4591 1.41895233340961e-08
4592 1.54431401000465e-08
4593 1.89179552023688e-08
4594 1.32282469422762e-08
4595 1.5295960054118e-08
4596 1.39609674931762e-08
4597 1.36037492382002e-08
4598 1.93311642249228e-08
4599 1.47178038645279e-08
4600 1.41745939430393e-08
4601 1.93116918012493e-08
4602 1.70406906363496e-08
4603 1.37333193706013e-08
4604 1.63005395847904e-08
4605 1.35229036857254e-08
4606 1.42306744166376e-08
4607 1.96351592762767e-08
4608 1.22796413037918e-08
4609 1.68142495482471e-08
4610 1.24433965353887e-08
4611 1.95961131765898e-08
4612 1.29527935044393e-08
4613 1.47456669097323e-08
4614 1.6228394628115e-08
4615 1.38812596972571e-08
4616 1.39148301769865e-08
4617 1.90521127763077e-08
4618 1.15121165933374e-08
4619 1.60663340409428e-08
4620 1.32208839431769e-08
4621 1.95164986394047e-08
4622 1.43940175334478e-08
4623 1.38717757280915e-08
4624 1.9003929097039e-08
4625 1.25431416364563e-08
4626 1.71215415178949e-08
4627 1.25609131984561e-08
4628 1.91433677798614e-08
4629 1.28053869730138e-08
4630 1.93952516269746e-08
4631 1.78618169144329e-08
4632 1.28071482308201e-08
4633 1.92710416513364e-08
4634 1.91826092788006e-08
4635 1.22138859026677e-08
4636 1.85848758604834e-08
4637 1.41007925336112e-08
4638 1.39711788804675e-08
4639 1.36981475051812e-08
4640 1.36300313258175e-08
4641 1.43341232217153e-08
4642 1.92563582857019e-08
4643 1.24555770142365e-08
4644 1.76433996301739e-08
4645 1.24006236390528e-08
4646 1.9081983992919e-08
4647 1.4282135474275e-08
4648 1.89854425514113e-08
4649 1.23484449332523e-08
4650 1.58803512562145e-08
4651 1.23470611512744e-08
4652 1.90657356569091e-08
4653 1.85721571455133e-08
4654 1.87173121446449e-08
4655 1.865175391913e-08
4656 1.40813289917219e-08
4657 1.42220750731781e-08
4658 1.8891462616466e-08
4659 1.86328090734378e-08
4660 1.44280969394117e-08
4661 1.4031316553087e-08
4662 1.53584753803671e-08
4663 1.40519613722745e-08
4664 1.63435842637227e-08
4665 1.339582578197e-08
4666 1.8505353693854e-08
4667 1.19002701026716e-08
4668 1.37437989877753e-08
4669 1.49032963747686e-08
4670 1.93849523100198e-08
4671 1.16617222545301e-08
4672 1.4838299478015e-08
4673 1.86150117542638e-08
4674 1.90959088541831e-08
4675 1.61530859799086e-08
4676 1.26830181912396e-08
4677 1.82153687688924e-08
4678 1.28352697359446e-08
4679 1.8459529016468e-08
4680 1.31585347062924e-08
4681 1.79528534260953e-08
4682 1.35126523304052e-08
4683 1.92963405254432e-08
4684 1.23974102095303e-08
4685 1.66840283810643e-08
4686 1.21473426872853e-08
4687 1.77999321948619e-08
4688 1.78631331948509e-08
4689 1.28997292847544e-08
4690 1.86255917355993e-08
4691 1.87314590505139e-08
4692 1.25738424117117e-08
4693 1.81985644331917e-08
4694 1.35848408078232e-08
4695 1.87614368485356e-08
4696 1.85957826914773e-08
4697 1.39669396048703e-08
4698 1.62901159228568e-08
4699 1.27480008771386e-08
4700 1.57467106021159e-08
4701 1.29330022247132e-08
4702 1.84129067548611e-08
4703 1.22079475417536e-08
4704 1.67299951669975e-08
4705 1.87763102843519e-08
4706 1.80066805910428e-08
4707 1.36239153292195e-08
4708 1.52230761329974e-08
4709 1.87984170452182e-08
4710 1.34901467774284e-08
4711 1.42984806217328e-08
4712 1.45500420600797e-08
4713 1.83501498440819e-08
4714 1.21083241211295e-08
4715 1.28797283949211e-08
4716 1.53397152757861e-08
4717 1.36352058532907e-08
4718 1.9169187126522e-08
4719 1.32276571918055e-08
4720 1.80642327762826e-08
4721 1.30520643182308e-08
4722 1.48814729428182e-08
4723 1.21771890349009e-08
4724 1.69185891962798e-08
4725 1.18464509313299e-08
4726 1.73495120492362e-08
4727 1.39732883042143e-08
4728 1.4271213544248e-08
4729 1.27194956789367e-08
4730 1.73437797457154e-08
4731 1.24788863686831e-08
4732 1.45712126808917e-08
4733 1.80628756396572e-08
4734 1.13340092866565e-08
4735 1.56744519586027e-08
4736 1.76104837379398e-08
4737 1.39154208156356e-08
4738 1.22191501361613e-08
4739 1.75246359646053e-08
4740 1.32133548547131e-08
4741 1.47483731893772e-08
4742 1.53361430221821e-08
4743 1.2111376790358e-08
4744 1.63828630661556e-08
4745 1.16459624166509e-08
4746 1.75010921310559e-08
4747 1.14875122747549e-08
4748 1.29464137188506e-08
4749 1.36391484772957e-08
4750 1.27959225437735e-08
4751 1.69907323765983e-08
4752 1.13060609763238e-08
4753 2.02769800949909e-08
4754 1.39692124534463e-08
4755 1.57020494384597e-08
4756 1.36718929510948e-08
4757 1.88999553785152e-08
4758 1.20278560444831e-08
4759 1.86073698671407e-08
4760 1.33999851215094e-08
4761 1.85189108492523e-08
4762 1.25158123864821e-08
4763 1.51673340553771e-08
4764 1.29810322491153e-08
4765 1.90467606131506e-08
4766 1.27584129927527e-08
4767 1.40887186361738e-08
4768 1.76386301120601e-08
4769 1.32554633935911e-08
4770 1.15599014804957e-08
4771 1.7027877774467e-08
4772 1.20781464829633e-08
4773 1.71280944982755e-08
4774 1.30743700310632e-08
4775 1.81498744922237e-08
4776 1.27419248485694e-08
4777 1.22855512429965e-08
4778 1.59321729142903e-08
4779 1.18542118343612e-08
4780 1.54728656553971e-08
4781 1.24478045648857e-08
4782 1.78862613608999e-08
4783 1.24089263309202e-08
4784 1.22501768728966e-08
4785 1.48599363924973e-08
4786 1.61712616630894e-08
4787 1.03621395908249e-08
4788 1.50412855504101e-08
4789 1.16320757470589e-08
4790 1.68019784752005e-08
4791 1.21635874705817e-08
4792 1.65624651771168e-08
4793 1.33709958660688e-08
4794 1.74252825502208e-08
4795 1.21058771895832e-08
4796 1.56817261398601e-08
4797 1.23198731216689e-08
4798 1.67703237963224e-08
4799 1.17712755098864e-08
4800 1.67207687695736e-08
4801 1.43499390148349e-08
4802 1.74765446558922e-08
4803 1.21669243569045e-08
4804 1.60272310978371e-08
4805 1.75633640964179e-08
4806 1.17274430166958e-08
4807 1.7649659511676e-08
4808 1.27971624408474e-08
4809 1.82058776942995e-08
4810 1.29283064254082e-08
4811 1.19014291755093e-08
4812 1.63232201089158e-08
4813 1.1268612709614e-08
4814 1.7824861586746e-08
4815 1.45207357249433e-08
4816 1.71125602577149e-08
4817 1.21539569519769e-08
4818 1.27113208847618e-08
4819 1.42780622880423e-08
4820 1.13055049766331e-08
4821 1.24714718552354e-08
4822 1.15944311929184e-08
4823 1.66885349983659e-08
4824 1.2638654567354e-08
4825 1.27071722033634e-08
4826 1.24891119668291e-08
4827 1.22149463877008e-08
4828 1.69033746999503e-08
4829 1.37624542873027e-08
4830 1.23880701252688e-08
4831 1.72309615464883e-08
4832 1.43910448002771e-08
4833 1.24059580386415e-08
4834 1.7995507306523e-08
4835 1.14671729889437e-08
4836 1.30420279020882e-08
4837 1.24862511441393e-08
4838 1.30852804147708e-08
4839 1.19599175008034e-08
4840 1.79443269132662e-08
4841 1.20542189563366e-08
4842 1.43933149843178e-08
4843 1.34568773901833e-08
4844 1.72969549794288e-08
4845 1.19662368902596e-08
4846 1.75693948278877e-08
4847 1.28024399970172e-08
4848 1.7433233523434e-08
4849 1.16636780234103e-08
4850 1.85944291075657e-08
4851 1.20530563307852e-08
4852 1.20383747415076e-08
4853 1.53187773577201e-08
4854 1.13380771438187e-08
4855 1.9223252323286e-08
4856 1.26002248634904e-08
4857 1.22922898526667e-08
4858 1.34173232524404e-08
4859 1.36079840729053e-08
4860 1.72831668976414e-08
4861 1.28526709275434e-08
4862 1.38259874660207e-08
4863 1.82646680002563e-08
4864 1.02605923757437e-08
4865 1.4165688178025e-08
4866 1.44469858298635e-08
4867 1.23969297050053e-08
4868 1.81631030216067e-08
4869 1.31081732135385e-08
4870 1.45388705519167e-08
4871 1.90892119888986e-08
4872 9.18409703842826e-09
4873 1.78459469424297e-08
4874 1.52831667321607e-08
4875 1.33126603074629e-08
4876 1.88365376629918e-08
4877 1.02524513323488e-08
4878 1.79569727976059e-08
4879 1.19627321382154e-08
4880 2.18553353192874e-08
4881 1.17611813621465e-08
4882 1.78339920609005e-08
4883 9.60107815473066e-09
4884 1.80978982911029e-08
4885 1.29146879856989e-08
4886 1.63285402976499e-08
4887 1.75315015837896e-08
4888 1.36181386167777e-08
4889 1.86455331174784e-08
4890 1.31908555189852e-08
4891 1.87567295029112e-08
4892 1.3011585586753e-08
4893 1.62153721561253e-08
4894 1.39469928939207e-08
4895 1.9020331976094e-08
4896 1.27085462153786e-08
4897 1.77911072540837e-08
4898 1.13264757573006e-08
4899 2.12443254099526e-08
4900 1.27451196263451e-08
4901 1.93557987415716e-08
4902 1.0764645175243e-08
4903 1.60278386118762e-08
4904 9.49990308640736e-09
4905 1.69852540921056e-08
4906 1.68595963856433e-08
4907 1.28380390762572e-08
4908 2.03872385640125e-08
4909 1.27510304537282e-08
4910 1.46746144125132e-08
4911 1.1541332334275e-08
4912 2.13400834780941e-08
4913 1.41222766814053e-08
4914 2.07737631541249e-08
4915 1.12822018394354e-08
4916 1.33538451407844e-08
4917 1.51647103763253e-08
4918 1.32701609700803e-08
4919 2.08932817713503e-08
4920 1.15682059487199e-08
4921 1.60031685680906e-08
4922 1.48518024545297e-08
4923 1.47047041210158e-08
4924 1.78606907041967e-08
4925 1.07329256593403e-08
4926 1.63834670274809e-08
4927 1.09048539087553e-08
4928 1.85575323996545e-08
4929 1.23226318038405e-08
4930 1.7408789076967e-08
4931 1.06596536042503e-08
4932 1.84551733894978e-08
4933 1.09356710353836e-08
4934 1.85747204284326e-08
4935 1.0242399817173e-08
4936 1.72895688876906e-08
4937 1.55579940042117e-08
4938 1.20266276937286e-08
4939 1.844333574752e-08
4940 1.35231701392513e-08
4941 1.83138553211393e-08
4942 9.45346201319808e-09
4943 1.55503201426654e-08
4944 1.08867483916697e-08
4945 1.79142496392615e-08
4946 1.1081879414121e-08
4947 1.47246490556086e-08
4948 1.3445089486197e-08
4949 1.55803387968945e-08
4950 1.43849465672474e-08
4951 1.58468509425802e-08
4952 1.52010830589688e-08
4953 1.59599675697564e-08
4954 1.46420804369995e-08
4955 1.49238843505373e-08
4956 1.38553968298538e-08
4957 1.67022449204524e-08
4958 9.33801747038387e-09
4959 1.75890182418925e-08
4960 1.22706653726823e-08
4961 1.5151451648876e-08
4962 1.20012240145684e-08
4963 1.68057940896915e-08
4964 1.96219591686031e-08
4965 1.20799450442632e-08
4966 1.28826433964946e-08
4967 1.25612711343592e-08
4968 1.67081051216655e-08
4969 1.50861065861818e-08
4970 1.82007724447431e-08
4971 1.07740367738529e-08
4972 1.839873853271e-08
4973 1.27373978031642e-08
4974 1.79481105533341e-08
4975 1.26637527131379e-08
4976 1.69856537723945e-08
4977 1.13936282630789e-08
4978 1.67488778402003e-08
4979 1.20448770957182e-08
4980 1.68136811140585e-08
4981 1.42476945796943e-08
4982 1.76051742073469e-08
4983 1.06170077174284e-08
4984 1.87457143141501e-08
4985 1.71885599087318e-08
4986 1.51002232939845e-08
4987 1.6943317859841e-08
4988 1.15455263127728e-08
4989 1.7839649757434e-08
4990 1.33234978605401e-08
4991 1.37698004110121e-08
4992 1.70763172491206e-08
4993 1.04231281383704e-08
4994 1.76328054379837e-08
4995 1.33012454384129e-08
4996 1.70304357283158e-08
4997 1.07887521139105e-08
4998 1.71349121558251e-08
4999 1.12995621748269e-08
};
\addlegendentry{Test}
\end{groupplot}

\end{tikzpicture}

		\caption{Five experiments over different batch sizes with $\hy$ left and $\rare$ right. The batch size used for every experiment is given. Training and validation loss are shown over 5000 epochs.}
		\label{Fig:batch}
	\end{figure}
\end{center}
\begin{center}
	\begin{figure}[H]
		% This file was created by tikzplotlib v0.9.6.
\begin{tikzpicture}

\begin{groupplot}[group style={group size=1 by 6},
legend cell align={left},
legend style={fill opacity=1, draw opacity=1, text opacity=1, draw=white},
log basis y={10},
tick align=outside,
tick pos=left,
title style={at={(0.3,0.85)},anchor=north},
x grid style={white!69.0196078431373!black},
xlabel={Epoch},
x label style={yshift=13pt},
xmin=-99.95, xmax=5098.95,
xtick style={color=black},
xtick = {0,1000,4000,5000},
y grid style={white!69.0196078431373!black},
ylabel={MSE Loss},
ymode=log,
ytick style={color=black},
width=.45\textwidth,
height=.25\textwidth
]
\nextgroupplot[
title={ReLU/ReLU $\hy$},
ymin=4.22592827103361e-09, ymax=1e-05,
]
\addplot [semithick, black, dashed]
table {%
0 0.0100203707803739
1 0.000804774132739112
2 0.000233406998524515
3 0.000204813241622105
4 0.000181739909599855
5 0.000117703136994351
6 4.48823786857702e-05
7 2.31475917159401e-05
8 1.81876512783674e-05
9 1.7058010415127e-05
10 1.65393507215015e-05
11 1.60309398646348e-05
12 1.53881556998101e-05
13 1.45253996167583e-05
14 1.30188670735834e-05
15 1.1251937952153e-05
16 9.13349212150649e-06
17 6.95229251117269e-06
18 5.151605016259e-06
19 3.95319622744239e-06
20 3.20331916304539e-06
21 2.65279863319279e-06
22 2.18477159048902e-06
23 1.7720323091126e-06
24 1.42441317764863e-06
25 1.14797583037962e-06
26 9.41029840273444e-07
27 7.92864413046601e-07
28 6.88918911331271e-07
29 6.16465836179358e-07
30 5.65867125896347e-07
31 5.29357136288766e-07
32 5.02635520131634e-07
33 4.82327803153382e-07
34 4.66821059793787e-07
35 4.54095006602984e-07
36 4.43499885920851e-07
37 4.34328127199457e-07
38 4.26845726895664e-07
39 4.19764905727149e-07
40 4.13209149503935e-07
41 4.07375375361063e-07
42 4.01970613924441e-07
43 3.9704398941165e-07
44 3.92540362127214e-07
45 3.88513800857027e-07
46 3.84516496270137e-07
47 3.80903392269261e-07
48 3.77438481311998e-07
49 3.74151750148322e-07
50 3.71230860798377e-07
51 3.68441203701053e-07
52 3.65864376144387e-07
53 3.63523388562825e-07
54 3.61349108793263e-07
55 3.5924712799229e-07
56 3.57136839924443e-07
57 3.55236432638684e-07
58 3.53363448374999e-07
59 3.51589117852491e-07
60 3.49925739540957e-07
61 3.4831281436265e-07
62 3.46763777169912e-07
63 3.45204373633834e-07
64 3.43693474066598e-07
65 3.42170730938562e-07
66 3.40837916682268e-07
67 3.3947068214335e-07
68 3.38151740595372e-07
69 3.36998112660858e-07
70 3.35783096434561e-07
71 3.3470701107241e-07
72 3.33505297980707e-07
73 3.32485279116668e-07
74 3.31331152126246e-07
75 3.30302987549658e-07
76 3.29191363024783e-07
77 3.28182662925336e-07
78 3.27039430914056e-07
79 3.25872501461433e-07
80 3.24861340853744e-07
81 3.24064443249839e-07
82 3.22855214861306e-07
83 3.20642355856826e-07
84 3.19593377098748e-07
85 3.18234692406882e-07
86 3.16889471952742e-07
87 3.1550341245179e-07
88 3.14350359069238e-07
89 3.12762306745817e-07
90 3.11252230887682e-07
91 3.10209906017711e-07
92 3.08641283616851e-07
93 3.07392297409947e-07
94 3.06017163968164e-07
95 3.04657471795977e-07
96 3.03093222579598e-07
97 3.01457335745603e-07
98 2.99929277508326e-07
99 2.98199477083649e-07
100 2.9627303407942e-07
101 2.94244780057795e-07
102 2.92255114598738e-07
103 2.90224259662075e-07
104 2.88096848548136e-07
105 2.86033462291613e-07
106 2.83613843461872e-07
107 2.81237622610675e-07
108 2.78856686140472e-07
109 2.7587495379322e-07
110 2.73183387223597e-07
111 2.7037954438569e-07
112 2.677690974906e-07
113 2.64939648157991e-07
114 2.62052586444739e-07
115 2.58879510450782e-07
116 2.55568184772237e-07
117 2.52068093656632e-07
118 2.48567447123094e-07
119 2.45122989243285e-07
120 2.41518031155685e-07
121 2.37957590247362e-07
122 2.34421168943122e-07
123 2.30942578638427e-07
124 2.27502988138895e-07
125 2.24065043268773e-07
126 2.20677043478723e-07
127 2.17109392259829e-07
128 2.13788660438752e-07
129 2.10384372508976e-07
130 2.07049280135152e-07
131 2.03823567712647e-07
132 2.00667304736335e-07
133 1.97597840030816e-07
134 1.94612284341034e-07
135 1.91733551426765e-07
136 1.88917484738482e-07
137 1.86186264425459e-07
138 1.83514262762863e-07
139 1.80964058955624e-07
140 1.78536305451615e-07
141 1.76180753406641e-07
142 1.73895592335604e-07
143 1.71650572339921e-07
144 1.69523144057493e-07
145 1.673940089848e-07
146 1.65417000924428e-07
147 1.63564048440534e-07
148 1.61739631761471e-07
149 1.59841856784304e-07
150 1.58061459956116e-07
151 1.56300670826504e-07
152 1.54697464172848e-07
153 1.53129566820454e-07
154 1.51537485684194e-07
155 1.49979306188541e-07
156 1.48534035365522e-07
157 1.47023418812786e-07
158 1.45689690313588e-07
159 1.44322674942465e-07
160 1.43179886043399e-07
161 1.41879238974685e-07
162 1.40543663575965e-07
163 1.39331391549202e-07
164 1.38137065305077e-07
165 1.3692576391211e-07
166 1.35799491963962e-07
167 1.34633155408181e-07
168 1.33541575022544e-07
169 1.32480273121516e-07
170 1.31562459851686e-07
171 1.30470744918565e-07
172 1.29527837900767e-07
173 1.28638715337637e-07
174 1.27576185750655e-07
175 1.26736099037128e-07
176 1.25860877494555e-07
177 1.24944313939501e-07
178 1.24103731815817e-07
179 1.23280744939835e-07
180 1.22470221194382e-07
181 1.21633250697473e-07
182 1.20846504036098e-07
183 1.20027154441926e-07
184 1.19393105565191e-07
185 1.18552080553558e-07
186 1.17859651167729e-07
187 1.17158918280902e-07
188 1.16325007727403e-07
189 1.15625359391247e-07
190 1.14973369602422e-07
191 1.14272153888484e-07
192 1.13632471612002e-07
193 1.12920880283252e-07
194 1.12257663160031e-07
195 1.11612448977949e-07
196 1.10937563037261e-07
197 1.10331546380626e-07
198 1.09564226564496e-07
199 1.08881928020388e-07
200 1.08360683129582e-07
201 1.0774217730658e-07
202 1.07077761019259e-07
203 1.0643046788994e-07
204 1.05715581894206e-07
205 1.05183493801952e-07
206 1.04568850391118e-07
207 1.03958973987961e-07
208 1.03346175460928e-07
209 1.02674204032205e-07
210 1.02178800804875e-07
211 1.01517160032039e-07
212 1.00875297830871e-07
213 1.00264894602464e-07
214 9.96819857075337e-08
215 9.91675079835552e-08
216 9.85858090700731e-08
217 9.79929625795073e-08
218 9.73527426881837e-08
219 9.68819954620947e-08
220 9.62907740116314e-08
221 9.57082982599999e-08
222 9.51541999105743e-08
223 9.46151489094049e-08
224 9.40209110460444e-08
225 9.34342028857671e-08
226 9.28074675852209e-08
227 9.21798962583154e-08
228 9.17753659632758e-08
229 9.11187664009994e-08
230 9.0679217744416e-08
231 9.00985529015408e-08
232 8.94818408920806e-08
233 8.90078307382858e-08
234 8.84822255815543e-08
235 8.79338464758739e-08
236 8.74931345773433e-08
237 8.69877430704769e-08
238 8.65063521358067e-08
239 8.59542838003158e-08
240 8.54510071022219e-08
241 8.49490375443018e-08
242 8.44524240903866e-08
243 8.40120812592993e-08
244 8.35133415373335e-08
245 8.30598230114887e-08
246 8.24980826039656e-08
247 8.20630339037898e-08
248 8.16030424837244e-08
249 8.11039664516677e-08
250 8.06345585591117e-08
251 8.01677322930239e-08
252 7.96951736270479e-08
253 7.92244553999844e-08
254 7.88165762530824e-08
255 7.83379853768196e-08
256 7.79159724157985e-08
257 7.74836092487519e-08
258 7.70837218420084e-08
259 7.66532735441139e-08
260 7.62536160836547e-08
261 7.58168885179877e-08
262 7.5396560636154e-08
263 7.4994555712582e-08
264 7.45609351358034e-08
265 7.41637578562937e-08
266 7.37489488429333e-08
267 7.32900264459602e-08
268 7.29220937687458e-08
269 7.2496005550704e-08
270 7.21470249036571e-08
271 7.17776126188951e-08
272 7.13958881384258e-08
273 7.10393017286926e-08
274 7.06776846426394e-08
275 7.03406997191181e-08
276 6.99341964636169e-08
277 6.95363519649383e-08
278 6.92120696381693e-08
279 6.9000595398272e-08
280 6.87090604016838e-08
281 6.82834952594469e-08
282 6.81072717081399e-08
283 6.7795930908332e-08
284 6.74916253613489e-08
285 6.72163864638975e-08
286 6.69210227712824e-08
287 6.66307215215767e-08
288 6.63649115812959e-08
289 6.611467216322e-08
290 6.58401337156889e-08
291 6.55662246376032e-08
292 6.52956321647302e-08
293 6.50484721034239e-08
294 6.48015536737212e-08
295 6.44996430816391e-08
296 6.42538660331482e-08
297 6.39764336893833e-08
298 6.37201037001489e-08
299 6.34658699771862e-08
300 6.31809161126817e-08
301 6.29549665958073e-08
302 6.27074659820259e-08
303 6.24604442203136e-08
304 6.22764374220708e-08
305 6.19666107621875e-08
306 6.17161258817944e-08
307 6.15112326687317e-08
308 6.13742949733265e-08
309 6.10816945436632e-08
310 6.09146572627672e-08
311 6.07555255691672e-08
312 6.05104247295429e-08
313 6.02380848437889e-08
314 6.00925696390142e-08
315 6.00093042657512e-08
316 5.97745812229711e-08
317 5.96244386512623e-08
318 5.93303485980634e-08
319 5.91796597824157e-08
320 5.90019021866617e-08
321 5.88491705557637e-08
322 5.86765190666583e-08
323 5.84945752044597e-08
324 5.83620726812839e-08
325 5.82157938868733e-08
326 5.80678052464201e-08
327 5.78505099841919e-08
328 5.77309357778777e-08
329 5.75793936343771e-08
330 5.74306643550404e-08
331 5.72785670767129e-08
332 5.69808196073662e-08
333 5.69453430283584e-08
334 5.69711008655549e-08
335 5.67902680026489e-08
336 5.65470736726414e-08
337 5.64731215060288e-08
338 5.63437763811958e-08
339 5.62086356199476e-08
340 5.61232105600329e-08
341 5.59604299450456e-08
342 5.58428906878294e-08
343 5.57257563489344e-08
344 5.5595163812372e-08
345 5.55434330198334e-08
346 5.53132564669045e-08
347 5.52463121861191e-08
348 5.49532897808902e-08
349 5.50329203605759e-08
350 5.47262465444565e-08
351 5.47003615831709e-08
352 5.45557213122994e-08
353 5.46238869783e-08
354 5.4298301642941e-08
355 5.42628654742749e-08
356 5.41471632475865e-08
357 5.40218782332236e-08
358 5.4029500588415e-08
359 5.39917504382981e-08
360 5.37055214877569e-08
361 5.3583082218811e-08
362 5.35201921654238e-08
363 5.3415861337669e-08
364 5.33877581123665e-08
365 5.3274212775456e-08
366 5.3267067636531e-08
367 5.30640895142209e-08
368 5.30328026040472e-08
369 5.28952789027315e-08
370 5.29296157998349e-08
371 5.26644713381152e-08
372 5.27540678660898e-08
373 5.25488090756809e-08
374 5.2508663782147e-08
375 5.23537177703481e-08
376 5.21758182485677e-08
377 5.22565755500803e-08
378 5.215878251974e-08
379 5.20785746247476e-08
380 5.20079465891055e-08
381 5.1935669131753e-08
382 5.18541390175375e-08
383 5.17559614643392e-08
384 5.16866616675493e-08
385 5.16036972078027e-08
386 5.15351848624501e-08
387 5.15169361918755e-08
388 5.13266719461836e-08
389 5.1329285722046e-08
390 5.13243799866725e-08
391 5.11658285029526e-08
392 5.1029860481e-08
393 5.1039029342892e-08
394 5.10398908826204e-08
395 5.08714728482129e-08
396 5.07454394567119e-08
397 5.07735672092835e-08
398 5.0692138168662e-08
399 5.06133423423538e-08
400 5.06227111620738e-08
401 5.04986844291899e-08
402 5.04091065658407e-08
403 5.03750101144007e-08
404 5.03234832960331e-08
405 5.02467766010373e-08
406 5.01807029489321e-08
407 5.00615588148179e-08
408 5.0095141290063e-08
409 4.99809733653933e-08
410 4.99206214503722e-08
411 4.98937883719464e-08
412 4.97775932624123e-08
413 4.97755265922439e-08
414 4.96537246941209e-08
415 4.96510305343367e-08
416 4.95246470015864e-08
417 4.95275469727385e-08
418 4.9420320860083e-08
419 4.93994054686642e-08
420 4.93111752877251e-08
421 4.92599480488298e-08
422 4.91535935451815e-08
423 4.91476814077352e-08
424 4.90556028418521e-08
425 4.90358794447143e-08
426 4.8940908682793e-08
427 4.88241244172016e-08
428 4.89307670117345e-08
429 4.88215268314995e-08
430 4.87392381069984e-08
431 4.86856775794298e-08
432 4.86331837470377e-08
433 4.85498957663033e-08
434 4.8462127376192e-08
435 4.84590545757335e-08
436 4.84308208215101e-08
437 4.83527417787233e-08
438 4.81929405795611e-08
439 4.82223728792697e-08
440 4.82034279969401e-08
441 4.81781976189311e-08
442 4.80497977093375e-08
443 4.80726540104115e-08
444 4.79429839355117e-08
445 4.80049486577538e-08
446 4.79585454551401e-08
447 4.78944948749671e-08
448 4.78403879291545e-08
449 4.76728241951552e-08
450 4.7595303376502e-08
451 4.762364735722e-08
452 4.76063624499368e-08
453 4.75805470785406e-08
454 4.74250710862378e-08
455 4.74315670171421e-08
456 4.74453098682215e-08
457 4.72970924496607e-08
458 4.72280010443971e-08
459 4.7181106767713e-08
460 4.72790606145246e-08
461 4.71167990276911e-08
462 4.71004654862828e-08
463 4.72010396865841e-08
464 4.7147840800621e-08
465 4.69139487120351e-08
466 4.68883932636111e-08
467 4.69989050595476e-08
468 4.67882111518936e-08
469 4.6748664756846e-08
470 4.68561838720127e-08
471 4.66451858558425e-08
472 4.65988641429327e-08
473 4.67079986856689e-08
474 4.64543738585288e-08
475 4.64568390563524e-08
476 4.65372431972533e-08
477 4.63121366856001e-08
478 4.63191487480863e-08
479 4.6248935804627e-08
480 4.62441253510359e-08
481 4.63119600950801e-08
482 4.62656777258097e-08
483 4.62154438252149e-08
484 4.61695028703968e-08
485 4.61253618715407e-08
486 4.59739967830775e-08
487 4.59238526475669e-08
488 4.60252803415884e-08
489 4.58150144526659e-08
490 4.58130684899416e-08
491 4.57113088867533e-08
492 4.57587375419788e-08
493 4.56185640860252e-08
494 4.56835300426395e-08
495 4.55722717447316e-08
496 4.5568189454448e-08
497 4.56010420950559e-08
498 4.54683542188139e-08
499 4.54510399918728e-08
500 4.54643797722643e-08
501 4.53354552309992e-08
502 4.53422031259088e-08
503 4.52342659200688e-08
504 4.530372100775e-08
505 4.51572628465957e-08
506 4.52575523204324e-08
507 4.50945366101685e-08
508 4.50928355577584e-08
509 4.50825557283263e-08
510 4.49905017736452e-08
511 4.4964388278057e-08
512 4.49933488844145e-08
513 4.48794269469754e-08
514 4.49143720342438e-08
515 4.47947804589344e-08
516 4.48417835401393e-08
517 4.47303306099212e-08
518 4.47691903688874e-08
519 4.4651578058108e-08
520 4.468790535328e-08
521 4.46020412225856e-08
522 4.45780181590472e-08
523 4.45724093016686e-08
524 4.44991224801683e-08
525 4.44569556503627e-08
526 4.44933929786284e-08
527 4.43971456558501e-08
528 4.43536355605456e-08
529 4.43633021467438e-08
530 4.42999774452613e-08
531 4.42582151791715e-08
532 4.42541175922351e-08
533 4.42221077956439e-08
534 4.42140633396537e-08
535 4.41155876995669e-08
536 4.40866183466504e-08
537 4.40875260105944e-08
538 4.40198171456352e-08
539 4.39486241570197e-08
540 4.40176318734498e-08
541 4.39040320772133e-08
542 4.38488468763598e-08
543 4.38589634825792e-08
544 4.37987351271651e-08
545 4.38118320711389e-08
546 4.37512297546405e-08
547 4.37572114331264e-08
548 4.3651819700008e-08
549 4.3684689366108e-08
550 4.36934562957347e-08
551 4.35558082905185e-08
552 4.35645312368838e-08
553 4.36025922307959e-08
554 4.34834047928767e-08
555 4.34753921123576e-08
556 4.33874795306988e-08
557 4.33960154246904e-08
558 4.33486092539237e-08
559 4.33700110133994e-08
560 4.33062428935038e-08
561 4.32488406272302e-08
562 4.32187208128099e-08
563 4.32603261202313e-08
564 4.31853677447247e-08
565 4.31328445724688e-08
566 4.31304023595924e-08
567 4.30275241360967e-08
568 4.32038977464799e-08
569 4.29647016202406e-08
570 4.3073110626457e-08
571 4.29072486034521e-08
572 4.29462493414245e-08
573 4.30247489031643e-08
574 4.2927465772058e-08
575 4.29443984490963e-08
576 4.28379156689473e-08
577 4.28640074265907e-08
578 4.27715048558763e-08
579 4.28647378651892e-08
580 4.27158761557855e-08
581 4.27652355712294e-08
582 4.26906899491097e-08
583 4.26899345349341e-08
584 4.25798669703159e-08
585 4.25541621988934e-08
586 4.26455745488052e-08
587 4.25325880508698e-08
588 4.26382859857188e-08
589 4.25581909415662e-08
590 4.24886411063419e-08
591 4.2463335780285e-08
592 4.2497370588368e-08
593 4.24259101605706e-08
594 4.24671097305485e-08
595 4.23742386264081e-08
596 4.23636003075156e-08
597 4.23287207040879e-08
598 4.2361737849328e-08
599 4.22274559366276e-08
600 4.2247981894894e-08
601 4.22182995081233e-08
602 4.22244846374209e-08
603 4.22047124806202e-08
604 4.21921520481483e-08
605 4.20939293537259e-08
606 4.20988831337787e-08
607 4.2108349588732e-08
608 4.19990031643724e-08
609 4.20942374519395e-08
610 4.19590243849743e-08
611 4.2032044535123e-08
612 4.19055622338238e-08
613 4.20046055769063e-08
614 4.19872050763104e-08
615 4.18914970534701e-08
616 4.19155748934141e-08
617 4.18898193517148e-08
618 4.17898171654052e-08
619 4.18704921323343e-08
620 4.17477749097284e-08
621 4.17817580400381e-08
622 4.17134393784213e-08
623 4.17367689569126e-08
624 4.16671126994839e-08
625 4.17273583666056e-08
626 4.16187962302139e-08
627 4.16967885403796e-08
628 4.15955587254047e-08
629 4.16140461623193e-08
630 4.1586050502751e-08
631 4.1583756363428e-08
632 4.15410758813906e-08
633 4.15105908095903e-08
634 4.14263540700066e-08
635 4.14785282059071e-08
636 4.1441569720746e-08
637 4.13605365923164e-08
638 4.14311892706909e-08
639 4.13210339527126e-08
640 4.14121669847223e-08
641 4.14065858780344e-08
642 4.12276453956029e-08
643 4.13182403515844e-08
644 4.11873161725307e-08
645 4.13281825124567e-08
646 4.12119066179883e-08
647 4.11930100927549e-08
648 4.11502073369707e-08
649 4.11475700896613e-08
650 4.11760597831545e-08
651 4.11097536481275e-08
652 4.10073163761293e-08
653 4.11253845562509e-08
654 4.10778240071963e-08
655 4.10487950863736e-08
656 4.10833768940133e-08
657 4.10661919953981e-08
658 4.10206412841951e-08
659 4.09392566356193e-08
660 4.0975660198006e-08
661 4.09492561139757e-08
662 4.0938018240233e-08
663 4.07852325237101e-08
664 4.10047759402676e-08
665 4.07923337231164e-08
666 4.07361078222923e-08
667 4.08684930943704e-08
668 4.08093413668542e-08
669 4.07324929840858e-08
670 4.08917449803425e-08
671 4.08711350337043e-08
672 4.07491819689465e-08
673 4.0664190451567e-08
674 4.06797054517405e-08
675 4.06991289287895e-08
676 4.06965779813362e-08
677 4.06639406806963e-08
678 4.05459251588969e-08
679 4.06228209668402e-08
680 4.061655859533e-08
681 4.05309909599083e-08
682 4.05147774511505e-08
683 4.06040734000079e-08
684 4.05854678440232e-08
685 4.05466586375081e-08
686 4.0558345062891e-08
687 4.05029348267139e-08
688 4.04382669558778e-08
689 4.05269161671828e-08
690 4.03865163325356e-08
691 4.04077432725192e-08
692 4.02747469994136e-08
693 4.02946296660378e-08
694 4.04171883556792e-08
695 4.01163335785881e-08
696 4.0042646763272e-08
697 3.99368677133083e-08
698 4.00511724798491e-08
699 4.02523031772795e-08
700 4.01706395942725e-08
701 4.01246986454495e-08
702 4.01203526068095e-08
703 4.00152269566245e-08
704 4.01047952311995e-08
705 3.99937384947258e-08
706 3.99857937463111e-08
707 4.01461750998422e-08
708 4.00520198078347e-08
709 4.00572467134097e-08
710 3.99848564753835e-08
711 3.9980925950811e-08
712 4.00155793625068e-08
713 3.9974076591287e-08
714 3.99530649617397e-08
715 3.98863211448219e-08
716 3.99018468959511e-08
717 3.98976241258175e-08
718 3.98862783532739e-08
719 3.98101544751839e-08
720 3.92602403973541e-08
721 3.99253570548819e-08
722 3.9232854044613e-08
723 3.99022745243283e-08
724 3.92233980295309e-08
725 3.9797401049535e-08
726 3.91496635092636e-08
727 3.98031175357083e-08
728 3.9115704478121e-08
729 3.97445447861688e-08
730 3.90977548216931e-08
731 3.92112413587498e-08
732 3.92924036103359e-08
733 3.9285478724338e-08
734 3.92063141898547e-08
735 3.92784579150707e-08
736 3.9191171535391e-08
737 3.91744818688533e-08
738 3.91488898867642e-08
739 3.91167747868515e-08
740 3.90887675743023e-08
741 3.90808423897937e-08
742 3.90685375784017e-08
743 3.90593482348933e-08
744 3.90197266475489e-08
745 3.89882012195031e-08
746 3.89788299854921e-08
747 3.89827458791014e-08
748 3.89358410215479e-08
749 3.92459322493544e-08
750 3.90863738100045e-08
751 3.88044922727815e-08
752 3.88524170242821e-08
753 3.91474470751163e-08
754 3.86916844621332e-08
755 3.90433544401603e-08
756 3.89338279291884e-08
757 3.9017254453988e-08
758 3.88897600447358e-08
759 3.88535167794579e-08
760 3.88480676396519e-08
761 3.87952367533195e-08
762 3.84959176822353e-08
763 3.88687257517617e-08
764 3.88171320708786e-08
765 3.87895294946539e-08
766 3.87069647853622e-08
767 3.86873451687642e-08
768 3.88213264084225e-08
769 3.86265001899311e-08
770 3.87385888702685e-08
771 3.85924473995658e-08
772 3.86444353293047e-08
773 3.85417486628015e-08
774 3.86435330619328e-08
775 3.84750776021292e-08
776 3.86007968111723e-08
777 3.84652230933646e-08
778 3.85562722882149e-08
779 3.83896940236994e-08
780 3.8531226222549e-08
781 3.8353836353977e-08
782 3.84570575635568e-08
783 3.840059043414e-08
784 3.84966493609529e-08
785 3.83798043406269e-08
786 3.84553286618683e-08
787 3.82617532512786e-08
788 3.84131992781089e-08
789 3.82484448604448e-08
790 3.83532753813753e-08
791 3.82023424765343e-08
792 3.82684551420986e-08
793 3.82167830630387e-08
794 3.82565470895546e-08
795 3.81377650009451e-08
796 3.81865733791731e-08
797 3.81253235639534e-08
798 3.81474441151219e-08
799 3.80699214521663e-08
800 3.81087275815872e-08
801 3.79952563247521e-08
802 3.80470373619879e-08
803 3.79510837074104e-08
804 3.79636832803509e-08
805 3.79814924531541e-08
806 3.79039302149664e-08
807 3.79669950423267e-08
808 3.78116216033231e-08
809 3.79206982905611e-08
810 3.77899681960336e-08
811 3.78194103876073e-08
812 3.81854946742699e-08
813 3.81646652489742e-08
814 3.7573540139002e-08
815 3.65901056254003e-08
816 3.63524741209265e-08
817 3.59769439732816e-08
818 3.57085172453253e-08
819 3.56828577379176e-08
820 3.5385676594224e-08
821 3.53987269365863e-08
822 3.52422148142129e-08
823 3.51192899206154e-08
824 3.50551919157116e-08
825 3.49317499503954e-08
826 3.47969734912912e-08
827 3.47179210158366e-08
828 3.45986650243102e-08
829 3.45941848680642e-08
830 3.44693762261983e-08
831 3.423700789984e-08
832 3.4282399941965e-08
833 3.41767939144866e-08
834 3.42261691261836e-08
835 3.40826578582032e-08
836 3.40729505055037e-08
837 3.40467944504042e-08
838 3.39166688037018e-08
839 3.39900544112926e-08
840 3.38416722025503e-08
841 3.39124175501304e-08
842 3.38303806508566e-08
843 3.37260163572051e-08
844 3.37362954523357e-08
845 3.36370150264553e-08
846 3.36086106140776e-08
847 3.36861863399207e-08
848 3.34454853816446e-08
849 3.35046054471544e-08
850 3.34346212724768e-08
851 3.34151683492046e-08
852 3.33973003361354e-08
853 3.33638261424962e-08
854 3.33317901457519e-08
855 3.32968606222561e-08
856 3.32973886207899e-08
857 3.32153409041691e-08
858 3.32032162664309e-08
859 3.32553122495138e-08
860 3.30801271990477e-08
861 3.31146156460349e-08
862 3.30898388980483e-08
863 3.30885672035208e-08
864 3.31906975705731e-08
865 3.29667822200896e-08
866 3.30311228009528e-08
867 3.2983688708077e-08
868 3.29472081652682e-08
869 3.29708091664216e-08
870 3.29311217872608e-08
871 3.27717321959575e-08
872 3.29471660680536e-08
873 3.27071704533388e-08
874 3.29019042863976e-08
875 3.27298975832946e-08
876 3.27326783160498e-08
877 3.28138998182403e-08
878 3.27302541280883e-08
879 3.25760602757352e-08
880 3.27516906855063e-08
881 3.26412456235214e-08
882 3.25057795911299e-08
883 3.26058003423668e-08
884 3.26391818756822e-08
885 3.25054162533256e-08
886 3.2399280751827e-08
887 3.25846443811617e-08
888 3.23446345609213e-08
889 3.25172295942977e-08
890 3.23314930765761e-08
891 3.23898347209806e-08
892 3.24418697994489e-08
893 3.23854295118942e-08
894 3.23140623761375e-08
895 3.22769484006891e-08
896 3.23795855721176e-08
897 3.2280080317415e-08
898 3.21480119098716e-08
899 3.22407689266058e-08
900 3.22983426233048e-08
901 3.21877479145805e-08
902 3.21111416605735e-08
903 3.21559959661766e-08
904 3.21044505109391e-08
905 3.22067429359141e-08
906 3.21252567041519e-08
907 3.21144227402037e-08
908 3.20933311059868e-08
909 3.20700041098743e-08
910 3.20513309743919e-08
911 3.20026052951228e-08
912 3.20247414700159e-08
913 3.18739256734979e-08
914 3.20709130077201e-08
915 3.19617564468189e-08
916 3.18434775450349e-08
917 3.20174672059359e-08
918 3.19205940753076e-08
919 3.18893281741062e-08
920 3.18865999238138e-08
921 3.18362135531824e-08
922 3.1860745426826e-08
923 3.17905544460206e-08
924 3.18268102439401e-08
925 3.17580529944905e-08
926 3.17880311884178e-08
927 3.17246402457005e-08
928 3.17670487743893e-08
929 3.16368097945663e-08
930 3.18061655193436e-08
931 3.16317428250557e-08
932 3.17676397580957e-08
933 3.16853365209813e-08
934 3.16809619340663e-08
935 3.15172888534221e-08
936 3.16859009839021e-08
937 3.14860678329865e-08
938 3.15561344459425e-08
939 3.16247325848895e-08
940 3.15099529502039e-08
941 3.13841611052634e-08
942 3.15877956957866e-08
943 3.1493440201702e-08
944 3.14709752693432e-08
945 3.14630571265351e-08
946 3.14080401571104e-08
947 3.14501722114002e-08
948 3.12820876682274e-08
949 3.13952414632279e-08
950 3.14650470991662e-08
951 3.13745178130986e-08
952 3.13687177024846e-08
953 3.12374784652292e-08
954 3.13976187216181e-08
955 3.13191495509013e-08
956 3.12932713739666e-08
957 3.12611803041651e-08
958 3.12809924634117e-08
959 3.1153535526407e-08
960 3.11781208839346e-08
961 3.12003939728633e-08
962 3.12717407719987e-08
963 3.11829169312716e-08
964 3.10803558194461e-08
965 3.12244510807247e-08
966 3.10966734597962e-08
967 3.11715187399919e-08
968 3.10546395683176e-08
969 3.11292940655683e-08
970 3.09615818148146e-08
971 3.1001324094504e-08
972 3.11489360445005e-08
973 3.10372384206836e-08
974 3.09230252002468e-08
975 3.11253006939349e-08
976 3.09935374882553e-08
977 3.09748591444947e-08
978 3.09305360299827e-08
979 3.09293649609632e-08
980 3.10228222470954e-08
981 3.08134413333772e-08
982 3.10043579785368e-08
983 3.09191341405501e-08
984 3.07495049691742e-08
985 3.10026101397742e-08
986 3.08775222723767e-08
987 3.09234736521891e-08
988 3.08217316118409e-08
989 3.07712406573124e-08
990 3.07177629528255e-08
991 3.08560425448245e-08
992 3.08858671422829e-08
993 3.06045478954964e-08
994 3.07744305965141e-08
995 3.06356688080101e-08
996 3.07966848720476e-08
997 3.08043280679016e-08
998 3.07508845798221e-08
999 3.06857237222236e-08
1000 3.05849193680263e-08
1001 3.06951073940631e-08
1002 3.07915475965537e-08
1003 3.05978644588034e-08
1004 3.06568467360879e-08
1005 3.0587569365581e-08
1006 3.06600975916726e-08
1007 3.05796522417356e-08
1008 3.05679823409122e-08
1009 3.07112359227979e-08
1010 3.04161806989889e-08
1011 3.04010284015721e-08
1012 3.0710449728133e-08
1013 3.03614319718815e-08
1014 3.0486503332261e-08
1015 3.04554807502022e-08
1016 3.06738826632014e-08
1017 3.03221969235601e-08
1018 3.05047384254742e-08
1019 3.04764025820248e-08
1020 3.03430611832134e-08
1021 3.05363844588236e-08
1022 3.03490142383733e-08
1023 3.03541103412464e-08
1024 3.0350506067478e-08
1025 3.04258657815026e-08
1026 3.04109658939566e-08
1027 3.02237794176374e-08
1028 3.02163328349536e-08
1029 3.03826733671908e-08
1030 3.00949179441456e-08
1031 3.03287617460235e-08
1032 3.03897720539403e-08
1033 3.01648178122615e-08
1034 3.01824942867679e-08
1035 3.01030665326163e-08
1036 3.02019988438484e-08
1037 3.02607490887397e-08
1038 3.0083693536298e-08
1039 3.00767928722578e-08
1040 3.01605709804864e-08
1041 3.01260105306511e-08
1042 3.00811490177022e-08
1043 3.0178696086125e-08
1044 3.00283993803774e-08
1045 3.00090746072623e-08
1046 2.99317926124587e-08
1047 2.99523463651585e-08
1048 2.98885366059043e-08
1049 2.99568394199756e-08
1050 2.9854256692774e-08
1051 2.99151012674859e-08
1052 2.98441781005909e-08
1053 2.99138052322068e-08
1054 2.97977018779427e-08
1055 2.9869781805747e-08
1056 2.97668950082208e-08
1057 2.98498081863219e-08
1058 2.97531454807043e-08
1059 2.98123950659424e-08
1060 2.97268697324604e-08
1061 2.98023559506255e-08
1062 2.97021841353384e-08
1063 2.97903295769508e-08
1064 2.96816697387126e-08
1065 2.97461217266815e-08
1066 2.96144233393747e-08
1067 2.97699852769284e-08
1068 2.9674038795946e-08
1069 2.96910897756852e-08
1070 2.96343538623844e-08
1071 2.96530967520692e-08
1072 2.96041174416839e-08
1073 2.96516355571264e-08
1074 2.95837091996987e-08
1075 2.95929973161968e-08
1076 2.95957112834699e-08
1077 2.9577447886675e-08
1078 2.95488280674938e-08
1079 2.95757277828379e-08
1080 2.95118454938237e-08
1081 2.9555969897288e-08
1082 2.95122129543302e-08
1083 2.95093192397289e-08
1084 2.94707942634087e-08
1085 2.95261192420071e-08
1086 2.94164091880766e-08
1087 2.94805585483449e-08
1088 2.9415598328475e-08
1089 2.94592224465706e-08
1090 2.94291397344715e-08
1091 2.94307658093018e-08
1092 2.9400090245435e-08
1093 2.94155818754138e-08
1094 2.93891368365884e-08
1095 2.94202607102889e-08
1096 2.93360474992177e-08
1097 2.93601356136097e-08
1098 2.93665330011184e-08
1099 2.93507964790329e-08
1100 2.92905636796181e-08
1101 2.9371093057895e-08
1102 2.91601409569964e-08
1103 2.91852859557462e-08
1104 2.91869845554515e-08
1105 2.91245154733222e-08
1106 2.90951183321564e-08
1107 2.91660408890948e-08
1108 2.90830172693379e-08
1109 2.90439950918486e-08
1110 2.90562182634346e-08
1111 2.90426306830405e-08
1112 2.90259247235358e-08
1113 2.91164950767531e-08
1114 2.89955620127458e-08
1115 2.91094064490682e-08
1116 2.89089899438988e-08
1117 2.88764398810315e-08
1118 2.90805093516866e-08
1119 2.89402654694015e-08
1120 2.89699908098129e-08
1121 2.88327293753721e-08
1122 2.9026894339701e-08
1123 2.88549216087564e-08
1124 2.88299131263603e-08
1125 2.8879121537928e-08
1126 2.87340795097624e-08
1127 2.89256255263659e-08
1128 2.8890983134966e-08
1129 2.87465757817795e-08
1130 2.87597266421802e-08
1131 2.87958622353646e-08
1132 2.88930797871512e-08
1133 2.87564502474691e-08
1134 2.87152621929465e-08
1135 2.86900354242725e-08
1136 2.86630133181642e-08
1137 2.86773116362493e-08
1138 2.86810715015573e-08
1139 2.86896707750728e-08
1140 2.85866763618525e-08
1141 2.87214518839285e-08
1142 2.86484862843661e-08
1143 2.85242965771815e-08
1144 2.86587195925669e-08
1145 2.86087227086007e-08
1146 2.85385033473862e-08
1147 2.85747194894714e-08
1148 2.85290424240081e-08
1149 2.85720284265434e-08
1150 2.84692555023724e-08
1151 2.85684671268438e-08
1152 2.84486173223186e-08
1153 2.85586035138063e-08
1154 2.84415622642253e-08
1155 2.85150501411202e-08
1156 2.84693922354418e-08
1157 2.85643343724828e-08
1158 2.83975539154824e-08
1159 2.85161665201095e-08
1160 2.84336968991994e-08
1161 2.8485795842137e-08
1162 2.84763115316888e-08
1163 2.84438971787093e-08
1164 2.8385458343827e-08
1165 2.8438733348235e-08
1166 2.83997722694274e-08
1167 2.84089542685884e-08
1168 2.83702699110933e-08
1169 2.84225286131257e-08
1170 2.82381800547071e-08
1171 2.8530476648525e-08
1172 2.83172495434414e-08
1173 2.83554983584455e-08
1174 2.82993960682987e-08
1175 2.83578415005703e-08
1176 2.82905377697595e-08
1177 2.83219319414663e-08
1178 2.83274591679206e-08
1179 2.8252198233325e-08
1180 2.83276562100809e-08
1181 2.82264044710878e-08
1182 2.829386113401e-08
1183 2.82179707755681e-08
1184 2.82846571391104e-08
1185 2.82542824594501e-08
1186 2.82553951569398e-08
1187 2.82158674986821e-08
1188 2.82320950297787e-08
1189 2.81382306916989e-08
1190 2.83285115438847e-08
1191 2.80646509067672e-08
1192 2.82547956418355e-08
1193 2.81147509526569e-08
1194 2.82090371723775e-08
1195 2.8146943632068e-08
1196 2.81341773733512e-08
1197 2.8125225173925e-08
1198 2.82077317930174e-08
1199 2.80683398905524e-08
1200 2.81802916952678e-08
1201 2.80501366327623e-08
1202 2.81275317233209e-08
1203 2.80889624426628e-08
1204 2.81442289620237e-08
1205 2.80124437344043e-08
1206 2.81393104133443e-08
1207 2.80073214065091e-08
1208 2.806160336144e-08
1209 2.80339730818202e-08
1210 2.81052206372312e-08
1211 2.79788635899614e-08
1212 2.80558186764868e-08
1213 2.79707317880629e-08
1214 2.80378089152489e-08
1215 2.79671223017974e-08
1216 2.80741532281414e-08
1217 2.79458950749323e-08
1218 2.79899226360847e-08
1219 2.79097412119178e-08
1220 2.80266013246155e-08
1221 2.7924007930169e-08
1222 2.79731733519029e-08
1223 2.79044066653356e-08
1224 2.79792967254888e-08
1225 2.78832123170591e-08
1226 2.79482916103468e-08
1227 2.78585104571682e-08
1228 2.79605653135118e-08
1229 2.78224254639525e-08
1230 2.79387713770785e-08
1231 2.78559516411203e-08
1232 2.79456595810856e-08
1233 2.77915579818266e-08
1234 2.79126967654975e-08
1235 2.77924110516636e-08
1236 2.78980325179035e-08
1237 2.77728283124556e-08
1238 2.78886646813969e-08
1239 2.7755225526338e-08
1240 2.78515167051463e-08
1241 2.77511030826183e-08
1242 2.78217790397051e-08
1243 2.77566279220931e-08
1244 2.78442419703317e-08
1245 2.76958112321868e-08
1246 2.78262109389971e-08
1247 2.7688694986594e-08
1248 2.77939684669715e-08
1249 2.76893242610043e-08
1250 2.78020070725304e-08
1251 2.76396248644062e-08
1252 2.76914152510521e-08
1253 2.75508076308384e-08
1254 2.76116805617566e-08
1255 2.75856785427742e-08
1256 2.76168204422778e-08
1257 2.75110727929739e-08
1258 2.76258702138765e-08
1259 2.75105976592638e-08
1260 2.76086551929122e-08
1261 2.75757186531944e-08
1262 2.75369093647448e-08
1263 2.7464534510635e-08
1264 2.75558570119738e-08
1265 2.75134755640671e-08
1266 2.7547593340671e-08
1267 2.75883199576388e-08
1268 2.75333798069877e-08
1269 2.75758592920283e-08
1270 2.74171846306182e-08
1271 2.75005644074788e-08
1272 2.74614426238884e-08
1273 2.74706211786935e-08
1274 2.7473375270981e-08
1275 2.74601243142047e-08
1276 2.74310998547023e-08
1277 2.74598891298883e-08
1278 2.74513029974166e-08
1279 2.74202478756091e-08
1280 2.73957542971104e-08
1281 2.74318693513909e-08
1282 2.73660305276646e-08
1283 2.74229521657343e-08
1284 2.73248000866033e-08
1285 2.74032418219328e-08
1286 2.7308705644602e-08
1287 2.74399347732768e-08
1288 2.72856188883175e-08
1289 2.74091365204399e-08
1290 2.72775656611213e-08
1291 2.73821944671848e-08
1292 2.7261844531612e-08
1293 2.73917060595874e-08
1294 2.7251368006187e-08
1295 2.73511874209209e-08
1296 2.72285165521247e-08
1297 2.73508772270503e-08
1298 2.72142886030835e-08
1299 2.73161966195534e-08
1300 2.71970410554712e-08
1301 2.72953961759637e-08
1302 2.72094817348467e-08
1303 2.72951230473328e-08
1304 2.73140372064606e-08
1305 2.72306232695918e-08
1306 2.72076886678185e-08
1307 2.72196793065227e-08
1308 2.72296521479642e-08
1309 2.72178601339323e-08
1310 2.71753359373239e-08
1311 2.72437480932641e-08
1312 2.71077220530547e-08
1313 2.72499756945876e-08
1314 2.7084854697268e-08
1315 2.71867024483896e-08
1316 2.70541372853206e-08
1317 2.71486870000448e-08
1318 2.70609692991641e-08
1319 2.71304067716027e-08
1320 2.70523380450083e-08
1321 2.71450166942078e-08
1322 2.70168355298583e-08
1323 2.71455488032313e-08
1324 2.70228888437085e-08
1325 2.71202675803384e-08
1326 2.69859410924056e-08
1327 2.707917820155e-08
1328 2.70837492035714e-08
1329 2.70336457397491e-08
1330 2.69622923154689e-08
1331 2.7076119217373e-08
1332 2.69638941994277e-08
1333 2.70170624643296e-08
1334 2.69312912688502e-08
1335 2.6988229304914e-08
1336 2.6932237163102e-08
1337 2.69473116440988e-08
1338 2.69470512126446e-08
1339 2.6960322176306e-08
1340 2.69340871090762e-08
1341 2.69892785258463e-08
1342 2.68624548875884e-08
1343 2.69670326757954e-08
1344 2.68667063425543e-08
1345 2.69292446435365e-08
1346 2.68909888483382e-08
1347 2.69019783518676e-08
1348 2.68751453578187e-08
1349 2.68714408482307e-08
1350 2.68316746305786e-08
1351 2.68887214529556e-08
1352 2.67950900396663e-08
1353 2.69178521361546e-08
1354 2.67856674740496e-08
1355 2.68712771120949e-08
1356 2.68168407833169e-08
1357 2.67986501467643e-08
1358 2.67518870631367e-08
1359 2.68532793721299e-08
1360 2.67612147728524e-08
1361 2.68033652017774e-08
1362 2.682695137457e-08
1363 2.68070964808409e-08
1364 2.67005742904391e-08
1365 2.68048030742651e-08
1366 2.67690985613633e-08
1367 2.67691424451488e-08
1368 2.67161788656267e-08
1369 2.67376284923682e-08
1370 2.66742251890939e-08
1371 2.67159933748928e-08
1372 2.668146423912e-08
1373 2.67331322572062e-08
1374 2.67290414213139e-08
1375 2.66969376738757e-08
1376 2.67120314687563e-08
1377 2.66980043686083e-08
1378 2.66547349307977e-08
1379 2.67171147112499e-08
1380 2.65658991784878e-08
1381 2.67076104816244e-08
1382 2.65631655469445e-08
1383 2.67261858477585e-08
1384 2.65931624754412e-08
1385 2.66961606045779e-08
1386 2.65581495124678e-08
1387 2.6626582581013e-08
1388 2.65767963072427e-08
1389 2.66385226483923e-08
1390 2.65417179556859e-08
1391 2.6689589492479e-08
1392 2.65300862514195e-08
1393 2.66342269907849e-08
1394 2.65107313983659e-08
1395 2.66299601827669e-08
1396 2.65195367314952e-08
1397 2.65895765954038e-08
1398 2.64826148761443e-08
1399 2.65228059459499e-08
1400 2.65632195965981e-08
1401 2.65430245931508e-08
1402 2.64984142106517e-08
1403 2.65345269285788e-08
1404 2.64567107237657e-08
1405 2.65453605423627e-08
1406 2.64625545915997e-08
1407 2.65667113243939e-08
1408 2.64130614384683e-08
1409 2.6544427019326e-08
1410 2.63939033238181e-08
1411 2.65529515324214e-08
1412 2.64022698062139e-08
1413 2.65056721306944e-08
1414 2.639183188613e-08
1415 2.65740969982442e-08
1416 2.63534860884818e-08
1417 2.64392411857273e-08
1418 2.63943164968694e-08
1419 2.64494847026509e-08
1420 2.64151482076702e-08
1421 2.63820274231108e-08
1422 2.63828014519518e-08
1423 2.64418079167772e-08
1424 2.63561730315942e-08
1425 2.63931352364466e-08
1426 2.63515135232595e-08
1427 2.64089425972003e-08
1428 2.63346389650643e-08
1429 2.63701540630379e-08
1430 2.63403834073195e-08
1431 2.64673682233152e-08
1432 2.62741409136069e-08
1433 2.6379893786288e-08
1434 2.6281273702855e-08
1435 2.6347083239564e-08
1436 2.62707944818086e-08
1437 2.63546162524442e-08
1438 2.62702201545562e-08
1439 2.63219942071569e-08
1440 2.62717606926977e-08
1441 2.62950675626872e-08
1442 2.62002370117198e-08
1443 2.63072860984259e-08
1444 2.62120663494514e-08
1445 2.62911485293671e-08
1446 2.62120315523973e-08
1447 2.63038600520904e-08
1448 2.62036884501882e-08
1449 2.63114513954221e-08
1450 2.62205268339866e-08
1451 2.62469215706052e-08
1452 2.62025780581876e-08
1453 2.62181113130922e-08
1454 2.61440442586025e-08
1455 2.62297699129554e-08
1456 2.61504798440004e-08
1457 2.61674217350016e-08
1458 2.61403198655863e-08
1459 2.61935967138527e-08
1460 2.61788943973773e-08
1461 2.61265588521464e-08
1462 2.61800282275182e-08
1463 2.61088374584606e-08
1464 2.61686794400529e-08
1465 2.6063444393376e-08
1466 2.6173183785172e-08
1467 2.60870143473646e-08
1468 2.6104188260323e-08
1469 2.60681283794639e-08
1470 2.61082592394324e-08
1471 2.60957278259699e-08
1472 2.60492620858255e-08
1473 2.60042955690931e-08
1474 2.61190280643753e-08
1475 2.60817421522397e-08
1476 2.60536559570923e-08
1477 2.60056077241444e-08
1478 2.60533915403816e-08
1479 2.60199981476639e-08
1480 2.60494127555244e-08
1481 2.60101184521577e-08
1482 2.6029138680661e-08
1483 2.59896815231198e-08
1484 2.59775222863912e-08
1485 2.5971451997453e-08
1486 2.59736563841262e-08
1487 2.59587701725295e-08
1488 2.59636261410456e-08
1489 2.60434939298726e-08
1490 2.59921797090001e-08
1491 2.59690754709219e-08
1492 2.59314103121522e-08
1493 2.60190717191744e-08
1494 2.59671746087875e-08
1495 2.59293350630685e-08
1496 2.59035487386594e-08
1497 2.5896951106219e-08
1498 2.59728864941966e-08
1499 2.58635465100188e-08
1500 2.59243097544548e-08
1501 2.5939033475364e-08
1502 2.59009187648651e-08
1503 2.59112327902766e-08
1504 2.58533961263741e-08
1505 2.58944842550513e-08
1506 2.58754345587864e-08
1507 2.58702707547354e-08
1508 2.58619607207411e-08
1509 2.59055578624157e-08
1510 2.58619402679905e-08
1511 2.59293988507103e-08
1512 2.58015787772958e-08
1513 2.58592152500015e-08
1514 2.58578162766199e-08
1515 2.57535662566699e-08
1516 2.58752404274087e-08
1517 2.57744878600086e-08
1518 2.58830533319543e-08
1519 2.57864697106314e-08
1520 2.5825459418094e-08
1521 2.57866120241257e-08
1522 2.58755419542123e-08
1523 2.57449763561013e-08
1524 2.57818128501786e-08
1525 2.57756856707481e-08
1526 2.57856876078044e-08
1527 2.57712816733591e-08
1528 2.58106783834755e-08
1529 2.57785605592975e-08
1530 2.57580837965676e-08
1531 2.57589294985205e-08
1532 2.57484868799729e-08
1533 2.57496422797487e-08
1534 2.57832189780594e-08
1535 2.56963850973069e-08
1536 2.57286245108901e-08
1537 2.57365473894655e-08
1538 2.57217185202219e-08
1539 2.574453999582e-08
1540 2.57255443285587e-08
1541 2.57851379745766e-08
1542 2.57914987513708e-08
1543 2.56956864290725e-08
1544 2.56693255686891e-08
1545 2.57407004242705e-08
1546 2.57346163130556e-08
1547 2.56542793732883e-08
1548 2.57380526735851e-08
1549 2.56558604121437e-08
1550 2.5741019812342e-08
1551 2.56347480485086e-08
1552 2.56324533887131e-08
1553 2.56029327685425e-08
1554 2.57085816255387e-08
1555 2.56021385303118e-08
1556 2.57099247413795e-08
1557 2.55758836791298e-08
1558 2.55990499686831e-08
1559 2.56165157963562e-08
1560 2.5613599950125e-08
1561 2.56562107359137e-08
1562 2.55515333011935e-08
1563 2.55889130968967e-08
1564 2.55882895323634e-08
1565 2.55553134302922e-08
1566 2.56448554984967e-08
1567 2.55300927425317e-08
1568 2.5484172194945e-08
1569 2.55320845468088e-08
1570 2.5557833850165e-08
1571 2.55109285851596e-08
1572 2.55932649950719e-08
1573 2.54553951382075e-08
1574 2.54890943021113e-08
1575 2.55498554064815e-08
1576 2.55423255386411e-08
1577 2.54666263650449e-08
1578 2.54849171854588e-08
1579 2.55199252250371e-08
1580 2.54166074569184e-08
1581 2.54724638297787e-08
1582 2.54799579979537e-08
1583 2.54478850378881e-08
1584 2.55261158477182e-08
1585 2.54222208733168e-08
1586 2.54628673754809e-08
1587 2.53717690030797e-08
1588 2.54880307588579e-08
1589 2.53430988854664e-08
1590 2.54881334222912e-08
1591 2.53818113198356e-08
1592 2.53497975204464e-08
1593 2.53458205323742e-08
1594 2.53769703548734e-08
1595 2.53030008972033e-08
1596 2.54058133382085e-08
1597 2.53260126990007e-08
1598 2.53742469382701e-08
1599 2.52720515336868e-08
1600 2.54122221439168e-08
1601 2.52885547697979e-08
1602 2.5350369145194e-08
1603 2.52774807227851e-08
1604 2.53431673618021e-08
1605 2.52248587382375e-08
1606 2.52530473985324e-08
1607 2.53868175623673e-08
1608 2.52372129299072e-08
1609 2.52240496687595e-08
1610 2.52886458551593e-08
1611 2.5265821172793e-08
1612 2.53233026719091e-08
1613 2.53385266750783e-08
1614 2.51688830712471e-08
1615 2.5175642362818e-08
1616 2.52543050032195e-08
1617 2.52130242253124e-08
1618 2.52619341751803e-08
1619 2.52156624240829e-08
1620 2.5109954329805e-08
1621 2.52229770003609e-08
1622 2.52059153025286e-08
1623 2.52446715458543e-08
1624 2.51265916400012e-08
1625 2.5199067354098e-08
1626 2.51789562559779e-08
1627 2.51590148541148e-08
1628 2.51411608995777e-08
1629 2.51907186785694e-08
1630 2.51579845123029e-08
1631 2.51741601078326e-08
1632 2.50821755771735e-08
1633 2.51028333793091e-08
1634 2.50710885629779e-08
1635 2.51229134886621e-08
1636 2.5046430813358e-08
1637 2.51858946596073e-08
1638 2.50887524431143e-08
1639 2.51043294530184e-08
1640 2.50642863079964e-08
1641 2.5038503814967e-08
1642 2.50507165431291e-08
1643 2.49821878062217e-08
1644 2.5030104640944e-08
1645 2.50258278180482e-08
1646 2.50802786878612e-08
1647 2.49477290812372e-08
1648 2.51247045446323e-08
1649 2.49140795449687e-08
1650 2.50308299831747e-08
1651 2.49392274747073e-08
1652 2.49882043770189e-08
1653 2.49733996728896e-08
1654 2.49484582250936e-08
1655 2.49982601605581e-08
1656 2.48884662572557e-08
1657 2.50305580005161e-08
1658 2.4804488768515e-08
1659 2.49888621322025e-08
1660 2.49069933551116e-08
1661 2.49149581432739e-08
1662 2.47720159296527e-08
1663 2.5007104345498e-08
1664 2.4827264957894e-08
1665 2.48252995056486e-08
1666 2.4832714202061e-08
1667 2.49337027991015e-08
1668 2.47650675926359e-08
1669 2.48962143887255e-08
1670 2.48261632092017e-08
1671 2.48033215459742e-08
1672 2.482399248116e-08
1673 2.48076194444558e-08
1674 2.4711193453486e-08
1675 2.47537939915832e-08
1676 2.4782919900046e-08
1677 2.47786509636194e-08
1678 2.48522657814165e-08
1679 2.47363684919843e-08
1680 2.47744284699314e-08
1681 2.46972620079333e-08
1682 2.47484435346168e-08
1683 2.47679156938352e-08
1684 2.46590080816844e-08
1685 2.46350603514411e-08
1686 2.47259276031642e-08
1687 2.46587390344555e-08
1688 2.46365423002271e-08
1689 2.463198849878e-08
1690 2.46687975914428e-08
1691 2.46290756740208e-08
1692 2.45817094126632e-08
1693 2.46325750924425e-08
1694 2.46094450926382e-08
1695 2.45466855557375e-08
1696 2.48048124142919e-08
1697 2.45247348995559e-08
1698 2.45267452485542e-08
1699 2.45116769017617e-08
1700 2.44986675741465e-08
1701 2.44930232770058e-08
1702 2.45655066326567e-08
1703 2.44693643552951e-08
1704 2.45326079044705e-08
1705 2.44180791870718e-08
1706 2.45298583229081e-08
1707 2.44762399144172e-08
1708 2.44597239995592e-08
1709 2.4405237680325e-08
1710 2.44106052169624e-08
1711 2.44446382535735e-08
1712 2.43895619294054e-08
1713 2.43730712999879e-08
1714 2.43738829515117e-08
1715 2.44104879998375e-08
1716 2.43540433537692e-08
1717 2.43534642002663e-08
1718 2.43401534938403e-08
1719 2.41876730768809e-08
1720 2.44230157682157e-08
1721 2.42033891151294e-08
1722 2.43481529218492e-08
1723 2.4240991973512e-08
1724 2.43357160010227e-08
1725 2.42579451370517e-08
1726 2.42070601459421e-08
1727 2.42526574996127e-08
1728 2.41599430477191e-08
1729 2.42280949742657e-08
1730 2.41961090019505e-08
1731 2.41846710808957e-08
1732 2.41516230115035e-08
1733 2.41557289346606e-08
1734 2.40734404595155e-08
1735 2.41458685728002e-08
1736 2.41280690034262e-08
1737 2.41055745453522e-08
1738 2.40225615248946e-08
1739 2.41086083370146e-08
1740 2.40395624587775e-08
1741 2.40547246470602e-08
1742 2.40433879989954e-08
1743 2.40893297108746e-08
1744 2.40032318894468e-08
1745 2.40127931938616e-08
1746 2.40491842855572e-08
1747 2.40321658446785e-08
1748 2.39377849093891e-08
1749 2.40456509505993e-08
1750 2.39884333910201e-08
1751 2.39878925689663e-08
1752 2.40398992688018e-08
1753 2.39216645947504e-08
1754 2.39329331541693e-08
1755 2.39720510926134e-08
1756 2.386612318106e-08
1757 2.39433394078059e-08
1758 2.39898603007216e-08
1759 2.39145039262478e-08
1760 2.38960007218347e-08
1761 2.39374265877945e-08
1762 2.38642671142042e-08
1763 2.3887529337463e-08
1764 2.38900510782791e-08
1765 2.39133699306837e-08
1766 2.39088164568635e-08
1767 2.38610462997002e-08
1768 2.38914447012739e-08
1769 2.39151616225897e-08
1770 2.37980656133674e-08
1771 2.38233482914296e-08
1772 2.37914088700997e-08
1773 2.38921647568446e-08
1774 2.39599409324054e-08
1775 2.37775934848194e-08
1776 2.36949280572052e-08
1777 2.36993129238972e-08
1778 2.38039129568657e-08
1779 2.36750252204931e-08
1780 2.3764288674899e-08
1781 2.36920999154844e-08
1782 2.37819539095074e-08
1783 2.37479941259666e-08
1784 2.37953344350839e-08
1785 2.36791987018314e-08
1786 2.3693970970684e-08
1787 2.36462272973359e-08
1788 2.367137828585e-08
1789 2.36934520903009e-08
1790 2.36886701767691e-08
1791 2.36257424248709e-08
1792 2.36864730547293e-08
1793 2.35649809614547e-08
1794 2.37476086366595e-08
1795 2.36645480325981e-08
1796 2.35534005836868e-08
1797 2.37170362135108e-08
1798 2.35541381015114e-08
1799 2.36437476458518e-08
1800 2.36628805680983e-08
1801 2.35285389036388e-08
1802 2.35745353949213e-08
1803 2.35864440087941e-08
1804 2.35354600115478e-08
1805 2.35537980357581e-08
1806 2.35468467599809e-08
1807 2.35003564372782e-08
1808 2.35451013959054e-08
1809 2.34816762405554e-08
1810 2.35177400442765e-08
1811 2.34690398264181e-08
1812 2.34476795630822e-08
1813 2.34655274522488e-08
1814 2.3500381192032e-08
1815 2.3535263519614e-08
1816 2.34153387008718e-08
1817 2.3414497220875e-08
1818 2.35174032237051e-08
1819 2.33705606581536e-08
1820 2.33844401338334e-08
1821 2.35604134851686e-08
1822 2.33603627216583e-08
1823 2.34445013722606e-08
1824 2.3377803015201e-08
1825 2.3392472220829e-08
1826 2.33448904304323e-08
1827 2.35200267421076e-08
1828 2.33159129476856e-08
1829 2.34421437540488e-08
1830 2.32921291545285e-08
1831 2.33717511622933e-08
1832 2.33867352599226e-08
1833 2.34012203620404e-08
1834 2.33901917393586e-08
1835 2.34589754219972e-08
1836 2.32627995133683e-08
1837 2.3300823722372e-08
1838 2.32908854007619e-08
1839 2.33678703106532e-08
1840 2.32798954836744e-08
1841 2.32181087935768e-08
1842 2.33799171944771e-08
1843 2.32077268419451e-08
1844 2.33761990007242e-08
1845 2.31862677320116e-08
1846 2.31736525693194e-08
1847 2.33179169381037e-08
1848 2.33278682110205e-08
1849 2.32542586968698e-08
1850 2.3244013794832e-08
1851 2.31637634646731e-08
1852 2.32591347202327e-08
1853 2.31644695424249e-08
1854 2.32325074217554e-08
1855 2.32338103863805e-08
1856 2.31349299986139e-08
1857 2.32366017319796e-08
1858 2.32448473046487e-08
1859 2.31280871128003e-08
1860 2.31943313713234e-08
1861 2.31257678833252e-08
1862 2.31520340027158e-08
1863 2.32129459193375e-08
1864 2.32138621496514e-08
1865 2.30728194177132e-08
1866 2.31305570703721e-08
1867 2.30838813792822e-08
1868 2.32127143100502e-08
1869 2.31040951343964e-08
1870 2.31242507369167e-08
1871 2.30812865423813e-08
1872 2.32042603435234e-08
1873 2.30412772489474e-08
1874 2.30297349521624e-08
1875 2.30618350836043e-08
1876 2.30034682577873e-08
1877 2.29985358585205e-08
1878 2.30455711061062e-08
1879 2.30231990411278e-08
1880 2.30158398639713e-08
1881 2.29943519931552e-08
1882 2.30569542458703e-08
1883 2.29910579762072e-08
1884 2.30076582894423e-08
1885 2.30295842539308e-08
1886 2.29867355844293e-08
1887 2.29712790306147e-08
1888 2.29814925351013e-08
1889 2.28878047386916e-08
1890 2.31184019878761e-08
1891 2.2938828566188e-08
1892 2.28860113807849e-08
1893 2.30027509090514e-08
1894 2.30525317352903e-08
1895 2.2912090886229e-08
1896 2.29118955221708e-08
1897 2.29548558592363e-08
1898 2.30045011436753e-08
1899 2.28699446811476e-08
1900 2.2906253878352e-08
1901 2.29294943525638e-08
1902 2.28715233071553e-08
1903 2.29277913278425e-08
1904 2.29914662907005e-08
1905 2.29085236715942e-08
1906 2.28282335726915e-08
1907 2.28099834687079e-08
1908 2.29106360248865e-08
1909 2.2806012296761e-08
1910 2.27721448065044e-08
1911 2.27689714434876e-08
1912 2.28225989480357e-08
1913 2.28492371795452e-08
1914 2.27684530110794e-08
1915 2.27869540908587e-08
1916 2.28292624796511e-08
1917 2.27903346583513e-08
1918 2.28288128616416e-08
1919 2.2803344242206e-08
1920 2.2757094650272e-08
1921 2.27472295464271e-08
1922 2.27570448968484e-08
1923 2.27622265023264e-08
1924 2.27460618508202e-08
1925 2.27478555590022e-08
1926 2.27094023641516e-08
1927 2.27460688397851e-08
1928 2.26902557310948e-08
1929 2.26278975810068e-08
1930 2.27276322017911e-08
1931 2.27266529391201e-08
1932 2.27564261687796e-08
1933 2.26537722594067e-08
1934 2.28231448519134e-08
1935 2.26657214924542e-08
1936 2.26240283716095e-08
1937 2.2657589533237e-08
1938 2.26192776787704e-08
1939 2.25960648202461e-08
1940 2.26404644841915e-08
1941 2.25911279024826e-08
1942 2.26255037567569e-08
1943 2.25878419739534e-08
1944 2.25856182946016e-08
1945 2.25880397732858e-08
1946 2.25435529694895e-08
1947 2.2592362744045e-08
1948 2.25973953419389e-08
1949 2.25706189818098e-08
1950 2.25286917922007e-08
1951 2.2560563444407e-08
1952 2.25449517837761e-08
1953 2.254852836725e-08
1954 2.25851740464078e-08
1955 2.25290218659469e-08
1956 2.25222471673669e-08
1957 2.25222911707235e-08
1958 2.25181517867767e-08
1959 2.25503644648573e-08
1960 2.25201958045096e-08
1961 2.24958337570191e-08
1962 2.24942612115919e-08
1963 2.24652323456143e-08
1964 2.24707452702866e-08
1965 2.25064278625009e-08
1966 2.24228053334885e-08
1967 2.24896412309716e-08
1968 2.25261524087195e-08
1969 2.24910273989298e-08
1970 2.25352680647273e-08
1971 2.23985622265221e-08
1972 2.24413081973962e-08
1973 2.24625321428862e-08
1974 2.2369830481983e-08
1975 2.24129387695893e-08
1976 2.24145252120111e-08
1977 2.24067873968448e-08
1978 2.23755813701931e-08
1979 2.2452496764136e-08
1980 2.24139565802028e-08
1981 2.23346952693548e-08
1982 2.23652485501491e-08
1983 2.23466662223615e-08
1984 2.23670080684579e-08
1985 2.23652831351506e-08
1986 2.23270944250675e-08
1987 2.23655415937385e-08
1988 2.23103166864247e-08
1989 2.23365189001568e-08
1990 2.22840693165915e-08
1991 2.23295259945688e-08
1992 2.23229540591285e-08
1993 2.22882632262555e-08
1994 2.23217260689745e-08
1995 2.2288635878942e-08
1996 2.23059030755035e-08
1997 2.22893868117025e-08
1998 2.22572439153579e-08
1999 2.22539434157065e-08
2000 2.23226808211407e-08
2001 2.22700633812911e-08
2002 2.2245495608586e-08
2003 2.23102269221176e-08
2004 2.22465572199404e-08
2005 2.22432480866397e-08
2006 2.22603760434703e-08
2007 2.23174321950514e-08
2008 2.22132968498157e-08
2009 2.21915402841555e-08
2010 2.22322136487207e-08
2011 2.22236613044435e-08
2012 2.21921305966211e-08
2013 2.22228244486367e-08
2014 2.22366098360238e-08
2015 2.22323169466465e-08
2016 2.22399842408905e-08
2017 2.21615738060787e-08
2018 2.21414971054434e-08
2019 2.22403172612795e-08
2020 2.21693916028221e-08
2021 2.22526086651742e-08
2022 2.21830127312694e-08
2023 2.21794891137606e-08
2024 2.2271102793292e-08
2025 2.20913839299453e-08
2026 2.2071727579398e-08
2027 2.2270916460676e-08
2028 2.20976341008816e-08
2029 2.22357759330771e-08
2030 2.21347963170748e-08
2031 2.21298193826547e-08
2032 2.21491032411647e-08
2033 2.20735455936927e-08
2034 2.20330497155974e-08
2035 2.20893572352443e-08
2036 2.2106730897864e-08
2037 2.21588734297118e-08
2038 2.20696087683825e-08
2039 2.20438265455813e-08
2040 2.22087459433862e-08
2041 2.20131875822149e-08
2042 2.20708119479385e-08
2043 2.21025779657547e-08
2044 2.20864324935732e-08
2045 2.20651458555654e-08
2046 2.21050329899386e-08
2047 2.1947304306047e-08
2048 2.20521416181008e-08
2049 2.19590598039554e-08
2050 2.19763253890282e-08
2051 2.19821014987298e-08
2052 2.20134539591355e-08
2053 2.21085936916721e-08
2054 2.19982977300859e-08
2055 2.19808766176444e-08
2056 2.19790304775147e-08
2057 2.20133829070823e-08
2058 2.19943587312921e-08
2059 2.19193972296194e-08
2060 2.1928183602471e-08
2061 2.21022240259883e-08
2062 2.20284773757573e-08
2063 2.19128739359364e-08
2064 2.20020169772184e-08
2065 2.19214694057168e-08
2066 2.19606643441228e-08
2067 2.19205204609052e-08
2068 2.19785568320585e-08
2069 2.18797887843181e-08
2070 2.20708186717822e-08
2071 2.1934255931888e-08
2072 2.21142216724202e-08
2073 2.1854822545575e-08
2074 2.19258557578872e-08
2075 2.18869762927953e-08
2076 2.18984062120287e-08
2077 2.19710850187793e-08
2078 2.18250608986503e-08
2079 2.18897178998478e-08
2080 2.19345449048447e-08
2081 2.18770524855749e-08
2082 2.17895557074188e-08
2083 2.19194317780946e-08
2084 2.19741244864569e-08
2085 2.18545173008566e-08
2086 2.18416959035794e-08
2087 2.18744099325097e-08
2088 2.17993858577792e-08
2089 2.17948552588831e-08
2090 2.19634637216304e-08
2091 2.18518318623184e-08
2092 2.18609366718914e-08
2093 2.18786789509817e-08
2094 2.17591270520456e-08
2095 2.1873878081502e-08
2096 2.17038928392865e-08
2097 2.18764800656857e-08
2098 2.16660000250668e-08
2099 2.19192115401556e-08
2100 2.17394353291267e-08
2101 2.1718644214852e-08
2102 2.17480920075008e-08
2103 2.1865247842312e-08
2104 2.17631101350202e-08
2105 2.17847000882898e-08
2106 2.17416174108642e-08
2107 2.18147293966631e-08
2108 2.18082153928822e-08
2109 2.16872862800477e-08
2110 2.1784051019047e-08
2111 2.16748848491655e-08
2112 2.16901352910748e-08
2113 2.17088824613354e-08
2114 2.17155657972201e-08
2115 2.1694303749209e-08
2116 2.15003888572873e-08
2117 2.16719654192454e-08
2118 2.16778958749275e-08
2119 2.17190033424686e-08
2120 2.16902540299824e-08
2121 2.17536752842307e-08
2122 2.15694031296954e-08
2123 2.15970567728485e-08
2124 2.16237402551611e-08
2125 2.15537884314942e-08
2126 2.15923142095065e-08
2127 2.15156314758991e-08
2128 2.15604345035203e-08
2129 2.15547457267373e-08
2130 2.15436568798966e-08
2131 2.15684466996491e-08
2132 2.17468110176311e-08
2133 2.1541522845836e-08
2134 2.14346567517509e-08
2135 2.17072883902381e-08
2136 2.16185935359992e-08
2137 2.15718384584118e-08
2138 2.16074339957073e-08
2139 2.16049623813497e-08
2140 2.15513346578655e-08
2141 2.15450687912622e-08
2142 2.16629230129772e-08
2143 2.15089451633954e-08
2144 2.17456783289105e-08
2145 2.1525993725291e-08
2146 2.16114060724859e-08
2147 2.15636518094975e-08
2148 2.14384482096763e-08
2149 2.14745205263656e-08
2150 2.1473447071263e-08
2151 2.15140515545942e-08
2152 2.1566694044739e-08
2153 2.13857598964085e-08
2154 2.13657815475443e-08
2155 2.17244715348119e-08
2156 2.13644984647932e-08
2157 2.13851005372856e-08
2158 2.14759026639921e-08
2159 2.15159902910989e-08
2160 2.14516313790103e-08
2161 2.12769932796952e-08
2162 2.13954264091099e-08
2163 2.14050618122519e-08
2164 2.1526445247777e-08
2165 2.16061556137515e-08
2166 2.12836995182597e-08
2167 2.13077798323358e-08
2168 2.13498276435686e-08
2169 2.13503385618763e-08
2170 2.14678203062091e-08
2171 2.14626566751308e-08
2172 2.14220398315756e-08
2173 2.14058057562694e-08
2174 2.13001622165754e-08
2175 2.14940964510246e-08
2176 2.13816219041263e-08
2177 2.12758368766108e-08
2178 2.13251220674282e-08
2179 2.1360441101681e-08
2180 2.1500986224221e-08
2181 2.12821683305275e-08
2182 2.13451973856138e-08
2183 2.15288208661457e-08
2184 2.1418108967608e-08
2185 2.14605191922734e-08
2186 2.13719124396716e-08
2187 2.1347239244518e-08
2188 2.12678699772484e-08
2189 2.14241818458083e-08
2190 2.12099305931801e-08
2191 2.12412838086484e-08
2192 2.14370682196652e-08
2193 2.14181332371943e-08
2194 2.14802132696423e-08
2195 2.11852011462454e-08
2196 2.13625568963316e-08
2197 2.14412028635147e-08
2198 2.13232902046556e-08
2199 2.14582782052108e-08
2200 2.10801869179322e-08
2201 2.12271797008778e-08
2202 2.1172506389e-08
2203 2.1164087123493e-08
2204 2.12852432903698e-08
2205 2.12581468902462e-08
2206 2.13653359810717e-08
2207 2.14220549655808e-08
2208 2.13889305228365e-08
2209 2.12091392998159e-08
2210 2.13563847586418e-08
2211 2.13417340597655e-08
2212 2.12215616283151e-08
2213 2.13934315456132e-08
2214 2.12461703489319e-08
2215 2.13292426424205e-08
2216 2.12281911653456e-08
2217 2.1160614505078e-08
2218 2.1168752263212e-08
2219 2.11528177059828e-08
2220 2.12384481477201e-08
2221 2.13018055978953e-08
2222 2.12341968051089e-08
2223 2.12572202079597e-08
2224 2.12806644795949e-08
2225 2.12522543742155e-08
2226 2.12018886663889e-08
2227 2.12699114876536e-08
2228 2.12809096652444e-08
2229 2.12160409599482e-08
2230 2.12382754305462e-08
2231 2.10872997168376e-08
2232 2.11576268218661e-08
2233 2.12950292967085e-08
2234 2.09739176484369e-08
2235 2.10336007017142e-08
2236 2.11225769548573e-08
2237 2.1139167419304e-08
2238 2.11231255701216e-08
2239 2.10535993029337e-08
2240 2.10951371237833e-08
2241 2.1129782371565e-08
2242 2.12043984321175e-08
2243 2.10951547570115e-08
2244 2.11000228061975e-08
2245 2.10882332541962e-08
2246 2.117202772145e-08
2247 2.11471535518371e-08
2248 2.1092574179149e-08
2249 2.11634041810127e-08
2250 2.11000339063183e-08
2251 2.1136534211208e-08
2252 2.10285632432639e-08
2253 2.11264763433361e-08
2254 2.11802693340646e-08
2255 2.10865082933553e-08
2256 2.11543266788183e-08
2257 2.10474090024393e-08
2258 2.12177742591191e-08
2259 2.08566704936741e-08
2260 2.09046447621208e-08
2261 2.10294149225465e-08
2262 2.10674837889391e-08
2263 2.08958571750184e-08
2264 2.08130254295558e-08
2265 2.0910223634929e-08
2266 2.08151580516347e-08
2267 2.08033379087702e-08
2268 2.09013434294691e-08
2269 2.10104543767464e-08
2270 2.08735905409263e-08
2271 2.10061137976281e-08
2272 2.10972762164641e-08
2273 2.08688029331716e-08
2274 2.09600264170495e-08
2275 2.09740238381606e-08
2276 2.0987110615267e-08
2277 2.11606286778521e-08
2278 2.08706568003425e-08
2279 2.1085532518561e-08
2280 2.08570335595848e-08
2281 2.10120436595629e-08
2282 2.09659259328143e-08
2283 2.09116985528945e-08
2284 2.08864725587121e-08
2285 2.09626147125164e-08
2286 2.10259550524894e-08
2287 2.09255250339657e-08
2288 2.07928336251051e-08
2289 2.09901256332801e-08
2290 2.09469913993177e-08
2291 2.07221200410057e-08
2292 2.08148998755986e-08
2293 2.06592273332973e-08
2294 2.07531016296336e-08
2295 2.09109414007758e-08
2296 2.10334380899013e-08
2297 2.06368007275026e-08
2298 2.07938652038164e-08
2299 2.08961482254244e-08
2300 2.08115078664806e-08
2301 2.07296663464751e-08
2302 2.07556984911372e-08
2303 2.08276259773266e-08
2304 2.0726682162242e-08
2305 2.0971695392058e-08
2306 2.06151765345108e-08
2307 2.07054306639032e-08
2308 2.06932665693937e-08
2309 2.06327882512625e-08
2310 2.06063413294988e-08
2311 2.06538889658825e-08
2312 2.06381841940662e-08
2313 2.06351507905378e-08
2314 2.06717195921913e-08
2315 2.05884025911152e-08
2316 2.06161210832834e-08
2317 2.06644645253107e-08
2318 2.05741929143466e-08
2319 2.05969636023484e-08
2320 2.05881163310995e-08
2321 2.0615154925463e-08
2322 2.05561253706632e-08
2323 2.06194556782169e-08
2324 2.05372051138575e-08
2325 2.05072568201237e-08
2326 2.05895532404687e-08
2327 2.05813590542503e-08
2328 2.05296064179628e-08
2329 2.06116600934836e-08
2330 2.06226308191448e-08
2331 2.06035780403457e-08
2332 2.06260816020265e-08
2333 2.05197109662825e-08
2334 2.05845095951895e-08
2335 2.05079321319301e-08
2336 2.0459444995069e-08
2337 2.06522794589103e-08
2338 2.06797789057411e-08
2339 2.06618874686315e-08
2340 2.05935691527381e-08
2341 2.07135691394633e-08
2342 2.05363965340988e-08
2343 2.04934757049857e-08
2344 2.05014219385458e-08
2345 2.04932267506841e-08
2346 2.0496217052135e-08
2347 2.04868528596025e-08
2348 2.05015108493134e-08
2349 2.05140642022261e-08
2350 2.04469419051723e-08
2351 2.04812738077154e-08
2352 2.05247033580047e-08
2353 2.06242477367447e-08
2354 2.04919501665124e-08
2355 2.05949454945564e-08
2356 2.05778727254691e-08
2357 2.05255757628198e-08
2358 2.05481375128436e-08
2359 2.05445184771058e-08
2360 2.04938262220455e-08
2361 2.0480317213134e-08
2362 2.05674024887914e-08
2363 2.06032022170843e-08
2364 2.04524606631784e-08
2365 2.06534358551114e-08
2366 2.06098598420912e-08
2367 2.04000667671789e-08
2368 2.04509334144065e-08
2369 2.03864680202726e-08
2370 2.0580097555345e-08
2371 2.04223023977379e-08
2372 2.03981220491256e-08
2373 2.0329827756238e-08
2374 2.04625269258907e-08
2375 2.04059410090496e-08
2376 2.04363408284891e-08
2377 2.04715627223351e-08
2378 2.0600378592972e-08
2379 2.03777195829646e-08
2380 2.05672823693215e-08
2381 2.03479337721468e-08
2382 2.04780332464516e-08
2383 2.04184292345033e-08
2384 2.04257310036926e-08
2385 2.04887470862669e-08
2386 2.03643549649968e-08
2387 2.04143737407847e-08
2388 2.0503523977089e-08
2389 2.0471781257192e-08
2390 2.04757548055712e-08
2391 2.03423114847334e-08
2392 2.03322286066454e-08
2393 2.0470550051721e-08
2394 2.04978137181566e-08
2395 2.02876855485146e-08
2396 2.0362135128793e-08
2397 2.0330852184669e-08
2398 2.02183743446271e-08
2399 2.02080478010069e-08
2400 2.031739926589e-08
2401 2.02351022899894e-08
2402 2.0275359244537e-08
2403 2.02226415965123e-08
2404 2.04324344104112e-08
2405 2.04210565251106e-08
2406 2.03754234128217e-08
2407 2.03708029729022e-08
2408 2.01971183434457e-08
2409 2.02436498789593e-08
2410 2.03640171806407e-08
2411 2.03360655787188e-08
2412 2.03569023173866e-08
2413 2.02196362600882e-08
2414 2.03227364861114e-08
2415 2.02359138921082e-08
2416 2.02705380918378e-08
2417 2.01939074697677e-08
2418 2.02762721454031e-08
2419 2.0184549299529e-08
2420 2.02371267314927e-08
2421 2.0362657224271e-08
2422 2.01137922997141e-08
2423 2.01836637422348e-08
2424 2.02116505483296e-08
2425 2.01707240617965e-08
2426 2.01963052598453e-08
2427 2.01759308920968e-08
2428 2.02052236403238e-08
2429 2.01818238251272e-08
2430 2.02295466976699e-08
2431 2.01785064261051e-08
2432 2.01896810738678e-08
2433 2.02472780345131e-08
2434 2.01121689645811e-08
2435 2.0002734222424e-08
2436 2.01786693639905e-08
2437 2.00328172167419e-08
2438 2.01869232652196e-08
2439 2.0216683178309e-08
2440 2.01004754508682e-08
2441 2.01352420660106e-08
2442 2.01214227076862e-08
2443 2.02907755302295e-08
2444 2.01888083773971e-08
2445 2.00498721723008e-08
2446 2.00234829393908e-08
2447 2.0138791238189e-08
2448 2.02706976099032e-08
2449 2.04089605194691e-08
2450 2.01142564394408e-08
2451 2.029195430997e-08
2452 1.9963991836347e-08
2453 2.03506000604481e-08
2454 2.00803789399684e-08
2455 2.01001894382102e-08
2456 2.00348124724803e-08
2457 2.02524988504038e-08
2458 2.00835202320837e-08
2459 2.00353023532829e-08
2460 2.00554692064703e-08
2461 2.01441666671798e-08
2462 1.99923212486075e-08
2463 2.00399472199964e-08
2464 2.02629155997558e-08
2465 2.01303359179672e-08
2466 1.99948810540862e-08
2467 2.01417168653739e-08
2468 1.99768624996599e-08
2469 2.00871983363493e-08
2470 2.02304411879295e-08
2471 2.01176260723468e-08
2472 2.01784267128691e-08
2473 2.01431045167011e-08
2474 2.00816979675222e-08
2475 2.02951028819287e-08
2476 2.01052352437614e-08
2477 1.98918383063074e-08
2478 2.00135162782278e-08
2479 1.98948905668628e-08
2480 2.02197574836749e-08
2481 1.99154493077947e-08
2482 1.99438386245809e-08
2483 2.01730880430295e-08
2484 1.98641305158009e-08
2485 2.01405609295824e-08
2486 2.01410811637714e-08
2487 2.00415175606317e-08
2488 1.9979381183477e-08
2489 1.9987783280162e-08
2490 1.98548707497359e-08
2491 2.01047592288672e-08
2492 1.99794849227164e-08
2493 1.99254974333485e-08
2494 1.98924118007815e-08
2495 1.98990190289905e-08
2496 1.98864121758735e-08
2497 2.00365053005802e-08
2498 2.02201978563021e-08
2499 1.99230935492567e-08
2500 1.98639759154684e-08
2501 2.0116429191197e-08
2502 1.99074230300411e-08
2503 1.99468813990578e-08
2504 1.98070968032704e-08
2505 1.97637390710881e-08
2506 1.98426419213416e-08
2507 1.98550068448711e-08
2508 1.98009676448674e-08
2509 2.00077328891268e-08
2510 1.99489836163469e-08
2511 1.97831241182111e-08
2512 2.01522518141761e-08
2513 1.98300474332225e-08
2514 2.00738759782437e-08
2515 2.00511501859957e-08
2516 1.98681250616062e-08
2517 1.97931863638035e-08
2518 2.01569256216283e-08
2519 1.98471693935165e-08
2520 2.01030965149007e-08
2521 1.97443741556391e-08
2522 1.98029750214657e-08
2523 1.99416919482998e-08
2524 1.98025478019836e-08
2525 1.97772660008866e-08
2526 1.97251173595925e-08
2527 1.97369388945656e-08
2528 1.97729069794095e-08
2529 2.01044773560088e-08
2530 1.97274126529923e-08
2531 1.99918441922131e-08
2532 1.97988068263433e-08
2533 1.98821931718385e-08
2534 1.9667480643526e-08
2535 2.031386159973e-08
2536 1.96899261637462e-08
2537 1.99597941490381e-08
2538 1.96109778178855e-08
2539 1.97290397433436e-08
2540 1.98208059918326e-08
2541 2.00805892232037e-08
2542 1.98499868154789e-08
2543 2.00105734383671e-08
2544 1.97702195900984e-08
2545 1.98000046863767e-08
2546 1.9710005380924e-08
2547 2.01256096854507e-08
2548 1.95668964015905e-08
2549 1.96177060460823e-08
2550 1.98788982591624e-08
2551 1.94847468191206e-08
2552 1.98075291993893e-08
2553 1.99318373365065e-08
2554 1.97944656904481e-08
2555 1.97955231850955e-08
2556 1.98460611479101e-08
2557 1.9780052260554e-08
2558 1.99121647074252e-08
2559 1.96677850359261e-08
2560 1.96947781216927e-08
2561 1.98317713600016e-08
2562 1.98529111202772e-08
2563 1.95191643428405e-08
2564 1.96745186203939e-08
2565 1.98695926555414e-08
2566 1.98115246773378e-08
2567 1.96721748433326e-08
2568 1.95382870424865e-08
2569 1.9677693353648e-08
2570 1.94980092447228e-08
2571 1.96546210834425e-08
2572 1.98640584437859e-08
2573 1.97887652834128e-08
2574 1.96295153376269e-08
2575 1.94327399238858e-08
2576 1.99126871374133e-08
2577 1.97087433565502e-08
2578 1.98971041246443e-08
2579 1.96219722028435e-08
2580 1.96147498687793e-08
2581 1.9722284808843e-08
2582 1.97028732338289e-08
2583 1.9611014205001e-08
2584 1.97303902251811e-08
2585 1.96754453263281e-08
2586 1.95059529732866e-08
2587 1.9941555003733e-08
2588 1.98395270925023e-08
2589 1.94408586080552e-08
2590 1.9602389481066e-08
2591 1.95612145992285e-08
2592 1.99961591730302e-08
2593 1.9840745498656e-08
2594 1.94528795364191e-08
2595 1.95085220988656e-08
2596 1.94600168124115e-08
2597 1.9662396622766e-08
2598 1.96706331214713e-08
2599 1.96112935053616e-08
2600 1.94994918130131e-08
2601 1.98149331974706e-08
2602 1.94583347374877e-08
2603 1.94882284056952e-08
2604 1.95853576913096e-08
2605 1.95786500922779e-08
2606 1.94736889576053e-08
2607 1.95243578205995e-08
2608 1.98506902061579e-08
2609 1.96467150844759e-08
2610 1.93457868347124e-08
2611 1.98317331875364e-08
2612 1.93716925449561e-08
2613 1.95034596193189e-08
2614 1.95276735815986e-08
2615 1.94794768595408e-08
2616 1.97148735174846e-08
2617 1.92140377069938e-08
2618 1.94796589028101e-08
2619 1.96490333970178e-08
2620 1.92691438143378e-08
2621 1.95921375615304e-08
2622 1.94748969167824e-08
2623 1.95055772813646e-08
2624 1.94177287969888e-08
2625 1.94594220720434e-08
2626 1.94873466323742e-08
2627 1.94955271491937e-08
2628 1.93141004917186e-08
2629 1.93331402892349e-08
2630 1.9417642185604e-08
2631 1.96154035437823e-08
2632 1.93187191247501e-08
2633 1.95262257796092e-08
2634 1.96313245742763e-08
2635 1.95984239595992e-08
2636 1.9281537763649e-08
2637 1.94427069899472e-08
2638 1.95721006622795e-08
2639 1.94645499683732e-08
2640 1.98635944462788e-08
2641 1.95393808529731e-08
2642 1.96178649551904e-08
2643 1.97230264386006e-08
2644 1.93696505227381e-08
2645 1.94948698838404e-08
2646 1.94527879300299e-08
2647 1.94392154303502e-08
2648 1.95175417463389e-08
2649 1.94212183148856e-08
2650 1.94527200089167e-08
2651 1.96596269562699e-08
2652 1.96362530161531e-08
2653 1.95913431275674e-08
2654 1.94172518270808e-08
2655 1.95971774710202e-08
2656 1.95818449612029e-08
2657 1.97002802848623e-08
2658 1.91612228037652e-08
2659 1.95976423102984e-08
2660 1.9385444015918e-08
2661 1.93722257795237e-08
2662 1.93898239754242e-08
2663 1.94088984483143e-08
2664 1.93801718911812e-08
2665 1.95084829309744e-08
2666 1.99203797911496e-08
2667 1.93217113493294e-08
2668 1.94663501772441e-08
2669 1.93913242034771e-08
2670 1.97106447973328e-08
2671 1.94098401711251e-08
2672 1.95607467343706e-08
2673 1.93671782635629e-08
2674 1.97182440899724e-08
2675 1.93925520176608e-08
2676 1.97613722074941e-08
2677 1.9361786371852e-08
2678 1.93513591763006e-08
2679 1.93015722995682e-08
2680 1.93362761335969e-08
2681 1.94534594920581e-08
2682 1.92577550545137e-08
2683 1.93822403592447e-08
2684 1.94365950176856e-08
2685 1.92947821562317e-08
2686 1.93518036686324e-08
2687 1.93730154646188e-08
2688 1.94895532561468e-08
2689 1.94635375926344e-08
2690 1.93923660716244e-08
2691 1.92711306303828e-08
2692 1.93457975422584e-08
2693 1.95239340390341e-08
2694 1.95239029030514e-08
2695 1.94204709795809e-08
2696 1.94856691696499e-08
2697 1.9367611929999e-08
2698 1.91987778582758e-08
2699 1.93351531320163e-08
2700 1.92787572733666e-08
2701 1.92905271948485e-08
2702 1.92599491545264e-08
2703 1.93820182935456e-08
2704 1.92426385128286e-08
2705 1.93008339788303e-08
2706 1.93154640579785e-08
2707 1.93303733290184e-08
2708 1.91772776158006e-08
2709 1.91439244157099e-08
2710 1.93176189633437e-08
2711 1.92109818185626e-08
2712 1.93482089931862e-08
2713 1.97538734653113e-08
2714 1.91551675398083e-08
2715 1.96013351778657e-08
2716 1.90583500333519e-08
2717 1.93089893818499e-08
2718 1.96342777826208e-08
2719 1.89323362421723e-08
2720 1.92232023855654e-08
2721 1.94425546946597e-08
2722 1.93601258923204e-08
2723 1.92111067232048e-08
2724 1.92219170580543e-08
2725 1.925897887467e-08
2726 1.93617872829011e-08
2727 1.93629907182524e-08
2728 1.93403149330029e-08
2729 1.95032805174744e-08
2730 1.92460064509792e-08
2731 1.93734016018521e-08
2732 1.91572797665351e-08
2733 1.93785653784806e-08
2734 1.93335950420259e-08
2735 1.93135096496766e-08
2736 1.93945377086058e-08
2737 1.90727356554898e-08
2738 1.91765191358551e-08
2739 1.92642177417879e-08
2740 1.92926931097404e-08
2741 1.93844254731124e-08
2742 1.91837269535311e-08
2743 1.91763318555527e-08
2744 1.91741038922943e-08
2745 1.95953703682195e-08
2746 1.89900022180556e-08
2747 1.92256428301896e-08
2748 1.90725371214118e-08
2749 1.90671780514462e-08
2750 1.9182109815441e-08
2751 1.94773677261173e-08
2752 1.89800135038665e-08
2753 1.92383598304202e-08
2754 1.93933742914654e-08
2755 1.88399743731305e-08
2756 1.93308752085564e-08
2757 1.90070405222675e-08
2758 1.92221983769114e-08
2759 1.91193010259383e-08
2760 1.94185728239482e-08
2761 1.90616755005868e-08
2762 1.94191948611477e-08
2763 1.90011288000447e-08
2764 1.90874019003751e-08
2765 1.93723542970536e-08
2766 1.92774418094066e-08
2767 1.90642814713593e-08
2768 1.94742947446924e-08
2769 1.91401554128223e-08
2770 1.92342635670917e-08
2771 1.92008288948387e-08
2772 1.90408170153145e-08
2773 1.90258049416192e-08
2774 1.95572811043654e-08
2775 1.91123258175274e-08
2776 1.92629194513128e-08
2777 1.892773919554e-08
2778 1.93928725406023e-08
2779 1.92615619784009e-08
2780 1.90513135390757e-08
2781 1.93190106830832e-08
2782 1.88971522924186e-08
2783 1.91187165289319e-08
2784 1.91121344597089e-08
2785 1.89717028126202e-08
2786 1.89745632106497e-08
2787 1.93710925909807e-08
2788 1.90544762010969e-08
2789 1.90130649434517e-08
2790 1.91930197073198e-08
2791 1.9047803865635e-08
2792 1.89624619958417e-08
2793 1.93671279671293e-08
2794 1.88690631963695e-08
2795 1.8955376270946e-08
2796 1.89255254701148e-08
2797 1.89946718581746e-08
2798 1.89622590418548e-08
2799 1.91007080786454e-08
2800 1.89257048824887e-08
2801 1.91339035721105e-08
2802 1.91929317828743e-08
2803 1.89920500166529e-08
2804 1.91329871019885e-08
2805 1.90428316573232e-08
2806 1.90603527451261e-08
2807 1.91278708032705e-08
2808 1.91654561593202e-08
2809 1.90472104719674e-08
2810 1.92043887504711e-08
2811 1.89265989209986e-08
2812 1.92372984704203e-08
2813 1.90431995490403e-08
2814 1.90222512383986e-08
2815 1.93554878996638e-08
2816 1.89147675231371e-08
2817 1.93158836514495e-08
2818 1.90114390277163e-08
2819 1.92107276000275e-08
2820 1.88600181635357e-08
2821 1.90032537672469e-08
2822 1.91738980219736e-08
2823 1.89378854323907e-08
2824 1.91456972794546e-08
2825 1.91125888525656e-08
2826 1.8943358229162e-08
2827 1.90972315720073e-08
2828 1.93409627957664e-08
2829 1.88331253714313e-08
2830 1.89675313515547e-08
2831 1.91152913482373e-08
2832 1.89104252721339e-08
2833 1.89943700331652e-08
2834 1.90485567971854e-08
2835 1.90421350880765e-08
2836 1.91860594471738e-08
2837 1.88426623138893e-08
2838 1.89440394288143e-08
2839 1.91110221884339e-08
2840 1.88734961658055e-08
2841 1.89346014342062e-08
2842 1.93575458850015e-08
2843 1.88133200426099e-08
2844 1.9122822855655e-08
2845 1.90601873332197e-08
2846 1.88143649877404e-08
2847 1.9132898754215e-08
2848 1.90696981600125e-08
2849 1.89541091546452e-08
2850 1.88604528210679e-08
2851 1.90388530831864e-08
2852 1.89921621537303e-08
2853 1.89301736842618e-08
2854 1.90942875230027e-08
2855 1.88982636569746e-08
2856 1.8853128309404e-08
2857 1.90810862198409e-08
2858 1.87758884823142e-08
2859 1.91097591739631e-08
2860 1.88997156018722e-08
2861 1.87847704324895e-08
2862 1.93018671019729e-08
2863 1.86911878325979e-08
2864 1.89512180676088e-08
2865 1.85829817012095e-08
2866 1.86927553287308e-08
2867 1.8891656117459e-08
2868 1.88827057830965e-08
2869 1.86322782464998e-08
2870 1.90156694177546e-08
2871 1.92061965295087e-08
2872 1.84676743721912e-08
2873 1.92857054898132e-08
2874 1.84896172539251e-08
2875 1.8651436017425e-08
2876 1.892233423062e-08
2877 1.88877059694725e-08
2878 1.83990380013821e-08
2879 1.85617007987249e-08
2880 1.84144981161261e-08
2881 1.8501734537546e-08
2882 1.85085493329407e-08
2883 1.84572295784946e-08
2884 1.84139399830308e-08
2885 1.84508054316135e-08
2886 1.85006572042123e-08
2887 1.8427929604492e-08
2888 1.8472672536296e-08
2889 1.84872699320326e-08
2890 1.8974319512477e-08
2891 1.83574298260325e-08
2892 1.86250448702685e-08
2893 1.84301143562049e-08
2894 1.84264305699289e-08
2895 1.84268455035719e-08
2896 1.88665457224735e-08
2897 1.8433759199632e-08
2898 1.85802255520118e-08
2899 1.8513034633072e-08
2900 1.85078834070707e-08
2901 1.86539577358147e-08
2902 1.84899994279952e-08
2903 1.8385110839847e-08
2904 1.848858310316e-08
2905 1.84114802750379e-08
2906 1.85854180340117e-08
2907 1.85530174077186e-08
2908 1.83378339454476e-08
2909 1.85014383173865e-08
2910 1.83311526036345e-08
2911 1.83797876717184e-08
2912 1.8311226527179e-08
2913 1.86209947429239e-08
2914 1.83225607747195e-08
2915 1.83110141682707e-08
2916 1.83740971572144e-08
2917 1.84347042149202e-08
2918 1.83561748479111e-08
2919 1.84605564162998e-08
2920 1.83698158121581e-08
2921 1.84336107489358e-08
2922 1.85649839213875e-08
2923 1.8447798973531e-08
2924 1.84041487215625e-08
2925 1.83406019311771e-08
2926 1.85275767171644e-08
2927 1.84701738719095e-08
2928 1.82794990314372e-08
2929 1.86097726464496e-08
2930 1.83274162844915e-08
2931 1.84034931742794e-08
2932 1.83251795706774e-08
2933 1.84724898116828e-08
2934 1.82590047631548e-08
2935 1.85060122647185e-08
2936 1.82890278804804e-08
2937 1.83506722123417e-08
2938 1.83764808965314e-08
2939 1.84318369945702e-08
2940 1.87393136601433e-08
2941 1.8345688211685e-08
2942 1.83463589573662e-08
2943 1.84001402391276e-08
2944 1.8351771990055e-08
2945 1.814000765199e-08
2946 1.83244593432441e-08
2947 1.85970988825224e-08
2948 1.82890176639861e-08
2949 1.82838600037982e-08
2950 1.83119796897113e-08
2951 1.82401492166973e-08
2952 1.84692094278849e-08
2953 1.81988454216464e-08
2954 1.85410871808012e-08
2955 1.82222676197163e-08
2956 1.82099440217964e-08
2957 1.82891190534384e-08
2958 1.82805040515044e-08
2959 1.82181372719858e-08
2960 1.8318953433516e-08
2961 1.81440403163036e-08
2962 1.82307079625854e-08
2963 1.83046236710283e-08
2964 1.82550582721319e-08
2965 1.84131389892039e-08
2966 1.8260258351277e-08
2967 1.83388870746981e-08
2968 1.88001352662281e-08
2969 1.8132798062509e-08
2970 1.85301189031817e-08
2971 1.85792675644336e-08
2972 1.83861133562502e-08
2973 1.86738113255869e-08
2974 1.80652150258975e-08
2975 1.82557658283122e-08
2976 1.87464302111584e-08
2977 1.84426829653583e-08
2978 1.84145774809741e-08
2979 1.81498273423841e-08
2980 1.84282964197369e-08
2981 1.82323520145911e-08
2982 1.84691356722722e-08
2983 1.82244655258512e-08
2984 1.82745498099601e-08
2985 1.83845566789076e-08
2986 1.85025380617931e-08
2987 1.84278381273328e-08
2988 1.82181625606459e-08
2989 1.87531027311394e-08
2990 1.81906914825669e-08
2991 1.82348392098186e-08
2992 1.82680686163783e-08
2993 1.86436779943122e-08
2994 1.82476030234913e-08
2995 1.84541872171318e-08
2996 1.85295915998696e-08
2997 1.83577590190942e-08
2998 1.82692405418239e-08
2999 1.86304676405014e-08
3000 1.0183755847229e-08
3001 1.01726911010194e-08
3002 1.0230697707482e-08
3003 1.02618209026836e-08
3004 1.0266539903013e-08
3005 1.02640824490088e-08
3006 1.0259656093338e-08
3007 1.02527672718386e-08
3008 1.02476989568071e-08
3009 1.02430859649366e-08
3010 1.02388586855143e-08
3011 1.02341058827796e-08
3012 1.02320800420691e-08
3013 1.02278864036326e-08
3014 1.02242734839955e-08
3015 1.02207726591988e-08
3016 1.0217371266108e-08
3017 1.02143516927389e-08
3018 1.02117037641195e-08
3019 1.02085433555665e-08
3020 1.02060976260449e-08
3021 1.02030850899631e-08
3022 1.02010547333165e-08
3023 1.01978682762183e-08
3024 1.01960586994937e-08
3025 1.01935609653908e-08
3026 1.01913863562036e-08
3027 1.01890765827894e-08
3028 1.01867231816349e-08
3029 1.01847083139317e-08
3030 1.01826615753475e-08
3031 1.01805763978585e-08
3032 1.01785806573551e-08
3033 1.01766529245761e-08
3034 1.01746332322322e-08
3035 1.01728325044986e-08
3036 1.01708010256663e-08
3037 1.01711360788181e-08
3038 1.01673306505401e-08
3039 1.01683459576216e-08
3040 1.0163766066211e-08
3041 1.01635830691074e-08
3042 1.01613579101198e-08
3043 1.01605445940248e-08
3044 1.01574916658992e-08
3045 1.01564121090178e-08
3046 1.01546408394007e-08
3047 1.01511555489139e-08
3048 1.01516966986917e-08
3049 1.01497085213884e-08
3050 1.01494805212071e-08
3051 1.01475696099446e-08
3052 1.01458644918839e-08
3053 1.0144314030909e-08
3054 1.01429400518327e-08
3055 1.0141434306693e-08
3056 1.01401947464122e-08
3057 1.01388327741619e-08
3058 1.01366200015718e-08
3059 1.01354741069287e-08
3060 1.01337815548233e-08
3061 1.01322585981459e-08
3062 1.01311428050699e-08
3063 1.01288836924854e-08
3064 1.01269948450458e-08
3065 1.01255272606376e-08
3066 1.01234745900694e-08
3067 1.01219225162388e-08
3068 1.0120024780437e-08
3069 1.01183219435447e-08
3070 1.01171107112427e-08
3071 1.01149853137458e-08
3072 1.01134209625695e-08
3073 1.01125715332648e-08
3074 1.01113542742143e-08
3075 1.01092170705155e-08
3076 1.01075335880094e-08
3077 1.01061942758551e-08
3078 1.01051258710252e-08
3079 1.01040873949215e-08
3080 1.01020869835849e-08
3081 1.01000955112576e-08
3082 1.01013604088332e-08
3083 1.0097404622704e-08
3084 1.00981518744644e-08
3085 1.0095714809373e-08
3086 1.00920167019972e-08
3087 1.00914926150011e-08
3088 1.00898233123259e-08
3089 1.00885394319489e-08
3090 1.00870840227738e-08
3091 1.00858572851317e-08
3092 1.00843827018082e-08
3093 1.00830845453856e-08
3094 1.0083208929465e-08
3095 1.0080166728807e-08
3096 1.00795090936245e-08
3097 1.0078241881914e-08
3098 1.00764023276273e-08
3099 1.00765281240214e-08
3100 1.00731999680623e-08
3101 1.00763591174557e-08
3102 1.00728387193577e-08
3103 1.00700430551923e-08
3104 1.00684051298691e-08
3105 1.00673582798805e-08
3106 1.00660616570228e-08
3107 1.00647204094306e-08
3108 1.00637779162638e-08
3109 1.0062609632086e-08
3110 1.00605996120398e-08
3111 1.00600496770534e-08
3112 1.00605571296056e-08
3113 1.00570224050509e-08
3114 1.00560388125309e-08
3115 1.00545521720258e-08
3116 1.0055043240316e-08
3117 1.0053488706023e-08
3118 1.00530580059571e-08
3119 1.00514519529721e-08
3120 1.00501312103804e-08
3121 1.00487080858455e-08
3122 1.00496216680623e-08
3123 1.00453225353805e-08
3124 1.00449323479981e-08
3125 1.00429659624576e-08
3126 1.00424630365761e-08
3127 1.00404339474849e-08
3128 1.00399395021669e-08
3129 1.0038406294835e-08
3130 1.00363637589376e-08
3131 1.00370914812714e-08
3132 1.00346640065085e-08
3133 1.0033312091734e-08
3134 1.00315206443546e-08
3135 1.00312006529601e-08
3136 1.00297656126999e-08
3137 1.00275035174943e-08
3138 1.00283493860709e-08
3139 1.00260288620133e-08
3140 1.00245837984445e-08
3141 1.00227176616924e-08
3142 1.00225072408588e-08
3143 1.00203647376979e-08
3144 1.00211211736811e-08
3145 1.00188086949726e-08
3146 1.00175916213849e-08
3147 1.00163231118514e-08
3148 1.00117440505129e-08
3149 1.00110039239071e-08
3150 1.00111639788833e-08
3151 1.00106349508397e-08
3152 1.00072952973831e-08
3153 1.00058542157194e-08
3154 1.00053844903567e-08
3155 1.00045406365296e-08
3156 1.00036641217383e-08
3157 1.00017050142787e-08
3158 1.00002610662758e-08
3159 9.99916622074215e-09
3160 9.99798769844851e-09
3161 9.99848239587042e-09
3162 9.9966254431122e-09
3163 9.99234604816962e-09
3164 9.99401607949813e-09
3165 9.9910037933762e-09
3166 9.98966586625294e-09
3167 9.98830905248049e-09
3168 9.98827734073615e-09
3169 9.98673531703992e-09
3170 9.98581031465273e-09
3171 9.98576396446515e-09
3172 9.98397545675517e-09
3173 9.98196081867381e-09
3174 9.98127473567784e-09
3175 9.98038514359728e-09
3176 9.97894320237114e-09
3177 9.97781027623629e-09
3178 9.97672838460445e-09
3179 9.97545769441249e-09
3180 9.97416782069771e-09
3181 9.97345896987245e-09
3182 9.97285542947424e-09
3183 9.97104699855533e-09
3184 9.96992693452076e-09
3185 9.96996842376335e-09
3186 9.96782901213633e-09
3187 9.96510460246625e-09
3188 9.96491037084934e-09
3189 9.96434374616467e-09
3190 9.9631540398451e-09
3191 9.96172555225994e-09
3192 9.96077822869312e-09
3193 9.96008053535252e-09
3194 9.95833187620448e-09
3195 9.95668546159567e-09
3196 9.9568587933474e-09
3197 9.95364111865127e-09
3198 9.95358304035215e-09
3199 9.95194602403798e-09
3200 9.95146242863815e-09
3201 9.94956355154786e-09
3202 9.94953865892517e-09
3203 9.94667355120466e-09
3204 9.94622412039958e-09
3205 9.94607270250053e-09
3206 9.94328295226415e-09
3207 9.94281803200908e-09
3208 9.94329687051959e-09
3209 9.94098260927861e-09
3210 9.94079897981281e-09
3211 9.93800811222023e-09
3212 9.93683504572906e-09
3213 9.93558638370112e-09
3214 9.93350998490516e-09
3215 9.93300659540147e-09
3216 9.93210090669688e-09
3217 9.92989005477174e-09
3218 9.92974143180364e-09
3219 9.92746595646077e-09
3220 9.92704014197293e-09
3221 9.92569717716413e-09
3222 9.92447490012066e-09
3223 9.92269744883728e-09
3224 9.92288236694588e-09
3225 9.92040738236055e-09
3226 9.91856814016451e-09
3227 9.91691510004999e-09
3228 9.91850115077825e-09
3229 9.91209485187022e-09
3230 9.91222777737416e-09
3231 9.91175317780618e-09
3232 9.91024417138636e-09
3233 9.90724444280278e-09
3234 9.9069382401562e-09
3235 9.90813919971023e-09
3236 9.90558023911819e-09
3237 9.90181518314981e-09
3238 9.90291286654538e-09
3239 9.89916622108622e-09
3240 9.89870010967286e-09
3241 9.89740676753115e-09
3242 9.8959562257056e-09
3243 9.89477093885083e-09
3244 9.89341344716233e-09
3245 9.89287681624207e-09
3246 9.89052082657482e-09
3247 9.89093374893779e-09
3248 9.88948523797384e-09
3249 9.88912169792855e-09
3250 9.88748968937247e-09
3251 9.88646278837768e-09
3252 9.8857625408233e-09
3253 9.88327658722754e-09
3254 9.88228406427483e-09
3255 9.88165090388959e-09
3256 9.87998902078624e-09
3257 9.87744243832744e-09
3258 9.87682260986111e-09
3259 9.87602949548139e-09
3260 9.87547265575428e-09
3261 9.87571046943969e-09
3262 9.87140737510051e-09
3263 9.87016277421982e-09
3264 9.86984525343237e-09
3265 9.86855463393221e-09
3266 9.86656999438229e-09
3267 9.86570419003741e-09
3268 9.86366934067895e-09
3269 9.86273209910293e-09
3270 9.86150568071942e-09
3271 9.86020287191414e-09
3272 9.85885657744934e-09
3273 9.85516356524774e-09
3274 9.85468398072287e-09
3275 9.85199285347127e-09
3276 9.85044360607423e-09
3277 9.84928213688951e-09
3278 9.84870993969711e-09
3279 9.84736231017524e-09
3280 9.84611261296842e-09
3281 9.8452046523867e-09
3282 9.84304427734117e-09
3283 9.84230906143913e-09
3284 9.84075670533435e-09
3285 9.83987296350464e-09
3286 9.83855196630412e-09
3287 9.83896539077933e-09
3288 9.83673404781477e-09
3289 9.83486305114467e-09
3290 9.83344731605995e-09
3291 9.83446786371328e-09
3292 9.83200170506099e-09
3293 9.82967082238662e-09
3294 9.82962388440534e-09
3295 9.82878272960674e-09
3296 9.82745026954779e-09
3297 9.82480899786031e-09
3298 9.82518745575728e-09
3299 9.82249871805951e-09
3300 9.82283476070461e-09
3301 9.82013006811666e-09
3302 9.82047597980484e-09
3303 9.81773934988217e-09
3304 9.81813144166482e-09
3305 9.81532072419355e-09
3306 9.81593162176914e-09
3307 9.81327388077657e-09
3308 9.8119764410376e-09
3309 9.81251791619397e-09
3310 9.80984757649461e-09
3311 9.8086114971585e-09
3312 9.80742606196405e-09
3313 9.8059762228403e-09
3314 9.80586720839005e-09
3315 9.80595438344195e-09
3316 9.80313152337819e-09
3317 9.80357052528941e-09
3318 9.80086715232642e-09
3319 9.80140024370302e-09
3320 9.79823490054876e-09
3321 9.79777549556249e-09
3322 9.79817366247165e-09
3323 9.79505094556021e-09
3324 9.79466267392315e-09
3325 9.79327808327268e-09
3326 9.79379598819302e-09
3327 9.79097911864291e-09
3328 9.79142596043564e-09
3329 9.79064353906489e-09
3330 9.78713628485561e-09
3331 9.78671890539129e-09
3332 9.78557608490027e-09
3333 9.78445816536555e-09
3334 9.78336107898547e-09
3335 9.78230301838801e-09
3336 9.7812058619251e-09
3337 9.78013785311871e-09
3338 9.77911394980935e-09
3339 9.77797792843516e-09
3340 9.77822030619191e-09
3341 9.77527450923832e-09
3342 9.7741534672291e-09
3343 9.77382355046852e-09
3344 9.77264095773644e-09
3345 9.77185121568691e-09
3346 9.77013254267745e-09
3347 9.77095298169972e-09
3348 9.76928718993825e-09
3349 9.76809881442914e-09
3350 9.76703678682439e-09
3351 9.76579122768939e-09
3352 9.76470523333939e-09
3353 9.76382985069107e-09
3354 9.75990775598651e-09
3355 9.75870870471851e-09
3356 9.76053810750055e-09
3357 9.75662558069801e-09
3358 9.75798234835457e-09
3359 9.7537497283362e-09
3360 9.75236579359162e-09
3361 9.75121186902461e-09
3362 9.75029282893058e-09
3363 9.75033656892832e-09
3364 9.74934858839066e-09
3365 9.74693879503274e-09
3366 9.74611256678026e-09
3367 9.74524126379217e-09
3368 9.74478852022731e-09
3369 9.74383506151949e-09
3370 9.74309815474084e-09
3371 9.74093441154233e-09
3372 9.7403716097233e-09
3373 9.73873839638095e-09
3374 9.73753833626023e-09
3375 9.73834288148928e-09
3376 9.73599251789803e-09
3377 9.73380382657263e-09
3378 9.73297368135617e-09
3379 9.73207854164382e-09
3380 9.73184569939833e-09
3381 9.73171484259378e-09
3382 9.72845575117343e-09
3383 9.72782159841534e-09
3384 9.72751162192642e-09
3385 9.72744308559009e-09
3386 9.72647949355754e-09
3387 9.72405640339846e-09
3388 9.72320857750564e-09
3389 9.72207584230833e-09
3390 9.72001768304032e-09
3391 9.72027123492814e-09
3392 9.72003655480558e-09
3393 9.71905592066918e-09
3394 9.71619482206809e-09
3395 9.71599195852269e-09
3396 9.71464966233182e-09
3397 9.71280202416436e-09
3398 9.71297872819077e-09
3399 9.71240582698513e-09
3400 9.71137200792699e-09
3401 9.70941466760244e-09
3402 9.70874864610061e-09
3403 9.70659916200989e-09
3404 9.70672462158317e-09
3405 9.70641850690096e-09
3406 9.70450347888285e-09
3407 9.70362493885046e-09
3408 9.70257285157205e-09
3409 9.70144878317814e-09
3410 9.70012447013002e-09
3411 9.6976953068012e-09
3412 9.69916605856464e-09
3413 9.69696388827823e-09
3414 9.69439623604851e-09
3415 9.69443313159069e-09
3416 9.69402558505944e-09
3417 9.69220442659929e-09
3418 9.69176162675389e-09
3419 9.68983962446046e-09
3420 9.68769851983353e-09
3421 9.6878636689085e-09
3422 9.68669180819587e-09
3423 9.6859532258825e-09
3424 9.68348101674371e-09
3425 9.68324364079148e-09
3426 9.68234225686565e-09
3427 9.68108933106782e-09
3428 9.68004162434088e-09
3429 9.67895410185216e-09
3430 9.67813213936797e-09
3431 9.67728721432831e-09
3432 9.67448721955755e-09
3433 9.67565362418321e-09
3434 9.67361533590894e-09
3435 9.67282532406133e-09
3436 9.67202732173739e-09
3437 9.67125329242452e-09
3438 9.66939425325192e-09
3439 9.66868869898413e-09
3440 9.66756461692059e-09
3441 9.66700685650634e-09
3442 9.66582765067614e-09
3443 9.66479644276541e-09
3444 9.66279738574133e-09
3445 9.6608550263777e-09
3446 9.65990834966846e-09
3447 9.65920030028544e-09
3448 9.65727840976371e-09
3449 9.65877123195596e-09
3450 9.65669306419309e-09
3451 9.65609793145139e-09
3452 9.6546606420389e-09
3453 9.65357124600719e-09
3454 9.6528511239341e-09
3455 9.65085773614593e-09
3456 9.65033969835966e-09
3457 9.64937319434395e-09
3458 9.64868327805318e-09
3459 9.64679235830601e-09
3460 9.64629252712956e-09
3461 9.64498196764224e-09
3462 9.64369311570734e-09
3463 9.64329961206101e-09
3464 9.6422494557033e-09
3465 9.64071989988796e-09
3466 9.63989729314829e-09
3467 9.63927910118539e-09
3468 9.6379496044019e-09
3469 9.63719070102426e-09
3470 9.63608397819976e-09
3471 9.6348002194685e-09
3472 9.63462009857807e-09
3473 9.6332013657871e-09
3474 9.63291021307405e-09
3475 9.63166019744138e-09
3476 9.63025722878347e-09
3477 9.62861387975716e-09
3478 9.62834983129351e-09
3479 9.62623569182586e-09
3480 9.62551543755991e-09
3481 9.62534175603241e-09
3482 9.62316000679281e-09
3483 9.62265167995385e-09
3484 9.62245384876492e-09
3485 9.62115807075919e-09
3486 9.62028466453685e-09
3487 9.61914578993805e-09
3488 9.61827461302966e-09
3489 9.61718753873103e-09
3490 9.61622451763761e-09
3491 9.61515357007087e-09
3492 9.61458017050693e-09
3493 9.6130732680777e-09
3494 9.61160096047547e-09
3495 9.61175048738022e-09
3496 9.61028739485809e-09
3497 9.60875912542347e-09
3498 9.60820359706405e-09
3499 9.6065331208553e-09
3500 9.60589164634984e-09
3501 9.60515879958046e-09
3502 9.60498965558998e-09
3503 9.6038490415562e-09
3504 9.60300893324689e-09
3505 9.60217312639261e-09
3506 9.60099843631129e-09
3507 9.6002174317239e-09
3508 9.59939410711402e-09
3509 9.59796903076543e-09
3510 9.59744328918322e-09
3511 9.59600794849663e-09
3512 9.59540453325525e-09
3513 9.59442092451218e-09
3514 9.59308608679266e-09
3515 9.59217619197345e-09
3516 9.58890048623207e-09
3517 9.59035165440381e-09
3518 9.58943520135841e-09
3519 9.58826998789797e-09
3520 9.58741843446603e-09
3521 9.58628914070336e-09
3522 9.58532808036094e-09
3523 9.58439512504378e-09
3524 9.58618651156207e-09
3525 9.58283935151988e-09
3526 9.58207031440811e-09
3527 9.5809763434665e-09
3528 9.57983609217727e-09
3529 9.57903571522506e-09
3530 9.57905724056213e-09
3531 9.57700370878001e-09
3532 9.575365577566e-09
3533 9.57412940263275e-09
3534 9.57504245469726e-09
3535 9.57201884850545e-09
3536 9.57175225616003e-09
3537 9.56991727311951e-09
3538 9.57026058128613e-09
3539 9.56875547140218e-09
3540 9.56631878613889e-09
3541 9.56476667066802e-09
3542 9.56542154294249e-09
3543 9.56245520369881e-09
3544 9.56306214242031e-09
3545 9.56034355020785e-09
3546 9.56124566352939e-09
3547 9.55839880992238e-09
3548 9.55905746088947e-09
3549 9.55835862238852e-09
3550 9.55472847642691e-09
3551 9.55344445698753e-09
3552 9.55448329292841e-09
3553 9.55350173359459e-09
3554 9.55240012556063e-09
3555 9.55077574722046e-09
3556 9.55017288791632e-09
3557 9.54847701390565e-09
3558 9.54811160614605e-09
3559 9.54642743448281e-09
3560 9.54600354242246e-09
3561 9.54422583906989e-09
3562 9.54389836020642e-09
3563 9.54229347525232e-09
3564 9.5418156573629e-09
3565 9.54012656548958e-09
3566 9.53975545412572e-09
3567 9.53800323266751e-09
3568 9.5377014944914e-09
3569 9.53602963144756e-09
3570 9.53548324524739e-09
3571 9.53400775106222e-09
3572 9.53339670340531e-09
3573 9.53180830674805e-09
3574 9.53151744532282e-09
3575 9.52985838494763e-09
3576 9.52936445688946e-09
3577 9.52816618398494e-09
3578 9.52724943607125e-09
3579 9.52463857450958e-09
3580 9.52395228473457e-09
3581 9.52448439097559e-09
3582 9.52274060039143e-09
3583 9.5215672500093e-09
3584 9.52061832128898e-09
3585 9.51900836152625e-09
3586 9.51898933179707e-09
3587 9.51593105790599e-09
3588 9.51673521691621e-09
3589 9.51591042683969e-09
3590 9.51459732174281e-09
3591 9.5138501368594e-09
3592 9.51307213502978e-09
3593 9.51065825580133e-09
3594 9.51135838417327e-09
3595 9.50867501817165e-09
3596 9.50925448564566e-09
3597 9.50791747744706e-09
3598 9.50646417902795e-09
3599 9.5055426466914e-09
3600 9.50463423240316e-09
3601 9.50412325356059e-09
3602 9.50332953157662e-09
3603 9.50230065507179e-09
3604 9.49966170170347e-09
3605 9.50015459837833e-09
3606 9.49864561408664e-09
3607 9.49818236040817e-09
3608 9.49709579447372e-09
3609 9.496197977861e-09
3610 9.49520575828366e-09
3611 9.49487867923926e-09
3612 9.49322274405168e-09
3613 9.49289312320023e-09
3614 9.49127674085282e-09
3615 9.49104767495462e-09
3616 9.48718275228105e-09
3617 9.48723174612476e-09
3618 9.48760118690828e-09
3619 9.48440154623781e-09
3620 9.48546849803555e-09
3621 9.48256656621571e-09
3622 9.4813064276944e-09
3623 9.48289061215857e-09
3624 9.48171822474225e-09
3625 9.48045478564585e-09
3626 9.47731056424178e-09
3627 9.47732957086445e-09
3628 9.47799394530879e-09
3629 9.47681193039923e-09
3630 9.47582767103078e-09
3631 9.47622086956701e-09
3632 9.47406330361472e-09
3633 9.47228803062361e-09
3634 9.47134892916729e-09
3635 9.47015052944755e-09
3636 9.46972958542486e-09
3637 9.47190187089442e-09
3638 9.47084307627993e-09
3639 9.46985948884621e-09
3640 9.46878137231266e-09
3641 9.46618008312861e-09
3642 9.46700710709181e-09
3643 9.46345331724607e-09
3644 9.46505686642146e-09
3645 9.46172215329777e-09
3646 9.46295977471978e-09
3647 9.46017034307578e-09
3648 9.46134951871486e-09
3649 9.45770096312526e-09
3650 9.45947605510145e-09
3651 9.45820835416417e-09
3652 9.45568473382369e-09
3653 9.4560582861794e-09
3654 9.45383885454787e-09
3655 9.45271303856576e-09
3656 9.45408811421633e-09
3657 9.45292193475777e-09
3658 9.45219801629787e-09
3659 9.44868584037972e-09
3660 9.4476384950512e-09
3661 9.44928766507397e-09
3662 9.44759091419528e-09
3663 9.44685939504447e-09
3664 9.44384091531442e-09
3665 9.44282497006982e-09
3666 9.44395673103254e-09
3667 9.44327543708073e-09
3668 9.44017385322293e-09
3669 9.43902729681062e-09
3670 9.4401093898161e-09
3671 9.43940724138642e-09
3672 9.43849682943226e-09
3673 9.43528731502802e-09
3674 9.43397009244701e-09
3675 9.43559440268887e-09
3676 9.4325351643193e-09
3677 9.43115299597425e-09
3678 9.43261400472573e-09
3679 9.43198883280455e-09
3680 9.4305169023659e-09
3681 9.42749123856307e-09
3682 9.42855129513337e-09
3683 9.42825191653507e-09
3684 9.42718380884944e-09
3685 9.42488426671634e-09
3686 9.42515741896716e-09
3687 9.42220036350555e-09
3688 9.42105275061195e-09
3689 9.42012030167444e-09
3690 9.41907677778908e-09
3691 9.42038960674574e-09
3692 9.41676395681962e-09
3693 9.41844811153114e-09
3694 9.41584150648128e-09
3695 9.4144680510555e-09
3696 9.41370308008005e-09
3697 9.41281674644162e-09
3698 9.41364260947236e-09
3699 9.41059553193763e-09
3700 9.4099708724607e-09
3701 9.40892953518041e-09
3702 9.40830409411342e-09
3703 9.4078060308006e-09
3704 9.41355926097065e-09
3705 9.40447669878219e-09
3706 9.41101793754812e-09
3707 9.40970162967292e-09
3708 9.40853894586952e-09
3709 9.40771955511982e-09
3710 9.40568459487784e-09
3711 9.40484254623175e-09
3712 9.4038858153081e-09
3713 9.40309644925641e-09
3714 9.40260112248376e-09
3715 9.40172233635655e-09
3716 9.40052091389504e-09
3717 9.39849005072296e-09
3718 9.39795896667123e-09
3719 9.39664256854183e-09
3720 9.39695866708501e-09
3721 9.3937019998297e-09
3722 9.39536822752024e-09
3723 9.39210432556597e-09
3724 9.39340174139352e-09
3725 9.3922965097859e-09
3726 9.3919315012625e-09
3727 9.38901746646431e-09
3728 9.38936641674271e-09
3729 9.38801129163336e-09
3730 9.38523476853165e-09
3731 9.38673050727284e-09
3732 9.38575450278728e-09
3733 9.38300927241786e-09
3734 9.38271244533412e-09
3735 9.3838124858997e-09
3736 9.38187385582379e-09
3737 9.38046711943669e-09
3738 9.38097071166016e-09
3739 9.37881134382978e-09
3740 9.37785227013432e-09
3741 9.37710640128153e-09
3742 9.37489568270805e-09
3743 9.37320963292215e-09
3744 9.3730471939571e-09
3745 9.37168252228976e-09
3746 9.37078086818421e-09
3747 9.36960992720115e-09
3748 9.36843695155398e-09
3749 9.36770583011282e-09
3750 9.36629055265509e-09
3751 9.36564044337618e-09
3752 9.36423732685043e-09
3753 9.36325042774633e-09
3754 9.36224554905218e-09
3755 9.36131332369278e-09
3756 9.36347395217446e-09
3757 9.36077192941614e-09
3758 9.36097938795261e-09
3759 9.35670260201332e-09
3760 9.35907092704247e-09
3761 9.35683396351139e-09
3762 9.35695609914633e-09
3763 9.352711323872e-09
3764 9.35528065054009e-09
3765 9.35271323330988e-09
3766 9.35293241718926e-09
3767 9.34876150437275e-09
3768 9.35115111889984e-09
3769 9.34852245391971e-09
3770 9.34914498153944e-09
3771 9.3456457223029e-09
3772 9.34448692534173e-09
3773 9.34611374948241e-09
3774 9.34481965950834e-09
3775 9.34357026294297e-09
3776 9.33973603376576e-09
3777 9.33918477438039e-09
3778 9.33913964958066e-09
3779 9.33789902237553e-09
3780 9.33654228904568e-09
3781 9.33596637533313e-09
3782 9.33474774278409e-09
3783 9.33401291778463e-09
3784 9.33273534862306e-09
3785 9.33175243543471e-09
3786 9.3330136080455e-09
3787 9.33231977207072e-09
3788 9.33114823593872e-09
3789 9.33002823632484e-09
3790 9.32629892255038e-09
3791 9.32862976428528e-09
3792 9.32731751163846e-09
3793 9.32624602553028e-09
3794 9.32520010181459e-09
3795 9.32154997791701e-09
3796 9.3236305426514e-09
3797 9.32280692333282e-09
3798 9.32167441325404e-09
3799 9.32034887886346e-09
3800 9.31864494190665e-09
3801 9.31910622858218e-09
3802 9.31782272135112e-09
3803 9.31654556145245e-09
3804 9.31624600739728e-09
3805 9.31497841138995e-09
3806 9.31401861994918e-09
3807 9.3107176356716e-09
3808 9.30991476674747e-09
3809 9.30890960963715e-09
3810 9.3082606497305e-09
3811 9.30720653452566e-09
3812 9.30623232076044e-09
3813 9.30543115949117e-09
3814 9.30427534237554e-09
3815 9.30349019468563e-09
3816 9.30235583859046e-09
3817 9.30154953948659e-09
3818 9.30053969044886e-09
3819 9.29953310908893e-09
3820 9.29868847969473e-09
3821 9.2978391492346e-09
3822 9.29693031614942e-09
3823 9.29542712781867e-09
3824 9.29484828337407e-09
3825 9.29408997572823e-09
3826 9.29315948581394e-09
3827 9.29156790882568e-09
3828 9.29033371449561e-09
3829 9.29023850349681e-09
3830 9.28935996923064e-09
3831 9.2884743119262e-09
3832 9.28751025086605e-09
3833 9.28678654427834e-09
3834 9.28554106212343e-09
3835 9.28422640209675e-09
3836 9.28436634808211e-09
3837 9.28219019052934e-09
3838 9.28244145206175e-09
3839 9.27919078451411e-09
3840 9.27970989132026e-09
3841 9.27890793827463e-09
3842 9.2766537294689e-09
3843 9.27724922087508e-09
3844 9.27529069474214e-09
3845 9.27611154997315e-09
3846 9.27344829004112e-09
3847 9.2743803653747e-09
3848 9.27183115236041e-09
3849 9.27150857073927e-09
3850 9.26945885124791e-09
3851 9.27064010364936e-09
3852 9.26805935321656e-09
3853 9.26771462265302e-09
3854 9.26589007664569e-09
3855 9.26563469744873e-09
3856 9.26345821419122e-09
3857 9.26409055188732e-09
3858 9.2623380389123e-09
3859 9.26224573678108e-09
3860 9.26062383242027e-09
3861 9.26037671846597e-09
3862 9.25874612842797e-09
3863 9.25848180956257e-09
3864 9.25689646157774e-09
3865 9.25665944009196e-09
3866 9.25467077205616e-09
3867 9.2538597391903e-09
3868 9.25432707373636e-09
3869 9.25340015998227e-09
3870 9.250978752047e-09
3871 9.25005384825456e-09
3872 9.25071682166684e-09
3873 9.24843214807763e-09
3874 9.24835022286796e-09
3875 9.24694798472292e-09
3876 9.24675125239111e-09
3877 9.24490721908311e-09
3878 9.24550124548756e-09
3879 9.24285150340209e-09
3880 9.24296664519403e-09
3881 9.24159204052088e-09
3882 9.24159842133343e-09
3883 9.23952014845325e-09
3884 9.23919267124818e-09
3885 9.23941102216191e-09
3886 9.23703804057985e-09
3887 9.23557393220364e-09
3888 9.235709609548e-09
3889 9.23621120404311e-09
3890 9.23491485933792e-09
3891 9.23401631810344e-09
3892 9.23295798703483e-09
3893 9.23211457089929e-09
3894 9.23019901734323e-09
3895 9.22920046487336e-09
3896 9.22833077468543e-09
3897 9.22738696172015e-09
3898 9.2264615183385e-09
3899 9.22550451194076e-09
3900 9.22469979191404e-09
3901 9.22372221174445e-09
3902 9.22282512084288e-09
3903 9.22189302796084e-09
3904 9.22104452628913e-09
3905 9.22011307876586e-09
3906 9.21926747002311e-09
3907 9.2183484339467e-09
3908 9.21737185800159e-09
3909 9.21655952625111e-09
3910 9.21345839053483e-09
3911 9.2137249791055e-09
3912 9.21110858457391e-09
3913 9.21178339818862e-09
3914 9.20863694371665e-09
3915 9.21000020762813e-09
3916 9.20734693916209e-09
3917 9.20829156139669e-09
3918 9.20571072536569e-09
3919 9.20621512346537e-09
3920 9.20464990001452e-09
3921 9.20314572645103e-09
3922 9.20390079420241e-09
3923 9.20088088894599e-09
3924 9.20076138204856e-09
3925 9.19953258202849e-09
3926 9.20023935360725e-09
3927 9.19756162221647e-09
3928 9.19821414640393e-09
3929 9.1959209020942e-09
3930 9.19508718794787e-09
3931 9.19530210230501e-09
3932 9.19317842199702e-09
3933 9.19343640743558e-09
3934 9.19130117212064e-09
3935 9.19088120439338e-09
3936 9.18861025438894e-09
3937 9.18924202336635e-09
3938 9.18693107060386e-09
3939 9.18747840991663e-09
3940 9.18493046781627e-09
3941 9.18569139916892e-09
3942 9.18342522097243e-09
3943 9.18335456991798e-09
3944 9.18363013755613e-09
3945 9.18158049557222e-09
3946 9.18153372179986e-09
3947 9.17970583096811e-09
3948 9.17807064197001e-09
3949 9.17819403509906e-09
3950 9.1772249532357e-09
3951 9.17622590723005e-09
3952 9.17574334855648e-09
3953 9.17458745419708e-09
3954 9.17281164954792e-09
3955 9.17271410851711e-09
3956 9.17192562036734e-09
3957 9.17096812697027e-09
3958 9.17055094228764e-09
3959 9.16940091268831e-09
3960 9.16769203583578e-09
3961 9.16663803566392e-09
3962 9.16679973846091e-09
3963 9.16627599121395e-09
3964 9.16540266725219e-09
3965 9.164453983114e-09
3966 9.16358997935657e-09
3967 9.16278582885344e-09
3968 9.16194471633353e-09
3969 9.16116899073866e-09
3970 9.16015201552628e-09
3971 9.15935262704032e-09
3972 9.15852253333821e-09
3973 9.1577283840294e-09
3974 9.15676082887534e-09
3975 9.15607593018203e-09
3976 9.15511321351176e-09
3977 9.1543433906327e-09
3978 9.15338936147147e-09
3979 9.1526826902083e-09
3980 9.15177007049239e-09
3981 9.15102539511548e-09
3982 9.15076930920605e-09
3983 9.14925161277225e-09
3984 9.14902001256796e-09
3985 9.14824197847247e-09
3986 9.14673756698819e-09
3987 9.14656443682521e-09
3988 9.14571591107555e-09
3989 9.14487832948813e-09
3990 9.14406529697182e-09
3991 9.1433027423482e-09
3992 9.14248263924472e-09
3993 9.14086070674669e-09
3994 9.14044824577853e-09
3995 9.13986452723975e-09
3996 9.13843540375353e-09
3997 9.13662400351539e-09
3998 9.13572013814395e-09
3999 9.13603450474615e-09
4000 9.13402748885933e-09
4001 9.13314350638877e-09
4002 9.13347349328075e-09
4003 9.13157438580531e-09
4004 9.13075555233206e-09
4005 9.12983140779339e-09
4006 9.13030176639962e-09
4007 9.12906826771448e-09
4008 9.12736293005867e-09
4009 9.12629514450425e-09
4010 9.12644407915358e-09
4011 9.12588522618968e-09
4012 9.12485849149242e-09
4013 9.12416088764273e-09
4014 9.12343478180755e-09
4015 9.12153413419603e-09
4016 9.12161035704351e-09
4017 9.12089520922754e-09
4018 9.1196581255143e-09
4019 9.11914853247986e-09
4020 9.1175264955029e-09
4021 9.11567241706351e-09
4022 9.11676220976404e-09
4023 9.1162394412897e-09
4024 9.11547859588913e-09
4025 9.11398437364319e-09
4026 9.1141955969945e-09
4027 9.11227282416466e-09
4028 9.11234266979394e-09
4029 9.11169202463813e-09
4030 9.10868444103235e-09
4031 9.10865838280039e-09
4032 9.10833167878239e-09
4033 9.10704751972552e-09
4034 9.10659218051474e-09
4035 9.10550701647278e-09
4036 9.10362860684355e-09
4037 9.1042123911908e-09
4038 9.10348919347381e-09
4039 9.10063497486235e-09
4040 9.10261911066246e-09
4041 9.09966773788434e-09
4042 9.09804278677318e-09
4043 9.09720082729187e-09
4044 9.09712228747139e-09
4045 9.09550928773067e-09
4046 9.09559654635461e-09
4047 9.09379277599237e-09
4048 9.09414674930831e-09
4049 9.09227713571936e-09
4050 9.09173065252733e-09
4051 9.09130157216581e-09
4052 9.0916532465013e-09
4053 9.08985415506297e-09
4054 9.09109424684223e-09
4055 9.08948476503052e-09
4056 9.08943381817268e-09
4057 9.08892737327277e-09
4058 9.08710115159195e-09
4059 9.08745861439436e-09
4060 9.08442935835391e-09
4061 9.08436471819263e-09
4062 9.08271496057045e-09
4063 9.08300147157981e-09
4064 9.08117277528126e-09
4065 9.08223890097593e-09
4066 9.0819864484587e-09
4067 9.07895508258511e-09
4068 9.0795044409564e-09
4069 9.07821997219588e-09
4070 9.07634395518053e-09
4071 9.07743119878818e-09
4072 9.07495407810599e-09
4073 9.07360060824897e-09
4074 9.07431627954897e-09
4075 9.0735078816101e-09
4076 9.07306369298672e-09
4077 9.07269099428842e-09
4078 9.07119608837409e-09
4079 9.06867890103358e-09
4080 9.06969333839408e-09
4081 9.06888082131269e-09
4082 9.06576099134887e-09
4083 9.06394001525673e-09
4084 9.06403991943888e-09
4085 9.06326660959911e-09
4086 9.06234318255594e-09
4087 9.0614207199427e-09
4088 9.05956535212499e-09
4089 9.05966741174963e-09
4090 9.05934360465738e-09
4091 9.05849581590151e-09
4092 9.05631166515236e-09
4093 9.05697225003776e-09
4094 9.05570951728607e-09
4095 9.05515024897385e-09
4096 9.05432115046873e-09
4097 9.05357733209378e-09
4098 9.05289298105266e-09
4099 9.05177069884106e-09
4100 9.05119387784731e-09
4101 9.05035931401954e-09
4102 9.04956510079658e-09
4103 9.04755188414763e-09
4104 9.0482246629317e-09
4105 9.04738247125419e-09
4106 9.04633492281731e-09
4107 9.04587611790658e-09
4108 9.0448946459612e-09
4109 9.04424570492834e-09
4110 9.04340473122794e-09
4111 9.04272286737784e-09
4112 9.04178104394609e-09
4113 9.04113669054213e-09
4114 9.04039970221759e-09
4115 9.03956163636477e-09
4116 9.03882121445349e-09
4117 9.0371105694112e-09
4118 9.03736766836444e-09
4119 9.03658708192867e-09
4120 9.03576621961999e-09
4121 9.03501752597818e-09
4122 9.03419905882302e-09
4123 9.03358106944113e-09
4124 9.03265862668701e-09
4125 9.03195426244646e-09
4126 9.03114919882653e-09
4127 9.03039407397499e-09
4128 9.02983825714493e-09
4129 9.02899792763062e-09
4130 9.02838638638243e-09
4131 9.02694487753958e-09
4132 9.02665488525245e-09
4133 9.02580838788025e-09
4134 9.02510746347146e-09
4135 9.02446897648596e-09
4136 9.02367710925445e-09
4137 9.02251664073456e-09
4138 9.02221809614967e-09
4139 9.02131376055715e-09
4140 9.02079077250845e-09
4141 9.01972897970893e-09
4142 9.01862311793178e-09
4143 9.01851773014301e-09
4144 9.01774065732025e-09
4145 9.01681527529924e-09
4146 9.01643270705826e-09
4147 9.0154567920428e-09
4148 9.01489184124138e-09
4149 9.01379064870839e-09
4150 9.01367665381847e-09
4151 9.01268898353264e-09
4152 9.01165407724003e-09
4153 9.01146551334636e-09
4154 9.01037023089829e-09
4155 9.0099909098515e-09
4156 9.00941769674951e-09
4157 9.0086584180471e-09
4158 9.00826957004469e-09
4159 9.00686864607064e-09
4160 9.0057908289573e-09
4161 9.00566703674471e-09
4162 9.00451145118691e-09
4163 9.00431719082911e-09
4164 9.00391411764318e-09
4165 9.00156729096574e-09
4166 9.001770064819e-09
4167 9.00077494634527e-09
4168 9.00053916032956e-09
4169 8.99984706849433e-09
4170 8.99909051888664e-09
4171 8.99660675258984e-09
4172 8.99580060755717e-09
4173 8.99432007445827e-09
4174 8.99460884051101e-09
4175 8.99270908865518e-09
4176 8.99305082317781e-09
4177 8.99239232903665e-09
4178 8.99059834150173e-09
4179 8.99088269940007e-09
4180 8.98996085550024e-09
4181 8.98991195417281e-09
4182 8.98854449179337e-09
4183 8.98815155146726e-09
4184 8.98616466049607e-09
4185 8.98634272628041e-09
4186 8.98555481740337e-09
4187 8.98508108221341e-09
4188 8.98433574410967e-09
4189 8.98359824832995e-09
4190 8.98285217841577e-09
4191 8.98209377614423e-09
4192 8.98105000676774e-09
4193 8.98065692958583e-09
4194 8.97955766381714e-09
4195 8.9792502771871e-09
4196 8.97844624515171e-09
4197 8.97772096155464e-09
4198 8.97688135425429e-09
4199 8.97657125754209e-09
4200 8.97696220240885e-09
4201 8.97770358946476e-09
4202 8.97690921937422e-09
4203 8.97608080360074e-09
4204 8.97527790568592e-09
4205 8.97497873511566e-09
4206 8.97383778644678e-09
4207 8.97310335805368e-09
4208 8.97260311905762e-09
4209 8.97166961700419e-09
4210 8.97106368903666e-09
4211 8.97027132740896e-09
4212 8.96944251475157e-09
4213 8.96847451407196e-09
4214 8.96776644353226e-09
4215 8.96712359072654e-09
4216 8.966447805063e-09
4217 8.96563082897206e-09
4218 8.9649814035489e-09
4219 8.96398058197051e-09
4220 8.9640652122247e-09
4221 8.96307140141328e-09
4222 8.96241508432233e-09
4223 8.96148784973461e-09
4224 8.96049575955377e-09
4225 8.9594112861538e-09
4226 8.96017608005562e-09
4227 8.95859384347775e-09
4228 8.95748908814192e-09
4229 8.95639440470775e-09
4230 8.95598475857823e-09
4231 8.95633869370249e-09
4232 8.95500017900758e-09
4233 8.95427801776011e-09
4234 8.95353970219864e-09
4235 8.95281926134583e-09
4236 8.95216080781802e-09
4237 8.95144028918715e-09
4238 8.95068623461837e-09
4239 8.9500198903053e-09
4240 8.94923798529235e-09
4241 8.94859818700161e-09
4242 8.94768658699779e-09
4243 8.94614942177629e-09
4244 8.94594826889722e-09
4245 8.94653876719331e-09
4246 8.94433954584661e-09
4247 8.94461090875792e-09
4248 8.94255105796105e-09
4249 8.94226915443969e-09
4250 8.94163558641525e-09
4251 8.9420101326218e-09
4252 8.94079513152901e-09
4253 8.94009131997137e-09
4254 8.9394256162223e-09
4255 8.93869162672811e-09
4256 8.93811377588455e-09
4257 8.93708492617773e-09
4258 8.93545572037274e-09
4259 8.93535601004941e-09
4260 8.93562284690758e-09
4261 8.93455972122981e-09
4262 8.93379857164894e-09
4263 8.93307636169738e-09
4264 8.93229500066595e-09
4265 8.93156394076583e-09
4266 8.93143176804861e-09
4267 8.93019497661546e-09
4268 8.92964860558371e-09
4269 8.92885785094138e-09
4270 8.92812495145723e-09
4271 8.92755696001862e-09
4272 8.92628823861291e-09
4273 8.92605203976077e-09
4274 8.92539208680737e-09
4275 8.92440013780527e-09
4276 8.92373679099723e-09
4277 8.92315068621446e-09
4278 8.92235888363263e-09
4279 8.92157788879439e-09
4280 8.92097921668944e-09
4281 8.92001096306633e-09
4282 8.91873901256307e-09
4283 8.91948873470855e-09
4284 8.91814107593231e-09
4285 8.91773904157511e-09
4286 8.91679698864639e-09
4287 8.91518644473982e-09
4288 8.9160754075876e-09
4289 8.91480075119328e-09
4290 8.91444785105361e-09
4291 8.91321340860257e-09
4292 8.91266340197899e-09
4293 8.9110454213337e-09
4294 8.91199303054008e-09
4295 8.90966775755203e-09
4296 8.9106265390923e-09
4297 8.90971730693746e-09
4298 8.90783175309939e-09
4299 8.90843172796085e-09
4300 8.90719282087976e-09
4301 8.90660340560623e-09
4302 8.90568278714976e-09
4303 8.90567601781467e-09
4304 8.90471764394135e-09
4305 8.90290742569522e-09
4306 8.90390032349392e-09
4307 8.90048560972523e-09
4308 8.90037092218104e-09
4309 8.89920226618962e-09
4310 8.89882749213061e-09
4311 8.89726649477146e-09
4312 8.89615117334597e-09
4313 8.89633128269701e-09
4314 8.89538848359434e-09
4315 8.89554929060093e-09
4316 8.893845630846e-09
4317 8.89345835391325e-09
4318 8.89232777039756e-09
4319 8.89256504024022e-09
4320 8.89120537185539e-09
4321 8.89070632244943e-09
4322 8.8894089965777e-09
4323 8.88916720632321e-09
4324 8.8888233975154e-09
4325 8.88771944596023e-09
4326 8.8869576876302e-09
4327 8.88638837334005e-09
4328 8.88567606843749e-09
4329 8.88509552136874e-09
4330 8.88439044920836e-09
4331 8.88371621771134e-09
4332 8.88302247299688e-09
4333 8.88228250418843e-09
4334 8.88160839204039e-09
4335 8.8808284976552e-09
4336 8.88026378336598e-09
4337 8.87945819944697e-09
4338 8.87877754941063e-09
4339 8.87805659686297e-09
4340 8.87754306184985e-09
4341 8.87707541596949e-09
4342 8.8761129339171e-09
4343 8.87550829004741e-09
4344 8.87511245468292e-09
4345 8.87441390415994e-09
4346 8.87366841206827e-09
4347 8.87319455859098e-09
4348 8.87243199474558e-09
4349 8.87140976637418e-09
4350 8.87105194124355e-09
4351 8.87042238962438e-09
4352 8.8695433388894e-09
4353 8.87016886257086e-09
4354 8.87010100229096e-09
4355 8.86714926859322e-09
4356 8.86699133097191e-09
4357 8.86691071712298e-09
4358 8.866456837682e-09
4359 8.8649438694452e-09
4360 8.8632210573647e-09
4361 8.8633143092362e-09
4362 8.86309250672235e-09
4363 8.86349948885784e-09
4364 8.86293240398572e-09
4365 8.86110901700959e-09
4366 8.86052862114628e-09
4367 8.86020856514375e-09
4368 8.85941065156132e-09
4369 8.85910197265249e-09
4370 8.85869953496515e-09
4371 8.85743240853359e-09
4372 8.85714032191481e-09
4373 8.85550838169297e-09
4374 8.85541013022351e-09
4375 8.85406265633409e-09
4376 8.85385370293784e-09
4377 8.85293315733976e-09
4378 8.8526316465512e-09
4379 8.85148072835712e-09
4380 8.85138802244473e-09
4381 8.85008681774518e-09
4382 8.8500786889073e-09
4383 8.84880284807849e-09
4384 8.84889863669858e-09
4385 8.84751621377938e-09
4386 8.8474309193759e-09
4387 8.84617404606908e-09
4388 8.84622645828948e-09
4389 8.84499550557472e-09
4390 8.84485260681245e-09
4391 8.84358838593863e-09
4392 8.84351586647597e-09
4393 8.84238042159202e-09
4394 8.84231552733539e-09
4395 8.842714191791e-09
4396 8.84230395495345e-09
4397 8.8413021537212e-09
4398 8.84101641459439e-09
4399 8.83995712355062e-09
4400 8.83971526347699e-09
4401 8.8386059781953e-09
4402 8.8384570888847e-09
4403 8.83741608950078e-09
4404 8.8369404417088e-09
4405 8.83612920582477e-09
4406 8.83576240269523e-09
4407 8.83458494824185e-09
4408 8.83443229800912e-09
4409 8.83343026959055e-09
4410 8.83296198237599e-09
4411 8.83229369447874e-09
4412 8.83183871079607e-09
4413 8.83083327301443e-09
4414 8.83061944947222e-09
4415 8.83005898091871e-09
4416 8.82912716019396e-09
4417 8.82855050262504e-09
4418 8.82766931364215e-09
4419 8.82714250906302e-09
4420 8.82644846676717e-09
4421 8.82560438591334e-09
4422 8.82520561056727e-09
4423 8.82430343886187e-09
4424 8.82384353586424e-09
4425 8.82312001982549e-09
4426 8.82226579646678e-09
4427 8.82164007330599e-09
4428 8.82112441314559e-09
4429 8.8201681851946e-09
4430 8.81974382340583e-09
4431 8.81892154021291e-09
4432 8.81843156909368e-09
4433 8.81766289393543e-09
4434 8.81672537764167e-09
4435 8.81644811596277e-09
4436 8.81572190467722e-09
4437 8.81497531048103e-09
4438 8.81447346807313e-09
4439 8.81362354995807e-09
4440 8.81305774082935e-09
4441 8.81236076253483e-09
4442 8.81185533528617e-09
4443 8.81104278294131e-09
4444 8.81042283710359e-09
4445 8.80964840627857e-09
4446 8.80921277430363e-09
4447 8.80832114607405e-09
4448 8.80787553712847e-09
4449 8.80701806766499e-09
4450 8.80659173322501e-09
4451 8.80567757098605e-09
4452 8.80517508896478e-09
4453 8.80450410460643e-09
4454 8.80387555939749e-09
4455 8.80326725596903e-09
4456 8.80253102298473e-09
4457 8.80225303743348e-09
4458 8.80127038544592e-09
4459 8.80078808170731e-09
4460 8.7999589101842e-09
4461 8.79945796139481e-09
4462 8.79842003822379e-09
4463 8.79811351255783e-09
4464 8.79736062781022e-09
4465 8.79701393137478e-09
4466 8.79603833264797e-09
4467 8.79580499777571e-09
4468 8.794765165826e-09
4469 8.79428293022733e-09
4470 8.79523880813193e-09
4471 8.79357472634984e-09
4472 8.79355609206822e-09
4473 8.79402946405566e-09
4474 8.79262878396292e-09
4475 8.79153456733589e-09
4476 8.79147164387084e-09
4477 8.79024155051728e-09
4478 8.78935039426432e-09
4479 8.78927697361281e-09
4480 8.78863340651737e-09
4481 8.78803294393493e-09
4482 8.78727693559567e-09
4483 8.7866873191289e-09
4484 8.78615818565576e-09
4485 8.78534517101404e-09
4486 8.7847355301901e-09
4487 8.784231768387e-09
4488 8.78332383170283e-09
4489 8.78287271790373e-09
4490 8.78204196039234e-09
4491 8.781653790986e-09
4492 8.7814629320393e-09
4493 8.78094338755547e-09
4494 8.7804618637069e-09
4495 8.77954854995588e-09
4496 8.77891906202882e-09
4497 8.77811576840276e-09
4498 8.77707448368459e-09
4499 8.77621632783265e-09
4500 8.77627930132713e-09
4501 8.7751503776573e-09
4502 8.77478042314583e-09
4503 8.77422636717806e-09
4504 8.77350601802967e-09
4505 8.77308025792195e-09
4506 8.7726289985518e-09
4507 8.77125525954037e-09
4508 8.77130898395417e-09
4509 8.76978264750111e-09
4510 8.76946136386253e-09
4511 8.76890010486492e-09
4512 8.76871793657752e-09
4513 8.76803375159107e-09
4514 8.76710590833746e-09
4515 8.76582416419608e-09
4516 8.7652298075766e-09
4517 8.76469015913262e-09
4518 8.76497363797718e-09
4519 8.76385267142149e-09
4520 8.76268160002192e-09
4521 8.76232939280713e-09
4522 8.7615258415677e-09
4523 8.76104399145927e-09
4524 8.7602451697838e-09
4525 8.75911414281033e-09
4526 8.76008926030403e-09
4527 8.75899057092844e-09
4528 8.75769092161666e-09
4529 8.75648848983751e-09
4530 8.75725479851375e-09
4531 8.75586264075662e-09
4532 8.75459268241674e-09
4533 8.75565210461254e-09
4534 8.75433283804156e-09
4535 8.7532045265834e-09
4536 8.75318328433933e-09
4537 8.75198294521956e-09
4538 8.75153982413113e-09
4539 8.75070343427098e-09
4540 8.75061716743702e-09
4541 8.7501769963777e-09
4542 8.74905800753861e-09
4543 8.74869557181135e-09
4544 8.74781091155657e-09
4545 8.74739139023612e-09
4546 8.74654694865695e-09
4547 8.74582042895838e-09
4548 8.74504271781984e-09
4549 8.74436347351321e-09
4550 8.74369803720298e-09
4551 8.74296605284791e-09
4552 8.74234501249682e-09
4553 8.74170744025182e-09
4554 8.74103584929536e-09
4555 8.74035627195652e-09
4556 8.7398140719841e-09
4557 8.7393257670737e-09
4558 8.73841961249178e-09
4559 8.73771293850162e-09
4560 8.73725372153855e-09
4561 8.73678672622113e-09
4562 8.73604037651687e-09
4563 8.73493831530375e-09
4564 8.7345849136139e-09
4565 8.73404449461962e-09
4566 8.73365093272821e-09
4567 8.73290091525647e-09
4568 8.73226278195333e-09
4569 8.73142917861419e-09
4570 8.73082954870597e-09
4571 8.73029931534874e-09
4572 8.72928402729373e-09
4573 8.72891311912149e-09
4574 8.72820563577875e-09
4575 8.72789770038629e-09
4576 8.72707420717517e-09
4577 8.72635182452147e-09
4578 8.72591393268757e-09
4579 8.72513199728225e-09
4580 8.72472866841889e-09
4581 8.72394812241606e-09
4582 8.72324504798405e-09
4583 8.72282327026713e-09
4584 8.72217023665384e-09
4585 8.72137988493227e-09
4586 8.72063681685298e-09
4587 8.72008072617947e-09
4588 8.71948672132722e-09
4589 8.71885850907417e-09
4590 8.71821129617284e-09
4591 8.7176086489768e-09
4592 8.71703292915776e-09
4593 8.71639826847959e-09
4594 8.71569299840808e-09
4595 8.71533584392847e-09
4596 8.71479460094055e-09
4597 8.71394683370219e-09
4598 8.71242145610018e-09
4599 8.71256817074062e-09
4600 8.71237407795711e-09
4601 8.71115234348657e-09
4602 8.71092498793552e-09
4603 8.70963456166274e-09
4604 8.70937363928182e-09
4605 8.70963638235217e-09
4606 8.708220026174e-09
4607 8.70794539731962e-09
4608 8.70685985370628e-09
4609 8.70664962805689e-09
4610 8.70619140190543e-09
4611 8.70555031128734e-09
4612 8.70495943578603e-09
4613 8.70421772272007e-09
4614 8.70387680815266e-09
4615 8.70309532824409e-09
4616 8.70264917984115e-09
4617 8.70189592804688e-09
4618 8.70132281974995e-09
4619 8.70096949738553e-09
4620 8.70012496099332e-09
4621 8.69960696230077e-09
4622 8.6989755178693e-09
4623 8.69839059415567e-09
4624 8.69769822455652e-09
4625 8.69727899625861e-09
4626 8.69650286283719e-09
4627 8.69620743777083e-09
4628 8.69535324594245e-09
4629 8.6948558308958e-09
4630 8.69412544610149e-09
4631 8.69357632043988e-09
4632 8.69300221151975e-09
4633 8.6926707196372e-09
4634 8.69182659109236e-09
4635 8.6912975397202e-09
4636 8.69072682252447e-09
4637 8.68971169819266e-09
4638 8.68931720111876e-09
4639 8.68842973242301e-09
4640 8.68818004561012e-09
4641 8.68734816903555e-09
4642 8.68698981910943e-09
4643 8.68615433305409e-09
4644 8.68582053665384e-09
4645 8.685402182948e-09
4646 8.68473557612964e-09
4647 8.68408931853359e-09
4648 8.68363852565834e-09
4649 8.68319679619978e-09
4650 8.68241203606096e-09
4651 8.68201106585253e-09
4652 8.68125745664972e-09
4653 8.68063307046113e-09
4654 8.6803196859761e-09
4655 8.6795364719755e-09
4656 8.67902182100089e-09
4657 8.67806961775408e-09
4658 8.67780262747003e-09
4659 8.67680258261061e-09
4660 8.67654120981914e-09
4661 8.67565984326996e-09
4662 8.67549794146549e-09
4663 8.6744134627087e-09
4664 8.674323417876e-09
4665 8.67373064366433e-09
4666 8.67333321379754e-09
4667 8.67237267698079e-09
4668 8.67204522631698e-09
4669 8.67109005783101e-09
4670 8.67096171440485e-09
4671 8.66974227708983e-09
4672 8.66962298747692e-09
4673 8.66912550829407e-09
4674 8.66792595596855e-09
4675 8.66748631810077e-09
4676 8.66764531228664e-09
4677 8.66682175072742e-09
4678 8.66632080870344e-09
4679 8.66577011233993e-09
4680 8.66520428260964e-09
4681 8.66462020676639e-09
4682 8.66405583891533e-09
4683 8.6634821756526e-09
4684 8.66290880531528e-09
4685 8.66234705568236e-09
4686 8.66176266953178e-09
4687 8.66119082339883e-09
4688 8.66060049595219e-09
4689 8.66006788378254e-09
4690 8.6595056952854e-09
4691 8.65892877617569e-09
4692 8.65838262054802e-09
4693 8.65777798886302e-09
4694 8.65713696227705e-09
4695 8.65657801465969e-09
4696 8.65602633290386e-09
4697 8.65545546015201e-09
4698 8.65488908005641e-09
4699 8.65434635393542e-09
4700 8.65374731907204e-09
4701 8.6532169928169e-09
4702 8.65265486984374e-09
4703 8.65206981016942e-09
4704 8.65150053294683e-09
4705 8.65092662753752e-09
4706 8.65040255355887e-09
4707 8.64982284269622e-09
4708 8.64926460269338e-09
4709 8.64870473771956e-09
4710 8.64812428744144e-09
4711 8.64760394499869e-09
4712 8.64704176038039e-09
4713 8.64646012950498e-09
4714 8.64592879283588e-09
4715 8.64548630222922e-09
4716 8.64473712561264e-09
4717 8.64427157641623e-09
4718 8.64370142290849e-09
4719 8.64313221508872e-09
4720 8.64264249843261e-09
4721 8.64213083316689e-09
4722 8.64147119356007e-09
4723 8.64093467602828e-09
4724 8.64037411291846e-09
4725 8.63981612069659e-09
4726 8.6392744533606e-09
4727 8.63871145417167e-09
4728 8.63816438186754e-09
4729 8.63753953473517e-09
4730 8.63704537153176e-09
4731 8.63658343257245e-09
4732 8.63596980474335e-09
4733 8.63542186441135e-09
4734 8.63487648928435e-09
4735 8.63429579958663e-09
4736 8.63371267740415e-09
4737 8.63318028838933e-09
4738 8.63273032395556e-09
4739 8.63222611683506e-09
4740 8.63166659694631e-09
4741 8.63109000723977e-09
4742 8.6306344203313e-09
4743 8.63011521307078e-09
4744 8.62950503151272e-09
4745 8.6289729318012e-09
4746 8.62840946477816e-09
4747 8.62794075812134e-09
4748 8.62739322361056e-09
4749 8.62682799975673e-09
4750 8.62627186556941e-09
4751 8.62571988966693e-09
4752 8.62517638300364e-09
4753 8.62464965908916e-09
4754 8.62411582980282e-09
4755 8.62358983262646e-09
4756 8.62301131492743e-09
4757 8.62249490377748e-09
4758 8.62196336884335e-09
4759 8.62140247800264e-09
4760 8.62094155518883e-09
4761 8.62042474960439e-09
4762 8.6198019492098e-09
4763 8.61935171762862e-09
4764 8.61880710861102e-09
4765 8.61826343061256e-09
4766 8.61773254166864e-09
4767 8.61713000528674e-09
4768 8.61668006321703e-09
4769 8.61615320080222e-09
4770 8.61563116148767e-09
4771 8.61510856840081e-09
4772 8.61460402347414e-09
4773 8.61362087188622e-09
4774 8.61277457474968e-09
4775 8.61218859923851e-09
4776 8.61192743741024e-09
4777 8.61147065270296e-09
4778 8.61105169232962e-09
4779 8.61053016644464e-09
4780 8.60996286437127e-09
4781 8.6094124002456e-09
4782 8.60923195545454e-09
4783 8.60834768601909e-09
4784 8.60827871099201e-09
4785 8.60727103337011e-09
4786 8.60725467660695e-09
4787 8.60628148881271e-09
4788 8.60577735077583e-09
4789 8.60567159152725e-09
4790 8.60464166356489e-09
4791 8.60462097572456e-09
4792 8.60359878929878e-09
4793 8.60358463344174e-09
4794 8.60256115371022e-09
4795 8.60251190298172e-09
4796 8.60144831198867e-09
4797 8.60156932452893e-09
4798 8.6004435425821e-09
4799 8.60040732955647e-09
4800 8.59940488091848e-09
4801 8.5993285712585e-09
4802 8.59840103619586e-09
4803 8.59801625005541e-09
4804 8.59712704893989e-09
4805 8.5989766552344e-09
4806 8.59791335341281e-09
4807 8.59752676054398e-09
4808 8.59725138734058e-09
4809 8.59672903279901e-09
4810 8.59603227004735e-09
4811 8.5957779507595e-09
4812 8.59485939146143e-09
4813 8.59420912428799e-09
4814 8.59363485590514e-09
4815 8.59317161434198e-09
4816 8.59255751167742e-09
4817 8.59186283310581e-09
4818 8.59152046910755e-09
4819 8.59092450324755e-09
4820 8.58999083739459e-09
4821 8.59017470652979e-09
4822 8.58964213332897e-09
4823 8.58908812521875e-09
4824 8.58853125826342e-09
4825 8.58795468849927e-09
4826 8.58746445737274e-09
4827 8.58691871286066e-09
4828 8.58638974862713e-09
4829 8.58585800733724e-09
4830 8.5853174249112e-09
4831 8.58459447034693e-09
4832 8.58432210077559e-09
4833 8.58357800260362e-09
4834 8.58320493499209e-09
4835 8.58249962331498e-09
4836 8.58212741142866e-09
4837 8.58146741093691e-09
4838 8.58112637096981e-09
4839 8.58038762688307e-09
4840 8.57968762741496e-09
4841 8.57947196864633e-09
4842 8.57867199182544e-09
4843 8.57818494042623e-09
4844 8.57763417033697e-09
4845 8.57706888175019e-09
4846 8.57669052797827e-09
4847 8.57612314862644e-09
4848 8.57562819369523e-09
4849 8.57509478337931e-09
4850 8.57450678752592e-09
4851 8.57413090200182e-09
4852 8.57347812483616e-09
4853 8.57287453816247e-09
4854 8.57238443668223e-09
4855 8.5720821102267e-09
4856 8.57189404162434e-09
4857 8.57088718198701e-09
4858 8.57129581741112e-09
4859 8.57031805326369e-09
4860 8.56905498679977e-09
4861 8.56881160411771e-09
4862 8.56843690536652e-09
4863 8.56817451465319e-09
4864 8.56713304523554e-09
4865 8.56727462836271e-09
4866 8.56602684497915e-09
4867 8.56555208034876e-09
4868 8.56536029774507e-09
4869 8.56500419064987e-09
4870 8.56399280069892e-09
4871 8.56458393844878e-09
4872 8.56357424362103e-09
4873 8.56268582750952e-09
4874 8.56204393288179e-09
4875 8.56156959681831e-09
4876 8.5612480988928e-09
4877 8.56093318769846e-09
4878 8.56001214370045e-09
4879 8.55995444999647e-09
4880 8.55898258949467e-09
4881 8.55887948277706e-09
4882 8.55815955011496e-09
4883 8.55803363670332e-09
4884 8.55682549534553e-09
4885 8.55641313392275e-09
4886 8.55601129526318e-09
4887 8.55633722954274e-09
4888 8.55460185033435e-09
4889 8.55426610230081e-09
4890 8.55413075241673e-09
4891 8.55345115124972e-09
4892 8.55338185876131e-09
4893 8.55222783292808e-09
4894 8.55176571463306e-09
4895 8.55167128406553e-09
4896 8.55099410973503e-09
4897 8.55119722958236e-09
4898 8.54955577414551e-09
4899 8.5492603149051e-09
4900 8.5489372657413e-09
4901 8.54852792701866e-09
4902 8.54855772222662e-09
4903 8.54738182688719e-09
4904 8.54682431664783e-09
4905 8.54648329441654e-09
4906 8.54603393314612e-09
4907 8.5455342801119e-09
4908 8.54569417352985e-09
4909 8.54468910187201e-09
4910 8.54394783186829e-09
4911 8.54373435339034e-09
4912 8.5429900187825e-09
4913 8.54280694886911e-09
4914 8.54168822299328e-09
4915 8.5413427987624e-09
4916 8.54112577187288e-09
4917 8.54055874628667e-09
4918 8.54069542540609e-09
4919 8.54010218477586e-09
4920 8.53925350002838e-09
4921 8.53856113187945e-09
4922 8.53822554780503e-09
4923 8.53760832451866e-09
4924 8.53788917778986e-09
4925 8.53703117030535e-09
4926 8.53623377385093e-09
4927 8.53563774164123e-09
4928 8.53535348868673e-09
4929 8.53460344821255e-09
4930 8.53408182066584e-09
4931 8.53358451256142e-09
4932 8.53333632085118e-09
4933 8.53262510543823e-09
4934 8.53204820822073e-09
4935 8.53151192817259e-09
4936 8.53152530089218e-09
4937 8.53067446100059e-09
4938 8.53068872235657e-09
4939 8.53020498296081e-09
4940 8.52942688014946e-09
4941 8.52911897988168e-09
4942 8.52871700045971e-09
4943 8.52806712001164e-09
4944 8.52778870422122e-09
4945 8.52718304891059e-09
4946 8.52685717657453e-09
4947 8.52597359103646e-09
4948 8.52534580486619e-09
4949 8.52518646967532e-09
4950 8.52444303636041e-09
4951 8.52392883288894e-09
4952 8.52384070380852e-09
4953 8.5229399746159e-09
4954 8.5225303259745e-09
4955 8.52248586878657e-09
4956 8.5215768578506e-09
4957 8.52096567275501e-09
4958 8.52066629029175e-09
4959 8.52035099894316e-09
4960 8.51964291022356e-09
4961 8.51902966820389e-09
4962 8.51873915155149e-09
4963 8.51873454956603e-09
4964 8.51778585982121e-09
4965 8.51737106721123e-09
4966 8.51733001096255e-09
4967 8.51638489702405e-09
4968 8.51596035072316e-09
4969 8.51584731279692e-09
4970 8.51496897217446e-09
4971 8.51449201530463e-09
4972 8.51471722218555e-09
4973 8.51336526235674e-09
4974 8.51345486597988e-09
4975 8.51246609006456e-09
4976 8.5123733779488e-09
4977 8.51205991848902e-09
4978 8.51106323815354e-09
4979 8.51094422272408e-09
4980 8.51038566349049e-09
4981 8.51021704853516e-09
4982 8.50904898826166e-09
4983 8.50909153233409e-09
4984 8.50869337575e-09
4985 8.50825857439536e-09
4986 8.50787657697555e-09
4987 8.50729915605197e-09
4988 8.50686705259479e-09
4989 8.50665324818312e-09
4990 8.5051393416205e-09
4991 8.50550700179364e-09
4992 8.50497590487026e-09
4993 8.5046087461918e-09
4994 8.50406552518196e-09
4995 8.50351453081377e-09
4996 8.50318883780649e-09
4997 8.5026531529489e-09
4998 8.50215662138998e-09
4999 8.50142493778738e-09
};
\addlegendentry{Train}
\addplot [semithick, black]
table {%
0 0.00164698576554656
1 0.000319997867336497
2 0.000236948311794549
3 0.000220637521124445
4 0.000176255474798381
5 7.65787553973496e-05
6 3.01542695524404e-05
7 1.97536865016446e-05
8 1.76177181856474e-05
9 1.70167404576205e-05
10 1.65528981597163e-05
11 1.59899482241599e-05
12 1.52104103108286e-05
13 1.41408991112257e-05
14 1.23904374049744e-05
15 1.04280061350437e-05
16 8.21452795207733e-06
17 6.20531409367686e-06
18 4.68678945253487e-06
19 3.72375757251575e-06
20 3.07429809254245e-06
21 2.57832039096684e-06
22 2.14386341212958e-06
23 1.74032732047635e-06
24 1.3888226249037e-06
25 1.10497762761952e-06
26 9.05191484434908e-07
27 7.74379032009165e-07
28 6.87878639382689e-07
29 6.28580210104701e-07
30 5.86219755405182e-07
31 5.59082707241032e-07
32 5.39234520147147e-07
33 5.22027278293535e-07
34 5.0740044343911e-07
35 4.92628771553427e-07
36 4.81738027247047e-07
37 4.70517221629052e-07
38 4.6088049998616e-07
39 4.51718108251953e-07
40 4.42773313125144e-07
41 4.34468773846675e-07
42 4.26487844151779e-07
43 4.19876300838951e-07
44 4.13272999821857e-07
45 4.06589606427588e-07
46 4.01280146888894e-07
47 3.956403702432e-07
48 3.91526981502466e-07
49 3.87004092772258e-07
50 3.82983813551618e-07
51 3.78600816475227e-07
52 3.7508536365749e-07
53 3.71204151861093e-07
54 3.68257303762221e-07
55 3.65195290896736e-07
56 3.62439749324039e-07
57 3.59941452643397e-07
58 3.5765449979408e-07
59 3.55654549366591e-07
60 3.53936286501266e-07
61 3.52291493754819e-07
62 3.50126612147506e-07
63 3.49030443658194e-07
64 3.47310219694918e-07
65 3.45994806139061e-07
66 3.44528473306127e-07
67 3.4332776976953e-07
68 3.4204990129183e-07
69 3.40738637305549e-07
70 3.39501951884813e-07
71 3.3822669820438e-07
72 3.37002347805537e-07
73 3.35694750219773e-07
74 3.34555437575546e-07
75 3.33259720264323e-07
76 3.32090650090322e-07
77 3.30928685343679e-07
78 3.29778259811064e-07
79 3.28580597397377e-07
80 3.27362386087771e-07
81 3.25977140391842e-07
82 3.22699179378105e-07
83 3.26314619769619e-07
84 3.25261112266162e-07
85 3.24050461131264e-07
86 3.23284325531858e-07
87 3.2227382007477e-07
88 3.20850290336239e-07
89 3.19364858114568e-07
90 3.17840203933883e-07
91 3.1614536055713e-07
92 3.14485873786907e-07
93 3.12343502173462e-07
94 3.1089109597815e-07
95 3.09714266677474e-07
96 3.09784439878058e-07
97 3.08026244510984e-07
98 3.05709249914798e-07
99 3.03319154681958e-07
100 3.03452992511666e-07
101 3.01160156368496e-07
102 2.98373606710811e-07
103 2.95939599936901e-07
104 2.93354020186598e-07
105 2.90972906213938e-07
106 2.88018441096938e-07
107 2.85119057252814e-07
108 2.8195572099321e-07
109 2.74036437986069e-07
110 2.7163579829903e-07
111 2.6903015282187e-07
112 2.66102688328829e-07
113 2.6172654088441e-07
114 2.5965877625822e-07
115 2.56880838378493e-07
116 2.5366102818225e-07
117 2.50210661079109e-07
118 2.4618285010547e-07
119 2.42769203850912e-07
120 2.39415442138124e-07
121 2.35446592000699e-07
122 2.31903612757378e-07
123 2.28984632144602e-07
124 2.25479993787303e-07
125 2.22112831238519e-07
126 2.19055479533381e-07
127 2.14684462207515e-07
128 2.12261284104898e-07
129 2.09541894946597e-07
130 2.07182679901052e-07
131 2.04492849320559e-07
132 2.01855030468323e-07
133 1.99226207087122e-07
134 1.96644407424174e-07
135 1.9414143537233e-07
136 1.91756654999153e-07
137 1.89360420677076e-07
138 1.86735576335195e-07
139 1.8443751059749e-07
140 1.8217016872768e-07
141 1.79984937176414e-07
142 1.77844299287244e-07
143 1.75728317231005e-07
144 1.73547974213761e-07
145 1.71369947565836e-07
146 1.69402966321286e-07
147 1.67648650517549e-07
148 1.6586298556831e-07
149 1.64040145023137e-07
150 1.62101116529811e-07
151 1.60563317308515e-07
152 1.58959423401939e-07
153 1.57462437755385e-07
154 1.59361178475592e-07
155 1.59249395892402e-07
156 1.57544420176237e-07
157 1.57290841684699e-07
158 1.54884062908422e-07
159 1.55249153976911e-07
160 1.56474825985242e-07
161 1.54940849483864e-07
162 1.52035198652811e-07
163 1.52815118781291e-07
164 1.52237035422331e-07
165 1.51195678199656e-07
166 1.4947650583963e-07
167 1.48170002489678e-07
168 1.47557500440598e-07
169 1.46665271927304e-07
170 1.45615032920432e-07
171 1.42430096161661e-07
172 1.41164662181836e-07
173 1.41176485612959e-07
174 1.39303196533547e-07
175 1.38348127620702e-07
176 1.40106777735127e-07
177 1.37104436248592e-07
178 1.36529436645105e-07
179 1.35934826062112e-07
180 1.35618421381878e-07
181 1.34644196236877e-07
182 1.33896634224584e-07
183 1.31675150782939e-07
184 1.33293767135001e-07
185 1.32362899307736e-07
186 1.31537234437928e-07
187 1.30976260948046e-07
188 1.27931016891125e-07
189 1.26801552369216e-07
190 1.27696139884392e-07
191 1.27300239682882e-07
192 1.26404998468388e-07
193 1.25535194683835e-07
194 1.25473974321721e-07
195 1.24644799370799e-07
196 1.23340853974696e-07
197 1.22088863463432e-07
198 1.19509152796127e-07
199 1.20015229754245e-07
200 1.19360194617002e-07
201 1.1861867932339e-07
202 1.17375485331195e-07
203 1.14925214234063e-07
204 1.1526839216458e-07
205 1.1517010989337e-07
206 1.14035728415729e-07
207 1.13351667607731e-07
208 1.11875564812181e-07
209 1.10931715369134e-07
210 1.10366059402622e-07
211 1.09226448330446e-07
212 1.10114768858693e-07
213 1.07530155446511e-07
214 1.0859040600053e-07
215 1.0775998759982e-07
216 1.06765234875184e-07
217 1.06241557773501e-07
218 1.05272519590471e-07
219 1.04567291714375e-07
220 1.03734791423449e-07
221 1.03004005325147e-07
222 1.02120452538657e-07
223 1.01078882153161e-07
224 9.94110180840835e-08
225 9.85956205568073e-08
226 9.75897549437832e-08
227 9.64373541023633e-08
228 9.65250848139476e-08
229 9.57442338744841e-08
230 9.51906500290534e-08
231 9.40623650080852e-08
232 9.26473475715284e-08
233 9.24094933907327e-08
234 9.13957691750511e-08
235 9.07037218667028e-08
236 9.03436685462111e-08
237 8.96314631404493e-08
238 8.88739393190008e-08
239 8.80153692150998e-08
240 8.72452048383821e-08
241 8.6463629145328e-08
242 8.56795594472715e-08
243 8.53890327334739e-08
244 8.45557224238291e-08
245 8.385249117282e-08
246 8.35232754070603e-08
247 8.26959052346865e-08
248 8.19621988057406e-08
249 8.11802749467461e-08
250 8.04907216434003e-08
251 7.98875348095862e-08
252 7.9211304182536e-08
253 7.86082878789784e-08
254 7.79706823550441e-08
255 7.73179849034022e-08
256 7.66270247254397e-08
257 7.59548086648465e-08
258 7.54183346884929e-08
259 7.47584252280831e-08
260 7.41539949444814e-08
261 7.35865555157034e-08
262 7.2815524276848e-08
263 7.22699198263399e-08
264 7.15852479515888e-08
265 7.10905965206621e-08
266 7.06876264189304e-08
267 7.00413451681925e-08
268 6.94655781785514e-08
269 6.91017731924148e-08
270 6.8653754681236e-08
271 6.81768739241306e-08
272 6.77103670909673e-08
273 6.71886439818081e-08
274 6.66062547338697e-08
275 6.61177921301714e-08
276 6.57482175370205e-08
277 6.598875046393e-08
278 6.56884253658063e-08
279 6.53391225569067e-08
280 6.50292477644143e-08
281 6.47076277004999e-08
282 6.42770814351934e-08
283 6.3964264995775e-08
284 6.35621901778904e-08
285 6.32328962524298e-08
286 6.28341467745486e-08
287 6.25250393682109e-08
288 6.21705709136222e-08
289 6.18642204130992e-08
290 6.15204669429659e-08
291 6.1201660628285e-08
292 6.07711996281068e-08
293 6.05477978865565e-08
294 6.01625558260821e-08
295 5.96896185811602e-08
296 5.9325522272502e-08
297 5.84871209241555e-08
298 5.87735051738036e-08
299 5.84216479637689e-08
300 5.7624308880122e-08
301 5.74274636733207e-08
302 5.70121798659784e-08
303 5.68147200397107e-08
304 5.64474227360279e-08
305 5.62210082932779e-08
306 5.59792567855766e-08
307 5.57607151563388e-08
308 5.55322685613646e-08
309 5.53506254163949e-08
310 5.5488950323479e-08
311 5.46254845801286e-08
312 5.45423191056216e-08
313 5.42897566901956e-08
314 5.41399067799375e-08
315 5.39682432076916e-08
316 5.37646798193236e-08
317 5.36151247843009e-08
318 5.33532364954681e-08
319 5.33020028115061e-08
320 5.3079926232158e-08
321 5.29975210383782e-08
322 5.27932400018472e-08
323 5.26880299389632e-08
324 5.25520569283344e-08
325 5.24407326452092e-08
326 5.23460492729555e-08
327 5.22108436484814e-08
328 5.20453049546177e-08
329 5.1936673628461e-08
330 5.18179064101787e-08
331 5.17650242670697e-08
332 5.25824930264207e-08
333 5.24959347103504e-08
334 5.1518942001394e-08
335 5.12978353128801e-08
336 5.12387750006837e-08
337 5.11817468407116e-08
338 5.10771975825719e-08
339 5.0986951549703e-08
340 5.07418036477247e-08
341 5.06927548826752e-08
342 5.06216011331162e-08
343 5.05195174582695e-08
344 5.04298007797388e-08
345 5.0386717020956e-08
346 5.02457950801727e-08
347 5.01794197305117e-08
348 5.12218285564359e-08
349 5.00873937880897e-08
350 5.10635693729e-08
351 5.12296587373839e-08
352 4.97891790018912e-08
353 4.97563448220717e-08
354 5.07008124372987e-08
355 5.08808781773951e-08
356 4.95528418298363e-08
357 5.04843313819947e-08
358 4.93753340435887e-08
359 4.93178582416931e-08
360 5.03834804987946e-08
361 5.02723445094944e-08
362 4.91940497227006e-08
363 5.0111204075165e-08
364 4.91116267653524e-08
365 5.00571246675463e-08
366 4.88491345151942e-08
367 4.9818403624613e-08
368 4.8698456822649e-08
369 4.96786256576343e-08
370 4.84938880163099e-08
371 4.95297953762019e-08
372 4.85343534251115e-08
373 4.93762470910042e-08
374 4.83574567056166e-08
375 4.84850914972412e-08
376 4.89969487205144e-08
377 4.88418230304433e-08
378 4.87380056313214e-08
379 4.86635549634684e-08
380 4.85520139648088e-08
381 4.8424514176304e-08
382 4.83482693880433e-08
383 4.82885411656753e-08
384 4.8226205251467e-08
385 4.81483937164739e-08
386 4.81014943432001e-08
387 4.80775277367229e-08
388 4.80558668414233e-08
389 4.80061927987663e-08
390 4.79162771682695e-08
391 4.8029932031568e-08
392 4.78508717094428e-08
393 4.77354973327238e-08
394 4.76785615433073e-08
395 4.78276227511287e-08
396 4.76377515212789e-08
397 4.75728647586493e-08
398 4.74266883543351e-08
399 4.75272123878767e-08
400 4.74306780517963e-08
401 4.7537362490857e-08
402 4.729901448286e-08
403 4.73678447576731e-08
404 4.72067611667626e-08
405 4.71697276793748e-08
406 4.71277132874093e-08
407 4.72002206208799e-08
408 4.70062460067311e-08
409 4.70195615775992e-08
410 4.68949217236059e-08
411 4.69312517736853e-08
412 4.68223539940027e-08
413 4.68030307843037e-08
414 4.6711377876818e-08
415 4.67148453253685e-08
416 4.66301024459881e-08
417 4.6657401497896e-08
418 4.64573481906427e-08
419 4.65206646538263e-08
420 4.63341294221209e-08
421 4.63864004984771e-08
422 4.6256481311957e-08
423 4.63007872042454e-08
424 4.60931062207237e-08
425 4.61144509245059e-08
426 4.61093527803769e-08
427 4.60487434850165e-08
428 4.59030857768994e-08
429 4.59266864538677e-08
430 4.58187017216005e-08
431 4.58446614004515e-08
432 4.5598365971955e-08
433 4.55843114366417e-08
434 4.5280987848173e-08
435 4.53986821469243e-08
436 4.50713812938375e-08
437 4.51555237646062e-08
438 4.51721362537683e-08
439 4.49107382394232e-08
440 4.50456845157987e-08
441 4.47792984914486e-08
442 4.49460912932409e-08
443 4.46700845202486e-08
444 4.48267734043384e-08
445 4.45742394106219e-08
446 4.44892478412839e-08
447 4.44554792977669e-08
448 4.44227659102125e-08
449 4.45367298596011e-08
450 4.45788188585539e-08
451 4.42608225625918e-08
452 4.43657448556678e-08
453 4.41401759587734e-08
454 4.42926548771538e-08
455 4.40102354559713e-08
456 4.40062741802194e-08
457 4.41519425464776e-08
458 4.38663363411251e-08
459 4.40474146046199e-08
460 4.384657259493e-08
461 4.393638519673e-08
462 4.36305782613999e-08
463 4.35774012430556e-08
464 4.3508798341918e-08
465 4.36689617799857e-08
466 4.32934257332818e-08
467 4.33420197509804e-08
468 4.35311022783935e-08
469 4.31654960664218e-08
470 4.32011368900476e-08
471 4.34352607214805e-08
472 4.30432365305933e-08
473 4.31247286769576e-08
474 4.34332072529742e-08
475 4.29568096649291e-08
476 4.30651745375599e-08
477 4.33117719467191e-08
478 4.29135909030265e-08
479 4.30257358630115e-08
480 4.28594368884205e-08
481 4.28997672941023e-08
482 4.27109512202151e-08
483 4.26810373710396e-08
484 4.26050661417321e-08
485 4.26817550192027e-08
486 4.29373763211061e-08
487 4.25371453616208e-08
488 4.25963300187959e-08
489 4.27290274274128e-08
490 4.24075032867677e-08
491 4.24662225384509e-08
492 4.23236912183711e-08
493 4.23555235329331e-08
494 4.23161772289404e-08
495 4.23461266052527e-08
496 4.22569037539233e-08
497 4.22141823719357e-08
498 4.22786392562102e-08
499 4.21438670628049e-08
500 4.20906651754649e-08
501 4.2154727708521e-08
502 4.20215542362712e-08
503 4.19257020212171e-08
504 4.18961967341147e-08
505 4.19221137804016e-08
506 4.18246628441921e-08
507 4.17396783802815e-08
508 4.17484962156323e-08
509 4.17245402672961e-08
510 4.16391010560346e-08
511 4.16461247709776e-08
512 4.17969978627752e-08
513 4.15690912802802e-08
514 4.16920151735667e-08
515 4.1480120671622e-08
516 4.1635491498937e-08
517 4.13860128389842e-08
518 4.15494838534869e-08
519 4.1323954036443e-08
520 4.13708463042894e-08
521 4.12226022206141e-08
522 4.1155995944564e-08
523 4.11925995535967e-08
524 4.10954434926225e-08
525 4.10614084955796e-08
526 4.10654976690239e-08
527 4.10088070168513e-08
528 4.09683984514686e-08
529 4.10760208069405e-08
530 4.09527416422861e-08
531 4.09168521287029e-08
532 4.09196232453723e-08
533 4.08564275744538e-08
534 4.08788665140492e-08
535 4.07940667912499e-08
536 4.07480378328273e-08
537 4.07864746421183e-08
538 4.07114839617861e-08
539 4.06733242641621e-08
540 4.06921394358051e-08
541 4.06139797348715e-08
542 4.05891213972609e-08
543 4.06392537399825e-08
544 4.0563392644799e-08
545 4.04921642882528e-08
546 4.05381150869744e-08
547 4.05588878038543e-08
548 4.04247089136334e-08
549 4.03889153233195e-08
550 4.04924733743428e-08
551 4.03404563087406e-08
552 4.03245365987459e-08
553 4.02962250234395e-08
554 4.03650446401116e-08
555 4.02424475964835e-08
556 4.02290787349102e-08
557 4.03510185265077e-08
558 4.03162161433102e-08
559 4.01522051163283e-08
560 4.01468973620922e-08
561 4.02515745179244e-08
562 4.01594562049468e-08
563 4.00017903245953e-08
564 4.0004259460602e-08
565 4.01506348168823e-08
566 4.01539921313088e-08
567 4.0035384785142e-08
568 3.99237691794951e-08
569 3.99357524827337e-08
570 3.98115105326724e-08
571 4.0139919832427e-08
572 3.9861962619625e-08
573 3.97865456136515e-08
574 3.98920718680529e-08
575 3.97120913930848e-08
576 3.97134840568469e-08
577 3.96491692811196e-08
578 3.97195982770882e-08
579 3.9614334923499e-08
580 3.96280199765897e-08
581 3.95634387473365e-08
582 3.96087145304591e-08
583 3.95334716074558e-08
584 3.9844088917107e-08
585 3.96747275033249e-08
586 3.95049504220424e-08
587 3.94145160953485e-08
588 3.93372197038389e-08
589 3.93796497633048e-08
590 3.94203709674912e-08
591 3.94511125989538e-08
592 3.93782997321068e-08
593 3.9331808920906e-08
594 3.92792891545923e-08
595 3.93414509858303e-08
596 3.91813550493225e-08
597 3.9143134955566e-08
598 3.92022521111812e-08
599 3.9152205033588e-08
600 3.91844885427872e-08
601 3.90877801237366e-08
602 3.92186514375226e-08
603 3.91405663435762e-08
604 3.90802163963144e-08
605 3.89828649360879e-08
606 3.8992066464516e-08
607 3.89630052666234e-08
608 3.90158767515913e-08
609 3.9024950382327e-08
610 3.89051493243642e-08
611 3.90118373161386e-08
612 3.88502989778772e-08
613 3.89229626307497e-08
614 3.87695209269623e-08
615 3.88658598637903e-08
616 3.8837846716433e-08
617 3.88765464265362e-08
618 3.87698975146122e-08
619 3.87753189556861e-08
620 3.87319474270953e-08
621 3.86024829879261e-08
622 3.8716631678426e-08
623 3.86065615032294e-08
624 3.86701621835073e-08
625 3.86479435121601e-08
626 3.86079577197052e-08
627 3.87495973086516e-08
628 3.85118212875568e-08
629 3.85887481968439e-08
630 3.85458349683176e-08
631 3.85146528003588e-08
632 3.8454679440747e-08
633 3.85948553116577e-08
634 3.84280625098654e-08
635 3.84290643751228e-08
636 3.85257976631692e-08
637 3.88427103814593e-08
638 3.86801488616584e-08
639 3.86532548191099e-08
640 3.867838316296e-08
641 3.8716940764516e-08
642 3.84166760625249e-08
643 3.8376203548296e-08
644 3.8705028515551e-08
645 3.8322248485656e-08
646 3.81924785131105e-08
647 3.83108478274607e-08
648 3.85100697997132e-08
649 3.81794897919008e-08
650 3.84150347088053e-08
651 3.81364770873915e-08
652 3.84413674225925e-08
653 3.84740275194417e-08
654 3.81843641150681e-08
655 3.82926437225706e-08
656 3.82211666760668e-08
657 3.81233284940663e-08
658 3.83116933733163e-08
659 3.79905422676075e-08
660 3.81645435254541e-08
661 3.79733222644063e-08
662 3.81235665258828e-08
663 3.81851030795133e-08
664 3.81887019784699e-08
665 3.80920788245476e-08
666 3.83026232952943e-08
667 3.81513629577057e-08
668 3.81913096703101e-08
669 3.80974789493393e-08
670 3.80499649565991e-08
671 3.82418612332458e-08
672 3.80674016753346e-08
673 3.80084266282665e-08
674 3.81561875428815e-08
675 3.75557291931727e-08
676 3.80171307767796e-08
677 3.79018452179025e-08
678 3.74794986157667e-08
679 3.79710733966476e-08
680 3.78365463404862e-08
681 3.74761732757634e-08
682 3.77506736981559e-08
683 3.77182587385505e-08
684 3.773861578793e-08
685 3.77093769543535e-08
686 3.71986459413165e-08
687 3.7268389263545e-08
688 3.71659325537621e-08
689 3.72336117493433e-08
690 3.72001274229206e-08
691 3.6922994439692e-08
692 3.72137058946009e-08
693 3.71107233831935e-08
694 3.48753985690564e-08
695 3.84749441195709e-08
696 3.83905742751267e-08
697 3.82790190656124e-08
698 3.83984612994936e-08
699 3.76795235013105e-08
700 3.78738178596905e-08
701 3.78485616181479e-08
702 3.75274211705801e-08
703 3.76679736291408e-08
704 3.74368092082022e-08
705 3.821557825745e-08
706 3.76794879741738e-08
707 3.75931641372063e-08
708 3.77715601018735e-08
709 3.76858508843725e-08
710 3.7936736418942e-08
711 3.75967204035987e-08
712 3.77485314118076e-08
713 3.7409712660974e-08
714 3.77491886638381e-08
715 3.77323274847186e-08
716 3.76262541124106e-08
717 3.76875135543742e-08
718 3.76321551698311e-08
719 3.78570916836907e-08
720 3.79309419429319e-08
721 3.79973315034476e-08
722 3.80692100065971e-08
723 3.79789604210146e-08
724 3.87106346977362e-08
725 3.86722973644282e-08
726 3.87107839117107e-08
727 3.85952496628761e-08
728 3.86883804992522e-08
729 3.83428400141383e-08
730 3.85237477473765e-08
731 3.82085403316523e-08
732 3.85598255547848e-08
733 3.83900093936518e-08
734 3.84390332897055e-08
735 3.85661458324194e-08
736 3.84417866428066e-08
737 3.84684888388165e-08
738 3.84120184548919e-08
739 3.83885421229024e-08
740 3.82870126713897e-08
741 3.83531819636573e-08
742 3.8289464043828e-08
743 3.72409374449489e-08
744 3.70876982458412e-08
745 3.7131545838065e-08
746 3.69957717794023e-08
747 3.70229109591946e-08
748 3.69431454316782e-08
749 3.67835291115171e-08
750 3.66278030128342e-08
751 3.6802255465318e-08
752 3.67300465597964e-08
753 3.64320129619955e-08
754 3.65394150492193e-08
755 3.6273661407904e-08
756 3.61328673648131e-08
757 3.61664866943556e-08
758 3.6076624354564e-08
759 3.60800669341188e-08
760 3.60234437835061e-08
761 3.60645699970519e-08
762 3.60151481970661e-08
763 3.6066559516712e-08
764 3.57774467829586e-08
765 3.59543683714492e-08
766 3.56252201072493e-08
767 3.55494869097583e-08
768 3.53663409669025e-08
769 3.51710234269831e-08
770 3.49276376709895e-08
771 3.49859199388902e-08
772 3.48167823460699e-08
773 3.48335085220697e-08
774 3.46921460447902e-08
775 3.47126380972895e-08
776 3.45156614400821e-08
777 3.45994131123462e-08
778 3.44085115955295e-08
779 3.44757786763239e-08
780 3.43504389377358e-08
781 3.43737980301739e-08
782 3.42424222310456e-08
783 3.47595019434266e-08
784 3.44956880837799e-08
785 3.44385711059658e-08
786 3.42644419504268e-08
787 3.416972660375e-08
788 3.41443815443654e-08
789 3.41251542579357e-08
790 3.39006831495681e-08
791 3.40513111041219e-08
792 3.3809222088621e-08
793 3.3947614497265e-08
794 3.34812959579267e-08
795 3.37012373563539e-08
796 3.35129897166553e-08
797 3.37025554131287e-08
798 3.35335528234282e-08
799 3.35985816946049e-08
800 3.34952936498212e-08
801 3.34738849971927e-08
802 3.33315242073695e-08
803 3.33196972235328e-08
804 3.34096235121706e-08
805 3.29582974245568e-08
806 3.29581126834455e-08
807 3.29176259583619e-08
808 3.30181997298951e-08
809 3.28853175801669e-08
810 3.29048290836909e-08
811 3.28541958083406e-08
812 3.49126594301197e-08
813 3.43210544428985e-08
814 3.43279751291448e-08
815 3.34082983499684e-08
816 3.2497801782938e-08
817 3.15533625894204e-08
818 3.12930197310379e-08
819 3.08424148443009e-08
820 3.02812424024523e-08
821 3.00652374107813e-08
822 2.98015354616155e-08
823 2.96463511517686e-08
824 2.94339415063405e-08
825 2.92194943796176e-08
826 2.90407573544371e-08
827 2.89576824741289e-08
828 2.8904910465144e-08
829 2.8707718868759e-08
830 2.85725150206417e-08
831 2.86593859755158e-08
832 2.8614985936315e-08
833 2.84590075949609e-08
834 2.81899943388453e-08
835 2.83888486052319e-08
836 2.81270438051706e-08
837 2.79727228047477e-08
838 2.79726375396194e-08
839 2.79978848993778e-08
840 2.78119998142756e-08
841 2.76520708553107e-08
842 2.75851412823158e-08
843 2.76897775819407e-08
844 2.74225087082414e-08
845 2.75885785328001e-08
846 2.7543970659849e-08
847 2.72978422088954e-08
848 2.74148170831268e-08
849 2.73315681198483e-08
850 2.73354299196171e-08
851 2.73068110345775e-08
852 2.72527866940209e-08
853 2.72350444419089e-08
854 2.71557798470212e-08
855 2.71330442558337e-08
856 2.70840772031988e-08
857 2.70645426070359e-08
858 2.69929731899765e-08
859 2.68564619432254e-08
860 2.70012350256366e-08
861 2.69705111577423e-08
862 2.69479674130935e-08
863 2.67922519725516e-08
864 2.65977604385625e-08
865 2.66451767316767e-08
866 2.66339981180863e-08
867 2.6577104961234e-08
868 2.65210093886026e-08
869 2.64333159805119e-08
870 2.63976058789694e-08
871 2.65062070070599e-08
872 2.64391779580819e-08
873 2.64223487533854e-08
874 2.63666013466946e-08
875 2.63756163576545e-08
876 2.63753889839791e-08
877 2.62259600702919e-08
878 2.6279680653829e-08
879 2.63281094703416e-08
880 2.62027590736125e-08
881 2.62309960419316e-08
882 2.62587018795557e-08
883 2.6243567319284e-08
884 2.61047610194964e-08
885 2.61538293244712e-08
886 2.62223469604805e-08
887 2.6081306003789e-08
888 2.62343267110055e-08
889 2.60252868145017e-08
890 2.61501202913905e-08
891 2.6158598842585e-08
892 2.60215102798611e-08
893 2.60257984052714e-08
894 2.59995065476915e-08
895 2.6109168160815e-08
896 2.59485908316037e-08
897 2.59665657864616e-08
898 2.60578385535837e-08
899 2.60250185846189e-08
900 2.59485410936122e-08
901 2.59353534204365e-08
902 2.59938932600789e-08
903 2.59842689587231e-08
904 2.58835886057796e-08
905 2.58436774203119e-08
906 2.58570729272378e-08
907 2.58975898503877e-08
908 2.59009702574531e-08
909 2.58730956659292e-08
910 2.58643240158563e-08
911 2.58659440532938e-08
912 2.58551899889881e-08
913 2.59053258844233e-08
914 2.57747885257231e-08
915 2.57687435833986e-08
916 2.58293297861201e-08
917 2.57682497561973e-08
918 2.57335361908417e-08
919 2.57354830779377e-08
920 2.56786965024958e-08
921 2.57145593707264e-08
922 2.56976715462542e-08
923 2.56957228828014e-08
924 2.56709817847423e-08
925 2.56568384315869e-08
926 2.55943692906158e-08
927 2.55957672834484e-08
928 2.55235121926489e-08
929 2.56566146106252e-08
930 2.54726479909095e-08
931 2.55950443062147e-08
932 2.54349821204869e-08
933 2.54449545877833e-08
934 2.54803307342399e-08
935 2.54750638362111e-08
936 2.54098644347778e-08
937 2.54926373344233e-08
938 2.54206540262203e-08
939 2.53008369810459e-08
940 2.53637981728616e-08
941 2.54034251412349e-08
942 2.52550709234356e-08
943 2.52841374503987e-08
944 2.53514471637573e-08
945 2.54070329219758e-08
946 2.53809115946524e-08
947 2.53168721542352e-08
948 2.53143568329506e-08
949 2.5253827473648e-08
950 2.51826914876574e-08
951 2.51616896207452e-08
952 2.52184868543281e-08
953 2.52860949956357e-08
954 2.52271394884929e-08
955 2.52028335978594e-08
956 2.51933212069844e-08
957 2.51867753320312e-08
958 2.51878589097032e-08
959 2.51925253991203e-08
960 2.50973659632336e-08
961 2.5147588900154e-08
962 2.50883722685558e-08
963 2.5120998614625e-08
964 2.51689407093636e-08
965 2.5029338601712e-08
966 2.51003733353627e-08
967 2.49141862695978e-08
968 2.49650504713372e-08
969 2.49352059000785e-08
970 2.49164475718544e-08
971 2.51571616871615e-08
972 2.5003721759731e-08
973 2.49860150347558e-08
974 2.50709284443928e-08
975 2.49155878151441e-08
976 2.50151988012703e-08
977 2.49690295106575e-08
978 2.49734295465487e-08
979 2.48055300744454e-08
980 2.47752200976947e-08
981 2.48888465392838e-08
982 2.47428069144462e-08
983 2.4844922563716e-08
984 2.48453773110668e-08
985 2.48461695662172e-08
986 2.48115021861395e-08
987 2.47597160552004e-08
988 2.46045175344989e-08
989 2.46419418203914e-08
990 2.48108502631794e-08
991 2.46757103639084e-08
992 2.44624338563426e-08
993 2.47835956201925e-08
994 2.47974920597471e-08
995 2.48857787710222e-08
996 2.4741437343323e-08
997 2.47377816009475e-08
998 2.47296263466978e-08
999 2.46611868703894e-08
1000 2.48952005676983e-08
1001 2.47261766617157e-08
1002 2.46325271291425e-08
1003 2.46920279778351e-08
1004 2.46409577187023e-08
1005 2.46748950161191e-08
1006 2.46779414680987e-08
1007 2.47087328375528e-08
1008 2.46926976643635e-08
1009 2.43755788886801e-08
1010 2.46168756490306e-08
1011 2.47406681808116e-08
1012 2.4372992513122e-08
1013 2.46448124130438e-08
1014 2.46537332770913e-08
1015 2.473307603168e-08
1016 2.43801832056079e-08
1017 2.45466420523144e-08
1018 2.45579023783193e-08
1019 2.4564288381157e-08
1020 2.47822615762061e-08
1021 2.45444695678998e-08
1022 2.43970657010095e-08
1023 2.45354261352304e-08
1024 2.4534225318007e-08
1025 2.45671873955189e-08
1026 2.43328219795558e-08
1027 2.47470772762881e-08
1028 2.46548488291864e-08
1029 2.43522766396609e-08
1030 2.45505873408547e-08
1031 2.46338256459921e-08
1032 2.43039597336292e-08
1033 2.42586253307309e-08
1034 2.4303021817218e-08
1035 2.44690667727809e-08
1036 2.43042688197193e-08
1037 2.41529534150686e-08
1038 2.43614319828112e-08
1039 2.42748967593798e-08
1040 2.4408386423147e-08
1041 2.44133691040815e-08
1042 2.46285445371086e-08
1043 2.44004425553612e-08
1044 2.45404052634512e-08
1045 2.44886564360058e-08
1046 2.44700668616815e-08
1047 2.44677682559313e-08
1048 2.45170230783742e-08
1049 2.44149536143823e-08
1050 2.44610216526553e-08
1051 2.44594424714251e-08
1052 2.44852706998699e-08
1053 2.45013342947686e-08
1054 2.44468978394252e-08
1055 2.44366891166692e-08
1056 2.44488962408695e-08
1057 2.43786573150828e-08
1058 2.44207001287577e-08
1059 2.43895623697199e-08
1060 2.43745681416385e-08
1061 2.43253168719093e-08
1062 2.43416842238275e-08
1063 2.43366375940468e-08
1064 2.43218956086366e-08
1065 2.43101858643513e-08
1066 2.45540157095547e-08
1067 2.4328848269306e-08
1068 2.43199806959637e-08
1069 2.43177975534081e-08
1070 2.43475337668997e-08
1071 2.43318414305804e-08
1072 2.43098359220539e-08
1073 2.42857822740916e-08
1074 2.4308729251743e-08
1075 2.42860043186965e-08
1076 2.42602435918116e-08
1077 2.42659403681955e-08
1078 2.42856881271791e-08
1079 2.42601458921854e-08
1080 2.4247828633861e-08
1081 2.42101592107247e-08
1082 2.42439188724575e-08
1083 2.42181616982862e-08
1084 2.42072708545038e-08
1085 2.41847608606349e-08
1086 2.4195333736543e-08
1087 2.41591298077992e-08
1088 2.4144531707293e-08
1089 2.41684823265587e-08
1090 2.41549908963634e-08
1091 2.40900561720991e-08
1092 2.41358257824231e-08
1093 2.41278161894343e-08
1094 2.41303581560715e-08
1095 2.41039526116538e-08
1096 2.41332518413628e-08
1097 2.40936248729895e-08
1098 2.40981936627804e-08
1099 2.41261250977232e-08
1100 2.41234552333935e-08
1101 2.40889459490745e-08
1102 2.4435363954467e-08
1103 2.43581776970814e-08
1104 2.40279280916411e-08
1105 2.43145787948151e-08
1106 2.43211530914778e-08
1107 2.4048048885561e-08
1108 2.39317277106466e-08
1109 2.404005172707e-08
1110 2.42284237117474e-08
1111 2.42300348674007e-08
1112 2.43998830029568e-08
1113 2.43284805634403e-08
1114 2.40390498618126e-08
1115 2.39032580395815e-08
1116 2.38167388033617e-08
1117 2.42694220276007e-08
1118 2.42260664862215e-08
1119 2.4223229644349e-08
1120 2.42515554305101e-08
1121 2.43539979294383e-08
1122 2.40739712609184e-08
1123 2.44870186349999e-08
1124 2.47204496872655e-08
1125 2.44061020282516e-08
1126 2.43569608926464e-08
1127 2.41471997952658e-08
1128 2.42548985340818e-08
1129 2.46345628340805e-08
1130 2.43991298276569e-08
1131 2.46131595105226e-08
1132 2.38942128305553e-08
1133 2.38101200977781e-08
1134 2.42425244323385e-08
1135 2.43336746308387e-08
1136 2.4288908662129e-08
1137 2.42924738103056e-08
1138 2.42225599578205e-08
1139 2.42724080834478e-08
1140 2.39646293920259e-08
1141 2.42751383439099e-08
1142 2.43015332301866e-08
1143 2.42993021259963e-08
1144 2.43049775860982e-08
1145 2.41920457000333e-08
1146 2.43186342174795e-08
1147 2.42609594636178e-08
1148 2.43209665740096e-08
1149 2.41515856203023e-08
1150 2.42511966064285e-08
1151 2.41412223545012e-08
1152 2.42312729881178e-08
1153 2.41966535696747e-08
1154 2.42623698909483e-08
1155 2.43080240380777e-08
1156 2.43400375410374e-08
1157 2.4173356649726e-08
1158 2.43997781979033e-08
1159 2.42038691311564e-08
1160 2.4182060798239e-08
1161 2.43586395498596e-08
1162 2.41951507717886e-08
1163 2.42748381396041e-08
1164 2.4269143139577e-08
1165 2.43104523178772e-08
1166 2.42512108172832e-08
1167 2.42861002419659e-08
1168 2.41744899653895e-08
1169 2.42346782641789e-08
1170 2.38423041309943e-08
1171 2.42092834668028e-08
1172 2.4199604098385e-08
1173 2.40820412500398e-08
1174 2.42479618606239e-08
1175 2.40113475769022e-08
1176 2.4185656144482e-08
1177 2.41588153926386e-08
1178 2.41674182888119e-08
1179 2.4142320143028e-08
1180 2.40422775021898e-08
1181 2.41612863050022e-08
1182 2.42153728180483e-08
1183 2.42180497878053e-08
1184 2.41724560368084e-08
1185 2.45250983965661e-08
1186 2.39655673084371e-08
1187 2.38082797920924e-08
1188 2.39135022894743e-08
1189 2.38795188067797e-08
1190 2.36431318967334e-08
1191 2.38583357514699e-08
1192 2.38028512455912e-08
1193 2.38326585133564e-08
1194 2.37079387233052e-08
1195 2.37672743708117e-08
1196 2.41129143319085e-08
1197 2.39994388806508e-08
1198 2.3976452823149e-08
1199 2.41302728909432e-08
1200 2.40044730759337e-08
1201 2.40005686436007e-08
1202 2.41188189420427e-08
1203 2.39695321369027e-08
1204 2.39488500142215e-08
1205 2.39403163959651e-08
1206 2.3851786323803e-08
1207 2.3897257506178e-08
1208 2.3978012464454e-08
1209 2.39323387773993e-08
1210 2.3820883043868e-08
1211 2.39259794199143e-08
1212 2.39486528386124e-08
1213 2.39408954882947e-08
1214 2.39363089349354e-08
1215 2.39217730069186e-08
1216 2.38266100183182e-08
1217 2.3859827891215e-08
1218 2.38635102789431e-08
1219 2.40394530948151e-08
1220 2.38773623095767e-08
1221 2.38583801603909e-08
1222 2.39515447475469e-08
1223 2.38868445023854e-08
1224 2.38142980890643e-08
1225 2.38038886379854e-08
1226 2.3857770869995e-08
1227 2.36932891084507e-08
1228 2.3773797153126e-08
1229 2.38002293428963e-08
1230 2.38069706171018e-08
1231 2.35542270132783e-08
1232 2.38322925838474e-08
1233 2.37876616182575e-08
1234 2.38255086770778e-08
1235 2.3749263888817e-08
1236 2.36343087323121e-08
1237 2.37675976677565e-08
1238 2.37389361501528e-08
1239 2.37126922542075e-08
1240 2.37267077096703e-08
1241 2.36777903950269e-08
1242 2.36908999085017e-08
1243 2.3488672340477e-08
1244 2.3657205971972e-08
1245 2.36698269873159e-08
1246 2.36948398679715e-08
1247 2.36525252717001e-08
1248 2.36530883768182e-08
1249 2.3586018471633e-08
1250 2.36493402638871e-08
1251 2.37492265853234e-08
1252 2.38435884369892e-08
1253 2.38130244412105e-08
1254 2.38819701792181e-08
1255 2.38156498966191e-08
1256 2.37299211391928e-08
1257 2.31199130951154e-08
1258 2.37821176085617e-08
1259 2.38533850449585e-08
1260 2.3830594386709e-08
1261 2.36597710312481e-08
1262 2.38307702460361e-08
1263 2.38725519352556e-08
1264 2.38518875761429e-08
1265 2.38019755016694e-08
1266 2.36779946760635e-08
1267 2.36132589037652e-08
1268 2.33625225831702e-08
1269 2.32387904475218e-08
1270 2.32814070244558e-08
1271 2.32645547271204e-08
1272 2.32997034999016e-08
1273 2.31438335163148e-08
1274 2.33180728059779e-08
1275 2.32235315422713e-08
1276 2.32841710356979e-08
1277 2.32091217355901e-08
1278 2.32709442826717e-08
1279 2.32106511788288e-08
1280 2.33138930383348e-08
1281 2.32035599623259e-08
1282 2.32531736088504e-08
1283 2.31519887705645e-08
1284 2.32463577276576e-08
1285 2.32054446769325e-08
1286 2.32123475996104e-08
1287 2.32289849577683e-08
1288 2.32250361165143e-08
1289 2.32386589971156e-08
1290 2.32318591031344e-08
1291 2.32756143248025e-08
1292 2.32530172894485e-08
1293 2.32344454786926e-08
1294 2.32264891764089e-08
1295 2.32350245710222e-08
1296 2.32248247300504e-08
1297 2.3237459956249e-08
1298 2.32097026042766e-08
1299 2.3250080971593e-08
1300 2.31923014126778e-08
1301 2.3239019597554e-08
1302 2.32313599468625e-08
1303 2.32183197113045e-08
1304 2.31596857247496e-08
1305 2.31724985866322e-08
1306 2.32259989019212e-08
1307 2.31562893304726e-08
1308 2.32613484030253e-08
1309 2.31904095926438e-08
1310 2.31565664421396e-08
1311 2.23351541706052e-08
1312 2.22964935403525e-08
1313 2.22403127025927e-08
1314 2.23927081322017e-08
1315 2.2348185524379e-08
1316 2.23380833830333e-08
1317 2.24039631291362e-08
1318 2.23403908705677e-08
1319 2.240557961386e-08
1320 2.23365628215788e-08
1321 2.23108731489674e-08
1322 2.23206590987957e-08
1323 2.22780798253552e-08
1324 2.22396536742053e-08
1325 2.2340973515611e-08
1326 2.22530101012808e-08
1327 2.36225137228985e-08
1328 2.23094538398527e-08
1329 2.23725571402156e-08
1330 2.22666187710274e-08
1331 2.2305977509518e-08
1332 2.22767670976509e-08
1333 2.26285905569057e-08
1334 2.23078160388468e-08
1335 2.23479386107783e-08
1336 2.22803340221844e-08
1337 2.32913688336112e-08
1338 2.27975025524074e-08
1339 2.27871055358264e-08
1340 2.27235563698969e-08
1341 2.27311431899579e-08
1342 2.31002807993264e-08
1343 2.27226308879835e-08
1344 2.31954970786319e-08
1345 2.30570282866438e-08
1346 2.2775672903208e-08
1347 2.31043184584223e-08
1348 2.27349623571627e-08
1349 2.31169217101979e-08
1350 2.30907843956629e-08
1351 2.26546781334491e-08
1352 2.27518022200002e-08
1353 2.26286935856024e-08
1354 2.29795968920143e-08
1355 2.26565717298399e-08
1356 2.26424656801782e-08
1357 2.30054730820939e-08
1358 2.2998149162845e-08
1359 2.27011476283678e-08
1360 2.29709709032022e-08
1361 2.30023484704134e-08
1362 2.26771401656833e-08
1363 2.29510597193894e-08
1364 2.23160050438764e-08
1365 2.26399130554e-08
1366 2.26169039052593e-08
1367 2.25584368962473e-08
1368 2.3090429124295e-08
1369 2.30107506382637e-08
1370 2.27662173557519e-08
1371 2.2674942812273e-08
1372 2.30041976578832e-08
1373 2.29906067517049e-08
1374 2.2614418782041e-08
1375 2.31030199415727e-08
1376 2.25964491562536e-08
1377 2.25387193353299e-08
1378 2.25458016700486e-08
1379 2.25940937070845e-08
1380 2.29204228929802e-08
1381 2.22565645913164e-08
1382 2.29211725155665e-08
1383 2.26453416019012e-08
1384 2.25378453677649e-08
1385 2.25570655487672e-08
1386 2.29167813614595e-08
1387 2.29383392280624e-08
1388 2.26259242452898e-08
1389 2.25626557437408e-08
1390 2.29028813691912e-08
1391 2.25987726309995e-08
1392 2.25118128582835e-08
1393 2.25277521082035e-08
1394 2.2850134229202e-08
1395 2.25956959809537e-08
1396 2.25098197859097e-08
1397 2.25172964718467e-08
1398 2.29124363926303e-08
1399 2.28773888721889e-08
1400 2.25782130769403e-08
1401 2.30525944999727e-08
1402 2.25374208184803e-08
1403 2.25182841262495e-08
1404 2.28949836866832e-08
1405 2.25470273562678e-08
1406 2.24350404920415e-08
1407 2.28249099620825e-08
1408 2.24373586377169e-08
1409 2.24939746829023e-08
1410 2.28361596299465e-08
1411 2.25179306312384e-08
1412 2.24569376428008e-08
1413 2.24725091868549e-08
1414 2.28914043276518e-08
1415 2.24102212342814e-08
1416 2.23991847292382e-08
1417 2.28904006860375e-08
1418 2.24721503627734e-08
1419 2.28562200277338e-08
1420 2.24062670639569e-08
1421 2.2769501839548e-08
1422 2.24872476195515e-08
1423 2.24385541258698e-08
1424 2.24252580949269e-08
1425 2.20454001720327e-08
1426 2.25383676166757e-08
1427 2.24371223822573e-08
1428 2.23976179825058e-08
1429 2.21031406510974e-08
1430 2.24934524339915e-08
1431 2.2442268488021e-08
1432 2.23262066612051e-08
1433 2.2406254629459e-08
1434 2.23395009157912e-08
1435 2.19559055381069e-08
1436 2.24263985160178e-08
1437 2.24359180123201e-08
1438 2.23173124425102e-08
1439 2.27373710970369e-08
1440 2.24431957462912e-08
1441 2.19069189455467e-08
1442 2.25077272375529e-08
1443 2.28243166588982e-08
1444 2.27899477067695e-08
1445 2.2694276680113e-08
1446 2.23783889197193e-08
1447 2.23468301641105e-08
1448 2.23063665316658e-08
1449 2.27077094905326e-08
1450 2.2282074851887e-08
1451 2.22938894012259e-08
1452 2.22683027573112e-08
1453 2.22255387427595e-08
1454 2.25617586835369e-08
1455 2.23489013961853e-08
1456 2.18980247268519e-08
1457 2.25842207157712e-08
1458 2.15813216186689e-08
1459 2.21429683477936e-08
1460 2.21037161907134e-08
1461 2.22106653069432e-08
1462 2.21796057076062e-08
1463 2.2153670897751e-08
1464 2.27097540772547e-08
1465 2.19824212166486e-08
1466 2.21665032995588e-08
1467 2.25226628458586e-08
1468 2.21741380812546e-08
1469 2.26336727138232e-08
1470 2.21226077457004e-08
1471 2.21582467929693e-08
1472 2.19170654958134e-08
1473 2.15980975326602e-08
1474 2.2191477100364e-08
1475 2.21360902941115e-08
1476 2.16907736216854e-08
1477 2.20195417455216e-08
1478 2.25136478348986e-08
1479 2.25076401960678e-08
1480 2.21185327831108e-08
1481 2.21499725228114e-08
1482 2.20994298416599e-08
1483 2.16678106568224e-08
1484 2.25949143839443e-08
1485 2.19829789926962e-08
1486 2.22581117981235e-08
1487 2.25008456311571e-08
1488 2.25333565140318e-08
1489 2.21139035971873e-08
1490 2.20434976938577e-08
1491 2.21207372419485e-08
1492 2.20320721666667e-08
1493 2.20643059378745e-08
1494 2.20131344264018e-08
1495 2.24739569176791e-08
1496 2.19501679055156e-08
1497 2.24874430188038e-08
1498 2.23987122183189e-08
1499 2.2484675454848e-08
1500 2.24238512203101e-08
1501 2.20383853388739e-08
1502 2.19932410061574e-08
1503 2.1969354335738e-08
1504 2.24667999759731e-08
1505 2.19442277682447e-08
1506 2.23586305025947e-08
1507 2.20961808850006e-08
1508 2.20125961902795e-08
1509 2.20124789507281e-08
1510 2.19743263585315e-08
1511 2.19723599315103e-08
1512 2.19797726686011e-08
1513 2.20045226484444e-08
1514 2.19335412054988e-08
1515 2.24137135518276e-08
1516 2.20264748662657e-08
1517 2.19921822974811e-08
1518 2.1971688468625e-08
1519 2.19303384341174e-08
1520 2.19392770617333e-08
1521 2.13594475440004e-08
1522 2.18923297268248e-08
1523 2.19303633031132e-08
1524 2.19164490999901e-08
1525 2.19167670678644e-08
1526 2.19269651324794e-08
1527 2.18885567448979e-08
1528 2.18958433606531e-08
1529 2.18893720926872e-08
1530 2.19223608155517e-08
1531 2.19006519586173e-08
1532 2.22837428509592e-08
1533 2.19295568371081e-08
1534 2.18242544036684e-08
1535 2.1886741308208e-08
1536 2.19117701760752e-08
1537 2.18402487206504e-08
1538 2.19037037396674e-08
1539 2.18872635571188e-08
1540 2.17207087871429e-08
1541 2.21056293270294e-08
1542 2.18570903598447e-08
1543 2.1846977560358e-08
1544 2.21718998716369e-08
1545 2.18224283088375e-08
1546 2.21633733588078e-08
1547 2.22801403992889e-08
1548 2.17602345031764e-08
1549 2.23159108969639e-08
1550 2.18411173591448e-08
1551 2.18282174557771e-08
1552 2.21491074370306e-08
1553 2.22337650512827e-08
1554 2.17968825211301e-08
1555 2.19239293386408e-08
1556 2.17695603765833e-08
1557 2.20655600458031e-08
1558 2.2222435447361e-08
1559 2.17890328002568e-08
1560 2.22481677525366e-08
1561 2.18593498857445e-08
1562 2.20882920842769e-08
1563 2.18388684913862e-08
1564 2.18145128627611e-08
1565 2.26110383749756e-08
1566 2.17574900318596e-08
1567 2.18086810832574e-08
1568 2.20733813449669e-08
1569 2.11768735880469e-08
1570 2.2174111435902e-08
1571 2.16135802588724e-08
1572 2.18264446516514e-08
1573 2.17259596979602e-08
1574 2.19943299129e-08
1575 2.1715797160482e-08
1576 2.15737845365993e-08
1577 2.21698144287075e-08
1578 2.11731734367504e-08
1579 2.15938662506687e-08
1580 2.10374366815813e-08
1581 2.10798951627567e-08
1582 2.16941504760371e-08
1583 2.25209131343718e-08
1584 2.15847997253604e-08
1585 2.16885638337772e-08
1586 2.15472937270533e-08
1587 2.15846860385227e-08
1588 2.17313651518225e-08
1589 2.13342143950968e-08
1590 2.15963726901691e-08
1591 2.16603037728191e-08
1592 2.14113260454951e-08
1593 2.20726938948701e-08
1594 2.20475229184558e-08
1595 2.11283754936176e-08
1596 2.16498339256077e-08
1597 2.10342552264819e-08
1598 2.16717648271469e-08
1599 2.1572903463607e-08
1600 2.1480589751377e-08
1601 2.08855812644515e-08
1602 2.14490061267725e-08
1603 2.09280663909794e-08
1604 2.16194031565919e-08
1605 2.18432276710701e-08
1606 2.18896225590015e-08
1607 2.13686082162212e-08
1608 2.14070645654374e-08
1609 2.17368985033772e-08
1610 2.16841957723091e-08
1611 2.12463895366e-08
1612 2.14109299179199e-08
1613 2.19308198268209e-08
1614 2.17667235347108e-08
1615 2.17087023912654e-08
1616 2.17958682213748e-08
1617 2.17818811876214e-08
1618 2.13655759750964e-08
1619 2.14606892257052e-08
1620 2.16556568233273e-08
1621 2.12766426699318e-08
1622 2.13694715256452e-08
1623 2.12169659619121e-08
1624 2.07828616538563e-08
1625 2.14794422248588e-08
1626 2.13832080930842e-08
1627 2.19606022255903e-08
1628 2.10637765007959e-08
1629 2.13088338085754e-08
1630 2.13819131289483e-08
1631 2.12672066624009e-08
1632 2.11910862191189e-08
1633 2.08096579967787e-08
1634 2.07904360394195e-08
1635 2.13390300984884e-08
1636 2.18795115358716e-08
1637 2.12358610696128e-08
1638 2.12182307279818e-08
1639 2.1212631651224e-08
1640 2.06674837244236e-08
1641 2.10647463916303e-08
1642 2.10391846167113e-08
1643 2.13922071168327e-08
1644 2.14463877767912e-08
1645 2.07159374099319e-08
1646 2.09668780115635e-08
1647 2.05293115840277e-08
1648 2.10311803527929e-08
1649 2.05858938784331e-08
1650 2.10841655245986e-08
1651 2.10128874300608e-08
1652 2.08510400057094e-08
1653 2.12196731297354e-08
1654 2.05080432635896e-08
1655 2.12412576416909e-08
1656 2.05130437080925e-08
1657 2.07491908099655e-08
1658 2.08048369643166e-08
1659 2.07600923118889e-08
1660 2.10766639696658e-08
1661 2.07650945327487e-08
1662 2.13640767299239e-08
1663 2.06631156629555e-08
1664 2.06677803760158e-08
1665 2.09829504882464e-08
1666 2.0930228217253e-08
1667 2.05662882279967e-08
1668 2.08945110102832e-08
1669 2.06957082582448e-08
1670 2.06367705146704e-08
1671 2.04169072759441e-08
1672 2.05192236535368e-08
1673 2.07520791661864e-08
1674 2.07933226192836e-08
1675 2.02725747300292e-08
1676 2.01753262984994e-08
1677 2.00399981054034e-08
1678 2.0234207198655e-08
1679 2.01543581823671e-08
1680 2.03874339632648e-08
1681 2.01027283708299e-08
1682 1.99284393431753e-08
1683 2.00315817266983e-08
1684 2.00165981567579e-08
1685 1.98343261814671e-08
1686 2.02103951352228e-08
1687 1.99441743120587e-08
1688 2.00352481272148e-08
1689 1.97820178016173e-08
1690 1.97500362730807e-08
1691 1.99823144697575e-08
1692 1.94955962484755e-08
1693 1.99513188192668e-08
1694 1.99031280345707e-08
1695 2.0297651559531e-08
1696 1.97366478715821e-08
1697 1.99298391123648e-08
1698 1.99317309323988e-08
1699 1.99437302228489e-08
1700 1.98044194377189e-08
1701 1.99755412211289e-08
1702 1.98505052395603e-08
1703 2.00159480101547e-08
1704 1.99304963643954e-08
1705 2.00026484264981e-08
1706 1.97998826223511e-08
1707 1.97772465071466e-08
1708 1.98953031826932e-08
1709 1.99790051169657e-08
1710 1.99715408655265e-08
1711 1.95684837223098e-08
1712 1.98260323713839e-08
1713 1.99492280472668e-08
1714 1.9944968343566e-08
1715 1.95812983605492e-08
1716 1.97926119938074e-08
1717 1.99399163847147e-08
1718 1.97722371808595e-08
1719 1.98613001600734e-08
1720 1.98635010661974e-08
1721 1.95935037083927e-08
1722 1.96263325591417e-08
1723 1.98265688311494e-08
1724 1.98732639233867e-08
1725 1.98797636130621e-08
1726 1.93377935886474e-08
1727 1.98861194178335e-08
1728 1.998309429041e-08
1729 1.98680147889263e-08
1730 1.99031564562802e-08
1731 2.00090006785558e-08
1732 1.98767331482941e-08
1733 2.00169463226985e-08
1734 1.99916456722349e-08
1735 1.98694589670367e-08
1736 1.97961629311294e-08
1737 2.01342871264387e-08
1738 1.99663645616965e-08
1739 1.99249452492722e-08
1740 1.99782856924458e-08
1741 2.00035774611251e-08
1742 1.9834393683027e-08
1743 1.95183282869493e-08
1744 1.96134610774834e-08
1745 1.99295921987641e-08
1746 1.97677803015495e-08
1747 1.97171470261992e-08
1748 1.98766407777384e-08
1749 1.97404137480817e-08
1750 1.98718250743468e-08
1751 1.99407299561472e-08
1752 1.9745696633322e-08
1753 1.98715657262483e-08
1754 1.98898160164163e-08
1755 2.01278744782485e-08
1756 1.98673184570453e-08
1757 1.98116936189763e-08
1758 1.97321181616417e-08
1759 1.95118747825518e-08
1760 1.98645881965831e-08
1761 1.96029805721309e-08
1762 1.97357277187393e-08
1763 1.97119121025935e-08
1764 1.97955198899535e-08
1765 1.93655758096156e-08
1766 1.97062330897779e-08
1767 1.95931004753902e-08
1768 1.97110576749537e-08
1769 1.91768663171388e-08
1770 1.97586977890296e-08
1771 1.96010834230265e-08
1772 1.9642190096647e-08
1773 1.94864622216073e-08
1774 1.93835205664072e-08
1775 1.95001739200507e-08
1776 1.97137985935569e-08
1777 1.98203107260042e-08
1778 1.98178824462047e-08
1779 1.96773015659346e-08
1780 1.9577905518986e-08
1781 1.95497058541605e-08
1782 1.93876790177683e-08
1783 1.95632736676998e-08
1784 1.91657214543284e-08
1785 1.96034246613408e-08
1786 1.94861762281562e-08
1787 1.97161238446597e-08
1788 1.96312193168069e-08
1789 1.94839948619574e-08
1790 1.94385734175739e-08
1791 1.9514317273206e-08
1792 1.93401223924639e-08
1793 1.97740099849852e-08
1794 1.94474907289077e-08
1795 1.9351590552219e-08
1796 1.94471585501788e-08
1797 1.92390849917956e-08
1798 1.94792555419099e-08
1799 1.94350153748246e-08
1800 1.92410674060284e-08
1801 1.92944042964882e-08
1802 1.93499349876447e-08
1803 1.93698284078891e-08
1804 1.93907609968846e-08
1805 1.93951184002117e-08
1806 1.96741787306109e-08
1807 1.90072917405359e-08
1808 1.91488709333498e-08
1809 1.93512885715563e-08
1810 1.94798346342395e-08
1811 1.92631564033263e-08
1812 1.95307947592482e-08
1813 1.92872136040023e-08
1814 1.99946352807956e-08
1815 1.92899420881076e-08
1816 1.93349425359202e-08
1817 1.9287604402507e-08
1818 1.86737008078808e-08
1819 1.90868174598791e-08
1820 1.93883504806536e-08
1821 1.8611403973523e-08
1822 1.92038029922514e-08
1823 1.9134622775141e-08
1824 1.91671993832188e-08
1825 1.92118339015224e-08
1826 1.92380618102561e-08
1827 1.88945215029435e-08
1828 1.93750544497107e-08
1829 1.91374063263083e-08
1830 1.88680591151069e-08
1831 1.96635259186451e-08
1832 1.98327860800873e-08
1833 1.92973406143437e-08
1834 1.90365430086104e-08
1835 1.92182412206421e-08
1836 1.92517113362101e-08
1837 1.97916492084005e-08
1838 1.96348981518213e-08
1839 1.92668121457018e-08
1840 1.91055971043852e-08
1841 1.91764915058457e-08
1842 1.91773885660496e-08
1843 1.9709329279749e-08
1844 1.98317327004816e-08
1845 1.94710541023824e-08
1846 1.92762499295895e-08
1847 1.90278726108772e-08
1848 1.8862428063926e-08
1849 1.8963291381624e-08
1850 1.92666291809473e-08
1851 1.91198097354572e-08
1852 1.85636555016799e-08
1853 1.91482456557424e-08
1854 1.88478619378429e-08
1855 1.92942017918085e-08
1856 1.90250428744321e-08
1857 1.93407529991418e-08
1858 1.95908054223537e-08
1859 1.97907894516902e-08
1860 1.89831066421675e-08
1861 1.91403710658733e-08
1862 1.87398931927873e-08
1863 1.87767170700681e-08
1864 1.88833624292783e-08
1865 1.90760491847186e-08
1866 1.90695548241138e-08
1867 1.90442950298575e-08
1868 1.90388860232815e-08
1869 1.88566158243475e-08
1870 1.88242914589409e-08
1871 1.87140223317783e-08
1872 1.87379338711935e-08
1873 1.91625328938017e-08
1874 1.92912139596046e-08
1875 1.92845615032411e-08
1876 1.87187243483322e-08
1877 1.9225044667337e-08
1878 1.91374702751546e-08
1879 1.889472578398e-08
1880 1.91198203935983e-08
1881 1.92963049983064e-08
1882 1.90653572929023e-08
1883 1.90834406055274e-08
1884 1.91088460610445e-08
1885 1.89370474856787e-08
1886 1.90788416176702e-08
1887 1.87867339462855e-08
1888 1.89064124356264e-08
1889 1.89817921381064e-08
1890 1.89729885136103e-08
1891 1.83217068183694e-08
1892 1.90203817140855e-08
1893 1.89694535634999e-08
1894 1.89018543039765e-08
1895 1.92376372609715e-08
1896 1.91044442487964e-08
1897 1.8886916919314e-08
1898 1.88408524337547e-08
1899 1.90032007907348e-08
1900 1.89531128569342e-08
1901 1.89298905439728e-08
1902 1.87903523851674e-08
1903 1.86975306348813e-08
1904 1.84573298867008e-08
1905 1.83071708903526e-08
1906 1.90082491968724e-08
1907 1.90876967565146e-08
1908 1.87260909001452e-08
1909 1.84453643470306e-08
1910 1.81713648572668e-08
1911 1.89110131998405e-08
1912 1.8957772240924e-08
1913 1.82102137813445e-08
1914 1.83320221225358e-08
1915 1.87185822397851e-08
1916 1.87442950050354e-08
1917 1.86781097255562e-08
1918 1.87115745120536e-08
1919 1.85016837406238e-08
1920 1.88066557882394e-08
1921 1.85641724215202e-08
1922 1.8655956779412e-08
1923 1.87308142329812e-08
1924 1.85881532388521e-08
1925 1.87132727091921e-08
1926 1.82071371312986e-08
1927 1.80332584420739e-08
1928 1.87515798444338e-08
1929 1.86918338584974e-08
1930 1.88185431682086e-08
1931 1.85107769112847e-08
1932 1.86677411306846e-08
1933 1.85541786379417e-08
1934 1.84418293969202e-08
1935 1.85086808102142e-08
1936 1.8618715458274e-08
1937 1.85816961817409e-08
1938 1.84763120358866e-08
1939 1.85478850056597e-08
1940 1.8688806946443e-08
1941 1.85110593520221e-08
1942 1.85540596220335e-08
1943 1.85466664248679e-08
1944 1.79150561052666e-08
1945 1.85671922281472e-08
1946 1.86609607766286e-08
1947 1.87329316503337e-08
1948 1.85095760940612e-08
1949 1.85385147233319e-08
1950 1.84224067112382e-08
1951 1.79345018835875e-08
1952 1.85299686705775e-08
1953 1.83435613365646e-08
1954 1.82756512145943e-08
1955 1.79573280689738e-08
1956 1.83074888582269e-08
1957 1.84211401688117e-08
1958 1.85998363377848e-08
1959 1.85557968990224e-08
1960 1.83823871680033e-08
1961 1.84169124395339e-08
1962 1.83291142263897e-08
1963 1.85075066383433e-08
1964 1.86799855583786e-08
1965 1.8483209629494e-08
1966 1.83993673630312e-08
1967 1.82413621985233e-08
1968 1.75892278519996e-08
1969 1.7672455498996e-08
1970 1.828328599629e-08
1971 1.85018667053782e-08
1972 1.84030337635477e-08
1973 1.86025612691765e-08
1974 1.84687607429623e-08
1975 1.83594330849246e-08
1976 1.86358075637827e-08
1977 1.83866948333389e-08
1978 1.86668973611859e-08
1979 1.82877055721065e-08
1980 1.82923685088099e-08
1981 1.84462489727366e-08
1982 1.85207973402157e-08
1983 1.82768644663156e-08
1984 1.84450694717953e-08
1985 1.85063555591114e-08
1986 1.83521500218831e-08
1987 1.83590120883537e-08
1988 1.82548216542955e-08
1989 1.82499189094187e-08
1990 1.83262649500193e-08
1991 1.82202430920597e-08
1992 1.83469737180531e-08
1993 1.82975483653536e-08
1994 1.83108976870017e-08
1995 1.82816748406367e-08
1996 1.82590422781459e-08
1997 1.82640675916446e-08
1998 1.82994135400349e-08
1999 1.84204704822832e-08
2000 1.83166051215267e-08
2001 1.82825132810649e-08
2002 1.83274213583218e-08
2003 1.81005059829431e-08
2004 1.81476984550955e-08
2005 1.83189392544136e-08
2006 1.82285422312134e-08
2007 1.78046288823452e-08
2008 1.80182624376357e-08
2009 1.82627406530855e-08
2010 1.82545889515495e-08
2011 1.83130559605615e-08
2012 1.80468369137543e-08
2013 1.82464905407187e-08
2014 1.81644725927299e-08
2015 1.81706951707383e-08
2016 1.81435009238839e-08
2017 1.81458688075509e-08
2018 1.82381114655072e-08
2019 1.80667711902061e-08
2020 1.77005450296974e-08
2021 1.79381363096809e-08
2022 1.81386639042103e-08
2023 1.80648118686122e-08
2024 1.78794739014165e-08
2025 1.80258989956883e-08
2026 1.81608754701301e-08
2027 1.80386159343016e-08
2028 1.77434973380741e-08
2029 1.80465953292241e-08
2030 1.74806107366976e-08
2031 1.82079507027311e-08
2032 1.80055117482425e-08
2033 1.8147261471313e-08
2034 1.81607155980146e-08
2035 1.83043979973263e-08
2036 1.83282384824679e-08
2037 1.84788326862417e-08
2038 1.81355410688866e-08
2039 1.83817654431095e-08
2040 1.8065444251647e-08
2041 1.81999268988875e-08
2042 1.79560970536841e-08
2043 1.80441048769353e-08
2044 1.824736095557e-08
2045 1.79848651526981e-08
2046 1.74374399364297e-08
2047 1.80985058051419e-08
2048 1.80036767716274e-08
2049 1.81450872105415e-08
2050 1.81076558192217e-08
2051 1.80876309485711e-08
2052 1.81236785579131e-08
2053 1.76766885573443e-08
2054 1.81790422715267e-08
2055 1.79888353102342e-08
2056 1.75712635552827e-08
2057 1.81314678826539e-08
2058 1.8011601099488e-08
2059 1.78570740416717e-08
2060 1.81007386856891e-08
2061 1.8088130104843e-08
2062 1.78994099542251e-08
2063 1.80397865534587e-08
2064 1.74407066566573e-08
2065 1.80637247382265e-08
2066 1.78552035379198e-08
2067 1.76038756904973e-08
2068 1.73131748937294e-08
2069 1.79870607297516e-08
2070 1.80225363521913e-08
2071 1.79608630190842e-08
2072 1.72717360413799e-08
2073 1.77506933596305e-08
2074 1.80607333533089e-08
2075 1.81343917660115e-08
2076 1.79644477071861e-08
2077 1.79500592167869e-08
2078 1.79904464658875e-08
2079 1.79681389766984e-08
2080 1.74892651472192e-08
2081 1.80339281286024e-08
2082 1.78386496685334e-08
2083 1.79568804270502e-08
2084 1.80263413085413e-08
2085 1.78831438546467e-08
2086 1.8010295477211e-08
2087 1.78209980106203e-08
2088 1.7953189157538e-08
2089 1.75193726192902e-08
2090 1.77973031867396e-08
2091 1.80051706877293e-08
2092 1.7886138792278e-08
2093 1.78155730168328e-08
2094 1.80347381473211e-08
2095 1.77816268376318e-08
2096 1.7684312680899e-08
2097 1.77732779604867e-08
2098 1.80136563443511e-08
2099 1.79682615453203e-08
2100 1.7879170144397e-08
2101 1.77993868533122e-08
2102 1.80568431318306e-08
2103 1.72833036771181e-08
2104 1.80337362820637e-08
2105 1.81361787809919e-08
2106 1.74927503593381e-08
2107 1.78232166803127e-08
2108 1.78005894468924e-08
2109 1.7057333323578e-08
2110 1.78641315073946e-08
2111 1.80156991547165e-08
2112 1.80336048316576e-08
2113 1.81926882447669e-08
2114 1.80250072645549e-08
2115 1.78284533802753e-08
2116 1.80971468921598e-08
2117 1.82004491477983e-08
2118 1.80520753900737e-08
2119 1.80238135527588e-08
2120 1.79489099139118e-08
2121 1.77770811404798e-08
2122 1.79555055268565e-08
2123 1.79175714265511e-08
2124 1.80899757395991e-08
2125 1.80273644900808e-08
2126 1.82105424073598e-08
2127 1.80722370402009e-08
2128 1.79302208636045e-08
2129 1.82545338844875e-08
2130 1.7982522138027e-08
2131 1.80797119497811e-08
2132 1.79117360943337e-08
2133 1.80626269496997e-08
2134 1.79640515796109e-08
2135 1.78828472030546e-08
2136 1.76077250557682e-08
2137 1.72237477613635e-08
2138 1.71024687745103e-08
2139 1.78658385863173e-08
2140 1.76530470241687e-08
2141 1.78075332257777e-08
2142 1.79317396487022e-08
2143 1.78793939653588e-08
2144 1.76871282064894e-08
2145 1.74733418845108e-08
2146 1.72677854237691e-08
2147 1.7931421680828e-08
2148 1.78785448667895e-08
2149 1.73493752697595e-08
2150 1.79938606237329e-08
2151 1.72821970068071e-08
2152 1.79804917621595e-08
2153 1.79150649870508e-08
2154 1.70795377840705e-08
2155 1.69388894022404e-08
2156 1.74651813011906e-08
2157 1.73522050062047e-08
2158 1.69742921940497e-08
2159 1.69176850306485e-08
2160 1.72757417260527e-08
2161 1.75838295035646e-08
2162 1.78495458413863e-08
2163 1.70374701013998e-08
2164 1.68569265213137e-08
2165 1.76238224014469e-08
2166 1.75422858461616e-08
2167 1.77449290816867e-08
2168 1.78014527563164e-08
2169 1.67486433610975e-08
2170 1.67778306803257e-08
2171 1.6822424342422e-08
2172 1.73059770958162e-08
2173 1.72986940327746e-08
2174 1.67186993138557e-08
2175 1.68371911968279e-08
2176 1.68301834690965e-08
2177 1.7346760472492e-08
2178 1.74804650754368e-08
2179 1.69508478364833e-08
2180 1.70506169183682e-08
2181 1.75418843895159e-08
2182 1.68746758788529e-08
2183 1.71306115959169e-08
2184 1.69309153363884e-08
2185 1.70066591920204e-08
2186 1.70213478867254e-08
2187 1.75112138123268e-08
2188 1.70453500203394e-08
2189 1.74387615459182e-08
2190 1.75510255218114e-08
2191 1.68595288840834e-08
2192 1.71775624835391e-08
2193 1.72580012502976e-08
2194 1.72329279735095e-08
2195 1.71187153341634e-08
2196 1.69632645707907e-08
2197 1.70732370463611e-08
2198 1.69741021238679e-08
2199 1.70636713647809e-08
2200 1.74283201204162e-08
2201 1.7472387980888e-08
2202 1.75296328563945e-08
2203 1.7436573074292e-08
2204 1.68282543455689e-08
2205 1.68133489353295e-08
2206 1.68851226334255e-08
2207 1.70016569711606e-08
2208 1.70806568888793e-08
2209 1.69482383682862e-08
2210 1.6805799418762e-08
2211 1.68950187173778e-08
2212 1.68466058880767e-08
2213 1.70074141436771e-08
2214 1.68970046843242e-08
2215 1.68910609943396e-08
2216 1.68346172557676e-08
2217 1.74948500131222e-08
2218 1.73698975203251e-08
2219 1.67545195495222e-08
2220 1.70662861620485e-08
2221 1.71078760047294e-08
2222 1.69191292087589e-08
2223 1.68766671748699e-08
2224 1.69380793835217e-08
2225 1.68534306510537e-08
2226 1.68380758225339e-08
2227 1.69261031857104e-08
2228 1.67929599115269e-08
2229 1.69208380640384e-08
2230 1.69360472312974e-08
2231 1.6919017298278e-08
2232 1.69102598590598e-08
2233 1.69063429922289e-08
2234 1.74256395979455e-08
2235 1.73699152838935e-08
2236 1.67849343313264e-08
2237 1.68980829329257e-08
2238 1.69375660163951e-08
2239 1.75092704779445e-08
2240 1.69019109819146e-08
2241 1.69576885866718e-08
2242 1.69788219039901e-08
2243 1.67619642610362e-08
2244 1.68392979560394e-08
2245 1.68824563218095e-08
2246 1.69777454317455e-08
2247 1.69995875154427e-08
2248 1.68055684923729e-08
2249 1.68585803095311e-08
2250 1.67441616127917e-08
2251 1.68287517254839e-08
2252 1.67896061498141e-08
2253 1.65934039841886e-08
2254 1.68117360033193e-08
2255 1.68343845530217e-08
2256 1.67894835811921e-08
2257 1.68050213744664e-08
2258 1.74519350082392e-08
2259 1.74260392782344e-08
2260 1.72867196113202e-08
2261 1.66547327040689e-08
2262 1.71843907992297e-08
2263 1.74197030133882e-08
2264 1.70991683035027e-08
2265 1.71291212325286e-08
2266 1.73579550732939e-08
2267 1.7100425964145e-08
2268 1.64848987793675e-08
2269 1.65699329812696e-08
2270 1.65155071840672e-08
2271 1.65871476554003e-08
2272 1.67797331585007e-08
2273 1.6685635984004e-08
2274 1.66356919351074e-08
2275 1.66889240205137e-08
2276 1.69697464968976e-08
2277 1.68502722885933e-08
2278 1.66622466934996e-08
2279 1.6771588562392e-08
2280 1.67242983906135e-08
2281 1.66496487707946e-08
2282 1.66652647237697e-08
2283 1.66703149062641e-08
2284 1.65369744564714e-08
2285 1.66050959649056e-08
2286 1.67566067688085e-08
2287 1.70977543234585e-08
2288 1.65552993536267e-08
2289 1.6684927217625e-08
2290 1.74263519170381e-08
2291 1.71409588745064e-08
2292 1.71286060890452e-08
2293 1.71231846479714e-08
2294 1.65410511954178e-08
2295 1.6633030952562e-08
2296 1.75186976036912e-08
2297 1.71956973105125e-08
2298 1.64787028467117e-08
2299 1.64427493842823e-08
2300 1.71157381601006e-08
2301 1.7195016965843e-08
2302 1.620746736819e-08
2303 1.71748055777243e-08
2304 1.6604637664841e-08
2305 1.7419594655621e-08
2306 1.71726064479572e-08
2307 1.69721907639087e-08
2308 1.70834564272582e-08
2309 1.67752709501201e-08
2310 1.72361396266751e-08
2311 1.69864815546816e-08
2312 1.69125407012416e-08
2313 1.6884154518948e-08
2314 1.69482579082114e-08
2315 1.66610512053467e-08
2316 1.70811542687943e-08
2317 1.68568572433969e-08
2318 1.69245435444054e-08
2319 1.70624190332092e-08
2320 1.68269007616573e-08
2321 1.66191789219283e-08
2322 1.66538232093671e-08
2323 1.69508229674875e-08
2324 1.69496168211936e-08
2325 1.69095883961745e-08
2326 1.69048011144923e-08
2327 1.69526543913889e-08
2328 1.642524338763e-08
2329 1.70105138863619e-08
2330 1.6759686971568e-08
2331 1.66097979814595e-08
2332 1.6705158145669e-08
2333 1.68455294158321e-08
2334 1.70191132298214e-08
2335 1.64318283424336e-08
2336 1.64895652687846e-08
2337 1.68686611345947e-08
2338 1.63542868136801e-08
2339 1.70474478977667e-08
2340 1.62488955623985e-08
2341 1.70660499065889e-08
2342 1.70135550092709e-08
2343 1.68835470049089e-08
2344 1.66518567823459e-08
2345 1.67469593748137e-08
2346 1.68280624990302e-08
2347 1.63392250840388e-08
2348 1.69547416106752e-08
2349 1.69081513234914e-08
2350 1.64952496106707e-08
2351 1.57350577012494e-08
2352 1.59235629126897e-08
2353 1.60581894448342e-08
2354 1.61587259128737e-08
2355 1.62328284147861e-08
2356 1.69322671439431e-08
2357 1.63332156688512e-08
2358 1.68786513654595e-08
2359 1.69151004314472e-08
2360 1.68909046749377e-08
2361 1.61384559049793e-08
2362 1.63198876634851e-08
2363 1.62767825884202e-08
2364 1.62084994315137e-08
2365 1.62594169239583e-08
2366 1.68203655448451e-08
2367 1.67593512401254e-08
2368 1.68453357929366e-08
2369 1.60199373766545e-08
2370 1.69670677507838e-08
2371 1.68333897931916e-08
2372 1.71690786032741e-08
2373 1.62096149836088e-08
2374 1.70263874110788e-08
2375 1.67421863039863e-08
2376 1.62201594378075e-08
2377 1.62269184755814e-08
2378 1.70024350154563e-08
2379 1.63629199079196e-08
2380 1.68368128328211e-08
2381 1.67218860980256e-08
2382 1.61152069466652e-08
2383 1.60736224330549e-08
2384 1.60023869710813e-08
2385 1.61779514229465e-08
2386 1.67330096445539e-08
2387 1.61685171917725e-08
2388 1.62528586145072e-08
2389 1.62343063436765e-08
2390 1.69884177836366e-08
2391 1.68318425863845e-08
2392 1.55456962858125e-08
2393 1.62374238499297e-08
2394 1.68215077422929e-08
2395 1.65828559772763e-08
2396 1.68704765712846e-08
2397 1.66670517387502e-08
2398 1.66106186583193e-08
2399 1.65897109383195e-08
2400 1.6648714407097e-08
2401 1.65193583256951e-08
2402 1.64329776453087e-08
2403 1.57205555240125e-08
2404 1.59971325075503e-08
2405 1.59162620860798e-08
2406 1.60823852013436e-08
2407 1.66993583405883e-08
2408 1.68105263043117e-08
2409 1.65378999383847e-08
2410 1.60136508497999e-08
2411 1.5819400900341e-08
2412 1.62820672500175e-08
2413 1.55732831075284e-08
2414 1.59587489889645e-08
2415 1.59934252508265e-08
2416 1.6602491825779e-08
2417 1.65320255263168e-08
2418 1.56685153740455e-08
2419 1.57040371817629e-08
2420 1.58072861466962e-08
2421 1.59064210691895e-08
2422 1.635538282585e-08
2423 1.61439182022605e-08
2424 1.53918158218858e-08
2425 1.63037370271013e-08
2426 1.66650142574554e-08
2427 1.56120076866273e-08
2428 1.55220334363548e-08
2429 1.57002624234792e-08
2430 1.57690926982923e-08
2431 1.58349706680383e-08
2432 1.56037440746104e-08
2433 1.64717217643329e-08
2434 1.64081743747602e-08
2435 1.58973403330265e-08
2436 1.5375453799038e-08
2437 1.56991930566619e-08
2438 1.62882081156113e-08
2439 1.58331765476305e-08
2440 1.57402197942247e-08
2441 1.54752761716281e-08
2442 1.65188076550749e-08
2443 1.62464406372465e-08
2444 1.66344040763988e-08
2445 1.59758730688964e-08
2446 1.565173057827e-08
2447 1.56103219239867e-08
2448 1.54011345898653e-08
2449 1.67887055368965e-08
2450 1.58410387030017e-08
2451 1.64334768015806e-08
2452 1.56652895100251e-08
2453 1.55003334612047e-08
2454 1.55786601396812e-08
2455 1.58554875895334e-08
2456 1.61513842300565e-08
2457 1.58657709192767e-08
2458 1.67553935170872e-08
2459 1.69234990465839e-08
2460 1.56292845332473e-08
2461 1.58262203342474e-08
2462 1.64856626128085e-08
2463 1.57500110731235e-08
2464 1.56874744305924e-08
2465 1.59313753300694e-08
2466 1.54725423584523e-08
2467 1.67197757861004e-08
2468 1.6695711479997e-08
2469 1.60609996413541e-08
2470 1.58387170046126e-08
2471 1.58188626642186e-08
2472 1.58858242116366e-08
2473 1.58624420265596e-08
2474 1.58034474395663e-08
2475 1.60794346726334e-08
2476 1.54556278886275e-08
2477 1.56365711490025e-08
2478 1.55871724416556e-08
2479 1.52276129483653e-08
2480 1.60068154286819e-08
2481 1.51329899722441e-08
2482 1.53777381939335e-08
2483 1.68761626895275e-08
2484 1.55815342850474e-08
2485 1.57154005364646e-08
2486 1.59899968821264e-08
2487 1.64085101062028e-08
2488 1.56894603975388e-08
2489 1.64909277344805e-08
2490 1.56019890340531e-08
2491 1.68213141193974e-08
2492 1.58232236202593e-08
2493 1.65704623356078e-08
2494 1.58773492131559e-08
2495 1.55838399962249e-08
2496 1.56139012830181e-08
2497 1.54726631507174e-08
2498 1.5951256315816e-08
2499 1.60146989003351e-08
2500 1.53353187926086e-08
2501 1.68384524101839e-08
2502 1.63196833824486e-08
2503 1.59695243695523e-08
2504 1.57578430304284e-08
2505 1.59491442275339e-08
2506 1.58957611517963e-08
2507 1.50029553225295e-08
2508 1.54310431099702e-08
2509 1.55778501209625e-08
2510 1.60752975375544e-08
2511 1.54003512164991e-08
2512 1.58742086142638e-08
2513 1.58167310360113e-08
2514 1.65425664278018e-08
2515 1.65261155871121e-08
2516 1.55583101957291e-08
2517 1.52002463948975e-08
2518 1.59181965386779e-08
2519 1.53986992046384e-08
2520 1.61600048897981e-08
2521 1.52870658354232e-08
2522 1.57972177561305e-08
2523 1.56871315937224e-08
2524 1.53840016281492e-08
2525 1.62235647138687e-08
2526 1.49466110599406e-08
2527 1.56839181641999e-08
2528 1.51611505572191e-08
2529 1.64424136528396e-08
2530 1.52422376942241e-08
2531 1.55382480215849e-08
2532 1.58414863449252e-08
2533 1.61165214507264e-08
2534 1.53386494616825e-08
2535 1.59176316572029e-08
2536 1.53571875216585e-08
2537 1.57858597304994e-08
2538 1.55711532556779e-08
2539 1.5301059974604e-08
2540 1.53917625311806e-08
2541 1.59854351977629e-08
2542 1.56375214999116e-08
2543 1.67198326295193e-08
2544 1.61385447228213e-08
2545 1.57426267577421e-08
2546 1.53284638315654e-08
2547 1.67679843343649e-08
2548 1.55637991383628e-08
2549 1.53938000124754e-08
2550 1.60342992217011e-08
2551 1.53872132813149e-08
2552 1.54382249206719e-08
2553 1.63613584902578e-08
2554 1.62640834133754e-08
2555 1.52646535411805e-08
2556 1.54278350095183e-08
2557 1.57486486074276e-08
2558 1.64372817579306e-08
2559 1.60305599905541e-08
2560 1.54539758767669e-08
2561 1.52954466869915e-08
2562 1.529848248083e-08
2563 1.54129189411378e-08
2564 1.50934695852811e-08
2565 1.54199693014334e-08
2566 1.55377204436036e-08
2567 1.59029909241326e-08
2568 1.5578928369564e-08
2569 1.58330646371496e-08
2570 1.52718744317326e-08
2571 1.52394274977041e-08
2572 1.64766493782054e-08
2573 1.56678385820896e-08
2574 1.59053605841564e-08
2575 1.51483110499839e-08
2576 1.69945888472967e-08
2577 1.55212660502002e-08
2578 1.66360880626826e-08
2579 1.57123825061944e-08
2580 1.53507038191947e-08
2581 1.54090269433027e-08
2582 1.54112758110614e-08
2583 1.52917678519771e-08
2584 1.53118158152665e-08
2585 1.54443497990542e-08
2586 1.50566226153614e-08
2587 1.52737626990529e-08
2588 1.60385624781156e-08
2589 1.51409516035983e-08
2590 1.54223904758055e-08
2591 1.50225254458292e-08
2592 1.50655505848363e-08
2593 1.59818682732293e-08
2594 1.55197064088952e-08
2595 1.52462771296769e-08
2596 1.55283412794915e-08
2597 1.54476609282028e-08
2598 1.55099844079132e-08
2599 1.55016941505437e-08
2600 1.53954697879044e-08
2601 1.53636285915582e-08
2602 1.53585482109975e-08
2603 1.52945265341486e-08
2604 1.52036658818133e-08
2605 1.59142192757145e-08
2606 1.52250603235871e-08
2607 1.53708548111808e-08
2608 1.61757025551879e-08
2609 1.63609215064753e-08
2610 1.52683217180538e-08
2611 1.54213513070545e-08
2612 1.5679603393437e-08
2613 1.51627403965904e-08
2614 1.55896930920107e-08
2615 1.53670427494035e-08
2616 1.58604223088332e-08
2617 1.51597880915233e-08
2618 1.53115937706616e-08
2619 1.57478670104183e-08
2620 1.52669805686401e-08
2621 1.56329083011997e-08
2622 1.52838222078344e-08
2623 1.48948622324951e-08
2624 1.5978354639401e-08
2625 1.50605394821923e-08
2626 1.52439625367151e-08
2627 1.53706771754969e-08
2628 1.51755497057593e-08
2629 1.50334678039599e-08
2630 1.52664796360114e-08
2631 1.59850568337561e-08
2632 1.49359475898336e-08
2633 1.48597862903443e-08
2634 1.49053089870677e-08
2635 1.61535371745458e-08
2636 1.53114374512597e-08
2637 1.49802730220472e-08
2638 1.55427137826791e-08
2639 1.54938977203756e-08
2640 1.54806247820716e-08
2641 1.54092987258991e-08
2642 1.5113490903218e-08
2643 1.5369868933135e-08
2644 1.54695882770284e-08
2645 1.53211630049555e-08
2646 1.56350470348343e-08
2647 1.54622394887838e-08
2648 1.54161217125193e-08
2649 1.54802872742721e-08
2650 1.48750620709848e-08
2651 1.50392924780363e-08
2652 1.51383900970359e-08
2653 1.55121622213983e-08
2654 1.54907766614087e-08
2655 1.56062416323266e-08
2656 1.55490003095338e-08
2657 1.58078350409596e-08
2658 1.52685579735135e-08
2659 1.53710608685742e-08
2660 1.54540842345341e-08
2661 1.53530272939406e-08
2662 1.52327075397807e-08
2663 1.5610417847256e-08
2664 1.55055239758894e-08
2665 1.54765125159884e-08
2666 1.5597045432969e-08
2667 1.53045540685071e-08
2668 1.52512402706861e-08
2669 1.48674947908489e-08
2670 1.52749812798447e-08
2671 1.54752584080597e-08
2672 1.53444901229705e-08
2673 1.49311709662925e-08
2674 1.50318477665223e-08
2675 1.49511389935242e-08
2676 1.51137697912418e-08
2677 1.53059449559123e-08
2678 1.54844830291267e-08
2679 1.51897392441924e-08
2680 1.52835291089559e-08
2681 1.49478758260102e-08
2682 1.50401717746718e-08
2683 1.53992321116903e-08
2684 1.55607278173875e-08
2685 1.52856021173875e-08
2686 1.5540370768008e-08
2687 1.53375427913716e-08
2688 1.49810901461933e-08
2689 1.50858845415769e-08
2690 1.56598130018892e-08
2691 1.52769761285754e-08
2692 1.54533861262962e-08
2693 1.52080126269993e-08
2694 1.48711905012533e-08
2695 1.52502668271381e-08
2696 1.52833994349066e-08
2697 1.53296273452952e-08
2698 1.52167789480018e-08
2699 1.52857300150799e-08
2700 1.53341499498083e-08
2701 1.52653178986384e-08
2702 1.53843746630855e-08
2703 1.58052788634677e-08
2704 1.54478492220278e-08
2705 1.53511354739067e-08
2706 1.52696824073928e-08
2707 1.51767665101943e-08
2708 1.51979495655041e-08
2709 1.52169796763246e-08
2710 1.51868100317643e-08
2711 1.53533967761632e-08
2712 1.50528496334346e-08
2713 1.534338878173e-08
2714 1.49217935785373e-08
2715 1.54428079213176e-08
2716 1.54559085530082e-08
2717 1.50734127402075e-08
2718 1.54782551220478e-08
2719 1.51436569950647e-08
2720 1.54973935906355e-08
2721 1.50997117032148e-08
2722 1.5291369948045e-08
2723 1.53008539172106e-08
2724 1.52283892163041e-08
2725 1.49546810490619e-08
2726 1.53528656454682e-08
2727 1.55186263839369e-08
2728 1.53258117308042e-08
2729 1.52755248450376e-08
2730 1.53739598829361e-08
2731 1.51442929308132e-08
2732 1.52642325446095e-08
2733 1.51577914664358e-08
2734 1.49660799309004e-08
2735 1.49862096066045e-08
2736 1.51908547962876e-08
2737 1.52894852334384e-08
2738 1.52385712937075e-08
2739 1.54348107628266e-08
2740 1.52481529624993e-08
2741 1.52614187953759e-08
2742 1.51318566565806e-08
2743 1.54185020306841e-08
2744 1.4902374445569e-08
2745 1.54405750407705e-08
2746 1.5303990963389e-08
2747 1.53242218914329e-08
2748 1.5265399611053e-08
2749 1.51149670557515e-08
2750 1.48456411608322e-08
2751 1.55117714228936e-08
2752 1.53036854300126e-08
2753 1.48926950771511e-08
2754 1.52943702147468e-08
2755 1.51055967734237e-08
2756 1.49766208323854e-08
2757 1.49045380481994e-08
2758 1.50236072471444e-08
2759 1.47654857229895e-08
2760 1.50591894509944e-08
2761 1.49370134039373e-08
2762 1.52201486969261e-08
2763 1.50226657780195e-08
2764 1.4771812217873e-08
2765 1.50364289908111e-08
2766 1.49634331592097e-08
2767 1.48999639293379e-08
2768 1.50537715626342e-08
2769 1.50538941312561e-08
2770 1.49670214000253e-08
2771 1.52308405887425e-08
2772 1.50052699154912e-08
2773 1.48814338629677e-08
2774 1.50793262321258e-08
2775 1.50469396942299e-08
2776 1.51380561419501e-08
2777 1.50950096866609e-08
2778 1.49148373651542e-08
2779 1.52656909335747e-08
2780 1.50391059605681e-08
2781 1.53741179786948e-08
2782 1.52066483849467e-08
2783 1.53432750948923e-08
2784 1.50242360774655e-08
2785 1.50732706316603e-08
2786 1.50902170759082e-08
2787 1.50678634014412e-08
2788 1.5232549444022e-08
2789 1.51493839695149e-08
2790 1.5175618983676e-08
2791 1.51167807160846e-08
2792 1.47568917086005e-08
2793 1.50611860760819e-08
2794 1.51631880385139e-08
2795 1.52298476052692e-08
2796 1.49823957684703e-08
2797 1.51841188511526e-08
2798 1.53892223409002e-08
2799 1.52691637111957e-08
2800 1.48114143172506e-08
2801 1.48746384098786e-08
2802 1.50095313955489e-08
2803 1.5217285209701e-08
2804 1.5194212110714e-08
2805 1.48525911569664e-08
2806 1.52411772091909e-08
2807 1.50233514517595e-08
2808 1.50939918341919e-08
2809 1.49165249041516e-08
2810 1.52940540232294e-08
2811 1.48089958074138e-08
2812 1.5247437090693e-08
2813 1.51619019561622e-08
2814 1.50263783638138e-08
2815 1.49559422624179e-08
2816 1.4692993488552e-08
2817 1.52641863593317e-08
2818 1.50114036756577e-08
2819 1.47618139934025e-08
2820 1.5039073986145e-08
2821 1.49817136474439e-08
2822 1.48135148592132e-08
2823 1.49898280454863e-08
2824 1.52461971936191e-08
2825 1.49847920738466e-08
2826 1.48418157763786e-08
2827 1.51427670402882e-08
2828 1.53381769507632e-08
2829 1.52589816337922e-08
2830 1.52684318521779e-08
2831 1.51404044856918e-08
2832 1.51428221073502e-08
2833 1.49625591916447e-08
2834 1.51335139975117e-08
2835 1.50192693837425e-08
2836 1.50767487383519e-08
2837 1.5120559027082e-08
2838 1.47854422039018e-08
2839 1.51872825426835e-08
2840 1.48724241810783e-08
2841 1.50492844852579e-08
2842 1.48776919672855e-08
2843 1.47297773978039e-08
2844 1.52440069456361e-08
2845 1.50403636212104e-08
2846 1.49913095270904e-08
2847 1.51331516207165e-08
2848 1.4925324975934e-08
2849 1.52427457322801e-08
2850 1.52419588062003e-08
2851 1.51449661700553e-08
2852 1.52942227771291e-08
2853 1.51546561966143e-08
2854 1.48589762716256e-08
2855 1.51917518564915e-08
2856 1.48375489672503e-08
2857 1.51090517874763e-08
2858 1.45786884786503e-08
2859 1.50403440812852e-08
2860 1.50566616952119e-08
2861 1.49303982510673e-08
2862 1.52220209770348e-08
2863 1.48689691670256e-08
2864 1.48698697799432e-08
2865 1.44757965614417e-08
2866 1.42152112303506e-08
2867 1.46312109094993e-08
2868 1.47112393378279e-08
2869 1.4279088134117e-08
2870 1.56900821224326e-08
2871 1.49390526615889e-08
2872 1.41427047850584e-08
2873 1.50676324750521e-08
2874 1.47617251755605e-08
2875 1.44578740091106e-08
2876 1.44808378621519e-08
2877 1.49598040621868e-08
2878 1.43484406578409e-08
2879 1.4450104224295e-08
2880 1.4304550433053e-08
2881 1.43653595685578e-08
2882 1.42636791267137e-08
2883 1.43120431062016e-08
2884 1.43372345107196e-08
2885 1.43125120644072e-08
2886 1.41248737151045e-08
2887 1.42940272951364e-08
2888 1.43067815372433e-08
2889 1.43795384488499e-08
2890 1.49256553783061e-08
2891 1.43614231618017e-08
2892 1.42487674992253e-08
2893 1.45457246247815e-08
2894 1.41080116478065e-08
2895 1.42047840157034e-08
2896 1.51620245247841e-08
2897 1.41886733473484e-08
2898 1.39182647629355e-08
2899 1.43336347235845e-08
2900 1.39335201154722e-08
2901 1.39752138750282e-08
2902 1.46584868687683e-08
2903 1.44017322512013e-08
2904 1.41009222076605e-08
2905 1.4130349335062e-08
2906 1.41971803202523e-08
2907 1.46547600721192e-08
2908 1.43408440678172e-08
2909 1.46282905788553e-08
2910 1.42961802396258e-08
2911 1.41116247576178e-08
2912 1.41685019272586e-08
2913 1.42655531831792e-08
2914 1.41903573336322e-08
2915 1.43798208895873e-08
2916 1.42335299102569e-08
2917 1.43723690726461e-08
2918 1.41898102157256e-08
2919 1.44768392829064e-08
2920 1.41260390051912e-08
2921 1.43081768655406e-08
2922 1.42314382500786e-08
2923 1.44092222598147e-08
2924 1.40926736946767e-08
2925 1.40952414184881e-08
2926 1.42115341716931e-08
2927 1.42430067739951e-08
2928 1.40889024891067e-08
2929 1.4035690831804e-08
2930 1.40393030534369e-08
2931 1.40945193294328e-08
2932 1.436668473076e-08
2933 1.44741099106227e-08
2934 1.41608476056376e-08
2935 1.40334384113316e-08
2936 1.39791760389585e-08
2937 1.40752209887296e-08
2938 1.41119604890605e-08
2939 1.43998324375616e-08
2940 1.44038700966576e-08
2941 1.39839801960306e-08
2942 1.43132492524956e-08
2943 1.38493856383093e-08
2944 1.4176969820312e-08
2945 1.41527998209767e-08
2946 1.39696663126188e-08
2947 1.43520804130048e-08
2948 1.41899656469491e-08
2949 1.38504674396245e-08
2950 1.44011211844486e-08
2951 1.40755895827738e-08
2952 1.39184130887315e-08
2953 1.42373437483911e-08
2954 1.42618095111402e-08
2955 1.42639722255922e-08
2956 1.39937350596142e-08
2957 1.41812943610375e-08
2958 1.40592240072124e-08
2959 1.42553329141037e-08
2960 1.41472353831773e-08
2961 1.39956197742208e-08
2962 1.4005130388739e-08
2963 1.40242999435714e-08
2964 1.4665347158882e-08
2965 1.47312544385159e-08
2966 1.42765737010109e-08
2967 1.44598990559075e-08
2968 1.71025753559206e-08
2969 1.45319623001683e-08
2970 1.43028922039434e-08
2971 1.49077372668671e-08
2972 1.46032661518802e-08
2973 1.54410031427687e-08
2974 1.53106753941756e-08
2975 1.39262095188997e-08
2976 1.75277072855806e-08
2977 1.63115583262652e-08
2978 1.46343728246734e-08
2979 1.49464280951861e-08
2980 1.49091263779155e-08
2981 1.39816895838862e-08
2982 1.42470533148753e-08
2983 1.5512950923835e-08
2984 1.52735495362322e-08
2985 1.50582284419443e-08
2986 1.41835911904309e-08
2987 1.44424259218567e-08
2988 1.43693563714464e-08
2989 1.66292135617141e-08
2990 1.5712974033022e-08
2991 1.43953631237537e-08
2992 1.41769911365941e-08
2993 1.63259183949549e-08
2994 1.40511806634436e-08
2995 1.42164697791713e-08
2996 1.42637928135514e-08
2997 1.42340175202094e-08
2998 1.394467208371e-08
2999 1.64112012868145e-08
3000 1.1294223334346e-08
3001 1.13244889021757e-08
3002 1.12691980191926e-08
3003 1.13040723448421e-08
3004 1.13129923207111e-08
3005 1.13049454242287e-08
3006 1.13149818403713e-08
3007 1.13125908640654e-08
3008 1.13096811915625e-08
3009 1.13063078899245e-08
3010 1.13060121265107e-08
3011 1.1294915225335e-08
3012 1.13003952861845e-08
3013 1.12972839971803e-08
3014 1.12946239028133e-08
3015 1.12909841476494e-08
3016 1.1288460832759e-08
3017 1.12860796264158e-08
3018 1.12828502096818e-08
3019 1.12812408303853e-08
3020 1.12771605387252e-08
3021 1.12721378897618e-08
3022 1.12722222667117e-08
3023 1.12663682827474e-08
3024 1.12639098048817e-08
3025 1.12622489112368e-08
3026 1.12611839853116e-08
3027 1.12588365297484e-08
3028 1.12590408107849e-08
3029 1.12541380659081e-08
3030 1.12534896956618e-08
3031 1.12501812310484e-08
3032 1.1248603826175e-08
3033 1.12470344149074e-08
3034 1.12455138534528e-08
3035 1.12458566903229e-08
3036 1.1242083708396e-08
3037 1.12477174241121e-08
3038 1.12441727040391e-08
3039 1.12452260836449e-08
3040 1.12420286413339e-08
3041 1.12543787622599e-08
3042 1.1250794074158e-08
3043 1.12382254613408e-08
3044 1.12539702001868e-08
3045 1.12302744881276e-08
3046 1.12329656687393e-08
3047 1.12444080713203e-08
3048 1.12269438190538e-08
3049 1.12260627460614e-08
3050 1.12217986014684e-08
3051 1.1221366058578e-08
3052 1.12187032996758e-08
3053 1.12199867174922e-08
3054 1.12158708986954e-08
3055 1.12139479924167e-08
3056 1.12130926765985e-08
3057 1.12114335593105e-08
3058 1.12348867986611e-08
3059 1.12282378950113e-08
3060 1.12268283558592e-08
3061 1.12299609611455e-08
3062 1.12382121386645e-08
3063 1.12370628357894e-08
3064 1.12326006274088e-08
3065 1.12252926953715e-08
3066 1.12235474247768e-08
3067 1.12211679947904e-08
3068 1.121177106711e-08
3069 1.12023608167533e-08
3070 1.12047544575944e-08
3071 1.12056062206989e-08
3072 1.11972902061552e-08
3073 1.11969038485427e-08
3074 1.12056603995825e-08
3075 1.11943707636897e-08
3076 1.11968256888417e-08
3077 1.11953797343745e-08
3078 1.11732907370765e-08
3079 1.11926352630576e-08
3080 1.11887432652225e-08
3081 1.11841611527552e-08
3082 1.11658069457121e-08
3083 1.11656364154555e-08
3084 1.11637072919279e-08
3085 1.11590141571583e-08
3086 1.11767359811665e-08
3087 1.11779110412158e-08
3088 1.11764109078649e-08
3089 1.11751079501232e-08
3090 1.11739018038293e-08
3091 1.11727898044478e-08
3092 1.11713207573416e-08
3093 1.1169154490176e-08
3094 1.11494173893334e-08
3095 1.11474500741338e-08
3096 1.11454561135815e-08
3097 1.11453291040675e-08
3098 1.11608899899807e-08
3099 1.11425935145348e-08
3100 1.11568780880589e-08
3101 1.11438565042477e-08
3102 1.11372724376224e-08
3103 1.1134638988608e-08
3104 1.11543725367369e-08
3105 1.11525917390054e-08
3106 1.11512861167284e-08
3107 1.11499787180946e-08
3108 1.11502718169731e-08
3109 1.11474562913827e-08
3110 1.11422426840591e-08
3111 1.11449720563428e-08
3112 1.11223972254493e-08
3113 1.11446594175391e-08
3114 1.114370906663e-08
3115 1.11389377721594e-08
3116 1.11411377901049e-08
3117 1.11322089324517e-08
3118 1.11368709809767e-08
3119 1.11372635558382e-08
3120 1.11355351606335e-08
3121 1.11309823580541e-08
3122 1.11145350700781e-08
3123 1.11322338014475e-08
3124 1.11121609691622e-08
3125 1.11300302307882e-08
3126 1.11096865040849e-08
3127 1.11250750833847e-08
3128 1.11287423720796e-08
3129 1.11230598065504e-08
3130 1.11205498143363e-08
3131 1.11182600903703e-08
3132 1.11234266242377e-08
3133 1.11146594150568e-08
3134 1.11139950575989e-08
3135 1.11210729514255e-08
3136 1.11132232305522e-08
3137 1.11104014877128e-08
3138 1.11095950217077e-08
3139 1.11154756510246e-08
3140 1.11101368105437e-08
3141 1.11077040898522e-08
3142 1.11096616350892e-08
3143 1.11031095428871e-08
3144 1.11036033700884e-08
3145 1.11092148813441e-08
3146 1.10844959877454e-08
3147 1.10646922735214e-08
3148 1.1078848949353e-08
3149 1.10782956141975e-08
3150 1.10766054106648e-08
3151 1.10752198523301e-08
3152 1.10748636927838e-08
3153 1.10695834720786e-08
3154 1.10702416122876e-08
3155 1.10727613744643e-08
3156 1.10655840046547e-08
3157 1.10672084829844e-08
3158 1.10679154730065e-08
3159 1.10694493571373e-08
3160 1.10708358036504e-08
3161 1.10675806297422e-08
3162 1.10466862324188e-08
3163 1.10630287153413e-08
3164 1.10485816051664e-08
3165 1.10455546931121e-08
3166 1.1042077474599e-08
3167 1.10548219467432e-08
3168 1.10524229768316e-08
3169 1.10537659026022e-08
3170 1.10402567088386e-08
3171 1.10391340513161e-08
3172 1.10383790996593e-08
3173 1.10366951133756e-08
3174 1.1045937498011e-08
3175 1.10483213688894e-08
3176 1.10463052038767e-08
3177 1.10454276835981e-08
3178 1.10425339983067e-08
3179 1.10409006381929e-08
3180 1.10386499940773e-08
3181 1.10399565045327e-08
3182 1.10359863469967e-08
3183 1.10380051765446e-08
3184 1.10363096439414e-08
3185 1.10198659086791e-08
3186 1.10185816026842e-08
3187 1.1031646707238e-08
3188 1.10279527731905e-08
3189 1.10270770292686e-08
3190 1.10295417243833e-08
3191 1.10252225127283e-08
3192 1.10234559258515e-08
3193 1.10278204346059e-08
3194 1.1021753287821e-08
3195 1.10215685467097e-08
3196 1.10059001912077e-08
3197 1.10140607745279e-08
3198 1.10147171383801e-08
3199 1.10135651709697e-08
3200 1.10119762197769e-08
3201 1.10134248387794e-08
3202 1.09995790253947e-08
3203 1.10083115956172e-08
3204 1.10098712369222e-08
3205 1.09959739091892e-08
3206 1.10049445112281e-08
3207 1.10056985747065e-08
3208 1.09922417834696e-08
3209 1.10014957144244e-08
3210 1.0988814302948e-08
3211 1.09859019659098e-08
3212 1.09860867070211e-08
3213 1.09837117179268e-08
3214 1.09831255201698e-08
3215 1.09964828354236e-08
3216 1.09767235301206e-08
3217 1.0995351296117e-08
3218 1.09744267007272e-08
3219 1.09930189395868e-08
3220 1.09856861385538e-08
3221 1.09844640050483e-08
3222 1.09842810402938e-08
3223 1.09685593940867e-08
3224 1.09681819182583e-08
3225 1.09860920360916e-08
3226 1.09803641734629e-08
3227 1.0977759146158e-08
3228 1.09641620227308e-08
3229 1.09583266905133e-08
3230 1.09558317973324e-08
3231 1.09566471451217e-08
3232 1.09559223915312e-08
3233 1.09512647838983e-08
3234 1.0949797513149e-08
3235 1.09470974507531e-08
3236 1.09465831954481e-08
3237 1.09359898914363e-08
3238 1.09338218479138e-08
3239 1.09300302142401e-08
3240 1.09304929551968e-08
3241 1.09281286242435e-08
3242 1.09271507398034e-08
3243 1.09256550473447e-08
3244 1.09247215718256e-08
3245 1.09238351697627e-08
3246 1.09224318478596e-08
3247 1.09208704301977e-08
3248 1.09316520280345e-08
3249 1.09177058504883e-08
3250 1.09353548438662e-08
3251 1.09371987022655e-08
3252 1.09150857241502e-08
3253 1.09253699420719e-08
3254 1.09272519921433e-08
3255 1.09123581282233e-08
3256 1.09106865764375e-08
3257 1.09224442823574e-08
3258 1.09218341037831e-08
3259 1.09213802446106e-08
3260 1.0904923186672e-08
3261 1.09173656781536e-08
3262 1.09160547268061e-08
3263 1.0910973458067e-08
3264 1.09161240047229e-08
3265 1.09148805549353e-08
3266 1.0912145853581e-08
3267 1.08929345543629e-08
3268 1.09092352928997e-08
3269 1.09091926603355e-08
3270 1.09072084697459e-08
3271 1.09064925979396e-08
3272 1.09057367581045e-08
3273 1.08875939375253e-08
3274 1.08958433386874e-08
3275 1.08848015045737e-08
3276 1.09013020832549e-08
3277 1.08977138424393e-08
3278 1.088172929542e-08
3279 1.08804636411719e-08
3280 1.08801474496545e-08
3281 1.08771258666707e-08
3282 1.08932516340587e-08
3283 1.08761586403716e-08
3284 1.08753459571176e-08
3285 1.087411938272e-08
3286 1.08731477155288e-08
3287 1.08875699567079e-08
3288 1.0886706647284e-08
3289 1.08726299075101e-08
3290 1.08856523794998e-08
3291 1.08832880485465e-08
3292 1.086650947002e-08
3293 1.08655067165842e-08
3294 1.08804281140351e-08
3295 1.0868113520246e-08
3296 1.0880152778725e-08
3297 1.08616093896785e-08
3298 1.08765547679468e-08
3299 1.08604005788493e-08
3300 1.08752908900556e-08
3301 1.08577156154865e-08
3302 1.08733075876444e-08
3303 1.08555564537482e-08
3304 1.08713003044159e-08
3305 1.08529869535801e-08
3306 1.08562128176004e-08
3307 1.08520525898825e-08
3308 1.08499103035342e-08
3309 1.08529274456259e-08
3310 1.0848784093298e-08
3311 1.08491127193133e-08
3312 1.08478515059574e-08
3313 1.08467794746048e-08
3314 1.08472351101341e-08
3315 1.08439621726575e-08
3316 1.08427533618283e-08
3317 1.08420215028104e-08
3318 1.08406812415751e-08
3319 1.08398285902922e-08
3320 1.08389919262208e-08
3321 1.08503277473915e-08
3322 1.0837663211305e-08
3323 1.08358522155072e-08
3324 1.08526609921e-08
3325 1.08487645533728e-08
3326 1.08335349580102e-08
3327 1.08315711955242e-08
3328 1.08308197965812e-08
3329 1.08451594371672e-08
3330 1.08292939060561e-08
3331 1.08282742772303e-08
3332 1.08271693832762e-08
3333 1.08265370002414e-08
3334 1.08257802722278e-08
3335 1.08242375063128e-08
3336 1.08232436346611e-08
3337 1.08231157369687e-08
3338 1.08216005045847e-08
3339 1.0820468965278e-08
3340 1.08185824743146e-08
3341 1.08192068637436e-08
3342 1.08278603860867e-08
3343 1.08192574899135e-08
3344 1.08167572676621e-08
3345 1.08150270961005e-08
3346 1.08296012157894e-08
3347 1.08143485277878e-08
3348 1.08131814613444e-08
3349 1.08124496023265e-08
3350 1.08118367592169e-08
3351 1.08107807150759e-08
3352 1.08097335527191e-08
3353 1.08021041000939e-08
3354 1.07967128570863e-08
3355 1.08098632267684e-08
3356 1.08157944822551e-08
3357 1.07969135854091e-08
3358 1.08120268293987e-08
3359 1.07909663427108e-08
3360 1.07911048985443e-08
3361 1.07903455059954e-08
3362 1.07932489612494e-08
3363 1.07886792832801e-08
3364 1.07919539971135e-08
3365 1.0790721205467e-08
3366 1.07894040368706e-08
3367 1.07898516787941e-08
3368 1.0785024429083e-08
3369 1.07883124655928e-08
3370 1.07905853141688e-08
3371 1.07871960253192e-08
3372 1.07852189401569e-08
3373 1.0800498273511e-08
3374 1.07970912210931e-08
3375 1.07831104045886e-08
3376 1.07821254147211e-08
3377 1.07792077486124e-08
3378 1.07784332570304e-08
3379 1.07795008474909e-08
3380 1.07934745585681e-08
3381 1.07784225988894e-08
3382 1.07754933864612e-08
3383 1.07769455581774e-08
3384 1.07899609247397e-08
3385 1.07758513223644e-08
3386 1.07745341537679e-08
3387 1.07775743884986e-08
3388 1.07751167988113e-08
3389 1.0772409630988e-08
3390 1.07685789174639e-08
3391 1.07830659956676e-08
3392 1.07697237794468e-08
3393 1.07810507188333e-08
3394 1.07675868221691e-08
3395 1.07832747175962e-08
3396 1.07665902859821e-08
3397 1.07630251378055e-08
3398 1.07796029880092e-08
3399 1.07814397409811e-08
3400 1.07649666958309e-08
3401 1.07608499888556e-08
3402 1.07615472089151e-08
3403 1.07584696706908e-08
3404 1.07759996481605e-08
3405 1.07619664291292e-08
3406 1.07617559308437e-08
3407 1.07609690047639e-08
3408 1.07600914844852e-08
3409 1.07688862271971e-08
3410 1.07530198079075e-08
3411 1.07633670864971e-08
3412 1.07564313012176e-08
3413 1.07513686842253e-08
3414 1.07613260524886e-08
3415 1.07473043797768e-08
3416 1.07630828694028e-08
3417 1.07466915366672e-08
3418 1.07467954535423e-08
3419 1.07454329878465e-08
3420 1.07564996909559e-08
3421 1.07460493836697e-08
3422 1.0756068036244e-08
3423 1.07441726626689e-08
3424 1.07454800613027e-08
3425 1.07423598905143e-08
3426 1.07519051439908e-08
3427 1.07391135983903e-08
3428 1.07374171776087e-08
3429 1.07370530244566e-08
3430 1.07504831703409e-08
3431 1.07374473756749e-08
3432 1.07525677250919e-08
3433 1.07346407318687e-08
3434 1.07331254994847e-08
3435 1.07304876095782e-08
3436 1.07457562847912e-08
3437 1.07329638510123e-08
3438 1.07305018204329e-08
3439 1.07321262987625e-08
3440 1.07435393914557e-08
3441 1.0733183231082e-08
3442 1.07429425355576e-08
3443 1.07237960733642e-08
3444 1.07255173631415e-08
3445 1.07369917401456e-08
3446 1.07249098491025e-08
3447 1.07295408113828e-08
3448 1.07348032685195e-08
3449 1.07223057099759e-08
3450 1.07220037293132e-08
3451 1.07225490708629e-08
3452 1.0721032062122e-08
3453 1.07340536459333e-08
3454 1.07160307294407e-08
3455 1.07165183393931e-08
3456 1.07173656616055e-08
3457 1.07312319030939e-08
3458 1.07179651820388e-08
3459 1.07139479510465e-08
3460 1.07135784688239e-08
3461 1.07313287145416e-08
3462 1.07112940739285e-08
3463 1.07265893944941e-08
3464 1.0716198595162e-08
3465 1.0709479525417e-08
3466 1.07089768164315e-08
3467 1.07441628927063e-08
3468 1.07417026384837e-08
3469 1.07055830866898e-08
3470 1.07383186787047e-08
3471 1.0705143438372e-08
3472 1.07423856476885e-08
3473 1.07410382810258e-08
3474 1.07367599255781e-08
3475 1.0738209432759e-08
3476 1.0704114927762e-08
3477 1.07318207653861e-08
3478 1.07011768335497e-08
3479 1.06976809632897e-08
3480 1.07296020956937e-08
3481 1.06984288095191e-08
3482 1.06949391565081e-08
3483 1.07283613104414e-08
3484 1.07312772001933e-08
3485 1.07305728747065e-08
3486 1.0729875654647e-08
3487 1.07287547734813e-08
3488 1.07280611061356e-08
3489 1.07291961981559e-08
3490 1.07264712667643e-08
3491 1.07276543204193e-08
3492 1.07244151337227e-08
3493 1.06879429750961e-08
3494 1.07184696673812e-08
3495 1.07177893227117e-08
3496 1.0721463716834e-08
3497 1.07157305251349e-08
3498 1.06807194200087e-08
3499 1.06832800383927e-08
3500 1.071787458784e-08
3501 1.07209539024211e-08
3502 1.0718687271094e-08
3503 1.07192583698179e-08
3504 1.07184749964517e-08
3505 1.07153255157755e-08
3506 1.07173585561782e-08
3507 1.07139710436854e-08
3508 1.07132507309871e-08
3509 1.07153024231366e-08
3510 1.07118776071502e-08
3511 1.07106306046489e-08
3512 1.07102886559574e-08
3513 1.07096536083873e-08
3514 1.07112114733354e-08
3515 1.07078781397263e-08
3516 1.07011306482718e-08
3517 1.07039230812234e-08
3518 1.07034194840594e-08
3519 1.07025419637807e-08
3520 1.07021786988071e-08
3521 1.07013846672999e-08
3522 1.07004662908139e-08
3523 1.06992210646695e-08
3524 1.0671137751217e-08
3525 1.06779722841566e-08
3526 1.06771320673715e-08
3527 1.06755209117182e-08
3528 1.06812239053511e-08
3529 1.06777306996264e-08
3530 1.06697397583844e-08
3531 1.06740536409689e-08
3532 1.06907984687155e-08
3533 1.06679731715076e-08
3534 1.06904050056755e-08
3535 1.068644017721e-08
3536 1.0674754413742e-08
3537 1.06543662781178e-08
3538 1.06574296054873e-08
3539 1.06869109117724e-08
3540 1.06846833602958e-08
3541 1.0656747484461e-08
3542 1.06548352363234e-08
3543 1.065327470684e-08
3544 1.06525090970422e-08
3545 1.0651708848286e-08
3546 1.06835242874581e-08
3547 1.06495434692988e-08
3548 1.06818731637759e-08
3549 1.06769757479697e-08
3550 1.06515010145358e-08
3551 1.06430713131545e-08
3552 1.06494413287805e-08
3553 1.06777822139748e-08
3554 1.06459507875911e-08
3555 1.06454631776387e-08
3556 1.06451905068639e-08
3557 1.06442339387058e-08
3558 1.06433457602861e-08
3559 1.0643330661253e-08
3560 1.06420436907229e-08
3561 1.06424886681111e-08
3562 1.06418971412836e-08
3563 1.06407114230933e-08
3564 1.06392370469166e-08
3565 1.06394670851273e-08
3566 1.0637877245756e-08
3567 1.06385042997204e-08
3568 1.06372226582607e-08
3569 1.06366986329931e-08
3570 1.06351132345139e-08
3571 1.06351922823933e-08
3572 1.06345350303627e-08
3573 1.0635388569824e-08
3574 1.06324247184375e-08
3575 1.06337489924613e-08
3576 1.06634354679613e-08
3577 1.06334514526907e-08
3578 1.06604733929316e-08
3579 1.06593871507243e-08
3580 1.06352651130237e-08
3581 1.06303756908233e-08
3582 1.06299777868912e-08
3583 1.06296109692039e-08
3584 1.06281650147366e-08
3585 1.06341566663559e-08
3586 1.06256239362779e-08
3587 1.06237871833059e-08
3588 1.06361088825224e-08
3589 1.06319175685599e-08
3590 1.0634465752446e-08
3591 1.06309991920739e-08
3592 1.06505888552988e-08
3593 1.0618471435464e-08
3594 1.06511741648774e-08
3595 1.06261781596118e-08
3596 1.06206581307333e-08
3597 1.06203197347554e-08
3598 1.06199244953586e-08
3599 1.06174606884224e-08
3600 1.06457012094552e-08
3601 1.06447988201808e-08
3602 1.06439186353668e-08
3603 1.064355714675e-08
3604 1.06126956112007e-08
3605 1.06133981603307e-08
3606 1.06133697386213e-08
3607 1.06114432796289e-08
3608 1.06112985065465e-08
3609 1.06097211016731e-08
3610 1.06393862608911e-08
3611 1.06077449046893e-08
3612 1.06381810027756e-08
3613 1.06063593463546e-08
3614 1.06362403329285e-08
3615 1.0604646050183e-08
3616 1.0634764180395e-08
3617 1.06312008085752e-08
3618 1.06035598079757e-08
3619 1.06001172284209e-08
3620 1.06032809199519e-08
3621 1.06032436164583e-08
3622 1.0602936306725e-08
3623 1.06000781485704e-08
3624 1.05972501884821e-08
3625 1.05981428077939e-08
3626 1.0625162971678e-08
3627 1.05950812567812e-08
3628 1.05965849428458e-08
3629 1.05952491225025e-08
3630 1.062384669126e-08
3631 1.05929593985365e-08
3632 1.05946957873471e-08
3633 1.05943271933029e-08
3634 1.05917585813131e-08
3635 1.05931654559299e-08
3636 1.06192494797597e-08
3637 1.06172262093196e-08
3638 1.06162936219789e-08
3639 1.06154489643018e-08
3640 1.0614484402538e-08
3641 1.05872475231195e-08
3642 1.05864863542138e-08
3643 1.05827595575647e-08
3644 1.05854267573591e-08
3645 1.06107931330257e-08
3646 1.06088968720996e-08
3647 1.05789323967542e-08
3648 1.05818704909666e-08
3649 1.05795612270754e-08
3650 1.05806785555274e-08
3651 1.06080779715967e-08
3652 1.0576656883643e-08
3653 1.06062065796664e-08
3654 1.05752206991383e-08
3655 1.05755315615852e-08
3656 1.06059800941694e-08
3657 1.06035606961541e-08
3658 1.06000257460437e-08
3659 1.05709787590058e-08
3660 1.05754711654527e-08
3661 1.06017710166384e-08
3662 1.05726920551774e-08
3663 1.06001190047778e-08
3664 1.05676782879982e-08
3665 1.05680797446439e-08
3666 1.05956736717872e-08
3667 1.05934780947337e-08
3668 1.05646265069481e-08
3669 1.05648787496193e-08
3670 1.05925019866504e-08
3671 1.05920099358059e-08
3672 1.05888000589971e-08
3673 1.05608872758012e-08
3674 1.05619033519133e-08
3675 1.05887139056904e-08
3676 1.05594510912965e-08
3677 1.05598632060833e-08
3678 1.05871276190328e-08
3679 1.05597193211793e-08
3680 1.05875352929274e-08
3681 1.0556703067266e-08
3682 1.05874606859402e-08
3683 1.0582186682484e-08
3684 1.05560395979865e-08
3685 1.05861408528085e-08
3686 1.05831201580031e-08
3687 1.05497219848871e-08
3688 1.05509982972762e-08
3689 1.05508588532643e-08
3690 1.05488533463927e-08
3691 1.05516182458132e-08
3692 1.05470823186238e-08
3693 1.05787849591366e-08
3694 1.05443787035142e-08
3695 1.05457429455669e-08
3696 1.05449160514581e-08
3697 1.05462314436977e-08
3698 1.0546587603244e-08
3699 1.05435624675465e-08
3700 1.05427009344794e-08
3701 1.05422248708464e-08
3702 1.05407034212135e-08
3703 1.0571618247468e-08
3704 1.05370112635228e-08
3705 1.05391446680869e-08
3706 1.05372590653019e-08
3707 1.05363735514175e-08
3708 1.05356621205033e-08
3709 1.05369215575024e-08
3710 1.05387636395449e-08
3711 1.05340909328788e-08
3712 1.05342623513138e-08
3713 1.05332107480649e-08
3714 1.05319237775348e-08
3715 1.05310498099698e-08
3716 1.05331743327497e-08
3717 1.05595967525574e-08
3718 1.05606279277026e-08
3719 1.05628927826729e-08
3720 1.05542463657571e-08
3721 1.05275272943572e-08
3722 1.05570885367001e-08
3723 1.0524172644466e-08
3724 1.05560440388786e-08
3725 1.05577946385438e-08
3726 1.05214414958255e-08
3727 1.05543671580222e-08
3728 1.05489057489194e-08
3729 1.05473123568345e-08
3730 1.05188435739478e-08
3731 1.05512585335532e-08
3732 1.05425161933681e-08
3733 1.0529808136539e-08
3734 1.05366639857607e-08
3735 1.05257713656215e-08
3736 1.05386765980597e-08
3737 1.05344604151014e-08
3738 1.05240829384456e-08
3739 1.05231103830761e-08
3740 1.05227053737167e-08
3741 1.05203161737677e-08
3742 1.05298125774311e-08
3743 1.05290114404966e-08
3744 1.05225712587753e-08
3745 1.05231094948977e-08
3746 1.05235935521364e-08
3747 1.05216191315094e-08
3748 1.05210835599223e-08
3749 1.05210036238645e-08
3750 1.05201669597932e-08
3751 1.05192690114109e-08
3752 1.05181099385732e-08
3753 1.05179323028892e-08
3754 1.0517470450111e-08
3755 1.05172084374772e-08
3756 1.05142614614806e-08
3757 1.05246220627464e-08
3758 1.0513333315032e-08
3759 1.05230135716283e-08
3760 1.05116324533583e-08
3761 1.05212851764236e-08
3762 1.05108073356064e-08
3763 1.05202992983777e-08
3764 1.05085078416778e-08
3765 1.05187609733548e-08
3766 1.05079216439208e-08
3767 1.05168993513871e-08
3768 1.0506072456451e-08
3769 1.05168407316114e-08
3770 1.05045687703864e-08
3771 1.05175042008909e-08
3772 1.05062847310933e-08
3773 1.05039967834841e-08
3774 1.05022968099888e-08
3775 1.05023802987603e-08
3776 1.05113056036998e-08
3777 1.05097219815775e-08
3778 1.05010755646617e-08
3779 1.05023802987603e-08
3780 1.04993382876728e-08
3781 1.05008748363389e-08
3782 1.04982920134944e-08
3783 1.04992894378597e-08
3784 1.04984119175811e-08
3785 1.05055812937849e-08
3786 1.05066426669964e-08
3787 1.04945749868079e-08
3788 1.04922399657426e-08
3789 1.04917941001759e-08
3790 1.05019744012225e-08
3791 1.049181985735e-08
3792 1.04898276731547e-08
3793 1.04890105490085e-08
3794 1.04884305685005e-08
3795 1.04977200265921e-08
3796 1.04883701723679e-08
3797 1.04862571959075e-08
3798 1.04866124672753e-08
3799 1.04847712734113e-08
3800 1.04922426302778e-08
3801 1.04847615034487e-08
3802 1.04825916835694e-08
3803 1.04944044565514e-08
3804 1.04811910262015e-08
3805 1.04820996327248e-08
3806 1.04812292178735e-08
3807 1.04915356402557e-08
3808 1.04929362976236e-08
3809 1.04924691157748e-08
3810 1.0492025026565e-08
3811 1.04945554468827e-08
3812 1.04908197684495e-08
3813 1.04897228681011e-08
3814 1.04898649766483e-08
3815 1.04887352136984e-08
3816 1.04918234100637e-08
3817 1.04881126006262e-08
3818 1.04874997575166e-08
3819 1.04870041539584e-08
3820 1.04862616367996e-08
3821 1.04845270243459e-08
3822 1.04841193504512e-08
3823 1.04844000148319e-08
3824 1.04824469104869e-08
3825 1.04829496194725e-08
3826 1.04813269174997e-08
3827 1.04705790704429e-08
3828 1.04809689815966e-08
3829 1.04794981581335e-08
3830 1.04790265353927e-08
3831 1.04785602417223e-08
3832 1.04777271303647e-08
3833 1.04767785558124e-08
3834 1.04759951824462e-08
3835 1.04645598852926e-08
3836 1.04629425123903e-08
3837 1.04709130255287e-08
3838 1.04620987428916e-08
3839 1.04607353890174e-08
3840 1.0467066324793e-08
3841 1.04657011945619e-08
3842 1.04638937514778e-08
3843 1.04699928726859e-08
3844 1.04569179981695e-08
3845 1.04678399281966e-08
3846 1.04559134683768e-08
3847 1.04654924726333e-08
3848 1.04577022597141e-08
3849 1.04651380894438e-08
3850 1.04512629661713e-08
3851 1.04623083529987e-08
3852 1.04543156353998e-08
3853 1.04620845320369e-08
3854 1.0449086929043e-08
3855 1.04513366849801e-08
3856 1.04526032274066e-08
3857 1.04578479209749e-08
3858 1.0449927145828e-08
3859 1.04558219859996e-08
3860 1.04486419516547e-08
3861 1.04544080059554e-08
3862 1.04469206618774e-08
3863 1.0453052645687e-08
3864 1.04457766880728e-08
3865 1.04527950739453e-08
3866 1.04398054645571e-08
3867 1.04387503085945e-08
3868 1.04389119570669e-08
3869 1.04476942652809e-08
3870 1.0436948194581e-08
3871 1.04361701502853e-08
3872 1.04360040609208e-08
3873 1.04391260080661e-08
3874 1.04454329630244e-08
3875 1.04366497666319e-08
3876 1.04450741389428e-08
3877 1.04320774241273e-08
3878 1.0432478880773e-08
3879 1.04348174545521e-08
3880 1.04415436297245e-08
3881 1.04303481407442e-08
3882 1.04289545888037e-08
3883 1.0432186670073e-08
3884 1.04395221356413e-08
3885 1.0439735298462e-08
3886 1.04300825753967e-08
3887 1.04295221348139e-08
3888 1.04366018049973e-08
3889 1.04333022221681e-08
3890 1.04330943884179e-08
3891 1.04334452188937e-08
3892 1.04325881267187e-08
3893 1.04320756477705e-08
3894 1.04368282904943e-08
3895 1.04361381758622e-08
3896 1.04355235563958e-08
3897 1.04348583107594e-08
3898 1.04343644835581e-08
3899 1.04341904005878e-08
3900 1.04332453787492e-08
3901 1.04326858263448e-08
3902 1.04328421457467e-08
3903 1.04321777882888e-08
3904 1.04317612326099e-08
3905 1.04309734183516e-08
3906 1.0430620811519e-08
3907 1.04300470482599e-08
3908 1.04294191061172e-08
3909 1.04290602820356e-08
3910 1.04131219202941e-08
3911 1.04113224708158e-08
3912 1.04116590904368e-08
3913 1.04116244514785e-08
3914 1.04099173725558e-08
3915 1.04088959673732e-08
3916 1.04088799801616e-08
3917 1.04177537707528e-08
3918 1.04075850160257e-08
3919 1.04059978411897e-08
3920 1.04159409985982e-08
3921 1.04057367167343e-08
3922 1.04148156765405e-08
3923 1.04044710624862e-08
3924 1.04140145396059e-08
3925 1.0403251593516e-08
3926 1.04138457857061e-08
3927 1.04020969615704e-08
3928 1.04114494803298e-08
3929 1.04090842611981e-08
3930 1.04011315116281e-08
3931 1.04096713471336e-08
3932 1.0399844541098e-08
3933 1.04099200370911e-08
3934 1.0397222638403e-08
3935 1.03992050526358e-08
3936 1.03958930353087e-08
3937 1.03981649957063e-08
3938 1.03947099816537e-08
3939 1.03975681398083e-08
3940 1.03933581740989e-08
3941 1.03964756803521e-08
3942 1.04020170255126e-08
3943 1.03923962768704e-08
3944 1.04016724122857e-08
3945 1.039095209876e-08
3946 1.039988273277e-08
3947 1.03898036840633e-08
3948 1.03909369997268e-08
3949 1.03987289890028e-08
3950 1.0387926074884e-08
3951 1.03970378972917e-08
3952 1.03968469389315e-08
3953 1.0386599136325e-08
3954 1.03864099543216e-08
3955 1.03947686014294e-08
3956 1.03847535015689e-08
3957 1.0393849336765e-08
3958 1.03928634587191e-08
3959 1.03832942244253e-08
3960 1.03881436785969e-08
3961 1.03820401164967e-08
3962 1.03918695870675e-08
3963 1.03900381631661e-08
3964 1.03898676329095e-08
3965 1.03887023428229e-08
3966 1.03882991098203e-08
3967 1.0387710247528e-08
3968 1.03860875455553e-08
3969 1.03864836731304e-08
3970 1.03863486700106e-08
3971 1.03856105937439e-08
3972 1.03852677568739e-08
3973 1.03843689203131e-08
3974 1.03838981857507e-08
3975 1.03816439889215e-08
3976 1.03829602693395e-08
3977 1.03807380469334e-08
3978 1.03813411200804e-08
3979 1.0380775350427e-08
3980 1.03811359508654e-08
3981 1.03798285522316e-08
3982 1.037917751745e-08
3983 1.03769419723676e-08
3984 1.03779136395588e-08
3985 1.03769437487244e-08
3986 1.03773833970422e-08
3987 1.03763033720838e-08
3988 1.03750492641552e-08
3989 1.03744168811204e-08
3990 1.03736930157083e-08
3991 1.0373182313117e-08
3992 1.03734922873855e-08
3993 1.03728750033838e-08
3994 1.0370532876891e-08
3995 1.03707078480397e-08
3996 1.03706980780771e-08
3997 1.03718464927738e-08
3998 1.03701944809131e-08
3999 1.03699733244866e-08
4000 1.03698853948231e-08
4001 1.03697024300686e-08
4002 1.03677573193295e-08
4003 1.03685433572309e-08
4004 1.03676205398529e-08
4005 1.03672270768129e-08
4006 1.0365301505999e-08
4007 1.03650208416184e-08
4008 1.03653512439905e-08
4009 1.03642374682522e-08
4010 1.03636779158478e-08
4011 1.03629407277595e-08
4012 1.03622399549863e-08
4013 1.03616422109098e-08
4014 1.03618242874859e-08
4015 1.03616173419141e-08
4016 1.03607069590339e-08
4017 1.03599022693857e-08
4018 1.03593915667943e-08
4019 1.03586232924613e-08
4020 1.0351973500633e-08
4021 1.0357907420655e-08
4022 1.03568797982234e-08
4023 1.03592014966125e-08
4024 1.03558281949745e-08
4025 1.03556994091036e-08
4026 1.03545483298717e-08
4027 1.03550261698615e-08
4028 1.03539150586585e-08
4029 1.0352930068791e-08
4030 1.0346862922006e-08
4031 1.03515178651037e-08
4032 1.03458148714708e-08
4033 1.03515978011615e-08
4034 1.03506989646007e-08
4035 1.03444106613892e-08
4036 1.03487369784716e-08
4037 1.03496340386755e-08
4038 1.03430544129424e-08
4039 1.03485158220451e-08
4040 1.03337818302407e-08
4041 1.03418891228557e-08
4042 1.0341233647182e-08
4043 1.03403712259365e-08
4044 1.03414485863595e-08
4045 1.03393302808286e-08
4046 1.03392654438039e-08
4047 1.03377981730546e-08
4048 1.03381347926756e-08
4049 1.033733987299e-08
4050 1.0336781208764e-08
4051 1.03413002605635e-08
4052 1.03355404235117e-08
4053 1.0340846401391e-08
4054 1.03284207852994e-08
4055 1.03391375461115e-08
4056 1.03388178018804e-08
4057 1.03373274384921e-08
4058 1.03372714832517e-08
4059 1.03269082174506e-08
4060 1.03360306979994e-08
4061 1.03328812173231e-08
4062 1.03348565261285e-08
4063 1.03317727706553e-08
4064 1.03338928525432e-08
4065 1.03336601497972e-08
4066 1.03232506987183e-08
4067 1.033218044455e-08
4068 1.03316590838176e-08
4069 1.03296597941949e-08
4070 1.03304147458516e-08
4071 1.03306643239875e-08
4072 1.0324487931257e-08
4073 1.03242339122289e-08
4074 1.03287511876715e-08
4075 1.03233510628797e-08
4076 1.03290576092263e-08
4077 1.03269170992348e-08
4078 1.03199990775238e-08
4079 1.0320809984421e-08
4080 1.03257074002272e-08
4081 1.0317872778387e-08
4082 1.03228954273504e-08
4083 1.0317358523082e-08
4084 1.03235944237667e-08
4085 1.03154533803718e-08
4086 1.03185868738365e-08
4087 1.03189572442375e-08
4088 1.03186774680353e-08
4089 1.03145429974916e-08
4090 1.0317115162195e-08
4091 1.03169659482205e-08
4092 1.03149382368883e-08
4093 1.03163291242936e-08
4094 1.03126280848187e-08
4095 1.03147970165196e-08
4096 1.03145971763752e-08
4097 1.03139372598093e-08
4098 1.03133732665128e-08
4099 1.03104671467236e-08
4100 1.03132480333556e-08
4101 1.03121466921152e-08
4102 1.03120525452027e-08
4103 1.03091029046709e-08
4104 1.03087662850498e-08
4105 1.03101838178077e-08
4106 1.03071293722223e-08
4107 1.03092832048901e-08
4108 1.03087280933778e-08
4109 1.03081410074424e-08
4110 1.03077066881951e-08
4111 1.03067714363192e-08
4112 1.03068655832317e-08
4113 1.03064357048765e-08
4114 1.0305678976863e-08
4115 1.03053796607355e-08
4116 1.03072901325163e-08
4117 1.03016777330822e-08
4118 1.03040811438859e-08
4119 1.0303389252897e-08
4120 1.03023625186438e-08
4121 1.03022070874204e-08
4122 1.03019548447492e-08
4123 1.0300767350202e-08
4124 1.03006501106506e-08
4125 1.03003703344484e-08
4126 1.02985318051196e-08
4127 1.02997379514136e-08
4128 1.02991855044365e-08
4129 1.0299783248513e-08
4130 1.03026058795308e-08
4131 1.02993711337263e-08
4132 1.02983817029667e-08
4133 1.02955874936583e-08
4134 1.02968993331842e-08
4135 1.02962420811537e-08
4136 1.0296437480406e-08
4137 1.02918482625114e-08
4138 1.02940829194154e-08
4139 1.02947455005165e-08
4140 1.02974127003108e-08
4141 1.02953716663023e-08
4142 1.02930997059048e-08
4143 1.0292516172683e-08
4144 1.02953592318045e-08
4145 1.02951274172369e-08
4146 1.0287312335322e-08
4147 1.02899981868632e-08
4148 1.02931547729668e-08
4149 1.02923438660696e-08
4150 1.02869455176346e-08
4151 1.02919957001291e-08
4152 1.02910417965063e-08
4153 1.02842987459439e-08
4154 1.02904209597909e-08
4155 1.02947277369481e-08
4156 1.02893809028615e-08
4157 1.0286393958836e-08
4158 1.02768478171811e-08
4159 1.02829602610655e-08
4160 1.02875921115242e-08
4161 1.02867527829176e-08
4162 1.02805000068429e-08
4163 1.02858761508173e-08
4164 1.02747153007954e-08
4165 1.02854809114206e-08
4166 1.02778914268242e-08
4167 1.02840846949448e-08
4168 1.02877404373203e-08
4169 1.02831130277536e-08
4170 1.02793684675362e-08
4171 1.02768993315294e-08
4172 1.02752633068803e-08
4173 1.02748263230978e-08
4174 1.02725374873103e-08
4175 1.02736814611148e-08
4176 1.02756976261276e-08
4177 1.02706430027411e-08
4178 1.02722497175023e-08
4179 1.02704653670571e-08
4180 1.02680965952118e-08
4181 1.02722985673154e-08
4182 1.02727906181599e-08
4183 1.0268315087103e-08
4184 1.0269371131244e-08
4185 1.02711652516518e-08
4186 1.0271478778634e-08
4187 1.02710560057062e-08
4188 1.0270145622826e-08
4189 1.02697459425372e-08
4190 1.02692965242568e-08
4191 1.02676169788651e-08
4192 1.02683586078456e-08
4193 1.02668922252747e-08
4194 1.0267591221691e-08
4195 1.02667989665406e-08
4196 1.02661115164437e-08
4197 1.02645891786324e-08
4198 1.02657367051506e-08
4199 1.02604502671966e-08
4200 1.02472972329792e-08
4201 1.02475814500735e-08
4202 1.02469863705323e-08
4203 1.02464579043726e-08
4204 1.02483097563777e-08
4205 1.02458859174703e-08
4206 1.02448325378646e-08
4207 1.02478443508858e-08
4208 1.0247521053941e-08
4209 1.02483834751865e-08
4210 1.02477848429317e-08
4211 1.0246005821557e-08
4212 1.02419885905647e-08
4213 1.02420134595604e-08
4214 1.02413446612104e-08
4215 1.02426040982095e-08
4216 1.02401793711238e-08
4217 1.02445643079818e-08
4218 1.02393178380566e-08
4219 1.02363717502385e-08
4220 1.02281525471426e-08
4221 1.02275281577136e-08
4222 1.02267412316337e-08
4223 1.02268398194383e-08
4224 1.02372590404798e-08
4225 1.02315764749505e-08
4226 1.02253485678716e-08
4227 1.02246229261027e-08
4228 1.0235210901044e-08
4229 1.02322763595453e-08
4230 1.02314947625359e-08
4231 1.02230437448725e-08
4232 1.02221129338886e-08
4233 1.02215906849779e-08
4234 1.02210657715318e-08
4235 1.02210266916813e-08
4236 1.02202850627009e-08
4237 1.02196748841266e-08
4238 1.0219011414847e-08
4239 1.02185833128488e-08
4240 1.02176551664002e-08
4241 1.02173656202353e-08
4242 1.02182005079499e-08
4243 1.02288248982063e-08
4244 1.02243156163695e-08
4245 1.02146318070595e-08
4246 1.02281569880347e-08
4247 1.02149675385022e-08
4248 1.02244346322777e-08
4249 1.02240216293126e-08
4250 1.02218846720348e-08
4251 1.02133403956373e-08
4252 1.02125268242048e-08
4253 1.02122088563306e-08
4254 1.02115098599143e-08
4255 1.02113153488403e-08
4256 1.02111252786585e-08
4257 1.02106563204529e-08
4258 1.02218544739685e-08
4259 1.02170272242574e-08
4260 1.02088737463646e-08
4261 1.02083790309848e-08
4262 1.02075272678803e-08
4263 1.02072199581471e-08
4264 1.02063451024037e-08
4265 1.02119255274147e-08
4266 1.02060333517784e-08
4267 1.02056656459126e-08
4268 1.02050341510562e-08
4269 1.02041761707028e-08
4270 1.02037054361404e-08
4271 1.02034256599381e-08
4272 1.02006971758328e-08
4273 1.0209676659656e-08
4274 1.02002104540588e-08
4275 1.02001012081132e-08
4276 1.02074242391836e-08
4277 1.01985104805635e-08
4278 1.01979953370801e-08
4279 1.01973256505516e-08
4280 1.01968851140555e-08
4281 1.01958974596528e-08
4282 1.02029886761557e-08
4283 1.01962971399416e-08
4284 1.02031521009849e-08
4285 1.01947197350682e-08
4286 1.01944142016919e-08
4287 1.02004129587385e-08
4288 1.01942436714353e-08
4289 1.02032906568184e-08
4290 1.01931032503444e-08
4291 1.01922621453809e-08
4292 1.01918784523036e-08
4293 1.01975263788745e-08
4294 1.0190624344375e-08
4295 1.01968389287777e-08
4296 1.01902228877293e-08
4297 1.01897050797106e-08
4298 1.01975112798414e-08
4299 1.0187947374618e-08
4300 1.01881090230904e-08
4301 1.01880814895594e-08
4302 1.01948867126112e-08
4303 1.01865857971006e-08
4304 1.01863077972553e-08
4305 1.01921138195848e-08
4306 1.01852339895458e-08
4307 1.01914885419774e-08
4308 1.01859445322816e-08
4309 1.01949826358805e-08
4310 1.01857979828424e-08
4311 1.01945207831022e-08
4312 1.0191453903019e-08
4313 1.01849710887336e-08
4314 1.01962820409085e-08
4315 1.01844950251007e-08
4316 1.01930970330955e-08
4317 1.01830881504839e-08
4318 1.01891037829205e-08
4319 1.01832258181389e-08
4320 1.0193002886183e-08
4321 1.01818464770531e-08
4322 1.01889465753402e-08
4323 1.01897379423121e-08
4324 1.0183860865709e-08
4325 1.01806945096428e-08
4326 1.0187936716477e-08
4327 1.01862029922017e-08
4328 1.01878496749919e-08
4329 1.01852339895458e-08
4330 1.0186576027138e-08
4331 1.01857349221746e-08
4332 1.01859995993436e-08
4333 1.01860013757005e-08
4334 1.01845785138721e-08
4335 1.01846406863615e-08
4336 1.01841299837702e-08
4337 1.0183391019325e-08
4338 1.0183606846681e-08
4339 1.01831520993301e-08
4340 1.0181958387534e-08
4341 1.01743466984772e-08
4342 1.01732178237057e-08
4343 1.0173231146382e-08
4344 1.01729682455698e-08
4345 1.01729060730804e-08
4346 1.01723127698961e-08
4347 1.01719530576361e-08
4348 1.01715640354882e-08
4349 1.01706083555086e-08
4350 1.01704351607168e-08
4351 1.0169813435823e-08
4352 1.01696171483923e-08
4353 1.01791641782256e-08
4354 1.01635864169225e-08
4355 1.01735029289785e-08
4356 1.01693471421527e-08
4357 1.01687689380014e-08
4358 1.01662189777585e-08
4359 1.01659871631909e-08
4360 1.01720054601628e-08
4361 1.01732791080167e-08
4362 1.01650865502734e-08
4363 1.01744745961696e-08
4364 1.01547783515343e-08
4365 1.01638368832369e-08
4366 1.01597477097926e-08
4367 1.0162408692338e-08
4368 1.01580122091605e-08
4369 1.0156872676248e-08
4370 1.0159532770615e-08
4371 1.01564383570008e-08
4372 1.01476604896789e-08
4373 1.01521093753831e-08
4374 1.01471977487222e-08
4375 1.01515267303398e-08
4376 1.01461425927596e-08
4377 1.01526396178997e-08
4378 1.01451433920374e-08
4379 1.01488026871266e-08
4380 1.01440340571912e-08
4381 1.01478718761427e-08
4382 1.01431218979542e-08
4383 1.01477777292303e-08
4384 1.01423873744011e-08
4385 1.01463601964724e-08
4386 1.01407016117605e-08
4387 1.01453299095056e-08
4388 1.01403232477537e-08
4389 1.01437080957112e-08
4390 1.01389359130621e-08
4391 1.01434469712558e-08
4392 1.01380139838625e-08
4393 1.01423429654801e-08
4394 1.01370822847002e-08
4395 1.01441601785268e-08
4396 1.01371719907206e-08
4397 1.01438937250009e-08
4398 1.01361825599611e-08
4399 1.01400816632236e-08
4400 1.01358530457674e-08
4401 1.01426325116449e-08
4402 1.01347881198421e-08
4403 1.01386383732915e-08
4404 1.01342889635703e-08
4405 1.01374908467733e-08
4406 1.01330153157164e-08
4407 1.013654848947e-08
4408 1.01320400958116e-08
4409 1.01358033077759e-08
4410 1.01308934574718e-08
4411 1.01378851979916e-08
4412 1.01302495281175e-08
4413 1.01337036539917e-08
4414 1.01300088317657e-08
4415 1.01377670702618e-08
4416 1.01291677268023e-08
4417 1.01376667061004e-08
4418 1.0128784033725e-08
4419 1.01372910066289e-08
4420 1.01281178999102e-08
4421 1.01351567138863e-08
4422 1.01275556829705e-08
4423 1.01337480629127e-08
4424 1.0126566252211e-08
4425 1.01353085923961e-08
4426 1.01257118245712e-08
4427 1.01339656666255e-08
4428 1.01246406813971e-08
4429 1.01330579482806e-08
4430 1.0124022509217e-08
4431 1.01319912459985e-08
4432 1.01227852766783e-08
4433 1.01313375466816e-08
4434 1.01235979599323e-08
4435 1.01297379373477e-08
4436 1.01205719360564e-08
4437 1.01291428578065e-08
4438 1.01195656299069e-08
4439 1.01280077657862e-08
4440 1.01183372791525e-08
4441 1.01262669360835e-08
4442 1.01175521294294e-08
4443 1.01255848150572e-08
4444 1.01165689159188e-08
4445 1.01246167005797e-08
4446 1.01153219134176e-08
4447 1.01232879856639e-08
4448 1.01142028086088e-08
4449 1.01219814752085e-08
4450 1.01129273843981e-08
4451 1.0120759341703e-08
4452 1.01118740047923e-08
4453 1.01184625123096e-08
4454 1.01130854801568e-08
4455 1.01181125700123e-08
4456 1.0109532766478e-08
4457 1.01176311773088e-08
4458 1.01088346582401e-08
4459 1.0116649740155e-08
4460 1.01078327929827e-08
4461 1.01152330955756e-08
4462 1.01090265047787e-08
4463 1.01148156517183e-08
4464 1.01056567558544e-08
4465 1.01133981189605e-08
4466 1.01043315936522e-08
4467 1.01127213270047e-08
4468 1.01053156953412e-08
4469 1.01116031103743e-08
4470 1.01102859417779e-08
4471 1.01028385657287e-08
4472 1.00956443205291e-08
4473 1.01028847510065e-08
4474 1.00897050714366e-08
4475 1.00950741099837e-08
4476 1.0084971968638e-08
4477 1.00926014212632e-08
4478 1.00839736560943e-08
4479 1.00829646854095e-08
4480 1.00818988713058e-08
4481 1.00812442838105e-08
4482 1.00828110305429e-08
4483 1.00820969350934e-08
4484 1.00817354464766e-08
4485 1.0080934309542e-08
4486 1.00795993773772e-08
4487 1.007954164578e-08
4488 1.00788382084716e-08
4489 1.00783195122744e-08
4490 1.00771062605531e-08
4491 1.00689687698718e-08
4492 1.00783141832039e-08
4493 1.0065576816487e-08
4494 1.00653076984258e-08
4495 1.00668557934114e-08
4496 1.00640917821693e-08
4497 1.0064808542154e-08
4498 1.00728616558854e-08
4499 1.00632293609237e-08
4500 1.00591348584089e-08
4501 1.00704626859738e-08
4502 1.00602965957819e-08
4503 1.006882932586e-08
4504 1.0056523613855e-08
4505 1.00564472305109e-08
4506 1.00542436598516e-08
4507 1.00598871455304e-08
4508 1.00506323263971e-08
4509 1.00642241207538e-08
4510 1.0055135390985e-08
4511 1.00573540606774e-08
4512 1.00475769926334e-08
4513 1.00499688571176e-08
4514 1.00497201671601e-08
4515 1.00582857598397e-08
4516 1.00517443257786e-08
4517 1.00595478613741e-08
4518 1.0046623977189e-08
4519 1.00473620534558e-08
4520 1.00571782013503e-08
4521 1.00461861052281e-08
4522 1.00541086567318e-08
4523 1.00443804385009e-08
4524 1.00530384017361e-08
4525 1.0052464638477e-08
4526 1.00413979353675e-08
4527 1.00421662097006e-08
4528 1.00523438462119e-08
4529 1.00507282496665e-08
4530 1.00403783065417e-08
4531 1.0048471388302e-08
4532 1.00488728449477e-08
4533 1.00369144107049e-08
4534 1.00377279821373e-08
4535 1.00487502763258e-08
4536 1.00388737322987e-08
4537 1.00327310903481e-08
4538 1.00299999417075e-08
4539 1.00277546266625e-08
4540 1.00246397849446e-08
4541 1.00225925336872e-08
4542 1.00190140628342e-08
4543 1.00202166564145e-08
4544 1.00175165940186e-08
4545 1.00177661721546e-08
4546 1.00172412587085e-08
4547 1.00164907479439e-08
4548 1.00159924798504e-08
4549 1.00155546078895e-08
4550 1.00150536752608e-08
4551 1.00143644488071e-08
4552 1.00135837399762e-08
4553 1.00134069924707e-08
4554 1.00123846991096e-08
4555 1.00116714918386e-08
4556 1.00102308664418e-08
4557 1.0011517836972e-08
4558 1.00107140355021e-08
4559 1.00093124899558e-08
4560 1.00075103759423e-08
4561 1.00092236721139e-08
4562 1.00063743957435e-08
4563 1.00110373324469e-08
4564 1.00071231301513e-08
4565 1.00047428119865e-08
4566 1.00041228634495e-08
4567 1.00033741290417e-08
4568 1.00023775928548e-08
4569 1.00042099049347e-08
4570 1.00016128712355e-08
4571 1.0002204398063e-08
4572 1.00023083149381e-08
4573 1.00007211401021e-08
4574 9.99945815038927e-09
4575 9.99878579932556e-09
4576 1.00004928782482e-08
4577 9.99787541644537e-09
4578 9.99968108317262e-09
4579 9.99678384516756e-09
4580 9.99580418437063e-09
4581 9.99768179354987e-09
4582 9.99512028698746e-09
4583 9.99412286262213e-09
4584 9.99362192999342e-09
4585 9.99539473411915e-09
4586 9.99470106677336e-09
4587 9.99394966783029e-09
4588 9.99342475438425e-09
4589 9.99333060747176e-09
4590 9.99263960466124e-09
4591 9.99204541329846e-09
4592 9.99149474267824e-09
4593 9.99100446819057e-09
4594 9.98825377962476e-09
4595 9.98765870008356e-09
4596 9.98723770351262e-09
4597 9.98862148549051e-09
4598 9.99970506398995e-09
4599 9.98638149951603e-09
4600 9.98684779318637e-09
4601 9.98537164065283e-09
4602 9.98621185743787e-09
4603 9.98678828523225e-09
4604 9.98645788286012e-09
4605 9.98404647845064e-09
4606 9.98378091310315e-09
4607 9.98357929660187e-09
4608 9.98197702273274e-09
4609 9.98189264578286e-09
4610 9.98123805828754e-09
4611 9.98066518320684e-09
4612 9.98004434649147e-09
4613 9.9813712850505e-09
4614 9.97893323528842e-09
4615 9.98038363064779e-09
4616 9.97819427084323e-09
4617 9.97966154159258e-09
4618 9.97866234087041e-09
4619 9.97584592710155e-09
4620 9.97600047014657e-09
4621 9.97541249603273e-09
4622 9.97499860488915e-09
4623 9.97458293738873e-09
4624 9.97685312142949e-09
4625 9.97351534692825e-09
4626 9.97511850897581e-09
4627 9.97157556525963e-09
4628 9.97209959052725e-09
4629 9.97147342474136e-09
4630 9.97342652908628e-09
4631 9.97300997340744e-09
4632 9.97260940494016e-09
4633 9.96891813542788e-09
4634 9.96910021200392e-09
4635 9.96845717082806e-09
4636 9.97075133568615e-09
4637 9.96721993828942e-09
4638 9.96951321496908e-09
4639 9.96738247494022e-09
4640 9.96878846137861e-09
4641 9.96480231663099e-09
4642 9.96782745232849e-09
4643 9.96432714117645e-09
4644 9.96486004822827e-09
4645 9.96396298802438e-09
4646 9.96586813073463e-09
4647 9.96506965833532e-09
4648 9.96523130680771e-09
4649 9.96152138554862e-09
4650 9.96412996556728e-09
4651 9.96060833813317e-09
4652 9.96295668187486e-09
4653 9.96214044590715e-09
4654 9.95907267764551e-09
4655 9.96147697662764e-09
4656 9.96079840831499e-09
4657 9.95715510043738e-09
4658 9.95999815955884e-09
4659 9.95627225108819e-09
4660 9.95936222381033e-09
4661 9.95564430894547e-09
4662 9.95827154071094e-09
4663 9.95805571335495e-09
4664 9.9550687693295e-09
4665 9.95664173331079e-09
4666 9.95600846209754e-09
4667 9.95267956938051e-09
4668 9.95504567669059e-09
4669 9.9518091545292e-09
4670 9.9540544695742e-09
4671 9.9505781392395e-09
4672 9.95083304644595e-09
4673 9.94975657420127e-09
4674 9.95300730721738e-09
4675 9.94959759026415e-09
4676 9.94888704752839e-09
4677 9.94852378255473e-09
4678 9.94798554643239e-09
4679 9.94729276726503e-09
4680 9.94690907418772e-09
4681 9.94642679330582e-09
4682 9.9458397073704e-09
4683 9.94526594411127e-09
4684 9.94476589966098e-09
4685 9.94461402115121e-09
4686 9.9442889478496e-09
4687 9.94359439232539e-09
4688 9.94337145954205e-09
4689 9.94276927457349e-09
4690 9.94250459740442e-09
4691 9.94183491087597e-09
4692 9.94150806121752e-09
4693 9.94098225959306e-09
4694 9.94024684786154e-09
4695 9.94013404920224e-09
4696 9.93962956385985e-09
4697 9.93909932134329e-09
4698 9.93867832477235e-09
4699 9.93823601191934e-09
4700 9.93783544345206e-09
4701 9.93734161625071e-09
4702 9.93663284987178e-09
4703 9.93648452407569e-09
4704 9.93597737419805e-09
4705 9.93531390491853e-09
4706 9.93487692113604e-09
4707 9.93440796293044e-09
4708 9.93399673632211e-09
4709 9.93369564383784e-09
4710 9.93311655150819e-09
4711 9.9327968072771e-09
4712 9.93211823896445e-09
4713 9.93158710826947e-09
4714 9.93145388150651e-09
4715 9.93046622710381e-09
4716 9.93020510264842e-09
4717 9.92964022117349e-09
4718 9.92970594637654e-09
4719 9.92883197881156e-09
4720 9.92857351889143e-09
4721 9.92779281006051e-09
4722 9.92733806270962e-09
4723 9.92733095728227e-09
4724 9.92662219090334e-09
4725 9.92637261276741e-09
4726 9.92585924564082e-09
4727 9.92521620446496e-09
4728 9.92453852433073e-09
4729 9.9243839812857e-09
4730 9.92444437741824e-09
4731 9.92367343854994e-09
4732 9.92308457625768e-09
4733 9.92263782251257e-09
4734 9.92227988660943e-09
4735 9.92165283264512e-09
4736 9.9211829862611e-09
4737 9.92080373407589e-09
4738 9.92036319757972e-09
4739 9.91977699982272e-09
4740 9.91953186257888e-09
4741 9.91913573500369e-09
4742 9.91865256594338e-09
4743 9.91868454036648e-09
4744 9.91763560165282e-09
4745 9.91728121846336e-09
4746 9.91693394070126e-09
4747 9.91656357030024e-09
4748 9.91582194131979e-09
4749 9.91557413954069e-09
4750 9.91544446549142e-09
4751 9.91466730937418e-09
4752 9.91405091355091e-09
4753 9.91370097125355e-09
4754 9.91318582777012e-09
4755 9.91275594941499e-09
4756 9.91215731716011e-09
4757 9.91194326616096e-09
4758 9.9113499629766e-09
4759 9.91082593770898e-09
4760 9.91023796359514e-09
4761 9.91019799556625e-09
4762 9.90946791290526e-09
4763 9.90916948495624e-09
4764 9.90829995828335e-09
4765 9.90829729374809e-09
4766 9.90723414417971e-09
4767 9.90706006120945e-09
4768 9.90670923073367e-09
4769 9.90643655995882e-09
4770 9.90586723759179e-09
4771 9.90489912311432e-09
4772 9.9048849122596e-09
4773 9.90794912780757e-09
4774 9.90763737718225e-09
4775 9.91014559303949e-09
4776 9.90273463230551e-09
4777 9.90546578094609e-09
4778 9.90483428608968e-09
4779 9.90405357725876e-09
4780 9.90381909815596e-09
4781 9.8990868835358e-09
4782 9.8984509477873e-09
4783 9.90211113105488e-09
4784 9.89736204104474e-09
4785 9.90068027562074e-09
4786 9.89644188820193e-09
4787 9.89603687884255e-09
4788 9.89999371370232e-09
4789 9.89489556957324e-09
4790 9.89872805945424e-09
4791 9.89384307814589e-09
4792 9.89808857099206e-09
4793 9.89297976872194e-09
4794 9.89685666752393e-09
4795 9.89197079803716e-09
4796 9.89624915348486e-09
4797 9.89138282392332e-09
4798 9.89539383766669e-09
4799 9.89044046662002e-09
4800 9.89414239427333e-09
4801 9.88912507438044e-09
4802 9.89370985138294e-09
4803 9.89014470320626e-09
4804 9.89235449111447e-09
4805 9.88412640623437e-09
4806 9.88214488018002e-09
4807 9.88252146782997e-09
4808 9.88221415809676e-09
4809 9.87914194894302e-09
4810 9.87842696531516e-09
4811 9.87806991759044e-09
4812 9.87958248543919e-09
4813 9.87782655670344e-09
4814 9.8786632207748e-09
4815 9.87699610988102e-09
4816 9.87661419316055e-09
4817 9.87589121592691e-09
4818 9.87598358648256e-09
4819 9.87677317709768e-09
4820 9.87726433976377e-09
4821 9.8746868459898e-09
4822 9.87398429685982e-09
4823 9.87371873151233e-09
4824 9.87296733256926e-09
4825 9.8727754860306e-09
4826 9.87209780589637e-09
4827 9.87149562092782e-09
4828 9.87112169781312e-09
4829 9.87067405588959e-09
4830 9.87009940445205e-09
4831 9.8717967134121e-09
4832 9.86957982007652e-09
4833 9.87096004934074e-09
4834 9.86889769905019e-09
4835 9.86984272088876e-09
4836 9.8676897763994e-09
4837 9.86918902157186e-09
4838 9.86688686310799e-09
4839 9.8686365745948e-09
4840 9.86771819810883e-09
4841 9.86559189897207e-09
4842 9.86664705493467e-09
4843 9.86709380867978e-09
4844 9.86624826282423e-09
4845 9.865327221803e-09
4846 9.86522152857106e-09
4847 9.86493198240623e-09
4848 9.86449677498058e-09
4849 9.86407044933912e-09
4850 9.86311743389479e-09
4851 9.86322579166199e-09
4852 9.86286963211569e-09
4853 9.86272041814118e-09
4854 9.86201964536804e-09
4855 9.86171677652692e-09
4856 9.86048398488037e-09
4857 9.86055681551079e-09
4858 9.85877512960087e-09
4859 9.85922365970282e-09
4860 9.85878489956349e-09
4861 9.85882753212763e-09
4862 9.85823866983537e-09
4863 9.85728298985578e-09
4864 9.85694281752103e-09
4865 9.85689219135111e-09
4866 9.85630954630778e-09
4867 9.85593828772835e-09
4868 9.8546690807666e-09
4869 9.85472148329336e-09
4870 9.85434134292973e-09
4871 9.85281900511836e-09
4872 9.85267956110647e-09
4873 9.85303660883119e-09
4874 9.85284032140044e-09
4875 9.85193970848286e-09
4876 9.85149561927301e-09
4877 9.85100889749901e-09
4878 9.85082060367404e-09
4879 9.85011627818722e-09
4880 9.8498134093461e-09
4881 9.84984271923395e-09
4882 9.8488488475823e-09
4883 9.84837367212776e-09
4884 9.847978432731e-09
4885 9.84748460552964e-09
4886 9.84675185833339e-09
4887 9.84559900274462e-09
4888 9.84583348184742e-09
4889 9.84592674058149e-09
4890 9.84418324634362e-09
4891 9.84396741898763e-09
4892 9.84316539387464e-09
4893 9.84376935520004e-09
4894 9.84317782837252e-09
4895 9.84179848728672e-09
4896 9.84212711330201e-09
4897 9.84124426395283e-09
4898 9.84150361205138e-09
4899 9.84129666647959e-09
4900 9.84045023244562e-09
4901 9.83997416881266e-09
4902 9.8396872871831e-09
4903 9.83952919142439e-09
4904 9.83880799054759e-09
4905 9.83805303889085e-09
4906 9.8381454094465e-09
4907 9.83758674522051e-09
4908 9.83707426627234e-09
4909 9.83710801705229e-09
4910 9.83650938479741e-09
4911 9.83480497041e-09
4912 9.83573578139385e-09
4913 9.8337542553395e-09
4914 9.83430847867339e-09
4915 9.83417525191044e-09
4916 9.83337766768955e-09
4917 9.83321690739558e-09
4918 9.83148851219084e-09
4919 9.83198500392746e-09
4920 9.83205783455787e-09
4921 9.83158976453069e-09
4922 9.82985248754176e-09
4923 9.83054260217386e-09
4924 9.82922987446955e-09
4925 9.82929648785102e-09
4926 9.82885861589011e-09
4927 9.82886305678221e-09
4928 9.82693837414672e-09
4929 9.82751746647637e-09
4930 9.82773151747551e-09
4931 9.82701031659872e-09
4932 9.82503856050698e-09
4933 9.82573133967435e-09
4934 9.82579972941267e-09
4935 9.82524017700825e-09
4936 9.82364678492331e-09
4937 9.8241512702657e-09
4938 9.82389636305925e-09
4939 9.82348069555883e-09
4940 9.82329861898279e-09
4941 9.82294245943649e-09
4942 9.82197168042376e-09
4943 9.82191750154016e-09
4944 9.82155334838808e-09
4945 9.82111103553507e-09
4946 9.82050973874493e-09
4947 9.82040582186983e-09
4948 9.81993153459371e-09
4949 9.81799175292508e-09
4950 9.81863923499304e-09
4951 9.81850778458693e-09
4952 9.81765602148243e-09
4953 9.81758319085202e-09
4954 9.81740910788176e-09
4955 9.8154995242794e-09
4956 9.81622871876198e-09
4957 9.81539383104746e-09
4958 9.81554038048671e-09
4959 9.81318937220976e-09
4960 9.81420456014348e-09
4961 9.81401715449692e-09
4962 9.81370718022845e-09
4963 9.8114822932871e-09
4964 9.81260583898802e-09
4965 9.81219372420128e-09
4966 9.81067582728201e-09
4967 9.81115366727181e-09
4968 9.81133574384785e-09
4969 9.80912329140438e-09
4970 9.80956738061423e-09
4971 9.80921033288951e-09
4972 9.80728298571876e-09
4973 9.80858061438994e-09
4974 9.80596492894392e-09
4975 9.80524372806713e-09
4976 9.80694192520559e-09
4977 9.806915279853e-09
4978 9.80625092239507e-09
4979 9.8057970632226e-09
4980 9.80584946574936e-09
4981 9.80281011919715e-09
4982 9.80257652827277e-09
4983 9.80380132631353e-09
4984 9.80399139649535e-09
4985 9.803768463712e-09
4986 9.80296377406376e-09
4987 9.8024877104308e-09
4988 9.80199565958628e-09
4989 9.79995373739939e-09
4990 9.8014734106755e-09
4991 9.80072378808927e-09
4992 9.7998729131632e-09
4993 9.79973702186498e-09
4994 9.79932845979192e-09
4995 9.79918191035267e-09
4996 9.79830883096611e-09
4997 9.79842962323119e-09
4998 9.79780878651582e-09
4999 9.79764092079449e-09
};
\addlegendentry{Test}

\nextgroupplot[
title={ELU/ELU $\hy$},
ymin=2.15131329810344e-09, ymax=1e-05,
]
\addplot [semithick, black, dashed]
table {%
0 0.00910873118822928
1 0.000609745856905647
2 0.000226115653555098
3 0.00020928675252253
4 0.000201760654364989
5 0.000182192063790353
6 0.000122173222587662
7 3.77127475366024e-05
8 1.85704894017533e-05
9 1.76400852603678e-05
10 1.74148618003045e-05
11 1.72507559139632e-05
12 1.712063632867e-05
13 1.70181753465215e-05
14 1.69379733252271e-05
15 1.68698922159791e-05
16 1.68049343276095e-05
17 1.67366979644186e-05
18 1.66603751946752e-05
19 1.65715753648357e-05
20 1.64631188539914e-05
21 1.63233637392466e-05
22 1.61321477939538e-05
23 1.58542895419345e-05
24 1.54250281558745e-05
25 1.47211584633311e-05
26 1.35094864149323e-05
27 1.14155001096066e-05
28 8.21599733812661e-06
29 4.88212171740798e-06
30 2.9931908371843e-06
31 2.3368413143956e-06
32 2.14548355765309e-06
33 2.08020611854387e-06
34 2.0414346597093e-06
35 2.00810549243613e-06
36 1.97581463568319e-06
37 1.94319220959827e-06
38 1.90945694968825e-06
39 1.87404797928004e-06
40 1.83647712763424e-06
41 1.79627180410868e-06
42 1.75293845545887e-06
43 1.70592917156931e-06
44 1.65460944280227e-06
45 1.59834871021225e-06
46 1.53643272483883e-06
47 1.46814955547292e-06
48 1.39292569756044e-06
49 1.31039823249246e-06
50 1.22095474923611e-06
51 1.12583976643954e-06
52 1.02745695191686e-06
53 9.29424907393894e-07
54 8.35822139464426e-07
55 7.50716643359794e-07
56 6.77083482623075e-07
57 6.16292502904514e-07
58 5.67982662353472e-07
59 5.30816838532289e-07
60 5.0336641819726e-07
61 4.8177558991469e-07
62 4.65373241883782e-07
63 4.52705868896786e-07
64 4.42715843071539e-07
65 4.34665495060216e-07
66 4.28309443735841e-07
67 4.22777524212492e-07
68 4.18197420929189e-07
69 4.1420678153159e-07
70 4.10772284324068e-07
71 4.07633282529574e-07
72 4.04841389157085e-07
73 4.02244891288817e-07
74 3.99878833495571e-07
75 3.97653953354116e-07
76 3.95590946237867e-07
77 3.93612962883871e-07
78 3.91750816391756e-07
79 3.89942248753883e-07
80 3.88226107051359e-07
81 3.86546921709474e-07
82 3.84940644864606e-07
83 3.8335462610295e-07
84 3.81809160527879e-07
85 3.80267510058374e-07
86 3.78869491577838e-07
87 3.77449670964758e-07
88 3.7606856683059e-07
89 3.74674842832157e-07
90 3.73285449726701e-07
91 3.72073191396538e-07
92 3.70850065895212e-07
93 3.69631142368654e-07
94 3.68451103707201e-07
95 3.67281746393111e-07
96 3.66147496464464e-07
97 3.65006727432338e-07
98 3.63908068830021e-07
99 3.62827802200982e-07
100 3.61766052517254e-07
101 3.60721763856375e-07
102 3.59694935648136e-07
103 3.5868496932423e-07
104 3.57691688814832e-07
105 3.56715518936568e-07
106 3.55783720228153e-07
107 3.55017173470529e-07
108 3.53748252008401e-07
109 3.53165873041128e-07
110 3.51910488999962e-07
111 3.51336131382496e-07
112 3.50143459036723e-07
113 3.49593837102091e-07
114 3.48375345764929e-07
115 3.47700964986153e-07
116 3.4694808818525e-07
117 3.45967220160226e-07
118 3.45316693170616e-07
119 3.4430375654626e-07
120 3.43656944657766e-07
121 3.4282238764316e-07
122 3.42036952476477e-07
123 3.4127347167523e-07
124 3.4050429313659e-07
125 3.39776207439968e-07
126 3.39035723789571e-07
127 3.38314922951e-07
128 3.37582298867822e-07
129 3.36858628544157e-07
130 3.36143700343072e-07
131 3.35430263297631e-07
132 3.34739984557686e-07
133 3.34033891368435e-07
134 3.33335630279485e-07
135 3.32644565162354e-07
136 3.31976251558785e-07
137 3.31290471404344e-07
138 3.3060059856016e-07
139 3.2994552519483e-07
140 3.29227688695255e-07
141 3.285848197212e-07
142 3.28003076202066e-07
143 3.27208561939685e-07
144 3.2664323476439e-07
145 3.25971153596782e-07
146 3.25325294099343e-07
147 3.24751496277642e-07
148 3.23999899901395e-07
149 3.23499543178229e-07
150 3.22714767468923e-07
151 3.22203879800753e-07
152 3.2147450855291e-07
153 3.20877223201066e-07
154 3.20284300390661e-07
155 3.1958659523923e-07
156 3.19076589666345e-07
157 3.1833390814473e-07
158 3.17841621177095e-07
159 3.17094955545549e-07
160 3.16614053678776e-07
161 3.15865697770867e-07
162 3.15394255261836e-07
163 3.14645005618175e-07
164 3.14180908841166e-07
165 3.13435735749046e-07
166 3.1297286058507e-07
167 3.1223753965115e-07
168 3.11774445730606e-07
169 3.11059446628903e-07
170 3.10620759647051e-07
171 3.09868839035943e-07
172 3.09388536251731e-07
173 3.08670241224718e-07
174 3.08190931511909e-07
175 3.07475454814465e-07
176 3.06995868235127e-07
177 3.06284878075047e-07
178 3.05802361828533e-07
179 3.05097637967933e-07
180 3.04611203749694e-07
181 3.03911679240265e-07
182 3.03422476172877e-07
183 3.02729155926329e-07
184 3.02234984508765e-07
185 3.01547979468353e-07
186 3.01048158093842e-07
187 3.00368387608607e-07
188 2.99862715540833e-07
189 2.99188434001785e-07
190 2.98678362542937e-07
191 2.98009485025119e-07
192 2.97494839860413e-07
193 2.9683159467897e-07
194 2.96312122694431e-07
195 2.95654998820005e-07
196 2.95133088095589e-07
197 2.94487388347697e-07
198 2.93964533622848e-07
199 2.93327716899938e-07
200 2.92788914087794e-07
201 2.92146143189953e-07
202 2.91598398439952e-07
203 2.90960419141228e-07
204 2.90403634161329e-07
205 2.89768935465062e-07
206 2.89204268053567e-07
207 2.88572649923324e-07
208 2.87999953381579e-07
209 2.8737065309592e-07
210 2.86788473156108e-07
211 2.86160286083081e-07
212 2.85573193703925e-07
213 2.84942797035725e-07
214 2.84347521875894e-07
215 2.83718005390732e-07
216 2.83115036473269e-07
217 2.82484011762563e-07
218 2.81872426142193e-07
219 2.81238399498562e-07
220 2.80613893436588e-07
221 2.79968963042165e-07
222 2.79331125004667e-07
223 2.78675701077091e-07
224 2.78026811237453e-07
225 2.77360761801404e-07
226 2.7669761004212e-07
227 2.76017224213554e-07
228 2.75337837797984e-07
229 2.74642300335692e-07
230 2.73942667029736e-07
231 2.73227399879339e-07
232 2.72505925390298e-07
233 2.71770205285904e-07
234 2.71041351767032e-07
235 2.70317266518383e-07
236 2.69594145939855e-07
237 2.68868619520468e-07
238 2.68148256745704e-07
239 2.67425019995216e-07
240 2.66706779272319e-07
241 2.65985430190518e-07
242 2.6526871702437e-07
243 2.64550053151069e-07
244 2.63834130437957e-07
245 2.63117801182489e-07
246 2.6240114202114e-07
247 2.61683729384288e-07
248 2.6096578799617e-07
249 2.60242030305058e-07
250 2.59512064982204e-07
251 2.58781868761382e-07
252 2.58073532706682e-07
253 2.57355696239081e-07
254 2.56652203660224e-07
255 2.55937999370737e-07
256 2.55239712509514e-07
257 2.54532647970152e-07
258 2.5384267817774e-07
259 2.53134657959286e-07
260 2.52412705839866e-07
261 2.51621666603086e-07
262 2.50940967105429e-07
263 2.50306754625562e-07
264 2.49616739551151e-07
265 2.48932251938783e-07
266 2.48248313929444e-07
267 2.47569354955068e-07
268 2.46886392775814e-07
269 2.46209278873977e-07
270 2.45511361018025e-07
271 2.44837164139255e-07
272 2.44255934939375e-07
273 2.43512464516371e-07
274 2.42904860696669e-07
275 2.42178827032369e-07
276 2.41512560640089e-07
277 2.40911707409808e-07
278 2.40249957386141e-07
279 2.39663861239414e-07
280 2.3898982782633e-07
281 2.3843902362497e-07
282 2.3773600106658e-07
283 2.37252356314777e-07
284 2.36485314126789e-07
285 2.36101468833461e-07
286 2.35242485866038e-07
287 2.3497062750355e-07
288 2.34019252705941e-07
289 2.33838845263179e-07
290 2.32827311496209e-07
291 2.32724483892e-07
292 2.31664282800637e-07
293 2.31605044985983e-07
294 2.30492883662059e-07
295 2.30495727576496e-07
296 2.29343666394044e-07
297 2.29393584628923e-07
298 2.28192804073046e-07
299 2.28355417941195e-07
300 2.27061974412024e-07
301 2.27345367782128e-07
302 2.25928060619829e-07
303 2.26164229367498e-07
304 2.24846075738583e-07
305 2.25136948570182e-07
306 2.23781520237942e-07
307 2.24169945815866e-07
308 2.22625489150197e-07
309 2.23126716633715e-07
310 2.2153117664736e-07
311 2.22142542744663e-07
312 2.20411288356814e-07
313 2.2118749406097e-07
314 2.19272853204444e-07
315 2.20290084039121e-07
316 2.18118038642245e-07
317 2.19453813895321e-07
318 2.16954384194779e-07
319 2.1867407853815e-07
320 2.15756116197952e-07
321 2.17925215186376e-07
322 2.14529789821505e-07
323 2.17233268707595e-07
324 2.13341055951055e-07
325 2.16500755873206e-07
326 2.12194866185733e-07
327 2.15703804691714e-07
328 2.11034177996616e-07
329 2.1492024225811e-07
330 2.09928635306333e-07
331 2.14020260522751e-07
332 2.08839821788764e-07
333 2.13154887833689e-07
334 2.07846662173772e-07
335 2.12254508063836e-07
336 2.06886391307393e-07
337 2.11374859498292e-07
338 2.0589912135538e-07
339 2.10501552636799e-07
340 2.04963402429925e-07
341 2.09592284004856e-07
342 2.0407193025207e-07
343 2.08705889257033e-07
344 2.03088715438504e-07
345 2.0782616826498e-07
346 2.02234135513457e-07
347 2.06966331523617e-07
348 2.01216544728489e-07
349 2.06095453999744e-07
350 2.00309206900506e-07
351 2.0524782386655e-07
352 1.99404306633788e-07
353 2.04410877806538e-07
354 1.98541788491546e-07
355 2.03560915850609e-07
356 1.97618780640951e-07
357 2.02679983513754e-07
358 1.96738877444425e-07
359 2.01796065732118e-07
360 1.95854132704731e-07
361 2.00930533672761e-07
362 1.94980844037573e-07
363 2.00048619271698e-07
364 1.94113974314902e-07
365 1.99166747185231e-07
366 1.93243412925703e-07
367 1.98279761216114e-07
368 1.92369986946161e-07
369 1.97404114308242e-07
370 1.91500220169338e-07
371 1.96486192899137e-07
372 1.9058559656937e-07
373 1.95680430014278e-07
374 1.89547050892358e-07
375 1.94734120019024e-07
376 1.88728256455661e-07
377 1.93847528218605e-07
378 1.87866179281571e-07
379 1.92963158231585e-07
380 1.87006319163352e-07
381 1.92178125450582e-07
382 1.86299310083626e-07
383 1.91180663248414e-07
384 1.85394783230652e-07
385 1.90289948329703e-07
386 1.84546415505249e-07
387 1.89377310888084e-07
388 1.83712015462412e-07
389 1.88476378594427e-07
390 1.82901871763175e-07
391 1.87537727713849e-07
392 1.8205573492569e-07
393 1.86591090197474e-07
394 1.81225562663201e-07
395 1.85640647313523e-07
396 1.80396322881915e-07
397 1.84677284721158e-07
398 1.79562350998363e-07
399 1.83689282254651e-07
400 1.78702010817045e-07
401 1.8266751202356e-07
402 1.77869857912683e-07
403 1.81676614332993e-07
404 1.77071672806672e-07
405 1.8067265834798e-07
406 1.7628239487788e-07
407 1.79661808141773e-07
408 1.75493516862968e-07
409 1.78639432346639e-07
410 1.74712384035658e-07
411 1.7760212733231e-07
412 1.73938942235274e-07
413 1.76549594911091e-07
414 1.73174631858952e-07
415 1.7548011363866e-07
416 1.72416734952829e-07
417 1.74392432890969e-07
418 1.71665416905942e-07
419 1.73293695463705e-07
420 1.70913532452133e-07
421 1.72190555172147e-07
422 1.70155328448995e-07
423 1.71088398883246e-07
424 1.69384724817423e-07
425 1.6999564980491e-07
426 1.68593118554661e-07
427 1.68920645925219e-07
428 1.67772851292547e-07
429 1.67867018384449e-07
430 1.66924397404511e-07
431 1.66834841164754e-07
432 1.66048756686799e-07
433 1.65817401547752e-07
434 1.65146106369196e-07
435 1.64815683582287e-07
436 1.64219003249144e-07
437 1.63830883698868e-07
438 1.63270771801116e-07
439 1.62844511029991e-07
440 1.62315625001686e-07
441 1.61858689973648e-07
442 1.61347430706282e-07
443 1.60869754056048e-07
444 1.60370602095838e-07
445 1.59880080877528e-07
446 1.59387139238021e-07
447 1.58888579740779e-07
448 1.58389085324284e-07
449 1.57889349434281e-07
450 1.57392087157149e-07
451 1.56881891006044e-07
452 1.56382942425815e-07
453 1.55870386374435e-07
454 1.55370121056375e-07
455 1.54854715478159e-07
456 1.5434626370725e-07
457 1.53836964150944e-07
458 1.5332023839898e-07
459 1.52809146083843e-07
460 1.52290635428987e-07
461 1.51777417288113e-07
462 1.51256619123519e-07
463 1.50743200443504e-07
464 1.50224576495273e-07
465 1.49703913201016e-07
466 1.49187510396054e-07
467 1.48666245807672e-07
468 1.48148264761172e-07
469 1.47629785664627e-07
470 1.47109597422812e-07
471 1.46590411123526e-07
472 1.46069636575152e-07
473 1.4554717108517e-07
474 1.45026536670922e-07
475 1.44503935706641e-07
476 1.43980434521129e-07
477 1.43459039176008e-07
478 1.4293688748257e-07
479 1.42414599836549e-07
480 1.41894372295059e-07
481 1.41373122020116e-07
482 1.40851017240262e-07
483 1.4032342264958e-07
484 1.39794656070169e-07
485 1.39262761260195e-07
486 1.38728959420131e-07
487 1.38190931358917e-07
488 1.3765188413517e-07
489 1.3710272604861e-07
490 1.36557166189988e-07
491 1.36032957340326e-07
492 1.3547122283919e-07
493 1.34920152782048e-07
494 1.34359218542546e-07
495 1.33798556904097e-07
496 1.33232892942647e-07
497 1.32671021269903e-07
498 1.32112207170909e-07
499 1.3155395936737e-07
500 1.30982579734873e-07
501 1.3041306415662e-07
502 1.29844484026975e-07
503 1.29277537663341e-07
504 1.28711126163861e-07
505 1.28145600045926e-07
506 1.27584830133198e-07
507 1.27021745593314e-07
508 1.26463661667575e-07
509 1.25903905734415e-07
510 1.2534616153248e-07
511 1.24789376173329e-07
512 1.24232137572289e-07
513 1.23678940107563e-07
514 1.23121907297463e-07
515 1.22569886766843e-07
516 1.22016126370195e-07
517 1.21462649019133e-07
518 1.20912103982018e-07
519 1.20359257066216e-07
520 1.19807195483101e-07
521 1.19254567549465e-07
522 1.18704734446862e-07
523 1.18152507312619e-07
524 1.17601618795149e-07
525 1.17049524191781e-07
526 1.16498606898663e-07
527 1.1594560230721e-07
528 1.1539541250194e-07
529 1.14839650456666e-07
530 1.14299186475986e-07
531 1.13721278967915e-07
532 1.13225313436605e-07
533 1.12574150306077e-07
534 1.12142330794818e-07
535 1.11469481363091e-07
536 1.11053104849379e-07
537 1.10362927991048e-07
538 1.09958077175953e-07
539 1.09264415541599e-07
540 1.08865300195493e-07
541 1.08166282858146e-07
542 1.07775812943078e-07
543 1.07070644970619e-07
544 1.0668878764486e-07
545 1.05977159879256e-07
546 1.05609564924336e-07
547 1.0488335588521e-07
548 1.04536262457877e-07
549 1.03789867243353e-07
550 1.03467222786868e-07
551 1.02714955725336e-07
552 1.02403534433471e-07
553 1.01626940944843e-07
554 1.01224919883958e-07
555 1.00800621471109e-07
556 1.00031672113143e-07
557 9.96201814911224e-08
558 9.92216055468731e-08
559 9.84652276763143e-08
560 9.81729762079553e-08
561 9.74233618382669e-08
562 9.70283004915196e-08
563 9.66483757300907e-08
564 9.59018776240939e-08
565 9.54919310549585e-08
566 9.49711725266589e-08
567 9.45208302232103e-08
568 9.41541574426275e-08
569 9.34617425127904e-08
570 9.32091587899642e-08
571 9.24952334555584e-08
572 9.22420170956784e-08
573 9.15320254768481e-08
574 9.12970189932416e-08
575 9.05803297559515e-08
576 9.03727614880445e-08
577 8.96524649998121e-08
578 8.94715857451267e-08
579 8.8745363489906e-08
580 8.85594756305785e-08
581 8.78388978691191e-08
582 8.77083599015727e-08
583 8.69546982116276e-08
584 8.6821377236479e-08
585 8.60906188688837e-08
586 8.59689028080979e-08
587 8.52374034590042e-08
588 8.51258055751813e-08
589 8.44033756166596e-08
590 8.42989182192611e-08
591 8.35863592594244e-08
592 8.34883436633405e-08
593 8.278574747278e-08
594 8.26929105612884e-08
595 8.200204611164e-08
596 8.19147339283788e-08
597 8.12363396525129e-08
598 8.11526954582753e-08
599 8.04880012910125e-08
600 8.04050021332436e-08
601 7.975648654579e-08
602 7.96691973978803e-08
603 7.90382191957484e-08
604 7.89409023695953e-08
605 7.83263140800017e-08
606 7.82160037364754e-08
607 7.7612791961279e-08
608 7.75068790823674e-08
609 7.69330309404737e-08
610 7.68305694278659e-08
611 7.62707930701545e-08
612 7.61636131545451e-08
613 7.56181874379536e-08
614 7.55073843805576e-08
615 7.49788683149788e-08
616 7.4867491381081e-08
617 7.4360185162714e-08
618 7.42530953936615e-08
619 7.3767125994717e-08
620 7.36602117461516e-08
621 7.31886797051828e-08
622 7.30797154231411e-08
623 7.26187820210278e-08
624 7.25005995478867e-08
625 7.20473523228016e-08
626 7.19447707071552e-08
627 7.15176362731107e-08
628 7.14182936065733e-08
629 7.10000289316781e-08
630 7.0908226880384e-08
631 7.04973247138696e-08
632 7.04067585988177e-08
633 7.00078572748453e-08
634 6.99180609884387e-08
635 6.95289952714617e-08
636 6.94384234654066e-08
637 6.9061135985482e-08
638 6.89652722161149e-08
639 6.86083693417494e-08
640 6.84970160582665e-08
641 6.81668963884441e-08
642 6.80384746867269e-08
643 6.7734774384931e-08
644 6.76092599007916e-08
645 6.73110683102607e-08
646 6.71926672111134e-08
647 6.68984850595855e-08
648 6.67816522104125e-08
649 6.64870843358756e-08
650 6.63711184416549e-08
651 6.60810101131837e-08
652 6.59657048402984e-08
653 6.56866973085535e-08
654 6.55759929757416e-08
655 6.53026956984881e-08
656 6.51932304984726e-08
657 6.49289993028823e-08
658 6.4827166278647e-08
659 6.45651463599961e-08
660 6.44628452843854e-08
661 6.42114840623265e-08
662 6.41159067922281e-08
663 6.38648480748571e-08
664 6.37740486402194e-08
665 6.35250453262515e-08
666 6.34462281858816e-08
667 6.319278772704e-08
668 6.31262136563393e-08
669 6.2865469545148e-08
670 6.2825174382386e-08
671 6.25382942343045e-08
672 6.25422533171438e-08
673 6.22091046231787e-08
674 6.22818537601155e-08
675 6.18904876561466e-08
676 6.19963033559046e-08
677 6.15927389566728e-08
678 6.17132240430252e-08
679 6.12998439448731e-08
680 6.14291955538526e-08
681 6.10178890925184e-08
682 6.11494645990085e-08
683 6.0739638199081e-08
684 6.0871940342988e-08
685 6.04715232945807e-08
686 6.05966133671032e-08
687 6.02072790738539e-08
688 6.03265110679096e-08
689 5.99506798493188e-08
690 6.00591114574556e-08
691 5.96989461629782e-08
692 5.97939095139388e-08
693 5.94544394236785e-08
694 5.95287296119018e-08
695 5.92176029705627e-08
696 5.92681732167755e-08
697 5.89927672396406e-08
698 5.89997149349486e-08
699 5.87765897481773e-08
700 5.87459881371544e-08
701 5.85422053860363e-08
702 5.85161215873597e-08
703 5.83180489188706e-08
704 5.82806156623761e-08
705 5.80908911640421e-08
706 5.80503091809348e-08
707 5.78645380371334e-08
708 5.7835905443504e-08
709 5.76524342186246e-08
710 5.76205384863471e-08
711 5.74384034681685e-08
712 5.74059486706524e-08
713 5.72295479024376e-08
714 5.71966771305554e-08
715 5.70209762487117e-08
716 5.69907013847626e-08
717 5.68163436782676e-08
718 5.67862968896549e-08
719 5.66144961897663e-08
720 5.65886158985229e-08
721 5.64135613512207e-08
722 5.63895645551149e-08
723 5.6218089949045e-08
724 5.61969341505986e-08
725 5.6026323995706e-08
726 5.60026893907661e-08
727 5.58364688898472e-08
728 5.58111539863404e-08
729 5.5652398694761e-08
730 5.56209907918159e-08
731 5.54704883657386e-08
732 5.54366478060508e-08
733 5.52870641379322e-08
734 5.52551371866272e-08
735 5.51136913498063e-08
736 5.50688335865246e-08
737 5.49420339299722e-08
738 5.48877234427536e-08
739 5.47720508614624e-08
740 5.47118351987841e-08
741 5.46030801746866e-08
742 5.45343820004263e-08
743 5.44419403500918e-08
744 5.43604059970448e-08
745 5.42773940979124e-08
746 5.41937167226614e-08
747 5.41120177821597e-08
748 5.40240133370951e-08
749 5.39499433052093e-08
750 5.38641538088136e-08
751 5.3785752300417e-08
752 5.37064132530851e-08
753 5.36189898725503e-08
754 5.35580583331807e-08
755 5.34498272002359e-08
756 5.34122839290774e-08
757 5.32811715490133e-08
758 5.32715958496688e-08
759 5.31100737473089e-08
760 5.31325690058537e-08
761 5.29412010825681e-08
762 5.29937458866314e-08
763 5.27755268540098e-08
764 5.28511304529466e-08
765 5.26118110384388e-08
766 5.2707415771458e-08
767 5.2451751401783e-08
768 5.25613412594161e-08
769 5.22950260317678e-08
770 5.24109049830912e-08
771 5.21375849693584e-08
772 5.22597758232379e-08
773 5.19740166282645e-08
774 5.21008350404806e-08
775 5.18060584968616e-08
776 5.19246378023475e-08
777 5.15987483891145e-08
778 5.1666059291744e-08
779 5.11396677738141e-08
780 5.09309903433675e-08
781 5.070108830596e-08
782 5.08452501883916e-08
783 5.05390538014794e-08
784 5.07760783490241e-08
785 5.03827694635639e-08
786 5.06493657734808e-08
787 5.02209073185433e-08
788 5.05073963310565e-08
789 5.00941638712504e-08
790 5.03660730042377e-08
791 4.9928958408163e-08
792 5.0229148941483e-08
793 4.98070942693829e-08
794 5.00866447468162e-08
795 4.96427919070008e-08
796 4.99575418091425e-08
797 4.95180271795626e-08
798 4.98200424461359e-08
799 4.93662174250176e-08
800 4.969271174593e-08
801 4.92293564113844e-08
802 4.95613212363466e-08
803 4.90927465679469e-08
804 4.94341898766848e-08
805 4.8962338753622e-08
806 4.92962012077491e-08
807 4.88342468911718e-08
808 4.91662839321449e-08
809 4.8706836045298e-08
810 4.90286509480242e-08
811 4.85812343868908e-08
812 4.8898804255515e-08
813 4.84608492530292e-08
814 4.87612914179536e-08
815 4.83372215347444e-08
816 4.86283425775724e-08
817 4.82176576852567e-08
818 4.84984222102192e-08
819 4.80947184571257e-08
820 4.83884056921013e-08
821 4.79643765713966e-08
822 4.82514634505549e-08
823 4.786286916314e-08
824 4.81930590972013e-08
825 4.77305301345154e-08
826 4.79882777919549e-08
827 4.76428432794052e-08
828 4.79558485766418e-08
829 4.75619032416219e-08
830 4.77019328584127e-08
831 4.74664621337162e-08
832 4.76364175647781e-08
833 4.7315132598591e-08
834 4.75617626829461e-08
835 4.72442873258316e-08
836 4.73608982218821e-08
837 4.70962611127668e-08
838 4.74102336891491e-08
839 4.70292851981746e-08
840 4.71169006859284e-08
841 4.69010957568905e-08
842 4.7087832713899e-08
843 4.67203560350349e-08
844 4.72305159151531e-08
845 4.66458249661628e-08
846 4.71343387116185e-08
847 4.65593564378253e-08
848 4.6790860336543e-08
849 4.6404187736826e-08
850 4.69320046279087e-08
851 4.62981982898292e-08
852 4.7140760052411e-08
853 4.6807106636404e-08
854 4.71874234189151e-08
855 4.65769806656802e-08
856 4.7037976529829e-08
857 4.65165233047848e-08
858 4.70825730198854e-08
859 4.63012604798863e-08
860 4.64477903836524e-08
861 4.63180600691615e-08
862 4.58024637128851e-08
863 4.64934135249173e-08
864 4.65381021728106e-08
865 4.63176524068132e-08
866 4.64437840372867e-08
867 4.62149186311045e-08
868 4.63682810210031e-08
869 4.60753426971827e-08
870 4.62499963198582e-08
871 4.60766903995768e-08
872 4.62224774477882e-08
873 4.58541374261934e-08
874 4.61067156480865e-08
875 4.58478246421201e-08
876 4.5884309760158e-08
877 4.59529978877882e-08
878 4.55871406348507e-08
879 4.58784386374589e-08
880 4.5667767301083e-08
881 4.5680597254627e-08
882 4.54420239437425e-08
883 4.59241804962041e-08
884 4.50925070967223e-08
885 4.56945805877851e-08
886 4.52072701873085e-08
887 4.58509482745839e-08
888 4.52584502954601e-08
889 4.43243180949793e-08
890 4.64486479034853e-08
891 4.45116719411942e-08
892 4.45616377904301e-08
893 4.44080412402847e-08
894 4.5494661742751e-08
895 4.49855049016179e-08
896 4.40670903798512e-08
897 4.60959658354909e-08
898 4.44306348592605e-08
899 4.43368965901136e-08
900 4.42158254623237e-08
901 4.40077675678552e-08
902 4.55794610156879e-08
903 4.42823908310963e-08
904 4.40999667241648e-08
905 4.40428489749678e-08
906 4.38210686120044e-08
907 4.4619069418772e-08
908 4.48908332790765e-08
909 4.41192812412616e-08
910 4.50115961840858e-08
911 4.41663097840994e-08
912 4.36110532895384e-08
913 4.43936824718882e-08
914 4.32697334149523e-08
915 4.43659552500364e-08
916 4.48392016774157e-08
917 4.38905434081427e-08
918 4.31756264469563e-08
919 4.4164485085485e-08
920 4.44080118788825e-08
921 4.39720927065679e-08
922 4.50593238130459e-08
923 4.33132322190666e-08
924 4.3152707087657e-08
925 4.34417566030909e-08
926 4.30168683145382e-08
927 4.45951901184571e-08
928 4.34459724996117e-08
929 4.30732527951783e-08
930 4.27554815733977e-08
931 4.45986585022595e-08
932 4.2944613349416e-08
933 4.29427738848354e-08
934 4.42668034419391e-08
935 4.28278298514329e-08
936 4.27034440591001e-08
937 4.42047735147444e-08
938 4.28923833268868e-08
939 4.24764304405301e-08
940 4.40100071203986e-08
941 4.28260386340362e-08
942 4.23722407059124e-08
943 4.40801062469021e-08
944 4.27601649453102e-08
945 4.21787860953593e-08
946 4.40283578799772e-08
947 4.24217918260084e-08
948 4.22775848365475e-08
949 4.35594636369885e-08
950 4.30093637220352e-08
951 4.18808191930964e-08
952 4.35889404792888e-08
953 4.24896205633019e-08
954 4.18517989566602e-08
955 4.36951258977469e-08
956 4.23617496996265e-08
957 4.17838801816117e-08
958 4.34478595114918e-08
959 4.20496632451695e-08
960 4.27554937920682e-08
961 4.32155334777473e-08
962 4.20422569797907e-08
963 4.21405542572195e-08
964 4.23354678025234e-08
965 4.17031197943452e-08
966 4.34455043387683e-08
967 4.1659718565068e-08
968 4.20793141540088e-08
969 4.31104823144235e-08
970 4.17673958623688e-08
971 4.16381967616175e-08
972 4.2903401572536e-08
973 4.17128916305742e-08
974 4.23854257016743e-08
975 4.24743603180122e-08
976 4.30717884076781e-08
977 4.1183200857553e-08
978 4.32501466933743e-08
979 4.10873004477175e-08
980 4.25035724820333e-08
981 4.17371629013452e-08
982 4.27189579301768e-08
983 4.10806245261242e-08
984 4.31958511135644e-08
985 4.07558195352209e-08
986 4.28099649452118e-08
987 4.11593669857879e-08
988 4.26758828029339e-08
989 4.09947049133663e-08
990 4.26853393180604e-08
991 4.07181435524429e-08
992 4.29036166091912e-08
993 4.07072674728504e-08
994 4.28811957138286e-08
995 4.05773672738086e-08
996 4.28060460404556e-08
997 4.05610250222566e-08
998 4.27468195958181e-08
999 4.04632773021163e-08
1000 4.27299486243093e-08
1001 4.04121291035509e-08
1002 4.26779641877761e-08
1003 4.03394665364765e-08
1004 4.26627628844489e-08
1005 4.0286148036639e-08
1006 4.25290758003793e-08
1007 4.02589960695199e-08
1008 4.25381400879221e-08
1009 4.01885415983738e-08
1010 4.24660985658409e-08
1011 4.01438636332685e-08
1012 4.24485220245074e-08
1013 4.00718026030145e-08
1014 4.23615404867572e-08
1015 4.00467782026848e-08
1016 4.23387130994435e-08
1017 3.99757950382273e-08
1018 4.22594183004943e-08
1019 3.99409444085119e-08
1020 4.22253195710987e-08
1021 3.98862585784254e-08
1022 4.21489688553489e-08
1023 3.9852986675859e-08
1024 4.21068926572676e-08
1025 3.98000710630786e-08
1026 4.20413870547698e-08
1027 3.9757466082202e-08
1028 4.19848553392388e-08
1029 3.97192811840696e-08
1030 4.19250835683727e-08
1031 3.96739397363621e-08
1032 4.1857311639415e-08
1033 3.96416663195343e-08
1034 4.17974218476225e-08
1035 3.95884808062075e-08
1036 4.1733213536288e-08
1037 3.95713709249179e-08
1038 4.16643053615662e-08
1039 3.95315174763766e-08
1040 4.15991043214436e-08
1041 3.94911031365019e-08
1042 4.15334052252803e-08
1043 3.94534136256564e-08
1044 4.14652013220884e-08
1045 3.94269212791176e-08
1046 4.14003687951858e-08
1047 3.93776915028621e-08
1048 4.13339844522742e-08
1049 3.93516120631432e-08
1050 4.12745262545755e-08
1051 3.92988809239103e-08
1052 4.12229296271605e-08
1053 3.92487944465003e-08
1054 4.11731776845947e-08
1055 3.92031624047995e-08
1056 4.11262506962995e-08
1057 3.9147678477125e-08
1058 4.11185755775367e-08
1059 3.909696892479e-08
1060 4.10918464834875e-08
1061 3.90264453533629e-08
1062 4.10430797987971e-08
1063 3.90292793512881e-08
1064 4.09761944335507e-08
1065 3.89350281817524e-08
1066 4.0931437848446e-08
1067 3.89525137356461e-08
1068 4.08579268733611e-08
1069 3.88532522139773e-08
1070 4.08154758526624e-08
1071 3.88702020219789e-08
1072 4.07397455493008e-08
1073 3.87722629138221e-08
1074 4.06998526518709e-08
1075 3.87994773181077e-08
1076 4.06251230262722e-08
1077 3.86801651226509e-08
1078 4.05883645884231e-08
1079 3.8752763212635e-08
1080 4.0513093535699e-08
1081 3.85511696083807e-08
1082 4.04777392764011e-08
1083 3.87675622648409e-08
1084 4.0405457918169e-08
1085 3.83645849264536e-08
1086 4.03067729066731e-08
1087 3.91628480804762e-08
1088 4.00180618205237e-08
1089 3.84710690177759e-08
1090 4.03077213140257e-08
1091 3.82324863801298e-08
1092 4.01813395123352e-08
1093 3.89168062937539e-08
1094 4.01605840696373e-08
1095 3.7915976798697e-08
1096 3.98478661842727e-08
1097 3.92424242197098e-08
1098 3.88696247073383e-08
1099 3.90609446256907e-08
1100 3.93340511148921e-08
1101 3.90895691984028e-08
1102 3.8528754068734e-08
1103 3.88742376933315e-08
1104 3.98589417938133e-08
1105 3.80892686011602e-08
1106 3.97320553302549e-08
1107 3.81984097035382e-08
1108 3.97533506573122e-08
1109 3.79062379123329e-08
1110 3.96894328658082e-08
1111 3.86386002315664e-08
1112 3.91515560651534e-08
1113 3.86663022629019e-08
1114 3.85828142663858e-08
1115 3.87130261279012e-08
1116 3.8935142086638e-08
1117 3.86838498662723e-08
1118 3.82479957072857e-08
1119 3.84745883599269e-08
1120 3.93554396092011e-08
1121 3.75816245370419e-08
1122 3.94227395092983e-08
1123 3.85049670958981e-08
1124 3.81680191210432e-08
1125 3.8334941284468e-08
1126 3.90699689245544e-08
1127 3.76596343660829e-08
1128 3.93122127224643e-08
1129 3.7587419734697e-08
1130 3.92520936594032e-08
1131 3.76501106822413e-08
1132 3.91913236594821e-08
1133 3.74090677757177e-08
1134 3.91448701662078e-08
1135 3.80804265660561e-08
1136 3.84063218841213e-08
1137 3.82400579510556e-08
1138 3.78586986942153e-08
1139 3.80297115236683e-08
1140 3.87660340883667e-08
1141 3.73182759296764e-08
1142 3.89975496390971e-08
1143 3.74393867377254e-08
1144 3.89124110411476e-08
1145 3.70714172650555e-08
1146 3.8811471700706e-08
1147 3.79718514722516e-08
1148 3.79922563418234e-08
1149 3.7845968867023e-08
1150 3.79388700080252e-08
1151 3.76981723979597e-08
1152 3.78387489814447e-08
1153 3.7955933052114e-08
1154 3.76443482019706e-08
1155 3.80283129652881e-08
1156 3.77125849990723e-08
1157 3.79542007657996e-08
1158 3.75663008229665e-08
1159 3.78977867765418e-08
1160 3.76296380695251e-08
1161 3.78261497153698e-08
1162 3.75212498742172e-08
1163 3.77978018493152e-08
1164 3.75708452615697e-08
1165 3.77242741504968e-08
1166 3.74423257947232e-08
1167 3.77243825293583e-08
1168 3.74375079283951e-08
1169 3.75169469293279e-08
1170 3.74108695131437e-08
1171 3.7623372010076e-08
1172 3.74167719408014e-08
1173 3.75797276406242e-08
1174 3.718354500859e-08
1175 3.75317135699049e-08
1176 3.74421764814947e-08
1177 3.75018722489351e-08
1178 3.70139569580985e-08
1179 3.73595444862485e-08
1180 3.75213447751932e-08
1181 3.74466535553175e-08
1182 3.69588913868846e-08
1183 3.74290346452089e-08
1184 3.73939883140384e-08
1185 3.73022472037743e-08
1186 3.69541417095665e-08
1187 3.72575122018493e-08
1188 3.71279782613865e-08
1189 3.73608549038007e-08
1190 3.67604585449488e-08
1191 3.7290604941953e-08
1192 3.72089256508446e-08
1193 3.71283532294431e-08
1194 3.68373189316706e-08
1195 3.70494816444022e-08
1196 3.69541701976672e-08
1197 3.72213392418885e-08
1198 3.65853144994599e-08
1199 3.7018240613973e-08
1200 3.7158504283985e-08
1201 3.69482503375096e-08
1202 3.66696542546396e-08
1203 3.70401258822373e-08
1204 3.66399928568573e-08
1205 3.68615384838122e-08
1206 3.67779690690906e-08
1207 3.69739429491034e-08
1208 3.64260300100838e-08
1209 3.69723293920376e-08
1210 3.65574008960312e-08
1211 3.67570060659794e-08
1212 3.64689266358909e-08
1213 3.67763387407738e-08
1214 3.6650992502052e-08
1215 3.68165952460142e-08
1216 3.62815226697677e-08
1217 3.67674327113043e-08
1218 3.64233859269092e-08
1219 3.6754917352777e-08
1220 3.62181220181501e-08
1221 3.65405553677256e-08
1222 3.65886713769026e-08
1223 3.65959830010976e-08
1224 3.61485197242217e-08
1225 3.65071015639984e-08
1226 3.64223163025201e-08
1227 3.65275591136527e-08
1228 3.60837131030411e-08
1229 3.64345533634403e-08
1230 3.61849622330102e-08
1231 3.63131912968218e-08
1232 3.63911332481592e-08
1233 3.63573658106464e-08
1234 3.59780509251451e-08
1235 3.61962368236046e-08
1236 3.63399896128325e-08
1237 3.61691616781457e-08
1238 3.60058544561781e-08
1239 3.62156836430927e-08
1240 3.62829802422748e-08
1241 3.60647978119299e-08
1242 3.59491996895223e-08
1243 3.61326473443668e-08
1244 3.59428475475987e-08
1245 3.60257854317147e-08
1246 3.61789064555751e-08
1247 3.60555042815403e-08
1248 3.57487688804614e-08
1249 3.59822963795775e-08
1250 3.58888610370034e-08
1251 3.59235262432644e-08
1252 3.58046700654757e-08
1253 3.58797421264878e-08
1254 3.58858821241093e-08
1255 3.59949496193668e-08
1256 3.5751773865389e-08
1257 3.57193580378112e-08
1258 3.58208565764961e-08
1259 3.56834971522169e-08
1260 3.58642176883084e-08
1261 3.56595690900186e-08
1262 3.55785111318951e-08
1263 3.56882661018965e-08
1264 3.56927828581988e-08
1265 3.5884795125618e-08
1266 3.55324416418945e-08
1267 3.586218964724e-08
1268 3.53195645010906e-08
1269 3.56910825300982e-08
1270 3.56388955931042e-08
1271 3.54314882395901e-08
1272 3.56599753494891e-08
1273 3.5548160651544e-08
1274 3.54983692987254e-08
1275 3.53737859193348e-08
1276 3.54710382335366e-08
1277 3.54036549983228e-08
1278 3.54200858794229e-08
1279 3.55653264714206e-08
1280 3.52232515496986e-08
1281 3.54038541130475e-08
1282 3.54053775846186e-08
1283 3.5379794202095e-08
1284 3.53434150293097e-08
1285 3.53505066701398e-08
1286 3.52095623996629e-08
1287 3.53226289933062e-08
1288 3.49805382910962e-08
1289 3.53784891831133e-08
1290 3.52981846669653e-08
1291 3.50955279242449e-08
1292 3.5412242671562e-08
1293 3.51114979932898e-08
1294 3.51138129306428e-08
1295 3.51708165409192e-08
1296 3.53309674343549e-08
1297 3.49931639607171e-08
1298 3.49261906340548e-08
1299 3.51941208662776e-08
1300 3.50559200756884e-08
1301 3.50972788802917e-08
1302 3.47548599011205e-08
1303 3.51791090178466e-08
1304 3.49079306267708e-08
1305 3.51403563130237e-08
1306 3.47397542377337e-08
1307 3.51132735780801e-08
1308 3.48190742529564e-08
1309 3.51374602909349e-08
1310 3.46148836871674e-08
1311 3.51486848297888e-08
1312 3.46518175051713e-08
1313 3.50033217002554e-08
1314 3.47069381096965e-08
1315 3.48100159222042e-08
1316 3.47630701573731e-08
1317 3.48235085854132e-08
1318 3.45164271569054e-08
1319 3.48606052611444e-08
1320 3.4643512284882e-08
1321 3.47785003333989e-08
1322 3.46833280364667e-08
1323 3.45590176504818e-08
1324 3.48977551154483e-08
1325 3.45952824707396e-08
1326 3.47436602867734e-08
1327 3.45223040008058e-08
1328 3.48767793407756e-08
1329 3.43161097184908e-08
1330 3.46684724523705e-08
1331 3.45616668697701e-08
1332 3.45378077770686e-08
1333 3.45620899371291e-08
1334 3.44023326344089e-08
1335 3.44542364902889e-08
1336 3.44973347903288e-08
1337 3.44071649855948e-08
1338 3.44875939382039e-08
1339 3.43630837280617e-08
1340 3.43337636972141e-08
1341 3.444861294577e-08
1342 3.44723666754199e-08
1343 3.42383481224395e-08
1344 3.45617301631407e-08
1345 3.41157194729824e-08
1346 3.4638309587276e-08
1347 3.39558007931284e-08
1348 3.47172532726425e-08
1349 3.388770908197e-08
1350 3.4536734013102e-08
1351 3.40253962922699e-08
1352 3.43907379298791e-08
1353 3.40630415300858e-08
1354 3.43409892540336e-08
1355 3.39571578247266e-08
1356 3.44001613632461e-08
1357 3.37563216341596e-08
1358 3.44834215859624e-08
1359 3.37359412467819e-08
1360 3.43105241760178e-08
1361 3.37929588303254e-08
1362 3.44117942410183e-08
1363 3.36776799392258e-08
1364 3.42182263191049e-08
1365 3.38278771760958e-08
1366 3.43481036400028e-08
1367 3.37360004842857e-08
1368 3.40498749291651e-08
1369 3.38249392444201e-08
1370 3.41004061708006e-08
1371 3.37479707468447e-08
1372 3.40955877342619e-08
1373 3.36662054882808e-08
1374 3.40798299616196e-08
1375 3.36679851611343e-08
1376 3.40360484794022e-08
1377 3.3536127971745e-08
1378 3.40914937793091e-08
1379 3.3456283006883e-08
1380 3.39613921847004e-08
1381 3.35075216089109e-08
1382 3.41112799884247e-08
1383 3.33338982325682e-08
1384 3.38733174285455e-08
1385 3.35016972770052e-08
1386 3.38693457502259e-08
1387 3.33648018135957e-08
1388 3.40446892057589e-08
1389 3.3327209000511e-08
1390 3.37428649888061e-08
1391 3.35405369600306e-08
1392 3.36141792340872e-08
1393 3.36067643134097e-08
1394 3.3636024307393e-08
1395 3.34493304208205e-08
1396 3.34301786228952e-08
1397 3.38063033493707e-08
1398 3.31745880581824e-08
1399 3.35089047935888e-08
1400 3.35428053022113e-08
1401 3.3241576006704e-08
1402 3.35911162352431e-08
1403 3.32421803996841e-08
1404 3.34744964909373e-08
1405 3.32588677474099e-08
1406 3.34748396362272e-08
1407 3.32123673858486e-08
1408 3.35841451528651e-08
1409 3.29762135096168e-08
1410 3.35211065964902e-08
1411 3.30881599857058e-08
1412 3.34869214122513e-08
1413 3.29546157122884e-08
1414 3.34852538863562e-08
1415 3.29146732616081e-08
1416 3.35289683048856e-08
1417 3.28356670629137e-08
1418 3.34587313388646e-08
1419 3.29433338954743e-08
1420 3.31640357924545e-08
1421 3.31994754168008e-08
1422 3.30316937151576e-08
1423 3.31766473313788e-08
1424 3.30299878972262e-08
1425 3.30808066757449e-08
1426 3.31074526467301e-08
1427 3.28682474775022e-08
1428 3.31815778860101e-08
1429 3.27551874830245e-08
1430 3.32461427974273e-08
1431 3.27960260542159e-08
1432 3.30453306394141e-08
1433 3.28542218246408e-08
1434 3.30718195626911e-08
1435 3.27807115261258e-08
1436 3.2998471771517e-08
1437 3.27710859435726e-08
1438 3.30351101953674e-08
1439 3.26944683253227e-08
1440 3.30005531192779e-08
1441 3.2714570348924e-08
1442 3.27641006916402e-08
1443 3.27163853910406e-08
1444 3.29003166095188e-08
1445 3.27233085672951e-08
1446 3.2770314609909e-08
1447 3.25788508537261e-08
1448 3.28944147844901e-08
1449 3.26895028786023e-08
1450 3.26541508290923e-08
1451 3.26296479036259e-08
1452 3.25772027625071e-08
1453 3.27451836905635e-08
1454 3.25215066527829e-08
1455 3.24982764772219e-08
1456 3.26994956194593e-08
1457 3.24936122879649e-08
1458 3.25691086446866e-08
1459 3.27141446565538e-08
1460 3.25313839180108e-08
1461 3.23293394122182e-08
1462 3.29240009591114e-08
1463 3.1972967167837e-08
1464 3.26433938804271e-08
1465 3.24267056723837e-08
1466 3.26777966526137e-08
1467 3.22411679520851e-08
1468 3.23846756811186e-08
1469 3.25577478497152e-08
1470 3.22156423748776e-08
1471 3.22870053943713e-08
1472 3.24509630238445e-08
1473 3.22651264352825e-08
1474 3.24442693868665e-08
1475 3.21661457143296e-08
1476 3.2278703145483e-08
1477 3.23713256096259e-08
1478 3.23221847264854e-08
1479 3.23056710285208e-08
1480 3.22168030530978e-08
1481 3.22653304600795e-08
1482 3.21724512291066e-08
1483 3.22539594670435e-08
1484 3.21417783171185e-08
1485 3.20743910681154e-08
1486 3.20907653370561e-08
1487 3.23753402549087e-08
1488 3.19741319569911e-08
1489 3.22770083451829e-08
1490 3.20112305540743e-08
1491 3.20919209824133e-08
1492 3.20909996689434e-08
1493 3.19539636597366e-08
1494 3.23699480422324e-08
1495 3.15685800247323e-08
1496 3.22423894858481e-08
1497 3.17018097957256e-08
1498 3.22658592073477e-08
1499 3.1738932706471e-08
1500 3.19262239785179e-08
1501 3.21666983282842e-08
1502 3.15721388584045e-08
1503 3.24982315993427e-08
1504 3.16135727678191e-08
1505 3.19452588868341e-08
1506 3.17172755694983e-08
1507 3.17515578467376e-08
1508 3.20549956496219e-08
1509 3.16167812741686e-08
1510 3.21424338893816e-08
1511 3.15956540366535e-08
1512 3.19696719378593e-08
1513 3.13614742595281e-08
1514 3.21910758160904e-08
1515 3.12462549953985e-08
1516 3.15915199196048e-08
1517 3.20538140301529e-08
1518 3.19378515369895e-08
1519 3.15426892290027e-08
1520 3.15790151710349e-08
1521 3.19844912362033e-08
1522 3.14388356643658e-08
1523 3.17918059338229e-08
1524 3.14135140188077e-08
1525 3.19427933814787e-08
1526 3.11702899664645e-08
1527 3.13312370066043e-08
1528 3.1989842606106e-08
1529 3.15396160401882e-08
1530 3.1173040289767e-08
1531 3.19085594947532e-08
1532 3.15818990073158e-08
1533 3.13086915293415e-08
1534 3.13468418267071e-08
1535 3.19530522092659e-08
1536 3.10742168903033e-08
1537 3.15904038118431e-08
1538 3.1249940243061e-08
1539 3.17331301366019e-08
1540 3.10484056740279e-08
1541 3.15092658986771e-08
1542 3.1283911686053e-08
1543 3.14378491722556e-08
1544 3.12053203845863e-08
1545 3.17367922616496e-08
1546 3.12408100825445e-08
1547 3.13381395016243e-08
1548 3.13273264389569e-08
1549 3.10843438382502e-08
1550 3.14114467188098e-08
1551 3.14006350223828e-08
1552 3.11790743557916e-08
1553 3.13302012922012e-08
1554 3.11905606196206e-08
1555 3.12007256800806e-08
1556 3.10478101676059e-08
1557 3.12550473059447e-08
1558 3.14154626855911e-08
1559 3.12248366192147e-08
1560 3.12089515224212e-08
1561 3.11156226473486e-08
1562 3.10290530285418e-08
1563 3.11326111877008e-08
1564 3.10279326108942e-08
1565 3.12924699327199e-08
1566 3.08966119083598e-08
1567 3.09142724255196e-08
1568 3.10216705675659e-08
1569 3.10812845006891e-08
1570 3.08994444206956e-08
1571 3.12072110690842e-08
1572 3.09570696744865e-08
1573 3.09896040631141e-08
1574 3.0884449235935e-08
1575 3.12280919659846e-08
1576 3.08330452959416e-08
1577 3.09225042408601e-08
1578 3.10408838414267e-08
1579 3.0810182291785e-08
1580 3.1049798059235e-08
1581 3.08061128599935e-08
1582 3.08197663948517e-08
1583 3.10480353669096e-08
1584 3.09054300119405e-08
1585 3.06586118382635e-08
1586 3.09303539028916e-08
1587 3.09501648557697e-08
1588 3.08567718448893e-08
1589 3.07631659867136e-08
1590 3.08256723388078e-08
1591 3.06958108767796e-08
1592 3.08923288295793e-08
1593 3.06311618591959e-08
1594 3.11513324735557e-08
1595 3.04206671938312e-08
1596 3.08442341636628e-08
1597 3.03653836745132e-08
1598 3.13429614414717e-08
1599 3.01638954862682e-08
1600 3.06989650952216e-08
1601 3.07196508666507e-08
1602 3.06161991402432e-08
1603 3.08913777294917e-08
1604 3.03472151889084e-08
1605 3.1062647964708e-08
1606 3.03775108402293e-08
1607 3.06894248011114e-08
1608 3.04323372879445e-08
1609 3.06642461627149e-08
1610 3.03791974817225e-08
1611 3.0891793580845e-08
1612 3.03127548350046e-08
1613 3.04127961270328e-08
1614 3.03896146698346e-08
1615 3.06968856979006e-08
1616 3.03740412288533e-08
1617 3.05450991550105e-08
1618 3.03564756705121e-08
1619 3.02401184468781e-08
1620 3.05187009714825e-08
1621 3.05969505395298e-08
1622 3.0013088962666e-08
1623 3.06500275364385e-08
1624 3.03417380521642e-08
1625 3.04860919410066e-08
1626 3.01646947979961e-08
1627 3.08061626093092e-08
1628 3.00847742840116e-08
1629 3.02719367244464e-08
1630 3.03650749561291e-08
1631 3.02133341898436e-08
1632 3.05300305581957e-08
1633 3.02266089599046e-08
1634 3.03931342012786e-08
1635 3.00190045432469e-08
1636 3.0681260837162e-08
1637 2.98879404966312e-08
1638 3.04325515726456e-08
1639 2.99571006147081e-08
1640 3.05221119868415e-08
1641 3.01404337345046e-08
1642 3.02387083871425e-08
1643 3.01415838394048e-08
1644 3.01509072025974e-08
1645 3.00901763278238e-08
1646 3.01657329035976e-08
1647 3.01989419460025e-08
1648 3.02108742061824e-08
1649 2.98662448653397e-08
1650 3.02140952667118e-08
1651 3.02677257535411e-08
1652 3.00331704887746e-08
1653 3.0035402220685e-08
1654 3.01288275775802e-08
1655 2.98623007680687e-08
1656 3.0298858539779e-08
1657 2.95936575652744e-08
1658 3.04700784578005e-08
1659 2.98807118436217e-08
1660 2.99443354765394e-08
1661 3.00142077390708e-08
1662 3.01286804621492e-08
1663 2.97925130219312e-08
1664 2.99405720667334e-08
1665 3.01816302729074e-08
1666 2.96698392631134e-08
1667 3.01430493416799e-08
1668 2.9919053380234e-08
1669 3.00356086299081e-08
1670 2.99879313873097e-08
1671 2.99052026758728e-08
1672 2.97565512951126e-08
1673 2.99389689960572e-08
1674 2.9850811568588e-08
1675 2.99463829016577e-08
1676 2.9706423457454e-08
1677 3.00419178950184e-08
1678 2.95818143011273e-08
1679 2.99735478608287e-08
1680 2.96403447517246e-08
1681 3.00199450714578e-08
1682 2.95126349507679e-08
1683 3.00880834201989e-08
1684 2.94162198037906e-08
1685 3.01517201023449e-08
1686 2.94138626025786e-08
1687 3.00440207152697e-08
1688 2.94452661956379e-08
1689 2.9844342988361e-08
1690 2.95095475534302e-08
1691 2.99506034984187e-08
1692 2.94863080226859e-08
1693 2.9744789529107e-08
1694 2.94056087365435e-08
1695 3.00157826014225e-08
1696 2.93328460815312e-08
1697 2.97165815964506e-08
1698 2.94441480355179e-08
1699 2.96894913124346e-08
1700 2.94709950412475e-08
1701 2.98680186437972e-08
1702 2.94951295540802e-08
1703 2.97990709271501e-08
1704 2.95119912792075e-08
1705 2.97458733299338e-08
1706 2.9345659686153e-08
1707 2.98277386596713e-08
1708 2.93011779672891e-08
1709 2.96157191699287e-08
1710 2.91772074640928e-08
1711 2.98928105374463e-08
1712 2.92821565728296e-08
1713 2.97651262918253e-08
1714 2.92898519608231e-08
1715 2.9716068353558e-08
1716 2.91915443610957e-08
1717 2.96187972840256e-08
1718 2.92803978042544e-08
1719 2.95298066380623e-08
1720 2.93100272636959e-08
1721 2.9598966068356e-08
1722 2.91130616922874e-08
1723 2.96449152716871e-08
1724 2.91193483162866e-08
1725 2.96551769712883e-08
1726 2.91300307412312e-08
1727 2.9581101386511e-08
1728 2.91212717298261e-08
1729 2.95984702739505e-08
1730 2.90955912931601e-08
1731 2.95055903016817e-08
1732 2.91013434327025e-08
1733 2.95362727116277e-08
1734 2.90351727733063e-08
1735 2.94662024754588e-08
1736 2.90570307166504e-08
1737 2.95331179864799e-08
1738 2.90139223372288e-08
1739 2.9520691711471e-08
1740 2.90207045349211e-08
1741 2.94589723058847e-08
1742 2.89933667459996e-08
1743 2.94836793951481e-08
1744 2.89781491518148e-08
1745 2.94477086669742e-08
1746 2.89499748004074e-08
1747 2.94491256924712e-08
1748 2.88984847951479e-08
1749 2.94611276812073e-08
1750 2.89095870885658e-08
1751 2.93348227019541e-08
1752 2.89555775953021e-08
1753 2.93437094910365e-08
1754 2.8908022490115e-08
1755 2.9352781000358e-08
1756 2.88687312918201e-08
1757 2.93545815844842e-08
1758 2.87841623641816e-08
1759 2.93847623384913e-08
1760 2.87214249584666e-08
1761 2.94661273322339e-08
1762 2.87386584665317e-08
1763 2.92699354996762e-08
1764 2.89910768999091e-08
1765 2.91418711215741e-08
1766 2.89646411755218e-08
1767 2.89577202950975e-08
1768 2.90285390830425e-08
1769 2.89461643036004e-08
1770 2.90425602554878e-08
1771 2.89482423089238e-08
1772 2.90144145922433e-08
1773 2.89514270541691e-08
1774 2.89254297364261e-08
1775 2.89098596077952e-08
1776 2.9001373429538e-08
1777 2.8860782484208e-08
1778 2.89481048878493e-08
1779 2.89100678317888e-08
1780 2.89109472720872e-08
1781 2.88706676115913e-08
1782 2.88940602720666e-08
1783 2.88739283945683e-08
1784 2.88817522211549e-08
1785 2.8814721495185e-08
1786 2.88903650553785e-08
1787 2.87699370731254e-08
1788 2.89397249707557e-08
1789 2.87723938907636e-08
1790 2.8882882537884e-08
1791 2.87354100699799e-08
1792 2.88576438082222e-08
1793 2.87447967127763e-08
1794 2.88553758586163e-08
1795 2.87169490470474e-08
1796 2.8821428203929e-08
1797 2.8718048178944e-08
1798 2.88302827794462e-08
1799 2.86761846478578e-08
1800 2.88018630041353e-08
1801 2.86809312777248e-08
1802 2.87835999358554e-08
1803 2.86609404341887e-08
1804 2.88004398659725e-08
1805 2.86057755661817e-08
1806 2.87639194842537e-08
1807 2.86621147858046e-08
1808 2.88229583376154e-08
1809 2.86427126593436e-08
1810 2.87038296148401e-08
1811 2.86750976434824e-08
1812 2.86897302604894e-08
1813 2.87335725278615e-08
1814 2.86596542800055e-08
1815 2.86370677495817e-08
1816 2.86670735343275e-08
1817 2.86675605060127e-08
1818 2.86174054149324e-08
1819 2.86354259835253e-08
1820 2.86247094161141e-08
1821 2.86303821362965e-08
1822 2.86132956581753e-08
1823 2.85523704036672e-08
1824 2.86544601961758e-08
1825 2.85209519835794e-08
1826 2.86083222796885e-08
1827 2.85296959204873e-08
1828 2.85941852237182e-08
1829 2.85113115247038e-08
1830 2.85450484179828e-08
1831 2.85502858307085e-08
1832 2.85154847757818e-08
1833 2.85581509252841e-08
1834 2.84599589032153e-08
1835 2.85444441940896e-08
1836 2.84778920822681e-08
1837 2.84889947964606e-08
1838 2.84825507329556e-08
1839 2.85074379221539e-08
1840 2.84070530515157e-08
1841 2.85104067466735e-08
1842 2.84286074795359e-08
1843 2.84686074896179e-08
1844 2.83903177904499e-08
1845 2.84567667836466e-08
1846 2.84159504959636e-08
1847 2.84276843726117e-08
1848 2.83820725681672e-08
1849 2.83960707404551e-08
1850 2.84351361554691e-08
1851 2.83582247120329e-08
1852 2.83307276677247e-08
1853 2.83470490697324e-08
1854 2.84503315898244e-08
1855 2.81413010582154e-08
1856 2.84089205692117e-08
1857 2.81731214040182e-08
1858 2.84147041631488e-08
1859 2.81951738266839e-08
1860 2.85769780116185e-08
1861 2.80768172399304e-08
1862 2.83120657895708e-08
1863 2.81799143549977e-08
1864 2.81546620758411e-08
1865 2.84330994950732e-08
1866 2.79930161705133e-08
1867 2.84958881245911e-08
1868 2.80699467045675e-08
1869 2.85044811980661e-08
1870 2.79219912039386e-08
1871 2.82913127387374e-08
1872 2.80657009776863e-08
1873 2.84742713815289e-08
1874 2.76952472236802e-08
1875 2.85615340666068e-08
1876 2.79888649682425e-08
1877 2.82644637813467e-08
1878 2.81692288105484e-08
1879 2.80200388307339e-08
1880 2.84548992192102e-08
1881 2.78268005671212e-08
1882 2.82369091505474e-08
1883 2.80259703283603e-08
1884 2.81742421784914e-08
1885 2.80722549146351e-08
1886 2.8168204479595e-08
1887 2.80357992243907e-08
1888 2.82220051129878e-08
1889 2.78386170422529e-08
1890 2.82381056783132e-08
1891 2.79726643376277e-08
1892 2.81530515066075e-08
1893 2.80172824035363e-08
1894 2.81647439922539e-08
1895 2.78695168596244e-08
1896 2.81577736882532e-08
1897 2.79167145561532e-08
1898 2.81193654034784e-08
1899 2.79555512184837e-08
1900 2.81836842952465e-08
1901 2.79115874486369e-08
1902 2.80382950762492e-08
1903 2.77845530376464e-08
1904 2.81058861453243e-08
1905 2.78434301965724e-08
1906 2.80952526306999e-08
1907 2.78772484703094e-08
1908 2.7910838624523e-08
1909 2.79140897856411e-08
1910 2.80507706441613e-08
1911 2.78149746077982e-08
1912 2.79746955272886e-08
1913 2.78258460854053e-08
1914 2.79509257455901e-08
1915 2.80061262886155e-08
1916 2.76616533819718e-08
1917 2.80159410314118e-08
1918 2.78762572926139e-08
1919 2.80654415376613e-08
1920 2.77664917701648e-08
1921 2.78388766892235e-08
1922 2.77649374592626e-08
1923 2.79406277783334e-08
1924 2.77138576167957e-08
1925 2.78174022139144e-08
1926 2.78441692282971e-08
1927 2.7884032670511e-08
1928 2.76780005569188e-08
1929 2.79420653087614e-08
1930 2.76206666035561e-08
1931 2.78377990651224e-08
1932 2.7747019563984e-08
1933 2.77605011108628e-08
1934 2.77782785972347e-08
1935 2.77782524303882e-08
1936 2.77386213448638e-08
1937 2.78679697833795e-08
1938 2.75968462246645e-08
1939 2.7802394510501e-08
1940 2.76139188162272e-08
1941 2.78232114501176e-08
1942 2.76407597413764e-08
1943 2.76555932076095e-08
1944 2.76894087380963e-08
1945 2.76383396146107e-08
1946 2.76525843970754e-08
1947 2.7547312206333e-08
1948 2.78561835264091e-08
1949 2.72184666676445e-08
1950 2.79837305676711e-08
1951 2.72764077393628e-08
1952 2.77502072469904e-08
1953 2.74658433496722e-08
1954 2.76350619990984e-08
1955 2.74834926292655e-08
1956 2.77703428405163e-08
1957 2.73822490862718e-08
1958 2.76648682230318e-08
1959 2.75285422129956e-08
1960 2.7517474981309e-08
1961 2.77807000030883e-08
1962 2.72865192216809e-08
1963 2.76738044887592e-08
1964 2.72302881590969e-08
1965 2.77631589454819e-08
1966 2.71979854622462e-08
1967 2.75152699452663e-08
1968 2.74540191560479e-08
1969 2.7354832343951e-08
1970 2.75428862149818e-08
1971 2.72600241590926e-08
1972 2.7694294288616e-08
1973 2.734953858885e-08
1974 2.7585379392514e-08
1975 2.72386201766173e-08
1976 2.75666551727838e-08
1977 2.72274999476574e-08
1978 2.76446575052658e-08
1979 2.73583816555734e-08
1980 2.74170862843981e-08
1981 2.74149640161481e-08
1982 2.74032152486337e-08
1983 2.71949602653754e-08
1984 2.74454899827914e-08
1985 2.70868547985748e-08
1986 2.74771446181488e-08
1987 2.72332537240017e-08
1988 2.75098129394191e-08
1989 2.71462741859851e-08
1990 2.74062934801922e-08
1991 2.72075919425241e-08
1992 2.74464382092887e-08
1993 2.71345239887255e-08
1994 2.75193870160617e-08
1995 2.71962295241845e-08
1996 2.71853222542084e-08
1997 2.74906403482378e-08
1998 2.71997422995884e-08
1999 2.71554924926587e-08
2000 2.72923052737628e-08
2001 2.71555541888624e-08
2002 2.7343076539732e-08
2003 2.71850413809949e-08
2004 2.72206369966721e-08
2005 2.71979019369484e-08
2006 2.71689422689514e-08
2007 2.72840457817836e-08
2008 2.7139288049316e-08
2009 2.71783580312324e-08
2010 2.7075743162186e-08
2011 2.73371874149886e-08
2012 2.69872172143915e-08
2013 2.7352317583218e-08
2014 2.72661547076058e-08
2015 2.68987688837319e-08
2016 2.72170148187811e-08
2017 2.70337544907573e-08
2018 2.71608146023006e-08
2019 2.69727508536155e-08
2020 2.72305180839538e-08
2021 2.69993883562636e-08
2022 2.71720162174915e-08
2023 2.72971646346765e-08
2024 2.69632801285224e-08
2025 2.71142925770684e-08
2026 2.70785174282251e-08
2027 2.69822333335279e-08
2028 2.7043413109662e-08
2029 2.71523868154988e-08
2030 2.71187230017622e-08
2031 2.70622780208507e-08
2032 2.71771684617894e-08
2033 2.6946226967195e-08
2034 2.71791589436798e-08
2035 2.69035627923353e-08
2036 2.71516469811939e-08
2037 2.67654523954386e-08
2038 2.71098815207704e-08
2039 2.6787939154671e-08
2040 2.72307421279594e-08
2041 2.67190858441735e-08
2042 2.70797439707593e-08
2043 2.67798855211332e-08
2044 2.70123376838738e-08
2045 2.68136480624515e-08
2046 2.71423107975899e-08
2047 2.66453774991904e-08
2048 2.70992561179906e-08
2049 2.69146044150625e-08
2050 2.67918995507932e-08
2051 2.69327816432297e-08
2052 2.71112587816313e-08
2053 2.6702583929672e-08
2054 2.71145429283637e-08
2055 2.68141127603982e-08
2056 2.67349629380353e-08
2057 2.69777863329823e-08
2058 2.67242162910186e-08
2059 2.692758739542e-08
2060 2.67243413649698e-08
2061 2.68493920805302e-08
2062 2.67379838256865e-08
2063 2.68868082161644e-08
2064 2.6725898438662e-08
2065 2.67225531319637e-08
2066 2.68363122922777e-08
2067 2.6809610595091e-08
2068 2.69126677207021e-08
2069 2.67277010300715e-08
2070 2.68672714552887e-08
2071 2.68130839158331e-08
2072 2.68664986528e-08
2073 2.66703793802003e-08
2074 2.67429273140829e-08
2075 2.66333447996869e-08
2076 2.67048125058844e-08
2077 2.66422005992251e-08
2078 2.66952690063382e-08
2079 2.65630149708373e-08
2080 2.67208094458793e-08
2081 2.66889873087317e-08
2082 2.66928141376965e-08
2083 2.67057458025466e-08
2084 2.6718794915892e-08
2085 2.66732199498687e-08
2086 2.67065685340961e-08
2087 2.64809997155835e-08
2088 2.66985010315413e-08
2089 2.65920959775512e-08
2090 2.6571828728672e-08
2091 2.67498305476233e-08
2092 2.65671661345834e-08
2093 2.65647860002716e-08
2094 2.6656942404002e-08
2095 2.65537856192211e-08
2096 2.66882635336918e-08
2097 2.6428821756963e-08
2098 2.66520526736036e-08
2099 2.64571606768405e-08
2100 2.66970545806933e-08
2101 2.62347642864258e-08
2102 2.67067309445146e-08
2103 2.63921191305805e-08
2104 2.66209217558711e-08
2105 2.65951154464483e-08
2106 2.64927495380318e-08
2107 2.63933687952855e-08
2108 2.65293050782933e-08
2109 2.6352248407302e-08
2110 2.68245916945631e-08
2111 2.6182411557496e-08
2112 2.64368590193742e-08
2113 2.63451010277249e-08
2114 2.6565556750513e-08
2115 2.62806814089744e-08
2116 2.65711037475969e-08
2117 2.6300495878373e-08
2118 2.63075733899498e-08
2119 2.6302205163975e-08
2120 2.65076500494121e-08
2121 2.633989499079e-08
2122 2.62616623377676e-08
2123 2.64197311771186e-08
2124 2.62514149689252e-08
2125 2.63232450160578e-08
2126 2.63081894881223e-08
2127 2.62243476435442e-08
2128 2.62006930389358e-08
2129 2.63381702988452e-08
2130 2.63199672061454e-08
2131 2.63315378989937e-08
2132 2.60796780520156e-08
2133 2.64908317727519e-08
2134 2.61578654882078e-08
2135 2.62751397424044e-08
2136 2.61854049431465e-08
2137 2.64216774769066e-08
2138 2.59666202407915e-08
2139 2.60873849311505e-08
2140 2.6328025697242e-08
2141 2.60230816101492e-08
2142 2.60618617807795e-08
2143 2.6097852648399e-08
2144 2.61612148639845e-08
2145 2.6103156947066e-08
2146 2.62439954810123e-08
2147 2.61186197446639e-08
2148 2.64545096806401e-08
2149 2.58800342617382e-08
2150 2.64592208678582e-08
2151 2.58603347245145e-08
2152 2.63430576274981e-08
2153 2.60243566869711e-08
2154 2.60518093995188e-08
2155 2.64200582832386e-08
2156 2.58402760590082e-08
2157 2.59886543857446e-08
2158 2.60621049472665e-08
2159 2.60735640865706e-08
2160 2.5917039607859e-08
2161 2.61703958879256e-08
2162 2.59323871186767e-08
2163 2.63132560619495e-08
2164 2.5786133114325e-08
2165 2.59146141762256e-08
2166 2.58569229268923e-08
2167 2.61878505731783e-08
2168 2.58723742799782e-08
2169 2.57577664402042e-08
2170 2.61584087332123e-08
2171 2.57379296700888e-08
2172 2.60544759925763e-08
2173 2.58436806500617e-08
2174 2.57097383233873e-08
2175 2.58273104826179e-08
2176 2.58696963674199e-08
2177 2.59990834103885e-08
2178 2.58629285561085e-08
2179 2.58603836945648e-08
2180 2.59424691780863e-08
2181 2.57751499059822e-08
2182 2.58272775252033e-08
2183 2.55596184821627e-08
2184 2.57837338517586e-08
2185 2.56441579505884e-08
2186 2.59210781171637e-08
2187 2.56581230277941e-08
2188 2.57817115205672e-08
2189 2.61190129825506e-08
2190 2.55343306577727e-08
2191 2.56723093604583e-08
2192 2.58199670375436e-08
2193 2.55652805823958e-08
2194 2.6076081243942e-08
2195 2.54054181675345e-08
2196 2.57912519314729e-08
2197 2.58276305409311e-08
2198 2.57211982368499e-08
2199 2.54948145020917e-08
2200 2.58205132490641e-08
2201 2.56633640591808e-08
2202 2.56207926825347e-08
2203 2.57046057027255e-08
2204 2.54851829578628e-08
2205 2.56114566709043e-08
2206 2.54448907760496e-08
2207 2.55865941957145e-08
2208 2.57824202346546e-08
2209 2.55121874339626e-08
2210 2.58300771676101e-08
2211 2.54250005627954e-08
2212 2.54473702286928e-08
2213 2.55239902247073e-08
2214 2.5379639417511e-08
2215 2.53389686404315e-08
2216 2.58537233591705e-08
2217 2.52310298446412e-08
2218 2.55165972693483e-08
2219 2.55604542970245e-08
2220 2.52295341311992e-08
2221 2.54790317583131e-08
2222 2.52524929625908e-08
2223 2.54554390101136e-08
2224 2.52456949836333e-08
2225 2.53937330390031e-08
2226 2.53626129673723e-08
2227 2.53909380605322e-08
2228 2.51599985888751e-08
2229 2.53210951024485e-08
2230 2.51345728738972e-08
2231 2.53944233246095e-08
2232 2.51276161793434e-08
2233 2.53710103598204e-08
2234 2.53884008933891e-08
2235 2.51119859445703e-08
2236 2.51305332568119e-08
2237 2.52838136940436e-08
2238 2.51641290076599e-08
2239 2.51359226521863e-08
2240 2.53013244537748e-08
2241 2.5140146976077e-08
2242 2.52062881965776e-08
2243 2.52929482572606e-08
2244 2.51076507937764e-08
2245 2.51377086517568e-08
2246 2.49997952729286e-08
2247 2.51986456786257e-08
2248 2.52507241071953e-08
2249 2.49492135795393e-08
2250 2.51681206331344e-08
2251 2.50605483279598e-08
2252 2.51425410285888e-08
2253 2.51992023317937e-08
2254 2.48284821632305e-08
2255 2.52231482674725e-08
2256 2.50579429773579e-08
2257 2.5014801254275e-08
2258 2.4977879482968e-08
2259 2.50947401004842e-08
2260 2.50822766630909e-08
2261 2.49791428434953e-08
2262 2.50204303517965e-08
2263 2.48978438769365e-08
2264 2.52372374068832e-08
2265 2.48582671122932e-08
2266 2.51037968642676e-08
2267 2.50295701057501e-08
2268 2.4668116463733e-08
2269 2.50682289163473e-08
2270 2.47820097286633e-08
2271 2.50887144577217e-08
2272 2.47181512362804e-08
2273 2.48401479031601e-08
2274 2.4957795192404e-08
2275 2.4843416983833e-08
2276 2.49242987290232e-08
2277 2.49573915670487e-08
2278 2.50188375764404e-08
2279 2.47877976383704e-08
2280 2.47698591319123e-08
2281 2.50045192335957e-08
2282 2.46541484897334e-08
2283 2.49943239200023e-08
2284 2.48154356893293e-08
2285 2.45682213945164e-08
2286 2.46782551459601e-08
2287 2.4707798004453e-08
2288 2.46860907356927e-08
2289 2.45217365186789e-08
2290 2.47804706533516e-08
2291 2.45768751225839e-08
2292 2.48096288630872e-08
2293 2.45415590113263e-08
2294 2.45389787362216e-08
2295 2.47591085512644e-08
2296 2.46786250240882e-08
2297 2.45683603034008e-08
2298 2.47423050130369e-08
2299 2.4583247976051e-08
2300 2.45916111313305e-08
2301 2.44802352294959e-08
2302 2.46799525768226e-08
2303 2.45890379370062e-08
2304 2.43639086604519e-08
2305 2.45927366436804e-08
2306 2.44923043756229e-08
2307 2.45723117580088e-08
2308 2.4539214154129e-08
2309 2.46436878724721e-08
2310 2.43243865593046e-08
2311 2.44764552224108e-08
2312 2.45114275608804e-08
2313 2.44807345927134e-08
2314 2.47053745088266e-08
2315 2.40522010285238e-08
2316 2.46058991004716e-08
2317 2.44938823773522e-08
2318 2.46791576951066e-08
2319 2.41293650147245e-08
2320 2.45140281019962e-08
2321 2.44276690982703e-08
2322 2.47232737268233e-08
2323 2.42012104908484e-08
2324 2.41585249018961e-08
2325 2.43462986700971e-08
2326 2.46723207447674e-08
2327 2.43545143668866e-08
2328 2.44060038973037e-08
2329 2.4227754334194e-08
2330 2.45686870880002e-08
2331 2.44584682352889e-08
2332 2.40852233889255e-08
2333 2.42108266848007e-08
2334 2.44560455128218e-08
2335 2.43064516279867e-08
2336 2.40569127543111e-08
2337 2.43913462201206e-08
2338 2.4548581276318e-08
2339 2.40078893289919e-08
2340 2.42718183303126e-08
2341 2.42403807109159e-08
2342 2.42192768017402e-08
2343 2.44463921496019e-08
2344 2.4371403367196e-08
2345 2.41570107065714e-08
2346 2.41420894855349e-08
2347 2.41371265018442e-08
2348 2.42406693132802e-08
2349 2.40701884776096e-08
2350 2.44447950601412e-08
2351 2.40263063678858e-08
2352 2.43574176440609e-08
2353 2.39297841448938e-08
2354 2.41682305136548e-08
2355 2.42324692864004e-08
2356 2.38721546609311e-08
2357 2.41053315984674e-08
2358 2.42745604874806e-08
2359 2.37143837976683e-08
2360 2.42749455411362e-08
2361 2.40323371335505e-08
2362 2.40636270391059e-08
2363 2.40007856794344e-08
2364 2.39410083752167e-08
2365 2.40832710322048e-08
2366 2.38361972570988e-08
2367 2.41940059403367e-08
2368 2.43675453980297e-08
2369 2.39279959614036e-08
2370 2.3788605151065e-08
2371 2.42909428905813e-08
2372 2.40194182016262e-08
2373 2.38929172280411e-08
2374 2.36186886304335e-08
2375 2.42169606777809e-08
2376 2.37381581391638e-08
2377 2.38381824894107e-08
2378 2.4120291982288e-08
2379 2.36840647871439e-08
2380 2.39832997376288e-08
2381 2.37950063459724e-08
2382 2.39057641252538e-08
2383 2.39910060667592e-08
2384 2.40105806440516e-08
2385 2.37462564928181e-08
2386 2.38948061467292e-08
2387 2.39259193692831e-08
2388 2.38449897284054e-08
2389 2.39228535718894e-08
2390 2.38345635298343e-08
2391 2.35170807632024e-08
2392 2.40149993474548e-08
2393 2.38945121403544e-08
2394 2.34389194467832e-08
2395 2.39535861661366e-08
2396 2.359979196076e-08
2397 2.38728033393754e-08
2398 2.37392189030849e-08
2399 2.35290826458012e-08
2400 2.38531351144378e-08
2401 2.36289816203383e-08
2402 2.3745056767277e-08
2403 2.38226408725284e-08
2404 2.38547356188334e-08
2405 2.35122026112089e-08
2406 2.35995093265107e-08
2407 2.37519401136144e-08
2408 2.36855606734476e-08
2409 2.37508018585686e-08
2410 2.36903421401147e-08
2411 2.36661098035329e-08
2412 2.3493669066732e-08
2413 2.35398501199358e-08
2414 2.3682543910164e-08
2415 2.37608082815921e-08
2416 2.35254008217334e-08
2417 2.36843839075407e-08
2418 2.33259677799857e-08
2419 2.36780022121463e-08
2420 2.33339458618476e-08
2421 2.37102504970776e-08
2422 2.33165582852468e-08
2423 2.38323498148452e-08
2424 2.34895988692196e-08
2425 2.34924433546446e-08
2426 2.35635000811474e-08
2427 2.35051996922975e-08
2428 2.38028609425012e-08
2429 2.32266944169002e-08
2430 2.34522336824972e-08
2431 2.34091407097514e-08
2432 2.34807048780095e-08
2433 2.3529237695441e-08
2434 2.32522928578227e-08
2435 2.35786149114237e-08
2436 2.33846638356683e-08
2437 2.32811601458272e-08
2438 2.34430761484328e-08
2439 2.34584473448596e-08
2440 2.33446093450551e-08
2441 2.36639160959839e-08
2442 2.33878339980231e-08
2443 2.34242001768692e-08
2444 2.33277858342484e-08
2445 2.3304189462392e-08
2446 2.34521863119452e-08
2447 2.38144130533247e-08
2448 2.33093247530292e-08
2449 2.31712666899364e-08
2450 2.33873245081284e-08
2451 2.31583864150897e-08
2452 2.32474348142997e-08
2453 2.32795875144687e-08
2454 2.32090008022157e-08
2455 2.3526961829945e-08
2456 2.31688575711431e-08
2457 2.31532880373697e-08
2458 2.32758543863287e-08
2459 2.3464947323748e-08
2460 2.32588246630305e-08
2461 2.33327336570666e-08
2462 2.31436582026623e-08
2463 2.34774955386596e-08
2464 2.30998603958366e-08
2465 2.33332782005879e-08
2466 2.30082625544137e-08
2467 2.33853888215174e-08
2468 2.3062126546014e-08
2469 2.31375217457641e-08
2470 2.3505887708386e-08
2471 2.32261680744861e-08
2472 2.27963600714087e-08
2473 2.32674731185378e-08
2474 2.31487733831637e-08
2475 2.34039778409523e-08
2476 2.31448667394885e-08
2477 2.31551294490595e-08
2478 2.33541727887587e-08
2479 2.27975817616022e-08
2480 2.33066412086425e-08
2481 2.29356173566675e-08
2482 2.30228765200069e-08
2483 2.29835865445116e-08
2484 2.33212194921117e-08
2485 2.29533833339124e-08
2486 2.3076094337271e-08
2487 2.30580777467182e-08
2488 2.31683604829946e-08
2489 2.2988867478535e-08
2490 2.32255725927111e-08
2491 2.33504276786478e-08
2492 2.27773934584619e-08
2493 2.2878883023103e-08
2494 2.31772807191e-08
2495 2.29489971993457e-08
2496 2.3089784548791e-08
2497 2.30385220116869e-08
2498 2.30002738180746e-08
2499 2.2706618851509e-08
2500 2.32201756179551e-08
2501 2.3162415329514e-08
2502 2.27748865998523e-08
2503 2.30041495173916e-08
2504 2.29320395110921e-08
2505 2.3070865948438e-08
2506 2.27397749540392e-08
2507 2.29157584870077e-08
2508 2.3073289437292e-08
2509 2.28828151225269e-08
2510 2.29086525810329e-08
2511 2.29544854982722e-08
2512 2.27752389555524e-08
2513 2.29521924509646e-08
2514 2.29643577804861e-08
2515 2.2800276307966e-08
2516 2.30483034151519e-08
2517 2.26487635018868e-08
2518 2.30186879446315e-08
2519 2.27750440500163e-08
2520 2.30139096950577e-08
2521 2.27551694278461e-08
2522 2.30770813468562e-08
2523 2.24041322015589e-08
2524 2.31188931726356e-08
2525 2.26910586941242e-08
2526 2.28475513981419e-08
2527 2.293577825907e-08
2528 2.27610523743094e-08
2529 2.26154561810965e-08
2530 2.29685799196799e-08
2531 2.27212359890094e-08
2532 2.26816864317136e-08
2533 2.30327456678392e-08
2534 2.25036678732637e-08
2535 2.28880435254597e-08
2536 2.26454233206441e-08
2537 2.27776264304369e-08
2538 2.29894680287002e-08
2539 2.246656125382e-08
2540 2.27994402726139e-08
2541 2.27569793115334e-08
2542 2.26459948361457e-08
2543 2.27494156809227e-08
2544 2.27728369321945e-08
2545 2.2658548117227e-08
2546 2.25907205527731e-08
2547 2.27374034016403e-08
2548 2.23851454975721e-08
2549 2.29495031578431e-08
2550 2.27441718397836e-08
2551 2.24439858776959e-08
2552 2.2537874578954e-08
2553 2.27507121272064e-08
2554 2.2786319687107e-08
2555 2.24557449085561e-08
2556 2.27389567298841e-08
2557 2.22866040755498e-08
2558 2.28099481628385e-08
2559 2.23122356458605e-08
2560 2.26750570657686e-08
2561 2.25931033115412e-08
2562 2.28818101855044e-08
2563 2.23617512192797e-08
2564 2.2539183636483e-08
2565 2.24624037591381e-08
2566 2.26724263895361e-08
2567 2.26830952867463e-08
2568 2.2367354813535e-08
2569 2.28820275084418e-08
2570 2.22732842882767e-08
2571 2.24562224565572e-08
2572 2.25507881693732e-08
2573 2.25396383760623e-08
2574 2.23846658411464e-08
2575 2.26247511961875e-08
2576 2.24645894584263e-08
2577 2.21692719942768e-08
2578 2.26578035196212e-08
2579 2.26797868360107e-08
2580 2.21111498411863e-08
2581 2.25911442851556e-08
2582 2.2456911021429e-08
2583 2.22323711990269e-08
2584 2.24921032261349e-08
2585 2.24108896503816e-08
2586 2.23932708370844e-08
2587 2.24723413970729e-08
2588 2.25325376713714e-08
2589 2.23382307730224e-08
2590 2.23223578870169e-08
2591 2.22299902193912e-08
2592 2.2576273312036e-08
2593 2.25325398638399e-08
2594 2.21906562741792e-08
2595 2.23592021721952e-08
2596 2.22983985632697e-08
2597 2.24844538259106e-08
2598 2.25500004339407e-08
2599 2.20654595094461e-08
2600 2.23972977086984e-08
2601 2.24332999005927e-08
2602 2.21962551758548e-08
2603 2.22244724109588e-08
2604 2.22814627445223e-08
2605 2.23258226063061e-08
2606 2.22840070510655e-08
2607 2.22172993425973e-08
2608 2.23905447046535e-08
2609 2.21110083780118e-08
2610 2.24762553314717e-08
2611 2.23341108434028e-08
2612 2.20608020631285e-08
2613 2.23824916630866e-08
2614 2.24266811260643e-08
2615 2.19931340975643e-08
2616 2.22873426750647e-08
2617 2.20972637569039e-08
2618 2.21680443894812e-08
2619 2.23499957857642e-08
2620 2.2103976390575e-08
2621 2.2492441831834e-08
2622 2.18630157921007e-08
2623 2.25030927956116e-08
2624 2.22468022773592e-08
2625 2.19959961762495e-08
2626 2.21206892335735e-08
2627 2.21579391068705e-08
2628 2.24221424390825e-08
2629 2.1907350631678e-08
2630 2.21296933515802e-08
2631 2.21707624118439e-08
2632 2.20141900325599e-08
2633 2.24023495765113e-08
2634 2.18760772956461e-08
2635 2.21651778881116e-08
2636 2.21950327914389e-08
2637 2.20721000344648e-08
2638 2.18939867733248e-08
2639 2.22790405918216e-08
2640 2.19447996443467e-08
2641 2.20885836459406e-08
2642 2.19997766108815e-08
2643 2.2141349599103e-08
2644 2.18632634163551e-08
2645 2.22002051489811e-08
2646 2.20509768171784e-08
2647 2.21332566158194e-08
2648 2.18901986721587e-08
2649 2.21858192034352e-08
2650 2.18747669256114e-08
2651 2.21586078094083e-08
2652 2.21097660504377e-08
2653 2.18618457054065e-08
2654 2.225594516303e-08
2655 2.18418156766598e-08
2656 2.20563793790207e-08
2657 2.19408875249405e-08
2658 2.19898242861394e-08
2659 2.22171856857356e-08
2660 2.17393562003121e-08
2661 2.20485527583358e-08
2662 2.19422496924304e-08
2663 2.19848556756164e-08
2664 2.18468799164651e-08
2665 2.20149985724616e-08
2666 2.18344332503229e-08
2667 2.21509638441697e-08
2668 2.16297978913316e-08
2669 2.20466962435051e-08
2670 2.18094102701905e-08
2671 2.19818158427865e-08
2672 2.18304470384068e-08
2673 2.20692965436919e-08
2674 2.16766322359563e-08
2675 2.20062077228578e-08
2676 2.17228471909303e-08
2677 2.20232983311197e-08
2678 2.17030870368617e-08
2679 2.20016403510437e-08
2680 2.17066833529955e-08
2681 2.19968900664336e-08
2682 2.16808640027821e-08
2683 2.19620595293124e-08
2684 2.16822807863615e-08
2685 2.18542008161293e-08
2686 2.17116153179431e-08
2687 2.19198764738193e-08
2688 2.17316397267409e-08
2689 2.19141442104887e-08
2690 2.15770575107133e-08
2691 2.19626491162472e-08
2692 2.15928859697012e-08
2693 2.19363880610279e-08
2694 2.16023183399194e-08
2695 2.18682453935193e-08
2696 2.16866266594673e-08
2697 2.1876382333641e-08
2698 2.16173817838561e-08
2699 2.18869737044214e-08
2700 2.16191818794842e-08
2701 2.17298630771356e-08
2702 2.1958151346424e-08
2703 2.14272440950403e-08
2704 2.18952960771013e-08
2705 2.1432953788536e-08
2706 2.19852914962182e-08
2707 2.16157584503884e-08
2708 2.15875839870705e-08
2709 2.1709726738206e-08
2710 2.1804746357712e-08
2711 2.14527484011517e-08
2712 2.2087223859435e-08
2713 2.13234500126003e-08
2714 2.16877154151085e-08
2715 2.18082084947335e-08
2716 2.14919671278624e-08
2717 2.14734270217454e-08
2718 2.20126428912559e-08
2719 2.12960168978205e-08
2720 2.16179672304362e-08
2721 2.18050500213618e-08
2722 2.13587680114591e-08
2723 2.16793527960668e-08
2724 2.1676737933185e-08
2725 2.13991680180214e-08
2726 2.19782402607338e-08
2727 2.12714230316546e-08
2728 2.16628342857295e-08
2729 2.15602670589066e-08
2730 2.14891404483053e-08
2731 2.1629242161314e-08
2732 2.13946915827989e-08
2733 2.14037703600844e-08
2734 2.20050994569343e-08
2735 2.11519441046804e-08
2736 2.1382382196955e-08
2737 2.18740581242605e-08
2738 2.15387623804242e-08
2739 2.13812137597191e-08
2740 2.16546166065257e-08
2741 2.14083871490356e-08
2742 2.17942334668209e-08
2743 2.1391634416279e-08
2744 2.15134824446128e-08
2745 2.15823330017573e-08
2746 2.14951955863629e-08
2747 2.15258806600671e-08
2748 2.15886304596458e-08
2749 2.1240161539593e-08
2750 2.17736187139161e-08
2751 2.11157010198137e-08
2752 2.15483606685885e-08
2753 2.13739693412096e-08
2754 2.15969424579576e-08
2755 2.12468669701371e-08
2756 2.17024353511563e-08
2757 2.12280348171889e-08
2758 2.12913424734174e-08
2759 2.13356734654058e-08
2760 2.161989248195e-08
2761 2.13659280253742e-08
2762 2.11037030785066e-08
2763 2.18397089056799e-08
2764 2.09500760358639e-08
2765 2.13407769272811e-08
2766 2.15347683915068e-08
2767 2.11553098320438e-08
2768 2.12680674428478e-08
2769 2.17624241564129e-08
2770 2.10346808997564e-08
2771 2.13280684925321e-08
2772 2.12182242312897e-08
2773 2.14931863009582e-08
2774 2.15045220112176e-08
2775 2.12826817933554e-08
2776 2.16088722156949e-08
2777 2.11512194033814e-08
2778 2.12283115851308e-08
2779 2.13014376975185e-08
2780 2.14406397676115e-08
2781 2.11777798546642e-08
2782 2.12958067264957e-08
2783 2.12149546785501e-08
2784 2.13248050422443e-08
2785 2.11585568510309e-08
2786 2.13714884101934e-08
2787 2.13474611182596e-08
2788 2.11747684017993e-08
2789 2.10898111423496e-08
2790 2.16097412829575e-08
2791 2.09964261762563e-08
2792 2.12016913765378e-08
2793 2.08443625925314e-08
2794 2.1641241928827e-08
2795 2.10172715220391e-08
2796 2.11655648579834e-08
2797 2.13685781821349e-08
2798 2.11619510855909e-08
2799 2.12240023944599e-08
2800 2.12801892450765e-08
2801 2.10240411846474e-08
2802 2.10390893263801e-08
2803 2.14507873302905e-08
2804 2.11936615145403e-08
2805 2.13796090321461e-08
2806 2.10422585140702e-08
2807 2.13654983204359e-08
2808 2.0865069527698e-08
2809 2.0996876238133e-08
2810 2.13852986934437e-08
2811 2.09527553227673e-08
2812 2.12596352773264e-08
2813 2.09909139753384e-08
2814 2.10203283406152e-08
2815 2.09922711755794e-08
2816 2.13425501678355e-08
2817 2.0947549218886e-08
2818 2.10737821169094e-08
2819 2.12248333183451e-08
2820 2.0845617486942e-08
2821 2.10284565064223e-08
2822 2.14101778284181e-08
2823 2.1060570471243e-08
2824 2.07421293185739e-08
2825 2.11829489723536e-08
2826 2.09955282700625e-08
2827 2.08202884676156e-08
2828 2.09955306229581e-08
2829 2.11286502670438e-08
2830 2.0809312946346e-08
2831 2.09842821798922e-08
2832 2.13250442380186e-08
2833 2.08590475810899e-08
2834 2.10794755709509e-08
2835 2.09432801732135e-08
2836 2.13080230091478e-08
2837 2.07432623127168e-08
2838 2.07340783161536e-08
2839 2.13081423973138e-08
2840 2.11650496760862e-08
2841 2.06468779158087e-08
2842 2.08794420765157e-08
2843 2.11697167434854e-08
2844 2.08915443494773e-08
2845 2.08555595295756e-08
2846 2.123129251419e-08
2847 2.09581941683634e-08
2848 2.09571038960465e-08
2849 2.07465523311967e-08
2850 2.12256186258308e-08
2851 2.07333343399396e-08
2852 2.08858252197519e-08
2853 2.09638737050932e-08
2854 2.06660064869801e-08
2855 2.11514811478963e-08
2856 2.09814890448357e-08
2857 2.08525109380853e-08
2858 2.07497924864519e-08
2859 2.09359112730567e-08
2860 2.093559929639e-08
2861 2.06009404963847e-08
2862 2.12100302238172e-08
2863 2.05980158447527e-08
2864 2.06186681521725e-08
2865 2.08862542088184e-08
2866 2.0936469918853e-08
2867 2.07627926368525e-08
2868 2.06527168982174e-08
2869 2.11511566502454e-08
2870 2.06822383124194e-08
2871 2.07941613930007e-08
2872 2.07559188144524e-08
2873 2.10543842035227e-08
2874 2.05840845423078e-08
2875 2.06535594552415e-08
2876 2.05974685043575e-08
2877 2.10835172055335e-08
2878 2.07733508724717e-08
2879 2.06281092630656e-08
2880 2.06996696985318e-08
2881 2.05928867468286e-08
2882 2.11614455344344e-08
2883 2.08973885852393e-08
2884 2.05854652760573e-08
2885 2.08793464357981e-08
2886 2.04275294294343e-08
2887 2.09185582905613e-08
2888 2.09370974737499e-08
2889 2.06647544062077e-08
2890 2.05757106733762e-08
2891 2.06192735664468e-08
2892 2.0846475898284e-08
2893 2.08889525464828e-08
2894 2.05483148564278e-08
2895 2.03616882594737e-08
2896 2.11555524313178e-08
2897 2.04765718255784e-08
2898 2.07763776445269e-08
2899 2.04691062906104e-08
2900 2.05173356947475e-08
2901 2.08954627918256e-08
2902 2.05753342683579e-08
2903 2.0665008421239e-08
2904 2.08193014182845e-08
2905 2.06230431275545e-08
2906 2.03956048585585e-08
2907 2.05801228722047e-08
2908 2.07224730037758e-08
2909 2.05524799213874e-08
2910 2.03921579357003e-08
2911 2.10132959274079e-08
2912 2.06421588265782e-08
2913 2.04563283952019e-08
2914 2.06446268881111e-08
2915 2.042753977749e-08
2916 2.03825037704553e-08
2917 2.07875830272597e-08
2918 2.08011483526782e-08
2919 2.06101566817551e-08
2920 2.03525157722595e-08
2921 2.04899607394449e-08
2922 2.06237808270115e-08
2923 2.06998062985964e-08
2924 2.07112286016997e-08
2925 2.03550020916321e-08
2926 2.08644582602169e-08
2927 2.0412516270496e-08
2928 2.05557076490281e-08
2929 2.03394948298241e-08
2930 2.06532547964988e-08
2931 2.02801794334517e-08
2932 2.07597828361106e-08
2933 2.02550767699483e-08
2934 2.07283404592973e-08
2935 2.03673894075829e-08
2936 2.04835833514938e-08
2937 2.050505140061e-08
2938 2.07645688586888e-08
2939 2.04239343837287e-08
2940 2.03455438597677e-08
2941 2.03069019342195e-08
2942 2.04494061420979e-08
2943 2.06728565901315e-08
2944 2.06858318181347e-08
2945 2.01608464037717e-08
2946 2.03585147005025e-08
2947 2.03383927517287e-08
2948 2.02021793508367e-08
2949 2.07816830462004e-08
2950 2.03949181846186e-08
2951 2.01812400221213e-08
2952 2.06846635195657e-08
2953 2.04860461354706e-08
2954 2.01512241445601e-08
2955 2.01811535165408e-08
2956 2.08092308461305e-08
2957 2.02434988983979e-08
2958 2.02048230710794e-08
2959 2.06058479856885e-08
2960 1.99582070142812e-08
2961 2.05431889630514e-08
2962 2.00417001928743e-08
2963 2.04876175844415e-08
2964 2.03843026681527e-08
2965 2.02367835424599e-08
2966 2.0017887232382e-08
2967 2.07091444832663e-08
2968 2.0414774478672e-08
2969 2.00275599662181e-08
2970 2.05543230432648e-08
2971 2.01226742382321e-08
2972 2.02409207448184e-08
2973 2.04162865737878e-08
2974 1.99240214889818e-08
2975 2.0678536702623e-08
2976 2.01035833045093e-08
2977 2.0198737493593e-08
2978 2.01368978089977e-08
2979 2.04561470206199e-08
2980 2.00458302264117e-08
2981 2.04844669385862e-08
2982 1.9938557464072e-08
2983 2.04197379505278e-08
2984 1.99995643944684e-08
2985 2.05319174665375e-08
2986 2.01686241813048e-08
2987 2.00977989602835e-08
2988 2.01755669919645e-08
2989 1.9967120910791e-08
2990 2.05835766882156e-08
2991 1.99111770760041e-08
2992 2.06249766698807e-08
2993 1.99829121635409e-08
2994 2.03047739631979e-08
2995 1.99240458417238e-08
2996 2.04638612101293e-08
2997 1.99414408255105e-08
2998 2.02041063185332e-08
2999 2.03275416282844e-08
3000 6.40849155449097e-09
3001 6.44593619243683e-09
3002 6.57360516362337e-09
3003 6.68246178416043e-09
3004 6.74533369963981e-09
3005 6.76387344802909e-09
3006 6.76746779088022e-09
3007 6.76727325323034e-09
3008 6.76617789878431e-09
3009 6.76474149434447e-09
3010 6.76337173798058e-09
3011 6.76191529822012e-09
3012 6.76054089444877e-09
3013 6.75913523111305e-09
3014 6.75778932889004e-09
3015 6.75639885387713e-09
3016 6.75503333417982e-09
3017 6.75370958334864e-09
3018 6.75237115416172e-09
3019 6.751034204458e-09
3020 6.7497613292361e-09
3021 6.74844434668531e-09
3022 6.74717952499348e-09
3023 6.74589272275927e-09
3024 6.74460228729246e-09
3025 6.74334003390731e-09
3026 6.74208095853557e-09
3027 6.74077154244612e-09
3028 6.73952707073988e-09
3029 6.73832801495466e-09
3030 6.73705838467875e-09
3031 6.73581130981871e-09
3032 6.73459843025803e-09
3033 6.73335089490523e-09
3034 6.7321805229531e-09
3035 6.73095337258567e-09
3036 6.7297471345179e-09
3037 6.728545321813e-09
3038 6.7272908183813e-09
3039 6.72613310152836e-09
3040 6.7249706454664e-09
3041 6.72374349182381e-09
3042 6.72260212855602e-09
3043 6.72140331071935e-09
3044 6.72024758301137e-09
3045 6.71911485608523e-09
3046 6.71796012678694e-09
3047 6.71676912275543e-09
3048 6.71557973241921e-09
3049 6.71441844124487e-09
3050 6.71328964506335e-09
3051 6.71215020990601e-09
3052 6.71104124988497e-09
3053 6.7098591183673e-09
3054 6.70873234112634e-09
3055 6.70763605609137e-09
3056 6.70649359847675e-09
3057 6.70537963838591e-09
3058 6.70424569285122e-09
3059 6.70314587289389e-09
3060 6.70202460983926e-09
3061 6.70089716557631e-09
3062 6.69977885214834e-09
3063 6.69865599443875e-09
3064 6.69758216342853e-09
3065 6.69641948015942e-09
3066 6.69535682686073e-09
3067 6.6942718584817e-09
3068 6.69315798103309e-09
3069 6.69204937325807e-09
3070 6.69098377549526e-09
3071 6.68988921848079e-09
3072 6.68878032415721e-09
3073 6.68773357881403e-09
3074 6.68661127452286e-09
3075 6.68554577086533e-09
3076 6.6844940415367e-09
3077 6.68334880220278e-09
3078 6.68233697788401e-09
3079 6.68125869818237e-09
3080 6.68016170024543e-09
3081 6.67906895650272e-09
3082 6.67802489763558e-09
3083 6.67698966091979e-09
3084 6.67588884814552e-09
3085 6.67485096614295e-09
3086 6.67373806567673e-09
3087 6.67273853939554e-09
3088 6.67163359596712e-09
3089 6.67056688929968e-09
3090 6.66951340932975e-09
3091 6.66847951233396e-09
3092 6.66747036309756e-09
3093 6.66639314068906e-09
3094 6.66537467133854e-09
3095 6.66430119992956e-09
3096 6.66327143018786e-09
3097 6.66219094577758e-09
3098 6.66117402621513e-09
3099 6.66012797298499e-09
3100 6.6591122592774e-09
3101 6.65806552728465e-09
3102 6.65705813122919e-09
3103 6.65595963027232e-09
3104 6.65497834775874e-09
3105 6.65388494011654e-09
3106 6.65292821898367e-09
3107 6.65189396287236e-09
3108 6.65085051416991e-09
3109 6.64983385476048e-09
3110 6.64881345922852e-09
3111 6.64776710827819e-09
3112 6.64676495756855e-09
3113 6.64573211148212e-09
3114 6.64467145759795e-09
3115 6.64366244958192e-09
3116 6.64266534747271e-09
3117 6.64165192282296e-09
3118 6.64061781886771e-09
3119 6.63964208479084e-09
3120 6.63857419148928e-09
3121 6.6376166985016e-09
3122 6.63661619230982e-09
3123 6.63559262129021e-09
3124 6.63456765587211e-09
3125 6.63360710242977e-09
3126 6.63256608748886e-09
3127 6.63153161495345e-09
3128 6.63057921487797e-09
3129 6.62953300423208e-09
3130 6.62858974374847e-09
3131 6.62755071584309e-09
3132 6.62656813192608e-09
3133 6.62555825113598e-09
3134 6.62458800276822e-09
3135 6.62354037636592e-09
3136 6.62254589106626e-09
3137 6.62160006886792e-09
3138 6.6205986607587e-09
3139 6.6196029893939e-09
3140 6.61855662088817e-09
3141 6.61759382454552e-09
3142 6.61662621895265e-09
3143 6.61559957575164e-09
3144 6.6146295789743e-09
3145 6.61365435945804e-09
3146 6.61259629641808e-09
3147 6.61171410984429e-09
3148 6.61064567461511e-09
3149 6.60966903809346e-09
3150 6.60869728597024e-09
3151 6.60771063740184e-09
3152 6.60668990463964e-09
3153 6.60573999546055e-09
3154 6.60476466342319e-09
3155 6.60382824213546e-09
3156 6.60282320240346e-09
3157 6.6017863945278e-09
3158 6.60085147524692e-09
3159 6.59981439404822e-09
3160 6.59885815994243e-09
3161 6.59789047020853e-09
3162 6.59689467283342e-09
3163 6.59596433300047e-09
3164 6.59497125193864e-09
3165 6.59403317021523e-09
3166 6.59305757813589e-09
3167 6.59205138478502e-09
3168 6.59108827202881e-09
3169 6.59013392043106e-09
3170 6.58912584904081e-09
3171 6.588185691353e-09
3172 6.58722246127397e-09
3173 6.58624130316088e-09
3174 6.58534254027732e-09
3175 6.58433099623434e-09
3176 6.58336935062398e-09
3177 6.58242081076621e-09
3178 6.58145629563178e-09
3179 6.58048957112578e-09
3180 6.5795965292631e-09
3181 6.57852746785426e-09
3182 6.5776096982717e-09
3183 6.57664682912618e-09
3184 6.57574554033258e-09
3185 6.57475821781717e-09
3186 6.57378967180988e-09
3187 6.57283321449376e-09
3188 6.57184807588418e-09
3189 6.570874032788e-09
3190 6.56992450012717e-09
3191 6.56901619827677e-09
3192 6.56803075252399e-09
3193 6.56713113902918e-09
3194 6.56621681445479e-09
3195 6.56522743008869e-09
3196 6.56428946478604e-09
3197 6.56331854061165e-09
3198 6.56234761245433e-09
3199 6.56143010063781e-09
3200 6.56047263730697e-09
3201 6.55955740298808e-09
3202 6.55862528453621e-09
3203 6.55762874983423e-09
3204 6.55669526694602e-09
3205 6.55582214249828e-09
3206 6.55483528688716e-09
3207 6.55393573349705e-09
3208 6.55296127628768e-09
3209 6.55201140603578e-09
3210 6.55108305362695e-09
3211 6.55012231058627e-09
3212 6.54921262863151e-09
3213 6.5482797197286e-09
3214 6.54734464271278e-09
3215 6.54639594196982e-09
3216 6.54543541478425e-09
3217 6.54453944715083e-09
3218 6.54358756239926e-09
3219 6.54264704844087e-09
3220 6.54167044009113e-09
3221 6.54081502281245e-09
3222 6.53977201396649e-09
3223 6.53888424384519e-09
3224 6.53793076126064e-09
3225 6.5370563955558e-09
3226 6.53608103300118e-09
3227 6.53517322474206e-09
3228 6.5342306014432e-09
3229 6.53328303074074e-09
3230 6.53231440626845e-09
3231 6.53142027287612e-09
3232 6.53048510529386e-09
3233 6.52956944190153e-09
3234 6.52864486946902e-09
3235 6.52771502812621e-09
3236 6.5267525977547e-09
3237 6.5258573088145e-09
3238 6.52495250867691e-09
3239 6.52404836580522e-09
3240 6.52308385380718e-09
3241 6.52217174207381e-09
3242 6.5212117023733e-09
3243 6.52032029692384e-09
3244 6.51940405459184e-09
3245 6.5184608041835e-09
3246 6.51755041675139e-09
3247 6.51662588910251e-09
3248 6.51570464990647e-09
3249 6.51475451587946e-09
3250 6.51386047230418e-09
3251 6.51291417753941e-09
3252 6.51201976265003e-09
3253 6.5110641519206e-09
3254 6.51017634820117e-09
3255 6.50925979583938e-09
3256 6.50835718987719e-09
3257 6.50744079293275e-09
3258 6.50651554764958e-09
3259 6.50557811489316e-09
3260 6.50471200833169e-09
3261 6.5037884063035e-09
3262 6.50285900906378e-09
3263 6.5019790919385e-09
3264 6.50105690562508e-09
3265 6.50008892773279e-09
3266 6.49926564900982e-09
3267 6.49826245378848e-09
3268 6.49735216391723e-09
3269 6.49644834288532e-09
3270 6.49556019478859e-09
3271 6.49465942270666e-09
3272 6.4937137579657e-09
3273 6.4928437799594e-09
3274 6.49190932779098e-09
3275 6.491031114636e-09
3276 6.49007679329183e-09
3277 6.4892005810363e-09
3278 6.48827843004185e-09
3279 6.48731403442293e-09
3280 6.48644764660034e-09
3281 6.48558804632049e-09
3282 6.48462133648331e-09
3283 6.48370836731083e-09
3284 6.48286080780214e-09
3285 6.48193558928922e-09
3286 6.48101236452869e-09
3287 6.4800846765467e-09
3288 6.47921827551246e-09
3289 6.47827204244833e-09
3290 6.47737347896082e-09
3291 6.47647121029826e-09
3292 6.47560427986804e-09
3293 6.47466170869415e-09
3294 6.47376868789795e-09
3295 6.47290331358408e-09
3296 6.47200254542957e-09
3297 6.47108152500297e-09
3298 6.47018290941825e-09
3299 6.46931633059566e-09
3300 6.46837188815474e-09
3301 6.46747042946927e-09
3302 6.46660950281985e-09
3303 6.46569960920107e-09
3304 6.46479689522805e-09
3305 6.46385390175308e-09
3306 6.46301887634315e-09
3307 6.46214175113735e-09
3308 6.46121177577674e-09
3309 6.4602962151078e-09
3310 6.45938331811369e-09
3311 6.45852730433993e-09
3312 6.45764334424037e-09
3313 6.45677120919563e-09
3314 6.45587696727901e-09
3315 6.45496624955555e-09
3316 6.45409601696123e-09
3317 6.45318875247547e-09
3318 6.45231562514115e-09
3319 6.45142355963635e-09
3320 6.45049626564476e-09
3321 6.44962158098672e-09
3322 6.44874612937274e-09
3323 6.44785935179082e-09
3324 6.44699549377792e-09
3325 6.44606197906794e-09
3326 6.44522225221045e-09
3327 6.44426621686234e-09
3328 6.44341054413522e-09
3329 6.44252787569688e-09
3330 6.44167826593933e-09
3331 6.44074444619558e-09
3332 6.43985321061025e-09
3333 6.43895278563955e-09
3334 6.43808634238907e-09
3335 6.43721732832003e-09
3336 6.43633739132177e-09
3337 6.43543315863304e-09
3338 6.4345743698152e-09
3339 6.4336887566907e-09
3340 6.43278211009957e-09
3341 6.43190288657225e-09
3342 6.43106233316759e-09
3343 6.43019555075786e-09
3344 6.42928798967601e-09
3345 6.4284102491291e-09
3346 6.42755195599809e-09
3347 6.42662899474899e-09
3348 6.42577314075021e-09
3349 6.42486711634194e-09
3350 6.42401901983225e-09
3351 6.42309002525654e-09
3352 6.42223810747022e-09
3353 6.42140225162735e-09
3354 6.42047356992637e-09
3355 6.41955802385685e-09
3356 6.4187188672099e-09
3357 6.41787150768014e-09
3358 6.41698335317187e-09
3359 6.41609640607277e-09
3360 6.41522982619547e-09
3361 6.4143308229625e-09
3362 6.41346047119962e-09
3363 6.41259858923104e-09
3364 6.41177161925244e-09
3365 6.41084591387897e-09
3366 6.40997761129636e-09
3367 6.4091312495379e-09
3368 6.40821446587503e-09
3369 6.40741233830233e-09
3370 6.40652148935217e-09
3371 6.40565288261008e-09
3372 6.40475848294464e-09
3373 6.40393980823328e-09
3374 6.40301290497081e-09
3375 6.40213884864349e-09
3376 6.40126144754727e-09
3377 6.40037912735814e-09
3378 6.39953000822202e-09
3379 6.39864385050537e-09
3380 6.3978395058728e-09
3381 6.39694293527726e-09
3382 6.39606847183116e-09
3383 6.39514978062472e-09
3384 6.39432588367406e-09
3385 6.39344259867336e-09
3386 6.39259876557274e-09
3387 6.39168661706324e-09
3388 6.39081821159071e-09
3389 6.38997612990855e-09
3390 6.38914762067511e-09
3391 6.38821969335679e-09
3392 6.38736310749899e-09
3393 6.38653467578687e-09
3394 6.38563886549981e-09
3395 6.38476911811459e-09
3396 6.38397850417916e-09
3397 6.38306510154785e-09
3398 6.3822210587261e-09
3399 6.38133841636412e-09
3400 6.3805316345561e-09
3401 6.37960059600429e-09
3402 6.3787747298788e-09
3403 6.37790874034871e-09
3404 6.37701282796277e-09
3405 6.3761837818116e-09
3406 6.37531112636369e-09
3407 6.37444718869229e-09
3408 6.37354743937557e-09
3409 6.3727426250354e-09
3410 6.37182906161604e-09
3411 6.37100079488306e-09
3412 6.37010475484923e-09
3413 6.3693391890024e-09
3414 6.36840053776233e-09
3415 6.36759211972149e-09
3416 6.36668380714356e-09
3417 6.36583002308622e-09
3418 6.36498927644724e-09
3419 6.36413763314969e-09
3420 6.36327270234216e-09
3421 6.36240439630398e-09
3422 6.36153124076999e-09
3423 6.3606916723552e-09
3424 6.35993298947479e-09
3425 6.35899781667448e-09
3426 6.35816444016779e-09
3427 6.3573017539259e-09
3428 6.35645183727496e-09
3429 6.35565197518173e-09
3430 6.35471699882151e-09
3431 6.35390843407857e-09
3432 6.35301574514191e-09
3433 6.35217419168999e-09
3434 6.35130088960656e-09
3435 6.35051867602099e-09
3436 6.34964037263264e-09
3437 6.34874127368457e-09
3438 6.34785553896289e-09
3439 6.34707190659556e-09
3440 6.34618837429268e-09
3441 6.34530450459303e-09
3442 6.34449318863417e-09
3443 6.343643890655e-09
3444 6.34283062372898e-09
3445 6.3419361632927e-09
3446 6.34112406137921e-09
3447 6.34025687669404e-09
3448 6.3394240083503e-09
3449 6.3385427012258e-09
3450 6.33774743331872e-09
3451 6.33687586285014e-09
3452 6.33597956754828e-09
3453 6.33514274109293e-09
3454 6.33430477198216e-09
3455 6.3335050758534e-09
3456 6.33259577635659e-09
3457 6.33174169809014e-09
3458 6.33086904998359e-09
3459 6.33006384913315e-09
3460 6.32924850381411e-09
3461 6.32838583354556e-09
3462 6.32750331706899e-09
3463 6.32669931975582e-09
3464 6.32585128951257e-09
3465 6.32499897754157e-09
3466 6.32412277354333e-09
3467 6.3232842081179e-09
3468 6.32250234530729e-09
3469 6.32162497891942e-09
3470 6.32071561992853e-09
3471 6.31993278649157e-09
3472 6.31907285197908e-09
3473 6.31820082178103e-09
3474 6.31739124454245e-09
3475 6.31655107430351e-09
3476 6.31571230098882e-09
3477 6.31488411713399e-09
3478 6.31404205636565e-09
3479 6.31321913860394e-09
3480 6.31231132799948e-09
3481 6.31153473326651e-09
3482 6.31066412883074e-09
3483 6.30982899658961e-09
3484 6.30902147312484e-09
3485 6.30817386589044e-09
3486 6.30737148020477e-09
3487 6.30645885060788e-09
3488 6.30566780532304e-09
3489 6.30480735014372e-09
3490 6.30395617473578e-09
3491 6.3031641636957e-09
3492 6.30232121331953e-09
3493 6.30145729604847e-09
3494 6.30062769008122e-09
3495 6.29978719905722e-09
3496 6.29895256938629e-09
3497 6.2981074335916e-09
3498 6.29730478611534e-09
3499 6.2963987927378e-09
3500 6.29558374601324e-09
3501 6.29473544466741e-09
3502 6.29392095455317e-09
3503 6.29308103790305e-09
3504 6.2922530121301e-09
3505 6.2914058086283e-09
3506 6.29057686890588e-09
3507 6.28972685441653e-09
3508 6.2889296868901e-09
3509 6.28805240041053e-09
3510 6.28723830435607e-09
3511 6.28639730043679e-09
3512 6.28558374100863e-09
3513 6.28472071158292e-09
3514 6.28391389818905e-09
3515 6.28308089070662e-09
3516 6.28230631838367e-09
3517 6.28139350762613e-09
3518 6.28061276641834e-09
3519 6.27978896050596e-09
3520 6.27893884842801e-09
3521 6.27809142525471e-09
3522 6.27733022842691e-09
3523 6.27647935580455e-09
3524 6.27561119966036e-09
3525 6.2748077884478e-09
3526 6.27398536097445e-09
3527 6.2731232123997e-09
3528 6.27230627474329e-09
3529 6.2714794041574e-09
3530 6.27069044523143e-09
3531 6.2698321504906e-09
3532 6.26900754571724e-09
3533 6.26815687648774e-09
3534 6.26736632973468e-09
3535 6.26654619799438e-09
3536 6.2657259542187e-09
3537 6.26488419766535e-09
3538 6.26401431848278e-09
3539 6.26322233822363e-09
3540 6.26238682102653e-09
3541 6.26160062018599e-09
3542 6.2607955481353e-09
3543 6.25992993689983e-09
3544 6.25907519774538e-09
3545 6.25824829933452e-09
3546 6.25745255979082e-09
3547 6.25661500334995e-09
3548 6.25578542386152e-09
3549 6.25495107298146e-09
3550 6.25413304304601e-09
3551 6.25334271339706e-09
3552 6.25248884392193e-09
3553 6.25168179192725e-09
3554 6.25087537116376e-09
3555 6.25003970991522e-09
3556 6.24918894262527e-09
3557 6.24842295522676e-09
3558 6.24757244500895e-09
3559 6.24671895579909e-09
3560 6.24590531522751e-09
3561 6.24512110070086e-09
3562 6.24426864291594e-09
3563 6.24347909514156e-09
3564 6.24264707509148e-09
3565 6.24177375724289e-09
3566 6.24100717602771e-09
3567 6.24018376679802e-09
3568 6.23935386943886e-09
3569 6.2384983586794e-09
3570 6.23765898163931e-09
3571 6.23686234842158e-09
3572 6.23605282815132e-09
3573 6.23522756944273e-09
3574 6.23439387359426e-09
3575 6.23356442051859e-09
3576 6.23275331704254e-09
3577 6.23196021488914e-09
3578 6.23109754224749e-09
3579 6.23031934307572e-09
3580 6.22952105415464e-09
3581 6.2286757761959e-09
3582 6.22786433492062e-09
3583 6.2270347226806e-09
3584 6.22622405800632e-09
3585 6.22539865652305e-09
3586 6.22458231906708e-09
3587 6.22375777570294e-09
3588 6.22296607363793e-09
3589 6.2221264838791e-09
3590 6.22134239623706e-09
3591 6.22052057026867e-09
3592 6.21972485394251e-09
3593 6.21889573748646e-09
3594 6.21805744947801e-09
3595 6.21719324311876e-09
3596 6.21648020357468e-09
3597 6.21561337142695e-09
3598 6.21480321010004e-09
3599 6.21398927490302e-09
3600 6.21315523044452e-09
3601 6.21232718846232e-09
3602 6.211608802556e-09
3603 6.21071100899429e-09
3604 6.20992051592051e-09
3605 6.20911225979182e-09
3606 6.20827967659499e-09
3607 6.20748962704143e-09
3608 6.2066315119208e-09
3609 6.20585222763093e-09
3610 6.20502293556535e-09
3611 6.20420489463869e-09
3612 6.20338722373548e-09
3613 6.20261350714468e-09
3614 6.20177128114741e-09
3615 6.20098353902587e-09
3616 6.20017798041994e-09
3617 6.19936601027604e-09
3618 6.19854980862811e-09
3619 6.19767094536816e-09
3620 6.19692834916141e-09
3621 6.19609455319858e-09
3622 6.19531802618922e-09
3623 6.19454690820032e-09
3624 6.19366395911147e-09
3625 6.19285795472324e-09
3626 6.19205184407279e-09
3627 6.19127581663603e-09
3628 6.19044246534528e-09
3629 6.18964953466583e-09
3630 6.18885473517572e-09
3631 6.18805516056087e-09
3632 6.18722728198962e-09
3633 6.18642857576346e-09
3634 6.18561317321242e-09
3635 6.18481959153983e-09
3636 6.18401123138324e-09
3637 6.18318048996314e-09
3638 6.18245424797992e-09
3639 6.18153212519901e-09
3640 6.18078928270316e-09
3641 6.18003035930681e-09
3642 6.17913378840596e-09
3643 6.17842116923395e-09
3644 6.17756765933231e-09
3645 6.17679367001522e-09
3646 6.17596851665292e-09
3647 6.17518533393979e-09
3648 6.17440173970862e-09
3649 6.1735798479734e-09
3650 6.17278007748745e-09
3651 6.17197684625592e-09
3652 6.17115642699562e-09
3653 6.17036597895526e-09
3654 6.1696133744904e-09
3655 6.16880673164066e-09
3656 6.16799359245968e-09
3657 6.16718890669721e-09
3658 6.1663190064204e-09
3659 6.16559883816203e-09
3660 6.16476235872077e-09
3661 6.16399517687494e-09
3662 6.16314284059005e-09
3663 6.16239381781569e-09
3664 6.1615718132263e-09
3665 6.16073363349234e-09
3666 6.15998053479783e-09
3667 6.15912817444886e-09
3668 6.15832652402226e-09
3669 6.15757598536326e-09
3670 6.15672872633644e-09
3671 6.15595370732913e-09
3672 6.15520293624494e-09
3673 6.15435036753487e-09
3674 6.15358933758747e-09
3675 6.15276467630377e-09
3676 6.151954429226e-09
3677 6.1511679932541e-09
3678 6.15036861610629e-09
3679 6.14955668150341e-09
3680 6.14879666542556e-09
3681 6.14795657072342e-09
3682 6.14717397177944e-09
3683 6.146371235069e-09
3684 6.14553532703177e-09
3685 6.14479614094443e-09
3686 6.14399184376002e-09
3687 6.1431954010982e-09
3688 6.14242434615608e-09
3689 6.14159277441406e-09
3690 6.14081678768086e-09
3691 6.13999911419638e-09
3692 6.13922449359261e-09
3693 6.13841670629722e-09
3694 6.13759635438582e-09
3695 6.13678648932192e-09
3696 6.13605425876795e-09
3697 6.13517811154374e-09
3698 6.13450265854465e-09
3699 6.13362540345663e-09
3700 6.13284576876649e-09
3701 6.13204564002545e-09
3702 6.13123216090805e-09
3703 6.13045055113115e-09
3704 6.12966137211735e-09
3705 6.12889699194708e-09
3706 6.12806625860385e-09
3707 6.12730697117925e-09
3708 6.12649710687863e-09
3709 6.12575289860029e-09
3710 6.12494197073377e-09
3711 6.12415489159579e-09
3712 6.12336350039322e-09
3713 6.12254630082132e-09
3714 6.12179577376415e-09
3715 6.1209684365654e-09
3716 6.12019809874409e-09
3717 6.11939371346348e-09
3718 6.11865152670699e-09
3719 6.11781646581155e-09
3720 6.11701727594449e-09
3721 6.11625845932384e-09
3722 6.11548841344955e-09
3723 6.1146641754245e-09
3724 6.11389298836584e-09
3725 6.11312036376266e-09
3726 6.11233175576431e-09
3727 6.11152955898309e-09
3728 6.11071181712275e-09
3729 6.10992415141232e-09
3730 6.10916867407885e-09
3731 6.10834876524358e-09
3732 6.10761813371608e-09
3733 6.10681870397145e-09
3734 6.10603193906822e-09
3735 6.10524291312642e-09
3736 6.10442957862944e-09
3737 6.10370011398798e-09
3738 6.10290169351935e-09
3739 6.10208346951691e-09
3740 6.10130711120593e-09
3741 6.10055234739892e-09
3742 6.09973782006445e-09
3743 6.09891398262175e-09
3744 6.0981621407552e-09
3745 6.09741034157674e-09
3746 6.09662393814825e-09
3747 6.09582526278629e-09
3748 6.09505785552356e-09
3749 6.09426745960817e-09
3750 6.09349445702956e-09
3751 6.09270068870071e-09
3752 6.09195734724288e-09
3753 6.09115146968375e-09
3754 6.09033520870828e-09
3755 6.08956637204727e-09
3756 6.08878730114226e-09
3757 6.08797592552279e-09
3758 6.08719238177902e-09
3759 6.08647728171652e-09
3760 6.08566501519858e-09
3761 6.08491889707297e-09
3762 6.08408549827855e-09
3763 6.08328207832298e-09
3764 6.08253367528899e-09
3765 6.08178079093691e-09
3766 6.08097969100052e-09
3767 6.08016199002415e-09
3768 6.07937559404803e-09
3769 6.07862362116129e-09
3770 6.07780691050386e-09
3771 6.07706563214572e-09
3772 6.07630665806769e-09
3773 6.07552788654819e-09
3774 6.07469386205983e-09
3775 6.07395910673381e-09
3776 6.07310803449335e-09
3777 6.07242487563053e-09
3778 6.07160408998275e-09
3779 6.07084154871651e-09
3780 6.07003781842586e-09
3781 6.06928057567124e-09
3782 6.06850814148518e-09
3783 6.06776403860865e-09
3784 6.06701981452351e-09
3785 6.0661950307267e-09
3786 6.06537250961991e-09
3787 6.06462966898369e-09
3788 6.06389616131242e-09
3789 6.06307980292875e-09
3790 6.0622713649594e-09
3791 6.06155810997655e-09
3792 6.06073395661988e-09
3793 6.06000513715677e-09
3794 6.05922396583469e-09
3795 6.0584578995132e-09
3796 6.05767980849103e-09
3797 6.05691716970558e-09
3798 6.05614494912643e-09
3799 6.05537093205377e-09
3800 6.05457380473129e-09
3801 6.05385909989431e-09
3802 6.05303907265375e-09
3803 6.05221814313495e-09
3804 6.05146003512802e-09
3805 6.05074652304527e-09
3806 6.04998507224008e-09
3807 6.0491311286992e-09
3808 6.04839602427742e-09
3809 6.047599668213e-09
3810 6.04687557526073e-09
3811 6.04605767998145e-09
3812 6.04531121616014e-09
3813 6.04454864762405e-09
3814 6.04375741829199e-09
3815 6.0430210015866e-09
3816 6.04222406273835e-09
3817 6.04143525327616e-09
3818 6.04065080329896e-09
3819 6.03993477914233e-09
3820 6.03915102963259e-09
3821 6.03838648177701e-09
3822 6.03762994433321e-09
3823 6.03683971669988e-09
3824 6.03610353848427e-09
3825 6.03530644838202e-09
3826 6.03451889720497e-09
3827 6.03377652524939e-09
3828 6.03300320760336e-09
3829 6.03224309959904e-09
3830 6.03146085254025e-09
3831 6.03066152612963e-09
3832 6.02992047012141e-09
3833 6.02915105946122e-09
3834 6.0284001731914e-09
3835 6.02764815445245e-09
3836 6.02687093820276e-09
3837 6.02608216238032e-09
3838 6.02532312848902e-09
3839 6.02455294737569e-09
3840 6.02383473907731e-09
3841 6.02302459938586e-09
3842 6.02231001776976e-09
3843 6.02146633638112e-09
3844 6.02068936873812e-09
3845 6.02003710380161e-09
3846 6.01920509749054e-09
3847 6.01844817248176e-09
3848 6.01767670684039e-09
3849 6.01693001221759e-09
3850 6.01614511423765e-09
3851 6.01540686868962e-09
3852 6.01460159721512e-09
3853 6.01388384550983e-09
3854 6.01305310220235e-09
3855 6.01234097660774e-09
3856 6.01153955130662e-09
3857 6.01082043186207e-09
3858 6.01000974949362e-09
3859 6.00927863335376e-09
3860 6.00853276422342e-09
3861 6.00773294084922e-09
3862 6.00701490792444e-09
3863 6.00622509247528e-09
3864 6.00547929614781e-09
3865 6.00468804043408e-09
3866 6.00395484157135e-09
3867 6.00322267650666e-09
3868 6.00246191027887e-09
3869 6.00167406819563e-09
3870 6.00092937472208e-09
3871 6.00018527452395e-09
3872 5.99946254499495e-09
3873 5.998624858089e-09
3874 5.99791947571149e-09
3875 5.99720862899666e-09
3876 5.99636782917801e-09
3877 5.99564032877087e-09
3878 5.99479847603057e-09
3879 5.99411813279049e-09
3880 5.99335543939594e-09
3881 5.99258302091954e-09
3882 5.99184228770866e-09
3883 5.99101992004858e-09
3884 5.99026282546711e-09
3885 5.98957556241508e-09
3886 5.98877301871692e-09
3887 5.9879947921504e-09
3888 5.98724306999365e-09
3889 5.98647699580346e-09
3890 5.9857561825194e-09
3891 5.98496276471572e-09
3892 5.98425140807546e-09
3893 5.98346088823143e-09
3894 5.98270755046426e-09
3895 5.98190056652625e-09
3896 5.98120267987501e-09
3897 5.98045643873668e-09
3898 5.97966703273778e-09
3899 5.97891281584051e-09
3900 5.97816839288667e-09
3901 5.97739938835995e-09
3902 5.97668738717971e-09
3903 5.97585356616748e-09
3904 5.97517115995261e-09
3905 5.97439412912404e-09
3906 5.97363410109741e-09
3907 5.97286295819788e-09
3908 5.97214576061877e-09
3909 5.97133936737493e-09
3910 5.9706869518783e-09
3911 5.96986537149124e-09
3912 5.96911583021498e-09
3913 5.96835735039436e-09
3914 5.96760055364409e-09
3915 5.96687325771228e-09
3916 5.96612589810752e-09
3917 5.96536432784234e-09
3918 5.96459848373199e-09
3919 5.96383690266988e-09
3920 5.96307669854801e-09
3921 5.96238682450123e-09
3922 5.96161839221121e-09
3923 5.96084850162937e-09
3924 5.96012428293047e-09
3925 5.95934987678015e-09
3926 5.95860593381736e-09
3927 5.95782904312669e-09
3928 5.95711747740368e-09
3929 5.95633540326212e-09
3930 5.95558479447866e-09
3931 5.95486097255959e-09
3932 5.95405810230321e-09
3933 5.95335359983296e-09
3934 5.95261789919366e-09
3935 5.95183173539293e-09
3936 5.95105176789956e-09
3937 5.95036844004693e-09
3938 5.94955165887645e-09
3939 5.94882717007417e-09
3940 5.94806678555493e-09
3941 5.94730327714565e-09
3942 5.94657269543941e-09
3943 5.94581804061467e-09
3944 5.94508084453271e-09
3945 5.94435192458054e-09
3946 5.94354200317282e-09
3947 5.94278854025576e-09
3948 5.94206854434565e-09
3949 5.94132503194122e-09
3950 5.94059793811164e-09
3951 5.93984087290944e-09
3952 5.93906870492711e-09
3953 5.93837084238158e-09
3954 5.93755588466915e-09
3955 5.93683754512853e-09
3956 5.93605231059835e-09
3957 5.93532263437613e-09
3958 5.93462739252593e-09
3959 5.93378511744291e-09
3960 5.93304776846937e-09
3961 5.93232959218704e-09
3962 5.93160779921442e-09
3963 5.93083800616567e-09
3964 5.93012249271163e-09
3965 5.92933746142166e-09
3966 5.92864327080001e-09
3967 5.92788475152484e-09
3968 5.92707881386101e-09
3969 5.92636854972184e-09
3970 5.9256397748203e-09
3971 5.92486747057197e-09
3972 5.92415674036117e-09
3973 5.92342095075138e-09
3974 5.92266029000865e-09
3975 5.92185473705098e-09
3976 5.92116539259446e-09
3977 5.92040006446026e-09
3978 5.91968909736951e-09
3979 5.91892992980736e-09
3980 5.91813460434909e-09
3981 5.91744762752644e-09
3982 5.91670855576432e-09
3983 5.91590803217246e-09
3984 5.91522839446523e-09
3985 5.91447880483875e-09
3986 5.91375974294539e-09
3987 5.91293109762636e-09
3988 5.91224432028303e-09
3989 5.91148225165261e-09
3990 5.91072425178141e-09
3991 5.91000401159447e-09
3992 5.9092428408708e-09
3993 5.90852643565787e-09
3994 5.90777746292681e-09
3995 5.90701723407472e-09
3996 5.90630565890093e-09
3997 5.90554544176169e-09
3998 5.90482736961828e-09
3999 5.90407116056457e-09
4000 5.90333094818707e-09
4001 5.90258107507902e-09
4002 5.90183673547517e-09
4003 5.90109852985354e-09
4004 5.90034545452922e-09
4005 5.89962376534858e-09
4006 5.89890136111493e-09
4007 5.8981307225342e-09
4008 5.89737311257332e-09
4009 5.89662011268866e-09
4010 5.89589826506531e-09
4011 5.8951715266875e-09
4012 5.89442078238744e-09
4013 5.89370840263503e-09
4014 5.89293131900148e-09
4015 5.89221234910398e-09
4016 5.89148683842466e-09
4017 5.89076146269296e-09
4018 5.88999746531371e-09
4019 5.88923364758243e-09
4020 5.88851160400472e-09
4021 5.88776159690663e-09
4022 5.88700754064475e-09
4023 5.88626439126938e-09
4024 5.88557209071916e-09
4025 5.88482358623854e-09
4026 5.88406991772206e-09
4027 5.8833014918297e-09
4028 5.88257800325509e-09
4029 5.88182892434508e-09
4030 5.88110996253832e-09
4031 5.88039616604419e-09
4032 5.87961606283993e-09
4033 5.87895030465524e-09
4034 5.87820202430089e-09
4035 5.87742490774923e-09
4036 5.87674979961317e-09
4037 5.87594419786119e-09
4038 5.8752639762738e-09
4039 5.87446206858078e-09
4040 5.87379552056955e-09
4041 5.87302743287887e-09
4042 5.87223231193756e-09
4043 5.87156025906588e-09
4044 5.87086804516856e-09
4045 5.87004742481911e-09
4046 5.86938172079943e-09
4047 5.86857825086995e-09
4048 5.86789147344335e-09
4049 5.86713861078225e-09
4050 5.86640498093094e-09
4051 5.86564803942147e-09
4052 5.86494317977859e-09
4053 5.86419233999935e-09
4054 5.86350533698932e-09
4055 5.86274436027712e-09
4056 5.86202764554789e-09
4057 5.86133075461404e-09
4058 5.86056261500656e-09
4059 5.85980770170802e-09
4060 5.85906061062458e-09
4061 5.85839699220048e-09
4062 5.85761188641454e-09
4063 5.85689756302243e-09
4064 5.8562081624719e-09
4065 5.85542956285656e-09
4066 5.85475748914044e-09
4067 5.85398400008985e-09
4068 5.8532623713331e-09
4069 5.85256520226063e-09
4070 5.85176442279012e-09
4071 5.85107083489467e-09
4072 5.85028065777649e-09
4073 5.84962292066171e-09
4074 5.84886959455189e-09
4075 5.84808478343302e-09
4076 5.84739790397693e-09
4077 5.84667810121786e-09
4078 5.84596426364548e-09
4079 5.84515954851761e-09
4080 5.84445682529078e-09
4081 5.84377939281955e-09
4082 5.84296037650645e-09
4083 5.84229258045188e-09
4084 5.84154609144238e-09
4085 5.84085126777156e-09
4086 5.84008882481557e-09
4087 5.83936367767879e-09
4088 5.83866089305662e-09
4089 5.83792321112719e-09
4090 5.83717373889292e-09
4091 5.83647914879915e-09
4092 5.83578703153287e-09
4093 5.8350255967704e-09
4094 5.83428335064473e-09
4095 5.83356300981608e-09
4096 5.83281955085502e-09
4097 5.8320980815818e-09
4098 5.8312853410819e-09
4099 5.83060914707834e-09
4100 5.82990679735829e-09
4101 5.82919219561939e-09
4102 5.828410872738e-09
4103 5.82771433403628e-09
4104 5.8270179204567e-09
4105 5.82625636495748e-09
4106 5.82553768056771e-09
4107 5.82479406861791e-09
4108 5.82406807668467e-09
4109 5.82338461582732e-09
4110 5.82262168168091e-09
4111 5.82191670098209e-09
4112 5.82117001299287e-09
4113 5.82045688019006e-09
4114 5.81974760968662e-09
4115 5.81896992683795e-09
4116 5.81829650905807e-09
4117 5.81757737211364e-09
4118 5.81686386451341e-09
4119 5.81613311642637e-09
4120 5.81538984513463e-09
4121 5.81469057417405e-09
4122 5.81393842769007e-09
4123 5.81318879885884e-09
4124 5.81252066723936e-09
4125 5.81179522396447e-09
4126 5.81107088490351e-09
4127 5.81033466065528e-09
4128 5.80962354725101e-09
4129 5.80893966060925e-09
4130 5.80821131609954e-09
4131 5.80744143363621e-09
4132 5.80675053911406e-09
4133 5.80602541153108e-09
4134 5.80525261777154e-09
4135 5.80461234485352e-09
4136 5.8038438175978e-09
4137 5.80314405623783e-09
4138 5.8024452819494e-09
4139 5.80167024778755e-09
4140 5.80097122887535e-09
4141 5.80024559698744e-09
4142 5.7995290450169e-09
4143 5.79883916705659e-09
4144 5.79806366976521e-09
4145 5.7973930174815e-09
4146 5.79662664709768e-09
4147 5.7959457062523e-09
4148 5.7951964862607e-09
4149 5.79450158885719e-09
4150 5.793780527244e-09
4151 5.79305025949495e-09
4152 5.79229698932648e-09
4153 5.79165700138884e-09
4154 5.79094933941815e-09
4155 5.79015677169836e-09
4156 5.78944449504404e-09
4157 5.78875743309504e-09
4158 5.78805979827013e-09
4159 5.78730014566542e-09
4160 5.786677747463e-09
4161 5.7858826116447e-09
4162 5.78514950949627e-09
4163 5.78441348388081e-09
4164 5.78369992747241e-09
4165 5.783051521005e-09
4166 5.7823190553058e-09
4167 5.78159689806901e-09
4168 5.78087744507183e-09
4169 5.78020198521711e-09
4170 5.77942922774799e-09
4171 5.77873581757149e-09
4172 5.7780178208261e-09
4173 5.77732311300705e-09
4174 5.77661679615482e-09
4175 5.77590325680222e-09
4176 5.77515974091447e-09
4177 5.77446002533733e-09
4178 5.77373357713018e-09
4179 5.77301604776093e-09
4180 5.77237062950042e-09
4181 5.77160754483552e-09
4182 5.77093495386649e-09
4183 5.77017215930287e-09
4184 5.76948914511599e-09
4185 5.76877148010524e-09
4186 5.76808912465532e-09
4187 5.76732849032202e-09
4188 5.766650380018e-09
4189 5.76600388001169e-09
4190 5.76522970575921e-09
4191 5.76456880682641e-09
4192 5.76381980521568e-09
4193 5.76316500341956e-09
4194 5.76243694344614e-09
4195 5.7617431168111e-09
4196 5.76103812864603e-09
4197 5.76029501037079e-09
4198 5.75960631207795e-09
4199 5.75888611829833e-09
4200 5.7582432419212e-09
4201 5.75744233766162e-09
4202 5.75674694142103e-09
4203 5.7560624045494e-09
4204 5.75537470177967e-09
4205 5.75464103856616e-09
4206 5.75393679215497e-09
4207 5.75327403763703e-09
4208 5.75254536455683e-09
4209 5.7518559904296e-09
4210 5.75111663946026e-09
4211 5.75039112633846e-09
4212 5.74974792411087e-09
4213 5.74900713615212e-09
4214 5.74826442464871e-09
4215 5.74761236535326e-09
4216 5.74685779337891e-09
4217 5.74619478661831e-09
4218 5.74546551539157e-09
4219 5.74478539044909e-09
4220 5.74403792305556e-09
4221 5.74331021342689e-09
4222 5.74264403205293e-09
4223 5.74191838817462e-09
4224 5.74121129871819e-09
4225 5.74051788017338e-09
4226 5.73984307553066e-09
4227 5.73909433400355e-09
4228 5.73841767483041e-09
4229 5.73772558448704e-09
4230 5.73699091334368e-09
4231 5.73631645228723e-09
4232 5.73561373702625e-09
4233 5.73483124570451e-09
4234 5.73422253972999e-09
4235 5.73347863380702e-09
4236 5.73275878688884e-09
4237 5.73206761646239e-09
4238 5.73136346773695e-09
4239 5.73062972089589e-09
4240 5.72993157986479e-09
4241 5.72926414585395e-09
4242 5.72857036307273e-09
4243 5.72784498226175e-09
4244 5.72716890161196e-09
4245 5.72643989658894e-09
4246 5.72574832558015e-09
4247 5.72502241076578e-09
4248 5.72429861750434e-09
4249 5.72359746171813e-09
4250 5.7229239876222e-09
4251 5.72220346535535e-09
4252 5.72151586485103e-09
4253 5.7208325657393e-09
4254 5.72010568253289e-09
4255 5.71937228827091e-09
4256 5.71873001140033e-09
4257 5.71803693279194e-09
4258 5.71729631969331e-09
4259 5.71664368810565e-09
4260 5.71591405250371e-09
4261 5.71519605491178e-09
4262 5.71451844194604e-09
4263 5.71383580735829e-09
4264 5.7131342032779e-09
4265 5.71240941779627e-09
4266 5.71174301029764e-09
4267 5.71105730808041e-09
4268 5.7103291356414e-09
4269 5.70960356408023e-09
4270 5.70892983367544e-09
4271 5.70821226648921e-09
4272 5.70752516033945e-09
4273 5.70684786785347e-09
4274 5.70617191784917e-09
4275 5.70543483932595e-09
4276 5.70477176180251e-09
4277 5.70399710977942e-09
4278 5.70334844612885e-09
4279 5.70264209126536e-09
4280 5.70196820517954e-09
4281 5.7012568457776e-09
4282 5.70058167954912e-09
4283 5.69989293781881e-09
4284 5.69921644179294e-09
4285 5.69849409612355e-09
4286 5.69778960281264e-09
4287 5.69706832394268e-09
4288 5.69638793648797e-09
4289 5.69565860604471e-09
4290 5.69500540258117e-09
4291 5.69432770407274e-09
4292 5.69360757508852e-09
4293 5.69291289126417e-09
4294 5.69221800061914e-09
4295 5.69150336662827e-09
4296 5.69087332784324e-09
4297 5.69015058500544e-09
4298 5.68948415581583e-09
4299 5.68876888520919e-09
4300 5.68805564084285e-09
4301 5.68740436829307e-09
4302 5.68665198188989e-09
4303 5.68596582183056e-09
4304 5.68532034157798e-09
4305 5.68459814669076e-09
4306 5.68386775169627e-09
4307 5.68324319966118e-09
4308 5.68256838202885e-09
4309 5.68183661589505e-09
4310 5.68116774965488e-09
4311 5.6804743242822e-09
4312 5.67975881594906e-09
4313 5.67911109285169e-09
4314 5.67846453430887e-09
4315 5.67772308836256e-09
4316 5.67707851291399e-09
4317 5.67636283697881e-09
4318 5.67567552052506e-09
4319 5.67498269817002e-09
4320 5.67427121046793e-09
4321 5.67364615675081e-09
4322 5.67286269834155e-09
4323 5.67224066136407e-09
4324 5.67151349190054e-09
4325 5.67080747759796e-09
4326 5.67018815458842e-09
4327 5.66950122092569e-09
4328 5.66875147101076e-09
4329 5.66810828581121e-09
4330 5.6674084788072e-09
4331 5.66670270177316e-09
4332 5.66606009727577e-09
4333 5.66535030102622e-09
4334 5.66466233006824e-09
4335 5.66401188349663e-09
4336 5.66331306700585e-09
4337 5.66263165543979e-09
4338 5.66192056318526e-09
4339 5.66125157037967e-09
4340 5.66059501386218e-09
4341 5.6598330629154e-09
4342 5.6592029920588e-09
4343 5.65848336712971e-09
4344 5.65781998923542e-09
4345 5.65710123259788e-09
4346 5.6564450678781e-09
4347 5.65579539499317e-09
4348 5.65507207254956e-09
4349 5.65435932277369e-09
4350 5.65369859337195e-09
4351 5.65303300030517e-09
4352 5.65232623629675e-09
4353 5.65157807227989e-09
4354 5.65094686216794e-09
4355 5.65032068933213e-09
4356 5.64958889293088e-09
4357 5.64891133561507e-09
4358 5.64819942917849e-09
4359 5.64752960298398e-09
4360 5.64688974566407e-09
4361 5.64615589279671e-09
4362 5.64549155415706e-09
4363 5.64485802558024e-09
4364 5.64407175333848e-09
4365 5.64345627536433e-09
4366 5.64276478534631e-09
4367 5.64209406909988e-09
4368 5.64134161648577e-09
4369 5.64074254416624e-09
4370 5.64006353495627e-09
4371 5.63935309980113e-09
4372 5.63868363601117e-09
4373 5.63798807254323e-09
4374 5.63732331707034e-09
4375 5.63661133422266e-09
4376 5.63594355976194e-09
4377 5.63526579099027e-09
4378 5.63460444277797e-09
4379 5.633874153671e-09
4380 5.63327571216632e-09
4381 5.63259570753039e-09
4382 5.63186007418448e-09
4383 5.63122629877733e-09
4384 5.63056903762904e-09
4385 5.62981947928309e-09
4386 5.6292311781253e-09
4387 5.62850208246646e-09
4388 5.62780449560318e-09
4389 5.62712950982758e-09
4390 5.6264619730656e-09
4391 5.62576684506877e-09
4392 5.62511646975961e-09
4393 5.62444487185731e-09
4394 5.62376139588705e-09
4395 5.62307182093436e-09
4396 5.62239025560241e-09
4397 5.6217071279091e-09
4398 5.62102643919538e-09
4399 5.62036128964882e-09
4400 5.61966718740092e-09
4401 5.61897007991807e-09
4402 5.61828987449831e-09
4403 5.61763040403396e-09
4404 5.61693744707825e-09
4405 5.61629990440682e-09
4406 5.61557564794646e-09
4407 5.61491875832043e-09
4408 5.6142519635205e-09
4409 5.61358305026238e-09
4410 5.61288325894027e-09
4411 5.61222928616156e-09
4412 5.61152862060821e-09
4413 5.61089574965812e-09
4414 5.61020814375535e-09
4415 5.6095126433342e-09
4416 5.60882274973362e-09
4417 5.60819043993188e-09
4418 5.60748473897588e-09
4419 5.60682471190121e-09
4420 5.60616482080112e-09
4421 5.60546661076966e-09
4422 5.60480012989917e-09
4423 5.60414454979508e-09
4424 5.60344487743336e-09
4425 5.60285058186227e-09
4426 5.60211332266414e-09
4427 5.60144896212533e-09
4428 5.6007937457303e-09
4429 5.60008956648761e-09
4430 5.59944213902874e-09
4431 5.59875470740323e-09
4432 5.59808770705938e-09
4433 5.59737880584388e-09
4434 5.59673629321744e-09
4435 5.59606768717191e-09
4436 5.59542190538276e-09
4437 5.59467916554091e-09
4438 5.59411665042309e-09
4439 5.59340121659979e-09
4440 5.59269438842047e-09
4441 5.59209180779618e-09
4442 5.59135923312859e-09
4443 5.59072133576866e-09
4444 5.59002098683703e-09
4445 5.58935621579326e-09
4446 5.58866565525395e-09
4447 5.58806963163461e-09
4448 5.58736551564687e-09
4449 5.58669640850218e-09
4450 5.58600568510703e-09
4451 5.5853638626091e-09
4452 5.58465935512897e-09
4453 5.58403920764006e-09
4454 5.58334547352823e-09
4455 5.58271035919211e-09
4456 5.58200438004197e-09
4457 5.581351389089e-09
4458 5.58068753928054e-09
4459 5.58001171840905e-09
4460 5.57932886460777e-09
4461 5.57865878243746e-09
4462 5.5780106188591e-09
4463 5.57735809118831e-09
4464 5.5766785201361e-09
4465 5.57598838021867e-09
4466 5.57529649469757e-09
4467 5.57468193468358e-09
4468 5.57403454268246e-09
4469 5.57334257139663e-09
4470 5.57263656961182e-09
4471 5.57200814374492e-09
4472 5.57132739964494e-09
4473 5.57069982348724e-09
4474 5.57000195798574e-09
4475 5.56934756744398e-09
4476 5.56867874555722e-09
4477 5.56797440035006e-09
4478 5.56735277643106e-09
4479 5.56672259778568e-09
4480 5.56602964266184e-09
4481 5.56538451473854e-09
4482 5.56471430357419e-09
4483 5.56407803710413e-09
4484 5.56334643002365e-09
4485 5.56277140026373e-09
4486 5.56202711843312e-09
4487 5.5614427982853e-09
4488 5.56074718971455e-09
4489 5.56002464342786e-09
4490 5.55937407639706e-09
4491 5.55875066884926e-09
4492 5.55806554210225e-09
4493 5.55742331483089e-09
4494 5.55673733415585e-09
4495 5.55609886160324e-09
4496 5.55543111510626e-09
4497 5.55478315854285e-09
4498 5.55409642657989e-09
4499 5.55340990762709e-09
4500 5.55276420131923e-09
4501 5.55208629028636e-09
4502 5.55144918369666e-09
4503 5.55082237951865e-09
4504 5.55006944014114e-09
4505 5.54945206439339e-09
4506 5.54880533028268e-09
4507 5.5481651803635e-09
4508 5.54744337834256e-09
4509 5.5468144884302e-09
4510 5.54612194877957e-09
4511 5.54553087771248e-09
4512 5.54483276574147e-09
4513 5.54419099525749e-09
4514 5.54347933112709e-09
4515 5.54290390666901e-09
4516 5.54219695868274e-09
4517 5.54154911688864e-09
4518 5.54089907553457e-09
4519 5.54022773266438e-09
4520 5.53952024268112e-09
4521 5.53890375372401e-09
4522 5.53829750242485e-09
4523 5.53756797132265e-09
4524 5.53689667380508e-09
4525 5.5362799148001e-09
4526 5.53562717094114e-09
4527 5.53494992053261e-09
4528 5.53428316442395e-09
4529 5.53365381689153e-09
4530 5.53297095842731e-09
4531 5.53233221967098e-09
4532 5.53164884395385e-09
4533 5.53098048976242e-09
4534 5.53038270861839e-09
4535 5.52971699267768e-09
4536 5.52903663585125e-09
4537 5.52835051871592e-09
4538 5.52768282982563e-09
4539 5.52707305355449e-09
4540 5.52637194660421e-09
4541 5.52576129630999e-09
4542 5.52511872316253e-09
4543 5.52441214107802e-09
4544 5.52375472422095e-09
4545 5.52311721316312e-09
4546 5.52246816251656e-09
4547 5.52183334508183e-09
4548 5.52114918860036e-09
4549 5.52049342231187e-09
4550 5.51987053633296e-09
4551 5.51917914044797e-09
4552 5.51853878777431e-09
4553 5.51788595690161e-09
4554 5.51724046723989e-09
4555 5.51656323342919e-09
4556 5.51589951593157e-09
4557 5.51526875415542e-09
4558 5.51463236059258e-09
4559 5.51397128505104e-09
4560 5.51332894103973e-09
4561 5.51267191287175e-09
4562 5.51202447350574e-09
4563 5.51140865452659e-09
4564 5.51073675497671e-09
4565 5.51007416611793e-09
4566 5.50944395953656e-09
4567 5.50877320870669e-09
4568 5.50811456238864e-09
4569 5.50747437197408e-09
4570 5.50679521069131e-09
4571 5.50613170907654e-09
4572 5.50552496832168e-09
4573 5.50490587346297e-09
4574 5.50418366525307e-09
4575 5.50360237225322e-09
4576 5.50285890917046e-09
4577 5.50217121356167e-09
4578 5.50161061703414e-09
4579 5.50091628993832e-09
4580 5.50032140823886e-09
4581 5.49961494018814e-09
4582 5.49899896111483e-09
4583 5.49828077875403e-09
4584 5.49764835165723e-09
4585 5.49703487934261e-09
4586 5.49639214661446e-09
4587 5.4957245067544e-09
4588 5.49511079989129e-09
4589 5.49446802892983e-09
4590 5.49376983334515e-09
4591 5.49314643671916e-09
4592 5.49248986857209e-09
4593 5.491806252228e-09
4594 5.49118490651701e-09
4595 5.49054440511509e-09
4596 5.48990255215542e-09
4597 5.48922459563117e-09
4598 5.48858156060317e-09
4599 5.48789544793649e-09
4600 5.48728336403903e-09
4601 5.48662061113092e-09
4602 5.48599977422126e-09
4603 5.48539555579597e-09
4604 5.48472386363574e-09
4605 5.48410973860713e-09
4606 5.48335196205729e-09
4607 5.48277270928654e-09
4608 5.48211252367203e-09
4609 5.48149825729816e-09
4610 5.48084807457105e-09
4611 5.48019733383198e-09
4612 5.47952540842778e-09
4613 5.47883222370982e-09
4614 5.47823144057824e-09
4615 5.47757119263859e-09
4616 5.47696623613703e-09
4617 5.47631368252866e-09
4618 5.47568060312031e-09
4619 5.47499071949786e-09
4620 5.47439515928561e-09
4621 5.47374627883623e-09
4622 5.47305016994348e-09
4623 5.47242534440495e-09
4624 5.4717883339328e-09
4625 5.47118492853771e-09
4626 5.47049745870665e-09
4627 5.46982817746511e-09
4628 5.4692309334331e-09
4629 5.46854083724457e-09
4630 5.46795176298298e-09
4631 5.46724012898125e-09
4632 5.46664722173018e-09
4633 5.4660092747294e-09
4634 5.46534590957493e-09
4635 5.46470541552824e-09
4636 5.46408442832214e-09
4637 5.46343950880157e-09
4638 5.46281149053918e-09
4639 5.46215453327281e-09
4640 5.46149526262085e-09
4641 5.46085412088393e-09
4642 5.46018712344054e-09
4643 5.45957549076548e-09
4644 5.45893724221425e-09
4645 5.45825113368315e-09
4646 5.4576784521837e-09
4647 5.45700010710914e-09
4648 5.45638491468436e-09
4649 5.45571725210636e-09
4650 5.45513274891052e-09
4651 5.4543956469616e-09
4652 5.45377811955738e-09
4653 5.45316203640067e-09
4654 5.45253196038154e-09
4655 5.45190748989233e-09
4656 5.45126184259281e-09
4657 5.45060773163297e-09
4658 5.44998825512122e-09
4659 5.44936144297736e-09
4660 5.44868429055989e-09
4661 5.44804757433848e-09
4662 5.44740912500341e-09
4663 5.4467344097614e-09
4664 5.44613264603921e-09
4665 5.44547891799529e-09
4666 5.44481087480508e-09
4667 5.4441832850749e-09
4668 5.44355312542821e-09
4669 5.44292439561e-09
4670 5.44226782556168e-09
4671 5.44157559748126e-09
4672 5.44098024284068e-09
4673 5.44038858416029e-09
4674 5.43969922198184e-09
4675 5.43908688444006e-09
4676 5.4384117480627e-09
4677 5.43775691888571e-09
4678 5.43718179431274e-09
4679 5.43649685312564e-09
4680 5.43589081927753e-09
4681 5.43525579577153e-09
4682 5.43464910267299e-09
4683 5.43399002950196e-09
4684 5.43331701195748e-09
4685 5.43271126096645e-09
4686 5.43206002537322e-09
4687 5.43143659362255e-09
4688 5.43083648815723e-09
4689 5.43017843392113e-09
4690 5.42953717094785e-09
4691 5.42887873793141e-09
4692 5.42825636044852e-09
4693 5.42762388849871e-09
4694 5.42701697553438e-09
4695 5.42638170007714e-09
4696 5.4257450043671e-09
4697 5.4250848819104e-09
4698 5.42442830461787e-09
4699 5.42380952697763e-09
4700 5.42317806057069e-09
4701 5.42244080294074e-09
4702 5.42190927334263e-09
4703 5.42125247827985e-09
4704 5.42063970192241e-09
4705 5.41998072303707e-09
4706 5.4193514969908e-09
4707 5.41873267166648e-09
4708 5.41806380623122e-09
4709 5.41748532069397e-09
4710 5.4168590046394e-09
4711 5.41616405988488e-09
4712 5.41557022930295e-09
4713 5.41488518762678e-09
4714 5.41427843978037e-09
4715 5.41364010764322e-09
4716 5.41306117853024e-09
4717 5.41234767452436e-09
4718 5.4117153926031e-09
4719 5.41109834784059e-09
4720 5.4104827568735e-09
4721 5.40979021766697e-09
4722 5.40916805691349e-09
4723 5.40860144894462e-09
4724 5.4079794938322e-09
4725 5.40730855176641e-09
4726 5.40666233499187e-09
4727 5.40603655672933e-09
4728 5.40538246505962e-09
4729 5.40480797864284e-09
4730 5.40417221532585e-09
4731 5.4035524281737e-09
4732 5.40289258997573e-09
4733 5.40225685026485e-09
4734 5.40164667980902e-09
4735 5.4010256028969e-09
4736 5.40040711205503e-09
4737 5.39975782168356e-09
4738 5.39913962262217e-09
4739 5.39848931771503e-09
4740 5.39793109967379e-09
4741 5.39724130342589e-09
4742 5.39668049540087e-09
4743 5.39600279426955e-09
4744 5.39536626073534e-09
4745 5.39470748223136e-09
4746 5.39415196162962e-09
4747 5.39347717176675e-09
4748 5.39287355258433e-09
4749 5.39227636536799e-09
4750 5.39163261883968e-09
4751 5.39104762393994e-09
4752 5.3903815261519e-09
4753 5.38976456651574e-09
4754 5.38906900277025e-09
4755 5.38846490581724e-09
4756 5.38787079733261e-09
4757 5.3872686540668e-09
4758 5.38667154283135e-09
4759 5.38601068196531e-09
4760 5.38533299500321e-09
4761 5.38474865258154e-09
4762 5.38411407924322e-09
4763 5.38349702594587e-09
4764 5.3828712826276e-09
4765 5.38221509981118e-09
4766 5.38159692475837e-09
4767 5.3809849236558e-09
4768 5.38035638858458e-09
4769 5.3797352455448e-09
4770 5.3790904230161e-09
4771 5.37847510620471e-09
4772 5.37786223288317e-09
4773 5.37724023040587e-09
4774 5.37660440227961e-09
4775 5.37601636099727e-09
4776 5.3753431921294e-09
4777 5.37475348476313e-09
4778 5.37411846975033e-09
4779 5.37346117686355e-09
4780 5.37287660275221e-09
4781 5.37227810144814e-09
4782 5.37163624751702e-09
4783 5.37099861207257e-09
4784 5.37040143261391e-09
4785 5.36973210300828e-09
4786 5.36916477181237e-09
4787 5.36852260020482e-09
4788 5.36788077480643e-09
4789 5.36727532195191e-09
4790 5.36662550737477e-09
4791 5.36605540225887e-09
4792 5.36536902449869e-09
4793 5.36477326329443e-09
4794 5.3641814385802e-09
4795 5.36353040120341e-09
4796 5.36289353700314e-09
4797 5.36230903756818e-09
4798 5.36165500697461e-09
4799 5.36107581458611e-09
4800 5.36042336896081e-09
4801 5.3598122665699e-09
4802 5.35916835467387e-09
4803 5.3585014587465e-09
4804 5.35794084872976e-09
4805 5.35727584365098e-09
4806 5.35665740054869e-09
4807 5.35608682042388e-09
4808 5.35546847342527e-09
4809 5.35481355143363e-09
4810 5.35420436388601e-09
4811 5.35355759935519e-09
4812 5.3529535491148e-09
4813 5.35235696511038e-09
4814 5.35167314302809e-09
4815 5.35109166389935e-09
4816 5.35044574152821e-09
4817 5.34986724727571e-09
4818 5.34922400281801e-09
4819 5.34858877584976e-09
4820 5.34799510444606e-09
4821 5.34738293600512e-09
4822 5.34672260643632e-09
4823 5.34612276302526e-09
4824 5.3455275466352e-09
4825 5.34491961250161e-09
4826 5.34424596804195e-09
4827 5.34364985310676e-09
4828 5.34301064256115e-09
4829 5.34242081787206e-09
4830 5.34179815310509e-09
4831 5.34117288092384e-09
4832 5.34057596582316e-09
4833 5.33994128280868e-09
4834 5.33928958371122e-09
4835 5.33868176964825e-09
4836 5.33808771355226e-09
4837 5.33746418804326e-09
4838 5.33682360846777e-09
4839 5.33619245761396e-09
4840 5.33560186140314e-09
4841 5.33502237100303e-09
4842 5.33438746899706e-09
4843 5.33374175092083e-09
4844 5.33315949929891e-09
4845 5.33254823936735e-09
4846 5.33192020092665e-09
4847 5.33132014857163e-09
4848 5.33073102426673e-09
4849 5.3300998761191e-09
4850 5.32950200043958e-09
4851 5.32885664131233e-09
4852 5.32821208207301e-09
4853 5.32763103779088e-09
4854 5.32700955417631e-09
4855 5.32637626078636e-09
4856 5.32580683429429e-09
4857 5.32517155622803e-09
4858 5.32454462663645e-09
4859 5.32391718309488e-09
4860 5.32335132960582e-09
4861 5.32269335112856e-09
4862 5.3220849157154e-09
4863 5.3214412848307e-09
4864 5.320863060293e-09
4865 5.32020549071088e-09
4866 5.31964167833909e-09
4867 5.31901070877083e-09
4868 5.31845890278104e-09
4869 5.31784423249415e-09
4870 5.31716780008407e-09
4871 5.31654365971967e-09
4872 5.31593545470554e-09
4873 5.31533602975143e-09
4874 5.31471565594355e-09
4875 5.31408454723659e-09
4876 5.31350760397953e-09
4877 5.31287737122466e-09
4878 5.31223105011691e-09
4879 5.311663459448e-09
4880 5.31105887043026e-09
4881 5.31038888923474e-09
4882 5.30981275341513e-09
4883 5.309167452755e-09
4884 5.30856406313895e-09
4885 5.30801339253262e-09
4886 5.30733663038629e-09
4887 5.30672207757488e-09
4888 5.30611107647794e-09
4889 5.30548185932733e-09
4890 5.30491577271919e-09
4891 5.30432452131024e-09
4892 5.303686082897e-09
4893 5.3030726563652e-09
4894 5.30243260507546e-09
4895 5.30188879462046e-09
4896 5.3012248731471e-09
4897 5.3006085423829e-09
4898 5.29998925193065e-09
4899 5.29942686348928e-09
4900 5.29875882374076e-09
4901 5.29819874602044e-09
4902 5.29759131027985e-09
4903 5.29695584663981e-09
4904 5.29638603750937e-09
4905 5.29569291297938e-09
4906 5.29515772411393e-09
4907 5.294568530545e-09
4908 5.29390532887086e-09
4909 5.2932791220206e-09
4910 5.29270164299767e-09
4911 5.29208367337775e-09
4912 5.29150167320747e-09
4913 5.29087038801668e-09
4914 5.2902494509649e-09
4915 5.28964641101359e-09
4916 5.28898860598093e-09
4917 5.28841168138949e-09
4918 5.28776434005618e-09
4919 5.28715929491719e-09
4920 5.28654901608971e-09
4921 5.28593830964597e-09
4922 5.28532155648354e-09
4923 5.28471338472059e-09
4924 5.28409564966303e-09
4925 5.28354039876222e-09
4926 5.2828965455276e-09
4927 5.28229174451278e-09
4928 5.28168344782198e-09
4929 5.28104304527155e-09
4930 5.28045621955853e-09
4931 5.27982257685078e-09
4932 5.27925411593355e-09
4933 5.27867302753393e-09
4934 5.27803949437744e-09
4935 5.27741912544066e-09
4936 5.27678100605e-09
4937 5.2762183808952e-09
4938 5.27562201151077e-09
4939 5.27497254035336e-09
4940 5.27439789159423e-09
4941 5.2737449820206e-09
4942 5.27319372517487e-09
4943 5.27256756072125e-09
4944 5.27192649292518e-09
4945 5.27132971499256e-09
4946 5.27070407931041e-09
4947 5.2701188940768e-09
4948 5.26946506723691e-09
4949 5.26886051689657e-09
4950 5.26823203678139e-09
4951 5.26764843507277e-09
4952 5.26704418299384e-09
4953 5.26643017277617e-09
4954 5.265814473146e-09
4955 5.26522465818524e-09
4956 5.26464038023977e-09
4957 5.26397849903715e-09
4958 5.26340430757888e-09
4959 5.262799276054e-09
4960 5.26219009996942e-09
4961 5.26157364384716e-09
4962 5.26101388624578e-09
4963 5.26037837450533e-09
4964 5.25978762382084e-09
4965 5.25914370366753e-09
4966 5.25859165735609e-09
4967 5.25798449335646e-09
4968 5.25742256966832e-09
4969 5.25670904086284e-09
4970 5.25617419344648e-09
4971 5.25555465624716e-09
4972 5.25494506362079e-09
4973 5.25430822721773e-09
4974 5.25375512115678e-09
4975 5.25311509916304e-09
4976 5.25249762403646e-09
4977 5.25190342907933e-09
4978 5.25132586130794e-09
4979 5.25067681506064e-09
4980 5.25010306781648e-09
4981 5.24953017659591e-09
4982 5.248888186829e-09
4983 5.24831754505906e-09
4984 5.24768378347418e-09
4985 5.24712860038024e-09
4986 5.24651431237089e-09
4987 5.24590490996735e-09
4988 5.24530570968074e-09
4989 5.24473443311302e-09
4990 5.2440983713542e-09
4991 5.2434746589114e-09
4992 5.24289870326711e-09
4993 5.24225277501178e-09
4994 5.24172283931923e-09
4995 5.24110797589517e-09
4996 5.24049778838354e-09
4997 5.23991623112285e-09
4998 5.2393101044601e-09
4999 5.2386673640159e-09
};
\addlegendentry{Train}
\addplot [semithick, black]
table {%
0 0.00123841257300228
1 0.000289466552203521
2 0.000237459084019065
3 0.000231174999498762
4 0.000218906920053996
5 0.000181923998752609
6 7.93823273852468e-05
7 2.11141214094823e-05
8 1.80101160367485e-05
9 1.77777565113502e-05
10 1.76114208443323e-05
11 1.74742326635169e-05
12 1.73643493326381e-05
13 1.72833279066253e-05
14 1.7221185771632e-05
15 1.71623432834167e-05
16 1.70983585121576e-05
17 1.70253661053721e-05
18 1.69403283507563e-05
19 1.68450897035655e-05
20 1.67248927027686e-05
21 1.65652054420207e-05
22 1.63404110935517e-05
23 1.59976243594429e-05
24 1.5440991774085e-05
25 1.45002368299174e-05
26 1.28874044094118e-05
27 1.02149424492382e-05
28 6.57677901472198e-06
29 3.70723364540027e-06
30 2.53792768489802e-06
31 2.18933200812899e-06
32 2.08510164156905e-06
33 2.03365721063165e-06
34 1.99685564439278e-06
35 1.96387077266991e-06
36 1.93222649613745e-06
37 1.90050513992901e-06
38 1.8678206288314e-06
39 1.8335252889301e-06
40 1.79707421921194e-06
41 1.75799505086616e-06
42 1.71574765772675e-06
43 1.6698410263416e-06
44 1.6201006474148e-06
45 1.56655414684792e-06
46 1.50876144289214e-06
47 1.44507089316903e-06
48 1.37333290695096e-06
49 1.29354111777502e-06
50 1.20605261599849e-06
51 1.11351243958779e-06
52 1.01710554645251e-06
53 9.23336870073399e-07
54 8.3821379348592e-07
55 7.6335828680385e-07
56 6.97083635259332e-07
57 6.39880738617649e-07
58 5.9618645309456e-07
59 5.6386102187389e-07
60 5.41307542789582e-07
61 5.25914401805494e-07
62 5.12792894369341e-07
63 5.01673355302046e-07
64 4.92857793688017e-07
65 4.84957922708418e-07
66 4.78733852560254e-07
67 4.7331113250948e-07
68 4.67897848466237e-07
69 4.63399828731781e-07
70 4.59486670933984e-07
71 4.55968631740689e-07
72 4.52807114470488e-07
73 4.49909151711836e-07
74 4.47254905111549e-07
75 4.44763827545103e-07
76 4.42443422343786e-07
77 4.40233066001383e-07
78 4.38128154200967e-07
79 4.36120132008e-07
80 4.34170743801587e-07
81 4.32290107710287e-07
82 4.30465405543146e-07
83 4.2874222572209e-07
84 4.27082994747252e-07
85 4.25037626428093e-07
86 4.23276588890076e-07
87 4.21510037540429e-07
88 4.19670641349512e-07
89 4.17728102775072e-07
90 4.15852781543435e-07
91 4.1418897467338e-07
92 4.12570500429865e-07
93 4.10953504115241e-07
94 4.09513262411565e-07
95 4.08050738087695e-07
96 4.0622873598295e-07
97 4.04738159431872e-07
98 4.03309371677096e-07
99 4.01870806854276e-07
100 4.0048095684142e-07
101 3.9909133420224e-07
102 3.97724647882569e-07
103 3.96377600964115e-07
104 3.95046441781233e-07
105 3.93728413428107e-07
106 3.92425079098757e-07
107 3.91163581525689e-07
108 3.89834639236142e-07
109 3.88628365044497e-07
110 3.87396084988723e-07
111 3.86114777484181e-07
112 3.84910691764162e-07
113 3.83698193218152e-07
114 3.82530203069109e-07
115 3.81392055714969e-07
116 3.80263401211778e-07
117 3.79166380071183e-07
118 3.78147404944684e-07
119 3.77138547946743e-07
120 3.76077764485672e-07
121 3.75121629758723e-07
122 3.74176806872129e-07
123 3.73256852981285e-07
124 3.72286734773297e-07
125 3.71325711512327e-07
126 3.70353205880747e-07
127 3.69426288671093e-07
128 3.68474502465688e-07
129 3.67563501413315e-07
130 3.6663863056674e-07
131 3.65752100606187e-07
132 3.64910704320209e-07
133 3.64000101171769e-07
134 3.63089213806234e-07
135 3.62163575573504e-07
136 3.61202154408602e-07
137 3.60338191285337e-07
138 3.5954016652795e-07
139 3.58638260422595e-07
140 3.57969611286535e-07
141 3.56984600102805e-07
142 3.56155851477524e-07
143 3.55481517999579e-07
144 3.54525269585793e-07
145 3.53743388359362e-07
146 3.52939593994961e-07
147 3.5217971117163e-07
148 3.51368981910127e-07
149 3.50618677202874e-07
150 3.49820794554034e-07
151 3.49069608773789e-07
152 3.48276643080681e-07
153 3.475521737073e-07
154 3.46761083847014e-07
155 3.46033289133629e-07
156 3.45202835205782e-07
157 3.44443861877153e-07
158 3.43614033226913e-07
159 3.42848977652466e-07
160 3.42031142963606e-07
161 3.41273761250704e-07
162 3.40478464977423e-07
163 3.39741717425568e-07
164 3.38944886379977e-07
165 3.38221923357196e-07
166 3.37441491637946e-07
167 3.36716226456701e-07
168 3.35980331556129e-07
169 3.35335045065221e-07
170 3.34893940134862e-07
171 3.34127975065712e-07
172 3.33359054138782e-07
173 3.32615570641792e-07
174 3.31844631773492e-07
175 3.31086226879052e-07
176 3.3031886914614e-07
177 3.29550118749466e-07
178 3.28780828340314e-07
179 3.28007672578678e-07
180 3.27248187659279e-07
181 3.26462213706691e-07
182 3.25697698144722e-07
183 3.24907546200848e-07
184 3.2414547490589e-07
185 3.23349325981326e-07
186 3.22592313750647e-07
187 3.2178533615479e-07
188 3.21029176575394e-07
189 3.20232715012025e-07
190 3.19473883791943e-07
191 3.18679610700201e-07
192 3.17929789162008e-07
193 3.17134549732145e-07
194 3.16410876166628e-07
195 3.15643148951494e-07
196 3.14969696546541e-07
197 3.14281550117812e-07
198 3.13658574668807e-07
199 3.12891245357605e-07
200 3.12124626589139e-07
201 3.11289937826587e-07
202 3.10517464185978e-07
203 3.09679592191969e-07
204 3.08894527734083e-07
205 3.0805486517238e-07
206 3.07274916622191e-07
207 3.06433520336213e-07
208 3.05649365373029e-07
209 3.04817319829453e-07
210 3.04027452102673e-07
211 3.03200977214146e-07
212 3.02419863373871e-07
213 3.01600294960735e-07
214 3.00813724152249e-07
215 3.00009389775369e-07
216 2.99227082223297e-07
217 2.98434940759762e-07
218 2.97673892646344e-07
219 2.96900736884709e-07
220 2.96179791803297e-07
221 2.9544514745794e-07
222 2.94744893380994e-07
223 2.94038756010195e-07
224 2.9334194096009e-07
225 2.92646575417166e-07
226 2.9195777528912e-07
227 2.91256839091147e-07
228 2.90562724103438e-07
229 2.89861247892986e-07
230 2.89170941414341e-07
231 2.88468811504572e-07
232 2.87764436279758e-07
233 2.87003047105827e-07
234 2.86181688124998e-07
235 2.85321135606864e-07
236 2.84464050537281e-07
237 2.83607192841373e-07
238 2.82750505675722e-07
239 2.81898991261187e-07
240 2.81047675798618e-07
241 2.80204005775886e-07
242 2.79367412758802e-07
243 2.78522037433504e-07
244 2.77682346450092e-07
245 2.76852006209083e-07
246 2.76031386192699e-07
247 2.75213466238711e-07
248 2.74394068355832e-07
249 2.73569014552777e-07
250 2.72696695446939e-07
251 2.71819772024173e-07
252 2.71018677722168e-07
253 2.70176855110549e-07
254 2.69364193172805e-07
255 2.68519102064602e-07
256 2.67719855173709e-07
257 2.66940475057709e-07
258 2.66125709913467e-07
259 2.65152976908212e-07
260 2.6431825972395e-07
261 2.63510742115614e-07
262 2.62680316609476e-07
263 2.6191116830887e-07
264 2.61150660207932e-07
265 2.60392170048362e-07
266 2.59644707512052e-07
267 2.58904861993869e-07
268 2.58194177149562e-07
269 2.57544286341727e-07
270 2.56797790143537e-07
271 2.56186240221723e-07
272 2.55419564609838e-07
273 2.54842206004469e-07
274 2.54028662993733e-07
275 2.53147504736262e-07
276 2.52298605118995e-07
277 2.51673412776654e-07
278 2.51005218387945e-07
279 2.50304964310999e-07
280 2.49601896484819e-07
281 2.49001033125751e-07
282 2.48359839361001e-07
283 2.47780434392553e-07
284 2.47088593141598e-07
285 2.46494124667151e-07
286 2.45765647832741e-07
287 2.45162794954012e-07
288 2.44506480839846e-07
289 2.43939638266966e-07
290 2.43211218275974e-07
291 2.42436840380833e-07
292 2.41584785953819e-07
293 2.41045484017377e-07
294 2.40350175317872e-07
295 2.39575967952987e-07
296 2.38931818330457e-07
297 2.38516918216192e-07
298 2.37661581081738e-07
299 2.36649526641486e-07
300 2.3611991650796e-07
301 2.35179456353762e-07
302 2.34385268527149e-07
303 2.33973011631861e-07
304 2.33185318165852e-07
305 2.32854930004578e-07
306 2.32331231586613e-07
307 2.3189527098566e-07
308 2.30899061648415e-07
309 2.30609430218465e-07
310 2.29786323302505e-07
311 2.29530314754811e-07
312 2.28693480153197e-07
313 2.28471705554512e-07
314 2.27562338750431e-07
315 2.27380724027171e-07
316 2.26411373205337e-07
317 2.26314483597889e-07
318 2.25176989943066e-07
319 2.25165294409635e-07
320 2.23925354703169e-07
321 2.24054403474838e-07
322 2.22721183718022e-07
323 2.23060609982895e-07
324 2.21480021878051e-07
325 2.21944560507836e-07
326 2.20401361161748e-07
327 2.20871754663676e-07
328 2.19132374468245e-07
329 2.1990628340518e-07
330 2.18667736362477e-07
331 2.1919700543549e-07
332 2.17410686786934e-07
333 2.18035737020728e-07
334 2.16388528428979e-07
335 2.16930715168928e-07
336 2.15514944557071e-07
337 2.15852324458865e-07
338 2.1426259877444e-07
339 2.14751310068095e-07
340 2.13569279594594e-07
341 2.13726877973386e-07
342 2.11990183629496e-07
343 2.12948677358327e-07
344 2.11138470262995e-07
345 2.11579703091047e-07
346 2.10031657843501e-07
347 2.10638503972405e-07
348 2.08661006695365e-07
349 2.09860417044183e-07
350 2.07541276608936e-07
351 2.08787398037202e-07
352 2.06459674245707e-07
353 2.07089158266172e-07
354 2.05769353556207e-07
355 2.06108211386891e-07
356 2.04360489419741e-07
357 2.04895258093529e-07
358 2.03302292334229e-07
359 2.03789269903609e-07
360 2.02219354150657e-07
361 2.02665958681791e-07
362 2.01107127395517e-07
363 2.01553561396395e-07
364 2.00058181576424e-07
365 2.00446947928867e-07
366 1.98991287447825e-07
367 1.99343091367155e-07
368 1.9788781457919e-07
369 1.98230452497228e-07
370 1.96962716358939e-07
371 1.97173761762315e-07
372 1.95663460544893e-07
373 1.96277213149187e-07
374 1.94603330783139e-07
375 1.95312850337359e-07
376 1.93559245076358e-07
377 1.94207956383252e-07
378 1.92494809425625e-07
379 1.93122048131045e-07
380 1.91429876394977e-07
381 1.9207966772683e-07
382 1.90602804650553e-07
383 1.91033237229021e-07
384 1.89433436048603e-07
385 1.8996024664375e-07
386 1.88321791938506e-07
387 1.88878914286761e-07
388 1.8724577444118e-07
389 1.87746735491601e-07
390 1.86107413924219e-07
391 1.86630089160644e-07
392 1.85024418897228e-07
393 1.85532002205946e-07
394 1.83937359565789e-07
395 1.84425758220641e-07
396 1.82853142405293e-07
397 1.83312621970799e-07
398 1.81769408413857e-07
399 1.82186155939235e-07
400 1.80671150928902e-07
401 1.81063541049298e-07
402 1.79593072857642e-07
403 1.79925876864218e-07
404 1.78492726377044e-07
405 1.78771358605445e-07
406 1.77382233346179e-07
407 1.77601819473239e-07
408 1.7627073134463e-07
409 1.76415468899904e-07
410 1.75149210690506e-07
411 1.75218431763824e-07
412 1.74026240529201e-07
413 1.74010921227818e-07
414 1.7289411857746e-07
415 1.72795267872061e-07
416 1.71759012346229e-07
417 1.7156757792236e-07
418 1.70613816408149e-07
419 1.7034034272001e-07
420 1.69466886745795e-07
421 1.69106527891927e-07
422 1.68303600389663e-07
423 1.67881466950348e-07
424 1.6714287198738e-07
425 1.66655070188426e-07
426 1.65971911769702e-07
427 1.65436006227537e-07
428 1.64790861845177e-07
429 1.64228126209309e-07
430 1.63610764047917e-07
431 1.63027110033909e-07
432 1.62419723892526e-07
433 1.61826136491072e-07
434 1.61227575290468e-07
435 1.60629383572086e-07
436 1.60024342221732e-07
437 1.59419101919411e-07
438 1.5881641957094e-07
439 1.58205168077075e-07
440 1.57599586714241e-07
441 1.56985421995159e-07
442 1.56375335791381e-07
443 1.55754321440327e-07
444 1.55140085666972e-07
445 1.54523021933528e-07
446 1.53904508692904e-07
447 1.53291679794165e-07
448 1.52668306441228e-07
449 1.52052123780777e-07
450 1.51438015905114e-07
451 1.50828952882875e-07
452 1.50212571270458e-07
453 1.49604886701127e-07
454 1.48998367421882e-07
455 1.4839469031358e-07
456 1.47790188975705e-07
457 1.4718840191108e-07
458 1.46586003779703e-07
459 1.45990540545426e-07
460 1.45389805084051e-07
461 1.44786952205322e-07
462 1.44185634098903e-07
463 1.43573188893242e-07
464 1.42958171522878e-07
465 1.42335522923531e-07
466 1.41709719514438e-07
467 1.41086147209535e-07
468 1.40475393095585e-07
469 1.39867594839416e-07
470 1.3927707698258e-07
471 1.38699917329177e-07
472 1.38127589366377e-07
473 1.37567553792906e-07
474 1.37006082923108e-07
475 1.36454460175628e-07
476 1.35905622755672e-07
477 1.35362185460508e-07
478 1.34834252207838e-07
479 1.34311406441157e-07
480 1.33801677293377e-07
481 1.33301853111334e-07
482 1.32810654918103e-07
483 1.32328054291975e-07
484 1.31849432705167e-07
485 1.31374136458362e-07
486 1.30905107198487e-07
487 1.30439971712804e-07
488 1.29978019458576e-07
489 1.29535962400951e-07
490 1.29112038393941e-07
491 1.2865127985151e-07
492 1.28209649119526e-07
493 1.27759932411209e-07
494 1.27311253095286e-07
495 1.26861124272182e-07
496 1.26405012679243e-07
497 1.25936551853556e-07
498 1.25435349218606e-07
499 1.24947021618027e-07
500 1.24479612395589e-07
501 1.24003975088272e-07
502 1.23528351991808e-07
503 1.2305051200201e-07
504 1.22575883665377e-07
505 1.22102704835925e-07
506 1.21626626992111e-07
507 1.21151558118981e-07
508 1.20677213999443e-07
509 1.20196261832461e-07
510 1.1971832236668e-07
511 1.19243054541585e-07
512 1.18763814782596e-07
513 1.18281064942494e-07
514 1.17802123611455e-07
515 1.17319004289129e-07
516 1.16835536800863e-07
517 1.16358805257732e-07
518 1.15878201256692e-07
519 1.15393852695433e-07
520 1.14914179505377e-07
521 1.14437028742032e-07
522 1.13956957648043e-07
523 1.1348566175684e-07
524 1.13005341972894e-07
525 1.12532319462844e-07
526 1.12060668300273e-07
527 1.11585670481418e-07
528 1.11117245182868e-07
529 1.10646794837521e-07
530 1.10185858659406e-07
531 1.09709858975293e-07
532 1.0927029592267e-07
533 1.08767871154214e-07
534 1.08350491245801e-07
535 1.07851072073117e-07
536 1.07442915009415e-07
537 1.06931018706291e-07
538 1.06532787924607e-07
539 1.06028409163628e-07
540 1.05633063185451e-07
541 1.05128208360838e-07
542 1.04742241546774e-07
543 1.04235631681604e-07
544 1.03855690269938e-07
545 1.03342941315532e-07
546 1.02972833815329e-07
547 1.02456446882115e-07
548 1.02106746169284e-07
549 1.01573917277165e-07
550 1.0122548843583e-07
551 1.00680665582331e-07
552 1.00356366772303e-07
553 9.97892328769012e-08
554 9.93497550894062e-08
555 9.90453372651245e-08
556 9.84282948479631e-08
557 9.79672662992925e-08
558 9.76764269466912e-08
559 9.70553628576454e-08
560 9.66542117453173e-08
561 9.60607735578378e-08
562 9.56306038801813e-08
563 9.518420540644e-08
564 9.46035640936316e-08
565 9.41253546216103e-08
566 9.35986719241555e-08
567 9.31504047230192e-08
568 9.25968279830158e-08
569 9.21299800893394e-08
570 9.16295661568256e-08
571 9.1124405798837e-08
572 9.06298325276111e-08
573 9.00914258750163e-08
574 8.96233842695437e-08
575 8.90711078227469e-08
576 8.86322837345688e-08
577 8.80859545304702e-08
578 8.76744223887727e-08
579 8.71177832095782e-08
580 8.66688978362617e-08
581 8.61226752135735e-08
582 8.5792628112813e-08
583 8.51596126949516e-08
584 8.47691694616515e-08
585 8.41827798581107e-08
586 8.38129992075665e-08
587 8.32188504773512e-08
588 8.28400175123534e-08
589 8.2266510048612e-08
590 8.18859717810483e-08
591 8.13299223523245e-08
592 8.0947081926297e-08
593 8.04203637017054e-08
594 8.00384754029437e-08
595 7.95294567978999e-08
596 7.91615590856054e-08
597 7.86684424269879e-08
598 7.83147982019727e-08
599 7.78406672452547e-08
600 7.74968142991384e-08
601 7.7040638757353e-08
602 7.67114300970206e-08
603 7.6276826632693e-08
604 7.59548228757012e-08
605 7.55317941525391e-08
606 7.52119930780282e-08
607 7.48171018472021e-08
608 7.4482194634129e-08
609 7.40578229851963e-08
610 7.36803826839605e-08
611 7.32565865746437e-08
612 7.2880617096871e-08
613 7.24587820855049e-08
614 7.20924120400923e-08
615 7.16710175652224e-08
616 7.13107795036194e-08
617 7.08883689526374e-08
618 7.05390590383104e-08
619 7.01241802403274e-08
620 6.97868287602432e-08
621 6.93760142667088e-08
622 6.90610875153652e-08
623 6.86556234086311e-08
624 6.83334633322374e-08
625 6.78724987324131e-08
626 6.75339606459602e-08
627 6.71142359465193e-08
628 6.68249739987914e-08
629 6.64158150698313e-08
630 6.61406858171176e-08
631 6.57388810054726e-08
632 6.54759872986688e-08
633 6.50842721938716e-08
634 6.48320082063947e-08
635 6.44472351041259e-08
636 6.41961861447271e-08
637 6.38269312958073e-08
638 6.35799537462844e-08
639 6.32205114925455e-08
640 6.29732994639198e-08
641 6.26384206725561e-08
642 6.23929210519236e-08
643 6.20930364902961e-08
644 6.18717947986624e-08
645 6.16089366189954e-08
646 6.14328214965099e-08
647 6.11963386631942e-08
648 6.10056574146256e-08
649 6.07278849429349e-08
650 6.05086611926708e-08
651 6.02154841544689e-08
652 5.99965872538633e-08
653 5.97032396854047e-08
654 5.94874727255501e-08
655 5.92032058932546e-08
656 5.89932618311195e-08
657 5.87172443999862e-08
658 5.85152868382011e-08
659 5.82463250964338e-08
660 5.80507624192705e-08
661 5.77896912545839e-08
662 5.76016425668513e-08
663 5.73476661713812e-08
664 5.71565657025985e-08
665 5.69163596253475e-08
666 5.67297924192189e-08
667 5.64975515260357e-08
668 5.63127429131782e-08
669 5.60865593968174e-08
670 5.59058364046905e-08
671 5.56812516094851e-08
672 5.55156631776299e-08
673 5.52868897329972e-08
674 5.5166395895867e-08
675 5.48973062564073e-08
676 5.47895844249524e-08
677 5.45091367598616e-08
678 5.44108367250828e-08
679 5.41255396058204e-08
680 5.40259748049721e-08
681 5.3747012174199e-08
682 5.3643859132535e-08
683 5.33687583015308e-08
684 5.3262748878069e-08
685 5.30012229660315e-08
686 5.28945065525477e-08
687 5.26361034758338e-08
688 5.25281080854256e-08
689 5.22765795096802e-08
690 5.21666336794624e-08
691 5.19226901474212e-08
692 5.18179632535976e-08
693 5.15734441819404e-08
694 5.14686995245484e-08
695 5.12348385939276e-08
696 5.11269071523657e-08
697 5.09036439666488e-08
698 5.07888948675372e-08
699 5.05894490743231e-08
700 5.04579880100664e-08
701 5.02642834021572e-08
702 5.01257702012481e-08
703 4.99450116819844e-08
704 4.98027006301527e-08
705 4.96232495095228e-08
706 4.94766965175586e-08
707 4.92915752658973e-08
708 4.91525540269322e-08
709 4.89742753018163e-08
710 4.88442672974543e-08
711 4.86694808898847e-08
712 4.85388760296246e-08
713 4.83631765746395e-08
714 4.8236437066862e-08
715 4.80666351165837e-08
716 4.79390998009421e-08
717 4.77661075137803e-08
718 4.76438266616697e-08
719 4.74748738099606e-08
720 4.73522803190463e-08
721 4.71850363226167e-08
722 4.70668943819419e-08
723 4.69049865614579e-08
724 4.67860559183464e-08
725 4.66245602126492e-08
726 4.6509143203366e-08
727 4.6351825488955e-08
728 4.62377052201646e-08
729 4.60821425463109e-08
730 4.59742039993216e-08
731 4.5816584304248e-08
732 4.57115092444838e-08
733 4.55567921164857e-08
734 4.54576145614283e-08
735 4.5301572271228e-08
736 4.52046116095062e-08
737 4.50494042070204e-08
738 4.49567316707089e-08
739 4.48028494304253e-08
740 4.47120527269362e-08
741 4.45583872021871e-08
742 4.44751790951159e-08
743 4.43218937107304e-08
744 4.42398437883185e-08
745 4.40879119878446e-08
746 4.40088676612049e-08
747 4.38602398844523e-08
748 4.37852101242697e-08
749 4.36360210187559e-08
750 4.35591047676098e-08
751 4.34169322716116e-08
752 4.33412488121121e-08
753 4.32057163379795e-08
754 4.31249098653552e-08
755 4.30010160812344e-08
756 4.29119637601616e-08
757 4.28008668507118e-08
758 4.27015578452483e-08
759 4.26025934530117e-08
760 4.24985451275006e-08
761 4.24098587359367e-08
762 4.22950918732568e-08
763 4.22210888473273e-08
764 4.2106790942853e-08
765 4.20365751097052e-08
766 4.19164294385155e-08
767 4.18562713377924e-08
768 4.17366656790819e-08
769 4.16763370481021e-08
770 4.15615701854222e-08
771 4.15035721346158e-08
772 4.13890894890301e-08
773 4.13342355898294e-08
774 4.12260376947415e-08
775 4.11767935304397e-08
776 4.10769906977748e-08
777 4.10399501049596e-08
778 4.09709173254669e-08
779 4.11461336113916e-08
780 4.28885158498815e-08
781 4.28764579396557e-08
782 4.28046185163566e-08
783 4.24929389453155e-08
784 4.25147703708717e-08
785 4.23423287543301e-08
786 4.24275121702067e-08
787 4.22150066015092e-08
788 4.22579269354628e-08
789 4.21517931670223e-08
790 4.22686134982087e-08
791 4.20033252623853e-08
792 4.2114898235468e-08
793 4.19893382286318e-08
794 4.21449399823359e-08
795 4.18050376538304e-08
796 4.20182679761183e-08
797 4.17971541821771e-08
798 4.20263930323017e-08
799 4.16353493903898e-08
800 4.196597558348e-08
801 4.15848653290141e-08
802 4.1929077099212e-08
803 4.14620728861337e-08
804 4.19009396068759e-08
805 4.14360066258723e-08
806 4.18592556172825e-08
807 4.13710878888196e-08
808 4.18333563345641e-08
809 4.13486418437969e-08
810 4.17868619706496e-08
811 4.1306389420015e-08
812 4.17491605730902e-08
813 4.12906047131401e-08
814 4.16762766519696e-08
815 4.12456166998254e-08
816 4.15601029146728e-08
817 4.11762215435374e-08
818 4.14568752660216e-08
819 4.10959195562555e-08
820 4.13594243298121e-08
821 4.09103648735254e-08
822 4.13030427637295e-08
823 4.07686826520148e-08
824 4.13907343954634e-08
825 4.0834027714709e-08
826 4.11871781125228e-08
827 4.04365323447564e-08
828 4.14637000289986e-08
829 4.0866027006814e-08
830 4.12762659607324e-08
831 4.06910309891373e-08
832 4.12728837773102e-08
833 4.05149265247928e-08
834 4.13515479635862e-08
835 4.08286489061993e-08
836 4.11164542413189e-08
837 4.02943136634804e-08
838 4.13537506460671e-08
839 4.08698355158776e-08
840 4.10047569232574e-08
841 4.04673343723516e-08
842 4.07917823963544e-08
843 3.95534272001896e-08
844 4.06072295788817e-08
845 3.94961574556874e-08
846 4.07994349416185e-08
847 4.01934450167118e-08
848 4.04261015773955e-08
849 3.92076273669772e-08
850 4.03010496086154e-08
851 3.85393263968581e-08
852 3.78197988482043e-08
853 3.80159583812656e-08
854 3.7647254202966e-08
855 3.82773919227475e-08
856 3.76248827649306e-08
857 3.79325797439378e-08
858 3.73557185184836e-08
859 3.86479044323096e-08
860 3.891408795198e-08
861 3.99624475733162e-08
862 3.77880766677663e-08
863 3.73274708920235e-08
864 3.74777791023462e-08
865 3.74084159204813e-08
866 3.74113398038389e-08
867 3.73798911823542e-08
868 3.71713895219727e-08
869 3.72858686148447e-08
870 3.72327662034877e-08
871 3.71519703890044e-08
872 3.70409978245334e-08
873 3.7161598243074e-08
874 3.70342299049753e-08
875 3.70264920945829e-08
876 3.6807332293165e-08
877 3.68001167316834e-08
878 3.7070268632533e-08
879 3.66974290955113e-08
880 3.69178252412894e-08
881 3.66331533996345e-08
882 3.66964094666855e-08
883 3.64543737418899e-08
884 3.70849200237444e-08
885 3.66197667744927e-08
886 3.65094479093386e-08
887 3.58453853266383e-08
888 3.93497820994071e-08
889 3.64966759036633e-08
890 3.54373526079144e-08
891 3.97264336982062e-08
892 3.89333933981106e-08
893 3.62269112486047e-08
894 3.54100855304296e-08
895 3.90675474193358e-08
896 3.62965835165596e-08
897 3.50618698519156e-08
898 3.95711019507416e-08
899 3.91974346314328e-08
900 3.85543081904416e-08
901 3.58746987672021e-08
902 3.49513520347955e-08
903 3.93355605865509e-08
904 3.89841439130123e-08
905 3.86827920806354e-08
906 3.62305812018349e-08
907 3.53573241795857e-08
908 3.52157663030539e-08
909 3.58019178747782e-08
910 3.49654172282499e-08
911 3.8540665769915e-08
912 3.72209960630698e-08
913 3.87752692176946e-08
914 3.63610581644025e-08
915 3.51003137666339e-08
916 3.4708673268824e-08
917 3.85080340947752e-08
918 3.59683767214847e-08
919 3.45946844504397e-08
920 3.45500197340698e-08
921 3.47287887336734e-08
922 3.47120341359641e-08
923 3.84084799520679e-08
924 3.67585712979235e-08
925 3.78363012032423e-08
926 3.48920146109322e-08
927 3.38452075254736e-08
928 3.81578075803191e-08
929 3.7286628895572e-08
930 3.42384005591612e-08
931 3.37840582176341e-08
932 3.72813495630453e-08
933 3.46288793195981e-08
934 3.35059944234217e-08
935 3.73689914567876e-08
936 3.44539188290582e-08
937 3.34939045387728e-08
938 3.74930486657377e-08
939 3.42809549636058e-08
940 3.34454099970571e-08
941 3.7450682555118e-08
942 3.45472770391098e-08
943 3.3169953894685e-08
944 3.71504071949857e-08
945 3.37158070351506e-08
946 3.33542615749138e-08
947 3.67004027168605e-08
948 3.38644277064759e-08
949 3.28991838216552e-08
950 3.69758552665189e-08
951 3.32539507041929e-08
952 3.35122045669323e-08
953 3.68172301534742e-08
954 3.32951408665849e-08
955 3.34357928011286e-08
956 3.60891654338502e-08
957 3.35626637593123e-08
958 3.27242126729743e-08
959 3.44284138975581e-08
960 3.34671845791945e-08
961 3.26403934991504e-08
962 3.52411930748531e-08
963 3.49293109991322e-08
964 3.47263409139487e-08
965 3.27921227949446e-08
966 3.25499058817513e-08
967 3.48064688182603e-08
968 3.30226086475705e-08
969 3.30550520288853e-08
970 3.54984273087666e-08
971 3.22433209021256e-08
972 3.30936700265738e-08
973 3.36503767073282e-08
974 3.26200151334888e-08
975 3.23197966167754e-08
976 3.22601962920999e-08
977 3.34859286965639e-08
978 3.34748371244586e-08
979 3.34585905648055e-08
980 3.29317799696582e-08
981 3.21779793921451e-08
982 3.2898523016911e-08
983 3.28685807460261e-08
984 3.27762741392235e-08
985 3.18919006758733e-08
986 3.28050973052996e-08
987 3.17904778057709e-08
988 3.27353824047805e-08
989 3.32693161908537e-08
990 3.28773346325306e-08
991 3.15524779637144e-08
992 3.25196793937721e-08
993 3.15338759548922e-08
994 3.24719415800701e-08
995 3.14844115223423e-08
996 3.23690123593678e-08
997 3.14491899189306e-08
998 3.23896749421237e-08
999 3.13312078503714e-08
1000 3.23056177364833e-08
1001 3.12508952049484e-08
1002 3.2182338571829e-08
1003 3.11677545994371e-08
1004 3.2108960823507e-08
1005 3.10556522720162e-08
1006 3.21520658985719e-08
1007 3.10097512112861e-08
1008 3.20340660664442e-08
1009 3.09936680764622e-08
1010 3.19873265652859e-08
1011 3.09523926489419e-08
1012 3.19084030309114e-08
1013 3.08833847384449e-08
1014 3.18450936731551e-08
1015 3.08500851531335e-08
1016 3.17814361494584e-08
1017 3.07774143948336e-08
1018 3.17228305846129e-08
1019 3.07394678600303e-08
1020 3.16569419567259e-08
1021 3.06871790201058e-08
1022 3.15940695827521e-08
1023 3.06509839731461e-08
1024 3.15267847383893e-08
1025 3.059415476514e-08
1026 3.14674508672397e-08
1027 3.0547678164794e-08
1028 3.14074632967731e-08
1029 3.05035214864802e-08
1030 3.1347592965858e-08
1031 3.04515275217909e-08
1032 3.12856478501544e-08
1033 3.04105078896555e-08
1034 3.12287724568705e-08
1035 3.0354257773979e-08
1036 3.11668522101627e-08
1037 3.03188052441783e-08
1038 3.11116785667309e-08
1039 3.02750713387923e-08
1040 3.10583914142626e-08
1041 3.02239442362406e-08
1042 3.10040384476906e-08
1043 3.0170561160503e-08
1044 3.09449745827806e-08
1045 3.01282163661654e-08
1046 3.08989278607896e-08
1047 3.0071710455104e-08
1048 3.08418712791081e-08
1049 3.00262250618744e-08
1050 3.07988266001757e-08
1051 2.99716376161996e-08
1052 3.07658147846723e-08
1053 2.99184534924279e-08
1054 3.0729328415191e-08
1055 2.98599047710013e-08
1056 3.0718190657808e-08
1057 2.97733286913626e-08
1058 3.07432692636667e-08
1059 2.96918276632141e-08
1060 3.06773095815061e-08
1061 2.96273992006491e-08
1062 3.06474809974588e-08
1063 2.96163893409584e-08
1064 3.05666603139798e-08
1065 2.95286302076647e-08
1066 3.05454683768858e-08
1067 2.95256779025976e-08
1068 3.04396507999627e-08
1069 2.94362756392275e-08
1070 3.04399314643433e-08
1071 2.94250472876456e-08
1072 3.03018232727936e-08
1073 2.93495610037553e-08
1074 3.03197644768716e-08
1075 2.93350463920206e-08
1076 3.01721989615089e-08
1077 2.92465465179248e-08
1078 3.021536798542e-08
1079 2.92569453108626e-08
1080 3.00576452616497e-08
1081 2.91283654973995e-08
1082 3.01011340297919e-08
1083 2.91971922194989e-08
1084 2.99356166522102e-08
1085 2.8990310596555e-08
1086 3.00906748407215e-08
1087 2.90283175274908e-08
1088 3.00353732995973e-08
1089 2.89936910036204e-08
1090 2.98520461683438e-08
1091 2.88225834310651e-08
1092 2.99292075567337e-08
1093 2.88866122133413e-08
1094 2.98627540473717e-08
1095 2.85426864365945e-08
1096 2.97287900963283e-08
1097 2.87615158356402e-08
1098 2.97186062425681e-08
1099 2.86455517084505e-08
1100 2.98342257565309e-08
1101 2.8256174289254e-08
1102 2.93447257604385e-08
1103 2.87341244131767e-08
1104 2.94742790174496e-08
1105 2.84344654488677e-08
1106 2.9505388354778e-08
1107 2.82894649927812e-08
1108 2.92375368360354e-08
1109 2.81548917513419e-08
1110 2.94086728303e-08
1111 2.81496159715289e-08
1112 2.94545774437438e-08
1113 2.79748402221003e-08
1114 2.92099873178131e-08
1115 2.79398459923641e-08
1116 2.92390662792741e-08
1117 2.77906124779292e-08
1118 2.9155923897406e-08
1119 2.81215193354001e-08
1120 2.90456103613224e-08
1121 2.77232778955749e-08
1122 2.91450596989762e-08
1123 2.77511755797377e-08
1124 2.91010806563463e-08
1125 2.8033310783826e-08
1126 2.89838197886638e-08
1127 2.76769700491286e-08
1128 2.88644805834792e-08
1129 2.77392722125569e-08
1130 2.89374817441512e-08
1131 2.77247611535358e-08
1132 2.8887400915778e-08
1133 2.76447753577713e-08
1134 2.89343091708361e-08
1135 2.76545506494585e-08
1136 2.87190680126059e-08
1137 2.7473646468934e-08
1138 2.88859069996761e-08
1139 2.77574123686009e-08
1140 2.87116108665941e-08
1141 2.74108735709433e-08
1142 2.86556200990162e-08
1143 2.74945541889338e-08
1144 2.86655037484707e-08
1145 2.74160711910554e-08
1146 2.86107955105308e-08
1147 2.74607749872757e-08
1148 2.86923498293845e-08
1149 2.72647646681889e-08
1150 2.8728248224752e-08
1151 2.753366246111e-08
1152 2.85986292425378e-08
1153 2.72533924317031e-08
1154 2.82711436483396e-08
1155 2.73383786719705e-08
1156 2.84782579740295e-08
1157 2.72924189914647e-08
1158 2.86307173524847e-08
1159 2.71996363210292e-08
1160 2.83642140885831e-08
1161 2.71963269682374e-08
1162 2.83623204921923e-08
1163 2.71654521100118e-08
1164 2.83043295468133e-08
1165 2.71462177181547e-08
1166 2.8356959447251e-08
1167 2.70750657449526e-08
1168 2.82401426687784e-08
1169 2.69484452530833e-08
1170 2.83691896640903e-08
1171 2.70620699183155e-08
1172 2.82270971041498e-08
1173 2.69603788183304e-08
1174 2.82803522821951e-08
1175 2.6979119382986e-08
1176 2.81766734389066e-08
1177 2.69097402139096e-08
1178 2.81986665129352e-08
1179 2.68165845085377e-08
1180 2.7813975123081e-08
1181 2.6852330137217e-08
1182 2.81273688784722e-08
1183 2.67936943743052e-08
1184 2.80369434335626e-08
1185 2.68210822440551e-08
1186 2.79277916348519e-08
1187 2.66221853451043e-08
1188 2.80024945453761e-08
1189 2.66461217535152e-08
1190 2.79817307102803e-08
1191 2.66897099976404e-08
1192 2.78448748503024e-08
1193 2.66836650553159e-08
1194 2.77876264220822e-08
1195 2.64489816714786e-08
1196 2.78555436494798e-08
1197 2.64947033201679e-08
1198 2.78570215783702e-08
1199 2.64461181842535e-08
1200 2.76369505058938e-08
1201 2.65444235481027e-08
1202 2.76968368240205e-08
1203 2.64697099794375e-08
1204 2.76584213310116e-08
1205 2.63365684816108e-08
1206 2.75243277059189e-08
1207 2.63639741149291e-08
1208 2.76351919126228e-08
1209 2.62409454165891e-08
1210 2.73929217087243e-08
1211 2.6186604884515e-08
1212 2.7473491925889e-08
1213 2.62134971507066e-08
1214 2.72154370151156e-08
1215 2.6203151648474e-08
1216 2.74617573126079e-08
1217 2.61605102025442e-08
1218 2.72291416081316e-08
1219 2.61296584369575e-08
1220 2.73399294314913e-08
1221 2.60652353034629e-08
1222 2.70044999695074e-08
1223 2.60413273167615e-08
1224 2.72540390255926e-08
1225 2.59758650145159e-08
1226 2.66212758504025e-08
1227 2.59725094764462e-08
1228 2.72048623628507e-08
1229 2.585665015431e-08
1230 2.70802438251394e-08
1231 2.59169663507919e-08
1232 2.6982636569528e-08
1233 2.57921559665419e-08
1234 2.70270383850857e-08
1235 2.57606895814888e-08
1236 2.68494755317761e-08
1237 2.58385934870375e-08
1238 2.69018585186132e-08
1239 2.57586290075551e-08
1240 2.64909889580167e-08
1241 2.58014516418825e-08
1242 2.68352344789946e-08
1243 2.5760753530335e-08
1244 2.62780375237526e-08
1245 2.59499124410922e-08
1246 2.6739609637616e-08
1247 2.5547700843731e-08
1248 2.63219899210299e-08
1249 2.57260985847552e-08
1250 2.67810147391856e-08
1251 2.5705848116786e-08
1252 2.58826453602978e-08
1253 2.5460840546998e-08
1254 2.67075197513122e-08
1255 2.56452636904214e-08
1256 2.5929264069191e-08
1257 2.56366377016093e-08
1258 2.68881343856719e-08
1259 2.56414658394988e-08
1260 2.56635921402903e-08
1261 2.55792951264766e-08
1262 2.63806594347216e-08
1263 2.55382701652707e-08
1264 2.53127616645088e-08
1265 2.56996059988523e-08
1266 2.54162024759808e-08
1267 2.63552717427729e-08
1268 2.53964866914203e-08
1269 2.54901344476366e-08
1270 2.59257593171469e-08
1271 2.56153214195365e-08
1272 2.59439119076887e-08
1273 2.53626453172728e-08
1274 2.57030023931293e-08
1275 2.54714453973293e-08
1276 2.59810679636985e-08
1277 2.54120280374082e-08
1278 2.5145476811872e-08
1279 2.57573962159086e-08
1280 2.53231409175214e-08
1281 2.54394567633653e-08
1282 2.57781955781411e-08
1283 2.52892853325193e-08
1284 2.54364387330952e-08
1285 2.51480951618532e-08
1286 2.58116603646386e-08
1287 2.54913103958643e-08
1288 2.50334011298037e-08
1289 2.50716478689128e-08
1290 2.59429153715018e-08
1291 2.53887897372351e-08
1292 2.54519196829506e-08
1293 2.51496974357224e-08
1294 2.52243737008939e-08
1295 2.52500989006421e-08
1296 2.58321808388473e-08
1297 2.51184939514815e-08
1298 2.49328095947021e-08
1299 2.51286227381797e-08
1300 2.56158863010114e-08
1301 2.53360301627481e-08
1302 2.48529108404227e-08
1303 2.51653453631207e-08
1304 2.48398741575784e-08
1305 2.57097116929117e-08
1306 2.4873083148691e-08
1307 2.49628033799354e-08
1308 2.47971101430267e-08
1309 2.57397836378459e-08
1310 2.4813122223577e-08
1311 2.50770142429246e-08
1312 2.49469476187869e-08
1313 2.49886529246623e-08
1314 2.50931595502379e-08
1315 2.51452902944038e-08
1316 2.54425280843407e-08
1317 2.53751935019864e-08
1318 2.4760263173107e-08
1319 2.58466510416611e-08
1320 2.49296974175195e-08
1321 2.49420004649892e-08
1322 2.55392027526113e-08
1323 2.50068392659841e-08
1324 2.52009506596096e-08
1325 2.49563516518947e-08
1326 2.51828193853498e-08
1327 2.50284895031427e-08
1328 2.53874663513898e-08
1329 2.48123637192066e-08
1330 2.47559341914894e-08
1331 2.47881892789792e-08
1332 2.50815794800019e-08
1333 2.51361882419587e-08
1334 2.47707774292394e-08
1335 2.47289975163767e-08
1336 2.53113121573278e-08
1337 2.48541063285757e-08
1338 2.47650060458682e-08
1339 2.4944569076979e-08
1340 2.44231852519761e-08
1341 2.47972877787106e-08
1342 2.53143426220959e-08
1343 2.47181137780217e-08
1344 2.51010146001818e-08
1345 2.45448532609771e-08
1346 2.52965079994283e-08
1347 2.45473241733407e-08
1348 2.46413662807754e-08
1349 2.46058995401199e-08
1350 2.54772718477625e-08
1351 2.48136586833425e-08
1352 2.48401654801e-08
1353 2.48850771100706e-08
1354 2.47961082777692e-08
1355 2.481325545034e-08
1356 2.48273526182174e-08
1357 2.43555042800381e-08
1358 2.51023486441682e-08
1359 2.45571296630942e-08
1360 2.45860771741491e-08
1361 2.43790019283097e-08
1362 2.48743710073995e-08
1363 2.43920634801498e-08
1364 2.5173006790169e-08
1365 2.45000322252054e-08
1366 2.45432580925353e-08
1367 2.49997995638296e-08
1368 2.47738540792852e-08
1369 2.47269689168661e-08
1370 2.47370088857224e-08
1371 2.46324241004459e-08
1372 2.47452796031666e-08
1373 2.45667859388732e-08
1374 2.4824680977531e-08
1375 2.46300793094179e-08
1376 2.49007587882488e-08
1377 2.4567938794462e-08
1378 2.54012118006131e-08
1379 2.45844766766368e-08
1380 2.48284166559642e-08
1381 2.45060487458204e-08
1382 2.49019045384102e-08
1383 2.46383820012852e-08
1384 2.45266402743027e-08
1385 2.43175950487284e-08
1386 2.49234197724491e-08
1387 2.4102545737037e-08
1388 2.47481430903917e-08
1389 2.44899851509217e-08
1390 2.45299904833018e-08
1391 2.46787408286764e-08
1392 2.45051428038323e-08
1393 2.47668179298444e-08
1394 2.44410998107014e-08
1395 2.49193341517184e-08
1396 2.43223325924191e-08
1397 2.55463756815288e-08
1398 2.39293953541164e-08
1399 2.43195543703223e-08
1400 2.45367282047937e-08
1401 2.44375684133047e-08
1402 2.44954030392819e-08
1403 2.46240023926703e-08
1404 2.41818760571277e-08
1405 2.43091076157498e-08
1406 2.45273099608312e-08
1407 2.44806965810085e-08
1408 2.49685729869498e-08
1409 2.42312072629147e-08
1410 2.47337315073537e-08
1411 2.39876083441004e-08
1412 2.49583287370569e-08
1413 2.44219080514085e-08
1414 2.4896509742689e-08
1415 2.43039259828493e-08
1416 2.49882283753777e-08
1417 2.42422792950947e-08
1418 2.46401850034772e-08
1419 2.36050858859471e-08
1420 2.43450433146108e-08
1421 2.41582789328731e-08
1422 2.43884628048363e-08
1423 2.41936550793298e-08
1424 2.44933371362777e-08
1425 2.38645370131962e-08
1426 2.47102676098621e-08
1427 2.38647039907391e-08
1428 2.45016877897797e-08
1429 2.37853043927316e-08
1430 2.45215527883147e-08
1431 2.39369821741775e-08
1432 2.45087576900005e-08
1433 2.36575985468335e-08
1434 2.46621549848669e-08
1435 2.38277344521975e-08
1436 2.45333815485083e-08
1437 2.3556308903494e-08
1438 2.46410571946853e-08
1439 2.37503936517669e-08
1440 2.44011957306611e-08
1441 2.39209185792788e-08
1442 2.45670470633286e-08
1443 2.372591545452e-08
1444 2.43355451345906e-08
1445 2.38017090481435e-08
1446 2.45164848422519e-08
1447 2.35970851747425e-08
1448 2.44812508043424e-08
1449 2.47834996969232e-08
1450 2.35460095865392e-08
1451 2.5035001627316e-08
1452 2.35578117013802e-08
1453 2.47371030326349e-08
1454 2.36751720450457e-08
1455 2.44547493366554e-08
1456 2.32458408078173e-08
1457 2.42880133782819e-08
1458 2.4291994193959e-08
1459 2.40056134970246e-08
1460 2.46019027372313e-08
1461 2.35769430645405e-08
1462 2.48224534260544e-08
1463 2.3418381900342e-08
1464 2.44248479219777e-08
1465 2.38066739655096e-08
1466 2.39493154197135e-08
1467 2.41315181170876e-08
1468 2.4258701714075e-08
1469 2.41026647529452e-08
1470 2.41189059835278e-08
1471 2.38303670130335e-08
1472 2.43899762608635e-08
1473 2.33614567690665e-08
1474 2.4357385441931e-08
1475 2.37285302517876e-08
1476 2.43226736529323e-08
1477 2.38253203832528e-08
1478 2.38106885319667e-08
1479 2.47097986516565e-08
1480 2.33130990068275e-08
1481 2.42952342688341e-08
1482 2.33257377857399e-08
1483 2.48140104019967e-08
1484 2.31758505719881e-08
1485 2.43380391395931e-08
1486 2.34366623885762e-08
1487 2.47416931387079e-08
1488 2.31770691527799e-08
1489 2.43285800394233e-08
1490 2.31160264263508e-08
1491 2.44165807572472e-08
1492 2.34254642350606e-08
1493 2.44936355642267e-08
1494 2.36608705961316e-08
1495 2.36995756353053e-08
1496 2.42369537772902e-08
1497 2.33069350485948e-08
1498 2.46346711918477e-08
1499 2.34077042193803e-08
1500 2.40329640632808e-08
1501 2.3407375593365e-08
1502 2.33611778810427e-08
1503 2.42319160292936e-08
1504 2.3027018514199e-08
1505 2.43213307271617e-08
1506 2.31846755127663e-08
1507 2.45820324096258e-08
1508 2.34253860753597e-08
1509 2.31525270066868e-08
1510 2.47582185863848e-08
1511 2.31838495068359e-08
1512 2.44390783166182e-08
1513 2.31530279393155e-08
1514 2.48629632437769e-08
1515 2.30688161906301e-08
1516 2.32691927948281e-08
1517 2.40438886578431e-08
1518 2.34673827037568e-08
1519 2.37875301678514e-08
1520 2.31080807822082e-08
1521 2.46400073677933e-08
1522 2.31054162469491e-08
1523 2.4220740968417e-08
1524 2.30731345141066e-08
1525 2.42701183594818e-08
1526 2.28623164844066e-08
1527 2.3215836364443e-08
1528 2.45345166405286e-08
1529 2.26918803747367e-08
1530 2.33466845855901e-08
1531 2.4035694323743e-08
1532 2.3116809799717e-08
1533 2.36137704945349e-08
1534 2.31576375853138e-08
1535 2.45448035229856e-08
1536 2.26803411607079e-08
1537 2.43788953468993e-08
1538 2.29937100471034e-08
1539 2.43587052750627e-08
1540 2.26614194076546e-08
1541 2.39373463273296e-08
1542 2.38841533217737e-08
1543 2.30909655840605e-08
1544 2.31679635476212e-08
1545 2.36415136356527e-08
1546 2.28829755144488e-08
1547 2.37860717788863e-08
1548 2.28336478613755e-08
1549 2.28516370270881e-08
1550 2.33625936374438e-08
1551 2.37918076351207e-08
1552 2.28511094491068e-08
1553 2.26129852620716e-08
1554 2.39954385250485e-08
1555 2.26945733317052e-08
1556 2.29435741516681e-08
1557 2.41268924838778e-08
1558 2.31707701914274e-08
1559 2.2684178091481e-08
1560 2.36928894281618e-08
1561 2.26602256958586e-08
1562 2.39289370540519e-08
1563 2.26297220962124e-08
1564 2.26625580523887e-08
1565 2.39439437166311e-08
1566 2.2577596681117e-08
1567 2.25464393821539e-08
1568 2.39248798550307e-08
1569 2.26230536526373e-08
1570 2.29948060592733e-08
1571 2.37543016368136e-08
1572 2.26836398553587e-08
1573 2.25659135821843e-08
1574 2.38700632593236e-08
1575 2.22869829258343e-08
1576 2.26528484859045e-08
1577 2.36445494294912e-08
1578 2.30023360359155e-08
1579 2.26749552467709e-08
1580 2.41604389827899e-08
1581 2.22404690219946e-08
1582 2.27454837187224e-08
1583 2.37508270828357e-08
1584 2.23833627188696e-08
1585 2.32403589706109e-08
1586 2.35195187769932e-08
1587 2.26054801544251e-08
1588 2.22427694041016e-08
1589 2.38844553024364e-08
1590 2.22016129924896e-08
1591 2.28742624841516e-08
1592 2.33106423053187e-08
1593 2.28472707419769e-08
1594 2.34356125616841e-08
1595 2.23837552937312e-08
1596 2.24041460938906e-08
1597 2.26449152762598e-08
1598 2.39261108703204e-08
1599 2.22349996192861e-08
1600 2.25501750605872e-08
1601 2.25250236240981e-08
1602 2.32238601682866e-08
1603 2.26287113491708e-08
1604 2.26644853995595e-08
1605 2.36737029979395e-08
1606 2.22204015187799e-08
1607 2.27989591650157e-08
1608 2.29699423925922e-08
1609 2.3023655870702e-08
1610 2.31868089173304e-08
1611 2.22396288052096e-08
1612 2.23141594091203e-08
1613 2.22669154226196e-08
1614 2.35549997285034e-08
1615 2.20773266335073e-08
1616 2.2517316011772e-08
1617 2.22424816342937e-08
1618 2.34364847528923e-08
1619 2.20518501237166e-08
1620 2.37630644051023e-08
1621 2.21036451364398e-08
1622 2.22852030162812e-08
1623 2.36828938682265e-08
1624 2.2408146449493e-08
1625 2.22523208748271e-08
1626 2.23041265456914e-08
1627 2.35201422782438e-08
1628 2.19249933763876e-08
1629 2.22422720241866e-08
1630 2.21302780545329e-08
1631 2.32765042795791e-08
1632 2.22535838645399e-08
1633 2.23384315489739e-08
1634 2.21568274838546e-08
1635 2.24429612671884e-08
1636 2.34203447746495e-08
1637 2.19478888396907e-08
1638 2.24245528812617e-08
1639 2.22099671987053e-08
1640 2.21391065480248e-08
1641 2.35388810665427e-08
1642 2.17904858601514e-08
1643 2.25211405080472e-08
1644 2.18025064668836e-08
1645 2.27604299851691e-08
1646 2.28815721925457e-08
1647 2.22035652086561e-08
1648 2.20793285876653e-08
1649 2.20516085391864e-08
1650 2.21248157572518e-08
1651 2.32196928351414e-08
1652 2.18162252707543e-08
1653 2.22698250951225e-08
1654 2.19427871428479e-08
1655 2.21876330641635e-08
1656 2.18427462783666e-08
1657 2.26393215285725e-08
1658 2.30823236080369e-08
1659 2.19487521491146e-08
1660 2.23185470105136e-08
1661 2.19343103680103e-08
1662 2.33394015225485e-08
1663 2.16802096275615e-08
1664 2.22311715702972e-08
1665 2.17626414666938e-08
1666 2.2728874782274e-08
1667 2.27979572997583e-08
1668 2.22012310757691e-08
1669 2.18728395395829e-08
1670 2.2144700295712e-08
1671 2.17354347853416e-08
1672 2.22565432750343e-08
1673 2.17628688403693e-08
1674 2.34517756325658e-08
1675 2.16432702870861e-08
1676 2.22240235103754e-08
1677 2.16270112929351e-08
1678 2.22097060742499e-08
1679 2.17699120952375e-08
1680 2.21987086490572e-08
1681 2.17878586283859e-08
1682 2.20554454699595e-08
1683 2.17430820015352e-08
1684 2.22401741467593e-08
1685 2.17504307897798e-08
1686 2.20069793499533e-08
1687 2.1738747690847e-08
1688 2.20998401800898e-08
1689 2.1705002239969e-08
1690 2.20686757756994e-08
1691 2.16014051090951e-08
1692 2.24273879467773e-08
1693 2.1493287150065e-08
1694 2.19542481971757e-08
1695 2.17716724648653e-08
1696 2.18787512551444e-08
1697 2.1550128792569e-08
1698 2.30562910985554e-08
1699 2.18095248527561e-08
1700 2.24122196357257e-08
1701 2.25360405892161e-08
1702 2.26858922758311e-08
1703 2.1656257231939e-08
1704 2.21675140466004e-08
1705 2.15922941748659e-08
1706 2.20741895873289e-08
1707 2.15463558106421e-08
1708 2.19737241735629e-08
1709 2.16187068247109e-08
1710 2.22108376135566e-08
1711 2.25699103850729e-08
1712 2.16236326622266e-08
1713 2.17539604108197e-08
1714 2.15903916966909e-08
1715 2.16442472833478e-08
1716 2.1469983124689e-08
1717 2.18904219195792e-08
1718 2.15711377649086e-08
1719 2.17777351707582e-08
1720 2.15611990483922e-08
1721 2.15660627134184e-08
1722 2.15088817867581e-08
1723 2.16900968297296e-08
1724 2.1605321975926e-08
1725 2.17772413435569e-08
1726 2.14815099042198e-08
1727 2.16750635217977e-08
1728 2.15651176915799e-08
1729 2.15538218384381e-08
1730 2.14874891213412e-08
1731 2.17274749303442e-08
1732 2.14815543131408e-08
1733 2.15434212691434e-08
1734 2.14980797608177e-08
1735 2.15841264861183e-08
1736 2.15219735366645e-08
1737 2.1482447820631e-08
1738 2.14946300758356e-08
1739 2.15661568603309e-08
1740 2.14109387997041e-08
1741 2.14884003923999e-08
1742 2.14327577907625e-08
1743 2.15159570160495e-08
1744 2.14364632711295e-08
1745 2.14290345468271e-08
1746 2.13958077921461e-08
1747 2.14851780810932e-08
1748 2.13736139897946e-08
1749 2.14949125165731e-08
1750 2.13013642280657e-08
1751 2.13255777481436e-08
1752 2.12964899048984e-08
1753 2.15088356014803e-08
1754 2.13095212586722e-08
1755 2.12922728337617e-08
1756 2.12679918121239e-08
1757 2.14027071621103e-08
1758 2.1264705551971e-08
1759 2.14117061858587e-08
1760 2.10104449394066e-08
1761 2.13564455009418e-08
1762 2.11964952256949e-08
1763 2.13568966955791e-08
1764 2.12725570492012e-08
1765 2.14231921091823e-08
1766 2.11072421762992e-08
1767 2.11475708056241e-08
1768 2.13044035746179e-08
1769 2.11055013465966e-08
1770 2.12299600121924e-08
1771 2.12082458261875e-08
1772 2.10791029076063e-08
1773 2.12622097706117e-08
1774 2.10339958783834e-08
1775 2.11394386440134e-08
1776 2.12272883715059e-08
1777 2.10632951080925e-08
1778 2.11030908303655e-08
1779 2.11922053239277e-08
1780 2.09617212476587e-08
1781 2.11317683351808e-08
1782 2.09714841048481e-08
1783 2.12213091543845e-08
1784 2.09798116657112e-08
1785 2.10486597040926e-08
1786 2.10953832180394e-08
1787 2.09493133951355e-08
1788 2.10871373695909e-08
1789 2.10410426859653e-08
1790 2.10268744638142e-08
1791 2.09417390095723e-08
1792 2.10600976657815e-08
1793 2.09250750060619e-08
1794 2.10286863477904e-08
1795 2.09288941732666e-08
1796 2.10136796852112e-08
1797 2.09054160649202e-08
1798 2.09872599299388e-08
1799 2.09032045006552e-08
1800 2.10028687774866e-08
1801 2.08854906702527e-08
1802 2.09476276324949e-08
1803 2.08731076867252e-08
1804 2.1019234353048e-08
1805 2.08699280079827e-08
1806 2.09233519399277e-08
1807 2.08383106325982e-08
1808 2.10041601889088e-08
1809 2.09023482966586e-08
1810 2.09729016376059e-08
1811 2.08690345004925e-08
1812 2.08808543789019e-08
1813 2.09464090517031e-08
1814 2.09278034901672e-08
1815 2.08827053427285e-08
1816 2.08958219616306e-08
1817 2.09029646924819e-08
1818 2.09045882826331e-08
1819 2.08671835366658e-08
1820 2.08662438438978e-08
1821 2.08693400338689e-08
1822 2.08408632573764e-08
1823 2.08187049821618e-08
1824 2.0980310821983e-08
1825 2.07921946326906e-08
1826 2.08712798155375e-08
1827 2.07971311283472e-08
1828 2.09227497549591e-08
1829 2.07824477627128e-08
1830 2.08185966243946e-08
1831 2.08053059225222e-08
1832 2.08286756731013e-08
1833 2.08219770314599e-08
1834 2.07254515771638e-08
1835 2.08241175414514e-08
1836 2.07597761203715e-08
1837 2.07759018877596e-08
1838 2.0777500608915e-08
1839 2.0789203247773e-08
1840 2.07165431476142e-08
1841 2.07728714229916e-08
1842 2.07446113620335e-08
1843 2.07866630574927e-08
1844 2.07508197291872e-08
1845 2.07417922837294e-08
1846 2.07508161764736e-08
1847 2.07632258053536e-08
1848 2.07167616395054e-08
1849 2.07019237308259e-08
1850 2.07648991334963e-08
1851 2.07218135983567e-08
1852 2.079808503197e-08
1853 2.06472687835912e-08
1854 2.0866362859806e-08
1855 2.05457588720037e-08
1856 2.10379393905669e-08
1857 2.0561738978131e-08
1858 2.07448334066385e-08
1859 2.05319654611458e-08
1860 2.08917629862526e-08
1861 2.0191844640749e-08
1862 2.07485246761507e-08
1863 2.05950581033676e-08
1864 2.05861994118095e-08
1865 2.06648369527329e-08
1866 2.08473238672013e-08
1867 2.03618082394996e-08
1868 2.05808952102871e-08
1869 2.06683985481959e-08
1870 2.01705105951078e-08
1871 2.07921644346243e-08
1872 2.0740149153653e-08
1873 2.06217354303817e-08
1874 2.02724343978389e-08
1875 2.07632577797767e-08
1876 2.07080166347851e-08
1877 2.0450865889643e-08
1878 2.05142800524527e-08
1879 2.05871995007101e-08
1880 2.06346282283221e-08
1881 2.0365133579503e-08
1882 2.06546513226158e-08
1883 2.06915764522364e-08
1884 2.01549799072609e-08
1885 2.07896810877628e-08
1886 2.02322887332684e-08
1887 2.06230623689407e-08
1888 2.045038982601e-08
1889 2.03677732457663e-08
1890 2.02224246237392e-08
1891 2.04410479653916e-08
1892 2.01699368318486e-08
1893 2.05735819491792e-08
1894 2.01462881932457e-08
1895 2.04781773760487e-08
1896 2.05603409852984e-08
1897 2.03039718371656e-08
1898 2.02305621144205e-08
1899 2.06251744572228e-08
1900 2.00955483364851e-08
1901 2.05101446937306e-08
1902 2.03024921319184e-08
1903 2.03429770806451e-08
1904 2.0267389544415e-08
1905 2.0419209434408e-08
1906 2.0054779170664e-08
1907 2.04639221124125e-08
1908 2.00903471636593e-08
1909 2.05595700464301e-08
1910 2.00772234393298e-08
1911 2.04641530388017e-08
1912 2.05534504971183e-08
1913 2.01721128689769e-08
1914 2.00961078888895e-08
1915 2.06187849016715e-08
1916 2.00399732364076e-08
1917 2.0478838180793e-08
1918 2.00363263758163e-08
1919 2.07704164978395e-08
1920 2.0138122280855e-08
1921 2.03620444949593e-08
1922 1.99678051870933e-08
1923 2.03951575628025e-08
1924 2.02504892854449e-08
1925 2.02714875996435e-08
1926 2.01503915775447e-08
1927 2.0327371785811e-08
1928 2.00186391907664e-08
1929 2.05263486208196e-08
1930 1.98785539140545e-08
1931 2.02404795146549e-08
1932 2.00860306165396e-08
1933 2.03998578029996e-08
1934 1.98973530984858e-08
1935 2.04558183725112e-08
1936 2.00641405712076e-08
1937 2.04602290665434e-08
1938 2.01903205265808e-08
1939 2.04080699006681e-08
1940 2.01650340869719e-08
1941 2.03584491487163e-08
1942 2.01007690492361e-08
1943 2.0314496751439e-08
1944 2.00913259362778e-08
1945 2.022366629717e-08
1946 1.99974685699544e-08
1947 2.02307308683203e-08
1948 2.01607619487731e-08
1949 2.00079330880953e-08
1950 2.03481604899025e-08
1951 1.98959302366575e-08
1952 2.01455208070911e-08
1953 1.9950995522322e-08
1954 2.00156780039151e-08
1955 2.01146104217287e-08
1956 2.00769552094471e-08
1957 1.99288052726843e-08
1958 2.00183887244521e-08
1959 1.97168024129724e-08
1960 2.04040837559205e-08
1961 1.98613019364302e-08
1962 1.99149940982579e-08
1963 2.01730063764671e-08
1964 1.97592200379404e-08
1965 2.01323633319817e-08
1966 1.97698355464127e-08
1967 1.96951059905359e-08
1968 1.97126119871882e-08
1969 2.04679952986453e-08
1970 1.99040108839199e-08
1971 1.98502245751797e-08
1972 1.99940437539681e-08
1973 1.99548022550289e-08
1974 2.01359586782246e-08
1975 1.96931786433652e-08
1976 2.02814085525915e-08
1977 1.96360101512028e-08
1978 1.98135161610935e-08
1979 1.99752232532546e-08
1980 1.98423872888043e-08
1981 1.98211189683661e-08
1982 1.98079650459704e-08
1983 1.9885284530119e-08
1984 1.99351717355967e-08
1985 1.99526066779754e-08
1986 1.98355838421094e-08
1987 1.98480236690557e-08
1988 1.97334042439934e-08
1989 2.01930099308356e-08
1990 1.97124698786411e-08
1991 2.00355536605912e-08
1992 1.97531928591843e-08
1993 1.98986107591281e-08
1994 1.98621297187174e-08
1995 1.96657321538396e-08
1996 1.97864356010768e-08
1997 1.98908196580305e-08
1998 1.96448404210514e-08
1999 1.97984970640164e-08
2000 1.96608063163239e-08
2001 1.97336547103077e-08
2002 1.9593533906459e-08
2003 1.98504412907141e-08
2004 1.95272953362746e-08
2005 1.98110079452363e-08
2006 1.95633909072512e-08
2007 1.97991809613995e-08
2008 1.92812983357271e-08
2009 1.96363085791518e-08
2010 1.98174294752107e-08
2011 1.98649949822993e-08
2012 1.95751113096776e-08
2013 1.96382838879572e-08
2014 2.00886240975251e-08
2015 1.95098976973895e-08
2016 1.99149674529053e-08
2017 2.00363654556668e-08
2018 1.96266771723685e-08
2019 1.95977349903842e-08
2020 1.96332301527491e-08
2021 1.9660230776708e-08
2022 1.99145766544007e-08
2023 1.95190743568219e-08
2024 1.98439362719682e-08
2025 1.94806730746677e-08
2026 1.98684180219288e-08
2027 1.95939371394616e-08
2028 1.97580138916464e-08
2029 1.94150917565139e-08
2030 1.96428509013913e-08
2031 1.96497431659282e-08
2032 1.94629716787631e-08
2033 1.97861282913436e-08
2034 1.95450482465276e-08
2035 1.97639167254238e-08
2036 1.95486258292021e-08
2037 1.97848137872825e-08
2038 1.95388398793739e-08
2039 1.94370759487583e-08
2040 1.95507769973347e-08
2041 1.94018863197698e-08
2042 1.96191276558011e-08
2043 1.94116918095233e-08
2044 1.99550491686296e-08
2045 1.96776905880824e-08
2046 1.96677572006365e-08
2047 1.93328766329159e-08
2048 1.95373068834215e-08
2049 1.93248421709313e-08
2050 1.94870022340865e-08
2051 1.95936546987241e-08
2052 1.95285210224938e-08
2053 1.94915639184501e-08
2054 1.96589411416426e-08
2055 1.92831208778443e-08
2056 1.97004439428383e-08
2057 1.94208347181757e-08
2058 1.9618719093728e-08
2059 1.93577491813812e-08
2060 1.98645579985168e-08
2061 1.93214653165796e-08
2062 1.97502441068309e-08
2063 1.92777012131273e-08
2064 1.97075031849181e-08
2065 1.92896987272206e-08
2066 1.95892049248414e-08
2067 1.9146579433027e-08
2068 1.92272366916768e-08
2069 1.94761469174409e-08
2070 1.92711286928215e-08
2071 1.94080591597867e-08
2072 1.95343670128523e-08
2073 1.92799305409608e-08
2074 1.925033998873e-08
2075 1.93650748769869e-08
2076 1.91004136951278e-08
2077 1.92765448048249e-08
2078 1.89606552680743e-08
2079 1.92620852601522e-08
2080 1.9316827248872e-08
2081 1.92674729504461e-08
2082 1.9295557152077e-08
2083 1.91358537904307e-08
2084 1.9386586558312e-08
2085 1.90494500174054e-08
2086 1.91836040386306e-08
2087 1.89660749327913e-08
2088 1.92078282168495e-08
2089 1.92596480985685e-08
2090 1.91613391820056e-08
2091 1.93018756533547e-08
2092 1.92637035212329e-08
2093 1.91272437888301e-08
2094 1.91338571653432e-08
2095 1.90919138276513e-08
2096 1.9357242919682e-08
2097 1.9145517171637e-08
2098 1.89021882590623e-08
2099 1.90061442140177e-08
2100 1.90002165112446e-08
2101 1.93200708764607e-08
2102 1.89639131065178e-08
2103 1.92964577649946e-08
2104 1.89079756296451e-08
2105 1.8999802620101e-08
2106 1.91686471140429e-08
2107 1.9242962778776e-08
2108 1.90456006521345e-08
2109 1.9074894552773e-08
2110 1.91870164201191e-08
2111 1.89074267353817e-08
2112 1.88115194532656e-08
2113 1.9083135072151e-08
2114 1.89290148000509e-08
2115 1.88921447374923e-08
2116 1.91197173649016e-08
2117 1.86733934981476e-08
2118 1.92459754799756e-08
2119 1.89750455348303e-08
2120 1.91041511499179e-08
2121 1.89168503084147e-08
2122 1.9063579159706e-08
2123 1.88894482278101e-08
2124 1.91228721746484e-08
2125 1.86315407546545e-08
2126 1.9012567520349e-08
2127 1.88345747886842e-08
2128 1.89243163362107e-08
2129 1.90632736263296e-08
2130 1.92768236928487e-08
2131 1.88322779592909e-08
2132 1.89135853645439e-08
2133 1.89587989751772e-08
2134 1.89433482233881e-08
2135 1.87021207409543e-08
2136 1.87731448164641e-08
2137 1.87693842690351e-08
2138 1.9153359787083e-08
2139 1.88424174041302e-08
2140 1.89428526198299e-08
2141 1.8930219169988e-08
2142 1.91216500411429e-08
2143 1.90547329026458e-08
2144 1.87858049116585e-08
2145 1.85499136051703e-08
2146 1.88835063141823e-08
2147 1.87172126686619e-08
2148 1.87031261589254e-08
2149 1.8897690523545e-08
2150 1.85987669709675e-08
2151 1.89795716920571e-08
2152 1.90623890006236e-08
2153 1.89303435149668e-08
2154 1.86742123986505e-08
2155 1.88473929796373e-08
2156 1.87741200363689e-08
2157 1.89070608058728e-08
2158 1.90725693016702e-08
2159 1.87649558114344e-08
2160 1.87502049442401e-08
2161 1.86616055941613e-08
2162 1.85342052816395e-08
2163 1.87006854446281e-08
2164 1.84536954606074e-08
2165 1.89527078475749e-08
2166 1.85798914031921e-08
2167 1.88119901878281e-08
2168 1.87986604061052e-08
2169 1.84799961999715e-08
2170 1.85331057167559e-08
2171 1.8650835542644e-08
2172 1.87211668389864e-08
2173 1.85931909868486e-08
2174 1.85910398187161e-08
2175 1.83586941204794e-08
2176 1.86049842199054e-08
2177 1.84895974086885e-08
2178 1.84806747682842e-08
2179 1.86012769631816e-08
2180 1.8617056340986e-08
2181 1.88191453531772e-08
2182 1.83932229447237e-08
2183 1.85423250087524e-08
2184 1.83373387585561e-08
2185 1.84799322511253e-08
2186 1.8686694858161e-08
2187 1.85948856312734e-08
2188 1.88137363466012e-08
2189 1.84384880697053e-08
2190 1.86330542106816e-08
2191 1.84887465337624e-08
2192 1.83229289518749e-08
2193 1.8720308858633e-08
2194 1.84802768643522e-08
2195 1.85535462549069e-08
2196 1.86533473112149e-08
2197 1.86467286056313e-08
2198 1.83642701045983e-08
2199 1.86505015875582e-08
2200 1.83073627368913e-08
2201 1.88240321108424e-08
2202 1.83406498877048e-08
2203 1.87864763745438e-08
2204 1.82423711692081e-08
2205 1.84146760062731e-08
2206 1.84776371980888e-08
2207 1.84330648522746e-08
2208 1.84398487590443e-08
2209 1.84685884363489e-08
2210 1.84919421997165e-08
2211 1.84839237249435e-08
2212 1.84763919719444e-08
2213 1.84238935219128e-08
2214 1.85212716274918e-08
2215 1.85738482372244e-08
2216 1.82432078332795e-08
2217 1.85348820735953e-08
2218 1.83881070370262e-08
2219 1.82498904877093e-08
2220 1.83165056455437e-08
2221 1.82271460147376e-08
2222 1.83362711680957e-08
2223 1.84224884236528e-08
2224 1.82263750758693e-08
2225 1.83095156813806e-08
2226 1.8281770763906e-08
2227 1.8352073638539e-08
2228 1.8350192476646e-08
2229 1.82900059542135e-08
2230 1.82325106123926e-08
2231 1.85609394520725e-08
2232 1.81540027455185e-08
2233 1.83477801840581e-08
2234 1.831993046153e-08
2235 1.84105317657668e-08
2236 1.81504447027692e-08
2237 1.83237744977305e-08
2238 1.80823089834803e-08
2239 1.8665922141281e-08
2240 1.80822272710657e-08
2241 1.83638331208158e-08
2242 1.82186958852526e-08
2243 1.84271566894267e-08
2244 1.83521873253767e-08
2245 1.79739192418538e-08
2246 1.83535160402926e-08
2247 1.82347399402261e-08
2248 1.83164221567722e-08
2249 1.84280555259875e-08
2250 1.82324040309823e-08
2251 1.8471927987207e-08
2252 1.80200743216119e-08
2253 1.81539885346638e-08
2254 1.80956138962074e-08
2255 1.83971522460524e-08
2256 1.90580582426492e-08
2257 1.7872578084166e-08
2258 1.82327095643586e-08
2259 1.81339760985111e-08
2260 1.82514305890891e-08
2261 1.82984791763374e-08
2262 1.79190795535078e-08
2263 1.87584205946223e-08
2264 1.78647834303547e-08
2265 1.829538653908e-08
2266 1.8275356339359e-08
2267 1.80560562057508e-08
2268 1.81109847119387e-08
2269 1.83903452466438e-08
2270 1.84660606805664e-08
2271 1.80676771321941e-08
2272 1.86272064439663e-08
2273 1.79619146223331e-08
2274 1.79603354411029e-08
2275 1.84529742597306e-08
2276 1.81179728997449e-08
2277 1.88928979127923e-08
2278 1.76782659622177e-08
2279 1.82119048730556e-08
2280 1.79837140734662e-08
2281 1.80729227139409e-08
2282 1.8019553849058e-08
2283 1.80570065566599e-08
2284 1.81313826175256e-08
2285 1.80280537165345e-08
2286 1.80008061789749e-08
2287 1.81929209475129e-08
2288 1.81741253157952e-08
2289 1.80812307348788e-08
2290 1.81637833662762e-08
2291 1.83470607595382e-08
2292 1.84242061607165e-08
2293 1.77064549689021e-08
2294 1.77578023397018e-08
2295 1.83818844590178e-08
2296 1.78439787390516e-08
2297 1.78534058647983e-08
2298 1.90119973098035e-08
2299 1.76614580738033e-08
2300 1.80054495757531e-08
2301 1.78096470904165e-08
2302 1.83346884341518e-08
2303 1.79717432047255e-08
2304 1.78027139696724e-08
2305 1.80486772194399e-08
2306 1.79415149403894e-08
2307 1.79456272064726e-08
2308 1.84622379606481e-08
2309 1.75808079205808e-08
2310 1.77822201408162e-08
2311 1.79372143804812e-08
2312 1.77438685966536e-08
2313 1.85183708367731e-08
2314 1.77189161121305e-08
2315 1.79870607297516e-08
2316 1.79455525994854e-08
2317 1.85455757417685e-08
2318 1.75344823105661e-08
2319 1.76368430970797e-08
2320 1.77793051392428e-08
2321 1.87911783910977e-08
2322 1.80273929117902e-08
2323 1.79331500760327e-08
2324 1.75250338685373e-08
2325 1.77095813569395e-08
2326 1.81723205372464e-08
2327 1.80693859874737e-08
2328 1.77056609373949e-08
2329 1.77593957317868e-08
2330 1.80042469821728e-08
2331 1.76077179503409e-08
2332 1.7466680546363e-08
2333 1.78892740620995e-08
2334 1.78551218255052e-08
2335 1.77534271728064e-08
2336 1.80809038852203e-08
2337 1.81018293687885e-08
2338 1.77598096229303e-08
2339 1.77356938024786e-08
2340 1.76173955424019e-08
2341 1.76384222783099e-08
2342 1.872538568648e-08
2343 1.76538001994686e-08
2344 1.77663093126057e-08
2345 1.75109100553072e-08
2346 1.76036394350376e-08
2347 1.7839317578705e-08
2348 1.75306702487887e-08
2349 1.82238419910163e-08
2350 1.80201684685244e-08
2351 1.83218009652819e-08
2352 1.83482189441975e-08
2353 1.79450232451472e-08
2354 1.78190084909602e-08
2355 1.76535337459427e-08
2356 1.77492935904411e-08
2357 1.77983139337812e-08
2358 1.81963333290014e-08
2359 1.78355268332098e-08
2360 1.76125602990851e-08
2361 1.81438633006792e-08
2362 1.75531358337366e-08
2363 1.74187384516244e-08
2364 1.75358874088261e-08
2365 1.74030958532967e-08
2366 1.74121641549618e-08
2367 1.78903167835642e-08
2368 1.81214172556565e-08
2369 1.79240746689402e-08
2370 1.79030692493143e-08
2371 1.89528197580557e-08
2372 1.76703505161413e-08
2373 1.79164576508128e-08
2374 1.76651102634651e-08
2375 1.81329617987558e-08
2376 1.75105636657236e-08
2377 1.74483059112163e-08
2378 1.78663750460828e-08
2379 1.79549655143774e-08
2380 1.77687091706957e-08
2381 1.7580385147653e-08
2382 1.78695174213317e-08
2383 1.84722122043013e-08
2384 1.75898513532502e-08
2385 1.78066592582127e-08
2386 1.78775128034658e-08
2387 1.83791009078504e-08
2388 1.76143615249202e-08
2389 1.78966814701198e-08
2390 1.79299082248008e-08
2391 1.76897181347613e-08
2392 1.81667818566211e-08
2393 1.77694303715725e-08
2394 1.76251955252837e-08
2395 1.78356600599727e-08
2396 1.75744752084483e-08
2397 1.823005035817e-08
2398 1.78908123871224e-08
2399 1.75330061580325e-08
2400 1.78544308226947e-08
2401 1.7384266470799e-08
2402 1.77318213445687e-08
2403 1.72507110818287e-08
2404 1.7974768340423e-08
2405 1.75061991569692e-08
2406 1.74057923629789e-08
2407 1.78649486315408e-08
2408 1.80805397320682e-08
2409 1.7742889824035e-08
2410 1.84428774474554e-08
2411 1.72207563764459e-08
2412 1.77271424206538e-08
2413 1.76395644757577e-08
2414 1.7498347659739e-08
2415 1.74730825364122e-08
2416 1.77228933750939e-08
2417 1.76071655033638e-08
2418 1.70529261822594e-08
2419 1.78347914214783e-08
2420 1.72120717678581e-08
2421 1.72994010227967e-08
2422 1.8077699337482e-08
2423 1.75280625569485e-08
2424 1.79638082187239e-08
2425 1.78700272357446e-08
2426 1.74039929135006e-08
2427 1.82089063827107e-08
2428 1.76954095820747e-08
2429 1.75477321562312e-08
2430 1.73949601389722e-08
2431 1.74577063916104e-08
2432 1.8303483173554e-08
2433 1.72153651334384e-08
2434 1.739507382581e-08
2435 1.78612804546674e-08
2436 1.7224250470349e-08
2437 1.73913878853682e-08
2438 1.79853127946217e-08
2439 1.74878831415981e-08
2440 1.75050303141688e-08
2441 1.80995360921088e-08
2442 1.74816712217307e-08
2443 1.75877943320302e-08
2444 1.77573440396372e-08
2445 1.71078902155841e-08
2446 1.79692882795734e-08
2447 1.73125336289104e-08
2448 1.75197687468653e-08
2449 1.72470091541754e-08
2450 1.76233765358802e-08
2451 1.73578182938172e-08
2452 1.76359460368758e-08
2453 1.71455027953016e-08
2454 1.74036447475601e-08
2455 1.7585739087167e-08
2456 1.71296239415142e-08
2457 1.69152070128575e-08
2458 1.69214384726502e-08
2459 1.79300432279206e-08
2460 1.74213301562531e-08
2461 1.78522689964211e-08
2462 1.81361876627761e-08
2463 1.74166068234172e-08
2464 1.7365186621987e-08
2465 1.79420300838729e-08
2466 1.72319865043846e-08
2467 1.72404099885171e-08
2468 1.72057870173603e-08
2469 1.68997527083548e-08
2470 1.69881761991064e-08
2471 1.71861938014217e-08
2472 1.77125993872096e-08
2473 1.69587632825596e-08
2474 1.74644441131022e-08
2475 1.70165908031095e-08
2476 1.74497571947541e-08
2477 1.82468316012319e-08
2478 1.78628702940387e-08
2479 1.69611116263013e-08
2480 1.73892278354515e-08
2481 1.685174666477e-08
2482 1.75894712128866e-08
2483 1.7408932961871e-08
2484 1.73435772410357e-08
2485 1.73404721692805e-08
2486 1.73320149343681e-08
2487 1.68919207510498e-08
2488 1.67464229150482e-08
2489 1.71006160343268e-08
2490 1.78684036455934e-08
2491 1.78095831415703e-08
2492 1.71021206085697e-08
2493 1.71923488778702e-08
2494 1.74912191397425e-08
2495 1.79207120254432e-08
2496 1.71751324273828e-08
2497 1.76206587099159e-08
2498 1.72640870488294e-08
2499 1.70800014132055e-08
2500 1.73573351247569e-08
2501 1.72229537298563e-08
2502 1.75512546718437e-08
2503 1.7309352173811e-08
2504 1.78640569004074e-08
2505 1.7341859503972e-08
2506 1.70428702261916e-08
2507 1.69890608248124e-08
2508 1.74708940647861e-08
2509 1.71800618176121e-08
2510 1.7292650866807e-08
2511 1.73246181844888e-08
2512 1.72367862205647e-08
2513 1.76838863552575e-08
2514 1.73443428508335e-08
2515 1.80782908643096e-08
2516 1.75054530870966e-08
2517 1.70743170713195e-08
2518 1.71575464946727e-08
2519 1.80583317188621e-08
2520 1.73752212617728e-08
2521 1.7088082060468e-08
2522 1.72599143866137e-08
2523 1.68056164540076e-08
2524 1.71615894828392e-08
2525 1.73276042403359e-08
2526 1.76829590969874e-08
2527 1.748389166778e-08
2528 1.73894516564133e-08
2529 1.70369975904805e-08
2530 1.74075260872542e-08
2531 1.72426322109231e-08
2532 1.76496754988875e-08
2533 1.75116241507567e-08
2534 1.68242593190371e-08
2535 1.68486771201515e-08
2536 1.74178733658437e-08
2537 1.74238490302514e-08
2538 1.74533134611465e-08
2539 1.67161307018659e-08
2540 1.72313008306446e-08
2541 1.78713666088015e-08
2542 1.68333240679885e-08
2543 1.72316987345766e-08
2544 1.74357790427848e-08
2545 1.72477836457574e-08
2546 1.75296150928261e-08
2547 1.73764274080668e-08
2548 1.69264247062983e-08
2549 1.75033605387398e-08
2550 1.72428578082418e-08
2551 1.71148855088177e-08
2552 1.67548108720439e-08
2553 1.72666005937572e-08
2554 1.72432557121738e-08
2555 1.68758074181596e-08
2556 1.71672258630906e-08
2557 1.69832183871677e-08
2558 1.77526402467265e-08
2559 1.70412164379741e-08
2560 1.71183369701566e-08
2561 1.75110077549334e-08
2562 1.70913310171272e-08
2563 1.74484071635561e-08
2564 1.72494374339749e-08
2565 1.71244725066799e-08
2566 1.72709118118064e-08
2567 1.70793619247434e-08
2568 1.66333045115152e-08
2569 1.7162653520586e-08
2570 1.69784790671201e-08
2571 1.7007049990525e-08
2572 1.73435772410357e-08
2573 1.71126632864116e-08
2574 1.67144449392254e-08
2575 1.66588556282932e-08
2576 1.70169531799047e-08
2577 1.70231420071332e-08
2578 1.68937397404534e-08
2579 1.69637601743489e-08
2580 1.63886006987468e-08
2581 1.67649485405263e-08
2582 1.72048189028828e-08
2583 1.69223568491361e-08
2584 1.70783778230543e-08
2585 1.69323222110052e-08
2586 1.66962816905425e-08
2587 1.67273697115888e-08
2588 1.72305725243405e-08
2589 1.70151039924349e-08
2590 1.69110379033555e-08
2591 1.67627192126929e-08
2592 1.72249841057237e-08
2593 1.74279151110568e-08
2594 1.67220903790621e-08
2595 1.69686273920888e-08
2596 1.70763385654027e-08
2597 1.72804419662498e-08
2598 1.7268748209176e-08
2599 1.68283591506224e-08
2600 1.70303344759759e-08
2601 1.71040035468195e-08
2602 1.66227955844533e-08
2603 1.70452807424226e-08
2604 1.68965783586827e-08
2605 1.71259326720019e-08
2606 1.67598965816751e-08
2607 1.67320219901512e-08
2608 1.72241207962998e-08
2609 1.69523381998715e-08
2610 1.65966973497689e-08
2611 1.64627405041529e-08
2612 1.69438649777476e-08
2613 1.70811045308028e-08
2614 1.6688975534862e-08
2615 1.69857390375228e-08
2616 1.6997615759351e-08
2617 1.67957061592006e-08
2618 1.67636127201831e-08
2619 1.66840266047075e-08
2620 1.70464193871567e-08
2621 1.65085562997547e-08
2622 1.6725122620187e-08
2623 1.70802678667314e-08
2624 1.71909508850376e-08
2625 1.69397988969422e-08
2626 1.71321463682261e-08
2627 1.66341234120182e-08
2628 1.66299720660845e-08
2629 1.66299702897277e-08
2630 1.66731553008503e-08
2631 1.67667799644278e-08
2632 1.65816942399033e-08
2633 1.71559157990941e-08
2634 1.66692899483678e-08
2635 1.68320539728484e-08
2636 1.71013496697014e-08
2637 1.67205520540392e-08
2638 1.67502296477551e-08
2639 1.71234830759204e-08
2640 1.67680020979333e-08
2641 1.64317714990148e-08
2642 1.63484799031721e-08
2643 1.6831704030551e-08
2644 1.64141837899479e-08
2645 1.66533684620163e-08
2646 1.70866023552207e-08
2647 1.65229288029423e-08
2648 1.61585109736961e-08
2649 1.66338054441439e-08
2650 1.66029465731299e-08
2651 1.65578342148365e-08
2652 1.67333755740628e-08
2653 1.62727999963863e-08
2654 1.68369780340072e-08
2655 1.64204330133089e-08
2656 1.6790380641396e-08
2657 1.6372450062363e-08
2658 1.63473323766539e-08
2659 1.64432556459815e-08
2660 1.68777241071894e-08
2661 1.66192215544925e-08
2662 1.66326810102646e-08
2663 1.64350897335908e-08
2664 1.65056075474013e-08
2665 1.66699170023321e-08
2666 1.69729919008432e-08
2667 1.64176476857847e-08
2668 1.66083999886268e-08
2669 1.67102971460054e-08
2670 1.65634386206648e-08
2671 1.64776725597449e-08
2672 1.66090554643006e-08
2673 1.63765587757325e-08
2674 1.65741393942653e-08
2675 1.6460870000401e-08
2676 1.65642504157404e-08
2677 1.63979656520041e-08
2678 1.65951021813271e-08
2679 1.6415487635868e-08
2680 1.65127307383273e-08
2681 1.62612021625819e-08
2682 1.64689364368087e-08
2683 1.64324713836095e-08
2684 1.63007296549722e-08
2685 1.59583848358125e-08
2686 1.64612163899847e-08
2687 1.61996194236735e-08
2688 1.63425983856769e-08
2689 1.64077551545461e-08
2690 1.63451012724636e-08
2691 1.63482649639946e-08
2692 1.6142086778359e-08
2693 1.62639892664629e-08
2694 1.61764361905625e-08
2695 1.62695226180176e-08
2696 1.60264530535414e-08
2697 1.63088937910061e-08
2698 1.61934181619472e-08
2699 1.60722954944958e-08
2700 1.61160862433007e-08
2701 1.5827518851097e-08
2702 1.64686859704943e-08
2703 1.59867070692599e-08
2704 1.66000528878385e-08
2705 1.60776494340098e-08
2706 1.59673874122745e-08
2707 1.63035061007122e-08
2708 1.63703450795083e-08
2709 1.59530788579332e-08
2710 1.61053286262813e-08
2711 1.59199409210942e-08
2712 1.64122990753413e-08
2713 1.57348445384287e-08
2714 1.60126223391899e-08
2715 1.59916471176302e-08
2716 1.62076201348782e-08
2717 1.58656288107295e-08
2718 1.61721338542975e-08
2719 1.5897688498967e-08
2720 1.60446891328547e-08
2721 1.57099009356898e-08
2722 1.60508939472948e-08
2723 1.58106114866996e-08
2724 1.60414828087596e-08
2725 1.58685775630829e-08
2726 1.63831295196815e-08
2727 1.56931445616237e-08
2728 1.61943916054952e-08
2729 1.5920761597954e-08
2730 1.59892312723287e-08
2731 1.58517963200211e-08
2732 1.59314499370566e-08
2733 1.57854085358622e-08
2734 1.60983191221931e-08
2735 1.54606532021262e-08
2736 1.56658206407201e-08
2737 1.56601007716972e-08
2738 1.5966065802786e-08
2739 1.57092596708708e-08
2740 1.57587578542007e-08
2741 1.56613211288459e-08
2742 1.6252053924859e-08
2743 1.56293396003093e-08
2744 1.53783350498315e-08
2745 1.56798272143988e-08
2746 1.59727999715642e-08
2747 1.56229020831233e-08
2748 1.58598911781382e-08
2749 1.56220902880477e-08
2750 1.62805857684134e-08
2751 1.54888919468021e-08
2752 1.58530397698087e-08
2753 1.57064654615624e-08
2754 1.58885065104641e-08
2755 1.55243728983123e-08
2756 1.63854920742779e-08
2757 1.59437512081695e-08
2758 1.58157913432433e-08
2759 1.56821435837173e-08
2760 1.57730681848989e-08
2761 1.62514428581062e-08
2762 1.56891726277308e-08
2763 1.62118674040812e-08
2764 1.54862362933272e-08
2765 1.52103254436042e-08
2766 1.53924357704227e-08
2767 1.57681760981632e-08
2768 1.54977666255718e-08
2769 1.56384345473271e-08
2770 1.53467798469364e-08
2771 1.54201735824699e-08
2772 1.55507002830291e-08
2773 1.51833567940685e-08
2774 1.55974078097643e-08
2775 1.56070534274022e-08
2776 1.51022394589972e-08
2777 1.54112349548541e-08
2778 1.53350097065186e-08
2779 1.55313859551143e-08
2780 1.53357415655364e-08
2781 1.54529100626632e-08
2782 1.55613264496424e-08
2783 1.56360115965981e-08
2784 1.57645470011403e-08
2785 1.55493005138396e-08
2786 1.56952300045532e-08
2787 1.53756900544977e-08
2788 1.57154413926719e-08
2789 1.55173616178672e-08
2790 1.61545781196537e-08
2791 1.55003121449226e-08
2792 1.55936934476131e-08
2793 1.53837991234695e-08
2794 1.51992818331337e-08
2795 1.52959369614791e-08
2796 1.52866856950595e-08
2797 1.53328851837387e-08
2798 1.52233869954443e-08
2799 1.55545034630222e-08
2800 1.59424047296852e-08
2801 1.52990651258733e-08
2802 1.5392730645658e-08
2803 1.52178021295413e-08
2804 1.59129207588649e-08
2805 1.55494994658056e-08
2806 1.53140451431e-08
2807 1.57299648861908e-08
2808 1.50862025094511e-08
2809 1.53546384495939e-08
2810 1.51141996695969e-08
2811 1.54657726625373e-08
2812 1.54084975889646e-08
2813 1.5324198798794e-08
2814 1.54827741738472e-08
2815 1.53619019727103e-08
2816 1.54965871246304e-08
2817 1.49920698078176e-08
2818 1.53562318416789e-08
2819 1.52513912610175e-08
2820 1.51660799474485e-08
2821 1.50968517687033e-08
2822 1.51720556118562e-08
2823 1.516054304318e-08
2824 1.53153560944475e-08
2825 1.54856962808481e-08
2826 1.53904586852605e-08
2827 1.5435558609056e-08
2828 1.51699044437237e-08
2829 1.522519355035e-08
2830 1.50948924471095e-08
2831 1.50465222503726e-08
2832 1.49710075447729e-08
2833 1.5127438857121e-08
2834 1.51909009815654e-08
2835 1.53878119135697e-08
2836 1.53259662738492e-08
2837 1.51537093984189e-08
2838 1.50510448548857e-08
2839 1.49724055376055e-08
2840 1.50481369587396e-08
2841 1.49796974824312e-08
2842 1.50720893543621e-08
2843 1.50786849673068e-08
2844 1.49468011301224e-08
2845 1.50454901870489e-08
2846 1.47123797589188e-08
2847 1.49296681684064e-08
2848 1.52122066054972e-08
2849 1.50662557985015e-08
2850 1.53020476290067e-08
2851 1.51249572866163e-08
2852 1.49067638233191e-08
2853 1.51146242188815e-08
2854 1.53799550872691e-08
2855 1.49101513358119e-08
2856 1.48988972270558e-08
2857 1.49234953283894e-08
2858 1.50429304568434e-08
2859 1.50954768685096e-08
2860 1.52816905796271e-08
2861 1.5194057567669e-08
2862 1.47391254756712e-08
2863 1.52401007369463e-08
2864 1.49248915448652e-08
2865 1.49649217462411e-08
2866 1.49611238953185e-08
2867 1.49055914278051e-08
2868 1.53628203491962e-08
2869 1.47549634732513e-08
2870 1.49751180344992e-08
2871 1.49433390106424e-08
2872 1.49286591977216e-08
2873 1.50536223486597e-08
2874 1.48596583926519e-08
2875 1.50501779927481e-08
2876 1.51864085751185e-08
2877 1.45283420849296e-08
2878 1.50611132454515e-08
2879 1.51010741689106e-08
2880 1.49257370907208e-08
2881 1.45486351854629e-08
2882 1.42788474377653e-08
2883 1.51385481927946e-08
2884 1.4841103457286e-08
2885 1.47501433289676e-08
2886 1.47435894604087e-08
2887 1.45759786462918e-08
2888 1.47309719977784e-08
2889 1.46058463101895e-08
2890 1.47467913436117e-08
2891 1.49408663219219e-08
2892 1.50705279367003e-08
2893 1.47871519473597e-08
2894 1.48167877966898e-08
2895 1.48209524652998e-08
2896 1.4557836713891e-08
2897 1.45419267738589e-08
2898 1.4802063574848e-08
2899 1.4691289962343e-08
2900 1.47552112750304e-08
2901 1.47530139216201e-08
2902 1.42744678299778e-08
2903 1.46299248271475e-08
2904 1.47582346343711e-08
2905 1.47598813171612e-08
2906 1.47016585572146e-08
2907 1.45948018115405e-08
2908 1.4583768859211e-08
2909 1.47597933874977e-08
2910 1.47603778088978e-08
2911 1.43647733708008e-08
2912 1.47471235223406e-08
2913 1.47897347702042e-08
2914 1.46206344808775e-08
2915 1.48011274347937e-08
2916 1.47118610627217e-08
2917 1.45609480028952e-08
2918 1.43641560867991e-08
2919 1.47058480948203e-08
2920 1.45356429115395e-08
2921 1.43282967712821e-08
2922 1.47294860752822e-08
2923 1.44812499769387e-08
2924 1.46445451321142e-08
2925 1.44431604454098e-08
2926 1.43855967138506e-08
2927 1.46391743172103e-08
2928 1.44346943287132e-08
2929 1.47110741366419e-08
2930 1.45159217979085e-08
2931 1.46227288055911e-08
2932 1.40678375615266e-08
2933 1.42588536533594e-08
2934 1.4228192846133e-08
2935 1.45010989882621e-08
2936 1.43142750985703e-08
2937 1.44385765565858e-08
2938 1.47245282633435e-08
2939 1.43033878075016e-08
2940 1.4486162491778e-08
2941 1.46017500313178e-08
2942 1.43264466956339e-08
2943 1.46271945666854e-08
2944 1.40258018532791e-08
2945 1.47852263765458e-08
2946 1.41478322390753e-08
2947 1.44479219699178e-08
2948 1.4171992468448e-08
2949 1.42527136759441e-08
2950 1.44123433187815e-08
2951 1.44021603531996e-08
2952 1.43035014943393e-08
2953 1.4293827454992e-08
2954 1.43941134567172e-08
2955 1.44357183984312e-08
2956 1.41897302796679e-08
2957 1.43551304176981e-08
2958 1.46687284541258e-08
2959 1.41806069109407e-08
2960 1.44704292992515e-08
2961 1.41895597494113e-08
2962 1.42957201632044e-08
2963 1.4225742361873e-08
2964 1.43856304646306e-08
2965 1.43571377009266e-08
2966 1.43929304030621e-08
2967 1.43041773981167e-08
2968 1.42252254420328e-08
2969 1.40860452191305e-08
2970 1.41341107706694e-08
2971 1.42877638609207e-08
2972 1.41598697211975e-08
2973 1.40440725715507e-08
2974 1.44465648332925e-08
2975 1.42590126372966e-08
2976 1.41237004314121e-08
2977 1.41572478185026e-08
2978 1.42090641475079e-08
2979 1.43529339524662e-08
2980 1.39628815176707e-08
2981 1.40989939723113e-08
2982 1.41184299806696e-08
2983 1.39336391313805e-08
2984 1.4151928517947e-08
2985 1.37868241267824e-08
2986 1.40147795590906e-08
2987 1.40080462784908e-08
2988 1.40654563551834e-08
2989 1.38248319458967e-08
2990 1.40276910087778e-08
2991 1.40670231019158e-08
2992 1.40979476981329e-08
2993 1.35314133231645e-08
2994 1.39483820049691e-08
2995 1.39982843094799e-08
2996 1.40457094843782e-08
2997 1.40323139774523e-08
2998 1.40660825209693e-08
2999 1.40574396567672e-08
3000 5.66036417737337e-09
3001 5.72974023782535e-09
3002 5.80295234087203e-09
3003 5.82456749498306e-09
3004 5.79255665655865e-09
3005 5.77928549461149e-09
3006 5.77456793493525e-09
3007 5.77194292361582e-09
3008 5.76993297585204e-09
3009 5.76808867336354e-09
3010 5.76621550507639e-09
3011 5.76470338131685e-09
3012 5.76292746856666e-09
3013 5.76151126807645e-09
3014 5.7596833968887e-09
3015 5.7581681645047e-09
3016 5.75670799918271e-09
3017 5.7550799681394e-09
3018 5.75351277731784e-09
3019 5.75204062158718e-09
3020 5.75035219441133e-09
3021 5.74908209927116e-09
3022 5.7476592374428e-09
3023 5.74630609762039e-09
3024 5.74484415594156e-09
3025 5.74358693938848e-09
3026 5.74220537785664e-09
3027 5.74085268212343e-09
3028 5.73939429315828e-09
3029 5.73794078917444e-09
3030 5.73691938399179e-09
3031 5.73546055093743e-09
3032 5.7341336123784e-09
3033 5.73275338311419e-09
3034 5.73156588856705e-09
3035 5.73014746763079e-09
3036 5.72894709449656e-09
3037 5.72771030604713e-09
3038 5.72627456563168e-09
3039 5.72507241614062e-09
3040 5.72387204300639e-09
3041 5.72266278808797e-09
3042 5.72151748201577e-09
3043 5.72020653066829e-09
3044 5.71892000422736e-09
3045 5.71765124135482e-09
3046 5.71650016212288e-09
3047 5.71531133530812e-09
3048 5.71400393667432e-09
3049 5.71283953476609e-09
3050 5.711726203117e-09
3051 5.71055691622746e-09
3052 5.70941871558261e-09
3053 5.70818592393607e-09
3054 5.70709390856905e-09
3055 5.70600100502361e-09
3056 5.70472868943739e-09
3057 5.70357805429467e-09
3058 5.7024887034629e-09
3059 5.70124925047821e-09
3060 5.69996716492938e-09
3061 5.69894087476541e-09
3062 5.69771785308149e-09
3063 5.69663249905261e-09
3064 5.69551694695747e-09
3065 5.69435254504924e-09
3066 5.69336044975444e-09
3067 5.69213920442735e-09
3068 5.69089442237214e-09
3069 5.68988056670605e-09
3070 5.68869307215891e-09
3071 5.68771518771882e-09
3072 5.68661473465681e-09
3073 5.68558933267127e-09
3074 5.68462477090748e-09
3075 5.68337155115728e-09
3076 5.68232305653282e-09
3077 5.68119062904771e-09
3078 5.68015767754559e-09
3079 5.67900348968919e-09
3080 5.67799585127204e-09
3081 5.67698155151675e-09
3082 5.67587710165185e-09
3083 5.67466873491185e-09
3084 5.67369085047176e-09
3085 5.67281999153124e-09
3086 5.67159963438257e-09
3087 5.67061997358564e-09
3088 5.66967672810392e-09
3089 5.66857538686349e-09
3090 5.66735103291194e-09
3091 5.66655966593999e-09
3092 5.66541702440304e-09
3093 5.66455371497909e-09
3094 5.66313573813204e-09
3095 5.66236790788821e-09
3096 5.66115287981006e-09
3097 5.66026159276589e-09
3098 5.65916113970388e-09
3099 5.65814017861044e-09
3100 5.65722535483815e-09
3101 5.65610713820774e-09
3102 5.65517055406417e-09
3103 5.65392799245501e-09
3104 5.65306823574474e-09
3105 5.65220581449921e-09
3106 5.65117685979999e-09
3107 5.65012614472948e-09
3108 5.64910829226051e-09
3109 5.64832891569722e-09
3110 5.64706681416283e-09
3111 5.64603563901755e-09
3112 5.64516700052309e-09
3113 5.64429170069047e-09
3114 5.64314817097511e-09
3115 5.64208457731752e-09
3116 5.64131097391396e-09
3117 5.6403477444178e-09
3118 5.63938229447558e-09
3119 5.63829782862513e-09
3120 5.63721247459625e-09
3121 5.63637625461411e-09
3122 5.63534507946883e-09
3123 5.63433166789196e-09
3124 5.63349145110692e-09
3125 5.63242741336012e-09
3126 5.63163604638817e-09
3127 5.63046276269574e-09
3128 5.62960300598547e-09
3129 5.6285673899481e-09
3130 5.62778401658193e-09
3131 5.62656143898721e-09
3132 5.62576252249869e-09
3133 5.6247890789507e-09
3134 5.62395019443329e-09
3135 5.62297008954715e-09
3136 5.62200419551573e-09
3137 5.62092417055737e-09
3138 5.619852583294e-09
3139 5.6189324304512e-09
3140 5.61808422006038e-09
3141 5.61728663583949e-09
3142 5.61622082173585e-09
3143 5.61528290532465e-09
3144 5.61423707523545e-09
3145 5.61338486804175e-09
3146 5.61257396114456e-09
3147 5.61138779886505e-09
3148 5.61062885040542e-09
3149 5.609662956374e-09
3150 5.6085855959509e-09
3151 5.60756774348192e-09
3152 5.60666446602909e-09
3153 5.6059796804675e-09
3154 5.60499202606479e-09
3155 5.60390800430355e-09
3156 5.6030313722033e-09
3157 5.60220714262982e-09
3158 5.60113333492041e-09
3159 5.60016522044293e-09
3160 5.59941648603512e-09
3161 5.59831425661628e-09
3162 5.59739410377347e-09
3163 5.59649615539115e-09
3164 5.59558488433254e-09
3165 5.59470825223229e-09
3166 5.59383339648889e-09
3167 5.5928826014906e-09
3168 5.59191049021024e-09
3169 5.59099433417032e-09
3170 5.59010437939378e-09
3171 5.5892712680361e-09
3172 5.58836132924512e-09
3173 5.5874949111967e-09
3174 5.58655965932076e-09
3175 5.58576962461643e-09
3176 5.58479706924686e-09
3177 5.58399504413387e-09
3178 5.58280399687305e-09
3179 5.58207613465811e-09
3180 5.58106005854597e-09
3181 5.58023449670486e-09
3182 5.57930679434548e-09
3183 5.57826274061313e-09
3184 5.57765256203879e-09
3185 5.57665069678137e-09
3186 5.5758357930813e-09
3187 5.57485391183832e-09
3188 5.57390933408897e-09
3189 5.57337020978821e-09
3190 5.57227064490462e-09
3191 5.57145263258008e-09
3192 5.57046941906947e-09
3193 5.56971313514509e-09
3194 5.56874457657841e-09
3195 5.56807133733628e-09
3196 5.5670534848673e-09
3197 5.56596146950028e-09
3198 5.56521317918168e-09
3199 5.56434409659801e-09
3200 5.56331825052325e-09
3201 5.56245449701009e-09
3202 5.56170354215624e-09
3203 5.56091261927349e-09
3204 5.56002666129984e-09
3205 5.55918555633639e-09
3206 5.55836798810105e-09
3207 5.5572160206907e-09
3208 5.55651036293625e-09
3209 5.55554846570772e-09
3210 5.55462387197281e-09
3211 5.55379653377486e-09
3212 5.55298296234241e-09
3213 5.5521436337358e-09
3214 5.55103252253275e-09
3215 5.55025270188025e-09
3216 5.54953905052002e-09
3217 5.54874279856676e-09
3218 5.54775159145038e-09
3219 5.54685231080043e-09
3220 5.54615064984887e-09
3221 5.54501911054217e-09
3222 5.54437074029579e-09
3223 5.54339996128306e-09
3224 5.54267964858468e-09
3225 5.54168178013015e-09
3226 5.54077050907154e-09
3227 5.53984547124742e-09
3228 5.53912027356773e-09
3229 5.53831780436553e-09
3230 5.53737455888381e-09
3231 5.53661472224576e-09
3232 5.53562795602147e-09
3233 5.53498180622114e-09
3234 5.53386403367995e-09
3235 5.53310952611241e-09
3236 5.53227241795184e-09
3237 5.53159607008524e-09
3238 5.53061285657463e-09
3239 5.5296962564455e-09
3240 5.52879253490346e-09
3241 5.52795720309973e-09
3242 5.52736567627221e-09
3243 5.52647971829856e-09
3244 5.52552403831896e-09
3245 5.52478773840903e-09
3246 5.52369927575569e-09
3247 5.5229647522026e-09
3248 5.52196910419411e-09
3249 5.5212834304541e-09
3250 5.52061196756881e-09
3251 5.51952572536152e-09
3252 5.51896883749237e-09
3253 5.51788081892823e-09
3254 5.51708190243971e-09
3255 5.51604495413471e-09
3256 5.51548628990872e-09
3257 5.51451817543125e-09
3258 5.51357137723585e-09
3259 5.51291678974053e-09
3260 5.51193934938965e-09
3261 5.51116752234293e-09
3262 5.51018741745679e-09
3263 5.5094346862461e-09
3264 5.50875478566581e-09
3265 5.50782930375249e-09
3266 5.5070317195316e-09
3267 5.50631318319006e-09
3268 5.50529088982898e-09
3269 5.50448131519943e-09
3270 5.5038715807143e-09
3271 5.50271961330395e-09
3272 5.50207746030651e-09
3273 5.50119949593864e-09
3274 5.50034373603125e-09
3275 5.49951550965488e-09
3276 5.49880763145438e-09
3277 5.49778222946884e-09
3278 5.49710055253172e-09
3279 5.49615064571185e-09
3280 5.49527268134398e-09
3281 5.49466916410779e-09
3282 5.49380674286226e-09
3283 5.4929429893491e-09
3284 5.49223999612991e-09
3285 5.49143530648166e-09
3286 5.49056489163036e-09
3287 5.48957101997871e-09
3288 5.48892220564312e-09
3289 5.48811129874593e-09
3290 5.48731460270346e-09
3291 5.4864011111988e-09
3292 5.48575229686321e-09
3293 5.48494982766101e-09
3294 5.48396794641803e-09
3295 5.48332090843928e-09
3296 5.48235634667549e-09
3297 5.48163647806632e-09
3298 5.48087752960669e-09
3299 5.48020473445376e-09
3300 5.47908118875284e-09
3301 5.47844702936118e-09
3302 5.47765610647843e-09
3303 5.47683542961863e-09
3304 5.47597078792705e-09
3305 5.47518785865009e-09
3306 5.47444978238332e-09
3307 5.47353229407577e-09
3308 5.47271294948359e-09
3309 5.47186251864673e-09
3310 5.47105027948191e-09
3311 5.47034817444114e-09
3312 5.46953371483028e-09
3313 5.46864464823216e-09
3314 5.46795719813531e-09
3315 5.46709033599768e-09
3316 5.46625233965869e-09
3317 5.46548939439617e-09
3318 5.46470513285158e-09
3319 5.46417489033502e-09
3320 5.46335554574284e-09
3321 5.4624340606324e-09
3322 5.46160361380998e-09
3323 5.46083267494168e-09
3324 5.4599857968185e-09
3325 5.45911893468087e-09
3326 5.45847678168343e-09
3327 5.45760681092133e-09
3328 5.45678835450758e-09
3329 5.4560724827013e-09
3330 5.45520517647446e-09
3331 5.45464029499954e-09
3332 5.45341460878035e-09
3333 5.45287415221196e-09
3334 5.45199441148725e-09
3335 5.45136158436321e-09
3336 5.45046141553485e-09
3337 5.44966827220605e-09
3338 5.44891598508457e-09
3339 5.44806377789087e-09
3340 5.44710543337601e-09
3341 5.44641354238706e-09
3342 5.44561551407696e-09
3343 5.44501554955445e-09
3344 5.44402656288412e-09
3345 5.44323475182296e-09
3346 5.44247580336332e-09
3347 5.44169109772952e-09
3348 5.4409601268901e-09
3349 5.44016787173973e-09
3350 5.439197092727e-09
3351 5.43869083102777e-09
3352 5.43785327877799e-09
3353 5.43717870726823e-09
3354 5.43616529569135e-09
3355 5.43533840158261e-09
3356 5.43491296411958e-09
3357 5.4339990285257e-09
3358 5.43307976386131e-09
3359 5.43244116357755e-09
3360 5.43156675192336e-09
3361 5.43075007186644e-09
3362 5.42990719054615e-09
3363 5.42918732193698e-09
3364 5.42847322648754e-09
3365 5.42779154955042e-09
3366 5.42696865224457e-09
3367 5.42610267828536e-09
3368 5.42536993108911e-09
3369 5.42473221898376e-09
3370 5.42399103409252e-09
3371 5.42312950102541e-09
3372 5.4222542011928e-09
3373 5.42162625905007e-09
3374 5.4207527355743e-09
3375 5.42007549952928e-09
3376 5.41914602081306e-09
3377 5.41852873681137e-09
3378 5.4176148012175e-09
3379 5.41687272814784e-09
3380 5.41618216942652e-09
3381 5.41539391107904e-09
3382 5.4146438444036e-09
3383 5.41402966902638e-09
3384 5.41306732770863e-09
3385 5.41242828333566e-09
3386 5.41168043710627e-09
3387 5.4108295621802e-09
3388 5.41017941557698e-09
3389 5.40946087923544e-09
3390 5.40858291486757e-09
3391 5.40779687696613e-09
3392 5.40700328954813e-09
3393 5.4063709065133e-09
3394 5.40555422645639e-09
3395 5.40483657829327e-09
3396 5.40433209295088e-09
3397 5.40335953758131e-09
3398 5.40248246139186e-09
3399 5.40182742980733e-09
3400 5.40111955160683e-09
3401 5.40019451378271e-09
3402 5.39934763565952e-09
3403 5.39873745708519e-09
3404 5.39781153108265e-09
3405 5.39705258262302e-09
3406 5.39642863728318e-09
3407 5.39563549395439e-09
3408 5.39487698958396e-09
3409 5.39423172796205e-09
3410 5.39337818850072e-09
3411 5.39263478316343e-09
3412 5.3920157228049e-09
3413 5.39118527598248e-09
3414 5.39040501124077e-09
3415 5.38958522255939e-09
3416 5.38884448175736e-09
3417 5.38812283679135e-09
3418 5.38723066156876e-09
3419 5.38660982485339e-09
3420 5.38579580933174e-09
3421 5.38521804926972e-09
3422 5.38421129903099e-09
3423 5.38369882008283e-09
3424 5.38278133177528e-09
3425 5.38219779855353e-09
3426 5.38127986615677e-09
3427 5.38055822119077e-09
3428 5.37989119919757e-09
3429 5.3793018928161e-09
3430 5.37836752911858e-09
3431 5.37751443374646e-09
3432 5.37696909219676e-09
3433 5.37609556872098e-09
3434 5.37536859468446e-09
3435 5.37478062057062e-09
3436 5.37386268817386e-09
3437 5.37317657034464e-09
3438 5.3724487081297e-09
3439 5.37161648495044e-09
3440 5.37086464191816e-09
3441 5.37016164869897e-09
3442 5.36955546692752e-09
3443 5.36875743861742e-09
3444 5.36805577766586e-09
3445 5.36743760548575e-09
3446 5.36656230565313e-09
3447 5.36571231890548e-09
3448 5.36517896776445e-09
3449 5.36438804488171e-09
3450 5.36352917634986e-09
3451 5.36287192431928e-09
3452 5.36217115154614e-09
3453 5.36138466955549e-09
3454 5.36063371470163e-09
3455 5.35988986527514e-09
3456 5.35912114685289e-09
3457 5.35836042203641e-09
3458 5.35771382814687e-09
3459 5.35698863046719e-09
3460 5.35608579710356e-09
3461 5.35567368231682e-09
3462 5.35496802456237e-09
3463 5.354031884508e-09
3464 5.35325916928286e-09
3465 5.35256017286656e-09
3466 5.3519038090144e-09
3467 5.35108890531433e-09
3468 5.35026600800848e-09
3469 5.34956301478928e-09
3470 5.34900834736618e-09
3471 5.34815791652932e-09
3472 5.34720090428209e-09
3473 5.34669286622602e-09
3474 5.34595478995925e-09
3475 5.34515143257863e-09
3476 5.34446309430336e-09
3477 5.3435540436908e-09
3478 5.34297983634247e-09
3479 5.34229949167297e-09
3480 5.34179989131189e-09
3481 5.34076827207741e-09
3482 5.34013988584547e-09
3483 5.33932542623461e-09
3484 5.3386171039449e-09
3485 5.33816013614796e-09
3486 5.33717470219131e-09
3487 5.3365027952168e-09
3488 5.33576072214714e-09
3489 5.33510791100866e-09
3490 5.33443511585574e-09
3491 5.33357136234258e-09
3492 5.33275867908856e-09
3493 5.33227506593903e-09
3494 5.33137800573513e-09
3495 5.33078692299682e-09
3496 5.32972155298239e-09
3497 5.32911892392463e-09
3498 5.32853050572157e-09
3499 5.32781374573688e-09
3500 5.32714672374368e-09
3501 5.32632205008099e-09
3502 5.32562172139706e-09
3503 5.32495914029596e-09
3504 5.32422461674287e-09
3505 5.32357002924755e-09
3506 5.3225623908304e-09
3507 5.3218727202875e-09
3508 5.32134603048462e-09
3509 5.32063992864096e-09
3510 5.31981170226459e-09
3511 5.3190820636928e-09
3512 5.31837995865203e-09
3513 5.3175264191907e-09
3514 5.31723198804457e-09
3515 5.31627186717287e-09
3516 5.31551602733771e-09
3517 5.31486543664528e-09
3518 5.31415267346347e-09
3519 5.31346167065294e-09
3520 5.31277066784241e-09
3521 5.31203792064616e-09
3522 5.31152055671669e-09
3523 5.31054800134712e-09
3524 5.30983879087898e-09
3525 5.30937649401153e-09
3526 5.30850607916022e-09
3527 5.30779420415683e-09
3528 5.3070534633548e-09
3529 5.3064579397244e-09
3530 5.30556798494786e-09
3531 5.30473664994702e-09
3532 5.30422550326648e-09
3533 5.30349053562418e-09
3534 5.30262989073549e-09
3535 5.30203259074824e-09
3536 5.30149302235827e-09
3537 5.30061683434724e-09
3538 5.29995336506772e-09
3539 5.29919264025125e-09
3540 5.29846744257156e-09
3541 5.29769561552484e-09
3542 5.29708765739656e-09
3543 5.2964672647704e-09
3544 5.29567056872793e-09
3545 5.29496269052743e-09
3546 5.2941500072734e-09
3547 5.29360821843738e-09
3548 5.29277244254445e-09
3549 5.29206944932525e-09
3550 5.29163379781039e-09
3551 5.29055066422757e-09
3552 5.29014787531423e-09
3553 5.28949684053259e-09
3554 5.28860955029131e-09
3555 5.28784482867195e-09
3556 5.28712273961673e-09
3557 5.2866253597017e-09
3558 5.28585353265498e-09
3559 5.28512966724293e-09
3560 5.28433874436018e-09
3561 5.28369614727353e-09
3562 5.28284926915035e-09
3563 5.28235810648425e-09
3564 5.28163557333983e-09
3565 5.28078292205691e-09
3566 5.28005594802039e-09
3567 5.27948884609941e-09
3568 5.27852428433562e-09
3569 5.27806687244947e-09
3570 5.27696242258457e-09
3571 5.27655519277914e-09
3572 5.27578292164321e-09
3573 5.27495691571289e-09
3574 5.27445154219208e-09
3575 5.27386667670271e-09
3576 5.27299581776219e-09
3577 5.27212851153536e-09
3578 5.2716302434419e-09
3579 5.27091126301116e-09
3580 5.2702504582669e-09
3581 5.26947863122018e-09
3582 5.26883248141985e-09
3583 5.26800025824059e-09
3584 5.26736920747339e-09
3585 5.26674082124146e-09
3586 5.26584287285914e-09
3587 5.26529841948786e-09
3588 5.26473442619135e-09
3589 5.26406962464421e-09
3590 5.26321564109367e-09
3591 5.26258459032647e-09
3592 5.26188914662384e-09
3593 5.26116972210389e-09
3594 5.26066168404782e-09
3595 5.25991739053211e-09
3596 5.25914511939618e-09
3597 5.25821430841233e-09
3598 5.2577302511736e-09
3599 5.25714094479213e-09
3600 5.25621812741406e-09
3601 5.25558929709291e-09
3602 5.25492716008102e-09
3603 5.25439647347525e-09
3604 5.2534474548338e-09
3605 5.25293364361801e-09
3606 5.25209298274376e-09
3607 5.25141974350163e-09
3608 5.25073495794004e-09
3609 5.25022780806239e-09
3610 5.2493986935076e-09
3611 5.24870547025102e-09
3612 5.24801713197576e-09
3613 5.24702947757305e-09
3614 5.2466035960208e-09
3615 5.24591436956712e-09
3616 5.24508436683391e-09
3617 5.24430987525193e-09
3618 5.2437179043352e-09
3619 5.24313303884583e-09
3620 5.24228793707948e-09
3621 5.24184029515595e-09
3622 5.24108756394526e-09
3623 5.24039300842105e-09
3624 5.23989829304128e-09
3625 5.23916510175582e-09
3626 5.238596223478e-09
3627 5.23767917925966e-09
3628 5.23701082499883e-09
3629 5.23642418315262e-09
3630 5.23573051580684e-09
3631 5.23502663440922e-09
3632 5.23436050059445e-09
3633 5.23381515904475e-09
3634 5.23310728084425e-09
3635 5.23224796822319e-09
3636 5.23154897180689e-09
3637 5.23082244185957e-09
3638 5.23024379361914e-09
3639 5.22949550330054e-09
3640 5.22875520658772e-09
3641 5.2279789386489e-09
3642 5.2273656514501e-09
3643 5.22687626514085e-09
3644 5.22611731668121e-09
3645 5.22558130100492e-09
3646 5.22490806176279e-09
3647 5.22402121561072e-09
3648 5.22339549391404e-09
3649 5.22272047831507e-09
3650 5.22179144368806e-09
3651 5.22127674429385e-09
3652 5.22066478936267e-09
3653 5.21990184410015e-09
3654 5.21925569429982e-09
3655 5.21851584167621e-09
3656 5.21797360875098e-09
3657 5.21727150371021e-09
3658 5.21660981078753e-09
3659 5.21573495504413e-09
3660 5.21526866137378e-09
3661 5.21442400369665e-09
3662 5.21364107441968e-09
3663 5.2128887872982e-09
3664 5.21241139139761e-09
3665 5.21163601163721e-09
3666 5.2111306381164e-09
3667 5.21044807300086e-09
3668 5.20965937056417e-09
3669 5.2090882718403e-09
3670 5.20835952144694e-09
3671 5.20793008718101e-09
3672 5.20722531760498e-09
3673 5.20641574297542e-09
3674 5.20578247176218e-09
3675 5.20515586188708e-09
3676 5.20429743744444e-09
3677 5.20357668065685e-09
3678 5.2030513231216e-09
3679 5.20217069421847e-09
3680 5.20161425043852e-09
3681 5.20104093126861e-09
3682 5.20019893812673e-09
3683 5.19949905353201e-09
3684 5.19887999317348e-09
3685 5.19837861645556e-09
3686 5.19760900985489e-09
3687 5.19686782496365e-09
3688 5.19610576787954e-09
3689 5.19564613554735e-09
3690 5.19471710092034e-09
3691 5.19441867297132e-09
3692 5.19367482354482e-09
3693 5.19293541501042e-09
3694 5.19217380201553e-09
3695 5.19150189504103e-09
3696 5.19090592732141e-09
3697 5.1902473430232e-09
3698 5.18982101738175e-09
3699 5.1889128549476e-09
3700 5.18824494477599e-09
3701 5.18762899304193e-09
3702 5.18703613394678e-09
3703 5.18650633551943e-09
3704 5.18584553077517e-09
3705 5.18500753443618e-09
3706 5.18433340701563e-09
3707 5.1835700176639e-09
3708 5.18316634057214e-09
3709 5.18240295122041e-09
3710 5.18175546915245e-09
3711 5.18118392633937e-09
3712 5.18035525587379e-09
3713 5.17994402926547e-09
3714 5.17892972951017e-09
3715 5.17849985115504e-09
3716 5.17769294106074e-09
3717 5.177208883822e-09
3718 5.17639842101403e-09
3719 5.17557596779739e-09
3720 5.17498799368354e-09
3721 5.17433029756376e-09
3722 5.1737325534873e-09
3723 5.17312592762664e-09
3724 5.17246245834713e-09
3725 5.17186871107356e-09
3726 5.17114218112624e-09
3727 5.17071274686032e-09
3728 5.16987297416449e-09
3729 5.16937026517894e-09
3730 5.1687822910651e-09
3731 5.16784837145678e-09
3732 5.16736564648568e-09
3733 5.1666182443455e-09
3734 5.16592901789181e-09
3735 5.1651394272767e-09
3736 5.16457010490967e-09
3737 5.16383691362421e-09
3738 5.16311216003373e-09
3739 5.16273157558089e-09
3740 5.16205433953587e-09
3741 5.16113374260385e-09
3742 5.16064835309749e-09
3743 5.15994624805671e-09
3744 5.15923748167779e-09
3745 5.15862508265741e-09
3746 5.15793852073898e-09
3747 5.15755571584009e-09
3748 5.156749249835e-09
3749 5.1560413716345e-09
3750 5.15541653811624e-09
3751 5.1544635226719e-09
3752 5.15386799904149e-09
3753 5.15343323570505e-09
3754 5.15271514345272e-09
3755 5.15209253038051e-09
3756 5.15146458823779e-09
3757 5.15079223717407e-09
3758 5.14999598522081e-09
3759 5.14952080976627e-09
3760 5.14870013290647e-09
3761 5.1481290341826e-09
3762 5.14740694512739e-09
3763 5.14670217555135e-09
3764 5.14615239310956e-09
3765 5.14549958197108e-09
3766 5.14482945135342e-09
3767 5.14424414177483e-09
3768 5.14371256699064e-09
3769 5.14265252604673e-09
3770 5.14222842085132e-09
3771 5.1415960378165e-09
3772 5.14089082415126e-09
3773 5.14029885323453e-09
3774 5.13957854053615e-09
3775 5.13917619571203e-09
3776 5.138363512458e-09
3777 5.13793008138919e-09
3778 5.13706011062709e-09
3779 5.13639397681231e-09
3780 5.13562303794401e-09
3781 5.1350408369899e-09
3782 5.13450970629492e-09
3783 5.13391151812925e-09
3784 5.13319031725246e-09
3785 5.13244513555833e-09
3786 5.13198594731534e-09
3787 5.13110709476905e-09
3788 5.13049203121341e-09
3789 5.12973663546745e-09
3790 5.12925435458556e-09
3791 5.1285393709577e-09
3792 5.12807885044708e-09
3793 5.12724707135703e-09
3794 5.12665376817267e-09
3795 5.1258930433562e-09
3796 5.12518694151254e-09
3797 5.12471975966378e-09
3798 5.12409359387789e-09
3799 5.12334530355929e-09
3800 5.12253617301894e-09
3801 5.12202191771394e-09
3802 5.12155740040043e-09
3803 5.12091613558141e-09
3804 5.12013942355338e-09
3805 5.11943731851261e-09
3806 5.11881959042171e-09
3807 5.11842435102494e-09
3808 5.11753484033761e-09
3809 5.11706055306149e-09
3810 5.11641973233168e-09
3811 5.11568787331385e-09
3812 5.11529174573866e-09
3813 5.11436759609296e-09
3814 5.11349362852798e-09
3815 5.11313169582195e-09
3816 5.11248599011083e-09
3817 5.11185715978968e-09
3818 5.11108000367244e-09
3819 5.11052178353566e-09
3820 5.10998132696727e-09
3821 5.10923214847026e-09
3822 5.10856956736916e-09
3823 5.10793540797749e-09
3824 5.10725106650511e-09
3825 5.1067905459945e-09
3826 5.10598674452467e-09
3827 5.10554842847455e-09
3828 5.10488229465977e-09
3829 5.10421660493421e-09
3830 5.10357089922309e-09
3831 5.10262543329532e-09
3832 5.1021333824508e-09
3833 5.10153430610671e-09
3834 5.10101205719593e-09
3835 5.10017805765983e-09
3836 5.09959274808125e-09
3837 5.09890130118151e-09
3838 5.09822584149333e-09
3839 5.09801534320786e-09
3840 5.0971737941552e-09
3841 5.09669950687908e-09
3842 5.09589126451715e-09
3843 5.09539521686975e-09
3844 5.09471043130816e-09
3845 5.09406428150783e-09
3846 5.0932289497041e-09
3847 5.09280617677632e-09
3848 5.09227282563529e-09
3849 5.09176967256053e-09
3850 5.09093078804312e-09
3851 5.0903121717738e-09
3852 5.08969488777211e-09
3853 5.0891308944756e-09
3854 5.08824493650195e-09
3855 5.08785635844333e-09
3856 5.08716802016806e-09
3857 5.08623099193528e-09
3858 5.08589081960054e-09
3859 5.08514697017404e-09
3860 5.08453856795654e-09
3861 5.08406605703726e-09
3862 5.08319830672121e-09
3863 5.08257658182742e-09
3864 5.08203346072378e-09
3865 5.08148723099566e-09
3866 5.08077890870595e-09
3867 5.08023001444258e-09
3868 5.07928987758532e-09
3869 5.07879471811634e-09
3870 5.07841591002034e-09
3871 5.07753394884958e-09
3872 5.07704900343242e-09
3873 5.07629138724042e-09
3874 5.07552799788868e-09
3875 5.07514119618691e-09
3876 5.07459363419116e-09
3877 5.07372899249958e-09
3878 5.07313213660154e-09
3879 5.07240294211897e-09
3880 5.07198771870776e-09
3881 5.07108266489809e-09
3882 5.07064745747243e-09
3883 5.07011232997456e-09
3884 5.06916331133311e-09
3885 5.06880137862709e-09
3886 5.06803621291851e-09
3887 5.06740827077579e-09
3888 5.06699926461351e-09
3889 5.06625141838413e-09
3890 5.06560704494063e-09
3891 5.06508746056511e-09
3892 5.06434405522782e-09
3893 5.06372233033403e-09
3894 5.06311170767049e-09
3895 5.06253883258978e-09
3896 5.06199526739692e-09
3897 5.06113284615139e-09
3898 5.06059150140459e-09
3899 5.06004749212252e-09
3900 5.05933828165439e-09
3901 5.05846076137573e-09
3902 5.05809572004523e-09
3903 5.05750152868245e-09
3904 5.05672526074363e-09
3905 5.05610575629589e-09
3906 5.05538855222198e-09
3907 5.05496133840211e-09
3908 5.05424013752531e-09
3909 5.05375830073262e-09
3910 5.05311170684308e-09
3911 5.0524135986052e-09
3912 5.05146280360691e-09
3913 5.05111419357718e-09
3914 5.0505786219901e-09
3915 5.04980235405128e-09
3916 5.04926722655341e-09
3917 5.04880137697228e-09
3918 5.04797625922038e-09
3919 5.04739006146337e-09
3920 5.04680741642005e-09
3921 5.04629049657979e-09
3922 5.04550135005388e-09
3923 5.04485120345066e-09
3924 5.04411890034362e-09
3925 5.04377961618729e-09
3926 5.04310548876674e-09
3927 5.04254638045154e-09
3928 5.04182118277186e-09
3929 5.04113062405054e-09
3930 5.04051200778122e-09
3931 5.03985697619669e-09
3932 5.03927566342099e-09
3933 5.03867569889849e-09
3934 5.03809216567674e-09
3935 5.03747088487216e-09
3936 5.03686736763598e-09
3937 5.03605068757906e-09
3938 5.03548847063939e-09
3939 5.03495822812283e-09
3940 5.03417263431061e-09
3941 5.03361219372778e-09
3942 5.03296604392744e-09
3943 5.03258412720697e-09
3944 5.03172215005065e-09
3945 5.0311470545239e-09
3946 5.03069452761906e-09
3947 5.02992492101839e-09
3948 5.02911090549674e-09
3949 5.02860686424356e-09
3950 5.02797359303031e-09
3951 5.02728925155793e-09
3952 5.02684693870492e-09
3953 5.02609820429711e-09
3954 5.02544938996152e-09
3955 5.02489738707368e-09
3956 5.02438135541183e-09
3957 5.02371610977548e-09
3958 5.02298158622239e-09
3959 5.02244912325978e-09
3960 5.02206898289614e-09
3961 5.02130692581204e-09
3962 5.02068076002615e-09
3963 5.02013897119014e-09
3964 5.01940622399388e-09
3965 5.01882935211029e-09
3966 5.01826669108141e-09
3967 5.01767383198626e-09
3968 5.01705299527089e-09
3969 5.01641750361159e-09
3970 5.01567809507719e-09
3971 5.0153583508461e-09
3972 5.01458918833464e-09
3973 5.01395236440771e-09
3974 5.0134332241214e-09
3975 5.01264230123866e-09
3976 5.01207342296084e-09
3977 5.01149788334487e-09
3978 5.01084729265244e-09
3979 5.01036057087845e-09
3980 5.00959762561592e-09
3981 5.00907493261593e-09
3982 5.00825514393455e-09
3983 5.00771468736616e-09
3984 5.00732078023702e-09
3985 5.00648944523618e-09
3986 5.005849956774e-09
3987 5.00539831804758e-09
3988 5.00453412044521e-09
3989 5.00425656468906e-09
3990 5.00354024879357e-09
3991 5.00311037043843e-09
3992 5.00228436450811e-09
3993 5.00166796868484e-09
3994 5.0011386143467e-09
3995 5.00034458283949e-09
3996 4.99977392820483e-09
3997 4.99915309148946e-09
3998 4.99856600555404e-09
3999 4.99800778541726e-09
4000 4.99745489435099e-09
4001 4.99680519183698e-09
4002 4.99606844783784e-09
4003 4.99549335231109e-09
4004 4.99502306183786e-09
4005 4.99431873635103e-09
4006 4.9936814683349e-09
4007 4.99298336009701e-09
4008 4.99244379170705e-09
4009 4.99182251090247e-09
4010 4.99132424280901e-09
4011 4.99075714088804e-09
4012 4.99014296551081e-09
4013 4.9893351672381e-09
4014 4.98883112598492e-09
4015 4.98816365990251e-09
4016 4.98783059299512e-09
4017 4.9867656670699e-09
4018 4.98626695488724e-09
4019 4.98586061326023e-09
4020 4.98517938041232e-09
4021 4.98456875774878e-09
4022 4.98410290816764e-09
4023 4.98358154743528e-09
4024 4.98269248083716e-09
4025 4.98219643318976e-09
4026 4.98186114583632e-09
4027 4.98098806644975e-09
4028 4.98050090058655e-09
4029 4.97996444082105e-09
4030 4.97908914098844e-09
4031 4.9786050837497e-09
4032 4.9780686239842e-09
4033 4.97740471061547e-09
4034 4.97664087717453e-09
4035 4.97628249718218e-09
4036 4.9755843889443e-09
4037 4.97488716888483e-09
4038 4.97432361967753e-09
4039 4.97389107678714e-09
4040 4.97300467472428e-09
4041 4.97242824692989e-09
4042 4.97186691816864e-09
4043 4.97147523148556e-09
4044 4.97080154815421e-09
4045 4.97004393196221e-09
4046 4.96946528372177e-09
4047 4.96890795176341e-09
4048 4.96830043772434e-09
4049 4.96769558822052e-09
4050 4.96704188890362e-09
4051 4.96625318646693e-09
4052 4.96587926335224e-09
4053 4.96533569815938e-09
4054 4.96472107869295e-09
4055 4.96418373074903e-09
4056 4.96356333812287e-09
4057 4.96308949493596e-09
4058 4.96229901614242e-09
4059 4.96193841570403e-09
4060 4.96101204561228e-09
4061 4.96065588606598e-09
4062 4.96003993433192e-09
4063 4.9595114681722e-09
4064 4.95886576246107e-09
4065 4.95815299927926e-09
4066 4.95751351081708e-09
4067 4.9567479010193e-09
4068 4.95641128139823e-09
4069 4.95573271308558e-09
4070 4.95522067822662e-09
4071 4.95477125994626e-09
4072 4.95397323163616e-09
4073 4.95336260897261e-09
4074 4.95271690326149e-09
4075 4.95226126773218e-09
4076 4.95131313726915e-09
4077 4.95091656560476e-09
4078 4.95033791736432e-09
4079 4.94974816689364e-09
4080 4.94906426951047e-09
4081 4.94876672973987e-09
4082 4.94794427652323e-09
4083 4.947194653937e-09
4084 4.94672747208824e-09
4085 4.94625496116896e-09
4086 4.94555596475266e-09
4087 4.94490581814944e-09
4088 4.94425655972464e-09
4089 4.9437005600339e-09
4090 4.94312724086399e-09
4091 4.94237806236697e-09
4092 4.94174079435084e-09
4093 4.94135354855985e-09
4094 4.94053420396767e-09
4095 4.94000307327269e-09
4096 4.93941954005095e-09
4097 4.93865259798554e-09
4098 4.93835550230415e-09
4099 4.93762630782157e-09
4100 4.93704366277825e-09
4101 4.93650276212065e-09
4102 4.93592100525575e-09
4103 4.93537877233052e-09
4104 4.93459939576724e-09
4105 4.93439511473071e-09
4106 4.93345542196266e-09
4107 4.93330176709605e-09
4108 4.9324335726908e-09
4109 4.93161644854467e-09
4110 4.93095253517595e-09
4111 4.93050977823373e-09
4112 4.93011764746143e-09
4113 4.9293813475515e-09
4114 4.92880047886501e-09
4115 4.92826979225924e-09
4116 4.92727325607234e-09
4117 4.92708096544447e-09
4118 4.92628071668832e-09
4119 4.92561058607066e-09
4120 4.92522689299335e-09
4121 4.92462426393558e-09
4122 4.92367613347255e-09
4123 4.92344254254817e-09
4124 4.92277330010893e-09
4125 4.92241891691947e-09
4126 4.92171770005712e-09
4127 4.92098628512849e-09
4128 4.92052620870709e-09
4129 4.92001417384813e-09
4130 4.91908647148875e-09
4131 4.91884755149385e-09
4132 4.91803442415062e-09
4133 4.9175330474327e-09
4134 4.91721108275556e-09
4135 4.91655738343866e-09
4136 4.91585261386263e-09
4137 4.91490981247011e-09
4138 4.91440976801982e-09
4139 4.9138000335347e-09
4140 4.91324714246844e-09
4141 4.9128670021048e-09
4142 4.91208762554152e-09
4143 4.91136731284314e-09
4144 4.91092544407934e-09
4145 4.91043383732404e-09
4146 4.90974105815667e-09
4147 4.90916152173781e-09
4148 4.90863838464861e-09
4149 4.90801266295193e-09
4150 4.90739715530708e-09
4151 4.90674345599018e-09
4152 4.9063384466308e-09
4153 4.90573537348382e-09
4154 4.90495599692053e-09
4155 4.90428408994603e-09
4156 4.90371743211426e-09
4157 4.90341367509473e-09
4158 4.90246243600723e-09
4159 4.90223639459941e-09
4160 4.90151297327657e-09
4161 4.90096940808371e-09
4162 4.90037210809646e-09
4163 4.89998486230547e-09
4164 4.89918905444142e-09
4165 4.89852292062665e-09
4166 4.89809970360966e-09
4167 4.89742646436753e-09
4168 4.89668661174392e-09
4169 4.89651430513049e-09
4170 4.8956172449266e-09
4171 4.89515583623756e-09
4172 4.89442308904131e-09
4173 4.8937915941849e-09
4174 4.89328355612884e-09
4175 4.89263873859613e-09
4176 4.89214047050268e-09
4177 4.89162932382214e-09
4178 4.89073359588588e-09
4179 4.89016782623253e-09
4180 4.88971707568453e-09
4181 4.88914508878224e-09
4182 4.88845586232856e-09
4183 4.88791318531412e-09
4184 4.88726437097853e-09
4185 4.88684115396154e-09
4186 4.88606710646877e-09
4187 4.88577933666079e-09
4188 4.88508611340421e-09
4189 4.8844812639004e-09
4190 4.88393681052912e-09
4191 4.88321783009837e-09
4192 4.88267959397604e-09
4193 4.88195439629635e-09
4194 4.88164619838471e-09
4195 4.88086904226748e-09
4196 4.88027396272628e-09
4197 4.8797157425895e-09
4198 4.87927254155807e-09
4199 4.8783905803873e-09
4200 4.87810947191747e-09
4201 4.87753615274755e-09
4202 4.87694373774161e-09
4203 4.87632911827518e-09
4204 4.87579532304494e-09
4205 4.87519224989796e-09
4206 4.87469042909083e-09
4207 4.87416995653689e-09
4208 4.87356421885465e-09
4209 4.87293094764141e-09
4210 4.87227547196767e-09
4211 4.87166795792859e-09
4212 4.87106666113846e-09
4213 4.87055729081476e-09
4214 4.8701580546151e-09
4215 4.86952744793712e-09
4216 4.86888129813678e-09
4217 4.86838036550807e-09
4218 4.86745399541633e-09
4219 4.86711249081395e-09
4220 4.86647833142229e-09
4221 4.86577489411388e-09
4222 4.86540452371287e-09
4223 4.86497464535773e-09
4224 4.86430007384797e-09
4225 4.86345852479531e-09
4226 4.86283857625835e-09
4227 4.86260143262029e-09
4228 4.86180118386415e-09
4229 4.86144591249627e-09
4230 4.86088058693213e-09
4231 4.86043472136544e-09
4232 4.85975792940962e-09
4233 4.85921791693045e-09
4234 4.85835105479282e-09
4235 4.85794293680897e-09
4236 4.85732432053965e-09
4237 4.85675544226183e-09
4238 4.8559560816841e-09
4239 4.85546047812591e-09
4240 4.8549058107028e-09
4241 4.85447682052609e-09
4242 4.85357443125167e-09
4243 4.8532466934148e-09
4244 4.85259832316842e-09
4245 4.85202189537404e-09
4246 4.85145168482859e-09
4247 4.85076379064253e-09
4248 4.85019313600787e-09
4249 4.84966022895605e-09
4250 4.84923479149302e-09
4251 4.84866324867994e-09
4252 4.8478723257972e-09
4253 4.84725370952788e-09
4254 4.84669326894505e-09
4255 4.84641526909968e-09
4256 4.84574025350071e-09
4257 4.84497064690004e-09
4258 4.84436846193148e-09
4259 4.8439141586698e-09
4260 4.84346207585418e-09
4261 4.8428456800309e-09
4262 4.84208584339285e-09
4263 4.84139617284995e-09
4264 4.84109019538437e-09
4265 4.84056039695702e-09
4266 4.83996265288056e-09
4267 4.83925699512611e-09
4268 4.83871698264693e-09
4269 4.83843409782025e-09
4270 4.83777196080837e-09
4271 4.83698148201483e-09
4272 4.83663153971747e-09
4273 4.83572026865886e-09
4274 4.83524953409642e-09
4275 4.83482409663338e-09
4276 4.83430762088233e-09
4277 4.8336437075136e-09
4278 4.83318673971667e-09
4279 4.8323949286555e-09
4280 4.831894440116e-09
4281 4.8312993605748e-09
4282 4.83081397106844e-09
4283 4.83044360066742e-09
4284 4.82945283764025e-09
4285 4.82907314136582e-09
4286 4.82863615758333e-09
4287 4.82809525692574e-09
4288 4.8273540720345e-09
4289 4.82668482959525e-09
4290 4.82611639540664e-09
4291 4.82551198999204e-09
4292 4.82506168353325e-09
4293 4.82444306726393e-09
4294 4.82390305478475e-09
4295 4.82328932349674e-09
4296 4.82273021518154e-09
4297 4.82235407162079e-09
4298 4.82138684532174e-09
4299 4.82086015551886e-09
4300 4.82035655835489e-09
4301 4.81999506973807e-09
4302 4.81913575711701e-09
4303 4.81850292999297e-09
4304 4.81816408992586e-09
4305 4.8176702627245e-09
4306 4.81712802979928e-09
4307 4.81628914528187e-09
4308 4.81582818068205e-09
4309 4.81518203088172e-09
4310 4.81475659341868e-09
4311 4.81413842123857e-09
4312 4.81335860058607e-09
4313 4.81289896825388e-09
4314 4.81255391093782e-09
4315 4.81199480262262e-09
4316 4.81139750263537e-09
4317 4.81068562763198e-09
4318 4.81028727961075e-09
4319 4.8095545324145e-09
4320 4.80920192558187e-09
4321 4.80846518158273e-09
4322 4.80780348866006e-09
4323 4.80740158792514e-09
4324 4.80695305782319e-09
4325 4.80638684408063e-09
4326 4.80556394677478e-09
4327 4.80518513867878e-09
4328 4.80454831475186e-09
4329 4.80416284531771e-09
4330 4.8034278776754e-09
4331 4.80281592274423e-09
4332 4.80226658439165e-09
4333 4.80167994254543e-09
4334 4.80147788195495e-09
4335 4.80047779305437e-09
4336 4.80018469417587e-09
4337 4.79951145493374e-09
4338 4.79884931792185e-09
4339 4.79824713295329e-09
4340 4.79770090322518e-09
4341 4.79722439550301e-09
4342 4.79658091023794e-09
4343 4.79617057180803e-09
4344 4.79572470624134e-09
4345 4.79480632975537e-09
4346 4.79459760782674e-09
4347 4.79363171379532e-09
4348 4.79323025714962e-09
4349 4.79265382935523e-09
4350 4.79203876579959e-09
4351 4.79163198008337e-09
4352 4.79116257778855e-09
4353 4.79061679214965e-09
4354 4.7898978117189e-09
4355 4.78927653091432e-09
4356 4.78905537448782e-09
4357 4.78839945472487e-09
4358 4.78764183853286e-09
4359 4.78704098583194e-09
4360 4.78627182332048e-09
4361 4.78583483953798e-09
4362 4.78552619753714e-09
4363 4.78484762922449e-09
4364 4.78403050507836e-09
4365 4.78358241906562e-09
4366 4.7830988059161e-09
4367 4.78255524072324e-09
4368 4.78195483211152e-09
4369 4.78142281323812e-09
4370 4.78101780387874e-09
4371 4.78022643690679e-09
4372 4.77963713052532e-09
4373 4.77935824250153e-09
4374 4.77871964221777e-09
4375 4.77788031361115e-09
4376 4.77761163963919e-09
4377 4.77694062084311e-09
4378 4.77644679364175e-09
4379 4.77595873960013e-09
4380 4.77531880704873e-09
4381 4.77485428973523e-09
4382 4.77423789391196e-09
4383 4.7738204500547e-09
4384 4.77295891698759e-09
4385 4.77250727826117e-09
4386 4.7718549112119e-09
4387 4.77148853761378e-09
4388 4.77061634640563e-09
4389 4.77009720611932e-09
4390 4.76967842999443e-09
4391 4.76904782331644e-09
4392 4.76847628050336e-09
4393 4.767711558884e-09
4394 4.76750683375826e-09
4395 4.76680428462828e-09
4396 4.766223860031e-09
4397 4.76585260145157e-09
4398 4.76539341320859e-09
4399 4.76464245835473e-09
4400 4.76396699866655e-09
4401 4.76346873057309e-09
4402 4.76296335705229e-09
4403 4.76242245639469e-09
4404 4.76195749499198e-09
4405 4.76123318549071e-09
4406 4.76075312505486e-09
4407 4.7601687036547e-09
4408 4.75930228560628e-09
4409 4.75889505580085e-09
4410 4.75809525113391e-09
4411 4.75805839172949e-09
4412 4.75728612059356e-09
4413 4.75661554588669e-09
4414 4.75617056849842e-09
4415 4.7556061311127e-09
4416 4.75498307395128e-09
4417 4.75470818273038e-09
4418 4.75372763375503e-09
4419 4.75343941985784e-09
4420 4.75262940113907e-09
4421 4.75217554196661e-09
4422 4.75162176272192e-09
4423 4.75124606325039e-09
4424 4.75060524252058e-09
4425 4.74978278930394e-09
4426 4.74922456916715e-09
4427 4.7487938026336e-09
4428 4.74840078368288e-09
4429 4.74767070102189e-09
4430 4.74720440735155e-09
4431 4.74652672721732e-09
4432 4.74602490641018e-09
4433 4.74538452976958e-09
4434 4.7448667217509e-09
4435 4.74411754325388e-09
4436 4.74387995552661e-09
4437 4.743347492564e-09
4438 4.74280303919272e-09
4439 4.74204364664388e-09
4440 4.74150496643233e-09
4441 4.74098360569997e-09
4442 4.7407189285309e-09
4443 4.74007277873056e-09
4444 4.73956562885292e-09
4445 4.7388346580135e-09
4446 4.73824135482914e-09
4447 4.73785721766262e-09
4448 4.73726746719194e-09
4449 4.73653782862016e-09
4450 4.73579619963971e-09
4451 4.7352863852268e-09
4452 4.73480588070174e-09
4453 4.73422634428289e-09
4454 4.73362460340354e-09
4455 4.73324712757517e-09
4456 4.7328696517468e-09
4457 4.73207117934749e-09
4458 4.73139794010535e-09
4459 4.7311203843492e-09
4460 4.73026995351233e-09
4461 4.72981476207224e-09
4462 4.72916639182586e-09
4463 4.72883421309689e-09
4464 4.72839012388704e-09
4465 4.72752637037388e-09
4466 4.72725858458034e-09
4467 4.72674699381059e-09
4468 4.72605377055402e-09
4469 4.72542716067892e-09
4470 4.72477656998649e-09
4471 4.72456163080892e-09
4472 4.72395589312669e-09
4473 4.72325822897801e-09
4474 4.72266625806128e-09
4475 4.72214312097208e-09
4476 4.72155115005535e-09
4477 4.72091654657447e-09
4478 4.72043870658467e-09
4479 4.71980898808511e-09
4480 4.7193173813298e-09
4481 4.71865968521001e-09
4482 4.71840300164672e-09
4483 4.71746197661105e-09
4484 4.7173922546051e-09
4485 4.7164272487521e-09
4486 4.71610661634259e-09
4487 4.71563676995856e-09
4488 4.71506078625339e-09
4489 4.71439021154652e-09
4490 4.71393279966037e-09
4491 4.71305350302487e-09
4492 4.71306726979037e-09
4493 4.71202810103932e-09
4494 4.71199879115147e-09
4495 4.71135042090509e-09
4496 4.71082906017273e-09
4497 4.71011896152618e-09
4498 4.709714396256e-09
4499 4.70920058504021e-09
4500 4.70863303903002e-09
4501 4.70823202647352e-09
4502 4.70760719295527e-09
4503 4.70678207520336e-09
4504 4.70646011052622e-09
4505 4.70592009804705e-09
4506 4.70546002162564e-09
4507 4.70457584000883e-09
4508 4.70401895213968e-09
4509 4.70368011207256e-09
4510 4.70281547038098e-09
4511 4.70279859499101e-09
4512 4.70206451552713e-09
4513 4.70150318676588e-09
4514 4.70106531480496e-09
4515 4.70042094136147e-09
4516 4.69972905037253e-09
4517 4.69917260659258e-09
4518 4.69854111173618e-09
4519 4.69790517598767e-09
4520 4.69743799413891e-09
4521 4.6969685918441e-09
4522 4.69637706501658e-09
4523 4.69594363394776e-09
4524 4.69541294734199e-09
4525 4.69468108832416e-09
4526 4.69414107584498e-09
4527 4.69352956500302e-09
4528 4.69324845653318e-09
4529 4.6925685559529e-09
4530 4.69193972563176e-09
4531 4.69181626883142e-09
4532 4.69096050892404e-09
4533 4.69029926009057e-09
4534 4.69002969794019e-09
4535 4.68946170784079e-09
4536 4.68905403394615e-09
4537 4.68837324518745e-09
4538 4.68796823582807e-09
4539 4.68732030967089e-09
4540 4.68675542819597e-09
4541 4.6859907065766e-09
4542 4.68580019230558e-09
4543 4.68494532057662e-09
4544 4.68433913880517e-09
4545 4.68387240104562e-09
4546 4.68337724157664e-09
4547 4.68285366039822e-09
4548 4.68238514628183e-09
4549 4.68178740220537e-09
4550 4.68107064222067e-09
4551 4.68075400661405e-09
4552 4.6800030517602e-09
4553 4.67969396567014e-09
4554 4.67913974233625e-09
4555 4.67846694718332e-09
4556 4.67795491232437e-09
4557 4.67745042698198e-09
4558 4.67674166060306e-09
4559 4.67622252031674e-09
4560 4.67588145980358e-09
4561 4.67523664227087e-09
4562 4.67497507372627e-09
4563 4.67410643523181e-09
4564 4.67381733315619e-09
4565 4.6728931835105e-09
4566 4.67264937853429e-09
4567 4.67234562151475e-09
4568 4.67174876561671e-09
4569 4.67087080124884e-09
4570 4.67020422334485e-09
4571 4.6697148370356e-09
4572 4.66958738343237e-09
4573 4.66895011541624e-09
4574 4.66822358546892e-09
4575 4.66788918629391e-09
4576 4.66738736548677e-09
4577 4.66667327003734e-09
4578 4.66620342365331e-09
4579 4.66555505340693e-09
4580 4.66531213660915e-09
4581 4.66454208591927e-09
4582 4.66387550801528e-09
4583 4.66357308326337e-09
4584 4.66282168432031e-09
4585 4.66229321816058e-09
4586 4.6618384708097e-09
4587 4.66132688003995e-09
4588 4.66084948413936e-09
4589 4.66018157396775e-09
4590 4.66003191590403e-09
4591 4.65928584603148e-09
4592 4.65842742158884e-09
4593 4.65793714710117e-09
4594 4.65761784695928e-09
4595 4.65685312533992e-09
4596 4.65656091463984e-09
4597 4.65573224417426e-09
4598 4.65507321578684e-09
4599 4.6547783405515e-09
4600 4.65445459951752e-09
4601 4.65396832183274e-09
4602 4.65341765121252e-09
4603 4.65265603821763e-09
4604 4.65186777987014e-09
4605 4.65187088849461e-09
4606 4.65103910940456e-09
4607 4.65034055707747e-09
4608 4.65013094697042e-09
4609 4.64942484512676e-09
4610 4.64914151621088e-09
4611 4.64848692871556e-09
4612 4.64783855846918e-09
4613 4.64732163862891e-09
4614 4.64710225855924e-09
4615 4.64637572861193e-09
4616 4.64584992698747e-09
4617 4.64531435540039e-09
4618 4.64484628537321e-09
4619 4.64400784494501e-09
4620 4.64339544592463e-09
4621 4.6429344813248e-09
4622 4.64255611731801e-09
4623 4.64186555859669e-09
4624 4.64130422983544e-09
4625 4.64125760046841e-09
4626 4.64026017610308e-09
4627 4.63974325626282e-09
4628 4.63910332371142e-09
4629 4.63876981271483e-09
4630 4.6383177298992e-09
4631 4.63770044589751e-09
4632 4.63719640464433e-09
4633 4.6369064143903e-09
4634 4.63624294511078e-09
4635 4.63579530318725e-09
4636 4.63515936743875e-09
4637 4.63472593636993e-09
4638 4.63401939043706e-09
4639 4.63331462086103e-09
4640 4.63290650287718e-09
4641 4.63253435611932e-09
4642 4.63214266943623e-09
4643 4.63128158045834e-09
4644 4.63085081392478e-09
4645 4.63029836694773e-09
4646 4.62982052695793e-09
4647 4.62946259105479e-09
4648 4.62862281835896e-09
4649 4.62813476431734e-09
4650 4.62764138120519e-09
4651 4.62719018656799e-09
4652 4.62656668531736e-09
4653 4.62603111373028e-09
4654 4.62577709470224e-09
4655 4.62487648178467e-09
4656 4.62435112424942e-09
4657 4.62381244403787e-09
4658 4.62336924300644e-09
4659 4.62277283119761e-09
4660 4.62256277700135e-09
4661 4.6216590554593e-09
4662 4.62124249978046e-09
4663 4.62079707830299e-09
4664 4.62014426716451e-09
4665 4.61954963171252e-09
4666 4.61924098971167e-09
4667 4.61843630006342e-09
4668 4.6182773161263e-09
4669 4.61726124001416e-09
4670 4.61686822106344e-09
4671 4.61658622441519e-09
4672 4.61603555379497e-09
4673 4.61543869789693e-09
4674 4.61469706891648e-09
4675 4.61422811071088e-09
4676 4.61376403748659e-09
4677 4.61319249467351e-09
4678 4.61308768961999e-09
4679 4.61230298398618e-09
4680 4.61173987886809e-09
4681 4.61125670980778e-09
4682 4.61076865576615e-09
4683 4.61047244826318e-09
4684 4.60969262761068e-09
4685 4.6090664618248e-09
4686 4.60846827365913e-09
4687 4.60808191604656e-09
4688 4.60745663843909e-09
4689 4.60696236714853e-09
4690 4.60653382106102e-09
4691 4.6056629621205e-09
4692 4.60548443825815e-09
4693 4.60495597209842e-09
4694 4.60440663374584e-09
4695 4.60371651911373e-09
4696 4.60331683882487e-09
4697 4.60286120329556e-09
4698 4.6023265198869e-09
4699 4.60193483320381e-09
4700 4.60112925537715e-09
4701 4.60051108319703e-09
4702 4.60007543168217e-09
4703 4.59972637756323e-09
4704 4.59934534902118e-09
4705 4.59878624070598e-09
4706 4.59796867247064e-09
4707 4.59777460548594e-09
4708 4.59683535680711e-09
4709 4.59627047533218e-09
4710 4.59584681422598e-09
4711 4.59576776634663e-09
4712 4.59494531312998e-09
4713 4.59401849894903e-09
4714 4.5937422754605e-09
4715 4.59305349309602e-09
4716 4.59291893406544e-09
4717 4.59200411029315e-09
4718 4.59160753862875e-09
4719 4.59085613968568e-09
4720 4.59079085857184e-09
4721 4.5901220602218e-09
4722 4.58963134164492e-09
4723 4.58896121102725e-09
4724 4.58861926233567e-09
4725 4.58797400071376e-09
4726 4.58766802324817e-09
4727 4.58682114512499e-09
4728 4.58658089286246e-09
4729 4.58576288053791e-09
4730 4.585360979803e-09
4731 4.5852437402516e-09
4732 4.58450077900352e-09
4733 4.58407312109443e-09
4734 4.58350157828136e-09
4735 4.58307880535358e-09
4736 4.58243754053456e-09
4737 4.58201077080389e-09
4738 4.58155291482853e-09
4739 4.58082327625675e-09
4740 4.5803294490554e-09
4741 4.57989912661105e-09
4742 4.57968907241479e-09
4743 4.57874316239781e-09
4744 4.57836080158813e-09
4745 4.57780524598661e-09
4746 4.57752413751678e-09
4747 4.57672033604695e-09
4748 4.57638504869351e-09
4749 4.57581572632648e-09
4750 4.57528370745308e-09
4751 4.57471349690763e-09
4752 4.57424187416677e-09
4753 4.57358773076066e-09
4754 4.57297533174028e-09
4755 4.57261029040978e-09
4756 4.57217153027045e-09
4757 4.57152671273775e-09
4758 4.57122650843189e-09
4759 4.57033610956614e-09
4760 4.56978721530277e-09
4761 4.56931736891875e-09
4762 4.56899451606319e-09
4763 4.56857085495699e-09
4764 4.56783366686864e-09
4765 4.56759696731979e-09
4766 4.56693394212948e-09
4767 4.56638593604453e-09
4768 4.56595783404623e-09
4769 4.56515136804114e-09
4770 4.56498661094429e-09
4771 4.56454740671575e-09
4772 4.56389281922043e-09
4773 4.56310900176504e-09
4774 4.5626564748602e-09
4775 4.56222259970218e-09
4776 4.56172166707347e-09
4777 4.56139925830712e-09
4778 4.56093340872599e-09
4779 4.56021131967077e-09
4780 4.55965176726636e-09
4781 4.55899451523578e-09
4782 4.55874227256459e-09
4783 4.55839765933774e-09
4784 4.55792292797241e-09
4785 4.5571604267991e-09
4786 4.55680559952043e-09
4787 4.55619897365978e-09
4788 4.55581350422563e-09
4789 4.55522286557652e-09
4790 4.55463666781952e-09
4791 4.55439375102173e-09
4792 4.55364856932761e-09
4793 4.55335102955701e-09
4794 4.55254101083824e-09
4795 4.55194015813731e-09
4796 4.55158710721548e-09
4797 4.55112969532934e-09
4798 4.55052351355789e-09
4799 4.54979032227243e-09
4800 4.54927739923505e-09
4801 4.54899717894364e-09
4802 4.54842563613056e-09
4803 4.54790738402266e-09
4804 4.54758364298868e-09
4805 4.54686421846873e-09
4806 4.54636905899974e-09
4807 4.54595072696407e-09
4808 4.5452805963464e-09
4809 4.54499105018158e-09
4810 4.54426407614505e-09
4811 4.5438603990533e-09
4812 4.54329951438126e-09
4813 4.54283721751381e-09
4814 4.54219595269478e-09
4815 4.54175985709071e-09
4816 4.54165771657244e-09
4817 4.5408037330219e-09
4818 4.5405239568197e-09
4819 4.53974147163194e-09
4820 4.53918591603042e-09
4821 4.53859350102448e-09
4822 4.53824133828107e-09
4823 4.53774839925813e-09
4824 4.53711246350963e-09
4825 4.53681714418508e-09
4826 4.53609416695144e-09
4827 4.53579040993191e-09
4828 4.53511583842214e-09
4829 4.53467530192597e-09
4830 4.53402870803643e-09
4831 4.53339499273397e-09
4832 4.53302151370849e-09
4833 4.53264137334486e-09
4834 4.53200010852584e-09
4835 4.53134729738736e-09
4836 4.53109150200248e-09
4837 4.53066784089629e-09
4838 4.52999371347573e-09
4839 4.52969750597276e-09
4840 4.52898030189886e-09
4841 4.52839854503395e-09
4842 4.52789228333472e-09
4843 4.5274584081767e-09
4844 4.5269761272948e-09
4845 4.52607462619881e-09
4846 4.52584725252336e-09
4847 4.52569226538913e-09
4848 4.52482629142992e-09
4849 4.52423298824556e-09
4850 4.52365567227275e-09
4851 4.52333992484455e-09
4852 4.5226795641895e-09
4853 4.5221337785506e-09
4854 4.52190151989385e-09
4855 4.52140591633565e-09
4856 4.52093207314874e-09
4857 4.52032011821757e-09
4858 4.51978099391681e-09
4859 4.51922721467213e-09
4860 4.51878623408675e-09
4861 4.51804194057104e-09
4862 4.51781234644955e-09
4863 4.51751080987606e-09
4864 4.51677273360929e-09
4865 4.51619497354727e-09
4866 4.51561676939605e-09
4867 4.51525528077923e-09
4868 4.51507675691687e-09
4869 4.51430981485146e-09
4870 4.51360149256175e-09
4871 4.5132160231276e-09
4872 4.5124179948175e-09
4873 4.51227410991351e-09
4874 4.51196546791266e-09
4875 4.51137482926356e-09
4876 4.51079174013103e-09
4877 4.51022286185321e-09
4878 4.50952075681244e-09
4879 4.50934045659324e-09
4880 4.5087120703613e-09
4881 4.50819914732392e-09
4882 4.50795711870455e-09
4883 4.50709203292377e-09
4884 4.50680159858052e-09
4885 4.50617809732989e-09
4886 4.50586590261537e-09
4887 4.50543025110051e-09
4888 4.50472725788131e-09
4889 4.50420012398922e-09
4890 4.50387416250919e-09
4891 4.50313786259926e-09
4892 4.50287940267913e-09
4893 4.50237847005042e-09
4894 4.50163994969444e-09
4895 4.50105641647269e-09
4896 4.50042669797313e-09
4897 4.50013670771909e-09
4898 4.49967796356532e-09
4899 4.49893233778198e-09
4900 4.49844117511589e-09
4901 4.49815118486185e-09
4902 4.49750547915073e-09
4903 4.49708537075821e-09
4904 4.49654402601141e-09
4905 4.49609593999867e-09
4906 4.49564385718304e-09
4907 4.49503989585764e-09
4908 4.49456605267073e-09
4909 4.49411663439037e-09
4910 4.49354775611255e-09
4911 4.49291670534535e-09
4912 4.49227322008028e-09
4913 4.49197967711257e-09
4914 4.49132553370646e-09
4915 4.49116033252039e-09
4916 4.49033032978718e-09
4917 4.49023529469628e-09
4918 4.48953008103103e-09
4919 4.48909398542696e-09
4920 4.48856285473198e-09
4921 4.48821690923751e-09
4922 4.48775727690531e-09
4923 4.48711423572945e-09
4924 4.48636328087559e-09
4925 4.4859760350846e-09
4926 4.48553461041001e-09
4927 4.48497372573797e-09
4928 4.48478676418063e-09
4929 4.48394388286033e-09
4930 4.48369652517044e-09
4931 4.48295800481446e-09
4932 4.48256898266663e-09
4933 4.48220927040666e-09
4934 4.48172254863266e-09
4935 4.48086190374397e-09
4936 4.48052883683658e-09
4937 4.48009052078646e-09
4938 4.47978543149929e-09
4939 4.47877068765479e-09
4940 4.47862147368028e-09
4941 4.47831238759022e-09
4942 4.47779191503628e-09
4943 4.47698855765566e-09
4944 4.47640591261234e-09
4945 4.47592940489017e-09
4946 4.47559012073384e-09
4947 4.47504477918415e-09
4948 4.47442127793352e-09
4949 4.47389680857668e-09
4950 4.47347270338128e-09
4951 4.47300818606777e-09
4952 4.47257075819607e-09
4953 4.472170633818e-09
4954 4.47138059911367e-09
4955 4.47093695399303e-09
4956 4.47077486143144e-09
4957 4.47007275639066e-09
4958 4.46964820710605e-09
4959 4.46921299968039e-09
4960 4.46872494563877e-09
4961 4.46810988208313e-09
4962 4.46727188574414e-09
4963 4.46698189549011e-09
4964 4.46613723781297e-09
4965 4.46578996005087e-09
4966 4.46547865351477e-09
4967 4.46497772088605e-09
4968 4.46468817472123e-09
4969 4.46393455533212e-09
4970 4.46350201244172e-09
4971 4.46283721089458e-09
4972 4.46255654651395e-09
4973 4.46191217307046e-09
4974 4.46148851196426e-09
4975 4.46086767524889e-09
4976 4.46063319614609e-09
4977 4.46017445199232e-09
4978 4.45921122249615e-09
4979 4.45887904376718e-09
4980 4.45850822927696e-09
4981 4.45783543412404e-09
4982 4.45735359733135e-09
4983 4.4568961854452e-09
4984 4.4563819301402e-09
4985 4.45613457245031e-09
4986 4.45530190518184e-09
4987 4.45475167865084e-09
4988 4.45450787367463e-09
4989 4.45395764714362e-09
4990 4.45358150358288e-09
4991 4.4528292164614e-09
4992 4.45248904412665e-09
4993 4.45178427455062e-09
4994 4.45144054950219e-09
4995 4.45093339962455e-09
4996 4.45056702602642e-09
4997 4.45001413496016e-09
4998 4.44931158583017e-09
4999 4.44901004925669e-09
};
\addlegendentry{Test}

\nextgroupplot[
title={Tanh/Tanh $\hy$},
ymin=3.96106582785004e-09, ymax=1e-05,
]
\addplot [semithick, black, dashed]
table {%
0 0.00654343567288015
1 0.000485908280519652
2 0.000207422003950342
3 0.000157615580404126
4 7.13584981224358e-05
5 2.28453118844527e-05
6 1.69666370386778e-05
7 1.65288759855855e-05
8 1.63117186082928e-05
9 1.60866644785358e-05
10 1.58059715671186e-05
11 1.54119263284542e-05
12 1.48282182186108e-05
13 1.39421518017286e-05
14 1.25994214990328e-05
15 1.0670800851301e-05
16 8.23958302328265e-06
17 5.77978367027754e-06
18 3.92833707920204e-06
19 2.90424916337884e-06
20 2.47166672740562e-06
21 2.32685799853627e-06
22 2.28410224540454e-06
23 2.26542131801821e-06
24 2.25025970055626e-06
25 2.23557102229321e-06
26 2.22120366889911e-06
27 2.20720070982949e-06
28 2.19355907121965e-06
29 2.18025100001285e-06
30 2.16726092809338e-06
31 2.15455599883896e-06
32 2.14210295983008e-06
33 2.1298539515513e-06
34 2.117754979718e-06
35 2.10579718061865e-06
36 2.09395216067065e-06
37 2.08218557888173e-06
38 2.07047087546641e-06
39 2.05878567444628e-06
40 2.04710556050358e-06
41 2.03539925949414e-06
42 2.02363844017128e-06
43 2.01178690603854e-06
44 1.9998051397927e-06
45 1.98765137003676e-06
46 1.97528167222316e-06
47 1.96264442334382e-06
48 1.94969178154381e-06
49 1.93636949588694e-06
50 1.92261817467099e-06
51 1.90837259320631e-06
52 1.89355616657494e-06
53 1.87807652605798e-06
54 1.86182353935394e-06
55 1.84467842287006e-06
56 1.826512008968e-06
57 1.80717322729151e-06
58 1.78647779338803e-06
59 1.76420641691877e-06
60 1.74014592991512e-06
61 1.71401796528414e-06
62 1.68555479017485e-06
63 1.65439963240033e-06
64 1.62018014234633e-06
65 1.58255340058489e-06
66 1.54114349800949e-06
67 1.49545071685608e-06
68 1.44521948431375e-06
69 1.39046002982113e-06
70 1.33123557488446e-06
71 1.26821421299184e-06
72 1.20262092949019e-06
73 1.13566727632275e-06
74 1.06912531316183e-06
75 1.00473530963541e-06
76 9.43728451293424e-07
77 8.87000563501772e-07
78 8.34961619414543e-07
79 7.87461629212061e-07
80 7.4437716289566e-07
81 7.05477165315216e-07
82 6.70521081513442e-07
83 6.39114614102709e-07
84 6.1086666972443e-07
85 5.85695171032796e-07
86 5.63292544704552e-07
87 5.43434041961888e-07
88 5.25889986707995e-07
89 5.10459320146595e-07
90 4.96889565578584e-07
91 4.84971236652143e-07
92 4.74529361291331e-07
93 4.65371475881327e-07
94 4.57362005390038e-07
95 4.50280618004228e-07
96 4.44024627961781e-07
97 4.38492238890475e-07
98 4.33607730265351e-07
99 4.29236485491202e-07
100 4.25284243254609e-07
101 4.21693747167495e-07
102 4.1841797533948e-07
103 4.15401190796416e-07
104 4.12599762457333e-07
105 4.09978367468256e-07
106 4.07510258073174e-07
107 4.05177839880366e-07
108 4.02962180444888e-07
109 4.00850035136457e-07
110 3.98830905432135e-07
111 3.96893529918785e-07
112 3.95030828403975e-07
113 3.93235835105088e-07
114 3.91500794531652e-07
115 3.89820894140414e-07
116 3.88191026091889e-07
117 3.86605555554809e-07
118 3.85061056213942e-07
119 3.83555210852649e-07
120 3.82086712701124e-07
121 3.80657602834233e-07
122 3.79269002486637e-07
123 3.77920333367143e-07
124 3.76602971179807e-07
125 3.7531351818032e-07
126 3.74049646708841e-07
127 3.72809808991903e-07
128 3.71592730216364e-07
129 3.70396633783798e-07
130 3.69220572176587e-07
131 3.6806381982224e-07
132 3.66924148268666e-07
133 3.65802939112925e-07
134 3.64697423986371e-07
135 3.63608003251414e-07
136 3.62533653200714e-07
137 3.61472450784106e-07
138 3.60425725672187e-07
139 3.59393445526379e-07
140 3.58374829511732e-07
141 3.57368513553169e-07
142 3.56376989387286e-07
143 3.55397220124232e-07
144 3.54429519155985e-07
145 3.53471457911425e-07
146 3.52524334344295e-07
147 3.51585602933468e-07
148 3.50655021840396e-07
149 3.49729907714646e-07
150 3.48808957058466e-07
151 3.47884689032796e-07
152 3.46922544293804e-07
153 3.45965674638293e-07
154 3.45106049630672e-07
155 3.44216265547814e-07
156 3.43336050310228e-07
157 3.42461601885447e-07
158 3.41593299680198e-07
159 3.40730502047748e-07
160 3.39874938530826e-07
161 3.39024043189085e-07
162 3.38181624485401e-07
163 3.37342835329935e-07
164 3.3650968060428e-07
165 3.35680949767436e-07
166 3.34856202321987e-07
167 3.34035885060757e-07
168 3.33219544403462e-07
169 3.32406706760935e-07
170 3.31596697733971e-07
171 3.3079267729974e-07
172 3.29989642537498e-07
173 3.29192634497488e-07
174 3.28396726185787e-07
175 3.27608298794857e-07
176 3.26817724571882e-07
177 3.2603876213777e-07
178 3.25252518813102e-07
179 3.24484984002282e-07
180 3.23698432209696e-07
181 3.22945399817698e-07
182 3.22157589348393e-07
183 3.21418464626078e-07
184 3.20628794296951e-07
185 3.19899834835269e-07
186 3.19117342627351e-07
187 3.18390289990589e-07
188 3.17618980595569e-07
189 3.16890480029031e-07
190 3.16131414415111e-07
191 3.15400256297949e-07
192 3.14651438262814e-07
193 3.13918219841725e-07
194 3.13179027497057e-07
195 3.12445891680113e-07
196 3.11712043778556e-07
197 3.10981938930865e-07
198 3.10252355401275e-07
199 3.09525433879188e-07
200 3.08799212573874e-07
201 3.08075061667523e-07
202 3.07351905515674e-07
203 3.0663059314584e-07
204 3.05909717122077e-07
205 3.05191485571399e-07
206 3.04472894930186e-07
207 3.03754995574224e-07
208 3.03038211212581e-07
209 3.02322759603513e-07
210 3.0160823698111e-07
211 3.00893162044957e-07
212 3.00180042879461e-07
213 2.99466500337076e-07
214 2.9875409537361e-07
215 2.98043207335397e-07
216 2.9733232017648e-07
217 2.96623261061058e-07
218 2.95914617140625e-07
219 2.95207125171615e-07
220 2.94500957895849e-07
221 2.93796146975112e-07
222 2.93092035107634e-07
223 2.92389998058695e-07
224 2.91689033600662e-07
225 2.90989939241371e-07
226 2.90293120281504e-07
227 2.8959689511332e-07
228 2.88900860075714e-07
229 2.88205797009056e-07
230 2.87512183918537e-07
231 2.86819817150175e-07
232 2.86128356352222e-07
233 2.85438533031979e-07
234 2.84749359392933e-07
235 2.84061189626428e-07
236 2.83373693820366e-07
237 2.82688323197888e-07
238 2.82003440533529e-07
239 2.8131985223645e-07
240 2.80637323241351e-07
241 2.79956116083113e-07
242 2.79275953444014e-07
243 2.78597744701514e-07
244 2.77919576022967e-07
245 2.77243200097033e-07
246 2.76569059673548e-07
247 2.75894991759174e-07
248 2.75222664896546e-07
249 2.74552201592115e-07
250 2.73881326403824e-07
251 2.73214062195493e-07
252 2.72545827916915e-07
253 2.71880231364818e-07
254 2.71215257267698e-07
255 2.70552262209733e-07
256 2.69890005375473e-07
257 2.69229412335648e-07
258 2.68567616190651e-07
259 2.67911075015448e-07
260 2.67251030171778e-07
261 2.66599237949805e-07
262 2.65937300515162e-07
263 2.65294282209894e-07
264 2.64626396074164e-07
265 2.63999098761403e-07
266 2.6331053020634e-07
267 2.62726188602791e-07
268 2.61978885979453e-07
269 2.61480665873393e-07
270 2.6064629784095e-07
271 2.60237155825571e-07
272 2.59361444123307e-07
273 2.58999699609852e-07
274 2.58185796182531e-07
275 2.57729135173079e-07
276 2.56987841933842e-07
277 2.56750742385847e-07
278 2.55610515845106e-07
279 2.55465471743044e-07
280 2.54382805022324e-07
281 2.54202267119474e-07
282 2.53144833933128e-07
283 2.52935429711165e-07
284 2.51917419709535e-07
285 2.51670024714379e-07
286 2.50681940528352e-07
287 2.50406462988018e-07
288 2.49455537904453e-07
289 2.4917096630972e-07
290 2.48252998704679e-07
291 2.47938017249538e-07
292 2.47057162011011e-07
293 2.46708501997261e-07
294 2.45870881522947e-07
295 2.45482539285469e-07
296 2.44695966372888e-07
297 2.44261895036679e-07
298 2.4352926588822e-07
299 2.43049716943666e-07
300 2.42369463347991e-07
301 2.41849556799956e-07
302 2.41210650944623e-07
303 2.40663764028426e-07
304 2.40052424756776e-07
305 2.39489588747865e-07
306 2.38895252559601e-07
307 2.38324340413598e-07
308 2.37744068629198e-07
309 2.37166125034882e-07
310 2.3659776758933e-07
311 2.36013066209573e-07
312 2.35458859589421e-07
313 2.34868595221371e-07
314 2.34326722230449e-07
315 2.33730530287524e-07
316 2.33207710401473e-07
317 2.32606544398273e-07
318 2.32098612314324e-07
319 2.31497655856394e-07
320 2.31006551127066e-07
321 2.30405722920324e-07
322 2.29929848841515e-07
323 2.29332777503366e-07
324 2.28867206917727e-07
325 2.28271482190401e-07
326 2.27808194830814e-07
327 2.27214381531837e-07
328 2.2675271341388e-07
329 2.26166697360419e-07
330 2.25704237100999e-07
331 2.25131501872866e-07
332 2.24662055149594e-07
333 2.24104979605677e-07
334 2.23630234236438e-07
335 2.23088263652471e-07
336 2.22607321643586e-07
337 2.22080438218697e-07
338 2.21591253615649e-07
339 2.21077630476074e-07
340 2.20584503468402e-07
341 2.20077854980794e-07
342 2.19579505015943e-07
343 2.19076213370251e-07
344 2.18575313136959e-07
345 2.18075076926283e-07
346 2.17576729304447e-07
347 2.17080177741025e-07
348 2.16586935366969e-07
349 2.16093505787995e-07
350 2.1560321666847e-07
351 2.15114347476231e-07
352 2.1462666975669e-07
353 2.14139794342927e-07
354 2.13654950691478e-07
355 2.13169798187529e-07
356 2.12686073074231e-07
357 2.12202257154459e-07
358 2.1172077878262e-07
359 2.11240427614001e-07
360 2.10760426584855e-07
361 2.10282472173695e-07
362 2.09808594819449e-07
363 2.09335428228208e-07
364 2.08866207214164e-07
365 2.08396717728654e-07
366 2.07931096295688e-07
367 2.07466985289173e-07
368 2.07002263066691e-07
369 2.065412411838e-07
370 2.06080855801538e-07
371 2.05621229704356e-07
372 2.05163119900931e-07
373 2.04706189015802e-07
374 2.04249882640894e-07
375 2.03794316285588e-07
376 2.0334069215e-07
377 2.02887090011039e-07
378 2.02435151773095e-07
379 2.01983189713317e-07
380 2.01532586169506e-07
381 2.01082712049327e-07
382 2.00632185448413e-07
383 2.00183520102648e-07
384 1.99735407355739e-07
385 1.99286234018103e-07
386 1.98837988977019e-07
387 1.98390685155481e-07
388 1.97941589735962e-07
389 1.97488608009166e-07
390 1.97023151987885e-07
391 1.96549280763847e-07
392 1.96098338066442e-07
393 1.95667065535865e-07
394 1.95225323280646e-07
395 1.94789039999144e-07
396 1.94354390741225e-07
397 1.93926770789332e-07
398 1.93504265896927e-07
399 1.93088708395628e-07
400 1.92652328695253e-07
401 1.92200173233736e-07
402 1.91752935390532e-07
403 1.91309916386828e-07
404 1.90866160694192e-07
405 1.90422613313146e-07
406 1.89978434049998e-07
407 1.89535246550854e-07
408 1.89091257378315e-07
409 1.88647408840836e-07
410 1.88202989032327e-07
411 1.87757702891922e-07
412 1.87314143301087e-07
413 1.86868340465551e-07
414 1.86422393489316e-07
415 1.85975031335772e-07
416 1.85529953551367e-07
417 1.85081949108046e-07
418 1.84634740622691e-07
419 1.84186540169051e-07
420 1.83738712733472e-07
421 1.83290373234613e-07
422 1.82841632353714e-07
423 1.82393744974085e-07
424 1.81948058440184e-07
425 1.81500539551571e-07
426 1.81045163843763e-07
427 1.80591471887759e-07
428 1.80137768171384e-07
429 1.79682281578053e-07
430 1.792262350504e-07
431 1.78769786080224e-07
432 1.78312051881235e-07
433 1.77853697403307e-07
434 1.77393323143704e-07
435 1.76934255969208e-07
436 1.76471509079512e-07
437 1.7600974925891e-07
438 1.75546597866116e-07
439 1.75081342117522e-07
440 1.74615554364266e-07
441 1.74147637413746e-07
442 1.73679999741161e-07
443 1.73208302182815e-07
444 1.72735885518804e-07
445 1.7226177070917e-07
446 1.71787832123016e-07
447 1.71312425595538e-07
448 1.70838459502498e-07
449 1.70366316806003e-07
450 1.69894396605308e-07
451 1.69423522009105e-07
452 1.68951246934057e-07
453 1.68479684976575e-07
454 1.68006595358428e-07
455 1.67535015398457e-07
456 1.67060885844172e-07
457 1.6658764003008e-07
458 1.6611330570182e-07
459 1.65638124284939e-07
460 1.6516312598025e-07
461 1.64687279730735e-07
462 1.64211500558231e-07
463 1.63735308413848e-07
464 1.63258260690746e-07
465 1.62781177992066e-07
466 1.62303859923441e-07
467 1.61825787317937e-07
468 1.6134802779888e-07
469 1.6086950273575e-07
470 1.6039103929355e-07
471 1.59912292859055e-07
472 1.59432665188142e-07
473 1.58953860348809e-07
474 1.58474358577187e-07
475 1.5799460627175e-07
476 1.57514791285784e-07
477 1.5703566580072e-07
478 1.56554733160341e-07
479 1.56074302450016e-07
480 1.55593067061588e-07
481 1.55113129664208e-07
482 1.54632000370469e-07
483 1.54151178477058e-07
484 1.53669260607003e-07
485 1.53187314540837e-07
486 1.52704888612298e-07
487 1.52221006323394e-07
488 1.51737356313575e-07
489 1.51251054476553e-07
490 1.50762763762913e-07
491 1.50269890489163e-07
492 1.49771339610361e-07
493 1.4927455953373e-07
494 1.48787812362272e-07
495 1.48307667065417e-07
496 1.47828614445622e-07
497 1.47349851179257e-07
498 1.46870894659479e-07
499 1.46392094303671e-07
500 1.45913413195586e-07
501 1.45434261649591e-07
502 1.44956060585955e-07
503 1.44477489561368e-07
504 1.43998945815405e-07
505 1.43521449814443e-07
506 1.43043002921672e-07
507 1.42564729001649e-07
508 1.42086519145934e-07
509 1.41608616944566e-07
510 1.41130135798306e-07
511 1.40651014412185e-07
512 1.4017236956132e-07
513 1.39693428771182e-07
514 1.39212827028778e-07
515 1.38732846497014e-07
516 1.38251854609806e-07
517 1.37772574764661e-07
518 1.37291288698549e-07
519 1.36810785048347e-07
520 1.36330735413637e-07
521 1.35851123255648e-07
522 1.35372446075088e-07
523 1.34894040103362e-07
524 1.34415328345128e-07
525 1.33938308774617e-07
526 1.33460033349131e-07
527 1.32981744422445e-07
528 1.32503411599316e-07
529 1.32025094291777e-07
530 1.31547228924056e-07
531 1.31067467384227e-07
532 1.30588859998504e-07
533 1.3010997337215e-07
534 1.29631273326325e-07
535 1.29152337644989e-07
536 1.28673126432055e-07
537 1.2819489781446e-07
538 1.27715784918436e-07
539 1.27237736037067e-07
540 1.26760411292004e-07
541 1.26282405088496e-07
542 1.25805200305251e-07
543 1.25329263269691e-07
544 1.24852846218992e-07
545 1.24377427340505e-07
546 1.23902954438648e-07
547 1.23428877776277e-07
548 1.22955568608418e-07
549 1.22482554951731e-07
550 1.2201199472095e-07
551 1.21540685890764e-07
552 1.2107076932466e-07
553 1.20602028941263e-07
554 1.20134234953717e-07
555 1.19667664742806e-07
556 1.19202871323054e-07
557 1.18737262035928e-07
558 1.18274229378379e-07
559 1.1781213765083e-07
560 1.1735137402713e-07
561 1.1689187593511e-07
562 1.16432271460631e-07
563 1.15974628825732e-07
564 1.15518408073889e-07
565 1.15061658253701e-07
566 1.14604577584387e-07
567 1.14144489917756e-07
568 1.13679369585284e-07
569 1.13223577468347e-07
570 1.12776595261543e-07
571 1.12333362165096e-07
572 1.11891631068062e-07
573 1.11452094691877e-07
574 1.11013477730459e-07
575 1.10577713248183e-07
576 1.10141514598538e-07
577 1.09708848665235e-07
578 1.09276275407133e-07
579 1.08845568492555e-07
580 1.08417194387478e-07
581 1.07989908263662e-07
582 1.07564401645277e-07
583 1.07139076887641e-07
584 1.06718479485401e-07
585 1.06296893346958e-07
586 1.05877330211612e-07
587 1.05460283234216e-07
588 1.05044349107075e-07
589 1.04630599348177e-07
590 1.04218622619001e-07
591 1.03807702905634e-07
592 1.03400331751935e-07
593 1.02992864587037e-07
594 1.025879418588e-07
595 1.02185728641402e-07
596 1.01784272763972e-07
597 1.01385488489036e-07
598 1.00987388384155e-07
599 1.00592585710046e-07
600 1.00199061012773e-07
601 9.98066531590069e-08
602 9.94173352655281e-08
603 9.90286707089894e-08
604 9.86410001937443e-08
605 9.82563890969956e-08
606 9.78724784781448e-08
607 9.74892696001817e-08
608 9.71051299112347e-08
609 9.671949923451e-08
610 9.63720074818042e-08
611 9.59829339270968e-08
612 9.56143055041991e-08
613 9.52564187670646e-08
614 9.48728241265684e-08
615 9.45119139661443e-08
616 9.41150417830272e-08
617 9.37594564609512e-08
618 9.33956799125113e-08
619 9.302844256176e-08
620 9.26610014007068e-08
621 9.23087095037189e-08
622 9.19615625649683e-08
623 9.16135621302772e-08
624 9.1267812929452e-08
625 9.09227643166588e-08
626 9.05805061313103e-08
627 9.02392151411746e-08
628 8.99010125916533e-08
629 8.95651503691752e-08
630 8.92327192749676e-08
631 8.89020074970048e-08
632 8.85723249064441e-08
633 8.82446401506698e-08
634 8.79182471331674e-08
635 8.7592993391894e-08
636 8.72681556645105e-08
637 8.69452763616607e-08
638 8.6623138481734e-08
639 8.63034395979234e-08
640 8.59845733396369e-08
641 8.56682345342641e-08
642 8.53520607129177e-08
643 8.50386151634019e-08
644 8.47291818075746e-08
645 8.44210575294824e-08
646 8.4113685734799e-08
647 8.38080190748158e-08
648 8.3506342261952e-08
649 8.32061577473731e-08
650 8.29093117169499e-08
651 8.26140750609383e-08
652 8.23221834140853e-08
653 8.2034726383462e-08
654 8.17427157389083e-08
655 8.14694086197498e-08
656 8.11698344023171e-08
657 8.09189147537026e-08
658 8.05996960191457e-08
659 8.03788801202465e-08
660 8.00394319777631e-08
661 7.98369317287495e-08
662 7.94904840635269e-08
663 7.92851293121011e-08
664 7.89538001675361e-08
665 7.87578683882906e-08
666 7.84505019013615e-08
667 7.82385183049072e-08
668 7.79474063223873e-08
669 7.77284125121902e-08
670 7.74507674647929e-08
671 7.72265958111085e-08
672 7.69600804018111e-08
673 7.67333297888584e-08
674 7.64757813018768e-08
675 7.62448548274897e-08
676 7.5986254328253e-08
677 7.56778238524269e-08
678 7.54528691206957e-08
679 7.52590394985653e-08
680 7.50320240512359e-08
681 7.48008213404816e-08
682 7.4577760523109e-08
683 7.43526471085332e-08
684 7.41311021359436e-08
685 7.39112878025416e-08
686 7.36934269256473e-08
687 7.34761096774505e-08
688 7.32620193373279e-08
689 7.3049240830958e-08
690 7.28386344337473e-08
691 7.26291033683601e-08
692 7.24218250574005e-08
693 7.22167172071408e-08
694 7.20136896661394e-08
695 7.18116182714112e-08
696 7.16138920986431e-08
697 7.14220284692146e-08
698 7.1287787827945e-08
699 7.09544075507829e-08
700 7.08295406992576e-08
701 7.06765867670711e-08
702 7.05079560989752e-08
703 7.01743510145292e-08
704 7.00866906244713e-08
705 6.99659532958918e-08
706 6.961003983319e-08
707 6.95447170238062e-08
708 6.93873486956598e-08
709 6.90594919503518e-08
710 6.90171652082405e-08
711 6.8737124314211e-08
712 6.86685795838748e-08
713 6.83394192297904e-08
714 6.83053064629569e-08
715 6.7990989379485e-08
716 6.79506656404705e-08
717 6.76356702191683e-08
718 6.75992196197406e-08
719 6.72859656223324e-08
720 6.7253870859485e-08
721 6.69441731790954e-08
722 6.69154214434364e-08
723 6.66119885353034e-08
724 6.65853865351451e-08
725 6.62852725614727e-08
726 6.62663511232786e-08
727 6.59651626926383e-08
728 6.59541358452032e-08
729 6.5654153895256e-08
730 6.56498363502678e-08
731 6.534704972605e-08
732 6.53503122780563e-08
733 6.50488972260277e-08
734 6.50554422434446e-08
735 6.47556232511448e-08
736 6.47624760454768e-08
737 6.44688017272799e-08
738 6.44695299172326e-08
739 6.4185147623963e-08
740 6.41783612858049e-08
741 6.39048666468689e-08
742 6.38900750691285e-08
743 6.36279731431877e-08
744 6.36093776202884e-08
745 6.33605118340874e-08
746 6.33392026316848e-08
747 6.31077102934974e-08
748 6.30728996329211e-08
749 6.28615012079337e-08
750 6.28168769223336e-08
751 6.26200582125591e-08
752 6.25641242737451e-08
753 6.23849001049059e-08
754 6.23158023396186e-08
755 6.21541869989528e-08
756 6.20724984954357e-08
757 6.19287757501752e-08
758 6.18316953664433e-08
759 6.17077707727454e-08
760 6.15945624402592e-08
761 6.14919786268153e-08
762 6.13610307613754e-08
763 6.12795431265134e-08
764 6.11323800057129e-08
765 6.10712725208629e-08
766 6.09095760690792e-08
767 6.08661018213219e-08
768 6.06902153075239e-08
769 6.06670548362054e-08
770 6.04737630180985e-08
771 6.04745997629941e-08
772 6.02624085708214e-08
773 6.02814231247528e-08
774 6.00648317883312e-08
775 6.0083001129474e-08
776 5.98696125884146e-08
777 5.98955775621413e-08
778 5.96791361799909e-08
779 5.97068735217476e-08
780 5.94918350151907e-08
781 5.9524628548413e-08
782 5.93093024052749e-08
783 5.93448672079333e-08
784 5.91288348643637e-08
785 5.9166525354204e-08
786 5.89506209087709e-08
787 5.89932359460477e-08
788 5.87761983008583e-08
789 5.88199875619644e-08
790 5.86028671261118e-08
791 5.86508443323197e-08
792 5.84336438591571e-08
793 5.84826963025975e-08
794 5.8266525535533e-08
795 5.83178614852464e-08
796 5.81001802766323e-08
797 5.81561528587571e-08
798 5.79356941918796e-08
799 5.79944080119077e-08
800 5.77714268201746e-08
801 5.78341095618384e-08
802 5.76057250882833e-08
803 5.76648077372788e-08
804 5.74165745974575e-08
805 5.74490718312504e-08
806 5.72400144420726e-08
807 5.73149907028281e-08
808 5.70946464173439e-08
809 5.71767375632248e-08
810 5.69464096402683e-08
811 5.7038702319101e-08
812 5.67859353264311e-08
813 5.69155817728983e-08
814 5.66271774657956e-08
815 5.68075726792294e-08
816 5.64891339007723e-08
817 5.66533746941822e-08
818 5.63678197651463e-08
819 5.65173290629772e-08
820 5.62259864715919e-08
821 5.63862296152351e-08
822 5.60883605738027e-08
823 5.62249436613094e-08
824 5.59292739872319e-08
825 5.61130857845704e-08
826 5.57882062324211e-08
827 5.59937283264311e-08
828 5.56599741465824e-08
829 5.5868128804093e-08
830 5.55361720313563e-08
831 5.57434553427605e-08
832 5.54149912366242e-08
833 5.5621624525326e-08
834 5.52962308595362e-08
835 5.5500655336882e-08
836 5.51805862483157e-08
837 5.53808103891384e-08
838 5.50673661465773e-08
839 5.52627373271442e-08
840 5.49573642381063e-08
841 5.51447561711882e-08
842 5.48479763220477e-08
843 5.50293441534677e-08
844 5.47406525504002e-08
845 5.49150648425645e-08
846 5.46350518795258e-08
847 5.48013159886729e-08
848 5.4532928758011e-08
849 5.46883944743115e-08
850 5.44283112335897e-08
851 5.4577482136331e-08
852 5.43278563358562e-08
853 5.44673397384621e-08
854 5.4229839598019e-08
855 5.43589250425747e-08
856 5.41456614246982e-08
857 5.42440322304216e-08
858 5.4051852814041e-08
859 5.41467765793335e-08
860 5.39706939743923e-08
861 5.40178930337198e-08
862 5.38876425726364e-08
863 5.3939229908373e-08
864 5.38062416044482e-08
865 5.37940685365612e-08
866 5.37390810686489e-08
867 5.36812915279405e-08
868 5.36335549337075e-08
869 5.36195243161508e-08
870 5.35274003632402e-08
871 5.35427860421933e-08
872 5.3435499437704e-08
873 5.3434117004425e-08
874 5.33541400344895e-08
875 5.33171705257018e-08
876 5.32952196015124e-08
877 5.32156843169318e-08
878 5.31968725034915e-08
879 5.31463093835782e-08
880 5.30818221586848e-08
881 5.30505457656538e-08
882 5.30262958635674e-08
883 5.29505943331898e-08
884 5.29278514367348e-08
885 5.28804488859613e-08
886 5.28581154579655e-08
887 5.27466267423726e-08
888 5.27594299462031e-08
889 5.27023473457788e-08
890 5.26887610075022e-08
891 5.25881852908583e-08
892 5.26006007195434e-08
893 5.25359454606189e-08
894 5.24991035901756e-08
895 5.24596571676028e-08
896 5.24150896943354e-08
897 5.23761161841385e-08
898 5.23353466257426e-08
899 5.22948481602725e-08
900 5.22540154959117e-08
901 5.2215049095139e-08
902 5.21748683226519e-08
903 5.21343384649597e-08
904 5.20942852220152e-08
905 5.20541773680883e-08
906 5.20151927463353e-08
907 5.19751245540157e-08
908 5.19355657120357e-08
909 5.18966218456818e-08
910 5.18557787256846e-08
911 5.18168745302638e-08
912 5.17766250487384e-08
913 5.17364828698241e-08
914 5.16958491281194e-08
915 5.16568028827713e-08
916 5.16153260474717e-08
917 5.15677126013969e-08
918 5.16053249586257e-08
919 5.14220684180167e-08
920 5.15367411941803e-08
921 5.13420854098001e-08
922 5.14600132417442e-08
923 5.12636939711264e-08
924 5.13826281980378e-08
925 5.11855523588345e-08
926 5.13065476055452e-08
927 5.11088295676032e-08
928 5.12339722700261e-08
929 5.10355327030254e-08
930 5.11629514985401e-08
931 5.0962218024253e-08
932 5.10912907483174e-08
933 5.0890339667653e-08
934 5.1020086868192e-08
935 5.08200422904004e-08
936 5.09492797546862e-08
937 5.07498267454665e-08
938 5.08788590680709e-08
939 5.06786306249118e-08
940 5.08101164360397e-08
941 5.06088055560738e-08
942 5.0740749008682e-08
943 5.05401167596453e-08
944 5.06728751012986e-08
945 5.04713377957877e-08
946 5.0604395456233e-08
947 5.04028742320806e-08
948 5.05367274752366e-08
949 5.03359578223339e-08
950 5.0469086040561e-08
951 5.02684399186215e-08
952 5.04024024903238e-08
953 5.02014500485259e-08
954 5.03376540677003e-08
955 5.01344206980114e-08
956 5.02699580628985e-08
957 5.00687480240458e-08
958 5.02067344938517e-08
959 5.00032723107502e-08
960 5.01411674056484e-08
961 4.993789211305e-08
962 5.00775792136032e-08
963 4.98727141513555e-08
964 5.00144688655446e-08
965 4.98076341264309e-08
966 4.99514265084322e-08
967 4.97422680760806e-08
968 4.98900909486899e-08
969 4.96782990753175e-08
970 4.98281040695048e-08
971 4.96150873332546e-08
972 4.97673570309942e-08
973 4.95515360405818e-08
974 4.97076206773528e-08
975 4.94876012342438e-08
976 4.96468727808619e-08
977 4.9425818874127e-08
978 4.95845301755615e-08
979 4.93638130905083e-08
980 4.9524567196535e-08
981 4.9302167807852e-08
982 4.94548772187464e-08
983 4.92445121107465e-08
984 4.94037069507236e-08
985 4.91806125606509e-08
986 4.93110376518224e-08
987 4.91369962891675e-08
988 4.92832275797994e-08
989 4.90598947116005e-08
990 4.91819723338338e-08
991 4.90233589158784e-08
992 4.91654577219336e-08
993 4.89408455806029e-08
994 4.90383735196609e-08
995 4.89289344018928e-08
996 4.89626104522856e-08
997 4.89007945354558e-08
998 4.88691589941581e-08
999 4.89038606104053e-08
1000 4.87678076472875e-08
1001 4.88888815297628e-08
1002 4.8678789477119e-08
1003 4.87917609959254e-08
1004 4.86519707028066e-08
1005 4.87628180403377e-08
1006 4.85702752199835e-08
1007 4.87219789508941e-08
1008 4.85065205402258e-08
1009 4.85748193383984e-08
1010 4.85206731268839e-08
1011 4.85827093763547e-08
1012 4.83995220292854e-08
1013 4.85567348609273e-08
1014 4.83344737145508e-08
1015 4.838853278466e-08
1016 4.84357933552282e-08
1017 4.82604617997318e-08
1018 4.84109966856217e-08
1019 4.81956278859386e-08
1020 4.82657424947242e-08
1021 4.82081189905337e-08
1022 4.82576679905478e-08
1023 4.81067591913131e-08
1024 4.82584157786015e-08
1025 4.80239344082811e-08
1026 4.80826359749997e-08
1027 4.81299472254726e-08
1028 4.79590989868051e-08
1029 4.81208632483465e-08
1030 4.78882899530575e-08
1031 4.79472773950995e-08
1032 4.79956309429319e-08
1033 4.78217487303034e-08
1034 4.79817632079715e-08
1035 4.77537845871012e-08
1036 4.78127988663424e-08
1037 4.78588182626361e-08
1038 4.76885983355935e-08
1039 4.78512086203864e-08
1040 4.76182612709763e-08
1041 4.76817949293107e-08
1042 4.77279329818003e-08
1043 4.7551650633082e-08
1044 4.77038137147723e-08
1045 4.74899373212878e-08
1046 4.75521663898526e-08
1047 4.75884242212565e-08
1048 4.74223451147893e-08
1049 4.75544842846176e-08
1050 4.73671052048097e-08
1051 4.7474058231689e-08
1052 4.73535406055881e-08
1053 4.73839164008005e-08
1054 4.73204380750936e-08
1055 4.73799681364184e-08
1056 4.72209253628009e-08
1057 4.74041406715031e-08
1058 4.71318410504296e-08
1059 4.72140234135665e-08
1060 4.72585653557012e-08
1061 4.70833885612976e-08
1062 4.72017601120811e-08
1063 4.70516651680519e-08
1064 4.72095963033325e-08
1065 4.69610770923445e-08
1066 4.70393277978154e-08
1067 4.70782565278327e-08
1068 4.69067375523657e-08
1069 4.69972557772813e-08
1070 4.69206534607913e-08
1071 4.69242574143713e-08
1072 4.68602168277599e-08
1073 4.6930865402306e-08
1074 4.67560896666441e-08
1075 4.68911945055162e-08
1076 4.67075215779822e-08
1077 4.67902788026109e-08
1078 4.6712712272523e-08
1079 4.67869231464135e-08
1080 4.66037245550055e-08
1081 4.66871860895779e-08
1082 4.66143022728183e-08
1083 4.66841574944254e-08
1084 4.65100386368711e-08
1085 4.66045625056033e-08
1086 4.65203283637194e-08
1087 4.65807436698107e-08
1088 4.64184330910555e-08
1089 4.65871385713079e-08
1090 4.63419261849118e-08
1091 4.64261830397383e-08
1092 4.64582094958921e-08
1093 4.62963434162411e-08
1094 4.63626263682571e-08
1095 4.64236687829356e-08
1096 4.62019541753733e-08
1097 4.62764474606914e-08
1098 4.63193064095258e-08
1099 4.61498115362779e-08
1100 4.62201399660245e-08
1101 4.63214942985069e-08
1102 4.60343970458688e-08
1103 4.61352567251438e-08
1104 4.61794144199779e-08
1105 4.60055624649236e-08
1106 4.60785714673229e-08
1107 4.61729846290559e-08
1108 4.58963749800922e-08
1109 4.59947537398975e-08
1110 4.60415953775772e-08
1111 4.58629434949565e-08
1112 4.59370120875491e-08
1113 4.59861009356466e-08
1114 4.57839293743767e-08
1115 4.58552543816104e-08
1116 4.59026546835162e-08
1117 4.57243942730834e-08
1118 4.57942950990997e-08
1119 4.58346378500174e-08
1120 4.5654627903069e-08
1121 4.57187205813447e-08
1122 4.57675753451348e-08
1123 4.55858099854822e-08
1124 4.56541818132372e-08
1125 4.57044878987567e-08
1126 4.55138019002099e-08
1127 4.55836518651331e-08
1128 4.56453211339358e-08
1129 4.54401674034877e-08
1130 4.5518123790167e-08
1131 4.558594670101e-08
1132 4.53674382669966e-08
1133 4.54510139777931e-08
1134 4.55289328815667e-08
1135 4.52934265138261e-08
1136 4.53861377591735e-08
1137 4.54713292152853e-08
1138 4.52197046323377e-08
1139 4.53202367651784e-08
1140 4.54248807173485e-08
1141 4.51384330100169e-08
1142 4.52585044086185e-08
1143 4.53857395621426e-08
1144 4.50552344068988e-08
1145 4.51985150826673e-08
1146 4.53219113478731e-08
1147 4.49872422527697e-08
1148 4.51341876575029e-08
1149 4.52403307447113e-08
1150 4.49293227671177e-08
1151 4.5068676743476e-08
1152 4.5128209712253e-08
1153 4.48971456408209e-08
1154 4.49972035259094e-08
1155 4.50596218160015e-08
1156 4.48372273051945e-08
1157 4.49280534704499e-08
1158 4.49769723331883e-08
1159 4.47843381587987e-08
1160 4.48655408797904e-08
1161 4.48516417357325e-08
1162 4.47734229167551e-08
1163 4.4959484924334e-08
1164 4.46214504128406e-08
1165 4.47655585174012e-08
1166 4.47697761543076e-08
1167 4.46386138412969e-08
1168 4.47137496100147e-08
1169 4.46242948064501e-08
1170 4.48199990059628e-08
1171 4.44659242369028e-08
1172 4.46210251241475e-08
1173 4.46126212225728e-08
1174 4.44990734909112e-08
1175 4.45693721093932e-08
1176 4.44717006113926e-08
1177 4.46581146991143e-08
1178 4.43266606637849e-08
1179 4.44736128688561e-08
1180 4.44451197050277e-08
1181 4.43764751252296e-08
1182 4.45048887622868e-08
1183 4.42568538092658e-08
1184 4.43657212549908e-08
1185 4.43282067541517e-08
1186 4.42919613496962e-08
1187 4.42982214297061e-08
1188 4.42464013277188e-08
1189 4.42553207204988e-08
1190 4.42103522906301e-08
1191 4.42147427683892e-08
1192 4.4166482972674e-08
1193 4.41701242324122e-08
1194 4.41342276040757e-08
1195 4.41222085387771e-08
1196 4.40963816163986e-08
1197 4.40822689919962e-08
1198 4.40574347524425e-08
1199 4.40398749290694e-08
1200 4.40167168935801e-08
1201 4.40003319481797e-08
1202 4.39760138126743e-08
1203 4.39592984688986e-08
1204 4.3935785319249e-08
1205 4.39197810884195e-08
1206 4.38956175781158e-08
1207 4.38791637473734e-08
1208 4.38546177705668e-08
1209 4.38383845113854e-08
1210 4.38153891284543e-08
1211 4.37984809451564e-08
1212 4.37759807270233e-08
1213 4.37571804696724e-08
1214 4.37365284384761e-08
1215 4.37171685763182e-08
1216 4.36957009872785e-08
1217 4.36778792121117e-08
1218 4.36560585348467e-08
1219 4.36357604769455e-08
1220 4.36168493114231e-08
1221 4.35957581386148e-08
1222 4.35770911817457e-08
1223 4.35558541171943e-08
1224 4.35375799328064e-08
1225 4.35163603622613e-08
1226 4.34967047104884e-08
1227 4.34783101046499e-08
1228 4.34573678815831e-08
1229 4.34389331576135e-08
1230 4.34191404243833e-08
1231 4.33997990834101e-08
1232 4.33798322974077e-08
1233 4.33611475192386e-08
1234 4.334106535131e-08
1235 4.33222925866339e-08
1236 4.33026241979384e-08
1237 4.32835674306631e-08
1238 4.32647019383836e-08
1239 4.32442290851931e-08
1240 4.32265429619605e-08
1241 4.32059835318022e-08
1242 4.31881624938235e-08
1243 4.31677529522112e-08
1244 4.31491857693089e-08
1245 4.31301928363048e-08
1246 4.31108619491916e-08
1247 4.30924408179045e-08
1248 4.30725470024207e-08
1249 4.30541755074287e-08
1250 4.30354042948444e-08
1251 4.3015966687765e-08
1252 4.29971511270999e-08
1253 4.29781415305364e-08
1254 4.29583231977837e-08
1255 4.2940882261977e-08
1256 4.29209946628539e-08
1257 4.29028587909386e-08
1258 4.28841289954462e-08
1259 4.28651859354368e-08
1260 4.28469881561533e-08
1261 4.2826846850641e-08
1262 4.28090867261588e-08
1263 4.27897579680092e-08
1264 4.27716712914083e-08
1265 4.27524199464457e-08
1266 4.27342435558309e-08
1267 4.27164608067798e-08
1268 4.26961070219001e-08
1269 4.26790580472236e-08
1270 4.26592356514988e-08
1271 4.2641870187321e-08
1272 4.26222762084372e-08
1273 4.2604515262834e-08
1274 4.25857381292083e-08
1275 4.25670609605078e-08
1276 4.25487079573905e-08
1277 4.25308824825166e-08
1278 4.25115455957581e-08
1279 4.24943768329022e-08
1280 4.24754123424798e-08
1281 4.24559229657007e-08
1282 4.2439848683351e-08
1283 4.24201248767631e-08
1284 4.2402465289193e-08
1285 4.2382986272127e-08
1286 4.23661506361306e-08
1287 4.23466303844933e-08
1288 4.23286165758263e-08
1289 4.23103741695741e-08
1290 4.22926291019365e-08
1291 4.22726675384233e-08
1292 4.22569251634641e-08
1293 4.22374950983517e-08
1294 4.22200426473118e-08
1295 4.22008607623425e-08
1296 4.21829286545439e-08
1297 4.21655705760138e-08
1298 4.21487664983289e-08
1299 4.21284588152382e-08
1300 4.21122841252064e-08
1301 4.2093314612579e-08
1302 4.20770026137163e-08
1303 4.20576032520437e-08
1304 4.20401416634242e-08
1305 4.2023279007708e-08
1306 4.20050915357351e-08
1307 4.19872964956269e-08
1308 4.19702719365311e-08
1309 4.19536296503154e-08
1310 4.19350717444367e-08
1311 4.19189012550447e-08
1312 4.19010634402639e-08
1313 4.1884574701001e-08
1314 4.18665627468506e-08
1315 4.18491989164771e-08
1316 4.18298914484083e-08
1317 4.1813679301228e-08
1318 4.17945479616044e-08
1319 4.17778044266015e-08
1320 4.17593771211422e-08
1321 4.17419007647801e-08
1322 4.17239348817766e-08
1323 4.17067846067987e-08
1324 4.16881643929834e-08
1325 4.16712006896525e-08
1326 4.16534554670278e-08
1327 4.16346532028378e-08
1328 4.16181836389029e-08
1329 4.15993809355086e-08
1330 4.1582793582684e-08
1331 4.15638670778762e-08
1332 4.15463153049522e-08
1333 4.1529371452409e-08
1334 4.1511213624279e-08
1335 4.14932293235637e-08
1336 4.14754603443868e-08
1337 4.14577362803925e-08
1338 4.14392000478436e-08
1339 4.14221559763561e-08
1340 4.14038587170928e-08
1341 4.13861402472904e-08
1342 4.13679803217271e-08
1343 4.13491417847744e-08
1344 4.13304730098041e-08
1345 4.1312159493323e-08
1346 4.12907477982394e-08
1347 4.12684552593134e-08
1348 4.12400396965218e-08
1349 4.11989697859383e-08
1350 4.11499909525048e-08
1351 4.11332379210982e-08
1352 4.11417757955057e-08
1353 4.11118472296579e-08
1354 4.11166389220341e-08
1355 4.10776479404795e-08
1356 4.10870591043277e-08
1357 4.10426129269226e-08
1358 4.10580065750565e-08
1359 4.10095050820125e-08
1360 4.10252063729466e-08
1361 4.09756878978484e-08
1362 4.09947615040984e-08
1363 4.09420616533929e-08
1364 4.09637133174101e-08
1365 4.0909342728046e-08
1366 4.09314579559172e-08
1367 4.08767912905006e-08
1368 4.09003115153794e-08
1369 4.08435944860841e-08
1370 4.08686879338482e-08
1371 4.08102410065503e-08
1372 4.08386024304264e-08
1373 4.07768890093863e-08
1374 4.08075619464654e-08
1375 4.07431551106008e-08
1376 4.07768074865977e-08
1377 4.0711062904375e-08
1378 4.07449306627505e-08
1379 4.06788482778531e-08
1380 4.07135424991267e-08
1381 4.06467689630929e-08
1382 4.06822184908506e-08
1383 4.06130117189818e-08
1384 4.06528931451433e-08
1385 4.05803005767424e-08
1386 4.06208433409994e-08
1387 4.05481369378791e-08
1388 4.05900374493839e-08
1389 4.05161556553679e-08
1390 4.05583448710445e-08
1391 4.04836470888625e-08
1392 4.05282369926319e-08
1393 4.04515517735682e-08
1394 4.04952700572281e-08
1395 4.04205783794964e-08
1396 4.04647250711321e-08
1397 4.03891499063391e-08
1398 4.04335190387073e-08
1399 4.03569757390088e-08
1400 4.04022816362648e-08
1401 4.03250251692633e-08
1402 4.03707202387693e-08
1403 4.02936174808133e-08
1404 4.03403649662515e-08
1405 4.02616842642978e-08
1406 4.03075232635608e-08
1407 4.0232398277773e-08
1408 4.02760213029651e-08
1409 4.02004070436668e-08
1410 4.024627556376e-08
1411 4.01692901355055e-08
1412 4.0214798551208e-08
1413 4.01373648597492e-08
1414 4.0185044419605e-08
1415 4.0106365463366e-08
1416 4.01525595017382e-08
1417 4.00765142187787e-08
1418 4.01212449445687e-08
1419 4.00448866844094e-08
1420 4.00911413205307e-08
1421 4.00150111499187e-08
1422 4.00588288229642e-08
1423 3.99843860972027e-08
1424 4.00290362208011e-08
1425 3.99528242596148e-08
1426 3.99974064815289e-08
1427 3.99234770784851e-08
1428 3.99658919469914e-08
1429 3.98937401029364e-08
1430 3.99339985563962e-08
1431 3.98629752540458e-08
1432 3.99040120107053e-08
1433 3.98324837198594e-08
1434 3.98723136942891e-08
1435 3.98029013291001e-08
1436 3.98413128732678e-08
1437 3.97725613021649e-08
1438 3.9810280280772e-08
1439 3.9742035133461e-08
1440 3.97790540298537e-08
1441 3.97129826796849e-08
1442 3.97477542786184e-08
1443 3.96830390814174e-08
1444 3.97177059241471e-08
1445 3.96522852783576e-08
1446 3.96855588649103e-08
1447 3.96223630589887e-08
1448 3.9656171483049e-08
1449 3.95935556216642e-08
1450 3.96247457383758e-08
1451 3.95643775381593e-08
1452 3.95923137870291e-08
1453 3.95348077528723e-08
1454 3.95624264384331e-08
1455 3.95054021269381e-08
1456 3.95318612183004e-08
1457 3.94757773620746e-08
1458 3.95018891987675e-08
1459 3.94463005131129e-08
1460 3.94710260689379e-08
1461 3.94177797380379e-08
1462 3.94402749059886e-08
1463 3.93887644580815e-08
1464 3.94104165652998e-08
1465 3.93597498282716e-08
1466 3.93800524198795e-08
1467 3.93307202148918e-08
1468 3.93499881159443e-08
1469 3.93013747683746e-08
1470 3.93209741109679e-08
1471 3.92731979745875e-08
1472 3.92899152616977e-08
1473 3.9245288319556e-08
1474 3.92597484655255e-08
1475 3.92165975005021e-08
1476 3.92294252806735e-08
1477 3.91881370545022e-08
1478 3.91997712307912e-08
1479 3.91596822546525e-08
1480 3.91711969944808e-08
1481 3.91295282424764e-08
1482 3.91415172562581e-08
1483 3.91031562885225e-08
1484 3.91120078790941e-08
1485 3.90724187395719e-08
1486 3.90819623024008e-08
1487 3.90466638169329e-08
1488 3.90528545297641e-08
1489 3.90184355190115e-08
1490 3.90230449864859e-08
1491 3.8989114254262e-08
1492 3.89938363398734e-08
1493 3.89609338282693e-08
1494 3.8965046679218e-08
1495 3.89320704043961e-08
1496 3.89367111501837e-08
1497 3.89048507105638e-08
1498 3.89062329793077e-08
1499 3.88755143789776e-08
1500 3.88782684079825e-08
1501 3.88492461260714e-08
1502 3.88487125579928e-08
1503 3.882107363129e-08
1504 3.88198546910079e-08
1505 3.8792297518464e-08
1506 3.87915394064464e-08
1507 3.87644253814212e-08
1508 3.87625891118404e-08
1509 3.87373462036322e-08
1510 3.87330289728371e-08
1511 3.87095084766198e-08
1512 3.87050264287758e-08
1513 3.86825100942012e-08
1514 3.86766131537097e-08
1515 3.86547776058954e-08
1516 3.86474578821794e-08
1517 3.86276070527547e-08
1518 3.86197996971038e-08
1519 3.85992827942605e-08
1520 3.85914998493053e-08
1521 3.85722565494628e-08
1522 3.85639128530624e-08
1523 3.85442833412686e-08
1524 3.85363514947556e-08
1525 3.85171662837802e-08
1526 3.85080932139292e-08
1527 3.8489539925024e-08
1528 3.84805524418219e-08
1529 3.84625639411329e-08
1530 3.8453796457727e-08
1531 3.84345056982927e-08
1532 3.84264690811431e-08
1533 3.84068450620667e-08
1534 3.83989320902334e-08
1535 3.83814077111744e-08
1536 3.83706286632979e-08
1537 3.83543023274591e-08
1538 3.83431685944124e-08
1539 3.83275558684115e-08
1540 3.83173806226544e-08
1541 3.83001840560393e-08
1542 3.82890045800277e-08
1543 3.82744699680693e-08
1544 3.82620330541261e-08
1545 3.82479333318475e-08
1546 3.82350660577124e-08
1547 3.8220931813937e-08
1548 3.8209084508134e-08
1549 3.81943544021102e-08
1550 3.81825717128415e-08
1551 3.81674761626982e-08
1552 3.81556987565368e-08
1553 3.81410430736473e-08
1554 3.81292905791142e-08
1555 3.81150503439009e-08
1556 3.81020034676549e-08
1557 3.80886615360954e-08
1558 3.8077460486452e-08
1559 3.80618457567206e-08
1560 3.8050609541429e-08
1561 3.8035625539834e-08
1562 3.80255649592431e-08
1563 3.80085830316013e-08
1564 3.79994917865112e-08
1565 3.79823041862792e-08
1566 3.79735357685096e-08
1567 3.79579025362453e-08
1568 3.79469946643063e-08
1569 3.79315737859542e-08
1570 3.792112937262e-08
1571 3.79067996707505e-08
1572 3.78951095556523e-08
1573 3.7881861032929e-08
1574 3.78679961465789e-08
1575 3.78564453675789e-08
1576 3.78430116505513e-08
1577 3.78315891831349e-08
1578 3.78172963888535e-08
1579 3.78055829499679e-08
1580 3.77913593332391e-08
1581 3.7781269575099e-08
1582 3.77647789939761e-08
1583 3.77567025013903e-08
1584 3.77396430615295e-08
1585 3.77308680064026e-08
1586 3.77154301540727e-08
1587 3.77057533293978e-08
1588 3.76901064065294e-08
1589 3.76810962317009e-08
1590 3.76644921569191e-08
1591 3.76561058184066e-08
1592 3.76396580314964e-08
1593 3.76302235338688e-08
1594 3.76163999689094e-08
1595 3.76033765654427e-08
1596 3.75913931913718e-08
1597 3.75794513032268e-08
1598 3.75670147638729e-08
1599 3.75540705788158e-08
1600 3.75420552594097e-08
1601 3.75292471328503e-08
1602 3.7517432028622e-08
1603 3.75051622791833e-08
1604 3.74914503642465e-08
1605 3.74810370047385e-08
1606 3.74675792018664e-08
1607 3.74573131722844e-08
1608 3.74421283713122e-08
1609 3.74315780542478e-08
1610 3.74172313735155e-08
1611 3.740761071791e-08
1612 3.73926290555548e-08
1613 3.73817259835541e-08
1614 3.73687607195805e-08
1615 3.73564960325368e-08
1616 3.73432568654408e-08
1617 3.7330406497782e-08
1618 3.73173713432706e-08
1619 3.73001394347039e-08
1620 3.72846569367535e-08
1621 3.72757625277664e-08
1622 3.72612479271339e-08
1623 3.72544635558469e-08
1624 3.7233945430426e-08
1625 3.72363016580835e-08
1626 3.72021272732059e-08
1627 3.72230411622709e-08
1628 3.71682197646273e-08
1629 3.72134055914852e-08
1630 3.71410362836055e-08
1631 3.7197649680909e-08
1632 3.7103907787106e-08
1633 3.71844661886112e-08
1634 3.70942017582365e-08
1635 3.71536662944205e-08
1636 3.70521398300738e-08
1637 3.71393648936902e-08
1638 3.70397694697822e-08
1639 3.71169938464533e-08
1640 3.70182559397136e-08
1641 3.70901052393346e-08
1642 3.70064290085015e-08
1643 3.70373715135042e-08
1644 3.69849409265211e-08
1645 3.70311027833026e-08
1646 3.69396235080366e-08
1647 3.70071264510496e-08
1648 3.69215419406821e-08
1649 3.69835203375413e-08
1650 3.68969395017071e-08
1651 3.69482735975701e-08
1652 3.68932216820994e-08
1653 3.69431068134585e-08
1654 3.69038644054687e-08
1655 3.68573845253017e-08
1656 3.69410358072031e-08
1657 3.68001965094233e-08
1658 3.69465624012744e-08
1659 3.67578752145103e-08
1660 3.69416040515436e-08
1661 3.67243197669787e-08
1662 3.69255605208174e-08
1663 3.67000692031993e-08
1664 3.68983167060577e-08
1665 3.66818187516049e-08
1666 3.68685834033489e-08
1667 3.66608709159166e-08
1668 3.68419083600635e-08
1669 3.663887923544e-08
1670 3.68181082264218e-08
1671 3.66141354632976e-08
1672 3.67947784314371e-08
1673 3.6589255341779e-08
1674 3.67701646970531e-08
1675 3.65680700480375e-08
1676 3.67450344309628e-08
1677 3.65458838758048e-08
1678 3.67180482274687e-08
1679 3.6525476994953e-08
1680 3.66912007627729e-08
1681 3.65043138574883e-08
1682 3.66542927252578e-08
1683 3.64918327104835e-08
1684 3.65999942428807e-08
1685 3.6511242445858e-08
1686 3.65832107540598e-08
1687 3.64761625886167e-08
1688 3.65499099701516e-08
1689 3.64689406757712e-08
1690 3.6539733676122e-08
1691 3.64247123194605e-08
1692 3.65023932467068e-08
1693 3.63711807411793e-08
1694 3.65802523210057e-08
1695 3.62907836208848e-08
1696 3.64615658401402e-08
1697 3.63391326454465e-08
1698 3.64134665451488e-08
1699 3.63590975471784e-08
1700 3.64361296421034e-08
1701 3.63108052852112e-08
1702 3.63084534662494e-08
1703 3.63491252337589e-08
1704 3.63333256616638e-08
1705 3.62856251556742e-08
1706 3.63564120473558e-08
1707 3.62888553688823e-08
1708 3.62450932822611e-08
1709 3.62537099731775e-08
1710 3.62115651089123e-08
1711 3.62579667838592e-08
1712 3.62617183338187e-08
1713 3.62311279535899e-08
1714 3.6172962205816e-08
1715 3.61814272471506e-08
1716 3.61693494423943e-08
1717 3.61759446387833e-08
1718 3.62177028645494e-08
1719 3.6173893311231e-08
1720 3.61488188671544e-08
1721 3.61537691750957e-08
1722 3.61318289228141e-08
1723 3.61259254264557e-08
1724 3.6165967184365e-08
1725 3.61099847310253e-08
1726 3.61947166012122e-08
1727 3.60202984810432e-08
1728 3.61474846493071e-08
1729 3.61379211422097e-08
1730 3.60076788692432e-08
1731 3.61505832409126e-08
1732 3.59869659589673e-08
1733 3.61363344905108e-08
1734 3.5928362938753e-08
1735 3.61437071831894e-08
1736 3.59215346648067e-08
1737 3.614313658451e-08
1738 3.5846024857733e-08
1739 3.60516781421349e-08
1740 3.59832959211381e-08
1741 3.5950038974164e-08
1742 3.59868580457334e-08
1743 3.59164443020088e-08
1744 3.6033117810419e-08
1745 3.58246843776833e-08
1746 3.60501437044469e-08
1747 3.58905773170104e-08
1748 3.58409230885037e-08
1749 3.60414546780596e-08
1750 3.59220942933725e-08
1751 3.58356852339092e-08
1752 3.5954147092454e-08
1753 3.58587194972948e-08
1754 3.59145327935018e-08
1755 3.59093112318742e-08
1756 3.58975965917274e-08
1757 3.58735034460267e-08
1758 3.58781918003981e-08
1759 3.58506787645485e-08
1760 3.58566485743061e-08
1761 3.58346163933376e-08
1762 3.58303522978165e-08
1763 3.58083198370718e-08
1764 3.58068862111871e-08
1765 3.57868174158948e-08
1766 3.57888523885297e-08
1767 3.5768169392103e-08
1768 3.57704918059198e-08
1769 3.57534379962221e-08
1770 3.57662167442818e-08
1771 3.57616642010505e-08
1772 3.57829811028498e-08
1773 3.57532252208692e-08
1774 3.57642854507123e-08
1775 3.57359315346084e-08
1776 3.5757916897472e-08
1777 3.57204477792195e-08
1778 3.57261601964254e-08
1779 3.5699540543499e-08
1780 3.57225790468263e-08
1781 3.56961902820974e-08
1782 3.57055259849748e-08
1783 3.56580612204116e-08
1784 3.56714967144622e-08
1785 3.56487262735961e-08
1786 3.56665286367353e-08
1787 3.56272648591727e-08
1788 3.56349370762565e-08
1789 3.5594027940844e-08
1790 3.56234906890851e-08
1791 3.55816229082873e-08
1792 3.56108817225476e-08
1793 3.55491287915566e-08
1794 3.55826741196275e-08
1795 3.55278083932209e-08
1796 3.55778157490327e-08
1797 3.55067167376877e-08
1798 3.55502722675372e-08
1799 3.54785129510482e-08
1800 3.55388428756598e-08
1801 3.5466460728939e-08
1802 3.5511302629132e-08
1803 3.54386202630064e-08
1804 3.55074010847112e-08
1805 3.54084363172191e-08
1806 3.54745530120049e-08
1807 3.53960640468998e-08
1808 3.54767380101872e-08
1809 3.53798920926796e-08
1810 3.54283397321353e-08
1811 3.53549797600383e-08
1812 3.54428463980039e-08
1813 3.53210542292093e-08
1814 3.53862454232079e-08
1815 3.53152355003772e-08
1816 3.53748624857264e-08
1817 3.53731073794439e-08
1818 3.52559731373603e-08
1819 3.53448708365622e-08
1820 3.52578297921902e-08
1821 3.53324026312052e-08
1822 3.53225658333844e-08
1823 3.51956875401793e-08
1824 3.53313657031062e-08
1825 3.51875335604657e-08
1826 3.5318774788573e-08
1827 3.52337659070834e-08
1828 3.51871847887875e-08
1829 3.52708397519663e-08
1830 3.51362191501892e-08
1831 3.53207218841689e-08
1832 3.50724417028658e-08
1833 3.51897822361558e-08
1834 3.50027108408923e-08
1835 3.50841573908411e-08
1836 3.50284988879235e-08
1837 3.50446092520773e-08
1838 3.50817544534365e-08
1839 3.5133075329874e-08
1840 3.51331583718917e-08
1841 3.50597291993715e-08
1842 3.50394566199785e-08
1843 3.50715432531068e-08
1844 3.51166261809421e-08
1845 3.50453870088252e-08
1846 3.4995355761902e-08
1847 3.51026898339768e-08
1848 3.49231568930186e-08
1849 3.51629038537471e-08
1850 3.4959558258052e-08
1851 3.49947070290568e-08
1852 3.50596751836907e-08
1853 3.48972750137122e-08
1854 3.51496939126061e-08
1855 3.48191944168352e-08
1856 3.50449844903622e-08
1857 3.49763022078342e-08
1858 3.48941517305246e-08
1859 3.50054902951147e-08
1860 3.4761637942804e-08
1861 3.48312861890854e-08
1862 3.48924258533057e-08
1863 3.46799293511335e-08
1864 3.50132941215886e-08
1865 3.48346893799256e-08
1866 3.4723380212931e-08
1867 3.49690854448692e-08
1868 3.4831107904143e-08
1869 3.48928369542367e-08
1870 3.48093296171914e-08
1871 3.4730148819051e-08
1872 3.48578429816282e-08
1873 3.46692226642631e-08
1874 3.48651333910155e-08
1875 3.48122755589042e-08
1876 3.45417646190338e-08
1877 3.48917237720237e-08
1878 3.46855100403776e-08
1879 3.47623265746222e-08
1880 3.4770677705076e-08
1881 3.46083375932782e-08
1882 3.48883133891587e-08
1883 3.44789870301287e-08
1884 3.4910494641105e-08
1885 3.44220385259941e-08
1886 3.48641777774272e-08
1887 3.45205472143117e-08
1888 3.47393328870016e-08
1889 3.45348889225772e-08
1890 3.46892453360059e-08
1891 3.47367870832116e-08
1892 3.44435412287236e-08
1893 3.46712013783446e-08
1894 3.44146259043665e-08
1895 3.48256828657245e-08
1896 3.42912743196422e-08
1897 3.46407793014869e-08
1898 3.45247513757752e-08
1899 3.45269794996828e-08
1900 3.46897291647608e-08
1901 3.42994929471985e-08
1902 3.46000019468828e-08
1903 3.43318886304456e-08
1904 3.46537392501567e-08
1905 3.43861308413906e-08
1906 3.44278337578441e-08
1907 3.46617283610851e-08
1908 3.41892558162193e-08
1909 3.45971863142669e-08
1910 3.43984418442744e-08
1911 3.44372842688845e-08
1912 3.44335990649647e-08
1913 3.43332195849033e-08
1914 3.45267562418261e-08
1915 3.42017907495329e-08
1916 3.45516426845105e-08
1917 3.41652257862535e-08
1918 3.45622661794831e-08
1919 3.41626167454923e-08
1920 3.43533801736573e-08
1921 3.44350808909599e-08
1922 3.41708872542146e-08
1923 3.45379818018632e-08
1924 3.39816569099316e-08
1925 3.44767674305135e-08
1926 3.41525129974318e-08
1927 3.43339821451405e-08
1928 3.41731866060346e-08
1929 3.42139256881424e-08
1930 3.43512957607928e-08
1931 3.40458595562421e-08
1932 3.44320516183494e-08
1933 3.40011445332244e-08
1934 3.42519601908009e-08
1935 3.42367627899076e-08
1936 3.41465848749944e-08
1937 3.42123699863528e-08
1938 3.40252847681466e-08
1939 3.43744207442764e-08
1940 3.38951168292745e-08
1941 3.41756571569096e-08
1942 3.41257875651202e-08
1943 3.40375308052199e-08
1944 3.42246444695604e-08
1945 3.40123406719073e-08
1946 3.40918144456914e-08
1947 3.40105244296396e-08
1948 3.41725493420153e-08
1949 3.38115782587689e-08
1950 3.42990732593584e-08
1951 3.3789214409552e-08
1952 3.39640925937079e-08
1953 3.41564013957019e-08
1954 3.37296851460334e-08
1955 3.40273826202608e-08
1956 3.38076859012215e-08
1957 3.39117693746172e-08
1958 3.40950339727808e-08
1959 3.36624779877503e-08
1960 3.40643316034672e-08
1961 3.36269093552666e-08
1962 3.39779633056203e-08
1963 3.37779159016449e-08
1964 3.37904144358525e-08
1965 3.40103438301043e-08
1966 3.37005075947694e-08
1967 3.39588195890084e-08
1968 3.3682246082023e-08
1969 3.39970558214109e-08
1970 3.35181762323167e-08
1971 3.38420512944193e-08
1972 3.3746243113697e-08
1973 3.37108448862367e-08
1974 3.36869655612748e-08
1975 3.3721706593326e-08
1976 3.36742126980649e-08
1977 3.36415274422297e-08
1978 3.37400929177889e-08
1979 3.37014334439445e-08
1980 3.3721535946718e-08
1981 3.36068610711227e-08
1982 3.38414277238908e-08
1983 3.35824224322323e-08
1984 3.37673894292845e-08
1985 3.35129928596967e-08
1986 3.37354536876777e-08
1987 3.35769944204145e-08
1988 3.36766722541793e-08
1989 3.34562720052389e-08
1990 3.37194521717876e-08
1991 3.34472689920151e-08
1992 3.37757763340107e-08
1993 3.34775050716551e-08
1994 3.36171808743568e-08
1995 3.34757355566762e-08
1996 3.37168185142733e-08
1997 3.32803695666239e-08
1998 3.36355107712905e-08
1999 3.35371856965416e-08
2000 3.34837224489348e-08
2001 3.34977289446758e-08
2002 3.35932434123443e-08
2003 3.3451523160366e-08
2004 3.33993363714757e-08
2005 3.34994830712976e-08
2006 3.35199768164429e-08
2007 3.34914656783258e-08
2008 3.3434613587735e-08
2009 3.34294251400635e-08
2010 3.34897762519493e-08
2011 3.33994225814038e-08
2012 3.34915289124105e-08
2013 3.33792630298202e-08
2014 3.35162036579906e-08
2015 3.32972283769717e-08
2016 3.34152964216461e-08
2017 3.33880450851254e-08
2018 3.34200951344066e-08
2019 3.33940664516419e-08
2020 3.34399952524045e-08
2021 3.32268468636876e-08
2022 3.33796657758789e-08
2023 3.33231621942875e-08
2024 3.34264242543014e-08
2025 3.3238499475674e-08
2026 3.33946271053875e-08
2027 3.32610017566015e-08
2028 3.334340389749e-08
2029 3.32690691005055e-08
2030 3.33340006073435e-08
2031 3.32213067388842e-08
2032 3.33465467832195e-08
2033 3.32201027926082e-08
2034 3.33096498787988e-08
2035 3.32458527805279e-08
2036 3.33301597650326e-08
2037 3.31643178113072e-08
2038 3.32792589479158e-08
2039 3.32395127389251e-08
2040 3.33066642799196e-08
2041 3.30655313498518e-08
2042 3.32711471866531e-08
2043 3.32403864811148e-08
2044 3.33787345376813e-08
2045 3.3089639337458e-08
2046 3.33793505942204e-08
2047 3.30331862767874e-08
2048 3.3221809190298e-08
2049 3.30550539431318e-08
2050 3.32230710362591e-08
2051 3.30392389300549e-08
2052 3.32068704584909e-08
2053 3.31382339826547e-08
2054 3.31147876366789e-08
2055 3.30406054656063e-08
2056 3.32303542776025e-08
2057 3.29955358449041e-08
2058 3.31803106470296e-08
2059 3.3007643316374e-08
2060 3.32414920420909e-08
2061 3.32322938043639e-08
2062 3.31457147575431e-08
2063 3.30165394812898e-08
2064 3.30260070577904e-08
2065 3.29629907382944e-08
2066 3.31892257401378e-08
2067 3.28541673515481e-08
2068 3.32052875084976e-08
2069 3.30590627519722e-08
2070 3.32136624656698e-08
2071 3.27951492755663e-08
2072 3.31568837035334e-08
2073 3.28287069240041e-08
2074 3.31295375499696e-08
2075 3.29876104356863e-08
2076 3.32610871440764e-08
2077 3.2742719116241e-08
2078 3.32115435077718e-08
2079 3.27306859539966e-08
2080 3.30432889712462e-08
2081 3.30348801658165e-08
2082 3.31043083923177e-08
2083 3.28345843929601e-08
2084 3.3072484731056e-08
2085 3.2670893715725e-08
2086 3.31368020920486e-08
2087 3.30423670331648e-08
2088 3.29051677070424e-08
2089 3.27491607317043e-08
2090 3.29352375865177e-08
2091 3.27973194917952e-08
2092 3.31829713757781e-08
2093 3.29576320665659e-08
2094 3.28874390469469e-08
2095 3.28603219401646e-08
2096 3.29765542443905e-08
2097 3.28407258725072e-08
2098 3.28144218606496e-08
2099 3.28009136119078e-08
2100 3.29327306336502e-08
2101 3.26628217932612e-08
2102 3.31286806827258e-08
2103 3.28793158943519e-08
2104 3.28677318948145e-08
2105 3.27618392268469e-08
2106 3.28940477596351e-08
2107 3.28127691064939e-08
2108 3.28145996295603e-08
2109 3.2625019193766e-08
2110 3.31340133132851e-08
2111 3.24143913432717e-08
2112 3.30977469595872e-08
2113 3.23656868754796e-08
2114 3.30988262833287e-08
2115 3.23744066792475e-08
2116 3.31689648838118e-08
2117 3.22432142556561e-08
2118 3.31629498735442e-08
2119 3.25897988873169e-08
2120 3.29370182274857e-08
2121 3.25571698407412e-08
2122 3.28980497448228e-08
2123 3.23938346014074e-08
2124 3.2958260428817e-08
2125 3.2376695355385e-08
2126 3.30412009976744e-08
2127 3.24693364521789e-08
2128 3.29011151631864e-08
2129 3.26259471872259e-08
2130 3.26981523663949e-08
2131 3.24160226377046e-08
2132 3.30753469459655e-08
2133 3.23326995455009e-08
2134 3.27451560131475e-08
2135 3.24537409288617e-08
2136 3.27079974573596e-08
2137 3.27492899026005e-08
2138 3.26822255205528e-08
2139 3.22625722897207e-08
2140 3.29567684484999e-08
2141 3.23065406691025e-08
2142 3.27858046911267e-08
2143 3.22606489484567e-08
2144 3.27441509406778e-08
2145 3.25056728611717e-08
2146 3.27332965281979e-08
2147 3.23050296611171e-08
2148 3.27704513349847e-08
2149 3.23413660987537e-08
2150 3.26451209191614e-08
2151 3.25682107462644e-08
2152 3.21741501754236e-08
2153 3.30813678659503e-08
2154 3.18496485118835e-08
2155 3.30703251849585e-08
2156 3.1904667396887e-08
2157 3.2918905252588e-08
2158 3.22646586314868e-08
2159 3.27879263068986e-08
2160 3.23418088954419e-08
2161 3.26100478249547e-08
2162 3.24397676096932e-08
2163 3.2611193982568e-08
2164 3.21175400164186e-08
2165 3.27734409675262e-08
2166 3.2199755271467e-08
2167 3.26219852306853e-08
2168 3.26386590541183e-08
2169 3.20663745811967e-08
2170 3.2495044357761e-08
2171 3.2436411794734e-08
2172 3.23354420457278e-08
2173 3.23704389919577e-08
2174 3.21458372491978e-08
2175 3.2441221138324e-08
2176 3.20708230980848e-08
2177 3.2635733086428e-08
2178 3.20415828816145e-08
2179 3.25165463883614e-08
2180 3.21999797310291e-08
2181 3.24739686263786e-08
2182 3.19757311566526e-08
2183 3.25929106441691e-08
2184 3.20657797265866e-08
2185 3.24406078344719e-08
2186 3.196263847971e-08
2187 3.26077944201586e-08
2188 3.20838482714692e-08
2189 3.23854361261588e-08
2190 3.20602547159154e-08
2191 3.25260074880429e-08
2192 3.18791186206813e-08
2193 3.24590285638582e-08
2194 3.19369187511409e-08
2195 3.25381955821324e-08
2196 3.19409678748439e-08
2197 3.26318861696429e-08
2198 3.19371886423614e-08
2199 3.24435155125702e-08
2200 3.18569386026368e-08
2201 3.25033349819481e-08
2202 3.19456034625354e-08
2203 3.21657689403843e-08
2204 3.1990552955552e-08
2205 3.24120122840998e-08
2206 3.17956429876087e-08
2207 3.24035743632933e-08
2208 3.17964725939923e-08
2209 3.25598268808847e-08
2210 3.17669261769016e-08
2211 3.24955493913315e-08
2212 3.17166794656654e-08
2213 3.26556057796967e-08
2214 3.17810889469694e-08
2215 3.22692835319494e-08
2216 3.18492116988534e-08
2217 3.25256413979957e-08
2218 3.15830623154323e-08
2219 3.24100121320559e-08
2220 3.16367989812161e-08
2221 3.24587238389462e-08
2222 3.16242590929772e-08
2223 3.22612781260556e-08
2224 3.16952122756398e-08
2225 3.25495722821589e-08
2226 3.16532298418881e-08
2227 3.21995029792799e-08
2228 3.1768009069566e-08
2229 3.23811024851572e-08
2230 3.1476647483597e-08
2231 3.261784487818e-08
2232 3.1505319032954e-08
2233 3.25128204521352e-08
2234 3.15630303586722e-08
2235 3.22507479399992e-08
2236 3.15124778260678e-08
2237 3.25256665272278e-08
2238 3.13798697171297e-08
2239 3.25788974251395e-08
2240 3.14430379695363e-08
2241 3.2418215133001e-08
2242 3.14874376772245e-08
2243 3.255301749161e-08
2244 3.12919261304945e-08
2245 3.24201250094802e-08
2246 3.14160069274649e-08
2247 3.25182044416117e-08
2248 3.12993664262073e-08
2249 3.26717698895251e-08
2250 3.14506037450979e-08
2251 3.23562770370156e-08
2252 3.14265883585296e-08
2253 3.2108432017397e-08
2254 3.17087675878458e-08
2255 3.18087132513689e-08
2256 3.21382812067394e-08
2257 3.14359469226755e-08
2258 3.22538589443422e-08
2259 3.15326007316319e-08
2260 3.2361325969843e-08
2261 3.12303805625191e-08
2262 3.20403497009458e-08
2263 3.13875647603989e-08
2264 3.24522503476477e-08
2265 3.12688046117948e-08
2266 3.21586499185678e-08
2267 3.13445145301294e-08
2268 3.16648224933935e-08
2269 3.21668293195021e-08
2270 3.16816961227762e-08
2271 3.17133949740978e-08
2272 3.20170310956769e-08
2273 3.12695001187802e-08
2274 3.21192412939819e-08
2275 3.13151676154533e-08
2276 3.23167740439168e-08
2277 3.11366384315726e-08
2278 3.25034701367244e-08
2279 3.11616506869505e-08
2280 3.23790451166595e-08
2281 3.10582291946915e-08
2282 3.23484439277433e-08
2283 3.1124692583262e-08
2284 3.19202658320972e-08
2285 3.09971323675384e-08
2286 3.21898452968483e-08
2287 3.13374243432474e-08
2288 3.22405919415125e-08
2289 3.09353062268869e-08
2290 3.20255581565121e-08
2291 3.11426655350555e-08
2292 3.21963654572599e-08
2293 3.10895650093013e-08
2294 3.19393686876168e-08
2295 3.13455225002812e-08
2296 3.19657601459689e-08
2297 3.11141873319265e-08
2298 3.19334483043221e-08
2299 3.16667562796358e-08
2300 3.14069336015965e-08
2301 3.17510070295679e-08
2302 3.14601206443665e-08
2303 3.13386986059427e-08
2304 3.20622193550335e-08
2305 3.10310714621842e-08
2306 3.21561535638892e-08
2307 3.11990487271707e-08
2308 3.15581026866241e-08
2309 3.16102768340709e-08
2310 3.1822508624435e-08
2311 3.10499534670328e-08
2312 3.17553136579551e-08
2313 3.12534422808319e-08
2314 3.15225190550272e-08
2315 3.19251580557234e-08
2316 3.119484380254e-08
2317 3.17558537206164e-08
2318 3.1212911657752e-08
2319 3.16546709444587e-08
2320 3.1210600139886e-08
2321 3.18882714853697e-08
2322 3.09860295117925e-08
2323 3.19172080998076e-08
2324 3.13187132734427e-08
2325 3.16171008467592e-08
2326 3.11232840786158e-08
2327 3.13333890380463e-08
2328 3.18597351565142e-08
2329 3.10284642723868e-08
2330 3.18709197801859e-08
2331 3.09694669577976e-08
2332 3.17494839570109e-08
2333 3.09161939200386e-08
2334 3.17170276833423e-08
2335 3.10897199233828e-08
2336 3.1424920223122e-08
2337 3.12594655742515e-08
2338 3.16510378131074e-08
2339 3.10564713390526e-08
2340 3.17638934470565e-08
2341 3.12280598029346e-08
2342 3.14235863774215e-08
2343 3.14603389066637e-08
2344 3.09978600809835e-08
2345 3.15704852469345e-08
2346 3.1027982544618e-08
2347 3.18673088686161e-08
2348 3.11249027802329e-08
2349 3.15377393975691e-08
2350 3.11431826558461e-08
2351 3.15079897921233e-08
2352 3.10416320452589e-08
2353 3.1339207809955e-08
2354 3.11412416644785e-08
2355 3.15252837124191e-08
2356 3.11640095642662e-08
2357 3.13355235981305e-08
2358 3.13194789884896e-08
2359 3.16221979528297e-08
2360 3.12385427605477e-08
2361 3.14628353990099e-08
2362 3.10049937657375e-08
2363 3.14427810450546e-08
2364 3.08888489966019e-08
2365 3.11140388713493e-08
2366 3.1178503407725e-08
2367 3.16366511194932e-08
2368 3.08208043922065e-08
2369 3.12942085685108e-08
2370 3.14959649310165e-08
2371 3.14021426697231e-08
2372 3.14146718840558e-08
2373 3.07303681621462e-08
2374 3.16343668711472e-08
2375 3.07260841394541e-08
2376 3.09417613280072e-08
2377 3.13381634413634e-08
2378 3.07444552869463e-08
2379 3.16317664346144e-08
2380 3.07539122892386e-08
2381 3.0999860245462e-08
2382 3.125141961835e-08
2383 3.10767306606063e-08
2384 3.12708234513348e-08
2385 3.08674471822812e-08
2386 3.13089248558107e-08
2387 3.08572442824229e-08
2388 3.11411535320882e-08
2389 3.10541400432296e-08
2390 3.10686820199635e-08
2391 3.09356508136904e-08
2392 3.11975894653482e-08
2393 3.06797180085194e-08
2394 3.15677014040006e-08
2395 3.0598887730493e-08
2396 3.11702334270247e-08
2397 3.10782449022273e-08
2398 3.082476975802e-08
2399 3.11132120569546e-08
2400 3.10957543334656e-08
2401 3.1178981522606e-08
2402 3.11380109194737e-08
2403 3.16260067922958e-08
2404 3.12608490928845e-08
2405 3.1604642218408e-08
2406 3.10750220564593e-08
2407 3.13370448026262e-08
2408 3.10938801260097e-08
2409 3.1576289974522e-08
2410 3.11730403425026e-08
2411 3.13812800846192e-08
2412 3.12816689307382e-08
2413 3.12131388509052e-08
2414 3.09342819905245e-08
2415 3.13547145625037e-08
2416 3.1507577141765e-08
2417 3.15128589336577e-08
2418 3.14637902270043e-08
2419 3.14080123503535e-08
2420 3.12826816382117e-08
2421 3.1712855695365e-08
2422 3.10352633110522e-08
2423 3.13634617306047e-08
2424 3.12730412153073e-08
2425 3.12658645333919e-08
2426 3.14340152784975e-08
2427 3.1418984206999e-08
2428 3.11160990558168e-08
2429 3.14126571266948e-08
2430 3.16263011326257e-08
2431 3.14679166926179e-08
2432 3.12361977895526e-08
2433 3.13241358125271e-08
2434 3.1087585111278e-08
2435 3.11315034413617e-08
2436 3.13191868482887e-08
2437 3.11836345234795e-08
2438 3.14748872888293e-08
2439 3.13400547324871e-08
2440 3.14577395292837e-08
2441 3.0914446997321e-08
2442 3.12290686361738e-08
2443 3.09604518726747e-08
2444 3.12735323330093e-08
2445 3.11599785485228e-08
2446 3.10256022323374e-08
2447 3.11477708783148e-08
2448 3.1069192365063e-08
2449 3.14866411620374e-08
2450 3.09391846272655e-08
2451 3.13522745450934e-08
2452 3.0895515921503e-08
2453 3.11458153209321e-08
2454 3.15506414013678e-08
2455 3.08394490837749e-08
2456 3.10343281650915e-08
2457 3.10406793542217e-08
2458 3.09975037868782e-08
2459 3.06173629361917e-08
2460 3.08215722442107e-08
2461 3.10997265429158e-08
2462 3.07418998511944e-08
2463 3.12818175752794e-08
2464 3.05382900446238e-08
2465 3.14617286437668e-08
2466 3.09667384645884e-08
2467 3.10415457726032e-08
2468 3.09931197600699e-08
2469 3.08602082650378e-08
2470 3.13337271251601e-08
2471 3.10470671857077e-08
2472 3.10633428892704e-08
2473 3.09452172654323e-08
2474 3.10791597749605e-08
2475 3.09968556205797e-08
2476 3.13143940310345e-08
2477 3.09463518753939e-08
2478 3.12283249759293e-08
2479 3.08342842222364e-08
2480 3.1134256700871e-08
2481 3.1080722336374e-08
2482 3.0790072627962e-08
2483 3.14123116510467e-08
2484 3.09535827949237e-08
2485 3.1618187818272e-08
2486 3.04800664117089e-08
2487 3.13070377585545e-08
2488 3.09693871665129e-08
2489 3.11783970231572e-08
2490 3.07914169055401e-08
2491 3.10931086670019e-08
2492 3.06806002173809e-08
2493 3.13774903957231e-08
2494 3.06443373193632e-08
2495 3.11483545998303e-08
2496 3.09069538504403e-08
2497 3.10266572155493e-08
2498 3.09286344228887e-08
2499 3.0690172743375e-08
2500 3.12397876340853e-08
2501 3.07110231549945e-08
2502 3.12349320547023e-08
2503 3.06934112996871e-08
2504 3.11993896269236e-08
2505 3.07884744588094e-08
2506 3.10974682173892e-08
2507 3.07570430378989e-08
2508 3.13362458873589e-08
2509 3.0595693075619e-08
2510 3.11758077864432e-08
2511 3.07535782002599e-08
2512 3.13882401162813e-08
2513 3.02975248293036e-08
2514 3.14497846196637e-08
2515 3.02637908230485e-08
2516 3.17420064915863e-08
2517 3.00899573826285e-08
2518 3.14898262591123e-08
2519 3.04310298573363e-08
2520 3.13062611622117e-08
2521 3.04099334279551e-08
2522 3.12915237070666e-08
2523 3.0532195952504e-08
2524 3.11354551855159e-08
2525 3.07379254820273e-08
2526 3.12254568848314e-08
2527 3.03016267684253e-08
2528 3.1559048915164e-08
2529 3.0220535184089e-08
2530 3.15610920305076e-08
2531 3.01697529481171e-08
2532 3.13191432629312e-08
2533 3.02816960408014e-08
2534 3.14982902911432e-08
2535 3.02240964008549e-08
2536 3.12559418381708e-08
2537 3.04802791808445e-08
2538 3.12601991314665e-08
2539 3.03136950664529e-08
2540 3.10720058388503e-08
2541 3.06923042935336e-08
2542 3.09636612709774e-08
2543 3.06478241515196e-08
2544 3.10713658150386e-08
2545 3.05002346778771e-08
2546 3.11229537448554e-08
2547 3.04397509072185e-08
2548 3.08344730530763e-08
2549 3.04420694839935e-08
2550 3.12035138854139e-08
2551 3.03722482448698e-08
2552 3.09906328899157e-08
2553 3.06479859786268e-08
2554 3.09805894863624e-08
2555 3.04368013908096e-08
2556 3.08492807280514e-08
2557 3.07842400815161e-08
2558 3.07348395838236e-08
2559 3.08171203243779e-08
2560 3.08255665477652e-08
2561 3.0515047374946e-08
2562 3.08628418234091e-08
2563 3.07872564631051e-08
2564 3.06952058670706e-08
2565 3.05590787279542e-08
2566 3.078735985651e-08
2567 3.08451047961178e-08
2568 3.04156234146591e-08
2569 3.0895649437479e-08
2570 3.05406900957816e-08
2571 3.10049352509889e-08
2572 3.04660523967604e-08
2573 3.07224366131731e-08
2574 3.05096559460871e-08
2575 3.07849446593567e-08
2576 3.04521241343281e-08
2577 3.0976396590221e-08
2578 3.03015326948985e-08
2579 3.10014670771297e-08
2580 3.06843614925612e-08
2581 3.05029753218111e-08
2582 3.06268429474121e-08
2583 3.05904233544041e-08
2584 3.0406073026179e-08
2585 3.09174726323969e-08
2586 3.03174771656423e-08
2587 3.10430368180015e-08
2588 3.02938810670117e-08
2589 3.06900229755103e-08
2590 3.05839586348666e-08
2591 3.06006527258651e-08
2592 3.06680918671365e-08
2593 3.04421175125746e-08
2594 3.07327831408077e-08
2595 3.0337708944228e-08
2596 3.08992172008971e-08
2597 3.02224856638666e-08
2598 3.09506820821426e-08
2599 3.00870812135479e-08
2600 3.11504111798477e-08
2601 2.99604575700885e-08
2602 3.11691091818833e-08
2603 2.99835661563552e-08
2604 3.12957854673446e-08
2605 2.99611646387143e-08
2606 3.13648008555267e-08
2607 2.96515699874655e-08
2608 3.12635582645493e-08
2609 2.92589761575757e-08
2610 3.13999887123773e-08
2611 3.00129039830788e-08
2612 3.11284388297972e-08
2613 2.95581484294205e-08
2614 3.14110385473132e-08
2615 2.97860616850221e-08
2616 3.11710941708343e-08
2617 2.99667234090473e-08
2618 3.08254877132708e-08
2619 3.02474168226707e-08
2620 3.07721406100914e-08
2621 3.0040425902933e-08
2622 3.07451341114984e-08
2623 3.05606377039025e-08
2624 3.01874934958146e-08
2625 3.06861569092653e-08
2626 3.0304098253664e-08
2627 3.03798990435311e-08
2628 3.0902414170475e-08
2629 2.93923732883616e-08
2630 3.08019718398e-08
2631 3.00589775228399e-08
2632 3.0423093518861e-08
2633 3.02032744627923e-08
2634 3.05156111035654e-08
2635 3.02597019844475e-08
2636 3.02557953550942e-08
2637 3.020378707419e-08
2638 3.03234275875752e-08
2639 3.00721326930153e-08
2640 3.08429250590603e-08
2641 3.00005141489201e-08
2642 3.0677437352078e-08
2643 3.01336198675894e-08
2644 3.03399984517005e-08
2645 3.05868281958999e-08
2646 2.9813303599191e-08
2647 3.14747699282636e-08
2648 2.89836899395324e-08
2649 3.08873424667011e-08
2650 3.01832217205478e-08
2651 3.04273814407674e-08
2652 3.02553365915115e-08
2653 3.03533392423949e-08
2654 3.02929203532809e-08
2655 3.02785417484186e-08
2656 3.00776197227348e-08
2657 3.03452248084923e-08
2658 3.03040437760194e-08
2659 3.0124222098471e-08
2660 3.04327978557506e-08
2661 3.0103110815749e-08
2662 3.00810269978635e-08
2663 2.96792699965298e-08
2664 3.13579428538047e-08
2665 2.92851778225245e-08
2666 3.10369001769173e-08
2667 2.91622220869314e-08
2668 3.09482512765857e-08
2669 2.9773129551991e-08
2670 3.04129748104343e-08
2671 3.02291785683195e-08
2672 3.02724007483768e-08
2673 2.9912922792974e-08
2674 3.06653184679595e-08
2675 2.92836748340131e-08
2676 3.11920625973849e-08
2677 2.95112655961383e-08
2678 3.03850943397244e-08
2679 2.97199169435602e-08
2680 3.11878380975239e-08
2681 2.91912674184047e-08
2682 3.10538704614283e-08
2683 2.96897835662113e-08
2684 3.07833837317473e-08
2685 3.00552166396795e-08
2686 3.04544491238623e-08
2687 2.97442789868319e-08
2688 3.09007629972768e-08
2689 2.93425085996502e-08
2690 3.13986689676193e-08
2691 2.87305522271852e-08
2692 3.07916186477186e-08
2693 3.00461978386402e-08
2694 3.02486959394832e-08
2695 3.08202435813643e-08
2696 2.96968955579846e-08
2697 3.02117952032388e-08
2698 3.02873009762061e-08
2699 2.92068386984479e-08
2700 3.11443441690651e-08
2701 2.89534788338042e-08
2702 3.06017615968601e-08
2703 3.02870858210902e-08
2704 2.96039041527418e-08
2705 2.98031940386689e-08
2706 3.04395922841039e-08
2707 2.94266327425241e-08
2708 3.03480564610714e-08
2709 2.97898043785105e-08
2710 3.06515685360997e-08
2711 2.99750744690019e-08
2712 3.05403619501599e-08
2713 2.93434331765097e-08
2714 3.10075060671044e-08
2715 2.99942644014228e-08
2716 3.01790401111557e-08
2717 2.97421869540626e-08
2718 3.08827633438424e-08
2719 2.93871986571936e-08
2720 3.01712381705066e-08
2721 2.98124061227645e-08
2722 3.07547400786312e-08
2723 2.94518431892543e-08
2724 2.99442274775963e-08
2725 3.053641105355e-08
2726 2.95928627124242e-08
2727 3.06217647635387e-08
2728 2.98280354391611e-08
2729 3.01232456236811e-08
2730 2.95819893892935e-08
2731 3.12087343913303e-08
2732 2.94409466101708e-08
2733 3.05907258455473e-08
2734 2.91202506166321e-08
2735 3.09376336726785e-08
2736 2.92441477354188e-08
2737 3.07773032924841e-08
2738 2.87969219245898e-08
2739 3.07308355257385e-08
2740 2.92001241108952e-08
2741 3.08831176732971e-08
2742 2.93397344468538e-08
2743 3.0590261737129e-08
2744 2.95153181154584e-08
2745 3.1061265720167e-08
2746 2.92077328485352e-08
2747 3.03595714226113e-08
2748 3.02807259082716e-08
2749 2.99411995434706e-08
2750 3.01459981926211e-08
2751 2.92655954119425e-08
2752 3.08908133888064e-08
2753 2.89011269181128e-08
2754 3.13123940128834e-08
2755 2.92019416691103e-08
2756 3.01307712893273e-08
2757 2.93623555146505e-08
2758 3.14548624181787e-08
2759 2.87850643723164e-08
2760 3.03471446063686e-08
2761 2.93128200697934e-08
2762 3.16337745915884e-08
2763 2.80013558711101e-08
2764 3.13898184701733e-08
2765 2.94253233109609e-08
2766 3.11104402517248e-08
2767 2.92716897558609e-08
2768 3.08302987103159e-08
2769 2.87935793974459e-08
2770 3.00415875664761e-08
2771 2.95797695921696e-08
2772 3.07864919111278e-08
2773 2.89376215066861e-08
2774 3.05887544966854e-08
2775 2.88891557224114e-08
2776 3.10735541426688e-08
2777 2.87686724980141e-08
2778 3.01759727769912e-08
2779 3.03198275695138e-08
2780 2.91673366150658e-08
2781 3.01437768274182e-08
2782 2.97135588610553e-08
2783 3.0088701149289e-08
2784 2.9709644008169e-08
2785 2.93313660274519e-08
2786 3.13266982351346e-08
2787 2.94283235027226e-08
2788 2.93596664977525e-08
2789 3.03995840356119e-08
2790 2.91468613553558e-08
2791 3.02954255736454e-08
2792 2.88821918389415e-08
2793 3.10093474861217e-08
2794 2.91949324631169e-08
2795 2.95441846783762e-08
2796 3.07516174810063e-08
2797 2.95300454121739e-08
2798 2.99616413443893e-08
2799 2.95246382179259e-08
2800 2.9655125492356e-08
2801 2.96534356341027e-08
2802 3.03618632938596e-08
2803 2.93385282487124e-08
2804 2.97483296710643e-08
2805 2.97721746189694e-08
2806 2.96661720902147e-08
2807 3.00797485245363e-08
2808 3.01221883403091e-08
2809 3.08174421909113e-08
2810 2.9841145610332e-08
2811 3.06853309318678e-08
2812 2.97832336274562e-08
2813 3.04894149676382e-08
2814 3.04499782874945e-08
2815 3.02269014526013e-08
2816 3.03674395919495e-08
2817 3.00496695395669e-08
2818 3.03175364632091e-08
2819 3.01951644676723e-08
2820 3.03144070078476e-08
2821 3.00719932979643e-08
2822 2.98872187804999e-08
2823 2.97847066936807e-08
2824 3.07693909064044e-08
2825 2.9864799988899e-08
2826 3.00249269735708e-08
2827 2.99321353470772e-08
2828 3.10271121778394e-08
2829 2.96140740800865e-08
2830 3.04411110551017e-08
2831 2.99901704134964e-08
2832 3.03875774156914e-08
2833 2.95848761442397e-08
2834 3.05438481412734e-08
2835 3.08759409459736e-08
2836 2.95599189543694e-08
2837 3.0211295529492e-08
2838 3.0747615627047e-08
2839 2.9869142288752e-08
2840 3.04953748877512e-08
2841 2.96397086108069e-08
2842 3.03725929280407e-08
2843 3.00885286664832e-08
2844 3.00705107834087e-08
2845 3.06329187435006e-08
2846 2.96317419279379e-08
2847 3.00403704396324e-08
2848 3.0008796229164e-08
2849 2.97508405038283e-08
2850 2.98379933466597e-08
2851 3.01924356720384e-08
2852 3.06118632749097e-08
2853 2.9218375007245e-08
2854 3.02018128063297e-08
2855 2.99016858549272e-08
2856 3.04718545627303e-08
2857 2.93275386267222e-08
2858 3.01755835104833e-08
2859 3.00002698984114e-08
2860 2.99635612147631e-08
2861 2.9927953717368e-08
2862 3.05264667795901e-08
2863 3.03320198710155e-08
2864 2.97671063652638e-08
2865 3.02339402591389e-08
2866 3.00254801773869e-08
2867 3.01414917248666e-08
2868 2.95447685393357e-08
2869 3.05562268447801e-08
2870 2.97485378843998e-08
2871 3.02313311261182e-08
2872 2.94816310139057e-08
2873 3.08328494461607e-08
2874 2.94303194030565e-08
2875 3.00683954851388e-08
2876 3.00148836750447e-08
2877 3.01245541097783e-08
2878 2.99606432618837e-08
2879 3.0320492878988e-08
2880 3.0090809986727e-08
2881 2.99756684490893e-08
2882 2.99208036064202e-08
2883 3.00536987541955e-08
2884 2.96635140840662e-08
2885 2.98708604775655e-08
2886 2.97175998965615e-08
2887 2.98345021303437e-08
2888 2.98459211258706e-08
2889 3.05587463849122e-08
2890 2.94302256796941e-08
2891 3.02539981432703e-08
2892 3.02326190324553e-08
2893 2.96479257867466e-08
2894 3.01593435423353e-08
2895 2.98447380652211e-08
2896 3.01157043552935e-08
2897 2.96811344009384e-08
2898 2.98265569683709e-08
2899 3.04087652166496e-08
2900 2.9773933025834e-08
2901 2.96000702226795e-08
2902 3.01181795324679e-08
2903 2.97845695773624e-08
2904 2.99198844405657e-08
2905 2.936822757027e-08
2906 3.08102235172525e-08
2907 2.93956256525174e-08
2908 2.97442584427099e-08
2909 2.99353175291506e-08
2910 2.95857488409323e-08
2911 2.97653062385494e-08
2912 2.9994501271835e-08
2913 2.92622536524068e-08
2914 3.05967503270166e-08
2915 2.93355425057262e-08
2916 3.01699639793096e-08
2917 2.97794769605897e-08
2918 2.9823608771129e-08
2919 2.96272451449919e-08
2920 3.00517382851861e-08
2921 2.94350455285475e-08
2922 3.04491430559439e-08
2923 2.96217179212022e-08
2924 2.99532024078397e-08
2925 2.90375496527773e-08
2926 3.06572493886659e-08
2927 2.90367586965878e-08
2928 3.06155740752434e-08
2929 2.93784848101053e-08
2930 3.0662706354545e-08
2931 2.89213779962605e-08
2932 3.02593410075325e-08
2933 2.9404636086694e-08
2934 3.0604953070168e-08
2935 2.89432003864887e-08
2936 3.00651323071888e-08
2937 2.9916686605902e-08
2938 2.9389192259921e-08
2939 2.97772740488478e-08
2940 3.00510902242479e-08
2941 2.8880541241616e-08
2942 3.06955651859786e-08
2943 2.88115905148212e-08
2944 3.01322200266796e-08
2945 2.90386839576495e-08
2946 3.04915254443205e-08
2947 2.95028359612592e-08
2948 3.00358011423585e-08
2949 2.98408934561456e-08
2950 2.94010663064448e-08
2951 3.02666170768351e-08
2952 2.88903330367685e-08
2953 3.03421128373671e-08
2954 2.94749148435081e-08
2955 3.00094201984846e-08
2956 2.87966685512675e-08
2957 3.04985406401892e-08
2958 2.87072389459908e-08
2959 3.03508502366157e-08
2960 2.94627191557151e-08
2961 3.052354272437e-08
2962 2.88008919672178e-08
2963 3.02096326415535e-08
2964 2.9181316504423e-08
2965 2.99309994258312e-08
2966 2.95488806141275e-08
2967 2.9556479886339e-08
2968 2.9797419580313e-08
2969 3.0270489753681e-08
2970 2.89987011853343e-08
2971 2.99435187509633e-08
2972 2.95964997739651e-08
2973 2.98927701355423e-08
2974 2.88056784466528e-08
2975 3.01303747365367e-08
2976 2.9058389521941e-08
2977 3.01704017898752e-08
2978 2.90198923819096e-08
2979 3.02237692444418e-08
2980 2.92311390949207e-08
2981 3.02752731166089e-08
2982 2.84788139214287e-08
2983 2.99612344359934e-08
2984 2.91533223537588e-08
2985 3.02469809451145e-08
2986 2.92148173877127e-08
2987 2.94593813241484e-08
2988 3.0242823223392e-08
2989 2.92364651265364e-08
2990 2.98012354676969e-08
2991 2.90721084056367e-08
2992 3.04038320295685e-08
2993 2.84168786459649e-08
2994 3.05301749702336e-08
2995 2.83147289776853e-08
2996 3.01417422444716e-08
2997 2.96399062642561e-08
2998 2.98488833369026e-08
2999 2.90642102948047e-08
3000 9.86126523552788e-09
3001 9.90268400145877e-09
3002 1.00913076529693e-08
3003 1.0235416597032e-08
3004 1.03031717516924e-08
3005 1.03234355331994e-08
3006 1.03272247138284e-08
3007 1.0326839838759e-08
3008 1.0325501374267e-08
3009 1.03239923431447e-08
3010 1.03224626789092e-08
3011 1.03210847994042e-08
3012 1.031969614812e-08
3013 1.03183256818448e-08
3014 1.03170293388674e-08
3015 1.03157900309472e-08
3016 1.03146057779946e-08
3017 1.03133720475365e-08
3018 1.03122190338797e-08
3019 1.0311092674499e-08
3020 1.03099883004276e-08
3021 1.03089141070406e-08
3022 1.03078650064425e-08
3023 1.03067928138856e-08
3024 1.03057318988145e-08
3025 1.03047201977999e-08
3026 1.03037577507265e-08
3027 1.03027583765528e-08
3028 1.03018446705461e-08
3029 1.03008099405291e-08
3030 1.02998753454334e-08
3031 1.0298943096225e-08
3032 1.0298064601337e-08
3033 1.02971448702818e-08
3034 1.02962023305966e-08
3035 1.02953360294686e-08
3036 1.02944332743549e-08
3037 1.02935523110526e-08
3038 1.0292728512197e-08
3039 1.0291881556497e-08
3040 1.02910453107924e-08
3041 1.02901784871448e-08
3042 1.02893118791997e-08
3043 1.02885317455267e-08
3044 1.02877088748732e-08
3045 1.02868849016016e-08
3046 1.02860720535558e-08
3047 1.02852643415821e-08
3048 1.02845022725909e-08
3049 1.02836745814669e-08
3050 1.02829134784807e-08
3051 1.0282160571029e-08
3052 1.02813525670459e-08
3053 1.02805792132343e-08
3054 1.02797952743414e-08
3055 1.02790583043841e-08
3056 1.02782833974607e-08
3057 1.02775323072013e-08
3058 1.02768038199585e-08
3059 1.0276109665211e-08
3060 1.02753196407832e-08
3061 1.02746179571067e-08
3062 1.02738443967243e-08
3063 1.02731544318752e-08
3064 1.02724320568265e-08
3065 1.02717017133533e-08
3066 1.02709588475219e-08
3067 1.02702632894594e-08
3068 1.02695501667804e-08
3069 1.02688239446147e-08
3070 1.02681720880252e-08
3071 1.02674714935261e-08
3072 1.02667784873042e-08
3073 1.0266063915236e-08
3074 1.02653978622594e-08
3075 1.02646817443236e-08
3076 1.02639819081346e-08
3077 1.02633395196727e-08
3078 1.02626339922793e-08
3079 1.02619348434155e-08
3080 1.02611976574088e-08
3081 1.02605777678386e-08
3082 1.02599683586835e-08
3083 1.0259237190148e-08
3084 1.0258606002421e-08
3085 1.0257908667613e-08
3086 1.0257261545972e-08
3087 1.02565713870281e-08
3088 1.02559366320296e-08
3089 1.02552907984832e-08
3090 1.02546320005018e-08
3091 1.02539801753801e-08
3092 1.02533132447999e-08
3093 1.02526996210855e-08
3094 1.02520263136063e-08
3095 1.02514001976001e-08
3096 1.02508107121951e-08
3097 1.02501214272813e-08
3098 1.02494874546361e-08
3099 1.02488283477004e-08
3100 1.0248235828382e-08
3101 1.02476261613568e-08
3102 1.02469768097357e-08
3103 1.02463700196107e-08
3104 1.02457756884014e-08
3105 1.02451507877563e-08
3106 1.02445269134499e-08
3107 1.02438800083093e-08
3108 1.02432631161331e-08
3109 1.02426068859102e-08
3110 1.024201915302e-08
3111 1.02413971408838e-08
3112 1.02407963877976e-08
3113 1.02401989584625e-08
3114 1.02395868337296e-08
3115 1.02389315394594e-08
3116 1.02383475622456e-08
3117 1.02377594657782e-08
3118 1.02371667143053e-08
3119 1.02365255745876e-08
3120 1.0235948503988e-08
3121 1.023532337565e-08
3122 1.02347483564649e-08
3123 1.02341591430577e-08
3124 1.02335553619284e-08
3125 1.0232974630528e-08
3126 1.02323386111172e-08
3127 1.02318140833696e-08
3128 1.02311881054959e-08
3129 1.02305807378467e-08
3130 1.02300007799871e-08
3131 1.02294290515656e-08
3132 1.02288351746041e-08
3133 1.02282541056403e-08
3134 1.02276805161033e-08
3135 1.02271170203325e-08
3136 1.02265300535725e-08
3137 1.02259135547345e-08
3138 1.0225379402963e-08
3139 1.02247782831841e-08
3140 1.02242306119146e-08
3141 1.02236224035282e-08
3142 1.0223063080822e-08
3143 1.02224891572328e-08
3144 1.02218702791035e-08
3145 1.02213766035447e-08
3146 1.02207555773975e-08
3147 1.02201888776618e-08
3148 1.0219661052413e-08
3149 1.02190721005604e-08
3150 1.02184966997224e-08
3151 1.02179060133267e-08
3152 1.02173405430658e-08
3153 1.02167813465712e-08
3154 1.0216231131642e-08
3155 1.02156945527218e-08
3156 1.02150971185641e-08
3157 1.02145210525775e-08
3158 1.02139697859202e-08
3159 1.02133942341612e-08
3160 1.02128351390091e-08
3161 1.02122869766325e-08
3162 1.02117157241913e-08
3163 1.02111637566502e-08
3164 1.02106068058869e-08
3165 1.02100019733556e-08
3166 1.02095176492192e-08
3167 1.02089694602528e-08
3168 1.02083670640851e-08
3169 1.02078532397618e-08
3170 1.02072892623831e-08
3171 1.02067113943241e-08
3172 1.02061691263505e-08
3173 1.02056074330129e-08
3174 1.02050728478104e-08
3175 1.02045111488938e-08
3176 1.02039984629515e-08
3177 1.02034037907719e-08
3178 1.02028789727426e-08
3179 1.02023390902567e-08
3180 1.02017614601116e-08
3181 1.02012642054367e-08
3182 1.02006950200365e-08
3183 1.0200155417632e-08
3184 1.0199625499481e-08
3185 1.01990397297566e-08
3186 1.01985268882027e-08
3187 1.01979851177547e-08
3188 1.01974714762296e-08
3189 1.01969023722018e-08
3190 1.01963582376796e-08
3191 1.0195864344531e-08
3192 1.01952871879793e-08
3193 1.01947571288161e-08
3194 1.01941912953596e-08
3195 1.01936819097495e-08
3196 1.01931403486827e-08
3197 1.01926273688782e-08
3198 1.01920881618936e-08
3199 1.01915811585793e-08
3200 1.01910264933905e-08
3201 1.01904759011867e-08
3202 1.01899675534062e-08
3203 1.01894105201256e-08
3204 1.01888737125272e-08
3205 1.01883306523956e-08
3206 1.01878693921434e-08
3207 1.01873326174076e-08
3208 1.01867710301726e-08
3209 1.01862781319156e-08
3210 1.01857182931223e-08
3211 1.01851956565008e-08
3212 1.01846850444828e-08
3213 1.01841572605829e-08
3214 1.01836413286469e-08
3215 1.01830628464541e-08
3216 1.0182581879542e-08
3217 1.01820269607089e-08
3218 1.01815241219591e-08
3219 1.01810163715202e-08
3220 1.0180491013527e-08
3221 1.01799460543797e-08
3222 1.01794785042422e-08
3223 1.0178913244438e-08
3224 1.01784123550733e-08
3225 1.01779146849043e-08
3226 1.01774232251425e-08
3227 1.01768725823334e-08
3228 1.01763756119658e-08
3229 1.01758552229977e-08
3230 1.01753199641189e-08
3231 1.0174818775209e-08
3232 1.01742710005084e-08
3233 1.01737137470359e-08
3234 1.01732412940633e-08
3235 1.01727557503262e-08
3236 1.01722414996355e-08
3237 1.01717050769098e-08
3238 1.01712156910586e-08
3239 1.01707070809948e-08
3240 1.01701768057821e-08
3241 1.01696314655231e-08
3242 1.01691519777611e-08
3243 1.01686625135836e-08
3244 1.01681155172326e-08
3245 1.01676203206891e-08
3246 1.01671077396906e-08
3247 1.01666141895107e-08
3248 1.01661047734319e-08
3249 1.01655707548873e-08
3250 1.01650498697467e-08
3251 1.01645812538645e-08
3252 1.01640663350555e-08
3253 1.0163550926888e-08
3254 1.01630530888533e-08
3255 1.01625241695213e-08
3256 1.01620963091423e-08
3257 1.01615424850515e-08
3258 1.01610453232329e-08
3259 1.01605161158813e-08
3260 1.01600398985172e-08
3261 1.01595145110614e-08
3262 1.0159012184062e-08
3263 1.0158522609105e-08
3264 1.01580326659148e-08
3265 1.01575461832343e-08
3266 1.01569987043179e-08
3267 1.01565008545218e-08
3268 1.01560312399454e-08
3269 1.01555251494564e-08
3270 1.01549925410199e-08
3271 1.01545436845679e-08
3272 1.01540372050915e-08
3273 1.01535342729511e-08
3274 1.01530017014365e-08
3275 1.01525300573654e-08
3276 1.01520109560271e-08
3277 1.01515255557724e-08
3278 1.01510521948664e-08
3279 1.01505518055531e-08
3280 1.0150032914491e-08
3281 1.01495013392142e-08
3282 1.01490331644374e-08
3283 1.01485339231502e-08
3284 1.01480380076749e-08
3285 1.01475292905995e-08
3286 1.01470524597538e-08
3287 1.01465244215157e-08
3288 1.01460336608544e-08
3289 1.01455570271775e-08
3290 1.01450312459603e-08
3291 1.01445559782323e-08
3292 1.01440842397993e-08
3293 1.01435616704434e-08
3294 1.01430789043247e-08
3295 1.01425874754896e-08
3296 1.01421014555916e-08
3297 1.01416183015055e-08
3298 1.01411414432859e-08
3299 1.01405974968008e-08
3300 1.01401243703877e-08
3301 1.0139628888059e-08
3302 1.01391500438447e-08
3303 1.01386614347124e-08
3304 1.01381839902187e-08
3305 1.01377273903427e-08
3306 1.01371935770506e-08
3307 1.01367273251385e-08
3308 1.01362360324445e-08
3309 1.01357225368373e-08
3310 1.01352332043392e-08
3311 1.01347427636234e-08
3312 1.01342910847901e-08
3313 1.01338019150785e-08
3314 1.01333151031544e-08
3315 1.01328102834544e-08
3316 1.01323078938592e-08
3317 1.01318324038924e-08
3318 1.01313214626655e-08
3319 1.01308559534649e-08
3320 1.01303546066675e-08
3321 1.01299017600184e-08
3322 1.01294177070749e-08
3323 1.01289222809581e-08
3324 1.01284332145526e-08
3325 1.01279796443712e-08
3326 1.01274702343848e-08
3327 1.01269884336327e-08
3328 1.01265186964669e-08
3329 1.01260178896403e-08
3330 1.01255569209535e-08
3331 1.01250581094545e-08
3332 1.01246200015573e-08
3333 1.01241214479006e-08
3334 1.01236170205674e-08
3335 1.01231068033864e-08
3336 1.01226387084138e-08
3337 1.01221880517005e-08
3338 1.01216603275098e-08
3339 1.01212473881535e-08
3340 1.01207071525125e-08
3341 1.01202628292052e-08
3342 1.01197829889682e-08
3343 1.01193263456964e-08
3344 1.01188278992803e-08
3345 1.01183197433602e-08
3346 1.01178882186762e-08
3347 1.01174218489694e-08
3348 1.01169156852196e-08
3349 1.01163832712109e-08
3350 1.01159038511933e-08
3351 1.01154466690054e-08
3352 1.01149826950211e-08
3353 1.0114504100843e-08
3354 1.01140476803027e-08
3355 1.01135772433633e-08
3356 1.01130666207636e-08
3357 1.01126182745007e-08
3358 1.01121278721986e-08
3359 1.01117170248355e-08
3360 1.01111561060171e-08
3361 1.01107159323696e-08
3362 1.01103069783393e-08
3363 1.01097581364645e-08
3364 1.01093157300822e-08
3365 1.01088418064457e-08
3366 1.01084221711331e-08
3367 1.01078636978724e-08
3368 1.01074600616091e-08
3369 1.01069896932121e-08
3370 1.01064726870242e-08
3371 1.01060066094519e-08
3372 1.0105538411638e-08
3373 1.01050356758822e-08
3374 1.01045835209021e-08
3375 1.01041142658714e-08
3376 1.01036467686152e-08
3377 1.0103167909857e-08
3378 1.01026597572884e-08
3379 1.01022048218727e-08
3380 1.01017649052765e-08
3381 1.01012801478687e-08
3382 1.01007990530588e-08
3383 1.01003086675974e-08
3384 1.00998643972547e-08
3385 1.00993570660568e-08
3386 1.00989329508763e-08
3387 1.00984292918105e-08
3388 1.00979558559783e-08
3389 1.00975014215507e-08
3390 1.00970242065193e-08
3391 1.00965340370726e-08
3392 1.00960658184282e-08
3393 1.00956153777851e-08
3394 1.00951542549993e-08
3395 1.00946935768162e-08
3396 1.00942183502498e-08
3397 1.00937410861951e-08
3398 1.00932616620072e-08
3399 1.00928375243101e-08
3400 1.00923509416331e-08
3401 1.00918906953129e-08
3402 1.00914376768776e-08
3403 1.00909317833908e-08
3404 1.00904812095209e-08
3405 1.00900472533513e-08
3406 1.00895288053515e-08
3407 1.00890852579583e-08
3408 1.00886354944332e-08
3409 1.00881433373126e-08
3410 1.00877244896408e-08
3411 1.00872064490026e-08
3412 1.0086760469076e-08
3413 1.00863163254097e-08
3414 1.00858467939682e-08
3415 1.0085364397068e-08
3416 1.00849399825506e-08
3417 1.00844430399732e-08
3418 1.00839938632288e-08
3419 1.00835274540606e-08
3420 1.00830742697094e-08
3421 1.00826049939245e-08
3422 1.00821309354307e-08
3423 1.00816527809564e-08
3424 1.00812143593171e-08
3425 1.00807354983593e-08
3426 1.00802827167554e-08
3427 1.00798421891896e-08
3428 1.0079336195061e-08
3429 1.00788741658406e-08
3430 1.00784040945739e-08
3431 1.00779596950082e-08
3432 1.00774899635669e-08
3433 1.00771091658819e-08
3434 1.00765677318593e-08
3435 1.00761277370062e-08
3436 1.00756918250886e-08
3437 1.00751944337243e-08
3438 1.00747598690415e-08
3439 1.00742754311003e-08
3440 1.00738410084081e-08
3441 1.00734070594827e-08
3442 1.00728939760666e-08
3443 1.00724813724903e-08
3444 1.00720448652503e-08
3445 1.00715923824282e-08
3446 1.00710912508753e-08
3447 1.00706502198997e-08
3448 1.00701519338345e-08
3449 1.00697482014814e-08
3450 1.00692756184045e-08
3451 1.0068787914444e-08
3452 1.00683533509269e-08
3453 1.00678941279547e-08
3454 1.00674357159589e-08
3455 1.00669818414237e-08
3456 1.00665611814169e-08
3457 1.00660761270036e-08
3458 1.00656104480845e-08
3459 1.00651743070862e-08
3460 1.00647157449743e-08
3461 1.00642749476312e-08
3462 1.00638144120493e-08
3463 1.00633801706843e-08
3464 1.00629339263095e-08
3465 1.00624987591294e-08
3466 1.0062018015719e-08
3467 1.00615865528467e-08
3468 1.00611311180249e-08
3469 1.00606926914451e-08
3470 1.0060279464548e-08
3471 1.0059765778829e-08
3472 1.00593145554093e-08
3473 1.00588663874898e-08
3474 1.00584536070827e-08
3475 1.00579764444053e-08
3476 1.00575292676583e-08
3477 1.0057111344941e-08
3478 1.00566108890768e-08
3479 1.00562023745213e-08
3480 1.00557627596073e-08
3481 1.0055269093194e-08
3482 1.00548436670692e-08
3483 1.00543985651139e-08
3484 1.00539466236227e-08
3485 1.00535233166499e-08
3486 1.00530799820239e-08
3487 1.00526169729345e-08
3488 1.00521710345927e-08
3489 1.00517066829175e-08
3490 1.0051267821283e-08
3491 1.00508254951351e-08
3492 1.00503985763292e-08
3493 1.00499575766966e-08
3494 1.00495125944441e-08
3495 1.00490815443041e-08
3496 1.00485974554518e-08
3497 1.00481827693816e-08
3498 1.00477019710707e-08
3499 1.00473041691612e-08
3500 1.00468003126214e-08
3501 1.00464022125615e-08
3502 1.00459278891477e-08
3503 1.00454727863381e-08
3504 1.00450495723881e-08
3505 1.00446211025867e-08
3506 1.00441641882679e-08
3507 1.00437169687009e-08
3508 1.00432898549399e-08
3509 1.00428135820646e-08
3510 1.00423909486919e-08
3511 1.00419130425328e-08
3512 1.00415287786551e-08
3513 1.00410604361514e-08
3514 1.00406144476692e-08
3515 1.00401437244033e-08
3516 1.00397124468202e-08
3517 1.00392694163468e-08
3518 1.00388414977365e-08
3519 1.00383617974431e-08
3520 1.00379291967378e-08
3521 1.00374785303864e-08
3522 1.00370250366091e-08
3523 1.00365788050852e-08
3524 1.00361540745289e-08
3525 1.003569504332e-08
3526 1.00352421788796e-08
3527 1.00348175584714e-08
3528 1.00344236382346e-08
3529 1.00339533018676e-08
3530 1.00335146460823e-08
3531 1.00330557687642e-08
3532 1.00326332694578e-08
3533 1.00322152181212e-08
3534 1.00317648848366e-08
3535 1.00313065838006e-08
3536 1.00308903326282e-08
3537 1.00304383802916e-08
3538 1.00299836787859e-08
3539 1.0029532296639e-08
3540 1.00291284843568e-08
3541 1.00286367407318e-08
3542 1.00282170069979e-08
3543 1.00277923204967e-08
3544 1.00273093386685e-08
3545 1.00269030170325e-08
3546 1.00264206005021e-08
3547 1.00260223650714e-08
3548 1.00255361468668e-08
3549 1.00251256835923e-08
3550 1.00246685195149e-08
3551 1.00242139904269e-08
3552 1.00237655618479e-08
3553 1.00233248026965e-08
3554 1.00229023148879e-08
3555 1.00224911760635e-08
3556 1.00220278915278e-08
3557 1.00215835426229e-08
3558 1.00211152758919e-08
3559 1.00207240542188e-08
3560 1.00202684822567e-08
3561 1.00198446014929e-08
3562 1.00193577398308e-08
3563 1.00189325232605e-08
3564 1.00185059419416e-08
3565 1.00180627172208e-08
3566 1.00176323822032e-08
3567 1.00172021907236e-08
3568 1.00167355904512e-08
3569 1.00163112331658e-08
3570 1.00158805991166e-08
3571 1.00154447433692e-08
3572 1.00149941361927e-08
3573 1.00145974652466e-08
3574 1.00141423583291e-08
3575 1.00137258286018e-08
3576 1.00132890243285e-08
3577 1.00128270095964e-08
3578 1.0012415685122e-08
3579 1.00119596613824e-08
3580 1.0011562426103e-08
3581 1.00110955356947e-08
3582 1.00106587448343e-08
3583 1.00102580258346e-08
3584 1.0009781677707e-08
3585 1.0009367367142e-08
3586 1.00089397629399e-08
3587 1.00084966330113e-08
3588 1.00080667564117e-08
3589 1.00076481801617e-08
3590 1.00072001752369e-08
3591 1.00067675318991e-08
3592 1.00063491395991e-08
3593 1.00059211587677e-08
3594 1.00054773253949e-08
3595 1.00050483827288e-08
3596 1.00046272806589e-08
3597 1.00041731688136e-08
3598 1.00036959114549e-08
3599 1.00032777238176e-08
3600 1.00028670293253e-08
3601 1.00024214200883e-08
3602 1.00019698170903e-08
3603 1.00015669523695e-08
3604 1.00010729572539e-08
3605 1.00006891428778e-08
3606 1.00002602878915e-08
3607 9.99980154090668e-09
3608 9.99936612958163e-09
3609 9.99895565279713e-09
3610 9.99849983074341e-09
3611 9.99806860386976e-09
3612 9.99758893946434e-09
3613 9.99716640575493e-09
3614 9.9967446855545e-09
3615 9.99635507790281e-09
3616 9.99594867413106e-09
3617 9.99547351687724e-09
3618 9.99503628984488e-09
3619 9.99458629317301e-09
3620 9.99424266568544e-09
3621 9.99374778826168e-09
3622 9.9933288815815e-09
3623 9.99290337332787e-09
3624 9.99246774517626e-09
3625 9.99202661449566e-09
3626 9.99165438415189e-09
3627 9.99119871292198e-09
3628 9.99077428325695e-09
3629 9.99034666426812e-09
3630 9.98994359976274e-09
3631 9.98945035845383e-09
3632 9.98906806815719e-09
3633 9.98863975452263e-09
3634 9.98819896588093e-09
3635 9.98771251042863e-09
3636 9.9873001420045e-09
3637 9.98687426474049e-09
3638 9.9864501167668e-09
3639 9.98603158278155e-09
3640 9.98558318061199e-09
3641 9.98518082494238e-09
3642 9.98471431003928e-09
3643 9.98435790334345e-09
3644 9.98389619838447e-09
3645 9.98345286444546e-09
3646 9.98303952981505e-09
3647 9.98260908291471e-09
3648 9.98219025618446e-09
3649 9.98171731806524e-09
3650 9.98135751013179e-09
3651 9.98091396342549e-09
3652 9.98044471414716e-09
3653 9.98006941650004e-09
3654 9.97957422531337e-09
3655 9.97921111001165e-09
3656 9.97878035743222e-09
3657 9.97834506438061e-09
3658 9.97791782517132e-09
3659 9.97747104717478e-09
3660 9.97704340263e-09
3661 9.97665679990101e-09
3662 9.9762373199988e-09
3663 9.97580612210197e-09
3664 9.97538885594979e-09
3665 9.97497925158547e-09
3666 9.9745226821088e-09
3667 9.97408179004289e-09
3668 9.973697964967e-09
3669 9.97325390073023e-09
3670 9.97286579884626e-09
3671 9.97237463871981e-09
3672 9.97198374026731e-09
3673 9.97154563424363e-09
3674 9.97111716826565e-09
3675 9.97071184782178e-09
3676 9.97029832796453e-09
3677 9.96983476773267e-09
3678 9.96945546533762e-09
3679 9.9690161530705e-09
3680 9.9685995663748e-09
3681 9.96818025452567e-09
3682 9.96772543969404e-09
3683 9.96733587480675e-09
3684 9.9668891604468e-09
3685 9.96650564781543e-09
3686 9.96608213954531e-09
3687 9.96565583830794e-09
3688 9.96519291453918e-09
3689 9.96477372427335e-09
3690 9.96439418429751e-09
3691 9.96392071987012e-09
3692 9.96354617129741e-09
3693 9.9631131109737e-09
3694 9.96271630161055e-09
3695 9.96225481900842e-09
3696 9.96182965366105e-09
3697 9.96142278555934e-09
3698 9.96100569561342e-09
3699 9.96057295932912e-09
3700 9.96020160354966e-09
3701 9.95972727529243e-09
3702 9.95931175543086e-09
3703 9.95891570437779e-09
3704 9.95850060128006e-09
3705 9.95808332286685e-09
3706 9.95767853642637e-09
3707 9.95723894904599e-09
3708 9.95684224308624e-09
3709 9.95644474205415e-09
3710 9.95597876707333e-09
3711 9.95558069186858e-09
3712 9.95519751948587e-09
3713 9.95477264458672e-09
3714 9.95435894792646e-09
3715 9.95393977838016e-09
3716 9.95350253094746e-09
3717 9.95309228198765e-09
3718 9.9527137601696e-09
3719 9.9522959539286e-09
3720 9.95185523219866e-09
3721 9.95137941072616e-09
3722 9.95101830086886e-09
3723 9.95057642356328e-09
3724 9.95014146808193e-09
3725 9.94975118893959e-09
3726 9.94935146683695e-09
3727 9.9489101406322e-09
3728 9.94847721010744e-09
3729 9.9481406237445e-09
3730 9.94765487005028e-09
3731 9.94725391111756e-09
3732 9.94681068115788e-09
3733 9.94644662305477e-09
3734 9.94598892280207e-09
3735 9.94559155453184e-09
3736 9.94516356759428e-09
3737 9.94478237870827e-09
3738 9.94433017016683e-09
3739 9.94392807741884e-09
3740 9.94347810001628e-09
3741 9.94312215720633e-09
3742 9.94267975891472e-09
3743 9.94228644161238e-09
3744 9.94187301598531e-09
3745 9.94146194299023e-09
3746 9.94102656596413e-09
3747 9.94062174184546e-09
3748 9.94021480092699e-09
3749 9.9397936287951e-09
3750 9.93941359375e-09
3751 9.93893666186019e-09
3752 9.93857317650726e-09
3753 9.93814536916909e-09
3754 9.93772279302835e-09
3755 9.93731448598045e-09
3756 9.93691546505998e-09
3757 9.93647183499674e-09
3758 9.93604086238498e-09
3759 9.93565576991318e-09
3760 9.93522804694502e-09
3761 9.93481017266817e-09
3762 9.93441656938626e-09
3763 9.93398143949581e-09
3764 9.93364457572282e-09
3765 9.93313826596481e-09
3766 9.93273104015618e-09
3767 9.93232310792652e-09
3768 9.93192735247728e-09
3769 9.93149571150431e-09
3770 9.93105532710176e-09
3771 9.93067174957091e-09
3772 9.93024084028349e-09
3773 9.92988556521718e-09
3774 9.92942381271983e-09
3775 9.92900874631497e-09
3776 9.92861712454524e-09
3777 9.92818714824068e-09
3778 9.92779923447706e-09
3779 9.92735974664205e-09
3780 9.92698089976457e-09
3781 9.92651267326533e-09
3782 9.92609994120847e-09
3783 9.92568944247624e-09
3784 9.92529554694893e-09
3785 9.92487851066148e-09
3786 9.92447868051333e-09
3787 9.92405538362273e-09
3788 9.92362957110554e-09
3789 9.92323921954197e-09
3790 9.9228399495499e-09
3791 9.92240869331779e-09
3792 9.92203295255983e-09
3793 9.92159140459498e-09
3794 9.92120017816872e-09
3795 9.92079098635634e-09
3796 9.92038952228602e-09
3797 9.91998692717599e-09
3798 9.91953666246853e-09
3799 9.91913208057277e-09
3800 9.91873084991296e-09
3801 9.91833013916366e-09
3802 9.91789636500123e-09
3803 9.91747424192862e-09
3804 9.91709349729142e-09
3805 9.9166331050185e-09
3806 9.91623488252186e-09
3807 9.91583484064723e-09
3808 9.9154938009785e-09
3809 9.9150107321852e-09
3810 9.91464114915436e-09
3811 9.91421953537575e-09
3812 9.9138160744236e-09
3813 9.91339076703707e-09
3814 9.91298240612254e-09
3815 9.91257912071053e-09
3816 9.91214553094227e-09
3817 9.91180254634322e-09
3818 9.91132525551136e-09
3819 9.91097031814714e-09
3820 9.91056800030837e-09
3821 9.91009572971768e-09
3822 9.90969041176487e-09
3823 9.9093109382567e-09
3824 9.90889271569595e-09
3825 9.90849690148121e-09
3826 9.90807693409396e-09
3827 9.90764233940733e-09
3828 9.90723055614706e-09
3829 9.90685228623861e-09
3830 9.90641024405797e-09
3831 9.90600214663406e-09
3832 9.90563263691957e-09
3833 9.90518510335381e-09
3834 9.90477723718242e-09
3835 9.90439926507741e-09
3836 9.90398613107124e-09
3837 9.90357800306069e-09
3838 9.90314127288783e-09
3839 9.90274814900022e-09
3840 9.90232574644284e-09
3841 9.90192641063537e-09
3842 9.90154845403879e-09
3843 9.90108749811952e-09
3844 9.9007083131028e-09
3845 9.9002861842501e-09
3846 9.89987095135342e-09
3847 9.89950700236442e-09
3848 9.89909630517288e-09
3849 9.89867167975472e-09
3850 9.89823308660681e-09
3851 9.89786901891415e-09
3852 9.89747017915432e-09
3853 9.89703595820346e-09
3854 9.89664092308079e-09
3855 9.89621449504902e-09
3856 9.8958026174753e-09
3857 9.89542447726172e-09
3858 9.89497361673225e-09
3859 9.89459541058529e-09
3860 9.89417955882949e-09
3861 9.89379692379472e-09
3862 9.89335397502678e-09
3863 9.89295009112823e-09
3864 9.89255896541308e-09
3865 9.89213765355962e-09
3866 9.89178609815677e-09
3867 9.89135841108352e-09
3868 9.89090114836971e-09
3869 9.89058435090645e-09
3870 9.89012993738575e-09
3871 9.88971488957713e-09
3872 9.88936868662887e-09
3873 9.8889473513844e-09
3874 9.88855549543394e-09
3875 9.88815656300518e-09
3876 9.88771417548967e-09
3877 9.88734020979792e-09
3878 9.88688806922988e-09
3879 9.88652119034844e-09
3880 9.88613350653977e-09
3881 9.88578024934544e-09
3882 9.88530773223667e-09
3883 9.88490431547834e-09
3884 9.88449413612952e-09
3885 9.8840995812477e-09
3886 9.88368296499231e-09
3887 9.88328334760452e-09
3888 9.88287303866492e-09
3889 9.88245754295763e-09
3890 9.88208365420434e-09
3891 9.88167911634974e-09
3892 9.88126988382687e-09
3893 9.88085338673311e-09
3894 9.88050745430458e-09
3895 9.88003479155536e-09
3896 9.87966903167464e-09
3897 9.87923872272645e-09
3898 9.87882217433345e-09
3899 9.87844450466707e-09
3900 9.87800313238113e-09
3901 9.87761692295558e-09
3902 9.87723169163651e-09
3903 9.87681352875025e-09
3904 9.87643697027835e-09
3905 9.87600503701141e-09
3906 9.87559992809278e-09
3907 9.87521407360553e-09
3908 9.87481388538963e-09
3909 9.87440428242697e-09
3910 9.87401579320701e-09
3911 9.87359903014545e-09
3912 9.87320357852262e-09
3913 9.87280506998395e-09
3914 9.8724016655144e-09
3915 9.87201474426241e-09
3916 9.87158118247872e-09
3917 9.87122464579659e-09
3918 9.87080635638654e-09
3919 9.87040968115221e-09
3920 9.87000001841792e-09
3921 9.86962430005872e-09
3922 9.86917572797646e-09
3923 9.8687977979281e-09
3924 9.86836780182687e-09
3925 9.86798944036665e-09
3926 9.86759573939205e-09
3927 9.86715988258308e-09
3928 9.8667861198401e-09
3929 9.86636423414705e-09
3930 9.86599437691194e-09
3931 9.86557200610694e-09
3932 9.86522230659453e-09
3933 9.86480965481373e-09
3934 9.86438081149177e-09
3935 9.86404522604345e-09
3936 9.86362440925231e-09
3937 9.8632113674918e-09
3938 9.86278829517157e-09
3939 9.8624087117305e-09
3940 9.8620324817042e-09
3941 9.86162329194573e-09
3942 9.86120888285536e-09
3943 9.86080356073921e-09
3944 9.86043725375446e-09
3945 9.86007352211937e-09
3946 9.85965487618895e-09
3947 9.85925559652406e-09
3948 9.85886614247866e-09
3949 9.85842752173477e-09
3950 9.85805153651265e-09
3951 9.8576629025543e-09
3952 9.85728667620561e-09
3953 9.85681731125593e-09
3954 9.85645364547788e-09
3955 9.85608076179562e-09
3956 9.85567052463887e-09
3957 9.8552880587327e-09
3958 9.8548625517697e-09
3959 9.85451584887415e-09
3960 9.85409291611589e-09
3961 9.85371548443276e-09
3962 9.85327886304094e-09
3963 9.85295745328102e-09
3964 9.85247777777337e-09
3965 9.85211070077263e-09
3966 9.85167809861715e-09
3967 9.8513313725665e-09
3968 9.8509591066262e-09
3969 9.85049648470626e-09
3970 9.85013704835058e-09
3971 9.84977074183768e-09
3972 9.84935683370075e-09
3973 9.84895340087194e-09
3974 9.84855893118586e-09
3975 9.84816355772272e-09
3976 9.84774981636211e-09
3977 9.84738366718863e-09
3978 9.84697667096718e-09
3979 9.84656576614312e-09
3980 9.8461754430082e-09
3981 9.8457755695891e-09
3982 9.84537260657892e-09
3983 9.84497867798778e-09
3984 9.84459904523893e-09
3985 9.84419268027642e-09
3986 9.84379698736049e-09
3987 9.84340517891369e-09
3988 9.84303956957922e-09
3989 9.84265076714452e-09
3990 9.84225767020758e-09
3991 9.84181288366665e-09
3992 9.84142532402948e-09
3993 9.8410491731829e-09
3994 9.8406390275016e-09
3995 9.84025218196682e-09
3996 9.83989327545814e-09
3997 9.83946044821882e-09
3998 9.83903194525654e-09
3999 9.8386565752992e-09
4000 9.83826864945497e-09
4001 9.83786653718788e-09
4002 9.83749848521703e-09
4003 9.83713360395294e-09
4004 9.83667502086477e-09
4005 9.83631591467554e-09
4006 9.83590082839769e-09
4007 9.83550763897223e-09
4008 9.83513917737766e-09
4009 9.83475864101135e-09
4010 9.83437862328573e-09
4011 9.83397674913372e-09
4012 9.83356296728466e-09
4013 9.8331546442218e-09
4014 9.83274239513277e-09
4015 9.83239651976275e-09
4016 9.83200112713439e-09
4017 9.83162908501506e-09
4018 9.83119810119004e-09
4019 9.83082583574935e-09
4020 9.83041062043583e-09
4021 9.83001546980139e-09
4022 9.82966295806631e-09
4023 9.82923728316126e-09
4024 9.82887743162381e-09
4025 9.82842630852659e-09
4026 9.82811220520685e-09
4027 9.82771391571519e-09
4028 9.82728131285887e-09
4029 9.82689365302408e-09
4030 9.8265257289995e-09
4031 9.8261395455046e-09
4032 9.82576105639649e-09
4033 9.82533997264529e-09
4034 9.82492843074057e-09
4035 9.82455759252626e-09
4036 9.82413972946267e-09
4037 9.82376215486608e-09
4038 9.82338323484572e-09
4039 9.82297671512505e-09
4040 9.82261184920286e-09
4041 9.82217775793992e-09
4042 9.82180320718146e-09
4043 9.82142158274801e-09
4044 9.82100378241896e-09
4045 9.82067454686347e-09
4046 9.82023748913319e-09
4047 9.81988536418593e-09
4048 9.81946822658369e-09
4049 9.81903862342509e-09
4050 9.81867903803585e-09
4051 9.81828937166029e-09
4052 9.81790898407775e-09
4053 9.81748787515224e-09
4054 9.8171274900069e-09
4055 9.81672421788288e-09
4056 9.81630539415718e-09
4057 9.81592859489871e-09
4058 9.81557512256398e-09
4059 9.81515581296305e-09
4060 9.81479514175487e-09
4061 9.81437589477052e-09
4062 9.81399900889385e-09
4063 9.81356149989954e-09
4064 9.81319675587983e-09
4065 9.81280625547004e-09
4066 9.81240109121373e-09
4067 9.81202336930442e-09
4068 9.81166578271214e-09
4069 9.81126142951538e-09
4070 9.81087189078772e-09
4071 9.81047692071718e-09
4072 9.81007508822629e-09
4073 9.80970461971625e-09
4074 9.80931641938326e-09
4075 9.80889762925569e-09
4076 9.8085131077924e-09
4077 9.80811268813664e-09
4078 9.80773388800649e-09
4079 9.80736300822821e-09
4080 9.80693260110854e-09
4081 9.80655836414074e-09
4082 9.8061507986108e-09
4083 9.80580757783262e-09
4084 9.80538778589529e-09
4085 9.80504447519598e-09
4086 9.80460648730003e-09
4087 9.80426538785273e-09
4088 9.80377623508583e-09
4089 9.80345457061993e-09
4090 9.80306218018728e-09
4091 9.80267903592097e-09
4092 9.8022928587127e-09
4093 9.80189268839221e-09
4094 9.80149659258328e-09
4095 9.80111456663768e-09
4096 9.80068168553866e-09
4097 9.80034907831912e-09
4098 9.79993590063261e-09
4099 9.79959631753485e-09
4100 9.7991745440576e-09
4101 9.79878824605013e-09
4102 9.79840164671425e-09
4103 9.79800415860932e-09
4104 9.79762535697343e-09
4105 9.79728377691752e-09
4106 9.79686953014175e-09
4107 9.79645068224094e-09
4108 9.7961085663012e-09
4109 9.79568966651134e-09
4110 9.79532750253892e-09
4111 9.79491579330971e-09
4112 9.79453605185615e-09
4113 9.79416761360402e-09
4114 9.7937349572072e-09
4115 9.79336115129043e-09
4116 9.79296549508818e-09
4117 9.79260110502145e-09
4118 9.79218221489747e-09
4119 9.79178864397162e-09
4120 9.79140855832117e-09
4121 9.7910139963131e-09
4122 9.79061197522307e-09
4123 9.79026520619985e-09
4124 9.78985850447894e-09
4125 9.78946304231593e-09
4126 9.78914290759164e-09
4127 9.78871249265184e-09
4128 9.78832265975671e-09
4129 9.7879160791195e-09
4130 9.78758262671492e-09
4131 9.78715581086836e-09
4132 9.78676140676871e-09
4133 9.78638276171589e-09
4134 9.7860069676256e-09
4135 9.78561836532249e-09
4136 9.78523566157879e-09
4137 9.7848278141216e-09
4138 9.78447065447946e-09
4139 9.78407607051673e-09
4140 9.78369083651925e-09
4141 9.78329223645658e-09
4142 9.78293800891344e-09
4143 9.78252394845391e-09
4144 9.78216486556549e-09
4145 9.78176027625205e-09
4146 9.78138583281052e-09
4147 9.78100568399531e-09
4148 9.78063017224862e-09
4149 9.78023136304074e-09
4150 9.77981711450943e-09
4151 9.7793956794906e-09
4152 9.77910951425326e-09
4153 9.77865899529778e-09
4154 9.7782936253829e-09
4155 9.77788212287722e-09
4156 9.77752986551506e-09
4157 9.77714792894935e-09
4158 9.77673906383397e-09
4159 9.7763920237498e-09
4160 9.77598247623584e-09
4161 9.77559534236228e-09
4162 9.77525776047622e-09
4163 9.77479546978438e-09
4164 9.77444512875736e-09
4165 9.77403540632776e-09
4166 9.7736433225179e-09
4167 9.77331174294854e-09
4168 9.77291607590774e-09
4169 9.77252782000809e-09
4170 9.7721163915751e-09
4171 9.77171680007632e-09
4172 9.77133600121166e-09
4173 9.77097065788662e-09
4174 9.77057303885864e-09
4175 9.77020893312003e-09
4176 9.76981070637678e-09
4177 9.76937035134656e-09
4178 9.76903451970629e-09
4179 9.76864832288871e-09
4180 9.76828905591837e-09
4181 9.76789568986336e-09
4182 9.7675092703281e-09
4183 9.76711240091577e-09
4184 9.76675114578579e-09
4185 9.76635989022312e-09
4186 9.76598082010754e-09
4187 9.7656060077389e-09
4188 9.76522894367837e-09
4189 9.76487005454468e-09
4190 9.76448082452147e-09
4191 9.76407587082589e-09
4192 9.76373386819135e-09
4193 9.76333224323583e-09
4194 9.7629686639511e-09
4195 9.76257908361783e-09
4196 9.7621700741457e-09
4197 9.76182990757718e-09
4198 9.76147874182098e-09
4199 9.76102771061554e-09
4200 9.76064537106663e-09
4201 9.76031538268279e-09
4202 9.75992358521333e-09
4203 9.75954167956733e-09
4204 9.75913629405745e-09
4205 9.75873982275027e-09
4206 9.75838055478073e-09
4207 9.7579950940202e-09
4208 9.75758669149313e-09
4209 9.75722931341461e-09
4210 9.75685548720156e-09
4211 9.7564647970616e-09
4212 9.75608074554185e-09
4213 9.75568884831884e-09
4214 9.75530534899627e-09
4215 9.7549334726818e-09
4216 9.75456274720371e-09
4217 9.75414402128866e-09
4218 9.75380333596149e-09
4219 9.75337910785051e-09
4220 9.7530387940456e-09
4221 9.7526833141362e-09
4222 9.75225514055628e-09
4223 9.75190170356133e-09
4224 9.75152393253159e-09
4225 9.75112176106185e-09
4226 9.75076124500041e-09
4227 9.75035095337334e-09
4228 9.75001962167515e-09
4229 9.74957513272951e-09
4230 9.74923763701757e-09
4231 9.74888385017747e-09
4232 9.74845146150399e-09
4233 9.7480576295092e-09
4234 9.74772429183229e-09
4235 9.74729336898628e-09
4236 9.74695529245423e-09
4237 9.74654147585519e-09
4238 9.74621325158798e-09
4239 9.74583889954556e-09
4240 9.74541839570547e-09
4241 9.74504418508465e-09
4242 9.74466771430649e-09
4243 9.74429310261066e-09
4244 9.74391397961377e-09
4245 9.7435644622404e-09
4246 9.74315866604514e-09
4247 9.7427693501323e-09
4248 9.74240307827917e-09
4249 9.74202707332977e-09
4250 9.74160661784684e-09
4251 9.74127773967909e-09
4252 9.7407904071159e-09
4253 9.74051154194189e-09
4254 9.74012246401229e-09
4255 9.73971439118676e-09
4256 9.73935074138205e-09
4257 9.73896372633704e-09
4258 9.73859943225602e-09
4259 9.73823371893529e-09
4260 9.73785674887595e-09
4261 9.73743748441253e-09
4262 9.73705298250999e-09
4263 9.73674242233519e-09
4264 9.73630613169268e-09
4265 9.7359642820885e-09
4266 9.73554546831318e-09
4267 9.73518706937071e-09
4268 9.73484139225878e-09
4269 9.73440535066705e-09
4270 9.73408171315443e-09
4271 9.7337112787213e-09
4272 9.73330590276628e-09
4273 9.73291488329947e-09
4274 9.73254697705928e-09
4275 9.73224275916534e-09
4276 9.73180480885044e-09
4277 9.73143125832659e-09
4278 9.73104242518036e-09
4279 9.73070878456522e-09
4280 9.7302906786681e-09
4281 9.72993460355426e-09
4282 9.72958329883977e-09
4283 9.72914924900897e-09
4284 9.72885316041783e-09
4285 9.72842449913081e-09
4286 9.72806213524885e-09
4287 9.7276817241157e-09
4288 9.72730226299345e-09
4289 9.72687497607233e-09
4290 9.72658459597736e-09
4291 9.7261552994693e-09
4292 9.72576209731096e-09
4293 9.72542591764353e-09
4294 9.72503087142557e-09
4295 9.72461621853021e-09
4296 9.72423883151274e-09
4297 9.72387056734358e-09
4298 9.72350452518866e-09
4299 9.72317091910147e-09
4300 9.7227807299774e-09
4301 9.72238027099198e-09
4302 9.72200790633204e-09
4303 9.7216473048875e-09
4304 9.72124993654094e-09
4305 9.7208817351549e-09
4306 9.72048369969614e-09
4307 9.72011770111747e-09
4308 9.71975298286881e-09
4309 9.71938398661859e-09
4310 9.71900647575574e-09
4311 9.71861850198458e-09
4312 9.71824355044948e-09
4313 9.71789650984489e-09
4314 9.71744963660509e-09
4315 9.71710798239323e-09
4316 9.71673351428393e-09
4317 9.71637510898543e-09
4318 9.71598160241782e-09
4319 9.71563672924614e-09
4320 9.71522534951724e-09
4321 9.71485858480836e-09
4322 9.7144793281892e-09
4323 9.71407559437198e-09
4324 9.71373240402779e-09
4325 9.71333847654687e-09
4326 9.71299989477314e-09
4327 9.7126070660522e-09
4328 9.71222984355591e-09
4329 9.71185032660332e-09
4330 9.71146393056316e-09
4331 9.71101738300728e-09
4332 9.71072114264865e-09
4333 9.71033892197687e-09
4334 9.70994396989194e-09
4335 9.70956794085071e-09
4336 9.70923498178761e-09
4337 9.70887748964755e-09
4338 9.70842624214985e-09
4339 9.70803055227315e-09
4340 9.70771036371693e-09
4341 9.70732101884808e-09
4342 9.70693506893022e-09
4343 9.70655997088732e-09
4344 9.70620837709157e-09
4345 9.70581456726655e-09
4346 9.7054631693419e-09
4347 9.7051117102856e-09
4348 9.7046989352631e-09
4349 9.70431160550456e-09
4350 9.70396818716912e-09
4351 9.70361873398745e-09
4352 9.70319954539389e-09
4353 9.70287171370848e-09
4354 9.70244944781262e-09
4355 9.70207595836492e-09
4356 9.70170361019179e-09
4357 9.70137968214768e-09
4358 9.70099656339568e-09
4359 9.70061897022367e-09
4360 9.70024647654194e-09
4361 9.69984428847437e-09
4362 9.69951454320162e-09
4363 9.69919798001623e-09
4364 9.69875133485787e-09
4365 9.69842176550689e-09
4366 9.69801988661562e-09
4367 9.69762323523027e-09
4368 9.69726760068762e-09
4369 9.69694273390831e-09
4370 9.6965441875041e-09
4371 9.69618344482531e-09
4372 9.69578018326922e-09
4373 9.69540615824299e-09
4374 9.69505124639308e-09
4375 9.69469101578385e-09
4376 9.69427571109738e-09
4377 9.69397655964377e-09
4378 9.69352237577964e-09
4379 9.69320622194042e-09
4380 9.69282078410599e-09
4381 9.69246960660225e-09
4382 9.69206252489363e-09
4383 9.69173412004171e-09
4384 9.69132685881696e-09
4385 9.69099249827077e-09
4386 9.69061409722416e-09
4387 9.69026081842211e-09
4388 9.68987407642258e-09
4389 9.68953019681029e-09
4390 9.68912055124976e-09
4391 9.68875601024127e-09
4392 9.68839528366072e-09
4393 9.68803194855566e-09
4394 9.68766918781755e-09
4395 9.68730361276121e-09
4396 9.68696416291798e-09
4397 9.68655894952736e-09
4398 9.68618617607636e-09
4399 9.68585076610573e-09
4400 9.68544893546058e-09
4401 9.68508922095934e-09
4402 9.68470185738063e-09
4403 9.68435227400449e-09
4404 9.68397574084567e-09
4405 9.68360201851481e-09
4406 9.6832408232328e-09
4407 9.68286372313859e-09
4408 9.68252114027374e-09
4409 9.68212823547476e-09
4410 9.68176952929195e-09
4411 9.68135372140383e-09
4412 9.68104223281196e-09
4413 9.68068651334419e-09
4414 9.68027102601909e-09
4415 9.67995018397172e-09
4416 9.67953618010581e-09
4417 9.67916417802389e-09
4418 9.67880807905414e-09
4419 9.67845588890209e-09
4420 9.67804956064633e-09
4421 9.67769979116212e-09
4422 9.67732358517909e-09
4423 9.67701530338316e-09
4424 9.67655153204239e-09
4425 9.67623483801722e-09
4426 9.67584646092284e-09
4427 9.67552257659376e-09
4428 9.67513164285005e-09
4429 9.67475421440739e-09
4430 9.67439311783808e-09
4431 9.67403294885316e-09
4432 9.67366835178535e-09
4433 9.67330899908497e-09
4434 9.67291254319602e-09
4435 9.67255448941495e-09
4436 9.67217346900529e-09
4437 9.67179637802185e-09
4438 9.67144130339243e-09
4439 9.67110017168621e-09
4440 9.67072220958015e-09
4441 9.6703156689526e-09
4442 9.66998862458185e-09
4443 9.66960460459937e-09
4444 9.66921656447156e-09
4445 9.66885601639406e-09
4446 9.66849933540376e-09
4447 9.66810896454312e-09
4448 9.66778857140749e-09
4449 9.66736371082327e-09
4450 9.66700402032367e-09
4451 9.6666284170599e-09
4452 9.66628039834189e-09
4453 9.66591729286564e-09
4454 9.66553072417886e-09
4455 9.66518883914469e-09
4456 9.66480917478918e-09
4457 9.66442602783751e-09
4458 9.664057385117e-09
4459 9.66369646029225e-09
4460 9.66336791461547e-09
4461 9.66296675592587e-09
4462 9.66261304760629e-09
4463 9.66228623190679e-09
4464 9.66183730953529e-09
4465 9.66151361626172e-09
4466 9.66114674156443e-09
4467 9.66074188828159e-09
4468 9.66044133039745e-09
4469 9.66005362994238e-09
4470 9.65970666322313e-09
4471 9.65930201052983e-09
4472 9.65891229182808e-09
4473 9.65858306668094e-09
4474 9.65817464124857e-09
4475 9.65779945404782e-09
4476 9.65740661346137e-09
4477 9.6570800758658e-09
4478 9.65669840411604e-09
4479 9.65632637059399e-09
4480 9.65595018904281e-09
4481 9.65562506755752e-09
4482 9.65524549187413e-09
4483 9.65483296225256e-09
4484 9.65448016621712e-09
4485 9.6541049714946e-09
4486 9.65376145713875e-09
4487 9.65340352206828e-09
4488 9.65302208802421e-09
4489 9.65266869415449e-09
4490 9.65227733234347e-09
4491 9.65195210929359e-09
4492 9.65156847961002e-09
4493 9.65120662076158e-09
4494 9.65082163861125e-09
4495 9.65046124559721e-09
4496 9.65009490550006e-09
4497 9.64970494222311e-09
4498 9.64937759174306e-09
4499 9.64900247622802e-09
4500 9.64863950246586e-09
4501 9.64824579467033e-09
4502 9.64791601498066e-09
4503 9.64749620745164e-09
4504 9.64716553209649e-09
4505 9.6467796998137e-09
4506 9.64641630895463e-09
4507 9.64602705562367e-09
4508 9.64572336300401e-09
4509 9.64528241729351e-09
4510 9.64497423115024e-09
4511 9.64460260654415e-09
4512 9.64425281632653e-09
4513 9.64387173920528e-09
4514 9.64349356653849e-09
4515 9.64314865511962e-09
4516 9.64277679300907e-09
4517 9.64235189517687e-09
4518 9.64204464060092e-09
4519 9.64170008604243e-09
4520 9.64127708436013e-09
4521 9.64089168696558e-09
4522 9.64056589736195e-09
4523 9.64022810956422e-09
4524 9.63978372101742e-09
4525 9.63944548680595e-09
4526 9.63907440281603e-09
4527 9.63869838233739e-09
4528 9.6383420988902e-09
4529 9.63798892080619e-09
4530 9.6376119562494e-09
4531 9.6372478329762e-09
4532 9.63686117518708e-09
4533 9.63649045423315e-09
4534 9.63612660760671e-09
4535 9.63581400228314e-09
4536 9.63541834202164e-09
4537 9.63505659067054e-09
4538 9.63473026160261e-09
4539 9.63434048207451e-09
4540 9.63399015733307e-09
4541 9.63358012976562e-09
4542 9.63321919005e-09
4543 9.63287778347338e-09
4544 9.6325008792919e-09
4545 9.63213065750812e-09
4546 9.6317732684037e-09
4547 9.63142048736321e-09
4548 9.63103751740191e-09
4549 9.63070531548316e-09
4550 9.63031095590344e-09
4551 9.62993412855634e-09
4552 9.62957666111874e-09
4553 9.62922026390839e-09
4554 9.6288601744085e-09
4555 9.62847223721919e-09
4556 9.6281444187038e-09
4557 9.62777219910838e-09
4558 9.62741746270146e-09
4559 9.62701393823762e-09
4560 9.62666487495933e-09
4561 9.62632154947324e-09
4562 9.62595019616402e-09
4563 9.62557635279804e-09
4564 9.62521909450564e-09
4565 9.62483861413621e-09
4566 9.62448875538313e-09
4567 9.62413636014514e-09
4568 9.623770882497e-09
4569 9.6234430459266e-09
4570 9.62302216923894e-09
4571 9.62269064137128e-09
4572 9.62231271844222e-09
4573 9.62194485005369e-09
4574 9.62158653682044e-09
4575 9.62120684348117e-09
4576 9.62084182967726e-09
4577 9.62047930908733e-09
4578 9.62014226683211e-09
4579 9.61977210100356e-09
4580 9.61940614459256e-09
4581 9.61900578894809e-09
4582 9.61864808476931e-09
4583 9.61830680087233e-09
4584 9.61794303355051e-09
4585 9.61755114880364e-09
4586 9.61718636250525e-09
4587 9.61684189691031e-09
4588 9.61648026360368e-09
4589 9.61609199413849e-09
4590 9.61575900270545e-09
4591 9.61537929383693e-09
4592 9.61504753538289e-09
4593 9.61466772832903e-09
4594 9.61431469591323e-09
4595 9.61393367113206e-09
4596 9.61357625354348e-09
4597 9.61319644644104e-09
4598 9.6128548858973e-09
4599 9.61246848844854e-09
4600 9.61209670293645e-09
4601 9.61179554696823e-09
4602 9.6114052396748e-09
4603 9.61102190705915e-09
4604 9.61069923458707e-09
4605 9.61032201418632e-09
4606 9.60996114253432e-09
4607 9.60956438705862e-09
4608 9.60920494905843e-09
4609 9.60883688403552e-09
4610 9.60849805774211e-09
4611 9.60813301677937e-09
4612 9.60781330010263e-09
4613 9.60741730589298e-09
4614 9.60706073838829e-09
4615 9.60672981704935e-09
4616 9.60636882776228e-09
4617 9.60599136317492e-09
4618 9.60558922727395e-09
4619 9.60525731135559e-09
4620 9.60493185717809e-09
4621 9.60455709025226e-09
4622 9.60419803099771e-09
4623 9.60383494987699e-09
4624 9.60343290126037e-09
4625 9.60315168225395e-09
4626 9.60275803572191e-09
4627 9.6023779518617e-09
4628 9.60200640535286e-09
4629 9.60165567560206e-09
4630 9.60129131136883e-09
4631 9.6009028770494e-09
4632 9.60055859903358e-09
4633 9.60019925385497e-09
4634 9.59983372177814e-09
4635 9.59950856323222e-09
4636 9.5991620037289e-09
4637 9.59876998302828e-09
4638 9.59843415010431e-09
4639 9.5980156551087e-09
4640 9.59772487443833e-09
4641 9.59730839304035e-09
4642 9.59696451922898e-09
4643 9.59660458915018e-09
4644 9.59625335558018e-09
4645 9.59590038936142e-09
4646 9.59553369659499e-09
4647 9.5951788978213e-09
4648 9.59482724088162e-09
4649 9.59443408840577e-09
4650 9.59405877959396e-09
4651 9.59373066094366e-09
4652 9.59334071130857e-09
4653 9.59299706358024e-09
4654 9.59265785661911e-09
4655 9.59223143462418e-09
4656 9.59192969617745e-09
4657 9.59154868613449e-09
4658 9.59119524723134e-09
4659 9.59084728057585e-09
4660 9.59051581416598e-09
4661 9.59012096678896e-09
4662 9.58974929089057e-09
4663 9.5893804254385e-09
4664 9.58904098997265e-09
4665 9.58869989167027e-09
4666 9.58834088855137e-09
4667 9.58797559923175e-09
4668 9.58756947067735e-09
4669 9.58724893680013e-09
4670 9.58687911558481e-09
4671 9.58651170492042e-09
4672 9.58617586236526e-09
4673 9.58579190758757e-09
4674 9.58543518007893e-09
4675 9.58507917948881e-09
4676 9.58471162730568e-09
4677 9.58434828054328e-09
4678 9.5840096619379e-09
4679 9.58362896091164e-09
4680 9.58332403772078e-09
4681 9.58294559898903e-09
4682 9.58258329246398e-09
4683 9.58222238990614e-09
4684 9.58181323729851e-09
4685 9.5815023990059e-09
4686 9.58117056371749e-09
4687 9.58076653010537e-09
4688 9.58039463355015e-09
4689 9.5800311109423e-09
4690 9.57973069105888e-09
4691 9.57935577689667e-09
4692 9.57900758204178e-09
4693 9.57866740770169e-09
4694 9.57825099347914e-09
4695 9.5779073487276e-09
4696 9.5775347867047e-09
4697 9.57717554280618e-09
4698 9.57680276830741e-09
4699 9.57642856140584e-09
4700 9.57608604020693e-09
4701 9.5757197453028e-09
4702 9.57534880935418e-09
4703 9.57501446682829e-09
4704 9.57464718327056e-09
4705 9.57424035587934e-09
4706 9.57388618203631e-09
4707 9.57356582866054e-09
4708 9.57322087413032e-09
4709 9.57285146797882e-09
4710 9.57248439650837e-09
4711 9.57212161714627e-09
4712 9.57176893692097e-09
4713 9.5714328221877e-09
4714 9.57102154865164e-09
4715 9.57071852142333e-09
4716 9.57033099901333e-09
4717 9.5699823251319e-09
4718 9.5696213347693e-09
4719 9.56924683503252e-09
4720 9.56888983671272e-09
4721 9.56855323907407e-09
4722 9.56812791063494e-09
4723 9.56786374686769e-09
4724 9.56745812921711e-09
4725 9.56711019806794e-09
4726 9.56673463844288e-09
4727 9.56636565328101e-09
4728 9.56605098002133e-09
4729 9.56564369386514e-09
4730 9.56530643526909e-09
4731 9.56496994390654e-09
4732 9.56458724953035e-09
4733 9.56423259081823e-09
4734 9.56387190993724e-09
4735 9.56355697763478e-09
4736 9.56320128808752e-09
4737 9.56283182546036e-09
4738 9.56245354133411e-09
4739 9.56208713025902e-09
4740 9.56174950171584e-09
4741 9.56140544604994e-09
4742 9.56102490649868e-09
4743 9.56068846518637e-09
4744 9.56032883785846e-09
4745 9.55994209098782e-09
4746 9.5595850218913e-09
4747 9.55925275456654e-09
4748 9.55887638182107e-09
4749 9.5585135713519e-09
4750 9.5581319701013e-09
4751 9.5578109800265e-09
4752 9.55743896868116e-09
4753 9.55708554890161e-09
4754 9.55674110039023e-09
4755 9.55640945455444e-09
4756 9.55599833125237e-09
4757 9.55566128960778e-09
4758 9.55530036592794e-09
4759 9.55492266686148e-09
4760 9.55459709705425e-09
4761 9.55421348519669e-09
4762 9.55388921114464e-09
4763 9.55350607182576e-09
4764 9.55315858654909e-09
4765 9.55282341468661e-09
4766 9.55241595695933e-09
4767 9.55202534936445e-09
4768 9.55170859351373e-09
4769 9.55139045206998e-09
4770 9.55099114885449e-09
4771 9.55064423031199e-09
4772 9.55028893137588e-09
4773 9.54993568217516e-09
4774 9.54956926305789e-09
4775 9.54924857054368e-09
4776 9.54885768184033e-09
4777 9.54848665776081e-09
4778 9.54815170512574e-09
4779 9.54779192640504e-09
4780 9.54740918424762e-09
4781 9.54706779056347e-09
4782 9.54669590349372e-09
4783 9.54636643074602e-09
4784 9.54599951796808e-09
4785 9.54563140379006e-09
4786 9.5453191477704e-09
4787 9.54495953952444e-09
4788 9.54458811627118e-09
4789 9.54423550484501e-09
4790 9.54388310485388e-09
4791 9.54350655155145e-09
4792 9.54317342768268e-09
4793 9.54281199221085e-09
4794 9.54243696168339e-09
4795 9.54207177966682e-09
4796 9.54174002098379e-09
4797 9.54138011159678e-09
4798 9.5410092697118e-09
4799 9.54066911811741e-09
4800 9.54027487710951e-09
4801 9.53996253539452e-09
4802 9.53960161106243e-09
4803 9.53925123630545e-09
4804 9.53890784505251e-09
4805 9.5385517495869e-09
4806 9.53819777704812e-09
4807 9.53781724345737e-09
4808 9.53748145076511e-09
4809 9.53709812531733e-09
4810 9.53680351747921e-09
4811 9.5364121597899e-09
4812 9.53603748284071e-09
4813 9.53571091069638e-09
4814 9.53534968844289e-09
4815 9.53502968016606e-09
4816 9.53464098922552e-09
4817 9.534276597431e-09
4818 9.53399193216048e-09
4819 9.53358936837956e-09
4820 9.53322706698234e-09
4821 9.53286890433003e-09
4822 9.53253000406107e-09
4823 9.53219782854481e-09
4824 9.5318371625755e-09
4825 9.53148828499589e-09
4826 9.53114011843087e-09
4827 9.53075538169534e-09
4828 9.53043994022379e-09
4829 9.53006788841776e-09
4830 9.52971040300843e-09
4831 9.5293453236181e-09
4832 9.52901592750355e-09
4833 9.52863823316247e-09
4834 9.52831683107003e-09
4835 9.52790350263605e-09
4836 9.52759658472829e-09
4837 9.52723170818959e-09
4838 9.52685552687432e-09
4839 9.52650147480888e-09
4840 9.52618494393792e-09
4841 9.52580369425332e-09
4842 9.52540045146694e-09
4843 9.5251504540067e-09
4844 9.52473401117509e-09
4845 9.52440974563012e-09
4846 9.52403216653019e-09
4847 9.52369561915689e-09
4848 9.52332911019482e-09
4849 9.52292257903192e-09
4850 9.52263798310871e-09
4851 9.52227644204773e-09
4852 9.52199351845345e-09
4853 9.52155437804886e-09
4854 9.52122187124205e-09
4855 9.52085202448466e-09
4856 9.52049681497702e-09
4857 9.52017237235148e-09
4858 9.51978591854502e-09
4859 9.51943974972225e-09
4860 9.51911879115003e-09
4861 9.51874394389957e-09
4862 9.51833757818071e-09
4863 9.51803961671965e-09
4864 9.51769469838964e-09
4865 9.51730052804744e-09
4866 9.51698586762473e-09
4867 9.51661413937238e-09
4868 9.51627202209343e-09
4869 9.51588751436916e-09
4870 9.51556105963786e-09
4871 9.51520624629249e-09
4872 9.5148540913137e-09
4873 9.51451686904969e-09
4874 9.51417099575441e-09
4875 9.51380338866381e-09
4876 9.51345030032052e-09
4877 9.51310197591648e-09
4878 9.51273054701496e-09
4879 9.51237341262345e-09
4880 9.51206803419952e-09
4881 9.51167042081286e-09
4882 9.51133842663071e-09
4883 9.51098868762212e-09
4884 9.5106535945369e-09
4885 9.51026705926783e-09
4886 9.50992718896926e-09
4887 9.50959058858974e-09
4888 9.50925373777861e-09
4889 9.50889655087356e-09
4890 9.50853013577391e-09
4891 9.50821122047002e-09
4892 9.50784536029453e-09
4893 9.50749437811371e-09
4894 9.50714797526286e-09
4895 9.50680495647588e-09
4896 9.50644506330506e-09
4897 9.50603383344933e-09
4898 9.50569111451277e-09
4899 9.50531398100779e-09
4900 9.5050420561224e-09
4901 9.50463882643665e-09
4902 9.50429391384511e-09
4903 9.50390020101194e-09
4904 9.50360343316553e-09
4905 9.50323192833863e-09
4906 9.5028731018007e-09
4907 9.50254301111675e-09
4908 9.50218979117734e-09
4909 9.50183921575448e-09
4910 9.50148892792796e-09
4911 9.50111252958491e-09
4912 9.50075555922886e-09
4913 9.50039302863998e-09
4914 9.5000692941008e-09
4915 9.49975226046534e-09
4916 9.49936348134533e-09
4917 9.49900214125554e-09
4918 9.4986708280842e-09
4919 9.49831006260421e-09
4920 9.49794715002228e-09
4921 9.49758644715887e-09
4922 9.49724893028331e-09
4923 9.49688349629468e-09
4924 9.4965497907662e-09
4925 9.49618475925423e-09
4926 9.49580961806529e-09
4927 9.49546758797354e-09
4928 9.49515208997082e-09
4929 9.49476642659458e-09
4930 9.49445358469636e-09
4931 9.49406915638773e-09
4932 9.49372236565271e-09
4933 9.4933965261515e-09
4934 9.49301242893913e-09
4935 9.49265779282005e-09
4936 9.4923290807486e-09
4937 9.49195971222672e-09
4938 9.49162328967718e-09
4939 9.49124547017233e-09
4940 9.49087844091118e-09
4941 9.49051887288999e-09
4942 9.49023533173415e-09
4943 9.48986343359687e-09
4944 9.4895381918883e-09
4945 9.48913809466212e-09
4946 9.4888026901524e-09
4947 9.48844377262475e-09
4948 9.48809178726029e-09
4949 9.48774207793629e-09
4950 9.48739925769187e-09
4951 9.48702214673136e-09
4952 9.48669650879114e-09
4953 9.48633655442621e-09
4954 9.48600053717746e-09
4955 9.48565812419061e-09
4956 9.48529599773679e-09
4957 9.48492552105967e-09
4958 9.4845957909831e-09
4959 9.48421204410854e-09
4960 9.48391784461045e-09
4961 9.48355185467764e-09
4962 9.48316670484201e-09
4963 9.48283839592723e-09
4964 9.48244994731368e-09
4965 9.48212755314676e-09
4966 9.48178264235239e-09
4967 9.48143568511167e-09
4968 9.48103511085041e-09
4969 9.48072761190744e-09
4970 9.48035578712814e-09
4971 9.47997993454297e-09
4972 9.47971718064794e-09
4973 9.47933350449459e-09
4974 9.4789610633958e-09
4975 9.47863459962311e-09
4976 9.47829954191931e-09
4977 9.47790967629342e-09
4978 9.47755611414164e-09
4979 9.47723432064701e-09
4980 9.47685844897989e-09
4981 9.47653297825313e-09
4982 9.47615324532053e-09
4983 9.47582513993045e-09
4984 9.47549073188059e-09
4985 9.47514413435907e-09
4986 9.47477421648496e-09
4987 9.47443414386212e-09
4988 9.47409128004839e-09
4989 9.47371382968576e-09
4990 9.47337305354928e-09
4991 9.47305148524952e-09
4992 9.47263813431964e-09
4993 9.47229143950096e-09
4994 9.47198768629087e-09
4995 9.47161903271099e-09
4996 9.4712740991848e-09
4997 9.47089355037706e-09
4998 9.47058177831056e-09
4999 9.47023015248488e-09
};
\addlegendentry{Train}
\addplot [semithick, black]
table {%
0 0.000998644391074777
1 0.000270246004220098
2 0.000208323588594794
3 0.000133029141579755
4 3.74842529708985e-05
5 1.81879986485001e-05
6 1.70022212842014e-05
7 1.67462058016099e-05
8 1.6501871868968e-05
9 1.62278338393662e-05
10 1.59051123773679e-05
11 1.54437602759572e-05
12 1.4756367818336e-05
13 1.36934404508793e-05
14 1.20442145998823e-05
15 9.73151509242598e-06
16 7.05154889146797e-06
17 4.72456667921506e-06
18 3.26413396578573e-06
19 2.5896779334289e-06
20 2.33866262533411e-06
21 2.26059000851819e-06
22 2.233736722701e-06
23 2.21663640331826e-06
24 2.200981725764e-06
25 2.18593390854949e-06
26 2.17144338421349e-06
27 2.15751106225071e-06
28 2.14410283660982e-06
29 2.13109933611122e-06
30 2.11849123843422e-06
31 2.10625444196921e-06
32 2.09418658414506e-06
33 2.08229948839289e-06
34 2.07073867386498e-06
35 2.05935020858306e-06
36 2.04803905035078e-06
37 2.03677905119548e-06
38 2.02558135242725e-06
39 2.01446914616099e-06
40 2.00347039935878e-06
41 1.99257078747905e-06
42 1.98172710952349e-06
43 1.97087183551048e-06
44 1.95994243767927e-06
45 1.94886524695903e-06
46 1.93758387467824e-06
47 1.92602101378725e-06
48 1.91410754268873e-06
49 1.90177206604858e-06
50 1.88897013231326e-06
51 1.8757075395115e-06
52 1.86194563411846e-06
53 1.84761734089989e-06
54 1.83258441666112e-06
55 1.81661971510039e-06
56 1.79943958755757e-06
57 1.78080063051311e-06
58 1.76067590018647e-06
59 1.73901469224802e-06
60 1.71560805029003e-06
61 1.69004511008097e-06
62 1.66214863384084e-06
63 1.6315472066708e-06
64 1.59691205681156e-06
65 1.55754514707951e-06
66 1.51337712850363e-06
67 1.46473121276358e-06
68 1.4116708371148e-06
69 1.35430991576868e-06
70 1.29253430714016e-06
71 1.22733911211981e-06
72 1.15978036774322e-06
73 1.08979816104693e-06
74 1.02000649349065e-06
75 9.5339288463947e-07
76 8.93179162630986e-07
77 8.38761536670063e-07
78 7.87376734479039e-07
79 7.42687348065374e-07
80 7.04575711552025e-07
81 6.72178771310428e-07
82 6.43238934117107e-07
83 6.16670035924471e-07
84 5.91558546148008e-07
85 5.6959930816447e-07
86 5.50759011730406e-07
87 5.34794139639416e-07
88 5.21344645676436e-07
89 5.09548954141792e-07
90 4.99481245697098e-07
91 4.90688876197964e-07
92 4.82441691929125e-07
93 4.75147260203812e-07
94 4.68841108158813e-07
95 4.63531449668153e-07
96 4.5881625965194e-07
97 4.54350811196491e-07
98 4.49793787993258e-07
99 4.4551441646945e-07
100 4.41872089140816e-07
101 4.38729983898156e-07
102 4.35715406865711e-07
103 4.32669622796311e-07
104 4.29577710292506e-07
105 4.26466044700646e-07
106 4.23395846382846e-07
107 4.20446838234056e-07
108 4.17674669961343e-07
109 4.15115181340298e-07
110 4.12758538459457e-07
111 4.10577996490247e-07
112 4.08535100859808e-07
113 4.06615725978554e-07
114 4.04803785158947e-07
115 4.03074920996005e-07
116 4.01423704943227e-07
117 3.99855707655661e-07
118 3.98392188571961e-07
119 3.97049092271118e-07
120 3.95820308085604e-07
121 3.94674685821883e-07
122 3.93554444144684e-07
123 3.92462396803239e-07
124 3.91398714327806e-07
125 3.90368342095826e-07
126 3.89367443176525e-07
127 3.88394823858107e-07
128 3.87444032412532e-07
129 3.86533230312125e-07
130 3.85640504418916e-07
131 3.84777592898899e-07
132 3.83937106107624e-07
133 3.83133709647154e-07
134 3.82346570404479e-07
135 3.81591974019102e-07
136 3.80853123260749e-07
137 3.80145593226189e-07
138 3.79459095256607e-07
139 3.78783965970797e-07
140 3.7813515518792e-07
141 3.77493762471204e-07
142 3.76865841644758e-07
143 3.76249886357982e-07
144 3.7563376054095e-07
145 3.75028292864954e-07
146 3.74412934434076e-07
147 3.73807722553465e-07
148 3.73192932556776e-07
149 3.72573651929997e-07
150 3.71951585975694e-07
151 3.71337421256612e-07
152 3.70795419257774e-07
153 3.70453420828198e-07
154 3.69801341548737e-07
155 3.69170521707929e-07
156 3.6850082096862e-07
157 3.67841096249322e-07
158 3.67179438853782e-07
159 3.66524943729019e-07
160 3.65862405260486e-07
161 3.65189009698952e-07
162 3.6448702189773e-07
163 3.63752491239211e-07
164 3.62992182090238e-07
165 3.62200324843798e-07
166 3.61395620984695e-07
167 3.60573864099933e-07
168 3.59725618181983e-07
169 3.58859892912733e-07
170 3.57984447418858e-07
171 3.57089959379664e-07
172 3.56197745077225e-07
173 3.55263296114572e-07
174 3.54348145492622e-07
175 3.53375810391299e-07
176 3.52481862364584e-07
177 3.51415309296499e-07
178 3.50602732623884e-07
179 3.49355588014078e-07
180 3.48725023968655e-07
181 3.47118771060195e-07
182 3.46825345332036e-07
183 3.44816186270691e-07
184 3.44879509839302e-07
185 3.42734637115427e-07
186 3.42860118962562e-07
187 3.40886401772877e-07
188 3.40797669196036e-07
189 3.39088757073114e-07
190 3.38674851718679e-07
191 3.37264822292127e-07
192 3.36564426106634e-07
193 3.35385379912623e-07
194 3.34509479671397e-07
195 3.3347993166899e-07
196 3.32551621795574e-07
197 3.31579258272541e-07
198 3.30649811530748e-07
199 3.29720137415279e-07
200 3.28802855165122e-07
201 3.278923088601e-07
202 3.26993472299364e-07
203 3.26096426306322e-07
204 3.25214841723209e-07
205 3.24329334944196e-07
206 3.23463694940074e-07
207 3.2259663385048e-07
208 3.21740316167052e-07
209 3.20886499594053e-07
210 3.20041948498329e-07
211 3.19209334520565e-07
212 3.18382376462978e-07
213 3.17567469210189e-07
214 3.16758985263732e-07
215 3.1596371741216e-07
216 3.15162765218702e-07
217 3.14360931952251e-07
218 3.13552760644598e-07
219 3.12731089024965e-07
220 3.11886452664112e-07
221 3.1103181186154e-07
222 3.10149289362016e-07
223 3.09257103481286e-07
224 3.08356902678497e-07
225 3.07483333017444e-07
226 3.06638696656591e-07
227 3.05809180645156e-07
228 3.04985803722957e-07
229 3.0416683216572e-07
230 3.03343085761298e-07
231 3.02532782825438e-07
232 3.0171932507983e-07
233 3.00913598039187e-07
234 3.00103494055293e-07
235 2.99305867201838e-07
236 2.98515288932322e-07
237 2.97717946295961e-07
238 2.96931290222346e-07
239 2.961427298942e-07
240 2.95370227831881e-07
241 2.94583315962882e-07
242 2.93815531904329e-07
243 2.93045133048508e-07
244 2.92272972046703e-07
245 2.91503454263875e-07
246 2.90740558739344e-07
247 2.89970188305233e-07
248 2.8920740646754e-07
249 2.88448319452073e-07
250 2.87694120970627e-07
251 2.86935772919605e-07
252 2.86183166053888e-07
253 2.8543212238219e-07
254 2.84681391349295e-07
255 2.83935634115551e-07
256 2.83189564243003e-07
257 2.82452475630635e-07
258 2.81719337635877e-07
259 2.80988274425908e-07
260 2.80262639762441e-07
261 2.79540984138293e-07
262 2.7883010034202e-07
263 2.78112139540099e-07
264 2.77411714932896e-07
265 2.76701769053034e-07
266 2.76015526878837e-07
267 2.75303108310254e-07
268 2.74636448693855e-07
269 2.73896745284219e-07
270 2.73256347327333e-07
271 2.72495185527077e-07
272 2.71843958898899e-07
273 2.71198246082349e-07
274 2.71974869292535e-07
275 2.70233300625478e-07
276 2.71764861281554e-07
277 2.68582510898341e-07
278 2.70530534862701e-07
279 2.67214062432686e-07
280 2.69233368044297e-07
281 2.65841890723095e-07
282 2.67884018967379e-07
283 2.64479041334198e-07
284 2.66430191686595e-07
285 2.62930853978105e-07
286 2.6467503744243e-07
287 2.61081680719144e-07
288 2.628931667914e-07
289 2.5973864126172e-07
290 2.61533529055669e-07
291 2.58599698099715e-07
292 2.60203620428001e-07
293 2.57527858593676e-07
294 2.58879680359314e-07
295 2.56484753435871e-07
296 2.57552187576948e-07
297 2.5547885229571e-07
298 2.56218527283636e-07
299 2.54492562135056e-07
300 2.5491647193121e-07
301 2.53467362654192e-07
302 2.53651847970104e-07
303 2.52410927714664e-07
304 2.52434318781525e-07
305 2.51291027097977e-07
306 2.51254760996744e-07
307 2.50153902925376e-07
308 2.50085150810264e-07
309 2.48982814810006e-07
310 2.48918297529599e-07
311 2.4778503870948e-07
312 2.47733083824642e-07
313 2.46570465378682e-07
314 2.46514360924266e-07
315 2.45316670088869e-07
316 2.45250447505896e-07
317 2.44029024543124e-07
318 2.43948079514666e-07
319 2.42676094330818e-07
320 2.4259867359433e-07
321 2.41315518678675e-07
322 2.41217435359431e-07
323 2.39944455415753e-07
324 2.39823663150673e-07
325 2.38593088397465e-07
326 2.38406784092149e-07
327 2.37241678746614e-07
328 2.36981861689856e-07
329 2.35909212165097e-07
330 2.35546167459688e-07
331 2.34562676837413e-07
332 2.34097711881986e-07
333 2.33193716780988e-07
334 2.32627115792639e-07
335 2.31773697123572e-07
336 2.31123962635138e-07
337 2.30312807047994e-07
338 2.29587982403245e-07
339 2.28808517022117e-07
340 2.28053693263064e-07
341 2.27294862042982e-07
342 2.26553581228472e-07
343 2.25816307874993e-07
344 2.25113211627104e-07
345 2.24406946358613e-07
346 2.23717151470737e-07
347 2.23014041011993e-07
348 2.22306553609997e-07
349 2.21591449189873e-07
350 2.20869608824614e-07
351 2.20149757979016e-07
352 2.19455046135408e-07
353 2.18806405882788e-07
354 2.18183075162415e-07
355 2.17594660512077e-07
356 2.17028130578001e-07
357 2.16481481629671e-07
358 2.15966338146245e-07
359 2.15460445929239e-07
360 2.14978527424137e-07
361 2.14517100971534e-07
362 2.14064911574496e-07
363 2.13619713917979e-07
364 2.13161470696832e-07
365 2.12709323932359e-07
366 2.12254192888395e-07
367 2.11797441806993e-07
368 2.11349814094319e-07
369 2.10898861041642e-07
370 2.10449343285291e-07
371 2.10006348311254e-07
372 2.09568554510042e-07
373 2.0912831644182e-07
374 2.08692142678046e-07
375 2.08262719070262e-07
376 2.07836009735729e-07
377 2.07419958542232e-07
378 2.06999544616338e-07
379 2.06582171813352e-07
380 2.06171279160117e-07
381 2.05767051397743e-07
382 2.05359285132545e-07
383 2.04956762672737e-07
384 2.04559995609088e-07
385 2.04156648919707e-07
386 2.03760535555375e-07
387 2.03364123763095e-07
388 2.02965580342607e-07
389 2.02565530571519e-07
390 2.02152648398624e-07
391 2.01747610617531e-07
392 2.01384423803574e-07
393 2.00926976390292e-07
394 2.00476975464881e-07
395 2.00010376261162e-07
396 1.99514190057926e-07
397 1.98978469256872e-07
398 1.98402887008342e-07
399 1.97874541640886e-07
400 1.97433607240782e-07
401 1.97018010794636e-07
402 1.96630367099715e-07
403 1.96247327721721e-07
404 1.95871820096727e-07
405 1.95489832321982e-07
406 1.95100426481076e-07
407 1.94723568824884e-07
408 1.94346753801256e-07
409 1.93964538652835e-07
410 1.93591262132031e-07
411 1.93204812148906e-07
412 1.92827258160833e-07
413 1.92440651858306e-07
414 1.92057200365525e-07
415 1.91674871530267e-07
416 1.91285664641327e-07
417 1.90896230378712e-07
418 1.90506185049344e-07
419 1.90116267617668e-07
420 1.89720211096756e-07
421 1.89321809784815e-07
422 1.88918761523382e-07
423 1.88514135857076e-07
424 1.8809956259247e-07
425 1.87681081342816e-07
426 1.87266735451885e-07
427 1.86852091133005e-07
428 1.86426660775396e-07
429 1.8600108830924e-07
430 1.85567301969058e-07
431 1.85129010787932e-07
432 1.84685873705348e-07
433 1.84244498768749e-07
434 1.83789722996153e-07
435 1.83329703418167e-07
436 1.82868845399753e-07
437 1.82406566295867e-07
438 1.8192842787812e-07
439 1.8145739488773e-07
440 1.80976044816816e-07
441 1.80493486823252e-07
442 1.80002473371133e-07
443 1.79505661890289e-07
444 1.79013511569792e-07
445 1.7851088784937e-07
446 1.77991665850641e-07
447 1.77483698848846e-07
448 1.76953420805148e-07
449 1.76435804633002e-07
450 1.7590620871033e-07
451 1.75376158040308e-07
452 1.74844061007207e-07
453 1.74305910149997e-07
454 1.73773855749459e-07
455 1.73232081124297e-07
456 1.72698051414955e-07
457 1.72151814581412e-07
458 1.71599708664871e-07
459 1.71059014064667e-07
460 1.70515065178733e-07
461 1.69966313023906e-07
462 1.69417631923352e-07
463 1.68869519256987e-07
464 1.6832022708968e-07
465 1.67766927461344e-07
466 1.67213713098135e-07
467 1.66662033507237e-07
468 1.66101472132141e-07
469 1.65550602559961e-07
470 1.64998368745728e-07
471 1.64445040695682e-07
472 1.63886909376743e-07
473 1.63338256697898e-07
474 1.62780452228617e-07
475 1.62223557254038e-07
476 1.61671181331258e-07
477 1.6110971046146e-07
478 1.60552062311581e-07
479 1.59993859938368e-07
480 1.59437178126609e-07
481 1.58880183676047e-07
482 1.58325818233607e-07
483 1.57767161113043e-07
484 1.5720968349342e-07
485 1.56653229055337e-07
486 1.56089996039555e-07
487 1.55531708401213e-07
488 1.54973363919453e-07
489 1.54414749431453e-07
490 1.53851189566012e-07
491 1.53275280467824e-07
492 1.52686240539879e-07
493 1.5209315051834e-07
494 1.51532290715295e-07
495 1.50994665659709e-07
496 1.50463023373959e-07
497 1.49930343695814e-07
498 1.4941089432341e-07
499 1.48887522755103e-07
500 1.48378632047752e-07
501 1.47866870747748e-07
502 1.47359116908774e-07
503 1.46856734772882e-07
504 1.4635860168255e-07
505 1.45861122291535e-07
506 1.45368332482576e-07
507 1.44878825381056e-07
508 1.44395826850996e-07
509 1.43914903105724e-07
510 1.43440473721057e-07
511 1.42962960580917e-07
512 1.42492353916168e-07
513 1.42019601412358e-07
514 1.41553712751374e-07
515 1.41087468819023e-07
516 1.40626681854883e-07
517 1.40162924822107e-07
518 1.39697505119329e-07
519 1.39244519914428e-07
520 1.38773089020106e-07
521 1.38314661057848e-07
522 1.37853803039434e-07
523 1.37386649612381e-07
524 1.36918828275157e-07
525 1.3645684759922e-07
526 1.35983015070451e-07
527 1.35512564725104e-07
528 1.35043563886939e-07
529 1.34575088850397e-07
530 1.34093070869312e-07
531 1.33619678877039e-07
532 1.33138755131768e-07
533 1.32658584561796e-07
534 1.32167897959334e-07
535 1.31679229298243e-07
536 1.31185771579112e-07
537 1.30683488919203e-07
538 1.30182968405279e-07
539 1.29672798721003e-07
540 1.29159133166468e-07
541 1.28637793750386e-07
542 1.28110485775323e-07
543 1.2758049194872e-07
544 1.27043378483904e-07
545 1.26498164831901e-07
546 1.25947295259721e-07
547 1.25392332961383e-07
548 1.24825959346708e-07
549 1.24251911870488e-07
550 1.23677295960078e-07
551 1.23088682357775e-07
552 1.22500495081113e-07
553 1.21900882277259e-07
554 1.21299947863918e-07
555 1.20683552040646e-07
556 1.20066104614125e-07
557 1.19441210699733e-07
558 1.18813382243843e-07
559 1.1817620304555e-07
560 1.1753638062828e-07
561 1.16888479340105e-07
562 1.16228861202217e-07
563 1.15560609970089e-07
564 1.14879441071025e-07
565 1.14187407973532e-07
566 1.13465553397418e-07
567 1.12703197885367e-07
568 1.1193458959724e-07
569 1.11215435083523e-07
570 1.1053181481202e-07
571 1.09867528408358e-07
572 1.09201145903626e-07
573 1.08533640741371e-07
574 1.07871052534847e-07
575 1.07206673760629e-07
576 1.06543474487353e-07
577 1.05880680223436e-07
578 1.052148164149e-07
579 1.04556107771714e-07
580 1.03898848635708e-07
581 1.03241632132267e-07
582 1.02587492278872e-07
583 1.01941189711852e-07
584 1.01298503807357e-07
585 1.00658091639616e-07
586 1.00028252347784e-07
587 9.93969635487701e-08
588 9.87761481496818e-08
589 9.81633121455161e-08
590 9.75587113316578e-08
591 9.6964669182853e-08
592 9.63810933285458e-08
593 9.58000683226601e-08
594 9.52325720504632e-08
595 9.46789953104599e-08
596 9.41315647651209e-08
597 9.36016348873636e-08
598 9.30797341425205e-08
599 9.25639369597775e-08
600 9.20659957159842e-08
601 9.1575429905788e-08
602 9.11013842141983e-08
603 9.06318931015448e-08
604 9.01757388760416e-08
605 8.97310101777293e-08
606 8.92952627395971e-08
607 8.88639135609992e-08
608 8.84456170524572e-08
609 8.80519479551367e-08
610 8.7606117915584e-08
611 8.72181118438675e-08
612 8.68645528839807e-08
613 8.64589182469899e-08
614 8.6106311414369e-08
615 8.58478585996636e-08
616 8.54858086540844e-08
617 8.51165395943099e-08
618 8.47539425308241e-08
619 8.44125906951376e-08
620 8.40996037254627e-08
621 8.38320914908763e-08
622 8.35318303415988e-08
623 8.31974205084407e-08
624 8.2862840145026e-08
625 8.25246928570778e-08
626 8.21801720007898e-08
627 8.18310184058646e-08
628 8.14921037317617e-08
629 8.11656164501073e-08
630 8.08519686756881e-08
631 8.05593884933842e-08
632 8.02805573130172e-08
633 8.00137982537308e-08
634 7.97585641976184e-08
635 7.95245185258864e-08
636 7.93043923863479e-08
637 7.91020937640496e-08
638 7.89214098517732e-08
639 7.87623193332365e-08
640 7.86203386837769e-08
641 7.85065523700723e-08
642 7.83948621574382e-08
643 7.8295173011611e-08
644 7.81931390747559e-08
645 7.80852005277666e-08
646 7.79648630100382e-08
647 7.78282682745157e-08
648 7.76687016923461e-08
649 7.75010349229888e-08
650 7.7325807978923e-08
651 7.71399939480943e-08
652 7.69516006471349e-08
653 7.67722667660564e-08
654 7.66017862474655e-08
655 7.64199441505298e-08
656 7.62749934324347e-08
657 7.60974927516145e-08
658 7.59821574547459e-08
659 7.58028164682401e-08
660 7.57128404416108e-08
661 7.55306075461704e-08
662 7.5463162829692e-08
663 7.52924904645624e-08
664 7.52026636519076e-08
665 7.50222923784349e-08
666 7.49179704939706e-08
667 7.47314956583978e-08
668 7.4610930766994e-08
669 7.44332879776266e-08
670 7.43052055440785e-08
671 7.41256229730425e-08
672 7.39874579380739e-08
673 7.38178940196121e-08
674 7.3658874555349e-08
675 7.34890406306477e-08
676 7.33203791014603e-08
677 7.31174480961272e-08
678 7.30619191813275e-08
679 7.28991054188555e-08
680 7.27391764598906e-08
681 7.25865092476852e-08
682 7.24208675251248e-08
683 7.22646618100953e-08
684 7.20990058766802e-08
685 7.19292643225344e-08
686 7.17637647085212e-08
687 7.15892909397553e-08
688 7.14133037149622e-08
689 7.12362862032023e-08
690 7.10623737631977e-08
691 7.08833169937861e-08
692 7.07066689642488e-08
693 7.05243863308169e-08
694 7.03381530797742e-08
695 7.01570783689931e-08
696 6.99723585739775e-08
697 6.9785691891866e-08
698 6.95494861702173e-08
699 6.94351527386061e-08
700 6.92248818268126e-08
701 6.90184904783564e-08
702 6.87840540081197e-08
703 6.86723709009129e-08
704 6.84569343434305e-08
705 6.81825866877261e-08
706 6.80683385212433e-08
707 6.78571723256027e-08
708 6.75812756867344e-08
709 6.7454557495239e-08
710 6.72078783736652e-08
711 6.70277273684405e-08
712 6.67491306671764e-08
713 6.66024320139513e-08
714 6.63455921312561e-08
715 6.61672032720162e-08
716 6.59170211747551e-08
717 6.57347811738873e-08
718 6.5482289812735e-08
719 6.53031335673404e-08
720 6.50576694738447e-08
721 6.48814193482394e-08
722 6.46335465148695e-08
723 6.44631512614069e-08
724 6.42175095322273e-08
725 6.40528412532149e-08
726 6.38061194990769e-08
727 6.36484998040032e-08
728 6.34034336144396e-08
729 6.32519032706114e-08
730 6.30090823960927e-08
731 6.28676630753944e-08
732 6.26250127311323e-08
733 6.24958502726258e-08
734 6.22601703526016e-08
735 6.21347240326031e-08
736 6.19007636259994e-08
737 6.17802342617324e-08
738 6.15453430441448e-08
739 6.14245223573562e-08
740 6.11757968727034e-08
741 6.10755819252518e-08
742 6.07987686862543e-08
743 6.07280696840462e-08
744 6.0414620861593e-08
745 6.03473111482344e-08
746 5.99789444777343e-08
747 5.99067888629179e-08
748 5.95223959010127e-08
749 5.94629767647348e-08
750 5.90811382039647e-08
751 5.90395075050765e-08
752 5.86572106442418e-08
753 5.8627023236113e-08
754 5.82538604021465e-08
755 5.8235936961637e-08
756 5.78708849729992e-08
757 5.7856784252408e-08
758 5.75044722950224e-08
759 5.74993919144617e-08
760 5.71561251661024e-08
761 5.71491085565867e-08
762 5.68233566866638e-08
763 5.68100126940863e-08
764 5.65012108211249e-08
765 5.64816247106137e-08
766 5.61861668302299e-08
767 5.616539411335e-08
768 5.58734711830766e-08
769 5.58423565166777e-08
770 5.55637527099861e-08
771 5.55210668551354e-08
772 5.52475292181498e-08
773 5.52039516321656e-08
774 5.49291812035335e-08
775 5.49052998621846e-08
776 5.46324763206485e-08
777 5.4605969523891e-08
778 5.43348761539164e-08
779 5.43176135181511e-08
780 5.40619602418246e-08
781 5.40384945679762e-08
782 5.37897939523191e-08
783 5.37707336434323e-08
784 5.35312842941948e-08
785 5.3520995635381e-08
786 5.32829069754825e-08
787 5.32757411519924e-08
788 5.30461967684914e-08
789 5.30457775482773e-08
790 5.28244825659385e-08
791 5.28245429620711e-08
792 5.26082928331562e-08
793 5.26112344800822e-08
794 5.24034540205776e-08
795 5.24088150655189e-08
796 5.22089607102316e-08
797 5.22141831993395e-08
798 5.20198071285449e-08
799 5.20298861772517e-08
800 5.18336591426305e-08
801 5.184883988818e-08
802 5.16522433713362e-08
803 5.16436315933788e-08
804 5.1341626061685e-08
805 5.11398425828702e-08
806 5.09159718831143e-08
807 5.09489623823356e-08
808 5.07616526590482e-08
809 5.08183397585071e-08
810 5.06282198386998e-08
811 5.06835959868113e-08
812 5.04933908018756e-08
813 5.05435515663066e-08
814 5.0360768000246e-08
815 5.0390909223097e-08
816 5.02241874755782e-08
817 5.02723835893448e-08
818 5.00975225747879e-08
819 5.01462658064611e-08
820 4.99803611830885e-08
821 5.00355952226528e-08
822 4.98905663448568e-08
823 5.00061538843966e-08
824 4.97889445227884e-08
825 4.99057755121157e-08
826 4.96922929471566e-08
827 4.98030914286574e-08
828 4.95991514526395e-08
829 4.97107883745684e-08
830 4.94986878152304e-08
831 4.96175047715042e-08
832 4.94041678678059e-08
833 4.95259726562836e-08
834 4.93166609771833e-08
835 4.94371690251683e-08
836 4.92347034253271e-08
837 4.93586753691488e-08
838 4.91462834872891e-08
839 4.92683227548696e-08
840 4.9062521156884e-08
841 4.91884399878018e-08
842 4.89810929593659e-08
843 4.91051466156023e-08
844 4.89079887699972e-08
845 4.90262692665056e-08
846 4.88306248769277e-08
847 4.89439742068498e-08
848 4.87585012365344e-08
849 4.88645142127098e-08
850 4.86887614670195e-08
851 4.87871396614992e-08
852 4.86138311828199e-08
853 4.87087881140269e-08
854 4.8552763587395e-08
855 4.86594124993189e-08
856 4.84803521771937e-08
857 4.85708930852979e-08
858 4.83901274606069e-08
859 4.84737618933195e-08
860 4.83825814967531e-08
861 4.83558295627518e-08
862 4.82726854045268e-08
863 4.83115769611686e-08
864 4.82673172541581e-08
865 4.81823434483886e-08
866 4.81997624035557e-08
867 4.8185501810849e-08
868 4.80858162177356e-08
869 4.81069406532697e-08
870 4.80258499635511e-08
871 4.80471022967777e-08
872 4.79633293082316e-08
873 4.79907136252677e-08
874 4.79451891521876e-08
875 4.78755488586557e-08
876 4.7882060982829e-08
877 4.7813152548315e-08
878 4.78590358454767e-08
879 4.78255870461908e-08
880 4.77445745161731e-08
881 4.77133603737911e-08
882 4.77109800556264e-08
883 4.76597143972413e-08
884 4.76345292099722e-08
885 4.76459085518854e-08
886 4.7639542088973e-08
887 4.75391601639785e-08
888 4.75345487416234e-08
889 4.74952983609001e-08
890 4.75149235512617e-08
891 4.74250398951881e-08
892 4.74375347891964e-08
893 4.74002916917016e-08
894 4.73699230951752e-08
895 4.73519676802425e-08
896 4.73240753251503e-08
897 4.73008903156824e-08
898 4.72749555058272e-08
899 4.72544243734774e-08
900 4.72290970776612e-08
901 4.72006256302393e-08
902 4.71774264099167e-08
903 4.71549519431846e-08
904 4.71306904614721e-08
905 4.7106645695294e-08
906 4.70852050682424e-08
907 4.7056087026931e-08
908 4.70360035365047e-08
909 4.70151348963554e-08
910 4.6990695778959e-08
911 4.69627394750205e-08
912 4.69474237263512e-08
913 4.69173428996328e-08
914 4.68992595870077e-08
915 4.68748027060428e-08
916 4.68463987601808e-08
917 4.6819607746329e-08
918 4.6842437484429e-08
919 4.68410839005173e-08
920 4.68254519603306e-08
921 4.68017162802425e-08
922 4.67902587786284e-08
923 4.67747334198521e-08
924 4.67652583324707e-08
925 4.67537049075872e-08
926 4.6736840175754e-08
927 4.67285339311729e-08
928 4.67011176397136e-08
929 4.66909462204512e-08
930 4.66553906619538e-08
931 4.66531666631909e-08
932 4.66119622899441e-08
933 4.66062139992118e-08
934 4.65731453402896e-08
935 4.65699976359701e-08
936 4.6534538000742e-08
937 4.65324170306758e-08
938 4.64914506892455e-08
939 4.64899443386457e-08
940 4.6449912360913e-08
941 4.64491947127499e-08
942 4.64124774168795e-08
943 4.64109710662797e-08
944 4.63709781683974e-08
945 4.63689744378826e-08
946 4.63389788762925e-08
947 4.63328220234871e-08
948 4.62957210345394e-08
949 4.62925520139379e-08
950 4.62543425783224e-08
951 4.62493723318858e-08
952 4.62157245806338e-08
953 4.62116567234716e-08
954 4.61791955785884e-08
955 4.61694753539632e-08
956 4.61396645334844e-08
957 4.61350815328387e-08
958 4.61034161958196e-08
959 4.60912623623244e-08
960 4.60612490371659e-08
961 4.60561402348958e-08
962 4.60276226021961e-08
963 4.60141578173534e-08
964 4.59854021528372e-08
965 4.59743141334457e-08
966 4.5946990212542e-08
967 4.59300615318625e-08
968 4.5913232327166e-08
969 4.58950069059938e-08
970 4.58755948784528e-08
971 4.58503208733418e-08
972 4.58332003461237e-08
973 4.58064732811181e-08
974 4.57969484557452e-08
975 4.5774719126257e-08
976 4.57610056514568e-08
977 4.57335040948692e-08
978 4.57305731060842e-08
979 4.56946906979283e-08
980 4.5687293948049e-08
981 4.56509674506833e-08
982 4.565038480564e-08
983 4.56272211124542e-08
984 4.56108999458138e-08
985 4.55703634827387e-08
986 4.56011939320433e-08
987 4.55645619013012e-08
988 4.55383180053559e-08
989 4.54961366358475e-08
990 4.5536889814457e-08
991 4.54850876963064e-08
992 4.54601583044223e-08
993 4.54183783915596e-08
994 4.54795490156812e-08
995 4.53620572216096e-08
996 4.54213413547677e-08
997 4.53293367286278e-08
998 4.53187070093009e-08
999 4.53535946576267e-08
1000 4.52712143328426e-08
1001 4.52798829542189e-08
1002 4.52506903059202e-08
1003 4.52992807709052e-08
1004 4.52028849906583e-08
1005 4.52267343575841e-08
1006 4.51905037834877e-08
1007 4.51756818620197e-08
1008 4.51304806858843e-08
1009 4.51539321488781e-08
1010 4.50686457043048e-08
1011 4.50886048497523e-08
1012 4.5072017229586e-08
1013 4.5057781505875e-08
1014 4.50123529560642e-08
1015 4.50107364713404e-08
1016 4.49937544999557e-08
1017 4.49782824318845e-08
1018 4.49621744280648e-08
1019 4.49132606661351e-08
1020 4.49212045339209e-08
1021 4.48654766671552e-08
1022 4.48963319854556e-08
1023 4.48372325934088e-08
1024 4.48283117293613e-08
1025 4.4806945709297e-08
1026 4.47652865886994e-08
1027 4.47846346673941e-08
1028 4.47537509273843e-08
1029 4.47502408462697e-08
1030 4.47104504530671e-08
1031 4.4658023057309e-08
1032 4.46850378921226e-08
1033 4.46602541614993e-08
1034 4.46577956836336e-08
1035 4.46091803496529e-08
1036 4.45687469152745e-08
1037 4.4593019055128e-08
1038 4.45612009514207e-08
1039 4.45588455022516e-08
1040 4.45224017653345e-08
1041 4.44629648654882e-08
1042 4.44873506921795e-08
1043 4.44570744662087e-08
1044 4.44688978973318e-08
1045 4.44062848714566e-08
1046 4.43932073324049e-08
1047 4.43880203704339e-08
1048 4.43583587639296e-08
1049 4.43860628251969e-08
1050 4.43139391848035e-08
1051 4.43838956698528e-08
1052 4.42612204665238e-08
1053 4.43049117393457e-08
1054 4.42216254725736e-08
1055 4.42521894683523e-08
1056 4.41978542653487e-08
1057 4.42122214394658e-08
1058 4.41992398236835e-08
1059 4.411408838223e-08
1060 4.41461587286085e-08
1061 4.41121059679972e-08
1062 4.41646399451656e-08
1063 4.40584919658704e-08
1064 4.40877805374384e-08
1065 4.40607870189069e-08
1066 4.39871143953496e-08
1067 4.4015326494673e-08
1068 4.39724523459972e-08
1069 4.40467324835936e-08
1070 4.39372342952993e-08
1071 4.39944756180921e-08
1072 4.38955076731418e-08
1073 4.39097789239895e-08
1074 4.38609930597522e-08
1075 4.39102478821951e-08
1076 4.38183320738972e-08
1077 4.38930918278402e-08
1078 4.37687468490822e-08
1079 4.37956764187675e-08
1080 4.37468656855344e-08
1081 4.38146905423764e-08
1082 4.37142873010998e-08
1083 4.37278089293613e-08
1084 4.3677530925379e-08
1085 4.37677840636752e-08
1086 4.36505480649885e-08
1087 4.37080451831662e-08
1088 4.36089635513781e-08
1089 4.36461675690225e-08
1090 4.36094538258658e-08
1091 4.35553459965377e-08
1092 4.35595701731017e-08
1093 4.35237907936425e-08
1094 4.35444214019753e-08
1095 4.35291340750155e-08
1096 4.3527435877877e-08
1097 4.34543103722262e-08
1098 4.34480043054464e-08
1099 4.34154436845802e-08
1100 4.34426361550777e-08
1101 4.34186233633227e-08
1102 4.34375948543675e-08
1103 4.33562128421272e-08
1104 4.33545928046897e-08
1105 4.33226610141446e-08
1106 4.33374545139031e-08
1107 4.33052065318407e-08
1108 4.33385700659983e-08
1109 4.32648619153042e-08
1110 4.32467608391107e-08
1111 4.32267732719538e-08
1112 4.32330224953148e-08
1113 4.31911537646101e-08
1114 4.32237499126131e-08
1115 4.31568594194687e-08
1116 4.31432205516558e-08
1117 4.3130508942113e-08
1118 4.31300861691852e-08
1119 4.30933404516054e-08
1120 4.31050537486044e-08
1121 4.30745501489582e-08
1122 4.30491517988685e-08
1123 4.30550173291522e-08
1124 4.3033306695861e-08
1125 4.29945785640484e-08
1126 4.30164419640278e-08
1127 4.29833093562593e-08
1128 4.29476543217788e-08
1129 4.29771382925992e-08
1130 4.2941746158931e-08
1131 4.29017532610487e-08
1132 4.29345838881545e-08
1133 4.28959374687565e-08
1134 4.28554649545276e-08
1135 4.28942996677506e-08
1136 4.28418474029968e-08
1137 4.28088569037754e-08
1138 4.28515996020451e-08
1139 4.28100968008494e-08
1140 4.27547242054516e-08
1141 4.28117417072826e-08
1142 4.27651265511031e-08
1143 4.27295141491868e-08
1144 4.27654214263384e-08
1145 4.27289279514298e-08
1146 4.26892583504923e-08
1147 4.27192468066551e-08
1148 4.26873008052553e-08
1149 4.26342268156077e-08
1150 4.26752073678927e-08
1151 4.26525907926134e-08
1152 4.25890753774638e-08
1153 4.26204742609571e-08
1154 4.26085264848552e-08
1155 4.25437640672044e-08
1156 4.25763069245022e-08
1157 4.25608064347216e-08
1158 4.24971844381616e-08
1159 4.25170441076261e-08
1160 4.25215951338487e-08
1161 4.25103827694784e-08
1162 4.24560511191885e-08
1163 4.24267057042016e-08
1164 4.24579731372887e-08
1165 4.24416164435115e-08
1166 4.23988915088103e-08
1167 4.23906598712165e-08
1168 4.24180193192569e-08
1169 4.23623021106323e-08
1170 4.23219574940958e-08
1171 4.23595452048176e-08
1172 4.23382431335995e-08
1173 4.23122763493211e-08
1174 4.22832613367063e-08
1175 4.23125463555607e-08
1176 4.22660093590821e-08
1177 4.22124060150963e-08
1178 4.225457317375e-08
1179 4.22370511898862e-08
1180 4.22414494494205e-08
1181 4.21990620225188e-08
1182 4.2132651145721e-08
1183 4.21843502351749e-08
1184 4.21703916231309e-08
1185 4.21766053193551e-08
1186 4.21401438188695e-08
1187 4.21438208775271e-08
1188 4.21183230514544e-08
1189 4.21177368536974e-08
1190 4.2105131825565e-08
1191 4.20908534692899e-08
1192 4.20653876176402e-08
1193 4.20657357835807e-08
1194 4.20494359332224e-08
1195 4.2031754077243e-08
1196 4.20290575675608e-08
1197 4.20090451314081e-08
1198 4.19916936778009e-08
1199 4.19848653621102e-08
1200 4.19663024331385e-08
1201 4.19549230912253e-08
1202 4.19341787960548e-08
1203 4.19289705178016e-08
1204 4.19165644416353e-08
1205 4.19020800279668e-08
1206 4.18848742356204e-08
1207 4.18695087489596e-08
1208 4.18617247532893e-08
1209 4.18467571705605e-08
1210 4.18309369365488e-08
1211 4.18211882902142e-08
1212 4.18085939202228e-08
1213 4.17946957043114e-08
1214 4.17808827535282e-08
1215 4.17704306698852e-08
1216 4.17557650678191e-08
1217 4.17373371419671e-08
1218 4.17255030527031e-08
1219 4.17129513152759e-08
1220 4.1696392116819e-08
1221 4.16757828247682e-08
1222 4.16693204385865e-08
1223 4.16562642158169e-08
1224 4.16361984889591e-08
1225 4.16265670821758e-08
1226 4.16146512804971e-08
1227 4.15927488006673e-08
1228 4.15843928180948e-08
1229 4.15671266296158e-08
1230 4.1555455965181e-08
1231 4.15457748204062e-08
1232 4.15347152227241e-08
1233 4.15205931858509e-08
1234 4.15035188439106e-08
1235 4.14954151040092e-08
1236 4.14797689529678e-08
1237 4.14696437189832e-08
1238 4.14583354313436e-08
1239 4.14496241774032e-08
1240 4.14364507150822e-08
1241 4.14189074149363e-08
1242 4.14015417504743e-08
1243 4.13964187373494e-08
1244 4.13786374053871e-08
1245 4.13675742549913e-08
1246 4.13591578762862e-08
1247 4.13430569778939e-08
1248 4.13311624924972e-08
1249 4.13147915878653e-08
1250 4.12977705366302e-08
1251 4.12907965596787e-08
1252 4.12803871085998e-08
1253 4.12688549999984e-08
1254 4.1253020555132e-08
1255 4.124432706476e-08
1256 4.12337044508604e-08
1257 4.12151663908844e-08
1258 4.12020817464054e-08
1259 4.11884748530156e-08
1260 4.11749283557583e-08
1261 4.11670200151093e-08
1262 4.11543403799897e-08
1263 4.11425524760034e-08
1264 4.11303098246663e-08
1265 4.11217122575636e-08
1266 4.1103600523229e-08
1267 4.10985023791e-08
1268 4.10764826597187e-08
1269 4.10696259223187e-08
1270 4.10577065679263e-08
1271 4.10462384081711e-08
1272 4.10351859159164e-08
1273 4.10240730275291e-08
1274 4.10096099301427e-08
1275 4.09968485826084e-08
1276 4.09889828745236e-08
1277 4.09811704571439e-08
1278 4.09721110372629e-08
1279 4.09513809529471e-08
1280 4.09404812273806e-08
1281 4.09289278024971e-08
1282 4.09173104287675e-08
1283 4.0908798126793e-08
1284 4.08992733014202e-08
1285 4.08911056126726e-08
1286 4.08795806094986e-08
1287 4.08673699325846e-08
1288 4.08583993305456e-08
1289 4.08467215606834e-08
1290 4.08394420503555e-08
1291 4.08274409835485e-08
1292 4.08126261675079e-08
1293 4.08050553346584e-08
1294 4.07996765261487e-08
1295 4.07880058617138e-08
1296 4.07776852284769e-08
1297 4.07700646576359e-08
1298 4.07547275926845e-08
1299 4.07494411547304e-08
1300 4.07341076424927e-08
1301 4.07269666879984e-08
1302 4.07167881633086e-08
1303 4.07041262917573e-08
1304 4.06966691457455e-08
1305 4.06887039616777e-08
1306 4.06819786746837e-08
1307 4.06698177357612e-08
1308 4.06554043763663e-08
1309 4.06471443170631e-08
1310 4.06401809982526e-08
1311 4.06254336837719e-08
1312 4.06216216219946e-08
1313 4.06072402370228e-08
1314 4.05991684715445e-08
1315 4.05947027104503e-08
1316 4.05835507422125e-08
1317 4.05716704676706e-08
1318 4.05644193790522e-08
1319 4.05557436522486e-08
1320 4.05454514407211e-08
1321 4.05460980346106e-08
1322 4.05316846752157e-08
1323 4.05290521143797e-08
1324 4.05203799402898e-08
1325 4.05141449277835e-08
1326 4.05132354330817e-08
1327 4.05021900462543e-08
1328 4.0497504016912e-08
1329 4.04902671391483e-08
1330 4.04733597747509e-08
1331 4.04779569862512e-08
1332 4.04624458383296e-08
1333 4.0463337569463e-08
1334 4.0455141459006e-08
1335 4.04471869330791e-08
1336 4.04416624633086e-08
1337 4.0436415105205e-08
1338 4.04336120141124e-08
1339 4.04253093222451e-08
1340 4.0423032032777e-08
1341 4.04136031306734e-08
1342 4.04165803047363e-08
1343 4.04053999147891e-08
1344 4.04004794063439e-08
1345 4.0390823130565e-08
1346 4.03914910407366e-08
1347 4.03772375534572e-08
1348 4.03620603606214e-08
1349 4.0320276895045e-08
1350 4.02434174873179e-08
1351 4.02001987254152e-08
1352 4.01965039031893e-08
1353 4.01852844333916e-08
1354 4.01920310366677e-08
1355 4.01813586847766e-08
1356 4.01844886255276e-08
1357 4.017045895921e-08
1358 4.01717237252797e-08
1359 4.01593283072543e-08
1360 4.01619217882399e-08
1361 4.01426945018102e-08
1362 4.01476718536742e-08
1363 4.01255881854468e-08
1364 4.01349815604135e-08
1365 4.01137647543237e-08
1366 4.01178255060586e-08
1367 4.00990991522576e-08
1368 4.01005344485839e-08
1369 4.00806996481151e-08
1370 4.00834139213657e-08
1371 4.00660553623311e-08
1372 4.00716153592384e-08
1373 4.00477802031673e-08
1374 4.00538979761222e-08
1375 4.00377970777299e-08
1376 4.00357222929415e-08
1377 4.00216428886324e-08
1378 4.00186657145696e-08
1379 4.00070589989809e-08
1380 4.00025434998952e-08
1381 3.99969550812784e-08
1382 3.99846875609455e-08
1383 3.99727113631343e-08
1384 3.99651014504343e-08
1385 3.99597759326298e-08
1386 3.99385378102579e-08
1387 3.99477215751176e-08
1388 3.99236803616532e-08
1389 3.99238793136192e-08
1390 3.99077890733679e-08
1391 3.99127770833729e-08
1392 3.98854425043282e-08
1393 3.98903701182007e-08
1394 3.98656112565732e-08
1395 3.98758679409639e-08
1396 3.98473609664052e-08
1397 3.98600690232342e-08
1398 3.98256858602508e-08
1399 3.98383619426568e-08
1400 3.98036910098654e-08
1401 3.98203106044548e-08
1402 3.97835577814476e-08
1403 3.97933206386369e-08
1404 3.97632220483501e-08
1405 3.97763137982565e-08
1406 3.97440231836299e-08
1407 3.97569230869976e-08
1408 3.97207529090338e-08
1409 3.97402004637115e-08
1410 3.96918444778294e-08
1411 3.97216410874535e-08
1412 3.96715087447319e-08
1413 3.97025914367077e-08
1414 3.9649773242445e-08
1415 3.96755872600352e-08
1416 3.96314092654393e-08
1417 3.96533934576837e-08
1418 3.96001631486342e-08
1419 3.96296577775956e-08
1420 3.95760046956184e-08
1421 3.96131980551218e-08
1422 3.95521517759789e-08
1423 3.95853341217389e-08
1424 3.95270234321288e-08
1425 3.95596160274181e-08
1426 3.95002111019949e-08
1427 3.9538225138358e-08
1428 3.94771539902194e-08
1429 3.95145356435478e-08
1430 3.94525336844254e-08
1431 3.94852293084114e-08
1432 3.94270216474979e-08
1433 3.94680235160649e-08
1434 3.9397100692895e-08
1435 3.94352923649421e-08
1436 3.93690022804094e-08
1437 3.94106223211566e-08
1438 3.93449468560902e-08
1439 3.93823818001238e-08
1440 3.93171610824083e-08
1441 3.9352293867978e-08
1442 3.92872330223781e-08
1443 3.932506231763e-08
1444 3.92545977945247e-08
1445 3.929462266683e-08
1446 3.92264176696244e-08
1447 3.92626553491482e-08
1448 3.91990866432934e-08
1449 3.92294161599693e-08
1450 3.91632930529795e-08
1451 3.91949832589944e-08
1452 3.91298335955526e-08
1453 3.91703025570678e-08
1454 3.91022076939862e-08
1455 3.91326260285041e-08
1456 3.90651742065984e-08
1457 3.90961858443006e-08
1458 3.90309864428673e-08
1459 3.90614438572356e-08
1460 3.89959033952891e-08
1461 3.90259096150203e-08
1462 3.89642025311332e-08
1463 3.89887091500896e-08
1464 3.89293610680852e-08
1465 3.89550329771282e-08
1466 3.88959087160856e-08
1467 3.891377886589e-08
1468 3.88542140683512e-08
1469 3.88811223217544e-08
1470 3.88232734849225e-08
1471 3.88439822529563e-08
1472 3.87866343487531e-08
1473 3.88015131136399e-08
1474 3.87498069187586e-08
1475 3.87615521901807e-08
1476 3.87225966846927e-08
1477 3.87310983285261e-08
1478 3.86842025079659e-08
1479 3.86926046758163e-08
1480 3.8647545608228e-08
1481 3.86528107299e-08
1482 3.86125016405003e-08
1483 3.86114500372514e-08
1484 3.85749672204838e-08
1485 3.857228847437e-08
1486 3.85468439390024e-08
1487 3.85348464249091e-08
1488 3.85039840011814e-08
1489 3.84959122357031e-08
1490 3.84583316304088e-08
1491 3.84585412405158e-08
1492 3.84264495778552e-08
1493 3.84202145653489e-08
1494 3.83871707754224e-08
1495 3.83811418203095e-08
1496 3.83530931458154e-08
1497 3.83418345961672e-08
1498 3.83187845898192e-08
1499 3.83023639471958e-08
1500 3.82707803225912e-08
1501 3.82658953412829e-08
1502 3.82389728770249e-08
1503 3.82218914296573e-08
1504 3.81973563889915e-08
1505 3.81852522934878e-08
1506 3.81575411267931e-08
1507 3.81493059364857e-08
1508 3.81150826456178e-08
1509 3.81071636468278e-08
1510 3.807804915823e-08
1511 3.80675260203134e-08
1512 3.80445221992431e-08
1513 3.80264104649086e-08
1514 3.80028453150771e-08
1515 3.7992062829062e-08
1516 3.79652433935007e-08
1517 3.7951657816393e-08
1518 3.79239679659804e-08
1519 3.79038027631395e-08
1520 3.7884309023184e-08
1521 3.78677675882955e-08
1522 3.78460889294274e-08
1523 3.78273412593444e-08
1524 3.78073394813327e-08
1525 3.77883004887281e-08
1526 3.77654814087691e-08
1527 3.77497073600352e-08
1528 3.77308673193966e-08
1529 3.7707479805249e-08
1530 3.76872577589893e-08
1531 3.76676823066191e-08
1532 3.763975442439e-08
1533 3.76252700107216e-08
1534 3.76143844960097e-08
1535 3.75912705408155e-08
1536 3.75708282263076e-08
1537 3.75497926086155e-08
1538 3.75245043926498e-08
1539 3.75082933601334e-08
1540 3.74876343300912e-08
1541 3.74664175240014e-08
1542 3.74487214571673e-08
1543 3.7421692411499e-08
1544 3.74021347226972e-08
1545 3.73877604431527e-08
1546 3.73629127636832e-08
1547 3.73482542670445e-08
1548 3.73257407204619e-08
1549 3.7302037014797e-08
1550 3.7283211185013e-08
1551 3.7263500729523e-08
1552 3.72439572515759e-08
1553 3.7218793380589e-08
1554 3.72041171203819e-08
1555 3.71823354328171e-08
1556 3.71641633023501e-08
1557 3.71425485923282e-08
1558 3.71151820388604e-08
1559 3.70952477624087e-08
1560 3.70744359656783e-08
1561 3.70532262650158e-08
1562 3.70411079586574e-08
1563 3.70098085511472e-08
1564 3.69897819041398e-08
1565 3.6970323691321e-08
1566 3.6953224480385e-08
1567 3.69290127366639e-08
1568 3.69111106124365e-08
1569 3.6890199339723e-08
1570 3.68646482229451e-08
1571 3.68467389932903e-08
1572 3.68242289994214e-08
1573 3.68013743923257e-08
1574 3.6779692180744e-08
1575 3.67615129448495e-08
1576 3.67323202965508e-08
1577 3.67138319745663e-08
1578 3.66941605989268e-08
1579 3.6674492776001e-08
1580 3.6647648471444e-08
1581 3.66309755861494e-08
1582 3.6609687725786e-08
1583 3.65843959571066e-08
1584 3.65589798434485e-08
1585 3.65431525040094e-08
1586 3.65213459474489e-08
1587 3.64977132960576e-08
1588 3.64775800676398e-08
1589 3.64550558629162e-08
1590 3.64255328122454e-08
1591 3.64045469325447e-08
1592 3.63860692687012e-08
1593 3.63583403384382e-08
1594 3.63354608623467e-08
1595 3.63125707281142e-08
1596 3.62949457155537e-08
1597 3.62761092276287e-08
1598 3.62480037097157e-08
1599 3.62262966291382e-08
1600 3.62016052690706e-08
1601 3.61768321965883e-08
1602 3.61571892426582e-08
1603 3.61322349817783e-08
1604 3.61090144451737e-08
1605 3.60864227388902e-08
1606 3.60586192016399e-08
1607 3.60393350717914e-08
1608 3.60195997473056e-08
1609 3.59952636586058e-08
1610 3.59735246036053e-08
1611 3.59496041824059e-08
1612 3.59189584742126e-08
1613 3.59041365527446e-08
1614 3.58774485675895e-08
1615 3.58532901145736e-08
1616 3.58291742941219e-08
1617 3.58090836982683e-08
1618 3.57787008908872e-08
1619 3.5762333538969e-08
1620 3.57310554477408e-08
1621 3.57192924127503e-08
1622 3.56822731362172e-08
1623 3.56679770163737e-08
1624 3.56394735945287e-08
1625 3.56195108963675e-08
1626 3.55918707839464e-08
1627 3.55697231668728e-08
1628 3.55401716944925e-08
1629 3.55112526051471e-08
1630 3.55007543362262e-08
1631 3.54559261950271e-08
1632 3.54348266284887e-08
1633 3.53991680412946e-08
1634 3.54153755210973e-08
1635 3.53499451932748e-08
1636 3.53441720335468e-08
1637 3.53012374887385e-08
1638 3.53117606266551e-08
1639 3.52498190636652e-08
1640 3.52583242602122e-08
1641 3.51997435643625e-08
1642 3.52013991289368e-08
1643 3.51222873007373e-08
1644 3.52157840666223e-08
1645 3.50536595306039e-08
1646 3.51458133707183e-08
1647 3.50127820070156e-08
1648 3.50959723505184e-08
1649 3.49690374434886e-08
1650 3.49935440624449e-08
1651 3.49593456405728e-08
1652 3.50031790219418e-08
1653 3.48464475052879e-08
1654 3.48932793770018e-08
1655 3.48977060582456e-08
1656 3.48741338029868e-08
1657 3.48692097418279e-08
1658 3.4864541476054e-08
1659 3.4813684379742e-08
1660 3.48273800909737e-08
1661 3.47828184033006e-08
1662 3.47730484406839e-08
1663 3.47540343170749e-08
1664 3.47106841047662e-08
1665 3.47330342265195e-08
1666 3.46706450216061e-08
1667 3.46957058638964e-08
1668 3.46233335335455e-08
1669 3.46569244413786e-08
1670 3.45896822295799e-08
1671 3.46153150587725e-08
1672 3.45482291663757e-08
1673 3.45704016524451e-08
1674 3.45148976066412e-08
1675 3.4522166458828e-08
1676 3.44779991223731e-08
1677 3.44780808347878e-08
1678 3.4448390806574e-08
1679 3.44143735730995e-08
1680 3.44184378775481e-08
1681 3.43439694461267e-08
1682 3.43718120632275e-08
1683 3.42754660209721e-08
1684 3.42731354407988e-08
1685 3.42715651413528e-08
1686 3.4363477396937e-08
1687 3.42225874305768e-08
1688 3.41951533755491e-08
1689 3.41925279201405e-08
1690 3.42670531949807e-08
1691 3.41345227639067e-08
1692 3.41421326766067e-08
1693 3.40199974857569e-08
1694 3.41831700723105e-08
1695 3.4136764526238e-08
1696 3.415587812583e-08
1697 3.39013084271755e-08
1698 3.4019731032231e-08
1699 3.39690515716029e-08
1700 3.41352794919203e-08
1701 3.40728654180111e-08
1702 3.36060352879031e-08
1703 3.39974022267597e-08
1704 3.37770202918364e-08
1705 3.39112169456257e-08
1706 3.38700019142379e-08
1707 3.41598678232913e-08
1708 3.36830936475963e-08
1709 3.39897709977777e-08
1710 3.34775940302734e-08
1711 3.37925634141811e-08
1712 3.37240670944539e-08
1713 3.40194965531282e-08
1714 3.35396208583916e-08
1715 3.37861081334268e-08
1716 3.34328547069163e-08
1717 3.36345671314575e-08
1718 3.36812817636201e-08
1719 3.39331407417376e-08
1720 3.35583258959105e-08
1721 3.38425749646376e-08
1722 3.34622143327579e-08
1723 3.34501351062499e-08
1724 3.34122276512971e-08
1725 3.34344996133495e-08
1726 3.33689413878346e-08
1727 3.33460157264653e-08
1728 3.3319761172379e-08
1729 3.33196972235328e-08
1730 3.3296544188488e-08
1731 3.32382938950104e-08
1732 3.33001004548805e-08
1733 3.32077689790822e-08
1734 3.32464971108948e-08
1735 3.32098011313064e-08
1736 3.32771463718018e-08
1737 3.31899805416924e-08
1738 3.31676908160716e-08
1739 3.30801661618807e-08
1740 3.31104281769967e-08
1741 3.31280851639804e-08
1742 3.30245875090895e-08
1743 3.31300427092174e-08
1744 3.31017986354709e-08
1745 3.30665770320593e-08
1746 3.31485345839155e-08
1747 3.3058991988355e-08
1748 3.29840936785786e-08
1749 3.31216263305123e-08
1750 3.31051346336153e-08
1751 3.29767360085498e-08
1752 3.31017773191888e-08
1753 3.31028324751514e-08
1754 3.30926823721711e-08
1755 3.31478240411798e-08
1756 3.31716485391098e-08
1757 3.31133200859313e-08
1758 3.31299716549438e-08
1759 3.30613687538062e-08
1760 3.30832357064992e-08
1761 3.30463549857996e-08
1762 3.30597842435054e-08
1763 3.30020029082334e-08
1764 3.30168425932698e-08
1765 3.29706182355949e-08
1766 3.29917355657017e-08
1767 3.29337446203226e-08
1768 3.29570717383376e-08
1769 3.29110534380561e-08
1770 3.29419300726386e-08
1771 3.28280123085278e-08
1772 3.27864952964774e-08
1773 3.26482272328121e-08
1774 3.27076001838122e-08
1775 3.259020431301e-08
1776 3.26442588516329e-08
1777 3.25528048961132e-08
1778 3.25692148805956e-08
1779 3.24866569201276e-08
1780 3.25583400240248e-08
1781 3.24667581708127e-08
1782 3.25059268391215e-08
1783 3.24128137663138e-08
1784 3.24459037415181e-08
1785 3.23933448953539e-08
1786 3.24517692718018e-08
1787 3.23444915295568e-08
1788 3.23815143588035e-08
1789 3.22798818785941e-08
1790 3.2377151626406e-08
1791 3.22556310550226e-08
1792 3.23535047641599e-08
1793 3.21965494265442e-08
1794 3.22848023870392e-08
1795 3.21591500096474e-08
1796 3.23052624651154e-08
1797 3.21099840050465e-08
1798 3.22243742800765e-08
1799 3.20917301621648e-08
1800 3.21994768626155e-08
1801 3.20226156702574e-08
1802 3.21753788057322e-08
1803 3.20047703894488e-08
1804 3.21523820900893e-08
1805 3.19212283272918e-08
1806 3.2077814182685e-08
1807 3.19723021391383e-08
1808 3.20707762568873e-08
1809 3.18670352328354e-08
1810 3.20091864125516e-08
1811 3.18433279744568e-08
1812 3.20345669990729e-08
1813 3.17616937195453e-08
1814 3.18559045808797e-08
1815 3.1930927235635e-08
1816 3.17756132517388e-08
1817 3.19456674446883e-08
1818 3.1628967889219e-08
1819 3.17220028023257e-08
1820 3.18864721293721e-08
1821 3.16751744833255e-08
1822 3.18583452951771e-08
1823 3.15264365724488e-08
1824 3.17028003848918e-08
1825 3.1717995341296e-08
1826 3.16334620720227e-08
1827 3.17310870912024e-08
1828 3.14632586650987e-08
1829 3.16594146454463e-08
1830 3.1530007049696e-08
1831 3.17026120910668e-08
1832 3.13503285553907e-08
1833 3.13353503145208e-08
1834 3.13314600930426e-08
1835 3.13301029564172e-08
1836 3.13079198122068e-08
1837 3.1327008542803e-08
1838 3.13551531405665e-08
1839 3.14588746164191e-08
1840 3.15640846793031e-08
1841 3.11506802574968e-08
1842 3.11812335951345e-08
1843 3.15436174957995e-08
1844 3.12494812249042e-08
1845 3.14619050811871e-08
1846 3.10719840967977e-08
1847 3.12237240507329e-08
1848 3.14695682845922e-08
1849 3.11279428899525e-08
1850 3.13896606485287e-08
1851 3.09898560146848e-08
1852 3.12179544437186e-08
1853 3.12139327718342e-08
1854 3.11960590693161e-08
1855 3.1100395148087e-08
1856 3.11522967422206e-08
1857 3.09736698511642e-08
1858 3.13780219585169e-08
1859 3.08975032226044e-08
1860 3.08383150127156e-08
1861 3.07832124235574e-08
1862 3.079187749222e-08
1863 3.10678451853619e-08
1864 3.11215053727665e-08
1865 3.07533625232281e-08
1866 3.08402476889569e-08
1867 3.09911847296007e-08
1868 3.08604342080798e-08
1869 3.10915204693174e-08
1870 3.06831005048025e-08
1871 3.0687342444935e-08
1872 3.06315222076137e-08
1873 3.08383967251302e-08
1874 3.10279943960268e-08
1875 3.05420648771815e-08
1876 3.06009084738434e-08
1877 3.06779455172546e-08
1878 3.09324938996269e-08
1879 3.05066762962269e-08
1880 3.08724246167458e-08
1881 3.05010416923324e-08
1882 3.06880885148075e-08
1883 3.06001268768341e-08
1884 3.05332754635401e-08
1885 3.05212743967331e-08
1886 3.06314476006264e-08
1887 3.04979685950002e-08
1888 3.04529663708308e-08
1889 3.05684260126782e-08
1890 3.03519840372246e-08
1891 3.07046477132644e-08
1892 3.03113694144486e-08
1893 3.01762952403806e-08
1894 3.04935561246111e-08
1895 3.04098364267702e-08
1896 3.02191054402101e-08
1897 3.01319111883913e-08
1898 3.04563059216889e-08
1899 3.01939806490736e-08
1900 3.04652942872963e-08
1901 3.00785458762221e-08
1902 2.99091311717348e-08
1903 3.03911704691018e-08
1904 3.01398443980361e-08
1905 3.01016385151343e-08
1906 3.02462730417119e-08
1907 3.00535667463464e-08
1908 3.01027860416525e-08
1909 3.01394784685272e-08
1910 3.00940321551479e-08
1911 3.01231857235962e-08
1912 3.00140143849603e-08
1913 3.00150091447904e-08
1914 2.993643732907e-08
1915 3.00218090387716e-08
1916 2.97215869693446e-08
1917 2.99889251209606e-08
1918 2.99822744409539e-08
1919 2.9693056902147e-08
1920 2.98668396681023e-08
1921 2.99421394345245e-08
1922 2.97547373406815e-08
1923 2.98665945308585e-08
1924 2.96654114606554e-08
1925 2.9649333654902e-08
1926 2.9893399755565e-08
1927 2.9647928556642e-08
1928 2.9577549298665e-08
1929 2.9732289519302e-08
1930 2.95035231800966e-08
1931 2.97481523858778e-08
1932 2.95535205196984e-08
1933 2.9493973485728e-08
1934 2.95498381319703e-08
1935 2.95098452340881e-08
1936 2.95717494935843e-08
1937 2.9378377064404e-08
1938 2.96081168471574e-08
1939 2.94005726431124e-08
1940 2.93243314075653e-08
1941 2.93642905546676e-08
1942 2.94908613085454e-08
1943 2.92750712560519e-08
1944 2.93716677646216e-08
1945 2.9165546422405e-08
1946 2.90730550744911e-08
1947 2.92836510595862e-08
1948 2.9143690127853e-08
1949 2.93617556934578e-08
1950 2.912977592473e-08
1951 2.94808231160459e-08
1952 2.94089250729712e-08
1953 2.96880831029966e-08
1954 2.90387944801296e-08
1955 2.94665181144182e-08
1956 2.92256032707883e-08
1957 2.92757906805718e-08
1958 2.9695765846327e-08
1959 2.87230452755693e-08
1960 2.89461077329634e-08
1961 2.95184801046844e-08
1962 2.92251165490143e-08
1963 2.93142896623522e-08
1964 2.89976647138701e-08
1965 2.94374729037372e-08
1966 2.86355703593699e-08
1967 2.86718755404536e-08
1968 2.89975243816798e-08
1969 2.93822868258076e-08
1970 2.87114740871175e-08
1971 2.89733783631618e-08
1972 2.90524049262331e-08
1973 2.88413399829324e-08
1974 2.90893655829905e-08
1975 2.88563946071463e-08
1976 2.87708932233954e-08
1977 2.8933746065718e-08
1978 2.88055641561868e-08
1979 2.86765580170822e-08
1980 2.86618551115225e-08
1981 2.87397963205649e-08
1982 2.89275501330621e-08
1983 2.84831127572716e-08
1984 2.83142593815455e-08
1985 2.85573396041627e-08
1986 2.84304864095475e-08
1987 2.84369239267335e-08
1988 2.85295431723398e-08
1989 2.8736321766587e-08
1990 2.83120549227078e-08
1991 2.84026082653099e-08
1992 2.82791106087643e-08
1993 2.80880776415415e-08
1994 2.81519518807727e-08
1995 2.84240364578636e-08
1996 2.82637575566014e-08
1997 2.85879373507214e-08
1998 2.83531758071831e-08
1999 2.82439795995515e-08
2000 2.81821339598309e-08
2001 2.81953944636371e-08
2002 2.8050553879666e-08
2003 2.83899446174019e-08
2004 2.82298948661719e-08
2005 2.82162684328569e-08
2006 2.79395493407719e-08
2007 2.79340479636403e-08
2008 2.79105130118751e-08
2009 2.79566876315585e-08
2010 2.79086620480484e-08
2011 2.79861946950177e-08
2012 2.78596754554883e-08
2013 2.80673351227279e-08
2014 2.78806240316953e-08
2015 2.80344405467758e-08
2016 2.78660525765417e-08
2017 2.79070420106109e-08
2018 2.7814804681725e-08
2019 2.79724439167239e-08
2020 2.77772080892191e-08
2021 2.79729324148548e-08
2022 2.78237131112746e-08
2023 2.7856625450795e-08
2024 2.76744831495535e-08
2025 2.785338537592e-08
2026 2.77118576974544e-08
2027 2.78326055536127e-08
2028 2.76990856917791e-08
2029 2.77999294695519e-08
2030 2.76519074304815e-08
2031 2.77433507278602e-08
2032 2.76284968236951e-08
2033 2.76994054360102e-08
2034 2.76121348008473e-08
2035 2.77959149030949e-08
2036 2.76803522325508e-08
2037 2.76926126474564e-08
2038 2.75381726311252e-08
2039 2.77175438156974e-08
2040 2.74727227633775e-08
2041 2.76903335816314e-08
2042 2.74682037115781e-08
2043 2.76892091477521e-08
2044 2.74108646891591e-08
2045 2.77053953112727e-08
2046 2.76598033366326e-08
2047 2.73807607698018e-08
2048 2.74488183293897e-08
2049 2.74923763754487e-08
2050 2.73847380327652e-08
2051 2.74255445020799e-08
2052 2.7309328132219e-08
2053 2.73455249555354e-08
2054 2.74074025696791e-08
2055 2.73229154856836e-08
2056 2.71932272255526e-08
2057 2.73453348853536e-08
2058 2.72540994217252e-08
2059 2.73442104514743e-08
2060 2.73109037607355e-08
2061 2.72919873367528e-08
2062 2.69511790662591e-08
2063 2.73516747029134e-08
2064 2.71419651198812e-08
2065 2.72957247915429e-08
2066 2.69633240179701e-08
2067 2.72763074349314e-08
2068 2.71209366076164e-08
2069 2.72764353326238e-08
2070 2.67942859011328e-08
2071 2.72117901545244e-08
2072 2.68256901136965e-08
2073 2.73081095514272e-08
2074 2.70141402580748e-08
2075 2.73934279704235e-08
2076 2.64514046222075e-08
2077 2.71477276214682e-08
2078 2.66956305949861e-08
2079 2.71370108606561e-08
2080 2.70378546218808e-08
2081 2.7152486481441e-08
2082 2.6527077423566e-08
2083 2.70173856620204e-08
2084 2.66111381819201e-08
2085 2.69780091599614e-08
2086 2.67691007138637e-08
2087 2.68589843699374e-08
2088 2.65250612585533e-08
2089 2.70468678564839e-08
2090 2.67895252648032e-08
2091 2.68544297910012e-08
2092 2.64672621597128e-08
2093 2.67092961081516e-08
2094 2.66113371338861e-08
2095 2.68256670210576e-08
2096 2.64367479019256e-08
2097 2.66940851645359e-08
2098 2.6415341025654e-08
2099 2.69664930385716e-08
2100 2.66545470140045e-08
2101 2.66895359146702e-08
2102 2.61989541172625e-08
2103 2.63547335066505e-08
2104 2.64606381250587e-08
2105 2.65066351090582e-08
2106 2.62658836902574e-08
2107 2.65271182797733e-08
2108 2.62818478091731e-08
2109 2.66391460002069e-08
2110 2.63369788200407e-08
2111 2.66158117767645e-08
2112 2.6046276246916e-08
2113 2.63351758178487e-08
2114 2.6130692276638e-08
2115 2.64428194896027e-08
2116 2.62876866941042e-08
2117 2.6723107282578e-08
2118 2.61539021551016e-08
2119 2.62828034891527e-08
2120 2.59947263714366e-08
2121 2.64157264950882e-08
2122 2.57316106200278e-08
2123 2.62068535761273e-08
2124 2.62636046244324e-08
2125 2.67334510084538e-08
2126 2.57892445176822e-08
2127 2.66111115365675e-08
2128 2.60441535004929e-08
2129 2.62410360107879e-08
2130 2.59222137088955e-08
2131 2.61931738521071e-08
2132 2.58173429301678e-08
2133 2.60964228004923e-08
2134 2.56971031120656e-08
2135 2.64053525711461e-08
2136 2.61815618074479e-08
2137 2.61041641635984e-08
2138 2.57139021186958e-08
2139 2.61061305906196e-08
2140 2.56000109999377e-08
2141 2.57637484679663e-08
2142 2.5821366378409e-08
2143 2.58981245337964e-08
2144 2.59085322085184e-08
2145 2.61963961634137e-08
2146 2.53706566866185e-08
2147 2.57369165979071e-08
2148 2.55357619494134e-08
2149 2.64560249263468e-08
2150 2.54342999994606e-08
2151 2.62873083300974e-08
2152 2.5305602946446e-08
2153 2.62147548113489e-08
2154 2.6040126499538e-08
2155 2.61645016763623e-08
2156 2.57414143334245e-08
2157 2.59429011606471e-08
2158 2.55474699173419e-08
2159 2.56201282411439e-08
2160 2.57682692961225e-08
2161 2.56160177514175e-08
2162 2.58508912054367e-08
2163 2.54027199275697e-08
2164 2.59168793093068e-08
2165 2.52906406927877e-08
2166 2.57953054472182e-08
2167 2.51026506248309e-08
2168 2.57499390698968e-08
2169 2.55370178336989e-08
2170 2.55266296989021e-08
2171 2.56447219015854e-08
2172 2.57554706450946e-08
2173 2.54828211865288e-08
2174 2.52274681145082e-08
2175 2.54296566026824e-08
2176 2.53609897526985e-08
2177 2.53932501692589e-08
2178 2.57110990276033e-08
2179 2.52371279430008e-08
2180 2.56960728250988e-08
2181 2.50542555590982e-08
2182 2.5476337484065e-08
2183 2.53305110220481e-08
2184 2.57427199557014e-08
2185 2.50242795374334e-08
2186 2.54191689919026e-08
2187 2.53162912855487e-08
2188 2.56960728250988e-08
2189 2.51501646175711e-08
2190 2.57187142693738e-08
2191 2.46158187167111e-08
2192 2.53417216100615e-08
2193 2.52714009718602e-08
2194 2.572395452205e-08
2195 2.48607427977277e-08
2196 2.54900207607989e-08
2197 2.46869813480544e-08
2198 2.54348986317154e-08
2199 2.47797355967805e-08
2200 2.53143248585275e-08
2201 2.51097134196243e-08
2202 2.54846153069366e-08
2203 2.53932785909683e-08
2204 2.55623522349424e-08
2205 2.47758400462317e-08
2206 2.51119161021052e-08
2207 2.45688216438111e-08
2208 2.54265213328608e-08
2209 2.44589699605058e-08
2210 2.54153764700504e-08
2211 2.4563801659383e-08
2212 2.52783021181813e-08
2213 2.44704096985515e-08
2214 2.53518717130419e-08
2215 2.48731844010308e-08
2216 2.51572540577172e-08
2217 2.45467219883722e-08
2218 2.55410057548033e-08
2219 2.49872371682613e-08
2220 2.5574346196322e-08
2221 2.48347404863125e-08
2222 2.54145309241949e-08
2223 2.50624641040531e-08
2224 2.56129553122264e-08
2225 2.48336089470058e-08
2226 2.54615404315928e-08
2227 2.49926213058416e-08
2228 2.52048266702332e-08
2229 2.48189593321513e-08
2230 2.48821585557835e-08
2231 2.5074255560753e-08
2232 2.53003644701266e-08
2233 2.48770390953723e-08
2234 2.51527296768472e-08
2235 2.49331613133563e-08
2236 2.54894967355312e-08
2237 2.48486813347881e-08
2238 2.54106673480692e-08
2239 2.46466829167957e-08
2240 2.49263560903046e-08
2241 2.50456473338545e-08
2242 2.5709642414995e-08
2243 2.48104932154547e-08
2244 2.48769378430325e-08
2245 2.51419276509068e-08
2246 2.53396184035637e-08
2247 2.45401690079916e-08
2248 2.52109320086902e-08
2249 2.45800748643887e-08
2250 2.49533478324793e-08
2251 2.42367050873327e-08
2252 2.49066260948894e-08
2253 2.49253275796946e-08
2254 2.56172381085662e-08
2255 2.50318983319175e-08
2256 2.44386502146199e-08
2257 2.46446898444219e-08
2258 2.50760123776672e-08
2259 2.51500740233723e-08
2260 2.45430413770009e-08
2261 2.5556461835663e-08
2262 2.50309373228674e-08
2263 2.51467007217343e-08
2264 2.45011104738069e-08
2265 2.4520874220002e-08
2266 2.50207694563187e-08
2267 2.45448710245455e-08
2268 2.58234820194048e-08
2269 2.4032486223291e-08
2270 2.48078428910503e-08
2271 2.39790995948397e-08
2272 2.43678019984372e-08
2273 2.43802276145288e-08
2274 2.44535627302866e-08
2275 2.5104894163519e-08
2276 2.47251694673878e-08
2277 2.5188027663603e-08
2278 2.44166749041597e-08
2279 2.48716265360827e-08
2280 2.38335235991372e-08
2281 2.43589060033855e-08
2282 2.51651410820841e-08
2283 2.48136231562057e-08
2284 2.46798261827053e-08
2285 2.48799345570205e-08
2286 2.44587781139671e-08
2287 2.45960176670224e-08
2288 2.42566038366476e-08
2289 2.55463117326826e-08
2290 2.46208031740025e-08
2291 2.46601654652068e-08
2292 2.4284638300287e-08
2293 2.48723583951005e-08
2294 2.53581848852491e-08
2295 2.55454839503955e-08
2296 2.41208777396196e-08
2297 2.52394940503109e-08
2298 2.51822456220907e-08
2299 2.39366126919549e-08
2300 2.49059564083609e-08
2301 2.43411069078547e-08
2302 2.4880028703933e-08
2303 2.49805083285537e-08
2304 2.36740760328757e-08
2305 2.50324223571852e-08
2306 2.45540476839778e-08
2307 2.44128504078844e-08
2308 2.57055337016254e-08
2309 2.39016078040777e-08
2310 2.38765931470653e-08
2311 2.44029152440817e-08
2312 2.46182185748012e-08
2313 2.43784086251253e-08
2314 2.51217251445723e-08
2315 2.34445689528684e-08
2316 2.45784441688102e-08
2317 2.40191635469955e-08
2318 2.4638218576456e-08
2319 2.39349375874554e-08
2320 2.47769786909657e-08
2321 2.38963782095425e-08
2322 2.4304526391461e-08
2323 2.38840254240813e-08
2324 2.47516869222864e-08
2325 2.39497577325665e-08
2326 2.41678161927439e-08
2327 2.45075231219971e-08
2328 2.38905393246114e-08
2329 2.43206326189238e-08
2330 2.46397924286157e-08
2331 2.43201299099383e-08
2332 2.40385720218228e-08
2333 2.43951632228345e-08
2334 2.5198644948432e-08
2335 2.42042670350884e-08
2336 2.44360887080575e-08
2337 2.45378881658098e-08
2338 2.41560549341102e-08
2339 2.45357352213205e-08
2340 2.35631816281057e-08
2341 2.40037554277706e-08
2342 2.43480915429473e-08
2343 2.3721833386503e-08
2344 2.48480844788901e-08
2345 2.35888411026508e-08
2346 2.41430324621206e-08
2347 2.35269794757187e-08
2348 2.41879494211616e-08
2349 2.38349340264676e-08
2350 2.41071713702468e-08
2351 2.39092425857734e-08
2352 2.45703155599131e-08
2353 2.42728575017281e-08
2354 2.47744136316896e-08
2355 2.41361686192931e-08
2356 2.40583446498022e-08
2357 2.44475231170327e-08
2358 2.37407942194068e-08
2359 2.40909212578799e-08
2360 2.34965469303461e-08
2361 2.40708217802421e-08
2362 2.43717792614007e-08
2363 2.34896493367387e-08
2364 2.48114027101565e-08
2365 2.452256708807e-08
2366 2.32712835668281e-08
2367 2.38509869632253e-08
2368 2.43159501422952e-08
2369 2.39270789847978e-08
2370 2.25415757171277e-08
2371 2.3598937914926e-08
2372 2.34576429392064e-08
2373 2.41432100978045e-08
2374 2.35140333870731e-08
2375 2.4686785948802e-08
2376 2.41015367663522e-08
2377 2.42415776341431e-08
2378 2.4190146774572e-08
2379 2.30882033491753e-08
2380 2.36546426890527e-08
2381 2.3576705032724e-08
2382 2.43908750974242e-08
2383 2.42977957753965e-08
2384 2.38146355968638e-08
2385 2.40431461406843e-08
2386 2.36869936998119e-08
2387 2.49295428744745e-08
2388 2.37129782476586e-08
2389 2.41859741123562e-08
2390 2.38360833293427e-08
2391 2.38792008389055e-08
2392 2.40652724414758e-08
2393 2.43901716601158e-08
2394 2.285350753084e-08
2395 2.41111592913512e-08
2396 2.48120848311828e-08
2397 2.43937883226408e-08
2398 2.48459830487491e-08
2399 2.40843220922216e-08
2400 2.36642918594043e-08
2401 2.33330634813456e-08
2402 2.27730723167952e-08
2403 2.28215650821539e-08
2404 2.38962645227048e-08
2405 2.26269634140408e-08
2406 2.38320030376826e-08
2407 2.27298198041126e-08
2408 2.31541239514854e-08
2409 2.26743210873792e-08
2410 2.37551187609597e-08
2411 2.31905374903363e-08
2412 2.33515109471227e-08
2413 2.30782770671567e-08
2414 2.43073845496156e-08
2415 2.36767849770558e-08
2416 2.29529799611328e-08
2417 2.30221139929654e-08
2418 2.32195844773742e-08
2419 2.24893010880578e-08
2420 2.32490773299787e-08
2421 2.26085141719068e-08
2422 2.30809042989222e-08
2423 2.30376162591028e-08
2424 2.31073293832651e-08
2425 2.32477699313449e-08
2426 2.33584689368627e-08
2427 2.2569965452135e-08
2428 2.30373142784401e-08
2429 2.32656454102198e-08
2430 2.2606709393358e-08
2431 2.30059189476606e-08
2432 2.24781313562517e-08
2433 2.33895942614026e-08
2434 2.28614656094805e-08
2435 2.26162395478013e-08
2436 2.31262866634552e-08
2437 2.29366463599945e-08
2438 2.22618723455525e-08
2439 2.31501164904557e-08
2440 2.23671907662037e-08
2441 2.32315571224717e-08
2442 2.22299831875716e-08
2443 2.32995649440682e-08
2444 2.395503884145e-08
2445 2.30956125335524e-08
2446 2.39914790256535e-08
2447 2.31098891134707e-08
2448 2.3461529607971e-08
2449 2.22649489955984e-08
2450 2.315142921816e-08
2451 2.21944116418626e-08
2452 2.29830874332038e-08
2453 2.30334293860324e-08
2454 2.26723173568644e-08
2455 2.2656648113184e-08
2456 2.34643806606982e-08
2457 2.31091519253823e-08
2458 2.31313741494432e-08
2459 2.3497852552623e-08
2460 2.41219808572168e-08
2461 2.39585276062826e-08
2462 2.42798332550365e-08
2463 2.30441603576992e-08
2464 2.37484556464551e-08
2465 2.2325588489025e-08
2466 2.3448345487509e-08
2467 2.25853167279411e-08
2468 2.35905748269261e-08
2469 2.3510583702091e-08
2470 2.24238227986007e-08
2471 2.29717080912906e-08
2472 2.28533849622181e-08
2473 2.25557226229967e-08
2474 2.31833219288546e-08
2475 2.24251852642965e-08
2476 2.27118324147568e-08
2477 2.25036078660423e-08
2478 2.27296634847107e-08
2479 2.29373942062239e-08
2480 2.23383374020614e-08
2481 2.26765735078516e-08
2482 2.27827321452878e-08
2483 2.27457501722483e-08
2484 2.25922534013989e-08
2485 2.21703118086225e-08
2486 2.35323689423694e-08
2487 2.24411706994943e-08
2488 2.23352394357335e-08
2489 2.2249269093777e-08
2490 2.28142429392619e-08
2491 2.21774563158306e-08
2492 2.3060653830953e-08
2493 2.24263541070968e-08
2494 2.27927010598705e-08
2495 2.24690452910181e-08
2496 2.26028920025101e-08
2497 2.22056115717351e-08
2498 2.32156462942612e-08
2499 2.23327525361583e-08
2500 2.23107754493412e-08
2501 2.3229153711668e-08
2502 2.24012648430971e-08
2503 2.24333192022641e-08
2504 2.23837197665944e-08
2505 2.26137935044335e-08
2506 2.29114380800866e-08
2507 2.23332659032849e-08
2508 2.21501075259312e-08
2509 2.27365131166835e-08
2510 2.20852243160152e-08
2511 2.25060539094102e-08
2512 2.19794031863785e-08
2513 2.31391847904661e-08
2514 2.18871534229947e-08
2515 2.31884715873321e-08
2516 2.16652136231232e-08
2517 2.33359909174169e-08
2518 2.21028084723685e-08
2519 2.31187105015351e-08
2520 2.22926814785751e-08
2521 2.26331930974766e-08
2522 2.23546603450586e-08
2523 2.21430127567146e-08
2524 2.19928999456442e-08
2525 2.230191853414e-08
2526 2.19008864377201e-08
2527 2.25022027677824e-08
2528 2.19953300018005e-08
2529 2.24115073166331e-08
2530 2.18432525400658e-08
2531 2.25219878302596e-08
2532 2.21992806359594e-08
2533 2.2408290334397e-08
2534 2.18461337908593e-08
2535 2.25512817308982e-08
2536 2.18545519459212e-08
2537 2.20839666553729e-08
2538 2.20653042504182e-08
2539 2.24903704548751e-08
2540 2.21010640899522e-08
2541 2.17291713511258e-08
2542 2.22744667155439e-08
2543 2.18245634897585e-08
2544 2.21716582871068e-08
2545 2.20645528514751e-08
2546 2.20651372728753e-08
2547 2.20856399835156e-08
2548 2.24477894050779e-08
2549 2.21758575946751e-08
2550 2.17074518360505e-08
2551 2.24233946966024e-08
2552 2.18800106921435e-08
2553 2.19087841202281e-08
2554 2.20775859816058e-08
2555 2.22944205319209e-08
2556 2.21503437813908e-08
2557 2.17647233569096e-08
2558 2.21787477272528e-08
2559 2.15872297815167e-08
2560 2.20451337185068e-08
2561 2.191757886294e-08
2562 2.22567937413487e-08
2563 2.1607114319977e-08
2564 2.21669136379887e-08
2565 2.18983871036471e-08
2566 2.23547083066933e-08
2567 2.13900079870655e-08
2568 2.23725233894356e-08
2569 2.14473345749866e-08
2570 2.22930296445156e-08
2571 2.1535473848644e-08
2572 2.23268088461737e-08
2573 2.16727809032591e-08
2574 2.25797958108842e-08
2575 2.13115267655439e-08
2576 2.2393424004008e-08
2577 2.15354809540713e-08
2578 2.21026290603277e-08
2579 2.16967936950141e-08
2580 2.20588844968006e-08
2581 2.21122249399741e-08
2582 2.21425597857206e-08
2583 2.16786375517586e-08
2584 2.21494271812617e-08
2585 2.15679296644566e-08
2586 2.2003023403272e-08
2587 2.15430873140576e-08
2588 2.18922746597627e-08
2589 2.15833804162457e-08
2590 2.14424122901846e-08
2591 2.16317559420531e-08
2592 2.15410072001987e-08
2593 2.15612097065332e-08
2594 2.13883488697775e-08
2595 2.16174189660023e-08
2596 2.12298552071388e-08
2597 2.17190638807097e-08
2598 2.11734594302015e-08
2599 2.17822968551218e-08
2600 2.11434372232588e-08
2601 2.19698694792214e-08
2602 2.12125641496641e-08
2603 2.19023235104032e-08
2604 2.12955910683377e-08
2605 2.23881748695476e-08
2606 2.16521041096485e-08
2607 2.21786482512698e-08
2608 2.14571915790884e-08
2609 2.2252200082562e-08
2610 2.18908073890134e-08
2611 2.14043947011078e-08
2612 2.20548042051405e-08
2613 2.19402220835718e-08
2614 2.14753814731239e-08
2615 2.14808792975418e-08
2616 2.17069633379197e-08
2617 2.16237765471305e-08
2618 2.16025881627502e-08
2619 2.20330100830779e-08
2620 2.14670006215556e-08
2621 2.2887519435244e-08
2622 2.11407016337262e-08
2623 2.21981313330843e-08
2624 2.18887397096523e-08
2625 2.1672128980299e-08
2626 2.17512852174195e-08
2627 2.19066116358135e-08
2628 2.00483487589054e-08
2629 2.07627248727249e-08
2630 2.00191099253288e-08
2631 2.13704858254005e-08
2632 2.20071125767163e-08
2633 2.16629061355889e-08
2634 2.13930899661818e-08
2635 2.16309334888365e-08
2636 2.09490185199002e-08
2637 2.14807123199989e-08
2638 2.11791153503782e-08
2639 2.14143280885537e-08
2640 2.15285513860408e-08
2641 2.21434817149202e-08
2642 2.14329869407948e-08
2643 2.1138458095038e-08
2644 2.19345146490468e-08
2645 2.20175859766414e-08
2646 2.24923333291827e-08
2647 2.03201171444789e-08
2648 2.04284695826118e-08
2649 2.17990230311216e-08
2650 2.06902566191047e-08
2651 2.17321609596866e-08
2652 2.16369304695263e-08
2653 2.14699475975522e-08
2654 2.10372590458974e-08
2655 2.17320277329236e-08
2656 2.15693276572893e-08
2657 2.18599272017173e-08
2658 1.98558147701533e-08
2659 2.1682073025886e-08
2660 2.21693259305766e-08
2661 2.17289439774504e-08
2662 2.01722389903125e-08
2663 2.04500008038622e-08
2664 2.06996926266356e-08
2665 2.21648281950593e-08
2666 2.25087717353745e-08
2667 2.2728853465992e-08
2668 2.17763336252119e-08
2669 2.01824885692758e-08
2670 2.04104875223265e-08
2671 2.15828652727623e-08
2672 2.16291766719223e-08
2673 2.20226308300653e-08
2674 2.06535091251681e-08
2675 2.04924948121743e-08
2676 2.0710793080525e-08
2677 2.09003463425006e-08
2678 2.00370582348341e-08
2679 2.01445402581157e-08
2680 2.061078241411e-08
2681 2.17555946591119e-08
2682 2.15048405749485e-08
2683 2.19159481673614e-08
2684 2.11680202255593e-08
2685 2.12974402558075e-08
2686 2.09393693495485e-08
2687 2.05445402912119e-08
2688 2.07688639619619e-08
2689 2.1219456414201e-08
2690 1.98799821049533e-08
2691 2.02812771021854e-08
2692 2.06269330504938e-08
2693 2.11202006994426e-08
2694 2.20207656553839e-08
2695 1.96006801900239e-08
2696 2.06998738150332e-08
2697 2.12910045149783e-08
2698 2.17720774742247e-08
2699 2.36403288056408e-08
2700 2.15677644632706e-08
2701 2.02800070070452e-08
2702 2.1266110650231e-08
2703 2.00410035233745e-08
2704 1.9818344298983e-08
2705 1.98443430576845e-08
2706 2.06129531221677e-08
2707 2.13516297975502e-08
2708 2.10727630900465e-08
2709 2.08860466699434e-08
2710 2.05419912191473e-08
2711 2.10657358223898e-08
2712 2.01892653706182e-08
2713 2.18479492275492e-08
2714 2.00503809111297e-08
2715 2.05876506953473e-08
2716 2.03385397412603e-08
2717 2.14775788265342e-08
2718 1.97998115680775e-08
2719 1.97596943252165e-08
2720 2.01719227987951e-08
2721 2.08799022516359e-08
2722 2.09830464115157e-08
2723 2.1539355188338e-08
2724 2.0828336388945e-08
2725 1.94553777532747e-08
2726 1.96599749813231e-08
2727 2.04291232819287e-08
2728 2.08599946205368e-08
2729 2.02042151897786e-08
2730 1.94791187624332e-08
2731 2.00678353934336e-08
2732 1.98812522000935e-08
2733 1.93272882142992e-08
2734 2.0368540631921e-08
2735 1.88285067537208e-08
2736 2.09377155613311e-08
2737 2.04349550614324e-08
2738 2.07290682396888e-08
2739 1.98009288965295e-08
2740 2.08909298748949e-08
2741 2.00349603574068e-08
2742 2.17454658724137e-08
2743 2.10510719966805e-08
2744 2.13567954432392e-08
2745 1.93886950938804e-08
2746 2.06169108452059e-08
2747 2.08187209693733e-08
2748 1.92678744070918e-08
2749 2.05518464468923e-08
2750 2.03232399798026e-08
2751 2.05455812363198e-08
2752 1.93816251936596e-08
2753 2.18179234678928e-08
2754 1.97311553762347e-08
2755 2.04972288031513e-08
2756 2.11027852969892e-08
2757 2.08851851368763e-08
2758 1.81912156449471e-08
2759 2.1538658856457e-08
2760 2.00818579543238e-08
2761 2.14440429857632e-08
2762 1.85328481450142e-08
2763 2.22223164314528e-08
2764 1.92580564828404e-08
2765 2.06265031721387e-08
2766 1.9007689644468e-08
2767 2.03949568344797e-08
2768 2.04854497809492e-08
2769 2.10268353839638e-08
2770 2.10045438819861e-08
2771 2.22280025496957e-08
2772 1.90772215802326e-08
2773 2.18796269990662e-08
2774 1.93484765986796e-08
2775 2.07163299847934e-08
2776 1.9415665519773e-08
2777 2.14222914962647e-08
2778 2.14231636874729e-08
2779 1.9326504840933e-08
2780 2.11852508869015e-08
2781 2.09554773533682e-08
2782 2.04959462735133e-08
2783 2.06011083747626e-08
2784 1.94921803142734e-08
2785 2.20450786514448e-08
2786 2.03495584827351e-08
2787 1.92779143759481e-08
2788 1.96752054648641e-08
2789 1.98027354514352e-08
2790 2.14707043255657e-08
2791 2.03900558659598e-08
2792 1.97403320356671e-08
2793 1.87293984765802e-08
2794 2.08805133183887e-08
2795 2.04795558289561e-08
2796 1.9650917337799e-08
2797 2.12222364126546e-08
2798 2.04661851910259e-08
2799 1.95567828598087e-08
2800 1.98156797637239e-08
2801 2.06674659608552e-08
2802 1.92018418943007e-08
2803 2.16992006585315e-08
2804 2.07385646433522e-08
2805 2.22672582594896e-08
2806 1.99132195177754e-08
2807 2.06515995415657e-08
2808 2.00856042908981e-08
2809 1.90514199971403e-08
2810 1.9666469341928e-08
2811 1.77787899957593e-08
2812 2.01431209490011e-08
2813 1.83947648224603e-08
2814 1.85416375586556e-08
2815 1.95781684197982e-08
2816 1.86953386105415e-08
2817 1.98081924196458e-08
2818 1.84597350738613e-08
2819 1.92511873109424e-08
2820 1.93656664038144e-08
2821 1.95952836179458e-08
2822 1.94612095327784e-08
2823 2.11753388157376e-08
2824 1.79711641123959e-08
2825 2.04768877409833e-08
2826 2.01824637002801e-08
2827 2.12060857762708e-08
2828 1.81295760626199e-08
2829 1.9717282029319e-08
2830 1.84046911044788e-08
2831 1.99059240202359e-08
2832 1.7941740537708e-08
2833 2.12735571381018e-08
2834 1.93624902777856e-08
2835 1.85108497419151e-08
2836 1.97187386419273e-08
2837 2.0164756975305e-08
2838 1.92621580907826e-08
2839 1.91408684457883e-08
2840 1.93287021943434e-08
2841 1.9093498337952e-08
2842 1.90901729979487e-08
2843 1.82641652912707e-08
2844 2.0163341218904e-08
2845 1.79240782216539e-08
2846 2.02333794163678e-08
2847 1.90139139988332e-08
2848 1.98001348650223e-08
2849 1.82617263533302e-08
2850 1.90544753309041e-08
2851 1.95107148215357e-08
2852 1.75629590870585e-08
2853 2.04911909662542e-08
2854 1.84241208955882e-08
2855 1.99009750900814e-08
2856 1.85167916555429e-08
2857 2.06946566549959e-08
2858 1.89776283576748e-08
2859 2.02370298296728e-08
2860 1.88955073809893e-08
2861 1.82342532184521e-08
2862 1.95832203786495e-08
2863 1.96498159965586e-08
2864 1.93693772132519e-08
2865 1.85324982027169e-08
2866 1.98210390323084e-08
2867 1.79284729284745e-08
2868 2.09395558670167e-08
2869 1.83015398391717e-08
2870 2.03270911214304e-08
2871 1.82767259104821e-08
2872 2.03833927514552e-08
2873 1.78345853640849e-08
2874 1.97677270108443e-08
2875 1.88378947996171e-08
2876 1.93107574375517e-08
2877 1.89174258480307e-08
2878 1.85627548887624e-08
2879 1.81774133523049e-08
2880 1.87962267972352e-08
2881 1.9733519707188e-08
2882 1.9485238311745e-08
2883 1.91774045532611e-08
2884 1.9763549019558e-08
2885 1.87969888543194e-08
2886 1.95706206795876e-08
2887 1.78563102082308e-08
2888 2.03518002450664e-08
2889 1.88241440213233e-08
2890 1.88662951927654e-08
2891 1.92881639549114e-08
2892 1.88563920033857e-08
2893 1.96031493260307e-08
2894 1.94104057271716e-08
2895 1.9339136514418e-08
2896 1.91411846373057e-08
2897 1.97181098116062e-08
2898 1.88854549776352e-08
2899 1.81436803359247e-08
2900 1.9129313244548e-08
2901 1.97690788183991e-08
2902 1.86969195681286e-08
2903 1.85965021159973e-08
2904 1.93772606849052e-08
2905 2.02202716792499e-08
2906 1.7552025610712e-08
2907 1.99297787162322e-08
2908 1.92240943164279e-08
2909 1.93503097989378e-08
2910 1.89429023578214e-08
2911 1.86261335244353e-08
2912 1.93073876886274e-08
2913 2.08962234182763e-08
2914 1.79964505520047e-08
2915 1.97373957178115e-08
2916 1.87306579135793e-08
2917 1.83253892060975e-08
2918 1.86236874810675e-08
2919 1.91777527192016e-08
2920 1.84051582863276e-08
2921 1.94862650459982e-08
2922 1.81702208834622e-08
2923 1.97102334453803e-08
2924 1.86669470991774e-08
2925 1.9315432808753e-08
2926 1.85008222075567e-08
2927 1.94347773430081e-08
2928 1.80518782144645e-08
2929 1.97776088839419e-08
2930 1.81194614867763e-08
2931 1.98124201489236e-08
2932 1.78724217647641e-08
2933 1.95816447501329e-08
2934 1.73330061414845e-08
2935 1.96733207502575e-08
2936 1.81018950939915e-08
2937 1.88611064544375e-08
2938 1.83686825749874e-08
2939 1.92805629239956e-08
2940 1.77356813679808e-08
2941 2.03372874096885e-08
2942 1.75593761753134e-08
2943 2.02208028099449e-08
2944 1.88517379484665e-08
2945 1.96560865362017e-08
2946 1.71376406399304e-08
2947 1.87006552465618e-08
2948 1.86577207017535e-08
2949 1.86637887367169e-08
2950 1.97249629962926e-08
2951 1.78407741913134e-08
2952 2.02391881032327e-08
2953 1.82061743458917e-08
2954 1.84958484084063e-08
2955 1.79218861973141e-08
2956 2.02550811678748e-08
2957 1.81251760267287e-08
2958 2.0123190225263e-08
2959 1.78849628440503e-08
2960 1.9159996256235e-08
2961 1.7147135267237e-08
2962 1.9575361775992e-08
2963 1.76966974407833e-08
2964 1.94343172665867e-08
2965 1.80446431130576e-08
2966 1.86791098144568e-08
2967 1.84665456259836e-08
2968 1.87487110281381e-08
2969 1.72409055920753e-08
2970 1.91529903048604e-08
2971 1.83956885280168e-08
2972 1.79817245538061e-08
2973 1.84185449114693e-08
2974 2.03206500515307e-08
2975 1.8881065599885e-08
2976 1.95640659228502e-08
2977 1.84833712779664e-08
2978 1.93914804214046e-08
2979 1.7462621570985e-08
2980 1.92248421626573e-08
2981 1.70822076484001e-08
2982 1.94045686185973e-08
2983 1.83568342748686e-08
2984 1.94661939900698e-08
2985 1.7257333340126e-08
2986 1.88442541571021e-08
2987 1.94579410361939e-08
2988 1.79221597562673e-08
2989 1.844093588943e-08
2990 1.82945623095065e-08
2991 1.87603692580751e-08
2992 1.74362408955631e-08
2993 2.10287378621388e-08
2994 1.77277161839129e-08
2995 1.84623587529131e-08
2996 1.84716366646853e-08
2997 1.77392394107301e-08
2998 1.82776975776733e-08
2999 1.92737505955165e-08
3000 8.10947486939995e-09
3001 8.37022984256919e-09
3002 8.49825543269844e-09
3003 8.54340242995022e-09
3004 8.53520187860113e-09
3005 8.53064552330807e-09
3006 8.52645509752392e-09
3007 8.52285353403204e-09
3008 8.51948467328612e-09
3009 8.51628811915361e-09
3010 8.51342107921482e-09
3011 8.51066062068639e-09
3012 8.50848902445023e-09
3013 8.5058502463653e-09
3014 8.50352588344094e-09
3015 8.50157455545286e-09
3016 8.49969961080888e-09
3017 8.4975999570247e-09
3018 8.49593551066619e-09
3019 8.4940108280307e-09
3020 8.49233305899588e-09
3021 8.49097414601374e-09
3022 8.48918269014121e-09
3023 8.48780068452015e-09
3024 8.48628545213614e-09
3025 8.48499048800022e-09
3026 8.48371595196795e-09
3027 8.48228243199856e-09
3028 8.48118375529339e-09
3029 8.47983905316596e-09
3030 8.47881143073437e-09
3031 8.47768522049819e-09
3032 8.47642223078537e-09
3033 8.47541326010059e-09
3034 8.47437409134955e-09
3035 8.47319725494344e-09
3036 8.47217052069027e-09
3037 8.47106385037932e-09
3038 8.46984526958749e-09
3039 8.46912584506754e-09
3040 8.46822079125786e-09
3041 8.46721182057308e-09
3042 8.46613446014999e-09
3043 8.46567260737174e-09
3044 8.46449577096564e-09
3045 8.46352765648817e-09
3046 8.4627309604457e-09
3047 8.46158343392744e-09
3048 8.46083558769806e-09
3049 8.45983816333273e-09
3050 8.4591693649827e-09
3051 8.458452604998e-09
3052 8.45747827327159e-09
3053 8.4568068103863e-09
3054 8.45591596743134e-09
3055 8.45497005741436e-09
3056 8.45416892047979e-09
3057 8.4534708122419e-09
3058 8.45267145166417e-09
3059 8.45196534982051e-09
3060 8.4510629605461e-09
3061 8.4504137021213e-09
3062 8.44969161306608e-09
3063 8.44917202869055e-09
3064 8.44855740922412e-09
3065 8.4478664064136e-09
3066 8.44699332702703e-09
3067 8.446009225338e-09
3068 8.4454772064646e-09
3069 8.44481728989877e-09
3070 8.44411474076878e-09
3071 8.44373460040515e-09
3072 8.44292102897271e-09
3073 8.44226200058529e-09
3074 8.44169267821826e-09
3075 8.44110914499652e-09
3076 8.44052916448845e-09
3077 8.44018632761845e-09
3078 8.43927150384616e-09
3079 8.43863290356239e-09
3080 8.43784242476886e-09
3081 8.43752445689461e-09
3082 8.43690806107134e-09
3083 8.43632630420643e-09
3084 8.43566727581901e-09
3085 8.43496472668903e-09
3086 8.43445491227612e-09
3087 8.43374348136194e-09
3088 8.43333403111046e-09
3089 8.4326234883747e-09
3090 8.43203995515296e-09
3091 8.43142178297285e-09
3092 8.43109315695756e-09
3093 8.43029290820141e-09
3094 8.42988701066361e-09
3095 8.42922709409777e-09
3096 8.42867731165597e-09
3097 8.42837710735012e-09
3098 8.42748626439516e-09
3099 8.42701108894062e-09
3100 8.42627478903069e-09
3101 8.42603053996527e-09
3102 8.42519298771549e-09
3103 8.42454639382595e-09
3104 8.42422309688118e-09
3105 8.42362180009104e-09
3106 8.42318925720065e-09
3107 8.42262171119046e-09
3108 8.42189251670789e-09
3109 8.42145908563907e-09
3110 8.42087288788207e-09
3111 8.42029201919559e-09
3112 8.42013392343688e-09
3113 8.41932479289653e-09
3114 8.41874747692373e-09
3115 8.41845082533155e-09
3116 8.41764791204014e-09
3117 8.41735658951848e-09
3118 8.41670289020158e-09
3119 8.41626501824067e-09
3120 8.41578806642929e-09
3121 8.4153466417547e-09
3122 8.41485192637492e-09
3123 8.41422309605377e-09
3124 8.41363334558309e-09
3125 8.41308978039024e-09
3126 8.41259240047521e-09
3127 8.41211367230699e-09
3128 8.41169001120079e-09
3129 8.41111713612008e-09
3130 8.41063574341661e-09
3131 8.41010638907846e-09
3132 8.4097786512416e-09
3133 8.40921909883718e-09
3134 8.40863911832912e-09
3135 8.40808755953049e-09
3136 8.40759728504281e-09
3137 8.4072047101813e-09
3138 8.40652170097655e-09
3139 8.40589731154751e-09
3140 8.40568592508362e-09
3141 8.40516278799441e-09
3142 8.40440783633767e-09
3143 8.40411384928075e-09
3144 8.40356761955263e-09
3145 8.40308889138441e-09
3146 8.40279135161381e-09
3147 8.40222025288995e-09
3148 8.40201774821026e-09
3149 8.40142977409641e-09
3150 8.40081959552208e-09
3151 8.40030178750339e-09
3152 8.39990033085769e-09
3153 8.39942693175999e-09
3154 8.3991613664125e-09
3155 8.39862668300384e-09
3156 8.39808933505992e-09
3157 8.39758040882543e-09
3158 8.39726066459434e-09
3159 8.39687785969545e-09
3160 8.3963120900421e-09
3161 8.39587421808119e-09
3162 8.39546920872181e-09
3163 8.39504377125877e-09
3164 8.39448421885436e-09
3165 8.39421243625793e-09
3166 8.39344593828173e-09
3167 8.39312974676432e-09
3168 8.39263858409822e-09
3169 8.39199465474394e-09
3170 8.39161895527241e-09
3171 8.39133118546442e-09
3172 8.39079206116367e-09
3173 8.39039415723164e-09
3174 8.38983194029197e-09
3175 8.38927949331492e-09
3176 8.38873237540838e-09
3177 8.38849700812716e-09
3178 8.38783709156132e-09
3179 8.38758040799803e-09
3180 8.38721891938121e-09
3181 8.38666913693942e-09
3182 8.38643643419346e-09
3183 8.38594083063526e-09
3184 8.38536795555456e-09
3185 8.38485991749849e-09
3186 8.38438829475763e-09
3187 8.38391933655203e-09
3188 8.38363778399298e-09
3189 8.38296987382137e-09
3190 8.38243607859113e-09
3191 8.38196267949343e-09
3192 8.38163494165656e-09
3193 8.38152658388935e-09
3194 8.38131253289021e-09
3195 8.38033464845012e-09
3196 8.38018010540509e-09
3197 8.38004865499897e-09
3198 8.37960101307544e-09
3199 8.37880520521139e-09
3200 8.37848457280188e-09
3201 8.377786464564e-09
3202 8.37759017713324e-09
3203 8.37748537207972e-09
3204 8.37693203692425e-09
3205 8.3763529445946e-09
3206 8.37593194802366e-09
3207 8.3758342483975e-09
3208 8.37542479814601e-09
3209 8.37497982075774e-09
3210 8.37437319489709e-09
3211 8.37430125244509e-09
3212 8.37373281825649e-09
3213 8.37331359804239e-09
3214 8.3729041477909e-09
3215 8.37216873605939e-09
3216 8.37213409710102e-09
3217 8.3713755927306e-09
3218 8.371149995412e-09
3219 8.37076097326417e-09
3220 8.3704829734188e-09
3221 8.36998470532535e-09
3222 8.36949354265926e-09
3223 8.36917823932026e-09
3224 8.3685849361359e-09
3225 8.36808666804245e-09
3226 8.3679898565947e-09
3227 8.36761504530159e-09
3228 8.36724023400848e-09
3229 8.36648883506541e-09
3230 8.3664071226508e-09
3231 8.36588842645369e-09
3232 8.36524538527783e-09
3233 8.36510505308752e-09
3234 8.36475866350384e-09
3235 8.36443181384539e-09
3236 8.36401881088022e-09
3237 8.36350100286154e-09
3238 8.36318214680887e-09
3239 8.36259861358712e-09
3240 8.36230373835178e-09
3241 8.36193247977235e-09
3242 8.36180280572307e-09
3243 8.36113489555146e-09
3244 8.36072455712156e-09
3245 8.36036662121842e-09
3246 8.35975200175199e-09
3247 8.35962232770271e-09
3248 8.35918090302812e-09
3249 8.35875813010034e-09
3250 8.35817992594912e-09
3251 8.35802183019041e-09
3252 8.3575075748854e-09
3253 8.35707769653027e-09
3254 8.35662739007148e-09
3255 8.35638847007658e-09
3256 8.35612468108593e-09
3257 8.35585822756002e-09
3258 8.35520541642154e-09
3259 8.35463787041135e-09
3260 8.35452684810889e-09
3261 8.35413249689054e-09
3262 8.35395752574186e-09
3263 8.35349833749888e-09
3264 8.35315017155835e-09
3265 8.35243252339524e-09
3266 8.352241565035e-09
3267 8.35184810199507e-09
3268 8.35137470289737e-09
3269 8.35132407672745e-09
3270 8.35093683093646e-09
3271 8.35032309964845e-09
3272 8.34985769415653e-09
3273 8.34982749609026e-09
3274 8.34902369462043e-09
3275 8.34865332421941e-09
3276 8.3483504553783e-09
3277 8.34823588036215e-09
3278 8.34785573999852e-09
3279 8.34737878818714e-09
3280 8.34677660321859e-09
3281 8.34635471846923e-09
3282 8.34624191980993e-09
3283 8.34580138331376e-09
3284 8.34561042495352e-09
3285 8.34510860414639e-09
3286 8.34491409307248e-09
3287 8.34455349263408e-09
3288 8.34421864936985e-09
3289 8.34395130766552e-09
3290 8.34351077116935e-09
3291 8.34292368523393e-09
3292 8.34287128270716e-09
3293 8.34219537892977e-09
3294 8.34214652911669e-09
3295 8.34151947515238e-09
3296 8.34138802474627e-09
3297 8.3409972262416e-09
3298 8.34060909227219e-09
3299 8.34023161644382e-09
3300 8.34006197436565e-09
3301 8.33930524635207e-09
3302 8.33918711862225e-09
3303 8.33873770034188e-09
3304 8.33856095283636e-09
3305 8.33793567522889e-09
3306 8.33777047404283e-09
3307 8.33718249992899e-09
3308 8.33708746483808e-09
3309 8.33678548417538e-09
3310 8.33596036642348e-09
3311 8.33583335690946e-09
3312 8.33535196420598e-09
3313 8.33501978547702e-09
3314 8.33476931916266e-09
3315 8.33437230340905e-09
3316 8.33410584988314e-09
3317 8.33346902595622e-09
3318 8.33324165228078e-09
3319 8.33290414448129e-09
3320 8.3323365984711e-09
3321 8.33231705854587e-09
3322 8.33152213886024e-09
3323 8.33128677157902e-09
3324 8.3308622222944e-09
3325 8.33084090601233e-09
3326 8.33013125145499e-09
3327 8.32999536015677e-09
3328 8.32951219109646e-09
3329 8.32924396121371e-09
3330 8.32867375066826e-09
3331 8.32838331632502e-09
3332 8.32805557848815e-09
3333 8.32770385983395e-09
3334 8.32772606429444e-09
3335 8.32702884423497e-09
3336 8.32687785390362e-09
3337 8.32634672320864e-09
3338 8.32615754120525e-09
3339 8.32538837869379e-09
3340 8.32524715832506e-09
3341 8.32492652591554e-09
3342 8.32454016830297e-09
3343 8.32414670526305e-09
3344 8.32424351671079e-09
3345 8.32360758096229e-09
3346 8.32331270572695e-09
3347 8.32294499986119e-09
3348 8.32240054648992e-09
3349 8.32192981192748e-09
3350 8.321863198546e-09
3351 8.32141733297931e-09
3352 8.3210158763336e-09
3353 8.32058599797847e-09
3354 8.32016056051543e-09
3355 8.31997049033362e-09
3356 8.3194420241739e-09
3357 8.31913027354858e-09
3358 8.3186124655299e-09
3359 8.31850854865479e-09
3360 8.31818169899634e-09
3361 8.31781754584426e-09
3362 8.31754665142626e-09
3363 8.31735569306602e-09
3364 8.31687430036254e-09
3365 8.31649682453417e-09
3366 8.31637336773383e-09
3367 8.31592661398872e-09
3368 8.31544166857157e-09
3369 8.3153937069369e-09
3370 8.31524094024871e-09
3371 8.31446911320199e-09
3372 8.31435276182901e-09
3373 8.3137479123252e-09
3374 8.31349655783242e-09
3375 8.31325319694542e-09
3376 8.31274782342462e-09
3377 8.31228685882479e-09
3378 8.31202662254782e-09
3379 8.31179036708818e-09
3380 8.31145996471605e-09
3381 8.3108897541706e-09
3382 8.31053181826746e-09
3383 8.3100557546345e-09
3384 8.31007440638132e-09
3385 8.30973068133289e-09
3386 8.30916402350113e-09
3387 8.30891888625729e-09
3388 8.30858404299306e-09
3389 8.30818258634736e-09
3390 8.30772339810437e-09
3391 8.30724289357931e-09
3392 8.30706969878747e-09
3393 8.30666291307125e-09
3394 8.30647106653259e-09
3395 8.30603053003642e-09
3396 8.30537238982743e-09
3397 8.30525426209761e-09
3398 8.30489277348079e-09
3399 8.30453750211291e-09
3400 8.30434210286057e-09
3401 8.30357027581385e-09
3402 8.3035542886023e-09
3403 8.30310131760825e-09
3404 8.30278601426926e-09
3405 8.30245294736187e-09
3406 8.30198398915627e-09
3407 8.30154611719536e-09
3408 8.30155588715797e-09
3409 8.30132318441201e-09
3410 8.30062507617413e-09
3411 8.30032576004669e-09
3412 8.29994828421832e-09
3413 8.29964186266352e-09
3414 8.29917912170686e-09
3415 8.29881852126846e-09
3416 8.29887092379522e-09
3417 8.29846502625742e-09
3418 8.29815505198894e-09
3419 8.29780510969158e-09
3420 8.29730506524129e-09
3421 8.29697466286916e-09
3422 8.29672774926848e-09
3423 8.29638757693374e-09
3424 8.29615931507988e-09
3425 8.29574275940104e-09
3426 8.29556778825236e-09
3427 8.29482882380717e-09
3428 8.29441670902042e-09
3429 8.29434121385475e-09
3430 8.29357826859223e-09
3431 8.29347257536028e-09
3432 8.29350010889129e-09
3433 8.29296631366105e-09
3434 8.2926483457868e-09
3435 8.29192448037475e-09
3436 8.29155233361689e-09
3437 8.29144575220653e-09
3438 8.290998110283e-09
3439 8.29088264708844e-09
3440 8.2902138487384e-09
3441 8.28991986168148e-09
3442 8.28975199596016e-09
3443 8.28943846897801e-09
3444 8.2893301112108e-09
3445 8.28932300578344e-09
3446 8.28845614364582e-09
3447 8.28844459732636e-09
3448 8.28782198425415e-09
3449 8.28759549875713e-09
3450 8.28699597832383e-09
3451 8.28675261743683e-09
3452 8.28668778041219e-09
3453 8.2861291161862e-09
3454 8.28551627307661e-09
3455 8.28519297613184e-09
3456 8.28508373018622e-09
3457 8.28514057360508e-09
3458 8.28439805644621e-09
3459 8.28434743027628e-09
3460 8.28366975014205e-09
3461 8.28366797378521e-09
3462 8.28323720725166e-09
3463 8.28285173781751e-09
3464 8.28232860072831e-09
3465 8.28209678616076e-09
3466 8.28185164891693e-09
3467 8.28123170037998e-09
3468 8.28139601338762e-09
3469 8.28105850558813e-09
3470 8.28048829504269e-09
3471 8.28041990530437e-09
3472 8.27987367557625e-09
3473 8.27993407170879e-09
3474 8.27938340108858e-09
3475 8.27902013611492e-09
3476 8.27872703723642e-09
3477 8.2785502897309e-09
3478 8.27786461599089e-09
3479 8.27762747235283e-09
3480 8.27726154284392e-09
3481 8.27668689140637e-09
3482 8.27669133229847e-09
3483 8.27628632293909e-09
3484 8.2760571729068e-09
3485 8.27526491775643e-09
3486 8.27557045113281e-09
3487 8.27502510958311e-09
3488 8.27479684772925e-09
3489 8.27436341666044e-09
3490 8.27378165979553e-09
3491 8.27391044566639e-09
3492 8.27348056731125e-09
3493 8.27303114903088e-09
3494 8.27296453564941e-09
3495 8.27257995439368e-09
3496 8.2724458394523e-09
3497 8.27207013998077e-09
3498 8.27171664496973e-09
3499 8.27106738654493e-09
3500 8.27079471577008e-09
3501 8.27048829421528e-09
3502 8.27018720173101e-09
3503 8.27005042225437e-09
3504 8.26938695297486e-09
3505 8.2692404035356e-09
3506 8.26876789261632e-09
3507 8.26840995671319e-09
3508 8.26834689604539e-09
3509 8.26766477501906e-09
3510 8.26762391881175e-09
3511 8.26722601487972e-09
3512 8.26690271793495e-09
3513 8.26638135720259e-09
3514 8.26621526783811e-09
3515 8.26591772806751e-09
3516 8.26538304465885e-09
3517 8.26514501284237e-09
3518 8.26494428451952e-09
3519 8.26441581835979e-09
3520 8.26412982490865e-09
3521 8.26404900067246e-09
3522 8.26347346105649e-09
3523 8.26320523117374e-09
3524 8.2628064390633e-09
3525 8.26247070762065e-09
3526 8.2619591168509e-09
3527 8.26215185156798e-09
3528 8.26165091893927e-09
3529 8.26123702779569e-09
3530 8.26074941784327e-09
3531 8.2604261208985e-09
3532 8.26039148194013e-09
3533 8.26000867704124e-09
3534 8.25956369965297e-09
3535 8.25926349534711e-09
3536 8.25903700985009e-09
3537 8.25874124643633e-09
3538 8.25851298458247e-09
3539 8.25786550251451e-09
3540 8.25751556021714e-09
3541 8.2572428894423e-09
3542 8.25715851249242e-09
3543 8.25667800796737e-09
3544 8.25640356083568e-09
3545 8.25605805943042e-09
3546 8.25590085185013e-09
3547 8.25561130568531e-09
3548 8.25506152324351e-09
3549 8.25483414956807e-09
3550 8.25455881425796e-09
3551 8.25423196459951e-09
3552 8.25373813739816e-09
3553 8.25309331986546e-09
3554 8.25322121755789e-09
3555 8.25271762039392e-09
3556 8.25252399749843e-09
3557 8.25200707765816e-09
3558 8.25163581907873e-09
3559 8.25133472659445e-09
3560 8.25112866920108e-09
3561 8.25086754474569e-09
3562 8.25044477181791e-09
3563 8.25015078476099e-09
3564 8.24975199265054e-09
3565 8.24944113020365e-09
3566 8.24885137973297e-09
3567 8.24882029348828e-09
3568 8.24841706048574e-09
3569 8.2481674823498e-09
3570 8.24783619179925e-09
3571 8.24750667760554e-09
3572 8.2469231443838e-09
3573 8.24677748312297e-09
3574 8.24648616060131e-09
3575 8.24624368789273e-09
3576 8.24584045489019e-09
3577 8.24558110679163e-09
3578 8.24523738174321e-09
3579 8.24486612316377e-09
3580 8.24462986770413e-09
3581 8.24438206592504e-09
3582 8.24414225775172e-09
3583 8.24385448794374e-09
3584 8.24346457761749e-09
3585 8.24301782387238e-09
3586 8.2430728909344e-09
3587 8.24252932574154e-09
3588 8.24224422046882e-09
3589 8.24231793927765e-09
3590 8.2419369107356e-09
3591 8.24104784413748e-09
3592 8.24102297514173e-09
3593 8.24049362080359e-09
3594 8.24039414482058e-09
3595 8.23991896936604e-09
3596 8.2395290590398e-09
3597 8.23926527004915e-09
3598 8.23896684210013e-09
3599 8.23843482322673e-09
3600 8.23785750725392e-09
3601 8.23776602487669e-09
3602 8.23754930934228e-09
3603 8.23714163544764e-09
3604 8.23694179530321e-09
3605 8.23684853656914e-09
3606 8.23617707368385e-09
3607 8.23606782773822e-09
3608 8.23557311235845e-09
3609 8.23510681868811e-09
3610 8.23483059519958e-09
3611 8.23463253141199e-09
3612 8.23422130480367e-09
3613 8.23408718986229e-09
3614 8.23378698555643e-09
3615 8.2333571072013e-09
3616 8.23305157382492e-09
3617 8.23287749085466e-09
3618 8.23230994484447e-09
3619 8.23240231540012e-09
3620 8.23186141474253e-09
3621 8.23143153638739e-09
3622 8.23131873772809e-09
3623 8.23087820123192e-09
3624 8.23047674458621e-09
3625 8.23029377983175e-09
3626 8.22983547976719e-09
3627 8.22957524349022e-09
3628 8.22924217658283e-09
3629 8.22899171026847e-09
3630 8.22866752514528e-09
3631 8.2283166946695e-09
3632 8.22813461809346e-09
3633 8.22772694419882e-09
3634 8.22753953855226e-09
3635 8.22708923209348e-09
3636 8.22678902778762e-09
3637 8.22629253605101e-09
3638 8.22627033159051e-09
3639 8.22548695822434e-09
3640 8.2254763000833e-09
3641 8.22522139287685e-09
3642 8.2245925625557e-09
3643 8.2246014443399e-09
3644 8.2244122623365e-09
3645 8.22388823706888e-09
3646 8.2236955023518e-09
3647 8.22346812867636e-09
3648 8.22307200110117e-09
3649 8.22295476154977e-09
3650 8.2224689279542e-09
3651 8.22220247442829e-09
3652 8.22195556082761e-09
3653 8.22176815518105e-09
3654 8.22113754850307e-09
3655 8.22107359965685e-09
3656 8.22082224516407e-09
3657 8.22028223268489e-09
3658 8.22012591328303e-09
3659 8.21964452057955e-09
3660 8.21905832282255e-09
3661 8.21889489799332e-09
3662 8.21896062319638e-09
3663 8.21894730052009e-09
3664 8.21817458529495e-09
3665 8.21776335868662e-09
3666 8.21759904567898e-09
3667 8.21713630472232e-09
3668 8.21660606220576e-09
3669 8.21682277774016e-09
3670 8.21642132109446e-09
3671 8.2159905545609e-09
3672 8.21590173671893e-09
3673 8.2155215963553e-09
3674 8.21512724513696e-09
3675 8.21484924529159e-09
3676 8.21454815280731e-09
3677 8.21422574404096e-09
3678 8.21404810835702e-09
3679 8.21370704784385e-09
3680 8.21328605127292e-09
3681 8.21263235195602e-09
3682 8.21262791106392e-09
3683 8.21247603255415e-09
3684 8.21203549605798e-09
3685 8.21144308105204e-09
3686 8.21155943242502e-09
3687 8.21086754143607e-09
3688 8.21069878753633e-09
3689 8.21045897936301e-09
3690 8.2099491649501e-09
3691 8.20971557402572e-09
3692 8.20938339529675e-09
3693 8.2091915487581e-09
3694 8.20850765137493e-09
3695 8.20851386862387e-09
3696 8.20822521063747e-09
3697 8.20769674447774e-09
3698 8.20755374775217e-09
3699 8.20726775430103e-09
3700 8.20705992055082e-09
3701 8.2065767514905e-09
3702 8.20625789543783e-09
3703 8.20596746109459e-09
3704 8.2056885730708e-09
3705 8.20549139746163e-09
3706 8.20515388966214e-09
3707 8.20479950647268e-09
3708 8.20454015837413e-09
3709 8.20427370484822e-09
3710 8.20381895749733e-09
3711 8.20338375007168e-09
3712 8.20335621654067e-09
3713 8.2030737758032e-09
3714 8.20282686220253e-09
3715 8.20254264510822e-09
3716 8.20218915009718e-09
3717 8.20172196824842e-09
3718 8.2015585434192e-09
3719 8.20124945732914e-09
3720 8.20101409004792e-09
3721 8.20074674834359e-09
3722 8.20021295311335e-09
3723 8.19996959222635e-09
3724 8.19972889587461e-09
3725 8.19927148398847e-09
3726 8.19907874927139e-09
3727 8.19877943314395e-09
3728 8.19862666645577e-09
3729 8.19825007880581e-09
3730 8.19799872431304e-09
3731 8.19755552328161e-09
3732 8.19700485266139e-09
3733 8.19684942143795e-09
3734 8.19683787511849e-09
3735 8.19619572212105e-09
3736 8.19587864242521e-09
3737 8.19573653387806e-09
3738 8.1956210706835e-09
3739 8.19534040630288e-09
3740 8.19494694326295e-09
3741 8.19458723100297e-09
3742 8.19437318000382e-09
3743 8.19380741035047e-09
3744 8.19352674596985e-09
3745 8.19339973645583e-09
3746 8.1930062734159e-09
3747 8.19263501483647e-09
3748 8.19240053573367e-09
3749 8.19212786495882e-09
3750 8.19165890675322e-09
3751 8.19143597396987e-09
3752 8.19094658766062e-09
3753 8.19079204461559e-09
3754 8.19037992982885e-09
3755 8.19037992982885e-09
3756 8.18995626872265e-09
3757 8.18954859482801e-09
3758 8.18930434576259e-09
3759 8.18890999454425e-09
3760 8.18851741968274e-09
3761 8.18832912585776e-09
3762 8.18800049984247e-09
3763 8.18765766297247e-09
3764 8.18725354179151e-09
3765 8.18724377182889e-09
3766 8.18686363146526e-09
3767 8.18632095445082e-09
3768 8.18605272456807e-09
3769 8.18587242434887e-09
3770 8.18534484636757e-09
3771 8.18520362599884e-09
3772 8.18491674436927e-09
3773 8.18440604177795e-09
3774 8.18437140281958e-09
3775 8.18376122424525e-09
3776 8.18342460462418e-09
3777 8.18373280253581e-09
3778 8.1831910136998e-09
3779 8.18262257951119e-09
3780 8.18261014501331e-09
3781 8.18236500776948e-09
3782 8.18179302086719e-09
3783 8.18133027991053e-09
3784 8.18116419054604e-09
3785 8.18108158995301e-09
3786 8.18069079144834e-09
3787 8.18023426774062e-09
3788 8.18006462566245e-09
3789 8.17971201882983e-09
3790 8.17955037035745e-09
3791 8.17893130999892e-09
3792 8.17874390435236e-09
3793 8.17859291402101e-09
3794 8.17828738064463e-09
3795 8.17794632013147e-09
3796 8.17781398154693e-09
3797 8.17739387315441e-09
3798 8.17742851211278e-09
3799 8.1769018223099e-09
3800 8.17664158603293e-09
3801 8.17640888328697e-09
3802 8.17595147140082e-09
3803 8.17559264731926e-09
3804 8.17535195096752e-09
3805 8.17493805982394e-09
3806 8.17451972778827e-09
3807 8.17414758103041e-09
3808 8.17415735099303e-09
3809 8.17374257167103e-09
3810 8.17344680825727e-09
3811 8.17324963264809e-09
3812 8.17276024633884e-09
3813 8.17257994611964e-09
3814 8.17231438077215e-09
3815 8.17228862359798e-09
3816 8.17149636844761e-09
3817 8.17151768472968e-09
3818 8.17088796623011e-09
3819 8.17067213887412e-09
3820 8.17046963419443e-09
3821 8.17010814557761e-09
3822 8.16972089978663e-09
3823 8.16957790306105e-09
3824 8.16924572433209e-09
3825 8.16902900879768e-09
3826 8.16865597386141e-09
3827 8.16834599959293e-09
3828 8.16772427469914e-09
3829 8.16771006384442e-09
3830 8.16724554653092e-09
3831 8.16682188542472e-09
3832 8.16655365554197e-09
3833 8.16655809643407e-09
3834 8.16631740008233e-09
3835 8.16548784143833e-09
3836 8.16537948367113e-09
3837 8.1650082250917e-09
3838 8.1650037841996e-09
3839 8.16459433394812e-09
3840 8.16430745231855e-09
3841 8.16400547165586e-09
3842 8.1638811266771e-09
3843 8.1631599258003e-09
3844 8.16300538275527e-09
3845 8.16264034142478e-09
3846 8.16222911481645e-09
3847 8.16240675050039e-09
3848 8.16198575392946e-09
3849 8.16155409921748e-09
3850 8.16156031646642e-09
3851 8.16108869372556e-09
3852 8.16076362042395e-09
3853 8.16041989537553e-09
3854 8.16003975501189e-09
3855 8.15976441970179e-09
3856 8.15962852840357e-09
3857 8.15907785778336e-09
3858 8.15882916782584e-09
3859 8.15840550671965e-09
3860 8.15819856114786e-09
3861 8.15778378182586e-09
3862 8.15749867655313e-09
3863 8.15710343715637e-09
3864 8.15695422318186e-09
3865 8.15659362274346e-09
3866 8.15645151419631e-09
3867 8.15596390424389e-09
3868 8.15540879273158e-09
3869 8.15531464581909e-09
3870 8.15498957251748e-09
3871 8.15442469104255e-09
3872 8.15446821178512e-09
3873 8.15405787335521e-09
3874 8.15381806518189e-09
3875 8.15336775872311e-09
3876 8.15302847456678e-09
3877 8.1528082063187e-09
3878 8.15267409137732e-09
3879 8.15204082016407e-09
3880 8.15217315874861e-09
3881 8.15183032187861e-09
3882 8.15136935727878e-09
3883 8.15150791311225e-09
3884 8.15085954286587e-09
3885 8.15054779224056e-09
3886 8.15023692979366e-09
3887 8.15011080845807e-09
3888 8.14979195240539e-09
3889 8.14936651494236e-09
3890 8.14917111569002e-09
3891 8.14891532030515e-09
3892 8.14864176135188e-09
3893 8.14827938455664e-09
3894 8.14786371705623e-09
3895 8.14785217073677e-09
3896 8.147528873792e-09
3897 8.14721712316668e-09
3898 8.14673306592795e-09
3899 8.14653056124826e-09
3900 8.14591416542498e-09
3901 8.14583067665353e-09
3902 8.14542122640205e-09
3903 8.14530043413697e-09
3904 8.14503930968158e-09
3905 8.14462275400274e-09
3906 8.14466982745898e-09
3907 8.14425860085066e-09
3908 8.14373368740462e-09
3909 8.14359868428483e-09
3910 8.14340772592459e-09
3911 8.14313061425764e-09
3912 8.14274159210981e-09
3913 8.14227441026105e-09
3914 8.14202216758986e-09
3915 8.14191380982265e-09
3916 8.14131784210304e-09
3917 8.14107004032394e-09
3918 8.14075917787704e-09
3919 8.14026712703253e-09
3920 8.14037903751341e-09
3921 8.13974310176491e-09
3922 8.13948464184477e-09
3923 8.13938605404019e-09
3924 8.13901479546075e-09
3925 8.13860090431717e-09
3926 8.138232310273e-09
3927 8.13776157571056e-09
3928 8.13777489838685e-09
3929 8.13731926285755e-09
3930 8.13685208100878e-09
3931 8.13670553156953e-09
3932 8.13652611952875e-09
3933 8.13600209426113e-09
3934 8.13587597292553e-09
3935 8.13555356415918e-09
3936 8.13534484223055e-09
3937 8.13500378171739e-09
3938 8.13482348149819e-09
3939 8.13433231883209e-09
3940 8.13425238277432e-09
3941 8.13373013386354e-09
3942 8.13345302219659e-09
3943 8.1337736546061e-09
3944 8.13295653045998e-09
3945 8.13267941879303e-09
3946 8.13268741239881e-09
3947 8.13203548943875e-09
3948 8.13185696557639e-09
3949 8.13165801361038e-09
3950 8.13118994358319e-09
3951 8.13091549645151e-09
3952 8.13075473615754e-09
3953 8.13053535608788e-09
3954 8.13005129884914e-09
3955 8.12982303699528e-09
3956 8.12917289039206e-09
3957 8.12912581693581e-09
3958 8.12871991939801e-09
3959 8.12860623256029e-09
3960 8.12834599628331e-09
3961 8.12784684001144e-09
3962 8.12766298707857e-09
3963 8.12731038024594e-09
3964 8.12690981177866e-09
3965 8.12643730085938e-09
3966 8.12650657877612e-09
3967 8.12590439380756e-09
3968 8.12576761433093e-09
3969 8.12561928853484e-09
3970 8.12529243887639e-09
3971 8.12492384483221e-09
3972 8.12451084186705e-09
3973 8.12424083562746e-09
3974 8.12373812664191e-09
3975 8.12351963475066e-09
3976 8.12325939847369e-09
3977 8.12314127074387e-09
3978 8.12267675343037e-09
3979 8.1223623382698e-09
3980 8.12212608281015e-09
3981 8.12207900935391e-09
3982 8.12167844088663e-09
3983 8.12108869041595e-09
3984 8.12099010261136e-09
3985 8.12059752774985e-09
3986 8.1203879176428e-09
3987 8.12030087615767e-09
3988 8.11976885728427e-09
3989 8.11958589252981e-09
3990 8.11948019929787e-09
3991 8.11901923469804e-09
3992 8.11848455128938e-09
3993 8.11814793166832e-09
3994 8.11805911382635e-09
3995 8.11758038565813e-09
3996 8.11757505658761e-09
3997 8.11701017511268e-09
3998 8.11687073110079e-09
3999 8.11667710820529e-09
4000 8.11585643134549e-09
4001 8.11584932591813e-09
4002 8.11557399060803e-09
4003 8.11488209961908e-09
4004 8.1150544062325e-09
4005 8.11484834883913e-09
4006 8.11416800416964e-09
4007 8.11403921829879e-09
4008 8.11385980625801e-09
4009 8.11362355079837e-09
4010 8.11340683526396e-09
4011 8.11296629876779e-09
4012 8.11260569832939e-09
4013 8.11224154517731e-09
4014 8.11195022265565e-09
4015 8.11164646563611e-09
4016 8.11122990995727e-09
4017 8.11094214014929e-09
4018 8.11086042773468e-09
4019 8.1102617954798e-09
4020 8.10985500976358e-09
4021 8.10952283103461e-09
4022 8.10945888218839e-09
4023 8.10907074821898e-09
4024 8.10897926584175e-09
4025 8.10853073573981e-09
4026 8.10823674868288e-09
4027 8.10792322170073e-09
4028 8.10745959256565e-09
4029 8.10732903033795e-09
4030 8.10703681963787e-09
4031 8.10676858975512e-09
4032 8.10649680715869e-09
4033 8.10626765712641e-09
4034 8.10561129327425e-09
4035 8.10530398354103e-09
4036 8.10508193893611e-09
4037 8.10491318503637e-09
4038 8.10454015010009e-09
4039 8.10416445062856e-09
4040 8.10418132601853e-09
4041 8.10364397807462e-09
4042 8.10367595249772e-09
4043 8.10340861079339e-09
4044 8.10306310938813e-09
4045 8.10234634940343e-09
4046 8.10226907788092e-09
4047 8.10190936562094e-09
4048 8.10146527641109e-09
4049 8.10130362793871e-09
4050 8.10078581992002e-09
4051 8.10051226096675e-09
4052 8.10008327079004e-09
4053 8.09995448491918e-09
4054 8.09979994187415e-09
4055 8.0993016737807e-09
4056 8.09914446620041e-09
4057 8.09886913089031e-09
4058 8.09861955275437e-09
4059 8.09841615989626e-09
4060 8.09788502920128e-09
4061 8.09798272882745e-09
4062 8.09726685702117e-09
4063 8.09729883144428e-09
4064 8.09701283799313e-09
4065 8.09665312573316e-09
4066 8.09635114507046e-09
4067 8.09596567563631e-09
4068 8.09560063430581e-09
4069 8.0954514203313e-09
4070 8.09512812338653e-09
4071 8.09485989350378e-09
4072 8.09480038554966e-09
4073 8.09425415582155e-09
4074 8.09407385560235e-09
4075 8.09368749798978e-09
4076 8.09357647568731e-09
4077 8.09321143435682e-09
4078 8.09299915971451e-09
4079 8.09241296195751e-09
4080 8.09210209951061e-09
4081 8.09192624018351e-09
4082 8.09129296897027e-09
4083 8.09119704570094e-09
4084 8.09101230458964e-09
4085 8.09049804928463e-09
4086 8.09049893746305e-09
4087 8.090018432938e-09
4088 8.08964362164488e-09
4089 8.08929190299068e-09
4090 8.0889863696143e-09
4091 8.08876698954464e-09
4092 8.08797206985901e-09
4093 8.08803601870522e-09
4094 8.08776512428722e-09
4095 8.08742495195247e-09
4096 8.08721534184542e-09
4097 8.08673306096352e-09
4098 8.08654654349539e-09
4099 8.08594613488367e-09
4100 8.08601452462199e-09
4101 8.08554379005955e-09
4102 8.08560240983525e-09
4103 8.08505973282081e-09
4104 8.08475597580127e-09
4105 8.08440070443339e-09
4106 8.08435274279873e-09
4107 8.08398326057613e-09
4108 8.08347255798481e-09
4109 8.08305689048439e-09
4110 8.08291922282933e-09
4111 8.0827193826849e-09
4112 8.08223887815984e-09
4113 8.08216249481575e-09
4114 8.08182498701626e-09
4115 8.08168199029069e-09
4116 8.08142619490582e-09
4117 8.08106381811058e-09
4118 8.08063038704177e-09
4119 8.08041189515052e-09
4120 8.08022804221764e-09
4121 8.07975109040626e-09
4122 8.07945177427882e-09
4123 8.07928124402224e-09
4124 8.07854849682599e-09
4125 8.07845523809192e-09
4126 8.07848365980135e-09
4127 8.07796585178266e-09
4128 8.07777134070875e-09
4129 8.07745603736976e-09
4130 8.0767863508413e-09
4131 8.07695688109789e-09
4132 8.07659716883791e-09
4133 8.07618150133749e-09
4134 8.0759070542058e-09
4135 8.07571165495347e-09
4136 8.0754425368923e-09
4137 8.07528266477675e-09
4138 8.07495226240462e-09
4139 8.07429589855246e-09
4140 8.07405342584389e-09
4141 8.07409605840803e-09
4142 8.07356403953463e-09
4143 8.07346189901637e-09
4144 8.07302491523387e-09
4145 8.07278599523897e-09
4146 8.07248312639786e-09
4147 8.07238187405801e-09
4148 8.07172551020585e-09
4149 8.07144662218207e-09
4150 8.07134714619906e-09
4151 8.0708257854667e-09
4152 8.07074851394418e-09
4153 8.07057531915234e-09
4154 8.07022093596288e-09
4155 8.06968447619738e-09
4156 8.06950239962134e-09
4157 8.06912847650665e-09
4158 8.06914712825346e-09
4159 8.06860178670377e-09
4160 8.06871014447097e-09
4161 8.06797917363156e-09
4162 8.06747735282443e-09
4163 8.0677962088771e-09
4164 8.06705191536139e-09
4165 8.06671174302664e-09
4166 8.06645328310651e-09
4167 8.06637689976242e-09
4168 8.0661193280207e-09
4169 8.06557487464943e-09
4170 8.06518674068002e-09
4171 8.06513256179642e-09
4172 8.06490341176413e-09
4173 8.06444777623483e-09
4174 8.0641893163147e-09
4175 8.06381716955684e-09
4176 8.06361555305557e-09
4177 8.06311728496212e-09
4178 8.06291300392559e-09
4179 8.06267408393069e-09
4180 8.06243161122211e-09
4181 8.06219091487037e-09
4182 8.06212696602415e-09
4183 8.06129651920173e-09
4184 8.06121658314396e-09
4185 8.06101851935637e-09
4186 8.060740519511e-09
4187 8.06038968903522e-09
4188 8.06031330569112e-09
4189 8.05989408547703e-09
4190 8.05947220072767e-09
4191 8.05916844370813e-09
4192 8.05885402854756e-09
4193 8.05882027776761e-09
4194 8.05844369011766e-09
4195 8.05798450187467e-09
4196 8.05809641235555e-09
4197 8.05734678976933e-09
4198 8.05703415096559e-09
4199 8.05716116047961e-09
4200 8.05659539082626e-09
4201 8.05609623455439e-09
4202 8.05599409403612e-09
4203 8.05576849671752e-09
4204 8.05546651605482e-09
4205 8.05540611992228e-09
4206 8.05505173673282e-09
4207 8.05447086804634e-09
4208 8.05428701511346e-09
4209 8.05371858092485e-09
4210 8.05389355207353e-09
4211 8.05350097721202e-09
4212 8.0532203128314e-09
4213 8.05276823001577e-09
4214 8.05241562318315e-09
4215 8.05211097798519e-09
4216 8.05185695895716e-09
4217 8.05163757888749e-09
4218 8.05131605829956e-09
4219 8.0507902566751e-09
4220 8.05082311927663e-09
4221 8.05051580954341e-09
4222 8.05019606531232e-09
4223 8.04996336256636e-09
4224 8.04959388034376e-09
4225 8.04934163767257e-09
4226 8.04914357388498e-09
4227 8.04857158698269e-09
4228 8.04853428348906e-09
4229 8.04813993227071e-09
4230 8.0479320985205e-09
4231 8.04748179206172e-09
4232 8.04737787518661e-09
4233 8.04724820113734e-09
4234 8.04663891074142e-09
4235 8.0465882845715e-09
4236 8.04628275119512e-09
4237 8.04595501335825e-09
4238 8.04564326273294e-09
4239 8.04557487299462e-09
4240 8.04520361441519e-09
4241 8.04471866899803e-09
4242 8.04465116743813e-09
4243 8.04426925071766e-09
4244 8.0437976279768e-09
4245 8.04366440121385e-09
4246 8.04348854188675e-09
4247 8.04310662516627e-09
4248 8.04294586487231e-09
4249 8.04300981371853e-09
4250 8.04262789699806e-09
4251 8.04203548199212e-09
4252 8.04168465151633e-09
4253 8.04170152690631e-09
4254 8.04134714371685e-09
4255 8.04100963591736e-09
4256 8.04062416648321e-09
4257 8.04021649258857e-09
4258 8.03993938092162e-09
4259 8.03981947683496e-09
4260 8.03955479966589e-09
4261 8.03932920234729e-09
4262 8.0392350554348e-09
4263 8.03867461485197e-09
4264 8.03813993144331e-09
4265 8.0383513179072e-09
4266 8.03764166334986e-09
4267 8.0372197786005e-09
4268 8.03710431540594e-09
4269 8.03671795779337e-09
4270 8.03655542114257e-09
4271 8.03640087809754e-09
4272 8.03611754918165e-09
4273 8.03569921714598e-09
4274 8.03551802874836e-09
4275 8.03485367129042e-09
4276 8.03471777999221e-09
4277 8.03448063635415e-09
4278 8.03403565896588e-09
4279 8.03374433644422e-09
4280 8.03339972321737e-09
4281 8.03326205556232e-09
4282 8.03288990880446e-09
4283 8.03235611357422e-09
4284 8.03193245246803e-09
4285 8.03156297024543e-09
4286 8.03178057395826e-09
4287 8.03115263181553e-09
4288 8.03106825486566e-09
4289 8.03065614007892e-09
4290 8.0304660698971e-09
4291 8.02999977622676e-09
4292 8.02970312463458e-09
4293 8.02937538679771e-09
4294 8.02879540628965e-09
4295 8.0290591952803e-09
4296 8.02870570026926e-09
4297 8.02852628822848e-09
4298 8.0280537773092e-09
4299 8.02767008423189e-09
4300 8.02763633345194e-09
4301 8.02718158610105e-09
4302 8.02703770119706e-09
4303 8.02677124767115e-09
4304 8.02652522224889e-09
4305 8.0260251777986e-09
4306 8.02594879445451e-09
4307 8.02523558718349e-09
4308 8.02509703135001e-09
4309 8.02504018793115e-09
4310 8.02456145976294e-09
4311 8.02443089753524e-09
4312 8.02422484014187e-09
4313 8.02411026512573e-09
4314 8.0234237032073e-09
4315 8.02325583748598e-09
4316 8.02284993994817e-09
4317 8.02260036181224e-09
4318 8.02261901355905e-09
4319 8.02210120554037e-09
4320 8.02186672643757e-09
4321 8.02177879677402e-09
4322 8.02132227306629e-09
4323 8.02102739783095e-09
4324 8.02055133419799e-09
4325 8.0203985675098e-09
4326 8.02006461242399e-09
4327 8.0199384910884e-09
4328 8.01948463191593e-09
4329 8.01944999295756e-09
4330 8.01919952664321e-09
4331 8.01855382093208e-09
4332 8.01841970599071e-09
4333 8.01812838346905e-09
4334 8.01762389812666e-09
4335 8.01736099731443e-09
4336 8.01702881858546e-09
4337 8.01721711241044e-09
4338 8.01657940030509e-09
4339 8.01605271050221e-09
4340 8.01601007793806e-09
4341 8.01575694708845e-09
4342 8.01524890903238e-09
4343 8.01536526040536e-09
4344 8.01495048108336e-09
4345 8.01443889031361e-09
4346 8.01445398934675e-09
4347 8.014113817012e-09
4348 8.01333133182425e-09
4349 8.01346988765772e-09
4350 8.01331090372059e-09
4351 8.01265542804686e-09
4352 8.0125222012839e-09
4353 8.01218114077074e-09
4354 8.01183297483021e-09
4355 8.01169086628306e-09
4356 8.01131694316837e-09
4357 8.01116151194492e-09
4358 8.0108764066722e-09
4359 8.01056376786846e-09
4360 8.01040922482343e-09
4361 8.01019606200271e-09
4362 8.0099118449084e-09
4363 8.00934962796873e-09
4364 8.00890287422362e-09
4365 8.00880162188378e-09
4366 8.00855559646152e-09
4367 8.00813193535532e-09
4368 8.00798449773765e-09
4369 8.00752530949467e-09
4370 8.00716470905627e-09
4371 8.00692667723979e-09
4372 8.00664601285916e-09
4373 8.00632093955755e-09
4374 8.00598876082859e-09
4375 8.00595501004864e-09
4376 8.00567256931117e-09
4377 8.00515387311407e-09
4378 8.00490340679971e-09
4379 8.00469379669266e-09
4380 8.00467958583795e-09
4381 8.00411736889828e-09
4382 8.00390598243439e-09
4383 8.0034689986519e-09
4384 8.00321942051596e-09
4385 8.00275490320246e-09
4386 8.00262522915318e-09
4387 8.00236588105463e-09
4388 8.00213939555761e-09
4389 8.00194488448369e-09
4390 8.00182409221861e-09
4391 8.00105848242083e-09
4392 8.00090038666212e-09
4393 8.00062149863834e-09
4394 8.00024935188048e-09
4395 7.99977772913962e-09
4396 7.99992250222203e-09
4397 7.99948107754744e-09
4398 7.99894017688985e-09
4399 7.99897392766979e-09
4400 7.99863908440557e-09
4401 7.99854937838518e-09
4402 7.99777666316004e-09
4403 7.99791433081509e-09
4404 7.99774912962903e-09
4405 7.99745336621527e-09
4406 7.99704569232063e-09
4407 7.99660693218129e-09
4408 7.99621346914137e-09
4409 7.99612998036991e-09
4410 7.99600918810484e-09
4411 7.99560329056703e-09
4412 7.99534127793322e-09
4413 7.99483412805557e-09
4414 7.99480925905982e-09
4415 7.99444155319406e-09
4416 7.99412891439033e-09
4417 7.9937576558109e-09
4418 7.99372212867411e-09
4419 7.99345922786188e-09
4420 7.99310750920768e-09
4421 7.99281707486443e-09
4422 7.99278243590607e-09
4423 7.99234545212357e-09
4424 7.99203014878458e-09
4425 7.99165889020514e-09
4426 7.99131782969198e-09
4427 7.99106025795027e-09
4428 7.99088617498001e-09
4429 7.99035060339293e-09
4430 7.99026533826463e-09
4431 7.9899482585688e-09
4432 7.98960275716354e-09
4433 7.98960897441248e-09
4434 7.98917643152208e-09
4435 7.98889665531988e-09
4436 7.98859645101402e-09
4437 7.98851029770731e-09
4438 7.98796762069287e-09
4439 7.98786192746093e-09
4440 7.98725352524343e-09
4441 7.98720822814403e-09
4442 7.98687516123664e-09
4443 7.98648880362407e-09
4444 7.98623656095288e-09
4445 7.98589017136919e-09
4446 7.98558996706333e-09
4447 7.98555976899706e-09
4448 7.9851467660319e-09
4449 7.98485189079656e-09
4450 7.98457566730804e-09
4451 7.98404098389938e-09
4452 7.98413601899028e-09
4453 7.98382959743549e-09
4454 7.983248728749e-09
4455 7.98310573202343e-09
4456 7.9827886523276e-09
4457 7.9826580900999e-09
4458 7.98235433308037e-09
4459 7.98223531717213e-09
4460 7.98188537487476e-09
4461 7.9816526721288e-09
4462 7.9811881548153e-09
4463 7.98127075540833e-09
4464 7.98066324136926e-09
4465 7.98045185490537e-09
4466 7.98031862814241e-09
4467 7.9799979957329e-09
4468 7.97970844956808e-09
4469 7.97930965745763e-09
4470 7.97891530623929e-09
4471 7.9786275364313e-09
4472 7.9783477602291e-09
4473 7.97809285302264e-09
4474 7.97783439310251e-09
4475 7.9775723804687e-09
4476 7.97722332634976e-09
4477 7.9770536842716e-09
4478 7.97656785067602e-09
4479 7.97666555030219e-09
4480 7.97616728220873e-09
4481 7.97590082868282e-09
4482 7.97571786392837e-09
4483 7.97518939776864e-09
4484 7.9747994874424e-09
4485 7.97468580060468e-09
4486 7.97453658663017e-09
4487 7.97407917474402e-09
4488 7.97363242099891e-09
4489 7.9737354496956e-09
4490 7.97305865773978e-09
4491 7.97298937982305e-09
4492 7.97268295826825e-09
4493 7.9724058466013e-09
4494 7.97189869672366e-09
4495 7.97186228140845e-09
4496 7.97161892052145e-09
4497 7.97129473539826e-09
4498 7.97126809004567e-09
4499 7.97088794968204e-09
4500 7.97069432678654e-09
4501 7.97028842924874e-09
4502 7.97016674880524e-09
4503 7.96955923476617e-09
4504 7.96955124116039e-09
4505 7.96913290912471e-09
4506 7.96880605946626e-09
4507 7.96877142050789e-09
4508 7.96818255821563e-09
4509 7.96776511435837e-09
4510 7.96798538260646e-09
4511 7.96745780462516e-09
4512 7.96742671838047e-09
4513 7.96690891036178e-09
4514 7.96666999036688e-09
4515 7.96644794576196e-09
4516 7.9659585594527e-09
4517 7.96577648287666e-09
4518 7.96536614444676e-09
4519 7.96519827872544e-09
4520 7.96468313524201e-09
4521 7.96469379338305e-09
4522 7.96436783190302e-09
4523 7.96410848380447e-09
4524 7.9636413019557e-09
4525 7.96358534671526e-09
4526 7.96320964724373e-09
4527 7.96308530226497e-09
4528 7.96245736012224e-09
4529 7.96209054243491e-09
4530 7.96200527730662e-09
4531 7.96154253634995e-09
4532 7.96143595493959e-09
4533 7.96092347599142e-09
4534 7.96082577636525e-09
4535 7.96066945696339e-09
4536 7.96031862648761e-09
4537 7.96006549563799e-09
4538 7.95947663334573e-09
4539 7.95924393059977e-09
4540 7.95936028197275e-09
4541 7.95884247395406e-09
4542 7.95832466593538e-09
4543 7.95850230161932e-09
4544 7.95805377151737e-09
4545 7.95764343308747e-09
4546 7.95756260885128e-09
4547 7.95697729927269e-09
4548 7.95694088395749e-09
4549 7.95650034746131e-09
4550 7.95642574047406e-09
4551 7.95607668635512e-09
4552 7.95572496770092e-09
4553 7.9554673959592e-09
4554 7.95484922377909e-09
4555 7.95482346660492e-09
4556 7.95456589486321e-09
4557 7.95438115375191e-09
4558 7.95377097517758e-09
4559 7.95395749264571e-09
4560 7.95348853444011e-09
4561 7.95321231095159e-09
4562 7.95298848998982e-09
4563 7.9524102858386e-09
4564 7.95235965966867e-09
4565 7.95191823499408e-09
4566 7.95181875901108e-09
4567 7.95165266964659e-09
4568 7.95122634400514e-09
4569 7.95111976259477e-09
4570 7.95088173077829e-09
4571 7.95049270863046e-09
4572 7.95002730313854e-09
4573 7.94962051742232e-09
4574 7.94948551430252e-09
4575 7.94915244739514e-09
4576 7.948965929927e-09
4577 7.94849963625666e-09
4578 7.94830867789642e-09
4579 7.94788590496864e-09
4580 7.94800403269846e-09
4581 7.94759191791172e-09
4582 7.94735832698734e-09
4583 7.94694887673586e-09
4584 7.94683430171972e-09
4585 7.94643728596611e-09
4586 7.94607402099246e-09
4587 7.94596743958209e-09
4588 7.94550025773333e-09
4589 7.94512899915389e-09
4590 7.94466803455407e-09
4591 7.94473375975713e-09
4592 7.94432342132723e-09
4593 7.94425947248101e-09
4594 7.94373899992706e-09
4595 7.94362531308934e-09
4596 7.94317500663055e-09
4597 7.9429742783077e-09
4598 7.94277887905537e-09
4599 7.9427353583128e-09
4600 7.94214116695002e-09
4601 7.94185961439098e-09
4602 7.94167664963652e-09
4603 7.9412521003519e-09
4604 7.9411934805762e-09
4605 7.94063215181495e-09
4606 7.94054066943772e-09
4607 7.93995447168072e-09
4608 7.93996957071386e-09
4609 7.93959475942074e-09
4610 7.93937005028056e-09
4611 7.93886822947343e-09
4612 7.93866572479374e-09
4613 7.93833176970793e-09
4614 7.93846854918456e-09
4615 7.93790100317437e-09
4616 7.93760257522536e-09
4617 7.93755816630437e-09
4618 7.93717713776232e-09
4619 7.93703680557201e-09
4620 7.93681032007498e-09
4621 7.93639287621772e-09
4622 7.93610155369606e-09
4623 7.93573740054399e-09
4624 7.93537768828401e-09
4625 7.93508547758393e-09
4626 7.93497889617356e-09
4627 7.9346378356604e-09
4628 7.93438470481078e-09
4629 7.93416887745479e-09
4630 7.93379406616168e-09
4631 7.93356402795098e-09
4632 7.93304710811071e-09
4633 7.93300181101131e-09
4634 7.9330781943554e-09
4635 7.93261989429084e-09
4636 7.93233745355337e-09
4637 7.93182408642679e-09
4638 7.93187293623987e-09
4639 7.93114818264939e-09
4640 7.9311215372968e-09
4641 7.93116416986095e-09
4642 7.93063836823649e-09
4643 7.93022092437923e-09
4644 7.92992693732231e-09
4645 7.92976084795782e-09
4646 7.92953702699606e-09
4647 7.92918886105554e-09
4648 7.92890642031807e-09
4649 7.92858934062224e-09
4650 7.92836374330363e-09
4651 7.92806797988987e-09
4652 7.92785481706915e-09
4653 7.92759369261375e-09
4654 7.92735477261886e-09
4655 7.92702792296041e-09
4656 7.92712828712183e-09
4657 7.92641152713713e-09
4658 7.92636001278879e-09
4659 7.92595233889415e-09
4660 7.92579868402754e-09
4661 7.92559351481259e-09
4662 7.92521781534106e-09
4663 7.92485721490266e-09
4664 7.92462806487038e-09
4665 7.92440513208703e-09
4666 7.92435272956027e-09
4667 7.92357823797829e-09
4668 7.92347343292477e-09
4669 7.92331356080922e-09
4670 7.92314214237422e-09
4671 7.92267673688229e-09
4672 7.92240673064271e-09
4673 7.92243959324423e-09
4674 7.92216514611255e-09
4675 7.92152032857985e-09
4676 7.92156118478715e-09
4677 7.92083465483984e-09
4678 7.92085330658665e-09
4679 7.92054066778292e-09
4680 7.92047583075828e-09
4681 7.9201702973819e-09
4682 7.91983367776083e-09
4683 7.91967913471581e-09
4684 7.91944643196985e-09
4685 7.91883891793077e-09
4686 7.91865861771157e-09
4687 7.91842946767929e-09
4688 7.91836551883307e-09
4689 7.9179880430047e-09
4690 7.91756704643376e-09
4691 7.91739562799876e-09
4692 7.91708920644396e-09
4693 7.91659982013471e-09
4694 7.91645327069546e-09
4695 7.91621257434372e-09
4696 7.91576848513387e-09
4697 7.91565923918824e-09
4698 7.91538568023498e-09
4699 7.91516185927321e-09
4700 7.91470711192233e-09
4701 7.91463783400559e-09
4702 7.91436605140916e-09
4703 7.91397347654765e-09
4704 7.91371590480594e-09
4705 7.91341925321376e-09
4706 7.91281173917469e-09
4707 7.91286414170145e-09
4708 7.91250887033357e-09
4709 7.91214560535991e-09
4710 7.91199372685014e-09
4711 7.91185161830299e-09
4712 7.91168730529535e-09
4713 7.91118637266663e-09
4714 7.91100251973376e-09
4715 7.91054954873971e-09
4716 7.91041188108466e-09
4717 7.90976084630302e-09
4718 7.90986653953496e-09
4719 7.90963028407532e-09
4720 7.90923149196487e-09
4721 7.90896415026054e-09
4722 7.90861331978476e-09
4723 7.90836462982725e-09
4724 7.90800314121043e-09
4725 7.9078485981654e-09
4726 7.90759102642369e-09
4727 7.90721177423848e-09
4728 7.90689203000738e-09
4729 7.90670107164715e-09
4730 7.9066131419836e-09
4731 7.90607046496916e-09
4732 7.90598786437613e-09
4733 7.9058333213311e-09
4734 7.90558907226568e-09
4735 7.90507925785278e-09
4736 7.90500997993604e-09
4737 7.90468224209917e-09
4738 7.90441578857326e-09
4739 7.90402943096069e-09
4740 7.90362264524447e-09
4741 7.9036466260618e-09
4742 7.90307108644583e-09
4743 7.90305687559112e-09
4744 7.90262877359282e-09
4745 7.90235166192588e-09
4746 7.90190313182393e-09
4747 7.901924448106e-09
4748 7.90149012885877e-09
4749 7.90106824410941e-09
4750 7.90082665957925e-09
4751 7.90044918375088e-09
4752 7.90030441066847e-09
4753 7.89994381023007e-09
4754 7.89985410420968e-09
4755 7.8995485708333e-09
4756 7.89957610436431e-09
4757 7.89916043686389e-09
4758 7.89873411122244e-09
4759 7.89846588133969e-09
4760 7.89835041814513e-09
4761 7.89786724908481e-09
4762 7.8978352746617e-09
4763 7.89732457207037e-09
4764 7.89718779259374e-09
4765 7.89689647007208e-09
4766 7.89649678978321e-09
4767 7.89626142250199e-09
4768 7.89591236838305e-09
4769 7.89592569105935e-09
4770 7.89563348035927e-09
4771 7.89513343590897e-09
4772 7.89500198550286e-09
4773 7.89442999860057e-09
4774 7.89432785808231e-09
4775 7.89430210090813e-09
4776 7.89380116827942e-09
4777 7.89331089379175e-09
4778 7.89357734731766e-09
4779 7.89304799297952e-09
4780 7.89243692622676e-09
4781 7.89245380161674e-09
4782 7.89223086883339e-09
4783 7.89182852400927e-09
4784 7.89160559122593e-09
4785 7.89137999390732e-09
4786 7.89125831346382e-09
4787 7.89100429443579e-09
4788 7.89029197534319e-09
4789 7.89028309355899e-09
4790 7.89021914471277e-09
4791 7.88990828226588e-09
4792 7.8894268895624e-09
4793 7.88921195038483e-09
4794 7.88877052571024e-09
4795 7.88850851307643e-09
4796 7.88825982311891e-09
4797 7.88805287754712e-09
4798 7.88758924841204e-09
4799 7.88738585555393e-09
4800 7.88704479504077e-09
4801 7.88682630314952e-09
4802 7.88647280813848e-09
4803 7.88626586256669e-09
4804 7.88604292978334e-09
4805 7.88568321752336e-09
4806 7.88546472563212e-09
4807 7.88496912207393e-09
4808 7.88474441293374e-09
4809 7.8844406559142e-09
4810 7.88421239406034e-09
4811 7.88409959540104e-09
4812 7.88372211957267e-09
4813 7.88350362768142e-09
4814 7.88327803036282e-09
4815 7.88294585163385e-09
4816 7.88257903394651e-09
4817 7.88247422889299e-09
4818 7.88216247826767e-09
4819 7.88170861909521e-09
4820 7.8815372006602e-09
4821 7.881500785345e-09
4822 7.88106113702725e-09
4823 7.88085596781229e-09
4824 7.88065435131102e-09
4825 7.88031684351154e-09
4826 7.87997223028469e-09
4827 7.8800175273841e-09
4828 7.87963294612837e-09
4829 7.87941889512922e-09
4830 7.87900855669932e-09
4831 7.87856713202473e-09
4832 7.87840548355234e-09
4833 7.87814524727537e-09
4834 7.87804932400604e-09
4835 7.87754306230681e-09
4836 7.87715848105108e-09
4837 7.87719489636629e-09
4838 7.87674636626434e-09
4839 7.87655540790411e-09
4840 7.87629783616239e-09
4841 7.87598519735866e-09
4842 7.87563259052604e-09
4843 7.87538834146062e-09
4844 7.87513343425417e-09
4845 7.87488474429665e-09
4846 7.8744450959789e-09
4847 7.87419995873506e-09
4848 7.87400011859063e-09
4849 7.87350540321086e-09
4850 7.87346454700355e-09
4851 7.87320608708342e-09
4852 7.87284282210976e-09
4853 7.87268561452947e-09
4854 7.87219889275548e-09
4855 7.87217580011657e-09
4856 7.87190312934172e-09
4857 7.87111886779712e-09
4858 7.87101051002992e-09
4859 7.8707573791803e-09
4860 7.87072185204352e-09
4861 7.87050602468753e-09
4862 7.8699500249968e-09
4863 7.86997400581413e-09
4864 7.86957965459578e-09
4865 7.86942067065866e-09
4866 7.86892151438678e-09
4867 7.86878739944541e-09
4868 7.86838150190761e-09
4869 7.86828380228144e-09
4870 7.86787968110048e-09
4871 7.86770559813021e-09
4872 7.86727927248876e-09
4873 7.86697551546922e-09
4874 7.86687959219989e-09
4875 7.86668419294756e-09
4876 7.86611753511579e-09
4877 7.86601095370543e-09
4878 7.86551446196881e-09
4879 7.86549936293568e-09
4880 7.86510767625259e-09
4881 7.86487230897137e-09
4882 7.8647337531379e-09
4883 7.86428522303595e-09
4884 7.86405163211157e-09
4885 7.86376386230359e-09
4886 7.86326825874539e-09
4887 7.86331977309374e-09
4888 7.86320697443443e-09
4889 7.862835715855e-09
4890 7.86258880225432e-09
4891 7.86221754367489e-09
4892 7.86190490487115e-09
4893 7.86181963974286e-09
4894 7.86131604257889e-09
4895 7.86117393403174e-09
4896 7.86061082891365e-09
4897 7.86046339129598e-09
4898 7.86024401122631e-09
4899 7.85987985807424e-09
4900 7.85977505302071e-09
4901 7.85916753898164e-09
4902 7.85916043355428e-09
4903 7.85870213348971e-09
4904 7.85844100903432e-09
4905 7.85795162272507e-09
4906 7.85774467715328e-09
4907 7.85755105425778e-09
4908 7.85732456876076e-09
4909 7.85743292652796e-09
4910 7.85681919523995e-09
4911 7.85672593650588e-09
4912 7.85645237755261e-09
4913 7.85600118291541e-09
4914 7.85601361741328e-09
4915 7.85579423734362e-09
4916 7.85541409697998e-09
4917 7.85504550293581e-09
4918 7.85481901743879e-09
4919 7.8545356885229e-09
4920 7.85415199544559e-09
4921 7.85368481359683e-09
4922 7.85349296705817e-09
4923 7.85327536334535e-09
4924 7.85285703130967e-09
4925 7.85288634119752e-09
4926 7.85222908916694e-09
4927 7.85195553021367e-09
4928 7.85181608620178e-09
4929 7.85137466152719e-09
4930 7.85125031654843e-09
4931 7.85103893008454e-09
4932 7.85062592711938e-09
4933 7.85037279626977e-09
4934 7.85011255999279e-09
4935 7.84989673263681e-09
4936 7.84949261145584e-09
4937 7.84940201725703e-09
4938 7.84914178098006e-09
4939 7.84842324463852e-09
4940 7.84875364701065e-09
4941 7.84828024791295e-09
4942 7.84815412657736e-09
4943 7.84778997342528e-09
4944 7.84750930904465e-09
4945 7.84702525180592e-09
4946 7.84670994846692e-09
4947 7.84635023620694e-09
4948 7.84632270267593e-09
4949 7.84616194238197e-09
4950 7.84567433242955e-09
4951 7.84565301614748e-09
4952 7.84526044128597e-09
4953 7.84487674820866e-09
4954 7.84456144486967e-09
4955 7.84453657587392e-09
4956 7.84406761766832e-09
4957 7.8437514261509e-09
4958 7.84330289604895e-09
4959 7.84328513248056e-09
4960 7.84313414214921e-09
4961 7.84263853859102e-09
4962 7.84230635986205e-09
4963 7.84215270499544e-09
4964 7.84179299273546e-09
4965 7.84138798337608e-09
4966 7.84150344657064e-09
4967 7.84112330620701e-09
4968 7.84070497417133e-09
4969 7.84079734472698e-09
4970 7.84031328748824e-09
4971 7.84000953046871e-09
4972 7.83997489151034e-09
4973 7.83971376705495e-09
4974 7.8393691538281e-09
4975 7.8390671731654e-09
4976 7.8386568347355e-09
4977 7.83850406804731e-09
4978 7.83843834284426e-09
4979 7.8381336976463e-09
4980 7.8380706369785e-09
4981 7.83747466925888e-09
4982 7.83709630525209e-09
4983 7.83681564087146e-09
4984 7.83659359626654e-09
4985 7.83617970512296e-09
4986 7.83623743672024e-09
4987 7.83574716223256e-09
4988 7.83536169279841e-09
4989 7.83503040224787e-09
4990 7.83519116254183e-09
4991 7.83479858768032e-09
4992 7.83441578278143e-09
4993 7.83383757863021e-09
4994 7.83394504821899e-09
4995 7.83350095900914e-09
4996 7.83334730414253e-09
4997 7.83302667173302e-09
4998 7.83286235872538e-09
4999 7.83261899783838e-09
};
\addlegendentry{Test}

\nextgroupplot[
title={SiLU/SiLU $\hy$},
ymin=3.93337441067931e-09, ymax=1e-05,
]
\addplot [semithick, black, dashed]
table {%
0 0.00512535910750739
1 0.000639720292863785
2 0.000219938890388221
3 0.000213346519893094
4 0.000208158865601945
5 0.000195952031097477
6 0.000152308875522067
7 5.34084106944874e-05
8 1.90344308875012e-05
9 1.76928728920984e-05
10 1.69933794408053e-05
11 1.63289667107733e-05
12 1.56492851359644e-05
13 1.49146216191127e-05
14 1.40739459743031e-05
15 1.30518305153373e-05
16 1.17537783500978e-05
17 1.00939727726654e-05
18 8.08761118787515e-06
19 5.98911539312041e-06
20 4.26612420854156e-06
21 3.19983702638638e-06
22 2.63914656158093e-06
23 2.33226211698678e-06
24 2.14682576298841e-06
25 2.02127115965567e-06
26 1.92288998271195e-06
27 1.83512577777023e-06
28 1.75022869860797e-06
29 1.66454335781552e-06
30 1.57619800089748e-06
31 1.48410523198805e-06
32 1.3877845072976e-06
33 1.28747669272045e-06
34 1.18436522614829e-06
35 1.08077588593503e-06
36 9.79991697764859e-07
37 8.85640147600952e-07
38 8.00468675773658e-07
39 7.25543915011784e-07
40 6.60773282442406e-07
41 6.05634482356621e-07
42 5.58891621206925e-07
43 5.19801959237398e-07
44 4.87714476834711e-07
45 4.62020861180079e-07
46 4.41761609129898e-07
47 4.25855304039047e-07
48 4.13433966313548e-07
49 4.03620226693491e-07
50 3.95777241950057e-07
51 3.89383222982431e-07
52 3.8402304238172e-07
53 3.79404985786813e-07
54 3.75326995073699e-07
55 3.71653089698221e-07
56 3.68288298382424e-07
57 3.65167520923748e-07
58 3.62241256959273e-07
59 3.59471912839027e-07
60 3.56834591777933e-07
61 3.54309771839922e-07
62 3.51883008284481e-07
63 3.49541364891515e-07
64 3.47271387997061e-07
65 3.45088005731853e-07
66 3.43000523193915e-07
67 3.40982838067383e-07
68 3.39029774591104e-07
69 3.37132762814107e-07
70 3.35290269949162e-07
71 3.33493468216872e-07
72 3.31745307950548e-07
73 3.30051951679344e-07
74 3.284015409033e-07
75 3.26782234020051e-07
76 3.25196069828948e-07
77 3.23643581795707e-07
78 3.22130297353951e-07
79 3.20649117328209e-07
80 3.19195498762248e-07
81 3.17755583084001e-07
82 3.16340344395094e-07
83 3.15021212838218e-07
84 3.1363158957376e-07
85 3.1237283517882e-07
86 3.11002417124229e-07
87 3.09799599799376e-07
88 3.08450583211162e-07
89 3.07278441647796e-07
90 3.0597018807299e-07
91 3.0482873759663e-07
92 3.03560248890555e-07
93 3.02451470628462e-07
94 3.01221928006079e-07
95 3.00131844765517e-07
96 2.98932246043826e-07
97 2.97854985757517e-07
98 2.96677369453491e-07
99 2.9561638020148e-07
100 2.9448573675861e-07
101 2.93430182273902e-07
102 2.92324435917202e-07
103 2.91279103132425e-07
104 2.90190769073995e-07
105 2.89161364131374e-07
106 2.88097511269569e-07
107 2.87082401738914e-07
108 2.86041040110874e-07
109 2.85035680152035e-07
110 2.84014233980834e-07
111 2.83016880884723e-07
112 2.82013433952955e-07
113 2.81024617655667e-07
114 2.80035506780685e-07
115 2.79055244038595e-07
116 2.78078873843235e-07
117 2.77111074741043e-07
118 2.76148120613051e-07
119 2.75188796827308e-07
120 2.74232574894562e-07
121 2.73278425012791e-07
122 2.72326614147111e-07
123 2.71377344057022e-07
124 2.70428171480042e-07
125 2.69474434876571e-07
126 2.68521925981702e-07
127 2.67593293802548e-07
128 2.6666080187443e-07
129 2.65730644050777e-07
130 2.64801010358973e-07
131 2.63874162655497e-07
132 2.62947723508766e-07
133 2.62022960174413e-07
134 2.61099171100732e-07
135 2.60175490822689e-07
136 2.59252373139063e-07
137 2.58330095211257e-07
138 2.57406522473858e-07
139 2.56483853025458e-07
140 2.55560634132124e-07
141 2.54637509526034e-07
142 2.53713200612182e-07
143 2.5279052405125e-07
144 2.51865835264375e-07
145 2.50941995804332e-07
146 2.50012622307949e-07
147 2.49084289037249e-07
148 2.48155629841484e-07
149 2.47228093770424e-07
150 2.4629099740725e-07
151 2.45345893472582e-07
152 2.4439936212417e-07
153 2.43447966064458e-07
154 2.42492362325208e-07
155 2.41533418301465e-07
156 2.40571709577253e-07
157 2.39607331632641e-07
158 2.38638065584773e-07
159 2.37665288477906e-07
160 2.36690059448819e-07
161 2.35711158374841e-07
162 2.34726038057964e-07
163 2.33735358535547e-07
164 2.32736730353622e-07
165 2.31731472664976e-07
166 2.30718313017242e-07
167 2.29699605682754e-07
168 2.28673384111921e-07
169 2.27640834813414e-07
170 2.26601973203344e-07
171 2.25557661443609e-07
172 2.24507237876814e-07
173 2.23450801870584e-07
174 2.22387463367113e-07
175 2.21316986345421e-07
176 2.20238014497554e-07
177 2.19152528400102e-07
178 2.18058267534182e-07
179 2.16955014125553e-07
180 2.15843134796501e-07
181 2.1472208803619e-07
182 2.1359073764593e-07
183 2.12450471162917e-07
184 2.113001165851e-07
185 2.10139339293036e-07
186 2.08968759242367e-07
187 2.07787444593599e-07
188 2.06596403632631e-07
189 2.05395092175209e-07
190 2.04182489171423e-07
191 2.02957513021573e-07
192 2.01718675147511e-07
193 2.00462533553214e-07
194 1.99206355838921e-07
195 1.97930535630775e-07
196 1.96647579680409e-07
197 1.95343415001048e-07
198 1.94037127253743e-07
199 1.92702156665003e-07
200 1.91374526927302e-07
201 1.90006182620106e-07
202 1.88655334314891e-07
203 1.87255240977624e-07
204 1.85877720233485e-07
205 1.84449337802928e-07
206 1.83042574014802e-07
207 1.81594735715773e-07
208 1.80161775776622e-07
209 1.78700470622406e-07
210 1.77237373141903e-07
211 1.75756895926416e-07
212 1.74267538242123e-07
213 1.7278687706046e-07
214 1.71277531202385e-07
215 1.69746775081414e-07
216 1.68202808854545e-07
217 1.66649689767873e-07
218 1.65089282416986e-07
219 1.63516721144674e-07
220 1.6193505027795e-07
221 1.6034280974786e-07
222 1.58733543424106e-07
223 1.57143660539649e-07
224 1.55526813716111e-07
225 1.53902776613091e-07
226 1.52302885551237e-07
227 1.50706110165544e-07
228 1.49006862196099e-07
229 1.47454685760717e-07
230 1.45791819955665e-07
231 1.44188111466104e-07
232 1.42482930379995e-07
233 1.40918853346328e-07
234 1.39247935837528e-07
235 1.3765672550381e-07
236 1.35976252662484e-07
237 1.34443737075163e-07
238 1.32752790609647e-07
239 1.31238320117966e-07
240 1.29545683552834e-07
241 1.28040128382523e-07
242 1.26381496673655e-07
243 1.24921477826501e-07
244 1.23209933917234e-07
245 1.21859218824572e-07
246 1.20169844537443e-07
247 1.18785487768225e-07
248 1.17154279902998e-07
249 1.15844081549632e-07
250 1.14243283438231e-07
251 1.12869539721228e-07
252 1.11370678911094e-07
253 1.10005821498227e-07
254 1.08554840247077e-07
255 1.0723927804257e-07
256 1.05825512203062e-07
257 1.04439683281221e-07
258 1.03109958979442e-07
259 1.0175656946565e-07
260 1.00535790600986e-07
261 9.93190555167445e-08
262 9.79479783649317e-08
263 9.68807181367382e-08
264 9.56333660595554e-08
265 9.47063393743619e-08
266 9.33537782437099e-08
267 9.25881490672076e-08
268 9.12600109996831e-08
269 9.03511068202079e-08
270 8.92672997019339e-08
271 8.85352895090286e-08
272 8.73823008298302e-08
273 8.65892422092784e-08
274 8.56988069850573e-08
275 8.49862492144915e-08
276 8.40123275329674e-08
277 8.32407768363019e-08
278 8.2588906479053e-08
279 8.17604703411412e-08
280 8.10526001528977e-08
281 8.03234412396492e-08
282 7.98996261410778e-08
283 7.88819039589939e-08
284 7.83757943425556e-08
285 7.77175460591195e-08
286 7.73587728009062e-08
287 7.64051363546514e-08
288 7.59834665786663e-08
289 7.53605860528417e-08
290 7.49552834835576e-08
291 7.43421029483215e-08
292 7.39850573809875e-08
293 7.35203335766599e-08
294 7.29181973500026e-08
295 7.24447226816416e-08
296 7.22193399345095e-08
297 7.15187746314072e-08
298 7.14495858669117e-08
299 7.0701798197792e-08
300 7.05346177496402e-08
301 6.99676678412864e-08
302 6.97904336992572e-08
303 6.92364931818012e-08
304 6.89966972222678e-08
305 6.86904714868319e-08
306 6.82905602538142e-08
307 6.8082630654942e-08
308 6.75891537000872e-08
309 6.74066373984594e-08
310 6.70051516329373e-08
311 6.68603825300096e-08
312 6.63642619573856e-08
313 6.61811567201909e-08
314 6.58227906922271e-08
315 6.5561314119833e-08
316 6.53025389181217e-08
317 6.50143585874119e-08
318 6.47715931454407e-08
319 6.45538544046254e-08
320 6.42187231414404e-08
321 6.40240758080779e-08
322 6.37326394694604e-08
323 6.35879368093839e-08
324 6.32885389895144e-08
325 6.30014221130182e-08
326 6.28195169798396e-08
327 6.25809636614072e-08
328 6.2357067069474e-08
329 6.21392710971058e-08
330 6.19176045240089e-08
331 6.17066421564338e-08
332 6.15068206566427e-08
333 6.130771019075e-08
334 6.10625812624832e-08
335 6.08342512551374e-08
336 6.07151493823288e-08
337 6.04759866980054e-08
338 6.02517568508709e-08
339 6.01494727510143e-08
340 5.98663017354895e-08
341 5.97585069170847e-08
342 5.95000388692313e-08
343 5.93391995793446e-08
344 5.92122766445158e-08
345 5.89909608512684e-08
346 5.88098232161549e-08
347 5.86431092042261e-08
348 5.84403378320886e-08
349 5.83087105852087e-08
350 5.81350444375239e-08
351 5.79534860962738e-08
352 5.7813852894828e-08
353 5.76337226099177e-08
354 5.75199038577168e-08
355 5.73269674051247e-08
356 5.71927224619628e-08
357 5.70341215992976e-08
358 5.68740103354948e-08
359 5.66948075708673e-08
360 5.65757047032989e-08
361 5.64192751126846e-08
362 5.62794713316883e-08
363 5.61334107236355e-08
364 5.59943396880591e-08
365 5.58522607212453e-08
366 5.57168466412961e-08
367 5.55782733266952e-08
368 5.54453246235376e-08
369 5.53097664779401e-08
370 5.51787908698032e-08
371 5.5050641829979e-08
372 5.4910767282923e-08
373 5.47971402253822e-08
374 5.46763899236247e-08
375 5.45639124300834e-08
376 5.44512685274334e-08
377 5.43250151170405e-08
378 5.42041361715206e-08
379 5.4082853344628e-08
380 5.39652121305245e-08
381 5.38434164543133e-08
382 5.3730553506437e-08
383 5.3611088255856e-08
384 5.34982814337504e-08
385 5.33819620707199e-08
386 5.32728731834631e-08
387 5.31540187751478e-08
388 5.30522247996767e-08
389 5.29316528130597e-08
390 5.28344366965694e-08
391 5.27141018960009e-08
392 5.26188193368249e-08
393 5.25029954761003e-08
394 5.24085193347457e-08
395 5.22980297572584e-08
396 5.22072740380963e-08
397 5.20915161679181e-08
398 5.20075912127105e-08
399 5.1883218110671e-08
400 5.18106649614047e-08
401 5.1682791562202e-08
402 5.16058134969466e-08
403 5.14962644408978e-08
404 5.14178823427525e-08
405 5.12961585734661e-08
406 5.12250208266618e-08
407 5.11108862881748e-08
408 5.10333545611363e-08
409 5.09329501487965e-08
410 5.08469049085303e-08
411 5.07541093275332e-08
412 5.0662501996479e-08
413 5.05734375282341e-08
414 5.04819937068302e-08
415 5.03904563244895e-08
416 5.02987833188584e-08
417 5.02139634583898e-08
418 5.01289604888377e-08
419 5.00428157115262e-08
420 4.99588591502675e-08
421 4.98734515474997e-08
422 4.9791398823773e-08
423 4.97080778498749e-08
424 4.96261488112104e-08
425 4.9545461352718e-08
426 4.94634097518709e-08
427 4.93851515548371e-08
428 4.93040508950937e-08
429 4.92276187127327e-08
430 4.91459830065377e-08
431 4.90728447977151e-08
432 4.89916621029263e-08
433 4.8919921666668e-08
434 4.88382551804278e-08
435 4.87688811539044e-08
436 4.86878903431798e-08
437 4.86201276423959e-08
438 4.85394233418202e-08
439 4.84733231529422e-08
440 4.83947547762487e-08
441 4.83277504015334e-08
442 4.82512717994155e-08
443 4.81855269096609e-08
444 4.81104564982182e-08
445 4.80446353845476e-08
446 4.79713525265435e-08
447 4.79054179709504e-08
448 4.78350008101813e-08
449 4.7768693503647e-08
450 4.76997623837239e-08
451 4.7633225711774e-08
452 4.75666319874613e-08
453 4.75001135291642e-08
454 4.74334992579539e-08
455 4.73672447030005e-08
456 4.73002020031643e-08
457 4.7234266117302e-08
458 4.71703521534028e-08
459 4.71069812779668e-08
460 4.70430486563256e-08
461 4.69820097748208e-08
462 4.69189795351266e-08
463 4.68568622380428e-08
464 4.67964718273706e-08
465 4.67350066348793e-08
466 4.66744733269575e-08
467 4.66144975006522e-08
468 4.65547647885423e-08
469 4.64949027960682e-08
470 4.64354554727819e-08
471 4.63771795247148e-08
472 4.63193288857688e-08
473 4.62610056248813e-08
474 4.62037068047483e-08
475 4.61457267262322e-08
476 4.60889168591994e-08
477 4.60323398177032e-08
478 4.59761477409337e-08
479 4.59198834348307e-08
480 4.58642846326018e-08
481 4.58091596235999e-08
482 4.57548672805341e-08
483 4.56993846320586e-08
484 4.56457527606702e-08
485 4.55907539396172e-08
486 4.55382997603948e-08
487 4.54837911569861e-08
488 4.54308415496207e-08
489 4.53788900718344e-08
490 4.53258477637331e-08
491 4.52736155278632e-08
492 4.52216790411963e-08
493 4.51697403602846e-08
494 4.51189052459444e-08
495 4.50682541854874e-08
496 4.5016897950223e-08
497 4.49665437092239e-08
498 4.49167134943806e-08
499 4.48665760479638e-08
500 4.48170651148327e-08
501 4.47682651285586e-08
502 4.4718194329274e-08
503 4.46700904646047e-08
504 4.46213138494134e-08
505 4.4572991120706e-08
506 4.45249240486056e-08
507 4.44771913772346e-08
508 4.44302776292371e-08
509 4.43817818365222e-08
510 4.43354687329478e-08
511 4.42889323035711e-08
512 4.4242567534214e-08
513 4.41954662278032e-08
514 4.41496496543792e-08
515 4.41033595450779e-08
516 4.40579462845925e-08
517 4.40122859934888e-08
518 4.39666436846675e-08
519 4.39210295093417e-08
520 4.38756545437791e-08
521 4.3829840658649e-08
522 4.37851607155704e-08
523 4.37376176338766e-08
524 4.36937158185824e-08
525 4.36501119405186e-08
526 4.3604051679802e-08
527 4.35622673369274e-08
528 4.35167214121179e-08
529 4.3479630861043e-08
530 4.34306699363951e-08
531 4.33943745801191e-08
532 4.33436233171491e-08
533 4.33064388261073e-08
534 4.32649144062047e-08
535 4.32214885928151e-08
536 4.31817068664575e-08
537 4.3140906889505e-08
538 4.31001751122206e-08
539 4.30591774251976e-08
540 4.30189502065303e-08
541 4.29784749831352e-08
542 4.29376849524488e-08
543 4.28962190641702e-08
544 4.28534993848206e-08
545 4.28059555765969e-08
546 4.27596600602875e-08
547 4.27242738381128e-08
548 4.26861877684814e-08
549 4.26448522199419e-08
550 4.26076612305426e-08
551 4.25682858216092e-08
552 4.25299870701057e-08
553 4.24922333861488e-08
554 4.24535353182876e-08
555 4.24149737980883e-08
556 4.2378853457592e-08
557 4.23396832240908e-08
558 4.23031857641476e-08
559 4.22642045354582e-08
560 4.2229252803283e-08
561 4.21896648472586e-08
562 4.21545678257473e-08
563 4.211434591328e-08
564 4.20839432728748e-08
565 4.20395626596903e-08
566 4.20088389618822e-08
567 4.19668344446844e-08
568 4.19306516257212e-08
569 4.18941736723966e-08
570 4.18579670342645e-08
571 4.18232036387689e-08
572 4.17857382384312e-08
573 4.17530903520369e-08
574 4.17127207130719e-08
575 4.16837815035986e-08
576 4.16382409393901e-08
577 4.1614269504997e-08
578 4.15657251224832e-08
579 4.15414996821628e-08
580 4.14959152479355e-08
581 4.14782732589547e-08
582 4.1426270217082e-08
583 4.13927730793517e-08
584 4.13432459995811e-08
585 4.13131511369791e-08
586 4.12905497346738e-08
587 4.12403240463988e-08
588 4.12292937033154e-08
589 4.11576173404704e-08
590 4.11289096888812e-08
591 4.1092181203739e-08
592 4.10729122213294e-08
593 4.10050331824952e-08
594 4.09671166492487e-08
595 4.09642252008346e-08
596 4.09049783927085e-08
597 4.08650469587712e-08
598 4.08479956737207e-08
599 4.08201107242601e-08
600 4.0770035780513e-08
601 4.07398541595327e-08
602 4.07144987817354e-08
603 4.06992351891233e-08
604 4.06168458373379e-08
605 4.06324813901904e-08
606 4.05787787109979e-08
607 4.05673701138198e-08
608 4.04979805812555e-08
609 4.05343075871034e-08
610 4.04170279517668e-08
611 4.04473817681161e-08
612 4.03847957310965e-08
613 4.03854687971439e-08
614 4.03105578605789e-08
615 4.03211758837063e-08
616 4.02711263041144e-08
617 4.02443121116924e-08
618 4.01937752816384e-08
619 4.02053000623237e-08
620 4.01384296950802e-08
621 4.01338766511383e-08
622 4.00652393053313e-08
623 4.00734058936258e-08
624 4.0028524449065e-08
625 4.00084193783457e-08
626 3.99605991698859e-08
627 3.99212131676485e-08
628 3.99321293025334e-08
629 3.99008544833546e-08
630 3.98266357042321e-08
631 3.98083765069668e-08
632 3.98037232238746e-08
633 3.97488013028635e-08
634 3.97362548971714e-08
635 3.97388116550879e-08
636 3.96776020845646e-08
637 3.96516899907606e-08
638 3.96310476893369e-08
639 3.95621407414115e-08
640 3.95635408922956e-08
641 3.95415621308182e-08
642 3.94901924645197e-08
643 3.9470851290746e-08
644 3.93704731305178e-08
645 3.94242004677636e-08
646 3.9380626617902e-08
647 3.93760709627156e-08
648 3.92954699479642e-08
649 3.93255144497395e-08
650 3.92637690662667e-08
651 3.92287399546554e-08
652 3.92150501509203e-08
653 3.91708971272386e-08
654 3.91577712199886e-08
655 3.91186855672743e-08
656 3.90895668160862e-08
657 3.90523998563541e-08
658 3.90708400175299e-08
659 3.90171295661101e-08
660 3.90439261142017e-08
661 3.89361408517086e-08
662 3.89653273045187e-08
663 3.89041039128291e-08
664 3.88634109067532e-08
665 3.88728271758509e-08
666 3.88456340953969e-08
667 3.88148369947494e-08
668 3.87463509274877e-08
669 3.87684742806371e-08
670 3.86902384190702e-08
671 3.86507337759578e-08
672 3.86492579942388e-08
673 3.86507256073809e-08
674 3.85556497288775e-08
675 3.85773809201684e-08
676 3.85164368275071e-08
677 3.85501637789609e-08
678 3.84674434259491e-08
679 3.8464894366097e-08
680 3.84377841180505e-08
681 3.84424873376421e-08
682 3.83417732465752e-08
683 3.83853671648282e-08
684 3.83250011450897e-08
685 3.83185881356329e-08
686 3.82478062448399e-08
687 3.82493071482415e-08
688 3.82446606861375e-08
689 3.81893767389396e-08
690 3.81763915831002e-08
691 3.81116697114603e-08
692 3.81868264176521e-08
693 3.80648722224652e-08
694 3.80829763637625e-08
695 3.80396512462688e-08
696 3.79943731936283e-08
697 3.80012080869463e-08
698 3.7932516733008e-08
699 3.79715519096546e-08
700 3.78982097839486e-08
701 3.78940990919396e-08
702 3.78720904941066e-08
703 3.78584374001178e-08
704 3.77787794920703e-08
705 3.78112814432008e-08
706 3.77707338516942e-08
707 3.77212475977107e-08
708 3.77242630429375e-08
709 3.77706944854062e-08
710 3.75954540334789e-08
711 3.76812049711628e-08
712 3.75811326049913e-08
713 3.76185061039536e-08
714 3.75172929787393e-08
715 3.75564150703056e-08
716 3.74896634280919e-08
717 3.75255119473561e-08
718 3.74789598889258e-08
719 3.74350139449309e-08
720 3.74148668678131e-08
721 3.73566459559438e-08
722 3.73731356972939e-08
723 3.73402777960052e-08
724 3.73119163221691e-08
725 3.72511990600621e-08
726 3.7295318448427e-08
727 3.71837867301217e-08
728 3.72193962516754e-08
729 3.71828286362952e-08
730 3.71215692964721e-08
731 3.7189168493601e-08
732 3.70980424007072e-08
733 3.70959233113588e-08
734 3.70785614267621e-08
735 3.70161654805745e-08
736 3.70239491052082e-08
737 3.69850575570041e-08
738 3.69887663449475e-08
739 3.69204982784144e-08
740 3.68987612537897e-08
741 3.68831999495445e-08
742 3.68782351429786e-08
743 3.68650003355064e-08
744 3.68222094020432e-08
745 3.68458355957113e-08
746 3.67288878662109e-08
747 3.67294128489348e-08
748 3.6746161472978e-08
749 3.66657595156639e-08
750 3.6725184701103e-08
751 3.66292251239297e-08
752 3.66000139742084e-08
753 3.66645153748735e-08
754 3.65894051015392e-08
755 3.65155326063071e-08
756 3.65489783993311e-08
757 3.6518751294401e-08
758 3.6473308048901e-08
759 3.65592512932e-08
760 3.64209612679911e-08
761 3.63908045704076e-08
762 3.6410732806802e-08
763 3.63504204587972e-08
764 3.63405476251355e-08
765 3.6346382778607e-08
766 3.62924369979867e-08
767 3.63195842043584e-08
768 3.62280162975459e-08
769 3.62016681614286e-08
770 3.62693160482763e-08
771 3.61211819780438e-08
772 3.61771025054747e-08
773 3.61400587302052e-08
774 3.61910671202725e-08
775 3.60654778834846e-08
776 3.60207120010392e-08
777 3.60486690017536e-08
778 3.60169386228737e-08
779 3.60712034994659e-08
780 3.59277034853722e-08
781 3.59452601992327e-08
782 3.59571591637131e-08
783 3.58440765644286e-08
784 3.58656816453751e-08
785 3.58260307115055e-08
786 3.58631130195075e-08
787 3.57902872931692e-08
788 3.57665007612029e-08
789 3.57789381336682e-08
790 3.57529642449617e-08
791 3.57082706552436e-08
792 3.56471938974812e-08
793 3.56693951651943e-08
794 3.56430532738594e-08
795 3.5611782842393e-08
796 3.56394114848779e-08
797 3.55643606331224e-08
798 3.54813837866175e-08
799 3.55800257926253e-08
800 3.54600028076302e-08
801 3.54651919124427e-08
802 3.55103008723434e-08
803 3.54197809628864e-08
804 3.54651364052883e-08
805 3.54049746981167e-08
806 3.53994303087246e-08
807 3.53468599508799e-08
808 3.53436626987502e-08
809 3.52752838000958e-08
810 3.527892753874e-08
811 3.52890028213482e-08
812 3.52700175615395e-08
813 3.52290886674877e-08
814 3.51946809866721e-08
815 3.52187231299483e-08
816 3.51359171452126e-08
817 3.51099621084261e-08
818 3.51254463840656e-08
819 3.50684031926729e-08
820 3.50385578089529e-08
821 3.50610042586519e-08
822 3.49789337213968e-08
823 3.50222879405004e-08
824 3.49659763940746e-08
825 3.48980294573309e-08
826 3.5053130884477e-08
827 3.48634991520491e-08
828 3.48341823689369e-08
829 3.4854829361608e-08
830 3.48647214275522e-08
831 3.47919680558917e-08
832 3.48553106105376e-08
833 3.47867262175949e-08
834 3.47580600060793e-08
835 3.4672952173298e-08
836 3.47860660646626e-08
837 3.46618390646425e-08
838 3.46274734629448e-08
839 3.47796766114739e-08
840 3.45457911915537e-08
841 3.45866986992682e-08
842 3.46025863500188e-08
843 3.46315269511566e-08
844 3.45294306751676e-08
845 3.44900227623457e-08
846 3.45656304869824e-08
847 3.44062302634063e-08
848 3.44999894226206e-08
849 3.44395344810122e-08
850 3.44495174330328e-08
851 3.43760859810027e-08
852 3.43458260494423e-08
853 3.43531092340532e-08
854 3.43563574903838e-08
855 3.42991423688543e-08
856 3.4286182950094e-08
857 3.42747549829658e-08
858 3.42392738925668e-08
859 3.42260293024754e-08
860 3.41847520861638e-08
861 3.41663566405526e-08
862 3.41499111046195e-08
863 3.4124269285507e-08
864 3.40839970727203e-08
865 3.4092546089326e-08
866 3.41354070741984e-08
867 3.40046329011523e-08
868 3.41446566357595e-08
869 3.39986562943873e-08
870 3.39747080085884e-08
871 3.3969217808627e-08
872 3.39392058791299e-08
873 3.39446237779262e-08
874 3.39456544746763e-08
875 3.39410193680445e-08
876 3.38339090333317e-08
877 3.38528392603843e-08
878 3.38272439498466e-08
879 3.37981714161328e-08
880 3.37780764770868e-08
881 3.38158673666156e-08
882 3.37458799487544e-08
883 3.3738798416949e-08
884 3.37229120150884e-08
885 3.37218434771636e-08
886 3.36608702932173e-08
887 3.36748869445014e-08
888 3.36255121927698e-08
889 3.3640832446391e-08
890 3.3527197432659e-08
891 3.36553359896463e-08
892 3.35038193091108e-08
893 3.35240095749034e-08
894 3.35090889250766e-08
895 3.35178769835798e-08
896 3.34573977036623e-08
897 3.34803864053823e-08
898 3.34446718102122e-08
899 3.34004466080451e-08
900 3.34153850940488e-08
901 3.33361796570619e-08
902 3.33799680609648e-08
903 3.34205839945811e-08
904 3.33487114773101e-08
905 3.32757983303456e-08
906 3.33095706676056e-08
907 3.32445587429842e-08
908 3.3260399457169e-08
909 3.32424301258127e-08
910 3.31793135829317e-08
911 3.32130006863718e-08
912 3.31578208522254e-08
913 3.31730560291232e-08
914 3.31448633958953e-08
915 3.30978170650642e-08
916 3.30994676311924e-08
917 3.30772283209102e-08
918 3.31258080682462e-08
919 3.29929468952939e-08
920 3.30351108578375e-08
921 3.2969123199833e-08
922 3.29848581304137e-08
923 3.29555059539466e-08
924 3.29151660455329e-08
925 3.2901238810723e-08
926 3.29412571259402e-08
927 3.28485999007144e-08
928 3.28421872305418e-08
929 3.28479354203548e-08
930 3.28012383965515e-08
931 3.28169274607104e-08
932 3.2741879247955e-08
933 3.27941625219985e-08
934 3.26490763574716e-08
935 3.28098866725357e-08
936 3.27539589936032e-08
937 3.26300968468418e-08
938 3.2669780379635e-08
939 3.26650536127682e-08
940 3.26313285327062e-08
941 3.25922549032631e-08
942 3.25690336113738e-08
943 3.26193337975011e-08
944 3.25631379098912e-08
945 3.25134307784802e-08
946 3.2493718743809e-08
947 3.24932637650877e-08
948 3.25040437665347e-08
949 3.24538718509126e-08
950 3.24564682741002e-08
951 3.24308698359532e-08
952 3.23829902088013e-08
953 3.23903759721356e-08
954 3.23459033537699e-08
955 3.24026609904715e-08
956 3.23524301558686e-08
957 3.22495819856661e-08
958 3.23131245409947e-08
959 3.22575934924574e-08
960 3.22485058110722e-08
961 3.22259004682524e-08
962 3.22109836545792e-08
963 3.21933706503019e-08
964 3.21822476611056e-08
965 3.22375118883e-08
966 3.21348257096288e-08
967 3.21302099199894e-08
968 3.21646679443122e-08
969 3.20529273903913e-08
970 3.2095305846247e-08
971 3.20507243747326e-08
972 3.20263189150793e-08
973 3.20348189679631e-08
974 3.1967619644524e-08
975 3.20248964351677e-08
976 3.20039916791304e-08
977 3.1893715982334e-08
978 3.19911593756883e-08
979 3.19075326725704e-08
980 3.19175950397321e-08
981 3.18299088531981e-08
982 3.19064697664739e-08
983 3.18737218074627e-08
984 3.17728646129778e-08
985 3.17811323059525e-08
986 3.18595975260516e-08
987 3.17735149203413e-08
988 3.17355901529037e-08
989 3.17866288431778e-08
990 3.17344587766888e-08
991 3.16544083459602e-08
992 3.1696493013933e-08
993 3.16560740835081e-08
994 3.16454232299757e-08
995 3.16379881947215e-08
996 3.16198828236303e-08
997 3.16209391576505e-08
998 3.15445943780412e-08
999 3.15248153009984e-08
1000 3.15671288561026e-08
1001 3.15653732759769e-08
1002 3.1486298138983e-08
1003 3.14917549429916e-08
1004 3.14959243993274e-08
1005 3.14225086031072e-08
1006 3.14032558722532e-08
1007 3.14510833470116e-08
1008 3.14383059212275e-08
1009 3.13341170298243e-08
1010 3.13988074119864e-08
1011 3.13483433079798e-08
1012 3.13939164707788e-08
1013 3.12686472836443e-08
1014 3.13676214198644e-08
1015 3.12779616873371e-08
1016 3.12579513671585e-08
1017 3.12640433712819e-08
1018 3.12218638576223e-08
1019 3.12321012756467e-08
1020 3.12016049788255e-08
1021 3.12094458021495e-08
1022 3.11465812096179e-08
1023 3.12007417589966e-08
1024 3.10882525766942e-08
1025 3.11018311379696e-08
1026 3.11954421454708e-08
1027 3.11186570736144e-08
1028 3.10208953350211e-08
1029 3.10737896314084e-08
1030 3.10932772425998e-08
1031 3.09805628804227e-08
1032 3.09933690610942e-08
1033 3.09975762153858e-08
1034 3.09390157903211e-08
1035 3.10468650148721e-08
1036 3.09768428226054e-08
1037 3.08869529169753e-08
1038 3.08869133219813e-08
1039 3.09618943051815e-08
1040 3.09093685995077e-08
1041 3.08068660155314e-08
1042 3.09369747369948e-08
1043 3.0878051473282e-08
1044 3.07590795852741e-08
1045 3.08554710258813e-08
1046 3.07873164800965e-08
1047 3.07388783098439e-08
1048 3.07544524845715e-08
1049 3.06944290739963e-08
1050 3.06897345275781e-08
1051 3.06583013779393e-08
1052 3.06929363980135e-08
1053 3.06711412699778e-08
1054 3.06564304216605e-08
1055 3.06191292234192e-08
1056 3.06444950166629e-08
1057 3.05607509575312e-08
1058 3.0580447135109e-08
1059 3.05246333623943e-08
1060 3.0477304955312e-08
1061 3.0575820943235e-08
1062 3.05309383598074e-08
1063 3.04618555789782e-08
1064 3.05053144196155e-08
1065 3.04220337099537e-08
1066 3.04300843568805e-08
1067 3.04257979361067e-08
1068 3.03550047305867e-08
1069 3.03915044418401e-08
1070 3.04102267574269e-08
1071 3.03271673782746e-08
1072 3.03794662062096e-08
1073 3.03242255305092e-08
1074 3.03269385172333e-08
1075 3.02497034732063e-08
1076 3.03163291770492e-08
1077 3.02482073784027e-08
1078 3.02111568007968e-08
1079 3.02886141820835e-08
1080 3.02551260030759e-08
1081 3.01914545656201e-08
1082 3.01636139944383e-08
1083 3.02379236059025e-08
1084 3.01227817113281e-08
1085 3.01962920904453e-08
1086 3.00733019941157e-08
1087 3.01426415787454e-08
1088 3.00421594054967e-08
1089 3.01154378925528e-08
1090 3.00366786084583e-08
1091 3.00879787906716e-08
1092 3.00273977382748e-08
1093 3.00393286244427e-08
1094 3.00176634815408e-08
1095 3.00606073260212e-08
1096 2.99616168625283e-08
1097 2.99584565571775e-08
1098 2.99521959592486e-08
1099 2.99157640587522e-08
1100 2.99434689144951e-08
1101 2.98825295748095e-08
1102 2.99137599989452e-08
1103 2.98795699522625e-08
1104 2.99232847538189e-08
1105 2.98572016484977e-08
1106 2.9836360878055e-08
1107 2.98406288938846e-08
1108 2.98449088882435e-08
1109 2.9782655328825e-08
1110 2.98062649036757e-08
1111 2.97646195347445e-08
1112 2.98135506694441e-08
1113 2.97011274289494e-08
1114 2.97625047388506e-08
1115 2.96845131066981e-08
1116 2.97726100408902e-08
1117 2.96321092487783e-08
1118 2.96821713478002e-08
1119 2.9716878249042e-08
1120 2.96398640508011e-08
1121 2.96724888863009e-08
1122 2.95833488166419e-08
1123 2.96174087337509e-08
1124 2.9592739186679e-08
1125 2.95651498646299e-08
1126 2.96620232833611e-08
1127 2.95969772473592e-08
1128 2.95202518850735e-08
1129 2.95009997280937e-08
1130 2.95273796852014e-08
1131 2.95398222383003e-08
1132 2.94668363826078e-08
1133 2.94502930049401e-08
1134 2.94227303172523e-08
1135 2.94192700049933e-08
1136 2.94126346023216e-08
1137 2.93825748696186e-08
1138 2.94441490842345e-08
1139 2.93326962694795e-08
1140 2.93404019414689e-08
1141 2.93286700242534e-08
1142 2.93694256905397e-08
1143 2.93117682864663e-08
1144 2.93702780267413e-08
1145 2.92960152362687e-08
1146 2.928422151105e-08
1147 2.92368092847939e-08
1148 2.92593428101728e-08
1149 2.92192147587356e-08
1150 2.92315185711489e-08
1151 2.92022013156368e-08
1152 2.91991596904628e-08
1153 2.91639032915514e-08
1154 2.91899871188717e-08
1155 2.91333900651614e-08
1156 2.91412403065072e-08
1157 2.91340528256745e-08
1158 2.90947978860423e-08
1159 2.90943781106057e-08
1160 2.90806061347126e-08
1161 2.90828144485777e-08
1162 2.90560519369265e-08
1163 2.90279897704471e-08
1164 2.90374032483331e-08
1165 2.90260549529187e-08
1166 2.9007222638211e-08
1167 2.90102184120622e-08
1168 2.89753689619587e-08
1169 2.89906160174658e-08
1170 2.89860819278065e-08
1171 2.8949012844337e-08
1172 2.89633214027862e-08
1173 2.89077270528848e-08
1174 2.88916041907283e-08
1175 2.88768312823873e-08
1176 2.8897488423052e-08
1177 2.88625072145665e-08
1178 2.88591106540892e-08
1179 2.88311094110982e-08
1180 2.88150232049533e-08
1181 2.88119562813494e-08
1182 2.88276481029692e-08
1183 2.88237617035447e-08
1184 2.88207255260131e-08
1185 2.87771962448335e-08
1186 2.87729614448784e-08
1187 2.87527129309018e-08
1188 2.87741964974941e-08
1189 2.86909586476591e-08
1190 2.8776898476246e-08
1191 2.87088042256745e-08
1192 2.86769577106938e-08
1193 2.86891367864417e-08
1194 2.86541653324868e-08
1195 2.86352079852392e-08
1196 2.86543447645116e-08
1197 2.86104564672929e-08
1198 2.86138263420055e-08
1199 2.86266872332463e-08
1200 2.85726742980108e-08
1201 2.85416454144372e-08
1202 2.85717673993435e-08
1203 2.8553038802448e-08
1204 2.85341773478276e-08
1205 2.85154230301732e-08
1206 2.84981850203758e-08
1207 2.85097726961947e-08
1208 2.84713140604742e-08
1209 2.85099769048447e-08
1210 2.84635422380664e-08
1211 2.84728872351758e-08
1212 2.84411951245733e-08
1213 2.84294363736004e-08
1214 2.84091777557061e-08
1215 2.83957959807957e-08
1216 2.8379238436127e-08
1217 2.83599242383303e-08
1218 2.83436502513945e-08
1219 2.83070240068017e-08
1220 2.83415255157848e-08
1221 2.8324405676905e-08
1222 2.83105562287611e-08
1223 2.83152038457191e-08
1224 2.82464069751853e-08
1225 2.82652050695997e-08
1226 2.82772347262039e-08
1227 2.82598509151466e-08
1228 2.82610688199236e-08
1229 2.82804956669436e-08
1230 2.8200963249625e-08
1231 2.82574260250801e-08
1232 2.81832370062629e-08
1233 2.8229705213656e-08
1234 2.81566510763609e-08
1235 2.82107533705611e-08
1236 2.81402220972948e-08
1237 2.81743372615439e-08
1238 2.81104629406004e-08
1239 2.81653894381728e-08
1240 2.80918250811224e-08
1241 2.81316291863565e-08
1242 2.80607060836324e-08
1243 2.81200447941332e-08
1244 2.80461593596959e-08
1245 2.8079900619371e-08
1246 2.80145715229052e-08
1247 2.80727756513111e-08
1248 2.79989177079942e-08
1249 2.80361507912774e-08
1250 2.79687878083079e-08
1251 2.80270042579689e-08
1252 2.79496680034574e-08
1253 2.79911132474719e-08
1254 2.79219481047477e-08
1255 2.79818049725433e-08
1256 2.79019826883165e-08
1257 2.7945766227111e-08
1258 2.78766921525397e-08
1259 2.79345646855278e-08
1260 2.78552103144536e-08
1261 2.79021308020599e-08
1262 2.78296202442796e-08
1263 2.78894192775603e-08
1264 2.78064880886442e-08
1265 2.78594493141604e-08
1266 2.77831428473485e-08
1267 2.78419749347725e-08
1268 2.77616593020724e-08
1269 2.78134059854551e-08
1270 2.77378891503366e-08
1271 2.77955472873792e-08
1272 2.77157810575357e-08
1273 2.77698404876903e-08
1274 2.76930041006596e-08
1275 2.77490733618935e-08
1276 2.76696648774388e-08
1277 2.77264224552987e-08
1278 2.76454928614944e-08
1279 2.77050135135681e-08
1280 2.76228329938455e-08
1281 2.76805709160666e-08
1282 2.76009837735236e-08
1283 2.76573484412346e-08
1284 2.75792517032691e-08
1285 2.76307602191661e-08
1286 2.75612711455597e-08
1287 2.75947563229684e-08
1288 2.75462740962906e-08
1289 2.75817330669392e-08
1290 2.75194735582795e-08
1291 2.75538297839217e-08
1292 2.7497293136447e-08
1293 2.75346278230115e-08
1294 2.74708508781751e-08
1295 2.75173654765704e-08
1296 2.7445443819496e-08
1297 2.74968475774129e-08
1298 2.74246834024439e-08
1299 2.74708282457237e-08
1300 2.74036384459997e-08
1301 2.74552608338086e-08
1302 2.73810155816401e-08
1303 2.74218402401827e-08
1304 2.73609989031032e-08
1305 2.74078017229451e-08
1306 2.73360820689517e-08
1307 2.7391245490116e-08
1308 2.73151198942889e-08
1309 2.73574607337901e-08
1310 2.72923740195496e-08
1311 2.7344127197293e-08
1312 2.72695684303548e-08
1313 2.73204605831801e-08
1314 2.72496421799273e-08
1315 2.73030683771713e-08
1316 2.72281425902632e-08
1317 2.72729445238706e-08
1318 2.72050402069235e-08
1319 2.72561982492814e-08
1320 2.7183165894451e-08
1321 2.72326895455199e-08
1322 2.71647935653485e-08
1323 2.72152516810342e-08
1324 2.71396410638225e-08
1325 2.71876178405961e-08
1326 2.7120069418074e-08
1327 2.71676164763646e-08
1328 2.71007673288137e-08
1329 2.71466890356953e-08
1330 2.70781133008269e-08
1331 2.71232721138492e-08
1332 2.70578623846607e-08
1333 2.71025592882834e-08
1334 2.70368072982086e-08
1335 2.70817292087333e-08
1336 2.70154004881062e-08
1337 2.7060075667884e-08
1338 2.6994000692393e-08
1339 2.70393077801412e-08
1340 2.69736275482213e-08
1341 2.70176264267619e-08
1342 2.69518223905418e-08
1343 2.69972840042376e-08
1344 2.6930557227911e-08
1345 2.69762558444686e-08
1346 2.6910671505509e-08
1347 2.69617968112756e-08
1348 2.68871473954047e-08
1349 2.69460857338366e-08
1350 2.68663093022647e-08
1351 2.69099373985116e-08
1352 2.68516759184267e-08
1353 2.68916719738943e-08
1354 2.68386465783754e-08
1355 2.68629268938003e-08
1356 2.68326051403589e-08
1357 2.68330518480298e-08
1358 2.68239598002484e-08
1359 2.67867637332442e-08
1360 2.68111665550075e-08
1361 2.67723963200961e-08
1362 2.67929877358908e-08
1363 2.67475251203297e-08
1364 2.6767066438671e-08
1365 2.67157439023391e-08
1366 2.67541143750272e-08
1367 2.6714586649712e-08
1368 2.6702252333255e-08
1369 2.66839807760988e-08
1370 2.67443577549598e-08
1371 2.67112133200964e-08
1372 2.66426368132899e-08
1373 2.6684054009074e-08
1374 2.66356822687941e-08
1375 2.66503984271971e-08
1376 2.66074278033601e-08
1377 2.66424446810909e-08
1378 2.65923776212595e-08
1379 2.66194399430875e-08
1380 2.65707413051874e-08
1381 2.65988649150728e-08
1382 2.65487552513211e-08
1383 2.65784782902401e-08
1384 2.65304427347068e-08
1385 2.65583279333015e-08
1386 2.65087088024973e-08
1387 2.65368194736215e-08
1388 2.64900101826226e-08
1389 2.6515881483169e-08
1390 2.64746178332276e-08
1391 2.64947731691834e-08
1392 2.64521934322248e-08
1393 2.64673852914399e-08
1394 2.64386107005077e-08
1395 2.64515778083396e-08
1396 2.64131185923056e-08
1397 2.64150877553604e-08
1398 2.64418748040507e-08
1399 2.63694864671704e-08
1400 2.64130478049296e-08
1401 2.63582261420536e-08
1402 2.63718482490516e-08
1403 2.63783180739496e-08
1404 2.63271710676527e-08
1405 2.63258961303858e-08
1406 2.63788358789707e-08
1407 2.63368052764168e-08
1408 2.62823338311735e-08
1409 2.63353077014639e-08
1410 2.62877960423014e-08
1411 2.62965847023233e-08
1412 2.62919735464218e-08
1413 2.62477052813681e-08
1414 2.62937064313684e-08
1415 2.62362904186464e-08
1416 2.62599715110667e-08
1417 2.62191424619251e-08
1418 2.62408405411474e-08
1419 2.6202254513441e-08
1420 2.62240738935216e-08
1421 2.61784811972365e-08
1422 2.62095746064173e-08
1423 2.61560019778617e-08
1424 2.61913828953908e-08
1425 2.61348954559981e-08
1426 2.61744724193713e-08
1427 2.6116043898905e-08
1428 2.61538128779604e-08
1429 2.61001967994945e-08
1430 2.61301891270049e-08
1431 2.60824126869785e-08
1432 2.61117715745218e-08
1433 2.60615122547048e-08
1434 2.60882811067287e-08
1435 2.60453244853576e-08
1436 2.60679863270941e-08
1437 2.60250361194814e-08
1438 2.60364981641281e-08
1439 2.60317618472339e-08
1440 2.59970838918377e-08
1441 2.60176290920455e-08
1442 2.59761283196713e-08
1443 2.59725753324336e-08
1444 2.59454509681234e-08
1445 2.59938803218729e-08
1446 2.59628468354256e-08
1447 2.59396524942002e-08
1448 2.59214781747064e-08
1449 2.59542092015952e-08
1450 2.5902359783947e-08
1451 2.59125832994256e-08
1452 2.58846343379471e-08
1453 2.58667627250908e-08
1454 2.58778549228067e-08
1455 2.59144129556299e-08
1456 2.58706117711727e-08
1457 2.58301527402693e-08
1458 2.58529709513677e-08
1459 2.58587336856575e-08
1460 2.58111839291919e-08
1461 2.57897511624083e-08
1462 2.58444636375144e-08
1463 2.57706280996084e-08
1464 2.58437323178429e-08
1465 2.57411933388685e-08
1466 2.58295400171749e-08
1467 2.5708590662199e-08
1468 2.58644545443332e-08
1469 2.57058189899739e-08
1470 2.57632598493762e-08
1471 2.56932018412037e-08
1472 2.57879563143604e-08
1473 2.56557981525019e-08
1474 2.57281572871859e-08
1475 2.56784632869378e-08
1476 2.57139514258098e-08
1477 2.56263714838445e-08
1478 2.57541064141442e-08
1479 2.5632967230349e-08
1480 2.56729534919842e-08
1481 2.56001854642696e-08
1482 2.56637761487655e-08
1483 2.56262736414437e-08
1484 2.56008464502022e-08
1485 2.55945250274836e-08
1486 2.56324950974607e-08
1487 2.55268015476595e-08
1488 2.56400649619737e-08
1489 2.5511785475274e-08
1490 2.5588155606493e-08
1491 2.55604053644998e-08
1492 2.5496546235293e-08
1493 2.55439167373694e-08
1494 2.56069611535015e-08
1495 2.54183406566133e-08
1496 2.55189924188715e-08
1497 2.54879632191063e-08
1498 2.55330767762185e-08
1499 2.54071646060838e-08
1500 2.55215696295386e-08
1501 2.54048431337361e-08
1502 2.55306843680581e-08
1503 2.54071508200004e-08
1504 2.54499430548671e-08
1505 2.53732500697934e-08
1506 2.54715429618413e-08
1507 2.53949658837183e-08
1508 2.54022959664812e-08
1509 2.5337775912071e-08
1510 2.53975791395522e-08
1511 2.53854276021093e-08
1512 2.53148731619302e-08
1513 2.53830941191469e-08
1514 2.52940477677432e-08
1515 2.53858050117683e-08
1516 2.52729588808842e-08
1517 2.53631105315844e-08
1518 2.53016018926244e-08
1519 2.53141742055929e-08
1520 2.52379139999981e-08
1521 2.52926429534783e-08
1522 2.52654074619496e-08
1523 2.52658417766449e-08
1524 2.52822283154375e-08
1525 2.51930612603646e-08
1526 2.52585414808326e-08
1527 2.51862574285333e-08
1528 2.52442949033593e-08
1529 2.51939847318861e-08
1530 2.52133488415351e-08
1531 2.5146940347498e-08
1532 2.5189434449846e-08
1533 2.52195610358497e-08
1534 2.51246561175922e-08
1535 2.51449200679987e-08
1536 2.51359889548164e-08
1537 2.51350627797908e-08
1538 2.51131939722482e-08
1539 2.51892622168404e-08
1540 2.50857870407462e-08
1541 2.50648899894346e-08
1542 2.50798451765233e-08
1543 2.50954943487036e-08
1544 2.50775393489944e-08
1545 2.50697572462544e-08
1546 2.50376686403708e-08
1547 2.50493322330181e-08
1548 2.50432101888132e-08
1549 2.50288159844292e-08
1550 2.50035105238133e-08
1551 2.50328429669544e-08
1552 2.49807138836822e-08
1553 2.502240505875e-08
1554 2.49428952905362e-08
1555 2.49643558949408e-08
1556 2.49484237899722e-08
1557 2.49800373711695e-08
1558 2.49698211025606e-08
1559 2.49006044085176e-08
1560 2.49190906086438e-08
1561 2.492241703278e-08
1562 2.4917836957794e-08
1563 2.49022912134356e-08
1564 2.48888761958943e-08
1565 2.48946457753751e-08
1566 2.48678545170522e-08
1567 2.48490648929733e-08
1568 2.48651622113405e-08
1569 2.48721701537891e-08
1570 2.48229730286376e-08
1571 2.48236879019093e-08
1572 2.48155169743081e-08
1573 2.48312572341813e-08
1574 2.47911966338998e-08
1575 2.47896219611743e-08
1576 2.48247217071729e-08
1577 2.4789742736897e-08
1578 2.47587817768791e-08
1579 2.47566577116221e-08
1580 2.47695303657869e-08
1581 2.47676433904331e-08
1582 2.47186628534735e-08
1583 2.47210474436654e-08
1584 2.47467018520409e-08
1585 2.47340699378595e-08
1586 2.46807572601426e-08
1587 2.46916466211111e-08
1588 2.46927217215642e-08
1589 2.46772598100353e-08
1590 2.46556300977918e-08
1591 2.46516507306227e-08
1592 2.4683439244777e-08
1593 2.46173112868897e-08
1594 2.46435963219271e-08
1595 2.46323588799502e-08
1596 2.46131268112348e-08
1597 2.46211937391383e-08
1598 2.46009540252334e-08
1599 2.46136557497945e-08
1600 2.4564189781362e-08
1601 2.45825033888813e-08
1602 2.45738521279515e-08
1603 2.45586649976204e-08
1604 2.45413155487428e-08
1605 2.45586444036494e-08
1606 2.45448200187903e-08
1607 2.4538104840266e-08
1608 2.45132840138718e-08
1609 2.45314540627817e-08
1610 2.4477180243232e-08
1611 2.45142686903232e-08
1612 2.44904568151849e-08
1613 2.44784891881844e-08
1614 2.44764097784289e-08
1615 2.44349816589384e-08
1616 2.44733197899416e-08
1617 2.44558140167772e-08
1618 2.4436141651929e-08
1619 2.44235684290217e-08
1620 2.44064701292457e-08
1621 2.44143504186667e-08
1622 2.44237251668666e-08
1623 2.43914046300642e-08
1624 2.43962299066869e-08
1625 2.43861623472341e-08
1626 2.43528411302263e-08
1627 2.43726402904265e-08
1628 2.43248926481421e-08
1629 2.43622326043802e-08
1630 2.43396557834119e-08
1631 2.43319489714455e-08
1632 2.43299890653192e-08
1633 2.4320137978262e-08
1634 2.42750099982425e-08
1635 2.43130430689353e-08
1636 2.42814252702228e-08
1637 2.42732570094883e-08
1638 2.4283183536089e-08
1639 2.42604835165583e-08
1640 2.42631818300199e-08
1641 2.42471218552254e-08
1642 2.42472986214937e-08
1643 2.42233293438199e-08
1644 2.42291746970213e-08
1645 2.42277743290886e-08
1646 2.41999115103697e-08
1647 2.41952465226403e-08
1648 2.41828789345755e-08
1649 2.41910330147643e-08
1650 2.41750495301174e-08
1651 2.41844695354487e-08
1652 2.41217843396369e-08
1653 2.41458189131905e-08
1654 2.41428828139378e-08
1655 2.41465379410277e-08
1656 2.40961741021506e-08
1657 2.41304223909111e-08
1658 2.41070129914922e-08
1659 2.41133785418013e-08
1660 2.40634410960672e-08
1661 2.41115761269173e-08
1662 2.40546281353726e-08
1663 2.40622293019577e-08
1664 2.40146637668914e-08
1665 2.40711095388368e-08
1666 2.40442927463835e-08
1667 2.40265189690447e-08
1668 2.40145833291239e-08
1669 2.40169209116958e-08
1670 2.39657530711757e-08
1671 2.39975835256701e-08
1672 2.3963043397357e-08
1673 2.39960674744966e-08
1674 2.39489825102357e-08
1675 2.39718424329682e-08
1676 2.39497344093342e-08
1677 2.39381356071933e-08
1678 2.39413603558836e-08
1679 2.39382225836193e-08
1680 2.39105953427909e-08
1681 2.39114264092288e-08
1682 2.38930162094197e-08
1683 2.38964408800779e-08
1684 2.38833490370238e-08
1685 2.38706949912126e-08
1686 2.39069079881471e-08
1687 2.38462499455583e-08
1688 2.38514573582815e-08
1689 2.38599120515603e-08
1690 2.38461797252842e-08
1691 2.38494313620219e-08
1692 2.38113997472755e-08
1693 2.38570670304528e-08
1694 2.38001375950647e-08
1695 2.37886541734067e-08
1696 2.38048604839225e-08
1697 2.38387524448269e-08
1698 2.37525258979243e-08
1699 2.37582233083566e-08
1700 2.37725309131243e-08
1701 2.37826007470909e-08
1702 2.37138977824403e-08
1703 2.37592891455529e-08
1704 2.37286466179265e-08
1705 2.37121058112022e-08
1706 2.37209313950215e-08
1707 2.36918174987277e-08
1708 2.37369839671819e-08
1709 2.36887154468901e-08
1710 2.36741353710768e-08
1711 2.36932602725171e-08
1712 2.3690040995894e-08
1713 2.36387473249167e-08
1714 2.3682765228461e-08
1715 2.3649693638661e-08
1716 2.36441239231944e-08
1717 2.36267354649655e-08
1718 2.36182507041827e-08
1719 2.36523788390564e-08
1720 2.35820753384797e-08
1721 2.36204386727668e-08
1722 2.35822874260494e-08
1723 2.35878673257028e-08
1724 2.35643411545805e-08
1725 2.35942504002162e-08
1726 2.3538003506185e-08
1727 2.35547883120635e-08
1728 2.35376234052342e-08
1729 2.35661270220344e-08
1730 2.35034005133805e-08
1731 2.35470375707925e-08
1732 2.34974159751822e-08
1733 2.35213333269479e-08
1734 2.34801640012217e-08
1735 2.34998519097474e-08
1736 2.3492742331932e-08
1737 2.34870242135088e-08
1738 2.34516222111791e-08
1739 2.35031180259027e-08
1740 2.34120568253227e-08
1741 2.34892399996189e-08
1742 2.34083193575429e-08
1743 2.34671604544268e-08
1744 2.34010312089028e-08
1745 2.34173997678555e-08
1746 2.34056900998425e-08
1747 2.34292432421679e-08
1748 2.33731811104532e-08
1749 2.33763542500931e-08
1750 2.33961526298065e-08
1751 2.33842702136444e-08
1752 2.33293924126743e-08
1753 2.33839316214901e-08
1754 2.33554255231194e-08
1755 2.33420207882418e-08
1756 2.33356180516786e-08
1757 2.33335749280084e-08
1758 2.33027156899546e-08
1759 2.33244593029758e-08
1760 2.33021578498471e-08
1761 2.3313084637544e-08
1762 2.32565405312579e-08
1763 2.33168984556942e-08
1764 2.32298045199641e-08
1765 2.32831202021666e-08
1766 2.32656135545906e-08
1767 2.32703309193116e-08
1768 2.32378402780231e-08
1769 2.32466926570751e-08
1770 2.32094725896026e-08
1771 2.32405336739738e-08
1772 2.31986852637922e-08
1773 2.32274944147459e-08
1774 2.31874142364585e-08
1775 2.32149120880099e-08
1776 2.31851392901161e-08
1777 2.31927146354671e-08
1778 2.31358060995834e-08
1779 2.31842640040503e-08
1780 2.31418105667291e-08
1781 2.31669153517577e-08
1782 2.31163374717624e-08
1783 2.31599354723055e-08
1784 2.31054345266601e-08
1785 2.3140252213838e-08
1786 2.30880335381212e-08
1787 2.31249165778546e-08
1788 2.308726833955e-08
1789 2.31138644228857e-08
1790 2.30495659109264e-08
1791 2.31006559990865e-08
1792 2.30422045330858e-08
1793 2.30772343987606e-08
1794 2.30475943042707e-08
1795 2.30681594128868e-08
1796 2.30005245333009e-08
1797 2.30601321956625e-08
1798 2.2992266645705e-08
1799 2.30368094107369e-08
1800 2.29975887851008e-08
1801 2.3017819156812e-08
1802 2.29819369793605e-08
1803 2.30075660928586e-08
1804 2.29481920833718e-08
1805 2.29957483125487e-08
1806 2.29446615221951e-08
1807 2.29808585283653e-08
1808 2.29182354715141e-08
1809 2.29665052989825e-08
1810 2.29168229375354e-08
1811 2.29496333016277e-08
1812 2.28981681672513e-08
1813 2.29338247298427e-08
1814 2.28899680414507e-08
1815 2.29195899850154e-08
1816 2.28720363382173e-08
1817 2.29037124077625e-08
1818 2.28588492550141e-08
1819 2.2889778634072e-08
1820 2.28444056609156e-08
1821 2.28755266730163e-08
1822 2.28335176850614e-08
1823 2.28608122194718e-08
1824 2.28163847386664e-08
1825 2.28473173243815e-08
1826 2.28062247411565e-08
1827 2.28333931254765e-08
1828 2.27904469749518e-08
1829 2.28173798371145e-08
1830 2.27835949446753e-08
1831 2.28003338899052e-08
1832 2.27673431554276e-08
1833 2.27831832099179e-08
1834 2.27632154643498e-08
1835 2.27655942086624e-08
1836 2.27460333503959e-08
1837 2.27520244558965e-08
1838 2.27261886566188e-08
1839 2.27413570684787e-08
1840 2.27322526826779e-08
1841 2.27128195907644e-08
1842 2.27132873614755e-08
1843 2.27083153779839e-08
1844 2.26981406523663e-08
1845 2.26871346286073e-08
1846 2.26848630356935e-08
1847 2.26928114658298e-08
1848 2.26541671703995e-08
1849 2.26940951155719e-08
1850 2.26081935461586e-08
1851 2.26670784555516e-08
1852 2.26381892689531e-08
1853 2.26379104462104e-08
1854 2.26317797732412e-08
1855 2.26263871407895e-08
1856 2.26040701489794e-08
1857 2.26210371535585e-08
1858 2.25969976845874e-08
1859 2.26033115316993e-08
1860 2.25786682824802e-08
1861 2.25913581006765e-08
1862 2.25696883440207e-08
1863 2.25769364757822e-08
1864 2.25467265979606e-08
1865 2.2565178211309e-08
1866 2.25428093565405e-08
1867 2.25414282655212e-08
1868 2.25282127639392e-08
1869 2.25295804801018e-08
1870 2.25292637652252e-08
1871 2.25068205409462e-08
1872 2.25173038438387e-08
1873 2.24886506217103e-08
1874 2.25077479951707e-08
1875 2.24700938299405e-08
1876 2.24972172288007e-08
1877 2.24615193279432e-08
1878 2.24729500328813e-08
1879 2.2449316779305e-08
1880 2.24757804595521e-08
1881 2.24325223623367e-08
1882 2.24439117045838e-08
1883 2.24287491995545e-08
1884 2.24217320603515e-08
1885 2.24241532174041e-08
1886 2.24109995654587e-08
1887 2.24082915626367e-08
1888 2.23931373712905e-08
1889 2.24117247780153e-08
1890 2.2372305517826e-08
1891 2.23956851128637e-08
1892 2.23616559649198e-08
1893 2.23651333898234e-08
1894 2.23647490021905e-08
1895 2.23450119479773e-08
1896 2.23464230769688e-08
1897 2.23417689014793e-08
1898 2.23371286617313e-08
1899 2.23235108403053e-08
1900 2.23223212926227e-08
1901 2.23091417703714e-08
1902 2.22958896444192e-08
1903 2.23107210347573e-08
1904 2.22995549878924e-08
1905 2.22775507313644e-08
1906 2.22906983117221e-08
1907 2.22668443071727e-08
1908 2.22543186013668e-08
1909 2.22780352837626e-08
1910 2.22379670026251e-08
1911 2.22480200419151e-08
1912 2.22505137740248e-08
1913 2.22260662430251e-08
1914 2.22166666348222e-08
1915 2.22366415678632e-08
1916 2.22074011044793e-08
1917 2.22039356201353e-08
1918 2.22020418674251e-08
1919 2.21912392471824e-08
1920 2.21826895847599e-08
1921 2.21884106254011e-08
1922 2.21797555274295e-08
1923 2.21625245120372e-08
1924 2.21647159716065e-08
1925 2.21556553735569e-08
1926 2.21580461304827e-08
1927 2.21355967184866e-08
1928 2.21416850882949e-08
1929 2.21241215109247e-08
1930 2.21263815942674e-08
1931 2.21141203669006e-08
1932 2.21158009965361e-08
1933 2.20994018340637e-08
1934 2.21028089986142e-08
1935 2.20883868771171e-08
1936 2.2079992008095e-08
1937 2.20784564195498e-08
1938 2.20784151186981e-08
1939 2.20586177797077e-08
1940 2.20650800628608e-08
1941 2.20486972187794e-08
1942 2.20505990738973e-08
1943 2.20378017964373e-08
1944 2.20325296038659e-08
1945 2.20286499702516e-08
1946 2.20192976524425e-08
1947 2.20145179199438e-08
1948 2.20081947536199e-08
1949 2.20011602257708e-08
1950 2.19993929956308e-08
1951 2.19850873457439e-08
1952 2.19833163054295e-08
1953 2.1978743948603e-08
1954 2.19710677492602e-08
1955 2.1963683684012e-08
1956 2.19570959660853e-08
1957 2.19529090820236e-08
1958 2.19484774724998e-08
1959 2.19370060766666e-08
1960 2.19370169186606e-08
1961 2.19260845992775e-08
1962 2.19226941522432e-08
1963 2.19154773041241e-08
1964 2.19098743720059e-08
1965 2.1901946458458e-08
1966 2.19000646114464e-08
1967 2.18924472686899e-08
1968 2.18879824782636e-08
1969 2.18770404523116e-08
1970 2.1878113941165e-08
1971 2.18639915889884e-08
1972 2.18656095911651e-08
1973 2.18523012425198e-08
1974 2.18533016269618e-08
1975 2.1841532375777e-08
1976 2.18427513763997e-08
1977 2.18286352056118e-08
1978 2.18293231373234e-08
1979 2.18180280088243e-08
1980 2.18144673886922e-08
1981 2.18064747649627e-08
1982 2.18049071061266e-08
1983 2.17917752833308e-08
1984 2.1795977534933e-08
1985 2.17811728318029e-08
1986 2.1781566768353e-08
1987 2.17674100796983e-08
1988 2.17706055019029e-08
1989 2.17568857788231e-08
1990 2.1757892851082e-08
1991 2.17456928955362e-08
1992 2.1746234932174e-08
1993 2.17348543754547e-08
1994 2.1730707703782e-08
1995 2.17234175538539e-08
1996 2.17224528494264e-08
1997 2.1710373501016e-08
1998 2.170898094217e-08
1999 2.17013210115913e-08
2000 2.16946421386921e-08
2001 2.16891542015318e-08
2002 2.16823361416929e-08
2003 2.16778977183973e-08
2004 2.16708460858972e-08
2005 2.16666359945106e-08
2006 2.16551223376493e-08
2007 2.16575499916716e-08
2008 2.16449522191242e-08
2009 2.16435213244992e-08
2010 2.16348713197312e-08
2011 2.16302482637931e-08
2012 2.16222755899387e-08
2013 2.16224539190124e-08
2014 2.16069732182356e-08
2015 2.16113775033944e-08
2016 2.15954091181136e-08
2017 2.16008306057613e-08
2018 2.15833294472945e-08
2019 2.15865595325493e-08
2020 2.15733391610295e-08
2021 2.15769943359145e-08
2022 2.15569672694094e-08
2023 2.15671010795893e-08
2024 2.15474811673388e-08
2025 2.15541972032329e-08
2026 2.15345627319619e-08
2027 2.15410196280352e-08
2028 2.15237571352112e-08
2029 2.15345168727565e-08
2030 2.15106681566435e-08
2031 2.15253645506341e-08
2032 2.15001349516686e-08
2033 2.1517920305747e-08
2034 2.1487891494032e-08
2035 2.15120834391458e-08
2036 2.14737649384955e-08
2037 2.15061533565541e-08
2038 2.1452487630802e-08
2039 2.15034829514349e-08
2040 2.14346767141715e-08
2041 2.14966143057826e-08
2042 2.14179069170095e-08
2043 2.14907325161162e-08
2044 2.13990770208672e-08
2045 2.14852289456835e-08
2046 2.13866388427686e-08
2047 2.14730014042597e-08
2048 2.1372316124868e-08
2049 2.14655068129233e-08
2050 2.13597741495763e-08
2051 2.14546777566049e-08
2052 2.13450150890471e-08
2053 2.14445914035188e-08
2054 2.13313594377151e-08
2055 2.14372586469502e-08
2056 2.13196717945285e-08
2057 2.14247360211806e-08
2058 2.13055624451175e-08
2059 2.1417804491386e-08
2060 2.1289703296512e-08
2061 2.14093997233378e-08
2062 2.12762339632522e-08
2063 2.14022187951524e-08
2064 2.12640625206761e-08
2065 2.13917148988441e-08
2066 2.12494448128542e-08
2067 2.13845962904746e-08
2068 2.12352076048306e-08
2069 2.13696356577975e-08
2070 2.12271335868208e-08
2071 2.13581918119821e-08
2072 2.12141472600291e-08
2073 2.13496626846332e-08
2074 2.11955548796716e-08
2075 2.13416086363716e-08
2076 2.11903023122351e-08
2077 2.1325016231144e-08
2078 2.11869120060326e-08
2079 2.13060317381109e-08
2080 2.11849879761528e-08
2081 2.13014843066239e-08
2082 2.11548260946603e-08
2083 2.12917265480228e-08
2084 2.11498460532256e-08
2085 2.12714030898842e-08
2086 2.11624385548781e-08
2087 2.12514721817381e-08
2088 2.1154099666687e-08
2089 2.12416453538311e-08
2090 2.11203298183804e-08
2091 2.12520081929735e-08
2092 2.10877881657301e-08
2093 2.12523132180897e-08
2094 2.10677346701438e-08
2095 2.1239609513779e-08
2096 2.10663556415303e-08
2097 2.12244035631137e-08
2098 2.10592911085161e-08
2099 2.1214369558431e-08
2100 2.10455527161479e-08
2101 2.12056012817707e-08
2102 2.10277354591448e-08
2103 2.11964243952423e-08
2104 2.10108092252304e-08
2105 2.11797714728124e-08
2106 2.10208233536457e-08
2107 2.11387284829168e-08
2108 2.10561097280237e-08
2109 2.11106185262211e-08
2110 2.1055611373999e-08
2111 2.10913165926696e-08
2112 2.1049674885143e-08
2113 2.10730777337464e-08
2114 2.10520934743053e-08
2115 2.10548893129214e-08
2116 2.10402075451754e-08
2117 2.1042120524728e-08
2118 2.10419781482285e-08
2119 2.10340610717896e-08
2120 2.10044068583159e-08
2121 2.10191583012165e-08
2122 2.10327447889846e-08
2123 2.10001597678033e-08
2124 2.09956320718074e-08
2125 2.09977085903201e-08
2126 2.09895200322663e-08
2127 2.09962008227982e-08
2128 2.0975949219848e-08
2129 2.09803473124603e-08
2130 2.09804932109758e-08
2131 2.0963027486498e-08
2132 2.09587070795769e-08
2133 2.09524758458257e-08
2134 2.0966915682985e-08
2135 2.09296073316323e-08
2136 2.09548540843763e-08
2137 2.09230599195753e-08
2138 2.09380025809613e-08
2139 2.09187802638344e-08
2140 2.09233516796914e-08
2141 2.09114757286355e-08
2142 2.09236198106533e-08
2143 2.08941167633703e-08
2144 2.0903951723994e-08
2145 2.08885771804113e-08
2146 2.08923854395082e-08
2147 2.08741145277247e-08
2148 2.09078572265264e-08
2149 2.08621026300815e-08
2150 2.08659720940529e-08
2151 2.08878689313963e-08
2152 2.0846455313972e-08
2153 2.08525394605474e-08
2154 2.0893355698326e-08
2155 2.08270954066836e-08
2156 2.08490474539746e-08
2157 2.0825891786258e-08
2158 2.08589807608184e-08
2159 2.08129046647132e-08
2160 2.08226377044762e-08
2161 2.08088298225273e-08
2162 2.08534508686076e-08
2163 2.07827600427479e-08
2164 2.08175104757791e-08
2165 2.07907038758948e-08
2166 2.08185683551743e-08
2167 2.07710595475086e-08
2168 2.07745765646306e-08
2169 2.07851797173753e-08
2170 2.08141469426049e-08
2171 2.07543042527458e-08
2172 2.07562184099119e-08
2173 2.07568173172246e-08
2174 2.07656165897019e-08
2175 2.07642636098626e-08
2176 2.07334991307873e-08
2177 2.07385574499952e-08
2178 2.07278763918461e-08
2179 2.07315932505558e-08
2180 2.07728416164454e-08
2181 2.07035185289528e-08
2182 2.07399692190302e-08
2183 2.0697155476368e-08
2184 2.07194928459331e-08
2185 2.07134398411135e-08
2186 2.0682477093581e-08
2187 2.07077737356953e-08
2188 2.06824222777624e-08
2189 2.06908440539832e-08
2190 2.06802450566923e-08
2191 2.07249368427975e-08
2192 2.06152115067026e-08
2193 2.0748893895306e-08
2194 2.0596830709485e-08
2195 2.07452802542529e-08
2196 2.05545009497876e-08
2197 2.07414181962462e-08
2198 2.05675702448782e-08
2199 2.06980567742931e-08
2200 2.06142689797573e-08
2201 2.06187567634575e-08
2202 2.0644910309342e-08
2203 2.06083580055472e-08
2204 2.0632301118928e-08
2205 2.05880383331025e-08
2206 2.0657039231653e-08
2207 2.05427682670178e-08
2208 2.06383996603776e-08
2209 2.05416500415612e-08
2210 2.06211282883229e-08
2211 2.06406356336175e-08
2212 2.04871705956622e-08
2213 2.06637532750298e-08
2214 2.0512486405555e-08
2215 2.0594793426143e-08
2216 2.05510627566685e-08
2217 2.05330016339045e-08
2218 2.05944446359241e-08
2219 2.05083332362466e-08
2220 2.05767475437435e-08
2221 2.04617908711735e-08
2222 2.06046240706637e-08
2223 2.04842653440118e-08
2224 2.0513395774302e-08
2225 2.05650318763628e-08
2226 2.04392590956171e-08
2227 2.05927411476292e-08
2228 2.0428847707088e-08
2229 2.0532347291935e-08
2230 2.04898007293841e-08
2231 2.04550190526787e-08
2232 2.05370652396897e-08
2233 2.04499561275995e-08
2234 2.04969391436327e-08
2235 2.04444076309374e-08
2236 2.04955986866651e-08
2237 2.04616261330615e-08
2238 2.04724767759457e-08
2239 2.04162005731923e-08
2240 2.04643462624055e-08
2241 2.04814649774621e-08
2242 2.0337858317232e-08
2243 2.05245140217913e-08
2244 2.03081969068486e-08
2245 2.0510237259852e-08
2246 2.04139055140495e-08
2247 2.04103785301779e-08
2248 2.04396460980472e-08
2249 2.03633557394611e-08
2250 2.04684294560775e-08
2251 2.031334028052e-08
2252 2.0473238938723e-08
2253 2.03142812069679e-08
2254 2.04624848931245e-08
2255 2.02915782868129e-08
2256 2.04611496755214e-08
2257 2.02902796703208e-08
2258 2.04460661480943e-08
2259 2.02521207705031e-08
2260 2.04494011661338e-08
2261 2.03587672275574e-08
2262 2.02795527478039e-08
2263 2.04119608190334e-08
2264 2.03550036380618e-08
2265 2.0338984402124e-08
2266 2.03569044046059e-08
2267 2.03014055591422e-08
2268 2.03774343641161e-08
2269 2.02380104844524e-08
2270 2.04106485880984e-08
2271 2.01957923666729e-08
2272 2.04286760751016e-08
2273 2.01882030579914e-08
2274 2.03728088036814e-08
2275 2.02680649185605e-08
2276 2.03692853381066e-08
2277 2.0184359766251e-08
2278 2.03947855955122e-08
2279 2.01674845824451e-08
2280 2.03720353624814e-08
2281 2.01736453774859e-08
2282 2.03596099915271e-08
2283 2.01790565432236e-08
2284 2.03254818427756e-08
2285 2.01922910366137e-08
2286 2.03466658028062e-08
2287 2.01453734473556e-08
2288 2.03415745128055e-08
2289 2.01642573440797e-08
2290 2.03211244992896e-08
2291 2.01173527344944e-08
2292 2.03594045274369e-08
2293 2.01331584254727e-08
2294 2.03068763381387e-08
2295 2.01318424870589e-08
2296 2.02937204167308e-08
2297 2.00904794941947e-08
2298 2.02801864937485e-08
2299 2.01300278280803e-08
2300 2.0295876560994e-08
2301 2.00940146565243e-08
2302 2.02835993443062e-08
2303 2.00904099653676e-08
2304 2.02543154296797e-08
2305 2.01039614343701e-08
2306 2.0282950775774e-08
2307 2.00342223473093e-08
2308 2.02554112599396e-08
2309 2.01435948090078e-08
2310 2.01330314787418e-08
2311 2.02083632954708e-08
2312 2.00668100076573e-08
2313 2.02039738779192e-08
2314 2.00946381490597e-08
2315 2.01763274041422e-08
2316 2.01540125592259e-08
2317 2.00669482343652e-08
2318 2.01701365937779e-08
2319 2.00826732209003e-08
2320 2.01606988451952e-08
2321 2.00336450858485e-08
2322 2.01584586095316e-08
2323 2.00755442088529e-08
2324 2.01398786845441e-08
2325 2.00542164524031e-08
2326 2.01231333420426e-08
2327 2.0034502215549e-08
2328 2.01729211053436e-08
2329 2.0050683691708e-08
2330 2.01079022422723e-08
2331 2.00718693433299e-08
2332 2.01212576036447e-08
2333 1.99601781976644e-08
2334 2.02007976548013e-08
2335 1.99725146242669e-08
2336 2.01246614376926e-08
2337 1.99519223191835e-08
2338 2.0181543013309e-08
2339 1.99515154759022e-08
2340 2.0134716171738e-08
2341 1.99417249183709e-08
2342 2.01134434790751e-08
2343 1.99199673804329e-08
2344 2.01981070147572e-08
2345 1.98708658228908e-08
2346 2.01594137974603e-08
2347 1.9870977772396e-08
2348 2.01408327842323e-08
2349 1.9908012500014e-08
2350 2.01253628401843e-08
2351 1.99312719929012e-08
2352 2.0074211177612e-08
2353 1.98415951241926e-08
2354 2.01695905357457e-08
2355 1.98183922431872e-08
2356 2.01227382437552e-08
2357 1.98570779031959e-08
2358 2.00900091271716e-08
2359 1.98387703747027e-08
2360 2.01344670806569e-08
2361 1.98181906032602e-08
2362 2.00588448959205e-08
2363 1.99098924680285e-08
2364 2.00330409387828e-08
2365 1.99065683517685e-08
2366 2.0020223941708e-08
2367 1.98548218477423e-08
2368 2.00035548888478e-08
2369 1.98835546494358e-08
2370 1.99797192278472e-08
2371 1.98375348439694e-08
2372 2.00875848405113e-08
2373 1.9812472268288e-08
2374 2.00332153326155e-08
2375 1.98212271592113e-08
2376 2.00332581465901e-08
2377 1.98204904267585e-08
2378 1.9984234975734e-08
2379 1.98347139152566e-08
2380 1.99612496130097e-08
2381 1.97707618248e-08
2382 2.0004315585409e-08
2383 1.98095061997861e-08
2384 1.99927521510879e-08
2385 1.98515250393605e-08
2386 1.9937375159651e-08
2387 1.98117068513359e-08
2388 2.00240222073544e-08
2389 1.98210552741496e-08
2390 1.98711810269203e-08
2391 1.9911171370679e-08
2392 1.97619772748236e-08
2393 1.98856045410256e-08
2394 1.97860703683994e-08
2395 1.99236490522892e-08
2396 1.98624784943369e-08
2397 1.98943457216938e-08
2398 1.97388916371999e-08
2399 1.99018316417998e-08
2400 1.97790956841803e-08
2401 1.98262917385228e-08
2402 1.98531879463948e-08
2403 1.99187255027211e-08
2404 1.97463656866903e-08
2405 1.99253557424672e-08
2406 1.97354116843185e-08
2407 1.9871531546034e-08
2408 1.9757216573324e-08
2409 1.98985119077566e-08
2410 1.96953672622069e-08
2411 1.98574453397216e-08
2412 1.97577958950812e-08
2413 1.98949042393703e-08
2414 1.97052937777609e-08
2415 1.98619443207426e-08
2416 1.97361344833058e-08
2417 1.98387085071916e-08
2418 1.97285542435921e-08
2419 1.98373499747384e-08
2420 1.97584781537197e-08
2421 1.97573963304776e-08
2422 1.97606191897903e-08
2423 1.97702470755012e-08
2424 1.97484902940137e-08
2425 1.97685302073536e-08
2426 1.97438523346083e-08
2427 1.97694510641888e-08
2428 1.97210327158581e-08
2429 1.98112495961578e-08
2430 1.97273666380227e-08
2431 1.96890601471544e-08
2432 1.97880925085214e-08
2433 1.96778318666269e-08
2434 1.97789766865353e-08
2435 1.96626723901749e-08
2436 1.97849776314429e-08
2437 1.96600382569856e-08
2438 1.97538982802947e-08
2439 1.97404544735602e-08
2440 1.97059592891291e-08
2441 1.96471097941298e-08
2442 1.97557214598465e-08
2443 1.96330823465374e-08
2444 1.97589247991625e-08
2445 1.96238234317359e-08
2446 1.97408875645122e-08
2447 1.96832157430271e-08
2448 1.96455369561588e-08
2449 1.96642841972516e-08
2450 1.97671942240296e-08
2451 1.96047371466834e-08
2452 1.96802576052924e-08
2453 1.96413525949124e-08
2454 1.96575819984002e-08
2455 1.96412271520341e-08
2456 1.97347118503499e-08
2457 1.95864023340153e-08
2458 1.96448353597667e-08
2459 1.96706182964967e-08
2460 1.96352642102826e-08
2461 1.95829593013785e-08
2462 1.96948573565336e-08
2463 1.95951352158219e-08
2464 1.95952127419741e-08
2465 1.96090318411168e-08
2466 1.96445711426185e-08
2467 1.95816276935545e-08
2468 1.96074178548389e-08
2469 1.96659882491512e-08
2470 1.95661929521806e-08
2471 1.95709208266059e-08
2472 1.95860839728446e-08
2473 1.96185910507074e-08
2474 1.95628886445753e-08
2475 1.95785846187624e-08
2476 1.95880638035884e-08
2477 1.95689454794978e-08
2478 1.9573021400543e-08
2479 1.95984938505811e-08
2480 1.95604113755188e-08
2481 1.95136103741667e-08
2482 1.96358453684597e-08
2483 1.95100726530506e-08
2484 1.95278302873025e-08
2485 1.95718225858843e-08
2486 1.9504452592356e-08
2487 1.95661599375896e-08
2488 1.9514066734172e-08
2489 1.95434963043417e-08
2490 1.95293351885617e-08
2491 1.95453441831361e-08
2492 1.94574475179121e-08
2493 1.95738759209974e-08
2494 1.94652277153429e-08
2495 1.95368487910796e-08
2496 1.94861349038788e-08
2497 1.95077287675205e-08
2498 1.94804002138227e-08
2499 1.95582869801925e-08
2500 1.94696726175669e-08
2501 1.94532139711212e-08
2502 1.95258487275529e-08
2503 1.94536869114748e-08
2504 1.94936886298014e-08
2505 1.9448176115322e-08
2506 1.95066871261917e-08
2507 1.94318203075028e-08
2508 1.94778451081401e-08
2509 1.94615272485765e-08
2510 1.94557352452551e-08
2511 1.94748104237474e-08
2512 1.94140049046276e-08
2513 1.94695835331604e-08
2514 1.94537524078608e-08
2515 1.93956528821149e-08
2516 1.9481855360215e-08
2517 1.93936476032275e-08
2518 1.94787785729456e-08
2519 1.93719459350983e-08
2520 1.95034980666198e-08
2521 1.93978687406671e-08
2522 1.94033924676984e-08
2523 1.94093188343736e-08
2524 1.94171115584507e-08
2525 1.93997879078811e-08
2526 1.94405241474849e-08
2527 1.93229404543138e-08
2528 1.94759484733464e-08
2529 1.93110648613604e-08
2530 1.94577155224751e-08
2531 1.9324493209294e-08
2532 1.94209693799574e-08
2533 1.93679247881828e-08
2534 1.93712790353961e-08
2535 1.93870359157522e-08
2536 1.9329113803479e-08
2537 1.94043796891696e-08
2538 1.94204160605116e-08
2539 1.92999184137621e-08
2540 1.93433329177539e-08
2541 1.94136715080395e-08
2542 1.93267493198701e-08
2543 1.92983839465422e-08
2544 1.93762269458109e-08
2545 1.93832424123519e-08
2546 1.92928241902202e-08
2547 1.93194722795664e-08
2548 1.9341684259877e-08
2549 1.92990857024733e-08
2550 1.94216795368907e-08
2551 1.92712975501941e-08
2552 1.93287515893314e-08
2553 1.9254477589048e-08
2554 1.93462119125187e-08
2555 1.92636152797632e-08
2556 1.93235599760255e-08
2557 1.92968689519124e-08
2558 1.92690090572989e-08
2559 1.93095283022027e-08
2560 1.92884365233792e-08
2561 1.93475634913676e-08
2562 1.92706266788911e-08
2563 1.92535125757565e-08
2564 1.92429318716103e-08
2565 1.93122842179205e-08
2566 1.92811584630026e-08
2567 1.92013131457003e-08
2568 1.93534934537221e-08
2569 1.91937474798287e-08
2570 1.92372362915472e-08
2571 1.92665299901806e-08
2572 1.92173671904605e-08
2573 1.93352516392742e-08
2574 1.91593442574467e-08
2575 1.92285983012241e-08
2576 1.92591757726679e-08
2577 1.91782408964225e-08
2578 1.93079318855371e-08
2579 1.91652874535486e-08
2580 1.92148869257447e-08
2581 1.92147753849126e-08
2582 1.91859551019191e-08
2583 1.92281768882085e-08
2584 1.92177145628691e-08
2585 1.91596289710882e-08
2586 1.92158324500147e-08
2587 1.91722841994535e-08
2588 1.91877734790347e-08
2589 1.92074011549637e-08
2590 1.91446472867507e-08
2591 1.92126579268148e-08
2592 1.91807696792168e-08
2593 1.91483986166086e-08
2594 1.91781526640566e-08
2595 1.91795694869934e-08
2596 1.91128117673633e-08
2597 1.91952280433894e-08
2598 1.91202970693527e-08
2599 1.91975848139458e-08
2600 1.91074989174811e-08
2601 1.92383495161152e-08
2602 1.90899906604702e-08
2603 1.91184723279392e-08
2604 1.91748689933324e-08
2605 1.91305764551397e-08
2606 1.90760595522588e-08
2607 1.91678331517564e-08
2608 1.91271677503213e-08
2609 1.91250535273024e-08
2610 1.91289267655326e-08
2611 1.90839606468707e-08
2612 1.91138081296338e-08
2613 1.90929134520346e-08
2614 1.91232216844028e-08
2615 1.90637801300575e-08
2616 1.91135060376713e-08
2617 1.91653505005052e-08
2618 1.89989742930274e-08
2619 1.91654400530372e-08
2620 1.90956990797631e-08
2621 1.90638525839892e-08
2622 1.90557822724036e-08
2623 1.90341849253262e-08
2624 1.90938931244333e-08
2625 1.90383233671931e-08
2626 1.91671977086694e-08
2627 1.90904209639853e-08
2628 1.90774153548956e-08
2629 1.90088649277831e-08
2630 1.91249558540441e-08
2631 1.89722222894706e-08
2632 1.91608072196425e-08
2633 1.90139267833289e-08
2634 1.90326673362162e-08
2635 1.90426698577495e-08
2636 1.9047094232727e-08
2637 1.9049726863507e-08
2638 1.90697154990316e-08
2639 1.8963030132324e-08
2640 1.9082001935844e-08
2641 1.89583294514239e-08
2642 1.91084553418097e-08
2643 1.89983564117258e-08
2644 1.89808793613078e-08
2645 1.90222599671941e-08
2646 1.90001284609553e-08
2647 1.90317539785489e-08
2648 1.9034315386135e-08
2649 1.89514017124859e-08
2650 1.90081359447425e-08
2651 1.8996277550698e-08
2652 1.90055577604653e-08
2653 1.89405224727546e-08
2654 1.90218502306716e-08
2655 1.89169686584112e-08
2656 1.90606263263948e-08
2657 1.894458555185e-08
2658 1.89430484854869e-08
2659 1.89628649095996e-08
2660 1.89785844529888e-08
2661 1.89392277398226e-08
2662 1.89510219703037e-08
2663 1.89306419035029e-08
2664 1.89933665988851e-08
2665 1.89384965413875e-08
2666 1.89101276303894e-08
2667 1.89427330004044e-08
2668 1.89453083245805e-08
2669 1.89335478573138e-08
2670 1.89136935682122e-08
2671 1.89240158026682e-08
2672 1.89262379271526e-08
2673 1.89135659964812e-08
2674 1.89571175783776e-08
2675 1.88665969821922e-08
2676 1.89499545643623e-08
2677 1.88850563542786e-08
2678 1.89284676281876e-08
2679 1.90040440213313e-08
2680 1.89531462064241e-08
2681 1.88796177330031e-08
2682 1.89929881818651e-08
2683 1.89159973507103e-08
2684 1.89597159984123e-08
2685 1.8980727539919e-08
2686 1.89463739229123e-08
2687 1.88994577652868e-08
2688 1.89528260847727e-08
2689 1.89581339332667e-08
2690 1.89300065290277e-08
2691 1.88897053306492e-08
2692 1.89349195149346e-08
2693 1.89570557850849e-08
2694 1.89138856312998e-08
2695 1.88800782728316e-08
2696 1.89219100512061e-08
2697 1.89433282710705e-08
2698 1.89026322556796e-08
2699 1.8869204945704e-08
2700 1.89070005820469e-08
2701 1.89330544665967e-08
2702 1.88866112816877e-08
2703 1.88560234506419e-08
2704 1.88989033433895e-08
2705 1.89152073274412e-08
2706 1.88756499937548e-08
2707 1.88433102688013e-08
2708 1.88907559584006e-08
2709 1.88593705753304e-08
2710 1.88830145980146e-08
2711 1.88336774895603e-08
2712 1.88873517704691e-08
2713 1.8817446664654e-08
2714 1.88939968804314e-08
2715 1.88165099667126e-08
2716 1.88591914336289e-08
2717 1.8823308231386e-08
2718 1.88699790293345e-08
2719 1.88086905277829e-08
2720 1.88457036598422e-08
2721 1.88175643830446e-08
2722 1.88677513933699e-08
2723 1.88076022457073e-08
2724 1.88214204563941e-08
2725 1.88013271436027e-08
2726 1.88616259031993e-08
2727 1.87940035147327e-08
2728 1.8808171536433e-08
2729 1.87864244583502e-08
2730 1.88502335758955e-08
2731 1.87832028696877e-08
2732 1.87927544628153e-08
2733 1.87745439619857e-08
2734 1.88380901737228e-08
2735 1.87652013797224e-08
2736 1.87876166058976e-08
2737 1.87535729154753e-08
2738 1.88274294283253e-08
2739 1.87510988447492e-08
2740 1.87752702660582e-08
2741 1.87387452777465e-08
2742 1.88137953712197e-08
2743 1.87342859453987e-08
2744 1.87620859553039e-08
2745 1.8724656923208e-08
2746 1.87988058716337e-08
2747 1.87186595858035e-08
2748 1.87545565187852e-08
2749 1.87043755451866e-08
2750 1.87894821878642e-08
2751 1.87020911464053e-08
2752 1.8744570118967e-08
2753 1.8688009794543e-08
2754 1.87769963124751e-08
2755 1.86899637747984e-08
2756 1.87311081987218e-08
2757 1.86770906903111e-08
2758 1.87630713195386e-08
2759 1.86771689857368e-08
2760 1.87199838324048e-08
2761 1.86605878331192e-08
2762 1.87533573378085e-08
2763 1.86607551654894e-08
2764 1.8712406655963e-08
2765 1.86425722153438e-08
2766 1.87455188036534e-08
2767 1.86458423312996e-08
2768 1.87002205935816e-08
2769 1.86300186590382e-08
2770 1.87322028130099e-08
2771 1.86332959207225e-08
2772 1.86866882206926e-08
2773 1.86185419605023e-08
2774 1.8718452327815e-08
2775 1.86203154450282e-08
2776 1.86773882761004e-08
2777 1.86014602405749e-08
2778 1.87099314484795e-08
2779 1.86054979792738e-08
2780 1.86651307476637e-08
2781 1.85886612942276e-08
2782 1.86982395156132e-08
2783 1.85886928908974e-08
2784 1.8656509216064e-08
2785 1.85726555871901e-08
2786 1.8687612951207e-08
2787 1.85785611523936e-08
2788 1.86406463472677e-08
2789 1.85597588474584e-08
2790 1.86741592921313e-08
2791 1.85639083963296e-08
2792 1.86277123437328e-08
2793 1.85528271937074e-08
2794 1.86487326718776e-08
2795 1.85567999687608e-08
2796 1.86136513113944e-08
2797 1.85467893930036e-08
2798 1.86197624715279e-08
2799 1.85486334861595e-08
2800 1.86199627326689e-08
2801 1.85064511109556e-08
2802 1.8636139992978e-08
2803 1.85216018906975e-08
2804 1.85935914480684e-08
2805 1.85212945119084e-08
2806 1.85947655770846e-08
2807 1.85161983547455e-08
2808 1.86065658722701e-08
2809 1.84728547755197e-08
2810 1.86262313464081e-08
2811 1.84879315971997e-08
2812 1.85737217192061e-08
2813 1.84782886476498e-08
2814 1.86010576902484e-08
2815 1.84856333092509e-08
2816 1.85518974337162e-08
2817 1.84719113491272e-08
2818 1.85808926126407e-08
2819 1.84790524737077e-08
2820 1.85297925426875e-08
2821 1.84712215924865e-08
2822 1.85495293555382e-08
2823 1.8483668446645e-08
2824 1.84956115681922e-08
2825 1.84860195352465e-08
2826 1.84971129728595e-08
2827 1.84889699822444e-08
2828 1.85233674925289e-08
2829 1.84208313319667e-08
2830 1.85355038944124e-08
2831 1.84369966698683e-08
2832 1.85036315412113e-08
2833 1.84327025878339e-08
2834 1.85035424474789e-08
2835 1.84279332310355e-08
2836 1.84905746551944e-08
2837 1.84331243922586e-08
2838 1.85030435237987e-08
2839 1.8385710199853e-08
2840 1.85123251502661e-08
2841 1.84101541900183e-08
2842 1.84554945446003e-08
2843 1.84145044797024e-08
2844 1.84621657519646e-08
2845 1.83887964636531e-08
2846 1.84892414083482e-08
2847 1.84050985749806e-08
2848 1.84279279042965e-08
2849 1.83923706734701e-08
2850 1.84432959736136e-08
2851 1.84072519533451e-08
2852 1.83943753129245e-08
2853 1.84026452322295e-08
2854 1.84282473780772e-08
2855 1.84075532688732e-08
2856 1.83744898827332e-08
2857 1.84002169283382e-08
2858 1.83742667557718e-08
2859 1.83925416956665e-08
2860 1.83858151043825e-08
2861 1.83917873421979e-08
2862 1.83527963741881e-08
2863 1.83857967683831e-08
2864 1.83617946464265e-08
2865 1.83883455562306e-08
2866 1.83358746406181e-08
2867 1.83722780319195e-08
2868 1.83712641981804e-08
2869 1.83804795582665e-08
2870 1.83180274562211e-08
2871 1.83703075291586e-08
2872 1.83243079900919e-08
2873 1.83541828229217e-08
2874 1.83202035685115e-08
2875 1.83808608689162e-08
2876 1.83307433119229e-08
2877 1.83186224044229e-08
2878 1.83230846926286e-08
2879 1.83162015486849e-08
2880 1.83417611478998e-08
2881 1.83259452434248e-08
2882 1.82883686946633e-08
2883 1.83187866378276e-08
2884 1.8323747663529e-08
2885 1.83220224216907e-08
2886 1.82748242650854e-08
2887 1.83145506813531e-08
2888 1.82769117107462e-08
2889 1.83006481407899e-08
2890 1.82780229874835e-08
2891 1.82991604816829e-08
2892 1.82789040346076e-08
2893 1.82739052660974e-08
2894 1.82710748047876e-08
2895 1.82926385984827e-08
2896 1.82740726752395e-08
2897 1.82499518359691e-08
2898 1.82762173180362e-08
2899 1.82373317147988e-08
2900 1.82860119849471e-08
2901 1.82493755071511e-08
2902 1.82422020827966e-08
2903 1.82547731522598e-08
2904 1.82262198384908e-08
2905 1.82700169951144e-08
2906 1.82380860773712e-08
2907 1.82207730605732e-08
2908 1.82443986743719e-08
2909 1.82072735955829e-08
2910 1.82560290171718e-08
2911 1.82200882061223e-08
2912 1.82091396475625e-08
2913 1.8224757478924e-08
2914 1.81936037647601e-08
2915 1.82417472939234e-08
2916 1.82047566792964e-08
2917 1.81918968887862e-08
2918 1.82104819345663e-08
2919 1.8177849223977e-08
2920 1.82230637434944e-08
2921 1.81853486662109e-08
2922 1.81824873706771e-08
2923 1.81892034022413e-08
2924 1.81704993915677e-08
2925 1.81928486795435e-08
2926 1.8172714598863e-08
2927 1.81689724110812e-08
2928 1.81748297498618e-08
2929 1.81511079513363e-08
2930 1.81905961135764e-08
2931 1.81590570643708e-08
2932 1.81468327097312e-08
2933 1.8163119465564e-08
2934 1.81325120010012e-08
2935 1.81731836128018e-08
2936 1.8135637333605e-08
2937 1.81421094819778e-08
2938 1.81360908851347e-08
2939 1.81315874758226e-08
2940 1.81321171993654e-08
2941 1.81279405604418e-08
2942 1.8121918174574e-08
2943 1.81298237251215e-08
2944 1.8124287808452e-08
2945 1.81054904135891e-08
2946 1.81274316849445e-08
2947 1.80894148556487e-08
2948 1.81394171521188e-08
2949 1.80967924008901e-08
2950 1.80971432946486e-08
2951 1.80984948662255e-08
2952 1.8088280254458e-08
2953 1.8098232737962e-08
2954 1.80941835704607e-08
2955 1.80728270693375e-08
2956 1.80993437251531e-08
2957 1.80767848456664e-08
2958 1.80723052560228e-08
2959 1.80741118555594e-08
2960 1.80654662998436e-08
2961 1.80699249514582e-08
2962 1.80631582739621e-08
2963 1.80550175318261e-08
2964 1.80801691765375e-08
2965 1.80578640280871e-08
2966 1.80384268934075e-08
2967 1.80582754100711e-08
2968 1.80286749344449e-08
2969 1.8065834899772e-08
2970 1.8007381894003e-08
2971 1.8081630334843e-08
2972 1.80042599250418e-08
2973 1.80537470735875e-08
2974 1.7998187449686e-08
2975 1.80499357416153e-08
2976 1.8006322681674e-08
2977 1.80413878650421e-08
2978 1.79839614229405e-08
2979 1.8034674723777e-08
2980 1.7997966052008e-08
2981 1.80293387446739e-08
2982 1.79697331026962e-08
2983 1.80228711841313e-08
2984 1.79875324000656e-08
2985 1.8016358265871e-08
2986 1.79578788037094e-08
2987 1.80095079274634e-08
2988 1.79767100348394e-08
2989 1.80033089976495e-08
2990 1.79461032291384e-08
2991 1.79982290434699e-08
2992 1.79595963366586e-08
2993 1.79905074923492e-08
2994 1.79358786740469e-08
2995 1.79850079300992e-08
2996 1.79546725664337e-08
2997 1.79803649847377e-08
2998 1.79202572762049e-08
2999 1.79795691696572e-08
3000 9.58347351356741e-09
3001 9.6020748853419e-09
3002 9.66438031779412e-09
3003 9.70280146601193e-09
3004 9.7306929956395e-09
3005 9.74102751981343e-09
3006 9.74348267947234e-09
3007 9.74337125595609e-09
3008 9.74274358631066e-09
3009 9.74198575692115e-09
3010 9.74120520997462e-09
3011 9.74049312927466e-09
3012 9.73980632787419e-09
3013 9.73909706002141e-09
3014 9.73843927815077e-09
3015 9.73777593721303e-09
3016 9.73715753105342e-09
3017 9.73651144157983e-09
3018 9.73586478491412e-09
3019 9.73526335768665e-09
3020 9.73467126539479e-09
3021 9.73407754423922e-09
3022 9.73350953300395e-09
3023 9.73290111490332e-09
3024 9.73236085850815e-09
3025 9.73178982051348e-09
3026 9.7312261022886e-09
3027 9.73067445655257e-09
3028 9.73011961417913e-09
3029 9.72957927321966e-09
3030 9.72904108906963e-09
3031 9.72849802782089e-09
3032 9.72795819412231e-09
3033 9.72743545509663e-09
3034 9.72691790526031e-09
3035 9.72636647602471e-09
3036 9.7258457909255e-09
3037 9.72533985577062e-09
3038 9.72480645135276e-09
3039 9.7243055199453e-09
3040 9.72377103120037e-09
3041 9.7232313495399e-09
3042 9.72273637614429e-09
3043 9.72218287945831e-09
3044 9.72165382474161e-09
3045 9.72112338424452e-09
3046 9.72058687611493e-09
3047 9.72010689456737e-09
3048 9.71960830837421e-09
3049 9.71907414579892e-09
3050 9.71860633107446e-09
3051 9.7181071731442e-09
3052 9.71758707239007e-09
3053 9.71711464538977e-09
3054 9.71658810559189e-09
3055 9.71611784762044e-09
3056 9.71560816023786e-09
3057 9.71514348743763e-09
3058 9.71462933346823e-09
3059 9.71416109889905e-09
3060 9.71363364590805e-09
3061 9.71316477687417e-09
3062 9.71263951316664e-09
3063 9.71220455532606e-09
3064 9.71167346115742e-09
3065 9.71119317835084e-09
3066 9.71073606885475e-09
3067 9.71024906995521e-09
3068 9.70976809926832e-09
3069 9.70925319294241e-09
3070 9.70876876463245e-09
3071 9.70830064824652e-09
3072 9.70778758888763e-09
3073 9.70732221123455e-09
3074 9.70687500824463e-09
3075 9.70634564367856e-09
3076 9.70587548120017e-09
3077 9.70538964679274e-09
3078 9.70491438685023e-09
3079 9.70441687186962e-09
3080 9.70396871897983e-09
3081 9.70346620501189e-09
3082 9.70299225230148e-09
3083 9.70251994972249e-09
3084 9.70205041337519e-09
3085 9.70156771512576e-09
3086 9.70109616347387e-09
3087 9.70058295399895e-09
3088 9.70011825317946e-09
3089 9.69965089951674e-09
3090 9.69915656745535e-09
3091 9.69867696729715e-09
3092 9.69820447788844e-09
3093 9.69776304540759e-09
3094 9.69725108727198e-09
3095 9.69679226975328e-09
3096 9.69631600652304e-09
3097 9.69584070174834e-09
3098 9.69534293301932e-09
3099 9.69489520939726e-09
3100 9.69442390600511e-09
3101 9.69394617567759e-09
3102 9.69347368086348e-09
3103 9.69298964446225e-09
3104 9.69252911706825e-09
3105 9.69204776319482e-09
3106 9.69157825948808e-09
3107 9.69112934767757e-09
3108 9.69064696813154e-09
3109 9.69020252991615e-09
3110 9.68970048866036e-09
3111 9.6892671722984e-09
3112 9.68877790526179e-09
3113 9.68830777130919e-09
3114 9.68784845304521e-09
3115 9.68735403607951e-09
3116 9.68687681571906e-09
3117 9.68642485454224e-09
3118 9.68596924062759e-09
3119 9.68551345588431e-09
3120 9.68502002752364e-09
3121 9.68454799844115e-09
3122 9.68407838875668e-09
3123 9.68364955605122e-09
3124 9.68316547406145e-09
3125 9.68268506253839e-09
3126 9.68224235747828e-09
3127 9.68175981126002e-09
3128 9.68129883832658e-09
3129 9.68082247950613e-09
3130 9.68033483499803e-09
3131 9.67989893031068e-09
3132 9.67945991863645e-09
3133 9.67894661978164e-09
3134 9.67849259281978e-09
3135 9.67804290334662e-09
3136 9.67757997726026e-09
3137 9.67711467565746e-09
3138 9.67666768143805e-09
3139 9.6761734678999e-09
3140 9.67571858812716e-09
3141 9.67526251595408e-09
3142 9.6747915963244e-09
3143 9.67431543642122e-09
3144 9.67384139513583e-09
3145 9.67339331584688e-09
3146 9.67293175321154e-09
3147 9.67248795242376e-09
3148 9.67201225857139e-09
3149 9.67152862140636e-09
3150 9.67109561425539e-09
3151 9.67065146238733e-09
3152 9.67015670460397e-09
3153 9.66970462031036e-09
3154 9.66920629488083e-09
3155 9.66876480702761e-09
3156 9.66831535662016e-09
3157 9.66781846709064e-09
3158 9.66737204045887e-09
3159 9.66688841037844e-09
3160 9.66645642443836e-09
3161 9.66597866246255e-09
3162 9.66549958623247e-09
3163 9.66505693825864e-09
3164 9.66458282224136e-09
3165 9.66412962737084e-09
3166 9.6636576463055e-09
3167 9.66319010472366e-09
3168 9.66272697040804e-09
3169 9.66229287673037e-09
3170 9.66181909400898e-09
3171 9.66133516341894e-09
3172 9.66090237303208e-09
3173 9.66046160264661e-09
3174 9.65999562807518e-09
3175 9.65952995898162e-09
3176 9.65904645993526e-09
3177 9.65860601727375e-09
3178 9.65813844012314e-09
3179 9.65767237044029e-09
3180 9.65722432308413e-09
3181 9.65675483093759e-09
3182 9.65630613564139e-09
3183 9.65584364887723e-09
3184 9.65543275385089e-09
3185 9.65493366667652e-09
3186 9.65446969208861e-09
3187 9.65403687670091e-09
3188 9.6535800478692e-09
3189 9.65310619455845e-09
3190 9.65267684931853e-09
3191 9.65219332647177e-09
3192 9.65172146624727e-09
3193 9.65129638423601e-09
3194 9.65081951927182e-09
3195 9.65038285295067e-09
3196 9.64990845587349e-09
3197 9.64946579457698e-09
3198 9.64900872201663e-09
3199 9.64854502560203e-09
3200 9.64810760217122e-09
3201 9.64764752795638e-09
3202 9.64720494672777e-09
3203 9.64671985116683e-09
3204 9.6462876202838e-09
3205 9.6458178845335e-09
3206 9.64534667664135e-09
3207 9.64493109626569e-09
3208 9.64443782010271e-09
3209 9.64399904728602e-09
3210 9.64355192555749e-09
3211 9.64308443725942e-09
3212 9.64259767009118e-09
3213 9.64218030524017e-09
3214 9.6417415177269e-09
3215 9.64124946591788e-09
3216 9.64079075839452e-09
3217 9.64037407760049e-09
3218 9.63988665479698e-09
3219 9.63943509685317e-09
3220 9.6389970415528e-09
3221 9.6385365893556e-09
3222 9.63808984075221e-09
3223 9.63765681644135e-09
3224 9.63719553868231e-09
3225 9.63674567124734e-09
3226 9.63629872153399e-09
3227 9.63580699374356e-09
3228 9.63540291176734e-09
3229 9.63490914714787e-09
3230 9.63448961725788e-09
3231 9.63402107524714e-09
3232 9.63356593302461e-09
3233 9.63312118270471e-09
3234 9.63266478248848e-09
3235 9.63220207621934e-09
3236 9.63175737148103e-09
3237 9.63131002083145e-09
3238 9.630883006366e-09
3239 9.63039703105739e-09
3240 9.62996174393854e-09
3241 9.62948444319101e-09
3242 9.62904200242643e-09
3243 9.62860628258427e-09
3244 9.62815064153161e-09
3245 9.6277056925026e-09
3246 9.62723039041324e-09
3247 9.62678244353227e-09
3248 9.62633778142652e-09
3249 9.6258904422053e-09
3250 9.62543249070913e-09
3251 9.62500874588518e-09
3252 9.62454773743154e-09
3253 9.62410306046268e-09
3254 9.62363399221316e-09
3255 9.62321706098057e-09
3256 9.62275517835814e-09
3257 9.62230964025867e-09
3258 9.62185937390669e-09
3259 9.62137194839008e-09
3260 9.6209976594569e-09
3261 9.62050120473929e-09
3262 9.62005526346926e-09
3263 9.61961570026398e-09
3264 9.61917492256492e-09
3265 9.61872110797485e-09
3266 9.61829326606711e-09
3267 9.61783192589966e-09
3268 9.617399998052e-09
3269 9.61694951980008e-09
3270 9.61650127415808e-09
3271 9.61606068305282e-09
3272 9.61561716191634e-09
3273 9.61516758536674e-09
3274 9.61472460125207e-09
3275 9.614265931851e-09
3276 9.61383538637672e-09
3277 9.61338069112999e-09
3278 9.61294824274039e-09
3279 9.61252416814551e-09
3280 9.6120457682633e-09
3281 9.61161578218878e-09
3282 9.6111765810411e-09
3283 9.61070416412996e-09
3284 9.61028626663557e-09
3285 9.60983142103688e-09
3286 9.60939324701204e-09
3287 9.60895576697374e-09
3288 9.6084944929617e-09
3289 9.60806718298957e-09
3290 9.60759723133558e-09
3291 9.60718022951362e-09
3292 9.60672589987721e-09
3293 9.60625976307444e-09
3294 9.60583330388703e-09
3295 9.60537382762444e-09
3296 9.6049380179028e-09
3297 9.60452192155098e-09
3298 9.60406965504895e-09
3299 9.60360146753242e-09
3300 9.60318476262573e-09
3301 9.60271535457857e-09
3302 9.60228926877998e-09
3303 9.60183996051578e-09
3304 9.60137600442001e-09
3305 9.60094285949037e-09
3306 9.60050286598346e-09
3307 9.60002404076094e-09
3308 9.59962650694951e-09
3309 9.59917682910594e-09
3310 9.59870415882108e-09
3311 9.59831113835075e-09
3312 9.59785467612162e-09
3313 9.59737275964961e-09
3314 9.59696118239817e-09
3315 9.59648264677038e-09
3316 9.59605767647531e-09
3317 9.59561057978925e-09
3318 9.59516659863879e-09
3319 9.59474014021466e-09
3320 9.59428534135637e-09
3321 9.59382704600331e-09
3322 9.59338584043123e-09
3323 9.59294798044685e-09
3324 9.59252024120005e-09
3325 9.59205253473622e-09
3326 9.59159860946385e-09
3327 9.59118515225788e-09
3328 9.59073271699501e-09
3329 9.59031289016893e-09
3330 9.58983769394628e-09
3331 9.5894144671177e-09
3332 9.58897632636485e-09
3333 9.5885242792082e-09
3334 9.58809393775129e-09
3335 9.58765413015816e-09
3336 9.58720201590241e-09
3337 9.58673871984117e-09
3338 9.58632657575148e-09
3339 9.5858625700565e-09
3340 9.5854156380859e-09
3341 9.58498948629843e-09
3342 9.58454978679246e-09
3343 9.58409128051091e-09
3344 9.58368478762295e-09
3345 9.58324366968216e-09
3346 9.58277860561851e-09
3347 9.58232626392663e-09
3348 9.5818800441641e-09
3349 9.58146440522417e-09
3350 9.5810211698244e-09
3351 9.58058398164291e-09
3352 9.5801596992906e-09
3353 9.57968820477356e-09
3354 9.57927015235449e-09
3355 9.57881133926974e-09
3356 9.57836300449766e-09
3357 9.5779472664842e-09
3358 9.5775024848907e-09
3359 9.57705886130838e-09
3360 9.57664659057694e-09
3361 9.57617066593003e-09
3362 9.57574974714409e-09
3363 9.57531779739035e-09
3364 9.57485532348396e-09
3365 9.57443975908856e-09
3366 9.57400095908528e-09
3367 9.5735553155285e-09
3368 9.57310867085587e-09
3369 9.57266360846615e-09
3370 9.57224289671599e-09
3371 9.57180041104982e-09
3372 9.5713673018416e-09
3373 9.57093531359782e-09
3374 9.57049505510149e-09
3375 9.57006210170613e-09
3376 9.56962325176364e-09
3377 9.56918887246028e-09
3378 9.5687450099996e-09
3379 9.56831596338886e-09
3380 9.56786405544724e-09
3381 9.56746453457252e-09
3382 9.5670194655284e-09
3383 9.56656975905495e-09
3384 9.56613341383805e-09
3385 9.56571157947894e-09
3386 9.56525782005307e-09
3387 9.56482512681767e-09
3388 9.56436246175862e-09
3389 9.56395297140033e-09
3390 9.56351630298363e-09
3391 9.56309128165994e-09
3392 9.56261422046384e-09
3393 9.56219555509064e-09
3394 9.56175573226664e-09
3395 9.56133409194676e-09
3396 9.56089547557437e-09
3397 9.5604617982234e-09
3398 9.55999215052072e-09
3399 9.55956266294328e-09
3400 9.55913770334105e-09
3401 9.55867719457798e-09
3402 9.55827380259844e-09
3403 9.55782702760644e-09
3404 9.55738647985538e-09
3405 9.5569408521054e-09
3406 9.55650199886693e-09
3407 9.55608915190892e-09
3408 9.55563015065525e-09
3409 9.55522244042162e-09
3410 9.55475927259808e-09
3411 9.554308221707e-09
3412 9.55388585401756e-09
3413 9.55346658313566e-09
3414 9.55304347936142e-09
3415 9.55259981703232e-09
3416 9.55217387001856e-09
3417 9.55174024527133e-09
3418 9.5512946951537e-09
3419 9.55088737966681e-09
3420 9.55044793003734e-09
3421 9.54998282734487e-09
3422 9.54956096017173e-09
3423 9.54911397017116e-09
3424 9.54868965647587e-09
3425 9.5482582953485e-09
3426 9.54783022113354e-09
3427 9.54740865037607e-09
3428 9.54697686884887e-09
3429 9.54651899907899e-09
3430 9.54611516593151e-09
3431 9.54565583088235e-09
3432 9.54521838402583e-09
3433 9.54480830751142e-09
3434 9.54434588713859e-09
3435 9.54392692956163e-09
3436 9.54348816395445e-09
3437 9.54304356200891e-09
3438 9.54260418999098e-09
3439 9.5422051679811e-09
3440 9.54176117472921e-09
3441 9.54132794279972e-09
3442 9.54088415960896e-09
3443 9.54044540530524e-09
3444 9.54003610650206e-09
3445 9.53959708335783e-09
3446 9.5391732066602e-09
3447 9.53872095191266e-09
3448 9.53828263124124e-09
3449 9.53787166294712e-09
3450 9.53742566431326e-09
3451 9.53698975952877e-09
3452 9.53656562741739e-09
3453 9.53613160542544e-09
3454 9.53569461595305e-09
3455 9.53527708306284e-09
3456 9.5348555345931e-09
3457 9.53441542377725e-09
3458 9.5339886120524e-09
3459 9.53354751909857e-09
3460 9.53312275775442e-09
3461 9.53269644286825e-09
3462 9.5322403367154e-09
3463 9.53180781231716e-09
3464 9.53139837717859e-09
3465 9.53095045511804e-09
3466 9.5305337817278e-09
3467 9.53009070127353e-09
3468 9.52970741999182e-09
3469 9.5292435879149e-09
3470 9.52882393018273e-09
3471 9.52838156659253e-09
3472 9.52795298329173e-09
3473 9.52752505336413e-09
3474 9.52709494942555e-09
3475 9.52665502018668e-09
3476 9.52621987345553e-09
3477 9.52580987221424e-09
3478 9.52539640965838e-09
3479 9.52492342429917e-09
3480 9.52450609349731e-09
3481 9.52410090593325e-09
3482 9.52363794708144e-09
3483 9.52322114263632e-09
3484 9.52280194470995e-09
3485 9.522347913897e-09
3486 9.52193062674078e-09
3487 9.52151112328104e-09
3488 9.52107220306142e-09
3489 9.52065478453112e-09
3490 9.52020881991172e-09
3491 9.51979962059146e-09
3492 9.5193872183888e-09
3493 9.51892075901767e-09
3494 9.51851241322299e-09
3495 9.5180703584552e-09
3496 9.51765555288336e-09
3497 9.51720964674496e-09
3498 9.51681306107094e-09
3499 9.51637330696281e-09
3500 9.51595874633393e-09
3501 9.51552750068363e-09
3502 9.51510181941562e-09
3503 9.51467980649101e-09
3504 9.51423050387507e-09
3505 9.5138313189469e-09
3506 9.51338747822578e-09
3507 9.5129678098077e-09
3508 9.51253236827765e-09
3509 9.51212842166538e-09
3510 9.51166136367587e-09
3511 9.51126055991763e-09
3512 9.51080506837731e-09
3513 9.5103671980401e-09
3514 9.50997611277871e-09
3515 9.50954225337891e-09
3516 9.50916005514751e-09
3517 9.50868099463403e-09
3518 9.50825370770597e-09
3519 9.50784615420808e-09
3520 9.50740686459645e-09
3521 9.50698874326722e-09
3522 9.50656866538502e-09
3523 9.5060951055756e-09
3524 9.50570942669787e-09
3525 9.50526804460727e-09
3526 9.50485520462979e-09
3527 9.50440846270856e-09
3528 9.50400129261231e-09
3529 9.50357934467044e-09
3530 9.5031614395294e-09
3531 9.50270395111419e-09
3532 9.50227876101578e-09
3533 9.50188792886442e-09
3534 9.50143657067748e-09
3535 9.50098881976474e-09
3536 9.50059626424193e-09
3537 9.5001790927432e-09
3538 9.49971769859115e-09
3539 9.49930346873934e-09
3540 9.49889395716885e-09
3541 9.49845697007651e-09
3542 9.49803813170974e-09
3543 9.49762376133839e-09
3544 9.49719041996172e-09
3545 9.49676344989131e-09
3546 9.49634279580336e-09
3547 9.49592098575119e-09
3548 9.49552280540139e-09
3549 9.49504252425321e-09
3550 9.49463547449819e-09
3551 9.49423771825359e-09
3552 9.49378698344994e-09
3553 9.49337365224734e-09
3554 9.49293343723012e-09
3555 9.49253154491902e-09
3556 9.492085116386e-09
3557 9.49167156982789e-09
3558 9.49127875914807e-09
3559 9.49084712022902e-09
3560 9.49039697162313e-09
3561 9.49000940868305e-09
3562 9.48954986568912e-09
3563 9.48911814761116e-09
3564 9.48871496448539e-09
3565 9.48831204940215e-09
3566 9.48786030201265e-09
3567 9.48744988209932e-09
3568 9.48703833747455e-09
3569 9.4866180297068e-09
3570 9.4862004690402e-09
3571 9.48578473439904e-09
3572 9.48532496566207e-09
3573 9.48492887648672e-09
3574 9.48451358494251e-09
3575 9.48409400764599e-09
3576 9.48366294718783e-09
3577 9.48323413615948e-09
3578 9.48282085939944e-09
3579 9.48239195407152e-09
3580 9.48194636052335e-09
3581 9.48155969665576e-09
3582 9.48111932910084e-09
3583 9.48069399418078e-09
3584 9.48028269474183e-09
3585 9.47986263127865e-09
3586 9.4794270245474e-09
3587 9.4790268021297e-09
3588 9.47858338448682e-09
3589 9.47817693913028e-09
3590 9.47776891879748e-09
3591 9.47731811488939e-09
3592 9.47690855241517e-09
3593 9.47649388061828e-09
3594 9.47605796625117e-09
3595 9.47566181692255e-09
3596 9.47523362775787e-09
3597 9.47482084388829e-09
3598 9.47438456863625e-09
3599 9.4739985972217e-09
3600 9.47356304864533e-09
3601 9.47315606417837e-09
3602 9.47271680116352e-09
3603 9.47230534276144e-09
3604 9.47187178251124e-09
3605 9.47146295059154e-09
3606 9.47104390364856e-09
3607 9.47063981437957e-09
3608 9.47018830840113e-09
3609 9.46977936316928e-09
3610 9.46937050915614e-09
3611 9.4689293219305e-09
3612 9.46851401124188e-09
3613 9.46807918036224e-09
3614 9.46767375731295e-09
3615 9.4672592097847e-09
3616 9.46683365810747e-09
3617 9.46639097385016e-09
3618 9.46600831767258e-09
3619 9.46556166934315e-09
3620 9.46517138392111e-09
3621 9.46475114430023e-09
3622 9.46427920640175e-09
3623 9.46391134341168e-09
3624 9.46348817125464e-09
3625 9.46307360149418e-09
3626 9.46262326623898e-09
3627 9.46224018014835e-09
3628 9.46178593329294e-09
3629 9.46141817655122e-09
3630 9.4609892950584e-09
3631 9.46056481174878e-09
3632 9.46011862506396e-09
3633 9.45971078834124e-09
3634 9.45929129957807e-09
3635 9.45886700438187e-09
3636 9.45848159276258e-09
3637 9.45804931984373e-09
3638 9.45759260567725e-09
3639 9.45723197153664e-09
3640 9.45678799726957e-09
3641 9.45635850265608e-09
3642 9.45592574987802e-09
3643 9.45553570486785e-09
3644 9.45511220810241e-09
3645 9.45469608058108e-09
3646 9.45427801536669e-09
3647 9.45385653737529e-09
3648 9.45343033446899e-09
3649 9.45304128029123e-09
3650 9.45261468914688e-09
3651 9.45219741838726e-09
3652 9.45178490164428e-09
3653 9.45135753278176e-09
3654 9.45094961018328e-09
3655 9.45053348087865e-09
3656 9.45013087681745e-09
3657 9.44968424897857e-09
3658 9.44930692037271e-09
3659 9.44887897925267e-09
3660 9.44845896171109e-09
3661 9.44801677492391e-09
3662 9.44761144313494e-09
3663 9.44720026576501e-09
3664 9.44678678686806e-09
3665 9.44640066553176e-09
3666 9.44599131068313e-09
3667 9.44554389044339e-09
3668 9.44514184188228e-09
3669 9.44472521446915e-09
3670 9.44432473036494e-09
3671 9.44390744132134e-09
3672 9.44347700839593e-09
3673 9.44306322913507e-09
3674 9.44265857021065e-09
3675 9.44222338487149e-09
3676 9.4418050723688e-09
3677 9.44141949215521e-09
3678 9.44101254519297e-09
3679 9.44056774962454e-09
3680 9.44015039056056e-09
3681 9.43976184813317e-09
3682 9.4393655388908e-09
3683 9.43891382335776e-09
3684 9.43850627366932e-09
3685 9.43809720260064e-09
3686 9.43768812461387e-09
3687 9.4372543196844e-09
3688 9.43684995238087e-09
3689 9.43642565894021e-09
3690 9.43598963805414e-09
3691 9.43557565080688e-09
3692 9.4352117587862e-09
3693 9.43477321876246e-09
3694 9.43431209383949e-09
3695 9.43394568975187e-09
3696 9.43352407857534e-09
3697 9.43311588179341e-09
3698 9.43266455637193e-09
3699 9.43228876357921e-09
3700 9.43187426224318e-09
3701 9.43142969615091e-09
3702 9.43103094145498e-09
3703 9.43064068336524e-09
3704 9.43021984613907e-09
3705 9.42981189698544e-09
3706 9.42939927849745e-09
3707 9.42897405170617e-09
3708 9.42855697718542e-09
3709 9.42815940213615e-09
3710 9.42775633849119e-09
3711 9.42734312963517e-09
3712 9.42691582753868e-09
3713 9.42648979449651e-09
3714 9.42607466926371e-09
3715 9.42569906123292e-09
3716 9.42528015590582e-09
3717 9.42488072817188e-09
3718 9.42443219636296e-09
3719 9.42405856323159e-09
3720 9.42363113743544e-09
3721 9.42320347010334e-09
3722 9.42277715199058e-09
3723 9.42242580136837e-09
3724 9.42196957969682e-09
3725 9.42157252557113e-09
3726 9.42117999449404e-09
3727 9.42075604018489e-09
3728 9.42036913445521e-09
3729 9.41993911362377e-09
3730 9.41952183894368e-09
3731 9.41910914367683e-09
3732 9.41870014138646e-09
3733 9.41829184696735e-09
3734 9.41789504500107e-09
3735 9.41745568767971e-09
3736 9.41705652122982e-09
3737 9.41665286349064e-09
3738 9.41623125451374e-09
3739 9.41582448153733e-09
3740 9.41541379535782e-09
3741 9.41500121638683e-09
3742 9.4145916795102e-09
3743 9.41417914639142e-09
3744 9.41376769746788e-09
3745 9.4133532871285e-09
3746 9.41291898828162e-09
3747 9.412525683615e-09
3748 9.41211660684255e-09
3749 9.41172008352142e-09
3750 9.41128920790846e-09
3751 9.41085412869275e-09
3752 9.41047460575883e-09
3753 9.41007364874125e-09
3754 9.40966543179489e-09
3755 9.40923747050348e-09
3756 9.40881672677196e-09
3757 9.40845212891311e-09
3758 9.40799504624973e-09
3759 9.40761564372644e-09
3760 9.40720822326796e-09
3761 9.40680112129777e-09
3762 9.4063726459731e-09
3763 9.40598116360575e-09
3764 9.40554963416162e-09
3765 9.40514824872979e-09
3766 9.40476843783178e-09
3767 9.40434372502519e-09
3768 9.40391872821661e-09
3769 9.4035331680356e-09
3770 9.40310596986294e-09
3771 9.40271316660773e-09
3772 9.40225976622105e-09
3773 9.40189343307668e-09
3774 9.40146639220874e-09
3775 9.40106758613723e-09
3776 9.40067987394155e-09
3777 9.40026361417878e-09
3778 9.39982364443065e-09
3779 9.39942916231007e-09
3780 9.3990155481255e-09
3781 9.39863176681321e-09
3782 9.39819914584639e-09
3783 9.39781005421247e-09
3784 9.39737928854628e-09
3785 9.396960604445e-09
3786 9.39657862215892e-09
3787 9.39616619472644e-09
3788 9.39577725172364e-09
3789 9.39534811882081e-09
3790 9.39492950914611e-09
3791 9.39450330867536e-09
3792 9.39412076077228e-09
3793 9.39373223338147e-09
3794 9.39330218092255e-09
3795 9.39291518603502e-09
3796 9.39250269705455e-09
3797 9.39205458595077e-09
3798 9.39168627546449e-09
3799 9.39128890201091e-09
3800 9.39086689395741e-09
3801 9.39045063583221e-09
3802 9.39007852626683e-09
3803 9.38966704282229e-09
3804 9.38923563536392e-09
3805 9.38882931209722e-09
3806 9.388414682579e-09
3807 9.38802531544325e-09
3808 9.38763133610798e-09
3809 9.38721691129407e-09
3810 9.38680162157479e-09
3811 9.3863981983841e-09
3812 9.38596912713335e-09
3813 9.38556306086941e-09
3814 9.38518625077933e-09
3815 9.38477021070888e-09
3816 9.38435073540717e-09
3817 9.38396590915991e-09
3818 9.38357137229839e-09
3819 9.38314921914568e-09
3820 9.3827362643642e-09
3821 9.38233142605549e-09
3822 9.38192124904508e-09
3823 9.38153471653769e-09
3824 9.38110885992383e-09
3825 9.38072530184964e-09
3826 9.3803082028554e-09
3827 9.37987415564751e-09
3828 9.37947038825299e-09
3829 9.37908409398558e-09
3830 9.37868166112077e-09
3831 9.37825316946195e-09
3832 9.37784911524131e-09
3833 9.37745233985099e-09
3834 9.37705033569186e-09
3835 9.37663153557922e-09
3836 9.37625472688386e-09
3837 9.37582847611307e-09
3838 9.3754244899491e-09
3839 9.37500801797414e-09
3840 9.37461273565338e-09
3841 9.3742140334363e-09
3842 9.37381291576944e-09
3843 9.37341361185312e-09
3844 9.37299661723373e-09
3845 9.37261191973765e-09
3846 9.37219527339522e-09
3847 9.37178702873492e-09
3848 9.37139156452493e-09
3849 9.37097006826343e-09
3850 9.37058365926835e-09
3851 9.37016190313139e-09
3852 9.36977614381801e-09
3853 9.36935220957613e-09
3854 9.36893823892671e-09
3855 9.36856527566921e-09
3856 9.36813447681428e-09
3857 9.36772372561734e-09
3858 9.36733677246032e-09
3859 9.36690636174631e-09
3860 9.36651905264485e-09
3861 9.36612963671479e-09
3862 9.36571467370639e-09
3863 9.36529678535053e-09
3864 9.36489871274787e-09
3865 9.36451099318308e-09
3866 9.36408102499703e-09
3867 9.36367287533713e-09
3868 9.3632964627624e-09
3869 9.36287507324191e-09
3870 9.36247651211336e-09
3871 9.36210364033824e-09
3872 9.36167615785827e-09
3873 9.36126031689938e-09
3874 9.36088116875594e-09
3875 9.36050075844158e-09
3876 9.36005465473899e-09
3877 9.35965945850908e-09
3878 9.35924532940308e-09
3879 9.35883489789485e-09
3880 9.35845715351047e-09
3881 9.35803758887743e-09
3882 9.35764557378343e-09
3883 9.35722948385009e-09
3884 9.35683607840299e-09
3885 9.35641819806154e-09
3886 9.3560181336147e-09
3887 9.35566740010996e-09
3888 9.35524071311172e-09
3889 9.35481971865021e-09
3890 9.35441052424962e-09
3891 9.35401926991514e-09
3892 9.35361554453562e-09
3893 9.35323439088531e-09
3894 9.35282601907672e-09
3895 9.35240745411625e-09
3896 9.35200022098709e-09
3897 9.35160469259927e-09
3898 9.35121636430614e-09
3899 9.35078353731661e-09
3900 9.35038989061804e-09
3901 9.34999633484673e-09
3902 9.34957959505822e-09
3903 9.34919893848946e-09
3904 9.34877261113409e-09
3905 9.34836499471431e-09
3906 9.34797857574804e-09
3907 9.34759798895679e-09
3908 9.34718221406311e-09
3909 9.34678130354999e-09
3910 9.34635661996308e-09
3911 9.34597165495876e-09
3912 9.34561335496492e-09
3913 9.34515592158902e-09
3914 9.34476764608005e-09
3915 9.34438021456263e-09
3916 9.34396329990012e-09
3917 9.34357178980494e-09
3918 9.3431950271769e-09
3919 9.34276765158365e-09
3920 9.3423665362205e-09
3921 9.34196393535119e-09
3922 9.34157712949996e-09
3923 9.34116140300506e-09
3924 9.34076096810454e-09
3925 9.34035963876673e-09
3926 9.33999418492593e-09
3927 9.33956625530363e-09
3928 9.33915482470571e-09
3929 9.33875540292534e-09
3930 9.33837687392553e-09
3931 9.33796307994034e-09
3932 9.3375819852845e-09
3933 9.33715978378852e-09
3934 9.33678012735722e-09
3935 9.3363700357299e-09
3936 9.33595878463422e-09
3937 9.33558156134689e-09
3938 9.33516545224833e-09
3939 9.33477171942421e-09
3940 9.334362023522e-09
3941 9.33399576481808e-09
3942 9.33359222610175e-09
3943 9.33318723828069e-09
3944 9.33277969528135e-09
3945 9.33239721902235e-09
3946 9.33198315773015e-09
3947 9.33160007271505e-09
3948 9.33120570378171e-09
3949 9.33079325753095e-09
3950 9.33040707787325e-09
3951 9.32999128704093e-09
3952 9.32960342094058e-09
3953 9.32921148744797e-09
3954 9.32881885033765e-09
3955 9.32844823636064e-09
3956 9.32800802169037e-09
3957 9.32764203485231e-09
3958 9.32722630319488e-09
3959 9.32682940230078e-09
3960 9.32643933413552e-09
3961 9.32604839651296e-09
3962 9.3256632479749e-09
3963 9.32524690086534e-09
3964 9.32484978510839e-09
3965 9.32445763210388e-09
3966 9.32402772242658e-09
3967 9.32364565083693e-09
3968 9.32327302758523e-09
3969 9.32284621681101e-09
3970 9.32244963845058e-09
3971 9.32206442266076e-09
3972 9.32168462799282e-09
3973 9.32127064309785e-09
3974 9.32089692941979e-09
3975 9.3204698585897e-09
3976 9.32008884683977e-09
3977 9.31971516305446e-09
3978 9.31927206822974e-09
3979 9.31887562289485e-09
3980 9.31850432363268e-09
3981 9.31809653424709e-09
3982 9.31769578309954e-09
3983 9.31728868967113e-09
3984 9.31687870902448e-09
3985 9.31650859243433e-09
3986 9.31614506547579e-09
3987 9.3157212784703e-09
3988 9.3153548639674e-09
3989 9.31492246754317e-09
3990 9.31453845507552e-09
3991 9.3141504711805e-09
3992 9.31375335702644e-09
3993 9.31336068081545e-09
3994 9.31295771927632e-09
3995 9.31255875716297e-09
3996 9.31217646306381e-09
3997 9.31178781288927e-09
3998 9.31139704511003e-09
3999 9.31098802001151e-09
4000 9.31060552612772e-09
4001 9.31021277427585e-09
4002 9.30981769758921e-09
4003 9.30944308000226e-09
4004 9.30901908200582e-09
4005 9.30863467626941e-09
4006 9.30819010996897e-09
4007 9.30783888370562e-09
4008 9.30743655345317e-09
4009 9.30707654291096e-09
4010 9.30666645856254e-09
4011 9.30628618145413e-09
4012 9.30586992522325e-09
4013 9.30548937400078e-09
4014 9.30507578132678e-09
4015 9.30470965421898e-09
4016 9.30429824565204e-09
4017 9.30387524809839e-09
4018 9.30350260793661e-09
4019 9.30309627811266e-09
4020 9.30269641251369e-09
4021 9.30230542286331e-09
4022 9.30192755331516e-09
4023 9.30152346766827e-09
4024 9.30116461492908e-09
4025 9.30074257751018e-09
4026 9.30035780748184e-09
4027 9.29996339054523e-09
4028 9.29955555217105e-09
4029 9.29916850098134e-09
4030 9.29879329644723e-09
4031 9.29838180135922e-09
4032 9.29799602460146e-09
4033 9.29759490964077e-09
4034 9.2971835032804e-09
4035 9.29681629494028e-09
4036 9.29641189275493e-09
4037 9.29601490747001e-09
4038 9.29563425237229e-09
4039 9.29523385713449e-09
4040 9.29484194708147e-09
4041 9.29447340569656e-09
4042 9.29407607648536e-09
4043 9.29366687878186e-09
4044 9.29328360447373e-09
4045 9.29289184502941e-09
4046 9.2924931838767e-09
4047 9.29212499369003e-09
4048 9.29172916596527e-09
4049 9.29134127215792e-09
4050 9.290939994612e-09
4051 9.2905435844573e-09
4052 9.29019311035617e-09
4053 9.28975306310753e-09
4054 9.28936883173165e-09
4055 9.28895323789453e-09
4056 9.28860336976006e-09
4057 9.28818365871636e-09
4058 9.28778020620191e-09
4059 9.28742202581379e-09
4060 9.28702575992563e-09
4061 9.28661945909237e-09
4062 9.28624647677373e-09
4063 9.28583689767393e-09
4064 9.28544706843559e-09
4065 9.28509253145943e-09
4066 9.28467171075476e-09
4067 9.2842768667431e-09
4068 9.2839104215911e-09
4069 9.28350688671198e-09
4070 9.28313053856489e-09
4071 9.28271193395558e-09
4072 9.28235641198955e-09
4073 9.28195256780229e-09
4074 9.28152557888584e-09
4075 9.28115957395115e-09
4076 9.28078302835089e-09
4077 9.28038116075613e-09
4078 9.279982352145e-09
4079 9.27959696648412e-09
4080 9.27923256397595e-09
4081 9.2788021521864e-09
4082 9.27842033013326e-09
4083 9.27805178986552e-09
4084 9.27765742664288e-09
4085 9.27726675531426e-09
4086 9.27686168353953e-09
4087 9.27648364894285e-09
4088 9.27610015512975e-09
4089 9.27570136913247e-09
4090 9.27533499319594e-09
4091 9.27490963913147e-09
4092 9.27455103789942e-09
4093 9.27414712140889e-09
4094 9.27377790382877e-09
4095 9.27335941292018e-09
4096 9.27296885709639e-09
4097 9.27260506883298e-09
4098 9.2721694125511e-09
4099 9.27181559642887e-09
4100 9.27141049195806e-09
4101 9.2710401120577e-09
4102 9.27060913833572e-09
4103 9.27023562945217e-09
4104 9.26986338053298e-09
4105 9.26947707347026e-09
4106 9.26907984055703e-09
4107 9.26868032318645e-09
4108 9.26829878822311e-09
4109 9.2679120579503e-09
4110 9.267560645565e-09
4111 9.26712632055848e-09
4112 9.26673339766287e-09
4113 9.26634529414827e-09
4114 9.26594800704228e-09
4115 9.26556377656151e-09
4116 9.26519625375766e-09
4117 9.26479297778254e-09
4118 9.26439972741971e-09
4119 9.26401328841392e-09
4120 9.26362939945857e-09
4121 9.26324457227457e-09
4122 9.26285927711074e-09
4123 9.26246721087709e-09
4124 9.26207605996682e-09
4125 9.26167137733219e-09
4126 9.26130992639357e-09
4127 9.26089182737983e-09
4128 9.26051496449171e-09
4129 9.26013011170318e-09
4130 9.25972470566805e-09
4131 9.25935593511229e-09
4132 9.25895945249472e-09
4133 9.25856113560136e-09
4134 9.25817805148832e-09
4135 9.25779087074252e-09
4136 9.25739274650755e-09
4137 9.25702354146601e-09
4138 9.25663744612976e-09
4139 9.25625082046771e-09
4140 9.25586316418564e-09
4141 9.25543362238074e-09
4142 9.25507326568486e-09
4143 9.25470803905815e-09
4144 9.25428521061922e-09
4145 9.25390742187449e-09
4146 9.25352905198351e-09
4147 9.25316595033754e-09
4148 9.25275946734444e-09
4149 9.25236638141952e-09
4150 9.25194705023863e-09
4151 9.25161707079208e-09
4152 9.25120465839618e-09
4153 9.25080339406192e-09
4154 9.250427430052e-09
4155 9.25002806431013e-09
4156 9.24965396337935e-09
4157 9.24924543625749e-09
4158 9.24886704419675e-09
4159 9.24847012374885e-09
4160 9.24807866104632e-09
4161 9.24768948884491e-09
4162 9.24732959745739e-09
4163 9.24693244971214e-09
4164 9.24653312581875e-09
4165 9.24613546944564e-09
4166 9.24573495261066e-09
4167 9.2453612442478e-09
4168 9.24499414908103e-09
4169 9.24459870156602e-09
4170 9.2442103994464e-09
4171 9.24383574075344e-09
4172 9.24344319554876e-09
4173 9.24306585017159e-09
4174 9.24266089988302e-09
4175 9.24228469854904e-09
4176 9.24189856887914e-09
4177 9.24150386815564e-09
4178 9.24116376357226e-09
4179 9.24075163804411e-09
4180 9.2403384312767e-09
4181 9.23995355255058e-09
4182 9.23959319864415e-09
4183 9.23920972382974e-09
4184 9.23881554840134e-09
4185 9.23840897349565e-09
4186 9.23806652221998e-09
4187 9.23765649528396e-09
4188 9.23727496891791e-09
4189 9.23688565387243e-09
4190 9.23651029802186e-09
4191 9.23614706755532e-09
4192 9.23574320885884e-09
4193 9.23536155162524e-09
4194 9.23498800982359e-09
4195 9.23458558614587e-09
4196 9.23421211264375e-09
4197 9.23382575772347e-09
4198 9.23348187013145e-09
4199 9.23304833336247e-09
4200 9.23267143222717e-09
4201 9.23230133821618e-09
4202 9.23190456215972e-09
4203 9.23152260651899e-09
4204 9.23113758259653e-09
4205 9.2307263575564e-09
4206 9.23037806076221e-09
4207 9.22998967128191e-09
4208 9.22960380204951e-09
4209 9.22921175381536e-09
4210 9.22883804520963e-09
4211 9.22842397062945e-09
4212 9.2280550464674e-09
4213 9.22766440943773e-09
4214 9.22727984423166e-09
4215 9.22689973118646e-09
4216 9.2265131969721e-09
4217 9.22617258401759e-09
4218 9.22575457397434e-09
4219 9.22538899006398e-09
4220 9.2249719514173e-09
4221 9.22460511994239e-09
4222 9.22423166816594e-09
4223 9.22383265154064e-09
4224 9.22346340277019e-09
4225 9.22306413543572e-09
4226 9.22269416057248e-09
4227 9.22232652564303e-09
4228 9.22190567486519e-09
4229 9.22157195054618e-09
4230 9.22115225039655e-09
4231 9.22079086763949e-09
4232 9.22038265457892e-09
4233 9.22003503036478e-09
4234 9.21962538434629e-09
4235 9.21923616472448e-09
4236 9.21886633701435e-09
4237 9.21847583439106e-09
4238 9.21811965520769e-09
4239 9.21769689154334e-09
4240 9.21733689571158e-09
4241 9.21694769414477e-09
4242 9.21655821720102e-09
4243 9.21618381315159e-09
4244 9.21580678958645e-09
4245 9.21540759076661e-09
4246 9.21505469486661e-09
4247 9.21464080310547e-09
4248 9.21428820472442e-09
4249 9.21387855041395e-09
4250 9.21349425113405e-09
4251 9.21311313746564e-09
4252 9.21274109076664e-09
4253 9.21233922939607e-09
4254 9.21200291417734e-09
4255 9.21161541919679e-09
4256 9.21122637505961e-09
4257 9.21084831447677e-09
4258 9.21045011797322e-09
4259 9.21005060329494e-09
4260 9.20971718065128e-09
4261 9.20929560510597e-09
4262 9.20894684466184e-09
4263 9.20853358390983e-09
4264 9.20816847663902e-09
4265 9.20776643283799e-09
4266 9.20742210265557e-09
4267 9.20702690262315e-09
4268 9.20663859317605e-09
4269 9.20626009764947e-09
4270 9.20587053429567e-09
4271 9.20549734349102e-09
4272 9.20513422627417e-09
4273 9.20472696758212e-09
4274 9.20437140588398e-09
4275 9.20400468976124e-09
4276 9.20359909695884e-09
4277 9.20323180095273e-09
4278 9.20281854918659e-09
4279 9.20248670806256e-09
4280 9.20210420383288e-09
4281 9.20168747901851e-09
4282 9.20132340678781e-09
4283 9.20095291322837e-09
4284 9.20055987767981e-09
4285 9.20017303880638e-09
4286 9.1998178378197e-09
4287 9.19941968385851e-09
4288 9.19903908375153e-09
4289 9.19864407079168e-09
4290 9.19829115899468e-09
4291 9.19790707596541e-09
4292 9.19751733362978e-09
4293 9.19716323546233e-09
4294 9.19676446416856e-09
4295 9.19640105474795e-09
4296 9.19603145573683e-09
4297 9.19565388274313e-09
4298 9.19526148964594e-09
4299 9.19488154559056e-09
4300 9.19449583096366e-09
4301 9.19414597636003e-09
4302 9.19376343613831e-09
4303 9.19336128243214e-09
4304 9.192986637728e-09
4305 9.19259762276886e-09
4306 9.19223520681051e-09
4307 9.19184860126432e-09
4308 9.19147071971188e-09
4309 9.19111105923687e-09
4310 9.19069633074227e-09
4311 9.19035924295403e-09
4312 9.1899462121639e-09
4313 9.18960311024203e-09
4314 9.18921631270358e-09
4315 9.18885580779294e-09
4316 9.18844335590085e-09
4317 9.1880691975646e-09
4318 9.18770223113513e-09
4319 9.18733657895993e-09
4320 9.18693432507695e-09
4321 9.18655792883638e-09
4322 9.1861864337725e-09
4323 9.18580958551157e-09
4324 9.18542613330409e-09
4325 9.18506526603052e-09
4326 9.1847021346167e-09
4327 9.18428710727781e-09
4328 9.1839307564609e-09
4329 9.18355768984164e-09
4330 9.18317428363902e-09
4331 9.18280021330875e-09
4332 9.18243555784321e-09
4333 9.18204028060865e-09
4334 9.18167937816877e-09
4335 9.18127281273468e-09
4336 9.18093094746941e-09
4337 9.18052079636883e-09
4338 9.18019569595002e-09
4339 9.17976871297327e-09
4340 9.17939300453668e-09
4341 9.17903311284385e-09
4342 9.17865962346554e-09
4343 9.17826383825537e-09
4344 9.17790383623412e-09
4345 9.1774923575566e-09
4346 9.17711088235101e-09
4347 9.17678899840096e-09
4348 9.176403561309e-09
4349 9.17600137224223e-09
4350 9.17565516763558e-09
4351 9.17525897797117e-09
4352 9.17486485615265e-09
4353 9.17448458213205e-09
4354 9.1741442924953e-09
4355 9.17376573107004e-09
4356 9.17334949957632e-09
4357 9.17299686372525e-09
4358 9.17262847326822e-09
4359 9.1722410803935e-09
4360 9.17188449759548e-09
4361 9.17147148650488e-09
4362 9.17111024616862e-09
4363 9.17072696469956e-09
4364 9.17036321497478e-09
4365 9.16997960404703e-09
4366 9.16960884633083e-09
4367 9.16925828566007e-09
4368 9.16883546385472e-09
4369 9.16847397663262e-09
4370 9.16811114395205e-09
4371 9.16770290187996e-09
4372 9.16735308531536e-09
4373 9.16697948747519e-09
4374 9.16659427596667e-09
4375 9.1662467112813e-09
4376 9.16584102873125e-09
4377 9.16548231701814e-09
4378 9.16512001149922e-09
4379 9.16469508204648e-09
4380 9.16433757065793e-09
4381 9.16396528485158e-09
4382 9.16358762072939e-09
4383 9.16322360996341e-09
4384 9.16287286793771e-09
4385 9.16248435205158e-09
4386 9.16209967746773e-09
4387 9.16171294362139e-09
4388 9.16137475109185e-09
4389 9.16096825183399e-09
4390 9.16060037988581e-09
4391 9.16022432748825e-09
4392 9.15986523948586e-09
4393 9.15949684231893e-09
4394 9.15910412137982e-09
4395 9.15872752442481e-09
4396 9.15837436128025e-09
4397 9.15795761020216e-09
4398 9.15758424542556e-09
4399 9.1572218806138e-09
4400 9.1568354513849e-09
4401 9.15647166672973e-09
4402 9.15612211609129e-09
4403 9.15572192670272e-09
4404 9.15535402386258e-09
4405 9.15497382419223e-09
4406 9.15461625441982e-09
4407 9.15423075061733e-09
4408 9.15382249994101e-09
4409 9.1534625913936e-09
4410 9.15311733131613e-09
4411 9.15274623709134e-09
4412 9.15234744683357e-09
4413 9.15194737530906e-09
4414 9.15158460010335e-09
4415 9.15123988522171e-09
4416 9.1508622368161e-09
4417 9.15048329543783e-09
4418 9.1500955930815e-09
4419 9.14975832992659e-09
4420 9.14935131418604e-09
4421 9.14900605560043e-09
4422 9.14860713285143e-09
4423 9.14823824892108e-09
4424 9.14789164676438e-09
4425 9.14748809432986e-09
4426 9.14710110001132e-09
4427 9.14674998064163e-09
4428 9.14637349121866e-09
4429 9.14599887804485e-09
4430 9.14561622315097e-09
4431 9.14523533691869e-09
4432 9.14488465573321e-09
4433 9.14450277090389e-09
4434 9.1441378693366e-09
4435 9.1437631559721e-09
4436 9.14336542959249e-09
4437 9.14300641121496e-09
4438 9.14264820583988e-09
4439 9.14226409374358e-09
4440 9.14191138209203e-09
4441 9.14150599962138e-09
4442 9.14112903645237e-09
4443 9.14076717414142e-09
4444 9.14038289823171e-09
4445 9.14003836173516e-09
4446 9.13964288548619e-09
4447 9.1392698112619e-09
4448 9.13890507280718e-09
4449 9.13853980614998e-09
4450 9.13818229382468e-09
4451 9.13778526179937e-09
4452 9.13741994103962e-09
4453 9.13705032928869e-09
4454 9.13666773244559e-09
4455 9.13631931631631e-09
4456 9.13593151306846e-09
4457 9.13558806137083e-09
4458 9.1351698006531e-09
4459 9.13482316221292e-09
4460 9.13445651347378e-09
4461 9.13408079007e-09
4462 9.13370078013675e-09
4463 9.13334323125736e-09
4464 9.13296335391944e-09
4465 9.13257142162033e-09
4466 9.13222457612356e-09
4467 9.13184269959316e-09
4468 9.13149707612582e-09
4469 9.13111481681828e-09
4470 9.13073347681703e-09
4471 9.13036633696379e-09
4472 9.12998576887286e-09
4473 9.12964619677326e-09
4474 9.12924929708653e-09
4475 9.12889648461979e-09
4476 9.12854253613815e-09
4477 9.12816423789126e-09
4478 9.12777960051375e-09
4479 9.12740712248616e-09
4480 9.12702804541088e-09
4481 9.12666516404703e-09
4482 9.12629684932803e-09
4483 9.12594219136531e-09
4484 9.12554591975256e-09
4485 9.12522503933311e-09
4486 9.1248232497454e-09
4487 9.12443332087148e-09
4488 9.12407509854468e-09
4489 9.1237238439984e-09
4490 9.12333678559918e-09
4491 9.12294924778484e-09
4492 9.12260689047567e-09
4493 9.12222374415544e-09
4494 9.12186987247349e-09
4495 9.12147474268349e-09
4496 9.12111127041731e-09
4497 9.12075492031511e-09
4498 9.12038035388169e-09
4499 9.12000307689426e-09
4500 9.11964333721871e-09
4501 9.11928445002791e-09
4502 9.11890285826278e-09
4503 9.11853991598932e-09
4504 9.11818938260922e-09
4505 9.11780604805762e-09
4506 9.11743443899465e-09
4507 9.11707187730565e-09
4508 9.11669878670557e-09
4509 9.1163433481728e-09
4510 9.11597818400306e-09
4511 9.11558640809274e-09
4512 9.11525737948282e-09
4513 9.11489982836217e-09
4514 9.11451206538766e-09
4515 9.11412823239449e-09
4516 9.11374792958441e-09
4517 9.11338971684716e-09
4518 9.11304921952932e-09
4519 9.11268334767568e-09
4520 9.11226302895141e-09
4521 9.11193268030019e-09
4522 9.11155538680514e-09
4523 9.11121919162233e-09
4524 9.11084472174362e-09
4525 9.11046415898176e-09
4526 9.11010144018232e-09
4527 9.10973102541401e-09
4528 9.10937621202701e-09
4529 9.1090061268978e-09
4530 9.10861817353392e-09
4531 9.10827790370078e-09
4532 9.10792532866911e-09
4533 9.10754481250192e-09
4534 9.10718335019045e-09
4535 9.106826222266e-09
4536 9.10643746498269e-09
4537 9.10608100301163e-09
4538 9.1057119646909e-09
4539 9.10533650504142e-09
4540 9.10497688581813e-09
4541 9.10459907547262e-09
4542 9.10422540047889e-09
4543 9.10389181741494e-09
4544 9.1035182134061e-09
4545 9.10310205472914e-09
4546 9.10276212106459e-09
4547 9.10239680775721e-09
4548 9.10205572936251e-09
4549 9.10165507010674e-09
4550 9.10128581758929e-09
4551 9.10092938407464e-09
4552 9.10055013370048e-09
4553 9.10022908280628e-09
4554 9.0998213565785e-09
4555 9.0994634978428e-09
4556 9.09910119864521e-09
4557 9.09872189011618e-09
4558 9.09835429052652e-09
4559 9.09799235593311e-09
4560 9.09762752773075e-09
4561 9.09729176685331e-09
4562 9.09689768385097e-09
4563 9.09655266516374e-09
4564 9.09618125399109e-09
4565 9.0958120432319e-09
4566 9.09542883311548e-09
4567 9.09508918782442e-09
4568 9.09470964101378e-09
4569 9.09435572447881e-09
4570 9.09396179146066e-09
4571 9.09363034081595e-09
4572 9.09322034040733e-09
4573 9.09288189895191e-09
4574 9.09250734407868e-09
4575 9.09214717444151e-09
4576 9.09178674753791e-09
4577 9.0914291327529e-09
4578 9.09104261223592e-09
4579 9.0906750473338e-09
4580 9.09032591269593e-09
4581 9.08997780724174e-09
4582 9.08958179103581e-09
4583 9.0892467545442e-09
4584 9.08884676518312e-09
4585 9.08851240833536e-09
4586 9.08813540936859e-09
4587 9.08775823316166e-09
4588 9.08739499585337e-09
4589 9.08701376072657e-09
4590 9.08666279330478e-09
4591 9.08633218162785e-09
4592 9.08596902041148e-09
4593 9.0855219512867e-09
4594 9.08523038418291e-09
4595 9.08483159930973e-09
4596 9.08449415518153e-09
4597 9.08413748939435e-09
4598 9.08373940818053e-09
4599 9.08338629640459e-09
4600 9.08303666329741e-09
4601 9.08267632730025e-09
4602 9.08231477945998e-09
4603 9.08190581380697e-09
4604 9.08156367761259e-09
4605 9.0812010842406e-09
4606 9.08085082770849e-09
4607 9.08048486192303e-09
4608 9.08009298849349e-09
4609 9.07976442514335e-09
4610 9.0793826265298e-09
4611 9.07902621171547e-09
4612 9.07865227130211e-09
4613 9.07832245610612e-09
4614 9.07792074910513e-09
4615 9.07754519806348e-09
4616 9.07720163872278e-09
4617 9.07685829389798e-09
4618 9.07646754164798e-09
4619 9.07611303133104e-09
4620 9.0757641266967e-09
4621 9.07540027535181e-09
4622 9.07498901688009e-09
4623 9.0746305412906e-09
4624 9.07429663157822e-09
4625 9.07392983609534e-09
4626 9.07355620103495e-09
4627 9.07319312783844e-09
4628 9.0728413622912e-09
4629 9.07247796452792e-09
4630 9.07209559144334e-09
4631 9.07174772608876e-09
4632 9.07136811144349e-09
4633 9.07097375502097e-09
4634 9.07065611691071e-09
4635 9.07027369240199e-09
4636 9.06993035780512e-09
4637 9.06956553804045e-09
4638 9.06919915991816e-09
4639 9.06884746592479e-09
4640 9.06846867166494e-09
4641 9.06810042366341e-09
4642 9.06773448468984e-09
4643 9.0674061142687e-09
4644 9.06698241483206e-09
4645 9.06665859769923e-09
4646 9.06630135485281e-09
4647 9.06593294375258e-09
4648 9.06558917278949e-09
4649 9.06521619001288e-09
4650 9.06483943717562e-09
4651 9.06447102692887e-09
4652 9.0641326982166e-09
4653 9.06376585541047e-09
4654 9.06341713947933e-09
4655 9.06306050658251e-09
4656 9.06267249268372e-09
4657 9.06232574769683e-09
4658 9.0619466236036e-09
4659 9.06161514879072e-09
4660 9.06123135040182e-09
4661 9.06087021504409e-09
4662 9.06054203712869e-09
4663 9.06015725525566e-09
4664 9.05978524445156e-09
4665 9.05944049114926e-09
4666 9.0590796241255e-09
4667 9.05871277277065e-09
4668 9.05836673069371e-09
4669 9.05800488356506e-09
4670 9.05762756521489e-09
4671 9.05727843512894e-09
4672 9.056909271353e-09
4673 9.05656469177285e-09
4674 9.0561974586123e-09
4675 9.05584664352005e-09
4676 9.05548783316362e-09
4677 9.05512478530795e-09
4678 9.05475710933495e-09
4679 9.05438074179365e-09
4680 9.05405995971914e-09
4681 9.05365652387191e-09
4682 9.05331189104963e-09
4683 9.05295914420401e-09
4684 9.05258960395144e-09
4685 9.05221653181576e-09
4686 9.05187854961104e-09
4687 9.05152981470897e-09
4688 9.05116750855861e-09
4689 9.05081879018499e-09
4690 9.05043827859747e-09
4691 9.05007102974115e-09
4692 9.04973572283391e-09
4693 9.04936859572741e-09
4694 9.04899654214503e-09
4695 9.04863385415428e-09
4696 9.04828538485225e-09
4697 9.04792336976073e-09
4698 9.04755127453111e-09
4699 9.04717998874427e-09
4700 9.04683218862917e-09
4701 9.04646871847936e-09
4702 9.04612632982027e-09
4703 9.0457708691663e-09
4704 9.04538978241387e-09
4705 9.04504433648506e-09
4706 9.04468221139126e-09
4707 9.04433235358187e-09
4708 9.04395270599073e-09
4709 9.04360516007508e-09
4710 9.04322128436608e-09
4711 9.04286834543799e-09
4712 9.04252922512977e-09
4713 9.04216204984654e-09
4714 9.04180583974346e-09
4715 9.04146706704606e-09
4716 9.04108676812176e-09
4717 9.04073906077968e-09
4718 9.0403711156678e-09
4719 9.04001302431262e-09
4720 9.03963454798595e-09
4721 9.03926723708204e-09
4722 9.03895334702226e-09
4723 9.03857141402314e-09
4724 9.03823596330039e-09
4725 9.03786807534418e-09
4726 9.03749419541716e-09
4727 9.03714256444976e-09
4728 9.03678351913545e-09
4729 9.03644537476184e-09
4730 9.03607781516103e-09
4731 9.0357134688579e-09
4732 9.03533504476722e-09
4733 9.03502955502261e-09
4734 9.03463264274179e-09
4735 9.03429214060836e-09
4736 9.03395645861227e-09
4737 9.03357143079708e-09
4738 9.03323466589639e-09
4739 9.03286984883095e-09
4740 9.03250046720844e-09
4741 9.03216069900181e-09
4742 9.03180194528064e-09
4743 9.0313982538115e-09
4744 9.03111833321091e-09
4745 9.03071559401475e-09
4746 9.03037324371386e-09
4747 9.03000161101009e-09
4748 9.02965143682183e-09
4749 9.02927654410773e-09
4750 9.02896485698024e-09
4751 9.02860036248315e-09
4752 9.02820853585645e-09
4753 9.02788340642613e-09
4754 9.02751032348659e-09
4755 9.027160216106e-09
4756 9.02681745968553e-09
4757 9.02643950028564e-09
4758 9.02608305786146e-09
4759 9.02573374638588e-09
4760 9.02538738844355e-09
4761 9.02500355025315e-09
4762 9.02467056375367e-09
4763 9.02430584631686e-09
4764 9.02396933088329e-09
4765 9.02360673699781e-09
4766 9.02322663275112e-09
4767 9.02286764678517e-09
4768 9.02249732151472e-09
4769 9.02218451621223e-09
4770 9.0218132937428e-09
4771 9.02147379876206e-09
4772 9.02110193744254e-09
4773 9.02075371799621e-09
4774 9.02038555503082e-09
4775 9.0200396515791e-09
4776 9.01966675590643e-09
4777 9.01931053708116e-09
4778 9.01893759922695e-09
4779 9.01860646592562e-09
4780 9.01825736272788e-09
4781 9.01790995506968e-09
4782 9.01753938360728e-09
4783 9.01716262670382e-09
4784 9.01680510995151e-09
4785 9.01648247328413e-09
4786 9.0161028941313e-09
4787 9.01579712932199e-09
4788 9.01539890511838e-09
4789 9.01506018191611e-09
4790 9.01471106850432e-09
4791 9.0143334671583e-09
4792 9.01400550676335e-09
4793 9.01365885190575e-09
4794 9.01330144657125e-09
4795 9.01293429420025e-09
4796 9.0125715393255e-09
4797 9.01223459758016e-09
4798 9.01187572831252e-09
4799 9.01152741827199e-09
4800 9.01115030958743e-09
4801 9.01081584653296e-09
4802 9.01046159194896e-09
4803 9.01013391912259e-09
4804 9.00974631289769e-09
4805 9.00939806141449e-09
4806 9.00902058720288e-09
4807 9.00868335815957e-09
4808 9.00830686064585e-09
4809 9.00798495734323e-09
4810 9.00765221198418e-09
4811 9.0072571127045e-09
4812 9.00691241886159e-09
4813 9.00655075816714e-09
4814 9.00622184204364e-09
4815 9.00587359683319e-09
4816 9.00548633325088e-09
4817 9.00516391590112e-09
4818 9.00478981097352e-09
4819 9.00440793306923e-09
4820 9.00408353506077e-09
4821 9.00375907353368e-09
4822 9.00340470664368e-09
4823 9.00302286000604e-09
4824 9.00265091181851e-09
4825 9.00234011440293e-09
4826 9.00197130398322e-09
4827 9.00160493250146e-09
4828 9.0012593670849e-09
4829 9.00089742965349e-09
4830 9.00056083765616e-09
4831 9.0001938692838e-09
4832 8.99982097738589e-09
4833 8.99949088330187e-09
4834 8.99914318530648e-09
4835 8.99877667961957e-09
4836 8.99845055778864e-09
4837 8.99806712868073e-09
4838 8.99770778520909e-09
4839 8.99737680520873e-09
4840 8.9970443969134e-09
4841 8.99667099297369e-09
4842 8.99632912518267e-09
4843 8.99593950041966e-09
4844 8.99562361913325e-09
4845 8.99525022892561e-09
4846 8.99493083860586e-09
4847 8.99456888791422e-09
4848 8.99419471360463e-09
4849 8.99383888480765e-09
4850 8.99347375084292e-09
4851 8.99315020174568e-09
4852 8.99277953861355e-09
4853 8.99242524352723e-09
4854 8.99207509230671e-09
4855 8.99171265632964e-09
4856 8.99138118841403e-09
4857 8.99101231791044e-09
4858 8.99065384717818e-09
4859 8.99031140497858e-09
4860 8.98996207106262e-09
4861 8.98960080968764e-09
4862 8.98924508448834e-09
4863 8.98889759182869e-09
4864 8.9885656176722e-09
4865 8.98818839412119e-09
4866 8.98781721226857e-09
4867 8.98747406974723e-09
4868 8.98713829385056e-09
4869 8.98677199047404e-09
4870 8.98642941436073e-09
4871 8.98607881721219e-09
4872 8.98574642336711e-09
4873 8.9853490400707e-09
4874 8.98502552400607e-09
4875 8.98466283405855e-09
4876 8.98429545605553e-09
4877 8.98396425250136e-09
4878 8.98360478451821e-09
4879 8.98326294218599e-09
4880 8.98289363795643e-09
4881 8.98254725980804e-09
4882 8.98218610802595e-09
4883 8.98181568470893e-09
4884 8.9814591943925e-09
4885 8.98114743672768e-09
4886 8.98075929543426e-09
4887 8.98041535904101e-09
4888 8.98005646647257e-09
4889 8.97971130359859e-09
4890 8.97935544034653e-09
4891 8.97897606088119e-09
4892 8.97865233522033e-09
4893 8.97830615737352e-09
4894 8.9779401868835e-09
4895 8.97760384148058e-09
4896 8.97720859925968e-09
4897 8.97691027437406e-09
4898 8.97653545565979e-09
4899 8.97616079819849e-09
4900 8.9758360493318e-09
4901 8.97548413644408e-09
4902 8.97511690890404e-09
4903 8.97477544362091e-09
4904 8.974431043033e-09
4905 8.97408397686553e-09
4906 8.97371119989651e-09
4907 8.97339735516159e-09
4908 8.97299491961837e-09
4909 8.97268470814033e-09
4910 8.97229559084986e-09
4911 8.97199059386386e-09
4912 8.97160900042299e-09
4913 8.97126384549751e-09
4914 8.97091816405918e-09
4915 8.97057106469604e-09
4916 8.97022880914575e-09
4917 8.96985896289837e-09
4918 8.96951945039692e-09
4919 8.96915729054967e-09
4920 8.96881093090038e-09
4921 8.96843726843136e-09
4922 8.96811329420891e-09
4923 8.9677551017471e-09
4924 8.9674301252049e-09
4925 8.9670634717022e-09
4926 8.96670616349835e-09
4927 8.96635686564729e-09
4928 8.96600813192136e-09
4929 8.96564148055584e-09
4930 8.96534296845539e-09
4931 8.96497575651398e-09
4932 8.96463297940173e-09
4933 8.96428693990606e-09
4934 8.96392480745009e-09
4935 8.96355726954026e-09
4936 8.96322446427775e-09
4937 8.96287506637825e-09
4938 8.96251705482382e-09
4939 8.96216278578263e-09
4940 8.96182993407463e-09
4941 8.96149835575774e-09
4942 8.96111857179652e-09
4943 8.96076682162586e-09
4944 8.96042880186437e-09
4945 8.96008342231649e-09
4946 8.95971577381111e-09
4947 8.95937685428325e-09
4948 8.95902923874273e-09
4949 8.95866178187571e-09
4950 8.95835067842368e-09
4951 8.9579822003423e-09
4952 8.95762064332534e-09
4953 8.95726035362449e-09
4954 8.95692804554413e-09
4955 8.95657097058772e-09
4956 8.95624066706707e-09
4957 8.95591666577253e-09
4958 8.95552884778647e-09
4959 8.95518586646604e-09
4960 8.95485062315723e-09
4961 8.95451706907011e-09
4962 8.9541395678662e-09
4963 8.95377818655244e-09
4964 8.95344562207648e-09
4965 8.95309748996306e-09
4966 8.95277664134456e-09
4967 8.95239781793441e-09
4968 8.95205849313352e-09
4969 8.95170768685019e-09
4970 8.95136177877023e-09
4971 8.951006638825e-09
4972 8.95067683985562e-09
4973 8.9502998036825e-09
4974 8.94996003298135e-09
4975 8.9496150434791e-09
4976 8.9492539898546e-09
4977 8.9489207797909e-09
4978 8.94859147910001e-09
4979 8.94820682197095e-09
4980 8.94789419811706e-09
4981 8.94751610737085e-09
4982 8.94718372665762e-09
4983 8.94683296039436e-09
4984 8.94648041034618e-09
4985 8.94615068107452e-09
4986 8.9457595312363e-09
4987 8.94543433744427e-09
4988 8.94510807549972e-09
4989 8.94474058132921e-09
4990 8.94436627757109e-09
4991 8.94402753638321e-09
4992 8.94372474756836e-09
4993 8.94336276519372e-09
4994 8.9430185063813e-09
4995 8.94264544449089e-09
4996 8.9423287140955e-09
4997 8.94196665233993e-09
4998 8.94159321775806e-09
4999 8.94126040633034e-09
};
\addlegendentry{Train}
\addplot [semithick, black]
table {%
0 0.00149826530832797
1 0.000262599671259522
2 0.000240551977185532
3 0.00023630257055629
4 0.000228264354518615
5 0.000205098593141884
6 0.000116963608888909
7 2.26499942073133e-05
8 1.83952288352884e-05
9 1.76151625055354e-05
10 1.69153263414046e-05
11 1.6240877812379e-05
12 1.55344096128829e-05
13 1.47565006045625e-05
14 1.38324612635188e-05
15 1.26771610666765e-05
16 1.12021170934895e-05
17 9.34593117563054e-06
18 7.21429069017177e-06
19 5.20685762239737e-06
20 3.78919753529772e-06
21 3.00057899949024e-06
22 2.57294414041098e-06
23 2.32196907745674e-06
24 2.16187186197203e-06
25 2.04777984436078e-06
26 1.95307529793354e-06
27 1.86603779184225e-06
28 1.7809591099649e-06
29 1.69486759205029e-06
30 1.60585386765888e-06
31 1.51240033119393e-06
32 1.41311818424583e-06
33 1.30800344777526e-06
34 1.20025140404323e-06
35 1.09483153210022e-06
36 9.95739014797437e-07
37 9.05420790786593e-07
38 8.26019061150873e-07
39 7.57633415560122e-07
40 7.00093380601174e-07
41 6.52666130918078e-07
42 6.12428550539335e-07
43 5.76365835058823e-07
44 5.44270960745052e-07
45 5.17696491897368e-07
46 4.95558026614162e-07
47 4.77672642773541e-07
48 4.62971371462118e-07
49 4.51058298267526e-07
50 4.41093249037294e-07
51 4.32657401461256e-07
52 4.25503372980529e-07
53 4.1937437345041e-07
54 4.14083217492589e-07
55 4.09454344207916e-07
56 4.05256827207268e-07
57 4.01280146888894e-07
58 3.97439265498178e-07
59 3.93750866578557e-07
60 3.90228393598591e-07
61 3.86874916102897e-07
62 3.83702996487045e-07
63 3.80726703497203e-07
64 3.77988044419908e-07
65 3.75527548612808e-07
66 3.7316380030461e-07
67 3.7086252291374e-07
68 3.68577445897245e-07
69 3.66281909691679e-07
70 3.63983133411239e-07
71 3.61738102583331e-07
72 3.59638505642579e-07
73 3.57484481128267e-07
74 3.55278700681083e-07
75 3.53117400209157e-07
76 3.50999727061208e-07
77 3.48910589309526e-07
78 3.46900662862026e-07
79 3.44959744325024e-07
80 3.43008395020661e-07
81 3.40941312515497e-07
82 3.39159186069082e-07
83 3.37516524950843e-07
84 3.35850614874289e-07
85 3.34292934667246e-07
86 3.32712033923599e-07
87 3.312424894375e-07
88 3.29735513560081e-07
89 3.28316559716768e-07
90 3.26855257526404e-07
91 3.25432012004967e-07
92 3.2400578220404e-07
93 3.22650009820791e-07
94 3.21295374305919e-07
95 3.20001134923587e-07
96 3.18707350288605e-07
97 3.17477514499842e-07
98 3.16262827482205e-07
99 3.15077301138444e-07
100 3.13815490926572e-07
101 3.12622290721265e-07
102 3.11337373659626e-07
103 3.10093952293755e-07
104 3.08783768332432e-07
105 3.07524771869794e-07
106 3.06229765101307e-07
107 3.04986315313727e-07
108 3.03717939686976e-07
109 3.02472898283668e-07
110 3.01223138876594e-07
111 2.99976306905592e-07
112 2.98733567660747e-07
113 2.97486764111454e-07
114 2.96232173013777e-07
115 2.94973347081395e-07
116 2.93716709620639e-07
117 2.9247323141135e-07
118 2.91239018679335e-07
119 2.90012820869379e-07
120 2.8879773594781e-07
121 2.87591490177874e-07
122 2.8639064453273e-07
123 2.85188662019209e-07
124 2.83983979443292e-07
125 2.82783787497465e-07
126 2.8157899123471e-07
127 2.8037294441674e-07
128 2.79184092732976e-07
129 2.7800430757452e-07
130 2.76831769951968e-07
131 2.7566656513045e-07
132 2.74510227882274e-07
133 2.733548001288e-07
134 2.72205738838238e-07
135 2.71059292344944e-07
136 2.69917279638321e-07
137 2.68775806944177e-07
138 2.67639535422859e-07
139 2.66501160695043e-07
140 2.6536562813817e-07
141 2.64232141944376e-07
142 2.63100531583405e-07
143 2.6196659064226e-07
144 2.60838049825907e-07
145 2.59715335459987e-07
146 2.58591711599365e-07
147 2.57467661413102e-07
148 2.5634906819505e-07
149 2.55243776337011e-07
150 2.54116713449548e-07
151 2.5296947114839e-07
152 2.51807961149098e-07
153 2.50636560394923e-07
154 2.49461180601429e-07
155 2.48277501668781e-07
156 2.47093765892714e-07
157 2.45900082518347e-07
158 2.44698270535082e-07
159 2.43486169893004e-07
160 2.42272989225967e-07
161 2.41048184079773e-07
162 2.39817637748274e-07
163 2.38573960587019e-07
164 2.37317451023955e-07
165 2.36051647561908e-07
166 2.3476968635805e-07
167 2.33471610044944e-07
168 2.32164367730547e-07
169 2.30846012527763e-07
170 2.2950966638291e-07
171 2.28163642645995e-07
172 2.26809135028816e-07
173 2.25446271429064e-07
174 2.24079286681445e-07
175 2.22703889107834e-07
176 2.21325180405074e-07
177 2.19940773149574e-07
178 2.18550951558427e-07
179 2.17158699911124e-07
180 2.15763535038604e-07
181 2.14361236317018e-07
182 2.12959491818765e-07
183 2.11551480333583e-07
184 2.1014123774421e-07
185 2.08726987693808e-07
186 2.0731202710067e-07
187 2.05886166781966e-07
188 2.04452248908638e-07
189 2.03000453780078e-07
190 2.01527740273377e-07
191 2.00036069486487e-07
192 1.98546601382077e-07
193 1.97059023321344e-07
194 1.95555458049057e-07
195 1.94053811242156e-07
196 1.92539573617978e-07
197 1.91034772001331e-07
198 1.895009376085e-07
199 1.87987396316203e-07
200 1.86425964443515e-07
201 1.84902859245994e-07
202 1.83313943580288e-07
203 1.81771554252919e-07
204 1.80144581918285e-07
205 1.7856456224763e-07
206 1.76901465920309e-07
207 1.75282650616282e-07
208 1.73598621699966e-07
209 1.71941280768806e-07
210 1.70250913811287e-07
211 1.68551679280426e-07
212 1.66841985560495e-07
213 1.65108502869771e-07
214 1.63376398631954e-07
215 1.61620690164455e-07
216 1.59844219638217e-07
217 1.58055669885471e-07
218 1.56243814330992e-07
219 1.54421954334794e-07
220 1.5258862617884e-07
221 1.50748306282367e-07
222 1.48848855019423e-07
223 1.4702402495459e-07
224 1.45160683473478e-07
225 1.43244136552312e-07
226 1.41278860610328e-07
227 1.39394970233297e-07
228 1.37541789513307e-07
229 1.35642565624039e-07
230 1.33760210019318e-07
231 1.31936687353118e-07
232 1.30036426071456e-07
233 1.28212079175682e-07
234 1.26421824120371e-07
235 1.24577368865175e-07
236 1.22739038488362e-07
237 1.20979052553594e-07
238 1.1920160147838e-07
239 1.17487218176393e-07
240 1.15764464680979e-07
241 1.14042499888001e-07
242 1.1237582953072e-07
243 1.10741481762489e-07
244 1.09027006089946e-07
245 1.07464742882257e-07
246 1.05799045968524e-07
247 1.04293739866534e-07
248 1.02642516708329e-07
249 1.01164971511025e-07
250 9.95580577978217e-08
251 9.81460317461824e-08
252 9.66403064239785e-08
253 9.52678504972937e-08
254 9.38773965231121e-08
255 9.26080900853776e-08
256 9.13539395241969e-08
257 9.01337529057855e-08
258 8.92059190960026e-08
259 8.79614745485924e-08
260 8.69121237201398e-08
261 8.55753157225081e-08
262 8.44404937083709e-08
263 8.32751325674508e-08
264 8.22700840785728e-08
265 8.10618274726949e-08
266 8.0205346364437e-08
267 7.90906469205765e-08
268 7.82248292807708e-08
269 7.73756738681186e-08
270 7.64946435083402e-08
271 7.56640829990829e-08
272 7.49649728959412e-08
273 7.4191660814904e-08
274 7.33507690142687e-08
275 7.25684756730516e-08
276 7.19830168804947e-08
277 7.12045959971874e-08
278 7.0533786811211e-08
279 7.00737743386526e-08
280 6.93132946594233e-08
281 6.8709468337147e-08
282 6.80913601058819e-08
283 6.76247537967356e-08
284 6.69873685410494e-08
285 6.63987123061816e-08
286 6.58079173376791e-08
287 6.54010889888923e-08
288 6.48432987304659e-08
289 6.43995221594196e-08
290 6.37432506778168e-08
291 6.34849115499492e-08
292 6.27300735800418e-08
293 6.2191986671678e-08
294 6.17054851659304e-08
295 6.13283503980711e-08
296 6.07890626724839e-08
297 6.03446892455395e-08
298 5.99231952946866e-08
299 5.95790083934844e-08
300 5.91287161455512e-08
301 5.87801700646651e-08
302 5.8362637389564e-08
303 5.79759387164813e-08
304 5.76377487959689e-08
305 5.72932705722451e-08
306 5.69648719306315e-08
307 5.66107196675603e-08
308 5.62990365438054e-08
309 5.59235999730845e-08
310 5.5692844114219e-08
311 5.53881136511336e-08
312 5.5100407791997e-08
313 5.48110143938629e-08
314 5.46810525747787e-08
315 5.43433600341814e-08
316 5.42428573169218e-08
317 5.38312008302455e-08
318 5.36328990108359e-08
319 5.34042889910324e-08
320 5.32410560083463e-08
321 5.3000398736458e-08
322 5.28220134299318e-08
323 5.2641709658019e-08
324 5.24254311073946e-08
325 5.22462393348633e-08
326 5.21442409251449e-08
327 5.19050828984291e-08
328 5.1809298184935e-08
329 5.15714511095666e-08
330 5.14776488103053e-08
331 5.12534725771729e-08
332 5.11024573768282e-08
333 5.0945970997418e-08
334 5.08133410903611e-08
335 5.06706498981657e-08
336 5.05009722928662e-08
337 5.03462587175818e-08
338 5.02177357475375e-08
339 5.01013310838516e-08
340 5.00149468507516e-08
341 4.98691434813736e-08
342 4.97579755176503e-08
343 4.96341634459441e-08
344 4.95199543593117e-08
345 4.94064771316971e-08
346 4.93056866446295e-08
347 4.91965330695621e-08
348 4.91015370585046e-08
349 4.90723728319153e-08
350 4.89182383489606e-08
351 4.88217146710213e-08
352 4.8754987602706e-08
353 4.8638639782439e-08
354 4.84914330911579e-08
355 4.83974282872168e-08
356 4.82796025380594e-08
357 4.81838071664242e-08
358 4.80845372408112e-08
359 4.79824109334004e-08
360 4.79081521120861e-08
361 4.78000323766992e-08
362 4.76937458415705e-08
363 4.75768544561106e-08
364 4.74702268604688e-08
365 4.73515022747506e-08
366 4.72511381133245e-08
367 4.71265053647585e-08
368 4.70237218053171e-08
369 4.69070080555412e-08
370 4.67938257031619e-08
371 4.67007517102047e-08
372 4.65710741082148e-08
373 4.64643399311626e-08
374 4.63615990042854e-08
375 4.62116460653306e-08
376 4.61101237192452e-08
377 4.6002238462961e-08
378 4.58902675859463e-08
379 4.5777550639059e-08
380 4.56676545468326e-08
381 4.55582984670855e-08
382 4.5444952689877e-08
383 4.53358204310916e-08
384 4.52248123394838e-08
385 4.51138149060171e-08
386 4.5001847581716e-08
387 4.48953230147708e-08
388 4.47784351820246e-08
389 4.46767280948279e-08
390 4.45619434685796e-08
391 4.4457745929094e-08
392 4.43409575723308e-08
393 4.42367991126957e-08
394 4.41212435475791e-08
395 4.40200516038658e-08
396 4.39077645353336e-08
397 4.38107647937613e-08
398 4.36979092910406e-08
399 4.36050981988956e-08
400 4.34894360523685e-08
401 4.33984652659092e-08
402 4.32834106334212e-08
403 4.31894378039033e-08
404 4.30814957042003e-08
405 4.29864925877155e-08
406 4.28718891498647e-08
407 4.27833661831301e-08
408 4.2668279576219e-08
409 4.25793338365565e-08
410 4.24704076351645e-08
411 4.23839807695003e-08
412 4.22747667983003e-08
413 4.21789998483746e-08
414 4.20772678921821e-08
415 4.19795114225963e-08
416 4.18767953647148e-08
417 4.17805061658783e-08
418 4.16827887761428e-08
419 4.15863432579044e-08
420 4.14887999511393e-08
421 4.13973566537607e-08
422 4.13007192889836e-08
423 4.1206078549294e-08
424 4.11165927971524e-08
425 4.10223854885317e-08
426 4.09272153945039e-08
427 4.08399110085611e-08
428 4.07464284535308e-08
429 4.06592448598531e-08
430 4.05658653335195e-08
431 4.04796161035392e-08
432 4.03894055978071e-08
433 4.03039344121225e-08
434 4.02133402133131e-08
435 4.01334681043863e-08
436 4.00366957364895e-08
437 3.99666504335983e-08
438 3.98688833058713e-08
439 3.97959034614814e-08
440 3.96992341222813e-08
441 3.96292954008004e-08
442 3.95360082450225e-08
443 3.9465412271511e-08
444 3.93749886029582e-08
445 3.93004349064086e-08
446 3.92136527693765e-08
447 3.91413408351582e-08
448 3.90578520637064e-08
449 3.89839946990378e-08
450 3.89061156624848e-08
451 3.8833665172433e-08
452 3.87531855494672e-08
453 3.86809624330908e-08
454 3.86019856080111e-08
455 3.85329599339457e-08
456 3.84598273228676e-08
457 3.83814402482585e-08
458 3.83078599952569e-08
459 3.82355516137522e-08
460 3.81691513950955e-08
461 3.80970064384201e-08
462 3.80251421461253e-08
463 3.79582694165492e-08
464 3.78908957543445e-08
465 3.7819912535042e-08
466 3.7757153847906e-08
467 3.76863660278559e-08
468 3.76234154941812e-08
469 3.75565605281736e-08
470 3.74905759770172e-08
471 3.74267870029144e-08
472 3.73612500936815e-08
473 3.72988822050502e-08
474 3.72355053457341e-08
475 3.71745514371469e-08
476 3.71114765584935e-08
477 3.7050138956829e-08
478 3.69907020569826e-08
479 3.69287569412791e-08
480 3.68691672747445e-08
481 3.68101282788302e-08
482 3.67492063446662e-08
483 3.66914392202489e-08
484 3.66318069211502e-08
485 3.657524416667e-08
486 3.65124250834015e-08
487 3.6459006480527e-08
488 3.64046499612414e-08
489 3.63469325748156e-08
490 3.62913219476013e-08
491 3.62360630390413e-08
492 3.61823708772135e-08
493 3.61277123772652e-08
494 3.60740379790059e-08
495 3.6020480820298e-08
496 3.59657867932128e-08
497 3.59152849682687e-08
498 3.5862665725972e-08
499 3.58095313401918e-08
500 3.57600704603556e-08
501 3.57073481893622e-08
502 3.56546792090739e-08
503 3.56057050510117e-08
504 3.55557432385467e-08
505 3.5505944850911e-08
506 3.54571128013959e-08
507 3.54062059670923e-08
508 3.53567131128329e-08
509 3.53103821737477e-08
510 3.52637563594271e-08
511 3.52132012437778e-08
512 3.51664155573417e-08
513 3.5119281704965e-08
514 3.50710003260701e-08
515 3.50269324655983e-08
516 3.49781252850789e-08
517 3.49332580640294e-08
518 3.48905047076187e-08
519 3.48434383568019e-08
520 3.47993953653258e-08
521 3.47521336152568e-08
522 3.4705355034248e-08
523 3.46581998655893e-08
524 3.46146684648829e-08
525 3.45601698370501e-08
526 3.45201840445952e-08
527 3.44878650082592e-08
528 3.44258026530042e-08
529 3.44181891875905e-08
530 3.43420119008897e-08
531 3.43315242901099e-08
532 3.42549348886223e-08
533 3.42454136159631e-08
534 3.42031114541896e-08
535 3.41513732848853e-08
536 3.41246177981702e-08
537 3.40829053868674e-08
538 3.40421699718263e-08
539 3.40059784775804e-08
540 3.3967999968354e-08
541 3.39259784709611e-08
542 3.38894849960525e-08
543 3.38507213371031e-08
544 3.38068524285973e-08
545 3.37473906597552e-08
546 3.37041754505663e-08
547 3.36801910805207e-08
548 3.36484866636511e-08
549 3.36119825306014e-08
550 3.3575950908471e-08
551 3.35437917442505e-08
552 3.35087051439587e-08
553 3.34753522679421e-08
554 3.34420029446392e-08
555 3.34105330068724e-08
556 3.33708989330717e-08
557 3.334260156862e-08
558 3.33056782153562e-08
559 3.32760805576982e-08
560 3.32413314652058e-08
561 3.32075522635478e-08
562 3.31770415584742e-08
563 3.31413261278612e-08
564 3.31114726748183e-08
565 3.30784963864517e-08
566 3.30476979115701e-08
567 3.30159721784185e-08
568 3.29875824434112e-08
569 3.29520659647642e-08
570 3.29282734412573e-08
571 3.28920890524387e-08
572 3.28674190086531e-08
573 3.28309788244496e-08
574 3.28048983533336e-08
575 3.27709379632779e-08
576 3.27451594728245e-08
577 3.2712613062813e-08
578 3.26865858824021e-08
579 3.2656789272778e-08
580 3.2627678336894e-08
581 3.25985922700056e-08
582 3.25711866366873e-08
583 3.25456070982e-08
584 3.25098454823092e-08
585 3.24903908222041e-08
586 3.24612834390337e-08
587 3.24386988381775e-08
588 3.24127746864633e-08
589 3.23937676682817e-08
590 3.23732436413593e-08
591 3.23635482857298e-08
592 3.23421573966698e-08
593 3.23161941651051e-08
594 3.22934994301249e-08
595 3.22712310207862e-08
596 3.22414415165895e-08
597 3.22172049038727e-08
598 3.21867759112138e-08
599 3.21611324238802e-08
600 3.21346114162679e-08
601 3.21050137586099e-08
602 3.20748263504811e-08
603 3.20598907421754e-08
604 3.2017311468735e-08
605 3.19993525010887e-08
606 3.19677155857789e-08
607 3.19440403018234e-08
608 3.19203223853037e-08
609 3.18994111125903e-08
610 3.18567678903037e-08
611 3.18402477716972e-08
612 3.18018784639662e-08
613 3.17837205443539e-08
614 3.17486268386347e-08
615 3.17400257188183e-08
616 3.17197823562765e-08
617 3.16708010927869e-08
618 3.16432924307719e-08
619 3.16285557744322e-08
620 3.16030650537868e-08
621 3.15701775832622e-08
622 3.15405976891725e-08
623 3.15210115786613e-08
624 3.15038910514431e-08
625 3.14763326514367e-08
626 3.14256851652317e-08
627 3.14129806611163e-08
628 3.14176418214629e-08
629 3.13946486585337e-08
630 3.13599386458918e-08
631 3.13071168989154e-08
632 3.12874846031264e-08
633 3.12588070983111e-08
634 3.12449124351133e-08
635 3.12557872916841e-08
636 3.1237604503076e-08
637 3.1209697937129e-08
638 3.11818375564599e-08
639 3.11245251793935e-08
640 3.10921954849164e-08
641 3.10738990094706e-08
642 3.10473069475847e-08
643 3.10319414609239e-08
644 3.09890531013934e-08
645 3.09596437375603e-08
646 3.09468042303251e-08
647 3.09491348104984e-08
648 3.08824894545978e-08
649 3.08548742111725e-08
650 3.08630028200696e-08
651 3.08126786308094e-08
652 3.08075449595435e-08
653 3.07731617965601e-08
654 3.07635339424905e-08
655 3.07134726540426e-08
656 3.07392546972096e-08
657 3.07814111977223e-08
658 3.06564373886431e-08
659 3.06281791040419e-08
660 3.06262784022238e-08
661 3.05805869516007e-08
662 3.05727567706526e-08
663 3.0531495553987e-08
664 3.05307139569777e-08
665 3.05120053667451e-08
666 3.04942027185007e-08
667 3.0444436305288e-08
668 3.04436973408428e-08
669 3.04554603758334e-08
670 3.043760088417e-08
671 3.04561069697229e-08
672 3.03603648887929e-08
673 3.0372873993656e-08
674 3.03414324775986e-08
675 3.03141511892591e-08
676 3.02793523587752e-08
677 3.02517513262046e-08
678 3.02223099879484e-08
679 3.02156877296511e-08
680 3.02253688744258e-08
681 3.02284632880401e-08
682 3.01587803619441e-08
683 3.0152996544075e-08
684 3.01151636961094e-08
685 3.01349523113004e-08
686 3.0084276403386e-08
687 3.00536271424789e-08
688 3.0030410158588e-08
689 3.00044789014464e-08
690 3.00383042883823e-08
691 2.9999483786014e-08
692 2.993887093794e-08
693 2.99462676878193e-08
694 2.9943659995979e-08
695 2.99564639760774e-08
696 2.98820346245066e-08
697 2.98967037792863e-08
698 2.99079161436566e-08
699 2.98536910747771e-08
700 2.98142452948014e-08
701 2.98276781052209e-08
702 2.97529592074852e-08
703 2.97465057030877e-08
704 2.97204412191832e-08
705 2.96961122359107e-08
706 2.96420967771382e-08
707 2.96411268863039e-08
708 2.96640187968933e-08
709 2.96041484659781e-08
710 2.96882820549627e-08
711 2.96188193971147e-08
712 2.95885591583556e-08
713 2.95598372446193e-08
714 2.95977446995721e-08
715 2.95729289945257e-08
716 2.95110922365893e-08
717 2.94692430458099e-08
718 2.94521615984422e-08
719 2.95603062028249e-08
720 2.94363431407874e-08
721 2.94512005893921e-08
722 2.94046813564819e-08
723 2.94449016280396e-08
724 2.93637576476158e-08
725 2.93524706762582e-08
726 2.93011765961637e-08
727 2.92991177985868e-08
728 2.93221464886528e-08
729 2.9257593681109e-08
730 2.92592439166128e-08
731 2.92267241519539e-08
732 2.92620434549917e-08
733 2.92255890599336e-08
734 2.91991071321718e-08
735 2.92250703637364e-08
736 2.91377961758599e-08
737 2.91333250856951e-08
738 2.91047133060829e-08
739 2.9143922830599e-08
740 2.91363164706127e-08
741 2.91258057671939e-08
742 2.90753217058182e-08
743 2.90871025043771e-08
744 2.90884720755002e-08
745 2.90161867866345e-08
746 2.89760002658568e-08
747 2.89657986485281e-08
748 2.8947660268841e-08
749 2.8948191399536e-08
750 2.89943482556509e-08
751 2.89324990632167e-08
752 2.88953323490659e-08
753 2.88869710374229e-08
754 2.89085342330964e-08
755 2.88493158251413e-08
756 2.8876506519282e-08
757 2.8846597999177e-08
758 2.8840736021607e-08
759 2.87530053100227e-08
760 2.8838014642929e-08
761 2.88145898252878e-08
762 2.88461698971787e-08
763 2.88263475312078e-08
764 2.89063084579766e-08
765 2.87224857231649e-08
766 2.87277650556916e-08
767 2.87043508961915e-08
768 2.86952239747507e-08
769 2.87372969864919e-08
770 2.87035373247591e-08
771 2.87133037346621e-08
772 2.86386292458474e-08
773 2.85848038572567e-08
774 2.85092962570843e-08
775 2.85615904260794e-08
776 2.84661592075963e-08
777 2.87137105203783e-08
778 2.8543510666168e-08
779 2.84548065110357e-08
780 2.84664647409727e-08
781 2.85403789490601e-08
782 2.84507226666619e-08
783 2.84160428520863e-08
784 2.84138650386012e-08
785 2.84978654008228e-08
786 2.84841323860974e-08
787 2.84302998920793e-08
788 2.84376877601744e-08
789 2.84371903802594e-08
790 2.83022139058176e-08
791 2.8384523176328e-08
792 2.82766574599691e-08
793 2.83657062283282e-08
794 2.83834804548633e-08
795 2.83112093768523e-08
796 2.82477294888395e-08
797 2.82669621043397e-08
798 2.83593255545611e-08
799 2.82608709767374e-08
800 2.81842513771835e-08
801 2.84093815139386e-08
802 2.81666352464072e-08
803 2.81138916591317e-08
804 2.81032441762363e-08
805 2.81770962118344e-08
806 2.81083014641581e-08
807 2.82244236871065e-08
808 2.82110210747533e-08
809 2.80180714185008e-08
810 2.81287118042428e-08
811 2.81295360338163e-08
812 2.80469585334231e-08
813 2.83125327626976e-08
814 2.80973946331642e-08
815 2.79608922681973e-08
816 2.81847167826754e-08
817 2.80102430139095e-08
818 2.82434235998608e-08
819 2.79870935315785e-08
820 2.79736465103042e-08
821 2.79906657851825e-08
822 2.80466299074078e-08
823 2.79989915696888e-08
824 2.79431056071644e-08
825 2.78687206645145e-08
826 2.83087739916255e-08
827 2.78317067170519e-08
828 2.78096035088993e-08
829 2.78393788022413e-08
830 2.7916044587073e-08
831 2.77356271283224e-08
832 2.77823186678461e-08
833 2.772788931793e-08
834 2.78023151167872e-08
835 2.77797749248521e-08
836 2.77442566698483e-08
837 2.77137157667084e-08
838 2.76837557322551e-08
839 2.82823275910005e-08
840 2.77081539934443e-08
841 2.7735731933376e-08
842 2.7690637338651e-08
843 2.82856902344975e-08
844 2.76292855261318e-08
845 2.76017697586894e-08
846 2.75844680430737e-08
847 2.77516871705075e-08
848 2.75871538946149e-08
849 2.77682730143169e-08
850 2.75490616985508e-08
851 2.759725070689e-08
852 2.7676168912194e-08
853 2.786408082045e-08
854 2.75011036166006e-08
855 2.75026774687603e-08
856 2.75318328135654e-08
857 2.7522988332862e-08
858 2.7467939034409e-08
859 2.75288236650795e-08
860 2.7431516613774e-08
861 2.74237930142363e-08
862 2.74160942836943e-08
863 2.7407260461132e-08
864 2.74186220394768e-08
865 2.74184479565065e-08
866 2.73652105420297e-08
867 2.7504130528655e-08
868 2.73311560050615e-08
869 2.74226543695022e-08
870 2.74387037535462e-08
871 2.73453295562831e-08
872 2.75005440641962e-08
873 2.72976450332862e-08
874 2.72953819546728e-08
875 2.7242531785987e-08
876 2.72595794825747e-08
877 2.73283387031142e-08
878 2.72514046883998e-08
879 2.72888289742923e-08
880 2.72690652280971e-08
881 2.72088591657393e-08
882 2.74412848000338e-08
883 2.71741384949564e-08
884 2.71833968668034e-08
885 2.72215796570663e-08
886 2.71864060152893e-08
887 2.72198690254299e-08
888 2.72218709795879e-08
889 2.71577942356771e-08
890 2.7202514019109e-08
891 2.70129287827103e-08
892 2.70571440807998e-08
893 2.706153345855e-08
894 2.70283226910806e-08
895 2.69803468455621e-08
896 2.70067577190503e-08
897 2.6961886945287e-08
898 2.70681912439841e-08
899 2.69476121417256e-08
900 2.69165738586707e-08
901 2.71639883919761e-08
902 2.69732876034823e-08
903 2.69272835140555e-08
904 2.69243525252705e-08
905 2.69403148678293e-08
906 2.70088200693408e-08
907 2.70761990606161e-08
908 2.69275162168015e-08
909 2.68910014256107e-08
910 2.69664273133685e-08
911 2.68406097347906e-08
912 2.70044218098064e-08
913 2.69246154260827e-08
914 2.68947513148987e-08
915 2.68723798768633e-08
916 2.68457522878407e-08
917 2.6871132874362e-08
918 2.732383919124e-08
919 2.67329642866798e-08
920 2.67254680608175e-08
921 2.67382862517707e-08
922 2.67671538267678e-08
923 2.67822191091227e-08
924 2.67824553645823e-08
925 2.67963287114981e-08
926 2.67139128595772e-08
927 2.67109214746597e-08
928 2.67347441962329e-08
929 2.66624304856578e-08
930 2.67119375507718e-08
931 2.6780877959709e-08
932 2.66322679465247e-08
933 2.66677560034623e-08
934 2.66692623540621e-08
935 2.71664948314765e-08
936 2.65989026360103e-08
937 2.66306052765231e-08
938 2.66369362122987e-08
939 2.65494222162488e-08
940 2.66369966084312e-08
941 2.66061856990518e-08
942 2.66061803699813e-08
943 2.65976964897163e-08
944 2.65420059264443e-08
945 2.64983359699045e-08
946 2.65144688427199e-08
947 2.66690918238055e-08
948 2.65250506004122e-08
949 2.64944048922189e-08
950 2.64739199451469e-08
951 2.64730299903704e-08
952 2.64600785726543e-08
953 2.6407830588937e-08
954 2.64646491388021e-08
955 2.6396062224876e-08
956 2.64880899436548e-08
957 2.6429587407506e-08
958 2.64212971501365e-08
959 2.63782027332127e-08
960 2.63240327313952e-08
961 2.63361208396873e-08
962 2.63257255994631e-08
963 2.63125841115652e-08
964 2.63383377330229e-08
965 2.67604214343464e-08
966 2.62291006691839e-08
967 2.62370356551855e-08
968 2.63572044190141e-08
969 2.62974815257166e-08
970 2.61877044493986e-08
971 2.61987089800186e-08
972 2.61868891016093e-08
973 2.63086388230249e-08
974 2.62410164708626e-08
975 2.6220030591162e-08
976 2.62943871121024e-08
977 2.62319783672638e-08
978 2.61579291560565e-08
979 2.61533958934024e-08
980 2.61664911960224e-08
981 2.61790553679475e-08
982 2.61300314718937e-08
983 2.61985952931809e-08
984 2.61602615125867e-08
985 2.61227235398565e-08
986 2.60897508042035e-08
987 2.60811887642376e-08
988 2.60764814186132e-08
989 2.60339145796706e-08
990 2.61281947189218e-08
991 2.60498893567274e-08
992 2.60412917896247e-08
993 2.60154209286156e-08
994 2.6019034038427e-08
995 2.59736303576119e-08
996 2.60065178281366e-08
997 2.60180996747295e-08
998 2.59636898647386e-08
999 2.60048995670559e-08
1000 2.59803627500332e-08
1001 2.59509675970548e-08
1002 2.5921190527356e-08
1003 2.59096086807631e-08
1004 2.5971660377877e-08
1005 2.58914951700717e-08
1006 2.59382506584416e-08
1007 2.59445034345163e-08
1008 2.5968176942115e-08
1009 2.58949164333444e-08
1010 2.58044021705928e-08
1011 2.58027910149394e-08
1012 2.5952138216212e-08
1013 2.58464396551972e-08
1014 2.57910226508784e-08
1015 2.58373553663205e-08
1016 2.57458054875315e-08
1017 2.57392258617983e-08
1018 2.57404089154534e-08
1019 2.57558525618151e-08
1020 2.56829792988356e-08
1021 2.57295429406668e-08
1022 2.57245051926702e-08
1023 2.58561723143202e-08
1024 2.57358738764424e-08
1025 2.57437662298798e-08
1026 2.56327190584216e-08
1027 2.58238266326316e-08
1028 2.5694090410866e-08
1029 2.56605385828834e-08
1030 2.57501664435722e-08
1031 2.56470418236177e-08
1032 2.56318681834955e-08
1033 2.55953214178817e-08
1034 2.56208299020955e-08
1035 2.5536817105376e-08
1036 2.56608618798282e-08
1037 2.55779895041996e-08
1038 2.55578065377904e-08
1039 2.54963197221514e-08
1040 2.56732661796377e-08
1041 2.55555665518159e-08
1042 2.54980196956467e-08
1043 2.55389807080064e-08
1044 2.54796432841431e-08
1045 2.54174530311957e-08
1046 2.54758969475688e-08
1047 2.53950638295919e-08
1048 2.53312339992817e-08
1049 2.5410857418251e-08
1050 2.53561776020206e-08
1051 2.53474254918729e-08
1052 2.54477541261622e-08
1053 2.52982097492804e-08
1054 2.54225103191175e-08
1055 2.52913743281624e-08
1056 2.54479957106923e-08
1057 2.52922767174368e-08
1058 2.53197374178171e-08
1059 2.52178367077249e-08
1060 2.52274396927987e-08
1061 2.51839384901587e-08
1062 2.53160727936574e-08
1063 2.51703902165445e-08
1064 2.5322835384145e-08
1065 2.51535929862712e-08
1066 2.51324951960896e-08
1067 2.5175351581197e-08
1068 2.51109071314204e-08
1069 2.50770550991319e-08
1070 2.52159875202551e-08
1071 2.50630272091712e-08
1072 2.52397391875547e-08
1073 2.5037754980417e-08
1074 2.50893901210247e-08
1075 2.502510199065e-08
1076 2.50525946654534e-08
1077 2.49864804402478e-08
1078 2.5014760041131e-08
1079 2.49551828090944e-08
1080 2.5125444835794e-08
1081 2.49422740239424e-08
1082 2.49837004417941e-08
1083 2.50544598401348e-08
1084 2.49407730024132e-08
1085 2.50492160347449e-08
1086 2.48997800156303e-08
1087 2.49284237696656e-08
1088 2.48863738505634e-08
1089 2.49177034561399e-08
1090 2.4847023993857e-08
1091 2.49713103528393e-08
1092 2.48286315951418e-08
1093 2.48915945633144e-08
1094 2.48121310164606e-08
1095 2.49471980851013e-08
1096 2.4787837560325e-08
1097 2.48062850261022e-08
1098 2.47711700041009e-08
1099 2.47944829112612e-08
1100 2.47591387392276e-08
1101 2.47432900835065e-08
1102 2.47196396685467e-08
1103 2.47238087780488e-08
1104 2.48566092153624e-08
1105 2.46625369015874e-08
1106 2.47408333819976e-08
1107 2.46458338182265e-08
1108 2.48137101976909e-08
1109 2.46125821945498e-08
1110 2.47065656822087e-08
1111 2.46392239944271e-08
1112 2.4747642157763e-08
1113 2.45631550654934e-08
1114 2.46458871089317e-08
1115 2.46077931365107e-08
1116 2.45544917731877e-08
1117 2.45267912646341e-08
1118 2.4550816490887e-08
1119 2.47251161766826e-08
1120 2.452496694616e-08
1121 2.46535662995484e-08
1122 2.44522109227319e-08
1123 2.45194300418916e-08
1124 2.45275639798592e-08
1125 2.44945521643558e-08
1126 2.44042848152048e-08
1127 2.45900295681167e-08
1128 2.43880577954769e-08
1129 2.43802986688024e-08
1130 2.44646098934709e-08
1131 2.44853328723593e-08
1132 2.43554083567687e-08
1133 2.44170266228139e-08
1134 2.43233593266723e-08
1135 2.4334346093724e-08
1136 2.43067184158008e-08
1137 2.43289850487827e-08
1138 2.43299282942644e-08
1139 2.42850841658537e-08
1140 2.42287168106259e-08
1141 2.42625581847733e-08
1142 2.42661730709415e-08
1143 2.42393909388738e-08
1144 2.43584850068146e-08
1145 2.42309639020277e-08
1146 2.42510420633835e-08
1147 2.41811175527573e-08
1148 2.4222753580716e-08
1149 2.41387088095735e-08
1150 2.41883366669526e-08
1151 2.40941293583319e-08
1152 2.42233753056098e-08
1153 2.40748843083338e-08
1154 2.41705198078535e-08
1155 2.40582327393213e-08
1156 2.41471056483533e-08
1157 2.40253292815851e-08
1158 2.4150073940632e-08
1159 2.40284592223361e-08
1160 2.40878019752699e-08
1161 2.39671962276589e-08
1162 2.41011477442044e-08
1163 2.39690489678424e-08
1164 2.40355433334116e-08
1165 2.39197941453995e-08
1166 2.40795436923236e-08
1167 2.39085053976851e-08
1168 2.40657769268182e-08
1169 2.3899348278178e-08
1170 2.4105819562692e-08
1171 2.39317099470782e-08
1172 2.40236150972351e-08
1173 2.3869207055327e-08
1174 2.39265105506092e-08
1175 2.38275355002315e-08
1176 2.39265816048828e-08
1177 2.37894859367316e-08
1178 2.39885498132253e-08
1179 2.37846435879874e-08
1180 2.38705517574545e-08
1181 2.3774781254815e-08
1182 2.39585506989215e-08
1183 2.38173285538323e-08
1184 2.39821584813171e-08
1185 2.37838815309033e-08
1186 2.38218547110591e-08
1187 2.37667201474778e-08
1188 2.37750654719093e-08
1189 2.37719266493741e-08
1190 2.37156871918387e-08
1191 2.38507151806289e-08
1192 2.37016841708737e-08
1193 2.37998065699685e-08
1194 2.36473631787248e-08
1195 2.37884769660468e-08
1196 2.36095960559624e-08
1197 2.37279014214664e-08
1198 2.36341897164039e-08
1199 2.38076118819208e-08
1200 2.35656276714735e-08
1201 2.3614322941512e-08
1202 2.35203092557867e-08
1203 2.37583446249801e-08
1204 2.35388721847585e-08
1205 2.36314434687301e-08
1206 2.34666988063736e-08
1207 2.36735360203966e-08
1208 2.35474946208569e-08
1209 2.37246950973713e-08
1210 2.35083881250375e-08
1211 2.36364527950172e-08
1212 2.34584707214935e-08
1213 2.36498767236526e-08
1214 2.34153887390676e-08
1215 2.35682069416043e-08
1216 2.33366144186675e-08
1217 2.35494308498119e-08
1218 2.33563710594353e-08
1219 2.34424799572253e-08
1220 2.33448318454066e-08
1221 2.35543673454686e-08
1222 2.33817800676661e-08
1223 2.35064501197257e-08
1224 2.32903705210674e-08
1225 2.33708128405397e-08
1226 2.32865744465016e-08
1227 2.34585417757671e-08
1228 2.33350352374373e-08
1229 2.34905357388016e-08
1230 2.33132322335905e-08
1231 2.34806147858535e-08
1232 2.32891732565577e-08
1233 2.34568258150603e-08
1234 2.32637002994807e-08
1235 2.34420678424385e-08
1236 2.32400623190188e-08
1237 2.33785133474385e-08
1238 2.32211974093843e-08
1239 2.33890009582183e-08
1240 2.31940511241646e-08
1241 2.3351972799901e-08
1242 2.31710366449533e-08
1243 2.33463666177158e-08
1244 2.3144737681946e-08
1245 2.32854411308381e-08
1246 2.3124917092332e-08
1247 2.32884804773903e-08
1248 2.30971153314385e-08
1249 2.32468835292821e-08
1250 2.30750760721321e-08
1251 2.32372467934283e-08
1252 2.30479617613355e-08
1253 2.32040040515358e-08
1254 2.30276793189432e-08
1255 2.31903314329429e-08
1256 2.30006005352834e-08
1257 2.31500703051779e-08
1258 2.29796146555827e-08
1259 2.31386749760532e-08
1260 2.29562484577173e-08
1261 2.3102618484927e-08
1262 2.29285515018773e-08
1263 2.30898109521149e-08
1264 2.29051781985845e-08
1265 2.30650858412673e-08
1266 2.2879223848804e-08
1267 2.30437624537672e-08
1268 2.28569518867516e-08
1269 2.30181047555789e-08
1270 2.28325180984257e-08
1271 2.30000640755179e-08
1272 2.28064607199485e-08
1273 2.29699459453059e-08
1274 2.27817853470924e-08
1275 2.29509407034811e-08
1276 2.27584227019406e-08
1277 2.29226699843821e-08
1278 2.27349801207311e-08
1279 2.29046133171096e-08
1280 2.27161951471544e-08
1281 2.28800889345848e-08
1282 2.26912160172787e-08
1283 2.28603163066055e-08
1284 2.26651799550837e-08
1285 2.28379075650764e-08
1286 2.26496634780915e-08
1287 2.27700418520271e-08
1288 2.26283560778029e-08
1289 2.27411103281838e-08
1290 2.26020340221567e-08
1291 2.27312124678747e-08
1292 2.25753211680058e-08
1293 2.27364260751983e-08
1294 2.25552909682847e-08
1295 2.27176641942606e-08
1296 2.25306830969885e-08
1297 2.26972769468148e-08
1298 2.25038920831366e-08
1299 2.26830358940333e-08
1300 2.24752447763876e-08
1301 2.26474714537517e-08
1302 2.24601564013938e-08
1303 2.26152092608345e-08
1304 2.24343548183015e-08
1305 2.26104894807122e-08
1306 2.24143832383561e-08
1307 2.25772911477407e-08
1308 2.23934506493606e-08
1309 2.25422951416476e-08
1310 2.237136875749e-08
1311 2.25268870224227e-08
1312 2.23453415770791e-08
1313 2.25122640529207e-08
1314 2.23189502435162e-08
1315 2.24784351132712e-08
1316 2.22963301155232e-08
1317 2.24511342850064e-08
1318 2.22752447598396e-08
1319 2.24313918550934e-08
1320 2.22519318526793e-08
1321 2.24176570640111e-08
1322 2.22244036507391e-08
1323 2.23839577984108e-08
1324 2.22042384478982e-08
1325 2.23643876751112e-08
1326 2.21815579237727e-08
1327 2.23438529900477e-08
1328 2.21574403269642e-08
1329 2.23149463352001e-08
1330 2.21337828065771e-08
1331 2.22976908048622e-08
1332 2.21106262188187e-08
1333 2.22737437383103e-08
1334 2.20868141553865e-08
1335 2.22467519961356e-08
1336 2.2063835203312e-08
1337 2.22254250559217e-08
1338 2.20409113182995e-08
1339 2.22032632279934e-08
1340 2.2015367306949e-08
1341 2.2175958847015e-08
1342 2.19951719060418e-08
1343 2.2150009826305e-08
1344 2.19713882643191e-08
1345 2.21225580077089e-08
1346 2.19474376450535e-08
1347 2.21009646139692e-08
1348 2.19259952416451e-08
1349 2.20818190399541e-08
1350 2.19312443761055e-08
1351 2.19396945055905e-08
1352 2.19079048235926e-08
1353 2.19768558906708e-08
1354 2.1873722388932e-08
1355 2.19641620446964e-08
1356 2.18419433650752e-08
1357 2.19664109124551e-08
1358 2.18803464235862e-08
1359 2.18197335755121e-08
1360 2.18900311210746e-08
1361 2.17950795189381e-08
1362 2.18667253193416e-08
1363 2.17541611391425e-08
1364 2.17889333242738e-08
1365 2.17457927220721e-08
1366 2.17901323651404e-08
1367 2.17195808005499e-08
1368 2.1741557887367e-08
1369 2.17160245341574e-08
1370 2.16399858032901e-08
1371 2.17554774195605e-08
1372 2.16775504213729e-08
1373 2.1696816787653e-08
1374 2.16304218980667e-08
1375 2.16615561043909e-08
1376 2.16242739270456e-08
1377 2.16332303182298e-08
1378 2.15938307235319e-08
1379 2.16177724610134e-08
1380 2.15643769507778e-08
1381 2.16011208920008e-08
1382 2.15475779441476e-08
1383 2.15789803803546e-08
1384 2.15182325291607e-08
1385 2.15490523203243e-08
1386 2.14949107402163e-08
1387 2.15198703301667e-08
1388 2.14726281200228e-08
1389 2.15011137782994e-08
1390 2.14501749695728e-08
1391 2.14749391602709e-08
1392 2.14338431447914e-08
1393 2.15174242867988e-08
1394 2.14121200770023e-08
1395 2.14805115916761e-08
1396 2.1390555104972e-08
1397 2.14334470172162e-08
1398 2.15121644941973e-08
1399 2.13672972648737e-08
1400 2.13692086248329e-08
1401 2.13270894278139e-08
1402 2.14142552579233e-08
1403 2.13576196728127e-08
1404 2.13076365440656e-08
1405 2.12804476262818e-08
1406 2.1265192273745e-08
1407 2.14090007943923e-08
1408 2.12469259963655e-08
1409 2.1292180463206e-08
1410 2.12025739187993e-08
1411 2.11950474948708e-08
1412 2.12887290018671e-08
1413 2.11731592258957e-08
1414 2.13272457472158e-08
1415 2.1148998996523e-08
1416 2.11564863406011e-08
1417 2.11270947403364e-08
1418 2.11383337500592e-08
1419 2.11042419095975e-08
1420 2.11142552331012e-08
1421 2.10831441194159e-08
1422 2.10931059285713e-08
1423 2.10608188666583e-08
1424 2.10697717051289e-08
1425 2.10398702904513e-08
1426 2.10524113697375e-08
1427 2.1017486417918e-08
1428 2.10261532629374e-08
1429 2.0992931837327e-08
1430 2.10074500017754e-08
1431 2.09707735621123e-08
1432 2.09841690690382e-08
1433 2.09507646786733e-08
1434 2.09612132096026e-08
1435 2.09290771380211e-08
1436 2.09355555114143e-08
1437 2.08981010274556e-08
1438 2.10078798801305e-08
1439 2.09379411586497e-08
1440 2.08750741137465e-08
1441 2.08652739530635e-08
1442 2.08304467008702e-08
1443 2.08329247186612e-08
1444 2.09286525887364e-08
1445 2.08408277302397e-08
1446 2.08563033510245e-08
1447 2.0794423960524e-08
1448 2.07838866117527e-08
1449 2.08004369284254e-08
1450 2.07594688106383e-08
1451 2.07472208302306e-08
1452 2.07212309533134e-08
1453 2.07903578797186e-08
1454 2.07418047182273e-08
1455 2.06788826062621e-08
1456 2.0787112475773e-08
1457 2.06518109280296e-08
1458 2.06656096679581e-08
1459 2.06425507798258e-08
1460 2.06582733142113e-08
1461 2.07907984162148e-08
1462 2.06777812650216e-08
1463 2.06435739613653e-08
1464 2.06620054399309e-08
1465 2.0584888460462e-08
1466 2.06951362713426e-08
1467 2.0549052237584e-08
1468 2.05045758150391e-08
1469 2.06673522740175e-08
1470 2.05202681513583e-08
1471 2.05170618272632e-08
1472 2.04933012781794e-08
1473 2.05071071235352e-08
1474 2.05815027243261e-08
1475 2.05472918679561e-08
1476 2.04833199290988e-08
1477 2.04643662016224e-08
1478 2.04227230682363e-08
1479 2.05412700182706e-08
1480 2.04088852484574e-08
1481 2.04155643501736e-08
1482 2.03727914538376e-08
1483 2.04254213542754e-08
1484 2.03678993671019e-08
1485 2.03706385093483e-08
1486 2.04001224801686e-08
1487 2.03451069324956e-08
1488 2.03424210809544e-08
1489 2.03481711480435e-08
1490 2.03068797333117e-08
1491 2.03205754445435e-08
1492 2.02771079926833e-08
1493 2.02552961070523e-08
1494 2.02800087834021e-08
1495 2.02831849094309e-08
1496 2.023230472048e-08
1497 2.03707593016134e-08
1498 2.02580423547261e-08
1499 2.02337222532378e-08
1500 2.02017513828423e-08
1501 2.0203225759019e-08
1502 2.01635845797909e-08
1503 2.02985823705149e-08
1504 2.0154757862656e-08
1505 2.01568113311623e-08
1506 2.0130999089929e-08
1507 2.01927043974592e-08
1508 2.01241832087362e-08
1509 2.01040908365258e-08
1510 2.00589074239588e-08
1511 2.01235117458509e-08
1512 2.00816376860757e-08
1513 2.00798524474521e-08
1514 2.00532390692842e-08
1515 2.00447214382393e-08
1516 2.0041760251388e-08
1517 1.99964294012034e-08
1518 2.01196215243726e-08
1519 2.00092280522313e-08
1520 2.00137311168191e-08
1521 1.99545588941419e-08
1522 1.9999774281132e-08
1523 1.99523473298768e-08
1524 2.00005914052781e-08
1525 2.00257534999082e-08
1526 1.99503791264988e-08
1527 1.99148537660676e-08
1528 1.99111749310532e-08
1529 1.9892627989293e-08
1530 1.99551042356916e-08
1531 1.98928198358317e-08
1532 1.9841671416998e-08
1533 1.98883860491605e-08
1534 1.99196339423224e-08
1535 1.98201970391665e-08
1536 1.98640996984523e-08
1537 1.98094056713671e-08
1538 1.98060519096543e-08
1539 1.97805505308679e-08
1540 1.98090219782898e-08
1541 1.97941520951872e-08
1542 1.97453413619542e-08
1543 1.97637586296651e-08
1544 1.97085174846734e-08
1545 1.98089935565804e-08
1546 1.97532976642378e-08
1547 1.97200655804863e-08
1548 1.98678105078898e-08
1549 1.97232541410131e-08
1550 1.96871265956133e-08
1551 1.96901073223898e-08
1552 1.96588487710869e-08
1553 1.9700934217326e-08
1554 1.96421279241576e-08
1555 1.96471283686606e-08
1556 1.96043252742584e-08
1557 1.96047853506798e-08
1558 1.96529157392433e-08
1559 1.96045135680833e-08
1560 1.95883220754922e-08
1561 1.95705958105918e-08
1562 1.96134362084877e-08
1563 1.95489509025037e-08
1564 1.96231191296192e-08
1565 1.95837372984897e-08
1566 1.9558482833304e-08
1567 1.95202485286927e-08
1568 1.95128819768797e-08
1569 1.95550509118902e-08
1570 1.95049203455255e-08
1571 1.94953688748001e-08
1572 1.95061566898858e-08
1573 1.95272988889883e-08
1574 1.94576017520376e-08
1575 1.94452791646427e-08
1576 1.94050837620807e-08
1577 1.95181186768423e-08
1578 1.94031457567689e-08
1579 1.94214671012105e-08
1580 1.93609306364806e-08
1581 1.94706863965166e-08
1582 1.93683682425672e-08
1583 1.93735978371024e-08
1584 1.93196765252424e-08
1585 1.94284321963778e-08
1586 1.93154789940309e-08
1587 1.93530542702547e-08
1588 1.93096756362365e-08
1589 1.93084694899426e-08
1590 1.92897822159921e-08
1591 1.93249629631964e-08
1592 1.93026394867957e-08
1593 1.93048759200565e-08
1594 1.92362623607778e-08
1595 1.92643518914792e-08
1596 1.92407352272994e-08
1597 1.92317877178994e-08
1598 1.92710647439753e-08
1599 1.92747169336371e-08
1600 1.92358804440573e-08
1601 1.91835933804896e-08
1602 1.9190888878029e-08
1603 1.91945765948276e-08
1604 1.91565465712529e-08
1605 1.91706828189808e-08
1606 1.91641138513887e-08
1607 1.91937399307562e-08
1608 1.91397955262573e-08
1609 1.91714164543555e-08
1610 1.91099296387165e-08
1611 1.9109304361109e-08
1612 1.91664746296283e-08
1613 1.90903293173506e-08
1614 1.91229290180672e-08
1615 1.90794171572861e-08
1616 1.90712921011027e-08
1617 1.90908302499793e-08
1618 1.91808755545253e-08
1619 1.90616447071079e-08
1620 1.90414652934123e-08
1621 1.90009288303372e-08
1622 1.90151343559819e-08
1623 1.90482278838999e-08
1624 1.89794810978583e-08
1625 1.89937594541334e-08
1626 1.89487661117482e-08
1627 1.9006671791999e-08
1628 1.89537399108985e-08
1629 1.89878139877919e-08
1630 1.90907840647014e-08
1631 1.89605948719418e-08
1632 1.89387705518129e-08
1633 1.89680466888831e-08
1634 1.89185218602006e-08
1635 1.90637763353152e-08
1636 1.88770599152122e-08
1637 1.88820798996403e-08
1638 1.88574365012073e-08
1639 1.88725106653465e-08
1640 1.88228135300506e-08
1641 1.88554345470493e-08
1642 1.88063307149378e-08
1643 1.88463733508115e-08
1644 1.87988522526439e-08
1645 1.88192874617243e-08
1646 1.88185147464992e-08
1647 1.88289952518517e-08
1648 1.87911961546661e-08
1649 1.87887980729329e-08
1650 1.87693256492594e-08
1651 1.88132123213336e-08
1652 1.87495228232137e-08
1653 1.87915922822413e-08
1654 1.87435986731543e-08
1655 1.87647053451201e-08
1656 1.87336350876421e-08
1657 1.87642044124914e-08
1658 1.8680390567738e-08
1659 1.87499225035026e-08
1660 1.86720647832317e-08
1661 1.87270856599753e-08
1662 1.86653945632997e-08
1663 1.86974915550309e-08
1664 1.86395858747801e-08
1665 1.87583069077846e-08
1666 1.86273627633682e-08
1667 1.86382447253663e-08
1668 1.86133046753412e-08
1669 1.86335213925304e-08
1670 1.86216944086937e-08
1671 1.86096205112563e-08
1672 1.85569177801881e-08
1673 1.86058262130473e-08
1674 1.85672774932755e-08
1675 1.86595681128665e-08
1676 1.85354203097177e-08
1677 1.85669648544717e-08
1678 1.86101605237354e-08
1679 1.85402146968272e-08
1680 1.85239965588835e-08
1681 1.85211934677909e-08
1682 1.85201365354715e-08
1683 1.85224635629311e-08
1684 1.84853359286308e-08
1685 1.85189392709617e-08
1686 1.85425719223531e-08
1687 1.84900414978983e-08
1688 1.84506827594078e-08
1689 1.85446875633488e-08
1690 1.84278512449509e-08
1691 1.84601276487228e-08
1692 1.84378983192346e-08
1693 1.84440391848284e-08
1694 1.84212023413011e-08
1695 1.84335480213349e-08
1696 1.84048314366692e-08
1697 1.83917858720406e-08
1698 1.84024511185044e-08
1699 1.83613568793817e-08
1700 1.84386585999619e-08
1701 1.8361507869713e-08
1702 1.83737576264775e-08
1703 1.834238361198e-08
1704 1.8318326411304e-08
1705 1.83496613459511e-08
1706 1.83149673205207e-08
1707 1.83322779179207e-08
1708 1.83183761492955e-08
1709 1.82696417994066e-08
1710 1.83237567341621e-08
1711 1.82798256531669e-08
1712 1.82564452444467e-08
1713 1.82666539672027e-08
1714 1.8284746161612e-08
1715 1.82451103114545e-08
1716 1.81667889620485e-08
1717 1.81964345813412e-08
1718 1.82071318022281e-08
1719 1.81891124384492e-08
1720 1.81297163948102e-08
1721 1.82177188889909e-08
1722 1.81450694469731e-08
1723 1.81789570063984e-08
1724 1.8114885591558e-08
1725 1.81844672653142e-08
1726 1.81371024865484e-08
1727 1.81221153638944e-08
1728 1.81720611891478e-08
1729 1.81189978576413e-08
1730 1.81212271854747e-08
1731 1.81184489633779e-08
1732 1.8084270081431e-08
1733 1.80730470589197e-08
1734 1.80708923380735e-08
1735 1.79927042154304e-08
1736 1.8107442656401e-08
1737 1.80312795805548e-08
1738 1.80496151358511e-08
1739 1.79871317840252e-08
1740 1.80036554553453e-08
1741 1.79578591996687e-08
1742 1.7982124234095e-08
1743 1.79142016776268e-08
1744 1.79690626822548e-08
1745 1.78942034523288e-08
1746 1.79322778848245e-08
1747 1.79511925324505e-08
1748 1.79607511086033e-08
1749 1.79236945285766e-08
1750 1.79743278039268e-08
1751 1.78913168724648e-08
1752 1.79365340358117e-08
1753 1.78177881338115e-08
1754 1.78945711581946e-08
1755 1.78019128327378e-08
1756 1.78876184975252e-08
1757 1.77675474333228e-08
1758 1.79005024136814e-08
1759 1.77394756661897e-08
1760 1.7848963196343e-08
1761 1.77628187714163e-08
1762 1.78644050663479e-08
1763 1.7727117551658e-08
1764 1.78177259613221e-08
1765 1.77896453124049e-08
1766 1.7781072614298e-08
1767 1.76699685994208e-08
1768 1.78053802812883e-08
1769 1.76579053601245e-08
1770 1.78056485111711e-08
1771 1.76403247564849e-08
1772 1.7740832802815e-08
1773 1.76470607016199e-08
1774 1.77579160265395e-08
1775 1.7608225988397e-08
1776 1.772199631489e-08
1777 1.76166015108947e-08
1778 1.77318941751992e-08
1779 1.75749921282886e-08
1780 1.77109740207015e-08
1781 1.75760153098281e-08
1782 1.76668866203045e-08
1783 1.75352479203639e-08
1784 1.76848420352371e-08
1785 1.75267729218831e-08
1786 1.76229377757409e-08
1787 1.75114944767074e-08
1788 1.76285137598597e-08
1789 1.74751662029848e-08
1790 1.76141412566722e-08
1791 1.74855205870017e-08
1792 1.75751235786947e-08
1793 1.74587579948593e-08
1794 1.75788610334848e-08
1795 1.74329244373439e-08
1796 1.75641581279251e-08
1797 1.7435787924569e-08
1798 1.75102972121977e-08
1799 1.74128835794818e-08
1800 1.75087908615978e-08
1801 1.7396544649273e-08
1802 1.75141270375434e-08
1803 1.73741980802333e-08
1804 1.7489112380531e-08
1805 1.73688636806446e-08
1806 1.75047425443609e-08
1807 1.73541430115165e-08
1808 1.74590795154472e-08
1809 1.73415184434589e-08
1810 1.74685990117496e-08
1811 1.73317271645601e-08
1812 1.7427437271067e-08
1813 1.7306067690015e-08
1814 1.741855548687e-08
1815 1.72904250916872e-08
1816 1.74080287962397e-08
1817 1.72822254285165e-08
1818 1.73867125141669e-08
1819 1.72638330298014e-08
1820 1.73716419027414e-08
1821 1.72486309679698e-08
1822 1.73552194837612e-08
1823 1.72281087174042e-08
1824 1.7333597668312e-08
1825 1.7210812330859e-08
1826 1.73195005004345e-08
1827 1.71918035363205e-08
1828 1.72916170271264e-08
1829 1.71764344969461e-08
1830 1.72825718181002e-08
1831 1.71565979201205e-08
1832 1.72510592477693e-08
1833 1.71525513792403e-08
1834 1.72483094473819e-08
1835 1.71399747728174e-08
1836 1.7220058268208e-08
1837 1.71318106367835e-08
1838 1.71946652471888e-08
1839 1.7104827776393e-08
1840 1.72024048339381e-08
1841 1.71032876750132e-08
1842 1.71780705215951e-08
1843 1.70647833641624e-08
1844 1.71416889571674e-08
1845 1.70757505912889e-08
1846 1.71474496823976e-08
1847 1.7014317066355e-08
1848 1.70776530694638e-08
1849 1.70775020791325e-08
1850 1.70791203402132e-08
1851 1.70035470148377e-08
1852 1.70476752714421e-08
1853 1.69858882514973e-08
1854 1.70834244528351e-08
1855 1.69557363705053e-08
1856 1.70234599750074e-08
1857 1.69426357388147e-08
1858 1.70375606955986e-08
1859 1.69166582963953e-08
1860 1.69850267184302e-08
1861 1.69083094192501e-08
1862 1.70104144103789e-08
1863 1.68824101365317e-08
1864 1.69491176649217e-08
1865 1.68805627254187e-08
1866 1.69916347658727e-08
1867 1.68762799290789e-08
1868 1.69162355234675e-08
1869 1.68611737905167e-08
1870 1.69160152552195e-08
1871 1.6845076444838e-08
1872 1.68988734117193e-08
1873 1.6841061878381e-08
1874 1.68833391711587e-08
1875 1.68355143159715e-08
1876 1.68812182010925e-08
1877 1.6808852976169e-08
1878 1.68345675177761e-08
1879 1.68072222805904e-08
1880 1.68564042724029e-08
1881 1.67822644669968e-08
1882 1.67994134159244e-08
1883 1.67655667127065e-08
1884 1.67979976595234e-08
1885 1.67484905944093e-08
1886 1.687248563087e-08
1887 1.67339901935293e-08
1888 1.67408149565063e-08
1889 1.67707465692502e-08
1890 1.67203157985796e-08
1891 1.67583955601458e-08
1892 1.67036517950692e-08
1893 1.67481424284688e-08
1894 1.66847211602317e-08
1895 1.68191753857627e-08
1896 1.66840496973464e-08
1897 1.66677960322659e-08
1898 1.67823337449136e-08
1899 1.66505884635626e-08
1900 1.67048010979443e-08
1901 1.6653570966696e-08
1902 1.66730611539379e-08
1903 1.66364042542e-08
1904 1.67035825171524e-08
1905 1.66118034883311e-08
1906 1.6719116757713e-08
1907 1.65990883260747e-08
1908 1.66227938080965e-08
1909 1.66062985584858e-08
1910 1.66200795348459e-08
1911 1.65793387907343e-08
1912 1.6685392623117e-08
1913 1.65558837750268e-08
1914 1.65836482324266e-08
1915 1.65599551849027e-08
1916 1.65923008665914e-08
1917 1.6539445368835e-08
1918 1.66676699109303e-08
1919 1.65139741881148e-08
1920 1.65658669004642e-08
1921 1.65146207820044e-08
1922 1.66150861957703e-08
1923 1.64968252391873e-08
1924 1.65315494626839e-08
1925 1.648502667706e-08
1926 1.65702989107785e-08
1927 1.6475603104027e-08
1928 1.65576174993021e-08
1929 1.64689204495971e-08
1930 1.65437850085937e-08
1931 1.645907765635e-08
1932 1.65423230669148e-08
1933 1.64399747148991e-08
1934 1.65180722433433e-08
1935 1.64292615068007e-08
1936 1.64862559159928e-08
1937 1.64269913227599e-08
1938 1.6508762357148e-08
1939 1.64073341579751e-08
1940 1.64806532865214e-08
1941 1.6400765190383e-08
1942 1.64636322352862e-08
1943 1.63898974392396e-08
1944 1.64472240271607e-08
1945 1.63789390938973e-08
1946 1.64392179868855e-08
1947 1.63677729148048e-08
1948 1.6426049853635e-08
1949 1.63577471568033e-08
1950 1.6418772119664e-08
1951 1.6345849118693e-08
1952 1.6392935009435e-08
1953 1.63331907998554e-08
1954 1.63896523019957e-08
1955 1.63232947159031e-08
1956 1.63737077230053e-08
1957 1.63146296472405e-08
1958 1.63790723206603e-08
1959 1.63000599684437e-08
1960 1.63500892824686e-08
1961 1.62923541324744e-08
1962 1.63397313457381e-08
1963 1.62817901383505e-08
1964 1.63407438691365e-08
1965 1.62702669115333e-08
1966 1.63196691715939e-08
1967 1.62601576647603e-08
1968 1.63124926899627e-08
1969 1.62488493771207e-08
1970 1.63082489734734e-08
1971 1.62379301116289e-08
1972 1.62900786193632e-08
1973 1.62294178096545e-08
1974 1.62835167571984e-08
1975 1.62150417537532e-08
1976 1.62736384368145e-08
1977 1.62043107820864e-08
1978 1.62616800025717e-08
1979 1.61965054701341e-08
1980 1.6254265489124e-08
1981 1.61858171310314e-08
1982 1.62446340823408e-08
1983 1.61758180183824e-08
1984 1.62355409116799e-08
1985 1.61640780760308e-08
1986 1.62237725476189e-08
1987 1.61562727640785e-08
1988 1.62148587889988e-08
1989 1.6146234571579e-08
1990 1.62021844829496e-08
1991 1.61355799832563e-08
1992 1.61922582009311e-08
1993 1.61214295246737e-08
1994 1.61821027688802e-08
1995 1.61153650424239e-08
1996 1.61677782273273e-08
1997 1.61049769076271e-08
1998 1.61601327874905e-08
1999 1.60929278791855e-08
2000 1.6150947246274e-08
2001 1.608790078933e-08
2002 1.61403761467227e-08
2003 1.60787294589682e-08
2004 1.61276965116031e-08
2005 1.6073643749337e-08
2006 1.61189053216049e-08
2007 1.60646926872232e-08
2008 1.61070410342745e-08
2009 1.605471844357e-08
2010 1.60917750235967e-08
2011 1.6044387152192e-08
2012 1.60818967032128e-08
2013 1.60338515797775e-08
2014 1.60744644261968e-08
2015 1.60274069571642e-08
2016 1.60622981582037e-08
2017 1.60189497222518e-08
2018 1.6051654228022e-08
2019 1.60135638083148e-08
2020 1.60392019665778e-08
2021 1.60028825746394e-08
2022 1.6032563721069e-08
2023 1.59945496847058e-08
2024 1.6023891546979e-08
2025 1.5984438661576e-08
2026 1.60134057125561e-08
2027 1.59789781406516e-08
2028 1.6003816938337e-08
2029 1.59682720379806e-08
2030 1.59955959588842e-08
2031 1.59594613080571e-08
2032 1.59786956999142e-08
2033 1.59416231326759e-08
2034 1.59708903879618e-08
2035 1.59302651070448e-08
2036 1.59628719131888e-08
2037 1.59232520502428e-08
2038 1.59538871002951e-08
2039 1.59125885801359e-08
2040 1.59486308604073e-08
2041 1.59048649805982e-08
2042 1.59379904829393e-08
2043 1.58908104452848e-08
2044 1.59298707558264e-08
2045 1.58802038185968e-08
2046 1.59112421016516e-08
2047 1.58698174601568e-08
2048 1.58989070797588e-08
2049 1.58595376831272e-08
2050 1.58851705123197e-08
2051 1.58478989931155e-08
2052 1.58731126020939e-08
2053 1.58379869219516e-08
2054 1.58566404451221e-08
2055 1.58305528685787e-08
2056 1.58474406930509e-08
2057 1.58203725675321e-08
2058 1.58319970466891e-08
2059 1.58110662340505e-08
2060 1.58250585968744e-08
2061 1.58035788899724e-08
2062 1.58148747431142e-08
2063 1.57892454666353e-08
2064 1.58062452015884e-08
2065 1.57805519762633e-08
2066 1.57982533721679e-08
2067 1.57691495417112e-08
2068 1.57854636029242e-08
2069 1.57583279758455e-08
2070 1.57724748817145e-08
2071 1.5740592829161e-08
2072 1.57690447366576e-08
2073 1.57299044900583e-08
2074 1.57435646741533e-08
2075 1.57177595383473e-08
2076 1.57410582346529e-08
2077 1.57190509497696e-08
2078 1.57364503650115e-08
2079 1.5702100952808e-08
2080 1.57408326373343e-08
2081 1.56956279084852e-08
2082 1.57041792903101e-08
2083 1.56766652992246e-08
2084 1.56907837833842e-08
2085 1.56730486366996e-08
2086 1.57066253336779e-08
2087 1.56515245208766e-08
2088 1.56959281127911e-08
2089 1.56541251072895e-08
2090 1.56667052664261e-08
2091 1.56412660601291e-08
2092 1.56421524621919e-08
2093 1.56303681109193e-08
2094 1.56263144646118e-08
2095 1.56207242696382e-08
2096 1.56331623202277e-08
2097 1.56140647078473e-08
2098 1.56367256920475e-08
2099 1.56095385506205e-08
2100 1.56169175369314e-08
2101 1.56050159461074e-08
2102 1.56047050836605e-08
2103 1.55954058556063e-08
2104 1.55964077208637e-08
2105 1.5593915492218e-08
2106 1.55973705062706e-08
2107 1.55985642180667e-08
2108 1.56199480016994e-08
2109 1.55519703781692e-08
2110 1.56252877303587e-08
2111 1.5526810059896e-08
2112 1.56322332856007e-08
2113 1.55148356384416e-08
2114 1.56160382402959e-08
2115 1.55350772246265e-08
2116 1.56088351133121e-08
2117 1.55282489089359e-08
2118 1.55887018848944e-08
2119 1.55031116833015e-08
2120 1.55915316213395e-08
2121 1.55131836265809e-08
2122 1.55732795548147e-08
2123 1.54860089196518e-08
2124 1.55684141134316e-08
2125 1.55042449989651e-08
2126 1.55293733428152e-08
2127 1.547268979607e-08
2128 1.55392374523444e-08
2129 1.54797419327224e-08
2130 1.55315920125076e-08
2131 1.54602979307583e-08
2132 1.55252628530889e-08
2133 1.5471075087703e-08
2134 1.55201522744619e-08
2135 1.54450781053583e-08
2136 1.55002872759269e-08
2137 1.54492560966446e-08
2138 1.54836552468396e-08
2139 1.54264903073908e-08
2140 1.54780064320903e-08
2141 1.54294319543169e-08
2142 1.54698938104048e-08
2143 1.54112509420656e-08
2144 1.54518886574806e-08
2145 1.54128549922916e-08
2146 1.54642183503029e-08
2147 1.54040673550071e-08
2148 1.54353951842268e-08
2149 1.53633248345386e-08
2150 1.53720520756906e-08
2151 1.54265844543033e-08
2152 1.53767416577466e-08
2153 1.53332635477454e-08
2154 1.53382782031031e-08
2155 1.53747432563023e-08
2156 1.53306149996979e-08
2157 1.53245558465187e-08
2158 1.53209001041432e-08
2159 1.53515653522618e-08
2160 1.5336343750505e-08
2161 1.53177683870354e-08
2162 1.530674431649e-08
2163 1.53056873841706e-08
2164 1.52986636692276e-08
2165 1.52945958120654e-08
2166 1.52959689359022e-08
2167 1.53302526229027e-08
2168 1.52815307075116e-08
2169 1.53259200885714e-08
2170 1.52514711970753e-08
2171 1.53033301586447e-08
2172 1.52734216385397e-08
2173 1.52643355733062e-08
2174 1.52713131029714e-08
2175 1.52989212409693e-08
2176 1.5240889439383e-08
2177 1.52664938468661e-08
2178 1.52354235893881e-08
2179 1.52809409570409e-08
2180 1.52367487515903e-08
2181 1.52551589138739e-08
2182 1.52096824024284e-08
2183 1.52493591087932e-08
2184 1.52386778751179e-08
2185 1.52294834521172e-08
2186 1.52535442055068e-08
2187 1.52144448151148e-08
2188 1.52605856840182e-08
2189 1.52637156247692e-08
2190 1.51685135563184e-08
2191 1.52376724571468e-08
2192 1.51682879589998e-08
2193 1.5257789698353e-08
2194 1.51662913339123e-08
2195 1.52596868474575e-08
2196 1.51358161559756e-08
2197 1.52549404219826e-08
2198 1.52032413325287e-08
2199 1.52081476301191e-08
2200 1.5153696963921e-08
2201 1.51562318251308e-08
2202 1.51776244905477e-08
2203 1.51342227638906e-08
2204 1.52243266882124e-08
2205 1.50857921710212e-08
2206 1.52492010130345e-08
2207 1.51292223193877e-08
2208 1.51786139213073e-08
2209 1.51745762622113e-08
2210 1.50723042935397e-08
2211 1.51681209814569e-08
2212 1.50223335992905e-08
2213 1.51578056772905e-08
2214 1.51661723180041e-08
2215 1.51596744046856e-08
2216 1.51509720325294e-08
2217 1.50603458592968e-08
2218 1.52411203657721e-08
2219 1.50913379570738e-08
2220 1.51921852875603e-08
2221 1.51463517283901e-08
2222 1.51447245855252e-08
2223 1.51317127716766e-08
2224 1.51092720557244e-08
2225 1.51571963868946e-08
2226 1.50144270349983e-08
2227 1.51091494871025e-08
2228 1.50569068324558e-08
2229 1.51010102200644e-08
2230 1.51514427670918e-08
2231 1.50325796255402e-08
2232 1.50748622473884e-08
2233 1.50891512618045e-08
2234 1.50746775062771e-08
2235 1.50727377246085e-08
2236 1.50977328416957e-08
2237 1.50642538443435e-08
2238 1.51142511839453e-08
2239 1.50994026171247e-08
2240 1.50532084575161e-08
2241 1.5142768816645e-08
2242 1.50570844681397e-08
2243 1.51154893046623e-08
2244 1.50207934979107e-08
2245 1.50650620867054e-08
2246 1.50616283889349e-08
2247 1.50643497676128e-08
2248 1.51154022631772e-08
2249 1.50497676543182e-08
2250 1.51080890020694e-08
2251 1.50257957187705e-08
2252 1.50881227511945e-08
2253 1.49435148699695e-08
2254 1.50882701888122e-08
2255 1.49748071720524e-08
2256 1.52033567957233e-08
2257 1.4938443371193e-08
2258 1.51195180819741e-08
2259 1.50018557576459e-08
2260 1.4959127270231e-08
2261 1.50793884046152e-08
2262 1.49529597592846e-08
2263 1.49505918756176e-08
2264 1.49777630298331e-08
2265 1.500604618343e-08
2266 1.50265400122862e-08
2267 1.49789443071313e-08
2268 1.49935175386418e-08
2269 1.48719028203459e-08
2270 1.50394132703013e-08
2271 1.4954194327288e-08
2272 1.4921807789392e-08
2273 1.49641206093065e-08
2274 1.49870320598211e-08
2275 1.47836267672119e-08
2276 1.51268544357208e-08
2277 1.48466767768696e-08
2278 1.48791210463628e-08
2279 1.48736702954011e-08
2280 1.498115942411e-08
2281 1.48500198804413e-08
2282 1.48548515710445e-08
2283 1.48781023057154e-08
2284 1.49476182542685e-08
2285 1.47540051287365e-08
2286 1.4871917919379e-08
2287 1.48260097532216e-08
2288 1.49317305186969e-08
2289 1.4878581922062e-08
2290 1.50197383419481e-08
2291 1.47954972717912e-08
2292 1.47692880148043e-08
2293 1.48350443041068e-08
2294 1.49146757166818e-08
2295 1.4863672959109e-08
2296 1.48985623837916e-08
2297 1.46958614166692e-08
2298 1.49419623340918e-08
2299 1.47265195593604e-08
2300 1.49206567101601e-08
2301 1.47982959219917e-08
2302 1.48913610331647e-08
2303 1.47358552027299e-08
2304 1.48983554382198e-08
2305 1.48633905183715e-08
2306 1.46449803395399e-08
2307 1.48594674342917e-08
2308 1.47748577816742e-08
2309 1.4732198572176e-08
2310 1.47846135334362e-08
2311 1.47114578297192e-08
2312 1.4925269908872e-08
2313 1.48455931991975e-08
2314 1.47302587905074e-08
2315 1.48309977632266e-08
2316 1.48229304386405e-08
2317 1.47006229411772e-08
2318 1.49446268693509e-08
2319 1.46989211913251e-08
2320 1.4762604472196e-08
2321 1.46896717012623e-08
2322 1.47955683260648e-08
2323 1.47688163920634e-08
2324 1.47281067341964e-08
2325 1.46894807429021e-08
2326 1.46951215640456e-08
2327 1.47594088062419e-08
2328 1.45521132921544e-08
2329 1.48206407146745e-08
2330 1.46523833066681e-08
2331 1.47968055586034e-08
2332 1.45490561820338e-08
2333 1.48314169834407e-08
2334 1.45411274132812e-08
2335 1.47599248379038e-08
2336 1.47450203158428e-08
2337 1.45048817401516e-08
2338 1.45579983623634e-08
2339 1.46892134011978e-08
2340 1.46473428941363e-08
2341 1.46594070216111e-08
2342 1.46309560022928e-08
2343 1.45196823453375e-08
2344 1.45199390289008e-08
2345 1.4662477454408e-08
2346 1.45733398682069e-08
2347 1.45524428063482e-08
2348 1.45153196129399e-08
2349 1.46317162830201e-08
2350 1.45914631488608e-08
2351 1.45771528181626e-08
2352 1.46639091980205e-08
2353 1.44520821976357e-08
2354 1.45076768376384e-08
2355 1.45518335159522e-08
2356 1.45529028827696e-08
2357 1.46345477958221e-08
2358 1.44825405001825e-08
2359 1.4495630473732e-08
2360 1.44875720309301e-08
2361 1.45256544570316e-08
2362 1.45016434416334e-08
2363 1.46202436823728e-08
2364 1.44939171775604e-08
2365 1.4508107604172e-08
2366 1.46239900189471e-08
2367 1.44502161347759e-08
2368 1.45143523866409e-08
2369 1.45136720419714e-08
2370 1.45504728266133e-08
2371 1.44422411807454e-08
2372 1.44879424013311e-08
2373 1.44611833619024e-08
2374 1.46632004316416e-08
2375 1.43573535282826e-08
2376 1.43872043167903e-08
2377 1.46288146041229e-08
2378 1.45974707876917e-08
2379 1.43715928047072e-08
2380 1.46374805609639e-08
2381 1.43951304210077e-08
2382 1.46647396448429e-08
2383 1.4437667061884e-08
2384 1.45628282766097e-08
2385 1.42848968209819e-08
2386 1.45345024904486e-08
2387 1.42702152317042e-08
2388 1.44028593496159e-08
2389 1.4414871074564e-08
2390 1.43251890349916e-08
2391 1.43950398268089e-08
2392 1.44504825883018e-08
2393 1.4488922062128e-08
2394 1.43531320162538e-08
2395 1.43901734972474e-08
2396 1.44101282018028e-08
2397 1.43474867542182e-08
2398 1.43757947768108e-08
2399 1.44475231778074e-08
2400 1.43297738119941e-08
2401 1.46718006632796e-08
2402 1.44006726543466e-08
2403 1.44534810786467e-08
2404 1.42974023731313e-08
2405 1.43444758293754e-08
2406 1.43928025053697e-08
2407 1.44599558993264e-08
2408 1.42154590321297e-08
2409 1.43818805753426e-08
2410 1.42166971528468e-08
2411 1.44150629211026e-08
2412 1.4189148522803e-08
2413 1.43572229660549e-08
2414 1.42842848660507e-08
2415 1.42311353812374e-08
2416 1.42795828494968e-08
2417 1.43775924499323e-08
2418 1.43081448911175e-08
2419 1.43336436053687e-08
2420 1.43581706524287e-08
2421 1.42243559153599e-08
2422 1.43390801454757e-08
2423 1.42748284304162e-08
2424 1.44456615558397e-08
2425 1.42968215044448e-08
2426 1.42563223448633e-08
2427 1.42218841148178e-08
2428 1.42452387663639e-08
2429 1.42408307368669e-08
2430 1.42782461409752e-08
2431 1.42056375551647e-08
2432 1.42285916382434e-08
2433 1.41676190779094e-08
2434 1.42616984888377e-08
2435 1.41539331366403e-08
2436 1.4218934474286e-08
2437 1.41702738432059e-08
2438 1.42304283912154e-08
2439 1.42275062842145e-08
2440 1.41707570122662e-08
2441 1.41970115663526e-08
2442 1.42313067996724e-08
2443 1.41479024051705e-08
2444 1.41336373715717e-08
2445 1.40852431940175e-08
2446 1.41480249737924e-08
2447 1.42154936710881e-08
2448 1.40912108648195e-08
2449 1.41193341463008e-08
2450 1.41417846322156e-08
2451 1.40366260836799e-08
2452 1.41202622927494e-08
2453 1.41234552941683e-08
2454 1.41382763274578e-08
2455 1.40802223214109e-08
2456 1.4100971945652e-08
2457 1.4122369051961e-08
2458 1.40822828953446e-08
2459 1.41155460653408e-08
2460 1.40952911564796e-08
2461 1.40363418665856e-08
2462 1.41340583681426e-08
2463 1.40973126505628e-08
2464 1.40476963395031e-08
2465 1.3988439739876e-08
2466 1.40626097433483e-08
2467 1.40291689376681e-08
2468 1.40250984159707e-08
2469 1.4093406441873e-08
2470 1.39903395535157e-08
2471 1.40664431214077e-08
2472 1.39372184904119e-08
2473 1.39678535404641e-08
2474 1.39808262744623e-08
2475 1.39679112720614e-08
2476 1.39370586182963e-08
2477 1.39387434927585e-08
2478 1.40132829784534e-08
2479 1.39675053745236e-08
2480 1.40050895325317e-08
2481 1.39206770555234e-08
2482 1.40260292269545e-08
2483 1.3961208189528e-08
2484 1.39443141478068e-08
2485 1.39292239964561e-08
2486 1.3940175236371e-08
2487 1.39070355231752e-08
2488 1.39377576147126e-08
2489 1.39090143846943e-08
2490 1.39795455211811e-08
2491 1.39161118184461e-08
2492 1.38622517908971e-08
2493 1.38721292231025e-08
2494 1.38600757537688e-08
2495 1.39165701185107e-08
2496 1.38918609948746e-08
2497 1.38574991481732e-08
2498 1.39333957704935e-08
2499 1.38523077453101e-08
2500 1.38720128717296e-08
2501 1.38146267758543e-08
2502 1.38681732764212e-08
2503 1.38928912818415e-08
2504 1.37573747949205e-08
2505 1.38211859734838e-08
2506 1.37809186284699e-08
2507 1.38388873693884e-08
2508 1.37601849914404e-08
2509 1.37597142568779e-08
2510 1.36616353785257e-08
2511 1.38348568157198e-08
2512 1.373142044514e-08
2513 1.37178002290739e-08
2514 1.37267734956481e-08
2515 1.38180720199443e-08
2516 1.36999336319832e-08
2517 1.37165452329668e-08
2518 1.37300828484399e-08
2519 1.37878570782846e-08
2520 1.37354874141238e-08
2521 1.37040210290706e-08
2522 1.36638362846497e-08
2523 1.37849553993874e-08
2524 1.36433744302167e-08
2525 1.36231941283427e-08
2526 1.37028193236688e-08
2527 1.37062450278336e-08
2528 1.36791467042485e-08
2529 1.36224480584701e-08
2530 1.37325404381272e-08
2531 1.36272175765839e-08
2532 1.36690276875129e-08
2533 1.36029640884772e-08
2534 1.37593385574064e-08
2535 1.36758000479631e-08
2536 1.36162361386027e-08
2537 1.3615398586353e-08
2538 1.36074351786419e-08
2539 1.37027909019594e-08
2540 1.35515270116571e-08
2541 1.36886164625594e-08
2542 1.35825111158283e-08
2543 1.36527420480093e-08
2544 1.36288953456187e-08
2545 1.37110136577689e-08
2546 1.36367805936288e-08
2547 1.35826043745624e-08
2548 1.35060211903237e-08
2549 1.36628512947823e-08
2550 1.35991591321272e-08
2551 1.36618405477407e-08
2552 1.35213804597356e-08
2553 1.36378934811887e-08
2554 1.34590827371994e-08
2555 1.36040805287507e-08
2556 1.34471616064502e-08
2557 1.35279876189998e-08
2558 1.35131097422914e-08
2559 1.35227411490746e-08
2560 1.34463906675819e-08
2561 1.35817259661053e-08
2562 1.35710420678947e-08
2563 1.35220883379361e-08
2564 1.34796298567608e-08
2565 1.34214745983741e-08
2566 1.34988926703272e-08
2567 1.34915492111531e-08
2568 1.35807738388394e-08
2569 1.35082149910204e-08
2570 1.3462228665162e-08
2571 1.34946782637257e-08
2572 1.35036470894079e-08
2573 1.35142776969133e-08
2574 1.3508241636373e-08
2575 1.34082949188041e-08
2576 1.34196422862942e-08
2577 1.34659945416615e-08
2578 1.34834579057497e-08
2579 1.34650850469598e-08
2580 1.33377264788237e-08
2581 1.33770816646006e-08
2582 1.3358373074368e-08
2583 1.34022775100107e-08
2584 1.33645885469491e-08
2585 1.34616575664381e-08
2586 1.32995179313866e-08
2587 1.34138957719188e-08
2588 1.34177673416502e-08
2589 1.34098403492544e-08
2590 1.33485107411957e-08
2591 1.33728867979244e-08
2592 1.32635564753514e-08
2593 1.33375923638823e-08
2594 1.33399726820471e-08
2595 1.3265286646913e-08
2596 1.3367055906599e-08
2597 1.32934063756807e-08
2598 1.33418218695169e-08
2599 1.32214097448013e-08
2600 1.33091901943772e-08
2601 1.32747510761533e-08
2602 1.33066793139847e-08
2603 1.32633353189249e-08
2604 1.31889441590261e-08
2605 1.32165114408167e-08
2606 1.33495818843699e-08
2607 1.31578188344861e-08
2608 1.3349127137019e-08
2609 1.328071608242e-08
2610 1.35072246720824e-08
2611 1.31750184095836e-08
2612 1.31514035217606e-08
2613 1.32168480604378e-08
2614 1.3140531329725e-08
2615 1.32391200224902e-08
2616 1.32122561780079e-08
2617 1.31696920036006e-08
2618 1.32540529662606e-08
2619 1.31380017975857e-08
2620 1.32855637602347e-08
2621 1.32736337477013e-08
2622 1.31574342532303e-08
2623 1.31429809258066e-08
2624 1.32217099491072e-08
2625 1.31352084764558e-08
2626 1.32011868103632e-08
2627 1.32018209697549e-08
2628 1.32252102602592e-08
2629 1.3170357249237e-08
2630 1.32035937738806e-08
2631 1.3158599543317e-08
2632 1.3213703908832e-08
2633 1.31407142944795e-08
2634 1.32138264774539e-08
2635 1.30969919354129e-08
2636 1.30843789136748e-08
2637 1.32047199841168e-08
2638 1.31131994152156e-08
2639 1.31663835389872e-08
2640 1.31117392498936e-08
2641 1.30918262897239e-08
2642 1.31663311364605e-08
2643 1.30537944897924e-08
2644 1.31736817010619e-08
2645 1.30987078961198e-08
2646 1.30652395569086e-08
2647 1.31323183438781e-08
2648 1.30460486857942e-08
2649 1.3047421809631e-08
2650 1.30240138673798e-08
2651 1.30959341149151e-08
2652 1.3060872383619e-08
2653 1.30747892512773e-08
2654 1.30166339928905e-08
2655 1.30271136100646e-08
2656 1.30596928826776e-08
2657 1.30034374379306e-08
2658 1.30644206564057e-08
2659 1.30149331312168e-08
2660 1.31046116180755e-08
2661 1.30159847344657e-08
2662 1.30157804534292e-08
2663 1.30029595979408e-08
2664 1.30438682077738e-08
2665 1.29430182127521e-08
2666 1.30617099358687e-08
2667 1.2980576613586e-08
2668 1.30711104162629e-08
2669 1.30163178013731e-08
2670 1.30339978809957e-08
2671 1.30142847609704e-08
2672 1.29672113047263e-08
2673 1.30168498202465e-08
2674 1.29939108362009e-08
2675 1.30269022236007e-08
2676 1.29865167508569e-08
2677 1.30081305727003e-08
2678 1.29257831105178e-08
2679 1.28849793057384e-08
2680 1.29137580628935e-08
2681 1.29747554922233e-08
2682 1.2978397023744e-08
2683 1.30726878211362e-08
2684 1.3060971859602e-08
2685 1.30563568845332e-08
2686 1.30917214846704e-08
2687 1.3138772736454e-08
2688 1.30802924047657e-08
2689 1.30350521487799e-08
2690 1.30752662030886e-08
2691 1.31201813857729e-08
2692 1.30661774733198e-08
2693 1.30234134587681e-08
2694 1.30708519563427e-08
2695 1.31132438241366e-08
2696 1.30588038160795e-08
2697 1.30129871322993e-08
2698 1.3062967596511e-08
2699 1.31020669869031e-08
2700 1.30570052547796e-08
2701 1.29982309360344e-08
2702 1.30565904754576e-08
2703 1.31014985527145e-08
2704 1.3056022041269e-08
2705 1.29937856030438e-08
2706 1.30487052274475e-08
2707 1.31007444892361e-08
2708 1.30840334122695e-08
2709 1.3112470220733e-08
2710 1.30878570203663e-08
2711 1.31088659927059e-08
2712 1.30940751574826e-08
2713 1.30363719819115e-08
2714 1.30028121603232e-08
2715 1.30576065515697e-08
2716 1.30848896162661e-08
2717 1.30334676384791e-08
2718 1.30195623171403e-08
2719 1.30598909464652e-08
2720 1.30889539207146e-08
2721 1.30389592456481e-08
2722 1.29435511198039e-08
2723 1.30221460281632e-08
2724 1.30746675708338e-08
2725 1.3030504675271e-08
2726 1.29324080333504e-08
2727 1.3013875310719e-08
2728 1.30695365641031e-08
2729 1.30167672196535e-08
2730 1.29245520952281e-08
2731 1.30033814826902e-08
2732 1.30691635291669e-08
2733 1.30096697859017e-08
2734 1.29196271458909e-08
2735 1.30030439748907e-08
2736 1.30628725614201e-08
2737 1.30043575907735e-08
2738 1.29127375458893e-08
2739 1.29962707262621e-08
2740 1.3060252435082e-08
2741 1.29927819614295e-08
2742 1.29084183342343e-08
2743 1.29985604502281e-08
2744 1.3062658510421e-08
2745 1.29897879119767e-08
2746 1.29097426082581e-08
2747 1.29992550057523e-08
2748 1.30576403023497e-08
2749 1.29893624745137e-08
2750 1.29053345787611e-08
2751 1.29945778581941e-08
2752 1.30529516084721e-08
2753 1.29800810100278e-08
2754 1.29021842099064e-08
2755 1.29924133673853e-08
2756 1.30478010618162e-08
2757 1.29726913655759e-08
2758 1.29000499171639e-08
2759 1.29825803441008e-08
2760 1.30462751712912e-08
2761 1.29618813460297e-08
2762 1.29013955074697e-08
2763 1.29796910997015e-08
2764 1.30425101829701e-08
2765 1.29554766914453e-08
2766 1.28969590562633e-08
2767 1.29731505538189e-08
2768 1.30365265249566e-08
2769 1.29464048370664e-08
2770 1.28932615695021e-08
2771 1.29639747825649e-08
2772 1.30343940085709e-08
2773 1.29351089839247e-08
2774 1.28938415500102e-08
2775 1.29555601802167e-08
2776 1.30225936700867e-08
2777 1.29249881908322e-08
2778 1.28890054185149e-08
2779 1.29475612453689e-08
2780 1.30223707373034e-08
2781 1.29156099148986e-08
2782 1.28936976651062e-08
2783 1.29425128392313e-08
2784 1.30075639148686e-08
2785 1.29037660556719e-08
2786 1.28778889774139e-08
2787 1.2934138204912e-08
2788 1.30060762160156e-08
2789 1.28994743775479e-08
2790 1.28893260509244e-08
2791 1.29247519353726e-08
2792 1.29968791284796e-08
2793 1.29039534613185e-08
2794 1.29234098977804e-08
2795 1.28769137575091e-08
2796 1.29740262977407e-08
2797 1.29241461976903e-08
2798 1.29257857750531e-08
2799 1.29096147105656e-08
2800 1.29690613803746e-08
2801 1.28685266886919e-08
2802 1.29161561446267e-08
2803 1.28725510251115e-08
2804 1.2953687900108e-08
2805 1.28967245771605e-08
2806 1.29169874796276e-08
2807 1.29167929685536e-08
2808 1.29515180802287e-08
2809 1.28462938064899e-08
2810 1.28482309236233e-08
2811 1.28966703982769e-08
2812 1.29268364901236e-08
2813 1.28389130438222e-08
2814 1.28254855624732e-08
2815 1.28916113339983e-08
2816 1.28947217348241e-08
2817 1.28271588906159e-08
2818 1.28106396601879e-08
2819 1.28888251182957e-08
2820 1.28702408730419e-08
2821 1.28202648497222e-08
2822 1.28048958103477e-08
2823 1.28657724474124e-08
2824 1.28513306663081e-08
2825 1.28590222914227e-08
2826 1.28153345713145e-08
2827 1.27874191235833e-08
2828 1.28960380152421e-08
2829 1.27934143279163e-08
2830 1.28681447719714e-08
2831 1.28313484282216e-08
2832 1.28701653778762e-08
2833 1.28143931021896e-08
2834 1.28611024052816e-08
2835 1.28306680835522e-08
2836 1.28168116120264e-08
2837 1.28143815558701e-08
2838 1.2872805932318e-08
2839 1.27550929818199e-08
2840 1.2790662751172e-08
2841 1.28143842204054e-08
2842 1.28135297927656e-08
2843 1.27900730007013e-08
2844 1.28500428075995e-08
2845 1.27213759526512e-08
2846 1.27229320412425e-08
2847 1.28039898683596e-08
2848 1.27727792786914e-08
2849 1.27422712381531e-08
2850 1.27559296458912e-08
2851 1.27962058726894e-08
2852 1.27470025645948e-08
2853 1.27197070654006e-08
2854 1.26814709844325e-08
2855 1.28121833142814e-08
2856 1.27374155667326e-08
2857 1.27554793394324e-08
2858 1.27264287996809e-08
2859 1.27284920381499e-08
2860 1.27457076004589e-08
2861 1.27743966515936e-08
2862 1.27131736249453e-08
2863 1.27215464829078e-08
2864 1.27328938503979e-08
2865 1.2758659018175e-08
2866 1.26976749115215e-08
2867 1.26795045574113e-08
2868 1.26443211456717e-08
2869 1.27810171335341e-08
2870 1.26957786505955e-08
2871 1.27283223960717e-08
2872 1.27049508691357e-08
2873 1.26970824965156e-08
2874 1.26397310395987e-08
2875 1.26217898355208e-08
2876 1.27386341475244e-08
2877 1.26761632301964e-08
2878 1.26805765887639e-08
2879 1.26346444417891e-08
2880 1.26539347888865e-08
2881 1.27168355845697e-08
2882 1.26437180725247e-08
2883 1.26295240931995e-08
2884 1.25993251387513e-08
2885 1.27328734222942e-08
2886 1.26398402855443e-08
2887 1.26761428020927e-08
2888 1.26531976007982e-08
2889 1.26353434382054e-08
2890 1.26228494323755e-08
2891 1.26584005499808e-08
2892 1.26458097327031e-08
2893 1.26190231597434e-08
2894 1.25964172426052e-08
2895 1.26049188864386e-08
2896 1.26726407145838e-08
2897 1.26010100132135e-08
2898 1.26636461317275e-08
2899 1.25582477750186e-08
2900 1.26322499127696e-08
2901 1.26375390152589e-08
2902 1.25916370663504e-08
2903 1.26413750578536e-08
2904 1.25428627484325e-08
2905 1.25814958451542e-08
2906 1.26361507923889e-08
2907 1.25666801409352e-08
2908 1.26271411104995e-08
2909 1.25198038603003e-08
2910 1.25729320288315e-08
2911 1.26116717069635e-08
2912 1.2552668238186e-08
2913 1.26099148900494e-08
2914 1.25004593343192e-08
2915 1.25508128334673e-08
2916 1.26002808187309e-08
2917 1.25347954238464e-08
2918 1.25884440649315e-08
2919 1.24891181840781e-08
2920 1.25496288916338e-08
2921 1.25682024787466e-08
2922 1.25261081507233e-08
2923 1.25595294164782e-08
2924 1.2483499567395e-08
2925 1.25424310937206e-08
2926 1.2537653582001e-08
2927 1.25036754283769e-08
2928 1.2557872075547e-08
2929 1.24525278977217e-08
2930 1.24996608619199e-08
2931 1.25468329059686e-08
2932 1.24807986168207e-08
2933 1.25347892065975e-08
2934 1.24372840915044e-08
2935 1.25094814507065e-08
2936 1.25027552755341e-08
2937 1.24802310708105e-08
2938 1.24855112915156e-08
2939 1.24543770851915e-08
2940 1.24687717928396e-08
2941 1.24904460108155e-08
2942 1.24316850147466e-08
2943 1.24670131995686e-08
2944 1.24772538967477e-08
2945 1.24349188723727e-08
2946 1.25011210272419e-08
2947 1.23820367292637e-08
2948 1.24437926629639e-08
2949 1.24717107752303e-08
2950 1.24249845967483e-08
2951 1.2452662012663e-08
2952 1.23883383551515e-08
2953 1.24300845172343e-08
2954 1.24473675811032e-08
2955 1.23726948686453e-08
2956 1.2423743811496e-08
2957 1.24307728555095e-08
2958 1.24010899327232e-08
2959 1.24069945428573e-08
2960 1.23925643080725e-08
2961 1.23924372985584e-08
2962 1.24280052915537e-08
2963 1.23348486980035e-08
2964 1.23570060850398e-08
2965 1.24331709372427e-08
2966 1.23502310600543e-08
2967 1.23948407093621e-08
2968 1.23311636457402e-08
2969 1.24265691070491e-08
2970 1.22709646888097e-08
2971 1.23841665811142e-08
2972 1.23434853449567e-08
2973 1.23647030392249e-08
2974 1.23255654571608e-08
2975 1.23294707776722e-08
2976 1.22734844509864e-08
2977 1.2385357628375e-08
2978 1.22934942226038e-08
2979 1.23219079384285e-08
2980 1.22457342044413e-08
2981 1.23823191700012e-08
2982 1.22790240197901e-08
2983 1.23112986472051e-08
2984 1.22303411842495e-08
2985 1.23708101540387e-08
2986 1.22679422176475e-08
2987 1.22947110270388e-08
2988 1.22102887800679e-08
2989 1.23589893874509e-08
2990 1.22524035361948e-08
2991 1.22839489691273e-08
2992 1.22082068898521e-08
2993 1.23312231536943e-08
2994 1.22499104193707e-08
2995 1.22603509566943e-08
2996 1.21587824253311e-08
2997 1.23508350213797e-08
2998 1.22093215537689e-08
2999 1.22834125093618e-08
3000 8.17177170375771e-09
3001 8.23981594066936e-09
3002 8.33371860409216e-09
3003 8.36992697372807e-09
3004 8.37954772237026e-09
3005 8.38070945974323e-09
3006 8.37950242527086e-09
3007 8.37803604269993e-09
3008 8.37655989016639e-09
3009 8.37523650432104e-09
3010 8.37380831342216e-09
3011 8.37256752816984e-09
3012 8.37122104968557e-09
3013 8.37011171483937e-09
3014 8.36892777300591e-09
3015 8.36809732618349e-09
3016 8.36701463668987e-09
3017 8.36597013886831e-09
3018 8.36501357071029e-09
3019 8.36409697058116e-09
3020 8.36330205089553e-09
3021 8.36235258816487e-09
3022 8.36164559814279e-09
3023 8.36079028232461e-09
3024 8.36005309423626e-09
3025 8.35925639819379e-09
3026 8.35856361902643e-09
3027 8.3577686993408e-09
3028 8.35721980507742e-09
3029 8.35650038055746e-09
3030 8.35576141611227e-09
3031 8.35517965924737e-09
3032 8.35463431769767e-09
3033 8.35391666953456e-09
3034 8.35332869542071e-09
3035 8.35259506004604e-09
3036 8.35208524563313e-09
3037 8.3514617443825e-09
3038 8.3509519299696e-09
3039 8.35025204537487e-09
3040 8.3497120328957e-09
3041 8.34921909387276e-09
3042 8.34861069165527e-09
3043 8.34806712646241e-09
3044 8.34766211710303e-09
3045 8.34726776588468e-09
3046 8.3467819322891e-09
3047 8.34629876322879e-09
3048 8.34563795848453e-09
3049 8.34528890436559e-09
3050 8.34476043820587e-09
3051 8.34426305829084e-09
3052 8.34379942915575e-09
3053 8.34337221533588e-09
3054 8.3429494424081e-09
3055 8.34258706561286e-09
3056 8.34191826726283e-09
3057 8.34159585849648e-09
3058 8.34105229330362e-09
3059 8.34062419130532e-09
3060 8.33995450477687e-09
3061 8.33957880530534e-09
3062 8.33921376397484e-09
3063 8.33879987283126e-09
3064 8.33840108072081e-09
3065 8.33782376474801e-09
3066 8.3374400716707e-09
3067 8.33701729874292e-09
3068 8.33666824462398e-09
3069 8.33617708195789e-09
3070 8.33567082025866e-09
3071 8.33528623900293e-09
3072 8.33487856510828e-09
3073 8.33444246950421e-09
3074 8.33393976051866e-09
3075 8.33361202268179e-09
3076 8.33290414448129e-09
3077 8.33273361422471e-09
3078 8.33227620233856e-09
3079 8.33189250926125e-09
3080 8.33146440726296e-09
3081 8.33100965991207e-09
3082 8.33051227999704e-09
3083 8.33022006929696e-09
3084 8.32970581399195e-09
3085 8.32930791005992e-09
3086 8.329013923003e-09
3087 8.3284774632375e-09
3088 8.32809998740913e-09
3089 8.32769586622817e-09
3090 8.32728286326301e-09
3091 8.32678459516956e-09
3092 8.32650659532419e-09
3093 8.3259479310982e-09
3094 8.32560509422819e-09
3095 8.32510682613474e-09
3096 8.32464319699966e-09
3097 8.32427282659864e-09
3098 8.32380742110672e-09
3099 8.32351521040664e-09
3100 8.32308089115941e-09
3101 8.3225755176386e-09
3102 8.32226643154854e-09
3103 8.32179924969978e-09
3104 8.32147950546869e-09
3105 8.3210158763336e-09
3106 8.32070501388671e-09
3107 8.32021740393429e-09
3108 8.32004953821297e-09
3109 8.3195006439496e-09
3110 8.31895441422148e-09
3111 8.31867463801927e-09
3112 8.31833180114927e-09
3113 8.31763991016032e-09
3114 8.31747293261742e-09
3115 8.31702440251547e-09
3116 8.31675706081114e-09
3117 8.31625524000401e-09
3118 8.31597368744497e-09
3119 8.31556778990716e-09
3120 8.31503133014166e-09
3121 8.31471336226741e-09
3122 8.31429858294541e-09
3123 8.31395752243225e-09
3124 8.31352853225553e-09
3125 8.31319457716972e-09
3126 8.31277180424195e-09
3127 8.31240942744671e-09
3128 8.31203816886728e-09
3129 8.31151680813491e-09
3130 8.31108426524452e-09
3131 8.31073787566083e-09
3132 8.31045721128021e-09
3133 8.31001578660562e-09
3134 8.30970758869398e-09
3135 8.30928215123095e-09
3136 8.30877766588856e-09
3137 8.30835400478236e-09
3138 8.30813018382059e-09
3139 8.30765767290131e-09
3140 8.30714785848841e-09
3141 8.30699065090812e-09
3142 8.30653501537881e-09
3143 8.30606072810269e-09
3144 8.3056281852123e-09
3145 8.30530222373227e-09
3146 8.30477198121571e-09
3147 8.30451973854451e-09
3148 8.30393442896593e-09
3149 8.3036040265938e-09
3150 8.30320523448336e-09
3151 8.30293256370851e-09
3152 8.30235258320045e-09
3153 8.3021207686329e-09
3154 8.30167223853095e-09
3155 8.30135338247828e-09
3156 8.30079738278755e-09
3157 8.30036839261084e-09
3158 8.30023072495578e-09
3159 8.29980617567116e-09
3160 8.29945889790906e-09
3161 8.29898372245452e-09
3162 8.29847124350636e-09
3163 8.29819679637467e-09
3164 8.29778645794477e-09
3165 8.29743651564741e-09
3166 8.29709900784792e-09
3167 8.29658031165081e-09
3168 8.29618596043247e-09
3169 8.29588664430503e-09
3170 8.29557045278762e-09
3171 8.29516011435771e-09
3172 8.29475510499833e-09
3173 8.29437318827786e-09
3174 8.29386337386495e-09
3175 8.29355784048857e-09
3176 8.29318391737388e-09
3177 8.29291213477745e-09
3178 8.29241653121926e-09
3179 8.29190760498477e-09
3180 8.29168023130933e-09
3181 8.29123525392106e-09
3182 8.29079382924647e-09
3183 8.2904483278412e-09
3184 8.29006818747757e-09
3185 8.28967294808081e-09
3186 8.28925994511565e-09
3187 8.28896595805872e-09
3188 8.28855206691514e-09
3189 8.28812396491685e-09
3190 8.28754309623037e-09
3191 8.28735036151329e-09
3192 8.28694446397549e-09
3193 8.28678992093046e-09
3194 8.28605184466369e-09
3195 8.28580670741985e-09
3196 8.28538837538417e-09
3197 8.28513080364246e-09
3198 8.28467072722106e-09
3199 8.28419377540968e-09
3200 8.28365642746576e-09
3201 8.28337665126355e-09
3202 8.28311463862974e-09
3203 8.28250801276909e-09
3204 8.2821749458617e-09
3205 8.2819271440826e-09
3206 8.28149016030011e-09
3207 8.28123170037998e-09
3208 8.28068902336554e-09
3209 8.28019963705628e-09
3210 8.27994650620667e-09
3211 8.27966317729079e-09
3212 8.27923241075723e-09
3213 8.27878121612002e-09
3214 8.27827406624237e-09
3215 8.27786905688299e-09
3216 8.27755286536558e-09
3217 8.27722335117187e-09
3218 8.2768369935593e-09
3219 8.27646040590935e-09
3220 8.27602519848369e-09
3221 8.27555535209967e-09
3222 8.27525603597223e-09
3223 8.27496027255847e-09
3224 8.27443447093401e-09
3225 8.27408896952875e-09
3226 8.2736946183104e-09
3227 8.27341928300029e-09
3228 8.27294943661627e-09
3229 8.27257995439368e-09
3230 8.27231172451093e-09
3231 8.2717512839281e-09
3232 8.27153900928579e-09
3233 8.27110113732488e-09
3234 8.27078405762904e-09
3235 8.27038260098334e-09
3236 8.2699749270887e-09
3237 8.26956103594512e-09
3238 8.26913115758998e-09
3239 8.26880874882363e-09
3240 8.26838686407427e-09
3241 8.26799073649909e-09
3242 8.26769053219323e-09
3243 8.26729795733172e-09
3244 8.26693824507174e-09
3245 8.26649948493241e-09
3246 8.26598434144898e-09
3247 8.2656876898568e-09
3248 8.26511126206242e-09
3249 8.26483859128757e-09
3250 8.26436519218987e-09
3251 8.26408896870134e-09
3252 8.2636901765909e-09
3253 8.26346191473704e-09
3254 8.26289703326211e-09
3255 8.26260926345412e-09
3256 8.26217050331479e-09
3257 8.2617832575238e-09
3258 8.26142088072856e-09
3259 8.26118373709051e-09
3260 8.26065260639552e-09
3261 8.26028401235135e-09
3262 8.25981505414575e-09
3263 8.25947132909732e-09
3264 8.25917911839724e-09
3265 8.25874924004211e-09
3266 8.25822343841764e-09
3267 8.25796764303277e-09
3268 8.25761414802173e-09
3269 8.25707857643465e-09
3270 8.25679880023245e-09
3271 8.2565758674491e-09
3272 8.25606427667935e-09
3273 8.25568147178046e-09
3274 8.25531643044997e-09
3275 8.25481549782126e-09
3276 8.25453838615431e-09
3277 8.2541591339691e-09
3278 8.25367507673036e-09
3279 8.25338997145764e-09
3280 8.25311907703963e-09
3281 8.25276735838543e-09
3282 8.25225843215094e-09
3283 8.25200796583658e-09
3284 8.25158252837355e-09
3285 8.25104518042963e-09
3286 8.25061796660975e-09
3287 8.25035328944068e-09
3288 8.25001933435487e-09
3289 8.24966139845174e-09
3290 8.24928392262336e-09
3291 8.2487714436752e-09
3292 8.24851120739822e-09
3293 8.24801738019687e-09
3294 8.24773138674573e-09
3295 8.24737433902101e-09
3296 8.24688761724701e-09
3297 8.24655721487488e-09
3298 8.24620016715016e-09
3299 8.24578361147132e-09
3300 8.24535639765145e-09
3301 8.24502777163616e-09
3302 8.24460943960048e-09
3303 8.24407031529972e-09
3304 8.24387491604739e-09
3305 8.2435009929327e-09
3306 8.24312795799642e-09
3307 8.24266788157502e-09
3308 8.24232060381291e-09
3309 8.242054150287e-09
3310 8.2416029556498e-09
3311 8.24118551179254e-09
3312 8.2409403745487e-09
3313 8.24050871983673e-09
3314 8.24010726319102e-09
3315 8.23987100773138e-09
3316 8.23923329562604e-09
3317 8.23887180700922e-09
3318 8.2385236410687e-09
3319 8.23826784568382e-09
3320 8.23788504078493e-09
3321 8.23752444034653e-09
3322 8.2370945619914e-09
3323 8.23673040883932e-09
3324 8.23643375724714e-09
3325 8.23597012811206e-09
3326 8.23564949570255e-09
3327 8.2353812658198e-09
3328 8.23497448010357e-09
3329 8.2345339436074e-09
3330 8.23415113870851e-09
3331 8.23390333692942e-09
3332 8.23345658318431e-09
3333 8.23301338215288e-09
3334 8.23264656446554e-09
3335 8.23228241131346e-09
3336 8.23186230292094e-09
3337 8.23164736374338e-09
3338 8.23122991988612e-09
3339 8.2308053706015e-09
3340 8.23040569031264e-09
3341 8.23001222727271e-09
3342 8.2298177161988e-09
3343 8.22921908394392e-09
3344 8.22902102015632e-09
3345 8.2284969948887e-09
3346 8.22825541035854e-09
3347 8.22786816456755e-09
3348 8.22747558970605e-09
3349 8.22718337900596e-09
3350 8.22678725143078e-09
3351 8.22627832519629e-09
3352 8.22614865114701e-09
3353 8.22560775048942e-09
3354 8.2253590605319e-09
3355 8.22480661355485e-09
3356 8.22453305460158e-09
3357 8.22412093981484e-09
3358 8.22363865893294e-09
3359 8.22338019901281e-09
3360 8.22296453151239e-09
3361 8.2225097841615e-09
3362 8.2222344488514e-09
3363 8.22176815518105e-09
3364 8.2214004493153e-09
3365 8.22102030895167e-09
3366 8.22068013661692e-09
3367 8.22019963209186e-09
3368 8.22004686540367e-09
3369 8.21944912132722e-09
3370 8.21937362616154e-09
3371 8.21878387569086e-09
3372 8.21846057874609e-09
3373 8.21805024031619e-09
3374 8.21773848969087e-09
3375 8.2173023940868e-09
3376 8.21697021535783e-09
3377 8.21670020911824e-09
3378 8.21630763425674e-09
3379 8.21596657374357e-09
3380 8.21555445895683e-09
3381 8.21520806937315e-09
3382 8.21479062551589e-09
3383 8.21433498998658e-09
3384 8.21404455564334e-09
3385 8.21375500947852e-09
3386 8.213334901086e-09
3387 8.21306311848957e-09
3388 8.21253465232985e-09
3389 8.21218559821091e-09
3390 8.21174950260684e-09
3391 8.21146617369095e-09
3392 8.21110024418203e-09
3393 8.21066681311322e-09
3394 8.21040657683625e-09
3395 8.21009304985409e-09
3396 8.20963119707585e-09
3397 8.20926082667484e-09
3398 8.20876611129506e-09
3399 8.20852363858648e-09
3400 8.20803336409881e-09
3401 8.20768253362303e-09
3402 8.20732992679041e-09
3403 8.20703061066297e-09
3404 8.20661583134097e-09
3405 8.20627210629254e-09
3406 8.20588486050156e-09
3407 8.20553935909629e-09
3408 8.20518764044209e-09
3409 8.20460499539877e-09
3410 8.20442913607167e-09
3411 8.20405698931381e-09
3412 8.20354806307932e-09
3413 8.20333578843702e-09
3414 8.20291479186608e-09
3415 8.20253021061035e-09
3416 8.20202039619744e-09
3417 8.20181789151775e-09
3418 8.20140932944469e-09
3419 8.20100964915582e-09
3420 8.20069345763841e-09
3421 8.20044743221615e-09
3422 8.19992251877011e-09
3423 8.19953260844386e-09
3424 8.19917644889756e-09
3425 8.19896683879051e-09
3426 8.19851742051014e-09
3427 8.19814882646597e-09
3428 8.1978033250607e-09
3429 8.19737433488399e-09
3430 8.19708478871917e-09
3431 8.19665135765035e-09
3432 8.19633605431136e-09
3433 8.19590706413464e-09
3434 8.19564682785767e-09
3435 8.1953910324728e-09
3436 8.19489454073619e-09
3437 8.19444245792056e-09
3438 8.19421686060195e-09
3439 8.19372036886534e-09
3440 8.19336332114062e-09
3441 8.19304180055269e-09
3442 8.19279488695202e-09
3443 8.19242806926468e-09
3444 8.19198220369799e-09
3445 8.19167844667845e-09
3446 8.19130008267166e-09
3447 8.19089329695544e-09
3448 8.190593980828e-09
3449 8.19029555287898e-09
3450 8.1899313997269e-09
3451 8.18960632642529e-09
3452 8.18913026279233e-09
3453 8.18883005848647e-09
3454 8.18847656347543e-09
3455 8.18808931768444e-09
3456 8.18776690891809e-09
3457 8.18732015517298e-09
3458 8.18687340142787e-09
3459 8.18664602775243e-09
3460 8.18623924203621e-09
3461 8.18576584293851e-09
3462 8.18553491654939e-09
3463 8.18514944711524e-09
3464 8.18473022690114e-09
3465 8.18437850824694e-09
3466 8.18405787583742e-09
3467 8.18353473874822e-09
3468 8.18326739704389e-09
3469 8.18289613846446e-09
3470 8.18258527601756e-09
3471 8.18210210695725e-09
3472 8.18176815187144e-09
3473 8.18136758340415e-09
3474 8.18107093181197e-09
3475 8.18066681063101e-09
3476 8.18018186521385e-09
3477 8.17999534774572e-09
3478 8.1796294182368e-09
3479 8.17928302865312e-09
3480 8.17881762316119e-09
3481 8.17840906108813e-09
3482 8.17822520815525e-09
3483 8.17781398154693e-09
3484 8.17745071657328e-09
3485 8.1770821225291e-09
3486 8.17656431451041e-09
3487 8.1761353243337e-09
3488 8.1759097270151e-09
3489 8.17558021282139e-09
3490 8.17518497342462e-09
3491 8.1748456892683e-09
3492 8.17463341462599e-09
3493 8.17408185582735e-09
3494 8.17371947903212e-09
3495 8.17348322357248e-09
3496 8.17314571577299e-09
3497 8.17276823994462e-09
3498 8.17235523697946e-09
3499 8.17191292412645e-09
3500 8.17161893706952e-09
3501 8.17108958273138e-09
3502 8.17083378734651e-09
3503 8.17053091850539e-09
3504 8.1701525544986e-09
3505 8.16985146201432e-09
3506 8.16942780090812e-09
3507 8.1690210151919e-09
3508 8.16875367348757e-09
3509 8.16839218487075e-09
3510 8.16787437685207e-09
3511 8.16762835142981e-09
3512 8.16726775099141e-09
3513 8.16701728467706e-09
3514 8.1665412210441e-09
3515 8.16624723398718e-09
3516 8.16575074225057e-09
3517 8.16544076798209e-09
3518 8.16531553482491e-09
3519 8.16486345200929e-09
3520 8.16435363759638e-09
3521 8.16407919046469e-09
3522 8.16365197664481e-09
3523 8.16341039211466e-09
3524 8.16306844342307e-09
3525 8.16259326796853e-09
3526 8.16222556210278e-09
3527 8.16183121088443e-09
3528 8.16139422710194e-09
3529 8.16117573521069e-09
3530 8.16094747335683e-09
3531 8.16059309016737e-09
3532 8.16018363991589e-09
3533 8.15973510981394e-09
3534 8.1593816148029e-09
3535 8.15895884187512e-09
3536 8.15865686121242e-09
3537 8.15837086776128e-09
3538 8.15785039520733e-09
3539 8.15762923878083e-09
3540 8.15711764801108e-09
3541 8.15677303478424e-09
3542 8.15655720742825e-09
3543 8.15612999360837e-09
3544 8.15573741874687e-09
3545 8.15533596210116e-09
3546 8.15494871631017e-09
3547 8.15461476122437e-09
3548 8.15412004584459e-09
3549 8.15394063380381e-09
3550 8.15352585448181e-09
3551 8.1532096629644e-09
3552 8.15282064081657e-09
3553 8.15243339502558e-09
3554 8.15217759964071e-09
3555 8.15171663504088e-09
3556 8.15141731891345e-09
3557 8.15088085914795e-09
3558 8.15085243743852e-09
3559 8.15037459744872e-09
3560 8.14999356890667e-09
3561 8.14959122408254e-09
3562 8.14934075776819e-09
3563 8.14890466216411e-09
3564 8.14863909681662e-09
3565 8.148187014001e-09
3566 8.14779621549633e-09
3567 8.14753686739778e-09
3568 8.14711231811316e-09
3569 8.14684764094409e-09
3570 8.14647371782939e-09
3571 8.14621881062294e-09
3572 8.14571521345897e-09
3573 8.14536971205371e-09
3574 8.14505707324997e-09
3575 8.14470091370367e-09
3576 8.14428613438167e-09
3577 8.1440258981047e-09
3578 8.14346101662977e-09
3579 8.14320344488806e-09
3580 8.14298495299681e-09
3581 8.14261191806054e-09
3582 8.14216960520753e-09
3583 8.14189693443268e-09
3584 8.14146439154229e-09
3585 8.14102563140295e-09
3586 8.14081335676065e-09
3587 8.14045897357119e-09
3588 8.14002643068079e-09
3589 8.1397066864497e-09
3590 8.13933631604868e-09
3591 8.13895173479295e-09
3592 8.13866041227129e-09
3593 8.13836642521437e-09
3594 8.13803158195014e-09
3595 8.1375066685041e-09
3596 8.13732725646332e-09
3597 8.13682188294251e-09
3598 8.13664691179383e-09
3599 8.13620637529766e-09
3600 8.1357800496562e-09
3601 8.13537948118892e-09
3602 8.13516987108187e-09
3603 8.13476397354407e-09
3604 8.13438116864518e-09
3605 8.13403833177517e-09
3606 8.13370970575988e-09
3607 8.13327538651265e-09
3608 8.133011597522e-09
3609 8.13261902266049e-09
3610 8.1322992784294e-09
3611 8.13193334892048e-09
3612 8.13169442892558e-09
3613 8.13126899146255e-09
3614 8.13074496619492e-09
3615 8.13044742642433e-09
3616 8.13020406553733e-09
3617 8.12979550346427e-09
3618 8.12956990614566e-09
3619 8.12924305648721e-09
3620 8.12882117173785e-09
3621 8.12830069918391e-09
3622 8.1280839836495e-09
3623 8.1276105845518e-09
3624 8.12740630351527e-09
3625 8.12696754337594e-09
3626 8.12650924331138e-09
3627 8.12637335201316e-09
3628 8.12587863663339e-09
3629 8.12554379336916e-09
3630 8.12523026638701e-09
3631 8.1249007521933e-09
3632 8.12457212617801e-09
3633 8.12410494432925e-09
3634 8.1237239157872e-09
3635 8.12350897660963e-09
3636 8.12317324516698e-09
3637 8.122714056924e-09
3638 8.12243161618653e-09
3639 8.1220061787235e-09
3640 8.12165712460455e-09
3641 8.12131872862665e-09
3642 8.1208790803089e-09
3643 8.12067035838027e-09
3644 8.12032396879658e-09
3645 8.11993405847034e-09
3646 8.11954148360883e-09
3647 8.1190894007932e-09
3648 8.1188460399062e-09
3649 8.11840994430213e-09
3650 8.11807776557316e-09
3651 8.1177731203752e-09
3652 8.11735123562585e-09
3653 8.11705991310419e-09
3654 8.11663092292747e-09
3655 8.11635025854684e-09
3656 8.1159470255443e-09
3657 8.11556866153751e-09
3658 8.11537681499885e-09
3659 8.11500999731152e-09
3660 8.11458011895638e-09
3661 8.11434297531832e-09
3662 8.11394773592156e-09
3663 8.1137061513914e-09
3664 8.11323452865054e-09
3665 8.11293432434468e-09
3666 8.1126785289598e-09
3667 8.1123943118655e-09
3668 8.11186495752736e-09
3669 8.11156741775676e-09
3670 8.11120504096152e-09
3671 8.11094391650613e-09
3672 8.11046163562423e-09
3673 8.11017386581625e-09
3674 8.10985945065568e-09
3675 8.10946154672365e-09
3676 8.10914713156308e-09
3677 8.10875366852315e-09
3678 8.10852629484771e-09
3679 8.10798272965485e-09
3680 8.10770650616632e-09
3681 8.10745426349513e-09
3682 8.10694888997432e-09
3683 8.10666112016634e-09
3684 8.10629252612216e-09
3685 8.1058839640491e-09
3686 8.10561928688003e-09
3687 8.10528710815106e-09
3688 8.10490075053849e-09
3689 8.10454547917061e-09
3690 8.10419109598115e-09
3691 8.10387668082058e-09
3692 8.10347344781803e-09
3693 8.10329225942041e-09
3694 8.10280997853852e-09
3695 8.10247424709587e-09
3696 8.10214118018848e-09
3697 8.1018578512726e-09
3698 8.10138178763964e-09
3699 8.10113398586054e-09
3700 8.10060107880872e-09
3701 8.10038258691748e-09
3702 8.10001754558698e-09
3703 8.09963474068809e-09
3704 8.09940647883423e-09
3705 8.09897393594383e-09
3706 8.09859113104494e-09
3707 8.09825628778071e-09
3708 8.09778644139669e-09
3709 8.09762124021063e-09
3710 8.09720734906705e-09
3711 8.09685030134233e-09
3712 8.09644173926927e-09
3713 8.09609890239926e-09
3714 8.09583067251651e-09
3715 8.09549849378755e-09
3716 8.09511391253182e-09
3717 8.09461475625994e-09
3718 8.09443267968391e-09
3719 8.09412803448595e-09
3720 8.09373279508918e-09
3721 8.09345213070856e-09
3722 8.09297961978928e-09
3723 8.09267319823448e-09
3724 8.0923179268666e-09
3725 8.09204081519965e-09
3726 8.09161626591504e-09
3727 8.09133116064231e-09
3728 8.09092348674767e-09
3729 8.09065259232966e-09
3730 8.09029288006968e-09
3731 8.09005129553952e-09
3732 8.08948819042143e-09
3733 8.0891711107256e-09
3734 8.08878386493461e-09
3735 8.08850053601873e-09
3736 8.08803779506206e-09
3737 8.08774114346988e-09
3738 8.08738498392358e-09
3739 8.08697109278e-09
3740 8.08664246676472e-09
3741 8.08626143822266e-09
3742 8.08610600699922e-09
3743 8.08565037146991e-09
3744 8.08514144523542e-09
3745 8.08494338144783e-09
3746 8.08463873624987e-09
3747 8.08431366294826e-09
3748 8.08394595708251e-09
3749 8.08364752913349e-09
3750 8.08309952304853e-09
3751 8.08284461584208e-09
3752 8.08244404737479e-09
3753 8.08223887815984e-09
3754 8.08197775370445e-09
3755 8.0815292236025e-09
3756 8.08114819506045e-09
3757 8.08073696845213e-09
3758 8.08044120503837e-09
3759 8.07999711582852e-09
3760 8.0796400681038e-09
3761 8.07949795955665e-09
3762 8.07896771704009e-09
3763 8.07872080343941e-09
3764 8.07829891869005e-09
3765 8.07807154501461e-09
3766 8.07758571141903e-09
3767 8.07729794161105e-09
3768 8.07681121983705e-09
3769 8.07666200586254e-09
3770 8.07632449806306e-09
3771 8.07593014684471e-09
3772 8.07562106075466e-09
3773 8.07529332291779e-09
3774 8.0749513742262e-09
3775 8.07456501661363e-09
3776 8.07412892100956e-09
3777 8.07400812874448e-09
3778 8.07356670406989e-09
3779 8.07330380325766e-09
3780 8.0730080398439e-09
3781 8.07250710721519e-09
3782 8.07223177190508e-09
3783 8.07187117146668e-09
3784 8.07144129311155e-09
3785 8.07127520374706e-09
3786 8.07090838605973e-09
3787 8.07053002205294e-09
3788 8.07019784332397e-09
3789 8.06980793299772e-09
3790 8.06945177345142e-09
3791 8.06903699412942e-09
3792 8.06877675785245e-09
3793 8.06828914790003e-09
3794 8.0680759850793e-09
3795 8.06763988947523e-09
3796 8.06730326985416e-09
3797 8.06701550004618e-09
3798 8.06667888042512e-09
3799 8.06622857396633e-09
3800 8.06595412683464e-09
3801 8.06565658706404e-09
3802 8.0653164147293e-09
3803 8.06505706663074e-09
3804 8.06457833846252e-09
3805 8.0642932331898e-09
3806 8.06401168063076e-09
3807 8.06357913774036e-09
3808 8.0632336363351e-09
3809 8.0629405374566e-09
3810 8.06263233954496e-09
3811 8.06215982862568e-09
3812 8.06186939428244e-09
3813 8.06162159250334e-09
3814 8.06117217422297e-09
3815 8.06080446835722e-09
3816 8.06047584234193e-09
3817 8.06004951670047e-09
3818 8.05989319729861e-09
3819 8.05956190674806e-09
3820 8.05898281441841e-09
3821 8.0587971851287e-09
3822 8.05846589457815e-09
3823 8.0580759842519e-09
3824 8.05777666812446e-09
3825 8.05745337117969e-09
3826 8.05701994011088e-09
3827 8.05664424063934e-09
3828 8.05628097566569e-09
3829 8.05608557641335e-09
3830 8.05568678430291e-09
3831 8.05539368542441e-09
3832 8.05506505940912e-09
3833 8.05473554521541e-09
3834 8.05436339845755e-09
3835 8.05399480441338e-09
3836 8.05364752665128e-09
3837 8.05330646613811e-09
3838 8.053031130828e-09
3839 8.05269806392062e-09
3840 8.0523276935196e-09
3841 8.05195377040491e-09
3842 8.05161359807016e-09
3843 8.0513311573327e-09
3844 8.05082844834715e-09
3845 8.05055400121546e-09
3846 8.05032751571844e-09
3847 8.04995092806848e-09
3848 8.04975641699457e-09
3849 8.04918443009228e-09
3850 8.04888689032168e-09
3851 8.04855204705746e-09
3852 8.04829891620784e-09
3853 8.04782196439646e-09
3854 8.04748356841856e-09
3855 8.047296162772e-09
3856 8.04681743460378e-09
3857 8.04660427178305e-09
3858 8.04621347327839e-09
3859 8.04588395908468e-09
3860 8.04559263656301e-09
3861 8.04511568475164e-09
3862 8.0447941641637e-09
3863 8.04448596625207e-09
3864 8.04421063094196e-09
3865 8.04377187080263e-09
3866 8.04334288062591e-09
3867 8.04313149416203e-09
3868 8.04282151989355e-09
3869 8.04252131558769e-09
3870 8.04211452987147e-09
3871 8.04170330326315e-09
3872 8.04148125865822e-09
3873 8.04109134833197e-09
3874 8.04063926551635e-09
3875 8.04044919533453e-09
3876 8.04002553422833e-09
3877 8.03969246732095e-09
3878 8.03939315119351e-09
3879 8.03914623759283e-09
3880 8.03862842957415e-09
3881 8.03842414853762e-09
3882 8.03805733085028e-09
3883 8.03762478795988e-09
3884 8.0373370181519e-09
3885 8.03694089057672e-09
3886 8.0365660792836e-09
3887 8.0364488397322e-09
3888 8.03594879528191e-09
3889 8.03567079543654e-09
3890 8.03538391380698e-09
3891 8.03506328139747e-09
3892 8.03470090460223e-09
3893 8.03433319873648e-09
3894 8.03406408067531e-09
3895 8.03351518641193e-09
3896 8.03328603637965e-09
3897 8.03307020902366e-09
3898 8.03260569171016e-09
3899 8.03231969825902e-09
3900 8.03193866971696e-09
3901 8.03150701500499e-09
3902 8.03125921322589e-09
3903 8.03086219747229e-09
3904 8.03062061294213e-09
3905 8.03024313711376e-09
3906 8.02998290083679e-09
3907 8.0294784154944e-09
3908 8.02911248598548e-09
3909 8.02881494621488e-09
3910 8.02856270354368e-09
3911 8.02813993061591e-09
3912 8.02791255694046e-09
3913 8.02747557315797e-09
3914 8.02719046788525e-09
3915 8.02668953525654e-09
3916 8.02649235964736e-09
3917 8.02600208515969e-09
3918 8.02579247505264e-09
3919 8.02534039223701e-09
3920 8.02514055209258e-09
3921 8.02492294837975e-09
3922 8.02449484638146e-09
3923 8.02404809263635e-09
3924 8.02378075093202e-09
3925 8.02344235495411e-09
3926 8.02305244462787e-09
3927 8.02278954381563e-09
3928 8.02239785713255e-09
3929 8.02196975513425e-09
3930 8.02163224733476e-09
3931 8.02146882250554e-09
3932 8.02116151277232e-09
3933 8.02087019025066e-09
3934 8.02045008185814e-09
3935 8.02012678491337e-09
3936 8.01979371800599e-09
3937 8.01946686834754e-09
3938 8.0190458717766e-09
3939 8.01870836397711e-09
3940 8.01824562302045e-09
3941 8.01794275417933e-09
3942 8.01769850511391e-09
3943 8.01732635835606e-09
3944 8.01715405174264e-09
3945 8.01671085071121e-09
3946 8.01639377101537e-09
3947 8.0160837967469e-09
3948 8.01568056374435e-09
3949 8.01534216776645e-09
3950 8.01491939483867e-09
3951 8.01474531186841e-09
3952 8.01429056451752e-09
3953 8.01391486504599e-09
3954 8.01370880765262e-09
3955 8.01314925524821e-09
3956 8.01308530640199e-09
3957 8.01267496797209e-09
3958 8.01243604797719e-09
3959 8.0119990641947e-09
3960 8.01165533914627e-09
3961 8.01120059179539e-09
3962 8.01090838109531e-09
3963 8.01075650258554e-09
3964 8.01018185114799e-09
3965 8.00991362126524e-09
3966 8.00962940417094e-09
3967 8.00921107213526e-09
3968 8.00889488061785e-09
3969 8.0085591491752e-09
3970 8.00845256776483e-09
3971 8.00795874056348e-09
3972 8.00768429343179e-09
3973 8.00731836392288e-09
3974 8.00693822355925e-09
3975 8.00654120780564e-09
3976 8.0062374507861e-09
3977 8.00592303562553e-09
3978 8.00559796232392e-09
3979 8.00519472932137e-09
3980 8.00485722152189e-09
3981 8.00473021200787e-09
3982 8.00424970748281e-09
3983 8.00380206555928e-09
3984 8.00361821262641e-09
3985 8.00324695404697e-09
3986 8.00290411717697e-09
3987 8.0025808202322e-09
3988 8.00230193220841e-09
3989 8.00199995154571e-09
3990 8.00161270575472e-09
3991 8.00126365163578e-09
3992 8.00082666785329e-09
3993 8.00045185656018e-09
3994 8.00028576719569e-09
3995 7.99993404854149e-09
3996 7.99947574847693e-09
3997 7.99919686045314e-09
3998 7.99887533986521e-09
3999 7.99860178091194e-09
4000 7.99812305274372e-09
4001 7.99785659921781e-09
4002 7.99744004353897e-09
4003 7.99721533439879e-09
4004 7.99682098318044e-09
4005 7.99649413352199e-09
4006 7.99603494527901e-09
4007 7.99577737353729e-09
4008 7.9954576293062e-09
4009 7.99507748894257e-09
4010 7.99477639645829e-09
4011 7.99457300360018e-09
4012 7.99429766829007e-09
4013 7.9938571317939e-09
4014 7.99348942592815e-09
4015 7.99330379663843e-09
4016 7.99293253805899e-09
4017 7.99265897910573e-09
4018 7.99221311353904e-09
4019 7.9919093565195e-09
4020 7.9915531969732e-09
4021 7.99115351668434e-09
4022 7.99087285230371e-09
4023 7.99054689082368e-09
4024 7.99015786867585e-09
4025 7.98995003492564e-09
4026 7.98964538972768e-09
4027 7.98919863598258e-09
4028 7.9888931026062e-09
4029 7.98854404848726e-09
4030 7.98827493042609e-09
4031 7.98776511601318e-09
4032 7.98770471988064e-09
4033 7.98724020256714e-09
4034 7.98693555736918e-09
4035 7.98663091217122e-09
4036 7.9862196855629e-09
4037 7.98595412021541e-09
4038 7.98562638237854e-09
4039 7.98525956469121e-09
4040 7.98491139875068e-09
4041 7.98455435102596e-09
4042 7.98432875370736e-09
4043 7.98407295832249e-09
4044 7.98365729082207e-09
4045 7.98339083729616e-09
4046 7.98295918258418e-09
4047 7.98245824995547e-09
4048 7.98219179642956e-09
4049 7.98193777740153e-09
4050 7.98156740700051e-09
4051 7.98134269786033e-09
4052 7.98104426991131e-09
4053 7.98064903051454e-09
4054 7.98027866011353e-09
4055 7.97996690948821e-09
4056 7.97964272436502e-09
4057 7.9792767948561e-09
4058 7.97902810489859e-09
4059 7.97866128721125e-09
4060 7.97829979859443e-09
4061 7.97795962625969e-09
4062 7.97762922388756e-09
4063 7.97739385660634e-09
4064 7.97699417631748e-09
4065 7.97676413810677e-09
4066 7.97635646421213e-09
4067 7.97599941648741e-09
4068 7.97553134646023e-09
4069 7.97535815166839e-09
4070 7.97512811345769e-09
4071 7.97477550662506e-09
4072 7.97437937904988e-09
4073 7.97413601816288e-09
4074 7.973852689247e-09
4075 7.97338461921981e-09
4076 7.97313237654862e-09
4077 7.97294585908048e-09
4078 7.97244403827335e-09
4079 7.97214472214591e-09
4080 7.97184096512638e-09
4081 7.97151145093267e-09
4082 7.97122545748152e-09
4083 7.97088350878994e-09
4084 7.97062682522665e-09
4085 7.97016674880524e-09
4086 7.96995092144925e-09
4087 7.96958588011876e-09
4088 7.96910715195054e-09
4089 7.96875276876108e-09
4090 7.96865951002701e-09
4091 7.96822341442294e-09
4092 7.96794719093441e-09
4093 7.9675546160729e-09
4094 7.96710519779253e-09
4095 7.96668242486476e-09
4096 7.96653498724709e-09
4097 7.96628096821905e-09
4098 7.96586885343231e-09
4099 7.96555443827174e-09
4100 7.96542298786562e-09
4101 7.96498689226155e-09
4102 7.96458277108059e-09
4103 7.9642363814969e-09
4104 7.96410848380447e-09
4105 7.96345744902283e-09
4106 7.96323451623948e-09
4107 7.9629876026388e-09
4108 7.96261012681043e-09
4109 7.96240406941706e-09
4110 7.96196264474247e-09
4111 7.96164911776032e-09
4112 7.96133026170764e-09
4113 7.9610353864723e-09
4114 7.96070498410018e-09
4115 7.96033283734232e-09
4116 7.96002463943069e-09
4117 7.95975285683426e-09
4118 7.95928656316391e-09
4119 7.95923327245873e-09
4120 7.95875543246893e-09
4121 7.95832910682748e-09
4122 7.95803156705688e-09
4123 7.95776067263887e-09
4124 7.95734589331687e-09
4125 7.95706700529308e-09
4126 7.95664245600847e-09
4127 7.9564861366066e-09
4128 7.95622145943753e-09
4129 7.95567878242309e-09
4130 7.95563082078843e-09
4131 7.95517252072386e-09
4132 7.95465293634834e-09
4133 7.95448773516227e-09
4134 7.95431542854885e-09
4135 7.95374432982499e-09
4136 7.95352939064742e-09
4137 7.95325494351573e-09
4138 7.9529289820357e-09
4139 7.95251953178422e-09
4140 7.95220689298048e-09
4141 7.95188270785729e-09
4142 7.95161980704506e-09
4143 7.95130539188449e-09
4144 7.9508462036415e-09
4145 7.9506747852065e-09
4146 7.95027688127448e-09
4147 7.94988341823455e-09
4148 7.94978838314364e-09
4149 7.94944199355996e-09
4150 7.94898724620907e-09
4151 7.94890642197288e-09
4152 7.9483708503858e-09
4153 7.94809018600517e-09
4154 7.94780863344613e-09
4155 7.94750487642659e-09
4156 7.94704124729151e-09
4157 7.94680854454555e-09
4158 7.94643995050137e-09
4159 7.94612287080554e-09
4160 7.94570365059144e-09
4161 7.94542120985398e-09
4162 7.9450659384861e-09
4163 7.94470889076138e-09
4164 7.94446730623122e-09
4165 7.94419729999163e-09
4166 7.94387311486844e-09
4167 7.94348409272061e-09
4168 7.94321497465944e-09
4169 7.94288368410889e-09
4170 7.94248755653371e-09
4171 7.94209409349378e-09
4172 7.94167842599336e-09
4173 7.94153098837569e-09
4174 7.94114196622786e-09
4175 7.94084353827884e-09
4176 7.94045718066627e-09
4177 7.94014098914886e-09
4178 7.93979015867308e-09
4179 7.93962584566543e-09
4180 7.93928212061701e-09
4181 7.93878918159407e-09
4182 7.93860621683962e-09
4183 7.93813992316927e-09
4184 7.93790988495857e-09
4185 7.93747112481924e-09
4186 7.93739651783198e-09
4187 7.93690269063063e-09
4188 7.93669219234516e-09
4189 7.93621079964169e-09
4190 7.93589460812427e-09
4191 7.93564502998834e-09
4192 7.9354087745287e-09
4193 7.93510324115232e-09
4194 7.93461829573516e-09
4195 7.93452681335793e-09
4196 7.93391841114044e-09
4197 7.9337496572407e-09
4198 7.93348675642847e-09
4199 7.93301069279551e-09
4200 7.93283572164682e-09
4201 7.93244847585584e-09
4202 7.93198395854233e-09
4203 7.93174859126111e-09
4204 7.93151588851515e-09
4205 7.93115173536307e-09
4206 7.93102117313538e-09
4207 7.93068544169273e-09
4208 7.93011789568254e-09
4209 7.92986565301135e-09
4210 7.92953169792554e-09
4211 7.92919863101815e-09
4212 7.92891796663753e-09
4213 7.92844012664773e-09
4214 7.92828824813796e-09
4215 7.92792587134272e-09
4216 7.92763810153474e-09
4217 7.92736898347357e-09
4218 7.92697107954154e-09
4219 7.92665133531045e-09
4220 7.92639287539032e-09
4221 7.92611309918811e-09
4222 7.92575782782023e-09
4223 7.92542920180495e-09
4224 7.92527643511676e-09
4225 7.92476662070385e-09
4226 7.92447796271745e-09
4227 7.92414844852374e-09
4228 7.92392551574039e-09
4229 7.92354537537676e-09
4230 7.92325050014142e-09
4231 7.92271936944644e-09
4232 7.92259680082452e-09
4233 7.92214560618731e-09
4234 7.92187559994773e-09
4235 7.92157006657135e-09
4236 7.92146170880415e-09
4237 7.92104071223321e-09
4238 7.92057086584919e-09
4239 7.92037813113211e-09
4240 7.91996512816695e-09
4241 7.91967913471581e-09
4242 7.9194668600735e-09
4243 7.91915777398344e-09
4244 7.91884691153655e-09
4245 7.91853782544649e-09
4246 7.91828291824004e-09
4247 7.91794985133265e-09
4248 7.91743559602764e-09
4249 7.91719045878381e-09
4250 7.91686805001746e-09
4251 7.91642573716445e-09
4252 7.91644438891126e-09
4253 7.91598520066827e-09
4254 7.91567700275664e-09
4255 7.91529508603617e-09
4256 7.91505438968443e-09
4257 7.91471865824178e-09
4258 7.91440601943805e-09
4259 7.91397525290449e-09
4260 7.9137771891169e-09
4261 7.91327892102345e-09
4262 7.9131234898e-09
4263 7.91282239731572e-09
4264 7.91226995033867e-09
4265 7.91227616758761e-09
4266 7.91180809756042e-09
4267 7.91147947154514e-09
4268 7.91121390619764e-09
4269 7.9107653760957e-09
4270 7.91043675008041e-09
4271 7.91022269908126e-09
4272 7.9098096961161e-09
4273 7.90963916585952e-09
4274 7.90926257820956e-09
4275 7.90895438029793e-09
4276 7.90868437405834e-09
4277 7.90822607399377e-09
4278 7.90789211890797e-09
4279 7.90757592739055e-09
4280 7.90730680932938e-09
4281 7.90701282227246e-09
4282 7.90677656681282e-09
4283 7.90656251581368e-09
4284 7.90606602407706e-09
4285 7.90591592192413e-09
4286 7.90553666973892e-09
4287 7.90527643346195e-09
4288 7.90469290024021e-09
4289 7.90459075972194e-09
4290 7.90410936701846e-09
4291 7.90395127125976e-09
4292 7.90372478576273e-09
4293 7.90331355915441e-09
4294 7.90301513120539e-09
4295 7.90275933582052e-09
4296 7.90224063962341e-09
4297 7.90214293999725e-09
4298 7.90184273569139e-09
4299 7.90143950268885e-09
4300 7.90116860827084e-09
4301 7.90066412292845e-09
4302 7.9004589537135e-09
4303 7.90009568873984e-09
4304 7.89978216175768e-09
4305 7.89941534407035e-09
4306 7.8992146157475e-09
4307 7.89887177887749e-09
4308 7.89849874394122e-09
4309 7.89816478885541e-09
4310 7.8979569551052e-09
4311 7.89761323005678e-09
4312 7.89741161355551e-09
4313 7.89683252122586e-09
4314 7.89670639989026e-09
4315 7.89616461105425e-09
4316 7.89603848971865e-09
4317 7.89573739723437e-09
4318 7.89546028556742e-09
4319 7.89512455412478e-09
4320 7.89472132112223e-09
4321 7.89442999860057e-09
4322 7.89432519354705e-09
4323 7.89371057408061e-09
4324 7.89360221631341e-09
4325 7.89307197379685e-09
4326 7.89276111134996e-09
4327 7.89253462585293e-09
4328 7.89212162288777e-09
4329 7.8919653034859e-09
4330 7.89156739955388e-09
4331 7.89120235822338e-09
4332 7.89091991748592e-09
4333 7.89065612849527e-09
4334 7.89039322768303e-09
4335 7.89003351542306e-09
4336 7.88974396925823e-09
4337 7.88944642948763e-09
4338 7.88909826354711e-09
4339 7.88881049373913e-09
4340 7.88824383590736e-09
4341 7.88811860275018e-09
4342 7.88763365733303e-09
4343 7.88743115265333e-09
4344 7.88720111444263e-09
4345 7.88674370255649e-09
4346 7.88656162598045e-09
4347 7.88617793290314e-09
4348 7.88609177959643e-09
4349 7.88554554986831e-09
4350 7.88533682793968e-09
4351 7.88476217650214e-09
4352 7.8845259210425e-09
4353 7.88432252818438e-09
4354 7.88396636863808e-09
4355 7.88381981919883e-09
4356 7.88337395363214e-09
4357 7.88305065668737e-09
4358 7.88272469520734e-09
4359 7.88252840777659e-09
4360 7.88218024183607e-09
4361 7.88174236987516e-09
4362 7.8816180248964e-09
4363 7.88125209538748e-09
4364 7.88086484959649e-09
4365 7.88070142476727e-09
4366 7.88027954001791e-09
4367 7.88002108009778e-09
4368 7.8795894253858e-09
4369 7.879426888735e-09
4370 7.87905385379872e-09
4371 7.87868259521929e-09
4372 7.87834864013348e-09
4373 7.87805554125498e-09
4374 7.87778375865855e-09
4375 7.87727127971038e-09
4376 7.87712384209271e-09
4377 7.87685117131787e-09
4378 7.87647902456001e-09
4379 7.8763431332618e-09
4380 7.87586351691516e-09
4381 7.87559084614031e-09
4382 7.87524179202137e-09
4383 7.87485454623038e-09
4384 7.87450282757618e-09
4385 7.87427900661442e-09
4386 7.87395482149122e-09
4387 7.87381004840881e-09
4388 7.87332066209956e-09
4389 7.87311016381409e-09
4390 7.87280640679455e-09
4391 7.87236587029838e-09
4392 7.87212606212506e-09
4393 7.87171661187358e-09
4394 7.87146081648871e-09
4395 7.87113663136552e-09
4396 7.87093146215057e-09
4397 7.87049625472491e-09
4398 7.87025289383791e-09
4399 7.86990472789739e-09
4400 7.86950860032221e-09
4401 7.86939757801974e-09
4402 7.8689357252415e-09
4403 7.86856713119732e-09
4404 7.86852183409792e-09
4405 7.86799159158136e-09
4406 7.86788589834941e-09
4407 7.86737963665018e-09
4408 7.86710607769692e-09
4409 7.86676235264849e-09
4410 7.86646126016421e-09
4411 7.86601983548962e-09
4412 7.86584042344884e-09
4413 7.86546117126363e-09
4414 7.86506948458054e-09
4415 7.86493803417443e-09
4416 7.86467335700536e-09
4417 7.864251472256e-09
4418 7.86405163211157e-09
4419 7.86338993918889e-09
4420 7.8633179967369e-09
4421 7.86304177324837e-09
4422 7.8625292943002e-09
4423 7.86227705162901e-09
4424 7.86214116033079e-09
4425 7.86177789535714e-09
4426 7.86149900733335e-09
4427 7.86110376793658e-09
4428 7.8608701770122e-09
4429 7.86036302713455e-09
4430 7.86010723174968e-09
4431 7.85996601138095e-09
4432 7.8595085994948e-09
4433 7.85918086165793e-09
4434 7.85894727073355e-09
4435 7.85862308561036e-09
4436 7.85827491966984e-09
4437 7.85804399328072e-09
4438 7.85760168042771e-09
4439 7.85746934184317e-09
4440 7.85719045381938e-09
4441 7.85675080550163e-09
4442 7.8565509653572e-09
4443 7.85616283138779e-09
4444 7.85589016061294e-09
4445 7.8553830107353e-09
4446 7.85524179036656e-09
4447 7.85482345833088e-09
4448 7.85442644257728e-09
4449 7.85425768867754e-09
4450 7.85401077507686e-09
4451 7.85361731203693e-09
4452 7.85336240483048e-09
4453 7.85295739547109e-09
4454 7.85261811131477e-09
4455 7.85253195800806e-09
4456 7.85199816277782e-09
4457 7.85189779861639e-09
4458 7.85164466776678e-09
4459 7.85115172874384e-09
4460 7.85071030406925e-09
4461 7.85070053410664e-09
4462 7.85027776117886e-09
4463 7.85001219583137e-09
4464 7.84969156342186e-09
4465 7.84935583197921e-09
4466 7.84902365325024e-09
4467 7.84871456716019e-09
4468 7.84838682932332e-09
4469 7.84802445252808e-09
4470 7.84774645268271e-09
4471 7.84748355187048e-09
4472 7.8471371622868e-09
4473 7.84666021047542e-09
4474 7.84658649166659e-09
4475 7.84629605732334e-09
4476 7.84607223636158e-09
4477 7.84570630685266e-09
4478 7.84537590448053e-09
4479 7.8451352081288e-09
4480 7.84473197512625e-09
4481 7.8444228890362e-09
4482 7.84424081246016e-09
4483 7.84382514495974e-09
4484 7.84359510674904e-09
4485 7.84322207181276e-09
4486 7.84290499211693e-09
4487 7.84266873665729e-09
4488 7.84229481354259e-09
4489 7.84189513325373e-09
4490 7.84170861578559e-09
4491 7.84134002174142e-09
4492 7.84116149787906e-09
4493 7.84072096138289e-09
4494 7.84044118518068e-09
4495 7.84025555589096e-09
4496 7.8397883740422e-09
4497 7.83958320482725e-09
4498 7.83917197821893e-09
4499 7.83890374833618e-09
4500 7.83846143548317e-09
4501 7.83822784455879e-09
4502 7.83798448367179e-09
4503 7.83774911639057e-09
4504 7.83740183862847e-09
4505 7.83694531492074e-09
4506 7.83656251002185e-09
4507 7.83633513634641e-09
4508 7.83607134735576e-09
4509 7.8356903188137e-09
4510 7.83551623584344e-09
4511 7.83511211466248e-09
4512 7.8349779997211e-09
4513 7.83460851749851e-09
4514 7.83418929728441e-09
4515 7.83393083736428e-09
4516 7.83368303558518e-09
4517 7.83326825626318e-09
4518 7.83306841611875e-09
4519 7.83283748972963e-09
4520 7.83232767531672e-09
4521 7.83221665301426e-09
4522 7.83184628261324e-09
4523 7.83141551607969e-09
4524 7.83125475578572e-09
4525 7.83103626389448e-09
4526 7.83073250687494e-09
4527 7.83034437290553e-09
4528 7.83000242421394e-09
4529 7.82966935730656e-09
4530 7.82936560028702e-09
4531 7.82903786245015e-09
4532 7.82883002869994e-09
4533 7.82843478930317e-09
4534 7.82824827183504e-09
4535 7.82792053399817e-09
4536 7.82767006768381e-09
4537 7.82730147363964e-09
4538 7.82713183156147e-09
4539 7.82682274547142e-09
4540 7.82639464347312e-09
4541 7.82620901418341e-09
4542 7.82590170445019e-09
4543 7.82552422862182e-09
4544 7.82524445241961e-09
4545 7.82492115547484e-09
4546 7.82470710447569e-09
4547 7.82441134106193e-09
4548 7.8240942613661e-09
4549 7.82386155862014e-09
4550 7.82338371863034e-09
4551 7.82297693291412e-09
4552 7.82283837708064e-09
4553 7.82256037723528e-09
4554 7.82216780237377e-09
4555 7.82193865234149e-09
4556 7.82150255673741e-09
4557 7.82120235243156e-09
4558 7.82096432061508e-09
4559 7.82070319615968e-09
4560 7.82044651259639e-09
4561 7.820113445689e-09
4562 7.81989495379776e-09
4563 7.81952014250464e-09
4564 7.81922615544772e-09
4565 7.81883624512147e-09
4566 7.81856801523872e-09
4567 7.81827314000338e-09
4568 7.81806441807475e-09
4569 7.81761144708071e-09
4570 7.81745956857094e-09
4571 7.8170296902158e-09
4572 7.81677833572303e-09
4573 7.81649145409347e-09
4574 7.81615305811556e-09
4575 7.81580578035346e-09
4576 7.81551889872389e-09
4577 7.81535280935941e-09
4578 7.81494335910793e-09
4579 7.81452413889383e-09
4580 7.81446907183181e-09
4581 7.81412534678338e-09
4582 7.81382247794227e-09
4583 7.81335618427192e-09
4584 7.813134139667e-09
4585 7.81280817818697e-09
4586 7.81250975023795e-09
4587 7.8121376034801e-09
4588 7.81188180809522e-09
4589 7.81172548869336e-09
4590 7.81143860706379e-09
4591 7.81094122714876e-09
4592 7.81063747012922e-09
4593 7.8105326650757e-09
4594 7.81025200069507e-09
4595 7.80990472293297e-09
4596 7.80943754108421e-09
4597 7.8092119437656e-09
4598 7.80903786079534e-09
4599 7.80870479388796e-09
4600 7.80807596356681e-09
4601 7.80798714572484e-09
4602 7.80769049413266e-09
4603 7.8075528264776e-09
4604 7.80700482039265e-09
4605 7.80682896106555e-09
4606 7.80658293564329e-09
4607 7.80607045669512e-09
4608 7.80593989446743e-09
4609 7.80564324287525e-09
4610 7.8053075114326e-09
4611 7.80501974162462e-09
4612 7.80478703887866e-09
4613 7.80447617643176e-09
4614 7.80412534595598e-09
4615 7.80391218313525e-09
4616 7.80351516738165e-09
4617 7.80324960203416e-09
4618 7.80284725721003e-09
4619 7.80256481647257e-09
4620 7.80234188368922e-09
4621 7.8020656602007e-09
4622 7.80164111091608e-09
4623 7.80135867017862e-09
4624 7.80083819762467e-09
4625 7.8006863191149e-09
4626 7.80047759718627e-09
4627 7.80013564849469e-09
4628 7.79978037712681e-09
4629 7.79959385965867e-09
4630 7.79910891424151e-09
4631 7.79889663959921e-09
4632 7.79843833953464e-09
4633 7.79837883158052e-09
4634 7.79801023753635e-09
4635 7.79764519620585e-09
4636 7.79722064692123e-09
4637 7.79711140097561e-09
4638 7.79668063444205e-09
4639 7.79645148440977e-09
4640 7.79605802136984e-09
4641 7.79582087773179e-09
4642 7.79533770867147e-09
4643 7.79510855863919e-09
4644 7.79497355551939e-09
4645 7.79458453337156e-09
4646 7.79413511509119e-09
4647 7.79390862959417e-09
4648 7.7935355946579e-09
4649 7.79321407406997e-09
4650 7.79306752463071e-09
4651 7.79277797846589e-09
4652 7.7924449115585e-09
4653 7.79220332702835e-09
4654 7.79179831766896e-09
4655 7.79140574280746e-09
4656 7.79109132764688e-09
4657 7.79083908497569e-09
4658 7.79055753241664e-09
4659 7.79028663799863e-09
4660 7.79006015250161e-09
4661 7.78981945614987e-09
4662 7.78936826151266e-09
4663 7.78905917542261e-09
4664 7.78883890717452e-09
4665 7.7885404792255e-09
4666 7.7880670801278e-09
4667 7.78781394927819e-09
4668 7.78763897812951e-09
4669 7.78742492713036e-09
4670 7.78703856951779e-09
4671 7.78664421829944e-09
4672 7.7863493430641e-09
4673 7.78602693429775e-09
4674 7.78574893445239e-09
4675 7.78543007839971e-09
4676 7.785172506658e-09
4677 7.78492204034364e-09
4678 7.78453923544475e-09
4679 7.7842932100225e-09
4680 7.78390063516099e-09
4681 7.78360842446091e-09
4682 7.78332065465293e-09
4683 7.78299380499448e-09
4684 7.78278153035217e-09
4685 7.78242092991377e-09
4686 7.78223796515931e-09
4687 7.78193243178293e-09
4688 7.78152564606671e-09
4689 7.7812005727651e-09
4690 7.78104425336323e-09
4691 7.78059039419077e-09
4692 7.78045183835729e-09
4693 7.78016939761983e-09
4694 7.77983810706928e-09
4695 7.77926523198857e-09
4696 7.77928743644907e-09
4697 7.77874475943463e-09
4698 7.7785404783981e-09
4699 7.77843300880932e-09
4700 7.77808839558247e-09
4701 7.77762831916107e-09
4702 7.77755726488749e-09
4703 7.77694975084842e-09
4704 7.77675346341766e-09
4705 7.77649855621121e-09
4706 7.77626318892999e-09
4707 7.77579156618913e-09
4708 7.77553932351793e-09
4709 7.7752853044899e-09
4710 7.77505348992236e-09
4711 7.77466446777453e-09
4712 7.7744557458459e-09
4713 7.77413244890113e-09
4714 7.77400899210079e-09
4715 7.77363418080768e-09
4716 7.77340325441855e-09
4717 7.77295827703028e-09
4718 7.77280817487735e-09
4719 7.77234276938543e-09
4720 7.77200348522911e-09
4721 7.77180808597677e-09
4722 7.77149544717304e-09
4723 7.77121389461399e-09
4724 7.77073339008894e-09
4725 7.77067832302691e-09
4726 7.7704251921773e-09
4727 7.7701187706225e-09
4728 7.769726195761e-09
4729 7.76944819591563e-09
4730 7.76921194045599e-09
4731 7.76880249020451e-09
4732 7.76866659890629e-09
4733 7.76824382597852e-09
4734 7.76787167922066e-09
4735 7.76769581989356e-09
4736 7.76738406926825e-09
4737 7.76707143046451e-09
4738 7.76679431879757e-09
4739 7.76626496445942e-09
4740 7.76627206988678e-09
4741 7.76595143747727e-09
4742 7.76557218529206e-09
4743 7.76533326529716e-09
4744 7.76513608968799e-09
4745 7.76486963616208e-09
4746 7.76454101014679e-09
4747 7.76415109982054e-09
4748 7.76392283796667e-09
4749 7.76357467202615e-09
4750 7.76327269136345e-09
4751 7.76303199501172e-09
4752 7.76285258297094e-09
4753 7.76235431487748e-09
4754 7.76215358655463e-09
4755 7.76197239815701e-09
4756 7.76149189363196e-09
4757 7.76146968917146e-09
4758 7.76079378539407e-09
4759 7.76064812413324e-09
4760 7.76031505722585e-09
4761 7.76006814362518e-09
4762 7.75970310229468e-09
4763 7.75949882125815e-09
4764 7.75919506423861e-09
4765 7.7588611091528e-09
4766 7.75862307733632e-09
4767 7.75831665578153e-09
4768 7.75805464314772e-09
4769 7.75767716731934e-09
4770 7.75742314829131e-09
4771 7.75698349997356e-09
4772 7.75683339782063e-09
4773 7.75647812645275e-09
4774 7.75624364734995e-09
4775 7.75595054847145e-09
4776 7.75564235055981e-09
4777 7.75533948171869e-09
4778 7.75527730922931e-09
4779 7.75481545645107e-09
4780 7.75454722656832e-09
4781 7.7543029775029e-09
4782 7.75392194896085e-09
4783 7.75357111848507e-09
4784 7.75333752756069e-09
4785 7.75302755329221e-09
4786 7.75274511255475e-09
4787 7.75230013516648e-09
4788 7.75234543226588e-09
4789 7.7517148255879e-09
4790 7.75152297904924e-09
4791 7.75133734975952e-09
4792 7.75087549698128e-09
4793 7.75067476865843e-09
4794 7.75029995736531e-09
4795 7.74995712049531e-09
4796 7.74988695440015e-09
4797 7.74944197701188e-09
4798 7.74903075040356e-09
4799 7.74881936393967e-09
4800 7.74867903174936e-09
4801 7.74817632276381e-09
4802 7.74803421421666e-09
4803 7.74786634849534e-09
4804 7.747427588356e-09
4805 7.74697106464828e-09
4806 7.74678010628804e-09
4807 7.74647812562534e-09
4808 7.74614328236112e-09
4809 7.74623121202467e-09
4810 7.74559971716826e-09
4811 7.74517872059732e-09
4812 7.74504904654805e-09
4813 7.7448216728726e-09
4814 7.74442288076216e-09
4815 7.7441137946721e-09
4816 7.74393704716658e-09
4817 7.74358088762028e-09
4818 7.7432478207129e-09
4819 7.74312969298308e-09
4820 7.74266872838325e-09
4821 7.74227260080806e-09
4822 7.74191111219125e-09
4823 7.74182673524137e-09
4824 7.74155495264495e-09
4825 7.74133290804002e-09
4826 7.74088793065175e-09
4827 7.74064279340791e-09
4828 7.74027153482848e-09
4829 7.74002639758464e-09
4830 7.73981412294233e-09
4831 7.73952280042067e-09
4832 7.73934782927199e-09
4833 7.73896591255152e-09
4834 7.73857333769001e-09
4835 7.73816388743853e-09
4836 7.7379143093026e-09
4837 7.73755903793472e-09
4838 7.73732988790243e-09
4839 7.73720199021e-09
4840 7.73679253995851e-09
4841 7.73633423989395e-09
4842 7.73609976079115e-09
4843 7.73604114101545e-09
4844 7.73543984422531e-09
4845 7.73524799768666e-09
4846 7.73497355055497e-09
4847 7.73473551873849e-09
4848 7.73449837510043e-09
4849 7.73427188960341e-09
4850 7.73380559593306e-09
4851 7.73353026062296e-09
4852 7.7332709125244e-09
4853 7.73309860591098e-09
4854 7.73249553276401e-09
4855 7.73230635076061e-09
4856 7.73194486214379e-09
4857 7.73177610824405e-09
4858 7.731395079702e-09
4859 7.73125297115484e-09
4860 7.7309172397122e-09
4861 7.73057617919903e-09
4862 7.7303266010631e-09
4863 7.7300112977241e-09
4864 7.72969510620669e-09
4865 7.72938690829506e-09
4866 7.72899522161197e-09
4867 7.7288841993095e-09
4868 7.72837971396712e-09
4869 7.72833264051087e-09
4870 7.72778818713959e-09
4871 7.72762120959669e-09
4872 7.72713271146586e-09
4873 7.72704034091021e-09
4874 7.72667618775813e-09
4875 7.72644259683375e-09
4876 7.72615926791786e-09
4877 7.72577024577004e-09
4878 7.72561481454659e-09
4879 7.72533503834438e-09
4880 7.72517516622884e-09
4881 7.72463870646334e-09
4882 7.72439712193318e-09
4883 7.72411912208781e-09
4884 7.7237443107947e-09
4885 7.7235009499077e-09
4886 7.72310571051094e-09
4887 7.72269448390261e-09
4888 7.72256925074544e-09
4889 7.72206121268937e-09
4890 7.72200792198419e-09
4891 7.72165087425947e-09
4892 7.72125918757638e-09
4893 7.72089681078114e-09
4894 7.72087727085591e-09
4895 7.72048114328072e-09
4896 7.72021468975481e-09
4897 7.71984165481854e-09
4898 7.71958674761208e-09
4899 7.7192900960199e-09
4900 7.71897745721617e-09
4901 7.71883534866902e-09
4902 7.71864350213036e-09
4903 7.71814878675059e-09
4904 7.71787345144048e-09
4905 7.71775088281856e-09
4906 7.71735830795706e-09
4907 7.71695773948977e-09
4908 7.71669572685596e-09
4909 7.71655539466565e-09
4910 7.7163457845586e-09
4911 7.71588837267245e-09
4912 7.7156290245739e-09
4913 7.71535990651273e-09
4914 7.71517072450933e-09
4915 7.71483410488827e-09
4916 7.71426122980756e-09
4917 7.71414931932668e-09
4918 7.7139699072859e-09
4919 7.71368213747792e-09
4920 7.71321406745074e-09
4921 7.712973371099e-09
4922 7.71258257259433e-09
4923 7.71236674523834e-09
4924 7.71209851535559e-09
4925 7.71184893721966e-09
4926 7.71156472012535e-09
4927 7.7112378704669e-09
4928 7.71073249694609e-09
4929 7.71063835003361e-09
4930 7.71031682944567e-09
4931 7.71011343658756e-09
4932 7.7097768169665e-09
4933 7.70966401830719e-09
4934 7.70931496418825e-09
4935 7.70894192925198e-09
4936 7.70884067691213e-09
4937 7.70841346309226e-09
4938 7.70812480510585e-09
4939 7.70793207038878e-09
4940 7.70753505463517e-09
4941 7.70730945731657e-09
4942 7.70702168750859e-09
4943 7.70662644811182e-09
4944 7.70627295310078e-09
4945 7.70612373912627e-09
4946 7.70565478092067e-09
4947 7.70537855743214e-09
4948 7.70512009751201e-09
4949 7.70480035328092e-09
4950 7.70482522227667e-09
4951 7.7043083024364e-09
4952 7.70417685203029e-09
4953 7.70378427716878e-09
4954 7.70348851375502e-09
4955 7.70318209220022e-09
4956 7.70284547257916e-09
4957 7.7025452682733e-09
4958 7.7023019073863e-09
4959 7.70184627185699e-09
4960 7.70179742204391e-09
4961 7.70132047023253e-09
4962 7.7010486876361e-09
4963 7.70089503276949e-09
4964 7.70056285404053e-09
4965 7.70026353791309e-09
4966 7.70003349970239e-09
4967 7.69968888647554e-09
4968 7.69934338507028e-09
4969 7.69913111042797e-09
4970 7.69880870166162e-09
4971 7.69838237602016e-09
4972 7.69824470836511e-09
4973 7.69786367982306e-09
4974 7.69764696428865e-09
4975 7.69737074080012e-09
4976 7.69723662585875e-09
4977 7.69675878586895e-09
4978 7.69633778929801e-09
4979 7.6962516359913e-09
4980 7.69606778305842e-09
4981 7.69569208358689e-09
4982 7.69543184730992e-09
4983 7.69502328523686e-09
4984 7.6948358795903e-09
4985 7.69459429506014e-09
4986 7.69416796941869e-09
4987 7.69388641685964e-09
4988 7.69365726682736e-09
4989 7.69339258965829e-09
4990 7.69309504988769e-09
4991 7.69280550372287e-09
4992 7.69254615562431e-09
4993 7.69219177243485e-09
4994 7.69195729333205e-09
4995 7.69157004754106e-09
4996 7.6914092872471e-09
4997 7.69096519803725e-09
4998 7.6908550639132e-09
4999 7.69048646986903e-09
};
\addlegendentry{Test}

\nextgroupplot[
title={Leaky/Leaky $\hy$},
ymin=9.21401811266189e-09, ymax=1e-05,
]
\addplot [semithick, black, dashed]
table {%
0 0.00970589889993425
1 0.00180805058579426
2 0.00053494363282698
3 0.000153862731784102
4 0.000114152527980877
5 6.76707743036786e-05
6 3.3392439818158e-05
7 2.08889225408484e-05
8 1.82443546832474e-05
9 1.7598902446494e-05
10 1.71766906896238e-05
11 1.67210628037822e-05
12 1.61978157589857e-05
13 1.55729538005573e-05
14 1.48231377344246e-05
15 1.39422236418767e-05
16 1.29232076414922e-05
17 1.17589548138142e-05
18 1.0481614918092e-05
19 9.14011275261828e-06
20 7.80683961458806e-06
21 6.56800483672271e-06
22 5.50973685257361e-06
23 4.6554938270873e-06
24 3.99998192149553e-06
25 3.51565744910687e-06
26 3.15779680249051e-06
27 2.89185434403549e-06
28 2.69721916191656e-06
29 2.54623322447145e-06
30 2.41494825590394e-06
31 2.30035284144492e-06
32 2.1945408087447e-06
33 2.08836155395176e-06
34 1.97569650207896e-06
35 1.86930410387731e-06
36 1.75704304411894e-06
37 1.62137115319894e-06
38 1.50653931439138e-06
39 1.4289809606538e-06
40 1.35234770773351e-06
41 1.27698938501553e-06
42 1.20303228742813e-06
43 1.13050485111543e-06
44 1.05764440561984e-06
45 9.80331805909174e-07
46 9.11678865850973e-07
47 8.50644858161331e-07
48 7.96285356960524e-07
49 7.48363438708211e-07
50 7.05598038798882e-07
51 6.68760273459412e-07
52 6.36771871553421e-07
53 6.10350021375439e-07
54 5.89024306551167e-07
55 5.71628922534373e-07
56 5.57775213716027e-07
57 5.46799955687405e-07
58 5.38066000933313e-07
59 5.30859815043527e-07
60 5.25339434028993e-07
61 5.20358583402825e-07
62 5.1613411885576e-07
63 5.12229194661984e-07
64 5.0899610356403e-07
65 5.0570308454212e-07
66 5.02677079942337e-07
67 4.99604868305248e-07
68 4.96581587291267e-07
69 4.94064953813655e-07
70 4.91379027705463e-07
71 4.88916685000618e-07
72 4.86269379017301e-07
73 4.83705272296575e-07
74 4.81301906925857e-07
75 4.78923055737823e-07
76 4.76771956954281e-07
77 4.74490196225474e-07
78 4.7209232204537e-07
79 4.69964070610729e-07
80 4.67909233655028e-07
81 4.65755283825864e-07
82 4.63725108393653e-07
83 4.61327827576241e-07
84 4.59016333643092e-07
85 4.56934578874524e-07
86 4.54852877641798e-07
87 4.5259007699272e-07
88 4.50563996480469e-07
89 4.47973682419445e-07
90 4.44323234107813e-07
91 4.41500187370991e-07
92 4.38757546023183e-07
93 4.36352241578675e-07
94 4.33841257111212e-07
95 4.31359935619824e-07
96 4.29410778673756e-07
97 4.27173391368996e-07
98 4.24594970448666e-07
99 4.22465475939049e-07
100 4.19772324937284e-07
101 4.17588774165978e-07
102 4.15384662540319e-07
103 4.13058784264564e-07
104 4.10764400559316e-07
105 4.08554054951438e-07
106 4.06493387924201e-07
107 4.04274806561133e-07
108 4.01960743991125e-07
109 3.99679837734368e-07
110 3.9797386092566e-07
111 3.95735122706853e-07
112 3.93685214131523e-07
113 3.91607633016378e-07
114 3.89608959299181e-07
115 3.87760845089957e-07
116 3.85941880038487e-07
117 3.84247711247454e-07
118 3.82480024157417e-07
119 3.80933477140744e-07
120 3.7925515193038e-07
121 3.77692546605246e-07
122 3.76233337950005e-07
123 3.74744372216185e-07
124 3.73312708831008e-07
125 3.71707252742937e-07
126 3.70186882594936e-07
127 3.68789254855173e-07
128 3.6731110249022e-07
129 3.66007227299292e-07
130 3.64561002166752e-07
131 3.63021223397375e-07
132 3.61708902143043e-07
133 3.60228137103569e-07
134 3.58825112293459e-07
135 3.57627983460951e-07
136 3.56430741685898e-07
137 3.551274261957e-07
138 3.5418768657447e-07
139 3.5289912323222e-07
140 3.51495240697552e-07
141 3.50400543061902e-07
142 3.49252910334386e-07
143 3.48080966431574e-07
144 3.47082895499895e-07
145 3.45710785022213e-07
146 3.4491555486138e-07
147 3.43782772988632e-07
148 3.42652789360187e-07
149 3.41559308456141e-07
150 3.40544409895394e-07
151 3.39534924280294e-07
152 3.38590341951495e-07
153 3.37530742714698e-07
154 3.3639920771833e-07
155 3.35596417804673e-07
156 3.34473086150311e-07
157 3.332214469145e-07
158 3.31977728699862e-07
159 3.30650984280645e-07
160 3.29617379496661e-07
161 3.28326322135197e-07
162 3.27286427394213e-07
163 3.26109184591417e-07
164 3.25135797070608e-07
165 3.24078592137234e-07
166 3.23048046272589e-07
167 3.22166835482029e-07
168 3.21330634903028e-07
169 3.20210881185723e-07
170 3.1935232890401e-07
171 3.18348280593739e-07
172 3.1744231991393e-07
173 3.16485413780931e-07
174 3.15620722195931e-07
175 3.14707105463441e-07
176 3.13838885126039e-07
177 3.12879023859125e-07
178 3.11999759612824e-07
179 3.11089174006796e-07
180 3.10164426204906e-07
181 3.09349018608351e-07
182 3.08414381207989e-07
183 3.07473691092852e-07
184 3.06683962545407e-07
185 3.05935701624627e-07
186 3.05010748700596e-07
187 3.04028408447898e-07
188 3.03240499738955e-07
189 3.02342832384284e-07
190 3.01407129328801e-07
191 3.00453415591484e-07
192 2.99686023652157e-07
193 2.98829472519913e-07
194 2.97912213225615e-07
195 2.96742717518228e-07
196 2.95893517519019e-07
197 2.95037145874311e-07
198 2.94176043862393e-07
199 2.93282184920862e-07
200 2.92360600613328e-07
201 2.91372081416164e-07
202 2.90375379179864e-07
203 2.89385764110861e-07
204 2.88510931811814e-07
205 2.87635400978203e-07
206 2.86667393285533e-07
207 2.85718895230325e-07
208 2.84746593365526e-07
209 2.83755679178377e-07
210 2.82734113056904e-07
211 2.81715671999194e-07
212 2.80774115085869e-07
213 2.79582170701786e-07
214 2.78659138112758e-07
215 2.77652584527388e-07
216 2.76562480944875e-07
217 2.75399952169408e-07
218 2.74263207911574e-07
219 2.73208884168596e-07
220 2.72055773210766e-07
221 2.71010168850339e-07
222 2.69832017686156e-07
223 2.68645028732806e-07
224 2.67517912149628e-07
225 2.66373184805602e-07
226 2.65158519976438e-07
227 2.64083070879551e-07
228 2.62873957748155e-07
229 2.61699451248987e-07
230 2.60475515016623e-07
231 2.59257652292177e-07
232 2.5800912491647e-07
233 2.56738647133048e-07
234 2.5566933649479e-07
235 2.54463015128437e-07
236 2.5316156178512e-07
237 2.52030603979492e-07
238 2.50700988569719e-07
239 2.49562748960486e-07
240 2.48265386602498e-07
241 2.47048641147885e-07
242 2.45614053064003e-07
243 2.44342910470152e-07
244 2.4304139973097e-07
245 2.4183575610337e-07
246 2.40425370680697e-07
247 2.39101974386813e-07
248 2.37738637132168e-07
249 2.36381497659366e-07
250 2.35254081243141e-07
251 2.33659147772514e-07
252 2.32216949798136e-07
253 2.30906252286722e-07
254 2.29588458570795e-07
255 2.28026853362984e-07
256 2.26736650087211e-07
257 2.25267492070635e-07
258 2.23865500227127e-07
259 2.22328839725172e-07
260 2.2086532290011e-07
261 2.19622997039259e-07
262 2.18178978134809e-07
263 2.16562468192905e-07
264 2.15214916392981e-07
265 2.13642353396715e-07
266 2.10706997659571e-07
267 2.09527698688206e-07
268 2.0804054397594e-07
269 2.06603045741716e-07
270 2.05090186807588e-07
271 2.03426924184846e-07
272 2.01900563086888e-07
273 2.0026916530913e-07
274 1.98728370757628e-07
275 1.97143359210727e-07
276 1.9552304900472e-07
277 1.9391821720216e-07
278 1.92256614817587e-07
279 1.90688414898332e-07
280 1.8906991232015e-07
281 1.87534715123228e-07
282 1.8605883297429e-07
283 1.84365978888934e-07
284 1.82775365406229e-07
285 1.81314208655081e-07
286 1.79653099591448e-07
287 1.78117566788316e-07
288 1.7659953751803e-07
289 1.75032632857786e-07
290 1.73424296875169e-07
291 1.71749139413535e-07
292 1.70333953146695e-07
293 1.68610336031882e-07
294 1.67181499927338e-07
295 1.65632036291719e-07
296 1.6417937655433e-07
297 1.62506396328865e-07
298 1.61135499430642e-07
299 1.59483156234153e-07
300 1.58103971587664e-07
301 1.56518108622272e-07
302 1.55275500858476e-07
303 1.53759553196409e-07
304 1.52488830935216e-07
305 1.50940648304676e-07
306 1.4975186082955e-07
307 1.48347085526801e-07
308 1.46937822293136e-07
309 1.45595366286777e-07
310 1.44430607395662e-07
311 1.43088239875855e-07
312 1.41728224267279e-07
313 1.40345865033531e-07
314 1.39162570223394e-07
315 1.37946619868767e-07
316 1.36944757853641e-07
317 1.35631133708625e-07
318 1.34424402884914e-07
319 1.33327178768461e-07
320 1.32308650067525e-07
321 1.31206217794944e-07
322 1.30153199310978e-07
323 1.29013822490531e-07
324 1.27987969931631e-07
325 1.26935693110841e-07
326 1.26008437597136e-07
327 1.24980779136585e-07
328 1.2396903216283e-07
329 1.2298142208822e-07
330 1.21915663043204e-07
331 1.21027473551916e-07
332 1.20291009401363e-07
333 1.19476482187242e-07
334 1.18523257349956e-07
335 1.17798107964528e-07
336 1.16866300307716e-07
337 1.16038727878909e-07
338 1.1513289922549e-07
339 1.1435293050388e-07
340 1.13472959579752e-07
341 1.12756211923415e-07
342 1.11856567349911e-07
343 1.1112446830408e-07
344 1.10302843044252e-07
345 1.09604508459427e-07
346 1.0878063277131e-07
347 1.08080433121938e-07
348 1.07391274227098e-07
349 1.06601736842293e-07
350 1.05919710180036e-07
351 1.05364630620386e-07
352 1.04587742542517e-07
353 1.04062058566612e-07
354 1.03363381470434e-07
355 1.02831563674854e-07
356 1.02264846177036e-07
357 1.01657765209318e-07
358 1.01057458434806e-07
359 1.00622192580335e-07
360 1.00020378833499e-07
361 9.94859156677741e-08
362 9.88872357168269e-08
363 9.8401832949957e-08
364 9.80091213889089e-08
365 9.75125097553153e-08
366 9.70419437105185e-08
367 9.65661880636404e-08
368 9.61802824774871e-08
369 9.57830945140792e-08
370 9.53370379774832e-08
371 9.49670479548637e-08
372 9.45182376139719e-08
373 9.41611688638488e-08
374 9.37345073395157e-08
375 9.34255310567345e-08
376 9.30151580589289e-08
377 9.27025564227968e-08
378 9.23203986991084e-08
379 9.19943147223989e-08
380 9.16338995562427e-08
381 9.13577880279348e-08
382 9.09980395009669e-08
383 9.08167766762169e-08
384 9.05132136841402e-08
385 9.02282399377619e-08
386 8.98868770358163e-08
387 8.97514215085771e-08
388 8.94768087285058e-08
389 8.92521382467315e-08
390 8.88167266142048e-08
391 8.86400943369558e-08
392 8.83957845916328e-08
393 8.8198795153982e-08
394 8.80021500373473e-08
395 8.7619338907885e-08
396 8.75654684602267e-08
397 8.72160141360823e-08
398 8.708155998427e-08
399 8.66535380086297e-08
400 8.6648711641768e-08
401 8.63230351151323e-08
402 8.62196419038064e-08
403 8.59700189876911e-08
404 8.57085581214001e-08
405 8.56199245982125e-08
406 8.54258150231324e-08
407 8.5152232071728e-08
408 8.51114567406164e-08
409 8.4742358114287e-08
410 8.46988313538688e-08
411 8.44133877722086e-08
412 8.43485361325946e-08
413 8.41428343414385e-08
414 8.39144065905017e-08
415 8.38375027778504e-08
416 8.36565841075831e-08
417 8.3576623980175e-08
418 8.35158031060601e-08
419 8.32817607849812e-08
420 8.32134802708495e-08
421 8.29700624693786e-08
422 8.29621561502236e-08
423 8.24498850660049e-08
424 8.26580895116891e-08
425 8.24223596769258e-08
426 8.21986277945719e-08
427 8.21568602757949e-08
428 8.19910195666296e-08
429 8.17905892378512e-08
430 8.01686310598626e-08
431 8.04475111939595e-08
432 8.04594778851531e-08
433 8.01703875903392e-08
434 8.01445371179099e-08
435 7.98045946117654e-08
436 7.98175400262835e-08
437 7.96291442335395e-08
438 7.94358032214504e-08
439 7.93642707379583e-08
440 7.9185353175415e-08
441 7.90230892873112e-08
442 7.89110299028728e-08
443 7.8788322786405e-08
444 7.86060871815764e-08
445 7.85434604333446e-08
446 7.83391853445004e-08
447 7.82764915201462e-08
448 7.81137885947913e-08
449 7.79458635853381e-08
450 7.79017212928501e-08
451 7.77054176812619e-08
452 7.7718309820618e-08
453 7.74574940911599e-08
454 7.74245431474085e-08
455 7.72229545815684e-08
456 7.71986419709769e-08
457 7.69591549278914e-08
458 7.69834897615951e-08
459 7.67592703865283e-08
460 7.67418556533173e-08
461 7.65327644645097e-08
462 7.65142343848169e-08
463 7.63308579854183e-08
464 7.63103237857443e-08
465 7.61069157237948e-08
466 7.6081075414347e-08
467 7.5868917612798e-08
468 7.58814860128965e-08
469 7.56695027028798e-08
470 7.56700292079415e-08
471 7.54762571149037e-08
472 7.55289164922068e-08
473 7.53654523033553e-08
474 7.52029752453254e-08
475 7.52240584085406e-08
476 7.50213869493699e-08
477 7.50239565225908e-08
478 7.48517500346324e-08
479 7.47825479865405e-08
480 7.46191178511157e-08
481 7.44593824395512e-08
482 7.4505974568817e-08
483 7.42936726789711e-08
484 7.41189727777591e-08
485 7.40767599389791e-08
486 7.41111108684134e-08
487 7.39998440704159e-08
488 7.38070911967537e-08
489 7.37272064263994e-08
490 7.36981191338249e-08
491 7.36543510664323e-08
492 7.34680324987469e-08
493 7.34329971143755e-08
494 7.33736256359752e-08
495 7.33079067776199e-08
496 7.31875588255715e-08
497 7.32191119074965e-08
498 7.30541291293907e-08
499 7.29807664894278e-08
500 7.29627008519973e-08
501 7.29203128992939e-08
502 7.28145584980577e-08
503 7.28129898881491e-08
504 7.25987676157835e-08
505 7.26196995757267e-08
506 7.25275507551526e-08
507 7.24577761330725e-08
508 7.23331181640141e-08
509 7.22818026619443e-08
510 7.22270302626615e-08
511 7.21534494816378e-08
512 7.22452278900665e-08
513 7.19776541431294e-08
514 7.19541624423226e-08
515 7.19564919309246e-08
516 7.18359372737787e-08
517 7.17523348450211e-08
518 7.16904944761954e-08
519 7.16613938041011e-08
520 7.15020248556186e-08
521 7.13879331819545e-08
522 7.13150458953038e-08
523 7.12896584462719e-08
524 7.12264294673304e-08
525 7.117396892653e-08
526 7.10329861721704e-08
527 7.10939856141124e-08
528 7.0999948839745e-08
529 7.08725372602359e-08
530 7.09098652074136e-08
531 7.08470134804795e-08
532 7.07288346748491e-08
533 7.0762707771399e-08
534 7.06282466234054e-08
535 7.05860463170449e-08
536 7.0381364241312e-08
537 7.04992162183782e-08
538 7.03500237160259e-08
539 7.03570062303704e-08
540 7.01972436714104e-08
541 7.03213008570458e-08
542 7.01616861986665e-08
543 7.00654738712281e-08
544 7.00203281431833e-08
545 7.00190631772735e-08
546 6.99786024012106e-08
547 6.99044401626381e-08
548 6.97821301194335e-08
549 6.9812850459483e-08
550 6.96325664182318e-08
551 6.97237683318086e-08
552 6.96782237996629e-08
553 6.95456427786123e-08
554 6.95435396556032e-08
555 6.95263332952667e-08
556 6.93800421762703e-08
557 6.94501544726833e-08
558 6.93263874158578e-08
559 6.92905434227509e-08
560 6.92543211182617e-08
561 6.91206418068635e-08
562 6.9205453736787e-08
563 6.9105198475139e-08
564 6.91211410350778e-08
565 6.89648297262924e-08
566 6.87370333753989e-08
567 6.88212815358469e-08
568 6.87044843761253e-08
569 6.87062704500807e-08
570 6.86471668855937e-08
571 6.85599604417497e-08
572 6.85878096950976e-08
573 6.84119412737161e-08
574 6.83952206390615e-08
575 6.83635581426589e-08
576 6.8297033308129e-08
577 6.82498913757712e-08
578 6.81809522742327e-08
579 6.82162911023188e-08
580 6.81068230161674e-08
581 6.80535874681976e-08
582 6.80914935449195e-08
583 6.798267438346e-08
584 6.79730591326422e-08
585 6.79156719105567e-08
586 6.78491675698556e-08
587 6.78792947701457e-08
588 6.78033167658132e-08
589 6.77831752022851e-08
590 6.77460387050743e-08
591 6.76622645661151e-08
592 6.76676529267617e-08
593 6.75813300690109e-08
594 6.75473856031061e-08
595 6.75151984868005e-08
596 6.74718725601764e-08
597 6.74888686773567e-08
598 6.73758148375114e-08
599 6.7358173382992e-08
600 6.73599769001054e-08
601 6.72863610746699e-08
602 6.71397729810064e-08
603 6.72937254175832e-08
604 6.72156647800737e-08
605 6.71584187312835e-08
606 6.70856813156373e-08
607 6.70329563354777e-08
608 6.69801189967512e-08
609 6.70082923412974e-08
610 6.68632406080061e-08
611 6.68882090304468e-08
612 6.6918518476955e-08
613 6.6760964644752e-08
614 6.67138460661576e-08
615 6.66640332489532e-08
616 6.66917181537663e-08
617 6.65570751927635e-08
618 6.66009107546106e-08
619 6.65326858160231e-08
620 6.64788396003146e-08
621 6.64734327386896e-08
622 6.64639534215805e-08
623 6.63801962579314e-08
624 6.63584835001174e-08
625 6.63790395496466e-08
626 6.62504426085508e-08
627 6.6301942633018e-08
628 6.62037862950715e-08
629 6.61738796776667e-08
630 6.61707722047211e-08
631 6.61656287797019e-08
632 6.60504218714664e-08
633 6.60999970616416e-08
634 6.59529017794647e-08
635 6.5991073256999e-08
636 6.60268062675495e-08
637 6.59800936677612e-08
638 6.58299739306756e-08
639 6.59855076339255e-08
640 6.57637292855728e-08
641 6.58338429657679e-08
642 6.57838258797394e-08
643 6.57369215577575e-08
644 6.57267310006926e-08
645 6.57197598750159e-08
646 6.56464274626511e-08
647 6.56166033494721e-08
648 6.5589264786281e-08
649 6.55337399637634e-08
650 6.54597351998909e-08
651 6.55190819918161e-08
652 6.53671711343673e-08
653 6.54341834034344e-08
654 6.53374475880764e-08
655 6.54541347790971e-08
656 6.53283041416319e-08
657 6.53386854165827e-08
658 6.52509003300494e-08
659 6.52489100101406e-08
660 6.5194519571099e-08
661 6.51979950572201e-08
662 6.5134074333173e-08
663 6.51222841143717e-08
664 6.50258949737026e-08
665 6.50652132490048e-08
666 6.50030014406333e-08
667 6.49301066106833e-08
668 6.50541408777627e-08
669 6.48858130805063e-08
670 6.47985223496406e-08
671 6.48681474477719e-08
672 6.48651292689539e-08
673 6.47980316259566e-08
674 6.47978750631939e-08
675 6.47133112139375e-08
676 6.46949495408666e-08
677 6.46848629417551e-08
678 6.46224547524721e-08
679 6.46498994936184e-08
680 6.45798058271918e-08
681 6.45480129402021e-08
682 6.45158792316192e-08
683 6.45052920360545e-08
684 6.43831852944476e-08
685 6.44832248672955e-08
686 6.43528200672261e-08
687 6.42920847815365e-08
688 6.43704376959153e-08
689 6.42990820896028e-08
690 6.42682116569482e-08
691 6.42523304799081e-08
692 6.41919498098886e-08
693 6.41532949996471e-08
694 6.41456652012984e-08
695 6.41285483509169e-08
696 6.40903018644945e-08
697 6.40845740831342e-08
698 6.39696763240494e-08
699 6.40139193324174e-08
700 6.39480407238491e-08
701 6.39476983606002e-08
702 6.39062834864035e-08
703 6.38876109497755e-08
704 6.3842732314745e-08
705 6.37561405743004e-08
706 6.38295530266397e-08
707 6.37548299085022e-08
708 6.37100005855817e-08
709 6.37165179500077e-08
710 6.36561067621777e-08
711 6.36238246776166e-08
712 6.35874385681578e-08
713 6.35785145199907e-08
714 6.35364020471485e-08
715 6.34351862827121e-08
716 6.34850763829853e-08
717 6.34067028941754e-08
718 6.33155764959703e-08
719 6.34196870317183e-08
720 6.33152342821575e-08
721 6.32507653477443e-08
722 6.33207865321506e-08
723 6.32689532493735e-08
724 6.3199147378068e-08
725 6.31908428769812e-08
726 6.31219405209382e-08
727 6.31856921859963e-08
728 6.31304252072251e-08
729 6.30799289687012e-08
730 6.30383091506648e-08
731 6.29405888217249e-08
732 6.30452820638006e-08
733 6.29118837467413e-08
734 6.28920214622575e-08
735 6.28986836874734e-08
736 6.27898065235222e-08
737 6.28953021684087e-08
738 6.28337367358789e-08
739 6.27673915325477e-08
740 6.27361817711058e-08
741 6.26894262218958e-08
742 6.26358015938244e-08
743 6.27005437230377e-08
744 6.26238473844243e-08
745 6.25866033534539e-08
746 6.25659483064478e-08
747 6.25481874136913e-08
748 6.24014835999276e-08
749 6.25128552036447e-08
750 6.24484787470347e-08
751 6.23771457413103e-08
752 6.24068157832891e-08
753 6.23158061470175e-08
754 6.22528491722729e-08
755 6.23113013258347e-08
756 6.22827898553169e-08
757 6.22169656852023e-08
758 6.22109634689227e-08
759 6.21765033557153e-08
760 6.20473613734962e-08
761 6.211143187973e-08
762 6.20692108825871e-08
763 6.20607667134454e-08
764 6.20228810419565e-08
765 6.20184171900018e-08
766 6.19746137504595e-08
767 6.18212367731186e-08
768 6.19658091696174e-08
769 6.18998754555733e-08
770 6.18576117015213e-08
771 6.18386140860405e-08
772 6.1801137202222e-08
773 6.16586369106553e-08
774 6.17920668875005e-08
775 6.16885256703448e-08
776 6.16672279549757e-08
777 6.16736709573296e-08
778 6.15315953589324e-08
779 6.16124569225995e-08
780 6.15917684991807e-08
781 6.15643185994941e-08
782 6.15303788940036e-08
783 6.14922235480098e-08
784 6.14812394408304e-08
785 6.13309832597775e-08
786 6.1452013510932e-08
787 6.13454111220158e-08
788 6.13505914970336e-08
789 6.13379878333653e-08
790 6.12822920511569e-08
791 6.11244209178796e-08
792 6.1225341047022e-08
793 6.11589750516561e-08
794 6.1232677028844e-08
795 6.11766762224075e-08
796 6.11410010193936e-08
797 6.11189693153769e-08
798 6.10500932347602e-08
799 6.08868012592634e-08
800 6.10657806370263e-08
801 6.1039168222754e-08
802 6.09957116808335e-08
803 6.09655785486218e-08
804 6.09394084187809e-08
805 6.08779015083272e-08
806 6.09098593025159e-08
807 6.08669769850412e-08
808 6.08456565920346e-08
809 6.06780297480736e-08
810 6.08386823655049e-08
811 6.07874609612757e-08
812 6.07276583559635e-08
813 6.06982766149233e-08
814 6.06219758716442e-08
815 6.06330076315942e-08
816 6.05367168660109e-08
817 6.06277786936449e-08
818 6.05813880112382e-08
819 6.05644788389537e-08
820 6.05161777356145e-08
821 6.04427751471626e-08
822 6.04814636575579e-08
823 6.02963258444778e-08
824 6.04271242201637e-08
825 6.03292193297467e-08
826 6.03655443396356e-08
827 6.03354914370247e-08
828 6.02555346638223e-08
829 6.01711791201609e-08
830 6.02278121573097e-08
831 6.02385761045987e-08
832 6.01185552191463e-08
833 6.0113152229313e-08
834 6.01705394813745e-08
835 6.00472617848791e-08
836 6.00509226789114e-08
837 5.99089476984727e-08
838 6.00420345147068e-08
839 6.00365127854108e-08
840 5.99441434079306e-08
841 5.98882115785404e-08
842 5.98863508798342e-08
843 5.99171947452959e-08
844 5.98583175661638e-08
845 5.97875058716557e-08
846 5.97579243575286e-08
847 5.97835882685249e-08
848 5.97360105614175e-08
849 5.97985241304055e-08
850 5.97558805117693e-08
851 5.96798425258616e-08
852 5.95639113716295e-08
853 5.95980135507101e-08
854 5.96472634433187e-08
855 5.95621630226084e-08
856 5.95731600829819e-08
857 5.95331661237264e-08
858 5.95390944122531e-08
859 5.95346132339358e-08
860 5.93985212469228e-08
861 5.94433561562635e-08
862 5.93995407005554e-08
863 5.94136911860055e-08
864 5.94282628443299e-08
865 5.93856936657033e-08
866 5.93217910289656e-08
867 5.90957195012454e-08
868 5.91403347840469e-08
869 5.91378479941618e-08
870 5.91074831211014e-08
871 5.91396480003059e-08
872 5.90506952333758e-08
873 5.91592623990778e-08
874 5.91358197352676e-08
875 5.90405473535061e-08
876 5.88555053675499e-08
877 5.89702704620887e-08
878 5.892012632569e-08
879 5.89370534427314e-08
880 5.8966426509155e-08
881 5.88741147971028e-08
882 5.88844021631729e-08
883 5.8950771977706e-08
884 5.88337682947415e-08
885 5.89322189556984e-08
886 5.87829139027107e-08
887 5.8794822314745e-08
888 5.87940715039981e-08
889 5.87106427278883e-08
890 5.8777847530278e-08
891 5.86953889585296e-08
892 5.85144085472056e-08
893 5.86623867957492e-08
894 5.85369287970927e-08
895 5.85912112076592e-08
896 5.85235265351258e-08
897 5.85357141906773e-08
898 5.85202445262389e-08
899 5.85081019206513e-08
900 5.84229574815964e-08
901 5.83613078650735e-08
902 5.83566255205614e-08
903 5.83564739180531e-08
904 5.83306971628783e-08
905 5.83664424744779e-08
906 5.8307666046753e-08
907 5.849797087909e-08
908 5.82887392415188e-08
909 5.81852008121331e-08
910 5.82887773068475e-08
911 5.83045666855408e-08
912 5.83120288839556e-08
913 5.82373792803459e-08
914 5.81945947866203e-08
915 5.81924715361559e-08
916 5.81822555718592e-08
917 5.805343897336e-08
918 5.80355316084358e-08
919 5.79338858168743e-08
920 5.80230658364655e-08
921 5.79587659377268e-08
922 5.79736387493757e-08
923 5.79273756076137e-08
924 5.78992473059259e-08
925 5.78694356669907e-08
926 5.78458779456614e-08
927 5.79578451604945e-08
928 5.7821792193602e-08
929 5.7910798040961e-08
930 5.78416914838176e-08
931 5.79020163686828e-08
932 5.77740026515006e-08
933 5.78026956117306e-08
934 5.77202071743166e-08
935 5.78039967358279e-08
936 5.77217177466505e-08
937 5.77756746287239e-08
938 5.76963244394157e-08
939 5.77048842640426e-08
940 5.75919954322046e-08
941 5.76983796656272e-08
942 5.76027775172072e-08
943 5.76848660085449e-08
944 5.75632426922379e-08
945 5.7607234489554e-08
946 5.74944601023919e-08
947 5.76079507059735e-08
948 5.75194512584254e-08
949 5.75913350957524e-08
950 5.74302683864492e-08
951 5.75121183918892e-08
952 5.74649354958545e-08
953 5.7441447844031e-08
954 5.74179964922816e-08
955 5.74555825996192e-08
956 5.73249283695176e-08
957 5.74375231223723e-08
958 5.73465244804172e-08
959 5.74126961656596e-08
960 5.73022954684088e-08
961 5.72695670864043e-08
962 5.73000364754161e-08
963 5.73099002565414e-08
964 5.72801012490576e-08
965 5.72217221221916e-08
966 5.71917816385437e-08
967 5.71999200997819e-08
968 5.71235369033474e-08
969 5.71744833637311e-08
970 5.71309273018628e-08
971 5.70640138062295e-08
972 5.70156309198655e-08
973 5.70580207834848e-08
974 5.7108603013889e-08
975 5.69501883078694e-08
976 5.70283598975152e-08
977 5.70324406488076e-08
978 5.6976535151243e-08
979 5.7001208079388e-08
980 5.70607646612764e-08
981 5.69565076153911e-08
982 5.69556509277813e-08
983 5.700327155167e-08
984 5.68711590454019e-08
985 5.68934613849326e-08
986 5.68475560751569e-08
987 5.69136606256571e-08
988 5.68669831944035e-08
989 5.68440838033535e-08
990 5.68192451035898e-08
991 5.68689043021209e-08
992 5.67363414234467e-08
993 5.67916285687531e-08
994 5.67053673372619e-08
995 5.67811968632537e-08
996 5.67352148512779e-08
997 5.66970537088096e-08
998 5.66196106634997e-08
999 5.66692050651962e-08
1000 5.66844850493631e-08
1001 5.66546740785601e-08
1002 5.66183482137994e-08
1003 5.65497965459016e-08
1004 5.66132281971665e-08
1005 5.65922706241562e-08
1006 5.66058167492667e-08
1007 5.65954630624699e-08
1008 5.65705265038741e-08
1009 5.65934003973201e-08
1010 5.65061097090869e-08
1011 5.64929223421107e-08
1012 5.64967117790971e-08
1013 5.64914681633333e-08
1014 5.65070149092239e-08
1015 5.64898606907338e-08
1016 5.64656152024501e-08
1017 5.6435456974091e-08
1018 5.64081530489702e-08
1019 5.63689745640694e-08
1020 5.63311182824044e-08
1021 5.63244633677851e-08
1022 5.63033290930104e-08
1023 5.62848100622571e-08
1024 5.63232547086123e-08
1025 5.63255848025079e-08
1026 5.63029248903391e-08
1027 5.62864956368259e-08
1028 5.62446566469976e-08
1029 5.60077487341548e-08
1030 5.61985585092462e-08
1031 5.60686339314831e-08
1032 5.61414922659598e-08
1033 5.61508748198047e-08
1034 5.61649911590134e-08
1035 5.61016871374331e-08
1036 5.61412444950449e-08
1037 5.6049901693056e-08
1038 5.60943621872312e-08
1039 5.60909968456702e-08
1040 5.60337525767896e-08
1041 5.59519340452841e-08
1042 5.60460311547217e-08
1043 5.59294493356344e-08
1044 5.60057475536002e-08
1045 5.59239155113467e-08
1046 5.59557695625212e-08
1047 5.59095114884833e-08
1048 5.58004584625404e-08
1049 5.5748585706672e-08
1050 5.59113522704546e-08
1051 5.59092969254493e-08
1052 5.58179122363001e-08
1053 5.58270576447306e-08
1054 5.58357365043616e-08
1055 5.58090657285515e-08
1056 5.56586767148648e-08
1057 5.57159325933476e-08
1058 5.57629889137434e-08
1059 5.56448428767897e-08
1060 5.57203768640768e-08
1061 5.55949189970928e-08
1062 5.57101291336348e-08
1063 5.57119278274953e-08
1064 5.5669650093515e-08
1065 5.55762666336612e-08
1066 5.58037259568156e-08
1067 5.55877626799983e-08
1068 5.5617817831477e-08
1069 5.55588417832098e-08
1070 5.55686901984043e-08
1071 5.54470850586686e-08
1072 5.55398917050365e-08
1073 5.55288814065857e-08
1074 5.55233592374194e-08
1075 5.54534169425747e-08
1076 5.55202610392769e-08
1077 5.53687566007266e-08
1078 5.54668107195511e-08
1079 5.5407786668793e-08
1080 5.54484336008354e-08
1081 5.53784768508869e-08
1082 5.5374293062016e-08
1083 5.5362488512456e-08
1084 5.53585868954265e-08
1085 5.53735626924734e-08
1086 5.53795659365974e-08
1087 5.52643197990754e-08
1088 5.53259269180373e-08
1089 5.53049404161676e-08
1090 5.52768603321319e-08
1091 5.522837894123e-08
1092 5.5271254225886e-08
1093 5.51858191528698e-08
1094 5.52785630807406e-08
1095 5.51296742183904e-08
1096 5.52151763277831e-08
1097 5.52939536060393e-08
1098 5.50983198641664e-08
1099 5.51081640032258e-08
1100 5.51769430243709e-08
1101 5.51190495836629e-08
1102 5.51637745989098e-08
1103 5.50460813604214e-08
1104 5.52184073985273e-08
1105 5.49740829944856e-08
1106 5.50656448179421e-08
1107 5.50422099643288e-08
1108 5.49917055381322e-08
1109 5.50606466298564e-08
1110 5.49574804218356e-08
1111 5.50266251846399e-08
1112 5.49964004175063e-08
1113 5.49326122423199e-08
1114 5.49018342754781e-08
1115 5.49209604987499e-08
1116 5.49035444874502e-08
1117 5.48625377483347e-08
1118 5.48888668490122e-08
1119 5.4865639823154e-08
1120 5.48053473192756e-08
1121 5.48200272483257e-08
1122 5.48237966186971e-08
1123 5.47815699136756e-08
1124 5.47837474513813e-08
1125 5.47320887451796e-08
1126 5.47595869686557e-08
1127 5.47020052306113e-08
1128 5.475770998431e-08
1129 5.47279321678751e-08
1130 5.47163213744373e-08
1131 5.46741631550773e-08
1132 5.46701000865291e-08
1133 5.46210454055451e-08
1134 5.46406070451866e-08
1135 5.4713848895771e-08
1136 5.46239664211967e-08
1137 5.4599468491956e-08
1138 5.4537460198123e-08
1139 5.46046363856245e-08
1140 5.45502425846056e-08
1141 5.45608795896602e-08
1142 5.45185194538167e-08
1143 5.45179332369639e-08
1144 5.44988017427972e-08
1145 5.45271911922551e-08
1146 5.44278807927956e-08
1147 5.45151016695389e-08
1148 5.44072958126307e-08
1149 5.4476056236874e-08
1150 5.43997923048156e-08
1151 5.44375808391173e-08
1152 5.43816549067877e-08
1153 5.43993472723603e-08
1154 5.43573514022455e-08
1155 5.43662695275948e-08
1156 5.43325503112868e-08
1157 5.4341956686299e-08
1158 5.43137615567346e-08
1159 5.43405636954652e-08
1160 5.42664207399124e-08
1161 5.42928011759702e-08
1162 5.42540058146024e-08
1163 5.42979815425504e-08
1164 5.42120123319112e-08
1165 5.42746123910476e-08
1166 5.42010036685969e-08
1167 5.42357903998703e-08
1168 5.4158660495629e-08
1169 5.42107229710709e-08
1170 5.41718273736613e-08
1171 5.4195616829178e-08
1172 5.41343461035737e-08
1173 5.41595631708969e-08
1174 5.40981628547321e-08
1175 5.4222929526615e-08
1176 5.40921332627953e-08
1177 5.41029220784139e-08
1178 5.40741404633227e-08
1179 5.40510636015679e-08
1180 5.40402900648385e-08
1181 5.41046450266425e-08
1182 5.40004168707586e-08
1183 5.40367340635672e-08
1184 5.3972358765142e-08
1185 5.40025535258337e-08
1186 5.39507149683871e-08
1187 5.39948383417865e-08
1188 5.39336535161361e-08
1189 5.3953276248464e-08
1190 5.39108205990235e-08
1191 5.39461618604964e-08
1192 5.38891160615229e-08
1193 5.38887756456052e-08
1194 5.38812114243559e-08
1195 5.38292618577074e-08
1196 5.39513499586697e-08
1197 5.37784744423231e-08
1198 5.38692788909501e-08
1199 5.3778769285362e-08
1200 5.3848612250551e-08
1201 5.37724500371262e-08
1202 5.37545332743861e-08
1203 5.37798974202808e-08
1204 5.37482009701495e-08
1205 5.37587346598478e-08
1206 5.37141917873463e-08
1207 5.37985903108496e-08
1208 5.36934901627095e-08
1209 5.36852115740061e-08
1210 5.36662581893665e-08
1211 5.36751143365155e-08
1212 5.36717760983851e-08
1213 5.36867064380164e-08
1214 5.36946851779074e-08
1215 5.36364329573047e-08
1216 5.35877829974662e-08
1217 5.36497680403425e-08
1218 5.35540036725735e-08
1219 5.35855839629562e-08
1220 5.36748403938692e-08
1221 5.3499835099613e-08
1222 5.3565082313245e-08
1223 5.3528145090409e-08
1224 5.35156450520713e-08
1225 5.34780377352551e-08
1226 5.35775900147861e-08
1227 5.34450675508946e-08
1228 5.3532081158103e-08
1229 5.34484799570301e-08
1230 5.34847088353718e-08
1231 5.3396389503968e-08
1232 5.35009533677577e-08
1233 5.34090089359118e-08
1234 5.3538794104524e-08
1235 5.33558572104464e-08
1236 5.33622817651125e-08
1237 5.34435260248767e-08
1238 5.33726582028216e-08
1239 5.3373752196384e-08
1240 5.33223772090086e-08
1241 5.33304162475545e-08
1242 5.3340174749783e-08
1243 5.33383616678762e-08
1244 5.32803353834321e-08
1245 5.33337839840886e-08
1246 5.32646972217776e-08
1247 5.3273082305072e-08
1248 5.3277115255268e-08
1249 5.32220661377814e-08
1250 5.32352256066293e-08
1251 5.32306416340944e-08
1252 5.32075752628369e-08
1253 5.32127522268055e-08
1254 5.31772960925547e-08
1255 5.32354025637449e-08
1256 5.31577899993696e-08
1257 5.31863489909856e-08
1258 5.31739301783407e-08
1259 5.31520129336549e-08
1260 5.31128911187562e-08
1261 5.31650633377456e-08
1262 5.3078748172064e-08
1263 5.30879113478377e-08
1264 5.30665287259424e-08
1265 5.30910671794338e-08
1266 5.30286345234998e-08
1267 5.30944067831385e-08
1268 5.2996552303064e-08
1269 5.30642701659367e-08
1270 5.30098925748401e-08
1271 5.30201931341345e-08
1272 5.29580395893792e-08
1273 5.30400943035136e-08
1274 5.29289132418231e-08
1275 5.30097832849297e-08
1276 5.29321940703209e-08
1277 5.29588949826909e-08
1278 5.29590662583512e-08
1279 5.2933593230442e-08
1280 5.29032377725169e-08
1281 5.29309507539821e-08
1282 5.28854665797773e-08
1283 5.28931307421931e-08
1284 5.28910919652681e-08
1285 5.28571152762503e-08
1286 5.28505750081454e-08
1287 5.28360589409083e-08
1288 5.28313183263407e-08
1289 5.28378979047783e-08
1290 5.27873872202544e-08
1291 5.28019087788767e-08
1292 5.27936682179764e-08
1293 5.28109967092849e-08
1294 5.27407946440128e-08
1295 5.27633789555448e-08
1296 5.27466238160468e-08
1297 5.27594386816599e-08
1298 5.2739690353576e-08
1299 5.2724526645731e-08
1300 5.26558441811265e-08
1301 5.27471406850388e-08
1302 5.26185383036282e-08
1303 5.2676300462684e-08
1304 5.2656548022334e-08
1305 5.26302322974015e-08
1306 5.2652446096868e-08
1307 5.26160914668949e-08
1308 5.26439063781581e-08
1309 5.25796919681465e-08
1310 5.26251796886257e-08
1311 5.25294859095027e-08
1312 5.26168526990833e-08
1313 5.25229559222851e-08
1314 5.2623064262125e-08
1315 5.25058089582142e-08
1316 5.25274306579782e-08
1317 5.2502562804202e-08
1318 5.25012359997579e-08
1319 5.25099784969285e-08
1320 5.24725985320274e-08
1321 5.24704340192361e-08
1322 5.24511853874365e-08
1323 5.24634660930001e-08
1324 5.24346221788541e-08
1325 5.24140008968388e-08
1326 5.24214517203525e-08
1327 5.23919923689498e-08
1328 5.23994830072816e-08
1329 5.25067002170587e-08
1330 5.23399901168897e-08
1331 5.23582850726623e-08
1332 5.23544809658549e-08
1333 5.23255186057892e-08
1334 5.23109024908042e-08
1335 5.23019050557849e-08
1336 5.22851338933261e-08
1337 5.23001804599854e-08
1338 5.22768100179594e-08
1339 5.22706164480802e-08
1340 5.22456076383904e-08
1341 5.22590615137375e-08
1342 5.22345022206938e-08
1343 5.22542403027515e-08
1344 5.22233633057922e-08
1345 5.22077445808122e-08
1346 5.21985854482487e-08
1347 5.221022955193e-08
1348 5.21768047527793e-08
1349 5.21933401629315e-08
1350 5.21496036383073e-08
1351 5.21606369110472e-08
1352 5.2121885324441e-08
1353 5.21412548237787e-08
1354 5.20952553395571e-08
1355 5.21362798331371e-08
1356 5.20676156685607e-08
1357 5.21248725431356e-08
1358 5.20549203439913e-08
1359 5.20876014440397e-08
1360 5.20379835857732e-08
1361 5.20793015938992e-08
1362 5.19994202701213e-08
1363 5.20566865751526e-08
1364 5.19845450752676e-08
1365 5.20411467748172e-08
1366 5.19646758925241e-08
1367 5.19986268743278e-08
1368 5.19551504520876e-08
1369 5.19961018887738e-08
1370 5.19301477142076e-08
1371 5.19735804795918e-08
1372 5.19081874186256e-08
1373 5.19596817021295e-08
1374 5.18856594660111e-08
1375 5.19211312326462e-08
1376 5.18883034223983e-08
1377 5.19011188611085e-08
1378 5.18665606332736e-08
1379 5.18837491423341e-08
1380 5.18278379961767e-08
1381 5.1881902576989e-08
1382 5.18151558037694e-08
1383 5.18355194198961e-08
1384 5.18182380160326e-08
1385 5.18222509675592e-08
1386 5.18027687643752e-08
1387 5.1791721710881e-08
1388 5.17597140019532e-08
1389 5.1790866878898e-08
1390 5.1749841514015e-08
1391 5.17860623849842e-08
1392 5.1732624901879e-08
1393 5.17482425630433e-08
1394 5.17261477868125e-08
1395 5.17292117070411e-08
1396 5.16846085136802e-08
1397 5.17192070259576e-08
1398 5.16717382492704e-08
1399 5.16884851025079e-08
1400 5.16445491058892e-08
1401 5.16743433616185e-08
1402 5.16344545193892e-08
1403 5.1623291368541e-08
1404 5.15872321158461e-08
1405 5.16423754159945e-08
1406 5.15891810588531e-08
1407 5.16007358328796e-08
1408 5.1597953410587e-08
1409 5.16010372453302e-08
1410 5.15580635853663e-08
1411 5.15909553697735e-08
1412 5.15261738458506e-08
1413 5.1571532497352e-08
1414 5.15517714563263e-08
1415 5.15215848426731e-08
1416 5.15267383998097e-08
1417 5.15037891701819e-08
1418 5.14760010423831e-08
1419 5.15166705830072e-08
1420 5.14763177741351e-08
1421 5.14683556458451e-08
1422 5.14461205334271e-08
1423 5.14435745919695e-08
1424 5.14548093277778e-08
1425 5.14323264559913e-08
1426 5.14194988412431e-08
1427 5.14208782971259e-08
1428 5.1389421361403e-08
1429 5.13684335918807e-08
1430 5.13828864165955e-08
1431 5.14000143470028e-08
1432 5.13651105260582e-08
1433 5.13587700250451e-08
1434 5.1345501073774e-08
1435 5.13362597920342e-08
1436 5.1317477852475e-08
1437 5.13328335842722e-08
1438 5.12999184296703e-08
1439 5.13207211765287e-08
1440 5.12588433583261e-08
1441 5.12976539308596e-08
1442 5.12754038100027e-08
1443 5.12913811707705e-08
1444 5.12811785742251e-08
1445 5.12988932821479e-08
1446 5.12532579564073e-08
1447 5.12754560153539e-08
1448 5.12155732046438e-08
1449 5.1261845233519e-08
1450 5.12091223401345e-08
1451 5.12354992801001e-08
1452 5.11920040495717e-08
1453 5.12211860932421e-08
1454 5.11786784289647e-08
1455 5.12033408419654e-08
1456 5.11463212999885e-08
1457 5.11815939658877e-08
1458 5.11399497709064e-08
1459 5.11884561049669e-08
1460 5.11135797656159e-08
1461 5.11349903136171e-08
1462 5.11199577053478e-08
1463 5.11208085653969e-08
1464 5.1084529552714e-08
1465 5.10734278325042e-08
1466 5.10830446538435e-08
1467 5.10882994546602e-08
1468 5.10409648946108e-08
1469 5.10474680059936e-08
1470 5.10260420625297e-08
1471 5.10362966399391e-08
1472 5.10128880568672e-08
1473 5.10416975947336e-08
1474 5.09795331924412e-08
1475 5.09947537878919e-08
1476 5.0989018322678e-08
1477 5.0994799531745e-08
1478 5.09766745515616e-08
1479 5.09817717915251e-08
1480 5.09864782358704e-08
1481 5.10671181292999e-08
1482 5.08586636516206e-08
1483 5.0966033140476e-08
1484 5.09359231486428e-08
1485 5.10321652480439e-08
1486 5.09717925387676e-08
1487 5.0949878495965e-08
1488 5.08245761716708e-08
1489 5.09112507918186e-08
1490 5.09048521966093e-08
1491 5.08864645205609e-08
1492 5.09460310951226e-08
1493 5.09230464555976e-08
1494 5.07962721347788e-08
1495 5.08966424928037e-08
1496 5.09007347488932e-08
1497 5.0816347670013e-08
1498 5.08188491528117e-08
1499 5.08205427571795e-08
1500 5.09332608609192e-08
1501 5.09061565983071e-08
1502 5.08578044717822e-08
1503 5.09124100487313e-08
1504 5.08986231089192e-08
1505 5.09564273207808e-08
1506 5.08106332752778e-08
1507 5.07719079672064e-08
1508 5.08533047411941e-08
1509 5.0941920211045e-08
1510 5.0797984507911e-08
1511 5.07681500891977e-08
1512 5.08410948927285e-08
1513 5.08621259109887e-08
1514 5.08449263734168e-08
1515 5.06849187367209e-08
1516 5.06974482881173e-08
1517 5.07132786189413e-08
1518 5.0709158303297e-08
1519 5.07533661240345e-08
1520 5.08307505542316e-08
1521 5.08520256774503e-08
1522 5.03945050203125e-08
1523 5.05758223190433e-08
1524 5.06960862856065e-08
1525 5.08096817346537e-08
1526 5.05667376742291e-08
1527 5.08801916347945e-08
1528 5.05808628084026e-08
1529 5.07085696916931e-08
1530 5.06517680896046e-08
1531 5.07021993294554e-08
1532 5.07295979892408e-08
1533 5.0305159071673e-08
1534 5.06188656663742e-08
1535 5.07298375849174e-08
1536 5.05244011876904e-08
1537 5.05489823070526e-08
1538 5.0712561204147e-08
1539 5.03922601171602e-08
1540 5.06834563380743e-08
1541 5.05552856162605e-08
1542 5.025308222395e-08
1543 5.05889238986423e-08
1544 5.06124257229068e-08
1545 5.02745555017725e-08
1546 5.06449783779228e-08
1547 5.03829364573161e-08
1548 5.0601805432926e-08
1549 5.0411822584584e-08
1550 5.03143699177588e-08
1551 5.05933934866754e-08
1552 5.05138925077642e-08
1553 5.03538843730222e-08
1554 5.05393870429138e-08
1555 5.04129022347311e-08
1556 5.01909592616201e-08
1557 5.05333712716993e-08
1558 5.04709354340882e-08
1559 5.03410070222809e-08
1560 5.0135580218269e-08
1561 5.03529082780396e-08
1562 5.05627648657025e-08
1563 5.03602194403818e-08
1564 5.01467827132007e-08
1565 5.02475818571213e-08
1566 5.04071533489192e-08
1567 5.02361385048555e-08
1568 5.05098849228336e-08
1569 5.02086428801896e-08
1570 5.02080481059597e-08
1571 5.03917848893032e-08
1572 5.03335088337664e-08
1573 5.01357855702267e-08
1574 5.02485721374235e-08
1575 5.03020114612784e-08
1576 5.01225679583328e-08
1577 5.03559547415655e-08
1578 5.03086653689255e-08
1579 5.0139546865946e-08
1580 5.02290572226283e-08
1581 5.03902547623891e-08
1582 5.01929420180236e-08
1583 4.99704709608917e-08
1584 5.00880178671004e-08
1585 5.01805703683189e-08
1586 5.02898961411979e-08
1587 5.00145897561755e-08
1588 5.0365597534574e-08
1589 5.04306943704957e-08
1590 5.01424535572781e-08
1591 5.02610391481806e-08
1592 5.00213302834229e-08
1593 5.00111091303879e-08
1594 5.01867482793905e-08
1595 4.98971689939509e-08
1596 5.02173311143217e-08
1597 4.99615082671934e-08
1598 4.99774160678257e-08
1599 5.02650639955249e-08
1600 5.00560656035987e-08
1601 5.01799880971987e-08
1602 4.97990069110443e-08
1603 5.01958901168997e-08
1604 5.00458799752579e-08
1605 5.02721975446097e-08
1606 4.97744896894581e-08
1607 5.00123735098779e-08
1608 5.01471549103627e-08
1609 5.0212770529745e-08
1610 4.98636626795612e-08
1611 4.99233963764389e-08
1612 5.01882328827197e-08
1613 4.98130260677687e-08
1614 4.98262561436924e-08
1615 4.99816120771435e-08
1616 5.00621479795349e-08
1617 4.97950922297985e-08
1618 4.99407425766929e-08
1619 4.99515568248565e-08
1620 4.98092096259395e-08
1621 4.99669218037013e-08
1622 4.98127922718972e-08
1623 4.9859957098608e-08
1624 5.00485802688022e-08
1625 4.97191720811863e-08
1626 4.9993591124009e-08
1627 4.97374290593378e-08
1628 4.97885242760621e-08
1629 5.0063283712154e-08
1630 4.99228351513725e-08
1631 4.97312284850615e-08
1632 4.9946437731041e-08
1633 4.98313428523023e-08
1634 4.96957987845548e-08
1635 4.97507152112941e-08
1636 5.01056622661533e-08
1637 4.97437582467342e-08
1638 4.97994318535699e-08
1639 4.98668561550453e-08
1640 4.98309014935749e-08
1641 4.96174843862551e-08
1642 4.97203186795581e-08
1643 4.98562313080431e-08
1644 4.9575258878054e-08
1645 4.9669847156375e-08
1646 4.95276802099376e-08
1647 4.97555417711126e-08
1648 4.96403901106834e-08
1649 4.96753266305827e-08
1650 4.98745367840669e-08
1651 4.94080262423147e-08
1652 4.96770312554684e-08
1653 4.95646719167464e-08
1654 4.95706802494666e-08
1655 4.97250857898202e-08
1656 4.95506644624388e-08
1657 4.9611819757267e-08
1658 4.95395833088885e-08
1659 4.9522678881031e-08
1660 4.96775604204025e-08
1661 4.9637966908378e-08
1662 4.95121427122047e-08
1663 4.96835183740885e-08
1664 4.97190160528849e-08
1665 4.94427042423418e-08
1666 4.94760761968749e-08
1667 4.97263656857871e-08
1668 4.93746173140241e-08
1669 4.94845462477578e-08
1670 4.94707658391658e-08
1671 4.94998294529037e-08
1672 4.94944542532849e-08
1673 4.94974346489929e-08
1674 4.9374449349493e-08
1675 4.95355390257579e-08
1676 4.94648929794117e-08
1677 4.95896404753005e-08
1678 4.95864150582559e-08
1679 4.93567747470447e-08
1680 4.94111822584298e-08
1681 4.94075836321439e-08
1682 4.9538767342483e-08
1683 4.92900165605104e-08
1684 4.93521649780337e-08
1685 4.93246255914848e-08
1686 4.94356289724163e-08
1687 4.95450889830273e-08
1688 4.92048114131283e-08
1689 4.93729804167398e-08
1690 4.94137178423948e-08
1691 4.94164088880034e-08
1692 4.93181234668683e-08
1693 4.9338866319415e-08
1694 4.92926032062968e-08
1695 4.94779875468421e-08
1696 4.93248868587148e-08
1697 4.94337557470637e-08
1698 4.92223773060907e-08
1699 4.92859095637677e-08
1700 4.93213667844827e-08
1701 4.9256815894072e-08
1702 4.92573701020849e-08
1703 4.93099216127835e-08
1704 4.92011993593611e-08
1705 4.92930234061717e-08
1706 4.92618066756378e-08
1707 4.92400609131138e-08
1708 4.92943397119028e-08
1709 4.93229436966391e-08
1710 4.92008575918579e-08
1711 4.91244384330791e-08
1712 4.92355200325534e-08
1713 4.92389957804651e-08
1714 4.9105839227126e-08
1715 4.9227604347335e-08
1716 4.91824437212074e-08
1717 4.90909320935984e-08
1718 4.92162700791443e-08
1719 4.90675647530914e-08
1720 4.92354811738593e-08
1721 4.9158085248191e-08
1722 4.91985834631947e-08
1723 4.90943505730979e-08
1724 4.90916938464814e-08
1725 4.90739468446133e-08
1726 4.91182828892267e-08
1727 4.91439580068143e-08
1728 4.91803918398759e-08
1729 4.90204558487584e-08
1730 4.91749160114185e-08
1731 4.91018862249781e-08
1732 4.90344642860574e-08
1733 4.91543220497714e-08
1734 4.90886004287372e-08
1735 4.89613941019318e-08
1736 4.89373535716986e-08
1737 4.91108398534834e-08
1738 4.90501885974659e-08
1739 4.89504297747079e-08
1740 4.89879816172323e-08
1741 4.89024155838624e-08
1742 4.88942136684933e-08
1743 4.90710667355732e-08
1744 4.88911456486552e-08
1745 4.88911884855003e-08
1746 4.91719108330901e-08
1747 4.88324132563722e-08
1748 4.89473077092128e-08
1749 4.88331261228048e-08
1750 4.88395374149686e-08
1751 4.8908915826873e-08
1752 4.89473801625895e-08
1753 4.88791033685931e-08
1754 4.88650776666599e-08
1755 4.89037661350888e-08
1756 4.87953176733225e-08
1757 4.88693559856923e-08
1758 4.88616376614903e-08
1759 4.87400330271281e-08
1760 4.88404889655847e-08
1761 4.87652034766306e-08
1762 4.88573986650032e-08
1763 4.89843754458708e-08
1764 4.87757766987062e-08
1765 4.88237409279613e-08
1766 4.90089953248951e-08
1767 4.90574159126567e-08
1768 4.90433872528673e-08
1769 4.84551445114612e-08
1770 4.83720655761744e-08
1771 4.84308719661541e-08
1772 4.83975430110028e-08
1773 4.84065638468589e-08
1774 4.83722053918889e-08
1775 4.83428071660352e-08
1776 4.82039600513406e-08
1777 4.82413594531383e-08
1778 4.84531751283601e-08
1779 4.84298451493004e-08
1780 4.82626787956519e-08
1781 4.81581185591828e-08
1782 4.82139434858642e-08
1783 4.82322616737196e-08
1784 4.81508272742737e-08
1785 4.83750717144016e-08
1786 4.81910141532094e-08
1787 4.80593014997588e-08
1788 4.83817798959674e-08
1789 4.7974601125933e-08
1790 4.81321966370185e-08
1791 4.84522614312422e-08
1792 4.80924127723714e-08
1793 4.79958221490939e-08
1794 4.79933690187284e-08
1795 4.83174676890741e-08
1796 4.81041156108475e-08
1797 4.82631701801406e-08
1798 4.80782719867179e-08
1799 4.78644621730862e-08
1800 4.82473359011415e-08
1801 4.80247992182825e-08
1802 4.804872755515e-08
1803 4.82877316998476e-08
1804 4.79062234293171e-08
1805 4.78229227183213e-08
1806 4.7998227951096e-08
1807 4.79398147608112e-08
1808 4.84847651636144e-08
1809 4.78322675405707e-08
1810 4.78924228299871e-08
1811 4.81192225849636e-08
1812 4.78977575770756e-08
1813 4.78024723831183e-08
1814 4.78955515845758e-08
1815 4.81395769098558e-08
1816 4.80145171148916e-08
1817 4.79015749061951e-08
1818 4.79813596019341e-08
1819 4.81381871189068e-08
1820 4.79389815637443e-08
1821 4.78251941449237e-08
1822 4.78101998997449e-08
1823 4.80981993806751e-08
1824 4.78066121121223e-08
1825 4.78204957610195e-08
1826 4.81332224055997e-08
1827 4.77528597400134e-08
1828 4.80657663290973e-08
1829 4.77585996230268e-08
1830 4.78537864938566e-08
1831 4.76444625312489e-08
1832 4.78033914022014e-08
1833 4.79380278917141e-08
1834 4.76183067947833e-08
1835 4.77638419325022e-08
1836 4.8110980151117e-08
1837 4.79010310086014e-08
1838 4.77848871967002e-08
1839 4.75856482120562e-08
1840 4.75690733883205e-08
1841 4.77261100975124e-08
1842 4.76596879863678e-08
1843 4.77799678240132e-08
1844 4.76181486033234e-08
1845 4.78905884471192e-08
1846 4.7474871512021e-08
1847 4.76411165899471e-08
1848 4.77226509716377e-08
1849 4.75413446303641e-08
1850 4.77447413103071e-08
1851 4.75058637954273e-08
1852 4.77106518295134e-08
1853 4.7901436013964e-08
1854 4.7581423218368e-08
1855 4.76590975762026e-08
1856 4.74962387608802e-08
1857 4.76004957050691e-08
1858 4.78213878074563e-08
1859 4.73968475920294e-08
1860 4.7626042834148e-08
1861 4.77507780898279e-08
1862 4.73777211740245e-08
1863 4.7632405477227e-08
1864 4.77177836566423e-08
1865 4.74511624064533e-08
1866 4.76876372494939e-08
1867 4.76758358340934e-08
1868 4.75029759425816e-08
1869 4.73842300909855e-08
1870 4.76071071877637e-08
1871 4.75397495613983e-08
1872 4.73769207238739e-08
1873 4.75382215539e-08
1874 4.73301348076216e-08
1875 4.76182640896106e-08
1876 4.73150763422581e-08
1877 4.73962677531858e-08
1878 4.74932449561205e-08
1879 4.7350112575506e-08
1880 4.7469138505063e-08
1881 4.75127598009717e-08
1882 4.73880715996522e-08
1883 4.72886351436941e-08
1884 4.76160341100762e-08
1885 4.72299052081127e-08
1886 4.72689736121801e-08
1887 4.73987704019407e-08
1888 4.73880277114258e-08
1889 4.74488376429605e-08
1890 4.74056370121101e-08
1891 4.71747046999482e-08
1892 4.73929106430404e-08
1893 4.73544915149393e-08
1894 4.74664972005545e-08
1895 4.71767926624178e-08
1896 4.73878438764785e-08
1897 4.71573521774182e-08
1898 4.72946655607487e-08
1899 4.74082533352682e-08
1900 4.71666336046006e-08
1901 4.72946518623729e-08
1902 4.71396429650994e-08
1903 4.73633268529561e-08
1904 4.72409622824799e-08
1905 4.72542935066045e-08
1906 4.7355206238997e-08
1907 4.72889820262168e-08
1908 4.70946515787052e-08
1909 4.72419599963914e-08
1910 4.72480073678838e-08
1911 4.71089008144965e-08
1912 4.72086795424431e-08
1913 4.72237212574278e-08
1914 4.7217891045559e-08
1915 4.72114508374144e-08
1916 4.73033554078128e-08
1917 4.68768711059386e-08
1918 4.71707134968025e-08
1919 4.70739768843043e-08
1920 4.71431989426119e-08
1921 4.71322753898828e-08
1922 4.71437003126685e-08
1923 4.71882674366597e-08
1924 4.69759094392508e-08
1925 4.71105884318757e-08
1926 4.71072753729374e-08
1927 4.6922074327993e-08
1928 4.70625390505131e-08
1929 4.70438629427417e-08
1930 4.70751929366742e-08
1931 4.69375975222786e-08
1932 4.70356914359371e-08
1933 4.70803294181366e-08
1934 4.68892930096665e-08
1935 4.71013689093969e-08
1936 4.69976178198994e-08
1937 4.69167482664012e-08
1938 4.69058291154845e-08
1939 4.68882570592299e-08
1940 4.70047972935816e-08
1941 4.69105700207084e-08
1942 4.69820361617135e-08
1943 4.71511508239875e-08
1944 4.67755986206519e-08
1945 4.67508773043246e-08
1946 4.70142673856255e-08
1947 4.68356283507276e-08
1948 4.69498918735933e-08
1949 4.69734656545207e-08
1950 4.68745650932245e-08
1951 4.69209194964293e-08
1952 4.70943637553845e-08
1953 4.67000985928667e-08
1954 4.68633483694525e-08
1955 4.70060385640014e-08
1956 4.66989352314595e-08
1957 4.68110755622586e-08
1958 4.69185599880628e-08
1959 4.68360478518282e-08
1960 4.69008878232202e-08
1961 4.6739664067319e-08
1962 4.67873739857794e-08
1963 4.68853466533137e-08
1964 4.65745180793231e-08
1965 4.67159223831448e-08
1966 4.67909996737514e-08
1967 4.68523423053924e-08
1968 4.67658173841645e-08
1969 4.68444573105131e-08
1970 4.66528874671912e-08
1971 4.67753388495584e-08
1972 4.66843548090345e-08
1973 4.67722731150033e-08
1974 4.66792780922098e-08
1975 4.67248135858078e-08
1976 4.65581268263016e-08
1977 4.66911605550013e-08
1978 4.65119260377733e-08
1979 4.66687614688599e-08
1980 4.67342120051839e-08
1981 4.6472862074598e-08
1982 4.64967560644958e-08
1983 4.66084071788497e-08
1984 4.6710290615648e-08
1985 4.66779938690376e-08
1986 4.66273005998996e-08
1987 4.66633523976689e-08
1988 4.65072855360127e-08
1989 4.66324097725401e-08
1990 4.66083455468169e-08
1991 4.6489787751014e-08
1992 4.6539114262778e-08
1993 4.66552612312654e-08
1994 4.66092881497016e-08
1995 4.6571260949424e-08
1996 4.66490532897712e-08
1997 4.66923517572493e-08
1998 4.65063233003882e-08
1999 4.65668898150806e-08
2000 4.65791461141674e-08
2001 4.63558469843139e-08
2002 4.64268133582024e-08
2003 4.65517621415223e-08
2004 4.65469978057076e-08
2005 4.63803678216745e-08
2006 4.64267914979111e-08
2007 4.6369477489705e-08
2008 4.6379755810122e-08
2009 4.64533311788085e-08
2010 4.63548633882205e-08
2011 4.62614464273869e-08
2012 4.63423912953065e-08
2013 4.62537284648956e-08
2014 4.64716980788094e-08
2015 4.62507071297136e-08
2016 4.62783139312251e-08
2017 4.63622014665877e-08
2018 4.63396052268195e-08
2019 4.63001955954834e-08
2020 4.63551410250229e-08
2021 4.62517393604589e-08
2022 4.62446775877368e-08
2023 4.63458521022808e-08
2024 4.62775726508546e-08
2025 4.62184717038294e-08
2026 4.62804450194199e-08
2027 4.62228240780682e-08
2028 4.62604850546278e-08
2029 4.61226935439818e-08
2030 4.61355562635379e-08
2031 4.63348968200439e-08
2032 4.63259275784722e-08
2033 4.61640150619669e-08
2034 4.61094585999522e-08
2035 4.60406542959202e-08
2036 4.61083347957914e-08
2037 4.61054126035254e-08
2038 4.63725588366604e-08
2039 4.62413842636789e-08
2040 4.60953105545503e-08
2041 4.61067032559992e-08
2042 4.61903804649388e-08
2043 4.6112945952359e-08
2044 4.62045717783965e-08
2045 4.62253889579323e-08
2046 4.62263279068509e-08
2047 4.60808666304935e-08
2048 4.60754117355133e-08
2049 4.61207228368288e-08
2050 4.60719927264375e-08
2051 4.61108308142943e-08
2052 4.62001508165777e-08
2053 4.61495329406869e-08
2054 4.6051897899968e-08
2055 4.60806568594041e-08
2056 4.60645841955376e-08
2057 4.60813029350415e-08
2058 4.60784936333614e-08
2059 4.60662161814174e-08
2060 4.6058491000478e-08
2061 4.59501207217272e-08
2062 4.60526887349211e-08
2063 4.59152717684486e-08
2064 4.60742836216888e-08
2065 4.5917516523053e-08
2066 4.59599205178129e-08
2067 4.60472476417895e-08
2068 4.58855501253019e-08
2069 4.61533975488759e-08
2070 4.58639109641723e-08
2071 4.61360286976298e-08
2072 4.59123287019914e-08
2073 4.60316669081973e-08
2074 4.59167232882418e-08
2075 4.59227100868542e-08
2076 4.59939633576933e-08
2077 4.60956537764456e-08
2078 4.58957640847579e-08
2079 4.59047536949253e-08
2080 4.58197056145693e-08
2081 4.59694883947481e-08
2082 4.58634951048253e-08
2083 4.59412204374399e-08
2084 4.58630802582238e-08
2085 4.59265369043838e-08
2086 4.58174634978548e-08
2087 4.57986886843731e-08
2088 4.58946992900611e-08
2089 4.5863213595787e-08
2090 4.58997786823012e-08
2091 4.56763188869669e-08
2092 4.58765066562172e-08
2093 4.59415987463796e-08
2094 4.57378628455629e-08
2095 4.56678620304185e-08
2096 4.58331214165941e-08
2097 4.56780583228955e-08
2098 4.57638432032059e-08
2099 4.59961910195261e-08
2100 4.58768520525954e-08
2101 4.59076292760319e-08
2102 4.572498287847e-08
2103 4.59641179040293e-08
2104 4.57578371630962e-08
2105 4.57595304090841e-08
2106 4.57692226802919e-08
2107 4.5763833785406e-08
2108 4.56301890303656e-08
2109 4.56888486544127e-08
2110 4.57681611787386e-08
2111 4.58498433053567e-08
2112 4.57195502030494e-08
2113 4.56638092558581e-08
2114 4.58067014021957e-08
2115 4.56756031288474e-08
2116 4.57848456762999e-08
2117 4.56881659245401e-08
2118 4.55860393882013e-08
2119 4.54973891130805e-08
2120 4.57053565137144e-08
2121 4.55958295773051e-08
2122 4.57009696304134e-08
2123 4.55677521573605e-08
2124 4.55733149566928e-08
2125 4.56730044693376e-08
2126 4.55643322514465e-08
2127 4.56251133216234e-08
2128 4.56205895622563e-08
2129 4.56437355824679e-08
2130 4.5623967501518e-08
2131 4.5440542982389e-08
2132 4.56038107901069e-08
2133 4.54919128547449e-08
2134 4.54331361412486e-08
2135 4.57062814915865e-08
2136 4.56107554585028e-08
2137 4.55097469391763e-08
2138 4.56415316067993e-08
2139 4.54606332076501e-08
2140 4.5554332120501e-08
2141 4.54673629668445e-08
2142 4.55848409732784e-08
2143 4.54043944431159e-08
2144 4.55367216312119e-08
2145 4.55778706935916e-08
2146 4.54480739930929e-08
2147 4.55969603851969e-08
2148 4.54729450440894e-08
2149 4.55307565432328e-08
2150 4.54473209194894e-08
2151 4.53775274018575e-08
2152 4.5577666466512e-08
2153 4.53587320059512e-08
2154 4.53277390617401e-08
2155 4.54781982128782e-08
2156 4.5514691454418e-08
2157 4.53820370014402e-08
2158 4.5322864659969e-08
2159 4.53118803662722e-08
2160 4.55046859606512e-08
2161 4.54349782303964e-08
2162 4.5460159532329e-08
2163 4.53673738776139e-08
2164 4.54288693532145e-08
2165 4.54058318146711e-08
2166 4.54259778204236e-08
2167 4.54428779514959e-08
2168 4.53678106135946e-08
2169 4.54051214573425e-08
2170 4.54363549289294e-08
2171 4.5477503701763e-08
2172 4.56127216927893e-08
2173 4.52707362712523e-08
2174 4.53882025126529e-08
2175 4.55396605438807e-08
2176 4.52943667843542e-08
2177 4.52316351451376e-08
2178 4.52206225751706e-08
2179 4.53519855487805e-08
2180 4.54139122663122e-08
2181 4.53920038514521e-08
2182 4.52280437115427e-08
2183 4.53512575913084e-08
2184 4.54232716446956e-08
2185 4.56465465032974e-08
2186 4.51945223707284e-08
2187 4.52941543989116e-08
2188 4.53603716239481e-08
2189 4.53843546550647e-08
2190 4.57974968455233e-08
2191 4.52168170745892e-08
2192 4.52556940910842e-08
2193 4.51755759292016e-08
2194 4.53207252761878e-08
2195 4.52328091593568e-08
2196 4.52816736948858e-08
2197 4.51308670945671e-08
2198 4.51565469343773e-08
2199 4.52901173124687e-08
2200 4.52294524329044e-08
2201 4.52833290343069e-08
2202 4.51419843321421e-08
2203 4.51225218900397e-08
2204 4.53721582054367e-08
2205 4.5190289271746e-08
2206 4.54660019715281e-08
2207 4.49464755853946e-08
2208 4.52118522680234e-08
2209 4.51027405612425e-08
2210 4.51516307615751e-08
2211 4.52605340905965e-08
2212 4.50437535999271e-08
2213 4.51186938166259e-08
2214 4.50363290689371e-08
2215 4.51181390233035e-08
2216 4.53251993102199e-08
2217 4.5291155890137e-08
2218 4.51150939089207e-08
2219 4.51033136563694e-08
2220 4.50671773148947e-08
2221 4.5015281947336e-08
2222 4.51840119790603e-08
2223 4.50744694309169e-08
2224 4.5150842365782e-08
2225 4.4996166436384e-08
2226 4.49243135742794e-08
2227 4.51973202479916e-08
2228 4.52021880843478e-08
2229 4.50321107825502e-08
2230 4.50667011935302e-08
2231 4.49844349914574e-08
2232 4.50900079729255e-08
2233 4.47640744658528e-08
2234 4.51156779448514e-08
2235 4.50008366887911e-08
2236 4.49642369362291e-08
2237 4.4969703822062e-08
2238 4.49530043051816e-08
2239 4.49925863279521e-08
2240 4.48805584114531e-08
2241 4.49535721802619e-08
2242 4.49987586090828e-08
2243 4.48458332502621e-08
2244 4.4851826846104e-08
2245 4.4763231044076e-08
2246 4.50200388413258e-08
2247 4.48722181825012e-08
2248 4.50731743839583e-08
2249 4.50141317449582e-08
2250 4.48530259327118e-08
2251 4.48230965790497e-08
2252 4.47211205614195e-08
2253 4.48252576090713e-08
2254 4.50345914109196e-08
2255 4.48245281077231e-08
2256 4.47905144211891e-08
2257 4.48021180314839e-08
2258 4.4875631471708e-08
2259 4.4944469412167e-08
2260 4.4909303560603e-08
2261 4.46522786583525e-08
2262 4.49639577866368e-08
2263 4.47131666858613e-08
2264 4.47988591880666e-08
2265 4.46840315111441e-08
2266 4.48303770212988e-08
2267 4.49662681343277e-08
2268 4.45526201380275e-08
2269 4.49678848521984e-08
2270 4.47715059210818e-08
2271 4.48278176030659e-08
2272 4.4715654887284e-08
2273 4.48371400305625e-08
2274 4.4750491476675e-08
2275 4.47730444634864e-08
2276 4.48229116640775e-08
2277 4.46318724713901e-08
2278 4.47447973381809e-08
2279 4.48623655142644e-08
2280 4.47629534754412e-08
2281 4.46058803240668e-08
2282 4.4682332504653e-08
2283 4.4773762900352e-08
2284 4.48347736614618e-08
2285 4.47152579874377e-08
2286 4.46962442193222e-08
2287 4.48388918345977e-08
2288 4.45439746501464e-08
2289 4.45857605322342e-08
2290 4.46322387566145e-08
2291 4.46541780143583e-08
2292 4.46067910742087e-08
2293 4.46193770180514e-08
2294 4.46328668359808e-08
2295 4.4597017064163e-08
2296 4.46917165162208e-08
2297 4.45099966397411e-08
2298 4.46500585802312e-08
2299 4.43734450477073e-08
2300 4.4831809300705e-08
2301 4.4510508057094e-08
2302 4.45383341998173e-08
2303 4.44613747605516e-08
2304 4.46703985379493e-08
2305 4.45719406880762e-08
2306 4.4725349832353e-08
2307 4.438765146908e-08
2308 4.45753592299702e-08
2309 4.45284992682815e-08
2310 4.45383224123574e-08
2311 4.451041792497e-08
2312 4.44617240491496e-08
2313 4.47351868408941e-08
2314 4.44912026411615e-08
2315 4.44353582162993e-08
2316 4.43041602322669e-08
2317 4.47011651221185e-08
2318 4.43760647956815e-08
2319 4.47688907505572e-08
2320 4.45516791101053e-08
2321 4.44778998713868e-08
2322 4.47017914011472e-08
2323 4.43530053113594e-08
2324 4.45424973345432e-08
2325 4.45725033269007e-08
2326 4.4347009372725e-08
2327 4.44280983504797e-08
2328 4.44881294991983e-08
2329 4.4284242123771e-08
2330 4.43706591650717e-08
2331 4.45225306728858e-08
2332 4.44065406559613e-08
2333 4.45836743796502e-08
2334 4.42713627166746e-08
2335 4.43635038835932e-08
2336 4.43162004932773e-08
2337 4.43620881744877e-08
2338 4.44087194573228e-08
2339 4.43121117961187e-08
2340 4.42394418371794e-08
2341 4.44570222521978e-08
2342 4.42890623817416e-08
2343 4.44317816030715e-08
2344 4.44024153780642e-08
2345 4.42730442771211e-08
2346 4.43637839524502e-08
2347 4.44525371321447e-08
2348 4.43490693837756e-08
2349 4.42627169050525e-08
2350 4.42423172459794e-08
2351 4.4191631601409e-08
2352 4.44112410309394e-08
2353 4.43054345256044e-08
2354 4.41722358199836e-08
2355 4.42899028350041e-08
2356 4.42538003013748e-08
2357 4.43508918903657e-08
2358 4.40919218738234e-08
2359 4.43003781098739e-08
2360 4.42414677566116e-08
2361 4.43700311441031e-08
2362 4.43084736585497e-08
2363 4.43429407261942e-08
2364 4.42006843917042e-08
2365 4.42214442553102e-08
2366 4.42555618487273e-08
2367 4.43193930461661e-08
2368 4.41773650265986e-08
2369 4.42022209208304e-08
2370 4.41025413022711e-08
2371 4.42401583600205e-08
2372 4.43240974898895e-08
2373 4.42877900728167e-08
2374 4.42431453915315e-08
2375 4.42541741234592e-08
2376 4.4298191864911e-08
2377 4.42510717082456e-08
2378 4.40739128486634e-08
2379 4.41724975923652e-08
2380 4.41861725615222e-08
2381 4.41123148717804e-08
2382 4.41597322642373e-08
2383 4.39288536657756e-08
2384 4.41302286777745e-08
2385 4.42264847038132e-08
2386 4.38711333721198e-08
2387 4.41933806452077e-08
2388 4.41391524659274e-08
2389 4.40830848049689e-08
2390 4.40451770631345e-08
2391 4.42098049562301e-08
2392 4.4252984562787e-08
2393 4.40105314987171e-08
2394 4.41873500511836e-08
2395 4.39371763130136e-08
2396 4.40682821061245e-08
2397 4.41145267298104e-08
2398 4.39228839688166e-08
2399 4.40507534689161e-08
2400 4.3919635926537e-08
2401 4.4291284376774e-08
2402 4.39973455919063e-08
2403 4.388131790245e-08
2404 4.41610108961044e-08
2405 4.39750577865272e-08
2406 4.38594963561023e-08
2407 4.40136082626719e-08
2408 4.4014254804825e-08
2409 4.40393643477144e-08
2410 4.39057777354979e-08
2411 4.38918296141733e-08
2412 4.39479083436112e-08
2413 4.39249778574347e-08
2414 4.38647999934094e-08
2415 4.42065107015832e-08
2416 4.40010767737142e-08
2417 4.37852072641132e-08
2418 4.39602413242479e-08
2419 4.37761450042817e-08
2420 4.40058195279036e-08
2421 4.39380835266512e-08
2422 4.38850494981491e-08
2423 4.37557695645019e-08
2424 4.39611401097384e-08
2425 4.38505020441404e-08
2426 4.37987637655901e-08
2427 4.37177785332921e-08
2428 4.39232756941355e-08
2429 4.37960654304792e-08
2430 4.36991324537228e-08
2431 4.38735002383783e-08
2432 4.3834631319406e-08
2433 4.3662464687122e-08
2434 4.39630114390521e-08
2435 4.37437718463496e-08
2436 4.38571127645559e-08
2437 4.38904903865556e-08
2438 4.37610917305431e-08
2439 4.398620678292e-08
2440 4.3834075587057e-08
2441 4.39294785514743e-08
2442 4.38392691930467e-08
2443 4.37627961153986e-08
2444 4.38205069617226e-08
2445 4.36461714270475e-08
2446 4.3733753839037e-08
2447 4.37432626714251e-08
2448 4.3722236153787e-08
2449 4.37814430449546e-08
2450 4.36964600676326e-08
2451 4.36489499666681e-08
2452 4.37349724531355e-08
2453 4.35894327477371e-08
2454 4.3766191680561e-08
2455 4.3745831298736e-08
2456 4.36865796007435e-08
2457 4.35817517157044e-08
2458 4.37383640130573e-08
2459 4.36415758291453e-08
2460 4.36516589654712e-08
2461 4.36661659233284e-08
2462 4.35744165434571e-08
2463 4.35666136877622e-08
2464 4.36400367547218e-08
2465 4.34673254354756e-08
2466 4.37339868750719e-08
2467 4.36115432258521e-08
2468 4.35809167032009e-08
2469 4.36903371956454e-08
2470 4.35985932127903e-08
2471 4.37273329223498e-08
2472 4.35829480920358e-08
2473 4.35140676089407e-08
2474 4.34341966975005e-08
2475 4.34483956968101e-08
2476 4.36060975097519e-08
2477 4.36160489711845e-08
2478 4.34050632893701e-08
2479 4.35337977504613e-08
2480 4.36789969504581e-08
2481 4.3509068210712e-08
2482 4.36459831771874e-08
2483 4.34603906780406e-08
2484 4.35036468022787e-08
2485 4.3549345387639e-08
2486 4.34451773667632e-08
2487 4.33961681280959e-08
2488 4.34828934818832e-08
2489 4.35442063364544e-08
2490 4.34771105120024e-08
2491 4.34515323157747e-08
2492 4.35376639422724e-08
2493 4.33549625959984e-08
2494 4.35036718453574e-08
2495 4.33633225629393e-08
2496 4.34717605410917e-08
2497 4.32899897155892e-08
2498 4.34690226946977e-08
2499 4.3335148555812e-08
2500 4.3519820489557e-08
2501 4.3390225920259e-08
2502 4.3398435874753e-08
2503 4.34411671312862e-08
2504 4.33168307891751e-08
2505 4.35329904084902e-08
2506 4.33001943431766e-08
2507 4.35009083772364e-08
2508 4.32254403437415e-08
2509 4.34997012488392e-08
2510 4.33254157312657e-08
2511 4.33970199511524e-08
2512 4.33834460735749e-08
2513 4.33618571427097e-08
2514 4.33101965888749e-08
2515 4.33692022190346e-08
2516 4.31850976871928e-08
2517 4.34484183406969e-08
2518 4.31455028173655e-08
2519 4.32320974919076e-08
2520 4.32228371072263e-08
2521 4.34882115079027e-08
2522 4.3269318459993e-08
2523 4.30659902204145e-08
2524 4.33828088342025e-08
2525 4.31290241085236e-08
2526 4.33737870821904e-08
2527 4.3129789263574e-08
2528 4.3246262594554e-08
2529 4.32685127724763e-08
2530 4.31405297767196e-08
2531 4.32771816873956e-08
2532 4.34265541175982e-08
2533 4.31831660347992e-08
2534 4.30790884955456e-08
2535 4.31267364495724e-08
2536 4.32248241248878e-08
2537 4.31894528698518e-08
2538 4.30504724058256e-08
2539 4.33643568895636e-08
2540 4.32671381058736e-08
2541 4.3212855157293e-08
2542 4.29999558750893e-08
2543 4.32246327877195e-08
2544 4.3135933873506e-08
2545 4.31807882552704e-08
2546 4.31089721704136e-08
2547 4.32304578292797e-08
2548 4.33329655660231e-08
2549 4.30471978638547e-08
2550 4.31225483781272e-08
2551 4.31307291812733e-08
2552 4.31528849134111e-08
2553 4.30364681940354e-08
2554 4.31416561006426e-08
2555 4.32005296051585e-08
2556 4.29743517131875e-08
2557 4.31659283453012e-08
2558 4.28826820018102e-08
2559 4.31127786100216e-08
2560 4.30844127508045e-08
2561 4.3074335306148e-08
2562 4.2887439786421e-08
2563 4.31697002156728e-08
2564 4.28651027699622e-08
2565 4.31044941968661e-08
2566 4.28505581078209e-08
2567 4.30851325652348e-08
2568 4.2994670705232e-08
2569 4.29312661744952e-08
2570 4.30073204484849e-08
2571 4.29037053741865e-08
2572 4.30595451481608e-08
2573 4.2943645560678e-08
2574 4.28911425736089e-08
2575 4.30253047549734e-08
2576 4.30518255558621e-08
2577 4.28563962067141e-08
2578 4.31012881891846e-08
2579 4.32333487487835e-08
2580 4.30234523733919e-08
2581 4.28367280118636e-08
2582 4.30261347710292e-08
2583 4.29018177641183e-08
2584 4.28359630573194e-08
2585 4.30302422564921e-08
2586 4.28082530807838e-08
2587 4.30533975592784e-08
2588 4.27541426462064e-08
2589 4.29541895077268e-08
2590 4.28185412200133e-08
2591 4.29861421464839e-08
2592 4.28348061727313e-08
2593 4.28556797922308e-08
2594 4.28367904672378e-08
2595 4.28132919660928e-08
2596 4.28989055671902e-08
2597 4.27505087832181e-08
2598 4.28426874696797e-08
2599 4.27847191832864e-08
2600 4.29446382701482e-08
2601 4.27080201574892e-08
2602 4.29078902763891e-08
2603 4.28445640050512e-08
2604 4.28999663646401e-08
2605 4.28634868243627e-08
2606 4.28743129909925e-08
2607 4.27561419780176e-08
2608 4.28263878462509e-08
2609 4.27639861559914e-08
2610 4.28485263797018e-08
2611 4.2705320008718e-08
2612 4.28206807103759e-08
2613 4.27278915102747e-08
2614 4.27449832514082e-08
2615 4.26292990209909e-08
2616 4.28598571875494e-08
2617 4.26077800366187e-08
2618 4.26805902857819e-08
2619 4.28418532665287e-08
2620 4.26076370454442e-08
2621 4.27550809700694e-08
2622 4.27638792050988e-08
2623 4.26494591079241e-08
2624 4.28886162577857e-08
2625 4.26410767382368e-08
2626 4.2667977698585e-08
2627 4.25816886000518e-08
2628 4.27956647799377e-08
2629 4.26212598916287e-08
2630 4.27130545945786e-08
2631 4.25740246077222e-08
2632 4.25868340527824e-08
2633 4.25045973910798e-08
2634 4.26293795168231e-08
2635 4.28251954700531e-08
2636 4.2628848455406e-08
2637 4.2581056515445e-08
2638 4.25905649468206e-08
2639 4.26056078721704e-08
2640 4.24560648251138e-08
2641 4.24757302244227e-08
2642 4.24936659251252e-08
2643 4.25785605151496e-08
2644 4.24205893723162e-08
2645 4.25375470878286e-08
2646 4.25285139047382e-08
2647 4.24156821396959e-08
2648 4.24151400086892e-08
2649 4.25167958424399e-08
2650 4.25791177551815e-08
2651 4.24787105883784e-08
2652 4.24183968397163e-08
2653 4.25351009418762e-08
2654 4.24998151522527e-08
2655 4.25644913755185e-08
2656 4.23104255329321e-08
2657 4.23419286719184e-08
2658 4.23721078930406e-08
2659 4.23924826291611e-08
2660 4.24753155967572e-08
2661 4.25282251534931e-08
2662 4.24371322300665e-08
2663 4.23466023542485e-08
2664 4.24524660227466e-08
2665 4.24125273297271e-08
2666 4.25696008989895e-08
2667 4.23228072687909e-08
2668 4.24484515886281e-08
2669 4.23102240940665e-08
2670 4.23458679403854e-08
2671 4.22820567391735e-08
2672 4.22900860175268e-08
2673 4.22852178205702e-08
2674 4.22956679178021e-08
2675 4.23471214809901e-08
2676 4.24978137372189e-08
2677 4.22921374427787e-08
2678 4.2467268467572e-08
2679 4.22418377392653e-08
2680 4.22724079762737e-08
2681 4.22802805406519e-08
2682 4.22362672423127e-08
2683 4.23078827183065e-08
2684 4.22306411771434e-08
2685 4.22330224527645e-08
2686 4.23002290723673e-08
2687 4.23584254150011e-08
2688 4.20705785690867e-08
2689 4.21818656732853e-08
2690 4.22434323106291e-08
2691 4.22469244245605e-08
2692 4.23108884113343e-08
2693 4.20262149978257e-08
2694 4.21087712423329e-08
2695 4.22405492288558e-08
2696 4.21130276246906e-08
2697 4.20234660958307e-08
2698 4.22155093704468e-08
2699 4.22283302676796e-08
2700 4.21441249442989e-08
2701 4.20745299760661e-08
2702 4.21790135571864e-08
2703 4.19730708667299e-08
2704 4.23933220727868e-08
2705 4.20255212174592e-08
2706 4.21303965731923e-08
2707 4.20025208303088e-08
2708 4.20654860104897e-08
2709 4.19858040625609e-08
2710 4.20640241358683e-08
2711 4.20721945968428e-08
2712 4.21081930603862e-08
2713 4.21180130545373e-08
2714 4.21482906816095e-08
2715 4.19756678926575e-08
2716 4.19800006299376e-08
2717 4.20121476440993e-08
2718 4.21077760117683e-08
2719 4.20362768183136e-08
2720 4.22351243229979e-08
2721 4.22067896836964e-08
2722 4.19372000952656e-08
2723 4.21253655686904e-08
2724 4.20196409554041e-08
2725 4.19379031408873e-08
2726 4.21294168337916e-08
2727 4.19597811236461e-08
2728 4.19932298021397e-08
2729 4.20000418548394e-08
2730 4.2112345054468e-08
2731 4.20252438984026e-08
2732 4.19456900480064e-08
2733 4.2070328081012e-08
2734 4.2004069391588e-08
2735 4.20630153314949e-08
2736 4.19985320985194e-08
2737 4.20394944531388e-08
2738 4.19519854184536e-08
2739 4.20279459354411e-08
2740 4.1899608596685e-08
2741 4.20249386154925e-08
2742 4.19364232004948e-08
2743 4.19562084779113e-08
2744 4.18234714290033e-08
2745 4.19485051736945e-08
2746 4.19080673450534e-08
2747 4.18832803448233e-08
2748 4.18364681318284e-08
2749 4.1903237221641e-08
2750 4.19259979071995e-08
2751 4.18956955050564e-08
2752 4.18412857423611e-08
2753 4.18608301009371e-08
2754 4.18162851245629e-08
2755 4.18434475817353e-08
2756 4.18201285832254e-08
2757 4.19114769456552e-08
2758 4.1809281451588e-08
2759 4.18101330703635e-08
2760 4.18764411218575e-08
2761 4.1646737568346e-08
2762 4.17714012415082e-08
2763 4.16759613450779e-08
2764 4.17408138442266e-08
2765 4.18391050602818e-08
2766 4.17453871284224e-08
2767 4.18096827738967e-08
2768 4.17436525392834e-08
2769 4.17585243532859e-08
2770 4.17240994958767e-08
2771 4.17086258748256e-08
2772 4.16608608337921e-08
2773 4.16741815298938e-08
2774 4.17409740167685e-08
2775 4.16869910588868e-08
2776 4.16910277756255e-08
2777 4.17164583688123e-08
2778 4.16789493957737e-08
2779 4.15616017019893e-08
2780 4.16601913819648e-08
2781 4.15707475669436e-08
2782 4.17024863261783e-08
2783 4.15643184781267e-08
2784 4.16319588536496e-08
2785 4.16112928189882e-08
2786 4.16292877691848e-08
2787 4.15448363042525e-08
2788 4.16157083269475e-08
2789 4.15327630172158e-08
2790 4.16168269352646e-08
2791 4.15515886249551e-08
2792 4.15671045455035e-08
2793 4.16385892358928e-08
2794 4.15795863595303e-08
2795 4.15191692992867e-08
2796 4.15204369914601e-08
2797 4.15592618507876e-08
2798 4.15241522515597e-08
2799 4.15541555907062e-08
2800 4.14936846431235e-08
2801 4.15446303569933e-08
2802 4.14423355521443e-08
2803 4.15028484801461e-08
2804 4.13875351519266e-08
2805 4.15384393439577e-08
2806 4.14301099711523e-08
2807 4.15287521942087e-08
2808 4.14219682964845e-08
2809 4.16221176069431e-08
2810 4.13411441495537e-08
2811 4.14060786115833e-08
2812 4.14654703351225e-08
2813 4.14499127392443e-08
2814 4.1524792621761e-08
2815 4.13763319513372e-08
2816 4.14298990747408e-08
2817 4.13849543379197e-08
2818 4.13391237987781e-08
2819 4.14458930306783e-08
2820 4.13884993089031e-08
2821 4.13730549151392e-08
2822 4.13621432606703e-08
2823 4.14021156596167e-08
2824 4.1328227709192e-08
2825 4.1360601199969e-08
2826 4.13719691605419e-08
2827 4.13625871957812e-08
2828 4.13119251301275e-08
2829 4.1317218345549e-08
2830 4.13339260794121e-08
2831 4.12860413678828e-08
2832 4.13696091330351e-08
2833 4.13190805221841e-08
2834 4.1278126541533e-08
2835 4.1243517054923e-08
2836 4.1253163434618e-08
2837 4.12342440816449e-08
2838 4.12051714153705e-08
2839 4.12500318636155e-08
2840 4.1275298299448e-08
2841 4.11974975835783e-08
2842 4.12252143517922e-08
2843 4.11973863099213e-08
2844 4.12302889030158e-08
2845 4.11813294234342e-08
2846 4.12299112615422e-08
2847 4.12036522305925e-08
2848 4.12024941283118e-08
2849 4.1145166357337e-08
2850 4.12057823169221e-08
2851 4.11611768507125e-08
2852 4.12357149257581e-08
2853 4.11304637759624e-08
2854 4.12530909419395e-08
2855 4.10989836439057e-08
2856 4.11058738460923e-08
2857 4.11274516427529e-08
2858 4.11011550971452e-08
2859 4.11363418306721e-08
2860 4.11668838271595e-08
2861 4.10630720677485e-08
2862 4.11040624364034e-08
2863 4.10427992347806e-08
2864 4.10744042855793e-08
2865 4.1097369554377e-08
2866 4.10054455524111e-08
2867 4.09832455554593e-08
2868 4.11000989684052e-08
2869 4.10543590518841e-08
2870 4.0960475706564e-08
2871 4.10511988748752e-08
2872 4.10476342687094e-08
2873 4.10417025165088e-08
2874 4.10362377945006e-08
2875 4.09693163561098e-08
2876 4.09300903840837e-08
2877 4.10113478357399e-08
2878 4.10496301708196e-08
2879 4.0930906105352e-08
2880 4.10097583052327e-08
2881 4.09631287390244e-08
2882 4.09130316192385e-08
2883 4.09726077723604e-08
2884 4.09320799525936e-08
2885 4.09843024804513e-08
2886 4.09615596086432e-08
2887 4.0869323179038e-08
2888 4.08887854679296e-08
2889 4.08554153235041e-08
2890 4.10314327816685e-08
2891 4.08108414042818e-08
2892 4.08959802740139e-08
2893 4.07801838895328e-08
2894 4.09378358050549e-08
2895 4.07838710194675e-08
2896 4.09478470690949e-08
2897 4.08553608344242e-08
2898 4.08550460990753e-08
2899 4.08094625543587e-08
2900 4.09716187165232e-08
2901 4.08090136578831e-08
2902 4.08658354680291e-08
2903 4.09263727927378e-08
2904 4.06794354792517e-08
2905 4.0894471245112e-08
2906 4.08238855815757e-08
2907 4.06482711339251e-08
2908 4.10007361462128e-08
2909 4.06979853988521e-08
2910 4.07446961172564e-08
2911 4.0662697934124e-08
2912 4.08195244263609e-08
2913 4.0745882434301e-08
2914 4.07357808220876e-08
2915 4.06348824226743e-08
2916 4.06420931988727e-08
2917 4.08005498175879e-08
2918 4.06781669517464e-08
2919 4.07146691225879e-08
2920 4.06635877785444e-08
2921 4.06515134403485e-08
2922 4.071596378874e-08
2923 4.07776042421482e-08
2924 4.06619492889782e-08
2925 4.08890452074928e-08
2926 4.06538549162505e-08
2927 4.06171562239432e-08
2928 4.05863549224339e-08
2929 4.06253838345361e-08
2930 4.06655018108104e-08
2931 4.05036844095541e-08
2932 4.05328270551841e-08
2933 4.06199060838386e-08
2934 4.06408788267143e-08
2935 4.05827344942544e-08
2936 4.05774646583534e-08
2937 4.05439174397948e-08
2938 4.05705637802622e-08
2939 4.05192754813211e-08
2940 4.04930829851491e-08
2941 4.05450058948986e-08
2942 4.05656202295823e-08
2943 4.06053349699675e-08
2944 4.05590619241281e-08
2945 4.0474877501584e-08
2946 4.05063541042416e-08
2947 4.06613160011116e-08
2948 4.04966602303158e-08
2949 4.04943540550651e-08
2950 4.04732045891087e-08
2951 4.04790597592353e-08
2952 4.04524474935108e-08
2953 4.0489667748167e-08
2954 4.04844170358576e-08
2955 4.04846429402639e-08
2956 4.04264362983131e-08
2957 4.05386953792331e-08
2958 4.0435162842245e-08
2959 4.04143235597232e-08
2960 4.03971901881128e-08
2961 4.04430544407308e-08
2962 4.04318787505797e-08
2963 4.04450017041924e-08
2964 4.03497296324051e-08
2965 4.03986099537512e-08
2966 4.03619201616578e-08
2967 4.03518682303705e-08
2968 4.03425092436738e-08
2969 4.03983134369401e-08
2970 4.03235617789655e-08
2971 4.03376171385084e-08
2972 4.0241655002049e-08
2973 4.03119719369904e-08
2974 4.02970612414233e-08
2975 4.03173518901401e-08
2976 4.02485731674229e-08
2977 4.02875732135044e-08
2978 4.0291702237738e-08
2979 4.0267384969761e-08
2980 4.02578863192282e-08
2981 4.04372682238918e-08
2982 4.02390204021774e-08
2983 4.01634499038739e-08
2984 4.01625600658928e-08
2985 4.01305743351621e-08
2986 4.01202257880318e-08
2987 4.01698148406737e-08
2988 4.02786422399881e-08
2989 4.01490779677882e-08
2990 4.01688815891976e-08
2991 4.00839463641223e-08
2992 4.01731222348101e-08
2993 4.00895705006121e-08
2994 4.00773315853975e-08
2995 4.00954060943981e-08
2996 4.00427712627938e-08
2997 4.01198105559608e-08
2998 4.0029786712914e-08
2999 4.00128394204557e-08
3000 2.83502551693821e-08
3001 2.83863429029751e-08
3002 2.84979097692623e-08
3003 2.85550524515565e-08
3004 2.85754265546556e-08
3005 2.85798928767456e-08
3006 2.85791351113396e-08
3007 2.85768848633516e-08
3008 2.85741108296822e-08
3009 2.85714927406311e-08
3010 2.85686942478325e-08
3011 2.8565972056277e-08
3012 2.85633159889387e-08
3013 2.85610413976456e-08
3014 2.85582136141937e-08
3015 2.85555907215962e-08
3016 2.85529650184968e-08
3017 2.85507644663952e-08
3018 2.85478560281549e-08
3019 2.85454124336892e-08
3020 2.85427493492141e-08
3021 2.85404304613546e-08
3022 2.85376662892967e-08
3023 2.8535363046478e-08
3024 2.85326115989082e-08
3025 2.85301933090065e-08
3026 2.85278038201497e-08
3027 2.85252622693466e-08
3028 2.85226023829244e-08
3029 2.85202917634508e-08
3030 2.8517929230476e-08
3031 2.8515207765728e-08
3032 2.85128251749367e-08
3033 2.85104371548772e-08
3034 2.85078898364377e-08
3035 2.85057240811959e-08
3036 2.85029445344642e-08
3037 2.85005752054812e-08
3038 2.84982406976719e-08
3039 2.84955611856152e-08
3040 2.84931141266986e-08
3041 2.84905718737904e-08
3042 2.84880352125483e-08
3043 2.84858316670356e-08
3044 2.84831664640328e-08
3045 2.84806643116586e-08
3046 2.84786338412835e-08
3047 2.8475989690524e-08
3048 2.84733126194869e-08
3049 2.84707892534708e-08
3050 2.84684163431026e-08
3051 2.84658251000491e-08
3052 2.84636592654541e-08
3053 2.84610012775965e-08
3054 2.84584789063402e-08
3055 2.84560726694916e-08
3056 2.84534145119364e-08
3057 2.84509794256094e-08
3058 2.84486165358644e-08
3059 2.84461344368103e-08
3060 2.84429053962199e-08
3061 2.84403175389303e-08
3062 2.84379082943209e-08
3063 2.84352990699011e-08
3064 2.84327220944902e-08
3065 2.84302235976919e-08
3066 2.84275669168443e-08
3067 2.84251318350692e-08
3068 2.84224854780468e-08
3069 2.84201527931127e-08
3070 2.84174665480252e-08
3071 2.84150968183627e-08
3072 2.84124361936977e-08
3073 2.84098725816528e-08
3074 2.84076286069246e-08
3075 2.84049793251029e-08
3076 2.84026025229311e-08
3077 2.84001031233827e-08
3078 2.83971845711217e-08
3079 2.83951489451484e-08
3080 2.83927557812036e-08
3081 2.83895986449845e-08
3082 2.8387584830647e-08
3083 2.83852008089447e-08
3084 2.83820491005782e-08
3085 2.8379877825363e-08
3086 2.83773047138613e-08
3087 2.83748748770574e-08
3088 2.83723290137317e-08
3089 2.83693159344178e-08
3090 2.83669346506366e-08
3091 2.83645822034828e-08
3092 2.83622959322549e-08
3093 2.83596918332907e-08
3094 2.83569059018607e-08
3095 2.83544278772974e-08
3096 2.83519336388427e-08
3097 2.83493816570501e-08
3098 2.83468824937294e-08
3099 2.83443674290729e-08
3100 2.83417263224339e-08
3101 2.83392506361668e-08
3102 2.83365782942635e-08
3103 2.83339780549063e-08
3104 2.83314622032682e-08
3105 2.83288022591699e-08
3106 2.83260956912834e-08
3107 2.83236177663071e-08
3108 2.83210760795849e-08
3109 2.83185458018476e-08
3110 2.83159624569207e-08
3111 2.83132946655162e-08
3112 2.83106727780591e-08
3113 2.83078251133717e-08
3114 2.83051598234663e-08
3115 2.83027903819066e-08
3116 2.82998692263392e-08
3117 2.82973980087198e-08
3118 2.82948022079232e-08
3119 2.82917546347572e-08
3120 2.82892359342035e-08
3121 2.82866761801559e-08
3122 2.82840018107633e-08
3123 2.8281481133291e-08
3124 2.82786541434266e-08
3125 2.82761012550259e-08
3126 2.82735688756364e-08
3127 2.82708089455463e-08
3128 2.82683598926414e-08
3129 2.82654086874168e-08
3130 2.82628020871756e-08
3131 2.8260096056526e-08
3132 2.82579984838827e-08
3133 2.82547345155648e-08
3134 2.82520331898461e-08
3135 2.82493699907127e-08
3136 2.82468568645333e-08
3137 2.82440950881979e-08
3138 2.82415713820372e-08
3139 2.82386468281048e-08
3140 2.82359613233973e-08
3141 2.82334373684634e-08
3142 2.82305377947711e-08
3143 2.82280025527826e-08
3144 2.82250609280055e-08
3145 2.82224369123896e-08
3146 2.82197697408504e-08
3147 2.82169877516558e-08
3148 2.82141006341385e-08
3149 2.82114326627669e-08
3150 2.8208926503015e-08
3151 2.82059159636971e-08
3152 2.8203348691358e-08
3153 2.82006058954787e-08
3154 2.81978890673829e-08
3155 2.81950488563165e-08
3156 2.81921285512077e-08
3157 2.81895351707528e-08
3158 2.81864150801225e-08
3159 2.8183744732152e-08
3160 2.81809220023521e-08
3161 2.81781687247462e-08
3162 2.81750540331582e-08
3163 2.81726893746337e-08
3164 2.81696340666548e-08
3165 2.81670845860005e-08
3166 2.81638571687504e-08
3167 2.8161423022699e-08
3168 2.81584123330847e-08
3169 2.81556920042336e-08
3170 2.81530289777676e-08
3171 2.81500681454938e-08
3172 2.81473694031253e-08
3173 2.81441324897769e-08
3174 2.81415668073048e-08
3175 2.81385432254477e-08
3176 2.81357550959149e-08
3177 2.81327143276389e-08
3178 2.81301642879039e-08
3179 2.81271879367662e-08
3180 2.81243075186954e-08
3181 2.81211538222526e-08
3182 2.8118619028461e-08
3183 2.81156541451777e-08
3184 2.81126229349504e-08
3185 2.81096169164885e-08
3186 2.8106843468545e-08
3187 2.81037299301734e-08
3188 2.81009359178463e-08
3189 2.80981926243373e-08
3190 2.80950774654565e-08
3191 2.80922391997229e-08
3192 2.80893232412205e-08
3193 2.80865996347135e-08
3194 2.80835060314233e-08
3195 2.80805221077873e-08
3196 2.80774795680672e-08
3197 2.80748151957333e-08
3198 2.8071897572729e-08
3199 2.80684891652283e-08
3200 2.80658310912174e-08
3201 2.80624161962117e-08
3202 2.80594146311042e-08
3203 2.80568607436138e-08
3204 2.80535534783466e-08
3205 2.80507701028165e-08
3206 2.80476021153608e-08
3207 2.80445076005775e-08
3208 2.8042133811107e-08
3209 2.80386542302891e-08
3210 2.80355604216076e-08
3211 2.80323950063166e-08
3212 2.80296757848852e-08
3213 2.80263710619455e-08
3214 2.8023487832679e-08
3215 2.80204066448464e-08
3216 2.80176514385888e-08
3217 2.80141228510555e-08
3218 2.80109334068745e-08
3219 2.8008396590673e-08
3220 2.80049083857259e-08
3221 2.800190396271e-08
3222 2.79989114724821e-08
3223 2.7995663564484e-08
3224 2.7992683451078e-08
3225 2.79892129288195e-08
3226 2.79864830281251e-08
3227 2.7983058950215e-08
3228 2.7979980360443e-08
3229 2.79770366237997e-08
3230 2.79738841694188e-08
3231 2.7970808909733e-08
3232 2.79674781328287e-08
3233 2.79646147835488e-08
3234 2.79612278829833e-08
3235 2.7958288432689e-08
3236 2.79549545484092e-08
3237 2.79520611642659e-08
3238 2.79485100487531e-08
3239 2.79455561427722e-08
3240 2.79420656224272e-08
3241 2.79390623790787e-08
3242 2.79357131102165e-08
3243 2.79326924146339e-08
3244 2.79292761213856e-08
3245 2.79259745405991e-08
3246 2.79228662073694e-08
3247 2.79195143862432e-08
3248 2.7916407865175e-08
3249 2.79129623264041e-08
3250 2.79098352162221e-08
3251 2.79065822368918e-08
3252 2.79033237740034e-08
3253 2.78999927848245e-08
3254 2.78966715218654e-08
3255 2.78934378041829e-08
3256 2.78899504035646e-08
3257 2.78868391944409e-08
3258 2.78832429167764e-08
3259 2.78802065550032e-08
3260 2.78765032994954e-08
3261 2.78735223880888e-08
3262 2.78698794432541e-08
3263 2.78665642866605e-08
3264 2.78631413683506e-08
3265 2.78600077251723e-08
3266 2.78564900207945e-08
3267 2.78531075633137e-08
3268 2.78495585092908e-08
3269 2.78463471931889e-08
3270 2.78427960911376e-08
3271 2.78394290585682e-08
3272 2.78359120532701e-08
3273 2.78329221923002e-08
3274 2.78289651332864e-08
3275 2.78258010034949e-08
3276 2.78222689687746e-08
3277 2.78188672118984e-08
3278 2.78154182647428e-08
3279 2.78117970864133e-08
3280 2.78083339940172e-08
3281 2.78051293486348e-08
3282 2.78012729289512e-08
3283 2.77981100963998e-08
3284 2.7794508410256e-08
3285 2.77910864189823e-08
3286 2.77873880151491e-08
3287 2.77840957210451e-08
3288 2.77800196766542e-08
3289 2.77766855290573e-08
3290 2.77731718925645e-08
3291 2.77696711196984e-08
3292 2.7765746151126e-08
3293 2.77623955472761e-08
3294 2.77587158968862e-08
3295 2.77551915685292e-08
3296 2.77516845103853e-08
3297 2.77482002086771e-08
3298 2.77445377704633e-08
3299 2.77411685240814e-08
3300 2.77373842531847e-08
3301 2.77338053173704e-08
3302 2.77301246556505e-08
3303 2.77265768658663e-08
3304 2.77225307939999e-08
3305 2.77193671015252e-08
3306 2.77156244939958e-08
3307 2.77123001129198e-08
3308 2.77082734987666e-08
3309 2.77047697083699e-08
3310 2.77008044940186e-08
3311 2.76974189543922e-08
3312 2.76936665208338e-08
3313 2.76898347285059e-08
3314 2.76862054338634e-08
3315 2.76830008070494e-08
3316 2.7678883880139e-08
3317 2.76753670742091e-08
3318 2.76714559133417e-08
3319 2.76680211037661e-08
3320 2.76636838189848e-08
3321 2.76606648429989e-08
3322 2.7656475864557e-08
3323 2.7652738038425e-08
3324 2.76491467432471e-08
3325 2.76454558606753e-08
3326 2.76413666231989e-08
3327 2.76379095254242e-08
3328 2.76338627893391e-08
3329 2.76303936711519e-08
3330 2.76264083386712e-08
3331 2.76229006242468e-08
3332 2.76189003333427e-08
3333 2.76150553682186e-08
3334 2.76110998014001e-08
3335 2.76072933559912e-08
3336 2.76035524835738e-08
3337 2.75997200731848e-08
3338 2.75957559225937e-08
3339 2.75920774787941e-08
3340 2.75881155872293e-08
3341 2.75844225372524e-08
3342 2.75803902039795e-08
3343 2.75766571090352e-08
3344 2.75725711642028e-08
3345 2.75687956256543e-08
3346 2.75647861314454e-08
3347 2.75609533197385e-08
3348 2.75567250009601e-08
3349 2.75530202687169e-08
3350 2.75490564549397e-08
3351 2.75451835298479e-08
3352 2.75411336779496e-08
3353 2.75373993318673e-08
3354 2.75331776776189e-08
3355 2.75292942205074e-08
3356 2.75252787917957e-08
3357 2.75213201181845e-08
3358 2.75172647305755e-08
3359 2.75135300212281e-08
3360 2.75088859649775e-08
3361 2.7505409006312e-08
3362 2.75012615520154e-08
3363 2.74975334141336e-08
3364 2.74931119283173e-08
3365 2.74892050124154e-08
3366 2.74850143872629e-08
3367 2.74811876290471e-08
3368 2.74771629174275e-08
3369 2.74729177402733e-08
3370 2.74688769499043e-08
3371 2.74651380350321e-08
3372 2.74602319020323e-08
3373 2.7456645193713e-08
3374 2.74524176421265e-08
3375 2.74482717075586e-08
3376 2.74440191114778e-08
3377 2.74402718279942e-08
3378 2.74360840431531e-08
3379 2.74318800842777e-08
3380 2.74276105012572e-08
3381 2.7423498907303e-08
3382 2.74192109583904e-08
3383 2.74150829735198e-08
3384 2.74111009092026e-08
3385 2.74069458302273e-08
3386 2.74025842231518e-08
3387 2.73983851053483e-08
3388 2.73943226488105e-08
3389 2.7390105947378e-08
3390 2.73857314912196e-08
3391 2.73815754261442e-08
3392 2.7377640106685e-08
3393 2.7372961741573e-08
3394 2.73687899623176e-08
3395 2.73647776627972e-08
3396 2.73601414859659e-08
3397 2.73561252182586e-08
3398 2.73520571884567e-08
3399 2.73472580293865e-08
3400 2.73432451760591e-08
3401 2.73389293656556e-08
3402 2.73344466518721e-08
3403 2.73302218783689e-08
3404 2.73257463444809e-08
3405 2.73216968125489e-08
3406 2.73172005798017e-08
3407 2.73129370239877e-08
3408 2.73084538446045e-08
3409 2.73043999561851e-08
3410 2.72999262316553e-08
3411 2.72955824124621e-08
3412 2.7291326603035e-08
3413 2.72866951125106e-08
3414 2.72823388146604e-08
3415 2.72779038500981e-08
3416 2.72734812724607e-08
3417 2.72690702816658e-08
3418 2.72647752366517e-08
3419 2.72603157925599e-08
3420 2.72557549059749e-08
3421 2.72513211915237e-08
3422 2.72469128826669e-08
3423 2.72424220420231e-08
3424 2.72380153785445e-08
3425 2.72335194777262e-08
3426 2.72290555562438e-08
3427 2.72242150181345e-08
3428 2.72142569469058e-08
3429 2.71883496902048e-08
3430 2.71753869856284e-08
3431 2.71662512489057e-08
3432 2.71584990971241e-08
3433 2.71521362874283e-08
3434 2.7145311283755e-08
3435 2.7138748225547e-08
3436 2.71326542981842e-08
3437 2.71263676375755e-08
3438 2.71197359800746e-08
3439 2.71140719320662e-08
3440 2.70995455005085e-08
3441 2.70919425749694e-08
3442 2.70841772334884e-08
3443 2.7077536250808e-08
3444 2.70708904728523e-08
3445 2.70643635885137e-08
3446 2.705789760718e-08
3447 2.70515451510078e-08
3448 2.70450205575035e-08
3449 2.70387642539449e-08
3450 2.70327517137292e-08
3451 2.70265394623215e-08
3452 2.70203777691103e-08
3453 2.70143967246728e-08
3454 2.70081769689901e-08
3455 2.70021133652043e-08
3456 2.69960662065971e-08
3457 2.69900114439892e-08
3458 2.69840646907604e-08
3459 2.69777552162898e-08
3460 2.69717434192018e-08
3461 2.69657612129437e-08
3462 2.69600596988773e-08
3463 2.69534578620778e-08
3464 2.69477648195959e-08
3465 2.69417677145056e-08
3466 2.6935584225507e-08
3467 2.69295334937025e-08
3468 2.69238452818565e-08
3469 2.69176223055279e-08
3470 2.69115520637186e-08
3471 2.69051206313875e-08
3472 2.68993502155757e-08
3473 2.68933107664682e-08
3474 2.68872204058357e-08
3475 2.68811661470469e-08
3476 2.68751069484874e-08
3477 2.6868908751837e-08
3478 2.68630761773303e-08
3479 2.68561534513267e-08
3480 2.68507946199481e-08
3481 2.68444885516139e-08
3482 2.68384403152577e-08
3483 2.68327483003428e-08
3484 2.68262855822043e-08
3485 2.68204514631276e-08
3486 2.6814062528524e-08
3487 2.68083034289379e-08
3488 2.68021910272143e-08
3489 2.67960528370081e-08
3490 2.67896274962487e-08
3491 2.67837469124232e-08
3492 2.6777493517316e-08
3493 2.67716980643373e-08
3494 2.67654033089537e-08
3495 2.67594732716314e-08
3496 2.67528590006116e-08
3497 2.67470195033093e-08
3498 2.67407752003734e-08
3499 2.67346849731065e-08
3500 2.67283114097072e-08
3501 2.67224482247419e-08
3502 2.6716109627567e-08
3503 2.67096939292888e-08
3504 2.67038816877674e-08
3505 2.66974852777269e-08
3506 2.66914259220152e-08
3507 2.66851189911488e-08
3508 2.66790044562426e-08
3509 2.66726022467301e-08
3510 2.66661318198691e-08
3511 2.66604425892547e-08
3512 2.6653938763388e-08
3513 2.66478731883457e-08
3514 2.66412370502345e-08
3515 2.66353832962252e-08
3516 2.66287338568649e-08
3517 2.66225803552955e-08
3518 2.66164189862694e-08
3519 2.6610051257564e-08
3520 2.66037052015777e-08
3521 2.65974808249803e-08
3522 2.65910961594895e-08
3523 2.65852856350113e-08
3524 2.65783646055395e-08
3525 2.65717585620218e-08
3526 2.6565866655337e-08
3527 2.65593279285825e-08
3528 2.65531181284917e-08
3529 2.65468732638663e-08
3530 2.65406100445065e-08
3531 2.65341150648413e-08
3532 2.65276383807633e-08
3533 2.65212643987822e-08
3534 2.65150633922684e-08
3535 2.65083864702531e-08
3536 2.65021281859457e-08
3537 2.64958520898584e-08
3538 2.64892051097809e-08
3539 2.64827955669678e-08
3540 2.64766226449054e-08
3541 2.64700987661071e-08
3542 2.64631913466651e-08
3543 2.64572022241061e-08
3544 2.64503320731579e-08
3545 2.64441840346907e-08
3546 2.64373115632377e-08
3547 2.64310085761332e-08
3548 2.64242131166048e-08
3549 2.64180526091395e-08
3550 2.64116114393476e-08
3551 2.64049741255379e-08
3552 2.63984984547327e-08
3553 2.63922369634628e-08
3554 2.63855276452507e-08
3555 2.6378848894143e-08
3556 2.63722143759304e-08
3557 2.63657291116326e-08
3558 2.63592907425447e-08
3559 2.63529744756463e-08
3560 2.6346016089418e-08
3561 2.63396884742362e-08
3562 2.63331111892695e-08
3563 2.63261914610624e-08
3564 2.63200017906473e-08
3565 2.63130266061617e-08
3566 2.63064507017019e-08
3567 2.62997805628162e-08
3568 2.62934554863536e-08
3569 2.62863893715215e-08
3570 2.62800364802807e-08
3571 2.62732243364039e-08
3572 2.62665929812722e-08
3573 2.62601096643056e-08
3574 2.62533569588275e-08
3575 2.62463556728043e-08
3576 2.62399481159581e-08
3577 2.62333778826562e-08
3578 2.62267029403629e-08
3579 2.62197928543317e-08
3580 2.62131218001227e-08
3581 2.62067477242445e-08
3582 2.6199665512372e-08
3583 2.61929880246148e-08
3584 2.61864817598101e-08
3585 2.61795323956704e-08
3586 2.61731886365413e-08
3587 2.61661139442926e-08
3588 2.61597644589107e-08
3589 2.61528590138893e-08
3590 2.61460947322545e-08
3591 2.61392817684503e-08
3592 2.61324532588869e-08
3593 2.61258208702764e-08
3594 2.61188487922226e-08
3595 2.61120317431862e-08
3596 2.61054298741903e-08
3597 2.60984508450579e-08
3598 2.60918571404756e-08
3599 2.60849296970511e-08
3600 2.60783194952763e-08
3601 2.60712099582061e-08
3602 2.60646185511748e-08
3603 2.60575108961825e-08
3604 2.60506749721334e-08
3605 2.60439185980899e-08
3606 2.60375176881489e-08
3607 2.60300340635122e-08
3608 2.60235417334498e-08
3609 2.60163441337946e-08
3610 2.60096285560063e-08
3611 2.60027988755462e-08
3612 2.59960316764118e-08
3613 2.59889152691151e-08
3614 2.5982239707123e-08
3615 2.59750149291249e-08
3616 2.59683059008475e-08
3617 2.59612859707936e-08
3618 2.59546911444419e-08
3619 2.59474238292201e-08
3620 2.59407321753502e-08
3621 2.59338954857746e-08
3622 2.59267196826563e-08
3623 2.59199614608963e-08
3624 2.59127802763326e-08
3625 2.59060243506248e-08
3626 2.58989533120091e-08
3627 2.58920404089535e-08
3628 2.58850892989881e-08
3629 2.58781664289887e-08
3630 2.58711072867457e-08
3631 2.58573592776257e-08
3632 2.58480505642145e-08
3633 2.58409497489454e-08
3634 2.58335039132751e-08
3635 2.58256371168109e-08
3636 2.58180531762242e-08
3637 2.58114726176817e-08
3638 2.58036292051234e-08
3639 2.57960499542575e-08
3640 2.57888775271053e-08
3641 2.57812902954846e-08
3642 2.57746523629199e-08
3643 2.57667695803543e-08
3644 2.57598453251295e-08
3645 2.57519033096953e-08
3646 2.57442079518444e-08
3647 2.57377562175032e-08
3648 2.57300614442402e-08
3649 2.57221196388879e-08
3650 2.57155753366212e-08
3651 2.57078109825171e-08
3652 2.57003273050616e-08
3653 2.56927174523691e-08
3654 2.56854979632792e-08
3655 2.56784765247098e-08
3656 2.56707526318234e-08
3657 2.56638008616361e-08
3658 2.56561341083761e-08
3659 2.56489319357123e-08
3660 2.56414451090259e-08
3661 2.56336221447218e-08
3662 2.56260844861966e-08
3663 2.56190373359999e-08
3664 2.5611536370207e-08
3665 2.56046949426436e-08
3666 2.55966251277162e-08
3667 2.55887622636897e-08
3668 2.55822702907582e-08
3669 2.55741494695483e-08
3670 2.5567452894254e-08
3671 2.55595464753455e-08
3672 2.55524048508315e-08
3673 2.55444540958605e-08
3674 2.55376491247183e-08
3675 2.55301263249741e-08
3676 2.55224129040998e-08
3677 2.55152557720739e-08
3678 2.55080853469036e-08
3679 2.54999937635003e-08
3680 2.54921897772931e-08
3681 2.54834014546401e-08
3682 2.5474325967445e-08
3683 2.54662798157002e-08
3684 2.54579613577777e-08
3685 2.54501455353717e-08
3686 2.54419002302109e-08
3687 2.54337750657252e-08
3688 2.54261055278593e-08
3689 2.54183276570119e-08
3690 2.54103007405748e-08
3691 2.54022229334294e-08
3692 2.5394312079513e-08
3693 2.53866262510671e-08
3694 2.53787574790165e-08
3695 2.53709050464512e-08
3696 2.53628164741393e-08
3697 2.535537716572e-08
3698 2.53471587342013e-08
3699 2.53392707002253e-08
3700 2.53317594925695e-08
3701 2.53238408767509e-08
3702 2.53156833628365e-08
3703 2.53081308659309e-08
3704 2.5300436220177e-08
3705 2.52922042325021e-08
3706 2.52848160248453e-08
3707 2.5276771589533e-08
3708 2.52688406297275e-08
3709 2.52612659085161e-08
3710 2.52536373051559e-08
3711 2.52455129872153e-08
3712 2.52382343125246e-08
3713 2.52301251185416e-08
3714 2.52222049045847e-08
3715 2.52148011327669e-08
3716 2.52067229420117e-08
3717 2.5198626677575e-08
3718 2.51912350466288e-08
3719 2.51831225962229e-08
3720 2.51757058361834e-08
3721 2.51679459041254e-08
3722 2.51595949642969e-08
3723 2.51522836266505e-08
3724 2.51442913240807e-08
3725 2.51365554120586e-08
3726 2.51287552951451e-08
3727 2.51210199968543e-08
3728 2.51130198364591e-08
3729 2.51057839394109e-08
3730 2.5098037144955e-08
3731 2.50901829128569e-08
3732 2.50820537747998e-08
3733 2.50743863362546e-08
3734 2.50665805680006e-08
3735 2.5058967906999e-08
3736 2.50513229503868e-08
3737 2.50436890996963e-08
3738 2.50355755811449e-08
3739 2.50279256883701e-08
3740 2.5020206282339e-08
3741 2.50127144382495e-08
3742 2.50043925905385e-08
3743 2.4997298136431e-08
3744 2.49889407386294e-08
3745 2.49817974107558e-08
3746 2.49739707966212e-08
3747 2.49661498677722e-08
3748 2.49582758970746e-08
3749 2.49507823943396e-08
3750 2.49428115413342e-08
3751 2.49355119248396e-08
3752 2.49274763760021e-08
3753 2.49201548802314e-08
3754 2.49118669566062e-08
3755 2.49047997066265e-08
3756 2.48966242994142e-08
3757 2.48893993046728e-08
3758 2.48815628864363e-08
3759 2.48737520807396e-08
3760 2.48546371755809e-08
3761 2.48455099096379e-08
3762 2.48367046463416e-08
3763 2.48288777716377e-08
3764 2.48200980760838e-08
3765 2.481210879271e-08
3766 2.48041660889653e-08
3767 2.47957962890843e-08
3768 2.47877174539557e-08
3769 2.47796208612538e-08
3770 2.47714941206956e-08
3771 2.47613563804372e-08
3772 2.47520741320228e-08
3773 2.47431872614778e-08
3774 2.47340498185367e-08
3775 2.47256205853086e-08
3776 2.47168337059733e-08
3777 2.47082695480072e-08
3778 2.46998275717225e-08
3779 2.46908805463519e-08
3780 2.46824005653568e-08
3781 2.4674182144413e-08
3782 2.46653604952518e-08
3783 2.46573223765256e-08
3784 2.46483360235028e-08
3785 2.46404656180921e-08
3786 2.46319058955224e-08
3787 2.4623787603234e-08
3788 2.46153744197508e-08
3789 2.46067357244362e-08
3790 2.4598715438473e-08
3791 2.45903659632785e-08
3792 2.45820027563459e-08
3793 2.45736828819731e-08
3794 2.45656702994868e-08
3795 2.45568632945004e-08
3796 2.45490300203599e-08
3797 2.45406629491296e-08
3798 2.45322102577772e-08
3799 2.45244152600743e-08
3800 2.45159402388895e-08
3801 2.4507834207349e-08
3802 2.44995696278005e-08
3803 2.44912201609049e-08
3804 2.44834004246575e-08
3805 2.44746917649197e-08
3806 2.44670418457216e-08
3807 2.4459123794951e-08
3808 2.44506787047682e-08
3809 2.4442554057591e-08
3810 2.44347091604147e-08
3811 2.44265136299937e-08
3812 2.44180551920437e-08
3813 2.44107316473563e-08
3814 2.4402160576642e-08
3815 2.43944062891521e-08
3816 2.43863114652576e-08
3817 2.43783118420993e-08
3818 2.43703520471095e-08
3819 2.43622161746337e-08
3820 2.43545168416293e-08
3821 2.43467422494092e-08
3822 2.43384947825054e-08
3823 2.43307126956971e-08
3824 2.43227463303519e-08
3825 2.43147082802653e-08
3826 2.4307038308552e-08
3827 2.42990006645294e-08
3828 2.42910045013534e-08
3829 2.42831664677701e-08
3830 2.42751318773882e-08
3831 2.4267432610886e-08
3832 2.42595067494189e-08
3833 2.42517006477927e-08
3834 2.4243711929911e-08
3835 2.42359752747612e-08
3836 2.42281907540387e-08
3837 2.42203691923348e-08
3838 2.42123059663868e-08
3839 2.4204788012766e-08
3840 2.4196866999282e-08
3841 2.41892179008718e-08
3842 2.41812732634239e-08
3843 2.41732940450268e-08
3844 2.41657832484865e-08
3845 2.41581874746155e-08
3846 2.41503879483407e-08
3847 2.41426782732201e-08
3848 2.41349931168755e-08
3849 2.41274344867648e-08
3850 2.41196346826011e-08
3851 2.41119963135805e-08
3852 2.41043735049462e-08
3853 2.40962982792914e-08
3854 2.40891598629334e-08
3855 2.40812591355277e-08
3856 2.40739673585399e-08
3857 2.40661692224031e-08
3858 2.40522303822238e-08
3859 2.40363274103123e-08
3860 2.40245343482298e-08
3861 2.40139995280886e-08
3862 2.40040396816826e-08
3863 2.39947748576913e-08
3864 2.39861723516505e-08
3865 2.39766501961697e-08
3866 2.39680373430168e-08
3867 2.39590378700305e-08
3868 2.39506111150423e-08
3869 2.39421904605491e-08
3870 2.39324345349817e-08
3871 2.39209626259618e-08
3872 2.39105799919731e-08
3873 2.39004847402136e-08
3874 2.38912001624403e-08
3875 2.38815948102045e-08
3876 2.38722837107713e-08
3877 2.38629776795896e-08
3878 2.38543220829612e-08
3879 2.38450155702202e-08
3880 2.38364236565397e-08
3881 2.38275370375712e-08
3882 2.38187329531098e-08
3883 2.38100635927968e-08
3884 2.38005975898847e-08
3885 2.37918446129165e-08
3886 2.37834641491075e-08
3887 2.37745491199898e-08
3888 2.37661820400026e-08
3889 2.37572389557156e-08
3890 2.37487567317507e-08
3891 2.37396947378177e-08
3892 2.37313686109741e-08
3893 2.37220309512731e-08
3894 2.37140176030382e-08
3895 2.37050901647218e-08
3896 2.36959716293089e-08
3897 2.3688024149135e-08
3898 2.36789970405604e-08
3899 2.36705842438789e-08
3900 2.36626205968316e-08
3901 2.36538125201052e-08
3902 2.36456313122829e-08
3903 2.3636929741444e-08
3904 2.36286140065656e-08
3905 2.3620864678317e-08
3906 2.36119141485136e-08
3907 2.36043689930132e-08
3908 2.35956370535501e-08
3909 2.35874321185825e-08
3910 2.35792816459801e-08
3911 2.35715872773795e-08
3912 2.35628255166875e-08
3913 2.35549727766876e-08
3914 2.35473465890618e-08
3915 2.35386887119104e-08
3916 2.35310503528124e-08
3917 2.35230765800731e-08
3918 2.35135567206168e-08
3919 2.3503066589714e-08
3920 2.34935951868021e-08
3921 2.34843258235112e-08
3922 2.34759002304824e-08
3923 2.34674808659158e-08
3924 2.3459123568062e-08
3925 2.34506337757623e-08
3926 2.34426609898308e-08
3927 2.34348312498495e-08
3928 2.34262751233338e-08
3929 2.34187629972044e-08
3930 2.34107759923829e-08
3931 2.34031233337373e-08
3932 2.33949164521879e-08
3933 2.33872955998499e-08
3934 2.33798197206703e-08
3935 2.33716347638885e-08
3936 2.33641963480052e-08
3937 2.33564683743137e-08
3938 2.33451314059774e-08
3939 2.3334762538213e-08
3940 2.33251115722838e-08
3941 2.33166974213522e-08
3942 2.33084678312595e-08
3943 2.33001588341303e-08
3944 2.32919545627985e-08
3945 2.32842585956294e-08
3946 2.32758676092792e-08
3947 2.32679602460428e-08
3948 2.32601690138279e-08
3949 2.3251995691545e-08
3950 2.32409819236196e-08
3951 2.32312103307097e-08
3952 2.32225548404408e-08
3953 2.32135406023487e-08
3954 2.32047989754403e-08
3955 2.31964702352705e-08
3956 2.31885012938332e-08
3957 2.3180222834443e-08
3958 2.31716092653489e-08
3959 2.31639005594669e-08
3960 2.31556787376602e-08
3961 2.31480552899954e-08
3962 2.31397438147929e-08
3963 2.3131964282358e-08
3964 2.312368895524e-08
3965 2.31159711378548e-08
3966 2.31081825748519e-08
3967 2.30993213470865e-08
3968 2.30923541907485e-08
3969 2.3084785432767e-08
3970 2.30767655955005e-08
3971 2.30696171456546e-08
3972 2.30613592012152e-08
3973 2.30542095637776e-08
3974 2.30462577877483e-08
3975 2.30386330146021e-08
3976 2.30313970622509e-08
3977 2.30236042970816e-08
3978 2.30163010587203e-08
3979 2.30086055573314e-08
3980 2.30015738895556e-08
3981 2.29934396004938e-08
3982 2.29866923712602e-08
3983 2.29787376195695e-08
3984 2.29718288993902e-08
3985 2.29640685884824e-08
3986 2.29566296882366e-08
3987 2.29496699716558e-08
3988 2.29425375227327e-08
3989 2.29347243643624e-08
3990 2.29280108062779e-08
3991 2.29205966183754e-08
3992 2.29130306214631e-08
3993 2.2906379754134e-08
3994 2.28984627471396e-08
3995 2.28918095511732e-08
3996 2.28848276666582e-08
3997 2.28775817564669e-08
3998 2.28703980737766e-08
3999 2.28633585667021e-08
4000 2.28559210999346e-08
4001 2.28489864303461e-08
4002 2.28419050651435e-08
4003 2.28354940202125e-08
4004 2.28277927041831e-08
4005 2.28213105330499e-08
4006 2.28139035866465e-08
4007 2.28067885213457e-08
4008 2.28008004195812e-08
4009 2.27927668112471e-08
4010 2.27865791758847e-08
4011 2.27792497634466e-08
4012 2.27726791474092e-08
4013 2.27654018404438e-08
4014 2.27588474557699e-08
4015 2.27522791566709e-08
4016 2.27449466395774e-08
4017 2.27386146429004e-08
4018 2.27321134742414e-08
4019 2.27248517797318e-08
4020 2.27189667174471e-08
4021 2.2711455895677e-08
4022 2.27054356757017e-08
4023 2.26985279617037e-08
4024 2.26921136384089e-08
4025 2.26851210116952e-08
4026 2.26788071416711e-08
4027 2.26724577639531e-08
4028 2.26654119271641e-08
4029 2.26597254136818e-08
4030 2.26523951425417e-08
4031 2.26466442514311e-08
4032 2.2639671175384e-08
4033 2.26331942735775e-08
4034 2.26274698347656e-08
4035 2.26204025714494e-08
4036 2.26145485867774e-08
4037 2.26078870233237e-08
4038 2.26013435940531e-08
4039 2.25955746189499e-08
4040 2.25888346646053e-08
4041 2.25829149864409e-08
4042 2.25764457835453e-08
4043 2.25699409674845e-08
4044 2.25643435244616e-08
4045 2.255737304456e-08
4046 2.25513212146478e-08
4047 2.25457532955636e-08
4048 2.25387933212307e-08
4049 2.25331317407623e-08
4050 2.25265135788272e-08
4051 2.25208285919293e-08
4052 2.25147321112895e-08
4053 2.25083322590025e-08
4054 2.25024698392723e-08
4055 2.2496499511776e-08
4056 2.24900181807208e-08
4057 2.24841077932636e-08
4058 2.24786969938301e-08
4059 2.24718948937802e-08
4060 2.24665137372876e-08
4061 2.24602376075189e-08
4062 2.24542665543254e-08
4063 2.24485945259506e-08
4064 2.24422367875871e-08
4065 2.24365241510566e-08
4066 2.24309160704178e-08
4067 2.24246773133929e-08
4068 2.24187192174041e-08
4069 2.24133260659787e-08
4070 2.24071520314034e-08
4071 2.24012549027569e-08
4072 2.23960463173606e-08
4073 2.23894036813915e-08
4074 2.23838692925277e-08
4075 2.23787077225646e-08
4076 2.23724561783933e-08
4077 2.23670289289379e-08
4078 2.23607541691151e-08
4079 2.23554011319488e-08
4080 2.23498598222044e-08
4081 2.23437917870745e-08
4082 2.23383967215829e-08
4083 2.2332725600066e-08
4084 2.23273957538767e-08
4085 2.23211414945163e-08
4086 2.23160039840992e-08
4087 2.23102573929379e-08
4088 2.23043242873755e-08
4089 2.22988177852185e-08
4090 2.22933096216404e-08
4091 2.22879309489249e-08
4092 2.22819977092337e-08
4093 2.22764526804087e-08
4094 2.22712839049455e-08
4095 2.22651006027696e-08
4096 2.22599866847029e-08
4097 2.22545261503448e-08
4098 2.22488144164118e-08
4099 2.22439192271023e-08
4100 2.22378585266469e-08
4101 2.22326048484045e-08
4102 2.22273617671159e-08
4103 2.22213227983326e-08
4104 2.22158626783792e-08
4105 2.22107631045465e-08
4106 2.2205067117545e-08
4107 2.21997791240436e-08
4108 2.21942725038976e-08
4109 2.21887191862724e-08
4110 2.21832706418651e-08
4111 2.21783901986627e-08
4112 2.21727842435043e-08
4113 2.21673907347397e-08
4114 2.21623561760392e-08
4115 2.21571053267006e-08
4116 2.21517311250058e-08
4117 2.21463532636829e-08
4118 2.21414033275136e-08
4119 2.21361827992261e-08
4120 2.21305386757242e-08
4121 2.21257559877841e-08
4122 2.21203640113077e-08
4123 2.21152870763658e-08
4124 2.21098048553275e-08
4125 2.21050164215392e-08
4126 2.20995122604539e-08
4127 2.20947736417898e-08
4128 2.20891732025524e-08
4129 2.20842684547085e-08
4130 2.20793144147663e-08
4131 2.20738045777491e-08
4132 2.20690875945145e-08
4133 2.20637510908311e-08
4134 2.20590592383413e-08
4135 2.20539903592748e-08
4136 2.20487254204704e-08
4137 2.2043766965324e-08
4138 2.20389844130253e-08
4139 2.20341089810089e-08
4140 2.20286243074186e-08
4141 2.2024089275624e-08
4142 2.20191122479313e-08
4143 2.20139637404082e-08
4144 2.20092576532216e-08
4145 2.20040153861428e-08
4146 2.19995015884045e-08
4147 2.19944011566608e-08
4148 2.19893813390576e-08
4149 2.19844426730675e-08
4150 2.19797674649025e-08
4151 2.19749983375317e-08
4152 2.19697605945979e-08
4153 2.19652339012583e-08
4154 2.19600101989958e-08
4155 2.19555900895646e-08
4156 2.19505844209489e-08
4157 2.19456565119791e-08
4158 2.19408790589626e-08
4159 2.19360635088928e-08
4160 2.19320182441585e-08
4161 2.19264066705499e-08
4162 2.19217749246464e-08
4163 2.19172950171043e-08
4164 2.19124965637613e-08
4165 2.19076718082495e-08
4166 2.19028833996909e-08
4167 2.18983870279021e-08
4168 2.18938196970969e-08
4169 2.18891266133975e-08
4170 2.18841192969471e-08
4171 2.18795823402063e-08
4172 2.18749750920538e-08
4173 2.18701409556571e-08
4174 2.18656442955156e-08
4175 2.18610618036991e-08
4176 2.18566183946417e-08
4177 2.18517216644815e-08
4178 2.18472403079178e-08
4179 2.18424495153496e-08
4180 2.1837965907448e-08
4181 2.18334628523664e-08
4182 2.18290296862267e-08
4183 2.18247275890204e-08
4184 2.18200257832424e-08
4185 2.1815921747978e-08
4186 2.18111663052856e-08
4187 2.18062776465161e-08
4188 2.18024980719878e-08
4189 2.17977598619273e-08
4190 2.17933465478104e-08
4191 2.17891790510877e-08
4192 2.17846065317107e-08
4193 2.17799301079763e-08
4194 2.17758171940363e-08
4195 2.17710415318928e-08
4196 2.17666615902196e-08
4197 2.17625701433022e-08
4198 2.17577293011723e-08
4199 2.17534260668478e-08
4200 2.17492130673291e-08
4201 2.17446804060342e-08
4202 2.17404627547504e-08
4203 2.17361308779002e-08
4204 2.17317289424798e-08
4205 2.17271687095455e-08
4206 2.17229705666705e-08
4207 2.17184312733681e-08
4208 2.17143419188673e-08
4209 2.17100104468043e-08
4210 2.17059204169201e-08
4211 2.17013954493042e-08
4212 2.16969418498705e-08
4213 2.16930408127314e-08
4214 2.1688573335614e-08
4215 2.16844936560476e-08
4216 2.16804644335988e-08
4217 2.16757578025903e-08
4218 2.16719022659503e-08
4219 2.16674762725244e-08
4220 2.16635244948624e-08
4221 2.16590915667475e-08
4222 2.16547820097307e-08
4223 2.16506975848768e-08
4224 2.16464593072105e-08
4225 2.16423278811273e-08
4226 2.1638361530435e-08
4227 2.16339340342181e-08
4228 2.16295945417513e-08
4229 2.16258944074205e-08
4230 2.16215891935545e-08
4231 2.16177972716461e-08
4232 2.16115245500886e-08
4233 2.16050439157192e-08
4234 2.15995356204132e-08
4235 2.1594553334614e-08
4236 2.15896926095208e-08
4237 2.1584844634534e-08
4238 2.15803744523999e-08
4239 2.15755572202103e-08
4240 2.15712954895683e-08
4241 2.15664945551405e-08
4242 2.15621172377431e-08
4243 2.15578370320116e-08
4244 2.15529168448025e-08
4245 2.15489254022164e-08
4246 2.15442546899688e-08
4247 2.15398862897162e-08
4248 2.15355551109395e-08
4249 2.15312359118855e-08
4250 2.15263505264143e-08
4251 2.15222979883523e-08
4252 2.15181370863751e-08
4253 2.15135040210271e-08
4254 2.15094619645528e-08
4255 2.15045400394492e-08
4256 2.15007582223606e-08
4257 2.14960577486281e-08
4258 2.14918616925769e-08
4259 2.14872327455526e-08
4260 2.14833022268782e-08
4261 2.14788462280999e-08
4262 2.14750261379112e-08
4263 2.14702538452252e-08
4264 2.14661473557573e-08
4265 2.14618314229378e-08
4266 2.14573822739234e-08
4267 2.14540957256606e-08
4268 2.14491602930078e-08
4269 2.14448536331249e-08
4270 2.14408811542513e-08
4271 2.1436673756349e-08
4272 2.14321061740505e-08
4273 2.14277835542884e-08
4274 2.14237256287997e-08
4275 2.14196412433379e-08
4276 2.14152648035301e-08
4277 2.14112256888693e-08
4278 2.14070765973198e-08
4279 2.14027480079301e-08
4280 2.13986609173475e-08
4281 2.13946732784825e-08
4282 2.13904347167934e-08
4283 2.13862216940711e-08
4284 2.13821781518894e-08
4285 2.13779374150236e-08
4286 2.13743571325087e-08
4287 2.13696404076716e-08
4288 2.13657161950292e-08
4289 2.13616981485989e-08
4290 2.13579610788758e-08
4291 2.13529751044647e-08
4292 2.13493085922875e-08
4293 2.13449757905659e-08
4294 2.13407055098327e-08
4295 2.13369633126972e-08
4296 2.13327839086591e-08
4297 2.13286662716708e-08
4298 2.13252684668833e-08
4299 2.13208643798299e-08
4300 2.13164962260051e-08
4301 2.13125278147805e-08
4302 2.13089482129086e-08
4303 2.13046027263339e-08
4304 2.13007618928215e-08
4305 2.12968370200961e-08
4306 2.12926568580532e-08
4307 2.12887787354107e-08
4308 2.12849364972442e-08
4309 2.1280905497717e-08
4310 2.12769884131436e-08
4311 2.12733592724543e-08
4312 2.12692145069998e-08
4313 2.12649331293793e-08
4314 2.12613270167278e-08
4315 2.1257073983226e-08
4316 2.12532484067246e-08
4317 2.1249587805236e-08
4318 2.1245352628825e-08
4319 2.12416208437982e-08
4320 2.12376140649381e-08
4321 2.12332769724752e-08
4322 2.12297664508657e-08
4323 2.12255497106864e-08
4324 2.12219467109406e-08
4325 2.12181388481933e-08
4326 2.12136736666216e-08
4327 2.12103202602079e-08
4328 2.12061653380238e-08
4329 2.12024993880705e-08
4330 2.11988039587677e-08
4331 2.11947716116101e-08
4332 2.11908949139458e-08
4333 2.11875430787267e-08
4334 2.1183414781821e-08
4335 2.11793500524485e-08
4336 2.11763893396902e-08
4337 2.1172193551966e-08
4338 2.11687996941601e-08
4339 2.11650672173602e-08
4340 2.11611661151967e-08
4341 2.11575879991502e-08
4342 2.11538726628824e-08
4343 2.11473161489811e-08
4344 2.11383329230957e-08
4345 2.11293324215156e-08
4346 2.11245640547378e-08
4347 2.11137793576857e-08
4348 2.11057792423586e-08
4349 2.10989859144167e-08
4350 2.10927130893379e-08
4351 2.10872152704364e-08
4352 2.10824788580491e-08
4353 2.10761900470008e-08
4354 2.10755377455518e-08
4355 2.10673342155784e-08
4356 2.10603580577187e-08
4357 2.10541350455645e-08
4358 2.10479033335362e-08
4359 2.10419828484201e-08
4360 2.10331277555112e-08
4361 2.10250307661961e-08
4362 2.10182886858715e-08
4363 2.10119909504689e-08
4364 2.10055866021672e-08
4365 2.09993890221347e-08
4366 2.09933887549488e-08
4367 2.09875843360033e-08
4368 2.09810471320446e-08
4369 2.09757436174518e-08
4370 2.0969477792919e-08
4371 2.09635084361948e-08
4372 2.09576780297038e-08
4373 2.09524910350298e-08
4374 2.09464560364178e-08
4375 2.09414925573317e-08
4376 2.0935519362969e-08
4377 2.09304011413794e-08
4378 2.09247259066875e-08
4379 2.09195213280097e-08
4380 2.09142621237987e-08
4381 2.09090666170378e-08
4382 2.0903749895558e-08
4383 2.08944034536737e-08
4384 2.08873530948353e-08
4385 2.08813236716524e-08
4386 2.08750157105683e-08
4387 2.08694459161371e-08
4388 2.0863774602288e-08
4389 2.08580754397156e-08
4390 2.08527129730227e-08
4391 2.08471217631387e-08
4392 2.08415613078688e-08
4393 2.0836629144684e-08
4394 2.08310392791275e-08
4395 2.08258890105895e-08
4396 2.08205275674042e-08
4397 2.08154603076535e-08
4398 2.08102246361877e-08
4399 2.08048811492084e-08
4400 2.07998329741152e-08
4401 2.07947886776486e-08
4402 2.07895369912434e-08
4403 2.07845911175605e-08
4404 2.07793876583554e-08
4405 2.07745057689557e-08
4406 2.07693006023107e-08
4407 2.07643667036808e-08
4408 2.07594926993696e-08
4409 2.07543028261301e-08
4410 2.07496415467953e-08
4411 2.07443762880663e-08
4412 2.07398024637956e-08
4413 2.07346176867851e-08
4414 2.07299786424905e-08
4415 2.07250599705486e-08
4416 2.07203274071241e-08
4417 2.0715275122804e-08
4418 2.0710462827235e-08
4419 2.07053767749057e-08
4420 2.07016175956393e-08
4421 2.06966894217842e-08
4422 2.0691648150592e-08
4423 2.06869803248064e-08
4424 2.06823461199393e-08
4425 2.06776204927719e-08
4426 2.06707526769906e-08
4427 2.06652497953194e-08
4428 2.06596367986825e-08
4429 2.06546105209968e-08
4430 2.064905775475e-08
4431 2.06434301526295e-08
4432 2.06382971105271e-08
4433 2.06334640623917e-08
4434 2.06284245560448e-08
4435 2.06238047832305e-08
4436 2.06189955907973e-08
4437 2.06139287405732e-08
4438 2.06091964056257e-08
4439 2.06041204577623e-08
4440 2.0599426253716e-08
4441 2.0594491885427e-08
4442 2.05898451793238e-08
4443 2.05848655843097e-08
4444 2.05801069505934e-08
4445 2.05753112600751e-08
4446 2.05705477315105e-08
4447 2.05657021327271e-08
4448 2.0561082282218e-08
4449 2.05563491582558e-08
4450 2.05515752654123e-08
4451 2.05471974286733e-08
4452 2.05423315788253e-08
4453 2.0537758799892e-08
4454 2.05331697816918e-08
4455 2.05283212203131e-08
4456 2.05240253232167e-08
4457 2.05191695353268e-08
4458 2.05146965451958e-08
4459 2.05100713294992e-08
4460 2.05055458754946e-08
4461 2.0500361318572e-08
4462 2.04940547176291e-08
4463 2.04880002137725e-08
4464 2.04823987934172e-08
4465 2.04779326411914e-08
4466 2.04717266295701e-08
4467 2.04603262387304e-08
4468 2.04513482950502e-08
4469 2.04430694727559e-08
4470 2.04353835729504e-08
4471 2.04292090169728e-08
4472 2.04225843576239e-08
4473 2.04159457498632e-08
4474 2.04096509036911e-08
4475 2.04035393131796e-08
4476 2.03973361199705e-08
4477 2.03913513618509e-08
4478 2.03853004161342e-08
4479 2.03795248697042e-08
4480 2.03736342044708e-08
4481 2.03677234784511e-08
4482 2.03620524208692e-08
4483 2.0356264963689e-08
4484 2.03507295838262e-08
4485 2.03449560355479e-08
4486 2.03394531281542e-08
4487 2.03338694544225e-08
4488 2.03283246763553e-08
4489 2.03228225375829e-08
4490 2.03173910609256e-08
4491 2.03119897116966e-08
4492 2.0306437263376e-08
4493 2.03011257853963e-08
4494 2.02958109804946e-08
4495 2.02904041456259e-08
4496 2.02852200650444e-08
4497 2.02798032547941e-08
4498 2.02746708062862e-08
4499 2.02693820521155e-08
4500 2.0264180641355e-08
4501 2.02589588881169e-08
4502 2.02532421495483e-08
4503 2.02485712791495e-08
4504 2.02429226118933e-08
4505 2.02383744777879e-08
4506 2.02327311166345e-08
4507 2.02281229843976e-08
4508 2.02225810848056e-08
4509 2.02178674007653e-08
4510 2.02124910861079e-08
4511 2.02080629016499e-08
4512 2.02025395676375e-08
4513 2.01981648165067e-08
4514 2.0192721936832e-08
4515 2.0187552166373e-08
4516 2.01825805394851e-08
4517 2.01776064838316e-08
4518 2.01726272053004e-08
4519 2.01677989986049e-08
4520 2.01627755155726e-08
4521 2.01578330581986e-08
4522 2.01523818020855e-08
4523 2.01476236585951e-08
4524 2.01428835015921e-08
4525 2.01379219217784e-08
4526 2.01330449528941e-08
4527 2.01281351232263e-08
4528 2.01232979654131e-08
4529 2.01185507416046e-08
4530 2.01138084385682e-08
4531 2.01090102274343e-08
4532 2.01041506038696e-08
4533 2.00996052997504e-08
4534 2.00947054700418e-08
4535 2.00901137604598e-08
4536 2.00853327460143e-08
4537 2.00804762759965e-08
4538 2.00757256794831e-08
4539 2.00710918364655e-08
4540 2.00666390898774e-08
4541 2.00623711787834e-08
4542 2.00578421370307e-08
4543 2.00531730886883e-08
4544 2.0049021164148e-08
4545 2.00455306898772e-08
4546 2.0033559227764e-08
4547 2.00248971584244e-08
4548 2.00181876831851e-08
4549 2.00118867253463e-08
4550 2.00061296564136e-08
4551 2.00003060908949e-08
4552 1.99947724504601e-08
4553 1.99891531775936e-08
4554 1.99838381478162e-08
4555 1.99783155122729e-08
4556 1.99729821441508e-08
4557 1.99678194360425e-08
4558 1.99623644179386e-08
4559 1.99569659216359e-08
4560 1.99517415710865e-08
4561 1.99465990354253e-08
4562 1.99414152088073e-08
4563 1.99362701167605e-08
4564 1.99308864662212e-08
4565 1.992573730214e-08
4566 1.99206545019315e-08
4567 1.99155321716771e-08
4568 1.99105353357737e-08
4569 1.99055510032298e-08
4570 1.99004874624348e-08
4571 1.98955305653892e-08
4572 1.9890623326399e-08
4573 1.98856406964376e-08
4574 1.98807687450259e-08
4575 1.98757680637263e-08
4576 1.98709695737459e-08
4577 1.9866017449105e-08
4578 1.98611642578256e-08
4579 1.98563141162872e-08
4580 1.98514588214405e-08
4581 1.9846637474577e-08
4582 1.9841778712365e-08
4583 1.98370120133573e-08
4584 1.98321719889621e-08
4585 1.98275449920376e-08
4586 1.98227135715656e-08
4587 1.98178888709266e-08
4588 1.98132036317161e-08
4589 1.98085192156111e-08
4590 1.98037514654859e-08
4591 1.97991327482716e-08
4592 1.97944328077099e-08
4593 1.97897483899118e-08
4594 1.97851475989969e-08
4595 1.97804496741422e-08
4596 1.97757901376633e-08
4597 1.97711736236866e-08
4598 1.9766682602701e-08
4599 1.97620123042752e-08
4600 1.97573361185172e-08
4601 1.97529547175945e-08
4602 1.97483068333781e-08
4603 1.97436789981659e-08
4604 1.97392549001751e-08
4605 1.973479296144e-08
4606 1.97301539352351e-08
4607 1.97256750552183e-08
4608 1.97212782899053e-08
4609 1.97166962705303e-08
4610 1.97123504402236e-08
4611 1.9707949422465e-08
4612 1.97063198238423e-08
4613 1.96990751612086e-08
4614 1.96909131299633e-08
4615 1.96826685630314e-08
4616 1.96759957092063e-08
4617 1.96699619719742e-08
4618 1.966431237789e-08
4619 1.96596456433912e-08
4620 1.96534528205117e-08
4621 1.96478612040363e-08
4622 1.96423500002513e-08
4623 1.96367633396172e-08
4624 1.96315935692831e-08
4625 1.96261582169899e-08
4626 1.96209523340846e-08
4627 1.96156740596143e-08
4628 1.96104737931885e-08
4629 1.96052777448608e-08
4630 1.96002139220691e-08
4631 1.95953119672271e-08
4632 1.95902358989114e-08
4633 1.95851355493937e-08
4634 1.95801415854291e-08
4635 1.95751354490209e-08
4636 1.9570119459536e-08
4637 1.95652502663068e-08
4638 1.95602884004165e-08
4639 1.95554080314603e-08
4640 1.95504356536175e-08
4641 1.95456511050268e-08
4642 1.95407768115163e-08
4643 1.95359923336053e-08
4644 1.9531231950809e-08
4645 1.95264489869923e-08
4646 1.95216106209373e-08
4647 1.95168720005384e-08
4648 1.95121805486448e-08
4649 1.95074253343808e-08
4650 1.95028693018157e-08
4651 1.94981732976329e-08
4652 1.94934210726444e-08
4653 1.94888010958544e-08
4654 1.94840856200112e-08
4655 1.94794180448515e-08
4656 1.94749417455203e-08
4657 1.94702920052192e-08
4658 1.94656713862623e-08
4659 1.94611331011674e-08
4660 1.94566049253331e-08
4661 1.94519558683604e-08
4662 1.94474172948156e-08
4663 1.94429014340469e-08
4664 1.94384223266703e-08
4665 1.94337768267688e-08
4666 1.94294331403028e-08
4667 1.94249537599361e-08
4668 1.94187909130089e-08
4669 1.94133804267416e-08
4670 1.94080674011943e-08
4671 1.94029960140923e-08
4672 1.93977102781184e-08
4673 1.9392701496368e-08
4674 1.93877948874849e-08
4675 1.93830998708178e-08
4676 1.93782254694491e-08
4677 1.93732974641397e-08
4678 1.93685509287944e-08
4679 1.93637235306326e-08
4680 1.93590716966313e-08
4681 1.93543996459267e-08
4682 1.93496137104177e-08
4683 1.93450600048384e-08
4684 1.93403899884925e-08
4685 1.93356473436601e-08
4686 1.933109397316e-08
4687 1.93263827450579e-08
4688 1.93219668321032e-08
4689 1.93171918133478e-08
4690 1.93127451751096e-08
4691 1.93081481313639e-08
4692 1.93036628473864e-08
4693 1.92988540314853e-08
4694 1.92946520130577e-08
4695 1.92899025847626e-08
4696 1.92855873264314e-08
4697 1.92809410658329e-08
4698 1.92765322025307e-08
4699 1.92722400973455e-08
4700 1.92676811521797e-08
4701 1.92632058763909e-08
4702 1.92588535711524e-08
4703 1.92545412526224e-08
4704 1.92502486296292e-08
4705 1.92454959232757e-08
4706 1.9241451574864e-08
4707 1.92370230736871e-08
4708 1.92326232109491e-08
4709 1.92282776117908e-08
4710 1.92239796638072e-08
4711 1.92196866902611e-08
4712 1.9215557442151e-08
4713 1.92110232334342e-08
4714 1.92068593436179e-08
4715 1.92027026844899e-08
4716 1.91981157491722e-08
4717 1.91941753738961e-08
4718 1.91899269222501e-08
4719 1.91854694480548e-08
4720 1.91814463034456e-08
4721 1.91771937813889e-08
4722 1.91729957340209e-08
4723 1.91688045289296e-08
4724 1.9164350790371e-08
4725 1.91601931093938e-08
4726 1.91560915592803e-08
4727 1.91519128441148e-08
4728 1.91476988177508e-08
4729 1.91433880349229e-08
4730 1.91395536981859e-08
4731 1.9135144435467e-08
4732 1.91312824729672e-08
4733 1.91271162439943e-08
4734 1.91231008830062e-08
4735 1.91189669204866e-08
4736 1.91149606301177e-08
4737 1.91109026751041e-08
4738 1.91067945471968e-08
4739 1.91027263725108e-08
4740 1.90987443578067e-08
4741 1.90947717281648e-08
4742 1.90922204656441e-08
4743 1.90889108525882e-08
4744 1.9080301088617e-08
4745 1.90713627685052e-08
4746 1.90627480650823e-08
4747 1.90554829686412e-08
4748 1.90492812272985e-08
4749 1.9043859319598e-08
4750 1.90378230305183e-08
4751 1.90321972929342e-08
4752 1.90270670843845e-08
4753 1.9021210331524e-08
4754 1.90160483059393e-08
4755 1.90110031265262e-08
4756 1.90059157323119e-08
4757 1.90008621586829e-08
4758 1.89957067060903e-08
4759 1.89908034722713e-08
4760 1.89857316133801e-08
4761 1.89809228435672e-08
4762 1.89761993607956e-08
4763 1.89719110481046e-08
4764 1.89662697035325e-08
4765 1.89615716476715e-08
4766 1.89567769093291e-08
4767 1.89518368174935e-08
4768 1.89471679741121e-08
4769 1.89424355873796e-08
4770 1.89375429726357e-08
4771 1.8933053740719e-08
4772 1.89317594489619e-08
4773 1.8927062725424e-08
4774 1.89179079294888e-08
4775 1.89111031441841e-08
4776 1.89053424989316e-08
4777 1.88981767216961e-08
4778 1.88926578656018e-08
4779 1.88871093567688e-08
4780 1.88816059705776e-08
4781 1.88761736191895e-08
4782 1.88706432748126e-08
4783 1.88654919781733e-08
4784 1.8860984073471e-08
4785 1.88545912563481e-08
4786 1.88493650940397e-08
4787 1.88443778520592e-08
4788 1.88394415723547e-08
4789 1.88348343092282e-08
4790 1.88287199467674e-08
4791 1.88239070844504e-08
4792 1.88188647079063e-08
4793 1.88146118357407e-08
4794 1.88060452385452e-08
4795 1.88003506882972e-08
4796 1.87947383299969e-08
4797 1.87857239847128e-08
4798 1.87787362053854e-08
4799 1.87720250022921e-08
4800 1.87648580428773e-08
4801 1.87628917709681e-08
4802 1.87544877898876e-08
4803 1.87472444099224e-08
4804 1.87400408855481e-08
4805 1.87336401981653e-08
4806 1.87266677120213e-08
4807 1.87205763336751e-08
4808 1.87153011983049e-08
4809 1.87084352109013e-08
4810 1.87023334040498e-08
4811 1.86966135741345e-08
4812 1.86906482544241e-08
4813 1.86848999415429e-08
4814 1.86790707158158e-08
4815 1.86732587742072e-08
4816 1.86674634370526e-08
4817 1.86617843622033e-08
4818 1.86562362551462e-08
4819 1.86507235941519e-08
4820 1.86455196400787e-08
4821 1.8639211867566e-08
4822 1.86339266805974e-08
4823 1.86282114186948e-08
4824 1.8622772660376e-08
4825 1.86171633975579e-08
4826 1.86122386510862e-08
4827 1.86066637488935e-08
4828 1.86011299873334e-08
4829 1.85957413249699e-08
4830 1.85904574287327e-08
4831 1.85843313441486e-08
4832 1.85792738122553e-08
4833 1.85739888810127e-08
4834 1.85686274580404e-08
4835 1.85642901654043e-08
4836 1.85588109194168e-08
4837 1.85537950917053e-08
4838 1.85484974152966e-08
4839 1.85433322542894e-08
4840 1.85385044484399e-08
4841 1.85337933511498e-08
4842 1.85290576407149e-08
4843 1.85235856321048e-08
4844 1.85185986519426e-08
4845 1.85132452991538e-08
4846 1.8508525985228e-08
4847 1.85038594416181e-08
4848 1.84977258109731e-08
4849 1.84926879201047e-08
4850 1.84878131056776e-08
4851 1.84825415352058e-08
4852 1.84776599423814e-08
4853 1.84725041743883e-08
4854 1.84676638840808e-08
4855 1.84626938416199e-08
4856 1.84576925035124e-08
4857 1.84526758448128e-08
4858 1.84478101458024e-08
4859 1.84431678630081e-08
4860 1.84380685913227e-08
4861 1.84333586156077e-08
4862 1.84286695532193e-08
4863 1.84235622276152e-08
4864 1.8418807804288e-08
4865 1.84142852442115e-08
4866 1.84095987911664e-08
4867 1.84041982215222e-08
4868 1.84002313571574e-08
4869 1.83954089423011e-08
4870 1.83907034207315e-08
4871 1.83861380796541e-08
4872 1.83815480217092e-08
4873 1.83767677897073e-08
4874 1.83722102074513e-08
4875 1.83674527108463e-08
4876 1.83630530487811e-08
4877 1.83584807807585e-08
4878 1.83534887462922e-08
4879 1.83492754212006e-08
4880 1.8346001351241e-08
4881 1.83395030186329e-08
4882 1.83337533318645e-08
4883 1.83285335852018e-08
4884 1.83234012203909e-08
4885 1.83183869652132e-08
4886 1.83132941709851e-08
4887 1.83083728832606e-08
4888 1.83035156821193e-08
4889 1.82986278029207e-08
4890 1.82935871218676e-08
4891 1.82887665193948e-08
4892 1.82841518372584e-08
4893 1.82793222019972e-08
4894 1.82745392361683e-08
4895 1.8269724000361e-08
4896 1.82650450556016e-08
4897 1.82602615352023e-08
4898 1.82557976209363e-08
4899 1.82510077911258e-08
4900 1.82463448629988e-08
4901 1.82418396289102e-08
4902 1.82371489855226e-08
4903 1.82325258635402e-08
4904 1.82279194262669e-08
4905 1.82234398207054e-08
4906 1.82187865196831e-08
4907 1.82145403449657e-08
4908 1.82101056508538e-08
4909 1.82051612124395e-08
4910 1.82011586346009e-08
4911 1.81965597325778e-08
4912 1.81919169603278e-08
4913 1.81872603906147e-08
4914 1.81831025022755e-08
4915 1.81785050327038e-08
4916 1.81740136959568e-08
4917 1.81699781622302e-08
4918 1.81655344685672e-08
4919 1.81609516407139e-08
4920 1.81567472046368e-08
4921 1.81524577092945e-08
4922 1.81479342025725e-08
4923 1.81437345438545e-08
4924 1.81395298657905e-08
4925 1.81351607216745e-08
4926 1.81311777896348e-08
4927 1.81269592690742e-08
4928 1.81223955475623e-08
4929 1.81183200081286e-08
4930 1.81141671861118e-08
4931 1.81099826019049e-08
4932 1.81057350101543e-08
4933 1.81015221117076e-08
4934 1.80974700828423e-08
4935 1.80930618566694e-08
4936 1.80892353902479e-08
4937 1.80852982968743e-08
4938 1.80808064975529e-08
4939 1.80769892367261e-08
4940 1.80726557634098e-08
4941 1.80688367143883e-08
4942 1.80646742633933e-08
4943 1.80607041474767e-08
4944 1.80563563294822e-08
4945 1.80505277955212e-08
4946 1.80452234958967e-08
4947 1.80406360025531e-08
4948 1.80362877495038e-08
4949 1.80315901947142e-08
4950 1.80271808880028e-08
4951 1.80227342121975e-08
4952 1.80184810886636e-08
4953 1.80142262456856e-08
4954 1.8010041198474e-08
4955 1.80058114086223e-08
4956 1.80014494952641e-08
4957 1.79974807895805e-08
4958 1.79933732908227e-08
4959 1.79892242831298e-08
4960 1.79851456387631e-08
4961 1.79810373148043e-08
4962 1.79769134581315e-08
4963 1.7972572378358e-08
4964 1.79689135408612e-08
4965 1.79647560726442e-08
4966 1.79606099438984e-08
4967 1.79568770473232e-08
4968 1.79528234869469e-08
4969 1.79484885715758e-08
4970 1.79448032399804e-08
4971 1.79407453339137e-08
4972 1.7936702015095e-08
4973 1.79324936995656e-08
4974 1.79289614198191e-08
4975 1.79248600967602e-08
4976 1.79209598566987e-08
4977 1.79169418349362e-08
4978 1.79130859757209e-08
4979 1.79092867545749e-08
4980 1.79054806359602e-08
4981 1.79019327754548e-08
4982 1.7897650911522e-08
4983 1.78940678732947e-08
4984 1.7890239975199e-08
4985 1.78863506617999e-08
4986 1.78822393518269e-08
4987 1.7878853238909e-08
4988 1.78746145029218e-08
4989 1.78712705354994e-08
4990 1.78671140786124e-08
4991 1.78640106934386e-08
4992 1.78599143876162e-08
4993 1.7856676760325e-08
4994 1.78517059196415e-08
4995 1.78487168693425e-08
4996 1.78451070053792e-08
4997 1.78411537327117e-08
4998 1.7837802226528e-08
4999 1.78335952037423e-08
};
\addlegendentry{Train}
\addplot [semithick, black]
table {%
0 0.00212635193020105
1 0.00136229686904699
2 0.000194428663235158
3 0.000151769403601065
4 0.000100693017884623
5 5.05226162204053e-05
6 2.52276277024066e-05
7 1.92134793906007e-05
8 1.80913611984579e-05
9 1.76366011146456e-05
10 1.71936644619564e-05
11 1.66821464517852e-05
12 1.61090774781769e-05
13 1.54022400238318e-05
14 1.45564927152009e-05
15 1.35917607622105e-05
16 1.24715515994467e-05
17 1.12067100417335e-05
18 9.84514372248668e-06
19 8.45455087983282e-06
20 7.14357338438276e-06
21 5.97300459048711e-06
22 5.01655995321926e-06
23 4.26919814344728e-06
24 3.70366456081683e-06
25 3.2944365102594e-06
26 3.01768250210444e-06
27 2.81792881651199e-06
28 2.65821904577024e-06
29 2.52814766099618e-06
30 2.4061698695732e-06
31 2.29569172915944e-06
32 2.19128514800104e-06
33 2.08343954000156e-06
34 1.98871907741704e-06
35 1.89591582966386e-06
36 1.73765715771879e-06
37 1.55540749346983e-06
38 1.49404343119386e-06
39 1.42717169637763e-06
40 1.354028768219e-06
41 1.28089038753387e-06
42 1.20655181490292e-06
43 1.12998998247349e-06
44 1.04725120309013e-06
45 9.67146888797288e-07
46 8.98303767371544e-07
47 8.33326566862524e-07
48 7.76850640704652e-07
49 7.28806071492727e-07
50 6.86797704929631e-07
51 6.51631182790879e-07
52 6.17751254594623e-07
53 5.90171396197547e-07
54 5.68820951230009e-07
55 5.54833775368024e-07
56 5.47499610092927e-07
57 5.42892621524516e-07
58 5.3708424729848e-07
59 5.3217530648908e-07
60 5.28331838722806e-07
61 5.2487121138256e-07
62 5.21663707786502e-07
63 5.19199943482818e-07
64 5.16729528499127e-07
65 5.14838063736534e-07
66 5.13071029217826e-07
67 5.11469409048004e-07
68 5.10751760884887e-07
69 5.08838695623126e-07
70 5.07936249505292e-07
71 5.0627176051421e-07
72 4.96109635150788e-07
73 4.94115738547407e-07
74 4.91717855766183e-07
75 4.89903925426916e-07
76 4.88051796310174e-07
77 4.8592585244478e-07
78 4.83535018247494e-07
79 4.81121446682664e-07
80 4.80101164157531e-07
81 4.78746187582146e-07
82 4.77654907626857e-07
83 4.76811607086347e-07
84 4.75996074555951e-07
85 4.75265323984786e-07
86 4.70177326405974e-07
87 4.73703863690389e-07
88 4.72897170311626e-07
89 4.71180840122543e-07
90 4.73154784685903e-07
91 4.6879034698577e-07
92 4.72560543585132e-07
93 4.71250189093553e-07
94 4.62019983160644e-07
95 4.70519353257259e-07
96 4.68557431076988e-07
97 4.62861180494656e-07
98 4.64982605308251e-07
99 4.5973737883287e-07
100 4.59398250995946e-07
101 4.57165356237965e-07
102 4.55775619911947e-07
103 4.54610955102908e-07
104 4.51623776598353e-07
105 4.496694430145e-07
106 4.47726563379547e-07
107 4.45827652129083e-07
108 4.43179487774614e-07
109 4.41382809412971e-07
110 4.37953588061646e-07
111 4.36630045896891e-07
112 4.35597797832088e-07
113 4.3381433556533e-07
114 4.32227267310736e-07
115 4.29682870617398e-07
116 4.2848071757362e-07
117 4.27904069510987e-07
118 4.25126842173995e-07
119 4.23191494292041e-07
120 4.23061351284559e-07
121 4.20994723526746e-07
122 4.19127786699391e-07
123 4.15429241229504e-07
124 4.14148331628894e-07
125 4.11985354276112e-07
126 4.10510580195478e-07
127 4.08978735322307e-07
128 3.9788127992324e-07
129 3.95086061644179e-07
130 3.90568487773635e-07
131 3.93015426425336e-07
132 3.90281968520867e-07
133 3.8795457157903e-07
134 3.85923272006039e-07
135 3.84391057650646e-07
136 3.83566316486394e-07
137 3.82115814545614e-07
138 3.8056217022131e-07
139 3.78862210936859e-07
140 3.79664413685532e-07
141 3.75687704945449e-07
142 3.75233526028751e-07
143 3.72726617570152e-07
144 3.72910278656491e-07
145 3.69333037042452e-07
146 3.67999405170849e-07
147 3.68859190302828e-07
148 3.67523966815497e-07
149 3.63345151299654e-07
150 3.64539602060177e-07
151 3.63270686420947e-07
152 3.62288659516707e-07
153 3.5963745403933e-07
154 3.5968832889921e-07
155 3.57701111397546e-07
156 3.5650882068694e-07
157 3.54410190084309e-07
158 3.54392597046171e-07
159 3.56451238303634e-07
160 3.55774744775772e-07
161 3.55635023652212e-07
162 3.54594192231161e-07
163 3.50707864527067e-07
164 3.53705871702914e-07
165 3.52424478933244e-07
166 3.4895077760666e-07
167 3.49267821775356e-07
168 3.46715211207993e-07
169 3.45264936640888e-07
170 3.46834269748797e-07
171 3.48053475818233e-07
172 3.46881677160127e-07
173 3.46985473242967e-07
174 3.46108834037295e-07
175 3.46129297668085e-07
176 3.44918589689769e-07
177 3.44373290772637e-07
178 3.43984510209339e-07
179 3.43149054060632e-07
180 3.43016722581524e-07
181 3.43042529493687e-07
182 3.42021166943596e-07
183 3.41843161777433e-07
184 3.41553260341243e-07
185 3.41293826977562e-07
186 3.40373134122274e-07
187 3.40198624826371e-07
188 3.39003236149438e-07
189 3.38637732966163e-07
190 3.37080706458437e-07
191 3.36071764195367e-07
192 3.35818413077504e-07
193 3.35235057491445e-07
194 3.27360822893752e-07
195 3.32930852664504e-07
196 3.30985557184249e-07
197 3.30307187823564e-07
198 3.29086702777204e-07
199 3.27390267784722e-07
200 3.26547507256691e-07
201 3.2482807910128e-07
202 3.24150079222818e-07
203 3.22870562285971e-07
204 3.20562520528256e-07
205 3.20346515536585e-07
206 3.18708430313563e-07
207 3.14772876208735e-07
208 3.17539758043495e-07
209 3.15849575827087e-07
210 3.14492581310333e-07
211 3.13249444161556e-07
212 3.1150182167039e-07
213 3.1174036507764e-07
214 3.08928406411724e-07
215 3.08366679746541e-07
216 3.07614413941337e-07
217 3.06542631278717e-07
218 3.04784350646514e-07
219 3.03548773672446e-07
220 3.02242227689931e-07
221 3.01695735061003e-07
222 3.00415877063642e-07
223 2.989899030581e-07
224 2.97666503001892e-07
225 2.96025746138184e-07
226 2.95121054705305e-07
227 2.9392074907264e-07
228 2.92915359523249e-07
229 2.89782150275641e-07
230 2.90066111574561e-07
231 2.88925946279051e-07
232 2.8733424528582e-07
233 2.861780785679e-07
234 2.84871958911026e-07
235 2.83561178093805e-07
236 2.82586654520856e-07
237 2.81721639794341e-07
238 2.79411352721581e-07
239 2.78668409237071e-07
240 2.77030835604819e-07
241 2.75958171869206e-07
242 2.75571608199243e-07
243 2.71877354407479e-07
244 2.72216539087822e-07
245 2.70268799340556e-07
246 2.69329319735334e-07
247 2.6758408466776e-07
248 2.65322398718126e-07
249 2.61684220959069e-07
250 2.62192486388813e-07
251 2.62060979139278e-07
252 2.60246054040181e-07
253 2.5757805133253e-07
254 2.57992155638931e-07
255 2.55330434129064e-07
256 2.54726131743155e-07
257 2.54159346013694e-07
258 2.51216476954141e-07
259 2.50637441467916e-07
260 2.47677093057064e-07
261 2.47796634766928e-07
262 2.46482159127481e-07
263 2.44895318246563e-07
264 2.43721672177344e-07
265 2.42551351448128e-07
266 2.36857815139047e-07
267 2.36269329434435e-07
268 2.34723543712789e-07
269 2.33534933613555e-07
270 2.32142170375482e-07
271 2.30435560411024e-07
272 2.28252659439931e-07
273 2.26892908017362e-07
274 2.25520736307772e-07
275 2.23635368001851e-07
276 2.21953868617675e-07
277 2.19977692950124e-07
278 2.18738719581779e-07
279 2.17127393398187e-07
280 2.14378388818659e-07
281 2.12870645555086e-07
282 2.12533223020728e-07
283 2.11070002364977e-07
284 2.09327339462106e-07
285 2.07950577646443e-07
286 2.06046962603068e-07
287 2.04236016543291e-07
288 2.02282521399866e-07
289 2.00619652446221e-07
290 1.98206606683016e-07
291 1.96629514448432e-07
292 1.95381915091275e-07
293 1.94022433674945e-07
294 1.92552093380982e-07
295 1.86659377732212e-07
296 1.85600185886869e-07
297 1.88344870366564e-07
298 1.86576343708111e-07
299 1.84989431772919e-07
300 1.8286124259248e-07
301 1.80799915483476e-07
302 1.79070340777798e-07
303 1.7788741502045e-07
304 1.75554490056129e-07
305 1.74101060679277e-07
306 1.6919767631407e-07
307 1.67863760225373e-07
308 1.66020257097443e-07
309 1.64950819225851e-07
310 1.62922162871837e-07
311 1.6392569079926e-07
312 1.62288671390343e-07
313 1.59915188646664e-07
314 1.57490077867806e-07
315 1.55754008801523e-07
316 1.54222675519122e-07
317 1.52763604432948e-07
318 1.51086155142366e-07
319 1.49306330854415e-07
320 1.47689235063808e-07
321 1.4600260556108e-07
322 1.44406698154853e-07
323 1.42986692708291e-07
324 1.41297832101372e-07
325 1.40578293894578e-07
326 1.38192660870118e-07
327 1.36672852590891e-07
328 1.36492332103444e-07
329 1.35106233756233e-07
330 1.32453479295691e-07
331 1.3025051259774e-07
332 1.30812352949761e-07
333 1.29147423422182e-07
334 1.28771375784709e-07
335 1.2680231975537e-07
336 1.25178615917321e-07
337 1.24225081776785e-07
338 1.22762514820351e-07
339 1.2242954028352e-07
340 1.20275927883995e-07
341 1.19261855502373e-07
342 1.17956446388234e-07
343 1.16977602715451e-07
344 1.15601068273463e-07
345 1.1475108152581e-07
346 1.13774163423841e-07
347 1.12837341248451e-07
348 1.12457342993366e-07
349 1.1149711554026e-07
350 1.10542409004211e-07
351 1.09466583353424e-07
352 1.09088624355991e-07
353 1.08025552947311e-07
354 1.07347510436284e-07
355 1.06878019323631e-07
356 1.0568611941153e-07
357 1.05200690825313e-07
358 1.04092578112613e-07
359 1.042390849193e-07
360 1.03430430442586e-07
361 1.02548227687294e-07
362 1.01614702430197e-07
363 1.00609170772259e-07
364 1.00611913467219e-07
365 1.0006633743842e-07
366 9.92525244214448e-08
367 9.82997789833462e-08
368 9.78494441028488e-08
369 9.77575353999782e-08
370 9.75283853676956e-08
371 9.6895483636672e-08
372 9.62775814628003e-08
373 9.51883123434527e-08
374 9.54859302737532e-08
375 9.47747764712403e-08
376 9.44299571870033e-08
377 9.42465945286131e-08
378 9.35086248432526e-08
379 9.29652657077895e-08
380 9.18740923339101e-08
381 9.17821552093301e-08
382 9.19023861456481e-08
383 9.15121702860233e-08
384 9.09879886989984e-08
385 9.10487614191879e-08
386 9.05383998883735e-08
387 9.00645886758866e-08
388 8.97666865284918e-08
389 8.93966642934174e-08
390 8.90070737114002e-08
391 8.83842261600876e-08
392 8.82827038140022e-08
393 8.83223592040849e-08
394 8.77856862757653e-08
395 8.7525684477896e-08
396 8.74269474593348e-08
397 8.70254979190577e-08
398 8.64837446101774e-08
399 8.74912160497843e-08
400 8.63511999682487e-08
401 8.61796820572636e-08
402 8.58393747193986e-08
403 8.54762163271516e-08
404 8.50460679657772e-08
405 8.49265759939044e-08
406 8.51713011229549e-08
407 8.48243075779465e-08
408 8.41081373437191e-08
409 8.43323562094156e-08
410 8.37366442851817e-08
411 8.37420017774093e-08
412 8.35475688631959e-08
413 8.3450665044893e-08
414 8.35488762618297e-08
415 8.29230799581637e-08
416 8.2635054354796e-08
417 8.30019928343972e-08
418 8.26126367314828e-08
419 8.28034814048806e-08
420 8.19678334096352e-08
421 8.21340719880936e-08
422 8.18860783624586e-08
423 8.20616250507555e-08
424 8.11700289204964e-08
425 8.10934466244362e-08
426 8.08451758871342e-08
427 8.1394865958373e-08
428 8.05423994165722e-08
429 8.04262825226942e-08
430 8.21856502852825e-08
431 8.25271513349435e-08
432 8.22400565425596e-08
433 8.2255503741635e-08
434 8.1934686591012e-08
435 8.20052861172371e-08
436 8.16852647744781e-08
437 8.15552638755435e-08
438 8.13286646916822e-08
439 8.12293095009409e-08
440 8.10293414588159e-08
441 8.09286362368766e-08
442 8.07609126241005e-08
443 8.08726667855808e-08
444 8.05098281375649e-08
445 8.04429376444205e-08
446 8.03110253855266e-08
447 8.02856945369967e-08
448 8.03752939759761e-08
449 8.01904320724134e-08
450 8.00425254965376e-08
451 7.99636481474408e-08
452 7.96626196120087e-08
453 7.9541777608938e-08
454 7.94286663108323e-08
455 7.93341641269762e-08
456 7.90597809441351e-08
457 7.90569814057562e-08
458 7.89310306004154e-08
459 7.8788616519887e-08
460 7.87736880170087e-08
461 7.8631089195369e-08
462 7.85068934305855e-08
463 7.83242199986489e-08
464 7.79851632159989e-08
465 7.77956614683717e-08
466 7.7429469058643e-08
467 7.73076109794602e-08
468 7.70164447772004e-08
469 7.70694370544334e-08
470 7.68109345017365e-08
471 7.67822356806391e-08
472 7.63120482361046e-08
473 7.61127409987239e-08
474 7.59780363068785e-08
475 7.57882929747211e-08
476 7.56541993496285e-08
477 7.54362048382973e-08
478 7.54304636529923e-08
479 7.52606794662825e-08
480 7.53345688053741e-08
481 7.52114814872584e-08
482 7.47513482224349e-08
483 7.47708668313862e-08
484 7.47019868185816e-08
485 7.45531778534314e-08
486 7.44225800985987e-08
487 7.40037222612955e-08
488 7.42115844332147e-08
489 7.40053565095877e-08
490 7.45011377034643e-08
491 7.36161496206478e-08
492 7.46160893072556e-08
493 7.36362650854971e-08
494 7.34334619778565e-08
495 7.31875857695741e-08
496 7.33436138489196e-08
497 7.29168263546853e-08
498 7.27772686559547e-08
499 7.28735898292143e-08
500 7.27944780010148e-08
501 7.25606454921035e-08
502 7.27232745134643e-08
503 7.26717317434122e-08
504 7.23381390344002e-08
505 7.21271007364521e-08
506 7.21248767376892e-08
507 7.22814448295139e-08
508 7.226330467347e-08
509 7.21012014537337e-08
510 7.20330959325111e-08
511 7.1965139625263e-08
512 7.1609989049648e-08
513 7.19850561381463e-08
514 7.15590715572034e-08
515 7.16738952633023e-08
516 7.15271610829404e-08
517 7.14774444077193e-08
518 7.13070775759661e-08
519 7.1133214873953e-08
520 7.12549024228792e-08
521 7.12661503143863e-08
522 7.13276975261579e-08
523 7.13123000650739e-08
524 7.10464931330534e-08
525 7.11089924720909e-08
526 7.10214109744811e-08
527 7.06675322703632e-08
528 7.09229652784416e-08
529 7.05523106603323e-08
530 7.05747922324917e-08
531 7.04106568605312e-08
532 7.05836669112614e-08
533 7.02129554497333e-08
534 7.04834093312456e-08
535 7.04059033296289e-08
536 7.02656919315814e-08
537 7.00060809322167e-08
538 6.99441216056584e-08
539 6.96329678362417e-08
540 6.95762167879366e-08
541 6.94193502681628e-08
542 6.92283848024999e-08
543 6.95544812856497e-08
544 6.92769148713523e-08
545 6.94321471428339e-08
546 6.86411638639584e-08
547 6.83279068880438e-08
548 6.88404142579202e-08
549 6.88231551748686e-08
550 6.88289105710282e-08
551 6.85611070139203e-08
552 6.83424730141269e-08
553 6.86772310132255e-08
554 6.81588261386423e-08
555 6.8072267822572e-08
556 6.81833469684534e-08
557 6.80982239487093e-08
558 6.78834979339626e-08
559 6.77377229862941e-08
560 6.75848852438321e-08
561 6.78566962619698e-08
562 6.73964635211632e-08
563 6.73686528784856e-08
564 6.72743070140314e-08
565 6.71898092718948e-08
566 6.67543318400021e-08
567 6.67273880594621e-08
568 6.67829596068259e-08
569 6.6441302237763e-08
570 6.64202772782119e-08
571 6.64525856564069e-08
572 6.6304458812283e-08
573 6.49310862854691e-08
574 6.50098499477281e-08
575 6.4917990982849e-08
576 6.47230962158574e-08
577 6.46752837951681e-08
578 6.45924274067511e-08
579 6.47273026288531e-08
580 6.46097575440763e-08
581 6.44243556280344e-08
582 6.43857731574826e-08
583 6.4307990044199e-08
584 6.43397797261969e-08
585 6.41550812474634e-08
586 6.40477750835089e-08
587 6.39166231053423e-08
588 6.39430766113946e-08
589 6.39234869481697e-08
590 6.38723420820497e-08
591 6.36204404713681e-08
592 6.38721218138016e-08
593 6.35410586369289e-08
594 6.35393675452178e-08
595 6.33589607446083e-08
596 6.33516776815668e-08
597 6.33478762779305e-08
598 6.32829042501726e-08
599 6.31403125339602e-08
600 6.31896028835399e-08
601 6.29417087338879e-08
602 6.272330210777e-08
603 6.2882001827802e-08
604 6.3187179932811e-08
605 6.26345340037915e-08
606 6.24980032171152e-08
607 6.25987723879007e-08
608 6.24648208713552e-08
609 6.23197848881318e-08
610 6.25345109028785e-08
611 6.22205647005103e-08
612 6.20220887981304e-08
613 6.20373370452398e-08
614 6.21052436144964e-08
615 6.21273414935786e-08
616 6.21414883994476e-08
617 6.19979871885334e-08
618 6.19264568513245e-08
619 6.15174045037747e-08
620 6.18328357404607e-08
621 6.20644087234723e-08
622 6.16775963635519e-08
623 6.18817850295272e-08
624 6.18854869571805e-08
625 6.15430124639715e-08
626 6.13808950333805e-08
627 6.16820798882145e-08
628 6.17921713796932e-08
629 6.120565387846e-08
630 6.15528250591524e-08
631 6.11622539281598e-08
632 6.15145694382591e-08
633 6.09852151001178e-08
634 6.10229164976772e-08
635 6.11311534726156e-08
636 6.11614865420052e-08
637 6.10338375395258e-08
638 6.10182695481853e-08
639 6.0545005453605e-08
640 6.05069345738229e-08
641 6.04639467383095e-08
642 6.04053269626093e-08
643 6.04592926833902e-08
644 6.03422236622464e-08
645 6.05787420226989e-08
646 6.03150667188856e-08
647 6.01386602738785e-08
648 6.01390581778105e-08
649 6.00633924818794e-08
650 6.00771556946711e-08
651 6.00704908038097e-08
652 5.99290075342651e-08
653 5.99248508592609e-08
654 6.00989054078127e-08
655 5.9809906360897e-08
656 5.98326863610055e-08
657 5.97154041770409e-08
658 5.96956937215509e-08
659 5.96155160792478e-08
660 5.95197811037451e-08
661 5.96111746631323e-08
662 5.95359246347016e-08
663 5.93888849209634e-08
664 5.93218523192718e-08
665 5.92705760027457e-08
666 5.93645701485457e-08
667 5.92264228771455e-08
668 5.92234563612237e-08
669 5.9085436987516e-08
670 5.91380278081033e-08
671 5.90690696355978e-08
672 5.9010279329641e-08
673 5.89489417279765e-08
674 5.89522279881294e-08
675 5.88487623076617e-08
676 5.88527058198451e-08
677 5.87834207976812e-08
678 5.87317927625008e-08
679 5.87919046779461e-08
680 5.87995252487872e-08
681 5.86490713772037e-08
682 5.85811861242291e-08
683 5.84309347573253e-08
684 5.86097996801982e-08
685 5.83947077359426e-08
686 5.83247334873249e-08
687 5.82137218430034e-08
688 5.82032662066467e-08
689 5.823329374266e-08
690 5.8125948498855e-08
691 5.84139812076501e-08
692 5.79918193466256e-08
693 5.80074868139491e-08
694 5.79058898608764e-08
695 5.79088492713709e-08
696 5.77776688714948e-08
697 5.77583598726505e-08
698 5.77851757554981e-08
699 5.76926240114517e-08
700 5.7585950230532e-08
701 5.75896095256212e-08
702 5.77448560079574e-08
703 5.74752228033049e-08
704 5.74707996747748e-08
705 5.74345548898236e-08
706 5.73163099204521e-08
707 5.72608733762081e-08
708 5.72479059712805e-08
709 5.73457903385588e-08
710 5.71433886875639e-08
711 5.70956437684345e-08
712 5.70312010950147e-08
713 5.70146667655536e-08
714 5.71624383383096e-08
715 5.70199922833581e-08
716 5.69002445161004e-08
717 5.6788767466287e-08
718 5.67334801360175e-08
719 5.67781768268105e-08
720 5.66992213180129e-08
721 5.67357361092036e-08
722 5.6597372122269e-08
723 5.66178393057726e-08
724 5.65681226305514e-08
725 5.65826354659293e-08
726 5.66088083076011e-08
727 5.66848044059043e-08
728 5.64165425487317e-08
729 5.6437809092813e-08
730 5.63353559357438e-08
731 5.64220421495065e-08
732 5.6237830392547e-08
733 5.61857298464474e-08
734 5.61329400738941e-08
735 5.61136310750499e-08
736 5.61236035423462e-08
737 5.60247990222251e-08
738 5.60009851824361e-08
739 5.59605659589124e-08
740 5.58776065417987e-08
741 5.58591537469511e-08
742 5.58793544769287e-08
743 5.56971571086251e-08
744 5.57615287277713e-08
745 5.56990222833065e-08
746 5.56655876948753e-08
747 5.56207346846804e-08
748 5.57729826766717e-08
749 5.54694530308097e-08
750 5.5539022270068e-08
751 5.53967360872321e-08
752 5.54483428061303e-08
753 5.53062626806877e-08
754 5.54263195340354e-08
755 5.5251586417171e-08
756 5.52935297548629e-08
757 5.51814558491515e-08
758 5.52304406653548e-08
759 5.51032215412306e-08
760 5.5124409925611e-08
761 5.50307284186147e-08
762 5.49795764470673e-08
763 5.50429533063834e-08
764 5.4999190979288e-08
765 5.48931105015527e-08
766 5.49040528596834e-08
767 5.49804148874955e-08
768 5.49048699838295e-08
769 5.47627010405449e-08
770 5.47683463025805e-08
771 5.46866054662587e-08
772 5.46796457001619e-08
773 5.46852163552103e-08
774 5.46207594709358e-08
775 5.45640368443401e-08
776 5.45560006059986e-08
777 5.44747429387371e-08
778 5.43882947567909e-08
779 5.43512825856851e-08
780 5.43050191481598e-08
781 5.42973772610367e-08
782 5.42475255826957e-08
783 5.41849054513932e-08
784 5.41768585549107e-08
785 5.41675184706492e-08
786 5.4040995678406e-08
787 5.40174127650062e-08
788 5.39964553070149e-08
789 5.3981842995654e-08
790 5.39434559243546e-08
791 5.39427169599094e-08
792 5.3848705050541e-08
793 5.38129540927912e-08
794 5.37704067937739e-08
795 5.37655147070382e-08
796 5.37309858827939e-08
797 5.36974305020976e-08
798 5.36483817370481e-08
799 5.36447295473863e-08
800 5.36378479409905e-08
801 5.35959365777217e-08
802 5.35652482369642e-08
803 5.35460067396798e-08
804 5.35151798430888e-08
805 5.34221271664137e-08
806 5.34290904852242e-08
807 5.34199955382064e-08
808 5.33607682484671e-08
809 5.33347197517742e-08
810 5.33093071908297e-08
811 5.32890318538648e-08
812 5.32538173558805e-08
813 5.32245856277314e-08
814 5.31885469001736e-08
815 5.31570485406974e-08
816 5.31072537057753e-08
817 5.3113705433816e-08
818 5.30833474954306e-08
819 5.30600772208345e-08
820 5.30148369648487e-08
821 5.30287707078969e-08
822 5.29906927226875e-08
823 5.29303889607036e-08
824 5.29095594004048e-08
825 5.28727213122693e-08
826 5.28504067176527e-08
827 5.28272465771806e-08
828 5.27960253293713e-08
829 5.27293835261844e-08
830 5.2730051436356e-08
831 5.27009120787625e-08
832 5.26276977552698e-08
833 5.26134371625631e-08
834 5.26240206966122e-08
835 5.25758565572687e-08
836 5.25800416539823e-08
837 5.25294723274783e-08
838 5.25229673087324e-08
839 5.25120249506017e-08
840 5.24240526544872e-08
841 5.23597663004693e-08
842 5.23245518024851e-08
843 5.23242924543865e-08
844 5.23170555766228e-08
845 5.22399190572287e-08
846 5.22107157507889e-08
847 5.21641503326009e-08
848 5.21338989756259e-08
849 5.21447631740557e-08
850 5.21119858376551e-08
851 5.20492449140875e-08
852 5.20658041125444e-08
853 5.19628535755601e-08
854 5.19557623590572e-08
855 5.1901722031289e-08
856 5.19041165603085e-08
857 5.18502289992284e-08
858 5.18370733004758e-08
859 5.18101188617948e-08
860 5.18100797819443e-08
861 5.17362401808441e-08
862 5.16913978287903e-08
863 5.16827505236961e-08
864 5.16840437114752e-08
865 5.16487048685121e-08
866 5.16280387330426e-08
867 5.16078628720607e-08
868 5.15372100551303e-08
869 5.15117939414722e-08
870 5.14986489008606e-08
871 5.14462250578163e-08
872 5.14269231643993e-08
873 5.1401851663968e-08
874 5.13658093836966e-08
875 5.13154141401628e-08
876 5.13374445176851e-08
877 5.12776736627529e-08
878 5.12465554436403e-08
879 5.12056921309068e-08
880 5.11534459235463e-08
881 5.11219049315059e-08
882 5.11219653276385e-08
883 5.1097149622592e-08
884 5.11173610107107e-08
885 5.10632567340963e-08
886 5.10142896814614e-08
887 5.09928490544098e-08
888 5.09434485707061e-08
889 5.09054594033387e-08
890 5.09145721139248e-08
891 5.08324866643761e-08
892 5.08841289104112e-08
893 5.08312290037338e-08
894 5.07703141749971e-08
895 5.07815158812264e-08
896 5.0715630806053e-08
897 5.0693863329343e-08
898 5.06748563111614e-08
899 5.06910069475452e-08
900 5.06087936003041e-08
901 5.06455464233113e-08
902 5.06389170595867e-08
903 5.05371744452532e-08
904 5.05243420434454e-08
905 5.04930959266403e-08
906 5.04764372522004e-08
907 5.04654487087919e-08
908 5.04342274609826e-08
909 5.04251147503965e-08
910 5.04321064909163e-08
911 5.03975812193858e-08
912 5.03923480721369e-08
913 5.03107386862212e-08
914 5.03090156200869e-08
915 5.02719750272718e-08
916 5.02992740791797e-08
917 5.02490031806246e-08
918 5.03256778472405e-08
919 5.03186541322975e-08
920 5.02287598180828e-08
921 5.02023311810262e-08
922 5.01811925346374e-08
923 5.01743002701005e-08
924 5.01936554542226e-08
925 5.01394836760483e-08
926 5.00595014329974e-08
927 5.0075875890343e-08
928 4.99967214295793e-08
929 5.00471699638183e-08
930 4.99876904314078e-08
931 4.99717600632721e-08
932 4.99237913231809e-08
933 4.99310601753677e-08
934 4.98832761763879e-08
935 4.99380945484518e-08
936 4.98439263196815e-08
937 4.9822496350771e-08
938 4.97906711416363e-08
939 4.98197785248067e-08
940 4.97540426636078e-08
941 4.98051591080184e-08
942 4.97367622642741e-08
943 4.97578440672442e-08
944 4.96852550213589e-08
945 4.97363785711968e-08
946 4.96641980873846e-08
947 4.96795173887676e-08
948 4.95897580776727e-08
949 4.96287029250198e-08
950 4.95632157537784e-08
951 4.95705414493841e-08
952 4.95640399833519e-08
953 4.96192065213563e-08
954 4.9529894852185e-08
955 4.95176024628563e-08
956 4.94577285792275e-08
957 4.9456129858072e-08
958 4.94123497674082e-08
959 4.94272249795813e-08
960 4.93870615514425e-08
961 4.93945790935868e-08
962 4.93325238437592e-08
963 4.93295360115553e-08
964 4.9288559011984e-08
965 4.93080314356575e-08
966 4.92705893861967e-08
967 4.92929963513689e-08
968 4.92552345576769e-08
969 4.92846261579416e-08
970 4.92863776457853e-08
971 4.92507155058775e-08
972 4.92834537624276e-08
973 4.92476779356821e-08
974 4.92030522991627e-08
975 4.92080935998729e-08
976 4.91351883624702e-08
977 4.91233613786335e-08
978 4.90312359602285e-08
979 4.90454716839395e-08
980 4.89911968770684e-08
981 4.90316764967247e-08
982 4.89389329061396e-08
983 4.89655036517433e-08
984 4.89084293064934e-08
985 4.89620930466117e-08
986 4.88938631804103e-08
987 4.89040132833907e-08
988 4.8843904920659e-08
989 4.89240683521075e-08
990 4.87969487039663e-08
991 4.8838728616829e-08
992 4.8756625403712e-08
993 4.87984728181345e-08
994 4.87428373219245e-08
995 4.87723745834501e-08
996 4.87199010024142e-08
997 4.87377214142271e-08
998 4.87150302319606e-08
999 4.87264379955832e-08
1000 4.86565987500853e-08
1001 4.86511524400157e-08
1002 4.86192028859023e-08
1003 4.86831623902617e-08
1004 4.86069033911463e-08
1005 4.85977800224191e-08
1006 4.85525148974375e-08
1007 4.85393485405439e-08
1008 4.85012421336251e-08
1009 4.84662514566025e-08
1010 4.85133924144066e-08
1011 4.84681805801301e-08
1012 4.84795208421929e-08
1013 4.84835886993551e-08
1014 4.84208655393559e-08
1015 4.84102145037468e-08
1016 4.83774442727736e-08
1017 4.83831321673733e-08
1018 4.83574957854671e-08
1019 4.83560889108503e-08
1020 4.83451145782965e-08
1021 4.83230842007742e-08
1022 4.8282227993468e-08
1023 4.82144493219039e-08
1024 4.8248637085635e-08
1025 4.81734510060505e-08
1026 4.81363464643891e-08
1027 4.80256581170124e-08
1028 4.81557798082122e-08
1029 4.81970801047282e-08
1030 4.82576112403876e-08
1031 4.80985882234108e-08
1032 4.79712340961669e-08
1033 4.7924523016718e-08
1034 4.78751971400015e-08
1035 4.78753925392539e-08
1036 4.78374140300275e-08
1037 4.78318789021159e-08
1038 4.77856545444411e-08
1039 4.78279531535009e-08
1040 4.7783100143306e-08
1041 4.77405741605708e-08
1042 4.7739487030185e-08
1043 4.77530655018654e-08
1044 4.76845514185698e-08
1045 4.76813504235452e-08
1046 4.7682469528354e-08
1047 4.77559929379368e-08
1048 4.76845087860056e-08
1049 4.77892641015387e-08
1050 4.75888199957808e-08
1051 4.75607393468636e-08
1052 4.75590837822892e-08
1053 4.75508379338407e-08
1054 4.74984460652195e-08
1055 4.7628503807573e-08
1056 4.76028887419488e-08
1057 4.75111434639075e-08
1058 4.74588972565471e-08
1059 4.75452033299462e-08
1060 4.74197108246699e-08
1061 4.74542360962005e-08
1062 4.73579611082187e-08
1063 4.73744812268251e-08
1064 4.73154777580476e-08
1065 4.73418850788221e-08
1066 4.72334988899092e-08
1067 4.71854590955445e-08
1068 4.72626062730797e-08
1069 4.71680507985184e-08
1070 4.72391334938038e-08
1071 4.72561154651885e-08
1072 4.72470809143033e-08
1073 4.7073264397568e-08
1074 4.71093990483951e-08
1075 4.70570462596243e-08
1076 4.71077150621113e-08
1077 4.7082842513646e-08
1078 4.71059493634129e-08
1079 4.69913459255622e-08
1080 4.70483243475428e-08
1081 4.69661607382932e-08
1082 4.70286387610486e-08
1083 4.69078749176788e-08
1084 4.69880667708367e-08
1085 4.68453400515045e-08
1086 4.68801175657063e-08
1087 4.68580410029062e-08
1088 4.68947263243535e-08
1089 4.68377194806635e-08
1090 4.68647201046224e-08
1091 4.68252672192193e-08
1092 4.68213094961811e-08
1093 4.67950869165179e-08
1094 4.67490437472406e-08
1095 4.67714507124128e-08
1096 4.67499710055108e-08
1097 4.67369005718865e-08
1098 4.670971875953e-08
1099 4.66497205309224e-08
1100 4.6729518032862e-08
1101 4.66097027640444e-08
1102 4.6589267554964e-08
1103 4.66211957927953e-08
1104 4.65344314193317e-08
1105 4.65970373397795e-08
1106 4.66051801595313e-08
1107 4.65847342923098e-08
1108 4.65425671336561e-08
1109 4.65016825046405e-08
1110 4.65328291454625e-08
1111 4.6477143911261e-08
1112 4.6487087956848e-08
1113 4.6467288683516e-08
1114 4.64828602275702e-08
1115 4.64509781750166e-08
1116 4.64388172360941e-08
1117 4.63957867680165e-08
1118 4.64081963968965e-08
1119 4.63938825134846e-08
1120 4.63569378439388e-08
1121 4.63545077877825e-08
1122 4.63161740071882e-08
1123 4.63114382398544e-08
1124 4.64351046502998e-08
1125 4.62911593501758e-08
1126 4.63280720452985e-08
1127 4.62471874129733e-08
1128 4.62768454667639e-08
1129 4.62328415551383e-08
1130 4.62130032019559e-08
1131 4.62133940004605e-08
1132 4.6262432107369e-08
1133 4.62262157441273e-08
1134 4.62095428588327e-08
1135 4.6219771121514e-08
1136 4.62208049611945e-08
1137 4.62269760248546e-08
1138 4.62100899767393e-08
1139 4.61675497831493e-08
1140 4.61400944118395e-08
1141 4.61029827647508e-08
1142 4.61328575340758e-08
1143 4.61457503320162e-08
1144 4.60701166105082e-08
1145 4.60621940590045e-08
1146 4.60798759149839e-08
1147 4.6046835677771e-08
1148 4.60088038778395e-08
1149 4.60339144581212e-08
1150 4.60343123620532e-08
1151 4.59830395982408e-08
1152 4.59778242145603e-08
1153 4.59823112919366e-08
1154 4.59789823992196e-08
1155 4.59477540459829e-08
1156 4.59476829917094e-08
1157 4.59126425766954e-08
1158 4.59140458985985e-08
1159 4.58713174111836e-08
1160 4.5880724997005e-08
1161 4.58625777355337e-08
1162 4.58464235464362e-08
1163 4.58167761507866e-08
1164 4.58069600028921e-08
1165 4.57530013875385e-08
1166 4.58140334558266e-08
1167 4.57505500151001e-08
1168 4.57764457451049e-08
1169 4.57170514778227e-08
1170 4.57527669084357e-08
1171 4.57380373575234e-08
1172 4.57222704142168e-08
1173 4.56457094344387e-08
1174 4.57079138982408e-08
1175 4.56286457506394e-08
1176 4.56793323166949e-08
1177 4.56437305729196e-08
1178 4.56266668891203e-08
1179 4.55916122632516e-08
1180 4.56265958348467e-08
1181 4.55822544154216e-08
1182 4.55736319793232e-08
1183 4.55201636384572e-08
1184 4.55545148270176e-08
1185 4.5488665278981e-08
1186 4.55062014736995e-08
1187 4.5460758713034e-08
1188 4.55033877244659e-08
1189 4.54361313018126e-08
1190 4.54695765483848e-08
1191 4.53848016945813e-08
1192 4.54284077022749e-08
1193 4.53331452376915e-08
1194 4.54191990684194e-08
1195 4.53535164979257e-08
1196 4.53774120501294e-08
1197 4.53027908520198e-08
1198 4.53552573276284e-08
1199 4.52759572056038e-08
1200 4.53317632320704e-08
1201 4.52772965786608e-08
1202 4.52801387496038e-08
1203 4.52205561884966e-08
1204 4.52713457832488e-08
1205 4.52019115471103e-08
1206 4.52392505678745e-08
1207 4.51654713629068e-08
1208 4.52016202245886e-08
1209 4.51284734026558e-08
1210 4.51668817902373e-08
1211 4.51281323421426e-08
1212 4.51571722237532e-08
1213 4.50837376320123e-08
1214 4.51097825759916e-08
1215 4.50579342725632e-08
1216 4.51032882153868e-08
1217 4.50287096498414e-08
1218 4.50913120175755e-08
1219 4.49856649709091e-08
1220 4.50014141506472e-08
1221 4.50245103422731e-08
1222 4.4987203295932e-08
1223 4.49935946278401e-08
1224 4.49905286359353e-08
1225 4.48799397645416e-08
1226 4.49659474099917e-08
1227 4.4890388295471e-08
1228 4.49346941877593e-08
1229 4.48450840906389e-08
1230 4.4895067219386e-08
1231 4.48271642028431e-08
1232 4.4886007799505e-08
1233 4.47968702133039e-08
1234 4.47979431328349e-08
1235 4.48116601603488e-08
1236 4.48037908995502e-08
1237 4.47801404845904e-08
1238 4.46821637467565e-08
1239 4.47043042584028e-08
1240 4.47414727489104e-08
1241 4.46931345265966e-08
1242 4.47233539091485e-08
1243 4.46547900878613e-08
1244 4.46826184941074e-08
1245 4.46480541427263e-08
1246 4.45776642266082e-08
1247 4.4598174042676e-08
1248 4.46060255399061e-08
1249 4.46614905058595e-08
1250 4.46088392891397e-08
1251 4.46229080353078e-08
1252 4.45690666595056e-08
1253 4.45649064317877e-08
1254 4.45380834435127e-08
1255 4.44953478506704e-08
1256 4.45224372924713e-08
1257 4.44694414625246e-08
1258 4.45087877665173e-08
1259 4.44398011723024e-08
1260 4.4474436577957e-08
1261 4.4450413128061e-08
1262 4.43985186393547e-08
1263 4.44435634960882e-08
1264 4.44104237828924e-08
1265 4.44167334023859e-08
1266 4.43675034489388e-08
1267 4.42970282676924e-08
1268 4.43464536203919e-08
1269 4.44059651272255e-08
1270 4.43510792536017e-08
1271 4.43618723977579e-08
1272 4.42926548771538e-08
1273 4.42656258314855e-08
1274 4.42659704447124e-08
1275 4.42620766705204e-08
1276 4.42432082081723e-08
1277 4.41871286227524e-08
1278 4.42485443841178e-08
1279 4.41759091529548e-08
1280 4.42034107095424e-08
1281 4.41687255658962e-08
1282 4.41602381329176e-08
1283 4.41514593774173e-08
1284 4.4178506186654e-08
1285 4.40631922060675e-08
1286 4.40750156371905e-08
1287 4.40539906776394e-08
1288 4.40273595359031e-08
1289 4.40337402096702e-08
1290 4.41142589124865e-08
1291 4.4026741363723e-08
1292 4.4077310690227e-08
1293 4.39767511295486e-08
1294 4.40581437999299e-08
1295 4.40657466072025e-08
1296 4.39829257459223e-08
1297 4.40055742956247e-08
1298 4.40023129044675e-08
1299 4.40368133070024e-08
1300 4.39452598754997e-08
1301 4.38966978322242e-08
1302 4.39438458954555e-08
1303 4.39705338806107e-08
1304 4.39319336464905e-08
1305 4.39270770868916e-08
1306 4.38364509136591e-08
1307 4.38863203555684e-08
1308 4.38212914843916e-08
1309 4.38856559981105e-08
1310 4.38271818836711e-08
1311 4.38221547938156e-08
1312 4.37873772796138e-08
1313 4.36654019608795e-08
1314 4.38045404393961e-08
1315 4.38289688986515e-08
1316 4.37167919642434e-08
1317 4.37989697843477e-08
1318 4.36928253577662e-08
1319 4.37695106825231e-08
1320 4.36645315460282e-08
1321 4.37411920017894e-08
1322 4.36420322103004e-08
1323 4.37156550958662e-08
1324 4.36187370667085e-08
1325 4.37013873977321e-08
1326 4.3599573729125e-08
1327 4.36832969796797e-08
1328 4.35716387414686e-08
1329 4.36046008189805e-08
1330 4.36663789571412e-08
1331 4.35421654287893e-08
1332 4.36086260435786e-08
1333 4.34530882387207e-08
1334 4.35823892530607e-08
1335 4.35522018449319e-08
1336 4.35345164362388e-08
1337 4.35222737849017e-08
1338 4.35280718136255e-08
1339 4.3494452484083e-08
1340 4.35161915390836e-08
1341 4.34906795021561e-08
1342 4.35966107659169e-08
1343 4.35434337475726e-08
1344 4.34536353566273e-08
1345 4.3394795312679e-08
1346 4.34079190370085e-08
1347 4.35063682857617e-08
1348 4.33963265322745e-08
1349 4.33478106742768e-08
1350 4.3512713432392e-08
1351 4.33323705806288e-08
1352 4.33531752719318e-08
1353 4.33138644950759e-08
1354 4.33304805369517e-08
1355 4.32888462853498e-08
1356 4.3432279994704e-08
1357 4.32611564349372e-08
1358 4.3270958371977e-08
1359 4.32472830880215e-08
1360 4.32513687087521e-08
1361 4.32272599937278e-08
1362 4.32539835060197e-08
1363 4.32153832718996e-08
1364 4.32150848439505e-08
1365 4.31874376261021e-08
1366 4.31931361788429e-08
1367 4.31628208730217e-08
1368 4.31659650246274e-08
1369 4.31368647468844e-08
1370 4.30352820046664e-08
1371 4.30035633769421e-08
1372 4.31257802802065e-08
1373 4.30990816369103e-08
1374 4.31051461191601e-08
1375 4.30776019300083e-08
1376 4.30751647684247e-08
1377 4.30456132960444e-08
1378 4.30443805043979e-08
1379 4.30270681306411e-08
1380 4.30161826159292e-08
1381 4.28879118885561e-08
1382 4.30101430026752e-08
1383 4.298077982412e-08
1384 4.29722177841541e-08
1385 4.2958347989952e-08
1386 4.29646895838687e-08
1387 4.293481126183e-08
1388 4.29170263771539e-08
1389 4.29049187289365e-08
1390 4.2925552889983e-08
1391 4.29366409093745e-08
1392 4.29018847114548e-08
1393 4.28964703758083e-08
1394 4.28986730582892e-08
1395 4.28691642184731e-08
1396 4.28548041497834e-08
1397 4.28522497486483e-08
1398 4.28695940968282e-08
1399 4.28286028864022e-08
1400 4.28172519661985e-08
1401 4.28063202662088e-08
1402 4.28232205251788e-08
1403 4.27924788937162e-08
1404 4.278754772713e-08
1405 4.27404742708859e-08
1406 4.27392912172309e-08
1407 4.27301571903627e-08
1408 4.2763247165567e-08
1409 4.27354720500261e-08
1410 4.2707014813459e-08
1411 4.27086455090375e-08
1412 4.27421191773192e-08
1413 4.2715708303831e-08
1414 4.26964348321235e-08
1415 4.26664072961103e-08
1416 4.2686341572562e-08
1417 4.26639523709582e-08
1418 4.26596216129838e-08
1419 4.2646920661582e-08
1420 4.26679598319879e-08
1421 4.26293773614361e-08
1422 4.26061212976947e-08
1423 4.24961719147632e-08
1424 4.26227231287157e-08
1425 4.25788826419193e-08
1426 4.25848760698955e-08
1427 4.25710027229798e-08
1428 4.25867447972905e-08
1429 4.25573993823036e-08
1430 4.25384136804041e-08
1431 4.24741202209589e-08
1432 4.25269774950721e-08
1433 4.25774722145889e-08
1434 4.24085406791619e-08
1435 4.24755732808535e-08
1436 4.24935571174956e-08
1437 4.24465866899482e-08
1438 4.24810480126325e-08
1439 4.23898285362156e-08
1440 4.24427426537477e-08
1441 4.24035278001611e-08
1442 4.24086543659996e-08
1443 4.23848831587748e-08
1444 4.24112656105535e-08
1445 4.24140758070735e-08
1446 4.24074180216394e-08
1447 4.24124486642086e-08
1448 4.23929904513898e-08
1449 4.23906598712165e-08
1450 4.23862012155496e-08
1451 4.23573816021872e-08
1452 4.23807868799031e-08
1453 4.23479882272204e-08
1454 4.23549018080394e-08
1455 4.23308357255792e-08
1456 4.23312371822249e-08
1457 4.2309512338079e-08
1458 4.23603410126816e-08
1459 4.23458423881584e-08
1460 4.2335461358789e-08
1461 4.23150510187043e-08
1462 4.23468584642706e-08
1463 4.23466168797404e-08
1464 4.23344914679546e-08
1465 4.23382111591764e-08
1466 4.23350883238527e-08
1467 4.23153885265037e-08
1468 4.23040944497188e-08
1469 4.23385095871254e-08
1470 4.22946371259059e-08
1471 4.23311874442334e-08
1472 4.22918091658175e-08
1473 4.22752037820828e-08
1474 4.22390868948241e-08
1475 4.22331147831301e-08
1476 4.22522887788546e-08
1477 4.22516599485334e-08
1478 4.2232517927232e-08
1479 4.22610213490771e-08
1480 4.23299333363047e-08
1481 4.21853805221417e-08
1482 4.22628723129037e-08
1483 4.22117167886427e-08
1484 4.22284394119288e-08
1485 4.21345802692485e-08
1486 4.21795185445717e-08
1487 4.21750669943322e-08
1488 4.22215578055329e-08
1489 4.21547348139484e-08
1490 4.21427017727183e-08
1491 4.21443040465874e-08
1492 4.21062225086644e-08
1493 4.20924415323043e-08
1494 4.22301233982125e-08
1495 4.21029398012251e-08
1496 4.20305283910238e-08
1497 4.22127008903317e-08
1498 4.20812611423571e-08
1499 4.20970316383773e-08
1500 4.27096225052992e-08
1501 4.26863877578398e-08
1502 4.24725747905086e-08
1503 4.25913917467824e-08
1504 4.27192290430867e-08
1505 4.26185380320021e-08
1506 4.19396002371286e-08
1507 4.25156692074324e-08
1508 4.19502157456009e-08
1509 4.21278798512503e-08
1510 4.26350830196043e-08
1511 4.26432364974971e-08
1512 4.26249791019018e-08
1513 4.25829540517952e-08
1514 4.25612718402135e-08
1515 4.19583514599253e-08
1516 4.25621280442101e-08
1517 4.25821973237817e-08
1518 4.25629025357921e-08
1519 4.235412021103e-08
1520 4.17614813841283e-08
1521 4.19192680567448e-08
1522 4.19228882719835e-08
1523 4.20221830665923e-08
1524 4.18280876601784e-08
1525 4.20655936750336e-08
1526 4.21654640092584e-08
1527 4.22480219697263e-08
1528 4.18381773670262e-08
1529 4.24773070051288e-08
1530 4.24612132121638e-08
1531 4.18665671020335e-08
1532 4.20416945701163e-08
1533 4.18173122795906e-08
1534 4.18101855359509e-08
1535 4.18298817805862e-08
1536 4.2244110431966e-08
1537 4.19361150250097e-08
1538 4.19971577514389e-08
1539 4.17243768424669e-08
1540 4.15945606846435e-08
1541 4.19870254120269e-08
1542 4.17690415588368e-08
1543 4.22305852509908e-08
1544 4.18077554797947e-08
1545 4.17232186578076e-08
1546 4.20634336251169e-08
1547 4.16566763306037e-08
1548 4.14969321127501e-08
1549 4.18286667525081e-08
1550 4.18027212845118e-08
1551 4.21315746734763e-08
1552 4.15886844962188e-08
1553 4.22412185230314e-08
1554 4.14783301039279e-08
1555 4.18044052707955e-08
1556 4.16564098770777e-08
1557 4.19864392142699e-08
1558 4.17385628281863e-08
1559 4.1779447457202e-08
1560 4.1663231087341e-08
1561 4.16995149521426e-08
1562 4.19788364069973e-08
1563 4.19684056396363e-08
1564 4.15953920196444e-08
1565 4.17709813405054e-08
1566 4.13640037777441e-08
1567 4.21059418442837e-08
1568 4.16858405571929e-08
1569 4.13925285158712e-08
1570 4.20897485753358e-08
1571 4.17680858788572e-08
1572 4.19093524328673e-08
1573 4.1468030786973e-08
1574 4.15284873156452e-08
1575 4.17621741632956e-08
1576 4.14434921935936e-08
1577 4.16755128185287e-08
1578 4.18139087798863e-08
1579 4.13700824708485e-08
1580 4.14610212828848e-08
1581 4.16379322132343e-08
1582 4.15618508498028e-08
1583 4.15748715454356e-08
1584 4.18215719832915e-08
1585 4.14262117942599e-08
1586 4.12026714968761e-08
1587 4.1478163126385e-08
1588 4.17962802146121e-08
1589 4.18307628535786e-08
1590 4.20449417504187e-08
1591 4.16822274473816e-08
1592 4.13137186683343e-08
1593 4.15443430767937e-08
1594 4.15719476620779e-08
1595 4.14682830296442e-08
1596 4.16262544433721e-08
1597 4.1684590001978e-08
1598 4.13918641584132e-08
1599 4.14657392866502e-08
1600 4.14557526084991e-08
1601 4.14153120686933e-08
1602 4.14736689435813e-08
1603 4.11809395473028e-08
1604 4.20307024739941e-08
1605 4.1736619493804e-08
1606 4.14662402192789e-08
1607 4.12821279383024e-08
1608 4.14549319316393e-08
1609 4.15450571722431e-08
1610 4.11942728817394e-08
1611 4.21823678209421e-08
1612 4.14747951538175e-08
1613 4.11726439608628e-08
1614 4.15060483760499e-08
1615 4.13685867783897e-08
1616 4.19085530722896e-08
1617 4.13566993984205e-08
1618 4.12807459326814e-08
1619 4.13708534097168e-08
1620 4.18869561258362e-08
1621 4.17770102956183e-08
1622 4.13329672710461e-08
1623 4.12030267682439e-08
1624 4.17315995093759e-08
1625 4.1330014965979e-08
1626 4.14582359553606e-08
1627 4.10883309598375e-08
1628 4.19914947258349e-08
1629 4.18644283683989e-08
1630 4.13291871836918e-08
1631 4.12112690639788e-08
1632 4.1311079002071e-08
1633 4.15039984602572e-08
1634 4.12021599061063e-08
1635 4.13159071399605e-08
1636 4.15039629331204e-08
1637 4.10982643472835e-08
1638 4.14548502192247e-08
1639 4.13740011140362e-08
1640 4.16310150797017e-08
1641 4.11660963095528e-08
1642 4.11494767149634e-08
1643 4.1677843398702e-08
1644 4.11501304142803e-08
1645 4.13296525891838e-08
1646 4.12227834090118e-08
1647 4.11296845470588e-08
1648 4.1843588149959e-08
1649 4.11707681280404e-08
1650 4.13081053807218e-08
1651 4.11500877817161e-08
1652 4.11954914625312e-08
1653 4.12833287555259e-08
1654 4.13462721837732e-08
1655 4.14717291619127e-08
1656 4.11611935646761e-08
1657 4.12774809888106e-08
1658 4.11711731373998e-08
1659 4.13899172713172e-08
1660 4.09912175314275e-08
1661 4.08316367384032e-08
1662 4.11308533898591e-08
1663 4.10663965055846e-08
1664 4.1155011842875e-08
1665 4.10497449365721e-08
1666 4.12491765189316e-08
1667 4.11027869517966e-08
1668 4.10708231868284e-08
1669 4.10025791097723e-08
1670 4.10555109908728e-08
1671 4.1096456016021e-08
1672 4.11872456140827e-08
1673 4.09931679712372e-08
1674 4.09929938882669e-08
1675 4.09248492871939e-08
1676 4.09072065110649e-08
1677 4.07149229886272e-08
1678 4.12933474081001e-08
1679 4.09385130240025e-08
1680 4.08316012112664e-08
1681 4.10440925691091e-08
1682 4.1401445827205e-08
1683 4.10154648022854e-08
1684 4.09647853416573e-08
1685 4.10584917176493e-08
1686 4.09080520569205e-08
1687 4.11838527725195e-08
1688 4.08665030704469e-08
1689 4.1104076586862e-08
1690 4.08233837845273e-08
1691 4.06485334281115e-08
1692 4.09482865393329e-08
1693 4.11805096689477e-08
1694 4.07730631479808e-08
1695 4.11071425787668e-08
1696 4.12056273546568e-08
1697 4.12546974359884e-08
1698 4.08802556250976e-08
1699 4.09675351420447e-08
1700 4.07951645797766e-08
1701 4.08177669442011e-08
1702 4.0771567455522e-08
1703 4.06216997816955e-08
1704 4.07614351161101e-08
1705 4.11183940229876e-08
1706 4.06975217970285e-08
1707 4.07261566692796e-08
1708 4.11670662003871e-08
1709 4.10939620110184e-08
1710 4.081295656988e-08
1711 4.06913791550778e-08
1712 4.07890112796849e-08
1713 4.07039379979324e-08
1714 4.07789748635423e-08
1715 4.066956549309e-08
1716 4.09137612678023e-08
1717 4.08733313861376e-08
1718 4.11652152365605e-08
1719 4.06686382348198e-08
1720 4.09256735167673e-08
1721 4.06333242608525e-08
1722 4.05405877756948e-08
1723 4.06919760109758e-08
1724 4.08661904316432e-08
1725 4.07335818408683e-08
1726 4.06571487587826e-08
1727 4.0742090590129e-08
1728 4.06978841738237e-08
1729 4.08390086192867e-08
1730 4.06915461326207e-08
1731 4.06744256054026e-08
1732 4.07057285656265e-08
1733 4.10344327406165e-08
1734 4.10169569420304e-08
1735 4.07322424678114e-08
1736 4.0711917392855e-08
1737 4.0578214566267e-08
1738 4.05811633186204e-08
1739 4.07224973741904e-08
1740 4.06875351188773e-08
1741 4.04325959380003e-08
1742 4.07088833753733e-08
1743 4.10529601424514e-08
1744 4.07107343391999e-08
1745 4.05504643197219e-08
1746 4.10046752108428e-08
1747 4.06215470150073e-08
1748 4.05727718089111e-08
1749 4.05550153459444e-08
1750 4.06069915470653e-08
1751 4.06774418593159e-08
1752 4.0464819051067e-08
1753 4.03503968016139e-08
1754 4.07469364915869e-08
1755 4.06276825515306e-08
1756 4.04840179157873e-08
1757 4.06633482441521e-08
1758 4.05201951991785e-08
1759 4.05616127352459e-08
1760 4.07874019003884e-08
1761 4.06342977044005e-08
1762 4.06759284032887e-08
1763 4.08049700695301e-08
1764 4.04234903328415e-08
1765 4.07357561016397e-08
1766 4.11465777006015e-08
1767 4.07539104685384e-08
1768 4.07016820247463e-08
1769 4.0528263411943e-08
1770 4.10954079654857e-08
1771 4.10238669701357e-08
1772 4.10769587233517e-08
1773 4.07446485439777e-08
1774 4.06125266749768e-08
1775 4.09848048832373e-08
1776 4.09624512087703e-08
1777 4.10628118174827e-08
1778 4.10828810970543e-08
1779 4.08907006033132e-08
1780 4.12051477383102e-08
1781 4.19705195042752e-08
1782 4.11238119113477e-08
1783 4.10155429619863e-08
1784 4.16148466797495e-08
1785 4.09592715300278e-08
1786 4.11413090262158e-08
1787 4.12227869617254e-08
1788 4.1019074359383e-08
1789 4.11387226506577e-08
1790 4.10862099897713e-08
1791 4.10558698149543e-08
1792 4.0679807966626e-08
1793 4.1046096299624e-08
1794 4.11074232431474e-08
1795 4.10685174756509e-08
1796 4.08988185540693e-08
1797 4.08348483915688e-08
1798 4.08503488813494e-08
1799 4.08133615792394e-08
1800 4.1004827977531e-08
1801 4.08761238190891e-08
1802 4.09429077308232e-08
1803 4.07326155027476e-08
1804 4.05309457107705e-08
1805 4.06757578730321e-08
1806 4.0557530667229e-08
1807 4.06083628945453e-08
1808 4.07718268036206e-08
1809 4.03599180742731e-08
1810 4.08388913797353e-08
1811 4.09395397582557e-08
1812 4.06728304369608e-08
1813 4.07275813074648e-08
1814 4.06913400752273e-08
1815 4.06350011417089e-08
1816 4.03683166894098e-08
1817 4.04155535704831e-08
1818 4.07157010329229e-08
1819 4.05790849811183e-08
1820 4.04732567460542e-08
1821 4.05091640232058e-08
1822 4.08012859054452e-08
1823 4.06398825703036e-08
1824 4.02483948391819e-08
1825 4.05898035182872e-08
1826 4.05806446224233e-08
1827 4.06630249472073e-08
1828 4.04848385926471e-08
1829 4.06719387058274e-08
1830 4.02766033857915e-08
1831 4.05560811600481e-08
1832 4.05770776978898e-08
1833 4.0226183273262e-08
1834 4.05592288643675e-08
1835 4.10211953294493e-08
1836 4.03512601110378e-08
1837 4.03533313431126e-08
1838 4.04005042753397e-08
1839 4.04511659723994e-08
1840 4.072931503174e-08
1841 4.05280786708317e-08
1842 4.04885014404499e-08
1843 4.04929778596852e-08
1844 4.03628952483359e-08
1845 4.05158289140672e-08
1846 4.0046700178209e-08
1847 4.05306472828215e-08
1848 4.01992039655852e-08
1849 4.0266876055739e-08
1850 4.04714164403686e-08
1851 4.01258830606821e-08
1852 4.02185520442799e-08
1853 4.03931466053109e-08
1854 4.0343593354919e-08
1855 4.01675031014292e-08
1856 4.0489297248314e-08
1857 4.01986888221018e-08
1858 4.02035773561238e-08
1859 4.03630302514557e-08
1860 4.02733171256386e-08
1861 4.02374062957733e-08
1862 4.06749727233091e-08
1863 4.02208151228933e-08
1864 4.00879436313062e-08
1865 4.06338536151907e-08
1866 4.01931252724808e-08
1867 4.02117308340166e-08
1868 4.00414918999559e-08
1869 4.04168893908263e-08
1870 4.02063555782206e-08
1871 4.02315336600623e-08
1872 4.04495388295345e-08
1873 3.99852098098563e-08
1874 4.04540720921887e-08
1875 4.01317912235299e-08
1876 4.04066220482946e-08
1877 4.01737523247903e-08
1878 4.01222663981571e-08
1879 4.05063893538227e-08
1880 4.00377686560205e-08
1881 4.0266289857982e-08
1882 4.00318320714632e-08
1883 4.03053306285983e-08
1884 4.01421189621942e-08
1885 4.00849167192519e-08
1886 4.00302582193035e-08
1887 4.06026501309498e-08
1888 4.01185076270849e-08
1889 4.00484267970569e-08
1890 4.03409430305146e-08
1891 4.02681052946718e-08
1892 4.01452666665136e-08
1893 4.00505228981274e-08
1894 4.01452417975179e-08
1895 4.02301658652959e-08
1896 4.00616890772199e-08
1897 4.04925692976121e-08
1898 3.9933304663009e-08
1899 4.00944628609068e-08
1900 4.04723436986387e-08
1901 3.98855277694565e-08
1902 4.0213198104766e-08
1903 4.00727060423378e-08
1904 4.03236413148989e-08
1905 4.01506561331644e-08
1906 3.98196533524242e-08
1907 4.01040658459806e-08
1908 4.04769373574254e-08
1909 4.00411153123059e-08
1910 4.01372979297321e-08
1911 4.05179854112703e-08
1912 4.01827335849703e-08
1913 4.03026625406255e-08
1914 4.00122424082383e-08
1915 3.99303807796514e-08
1916 3.99342070522835e-08
1917 4.00813107148679e-08
1918 4.01559674401142e-08
1919 3.99588486743596e-08
1920 4.00154291924082e-08
1921 4.02259630050139e-08
1922 3.99996977762385e-08
1923 3.99526953742679e-08
1924 4.02371043151106e-08
1925 3.98792465716724e-08
1926 3.99828117281231e-08
1927 4.03859914399618e-08
1928 3.98700663595264e-08
1929 3.98021278158467e-08
1930 4.01444353315128e-08
1931 4.00476380946202e-08
1932 3.994616903924e-08
1933 4.00468422867561e-08
1934 3.97862578438435e-08
1935 4.00271424894072e-08
1936 3.98162534054336e-08
1937 3.99825736963066e-08
1938 3.97913098026947e-08
1939 3.9933770068501e-08
1940 4.00797546262766e-08
1941 3.99419448626759e-08
1942 3.9797136253128e-08
1943 3.9945970087274e-08
1944 3.99959105834569e-08
1945 4.01196942334536e-08
1946 4.013590526597e-08
1947 4.03397955039964e-08
1948 4.00614901252538e-08
1949 4.01010993300588e-08
1950 3.98223853892432e-08
1951 3.97580386390928e-08
1952 3.99400761352808e-08
1953 4.02076665295681e-08
1954 4.01320221499191e-08
1955 3.9673189178302e-08
1956 4.00535604683228e-08
1957 3.98089241571142e-08
1958 3.98953758917742e-08
1959 3.99674178197529e-08
1960 3.99094304270875e-08
1961 4.01032131946977e-08
1962 3.98914536958728e-08
1963 4.01045419096135e-08
1964 3.99209127976974e-08
1965 3.9698175413605e-08
1966 3.99448687460335e-08
1967 3.99558004460232e-08
1968 3.97665935736313e-08
1969 3.99937931661043e-08
1970 4.00993727112109e-08
1971 3.9891805414527e-08
1972 4.00193762573053e-08
1973 3.97569976939849e-08
1974 3.97628454607002e-08
1975 4.00763013885808e-08
1976 3.98887536334769e-08
1977 4.00948678702662e-08
1978 4.00201365380326e-08
1979 3.98803763346223e-08
1980 3.98996391481887e-08
1981 3.99647674953485e-08
1982 3.9959054731753e-08
1983 3.9904392679091e-08
1984 4.00557169655258e-08
1985 3.977401874522e-08
1986 3.97094694903899e-08
1987 3.99335853273897e-08
1988 3.99046768961853e-08
1989 3.96657640067133e-08
1990 4.0245794252769e-08
1991 4.0053624417169e-08
1992 4.0043094173825e-08
1993 3.96437620509005e-08
1994 3.99276345319777e-08
1995 3.98298780623918e-08
1996 3.97958075382121e-08
1997 3.98202466556086e-08
1998 3.96511801170618e-08
1999 3.98721198280327e-08
2000 3.98683717151016e-08
2001 3.97782073946473e-08
2002 3.97536652485542e-08
2003 3.97891213310686e-08
2004 3.98990600558591e-08
2005 4.00820034940352e-08
2006 3.98674693258272e-08
2007 4.01349744549861e-08
2008 3.99454123112264e-08
2009 3.97456645373495e-08
2010 4.01792874527018e-08
2011 3.97823072262327e-08
2012 3.95940986663845e-08
2013 3.9553633257583e-08
2014 3.99542727791413e-08
2015 4.0143337542986e-08
2016 3.98063733086929e-08
2017 3.99702564379822e-08
2018 3.96818897741014e-08
2019 3.97465456103419e-08
2020 3.98737007856198e-08
2021 3.97382855510386e-08
2022 3.9633192727706e-08
2023 4.00118231880242e-08
2024 3.95936652353157e-08
2025 3.99679223050953e-08
2026 3.95913843931339e-08
2027 3.9818228714239e-08
2028 3.97253643313888e-08
2029 3.9619333591645e-08
2030 3.97268813401297e-08
2031 3.99324768807219e-08
2032 3.99247035431927e-08
2033 3.97018773412583e-08
2034 3.98646662347346e-08
2035 3.97876434021782e-08
2036 3.97421331399528e-08
2037 3.98034600834762e-08
2038 3.98587403083184e-08
2039 3.96928925283646e-08
2040 3.99745552215336e-08
2041 3.98942283652559e-08
2042 3.97113737449217e-08
2043 3.98988113659016e-08
2044 3.9522110029111e-08
2045 3.93841759205316e-08
2046 3.97973280996666e-08
2047 3.95363066729715e-08
2048 3.9348776681436e-08
2049 3.98168218396222e-08
2050 3.98532833401077e-08
2051 4.01788327053509e-08
2052 3.95227353067185e-08
2053 3.97316419764593e-08
2054 3.97238153482249e-08
2055 3.98707626914074e-08
2056 3.97367792004388e-08
2057 3.9466197421234e-08
2058 3.95035044675751e-08
2059 3.94701658024132e-08
2060 3.93230514816878e-08
2061 3.94649930512969e-08
2062 3.96367951793763e-08
2063 3.95698798172361e-08
2064 3.9819898489668e-08
2065 3.96566157689904e-08
2066 3.95834334199208e-08
2067 3.96638100141899e-08
2068 3.96848278683137e-08
2069 3.96151946802092e-08
2070 3.95224120097737e-08
2071 3.9699536102944e-08
2072 3.93025594291885e-08
2073 3.95099810646116e-08
2074 3.99049184807154e-08
2075 3.94272987591648e-08
2076 3.96633872412622e-08
2077 3.96574257877091e-08
2078 3.94559727112664e-08
2079 3.99099313597162e-08
2080 3.95016819254579e-08
2081 3.9539873597505e-08
2082 3.96984667361266e-08
2083 3.97609092317452e-08
2084 3.95474017977904e-08
2085 3.963817007957e-08
2086 3.95988628554278e-08
2087 3.97825878906133e-08
2088 3.94840107276195e-08
2089 3.95859984791969e-08
2090 3.95804100605801e-08
2091 3.96475101638316e-08
2092 3.95626500448998e-08
2093 3.94353811827841e-08
2094 3.95448793710784e-08
2095 3.96666948176971e-08
2096 3.97575377064641e-08
2097 3.95597510305379e-08
2098 3.94138801596e-08
2099 3.94645738310828e-08
2100 3.94505299539105e-08
2101 3.93856502967083e-08
2102 3.97693717957281e-08
2103 3.94395058833652e-08
2104 3.9556265818419e-08
2105 3.93253927200021e-08
2106 3.96345285480493e-08
2107 3.95862080893039e-08
2108 3.95842150169301e-08
2109 3.95361858807064e-08
2110 3.9632496395825e-08
2111 3.91347008132925e-08
2112 3.95018062704366e-08
2113 3.9614334923499e-08
2114 3.9462381806743e-08
2115 3.94529706682079e-08
2116 3.93854691083106e-08
2117 3.96197563645728e-08
2118 3.93445169777351e-08
2119 3.93273147381024e-08
2120 3.95460446611651e-08
2121 3.93579817625778e-08
2122 3.93967276579588e-08
2123 3.95047976553542e-08
2124 3.960049710372e-08
2125 3.93409642640563e-08
2126 3.93206391890999e-08
2127 3.95865598079581e-08
2128 3.95494126337326e-08
2129 3.95657586693687e-08
2130 3.93983263791142e-08
2131 3.92601187115815e-08
2132 3.93268670961788e-08
2133 3.93837567003175e-08
2134 3.92622467870751e-08
2135 3.91990369053019e-08
2136 3.9597953360726e-08
2137 3.95110681949973e-08
2138 3.94172943174453e-08
2139 3.9533325946195e-08
2140 3.913775969977e-08
2141 3.94385146762488e-08
2142 3.91164398649835e-08
2143 3.95166530609004e-08
2144 3.94306560735913e-08
2145 3.91096008911518e-08
2146 3.95860269009063e-08
2147 3.94720878205135e-08
2148 3.92847212538072e-08
2149 3.94420354155045e-08
2150 3.9303792220835e-08
2151 3.92807493199143e-08
2152 3.92449450714594e-08
2153 3.9139827379131e-08
2154 3.91380048370138e-08
2155 3.93732477732556e-08
2156 3.93714039148563e-08
2157 3.90590315646477e-08
2158 3.91641528096898e-08
2159 3.8979258931704e-08
2160 3.93952888089188e-08
2161 3.92731571707827e-08
2162 3.92310113284111e-08
2163 3.92273840077451e-08
2164 3.92603993759622e-08
2165 3.94573760331696e-08
2166 3.9079711910972e-08
2167 3.94818293614208e-08
2168 3.93101728946021e-08
2169 3.93797634501425e-08
2170 3.94842594175771e-08
2171 3.94478938403608e-08
2172 3.92819288208557e-08
2173 3.93068937398766e-08
2174 3.92687802275304e-08
2175 3.9211347058199e-08
2176 3.92424510664569e-08
2177 3.91745729189097e-08
2178 3.91694605639259e-08
2179 3.9156109465921e-08
2180 3.95371877459638e-08
2181 3.90220264989694e-08
2182 3.92761059231361e-08
2183 3.92781949187793e-08
2184 3.92473396004789e-08
2185 3.91893237861041e-08
2186 3.91608310224001e-08
2187 3.96851440598311e-08
2188 3.94796302316536e-08
2189 3.93801329323651e-08
2190 3.89531855660152e-08
2191 3.9197686874104e-08
2192 3.91571255420331e-08
2193 3.91850178971254e-08
2194 3.92826784434419e-08
2195 3.89347185603128e-08
2196 3.90870695810008e-08
2197 3.9104598670292e-08
2198 3.90643144498881e-08
2199 3.89506489284486e-08
2200 3.91335746030563e-08
2201 3.95465278302254e-08
2202 3.92514927227694e-08
2203 3.92858581221844e-08
2204 3.92445969055188e-08
2205 3.94289259020297e-08
2206 3.89944183609714e-08
2207 3.91851173731084e-08
2208 3.9131808904358e-08
2209 3.93776744544994e-08
2210 3.91907519770029e-08
2211 3.94986265916941e-08
2212 3.92388415093592e-08
2213 3.92963777073874e-08
2214 3.92135852678166e-08
2215 3.9389380646071e-08
2216 3.90307199893414e-08
2217 3.90520860094057e-08
2218 3.90588610343912e-08
2219 3.88530381201235e-08
2220 3.8853166017816e-08
2221 3.89148659962757e-08
2222 3.88678920160146e-08
2223 3.88720202693094e-08
2224 3.90847709752506e-08
2225 3.91998931092985e-08
2226 3.90810654948837e-08
2227 3.92055490294752e-08
2228 3.88452363608849e-08
2229 3.90050374221573e-08
2230 3.91289241008508e-08
2231 3.92232060164588e-08
2232 3.90473253730761e-08
2233 3.90284320417322e-08
2234 3.9363310833096e-08
2235 3.90853323040119e-08
2236 3.90849308473662e-08
2237 3.90737007194275e-08
2238 3.85897322985329e-08
2239 3.90778751580001e-08
2240 3.89294356750725e-08
2241 3.91216588013776e-08
2242 3.87685581415553e-08
2243 3.92412999872249e-08
2244 3.90043481957036e-08
2245 3.89062293493225e-08
2246 3.88500431824923e-08
2247 3.90039005537801e-08
2248 3.90197065769371e-08
2249 3.88780314608539e-08
2250 3.92589107889307e-08
2251 3.93153989364237e-08
2252 3.89912422349425e-08
2253 3.92891088552005e-08
2254 3.89088334884491e-08
2255 3.89011702850439e-08
2256 3.92864123455183e-08
2257 3.8902861376755e-08
2258 3.89200316419647e-08
2259 3.92633694445976e-08
2260 3.89854157845093e-08
2261 3.9091425207971e-08
2262 3.90466432520498e-08
2263 3.90180403542217e-08
2264 3.91641989949676e-08
2265 3.89479417606253e-08
2266 3.91370349461795e-08
2267 3.88676610896255e-08
2268 3.88137095796992e-08
2269 3.87540488588911e-08
2270 3.94843198137096e-08
2271 3.88489453939656e-08
2272 3.88137593176907e-08
2273 3.89837033765161e-08
2274 3.92231349621852e-08
2275 3.90215504353364e-08
2276 3.8893791298733e-08
2277 3.903684131501e-08
2278 3.87872418627921e-08
2279 3.87018062042443e-08
2280 3.9170753751705e-08
2281 3.85777951805721e-08
2282 3.89596941374748e-08
2283 3.89472418760306e-08
2284 3.88082490587749e-08
2285 3.86892295978214e-08
2286 3.91016143908018e-08
2287 3.86850373956804e-08
2288 3.88640088999637e-08
2289 3.86134573204799e-08
2290 3.85962550808472e-08
2291 3.83966352046627e-08
2292 3.85825096316239e-08
2293 3.86139191732582e-08
2294 3.84236571449037e-08
2295 3.85475793507339e-08
2296 3.88408025742137e-08
2297 3.88492082947778e-08
2298 3.86915068872895e-08
2299 3.88333489809156e-08
2300 3.86255187834195e-08
2301 3.8841662330924e-08
2302 3.88308478704857e-08
2303 3.90805432459729e-08
2304 3.84859468738341e-08
2305 3.88220442459897e-08
2306 3.86482774672459e-08
2307 3.87440231008895e-08
2308 3.84454033053316e-08
2309 3.86265668339547e-08
2310 3.87945249258337e-08
2311 3.91009358224892e-08
2312 3.87138214819061e-08
2313 3.8353583420303e-08
2314 3.86175216249285e-08
2315 3.91744912064951e-08
2316 3.87679932600804e-08
2317 3.86414384934142e-08
2318 3.86574932065287e-08
2319 3.8521243084233e-08
2320 3.81884035505209e-08
2321 3.83026979022816e-08
2322 3.82949991717396e-08
2323 3.85077356668262e-08
2324 3.86957736964177e-08
2325 3.84536598119212e-08
2326 3.8956141423796e-08
2327 3.89306755721464e-08
2328 3.86212946068554e-08
2329 3.89561840563601e-08
2330 3.86587331036026e-08
2331 3.8650167510923e-08
2332 3.8426861692642e-08
2333 3.85712581874031e-08
2334 3.87625895825749e-08
2335 3.89290910618456e-08
2336 3.83098530676307e-08
2337 3.85233747124403e-08
2338 3.84673697340077e-08
2339 3.87356600128896e-08
2340 3.87233498599926e-08
2341 3.83917821977775e-08
2342 3.85906027133842e-08
2343 3.8295603133065e-08
2344 3.83117928492993e-08
2345 3.8244447608804e-08
2346 3.85050427098577e-08
2347 3.82487108652185e-08
2348 3.82205591620277e-08
2349 3.81661529047506e-08
2350 3.84096736638639e-08
2351 3.85336491603994e-08
2352 3.80937237309809e-08
2353 3.84839005107551e-08
2354 3.82341518445628e-08
2355 3.85626890420099e-08
2356 3.88574363796579e-08
2357 3.83697233985458e-08
2358 3.88394028050243e-08
2359 3.83796141534276e-08
2360 3.90945018580169e-08
2361 3.83984826157757e-08
2362 3.84746527970492e-08
2363 3.86071725699821e-08
2364 3.84669505137936e-08
2365 3.87976619720121e-08
2366 3.86473715252578e-08
2367 3.87405307833433e-08
2368 3.89584329241188e-08
2369 3.84660623353739e-08
2370 3.86689293918607e-08
2371 3.83500520229063e-08
2372 3.83803886450096e-08
2373 3.87094125642307e-08
2374 3.82895883888068e-08
2375 3.84567648836764e-08
2376 3.85457923357535e-08
2377 3.82670855003653e-08
2378 3.83492988476064e-08
2379 3.81546918504228e-08
2380 3.82052931513499e-08
2381 3.82946403476581e-08
2382 3.8307415906047e-08
2383 3.90396692750983e-08
2384 3.87985039651539e-08
2385 3.81335176768971e-08
2386 3.88868386380636e-08
2387 3.83635416767447e-08
2388 3.82614615546117e-08
2389 3.81650728797922e-08
2390 3.80067284311281e-08
2391 3.81595945952995e-08
2392 3.83131109060741e-08
2393 3.82145444177695e-08
2394 3.8448042971595e-08
2395 3.84682294907179e-08
2396 3.8460242990368e-08
2397 3.83384239910356e-08
2398 3.88470553502884e-08
2399 3.82326028613988e-08
2400 3.8346847475168e-08
2401 3.82699028023126e-08
2402 3.84852612000941e-08
2403 3.83220637445447e-08
2404 3.80783227171833e-08
2405 3.81415858896617e-08
2406 3.80772569030796e-08
2407 3.8014228209704e-08
2408 3.83455578401026e-08
2409 3.82190279424321e-08
2410 3.81533489246522e-08
2411 3.82781024654832e-08
2412 3.80920361919834e-08
2413 3.8116596101645e-08
2414 3.85094729438151e-08
2415 3.8098157517652e-08
2416 3.80808344857542e-08
2417 3.83149405536187e-08
2418 3.79389781812733e-08
2419 3.83846732177062e-08
2420 3.79937468153457e-08
2421 3.81065063947972e-08
2422 3.81806657401285e-08
2423 3.84418257226571e-08
2424 3.78156279623454e-08
2425 3.80207367811636e-08
2426 3.79355427071459e-08
2427 3.82602003412558e-08
2428 3.7841580535769e-08
2429 3.79187525822999e-08
2430 3.83451421726022e-08
2431 3.79333258138104e-08
2432 3.80282898504447e-08
2433 3.84807741227178e-08
2434 3.79915299220102e-08
2435 3.81668314730632e-08
2436 3.7836493049781e-08
2437 3.79421294383064e-08
2438 3.82747060712063e-08
2439 3.83934626313476e-08
2440 3.8223962661732e-08
2441 3.82492402195567e-08
2442 3.8149501335738e-08
2443 3.84684071264019e-08
2444 3.80290856583088e-08
2445 3.80026996538163e-08
2446 3.78707376569309e-08
2447 3.80101212726913e-08
2448 3.77866236078717e-08
2449 3.79896540891878e-08
2450 3.79992819432573e-08
2451 3.80658882193075e-08
2452 3.76961715176094e-08
2453 3.82822022970686e-08
2454 3.82423337441651e-08
2455 3.77624260750054e-08
2456 3.83084568511549e-08
2457 3.8396510859684e-08
2458 3.79890607860034e-08
2459 3.78901638953266e-08
2460 3.81073910205032e-08
2461 3.7809261499433e-08
2462 3.82241474028433e-08
2463 3.79382782966786e-08
2464 3.7747387438003e-08
2465 3.81872595767163e-08
2466 3.77566493625636e-08
2467 3.77804241225022e-08
2468 3.77161448739116e-08
2469 3.80233338148628e-08
2470 3.7952066378466e-08
2471 3.80348410544684e-08
2472 3.77576938603852e-08
2473 3.76769726528892e-08
2474 3.77322812994407e-08
2475 3.80709508362997e-08
2476 3.78257851707531e-08
2477 3.76920255007462e-08
2478 3.85562621829649e-08
2479 3.84149174692539e-08
2480 3.79689879537182e-08
2481 3.7835818034182e-08
2482 3.80444298286875e-08
2483 3.79972782127425e-08
2484 3.7671842534337e-08
2485 3.77468829526606e-08
2486 3.81248241865251e-08
2487 3.78074531681705e-08
2488 3.77025770603723e-08
2489 3.77944573415334e-08
2490 3.79517786086581e-08
2491 3.77526063743971e-08
2492 3.82401061926885e-08
2493 3.8133585178457e-08
2494 3.75839768196329e-08
2495 3.7702303501419e-08
2496 3.76137947455391e-08
2497 3.82518443586832e-08
2498 3.76618132236217e-08
2499 3.77169087073526e-08
2500 3.7477615677517e-08
2501 3.76008948421713e-08
2502 3.78153153235417e-08
2503 3.76231348298006e-08
2504 3.80094178353829e-08
2505 3.75441402411525e-08
2506 3.76116737754728e-08
2507 3.78916169552213e-08
2508 3.83740825782297e-08
2509 3.84407456976987e-08
2510 3.79254068150203e-08
2511 3.76884514707854e-08
2512 3.75552389186851e-08
2513 3.83088973876511e-08
2514 3.78341624696077e-08
2515 3.74666342395358e-08
2516 3.80809268563098e-08
2517 3.75850639500186e-08
2518 3.77221596181698e-08
2519 3.81008931071847e-08
2520 3.81572498042715e-08
2521 3.76503663801486e-08
2522 3.78550844004621e-08
2523 3.78986335647369e-08
2524 3.78552194035819e-08
2525 3.84165659284008e-08
2526 3.78622431185249e-08
2527 3.78138302892239e-08
2528 3.81713363140079e-08
2529 3.80205626981933e-08
2530 3.75993352008663e-08
2531 3.78155746716402e-08
2532 3.78198734551916e-08
2533 3.75808681951639e-08
2534 3.78196602923708e-08
2535 3.77395821260507e-08
2536 3.7520972995253e-08
2537 3.75317235068451e-08
2538 3.82234794926717e-08
2539 3.75853872469634e-08
2540 3.82396123654871e-08
2541 3.76233870724718e-08
2542 3.81591185316665e-08
2543 3.82275011645561e-08
2544 3.78979514437106e-08
2545 3.78169566772613e-08
2546 3.77302455945028e-08
2547 3.7584584333672e-08
2548 3.78974398529408e-08
2549 3.84118195029259e-08
2550 3.76775446397914e-08
2551 3.78036837389573e-08
2552 3.76218061148847e-08
2553 3.75138036190492e-08
2554 3.80460143389882e-08
2555 3.80712421588214e-08
2556 3.77913558224918e-08
2557 3.75611541869603e-08
2558 3.82446536661973e-08
2559 3.74763722277294e-08
2560 3.74771147448882e-08
2561 3.78529882993917e-08
2562 3.83176121943052e-08
2563 3.75697020160715e-08
2564 3.80699241020466e-08
2565 3.75796744833679e-08
2566 3.7796972662818e-08
2567 3.80003477573609e-08
2568 3.7808675301676e-08
2569 3.84017688759286e-08
2570 3.82232450135689e-08
2571 3.82441918134191e-08
2572 3.77358979619657e-08
2573 3.75444493272425e-08
2574 3.74323505525354e-08
2575 3.78678670642785e-08
2576 3.75290944987228e-08
2577 3.83077249921371e-08
2578 3.83153953009696e-08
2579 3.78597491135224e-08
2580 3.78570277348445e-08
2581 3.78473572482108e-08
2582 3.75539279673376e-08
2583 3.74356936561071e-08
2584 3.79160169927673e-08
2585 3.76858473316588e-08
2586 3.758624345096e-08
2587 3.76785109779121e-08
2588 3.80549742828862e-08
2589 3.76838151794345e-08
2590 3.74671813574423e-08
2591 3.74792499258092e-08
2592 3.7950684372845e-08
2593 3.73699009514894e-08
2594 3.7512290163022e-08
2595 3.77469291379384e-08
2596 3.77860516209694e-08
2597 3.81390989900865e-08
2598 3.7666289642857e-08
2599 3.76264743806587e-08
2600 3.77166955445318e-08
2601 3.80041065284331e-08
2602 3.72864654707428e-08
2603 3.81700182572331e-08
2604 3.7762607263403e-08
2605 3.77003424034683e-08
2606 3.76606905660992e-08
2607 3.79727254085083e-08
2608 3.80040816594374e-08
2609 3.82950524624448e-08
2610 3.81299329887952e-08
2611 3.81564753126895e-08
2612 3.72538160320346e-08
2613 3.75358126802894e-08
2614 3.76272843993775e-08
2615 3.80719775705529e-08
2616 3.74592268315155e-08
2617 3.74289932381089e-08
2618 3.77631899084463e-08
2619 3.72558588423999e-08
2620 3.71359973883045e-08
2621 3.76173332483631e-08
2622 3.72617954269572e-08
2623 3.73409960729987e-08
2624 3.7932622376502e-08
2625 3.72144164373367e-08
2626 3.73529438491005e-08
2627 3.79253677351699e-08
2628 3.73150506050024e-08
2629 3.73692117250357e-08
2630 3.74388591239949e-08
2631 3.75968269850091e-08
2632 3.74152868687361e-08
2633 3.70240194058624e-08
2634 3.82631526463229e-08
2635 3.77279185670432e-08
2636 3.74084834220412e-08
2637 3.81115548009348e-08
2638 3.80131091048952e-08
2639 3.72975925699848e-08
2640 3.78108104825969e-08
2641 3.79817457485387e-08
2642 3.80209463912706e-08
2643 3.78489062313747e-08
2644 3.77132707285455e-08
2645 3.79481939205562e-08
2646 3.76397970569542e-08
2647 3.8111629407922e-08
2648 3.75880588876498e-08
2649 3.82046927427382e-08
2650 3.76853570571711e-08
2651 3.77608806445551e-08
2652 3.82805929177721e-08
2653 3.78290145874871e-08
2654 3.79043001430546e-08
2655 3.77715672073009e-08
2656 3.72306097062847e-08
2657 3.72621364874703e-08
2658 3.77302811216396e-08
2659 3.82133400478324e-08
2660 3.81703131324684e-08
2661 3.7791298979073e-08
2662 3.77275028995427e-08
2663 3.83013905036478e-08
2664 3.77669344686637e-08
2665 3.834587403162e-08
2666 3.78723044036633e-08
2667 3.81776352753604e-08
2668 3.79094196034657e-08
2669 3.77671938167623e-08
2670 3.73976085654704e-08
2671 3.74836801597667e-08
2672 3.73466981784532e-08
2673 3.76983599892355e-08
2674 3.81300253593508e-08
2675 3.83389355818053e-08
2676 3.78421063373935e-08
2677 3.76610707064629e-08
2678 3.75909898764348e-08
2679 3.82359317541159e-08
2680 3.73990403090829e-08
2681 3.70220121226339e-08
2682 3.79641136305509e-08
2683 3.78203672823929e-08
2684 3.8157587312071e-08
2685 3.77619535640861e-08
2686 3.79064175604071e-08
2687 3.72699879847005e-08
2688 3.72363011535981e-08
2689 3.73918318530286e-08
2690 3.80308335934387e-08
2691 3.8254711398622e-08
2692 3.79436642106157e-08
2693 3.7355924575877e-08
2694 3.80473892391819e-08
2695 3.74803725833317e-08
2696 3.69166208713523e-08
2697 3.79994062882361e-08
2698 3.7369296990164e-08
2699 3.75511497452408e-08
2700 3.76909348176468e-08
2701 3.79849218745676e-08
2702 3.73952495635876e-08
2703 3.78957025759519e-08
2704 3.78847353488254e-08
2705 3.79159637020621e-08
2706 3.77417670449631e-08
2707 3.76433710869151e-08
2708 3.72954609417775e-08
2709 3.75760684789839e-08
2710 3.79324838206685e-08
2711 3.81225397916296e-08
2712 3.82186762237779e-08
2713 3.80509206365787e-08
2714 3.80104197006403e-08
2715 3.72213335708693e-08
2716 3.73055684121937e-08
2717 3.78505760068037e-08
2718 3.77464317580234e-08
2719 3.82914464580608e-08
2720 3.77981272947636e-08
2721 3.74296469374258e-08
2722 3.80436340208234e-08
2723 3.75356847825969e-08
2724 3.74995181573468e-08
2725 3.77462185952027e-08
2726 3.75272861674603e-08
2727 3.78440354609211e-08
2728 3.8232656152104e-08
2729 3.73762460981197e-08
2730 3.78171449710862e-08
2731 3.73477639925568e-08
2732 3.77233462245385e-08
2733 3.74494675270398e-08
2734 3.79012057294403e-08
2735 3.74194968344455e-08
2736 3.74967967786688e-08
2737 3.76866289286681e-08
2738 3.76159832171652e-08
2739 3.76387383482779e-08
2740 3.79195768118734e-08
2741 3.76117554878874e-08
2742 3.7299404453961e-08
2743 3.77480269264652e-08
2744 3.83494480615809e-08
2745 3.85217369114343e-08
2746 3.82614331329023e-08
2747 3.82947966670599e-08
2748 3.771273426878e-08
2749 3.77006372787037e-08
2750 3.7737134306326e-08
2751 3.8281932290829e-08
2752 3.79381397408451e-08
2753 3.73301780598467e-08
2754 3.74681299319946e-08
2755 3.74971058647589e-08
2756 3.79544466966308e-08
2757 3.75690092369041e-08
2758 3.82483236194275e-08
2759 3.81526703563395e-08
2760 3.75317235068451e-08
2761 3.8121783063616e-08
2762 3.81094373835822e-08
2763 3.81818203720741e-08
2764 3.75276272279734e-08
2765 3.74304924832813e-08
2766 3.80736828731187e-08
2767 3.8076013453292e-08
2768 3.81336384691622e-08
2769 3.8115146594464e-08
2770 3.79032911723698e-08
2771 3.81305618191163e-08
2772 3.79966493824213e-08
2773 3.84452754076392e-08
2774 3.76953437353222e-08
2775 3.81885634226364e-08
2776 3.81577187624771e-08
2777 3.82106151164407e-08
2778 3.83483609311952e-08
2779 3.82107820939837e-08
2780 3.81453908460117e-08
2781 3.82238418694669e-08
2782 3.82340097360157e-08
2783 3.81026090678915e-08
2784 3.80490483564699e-08
2785 3.7455471613157e-08
2786 3.81667213389392e-08
2787 3.82335656468058e-08
2788 3.8353871190111e-08
2789 3.82024616385479e-08
2790 3.82198379611509e-08
2791 3.82909703944279e-08
2792 3.82781877306115e-08
2793 3.83518816704509e-08
2794 3.75951678677211e-08
2795 3.79062079503001e-08
2796 3.81633249446622e-08
2797 3.8151448222834e-08
2798 3.82475384697045e-08
2799 3.80205840144754e-08
2800 3.82038614077373e-08
2801 3.80463802684972e-08
2802 3.82885723126947e-08
2803 3.7493045113024e-08
2804 3.81514162484109e-08
2805 3.81946918537324e-08
2806 3.81794365011956e-08
2807 3.81111071590112e-08
2808 3.82519651509483e-08
2809 3.75262345642113e-08
2810 3.74576316630737e-08
2811 3.74004152092766e-08
2812 3.75407651631576e-08
2813 3.73741997350407e-08
2814 3.80524980414521e-08
2815 3.82270464172052e-08
2816 3.75278190745121e-08
2817 3.74716506712502e-08
2818 3.81903255686211e-08
2819 3.82519260710978e-08
2820 3.8136406033118e-08
2821 3.77020796804572e-08
2822 3.8244657218911e-08
2823 3.76220121722781e-08
2824 3.83557789973565e-08
2825 3.83165357220605e-08
2826 3.82096487783201e-08
2827 3.81894764700519e-08
2828 3.8174587047024e-08
2829 3.81734643895015e-08
2830 3.79748144041514e-08
2831 3.81379621217093e-08
2832 3.74719846263361e-08
2833 3.74377258083314e-08
2834 3.73185713442581e-08
2835 3.73524322583307e-08
2836 3.79932707517128e-08
2837 3.79460836086309e-08
2838 3.79000582029221e-08
2839 3.79499844882503e-08
2840 3.74826996107913e-08
2841 3.81563545204244e-08
2842 3.78600013561936e-08
2843 3.80103841735036e-08
2844 3.73613069371004e-08
2845 3.7934182017807e-08
2846 3.7922202267282e-08
2847 3.81713505248626e-08
2848 3.81591149789529e-08
2849 3.75577577926833e-08
2850 3.80443374581318e-08
2851 3.80978235625662e-08
2852 3.80366103058805e-08
2853 3.81944040839244e-08
2854 3.76389621692397e-08
2855 3.8091819476449e-08
2856 3.80665383659107e-08
2857 3.81597473619877e-08
2858 3.75163331511885e-08
2859 3.80452611636883e-08
2860 3.74117838930488e-08
2861 3.82506648577419e-08
2862 3.7776864303396e-08
2863 3.77803104356644e-08
2864 3.80926259424541e-08
2865 3.78372462250809e-08
2866 3.7930188767632e-08
2867 3.80840177172104e-08
2868 3.79711018183571e-08
2869 3.77533098117055e-08
2870 3.72628825573429e-08
2871 3.77120841221767e-08
2872 3.77657087824446e-08
2873 3.73608415316085e-08
2874 3.78297180247955e-08
2875 3.7726493928858e-08
2876 3.74568003280729e-08
2877 3.75497748450471e-08
2878 3.72496451461757e-08
2879 3.81329527954222e-08
2880 3.77685083208235e-08
2881 3.77438205134695e-08
2882 3.80837406055434e-08
2883 3.76979620853035e-08
2884 3.78024900271612e-08
2885 3.78152087421313e-08
2886 3.77374895776939e-08
2887 3.76796833734261e-08
2888 3.77976263621349e-08
2889 3.79887197254902e-08
2890 3.70865578247503e-08
2891 3.78151590041398e-08
2892 3.77829003639363e-08
2893 3.7740065295111e-08
2894 3.76604560869964e-08
2895 3.79110502990443e-08
2896 3.74827280325007e-08
2897 3.74446216255819e-08
2898 3.73529616126689e-08
2899 3.80458047288812e-08
2900 3.75166386845649e-08
2901 3.73777346851512e-08
2902 3.75104249883407e-08
2903 3.76060711460013e-08
2904 3.76718105599139e-08
2905 3.79253926041656e-08
2906 3.7887062376285e-08
2907 3.78545266244146e-08
2908 3.71092312434484e-08
2909 3.7960241172641e-08
2910 3.76585624906056e-08
2911 3.72226267586484e-08
2912 3.69391628396443e-08
2913 3.70889488010562e-08
2914 3.71442823166035e-08
2915 3.715799223869e-08
2916 3.71002428778411e-08
2917 3.69345798389986e-08
2918 3.78232982711779e-08
2919 3.77308566612555e-08
2920 3.79741429412661e-08
2921 3.74806283787166e-08
2922 3.72204347343086e-08
2923 3.74904445266111e-08
2924 3.75559849885576e-08
2925 3.78669895439998e-08
2926 3.77998503608978e-08
2927 3.76439999172362e-08
2928 3.77649804761404e-08
2929 3.78738853612504e-08
2930 3.71926667241951e-08
2931 3.71047370606448e-08
2932 3.76011612956972e-08
2933 3.7725531143451e-08
2934 3.78638880249582e-08
2935 3.72159973949238e-08
2936 3.77962194875181e-08
2937 3.77386015770753e-08
2938 3.77366617954067e-08
2939 3.72857407171523e-08
2940 3.71569726098642e-08
2941 3.77441331522732e-08
2942 3.71316524194754e-08
2943 3.8009449809806e-08
2944 3.74152158144625e-08
2945 3.74413069437196e-08
2946 3.78301585612917e-08
2947 3.72434101336694e-08
2948 3.73221418215053e-08
2949 3.72914570334615e-08
2950 3.73524429164718e-08
2951 3.73031703304605e-08
2952 3.71602162374529e-08
2953 3.73673074705039e-08
2954 3.80602536154129e-08
2955 3.73699684530493e-08
2956 3.75856501477756e-08
2957 3.75004951536084e-08
2958 3.76908104726681e-08
2959 3.72715192042961e-08
2960 3.74460746854766e-08
2961 3.73043285151198e-08
2962 3.73802322428674e-08
2963 3.75377986472358e-08
2964 3.78650675258996e-08
2965 3.74316044826628e-08
2966 3.80225557705671e-08
2967 3.78489559693662e-08
2968 3.7921317641576e-08
2969 3.78484905638743e-08
2970 3.79536331251984e-08
2971 3.80134785871178e-08
2972 3.80807776423353e-08
2973 3.8191355855588e-08
2974 3.82468279269688e-08
2975 3.81706151131311e-08
2976 3.8502882659941e-08
2977 3.85665970270566e-08
2978 3.80439892921913e-08
2979 3.79668954053614e-08
2980 3.78951874324684e-08
2981 3.7534444885523e-08
2982 3.767156542267e-08
2983 3.79494196067753e-08
2984 3.7864410273869e-08
2985 3.78461777472694e-08
2986 3.81963047857425e-08
2987 3.81073697042211e-08
2988 3.79708424702585e-08
2989 3.79937254990637e-08
2990 3.80352496165415e-08
2991 3.80989604309434e-08
2992 3.80287765722187e-08
2993 3.85263554392168e-08
2994 3.78920361754354e-08
2995 3.7909792638402e-08
2996 3.85452949558385e-08
2997 3.79048401555337e-08
2998 3.80059965721102e-08
2999 3.79315423515436e-08
3000 2.97356432810147e-08
3001 2.97624804801444e-08
3002 2.98114883889866e-08
3003 2.98193256753621e-08
3004 2.98175315549543e-08
3005 2.98130906628558e-08
3006 2.98093283390699e-08
3007 2.98057472036817e-08
3008 2.98041165081031e-08
3009 2.9801352496861e-08
3010 2.98000415455135e-08
3011 2.97978530738874e-08
3012 2.97965883078177e-08
3013 2.97942470695034e-08
3014 2.97927229553352e-08
3015 2.9790623301551e-08
3016 2.97893318901288e-08
3017 2.97869551246777e-08
3018 2.97858626652214e-08
3019 2.97834965579113e-08
3020 2.97819067185401e-08
3021 2.97794713333133e-08
3022 2.97780839986217e-08
3023 2.97757818401578e-08
3024 2.97748439237466e-08
3025 2.97725435416396e-08
3026 2.97711615360186e-08
3027 2.97687794414969e-08
3028 2.97671647331299e-08
3029 2.97653119929464e-08
3030 2.97630453616193e-08
3031 2.97615123656669e-08
3032 2.97596187692761e-08
3033 2.97578051089431e-08
3034 2.97555846628939e-08
3035 2.97535649451675e-08
3036 2.97516216107852e-08
3037 2.97499003210078e-08
3038 2.97483762068396e-08
3039 2.97458804254802e-08
3040 2.97441697938439e-08
3041 2.97422904083078e-08
3042 2.97399580517776e-08
3043 2.97387021674922e-08
3044 2.97360891465814e-08
3045 2.97343500932357e-08
3046 2.97325293274753e-08
3047 2.97298914375688e-08
3048 2.97282642947039e-08
3049 2.97262978676827e-08
3050 2.97240365654261e-08
3051 2.97221518508195e-08
3052 2.97204465482537e-08
3053 2.97180289265953e-08
3054 2.97164177709419e-08
3055 2.9714691152094e-08
3056 2.97120390513328e-08
3057 2.97102147328587e-08
3058 2.97081417244272e-08
3059 2.97057081155572e-08
3060 2.97098665669182e-08
3061 2.97077686894909e-08
3062 2.97059674636557e-08
3063 2.9703276283044e-08
3064 2.97013791339396e-08
3065 2.96995761317476e-08
3066 2.96968760693517e-08
3067 2.96952507028436e-08
3068 2.96929272280977e-08
3069 2.9690877312305e-08
3070 2.96884827832855e-08
3071 2.96869124838395e-08
3072 2.96845321656747e-08
3073 2.96826438983544e-08
3074 2.96804074650936e-08
3075 2.96778281949628e-08
3076 2.96758972950784e-08
3077 2.96726803128422e-08
3078 2.96723197124038e-08
3079 2.96698310364718e-08
3080 2.96678184241728e-08
3081 2.96673690058924e-08
3082 2.96648039466163e-08
3083 2.96613151817837e-08
3084 2.96608266836529e-08
3085 2.96589615089715e-08
3086 2.96563609225586e-08
3087 2.96542168598535e-08
3088 2.96506854624567e-08
3089 2.96507263186641e-08
3090 2.96480688888323e-08
3091 2.96456104109666e-08
3092 2.96431430513167e-08
3093 2.96409368161221e-08
3094 2.96390609832997e-08
3095 2.96371212016311e-08
3096 2.96348510175903e-08
3097 2.9632454712214e-08
3098 2.96305202596159e-08
3099 2.96279321077009e-08
3100 2.96261344345794e-08
3101 2.96233935159762e-08
3102 2.96213080730467e-08
3103 2.96192546045404e-08
3104 2.96170128422091e-08
3105 2.96149220702091e-08
3106 2.96125932663927e-08
3107 2.96107796060596e-08
3108 2.9609127594199e-08
3109 2.96065074678609e-08
3110 2.96041431369076e-08
3111 2.96023277002178e-08
3112 2.96002209410062e-08
3113 2.95981585907157e-08
3114 2.95958972884591e-08
3115 2.95935151939375e-08
3116 2.95912165881873e-08
3117 2.95889641677149e-08
3118 2.95864008847957e-08
3119 2.95841697806054e-08
3120 2.95821127593854e-08
3121 2.95796738214449e-08
3122 2.95780484549368e-08
3123 2.95753856960346e-08
3124 2.95733073585325e-08
3125 2.95708719733057e-08
3126 2.9568589354767e-08
3127 2.95665145699786e-08
3128 2.95641022773907e-08
3129 2.95618729495573e-08
3130 2.95595512511682e-08
3131 2.95566557895199e-08
3132 2.95548172601912e-08
3133 2.95524724691631e-08
3134 2.95502609048981e-08
3135 2.9547784663464e-08
3136 2.95454576360044e-08
3137 2.95432887043034e-08
3138 2.95418800533298e-08
3139 2.95391551219382e-08
3140 2.95365865099484e-08
3141 2.95338100642084e-08
3142 2.9531845413544e-08
3143 2.95298381303155e-08
3144 2.95274258377276e-08
3145 2.95249904525008e-08
3146 2.95230542235458e-08
3147 2.95201001421219e-08
3148 2.95188122834134e-08
3149 2.95165740737957e-08
3150 2.95138686823293e-08
3151 2.951202482393e-08
3152 2.95099358282869e-08
3153 2.95070332612113e-08
3154 2.95047808407389e-08
3155 2.9502999154829e-08
3156 2.95004234374119e-08
3157 2.94984072723992e-08
3158 2.94957072100033e-08
3159 2.9493099518163e-08
3160 2.94912698706185e-08
3161 2.94887065876992e-08
3162 2.94863458094596e-08
3163 2.94840596382073e-08
3164 2.94818693902243e-08
3165 2.94787021459797e-08
3166 2.94767286135311e-08
3167 2.9474739093871e-08
3168 2.94718844884301e-08
3169 2.94694988411948e-08
3170 2.94675643885967e-08
3171 2.94648661025576e-08
3172 2.94627042762841e-08
3173 2.94603896833223e-08
3174 2.94575155379562e-08
3175 2.94551973922808e-08
3176 2.94526305566478e-08
3177 2.94511011134091e-08
3178 2.94478912366003e-08
3179 2.94454896021534e-08
3180 2.94431661274075e-08
3181 2.94406490297661e-08
3182 2.94384054910779e-08
3183 2.9435074822004e-08
3184 2.9433394388434e-08
3185 2.94306961023949e-08
3186 2.94281043977662e-08
3187 2.94260811273261e-08
3188 2.94230932951223e-08
3189 2.94206827788912e-08
3190 2.94177731063883e-08
3191 2.94175137582897e-08
3192 2.9414900737379e-08
3193 2.94114155252601e-08
3194 2.94092714625549e-08
3195 2.94074258277988e-08
3196 2.94050366278498e-08
3197 2.94018107638294e-08
3198 2.93996649247674e-08
3199 2.93974959930665e-08
3200 2.93948971830105e-08
3201 2.93921207372705e-08
3202 2.93896533776206e-08
3203 2.93870137113572e-08
3204 2.93852249200199e-08
3205 2.93821145191941e-08
3206 2.93798034789461e-08
3207 2.93773769755035e-08
3208 2.93744442103616e-08
3209 2.93714261800915e-08
3210 2.93694668584976e-08
3211 2.93672250961663e-08
3212 2.93644308868579e-08
3213 2.93615691759896e-08
3214 2.93588815480916e-08
3215 2.93568707121494e-08
3216 2.93540178830654e-08
3217 2.93517814498045e-08
3218 2.93491062564044e-08
3219 2.93456832167749e-08
3220 2.9343469876153e-08
3221 2.93412441010332e-08
3222 2.93381905436263e-08
3223 2.93358795033782e-08
3224 2.93328152878303e-08
3225 2.93307760301786e-08
3226 2.93276798402076e-08
3227 2.93254274197352e-08
3228 2.93231430248397e-08
3229 2.93194339917591e-08
3230 2.93175901333598e-08
3231 2.93148527674703e-08
3232 2.93119803984609e-08
3233 2.93093833647617e-08
3234 2.93067294876437e-08
3235 2.93035835596811e-08
3236 2.93020541164424e-08
3237 2.92993185269097e-08
3238 2.92969648540975e-08
3239 2.92942203827806e-08
3240 2.92914084099039e-08
3241 2.9289017433598e-08
3242 2.92864594797493e-08
3243 2.92831430215301e-08
3244 2.92806152657477e-08
3245 2.92785635735981e-08
3246 2.9274993096351e-08
3247 2.92730923945328e-08
3248 2.92700601534079e-08
3249 2.92673529855847e-08
3250 2.926456232899e-08
3251 2.92614288355253e-08
3252 2.92591781914098e-08
3253 2.92560269343767e-08
3254 2.92539930057956e-08
3255 2.92513924193827e-08
3256 2.92483992581083e-08
3257 2.92454167549749e-08
3258 2.92430950565858e-08
3259 2.92399811030464e-08
3260 2.92372508425842e-08
3261 2.92349291441951e-08
3262 2.92318524941493e-08
3263 2.92288575565181e-08
3264 2.92268467205758e-08
3265 2.92231039367152e-08
3266 2.92212831709548e-08
3267 2.92178103933338e-08
3268 2.92153252701155e-08
3269 2.92122486200697e-08
3270 2.92100370558046e-08
3271 2.92064772366984e-08
3272 2.920405961504e-08
3273 2.92008586200154e-08
3274 2.91984711964233e-08
3275 2.91955490894225e-08
3276 2.91934778573477e-08
3277 2.91901400828465e-08
3278 2.91872730429077e-08
3279 2.91834734156282e-08
3280 2.91816366626563e-08
3281 2.91778405880905e-08
3282 2.9175357241229e-08
3283 2.91723161183199e-08
3284 2.9170026394354e-08
3285 2.91665163132393e-08
3286 2.91643775796047e-08
3287 2.91612423097831e-08
3288 2.915911956336e-08
3289 2.91559718590406e-08
3290 2.91543500452462e-08
3291 2.91503496896439e-08
3292 2.91476691671733e-08
3293 2.91442461275437e-08
3294 2.91417201481181e-08
3295 2.91383859263306e-08
3296 2.91358439596934e-08
3297 2.91322557188778e-08
3298 2.91302448829356e-08
3299 2.91265074281455e-08
3300 2.91247808092976e-08
3301 2.9120849731612e-08
3302 2.91178405831261e-08
3303 2.91149007125568e-08
3304 2.91122610462935e-08
3305 2.91088468884482e-08
3306 2.91062978163836e-08
3307 2.91031163612843e-08
3308 2.91009527586539e-08
3309 2.90970714189598e-08
3310 2.90942310243736e-08
3311 2.90909483169344e-08
3312 2.90882766762479e-08
3313 2.90852728568325e-08
3314 2.90822548265623e-08
3315 2.90789827772642e-08
3316 2.90758634946542e-08
3317 2.90729840202175e-08
3318 2.90708701555786e-08
3319 2.90667152569313e-08
3320 2.90637256483706e-08
3321 2.90605655095533e-08
3322 2.90581532169654e-08
3323 2.90544441838847e-08
3324 2.90525008495024e-08
3325 2.90484258869128e-08
3326 2.90456902973801e-08
3327 2.90420629767141e-08
3328 2.90398549651627e-08
3329 2.90360961940905e-08
3330 2.90333108665664e-08
3331 2.90297634819581e-08
3332 2.90270598668485e-08
3333 2.90238375555418e-08
3334 2.9020915448541e-08
3335 2.90173467476507e-08
3336 2.90145933945496e-08
3337 2.90107102784987e-08
3338 2.90082109444256e-08
3339 2.9004711521452e-08
3340 2.90014163795149e-08
3341 2.89981105794368e-08
3342 2.89954353860367e-08
3343 2.89919768192703e-08
3344 2.89888433258056e-08
3345 2.89856387780674e-08
3346 2.89827983834812e-08
3347 2.89786719065432e-08
3348 2.89759594096495e-08
3349 2.89726802549239e-08
3350 2.89698007804873e-08
3351 2.89661254981866e-08
3352 2.89629955574355e-08
3353 2.89595760705197e-08
3354 2.89569008771196e-08
3355 2.89530728281306e-08
3356 2.89505379669208e-08
3357 2.89465624803142e-08
3358 2.89440151846065e-08
3359 2.89396666630637e-08
3360 2.89369772588088e-08
3361 2.89336146153119e-08
3362 2.89305130962703e-08
3363 2.89266086639373e-08
3364 2.89234858286136e-08
3365 2.89201462777555e-08
3366 2.89169737044404e-08
3367 2.89137407349926e-08
3368 2.89098167627344e-08
3369 2.89072232817489e-08
3370 2.89034183253989e-08
3371 2.88996631070404e-08
3372 2.88971229167601e-08
3373 2.88935719794381e-08
3374 2.88901684797338e-08
3375 2.88871486731068e-08
3376 2.88834751671629e-08
3377 2.88802084469353e-08
3378 2.88769772538444e-08
3379 2.88735115816507e-08
3380 2.887052197309e-08
3381 2.88667934000841e-08
3382 2.88630115363731e-08
3383 2.88602084452805e-08
3384 2.885639815986e-08
3385 2.88529111713842e-08
3386 2.88498309686247e-08
3387 2.88461787789629e-08
3388 2.88425248129442e-08
3389 2.88392776326418e-08
3390 2.88360695321899e-08
3391 2.88323231956156e-08
3392 2.8828690545879e-08
3393 2.882561567219e-08
3394 2.88219421662461e-08
3395 2.88184445196293e-08
3396 2.88152968153099e-08
3397 2.88116037694408e-08
3398 2.88086763333695e-08
3399 2.88046901886219e-08
3400 2.88011392512999e-08
3401 2.87979169399932e-08
3402 2.8794822526379e-08
3403 2.87904846629772e-08
3404 2.87870740578455e-08
3405 2.8783318839487e-08
3406 2.87799277742806e-08
3407 2.8776472760228e-08
3408 2.87731474202246e-08
3409 2.87695129941312e-08
3410 2.87657595521296e-08
3411 2.8762181969455e-08
3412 2.87583521441093e-08
3413 2.875544602432e-08
3414 2.87516588315384e-08
3415 2.87477881499854e-08
3416 2.87444752444799e-08
3417 2.87408692400959e-08
3418 2.87366148654655e-08
3419 2.87331083370646e-08
3420 2.87295947032362e-08
3421 2.87261041620468e-08
3422 2.87225851991479e-08
3423 2.87187038594539e-08
3424 2.87148829158923e-08
3425 2.87113390839977e-08
3426 2.87076549199128e-08
3427 2.87590715686292e-08
3428 2.87961068323739e-08
3429 2.88415638038941e-08
3430 2.8861009582215e-08
3431 2.88713604135182e-08
3432 2.88753518873364e-08
3433 2.8875543733875e-08
3434 2.88723516206346e-08
3435 2.88691648364647e-08
3436 2.88642034718123e-08
3437 2.88599313336135e-08
3438 2.88540764614709e-08
3439 2.8849330035996e-08
3440 2.88631110123561e-08
3441 2.88686319294129e-08
3442 2.88692696415183e-08
3443 2.8867823687051e-08
3444 2.8866816492723e-08
3445 2.8860611678283e-08
3446 2.88558155148166e-08
3447 2.88511685653248e-08
3448 2.88466335263138e-08
3449 2.88398478431873e-08
3450 2.88352133281933e-08
3451 2.88304171647269e-08
3452 2.88257968605876e-08
3453 2.88203523268749e-08
3454 2.88153678695835e-08
3455 2.88095822753576e-08
3456 2.88041110962922e-08
3457 2.87992882874732e-08
3458 2.87949575294988e-08
3459 2.87888095584776e-08
3460 2.87835479895193e-08
3461 2.87783894492577e-08
3462 2.87744068572238e-08
3463 2.87680119726019e-08
3464 2.87628729722655e-08
3465 2.87593593384372e-08
3466 2.87525399045307e-08
3467 2.87476442650814e-08
3468 2.87440933277594e-08
3469 2.87381940466958e-08
3470 2.87326820114231e-08
3471 2.8727860978961e-08
3472 2.87229280360179e-08
3473 2.87183716807249e-08
3474 2.87141386223766e-08
3475 2.87076513671991e-08
3476 2.87026811207625e-08
3477 2.86978867336529e-08
3478 2.86922041681237e-08
3479 2.86872285926165e-08
3480 2.86832140261595e-08
3481 2.86770731605657e-08
3482 2.86718826458809e-08
3483 2.86671237859082e-08
3484 2.86618551115225e-08
3485 2.86585333242328e-08
3486 2.86518471170893e-08
3487 2.86471415478218e-08
3488 2.86422885409365e-08
3489 2.86367036750335e-08
3490 2.86319465914175e-08
3491 2.86282748618305e-08
3492 2.86222459067176e-08
3493 2.86170678265307e-08
3494 2.8611932378908e-08
3495 2.86061112575453e-08
3496 2.85999401938852e-08
3497 2.85955650269898e-08
3498 2.85884453887775e-08
3499 2.8582455513515e-08
3500 2.85764283347589e-08
3501 2.85709251812705e-08
3502 2.85652461684549e-08
3503 2.85592545168356e-08
3504 2.85543997335935e-08
3505 2.85473991112895e-08
3506 2.85412138367747e-08
3507 2.85353713991299e-08
3508 2.85297296898079e-08
3509 2.85236385622056e-08
3510 2.85177677028514e-08
3511 2.85119821086255e-08
3512 2.85069710059815e-08
3513 2.84999881472459e-08
3514 2.84944388084796e-08
3515 2.84880936618492e-08
3516 2.84824128726768e-08
3517 2.84758954194331e-08
3518 2.846996771666e-08
3519 2.84652710291766e-08
3520 2.84580341514129e-08
3521 2.84525274452108e-08
3522 2.84464451993927e-08
3523 2.84418923968133e-08
3524 2.84343606438142e-08
3525 2.84287384744175e-08
3526 2.84223542479367e-08
3527 2.84165686537108e-08
3528 2.84102910086403e-08
3529 2.84042513953864e-08
3530 2.83991994365351e-08
3531 2.83921934851605e-08
3532 2.83860810412762e-08
3533 2.83802634726271e-08
3534 2.83743926132729e-08
3535 2.83678733836723e-08
3536 2.83621517382926e-08
3537 2.83560783742587e-08
3538 2.83496284225748e-08
3539 2.83435124259768e-08
3540 2.83381460519649e-08
3541 2.83307439730152e-08
3542 2.83237486797816e-08
3543 2.83189969252362e-08
3544 2.83117689292567e-08
3545 2.8305040089549e-08
3546 2.82991603484106e-08
3547 2.8292774345573e-08
3548 2.82869763168492e-08
3549 2.82811214447065e-08
3550 2.82758225722546e-08
3551 2.82688503716599e-08
3552 2.82631251735666e-08
3553 2.82577623522684e-08
3554 2.82507475191096e-08
3555 2.82444130306203e-08
3556 2.823772504712e-08
3557 2.82320673505865e-08
3558 2.82266743312221e-08
3559 2.82193823863963e-08
3560 2.82130585560481e-08
3561 2.82071361823455e-08
3562 2.82007643903626e-08
3563 2.81948011604527e-08
3564 2.81881717967281e-08
3565 2.81812280178428e-08
3566 2.81757479569933e-08
3567 2.81703353977036e-08
3568 2.8162766341211e-08
3569 2.81566041593351e-08
3570 2.81503549359741e-08
3571 2.81434449078688e-08
3572 2.81371814736531e-08
3573 2.81312182437432e-08
3574 2.8125944240287e-08
3575 2.81189898032608e-08
3576 2.81121934619932e-08
3577 2.81056422579695e-08
3578 2.81006666824624e-08
3579 2.80932201945916e-08
3580 2.80869318913801e-08
3581 2.80804535179868e-08
3582 2.80736998092834e-08
3583 2.80671219599071e-08
3584 2.80611995862046e-08
3585 2.80543162034519e-08
3586 2.80490599635641e-08
3587 2.80424004017732e-08
3588 2.80361884819058e-08
3589 2.80310548106399e-08
3590 2.80232708149697e-08
3591 2.80165934896104e-08
3592 2.80097651739197e-08
3593 2.800373444245e-08
3594 2.79981762218995e-08
3595 2.79902412358979e-08
3596 2.79837379935088e-08
3597 2.79771796840578e-08
3598 2.79710299366798e-08
3599 2.79650294032763e-08
3600 2.79577161421685e-08
3601 2.79509482226104e-08
3602 2.79448695295059e-08
3603 2.79382170731424e-08
3604 2.79320033769181e-08
3605 2.79260845559293e-08
3606 2.79187020169047e-08
3607 2.79116285639702e-08
3608 2.79051430851496e-08
3609 2.7898524379566e-08
3610 2.78923888430427e-08
3611 2.78866210123851e-08
3612 2.78791301155934e-08
3613 2.78719145541118e-08
3614 2.78658713881441e-08
3615 2.78592349189921e-08
3616 2.7852820494445e-08
3617 2.78463243574834e-08
3618 2.78411356191555e-08
3619 2.78325611446917e-08
3620 2.7826020598809e-08
3621 2.78199578929161e-08
3622 2.78127973984965e-08
3623 2.78064344882978e-08
3624 2.77995244601925e-08
3625 2.7793756629535e-08
3626 2.77852869601247e-08
3627 2.77794800496167e-08
3628 2.77726108777188e-08
3629 2.77658358527333e-08
3630 2.77586860164547e-08
3631 2.77533693804344e-08
3632 2.77456226882578e-08
3633 2.77358829237073e-08
3634 2.77274168070107e-08
3635 2.77171867679726e-08
3636 2.77074754251316e-08
3637 2.7697925730763e-08
3638 2.7689232240391e-08
3639 2.76799489995483e-08
3640 2.76712786018152e-08
3641 2.76631872964117e-08
3642 2.76538010268723e-08
3643 2.76450844438614e-08
3644 2.76370908380841e-08
3645 2.76282339228828e-08
3646 2.76196985282695e-08
3647 2.76116161046502e-08
3648 2.76032476875798e-08
3649 2.75945559735646e-08
3650 2.75863030196888e-08
3651 2.75784781678112e-08
3652 2.75705271945981e-08
3653 2.7562165882955e-08
3654 2.75536589100511e-08
3655 2.75454272724573e-08
3656 2.75373839286885e-08
3657 2.75298557284032e-08
3658 2.75214677714075e-08
3659 2.75135558780448e-08
3660 2.75062372878665e-08
3661 2.74976539316185e-08
3662 2.74900919805532e-08
3663 2.74815619150104e-08
3664 2.74740941108575e-08
3665 2.74663065624736e-08
3666 2.74582667714185e-08
3667 2.74494738050635e-08
3668 2.74417626400236e-08
3669 2.7433783245101e-08
3670 2.74259548405098e-08
3671 2.74177072157045e-08
3672 2.74103424402483e-08
3673 2.74021108026545e-08
3674 2.73946074713649e-08
3675 2.73863172139954e-08
3676 2.7378634470665e-08
3677 2.73711524556575e-08
3678 2.7363304511141e-08
3679 2.73559699337511e-08
3680 2.73479336954097e-08
3681 2.73394924477088e-08
3682 2.73307350084906e-08
3683 2.73219242785672e-08
3684 2.73133675676718e-08
3685 2.7304107419468e-08
3686 2.72966715897383e-08
3687 2.72862070715973e-08
3688 2.72770019904556e-08
3689 2.72686584423809e-08
3690 2.72601354822655e-08
3691 2.72501470277575e-08
3692 2.72425566549828e-08
3693 2.72327280725904e-08
3694 2.72242530741096e-08
3695 2.72151581270919e-08
3696 2.72072018248082e-08
3697 2.71983076061133e-08
3698 2.71894808889783e-08
3699 2.71817537367269e-08
3700 2.71729501122309e-08
3701 2.71651217076396e-08
3702 2.71577604848972e-08
3703 2.71484399405608e-08
3704 2.71399400730843e-08
3705 2.71322129208329e-08
3706 2.71238320692646e-08
3707 2.7115930834043e-08
3708 2.71075411006905e-08
3709 2.71014108932377e-08
3710 2.70937494661894e-08
3711 2.70855693429439e-08
3712 2.70774815902541e-08
3713 2.70694879844768e-08
3714 2.70614677333469e-08
3715 2.70528293100369e-08
3716 2.70452371609053e-08
3717 2.70373892163889e-08
3718 2.7029280147417e-08
3719 2.70215512188088e-08
3720 2.70131739199542e-08
3721 2.70057274320834e-08
3722 2.69975952704726e-08
3723 2.69895643612017e-08
3724 2.69832653998492e-08
3725 2.69739484082265e-08
3726 2.69667346231017e-08
3727 2.69585083145785e-08
3728 2.69501345684375e-08
3729 2.69426312371479e-08
3730 2.69358455540214e-08
3731 2.69269104791192e-08
3732 2.69191406943037e-08
3733 2.69113602513471e-08
3734 2.69050310919283e-08
3735 2.68958277871434e-08
3736 2.6889217963344e-08
3737 2.68801318981104e-08
3738 2.68737725406254e-08
3739 2.68646527246119e-08
3740 2.68568225436638e-08
3741 2.68499995570437e-08
3742 2.68413504755927e-08
3743 2.68347051246565e-08
3744 2.68261661773295e-08
3745 2.68182418494689e-08
3746 2.68114508372719e-08
3747 2.68024553662372e-08
3748 2.67945186038787e-08
3749 2.67874273873758e-08
3750 2.67793751618228e-08
3751 2.67716835367082e-08
3752 2.67641766527049e-08
3753 2.67579860491196e-08
3754 2.67492161754035e-08
3755 2.67426916167324e-08
3756 2.67336446313493e-08
3757 2.6726038271363e-08
3758 2.67194746328414e-08
3759 2.67109534490828e-08
3760 2.67087045813241e-08
3761 2.67029456324508e-08
3762 2.66968918083421e-08
3763 2.66874149446039e-08
3764 2.66753410471665e-08
3765 2.66643755963969e-08
3766 2.66528097370156e-08
3767 2.66420254746436e-08
3768 2.66317705666097e-08
3769 2.6621259863191e-08
3770 2.66052886388479e-08
3771 2.65918806974241e-08
3772 2.6579339618138e-08
3773 2.65675552668654e-08
3774 2.65555062384237e-08
3775 2.65442263724935e-08
3776 2.65318611525345e-08
3777 2.65207962257819e-08
3778 2.65098520912943e-08
3779 2.64993094134525e-08
3780 2.6488431004168e-08
3781 2.6477835035621e-08
3782 2.6468223168763e-08
3783 2.64590696019695e-08
3784 2.64492410195771e-08
3785 2.64385846548976e-08
3786 2.64292605578476e-08
3787 2.64190820331578e-08
3788 2.64083404033499e-08
3789 2.63995740823475e-08
3790 2.63901771546671e-08
3791 2.63805688405228e-08
3792 2.63702784053521e-08
3793 2.63616026785485e-08
3794 2.63523158849921e-08
3795 2.63430166569378e-08
3796 2.6333248470678e-08
3797 2.63241961562244e-08
3798 2.63155204294208e-08
3799 2.63059760641227e-08
3800 2.6297090727212e-08
3801 2.62870223366463e-08
3802 2.62789416893838e-08
3803 2.62696335795454e-08
3804 2.62607855461283e-08
3805 2.62520885030426e-08
3806 2.6242792827702e-08
3807 2.62344599377684e-08
3808 2.62247770166368e-08
3809 2.62165933406777e-08
3810 2.62079939972182e-08
3811 2.61994745898164e-08
3812 2.61906389908972e-08
3813 2.61821213598523e-08
3814 2.61734189876961e-08
3815 2.61654093947072e-08
3816 2.61566874826258e-08
3817 2.61489958575112e-08
3818 2.6140392961338e-08
3819 2.61318131578037e-08
3820 2.61232315779125e-08
3821 2.61141703816747e-08
3822 2.61062869100215e-08
3823 2.60979557964447e-08
3824 2.60893404657736e-08
3825 2.60814374541951e-08
3826 2.60728079126693e-08
3827 2.60641428440067e-08
3828 2.60562167397893e-08
3829 2.60474966040647e-08
3830 2.60393484552424e-08
3831 2.60309853672425e-08
3832 2.6022894061839e-08
3833 2.60146286734653e-08
3834 2.60066030932649e-08
3835 2.59980446060126e-08
3836 2.5989756125e-08
3837 2.59811052671921e-08
3838 2.59733710095134e-08
3839 2.59650150269408e-08
3840 2.59571795169222e-08
3841 2.59489549847558e-08
3842 2.59406860436684e-08
3843 2.5932992642197e-08
3844 2.59247894263126e-08
3845 2.5916765622469e-08
3846 2.59087862275464e-08
3847 2.59004337976876e-08
3848 2.58929500063232e-08
3849 2.58845389566886e-08
3850 2.58769645711254e-08
3851 2.58687258281043e-08
3852 2.58612615766651e-08
3853 2.58534278430034e-08
3854 2.5846043527622e-08
3855 2.58376573469832e-08
3856 2.58300403288558e-08
3857 2.58220484994354e-08
3858 2.58113193041254e-08
3859 2.58031818134441e-08
3860 2.57951597859574e-08
3861 2.57866759056924e-08
3862 2.57773127287919e-08
3863 2.57677790216349e-08
3864 2.57581085350012e-08
3865 2.57483065979613e-08
3866 2.57385828206225e-08
3867 2.57302570361162e-08
3868 2.57207499743117e-08
3869 2.57116905544308e-08
3870 2.57067611642015e-08
3871 2.57008068160758e-08
3872 2.56938434972653e-08
3873 2.56856651503767e-08
3874 2.56770764650582e-08
3875 2.56680365851025e-08
3876 2.5659160129976e-08
3877 2.5650301438418e-08
3878 2.56407517440493e-08
3879 2.56317491675873e-08
3880 2.56230006101532e-08
3881 2.56141845511593e-08
3882 2.56053542813106e-08
3883 2.55967957940584e-08
3884 2.55880436839107e-08
3885 2.55794265768827e-08
3886 2.55709231566925e-08
3887 2.55622438771752e-08
3888 2.55539607252331e-08
3889 2.55457717202034e-08
3890 2.55377514690736e-08
3891 2.55304417606794e-08
3892 2.55204906096651e-08
3893 2.55120564673916e-08
3894 2.55020431438879e-08
3895 2.54927527976179e-08
3896 2.54846241887208e-08
3897 2.54748968586682e-08
3898 2.54666669974313e-08
3899 2.54583145675724e-08
3900 2.5448231966152e-08
3901 2.54404657340501e-08
3902 2.54312073622032e-08
3903 2.54235299479433e-08
3904 2.54155416712365e-08
3905 2.54059653315153e-08
3906 2.53984122622342e-08
3907 2.5389155666744e-08
3908 2.53812082462446e-08
3909 2.53734455668564e-08
3910 2.53653897885897e-08
3911 2.5356637678442e-08
3912 2.53490224366715e-08
3913 2.53407073103062e-08
3914 2.53319125675944e-08
3915 2.53243346293175e-08
3916 2.53168828123762e-08
3917 2.53088430213211e-08
3918 2.53068908051546e-08
3919 2.53026772867315e-08
3920 2.52973819669933e-08
3921 2.52911878106943e-08
3922 2.52838230352381e-08
3923 2.52763303620895e-08
3924 2.52684522195068e-08
3925 2.52602365691246e-08
3926 2.52526302091383e-08
3927 2.52437217795887e-08
3928 2.52371350484282e-08
3929 2.5228443334413e-08
3930 2.52209186868413e-08
3931 2.52130298861175e-08
3932 2.52060505800955e-08
3933 2.51989700217337e-08
3934 2.51905483139581e-08
3935 2.51836169695707e-08
3936 2.51760887692853e-08
3937 2.51684966201537e-08
3938 2.51720280175505e-08
3939 2.51683953678139e-08
3940 2.51581262489253e-08
3941 2.5148617410764e-08
3942 2.51387461958075e-08
3943 2.51292320285756e-08
3944 2.51197427303396e-08
3945 2.51103831061528e-08
3946 2.51007872265063e-08
3947 2.5092063538068e-08
3948 2.50830964887427e-08
3949 2.50755523012458e-08
3950 2.50566554171883e-08
3951 2.50411460456235e-08
3952 2.50277150115608e-08
3953 2.50152343284071e-08
3954 2.50037395232994e-08
3955 2.49927207818246e-08
3956 2.49816096697941e-08
3957 2.4971091860948e-08
3958 2.49615723646457e-08
3959 2.49516372008429e-08
3960 2.4941504861431e-08
3961 2.49315874611966e-08
3962 2.49223521819886e-08
3963 2.49131986151951e-08
3964 2.49044127542675e-08
3965 2.4895724592966e-08
3966 2.48864822083306e-08
3967 2.4878845650278e-08
3968 2.48702800575984e-08
3969 2.48619258513827e-08
3970 2.48538096769835e-08
3971 2.48446845318995e-08
3972 2.48367051369769e-08
3973 2.48283704706864e-08
3974 2.48201885710841e-08
3975 2.48125164858948e-08
3976 2.48037999028838e-08
3977 2.47964688782076e-08
3978 2.47878109149724e-08
3979 2.4780437257732e-08
3980 2.47732856450966e-08
3981 2.47649882822998e-08
3982 2.47573908040977e-08
3983 2.47497720096135e-08
3984 2.47422491383986e-08
3985 2.47353160176544e-08
3986 2.47270079967166e-08
3987 2.47201761283122e-08
3988 2.47120190977057e-08
3989 2.47048745904976e-08
3990 2.46970675021885e-08
3991 2.46894025224265e-08
3992 2.46823006477825e-08
3993 2.46748328436297e-08
3994 2.46676030712933e-08
3995 2.46618281352085e-08
3996 2.46551685734175e-08
3997 2.46467255493599e-08
3998 2.46411762105936e-08
3999 2.46319995511612e-08
4000 2.46261873115827e-08
4001 2.46190374753041e-08
4002 2.46108751156271e-08
4003 2.46040663398617e-08
4004 2.45962326061999e-08
4005 2.45887630256902e-08
4006 2.45831408562935e-08
4007 2.45745628291161e-08
4008 2.45682354460541e-08
4009 2.45603022364094e-08
4010 2.45529463427374e-08
4011 2.45462352665982e-08
4012 2.45387656860885e-08
4013 2.45329712100784e-08
4014 2.45261446707445e-08
4015 2.45190108216775e-08
4016 2.45124223141602e-08
4017 2.45053648484372e-08
4018 2.44982523156523e-08
4019 2.44917792713295e-08
4020 2.44845139718564e-08
4021 2.44779183589117e-08
4022 2.44711646502083e-08
4023 2.446446423221e-08
4024 2.4457618152951e-08
4025 2.44510687252841e-08
4026 2.44438638219435e-08
4027 2.44365647716904e-08
4028 2.4430297784761e-08
4029 2.44234747981409e-08
4030 2.44173445906881e-08
4031 2.44102906776789e-08
4032 2.44050308850774e-08
4033 2.43976607805507e-08
4034 2.43910474040376e-08
4035 2.43847360081872e-08
4036 2.43779449959902e-08
4037 2.43719977532919e-08
4038 2.43653115461484e-08
4039 2.43580888792394e-08
4040 2.43521025566906e-08
4041 2.43468392113755e-08
4042 2.43388562637392e-08
4043 2.43326034876645e-08
4044 2.432621570847e-08
4045 2.43200783955899e-08
4046 2.43134383737242e-08
4047 2.43063560390055e-08
4048 2.43004070199504e-08
4049 2.42937066019522e-08
4050 2.428823364653e-08
4051 2.42818796181155e-08
4052 2.42749838008649e-08
4053 2.42689477403246e-08
4054 2.42628210855855e-08
4055 2.425611356216e-08
4056 2.42500082237029e-08
4057 2.42438140674039e-08
4058 2.4237586160325e-08
4059 2.42317579335349e-08
4060 2.42251285698103e-08
4061 2.42191724453278e-08
4062 2.42131541483559e-08
4063 2.42070417044715e-08
4064 2.42017659246585e-08
4065 2.41954882795881e-08
4066 2.41889210883528e-08
4067 2.41834801073537e-08
4068 2.41783872922952e-08
4069 2.41713014048628e-08
4070 2.41664626088323e-08
4071 2.41600801587083e-08
4072 2.41541862067152e-08
4073 2.41480453411214e-08
4074 2.41424427116499e-08
4075 2.41364297437485e-08
4076 2.41307862580697e-08
4077 2.4124622299837e-08
4078 2.41200748263282e-08
4079 2.41135449385865e-08
4080 2.41071749229604e-08
4081 2.41020536861924e-08
4082 2.40967121811764e-08
4083 2.40906032900057e-08
4084 2.40846702581621e-08
4085 2.40796094175266e-08
4086 2.40746622637289e-08
4087 2.40691893083067e-08
4088 2.40641284676713e-08
4089 2.40589166367045e-08
4090 2.40540583007487e-08
4091 2.40484734348456e-08
4092 2.40436826004498e-08
4093 2.40378827953691e-08
4094 2.40321202937821e-08
4095 2.40270008333709e-08
4096 2.40215634050855e-08
4097 2.40157547182207e-08
4098 2.40107862481409e-08
4099 2.40055211264689e-08
4100 2.40002950846474e-08
4101 2.39947670621632e-08
4102 2.3989205288899e-08
4103 2.39852511185745e-08
4104 2.39799167189858e-08
4105 2.39743336294396e-08
4106 2.39694042392102e-08
4107 2.396333442789e-08
4108 2.39585880024151e-08
4109 2.39541133595367e-08
4110 2.39481856567636e-08
4111 2.39430963944187e-08
4112 2.39376340971376e-08
4113 2.39328894480195e-08
4114 2.39272246460587e-08
4115 2.39229596132873e-08
4116 2.39172273097665e-08
4117 2.39117223799212e-08
4118 2.39065851559417e-08
4119 2.39016291203598e-08
4120 2.38967476917651e-08
4121 2.38912214456377e-08
4122 2.38865531798638e-08
4123 2.38806432406591e-08
4124 2.38766126869905e-08
4125 2.38710722300084e-08
4126 2.38660806672897e-08
4127 2.38610162739405e-08
4128 2.38561543852711e-08
4129 2.38514861194972e-08
4130 2.38458746082415e-08
4131 2.38414799014208e-08
4132 2.38358683901652e-08
4133 2.38298802912595e-08
4134 2.38239312722044e-08
4135 2.38185986489725e-08
4136 2.38123867291051e-08
4137 2.38059989499106e-08
4138 2.37995880780772e-08
4139 2.37929818069915e-08
4140 2.37869457464512e-08
4141 2.37808333025669e-08
4142 2.37741186737139e-08
4143 2.37681927472977e-08
4144 2.37620998433385e-08
4145 2.37557067350735e-08
4146 2.37492212562529e-08
4147 2.37434036876039e-08
4148 2.37368293909412e-08
4149 2.37306281292149e-08
4150 2.37250841195191e-08
4151 2.37184636375787e-08
4152 2.37125412638761e-08
4153 2.37067414587955e-08
4154 2.37002826253274e-08
4155 2.36939232678424e-08
4156 2.36884094562129e-08
4157 2.36823858301705e-08
4158 2.36760993033158e-08
4159 2.366990337066e-08
4160 2.36638300066261e-08
4161 2.36579005274962e-08
4162 2.36519088758769e-08
4163 2.36460468983068e-08
4164 2.36396999753197e-08
4165 2.3634077805923e-08
4166 2.36280737198058e-08
4167 2.36219612759214e-08
4168 2.36162929212469e-08
4169 2.36102568607066e-08
4170 2.36044090939913e-08
4171 2.36015864629735e-08
4172 2.35945343263211e-08
4173 2.35909656254307e-08
4174 2.35832615658182e-08
4175 2.35797514847036e-08
4176 2.35751187460664e-08
4177 2.35675887694242e-08
4178 2.35630999156911e-08
4179 2.35585666530369e-08
4180 2.35536070647413e-08
4181 2.35482318089453e-08
4182 2.35429382655639e-08
4183 2.35380603896829e-08
4184 2.35319070895912e-08
4185 2.35266721659855e-08
4186 2.3521412373384e-08
4187 2.35162449513382e-08
4188 2.35103776446977e-08
4189 2.35046293539654e-08
4190 2.34995223280521e-08
4191 2.34930759290819e-08
4192 2.34883774652417e-08
4193 2.34825527911653e-08
4194 2.347701411054e-08
4195 2.34715642477568e-08
4196 2.34658568132318e-08
4197 2.34604176085895e-08
4198 2.34549606403789e-08
4199 2.34493988671147e-08
4200 2.34439081481241e-08
4201 2.34376766883315e-08
4202 2.3432262352685e-08
4203 2.34262031995058e-08
4204 2.34213146654838e-08
4205 2.34154438061296e-08
4206 2.34106121155264e-08
4207 2.34049792879887e-08
4208 2.33993251441689e-08
4209 2.33937509364068e-08
4210 2.33880363964545e-08
4211 2.33829666740348e-08
4212 2.33766694890392e-08
4213 2.33715180542049e-08
4214 2.33658177251073e-08
4215 2.33613359768015e-08
4216 2.33554686701609e-08
4217 2.33496066925909e-08
4218 2.33443895325536e-08
4219 2.33388313120031e-08
4220 2.33335981647542e-08
4221 2.33285035733388e-08
4222 2.33229613399999e-08
4223 2.33175718733492e-08
4224 2.33117205539202e-08
4225 2.33062351640001e-08
4226 2.33009949113239e-08
4227 2.32958079493528e-08
4228 2.32901413710351e-08
4229 2.32850112524829e-08
4230 2.32798136323709e-08
4231 2.32743762040855e-08
4232 2.32470522831818e-08
4233 2.32267147737275e-08
4234 2.32106742714677e-08
4235 2.31974652820099e-08
4236 2.31862138377892e-08
4237 2.31759287316891e-08
4238 2.31668799699492e-08
4239 2.31585612908702e-08
4240 2.3150887429324e-08
4241 2.31429613251066e-08
4242 2.31363124214568e-08
4243 2.31288943552954e-08
4244 2.31223502566991e-08
4245 2.31159731356456e-08
4246 2.31092833757884e-08
4247 2.31023786767537e-08
4248 2.30966428205193e-08
4249 2.30905019549255e-08
4250 2.30842758242034e-08
4251 2.30779910737056e-08
4252 2.30722498884006e-08
4253 2.30664465306063e-08
4254 2.3060708898015e-08
4255 2.30545715851349e-08
4256 2.30489778374476e-08
4257 2.30431798087238e-08
4258 2.30374901377672e-08
4259 2.3032681539803e-08
4260 2.30264998180019e-08
4261 2.3020724881917e-08
4262 2.30141417034702e-08
4263 2.30095320574719e-08
4264 2.30041514726054e-08
4265 2.29983392330269e-08
4266 2.29934364881501e-08
4267 2.29872636481332e-08
4268 2.29820713570916e-08
4269 2.29775807270016e-08
4270 2.29714647304036e-08
4271 2.29665761963815e-08
4272 2.29613199564938e-08
4273 2.2956427869758e-08
4274 2.2951384792691e-08
4275 2.2945588540324e-08
4276 2.29402417062374e-08
4277 2.29348717795119e-08
4278 2.29301093668255e-08
4279 2.29247874017346e-08
4280 2.2920303877072e-08
4281 2.29142198548971e-08
4282 2.29095249437705e-08
4283 2.29044410104962e-08
4284 2.29003234153424e-08
4285 2.28952057312881e-08
4286 2.28891696707478e-08
4287 2.28843717309246e-08
4288 2.28792398360156e-08
4289 2.28748735509043e-08
4290 2.2869269145076e-08
4291 2.28646861444304e-08
4292 2.2859790504981e-08
4293 2.28552305969743e-08
4294 2.28511236599616e-08
4295 2.28458176820823e-08
4296 2.28403447266601e-08
4297 2.28355734321894e-08
4298 2.28303083105175e-08
4299 2.28259828816135e-08
4300 2.28221992415456e-08
4301 2.28165735194352e-08
4302 2.28120242695695e-08
4303 2.28068230967438e-08
4304 2.28019629844312e-08
4305 2.27974474853454e-08
4306 2.2792342235789e-08
4307 2.27878480529853e-08
4308 2.27834640043056e-08
4309 2.27784067163839e-08
4310 2.27730510005131e-08
4311 2.27687397824639e-08
4312 2.27647998229941e-08
4313 2.27597691804249e-08
4314 2.27555307930061e-08
4315 2.27515375428311e-08
4316 2.27468657243435e-08
4317 2.27417444875755e-08
4318 2.27370211547395e-08
4319 2.27326530932714e-08
4320 2.27278480480209e-08
4321 2.27244747463828e-08
4322 2.2719396142179e-08
4323 2.27147012310525e-08
4324 2.27102674443813e-08
4325 2.27060272806057e-08
4326 2.27020358067875e-08
4327 2.26980141349031e-08
4328 2.26929230962014e-08
4329 2.26881944342949e-08
4330 2.26835137340231e-08
4331 2.26791847524055e-08
4332 2.26747616238754e-08
4333 2.26701732941592e-08
4334 2.26659313540267e-08
4335 2.26620198162664e-08
4336 2.26569394357057e-08
4337 2.26523546587032e-08
4338 2.26477823161986e-08
4339 2.26435616923482e-08
4340 2.26389520463499e-08
4341 2.26337224518147e-08
4342 2.26292158345132e-08
4343 2.26094503119612e-08
4344 2.25447944757207e-08
4345 2.25163336864398e-08
4346 2.24786180780256e-08
4347 2.24510685598034e-08
4348 2.24303651208402e-08
4349 2.24125074055337e-08
4350 2.23989466974217e-08
4351 2.23871374771534e-08
4352 2.23864287107745e-08
4353 2.23744631711043e-08
4354 2.23533245247154e-08
4355 2.23335430149518e-08
4356 2.23182041736436e-08
4357 2.23042757596659e-08
4358 2.2292660162293e-08
4359 2.2282307554633e-08
4360 2.22625509138652e-08
4361 2.22452474218926e-08
4362 2.22312550590686e-08
4363 2.22188951681801e-08
4364 2.22078266887138e-08
4365 2.22020464235584e-08
4366 2.21937455080479e-08
4367 2.21857838766937e-08
4368 2.21777156639291e-08
4369 2.2169720281795e-08
4370 2.21620481966056e-08
4371 2.21545093381792e-08
4372 2.21470717320926e-08
4373 2.21397353783459e-08
4374 2.21324931715117e-08
4375 2.21256186705432e-08
4376 2.21185647575339e-08
4377 2.21116120968645e-08
4378 2.21046221327015e-08
4379 2.20977955933677e-08
4380 2.20906546388733e-08
4381 2.20840252751486e-08
4382 2.2077152550537e-08
4383 2.20621423352441e-08
4384 2.20488995950063e-08
4385 2.20379892112987e-08
4386 2.20288267627211e-08
4387 2.20200391254366e-08
4388 2.20116458393704e-08
4389 2.20038653964139e-08
4390 2.19968114834046e-08
4391 2.19892903885466e-08
4392 2.19830429415424e-08
4393 2.19756266517379e-08
4394 2.19690132752248e-08
4395 2.19626041797483e-08
4396 2.19562483749769e-08
4397 2.19497309217331e-08
4398 2.19433129444724e-08
4399 2.19372147114427e-08
4400 2.19310187787869e-08
4401 2.19250377853086e-08
4402 2.1919481341115e-08
4403 2.19131255363436e-08
4404 2.19073736928976e-08
4405 2.19011493385324e-08
4406 2.18955076292104e-08
4407 2.18895301884459e-08
4408 2.18836753163032e-08
4409 2.18783497984987e-08
4410 2.18722124856185e-08
4411 2.18670823670664e-08
4412 2.18612647984173e-08
4413 2.18558646736255e-08
4414 2.18504148108423e-08
4415 2.18448068523003e-08
4416 2.18392965933845e-08
4417 2.18340101554304e-08
4418 2.18279652131059e-08
4419 2.18235314264348e-08
4420 2.18183320299659e-08
4421 2.18126157136567e-08
4422 2.18082565339728e-08
4423 2.18031157572796e-08
4424 2.17973177285558e-08
4425 2.1791043636199e-08
4426 2.17718820749724e-08
4427 2.17639986033191e-08
4428 2.1745849565491e-08
4429 2.17362963184087e-08
4430 2.17229878529679e-08
4431 2.17144791037072e-08
4432 2.17067643859536e-08
4433 2.16995754698246e-08
4434 2.16921876017295e-08
4435 2.16851727685707e-08
4436 2.16781614881256e-08
4437 2.16712212619541e-08
4438 2.16633555538692e-08
4439 2.16574367328803e-08
4440 2.16497664240478e-08
4441 2.1643714376296e-08
4442 2.16366888849961e-08
4443 2.16301234701177e-08
4444 2.16242383999088e-08
4445 2.16176463396778e-08
4446 2.16110223050237e-08
4447 2.1609558586988e-08
4448 2.15990585417103e-08
4449 2.15927400404325e-08
4450 2.1591551657707e-08
4451 2.15875548548183e-08
4452 2.15760760369221e-08
4453 2.15736264408406e-08
4454 2.15673807701933e-08
4455 2.15619078147711e-08
4456 2.1555681684049e-08
4457 2.1550084383648e-08
4458 2.15442046425096e-08
4459 2.15386375401749e-08
4460 2.15327560226797e-08
4461 2.14879687376879e-08
4462 2.14686117772089e-08
4463 2.14530668785073e-08
4464 2.14401314480028e-08
4465 2.14168700551909e-08
4466 2.13860946729483e-08
4467 2.13630713119528e-08
4468 2.13448707597763e-08
4469 2.13296686979447e-08
4470 2.13166337914572e-08
4471 2.13055670883477e-08
4472 2.12950812539248e-08
4473 2.12856559045349e-08
4474 2.12762429896429e-08
4475 2.12672635058198e-08
4476 2.12588044945505e-08
4477 2.12510826713697e-08
4478 2.12433821644709e-08
4479 2.1235173619516e-08
4480 2.12286099809944e-08
4481 2.1221801205229e-08
4482 2.12146407108094e-08
4483 2.12082387207602e-08
4484 2.12012629674518e-08
4485 2.11947845940585e-08
4486 2.11882014156117e-08
4487 2.1181262965797e-08
4488 2.11754684897869e-08
4489 2.11692121609985e-08
4490 2.1162891883364e-08
4491 2.11561577145858e-08
4492 2.11505497560438e-08
4493 2.11441832931314e-08
4494 2.11377972902937e-08
4495 2.11323420984399e-08
4496 2.1126004057237e-08
4497 2.11208366351912e-08
4498 2.11148662998539e-08
4499 2.11092157087478e-08
4500 2.11035491304301e-08
4501 2.10977137982127e-08
4502 2.10919175458457e-08
4503 2.10861248461924e-08
4504 2.10803356992528e-08
4505 2.10748201112665e-08
4506 2.1068929711987e-08
4507 2.10633022135198e-08
4508 2.10574508940908e-08
4509 2.10522337340535e-08
4510 2.10467714367724e-08
4511 2.10411954526535e-08
4512 2.10364170527555e-08
4513 2.10313242376969e-08
4514 2.10255546306826e-08
4515 2.10208135342782e-08
4516 2.10151380741763e-08
4517 2.10101340769597e-08
4518 2.10045207893472e-08
4519 2.09990496102819e-08
4520 2.09933759265368e-08
4521 2.0987920734683e-08
4522 2.09820623098267e-08
4523 2.09774846382516e-08
4524 2.09712887055957e-08
4525 2.09658956862313e-08
4526 2.09607762258202e-08
4527 2.09559871677811e-08
4528 2.09501980208415e-08
4529 2.09448209886887e-08
4530 2.09399697581603e-08
4531 2.09341486367975e-08
4532 2.09291233232989e-08
4533 2.0923591748101e-08
4534 2.09186339361622e-08
4535 2.09132799966483e-08
4536 2.09088000246993e-08
4537 2.09032791076424e-08
4538 2.08982573468575e-08
4539 2.08971684401149e-08
4540 2.08927311007301e-08
4541 2.08884305408219e-08
4542 2.08831458792247e-08
4543 2.08786428146368e-08
4544 2.08744186380727e-08
4545 2.08484838282175e-08
4546 2.08297681325575e-08
4547 2.08164472326189e-08
4548 2.08050181527142e-08
4549 2.07953103625869e-08
4550 2.07869614854417e-08
4551 2.07787387296321e-08
4552 2.07716972511207e-08
4553 2.07653840789135e-08
4554 2.0758996299719e-08
4555 2.07524184503427e-08
4556 2.07467039103904e-08
4557 2.07401829044329e-08
4558 2.07343617830702e-08
4559 2.0729165939315e-08
4560 2.07234620575036e-08
4561 2.07182182521137e-08
4562 2.07122923256975e-08
4563 2.07067554214291e-08
4564 2.07014636544045e-08
4565 2.06960262261191e-08
4566 2.06908765676417e-08
4567 2.06855865769739e-08
4568 2.06803605351524e-08
4569 2.06749160014397e-08
4570 2.06699777294261e-08
4571 2.06647978728824e-08
4572 2.06595700547041e-08
4573 2.06549657377764e-08
4574 2.06505550437441e-08
4575 2.06458121709829e-08
4576 2.06409609404545e-08
4577 2.06364081378752e-08
4578 2.06316226325498e-08
4579 2.06268584435065e-08
4580 2.06219787912687e-08
4581 2.06172234840096e-08
4582 2.06129033841762e-08
4583 2.06083825560199e-08
4584 2.06033057281729e-08
4585 2.05989856283395e-08
4586 2.05944292730464e-08
4587 2.05900185790142e-08
4588 2.05855350543516e-08
4589 2.05808881048597e-08
4590 2.05762127336584e-08
4591 2.05716172985149e-08
4592 2.0566782055198e-08
4593 2.05621848436977e-08
4594 2.05582111334479e-08
4595 2.05534274044794e-08
4596 2.05487005189298e-08
4597 2.05439540934549e-08
4598 2.0539948408782e-08
4599 2.05351806670251e-08
4600 2.05308445799801e-08
4601 2.05262598029776e-08
4602 2.05217904891697e-08
4603 2.0517330057146e-08
4604 2.05129762065326e-08
4605 2.05088053206737e-08
4606 2.0504455022774e-08
4607 2.04996108976729e-08
4608 2.04956158711411e-08
4609 2.04908268131021e-08
4610 2.04864054609288e-08
4611 2.04825862937241e-08
4612 2.04731964714711e-08
4613 2.04622150334899e-08
4614 2.04474339682292e-08
4615 2.04355590227578e-08
4616 2.04265706571505e-08
4617 2.04188026486918e-08
4618 2.04125356617624e-08
4619 2.04073202780819e-08
4620 2.04002752468568e-08
4621 2.039338298232e-08
4622 2.03860039960091e-08
4623 2.03789571884272e-08
4624 2.03718677482811e-08
4625 2.03654551000909e-08
4626 2.03585432956288e-08
4627 2.03518482067011e-08
4628 2.03454941782866e-08
4629 2.03389980413249e-08
4630 2.03332497505926e-08
4631 2.0326819338834e-08
4632 2.03208969651314e-08
4633 2.03150936073371e-08
4634 2.03087875405572e-08
4635 2.03033057033508e-08
4636 2.02975840579711e-08
4637 2.02919743230723e-08
4638 2.02863930098829e-08
4639 2.0280662482719e-08
4640 2.02750527478202e-08
4641 2.02698533513512e-08
4642 2.02640979551916e-08
4643 2.02586427633378e-08
4644 2.02533154691764e-08
4645 2.0248110743637e-08
4646 2.02428083184714e-08
4647 2.02374881297374e-08
4648 2.02321857045717e-08
4649 2.02270040716712e-08
4650 2.0221774477136e-08
4651 2.02164809337546e-08
4652 2.02113490388456e-08
4653 2.02059418086264e-08
4654 2.02013268335577e-08
4655 2.0196202044076e-08
4656 2.01910772545943e-08
4657 2.01858956216938e-08
4658 2.01809982058876e-08
4659 2.01757650586387e-08
4660 2.01708960645419e-08
4661 2.01663308274647e-08
4662 2.01615861783466e-08
4663 2.01564365198692e-08
4664 2.01514538389347e-08
4665 2.01470093941225e-08
4666 2.01421084256026e-08
4667 2.01260412779902e-08
4668 2.01132834831697e-08
4669 2.01035472713329e-08
4670 2.00954151097221e-08
4671 2.0088105401328e-08
4672 2.00811793860112e-08
4673 2.00752570123086e-08
4674 2.00694625362985e-08
4675 2.0063804839765e-08
4676 2.00585876797277e-08
4677 2.00531822258654e-08
4678 2.00483825096853e-08
4679 2.00434335795308e-08
4680 2.00387031412674e-08
4681 2.00335232847237e-08
4682 2.00283363227527e-08
4683 2.00237799674596e-08
4684 2.00189873567069e-08
4685 2.00141165862533e-08
4686 2.00093399627121e-08
4687 2.00048742016179e-08
4688 1.99997476357794e-08
4689 1.99955199065016e-08
4690 1.99905905162723e-08
4691 1.99861496241738e-08
4692 1.99816003743081e-08
4693 1.9977475673727e-08
4694 1.99727097083269e-08
4695 1.99684695445512e-08
4696 1.99636112085955e-08
4697 1.99594385463797e-08
4698 1.99546370538428e-08
4699 1.99498444430901e-08
4700 1.99457605987163e-08
4701 1.99415026713723e-08
4702 1.99365324249356e-08
4703 1.99321767979654e-08
4704 1.99273344492212e-08
4705 1.99232257358517e-08
4706 1.99193017635935e-08
4707 1.99141982903939e-08
4708 1.99101091169496e-08
4709 1.99058192151824e-08
4710 1.99014138502207e-08
4711 1.98968024278656e-08
4712 1.98924787753185e-08
4713 1.98884659852183e-08
4714 1.9883934498921e-08
4715 1.98794012362669e-08
4716 1.98754168678761e-08
4717 1.98712069021667e-08
4718 1.98668637096944e-08
4719 1.98626324277029e-08
4720 1.98584579891303e-08
4721 1.98540686113802e-08
4722 1.98502085879682e-08
4723 1.98458103284338e-08
4724 1.98433554032817e-08
4725 1.98393959038867e-08
4726 1.98348981683694e-08
4727 1.98309582088996e-08
4728 1.98267464668334e-08
4729 1.98233749415522e-08
4730 1.98186178579363e-08
4731 1.98144576302184e-08
4732 1.98108978111122e-08
4733 1.98071390400401e-08
4734 1.98026199882406e-08
4735 1.97992253703205e-08
4736 1.97953120562033e-08
4737 1.97905603016579e-08
4738 1.97869507445603e-08
4739 1.97836094173454e-08
4740 1.97793603717855e-08
4741 1.97750580355205e-08
4742 1.97596108364451e-08
4743 1.97412450830825e-08
4744 1.97265528356638e-08
4745 1.97086276187974e-08
4746 1.96971612353991e-08
4747 1.96874747615539e-08
4748 1.9679797347294e-08
4749 1.96716545275422e-08
4750 1.96639842187096e-08
4751 1.96571239285959e-08
4752 1.96510292482799e-08
4753 1.96450038458806e-08
4754 1.96391685136632e-08
4755 1.96341307656667e-08
4756 1.96287057718791e-08
4757 1.96230089954952e-08
4758 1.96181719758215e-08
4759 1.96127043494698e-08
4760 1.96076896941122e-08
4761 1.96027052368208e-08
4762 1.95978522299356e-08
4763 1.95959017901259e-08
4764 1.95919991341498e-08
4765 1.95870146768584e-08
4766 1.95819929160734e-08
4767 1.95768290467413e-08
4768 1.95726084228909e-08
4769 1.95683185211237e-08
4770 1.95629485943982e-08
4771 1.95584846096608e-08
4772 1.95487537268946e-08
4773 1.95402183322813e-08
4774 1.9530963513148e-08
4775 1.95244176381948e-08
4776 1.95181737439043e-08
4777 1.95124574275951e-08
4778 1.95079099540862e-08
4779 1.95027780591772e-08
4780 1.94974809630821e-08
4781 1.94932088248834e-08
4782 1.94882758819404e-08
4783 1.94844300693831e-08
4784 1.94793923213865e-08
4785 1.94749425475038e-08
4786 1.94705638278947e-08
4787 1.94662366226339e-08
4788 1.94620728422024e-08
4789 1.94575946466102e-08
4790 1.94529388153342e-08
4791 1.94486471372102e-08
4792 1.9444202692398e-08
4793 1.94399785158339e-08
4794 1.94332816505494e-08
4795 1.94282634424781e-08
4796 1.94229095029641e-08
4797 1.94161682287586e-08
4798 1.9410588691926e-08
4799 1.94055029822948e-08
4800 1.94006037901318e-08
4801 1.93948626048268e-08
4802 1.93888638477802e-08
4803 1.93845313134489e-08
4804 1.93798186387539e-08
4805 1.93756442001813e-08
4806 1.93711198193114e-08
4807 1.93669738024482e-08
4808 1.93626572553285e-08
4809 1.93589606567457e-08
4810 1.93546583204807e-08
4811 1.9349633006982e-08
4812 1.93452045493814e-08
4813 1.93409235293984e-08
4814 1.93369746881444e-08
4815 1.93321589847528e-08
4816 1.93281053384453e-08
4817 1.93236200374258e-08
4818 1.93193283593018e-08
4819 1.93179872098881e-08
4820 1.93129388037505e-08
4821 1.930950510598e-08
4822 1.93043820928551e-08
4823 1.93014795257795e-08
4824 1.92961575606887e-08
4825 1.92918250263574e-08
4826 1.92887519290252e-08
4827 1.92846343338715e-08
4828 1.92795255316014e-08
4829 1.92766549389489e-08
4830 1.9272428986028e-08
4831 1.9267517359367e-08
4832 1.92636235851751e-08
4833 1.92596054660044e-08
4834 1.92558164968659e-08
4835 1.92531484088931e-08
4836 1.92493878614641e-08
4837 1.92457108028066e-08
4838 1.92418188049714e-08
4839 1.92382589858653e-08
4840 1.92340490201559e-08
4841 1.92318161396088e-08
4842 1.92296134571279e-08
4843 1.92245259711399e-08
4844 1.92220479533489e-08
4845 1.92167952661748e-08
4846 1.9213297619558e-08
4847 1.92105957808053e-08
4848 1.92075866323194e-08
4849 1.92026927692268e-08
4850 1.9199029921424e-08
4851 1.91953475336959e-08
4852 1.91918321235107e-08
4853 1.91881373012848e-08
4854 1.91848759101276e-08
4855 1.91812983274531e-08
4856 1.91776763358575e-08
4857 1.91733988685883e-08
4858 1.91700646468007e-08
4859 1.91661229109741e-08
4860 1.91629343504474e-08
4861 1.91591364995247e-08
4862 1.91554434536556e-08
4863 1.91520221903829e-08
4864 1.91483913170032e-08
4865 1.91448563668928e-08
4866 1.91414297745496e-08
4867 1.91377882430288e-08
4868 1.91346991584851e-08
4869 1.91312281572209e-08
4870 1.91280182804121e-08
4871 1.91246307679194e-08
4872 1.91213729294759e-08
4873 1.91181577235966e-08
4874 1.91146707351209e-08
4875 1.91114999381625e-08
4876 1.9108131965595e-08
4877 1.91047409003886e-08
4878 1.91013356243275e-08
4879 1.90967739399639e-08
4880 1.908518143523e-08
4881 1.90771523023159e-08
4882 1.90707378777688e-08
4883 1.90644762199099e-08
4884 1.90586835202566e-08
4885 1.90541289413204e-08
4886 1.90492794871489e-08
4887 1.90444957581803e-08
4888 1.90402005273427e-08
4889 1.9036159315533e-08
4890 1.90314519699086e-08
4891 1.90271496336436e-08
4892 1.90229378915774e-08
4893 1.90189766158255e-08
4894 1.90150419854263e-08
4895 1.90108018216506e-08
4896 1.90068902838902e-08
4897 1.90030249314077e-08
4898 1.89991702370662e-08
4899 1.89947435558224e-08
4900 1.89912174874962e-08
4901 1.89872828570969e-08
4902 1.89831919072958e-08
4903 1.89793798455185e-08
4904 1.89757081159314e-08
4905 1.89715638754251e-08
4906 1.89678566187013e-08
4907 1.89641067294133e-08
4908 1.89601543354456e-08
4909 1.89569249187116e-08
4910 1.89527895599895e-08
4911 1.89487341373251e-08
4912 1.89453199794798e-08
4913 1.89420816809616e-08
4914 1.89380884307866e-08
4915 1.89342355128019e-08
4916 1.89308320130976e-08
4917 1.89269844241835e-08
4918 1.89234015124384e-08
4919 1.89197972844113e-08
4920 1.89158608776552e-08
4921 1.8912356125611e-08
4922 1.89083770862908e-08
4923 1.89052791199629e-08
4924 1.89015842977369e-08
4925 1.88978539483742e-08
4926 1.88945534773666e-08
4927 1.88907023357388e-08
4928 1.88875564077762e-08
4929 1.88841529080719e-08
4930 1.88804420986344e-08
4931 1.88767561581926e-08
4932 1.88735853612343e-08
4933 1.88698336955895e-08
4934 1.8866288087338e-08
4935 1.88632967024205e-08
4936 1.8859342532096e-08
4937 1.88561664060671e-08
4938 1.88528410660638e-08
4939 1.88491942054725e-08
4940 1.88462756511854e-08
4941 1.8842364113425e-08
4942 1.88391826583256e-08
4943 1.88360207431515e-08
4944 1.88285120827913e-08
4945 1.88229041242494e-08
4946 1.88184241523004e-08
4947 1.88145250490379e-08
4948 1.88107893706047e-08
4949 1.88074196216803e-08
4950 1.8804005463835e-08
4951 1.88010211843448e-08
4952 1.87979143362327e-08
4953 1.87947328811333e-08
4954 1.8791329381429e-08
4955 1.87880520030603e-08
4956 1.878513522513e-08
4957 1.87820710095821e-08
4958 1.87789055416943e-08
4959 1.87759763292661e-08
4960 1.87725834877028e-08
4961 1.87697040132662e-08
4962 1.8766542098092e-08
4963 1.87640747384421e-08
4964 1.87605557755433e-08
4965 1.87572037901873e-08
4966 1.87547328778237e-08
4967 1.87547115615416e-08
4968 1.87516828731304e-08
4969 1.87497768422418e-08
4970 1.87465989398561e-08
4971 1.87435666987312e-08
4972 1.87404989304696e-08
4973 1.87380653215996e-08
4974 1.87349051827823e-08
4975 1.87312654276184e-08
4976 1.87276381069523e-08
4977 1.87237496618309e-08
4978 1.87204580726075e-08
4979 1.87166726561827e-08
4980 1.87128801343306e-08
4981 1.87093665005023e-08
4982 1.87055704259365e-08
4983 1.87012147989662e-08
4984 1.86975910310139e-08
4985 1.86938748925058e-08
4986 1.86905211307931e-08
4987 1.86863875484278e-08
4988 1.86836217608288e-08
4989 1.86794775203225e-08
4990 1.86741928587253e-08
4991 1.86727415751875e-08
4992 1.86703328353133e-08
4993 1.86670607860151e-08
4994 1.86639148580525e-08
4995 1.86591737616482e-08
4996 1.86546120772846e-08
4997 1.86516899702838e-08
4998 1.86478477104401e-08
4999 1.86446182937061e-08
};
\addlegendentry{Test}

\nextgroupplot[
title={ELU/Tanh $\hy$},
ymin=2.72697905567862e-09, ymax=1e-05,
]
\addplot [semithick, black, dashed]
table {%
0 0.0068044427804416
1 0.000987935652483429
2 0.000225396620639003
3 0.000211181294558628
4 0.000190665577460095
5 0.000114531633588285
6 2.57327268279823e-05
7 1.78330133804536e-05
8 1.76603466406533e-05
9 1.75185876513524e-05
10 1.73672080772889e-05
11 1.72098118941904e-05
12 1.70495317710078e-05
13 1.6883599081936e-05
14 1.67075088371291e-05
15 1.65136540624218e-05
16 1.62916584685604e-05
17 1.6027069549196e-05
18 1.56987227481267e-05
19 1.52761831169954e-05
20 1.47156991001367e-05
21 1.39512738390835e-05
22 1.28873365046474e-05
23 1.14061944530164e-05
24 9.42957606309847e-06
25 7.08693948361372e-06
26 4.85833765460342e-06
27 3.31017903490149e-06
28 2.58286284485365e-06
29 2.34762777767727e-06
30 2.2734544903571e-06
31 2.23242893408582e-06
32 2.19876880018788e-06
33 2.16858317361002e-06
34 2.14045606726287e-06
35 2.11376913666328e-06
36 2.08811495245342e-06
37 2.06304696908077e-06
38 2.03814828884319e-06
39 2.01309676586092e-06
40 1.98759894478684e-06
41 1.96136906667377e-06
42 1.93414744872555e-06
43 1.9056610456083e-06
44 1.87560180894053e-06
45 1.84363291499068e-06
46 1.80937699658301e-06
47 1.77242119412213e-06
48 1.73228951094373e-06
49 1.68843996087986e-06
50 1.64026084454605e-06
51 1.58707656885149e-06
52 1.52818533097943e-06
53 1.46295859740242e-06
54 1.39089941704462e-06
55 1.31175579059395e-06
56 1.2257418765671e-06
57 1.13404216841317e-06
58 1.03901775428206e-06
59 9.44005411890103e-07
60 8.53627235049004e-07
61 7.72048373825385e-07
62 7.02283876501397e-07
63 6.45814130688649e-07
64 6.02191083675763e-07
65 5.69407509731334e-07
66 5.45118875413309e-07
67 5.27100124962487e-07
68 5.1337629537862e-07
69 5.02493315535091e-07
70 4.93496831337481e-07
71 4.85809784626312e-07
72 4.79069109386998e-07
73 4.73042084589537e-07
74 4.67577825649101e-07
75 4.6256646316678e-07
76 4.57931924108479e-07
77 4.53617336397372e-07
78 4.49582470709586e-07
79 4.45794740077332e-07
80 4.4223101845553e-07
81 4.38870958374693e-07
82 4.35697073434937e-07
83 4.32694934799471e-07
84 4.29851367094614e-07
85 4.27153868781005e-07
86 4.24591637322891e-07
87 4.22153857824625e-07
88 4.19830452406345e-07
89 4.17610955677716e-07
90 4.15488239339012e-07
91 4.13453170914124e-07
92 4.11499483039535e-07
93 4.09620217451057e-07
94 4.07809157632499e-07
95 4.06062517487094e-07
96 4.04372303630218e-07
97 4.02736548243965e-07
98 4.01149918128674e-07
99 3.99608961437536e-07
100 3.98109407327318e-07
101 3.96649457658604e-07
102 3.95225558393442e-07
103 3.938350321393e-07
104 3.92476611500214e-07
105 3.91145812579374e-07
106 3.89843561931613e-07
107 3.88566465231221e-07
108 3.87313633633646e-07
109 3.86082127313614e-07
110 3.84872060124053e-07
111 3.83680042073564e-07
112 3.82506022703843e-07
113 3.8134724752048e-07
114 3.80206110619952e-07
115 3.79085432423487e-07
116 3.77987958748882e-07
117 3.76912769052495e-07
118 3.75857008047653e-07
119 3.74818283874312e-07
120 3.7379633455803e-07
121 3.72789614718094e-07
122 3.71796869974794e-07
123 3.70817565502612e-07
124 3.69851547898037e-07
125 3.68898378061644e-07
126 3.67956728482888e-07
127 3.67027407892628e-07
128 3.66108545415855e-07
129 3.65201959720451e-07
130 3.64305377749119e-07
131 3.63419743798055e-07
132 3.62544379338026e-07
133 3.61677678148098e-07
134 3.60820915955351e-07
135 3.59971666560099e-07
136 3.59131287929415e-07
137 3.58298628997389e-07
138 3.57473829509125e-07
139 3.56654225504016e-07
140 3.55843972983827e-07
141 3.55036455906621e-07
142 3.5423830617809e-07
143 3.5344245911606e-07
144 3.52658512959181e-07
145 3.51874383888173e-07
146 3.51109070117595e-07
147 3.50335553696901e-07
148 3.49590781964615e-07
149 3.4882158676286e-07
150 3.48099964107895e-07
151 3.47326596652486e-07
152 3.46635489724356e-07
153 3.4584663882864e-07
154 3.45196925652935e-07
155 3.44379535449413e-07
156 3.43759770993657e-07
157 3.42945219366619e-07
158 3.42332919576194e-07
159 3.41565096150909e-07
160 3.4096605793632e-07
161 3.40163367888557e-07
162 3.39575695747385e-07
163 3.38843873170624e-07
164 3.3825400612475e-07
165 3.37473343689609e-07
166 3.36910848250938e-07
167 3.36195340856271e-07
168 3.35634220220093e-07
169 3.34868982544201e-07
170 3.34335552162557e-07
171 3.33608109090555e-07
172 3.33074473319428e-07
173 3.32362749148274e-07
174 3.31835686975701e-07
175 3.31134220929563e-07
176 3.30635836275128e-07
177 3.29907645386207e-07
178 3.29395325469939e-07
179 3.28860443591594e-07
180 3.28152297248252e-07
181 3.27665974537261e-07
182 3.26983980544426e-07
183 3.26485480487193e-07
184 3.25975828039837e-07
185 3.25285461183711e-07
186 3.24852488794214e-07
187 3.24177403825843e-07
188 3.23703002205633e-07
189 3.23065476538709e-07
190 3.2258067581914e-07
191 3.22100789071911e-07
192 3.21440114055882e-07
193 3.20986355537123e-07
194 3.20351101268379e-07
195 3.19899433054438e-07
196 3.19406502038433e-07
197 3.18761708593129e-07
198 3.18321365432084e-07
199 3.17840663589308e-07
200 3.17202405211603e-07
201 3.16762135518012e-07
202 3.16307863414522e-07
203 3.15688933804736e-07
204 3.15282818637463e-07
205 3.14667858620332e-07
206 3.14269254658051e-07
207 3.1365493586577e-07
208 3.13221446387679e-07
209 3.12603156205959e-07
210 3.12171248979531e-07
211 3.11725262599261e-07
212 3.11123877009933e-07
213 3.10616951185061e-07
214 3.10173758281351e-07
215 3.0958231119893e-07
216 3.0919180993827e-07
217 3.08592635665406e-07
218 3.08051862625725e-07
219 3.07604884118184e-07
220 3.07164379538705e-07
221 3.06557293589194e-07
222 3.0599764572159e-07
223 3.05531530143455e-07
224 3.05064546140876e-07
225 3.0443517075085e-07
226 3.03843176337359e-07
227 3.03413651939977e-07
228 3.02732977637277e-07
229 3.02109153931696e-07
230 3.01538624782438e-07
231 3.00844466682548e-07
232 3.00149270593408e-07
233 2.9952951652934e-07
234 2.98843623965261e-07
235 2.9818708320839e-07
236 2.97531584822153e-07
237 2.96880718387627e-07
238 2.96232529208851e-07
239 2.95587677676679e-07
240 2.94945687612014e-07
241 2.94307340594102e-07
242 2.93670803870327e-07
243 2.93036672335489e-07
244 2.92404471275098e-07
245 2.9177354651555e-07
246 2.91145293652484e-07
247 2.90517610188168e-07
248 2.89891033116163e-07
249 2.89266453076031e-07
250 2.88642189901545e-07
251 2.88019018821473e-07
252 2.87397461543826e-07
253 2.86777511951719e-07
254 2.86159988399426e-07
255 2.8554404256198e-07
256 2.84929885739693e-07
257 2.84318159526009e-07
258 2.83706932512118e-07
259 2.83096527817328e-07
260 2.8248748591686e-07
261 2.81878875536101e-07
262 2.81271501164859e-07
263 2.80664655007534e-07
264 2.80058545843787e-07
265 2.79452437510486e-07
266 2.78847227755286e-07
267 2.78241735724549e-07
268 2.77637118733409e-07
269 2.77031209917844e-07
270 2.76426058882784e-07
271 2.75820950724537e-07
272 2.75215462239409e-07
273 2.74609646677249e-07
274 2.74003502894082e-07
275 2.73396927325642e-07
276 2.72790333824879e-07
277 2.7218300930798e-07
278 2.71575171817418e-07
279 2.70966976739118e-07
280 2.70358301008855e-07
281 2.6974962862969e-07
282 2.69139625775239e-07
283 2.68531006145878e-07
284 2.67921354444312e-07
285 2.67311808645587e-07
286 2.66702801319241e-07
287 2.66093596326122e-07
288 2.65484151767303e-07
289 2.64874959072792e-07
290 2.64265929414975e-07
291 2.63656760362885e-07
292 2.63047947415274e-07
293 2.62440205132108e-07
294 2.61834376333425e-07
295 2.61234587828341e-07
296 2.60663006662831e-07
297 2.6007084692381e-07
298 2.59446772472138e-07
299 2.58831162810225e-07
300 2.58213643395955e-07
301 2.57596498697943e-07
302 2.56976969128964e-07
303 2.56357924441453e-07
304 2.55736564646725e-07
305 2.55113889616076e-07
306 2.54489920756384e-07
307 2.53863566642032e-07
308 2.53237709550547e-07
309 2.52612063107449e-07
310 2.51986331604392e-07
311 2.51358824262304e-07
312 2.50730575544367e-07
313 2.50100880906601e-07
314 2.49471066962492e-07
315 2.4883553418853e-07
316 2.48211694741762e-07
317 2.47558036650553e-07
318 2.46989146162946e-07
319 2.46287622046815e-07
320 2.45647347057343e-07
321 2.4507522925532e-07
322 2.44362874790127e-07
323 2.4373130188593e-07
324 2.43109170144251e-07
325 2.42427668453615e-07
326 2.41863225337369e-07
327 2.41129484064295e-07
328 2.40543782832781e-07
329 2.39825131980176e-07
330 2.39264424707564e-07
331 2.38517778364589e-07
332 2.37951684513682e-07
333 2.37203515282225e-07
334 2.36640619187156e-07
335 2.35883415854232e-07
336 2.35323015432165e-07
337 2.34557179521033e-07
338 2.33999853598732e-07
339 2.33224480086669e-07
340 2.3267132332272e-07
341 2.31887462645908e-07
342 2.313386044559e-07
343 2.30547287857874e-07
344 2.30001155488324e-07
345 2.29203430544445e-07
346 2.28658366715706e-07
347 2.27857255882924e-07
348 2.27311416932885e-07
349 2.26507646228669e-07
350 2.25960524515401e-07
351 2.25155427718171e-07
352 2.24605682358714e-07
353 2.23800567854759e-07
354 2.23248687265531e-07
355 2.22443154406449e-07
356 2.2188811945334e-07
357 2.21085551046052e-07
358 2.20526377435526e-07
359 2.1972730316655e-07
360 2.19164076447065e-07
361 2.18368763357013e-07
362 2.17800369989973e-07
363 2.1701074295688e-07
364 2.16435914076385e-07
365 2.15653497108725e-07
366 2.15067366974075e-07
367 2.14299187332401e-07
368 2.13675126870427e-07
369 2.12963313114045e-07
370 2.12271749851389e-07
371 2.11629831291305e-07
372 2.10929483633215e-07
373 2.10244118959757e-07
374 2.09532955044445e-07
375 2.08955097439123e-07
376 2.08184493568631e-07
377 2.07501977502744e-07
378 2.06812969651082e-07
379 2.06374719739166e-07
380 2.05442317357196e-07
381 2.04807734508705e-07
382 2.04115722369558e-07
383 2.03441607051325e-07
384 2.02753264477273e-07
385 2.02054267989027e-07
386 2.01589800008506e-07
387 2.00679902699896e-07
388 2.00048975985467e-07
389 1.99351654827318e-07
390 1.98755503351578e-07
391 1.98025035046179e-07
392 1.97353598831107e-07
393 1.96657596961636e-07
394 1.96063384253264e-07
395 1.95337596596623e-07
396 1.94676387220483e-07
397 1.94007427632315e-07
398 1.93341489252319e-07
399 1.92674231854539e-07
400 1.92005433383713e-07
401 1.91337177272821e-07
402 1.90670236815293e-07
403 1.90004239358998e-07
404 1.89337834616943e-07
405 1.88673048296728e-07
406 1.88008902972214e-07
407 1.87344813357804e-07
408 1.86681486630036e-07
409 1.86019717159702e-07
410 1.85358188774742e-07
411 1.8469757660089e-07
412 1.84037118233782e-07
413 1.83379015263441e-07
414 1.82720765037558e-07
415 1.82063840983737e-07
416 1.8140724318183e-07
417 1.80752828653041e-07
418 1.80098742635693e-07
419 1.79445978972126e-07
420 1.78794383028702e-07
421 1.78143729942626e-07
422 1.77494109840026e-07
423 1.76847019575099e-07
424 1.76199498990393e-07
425 1.75554417700674e-07
426 1.74910252846772e-07
427 1.74267584975851e-07
428 1.7362639782359e-07
429 1.7298627702278e-07
430 1.72347510302373e-07
431 1.71710359096799e-07
432 1.7107381142889e-07
433 1.70439238579512e-07
434 1.69805105421972e-07
435 1.69171232990095e-07
436 1.68536686467213e-07
437 1.67899757092194e-07
438 1.6725232629522e-07
439 1.66586070915464e-07
440 1.65926770480773e-07
441 1.65299210098446e-07
442 1.64688025222404e-07
443 1.64070302756336e-07
444 1.63452665498554e-07
445 1.62836423465151e-07
446 1.62220879130537e-07
447 1.61608536667579e-07
448 1.60996612045494e-07
449 1.60386355569031e-07
450 1.59777931880711e-07
451 1.59170600587011e-07
452 1.58563807238998e-07
453 1.5795887250647e-07
454 1.57355349484689e-07
455 1.56751363729946e-07
456 1.56149042237708e-07
457 1.55546853874533e-07
458 1.54946670188671e-07
459 1.54344826585273e-07
460 1.53742566086024e-07
461 1.53140534616902e-07
462 1.52538637212984e-07
463 1.51941912351461e-07
464 1.51349955231694e-07
465 1.50760070448008e-07
466 1.50172018144268e-07
467 1.4958606140425e-07
468 1.49001463070952e-07
469 1.48418226839198e-07
470 1.47837152302266e-07
471 1.47257825193314e-07
472 1.4667935367374e-07
473 1.46103895772853e-07
474 1.45528678717532e-07
475 1.44956573275401e-07
476 1.44385633968902e-07
477 1.4381612917802e-07
478 1.4324925888598e-07
479 1.4268360465497e-07
480 1.42120521372746e-07
481 1.41558665061936e-07
482 1.40999219388149e-07
483 1.40441961286264e-07
484 1.39886501947295e-07
485 1.39333286759857e-07
486 1.38781694233536e-07
487 1.3823278528946e-07
488 1.37685159232959e-07
489 1.37140381885281e-07
490 1.36597732312893e-07
491 1.36056749799884e-07
492 1.35518481036989e-07
493 1.34982020377095e-07
494 1.34447113255831e-07
495 1.33915224532544e-07
496 1.33384630097488e-07
497 1.32856454632346e-07
498 1.32330822780524e-07
499 1.31806636836984e-07
500 1.31284090255868e-07
501 1.30763980748405e-07
502 1.30246829752601e-07
503 1.29730375683756e-07
504 1.29217228460377e-07
505 1.28704996510187e-07
506 1.28196225669619e-07
507 1.27689004731479e-07
508 1.2718482767049e-07
509 1.26682004887968e-07
510 1.26181783249635e-07
511 1.25683567053336e-07
512 1.25186702027769e-07
513 1.24691003541066e-07
514 1.24194272649092e-07
515 1.23698446605758e-07
516 1.23203019846585e-07
517 1.22715710661581e-07
518 1.22236882337035e-07
519 1.21762503003886e-07
520 1.21290019253095e-07
521 1.20818651061505e-07
522 1.20354686582225e-07
523 1.19893872003729e-07
524 1.1943725734298e-07
525 1.18981362119186e-07
526 1.18528570266285e-07
527 1.18077673145756e-07
528 1.17628993773877e-07
529 1.17181434576175e-07
530 1.16734565710708e-07
531 1.16285353078283e-07
532 1.15831272178291e-07
533 1.15375546053897e-07
534 1.14949878264881e-07
535 1.14533176847509e-07
536 1.14107333083702e-07
537 1.13684605390851e-07
538 1.13264269884006e-07
539 1.1284619255747e-07
540 1.12430727080337e-07
541 1.12017270327947e-07
542 1.11606848576251e-07
543 1.11198551459246e-07
544 1.10792170214147e-07
545 1.10389119472032e-07
546 1.09987413963175e-07
547 1.09588600465393e-07
548 1.09191490871297e-07
549 1.08797182371667e-07
550 1.08405259599209e-07
551 1.08015198761358e-07
552 1.07627680616673e-07
553 1.07241598081842e-07
554 1.06858484055117e-07
555 1.06477270250593e-07
556 1.06097972503605e-07
557 1.05720752290317e-07
558 1.05345368960563e-07
559 1.04972209482135e-07
560 1.04600414012168e-07
561 1.0423137826443e-07
562 1.03864460542447e-07
563 1.03499685367936e-07
564 1.03136966964179e-07
565 1.02775904555763e-07
566 1.02416667016492e-07
567 1.02059481801753e-07
568 1.01704257425617e-07
569 1.01350287618374e-07
570 1.00998063382818e-07
571 1.00648382500346e-07
572 1.00299927197511e-07
573 9.99539331028032e-08
574 9.96093717420265e-08
575 9.9268345787884e-08
576 9.89289892920908e-08
577 9.85915424767114e-08
578 9.82566088243431e-08
579 9.79234641804716e-08
580 9.75917273664528e-08
581 9.72622259438616e-08
582 9.69350184889528e-08
583 9.66088915079766e-08
584 9.62848850583065e-08
585 9.59625158984956e-08
586 9.56422093518761e-08
587 9.53232750662281e-08
588 9.50070964904626e-08
589 9.46927727203395e-08
590 9.43800146684382e-08
591 9.40692234423501e-08
592 9.37614309162171e-08
593 9.34556310547841e-08
594 9.31524472544965e-08
595 9.28498431829183e-08
596 9.25498069572761e-08
597 9.22526844608917e-08
598 9.19573504125637e-08
599 9.16632589582633e-08
600 9.13716600470238e-08
601 9.108194917129e-08
602 9.07931924101213e-08
603 9.05079032094136e-08
604 9.02237882698387e-08
605 8.99414453985337e-08
606 8.96604389470923e-08
607 8.93825706511642e-08
608 8.9104899848369e-08
609 8.88302947190667e-08
610 8.8557425407032e-08
611 8.82859328759089e-08
612 8.80163233873965e-08
613 8.77490679647153e-08
614 8.74827032575354e-08
615 8.72190286931307e-08
616 8.69561253526996e-08
617 8.66949550504259e-08
618 8.64363293722192e-08
619 8.61786819892352e-08
620 8.59232654213393e-08
621 8.56687729813288e-08
622 8.54160930021663e-08
623 8.51653868032543e-08
624 8.49162367928535e-08
625 8.46678727586259e-08
626 8.44219796762857e-08
627 8.41771226114396e-08
628 8.39338478471063e-08
629 8.36925922640397e-08
630 8.34526988873208e-08
631 8.32132910604244e-08
632 8.29762813694401e-08
633 8.27396381382073e-08
634 8.25054537645364e-08
635 8.22717540978068e-08
636 8.20401095062984e-08
637 8.18092323373598e-08
638 8.15798778193155e-08
639 8.13501416034867e-08
640 8.11231762569697e-08
641 8.08962111915612e-08
642 8.06701073607918e-08
643 8.04425890468252e-08
644 8.02152789005639e-08
645 7.99848919430524e-08
646 7.97521051341121e-08
647 7.95232860606454e-08
648 7.9299885154871e-08
649 7.90748397787588e-08
650 7.88442780814691e-08
651 7.86189288923822e-08
652 7.84105863083084e-08
653 7.82060815063268e-08
654 7.7998791526035e-08
655 7.77928346820112e-08
656 7.75885340318361e-08
657 7.73854382631889e-08
658 7.71825402199333e-08
659 7.69821353188149e-08
660 7.67818332985115e-08
661 7.6583953840359e-08
662 7.63863299892975e-08
663 7.61901965034006e-08
664 7.59947146025119e-08
665 7.58010361918693e-08
666 7.56084860271677e-08
667 7.54166281002888e-08
668 7.52256708782539e-08
669 7.50360361427838e-08
670 7.48481700099113e-08
671 7.4659674993871e-08
672 7.4474251547052e-08
673 7.42880370157018e-08
674 7.41044530596646e-08
675 7.39205680333477e-08
676 7.37374436843297e-08
677 7.35552307178367e-08
678 7.33743839900214e-08
679 7.31942774390149e-08
680 7.30148765026861e-08
681 7.28352497190166e-08
682 7.26576045999572e-08
683 7.24804955365421e-08
684 7.23026178994779e-08
685 7.21271193864226e-08
686 7.19515881191235e-08
687 7.17749864684869e-08
688 7.1600002068628e-08
689 7.14245343274023e-08
690 7.12497860626549e-08
691 7.10741321601205e-08
692 7.0899766929422e-08
693 7.07256892522601e-08
694 7.05551598891852e-08
695 7.03874036398489e-08
696 7.02198701461043e-08
697 7.00543032277068e-08
698 6.98887350898403e-08
699 6.97244399736618e-08
700 6.95601894062925e-08
701 6.93973425809347e-08
702 6.92351208555486e-08
703 6.90742251410192e-08
704 6.89131460895887e-08
705 6.8754103210722e-08
706 6.85943625371443e-08
707 6.84371024197716e-08
708 6.82789926531591e-08
709 6.81215539990454e-08
710 6.79655996229656e-08
711 6.78104609650454e-08
712 6.76551155414096e-08
713 6.75017252267374e-08
714 6.73484281716874e-08
715 6.71975721386886e-08
716 6.70449626793967e-08
717 6.68956768157258e-08
718 6.67455640441794e-08
719 6.6596373616612e-08
720 6.64466937241315e-08
721 6.62983401489292e-08
722 6.61497888896356e-08
723 6.60000767500613e-08
724 6.58507485091775e-08
725 6.57013144551577e-08
726 6.55519111019132e-08
727 6.54039533007911e-08
728 6.52576928090731e-08
729 6.51116981487121e-08
730 6.49680682611731e-08
731 6.48256109614387e-08
732 6.46842532017189e-08
733 6.45419998410723e-08
734 6.44014120472214e-08
735 6.42602540867898e-08
736 6.41173563025177e-08
737 6.39718171897563e-08
738 6.38165290274273e-08
739 6.36541439384963e-08
740 6.35174542691352e-08
741 6.3395585793824e-08
742 6.32650487641939e-08
743 6.31344177284277e-08
744 6.30045429965698e-08
745 6.28754417819621e-08
746 6.27462912565768e-08
747 6.26185647623068e-08
748 6.24901299128844e-08
749 6.2363542549182e-08
750 6.22375898946537e-08
751 6.21113655099315e-08
752 6.19860018580276e-08
753 6.18606964195578e-08
754 6.17367799038426e-08
755 6.16128218946521e-08
756 6.14903459585925e-08
757 6.13672723175362e-08
758 6.1244933050375e-08
759 6.11233507585851e-08
760 6.10021073996769e-08
761 6.08815818350372e-08
762 6.07612603622521e-08
763 6.06419458630825e-08
764 6.05219968856652e-08
765 6.04041066316618e-08
766 6.02850179975967e-08
767 6.01680605800681e-08
768 6.00501873608827e-08
769 5.99338462938803e-08
770 5.98168914618391e-08
771 5.97014466934276e-08
772 5.95864725818629e-08
773 5.94700580607288e-08
774 5.93566901687659e-08
775 5.92428580619142e-08
776 5.9128903034722e-08
777 5.90158158497722e-08
778 5.89039240752598e-08
779 5.87915749257206e-08
780 5.86811789990982e-08
781 5.85694494272815e-08
782 5.84589020156123e-08
783 5.83487354215428e-08
784 5.82398890842661e-08
785 5.81303057156646e-08
786 5.80217562622387e-08
787 5.7913421437128e-08
788 5.78065470309141e-08
789 5.76984978803985e-08
790 5.75921586780659e-08
791 5.74860301236768e-08
792 5.73804244798914e-08
793 5.72741698809942e-08
794 5.71706834513463e-08
795 5.70651127458355e-08
796 5.6961410829004e-08
797 5.68574359873786e-08
798 5.67545230318878e-08
799 5.6652280502778e-08
800 5.65491670241336e-08
801 5.64478893045894e-08
802 5.63457723519178e-08
803 5.62455163017894e-08
804 5.61444519724397e-08
805 5.60445270685328e-08
806 5.59447416832448e-08
807 5.58445142440078e-08
808 5.57461832184636e-08
809 5.56475811679391e-08
810 5.554955886522e-08
811 5.54511183414874e-08
812 5.53545884907081e-08
813 5.52565692895079e-08
814 5.51609370629258e-08
815 5.5064403705618e-08
816 5.49679192962671e-08
817 5.48734473211354e-08
818 5.47780923341179e-08
819 5.46831337389086e-08
820 5.45892322691977e-08
821 5.44944268487946e-08
822 5.44013181533742e-08
823 5.43076053589608e-08
824 5.42158178826746e-08
825 5.41220908583107e-08
826 5.40304361238419e-08
827 5.39387351357057e-08
828 5.38475973801411e-08
829 5.37557495854379e-08
830 5.36660751588869e-08
831 5.35741818588065e-08
832 5.34871109598711e-08
833 5.33942085603201e-08
834 5.33080520948026e-08
835 5.32161748862947e-08
836 5.31304745008399e-08
837 5.30385393759936e-08
838 5.29549602128476e-08
839 5.28627539138071e-08
840 5.27811531636502e-08
841 5.26874886661588e-08
842 5.26092542645351e-08
843 5.25132090620772e-08
844 5.24379689643695e-08
845 5.23389031887866e-08
846 5.22704054273238e-08
847 5.2165702059348e-08
848 5.21047457120449e-08
849 5.19912133241096e-08
850 5.19425157026188e-08
851 5.181690185041e-08
852 5.17834958595476e-08
853 5.16419673286528e-08
854 5.16255779940966e-08
855 5.14673508966723e-08
856 5.14695495688056e-08
857 5.12949348658864e-08
858 5.13121227980751e-08
859 5.11271897656318e-08
860 5.11539554690632e-08
861 5.0963630843448e-08
862 5.09963906529798e-08
863 5.08031038222168e-08
864 5.08383223321029e-08
865 5.06458373361607e-08
866 5.06814831975344e-08
867 5.04911526861829e-08
868 5.05257855680785e-08
869 5.03386471695233e-08
870 5.03724504983261e-08
871 5.0187841884819e-08
872 5.02188244866364e-08
873 5.00386859545454e-08
874 5.00660682964948e-08
875 4.98919429532396e-08
876 4.991492916262e-08
877 4.97463963171629e-08
878 4.9765185255124e-08
879 4.96020538811592e-08
880 4.96153395386223e-08
881 4.94598165758919e-08
882 4.94672567115106e-08
883 4.9318213607652e-08
884 4.9319979752882e-08
885 4.91789430852041e-08
886 4.91732147798185e-08
887 4.90397058359182e-08
888 4.90276717768356e-08
889 4.89030175310212e-08
890 4.88834662142335e-08
891 4.87665438171359e-08
892 4.87401403845844e-08
893 4.86320196926293e-08
894 4.85976597341953e-08
895 4.84975210759408e-08
896 4.84572781607184e-08
897 4.83638272270603e-08
898 4.83173787018121e-08
899 4.82314912495507e-08
900 4.81796820954727e-08
901 4.8098652510209e-08
902 4.80430680167565e-08
903 4.7966371821806e-08
904 4.79082717612656e-08
905 4.78349536203737e-08
906 4.77742538349535e-08
907 4.77042391202609e-08
908 4.76418837001002e-08
909 4.75742923560229e-08
910 4.75106121402558e-08
911 4.74442867079983e-08
912 4.73803675791196e-08
913 4.73161995833138e-08
914 4.72514346365749e-08
915 4.71875617866768e-08
916 4.71242941160188e-08
917 4.70600208759553e-08
918 4.69969111878132e-08
919 4.69337680730764e-08
920 4.6870963994472e-08
921 4.68089834104735e-08
922 4.67457260473481e-08
923 4.66836850219199e-08
924 4.6622122062967e-08
925 4.65601694097373e-08
926 4.64982714067919e-08
927 4.64368939356863e-08
928 4.63763013109908e-08
929 4.63146520799373e-08
930 4.62540211736417e-08
931 4.61932688682332e-08
932 4.61335446553246e-08
933 4.6073023221016e-08
934 4.60131941095732e-08
935 4.5953627911377e-08
936 4.58939187433227e-08
937 4.58345202278831e-08
938 4.57753630718205e-08
939 4.57163919493908e-08
940 4.56575366925005e-08
941 4.55994099604062e-08
942 4.55412367161845e-08
943 4.54831843268444e-08
944 4.54245441297019e-08
945 4.5367019303022e-08
946 4.53092023984247e-08
947 4.52529265573265e-08
948 4.51950916426913e-08
949 4.51376937102843e-08
950 4.50816519197428e-08
951 4.50249135892644e-08
952 4.49680519800655e-08
953 4.49120433581029e-08
954 4.48558285650602e-08
955 4.4800054096461e-08
956 4.47449207752193e-08
957 4.46888481555607e-08
958 4.46336262744929e-08
959 4.45785827891676e-08
960 4.45230591321621e-08
961 4.44689886018868e-08
962 4.44142280593418e-08
963 4.43597109074378e-08
964 4.43062488733048e-08
965 4.42507721067908e-08
966 4.41973315763455e-08
967 4.41433459394958e-08
968 4.40898167923809e-08
969 4.40369542551444e-08
970 4.39835507750619e-08
971 4.39305840373994e-08
972 4.38772599937742e-08
973 4.38248744758507e-08
974 4.37723909776366e-08
975 4.37199767735486e-08
976 4.36678064328344e-08
977 4.36151687859265e-08
978 4.3564099901694e-08
979 4.35120064943284e-08
980 4.34605361314677e-08
981 4.34089544174565e-08
982 4.33578468903573e-08
983 4.33070069441044e-08
984 4.32557199552264e-08
985 4.32049605243456e-08
986 4.31549700066203e-08
987 4.31043743696069e-08
988 4.30546081349181e-08
989 4.30033817819808e-08
990 4.29544000155602e-08
991 4.29044772967124e-08
992 4.28541212824651e-08
993 4.28051534120399e-08
994 4.27560843929697e-08
995 4.27069646795619e-08
996 4.26577118664007e-08
997 4.26090705758497e-08
998 4.25599185458481e-08
999 4.25119207780877e-08
1000 4.24638359601826e-08
1001 4.24153413554063e-08
1002 4.23669140765526e-08
1003 4.23192931462424e-08
1004 4.22719421036e-08
1005 4.22237158639316e-08
1006 4.21767394409578e-08
1007 4.21293841963433e-08
1008 4.20821377224989e-08
1009 4.20355602794853e-08
1010 4.19879502497889e-08
1011 4.19416763615121e-08
1012 4.1894891939398e-08
1013 4.18487484361574e-08
1014 4.18028833497619e-08
1015 4.17560911107895e-08
1016 4.17109000965254e-08
1017 4.16648426575517e-08
1018 4.1618916763797e-08
1019 4.15733806318208e-08
1020 4.15281082291052e-08
1021 4.14830878301409e-08
1022 4.14384049609584e-08
1023 4.13927768148081e-08
1024 4.13485286208104e-08
1025 4.1303422470218e-08
1026 4.1258878843875e-08
1027 4.12150146809065e-08
1028 4.11701768976425e-08
1029 4.11263026633968e-08
1030 4.10825128365211e-08
1031 4.10391674519062e-08
1032 4.09945505350784e-08
1033 4.09519170028183e-08
1034 4.09084967012419e-08
1035 4.08650029981583e-08
1036 4.08217910512487e-08
1037 4.07793272798607e-08
1038 4.07365234234014e-08
1039 4.0693789721935e-08
1040 4.06507034464987e-08
1041 4.06090082472055e-08
1042 4.05665589957582e-08
1043 4.05246946519888e-08
1044 4.04823947994792e-08
1045 4.04408930669842e-08
1046 4.03981976453416e-08
1047 4.03578708765284e-08
1048 4.03157993658088e-08
1049 4.02747506906831e-08
1050 4.02337493656812e-08
1051 4.01925506894329e-08
1052 4.01516144818981e-08
1053 4.01110262597815e-08
1054 4.00705578753602e-08
1055 4.0030026728699e-08
1056 3.99900804950448e-08
1057 3.99490832363458e-08
1058 3.9909818885242e-08
1059 3.98692488541297e-08
1060 3.98294636096974e-08
1061 3.97896840214074e-08
1062 3.97502910525205e-08
1063 3.97110191934225e-08
1064 3.96713200017107e-08
1065 3.96326765883526e-08
1066 3.95934312740209e-08
1067 3.95548202694052e-08
1068 3.95159515618193e-08
1069 3.94768951739177e-08
1070 3.94376635410865e-08
1071 3.94003437107493e-08
1072 3.93615893581334e-08
1073 3.93234357325412e-08
1074 3.92852971482505e-08
1075 3.92470344658946e-08
1076 3.92092457972559e-08
1077 3.91721733472572e-08
1078 3.91343785115517e-08
1079 3.90966476124799e-08
1080 3.9059586267598e-08
1081 3.90224841699816e-08
1082 3.89853192708234e-08
1083 3.89487054117676e-08
1084 3.89114733598817e-08
1085 3.8875248638659e-08
1086 3.88379446125686e-08
1087 3.88010305112374e-08
1088 3.87657544653752e-08
1089 3.8729323218023e-08
1090 3.86934236862313e-08
1091 3.86571627297716e-08
1092 3.86203069147051e-08
1093 3.85854523516915e-08
1094 3.85494564349997e-08
1095 3.85134470792803e-08
1096 3.84787864233793e-08
1097 3.84429535882358e-08
1098 3.8407954248143e-08
1099 3.83723052361784e-08
1100 3.83381536614991e-08
1101 3.83025280186189e-08
1102 3.82680123176549e-08
1103 3.82327426342588e-08
1104 3.81989031654761e-08
1105 3.8164004359853e-08
1106 3.81306543733029e-08
1107 3.80950833724913e-08
1108 3.8061194767236e-08
1109 3.80274299609518e-08
1110 3.79933653038567e-08
1111 3.79593403196932e-08
1112 3.79260635183787e-08
1113 3.78925312860545e-08
1114 3.78583058362469e-08
1115 3.78255653299409e-08
1116 3.77916498766062e-08
1117 3.7759242245361e-08
1118 3.77259271535646e-08
1119 3.76929184058294e-08
1120 3.76596022289011e-08
1121 3.762698212606e-08
1122 3.75944686428209e-08
1123 3.75615465084689e-08
1124 3.75297574806144e-08
1125 3.7496433816564e-08
1126 3.74644436307303e-08
1127 3.74326148449011e-08
1128 3.73999696623439e-08
1129 3.73687081780538e-08
1130 3.73364589789649e-08
1131 3.73047839443696e-08
1132 3.72732365367945e-08
1133 3.72414399596455e-08
1134 3.72102340140401e-08
1135 3.71782816335209e-08
1136 3.71469434483629e-08
1137 3.71158673753147e-08
1138 3.70848970174809e-08
1139 3.70540735228353e-08
1140 3.70226382027106e-08
1141 3.69923759155899e-08
1142 3.69611201247455e-08
1143 3.69300279738916e-08
1144 3.68998086306416e-08
1145 3.68699720034726e-08
1146 3.68393809284662e-08
1147 3.6808156525181e-08
1148 3.67792302082837e-08
1149 3.67484061944978e-08
1150 3.67180815474821e-08
1151 3.66886903244712e-08
1152 3.66583005682042e-08
1153 3.66287651147879e-08
1154 3.6599686666472e-08
1155 3.65693823127788e-08
1156 3.6540550222286e-08
1157 3.6510458441219e-08
1158 3.64814909821209e-08
1159 3.64517278870213e-08
1160 3.64231963656536e-08
1161 3.63937222378485e-08
1162 3.63647008723156e-08
1163 3.63361309316801e-08
1164 3.63070829818568e-08
1165 3.62782414202734e-08
1166 3.62499479882317e-08
1167 3.62211542850321e-08
1168 3.61929274474981e-08
1169 3.61639565629179e-08
1170 3.6135953866534e-08
1171 3.61074582093668e-08
1172 3.60796494309756e-08
1173 3.60510075396725e-08
1174 3.60238114647249e-08
1175 3.59951680337645e-08
1176 3.59671647320869e-08
1177 3.5939846639188e-08
1178 3.59119564141697e-08
1179 3.58847129786621e-08
1180 3.58570923835178e-08
1181 3.58290298225761e-08
1182 3.5802073337976e-08
1183 3.57746675350157e-08
1184 3.57474318548601e-08
1185 3.5720507786996e-08
1186 3.56925745863546e-08
1187 3.56669186158953e-08
1188 3.56388308557776e-08
1189 3.56120964264406e-08
1190 3.55855007228012e-08
1191 3.55595634766726e-08
1192 3.55318447839981e-08
1193 3.55058935364028e-08
1194 3.54784684635234e-08
1195 3.54532749036185e-08
1196 3.54252035996705e-08
1197 3.53999379747449e-08
1198 3.53744113232857e-08
1199 3.5347758216675e-08
1200 3.53211740269366e-08
1201 3.5295273275171e-08
1202 3.52689877312073e-08
1203 3.52439058601828e-08
1204 3.52178675853132e-08
1205 3.51920423529162e-08
1206 3.51662255406726e-08
1207 3.51408375867113e-08
1208 3.51154320770153e-08
1209 3.50897882779311e-08
1210 3.50647007287819e-08
1211 3.50387830854171e-08
1212 3.50134890600984e-08
1213 3.49890919808526e-08
1214 3.49634856930958e-08
1215 3.4937656489431e-08
1216 3.49134822521524e-08
1217 3.48886283081384e-08
1218 3.48632871798316e-08
1219 3.48383421540088e-08
1220 3.48141471231411e-08
1221 3.47883301343721e-08
1222 3.47644468641217e-08
1223 3.47392556738768e-08
1224 3.47156399218562e-08
1225 3.46905646724238e-08
1226 3.46663979429174e-08
1227 3.46420266470471e-08
1228 3.46176367638229e-08
1229 3.45936271244351e-08
1230 3.45696759127545e-08
1231 3.45444061851108e-08
1232 3.45210467591617e-08
1233 3.44975583081997e-08
1234 3.44735614747904e-08
1235 3.44493592037143e-08
1236 3.44254053965543e-08
1237 3.4401373557591e-08
1238 3.43778571234576e-08
1239 3.4354641828882e-08
1240 3.43305521002257e-08
1241 3.43076815896293e-08
1242 3.42833551441046e-08
1243 3.42604188705575e-08
1244 3.42372240074162e-08
1245 3.42127958072869e-08
1246 3.41901196450056e-08
1247 3.4165980303813e-08
1248 3.41434288939624e-08
1249 3.4120290277162e-08
1250 3.40969683445547e-08
1251 3.4073855900596e-08
1252 3.40508015812535e-08
1253 3.4027907200862e-08
1254 3.40050239258094e-08
1255 3.39827827520178e-08
1256 3.39595499769807e-08
1257 3.39359237344628e-08
1258 3.39145543530872e-08
1259 3.38909124049103e-08
1260 3.3867833027168e-08
1261 3.38457257706981e-08
1262 3.38228901344273e-08
1263 3.38004212490084e-08
1264 3.3777936396584e-08
1265 3.37557851279868e-08
1266 3.37325825989776e-08
1267 3.37107733217046e-08
1268 3.36880202138623e-08
1269 3.36660469555383e-08
1270 3.36428227611929e-08
1271 3.36216010141666e-08
1272 3.35990930353525e-08
1273 3.35766572701068e-08
1274 3.35548075105496e-08
1275 3.35321843123459e-08
1276 3.35112284346462e-08
1277 3.34880336183563e-08
1278 3.34660686112098e-08
1279 3.34440656257762e-08
1280 3.34222622775382e-08
1281 3.3400496676661e-08
1282 3.33780708492437e-08
1283 3.33568613306579e-08
1284 3.33346162706416e-08
1285 3.33126923470761e-08
1286 3.32911523537316e-08
1287 3.32686337336519e-08
1288 3.32477538411702e-08
1289 3.32256273272158e-08
1290 3.32038384069921e-08
1291 3.31828541717538e-08
1292 3.31606124712724e-08
1293 3.31397496824426e-08
1294 3.31185089783581e-08
1295 3.30965764474556e-08
1296 3.30749869918989e-08
1297 3.30537093014005e-08
1298 3.30325635162776e-08
1299 3.30114373836565e-08
1300 3.2989555056373e-08
1301 3.29691123412967e-08
1302 3.29480978047414e-08
1303 3.29259499360912e-08
1304 3.29069987901054e-08
1305 3.28838436096657e-08
1306 3.28625021952522e-08
1307 3.2842557148971e-08
1308 3.28219823129139e-08
1309 3.28004262262205e-08
1310 3.2779487927348e-08
1311 3.27594200280057e-08
1312 3.27379145372841e-08
1313 3.27168738167849e-08
1314 3.26964428927212e-08
1315 3.26758218729406e-08
1316 3.26544716027222e-08
1317 3.26346522436793e-08
1318 3.26134011285895e-08
1319 3.25931276172753e-08
1320 3.25723668779254e-08
1321 3.25519313193467e-08
1322 3.25313439464292e-08
1323 3.25115984738389e-08
1324 3.24906244300038e-08
1325 3.247033603615e-08
1326 3.24494723462632e-08
1327 3.2429574994941e-08
1328 3.24087358181124e-08
1329 3.23886687765285e-08
1330 3.23682455984908e-08
1331 3.23480375705731e-08
1332 3.2327858273673e-08
1333 3.2307160744649e-08
1334 3.2288030005212e-08
1335 3.22668284071792e-08
1336 3.22475789735765e-08
1337 3.22272875863394e-08
1338 3.22063714877086e-08
1339 3.21881238520838e-08
1340 3.2166476620743e-08
1341 3.21467919879304e-08
1342 3.21269791370149e-08
1343 3.21068357125309e-08
1344 3.20875228740913e-08
1345 3.20675762652822e-08
1346 3.20471151131763e-08
1347 3.2027775425103e-08
1348 3.2007873280504e-08
1349 3.19887264506047e-08
1350 3.19672563324769e-08
1351 3.19485104915973e-08
1352 3.19292231276691e-08
1353 3.19095986596185e-08
1354 3.18888235579795e-08
1355 3.18698087320435e-08
1356 3.1849482726054e-08
1357 3.18300938841887e-08
1358 3.18109970876002e-08
1359 3.1791071898768e-08
1360 3.1771604117603e-08
1361 3.17522525985492e-08
1362 3.17313117530471e-08
1363 3.17133023586269e-08
1364 3.16930914701086e-08
1365 3.16739978380998e-08
1366 3.1654503768852e-08
1367 3.16346231956643e-08
1368 3.16152571857575e-08
1369 3.15957956211754e-08
1370 3.15766907079684e-08
1371 3.15575369418397e-08
1372 3.1537720201591e-08
1373 3.15184250418987e-08
1374 3.14988311438391e-08
1375 3.14791483138066e-08
1376 3.1460161456609e-08
1377 3.14419114773035e-08
1378 3.14211735381598e-08
1379 3.14024833796278e-08
1380 3.13841668171388e-08
1381 3.13642475355813e-08
1382 3.13452486213617e-08
1383 3.13250223302752e-08
1384 3.13062454517787e-08
1385 3.12875881913755e-08
1386 3.12677945084605e-08
1387 3.12486534399525e-08
1388 3.12295657196593e-08
1389 3.12100121593328e-08
1390 3.11909057915116e-08
1391 3.11729260820126e-08
1392 3.11536782631183e-08
1393 3.11335500773335e-08
1394 3.11151998733106e-08
1395 3.10954964939381e-08
1396 3.10763195341401e-08
1397 3.10578546758311e-08
1398 3.10382693036182e-08
1399 3.10196349369019e-08
1400 3.10011698658741e-08
1401 3.0981225566995e-08
1402 3.09627692471892e-08
1403 3.0943027488739e-08
1404 3.09245523146817e-08
1405 3.09060063987676e-08
1406 3.08859032409625e-08
1407 3.08666235402377e-08
1408 3.08481175581576e-08
1409 3.0829102281027e-08
1410 3.08105085240484e-08
1411 3.07905240668482e-08
1412 3.0772272580748e-08
1413 3.0753320463317e-08
1414 3.07338509177857e-08
1415 3.07160048127475e-08
1416 3.06960767830766e-08
1417 3.06780550745067e-08
1418 3.06583731637389e-08
1419 3.06408354351362e-08
1420 3.06212899110481e-08
1421 3.06027603909076e-08
1422 3.05832419238428e-08
1423 3.05650213283659e-08
1424 3.05471764963094e-08
1425 3.05272334384377e-08
1426 3.05090074632641e-08
1427 3.0491127587462e-08
1428 3.04713421075409e-08
1429 3.04540138988951e-08
1430 3.04344762873665e-08
1431 3.04157823789453e-08
1432 3.03974920177197e-08
1433 3.0378841749501e-08
1434 3.03593514738854e-08
1435 3.03426683969654e-08
1436 3.03229727167675e-08
1437 3.0304468231046e-08
1438 3.02867868933188e-08
1439 3.02679857866472e-08
1440 3.02492281283318e-08
1441 3.02310039934639e-08
1442 3.02119886919083e-08
1443 3.01943036224994e-08
1444 3.01751457851829e-08
1445 3.0156794501357e-08
1446 3.01387282006793e-08
1447 3.01199302157329e-08
1448 3.01023974598191e-08
1449 3.00834239970182e-08
1450 3.00647281847866e-08
1451 3.00458121513802e-08
1452 3.00285106507037e-08
1453 3.00096546383966e-08
1454 2.99906652583282e-08
1455 2.99723595353907e-08
1456 2.99540399488762e-08
1457 2.99355549171487e-08
1458 2.99179630134372e-08
1459 2.98988014384438e-08
1460 2.98807815610758e-08
1461 2.98626967405458e-08
1462 2.98433059249259e-08
1463 2.98255218909027e-08
1464 2.98073929192455e-08
1465 2.97886290669958e-08
1466 2.97704149403444e-08
1467 2.97520646330707e-08
1468 2.97344999100613e-08
1469 2.97149172856503e-08
1470 2.96966790942488e-08
1471 2.96780276889397e-08
1472 2.96608193852688e-08
1473 2.96423715067107e-08
1474 2.96237175683167e-08
1475 2.96049823143019e-08
1476 2.95868160249402e-08
1477 2.95691333163095e-08
1478 2.95499615026174e-08
1479 2.95316940503998e-08
1480 2.9513504468337e-08
1481 2.94951920718489e-08
1482 2.9476856642674e-08
1483 2.94581650162051e-08
1484 2.94393227409095e-08
1485 2.9422570690274e-08
1486 2.94026726423979e-08
1487 2.93851123651656e-08
1488 2.93667496267247e-08
1489 2.93491650780187e-08
1490 2.93300409621722e-08
1491 2.93117718568325e-08
1492 2.92941688784243e-08
1493 2.92751093737831e-08
1494 2.92557991656839e-08
1495 2.92385237656845e-08
1496 2.92201997618147e-08
1497 2.92024531893809e-08
1498 2.91836127583878e-08
1499 2.9165412369192e-08
1500 2.91477156333375e-08
1501 2.91295078009846e-08
1502 2.91112184640507e-08
1503 2.90930324486904e-08
1504 2.90748828875032e-08
1505 2.90573428762819e-08
1506 2.90395605646587e-08
1507 2.90210812101677e-08
1508 2.90038560488703e-08
1509 2.89855333486244e-08
1510 2.89670353237348e-08
1511 2.89494639087451e-08
1512 2.89321546316224e-08
1513 2.89140921794218e-08
1514 2.88961061993831e-08
1515 2.88784363304817e-08
1516 2.88605872003078e-08
1517 2.88427506343059e-08
1518 2.88255827778361e-08
1519 2.88077940862053e-08
1520 2.87899787225765e-08
1521 2.87721438385624e-08
1522 2.87551624440496e-08
1523 2.8736553419284e-08
1524 2.8719940153632e-08
1525 2.87021816534061e-08
1526 2.86847415416069e-08
1527 2.86677928842405e-08
1528 2.86504343434135e-08
1529 2.86334675403399e-08
1530 2.86150446651146e-08
1531 2.85990137476322e-08
1532 2.85798742241106e-08
1533 2.85654919898182e-08
1534 2.85448502244101e-08
1535 2.85328611220326e-08
1536 2.85086630691378e-08
1537 2.8503431211746e-08
1538 2.84744364715817e-08
1539 2.8471757706372e-08
1540 2.84398793966023e-08
1541 2.84376399670716e-08
1542 2.84029466872671e-08
1543 2.84092086419996e-08
1544 2.83706312591558e-08
1545 2.83739691944174e-08
1546 2.83318062419546e-08
1547 2.83482883205455e-08
1548 2.83009349075769e-08
1549 2.83114081609526e-08
1550 2.8261475730762e-08
1551 2.82872523609257e-08
1552 2.82315169832792e-08
1553 2.8250846230149e-08
1554 2.81929186443097e-08
1555 2.82238405000523e-08
1556 2.8161298246232e-08
1557 2.81913826081226e-08
1558 2.81238224943969e-08
1559 2.81618087916158e-08
1560 2.80886102943523e-08
1561 2.81315516439395e-08
1562 2.80523670810329e-08
1563 2.81004023381382e-08
1564 2.80169122195417e-08
1565 2.80651120527775e-08
1566 2.79844124733142e-08
1567 2.80354515989067e-08
1568 2.79496075843433e-08
1569 2.8004040482088e-08
1570 2.79198149739646e-08
1571 2.79772743363438e-08
1572 2.78814672978189e-08
1573 2.79369472688806e-08
1574 2.78783418554696e-08
1575 2.78722878706006e-08
1576 2.78802553728141e-08
1577 2.78241476654451e-08
1578 2.78531429853679e-08
1579 2.77871881231295e-08
1580 2.78346230890847e-08
1581 2.77371837549278e-08
1582 2.77916115296595e-08
1583 2.77399092498687e-08
1584 2.77199701179676e-08
1585 2.77344857282369e-08
1586 2.76978281119611e-08
1587 2.76861653241145e-08
1588 2.76739711517759e-08
1589 2.76406291201958e-08
1590 2.76693590144372e-08
1591 2.75878937956442e-08
1592 2.76385192177209e-08
1593 2.75794396276119e-08
1594 2.75627759283026e-08
1595 2.75668321605416e-08
1596 2.75558552906841e-08
1597 2.75114846335356e-08
1598 2.75420857924757e-08
1599 2.74502718087266e-08
1600 2.75004017149527e-08
1601 2.74270197031523e-08
1602 2.74815034155829e-08
1603 2.73956593352942e-08
1604 2.74443683785375e-08
1605 2.73762672358124e-08
1606 2.74237509496622e-08
1607 2.73384402957877e-08
1608 2.738941653746e-08
1609 2.73109253214887e-08
1610 2.73599989246698e-08
1611 2.72915100667515e-08
1612 2.73048680106136e-08
1613 2.72685380802162e-08
1614 2.72831824554842e-08
1615 2.72405448975555e-08
1616 2.72411886566015e-08
1617 2.72163602699216e-08
1618 2.72448574103468e-08
1619 2.72078903711392e-08
1620 2.71331257457597e-08
1621 2.72338773454894e-08
1622 2.70687423897797e-08
1623 2.7232598497795e-08
1624 2.70352418858533e-08
1625 2.71780600391569e-08
1626 2.70878941404096e-08
1627 2.70299327815859e-08
1628 2.72005166392919e-08
1629 2.69563791923666e-08
1630 2.71082335685513e-08
1631 2.70248807459073e-08
1632 2.69544743802808e-08
1633 2.71588247380272e-08
1634 2.68636631290953e-08
1635 2.70261003076921e-08
1636 2.69857782932981e-08
1637 2.68727524446177e-08
1638 2.70615695256993e-08
1639 2.68179220554998e-08
1640 2.69593559514281e-08
1641 2.6851251458071e-08
1642 2.69067703264625e-08
1643 2.6849774118709e-08
1644 2.68532007186018e-08
1645 2.67810174858774e-08
1646 2.69304431730344e-08
1647 2.67131477065252e-08
1648 2.68735154214106e-08
1649 2.67778765457027e-08
1650 2.67049922761942e-08
1651 2.69052224146638e-08
1652 2.66295231894365e-08
1653 2.68216380359121e-08
1654 2.66509251443114e-08
1655 2.66998429763188e-08
1656 2.6695507691965e-08
1657 2.66778571977877e-08
1658 2.66696693389523e-08
1659 2.65654105486846e-08
1660 2.68224941000206e-08
1661 2.64926783115627e-08
1662 2.67036204257831e-08
1663 2.64982755973087e-08
1664 2.66594346618465e-08
1665 2.6476402981368e-08
1666 2.67520736076943e-08
1667 2.63855381750222e-08
1668 2.6619654329485e-08
1669 2.65007827755515e-08
1670 2.64435347454572e-08
1671 2.65941606683029e-08
1672 2.63714779176194e-08
1673 2.66196461218282e-08
1674 2.63279748855538e-08
1675 2.65143168072246e-08
1676 2.64342524989303e-08
1677 2.63499696668834e-08
1678 2.64617089775765e-08
1679 2.63312725559128e-08
1680 2.63874530401775e-08
1681 2.63021033128918e-08
1682 2.64659690920599e-08
1683 2.62263972183874e-08
1684 2.64027071141815e-08
1685 2.62678710265529e-08
1686 2.62855399526529e-08
1687 2.64704681742778e-08
1688 2.61312182234796e-08
1689 2.64372413585345e-08
1690 2.61167165429566e-08
1691 2.62745315187107e-08
1692 2.63552781705201e-08
1693 2.61686635238956e-08
1694 2.63244626106385e-08
1695 2.60788960382286e-08
1696 2.63592162248472e-08
1697 2.602089657322e-08
1698 2.61818120830259e-08
1699 2.6278978761507e-08
1700 2.60884683398466e-08
1701 2.61696131074185e-08
1702 2.60505593665528e-08
1703 2.60706995811599e-08
1704 2.62337403651447e-08
1705 2.59625556051635e-08
1706 2.62782771327519e-08
1707 2.58626834905851e-08
1708 2.60746833480319e-08
1709 2.61163358121763e-08
1710 2.59923882257596e-08
1711 2.61370301295472e-08
1712 2.58823056613622e-08
1713 2.61984318894459e-08
1714 2.57697144792068e-08
1715 2.60541698715588e-08
1716 2.58104008767912e-08
1717 2.61534841239364e-08
1718 2.57343917147157e-08
1719 2.5980354928512e-08
1720 2.58106123209867e-08
1721 2.59027715638904e-08
1722 2.58855993993112e-08
1723 2.58331370677212e-08
1724 2.60466672354909e-08
1725 2.5648251088306e-08
1726 2.59636449222445e-08
1727 2.5688613499053e-08
1728 2.58844434708472e-08
1729 2.56516415584329e-08
1730 2.59527993475839e-08
1731 2.5621062597958e-08
1732 2.58924282463546e-08
1733 2.55998183389394e-08
1734 2.59006712632903e-08
1735 2.55635776535978e-08
1736 2.58793645533384e-08
1737 2.55395494512811e-08
1738 2.58599348743616e-08
1739 2.55158007493872e-08
1740 2.58410802305153e-08
1741 2.54899587082136e-08
1742 2.58209358579009e-08
1743 2.54670331438778e-08
1744 2.57975656643428e-08
1745 2.54456769206435e-08
1746 2.57775053567055e-08
1747 2.54200451845765e-08
1748 2.57605040066e-08
1749 2.53946068757838e-08
1750 2.57398509466711e-08
1751 2.53727596850606e-08
1752 2.57189166366079e-08
1753 2.53494960957834e-08
1754 2.56976451775692e-08
1755 2.5327358580185e-08
1756 2.56794661532833e-08
1757 2.53012065465352e-08
1758 2.56605510391417e-08
1759 2.5279557481328e-08
1760 2.56389928594558e-08
1761 2.52559166833422e-08
1762 2.56206824944538e-08
1763 2.52330628260911e-08
1764 2.56010590462541e-08
1765 2.52094135033687e-08
1766 2.55806691924487e-08
1767 2.51888642355258e-08
1768 2.55602891390261e-08
1769 2.51649583733471e-08
1770 2.55422882207146e-08
1771 2.5143193475019e-08
1772 2.55224662848441e-08
1773 2.51190502997822e-08
1774 2.55040823824437e-08
1775 2.50974995366082e-08
1776 2.54842045721571e-08
1777 2.50748092385233e-08
1778 2.5466267260299e-08
1779 2.50509867973925e-08
1780 2.54486320983016e-08
1781 2.50284555858293e-08
1782 2.54302218496427e-08
1783 2.50047126570951e-08
1784 2.5411599816616e-08
1785 2.49844843753788e-08
1786 2.5393082076608e-08
1787 2.49595401196068e-08
1788 2.53762151856129e-08
1789 2.49386683177644e-08
1790 2.53562801033613e-08
1791 2.49164188323991e-08
1792 2.53391929683211e-08
1793 2.48945892467045e-08
1794 2.53215705043885e-08
1795 2.48714184454268e-08
1796 2.53035337429708e-08
1797 2.48503345479101e-08
1798 2.52862419317657e-08
1799 2.48269675615642e-08
1800 2.52689261298622e-08
1801 2.48078273079599e-08
1802 2.52495766546179e-08
1803 2.47852348036481e-08
1804 2.52331509102977e-08
1805 2.47649927049842e-08
1806 2.52131304057102e-08
1807 2.47440714531422e-08
1808 2.51981799360701e-08
1809 2.4722056149562e-08
1810 2.51820256635948e-08
1811 2.47015887282487e-08
1812 2.51656365315434e-08
1813 2.4681948031624e-08
1814 2.51463732557911e-08
1815 2.46619773756063e-08
1816 2.51313927669417e-08
1817 2.46418024272277e-08
1818 2.51180386644512e-08
1819 2.46240043610957e-08
1820 2.51096200568623e-08
1821 2.46105544134956e-08
1822 2.51164681881466e-08
1823 2.46223887641062e-08
1824 2.50097634224566e-08
1825 2.45928359414727e-08
1826 2.51460108383617e-08
1827 2.4649486370043e-08
1828 2.47151307151139e-08
1829 2.48896657639719e-08
1830 2.45930150533091e-08
1831 2.49031563626101e-08
1832 2.45679663808396e-08
1833 2.49377088479275e-08
1834 2.45344193703367e-08
1835 2.50291840413475e-08
1836 2.45797278002335e-08
1837 2.4691318643244e-08
1838 2.48152400614821e-08
1839 2.45531549438738e-08
1840 2.46972517263799e-08
1841 2.47099615673374e-08
1842 2.46394221812274e-08
1843 2.47075493466919e-08
1844 2.4608942156501e-08
1845 2.46913099577473e-08
1846 2.45908068658984e-08
1847 2.46763817817186e-08
1848 2.45743490925321e-08
1849 2.46622214898906e-08
1850 2.45579211521907e-08
1851 2.46450770449158e-08
1852 2.45406851440144e-08
1853 2.46293459780222e-08
1854 2.45233685101098e-08
1855 2.46142588475884e-08
1856 2.45044254312266e-08
1857 2.45976092334566e-08
1858 2.4488290730762e-08
1859 2.4581095515952e-08
1860 2.44703143379432e-08
1861 2.45646891083862e-08
1862 2.44521059213909e-08
1863 2.45485935543854e-08
1864 2.44336861370265e-08
1865 2.4532604029126e-08
1866 2.44163247167251e-08
1867 2.45164303662726e-08
1868 2.43978265379585e-08
1869 2.44990144495638e-08
1870 2.43793505489975e-08
1871 2.44827551296733e-08
1872 2.43628119218631e-08
1873 2.44665467095118e-08
1874 2.43427037645016e-08
1875 2.44505962034847e-08
1876 2.43257625316406e-08
1877 2.44331060723635e-08
1878 2.43078538826325e-08
1879 2.44173870915798e-08
1880 2.42899540427999e-08
1881 2.44018315262995e-08
1882 2.4269980444025e-08
1883 2.43843473851646e-08
1884 2.42524924964904e-08
1885 2.43676653121083e-08
1886 2.42350774701805e-08
1887 2.43521974808703e-08
1888 2.42167727024789e-08
1889 2.43354309591437e-08
1890 2.41994723120254e-08
1891 2.43200188160264e-08
1892 2.4179794092527e-08
1893 2.4302577244284e-08
1894 2.41619382073122e-08
1895 2.42878967076088e-08
1896 2.41444024959847e-08
1897 2.42705576779523e-08
1898 2.41255771422644e-08
1899 2.42551340731145e-08
1900 2.41080463274646e-08
1901 2.4237427823981e-08
1902 2.40898285017721e-08
1903 2.42242300679818e-08
1904 2.40713446695295e-08
1905 2.42056250501221e-08
1906 2.40539029885412e-08
1907 2.41930468539708e-08
1908 2.40363772630037e-08
1909 2.41728644180261e-08
1910 2.40198484935394e-08
1911 2.41619835966711e-08
1912 2.40039192294095e-08
1913 2.41418719041375e-08
1914 2.39870812928089e-08
1915 2.41352440831788e-08
1916 2.39690908780954e-08
1917 2.41156140307064e-08
1918 2.3952949463002e-08
1919 2.41056024649078e-08
1920 2.39204692868977e-08
1921 2.41220614842774e-08
1922 2.39180379595361e-08
1923 2.40338039902976e-08
1924 2.39020813450619e-08
1925 2.40636234097868e-08
1926 2.38241284187257e-08
1927 2.41883810689902e-08
1928 2.39014039686847e-08
1929 2.39162022779293e-08
1930 2.39476947831996e-08
1931 2.39549839251563e-08
1932 2.39413944609357e-08
1933 2.39192667490507e-08
1934 2.39108200945637e-08
1935 2.39014974059426e-08
1936 2.38946279640384e-08
1937 2.3885836782922e-08
1938 2.38769054559107e-08
1939 2.38683466065037e-08
1940 2.38619487242797e-08
1941 2.38534221344011e-08
1942 2.3843072010532e-08
1943 2.38354628423476e-08
1944 2.38295068408778e-08
1945 2.38186110057548e-08
1946 2.3812179015481e-08
1947 2.38031119759174e-08
1948 2.37954687060116e-08
1949 2.37866041958856e-08
1950 2.37783847109041e-08
1951 2.37704629836299e-08
1952 2.37618662072281e-08
1953 2.37551422613835e-08
1954 2.37467912029832e-08
1955 2.37360681437648e-08
1956 2.37302786357807e-08
1957 2.37219062185767e-08
1958 2.37137723085867e-08
1959 2.37051596286841e-08
1960 2.36974928420342e-08
1961 2.36900103431914e-08
1962 2.36809016526118e-08
1963 2.36726460211001e-08
1964 2.3666226280139e-08
1965 2.36554101996633e-08
1966 2.36502760379009e-08
1967 2.36408188528658e-08
1968 2.36330478532443e-08
1969 2.362489023644e-08
1970 2.36182834795207e-08
1971 2.36089736181944e-08
1972 2.36021777872963e-08
1973 2.35936539693116e-08
1974 2.35836080084706e-08
1975 2.35784570855602e-08
1976 2.35682205206311e-08
1977 2.35615062029737e-08
1978 2.35543869780974e-08
1979 2.35454227092147e-08
1980 2.35384669897698e-08
1981 2.35300944922967e-08
1982 2.35206700713864e-08
1983 2.35143910788382e-08
1984 2.35055819340912e-08
1985 2.3498775658326e-08
1986 2.34893337055508e-08
1987 2.34836132382288e-08
1988 2.34733361721418e-08
1989 2.3467782968245e-08
1990 2.34593475217926e-08
1991 2.34511516802316e-08
1992 2.34441271963481e-08
1993 2.34352264244508e-08
1994 2.3427440235646e-08
1995 2.34209353477954e-08
1996 2.34123358674454e-08
1997 2.34043026129394e-08
1998 2.33970418507257e-08
1999 2.33887423960466e-08
2000 2.33818435728672e-08
2001 2.33735276169433e-08
2002 2.3366241443723e-08
2003 2.33583957548467e-08
2004 2.33494243550014e-08
2005 2.33440131902629e-08
2006 2.33353320567842e-08
2007 2.33264657615306e-08
2008 2.33212889709788e-08
2009 2.33121358520494e-08
2010 2.33038694128496e-08
2011 2.3298055525478e-08
2012 2.32884727919824e-08
2013 2.32823354657796e-08
2014 2.32737858606447e-08
2015 2.32658584971013e-08
2016 2.32586878287089e-08
2017 2.32505139869943e-08
2018 2.32446895686023e-08
2019 2.3237160243661e-08
2020 2.32264632694656e-08
2021 2.32218953791774e-08
2022 2.32133122899381e-08
2023 2.32060175037185e-08
2024 2.3198230664434e-08
2025 2.31909507254491e-08
2026 2.31831495345336e-08
2027 2.31754111866822e-08
2028 2.31685561593586e-08
2029 2.31612527972214e-08
2030 2.31527821581423e-08
2031 2.31465282924948e-08
2032 2.31374864747558e-08
2033 2.31332264781781e-08
2034 2.31212739393305e-08
2035 2.31170367291922e-08
2036 2.31078574433052e-08
2037 2.31016844711718e-08
2038 2.30939941736619e-08
2039 2.3084983529098e-08
2040 2.30802306622202e-08
2041 2.30709032806864e-08
2042 2.30634513173067e-08
2043 2.30571183044148e-08
2044 2.30491163057955e-08
2045 2.30410147566973e-08
2046 2.30363884323737e-08
2047 2.30270775958274e-08
2048 2.3017954449811e-08
2049 2.30139861437939e-08
2050 2.30034622332731e-08
2051 2.30001819486736e-08
2052 2.29884755553744e-08
2053 2.29847774847469e-08
2054 2.29743383413084e-08
2055 2.2968883221397e-08
2056 2.29612979215643e-08
2057 2.29557349359366e-08
2058 2.29432010817598e-08
2059 2.29438755385836e-08
2060 2.29257980284281e-08
2061 2.29340787372134e-08
2062 2.29064237481369e-08
2063 2.29226662201709e-08
2064 2.2888138408228e-08
2065 2.29149739765511e-08
2066 2.28643546988927e-08
2067 2.29128248520061e-08
2068 2.28356033904475e-08
2069 2.29121740473737e-08
2070 2.28108338105049e-08
2071 2.29038626131661e-08
2072 2.27944310934314e-08
2073 2.28899290842799e-08
2074 2.27816389981594e-08
2075 2.28756454235812e-08
2076 2.27661341841712e-08
2077 2.28631618113262e-08
2078 2.27498813150095e-08
2079 2.28516302829274e-08
2080 2.27330775010026e-08
2081 2.28415730820775e-08
2082 2.27153455079065e-08
2083 2.2829225592047e-08
2084 2.27002034927093e-08
2085 2.28180378538667e-08
2086 2.26843944798327e-08
2087 2.28049087016569e-08
2088 2.26699559261467e-08
2089 2.27912293583321e-08
2090 2.26550066865228e-08
2091 2.27802872558858e-08
2092 2.26377473759865e-08
2093 2.27678935993403e-08
2094 2.26228504809356e-08
2095 2.27568641529841e-08
2096 2.2605093599859e-08
2097 2.27461373725202e-08
2098 2.25902698959279e-08
2099 2.27322228393279e-08
2100 2.25764610324308e-08
2101 2.27200079576662e-08
2102 2.2560204951616e-08
2103 2.27079176093881e-08
2104 2.25450550296324e-08
2105 2.26961179561336e-08
2106 2.25311378293513e-08
2107 2.26825223318405e-08
2108 2.25170187093671e-08
2109 2.26707011895533e-08
2110 2.2500137406789e-08
2111 2.26606710761468e-08
2112 2.24835057411887e-08
2113 2.26483555364476e-08
2114 2.24685801194502e-08
2115 2.26370369522666e-08
2116 2.24519153041447e-08
2117 2.26255630637606e-08
2118 2.24380999732654e-08
2119 2.2611146499929e-08
2120 2.24249494090722e-08
2121 2.2598842634225e-08
2122 2.24096035092991e-08
2123 2.25842420848998e-08
2124 2.23979371800631e-08
2125 2.25725710975011e-08
2126 2.23822088346592e-08
2127 2.25600471358556e-08
2128 2.23674521806494e-08
2129 2.25476996914553e-08
2130 2.23541785501213e-08
2131 2.25335728827325e-08
2132 2.23398676580944e-08
2133 2.25220147568317e-08
2134 2.23239299853573e-08
2135 2.25101922312065e-08
2136 2.2311164192379e-08
2137 2.2496526976834e-08
2138 2.22970740246797e-08
2139 2.24813339609886e-08
2140 2.22869564491246e-08
2141 2.24659764983537e-08
2142 2.22730435861518e-08
2143 2.24519808950108e-08
2144 2.22605461389636e-08
2145 2.24405051647558e-08
2146 2.22449875529884e-08
2147 2.24271025880407e-08
2148 2.22329327213e-08
2149 2.24120878955514e-08
2150 2.22205178250778e-08
2151 2.23977899832528e-08
2152 2.22071497979481e-08
2153 2.23830494062716e-08
2154 2.21977917074234e-08
2155 2.23652561173182e-08
2156 2.21873630371627e-08
2157 2.23479818275418e-08
2158 2.21774168766009e-08
2159 2.23325003114727e-08
2160 2.21673912965681e-08
2161 2.23169918023292e-08
2162 2.21545937867385e-08
2163 2.23059347367327e-08
2164 2.2139292777279e-08
2165 2.22905455480316e-08
2166 2.21297782730945e-08
2167 2.22731681100985e-08
2168 2.21212115802949e-08
2169 2.2253128893146e-08
2170 2.21139673031168e-08
2171 2.22344544259512e-08
2172 2.21055727206432e-08
2173 2.22195664972835e-08
2174 2.20924633798081e-08
2175 2.22103911237115e-08
2176 2.2073390196331e-08
2177 2.22028210006275e-08
2178 2.20562799105872e-08
2179 2.21828229659549e-08
2180 2.20559739545489e-08
2181 2.21476985361457e-08
2182 2.20681858640326e-08
2183 2.21092248183785e-08
2184 2.20708489446775e-08
2185 2.20978971134933e-08
2186 2.20451015926493e-08
2187 2.21243057167975e-08
2188 2.19891961227292e-08
2189 2.21489577533207e-08
2190 2.19460925339199e-08
2191 2.2103372356308e-08
2192 2.20246972272298e-08
2193 2.19879656345734e-08
2194 2.21181198108988e-08
2195 2.18733449183217e-08
2196 2.21000231340751e-08
2197 2.19546997425324e-08
2198 2.20123105646453e-08
2199 2.2043171156283e-08
2200 2.19090376484266e-08
2201 2.20819260091654e-08
2202 2.18556007481885e-08
2203 2.20262043447716e-08
2204 2.19656849901328e-08
2205 2.19032138062403e-08
2206 2.20470276026674e-08
2207 2.17937580176963e-08
2208 2.20208983210402e-08
2209 2.18828182489972e-08
2210 2.19520921999061e-08
2211 2.19224338430246e-08
2212 2.18814765240438e-08
2213 2.1972165204609e-08
2214 2.1797478002239e-08
2215 2.19762846562777e-08
2216 2.18214817694351e-08
2217 2.18891450122216e-08
2218 2.19230031789364e-08
2219 2.17701797642356e-08
2220 2.19368098928197e-08
2221 2.18016572435253e-08
2222 2.18590052342149e-08
2223 2.18821026075622e-08
2224 2.17510312451319e-08
2225 2.19046499430053e-08
2226 2.17600362066861e-08
2227 2.1837700366012e-08
2228 2.18368056583707e-08
2229 2.17380191372962e-08
2230 2.18590554269538e-08
2231 2.17421041889265e-08
2232 2.17989967800092e-08
2233 2.18062158132692e-08
2234 2.16943358539723e-08
2235 2.18469086461504e-08
2236 2.16730866724468e-08
2237 2.18153631227347e-08
2238 2.17089717265972e-08
2239 2.17736678673752e-08
2240 2.16895153798413e-08
2241 2.17732590575004e-08
2242 2.16738646513503e-08
2243 2.17553806820536e-08
2244 2.16838050155488e-08
2245 2.17115990022165e-08
2246 2.17040279398839e-08
2247 2.16580377939701e-08
2248 2.17300227404182e-08
2249 2.16226012048626e-08
2250 2.17222635339187e-08
2251 2.16329185256381e-08
2252 2.16762308473672e-08
2253 2.16569746605e-08
2254 2.16144256590667e-08
2255 2.16950149687367e-08
2256 2.15573609283481e-08
2257 2.17067036529794e-08
2258 2.15494167127295e-08
2259 2.16965862330776e-08
2260 2.15131650778133e-08
2261 2.17007604026431e-08
2262 2.14891168525133e-08
2263 2.17137086226948e-08
2264 2.14146434627249e-08
2265 2.17309007652933e-08
2266 2.14062068037979e-08
2267 2.17374103677059e-08
2268 2.13478446838833e-08
2269 2.17252740919438e-08
2270 2.1406830551296e-08
2271 2.16995126114394e-08
2272 2.13203126392392e-08
2273 2.16905051289018e-08
2274 2.13740914257743e-08
2275 2.17052550895991e-08
2276 2.12750839818643e-08
2277 2.16853103497394e-08
2278 2.13342425542384e-08
2279 2.16971662867493e-08
2280 2.12422474891172e-08
2281 2.16716945093953e-08
2282 2.13107273731028e-08
2283 2.1670708410193e-08
2284 2.12165235601303e-08
2285 2.16498120977349e-08
2286 2.12787557055671e-08
2287 2.16585138925751e-08
2288 2.11891966745981e-08
2289 2.16324126942613e-08
2290 2.12620302553201e-08
2291 2.16254907865476e-08
2292 2.11658004642956e-08
2293 2.16078458877833e-08
2294 2.12212837301662e-08
2295 2.16227942309022e-08
2296 2.11380620152024e-08
2297 2.15948645965192e-08
2298 2.12221881252805e-08
2299 2.1566951478702e-08
2300 2.11237205148818e-08
2301 2.15601337718585e-08
2302 2.11693798841672e-08
2303 2.15839948914942e-08
2304 2.10885989814225e-08
2305 2.15533027904113e-08
2306 2.11831498168058e-08
2307 2.15069865288076e-08
2308 2.10825460682518e-08
2309 2.15127711460372e-08
2310 2.11255988832315e-08
2311 2.15362857768353e-08
2312 2.103765067063e-08
2313 2.15147340694166e-08
2314 2.11282783468825e-08
2315 2.14710700501275e-08
2316 2.10268033031813e-08
2317 2.14783288832177e-08
2318 2.10698615941141e-08
2319 2.14973249339501e-08
2320 2.09897965025574e-08
2321 2.14724351019768e-08
2322 2.10906435992309e-08
2323 2.140927563965e-08
2324 2.09897652106994e-08
2325 2.14304464225545e-08
2326 2.10302054233225e-08
2327 2.14448432879211e-08
2328 2.0939083249738e-08
2329 2.14344997354621e-08
2330 2.10319792101066e-08
2331 2.1380418708139e-08
2332 2.09295537461074e-08
2333 2.14023264933916e-08
2334 2.09841979068592e-08
2335 2.13941446909338e-08
2336 2.08900664290246e-08
2337 2.13959961578025e-08
2338 2.09846169052508e-08
2339 2.13350301480064e-08
2340 2.08827951113655e-08
2341 2.13638201558286e-08
2342 2.09503684632795e-08
2343 2.1330159727273e-08
2344 2.08486522307316e-08
2345 2.13516043497952e-08
2346 2.09303573184272e-08
2347 2.12997252628799e-08
2348 2.08295851797935e-08
2349 2.13287494336134e-08
2350 2.09146646001912e-08
2351 2.126603116126e-08
2352 2.08106847049416e-08
2353 2.13055392490613e-08
2354 2.08879162814091e-08
2355 2.1248855059941e-08
2356 2.07858978590325e-08
2357 2.12879725894899e-08
2358 2.0875771936657e-08
2359 2.12088325570736e-08
2360 2.07709600026362e-08
2361 2.12600812498076e-08
2362 2.08539670110142e-08
2363 2.11857820483496e-08
2364 2.07480820126893e-08
2365 2.1240928907984e-08
2366 2.08339568925631e-08
2367 2.11578105496768e-08
2368 2.07283317578133e-08
2369 2.12177035718852e-08
2370 2.08163306859577e-08
2371 2.11254159590002e-08
2372 2.07121465007898e-08
2373 2.11931031994794e-08
2374 2.07970015022596e-08
2375 2.10973426942873e-08
2376 2.06929161394864e-08
2377 2.11690822322641e-08
2378 2.07764057451598e-08
2379 2.10714390500577e-08
2380 2.06731250167458e-08
2381 2.11459224545019e-08
2382 2.07569340858749e-08
2383 2.10433162186607e-08
2384 2.06533945331655e-08
2385 2.11226816554388e-08
2386 2.07386589150538e-08
2387 2.1014507655015e-08
2388 2.06355361864752e-08
2389 2.10981976684899e-08
2390 2.0717386623903e-08
2391 2.09881494244168e-08
2392 2.06166893864657e-08
2393 2.10726542416717e-08
2394 2.06965100754131e-08
2395 2.09628573273335e-08
2396 2.05965629046512e-08
2397 2.10500445769757e-08
2398 2.06765876735737e-08
2399 2.09367676972105e-08
2400 2.05774487397337e-08
2401 2.10244931869719e-08
2402 2.06553971981993e-08
2403 2.09128387637092e-08
2404 2.05560176880226e-08
2405 2.10027379512479e-08
2406 2.06374321668479e-08
2407 2.08820282159294e-08
2408 2.0540922479495e-08
2409 2.09760542019266e-08
2410 2.06142872468673e-08
2411 2.08601254721996e-08
2412 2.05190907578423e-08
2413 2.09543990471461e-08
2414 2.05964982779028e-08
2415 2.08304499977885e-08
2416 2.05015868770531e-08
2417 2.09294912465463e-08
2418 2.05732915767864e-08
2419 2.08077262167095e-08
2420 2.04814469131343e-08
2421 2.09054593489633e-08
2422 2.05526150465207e-08
2423 2.07836500558667e-08
2424 2.04612602987542e-08
2425 2.08829035267533e-08
2426 2.05280966147914e-08
2427 2.07628972981322e-08
2428 2.0439973116515e-08
2429 2.08599518366492e-08
2430 2.0503990975973e-08
2431 2.07438597286114e-08
2432 2.04163719098593e-08
2433 2.08399480566834e-08
2434 2.04795240710265e-08
2435 2.0733653324001e-08
2436 2.03881204511314e-08
2437 2.08209314723762e-08
2438 2.04427355506853e-08
2439 2.07489570520636e-08
2440 2.03603703273769e-08
2441 2.08009404606413e-08
2442 2.03823670203995e-08
2443 2.07516990463619e-08
2444 2.0353323932798e-08
2445 2.07685541770974e-08
2446 2.03483670786486e-08
2447 2.07370221858572e-08
2448 2.03360019278565e-08
2449 2.07413596685635e-08
2450 2.03288732122386e-08
2451 2.07164910087654e-08
2452 2.03162006401358e-08
2453 2.07173424396911e-08
2454 2.03088604853185e-08
2455 2.06963226833112e-08
2456 2.02975743436307e-08
2457 2.06927818776625e-08
2458 2.0289475963664e-08
2459 2.06744046470364e-08
2460 2.02784523111177e-08
2461 2.0669024814568e-08
2462 2.0270614669804e-08
2463 2.06514021963145e-08
2464 2.02605218786456e-08
2465 2.06460234229988e-08
2466 2.02555750250522e-08
2467 2.06257515807939e-08
2468 2.0246663043233e-08
2469 2.06219236604799e-08
2470 2.02498649900518e-08
2471 2.05982175844888e-08
2472 2.02583172349557e-08
2473 2.06661094387384e-08
2474 2.032325380108e-08
2475 2.06532631791267e-08
2476 2.0298535075236e-08
2477 2.06338005884765e-08
2478 2.0255964604865e-08
2479 2.06348306681647e-08
2480 2.03013822365206e-08
2481 2.06058501215356e-08
2482 2.02349598599216e-08
2483 2.06097716045628e-08
2484 2.02642749951298e-08
2485 2.05949076336198e-08
2486 2.02027564388807e-08
2487 2.05851596668527e-08
2488 2.02624819188868e-08
2489 2.05556591519329e-08
2490 2.01952863572163e-08
2491 2.05622800375505e-08
2492 2.02239927029613e-08
2493 2.054915861327e-08
2494 2.01606320031633e-08
2495 2.0540422058013e-08
2496 2.02185198954208e-08
2497 2.05129584360808e-08
2498 2.01526577116029e-08
2499 2.0516089801359e-08
2500 2.01842733713597e-08
2501 2.05036826637084e-08
2502 2.01213650743437e-08
2503 2.04991596222115e-08
2504 2.01673636320843e-08
2505 2.04787695269326e-08
2506 2.01043563331504e-08
2507 2.04750183905311e-08
2508 2.01416755717387e-08
2509 2.04612204172117e-08
2510 2.00822138936019e-08
2511 2.04581300726758e-08
2512 2.01127546916036e-08
2513 2.04437993842621e-08
2514 2.00597150582515e-08
2515 2.04406739833241e-08
2516 2.00856481972211e-08
2517 2.0423618185772e-08
2518 2.00399182612721e-08
2519 2.04201570090934e-08
2520 2.0062709205737e-08
2521 2.04026340853147e-08
2522 2.00205226902339e-08
2523 2.04004183600448e-08
2524 2.00360057583948e-08
2525 2.03805584786387e-08
2526 2.00029395991397e-08
2527 2.03770440087014e-08
2528 2.00148544244882e-08
2529 2.03593427156035e-08
2530 1.99837935201952e-08
2531 2.03558875011556e-08
2532 1.99910059260899e-08
2533 2.03380525555241e-08
2534 1.99677889362038e-08
2535 2.03322931323635e-08
2536 1.99695204921024e-08
2537 2.03166863823601e-08
2538 1.994853272691e-08
2539 2.0310494809328e-08
2540 1.99472959794278e-08
2541 2.02948262461033e-08
2542 1.99298855092511e-08
2543 2.02890087003249e-08
2544 1.99268913434469e-08
2545 2.02721297885011e-08
2546 1.99131677751652e-08
2547 2.02652384032609e-08
2548 1.99064588489728e-08
2549 2.02509573772325e-08
2550 1.98952052238299e-08
2551 2.02419112219632e-08
2552 1.98870208907298e-08
2553 2.02294330456931e-08
2554 1.98765735867079e-08
2555 2.02189622138249e-08
2556 1.98678518108508e-08
2557 2.02078491774449e-08
2558 1.9857653239308e-08
2559 2.01970032334131e-08
2560 1.98483451285814e-08
2561 2.01860596159564e-08
2562 1.98395634868342e-08
2563 2.01743565104717e-08
2564 1.98304991146925e-08
2565 2.01626999256632e-08
2566 1.98216151859887e-08
2567 2.01506073855828e-08
2568 1.98128804698161e-08
2569 2.01397814342252e-08
2570 1.98018437338465e-08
2571 2.01304078317754e-08
2572 1.97925207838789e-08
2573 2.01193663063037e-08
2574 1.9783358060077e-08
2575 2.01072185435081e-08
2576 1.97762960585379e-08
2577 2.00950141452383e-08
2578 1.97666298875632e-08
2579 2.00843183982835e-08
2580 1.97579166366646e-08
2581 2.0072033919849e-08
2582 1.97499449322569e-08
2583 2.0059592625743e-08
2584 1.97417579201886e-08
2585 2.00487971434571e-08
2586 1.97329968393278e-08
2587 2.00361336628596e-08
2588 1.97248415514384e-08
2589 2.00243637672459e-08
2590 1.9716936673575e-08
2591 2.00121142218679e-08
2592 1.97088772405651e-08
2593 2.00010009885343e-08
2594 1.97014631860837e-08
2595 1.99878377141749e-08
2596 1.96936004691617e-08
2597 1.99756175810473e-08
2598 1.96858341960926e-08
2599 1.99639591279555e-08
2600 1.96784744674883e-08
2601 1.99510851843776e-08
2602 1.96715392013491e-08
2603 1.99379014947931e-08
2604 1.96646088067576e-08
2605 1.99256783960378e-08
2606 1.96577623146066e-08
2607 1.99122463705459e-08
2608 1.9651490393735e-08
2609 1.9898775561189e-08
2610 1.96458210398598e-08
2611 1.98858858057038e-08
2612 1.96397674711024e-08
2613 1.98709001243413e-08
2614 1.9635543865637e-08
2615 1.98567440143904e-08
2616 1.96297313460603e-08
2617 1.98398522430754e-08
2618 1.96286630960163e-08
2619 1.98202892884858e-08
2620 1.96296023574627e-08
2621 1.97943475246376e-08
2622 1.96355335702059e-08
2623 1.97607083710638e-08
2624 1.96409681522125e-08
2625 1.97394314286337e-08
2626 1.96348314142059e-08
2627 1.97283744226562e-08
2628 1.9627031918823e-08
2629 1.97122436598196e-08
2630 1.96184488168161e-08
2631 1.97005467529321e-08
2632 1.96100537247501e-08
2633 1.96849140362554e-08
2634 1.96035064594646e-08
2635 1.9670003468808e-08
2636 1.95965474462101e-08
2637 1.96567288320848e-08
2638 1.9589144852894e-08
2639 1.96470494538969e-08
2640 1.9579184817764e-08
2641 1.96385721621395e-08
2642 1.95712248733937e-08
2643 1.96287162290698e-08
2644 1.95608623979604e-08
2645 1.96216993992859e-08
2646 1.95498429191865e-08
2647 1.96147514630596e-08
2648 1.95389653876443e-08
2649 1.96116152825221e-08
2650 1.95270729841379e-08
2651 1.96112621676514e-08
2652 1.95131417585204e-08
2653 1.96025242472642e-08
2654 1.95028078047166e-08
2655 1.95940923950477e-08
2656 1.94906158339503e-08
2657 1.95826736919535e-08
2658 1.94823370771591e-08
2659 1.95740158485114e-08
2660 1.94711321886976e-08
2661 1.95633263476713e-08
2662 1.94620604982987e-08
2663 1.9552689204505e-08
2664 1.94524712564537e-08
2665 1.95424244573417e-08
2666 1.94430974566062e-08
2667 1.95323217272447e-08
2668 1.94346206243701e-08
2669 1.95216995859671e-08
2670 1.94253052320237e-08
2671 1.95114192320656e-08
2672 1.94160399429322e-08
2673 1.95012482111512e-08
2674 1.94055278119887e-08
2675 1.9491439238406e-08
2676 1.93961838206302e-08
2677 1.94808592302032e-08
2678 1.93881433317999e-08
2679 1.94722231798838e-08
2680 1.9377699940315e-08
2681 1.94605896234323e-08
2682 1.93691139320773e-08
2683 1.94514078112329e-08
2684 1.93599222283458e-08
2685 1.94411501123204e-08
2686 1.93507681810479e-08
2687 1.94308334396931e-08
2688 1.9342003675038e-08
2689 1.94220502288678e-08
2690 1.93314456625737e-08
2691 1.94119786490754e-08
2692 1.93240653945104e-08
2693 1.9402258816914e-08
2694 1.9314450787622e-08
2695 1.93915329774752e-08
2696 1.93054280849259e-08
2697 1.93808603610179e-08
2698 1.92967676124756e-08
2699 1.93706894879853e-08
2700 1.92900885747083e-08
2701 1.93604378241341e-08
2702 1.92794349663794e-08
2703 1.93498769940348e-08
2704 1.92714661332305e-08
2705 1.93381037987272e-08
2706 1.92475262059677e-08
2707 1.93446859817481e-08
2708 1.919685205376e-08
2709 1.93501972819421e-08
2710 1.9217888437395e-08
2711 1.93441254654481e-08
2712 1.91780014077159e-08
2713 1.93281665379352e-08
2714 1.91952332176948e-08
2715 1.93262284990947e-08
2716 1.91591067773e-08
2717 1.93122038787408e-08
2718 1.91748971951622e-08
2719 1.93122657132783e-08
2720 1.9141942807166e-08
2721 1.92895012526284e-08
2722 1.91536470339759e-08
2723 1.92945913228826e-08
2724 1.91241438904921e-08
2725 1.92714666411575e-08
2726 1.9134900887674e-08
2727 1.92784855119754e-08
2728 1.91120874282191e-08
2729 1.92474738069937e-08
2730 1.91173609185302e-08
2731 1.92589131712273e-08
2732 1.91017281524486e-08
2733 1.92262885142513e-08
2734 1.90945881700655e-08
2735 1.92408678033607e-08
2736 1.90783319869992e-08
2737 1.9213501663784e-08
2738 1.90788139956544e-08
2739 1.92178375555407e-08
2740 1.90660585432934e-08
2741 1.91942591235561e-08
2742 1.90362720909887e-08
2743 1.92039304933722e-08
2744 1.90425889139423e-08
2745 1.91969572963524e-08
2746 1.90384348651706e-08
2747 1.917330706247e-08
2748 1.90227323894065e-08
2749 1.91656367144466e-08
2750 1.8995470883354e-08
2751 1.91375937278471e-08
2752 1.89570844227882e-08
2753 1.9123097986129e-08
2754 1.90942498690116e-08
2755 1.9104618542376e-08
2756 1.90076216137758e-08
2757 1.91144890318018e-08
2758 1.8959026659382e-08
2759 1.91276751115854e-08
2760 1.89636373816304e-08
2761 1.91195209449102e-08
2762 1.89565755271959e-08
2763 1.90996814460886e-08
2764 1.89596635370437e-08
2765 1.90831944474334e-08
2766 1.89304499008669e-08
2767 1.90860514290847e-08
2768 1.89348663706124e-08
2769 1.90662898169602e-08
2770 1.89116136029011e-08
2771 1.90696635621324e-08
2772 1.89149982254833e-08
2773 1.90470858184577e-08
2774 1.8892502596457e-08
2775 1.90515029476757e-08
2776 1.88967018193154e-08
2777 1.90244412340279e-08
2778 1.88633172361063e-08
2779 1.90483342454861e-08
2780 1.88735050412614e-08
2781 1.90144725761288e-08
2782 1.88603575013158e-08
2783 1.89830920385159e-08
2784 1.88252244109899e-08
2785 1.90781628166548e-08
2786 1.88602095871904e-08
2787 1.89505389347477e-08
2788 1.88292537370849e-08
2789 1.89922188951197e-08
2790 1.88264911321623e-08
2791 1.89647730993725e-08
2792 1.87748381100805e-08
2793 1.89591734907069e-08
2794 1.88990946422551e-08
2795 1.89024608997501e-08
2796 1.88200736556121e-08
2797 1.89188235328874e-08
2798 1.87671455161853e-08
2799 1.8913019109057e-08
2800 1.88789192305805e-08
2801 1.88973171196816e-08
2802 1.87816694311449e-08
2803 1.89263847791832e-08
2804 1.87754698174336e-08
2805 1.88866904529128e-08
2806 1.87208188593413e-08
2807 1.89240071194474e-08
2808 1.87244792276831e-08
2809 1.89078310433022e-08
2810 1.87474693401635e-08
2811 1.89026963069994e-08
2812 1.87120469352653e-08
2813 1.88545710212429e-08
2814 1.87051575712971e-08
2815 1.88853067329964e-08
2816 1.8716481446579e-08
2817 1.88564738128338e-08
2818 1.86767556942247e-08
2819 1.88633528949156e-08
2820 1.86860913690134e-08
2821 1.88818008310943e-08
2822 1.86547363804079e-08
2823 1.88314349147678e-08
2824 1.87336666945814e-08
2825 1.87918215048821e-08
2826 1.87095384384106e-08
2827 1.87705160569207e-08
2828 1.86609797069304e-08
2829 1.89192259014703e-08
2830 1.86127637133993e-08
2831 1.87784513674405e-08
2832 1.86263211955362e-08
2833 1.87955322276112e-08
2834 1.86532215278312e-08
2835 1.87511505214122e-08
2836 1.86142784455168e-08
2837 1.8901890900147e-08
2838 1.86003005088198e-08
2839 1.87207247308585e-08
2840 1.85999689671368e-08
2841 1.88116588429876e-08
2842 1.86236485733016e-08
2843 1.87224956581522e-08
2844 1.85532460746929e-08
2845 1.88213785947688e-08
2846 1.85772927968619e-08
2847 1.8716759290438e-08
2848 1.85553406976124e-08
2849 1.87587694353564e-08
2850 1.85921355573271e-08
2851 1.86794612765384e-08
2852 1.85327364383703e-08
2853 1.87552369983468e-08
2854 1.85783192103717e-08
2855 1.86707858961244e-08
2856 1.8492870291098e-08
2857 1.88494261944561e-08
2858 1.85158959455922e-08
2859 1.86307250563678e-08
2860 1.85157299783523e-08
2861 1.87243994604902e-08
2862 1.85280225004636e-08
2863 1.86203932991402e-08
2864 1.85110706644176e-08
2865 1.86687078549408e-08
2866 1.85270254887682e-08
2867 1.86439388286663e-08
2868 1.85020794063462e-08
2869 1.86391464013225e-08
2870 1.85226798848914e-08
2871 1.8596490814482e-08
2872 1.84286691657931e-08
2873 1.87666070927683e-08
2874 1.84253192190287e-08
2875 1.85579744615971e-08
2876 1.84715236910549e-08
2877 1.86392237737643e-08
2878 1.84766486559518e-08
2879 1.85375075278937e-08
2880 1.84428846510265e-08
2881 1.8696026571341e-08
2882 1.84268059689741e-08
2883 1.85392963168995e-08
2884 1.83693624162817e-08
2885 1.87464285201777e-08
2886 1.83957813947311e-08
2887 1.85217224860645e-08
2888 1.83657128250969e-08
2889 1.87145544859879e-08
2890 1.8366994588237e-08
2891 1.85066621514185e-08
2892 1.83534575973754e-08
2893 1.86861622208934e-08
2894 1.83571600375076e-08
2895 1.84911971431445e-08
2896 1.83344303523736e-08
2897 1.86744359508273e-08
2898 1.83402426497947e-08
2899 1.8477942048345e-08
2900 1.83149586981957e-08
2901 1.86605799283868e-08
2902 1.83331082690419e-08
2903 1.84612812588192e-08
2904 1.82899813349069e-08
2905 1.86525520232683e-08
2906 1.83149187885645e-08
2907 1.8452270154623e-08
2908 1.82693647299281e-08
2909 1.86377230864121e-08
2910 1.82971889420847e-08
2911 1.84361648659648e-08
2912 1.82463713915837e-08
2913 1.86244027581273e-08
2914 1.83105664889327e-08
2915 1.84025480522987e-08
2916 1.8228671085474e-08
2917 1.86157253082575e-08
2918 1.82921709896977e-08
2919 1.83852498285564e-08
2920 1.82176871856843e-08
2921 1.85907905339011e-08
2922 1.82747295219832e-08
2923 1.83723621411946e-08
2924 1.82001138880894e-08
2925 1.8542409364497e-08
2926 1.82677607429849e-08
2927 1.83609659967221e-08
2928 1.81830722190135e-08
2929 1.85602016088415e-08
2930 1.82371881215504e-08
2931 1.83426953488386e-08
2932 1.81692043650372e-08
2933 1.84325427455834e-08
2934 1.82118102810547e-08
2935 1.84065360551156e-08
2936 1.82004486125598e-08
2937 1.8350466820527e-08
2938 1.82289837667993e-08
2939 1.83618258299267e-08
2940 1.83111084559595e-08
2941 1.823703826076e-08
2942 1.82302816983393e-08
2943 1.82576863300143e-08
2944 1.83099139745568e-08
2945 1.83272040450433e-08
2946 1.81498055816798e-08
2947 1.82780583333209e-08
2948 1.82677343952831e-08
2949 1.82887870323611e-08
2950 1.81674809638377e-08
2951 1.82768768820507e-08
2952 1.8217276232857e-08
2953 1.82535753473489e-08
2954 1.83374859774599e-08
2955 1.80765683313044e-08
2956 1.82993732984471e-08
2957 1.82167860791616e-08
2958 1.81684270973426e-08
2959 1.82645260186076e-08
2960 1.82222725777503e-08
2961 1.81218763638524e-08
2962 1.8208197229086e-08
2963 1.81579276410382e-08
2964 1.82156342847239e-08
2965 1.81649191942634e-08
2966 1.81887161602168e-08
2967 1.81615053763684e-08
2968 1.81718248682961e-08
2969 1.81369494407502e-08
2970 1.81029135845057e-08
2971 1.82294339329259e-08
2972 1.82277477022152e-08
2973 1.80498374194871e-08
2974 1.81412001293291e-08
2975 1.81882471637085e-08
2976 1.81754188025574e-08
2977 1.80911548268714e-08
2978 1.81481945650264e-08
2979 1.80879302124071e-08
2980 1.8041081315423e-08
2981 1.82473143740536e-08
2982 1.79739540816515e-08
2983 1.81804224145266e-08
2984 1.80527924532603e-08
2985 1.81482268989397e-08
2986 1.80604804060902e-08
2987 1.81364917086757e-08
2988 1.79523346973687e-08
2989 1.82930655563407e-08
2990 1.79775819886263e-08
2991 1.81618285113316e-08
2992 1.79192399042405e-08
2993 1.81882235651409e-08
2994 1.80998666614585e-08
2995 1.79984050354687e-08
2996 1.80369838284067e-08
2997 1.80637509528125e-08
2998 1.81002331681723e-08
2999 1.80758074665999e-08
3000 6.45671414824939e-09
3001 6.51945477246618e-09
3002 6.64834056555291e-09
3003 6.72847049369707e-09
3004 6.76719513345581e-09
3005 6.77871023015142e-09
3006 6.78059446593271e-09
3007 6.78005095915146e-09
3008 6.77892900251276e-09
3009 6.77766885626152e-09
3010 6.77642397857448e-09
3011 6.7752337769289e-09
3012 6.7740363699359e-09
3013 6.77289925361813e-09
3014 6.77181737869514e-09
3015 6.7706601456774e-09
3016 6.76956174100463e-09
3017 6.76847818609216e-09
3018 6.76742153721266e-09
3019 6.76636976930378e-09
3020 6.76536080709833e-09
3021 6.76430448003085e-09
3022 6.76333908274096e-09
3023 6.76235278414261e-09
3024 6.76134853341837e-09
3025 6.76039912725357e-09
3026 6.75943139681612e-09
3027 6.75849389628058e-09
3028 6.75753867231121e-09
3029 6.75659081751656e-09
3030 6.75567961520851e-09
3031 6.7547285858699e-09
3032 6.75383964503229e-09
3033 6.75295120700081e-09
3034 6.75204971306576e-09
3035 6.75119441882754e-09
3036 6.75028021478175e-09
3037 6.74938175056206e-09
3038 6.74855887679293e-09
3039 6.74770469416552e-09
3040 6.74684466656084e-09
3041 6.7459628442651e-09
3042 6.7451902134863e-09
3043 6.74429028742207e-09
3044 6.74342543233175e-09
3045 6.74261163106926e-09
3046 6.74175048862025e-09
3047 6.74096202336882e-09
3048 6.7401577086984e-09
3049 6.73931185787846e-09
3050 6.73851434644657e-09
3051 6.73767180274509e-09
3052 6.73687475007123e-09
3053 6.73609168710954e-09
3054 6.73526657726797e-09
3055 6.73448496206486e-09
3056 6.73367293829719e-09
3057 6.732883937835e-09
3058 6.73207006103571e-09
3059 6.73129928131788e-09
3060 6.73051129353175e-09
3061 6.72974081504518e-09
3062 6.72898420871204e-09
3063 6.72816226436612e-09
3064 6.72737219757635e-09
3065 6.72659928212249e-09
3066 6.72586031896794e-09
3067 6.72510400655246e-09
3068 6.72430347907482e-09
3069 6.72356444937627e-09
3070 6.7228156329785e-09
3071 6.72204416168887e-09
3072 6.72122995225288e-09
3073 6.72057684261706e-09
3074 6.71977512287092e-09
3075 6.71899623536087e-09
3076 6.71830918423655e-09
3077 6.71751146173738e-09
3078 6.71674422639268e-09
3079 6.71601723599424e-09
3080 6.71525753390134e-09
3081 6.71454495722312e-09
3082 6.71381597268372e-09
3083 6.71301345472886e-09
3084 6.71231979725018e-09
3085 6.7115832142195e-09
3086 6.71087252299141e-09
3087 6.71011607523975e-09
3088 6.70938343488858e-09
3089 6.70865281862665e-09
3090 6.70793021792515e-09
3091 6.70718942916149e-09
3092 6.70644402604725e-09
3093 6.70573361423454e-09
3094 6.70501491105424e-09
3095 6.70426889244624e-09
3096 6.70356499148095e-09
3097 6.70284879278338e-09
3098 6.702132088518e-09
3099 6.70141101373478e-09
3100 6.70067157652887e-09
3101 6.69995761837239e-09
3102 6.69924867977012e-09
3103 6.69853053192337e-09
3104 6.69784492930703e-09
3105 6.6971082469669e-09
3106 6.69644911423239e-09
3107 6.6956713897226e-09
3108 6.69499332552059e-09
3109 6.69427873514761e-09
3110 6.69356438191826e-09
3111 6.69287039377353e-09
3112 6.69217104791453e-09
3113 6.69148909371309e-09
3114 6.6907406072042e-09
3115 6.69004179711108e-09
3116 6.6893601964324e-09
3117 6.68863908756534e-09
3118 6.68795090903729e-09
3119 6.68728589162115e-09
3120 6.68657357219549e-09
3121 6.68587517925479e-09
3122 6.68519747128171e-09
3123 6.68449678609129e-09
3124 6.68379854629197e-09
3125 6.68312393709014e-09
3126 6.68240925999486e-09
3127 6.68173022312646e-09
3128 6.68105112482109e-09
3129 6.68039008011601e-09
3130 6.67967703693595e-09
3131 6.67900759498963e-09
3132 6.67831271555785e-09
3133 6.67764831989437e-09
3134 6.67692179036339e-09
3135 6.67624771671926e-09
3136 6.67559359179837e-09
3137 6.67489446244673e-09
3138 6.67421924414879e-09
3139 6.67353147131011e-09
3140 6.67288423197854e-09
3141 6.67220754629883e-09
3142 6.67151778153963e-09
3143 6.67083081980213e-09
3144 6.67015463268195e-09
3145 6.66951307812247e-09
3146 6.66879778893348e-09
3147 6.66814888244938e-09
3148 6.66745602215246e-09
3149 6.66682146678588e-09
3150 6.66609433219723e-09
3151 6.66543436084188e-09
3152 6.66477967109502e-09
3153 6.66412440102071e-09
3154 6.66347374644882e-09
3155 6.66276547318934e-09
3156 6.66213035345475e-09
3157 6.66140308278051e-09
3158 6.66076643275615e-09
3159 6.66009902898501e-09
3160 6.65942120990637e-09
3161 6.65875902777779e-09
3162 6.658074757665e-09
3163 6.65740757194166e-09
3164 6.65676886757449e-09
3165 6.65604092887906e-09
3166 6.65541893313248e-09
3167 6.65475829009221e-09
3168 6.65409550164309e-09
3169 6.65341174389822e-09
3170 6.65273527385157e-09
3171 6.65207462047235e-09
3172 6.65143907821464e-09
3173 6.65076652527075e-09
3174 6.6501031789068e-09
3175 6.64946937212196e-09
3176 6.64879931054629e-09
3177 6.64814928143242e-09
3178 6.64750881104736e-09
3179 6.64681170273185e-09
3180 6.6461874566881e-09
3181 6.64551461071472e-09
3182 6.6448623875226e-09
3183 6.64420401484678e-09
3184 6.64355698885843e-09
3185 6.64286145317383e-09
3186 6.64226348164043e-09
3187 6.64156303098462e-09
3188 6.64091206602213e-09
3189 6.64024075545944e-09
3190 6.63960652591555e-09
3191 6.63894827002132e-09
3192 6.63832346681215e-09
3193 6.63763890761115e-09
3194 6.63697456861068e-09
3195 6.63635894192216e-09
3196 6.63569633542471e-09
3197 6.63507592491008e-09
3198 6.6343936049873e-09
3199 6.63378842306883e-09
3200 6.63308464465828e-09
3201 6.63243758755594e-09
3202 6.63175996473697e-09
3203 6.63121307228898e-09
3204 6.63053420893456e-09
3205 6.62986786123532e-09
3206 6.62920000735201e-09
3207 6.6285385404985e-09
3208 6.62787390744157e-09
3209 6.62723357623685e-09
3210 6.6265965260881e-09
3211 6.62592281482077e-09
3212 6.62525154813964e-09
3213 6.62467175113757e-09
3214 6.62402993055478e-09
3215 6.62333116656366e-09
3216 6.62265930395645e-09
3217 6.62203185571031e-09
3218 6.62142078253214e-09
3219 6.62076850288518e-09
3220 6.62015539537597e-09
3221 6.61947146606001e-09
3222 6.61881464968095e-09
3223 6.6182028621159e-09
3224 6.61752504581281e-09
3225 6.61691283436461e-09
3226 6.61625412068378e-09
3227 6.61563966081191e-09
3228 6.61494578020616e-09
3229 6.6143117112144e-09
3230 6.61368077221891e-09
3231 6.61304530309714e-09
3232 6.61239299873384e-09
3233 6.61175298369288e-09
3234 6.61104515344013e-09
3235 6.61046675104471e-09
3236 6.60982923149367e-09
3237 6.60919611544797e-09
3238 6.60854675083034e-09
3239 6.60792065801385e-09
3240 6.60726662785049e-09
3241 6.6066420312122e-09
3242 6.60600637078512e-09
3243 6.60534639559951e-09
3244 6.6047530523361e-09
3245 6.60409150982089e-09
3246 6.60346553187086e-09
3247 6.60278096531464e-09
3248 6.60221512081838e-09
3249 6.60153332575353e-09
3250 6.60089252146889e-09
3251 6.60027276236375e-09
3252 6.59958247009595e-09
3253 6.59899701616062e-09
3254 6.5983768531841e-09
3255 6.59770809936788e-09
3256 6.59707645069008e-09
3257 6.59646438093409e-09
3258 6.59584807047309e-09
3259 6.5951883288784e-09
3260 6.59456826108962e-09
3261 6.59390124924109e-09
3262 6.59328892586852e-09
3263 6.59265863688085e-09
3264 6.59205308635447e-09
3265 6.59140822582838e-09
3266 6.59075843895129e-09
3267 6.59014672903246e-09
3268 6.58954039979565e-09
3269 6.5888736452413e-09
3270 6.58824326715823e-09
3271 6.58764121139188e-09
3272 6.58700540738522e-09
3273 6.58640331094307e-09
3274 6.58573914705252e-09
3275 6.5851655464827e-09
3276 6.58452009240362e-09
3277 6.58385381208104e-09
3278 6.58327207152254e-09
3279 6.58261475394717e-09
3280 6.5820074537648e-09
3281 6.58135931091985e-09
3282 6.58072351862604e-09
3283 6.58011753258692e-09
3284 6.57945653448344e-09
3285 6.57886345925562e-09
3286 6.57821122718172e-09
3287 6.57761002788304e-09
3288 6.5769656853315e-09
3289 6.57634108766625e-09
3290 6.57573795861954e-09
3291 6.57511162520386e-09
3292 6.57444679816321e-09
3293 6.57384298276276e-09
3294 6.5731814996306e-09
3295 6.5726287292317e-09
3296 6.57197980233337e-09
3297 6.57134266458803e-09
3298 6.57073136160757e-09
3299 6.57010570252747e-09
3300 6.56948766492027e-09
3301 6.56888117953058e-09
3302 6.56822710103089e-09
3303 6.56760139351731e-09
3304 6.56698014137036e-09
3305 6.5663539248334e-09
3306 6.56575634776224e-09
3307 6.5650806498202e-09
3308 6.56451821745563e-09
3309 6.56388216505044e-09
3310 6.56326612519242e-09
3311 6.56264335080481e-09
3312 6.56204301609231e-09
3313 6.56139601967753e-09
3314 6.56077965673074e-09
3315 6.56013046727855e-09
3316 6.55951739050864e-09
3317 6.55888018333273e-09
3318 6.55829008056052e-09
3319 6.5576403766171e-09
3320 6.55704497742882e-09
3321 6.55641255153938e-09
3322 6.55580146548262e-09
3323 6.55518515448039e-09
3324 6.55458142453935e-09
3325 6.55391092629909e-09
3326 6.55333893248566e-09
3327 6.55271508108457e-09
3328 6.55207385362455e-09
3329 6.55150458135634e-09
3330 6.55088879375487e-09
3331 6.55023848891711e-09
3332 6.54961799055609e-09
3333 6.54897828909762e-09
3334 6.54836307115159e-09
3335 6.54773023363309e-09
3336 6.54714854683713e-09
3337 6.54648036083061e-09
3338 6.54591960018436e-09
3339 6.54528201479709e-09
3340 6.54468187998025e-09
3341 6.54403864457087e-09
3342 6.5434156036881e-09
3343 6.54285572304625e-09
3344 6.5422158140177e-09
3345 6.54159681387489e-09
3346 6.54094792500171e-09
3347 6.54035792060914e-09
3348 6.53975512289462e-09
3349 6.53915282768092e-09
3350 6.53854132191822e-09
3351 6.53792586248481e-09
3352 6.53732521020689e-09
3353 6.53672381838166e-09
3354 6.53610185687159e-09
3355 6.53550287330051e-09
3356 6.53489483551373e-09
3357 6.53428623668961e-09
3358 6.53366165859204e-09
3359 6.53304730623139e-09
3360 6.53241409029337e-09
3361 6.53187725543902e-09
3362 6.53125848007474e-09
3363 6.53061305250224e-09
3364 6.53006253761856e-09
3365 6.52938851755658e-09
3366 6.5288160484428e-09
3367 6.52817819518647e-09
3368 6.52757588276431e-09
3369 6.52697402475644e-09
3370 6.52637318769578e-09
3371 6.52575855711324e-09
3372 6.52518289151471e-09
3373 6.5245304735756e-09
3374 6.52392184814776e-09
3375 6.52330836199699e-09
3376 6.52271938536564e-09
3377 6.52209034682216e-09
3378 6.52146672633358e-09
3379 6.52086342736724e-09
3380 6.52028567993401e-09
3381 6.51967258918917e-09
3382 6.51908088022957e-09
3383 6.51845815297092e-09
3384 6.51785825560303e-09
3385 6.51725797300162e-09
3386 6.51666260391426e-09
3387 6.51601862751428e-09
3388 6.51543147264788e-09
3389 6.51483388454388e-09
3390 6.5142274378871e-09
3391 6.51363882442357e-09
3392 6.51300543801081e-09
3393 6.51238711477098e-09
3394 6.51179783420219e-09
3395 6.51120255429349e-09
3396 6.51058959830197e-09
3397 6.50998618430265e-09
3398 6.50936252480361e-09
3399 6.50879371501267e-09
3400 6.50814363643837e-09
3401 6.50755222680877e-09
3402 6.50696736177736e-09
3403 6.50635168700231e-09
3404 6.50579175623389e-09
3405 6.50512984226581e-09
3406 6.50454871338468e-09
3407 6.50393324366783e-09
3408 6.50331663169801e-09
3409 6.50273612544383e-09
3410 6.50212561399688e-09
3411 6.50154054496199e-09
3412 6.50089716446034e-09
3413 6.50030235960219e-09
3414 6.49974642040474e-09
3415 6.49911572900286e-09
3416 6.49850563420873e-09
3417 6.49792108607661e-09
3418 6.49733402829922e-09
3419 6.49675032959873e-09
3420 6.49610868201644e-09
3421 6.49551609306631e-09
3422 6.49489423537597e-09
3423 6.49431765410713e-09
3424 6.49371069226523e-09
3425 6.49313582302968e-09
3426 6.49255031351381e-09
3427 6.49192769491824e-09
3428 6.49133607391605e-09
3429 6.49071068013762e-09
3430 6.49013860645753e-09
3431 6.48951405926579e-09
3432 6.48891919022288e-09
3433 6.48835228095335e-09
3434 6.48776277790974e-09
3435 6.48714451415011e-09
3436 6.48653408270861e-09
3437 6.4859575126669e-09
3438 6.48536749019157e-09
3439 6.48476405123999e-09
3440 6.48415790087398e-09
3441 6.48358977427266e-09
3442 6.48297985546276e-09
3443 6.48236693272242e-09
3444 6.48177971980524e-09
3445 6.4812231560657e-09
3446 6.48060612784551e-09
3447 6.480034228179e-09
3448 6.47943428275233e-09
3449 6.47881220940116e-09
3450 6.47818821648827e-09
3451 6.47765564555647e-09
3452 6.47701931588696e-09
3453 6.47642412686389e-09
3454 6.47584551574654e-09
3455 6.47523277959305e-09
3456 6.47466095975158e-09
3457 6.47404319460698e-09
3458 6.47348012354498e-09
3459 6.47287540873898e-09
3460 6.47225148354968e-09
3461 6.47163107986293e-09
3462 6.47108505404115e-09
3463 6.47049529373112e-09
3464 6.46985092185581e-09
3465 6.46928942281411e-09
3466 6.46866329610807e-09
3467 6.46804376458476e-09
3468 6.46752787278326e-09
3469 6.46688809899376e-09
3470 6.4662993118636e-09
3471 6.46574383697529e-09
3472 6.46513369632895e-09
3473 6.46451536656656e-09
3474 6.46391945968716e-09
3475 6.46336025317273e-09
3476 6.46280121699427e-09
3477 6.46214845699544e-09
3478 6.46156742988013e-09
3479 6.46099170206749e-09
3480 6.46035794260591e-09
3481 6.4598033459734e-09
3482 6.45921354132384e-09
3483 6.45858352489592e-09
3484 6.45800605572622e-09
3485 6.45742364767388e-09
3486 6.45682743394271e-09
3487 6.45620987832496e-09
3488 6.45566877899295e-09
3489 6.45502178921176e-09
3490 6.45444520681882e-09
3491 6.45385298071732e-09
3492 6.453237090559e-09
3493 6.45264463382256e-09
3494 6.45207417759863e-09
3495 6.45146733566082e-09
3496 6.45087684111867e-09
3497 6.45027542467425e-09
3498 6.44973326514864e-09
3499 6.44913895751775e-09
3500 6.4485423138666e-09
3501 6.44794208813637e-09
3502 6.44736571854543e-09
3503 6.446758689993e-09
3504 6.44615321554465e-09
3505 6.44560255243232e-09
3506 6.44498134096116e-09
3507 6.4443975944517e-09
3508 6.44381402922778e-09
3509 6.44324179198408e-09
3510 6.4426207515289e-09
3511 6.44204096041101e-09
3512 6.44144340471164e-09
3513 6.44087187602616e-09
3514 6.44028104569705e-09
3515 6.43970024759299e-09
3516 6.43910851758078e-09
3517 6.43850431340798e-09
3518 6.43794868972203e-09
3519 6.43731212753018e-09
3520 6.43676052299036e-09
3521 6.43612304261632e-09
3522 6.43556561452652e-09
3523 6.43497536398363e-09
3524 6.43437718901574e-09
3525 6.43376976521004e-09
3526 6.43318594867326e-09
3527 6.43256403856651e-09
3528 6.43200952570033e-09
3529 6.43143502931931e-09
3530 6.43085597569482e-09
3531 6.43023451268876e-09
3532 6.42965335283241e-09
3533 6.42907216011346e-09
3534 6.42846648440942e-09
3535 6.42793405580822e-09
3536 6.42730609605457e-09
3537 6.42673603340471e-09
3538 6.42611753133571e-09
3539 6.42552484136916e-09
3540 6.42498617664522e-09
3541 6.42436851483463e-09
3542 6.42376091697372e-09
3543 6.42320035107447e-09
3544 6.42259471309026e-09
3545 6.42200440922891e-09
3546 6.42142028189907e-09
3547 6.42083481407207e-09
3548 6.42026818103991e-09
3549 6.41965216315044e-09
3550 6.41910217025199e-09
3551 6.41846888496667e-09
3552 6.41790842111078e-09
3553 6.41729020250947e-09
3554 6.41677984113132e-09
3555 6.4161104070537e-09
3556 6.41555478396449e-09
3557 6.41495631027722e-09
3558 6.41438912613035e-09
3559 6.41381498717175e-09
3560 6.41322129929511e-09
3561 6.41268493886682e-09
3562 6.41205062458516e-09
3563 6.41145570025314e-09
3564 6.41089130690164e-09
3565 6.41030615590454e-09
3566 6.40972303167509e-09
3567 6.40919126236594e-09
3568 6.40856536268664e-09
3569 6.40800377549322e-09
3570 6.40740943003149e-09
3571 6.40684053580809e-09
3572 6.40625798126182e-09
3573 6.4056802869944e-09
3574 6.40506838862909e-09
3575 6.40451759124916e-09
3576 6.40391074080426e-09
3577 6.40334678744803e-09
3578 6.40278738744948e-09
3579 6.40219430463051e-09
3580 6.40160583520455e-09
3581 6.40105654557688e-09
3582 6.40046273155115e-09
3583 6.3999218690991e-09
3584 6.39930727484861e-09
3585 6.39871019705862e-09
3586 6.39815236291863e-09
3587 6.39754204674603e-09
3588 6.39700565523149e-09
3589 6.39640275085229e-09
3590 6.39584036071783e-09
3591 6.39524143605796e-09
3592 6.39472720126433e-09
3593 6.39408826758059e-09
3594 6.39355237777584e-09
3595 6.39294498459841e-09
3596 6.39238093785854e-09
3597 6.39180821233876e-09
3598 6.39121679835153e-09
3599 6.39069687249916e-09
3600 6.39007016098314e-09
3601 6.38948956689644e-09
3602 6.388918082606e-09
3603 6.38837479625953e-09
3604 6.38779007801349e-09
3605 6.38720168061324e-09
3606 6.38663041968579e-09
3607 6.38605142652682e-09
3608 6.38547596988614e-09
3609 6.38491781358719e-09
3610 6.3843000402547e-09
3611 6.38375122515023e-09
3612 6.38315867591832e-09
3613 6.38263019717145e-09
3614 6.38200325427107e-09
3615 6.38143263492763e-09
3616 6.38085748301542e-09
3617 6.38028897455289e-09
3618 6.37974942671593e-09
3619 6.37914544408813e-09
3620 6.3785612656464e-09
3621 6.37802251364505e-09
3622 6.37742851596068e-09
3623 6.37684985126119e-09
3624 6.37628563322779e-09
3625 6.37571511900858e-09
3626 6.37513552832758e-09
3627 6.37456296201377e-09
3628 6.37395781850902e-09
3629 6.37339544380666e-09
3630 6.37283886835427e-09
3631 6.37221796902232e-09
3632 6.37168543991817e-09
3633 6.37108844120382e-09
3634 6.37051969942792e-09
3635 6.36994873033647e-09
3636 6.36937743894728e-09
3637 6.36879853478656e-09
3638 6.36820500655999e-09
3639 6.36765573597264e-09
3640 6.36708827631005e-09
3641 6.36649549459745e-09
3642 6.36591771584205e-09
3643 6.36537200203458e-09
3644 6.36478242553584e-09
3645 6.3642151392207e-09
3646 6.36362166060722e-09
3647 6.36306251033947e-09
3648 6.36250298970131e-09
3649 6.36192293909654e-09
3650 6.36137407400428e-09
3651 6.36079913604592e-09
3652 6.36024228785337e-09
3653 6.35963524783789e-09
3654 6.35906834413336e-09
3655 6.35852158085981e-09
3656 6.35796193974858e-09
3657 6.35737857988816e-09
3658 6.3567745551274e-09
3659 6.35621905235861e-09
3660 6.35562461911987e-09
3661 6.35506085486237e-09
3662 6.35444834116983e-09
3663 6.35390149943127e-09
3664 6.35336804370723e-09
3665 6.35273413121529e-09
3666 6.352180762767e-09
3667 6.35162296920566e-09
3668 6.35107899588094e-09
3669 6.35048165514851e-09
3670 6.34993176784615e-09
3671 6.34939230823028e-09
3672 6.34875927811585e-09
3673 6.34822304838856e-09
3674 6.34767499839428e-09
3675 6.34707910913968e-09
3676 6.34650389538804e-09
3677 6.34593009694873e-09
3678 6.34539687729974e-09
3679 6.34476444086318e-09
3680 6.34424618090046e-09
3681 6.34365982310481e-09
3682 6.34312135165682e-09
3683 6.34252707737426e-09
3684 6.34198312161882e-09
3685 6.34139687973045e-09
3686 6.34082601028152e-09
3687 6.34028209434145e-09
3688 6.33969399170542e-09
3689 6.33910751215994e-09
3690 6.33854544532242e-09
3691 6.33800526381179e-09
3692 6.33741319794323e-09
3693 6.33684510126242e-09
3694 6.3362975224468e-09
3695 6.33569745275842e-09
3696 6.33512970442396e-09
3697 6.33460583607248e-09
3698 6.33403140275213e-09
3699 6.33345264380858e-09
3700 6.33285602982814e-09
3701 6.33229240679101e-09
3702 6.33171615847805e-09
3703 6.33115829180853e-09
3704 6.33057612280108e-09
3705 6.33001735377003e-09
3706 6.32946015814706e-09
3707 6.32886559530366e-09
3708 6.32832699083707e-09
3709 6.32775493585036e-09
3710 6.32721439262907e-09
3711 6.3266148091351e-09
3712 6.32606428510596e-09
3713 6.32549577110619e-09
3714 6.32489392962676e-09
3715 6.3243618080161e-09
3716 6.32377134458795e-09
3717 6.32320140914189e-09
3718 6.32262994276767e-09
3719 6.32208470700835e-09
3720 6.32153063420682e-09
3721 6.32093007667256e-09
3722 6.32039036010945e-09
3723 6.3198582824775e-09
3724 6.31927697770929e-09
3725 6.3186757743583e-09
3726 6.31808609945217e-09
3727 6.31756095713365e-09
3728 6.31699918685058e-09
3729 6.3164473827898e-09
3730 6.31586493690661e-09
3731 6.3153226836643e-09
3732 6.31472863288351e-09
3733 6.31416123497708e-09
3734 6.31363316046241e-09
3735 6.31304379589137e-09
3736 6.31248225435588e-09
3737 6.3118980595106e-09
3738 6.3113505780199e-09
3739 6.31075098393719e-09
3740 6.31020270287774e-09
3741 6.30963591659317e-09
3742 6.30907416537818e-09
3743 6.30850474923894e-09
3744 6.30796250353227e-09
3745 6.3074176600425e-09
3746 6.30681661578647e-09
3747 6.30626560692293e-09
3748 6.30569308168738e-09
3749 6.30517420580068e-09
3750 6.30455418014486e-09
3751 6.30404102279492e-09
3752 6.30347160612832e-09
3753 6.30287447914157e-09
3754 6.302329120772e-09
3755 6.30179486175997e-09
3756 6.30121722421106e-09
3757 6.30065387181855e-09
3758 6.30010117538826e-09
3759 6.29948754057863e-09
3760 6.29893390971525e-09
3761 6.29837961504953e-09
3762 6.29783972280751e-09
3763 6.29723298661844e-09
3764 6.29668623207402e-09
3765 6.29615898703473e-09
3766 6.29556982562063e-09
3767 6.29505766899796e-09
3768 6.29443622331138e-09
3769 6.29384306964631e-09
3770 6.29332593014842e-09
3771 6.29274890980025e-09
3772 6.29216599765114e-09
3773 6.29164600797483e-09
3774 6.29107966604314e-09
3775 6.29048940270494e-09
3776 6.28994541390648e-09
3777 6.28938768477971e-09
3778 6.28881915827606e-09
3779 6.28827378792995e-09
3780 6.28772030536462e-09
3781 6.28717205393425e-09
3782 6.28659113059704e-09
3783 6.28604269282962e-09
3784 6.28548661484041e-09
3785 6.28494338392183e-09
3786 6.28437338105747e-09
3787 6.28380444260557e-09
3788 6.28325348148162e-09
3789 6.28269270097626e-09
3790 6.2821513817507e-09
3791 6.28157634288695e-09
3792 6.2810167148486e-09
3793 6.28044175608744e-09
3794 6.27991378603088e-09
3795 6.27938837612085e-09
3796 6.27880866312103e-09
3797 6.27821646208282e-09
3798 6.27768126003347e-09
3799 6.27709556740019e-09
3800 6.27656928756104e-09
3801 6.27595131239e-09
3802 6.27542269263093e-09
3803 6.27488774053442e-09
3804 6.27430621855107e-09
3805 6.2737637546717e-09
3806 6.27322891166848e-09
3807 6.27263654426335e-09
3808 6.27209399817197e-09
3809 6.27155246286926e-09
3810 6.27096440937447e-09
3811 6.27041260257977e-09
3812 6.26987139761004e-09
3813 6.26928625509227e-09
3814 6.26875977755015e-09
3815 6.26818475110702e-09
3816 6.26762084356136e-09
3817 6.26709355333599e-09
3818 6.26654327359755e-09
3819 6.26598124901789e-09
3820 6.2654122506417e-09
3821 6.26486438981555e-09
3822 6.26429177738586e-09
3823 6.2637020523254e-09
3824 6.26316334720323e-09
3825 6.2626346011424e-09
3826 6.26207884331376e-09
3827 6.26149441664003e-09
3828 6.26094370513586e-09
3829 6.26039124876354e-09
3830 6.25982811268411e-09
3831 6.2592797016453e-09
3832 6.25871784315502e-09
3833 6.25814936368319e-09
3834 6.25761489340959e-09
3835 6.25706583268215e-09
3836 6.25647814243846e-09
3837 6.25598186579368e-09
3838 6.25539836517086e-09
3839 6.25486620950411e-09
3840 6.25432103738832e-09
3841 6.25378813823296e-09
3842 6.25320555551479e-09
3843 6.25265863839208e-09
3844 6.25208807590594e-09
3845 6.25150509087069e-09
3846 6.25098709321204e-09
3847 6.25042990980151e-09
3848 6.24988463819587e-09
3849 6.2493658484486e-09
3850 6.24878521611472e-09
3851 6.24826725684202e-09
3852 6.24770428696297e-09
3853 6.24713615689221e-09
3854 6.24656030523418e-09
3855 6.24602035355359e-09
3856 6.24546505488544e-09
3857 6.24493631337653e-09
3858 6.24438077144462e-09
3859 6.24383034289466e-09
3860 6.24328719557587e-09
3861 6.24273574746648e-09
3862 6.24219232409073e-09
3863 6.241609905977e-09
3864 6.24105443421119e-09
3865 6.24052244362072e-09
3866 6.23998082469046e-09
3867 6.23943105500235e-09
3868 6.23884331571456e-09
3869 6.23833130433682e-09
3870 6.2377704135308e-09
3871 6.23721498789476e-09
3872 6.23665001582563e-09
3873 6.23608994587688e-09
3874 6.23554894159384e-09
3875 6.23503507989065e-09
3876 6.23443905625742e-09
3877 6.23390238098376e-09
3878 6.23336774509264e-09
3879 6.23280789077696e-09
3880 6.23224964423075e-09
3881 6.23170310420917e-09
3882 6.23114010066261e-09
3883 6.23062706796296e-09
3884 6.23003913489961e-09
3885 6.22947292842901e-09
3886 6.22892826641908e-09
3887 6.22837888671068e-09
3888 6.22781870789069e-09
3889 6.22730088761791e-09
3890 6.22676174119596e-09
3891 6.22615222499456e-09
3892 6.22561146025602e-09
3893 6.22508717021408e-09
3894 6.22449055394381e-09
3895 6.22396887824117e-09
3896 6.22341654431258e-09
3897 6.22285909945841e-09
3898 6.22229066474245e-09
3899 6.2217726337771e-09
3900 6.22119224603923e-09
3901 6.22067885568123e-09
3902 6.22009580127092e-09
3903 6.2195398140702e-09
3904 6.2190517512023e-09
3905 6.21845529445719e-09
3906 6.2178901515525e-09
3907 6.21735540198842e-09
3908 6.21681123458284e-09
3909 6.21624586594205e-09
3910 6.21572784299806e-09
3911 6.21517281451656e-09
3912 6.21460695467146e-09
3913 6.21407089558546e-09
3914 6.21354180736777e-09
3915 6.21298464315023e-09
3916 6.21245412274485e-09
3917 6.21188393395977e-09
3918 6.21135831263553e-09
3919 6.21077800447289e-09
3920 6.21028189623507e-09
3921 6.20968604574112e-09
3922 6.20915685731194e-09
3923 6.20859060591894e-09
3924 6.20809175364001e-09
3925 6.20752213402465e-09
3926 6.20698236956929e-09
3927 6.20641297043034e-09
3928 6.20586125342493e-09
3929 6.20532389676576e-09
3930 6.20476969369343e-09
3931 6.20427272790547e-09
3932 6.20366177193909e-09
3933 6.20311403976004e-09
3934 6.20255053783436e-09
3935 6.20206854692884e-09
3936 6.20151625738141e-09
3937 6.20092794223481e-09
3938 6.20039186444155e-09
3939 6.19986735662936e-09
3940 6.19929927662966e-09
3941 6.19876349766679e-09
3942 6.19821967794143e-09
3943 6.1976904645461e-09
3944 6.19711540617018e-09
3945 6.19658182508809e-09
3946 6.19603728603535e-09
3947 6.19547662504549e-09
3948 6.1949358947655e-09
3949 6.19442715354968e-09
3950 6.19385709524356e-09
3951 6.19327670853265e-09
3952 6.19272917559699e-09
3953 6.19220667326392e-09
3954 6.19167177791369e-09
3955 6.19115652018831e-09
3956 6.19059588967408e-09
3957 6.19003673137108e-09
3958 6.18950189235079e-09
3959 6.18894913395618e-09
3960 6.18838979267455e-09
3961 6.18784795444749e-09
3962 6.1872871998242e-09
3963 6.18678039449039e-09
3964 6.18622306396144e-09
3965 6.18568817534193e-09
3966 6.18514059079478e-09
3967 6.18460919568631e-09
3968 6.18409376350326e-09
3969 6.1835009700778e-09
3970 6.1829911143646e-09
3971 6.18242755383303e-09
3972 6.18188824409727e-09
3973 6.18136171999517e-09
3974 6.18080668010612e-09
3975 6.18026768410551e-09
3976 6.17972297889402e-09
3977 6.17916327357027e-09
3978 6.17864413174352e-09
3979 6.17807142724858e-09
3980 6.17755155127297e-09
3981 6.17702317977364e-09
3982 6.1764400699077e-09
3983 6.17595913576796e-09
3984 6.17538307927379e-09
3985 6.17484276943525e-09
3986 6.17429404312086e-09
3987 6.17374420118499e-09
3988 6.17321211836275e-09
3989 6.17268412542171e-09
3990 6.17213999762334e-09
3991 6.17159568197523e-09
3992 6.1710730013681e-09
3993 6.17046219653084e-09
3994 6.1699682344929e-09
3995 6.16942594583447e-09
3996 6.16887229637486e-09
3997 6.16831794192363e-09
3998 6.16779404978562e-09
3999 6.16728201403849e-09
4000 6.16672020245512e-09
4001 6.16613418832901e-09
4002 6.16563159122618e-09
4003 6.16507723508186e-09
4004 6.16451834491161e-09
4005 6.1639945931613e-09
4006 6.16342054396424e-09
4007 6.16290524037277e-09
4008 6.16237156991772e-09
4009 6.16182509453889e-09
4010 6.16126487351654e-09
4011 6.16074962668522e-09
4012 6.16018861727963e-09
4013 6.15967386187466e-09
4014 6.15914938335849e-09
4015 6.15860188019068e-09
4016 6.15808047435551e-09
4017 6.15754418860359e-09
4018 6.15698389302977e-09
4019 6.15646956114713e-09
4020 6.1559172378628e-09
4021 6.15535113952792e-09
4022 6.15482086852026e-09
4023 6.1542744186488e-09
4024 6.15375705408094e-09
4025 6.15318612300075e-09
4026 6.15264981451702e-09
4027 6.1521102921458e-09
4028 6.15156916132509e-09
4029 6.15104237769448e-09
4030 6.1504846503857e-09
4031 6.14995356193471e-09
4032 6.14939868344433e-09
4033 6.14887774137707e-09
4034 6.14832541596944e-09
4035 6.14781164146039e-09
4036 6.14724069071537e-09
4037 6.14670618882818e-09
4038 6.14618335638417e-09
4039 6.14567341382377e-09
4040 6.14505588150682e-09
4041 6.14453607901411e-09
4042 6.14399799853116e-09
4043 6.14345532362615e-09
4044 6.14290615576218e-09
4045 6.14238081582397e-09
4046 6.1418485994974e-09
4047 6.14132050549832e-09
4048 6.1407634989602e-09
4049 6.14022665246239e-09
4050 6.13967477819388e-09
4051 6.13917691970334e-09
4052 6.13863053797181e-09
4053 6.13807751401185e-09
4054 6.13755524481174e-09
4055 6.13700251975158e-09
4056 6.13648011119072e-09
4057 6.13592150483511e-09
4058 6.13540820064429e-09
4059 6.13490367105951e-09
4060 6.13433338479685e-09
4061 6.13380503289296e-09
4062 6.13327058007762e-09
4063 6.13270157145962e-09
4064 6.13220793138636e-09
4065 6.13164223049745e-09
4066 6.13111327009419e-09
4067 6.13056683373969e-09
4068 6.13004730590228e-09
4069 6.12950955750091e-09
4070 6.12897371747578e-09
4071 6.12842874962716e-09
4072 6.127906571049e-09
4073 6.12737061540802e-09
4074 6.1268079521587e-09
4075 6.12626543221306e-09
4076 6.12574292392642e-09
4077 6.1252187506522e-09
4078 6.12466883455343e-09
4079 6.12413286361913e-09
4080 6.12358663959478e-09
4081 6.12306279626496e-09
4082 6.12250230642986e-09
4083 6.12202516732385e-09
4084 6.12144445209795e-09
4085 6.12097208677054e-09
4086 6.12038531742909e-09
4087 6.11985771896417e-09
4088 6.11931601648963e-09
4089 6.11879413812966e-09
4090 6.11824369670111e-09
4091 6.11771888087143e-09
4092 6.11716131293516e-09
4093 6.11664614008633e-09
4094 6.11609800782453e-09
4095 6.11558390228861e-09
4096 6.11507206582651e-09
4097 6.11450766312138e-09
4098 6.11398166028287e-09
4099 6.11341999047499e-09
4100 6.11289338690868e-09
4101 6.11236761320633e-09
4102 6.11182129074683e-09
4103 6.11127470792616e-09
4104 6.11077228825085e-09
4105 6.11023793983811e-09
4106 6.10969101522141e-09
4107 6.10914706199173e-09
4108 6.10857394132969e-09
4109 6.1080845562278e-09
4110 6.10754017373039e-09
4111 6.10699467432085e-09
4112 6.10644333703947e-09
4113 6.10596571799793e-09
4114 6.10539393026965e-09
4115 6.10487253865921e-09
4116 6.10433922385023e-09
4117 6.10376263641965e-09
4118 6.10327780904696e-09
4119 6.10272879872364e-09
4120 6.10216452878731e-09
4121 6.10166939364609e-09
4122 6.10114191980371e-09
4123 6.10057118867469e-09
4124 6.1000686405327e-09
4125 6.09948779368985e-09
4126 6.09897348798072e-09
4127 6.09841604717887e-09
4128 6.09792118994035e-09
4129 6.09738634550772e-09
4130 6.09685907301816e-09
4131 6.09631492075324e-09
4132 6.09577977368769e-09
4133 6.09524893525892e-09
4134 6.09468398360402e-09
4135 6.0941318790475e-09
4136 6.09365384471316e-09
4137 6.09308646687401e-09
4138 6.09256159042615e-09
4139 6.09200193212034e-09
4140 6.09149425634392e-09
4141 6.09093146393402e-09
4142 6.09039106665155e-09
4143 6.08992604386738e-09
4144 6.08935965690227e-09
4145 6.08875737448389e-09
4146 6.08826788457695e-09
4147 6.08774772425935e-09
4148 6.08722013807628e-09
4149 6.08666468147889e-09
4150 6.08617194376715e-09
4151 6.08561854821554e-09
4152 6.08509990737693e-09
4153 6.08459064714573e-09
4154 6.08398795817755e-09
4155 6.08347416312938e-09
4156 6.08296406828801e-09
4157 6.08240552538164e-09
4158 6.08193031298443e-09
4159 6.08136085256117e-09
4160 6.08088552078723e-09
4161 6.08032742938081e-09
4162 6.07981480985065e-09
4163 6.07928912391142e-09
4164 6.07874981986556e-09
4165 6.07820904090228e-09
4166 6.07765222010448e-09
4167 6.07710615067869e-09
4168 6.0765887559544e-09
4169 6.07608590706687e-09
4170 6.07553266535055e-09
4171 6.07502316853081e-09
4172 6.07447181837084e-09
4173 6.07394314738885e-09
4174 6.07343424732787e-09
4175 6.07284390320706e-09
4176 6.07236347570372e-09
4177 6.07184018497353e-09
4178 6.07130770391429e-09
4179 6.07077793987609e-09
4180 6.07027149918116e-09
4181 6.06973702357849e-09
4182 6.06921657728132e-09
4183 6.06865575018822e-09
4184 6.06811228914816e-09
4185 6.0676050713665e-09
4186 6.06708591271987e-09
4187 6.06653441775906e-09
4188 6.06603130597072e-09
4189 6.06548395168383e-09
4190 6.06498549549084e-09
4191 6.06447063611348e-09
4192 6.06389232048199e-09
4193 6.06338703532106e-09
4194 6.06284863437223e-09
4195 6.06230845283384e-09
4196 6.06179621984171e-09
4197 6.06123254584534e-09
4198 6.06076064546546e-09
4199 6.06017996181152e-09
4200 6.05968625917719e-09
4201 6.05912835900668e-09
4202 6.05864067598771e-09
4203 6.05806866452174e-09
4204 6.05757556917941e-09
4205 6.05703326263252e-09
4206 6.05651095815507e-09
4207 6.05597355329834e-09
4208 6.05544222072318e-09
4209 6.05490627100802e-09
4210 6.05442348493301e-09
4211 6.05385151540572e-09
4212 6.05335531646267e-09
4213 6.05280413573661e-09
4214 6.05231406995699e-09
4215 6.05177634865894e-09
4216 6.05122584064477e-09
4217 6.05073615986274e-09
4218 6.05016713607631e-09
4219 6.04965715876593e-09
4220 6.04914626865793e-09
4221 6.04860599252854e-09
4222 6.04810602763961e-09
4223 6.04757382018095e-09
4224 6.04704858224447e-09
4225 6.04652747694667e-09
4226 6.04599942256867e-09
4227 6.04547396854116e-09
4228 6.04493244381332e-09
4229 6.04441677524992e-09
4230 6.04386155950154e-09
4231 6.04333283574232e-09
4232 6.04282218788499e-09
4233 6.04231797789179e-09
4234 6.04179620668221e-09
4235 6.04127084503914e-09
4236 6.04073473831068e-09
4237 6.04022495882817e-09
4238 6.03966417005164e-09
4239 6.03915820346357e-09
4240 6.03865247270074e-09
4241 6.03809719579296e-09
4242 6.03757900610735e-09
4243 6.03706856439068e-09
4244 6.03654935758813e-09
4245 6.03600249932112e-09
4246 6.03548284913713e-09
4247 6.03492802776773e-09
4248 6.03442526435349e-09
4249 6.03391143559617e-09
4250 6.0333745513369e-09
4251 6.03285408891374e-09
4252 6.03232924729913e-09
4253 6.03181683180021e-09
4254 6.031275373658e-09
4255 6.03077361473192e-09
4256 6.03023894578392e-09
4257 6.0297054796099e-09
4258 6.02915241167124e-09
4259 6.02865131164865e-09
4260 6.02813902846056e-09
4261 6.02762880740071e-09
4262 6.02708464123025e-09
4263 6.02655722457823e-09
4264 6.02601988285156e-09
4265 6.02550979433847e-09
4266 6.02502021547491e-09
4267 6.02448002884337e-09
4268 6.0239443243626e-09
4269 6.02343094777136e-09
4270 6.02286880614644e-09
4271 6.02237709740328e-09
4272 6.02185079733031e-09
4273 6.0212996696174e-09
4274 6.02081019904221e-09
4275 6.02030848474711e-09
4276 6.01975950265121e-09
4277 6.01923227482037e-09
4278 6.01870830636841e-09
4279 6.01818967919943e-09
4280 6.01764366479907e-09
4281 6.0171183074581e-09
4282 6.01665104547899e-09
4283 6.01608952986721e-09
4284 6.01559022575526e-09
4285 6.0150750564314e-09
4286 6.01454708915039e-09
4287 6.01402550545749e-09
4288 6.01348664104606e-09
4289 6.01295912047717e-09
4290 6.01240778551337e-09
4291 6.01190314929168e-09
4292 6.01141556864915e-09
4293 6.01086708985898e-09
4294 6.01034556266256e-09
4295 6.00980391728123e-09
4296 6.00930244447351e-09
4297 6.00878295901885e-09
4298 6.00822506315046e-09
4299 6.00775351268512e-09
4300 6.00719655693971e-09
4301 6.00669416085664e-09
4302 6.0061850557791e-09
4303 6.00562751508493e-09
4304 6.00511576501206e-09
4305 6.00459993017888e-09
4306 6.00406787228114e-09
4307 6.00355446468692e-09
4308 6.00303023581827e-09
4309 6.00249851517853e-09
4310 6.00202302004915e-09
4311 6.0014647957074e-09
4312 6.00092739218294e-09
4313 6.00043325700572e-09
4314 5.99990016238172e-09
4315 5.99936597306394e-09
4316 5.99885052539328e-09
4317 5.99835766054713e-09
4318 5.99784998366049e-09
4319 5.99726924546684e-09
4320 5.99681727347923e-09
4321 5.99626774760997e-09
4322 5.99577610613244e-09
4323 5.99522652357243e-09
4324 5.99469295542443e-09
4325 5.99416925013696e-09
4326 5.99367482344293e-09
4327 5.99315089293284e-09
4328 5.99257996333757e-09
4329 5.99211571118696e-09
4330 5.99159447065012e-09
4331 5.99108661035463e-09
4332 5.99054479996641e-09
4333 5.99003590394387e-09
4334 5.98951042424245e-09
4335 5.98901333939694e-09
4336 5.98846670135655e-09
4337 5.98797963112518e-09
4338 5.98744383600858e-09
4339 5.98692201134177e-09
4340 5.98642270283056e-09
4341 5.98591367338697e-09
4342 5.98539184620828e-09
4343 5.98483662035687e-09
4344 5.98434100868017e-09
4345 5.98382727971802e-09
4346 5.98328914569457e-09
4347 5.98279511844491e-09
4348 5.98226572752492e-09
4349 5.98175894689357e-09
4350 5.9812475115828e-09
4351 5.98073314191094e-09
4352 5.98018850159199e-09
4353 5.97972834467941e-09
4354 5.97916968038403e-09
4355 5.97865180171353e-09
4356 5.97812925685892e-09
4357 5.97759612479265e-09
4358 5.97708657514018e-09
4359 5.9765733132211e-09
4360 5.97606338741119e-09
4361 5.9755476686657e-09
4362 5.97504306208696e-09
4363 5.97452903992879e-09
4364 5.97396739669687e-09
4365 5.97347608857768e-09
4366 5.97295834302292e-09
4367 5.97241834374151e-09
4368 5.97188392886805e-09
4369 5.97138788990814e-09
4370 5.9708553990373e-09
4371 5.97033250719636e-09
4372 5.96985512456294e-09
4373 5.96931936436285e-09
4374 5.96882527481579e-09
4375 5.96825565475634e-09
4376 5.96777487632538e-09
4377 5.96722508744429e-09
4378 5.96674836363109e-09
4379 5.96621308962542e-09
4380 5.96567522290203e-09
4381 5.96516143681891e-09
4382 5.96467603841688e-09
4383 5.96417382575654e-09
4384 5.96360829739628e-09
4385 5.96311317628551e-09
4386 5.96256500633163e-09
4387 5.96207872546883e-09
4388 5.96153943280275e-09
4389 5.9610506304153e-09
4390 5.96050147454175e-09
4391 5.96001961418457e-09
4392 5.95948779737176e-09
4393 5.95897624838804e-09
4394 5.95845917748805e-09
4395 5.95796460764464e-09
4396 5.95739567350872e-09
4397 5.95691839520851e-09
4398 5.95639972103545e-09
4399 5.95587331805869e-09
4400 5.95536254911766e-09
4401 5.95486254831301e-09
4402 5.95433255329292e-09
4403 5.95381703706599e-09
4404 5.95330370198321e-09
4405 5.95276607427697e-09
4406 5.95227111394037e-09
4407 5.95176145117005e-09
4408 5.95123157365318e-09
4409 5.95072260178853e-09
4410 5.95019293314625e-09
4411 5.94969760743525e-09
4412 5.94913780188611e-09
4413 5.94864311125043e-09
4414 5.94811845655574e-09
4415 5.94758735612821e-09
4416 5.94710595518133e-09
4417 5.94656684715922e-09
4418 5.94609960127834e-09
4419 5.94555214464276e-09
4420 5.9450399980121e-09
4421 5.94452435875858e-09
4422 5.94398464021095e-09
4423 5.94354527962138e-09
4424 5.94298165085272e-09
4425 5.94248493761274e-09
4426 5.94197631632876e-09
4427 5.94141989136709e-09
4428 5.94088556529759e-09
4429 5.94040570464638e-09
4430 5.93987510670579e-09
4431 5.93937022003166e-09
4432 5.938854632348e-09
4433 5.93835021135691e-09
4434 5.937795989161e-09
4435 5.93728333680987e-09
4436 5.93680541076391e-09
4437 5.93623214578676e-09
4438 5.93578314488086e-09
4439 5.93526558853585e-09
4440 5.93472094885528e-09
4441 5.93424959378919e-09
4442 5.93371971485679e-09
4443 5.93318975317114e-09
4444 5.9327135238374e-09
4445 5.93214822737498e-09
4446 5.93161901406292e-09
4447 5.93115987079784e-09
4448 5.93062908435527e-09
4449 5.93005675891822e-09
4450 5.92958581263625e-09
4451 5.92908952923299e-09
4452 5.92858281646402e-09
4453 5.92805199386981e-09
4454 5.92752000408425e-09
4455 5.92704352522788e-09
4456 5.92648711938981e-09
4457 5.92597352427893e-09
4458 5.9254868394476e-09
4459 5.92494682495615e-09
4460 5.92442761547518e-09
4461 5.92391648560064e-09
4462 5.92340143519554e-09
4463 5.92286859497915e-09
4464 5.92236491946119e-09
4465 5.92186997791511e-09
4466 5.92135409351047e-09
4467 5.92086350292842e-09
4468 5.9202952683024e-09
4469 5.91982131885327e-09
4470 5.91928671138386e-09
4471 5.91876340337583e-09
4472 5.91826422255415e-09
4473 5.91776350035433e-09
4474 5.91725297999224e-09
4475 5.91672814428956e-09
4476 5.91621115873797e-09
4477 5.91567700682083e-09
4478 5.91517730900992e-09
4479 5.91466428023768e-09
4480 5.9142052219463e-09
4481 5.91367971400358e-09
4482 5.91315626446987e-09
4483 5.91263241203621e-09
4484 5.91216523512794e-09
4485 5.91160083890374e-09
4486 5.91110760264635e-09
4487 5.91060280420719e-09
4488 5.91009425561506e-09
4489 5.90957777850398e-09
4490 5.90904324840325e-09
4491 5.90856935006601e-09
4492 5.90801704197785e-09
4493 5.90752289003627e-09
4494 5.90703179352559e-09
4495 5.90648611217826e-09
4496 5.9060222161178e-09
4497 5.90549326064116e-09
4498 5.90498659709671e-09
4499 5.90449042091323e-09
4500 5.9039688784096e-09
4501 5.90345225125188e-09
4502 5.90295874335067e-09
4503 5.90241307749095e-09
4504 5.90192046884264e-09
4505 5.90139553696689e-09
4506 5.9009048508224e-09
4507 5.90040368368683e-09
4508 5.8998788500797e-09
4509 5.89936714068262e-09
4510 5.89888901753044e-09
4511 5.89837676652494e-09
4512 5.89789195183654e-09
4513 5.89734498858407e-09
4514 5.89686098684261e-09
4515 5.89633598674366e-09
4516 5.89582879910455e-09
4517 5.89524953743814e-09
4518 5.89482021731702e-09
4519 5.89429672709363e-09
4520 5.89378227965065e-09
4521 5.89329037084529e-09
4522 5.89274001053242e-09
4523 5.89226698982648e-09
4524 5.89173193478454e-09
4525 5.89125170807891e-09
4526 5.89070710568795e-09
4527 5.8902184277565e-09
4528 5.88973272705462e-09
4529 5.88920542715643e-09
4530 5.88870274463582e-09
4531 5.88816640686995e-09
4532 5.88769109358123e-09
4533 5.88717446919906e-09
4534 5.88666624130874e-09
4535 5.88616181167179e-09
4536 5.88563360313932e-09
4537 5.88514729170375e-09
4538 5.88463768713687e-09
4539 5.88413236264629e-09
4540 5.883626041564e-09
4541 5.88304768092685e-09
4542 5.8825982271471e-09
4543 5.88206753869558e-09
4544 5.8815954484398e-09
4545 5.88104219731433e-09
4546 5.88059474870839e-09
4547 5.8800898620065e-09
4548 5.87954791093914e-09
4549 5.87904172942577e-09
4550 5.87852421488066e-09
4551 5.87803848121904e-09
4552 5.87752611817816e-09
4553 5.8769958705962e-09
4554 5.87650580803622e-09
4555 5.87599876686329e-09
4556 5.8754675456435e-09
4557 5.8749738669206e-09
4558 5.87443795523057e-09
4559 5.8739836141658e-09
4560 5.87343092300907e-09
4561 5.87295004519928e-09
4562 5.87244581626289e-09
4563 5.87191709405799e-09
4564 5.87143985837646e-09
4565 5.87091344302071e-09
4566 5.87038638759363e-09
4567 5.86990456380443e-09
4568 5.86941292758658e-09
4569 5.86885514598789e-09
4570 5.86837363994452e-09
4571 5.86788136654193e-09
4572 5.86737057249598e-09
4573 5.86685951871335e-09
4574 5.86637538796397e-09
4575 5.86579755768035e-09
4576 5.86532776918058e-09
4577 5.86483248306291e-09
4578 5.86428374628467e-09
4579 5.86378118311304e-09
4580 5.86331006535712e-09
4581 5.86282780921932e-09
4582 5.86227701340758e-09
4583 5.86181664996577e-09
4584 5.86131169269533e-09
4585 5.86079320649691e-09
4586 5.86026383617155e-09
4587 5.85978347474037e-09
4588 5.85927232360506e-09
4589 5.85875674563585e-09
4590 5.85824769507026e-09
4591 5.85770937727714e-09
4592 5.85723105162028e-09
4593 5.85669661747057e-09
4594 5.85621914184209e-09
4595 5.85572075499641e-09
4596 5.85522054913357e-09
4597 5.85468644567766e-09
4598 5.85418802719062e-09
4599 5.85371588640582e-09
4600 5.85317879461811e-09
4601 5.85267691770308e-09
4602 5.85217337392696e-09
4603 5.85162526595129e-09
4604 5.85117682147529e-09
4605 5.85065243899341e-09
4606 5.85016349173573e-09
4607 5.84963017179196e-09
4608 5.84914421201954e-09
4609 5.84860907743012e-09
4610 5.84816424080425e-09
4611 5.84763623980022e-09
4612 5.84713683145222e-09
4613 5.84664454268691e-09
4614 5.84612526653705e-09
4615 5.84562455933912e-09
4616 5.84512748884325e-09
4617 5.84463080630093e-09
4618 5.84408513795709e-09
4619 5.84361831913738e-09
4620 5.84310288918866e-09
4621 5.84260466278408e-09
4622 5.8421109332546e-09
4623 5.84160900997388e-09
4624 5.84106658881034e-09
4625 5.84058084696082e-09
4626 5.84010511114896e-09
4627 5.83956260098717e-09
4628 5.83907482176738e-09
4629 5.83855326066751e-09
4630 5.83802770202924e-09
4631 5.8375617967843e-09
4632 5.83701548169391e-09
4633 5.83653198725498e-09
4634 5.83603386970777e-09
4635 5.83552053454173e-09
4636 5.83503032469679e-09
4637 5.83452513365501e-09
4638 5.83405300845496e-09
4639 5.83350967808816e-09
4640 5.8330607378837e-09
4641 5.83253318266197e-09
4642 5.83203468770832e-09
4643 5.83148090919916e-09
4644 5.83102053205997e-09
4645 5.83049106879507e-09
4646 5.82999972424669e-09
4647 5.82951335577342e-09
4648 5.82902561419019e-09
4649 5.82847392063823e-09
4650 5.82800796658511e-09
4651 5.82752367020434e-09
4652 5.82700564580318e-09
4653 5.82647194835584e-09
4654 5.82598468977513e-09
4655 5.82546448692212e-09
4656 5.82500395289454e-09
4657 5.82447668219521e-09
4658 5.82394631766514e-09
4659 5.82349205643928e-09
4660 5.82301630200344e-09
4661 5.82246809958942e-09
4662 5.82197137619089e-09
4663 5.8214746850721e-09
4664 5.82097283487182e-09
4665 5.82046991103036e-09
4666 5.81995558485149e-09
4667 5.81943328903378e-09
4668 5.81895228358997e-09
4669 5.81846647583484e-09
4670 5.8179321408558e-09
4671 5.81744949469665e-09
4672 5.81692576590021e-09
4673 5.81642520605663e-09
4674 5.81592098988781e-09
4675 5.81539823624189e-09
4676 5.81492994461419e-09
4677 5.8144314377534e-09
4678 5.81393043998235e-09
4679 5.81341151853487e-09
4680 5.81291656502614e-09
4681 5.81242148441075e-09
4682 5.81189262677251e-09
4683 5.81143127197092e-09
4684 5.81090293461095e-09
4685 5.8104079115745e-09
4686 5.80993336032898e-09
4687 5.80939960413696e-09
4688 5.80891898917246e-09
4689 5.80845701678157e-09
4690 5.80793008136959e-09
4691 5.80740341551977e-09
4692 5.80692092994051e-09
4693 5.80640174367708e-09
4694 5.80590900856048e-09
4695 5.80538253597274e-09
4696 5.80487337430158e-09
4697 5.80440690542416e-09
4698 5.80391948178105e-09
4699 5.80339957213793e-09
4700 5.8029149897082e-09
4701 5.80241294881323e-09
4702 5.80190986788909e-09
4703 5.80145007221533e-09
4704 5.80091316132458e-09
4705 5.80042071271492e-09
4706 5.79991931133617e-09
4707 5.79939381717409e-09
4708 5.79891024214685e-09
4709 5.79843464013074e-09
4710 5.79789641302897e-09
4711 5.79738782294226e-09
4712 5.79691931733295e-09
4713 5.79642497483546e-09
4714 5.79591203896113e-09
4715 5.7954081769257e-09
4716 5.79492817340266e-09
4717 5.79443488323006e-09
4718 5.7939245421551e-09
4719 5.79342308132391e-09
4720 5.79292754246397e-09
4721 5.7924184493352e-09
4722 5.791889575682e-09
4723 5.79143498580237e-09
4724 5.79092957186944e-09
4725 5.79040919539142e-09
4726 5.78991407415574e-09
4727 5.78944761386868e-09
4728 5.78894172781341e-09
4729 5.78843560608555e-09
4730 5.78792607551504e-09
4731 5.78743538395821e-09
4732 5.78691212467508e-09
4733 5.78644158441555e-09
4734 5.78592464245409e-09
4735 5.7854499252441e-09
4736 5.78493714811779e-09
4737 5.78444624833863e-09
4738 5.78393153594092e-09
4739 5.78345242800704e-09
4740 5.78293717752587e-09
4741 5.78244913250481e-09
4742 5.78194110451014e-09
4743 5.78147145505192e-09
4744 5.78097441161773e-09
4745 5.78043346914636e-09
4746 5.77994983579078e-09
4747 5.77946608888713e-09
4748 5.77896010876822e-09
4749 5.77848677875414e-09
4750 5.77797736568686e-09
4751 5.77747871778611e-09
4752 5.77696749631818e-09
4753 5.77648107479012e-09
4754 5.77596653426882e-09
4755 5.77548514643644e-09
4756 5.77497728927734e-09
4757 5.77448189469088e-09
4758 5.77404759603828e-09
4759 5.7734747246907e-09
4760 5.77300424914329e-09
4761 5.77253238961961e-09
4762 5.77202217376394e-09
4763 5.77150753784672e-09
4764 5.77103027696313e-09
4765 5.77052668079836e-09
4766 5.77006011825976e-09
4767 5.76949777267299e-09
4768 5.76904328303263e-09
4769 5.76853464424876e-09
4770 5.76803298768525e-09
4771 5.76754849637706e-09
4772 5.76707368059315e-09
4773 5.76654162473544e-09
4774 5.76610660571464e-09
4775 5.76559130462118e-09
4776 5.76509655868251e-09
4777 5.76455467526937e-09
4778 5.76404015537046e-09
4779 5.76358379106301e-09
4780 5.76310091937593e-09
4781 5.76259737818108e-09
4782 5.76208515726262e-09
4783 5.76159256461539e-09
4784 5.76108238255213e-09
4785 5.76065138897797e-09
4786 5.76011575699475e-09
4787 5.75964375972016e-09
4788 5.75911215129621e-09
4789 5.75861460690164e-09
4790 5.75813589503982e-09
4791 5.75763684938779e-09
4792 5.75714771336444e-09
4793 5.75662885042572e-09
4794 5.75614940138969e-09
4795 5.75572472312491e-09
4796 5.75514022208012e-09
4797 5.75468627918296e-09
4798 5.75418281112405e-09
4799 5.75365008320672e-09
4800 5.7531760635221e-09
4801 5.75265864334595e-09
4802 5.75217708570497e-09
4803 5.75168996783115e-09
4804 5.75118807365216e-09
4805 5.75069612107626e-09
4806 5.75019316712e-09
4807 5.74965957209073e-09
4808 5.74918420850889e-09
4809 5.748670388106e-09
4810 5.74820218621208e-09
4811 5.74772249556377e-09
4812 5.74720463887568e-09
4813 5.74669089459245e-09
4814 5.74621420824928e-09
4815 5.74569036858319e-09
4816 5.74521106749826e-09
4817 5.7447130782512e-09
4818 5.74423711131866e-09
4819 5.74373911348125e-09
4820 5.74322229929258e-09
4821 5.74274540186825e-09
4822 5.74221988271229e-09
4823 5.74175178898184e-09
4824 5.74124771900164e-09
4825 5.74075873815971e-09
4826 5.74028859870457e-09
4827 5.73975770272461e-09
4828 5.73924514050972e-09
4829 5.73875509879418e-09
4830 5.73825859571941e-09
4831 5.73777616862114e-09
4832 5.73729909721099e-09
4833 5.73674384510281e-09
4834 5.73626680959449e-09
4835 5.73577227970523e-09
4836 5.73530225139729e-09
4837 5.73480630484957e-09
4838 5.73428420796995e-09
4839 5.73381284808827e-09
4840 5.73332739868537e-09
4841 5.73281498254807e-09
4842 5.73232552336655e-09
4843 5.73181145141488e-09
4844 5.73132951801891e-09
4845 5.73081040490531e-09
4846 5.73032240379356e-09
4847 5.72985528478343e-09
4848 5.72933867672154e-09
4849 5.72885030687698e-09
4850 5.72835198676958e-09
4851 5.72785550126409e-09
4852 5.72732372122742e-09
4853 5.72685437814913e-09
4854 5.72635941362143e-09
4855 5.7258771616886e-09
4856 5.72538914225818e-09
4857 5.72489927107289e-09
4858 5.72439689203175e-09
4859 5.72389981765009e-09
4860 5.72338389669136e-09
4861 5.72292604542057e-09
4862 5.72242598739781e-09
4863 5.72193665582255e-09
4864 5.72145872175522e-09
4865 5.72093458678369e-09
4866 5.72043792934629e-09
4867 5.71995205581044e-09
4868 5.71945505616067e-09
4869 5.71898692372508e-09
4870 5.71843416168061e-09
4871 5.71797592464895e-09
4872 5.71748418541629e-09
4873 5.7169853766581e-09
4874 5.71645978611479e-09
4875 5.71597471143992e-09
4876 5.71548408707934e-09
4877 5.71501532369012e-09
4878 5.71450799219386e-09
4879 5.71401690183104e-09
4880 5.71351095626782e-09
4881 5.71303536618661e-09
4882 5.71253779181602e-09
4883 5.71203156604638e-09
4884 5.71151989503527e-09
4885 5.7110784403569e-09
4886 5.71057158579857e-09
4887 5.71008517230576e-09
4888 5.70957339086908e-09
4889 5.70909147580567e-09
4890 5.70860080609248e-09
4891 5.70812463272785e-09
4892 5.70760877735554e-09
4893 5.70712058525769e-09
4894 5.70663596163867e-09
4895 5.70607935235223e-09
4896 5.7056154068591e-09
4897 5.70513142167384e-09
4898 5.70466684436666e-09
4899 5.70410068158333e-09
4900 5.70363652728501e-09
4901 5.70312742825818e-09
4902 5.70268637793692e-09
4903 5.70211293821066e-09
4904 5.70169241394247e-09
4905 5.70119880413666e-09
4906 5.70066833378846e-09
4907 5.70015462564299e-09
4908 5.69970396559205e-09
4909 5.69921919588157e-09
4910 5.69869537493661e-09
4911 5.69826691503017e-09
4912 5.69771833014432e-09
4913 5.69720584414624e-09
4914 5.69674084896499e-09
4915 5.69627367545678e-09
4916 5.69574710646004e-09
4917 5.69527624245947e-09
4918 5.69479744412515e-09
4919 5.69427869914763e-09
4920 5.69376650193243e-09
4921 5.69330023278414e-09
4922 5.69283981827207e-09
4923 5.69232636876693e-09
4924 5.69181841887645e-09
4925 5.69133697658764e-09
4926 5.69086642837613e-09
4927 5.69032857616891e-09
4928 5.68986868310084e-09
4929 5.68936259355557e-09
4930 5.68888360730213e-09
4931 5.6883772007188e-09
4932 5.68788608883153e-09
4933 5.68741089661262e-09
4934 5.68694239705403e-09
4935 5.68643031702287e-09
4936 5.68597106262447e-09
4937 5.68544228858603e-09
4938 5.68495219942233e-09
4939 5.68444562178139e-09
4940 5.68396237433932e-09
4941 5.68348204132985e-09
4942 5.68299159593721e-09
4943 5.68249519208863e-09
4944 5.68203026461711e-09
4945 5.68150581790883e-09
4946 5.68104008841219e-09
4947 5.68055660896127e-09
4948 5.68006475744343e-09
4949 5.67955906567719e-09
4950 5.6790512460575e-09
4951 5.67856689191737e-09
4952 5.67809983482992e-09
4953 5.67756847619227e-09
4954 5.67712430138423e-09
4955 5.67664872470897e-09
4956 5.67614533360239e-09
4957 5.67565569267736e-09
4958 5.67518560097569e-09
4959 5.67465929038335e-09
4960 5.67414774924058e-09
4961 5.67365680226306e-09
4962 5.67320626464196e-09
4963 5.67274158301545e-09
4964 5.67220014331682e-09
4965 5.67171260885957e-09
4966 5.6712237670592e-09
4967 5.6707466180167e-09
4968 5.67023308473136e-09
4969 5.6697558949298e-09
4970 5.66926333772644e-09
4971 5.66878012778216e-09
4972 5.66826407930043e-09
4973 5.66781265515237e-09
4974 5.66731794009179e-09
4975 5.66685238433118e-09
4976 5.66635803860016e-09
4977 5.66582984229402e-09
4978 5.66533558599147e-09
4979 5.66486625375173e-09
4980 5.66437738050429e-09
4981 5.66389321804417e-09
4982 5.66341614509358e-09
4983 5.66291202112879e-09
4984 5.66244044086783e-09
4985 5.66196397293328e-09
4986 5.66146362723785e-09
4987 5.66096192708421e-09
4988 5.66049207632868e-09
4989 5.65996881403408e-09
4990 5.65948032919816e-09
4991 5.65899979412809e-09
4992 5.65851128607464e-09
4993 5.65800969690167e-09
4994 5.65751866166142e-09
4995 5.65700471423514e-09
4996 5.6565535464792e-09
4997 5.65605906802102e-09
4998 5.65555511165827e-09
4999 5.65506399112514e-09
};
\addlegendentry{Train}
\addplot [semithick, black]
table {%
0 0.00191779050510377
1 0.000294900150038302
2 0.000241628382354975
3 0.000229538694838993
4 0.000188198479008861
5 5.52264355064835e-05
6 1.83738829946378e-05
7 1.804286512197e-05
8 1.79411526914919e-05
9 1.77913352672476e-05
10 1.7625325199333e-05
11 1.7460746676079e-05
12 1.72935870068613e-05
13 1.71192095876904e-05
14 1.69283994182479e-05
15 1.67138405231526e-05
16 1.64650737133343e-05
17 1.616646295588e-05
18 1.57897684402997e-05
19 1.52973443618976e-05
20 1.4638360880781e-05
21 1.37287097459193e-05
22 1.24557209346676e-05
23 1.07071409729542e-05
24 8.46728016767884e-06
25 6.04208980803378e-06
26 4.05075388698606e-06
27 2.92566733151034e-06
28 2.46860145125538e-06
29 2.3126881387725e-06
30 2.25460621550155e-06
31 2.20989841182018e-06
32 2.17577826333581e-06
33 2.14634565054439e-06
34 2.11789279092045e-06
35 2.09041104426433e-06
36 2.06384606826759e-06
37 2.03784497898596e-06
38 2.0122042769799e-06
39 1.98659336092533e-06
40 1.96069890989747e-06
41 1.93421055882936e-06
42 1.90674938949087e-06
43 1.87794125849905e-06
44 1.84741918474174e-06
45 1.81491702733183e-06
46 1.78015136498288e-06
47 1.74266949670709e-06
48 1.70188036463514e-06
49 1.65722951805947e-06
50 1.60819104166876e-06
51 1.55408690716285e-06
52 1.49377069647016e-06
53 1.42729766139382e-06
54 1.35429525016662e-06
55 1.27421367324132e-06
56 1.18882599053904e-06
57 1.09960387817409e-06
58 1.00959209703433e-06
59 9.22397191516211e-07
60 8.40993322981376e-07
61 7.70315182307968e-07
62 7.12188580109796e-07
63 6.66760968215385e-07
64 6.31300508757704e-07
65 6.04208821641805e-07
66 5.82674715587927e-07
67 5.65897323667741e-07
68 5.52775532014493e-07
69 5.41434076239966e-07
70 5.31310774931626e-07
71 5.2213232493159e-07
72 5.13754912390141e-07
73 5.06071728523239e-07
74 4.99012116961239e-07
75 4.92505591864756e-07
76 4.86470298710628e-07
77 4.80830578908353e-07
78 4.75520209874958e-07
79 4.70481040792947e-07
80 4.65686639472551e-07
81 4.61094458614753e-07
82 4.56688979966202e-07
83 4.52445732435081e-07
84 4.48370258254727e-07
85 4.44449796077606e-07
86 4.40680906876878e-07
87 4.3706370433938e-07
88 4.33586876624759e-07
89 4.30241982485313e-07
90 4.27042863293536e-07
91 4.23970419660691e-07
92 4.21026413732761e-07
93 4.18204024299484e-07
94 4.15508225160011e-07
95 4.12931143500828e-07
96 4.10463457001242e-07
97 4.08110111038695e-07
98 4.0585845795249e-07
99 4.03709634611005e-07
100 4.01653636572519e-07
101 3.99690463837032e-07
102 3.97809316154962e-07
103 3.96011557768361e-07
104 3.94289912719614e-07
105 3.92637872437263e-07
106 3.91059188586951e-07
107 3.89548006296536e-07
108 3.88101426551657e-07
109 3.86718880918124e-07
110 3.85403353675429e-07
111 3.84144186682533e-07
112 3.82946126364914e-07
113 3.8180462524906e-07
114 3.80699106017346e-07
115 3.79642330017305e-07
116 3.78643960630143e-07
117 3.77708687437917e-07
118 3.7683332720917e-07
119 3.75999775314995e-07
120 3.75215847725485e-07
121 3.74465798813617e-07
122 3.73750879134604e-07
123 3.73065262238015e-07
124 3.7240386063786e-07
125 3.71760989992254e-07
126 3.71133296539483e-07
127 3.70524958270835e-07
128 3.69923981224929e-07
129 3.69327750604498e-07
130 3.68745190826303e-07
131 3.68165331110504e-07
132 3.67587233540689e-07
133 3.67020419389519e-07
134 3.66455822131684e-07
135 3.65891708042909e-07
136 3.65327707640972e-07
137 3.64760921911511e-07
138 3.64183875944946e-07
139 3.63593642305204e-07
140 3.62988629376559e-07
141 3.62353205218824e-07
142 3.61686971928066e-07
143 3.60990924264115e-07
144 3.60283593181521e-07
145 3.59588199216887e-07
146 3.58934755695373e-07
147 3.58325564775441e-07
148 3.57749854629219e-07
149 3.57187332156172e-07
150 3.56640043719381e-07
151 3.5611515158962e-07
152 3.55571330601379e-07
153 3.55064088353174e-07
154 3.54491589860118e-07
155 3.54002992253299e-07
156 3.53374048245314e-07
157 3.52886900145677e-07
158 3.52203755937808e-07
159 3.51660702335721e-07
160 3.50991200548378e-07
161 3.50440274132779e-07
162 3.49666464671827e-07
163 3.49017341250146e-07
164 3.48296282481897e-07
165 3.47684419921279e-07
166 3.46838874065725e-07
167 3.46159623632047e-07
168 3.45391697464947e-07
169 3.4474936683182e-07
170 3.43890917520184e-07
171 3.43260751378693e-07
172 3.42414352871856e-07
173 3.41787625757206e-07
174 3.40956461286623e-07
175 3.40327801495732e-07
176 3.39526707193727e-07
177 3.38896114726595e-07
178 3.38125943244449e-07
179 3.374238701781e-07
180 3.36802372657985e-07
181 3.36046724669359e-07
182 3.35441313836782e-07
183 3.34732021656237e-07
184 3.34082614017461e-07
185 3.33493119342165e-07
186 3.32820093262853e-07
187 3.32219741494555e-07
188 3.31566724298682e-07
189 3.31018298993513e-07
190 3.30373694623631e-07
191 3.29797046560998e-07
192 3.29231539808461e-07
193 3.2862732268768e-07
194 3.28083046952088e-07
195 3.27491676443969e-07
196 3.2692182116989e-07
197 3.26389312022002e-07
198 3.25803028999871e-07
199 3.25250368859997e-07
200 3.24723714584252e-07
201 3.24129956652541e-07
202 3.23585794603787e-07
203 3.23015200365262e-07
204 3.22465155022655e-07
205 3.21877422493344e-07
206 3.21321664387142e-07
207 3.20721426305681e-07
208 3.20143271892448e-07
209 3.19572166063153e-07
210 3.18957489753302e-07
211 3.18415402489336e-07
212 3.17882125955293e-07
213 3.17485813638996e-07
214 3.17155240736611e-07
215 3.16672384315098e-07
216 3.16189101567943e-07
217 3.15693029051545e-07
218 3.15182916210688e-07
219 3.14690538516516e-07
220 3.14220613972793e-07
221 3.13754128455912e-07
222 3.13230572146495e-07
223 3.12758032805505e-07
224 3.12274636371512e-07
225 3.11814631004381e-07
226 3.11253785412191e-07
227 3.10777608092394e-07
228 3.10308678308502e-07
229 3.097384535522e-07
230 3.09252982333419e-07
231 3.08761855194462e-07
232 3.08216783651005e-07
233 3.07623537310064e-07
234 3.07078408923189e-07
235 3.06517421222452e-07
236 3.05959332536077e-07
237 3.05403517586456e-07
238 3.04844405718541e-07
239 3.04285379115754e-07
240 3.03729592587842e-07
241 3.03173919746769e-07
242 3.02616342651163e-07
243 3.02061522461372e-07
244 3.01510311828679e-07
245 3.00964103416845e-07
246 3.00417326570823e-07
247 2.99885272170286e-07
248 2.99356315736077e-07
249 2.98838529033674e-07
250 2.98329666748032e-07
251 2.97834475304626e-07
252 2.97348321964819e-07
253 2.96876464744855e-07
254 2.9640870025105e-07
255 2.95947245376738e-07
256 2.9549303803833e-07
257 2.95031441055471e-07
258 2.94570469350219e-07
259 2.9410341539915e-07
260 2.93632638204144e-07
261 2.93149014396477e-07
262 2.9266681167428e-07
263 2.92167186444203e-07
264 2.91666793827972e-07
265 2.91166031729517e-07
266 2.90659301072083e-07
267 2.90144186010366e-07
268 2.89629980443351e-07
269 2.891036103847e-07
270 2.8857820666417e-07
271 2.88042116380893e-07
272 2.87504974494368e-07
273 2.86964308315873e-07
274 2.86417673578399e-07
275 2.85870555671863e-07
276 2.85314087022925e-07
277 2.84760744762025e-07
278 2.84195380118035e-07
279 2.83629901787208e-07
280 2.83063712913645e-07
281 2.82486865899045e-07
282 2.81901520793326e-07
283 2.81316886230343e-07
284 2.80715170219992e-07
285 2.80109048844679e-07
286 2.79497271549189e-07
287 2.78874296100184e-07
288 2.78242652029803e-07
289 2.77602822507106e-07
290 2.76950572697388e-07
291 2.76295850198949e-07
292 2.75629247425968e-07
293 2.74950366474513e-07
294 2.74245024911579e-07
295 2.73506657322287e-07
296 2.72807028522948e-07
297 2.7217868137086e-07
298 2.71534929652262e-07
299 2.70876199692793e-07
300 2.70214599140672e-07
301 2.69553510179321e-07
302 2.68886594767537e-07
303 2.6822127097148e-07
304 2.67552479726874e-07
305 2.66889543354409e-07
306 2.66220411049289e-07
307 2.65554916722976e-07
308 2.64875666289299e-07
309 2.64195932686562e-07
310 2.63513953768779e-07
311 2.62831662212193e-07
312 2.62149512764154e-07
313 2.61465771700387e-07
314 2.60789136063977e-07
315 2.60100762261573e-07
316 2.59437513250305e-07
317 2.58710599609913e-07
318 2.58030411259824e-07
319 2.57394759728413e-07
320 2.5667890213299e-07
321 2.56010167731802e-07
322 2.55364227541577e-07
323 2.54689354051152e-07
324 2.54039747460411e-07
325 2.53305813657789e-07
326 2.52669508427061e-07
327 2.52000262435104e-07
328 2.51353299063339e-07
329 2.50650003863484e-07
330 2.50033536985939e-07
331 2.49343401037549e-07
332 2.4872613835214e-07
333 2.48031511773661e-07
334 2.4742880100348e-07
335 2.46726017394394e-07
336 2.46132572101487e-07
337 2.45421603040086e-07
338 2.44837281115906e-07
339 2.44119121362019e-07
340 2.43538778477159e-07
341 2.42799615080003e-07
342 2.42225894453441e-07
343 2.41482581486707e-07
344 2.4090363126561e-07
345 2.40147130625701e-07
346 2.39572671034693e-07
347 2.38813072428457e-07
348 2.38237717553602e-07
349 2.37459360619141e-07
350 2.36883295201551e-07
351 2.36102025041873e-07
352 2.35517418900599e-07
353 2.34727693282366e-07
354 2.34137019106129e-07
355 2.33341538091736e-07
356 2.32745463790707e-07
357 2.31949655926655e-07
358 2.31338759704158e-07
359 2.30550142532593e-07
360 2.29939672635737e-07
361 2.2914366581972e-07
362 2.28531476409444e-07
363 2.27741423941552e-07
364 2.2711580527357e-07
365 2.26326164920465e-07
366 2.25692829758373e-07
367 2.24895430278593e-07
368 2.24233858148182e-07
369 2.23465974613646e-07
370 2.22698432139623e-07
371 2.21996742766351e-07
372 2.2121784581941e-07
373 2.20433420849986e-07
374 2.1963958829474e-07
375 2.18950091834813e-07
376 2.18072585767004e-07
377 2.17262780211058e-07
378 2.16479193682062e-07
379 2.15840159967229e-07
380 2.14936619613582e-07
381 2.14068592185868e-07
382 2.13215727740135e-07
383 2.12398688859139e-07
384 2.11577514619421e-07
385 2.10781976761609e-07
386 2.10112432341703e-07
387 2.09151721719536e-07
388 2.08305536375519e-07
389 2.07481519964858e-07
390 2.06796769930406e-07
391 2.05874755465629e-07
392 2.05027774313749e-07
393 2.0420415580702e-07
394 2.03525544861805e-07
395 2.02589575337697e-07
396 2.01758268758567e-07
397 2.00938771399706e-07
398 2.00128070559913e-07
399 1.99314754922852e-07
400 1.98494277015016e-07
401 1.97668626356062e-07
402 1.96844439415145e-07
403 1.96025055743121e-07
404 1.95204847841524e-07
405 1.94384597307362e-07
406 1.93563877814995e-07
407 1.92743272009466e-07
408 1.9192448519334e-07
409 1.91106991564993e-07
410 1.90289995316562e-07
411 1.89468735811715e-07
412 1.88647561571997e-07
413 1.87830750064677e-07
414 1.87012602737013e-07
415 1.8619466857217e-07
416 1.85372172722964e-07
417 1.84551637971708e-07
418 1.83726896807457e-07
419 1.82905552037482e-07
420 1.82084107791525e-07
421 1.81255387587953e-07
422 1.80432053298318e-07
423 1.79609898509625e-07
424 1.78784148374689e-07
425 1.77962476755056e-07
426 1.77137053469778e-07
427 1.76313761812708e-07
428 1.75491379650339e-07
429 1.7467101542934e-07
430 1.73858211383049e-07
431 1.73042110418464e-07
432 1.72235161244316e-07
433 1.71431821627266e-07
434 1.70630713114406e-07
435 1.69838415331469e-07
436 1.69054018783754e-07
437 1.68278276646561e-07
438 1.67483449331485e-07
439 1.66552013070032e-07
440 1.65494554948964e-07
441 1.64455514095607e-07
442 1.63564166655306e-07
443 1.62704409945036e-07
444 1.61850820745713e-07
445 1.60999377385451e-07
446 1.60152026751348e-07
447 1.59308385150325e-07
448 1.58458803412032e-07
449 1.57624398866574e-07
450 1.56785915805813e-07
451 1.55951568103774e-07
452 1.55124297407383e-07
453 1.54298248844498e-07
454 1.53477841990934e-07
455 1.52666160602166e-07
456 1.51857179275794e-07
457 1.51052162777887e-07
458 1.50259424458454e-07
459 1.49468831978083e-07
460 1.4869058873046e-07
461 1.47924666293875e-07
462 1.47168861985847e-07
463 1.46420788382784e-07
464 1.4567849859759e-07
465 1.44937303048209e-07
466 1.44209082009183e-07
467 1.43482310477339e-07
468 1.4276797344337e-07
469 1.4205866705197e-07
470 1.41355883442884e-07
471 1.40663914294237e-07
472 1.3997942005517e-07
473 1.39302613888503e-07
474 1.38635982693813e-07
475 1.37971355229638e-07
476 1.3731811066009e-07
477 1.36669768835418e-07
478 1.36029825625883e-07
479 1.35394188305327e-07
480 1.34766693804522e-07
481 1.34142666752268e-07
482 1.33525389856004e-07
483 1.3291410994043e-07
484 1.32306965383577e-07
485 1.31705064632115e-07
486 1.31106190792707e-07
487 1.305150050257e-07
488 1.29922412384076e-07
489 1.29335830933996e-07
490 1.28751551642381e-07
491 1.28175159375132e-07
492 1.27597786558908e-07
493 1.27023724871833e-07
494 1.26452306403735e-07
495 1.25881129520167e-07
496 1.25316461208058e-07
497 1.24754976127406e-07
498 1.24193235251369e-07
499 1.23628254300456e-07
500 1.23072666724511e-07
501 1.22508538424881e-07
502 1.2194627174722e-07
503 1.21384744034003e-07
504 1.20820189408732e-07
505 1.20255904789701e-07
506 1.19690653832549e-07
507 1.19124095476764e-07
508 1.185555689176e-07
509 1.17983795178134e-07
510 1.17412994882216e-07
511 1.16846443631857e-07
512 1.16271792194311e-07
513 1.15699933189717e-07
514 1.15129637379141e-07
515 1.14560364750105e-07
516 1.13985947791662e-07
517 1.13419893921218e-07
518 1.12860938372705e-07
519 1.12311930422493e-07
520 1.11767960220277e-07
521 1.11233532607002e-07
522 1.10705329348093e-07
523 1.10183364654404e-07
524 1.0966670771495e-07
525 1.09150889215925e-07
526 1.08640186624598e-07
527 1.08135303378276e-07
528 1.07633518098282e-07
529 1.07139342730989e-07
530 1.06650595910196e-07
531 1.06167455271589e-07
532 1.05682211426483e-07
533 1.05191681143424e-07
534 1.04705257797377e-07
535 1.04221520302872e-07
536 1.03747304081026e-07
537 1.03268895657038e-07
538 1.02801934076524e-07
539 1.02337857299517e-07
540 1.01872601021569e-07
541 1.0141420148102e-07
542 1.00962346039069e-07
543 1.00515251233446e-07
544 1.00065811636796e-07
545 9.96239890582729e-08
546 9.91848381204363e-08
547 9.8751215205084e-08
548 9.83170878043893e-08
549 9.78892984448976e-08
550 9.74627312189114e-08
551 9.703953907092e-08
552 9.66167661431427e-08
553 9.6196785648317e-08
554 9.57850900817903e-08
555 9.53740837417172e-08
556 9.49600007515983e-08
557 9.45527958151615e-08
558 9.41469551207774e-08
559 9.37413915380603e-08
560 9.33385067014569e-08
561 9.29376753333599e-08
562 9.25406311580446e-08
563 9.2147438124357e-08
564 9.17535132316516e-08
565 9.1365961907286e-08
566 9.09836117557461e-08
567 9.05993360333923e-08
568 9.02173056260835e-08
569 8.98418619499353e-08
570 8.94615368451923e-08
571 8.90900579975096e-08
572 8.87171012209365e-08
573 8.83438886489785e-08
574 8.79709816103968e-08
575 8.76042989261805e-08
576 8.72315979449922e-08
577 8.68653984298362e-08
578 8.64945306489062e-08
579 8.61304272348207e-08
580 8.57619468774828e-08
581 8.53928838751017e-08
582 8.50252632744741e-08
583 8.46537346887999e-08
584 8.4279491829875e-08
585 8.39101588212543e-08
586 8.35366051887831e-08
587 8.31600814876765e-08
588 8.27845241246905e-08
589 8.24062027504624e-08
590 8.20290466663209e-08
591 8.1653809047566e-08
592 8.12779745729131e-08
593 8.09049680583485e-08
594 8.05316204832707e-08
595 8.01592392463135e-08
596 7.97897072857268e-08
597 7.94215182509106e-08
598 7.90539829154113e-08
599 7.86852467626886e-08
600 7.83188696118486e-08
601 7.79513413817767e-08
602 7.75876358716232e-08
603 7.72190134057382e-08
604 7.68546826179772e-08
605 7.64939116493224e-08
606 7.61271579108325e-08
607 7.57649800675608e-08
608 7.54042517314701e-08
609 7.50413775563175e-08
610 7.46820987274077e-08
611 7.43270547332031e-08
612 7.39668593041642e-08
613 7.36116874122672e-08
614 7.32508738110482e-08
615 7.28979898667603e-08
616 7.2548793639271e-08
617 7.21993558272516e-08
618 7.18480634986918e-08
619 7.15031163167623e-08
620 7.1155966452352e-08
621 7.08142167127335e-08
622 7.04717351140971e-08
623 7.01302411698634e-08
624 6.9793777868199e-08
625 6.94524686650766e-08
626 6.91169859123875e-08
627 6.87850558733771e-08
628 6.84540140127865e-08
629 6.81270790892086e-08
630 6.77973588381064e-08
631 6.74733868777366e-08
632 6.71490667514263e-08
633 6.68269493075968e-08
634 6.6505720042187e-08
635 6.61888321928927e-08
636 6.58717667079145e-08
637 6.55585736808462e-08
638 6.52470220074974e-08
639 6.49402522867604e-08
640 6.46282032334966e-08
641 6.43253414978062e-08
642 6.40211936797641e-08
643 6.37220907151459e-08
644 6.34259009757443e-08
645 6.31363903380588e-08
646 6.28559746473911e-08
647 6.25899261308405e-08
648 6.23197209392856e-08
649 6.20474551737971e-08
650 6.17820390402812e-08
651 6.15072650589354e-08
652 6.12307644587418e-08
653 6.0967813908519e-08
654 6.07064620794517e-08
655 6.04492313982519e-08
656 6.01893930252118e-08
657 5.99313878524299e-08
658 5.96780083128579e-08
659 5.94304410128643e-08
660 5.91823088313959e-08
661 5.89313309262707e-08
662 5.86909827404725e-08
663 5.8450162043755e-08
664 5.82087089640027e-08
665 5.7972524558636e-08
666 5.77355088182685e-08
667 5.75031684491023e-08
668 5.72716665203643e-08
669 5.70414684375464e-08
670 5.68158462499468e-08
671 5.65898261584152e-08
672 5.63681723519949e-08
673 5.61471331650409e-08
674 5.59263853006087e-08
675 5.57085257923973e-08
676 5.54919630246786e-08
677 5.52768675277093e-08
678 5.5065267901e-08
679 5.48541478906373e-08
680 5.46422533886926e-08
681 5.44347074082907e-08
682 5.42303943973366e-08
683 5.40255413739033e-08
684 5.38274029793229e-08
685 5.36256514749311e-08
686 5.34303516985801e-08
687 5.32375423745179e-08
688 5.30460333436622e-08
689 5.2859849830611e-08
690 5.26798622502156e-08
691 5.25008516660819e-08
692 5.23309005018291e-08
693 5.21683212184598e-08
694 5.20123961678109e-08
695 5.18558955775461e-08
696 5.17003719835429e-08
697 5.15504652298659e-08
698 5.14031199827514e-08
699 5.12555438092477e-08
700 5.11156201810081e-08
701 5.09727584585562e-08
702 5.08385014086343e-08
703 5.07030790686258e-08
704 5.05718311671899e-08
705 5.04436563630861e-08
706 5.03218586800358e-08
707 5.02001178404043e-08
708 5.0082501701354e-08
709 4.99706764856001e-08
710 4.98575225549303e-08
711 4.9753410280573e-08
712 4.96507261971146e-08
713 4.95505787512229e-08
714 4.94523710869998e-08
715 4.93567569037623e-08
716 4.92638463356343e-08
717 4.91745382191766e-08
718 4.90876459480205e-08
719 4.9006388280759e-08
720 4.89295466366002e-08
721 4.88536677778484e-08
722 4.87827733763879e-08
723 4.87203379861967e-08
724 4.86621267725695e-08
725 4.86086833006993e-08
726 4.85605369249242e-08
727 4.8520465867341e-08
728 4.84741917716747e-08
729 4.84329483185775e-08
730 4.83909765591761e-08
731 4.83458784117374e-08
732 4.83023647745995e-08
733 4.82619313402211e-08
734 4.82159911996405e-08
735 4.81713655631211e-08
736 4.81344528679983e-08
737 4.80857522688893e-08
738 4.80378119505076e-08
739 4.80057522622701e-08
740 4.79864148417164e-08
741 4.79392276986346e-08
742 4.79002011388729e-08
743 4.78548010107716e-08
744 4.78169361883829e-08
745 4.77746802118872e-08
746 4.77388972797144e-08
747 4.7706770089917e-08
748 4.76698502893669e-08
749 4.7635815292324e-08
750 4.76039225816294e-08
751 4.75709533986901e-08
752 4.75368509000873e-08
753 4.75035015767844e-08
754 4.74742662959216e-08
755 4.74431018915311e-08
756 4.7411635506478e-08
757 4.73807482137545e-08
758 4.73526604594099e-08
759 4.73225121311316e-08
760 4.72914258864421e-08
761 4.72644430260516e-08
762 4.72351366909152e-08
763 4.72034180631908e-08
764 4.71757850561971e-08
765 4.71381760291933e-08
766 4.71121559542098e-08
767 4.70833718679842e-08
768 4.70486298809192e-08
769 4.70205243630062e-08
770 4.69871714869896e-08
771 4.69640006883765e-08
772 4.69266190350481e-08
773 4.68967691347189e-08
774 4.68619667515213e-08
775 4.68331151637358e-08
776 4.68056633451397e-08
777 4.67756571254085e-08
778 4.67418708183232e-08
779 4.67165186535112e-08
780 4.66748311112042e-08
781 4.66496885564993e-08
782 4.6618456650549e-08
783 4.65869227639359e-08
784 4.65557512541181e-08
785 4.65235139301967e-08
786 4.64976892544655e-08
787 4.64605278693853e-08
788 4.64319711568351e-08
789 4.63986502552416e-08
790 4.6361183336785e-08
791 4.63344953516298e-08
792 4.62947440382777e-08
793 4.62776164056322e-08
794 4.62252032207289e-08
795 4.62065763429109e-08
796 4.61599469758767e-08
797 4.61377567262389e-08
798 4.60907898514051e-08
799 4.60651286005032e-08
800 4.60218672060364e-08
801 4.59880098446774e-08
802 4.59482052406202e-08
803 4.59111468842366e-08
804 4.58718858453722e-08
805 4.58299425076802e-08
806 4.57907844975125e-08
807 4.57561668554263e-08
808 4.57126532182883e-08
809 4.56692497152744e-08
810 4.56271891380311e-08
811 4.55782398489646e-08
812 4.55389006503992e-08
813 4.54970816576861e-08
814 4.54457449450274e-08
815 4.54013857620339e-08
816 4.53505855091407e-08
817 4.53072139805499e-08
818 4.52561828012676e-08
819 4.5211592691885e-08
820 4.51520385524873e-08
821 4.51052493133375e-08
822 4.50513120142659e-08
823 4.50027037857126e-08
824 4.49481944997387e-08
825 4.48916566142543e-08
826 4.48376340500545e-08
827 4.47799521907655e-08
828 4.4724963288445e-08
829 4.46685284316573e-08
830 4.46081998006775e-08
831 4.45479031441209e-08
832 4.44916281594487e-08
833 4.44319461223586e-08
834 4.43680114869949e-08
835 4.43087309065504e-08
836 4.42442562587075e-08
837 4.41829648423209e-08
838 4.41175735943489e-08
839 4.40546834568067e-08
840 4.39865210921653e-08
841 4.39219220993436e-08
842 4.3853408016048e-08
843 4.37869260849766e-08
844 4.37196270297591e-08
845 4.36544596027488e-08
846 4.35816041033377e-08
847 4.35119567043785e-08
848 4.34405293958662e-08
849 4.33751630168899e-08
850 4.32947118156335e-08
851 4.32273274952877e-08
852 4.3155694129382e-08
853 4.30812043816786e-08
854 4.30109352578256e-08
855 4.29324948925114e-08
856 4.28670361429795e-08
857 4.27848085848836e-08
858 4.27155733007112e-08
859 4.26292423583163e-08
860 4.25733972519993e-08
861 4.24781845254074e-08
862 4.24245918395627e-08
863 4.23263202264934e-08
864 4.22748023254371e-08
865 4.21756638502302e-08
866 4.21280255125112e-08
867 4.2026467639289e-08
868 4.1977976650287e-08
869 4.18794989798243e-08
870 4.18305177163347e-08
871 4.17260963558874e-08
872 4.16821954729585e-08
873 4.15775858186862e-08
874 4.15309564516519e-08
875 4.14295691086863e-08
876 4.13810887778254e-08
877 4.1283360729949e-08
878 4.12324325793634e-08
879 4.11319014403944e-08
880 4.10845366616286e-08
881 4.09842080273393e-08
882 4.0934061473763e-08
883 4.08362268444762e-08
884 4.0782971666431e-08
885 4.06913081008042e-08
886 4.06365465721592e-08
887 4.05437745598647e-08
888 4.04861388858535e-08
889 4.03979107943542e-08
890 4.03395681303209e-08
891 4.02497697393756e-08
892 4.01915087877569e-08
893 4.01036217567707e-08
894 4.00412432099984e-08
895 3.9961101094832e-08
896 3.98971629067546e-08
897 3.98129920142765e-08
898 3.97453661094005e-08
899 3.9667337858873e-08
900 3.96011827774601e-08
901 3.95212822468238e-08
902 3.9452903166648e-08
903 3.9374349114496e-08
904 3.93089081285325e-08
905 3.92335302024094e-08
906 3.91588095283169e-08
907 3.90905086078419e-08
908 3.90145586948165e-08
909 3.89430852010264e-08
910 3.88722298794164e-08
911 3.87979568472474e-08
912 3.87281637870274e-08
913 3.8655741718685e-08
914 3.8586346562397e-08
915 3.8516265732369e-08
916 3.8443463523663e-08
917 3.83751803667565e-08
918 3.83036287132654e-08
919 3.82323825931508e-08
920 3.81639075897056e-08
921 3.80931020060871e-08
922 3.80234901342646e-08
923 3.79511071457728e-08
924 3.7885392600856e-08
925 3.78163882430727e-08
926 3.77457070044329e-08
927 3.76747806285493e-08
928 3.76101922938687e-08
929 3.75422182230523e-08
930 3.7469529701184e-08
931 3.74010831194482e-08
932 3.73371271678025e-08
933 3.72687267713445e-08
934 3.71996122794371e-08
935 3.71335282522978e-08
936 3.70681405570394e-08
937 3.70014205941516e-08
938 3.69334394179077e-08
939 3.68658241711728e-08
940 3.67993457928151e-08
941 3.67373864662568e-08
942 3.66697712195219e-08
943 3.66043870769772e-08
944 3.65378447497733e-08
945 3.6471107023317e-08
946 3.64082275439159e-08
947 3.63441365891504e-08
948 3.62822376587246e-08
949 3.62165444300899e-08
950 3.61545495763949e-08
951 3.60883518624178e-08
952 3.60269538646207e-08
953 3.59588092635477e-08
954 3.59001255390012e-08
955 3.58405003453299e-08
956 3.57795251204607e-08
957 3.57108049797716e-08
958 3.56518370381309e-08
959 3.55882576741351e-08
960 3.55271900787102e-08
961 3.54675933067483e-08
962 3.54070515129479e-08
963 3.53451454770948e-08
964 3.52852609353249e-08
965 3.52234366118864e-08
966 3.51668241194147e-08
967 3.51081901328598e-08
968 3.50473072785462e-08
969 3.49892914641714e-08
970 3.49289841494738e-08
971 3.48758000257021e-08
972 3.48152724427564e-08
973 3.47571926795354e-08
974 3.469851961313e-08
975 3.46419142260856e-08
976 3.45862076756021e-08
977 3.45287851644116e-08
978 3.44717641098669e-08
979 3.44185231426763e-08
980 3.43613457687297e-08
981 3.43087087628646e-08
982 3.42532509023386e-08
983 3.41987735907878e-08
984 3.41442252249635e-08
985 3.40927215347619e-08
986 3.4037970664258e-08
987 3.39839765217675e-08
988 3.3930039222696e-08
989 3.38771037888819e-08
990 3.3826189849151e-08
991 3.37730270416614e-08
992 3.37235626091115e-08
993 3.36720695770509e-08
994 3.3619961925524e-08
995 3.3569623525409e-08
996 3.3521697417882e-08
997 3.3468694482508e-08
998 3.34164624860023e-08
999 3.33695098220232e-08
1000 3.33214167369533e-08
1001 3.32700054173074e-08
1002 3.32228538013624e-08
1003 3.31723839508413e-08
1004 3.31282592469506e-08
1005 3.30800133951925e-08
1006 3.30290497174701e-08
1007 3.29824381140043e-08
1008 3.29348850414135e-08
1009 3.28893747791881e-08
1010 3.28451257303186e-08
1011 3.27978391112538e-08
1012 3.27509397379799e-08
1013 3.27058415905412e-08
1014 3.26581925946812e-08
1015 3.26149418583555e-08
1016 3.25691367208947e-08
1017 3.25293427749784e-08
1018 3.24767661652459e-08
1019 3.24371427495862e-08
1020 3.23905204879793e-08
1021 3.23472804097946e-08
1022 3.23045235006703e-08
1023 3.22590452128679e-08
1024 3.2220718537701e-08
1025 3.21739683784017e-08
1026 3.21319291174404e-08
1027 3.20872679537842e-08
1028 3.20486037708179e-08
1029 3.20084936333842e-08
1030 3.19632107448342e-08
1031 3.19224611189384e-08
1032 3.18802122478701e-08
1033 3.18412602950957e-08
1034 3.18015302980257e-08
1035 3.17606811961468e-08
1036 3.17194377430496e-08
1037 3.16808232980748e-08
1038 3.16406172373718e-08
1039 3.15980770437818e-08
1040 3.15609902656888e-08
1041 3.15228589897742e-08
1042 3.14844434967654e-08
1043 3.14467172302102e-08
1044 3.14057828632031e-08
1045 3.13705257326546e-08
1046 3.13333821111428e-08
1047 3.12942347591161e-08
1048 3.12548280589908e-08
1049 3.12166470450848e-08
1050 3.11813579401132e-08
1051 3.11463281832403e-08
1052 3.11075893932866e-08
1053 3.10732595210084e-08
1054 3.10378034384939e-08
1055 3.10012779891622e-08
1056 3.09670369347259e-08
1057 3.09298684442183e-08
1058 3.08973184814931e-08
1059 3.08605834220543e-08
1060 3.08256069558865e-08
1061 3.07918170960875e-08
1062 3.07581586866945e-08
1063 3.072170784435e-08
1064 3.06905754143827e-08
1065 3.06561034335573e-08
1066 3.06237311065161e-08
1067 3.0588392263553e-08
1068 3.05554266333274e-08
1069 3.05239993281248e-08
1070 3.04940606099535e-08
1071 3.04571621256855e-08
1072 3.04263103600988e-08
1073 3.03929681422233e-08
1074 3.03585245831073e-08
1075 3.03303728799165e-08
1076 3.02986826739016e-08
1077 3.02662748197235e-08
1078 3.0236350312407e-08
1079 3.02011748942732e-08
1080 3.01789846446354e-08
1081 3.01439477823351e-08
1082 3.01155616000415e-08
1083 3.00827451837904e-08
1084 3.00544833464755e-08
1085 3.00252658291811e-08
1086 2.99950890791933e-08
1087 2.99652214152957e-08
1088 2.99353892785348e-08
1089 2.9908331811157e-08
1090 2.98774587292883e-08
1091 2.98497795370167e-08
1092 2.981999003282e-08
1093 2.97894224843276e-08
1094 2.97640188051673e-08
1095 2.97345064126375e-08
1096 2.97053208697662e-08
1097 2.96762170393094e-08
1098 2.96518045672656e-08
1099 2.96212583350552e-08
1100 2.95952506945696e-08
1101 2.9568921533496e-08
1102 2.95402369232534e-08
1103 2.95099855662784e-08
1104 2.94852515736466e-08
1105 2.94565314362671e-08
1106 2.94311561788163e-08
1107 2.94035160663952e-08
1108 2.93786825977804e-08
1109 2.93502218084996e-08
1110 2.9325363470889e-08
1111 2.92965776083065e-08
1112 2.92744211094487e-08
1113 2.92468786966538e-08
1114 2.92178867766779e-08
1115 2.9192788630894e-08
1116 2.91699677745783e-08
1117 2.91407822317069e-08
1118 2.91169577337769e-08
1119 2.90908666045198e-08
1120 2.90633295207954e-08
1121 2.90418853410301e-08
1122 2.90153661097747e-08
1123 2.8988761613391e-08
1124 2.89641786110906e-08
1125 2.89377783957434e-08
1126 2.8913946792386e-08
1127 2.88894650424254e-08
1128 2.88646511137358e-08
1129 2.88377073331958e-08
1130 2.88129431424977e-08
1131 2.8789358452741e-08
1132 2.87647754504405e-08
1133 2.8740370083824e-08
1134 2.87161991963103e-08
1135 2.8691902187461e-08
1136 2.86664452175955e-08
1137 2.86407626504115e-08
1138 2.86180643627176e-08
1139 2.8597884949022e-08
1140 2.85719785608762e-08
1141 2.8546232044846e-08
1142 2.85251626763738e-08
1143 2.84984071896588e-08
1144 2.84777392778324e-08
1145 2.84535950356712e-08
1146 2.84286603147166e-08
1147 2.84056564936463e-08
1148 2.8383693617684e-08
1149 2.83592260785781e-08
1150 2.83367356246345e-08
1151 2.83106764698005e-08
1152 2.82893370950887e-08
1153 2.82653083161222e-08
1154 2.82421943609279e-08
1155 2.82174053012341e-08
1156 2.81937158064238e-08
1157 2.8168971155651e-08
1158 2.81455427852961e-08
1159 2.81248730971129e-08
1160 2.81048837535991e-08
1161 2.80792757934023e-08
1162 2.80564655952276e-08
1163 2.80333800617427e-08
1164 2.80137850694473e-08
1165 2.79923551005368e-08
1166 2.79649707835006e-08
1167 2.79455587559596e-08
1168 2.79211889164799e-08
1169 2.78999916503153e-08
1170 2.78748242266147e-08
1171 2.78520069230126e-08
1172 2.78344085558047e-08
1173 2.7806100533212e-08
1174 2.77871130549556e-08
1175 2.77663190217936e-08
1176 2.77451501773385e-08
1177 2.7719583073349e-08
1178 2.76975065105489e-08
1179 2.76789133835109e-08
1180 2.76557443612546e-08
1181 2.76332130511037e-08
1182 2.76099214602255e-08
1183 2.75895448709207e-08
1184 2.75698699425675e-08
1185 2.75466440768923e-08
1186 2.75256510917643e-08
1187 2.7503300970011e-08
1188 2.74803024780113e-08
1189 2.74607376837821e-08
1190 2.74399418742632e-08
1191 2.7418021630865e-08
1192 2.73935416572613e-08
1193 2.73780340620533e-08
1194 2.73553339980026e-08
1195 2.73293707664379e-08
1196 2.73121738558757e-08
1197 2.72893156960663e-08
1198 2.72689639757573e-08
1199 2.72453490879343e-08
1200 2.7226095156152e-08
1201 2.72040256987793e-08
1202 2.71819988739708e-08
1203 2.71646580785045e-08
1204 2.71435371956841e-08
1205 2.71214570801703e-08
1206 2.71024749309845e-08
1207 2.70805458058021e-08
1208 2.70591282713895e-08
1209 2.70403717195222e-08
1210 2.70197606511147e-08
1211 2.69978315259323e-08
1212 2.69772488792341e-08
1213 2.69557762777595e-08
1214 2.69358082505278e-08
1215 2.69149662557311e-08
1216 2.68948614490228e-08
1217 2.68752060605948e-08
1218 2.68558721927548e-08
1219 2.68356128430014e-08
1220 2.68147779536321e-08
1221 2.67918878193996e-08
1222 2.6773179229167e-08
1223 2.67530104736124e-08
1224 2.6735218483509e-08
1225 2.67142397092357e-08
1226 2.66958508632342e-08
1227 2.6676254094582e-08
1228 2.66523212388847e-08
1229 2.66356536826606e-08
1230 2.66135966597858e-08
1231 2.6593099278216e-08
1232 2.65782293951133e-08
1233 2.6555104781778e-08
1234 2.65371689067706e-08
1235 2.65169752822203e-08
1236 2.65000608123955e-08
1237 2.64775081859625e-08
1238 2.64582595832508e-08
1239 2.64357176149588e-08
1240 2.64211941214398e-08
1241 2.64020343365701e-08
1242 2.63808885847538e-08
1243 2.63617607743072e-08
1244 2.63439829950585e-08
1245 2.63230539587767e-08
1246 2.63059565241974e-08
1247 2.62839847664509e-08
1248 2.62647361637391e-08
1249 2.62443489162933e-08
1250 2.622614658776e-08
1251 2.62086192748257e-08
1252 2.61877080021122e-08
1253 2.61709107718389e-08
1254 2.61527492995128e-08
1255 2.61305324045225e-08
1256 2.6108725847962e-08
1257 2.60959041042952e-08
1258 2.60756802816786e-08
1259 2.60564441134647e-08
1260 2.60374921623452e-08
1261 2.60181316491526e-08
1262 2.60010182273618e-08
1263 2.59806203217749e-08
1264 2.59628247789578e-08
1265 2.59428460935851e-08
1266 2.59275925174052e-08
1267 2.59040149330758e-08
1268 2.58858943169571e-08
1269 2.58664698549183e-08
1270 2.5846818019204e-08
1271 2.58296459776375e-08
1272 2.58107508699368e-08
1273 2.57916692447679e-08
1274 2.57756695987155e-08
1275 2.57547103643674e-08
1276 2.57363623745732e-08
1277 2.5717472595943e-08
1278 2.57021834926263e-08
1279 2.56801016007557e-08
1280 2.56594425707135e-08
1281 2.56441108348326e-08
1282 2.56245353824625e-08
1283 2.56052530289708e-08
1284 2.55899408330151e-08
1285 2.55711967156458e-08
1286 2.55523744385755e-08
1287 2.55348382438569e-08
1288 2.55123602244112e-08
1289 2.54940317745422e-08
1290 2.54804728427871e-08
1291 2.54587249060023e-08
1292 2.54409595612515e-08
1293 2.54242937813842e-08
1294 2.54055834147948e-08
1295 2.53840060082666e-08
1296 2.53699603547375e-08
1297 2.53506939884574e-08
1298 2.5331949871088e-08
1299 2.5315735285858e-08
1300 2.52978242798463e-08
1301 2.52816914070308e-08
1302 2.5260851188591e-08
1303 2.52457166283193e-08
1304 2.52253649080103e-08
1305 2.5210104226403e-08
1306 2.51902267933701e-08
1307 2.51729961320279e-08
1308 2.5156857930142e-08
1309 2.51395562145262e-08
1310 2.51205509727015e-08
1311 2.51032101772353e-08
1312 2.50838247950469e-08
1313 2.50665372902859e-08
1314 2.50490064246378e-08
1315 2.50338683116524e-08
1316 2.50178455729611e-08
1317 2.4996763769991e-08
1318 2.49796698881255e-08
1319 2.49639402483126e-08
1320 2.494598483338e-08
1321 2.49270204477625e-08
1322 2.49106477667738e-08
1323 2.48952858328266e-08
1324 2.48780214207045e-08
1325 2.4858294978003e-08
1326 2.48425173765554e-08
1327 2.48230449528819e-08
1328 2.48064448982177e-08
1329 2.479008287537e-08
1330 2.47745610693073e-08
1331 2.47554119425786e-08
1332 2.47400500086314e-08
1333 2.47228033600777e-08
1334 2.47027713839998e-08
1335 2.46887381649685e-08
1336 2.46704541240206e-08
1337 2.4651590990743e-08
1338 2.46391511637967e-08
1339 2.46195241970781e-08
1340 2.4604361215097e-08
1341 2.45860469760828e-08
1342 2.45721132330345e-08
1343 2.45532643106117e-08
1344 2.45349021099628e-08
1345 2.45175293400735e-08
1346 2.45032527601552e-08
1347 2.44844731156491e-08
1348 2.44685232075881e-08
1349 2.44499620549732e-08
1350 2.44348115074899e-08
1351 2.44195526022395e-08
1352 2.43988580450605e-08
1353 2.43837394720003e-08
1354 2.43675692956913e-08
1355 2.43492159768266e-08
1356 2.43350140038956e-08
1357 2.43179609782374e-08
1358 2.43008440037329e-08
1359 2.42856348364739e-08
1360 2.42676634343297e-08
1361 2.42501361213954e-08
1362 2.42344047052256e-08
1363 2.42175168807535e-08
1364 2.41999487116118e-08
1365 2.41823716606859e-08
1366 2.41674005252435e-08
1367 2.41486901586541e-08
1368 2.41317081872694e-08
1369 2.41153337299238e-08
1370 2.40995738920446e-08
1371 2.40811335316948e-08
1372 2.40666064854622e-08
1373 2.40466917489357e-08
1374 2.40314967925315e-08
1375 2.40157636000049e-08
1376 2.39969768500714e-08
1377 2.39809256896706e-08
1378 2.39664181833632e-08
1379 2.39488606723626e-08
1380 2.39298305615421e-08
1381 2.3910619262324e-08
1382 2.38959572129716e-08
1383 2.38789681361595e-08
1384 2.38613075964622e-08
1385 2.38461765889042e-08
1386 2.38287984899443e-08
1387 2.38128077256761e-08
1388 2.37950636972073e-08
1389 2.37778330358651e-08
1390 2.37636061939384e-08
1391 2.37444321982139e-08
1392 2.37329231822514e-08
1393 2.37116033474649e-08
1394 2.36930635111321e-08
1395 2.3676163252162e-08
1396 2.36583499457765e-08
1397 2.36465247382966e-08
1398 2.36266153308407e-08
1399 2.36109194418077e-08
1400 2.35934347614375e-08
1401 2.35789219260596e-08
1402 2.35643149437692e-08
1403 2.35457040531628e-08
1404 2.35284627336796e-08
1405 2.35091732747605e-08
1406 2.349775307664e-08
1407 2.34788561925825e-08
1408 2.34622241634952e-08
1409 2.34444570423875e-08
1410 2.34305126411982e-08
1411 2.34115375974397e-08
1412 2.33949251082777e-08
1413 2.33790142800672e-08
1414 2.33645760516765e-08
1415 2.33476935562749e-08
1416 2.33335306631943e-08
1417 2.3312463071079e-08
1418 2.33009807004692e-08
1419 2.32834480584643e-08
1420 2.32653949439054e-08
1421 2.32508732267434e-08
1422 2.32350227946654e-08
1423 2.32184191872875e-08
1424 2.3203790888715e-08
1425 2.31853647392199e-08
1426 2.31703509712133e-08
1427 2.31536709804914e-08
1428 2.31375540948875e-08
1429 2.31237393677475e-08
1430 2.31054873012226e-08
1431 2.3093880585634e-08
1432 2.30747225771211e-08
1433 2.30598296013795e-08
1434 2.30460948102973e-08
1435 2.30284680213799e-08
1436 2.30118555322179e-08
1437 2.29973498022673e-08
1438 2.29837322507365e-08
1439 2.29656897943187e-08
1440 2.2950267464239e-08
1441 2.29352394853777e-08
1442 2.29202701262921e-08
1443 2.29031336118624e-08
1444 2.28880967512168e-08
1445 2.28757031095483e-08
1446 2.28567316185035e-08
1447 2.28438974403389e-08
1448 2.28273862035167e-08
1449 2.28144205749459e-08
1450 2.2797120635687e-08
1451 2.27827552379267e-08
1452 2.27685355014273e-08
1453 2.27514220796365e-08
1454 2.27375434036503e-08
1455 2.27229506322146e-08
1456 2.27069332225938e-08
1457 2.26909744327486e-08
1458 2.26777245870835e-08
1459 2.26625935795255e-08
1460 2.2647700603784e-08
1461 2.26301697381359e-08
1462 2.26178240581021e-08
1463 2.26035865580343e-08
1464 2.25895266936504e-08
1465 2.25753016280805e-08
1466 2.2560662671367e-08
1467 2.25470593306909e-08
1468 2.25302656531312e-08
1469 2.25173515389088e-08
1470 2.25020393429531e-08
1471 2.2488579887181e-08
1472 2.24756071531829e-08
1473 2.24596625741924e-08
1474 2.24442082696896e-08
1475 2.24334222309608e-08
1476 2.24182805652617e-08
1477 2.24054890196612e-08
1478 2.23919851549681e-08
1479 2.23773124474747e-08
1480 2.23630660656227e-08
1481 2.23495604245727e-08
1482 2.23399130305779e-08
1483 2.23232134999307e-08
1484 2.23095231177695e-08
1485 2.22981704212089e-08
1486 2.22846221475947e-08
1487 2.22708642638736e-08
1488 2.22577263286894e-08
1489 2.22461693510922e-08
1490 2.22304130659268e-08
1491 2.22209273204044e-08
1492 2.22060467791607e-08
1493 2.21959997048771e-08
1494 2.21834781655161e-08
1495 2.21686331514093e-08
1496 2.2157420787039e-08
1497 2.21464979688335e-08
1498 2.21323634974624e-08
1499 2.21191349680794e-08
1500 2.21083062967864e-08
1501 2.20956692942309e-08
1502 2.20859774913151e-08
1503 2.20724061250621e-08
1504 2.2059930770979e-08
1505 2.20501448211508e-08
1506 2.20351825674925e-08
1507 2.20235492065513e-08
1508 2.20103384407366e-08
1509 2.19995861527877e-08
1510 2.19877698270921e-08
1511 2.19750138086283e-08
1512 2.19647660060218e-08
1513 2.19543796475818e-08
1514 2.19426006253798e-08
1515 2.19303792903247e-08
1516 2.19186890859646e-08
1517 2.1905389502308e-08
1518 2.18984776978459e-08
1519 2.1886011225547e-08
1520 2.18739089064002e-08
1521 2.18640536786552e-08
1522 2.18533120488473e-08
1523 2.18433342524804e-08
1524 2.18323901179929e-08
1525 2.18230482573745e-08
1526 2.1812766703988e-08
1527 2.18018190167868e-08
1528 2.17917097700138e-08
1529 2.17822879733376e-08
1530 2.17733546747922e-08
1531 2.17611706432308e-08
1532 2.17545839120703e-08
1533 2.1741236366779e-08
1534 2.17369144905888e-08
1535 2.17204778607538e-08
1536 2.1723119303374e-08
1537 2.16947668718603e-08
1538 2.1703010943952e-08
1539 2.16738893499269e-08
1540 2.16838671462938e-08
1541 2.16628475158132e-08
1542 2.16731592672659e-08
1543 2.16362483485e-08
1544 2.16528235341684e-08
1545 2.16293400967515e-08
1546 2.1642929226573e-08
1547 2.160166090448e-08
1548 2.16183106971357e-08
1549 2.15965449967825e-08
1550 2.16160707111612e-08
1551 2.15696740468729e-08
1552 2.15921325263935e-08
1553 2.15643538581389e-08
1554 2.15864837116442e-08
1555 2.15416289250925e-08
1556 2.15667679270837e-08
1557 2.1537054806231e-08
1558 2.15620179488951e-08
1559 2.15226183541972e-08
1560 2.15553566107474e-08
1561 2.15142179627037e-08
1562 2.15516404722393e-08
1563 2.15141344739322e-08
1564 2.15521520630091e-08
1565 2.15203961317911e-08
1566 2.15482423016056e-08
1567 2.15155733229722e-08
1568 2.15535838066216e-08
1569 2.15233164624351e-08
1570 2.15563105143701e-08
1571 2.15111413126579e-08
1572 2.15710738160624e-08
1573 2.15256505953221e-08
1574 2.15129745129161e-08
1575 2.15571454020846e-08
1576 2.14591366898276e-08
1577 2.15642721457243e-08
1578 2.14679634069626e-08
1579 2.15356124044774e-08
1580 2.14542854592992e-08
1581 2.15381330548325e-08
1582 2.14709743318053e-08
1583 2.145614530491e-08
1584 2.15165290029518e-08
1585 2.14443307555712e-08
1586 2.14667092990339e-08
1587 2.14627995376304e-08
1588 2.14429469735933e-08
1589 2.14663504749524e-08
1590 2.13941966364928e-08
1591 2.1464629185175e-08
1592 2.14108801799284e-08
1593 2.13934665538318e-08
1594 2.14302868783989e-08
1595 2.14085478233983e-08
1596 2.13735535936621e-08
1597 2.14166320233744e-08
1598 2.13388329228792e-08
1599 2.14389785924141e-08
1600 2.13612860733292e-08
1601 2.14627107197884e-08
1602 2.13245634483883e-08
1603 2.14149160626675e-08
1604 2.13466080367652e-08
1605 2.14190940539538e-08
1606 2.12390656173511e-08
1607 2.13872066723297e-08
1608 2.12478088457146e-08
1609 2.13812896276977e-08
1610 2.12174455782588e-08
1611 2.12847659497584e-08
1612 2.13263486870119e-08
1613 2.12525375076211e-08
1614 2.12964685886163e-08
1615 2.11721538079246e-08
1616 2.13465973786242e-08
1617 2.11117487936008e-08
1618 2.12096349372359e-08
1619 2.1128448324248e-08
1620 2.12868851434678e-08
1621 2.10731716521195e-08
1622 2.13376090130168e-08
1623 2.09431973985374e-08
1624 2.13146904570749e-08
1625 2.09942943030228e-08
1626 2.11602042554659e-08
1627 2.13329531817408e-08
1628 2.07952162156744e-08
1629 2.12980513225602e-08
1630 2.08867216855424e-08
1631 2.11162376473339e-08
1632 2.13772182178218e-08
1633 2.07222665693507e-08
1634 2.13502957535638e-08
1635 2.08533315060322e-08
1636 2.10048156645826e-08
1637 2.13014175187709e-08
1638 2.07169605914714e-08
1639 2.13378434921196e-08
1640 2.07840180621588e-08
1641 2.09876080958793e-08
1642 2.10911057507701e-08
1643 2.09448351995434e-08
1644 2.10402202327487e-08
1645 2.10456736482456e-08
1646 2.07866275303559e-08
1647 2.13672066706749e-08
1648 2.07215844483244e-08
1649 2.09459614097796e-08
1650 2.12961985823767e-08
1651 2.06681978198731e-08
1652 2.12185309322876e-08
1653 2.07013091113595e-08
1654 2.09684962726442e-08
1655 2.08563442072318e-08
1656 2.10607389306006e-08
1657 2.08942889656782e-08
1658 2.09332871037304e-08
1659 2.11498498714491e-08
1660 2.06664267921042e-08
1661 2.11294697294306e-08
1662 2.06572412508876e-08
1663 2.09341486367975e-08
1664 2.07681463137988e-08
1665 2.1331834076932e-08
1666 2.06037871208764e-08
1667 2.11567652286249e-08
1668 2.06110666312043e-08
1669 2.07692725240349e-08
1670 2.1288066420766e-08
1671 2.05705532607681e-08
1672 2.10846824444388e-08
1673 2.05956318666267e-08
1674 2.1125870830474e-08
1675 2.05999448610328e-08
1676 2.07023287401853e-08
1677 2.12586659387171e-08
1678 2.05582502132984e-08
1679 2.07387973460982e-08
1680 2.07620303172007e-08
1681 2.1192523291802e-08
1682 2.05450820800479e-08
1683 2.10912602938151e-08
1684 2.0551318868911e-08
1685 2.07335659752061e-08
1686 2.08009094393447e-08
1687 2.05692014532133e-08
1688 2.08728963002613e-08
1689 2.05516048623622e-08
1690 2.1017863005568e-08
1691 2.0518413634818e-08
1692 2.04255510283247e-08
1693 2.06330508234487e-08
1694 2.05445367384982e-08
1695 2.08063433149164e-08
1696 2.05624512972236e-08
1697 2.10852473259138e-08
1698 2.04295691474954e-08
1699 2.03192218606318e-08
1700 2.05646006889992e-08
1701 2.03534789022797e-08
1702 2.10586161841775e-08
1703 2.06290113879959e-08
1704 2.02385130876337e-08
1705 2.0682135115635e-08
1706 2.05439807388075e-08
1707 2.09631245695618e-08
1708 2.02959196116126e-08
1709 2.0324728566834e-08
1710 2.05133652286804e-08
1711 2.03602272819126e-08
1712 2.05748200698963e-08
1713 2.04442383022752e-08
1714 2.07472563573674e-08
1715 2.02967953555344e-08
1716 2.06087111820352e-08
1717 2.04481320764671e-08
1718 2.09143919960297e-08
1719 2.02691214923334e-08
1720 2.03963228528892e-08
1721 2.04866577036e-08
1722 2.02916048408497e-08
1723 2.07222363712845e-08
1724 2.03324610481559e-08
1725 2.06017372050837e-08
1726 2.02580352492987e-08
1727 2.09671231488073e-08
1728 2.02271337457205e-08
1729 2.07564330167997e-08
1730 2.02991703446287e-08
1731 2.1013924822455e-08
1732 2.01903880281407e-08
1733 2.10495016972345e-08
1734 2.0214020679532e-08
1735 2.1014709972178e-08
1736 2.01935321797464e-08
1737 2.09966959374697e-08
1738 2.01813517008986e-08
1739 2.09758166391794e-08
1740 2.01681160660883e-08
1741 2.09547632579188e-08
1742 2.01559107182447e-08
1743 2.092803796927e-08
1744 2.01384864340071e-08
1745 2.09195150091546e-08
1746 2.01218224304966e-08
1747 2.08841797189052e-08
1748 2.01133669719411e-08
1749 2.08418331482108e-08
1750 2.00979641817867e-08
1751 2.08244106403299e-08
1752 2.00862437793603e-08
1753 2.07910311189607e-08
1754 2.00738305977666e-08
1755 2.07792574258292e-08
1756 2.00566052654949e-08
1757 2.0736598216331e-08
1758 2.0042822512778e-08
1759 2.07079757785777e-08
1760 2.00319991705555e-08
1761 2.06718357986801e-08
1762 2.0019506052904e-08
1763 2.06390762258479e-08
1764 2.00043466236366e-08
1765 2.06051673501406e-08
1766 1.99932514988177e-08
1767 2.05833003974476e-08
1768 1.99740703976659e-08
1769 2.05346690762553e-08
1770 1.99620782126431e-08
1771 2.05108001694043e-08
1772 1.99460110650307e-08
1773 2.04756407384821e-08
1774 1.99354470709068e-08
1775 2.04493098010516e-08
1776 1.99206215967251e-08
1777 2.04078745014158e-08
1778 1.99026466418672e-08
1779 2.03665013742693e-08
1780 1.98865102163381e-08
1781 2.03420800204412e-08
1782 1.9874013545973e-08
1783 2.0275630063793e-08
1784 1.98538643303436e-08
1785 2.02761611944879e-08
1786 1.98414262797542e-08
1787 2.0218212881673e-08
1788 1.98216749680569e-08
1789 2.01954097889256e-08
1790 1.98052756417155e-08
1791 2.01624725804095e-08
1792 1.97887377595407e-08
1793 2.01167207336539e-08
1794 1.9771169590399e-08
1795 2.00675174255593e-08
1796 1.97507503685301e-08
1797 2.00590903887132e-08
1798 1.9732368627956e-08
1799 1.99674730083643e-08
1800 1.97081693187329e-08
1801 1.99941982970131e-08
1802 1.96945997288367e-08
1803 1.98977261334221e-08
1804 1.96683096476136e-08
1805 1.99104732701016e-08
1806 1.96547400577174e-08
1807 1.98627674308227e-08
1808 1.96314449141255e-08
1809 1.97604173024502e-08
1810 1.96008862474173e-08
1811 1.97233447352119e-08
1812 1.95777456468704e-08
1813 1.9676091866927e-08
1814 1.95575520223201e-08
1815 1.96366745086607e-08
1816 1.95263982760707e-08
1817 1.95500415856031e-08
1818 1.9496530612173e-08
1819 1.94088247695845e-08
1820 1.94597973290911e-08
1821 1.92289881795205e-08
1822 1.94747951098861e-08
1823 1.92601330439857e-08
1824 1.97049399019988e-08
1825 1.92436537815865e-08
1826 1.93593532316072e-08
1827 1.95171629968627e-08
1828 1.96310256939114e-08
1829 1.94329405900362e-08
1830 1.98266452144935e-08
1831 1.94338660719495e-08
1832 1.9780516780088e-08
1833 1.94702387545931e-08
1834 1.92204083759862e-08
1835 1.9281786833858e-08
1836 1.93496116906999e-08
1837 1.92577900293145e-08
1838 1.93145304194786e-08
1839 1.97302245652509e-08
1840 1.91920186409789e-08
1841 1.89387172611077e-08
1842 1.94019733612549e-08
1843 1.89863449406857e-08
1844 1.93839042594846e-08
1845 1.89774240766383e-08
1846 1.93664337899691e-08
1847 1.89659274951737e-08
1848 1.93555127481204e-08
1849 1.89483220225384e-08
1850 1.93423268513016e-08
1851 1.89281283979881e-08
1852 1.93299598549856e-08
1853 1.89161326602516e-08
1854 1.93077251964269e-08
1855 1.8897324594036e-08
1856 1.92938038878765e-08
1857 1.88757098840142e-08
1858 1.92796445475096e-08
1859 1.886116685057e-08
1860 1.92648421659669e-08
1861 1.88371931386655e-08
1862 1.92433020629323e-08
1863 1.88183353344584e-08
1864 1.92302600510175e-08
1865 1.88030622183533e-08
1866 1.92100468865419e-08
1867 1.87819111374665e-08
1868 1.91931555093561e-08
1869 1.87643056648312e-08
1870 1.91779356839561e-08
1871 1.87442008581229e-08
1872 1.91616571498798e-08
1873 1.87296027576167e-08
1874 1.91430835627671e-08
1875 1.87081603542083e-08
1876 1.91298923368777e-08
1877 1.86945285918227e-08
1878 1.91135978155899e-08
1879 1.86715727323872e-08
1880 1.90912849973301e-08
1881 1.86577722161019e-08
1882 1.90776567876583e-08
1883 1.86391950762754e-08
1884 1.90658706600288e-08
1885 1.86209696551032e-08
1886 1.90479436668056e-08
1887 1.86109758715247e-08
1888 1.90334450422824e-08
1889 1.85879756031682e-08
1890 1.90158129242946e-08
1891 1.85811224184818e-08
1892 1.90045650327875e-08
1893 1.85587847312263e-08
1894 1.89861140142966e-08
1895 1.85476096703496e-08
1896 1.89697519914489e-08
1897 1.85292936549786e-08
1898 1.89541715656105e-08
1899 1.85092154936228e-08
1900 1.89367259650908e-08
1901 1.84951876036621e-08
1902 1.89187137067393e-08
1903 1.8482557706534e-08
1904 1.88967774761295e-08
1905 1.84549726611749e-08
1906 1.88706170689557e-08
1907 1.84280963821948e-08
1908 1.8845792482125e-08
1909 1.84011188508748e-08
1910 1.88021953562156e-08
1911 1.83691852839729e-08
1912 1.8767570608702e-08
1913 1.83395201247549e-08
1914 1.86909616672892e-08
1915 1.83063129099992e-08
1916 1.86396000856348e-08
1917 1.82545054627781e-08
1918 1.85779285288845e-08
1919 1.82593637987338e-08
1920 1.8466032258857e-08
1921 1.81082082661987e-08
1922 1.8572901439029e-08
1923 1.82265793569059e-08
1924 1.84204029807233e-08
1925 1.827845252933e-08
1926 1.80262720306246e-08
1927 1.84315638307453e-08
1928 1.88462205841233e-08
1929 1.88028685954578e-08
1930 1.87818809394003e-08
1931 1.88122708522087e-08
1932 1.88361948261218e-08
1933 1.88326918504345e-08
1934 1.88128410627542e-08
1935 1.87979800614357e-08
1936 1.8780037081001e-08
1937 1.87668689477505e-08
1938 1.87548945262961e-08
1939 1.87473574442265e-08
1940 1.87323418998631e-08
1941 1.87201809609405e-08
1942 1.87115176686348e-08
1943 1.87010549268507e-08
1944 1.86908923893725e-08
1945 1.86862898488016e-08
1946 1.86746511587899e-08
1947 1.86672881596905e-08
1948 1.86590796147357e-08
1949 1.86491373455056e-08
1950 1.86415824998676e-08
1951 1.86352000497436e-08
1952 1.86274657920649e-08
1953 1.86212840702638e-08
1954 1.86135089563777e-08
1955 1.86089366138731e-08
1956 1.86010264968672e-08
1957 1.85938144880993e-08
1958 1.85853945566805e-08
1959 1.85795911988862e-08
1960 1.85738020519466e-08
1961 1.8565634363199e-08
1962 1.85607191838244e-08
1963 1.85523756357497e-08
1964 1.85506277006198e-08
1965 1.85404509522868e-08
1966 1.85343917991077e-08
1967 1.85253465900814e-08
1968 1.85203674618606e-08
1969 1.85135480279541e-08
1970 1.85085902160154e-08
1971 1.85018933507308e-08
1972 1.8494580089623e-08
1973 1.84915069922909e-08
1974 1.84839130668024e-08
1975 1.84743562670064e-08
1976 1.84698496497049e-08
1977 1.84640676081926e-08
1978 1.84567756633669e-08
1979 1.84519990398258e-08
1980 1.84465296371172e-08
1981 1.84381772072584e-08
1982 1.84326118812805e-08
1983 1.84232469280232e-08
1984 1.84178503559451e-08
1985 1.84135888758874e-08
1986 1.84041457629291e-08
1987 1.83963972943957e-08
1988 1.83951787136039e-08
1989 1.83854513835513e-08
1990 1.83828028355038e-08
1991 1.83744575110723e-08
1992 1.83654282892576e-08
1993 1.83571113865355e-08
1994 1.83549175858388e-08
1995 1.83494659466987e-08
1996 1.83407458109741e-08
1997 1.83329689207312e-08
1998 1.83256148034161e-08
1999 1.83196284808673e-08
2000 1.83128427977408e-08
2001 1.83031154676883e-08
2002 1.83015842480927e-08
2003 1.82901338519059e-08
2004 1.82867285758448e-08
2005 1.82768751244566e-08
2006 1.82707378115765e-08
2007 1.82655526259623e-08
2008 1.82561841199913e-08
2009 1.82504962253915e-08
2010 1.82487447375479e-08
2011 1.82354984445965e-08
2012 1.82330808229381e-08
2013 1.82211756794004e-08
2014 1.82140098559103e-08
2015 1.8210075225511e-08
2016 1.82009358695723e-08
2017 1.81945054578136e-08
2018 1.81895387640907e-08
2019 1.81813355482063e-08
2020 1.81762693785004e-08
2021 1.81624297823646e-08
2022 1.81583725833434e-08
2023 1.8157813030939e-08
2024 1.81448029934472e-08
2025 1.81364310236631e-08
2026 1.81318746683701e-08
2027 1.81220460859777e-08
2028 1.81222095108069e-08
2029 1.81049948366763e-08
2030 1.81082953076839e-08
2031 1.8092720210916e-08
2032 1.80952000050638e-08
2033 1.80763350954294e-08
2034 1.80788077841498e-08
2035 1.80697607987668e-08
2036 1.80589321274738e-08
2037 1.80575394637117e-08
2038 1.80440835606532e-08
2039 1.80458190612853e-08
2040 1.80312760278412e-08
2041 1.80257515580706e-08
2042 1.80238206581862e-08
2043 1.80148020945126e-08
2044 1.80046004771839e-08
2045 1.80019696927047e-08
2046 1.79940187194916e-08
2047 1.79852932546964e-08
2048 1.79862311711076e-08
2049 1.79701036273627e-08
2050 1.79677019929159e-08
2051 1.79574239922431e-08
2052 1.79590671223195e-08
2053 1.79422006141294e-08
2054 1.79440622360971e-08
2055 1.79277197531746e-08
2056 1.79315051695994e-08
2057 1.79100005937016e-08
2058 1.79258705657048e-08
2059 1.78865509070647e-08
2060 1.79223587082333e-08
2061 1.78655117366588e-08
2062 1.79227832575179e-08
2063 1.78373724679659e-08
2064 1.79367045660683e-08
2065 1.7801545126872e-08
2066 1.79689081392098e-08
2067 1.7757423975695e-08
2068 1.80016090922663e-08
2069 1.7727305845483e-08
2070 1.80113470804599e-08
2071 1.77159620307066e-08
2072 1.79929244836785e-08
2073 1.77087944308596e-08
2074 1.79833392621731e-08
2075 1.76927912320934e-08
2076 1.79732548843958e-08
2077 1.76795520445694e-08
2078 1.79687305035259e-08
2079 1.76631669290828e-08
2080 1.79561681079576e-08
2081 1.76469612256369e-08
2082 1.79443109260546e-08
2083 1.76387295880431e-08
2084 1.79316383963624e-08
2085 1.76270944507451e-08
2086 1.79148678114416e-08
2087 1.76197829659941e-08
2088 1.79045844816983e-08
2089 1.76069026025516e-08
2090 1.788999881569e-08
2091 1.75929333323666e-08
2092 1.7873572843996e-08
2093 1.7582431510732e-08
2094 1.78587651333828e-08
2095 1.75726224682649e-08
2096 1.78391985627968e-08
2097 1.75628454002208e-08
2098 1.78270251893764e-08
2099 1.75537167024231e-08
2100 1.78128125583044e-08
2101 1.75419465620053e-08
2102 1.7795418472133e-08
2103 1.75313559225287e-08
2104 1.77780705712394e-08
2105 1.75221224196775e-08
2106 1.7761019321938e-08
2107 1.75091035004016e-08
2108 1.77443251203613e-08
2109 1.74986389822607e-08
2110 1.77294730008271e-08
2111 1.74876237934996e-08
2112 1.77159567016361e-08
2113 1.74780119266416e-08
2114 1.7692940446068e-08
2115 1.74694747556714e-08
2116 1.76786301153697e-08
2117 1.74603105307369e-08
2118 1.7658939199805e-08
2119 1.74525798257719e-08
2120 1.76378591731918e-08
2121 1.74391772134186e-08
2122 1.76218364345004e-08
2123 1.74318142143193e-08
2124 1.76102137317002e-08
2125 1.74203496072778e-08
2126 1.75892882481321e-08
2127 1.74104446415413e-08
2128 1.75724377271536e-08
2129 1.74021508314581e-08
2130 1.75525745049754e-08
2131 1.73941323566851e-08
2132 1.75353616072016e-08
2133 1.73821153026665e-08
2134 1.75195271623352e-08
2135 1.73741891984491e-08
2136 1.74949157383253e-08
2137 1.73680216875027e-08
2138 1.74789942519737e-08
2139 1.73582996865207e-08
2140 1.74625611748525e-08
2141 1.73455241281317e-08
2142 1.74454353185638e-08
2143 1.73446021989321e-08
2144 1.74322369872471e-08
2145 1.7333929847041e-08
2146 1.74139724862243e-08
2147 1.73250853663376e-08
2148 1.73942940051575e-08
2149 1.7312869360353e-08
2150 1.73762391142418e-08
2151 1.73090022315137e-08
2152 1.7357500325943e-08
2153 1.73002252523702e-08
2154 1.73439875794656e-08
2155 1.72932317354935e-08
2156 1.73291585525703e-08
2157 1.72814917931419e-08
2158 1.73156191607404e-08
2159 1.72765197703484e-08
2160 1.72981433621544e-08
2161 1.72632983463927e-08
2162 1.72862080205505e-08
2163 1.72533809461584e-08
2164 1.72690963751165e-08
2165 1.72506364748415e-08
2166 1.72473413329044e-08
2167 1.72450249635858e-08
2168 1.72378680218799e-08
2169 1.72366227957355e-08
2170 1.72252612173907e-08
2171 1.72267569098494e-08
2172 1.72171041867841e-08
2173 1.72170135925853e-08
2174 1.72098744144478e-08
2175 1.72058829406296e-08
2176 1.7190371792708e-08
2177 1.71973244533774e-08
2178 1.7168405364032e-08
2179 1.71998895126535e-08
2180 1.71542282600967e-08
2181 1.71943472793146e-08
2182 1.71584559893745e-08
2183 1.71880571997463e-08
2184 1.71771006307608e-08
2185 1.71631224787916e-08
2186 1.71641989510363e-08
2187 1.71400582615888e-08
2188 1.71157914508058e-08
2189 1.71453677921818e-08
2190 1.70764646867383e-08
2191 1.71856111563784e-08
2192 1.70937166643625e-08
2193 1.71522707148597e-08
2194 1.71312013463876e-08
2195 1.71246004043724e-08
2196 1.71762994938263e-08
2197 1.71381877578369e-08
2198 1.7094729187761e-08
2199 1.71118266223402e-08
2200 1.70279008671059e-08
2201 1.70981255820379e-08
2202 1.70153402478945e-08
2203 1.7135610264063e-08
2204 1.7063349844193e-08
2205 1.70461422754897e-08
2206 1.70959388867686e-08
2207 1.70606764271497e-08
2208 1.7090183490609e-08
2209 1.71186442798898e-08
2210 1.70135887600509e-08
2211 1.7065694635221e-08
2212 1.69804064142909e-08
2213 1.70719633985073e-08
2214 1.6954711412609e-08
2215 1.70459433235237e-08
2216 1.70027512069737e-08
2217 1.69797367277624e-08
2218 1.70840905866498e-08
2219 1.69637726088467e-08
2220 1.70113434450059e-08
2221 1.7007039332384e-08
2222 1.69300946595285e-08
2223 1.70655685138854e-08
2224 1.69357168289253e-08
2225 1.69833285212917e-08
2226 1.69835523422535e-08
2227 1.69090164092722e-08
2228 1.70491176731957e-08
2229 1.69173350883511e-08
2230 1.69543934447347e-08
2231 1.69567009322691e-08
2232 1.68959761737142e-08
2233 1.70041030145285e-08
2234 1.69345977241164e-08
2235 1.69060356824957e-08
2236 1.6981223538437e-08
2237 1.68313363246853e-08
2238 1.70155391998605e-08
2239 1.68629359365013e-08
2240 1.69687375262129e-08
2241 1.68623284224623e-08
2242 1.69532938798511e-08
2243 1.68492721996927e-08
2244 1.69384009041096e-08
2245 1.68696239200017e-08
2246 1.69161964436171e-08
2247 1.68950418100167e-08
2248 1.68629146202193e-08
2249 1.69025256013811e-08
2250 1.68108389431154e-08
2251 1.69033071983904e-08
2252 1.68342086936946e-08
2253 1.6886122722326e-08
2254 1.68757381402429e-08
2255 1.68187810345444e-08
2256 1.68956155732758e-08
2257 1.67371521087034e-08
2258 1.69114215964328e-08
2259 1.67426126296277e-08
2260 1.69040497155493e-08
2261 1.67049378774209e-08
2262 1.6955949533326e-08
2263 1.66913665111679e-08
2264 1.70865295245903e-08
2265 1.66618807639907e-08
2266 1.71389107350706e-08
2267 1.67193405786747e-08
2268 1.71160863260411e-08
2269 1.66486664454624e-08
2270 1.70147664846354e-08
2271 1.67587597132979e-08
2272 1.71020495542962e-08
2273 1.66712545990322e-08
2274 1.69367506686058e-08
2275 1.67692970620692e-08
2276 1.70750382721963e-08
2277 1.6659399193486e-08
2278 1.69108194114642e-08
2279 1.6761578791602e-08
2280 1.70602234561557e-08
2281 1.66415556890342e-08
2282 1.69029661378772e-08
2283 1.67629128355884e-08
2284 1.70442859825926e-08
2285 1.66355516029171e-08
2286 1.6885039144654e-08
2287 1.67495013414509e-08
2288 1.70293450452164e-08
2289 1.66123523825945e-08
2290 1.68803211408886e-08
2291 1.6749876152744e-08
2292 1.70089506923432e-08
2293 1.66180580407627e-08
2294 1.68566174352236e-08
2295 1.67267941719729e-08
2296 1.70017919742804e-08
2297 1.65725531076077e-08
2298 1.68505494002602e-08
2299 1.67200102652032e-08
2300 1.69742051525645e-08
2301 1.65996940637569e-08
2302 1.68237441755537e-08
2303 1.66960365532987e-08
2304 1.69684000184134e-08
2305 1.65314890665513e-08
2306 1.68141873757577e-08
2307 1.66832165859887e-08
2308 1.69433072017e-08
2309 1.65651581340853e-08
2310 1.67960383379295e-08
2311 1.66859841499445e-08
2312 1.69228169255575e-08
2313 1.65029199195033e-08
2314 1.68021472291002e-08
2315 1.66752425201366e-08
2316 1.69023923746181e-08
2317 1.65550027020345e-08
2318 1.67797793437785e-08
2319 1.66515068400486e-08
2320 1.689524786741e-08
2321 1.64667621760373e-08
2322 1.6758979981546e-08
2323 1.66368003817752e-08
2324 1.68754592522191e-08
2325 1.65196230028641e-08
2326 1.67536455819572e-08
2327 1.66481015639874e-08
2328 1.6852299111747e-08
2329 1.6442962547103e-08
2330 1.67609286449988e-08
2331 1.66354485742204e-08
2332 1.68316898196963e-08
2333 1.64885136655357e-08
2334 1.6747630837699e-08
2335 1.66324589656597e-08
2336 1.6817049086626e-08
2337 1.64198947771865e-08
2338 1.674630389914e-08
2339 1.66107270160865e-08
2340 1.68012039836185e-08
2341 1.64412270464709e-08
2342 1.67309934795412e-08
2343 1.66106577381697e-08
2344 1.67862008737529e-08
2345 1.64226339194329e-08
2346 1.67195555178523e-08
2347 1.6590490758972e-08
2348 1.67701745823479e-08
2349 1.63973297162556e-08
2350 1.67006035667328e-08
2351 1.65802944707139e-08
2352 1.67585785249003e-08
2353 1.64020139692411e-08
2354 1.66891833686122e-08
2355 1.65683946562467e-08
2356 1.67457550048766e-08
2357 1.63619926496494e-08
2358 1.6666799496079e-08
2359 1.65432272325461e-08
2360 1.67348908064469e-08
2361 1.63540399000794e-08
2362 1.66446803007148e-08
2363 1.6532034408101e-08
2364 1.67202678369449e-08
2365 1.63395199592742e-08
2366 1.66310734073249e-08
2367 1.65096736282067e-08
2368 1.67129599049076e-08
2369 1.63176316902991e-08
2370 1.6604360553174e-08
2371 1.64836002625179e-08
2372 1.66989373440174e-08
2373 1.62968536443486e-08
2374 1.65761022685729e-08
2375 1.64639182287374e-08
2376 1.66897073938799e-08
2377 1.62835593897626e-08
2378 1.65558731168858e-08
2379 1.64322653262161e-08
2380 1.66767240017407e-08
2381 1.62664690606107e-08
2382 1.65255773509898e-08
2383 1.64123949986106e-08
2384 1.6667296875994e-08
2385 1.62511657464393e-08
2386 1.65018754216817e-08
2387 1.63728532953655e-08
2388 1.66566955783765e-08
2389 1.62340718645737e-08
2390 1.64710769468002e-08
2391 1.63517430706861e-08
2392 1.66482685415303e-08
2393 1.62217226318262e-08
2394 1.64569016192218e-08
2395 1.63259894492285e-08
2396 1.66399836132314e-08
2397 1.62048490182087e-08
2398 1.64285705039902e-08
2399 1.63005644537861e-08
2400 1.66273323998212e-08
2401 1.6193354213101e-08
2402 1.64054814177916e-08
2403 1.6273920877552e-08
2404 1.66162212877907e-08
2405 1.61746029903043e-08
2406 1.63830033983459e-08
2407 1.62300963779671e-08
2408 1.66137024137925e-08
2409 1.61586513058865e-08
2410 1.63709632516884e-08
2411 1.62082329779878e-08
2412 1.66058917727696e-08
2413 1.61413282739886e-08
2414 1.6342928788049e-08
2415 1.61733755277282e-08
2416 1.65970437393526e-08
2417 1.61270499177135e-08
2418 1.63351057125283e-08
2419 1.6151156856381e-08
2420 1.65882276803586e-08
2421 1.61103468343526e-08
2422 1.63331037583703e-08
2423 1.61142406085446e-08
2424 1.65824332043485e-08
2425 1.609273425629e-08
2426 1.63471618463973e-08
2427 1.60902935419927e-08
2428 1.65717803923826e-08
2429 1.60721445041645e-08
2430 1.63834243949168e-08
2431 1.60677817717669e-08
2432 1.65609765900854e-08
2433 1.60459823206338e-08
2434 1.64490092657843e-08
2435 1.60266004911591e-08
2436 1.65474745017491e-08
2437 1.60052273656675e-08
2438 1.65186779810256e-08
2439 1.58225290647351e-08
2440 1.65291833553738e-08
2441 1.59355071360778e-08
2442 1.64988485096274e-08
2443 1.58804098759902e-08
2444 1.64956297510344e-08
2445 1.59005768551879e-08
2446 1.64885758380251e-08
2447 1.58778092895773e-08
2448 1.64837885563429e-08
2449 1.58781077175263e-08
2450 1.64764113463889e-08
2451 1.586251485719e-08
2452 1.64673306102259e-08
2453 1.58557877938392e-08
2454 1.64596674068207e-08
2455 1.58418842488572e-08
2456 1.64513114242482e-08
2457 1.58304906960893e-08
2458 1.64439413197215e-08
2459 1.58190278654047e-08
2460 1.64345745901073e-08
2461 1.58143826922696e-08
2462 1.64282312198338e-08
2463 1.5802433139811e-08
2464 1.64247389022876e-08
2465 1.57968074177006e-08
2466 1.64150275594466e-08
2467 1.57852060311825e-08
2468 1.64028381988146e-08
2469 1.57750523754885e-08
2470 1.63858562274299e-08
2471 1.57667585654053e-08
2472 1.63583262491329e-08
2473 1.59420050493964e-08
2474 1.63393742980134e-08
2475 1.57364379305136e-08
2476 1.62531854641657e-08
2477 1.57890660545945e-08
2478 1.62354805155474e-08
2479 1.58640354186446e-08
2480 1.59435433744193e-08
2481 1.55356225661762e-08
2482 1.62502598044512e-08
2483 1.58399746652549e-08
2484 1.60269966187343e-08
2485 1.55762034381723e-08
2486 1.61977578017058e-08
2487 1.585456210762e-08
2488 1.59113504594188e-08
2489 1.5546232745578e-08
2490 1.62299720329884e-08
2491 1.58012394280149e-08
2492 1.5979825462864e-08
2493 1.55433301785024e-08
2494 1.61737681025897e-08
2495 1.58238169234437e-08
2496 1.5877724024449e-08
2497 1.5511158579784e-08
2498 1.61979336610329e-08
2499 1.57738977435429e-08
2500 1.59462203441763e-08
2501 1.55101833598792e-08
2502 1.61566777734379e-08
2503 1.57925654775681e-08
2504 1.58875668176961e-08
2505 1.54809658425847e-08
2506 1.61616782179408e-08
2507 1.577647346096e-08
2508 1.59065454141682e-08
2509 1.54839074895108e-08
2510 1.61468278747634e-08
2511 1.57687285451402e-08
2512 1.59392161691585e-08
2513 1.55318922168135e-08
2514 1.61316950908486e-08
2515 1.57758908159167e-08
2516 1.59452131498483e-08
2517 1.55799106948962e-08
2518 1.61246056507025e-08
2519 1.57721995464044e-08
2520 1.59390882714661e-08
2521 1.55981219052137e-08
2522 1.61137911902642e-08
2523 1.57706061543195e-08
2524 1.59560187285024e-08
2525 1.56763491077072e-08
2526 1.61100572881878e-08
2527 1.5760413418775e-08
2528 1.59415787237549e-08
2529 1.56820707530869e-08
2530 1.6101925126577e-08
2531 1.57493644792339e-08
2532 1.59511941433266e-08
2533 1.5704102906966e-08
2534 1.60805644355833e-08
2535 1.57296700109555e-08
2536 1.59414739187014e-08
2537 1.57058561711665e-08
2538 1.60655702075019e-08
2539 1.57136383904799e-08
2540 1.59546544864497e-08
2541 1.57089345975692e-08
2542 1.60449946662311e-08
2543 1.56978874343849e-08
2544 1.59585908932058e-08
2545 1.56940291873298e-08
2546 1.60205146926273e-08
2547 1.56750097346503e-08
2548 1.59572746127878e-08
2549 1.56741872814337e-08
2550 1.59913984276727e-08
2551 1.56571324794186e-08
2552 1.59612376648965e-08
2553 1.56466715139914e-08
2554 1.59754698358938e-08
2555 1.56345052459983e-08
2556 1.59516311271091e-08
2557 1.56293893383008e-08
2558 1.59608219973961e-08
2559 1.56124553285508e-08
2560 1.59516790887437e-08
2561 1.56085704361431e-08
2562 1.59485171735696e-08
2563 1.55873145502028e-08
2564 1.5939177089308e-08
2565 1.55776280763575e-08
2566 1.59258206622326e-08
2567 1.55740043084052e-08
2568 1.59332795846012e-08
2569 1.55526791445482e-08
2570 1.59296398294373e-08
2571 1.55548658398175e-08
2572 1.59209303518537e-08
2573 1.55388910627607e-08
2574 1.59095598917247e-08
2575 1.55330770468254e-08
2576 1.59046642522753e-08
2577 1.55234491927558e-08
2578 1.59033213265047e-08
2579 1.55070214447051e-08
2580 1.58879114309229e-08
2581 1.55002872759269e-08
2582 1.58870445687853e-08
2583 1.54994364010008e-08
2584 1.58784061454753e-08
2585 1.54772923366409e-08
2586 1.58661386251424e-08
2587 1.54820778419662e-08
2588 1.58662700755485e-08
2589 1.54693733378508e-08
2590 1.58565267582844e-08
2591 1.54586849987481e-08
2592 1.58508388636847e-08
2593 1.54475845448587e-08
2594 1.58373012482116e-08
2595 1.54412465036557e-08
2596 1.58316186826823e-08
2597 1.5425460020424e-08
2598 1.58190704979688e-08
2599 1.5416622645148e-08
2600 1.58048081289053e-08
2601 1.5398725849991e-08
2602 1.57923700783158e-08
2603 1.53914889722273e-08
2604 1.57804880274171e-08
2605 1.53680197456652e-08
2606 1.57605857253884e-08
2607 1.53529668978081e-08
2608 1.57351482954482e-08
2609 1.53372621269909e-08
2610 1.57090731534026e-08
2611 1.53081085585427e-08
2612 1.5688495835775e-08
2613 1.52949137799396e-08
2614 1.56560950870244e-08
2615 1.52618895299383e-08
2616 1.56237138781989e-08
2617 1.52886929782881e-08
2618 1.55920290012546e-08
2619 1.52822696719568e-08
2620 1.55538888435558e-08
2621 1.53655967949362e-08
2622 1.55250106104177e-08
2623 1.54812660468906e-08
2624 1.55274531010718e-08
2625 1.55748232089081e-08
2626 1.55505102128473e-08
2627 1.56238080251114e-08
2628 1.55660053735573e-08
2629 1.56962105535285e-08
2630 1.55853552286089e-08
2631 1.57300554803896e-08
2632 1.55879948948723e-08
2633 1.57686130819457e-08
2634 1.55742725382879e-08
2635 1.57766297803619e-08
2636 1.55573776083884e-08
2637 1.57744874940136e-08
2638 1.55299648696428e-08
2639 1.57526081068227e-08
2640 1.55057353623533e-08
2641 1.57201149875164e-08
2642 1.54984913791623e-08
2643 1.56948569696169e-08
2644 1.5489545646119e-08
2645 1.56562638409241e-08
2646 1.54877177749313e-08
2647 1.56176014343146e-08
2648 1.54697623599986e-08
2649 1.5521013807529e-08
2650 1.54169459420928e-08
2651 1.53691672721834e-08
2652 1.54168926513876e-08
2653 1.53834260885333e-08
2654 1.53657886414749e-08
2655 1.53021364468486e-08
2656 1.53879149422664e-08
2657 1.53551074077996e-08
2658 1.53478190156875e-08
2659 1.52935353270323e-08
2660 1.5366019567864e-08
2661 1.53301478178491e-08
2662 1.53421861881498e-08
2663 1.52960435428895e-08
2664 1.53404773328703e-08
2665 1.53011452397323e-08
2666 1.53224384291661e-08
2667 1.52792729579687e-08
2668 1.53165853333803e-08
2669 1.52717607448949e-08
2670 1.52941392883577e-08
2671 1.52500625461016e-08
2672 1.52822874355252e-08
2673 1.52335477565657e-08
2674 1.5276393483532e-08
2675 1.52265151598385e-08
2676 1.52596850711006e-08
2677 1.52126897745575e-08
2678 1.52279984177994e-08
2679 1.51547840943067e-08
2680 1.52185428703433e-08
2681 1.51634100831188e-08
2682 1.51869929965187e-08
2683 1.51313788165908e-08
2684 1.51574521822795e-08
2685 1.50950825172913e-08
2686 1.51491601485532e-08
2687 1.50824490674495e-08
2688 1.51016710248086e-08
2689 1.50326506798137e-08
2690 1.5085131366277e-08
2691 1.50130663456594e-08
2692 1.50463570491866e-08
2693 1.49723575759708e-08
2694 1.50090979644801e-08
2695 1.49497569879031e-08
2696 1.4972499684518e-08
2697 1.49222803003113e-08
2698 1.49290588780104e-08
2699 1.48953924750117e-08
2700 1.48300740576701e-08
2701 1.48320840054339e-08
2702 1.47637173597559e-08
2703 1.48239527320015e-08
2704 1.46066243544851e-08
2705 1.48437253599809e-08
2706 1.45082710290012e-08
2707 1.4852114205155e-08
2708 1.51155443717244e-08
2709 1.51129118108884e-08
2710 1.45685969954457e-08
2711 1.47476804102098e-08
2712 1.50922581099167e-08
2713 1.50610439675347e-08
2714 1.46288376967618e-08
2715 1.47519667592633e-08
2716 1.509314984105e-08
2717 1.50432288847924e-08
2718 1.45584087007933e-08
2719 1.46914835852385e-08
2720 1.50692081035686e-08
2721 1.50136578724869e-08
2722 1.45661438466504e-08
2723 1.46715626314631e-08
2724 1.50601078274804e-08
2725 1.49927288362051e-08
2726 1.4484702326456e-08
2727 1.45850016508575e-08
2728 1.49974361818295e-08
2729 1.49133096982723e-08
2730 1.43868366109245e-08
2731 1.44956313619105e-08
2732 1.49134891103131e-08
2733 1.48451126946725e-08
2734 1.44486413944378e-08
2735 1.45229543946357e-08
2736 1.49543577521172e-08
2737 1.48308281211484e-08
2738 1.4300828077296e-08
2739 1.43981715439168e-08
2740 1.47688918872291e-08
2741 1.48700642910171e-08
2742 1.50420103040005e-08
2743 1.49519490122429e-08
2744 1.43211149605804e-08
2745 1.42490019783281e-08
2746 1.44388412337548e-08
2747 1.44107135113813e-08
2748 1.41520590801747e-08
2749 1.41383011964535e-08
2750 1.43678562380956e-08
2751 1.42290401683454e-08
2752 1.41938709674605e-08
2753 1.42099203515045e-08
2754 1.39932625486949e-08
2755 1.41282843202362e-08
2756 1.43794069984438e-08
2757 1.47213849999162e-08
2758 1.4956604843519e-08
2759 1.47741214817643e-08
2760 1.43396787777306e-08
2761 1.43895686477435e-08
2762 1.48270631328273e-08
2763 1.46031569059346e-08
2764 1.4089866162692e-08
2765 1.43445406664e-08
2766 1.48447680814456e-08
2767 1.46616008223077e-08
2768 1.40425342465278e-08
2769 1.43384548678682e-08
2770 1.48843408709354e-08
2771 1.47238958803086e-08
2772 1.4024259087364e-08
2773 1.43051854806231e-08
2774 1.48211967143652e-08
2775 1.46402960865544e-08
2776 1.39740690130452e-08
2777 1.4395918235266e-08
2778 1.49691086193116e-08
2779 1.48226986240729e-08
2780 1.40108546986539e-08
2781 1.40023681538537e-08
2782 1.40411735571888e-08
2783 1.39247307018309e-08
2784 1.40458871200622e-08
2785 1.37737297123408e-08
2786 1.44969432014364e-08
2787 1.47071457234915e-08
2788 1.48595287186026e-08
2789 1.46181085014518e-08
2790 1.39974858370806e-08
2791 1.4083390453834e-08
2792 1.40123130876191e-08
2793 1.39569911183912e-08
2794 1.43817642239696e-08
2795 1.43446223788146e-08
2796 1.39234863638649e-08
2797 1.4203333620344e-08
2798 1.40306664064838e-08
2799 1.40369769141557e-08
2800 1.36380418069848e-08
2801 1.37150202306202e-08
2802 1.45932821382644e-08
2803 1.45310812271759e-08
2804 1.37843390035641e-08
2805 1.4354268884631e-08
2806 1.4664147229837e-08
2807 1.46911682818995e-08
2808 1.46645957599389e-08
2809 1.44658214296101e-08
2810 1.3967053291708e-08
2811 1.37064048999491e-08
2812 1.44703689031189e-08
2813 1.46411958112935e-08
2814 1.46996050887083e-08
2815 1.44855798467347e-08
2816 1.37809736955319e-08
2817 1.42406255676519e-08
2818 1.45637315540625e-08
2819 1.45429366327221e-08
2820 1.4086880106845e-08
2821 1.35205180384901e-08
2822 1.37858844340144e-08
2823 1.37542048861405e-08
2824 1.42970746352944e-08
2825 1.42166545202826e-08
2826 1.36085755997328e-08
2827 1.36399878059024e-08
2828 1.40211593446793e-08
2829 1.3493878903148e-08
2830 1.36479894052854e-08
2831 1.41745646331515e-08
2832 1.4558676930676e-08
2833 1.44052858530586e-08
2834 1.36061109046182e-08
2835 1.3572526214034e-08
2836 1.42354208421125e-08
2837 1.34294513287614e-08
2838 1.35681297308565e-08
2839 1.35566287084998e-08
2840 1.47039340703259e-08
2841 1.40423406236323e-08
2842 1.35125182154638e-08
2843 1.35253719335537e-08
2844 1.45657299555069e-08
2845 1.34724604805569e-08
2846 1.35310216364815e-08
2847 1.35250646238205e-08
2848 1.47708956177439e-08
2849 1.38100357816029e-08
2850 1.34844642118992e-08
2851 1.35018556335353e-08
2852 1.4715038076929e-08
2853 1.38162645768602e-08
2854 1.34596476186744e-08
2855 1.35027002912125e-08
2856 1.39840103940969e-08
2857 1.33656863354759e-08
2858 1.34827562447981e-08
2859 1.34787905281541e-08
2860 1.45795517880742e-08
2861 1.36420581497987e-08
2862 1.34863249456885e-08
2863 1.41424658650635e-08
2864 1.41390064101188e-08
2865 1.40570914908267e-08
2866 1.3610017113308e-08
2867 1.40139606585876e-08
2868 1.39203297777613e-08
2869 1.38396547555431e-08
2870 1.3446278757101e-08
2871 1.34678979080149e-08
2872 1.3813557409037e-08
2873 1.33851374428673e-08
2874 1.34560318443278e-08
2875 1.35935342981952e-08
2876 1.36675373241246e-08
2877 1.36672300143914e-08
2878 1.34471056512098e-08
2879 1.34638167281764e-08
2880 1.40298261896987e-08
2881 1.33916246980448e-08
2882 1.34704327692248e-08
2883 1.34923290318056e-08
2884 1.3869199122496e-08
2885 1.3355884398436e-08
2886 1.34176536548125e-08
2887 1.34284814379271e-08
2888 1.38088456225205e-08
2889 1.33799780144273e-08
2890 1.3412903676624e-08
2891 1.34000908147414e-08
2892 1.37780533648879e-08
2893 1.33661455237188e-08
2894 1.33964768167516e-08
2895 1.33976874039377e-08
2896 1.37726230420299e-08
2897 1.33678712543883e-08
2898 1.33903652610456e-08
2899 1.34051614253394e-08
2900 1.38013653838698e-08
2901 1.33529951540368e-08
2902 1.33860895701332e-08
2903 1.34373276949873e-08
2904 1.38083455780702e-08
2905 1.335637112021e-08
2906 1.33866988605291e-08
2907 1.3441866286712e-08
2908 1.38168934071814e-08
2909 1.33561348647504e-08
2910 1.33847395389353e-08
2911 1.34457511791197e-08
2912 1.3932781151027e-08
2913 1.33107578292879e-08
2914 1.33836106641638e-08
2915 1.34834552412144e-08
2916 1.39693652201345e-08
2917 1.33095854337739e-08
2918 1.33839295202165e-08
2919 1.35030324699414e-08
2920 1.39648683727955e-08
2921 1.3314899405259e-08
2922 1.33847235517237e-08
2923 1.35157751657289e-08
2924 1.38523850168326e-08
2925 1.33104665067663e-08
2926 1.33791928647042e-08
2927 1.34792150774388e-08
2928 1.39130467147197e-08
2929 1.33034188110059e-08
2930 1.33794140211307e-08
2931 1.35135955758869e-08
2932 1.37926345900041e-08
2933 1.33111104361205e-08
2934 1.35217010921451e-08
2935 1.3290454958792e-08
2936 1.33021398340816e-08
2937 1.35267637091374e-08
2938 1.37222029295003e-08
2939 1.3341654891974e-08
2940 1.34289415143485e-08
2941 1.3457285064078e-08
2942 1.32417881104629e-08
2943 1.41849989532261e-08
2944 1.33141826452743e-08
2945 1.339923727528e-08
2946 1.33526842915899e-08
2947 1.35320608052325e-08
2948 1.32261162022473e-08
2949 1.32680488817982e-08
2950 1.33251480960439e-08
2951 1.34596565004585e-08
2952 1.36280737805805e-08
2953 1.32718831480361e-08
2954 1.34534010598486e-08
2955 1.33815740710475e-08
2956 1.34287612141293e-08
2957 1.32560602494891e-08
2958 1.35465363371168e-08
2959 1.33245636746437e-08
2960 1.34279156682737e-08
2961 1.34773747717531e-08
2962 1.32634445648705e-08
2963 1.35348612317898e-08
2964 1.33523538892177e-08
2965 1.34984468047605e-08
2966 1.34786217742544e-08
2967 1.31899815514203e-08
2968 1.32288304754979e-08
2969 1.32893260840206e-08
2970 1.41070080061922e-08
2971 1.32106512396035e-08
2972 1.33460540396868e-08
2973 1.32895694449076e-08
2974 1.34564199782972e-08
2975 1.31771304978656e-08
2976 1.32348958459261e-08
2977 1.3258019571083e-08
2978 1.33568631710546e-08
2979 1.35902391562581e-08
2980 1.33214177466812e-08
2981 1.32155761889408e-08
2982 1.33943549585069e-08
2983 1.32124977625381e-08
2984 1.34296698206526e-08
2985 1.3392490672004e-08
2986 1.31527464475312e-08
2987 1.32413804365683e-08
2988 1.36162983110921e-08
2989 1.32638975358645e-08
2990 1.32683259934652e-08
2991 1.33769395560535e-08
2992 1.36682718476777e-08
2993 1.31241986167652e-08
2994 1.33444162386809e-08
2995 1.33571607108252e-08
2996 1.33361561793777e-08
2997 1.35619986352253e-08
2998 1.32234552197019e-08
2999 1.33667077406585e-08
3000 6.15112671908946e-09
3001 6.25171159285287e-09
3002 6.35374197699434e-09
3003 6.40165120913139e-09
3004 6.42081632307168e-09
3005 6.42446096321692e-09
3006 6.42359720970376e-09
3007 6.42137365503004e-09
3008 6.41930419931214e-09
3009 6.41731956463332e-09
3010 6.41566844095109e-09
3011 6.4141900679715e-09
3012 6.41261843625784e-09
3013 6.41125996736491e-09
3014 6.40995123646348e-09
3015 6.4085541318093e-09
3016 6.40716146804721e-09
3017 6.40575326116277e-09
3018 6.40491260028853e-09
3019 6.40353547964878e-09
3020 6.40260733320019e-09
3021 6.4013496725579e-09
3022 6.40021857734041e-09
3023 6.39918740219514e-09
3024 6.39822372860976e-09
3025 6.39726316364886e-09
3026 6.3962550811425e-09
3027 6.39526520629374e-09
3028 6.39433217486385e-09
3029 6.39328190388255e-09
3030 6.39249764233796e-09
3031 6.3915841508333e-09
3032 6.39068042929125e-09
3033 6.38998942648072e-09
3034 6.38903951966086e-09
3035 6.38827346577386e-09
3036 6.38734087843318e-09
3037 6.38652331019784e-09
3038 6.38574171318851e-09
3039 6.38483710346804e-09
3040 6.3840324138198e-09
3041 6.38324992863204e-09
3042 6.38244834760826e-09
3043 6.38168184963206e-09
3044 6.381029926672e-09
3045 6.38006891762188e-09
3046 6.37938768477397e-09
3047 6.37862873631434e-09
3048 6.37784580703737e-09
3049 6.37723784890909e-09
3050 6.37642472156585e-09
3051 6.37571417883009e-09
3052 6.37480601639595e-09
3053 6.37430996874855e-09
3054 6.37354524712919e-09
3055 6.37282759896607e-09
3056 6.37220498589386e-09
3057 6.37148378501706e-09
3058 6.37088470867297e-09
3059 6.37008712445208e-09
3060 6.3694343133136e-09
3061 6.36881969384717e-09
3062 6.36820196575627e-09
3063 6.36753805238754e-09
3064 6.36666408482256e-09
3065 6.3660752225303e-09
3066 6.36533847853116e-09
3067 6.36475716575546e-09
3068 6.36408836740543e-09
3069 6.36340757864673e-09
3070 6.36280894639185e-09
3071 6.36227159844793e-09
3072 6.36165964351676e-09
3073 6.36113606233835e-09
3074 6.36048769209197e-09
3075 6.35989216846156e-09
3076 6.35927133174619e-09
3077 6.35869668030864e-09
3078 6.35797281489658e-09
3079 6.35726271625003e-09
3080 6.35669961113194e-09
3081 6.35618047084563e-09
3082 6.3554668194854e-09
3083 6.35482289013112e-09
3084 6.35444097341065e-09
3085 6.35372510160437e-09
3086 6.35299413076496e-09
3087 6.35256514058824e-09
3088 6.35181773844806e-09
3089 6.35139230098503e-09
3090 6.3506742087327e-09
3091 6.35010755090093e-09
3092 6.3496057300938e-09
3093 6.34897867612949e-09
3094 6.34850838565626e-09
3095 6.34784846909042e-09
3096 6.34728669623996e-09
3097 6.3466436550641e-09
3098 6.34598951165799e-09
3099 6.34543617650252e-09
3100 6.34504582208706e-09
3101 6.34433661161893e-09
3102 6.3438032604779e-09
3103 6.34323660264613e-09
3104 6.34268459975829e-09
3105 6.34195096438361e-09
3106 6.34139984967419e-09
3107 6.34090113749153e-09
3108 6.34031760426979e-09
3109 6.33983976427999e-09
3110 6.33923447068696e-09
3111 6.33859942311688e-09
3112 6.33826502394186e-09
3113 6.33757046841765e-09
3114 6.33693097995547e-09
3115 6.33639052338708e-09
3116 6.33590824250518e-09
3117 6.33524832593935e-09
3118 6.33494634527665e-09
3119 6.33427665874819e-09
3120 6.33375263348057e-09
3121 6.33313312903283e-09
3122 6.33270591521296e-09
3123 6.33228980362333e-09
3124 6.3316258902546e-09
3125 6.33119734416709e-09
3126 6.33055652343728e-09
3127 6.33012664508215e-09
3128 6.32950847290203e-09
3129 6.32905994280009e-09
3130 6.32849994630647e-09
3131 6.32803498490375e-09
3132 6.32730001726145e-09
3133 6.32693364366332e-09
3134 6.32629371111193e-09
3135 6.32584917781287e-09
3136 6.32519592258518e-09
3137 6.32469054906437e-09
3138 6.32426511160133e-09
3139 6.32346397466677e-09
3140 6.32318641891061e-09
3141 6.32267616040849e-09
3142 6.32214725015956e-09
3143 6.32164320890638e-09
3144 6.32107521880698e-09
3145 6.32051744275941e-09
3146 6.31990415556061e-09
3147 6.31944541140683e-09
3148 6.31894936375943e-09
3149 6.31844176979257e-09
3150 6.3179967924043e-09
3151 6.31742214096676e-09
3152 6.31687280261417e-09
3153 6.3163825281265e-09
3154 6.31582164345446e-09
3155 6.31541707818428e-09
3156 6.31500407521912e-09
3157 6.31429308839415e-09
3158 6.31391072758447e-09
3159 6.31343199941625e-09
3160 6.31290042463206e-09
3161 6.31219254643156e-09
3162 6.31195362643666e-09
3163 6.31120444793964e-09
3164 6.31082919255732e-09
3165 6.31037311293881e-09
3166 6.31000629525147e-09
3167 6.30941610069158e-09
3168 6.30877083906967e-09
3169 6.3083169798972e-09
3170 6.3076361911385e-09
3171 6.30717078564658e-09
3172 6.30659835465508e-09
3173 6.30612984053869e-09
3174 6.30564933601363e-09
3175 6.30498853126937e-09
3176 6.30464036532885e-09
3177 6.30422114511475e-09
3178 6.30359986431017e-09
3179 6.30313357063983e-09
3180 6.30241414611987e-09
3181 6.30211482999243e-09
3182 6.30155394532039e-09
3183 6.30102503507146e-09
3184 6.3005716199882e-09
3185 6.30020569047929e-09
3186 6.29966390164327e-09
3187 6.29912566552093e-09
3188 6.29858254441729e-09
3189 6.29798524443004e-09
3190 6.29747898273081e-09
3191 6.29715479760762e-09
3192 6.29649044014968e-09
3193 6.29597574075547e-09
3194 6.29563423615309e-09
3195 6.29495477966202e-09
3196 6.29443475119729e-09
3197 6.29395113804776e-09
3198 6.29345420222194e-09
3199 6.29288310349807e-09
3200 6.29231511339867e-09
3201 6.29194962797897e-09
3202 6.2913874110393e-09
3203 6.29094865089996e-09
3204 6.2905671782687e-09
3205 6.28997121054908e-09
3206 6.2896114982891e-09
3207 6.28901064558818e-09
3208 6.28845953087875e-09
3209 6.28785778999941e-09
3210 6.28749496911496e-09
3211 6.28698071380995e-09
3212 6.28648422207334e-09
3213 6.28592644602577e-09
3214 6.28547347503172e-09
3215 6.28500007593402e-09
3216 6.28446805706062e-09
3217 6.28399021707082e-09
3218 6.28344132280745e-09
3219 6.28294438698163e-09
3220 6.28265306445996e-09
3221 6.28223162379982e-09
3222 6.2815228574209e-09
3223 6.28104235289584e-09
3224 6.28067819974376e-09
3225 6.28003871128158e-09
3226 6.27956087129178e-09
3227 6.27926510787802e-09
3228 6.27876017844642e-09
3229 6.27825258447956e-09
3230 6.2776410736376e-09
3231 6.27717122725358e-09
3232 6.27674801023659e-09
3233 6.276099195901e-09
3234 6.27570129196897e-09
3235 6.27520835294604e-09
3236 6.27470875258496e-09
3237 6.27422780397069e-09
3238 6.27371310457647e-09
3239 6.27319529655779e-09
3240 6.27269036712619e-09
3241 6.27231955263596e-09
3242 6.2717981919036e-09
3243 6.27126217622731e-09
3244 6.27099439043377e-09
3245 6.27036556011262e-09
3246 6.26976826012537e-09
3247 6.26941831782801e-09
3248 6.2690075353089e-09
3249 6.26834806283227e-09
3250 6.26801410774647e-09
3251 6.2675376000243e-09
3252 6.26693452687732e-09
3253 6.266487329043e-09
3254 6.26586382779237e-09
3255 6.26544283122143e-09
3256 6.26497387301583e-09
3257 6.26448315443895e-09
3258 6.26399287995127e-09
3259 6.26366070122231e-09
3260 6.26313845231152e-09
3261 6.26260243663523e-09
3262 6.26205043374739e-09
3263 6.26173202178393e-09
3264 6.26115026491902e-09
3265 6.2607568018791e-09
3266 6.26017770954945e-09
3267 6.25969809320281e-09
3268 6.25920426600146e-09
3269 6.25867624393095e-09
3270 6.25814866594965e-09
3271 6.25773655116291e-09
3272 6.25725515845943e-09
3273 6.2566587466506e-09
3274 6.25634832829292e-09
3275 6.25577145640932e-09
3276 6.25522300623516e-09
3277 6.2546030576982e-09
3278 6.25403728804486e-09
3279 6.2537912626226e-09
3280 6.25323615111029e-09
3281 6.25288221201004e-09
3282 6.25241458607206e-09
3283 6.25180263114089e-09
3284 6.25131058029638e-09
3285 6.25087359651388e-09
3286 6.25044238589112e-09
3287 6.24974916263454e-09
3288 6.24931395520889e-09
3289 6.24886276057168e-09
3290 6.24827611872547e-09
3291 6.24792617642811e-09
3292 6.24740303933891e-09
3293 6.24700691176372e-09
3294 6.24650775549185e-09
3295 6.24608853527775e-09
3296 6.24549212346892e-09
3297 6.24495966050631e-09
3298 6.24461593545789e-09
3299 6.24404261628797e-09
3300 6.24361362611126e-09
3301 6.24300033891245e-09
3302 6.24255447334576e-09
3303 6.24197760146217e-09
3304 6.24160856332878e-09
3305 6.24101659241205e-09
3306 6.24039309116142e-09
3307 6.24009866001529e-09
3308 6.23953866352167e-09
3309 6.23910434427444e-09
3310 6.23863050108753e-09
3311 6.23816287514956e-09
3312 6.23770679553104e-09
3313 6.2371414699669e-09
3314 6.23670315391678e-09
3315 6.23614226924474e-09
3316 6.2357212726738e-09
3317 6.23530427290575e-09
3318 6.23483664696778e-09
3319 6.2343237239304e-09
3320 6.23380724817935e-09
3321 6.23327700566279e-09
3322 6.23286799950051e-09
3323 6.23239149177834e-09
3324 6.2317040416815e-09
3325 6.23147089484632e-09
3326 6.23091889195848e-09
3327 6.23047169412416e-09
3328 6.23023144186163e-09
3329 6.22975226960421e-09
3330 6.22923845838841e-09
3331 6.22865536925588e-09
3332 6.22805140793048e-09
3333 6.22761220370194e-09
3334 6.22717655218707e-09
3335 6.22675289108088e-09
3336 6.2261316102763e-09
3337 6.22584961362804e-09
3338 6.22521900695006e-09
3339 6.22464302324488e-09
3340 6.22424689566969e-09
3341 6.22385565307582e-09
3342 6.2234537523409e-09
3343 6.22288709450913e-09
3344 6.22245854842163e-09
3345 6.22180884590762e-09
3346 6.22154683327381e-09
3347 6.22082563239701e-09
3348 6.22027895857968e-09
3349 6.21989082461027e-09
3350 6.21951645740637e-09
3351 6.21902263020502e-09
3352 6.21849949311581e-09
3353 6.21813400769611e-09
3354 6.21762641372925e-09
3355 6.2173493020623e-09
3356 6.21673112988219e-09
3357 6.21626083940896e-09
3358 6.2157892166681e-09
3359 6.2152030189111e-09
3360 6.21466478278876e-09
3361 6.214404990601e-09
3362 6.21387741261969e-09
3363 6.21327611582956e-09
3364 6.21295681568768e-09
3365 6.21250118015837e-09
3366 6.21199847117282e-09
3367 6.21160456404368e-09
3368 6.21106099885083e-09
3369 6.21049967008958e-09
3370 6.21017681723401e-09
3371 6.20961815300802e-09
3372 6.2090896868483e-09
3373 6.20877349533089e-09
3374 6.20840490128671e-09
3375 6.20793061401059e-09
3376 6.20735907119752e-09
3377 6.2069074324711e-09
3378 6.20629725389676e-09
3379 6.20601880996219e-09
3380 6.20561113606755e-09
3381 6.20489526426127e-09
3382 6.20464390976849e-09
3383 6.20403417528337e-09
3384 6.20360074421455e-09
3385 6.20307227805483e-09
3386 6.20271656259774e-09
3387 6.20219076097328e-09
3388 6.20171247689427e-09
3389 6.20124840366998e-09
3390 6.20079809721119e-09
3391 6.20044282584331e-09
3392 6.19994233730381e-09
3393 6.19933437917553e-09
3394 6.19913897992319e-09
3395 6.19850926142362e-09
3396 6.19792706046951e-09
3397 6.19759354947291e-09
3398 6.19722939632084e-09
3399 6.19651929767429e-09
3400 6.19600681872612e-09
3401 6.19553741643131e-09
3402 6.19496276499376e-09
3403 6.19450979399971e-09
3404 6.19407858337695e-09
3405 6.19369489029964e-09
3406 6.19313489380602e-09
3407 6.19264550749676e-09
3408 6.19226225850866e-09
3409 6.191763546326e-09
3410 6.19142337399126e-09
3411 6.19081896857665e-09
3412 6.19048101668795e-09
3413 6.18996809365058e-09
3414 6.18949203001762e-09
3415 6.18902262772281e-09
3416 6.18869444579673e-09
3417 6.18814155473046e-09
3418 6.18761042403548e-09
3419 6.18717566069904e-09
3420 6.18666318175087e-09
3421 6.18637763238894e-09
3422 6.18572437716125e-09
3423 6.18539708341359e-09
3424 6.18496232007715e-09
3425 6.18447071332184e-09
3426 6.18406525987325e-09
3427 6.18348972025728e-09
3428 6.18310069810946e-09
3429 6.18252915529638e-09
3430 6.18203088720293e-09
3431 6.18165651999902e-09
3432 6.18118400907974e-09
3433 6.18073947578068e-09
3434 6.18026518850456e-09
3435 6.17985307371782e-09
3436 6.17925444146294e-09
3437 6.17885254072803e-09
3438 6.17824058579686e-09
3439 6.17796525048675e-09
3440 6.17751982900927e-09
3441 6.17702999861081e-09
3442 6.17653350687419e-09
3443 6.17605078190309e-09
3444 6.17563378213504e-09
3445 6.17523499002459e-09
3446 6.17478601583343e-09
3447 6.17441120454032e-09
3448 6.17389384061084e-09
3449 6.17333029140354e-09
3450 6.17275874859047e-09
3451 6.17240258904417e-09
3452 6.17206730169073e-09
3453 6.17163031790824e-09
3454 6.17101658662023e-09
3455 6.17067641428548e-09
3456 6.17005335712406e-09
3457 6.16968920397198e-09
3458 6.16925666108159e-09
3459 6.16886985937981e-09
3460 6.16835427180717e-09
3461 6.16783291107481e-09
3462 6.16751449911135e-09
3463 6.16699891153871e-09
3464 6.16645179363218e-09
3465 6.16593132107823e-09
3466 6.16553208487858e-09
3467 6.16493434080212e-09
3468 6.16454975954639e-09
3469 6.16411188758548e-09
3470 6.16371043093977e-09
3471 6.1631659775685e-09
3472 6.16267836761608e-09
3473 6.16220008353707e-09
3474 6.16184614443682e-09
3475 6.16143802645297e-09
3476 6.16100548356258e-09
3477 6.16047479695681e-09
3478 6.16005024767219e-09
3479 6.15953643645639e-09
3480 6.15897866040882e-09
3481 6.15855633157025e-09
3482 6.15819928384553e-09
3483 6.1576073129288e-09
3484 6.15726580832643e-09
3485 6.1567280162933e-09
3486 6.15634876410809e-09
3487 6.15593354069688e-09
3488 6.15527939729077e-09
3489 6.15492279365526e-09
3490 6.15455997277081e-09
3491 6.15396356096198e-09
3492 6.15342576892886e-09
3493 6.15319883934262e-09
3494 6.15273476611833e-09
3495 6.15219786226362e-09
3496 6.15184037044969e-09
3497 6.15133632919651e-09
3498 6.1508127480181e-09
3499 6.1503304671362e-09
3500 6.14986861435796e-09
3501 6.14935968812347e-09
3502 6.14899464679297e-09
3503 6.14848127966638e-09
3504 6.14810424792722e-09
3505 6.14758244310565e-09
3506 6.14721251679384e-09
3507 6.14666273435205e-09
3508 6.14626127770634e-09
3509 6.14577633228919e-09
3510 6.14538819831978e-09
3511 6.1447624766231e-09
3512 6.14443562696465e-09
3513 6.1438245602119e-09
3514 6.14350703642685e-09
3515 6.14310069479984e-09
3516 6.14237816165542e-09
3517 6.14202999571489e-09
3518 6.14157391609638e-09
3519 6.14115158725781e-09
3520 6.14073858429265e-09
3521 6.14025941203522e-09
3522 6.13969985963081e-09
3523 6.13917316982793e-09
3524 6.13878059496642e-09
3525 6.13827744189166e-09
3526 6.13783379677102e-09
3527 6.13738171395539e-09
3528 6.13679107530629e-09
3529 6.13639539182032e-09
3530 6.13614625777359e-09
3531 6.13536865756714e-09
3532 6.13496498047539e-09
3533 6.13448447595033e-09
3534 6.13401907045841e-09
3535 6.13358563938959e-09
3536 6.13310868757821e-09
3537 6.13265216387049e-09
3538 6.13212014499709e-09
3539 6.13162587370653e-09
3540 6.1312053212248e-09
3541 6.13073369848394e-09
3542 6.13047079767171e-09
3543 6.12981443381955e-09
3544 6.12942185895804e-09
3545 6.12901240870656e-09
3546 6.12853900960886e-09
3547 6.12793460419425e-09
3548 6.1276890228612e-09
3549 6.12706552161058e-09
3550 6.1265614803574e-09
3551 6.12629680318832e-09
3552 6.12581452230643e-09
3553 6.12524830856387e-09
3554 6.12491346529964e-09
3555 6.12456751980517e-09
3556 6.12406569899804e-09
3557 6.12370731900569e-09
3558 6.12329875693263e-09
3559 6.12264328125889e-09
3560 6.12229511531837e-09
3561 6.12180350856306e-09
3562 6.12142425637785e-09
3563 6.12091577423257e-09
3564 6.1204707968443e-09
3565 6.12006578748492e-09
3566 6.11948980377974e-09
3567 6.11884942713914e-09
3568 6.11852035703464e-09
3569 6.11824990670584e-09
3570 6.11767747571434e-09
3571 6.11732708932777e-09
3572 6.1168821119395e-09
3573 6.11648331982906e-09
3574 6.11574524356229e-09
3575 6.11536021821735e-09
3576 6.11493256030826e-09
3577 6.11453065957335e-09
3578 6.11406303363538e-09
3579 6.11371397951643e-09
3580 6.1131939510517e-09
3581 6.11272366057847e-09
3582 6.11225026148077e-09
3583 6.11190653643234e-09
3584 6.11143446960227e-09
3585 6.11084871593448e-09
3586 6.11044681519957e-09
3587 6.110024486361e-09
3588 6.10961015112821e-09
3589 6.10916606191836e-09
3590 6.10863670758022e-09
3591 6.10829253844258e-09
3592 6.10760775288099e-09
3593 6.10737638240266e-09
3594 6.10679062873487e-09
3595 6.10650285892689e-09
3596 6.10593264838144e-09
3597 6.10552675084364e-09
3598 6.10496764252844e-09
3599 6.10472383755223e-09
3600 6.10415806789888e-09
3601 6.10367756337382e-09
3602 6.10302519632455e-09
3603 6.10266770451062e-09
3604 6.1022698005786e-09
3605 6.10180039828379e-09
3606 6.10146511093035e-09
3607 6.10087935726256e-09
3608 6.10063644046477e-09
3609 6.10014883051235e-09
3610 6.09959549535688e-09
3611 6.09906081194822e-09
3612 6.09867978340617e-09
3613 6.09804162721161e-09
3614 6.09777828231017e-09
3615 6.09727113243252e-09
3616 6.09684747132633e-09
3617 6.09643224791512e-09
3618 6.0958229575192e-09
3619 6.09555783626092e-09
3620 6.09496098036288e-09
3621 6.09472516899245e-09
3622 6.09412964536205e-09
3623 6.09367933890326e-09
3624 6.09305672583105e-09
3625 6.09277250873674e-09
3626 6.09233286041899e-09
3627 6.09190342615307e-09
3628 6.09133721241051e-09
3629 6.09097439152606e-09
3630 6.09057115852352e-09
3631 6.09015726737994e-09
3632 6.08953731884299e-09
3633 6.08918693245641e-09
3634 6.08877748220493e-09
3635 6.08820949210553e-09
3636 6.08790529099679e-09
3637 6.08734129770028e-09
3638 6.0869727036561e-09
3639 6.08642780619562e-09
3640 6.08585803973938e-09
3641 6.0854579153613e-09
3642 6.08490502429504e-09
3643 6.08451200534432e-09
3644 6.0842224591795e-09
3645 6.08373529331629e-09
3646 6.08315797734349e-09
3647 6.08266503832056e-09
3648 6.08241190747094e-09
3649 6.08192562978616e-09
3650 6.08125416690086e-09
3651 6.08083583486518e-09
3652 6.08052141970461e-09
3653 6.07998318358227e-09
3654 6.07951733400114e-09
3655 6.07908523519995e-09
3656 6.07862427060013e-09
3657 6.0781450983427e-09
3658 6.07781380779215e-09
3659 6.07731198698502e-09
3660 6.07688654952199e-09
3661 6.0766436327242e-09
3662 6.07607608671401e-09
3663 6.07557115728241e-09
3664 6.07519901052456e-09
3665 6.07483841008616e-09
3666 6.07430061805303e-09
3667 6.07377881323146e-09
3668 6.07324679435806e-09
3669 6.07297323540479e-09
3670 6.07256467333173e-09
3671 6.07201044999783e-09
3672 6.07162187193921e-09
3673 6.07102856875485e-09
3674 6.07077099701314e-09
3675 6.07015682163592e-09
3676 6.06975802952547e-09
3677 6.06938188596473e-09
3678 6.06885386389422e-09
3679 6.06844574591037e-09
3680 6.06779693157478e-09
3681 6.06734795738362e-09
3682 6.06720496065805e-09
3683 6.06667160951702e-09
3684 6.06620131904378e-09
3685 6.06571548544821e-09
3686 6.06529004798517e-09
3687 6.06487127186028e-09
3688 6.06438410599708e-09
3689 6.06394445767933e-09
3690 6.06346395315427e-09
3691 6.062947033314e-09
3692 6.06261574276346e-09
3693 6.06210015519082e-09
3694 6.06169736627749e-09
3695 6.06108141454342e-09
3696 6.06072658726475e-09
3697 6.06031314021038e-09
3698 6.05972383382891e-09
3699 6.05928862640326e-09
3700 6.05885475124524e-09
3701 6.05841732337353e-09
3702 6.05795902330897e-09
3703 6.05759886695978e-09
3704 6.056968704371e-09
3705 6.05660455121892e-09
3706 6.05613958981621e-09
3707 6.05570393830135e-09
3708 6.05530603436932e-09
3709 6.05485528382133e-09
3710 6.05445027446194e-09
3711 6.05399019804054e-09
3712 6.05363004169135e-09
3713 6.05315797486128e-09
3714 6.05272409970325e-09
3715 6.0520903844008e-09
3716 6.05178218648916e-09
3717 6.05122840724448e-09
3718 6.0506972765495e-09
3719 6.0502376442173e-09
3720 6.04990146868545e-09
3721 6.04935923576022e-09
3722 6.04891958744247e-09
3723 6.04849903496074e-09
3724 6.04819083704911e-09
3725 6.0475504604085e-09
3726 6.04729999409415e-09
3727 6.04675509663366e-09
3728 6.04619421196162e-09
3729 6.04590510988601e-09
3730 6.04529981629298e-09
3731 6.04498717748925e-09
3732 6.04449157393105e-09
3733 6.04417715877048e-09
3734 6.04372996093616e-09
3735 6.04320371522249e-09
3736 6.04286753969063e-09
3737 6.04237948564901e-09
3738 6.04178662655386e-09
3739 6.04135630410951e-09
3740 6.04105787616049e-09
3741 6.04063465914351e-09
3742 6.04019323446892e-09
3743 6.03955552236357e-09
3744 6.03920735642305e-09
3745 6.03862471137973e-09
3746 6.03826855183343e-09
3747 6.03785421660064e-09
3748 6.03735905713165e-09
3749 6.03703975698977e-09
3750 6.03669780829819e-09
3751 6.03603034221578e-09
3752 6.0357785436338e-09
3753 6.03513727881477e-09
3754 6.03486416395072e-09
3755 6.03439165303143e-09
3756 6.03392047437978e-09
3757 6.03347105609942e-09
3758 6.03308514257606e-09
3759 6.03262861886833e-09
3760 6.0322351558284e-09
3761 6.03181993241719e-09
3762 6.03126082410199e-09
3763 6.0309863769703e-09
3764 6.03043481817167e-09
3765 6.03007821453616e-09
3766 6.02957817008587e-09
3767 6.02917005210202e-09
3768 6.02860961151919e-09
3769 6.02828009732548e-09
3770 6.02785199532718e-09
3771 6.02724448128811e-09
3772 6.0268878776526e-09
3773 6.02651129000265e-09
3774 6.02607874711225e-09
3775 6.02546412764582e-09
3776 6.02519012460334e-09
3777 6.02480110245551e-09
3778 6.02406258209953e-09
3779 6.02382987935357e-09
3780 6.02333471988459e-09
3781 6.02291105877839e-09
3782 6.02248872993982e-09
3783 6.02199090593558e-09
3784 6.02161787099931e-09
3785 6.02106098313016e-09
3786 6.02039973429669e-09
3787 6.02037220076568e-09
3788 6.01978200620579e-09
3789 6.01929395216416e-09
3790 6.01880634221175e-09
3791 6.01843330727547e-09
3792 6.01802918609451e-09
3793 6.01756378060259e-09
3794 6.01708594061279e-09
3795 6.01675598232987e-09
3796 6.01633942665103e-09
3797 6.01585847803676e-09
3798 6.01542904377084e-09
3799 6.01502447850066e-09
3800 6.01446625836388e-09
3801 6.01401284328063e-09
3802 6.01353766782609e-09
3803 6.01310601311411e-09
3804 6.01284178003425e-09
3805 6.0122760103809e-09
3806 6.01174710013197e-09
3807 6.01139982236987e-09
3808 6.01095084817871e-09
3809 6.01068261829596e-09
3810 6.01029581659418e-09
3811 6.00966387764856e-09
3812 6.00914384918383e-09
3813 6.00882366086353e-09
3814 6.00846705722802e-09
3815 6.00800076355767e-09
3816 6.00740346357043e-09
3817 6.00684701979048e-09
3818 6.00641580916772e-09
3819 6.00617822144045e-09
3820 6.00551874896382e-09
3821 6.00515903670384e-09
3822 6.00470206890691e-09
3823 6.00426464103521e-09
3824 6.00385519078372e-09
3825 6.00343508239121e-09
3826 6.00291061303437e-09
3827 6.00248872828502e-09
3828 6.00194560718137e-09
3829 6.00144955953397e-09
3830 6.00118132965122e-09
3831 6.00072480594349e-09
3832 6.00034022468776e-09
3833 5.99986771376848e-09
3834 5.99934457667928e-09
3835 5.99891025743204e-09
3836 5.99834093506502e-09
3837 5.9980607147736e-09
3838 5.99767258080419e-09
3839 5.99714500282289e-09
3840 5.99670491041593e-09
3841 5.99623106722902e-09
3842 5.99581673199623e-09
3843 5.99538907408714e-09
3844 5.99484462071587e-09
3845 5.99441296600389e-09
3846 5.99397242950772e-09
3847 5.9934621710056e-09
3848 5.99303717763178e-09
3849 5.99272009793594e-09
3850 5.9923572770515e-09
3851 5.99175198345847e-09
3852 5.99132388146018e-09
3853 5.99089311492662e-09
3854 5.9904952109946e-09
3855 5.99004623680344e-09
3856 5.98951643837609e-09
3857 5.98919758232341e-09
3858 5.98863758582979e-09
3859 5.98828897580006e-09
3860 5.98795812933872e-09
3861 5.98724758660296e-09
3862 5.98695359954604e-09
3863 5.98640514937188e-09
3864 5.98597393874911e-09
3865 5.98568616894113e-09
3866 5.98526739281624e-09
3867 5.98486193936765e-09
3868 5.98432858822662e-09
3869 5.98388893990887e-09
3870 5.98329652490293e-09
3871 5.98286975517226e-09
3872 5.98254157324618e-09
3873 5.98207972046794e-09
3874 5.98171334686981e-09
3875 5.98122529282819e-09
3876 5.98076876912046e-09
3877 5.98022475983839e-09
3878 5.9799498686175e-09
3879 5.97946003821903e-09
3880 5.97904747934308e-09
3881 5.9785030259718e-09
3882 5.97817173542126e-09
3883 5.9776508187781e-09
3884 5.9772631288979e-09
3885 5.97691895976027e-09
3886 5.97643934341363e-09
3887 5.97586868877897e-09
3888 5.9754281522828e-09
3889 5.974886807536e-09
3890 5.97457328055384e-09
3891 5.97403371216387e-09
3892 5.97357852072378e-09
3893 5.97336091701095e-09
3894 5.97279381508997e-09
3895 5.97228844156916e-09
3896 5.97188742901267e-09
3897 5.97140603630919e-09
3898 5.97087268516816e-09
3899 5.97044902406196e-09
3900 5.96995519686061e-09
3901 5.96941029940012e-09
3902 5.96915850081814e-09
3903 5.96862159696343e-09
3904 5.96826765786318e-09
3905 5.96777027794815e-09
3906 5.96731242197279e-09
3907 5.96694649246388e-09
3908 5.96643401351571e-09
3909 5.96597171664826e-09
3910 5.96552318654631e-09
3911 5.96519766915549e-09
3912 5.96474025726934e-09
3913 5.96424731824641e-09
3914 5.96386451334752e-09
3915 5.9633586957375e-09
3916 5.96300431254804e-09
3917 5.96245586237387e-09
3918 5.9620965942031e-09
3919 5.96162852417592e-09
3920 5.96126570329147e-09
3921 5.96077187609012e-09
3922 5.96026694665852e-09
3923 5.95988769447331e-09
3924 5.95951243909099e-09
3925 5.95891469501453e-09
3926 5.95861893160077e-09
3927 5.95815841109015e-09
3928 5.95768101518956e-09
3929 5.95738613995422e-09
3930 5.95683324888796e-09
3931 5.95623639298992e-09
3932 5.95585181173419e-09
3933 5.95538063308254e-09
3934 5.95500182498654e-09
3935 5.95461857599844e-09
3936 5.95415361459573e-09
3937 5.95378013557024e-09
3938 5.95326321572998e-09
3939 5.9527547335847e-09
3940 5.95256288704604e-09
3941 5.95205928988207e-09
3942 5.95161919747511e-09
3943 5.95121552038336e-09
3944 5.95061511177164e-09
3945 5.9503877380962e-09
3946 5.94974736145559e-09
3947 5.94940097187191e-09
3948 5.94889959515399e-09
3949 5.94866689240803e-09
3950 5.94815308119223e-09
3951 5.94770765971475e-09
3952 5.94718008173345e-09
3953 5.94678217780142e-09
3954 5.94631899275555e-09
3955 5.94577675983032e-09
3956 5.94537841180909e-09
3957 5.94520921382014e-09
3958 5.94465543457545e-09
3959 5.9441660482662e-09
3960 5.94373172901896e-09
3961 5.94315530122458e-09
3962 5.94273563692127e-09
3963 5.94237059559077e-09
3964 5.94187632430021e-09
3965 5.94157123501304e-09
3966 5.94106408513539e-09
3967 5.94075721949139e-09
3968 5.94033267020677e-09
3969 5.93978599638945e-09
3970 5.93923177305555e-09
3971 5.93880633559252e-09
3972 5.93853632935293e-09
3973 5.93793503256279e-09
3974 5.93751714461632e-09
3975 5.93720272945575e-09
3976 5.93669069459679e-09
3977 5.93617555111337e-09
3978 5.9358162829426e-09
3979 5.93529048131813e-09
3980 5.93489346556453e-09
3981 5.93453775010744e-09
3982 5.93410920401993e-09
3983 5.93350302224849e-09
3984 5.93318905117712e-09
3985 5.93285820471579e-09
3986 5.93238658197492e-09
3987 5.93176840979481e-09
3988 5.93129190207264e-09
3989 5.9310103495136e-09
3990 5.93057425390953e-09
3991 5.9299147814329e-09
3992 5.92969007229271e-09
3993 5.92922866360368e-09
3994 5.92863802495458e-09
3995 5.92844706659434e-09
3996 5.92789728415255e-09
3997 5.92735549531653e-09
3998 5.92703353063939e-09
3999 5.92653748299199e-09
4000 5.92603610627407e-09
4001 5.92562399148733e-09
4002 5.92530513543466e-09
4003 5.92485127626219e-09
4004 5.92447468861224e-09
4005 5.92406079746866e-09
4006 5.92364424178982e-09
4007 5.92300963830894e-09
4008 5.92261262255533e-09
4009 5.92219207007361e-09
4010 5.9216689329844e-09
4011 5.92138604815773e-09
4012 5.92088023054771e-09
4013 5.92044013814075e-09
4014 5.91986593079241e-09
4015 5.91945381600567e-09
4016 5.91901283542029e-09
4017 5.91871263111443e-09
4018 5.91818150041945e-09
4019 5.91773563485276e-09
4020 5.91756998957749e-09
4021 5.91679105710341e-09
4022 5.91651394543646e-09
4023 5.91596149845941e-09
4024 5.91568261043562e-09
4025 5.91517856918244e-09
4026 5.91466431387744e-09
4027 5.91437343544499e-09
4028 5.91380500125638e-09
4029 5.91314819331501e-09
4030 5.91291993146115e-09
4031 5.91246474002105e-09
4032 5.91212767631077e-09
4033 5.91156901208478e-09
4034 5.91116222636856e-09
4035 5.91092863544418e-09
4036 5.91025450802363e-09
4037 5.90988236126577e-09
4038 5.90949378320715e-09
4039 5.90922111243231e-09
4040 5.90854654092254e-09
4041 5.90807314182484e-09
4042 5.90766457975178e-09
4043 5.90715565351729e-09
4044 5.90684301471356e-09
4045 5.90634652297695e-09
4046 5.90591398008655e-09
4047 5.90556137325393e-09
4048 5.9050595524468e-09
4049 5.90454618532021e-09
4050 5.90427795543746e-09
4051 5.90370463626755e-09
4052 5.90335424988098e-09
4053 5.90291904245532e-09
4054 5.90237103637037e-09
4055 5.90202420269748e-09
4056 5.90154236590479e-09
4057 5.90105386777395e-09
4058 5.90079851647829e-09
4059 5.90027449121067e-09
4060 5.89985438281815e-09
4061 5.8994462648343e-09
4062 5.89891868685299e-09
4063 5.89863979882921e-09
4064 5.89807314099744e-09
4065 5.89760640323789e-09
4066 5.89719517662957e-09
4067 5.89680970719542e-09
4068 5.89644200132966e-09
4069 5.89595927635855e-09
4070 5.89555426699917e-09
4071 5.89508486470436e-09
4072 5.89461013333903e-09
4073 5.89417803453784e-09
4074 5.89385917848517e-09
4075 5.8933196100952e-09
4076 5.89297366460073e-09
4077 5.89241366810711e-09
4078 5.89202286960244e-09
4079 5.89172977072394e-09
4080 5.89130344508249e-09
4081 5.89080251245377e-09
4082 5.89030957343084e-09
4083 5.88977977500349e-09
4084 5.88941251322694e-09
4085 5.88898307896102e-09
4086 5.8885900600103e-09
4087 5.88806958745636e-09
4088 5.88769344389561e-09
4089 5.88716009275458e-09
4090 5.88670179269002e-09
4091 5.88635540310634e-09
4092 5.88596460460167e-09
4093 5.88536241963311e-09
4094 5.88510884469429e-09
4095 5.88473492157959e-09
4096 5.88405724144536e-09
4097 5.88377346844027e-09
4098 5.8833196092678e-09
4099 5.88285775648956e-09
4100 5.88245674393306e-09
4101 5.8819353832007e-09
4102 5.88152815339527e-09
4103 5.88104942522705e-09
4104 5.88051385363997e-09
4105 5.88019632985493e-09
4106 5.87974202659325e-09
4107 5.87944004593055e-09
4108 5.87877613256182e-09
4109 5.87835691234773e-09
4110 5.8781526313112e-09
4111 5.87746162850067e-09
4112 5.87712367661197e-09
4113 5.87666226792294e-09
4114 5.87611204139193e-09
4115 5.8756146614769e-09
4116 5.87528914408608e-09
4117 5.87480464275814e-09
4118 5.87450932343359e-09
4119 5.87404080931719e-09
4120 5.87357096293317e-09
4121 5.87321835610055e-09
4122 5.87253756734185e-09
4123 5.87220405634525e-09
4124 5.8720348583563e-09
4125 5.87157256148885e-09
4126 5.87082471525946e-09
4127 5.8703939487259e-09
4128 5.87003468055514e-09
4129 5.86971582450246e-09
4130 5.86920689826798e-09
4131 5.86867443530537e-09
4132 5.86820103620767e-09
4133 5.86769655086528e-09
4134 5.86727333384829e-09
4135 5.86705484195704e-09
4136 5.86642201483301e-09
4137 5.86598281060446e-09
4138 5.86556225812274e-09
4139 5.86522608259088e-09
4140 5.86472470587296e-09
4141 5.86428727800126e-09
4142 5.86381654343882e-09
4143 5.86338488872684e-09
4144 5.86293458226805e-09
4145 5.8626121735017e-09
4146 5.86212278719245e-09
4147 5.86162363092058e-09
4148 5.86127013590954e-09
4149 5.86081894127233e-09
4150 5.86053960915933e-09
4151 5.86005421965297e-09
4152 5.8595057694788e-09
4153 5.85916692941169e-09
4154 5.85864379232248e-09
4155 5.85830450816616e-09
4156 5.8578057959835e-09
4157 5.85731196878214e-09
4158 5.85713832990109e-09
4159 5.85670623109991e-09
4160 5.85616266590705e-09
4161 5.85569370770145e-09
4162 5.85529891239389e-09
4163 5.85497428318149e-09
4164 5.85440140810078e-09
4165 5.8540052805256e-09
4166 5.85349413384506e-09
4167 5.85310067080513e-09
4168 5.85269921415943e-09
4169 5.85215920168025e-09
4170 5.85175685685613e-09
4171 5.8513003331484e-09
4172 5.85095794036761e-09
4173 5.8506870459496e-09
4174 5.8501865574101e-09
4175 5.84961945548912e-09
4176 5.84917803081453e-09
4177 5.84879567000485e-09
4178 5.8484825871119e-09
4179 5.84805803782729e-09
4180 5.84772541500911e-09
4181 5.84709791695559e-09
4182 5.84687054328015e-09
4183 5.84634607392331e-09
4184 5.84586912211194e-09
4185 5.84526826941101e-09
4186 5.84501913536428e-09
4187 5.8444880046693e-09
4188 5.84403769821051e-09
4189 5.84362025435325e-09
4190 5.84328985198113e-09
4191 5.84300963168971e-09
4192 5.84247539237026e-09
4193 5.8417466419769e-09
4194 5.84171688799984e-09
4195 5.84118531321565e-09
4196 5.84065595887751e-09
4197 5.84012438409331e-09
4198 5.8397944258104e-09
4199 5.8394791224714e-09
4200 5.83906034634651e-09
4201 5.83853498881126e-09
4202 5.83826942346377e-09
4203 5.83771564421909e-09
4204 5.83726178504662e-09
4205 5.83687143063116e-09
4206 5.83643178231341e-09
4207 5.83593351421996e-09
4208 5.83557158151393e-09
4209 5.83507020479601e-09
4210 5.83467540948845e-09
4211 5.83427084421828e-09
4212 5.83387471664309e-09
4213 5.83335735271362e-09
4214 5.83289772038142e-09
4215 5.83233550344175e-09
4216 5.83199000203649e-09
4217 5.83183679125909e-09
4218 5.83122572450634e-09
4219 5.83074033499997e-09
4220 5.83030113077143e-09
4221 5.82999692966268e-09
4222 5.8295421823118e-09
4223 5.82899506440526e-09
4224 5.82857939690484e-09
4225 5.82812420546475e-09
4226 5.82766324086492e-09
4227 5.82729420273154e-09
4228 5.82690962147581e-09
4229 5.82637937895925e-09
4230 5.82598280729485e-09
4231 5.82552805994396e-09
4232 5.82521453296181e-09
4233 5.82483661304423e-09
4234 5.82428016926428e-09
4235 5.82384274139258e-09
4236 5.82341463939429e-09
4237 5.82288572914536e-09
4238 5.82242032365343e-09
4239 5.82212011934757e-09
4240 5.8216027554181e-09
4241 5.8211924169882e-09
4242 5.82082471112244e-09
4243 5.82014436645295e-09
4244 5.81988679471124e-09
4245 5.81949466393894e-09
4246 5.81901193896783e-09
4247 5.81862158455237e-09
4248 5.81810644106895e-09
4249 5.81767034546488e-09
4250 5.81733505811144e-09
4251 5.81682302325248e-09
4252 5.81635806184977e-09
4253 5.81588421866286e-09
4254 5.81562353829668e-09
4255 5.81513193154137e-09
4256 5.81475800842668e-09
4257 5.8142721748311e-09
4258 5.81381121023128e-09
4259 5.81338310823298e-09
4260 5.81290393597556e-09
4261 5.81247050490674e-09
4262 5.81208015049128e-09
4263 5.8116933487895e-09
4264 5.81117243214635e-09
4265 5.81098413832137e-09
4266 5.81044368175299e-09
4267 5.81005688005121e-09
4268 5.80958925411323e-09
4269 5.80916115211494e-09
4270 5.80858783294502e-09
4271 5.80837555830271e-09
4272 5.80771297720162e-09
4273 5.807360370369e-09
4274 5.80707526509627e-09
4275 5.80650105774794e-09
4276 5.80604586630784e-09
4277 5.80560888252535e-09
4278 5.8052775919748e-09
4279 5.80481884782102e-09
4280 5.80429437846419e-09
4281 5.80373793468425e-09
4282 5.8034337335755e-09
4283 5.80280401507594e-09
4284 5.80246695136566e-09
4285 5.8019775650564e-09
4286 5.8016120796367e-09
4287 5.80136205741155e-09
4288 5.80074166478539e-09
4289 5.80055203869279e-09
4290 5.80000980576756e-09
4291 5.79973624681429e-09
4292 5.7990918733708e-09
4293 5.79868242311932e-09
4294 5.79822589941159e-09
4295 5.79781822551695e-09
4296 5.79731596062061e-09
4297 5.79690651036913e-09
4298 5.79650771825868e-09
4299 5.79596903804713e-09
4300 5.79550185619837e-09
4301 5.79522030363933e-09
4302 5.79475090134451e-09
4303 5.79444581205735e-09
4304 5.79378278686704e-09
4305 5.79340175832499e-09
4306 5.79302472658583e-09
4307 5.79260817090699e-09
4308 5.79212100504378e-09
4309 5.79161296698771e-09
4310 5.79128212052638e-09
4311 5.79091485874983e-09
4312 5.79033399006335e-09
4313 5.78999603817465e-09
4314 5.7895070959546e-09
4315 5.78916470317381e-09
4316 5.78861802935648e-09
4317 5.7881721637898e-09
4318 5.78786174543211e-09
4319 5.78737546774732e-09
4320 5.78696468522821e-09
4321 5.78657122218829e-09
4322 5.78606718093511e-09
4323 5.78559733455108e-09
4324 5.78527536987394e-09
4325 5.78482017843385e-09
4326 5.78432413078644e-09
4327 5.78397640893513e-09
4328 5.78344083734805e-09
4329 5.78296477371509e-09
4330 5.7826041732767e-09
4331 5.7822391319462e-09
4332 5.78160674891137e-09
4333 5.78134695672361e-09
4334 5.78073144907876e-09
4335 5.7803681841051e-09
4336 5.78009906604393e-09
4337 5.77955017178056e-09
4338 5.77920333810766e-09
4339 5.77890002517734e-09
4340 5.77836267723342e-09
4341 5.77782843791397e-09
4342 5.77749670327421e-09
4343 5.77711345428611e-09
4344 5.77673953117142e-09
4345 5.77625991482478e-09
4346 5.7757918447976e-09
4347 5.77533532108987e-09
4348 5.77495962161834e-09
4349 5.77457726080866e-09
4350 5.77411674029804e-09
4351 5.77352698982736e-09
4352 5.77324099637622e-09
4353 5.77286662917231e-09
4354 5.7722977508945e-09
4355 5.77217074138048e-09
4356 5.77167336146545e-09
4357 5.7710898282437e-09
4358 5.77056402661924e-09
4359 5.77030778714516e-09
4360 5.76980108135672e-09
4361 5.76942449370677e-09
4362 5.76877434710354e-09
4363 5.76841685528962e-09
4364 5.76810821328877e-09
4365 5.76768233173652e-09
4366 5.7672031594791e-09
4367 5.76676262298292e-09
4368 5.76654146655642e-09
4369 5.76589265222083e-09
4370 5.76552494635507e-09
4371 5.76518299766349e-09
4372 5.76465053470088e-09
4373 5.76424685760912e-09
4374 5.76380365657769e-09
4375 5.76338576863122e-09
4376 5.76302117138994e-09
4377 5.76241276917244e-09
4378 5.76201175661595e-09
4379 5.76146508279862e-09
4380 5.76118353023958e-09
4381 5.7607425496542e-09
4382 5.76033443167034e-09
4383 5.75997649576721e-09
4384 5.75941916380884e-09
4385 5.75900216404079e-09
4386 5.75863179363978e-09
4387 5.75807757030589e-09
4388 5.75775604971795e-09
4389 5.75724223850216e-09
4390 5.75678082981312e-09
4391 5.75624037324474e-09
4392 5.75584113704508e-09
4393 5.75537351110711e-09
4394 5.7551972076908e-09
4395 5.75464476071375e-09
4396 5.7542099973773e-09
4397 5.75356429166618e-09
4398 5.75326142282506e-09
4399 5.75306335903747e-09
4400 5.75259884172397e-09
4401 5.7520681551182e-09
4402 5.75155789661608e-09
4403 5.75122482970869e-09
4404 5.75083891618533e-09
4405 5.75045566719723e-09
4406 5.75017811144107e-09
4407 5.74955150156597e-09
4408 5.74917091711313e-09
4409 5.748669096306e-09
4410 5.74833336486336e-09
4411 5.74779246420576e-09
4412 5.74748071358044e-09
4413 5.74688652221766e-09
4414 5.74649616780221e-09
4415 5.74621283888632e-09
4416 5.74575942380307e-09
4417 5.7452469448549e-09
4418 5.74484770865524e-09
4419 5.74444491974191e-09
4420 5.74387737373172e-09
4421 5.74359271254821e-09
4422 5.74310821122026e-09
4423 5.74269254371984e-09
4424 5.74240299755502e-09
4425 5.74186298507584e-09
4426 5.74155123445053e-09
4427 5.74109337847517e-09
4428 5.74048009127637e-09
4429 5.74005820652701e-09
4430 5.73972469553041e-09
4431 5.73927261271479e-09
4432 5.73879965770629e-09
4433 5.73841330009373e-09
4434 5.73797676040044e-09
4435 5.73735636777428e-09
4436 5.73712588547437e-09
4437 5.73679681536987e-09
4438 5.7361444483206e-09
4439 5.73590375196886e-09
4440 5.73535396952707e-09
4441 5.73492631161798e-09
4442 5.73456260255512e-09
4443 5.73415137594679e-09
4444 5.73388403424246e-09
4445 5.73321878860611e-09
4446 5.73273384318895e-09
4447 5.73248248869618e-09
4448 5.73204106402159e-09
4449 5.73163161377011e-09
4450 5.73101743839288e-09
4451 5.73056402330963e-09
4452 5.73033176465287e-09
4453 5.729904550833e-09
4454 5.72935121567753e-09
4455 5.72906655449401e-09
4456 5.7284843535399e-09
4457 5.72812508536913e-09
4458 5.72758640515758e-09
4459 5.72741720716863e-09
4460 5.72703617862658e-09
4461 5.7263878083802e-09
4462 5.72618308325445e-09
4463 5.72558489508879e-09
4464 5.72518388253229e-09
4465 5.72478864313553e-09
4466 5.72417713229356e-09
4467 5.72380809416018e-09
4468 5.72336666948559e-09
4469 5.72291725120522e-09
4470 5.72263569864617e-09
4471 5.72209790661304e-09
4472 5.72162939249665e-09
4473 5.72124925213302e-09
4474 5.72080427474475e-09
4475 5.72023317602088e-09
4476 5.72006886301324e-09
4477 5.71941161098266e-09
4478 5.71907365909397e-09
4479 5.71866820564537e-09
4480 5.71811664684674e-09
4481 5.71790437220443e-09
4482 5.71729552589773e-09
4483 5.71692826412118e-09
4484 5.71659830583826e-09
4485 5.71614355848737e-09
4486 5.71584291009231e-09
4487 5.71525005099716e-09
4488 5.71481661992834e-09
4489 5.71437341889691e-09
4490 5.7139852849275e-09
4491 5.71345015742963e-09
4492 5.71305935892497e-09
4493 5.71263036874825e-09
4494 5.71232794399634e-09
4495 5.71199354482133e-09
4496 5.71152058981284e-09
4497 5.7109281748069e-09
4498 5.71054004083749e-09
4499 5.71022917839059e-09
4500 5.70957947587658e-09
4501 5.70926728116206e-09
4502 5.70888802897684e-09
4503 5.70839553404312e-09
4504 5.70784886022579e-09
4505 5.70743630134984e-09
4506 5.70717206826998e-09
4507 5.70668445831757e-09
4508 5.70624747453508e-09
4509 5.70583091885624e-09
4510 5.70545832800917e-09
4511 5.70504354868717e-09
4512 5.70471270222583e-09
4513 5.70417180156824e-09
4514 5.70371216923604e-09
4515 5.70336400329552e-09
4516 5.70297453705848e-09
4517 5.70255043186307e-09
4518 5.70215652473394e-09
4519 5.70170577418594e-09
4520 5.70125013865663e-09
4521 5.7008833209693e-09
4522 5.70061642335418e-09
4523 5.69991609467024e-09
4524 5.69958835683337e-09
4525 5.69920599602369e-09
4526 5.69882807610611e-09
4527 5.6983675555955e-09
4528 5.69805491679176e-09
4529 5.69749136758446e-09
4530 5.69712232945108e-09
4531 5.69658542559637e-09
4532 5.6961497740815e-09
4533 5.69576608100419e-09
4534 5.69544900130836e-09
4535 5.69512348391754e-09
4536 5.694612337237e-09
4537 5.69417446527609e-09
4538 5.69383429294135e-09
4539 5.69334179800762e-09
4540 5.69289415608409e-09
4541 5.69241587200509e-09
4542 5.69214275714103e-09
4543 5.69157831975531e-09
4544 5.69124924965081e-09
4545 5.69104274816823e-09
4546 5.6903788347995e-09
4547 5.68998892447325e-09
4548 5.68942937206884e-09
4549 5.68897595698559e-09
4550 5.68863001149111e-09
4551 5.68828895097795e-09
4552 5.68778002474346e-09
4553 5.68723956817507e-09
4554 5.6870019804478e-09
4555 5.686524140458e-09
4556 5.6862581310213e-09
4557 5.68574343162709e-09
4558 5.68527314115386e-09
4559 5.68500668762795e-09
4560 5.68447777737902e-09
4561 5.68407765300094e-09
4562 5.68350966290154e-09
4563 5.68313973658974e-09
4564 5.6828359795702e-09
4565 5.68234392872569e-09
4566 5.68195934746996e-09
4567 5.68129454592281e-09
4568 5.68090907648866e-09
4569 5.6805005144156e-09
4570 5.68013414081747e-09
4571 5.67977398446828e-09
4572 5.67924685057619e-09
4573 5.67904079318282e-09
4574 5.67853541966201e-09
4575 5.67808511320322e-09
4576 5.6775566470435e-09
4577 5.67706814891267e-09
4578 5.67684654839695e-09
4579 5.67639046877844e-09
4580 5.6759921207572e-09
4581 5.67553337660343e-09
4582 5.67517943750317e-09
4583 5.67454172539783e-09
4584 5.67411451157795e-09
4585 5.67395774808688e-09
4586 5.67348434898918e-09
4587 5.6729581032755e-09
4588 5.67251756677933e-09
4589 5.67216673630355e-09
4590 5.67163205289489e-09
4591 5.67120927996712e-09
4592 5.67080915558904e-09
4593 5.67042457433331e-09
4594 5.66983970884394e-09
4595 5.66961899650664e-09
4596 5.66903812782016e-09
4597 5.66855895556273e-09
4598 5.6683684412917e-09
4599 5.66774804866554e-09
4600 5.66740654406317e-09
4601 5.66702862414559e-09
4602 5.66675462110311e-09
4603 5.6660014458032e-09
4604 5.66570790283549e-09
4605 5.6650817370496e-09
4606 5.6648965518491e-09
4607 5.66434810167493e-09
4608 5.6639724022034e-09
4609 5.66357849507426e-09
4610 5.66323832273952e-09
4611 5.66280711211675e-09
4612 5.6623585820148e-09
4613 5.661952684477e-09
4614 5.66148505853903e-09
4615 5.66092861475909e-09
4616 5.66050184502842e-09
4617 5.66014168867923e-09
4618 5.65980728950422e-09
4619 5.65930191598341e-09
4620 5.65876279168265e-09
4621 5.65837110499956e-09
4622 5.65796653972939e-09
4623 5.65767077631563e-09
4624 5.6572413420497e-09
4625 5.65681856912192e-09
4626 5.65647351180587e-09
4627 5.6559157357583e-09
4628 5.65554536535728e-09
4629 5.65514524097921e-09
4630 5.65477753511345e-09
4631 5.65409496999791e-09
4632 5.65388891260454e-09
4633 5.65331204072095e-09
4634 5.65306379485264e-09
4635 5.65253710504976e-09
4636 5.6523359326377e-09
4637 5.65167246335818e-09
4638 5.65135938046524e-09
4639 5.65096591742531e-09
4640 5.65036462063517e-09
4641 5.6499196432469e-09
4642 5.6495950140345e-09
4643 5.64920554779746e-09
4644 5.64876723174734e-09
4645 5.64847812967173e-09
4646 5.64803537272951e-09
4647 5.64764945920615e-09
4648 5.64695046278985e-09
4649 5.64681590375926e-09
4650 5.64634339283998e-09
4651 5.64594193619428e-09
4652 5.64532243174654e-09
4653 5.64487656617985e-09
4654 5.64451729800908e-09
4655 5.64407054426397e-09
4656 5.64363134003543e-09
4657 5.64322144569473e-09
4658 5.64282487403034e-09
4659 5.64238700206943e-09
4660 5.64202284891735e-09
4661 5.64147395465397e-09
4662 5.64125013369221e-09
4663 5.64074387199298e-09
4664 5.64042412776189e-09
4665 5.63992674784686e-09
4666 5.63944846376785e-09
4667 5.63889779314763e-09
4668 5.63870905523345e-09
4669 5.63809443576702e-09
4670 5.63780888640508e-09
4671 5.63735724767866e-09
4672 5.63692781341274e-09
4673 5.63640112360986e-09
4674 5.63608049120035e-09
4675 5.63555824228956e-09
4676 5.63515589746544e-09
4677 5.6347304600024e-09
4678 5.63432189792934e-09
4679 5.63395330388516e-09
4680 5.63341595594125e-09
4681 5.63308866219359e-09
4682 5.63260416086564e-09
4683 5.63225688310354e-09
4684 5.63168622846888e-09
4685 5.6312861040908e-09
4686 5.63086777205513e-09
4687 5.63040591927688e-09
4688 5.63010793541707e-09
4689 5.62973179185633e-09
4690 5.62932056524801e-09
4691 5.62893198718939e-09
4692 5.62850255292346e-09
4693 5.62808821769067e-09
4694 5.62763924349952e-09
4695 5.62705881890224e-09
4696 5.6267639436669e-09
4697 5.62628565958789e-09
4698 5.6259570335726e-09
4699 5.62544810733812e-09
4700 5.62513147173149e-09
4701 5.62462965092436e-09
4702 5.62408875026676e-09
4703 5.62378010826592e-09
4704 5.62331292641716e-09
4705 5.6228595113339e-09
4706 5.62261037728717e-09
4707 5.62215562993629e-09
4708 5.6218314448131e-09
4709 5.62127988601446e-09
4710 5.62089708111557e-09
4711 5.62049828900513e-09
4712 5.61993651615467e-09
4713 5.61960744605017e-09
4714 5.61896840167719e-09
4715 5.61873170212834e-09
4716 5.61842572466276e-09
4717 5.61780977292869e-09
4718 5.61744295524136e-09
4719 5.61686830380381e-09
4720 5.61671420484799e-09
4721 5.61622659489558e-09
4722 5.61578783475625e-09
4723 5.61518875841216e-09
4724 5.61502266904768e-09
4725 5.61455326675286e-09
4726 5.61406920951413e-09
4727 5.61365931517344e-09
4728 5.61313528990581e-09
4729 5.61275959043428e-09
4730 5.61241675356428e-09
4731 5.61213919780812e-09
4732 5.61145885313863e-09
4733 5.6111373325507e-09
4734 5.61064661397381e-09
4735 5.61028912215988e-09
4736 5.60986634923211e-09
4737 5.60920598857706e-09
4738 5.60889246159491e-09
4739 5.60843638197639e-09
4740 5.60799273685575e-09
4741 5.60748603106731e-09
4742 5.60705304408771e-09
4743 5.6067008813443e-09
4744 5.60620483369689e-09
4745 5.60584778597217e-09
4746 5.60530688531458e-09
4747 5.60512036784644e-09
4748 5.60465851506819e-09
4749 5.60411006489403e-09
4750 5.60365753798919e-09
4751 5.60335999821859e-09
4752 5.60286395057119e-09
4753 5.60232926716253e-09
4754 5.60229285184732e-09
4755 5.60172974672923e-09
4756 5.60126078852363e-09
4757 5.60070301247606e-09
4758 5.60036728103341e-09
4759 5.60004576044548e-09
4760 5.5993298886392e-09
4761 5.59913271303003e-09
4762 5.59861623727897e-09
4763 5.5982027902246e-09
4764 5.59776180963922e-09
4765 5.5974318513563e-09
4766 5.59695667590177e-09
4767 5.59658053234102e-09
4768 5.59611690320594e-09
4769 5.59572788105811e-09
4770 5.59533042121529e-09
4771 5.59498936070213e-09
4772 5.59444357506322e-09
4773 5.59413093625949e-09
4774 5.59367974162228e-09
4775 5.59328405813631e-09
4776 5.59287194334956e-09
4777 5.59245050268942e-09
4778 5.59205393102502e-09
4779 5.59166446478798e-09
4780 5.59138824129946e-09
4781 5.59079094131221e-09
4782 5.59042279135724e-09
4783 5.5900919448959e-09
4784 5.58958967999956e-09
4785 5.58906521064273e-09
4786 5.58871882105905e-09
4787 5.58836665831564e-09
4788 5.58789770011003e-09
4789 5.58745094636492e-09
4790 5.58717871967929e-09
4791 5.58677726303358e-09
4792 5.58627410995882e-09
4793 5.58553203688916e-09
4794 5.58527579741508e-09
4795 5.58503510106334e-09
4796 5.58445778509054e-09
4797 5.58396884287049e-09
4798 5.58362511782207e-09
4799 5.5832072298756e-09
4800 5.58276047613049e-09
4801 5.58235280223585e-09
4802 5.58202994938028e-09
4803 5.58169999109737e-09
4804 5.58115198501241e-09
4805 5.58077450918404e-09
4806 5.58033574904471e-09
4807 5.57994184191557e-09
4808 5.57947155144234e-09
4809 5.57887736007956e-09
4810 5.57841595139053e-09
4811 5.57817969593088e-09
4812 5.57773782716708e-09
4813 5.57732171557745e-09
4814 5.57692469982385e-09
4815 5.57654056265733e-09
4816 5.576029860066e-09
4817 5.57572876758172e-09
4818 5.57522383815012e-09
4819 5.57493606834214e-09
4820 5.57436719006432e-09
4821 5.57408874612975e-09
4822 5.57354162822321e-09
4823 5.573198347264e-09
4824 5.57286838898108e-09
4825 5.57229462572195e-09
4826 5.57181056848322e-09
4827 5.57152457503207e-09
4828 5.57122747935068e-09
4829 5.57070922724279e-09
4830 5.57050627847389e-09
4831 5.56984147692674e-09
4832 5.56952572949854e-09
4833 5.56901724735326e-09
4834 5.56853452238215e-09
4835 5.56817747465743e-09
4836 5.56786350358607e-09
4837 5.56734702783501e-09
4838 5.56688917185966e-09
4839 5.56645485261242e-09
4840 5.56604051737963e-09
4841 5.56554047292934e-09
4842 5.56515766803045e-09
4843 5.56473800372714e-09
4844 5.56425971964813e-09
4845 5.56385026939665e-09
4846 5.56351276159717e-09
4847 5.56309620591833e-09
4848 5.56263435314008e-09
4849 5.56217782943236e-09
4850 5.56156409814434e-09
4851 5.56143309182744e-09
4852 5.56083445957256e-09
4853 5.56040369303901e-09
4854 5.55992185624632e-09
4855 5.55957679893027e-09
4856 5.55903634236188e-09
4857 5.55864243523274e-09
4858 5.55826007442306e-09
4859 5.55790879985807e-09
4860 5.5574909119116e-09
4861 5.55698687065842e-09
4862 5.55669199542308e-09
4863 5.55599033447152e-09
4864 5.55560575321579e-09
4865 5.55508616884026e-09
4866 5.55485257791588e-09
4867 5.5542761501215e-09
4868 5.55393464551912e-09
4869 5.55354784381734e-09
4870 5.55300294635686e-09
4871 5.55261481238745e-09
4872 5.55222223752594e-09
4873 5.55170442950725e-09
4874 5.55126788981397e-09
4875 5.55090906573241e-09
4876 5.55039125771373e-09
4877 5.55012169556335e-09
4878 5.54976375966021e-09
4879 5.54923973439259e-09
4880 5.54884715953108e-09
4881 5.54844570288537e-09
4882 5.54800427821078e-09
4883 5.54754420178938e-09
4884 5.54718271317256e-09
4885 5.54670265273671e-09
4886 5.546445080995e-09
4887 5.54590817714029e-09
4888 5.54554713261268e-09
4889 5.54512080697123e-09
4890 5.54458612356257e-09
4891 5.5444027147189e-09
4892 5.54385959361525e-09
4893 5.54341283987014e-09
4894 5.54299361965604e-09
4895 5.54267653996021e-09
4896 5.5422790801174e-09
4897 5.54184698131621e-09
4898 5.54151702303329e-09
4899 5.5409130617079e-09
4900 5.54064083502226e-09
4901 5.54019896625846e-09
4902 5.53969492500528e-09
4903 5.53931922553375e-09
4904 5.53911094769433e-09
4905 5.53849099915737e-09
4906 5.53814283321685e-09
4907 5.5377391561251e-09
4908 5.53730394869945e-09
4909 5.53706147599087e-09
4910 5.53642909295604e-09
4911 5.53610446374364e-09
4912 5.53562129468332e-09
4913 5.53526469104781e-09
4914 5.53475754117017e-09
4915 5.53457102370203e-09
4916 5.53402346170628e-09
4917 5.53380807843951e-09
4918 5.53331602759499e-09
4919 5.53285950388727e-09
4920 5.53248424850494e-09
4921 5.53202728070801e-09
4922 5.53159029692551e-09
4923 5.53116930035458e-09
4924 5.53075851783547e-09
4925 5.53025358840387e-09
4926 5.52978063339538e-09
4927 5.52949597221186e-09
4928 5.52909940054747e-09
4929 5.52868284486863e-09
4930 5.5282081135033e-09
4931 5.52801315834017e-09
4932 5.52733903091962e-09
4933 5.52706591605556e-09
4934 5.52659562558233e-09
4935 5.52636736372847e-09
4936 5.52573897749653e-09
4937 5.52546053356195e-09
4938 5.52498091721532e-09
4939 5.52457724012356e-09
4940 5.5241642371584e-09
4941 5.52390133634617e-09
4942 5.52325740699189e-09
4943 5.52278400789419e-09
4944 5.52241186113633e-09
4945 5.52202950032665e-09
4946 5.52156009803184e-09
4947 5.52116796725954e-09
4948 5.52074830295624e-09
4949 5.52040191337255e-09
4950 5.51986545360705e-09
4951 5.51956036431989e-09
4952 5.51914336455184e-09
4953 5.51864420827997e-09
4954 5.51834888895542e-09
4955 5.51777024071498e-09
4956 5.51731460518567e-09
4957 5.51704104623241e-09
4958 5.51644463442358e-09
4959 5.51607248766572e-09
4960 5.51581225138875e-09
4961 5.51528112069377e-09
4962 5.51472512100304e-09
4963 5.51454215624858e-09
4964 5.51387202563092e-09
4965 5.51350698430042e-09
4966 5.51314238705913e-09
4967 5.51269296877877e-09
4968 5.51258283465472e-09
4969 5.51170487028685e-09
4970 5.51140688642704e-09
4971 5.51097789625032e-09
4972 5.51051604347208e-09
4973 5.51028644935059e-09
4974 5.50980105984422e-09
4975 5.50935785881279e-09
4976 5.50908652030557e-09
4977 5.50855050462928e-09
4978 5.50820278277797e-09
4979 5.50764189810593e-09
4980 5.50728262993516e-09
4981 5.50699041923508e-09
4982 5.50649437158768e-09
4983 5.50606360505412e-09
4984 5.50558132417223e-09
4985 5.5052060687899e-09
4986 5.5047686409182e-09
4987 5.50450973690886e-09
4988 5.50395418130734e-09
4989 5.50354428696664e-09
4990 5.5031228463065e-09
4991 5.50269874111109e-09
4992 5.50242207353335e-09
4993 5.50181411540507e-09
4994 5.50142553734645e-09
4995 5.50099876761578e-09
4996 5.50050494041443e-09
4997 5.50023626644247e-09
4998 5.49967138496754e-09
4999 5.49928369508734e-09
};
\addlegendentry{Test}

\end{groupplot}

\end{tikzpicture}

		% This file was created by tikzplotlib v0.9.6.
\begin{tikzpicture}

\begin{groupplot}[group style={group size=1 by 6},
legend cell align={left},
legend style={fill opacity=1, draw opacity=1, text opacity=1, draw=white},
log basis y={10},
tick align=outside,
tick pos=left,
title style={at={(0.3,0.85)},anchor=north},
x grid style={white!69.0196078431373!black},
xlabel={Epoch},
x label style={yshift=13pt},
xmin=-99.95, xmax=5098.95,
xtick style={color=black},
xtick = {0,1000,4000,5000},
y grid style={white!69.0196078431373!black},
ylabel={MSE Loss},
ymode=log,
ytick style={color=black},
width=.45\textwidth,
height=.25\textwidth
]
\nextgroupplot[
title={ReLU/ReLU $\rare$},
ymin=3.19305577949804e-09, ymax=1e-05,
]
\addplot [semithick, black, dashed]
table {%
0 0.00424815546080936
1 0.000435057971448259
2 0.000189762197962409
3 0.000167931812520692
4 0.000131270386419601
5 6.85811526345788e-05
6 2.9000374238052e-05
7 1.66094894836988e-05
8 9.62847914439635e-06
9 6.19399008834876e-06
10 4.6593035775544e-06
11 3.8813620451208e-06
12 3.37336417525336e-06
13 2.98629472329992e-06
14 2.69124156937295e-06
15 2.47254889869453e-06
16 2.31235084714143e-06
17 2.19397613464878e-06
18 2.10415537526387e-06
19 2.03294091955186e-06
20 1.97283657425729e-06
21 1.92094396080478e-06
22 1.87479287488301e-06
23 1.83091118531564e-06
24 1.78827610339738e-06
25 1.74649451493636e-06
26 1.70659933403172e-06
27 1.66835484443695e-06
28 1.63115397796432e-06
29 1.59410077338151e-06
30 1.55742810543202e-06
31 1.51993886863622e-06
32 1.48355703460901e-06
33 1.44726320039013e-06
34 1.41086987642325e-06
35 1.37507953356319e-06
36 1.34007832194882e-06
37 1.30434704133009e-06
38 1.26970690233819e-06
39 1.23465064763906e-06
40 1.20063609341514e-06
41 1.16763545761955e-06
42 1.13424724770539e-06
43 1.10079717212841e-06
44 1.0681530864538e-06
45 1.03548723613045e-06
46 1.00439113998618e-06
47 9.72032490011543e-07
48 9.40660798899984e-07
49 9.1058181492798e-07
50 8.81645077019044e-07
51 8.54980221841828e-07
52 8.28105153253489e-07
53 8.03589792571557e-07
54 7.79670177323055e-07
55 7.57659733725546e-07
56 7.35677339370966e-07
57 7.16398104238181e-07
58 6.98210022715529e-07
59 6.80927871634651e-07
60 6.64981120522157e-07
61 6.5085474341231e-07
62 6.37668469222419e-07
63 6.25875315380497e-07
64 6.14697549467635e-07
65 6.04722611166508e-07
66 5.95435091533147e-07
67 5.8659890211743e-07
68 5.78165960195776e-07
69 5.70661046538135e-07
70 5.63327344282882e-07
71 5.56679247502245e-07
72 5.50689328299825e-07
73 5.45219224630245e-07
74 5.39869871966303e-07
75 5.34274461964301e-07
76 5.29050406779419e-07
77 5.24338659815271e-07
78 5.19760935423719e-07
79 5.15059175292265e-07
80 5.10600914861214e-07
81 5.06577974020317e-07
82 5.02651408959664e-07
83 4.99018968165288e-07
84 4.95496781569571e-07
85 4.9227233207283e-07
86 4.88538983375975e-07
87 4.85445042285093e-07
88 4.81980573011853e-07
89 4.79309150421159e-07
90 4.76092043708576e-07
91 4.73506889818509e-07
92 4.71065732762455e-07
93 4.6826379077558e-07
94 4.65582789590968e-07
95 4.63223471980356e-07
96 4.60732275486819e-07
97 4.58262430568013e-07
98 4.55574308332984e-07
99 4.53036423635922e-07
100 4.50479949360982e-07
101 4.47766368344205e-07
102 4.4529767358803e-07
103 4.42895762352791e-07
104 4.40528223144554e-07
105 4.38393048110441e-07
106 4.35774731116112e-07
107 4.33268862661151e-07
108 4.30829531515187e-07
109 4.28441580671901e-07
110 4.2601796066144e-07
111 4.23485476195751e-07
112 4.21120904045225e-07
113 4.18565086617306e-07
114 4.16050392338718e-07
115 4.13626865812589e-07
116 4.11067495720729e-07
117 4.08723653706033e-07
118 4.05992933984578e-07
119 4.03508337585734e-07
120 4.0112196079356e-07
121 3.98603061775304e-07
122 3.96133076270289e-07
123 3.93572736692605e-07
124 3.91096142216441e-07
125 3.88498279255245e-07
126 3.85909240824489e-07
127 3.83463279005625e-07
128 3.80851771836888e-07
129 3.78345855477136e-07
130 3.75854788812191e-07
131 3.73143167253787e-07
132 3.70521531456447e-07
133 3.68087617330204e-07
134 3.65562920237039e-07
135 3.63034729165435e-07
136 3.6044595570317e-07
137 3.58023074587876e-07
138 3.55321015096166e-07
139 3.52404923903649e-07
140 3.49606269560709e-07
141 3.46884312122597e-07
142 3.44353602541148e-07
143 3.41331828073166e-07
144 3.38780055441745e-07
145 3.36294802357529e-07
146 3.33764976680584e-07
147 3.31071807261551e-07
148 3.28384308652119e-07
149 3.25897894962068e-07
150 3.22985789782848e-07
151 3.20340478060288e-07
152 3.17635469665234e-07
153 3.14474523980479e-07
154 3.11338886563206e-07
155 3.08522893956109e-07
156 3.05513705555427e-07
157 3.0255954671432e-07
158 2.99897978271346e-07
159 2.97140602443235e-07
160 2.94597975614153e-07
161 2.91773963335018e-07
162 2.89537810505891e-07
163 2.87213933535924e-07
164 2.84562977711289e-07
165 2.81421373513169e-07
166 2.78560743916856e-07
167 2.75842793138636e-07
168 2.73365475237597e-07
169 2.71083009492301e-07
170 2.68928278671865e-07
171 2.66628273354996e-07
172 2.64336555481748e-07
173 2.62530875373201e-07
174 2.60523335697371e-07
175 2.58115738395404e-07
176 2.5591376335754e-07
177 2.53917169628082e-07
178 2.51970870607465e-07
179 2.503769481077e-07
180 2.48665452598829e-07
181 2.46962304506937e-07
182 2.4557531019731e-07
183 2.44044473907579e-07
184 2.42630377179864e-07
185 2.41228858879339e-07
186 2.39680273862675e-07
187 2.38389664710681e-07
188 2.36874634801154e-07
189 2.35392372648135e-07
190 2.34151626226975e-07
191 2.32785328901031e-07
192 2.31603315338447e-07
193 2.30207622301748e-07
194 2.29304298972721e-07
195 2.28338629749913e-07
196 2.27273066699674e-07
197 2.26397170518311e-07
198 2.25501060617361e-07
199 2.24453001511193e-07
200 2.23295516258659e-07
201 2.22515688911074e-07
202 2.21515830054031e-07
203 2.20638683614993e-07
204 2.19982666687102e-07
205 2.19080753048218e-07
206 2.18199480036318e-07
207 2.17460962367966e-07
208 2.16596045035544e-07
209 2.1600066935612e-07
210 2.15096834487838e-07
211 2.14449176572984e-07
212 2.13727904929684e-07
213 2.12970434375848e-07
214 2.12246125419924e-07
215 2.11542017101607e-07
216 2.1081144660684e-07
217 2.10249034989118e-07
218 2.09626413298025e-07
219 2.08911659763444e-07
220 2.08243099075744e-07
221 2.07589092068972e-07
222 2.06845510300369e-07
223 2.06149647921094e-07
224 2.05533620890819e-07
225 2.049862049045e-07
226 2.04395188189821e-07
227 2.03881905893155e-07
228 2.03287663852691e-07
229 2.02651977271984e-07
230 2.02092363300821e-07
231 2.01510372449754e-07
232 2.00902295689254e-07
233 2.00287396341814e-07
234 1.99739031234714e-07
235 1.99169150191914e-07
236 1.98754382582322e-07
237 1.98254360452843e-07
238 1.97850015281276e-07
239 1.97092209088368e-07
240 1.96750689944736e-07
241 1.95671652516438e-07
242 1.95595526455428e-07
243 1.94767820600461e-07
244 1.94386144554137e-07
245 1.93681605946239e-07
246 1.93596427288334e-07
247 1.92714548366446e-07
248 1.92543469094808e-07
249 1.9203274666868e-07
250 1.91531600351347e-07
251 1.91138696211901e-07
252 1.90748648975614e-07
253 1.9020791297919e-07
254 1.89639417080834e-07
255 1.89055506933045e-07
256 1.88720974091616e-07
257 1.88542377409284e-07
258 1.88308799226711e-07
259 1.87431821763617e-07
260 1.87159406856985e-07
261 1.86969584752283e-07
262 1.86424541842811e-07
263 1.86144191478554e-07
264 1.8579510180583e-07
265 1.85335582211899e-07
266 1.84892544378812e-07
267 1.84616216797551e-07
268 1.83911076491405e-07
269 1.84031709732579e-07
270 1.83531840812101e-07
271 1.83045601508169e-07
272 1.82653509094521e-07
273 1.82254203043186e-07
274 1.82040355307223e-07
275 1.81644162298156e-07
276 1.81305120755759e-07
277 1.80835162066018e-07
278 1.80642715531754e-07
279 1.80261498904599e-07
280 1.79922487981266e-07
281 1.79648038706404e-07
282 1.79397216833799e-07
283 1.78949224420855e-07
284 1.7853098128473e-07
285 1.78167730435774e-07
286 1.77709466995246e-07
287 1.77853037551579e-07
288 1.77387857972278e-07
289 1.76953894010623e-07
290 1.76933787086497e-07
291 1.76405864086071e-07
292 1.76014165498728e-07
293 1.75808594848448e-07
294 1.75725479476974e-07
295 1.75182093085091e-07
296 1.74885820412918e-07
297 1.74566130304754e-07
298 1.74346736566022e-07
299 1.74047682578227e-07
300 1.73906863540019e-07
301 1.73389073951036e-07
302 1.73062626107168e-07
303 1.72921720276165e-07
304 1.72528606740663e-07
305 1.72300348916465e-07
306 1.72084555523533e-07
307 1.71836753033006e-07
308 1.71465894354128e-07
309 1.7112539198294e-07
310 1.70872920251419e-07
311 1.70605289532944e-07
312 1.70385468480028e-07
313 1.70130386010214e-07
314 1.69905583768504e-07
315 1.69592226269444e-07
316 1.69202845262895e-07
317 1.68877527317868e-07
318 1.68609611195691e-07
319 1.68337833686749e-07
320 1.68072871401748e-07
321 1.67761122098042e-07
322 1.67488987274034e-07
323 1.67191938224676e-07
324 1.6676562741047e-07
325 1.66676025508572e-07
326 1.6650844432764e-07
327 1.66246139484016e-07
328 1.66012244012137e-07
329 1.65670316984112e-07
330 1.65575543601193e-07
331 1.65224845542156e-07
332 1.65020554208084e-07
333 1.64668414582181e-07
334 1.64583939577412e-07
335 1.64019487439049e-07
336 1.64063422945837e-07
337 1.63705269515368e-07
338 1.63516173893541e-07
339 1.63155346640131e-07
340 1.62823385009503e-07
341 1.62456308550496e-07
342 1.62269940631177e-07
343 1.61946965445026e-07
344 1.61862433320614e-07
345 1.61575150711357e-07
346 1.61238239463124e-07
347 1.60965146624115e-07
348 1.60688239230211e-07
349 1.60439943021728e-07
350 1.60151473545866e-07
351 1.59810571005892e-07
352 1.59597665048139e-07
353 1.5928657716735e-07
354 1.59000839494716e-07
355 1.58756361704704e-07
356 1.58474408398668e-07
357 1.58128373139554e-07
358 1.57861737114295e-07
359 1.5761818485327e-07
360 1.57252084613368e-07
361 1.57008953265425e-07
362 1.56739590440935e-07
363 1.56457586719227e-07
364 1.56187528310703e-07
365 1.55909436531765e-07
366 1.55627319543328e-07
367 1.55392255344111e-07
368 1.55123535054535e-07
369 1.54823913240953e-07
370 1.54569770510982e-07
371 1.54281409834311e-07
372 1.53892298469849e-07
373 1.53696336122167e-07
374 1.53429735871846e-07
375 1.53320450371197e-07
376 1.52944868410998e-07
377 1.52739271817914e-07
378 1.52383835135161e-07
379 1.52206618250439e-07
380 1.51931956925644e-07
381 1.51672619558063e-07
382 1.51451936873492e-07
383 1.51142565835372e-07
384 1.50998153929827e-07
385 1.50683358535186e-07
386 1.50311609983689e-07
387 1.5012170769424e-07
388 1.49969492298396e-07
389 1.49643036039038e-07
390 1.49411542932132e-07
391 1.49249351797831e-07
392 1.49017345382862e-07
393 1.48795133026791e-07
394 1.48497611293319e-07
395 1.48207912142428e-07
396 1.47992186861234e-07
397 1.47962900229004e-07
398 1.47757954583216e-07
399 1.47158615888543e-07
400 1.47316339761616e-07
401 1.46993050953892e-07
402 1.46767840986328e-07
403 1.4652472835186e-07
404 1.46313419343791e-07
405 1.45987183353125e-07
406 1.45859760058542e-07
407 1.45577776073491e-07
408 1.45535278232689e-07
409 1.45201303537679e-07
410 1.44978553921593e-07
411 1.44777858900991e-07
412 1.4447704469811e-07
413 1.44304610669721e-07
414 1.43977022751152e-07
415 1.43891438745936e-07
416 1.43532466812424e-07
417 1.43551556297528e-07
418 1.43006378317434e-07
419 1.4288141313612e-07
420 1.42825717825534e-07
421 1.42618660201155e-07
422 1.42172071846414e-07
423 1.42003343532426e-07
424 1.41941651906397e-07
425 1.41615189935607e-07
426 1.41347903639843e-07
427 1.41151291103814e-07
428 1.40775774098323e-07
429 1.40562605778882e-07
430 1.40465662052947e-07
431 1.4027266404204e-07
432 1.39983828825851e-07
433 1.39687308895464e-07
434 1.39709143374134e-07
435 1.39419567323884e-07
436 1.39105666007389e-07
437 1.38915069965329e-07
438 1.38687596752618e-07
439 1.3857179635357e-07
440 1.38295893015172e-07
441 1.38336580099541e-07
442 1.37772480347742e-07
443 1.37426858945133e-07
444 1.37563303192589e-07
445 1.37025655363221e-07
446 1.37167833615059e-07
447 1.37067577048455e-07
448 1.36293280031285e-07
449 1.36455578779504e-07
450 1.36297026204435e-07
451 1.36085189196944e-07
452 1.35919055035405e-07
453 1.3558689242954e-07
454 1.35368424873494e-07
455 1.35016059238158e-07
456 1.34707753177921e-07
457 1.34607501285799e-07
458 1.34278737550009e-07
459 1.34129820615225e-07
460 1.33909046605929e-07
461 1.33702183328754e-07
462 1.33556590499406e-07
463 1.33230767301651e-07
464 1.33061737410678e-07
465 1.32760428043888e-07
466 1.32540664385505e-07
467 1.32401125131132e-07
468 1.3224408322543e-07
469 1.31760213176069e-07
470 1.31633250853191e-07
471 1.31573897540704e-07
472 1.3133436752133e-07
473 1.31147745498605e-07
474 1.30948124234642e-07
475 1.30738618299375e-07
476 1.30343254412413e-07
477 1.30392368987931e-07
478 1.29947172932354e-07
479 1.30137911068751e-07
480 1.29593972380437e-07
481 1.29413516233079e-07
482 1.29255094937886e-07
483 1.29176143447207e-07
484 1.28989702846471e-07
485 1.28682602106522e-07
486 1.28309985165664e-07
487 1.28418412822295e-07
488 1.2794559228313e-07
489 1.27700379753737e-07
490 1.27829166939541e-07
491 1.27474781594916e-07
492 1.27258703416011e-07
493 1.26978213038864e-07
494 1.2700297371504e-07
495 1.26633682834942e-07
496 1.26286786168439e-07
497 1.26416906477633e-07
498 1.26052876162852e-07
499 1.25703169519653e-07
500 1.25515927225983e-07
501 1.25490312502752e-07
502 1.25442690421806e-07
503 1.25166660824849e-07
504 1.2503167359279e-07
505 1.24835667635548e-07
506 1.24585203054473e-07
507 1.24426838409519e-07
508 1.24220908732653e-07
509 1.23944701840628e-07
510 1.23680672211801e-07
511 1.23558205577634e-07
512 1.23368014614655e-07
513 1.23172760624346e-07
514 1.22898605584787e-07
515 1.2282914784878e-07
516 1.22644103131897e-07
517 1.22399048727573e-07
518 1.22169854755771e-07
519 1.2197011452697e-07
520 1.21765031427135e-07
521 1.21627625884191e-07
522 1.21458049541445e-07
523 1.21074591386883e-07
524 1.21058107675598e-07
525 1.20843347651611e-07
526 1.20642153614181e-07
527 1.20233275749282e-07
528 1.20077605147983e-07
529 1.19917387844559e-07
530 1.19788355230632e-07
531 1.19557362430189e-07
532 1.19360559613479e-07
533 1.19143739851779e-07
534 1.18985526991189e-07
535 1.18754830760182e-07
536 1.18802140642416e-07
537 1.18344307644236e-07
538 1.18224777730269e-07
539 1.18028503412582e-07
540 1.17952973868629e-07
541 1.17901333386516e-07
542 1.17639091155475e-07
543 1.1760608464062e-07
544 1.17229968500165e-07
545 1.17215531762582e-07
546 1.16988303756838e-07
547 1.16838979349776e-07
548 1.167261327506e-07
549 1.16460594381973e-07
550 1.1629809991387e-07
551 1.16055068769327e-07
552 1.15945436412801e-07
553 1.15701884386255e-07
554 1.15568513667785e-07
555 1.15378380168618e-07
556 1.15193987124673e-07
557 1.15115225796547e-07
558 1.14837583672145e-07
559 1.14846747968134e-07
560 1.14571434920663e-07
561 1.14413328551333e-07
562 1.14290057815492e-07
563 1.14291693595892e-07
564 1.1407665795149e-07
565 1.13900465056105e-07
566 1.13713187002151e-07
567 1.13532236342362e-07
568 1.13432217936804e-07
569 1.13202125456624e-07
570 1.13029984564861e-07
571 1.12806458180792e-07
572 1.12678423133339e-07
573 1.12670988265506e-07
574 1.12417042705815e-07
575 1.12201884057583e-07
576 1.1212489194623e-07
577 1.12041663253848e-07
578 1.11780671218753e-07
579 1.11568305485754e-07
580 1.1144641916605e-07
581 1.11319070323468e-07
582 1.11191993488191e-07
583 1.1109366740758e-07
584 1.10961587661329e-07
585 1.10781350685762e-07
586 1.1077108177826e-07
587 1.10456322503616e-07
588 1.10368269762517e-07
589 1.10242662012539e-07
590 1.1002701650753e-07
591 1.09943567066395e-07
592 1.09672024450447e-07
593 1.09634569541939e-07
594 1.0942111865031e-07
595 1.09426639375521e-07
596 1.09185579104221e-07
597 1.09109003187946e-07
598 1.08956414348604e-07
599 1.08766505477753e-07
600 1.08620476003907e-07
601 1.08401997255925e-07
602 1.08338271625819e-07
603 1.08245001126939e-07
604 1.08100665461741e-07
605 1.07985800393173e-07
606 1.07822615289521e-07
607 1.07838162046292e-07
608 1.07565039846769e-07
609 1.07365186591046e-07
610 1.07421894533388e-07
611 1.07133104566337e-07
612 1.07088236734221e-07
613 1.07000189663253e-07
614 1.06946179559309e-07
615 1.06845362050301e-07
616 1.06639149829668e-07
617 1.06630261973883e-07
618 1.06369211130719e-07
619 1.06320316731967e-07
620 1.06199533131068e-07
621 1.058496820594e-07
622 1.0582316741381e-07
623 1.05607008592301e-07
624 1.05687747321781e-07
625 1.05627352880333e-07
626 1.05409666333323e-07
627 1.05177166855874e-07
628 1.05126961576829e-07
629 1.05154911782535e-07
630 1.04995140771891e-07
631 1.04593167380429e-07
632 1.04537151251805e-07
633 1.04365969002984e-07
634 1.04633588311032e-07
635 1.04431891563728e-07
636 1.04329726689834e-07
637 1.03817077296675e-07
638 1.04029273080286e-07
639 1.03883319226838e-07
640 1.03659715285254e-07
641 1.03632204269566e-07
642 1.0347754033635e-07
643 1.03432909956958e-07
644 1.03002718482514e-07
645 1.03179449867596e-07
646 1.0293789644944e-07
647 1.02938239052719e-07
648 1.02832593720237e-07
649 1.02667100980725e-07
650 1.02587415693023e-07
651 1.02394810394202e-07
652 1.02467665640482e-07
653 1.02090197457549e-07
654 1.02208137868764e-07
655 1.01928322978395e-07
656 1.01858369703223e-07
657 1.01739339322116e-07
658 1.01766021053606e-07
659 1.01507077264529e-07
660 1.01480419590239e-07
661 1.01365466246506e-07
662 1.01478488642393e-07
663 1.01053717484589e-07
664 1.00996056328739e-07
665 1.00925457119594e-07
666 1.00841014972985e-07
667 1.01013049273568e-07
668 1.00972159445156e-07
669 1.00545947580244e-07
670 1.00663446111149e-07
671 1.00517954888524e-07
672 1.00280442935308e-07
673 1.00339104722735e-07
674 1.00228785192336e-07
675 1.00031515623655e-07
676 1.00027088411281e-07
677 9.9821066743111e-08
678 9.97027345324142e-08
679 9.95527583320666e-08
680 9.94999709149624e-08
681 9.95043563776221e-08
682 9.94042648860827e-08
683 9.91736074364979e-08
684 9.91706083803834e-08
685 9.89917104581295e-08
686 9.89270071789505e-08
687 9.86590956113531e-08
688 9.88158502979175e-08
689 9.86687102697559e-08
690 9.85874294325839e-08
691 9.84947597935104e-08
692 9.80482202850297e-08
693 9.81082491540164e-08
694 9.80756486876899e-08
695 9.79669779122006e-08
696 9.7695889142102e-08
697 9.77400126087602e-08
698 9.76075688470246e-08
699 9.74170613670111e-08
700 9.74518792977719e-08
701 9.72980007283297e-08
702 9.71913457545881e-08
703 9.7082170745999e-08
704 9.69907208538956e-08
705 9.68001195271739e-08
706 9.68656873121532e-08
707 9.67395932720549e-08
708 9.6707195966772e-08
709 9.6494824692428e-08
710 9.62996833893115e-08
711 9.61940648345205e-08
712 9.61291992700453e-08
713 9.60066472659449e-08
714 9.6076507236198e-08
715 9.59144973089465e-08
716 9.5817598426251e-08
717 9.60235825662181e-08
718 9.58339071210901e-08
719 9.57792836704385e-08
720 9.56786730883508e-08
721 9.57049561574053e-08
722 9.55685915737803e-08
723 9.55855500128777e-08
724 9.54792902758683e-08
725 9.53052887586736e-08
726 9.519718598483e-08
727 9.5087806553984e-08
728 9.50329196420796e-08
729 9.49552710034496e-08
730 9.48612345927913e-08
731 9.46767182776753e-08
732 9.4691728571128e-08
733 9.45465531412282e-08
734 9.44030531044859e-08
735 9.44767099193555e-08
736 9.43224152241129e-08
737 9.40917895961846e-08
738 9.41753694760905e-08
739 9.39516255096251e-08
740 9.38889783919095e-08
741 9.38187775876287e-08
742 9.3698230727135e-08
743 9.36288434969512e-08
744 9.34874481668402e-08
745 9.34836371575543e-08
746 9.35212261410356e-08
747 9.34304055943969e-08
748 9.33346779508071e-08
749 9.32723561071214e-08
750 9.31780120065895e-08
751 9.30278716690935e-08
752 9.29624295604725e-08
753 9.29329705181559e-08
754 9.28192649523041e-08
755 9.27499729996839e-08
756 9.26559738809729e-08
757 9.25389097696794e-08
758 9.24657362268988e-08
759 9.24185982213999e-08
760 9.22568115262479e-08
761 9.21515668386164e-08
762 9.21159244464676e-08
763 9.20022454806713e-08
764 9.19474032317069e-08
765 9.18298043304233e-08
766 9.17356848608719e-08
767 9.16950475211564e-08
768 9.1572096961201e-08
769 9.15518643429536e-08
770 9.14264940394815e-08
771 9.13706081169252e-08
772 9.12531428314267e-08
773 9.12019186651847e-08
774 9.10379030614195e-08
775 9.10307679946598e-08
776 9.09358130565252e-08
777 9.08540213107045e-08
778 9.07468777260334e-08
779 9.07100761446955e-08
780 9.05459163238476e-08
781 9.05525267329566e-08
782 9.0461059213176e-08
783 9.03587971157194e-08
784 9.03230065043559e-08
785 9.03285514901597e-08
786 9.01226274496736e-08
787 9.00631505444949e-08
788 9.00727523669431e-08
789 8.98904938138045e-08
790 8.98675477580824e-08
791 8.96780498234051e-08
792 8.9627807966508e-08
793 8.96693063823228e-08
794 8.95617350309408e-08
795 8.95103472569048e-08
796 8.95010753314551e-08
797 8.93973705000661e-08
798 8.93141758977478e-08
799 8.92530253477908e-08
800 8.91564782334875e-08
801 8.90804227231534e-08
802 8.90008694480748e-08
803 8.89088404725236e-08
804 8.87471153512287e-08
805 8.8700519260243e-08
806 8.85524571261342e-08
807 8.83336850865391e-08
808 8.81912971846255e-08
809 8.80269781275977e-08
810 8.80591373944029e-08
811 8.78580359491288e-08
812 8.77974516639846e-08
813 8.77884599015388e-08
814 8.75556714645676e-08
815 8.75048296302694e-08
816 8.74006725943843e-08
817 8.73969709509481e-08
818 8.7267815632508e-08
819 8.71455074271665e-08
820 8.70839791926592e-08
821 8.70104569079544e-08
822 8.6909445168537e-08
823 8.6838467531436e-08
824 8.67497630969716e-08
825 8.6635362151366e-08
826 8.6536745074639e-08
827 8.63864295217454e-08
828 8.63494735754422e-08
829 8.63777932953447e-08
830 8.62112493913436e-08
831 8.60564367650696e-08
832 8.60005593947832e-08
833 8.59721020702864e-08
834 8.5872028559919e-08
835 8.57904467501669e-08
836 8.56934462678538e-08
837 8.56173097094626e-08
838 8.56384114831244e-08
839 8.56181937689549e-08
840 8.54273871166988e-08
841 8.53150348998355e-08
842 8.52226006293755e-08
843 8.51822445566697e-08
844 8.51332967308771e-08
845 8.50246671020294e-08
846 8.49421188999777e-08
847 8.48637727108859e-08
848 8.47561364403227e-08
849 8.47911395691625e-08
850 8.46884001197701e-08
851 8.45744160522521e-08
852 8.45191546638979e-08
853 8.44257204088983e-08
854 8.43355434683168e-08
855 8.42472466908717e-08
856 8.4179818669039e-08
857 8.40996491189294e-08
858 8.40115662330554e-08
859 8.39586472340947e-08
860 8.39118784132786e-08
861 8.37864074583194e-08
862 8.37147320056353e-08
863 8.36349355073374e-08
864 8.35844913997796e-08
865 8.35129116993905e-08
866 8.33020824644315e-08
867 8.32021442693787e-08
868 8.31664316707226e-08
869 8.31670102248161e-08
870 8.31106115541935e-08
871 8.29747060500097e-08
872 8.29244927427197e-08
873 8.28836039707959e-08
874 8.28411777109572e-08
875 8.26216974401106e-08
876 8.25222110938384e-08
877 8.2489922426543e-08
878 8.24058434223396e-08
879 8.23493402557496e-08
880 8.22584297939244e-08
881 8.22634361949959e-08
882 8.22391678529932e-08
883 8.21000549873219e-08
884 8.20762357056637e-08
885 8.20079722663181e-08
886 8.18654946468378e-08
887 8.1754687284441e-08
888 8.16654795934291e-08
889 8.15805218734589e-08
890 8.14826116730671e-08
891 8.14471428274288e-08
892 8.13841707008045e-08
893 8.13696238002315e-08
894 8.11725146112607e-08
895 8.11537045533051e-08
896 8.11443693389258e-08
897 8.10655367691204e-08
898 8.10169670857341e-08
899 8.09152031595595e-08
900 8.08616226515291e-08
901 8.07939668443502e-08
902 8.07879037787451e-08
903 8.0508861882489e-08
904 8.05595198647424e-08
905 8.04419422273384e-08
906 8.03604768808697e-08
907 8.031893246363e-08
908 8.02957432877172e-08
909 8.0145958041733e-08
910 8.01179137330799e-08
911 8.00710532793403e-08
912 7.9959131090046e-08
913 7.98904751420082e-08
914 7.9830681671389e-08
915 7.97576344857553e-08
916 7.96899211876756e-08
917 7.96050470790455e-08
918 7.94692431509425e-08
919 7.95173272072347e-08
920 7.94059602067243e-08
921 7.93304298114528e-08
922 7.92476691580113e-08
923 7.92225678973857e-08
924 7.91172064751322e-08
925 7.90336190923391e-08
926 7.90615830452701e-08
927 7.88859522597996e-08
928 7.88190420526469e-08
929 7.87461253128896e-08
930 7.86594789405903e-08
931 7.86428753096757e-08
932 7.85188502274714e-08
933 7.84753542486527e-08
934 7.84179332242729e-08
935 7.83179767362974e-08
936 7.82841796578282e-08
937 7.81979816637524e-08
938 7.81218274261697e-08
939 7.80749920723522e-08
940 7.79750973856075e-08
941 7.79054744910468e-08
942 7.786007244448e-08
943 7.77918249714737e-08
944 7.76965327378143e-08
945 7.77007382559702e-08
946 7.75908274155412e-08
947 7.75783005253494e-08
948 7.74390992064333e-08
949 7.74265347964764e-08
950 7.738464728968e-08
951 7.72121033363327e-08
952 7.7174738815966e-08
953 7.71054145536354e-08
954 7.71054562647144e-08
955 7.70545793367106e-08
956 7.69516728253983e-08
957 7.68473100163369e-08
958 7.67910543042483e-08
959 7.67189845252148e-08
960 7.66572477006733e-08
961 7.6589503073432e-08
962 7.65900868939795e-08
963 7.6447151777348e-08
964 7.64224602525232e-08
965 7.63342384622057e-08
966 7.62755497816237e-08
967 7.62802063549728e-08
968 7.6207270183648e-08
969 7.61248992251495e-08
970 7.60126810441797e-08
971 7.59968390351418e-08
972 7.59105584338116e-08
973 7.58637625239977e-08
974 7.57702422786721e-08
975 7.57076395885292e-08
976 7.56822166168014e-08
977 7.55790528166322e-08
978 7.55663203162449e-08
979 7.54569771452829e-08
980 7.54226025767579e-08
981 7.53525594041626e-08
982 7.5300878535689e-08
983 7.52316039007717e-08
984 7.51967590253422e-08
985 7.51351171799364e-08
986 7.50539641063419e-08
987 7.50298237859326e-08
988 7.49356098999066e-08
989 7.48993368686612e-08
990 7.4852713151774e-08
991 7.481069681603e-08
992 7.47366659070714e-08
993 7.4729546815977e-08
994 7.46709595107475e-08
995 7.45724176853813e-08
996 7.45250302669476e-08
997 7.44649617629989e-08
998 7.43910972547113e-08
999 7.43461227274977e-08
1000 7.43011735488963e-08
1001 7.42270937896805e-08
1002 7.41806760888863e-08
1003 7.41505089991712e-08
1004 7.39578361561577e-08
1005 7.40403963717107e-08
1006 7.39660941730946e-08
1007 7.38728591200832e-08
1008 7.38475541570693e-08
1009 7.38150589105757e-08
1010 7.37478238042044e-08
1011 7.36758206976162e-08
1012 7.36051445953123e-08
1013 7.36035174764282e-08
1014 7.34529367187875e-08
1015 7.36093785547176e-08
1016 7.34726492543913e-08
1017 7.34502854218277e-08
1018 7.34317435289178e-08
1019 7.33857053107911e-08
1020 7.32526547082557e-08
1021 7.32836796664138e-08
1022 7.32031796308874e-08
1023 7.31642384059761e-08
1024 7.31246259344509e-08
1025 7.29939173647054e-08
1026 7.28514399450653e-08
1027 7.29290611753974e-08
1028 7.28868463855115e-08
1029 7.28339745519335e-08
1030 7.2691349103593e-08
1031 7.2665702771868e-08
1032 7.25820170690561e-08
1033 7.26503909214138e-08
1034 7.25638171701348e-08
1035 7.24151115343297e-08
1036 7.24609690885281e-08
1037 7.23828685234018e-08
1038 7.22305575098225e-08
1039 7.23415152559248e-08
1040 7.22755292437149e-08
1041 7.21183847205431e-08
1042 7.2118539750754e-08
1043 7.20047615798691e-08
1044 7.20593657264068e-08
1045 7.20382992431823e-08
1046 7.19810790341668e-08
1047 7.17598934725228e-08
1048 7.17102248752877e-08
1049 7.18261230661099e-08
1050 7.16833981111442e-08
1051 7.16616585179075e-08
1052 7.15727650142384e-08
1053 7.15984927581736e-08
1054 7.13843806805503e-08
1055 7.14019629817209e-08
1056 7.14546056617138e-08
1057 7.139241516807e-08
1058 7.14251792950371e-08
1059 7.12724556277688e-08
1060 7.11637807644383e-08
1061 7.11638591006647e-08
1062 7.10929398308746e-08
1063 7.11124745222946e-08
1064 7.11474521075672e-08
1065 7.10004398534192e-08
1066 7.09540195316105e-08
1067 7.0946372017211e-08
1068 7.09095109572289e-08
1069 7.08105695905736e-08
1070 7.07592909774402e-08
1071 7.06968075521175e-08
1072 7.06422170455578e-08
1073 7.05835670331556e-08
1074 7.05603962207757e-08
1075 7.05664992841637e-08
1076 7.03633882925825e-08
1077 7.04617718865208e-08
1078 7.03788878242406e-08
1079 7.03569815208027e-08
1080 7.02714164939611e-08
1081 7.02771059899465e-08
1082 7.01269178671193e-08
1083 7.00804815962908e-08
1084 7.00705975009708e-08
1085 7.0113495148183e-08
1086 6.99565760562404e-08
1087 6.99128376093761e-08
1088 6.97206464457256e-08
1089 7.00492444696277e-08
1090 6.97898879824521e-08
1091 6.97596279850554e-08
1092 6.97030316334502e-08
1093 6.95568594952789e-08
1094 6.96896482157428e-08
1095 6.96533152093615e-08
1096 6.96018714476665e-08
1097 6.93415438721701e-08
1098 6.94261246729777e-08
1099 6.9411683999876e-08
1100 6.93545094496706e-08
1101 6.92737007288446e-08
1102 6.92807941511386e-08
1103 6.9265259929896e-08
1104 6.92470649237276e-08
1105 6.90106577914129e-08
1106 6.90894921189233e-08
1107 6.92025432769139e-08
1108 6.89520384591358e-08
1109 6.90589295617716e-08
1110 6.89092756909204e-08
1111 6.89497644084103e-08
1112 6.89229021215709e-08
1113 6.86581389086616e-08
1114 6.88446062180326e-08
1115 6.86919227308458e-08
1116 6.85680362315644e-08
1117 6.84963424710805e-08
1118 6.8618432728762e-08
1119 6.83736763280596e-08
1120 6.84453299677479e-08
1121 6.85403345708213e-08
1122 6.85212986639705e-08
1123 6.82943847589712e-08
1124 6.83792078730594e-08
1125 6.8463196318147e-08
1126 6.81532508206573e-08
1127 6.82153599504964e-08
1128 6.81068739336599e-08
1129 6.79805995766891e-08
1130 6.80408396529497e-08
1131 6.79831315537616e-08
1132 6.79248776163099e-08
1133 6.78405965166551e-08
1134 6.78594931606824e-08
1135 6.79554815670169e-08
1136 6.78942086671519e-08
1137 6.77442200165412e-08
1138 6.78065891521751e-08
1139 6.7711460662867e-08
1140 6.76124015361879e-08
1141 6.75676484658272e-08
1142 6.75973321082601e-08
1143 6.75834183863078e-08
1144 6.7484189505862e-08
1145 6.75027145045082e-08
1146 6.74636089992298e-08
1147 6.7308630904428e-08
1148 6.72394150635913e-08
1149 6.73436542422135e-08
1150 6.72807057069402e-08
1151 6.71731189143543e-08
1152 6.70366016763957e-08
1153 6.72000870540756e-08
1154 6.71135663024014e-08
1155 6.6967160430309e-08
1156 6.69617655644217e-08
1157 6.69030750615196e-08
1158 6.693093677268e-08
1159 6.68346773595729e-08
1160 6.67325956202891e-08
1161 6.66822676318457e-08
1162 6.68586716217057e-08
1163 6.6700530209074e-08
1164 6.65990173320807e-08
1165 6.65538460280501e-08
1166 6.66988891815379e-08
1167 6.63238014171341e-08
1168 6.63902377695535e-08
1169 6.63452517861707e-08
1170 6.63528232525135e-08
1171 6.62145075456255e-08
1172 6.62817998415832e-08
1173 6.60734775645988e-08
1174 6.61652625644216e-08
1175 6.60548495339253e-08
1176 6.60964534855424e-08
1177 6.60649043227046e-08
1178 6.5935210951773e-08
1179 6.59030277165851e-08
1180 6.6008169794296e-08
1181 6.58869965859399e-08
1182 6.57807045933989e-08
1183 6.58309211034602e-08
1184 6.56551506021952e-08
1185 6.56803366680947e-08
1186 6.55498246022734e-08
1187 6.55833262173111e-08
1188 6.56112815269339e-08
1189 6.55178517694477e-08
1190 6.54351079574234e-08
1191 6.54502536741042e-08
1192 6.54235976480511e-08
1193 6.54310504581979e-08
1194 6.53417851721461e-08
1195 6.52146223023564e-08
1196 6.52363873958617e-08
1197 6.5208121567295e-08
1198 6.5142295234244e-08
1199 6.5149204803383e-08
1200 6.50658437664209e-08
1201 6.503627643073e-08
1202 6.48895158525953e-08
1203 6.49485683246187e-08
1204 6.49016892955157e-08
1205 6.48258171269678e-08
1206 6.48252435606622e-08
1207 6.47334536492128e-08
1208 6.48235230442662e-08
1209 6.47103579816122e-08
1210 6.46169400975172e-08
1211 6.46040009772975e-08
1212 6.45964009331479e-08
1213 6.45626040411962e-08
1214 6.45085224824093e-08
1215 6.45539654440252e-08
1216 6.43867899696726e-08
1217 6.44361893649137e-08
1218 6.43542022342736e-08
1219 6.42912131747497e-08
1220 6.43482885978042e-08
1221 6.42083618609401e-08
1222 6.42678588067547e-08
1223 6.42589527874993e-08
1224 6.41194037138693e-08
1225 6.40963892812874e-08
1226 6.40592999705536e-08
1227 6.40705531436137e-08
1228 6.40328316314776e-08
1229 6.39417344137083e-08
1230 6.38602423501666e-08
1231 6.39045878805256e-08
1232 6.39255125831628e-08
1233 6.37725675372014e-08
1234 6.38228183402756e-08
1235 6.38113169169063e-08
1236 6.38635947178834e-08
1237 6.37996462142798e-08
1238 6.36743047841559e-08
1239 6.36894528178722e-08
1240 6.35809069549076e-08
1241 6.36978822363687e-08
1242 6.3474162486532e-08
1243 6.36207797426458e-08
1244 6.34197255526825e-08
1245 6.35193186619976e-08
1246 6.33794059226034e-08
1247 6.34731399318245e-08
1248 6.32032536924676e-08
1249 6.34499540257405e-08
1250 6.32396618087672e-08
1251 6.30699052892592e-08
1252 6.31945658826627e-08
1253 6.30215352761532e-08
1254 6.31556688102108e-08
1255 6.30286300704608e-08
1256 6.30034476392094e-08
1257 6.30715389355885e-08
1258 6.28382373706327e-08
1259 6.28713504802203e-08
1260 6.27701506836775e-08
1261 6.28232565647036e-08
1262 6.27822509462472e-08
1263 6.26673041523862e-08
1264 6.26396749028313e-08
1265 6.26213980257084e-08
1266 6.25454268208259e-08
1267 6.25569462444631e-08
1268 6.23866540123696e-08
1269 6.23526394791796e-08
1270 6.23752088053653e-08
1271 6.23522890235151e-08
1272 6.21200705182012e-08
1273 6.20883466853073e-08
1274 6.21294587057619e-08
1275 6.21233222255491e-08
1276 6.19859529453759e-08
1277 6.19365143819106e-08
1278 6.20169667948645e-08
1279 6.18077522578364e-08
1280 6.17902566035777e-08
1281 6.19055042141437e-08
1282 6.17424855711146e-08
1283 6.2034086537599e-08
1284 6.18363656683663e-08
1285 6.1697697092189e-08
1286 6.16570836002861e-08
1287 6.16246283859923e-08
1288 6.15599191386984e-08
1289 6.14336023987416e-08
1290 6.13666585780326e-08
1291 6.15037913567207e-08
1292 6.11852460496998e-08
1293 6.12455610664586e-08
1294 6.12064740506124e-08
1295 6.1052026177455e-08
1296 6.11019151313119e-08
1297 6.10293322418354e-08
1298 6.09308206589443e-08
1299 6.08138975357608e-08
1300 6.08387981184677e-08
1301 6.07879587419813e-08
1302 6.06693343911946e-08
1303 6.06567909711142e-08
1304 6.06426142544247e-08
1305 6.05653324707855e-08
1306 6.06082767080629e-08
1307 6.04393770291534e-08
1308 6.04896939431221e-08
1309 6.04056073594261e-08
1310 6.03689558027476e-08
1311 6.02497931745383e-08
1312 6.01671016906558e-08
1313 6.01090776557456e-08
1314 6.00250433762106e-08
1315 5.99432534613698e-08
1316 5.99434990899983e-08
1317 5.98659015582648e-08
1318 5.98802419584654e-08
1319 5.96550726572609e-08
1320 5.97129929935658e-08
1321 5.96045570997461e-08
1322 5.95505827949427e-08
1323 5.94509088067685e-08
1324 5.94053684896956e-08
1325 5.94670735809544e-08
1326 5.94028804039581e-08
1327 5.93484664896948e-08
1328 5.93746339450085e-08
1329 5.91540045800087e-08
1330 5.9209312960995e-08
1331 5.91441661410563e-08
1332 5.90712135615235e-08
1333 5.90431644029721e-08
1334 5.90230494381672e-08
1335 5.90510764564289e-08
1336 5.89811857500955e-08
1337 5.8914619266881e-08
1338 5.87326521559461e-08
1339 5.87874601971627e-08
1340 5.86959035904222e-08
1341 5.87712029482823e-08
1342 5.86549009948811e-08
1343 5.86055514255612e-08
1344 5.86176810024597e-08
1345 5.85600564972921e-08
1346 5.83989852969147e-08
1347 5.83184119582469e-08
1348 5.84289383578351e-08
1349 5.83555418320891e-08
1350 5.82471184782385e-08
1351 5.82195696612331e-08
1352 5.82573177920764e-08
1353 5.82421433681368e-08
1354 5.8094640336126e-08
1355 5.80573212358715e-08
1356 5.79755381986402e-08
1357 5.79935405742305e-08
1358 5.78301512756951e-08
1359 5.77900016884847e-08
1360 5.77503180858585e-08
1361 5.77386568458849e-08
1362 5.7715083824128e-08
1363 5.75903290069846e-08
1364 5.75839758121255e-08
1365 5.75773649162947e-08
1366 5.75100263837669e-08
1367 5.75181999389329e-08
1368 5.7399833553351e-08
1369 5.73604857412313e-08
1370 5.7282062932984e-08
1371 5.73346582974921e-08
1372 5.72301664427677e-08
1373 5.72098079909544e-08
1374 5.71307052856795e-08
1375 5.70882139525608e-08
1376 5.71394090140842e-08
1377 5.70364707228599e-08
1378 5.69832751726729e-08
1379 5.6949576501264e-08
1380 5.68304424697352e-08
1381 5.68713582693725e-08
1382 5.67677745988426e-08
1383 5.67926377526007e-08
1384 5.67729692666497e-08
1385 5.66700584463398e-08
1386 5.66311604219827e-08
1387 5.65383715556145e-08
1388 5.65314592382293e-08
1389 5.66098069518794e-08
1390 5.65256096205502e-08
1391 5.64862662741916e-08
1392 5.63976128167276e-08
1393 5.63276584912842e-08
1394 5.62703158073496e-08
1395 5.62374739545568e-08
1396 5.62794404008748e-08
1397 5.61624602655186e-08
1398 5.61800732628015e-08
1399 5.61649515744556e-08
1400 5.60704522114541e-08
1401 5.60843107861864e-08
1402 5.58994626649323e-08
1403 5.59434523452751e-08
1404 5.57970480694614e-08
1405 5.58966059691635e-08
1406 5.58720905865506e-08
1407 5.58338563458882e-08
1408 5.57844052950784e-08
1409 5.5700719733931e-08
1410 5.56335157839705e-08
1411 5.5668150586996e-08
1412 5.56463192529222e-08
1413 5.56828688913491e-08
1414 5.5602212571948e-08
1415 5.54994365051797e-08
1416 5.54698707548873e-08
1417 5.54372097725242e-08
1418 5.53900301922816e-08
1419 5.54059340864832e-08
1420 5.52727298215316e-08
1421 5.53452016021083e-08
1422 5.52793637420557e-08
1423 5.58608437262187e-08
1424 5.50065232052255e-08
1425 5.56680901322437e-08
1426 5.49380684735645e-08
1427 5.49962683105143e-08
1428 5.49877102118401e-08
1429 5.4953411149139e-08
1430 5.48556143575851e-08
1431 5.46677974786469e-08
1432 5.46737908062589e-08
1433 5.46321611754053e-08
1434 5.55443651295562e-08
1435 5.44890779492491e-08
1436 5.45200547206193e-08
1437 5.45390671962132e-08
1438 5.53605753412789e-08
1439 5.43416237026584e-08
1440 5.43902806398044e-08
1441 5.4335318519616e-08
1442 5.43245012374349e-08
1443 5.52091532348697e-08
1444 5.41781801546293e-08
1445 5.39627669255971e-08
1446 5.41382372452048e-08
1447 5.40851293115985e-08
1448 5.50999170343403e-08
1449 5.38587863148088e-08
1450 5.39440119617929e-08
1451 5.39007665465441e-08
1452 5.48335629493302e-08
1453 5.37595202318997e-08
1454 5.38281321995626e-08
1455 5.3778604537591e-08
1456 5.37574049910283e-08
1457 5.37004689449283e-08
1458 5.36915441124997e-08
1459 5.36478275403596e-08
1460 5.35478222740693e-08
1461 5.36172357419318e-08
1462 5.34386383153063e-08
1463 5.34407425440975e-08
1464 5.35264831258431e-08
1465 5.34112386731955e-08
1466 5.34819873574399e-08
1467 5.32715527339356e-08
1468 5.32867589710939e-08
1469 5.33015138755033e-08
1470 5.32579592005256e-08
1471 5.32196177163158e-08
1472 5.41950698655924e-08
1473 5.29254024943526e-08
1474 5.3194384522115e-08
1475 5.28867211775363e-08
1476 5.31012753217652e-08
1477 5.30268851330895e-08
1478 5.2949727295637e-08
1479 5.29493935710335e-08
1480 5.38228194444379e-08
1481 5.28410364202259e-08
1482 5.26985625555199e-08
1483 5.27674542705192e-08
1484 5.2762462102951e-08
1485 5.27264249408077e-08
1486 5.34397033229439e-08
1487 5.33777879190289e-08
1488 5.23944572772272e-08
1489 5.25963112654182e-08
1490 5.23724450558483e-08
1491 5.25009944505328e-08
1492 5.32762206679749e-08
1493 5.23547275057545e-08
1494 5.23309045314946e-08
1495 5.22561196190097e-08
1496 5.22878268212068e-08
1497 5.31502215341817e-08
1498 5.21724308599758e-08
1499 5.20520254805312e-08
1500 5.21257380525775e-08
1501 5.20818777731691e-08
1502 5.21016675070207e-08
1503 5.19742763502862e-08
1504 5.19903458271465e-08
1505 5.28737236735743e-08
1506 5.19539048626783e-08
1507 5.18270253935249e-08
1508 5.17494341942815e-08
1509 5.29073470763031e-08
1510 5.16187757506614e-08
1511 5.16982107150454e-08
1512 5.16267932360037e-08
1513 5.16153288150356e-08
1514 5.16290080336823e-08
1515 5.16791185964927e-08
1516 5.15384441008848e-08
1517 5.15312787858768e-08
1518 5.23390242208599e-08
1519 5.19592389758294e-08
1520 5.20337976270824e-08
1521 5.14038561760799e-08
1522 5.134364354209e-08
1523 5.12048398291221e-08
1524 5.12320874155314e-08
1525 5.11865637768949e-08
1526 5.12145263003028e-08
1527 5.12125970866251e-08
1528 5.17728659956696e-08
1529 5.16683190450173e-08
1530 5.10411777083775e-08
1531 5.16449285412612e-08
1532 5.16043047058545e-08
1533 5.0990909938875e-08
1534 5.16289535013037e-08
1535 5.14396469455392e-08
1536 5.07736270489723e-08
1537 5.15391878521676e-08
1538 5.07790329464797e-08
1539 5.06556261377078e-08
1540 5.13769595369418e-08
1541 5.05484742436835e-08
1542 5.14047425288489e-08
1543 5.05684131222317e-08
1544 5.11543066612497e-08
1545 5.08415655895433e-08
1546 5.11046939632998e-08
1547 5.05061048206024e-08
1548 5.08837524888506e-08
1549 5.12417258891062e-08
1550 5.02762175393912e-08
1551 5.0794699915091e-08
1552 5.06651042888606e-08
1553 5.10844087382445e-08
1554 5.01566998845071e-08
1555 5.06716255199713e-08
1556 5.04253640420949e-08
1557 5.06134860209784e-08
1558 5.0533651423379e-08
1559 5.05044981258251e-08
1560 5.04616558871795e-08
1561 5.03569211391941e-08
1562 5.03762540771113e-08
1563 5.03342929700779e-08
1564 5.06158088082742e-08
1565 5.02683812424465e-08
1566 5.02043239762884e-08
1567 5.01706918076827e-08
1568 5.01457514103798e-08
1569 5.01115362396831e-08
1570 5.00755003103315e-08
1571 5.00318587239867e-08
1572 5.00161078744732e-08
1573 4.99631263499722e-08
1574 4.99739924642029e-08
1575 5.004149594523e-08
1576 4.98911132078561e-08
1577 4.9855505620755e-08
1578 4.97995332593426e-08
1579 4.97460980897202e-08
1580 4.97305124507186e-08
1581 4.96993159702086e-08
1582 4.96697623861841e-08
1583 4.96408844559149e-08
1584 4.95977481862475e-08
1585 4.96668893248398e-08
1586 4.96186348755145e-08
1587 4.96679561607927e-08
1588 4.94469038274659e-08
1589 4.9508331778636e-08
1590 4.94810147126934e-08
1591 4.95191821414664e-08
1592 4.95041214128022e-08
1593 4.9470550635089e-08
1594 4.9198319116428e-08
1595 4.91571029574978e-08
1596 4.91541040408272e-08
1597 4.91158494404509e-08
1598 4.91379556510907e-08
1599 4.90288925552029e-08
1600 4.88214725224978e-08
1601 4.91758261489394e-08
1602 4.89393655644932e-08
1603 4.88874396658012e-08
1604 4.89221190451694e-08
1605 4.86702364375091e-08
1606 4.89828626144551e-08
1607 4.89377012264924e-08
1608 4.88711789250473e-08
1609 4.90344168242451e-08
1610 4.86136223383227e-08
1611 4.86476419920834e-08
1612 4.86414786089462e-08
1613 4.83624934086713e-08
1614 4.87191591473923e-08
1615 4.86545449653875e-08
1616 4.86134385626613e-08
1617 4.85535664709325e-08
1618 4.84904297395339e-08
1619 4.85108331411688e-08
1620 4.86505154091432e-08
1621 4.84719542881429e-08
1622 4.84510720513143e-08
1623 4.83607160037813e-08
1624 4.84924674535314e-08
1625 4.82475383241088e-08
1626 4.83415788501418e-08
1627 4.82652058342303e-08
1628 4.82231077767459e-08
1629 4.81576796080851e-08
1630 4.81280944311102e-08
1631 4.81316428913026e-08
1632 4.80706595280544e-08
1633 4.8014243942518e-08
1634 4.79631789267465e-08
1635 4.79682728253827e-08
1636 4.78868969833179e-08
1637 4.78799164258525e-08
1638 4.7875501349548e-08
1639 4.78515579147576e-08
1640 4.7741317938943e-08
1641 4.77634743432098e-08
1642 4.76854465261134e-08
1643 4.76719843649498e-08
1644 4.76438154146663e-08
1645 4.76122928407818e-08
1646 4.75365713392684e-08
1647 4.74722511909498e-08
1648 4.74760247892725e-08
1649 4.74335425377781e-08
1650 4.74457534078709e-08
1651 4.73615282281514e-08
1652 4.7302322318199e-08
1653 4.73026032437041e-08
1654 4.72761505263541e-08
1655 4.72012673351507e-08
1656 4.71702837097077e-08
1657 4.71602742977417e-08
1658 4.70847480067427e-08
1659 4.70854466869675e-08
1660 4.70680319639705e-08
1661 4.69694532712062e-08
1662 4.70704827852941e-08
1663 4.6935969405304e-08
1664 4.68931984842769e-08
1665 4.68407003313409e-08
1666 4.68132390847309e-08
1667 4.68603520733524e-08
1668 4.67818783800311e-08
1669 4.67471670124731e-08
1670 4.67208151824217e-08
1671 4.66211769984959e-08
1672 4.65725810321338e-08
1673 4.65707836556639e-08
1674 4.64978158989382e-08
1675 4.66064502413488e-08
1676 4.64903837515962e-08
1677 4.646031055211e-08
1678 4.64047998076644e-08
1679 4.63609589247405e-08
1680 4.63114752000671e-08
1681 4.62290743472593e-08
1682 4.62784372148306e-08
1683 4.62101741716126e-08
1684 4.63387395868153e-08
1685 4.6105132328389e-08
1686 4.61249871661629e-08
1687 4.60403387831931e-08
1688 4.60459118638568e-08
1689 4.60104884516532e-08
1690 4.59062408437738e-08
1691 4.59617401062928e-08
1692 4.58688467497304e-08
1693 4.57484605114189e-08
1694 4.57739473258556e-08
1695 4.58195020525132e-08
1696 4.57339603854301e-08
1697 4.56452549624231e-08
1698 4.55560837462343e-08
1699 4.55345729557521e-08
1700 4.54907226821177e-08
1701 4.54552821498311e-08
1702 4.54854339073663e-08
1703 4.5309266969884e-08
1704 4.53663402364413e-08
1705 4.53268480078428e-08
1706 4.5271537757241e-08
1707 4.53194358347453e-08
1708 4.52001485937714e-08
1709 4.49656771501772e-08
1710 4.51221844293137e-08
1711 4.50786902264078e-08
1712 4.50820006925312e-08
1713 4.51744173188828e-08
1714 4.49953082095611e-08
1715 4.50339304753911e-08
1716 4.48852542098699e-08
1717 4.47148809739062e-08
1718 4.484139848282e-08
1719 4.48372354648896e-08
1720 4.4593412767302e-08
1721 4.47152889830882e-08
1722 4.470271279855e-08
1723 4.44772484029521e-08
1724 4.45896091814113e-08
1725 4.46128938138557e-08
1726 4.43288910805251e-08
1727 4.45092448915219e-08
1728 4.45587767008426e-08
1729 4.42114359788803e-08
1730 4.43857996343944e-08
1731 4.42175007591139e-08
1732 4.43198873156803e-08
1733 4.43160459870917e-08
1734 4.41087908558124e-08
1735 4.42114121144144e-08
1736 4.42114282459549e-08
1737 4.39646281491512e-08
1738 4.41271766056239e-08
1739 4.41053074213826e-08
1740 4.38918300638136e-08
1741 4.40184853025549e-08
1742 4.38322659843671e-08
1743 4.40117752664726e-08
1744 4.39641748557484e-08
1745 4.36961448220252e-08
1746 4.39327743131379e-08
1747 4.38679700085487e-08
1748 4.38335038168702e-08
1749 4.35257657738219e-08
1750 4.37588040700376e-08
1751 4.36770811926834e-08
1752 4.36580516769425e-08
1753 4.34630559134064e-08
1754 4.35797193447662e-08
1755 4.35751068250667e-08
1756 4.33383575648705e-08
1757 4.32926921054566e-08
1758 4.34171986389842e-08
1759 4.34715928681051e-08
1760 4.31937474738575e-08
1761 4.31346886173856e-08
1762 4.31102381277526e-08
1763 4.31611334685833e-08
1764 4.30832246531843e-08
1765 4.30229588226361e-08
1766 4.29914141308885e-08
1767 4.31595478360691e-08
1768 4.29186729942188e-08
1769 4.28872370656919e-08
1770 4.28703530834795e-08
1771 4.29472220426774e-08
1772 4.28613716922932e-08
1773 4.27922874268738e-08
1774 4.27434352689993e-08
1775 4.28277787909437e-08
1776 4.27414838664042e-08
1777 4.28219206050073e-08
1778 4.26209375925524e-08
1779 4.25439587812271e-08
1780 4.2581038256273e-08
1781 4.27589287639307e-08
1782 4.24200570616762e-08
1783 4.24283918074586e-08
1784 4.24523390725184e-08
1785 4.24469322122256e-08
1786 4.237799283624e-08
1787 4.23890269236615e-08
1788 4.20511714476213e-08
1789 4.22462634861453e-08
1790 4.24036075354905e-08
1791 4.2396193231653e-08
1792 4.23174502546253e-08
1793 4.22981643697362e-08
1794 4.20998607983947e-08
1795 4.22871871847974e-08
1796 4.21888224368949e-08
1797 4.2030116071512e-08
1798 4.2132890683888e-08
1799 4.20416257065348e-08
1800 4.1994450431293e-08
1801 4.20020168192536e-08
1802 4.19924224677182e-08
1803 4.19967090041773e-08
1804 4.19544641210301e-08
1805 4.1879611197615e-08
1806 4.19031116289936e-08
1807 4.17681520752389e-08
1808 4.17727890815733e-08
1809 4.18646965645308e-08
1810 4.16421366975506e-08
1811 4.16004963317285e-08
1812 4.22059843172562e-08
1813 4.19020975601647e-08
1814 4.15764892593984e-08
1815 4.1486230410559e-08
1816 4.20074758578082e-08
1817 4.15044909818363e-08
1818 4.17396780370005e-08
1819 4.13964593262151e-08
1820 4.14202142695608e-08
1821 4.15559112565411e-08
1822 4.12933712321539e-08
1823 4.16198730452955e-08
1824 4.12458073419941e-08
1825 4.11860902205241e-08
1826 4.11836209099903e-08
1827 4.14986049372956e-08
1828 4.11673313953642e-08
1829 4.13477947791563e-08
1830 4.10693045762578e-08
1831 4.13305707525069e-08
1832 4.10021964336593e-08
1833 4.12961208735574e-08
1834 4.13382507775673e-08
1835 4.11930989629994e-08
1836 4.12205502846597e-08
1837 4.12543487215977e-08
1838 4.10979640190767e-08
1839 4.11400520343719e-08
1840 4.11307752012924e-08
1841 4.10984647083445e-08
1842 4.11155017814302e-08
1843 4.0928636751314e-08
1844 4.10878964856032e-08
1845 4.09982085325389e-08
1846 4.10542385202994e-08
1847 4.08798399957888e-08
1848 4.09632234315005e-08
1849 4.0960219213737e-08
1850 4.08276972292398e-08
1851 4.08438760799434e-08
1852 4.0834204678486e-08
1853 4.07731794545008e-08
1854 4.07487700200271e-08
1855 4.07271356062111e-08
1856 4.07031049109996e-08
1857 4.06991327648321e-08
1858 4.02436294795194e-08
1859 4.08422614381898e-08
1860 4.06212327148658e-08
1861 4.05444478324135e-08
1862 4.05133266991875e-08
1863 4.05296054957205e-08
1864 4.04251136347789e-08
1865 4.03364296954756e-08
1866 4.03730273870195e-08
1867 4.03147201848419e-08
1868 4.01298125676242e-08
1869 4.02079695196456e-08
1870 4.0200436534743e-08
1871 4.031377545477e-08
1872 4.03059737696942e-08
1873 3.97912478558027e-08
1874 4.00851223663068e-08
1875 4.02592817465575e-08
1876 3.97880366040937e-08
1877 3.99164561133425e-08
1878 4.00298525575771e-08
1879 4.01224596635608e-08
1880 3.95693227597249e-08
1881 3.97718833458072e-08
1882 3.98665757765926e-08
1883 3.99656347065402e-08
1884 3.94973886419514e-08
1885 3.97556775619812e-08
1886 3.97950630253163e-08
1887 3.94356658319772e-08
1888 3.96597871783477e-08
1889 3.97756706718155e-08
1890 3.98498992537633e-08
1891 3.96813739000912e-08
1892 3.93667483340465e-08
1893 3.94053095993385e-08
1894 3.97046880351581e-08
1895 3.92630611685263e-08
1896 3.9351154556222e-08
1897 3.91184238970332e-08
1898 3.93185085645342e-08
1899 3.94432853934035e-08
1900 3.91298829076625e-08
1901 3.92736387793136e-08
1902 3.89969104523935e-08
1903 3.92379157481138e-08
1904 3.940498883237e-08
1905 3.92304850826974e-08
1906 3.88871742122987e-08
1907 3.90779986818579e-08
1908 3.89258792306002e-08
1909 3.9197177895911e-08
1910 3.90775523642084e-08
1911 3.87195710724875e-08
1912 3.860753247098e-08
1913 3.9032133433814e-08
1914 3.87496831733003e-08
1915 3.91464211708659e-08
1916 3.88613546449257e-08
1917 3.88190315026016e-08
1918 3.85917411840353e-08
1919 3.88119680430066e-08
1920 3.85567130800624e-08
1921 3.86948424773159e-08
1922 3.85437137242484e-08
1923 3.86686439832751e-08
1924 3.84885208193353e-08
1925 3.85377112830376e-08
1926 3.85514012570809e-08
1927 3.83842766953357e-08
1928 3.84999805409514e-08
1929 3.82613729703607e-08
1930 3.83587074415104e-08
1931 3.84568532991736e-08
1932 3.81890716041333e-08
1933 3.83974050612856e-08
1934 3.81564609153173e-08
1935 3.8464893982848e-08
1936 3.84124107717376e-08
1937 3.822378795304e-08
1938 3.81349822764498e-08
1939 3.81537683047384e-08
1940 3.80480919890402e-08
1941 3.81127485749033e-08
1942 3.80251562845935e-08
1943 3.80692888914957e-08
1944 3.81332554582059e-08
1945 3.80694627648559e-08
1946 3.797680349793e-08
1947 3.78209455598189e-08
1948 3.80053365209854e-08
1949 3.79103785945745e-08
1950 3.78569811960716e-08
1951 3.77965821698467e-08
1952 3.78415061277337e-08
1953 3.79434541519785e-08
1954 3.75656447637596e-08
1955 3.78158379348292e-08
1956 3.75548735540576e-08
1957 3.73299903242419e-08
1958 3.75080810335326e-08
1959 3.7579084413597e-08
1960 3.74161736864664e-08
1961 3.74092821990857e-08
1962 3.73602369241333e-08
1963 3.74520507309128e-08
1964 3.73309944583511e-08
1965 3.7230935983068e-08
1966 3.73543194576165e-08
1967 3.73536711260058e-08
1968 3.73142840333074e-08
1969 3.72313086254295e-08
1970 3.72272458530887e-08
1971 3.72139039450659e-08
1972 3.69996537514794e-08
1973 3.71250478834462e-08
1974 3.70089724359701e-08
1975 3.70314365696345e-08
1976 3.70423673783371e-08
1977 3.67066811199912e-08
1978 3.70214932368107e-08
1979 3.68918402884333e-08
1980 3.67648855577052e-08
1981 3.6797833875557e-08
1982 3.69153554520274e-08
1983 3.67867054054116e-08
1984 3.66840611469854e-08
1985 3.66455030387236e-08
1986 3.67121352287114e-08
1987 3.66382385994513e-08
1988 3.6583155474279e-08
1989 3.66123663106954e-08
1990 3.65383543963205e-08
1991 3.66244089624601e-08
1992 3.65112085338737e-08
1993 3.65701390312445e-08
1994 3.64813148925336e-08
1995 3.64026337726919e-08
1996 3.63486011161207e-08
1997 3.64882124670451e-08
1998 3.64385194493444e-08
1999 3.62810838874239e-08
2000 3.6413163921889e-08
2001 3.63240699816636e-08
2002 3.61509250721426e-08
2003 3.62398078554982e-08
2004 3.6356993362574e-08
2005 3.60313769274256e-08
2006 3.6116813176168e-08
2007 3.59193238868016e-08
2008 3.59886349730054e-08
2009 3.60494001929723e-08
2010 3.57598130347192e-08
2011 3.59919460652947e-08
2012 3.58113562692886e-08
2013 3.59295682086369e-08
2014 3.58131049820187e-08
2015 3.58007826863904e-08
2016 3.57837239084446e-08
2017 3.56016501972256e-08
2018 3.58211246629381e-08
2019 3.55975198624847e-08
2020 3.54945878102697e-08
2021 3.54896341856481e-08
2022 3.56199790148004e-08
2023 3.54522949086444e-08
2024 3.54327876612714e-08
2025 3.54149614345545e-08
2026 3.55962086961981e-08
2027 3.53811339506294e-08
2028 3.53620078024086e-08
2029 3.53342337229279e-08
2030 3.5459588783926e-08
2031 3.52794878661911e-08
2032 3.52709393034445e-08
2033 3.52420326263925e-08
2034 3.51640069031767e-08
2035 3.52012844264671e-08
2036 3.52043623426113e-08
2037 3.51355246372975e-08
2038 3.51366712711965e-08
2039 3.50832336151186e-08
2040 3.52666978433724e-08
2041 3.50705063434376e-08
2042 3.50337685333102e-08
2043 3.49735072679991e-08
2044 3.5457436219577e-08
2045 3.51759981152e-08
2046 3.50585557136185e-08
2047 3.51371271620771e-08
2048 3.51937499196708e-08
2049 3.51273634437987e-08
2050 3.49897461293658e-08
2051 3.4991228961001e-08
2052 3.4950167963288e-08
2053 3.50216840856454e-08
2054 3.4889763122159e-08
2055 3.49107130408477e-08
2056 3.51693636275741e-08
2057 3.47509746179142e-08
2058 3.48671745138418e-08
2059 3.48428465359873e-08
2060 3.47583523367945e-08
2061 3.47029798903886e-08
2062 3.49971758377698e-08
2063 3.46797697465817e-08
2064 3.46916166464872e-08
2065 3.45275453708105e-08
2066 3.46239628465916e-08
2067 3.47523397894456e-08
2068 3.46002796249856e-08
2069 3.46659411167849e-08
2070 3.46381963547415e-08
2071 3.46021424157961e-08
2072 3.44254758712914e-08
2073 3.45763388669429e-08
2074 3.45508080290546e-08
2075 3.44897629442897e-08
2076 3.44769258433519e-08
2077 3.44424870779569e-08
2078 3.43670808222729e-08
2079 3.42575945371237e-08
2080 3.44870092878757e-08
2081 3.43114188550153e-08
2082 3.44427295968508e-08
2083 3.42837921452066e-08
2084 3.43078319411383e-08
2085 3.43755144567304e-08
2086 3.43766872969997e-08
2087 3.40985675317818e-08
2088 3.42263226240647e-08
2089 3.4148764433084e-08
2090 3.41347573742379e-08
2091 3.40950394814854e-08
2092 3.43015465582575e-08
2093 3.3996479381404e-08
2094 3.42440723186677e-08
2095 3.3871950419595e-08
2096 3.39247352010297e-08
2097 3.42200479881427e-08
2098 3.39075039685888e-08
2099 3.38780409956385e-08
2100 3.39329185614634e-08
2101 3.41149876796898e-08
2102 3.36924908923564e-08
2103 3.38929731340531e-08
2104 3.4023903292546e-08
2105 3.36713940782829e-08
2106 3.3965681874637e-08
2107 3.36106830320926e-08
2108 3.39314085655573e-08
2109 3.3633130922528e-08
2110 3.40042220834391e-08
2111 3.34897993590211e-08
2112 3.38712889269566e-08
2113 3.33927516762422e-08
2114 3.36119790960154e-08
2115 3.37675917569946e-08
2116 3.35260883721489e-08
2117 3.34937862871421e-08
2118 3.35731116942739e-08
2119 3.36656699988502e-08
2120 3.36141738213058e-08
2121 3.34974515268094e-08
2122 3.36283379662738e-08
2123 3.33282090330123e-08
2124 3.36539770953159e-08
2125 3.33599669506412e-08
2126 3.34232924918965e-08
2127 3.35010667229518e-08
2128 3.32196758670023e-08
2129 3.35005119640464e-08
2130 3.32270099949739e-08
2131 3.34661737406705e-08
2132 3.30040321701031e-08
2133 3.34270039665796e-08
2134 3.32723088727338e-08
2135 3.3218136962665e-08
2136 3.31160886282156e-08
2137 3.33421102087783e-08
2138 3.32155125395417e-08
2139 3.28345218405524e-08
2140 3.31659583876487e-08
2141 3.31464833851491e-08
2142 3.309384653849e-08
2143 3.30349601540547e-08
2144 3.27826673889309e-08
2145 3.30163808501815e-08
2146 3.30359714242334e-08
2147 3.28973562826462e-08
2148 3.29322646832608e-08
2149 3.28651995415985e-08
2150 3.24577260077952e-08
2151 3.29355025754374e-08
2152 3.29081793379871e-08
2153 3.31306454526281e-08
2154 3.29857774201692e-08
2155 3.30673627864098e-08
2156 3.2932205017655e-08
2157 3.30296026185284e-08
2158 3.27360353236106e-08
2159 3.29821073945524e-08
2160 3.28144974381939e-08
2161 3.23884837092336e-08
2162 3.25699728276341e-08
2163 3.28868882826239e-08
2164 3.27998488311998e-08
2165 3.29213108938298e-08
2166 3.25919729959878e-08
2167 3.27440121095091e-08
2168 3.26646335770953e-08
2169 3.26984212424275e-08
2170 3.23186314625801e-08
2171 3.27611404797867e-08
2172 3.24166678213889e-08
2173 3.26094638904983e-08
2174 3.27276850440317e-08
2175 3.24076339952573e-08
2176 3.25483237508806e-08
2177 3.2536808228345e-08
2178 3.25005549721702e-08
2179 3.23192131510552e-08
2180 3.2254837280421e-08
2181 3.25021259528491e-08
2182 3.23297607500717e-08
2183 3.23886049762301e-08
2184 3.20722664473028e-08
2185 3.23724597555142e-08
2186 3.22678428383849e-08
2187 3.21661710465104e-08
2188 3.21112742216467e-08
2189 3.22625108859498e-08
2190 3.22490097635075e-08
2191 3.19351947835056e-08
2192 3.20209782982417e-08
2193 3.21837305463646e-08
2194 3.21950949129235e-08
2195 3.21112284826786e-08
2196 3.18694044791901e-08
2197 3.19050818577971e-08
2198 3.18853274470232e-08
2199 3.2115809386779e-08
2200 3.19582190917433e-08
2201 3.17691529665431e-08
2202 3.17253637063253e-08
2203 3.17484069909391e-08
2204 3.1905419571876e-08
2205 3.17254757935537e-08
2206 3.19439278209099e-08
2207 3.17028070568881e-08
2208 3.17106617147012e-08
2209 3.20906427266898e-08
2210 3.17220334276058e-08
2211 3.18349615231472e-08
2212 3.18444469362689e-08
2213 3.15033174207446e-08
2214 3.20155159208024e-08
2215 3.18167369646183e-08
2216 3.15500147243242e-08
2217 3.18485476622499e-08
2218 3.15775525703277e-08
2219 3.19338432863692e-08
2220 3.14051419634254e-08
2221 3.19330119507022e-08
2222 3.14554243370235e-08
2223 3.17460773218148e-08
2224 3.18045273293155e-08
2225 3.13928229582761e-08
2226 3.16535894333558e-08
2227 3.16351853384411e-08
2228 3.17007393166868e-08
2229 3.23513729827418e-08
2230 3.17232405362411e-08
2231 3.16243534652649e-08
2232 3.16627541971926e-08
2233 3.14669909302623e-08
2234 3.12977985423846e-08
2235 3.14192460173501e-08
2236 3.13206627116092e-08
2237 3.15379551163453e-08
2238 3.12872843264422e-08
2239 3.11998226651955e-08
2240 3.12905851669321e-08
2241 3.1857403302693e-08
2242 3.12102452815211e-08
2243 3.12579395511658e-08
2244 3.10733516291073e-08
2245 3.1228873901501e-08
2246 3.10029057668526e-08
2247 3.14252705893026e-08
2248 3.13551995543282e-08
2249 3.11593808830501e-08
2250 3.15022922765529e-08
2251 3.08370502763999e-08
2252 3.12813532097334e-08
2253 3.08489621796415e-08
2254 3.14638936897982e-08
2255 3.0759933216773e-08
2256 3.13832713678686e-08
2257 3.06899919131354e-08
2258 3.1188226523593e-08
2259 3.0924674033761e-08
2260 3.11468043667773e-08
2261 3.0395291775509e-08
2262 3.12740838266379e-08
2263 3.07554480352135e-08
2264 3.09635712794076e-08
2265 3.12803194821853e-08
2266 3.06138077825757e-08
2267 3.08375489030954e-08
2268 3.11369378191984e-08
2269 3.04227304142035e-08
2270 3.0934683829642e-08
2271 3.06661762585758e-08
2272 3.08678255285244e-08
2273 3.0377320633157e-08
2274 3.0945275905081e-08
2275 3.06795972684348e-08
2276 3.06328698849168e-08
2277 3.06245630201651e-08
2278 3.0494991757779e-08
2279 3.04522897289772e-08
2280 3.05398675422008e-08
2281 3.09123711925707e-08
2282 3.06736243553818e-08
2283 3.04544909850435e-08
2284 3.04410221005913e-08
2285 3.10226697464167e-08
2286 3.03009898319218e-08
2287 3.0482176683666e-08
2288 3.06156718321571e-08
2289 3.08505799875292e-08
2290 3.05777179252509e-08
2291 3.05401474141043e-08
2292 3.07230801563918e-08
2293 3.07303998328123e-08
2294 3.0610532564701e-08
2295 3.05274524341481e-08
2296 3.03596123865102e-08
2297 3.06591289875868e-08
2298 3.0604599958961e-08
2299 3.08706175198292e-08
2300 3.09604884010106e-08
2301 3.10487929042758e-08
2302 3.0807440960734e-08
2303 3.08791906917794e-08
2304 3.06571056216676e-08
2305 3.07180194645262e-08
2306 3.05318458502235e-08
2307 3.04473784367154e-08
2308 3.05227695489574e-08
2309 3.03913002945855e-08
2310 3.04100893400161e-08
2311 3.03785204978091e-08
2312 3.03243381041263e-08
2313 3.03229449705178e-08
2314 3.03446455056644e-08
2315 3.02706110315576e-08
2316 3.02058567189789e-08
2317 3.00786314291202e-08
2318 3.01683133707709e-08
2319 3.03571895012844e-08
2320 3.00569900597569e-08
2321 3.01667675803863e-08
2322 3.01450562172345e-08
2323 3.02356181769436e-08
2324 3.00738143774737e-08
2325 3.02595886476631e-08
2326 3.00173324196962e-08
2327 2.98686788138269e-08
2328 3.00250092868382e-08
2329 3.00167538980212e-08
2330 2.98252617274564e-08
2331 2.99638970060467e-08
2332 2.97711724868144e-08
2333 2.98572227046545e-08
2334 2.9612594925732e-08
2335 2.97084853375651e-08
2336 2.9840809687709e-08
2337 2.9440202789166e-08
2338 2.96868932914141e-08
2339 2.9577195022723e-08
2340 2.94112094272325e-08
2341 2.96827471615302e-08
2342 2.93904147057322e-08
2343 2.97999130765003e-08
2344 3.0067314840565e-08
2345 2.95478043341735e-08
2346 2.96743743257721e-08
2347 2.99829176653166e-08
2348 3.00288792260961e-08
2349 2.94001491252249e-08
2350 2.97775997897265e-08
2351 2.95630843190065e-08
2352 2.97761872352087e-08
2353 2.95359331392575e-08
2354 2.98233178972485e-08
2355 2.92941863926366e-08
2356 2.94477104885171e-08
2357 2.93712257062317e-08
2358 2.89037029521655e-08
2359 2.93040397074673e-08
2360 2.92261232752722e-08
2361 2.94491996162272e-08
2362 2.92629523215293e-08
2363 2.96271481847743e-08
2364 2.95511033485685e-08
2365 2.92983735230568e-08
2366 2.93655462253462e-08
2367 2.92212903516553e-08
2368 2.95073164009452e-08
2369 2.92469281011343e-08
2370 2.92781764781314e-08
2371 2.95854871683598e-08
2372 2.93389192471683e-08
2373 2.91889776724563e-08
2374 2.91915277712551e-08
2375 2.90849864954534e-08
2376 2.93435922535945e-08
2377 2.89674596800626e-08
2378 2.87109808139174e-08
2379 2.88396909926547e-08
2380 2.88140590594121e-08
2381 2.8932750445243e-08
2382 2.8672327647028e-08
2383 2.88611239882552e-08
2384 2.87625871058239e-08
2385 2.87344044112015e-08
2386 2.86092139532013e-08
2387 2.82731462433183e-08
2388 2.84745006620124e-08
2389 2.84599026494359e-08
2390 2.83654674848588e-08
2391 2.85005522788762e-08
2392 2.82650452587685e-08
2393 2.84302706770045e-08
2394 2.83916708556653e-08
2395 2.84206758143224e-08
2396 2.82653879273287e-08
2397 2.86200814791915e-08
2398 2.79756449212964e-08
2399 2.83298191769976e-08
2400 2.84018445388945e-08
2401 2.81262990435849e-08
2402 2.82322812628077e-08
2403 2.81712190135508e-08
2404 2.83687439652791e-08
2405 2.80449171046193e-08
2406 2.80466383839606e-08
2407 2.8080739312264e-08
2408 2.80894177171476e-08
2409 2.81162059831974e-08
2410 2.7639779934141e-08
2411 2.7896001330463e-08
2412 2.76470895030911e-08
2413 2.7891319349882e-08
2414 2.78807225768674e-08
2415 2.75690553508401e-08
2416 2.78291824171717e-08
2417 2.77809523048234e-08
2418 2.74708811585089e-08
2419 2.79764561841311e-08
2420 2.76168482222783e-08
2421 2.73372512846759e-08
2422 2.74185101969415e-08
2423 2.77552611822607e-08
2424 2.71354274548052e-08
2425 2.7038381243516e-08
2426 2.7460726895745e-08
2427 2.74339617547525e-08
2428 2.74044265649032e-08
2429 2.73747461456697e-08
2430 2.72546961037623e-08
2431 2.74703244811381e-08
2432 2.73057461970261e-08
2433 2.722013761014e-08
2434 2.72014637081597e-08
2435 2.72047009211018e-08
2436 2.70942872893087e-08
2437 2.71587199038859e-08
2438 2.70851950172624e-08
2439 2.73545642390793e-08
2440 2.72362933064763e-08
2441 2.71533130715706e-08
2442 2.71446201582926e-08
2443 2.72013729782916e-08
2444 2.6856563746902e-08
2445 2.68636108911036e-08
2446 2.66855646728903e-08
2447 2.70875486678701e-08
2448 2.66138722000431e-08
2449 2.71488242962192e-08
2450 2.67972511980297e-08
2451 2.69251151843175e-08
2452 2.68302272445897e-08
2453 2.684285534027e-08
2454 2.68935802583137e-08
2455 2.67380309140197e-08
2456 2.66033114653119e-08
2457 2.70602423433353e-08
2458 2.64947411050542e-08
2459 2.69540791044687e-08
2460 2.68605577942171e-08
2461 2.66260391066364e-08
2462 2.64320170832999e-08
2463 2.69424418513076e-08
2464 2.63776349636036e-08
2465 2.67705230991844e-08
2466 2.65704375383979e-08
2467 2.64805778498189e-08
2468 2.63371661015643e-08
2469 2.64778497038876e-08
2470 2.66425277333227e-08
2471 2.63982346435654e-08
2472 2.6368936900667e-08
2473 2.63977352352729e-08
2474 2.62014059337901e-08
2475 2.67280655543711e-08
2476 2.61799536949159e-08
2477 2.63543819174483e-08
2478 2.6315925298448e-08
2479 2.62109579540759e-08
2480 2.6404012276382e-08
2481 2.6352784755046e-08
2482 2.59335988699316e-08
2483 2.61952822584455e-08
2484 2.61702666000119e-08
2485 2.60782937882009e-08
2486 2.62086788018756e-08
2487 2.59578073904532e-08
2488 2.60694062075295e-08
2489 2.58088497799847e-08
2490 2.5962422953052e-08
2491 2.60505682900813e-08
2492 2.56627147676713e-08
2493 2.57567992820729e-08
2494 2.60062394328298e-08
2495 2.58323260908799e-08
2496 2.60437643391231e-08
2497 2.61326758539404e-08
2498 2.53394382956351e-08
2499 2.61729268731248e-08
2500 2.59859562401488e-08
2501 2.57512531169901e-08
2502 2.59318095678918e-08
2503 2.55637663288955e-08
2504 2.59846324497381e-08
2505 2.58162046162802e-08
2506 2.59376087856644e-08
2507 2.54970700814816e-08
2508 2.58560014247955e-08
2509 2.56543949894716e-08
2510 2.56755626335536e-08
2511 2.57862876864579e-08
2512 2.54186070027806e-08
2513 2.57106572834065e-08
2514 2.54979691498569e-08
2515 2.6734462509781e-08
2516 2.59976562932973e-08
2517 2.58419687662048e-08
2518 2.58982349909953e-08
2519 2.57865449546646e-08
2520 2.5809843849478e-08
2521 2.5783806911317e-08
2522 2.57038866446901e-08
2523 2.54911985622108e-08
2524 2.55771760933055e-08
2525 2.56842061667761e-08
2526 2.55259637143013e-08
2527 2.53713241091802e-08
2528 2.5496558799798e-08
2529 2.55326455254057e-08
2530 2.53348976588796e-08
2531 2.53001487620086e-08
2532 2.52480030478175e-08
2533 2.53086449346629e-08
2534 2.5221428141764e-08
2535 2.51105947997043e-08
2536 2.50746044152539e-08
2537 2.51079448672087e-08
2538 2.51303002241077e-08
2539 2.51318071962103e-08
2540 2.49492377193405e-08
2541 2.49412222679268e-08
2542 2.49797202125368e-08
2543 2.47814357372533e-08
2544 2.48016916466476e-08
2545 2.49006879011748e-08
2546 2.47650398730315e-08
2547 2.4840029022366e-08
2548 2.46920252084948e-08
2549 2.45265741245504e-08
2550 2.46533254644188e-08
2551 2.46128376175658e-08
2552 2.45581436708608e-08
2553 2.43393138539272e-08
2554 2.43580680070465e-08
2555 2.4374771611102e-08
2556 2.42778300614255e-08
2557 2.44367897037634e-08
2558 2.42794183309414e-08
2559 2.41174014514733e-08
2560 2.4255246942495e-08
2561 2.42162095169807e-08
2562 2.4001970368337e-08
2563 2.40748040132299e-08
2564 2.38432140113876e-08
2565 2.38830960279657e-08
2566 2.39156853631961e-08
2567 2.40051071696001e-08
2568 2.38741336351378e-08
2569 2.41336719344343e-08
2570 2.38438951889464e-08
2571 2.3688980337e-08
2572 2.35715241932999e-08
2573 2.36680887535368e-08
2574 2.36411835597972e-08
2575 2.36169629985739e-08
2576 2.38963902143841e-08
2577 2.37196368719772e-08
2578 2.3749775719395e-08
2579 2.3541075587441e-08
2580 2.34832740449997e-08
2581 2.35385039299985e-08
2582 2.3423684830326e-08
2583 2.33857334080989e-08
2584 2.34218178678525e-08
2585 2.32924690832803e-08
2586 2.32051345550044e-08
2587 2.31408544436595e-08
2588 2.33801471786199e-08
2589 2.33546006818131e-08
2590 2.32362134944264e-08
2591 2.30391337345814e-08
2592 2.29919728884642e-08
2593 2.32143623990488e-08
2594 2.30415194606426e-08
2595 2.30324039971475e-08
2596 2.2996895645111e-08
2597 2.30859423772101e-08
2598 2.3090536291126e-08
2599 2.28821682968228e-08
2600 2.29089673543514e-08
2601 2.27712165804528e-08
2602 2.31754521466954e-08
2603 2.27647341184412e-08
2604 2.30274005736941e-08
2605 2.27841868587486e-08
2606 2.28847023372447e-08
2607 2.26721765796967e-08
2608 2.26952221198395e-08
2609 2.3249262842473e-08
2610 2.32103088237956e-08
2611 2.26676630730172e-08
2612 2.28844300003139e-08
2613 2.26067021067422e-08
2614 2.3065580187831e-08
2615 2.25419148971451e-08
2616 2.31400878845101e-08
2617 2.22178609818879e-08
2618 2.28409652074379e-08
2619 2.23842773658944e-08
2620 2.23628934007403e-08
2621 2.25324142870686e-08
2622 2.23960684393454e-08
2623 2.26288252043183e-08
2624 2.22876855846543e-08
2625 2.25082797209453e-08
2626 2.22429398877289e-08
2627 2.22247868530978e-08
2628 2.22537845722126e-08
2629 2.20453976058632e-08
2630 2.22750687519646e-08
2631 2.226725768395e-08
2632 2.19759938933084e-08
2633 2.21629132530765e-08
2634 2.18284958626302e-08
2635 2.25653579395413e-08
2636 2.18259448869773e-08
2637 2.20949279736704e-08
2638 2.20786486411217e-08
2639 2.21398857582766e-08
2640 2.16725648876093e-08
2641 2.18187260787595e-08
2642 2.21146541266037e-08
2643 2.18327709766708e-08
2644 2.1785268515373e-08
2645 2.17029070679331e-08
2646 2.19508227885523e-08
2647 2.20372122057277e-08
2648 2.18000849545597e-08
2649 2.20274858691027e-08
2650 2.15869861024398e-08
2651 2.18066856749743e-08
2652 2.194546407841e-08
2653 2.17792833760999e-08
2654 2.1806988765416e-08
2655 2.18031988574729e-08
2656 2.17678197316218e-08
2657 2.18446330286781e-08
2658 2.15885171364061e-08
2659 2.17332278584781e-08
2660 2.17504077051345e-08
2661 2.16615531647424e-08
2662 2.17817617411686e-08
2663 2.15548577149338e-08
2664 2.16023833174983e-08
2665 2.13889920979149e-08
2666 2.15208432381431e-08
2667 2.14509863609713e-08
2668 2.14189066956028e-08
2669 2.16560692547496e-08
2670 2.12651001256781e-08
2671 2.16920238491625e-08
2672 2.11771872704603e-08
2673 2.13549157199733e-08
2674 2.13648426390378e-08
2675 2.14095664383684e-08
2676 2.16051667822903e-08
2677 2.14518497472227e-08
2678 2.133088208911e-08
2679 2.15199504138841e-08
2680 2.11230121790473e-08
2681 2.13062419820442e-08
2682 2.11725342411651e-08
2683 2.11300315524632e-08
2684 2.13606556265233e-08
2685 2.11371184017928e-08
2686 2.10031486103102e-08
2687 2.06874135426638e-08
2688 2.15853671308164e-08
2689 2.07884747747133e-08
2690 2.10265144580113e-08
2691 2.11224460766601e-08
2692 2.10058246521427e-08
2693 2.08548400060238e-08
2694 2.09244102078454e-08
2695 2.08356631991036e-08
2696 2.11903811799807e-08
2697 2.08757604382992e-08
2698 2.09324387274723e-08
2699 2.07760425732229e-08
2700 2.09522612979462e-08
2701 2.08738387974527e-08
2702 2.07862274432813e-08
2703 2.0842470685789e-08
2704 2.06469758485817e-08
2705 2.11168308505982e-08
2706 2.05785636662181e-08
2707 2.07996464363092e-08
2708 2.07068920712317e-08
2709 2.04648796318097e-08
2710 2.07753538556954e-08
2711 2.04109722430346e-08
2712 2.06004386249514e-08
2713 2.04474564202695e-08
2714 2.05081607591584e-08
2715 2.06171228298579e-08
2716 2.06684780788002e-08
2717 2.03065584389872e-08
2718 2.0568184923464e-08
2719 2.0417277411422e-08
2720 2.03521169095389e-08
2721 2.0407672952194e-08
2722 2.01563740838129e-08
2723 2.10182648610058e-08
2724 2.03525413888794e-08
2725 2.02423251594031e-08
2726 2.02245509146604e-08
2727 2.03861948857575e-08
2728 2.02707240877231e-08
2729 2.03892629859759e-08
2730 2.03411820323129e-08
2731 1.99921570820383e-08
2732 2.05290009225312e-08
2733 2.07026886889228e-08
2734 2.00951594229171e-08
2735 2.02701758507118e-08
2736 2.06563739080234e-08
2737 1.99835283523075e-08
2738 2.05462115268062e-08
2739 2.02702377820607e-08
2740 2.00150243891972e-08
2741 2.01069811629484e-08
2742 2.03371275879771e-08
2743 2.02934718005476e-08
2744 1.98590161755163e-08
2745 2.03288148858949e-08
2746 2.02248711529407e-08
2747 2.01746792918289e-08
2748 1.99579749939893e-08
2749 2.0116905135259e-08
2750 2.02157144049142e-08
2751 1.96884036138822e-08
2752 1.96784810608808e-08
2753 2.02299996354682e-08
2754 1.99523078940889e-08
2755 1.97758027249417e-08
2756 2.00466449644665e-08
2757 1.96870227437973e-08
2758 1.96683655300234e-08
2759 1.98588728248517e-08
2760 1.99367471007683e-08
2761 1.98343930450928e-08
2762 1.96637002163325e-08
2763 1.98060724037052e-08
2764 1.9757032376555e-08
2765 2.00747799385947e-08
2766 1.96017201739007e-08
2767 2.01030950555126e-08
2768 1.95700601024473e-08
2769 1.9351623715691e-08
2770 1.98176694587993e-08
2771 1.95782774230491e-08
2772 1.95435542829081e-08
2773 1.96317882203978e-08
2774 1.95637230329115e-08
2775 1.96961039580223e-08
2776 1.94146379455251e-08
2777 1.95944965506456e-08
2778 1.9658275700607e-08
2779 1.94892811737901e-08
2780 1.90642859876355e-08
2781 1.97051321633168e-08
2782 1.92753389380851e-08
2783 1.97433402351344e-08
2784 1.9271451759284e-08
2785 1.94796834851463e-08
2786 1.93661833329806e-08
2787 1.94997931333152e-08
2788 1.94359070273542e-08
2789 1.94318495461143e-08
2790 1.94997010560805e-08
2791 1.94359166376668e-08
2792 1.91259410642353e-08
2793 1.92928663935721e-08
2794 1.94861758329168e-08
2795 1.96880793497112e-08
2796 1.91020278956788e-08
2797 1.90724253303909e-08
2798 1.89103791903289e-08
2799 1.94536639999132e-08
2800 1.95331791996711e-08
2801 1.91720652753524e-08
2802 1.9409672998405e-08
2803 1.91631918577961e-08
2804 1.92891519370519e-08
2805 1.92971476087322e-08
2806 1.93022236072427e-08
2807 1.91180620592313e-08
2808 1.95212799971589e-08
2809 1.9427653180859e-08
2810 1.86052106814216e-08
2811 1.90272997555585e-08
2812 1.92826162423998e-08
2813 1.91450080453404e-08
2814 1.93144183038285e-08
2815 1.9035254822608e-08
2816 1.90073321295614e-08
2817 1.92825461124979e-08
2818 1.88047883806775e-08
2819 1.8906337554192e-08
2820 1.92590043364138e-08
2821 1.903533235037e-08
2822 1.89831112666905e-08
2823 1.971063347872e-08
2824 1.85699279124929e-08
2825 1.88629247739325e-08
2826 1.91740650405947e-08
2827 1.90201879954888e-08
2828 1.89436289772527e-08
2829 1.89993370802721e-08
2830 1.88711616744275e-08
2831 1.89797809142522e-08
2832 1.90108940545386e-08
2833 1.90330303864172e-08
2834 1.87019767818875e-08
2835 1.89332421918209e-08
2836 1.8782662907002e-08
2837 1.89558339451779e-08
2838 1.92528054758778e-08
2839 1.84100350424377e-08
2840 1.90416376364411e-08
2841 1.92091135100547e-08
2842 1.88412747486044e-08
2843 1.87496504298057e-08
2844 1.88310255280211e-08
2845 1.88440721953231e-08
2846 1.87621896390322e-08
2847 1.88905361246938e-08
2848 1.85641030292505e-08
2849 1.87936667421784e-08
2850 1.87460493017433e-08
2851 1.87457324318796e-08
2852 1.85097948983692e-08
2853 1.87568957905615e-08
2854 1.86018845969027e-08
2855 1.85533397396576e-08
2856 1.86963073649515e-08
2857 1.86339718419859e-08
2858 1.85641377514756e-08
2859 1.86894850222608e-08
2860 1.84963503004898e-08
2861 1.84234747584444e-08
2862 1.84515163130783e-08
2863 1.85633486839976e-08
2864 1.85830888648209e-08
2865 1.84263571472165e-08
2866 1.85363191183008e-08
2867 1.83901993144886e-08
2868 1.84773119331627e-08
2869 1.85344651397745e-08
2870 1.84138402625766e-08
2871 1.84872063702102e-08
2872 1.84451550226949e-08
2873 1.83807767173416e-08
2874 1.8255105732834e-08
2875 1.83923090373295e-08
2876 1.85103941379161e-08
2877 1.8364775054236e-08
2878 1.81200314812724e-08
2879 1.81388031428309e-08
2880 1.88060430301729e-08
2881 1.82734131739526e-08
2882 1.82898266201104e-08
2883 1.82592212227828e-08
2884 1.80751178389116e-08
2885 1.85769061133989e-08
2886 1.81559534959685e-08
2887 1.80941684524427e-08
2888 1.8024061728239e-08
2889 1.8336542571662e-08
2890 1.82969443025538e-08
2891 1.811015756914e-08
2892 1.79760565255371e-08
2893 1.80385529868765e-08
2894 1.80602276829145e-08
2895 1.83735877070657e-08
2896 1.80966844212094e-08
2897 1.80054083955827e-08
2898 1.80232892543764e-08
2899 1.8175149354871e-08
2900 1.80619987870667e-08
2901 1.81123726854526e-08
2902 1.78957159158166e-08
2903 1.80209187123914e-08
2904 1.79515992502211e-08
2905 1.80051119758051e-08
2906 1.78678333950799e-08
2907 1.79231785759626e-08
2908 1.79078872000193e-08
2909 1.77984018254751e-08
2910 1.80556785349673e-08
2911 1.78401892505509e-08
2912 1.78946487323639e-08
2913 1.77430477064089e-08
2914 1.77961817706684e-08
2915 1.80106200833308e-08
2916 1.77494125312982e-08
2917 1.79046771526803e-08
2918 1.78799554835241e-08
2919 1.78875465903783e-08
2920 1.78121219458482e-08
2921 1.78119198099846e-08
2922 1.78618179820234e-08
2923 1.7730525326165e-08
2924 1.79090395131531e-08
2925 1.76572759322813e-08
2926 1.78240029191645e-08
2927 1.76799774778136e-08
2928 1.7801505416859e-08
2929 1.75314332302445e-08
2930 1.77866451416264e-08
2931 1.76441847317133e-08
2932 1.77537671530725e-08
2933 1.76924072083917e-08
2934 1.77922289112953e-08
2935 1.76331635517446e-08
2936 1.77812282164957e-08
2937 1.754459904868e-08
2938 1.7701661669367e-08
2939 1.78019185390621e-08
2940 1.75970719884511e-08
2941 1.75491800873395e-08
2942 1.77130117666735e-08
2943 1.77285693971907e-08
2944 1.80741084829794e-08
2945 1.70950936735359e-08
2946 1.78853249102051e-08
2947 1.7474040596932e-08
2948 1.79767405690257e-08
2949 1.75069532386551e-08
2950 1.79354322469294e-08
2951 1.75196348120021e-08
2952 1.76616111959849e-08
2953 1.76475425233136e-08
2954 1.78176516569817e-08
2955 1.73739031714693e-08
2956 1.7379642339943e-08
2957 1.77169672639366e-08
2958 1.72313705608662e-08
2959 1.76321726506057e-08
2960 1.73234341986905e-08
2961 1.78159296762015e-08
2962 1.76262225923818e-08
2963 1.78684287264197e-08
2964 1.76044132937925e-08
2965 1.71597879661256e-08
2966 1.73188528567181e-08
2967 1.76451884565942e-08
2968 1.73699941010685e-08
2969 1.73494314807954e-08
2970 1.82143977036642e-08
2971 1.68461002538756e-08
2972 1.72377352183339e-08
2973 1.75911249407079e-08
2974 1.81211559082684e-08
2975 1.72946383387718e-08
2976 1.74902650025288e-08
2977 1.73895226875942e-08
2978 1.75155812804917e-08
2979 1.74188945134546e-08
2980 1.76345759743679e-08
2981 1.75214475479635e-08
2982 1.73039358040139e-08
2983 1.71702972628918e-08
2984 1.74827459309412e-08
2985 1.75876961350241e-08
2986 1.74261819045896e-08
2987 1.73397259188857e-08
2988 1.78223604697658e-08
2989 1.69310498105979e-08
2990 1.72737637603726e-08
2991 1.71479391011253e-08
2992 1.75818245655712e-08
2993 1.78041986432786e-08
2994 1.70515273618665e-08
2995 1.71417901015936e-08
2996 1.72823542132772e-08
2997 1.73739464468525e-08
2998 1.77103557205127e-08
2999 1.69443195127128e-08
3000 7.40914938723858e-09
3001 7.47176058102372e-09
3002 7.55817807424064e-09
3003 7.60108902519985e-09
3004 7.61407092286781e-09
3005 7.61564237225509e-09
3006 7.61197762980115e-09
3007 7.60858923665175e-09
3008 7.60367902483128e-09
3009 7.59942432101979e-09
3010 7.59493826385516e-09
3011 7.59158210095201e-09
3012 7.58903926512122e-09
3013 7.58700774315668e-09
3014 7.58477599555052e-09
3015 7.58142646721527e-09
3016 7.57812269207292e-09
3017 7.57535355988548e-09
3018 7.57253245434186e-09
3019 7.56682975330825e-09
3020 7.55112208987441e-09
3021 7.54662350710422e-09
3022 7.54406756424408e-09
3023 7.54174956327236e-09
3024 7.54058469790153e-09
3025 7.53642009962419e-09
3026 7.53417070727125e-09
3027 7.53173747890246e-09
3028 7.53202131940911e-09
3029 7.52822873424652e-09
3030 7.52638198223132e-09
3031 7.52406389778471e-09
3032 7.5216984027765e-09
3033 7.51978134051756e-09
3034 7.51760373156329e-09
3035 7.51728101561211e-09
3036 7.51455089037512e-09
3037 7.51274662257051e-09
3038 7.51106116372857e-09
3039 7.50852450777384e-09
3040 7.50631586095285e-09
3041 7.50455645125159e-09
3042 7.50253208818552e-09
3043 7.50110762659517e-09
3044 7.49922431179939e-09
3045 7.49807456920659e-09
3046 7.49571521285741e-09
3047 7.49435313644742e-09
3048 7.49279511501333e-09
3049 7.49091221952103e-09
3050 7.48963352013854e-09
3051 7.4880026377927e-09
3052 7.4872573274376e-09
3053 7.48691691668246e-09
3054 7.48527670739463e-09
3055 7.48366486191809e-09
3056 7.48210034669239e-09
3057 7.48064571882978e-09
3058 7.4816638148123e-09
3059 7.47698137112951e-09
3060 7.47816541585289e-09
3061 7.47665855452273e-09
3062 7.47502930015242e-09
3063 7.47357175859276e-09
3064 7.47228152753188e-09
3065 7.47079716377497e-09
3066 7.46940238409433e-09
3067 7.4680219063622e-09
3068 7.4639592235054e-09
3069 7.46277045672594e-09
3070 7.46277224285274e-09
3071 7.45962029069513e-09
3072 7.46020835114269e-09
3073 7.45883025747573e-09
3074 7.45788594047614e-09
3075 7.45640362445621e-09
3076 7.45508593265332e-09
3077 7.45379769205579e-09
3078 7.45249968646389e-09
3079 7.45143272774806e-09
3080 7.44948379627353e-09
3081 7.44906068457507e-09
3082 7.44794204328436e-09
3083 7.44660115485629e-09
3084 7.44447164492679e-09
3085 7.4438365188223e-09
3086 7.44223453306991e-09
3087 7.44102591167251e-09
3088 7.43982361607609e-09
3089 7.43855861227993e-09
3090 7.43741992968727e-09
3091 7.43597634494941e-09
3092 7.435021891114e-09
3093 7.43315763465646e-09
3094 7.43205160604055e-09
3095 7.43100818763331e-09
3096 7.42981451277991e-09
3097 7.42862436205194e-09
3098 7.42735855331633e-09
3099 7.4263852676143e-09
3100 7.42528025848843e-09
3101 7.42397710812304e-09
3102 7.42291832184216e-09
3103 7.42161641770211e-09
3104 7.42057417371655e-09
3105 7.41932464824735e-09
3106 7.41757537495169e-09
3107 7.41646220794867e-09
3108 7.41512672346178e-09
3109 7.41357596921455e-09
3110 7.41240783445896e-09
3111 7.4118905518672e-09
3112 7.41064964394911e-09
3113 7.40927022170601e-09
3114 7.40821577867312e-09
3115 7.40708271171342e-09
3116 7.40610521456075e-09
3117 7.40486423875253e-09
3118 7.40379010820413e-09
3119 7.40280331787413e-09
3120 7.40147263326996e-09
3121 7.40035452395649e-09
3122 7.39926517545619e-09
3123 7.39813156527824e-09
3124 7.3968549633846e-09
3125 7.39587495816874e-09
3126 7.39495496128451e-09
3127 7.39509962136009e-09
3128 7.39217312081819e-09
3129 7.39325022404425e-09
3130 7.38961139425121e-09
3131 7.39087738667321e-09
3132 7.3872371026823e-09
3133 7.3884365781457e-09
3134 7.38486662074656e-09
3135 7.38596120837542e-09
3136 7.38231531892397e-09
3137 7.3831968861876e-09
3138 7.37995295070626e-09
3139 7.38087249353703e-09
3140 7.37770299966134e-09
3141 7.37858793294077e-09
3142 7.37535016050173e-09
3143 7.37631501494806e-09
3144 7.37300087924608e-09
3145 7.37399352165879e-09
3146 7.37073443497771e-09
3147 7.37054741907495e-09
3148 7.36903943461542e-09
3149 7.36847438835564e-09
3150 7.36664744963034e-09
3151 7.3661368690664e-09
3152 7.36431853019803e-09
3153 7.36379604805715e-09
3154 7.3620363746224e-09
3155 7.36211865959291e-09
3156 7.35984755194374e-09
3157 7.35872737692156e-09
3158 7.35807843287029e-09
3159 7.35806620845436e-09
3160 7.35697362831689e-09
3161 7.35555821276823e-09
3162 7.35489623765462e-09
3163 7.35406091441349e-09
3164 7.35264248465095e-09
3165 7.35148534718177e-09
3166 7.34921739441174e-09
3167 7.34952239003772e-09
3168 7.3455391161048e-09
3169 7.34437628416296e-09
3170 7.34389999307306e-09
3171 7.34029477707232e-09
3172 7.34118835779562e-09
3173 7.34080393600955e-09
3174 7.33874806825574e-09
3175 7.33707896745373e-09
3176 7.33602597384475e-09
3177 7.33494526157508e-09
3178 7.33386898790767e-09
3179 7.33370370400999e-09
3180 7.33161890788259e-09
3181 7.33006336706143e-09
3182 7.3302311656559e-09
3183 7.32811187133253e-09
3184 7.3264028725567e-09
3185 7.32555340289542e-09
3186 7.32447464844854e-09
3187 7.3232803133455e-09
3188 7.3227459626013e-09
3189 7.32088278948551e-09
3190 7.31956342049644e-09
3191 7.31805293795185e-09
3192 7.31783526088103e-09
3193 7.31586025357056e-09
3194 7.31488594950258e-09
3195 7.31360251637891e-09
3196 7.31310328157264e-09
3197 7.31162705631949e-09
3198 7.31039137551182e-09
3199 7.30944247888388e-09
3200 7.30852579484964e-09
3201 7.30774930444988e-09
3202 7.3062211591729e-09
3203 7.30462779451047e-09
3204 7.30427688894197e-09
3205 7.30290331944772e-09
3206 7.30156943802573e-09
3207 7.30095570068701e-09
3208 7.29935595203401e-09
3209 7.29875929138257e-09
3210 7.29697267590201e-09
3211 7.29693352070904e-09
3212 7.29499560564451e-09
3213 7.29499607100836e-09
3214 7.29282055821068e-09
3215 7.2928856652138e-09
3216 7.29087094222813e-09
3217 7.29024170968706e-09
3218 7.28891431715784e-09
3219 7.28811649659822e-09
3220 7.28674977099053e-09
3221 7.28633930245637e-09
3222 7.28500196296722e-09
3223 7.28418681182619e-09
3224 7.28290202996196e-09
3225 7.2845257214349e-09
3226 7.28093143904929e-09
3227 7.28002345640189e-09
3228 7.27929058748356e-09
3229 7.2779856808286e-09
3230 7.27739653706705e-09
3231 7.27605280391452e-09
3232 7.27519363431794e-09
3233 7.27359015782747e-09
3234 7.27323519643386e-09
3235 7.27168401773448e-09
3236 7.271333614986e-09
3237 7.269376565186e-09
3238 7.26968739891976e-09
3239 7.26720691052174e-09
3240 7.26794288041788e-09
3241 7.26724169923532e-09
3242 7.26617808911867e-09
3243 7.26571854108016e-09
3244 7.26478943915976e-09
3245 7.26404930073699e-09
3246 7.26323273704532e-09
3247 7.26180671374588e-09
3248 7.26131714068323e-09
3249 7.26034443167267e-09
3250 7.2591678515338e-09
3251 7.25845086453625e-09
3252 7.25747013453904e-09
3253 7.25588857443393e-09
3254 7.25556206886135e-09
3255 7.25458220192377e-09
3256 7.2530831925488e-09
3257 7.25275342715326e-09
3258 7.2511804969505e-09
3259 7.25058392933575e-09
3260 7.24939426828552e-09
3261 7.24871036421326e-09
3262 7.24764601238437e-09
3263 7.24643738705955e-09
3264 7.24575680617623e-09
3265 7.24457661850642e-09
3266 7.24390484425908e-09
3267 7.24273676426523e-09
3268 7.24209959300504e-09
3269 7.24130163898273e-09
3270 7.24030230482564e-09
3271 7.23969319771367e-09
3272 7.23823070304663e-09
3273 7.23745329586634e-09
3274 7.23581351812219e-09
3275 7.23491294792045e-09
3276 7.23376573794066e-09
3277 7.23282100519029e-09
3278 7.23213403373835e-09
3279 7.23110148827255e-09
3280 7.22977659364798e-09
3281 7.22853078830021e-09
3282 7.22755752828597e-09
3283 7.22638340311388e-09
3284 7.22568361644083e-09
3285 7.22443721910826e-09
3286 7.22372767876511e-09
3287 7.22305454446681e-09
3288 7.2218473314567e-09
3289 7.22114128484663e-09
3290 7.2200911814535e-09
3291 7.21914974415039e-09
3292 7.21740746638388e-09
3293 7.21777247436606e-09
3294 7.21666608941018e-09
3295 7.21552736551723e-09
3296 7.21486194318888e-09
3297 7.21390568246549e-09
3298 7.21301871443869e-09
3299 7.21076370278906e-09
3300 7.21148396996829e-09
3301 7.21054061635085e-09
3302 7.20940351224553e-09
3303 7.208665420394e-09
3304 7.20775316814803e-09
3305 7.20554984279254e-09
3306 7.20625002012532e-09
3307 7.20536194194488e-09
3308 7.20431417809697e-09
3309 7.2036778559631e-09
3310 7.20284343096178e-09
3311 7.20170078981341e-09
3312 7.19983887520481e-09
3313 7.20027923825639e-09
3314 7.19911909641269e-09
3315 7.1984822117982e-09
3316 7.19759926312569e-09
3317 7.19677188969203e-09
3318 7.19462663412163e-09
3319 7.19541227613141e-09
3320 7.19420080226574e-09
3321 7.19309750936425e-09
3322 7.19252245878765e-09
3323 7.19177831796924e-09
3324 7.19038296756458e-09
3325 7.18998428259066e-09
3326 7.18937466132052e-09
3327 7.1884863893229e-09
3328 7.1876888652761e-09
3329 7.18593094685416e-09
3330 7.1860429874393e-09
3331 7.1851942829021e-09
3332 7.18414498615538e-09
3333 7.18355725182196e-09
3334 7.18279002830113e-09
3335 7.18119610917944e-09
3336 7.18123917765878e-09
3337 7.18029849541835e-09
3338 7.17920296042229e-09
3339 7.178639271882e-09
3340 7.17726300444865e-09
3341 7.17728998005096e-09
3342 7.17617777666602e-09
3343 7.17547216266823e-09
3344 7.17454292636921e-09
3345 7.17312579133489e-09
3346 7.17327741507623e-09
3347 7.17241098402432e-09
3348 7.1710956230514e-09
3349 7.17106846954974e-09
3350 7.16917218918089e-09
3351 7.16923169548356e-09
3352 7.16849067089465e-09
3353 7.16775888096044e-09
3354 7.16598115713118e-09
3355 7.16579804486972e-09
3356 7.16537375385073e-09
3357 7.16441607903628e-09
3358 7.16360760052992e-09
3359 7.16183966494033e-09
3360 7.16198365348364e-09
3361 7.16127875688421e-09
3362 7.16017389514045e-09
3363 7.15969450541809e-09
3364 7.15808338565271e-09
3365 7.15840669701651e-09
3366 7.15742923253215e-09
3367 7.15663565518942e-09
3368 7.15578991311583e-09
3369 7.15529412675941e-09
3370 7.15360022754408e-09
3371 7.15402313199165e-09
3372 7.15280635465998e-09
3373 7.15243381534114e-09
3374 7.15165506988413e-09
3375 7.14880787298922e-09
3376 7.15005288140502e-09
3377 7.14923540910684e-09
3378 7.14837631171639e-09
3379 7.14728660086705e-09
3380 7.14681755090152e-09
3381 7.1446418643123e-09
3382 7.14410967644907e-09
3383 7.14385177787158e-09
3384 7.14330498709226e-09
3385 7.14211390281938e-09
3386 7.1419964435826e-09
3387 7.1404032195993e-09
3388 7.14058452451483e-09
3389 7.13882162878265e-09
3390 7.13831035757673e-09
3391 7.13776121225029e-09
3392 7.13762595798417e-09
3393 7.13582472661178e-09
3394 7.13561186048428e-09
3395 7.13548000010056e-09
3396 7.13443357554244e-09
3397 7.13389273368503e-09
3398 7.13238597119248e-09
3399 7.13240000380089e-09
3400 7.13097782509275e-09
3401 7.13045467719275e-09
3402 7.13020046506918e-09
3403 7.12860778510904e-09
3404 7.12860099043311e-09
3405 7.12766544223864e-09
3406 7.12589334314617e-09
3407 7.12580283579456e-09
3408 7.12553368496793e-09
3409 7.12428588650615e-09
3410 7.12402179532667e-09
3411 7.12224286313068e-09
3412 7.12211570449983e-09
3413 7.12199509830813e-09
3414 7.12032066475798e-09
3415 7.12022580658112e-09
3416 7.11852411471037e-09
3417 7.11786809025339e-09
3418 7.11804178317454e-09
3419 7.11659831895139e-09
3420 7.11625368952917e-09
3421 7.11502409445064e-09
3422 7.1145047285115e-09
3423 7.11371261861993e-09
3424 7.11356387053663e-09
3425 7.11177121735795e-09
3426 7.11207490562693e-09
3427 7.11077231209389e-09
3428 7.11062039887578e-09
3429 7.1092685012264e-09
3430 7.10818734013519e-09
3431 7.10843659612603e-09
3432 7.10698128565668e-09
3433 7.10688147818883e-09
3434 7.10553432295424e-09
3435 7.104402070926e-09
3436 7.10419770721948e-09
3437 7.10435986962799e-09
3438 7.10259725135443e-09
3439 7.10245707823132e-09
3440 7.10089148429904e-09
3441 7.0997255845473e-09
3442 7.09943477664177e-09
3443 7.09936840836978e-09
3444 7.09871619750113e-09
3445 7.09733939240065e-09
3446 7.09680945992774e-09
3447 7.09537645299241e-09
3448 7.09519597855146e-09
3449 7.09423317589442e-09
3450 7.0945381395876e-09
3451 7.09272355436941e-09
3452 7.09258442885274e-09
3453 7.09059687915847e-09
3454 7.09103327271865e-09
3455 7.09026696961434e-09
3456 7.09004259052959e-09
3457 7.08890344559843e-09
3458 7.08828997302013e-09
3459 7.08621436208701e-09
3460 7.0867673694075e-09
3461 7.08644155682225e-09
3462 7.08508079866332e-09
3463 7.0850926328081e-09
3464 7.08383554409719e-09
3465 7.0824693478716e-09
3466 7.08250300834878e-09
3467 7.08175296225411e-09
3468 7.08148047477708e-09
3469 7.08048031604347e-09
3470 7.07983362181752e-09
3471 7.07825653206706e-09
3472 7.0785940045337e-09
3473 7.07749080845754e-09
3474 7.07685765480304e-09
3475 7.07678815679846e-09
3476 7.07507877244218e-09
3477 7.07451235157364e-09
3478 7.07400409524772e-09
3479 7.07329895488495e-09
3480 7.07261456185659e-09
3481 7.07225456135496e-09
3482 7.07173080993773e-09
3483 7.07034464382528e-09
3484 7.0699465454932e-09
3485 7.06910831842211e-09
3486 7.06846501245806e-09
3487 7.06857153523477e-09
3488 7.06750341694651e-09
3489 7.06612623743719e-09
3490 7.06615094180074e-09
3491 7.06498818169032e-09
3492 7.06508854926569e-09
3493 7.06396645926144e-09
3494 7.06380910936444e-09
3495 7.06261198731017e-09
3496 7.06087069553274e-09
3497 7.06185280986704e-09
3498 7.06078665595189e-09
3499 7.06105223041709e-09
3500 7.05988922015954e-09
3501 7.05867662093018e-09
3502 7.05861369888761e-09
3503 7.05783990793962e-09
3504 7.05736385542277e-09
3505 7.05655963854912e-09
3506 7.05566331961338e-09
3507 7.05517665010313e-09
3508 7.05453004988532e-09
3509 7.05400092056163e-09
3510 7.05381670192129e-09
3511 7.05268019028649e-09
3512 7.05235375310365e-09
3513 7.05149190992349e-09
3514 7.05110001474052e-09
3515 7.05049078966735e-09
3516 7.04959893983725e-09
3517 7.04903282000569e-09
3518 7.04819470399853e-09
3519 7.04778811373707e-09
3520 7.04643645683956e-09
3521 7.04561306397178e-09
3522 7.04530810466397e-09
3523 7.04484816506368e-09
3524 7.04399729237193e-09
3525 7.0434348873466e-09
3526 7.04271670048939e-09
3527 7.04099336830411e-09
3528 7.04171765530948e-09
3529 7.04031497349156e-09
3530 7.03958315284581e-09
3531 7.03863941074745e-09
3532 7.03743062058226e-09
3533 7.03781358112054e-09
3534 7.03663490615736e-09
3535 7.03585177087851e-09
3536 7.03539877063009e-09
3537 7.03381970378059e-09
3538 7.03393319513579e-09
3539 7.03278552215136e-09
3540 7.03176634084257e-09
3541 7.02942376459847e-09
3542 7.03127468835996e-09
3543 7.02886040460093e-09
3544 7.02870852173354e-09
3545 7.02747580680341e-09
3546 7.02683723134701e-09
3547 7.02598274823774e-09
3548 7.02572269341284e-09
3549 7.02430878773974e-09
3550 7.02387512413472e-09
3551 7.02313717326764e-09
3552 7.02177267189463e-09
3553 7.02181375511479e-09
3554 7.02117961451365e-09
3555 7.02018657941506e-09
3556 7.0188345362987e-09
3557 7.01880025603951e-09
3558 7.01795775413794e-09
3559 7.01724499663214e-09
3560 7.01641424161181e-09
3561 7.01526690546905e-09
3562 7.01490090593282e-09
3563 7.01423824908676e-09
3564 7.013621677876e-09
3565 7.01215311153802e-09
3566 7.01195235325303e-09
3567 7.01144053916192e-09
3568 7.01079839098007e-09
3569 7.00967571649891e-09
3570 7.00867639612246e-09
3571 7.0080863316252e-09
3572 7.00785082687061e-09
3573 7.00712917832413e-09
3574 7.00631617558956e-09
3575 7.00458620556121e-09
3576 7.00564551202321e-09
3577 7.00476785886472e-09
3578 7.00249586288448e-09
3579 7.00278567390689e-09
3580 7.00238081756399e-09
3581 7.00155257621349e-09
3582 7.00082382866507e-09
3583 7.0000302377915e-09
3584 6.99890625652233e-09
3585 6.99800217569824e-09
3586 6.99927844548254e-09
3587 6.99625956086802e-09
3588 6.99658675999693e-09
3589 6.99656633000589e-09
3590 6.99494491260277e-09
3591 6.9945564556001e-09
3592 6.99388389176214e-09
3593 6.99280461957663e-09
3594 6.99219506218596e-09
3595 6.99244133610388e-09
3596 6.99104384191407e-09
3597 6.98945677242435e-09
3598 6.98963678916198e-09
3599 6.9876715488304e-09
3600 6.9874818395127e-09
3601 6.98693604260503e-09
3602 6.98573438813466e-09
3603 6.98510139980624e-09
3604 6.98357569187125e-09
3605 6.98415020145005e-09
3606 6.98389276392042e-09
3607 6.98109456145901e-09
3608 6.98201017917954e-09
3609 6.9802823913917e-09
3610 6.97993341214342e-09
3611 6.97907271794695e-09
3612 6.97913089998947e-09
3613 6.97806891535035e-09
3614 6.9760244883893e-09
3615 6.97689254322564e-09
3616 6.97553862227807e-09
3617 6.97544011998841e-09
3618 6.97474415023436e-09
3619 6.97284210984805e-09
3620 6.97315943394561e-09
3621 6.97228627684343e-09
3622 6.97141002738155e-09
3623 6.97082447623232e-09
3624 6.96922730061833e-09
3625 6.96898553242953e-09
3626 6.96765576575775e-09
3627 6.96764662562466e-09
3628 6.96606837918834e-09
3629 6.96417173763619e-09
3630 6.96558376649115e-09
3631 6.96363641429498e-09
3632 6.96324184400188e-09
3633 6.96219554766064e-09
3634 6.96157345840553e-09
3635 6.96094668169167e-09
3636 6.95958629157167e-09
3637 6.96051391058106e-09
3638 6.95871150246941e-09
3639 6.95833888997299e-09
3640 6.95761794818062e-09
3641 6.95692640124979e-09
3642 6.95430553522558e-09
3643 6.9561497832854e-09
3644 6.9537122169977e-09
3645 6.95426152411138e-09
3646 6.95307596880468e-09
3647 6.95158452979072e-09
3648 6.95153628457734e-09
3649 6.95075695650305e-09
3650 6.95042485296393e-09
3651 6.948266244311e-09
3652 6.94814647242414e-09
3653 6.94766729220086e-09
3654 6.94780977554543e-09
3655 6.94542674455068e-09
3656 6.94554127793345e-09
3657 6.94479097755607e-09
3658 6.94407660237206e-09
3659 6.94282292637571e-09
3660 6.94322898754651e-09
3661 6.94268814387111e-09
3662 6.94124865982526e-09
3663 6.94056083178074e-09
3664 6.93952727133518e-09
3665 6.93933376158529e-09
3666 6.93864493959973e-09
3667 6.93828414402575e-09
3668 6.93728051308351e-09
3669 6.93616868469027e-09
3670 6.93556104885973e-09
3671 6.93536585570642e-09
3672 6.93451386050981e-09
3673 6.93443047293518e-09
3674 6.93321728649499e-09
3675 6.93214782110374e-09
3676 6.93190417354328e-09
3677 6.93119152404831e-09
3678 6.93105807673955e-09
3679 6.93041444417564e-09
3680 6.92858013702502e-09
3681 6.9293683778171e-09
3682 6.92891433186349e-09
3683 6.92793937638603e-09
3684 6.93571129496695e-09
3685 6.9315755134719e-09
3686 6.93047791860968e-09
3687 6.92969887138017e-09
3688 6.92951439999756e-09
3689 6.92824875635212e-09
3690 6.92771398121128e-09
3691 6.92690070083768e-09
3692 6.92637633439852e-09
3693 6.92553071011959e-09
3694 6.92481332469075e-09
3695 6.92410188120329e-09
3696 6.92364654167332e-09
3697 6.92317447084645e-09
3698 6.92184784389926e-09
3699 6.92127120778541e-09
3700 6.92081475399398e-09
3701 6.91988510817532e-09
3702 6.91909579000893e-09
3703 6.9185060264515e-09
3704 6.9181416195857e-09
3705 6.91706080301058e-09
3706 6.91712747824536e-09
3707 6.91568052041402e-09
3708 6.91501772001613e-09
3709 6.91472256141024e-09
3710 6.91378895946448e-09
3711 6.91306031637418e-09
3712 6.91250503137564e-09
3713 6.91192027180154e-09
3714 6.91109521021305e-09
3715 6.91048016754348e-09
3716 6.90971040970267e-09
3717 6.90934581600022e-09
3718 6.9083545019416e-09
3719 6.90750649021132e-09
3720 6.90765714471409e-09
3721 6.9067497979608e-09
3722 6.90603444018778e-09
3723 6.90472081556204e-09
3724 6.90369389222401e-09
3725 6.90331322202031e-09
3726 6.90227168473334e-09
3727 6.90108521286814e-09
3728 6.90112088140571e-09
3729 6.90074396324236e-09
3730 6.89956120351887e-09
3731 6.89881305589168e-09
3732 6.89846539747574e-09
3733 6.89738497482162e-09
3734 6.89737904874255e-09
3735 6.8961149211344e-09
3736 6.89539703691011e-09
3737 6.89482595285507e-09
3738 6.89434160229541e-09
3739 6.89318945537587e-09
3740 6.8926521285817e-09
3741 6.89193444620984e-09
3742 6.89042934239048e-09
3743 6.89074544395207e-09
3744 6.89004162358897e-09
3745 6.88919196893678e-09
3746 6.88883730405598e-09
3747 6.88790823491492e-09
3748 6.88727640131559e-09
3749 6.88549685491646e-09
3750 6.88599841695836e-09
3751 6.88547161557806e-09
3752 6.88460546180436e-09
3753 6.88291762754312e-09
3754 6.88202892520917e-09
3755 6.88160170375651e-09
3756 6.88096382024661e-09
3757 6.88020959692393e-09
3758 6.87965613552222e-09
3759 6.87931549461784e-09
3760 6.87829243289917e-09
3761 6.87724108057453e-09
3762 6.87711568032878e-09
3763 6.87686174739854e-09
3764 6.87575579447208e-09
3765 6.87512081634645e-09
3766 6.87456559683719e-09
3767 6.87460434743714e-09
3768 6.87363809859232e-09
3769 6.87260507210852e-09
3770 6.8720516580717e-09
3771 6.87182862840752e-09
3772 6.87072538557709e-09
3773 6.86934497125258e-09
3774 6.86931211406339e-09
3775 6.869227982223e-09
3776 6.86816057639261e-09
3777 6.8670204084742e-09
3778 6.86705138271126e-09
3779 6.86670237159959e-09
3780 6.86562162156845e-09
3781 6.86515814353472e-09
3782 6.86450724035614e-09
3783 6.86455067649971e-09
3784 6.8637328839577e-09
3785 6.8626696362456e-09
3786 6.86207464437039e-09
3787 6.86164669097544e-09
3788 6.86025595060535e-09
3789 6.8603037381848e-09
3790 6.85967714840474e-09
3791 6.85925374224738e-09
3792 6.85837199322636e-09
3793 6.85782140637248e-09
3794 6.85768344522997e-09
3795 6.8577854388796e-09
3796 6.85577312388441e-09
3797 6.85539727324214e-09
3798 6.8552114082937e-09
3799 6.85447420345486e-09
3800 6.85361459691614e-09
3801 6.85430635155582e-09
3802 6.85253474076719e-09
3803 6.85246457590716e-09
3804 6.85131943291284e-09
3805 6.85092970605095e-09
3806 6.85008776200868e-09
3807 6.84945809457937e-09
3808 6.84885304998162e-09
3809 6.84835214585788e-09
3810 6.84791498303805e-09
3811 6.84702790333669e-09
3812 6.84611122987733e-09
3813 6.84567642586509e-09
3814 6.84509650805687e-09
3815 6.84447877503935e-09
3816 6.84445898371799e-09
3817 6.84363653113973e-09
3818 6.84295402450519e-09
3819 6.84228730880865e-09
3820 6.84157981716005e-09
3821 6.84111832086054e-09
3822 6.84046021111329e-09
3823 6.839763764796e-09
3824 6.83993583223963e-09
3825 6.83883345656278e-09
3826 6.83809602715357e-09
3827 6.83645131763222e-09
3828 6.83715901593496e-09
3829 6.83546649625499e-09
3830 6.83516786355309e-09
3831 6.83435950268874e-09
3832 6.83452187280609e-09
3833 6.83380143549905e-09
3834 6.83275135941741e-09
3835 6.8373529718696e-09
3836 6.83467157143458e-09
3837 6.83375676499565e-09
3838 6.83395939107823e-09
3839 6.8324965139116e-09
3840 6.83327358176611e-09
3841 6.83103115689532e-09
3842 6.83147068097911e-09
3843 6.83035505519292e-09
3844 6.82970217356915e-09
3845 6.82950807930072e-09
3846 6.82847395554187e-09
3847 6.8272893052751e-09
3848 6.82756485481661e-09
3849 6.8266636908676e-09
3850 6.82611238700392e-09
3851 6.82564324647195e-09
3852 6.82479735655828e-09
3853 6.8246128965832e-09
3854 6.82387395321837e-09
3855 6.82306367748298e-09
3856 6.8195242982072e-09
3857 6.82239866997558e-09
3858 6.82149131288334e-09
3859 6.81817804100426e-09
3860 6.82036229794625e-09
3861 6.8167815517467e-09
3862 6.81614430637911e-09
3863 6.81619341826034e-09
3864 6.81454751361166e-09
3865 6.81488236119188e-09
3866 6.81419883157008e-09
3867 6.81714685177304e-09
3868 6.81273499748836e-09
3869 6.81223871389081e-09
3870 6.81134706820297e-09
3871 6.81168277104349e-09
3872 6.81059176323162e-09
3873 6.81084371793872e-09
3874 6.80940780757588e-09
3875 6.80934516081311e-09
3876 6.80784495206421e-09
3877 6.80833419318405e-09
3878 6.80766594200744e-09
3879 6.80711364428599e-09
3880 6.80629280491729e-09
3881 6.80530676352986e-09
3882 6.80475833118865e-09
3883 6.80493502643043e-09
3884 6.8045799938854e-09
3885 6.80299425030728e-09
3886 6.80280045543824e-09
3887 6.80084921848845e-09
3888 6.80109527785577e-09
3889 6.80099899452724e-09
3890 6.80005163512798e-09
3891 6.79986683674305e-09
3892 6.79862684659083e-09
3893 6.79898425728909e-09
3894 6.79760025920628e-09
3895 6.79801899543253e-09
3896 6.79671174612373e-09
3897 6.79596880455435e-09
3898 6.7958848570665e-09
3899 6.7951493458035e-09
3900 6.79438025877332e-09
3901 6.79429714087187e-09
3902 6.79324386081148e-09
3903 6.79211249302036e-09
3904 6.79248164217605e-09
3905 6.79205761601465e-09
3906 6.7907205351242e-09
3907 6.79059094407797e-09
3908 6.78963044412062e-09
3909 6.78801817260988e-09
3910 6.7884258473927e-09
3911 6.78753311782188e-09
3912 6.78709449931925e-09
3913 6.78670613489529e-09
3914 6.78623599074946e-09
3915 6.78525587503864e-09
3916 6.78498444139919e-09
3917 6.78470592763158e-09
3918 6.78392987364662e-09
3919 6.7840143566783e-09
3920 6.78260623959703e-09
3921 6.78269577096546e-09
3922 6.78231418742292e-09
3923 6.78150182227555e-09
3924 6.78080450321195e-09
3925 6.78059898982786e-09
3926 6.77930811973648e-09
3927 6.77935453148593e-09
3928 6.77899145364036e-09
3929 6.7782610465722e-09
3930 6.77742296895101e-09
3931 6.77737743802476e-09
3932 6.77645831387297e-09
3933 6.77548277812356e-09
3934 6.77572132652682e-09
3935 6.77486680562833e-09
3936 6.7742682628158e-09
3937 6.77397616159758e-09
3938 6.77299505046081e-09
3939 6.77169402518718e-09
3940 6.7716846800514e-09
3941 6.77100322822588e-09
3942 6.77026630162281e-09
3943 6.76992459110182e-09
3944 6.76901083181136e-09
3945 6.76841428938479e-09
3946 6.76837846685929e-09
3947 6.76752044917461e-09
3948 6.76700563487231e-09
3949 6.76672087031316e-09
3950 6.76590391289478e-09
3951 6.76502683907842e-09
3952 6.76499975209299e-09
3953 6.76431901586172e-09
3954 6.76356437391556e-09
3955 6.76335601772216e-09
3956 6.7621690344466e-09
3957 6.76198899700331e-09
3958 6.7616508923618e-09
3959 6.76118265158232e-09
3960 6.76016932496526e-09
3961 6.76085846201824e-09
3962 6.75863457812176e-09
3963 6.75897660658287e-09
3964 6.75816238321358e-09
3965 6.75767286462126e-09
3966 6.7575806840664e-09
3967 6.75638560593272e-09
3968 6.755521844648e-09
3969 6.75564172558651e-09
3970 6.75486659972024e-09
3971 6.75427050145228e-09
3972 6.75413713567552e-09
3973 6.75319971399357e-09
3974 6.75258426124925e-09
3975 6.75244910874895e-09
3976 6.7516499326209e-09
3977 6.75100473510049e-09
3978 6.75104146935501e-09
3979 6.74988986811009e-09
3980 6.74926553538568e-09
3981 6.74888522492201e-09
3982 6.74839758554868e-09
3983 6.74784912152449e-09
3984 6.74711752975121e-09
3985 6.74591435660676e-09
3986 6.74689234350656e-09
3987 6.7462379182287e-09
3988 6.74558115358603e-09
3989 6.74575395463728e-09
3990 6.74380142783038e-09
3991 6.74420515457685e-09
3992 6.7435489339851e-09
3993 6.74267190321764e-09
3994 6.74248362926566e-09
3995 6.7417570376177e-09
3996 6.74100737188543e-09
3997 6.74082286065969e-09
3998 6.7403800029231e-09
3999 6.73972568618342e-09
4000 6.73930041976412e-09
4001 6.73858361301127e-09
4002 6.73806090310813e-09
4003 6.73767741406206e-09
4004 6.73714902453548e-09
4005 6.73629420684663e-09
4006 6.73582081804625e-09
4007 6.7356243487332e-09
4008 6.73477703934389e-09
4009 6.7343996368735e-09
4010 6.73402247923505e-09
4011 6.73343039946095e-09
4012 6.73267021646107e-09
4013 6.7323437051986e-09
4014 6.73136569674659e-09
4015 6.73089443323227e-09
4016 6.73059470222836e-09
4017 6.729797859234e-09
4018 6.72933236182949e-09
4019 6.72957456739065e-09
4020 6.72829657896723e-09
4021 6.72765419651444e-09
4022 6.72757439372507e-09
4023 6.72652035194066e-09
4024 6.72596789917657e-09
4025 6.72595817623189e-09
4026 6.72508735539978e-09
4027 6.72476380567111e-09
4028 6.72416157722344e-09
4029 6.72311426394567e-09
4030 6.72326137110546e-09
4031 6.72262195987317e-09
4032 6.72208730455315e-09
4033 6.7213091164281e-09
4034 6.7205486142391e-09
4035 6.72047456985736e-09
4036 6.71953203232323e-09
4037 6.7192355044654e-09
4038 6.71918028093132e-09
4039 6.71761265531867e-09
4040 6.71764772751382e-09
4041 6.71769876839368e-09
4042 6.71638274171715e-09
4043 6.71605721276614e-09
4044 6.7152967298395e-09
4045 6.71514198183343e-09
4046 6.71452686930307e-09
4047 6.71422345348283e-09
4048 6.71354315524841e-09
4049 6.71240098919224e-09
4050 6.71302630998738e-09
4051 6.71181384352681e-09
4052 6.71154040937649e-09
4053 6.71112360924042e-09
4054 6.71040403919798e-09
4055 6.70915648748327e-09
4056 6.7098458041287e-09
4057 6.70860273561946e-09
4058 6.70841359989849e-09
4059 6.70778041508835e-09
4060 6.70669679139757e-09
4061 6.70679083672465e-09
4062 6.70652487676227e-09
4063 6.70544800086825e-09
4064 6.70516062385718e-09
4065 6.70468748556474e-09
4066 6.70367765627511e-09
4067 6.70318377744839e-09
4068 6.70339839957401e-09
4069 6.70235864815016e-09
4070 6.7020256882544e-09
4071 6.70074785143193e-09
4072 6.70133899496883e-09
4073 6.70037387943967e-09
4074 6.70031345073596e-09
4075 6.6992832225693e-09
4076 6.69901993965993e-09
4077 6.69782493731286e-09
4078 6.69810165941642e-09
4079 6.69714685463951e-09
4080 6.69716068217585e-09
4081 6.69620265683624e-09
4082 6.695313522459e-09
4083 6.6954075936404e-09
4084 6.69470408788675e-09
4085 6.69399819107352e-09
4086 6.69397663911886e-09
4087 6.69292768133711e-09
4088 6.6921849830176e-09
4089 6.69225064701962e-09
4090 6.69178095560885e-09
4091 6.69082891707751e-09
4092 6.69069571165859e-09
4093 6.68988646358726e-09
4094 6.68884853714802e-09
4095 6.68915795674907e-09
4096 6.68849835558372e-09
4097 6.68769271833236e-09
4098 6.68765760548917e-09
4099 6.68600581187062e-09
4100 6.68648299771701e-09
4101 6.6861183346395e-09
4102 6.68497874431462e-09
4103 6.68473178498663e-09
4104 6.68425364383496e-09
4105 6.68295834997668e-09
4106 6.68320344797413e-09
4107 6.683075082628e-09
4108 6.68186436295071e-09
4109 6.68139463937123e-09
4110 6.68120097364888e-09
4111 6.68039881718263e-09
4112 6.67972202175737e-09
4113 6.67977480740822e-09
4114 6.67864784710537e-09
4115 6.67847131752275e-09
4116 6.67772058580984e-09
4117 6.67747245475248e-09
4118 6.67707902650416e-09
4119 6.67664420284098e-09
4120 6.67600136176893e-09
4121 6.67476695324909e-09
4122 6.67536887991038e-09
4123 6.67433234537163e-09
4124 6.67388782790712e-09
4125 6.6735719423533e-09
4126 6.67287496267099e-09
4127 6.67171532456323e-09
4128 6.67215829248946e-09
4129 6.67127634994269e-09
4130 6.67124826049525e-09
4131 6.67029503481631e-09
4132 6.66976292415522e-09
4133 6.66922081901666e-09
4134 6.6689026762129e-09
4135 6.668339608773e-09
4136 6.6680681247433e-09
4137 6.66733099614902e-09
4138 6.66570556959178e-09
4139 6.66576635156846e-09
4140 6.66500189476504e-09
4141 6.66440181364136e-09
4142 6.66389034538473e-09
4143 6.66304993644784e-09
4144 6.66255407158478e-09
4145 6.66220029657549e-09
4146 6.66140199678811e-09
4147 6.66046236107165e-09
4148 6.66042686693624e-09
4149 6.65970111836389e-09
4150 6.65952329342856e-09
4151 6.66067232822642e-09
4152 6.65927858524429e-09
4153 6.65853321486776e-09
4154 6.65505215417805e-09
4155 6.65521505115174e-09
4156 6.65427887942827e-09
4157 6.65398661013616e-09
4158 6.65321239053107e-09
4159 6.65272465599776e-09
4160 6.65159261974135e-09
4161 6.65133707029864e-09
4162 6.65076972199141e-09
4163 6.65005179552314e-09
4164 6.64972134707675e-09
4165 6.64913848281989e-09
4166 6.64803794485358e-09
4167 6.64775383382721e-09
4168 6.64706469312437e-09
4169 6.64647774716032e-09
4170 6.64701323732642e-09
4171 6.64476155730342e-09
4172 6.64484181744918e-09
4173 6.64425112008316e-09
4174 6.64358908675222e-09
4175 6.64326821168959e-09
4176 6.64241099337937e-09
4177 6.64192403830588e-09
4178 6.64138035531836e-09
4179 6.64075899474426e-09
4180 6.64029225526386e-09
4181 6.63971361521132e-09
4182 6.63911856449428e-09
4183 6.63847406009999e-09
4184 6.63750938971475e-09
4185 6.63783911485083e-09
4186 6.6367529896888e-09
4187 6.63675405788988e-09
4188 6.63591740790448e-09
4189 6.63537985566565e-09
4190 6.63471971940055e-09
4191 6.63394260770611e-09
4192 6.63366405348476e-09
4193 6.63332242799297e-09
4194 6.63244259296869e-09
4195 6.63200996670055e-09
4196 6.63193850454491e-09
4197 6.63085464359947e-09
4198 6.63060526774617e-09
4199 6.62922546790234e-09
4200 6.62924723079938e-09
4201 6.62859610550681e-09
4202 6.62871976112034e-09
4203 6.6275452131892e-09
4204 6.6273407002132e-09
4205 6.62654497889104e-09
4206 6.62630312388279e-09
4207 6.62563096583135e-09
4208 6.6251843000853e-09
4209 6.62369869290635e-09
4210 6.62421567551585e-09
4211 6.6237775016853e-09
4212 6.62270320075931e-09
4213 6.62257204014915e-09
4214 6.62131446062253e-09
4215 6.62135566806277e-09
4216 6.62107978049997e-09
4217 6.62058854987435e-09
4218 6.62004077919831e-09
4219 6.61820545169722e-09
4220 6.61876816107632e-09
4221 6.61854912611948e-09
4222 6.61791630575392e-09
4223 6.6170349014294e-09
4224 6.61695512665927e-09
4225 6.61551054494114e-09
4226 6.61562775536306e-09
4227 6.61490285526478e-09
4228 6.61466917431519e-09
4229 6.61338603087647e-09
4230 6.61405581434127e-09
4231 6.61273544898255e-09
4232 6.61265635160113e-09
4233 6.6118002901433e-09
4234 6.61095892375008e-09
4235 6.61076565773577e-09
4236 6.61002809890232e-09
4237 6.60981808978112e-09
4238 6.60921036178819e-09
4239 6.60864177336185e-09
4240 6.60731841972684e-09
4241 6.60817227904342e-09
4242 6.60724234848031e-09
4243 6.60671596142137e-09
4244 6.60553239106854e-09
4245 6.60559720661036e-09
4246 6.60504278578167e-09
4247 6.60476363707751e-09
4248 6.60411394932947e-09
4249 6.60332173484102e-09
4250 6.60229171814408e-09
4251 6.60190608965661e-09
4252 6.60178847756987e-09
4253 6.60125654002031e-09
4254 6.6006541605268e-09
4255 6.5996270058738e-09
4256 6.59974002706343e-09
4257 6.59918724973951e-09
4258 6.59895388847864e-09
4259 6.59825327446739e-09
4260 6.59742556277654e-09
4261 6.5964582159489e-09
4262 6.59655233586909e-09
4263 6.59597342479723e-09
4264 6.59590374306462e-09
4265 6.59506476344274e-09
4266 6.59451278016421e-09
4267 6.59354010554281e-09
4268 6.5930750660359e-09
4269 6.59305653326336e-09
4270 6.592382607154e-09
4271 6.59234277201881e-09
4272 6.59074013749372e-09
4273 6.59039127283434e-09
4274 6.59073605502325e-09
4275 6.59005473699348e-09
4276 6.58949858728386e-09
4277 6.58829044979103e-09
4278 6.5885725438336e-09
4279 6.58785346945023e-09
4280 6.58755433438629e-09
4281 6.58680531336053e-09
4282 6.58643387450863e-09
4283 6.58570227754507e-09
4284 6.58545552791046e-09
4285 6.58463387305519e-09
4286 6.58388151338063e-09
4287 6.58370799858088e-09
4288 6.58224549218711e-09
4289 6.58307697774807e-09
4290 6.58215083110258e-09
4291 6.58167061291792e-09
4292 6.58129291954135e-09
4293 6.5807355763281e-09
4294 6.5795205700242e-09
4295 6.58007632799162e-09
4296 6.5788291795793e-09
4297 6.57883622201538e-09
4298 6.57835064898915e-09
4299 6.57695990728679e-09
4300 6.57638550251305e-09
4301 6.57717785837453e-09
4302 6.57608135827548e-09
4303 6.57571830904591e-09
4304 6.57452844442241e-09
4305 6.57480941795852e-09
4306 6.57420342690951e-09
4307 6.57393475254897e-09
4308 6.57266517255228e-09
4309 6.57271748850941e-09
4310 6.57174080538614e-09
4311 6.57192602492029e-09
4312 6.57120161877411e-09
4313 6.57079760198187e-09
4314 6.57027616447758e-09
4315 6.56895703869675e-09
4316 6.56859927307407e-09
4317 6.56883978415734e-09
4318 6.56832411323471e-09
4319 6.56806709578572e-09
4320 6.56727067203933e-09
4321 6.56596393332209e-09
4322 6.56670662459169e-09
4323 6.56576733988612e-09
4324 6.56558897495219e-09
4325 6.56475170787707e-09
4326 6.56428151747657e-09
4327 6.56330000414762e-09
4328 6.56327922247957e-09
4329 6.5628010212232e-09
4330 6.56256304619462e-09
4331 6.56174682552024e-09
4332 6.56054143559348e-09
4333 6.56117515220045e-09
4334 6.56025972839425e-09
4335 6.55975014587917e-09
4336 6.55958125019007e-09
4337 6.55862668069718e-09
4338 6.55766460398721e-09
4339 6.55813951838669e-09
4340 6.55701088432548e-09
4341 6.55625323679743e-09
4342 6.55597989170087e-09
4343 6.5549838177692e-09
4344 6.55578035306525e-09
4345 6.55471823117482e-09
4346 6.55414670436283e-09
4347 6.55389293095776e-09
4348 6.55236916199953e-09
4349 6.5526698811158e-09
4350 6.55204510949248e-09
4351 6.55173093914996e-09
4352 6.55127445285675e-09
4353 6.55053236349457e-09
4354 6.54953531024904e-09
4355 6.54994102453066e-09
4356 6.54912944046682e-09
4357 6.5490628679693e-09
4358 6.54795856785961e-09
4359 6.54652711087889e-09
4360 6.54742915430495e-09
4361 6.54661864514516e-09
4362 6.54619144163648e-09
4363 6.54597000175616e-09
4364 6.54532550142806e-09
4365 6.54444941168564e-09
4366 6.54434181134034e-09
4367 6.54361543164783e-09
4368 6.54313306659715e-09
4369 6.54228780953836e-09
4370 6.54125207179279e-09
4371 6.54168052995063e-09
4372 6.54087467384656e-09
4373 6.5405646092892e-09
4374 6.53978124541543e-09
4375 6.53927773978935e-09
4376 6.53909270192932e-09
4377 6.5387691972757e-09
4378 6.53772955831744e-09
4379 6.53708284641119e-09
4380 6.53620842323843e-09
4381 6.53553943189e-09
4382 6.53494457764081e-09
4383 6.53468836027404e-09
4384 6.53370164342693e-09
4385 6.53593364992322e-09
4386 6.53187665876354e-09
4387 6.53554217060204e-09
4388 6.53477842933736e-09
4389 6.53441950164424e-09
4390 6.53078766742199e-09
4391 6.5331622904341e-09
4392 6.53139700096395e-09
4393 6.53042988815744e-09
4394 6.52957072980187e-09
4395 6.52932771297299e-09
4396 6.52912025142505e-09
4397 6.52844408499664e-09
4398 6.52816774046605e-09
4399 6.52690046419602e-09
4400 6.52685998281388e-09
4401 6.52627331744482e-09
4402 6.52474426206251e-09
4403 6.52548117092977e-09
4404 6.52388674780846e-09
4405 6.52364681424933e-09
4406 6.52322377352188e-09
4407 6.52257346350771e-09
4408 6.52215391977462e-09
4409 6.52063950454251e-09
4410 6.52085800849356e-09
4411 6.52039633282364e-09
4412 6.5194652351247e-09
4413 6.51796802711435e-09
4414 6.51899540775325e-09
4415 6.51792770391124e-09
4416 6.51724184440317e-09
4417 6.5159857116498e-09
4418 6.51654128218382e-09
4419 6.51570514977051e-09
4420 6.51451250646695e-09
4421 6.51498136688966e-09
4422 6.5137698635892e-09
4423 6.51359686575093e-09
4424 6.51182165173347e-09
4425 6.51249973528112e-09
4426 6.51137283190495e-09
4427 6.51115897062904e-09
4428 6.51050797681463e-09
4429 6.51025745129763e-09
4430 6.50954037337281e-09
4431 6.50853190874157e-09
4432 6.50815260640203e-09
4433 6.50750292149893e-09
4434 6.50743954209998e-09
4435 6.50605234606139e-09
4436 6.50505736021767e-09
4437 6.50515018939257e-09
4438 6.50440108851402e-09
4439 6.50395503290491e-09
4440 6.5034372191547e-09
4441 6.502815238979e-09
4442 6.50330218206208e-09
4443 6.50214610475874e-09
4444 6.50112072059228e-09
4445 6.50112830867522e-09
4446 6.50048153630345e-09
4447 6.49984787848279e-09
4448 6.49931584552343e-09
4449 6.4986458166022e-09
4450 6.49891888561405e-09
4451 6.49766554293441e-09
4452 6.4976923886545e-09
4453 6.49645793615594e-09
4454 6.49681957023973e-09
4455 6.49599823047964e-09
4456 6.49501096169902e-09
4457 6.49561736053295e-09
4458 6.4939129698488e-09
4459 6.49424461408865e-09
4460 6.49307139202748e-09
4461 6.49356956559932e-09
4462 6.49181394100917e-09
4463 6.49158553517326e-09
4464 6.49182932548065e-09
4465 6.49082442356896e-09
4466 6.49016975060035e-09
4467 6.48997678345808e-09
4468 6.48902633354487e-09
4469 6.48929240942842e-09
4470 6.48809899155001e-09
4471 6.48800018673845e-09
4472 6.48688374223216e-09
4473 6.48709474829201e-09
4474 6.48606088780868e-09
4475 6.48618266148315e-09
4476 6.48494364767493e-09
4477 6.48513158126029e-09
4478 6.48402499835166e-09
4479 6.4842743257576e-09
4480 6.4829675998912e-09
4481 6.48310194470425e-09
4482 6.48216441585803e-09
4483 6.48190973427665e-09
4484 6.48111631418313e-09
4485 6.4810292989409e-09
4486 6.48007292985975e-09
4487 6.48012712482771e-09
4488 6.47926771055185e-09
4489 6.47915853500824e-09
4490 6.47816342734531e-09
4491 6.47792335138619e-09
4492 6.47745372377162e-09
4493 6.47687432679678e-09
4494 6.47645059702329e-09
4495 6.47594954171493e-09
4496 6.47540106152311e-09
4497 6.47497072937819e-09
4498 6.47487860255813e-09
4499 6.4734519910209e-09
4500 6.47363259799472e-09
4501 6.47300392260519e-09
4502 6.47254582340773e-09
4503 6.47198258313386e-09
4504 6.47179977961154e-09
4505 6.4706600829828e-09
4506 6.47079699987729e-09
4507 6.47001141905468e-09
4508 6.46985582079818e-09
4509 6.46927430587863e-09
4510 6.46787746945432e-09
4511 6.46861116500308e-09
4512 6.46735678887234e-09
4513 6.46754759263402e-09
4514 6.46627915595888e-09
4515 6.46600509975059e-09
4516 6.46605007813272e-09
4517 6.46548552606097e-09
4518 6.46436354842517e-09
4519 6.4651494093293e-09
4520 6.46351347405316e-09
4521 6.46355993848269e-09
4522 6.46273507932948e-09
4523 6.46212716710892e-09
4524 6.46225867144412e-09
4525 6.46179575324035e-09
4526 6.46095097002475e-09
4527 6.4605572230314e-09
4528 6.45994710156417e-09
4529 6.45947898625376e-09
4530 6.45901711637808e-09
4531 6.4584979396487e-09
4532 6.4581045370049e-09
4533 6.45764055591524e-09
4534 6.45692983709811e-09
4535 6.45677475939743e-09
4536 6.45617754627126e-09
4537 6.45584221843631e-09
4538 6.45509654823018e-09
4539 6.45448699281015e-09
4540 6.45427403388188e-09
4541 6.45364176807273e-09
4542 6.45323022503697e-09
4543 6.45294431954324e-09
4544 6.45226628168127e-09
4545 6.45220938708113e-09
4546 6.45077408305272e-09
4547 6.45017288102012e-09
4548 6.45021844322691e-09
4549 6.44944790510749e-09
4550 6.44999814469749e-09
4551 6.44814581453324e-09
4552 6.44815443641422e-09
4553 6.44774086695776e-09
4554 6.44743692179706e-09
4555 6.44696394172528e-09
4556 6.4461063172816e-09
4557 6.44589574717858e-09
4558 6.44498383491066e-09
4559 6.44454713723264e-09
4560 6.44473459868178e-09
4561 6.44382759697193e-09
4562 6.44393226974238e-09
4563 6.44255422439788e-09
4564 6.44315896448922e-09
4565 6.44189009978147e-09
4566 6.44131152389982e-09
4567 6.44074915193138e-09
4568 6.44099137446508e-09
4569 6.44004195411718e-09
4570 6.44014135406379e-09
4571 6.43872818441305e-09
4572 6.43891503067373e-09
4573 6.43838938138574e-09
4574 6.43745629759496e-09
4575 6.43755422177761e-09
4576 6.4367530215742e-09
4577 6.43649632005305e-09
4578 6.43566644066562e-09
4579 6.43563180939555e-09
4580 6.43467695446431e-09
4581 6.43473514032322e-09
4582 6.43391799780291e-09
4583 6.43396350261116e-09
4584 6.4331288026076e-09
4585 6.43280723944262e-09
4586 6.43236875363939e-09
4587 6.43167133126954e-09
4588 6.43156078140861e-09
4589 6.43063563207646e-09
4590 6.43100333232172e-09
4591 6.42939359932204e-09
4592 6.42930231724292e-09
4593 6.42882049545213e-09
4594 6.42844057914538e-09
4595 6.42866566195888e-09
4596 6.42795157670961e-09
4597 6.42623271757126e-09
4598 6.4264109974066e-09
4599 6.42647127888085e-09
4600 6.42559753881444e-09
4601 6.42617626030184e-09
4602 6.42505669863624e-09
4603 6.42460197060324e-09
4604 6.42387642674214e-09
4605 6.42379498679013e-09
4606 6.42286478312604e-09
4607 6.42253823902872e-09
4608 6.42175782152032e-09
4609 6.4220375769336e-09
4610 6.42187710075959e-09
4611 6.42056871419383e-09
4612 6.42081920960991e-09
4613 6.41953499343206e-09
4614 6.41963410129287e-09
4615 6.41846716395444e-09
4616 6.41836421079456e-09
4617 6.41894963078482e-09
4618 6.41747713082952e-09
4619 6.417393007635e-09
4620 6.41664285191967e-09
4621 6.41626153045916e-09
4622 6.41580086947757e-09
4623 6.41520269482887e-09
4624 6.41529244339856e-09
4625 6.41375292187441e-09
4626 6.41389518218904e-09
4627 6.41290653181914e-09
4628 6.41282994627568e-09
4629 6.41192685862935e-09
4630 6.41188208111432e-09
4631 6.41084794215929e-09
4632 6.41021107455897e-09
4633 6.40915140996678e-09
4634 6.40960854281225e-09
4635 6.40825547704171e-09
4636 6.40844950539066e-09
4637 6.4073927587005e-09
4638 6.40731273858497e-09
4639 6.40705866521152e-09
4640 6.40673147515869e-09
4641 6.40591695218184e-09
4642 6.40535552769161e-09
4643 6.40521416944706e-09
4644 6.40438873124316e-09
4645 6.40361370292386e-09
4646 6.40368838317196e-09
4647 6.40362126617944e-09
4648 6.40269077242794e-09
4649 6.40181193826972e-09
4650 6.40202964950765e-09
4651 6.40140925479982e-09
4652 6.40064459031531e-09
4653 6.40047006743305e-09
4654 6.39986764160161e-09
4655 6.39992349199536e-09
4656 6.39874287498843e-09
4657 6.39865517181037e-09
4658 6.39773458677162e-09
4659 6.39777224428761e-09
4660 6.39656286283152e-09
4661 6.39580030083187e-09
4662 6.39606635431667e-09
4663 6.39592782564202e-09
4664 6.39473417182734e-09
4665 6.3949004351499e-09
4666 6.3940997235834e-09
4667 6.39332600629872e-09
4668 6.39285435498349e-09
4669 6.39330016104223e-09
4670 6.39229706879407e-09
4671 6.39102698653249e-09
4672 6.39197413383752e-09
4673 6.39062035868998e-09
4674 6.39073001400259e-09
4675 6.38885354135865e-09
4676 6.39010290058584e-09
4677 6.38885928533339e-09
4678 6.38870775399036e-09
4679 6.38775351174958e-09
4680 6.38770054330051e-09
4681 6.38709153284733e-09
4682 6.38727096419212e-09
4683 6.3861358941264e-09
4684 6.38600498496789e-09
4685 6.38530997544573e-09
4686 6.38512049140616e-09
4687 6.38439512629097e-09
4688 6.38412832439095e-09
4689 6.38314620460267e-09
4690 6.38363047189561e-09
4691 6.38209671033341e-09
4692 6.38270747367609e-09
4693 6.38162189209313e-09
4694 6.38224282011046e-09
4695 6.38054177569403e-09
4696 6.38049529209928e-09
4697 6.3797587953468e-09
4698 6.38006763088728e-09
4699 6.37879846207556e-09
4700 6.37901656459772e-09
4701 6.37835483353888e-09
4702 6.37815650385287e-09
4703 6.37697281984095e-09
4704 6.37688501130274e-09
4705 6.37599036314163e-09
4706 6.37657123923885e-09
4707 6.37512446038935e-09
4708 6.37573300225902e-09
4709 6.37408760498226e-09
4710 6.37505311805453e-09
4711 6.373206997215e-09
4712 6.3739025393944e-09
4713 6.37209803494576e-09
4714 6.37289391153262e-09
4715 6.37162979741368e-09
4716 6.37247151229203e-09
4717 6.37133199417328e-09
4718 6.37073817953693e-09
4719 6.36993006092323e-09
4720 6.37017121460359e-09
4721 6.36855248195567e-09
4722 6.3693021372796e-09
4723 6.36860054946398e-09
4724 6.36824935427993e-09
4725 6.36695636883899e-09
4726 6.36663368688839e-09
4727 6.36624027576527e-09
4728 6.36602488811311e-09
4729 6.36532937797751e-09
4730 6.36495049005648e-09
4731 6.3647512233006e-09
4732 6.36510648777122e-09
4733 6.3635990364963e-09
4734 6.36334102126213e-09
4735 6.3629859978348e-09
4736 6.36230208238275e-09
4737 6.36207727226779e-09
4738 6.36152552981051e-09
4739 6.36084737755394e-09
4740 6.3613848805405e-09
4741 6.36013176134875e-09
4742 6.35958612922594e-09
4743 6.35941015146579e-09
4744 6.35884233812778e-09
4745 6.35829422683531e-09
4746 6.35796266201805e-09
4747 6.35722307569531e-09
4748 6.35895642166207e-09
4749 6.35685369927697e-09
4750 6.35761677679481e-09
4751 6.35623941443175e-09
4752 6.35636247550697e-09
4753 6.35512754181e-09
4754 6.35485030157923e-09
4755 6.35547359738309e-09
4756 6.35437958826368e-09
4757 6.35322008646355e-09
4758 6.35354749088657e-09
4759 6.35245091536174e-09
4760 6.35264532876378e-09
4761 6.35203440520205e-09
4762 6.35572486333558e-09
4763 6.35400161753952e-09
4764 6.35359242528999e-09
4765 6.35254491580806e-09
4766 6.35260113800162e-09
4767 6.35207585687747e-09
4768 6.35167254044167e-09
4769 6.35148327179924e-09
4770 6.35032857938811e-09
4771 6.35037259387461e-09
4772 6.34906083134268e-09
4773 6.34937044609318e-09
4774 6.34846925867683e-09
4775 6.34889113022841e-09
4776 6.34779726066403e-09
4777 6.3479352565704e-09
4778 6.34669985001557e-09
4779 6.34699121646043e-09
4780 6.34585882459959e-09
4781 6.34566107253487e-09
4782 6.34515869660235e-09
4783 6.34526534652735e-09
4784 6.34456404216555e-09
4785 6.34433400521772e-09
4786 6.34346211415837e-09
4787 6.34328306928222e-09
4788 6.34251021496002e-09
4789 6.34190155499037e-09
4790 6.34197474184972e-09
4791 6.34116705398868e-09
4792 6.34139094610187e-09
4793 6.33976679534354e-09
4794 6.34028947103793e-09
4795 6.33901351780597e-09
4796 6.33926304423327e-09
4797 6.33840633211535e-09
4798 6.33811502530335e-09
4799 6.33734481943204e-09
4800 6.33788968927573e-09
4801 6.33657706740343e-09
4802 6.33658435966744e-09
4803 6.33549372015818e-09
4804 6.33564644696161e-09
4805 6.33477134839855e-09
4806 6.3344845158686e-09
4807 6.33369081370905e-09
4808 6.33398029564747e-09
4809 6.33302149756487e-09
4810 6.33292029714183e-09
4811 6.33190815636786e-09
4812 6.33209673575608e-09
4813 6.33139896435986e-09
4814 6.33076470840654e-09
4815 6.32978099086856e-09
4816 6.32945108310079e-09
4817 6.32888075895177e-09
4818 6.33275243958065e-09
4819 6.33112936972158e-09
4820 6.33136414544822e-09
4821 6.32999235135323e-09
4822 6.32883068613088e-09
4823 6.32837809086406e-09
4824 6.32843391672189e-09
4825 6.32782292811496e-09
4826 6.32715093797087e-09
4827 6.32667104762119e-09
4828 6.32661235040743e-09
4829 6.32542062573016e-09
4830 6.32554281451703e-09
4831 6.32480902890142e-09
4832 6.32411979005487e-09
4833 6.32410218431856e-09
4834 6.32329811371679e-09
4835 6.323018783852e-09
4836 6.32226980315509e-09
4837 6.32174578567291e-09
4838 6.32140072968912e-09
4839 6.32079998746926e-09
4840 6.32064812275401e-09
4841 6.32026117429463e-09
4842 6.31955104245241e-09
4843 6.31865005981669e-09
4844 6.31835099396127e-09
4845 6.31781226807793e-09
4846 6.31740458970076e-09
4847 6.31685699675755e-09
4848 6.31679218805747e-09
4849 6.31615475786551e-09
4850 6.31579399297533e-09
4851 6.31511014963226e-09
4852 6.31513824059238e-09
4853 6.31380275759041e-09
4854 6.31350744488557e-09
4855 6.31335160683477e-09
4856 6.31256925330559e-09
4857 6.31239722381771e-09
4858 6.31158854728919e-09
4859 6.31148489559086e-09
4860 6.31065732603631e-09
4861 6.30994821186615e-09
4862 6.30973724018158e-09
4863 6.3095690267384e-09
4864 6.30908035510747e-09
4865 6.30814257200629e-09
4866 6.30783551103242e-09
4867 6.30728554489035e-09
4868 6.30691913813819e-09
4869 6.30612600950597e-09
4870 6.30631399192727e-09
4871 6.30551765928855e-09
4872 6.3052815054282e-09
4873 6.30461175421537e-09
4874 6.30431758610883e-09
4875 6.30339700033455e-09
4876 6.30380670106623e-09
4877 6.30323829715895e-09
4878 6.30246163009496e-09
4879 6.30186945621558e-09
4880 6.3016086846307e-09
4881 6.30087715680627e-09
4882 6.30066586884692e-09
4883 6.30007321740511e-09
4884 6.29989703816725e-09
4885 6.29908180928285e-09
4886 6.29888287773106e-09
4887 6.29808306505664e-09
4888 6.29796272282046e-09
4889 6.29756949582783e-09
4890 6.29719096817416e-09
4891 6.2962017549073e-09
4892 6.29677547953067e-09
4893 6.29697112498884e-09
4894 6.29688203064582e-09
4895 6.29577240131296e-09
4896 6.2955448159735e-09
4897 6.29480180576458e-09
4898 6.29505823032672e-09
4899 6.2938199236473e-09
4900 6.2940648166282e-09
4901 6.2930914688647e-09
4902 6.29280264238807e-09
4903 6.29213569409981e-09
4904 6.29245163839831e-09
4905 6.2911692040174e-09
4906 6.29109141443707e-09
4907 6.29067566225439e-09
4908 6.29022161693915e-09
4909 6.28943225905454e-09
4910 6.28969589912265e-09
4911 6.28850913528267e-09
4912 6.28830519873447e-09
4913 6.28824010255602e-09
4914 6.28751728386223e-09
4915 6.28673846499173e-09
4916 6.28702197182085e-09
4917 6.28585419780447e-09
4918 6.28570817473584e-09
4919 6.28495008039853e-09
4920 6.2848533517873e-09
4921 6.28406821757965e-09
4922 6.28395109471269e-09
4923 6.28314959623399e-09
4924 6.28309815671468e-09
4925 6.28252322686096e-09
4926 6.28219042431155e-09
4927 6.2817039794133e-09
4928 6.28095780667859e-09
4929 6.28084169917997e-09
4930 6.27998865937451e-09
4931 6.27994038719659e-09
4932 6.2790832895121e-09
4933 6.27935600898411e-09
4934 6.27826166628431e-09
4935 6.27839878822312e-09
4936 6.27726248393634e-09
4937 6.2771674275014e-09
4938 6.27676019031298e-09
4939 6.27633853936271e-09
4940 6.27557995327987e-09
4941 6.27589154300612e-09
4942 6.27459326403568e-09
4943 6.27514531974238e-09
4944 6.27369494976549e-09
4945 6.27344586799639e-09
4946 6.27280263976182e-09
4947 6.27268222454236e-09
4948 6.27233885047707e-09
4949 6.27158845172004e-09
4950 6.2726329118773e-09
4951 6.27139841730129e-09
4952 6.2702650147628e-09
4953 6.26967496687725e-09
4954 6.26981338718024e-09
4955 6.26908447347307e-09
4956 6.26936114227206e-09
4957 6.26828997978002e-09
4958 6.26751218227306e-09
4959 6.26716243563863e-09
4960 6.26743298186294e-09
4961 6.26624316238389e-09
4962 6.26676997336761e-09
4963 6.26543851196815e-09
4964 6.26456818828003e-09
4965 6.26405261337781e-09
4966 6.26489236936478e-09
4967 6.26315302729163e-09
4968 6.26259970558374e-09
4969 6.26230773032033e-09
4970 6.26316371581115e-09
4971 6.26153254866535e-09
4972 6.26077389394297e-09
4973 6.26054658613151e-09
4974 6.26143482929109e-09
4975 6.25963114973815e-09
4976 6.25955619501517e-09
4977 6.25857492257687e-09
4978 6.25952182012335e-09
4979 6.25786146811469e-09
4980 6.25709882100256e-09
4981 6.25669642756455e-09
4982 6.25784165859955e-09
4983 6.25596128521477e-09
4984 6.25530744334524e-09
4985 6.25499068307445e-09
4986 6.25595795851475e-09
4987 6.25443464667697e-09
4988 6.25379842564278e-09
4989 6.25318987580725e-09
4990 6.2543001879134e-09
4991 6.25240352161716e-09
4992 6.251624262682e-09
4993 6.25257774664045e-09
4994 6.25109379744082e-09
4995 6.2506150988878e-09
4996 6.25111774395226e-09
4997 6.24973387679251e-09
4998 6.24921206053564e-09
4999 6.24993656986317e-09
};
\addlegendentry{Train}
\addplot [semithick, black]
table {%
0 0.00100090191699564
1 0.000221884401980788
2 0.000188961610547267
3 0.000162966694915667
4 0.000108854117570445
5 4.38603565271478e-05
6 2.29823308472987e-05
7 1.32174482132541e-05
8 7.74125055613695e-06
9 5.34401351615088e-06
10 4.26577616963186e-06
11 3.63888898391451e-06
12 3.15245620186033e-06
13 2.78147763310699e-06
14 2.50819653047074e-06
15 2.31533613259671e-06
16 2.17249225897831e-06
17 2.06817571779538e-06
18 1.98825978259265e-06
19 1.92393758879916e-06
20 1.87013267805014e-06
21 1.82007022431208e-06
22 1.77706749582285e-06
23 1.73414423443319e-06
24 1.69075656231144e-06
25 1.6499307093909e-06
26 1.61150740041194e-06
27 1.57439797021652e-06
28 1.53572443650773e-06
29 1.49804179727653e-06
30 1.46136085277249e-06
31 1.42719454743201e-06
32 1.39215421768313e-06
33 1.35817560931173e-06
34 1.32176342049206e-06
35 1.28751844385988e-06
36 1.24991277061781e-06
37 1.21578409562062e-06
38 1.18288005523937e-06
39 1.14989006760879e-06
40 1.10931068775244e-06
41 1.0771954066513e-06
42 1.04458626992709e-06
43 1.0116793873749e-06
44 9.77920308287139e-07
45 9.5280211098725e-07
46 9.11001961867441e-07
47 8.78395496783924e-07
48 8.468571195408e-07
49 8.18707519556483e-07
50 7.92250091308233e-07
51 7.68871302625485e-07
52 7.4592344390112e-07
53 7.25179859273339e-07
54 7.05582579030306e-07
55 6.86287989992707e-07
56 6.65287416268256e-07
57 6.48323748464463e-07
58 6.32670435152249e-07
59 6.20630487446761e-07
60 6.10581309956615e-07
61 6.00279577156471e-07
62 5.8730262253448e-07
63 5.78652361582499e-07
64 5.69007625017548e-07
65 5.62366665235459e-07
66 5.56027202947007e-07
67 5.42002169368061e-07
68 5.3605987204719e-07
69 5.30491661265842e-07
70 5.25640871273936e-07
71 5.20891319411021e-07
72 5.17925570875377e-07
73 5.14935607043299e-07
74 5.11565531269298e-07
75 5.08725008785404e-07
76 5.06085370943765e-07
77 5.03814874264208e-07
78 5.01943247854797e-07
79 5.02808177316183e-07
80 5.02887417042075e-07
81 5.03073863455938e-07
82 5.00774092415668e-07
83 4.99263762776536e-07
84 4.97990811254567e-07
85 4.94821222218889e-07
86 4.92937545004679e-07
87 4.86609621930256e-07
88 4.86829435430991e-07
89 4.80201208574726e-07
90 4.81879169456079e-07
91 4.79750269732904e-07
92 4.72186655997575e-07
93 4.69572853489808e-07
94 4.6831681288495e-07
95 4.65666971649625e-07
96 4.62684937474478e-07
97 4.60379226296936e-07
98 4.59130461649693e-07
99 4.55098103202545e-07
100 4.52583520882399e-07
101 4.50436544952026e-07
102 4.48598029834102e-07
103 4.46903243300767e-07
104 4.46721173830156e-07
105 4.44316555103796e-07
106 4.40909730059502e-07
107 4.39134169027966e-07
108 4.37386830753894e-07
109 4.36856396390795e-07
110 4.33336367677839e-07
111 4.31614495255417e-07
112 4.29947533575614e-07
113 4.26771066486253e-07
114 4.24940765242354e-07
115 4.22118830556428e-07
116 4.21180203602489e-07
117 4.19125171902124e-07
118 4.16551216630978e-07
119 4.14157739214716e-07
120 4.14738735798892e-07
121 4.12795458260007e-07
122 4.09647640253752e-07
123 4.08955401098865e-07
124 4.05647654133645e-07
125 4.02340305072357e-07
126 4.01820472006875e-07
127 3.99693249164557e-07
128 3.97553350239832e-07
129 3.9561791709275e-07
130 3.93061839076836e-07
131 3.90275658901373e-07
132 3.89529532185406e-07
133 3.86580637723455e-07
134 3.8414034975176e-07
135 3.81891823053593e-07
136 3.79654892412873e-07
137 3.7705237332375e-07
138 3.74040155293187e-07
139 3.68795042504644e-07
140 3.65347176511932e-07
141 3.61650762670251e-07
142 3.58488904339538e-07
143 3.56657750444356e-07
144 3.52771849065903e-07
145 3.49989988990274e-07
146 3.47429931935039e-07
147 3.44433487953211e-07
148 3.41470268949706e-07
149 3.40351192562593e-07
150 3.38757843110216e-07
151 3.35721523470056e-07
152 3.33404670982418e-07
153 3.28007445205003e-07
154 3.23345261676877e-07
155 3.20734130809797e-07
156 3.14971657644492e-07
157 3.11480789605412e-07
158 3.09946955212581e-07
159 3.07625072082374e-07
160 3.03870507423198e-07
161 3.00421078236468e-07
162 3.00300996514125e-07
163 2.98595352887787e-07
164 2.97476219657256e-07
165 2.94382175525243e-07
166 2.95024619845208e-07
167 2.93481832613907e-07
168 2.92109291422094e-07
169 2.91530113827321e-07
170 2.88466452502689e-07
171 2.86840958096946e-07
172 2.84982320408744e-07
173 2.832745451542e-07
174 2.82105673932165e-07
175 2.80375303418623e-07
176 2.78994491509366e-07
177 2.77207760746023e-07
178 2.75707634500577e-07
179 2.74338532335605e-07
180 2.72263434908382e-07
181 2.73067428224749e-07
182 2.69733533286853e-07
183 2.69942461272876e-07
184 2.66655689529216e-07
185 2.65981668690074e-07
186 2.62878359080787e-07
187 2.61565872961e-07
188 2.59889446851957e-07
189 2.58079609238848e-07
190 2.56958855970879e-07
191 2.55993171549562e-07
192 2.53991771614892e-07
193 2.53026058771866e-07
194 2.51614636681552e-07
195 2.50844664151373e-07
196 2.4866824333003e-07
197 2.47190399704778e-07
198 2.46944978243846e-07
199 2.44756137135482e-07
200 2.43242595843185e-07
201 2.42838808617307e-07
202 2.42145546280881e-07
203 2.40828654796132e-07
204 2.39974696114587e-07
205 2.38312296119148e-07
206 2.38361593574155e-07
207 2.37211608578036e-07
208 2.35697910966337e-07
209 2.34802342902185e-07
210 2.31137576633955e-07
211 2.30915787824415e-07
212 2.30060464900816e-07
213 2.29289241815422e-07
214 2.28512817557203e-07
215 2.27341217851063e-07
216 2.26686438509205e-07
217 2.26376457135302e-07
218 2.25285276655995e-07
219 2.24750934307849e-07
220 2.2360025297985e-07
221 2.21993602167458e-07
222 2.2147634126668e-07
223 2.19649237465092e-07
224 2.19218094343887e-07
225 2.16745888792502e-07
226 2.15781938095461e-07
227 2.16566007793517e-07
228 2.15212850207536e-07
229 2.13555324535264e-07
230 2.15123677094198e-07
231 2.12638568086732e-07
232 2.11676194794563e-07
233 2.10192879990245e-07
234 2.09583475907493e-07
235 2.08719683314484e-07
236 2.08317558758608e-07
237 2.07229248871954e-07
238 2.05581528689436e-07
239 2.04788889845986e-07
240 2.04225898414734e-07
241 2.04176529905453e-07
242 2.03463613956956e-07
243 2.02614188538064e-07
244 2.01990388859485e-07
245 2.0129535016622e-07
246 2.00540057448961e-07
247 2.00233188252241e-07
248 1.99024711378115e-07
249 1.98335030177077e-07
250 1.97963686332514e-07
251 1.97224593989631e-07
252 1.96668111129839e-07
253 1.95947990278e-07
254 1.95517515066967e-07
255 1.94994271396354e-07
256 1.94315774137976e-07
257 1.93829578165605e-07
258 1.93560651950975e-07
259 1.92562225720394e-07
260 1.92196480952589e-07
261 1.91523994885756e-07
262 1.91088403767026e-07
263 1.90489728879584e-07
264 1.90162012358996e-07
265 1.89428931207658e-07
266 1.88982056670284e-07
267 1.88350696816997e-07
268 1.87593954592558e-07
269 1.87264987516755e-07
270 1.86788653877556e-07
271 1.85635755656222e-07
272 1.85232352123421e-07
273 1.84739889164121e-07
274 1.84187896934418e-07
275 1.83468571890444e-07
276 1.83019551514008e-07
277 1.82451429964203e-07
278 1.82081237198872e-07
279 1.81498990059481e-07
280 1.81009696120782e-07
281 1.80776254410375e-07
282 1.80282256678765e-07
283 1.79823572921123e-07
284 1.79093746055514e-07
285 1.78755243496198e-07
286 1.78292040686756e-07
287 1.77755083541342e-07
288 1.77568992398847e-07
289 1.76979540356115e-07
290 1.76555332132011e-07
291 1.76271754526169e-07
292 1.75879449670902e-07
293 1.75341327235401e-07
294 1.74805663277766e-07
295 1.74599932734054e-07
296 1.74203577785192e-07
297 1.73804480141371e-07
298 1.73361570432462e-07
299 1.72941227560841e-07
300 1.72329606584753e-07
301 1.72065952597222e-07
302 1.71715086594304e-07
303 1.70684444356084e-07
304 1.70929254750263e-07
305 1.6999356944325e-07
306 1.69385344861439e-07
307 1.68912549725064e-07
308 1.6854774287367e-07
309 1.68349373552701e-07
310 1.68065582784038e-07
311 1.67537763218206e-07
312 1.67089112323993e-07
313 1.66700786508045e-07
314 1.66346751484525e-07
315 1.66007339430507e-07
316 1.65844994626241e-07
317 1.65594798318125e-07
318 1.65307824318006e-07
319 1.65020566100793e-07
320 1.64671050129073e-07
321 1.64621980047741e-07
322 1.64720546536046e-07
323 1.64612984576706e-07
324 1.64419034831553e-07
325 1.63520013529705e-07
326 1.63380292406146e-07
327 1.63161345767548e-07
328 1.62895375410699e-07
329 1.62544253612396e-07
330 1.61825909117397e-07
331 1.61806653409258e-07
332 1.61206159532412e-07
333 1.61076499693991e-07
334 1.60660420078784e-07
335 1.6053490980994e-07
336 1.59903606800071e-07
337 1.59993589932128e-07
338 1.59474666361348e-07
339 1.59007385036602e-07
340 1.58809726258369e-07
341 1.58749458023522e-07
342 1.58358332669195e-07
343 1.58140593953249e-07
344 1.57826988811394e-07
345 1.57596005578853e-07
346 1.5728451785435e-07
347 1.56990282107472e-07
348 1.56708253484794e-07
349 1.56615598712051e-07
350 1.5633196426279e-07
351 1.54850141598217e-07
352 1.54532060037127e-07
353 1.54092745674461e-07
354 1.53838939809248e-07
355 1.5358560290224e-07
356 1.53360190324747e-07
357 1.53149656512142e-07
358 1.53822441006923e-07
359 1.53248109313608e-07
360 1.52991077584375e-07
361 1.52692280153133e-07
362 1.52372763295716e-07
363 1.52040129819397e-07
364 1.51694351302467e-07
365 1.51356204014519e-07
366 1.51056184449772e-07
367 1.50859037262308e-07
368 1.5052083313094e-07
369 1.5019202237454e-07
370 1.49821332229294e-07
371 1.49927672055128e-07
372 1.49624398204651e-07
373 1.49263456705739e-07
374 1.48906849517516e-07
375 1.48642143926736e-07
376 1.48270913769011e-07
377 1.48229588603499e-07
378 1.47208481848793e-07
379 1.46981506077282e-07
380 1.47145840401208e-07
381 1.46937580325357e-07
382 1.46778376119983e-07
383 1.46585961147139e-07
384 1.46306987858225e-07
385 1.45952014918294e-07
386 1.44905897059289e-07
387 1.45546835028654e-07
388 1.44647685829113e-07
389 1.44327657380927e-07
390 1.44518281786077e-07
391 1.43281496889358e-07
392 1.43954878240038e-07
393 1.42787300205782e-07
394 1.43301775779037e-07
395 1.42025157856551e-07
396 1.4353140898038e-07
397 1.42818819881541e-07
398 1.43927039175651e-07
399 1.42141843184618e-07
400 1.43410431974189e-07
401 1.43064895041789e-07
402 1.4307556739368e-07
403 1.42595027341486e-07
404 1.42467669661528e-07
405 1.42391158419741e-07
406 1.42022685167831e-07
407 1.41918292229093e-07
408 1.40487941280298e-07
409 1.39819391620222e-07
410 1.41251121021924e-07
411 1.40688896976826e-07
412 1.39006445465384e-07
413 1.39129994636278e-07
414 1.40212080168567e-07
415 1.36741945766516e-07
416 1.39667761800411e-07
417 1.39285461386862e-07
418 1.37760409302246e-07
419 1.39002580112901e-07
420 1.38735245513999e-07
421 1.36103139425359e-07
422 1.36197911615454e-07
423 1.35827832536961e-07
424 1.35812854296091e-07
425 1.36850076160044e-07
426 1.33483140984936e-07
427 1.35047500293695e-07
428 1.33266183865999e-07
429 1.34528079342999e-07
430 1.34424297471014e-07
431 1.34505896198789e-07
432 1.32755289428133e-07
433 1.32935198848827e-07
434 1.32636003513653e-07
435 1.32464222701856e-07
436 1.32252168327796e-07
437 1.32663075191886e-07
438 1.3186945579946e-07
439 1.32829853782823e-07
440 1.32471029701264e-07
441 1.32413717324198e-07
442 1.32171635414124e-07
443 1.30773585738098e-07
444 1.32014591258667e-07
445 1.31478927301032e-07
446 1.30788080809907e-07
447 1.31026396843481e-07
448 1.29889087929769e-07
449 1.3020539313402e-07
450 1.28585696757e-07
451 1.29657735215005e-07
452 1.28757491779652e-07
453 1.28994770420832e-07
454 1.27710364949962e-07
455 1.28681691080601e-07
456 1.29268755699741e-07
457 1.28320081671518e-07
458 1.27915598113759e-07
459 1.27503867020096e-07
460 1.27419724549327e-07
461 1.27147927742044e-07
462 1.26984986081879e-07
463 1.2688742856426e-07
464 1.26058168348209e-07
465 1.26032276170918e-07
466 1.25491851576953e-07
467 1.25815816431896e-07
468 1.25446732113232e-07
469 1.24773450238536e-07
470 1.25542115370081e-07
471 1.24862509665036e-07
472 1.24479541341316e-07
473 1.25140928730616e-07
474 1.24322454553294e-07
475 1.24163378245612e-07
476 1.25131037975734e-07
477 1.23720766964652e-07
478 1.2429944717951e-07
479 1.23437246202229e-07
480 1.2273116567485e-07
481 1.22751416142819e-07
482 1.22717921158255e-07
483 1.22862076068486e-07
484 1.21229859928462e-07
485 1.21127811780752e-07
486 1.22018107617805e-07
487 1.21679292419685e-07
488 1.2143054561875e-07
489 1.21081967563441e-07
490 1.21583411782922e-07
491 1.22031380556109e-07
492 1.21371613204246e-07
493 1.19846490065356e-07
494 1.21119100526812e-07
495 1.19180086244342e-07
496 1.20860249808175e-07
497 1.19889293159758e-07
498 1.2013772732189e-07
499 1.19334899295609e-07
500 1.19165449063985e-07
501 1.18410532934377e-07
502 1.180951301194e-07
503 1.17972547286627e-07
504 1.17683185862916e-07
505 1.17477320316084e-07
506 1.17442034763826e-07
507 1.16940860550585e-07
508 1.16714275577579e-07
509 1.16477622214006e-07
510 1.16801949445744e-07
511 1.16051111831439e-07
512 1.1568319990829e-07
513 1.15398322009241e-07
514 1.1499295737849e-07
515 1.1491000151409e-07
516 1.14770919878993e-07
517 1.14215900737236e-07
518 1.14471383483306e-07
519 1.1358832807673e-07
520 1.14004208739971e-07
521 1.13357152997651e-07
522 1.13713532812199e-07
523 1.12814070973855e-07
524 1.15023603086684e-07
525 1.15752847307249e-07
526 1.15163651059902e-07
527 1.15262018596241e-07
528 1.14967647846242e-07
529 1.15318293580913e-07
530 1.15495510044639e-07
531 1.14383006177832e-07
532 1.14183642097032e-07
533 1.14031919906665e-07
534 1.14324976152602e-07
535 1.13615527652655e-07
536 1.13353870290211e-07
537 1.13831553960608e-07
538 1.13901172937858e-07
539 1.12589269463115e-07
540 1.12782302608139e-07
541 1.12491015613614e-07
542 1.123787072288e-07
543 1.11921316658936e-07
544 1.11053225282376e-07
545 1.10887519610969e-07
546 1.1096472007921e-07
547 1.10714502454812e-07
548 1.11888994069886e-07
549 1.10762684357724e-07
550 1.10518946883076e-07
551 1.10257616370291e-07
552 1.10012315701624e-07
553 1.1008772560217e-07
554 1.09865226249894e-07
555 1.09705126760673e-07
556 1.09410592585846e-07
557 1.09691839611514e-07
558 1.0990624588203e-07
559 1.09361145916864e-07
560 1.08581581059752e-07
561 1.09027389783023e-07
562 1.08888293937071e-07
563 1.07781211511337e-07
564 1.0742743938863e-07
565 1.0715122300553e-07
566 1.0694756014118e-07
567 1.07223904421971e-07
568 1.06496294449698e-07
569 1.06202925564958e-07
570 1.05949574447095e-07
571 1.0583379861373e-07
572 1.06306210057028e-07
573 1.0592847843327e-07
574 1.04710515813622e-07
575 1.04511030940557e-07
576 1.05114828841124e-07
577 1.0428571073362e-07
578 1.04516416854494e-07
579 1.04407973822163e-07
580 1.05005710793193e-07
581 1.04198591088789e-07
582 1.03813974305922e-07
583 1.04040481119227e-07
584 1.03211746704801e-07
585 1.0239071457363e-07
586 1.03190004097087e-07
587 1.02605746121753e-07
588 1.0183285326093e-07
589 1.01803628638208e-07
590 1.01893029125222e-07
591 1.02180095495896e-07
592 1.01588042866751e-07
593 1.01572005917205e-07
594 1.00972464167626e-07
595 1.0141245354589e-07
596 1.01150369857805e-07
597 1.00445738837607e-07
598 1.0037498299198e-07
599 1.00255753920919e-07
600 1.00587968177024e-07
601 1.00337679498352e-07
602 1.00176293926779e-07
603 9.95488491639662e-08
604 9.9919851948016e-08
605 9.96936506680868e-08
606 9.90534516631669e-08
607 9.91969031360895e-08
608 9.87872539326418e-08
609 9.87101458349571e-08
610 9.84698473871504e-08
611 9.84116184099548e-08
612 9.84964074746131e-08
613 9.83841843549271e-08
614 9.81674617150929e-08
615 9.80444383458234e-08
616 9.81222214591071e-08
617 9.88388322298306e-08
618 9.86881616427127e-08
619 9.77993011019862e-08
620 9.7857416392344e-08
621 9.76255023488193e-08
622 9.763507335947e-08
623 9.71393276927301e-08
624 9.77487744080463e-08
625 9.7600441506529e-08
626 9.72650440189682e-08
627 9.72071916294226e-08
628 9.69632978353729e-08
629 9.65685842402308e-08
630 9.6447301700664e-08
631 9.64904671718614e-08
632 9.65869233482408e-08
633 9.62086517120042e-08
634 9.59518331455911e-08
635 9.6272309235701e-08
636 9.59317887350153e-08
637 9.61755475259451e-08
638 9.61911155172857e-08
639 9.61439425850585e-08
640 9.61270529842295e-08
641 9.54077634673922e-08
642 9.52180414515169e-08
643 9.5175501257927e-08
644 9.57136236934275e-08
645 9.49956202589419e-08
646 9.48355562968572e-08
647 9.51779242086559e-08
648 9.46650118294201e-08
649 9.45477438563103e-08
650 9.48634379938085e-08
651 9.44344762388027e-08
652 9.42236937362395e-08
653 9.41765350148671e-08
654 9.40444451202893e-08
655 9.39762756502205e-08
656 9.39155313517404e-08
657 9.37638802156471e-08
658 9.35729218554116e-08
659 9.34916002393038e-08
660 9.3887905450174e-08
661 9.37624804464576e-08
662 9.3730200489972e-08
663 9.37949593549092e-08
664 9.35854842509798e-08
665 9.3556558056207e-08
666 9.34055961465674e-08
667 9.27818248896983e-08
668 9.31279302562871e-08
669 9.30554762135216e-08
670 9.37155135716239e-08
671 9.2868013723546e-08
672 9.32880936943548e-08
673 9.27463190691924e-08
674 9.3063214023914e-08
675 9.312391568983e-08
676 9.26251431110359e-08
677 9.3016659263867e-08
678 9.29059993381998e-08
679 9.28224253016197e-08
680 9.3208221585428e-08
681 9.31647150537174e-08
682 9.29717955955311e-08
683 9.29988814846183e-08
684 9.29337247157491e-08
685 9.29273156202726e-08
686 9.2814126162466e-08
687 9.27718701859703e-08
688 9.26188974403885e-08
689 9.25086354186533e-08
690 9.24717298289579e-08
691 9.23266938457346e-08
692 9.22844876072304e-08
693 9.23125256235835e-08
694 9.21466067893562e-08
695 9.20349805255682e-08
696 9.19980820413002e-08
697 9.18716835940359e-08
698 9.18110458769661e-08
699 9.16944955520194e-08
700 9.15789897248942e-08
701 9.15819171609655e-08
702 9.14981725941288e-08
703 9.148289592531e-08
704 9.17273581535483e-08
705 9.13073776587225e-08
706 9.12123994112335e-08
707 9.10928008579504e-08
708 9.09637947188457e-08
709 9.12307172029614e-08
710 9.07690065332645e-08
711 9.0993744095158e-08
712 9.10522928165847e-08
713 9.11576023554517e-08
714 9.1054843665006e-08
715 9.10641091422804e-08
716 9.10124313691085e-08
717 9.07044821474301e-08
718 9.04714241301008e-08
719 9.03343746472274e-08
720 9.19741012239683e-08
721 9.18932414606388e-08
722 9.16986095944594e-08
723 9.166814152195e-08
724 9.15739377660429e-08
725 9.16002846906849e-08
726 9.15495306230696e-08
727 9.14570179588736e-08
728 9.12780606654451e-08
729 9.12415742959638e-08
730 9.11410680259905e-08
731 9.107747445114e-08
732 9.09938719928505e-08
733 9.09213824229482e-08
734 9.08744510752513e-08
735 9.0595932533688e-08
736 9.06533230704554e-08
737 9.05752557400774e-08
738 9.04518699940127e-08
739 9.04442103433212e-08
740 9.03642458638387e-08
741 9.01895020888333e-08
742 9.02410661751674e-08
743 9.01178367485045e-08
744 8.99945860055595e-08
745 8.99018104405513e-08
746 8.98242689117978e-08
747 8.97518717124512e-08
748 8.95317242566307e-08
749 8.95392560096298e-08
750 8.94510776561219e-08
751 8.92577816102857e-08
752 8.92729374868395e-08
753 8.91726301688323e-08
754 8.89962734618166e-08
755 8.90389912910905e-08
756 8.89592541852835e-08
757 8.88739961624196e-08
758 8.86920616949283e-08
759 8.87090791934497e-08
760 8.86403768163291e-08
761 8.85736852751506e-08
762 8.8370683215544e-08
763 8.84094575326344e-08
764 8.82050414929836e-08
765 8.822917152429e-08
766 8.81637376437538e-08
767 8.79598829328643e-08
768 8.7985092989129e-08
769 8.78592629760533e-08
770 8.77342074545595e-08
771 8.77082797501316e-08
772 8.75181029869054e-08
773 8.73216166041857e-08
774 8.72369838589293e-08
775 8.70174474698615e-08
776 8.70731824420545e-08
777 8.69514877877009e-08
778 8.68223253291944e-08
779 8.6772523388845e-08
780 8.67425242745412e-08
781 8.65997265009355e-08
782 8.66059224335913e-08
783 8.63784421767377e-08
784 8.64191136429326e-08
785 8.62512905541735e-08
786 8.61030642340666e-08
787 8.62066329432309e-08
788 8.59225011140552e-08
789 8.60218136722324e-08
790 8.59036148881387e-08
791 8.57071711379831e-08
792 8.57965360978596e-08
793 8.56659809755911e-08
794 8.54809698580539e-08
795 8.53783035381639e-08
796 8.52959587405167e-08
797 8.51817461011706e-08
798 8.51323562756079e-08
799 8.50127506168974e-08
800 8.4942563205459e-08
801 8.4852672443958e-08
802 8.47547454441155e-08
803 8.4693873247943e-08
804 8.46260164166779e-08
805 8.49131467361985e-08
806 8.46169925239337e-08
807 8.42057090721937e-08
808 8.41684553165578e-08
809 8.41513596583354e-08
810 8.39890716974878e-08
811 8.39450464695801e-08
812 8.38554683468828e-08
813 8.36796374414916e-08
814 8.3747856649552e-08
815 8.35931004417034e-08
816 8.34525195614333e-08
817 8.33853022186304e-08
818 8.32709261544551e-08
819 8.30517521421825e-08
820 8.29501800581056e-08
821 8.28419146614578e-08
822 8.26996782166134e-08
823 8.28519191031774e-08
824 8.26705601753019e-08
825 8.2735375883658e-08
826 8.26648403062791e-08
827 8.25374613100394e-08
828 8.24598771487217e-08
829 8.24080501615754e-08
830 8.24001489263537e-08
831 8.21785022253607e-08
832 8.21467267542175e-08
833 8.20471726115102e-08
834 8.22272383516065e-08
835 8.22078831674844e-08
836 8.21473378209703e-08
837 8.19668173335231e-08
838 8.20497376707863e-08
839 8.17884497905652e-08
840 8.18327663409946e-08
841 8.16340275378025e-08
842 8.16680127968539e-08
843 8.15607847926003e-08
844 8.14940932514219e-08
845 8.13974381230764e-08
846 8.12694764817934e-08
847 8.1279601715778e-08
848 8.12268297067931e-08
849 8.09078173347189e-08
850 8.09869789009099e-08
851 8.10059859190915e-08
852 8.09256732736685e-08
853 8.08675935104475e-08
854 8.07975908401204e-08
855 8.0721939355044e-08
856 8.06011897225289e-08
857 8.05183404395393e-08
858 8.04517128472071e-08
859 8.04024367084821e-08
860 8.03705688667833e-08
861 8.03181450237389e-08
862 8.0244390687767e-08
863 8.01186530452469e-08
864 7.99803459017312e-08
865 7.99372230630979e-08
866 8.01482897827555e-08
867 8.01226747171313e-08
868 8.00056270122695e-08
869 7.97196193502714e-08
870 7.9629920435309e-08
871 7.97768180405001e-08
872 7.95133345832255e-08
873 7.93907730667343e-08
874 7.93382639585616e-08
875 7.95593919633575e-08
876 7.94931338532479e-08
877 7.92404719618389e-08
878 7.93931604903264e-08
879 7.92660515003263e-08
880 7.92306309449486e-08
881 7.89329988037935e-08
882 7.88009870689166e-08
883 7.87356171372267e-08
884 7.85968055083686e-08
885 7.85437350714346e-08
886 7.88049305811001e-08
887 7.85790277291198e-08
888 7.86838114663624e-08
889 7.86372069683239e-08
890 7.85737555020205e-08
891 7.85313218898409e-08
892 7.82085933792587e-08
893 7.8100747202825e-08
894 7.8325314234462e-08
895 7.80546045575647e-08
896 7.79553701590885e-08
897 7.78760096409314e-08
898 7.7823465005622e-08
899 7.77330484424965e-08
900 7.76521460466029e-08
901 7.76210171693492e-08
902 7.75093127458604e-08
903 7.7760120120729e-08
904 7.74422161953225e-08
905 7.7429568534626e-08
906 7.73689308175562e-08
907 7.75328459212687e-08
908 7.72031114593119e-08
909 7.71712365121857e-08
910 7.7090888339626e-08
911 7.70221859625053e-08
912 7.69851382642628e-08
913 7.69372405784452e-08
914 7.68606724932397e-08
915 7.68029053688224e-08
916 7.67617436281398e-08
917 7.66817791486574e-08
918 7.6627010514585e-08
919 7.66648753369736e-08
920 7.65236336519592e-08
921 7.64735403890882e-08
922 7.63973346806779e-08
923 7.63584537821771e-08
924 7.62985266078431e-08
925 7.62597380798979e-08
926 7.61441043550803e-08
927 7.61281739869446e-08
928 7.60610490146973e-08
929 7.59984644105316e-08
930 7.57786295935148e-08
931 7.58999831873552e-08
932 7.57852376409573e-08
933 7.57242730742291e-08
934 7.56539577650983e-08
935 7.56245199795558e-08
936 7.55625393367154e-08
937 7.54996563046006e-08
938 7.55747961989073e-08
939 7.53958318000514e-08
940 7.53303410760964e-08
941 7.52719415686443e-08
942 7.52137836457223e-08
943 7.51971427348508e-08
944 7.51557251987833e-08
945 7.51047082303558e-08
946 7.505871479907e-08
947 7.49893658280598e-08
948 7.4921793213889e-08
949 7.49031840996395e-08
950 7.48477546608228e-08
951 7.48028199382134e-08
952 7.47367749909245e-08
953 7.46955208796862e-08
954 7.46331565437686e-08
955 7.4568369257122e-08
956 7.45865520457301e-08
957 7.44951051956377e-08
958 7.44624770732116e-08
959 7.43919414958327e-08
960 7.43190824437079e-08
961 7.42655430485684e-08
962 7.41985672902956e-08
963 7.41302983442438e-08
964 7.40796508580388e-08
965 7.410029922994e-08
966 7.40223526918271e-08
967 7.39472199029478e-08
968 7.38579331027722e-08
969 7.38288292723155e-08
970 7.36825924718687e-08
971 7.35752436753501e-08
972 7.36332523842975e-08
973 7.35912308869047e-08
974 7.35768139747961e-08
975 7.34492004994536e-08
976 7.33664791141564e-08
977 7.32670812908509e-08
978 7.31749523197323e-08
979 7.31302023382341e-08
980 7.3047026205586e-08
981 7.29913409713845e-08
982 7.29135365418188e-08
983 7.28613755995866e-08
984 7.27908329167803e-08
985 7.27494295915676e-08
986 7.27130924360608e-08
987 7.26111011317698e-08
988 7.2595348399318e-08
989 7.251004774389e-08
990 7.24373805383038e-08
991 7.23837274563266e-08
992 7.23798549984167e-08
993 7.23938029523197e-08
994 7.23919626466341e-08
995 7.23767783483709e-08
996 7.23341884167894e-08
997 7.24543198771244e-08
998 7.24115665207137e-08
999 7.22090334193126e-08
1000 7.23518667200551e-08
1001 7.22980857403854e-08
1002 7.22374409178883e-08
1003 7.22572295330792e-08
1004 7.21446724583075e-08
1005 7.20112396379591e-08
1006 7.1936469225875e-08
1007 7.19383450586975e-08
1008 7.1907379606273e-08
1009 7.17530568294933e-08
1010 7.18489943096756e-08
1011 7.18320478654277e-08
1012 7.17786008408439e-08
1013 7.17686603479706e-08
1014 7.16610273343576e-08
1015 7.15798762485065e-08
1016 7.15866832479151e-08
1017 7.15303016818325e-08
1018 7.14549415192778e-08
1019 7.13082428660528e-08
1020 7.14621606334731e-08
1021 7.14446173333272e-08
1022 7.13772436711224e-08
1023 7.13303478505622e-08
1024 7.12881771391949e-08
1025 7.13217858105963e-08
1026 7.11779151174596e-08
1027 7.11465446556758e-08
1028 7.10838961026639e-08
1029 7.09787641994808e-08
1030 7.10057008745935e-08
1031 7.10625727151637e-08
1032 7.0923789508015e-08
1033 7.09066725335106e-08
1034 7.0927484330241e-08
1035 7.07806222521867e-08
1036 7.0751916325662e-08
1037 7.07767995322683e-08
1038 7.06562843788561e-08
1039 7.06090830249195e-08
1040 7.05783449461705e-08
1041 7.05487792629356e-08
1042 7.04993112776719e-08
1043 7.04589666611355e-08
1044 7.04121561057036e-08
1045 7.02110156680646e-08
1046 7.05137139789258e-08
1047 7.04518328120685e-08
1048 7.02970694987926e-08
1049 7.03792935041747e-08
1050 7.03114082512002e-08
1051 7.03416844771709e-08
1052 7.01915752188143e-08
1053 7.02066174085303e-08
1054 7.00436828537931e-08
1055 7.00567568401311e-08
1056 7.00909268402938e-08
1057 6.99871449683087e-08
1058 6.99824482808253e-08
1059 6.98941633459071e-08
1060 6.97781175063028e-08
1061 6.98487170325279e-08
1062 6.96578226211386e-08
1063 6.96695039437145e-08
1064 6.96803965638537e-08
1065 6.96072106620704e-08
1066 6.94692730007773e-08
1067 6.94780624144187e-08
1068 6.95008424145271e-08
1069 6.94155772862359e-08
1070 6.92922412781627e-08
1071 6.93823309916297e-08
1072 6.91703618826978e-08
1073 6.91812331865549e-08
1074 6.90660613145155e-08
1075 6.91583679213181e-08
1076 6.89962647015818e-08
1077 6.89594372715874e-08
1078 6.88128878323369e-08
1079 6.89387817942588e-08
1080 6.87813681565785e-08
1081 6.87082675199235e-08
1082 6.85921150989088e-08
1083 6.85446082115959e-08
1084 6.84449119603414e-08
1085 6.86310883679653e-08
1086 6.84872389911106e-08
1087 6.84104648485118e-08
1088 6.8582004075779e-08
1089 6.8541510245268e-08
1090 6.82991014855361e-08
1091 6.81916105804703e-08
1092 6.81282941172867e-08
1093 6.8073340742103e-08
1094 6.80394549590346e-08
1095 6.80092284710554e-08
1096 6.80876794945107e-08
1097 6.81939056335068e-08
1098 6.78999327874408e-08
1099 6.77804052884312e-08
1100 6.7754491794858e-08
1101 6.7736607434199e-08
1102 6.760944160078e-08
1103 6.75863276455857e-08
1104 6.76776110708488e-08
1105 6.78155558375693e-08
1106 6.75056313070854e-08
1107 6.73219417990367e-08
1108 6.73366855608037e-08
1109 6.72795579248486e-08
1110 6.71961970510893e-08
1111 6.71770621352152e-08
1112 6.70960034199197e-08
1113 6.73048461408143e-08
1114 6.70161810489844e-08
1115 6.69422064447645e-08
1116 6.68944366566393e-08
1117 6.69163569000375e-08
1118 6.67933193199133e-08
1119 6.69058337621209e-08
1120 6.67087576289305e-08
1121 6.65806041411088e-08
1122 6.65219701545539e-08
1123 6.65626700424582e-08
1124 6.62479848756448e-08
1125 6.65729089632805e-08
1126 6.6478179405749e-08
1127 6.6317355162937e-08
1128 6.62779697790938e-08
1129 6.6278808219522e-08
1130 6.62413768282022e-08
1131 6.61411974078874e-08
1132 6.61464483187046e-08
1133 6.60645440575536e-08
1134 6.58984262713602e-08
1135 6.58725340940691e-08
1136 6.58420802324144e-08
1137 6.59459757912373e-08
1138 6.57225029954134e-08
1139 6.57619310118207e-08
1140 6.57170815543395e-08
1141 6.56298340118155e-08
1142 6.5633976475965e-08
1143 6.55913083846826e-08
1144 6.56197727266772e-08
1145 6.55811547289886e-08
1146 6.54622454021592e-08
1147 6.53896421454192e-08
1148 6.52806235734715e-08
1149 6.52657519140121e-08
1150 6.52254783517492e-08
1151 6.5256472225883e-08
1152 6.51273097673766e-08
1153 6.51578631050143e-08
1154 6.51047002975247e-08
1155 6.50733014140314e-08
1156 6.50871854190882e-08
1157 6.50163229920508e-08
1158 6.49820890430419e-08
1159 6.49113687245517e-08
1160 6.48273399406207e-08
1161 6.48570477324029e-08
1162 6.47458122671196e-08
1163 6.47414637455768e-08
1164 6.4675958810767e-08
1165 6.46260573944346e-08
1166 6.45780176000699e-08
1167 6.49794529294923e-08
1168 6.492842175021e-08
1169 6.49151346010512e-08
1170 6.48159286242844e-08
1171 6.48684377324571e-08
1172 6.47705107326146e-08
1173 6.48362572519545e-08
1174 6.47473541448562e-08
1175 6.46969624540361e-08
1176 6.446212097444e-08
1177 6.463389468081e-08
1178 6.44763886725741e-08
1179 6.46696562967009e-08
1180 6.44001758587365e-08
1181 6.45447073566174e-08
1182 6.42184900812026e-08
1183 6.4414571454563e-08
1184 6.44256346049588e-08
1185 6.43564845859146e-08
1186 6.42795470184865e-08
1187 6.43395665633761e-08
1188 6.41523243416486e-08
1189 6.42257731442442e-08
1190 6.41983959326353e-08
1191 6.42218935809069e-08
1192 6.41835313786032e-08
1193 6.40366266679848e-08
1194 6.39554400549969e-08
1195 6.38808401731694e-08
1196 6.39260591128732e-08
1197 6.39405683955374e-08
1198 6.39209929431672e-08
1199 6.36898107586603e-08
1200 6.38162802601983e-08
1201 6.35539905147198e-08
1202 6.37026786876049e-08
1203 6.35051762287731e-08
1204 6.36511288121255e-08
1205 6.33242507319665e-08
1206 6.34528944942758e-08
1207 6.32021439628261e-08
1208 6.34185610692839e-08
1209 6.34441335023439e-08
1210 6.30903542742089e-08
1211 6.30423144798442e-08
1212 6.3067048472476e-08
1213 6.31093612923905e-08
1214 6.30505851972885e-08
1215 6.28712086836458e-08
1216 6.29277039365661e-08
1217 6.29087608672307e-08
1218 6.29956105058227e-08
1219 6.30067091833553e-08
1220 6.29360528137113e-08
1221 6.27299669986314e-08
1222 6.28891072551596e-08
1223 6.27352889637223e-08
1224 6.27675760256352e-08
1225 6.2694134328467e-08
1226 6.27901002303588e-08
1227 6.27606837610983e-08
1228 6.27514040729693e-08
1229 6.25813356691651e-08
1230 6.25584348767916e-08
1231 6.25954541533247e-08
1232 6.2642691034398e-08
1233 6.25607796678196e-08
1234 6.25708125312485e-08
1235 6.26807477033253e-08
1236 6.25537524001629e-08
1237 6.26747720389176e-08
1238 6.26198470854433e-08
1239 6.25355198735633e-08
1240 6.25729441594558e-08
1241 6.25208542714972e-08
1242 6.25226590500461e-08
1243 6.25636999984636e-08
1244 6.25045473157115e-08
1245 6.25272065235549e-08
1246 6.24286684569597e-08
1247 6.24929583636913e-08
1248 6.24514200353588e-08
1249 6.27279561626892e-08
1250 6.25480538474221e-08
1251 6.24387936909443e-08
1252 6.24564790996374e-08
1253 6.23543954247907e-08
1254 6.23616855932596e-08
1255 6.23657925302723e-08
1256 6.23425009393941e-08
1257 6.24577154439976e-08
1258 6.22679721118402e-08
1259 6.22051032905802e-08
1260 6.21010798340649e-08
1261 6.20052915678571e-08
1262 6.21181683868599e-08
1263 6.20848865651169e-08
1264 6.20121909378213e-08
1265 6.19694944248295e-08
1266 6.19347915176149e-08
1267 6.19018010183936e-08
1268 6.18654993900236e-08
1269 6.1248513816281e-08
1270 6.10674959489188e-08
1271 6.10445027859896e-08
1272 6.09816055430201e-08
1273 6.08969799031911e-08
1274 6.0882115349159e-08
1275 6.08957151371214e-08
1276 6.07955001896698e-08
1277 6.07990386924939e-08
1278 6.08232681997833e-08
1279 6.06944325909353e-08
1280 6.06806409564342e-08
1281 6.07107040195842e-08
1282 6.0623293052231e-08
1283 6.03967293955066e-08
1284 6.04791452474274e-08
1285 6.03300094326187e-08
1286 6.05319385726943e-08
1287 6.08220034337137e-08
1288 6.05102599138263e-08
1289 6.03914003249884e-08
1290 6.03763226081355e-08
1291 6.00698655262022e-08
1292 5.98849609900753e-08
1293 5.97453748696353e-08
1294 5.95621756360742e-08
1295 5.94141731369291e-08
1296 5.96678688680186e-08
1297 5.96756279946931e-08
1298 5.96552851561682e-08
1299 5.9726950496497e-08
1300 5.98052807276872e-08
1301 5.96957931975339e-08
1302 5.9868860091683e-08
1303 5.95529741076462e-08
1304 5.96184790424559e-08
1305 5.9627680570884e-08
1306 5.95811044945549e-08
1307 5.97641403032867e-08
1308 5.9523653561655e-08
1309 5.97050089368167e-08
1310 5.97032610016868e-08
1311 5.95691957983036e-08
1312 5.95662577040912e-08
1313 5.94355498151344e-08
1314 5.94963474043197e-08
1315 5.96042788458817e-08
1316 5.95949742887569e-08
1317 5.95423479410329e-08
1318 5.96746829728545e-08
1319 5.95083484711267e-08
1320 5.96805662667066e-08
1321 5.94941731435483e-08
1322 5.94442752799296e-08
1323 5.94929616681839e-08
1324 5.92524926901206e-08
1325 5.94209446092009e-08
1326 5.92826658873946e-08
1327 5.95548286241865e-08
1328 5.91725139997834e-08
1329 5.91941784477967e-08
1330 5.91562994145534e-08
1331 5.92170472657472e-08
1332 5.90742388340004e-08
1333 5.89714268528496e-08
1334 5.8914238820762e-08
1335 5.90800546262926e-08
1336 5.85639803318827e-08
1337 5.85584807311079e-08
1338 5.8436203431711e-08
1339 5.85032573496846e-08
1340 5.86071777775032e-08
1341 5.84003814196876e-08
1342 5.81977310787352e-08
1343 5.87018149644791e-08
1344 5.86701638383147e-08
1345 5.83418398036883e-08
1346 5.81578447622633e-08
1347 5.83927359798508e-08
1348 5.83359174299858e-08
1349 5.8137672453995e-08
1350 5.81997134929679e-08
1351 5.81865435833606e-08
1352 5.81160541912595e-08
1353 5.78192178579684e-08
1354 5.79643533171748e-08
1355 5.80142938133577e-08
1356 5.79669396927329e-08
1357 5.75113077161404e-08
1358 5.75865577445711e-08
1359 5.7360299621223e-08
1360 5.73901068889882e-08
1361 5.7261924979457e-08
1362 5.72037031076889e-08
1363 5.71770399915295e-08
1364 5.71056411047266e-08
1365 5.71405855964713e-08
1366 5.70975799973894e-08
1367 5.70094442764457e-08
1368 5.69168605579762e-08
1369 5.69268472361273e-08
1370 5.68234739262152e-08
1371 5.68368143660791e-08
1372 5.67141178464681e-08
1373 5.66763382892077e-08
1374 5.67067139911615e-08
1375 5.65594824308846e-08
1376 5.65569422406043e-08
1377 5.64164217564667e-08
1378 5.65022766352286e-08
1379 5.63975142142681e-08
1380 5.61994077941108e-08
1381 5.6268170567364e-08
1382 5.6248161683925e-08
1383 5.61447563995898e-08
1384 5.61784858632564e-08
1385 5.60093447177223e-08
1386 5.60352333422998e-08
1387 5.59021025026141e-08
1388 5.58273001161069e-08
1389 5.59997097582254e-08
1390 5.58535084849154e-08
1391 5.58132562389346e-08
1392 5.58372228454118e-08
1393 5.55883516994982e-08
1394 5.56492736336622e-08
1395 5.56007151431004e-08
1396 5.55180577066494e-08
1397 5.54852626066804e-08
1398 5.5426003342518e-08
1399 5.56030563814147e-08
1400 5.54828076815284e-08
1401 5.56942900686863e-08
1402 5.54021930554427e-08
1403 5.55414949587885e-08
1404 5.53532082392394e-08
1405 5.58691226615338e-08
1406 5.5805077892046e-08
1407 5.57863977235229e-08
1408 5.5592888514866e-08
1409 5.56322063971493e-08
1410 5.56339863067024e-08
1411 5.55867849527658e-08
1412 5.56732793199899e-08
1413 5.55920571798652e-08
1414 5.55305348370894e-08
1415 5.53660797208977e-08
1416 5.53569670103116e-08
1417 5.53132437630666e-08
1418 5.54542118891277e-08
1419 5.53104939626792e-08
1420 5.52541834508702e-08
1421 5.51580932039997e-08
1422 5.51557661765401e-08
1423 5.47215535107171e-08
1424 5.5432817447354e-08
1425 5.47133716111148e-08
1426 5.5499601359088e-08
1427 5.52837242651094e-08
1428 5.50384129383019e-08
1429 5.51019248007378e-08
1430 5.4995307863237e-08
1431 5.50081686867543e-08
1432 5.48774146125197e-08
1433 5.49656036241686e-08
1434 5.44391269841071e-08
1435 5.48578960035684e-08
1436 5.48457066429364e-08
1437 5.48287495405475e-08
1438 5.42124496405449e-08
1439 5.49024221641048e-08
1440 5.46825482672375e-08
1441 5.45296003906515e-08
1442 5.46051417416038e-08
1443 5.40306928087375e-08
1444 5.45759988312966e-08
1445 5.43762794791292e-08
1446 5.42803988423657e-08
1447 5.42557785365716e-08
1448 5.38155759954861e-08
1449 5.42643263656828e-08
1450 5.40857314490495e-08
1451 5.39824505096931e-08
1452 5.3565969437841e-08
1453 5.40319291530977e-08
1454 5.40255200576212e-08
1455 5.40463709342021e-08
1456 5.38185815912584e-08
1457 5.37048130411222e-08
1458 5.38247988401963e-08
1459 5.36032089826222e-08
1460 5.3473147687555e-08
1461 5.36861080036033e-08
1462 5.3434899172089e-08
1463 5.33215143150301e-08
1464 5.3872152960821e-08
1465 5.3581409531489e-08
1466 5.36282911411945e-08
1467 5.35902273668398e-08
1468 5.34323127965308e-08
1469 5.35012176783312e-08
1470 5.34836566146168e-08
1471 5.32414752285604e-08
1472 5.28796455512293e-08
1473 5.30714672208887e-08
1474 5.31604342768333e-08
1475 5.31089483502001e-08
1476 5.29726307263445e-08
1477 5.31073638398993e-08
1478 5.31103943046674e-08
1479 5.29207966337708e-08
1480 5.21250775875615e-08
1481 5.29498969115139e-08
1482 5.28734389604324e-08
1483 5.27891295121208e-08
1484 5.29587289577194e-08
1485 5.2682921136693e-08
1486 5.21704244249577e-08
1487 5.20705576434466e-08
1488 5.25901313608301e-08
1489 5.25341228296838e-08
1490 5.26705967729413e-08
1491 5.26209902318442e-08
1492 5.19724423497792e-08
1493 5.25180290367189e-08
1494 5.2690133145461e-08
1495 5.26919592402919e-08
1496 5.24606598162336e-08
1497 5.17638092389916e-08
1498 5.24075218777398e-08
1499 5.22736023356174e-08
1500 5.25535952533573e-08
1501 5.2311545317707e-08
1502 5.23688221676366e-08
1503 5.22279819392679e-08
1504 5.21622673943511e-08
1505 5.15841236392589e-08
1506 5.20534157999464e-08
1507 5.20202867448916e-08
1508 5.19339700133514e-08
1509 5.13830649140345e-08
1510 5.19658662767597e-08
1511 5.18396383597519e-08
1512 5.18491241052743e-08
1513 5.18547977890194e-08
1514 5.17332097160761e-08
1515 5.1992905980569e-08
1516 5.16887226353901e-08
1517 5.16697227226359e-08
1518 5.09106818924465e-08
1519 5.07078574685238e-08
1520 5.07832247365059e-08
1521 5.14077314051065e-08
1522 5.17486107298737e-08
1523 5.13302786941949e-08
1524 5.13919928835094e-08
1525 5.12669409147293e-08
1526 5.15161460157287e-08
1527 5.11879498787948e-08
1528 5.00678360992879e-08
1529 5.03542771923549e-08
1530 5.1082803764757e-08
1531 5.00499020006373e-08
1532 4.95226011310024e-08
1533 5.0972506215885e-08
1534 5.01855161871845e-08
1535 5.02337691443699e-08
1536 5.07404038785353e-08
1537 4.97283707545648e-08
1538 5.09945827786851e-08
1539 5.10549575949426e-08
1540 4.98331331755253e-08
1541 5.06150037438147e-08
1542 4.96800751648152e-08
1543 5.08043740410358e-08
1544 4.96492873480747e-08
1545 4.97041163782797e-08
1546 4.98267489490445e-08
1547 5.03251662564708e-08
1548 4.97177339298105e-08
1549 4.9546883928997e-08
1550 5.05959931729194e-08
1551 4.88294453759863e-08
1552 4.94835461495313e-08
1553 4.923312246774e-08
1554 5.04347248408976e-08
1555 4.918646823171e-08
1556 4.94863598987649e-08
1557 4.95202101546965e-08
1558 4.94621090751934e-08
1559 4.94497029990271e-08
1560 4.93774017229498e-08
1561 4.93330318818153e-08
1562 4.9345157293601e-08
1563 4.92926481854283e-08
1564 4.91356324516801e-08
1565 4.93440666105016e-08
1566 4.90265605890272e-08
1567 4.89541562842533e-08
1568 4.89246865242876e-08
1569 4.89173110906904e-08
1570 4.889855631518e-08
1571 4.88728595371413e-08
1572 4.8880870906487e-08
1573 4.88662443842713e-08
1574 4.88549005694949e-08
1575 4.87334403942441e-08
1576 4.87247220348763e-08
1577 4.87972116047786e-08
1578 4.88559628308849e-08
1579 4.87866600451525e-08
1580 4.8765230076242e-08
1581 4.87659121972683e-08
1582 4.86101150443119e-08
1583 4.85765063729104e-08
1584 4.85034448161059e-08
1585 4.83569628784153e-08
1586 4.82763802267527e-08
1587 4.85585935905419e-08
1588 4.83932396377895e-08
1589 4.81291486664759e-08
1590 4.81124473594718e-08
1591 4.82363446963063e-08
1592 4.78353250343844e-08
1593 4.83247113436391e-08
1594 4.81956412556883e-08
1595 4.82625246434054e-08
1596 4.83353765901029e-08
1597 4.80423238968797e-08
1598 4.79996096203195e-08
1599 4.80059618723772e-08
1600 4.77838426604649e-08
1601 4.79962238841836e-08
1602 4.78438941797776e-08
1603 4.77827519773655e-08
1604 4.76898343038101e-08
1605 4.76809702831815e-08
1606 4.75344990036319e-08
1607 4.7361663035872e-08
1608 4.74496708591232e-08
1609 4.74806753913981e-08
1610 4.74862389410191e-08
1611 4.74230645863827e-08
1612 4.75556696244439e-08
1613 4.74164814079359e-08
1614 4.72595154121791e-08
1615 4.70139660535551e-08
1616 4.70017873510642e-08
1617 4.69584371387555e-08
1618 4.70070204983131e-08
1619 4.70725716184006e-08
1620 4.7173784878396e-08
1621 4.68769769668143e-08
1622 4.67798173531264e-08
1623 4.68821035326528e-08
1624 4.66301983692574e-08
1625 4.67456864328142e-08
1626 4.63521594440408e-08
1627 4.66089673523129e-08
1628 4.6465519432104e-08
1629 4.66098981632967e-08
1630 4.63544651552183e-08
1631 4.64356730844884e-08
1632 4.63530369643195e-08
1633 4.62648515053843e-08
1634 4.62045868232508e-08
1635 4.61236950854982e-08
1636 4.62055567140851e-08
1637 4.61401974405362e-08
1638 4.6052299751409e-08
1639 4.6011010113034e-08
1640 4.60123743550866e-08
1641 4.59537083941086e-08
1642 4.59521807272267e-08
1643 4.5893898459326e-08
1644 4.58895392796421e-08
1645 4.57999682623722e-08
1646 4.57301396750154e-08
1647 4.57994531188888e-08
1648 4.5668237191876e-08
1649 4.56547155636144e-08
1650 4.57490862970644e-08
1651 4.57522020269607e-08
1652 4.56594584363756e-08
1653 4.54579129893773e-08
1654 4.54609008215812e-08
1655 4.55625972506368e-08
1656 4.53472743799921e-08
1657 4.55269777432932e-08
1658 4.53248851783883e-08
1659 4.54281234851805e-08
1660 4.52150601404355e-08
1661 4.5357971600879e-08
1662 4.50455388545379e-08
1663 4.49896724319387e-08
1664 4.52067254741451e-08
1665 4.52444837151234e-08
1666 4.51733761508422e-08
1667 4.50114683303582e-08
1668 4.47727259711428e-08
1669 4.47151009552726e-08
1670 4.4691105927086e-08
1671 4.48207302383707e-08
1672 4.46019328137481e-08
1673 4.477972481709e-08
1674 4.45737420307069e-08
1675 4.45438885776639e-08
1676 4.43879564215877e-08
1677 4.43489938106723e-08
1678 4.43143193251672e-08
1679 4.42866330274683e-08
1680 4.43738095157187e-08
1681 4.44078445127616e-08
1682 4.43925074478102e-08
1683 4.42910419451437e-08
1684 4.41394156780461e-08
1685 4.4319055092501e-08
1686 4.434218325855e-08
1687 4.42821743718014e-08
1688 4.40046399319272e-08
1689 4.38673595226646e-08
1690 4.41303242837421e-08
1691 4.41253860117286e-08
1692 4.38929355084383e-08
1693 4.37109477502418e-08
1694 4.40268053125692e-08
1695 4.3676667615955e-08
1696 4.36018687821615e-08
1697 4.38092691013026e-08
1698 4.35275602228558e-08
1699 4.35359091000009e-08
1700 4.34663718351658e-08
1701 4.35903046991371e-08
1702 4.36361098365978e-08
1703 4.34367777302214e-08
1704 4.33568985158672e-08
1705 4.34226983259123e-08
1706 4.32067075450959e-08
1707 4.30816449181748e-08
1708 4.320301272287e-08
1709 4.30582289823178e-08
1710 4.29961168890713e-08
1711 4.28822417575248e-08
1712 4.28686846021265e-08
1713 4.28420925402406e-08
1714 4.29513065114406e-08
1715 4.29951860780875e-08
1716 4.28630286819498e-08
1717 4.26910560236138e-08
1718 4.26960404809051e-08
1719 4.26000887898681e-08
1720 4.26456558955124e-08
1721 4.27217301535165e-08
1722 4.25279083060559e-08
1723 4.24844301960547e-08
1724 4.26136566034074e-08
1725 4.2554397339245e-08
1726 4.24407602395149e-08
1727 4.23009822725362e-08
1728 4.22349053508242e-08
1729 4.22044301728874e-08
1730 4.22198880301039e-08
1731 4.249689311564e-08
1732 4.21161914232471e-08
1733 4.21123900196108e-08
1734 4.23731663090621e-08
1735 4.20428101222114e-08
1736 4.19811421181748e-08
1737 4.22281516421208e-08
1738 4.19081302993618e-08
1739 4.18546903802053e-08
1740 4.20385326549422e-08
1741 4.18780707889255e-08
1742 4.20755377206206e-08
1743 4.21173602660474e-08
1744 4.20106651688457e-08
1745 4.15259151509417e-08
1746 4.1720802812506e-08
1747 4.18359427101223e-08
1748 4.18011971703436e-08
1749 4.13756353623285e-08
1750 4.15970049516545e-08
1751 4.16448884266174e-08
1752 4.15427656719203e-08
1753 4.15151930610591e-08
1754 4.16070804476476e-08
1755 4.16748093812203e-08
1756 4.14049345920375e-08
1757 4.13760332662605e-08
1758 4.15310843493444e-08
1759 4.14458476427626e-08
1760 4.12542995320564e-08
1761 4.12458014409367e-08
1762 4.12268938987381e-08
1763 4.10857126098563e-08
1764 4.10257392502444e-08
1765 4.09837177528516e-08
1766 4.11020160129283e-08
1767 4.07852702721812e-08
1768 4.1126618555154e-08
1769 4.08739744273134e-08
1770 4.09141662771617e-08
1771 4.11690628254746e-08
1772 4.08090343739786e-08
1773 4.07827549508966e-08
1774 4.07472207086812e-08
1775 4.05163120831276e-08
1776 4.05393194569115e-08
1777 4.05718587614956e-08
1778 4.0427821090816e-08
1779 4.04420461563859e-08
1780 4.05767863753681e-08
1781 4.02879933858458e-08
1782 4.02715691905087e-08
1783 4.03311091190517e-08
1784 3.99714501497783e-08
1785 3.99684552121471e-08
1786 3.99088335711895e-08
1787 4.00973334535593e-08
1788 3.96114394618508e-08
1789 3.9827408926385e-08
1790 3.95748109838223e-08
1791 3.96212982423094e-08
1792 3.93728143421868e-08
1793 3.92255827819099e-08
1794 3.94822841087716e-08
1795 3.92939547566584e-08
1796 3.92632593104736e-08
1797 3.94497483569012e-08
1798 3.9238976512479e-08
1799 3.9240060090151e-08
1800 3.94699242178831e-08
1801 3.91067516147814e-08
1802 3.9093361436926e-08
1803 3.90263501515165e-08
1804 3.90595502608448e-08
1805 3.88897838377034e-08
1806 3.87945924273936e-08
1807 3.90955889884026e-08
1808 3.89878884732298e-08
1809 3.87581842176132e-08
1810 3.89127414734958e-08
1811 3.90551768703062e-08
1812 3.96763013554846e-08
1813 3.95714323531138e-08
1814 3.93872987558552e-08
1815 3.92231491730399e-08
1816 3.93286399003046e-08
1817 3.94369692457985e-08
1818 3.93572392454189e-08
1819 3.92777117497189e-08
1820 3.93321499814192e-08
1821 3.90121179805192e-08
1822 3.9314134170354e-08
1823 3.92850658670341e-08
1824 3.94908461487375e-08
1825 3.92578165531177e-08
1826 3.9236432769485e-08
1827 3.91518071296559e-08
1828 3.91649699338359e-08
1829 3.9136509144555e-08
1830 3.93472951998319e-08
1831 3.87801577517166e-08
1832 3.94599517505867e-08
1833 3.90628329682841e-08
1834 3.88400103190634e-08
1835 3.90370722413991e-08
1836 3.90480074941024e-08
1837 3.87796923462247e-08
1838 3.8798336987611e-08
1839 3.88359779890379e-08
1840 3.90662862059798e-08
1841 3.8668318325108e-08
1842 3.86684106956636e-08
1843 3.89366263675583e-08
1844 3.87547380853448e-08
1845 3.87406373647536e-08
1846 3.85493095222955e-08
1847 3.85299472327461e-08
1848 3.85620992915392e-08
1849 3.84658349616984e-08
1850 3.8331528173785e-08
1851 3.83218328181556e-08
1852 3.83001221848644e-08
1853 3.82303078083623e-08
1854 3.82057159242777e-08
1855 3.81836784413281e-08
1856 3.83427796180058e-08
1857 3.82112368413345e-08
1858 3.854397334635e-08
1859 3.82066787096846e-08
1860 3.81001719063079e-08
1861 3.80095208640796e-08
1862 3.78284745750079e-08
1863 3.78950559820623e-08
1864 3.81046802999663e-08
1865 3.81744946764684e-08
1866 3.81974594176882e-08
1867 3.81625255840845e-08
1868 3.81769744706162e-08
1869 3.81849822872482e-08
1870 3.79660818339289e-08
1871 3.80449129977478e-08
1872 3.79707678632712e-08
1873 3.85150897841413e-08
1874 3.8050032458159e-08
1875 3.80515174924767e-08
1876 3.79978501996447e-08
1877 3.81223976830825e-08
1878 3.79441935649538e-08
1879 3.7945220299207e-08
1880 3.78996816152721e-08
1881 3.81487410550108e-08
1882 3.78562283742667e-08
1883 3.76297712989526e-08
1884 3.80846323366768e-08
1885 3.77603761592127e-08
1886 3.7756070270234e-08
1887 3.79221205548674e-08
1888 3.76243995958703e-08
1889 3.74920432477666e-08
1890 3.74603281727559e-08
1891 3.73309845258518e-08
1892 3.74814348447217e-08
1893 3.78110627252681e-08
1894 3.71967736612078e-08
1895 3.7418395493205e-08
1896 3.70427564178044e-08
1897 3.71905457541288e-08
1898 3.76311533045737e-08
1899 3.72750790234022e-08
1900 3.71423887202127e-08
1901 3.71765622730891e-08
1902 3.72857478225797e-08
1903 3.71756350148189e-08
1904 3.70189425780154e-08
1905 3.71401718268771e-08
1906 3.7142662279166e-08
1907 3.70422448270347e-08
1908 3.6760479105169e-08
1909 3.67364236808498e-08
1910 3.69138355438281e-08
1911 3.68695438623945e-08
1912 3.70101247426646e-08
1913 3.72308726070969e-08
1914 3.68637884662348e-08
1915 3.72893111943995e-08
1916 3.67461012729109e-08
1917 3.69386867760113e-08
1918 3.74956705684326e-08
1919 3.7316215895089e-08
1920 3.71627564277333e-08
1921 3.70707589070207e-08
1922 3.73551145571582e-08
1923 3.6807133341199e-08
1924 3.6861298013946e-08
1925 3.7124710416947e-08
1926 3.73925850283285e-08
1927 3.7383731665841e-08
1928 3.72009054672162e-08
1929 3.73719331037137e-08
1930 3.67471635343009e-08
1931 3.6998617503059e-08
1932 3.68025609986944e-08
1933 3.69938994992935e-08
1934 3.68336969813754e-08
1935 3.6754727261723e-08
1936 3.67600989648054e-08
1937 3.68027563979467e-08
1938 3.68778110271251e-08
1939 3.69498103225396e-08
1940 3.64017438414521e-08
1941 3.69398449606706e-08
1942 3.70920538728114e-08
1943 3.69871813177269e-08
1944 3.66617420866078e-08
1945 3.68504515790846e-08
1946 3.61855576613834e-08
1947 3.66246695193695e-08
1948 3.65394825507792e-08
1949 3.65171111127438e-08
1950 3.64721159940018e-08
1951 3.62484797733487e-08
1952 3.63026018135315e-08
1953 3.62190917257976e-08
1954 3.60858010139964e-08
1955 3.60604666127529e-08
1956 3.62690784072583e-08
1957 3.64636107974547e-08
1958 3.61566776518885e-08
1959 3.64400882801874e-08
1960 3.61957113170774e-08
1961 3.63251153601141e-08
1962 3.62083447669193e-08
1963 3.62127323683126e-08
1964 3.61843390805916e-08
1965 3.60752245853746e-08
1966 3.63051135821024e-08
1967 3.62632022188336e-08
1968 3.61675880355961e-08
1969 3.60870728854934e-08
1970 3.59945033778786e-08
1971 3.58435130465296e-08
1972 3.57915936888276e-08
1973 3.59368890201495e-08
1974 3.55548870345501e-08
1975 3.57974307974018e-08
1976 3.54826923398832e-08
1977 3.51862929903746e-08
1978 3.54415767844785e-08
1979 3.51279112464908e-08
1980 3.50152937755865e-08
1981 3.53580311696078e-08
1982 3.52526612346082e-08
1983 3.52644171641714e-08
1984 3.52357005795056e-08
1985 3.49709878832982e-08
1986 3.49359119411474e-08
1987 3.48665345484278e-08
1988 3.51408253607133e-08
1989 3.48414346262871e-08
1990 3.49616797734598e-08
1991 3.46956419150501e-08
1992 3.47360895602833e-08
1993 3.48346915757247e-08
1994 3.45919026756292e-08
1995 3.46044508603427e-08
1996 3.49443851632714e-08
1997 3.4552130045995e-08
1998 3.46737181189383e-08
1999 3.47053479288206e-08
2000 3.43657902135419e-08
2001 3.453709851442e-08
2002 3.41027082129131e-08
2003 3.4684472183244e-08
2004 3.43402604130461e-08
2005 3.38759491569363e-08
2006 3.43067512176276e-08
2007 3.37252288318268e-08
2008 3.41043921991968e-08
2009 3.38960468582172e-08
2010 3.37324799204453e-08
2011 3.42046178047895e-08
2012 3.36746808216049e-08
2013 3.38747625505675e-08
2014 3.35256409300655e-08
2015 3.37655485793675e-08
2016 3.34297567405883e-08
2017 3.32053389229259e-08
2018 3.32856444629215e-08
2019 3.3241295938069e-08
2020 3.31607949988211e-08
2021 3.31141016829406e-08
2022 3.31786225160613e-08
2023 3.30363789657895e-08
2024 3.29189013825726e-08
2025 3.28318137121641e-08
2026 3.30337606158082e-08
2027 3.29559703970972e-08
2028 3.30670211212691e-08
2029 3.2977197861328e-08
2030 3.16127426458479e-08
2031 3.20474562442996e-08
2032 3.26924762816816e-08
2033 3.24710285326546e-08
2034 3.24288613740009e-08
2035 3.25265396838859e-08
2036 3.26459144162072e-08
2037 3.08133252246989e-08
2038 3.25696341008097e-08
2039 3.07449177228136e-08
2040 3.26063975819579e-08
2041 3.05760430308055e-08
2042 3.24130837725534e-08
2043 3.21379189927029e-08
2044 3.27423848034414e-08
2045 3.22924726958718e-08
2046 3.23991429240778e-08
2047 3.23170503691017e-08
2048 3.23570894522618e-08
2049 3.23129718537984e-08
2050 3.23407007840615e-08
2051 3.22926965168335e-08
2052 3.24200826185006e-08
2053 3.22392885721001e-08
2054 3.22762225835049e-08
2055 3.23226565512869e-08
2056 3.20463371394908e-08
2057 3.21881472586938e-08
2058 3.19948227911482e-08
2059 3.23354711895263e-08
2060 3.1843491399286e-08
2061 3.203302867405e-08
2062 3.17923820603028e-08
2063 3.24500000203898e-08
2064 3.22828199728065e-08
2065 3.20063300307538e-08
2066 3.20595319180939e-08
2067 3.21109787648766e-08
2068 3.18577590974201e-08
2069 3.16083550444546e-08
2070 3.20875948034427e-08
2071 3.18820845279788e-08
2072 3.13100514404141e-08
2073 3.18861168580042e-08
2074 3.13246317773519e-08
2075 3.18501065521559e-08
2076 3.19245430091541e-08
2077 3.25853939386889e-08
2078 3.20600541670046e-08
2079 3.18856088199482e-08
2080 3.16918438159064e-08
2081 3.1944708211995e-08
2082 3.15937960237989e-08
2083 3.16043724524206e-08
2084 3.13681027819257e-08
2085 3.10835943651e-08
2086 3.12799635082683e-08
2087 3.16493213858848e-08
2088 3.13628554238221e-08
2089 3.15311936560647e-08
2090 3.16437294145544e-08
2091 3.12545260783281e-08
2092 3.19121760128382e-08
2093 3.14304067217108e-08
2094 3.15758335034388e-08
2095 3.14034771520255e-08
2096 3.09228163075659e-08
2097 3.2225695889565e-08
2098 3.18479393968119e-08
2099 3.2399128713223e-08
2100 3.13835997189926e-08
2101 3.21009814285844e-08
2102 3.19609227972251e-08
2103 3.122636726971e-08
2104 3.23475006780427e-08
2105 3.11977821354503e-08
2106 3.20611626136724e-08
2107 3.07153307232966e-08
2108 3.15714459020455e-08
2109 3.11628234328509e-08
2110 3.13093515558194e-08
2111 3.11874259750766e-08
2112 3.11925916207656e-08
2113 3.16330783789454e-08
2114 3.06881098310896e-08
2115 3.12541388325371e-08
2116 3.06386169768302e-08
2117 3.14659018840757e-08
2118 3.00840312661421e-08
2119 3.10643990530934e-08
2120 3.06681435802147e-08
2121 3.0729705002841e-08
2122 3.1313934556465e-08
2123 3.0846052823108e-08
2124 3.123565761598e-08
2125 3.16067456651581e-08
2126 3.07118845910281e-08
2127 3.11608552294729e-08
2128 3.04812530771414e-08
2129 3.15561692332267e-08
2130 3.08219156863743e-08
2131 3.09933696485132e-08
2132 3.01767286714494e-08
2133 3.06761620549878e-08
2134 3.16123802690527e-08
2135 3.14231307640966e-08
2136 3.05702165803723e-08
2137 3.09736840620189e-08
2138 3.07355492168426e-08
2139 3.01406828384643e-08
2140 3.06755758572308e-08
2141 3.04553253727136e-08
2142 3.05902645436618e-08
2143 3.06681577910695e-08
2144 3.07129823795549e-08
2145 3.02739771029792e-08
2146 3.0608578782676e-08
2147 3.07328704707288e-08
2148 3.03654381639262e-08
2149 3.10978833795161e-08
2150 2.9535948797843e-08
2151 2.96897120222184e-08
2152 3.03074116914104e-08
2153 3.01081470865938e-08
2154 2.99799900460584e-08
2155 3.0271767315071e-08
2156 3.04895841907182e-08
2157 2.99734850273126e-08
2158 3.04801091033369e-08
2159 2.99508435830376e-08
2160 3.09214520655132e-08
2161 3.04874561152246e-08
2162 3.13684580532936e-08
2163 3.01645250999627e-08
2164 3.02295006804343e-08
2165 2.97427451556587e-08
2166 2.98079463334489e-08
2167 2.96401836408222e-08
2168 2.97380786662416e-08
2169 2.96792581622185e-08
2170 2.98026598954948e-08
2171 2.97095166246208e-08
2172 2.96503976926488e-08
2173 3.00368618866287e-08
2174 2.89100618999782e-08
2175 2.96350286532743e-08
2176 2.95552080586958e-08
2177 3.04028340281093e-08
2178 2.88119359481698e-08
2179 2.99704794315403e-08
2180 2.99285609628441e-08
2181 2.93883690716257e-08
2182 2.90247523793141e-08
2183 2.91572899158155e-08
2184 2.94263973188436e-08
2185 2.96256370546644e-08
2186 2.89082180415789e-08
2187 2.90390858026512e-08
2188 2.98111864083239e-08
2189 2.95977216069332e-08
2190 2.97651485681172e-08
2191 2.94532398470437e-08
2192 2.9398773193634e-08
2193 2.93381710037011e-08
2194 2.94415691826089e-08
2195 2.96589419690463e-08
2196 2.89814021670054e-08
2197 2.95392776905601e-08
2198 2.9272426260718e-08
2199 2.92843562732514e-08
2200 2.91853883283011e-08
2201 2.90646084977197e-08
2202 2.90826243087849e-08
2203 2.90562134352967e-08
2204 2.94146271784257e-08
2205 2.88857666674858e-08
2206 2.90142576631069e-08
2207 2.89771602268729e-08
2208 2.96614928174677e-08
2209 2.87180306202117e-08
2210 2.89814199305738e-08
2211 2.88612227450358e-08
2212 2.86852710473795e-08
2213 2.88504260481659e-08
2214 2.86823222950261e-08
2215 2.86574763919134e-08
2216 2.85798034127538e-08
2217 2.91017965281526e-08
2218 2.8362224568923e-08
2219 2.87117210007182e-08
2220 2.85284862400204e-08
2221 2.84824643870252e-08
2222 2.90035906402863e-08
2223 2.81342558139386e-08
2224 2.87804571286188e-08
2225 2.89050596791185e-08
2226 2.79941350100898e-08
2227 2.79855623119829e-08
2228 2.78436527167969e-08
2229 2.86355827938678e-08
2230 2.77321507979877e-08
2231 2.72005618029425e-08
2232 2.76544298571935e-08
2233 2.83185777050221e-08
2234 2.80345453518294e-08
2235 2.8397950657677e-08
2236 2.79682570436535e-08
2237 2.84079018086913e-08
2238 2.83755934304963e-08
2239 2.75107243652428e-08
2240 2.7083290277119e-08
2241 2.70105751098981e-08
2242 2.77563962924887e-08
2243 2.79354352983319e-08
2244 2.78194303149348e-08
2245 2.7263425295132e-08
2246 2.67683581967049e-08
2247 3.06054381837839e-08
2248 2.76062444015679e-08
2249 2.7365052446271e-08
2250 2.68884345899778e-08
2251 2.86250827485901e-08
2252 2.7886335018934e-08
2253 2.6634776162382e-08
2254 2.75308025265986e-08
2255 2.70855338158071e-08
2256 2.71714402089174e-08
2257 2.72241305054877e-08
2258 2.71694009512657e-08
2259 2.68646616063961e-08
2260 2.83207164386567e-08
2261 2.72951865554205e-08
2262 2.8063915635812e-08
2263 2.74024944957318e-08
2264 2.68622564192356e-08
2265 2.90884223375087e-08
2266 2.75261324844678e-08
2267 2.69485802562031e-08
2268 2.94024804503579e-08
2269 2.76600271575944e-08
2270 2.70686388859076e-08
2271 2.63945754142014e-08
2272 2.74479514672521e-08
2273 2.73535309958106e-08
2274 2.63323585159014e-08
2275 2.73444182852245e-08
2276 2.64959982843038e-08
2277 2.65360569073891e-08
2278 2.65743089755688e-08
2279 2.64690456219796e-08
2280 2.62177302090549e-08
2281 2.94779383125388e-08
2282 2.71752522706947e-08
2283 2.61879229412898e-08
2284 2.58418584309084e-08
2285 2.81042051852864e-08
2286 2.63703654468372e-08
2287 2.66395261405705e-08
2288 2.56493564165794e-08
2289 2.80094454296886e-08
2290 2.77919287583472e-08
2291 2.79746288356364e-08
2292 2.72455906724645e-08
2293 2.80696390575486e-08
2294 2.75097438162675e-08
2295 2.87359167572276e-08
2296 2.55658214598498e-08
2297 2.76742344595959e-08
2298 2.82824608177634e-08
2299 2.77730958231359e-08
2300 2.71965561182697e-08
2301 2.78109038021057e-08
2302 2.66064539289346e-08
2303 2.72120228572703e-08
2304 2.85674524036494e-08
2305 2.81438730098671e-08
2306 2.79025975657987e-08
2307 2.90850739048665e-08
2308 2.86440648977759e-08
2309 2.95485218515523e-08
2310 2.98240685481233e-08
2311 2.82285128605508e-08
2312 2.85122219167988e-08
2313 2.8139593766241e-08
2314 2.81186576245318e-08
2315 2.84869337008331e-08
2316 2.81879568575505e-08
2317 2.8269834473349e-08
2318 2.7936827962094e-08
2319 2.87954229349907e-08
2320 2.90171797701078e-08
2321 2.83451733196216e-08
2322 3.03672678114708e-08
2323 2.90200024011256e-08
2324 3.01493585652679e-08
2325 2.89355526206236e-08
2326 2.84417698281914e-08
2327 2.77460010522645e-08
2328 3.07269409915989e-08
2329 2.77802971737628e-08
2330 2.90159771765275e-08
2331 2.8670786633711e-08
2332 2.99100619827186e-08
2333 2.80978511568719e-08
2334 2.87722219383113e-08
2335 2.81742966734555e-08
2336 2.99742524134672e-08
2337 2.96097155683128e-08
2338 2.88455090924344e-08
2339 2.85107830677589e-08
2340 2.88629298239584e-08
2341 2.87651538144473e-08
2342 3.09186169999975e-08
2343 2.88092909528359e-08
2344 2.77100919987561e-08
2345 2.76661413778356e-08
2346 2.84921313209452e-08
2347 2.75954565864822e-08
2348 2.7255959267336e-08
2349 2.80313230405227e-08
2350 2.85201977590077e-08
2351 2.81478609309715e-08
2352 2.83580003923589e-08
2353 2.83198389183781e-08
2354 2.90156272342301e-08
2355 2.76035407864583e-08
2356 2.88669212977766e-08
2357 2.96911970565361e-08
2358 2.76945755217639e-08
2359 2.78739538117634e-08
2360 2.92283797165283e-08
2361 2.78274221443553e-08
2362 2.88856689678596e-08
2363 2.72013984670139e-08
2364 3.02369755900145e-08
2365 2.84585652821079e-08
2366 2.79561671590045e-08
2367 2.81202954255377e-08
2368 3.01523144230487e-08
2369 2.87901968931692e-08
2370 2.75214642186938e-08
2371 3.00382829721002e-08
2372 2.9085610364632e-08
2373 2.7631605448164e-08
2374 2.66884043753635e-08
2375 2.82374053028889e-08
2376 2.73996363375772e-08
2377 3.03340854657108e-08
2378 2.78437450873525e-08
2379 2.59159786963892e-08
2380 2.62149182361782e-08
2381 2.96702307167607e-08
2382 2.7381952705241e-08
2383 2.85215904227698e-08
2384 2.79996186236531e-08
2385 2.67009259147244e-08
2386 2.87271610943662e-08
2387 3.17897921320309e-08
2388 2.82120478090064e-08
2389 2.80839707045288e-08
2390 2.79955134629972e-08
2391 2.8531420781519e-08
2392 2.73182685361917e-08
2393 2.61608654739121e-08
2394 2.65627431161874e-08
2395 2.68189346286363e-08
2396 2.55239953617092e-08
2397 3.01455678197726e-08
2398 2.59404924207729e-08
2399 2.76176184144106e-08
2400 3.01110603118104e-08
2401 2.59333692298469e-08
2402 2.78843597101286e-08
2403 2.73789879656761e-08
2404 2.74537210742665e-08
2405 2.70449564965247e-08
2406 2.74103140185389e-08
2407 2.75313958297829e-08
2408 2.85890777718123e-08
2409 2.72567479697727e-08
2410 2.71156714859444e-08
2411 2.56328416270435e-08
2412 2.56879619797701e-08
2413 2.53490579638083e-08
2414 2.69621036608214e-08
2415 2.76971370283263e-08
2416 2.57466528097439e-08
2417 2.81184640016363e-08
2418 2.55868108922641e-08
2419 2.80560730203661e-08
2420 2.88594694808353e-08
2421 2.84561902930136e-08
2422 2.63346073836601e-08
2423 2.79867506947085e-08
2424 2.55031586959831e-08
2425 2.75283262851644e-08
2426 2.51160727771094e-08
2427 2.88301720274831e-08
2428 2.72236242437884e-08
2429 2.81005956281888e-08
2430 2.80602758806481e-08
2431 2.90631554378251e-08
2432 2.81120335898777e-08
2433 2.81496213005994e-08
2434 2.83409402612733e-08
2435 2.83796381950197e-08
2436 2.80549130593499e-08
2437 2.72876636842057e-08
2438 2.81555401215883e-08
2439 2.73779701132071e-08
2440 2.83173520188029e-08
2441 2.73699356512225e-08
2442 2.75598992516279e-08
2443 2.76794391851354e-08
2444 2.79277347914331e-08
2445 2.73806239903251e-08
2446 2.75886709033557e-08
2447 2.74316800386032e-08
2448 2.74508895614645e-08
2449 2.7413287639888e-08
2450 2.68745345977095e-08
2451 2.70415370096089e-08
2452 2.73194427080625e-08
2453 2.69827467036521e-08
2454 2.66114206226575e-08
2455 2.68112465562353e-08
2456 2.7402625946138e-08
2457 2.5951562676596e-08
2458 2.80336944769033e-08
2459 2.61896246911419e-08
2460 2.72076849938685e-08
2461 2.67586912627849e-08
2462 2.77054823527578e-08
2463 2.61095127740418e-08
2464 2.6687148491078e-08
2465 2.56111150065408e-08
2466 2.66285891115103e-08
2467 2.64867754395937e-08
2468 2.79047522866449e-08
2469 2.58788226403794e-08
2470 2.71326570100427e-08
2471 2.69126534391262e-08
2472 2.64955737350192e-08
2473 2.42714701670366e-08
2474 2.52639669184873e-08
2475 2.69148205944703e-08
2476 2.79809242442752e-08
2477 2.75775171587611e-08
2478 2.78213274640393e-08
2479 2.45482709715361e-08
2480 2.66160622430789e-08
2481 2.64051163156864e-08
2482 2.39471642515809e-08
2483 2.76872427207309e-08
2484 2.79963607852096e-08
2485 2.37980266604154e-08
2486 2.75511986558286e-08
2487 2.74507154784942e-08
2488 2.80164691446316e-08
2489 2.39528148426871e-08
2490 2.35246950808232e-08
2491 2.76082818828627e-08
2492 2.67311452972763e-08
2493 2.45059901260447e-08
2494 2.7517403466959e-08
2495 2.65862265536043e-08
2496 2.68240238909812e-08
2497 2.6520197593527e-08
2498 2.33125287962821e-08
2499 2.55529482018346e-08
2500 2.53111096526482e-08
2501 2.24628902145696e-08
2502 2.67357744831997e-08
2503 2.24108411828183e-08
2504 2.63209098960715e-08
2505 2.23535217003246e-08
2506 2.65264308296764e-08
2507 2.24147420624377e-08
2508 2.54636400853769e-08
2509 2.56697241241e-08
2510 2.60770320892334e-08
2511 2.56246242003044e-08
2512 2.68725166563399e-08
2513 2.20680629325898e-08
2514 2.67356288219389e-08
2515 3.78979230220011e-08
2516 3.74825361859621e-08
2517 3.50435982454655e-08
2518 3.416143101731e-08
2519 3.55949154595692e-08
2520 3.39054331277566e-08
2521 3.55097995452525e-08
2522 3.22432001098605e-08
2523 3.35709664511796e-08
2524 3.31266285513721e-08
2525 3.37162653352152e-08
2526 3.14247685651026e-08
2527 3.26295435115753e-08
2528 3.34788339273473e-08
2529 3.25850209037526e-08
2530 3.04535667794426e-08
2531 3.24444364707688e-08
2532 3.30445004692592e-08
2533 3.16604449324132e-08
2534 3.22617275116954e-08
2535 3.1700906788501e-08
2536 3.08379561886341e-08
2537 3.22728013202322e-08
2538 3.12921812906097e-08
2539 3.10082981513915e-08
2540 3.25120019795122e-08
2541 3.25865308070661e-08
2542 3.17372155222984e-08
2543 3.37688348395204e-08
2544 3.46149029439857e-08
2545 3.17536716920586e-08
2546 3.18734123538889e-08
2547 3.07822141110137e-08
2548 3.37103109870895e-08
2549 3.3667490129119e-08
2550 3.31740324099883e-08
2551 3.39531887050271e-08
2552 3.13836601151252e-08
2553 3.55050318034955e-08
2554 3.49419906342519e-08
2555 3.36651773125141e-08
2556 3.35401182383066e-08
2557 3.37782992687607e-08
2558 3.26100533243334e-08
2559 3.23626743181649e-08
2560 3.51656765928965e-08
2561 3.13941015406272e-08
2562 3.49291191525936e-08
2563 3.43574697581062e-08
2564 3.38495311780207e-08
2565 3.25004627654835e-08
2566 3.19637898371639e-08
2567 3.16473816042162e-08
2568 3.26298561503791e-08
2569 3.19587982744451e-08
2570 3.25541869017343e-08
2571 3.18818855760128e-08
2572 3.33925100903798e-08
2573 3.24407807283933e-08
2574 3.51152067423754e-08
2575 3.25221591879199e-08
2576 3.33569829535918e-08
2577 3.29268914356362e-08
2578 3.32732597030372e-08
2579 3.13090389170156e-08
2580 2.99948723636589e-08
2581 3.00385387674851e-08
2582 3.16966470848001e-08
2583 3.41229906553053e-08
2584 3.07284153677756e-08
2585 2.93450828081632e-08
2586 2.98401054976694e-08
2587 3.1509223674675e-08
2588 3.27807505584587e-08
2589 3.11676373598857e-08
2590 2.99710549711563e-08
2591 2.96483833039929e-08
2592 3.05595229121991e-08
2593 2.89534920483447e-08
2594 2.92505966115186e-08
2595 2.81479479724567e-08
2596 2.9311648219732e-08
2597 2.91754798098509e-08
2598 2.95663635796473e-08
2599 2.87654273734006e-08
2600 2.88370536338789e-08
2601 2.9368987242151e-08
2602 2.90187358586991e-08
2603 2.87261165965447e-08
2604 2.90192989638172e-08
2605 2.93236190884727e-08
2606 2.88054415875649e-08
2607 2.88779400392514e-08
2608 2.87217698513587e-08
2609 2.99550535487469e-08
2610 3.13887369429722e-08
2611 3.07379792729989e-08
2612 2.65182737990699e-08
2613 3.02758387249469e-08
2614 3.06674472483337e-08
2615 3.06325844690036e-08
2616 2.61553623204236e-08
2617 3.0278812346296e-08
2618 2.66164192908036e-08
2619 2.54032688218331e-08
2620 2.91579151934229e-08
2621 2.98586435576453e-08
2622 3.18280015676464e-08
2623 2.60268411267361e-08
2624 2.58782488771203e-08
2625 2.46868623321461e-08
2626 2.53549483630877e-08
2627 2.62593804478684e-08
2628 2.63842654391055e-08
2629 2.60550514497027e-08
2630 2.42042741405157e-08
2631 2.59179131489873e-08
2632 2.50305038917986e-08
2633 2.8402331153643e-08
2634 2.82212120339409e-08
2635 2.98758493499918e-08
2636 2.89079498116962e-08
2637 2.7994088824812e-08
2638 2.76787659458932e-08
2639 2.56251677654973e-08
2640 2.66462318876393e-08
2641 2.73356342006537e-08
2642 2.87869053039458e-08
2643 2.96484099493455e-08
2644 2.68317297269505e-08
2645 2.8756087289139e-08
2646 2.4835763667852e-08
2647 2.77978688956182e-08
2648 2.41815829582492e-08
2649 2.5737559639083e-08
2650 2.61255088673806e-08
2651 2.59412669123549e-08
2652 2.59794159518378e-08
2653 2.44964173390372e-08
2654 2.60628976178623e-08
2655 2.40135680229514e-08
2656 2.76517742037186e-08
2657 2.28496528364985e-08
2658 2.60271395546852e-08
2659 2.30376056009618e-08
2660 2.23323297632305e-08
2661 2.39251889411207e-08
2662 2.21640519271205e-08
2663 2.1934502214549e-08
2664 2.20056755040332e-08
2665 2.17467057694876e-08
2666 2.24378382540635e-08
2667 2.21031388747406e-08
2668 2.20015774488047e-08
2669 2.19008384760855e-08
2670 2.24145431104716e-08
2671 2.13071587040758e-08
2672 2.15375592915734e-08
2673 2.14250928110005e-08
2674 2.11234212343925e-08
2675 2.11532498184397e-08
2676 2.10715977999598e-08
2677 2.09035864173757e-08
2678 2.09606128009909e-08
2679 2.02596375231678e-08
2680 2.09824495556177e-08
2681 1.97825755776648e-08
2682 2.09307469134501e-08
2683 2.09286543650933e-08
2684 2.05892458637891e-08
2685 2.06452295259396e-08
2686 1.98511198590268e-08
2687 1.98704821485762e-08
2688 1.95660945223608e-08
2689 2.01602254890076e-08
2690 1.93518623348155e-08
2691 1.95835045957438e-08
2692 1.88798239264543e-08
2693 1.9226847669529e-08
2694 1.99411420709339e-08
2695 1.87932975848071e-08
2696 1.8911372023922e-08
2697 1.8433967241549e-08
2698 1.7986293343597e-08
2699 1.7891469639153e-08
2700 1.81449877345585e-08
2701 1.78645525039656e-08
2702 1.77635239850815e-08
2703 1.77648917798479e-08
2704 1.82661956671382e-08
2705 1.81912795937933e-08
2706 1.77752088603711e-08
2707 1.78761734304089e-08
2708 1.81229058426879e-08
2709 1.77396817235831e-08
2710 1.78006303030998e-08
2711 1.79824528601102e-08
2712 1.84778965461874e-08
2713 1.72876468695904e-08
2714 1.71094232115365e-08
2715 1.68791007837399e-08
2716 1.77092527309242e-08
2717 1.6798798796458e-08
2718 1.73595289254536e-08
2719 1.77811170232189e-08
2720 1.85029360721956e-08
2721 1.8232716669786e-08
2722 1.68346119266971e-08
2723 1.62840905204575e-08
2724 1.67377329773899e-08
2725 1.67312261822872e-08
2726 1.86693505099811e-08
2727 1.75248597855671e-08
2728 1.68015947821232e-08
2729 1.6285191861698e-08
2730 1.64879416786334e-08
2731 1.92822984246277e-08
2732 1.74184915380238e-08
2733 1.63732085667334e-08
2734 1.72376548590591e-08
2735 1.62840834150302e-08
2736 1.63131783637027e-08
2737 1.58606212607992e-08
2738 1.59454884851584e-08
2739 1.59873554395062e-08
2740 1.63565037070157e-08
2741 1.56141197749093e-08
2742 1.59065187688157e-08
2743 1.56186761302024e-08
2744 1.61758411110213e-08
2745 1.61461155556708e-08
2746 1.60712865238111e-08
2747 1.5997633440179e-08
2748 1.58076574052757e-08
2749 1.59129278642922e-08
2750 1.55785606636982e-08
2751 1.6901198662822e-08
2752 1.65706381949349e-08
2753 1.55789621203439e-08
2754 1.53671670943822e-08
2755 1.58467230448878e-08
2756 1.54130948004649e-08
2757 1.56766084558058e-08
2758 1.49707233276786e-08
2759 1.5624664229108e-08
2760 1.49711816277431e-08
2761 1.55192054762665e-08
2762 1.55479735752806e-08
2763 1.53166190841603e-08
2764 1.63157558574767e-08
2765 1.50963774814272e-08
2766 1.53199337660226e-08
2767 1.48792533849473e-08
2768 1.56423940467221e-08
2769 1.49931231874234e-08
2770 1.51294141659264e-08
2771 1.57471014006205e-08
2772 1.50455914393888e-08
2773 1.54479362635129e-08
2774 1.59366013718909e-08
2775 1.50712864410707e-08
2776 1.46525902522399e-08
2777 1.61120876640553e-08
2778 1.47568179897917e-08
2779 1.58073731881814e-08
2780 1.46616194740545e-08
2781 1.48493803919791e-08
2782 1.64321800610878e-08
2783 1.63515831985706e-08
2784 1.47900536262568e-08
2785 1.49564698403992e-08
2786 1.57243835730014e-08
2787 1.46087071328793e-08
2788 1.45351517488734e-08
2789 1.46924588051434e-08
2790 1.43575507038918e-08
2791 1.44518708111718e-08
2792 1.48227208285334e-08
2793 1.60066271348569e-08
2794 1.52612731341151e-08
2795 1.43129668117581e-08
2796 1.43564857779666e-08
2797 1.51828611905103e-08
2798 1.46253302801824e-08
2799 1.44933700596539e-08
2800 1.44794176648588e-08
2801 1.49435859242431e-08
2802 1.46119729649286e-08
2803 1.47783776327515e-08
2804 1.41645548623615e-08
2805 1.41782816598379e-08
2806 1.4405297399378e-08
2807 1.48343417549768e-08
2808 1.48584868853163e-08
2809 1.44577301242066e-08
2810 1.61918105590075e-08
2811 1.49739385335579e-08
2812 1.61022146727419e-08
2813 1.49066163857015e-08
2814 1.39548905764286e-08
2815 1.46415075619188e-08
2816 1.52140788856059e-08
2817 1.4055898667209e-08
2818 1.50237013940568e-08
2819 1.44398448753691e-08
2820 1.40806024617746e-08
2821 1.53786352541374e-08
2822 1.5363873728802e-08
2823 1.43451908130032e-08
2824 1.50297108092445e-08
2825 1.55694426240416e-08
2826 1.52388857088681e-08
2827 1.52753401039263e-08
2828 1.49898617962663e-08
2829 1.53793600077279e-08
2830 1.54383155148707e-08
2831 1.53459929208566e-08
2832 1.47412180240281e-08
2833 1.39357618778035e-08
2834 1.59797135523831e-08
2835 1.61262025955011e-08
2836 1.50101211460196e-08
2837 1.48535450605891e-08
2838 1.43676723851627e-08
2839 1.47963987728872e-08
2840 1.47004390882444e-08
2841 1.39902329721053e-08
2842 1.40954252714209e-08
2843 1.38663907023329e-08
2844 1.45654004413132e-08
2845 1.42435352401549e-08
2846 1.41918548024478e-08
2847 1.39483562477949e-08
2848 1.36062361377753e-08
2849 1.38364182333817e-08
2850 1.37316158443923e-08
2851 1.35874564932692e-08
2852 1.38614755229582e-08
2853 1.38877433997209e-08
2854 1.37228841623482e-08
2855 1.34571056520372e-08
2856 1.45846401622407e-08
2857 1.4473204856813e-08
2858 1.36447351195557e-08
2859 1.48241507957891e-08
2860 1.35849704818725e-08
2861 1.41031071265729e-08
2862 1.4475422638327e-08
2863 1.3382474683965e-08
2864 1.42157094984441e-08
2865 1.41585676516343e-08
2866 1.37596281035712e-08
2867 1.39981208846507e-08
2868 1.44118121880865e-08
2869 1.34399948947816e-08
2870 1.33520510203766e-08
2871 1.34222757353086e-08
2872 1.36355229329865e-08
2873 1.33841906446719e-08
2874 1.56809978335559e-08
2875 1.47343541812006e-08
2876 1.46304870440872e-08
2877 1.25919648041872e-08
2878 1.43846907718626e-08
2879 1.5183273305297e-08
2880 1.4537296699757e-08
2881 1.32267103936101e-08
2882 1.58961039886663e-08
2883 1.34365452097995e-08
2884 1.50535885978798e-08
2885 1.44835281545852e-08
2886 1.42713441064757e-08
2887 1.47863259414294e-08
2888 1.44875631491459e-08
2889 1.48807508537629e-08
2890 1.30922606089712e-08
2891 1.31364332744965e-08
2892 1.51840211515264e-08
2893 1.37232509800356e-08
2894 1.37116114018454e-08
2895 1.31477664311319e-08
2896 1.52771644224003e-08
2897 1.53323629348279e-08
2898 1.47262237959467e-08
2899 1.45620129288204e-08
2900 1.46112864030101e-08
2901 1.31068382813737e-08
2902 1.48512535602663e-08
2903 1.31717996509906e-08
2904 1.48662833154845e-08
2905 1.43665452867481e-08
2906 1.35511601939697e-08
2907 1.4173137330431e-08
2908 1.48410919109665e-08
2909 1.39156570710952e-08
2910 1.33845095007246e-08
2911 1.34013200536742e-08
2912 1.50459236181177e-08
2913 1.54282453479482e-08
2914 1.4993746688674e-08
2915 1.4720838770188e-08
2916 1.45316372268667e-08
2917 1.53113077772105e-08
2918 1.53285331094821e-08
2919 1.4754823141061e-08
2920 1.46091672093007e-08
2921 1.31105410972054e-08
2922 1.41442617618281e-08
2923 1.29607977683577e-08
2924 1.47462762001282e-08
2925 1.3118852670857e-08
2926 1.48218726181426e-08
2927 1.47648959725188e-08
2928 1.48969192537152e-08
2929 1.47342991141386e-08
2930 1.47226835167658e-08
2931 1.48025112167716e-08
2932 1.49024081963489e-08
2933 1.31145032611357e-08
2934 1.32368098704205e-08
2935 1.35420243907447e-08
2936 1.37378224351892e-08
2937 1.55771964216456e-08
2938 1.59324091697499e-08
2939 1.51521568625412e-08
2940 1.35917481713932e-08
2941 1.38088545043047e-08
2942 1.33105642063924e-08
2943 1.49468846188938e-08
2944 1.52446641976667e-08
2945 1.38119498060973e-08
2946 1.36923823390589e-08
2947 1.3148661714979e-08
2948 1.44821159508979e-08
2949 1.32824231613426e-08
2950 1.35868321038402e-08
2951 1.37158453483721e-08
2952 1.29938104720395e-08
2953 1.68829572544382e-08
2954 1.41488083471586e-08
2955 1.36973961062381e-08
2956 1.33971544968858e-08
2957 1.33138735591842e-08
2958 1.37174565040254e-08
2959 1.31500534905626e-08
2960 1.4819960370005e-08
2961 1.51513344093246e-08
2962 1.62744857590269e-08
2963 1.35664492972865e-08
2964 1.33311930383684e-08
2965 1.34081332703317e-08
2966 1.46370435771814e-08
2967 1.36525457605785e-08
2968 1.28429551438103e-08
2969 1.32191200208354e-08
2970 1.44665968093705e-08
2971 1.35797337819099e-08
2972 1.39680906841022e-08
2973 1.48924677034756e-08
2974 1.41074512072237e-08
2975 1.60227831003112e-08
2976 1.54620209968925e-08
2977 1.61714712731964e-08
2978 1.39354439099293e-08
2979 1.38221398771066e-08
2980 1.3655543362745e-08
2981 1.60987578823324e-08
2982 1.57333897021772e-08
2983 1.37499709396138e-08
2984 1.37764857299771e-08
2985 1.52512455997567e-08
2986 1.45930192374522e-08
2987 1.47488945501095e-08
2988 1.35732678430145e-08
2989 1.31470772046782e-08
2990 1.44162779491808e-08
2991 1.32167343736e-08
2992 1.3968386447516e-08
2993 1.28096528939636e-08
2994 1.33785622580262e-08
2995 1.42897782495766e-08
2996 1.35065310047366e-08
2997 1.53874868402681e-08
2998 1.31774484657399e-08
2999 1.41603999637141e-08
3000 8.43196268363045e-09
3001 8.5258546889122e-09
3002 8.5462916743495e-09
3003 8.57128057418777e-09
3004 8.56739923449368e-09
3005 8.5725533338632e-09
3006 8.56679260863302e-09
3007 8.5631981505685e-09
3008 8.55887272166456e-09
3009 8.55462545246155e-09
3010 8.55139337119226e-09
3011 8.54820036977344e-09
3012 8.54546033934867e-09
3013 8.55138182487281e-09
3014 8.55475956740293e-09
3015 8.55196358173771e-09
3016 8.54888426715661e-09
3017 8.5712503761215e-09
3018 8.54385273640901e-09
3019 8.52762394032425e-09
3020 8.51685300062854e-09
3021 8.51422932157675e-09
3022 8.51044212879515e-09
3023 8.50775094818346e-09
3024 8.50345038827527e-09
3025 8.50085601911132e-09
3026 8.49793924118103e-09
3027 8.49560866100774e-09
3028 8.49300008098908e-09
3029 8.49045012074612e-09
3030 8.48791259500103e-09
3031 8.48746051218541e-09
3032 8.48518943996623e-09
3033 8.48246273221775e-09
3034 8.48259418262387e-09
3035 8.48519121632307e-09
3036 8.48256043184392e-09
3037 8.48138625997308e-09
3038 8.4792954879731e-09
3039 8.47618686350415e-09
3040 8.4738580596877e-09
3041 8.4730329419358e-09
3042 8.47104875134619e-09
3043 8.46904679718818e-09
3044 8.46673309240487e-09
3045 8.46435277424007e-09
3046 8.46317682601239e-09
3047 8.45862491161142e-09
3048 8.45667003090966e-09
3049 8.4544220513294e-09
3050 8.45266168170156e-09
3051 8.45047054554016e-09
3052 8.44720027259882e-09
3053 8.44364134167108e-09
3054 8.44122372001266e-09
3055 8.43869152333809e-09
3056 8.43670644457006e-09
3057 8.43471870126677e-09
3058 8.43208347589552e-09
3059 8.42888958629828e-09
3060 8.42682901236458e-09
3061 8.4247604448251e-09
3062 8.42286862479114e-09
3063 8.42102476639184e-09
3064 8.41864888911914e-09
3065 8.41662206596538e-09
3066 8.41490077618801e-09
3067 8.41331715406568e-09
3068 8.41511127447347e-09
3069 8.41518765781757e-09
3070 8.41310221488811e-09
3071 8.41112957061796e-09
3072 8.40918801259249e-09
3073 8.40730329798589e-09
3074 8.40310399041755e-09
3075 8.40086400444306e-09
3076 8.39896596716017e-09
3077 8.39716474132501e-09
3078 8.39543545794186e-09
3079 8.39370084548818e-09
3080 8.39238989414071e-09
3081 8.39233660343552e-09
3082 8.39117930695465e-09
3083 8.3897520042342e-09
3084 8.38745251030559e-09
3085 8.38756353260806e-09
3086 8.38632718824783e-09
3087 8.38488212195898e-09
3088 8.3835169917279e-09
3089 8.38195557406607e-09
3090 8.3809021944603e-09
3091 8.37840552492253e-09
3092 8.3782509818775e-09
3093 8.37566016542723e-09
3094 8.37439984024968e-09
3095 8.37275937470849e-09
3096 8.37209856996424e-09
3097 8.37034885847743e-09
3098 8.36897218192689e-09
3099 8.36686187000169e-09
3100 8.36616731447748e-09
3101 8.36409519422432e-09
3102 8.36373992285644e-09
3103 8.36172553420056e-09
3104 8.36123437153447e-09
3105 8.3592475164096e-09
3106 8.35932034704001e-09
3107 8.35787528075116e-09
3108 8.35750135763647e-09
3109 8.35660696196783e-09
3110 8.35517344199843e-09
3111 8.35341218419217e-09
3112 8.35295121959234e-09
3113 8.35264479803755e-09
3114 8.3508346904182e-09
3115 8.34959479334429e-09
3116 8.34917734948704e-09
3117 8.34752444944797e-09
3118 8.34615399014638e-09
3119 8.34566904472922e-09
3120 8.34398328208863e-09
3121 8.34268831795271e-09
3122 8.34252311676664e-09
3123 8.34043056840983e-09
3124 8.33923952114901e-09
3125 8.33785573917112e-09
3126 8.33962499058316e-09
3127 8.33526314636401e-09
3128 8.33543722933427e-09
3129 8.33627833429773e-09
3130 8.33619040463418e-09
3131 8.335893753042e-09
3132 8.33493896124082e-09
3133 8.33449398385255e-09
3134 8.33305158209896e-09
3135 8.33248048337509e-09
3136 8.33078050987979e-09
3137 8.33055047166908e-09
3138 8.32907787184922e-09
3139 8.32803781491975e-09
3140 8.32671620543124e-09
3141 8.32558111341086e-09
3142 8.3240410120311e-09
3143 8.32277091689093e-09
3144 8.32276469964199e-09
3145 8.32006463724611e-09
3146 8.32015256690966e-09
3147 8.31886914909319e-09
3148 8.31695157188506e-09
3149 8.31605539985958e-09
3150 8.31433055736852e-09
3151 8.31348589969139e-09
3152 8.31163138315105e-09
3153 8.3103666170814e-09
3154 8.30789836925305e-09
3155 8.30656787798034e-09
3156 8.30521518224714e-09
3157 8.30369639714945e-09
3158 8.30202306900674e-09
3159 8.30189605949272e-09
3160 8.30137381058194e-09
3161 8.29990653983259e-09
3162 8.29849078343159e-09
3163 8.29703239446644e-09
3164 8.29560242721072e-09
3165 8.29428170590063e-09
3166 8.29254176437644e-09
3167 8.29096080678937e-09
3168 8.29221047382589e-09
3169 8.29220425657695e-09
3170 8.29782820233049e-09
3171 8.29115709422013e-09
3172 8.29048651951325e-09
3173 8.29387047929231e-09
3174 8.28819324283359e-09
3175 8.2866611350596e-09
3176 8.28438473376991e-09
3177 8.28321411461275e-09
3178 8.28193336133154e-09
3179 8.28613178072146e-09
3180 8.28000512598237e-09
3181 8.27823409821349e-09
3182 8.28304358435616e-09
3183 8.28084711912425e-09
3184 8.27508017664513e-09
3185 8.27373991540981e-09
3186 8.27277357728917e-09
3187 8.27105495204705e-09
3188 8.2741049567403e-09
3189 8.27508994660775e-09
3190 8.27456236862645e-09
3191 8.26931056963076e-09
3192 8.27284996063327e-09
3193 8.2710842619349e-09
3194 8.2699527226282e-09
3195 8.26856805247189e-09
3196 8.2671078871499e-09
3197 8.26608825832409e-09
3198 8.2634601383802e-09
3199 8.2612467977583e-09
3200 8.25917023661304e-09
3201 8.2575271065366e-09
3202 8.25594170805743e-09
3203 8.25410673144233e-09
3204 8.25347257205067e-09
3205 8.25196178055876e-09
3206 8.2506517173897e-09
3207 8.24890999950867e-09
3208 8.2480413610142e-09
3209 8.24665757903631e-09
3210 8.24499490903463e-09
3211 8.24395307574832e-09
3212 8.24264834164978e-09
3213 8.24113399744419e-09
3214 8.23993762111286e-09
3215 8.23891710410862e-09
3216 8.23683699024969e-09
3217 8.23603940602879e-09
3218 8.23420975848421e-09
3219 8.23350188028371e-09
3220 8.23221490975357e-09
3221 8.23070855915375e-09
3222 8.22965695590483e-09
3223 8.22839396619202e-09
3224 8.22662027388787e-09
3225 8.22699419700257e-09
3226 8.22436341252342e-09
3227 8.22325940674773e-09
3228 8.22208079398479e-09
3229 8.22072898643e-09
3230 8.21981505083613e-09
3231 8.21847656595764e-09
3232 8.21767898173675e-09
3233 8.21658208138842e-09
3234 8.21547807561274e-09
3235 8.21438561615651e-09
3236 8.21308887566374e-09
3237 8.21122991823131e-09
3238 8.21072276835366e-09
3239 8.2162516790163e-09
3240 8.21623302726948e-09
3241 8.219069869142e-09
3242 8.22050072457614e-09
3243 8.22027601543596e-09
3244 8.21956014362968e-09
3245 8.21879808654558e-09
3246 8.21814349905026e-09
3247 8.21663448391519e-09
3248 8.21592660571469e-09
3249 8.21422663221938e-09
3250 8.21291923358558e-09
3251 8.21260126571133e-09
3252 8.21144752194414e-09
3253 8.21028667274959e-09
3254 8.20938339529675e-09
3255 8.20818879532226e-09
3256 8.20704126880401e-09
3257 8.20593015760096e-09
3258 8.2046787142076e-09
3259 8.20364753906233e-09
3260 8.20256751410398e-09
3261 8.20171486282106e-09
3262 8.20037104887206e-09
3263 8.19918977157386e-09
3264 8.19755729963845e-09
3265 8.19611578606327e-09
3266 8.19567258503184e-09
3267 8.19387580008879e-09
3268 8.19363243920179e-09
3269 8.19245382643885e-09
3270 8.19091194870225e-09
3271 8.18156919990543e-09
3272 8.17992162893688e-09
3273 8.18060463814163e-09
3274 8.18034529004308e-09
3275 8.17910716932602e-09
3276 8.17806888875339e-09
3277 8.17717094037107e-09
3278 8.17242007400409e-09
3279 8.17364753658012e-09
3280 8.17273893005677e-09
3281 8.17214207415873e-09
3282 8.17166956323945e-09
3283 8.17077694392765e-09
3284 8.17018008802961e-09
3285 8.16877410159123e-09
3286 8.16804579528707e-09
3287 8.16767098399396e-09
3288 8.16680323367791e-09
3289 8.16536349645958e-09
3290 8.16457745855814e-09
3291 8.16354273069919e-09
3292 8.16253464819283e-09
3293 8.16132139647152e-09
3294 8.1604820678649e-09
3295 8.16008327575446e-09
3296 8.15878387072644e-09
3297 8.15797562836451e-09
3298 8.15719669589043e-09
3299 8.15650746943675e-09
3300 8.1555269204614e-09
3301 8.15453837788027e-09
3302 8.15354006533653e-09
3303 8.15273715204512e-09
3304 8.15192535696951e-09
3305 8.15116951713435e-09
3306 8.15033729395509e-09
3307 8.14935230408764e-09
3308 8.14809553162377e-09
3309 8.14729084197552e-09
3310 8.14636180734851e-09
3311 8.14538037019474e-09
3312 8.14652079128564e-09
3313 8.1455677758413e-09
3314 8.14347167477081e-09
3315 8.14255862735536e-09
3316 8.14286327255331e-09
3317 8.1403470630903e-09
3318 8.13948819455845e-09
3319 8.13995093551512e-09
3320 8.13909917241062e-09
3321 8.13809641897478e-09
3322 8.13725797854659e-09
3323 8.13466360938264e-09
3324 8.13712919267573e-09
3325 8.13281619826967e-09
3326 8.13199196869618e-09
3327 8.13085332396213e-09
3328 8.13001221899867e-09
3329 8.13226819218471e-09
3330 8.12838418795536e-09
3331 8.12732903199276e-09
3332 8.12604472599787e-09
3333 8.12521250281861e-09
3334 8.1244317939877e-09
3335 8.12384737258753e-09
3336 8.12243960979231e-09
3337 8.12189160370735e-09
3338 8.12109313130804e-09
3339 8.12035771957653e-09
3340 8.11935052524859e-09
3341 8.1187918610226e-09
3342 8.11743561257572e-09
3343 8.11716560633613e-09
3344 8.11538924949673e-09
3345 8.11506595255196e-09
3346 8.11401523748145e-09
3347 8.11318390248061e-09
3348 8.11796141420018e-09
3349 8.11469558215094e-09
3350 8.11424882840583e-09
3351 8.11374523124186e-09
3352 8.11290412627841e-09
3353 8.11037192960384e-09
3354 8.11100253628183e-09
3355 8.10950151475254e-09
3356 8.10877942569732e-09
3357 8.10862488265229e-09
3358 8.10735123479844e-09
3359 8.10719047450448e-09
3360 8.10641864745776e-09
3361 8.10525335737111e-09
3362 8.10433675724198e-09
3363 8.10381095561752e-09
3364 8.10370526238557e-09
3365 8.10246802984693e-09
3366 8.10191913558356e-09
3367 8.10105227344593e-09
3368 8.10032840803387e-09
3369 8.09931322010016e-09
3370 8.09718070371446e-09
3371 8.09718070371446e-09
3372 8.09668865286994e-09
3373 8.09589817407641e-09
3374 8.09541500501609e-09
3375 8.0940809610297e-09
3376 8.09357825204415e-09
3377 8.09285172209684e-09
3378 8.09217404196261e-09
3379 8.0909607902413e-09
3380 8.0904785093594e-09
3381 8.08955658015975e-09
3382 8.08868438895161e-09
3383 8.08790190376385e-09
3384 8.08570543853193e-09
3385 8.08560685072734e-09
3386 8.08495492776728e-09
3387 8.08429678755829e-09
3388 8.08354627679364e-09
3389 8.08230815607658e-09
3390 8.08165001586758e-09
3391 8.08113842509783e-09
3392 8.08032218913013e-09
3393 8.07902811317263e-09
3394 8.07788502754647e-09
3395 8.07682543069177e-09
3396 8.07471067787446e-09
3397 8.07437139371814e-09
3398 8.07362088295349e-09
3399 8.07288813575724e-09
3400 8.07229216803762e-09
3401 8.07153099913194e-09
3402 8.07055045015659e-09
3403 8.06965605448795e-09
3404 8.06895794625007e-09
3405 8.06690625410056e-09
3406 8.06687250332061e-09
3407 8.0662632129247e-09
3408 8.06536437636396e-09
3409 8.06467603808869e-09
3410 8.06379585327477e-09
3411 8.0631794574515e-09
3412 8.0624449338984e-09
3413 8.06029287758747e-09
3414 8.06014810450506e-09
3415 8.05976085871407e-09
3416 8.05900679523575e-09
3417 8.0583824058067e-09
3418 8.05767541578462e-09
3419 8.0571380678407e-09
3420 8.05602073938871e-09
3421 8.05540434356544e-09
3422 8.05344768650684e-09
3423 8.05333399966912e-09
3424 8.05226552103022e-09
3425 8.05210742527152e-09
3426 8.05093236522225e-09
3427 8.05062949638113e-09
3428 8.04994293446271e-09
3429 8.04760524886206e-09
3430 8.04742317228602e-09
3431 8.04702882106767e-09
3432 8.0465838436794e-09
3433 8.04571076429283e-09
3434 8.04506505858171e-09
3435 8.04418398558937e-09
3436 8.0437931870847e-09
3437 8.04161270906434e-09
3438 8.04113930996664e-09
3439 8.04107269658516e-09
3440 8.039487298106e-09
3441 8.03928390524788e-09
3442 8.03912048041866e-09
3443 8.03721622588682e-09
3444 8.03565214368973e-09
3445 8.03581379216212e-09
3446 8.03548960703893e-09
3447 8.03470090460223e-09
3448 8.03385447056826e-09
3449 8.03396371651388e-09
3450 8.03141464444934e-09
3451 8.03131960935843e-09
3452 8.03055311138223e-09
3453 8.03021737993959e-09
3454 8.0295192717017e-09
3455 8.02922794918004e-09
3456 8.02800492749611e-09
3457 8.02565391921917e-09
3458 8.02590705006878e-09
3459 8.02547273082155e-09
3460 8.02464850124807e-09
3461 8.02393618215547e-09
3462 8.0229503041096e-09
3463 8.0214039854809e-09
3464 8.02136401745202e-09
3465 8.02088706564064e-09
3466 8.02030974966783e-09
3467 8.01960364782417e-09
3468 8.018919750441e-09
3469 8.016503016961e-09
3470 8.01686450557781e-09
3471 8.01638666558802e-09
3472 8.01519739468404e-09
3473 8.01536081951326e-09
3474 8.01455879440027e-09
3475 8.01215094270447e-09
3476 8.01254707027965e-09
3477 8.01218202894916e-09
3478 8.01155142227117e-09
3479 8.01093147373422e-09
3480 8.0102768862389e-09
3481 8.00828559022193e-09
3482 8.00769672792967e-09
3483 8.00787702814887e-09
3484 8.00705368675381e-09
3485 8.00659272215398e-09
3486 8.00588839666716e-09
3487 8.00359245545224e-09
3488 8.00355515195861e-09
3489 8.00400190570372e-09
3490 8.00282240476236e-09
3491 8.00225663510901e-09
3492 7.99992072586519e-09
3493 8.00004151813027e-09
3494 7.99960719888304e-09
3495 7.99905652826283e-09
3496 7.99853516753046e-09
3497 7.9973014877055e-09
3498 7.99528532269278e-09
3499 7.99512278604197e-09
3500 7.99461208345065e-09
3501 7.9946413933385e-09
3502 7.99361021819323e-09
3503 7.99142796381602e-09
3504 7.9920265960709e-09
3505 7.99135246865035e-09
3506 7.99094745929096e-09
3507 7.99100163817457e-09
3508 7.98872434870646e-09
3509 7.98838151183645e-09
3510 7.98834420834282e-09
3511 7.98756349951191e-09
3512 7.98666821566485e-09
3513 7.98474886209988e-09
3514 7.98491406328594e-09
3515 7.98427546300218e-09
3516 7.98379229394186e-09
3517 7.98300092696991e-09
3518 7.98132671064877e-09
3519 7.98114552225115e-09
3520 7.97954857745253e-09
3521 7.98032306903451e-09
3522 7.98218557918062e-09
3523 7.98167931748139e-09
3524 7.98244670363601e-09
3525 7.98104249355447e-09
3526 7.98162513859779e-09
3527 7.98152299807953e-09
3528 7.97971200228176e-09
3529 7.97982036004896e-09
3530 7.97935850727072e-09
3531 7.97857069301244e-09
3532 7.97821542164456e-09
3533 7.9759132631807e-09
3534 7.97584220890712e-09
3535 7.97518673323339e-09
3536 7.97429233756475e-09
3537 7.97397525786891e-09
3538 7.97335175661829e-09
3539 7.97179833256223e-09
3540 7.97157451160047e-09
3541 7.97370258709407e-09
3542 7.96961963089871e-09
3543 7.96937893454697e-09
3544 7.96836907568377e-09
3545 7.96951482584518e-09
3546 7.96965249350023e-09
3547 7.96811860936941e-09
3548 7.96839838557162e-09
3549 7.96788413026661e-09
3550 7.96722332552235e-09
3551 7.96640708955465e-09
3552 7.96575516659459e-09
3553 7.96498600408313e-09
3554 7.96276911074756e-09
3555 7.96231081068299e-09
3556 7.96187915597102e-09
3557 7.96082133547316e-09
3558 7.9599260516261e-09
3559 7.95903432049272e-09
3560 7.9580777523347e-09
3561 7.95596566405266e-09
3562 7.95606602821408e-09
3563 7.95489629723534e-09
3564 7.95431542854885e-09
3565 7.9533482022498e-09
3566 7.95445309620391e-09
3567 7.95167931499918e-09
3568 7.94910981483099e-09
3569 7.94896504174858e-09
3570 7.95011168008841e-09
3571 7.94935139936115e-09
3572 7.94642218693298e-09
3573 7.94775800727621e-09
3574 7.94667442960417e-09
3575 7.94503041134931e-09
3576 7.94243426582852e-09
3577 7.94424348526945e-09
3578 7.94296450834509e-09
3579 7.94093679701291e-09
3580 7.93917553920664e-09
3581 7.93932830589483e-09
3582 7.9368112082534e-09
3583 7.9381949902313e-09
3584 7.9354629534123e-09
3585 7.93669485688042e-09
3586 7.93384735686686e-09
3587 7.93349919092634e-09
3588 7.9342097336621e-09
3589 7.92959831130702e-09
3590 7.9312654222008e-09
3591 7.92859733422802e-09
3592 7.92670196148038e-09
3593 7.92891885481595e-09
3594 7.92684584638437e-09
3595 7.92330911991712e-09
3596 7.92569210261718e-09
3597 7.92531817950248e-09
3598 7.92446908093325e-09
3599 7.92300181018391e-09
3600 7.92269627680753e-09
3601 7.92511123393069e-09
3602 7.92248400216522e-09
3603 7.92125298687552e-09
3604 7.91718690607013e-09
3605 7.91871990202253e-09
3606 7.91987542214656e-09
3607 7.91956900059176e-09
3608 7.91476750805487e-09
3609 7.91653764764533e-09
3610 7.91524001897415e-09
3611 7.91458898419251e-09
3612 7.91319276771674e-09
3613 7.91192178439815e-09
3614 7.91269627598012e-09
3615 7.91387222420781e-09
3616 7.91253906839984e-09
3617 7.90975640541092e-09
3618 7.91019072465815e-09
3619 7.90879450818238e-09
3620 7.90842591413821e-09
3621 7.90772425318664e-09
3622 7.90624543611784e-09
3623 7.90432164166077e-09
3624 7.90429854902186e-09
3625 7.90468757116969e-09
3626 7.90230725300489e-09
3627 7.9035880062861e-09
3628 7.90215626267354e-09
3629 7.89954768265488e-09
3630 7.8999677910474e-09
3631 7.89953347180017e-09
3632 7.89847121041021e-09
3633 7.8974293771239e-09
3634 7.89674192702705e-09
3635 7.89595677730404e-09
3636 7.89363596709336e-09
3637 7.8934485614468e-09
3638 7.89262166733806e-09
3639 7.89180187865668e-09
3640 7.89083465235763e-09
3641 7.88998910650207e-09
3642 7.88880782920387e-09
3643 7.88819232155902e-09
3644 7.886992392514e-09
3645 7.88604381796176e-09
3646 7.88548604191419e-09
3647 7.88295650977489e-09
3648 7.88255949402128e-09
3649 7.88193776912749e-09
3650 7.88128406981059e-09
3651 7.88060905421162e-09
3652 7.87974308025241e-09
3653 7.8772899314572e-09
3654 7.87732723495083e-09
3655 7.87631559973079e-09
3656 7.87573561922272e-09
3657 7.87526666101712e-09
3658 7.87438292348952e-09
3659 7.87190757023382e-09
3660 7.87146614555922e-09
3661 7.87109843969347e-09
3662 7.87060194795686e-09
3663 7.86978660016757e-09
3664 7.87219267550654e-09
3665 7.86985498990589e-09
3666 7.86991893875211e-09
3667 7.86964093890674e-09
3668 7.86898635141142e-09
3669 7.86824472243097e-09
3670 7.86619125392463e-09
3671 7.86563791876915e-09
3672 7.86517340145565e-09
3673 7.8647923729136e-09
3674 7.86381271211667e-09
3675 7.86171483468934e-09
3676 7.86164378041576e-09
3677 7.86090126325689e-09
3678 7.86035769806404e-09
3679 7.85972265049395e-09
3680 7.85819853632574e-09
3681 7.85651632639883e-09
3682 7.8572179873504e-09
3683 7.85665310587547e-09
3684 7.85119791402167e-09
3685 7.84845344270479e-09
3686 7.84768516837175e-09
3687 7.84681475352045e-09
3688 7.84386156027495e-09
3689 7.84403297870995e-09
3690 7.84361020578217e-09
3691 7.8429760463905e-09
3692 7.84051312763268e-09
3693 7.84046783053327e-09
3694 7.84005660392495e-09
3695 7.83925813152564e-09
3696 7.83697196027333e-09
3697 7.83696219031071e-09
3698 7.83610598631412e-09
3699 7.83580578200826e-09
3700 7.83341747023769e-09
3701 7.83348230726233e-09
3702 7.83300357909411e-09
3703 7.83243336854866e-09
3704 7.83011433469483e-09
3705 7.83014630911794e-09
3706 7.82903786245015e-09
3707 7.82903164520121e-09
3708 7.82825182454872e-09
3709 7.82627118667278e-09
3710 7.82609088645358e-09
3711 7.82576670133039e-09
3712 7.82523468245699e-09
3713 7.82286502243323e-09
3714 7.82269715671191e-09
3715 7.82268738674929e-09
3716 7.822015923864e-09
3717 7.81988163112146e-09
3718 7.81958497952928e-09
3719 7.81914444303311e-09
3720 7.81861331233813e-09
3721 7.81709363906202e-09
3722 7.81811948513678e-09
3723 7.81783793257773e-09
3724 7.81843301211893e-09
3725 7.81709541541886e-09
3726 7.81712383712829e-09
3727 7.81625519863383e-09
3728 7.81604558852678e-09
3729 7.81410314232289e-09
3730 7.81360043333734e-09
3731 7.81329578813938e-09
3732 7.81309772435179e-09
3733 7.81264208882249e-09
3734 7.81055486953619e-09
3735 7.81023246076984e-09
3736 7.80992248650136e-09
3737 7.80890552221081e-09
3738 7.80692932522697e-09
3739 7.80648790055238e-09
3740 7.80581466131025e-09
3741 7.80520714727118e-09
3742 7.80285525081581e-09
3743 7.80292541691097e-09
3744 7.80242714881751e-09
3745 7.801863155521e-09
3746 7.79977860076997e-09
3747 7.79972442188637e-09
3748 7.79918440940719e-09
3749 7.79835218622793e-09
3750 7.79774023129676e-09
3751 7.79584130583544e-09
3752 7.79582531862388e-09
3753 7.79482434154488e-09
3754 7.79474618184395e-09
3755 7.79259679006827e-09
3756 7.79293962693828e-09
3757 7.79237208092809e-09
3758 7.79159758934611e-09
3759 7.78984166061036e-09
3760 7.78965691949907e-09
3761 7.78890285602074e-09
3762 7.78847208948719e-09
3763 7.7867010617183e-09
3764 7.78670194989672e-09
3765 7.78603581608195e-09
3766 7.78557307512528e-09
3767 7.78422837299786e-09
3768 7.78361197717459e-09
3769 7.78314923621792e-09
3770 7.78259590106245e-09
3771 7.78081066243885e-09
3772 7.78064901396647e-09
3773 7.7797759345799e-09
3774 7.7797714936878e-09
3775 7.7779001017575e-09
3776 7.77787612094016e-09
3777 7.77739028734459e-09
3778 7.77682007679914e-09
3779 7.77464315149246e-09
3780 7.77484654435057e-09
3781 7.77439090882126e-09
3782 7.77371766957913e-09
3783 7.7724271463353e-09
3784 7.77185338307618e-09
3785 7.77142350472104e-09
3786 7.77117126204985e-09
3787 7.76910535904562e-09
3788 7.76909470090459e-09
3789 7.76861419637953e-09
3790 7.76813990910341e-09
3791 7.76617525843903e-09
3792 7.7662232200737e-09
3793 7.76580577621644e-09
3794 7.7598212300245e-09
3795 7.75888153725646e-09
3796 7.75883179926495e-09
3797 7.75855646395485e-09
3798 7.75786102025222e-09
3799 7.7556103761367e-09
3800 7.75569564126499e-09
3801 7.75637865046974e-09
3802 7.75552688736525e-09
3803 7.75344144443579e-09
3804 7.75293074184447e-09
3805 7.75331177038652e-09
3806 7.75136932418263e-09
3807 7.75132491526165e-09
3808 7.75115527318349e-09
3809 7.75014363796345e-09
3810 7.74834685302039e-09
3811 7.74895614341631e-09
3812 7.74841968365081e-09
3813 7.74781927503909e-09
3814 7.74574449025067e-09
3815 7.74578889917166e-09
3816 7.74575426021329e-09
3817 7.74321851082505e-09
3818 7.7431128175931e-09
3819 7.74292541194654e-09
3820 7.74229036437646e-09
3821 7.74040653794827e-09
3822 7.7406285825532e-09
3823 7.74018982241387e-09
3824 7.73875008519553e-09
3825 7.73842323553708e-09
3826 7.73816921650905e-09
3827 7.73817010468747e-09
3828 7.73609620807747e-09
3829 7.7365305273247e-09
3830 7.73613884064162e-09
3831 7.73410668841734e-09
3832 7.73445485435786e-09
3833 7.73400898879117e-09
3834 7.73291386479968e-09
3835 7.73279573706986e-09
3836 7.73310571133834e-09
3837 7.73284281052611e-09
3838 7.73120856223386e-09
3839 7.73111175078611e-09
3840 7.73088881800277e-09
3841 7.73058950187533e-09
3842 7.72861508124834e-09
3843 7.72887176481163e-09
3844 7.72863728570883e-09
3845 7.72702613005549e-09
3846 7.72701103102236e-09
3847 7.72678099281165e-09
3848 7.72498509604702e-09
3849 7.72499930690174e-09
3850 7.72481101307676e-09
3851 7.72290942308018e-09
3852 7.72295116746591e-09
3853 7.72266339765793e-09
3854 7.72093589063161e-09
3855 7.72122010772591e-09
3856 7.7177162438602e-09
3857 7.71913732933172e-09
3858 7.71938868382449e-09
3859 7.71553754219667e-09
3860 7.71723662751356e-09
3861 7.71411468036831e-09
3862 7.713974348178e-09
3863 7.71214647699026e-09
3864 7.71231789542526e-09
3865 7.7119501895595e-09
3866 7.71153896295118e-09
3867 7.71282238076765e-09
3868 7.70961516849411e-09
3869 7.70950148165639e-09
3870 7.70777752734375e-09
3871 7.70777486280849e-09
3872 7.70746488854002e-09
3873 7.70563879370911e-09
3874 7.70604913213901e-09
3875 7.70546471073885e-09
3876 7.70377983627668e-09
3877 7.70390151672018e-09
3878 7.70328423271849e-09
3879 7.70163577357152e-09
3880 7.7019954858315e-09
3881 7.70144481521129e-09
3882 7.70113750547807e-09
3883 7.69918173659789e-09
3884 7.69911601139484e-09
3885 7.69901742359025e-09
3886 7.67033014881235e-09
3887 7.66995178480556e-09
3888 7.66926611106555e-09
3889 7.6672304061276e-09
3890 7.66747554337144e-09
3891 7.66726593326439e-09
3892 7.66553931441649e-09
3893 7.66567165300103e-09
3894 7.66567698207155e-09
3895 7.66356134107582e-09
3896 7.66373986493818e-09
3897 7.66338548174872e-09
3898 7.66169439003761e-09
3899 7.66196084356352e-09
3900 7.6614234956196e-09
3901 7.65981145178785e-09
3902 7.6598141163231e-09
3903 7.65956365000875e-09
3904 7.65784147205295e-09
3905 7.65807417479891e-09
3906 7.6576993635058e-09
3907 7.65602869989834e-09
3908 7.65606156249987e-09
3909 7.65599850183207e-09
3910 7.65435359539879e-09
3911 7.65447971673439e-09
3912 7.65424790216684e-09
3913 7.65262875290773e-09
3914 7.65250263157213e-09
3915 7.65203012065285e-09
3916 7.65052732276672e-09
3917 7.6506481150318e-09
3918 7.65029550819918e-09
3919 7.64871721514737e-09
3920 7.64881580295196e-09
3921 7.64825980326123e-09
3922 7.64680407883134e-09
3923 7.64683782961129e-09
3924 7.6465003218118e-09
3925 7.64473995218395e-09
3926 7.64489715976424e-09
3927 7.64458718549577e-09
3928 7.64287388932416e-09
3929 7.64310037482119e-09
3930 7.64275043252383e-09
3931 7.6409412130829e-09
3932 7.64115437590362e-09
3933 7.64095720029445e-09
3934 7.6392732140107e-09
3935 7.63947749504723e-09
3936 7.63906360390365e-09
3937 7.63729701702687e-09
3938 7.63739116393936e-09
3939 7.63703145167938e-09
3940 7.63532437275671e-09
3941 7.63563789973887e-09
3942 7.63552598925799e-09
3943 7.63344498722063e-09
3944 7.63373630974229e-09
3945 7.63339258469387e-09
3946 7.63166063677545e-09
3947 7.63194307751291e-09
3948 7.63170415751802e-09
3949 7.62979723845092e-09
3950 7.62998197956222e-09
3951 7.62952723221133e-09
3952 7.62772867091144e-09
3953 7.62794893915952e-09
3954 7.62788143759963e-09
3955 7.62612462068546e-09
3956 7.62615037785963e-09
3957 7.62604113191401e-09
3958 7.6242914204272e-09
3959 7.62429941403298e-09
3960 7.62218643757251e-09
3961 7.62251417540938e-09
3962 7.622623421355e-09
3963 7.62113749885884e-09
3964 7.62122365216555e-09
3965 7.62128848919019e-09
3966 7.61900409429472e-09
3967 7.61905205592939e-09
3968 7.61936735926838e-09
3969 7.61729079812312e-09
3970 7.61744622934657e-09
3971 7.61747553923442e-09
3972 7.61514851177481e-09
3973 7.61535989823869e-09
3974 7.61500018597872e-09
3975 7.6133810367196e-09
3976 7.61360130496769e-09
3977 7.61334373322597e-09
3978 7.61154872463976e-09
3979 7.61165619422854e-09
3980 7.6090751477409e-09
3981 7.60953611234072e-09
3982 7.60879448336027e-09
3983 7.60697016488621e-09
3984 7.60733342985986e-09
3985 7.60691953871628e-09
3986 7.60488294559991e-09
3987 7.60684404355061e-09
3988 7.60764340412834e-09
3989 7.60616547523796e-09
3990 7.60672769217763e-09
3991 7.60659712994993e-09
3992 7.60493534812667e-09
3993 7.60401874799754e-09
3994 7.60439800018275e-09
3995 7.60243068498312e-09
3996 7.60272644839688e-09
3997 7.60277796274522e-09
3998 7.60129115207064e-09
3999 7.60204343919213e-09
4000 7.60213314521252e-09
4001 7.60042961900353e-09
4002 7.60109486463989e-09
4003 7.60086926732129e-09
4004 7.59914176029497e-09
4005 7.5995130188744e-09
4006 7.59918172832386e-09
4007 7.59749330114801e-09
4008 7.59805018901716e-09
4009 7.59784501980221e-09
4010 7.59642571068753e-09
4011 7.59666018979033e-09
4012 7.59624718682517e-09
4013 7.59476570522111e-09
4014 7.59516716186681e-09
4015 7.59491758373088e-09
4016 7.59317231313617e-09
4017 7.59332952071645e-09
4018 7.59312346332308e-09
4019 7.59148921503083e-09
4020 7.59179563658563e-09
4021 7.59150609042081e-09
4022 7.58963114577682e-09
4023 7.58997131811157e-09
4024 7.590585937578e-09
4025 7.58880691620334e-09
4026 7.58924212362899e-09
4027 7.58890283947267e-09
4028 7.5872135241184e-09
4029 7.587511063889e-09
4030 7.58732010552876e-09
4031 7.58567342273864e-09
4032 7.585365224827e-09
4033 7.58510765308529e-09
4034 7.58351248464351e-09
4035 7.58366258679644e-09
4036 7.58365814590434e-09
4037 7.58319540494767e-09
4038 7.58232143738269e-09
4039 7.58171569970045e-09
4040 7.58145102253138e-09
4041 7.5804420518466e-09
4042 7.58034346404202e-09
4043 7.58030527236997e-09
4044 7.5783184172451e-09
4045 7.57853335642267e-09
4046 7.57820739494264e-09
4047 7.57657225847197e-09
4048 7.57674367690697e-09
4049 7.57662377282031e-09
4050 7.57479146074047e-09
4051 7.57537321760537e-09
4052 7.57494333925024e-09
4053 7.57343521229359e-09
4054 7.57351159563768e-09
4055 7.57326379385859e-09
4056 7.57156293218486e-09
4057 7.57179297039556e-09
4058 7.57146523255869e-09
4059 7.56982743155277e-09
4060 7.56990647943212e-09
4061 7.56982032612541e-09
4062 7.5681922950821e-09
4063 7.56835749626816e-09
4064 7.5680794964228e-09
4065 7.56647722255366e-09
4066 7.56653140143726e-09
4067 7.56641327370744e-09
4068 7.56468399032428e-09
4069 7.56493623299548e-09
4070 7.564869619614e-09
4071 7.56279749936084e-09
4072 7.56327622752906e-09
4073 7.56293516701589e-09
4074 7.56120321909748e-09
4075 7.56135598578567e-09
4076 7.56130624779416e-09
4077 7.55926699014253e-09
4078 7.55978746269648e-09
4079 7.55939311147813e-09
4080 7.55772422422751e-09
4081 7.5580706138112e-09
4082 7.55778728489531e-09
4083 7.55614415481887e-09
4084 7.55649232075939e-09
4085 7.55630669146967e-09
4086 7.55445395128618e-09
4087 7.5546129352233e-09
4088 7.55425588749858e-09
4089 7.55286766462859e-09
4090 7.55300177956997e-09
4091 7.55285611830914e-09
4092 7.55104689886821e-09
4093 7.55136131402878e-09
4094 7.55079465619701e-09
4095 7.54935847169236e-09
4096 7.54978302097697e-09
4097 7.54926876567197e-09
4098 7.54765139276969e-09
4099 7.54848805684105e-09
4100 7.54759899024293e-09
4101 7.5458661541461e-09
4102 7.54667617286486e-09
4103 7.54605622432791e-09
4104 7.54483053810873e-09
4105 7.54439177796939e-09
4106 7.54483053810873e-09
4107 7.54302931227357e-09
4108 7.54263052016313e-09
4109 7.54308260297876e-09
4110 7.54144657832967e-09
4111 7.54150164539169e-09
4112 7.54155848881055e-09
4113 7.53972884126597e-09
4114 7.5391577425421e-09
4115 7.53950679666104e-09
4116 7.53731654867806e-09
4117 7.53798801156336e-09
4118 7.53733253588962e-09
4119 7.53624185279023e-09
4120 7.53592566127281e-09
4121 7.53614060045038e-09
4122 7.53393170072059e-09
4123 7.53458540003749e-09
4124 7.53412265908082e-09
4125 7.53266249375883e-09
4126 7.53239426387609e-09
4127 7.53284368215645e-09
4128 7.5305779390078e-09
4129 7.53139151044024e-09
4130 7.52910889190161e-09
4131 7.53006190734595e-09
4132 7.52929096847765e-09
4133 7.52751638799509e-09
4134 7.52777751245048e-09
4135 7.52766915468328e-09
4136 7.52601714282264e-09
4137 7.52619033761448e-09
4138 7.52577822282774e-09
4139 7.52358175759582e-09
4140 7.52369722079038e-09
4141 7.52348672250491e-09
4142 7.52190132402575e-09
4143 7.52205497889236e-09
4144 7.52159756700621e-09
4145 7.52008411097904e-09
4146 7.5200414784149e-09
4147 7.5185626613461e-09
4148 7.51870476989325e-09
4149 7.51831752410226e-09
4150 7.51659179343278e-09
4151 7.51527373665795e-09
4152 7.51377182695023e-09
4153 7.51149276112528e-09
4154 7.51156736811254e-09
4155 7.51085948991204e-09
4156 7.50904050050849e-09
4157 7.5088921747124e-09
4158 7.508531574274e-09
4159 7.50794626469542e-09
4160 7.50630135826214e-09
4161 7.50637774160623e-09
4162 7.50597184406843e-09
4163 7.50519202341593e-09
4164 7.50508188929189e-09
4165 7.50338813304552e-09
4166 7.50328332799199e-09
4167 7.50282058703533e-09
4168 7.50277440175751e-09
4169 7.50233475343975e-09
4170 7.50016226902517e-09
4171 7.50036033281276e-09
4172 7.50028039675499e-09
4173 7.49983097847462e-09
4174 7.49925810339391e-09
4175 7.49733342075842e-09
4176 7.49756434714755e-09
4177 7.49724904380855e-09
4178 7.49695772128689e-09
4179 7.49630224561315e-09
4180 7.49436779301504e-09
4181 7.49469641903033e-09
4182 7.49433937130561e-09
4183 7.49365547392244e-09
4184 7.49343875838804e-09
4185 7.49162509805501e-09
4186 7.49148298950786e-09
4187 7.49129736021814e-09
4188 7.4908266256557e-09
4189 7.48901740621477e-09
4190 7.48936646033371e-09
4191 7.48878115075513e-09
4192 7.48839390496414e-09
4193 7.488107911513e-09
4194 7.48595496702364e-09
4195 7.48618678159119e-09
4196 7.48603667943826e-09
4197 7.48550910145696e-09
4198 7.48374606729385e-09
4199 7.483888175841e-09
4200 7.48365103220294e-09
4201 7.48311634879428e-09
4202 7.48145101425735e-09
4203 7.48155226659719e-09
4204 7.48112771731257e-09
4205 7.48077866319363e-09
4206 7.47917017207556e-09
4207 7.47923678545703e-09
4208 7.47894279840011e-09
4209 7.47835304792943e-09
4210 7.47681028201441e-09
4211 7.47711226267711e-09
4212 7.47668682521407e-09
4213 7.47642392440184e-09
4214 7.4744290756712e-09
4215 7.47454809157944e-09
4216 7.47441397663806e-09
4217 7.47394501843246e-09
4218 7.47236938991591e-09
4219 7.47243156240529e-09
4220 7.47201767126171e-09
4221 7.47164818903912e-09
4222 7.47006989598731e-09
4223 7.46999528900005e-09
4224 7.46978479071458e-09
4225 7.46938155771204e-09
4226 7.4676531625073e-09
4227 7.4678423445107e-09
4228 7.46751016578173e-09
4229 7.46715134170017e-09
4230 7.46537853757445e-09
4231 7.465781770577e-09
4232 7.46536077400606e-09
4233 7.46501971349289e-09
4234 7.46340766966114e-09
4235 7.46335793166963e-09
4236 7.46307726728901e-09
4237 7.46277972751841e-09
4238 7.46108774762888e-09
4239 7.46127692963228e-09
4240 7.46130712769855e-09
4241 7.45962935866373e-09
4242 7.45976613814037e-09
4243 7.45959471970536e-09
4244 7.45805373014718e-09
4245 7.45800576851252e-09
4246 7.45776063126868e-09
4247 7.45756345565951e-09
4248 7.45596029361195e-09
4249 7.45618056186004e-09
4250 7.45614059383115e-09
4251 7.45436512517017e-09
4252 7.45462980233924e-09
4253 7.45435624338597e-09
4254 7.45411554703423e-09
4255 7.45223172060605e-09
4256 7.45271133695269e-09
4257 7.45243955435626e-09
4258 7.45073602814728e-09
4259 7.45117656464345e-09
4260 7.45093498011329e-09
4261 7.44906047955851e-09
4262 7.44950412467915e-09
4263 7.44908623673268e-09
4264 7.4476038669502e-09
4265 7.44769046434612e-09
4266 7.44759143245233e-09
4267 7.44593675605643e-09
4268 7.44610861858064e-09
4269 7.44595274326798e-09
4270 7.44553485532151e-09
4271 7.44396677632153e-09
4272 7.44423100940139e-09
4273 7.44336814406665e-09
4274 7.44234540661637e-09
4275 7.44271488883896e-09
4276 7.44268691121874e-09
4277 7.44076222858325e-09
4278 7.44134087682369e-09
4279 7.44097672367161e-09
4280 7.43950723247622e-09
4281 7.43958805671241e-09
4282 7.4395019034057e-09
4283 7.43761585653147e-09
4284 7.43816563897326e-09
4285 7.43784811518822e-09
4286 7.43596917374134e-09
4287 7.43640438116699e-09
4288 7.4358728063828e-09
4289 7.43442640782632e-09
4290 7.43482608811519e-09
4291 7.43470618402853e-09
4292 7.43296846295038e-09
4293 7.43329753305488e-09
4294 7.43313011142277e-09
4295 7.43135286640495e-09
4296 7.43146566506425e-09
4297 7.4315176235018e-09
4298 7.42998373937098e-09
4299 7.43026173921635e-09
4300 7.42925587715604e-09
4301 7.42836503420108e-09
4302 7.42872297010422e-09
4303 7.42848049739564e-09
4304 7.42654515661911e-09
4305 7.42712424894876e-09
4306 7.42704120426652e-09
4307 7.42552241916883e-09
4308 7.42535277709067e-09
4309 7.42542116682898e-09
4310 7.42377093132518e-09
4311 7.42399475228694e-09
4312 7.42389838492841e-09
4313 7.42207761916802e-09
4314 7.42247729945689e-09
4315 7.42220818139572e-09
4316 7.41981898144672e-09
4317 7.42102246320542e-09
4318 7.42060279890211e-09
4319 7.41911110324622e-09
4320 7.41922567826236e-09
4321 7.41907779655548e-09
4322 7.41771133405678e-09
4323 7.41782635316213e-09
4324 7.41648431556996e-09
4325 7.41667793846545e-09
4326 7.41632444345441e-09
4327 7.41483452415537e-09
4328 7.414971303632e-09
4329 7.41476613441705e-09
4330 7.41332151221741e-09
4331 7.4135466654468e-09
4332 7.4132668892446e-09
4333 7.41180450347656e-09
4334 7.41209671417664e-09
4335 7.4118919890509e-09
4336 7.41032257778329e-09
4337 7.41042827101523e-09
4338 7.4104717917578e-09
4339 7.40794670051059e-09
4340 7.40895922390905e-09
4341 7.40670325072301e-09
4342 7.40765671025656e-09
4343 7.40782590824551e-09
4344 7.40583150360408e-09
4345 7.40644034991078e-09
4346 7.4060624299932e-09
4347 7.40386285613681e-09
4348 7.40471284288446e-09
4349 7.40261452136792e-09
4350 7.4036097252872e-09
4351 7.40358174766698e-09
4352 7.40182937164491e-09
4353 7.40233785379019e-09
4354 7.40204741944694e-09
4355 7.40044692193464e-09
4356 7.40064587390066e-09
4357 7.39851291342575e-09
4358 7.3992367788378e-09
4359 7.39860972487349e-09
4360 7.39778283076475e-09
4361 7.39801686577835e-09
4362 7.39775440905532e-09
4363 7.39474881328306e-09
4364 7.39622008083529e-09
4365 7.39495265023038e-09
4366 7.39524308457362e-09
4367 7.39411198935613e-09
4368 7.39223615653373e-09
4369 7.39237737690246e-09
4370 7.39213623646151e-09
4371 7.39073513500443e-09
4372 7.39081151834853e-09
4373 7.38943395361957e-09
4374 7.3896200269985e-09
4375 7.38781080755757e-09
4376 7.38560990143355e-09
4377 7.38585015369608e-09
4378 7.38581640291613e-09
4379 7.38360173002661e-09
4380 7.38388639121013e-09
4381 7.38344496653554e-09
4382 7.38238048469952e-09
4383 7.38248395748542e-09
4384 7.38252436960352e-09
4385 7.38450856019313e-09
4386 7.38048377968425e-09
4387 7.38577643488725e-09
4388 7.38585814730186e-09
4389 7.38455208093569e-09
4390 7.38122807319996e-09
4391 7.38145899958909e-09
4392 7.38026395552538e-09
4393 7.38060190741407e-09
4394 7.3803323452637e-09
4395 7.3798545052739e-09
4396 7.37824779051266e-09
4397 7.37843786069448e-09
4398 7.37823890872846e-09
4399 7.37643501835805e-09
4400 7.37642169568176e-09
4401 7.37615168944217e-09
4402 7.37359329150422e-09
4403 7.37418659468858e-09
4404 7.37420347007856e-09
4405 7.37138972084495e-09
4406 7.37207805912021e-09
4407 7.37195815503355e-09
4408 7.3702333125425e-09
4409 7.37023198027487e-09
4410 7.37012184615082e-09
4411 7.36822336477871e-09
4412 7.36823047020607e-09
4413 7.36725169758756e-09
4414 7.36589234051621e-09
4415 7.36611394103193e-09
4416 7.3657657750914e-09
4417 7.36330418860121e-09
4418 7.36387351096823e-09
4419 7.36365679543383e-09
4420 7.36153360492153e-09
4421 7.3619421669946e-09
4422 7.36043004323506e-09
4423 7.3606618578026e-09
4424 7.35958538555792e-09
4425 7.35838368015607e-09
4426 7.35808214358258e-09
4427 7.35727789802354e-09
4428 7.3560499913583e-09
4429 7.35587057931752e-09
4430 7.35455829570242e-09
4431 7.3544716983065e-09
4432 7.35441130217396e-09
4433 7.35217797753762e-09
4434 7.35381844307881e-09
4435 7.35257010830992e-09
4436 7.35225968995223e-09
4437 7.35250305083923e-09
4438 7.35031591148072e-09
4439 7.35099536797179e-09
4440 7.34887750653002e-09
4441 7.34879312958014e-09
4442 7.34907823485287e-09
4443 7.34752658715365e-09
4444 7.34728233808823e-09
4445 7.34541050206872e-09
4446 7.34605531960142e-09
4447 7.34386729206449e-09
4448 7.34475502639498e-09
4449 7.3441293046983e-09
4450 7.34195992890818e-09
4451 7.34240401811803e-09
4452 7.3412258494443e-09
4453 7.34114546929732e-09
4454 7.34000549229563e-09
4455 7.34067073793199e-09
4456 7.33962179921832e-09
4457 7.33767269167629e-09
4458 7.33796845509005e-09
4459 7.33699412336364e-09
4460 7.33700700195072e-09
4461 7.33552996123876e-09
4462 7.33568361610537e-09
4463 7.33494953664149e-09
4464 7.33316607437473e-09
4465 7.33420746357183e-09
4466 7.33212868198052e-09
4467 7.3330070904376e-09
4468 7.33100291583355e-09
4469 7.33232985439258e-09
4470 7.32905958145125e-09
4471 7.33093141747077e-09
4472 7.32906979550307e-09
4473 7.32982030626772e-09
4474 7.32917548873502e-09
4475 7.32697724714626e-09
4476 7.32772820200012e-09
4477 7.32689242610718e-09
4478 7.32679250603496e-09
4479 7.32485005983108e-09
4480 7.32526173052861e-09
4481 7.32448279805453e-09
4482 7.32423810489991e-09
4483 7.32381000290161e-09
4484 7.32195148955839e-09
4485 7.32320915020068e-09
4486 7.32058413888126e-09
4487 7.32194704866629e-09
4488 7.32000149383794e-09
4489 7.32097538147514e-09
4490 7.31814120413787e-09
4491 7.31975102752358e-09
4492 7.31760030348028e-09
4493 7.31790450458902e-09
4494 7.31660554365021e-09
4495 7.31755589455929e-09
4496 7.31502503015236e-09
4497 7.31546778709458e-09
4498 7.31500726658396e-09
4499 7.31416704979893e-09
4500 7.31510452212092e-09
4501 7.31290050737243e-09
4502 7.31321048164091e-09
4503 7.3118080479162e-09
4504 7.31267446596462e-09
4505 7.31015115107425e-09
4506 7.31153715349819e-09
4507 7.30954807792727e-09
4508 7.31003568787969e-09
4509 7.30903382262227e-09
4510 7.30897120604368e-09
4511 7.30740934429264e-09
4512 7.30782101499017e-09
4513 7.30712690355517e-09
4514 7.3058070704235e-09
4515 7.30502058843285e-09
4516 7.30557836448043e-09
4517 7.30483140642946e-09
4518 7.30445171015504e-09
4519 7.30288940431478e-09
4520 7.30325355746686e-09
4521 7.30383220570729e-09
4522 7.30105842450257e-09
4523 7.30243066016101e-09
4524 7.30014804162238e-09
4525 7.30156424211259e-09
4526 7.29940730082035e-09
4527 7.30004190430122e-09
4528 7.29829530143888e-09
4529 7.29976745716954e-09
4530 7.29697990919931e-09
4531 7.29831262091807e-09
4532 7.29646210118062e-09
4533 7.29668414578555e-09
4534 7.29535853949415e-09
4535 7.29576887792405e-09
4536 7.2941213069555e-09
4537 7.29532745324946e-09
4538 7.29276061761652e-09
4539 7.29426385959187e-09
4540 7.29150295697423e-09
4541 7.2931518602104e-09
4542 7.29137683563863e-09
4543 7.29160198886802e-09
4544 7.29034654867178e-09
4545 7.29117521913736e-09
4546 7.28870386268454e-09
4547 7.28978610808895e-09
4548 7.28854754328268e-09
4549 7.28907334490714e-09
4550 7.28743021483069e-09
4551 7.28777038716544e-09
4552 7.28672233663019e-09
4553 7.28706162078652e-09
4554 7.28569249375255e-09
4555 7.28612636891057e-09
4556 7.28444238262682e-09
4557 7.2857084809641e-09
4558 7.28343074740678e-09
4559 7.28414262241017e-09
4560 7.28275129091571e-09
4561 7.28335436406269e-09
4562 7.28199456290213e-09
4563 7.28230853397349e-09
4564 7.28105442604488e-09
4565 7.28134263994207e-09
4566 7.27921767307294e-09
4567 7.28019333706698e-09
4568 7.27908711084524e-09
4569 7.27949434065067e-09
4570 7.27820870238816e-09
4571 7.27828242119699e-09
4572 7.27640347975012e-09
4573 7.2779942072998e-09
4574 7.27616455975522e-09
4575 7.27648341580789e-09
4576 7.27529858579601e-09
4577 7.27550242274333e-09
4578 7.27443483228285e-09
4579 7.27409510403731e-09
4580 7.27317761572976e-09
4581 7.27363147490223e-09
4582 7.27245419440692e-09
4583 7.27281479484532e-09
4584 7.27147542178841e-09
4585 7.27204074735255e-09
4586 7.27071958195324e-09
4587 7.27127869026845e-09
4588 7.26958759855734e-09
4589 7.26997084754544e-09
4590 7.26880555745879e-09
4591 7.26906046466524e-09
4592 7.26726723243587e-09
4593 7.26778415227614e-09
4594 7.26653004434752e-09
4595 7.26726279154377e-09
4596 7.26625692948346e-09
4597 7.26572046971796e-09
4598 7.26494464586835e-09
4599 7.26518445404167e-09
4600 7.26356441660414e-09
4601 7.26331261802216e-09
4602 7.26300131148605e-09
4603 7.2632513337112e-09
4604 7.26189863797799e-09
4605 7.26321491839599e-09
4606 7.26094251390919e-09
4607 7.26226057068402e-09
4608 7.25951210256426e-09
4609 7.26009474760758e-09
4610 7.25939264256681e-09
4611 7.25989934835525e-09
4612 7.25862214778772e-09
4613 7.25892945752094e-09
4614 7.25730142647762e-09
4615 7.2571291198642e-09
4616 7.25655890931876e-09
4617 7.25665838530176e-09
4618 7.25583992888801e-09
4619 7.25613569230177e-09
4620 7.25505344689736e-09
4621 7.25555526770449e-09
4622 7.25620674657534e-09
4623 7.25750970431704e-09
4624 7.25552684599506e-09
4625 7.25718152239097e-09
4626 7.25553261915479e-09
4627 7.25632220976991e-09
4628 7.25476567708938e-09
4629 7.25539539558895e-09
4630 7.25340276730435e-09
4631 7.25438109583365e-09
4632 7.25355020492202e-09
4633 7.25451920757791e-09
4634 7.25338811236043e-09
4635 7.25376869681327e-09
4636 7.25157622838424e-09
4637 7.252322742346e-09
4638 7.25104909449215e-09
4639 7.25160687053972e-09
4640 7.24973192589573e-09
4641 7.24994775325172e-09
4642 7.24891346948198e-09
4643 7.24948812091952e-09
4644 7.24814297470289e-09
4645 7.24787208028488e-09
4646 7.24686888275983e-09
4647 7.24676763041998e-09
4648 7.24615256686434e-09
4649 7.24589188649816e-09
4650 7.24488602443785e-09
4651 7.24551352249136e-09
4652 7.2441443954574e-09
4653 7.24452586808866e-09
4654 7.24264426210652e-09
4655 7.24294135778791e-09
4656 7.24187199097059e-09
4657 7.24380377903344e-09
4658 7.24130666540646e-09
4659 7.24289916931298e-09
4660 7.24103177418556e-09
4661 7.24154558540135e-09
4662 7.23922388701226e-09
4663 7.23980786432321e-09
4664 7.23872872754328e-09
4665 7.24053172973527e-09
4666 7.23892457088482e-09
4667 7.23924875600801e-09
4668 7.23725745999104e-09
4669 7.23824511439375e-09
4670 7.23653936773871e-09
4671 7.23684934200719e-09
4672 7.23595494633855e-09
4673 7.23640081190524e-09
4674 7.23546422776167e-09
4675 7.23567383786872e-09
4676 7.23298487770307e-09
4677 7.2345325285994e-09
4678 7.23261051049917e-09
4679 7.23353110743119e-09
4680 7.23200743735219e-09
4681 7.23272863822899e-09
4682 7.23180892947539e-09
4683 7.23193682716783e-09
4684 7.23096471588747e-09
4685 7.23067028474134e-09
4686 7.22908843897585e-09
4687 7.23012849590532e-09
4688 7.22843029876685e-09
4689 7.22929627272606e-09
4690 7.22814696985097e-09
4691 7.22849202716702e-09
4692 7.22613524573035e-09
4693 7.22734938563008e-09
4694 7.22634263539135e-09
4695 7.22657444995889e-09
4696 7.22490200999459e-09
4697 7.22586479540155e-09
4698 7.22464132962841e-09
4699 7.22470705483147e-09
4700 7.22321669144321e-09
4701 7.22348714177201e-09
4702 7.22290716126395e-09
4703 7.22309145828604e-09
4704 7.22132620367688e-09
4705 7.2221024716157e-09
4706 7.22104998018835e-09
4707 7.22151760612633e-09
4708 7.21919946045091e-09
4709 7.2203034662266e-09
4710 7.21959692029372e-09
4711 7.21988113738803e-09
4712 7.21874426901081e-09
4713 7.21907111866926e-09
4714 7.21779347401252e-09
4715 7.21797199787488e-09
4716 7.2158177211179e-09
4717 7.21758697252994e-09
4718 7.21550152960049e-09
4719 7.21606907561068e-09
4720 7.21520665436515e-09
4721 7.21556947524959e-09
4722 7.21350090771011e-09
4723 7.2136625561825e-09
4724 7.21252080282397e-09
4725 7.21254700408736e-09
4726 7.21164505890215e-09
4727 7.21165394068635e-09
4728 7.21088788679936e-09
4729 7.21073778464643e-09
4730 7.20991311098373e-09
4731 7.21231963041191e-09
4732 7.2114354487951e-09
4733 7.21223214483757e-09
4734 7.21125914537879e-09
4735 7.21125692493274e-09
4736 7.21057835662009e-09
4737 7.21017778815281e-09
4738 7.20953430288773e-09
4739 7.20947035404151e-09
4740 7.20826554001519e-09
4741 7.20850623636693e-09
4742 7.207662022779e-09
4743 7.20756965222336e-09
4744 7.20661175179771e-09
4745 7.20641057938565e-09
4746 7.20562187694895e-09
4747 7.2055024169515e-09
4748 7.20620230154623e-09
4749 7.20729920189456e-09
4750 7.20698167810951e-09
4751 7.20735826575947e-09
4752 7.20576531776373e-09
4753 7.20626136541114e-09
4754 7.20506543316901e-09
4755 7.20524484520979e-09
4756 7.20414572441541e-09
4757 7.20521331487589e-09
4758 7.2038566223398e-09
4759 7.20459336633894e-09
4760 7.20313408919537e-09
4761 7.30157401207521e-09
4762 7.30059834808117e-09
4763 7.30034033225024e-09
4764 7.29925009324006e-09
4765 7.29848670388833e-09
4766 7.29799465304382e-09
4767 7.29695015522225e-09
4768 7.29687066325369e-09
4769 7.29603044646865e-09
4770 7.29618765404894e-09
4771 7.29600602156211e-09
4772 7.29501659080256e-09
4773 7.29479943117894e-09
4774 7.29422966472271e-09
4775 7.29450500003281e-09
4776 7.29436511193171e-09
4777 7.29249327591219e-09
4778 7.29288673895212e-09
4779 7.2927099914466e-09
4780 7.29263716081618e-09
4781 7.29166327317898e-09
4782 7.2921553240235e-09
4783 7.29110816166667e-09
4784 7.29084748130049e-09
4785 7.29059701498613e-09
4786 7.2903070247321e-09
4787 7.2895045555299e-09
4788 7.28810611860808e-09
4789 7.28894233859023e-09
4790 7.28815141570749e-09
4791 7.28794669058175e-09
4792 7.28753501988422e-09
4793 7.28647764347556e-09
4794 7.28674409700147e-09
4795 7.28724725007623e-09
4796 7.28646876169137e-09
4797 7.28655180637361e-09
4798 7.28458093846029e-09
4799 7.28493043666845e-09
4800 7.28527727034134e-09
4801 7.28504856439827e-09
4802 7.2841350728936e-09
4803 7.28417726136854e-09
4804 7.28350313394799e-09
4805 7.28336191357926e-09
4806 7.28137639072202e-09
4807 7.28206828171096e-09
4808 7.28237914415786e-09
4809 7.28214244460901e-09
4810 7.28106597236433e-09
4811 7.28000548733121e-09
4812 7.28075999489874e-09
4813 7.28038562769484e-09
4814 7.27967508495908e-09
4815 7.27520177434826e-09
4816 7.27395610411463e-09
4817 7.27183957494049e-09
4818 7.26753324187257e-09
4819 7.26610505097369e-09
4820 7.26369719927789e-09
4821 7.2635986114733e-09
4822 7.26185422905701e-09
4823 7.26133597694911e-09
4824 7.2611188173255e-09
4825 7.25998416939433e-09
4826 7.25996640582593e-09
4827 7.25760518349716e-09
4828 7.2581616272771e-09
4829 7.25664550671468e-09
4830 7.25527327105624e-09
4831 7.25506943410892e-09
4832 7.25436288817605e-09
4833 7.25276194657454e-09
4834 7.25243864962977e-09
4835 7.25099669196538e-09
4836 7.25067028639614e-09
4837 7.25063520334857e-09
4838 7.25000903756268e-09
4839 7.24842408317272e-09
4840 7.24767401649729e-09
4841 7.24738846713535e-09
4842 7.24554283237921e-09
4843 7.24642790217445e-09
4844 7.24487358993997e-09
4845 7.24408177887881e-09
4846 7.24374027427643e-09
4847 7.24358928394508e-09
4848 7.2428596453733e-09
4849 7.24278548247526e-09
4850 7.24189685996635e-09
4851 7.24142923402837e-09
4852 7.24003657026628e-09
4853 7.2405872408865e-09
4854 7.23928783585848e-09
4855 7.23910442701481e-09
4856 7.23807813685085e-09
4857 7.23806570235297e-09
4858 7.23692528126207e-09
4859 7.23663973190014e-09
4860 7.23592297191544e-09
4861 7.23470972019413e-09
4862 7.23463333685004e-09
4863 7.23444149031138e-09
4864 7.23311899264445e-09
4865 7.23352311382541e-09
4866 7.23238269273452e-09
4867 7.2323818045561e-09
4868 7.23157667081864e-09
4869 7.23044735195799e-09
4870 7.22953874543464e-09
4871 7.22948456655104e-09
4872 7.22844850642446e-09
4873 7.22833259914069e-09
4874 7.22774329275921e-09
4875 7.22727211410756e-09
4876 7.22611481762669e-09
4877 7.22620807636076e-09
4878 7.2258896643973e-09
4879 7.22549353682211e-09
4880 7.22425541610505e-09
4881 7.22440374190114e-09
4882 7.22353554749589e-09
4883 7.22306392475502e-09
4884 7.22244752893175e-09
4885 7.22231963123932e-09
4886 7.22126758390118e-09
4887 7.2212458235299e-09
4888 7.22037318823254e-09
4889 7.22072579506516e-09
4890 7.21950499382729e-09
4891 7.21942594594793e-09
4892 7.2207813062164e-09
4893 7.2240093906828e-09
4894 7.22380066875417e-09
4895 7.22512183415347e-09
4896 7.22514448270317e-09
4897 7.22598292313137e-09
4898 7.22541759756723e-09
4899 7.22555926202517e-09
4900 7.22482784709655e-09
4901 7.22554016618915e-09
4902 7.22440374190114e-09
4903 7.22480031356554e-09
4904 7.22454807089434e-09
4905 7.2243389048765e-09
4906 7.22354887017218e-09
4907 7.2237704706879e-09
4908 7.22338144854007e-09
4909 7.22324244861738e-09
4910 7.22280457665647e-09
4911 7.22254833718239e-09
4912 7.22167214917135e-09
4913 7.22199677838375e-09
4914 7.22161308530644e-09
4915 7.22161441757407e-09
4916 7.22115700568793e-09
4917 7.22133197683661e-09
4918 7.22002369002439e-09
4919 7.22022619470408e-09
4920 7.21942639003714e-09
4921 7.21992199359534e-09
4922 7.21916926238464e-09
4923 7.21927007063528e-09
4924 7.21801196590377e-09
4925 7.21831217020963e-09
4926 7.2182975152657e-09
4927 7.21737114517396e-09
4928 7.2175265763974e-09
4929 7.2168906406489e-09
4930 7.2168142573048e-09
4931 7.21569737294203e-09
4932 7.21619297650022e-09
4933 7.21563475636344e-09
4934 7.21567561257075e-09
4935 7.21490955868376e-09
4936 7.21490645005929e-09
4937 7.21409820769736e-09
4938 7.21229032052406e-09
4939 7.21287163329976e-09
4940 7.21324378005761e-09
4941 7.21223569755125e-09
4942 7.21265758230061e-09
4943 7.21080484211711e-09
4944 7.20918924557168e-09
4945 7.20945081411628e-09
4946 7.20994819403131e-09
4947 7.21056592212221e-09
4948 7.20988690972035e-09
4949 7.2101324910534e-09
4950 7.20879755888859e-09
4951 7.20595538794555e-09
4952 7.20683912547315e-09
4953 7.20854398394977e-09
4954 7.20733472903134e-09
4955 7.20709536494724e-09
4956 7.20664505848845e-09
4957 7.2045600596482e-09
4958 7.20517467911463e-09
4959 7.20556636579772e-09
4960 7.20489046202033e-09
4961 7.2053851774001e-09
4962 7.20511872387419e-09
4963 7.20091097861086e-09
4964 7.20220905137126e-09
4965 7.20261850162274e-09
4966 7.20215931337975e-09
4967 7.19985226993458e-09
4968 7.19999126985726e-09
4969 7.20141057897195e-09
4970 7.20220683092521e-09
4971 7.19733428411473e-09
4972 7.19893433753782e-09
4973 7.19985004948853e-09
4974 7.20039627921665e-09
4975 7.19656600978169e-09
4976 7.19693726836113e-09
4977 7.19874959642652e-09
4978 7.19890813627444e-09
4979 7.19523729486582e-09
4980 7.1961427927647e-09
4981 7.19675385951746e-09
4982 7.19570802942826e-09
4983 7.19334547483186e-09
4984 7.19456316744527e-09
4985 7.19473947086158e-09
4986 7.19437887042318e-09
4987 7.19118808945041e-09
4988 7.192801909639e-09
4989 7.19313408836797e-09
4990 7.19232273738157e-09
4991 7.1907728660392e-09
4992 7.19109927160844e-09
4993 7.19166282081574e-09
4994 7.1897088282924e-09
4995 7.19049175756936e-09
4996 7.18949078049036e-09
4997 7.18821935308256e-09
4998 7.18854709091943e-09
4999 7.18928205856173e-09
};
\addlegendentry{Test}

\nextgroupplot[
title={ELU/ELU $\rare$},
ymin=5.8568301326772e-09, ymax=1e-05,
]
\addplot [semithick, black, dashed]
table {%
0 0.00476444136322243
1 0.000287209001748124
2 0.000182133273687214
3 0.000163652311056467
4 0.000104352656668652
5 3.59188913384969e-05
6 2.71513295885484e-05
7 2.55558569300547e-05
8 2.32184630571908e-05
9 1.913609235433e-05
10 1.31369357849849e-05
11 8.06671850284601e-06
12 5.80813367032107e-06
13 5.06432934583501e-06
14 4.66482480633346e-06
15 4.25811618047334e-06
16 3.78560156306662e-06
17 3.26326517637199e-06
18 2.74738851393863e-06
19 2.30882097436336e-06
20 1.99422408720196e-06
21 1.80195170632125e-06
22 1.69239316190684e-06
23 1.62640572988515e-06
24 1.5815862651376e-06
25 1.548602872532e-06
26 1.52305749559112e-06
27 1.50235992007275e-06
28 1.48491101147741e-06
29 1.46970120535883e-06
30 1.45605174567542e-06
31 1.44349768420682e-06
32 1.43168866096133e-06
33 1.42037038907361e-06
34 1.40939222942649e-06
35 1.3986427675583e-06
36 1.38802548497452e-06
37 1.37745497154995e-06
38 1.3668557726696e-06
39 1.35614575357934e-06
40 1.34523499353278e-06
41 1.33402135269378e-06
42 1.32239029813164e-06
43 1.31021182193791e-06
44 1.29734851622842e-06
45 1.28365266165531e-06
46 1.26895666727833e-06
47 1.25307285060217e-06
48 1.23579656079187e-06
49 1.21691350692643e-06
50 1.19621497921685e-06
51 1.17349751567986e-06
52 1.14855621178478e-06
53 1.1212165755623e-06
54 1.09134073562345e-06
55 1.05883943617613e-06
56 1.02370652873063e-06
57 9.86051293612178e-07
58 9.46174484848683e-07
59 9.04584105178685e-07
60 8.61948530236134e-07
61 8.19150756800724e-07
62 7.77351648643787e-07
63 7.37519012362853e-07
64 7.00575846456886e-07
65 6.67252035761123e-07
66 6.37832278984618e-07
67 6.12391757444541e-07
68 5.90972821722602e-07
69 5.73321464923282e-07
70 5.59040760998997e-07
71 5.47403495593457e-07
72 5.38100507036177e-07
73 5.30902989762438e-07
74 5.25313168706632e-07
75 5.20845269576853e-07
76 5.17217641100132e-07
77 5.14426029806003e-07
78 5.12118985797727e-07
79 5.10243719801906e-07
80 5.08567690722828e-07
81 5.07017136095911e-07
82 5.05571204676869e-07
83 5.04209146084733e-07
84 5.02919981371974e-07
85 5.01691930635317e-07
86 5.00519327390592e-07
87 4.99396791523665e-07
88 4.98320389631601e-07
89 4.97285846208229e-07
90 4.96288930589728e-07
91 4.95327048888328e-07
92 4.94396838428557e-07
93 4.93495796462184e-07
94 4.92621536499982e-07
95 4.91771958786913e-07
96 4.90944786887937e-07
97 4.90138088443004e-07
98 4.89350383544362e-07
99 4.88579430756531e-07
100 4.87825167427403e-07
101 4.87085578523505e-07
102 4.86359469421771e-07
103 4.85645413514035e-07
104 4.84943153214346e-07
105 4.84251735604957e-07
106 4.83569449011512e-07
107 4.82895870373312e-07
108 4.82230397661709e-07
109 4.81572449823275e-07
110 4.80921556476943e-07
111 4.80277849744581e-07
112 4.79639639424434e-07
113 4.79007515391316e-07
114 4.78380398687861e-07
115 4.77757354351738e-07
116 4.77138436865587e-07
117 4.76523064016021e-07
118 4.75910465981499e-07
119 4.75300764547271e-07
120 4.74692531001963e-07
121 4.74086114374828e-07
122 4.73480651651315e-07
123 4.72876360424479e-07
124 4.72272341927038e-07
125 4.7166814273325e-07
126 4.71063715766107e-07
127 4.70459516327182e-07
128 4.69853648205287e-07
129 4.6924633786638e-07
130 4.68637013540274e-07
131 4.68025434667396e-07
132 4.6741139433415e-07
133 4.66795079367088e-07
134 4.66175605994934e-07
135 4.65553710533584e-07
136 4.64928501948236e-07
137 4.64300315664801e-07
138 4.63668755829261e-07
139 4.63033809319668e-07
140 4.62395510645308e-07
141 4.61752369913526e-07
142 4.61106113657905e-07
143 4.60453145432638e-07
144 4.59791298839107e-07
145 4.5911871204396e-07
146 4.58437270898671e-07
147 4.57749457043022e-07
148 4.57053023724896e-07
149 4.56350126507132e-07
150 4.55640013564462e-07
151 4.54923675308194e-07
152 4.54202604000642e-07
153 4.53476513325057e-07
154 4.52733631833624e-07
155 4.5197180902079e-07
156 4.51199394440493e-07
157 4.50418337283764e-07
158 4.49625844870738e-07
159 4.48821316448189e-07
160 4.48002295646788e-07
161 4.47169362768562e-07
162 4.46323745034505e-07
163 4.4546594391015e-07
164 4.44594557352573e-07
165 4.43705247782589e-07
166 4.42798313709858e-07
167 4.41875008485582e-07
168 4.40933764767948e-07
169 4.39975435053697e-07
170 4.38998427521042e-07
171 4.38003892604399e-07
172 4.36989101421759e-07
173 4.35956088230327e-07
174 4.34902528173353e-07
175 4.33826748160726e-07
176 4.32728797932214e-07
177 4.31606910723303e-07
178 4.30460019556378e-07
179 4.29288943070816e-07
180 4.28093117493589e-07
181 4.26871748359758e-07
182 4.25625092102422e-07
183 4.24353442177505e-07
184 4.23055193074617e-07
185 4.21731596704333e-07
186 4.20381192480335e-07
187 4.19003061644219e-07
188 4.17596254019159e-07
189 4.16157675513062e-07
190 4.14685380677682e-07
191 4.13179969148914e-07
192 4.1163460672955e-07
193 4.10047745226905e-07
194 4.08420175563862e-07
195 4.0675242045829e-07
196 4.05044442857161e-07
197 4.03294653271047e-07
198 4.01500316632664e-07
199 3.99659197940494e-07
200 3.97771593840801e-07
201 3.95838685902206e-07
202 3.93863643898484e-07
203 3.91850701365826e-07
204 3.8978323626182e-07
205 3.87650841425113e-07
206 3.85481064359539e-07
207 3.83267576953728e-07
208 3.81010536811388e-07
209 3.78711102758444e-07
210 3.76370464723053e-07
211 3.73988648805224e-07
212 3.71568461995153e-07
213 3.69108867340984e-07
214 3.66611476771084e-07
215 3.64076967914073e-07
216 3.61502240103562e-07
217 3.58890517373922e-07
218 3.56240534731711e-07
219 3.53557073381694e-07
220 3.50842083378211e-07
221 3.48100659650186e-07
222 3.45333307750551e-07
223 3.42543893372849e-07
224 3.39736114847611e-07
225 3.369143338805e-07
226 3.34083649160988e-07
227 3.31248908359782e-07
228 3.28416093058337e-07
229 3.25591502237543e-07
230 3.22780568902559e-07
231 3.19989789268504e-07
232 3.17225452279857e-07
233 3.14491629682223e-07
234 3.11792668754585e-07
235 3.09131197729684e-07
236 3.06517852294519e-07
237 3.03954773110071e-07
238 3.01446328777821e-07
239 2.98995648536859e-07
240 2.9660936257514e-07
241 2.94294683270024e-07
242 2.92052614168092e-07
243 2.89879255549685e-07
244 2.87774823903852e-07
245 2.85751411886359e-07
246 2.83823413107065e-07
247 2.8198172741245e-07
248 2.80221646313805e-07
249 2.78542331105136e-07
250 2.76940837141915e-07
251 2.75412190186231e-07
252 2.73951231561931e-07
253 2.7255024429973e-07
254 2.71203027819134e-07
255 2.69906216276539e-07
256 2.68660111850671e-07
257 2.67471191886415e-07
258 2.66344334089119e-07
259 2.65264654604636e-07
260 2.64230337340798e-07
261 2.63237087007262e-07
262 2.62282143792092e-07
263 2.61361334200849e-07
264 2.60470520708544e-07
265 2.59607627675962e-07
266 2.58770915285922e-07
267 2.57959423107579e-07
268 2.57173640277664e-07
269 2.56410442509569e-07
270 2.55663220180224e-07
271 2.54925171003784e-07
272 2.54196498946868e-07
273 2.53480272319528e-07
274 2.52776445068825e-07
275 2.52084416000997e-07
276 2.5140289786485e-07
277 2.50730220485274e-07
278 2.50066025321516e-07
279 2.49411215393458e-07
280 2.48764684752878e-07
281 2.48125664899845e-07
282 2.47494309446594e-07
283 2.46868687562518e-07
284 2.4624734290235e-07
285 2.45626904383123e-07
286 2.45009446378397e-07
287 2.44393325001369e-07
288 2.4378181800877e-07
289 2.43172055133911e-07
290 2.4256530100164e-07
291 2.41960511086603e-07
292 2.41357253369046e-07
293 2.40755882733978e-07
294 2.40156422573357e-07
295 2.39557949601732e-07
296 2.38959932607585e-07
297 2.38363551857468e-07
298 2.37768335594524e-07
299 2.37172530119345e-07
300 2.36577896417067e-07
301 2.35982732150752e-07
302 2.35387312364388e-07
303 2.34792562953778e-07
304 2.34196452163715e-07
305 2.33600236365206e-07
306 2.33003138595755e-07
307 2.32405039962025e-07
308 2.31806552119629e-07
309 2.31206627679725e-07
310 2.30605956792473e-07
311 2.30002627902515e-07
312 2.29398833879735e-07
313 2.28793283710615e-07
314 2.28186628827842e-07
315 2.27577188701922e-07
316 2.26967089908214e-07
317 2.26354965541908e-07
318 2.25740872699554e-07
319 2.25125758356093e-07
320 2.24507902021998e-07
321 2.23887036132631e-07
322 2.23265983901655e-07
323 2.22641690528214e-07
324 2.22015051130953e-07
325 2.21385545176034e-07
326 2.20754100204168e-07
327 2.20120507351318e-07
328 2.19483208324434e-07
329 2.18845289783332e-07
330 2.18203873785683e-07
331 2.17559767497377e-07
332 2.16915209387381e-07
333 2.16266716092761e-07
334 2.15616752402781e-07
335 2.14966548944595e-07
336 2.14315003595367e-07
337 2.1366344830831e-07
338 2.13011677022479e-07
339 2.12359941449414e-07
340 2.11707755698676e-07
341 2.1105626038409e-07
342 2.10403127845815e-07
343 2.0975012869684e-07
344 2.09096708919887e-07
345 2.08442651044471e-07
346 2.07787246781166e-07
347 2.07131656207338e-07
348 2.06474762439512e-07
349 2.05817381004891e-07
350 2.05158846203979e-07
351 2.04499910220868e-07
352 2.03839550005469e-07
353 2.03179163081657e-07
354 2.02517153533144e-07
355 2.01855124773331e-07
356 2.01191448346627e-07
357 2.0052761777567e-07
358 1.99862679419738e-07
359 1.99197282466201e-07
360 1.98530614780701e-07
361 1.97864685103077e-07
362 1.9719662074813e-07
363 1.9652928455649e-07
364 1.95861606787417e-07
365 1.95194053539183e-07
366 1.94526078958646e-07
367 1.93859603205837e-07
368 1.93192175157897e-07
369 1.92525448107439e-07
370 1.91858885287388e-07
371 1.91191966504967e-07
372 1.90525271993813e-07
373 1.89860205225045e-07
374 1.89196254019919e-07
375 1.88534281616803e-07
376 1.87874955693168e-07
377 1.87216973878357e-07
378 1.86560850582218e-07
379 1.85907541185415e-07
380 1.85255476946367e-07
381 1.84605074016631e-07
382 1.83957618423491e-07
383 1.83312013485093e-07
384 1.8266833139613e-07
385 1.82027810332386e-07
386 1.81389343944183e-07
387 1.8075267116302e-07
388 1.80119758550923e-07
389 1.7948883300889e-07
390 1.78861190883062e-07
391 1.78236034889423e-07
392 1.77614137101045e-07
393 1.76995685527537e-07
394 1.7637973798923e-07
395 1.75766713156555e-07
396 1.75158246918272e-07
397 1.74553110365849e-07
398 1.7395062255332e-07
399 1.73352543488825e-07
400 1.72758552045238e-07
401 1.72167235114706e-07
402 1.71580275167393e-07
403 1.70998021722824e-07
404 1.70419156979129e-07
405 1.69844977508937e-07
406 1.69273987524754e-07
407 1.68709117673416e-07
408 1.68147527970675e-07
409 1.67591753821306e-07
410 1.67039669641156e-07
411 1.6649350091047e-07
412 1.65953363407745e-07
413 1.65418042895382e-07
414 1.64888517518946e-07
415 1.64365401463584e-07
416 1.6384633507105e-07
417 1.63333670452737e-07
418 1.62825127954846e-07
419 1.62323263477759e-07
420 1.61825697531093e-07
421 1.61333425137578e-07
422 1.60846885428612e-07
423 1.60364141144065e-07
424 1.59888861375634e-07
425 1.59415983183919e-07
426 1.58949547924614e-07
427 1.58487470931057e-07
428 1.58030972794876e-07
429 1.57579107093575e-07
430 1.57132099761981e-07
431 1.56689492880169e-07
432 1.56253468179379e-07
433 1.55820597076861e-07
434 1.55392990940584e-07
435 1.54970411767152e-07
436 1.54552344437953e-07
437 1.54139077490179e-07
438 1.53729726214635e-07
439 1.53326030714496e-07
440 1.52926914107177e-07
441 1.52530817421059e-07
442 1.52141003624706e-07
443 1.51754471509946e-07
444 1.51372709933817e-07
445 1.50995253035191e-07
446 1.50622202646389e-07
447 1.50252689715913e-07
448 1.49886973209945e-07
449 1.49527585062259e-07
450 1.49169798584481e-07
451 1.48817853256489e-07
452 1.48468534118429e-07
453 1.48123280571788e-07
454 1.47781767492638e-07
455 1.4744515742926e-07
456 1.47110209938184e-07
457 1.46780914684008e-07
458 1.46453139789671e-07
459 1.46130646546894e-07
460 1.45810925787693e-07
461 1.45495181498312e-07
462 1.45181858957066e-07
463 1.44872856814082e-07
464 1.44566727250606e-07
465 1.44263553985446e-07
466 1.4396442981468e-07
467 1.4366765873941e-07
468 1.43373137677649e-07
469 1.43083048884662e-07
470 1.42794821993419e-07
471 1.42510143923857e-07
472 1.4222744771164e-07
473 1.41946990662944e-07
474 1.4167077216598e-07
475 1.41396124210935e-07
476 1.41123678225163e-07
477 1.40854651661471e-07
478 1.40587699712036e-07
479 1.4032314951784e-07
480 1.40060456583768e-07
481 1.39800100708243e-07
482 1.39542786960156e-07
483 1.39287484004491e-07
484 1.39034087026424e-07
485 1.38782402189008e-07
486 1.38533187829637e-07
487 1.3828599954735e-07
488 1.38041222668583e-07
489 1.37797628290137e-07
490 1.37555884436757e-07
491 1.37317337082443e-07
492 1.37079112704441e-07
493 1.36844074585696e-07
494 1.36609811836053e-07
495 1.36377500765228e-07
496 1.36146664001302e-07
497 1.35918237884169e-07
498 1.35691210005007e-07
499 1.35465777981736e-07
500 1.3524153702793e-07
501 1.35019593495933e-07
502 1.34798652975832e-07
503 1.34579642998567e-07
504 1.3436186366178e-07
505 1.34146148710101e-07
506 1.33931323725633e-07
507 1.3371848620114e-07
508 1.33507867792204e-07
509 1.33298477745569e-07
510 1.33092243059352e-07
511 1.32891398486379e-07
512 1.32696917072295e-07
513 1.32522255436029e-07
514 1.32349155681943e-07
515 1.32131431183069e-07
516 1.31927722039293e-07
517 1.31727828671657e-07
518 1.3152838934527e-07
519 1.31330402486363e-07
520 1.31134568035485e-07
521 1.30938400192626e-07
522 1.30744963525053e-07
523 1.30552471267364e-07
524 1.30360561046317e-07
525 1.30169620002896e-07
526 1.29979039062533e-07
527 1.29790439500788e-07
528 1.29602372294535e-07
529 1.29416423416906e-07
530 1.2923536692977e-07
531 1.29059579432322e-07
532 1.28893761988103e-07
533 1.28734874896175e-07
534 1.28556594769513e-07
535 1.28363371405094e-07
536 1.28177246439876e-07
537 1.27992216730988e-07
538 1.2780702206161e-07
539 1.27622994593679e-07
540 1.27438863803953e-07
541 1.27256288792843e-07
542 1.27072852159849e-07
543 1.26890750209441e-07
544 1.26709945813985e-07
545 1.26528047198704e-07
546 1.26347925257342e-07
547 1.26168113149916e-07
548 1.25988558561829e-07
549 1.25809680955768e-07
550 1.2563118560216e-07
551 1.25452847595575e-07
552 1.25275483849663e-07
553 1.25098409034408e-07
554 1.24922051846887e-07
555 1.24746229278827e-07
556 1.24570210558339e-07
557 1.24395052793602e-07
558 1.24220731204439e-07
559 1.24046164882152e-07
560 1.23872558492444e-07
561 1.23699761362062e-07
562 1.23526690148701e-07
563 1.23354043853219e-07
564 1.23181955468787e-07
565 1.23011397810124e-07
566 1.22839523724494e-07
567 1.22668641868273e-07
568 1.22498408559935e-07
569 1.22328809955441e-07
570 1.22158692985153e-07
571 1.21988966887088e-07
572 1.21820702569053e-07
573 1.21652054284382e-07
574 1.21483218614493e-07
575 1.21315357878515e-07
576 1.21148116337899e-07
577 1.209802841311e-07
578 1.208137726465e-07
579 1.20646254523571e-07
580 1.20480603477358e-07
581 1.20313416047235e-07
582 1.20148026471423e-07
583 1.19982083629289e-07
584 1.19816621514168e-07
585 1.19651504149498e-07
586 1.19486845242545e-07
587 1.19321473456946e-07
588 1.1915713027566e-07
589 1.18993552634006e-07
590 1.18828981759744e-07
591 1.18665931886142e-07
592 1.18501737660814e-07
593 1.18338479609292e-07
594 1.18175279399502e-07
595 1.18012474632057e-07
596 1.17849637964795e-07
597 1.17686911822013e-07
598 1.175244700935e-07
599 1.17362114524955e-07
600 1.17200307706788e-07
601 1.17037897220396e-07
602 1.16876867132465e-07
603 1.16714791792649e-07
604 1.16553782866902e-07
605 1.16392543889621e-07
606 1.16231044639203e-07
607 1.16070923669387e-07
608 1.15910125122021e-07
609 1.15750203398601e-07
610 1.15590060051129e-07
611 1.15430549982865e-07
612 1.15271244303994e-07
613 1.15111466217233e-07
614 1.14952653808675e-07
615 1.14794011314423e-07
616 1.14636169557425e-07
617 1.1447777123097e-07
618 1.14319268335272e-07
619 1.14161953170822e-07
620 1.14004547192792e-07
621 1.13846918361915e-07
622 1.13690153930701e-07
623 1.13533620243356e-07
624 1.13376968536194e-07
625 1.13220660934665e-07
626 1.13065189515371e-07
627 1.12909202069744e-07
628 1.1275323890203e-07
629 1.12598653509099e-07
630 1.12443625357006e-07
631 1.1228897591975e-07
632 1.12134507745498e-07
633 1.11980143985058e-07
634 1.1182607379423e-07
635 1.11672019239339e-07
636 1.11518923690213e-07
637 1.11365562485144e-07
638 1.11212768687974e-07
639 1.11059815516512e-07
640 1.10907335948429e-07
641 1.10754633197541e-07
642 1.10602291568895e-07
643 1.10450840402621e-07
644 1.10298577292944e-07
645 1.1014669817877e-07
646 1.09995515936934e-07
647 1.09843339915816e-07
648 1.0969279371853e-07
649 1.09540945763431e-07
650 1.09390486698224e-07
651 1.09238189243133e-07
652 1.09087927780749e-07
653 1.08937336623871e-07
654 1.08786092669622e-07
655 1.08636473592494e-07
656 1.0848593409607e-07
657 1.08335112119384e-07
658 1.08185772090152e-07
659 1.08036007549828e-07
660 1.07887040126542e-07
661 1.07737444104306e-07
662 1.07588886228527e-07
663 1.07439557815336e-07
664 1.07291704211754e-07
665 1.07143376038454e-07
666 1.06995318413183e-07
667 1.06848129044046e-07
668 1.06700548135219e-07
669 1.06553551952882e-07
670 1.06406202334597e-07
671 1.06260220211762e-07
672 1.06113506380012e-07
673 1.05967520988237e-07
674 1.0582156773653e-07
675 1.05676486029882e-07
676 1.05530475889459e-07
677 1.05386219429793e-07
678 1.05240682085395e-07
679 1.05096190287757e-07
680 1.04952218093146e-07
681 1.04808220077413e-07
682 1.04664291370682e-07
683 1.04521389627799e-07
684 1.0437822012932e-07
685 1.04236072287822e-07
686 1.04092943057577e-07
687 1.0395204272573e-07
688 1.03809851032644e-07
689 1.03669038661991e-07
690 1.03529236914923e-07
691 1.0338842281854e-07
692 1.03248966096814e-07
693 1.03109455399153e-07
694 1.02971135624852e-07
695 1.02833057749052e-07
696 1.02695250123741e-07
697 1.02557699454664e-07
698 1.02422627224819e-07
699 1.02285763839305e-07
700 1.02151368084336e-07
701 1.02017847460889e-07
702 1.01884583183498e-07
703 1.01754518659902e-07
704 1.01626311107772e-07
705 1.01502969955369e-07
706 1.01382425513385e-07
707 1.01258288738748e-07
708 1.01127622211816e-07
709 1.00996363517236e-07
710 1.00865741437417e-07
711 1.00735804393182e-07
712 1.00606095556621e-07
713 1.00476863825083e-07
714 1.00349162106106e-07
715 1.00221187839367e-07
716 1.00093577668936e-07
717 9.99678685222882e-08
718 9.9840921337524e-08
719 9.97158997675385e-08
720 9.95906663594504e-08
721 9.94663726960887e-08
722 9.93428112967543e-08
723 9.92195656723816e-08
724 9.90969913297057e-08
725 9.89749747324353e-08
726 9.88539189790138e-08
727 9.87329593629127e-08
728 9.86127621747102e-08
729 9.84940123895939e-08
730 9.83744557196964e-08
731 9.82572718446484e-08
732 9.81389049128367e-08
733 9.80216593648997e-08
734 9.79053874483604e-08
735 9.77901078051957e-08
736 9.76749700205559e-08
737 9.75604245225803e-08
738 9.74468859595845e-08
739 9.73329311864113e-08
740 9.72203658249349e-08
741 9.71087589176634e-08
742 9.6996623001111e-08
743 9.68871573028274e-08
744 9.67765988315605e-08
745 9.66665913879616e-08
746 9.65583119119096e-08
747 9.64494899808877e-08
748 9.63426108588905e-08
749 9.62350978257476e-08
750 9.6128775738169e-08
751 9.60226565398514e-08
752 9.59183127506868e-08
753 9.58138117921337e-08
754 9.57091826152556e-08
755 9.56064252952693e-08
756 9.55030073783725e-08
757 9.54024015920041e-08
758 9.52992291760601e-08
759 9.51996421463264e-08
760 9.5098727951104e-08
761 9.49987755913817e-08
762 9.48994145311133e-08
763 9.48003797356378e-08
764 9.47026082305236e-08
765 9.46049005854377e-08
766 9.45073468887614e-08
767 9.44106844533721e-08
768 9.43152188037644e-08
769 9.42190871358051e-08
770 9.41254612909503e-08
771 9.40308928223388e-08
772 9.3936125420857e-08
773 9.38435420012595e-08
774 9.37491349262132e-08
775 9.36539536140479e-08
776 9.35558058121622e-08
777 9.34518880320034e-08
778 9.33470724242547e-08
779 9.32482016602343e-08
780 9.31616052137763e-08
781 9.30747656440189e-08
782 9.29874449360213e-08
783 9.28997957849731e-08
784 9.28126603589874e-08
785 9.27270734196739e-08
786 9.26405464762858e-08
787 9.25545501795355e-08
788 9.24692513994963e-08
789 9.23827188317183e-08
790 9.22989614924319e-08
791 9.22135720298201e-08
792 9.21292302447085e-08
793 9.20454833202022e-08
794 9.19635640679317e-08
795 9.1881423939455e-08
796 9.18003402707868e-08
797 9.17214335154348e-08
798 9.16410416671454e-08
799 9.15627961197707e-08
800 9.14836974454936e-08
801 9.14065300055e-08
802 9.13287694044307e-08
803 9.1251877836207e-08
804 9.11757950876613e-08
805 9.10991983440468e-08
806 9.10235873443632e-08
807 9.09488255445012e-08
808 9.08737499298873e-08
809 9.07992002856517e-08
810 9.07259728686327e-08
811 9.06510821718776e-08
812 9.05779086810554e-08
813 9.05030697424536e-08
814 9.04310439620026e-08
815 9.03578223141466e-08
816 9.02830791118525e-08
817 9.02097550392611e-08
818 9.01334099006057e-08
819 9.00570043009807e-08
820 8.99779033796122e-08
821 8.98936099784642e-08
822 8.98022139210752e-08
823 8.97003447719591e-08
824 8.95939051201999e-08
825 8.94924575929323e-08
826 8.9402857770704e-08
827 8.93162613606613e-08
828 8.92303652029192e-08
829 8.91449576290171e-08
830 8.90717843553546e-08
831 8.90111509739278e-08
832 8.8952634066608e-08
833 8.88903048510414e-08
834 8.882857691761e-08
835 8.87667813720405e-08
836 8.87053647260494e-08
837 8.86437451859123e-08
838 8.85826805832046e-08
839 8.85212698911175e-08
840 8.84610560696331e-08
841 8.83997905241074e-08
842 8.8339696300288e-08
843 8.82802740851396e-08
844 8.82197859466416e-08
845 8.81613151149096e-08
846 8.81022303582668e-08
847 8.80431548733185e-08
848 8.79848105710401e-08
849 8.79262673163517e-08
850 8.78683738365638e-08
851 8.78112604261183e-08
852 8.77529224849738e-08
853 8.76972461538728e-08
854 8.7639054436206e-08
855 8.75834314117974e-08
856 8.75266014537246e-08
857 8.74705593618685e-08
858 8.74139314968048e-08
859 8.73597653345826e-08
860 8.73029686867532e-08
861 8.72492711119221e-08
862 8.71932684849419e-08
863 8.71388995236089e-08
864 8.70845817497745e-08
865 8.70303799276328e-08
866 8.69757650581171e-08
867 8.69227073807544e-08
868 8.68685449848527e-08
869 8.6815370408555e-08
870 8.67625306235631e-08
871 8.67090607505894e-08
872 8.66564838100103e-08
873 8.66045283136607e-08
874 8.65508199927589e-08
875 8.65000938992111e-08
876 8.64480161539838e-08
877 8.63961794110857e-08
878 8.63444138325953e-08
879 8.62929601068707e-08
880 8.62420939680142e-08
881 8.61907387217364e-08
882 8.61406429124578e-08
883 8.60896717576054e-08
884 8.6039211971034e-08
885 8.59888848583346e-08
886 8.59390998382281e-08
887 8.58890059891593e-08
888 8.58390873852422e-08
889 8.57906484221793e-08
890 8.57398850673263e-08
891 8.56908500161957e-08
892 8.56426351716344e-08
893 8.55923604552444e-08
894 8.55454410020329e-08
895 8.54963042362655e-08
896 8.54477663234654e-08
897 8.54001153687278e-08
898 8.53514826451018e-08
899 8.53046482416353e-08
900 8.52562820861103e-08
901 8.52089157130642e-08
902 8.51615646353388e-08
903 8.51142204596478e-08
904 8.50671637522282e-08
905 8.50212164578146e-08
906 8.49732173686668e-08
907 8.49276365486595e-08
908 8.48809312596899e-08
909 8.48352416258358e-08
910 8.47889131692092e-08
911 8.47430173771535e-08
912 8.46970855805651e-08
913 8.46515280898963e-08
914 8.46066990534133e-08
915 8.4561528482574e-08
916 8.45159371047899e-08
917 8.44715098740068e-08
918 8.44260488186421e-08
919 8.43812938957633e-08
920 8.43370266494503e-08
921 8.429230893503e-08
922 8.42482402854117e-08
923 8.42038745880735e-08
924 8.41587628723772e-08
925 8.41149075161418e-08
926 8.4070599779551e-08
927 8.4027523123531e-08
928 8.39824327467653e-08
929 8.3938424416008e-08
930 8.38946814178954e-08
931 8.38503988882522e-08
932 8.38056028094059e-08
933 8.37627899317184e-08
934 8.37177211843176e-08
935 8.36754834327991e-08
936 8.36308760336912e-08
937 8.35864601005376e-08
938 8.35431058874292e-08
939 8.34988899818256e-08
940 8.34549861794542e-08
941 8.34111209466748e-08
942 8.33683120338158e-08
943 8.33232243779847e-08
944 8.32809539192958e-08
945 8.32370073853284e-08
946 8.31929331432946e-08
947 8.3150344296179e-08
948 8.31067126525475e-08
949 8.30636147450825e-08
950 8.30198155141737e-08
951 8.29767426631101e-08
952 8.29344496908746e-08
953 8.28897307481036e-08
954 8.28472481617659e-08
955 8.2805573020206e-08
956 8.27615379312974e-08
957 8.27188732488438e-08
958 8.26758476275558e-08
959 8.26331254608625e-08
960 8.25904240455699e-08
961 8.25477972674626e-08
962 8.25049994559457e-08
963 8.24628652247306e-08
964 8.24201349143294e-08
965 8.23778710707934e-08
966 8.23350818692781e-08
967 8.22937043718497e-08
968 8.22511737270659e-08
969 8.22093699173188e-08
970 8.2166476085721e-08
971 8.21253117413434e-08
972 8.20830842394038e-08
973 8.20416957258985e-08
974 8.19994828318293e-08
975 8.1957906467256e-08
976 8.19162581935373e-08
977 8.18746849033936e-08
978 8.18330563103764e-08
979 8.17924423208716e-08
980 8.17503517183127e-08
981 8.1710411072855e-08
982 8.16680365947064e-08
983 8.16278677793214e-08
984 8.15867228713962e-08
985 8.15449063202855e-08
986 8.15057657295171e-08
987 8.14645723301588e-08
988 8.14233565278322e-08
989 8.13827976298143e-08
990 8.13439805549265e-08
991 8.13022304293476e-08
992 8.12621312769046e-08
993 8.12224284825191e-08
994 8.11814479138029e-08
995 8.114121890479e-08
996 8.1102004711564e-08
997 8.10606730339281e-08
998 8.10211181505594e-08
999 8.09806947597025e-08
1000 8.09415495757193e-08
1001 8.0900535489814e-08
1002 8.08607501778802e-08
1003 8.08201764725958e-08
1004 8.07810862446523e-08
1005 8.07393143618107e-08
1006 8.0699893064029e-08
1007 8.06594009317152e-08
1008 8.06180020882508e-08
1009 8.05772620084966e-08
1010 8.05357052207079e-08
1011 8.04933285398768e-08
1012 8.04499097242228e-08
1013 8.04038127864004e-08
1014 8.0359317265799e-08
1015 8.03118968808469e-08
1016 8.02706207982951e-08
1017 8.0231481752957e-08
1018 8.01921510515946e-08
1019 8.0154548860456e-08
1020 8.01162594630256e-08
1021 8.00781804501938e-08
1022 8.00403534029215e-08
1023 8.000204145997e-08
1024 7.99640715367644e-08
1025 7.99267004474302e-08
1026 7.9888819568108e-08
1027 7.98506517170061e-08
1028 7.98126850800607e-08
1029 7.97755923827026e-08
1030 7.97377969190549e-08
1031 7.97007548154483e-08
1032 7.96625830163933e-08
1033 7.96260147644112e-08
1034 7.95883693784916e-08
1035 7.95514813662912e-08
1036 7.95142040486141e-08
1037 7.9477416041307e-08
1038 7.94394249421515e-08
1039 7.94026307691098e-08
1040 7.93660000510954e-08
1041 7.93286426374884e-08
1042 7.92924608896683e-08
1043 7.92544332686162e-08
1044 7.92179343176436e-08
1045 7.91816609235774e-08
1046 7.91450519623815e-08
1047 7.91076069255325e-08
1048 7.90714808425186e-08
1049 7.9034229781616e-08
1050 7.89981413600316e-08
1051 7.89617429730605e-08
1052 7.89249196047059e-08
1053 7.88878583271035e-08
1054 7.88513648419809e-08
1055 7.88151344384147e-08
1056 7.87780299567053e-08
1057 7.87422999191101e-08
1058 7.87046755510446e-08
1059 7.86682481690448e-08
1060 7.86327942337017e-08
1061 7.8595652150959e-08
1062 7.85599234212064e-08
1063 7.85228303024077e-08
1064 7.84866945671148e-08
1065 7.84502489756811e-08
1066 7.84136482141484e-08
1067 7.83776649955215e-08
1068 7.83414057958787e-08
1069 7.83048673529407e-08
1070 7.82693138603463e-08
1071 7.82332721240841e-08
1072 7.81958018287732e-08
1073 7.81615477158937e-08
1074 7.81234699891442e-08
1075 7.80900494357084e-08
1076 7.8051801998491e-08
1077 7.80175754582224e-08
1078 7.79806709951814e-08
1079 7.7945476717467e-08
1080 7.79088027003105e-08
1081 7.78744320073521e-08
1082 7.78364004796472e-08
1083 7.78027595957909e-08
1084 7.77633265376387e-08
1085 7.77304422023839e-08
1086 7.76910361977023e-08
1087 7.76590209929928e-08
1088 7.7618477577257e-08
1089 7.75871935432271e-08
1090 7.75448993914907e-08
1091 7.7514964180736e-08
1092 7.74724494556622e-08
1093 7.74457772680925e-08
1094 7.74005958779078e-08
1095 7.73762076811124e-08
1096 7.73297699838693e-08
1097 7.73065878241219e-08
1098 7.72588790400874e-08
1099 7.72391551655538e-08
1100 7.71880761880617e-08
1101 7.71700130628439e-08
1102 7.71208982297544e-08
1103 7.70988002227746e-08
1104 7.70557281772888e-08
1105 7.70244234362671e-08
1106 7.69951153265502e-08
1107 7.69475330222313e-08
1108 7.69338299404509e-08
1109 7.68712702683416e-08
1110 7.68712486385326e-08
1111 7.6797682165175e-08
1112 7.68079419346535e-08
1113 7.67252501341353e-08
1114 7.67412354560903e-08
1115 7.6654727362957e-08
1116 7.66747389611133e-08
1117 7.65853943369699e-08
1118 7.66078809597559e-08
1119 7.65165651364086e-08
1120 7.65409124454663e-08
1121 7.64466321019341e-08
1122 7.64746762538238e-08
1123 7.63777494663387e-08
1124 7.64085295337225e-08
1125 7.63084385853041e-08
1126 7.63423084850778e-08
1127 7.62396642692842e-08
1128 7.62765124822096e-08
1129 7.61709134775579e-08
1130 7.62113622290173e-08
1131 7.61022189430527e-08
1132 7.61445287631446e-08
1133 7.60342963346261e-08
1134 7.60780043500553e-08
1135 7.59674034331859e-08
1136 7.60124104624005e-08
1137 7.58986066480283e-08
1138 7.59480944054758e-08
1139 7.58296932010794e-08
1140 7.58829837939068e-08
1141 7.57620333375897e-08
1142 7.5818246993542e-08
1143 7.56940808517292e-08
1144 7.57523499244073e-08
1145 7.56266764172508e-08
1146 7.56873495970645e-08
1147 7.55581276243866e-08
1148 7.56228720169005e-08
1149 7.54902632063015e-08
1150 7.55582611686734e-08
1151 7.54223576189261e-08
1152 7.54940499647638e-08
1153 7.53530733454966e-08
1154 7.54306848196329e-08
1155 7.52849017091606e-08
1156 7.53667452730866e-08
1157 7.52153269361067e-08
1158 7.53037835194981e-08
1159 7.51443019630926e-08
1160 7.52438100173336e-08
1161 7.50718001440997e-08
1162 7.51852829867783e-08
1163 7.49964152264582e-08
1164 7.51328439481114e-08
1165 7.49145452307864e-08
1166 7.50853080448977e-08
1167 7.48272438111375e-08
1168 7.50406603664544e-08
1169 7.47297045264883e-08
1170 7.49911240447432e-08
1171 7.46446312158699e-08
1172 7.49388492313763e-08
1173 7.45739121477129e-08
1174 7.4878261874467e-08
1175 7.45060180258328e-08
1176 7.48164679986729e-08
1177 7.44389210101115e-08
1178 7.47529606708319e-08
1179 7.43741202251513e-08
1180 7.46902475170508e-08
1181 7.43091737678281e-08
1182 7.46255670969198e-08
1183 7.42452323896359e-08
1184 7.4562683143764e-08
1185 7.41809923157089e-08
1186 7.44990430605608e-08
1187 7.41175106291347e-08
1188 7.44323042292105e-08
1189 7.40551479396778e-08
1190 7.4367367298267e-08
1191 7.39910432003832e-08
1192 7.43031953607254e-08
1193 7.392710779075e-08
1194 7.42384473833013e-08
1195 7.38629066592278e-08
1196 7.41745397174753e-08
1197 7.37979275626799e-08
1198 7.41113445892338e-08
1199 7.37336230913765e-08
1200 7.40474975968741e-08
1201 7.36693861993487e-08
1202 7.39836460357246e-08
1203 7.36043825446675e-08
1204 7.39196195913561e-08
1205 7.35412090024923e-08
1206 7.38558653510246e-08
1207 7.34761675473816e-08
1208 7.37935092067765e-08
1209 7.34123624652749e-08
1210 7.37296011819044e-08
1211 7.33474776228871e-08
1212 7.36660664495403e-08
1213 7.32843995083776e-08
1214 7.36024501668542e-08
1215 7.32210273870493e-08
1216 7.35382421324093e-08
1217 7.31570269507653e-08
1218 7.34753752431594e-08
1219 7.30936034405438e-08
1220 7.34119258309906e-08
1221 7.30309885381697e-08
1222 7.33481999521945e-08
1223 7.29677902806536e-08
1224 7.32857566376755e-08
1225 7.2903607545971e-08
1226 7.32223575046209e-08
1227 7.28420887929904e-08
1228 7.3158730709677e-08
1229 7.27800264577194e-08
1230 7.30946732125925e-08
1231 7.27187580040756e-08
1232 7.30311654419946e-08
1233 7.26566145790031e-08
1234 7.29679414130935e-08
1235 7.25956862288601e-08
1236 7.29039293683176e-08
1237 7.25341554519421e-08
1238 7.28388967714544e-08
1239 7.24722289902058e-08
1240 7.27748303532838e-08
1241 7.24105716496393e-08
1242 7.27097426804768e-08
1243 7.2349806597094e-08
1244 7.26463316305104e-08
1245 7.22876497285529e-08
1246 7.25820593328042e-08
1247 7.22257170788776e-08
1248 7.25175863576233e-08
1249 7.21650411497077e-08
1250 7.24528337021635e-08
1251 7.21038362681003e-08
1252 7.23885135207603e-08
1253 7.20432578733554e-08
1254 7.23228988090963e-08
1255 7.19826436521576e-08
1256 7.22582298890995e-08
1257 7.19206922972226e-08
1258 7.21939487873247e-08
1259 7.18591647534961e-08
1260 7.21283337210554e-08
1261 7.17987642484541e-08
1262 7.20635271080905e-08
1263 7.17386435347134e-08
1264 7.19983808641356e-08
1265 7.16756805905216e-08
1266 7.19338757562138e-08
1267 7.16163162726158e-08
1268 7.18679392903709e-08
1269 7.15543524838491e-08
1270 7.18027671431365e-08
1271 7.14951950782083e-08
1272 7.17372642689007e-08
1273 7.14331421405312e-08
1274 7.16720496609646e-08
1275 7.13727469390246e-08
1276 7.16059574199512e-08
1277 7.13123188629261e-08
1278 7.1540442148077e-08
1279 7.12529399473638e-08
1280 7.14731291493465e-08
1281 7.11937588486755e-08
1282 7.1407307384419e-08
1283 7.11335936882662e-08
1284 7.1341903093769e-08
1285 7.10736990376937e-08
1286 7.12756691461092e-08
1287 7.10148954348178e-08
1288 7.12091879335741e-08
1289 7.09561906049316e-08
1290 7.114294531263e-08
1291 7.08969962592843e-08
1292 7.107771942505e-08
1293 7.08395782333948e-08
1294 7.10100955627624e-08
1295 7.07819442018476e-08
1296 7.09448344859975e-08
1297 7.07243137325619e-08
1298 7.08802794437968e-08
1299 7.06659394953402e-08
1300 7.08161044795652e-08
1301 7.06079787542446e-08
1302 7.07532183210624e-08
1303 7.05499493867112e-08
1304 7.06885262309331e-08
1305 7.04918228477958e-08
1306 7.06255844986625e-08
1307 7.0433251694979e-08
1308 7.05609023230469e-08
1309 7.03739912735202e-08
1310 7.04957704338316e-08
1311 7.03149334033704e-08
1312 7.04303792680161e-08
1313 7.02563982901694e-08
1314 7.0363104064608e-08
1315 7.0196861794658e-08
1316 7.02960105456452e-08
1317 7.01375314695607e-08
1318 7.0227290708269e-08
1319 7.00780235778353e-08
1320 7.01546991574986e-08
1321 7.00157867388196e-08
1322 7.00806813369592e-08
1323 6.9954645868231e-08
1324 7.00106775763931e-08
1325 6.989431656379e-08
1326 6.99379779882658e-08
1327 6.9817922010218e-08
1328 6.99141700521988e-08
1329 6.97569993555192e-08
1330 6.98645290091449e-08
1331 6.96933877935013e-08
1332 6.97463601557757e-08
1333 6.97019751687567e-08
1334 6.96376211601901e-08
1335 6.96480466180205e-08
1336 6.95810839219657e-08
1337 6.95724689525612e-08
1338 6.95165723432201e-08
1339 6.95194224749063e-08
1340 6.94591008223444e-08
1341 6.94269431373851e-08
1342 6.94040936237705e-08
1343 6.93930153083944e-08
1344 6.93317237685509e-08
1345 6.92787606537682e-08
1346 6.93378378322507e-08
1347 6.91818963294288e-08
1348 6.92621056612719e-08
1349 6.91308065963181e-08
1350 6.91924304372371e-08
1351 6.90764301816138e-08
1352 6.91129786445366e-08
1353 6.90171253827643e-08
1354 6.90501236819685e-08
1355 6.89595626295336e-08
1356 6.89722542837057e-08
1357 6.89002572940023e-08
1358 6.89082975215971e-08
1359 6.88408420177478e-08
1360 6.88348685520257e-08
1361 6.87804339118969e-08
1362 6.87684462967741e-08
1363 6.87171408064735e-08
1364 6.87016910490001e-08
1365 6.86530376969863e-08
1366 6.86360637198735e-08
1367 6.85887801268326e-08
1368 6.85705773266765e-08
1369 6.85234398865031e-08
1370 6.85060982885677e-08
1371 6.84563842663355e-08
1372 6.84409100268901e-08
1373 6.83898970521568e-08
1374 6.83764299345135e-08
1375 6.83233545373252e-08
1376 6.83113883390707e-08
1377 6.82556391535805e-08
1378 6.82483689908864e-08
1379 6.81869943999214e-08
1380 6.81839015468366e-08
1381 6.81204813142156e-08
1382 6.8119086199081e-08
1383 6.80518584548206e-08
1384 6.8055547345125e-08
1385 6.79860363688167e-08
1386 6.79887639802867e-08
1387 6.79197789377195e-08
1388 6.79219148254084e-08
1389 6.78547634569426e-08
1390 6.78524061754615e-08
1391 6.77918333544802e-08
1392 6.77768235417542e-08
1393 6.77637188828406e-08
1394 6.76794987208851e-08
1395 6.76838195419194e-08
1396 6.7661250879647e-08
1397 6.75866840600392e-08
1398 6.75780386909519e-08
1399 6.75662513465181e-08
1400 6.74867108365618e-08
1401 6.74781066309915e-08
1402 6.74541093814707e-08
1403 6.74240666935777e-08
1404 6.73672247906154e-08
1405 6.7347021592612e-08
1406 6.73129528581118e-08
1407 6.72836272828103e-08
1408 6.72506616870017e-08
1409 6.72204164018364e-08
1410 6.71885564498265e-08
1411 6.71586354270559e-08
1412 6.71266302771922e-08
1413 6.70992237694623e-08
1414 6.70708892596128e-08
1415 6.70485213962735e-08
1416 6.70295912108543e-08
1417 6.70193552649856e-08
1418 6.69830244008907e-08
1419 6.69395383963156e-08
1420 6.69039950302874e-08
1421 6.68717521943307e-08
1422 6.683724714307e-08
1423 6.68046922487342e-08
1424 6.6769906155173e-08
1425 6.67378703225197e-08
1426 6.67027217393645e-08
1427 6.66696576918646e-08
1428 6.6634905646179e-08
1429 6.66026176463497e-08
1430 6.6567321566513e-08
1431 6.65340457093322e-08
1432 6.64992756083116e-08
1433 6.64657902653687e-08
1434 6.64308994995366e-08
1435 6.63976493107121e-08
1436 6.63620189755854e-08
1437 6.63294862923713e-08
1438 6.62937246844741e-08
1439 6.62600558980486e-08
1440 6.62247956237927e-08
1441 6.61911195880549e-08
1442 6.61551291081253e-08
1443 6.6121718219847e-08
1444 6.60860807588648e-08
1445 6.60530449678731e-08
1446 6.60166660502171e-08
1447 6.59829486977514e-08
1448 6.59467076409292e-08
1449 6.59138500065382e-08
1450 6.58767899279766e-08
1451 6.58432803399833e-08
1452 6.58068994665584e-08
1453 6.57729646040828e-08
1454 6.57365216749639e-08
1455 6.57029586985658e-08
1456 6.56664244318428e-08
1457 6.56318696590219e-08
1458 6.55950531078808e-08
1459 6.55600379846355e-08
1460 6.55245921628023e-08
1461 6.54890699318678e-08
1462 6.54525462344679e-08
1463 6.54184354997867e-08
1464 6.53817590339223e-08
1465 6.53458543902197e-08
1466 6.53096382601248e-08
1467 6.52753488816771e-08
1468 6.52376838932156e-08
1469 6.52032305565875e-08
1470 6.51649756377992e-08
1471 6.51319503521819e-08
1472 6.50934278172421e-08
1473 6.50587624302368e-08
1474 6.5020912770386e-08
1475 6.4986245105203e-08
1476 6.49486690305068e-08
1477 6.49141946853504e-08
1478 6.48763408142017e-08
1479 6.48411519406089e-08
1480 6.48037015689162e-08
1481 6.4768760798195e-08
1482 6.47309190431322e-08
1483 6.46956954324374e-08
1484 6.46585667305466e-08
1485 6.46232816507997e-08
1486 6.45865642416865e-08
1487 6.45519400581662e-08
1488 6.45129277891954e-08
1489 6.44793666273458e-08
1490 6.44394449866859e-08
1491 6.44052323743871e-08
1492 6.43650808118679e-08
1493 6.43281002115081e-08
1494 6.4286003170988e-08
1495 6.42495507690022e-08
1496 6.42124990815063e-08
1497 6.41772914260485e-08
1498 6.41377617838224e-08
1499 6.41039439073232e-08
1500 6.40643732032764e-08
1501 6.40301262979648e-08
1502 6.39917981919424e-08
1503 6.39573632379431e-08
1504 6.39186405302361e-08
1505 6.38847683727128e-08
1506 6.38474544216727e-08
1507 6.38115847104714e-08
1508 6.37745319935767e-08
1509 6.37395300397969e-08
1510 6.37039882107615e-08
1511 6.36649963809965e-08
1512 6.36346105005181e-08
1513 6.35910848667542e-08
1514 6.35656550223018e-08
1515 6.35155242236785e-08
1516 6.3493196442721e-08
1517 6.34416093423518e-08
1518 6.34209391896512e-08
1519 6.33670916387885e-08
1520 6.33485094057029e-08
1521 6.32909612390264e-08
1522 6.32782713467783e-08
1523 6.3212017867631e-08
1524 6.32119764052419e-08
1525 6.31313391834531e-08
1526 6.31359179541491e-08
1527 6.30571608692954e-08
1528 6.30654648761109e-08
1529 6.29812567951582e-08
1530 6.29744570246338e-08
1531 6.29148844306648e-08
1532 6.29280741253346e-08
1533 6.28321968987322e-08
1534 6.2799361391086e-08
1535 6.27930678613886e-08
1536 6.27425799706316e-08
1537 6.26730064201553e-08
1538 6.269428453054e-08
1539 6.26101973577597e-08
1540 6.25877191549051e-08
1541 6.25250583730086e-08
1542 6.25008228842816e-08
1543 6.24870518972642e-08
1544 6.24625505252219e-08
1545 6.23551860243765e-08
1546 6.23425156063284e-08
1547 6.23039307097173e-08
1548 6.22957413880521e-08
1549 6.22389972733117e-08
1550 6.21786498697929e-08
1551 6.21613043927383e-08
1552 6.21375715694761e-08
1553 6.20528720647329e-08
1554 6.2040987608647e-08
1555 6.20320081412551e-08
1556 6.19265369858901e-08
1557 6.19102981200825e-08
1558 6.1894668968332e-08
1559 6.18372144418622e-08
1560 6.18771531009088e-08
1561 6.17154843833845e-08
1562 6.18211534306745e-08
1563 6.16289074590881e-08
1564 6.17556631543614e-08
1565 6.15457859125534e-08
1566 6.16771917791858e-08
1567 6.14660165183523e-08
1568 6.1597350847542e-08
1569 6.13878953457103e-08
1570 6.15169014137606e-08
1571 6.13085481235842e-08
1572 6.14389783666347e-08
1573 6.12253851555167e-08
1574 6.13611144983928e-08
1575 6.11452378405097e-08
1576 6.12831616226828e-08
1577 6.10645409038213e-08
1578 6.1203811564603e-08
1579 6.09841270398093e-08
1580 6.11269653201241e-08
1581 6.09030024310009e-08
1582 6.104963206921e-08
1583 6.08225442082499e-08
1584 6.09721963269472e-08
1585 6.0742700896288e-08
1586 6.08937869777115e-08
1587 6.06639402089826e-08
1588 6.08149654657275e-08
1589 6.05847252721503e-08
1590 6.07383981687804e-08
1591 6.05060835328075e-08
1592 6.06598423842364e-08
1593 6.04286437293489e-08
1594 6.05820813555091e-08
1595 6.03509606071917e-08
1596 6.05027955180582e-08
1597 6.02736800350279e-08
1598 6.04260538947798e-08
1599 6.01961267467921e-08
1600 6.03479784420102e-08
1601 6.011953887608e-08
1602 6.02694119389291e-08
1603 6.00435106687947e-08
1604 6.01901555685735e-08
1605 5.9967548080575e-08
1606 6.01117892555791e-08
1607 5.98920840264583e-08
1608 6.00322398098818e-08
1609 5.98162515190737e-08
1610 5.99538779875175e-08
1611 5.97400722126551e-08
1612 5.98748851774467e-08
1613 5.96641585541846e-08
1614 5.97966919460013e-08
1615 5.95892282095889e-08
1616 5.97162576645438e-08
1617 5.9513393271704e-08
1618 5.96388672975401e-08
1619 5.94385278231968e-08
1620 5.95595537475901e-08
1621 5.93631851719678e-08
1622 5.94796705732925e-08
1623 5.92890015029823e-08
1624 5.93994416626664e-08
1625 5.92158084256056e-08
1626 5.93183008379405e-08
1627 5.91409715746671e-08
1628 5.92392956253462e-08
1629 5.90680417058564e-08
1630 5.9157350792205e-08
1631 5.89954411291949e-08
1632 5.90749584312711e-08
1633 5.89233953758317e-08
1634 5.89933236194717e-08
1635 5.88530896679096e-08
1636 5.89084077380342e-08
1637 5.87834741354598e-08
1638 5.88246625103928e-08
1639 5.87131923612816e-08
1640 5.87407328245249e-08
1641 5.86423724002749e-08
1642 5.86582772865718e-08
1643 5.85700126052302e-08
1644 5.85772581391808e-08
1645 5.84957905136463e-08
1646 5.84976455257902e-08
1647 5.84204297884305e-08
1648 5.84182950409406e-08
1649 5.83444807946698e-08
1650 5.83385117445268e-08
1651 5.82693339525875e-08
1652 5.82607283483583e-08
1653 5.81915474917594e-08
1654 5.81822983476421e-08
1655 5.81146691462919e-08
1656 5.81053379633545e-08
1657 5.80385381079296e-08
1658 5.80259813647288e-08
1659 5.79614929514527e-08
1660 5.79476879374319e-08
1661 5.7884129854191e-08
1662 5.78687597494465e-08
1663 5.78080841724393e-08
1664 5.77909750312244e-08
1665 5.77302060968954e-08
1666 5.77117497302382e-08
1667 5.76531422149529e-08
1668 5.76357309531872e-08
1669 5.75749956133187e-08
1670 5.75546880154931e-08
1671 5.74983891574732e-08
1672 5.74794329986084e-08
1673 5.74188316213586e-08
1674 5.73993617223323e-08
1675 5.73417539957433e-08
1676 5.73234548810753e-08
1677 5.72613663507582e-08
1678 5.72421914792898e-08
1679 5.71855802493637e-08
1680 5.71670860711748e-08
1681 5.71050141164875e-08
1682 5.7084367714122e-08
1683 5.70295267325882e-08
1684 5.70084421069872e-08
1685 5.69479245653426e-08
1686 5.69265640404382e-08
1687 5.68711093786867e-08
1688 5.68514103083118e-08
1689 5.67892752925125e-08
1690 5.67696430251452e-08
1691 5.67118437171388e-08
1692 5.66935493595544e-08
1693 5.66308193783449e-08
1694 5.6611147960961e-08
1695 5.65539178332131e-08
1696 5.65358012662998e-08
1697 5.6471037694461e-08
1698 5.64537724043745e-08
1699 5.63927282439813e-08
1700 5.63785449561038e-08
1701 5.63110744593587e-08
1702 5.6295190068667e-08
1703 5.62329479598667e-08
1704 5.62185478272248e-08
1705 5.61512749985837e-08
1706 5.61365572693262e-08
1707 5.60718037001529e-08
1708 5.60589472904383e-08
1709 5.59909387418678e-08
1710 5.59785994509454e-08
1711 5.59099312336109e-08
1712 5.58999428736939e-08
1713 5.58283276785865e-08
1714 5.58200227356309e-08
1715 5.57476780302935e-08
1716 5.57401185252715e-08
1717 5.56671843798817e-08
1718 5.56615447164788e-08
1719 5.55847915078012e-08
1720 5.55809227580362e-08
1721 5.55038834133548e-08
1722 5.55023399058108e-08
1723 5.54223671134224e-08
1724 5.54229257119232e-08
1725 5.53395330915585e-08
1726 5.53435286372306e-08
1727 5.52580756867371e-08
1728 5.52656458925327e-08
1729 5.51748073998048e-08
1730 5.51866722968697e-08
1731 5.50927735574191e-08
1732 5.5109692099542e-08
1733 5.50091379598605e-08
1734 5.50320511414348e-08
1735 5.49249938841534e-08
1736 5.49511325100838e-08
1737 5.48431016027529e-08
1738 5.48634302801609e-08
1739 5.47611333683307e-08
1740 5.47876530760938e-08
1741 5.46772301412979e-08
1742 5.46882963914364e-08
1743 5.45976105716939e-08
1744 5.4635328146535e-08
1745 5.45120997337278e-08
1746 5.44617731761399e-08
1747 5.44861210061143e-08
1748 5.43721089782956e-08
1749 5.44912571216472e-08
1750 5.42484918533148e-08
1751 5.43107595696313e-08
1752 5.42884023739809e-08
1753 5.41836630261194e-08
1754 5.41348215099546e-08
1755 5.4182345638587e-08
1756 5.40288127455213e-08
1757 5.40611828601101e-08
1758 5.39859488335459e-08
1759 5.39780799551082e-08
1760 5.38837344636889e-08
1761 5.38880924136897e-08
1762 5.38964851397594e-08
1763 5.3771111354628e-08
1764 5.37255082093679e-08
1765 5.37110039742217e-08
1766 5.3645456860707e-08
1767 5.37190197351656e-08
1768 5.35438715414394e-08
1769 5.35362007840767e-08
1770 5.34761928827621e-08
1771 5.35448288312423e-08
1772 5.33754491067739e-08
1773 5.33750805402633e-08
1774 5.33125139146051e-08
1775 5.33005192293601e-08
1776 5.32354687359238e-08
1777 5.32548275087308e-08
1778 5.31278574782412e-08
1779 5.31884018211137e-08
1780 5.305180049886e-08
1781 5.30406115513138e-08
1782 5.29898539145535e-08
1783 5.29582605652834e-08
1784 5.29055667355394e-08
1785 5.28744819048299e-08
1786 5.28276990290344e-08
1787 5.27909758711864e-08
1788 5.27431109906829e-08
1789 5.2714003496046e-08
1790 5.26658151720483e-08
1791 5.26194021657211e-08
1792 5.25789413559075e-08
1793 5.25357466027998e-08
1794 5.24955727376764e-08
1795 5.24523910025998e-08
1796 5.24112405653199e-08
1797 5.23700398664673e-08
1798 5.23279215647321e-08
1799 5.2287163559761e-08
1800 5.22453462745709e-08
1801 5.22035429408874e-08
1802 5.21622120710496e-08
1803 5.21204909631479e-08
1804 5.20784633812887e-08
1805 5.20376862809258e-08
1806 5.19960806872888e-08
1807 5.1954690319711e-08
1808 5.19118808375296e-08
1809 5.18711561747764e-08
1810 5.18285554043096e-08
1811 5.17881578265822e-08
1812 5.17463959543996e-08
1813 5.1705255818657e-08
1814 5.16631006699519e-08
1815 5.1621881555608e-08
1816 5.15810384844606e-08
1817 5.15392692750361e-08
1818 5.14983133244051e-08
1819 5.14579261894355e-08
1820 5.14145513244024e-08
1821 5.13760479123881e-08
1822 5.13337573493367e-08
1823 5.12942287347329e-08
1824 5.12553903888602e-08
1825 5.12158723182665e-08
1826 5.11781159620028e-08
1827 5.11434236547537e-08
1828 5.1114754670234e-08
1829 5.10719433388651e-08
1830 5.1023051816923e-08
1831 5.0981525274274e-08
1832 5.09403211195192e-08
1833 5.08995429138182e-08
1834 5.0858694124134e-08
1835 5.08166464654458e-08
1836 5.07767308897122e-08
1837 5.07356398813386e-08
1838 5.06957866264202e-08
1839 5.06541106188863e-08
1840 5.06137260964934e-08
1841 5.05718152306045e-08
1842 5.05325935931111e-08
1843 5.04907590226367e-08
1844 5.0450579222705e-08
1845 5.04102696265996e-08
1846 5.03699806144731e-08
1847 5.03289518953842e-08
1848 5.02890734739303e-08
1849 5.02479479687068e-08
1850 5.02073175376871e-08
1851 5.01670683030753e-08
1852 5.0127268263811e-08
1853 5.00869110267033e-08
1854 5.00462139307878e-08
1855 5.00059739692027e-08
1856 4.99663007440532e-08
1857 4.99258455342044e-08
1858 4.98857198025782e-08
1859 4.98449834824832e-08
1860 4.98055720412616e-08
1861 4.97659144831353e-08
1862 4.97261023504336e-08
1863 4.96857117981975e-08
1864 4.96451923028829e-08
1865 4.96055663341544e-08
1866 4.95652287559878e-08
1867 4.95271812126497e-08
1868 4.94856851327441e-08
1869 4.94480312660528e-08
1870 4.94065926130993e-08
1871 4.93672317616323e-08
1872 4.93284616212009e-08
1873 4.92884739458077e-08
1874 4.92495564516915e-08
1875 4.9209717488008e-08
1876 4.91701868421401e-08
1877 4.91311781516401e-08
1878 4.90910223178709e-08
1879 4.90523533622778e-08
1880 4.90123098200179e-08
1881 4.8974328321183e-08
1882 4.89341253526732e-08
1883 4.88955821187353e-08
1884 4.88563162601707e-08
1885 4.88171541364046e-08
1886 4.87788570024961e-08
1887 4.87390465324644e-08
1888 4.8699822375653e-08
1889 4.86619833037771e-08
1890 4.86231492144995e-08
1891 4.85845420628017e-08
1892 4.85444112188205e-08
1893 4.85065022379771e-08
1894 4.84684236101707e-08
1895 4.84287115134485e-08
1896 4.83913705666694e-08
1897 4.83526076764385e-08
1898 4.83137368929576e-08
1899 4.8275671446607e-08
1900 4.82365996488099e-08
1901 4.82002037176521e-08
1902 4.81609236202551e-08
1903 4.81231015760919e-08
1904 4.80854830748889e-08
1905 4.80473289172778e-08
1906 4.80103513269547e-08
1907 4.79733016618411e-08
1908 4.79357831530969e-08
1909 4.78989631784721e-08
1910 4.78629322628876e-08
1911 4.78283773990285e-08
1912 4.77951981183722e-08
1913 4.77663061153599e-08
1914 4.77487991981107e-08
1915 4.77646246301866e-08
1916 4.78029671242552e-08
1917 4.77596082255616e-08
1918 4.7705753885019e-08
1919 4.76701297720261e-08
1920 4.76321043154648e-08
1921 4.75941880520026e-08
1922 4.75570658542424e-08
1923 4.75188074982036e-08
1924 4.74813617099557e-08
1925 4.74435356885294e-08
1926 4.74064253888073e-08
1927 4.73689836995028e-08
1928 4.73302763541739e-08
1929 4.72945239242684e-08
1930 4.72559722726196e-08
1931 4.72187432669635e-08
1932 4.71809186866068e-08
1933 4.7143993401999e-08
1934 4.71067563290184e-08
1935 4.70691380560773e-08
1936 4.70324427723767e-08
1937 4.69945251184711e-08
1938 4.69578123318826e-08
1939 4.69207801190308e-08
1940 4.68836116112037e-08
1941 4.68461190377134e-08
1942 4.68096606218715e-08
1943 4.67733569444739e-08
1944 4.6735207820614e-08
1945 4.66986611931119e-08
1946 4.66620070889157e-08
1947 4.66252436619463e-08
1948 4.65893608647683e-08
1949 4.65516483152406e-08
1950 4.65151871300584e-08
1951 4.64795559742548e-08
1952 4.64424211732428e-08
1953 4.640683350976e-08
1954 4.63695721490964e-08
1955 4.63339684633191e-08
1956 4.62973117736354e-08
1957 4.62610671547736e-08
1958 4.62262692848547e-08
1959 4.61885549754015e-08
1960 4.61543553988974e-08
1961 4.61171560388607e-08
1962 4.60825307109225e-08
1963 4.60458336117853e-08
1964 4.60092340586193e-08
1965 4.59740251099738e-08
1966 4.59389464118054e-08
1967 4.59032827633266e-08
1968 4.58677202281876e-08
1969 4.58329639081434e-08
1970 4.5795988989461e-08
1971 4.57613676330126e-08
1972 4.57265454425482e-08
1973 4.56915518856071e-08
1974 4.5656116363979e-08
1975 4.56201222256425e-08
1976 4.55857986447761e-08
1977 4.55509627959039e-08
1978 4.55159573640174e-08
1979 4.54814051638053e-08
1980 4.54463913577285e-08
1981 4.54112818233376e-08
1982 4.53773602857588e-08
1983 4.53428593663041e-08
1984 4.53083351459327e-08
1985 4.52730330597895e-08
1986 4.52390940783864e-08
1987 4.52048363843716e-08
1988 4.51701135246729e-08
1989 4.51363818729789e-08
1990 4.51013985922621e-08
1991 4.50681954053955e-08
1992 4.50338536248651e-08
1993 4.49995147682181e-08
1994 4.49654328082971e-08
1995 4.49325563076641e-08
1996 4.4897817764511e-08
1997 4.48633359377837e-08
1998 4.48301050059996e-08
1999 4.47966383649856e-08
2000 4.4762951463273e-08
2001 4.47286947951042e-08
2002 4.46951008208352e-08
2003 4.46626557919494e-08
2004 4.46282368811168e-08
2005 4.45959317536548e-08
2006 4.45608276669063e-08
2007 4.45287004655626e-08
2008 4.44952614406802e-08
2009 4.44620801141049e-08
2010 4.44284509910631e-08
2011 4.43957876781198e-08
2012 4.43623545307581e-08
2013 4.43300486705489e-08
2014 4.42971152052607e-08
2015 4.42633960071603e-08
2016 4.42315200710475e-08
2017 4.41994888351616e-08
2018 4.4165705364474e-08
2019 4.41328745339753e-08
2020 4.41008706211221e-08
2021 4.40685614502279e-08
2022 4.40351364443536e-08
2023 4.40034003292844e-08
2024 4.3971217923211e-08
2025 4.39384419679278e-08
2026 4.39070923361484e-08
2027 4.38743232149541e-08
2028 4.38424226718581e-08
2029 4.38095272636829e-08
2030 4.37791324507941e-08
2031 4.37463130915638e-08
2032 4.3714059283495e-08
2033 4.36836003614438e-08
2034 4.36507637560091e-08
2035 4.36193877941804e-08
2036 4.35880758580254e-08
2037 4.35564619887252e-08
2038 4.35250337740278e-08
2039 4.34940683087248e-08
2040 4.34616182940495e-08
2041 4.34316143840263e-08
2042 4.33998045945572e-08
2043 4.33686962657553e-08
2044 4.33374875381531e-08
2045 4.33071642045313e-08
2046 4.32765868221274e-08
2047 4.3245596449637e-08
2048 4.3213707430656e-08
2049 4.31825447568812e-08
2050 4.3152142116476e-08
2051 4.3121598170881e-08
2052 4.30900646026799e-08
2053 4.30606618242457e-08
2054 4.30291414219575e-08
2055 4.2999457418702e-08
2056 4.2969000404014e-08
2057 4.29381534448048e-08
2058 4.29081413857446e-08
2059 4.28772673419786e-08
2060 4.2847651128497e-08
2061 4.28169565984682e-08
2062 4.27864515972765e-08
2063 4.27565868577062e-08
2064 4.27257373529777e-08
2065 4.26950957370664e-08
2066 4.26658473271502e-08
2067 4.26352508360317e-08
2068 4.26052309423497e-08
2069 4.25753244654992e-08
2070 4.25450404555328e-08
2071 4.25143573443698e-08
2072 4.24846889823804e-08
2073 4.24546701855988e-08
2074 4.24246307715315e-08
2075 4.23931744202299e-08
2076 4.23631614974163e-08
2077 4.23328723799798e-08
2078 4.23018022122079e-08
2079 4.22714549856984e-08
2080 4.22405776663304e-08
2081 4.22099574546309e-08
2082 4.21776878525293e-08
2083 4.21459010375713e-08
2084 4.21122162714838e-08
2085 4.20784552104259e-08
2086 4.2041940075066e-08
2087 4.20055053393931e-08
2088 4.19667591056161e-08
2089 4.1926916425794e-08
2090 4.18895211220516e-08
2091 4.18545753340105e-08
2092 4.18234219963232e-08
2093 4.17953496052448e-08
2094 4.17653814048791e-08
2095 4.17382159652036e-08
2096 4.17102396612634e-08
2097 4.16826362030776e-08
2098 4.16546680797047e-08
2099 4.16268248808471e-08
2100 4.15996525666706e-08
2101 4.15713304069421e-08
2102 4.15440511991605e-08
2103 4.15175954757707e-08
2104 4.14886466342246e-08
2105 4.14623651288082e-08
2106 4.14352316964184e-08
2107 4.14080371378134e-08
2108 4.13804907641868e-08
2109 4.13535566705647e-08
2110 4.13263974294864e-08
2111 4.13003947024038e-08
2112 4.12728050500633e-08
2113 4.12468751158634e-08
2114 4.12190053826755e-08
2115 4.11921155505368e-08
2116 4.11672737157254e-08
2117 4.11401712745452e-08
2118 4.11128625952273e-08
2119 4.10871443317085e-08
2120 4.10602485496625e-08
2121 4.1034772767734e-08
2122 4.10075683805466e-08
2123 4.098261720209e-08
2124 4.09548825008876e-08
2125 4.0930691099561e-08
2126 4.09031990513675e-08
2127 4.08783275256397e-08
2128 4.08522201986727e-08
2129 4.08256463586554e-08
2130 4.0801115138045e-08
2131 4.07756408247195e-08
2132 4.07495717147377e-08
2133 4.07233555965725e-08
2134 4.06977592040114e-08
2135 4.06733747144727e-08
2136 4.06479177756935e-08
2137 4.06214843580166e-08
2138 4.05977389741352e-08
2139 4.05720286815736e-08
2140 4.05461233223825e-08
2141 4.05223548125555e-08
2142 4.04968206204615e-08
2143 4.04719326563274e-08
2144 4.04463033127911e-08
2145 4.04219729279731e-08
2146 4.03972851921175e-08
2147 4.03728048170571e-08
2148 4.03475281154364e-08
2149 4.03235503325661e-08
2150 4.02991160992627e-08
2151 4.02739907023886e-08
2152 4.02496898384008e-08
2153 4.02259109288927e-08
2154 4.02015074674189e-08
2155 4.01768606055342e-08
2156 4.01526361790694e-08
2157 4.01292489562444e-08
2158 4.01053185585809e-08
2159 4.0080901421824e-08
2160 4.00572929653009e-08
2161 4.00332917900315e-08
2162 4.00089907564016e-08
2163 3.99860147832776e-08
2164 3.99622526479071e-08
2165 3.99380415490036e-08
2166 3.99156174477611e-08
2167 3.98916612032174e-08
2168 3.98679061737184e-08
2169 3.98451992555948e-08
2170 3.98204846550065e-08
2171 3.97992558096583e-08
2172 3.97746644549279e-08
2173 3.97512741727724e-08
2174 3.97292001284022e-08
2175 3.97048141635992e-08
2176 3.96816157741675e-08
2177 3.96590025211196e-08
2178 3.96354745726413e-08
2179 3.96133311770797e-08
2180 3.95903794845331e-08
2181 3.95656354821305e-08
2182 3.95451772514654e-08
2183 3.95199955200987e-08
2184 3.94990910481674e-08
2185 3.94753367105594e-08
2186 3.9453529696587e-08
2187 3.94289666161285e-08
2188 3.94081207164554e-08
2189 3.93841337680279e-08
2190 3.93629992587741e-08
2191 3.93382092540584e-08
2192 3.93185272953289e-08
2193 3.92910808035829e-08
2194 3.92772869113323e-08
2195 3.92397375676978e-08
2196 3.92414415610887e-08
2197 3.91796677523537e-08
2198 3.92032313221158e-08
2199 3.91394903145326e-08
2200 3.91636169423393e-08
2201 3.90745604303966e-08
2202 3.91278679892082e-08
2203 3.90399555492138e-08
2204 3.90838805599536e-08
2205 3.8979630671232e-08
2206 3.90485865326973e-08
2207 3.89365100055272e-08
2208 3.90056526695659e-08
2209 3.88879566806111e-08
2210 3.89694660514195e-08
2211 3.88384525460417e-08
2212 3.89297541518729e-08
2213 3.87912836821158e-08
2214 3.88884143642798e-08
2215 3.8747592962185e-08
2216 3.88467625462852e-08
2217 3.8703581437094e-08
2218 3.88048276085406e-08
2219 3.86600582382712e-08
2220 3.87627793396739e-08
2221 3.86152745539547e-08
2222 3.87214697092908e-08
2223 3.85737888382032e-08
2224 3.8680222962828e-08
2225 3.85284694037757e-08
2226 3.86390304600859e-08
2227 3.84871867695757e-08
2228 3.85989832154543e-08
2229 3.84433431031006e-08
2230 3.85575746815192e-08
2231 3.8401116427611e-08
2232 3.85180565936061e-08
2233 3.83576617668524e-08
2234 3.84776510840013e-08
2235 3.83138921904447e-08
2236 3.84391525378724e-08
2237 3.8273406789191e-08
2238 3.83977907016941e-08
2239 3.82304944306355e-08
2240 3.83592859383164e-08
2241 3.81879788351469e-08
2242 3.83188989747651e-08
2243 3.81484025608891e-08
2244 3.82788065005357e-08
2245 3.81059107419368e-08
2246 3.8238509044497e-08
2247 3.80650247255865e-08
2248 3.82011927588799e-08
2249 3.80223844231864e-08
2250 3.81615149276371e-08
2251 3.79828943422034e-08
2252 3.81213098692434e-08
2253 3.79427957795109e-08
2254 3.80828345827311e-08
2255 3.79011523565786e-08
2256 3.8043918811681e-08
2257 3.78627739014981e-08
2258 3.80045399372975e-08
2259 3.78210160461023e-08
2260 3.79673994235219e-08
2261 3.77820405734752e-08
2262 3.79279705020608e-08
2263 3.77409222318548e-08
2264 3.7890883809677e-08
2265 3.77027193230184e-08
2266 3.78502586100282e-08
2267 3.76633243122093e-08
2268 3.78125736064661e-08
2269 3.76241801238741e-08
2270 3.77743444404111e-08
2271 3.75835243504596e-08
2272 3.77385387819729e-08
2273 3.75454144432297e-08
2274 3.76988873300732e-08
2275 3.75064974798001e-08
2276 3.76603811491627e-08
2277 3.74700665757288e-08
2278 3.76239032284786e-08
2279 3.74299636076714e-08
2280 3.75852321079506e-08
2281 3.73917065288332e-08
2282 3.75480648449056e-08
2283 3.73554906598628e-08
2284 3.75107046726164e-08
2285 3.73168058973583e-08
2286 3.74717152880066e-08
2287 3.7279406129187e-08
2288 3.74354347543182e-08
2289 3.72415922971214e-08
2290 3.73984610884115e-08
2291 3.72044464769239e-08
2292 3.73617098934353e-08
2293 3.71643075949279e-08
2294 3.73264020163688e-08
2295 3.71289832652444e-08
2296 3.72873537357599e-08
2297 3.7093046917569e-08
2298 3.72508904940005e-08
2299 3.7055179295642e-08
2300 3.72140091586814e-08
2301 3.70173054027312e-08
2302 3.71789570787762e-08
2303 3.69809641376229e-08
2304 3.71429168759541e-08
2305 3.69441102665569e-08
2306 3.71048946656849e-08
2307 3.6909114619732e-08
2308 3.70686004025345e-08
2309 3.68730108171178e-08
2310 3.70339217514726e-08
2311 3.68360484341856e-08
2312 3.69975482925611e-08
2313 3.68002916393273e-08
2314 3.69608670438559e-08
2315 3.67639439464718e-08
2316 3.69251736631604e-08
2317 3.67288303402269e-08
2318 3.68901633596153e-08
2319 3.66928549686563e-08
2320 3.68535362778033e-08
2321 3.66590885456652e-08
2322 3.68164335595722e-08
2323 3.6623797097679e-08
2324 3.67824608713541e-08
2325 3.65886238613555e-08
2326 3.67462312622635e-08
2327 3.65530200809872e-08
2328 3.67132198948461e-08
2329 3.65155276518259e-08
2330 3.66783117424774e-08
2331 3.64815517222006e-08
2332 3.66421712438836e-08
2333 3.6448317870974e-08
2334 3.66080311957617e-08
2335 3.64116739635101e-08
2336 3.65722335677887e-08
2337 3.63789867980913e-08
2338 3.65374589637035e-08
2339 3.63447936688743e-08
2340 3.65021665400533e-08
2341 3.63116222983351e-08
2342 3.64670681012313e-08
2343 3.62766465271669e-08
2344 3.64340241358896e-08
2345 3.62410846777017e-08
2346 3.64007277311451e-08
2347 3.62070897903521e-08
2348 3.63668632292757e-08
2349 3.61735956335707e-08
2350 3.63306137414199e-08
2351 3.61416230463263e-08
2352 3.62970488065883e-08
2353 3.61062423261949e-08
2354 3.62629700285666e-08
2355 3.60747799792449e-08
2356 3.62283469921287e-08
2357 3.60411275970307e-08
2358 3.61952166501034e-08
2359 3.60076651264585e-08
2360 3.61616131967502e-08
2361 3.59736437829383e-08
2362 3.61289547314847e-08
2363 3.59404377161532e-08
2364 3.60936745353868e-08
2365 3.59096676061377e-08
2366 3.60594480603993e-08
2367 3.58751884430575e-08
2368 3.60253457021287e-08
2369 3.58438533187844e-08
2370 3.59923589550171e-08
2371 3.58115217706789e-08
2372 3.59583741618152e-08
2373 3.5779197174346e-08
2374 3.59251948984785e-08
2375 3.57452262780278e-08
2376 3.58902166324171e-08
2377 3.57156807959669e-08
2378 3.58582100323801e-08
2379 3.56812921933347e-08
2380 3.58240186386638e-08
2381 3.56489615289668e-08
2382 3.57905484951182e-08
2383 3.56177735629792e-08
2384 3.57585118280213e-08
2385 3.55842070689505e-08
2386 3.57235857206817e-08
2387 3.55538374443576e-08
2388 3.56918112602855e-08
2389 3.55226888286708e-08
2390 3.56561564074376e-08
2391 3.54923951122466e-08
2392 3.56231550613373e-08
2393 3.54602189434061e-08
2394 3.55889282275257e-08
2395 3.54306804069005e-08
2396 3.55577918806915e-08
2397 3.5396202979765e-08
2398 3.55261960112152e-08
2399 3.5367217859239e-08
2400 3.54923745917723e-08
2401 3.53345514869652e-08
2402 3.54594138438635e-08
2403 3.53046951997449e-08
2404 3.54270249496302e-08
2405 3.52729646957428e-08
2406 3.53949957294653e-08
2407 3.52413134971918e-08
2408 3.53623223485755e-08
2409 3.52123783637914e-08
2410 3.53295375070672e-08
2411 3.51814835788389e-08
2412 3.52972513333327e-08
2413 3.51497353623387e-08
2414 3.52656927993245e-08
2415 3.51200124883988e-08
2416 3.52330158097658e-08
2417 3.50895473983925e-08
2418 3.520001875712e-08
2419 3.50594606577381e-08
2420 3.51697617242408e-08
2421 3.50285431447439e-08
2422 3.51385076342581e-08
2423 3.49965910499961e-08
2424 3.51041963875254e-08
2425 3.49689857035429e-08
2426 3.50740305048447e-08
2427 3.49385292692794e-08
2428 3.50431642197613e-08
2429 3.49092060845102e-08
2430 3.50104608570767e-08
2431 3.48791038171647e-08
2432 3.49774652868007e-08
2433 3.485031450845e-08
2434 3.49466223714678e-08
2435 3.48199860451537e-08
2436 3.49161054757907e-08
2437 3.47916090581712e-08
2438 3.48849208569035e-08
2439 3.47581194466429e-08
2440 3.48555303295584e-08
2441 3.47294850806534e-08
2442 3.48226841504484e-08
2443 3.47030682719129e-08
2444 3.47907692210825e-08
2445 3.46738737029284e-08
2446 3.47593450022998e-08
2447 3.46420106658307e-08
2448 3.47291547670991e-08
2449 3.46151566916753e-08
2450 3.469967499381e-08
2451 3.45854134033985e-08
2452 3.46679996998667e-08
2453 3.45563529933202e-08
2454 3.46375327353599e-08
2455 3.45284110778721e-08
2456 3.46065165763854e-08
2457 3.44979093314102e-08
2458 3.45779171491145e-08
2459 3.44700095280537e-08
2460 3.4545373088779e-08
2461 3.44426802936226e-08
2462 3.45154431156125e-08
2463 3.4414049346676e-08
2464 3.44849597930796e-08
2465 3.43848280701664e-08
2466 3.44550928130793e-08
2467 3.4354174984319e-08
2468 3.44280574475775e-08
2469 3.43272049332377e-08
2470 3.43947017387158e-08
2471 3.4299837987195e-08
2472 3.43651122172162e-08
2473 3.42717122499003e-08
2474 3.43339454400571e-08
2475 3.42454171908813e-08
2476 3.4306402846962e-08
2477 3.42136969120332e-08
2478 3.42768697514373e-08
2479 3.41868404984957e-08
2480 3.42456015323123e-08
2481 3.41604675893059e-08
2482 3.4217632520761e-08
2483 3.41309523070876e-08
2484 3.41886018611071e-08
2485 3.41024483363483e-08
2486 3.41630019260464e-08
2487 3.40721411733469e-08
2488 3.41298680290869e-08
2489 3.40488622461166e-08
2490 3.41071972411733e-08
2491 3.40184642739771e-08
2492 3.40579258013563e-08
2493 3.39998166296596e-08
2494 3.40722725447051e-08
2495 3.39608833570537e-08
2496 3.39800728301753e-08
2497 3.40122165001233e-08
2498 3.39229559060961e-08
2499 3.39377450160327e-08
2500 3.39619018361326e-08
2501 3.38839800599899e-08
2502 3.38992406665461e-08
2503 3.38772759156392e-08
2504 3.38672879309776e-08
2505 3.38994664250691e-08
2506 3.38155905792448e-08
2507 3.38281896938852e-08
2508 3.38592922095593e-08
2509 3.37762003570496e-08
2510 3.37857350505288e-08
2511 3.38171884535221e-08
2512 3.37354736035245e-08
2513 3.37441375051206e-08
2514 3.37772934191349e-08
2515 3.36965343028517e-08
2516 3.37017122635785e-08
2517 3.37376206771545e-08
2518 3.36524033892616e-08
2519 3.36635046420675e-08
2520 3.36958018354316e-08
2521 3.36140216457004e-08
2522 3.36223681145498e-08
2523 3.36541885883612e-08
2524 3.35751311801857e-08
2525 3.35810401352887e-08
2526 3.36152149458258e-08
2527 3.35335594927599e-08
2528 3.35431043763101e-08
2529 3.35731242455672e-08
2530 3.34965065436066e-08
2531 3.35027278093314e-08
2532 3.35350023443759e-08
2533 3.34559073040808e-08
2534 3.34632049123762e-08
2535 3.34968481157105e-08
2536 3.34160171404641e-08
2537 3.34227565415013e-08
2538 3.34593592330457e-08
2539 3.33770105047648e-08
2540 3.33837192481035e-08
2541 3.34225050511172e-08
2542 3.33378725714262e-08
2543 3.33423693015256e-08
2544 3.33852544720514e-08
2545 3.32999086816166e-08
2546 3.33020906420067e-08
2547 3.33489475590199e-08
2548 3.32595432230853e-08
2549 3.32639517113265e-08
2550 3.33086214405753e-08
2551 3.32228978541238e-08
2552 3.32239520854927e-08
2553 3.32701143683778e-08
2554 3.31849169876719e-08
2555 3.31838095029013e-08
2556 3.32324532403838e-08
2557 3.31455203830266e-08
2558 3.31481596784755e-08
2559 3.31922482126323e-08
2560 3.31092247440878e-08
2561 3.31062615446243e-08
2562 3.31553535901641e-08
2563 3.30695873174136e-08
2564 3.30698338593116e-08
2565 3.31158429802692e-08
2566 3.30313724030695e-08
2567 3.30314510441632e-08
2568 3.30763766291931e-08
2569 3.29946823578453e-08
2570 3.29942109575931e-08
2571 3.30356640634299e-08
2572 3.29571301929121e-08
2573 3.29551230993097e-08
2574 3.29976481023842e-08
2575 3.29195021753215e-08
2576 3.29163240504471e-08
2577 3.2960025224682e-08
2578 3.28815670909144e-08
2579 3.28793827417684e-08
2580 3.29197469484122e-08
2581 3.28460829859267e-08
2582 3.28410651611044e-08
2583 3.28823732798078e-08
2584 3.28090226040523e-08
2585 3.28037557619787e-08
2586 3.28414766754825e-08
2587 3.27721243831292e-08
2588 3.27664418193763e-08
2589 3.28041945132362e-08
2590 3.27352000994985e-08
2591 3.27296441606784e-08
2592 3.27649130880126e-08
2593 3.26990452879095e-08
2594 3.26910971275574e-08
2595 3.27269730564517e-08
2596 3.26631284406353e-08
2597 3.26555227641023e-08
2598 3.26872843898762e-08
2599 3.26254890712363e-08
2600 3.26193064190683e-08
2601 3.26527234051355e-08
2602 3.25891094625774e-08
2603 3.25822411828547e-08
2604 3.26076714753931e-08
2605 3.25506176959145e-08
2606 3.2546782346099e-08
2607 3.2579471489047e-08
2608 3.25169941590708e-08
2609 3.2508287711952e-08
2610 3.25198214774503e-08
2611 3.24795105699494e-08
2612 3.24936851798796e-08
2613 3.24564131153338e-08
2614 3.24775002438216e-08
2615 3.24288700124242e-08
2616 3.24343770645719e-08
2617 3.24169322967194e-08
2618 3.24074330828594e-08
2619 3.2394708347816e-08
2620 3.23811150901854e-08
2621 3.23829448953816e-08
2622 3.23478387032061e-08
2623 3.23516476417041e-08
2624 3.23289995800558e-08
2625 3.23327782947302e-08
2626 3.23001074389317e-08
2627 3.2306351481548e-08
2628 3.22770493679236e-08
2629 3.22973422384187e-08
2630 3.22506266368094e-08
2631 3.22506179326609e-08
2632 3.22459845474299e-08
2633 3.22168141981827e-08
2634 3.22234616794148e-08
2635 3.21938474434624e-08
2636 3.22073619947183e-08
2637 3.21661849169708e-08
2638 3.21740986084507e-08
2639 3.21516366712515e-08
2640 3.21457684595572e-08
2641 3.21298236012346e-08
2642 3.21225018478088e-08
2643 3.21054347245386e-08
2644 3.21003277856669e-08
2645 3.20763508092625e-08
2646 3.20809485132578e-08
2647 3.2051059323468e-08
2648 3.20577033123826e-08
2649 3.20294384015263e-08
2650 3.20333908985226e-08
2651 3.20047273132396e-08
2652 3.20094331773824e-08
2653 3.19841115432595e-08
2654 3.19858306490062e-08
2655 3.1959727228692e-08
2656 3.1960925036767e-08
2657 3.19376033188767e-08
2658 3.19365907239799e-08
2659 3.19155822188222e-08
2660 3.19145097287255e-08
2661 3.18921206852174e-08
2662 3.18904983629409e-08
2663 3.18703480983729e-08
2664 3.18664834235705e-08
2665 3.18479216629974e-08
2666 3.18430331049946e-08
2667 3.18256398181838e-08
2668 3.18198369142486e-08
2669 3.18038012592226e-08
2670 3.17955980295714e-08
2671 3.17823734969913e-08
2672 3.17718719227145e-08
2673 3.17592284124757e-08
2674 3.17485402621109e-08
2675 3.17363728652431e-08
2676 3.1727100778145e-08
2677 3.1713353851881e-08
2678 3.17039758508386e-08
2679 3.16918497267338e-08
2680 3.16810639011678e-08
2681 3.16697991613601e-08
2682 3.16579568706565e-08
2683 3.16474663746291e-08
2684 3.16359110179576e-08
2685 3.16259083783521e-08
2686 3.16138997216164e-08
2687 3.16032050347026e-08
2688 3.15923573221966e-08
2689 3.15802467820703e-08
2690 3.15693109151916e-08
2691 3.15577690903623e-08
2692 3.15485198423282e-08
2693 3.15357284721429e-08
2694 3.15260980983112e-08
2695 3.15140953337512e-08
2696 3.15028304376241e-08
2697 3.14911714140997e-08
2698 3.14809012920136e-08
2699 3.14701589001487e-08
2700 3.14581552638415e-08
2701 3.14473010001315e-08
2702 3.14353658890099e-08
2703 3.14248987978161e-08
2704 3.14138937440589e-08
2705 3.14024867269502e-08
2706 3.13913195268967e-08
2707 3.13802737190727e-08
2708 3.13699184850691e-08
2709 3.13570313608125e-08
2710 3.134646436731e-08
2711 3.13348494698218e-08
2712 3.13240740283938e-08
2713 3.13129732836259e-08
2714 3.13014242609988e-08
2715 3.12909097659464e-08
2716 3.1278983606331e-08
2717 3.12678379588149e-08
2718 3.12573709870811e-08
2719 3.12449781825208e-08
2720 3.12341747785716e-08
2721 3.12234218378116e-08
2722 3.12116198988299e-08
2723 3.12007660117075e-08
2724 3.11890111626134e-08
2725 3.11781108823794e-08
2726 3.11669610844056e-08
2727 3.11557800043438e-08
2728 3.1143995313343e-08
2729 3.1131308746879e-08
2730 3.11215393939968e-08
2731 3.11099451408836e-08
2732 3.10987956484432e-08
2733 3.10879113798457e-08
2734 3.10775010490261e-08
2735 3.1066548852543e-08
2736 3.10552317435153e-08
2737 3.10436639101574e-08
2738 3.10341553668714e-08
2739 3.10222634318791e-08
2740 3.10112821519937e-08
2741 3.10006979398203e-08
2742 3.09896871639737e-08
2743 3.09781435392509e-08
2744 3.09673375102903e-08
2745 3.09567194785032e-08
2746 3.09453043998431e-08
2747 3.09352213800906e-08
2748 3.09237199340728e-08
2749 3.0913129666299e-08
2750 3.09021727331604e-08
2751 3.08918867384378e-08
2752 3.08795152927921e-08
2753 3.08696056774416e-08
2754 3.08582088237586e-08
2755 3.0847463626138e-08
2756 3.08367939343412e-08
2757 3.08252606262549e-08
2758 3.08158037212181e-08
2759 3.08038387122345e-08
2760 3.07937602168629e-08
2761 3.07815401341394e-08
2762 3.07716514695855e-08
2763 3.07609168515022e-08
2764 3.07496226867876e-08
2765 3.07392779790305e-08
2766 3.07281756963818e-08
2767 3.07176432019496e-08
2768 3.07060502415801e-08
2769 3.06959079647839e-08
2770 3.06853570557486e-08
2771 3.06743741664839e-08
2772 3.06635838680513e-08
2773 3.06522672399723e-08
2774 3.06415229243129e-08
2775 3.06315455977924e-08
2776 3.06209138498659e-08
2777 3.0609871845666e-08
2778 3.0599314606139e-08
2779 3.05882679527691e-08
2780 3.05771073074723e-08
2781 3.05671825517884e-08
2782 3.05563888329807e-08
2783 3.05457555169752e-08
2784 3.05342596305103e-08
2785 3.05246600604825e-08
2786 3.05138311289532e-08
2787 3.05024280282673e-08
2788 3.04925266858547e-08
2789 3.04812608327154e-08
2790 3.047101095488e-08
2791 3.04607455952066e-08
2792 3.04502735035683e-08
2793 3.04382189391106e-08
2794 3.04287786452306e-08
2795 3.04178188570425e-08
2796 3.04067645093831e-08
2797 3.03965438286369e-08
2798 3.03863413662064e-08
2799 3.03753913639682e-08
2800 3.0364924086701e-08
2801 3.03545141044914e-08
2802 3.03427916064081e-08
2803 3.03324581349962e-08
2804 3.03221695476807e-08
2805 3.03117269759845e-08
2806 3.03013721878465e-08
2807 3.02905341560411e-08
2808 3.02794228503878e-08
2809 3.02695959546462e-08
2810 3.02583002773638e-08
2811 3.02489162247177e-08
2812 3.02373029446024e-08
2813 3.02269120204812e-08
2814 3.02159184744077e-08
2815 3.02061810466547e-08
2816 3.01949972816296e-08
2817 3.01849799941856e-08
2818 3.01750513114207e-08
2819 3.01642766871169e-08
2820 3.01543111382863e-08
2821 3.01431434026611e-08
2822 3.01325977631883e-08
2823 3.01234045281262e-08
2824 3.01121514740821e-08
2825 3.01024697342278e-08
2826 3.0092127093706e-08
2827 3.00818444549655e-08
2828 3.00715218695124e-08
2829 3.00621647948418e-08
2830 3.00517368896358e-08
2831 3.00429490880383e-08
2832 3.00325212152508e-08
2833 3.00229261633866e-08
2834 3.00122255070257e-08
2835 3.00032062008349e-08
2836 2.99929830336332e-08
2837 2.99840864999012e-08
2838 2.99751647250268e-08
2839 2.99648802797314e-08
2840 2.99561671548432e-08
2841 2.99463089930008e-08
2842 2.99367011833418e-08
2843 2.99277901922856e-08
2844 2.99193765695982e-08
2845 2.9909646956483e-08
2846 2.99020157710217e-08
2847 2.98926340680517e-08
2848 2.98841636903191e-08
2849 2.98756132788292e-08
2850 2.98678197134805e-08
2851 2.98601145098942e-08
2852 2.98520503414501e-08
2853 2.9844781407995e-08
2854 2.98374647877964e-08
2855 2.98308983199824e-08
2856 2.98242274796578e-08
2857 2.98167388854687e-08
2858 2.98101716391663e-08
2859 2.9804449607429e-08
2860 2.97983350301401e-08
2861 2.97917550353688e-08
2862 2.9784429941504e-08
2863 2.97773711444016e-08
2864 2.97700739260165e-08
2865 2.97629050396431e-08
2866 2.97567151097411e-08
2867 2.97498581791622e-08
2868 2.97419335049121e-08
2869 2.97351089346698e-08
2870 2.9728201134116e-08
2871 2.97207949975231e-08
2872 2.97142329301892e-08
2873 2.97075723949547e-08
2874 2.97002369915589e-08
2875 2.96942682282975e-08
2876 2.96878995569294e-08
2877 2.96820978995527e-08
2878 2.96742300203157e-08
2879 2.96668128840771e-08
2880 2.96605453016241e-08
2881 2.9653600004842e-08
2882 2.96471459133585e-08
2883 2.9640884332327e-08
2884 2.96347978321343e-08
2885 2.96272693329769e-08
2886 2.96209001602321e-08
2887 2.96136511317435e-08
2888 2.96079805632843e-08
2889 2.96035045588283e-08
2890 2.95960052105748e-08
2891 2.9587874860848e-08
2892 2.95819559905652e-08
2893 2.95752625234513e-08
2894 2.95697982215515e-08
2895 2.9564276490035e-08
2896 2.95579032618676e-08
2897 2.95516005994756e-08
2898 2.95475034723935e-08
2899 2.95411857718086e-08
2900 2.95347757921505e-08
2901 2.95286956379925e-08
2902 2.95223924182686e-08
2903 2.95155818301751e-08
2904 2.95096576188314e-08
2905 2.95047154490469e-08
2906 2.94990023439468e-08
2907 2.94943681993232e-08
2908 2.94906700832875e-08
2909 2.94863494971764e-08
2910 2.9482089710875e-08
2911 2.94783131016274e-08
2912 2.94743884201587e-08
2913 2.9470267091547e-08
2914 2.94674865304323e-08
2915 2.94669061764452e-08
2916 2.94671260765433e-08
2917 2.94703784771144e-08
2918 2.94792424724299e-08
2919 2.94982158179913e-08
2920 2.95433990631366e-08
2921 2.96679719364867e-08
2922 2.99735680115987e-08
2923 2.99704980186721e-08
2924 2.99298639190315e-08
2925 2.9913061674991e-08
2926 2.98998778012205e-08
2927 2.98888628014193e-08
2928 2.98811910977026e-08
2929 2.98727152427958e-08
2930 2.98639984221971e-08
2931 2.9855128136802e-08
2932 2.98463924686132e-08
2933 2.98369512952146e-08
2934 2.98324739862466e-08
2935 2.98231561641771e-08
2936 2.98135776115593e-08
2937 2.9806692791734e-08
2938 2.97984553552233e-08
2939 2.9787661224967e-08
2940 2.97823761885585e-08
2941 2.97754857119248e-08
2942 2.97653787704188e-08
2943 2.97580193746594e-08
2944 2.97513365268287e-08
2945 2.97413673029379e-08
2946 2.97340043164951e-08
2947 2.97263560156136e-08
2948 2.97190793112634e-08
2949 2.97102922948156e-08
2950 2.97029908749025e-08
2951 2.96958305119333e-08
2952 2.96874747125475e-08
2953 2.96802132588869e-08
2954 2.96731956455076e-08
2955 2.96638096839441e-08
2956 2.96551658254263e-08
2957 2.96493087033056e-08
2958 2.9641030357741e-08
2959 2.96326622157839e-08
2960 2.962370547821e-08
2961 2.9616272136046e-08
2962 2.96098759151597e-08
2963 2.96018273109322e-08
2964 2.95942159411755e-08
2965 2.95868926962761e-08
2966 2.95787547117676e-08
2967 2.95689830931423e-08
2968 2.95642712173638e-08
2969 2.95557847618255e-08
2970 2.95477208775985e-08
2971 2.95407117647528e-08
2972 2.95328616088941e-08
2973 2.95254207824946e-08
2974 2.95169562061215e-08
2975 2.9509261042282e-08
2976 2.95025695891127e-08
2977 2.94945407850467e-08
2978 2.94857579434815e-08
2979 2.94791083974566e-08
2980 2.94707064487643e-08
2981 2.94640576532501e-08
2982 2.94552354487276e-08
2983 2.94476778981423e-08
2984 2.94400709095477e-08
2985 2.943274549283e-08
2986 2.94261883611036e-08
2987 2.94179006290474e-08
2988 2.94104908846737e-08
2989 2.94019847044691e-08
2990 2.9394516948944e-08
2991 2.93864757068274e-08
2992 2.93812565508311e-08
2993 2.93712871708429e-08
2994 2.93626006868664e-08
2995 2.9359744297075e-08
2996 2.9347050183981e-08
2997 2.9340126910915e-08
2998 2.93344557897868e-08
2999 2.93245456699509e-08
3000 1.34023260266319e-08
3001 1.34145304479388e-08
3002 1.3570035164584e-08
3003 1.3657781875509e-08
3004 1.36927612434945e-08
3005 1.36999826725048e-08
3006 1.36983491198783e-08
3007 1.36944627017743e-08
3008 1.36900324207989e-08
3009 1.36857170100479e-08
3010 1.36815652318767e-08
3011 1.36775603749306e-08
3012 1.36736736425502e-08
3013 1.36699623308201e-08
3014 1.36663369354473e-08
3015 1.36628538674188e-08
3016 1.36593677489416e-08
3017 1.36559765970823e-08
3018 1.36527166239853e-08
3019 1.36494098626205e-08
3020 1.36462001064375e-08
3021 1.36430613217609e-08
3022 1.36399449670865e-08
3023 1.36368945463927e-08
3024 1.36338320139984e-08
3025 1.36308256814821e-08
3026 1.36278548269197e-08
3027 1.36248698510422e-08
3028 1.36219725216935e-08
3029 1.36190984258155e-08
3030 1.36161848242888e-08
3031 1.3613360548892e-08
3032 1.36105423934607e-08
3033 1.36077247897826e-08
3034 1.36049083408196e-08
3035 1.36021643961093e-08
3036 1.35994466785028e-08
3037 1.35966972461266e-08
3038 1.35940025252634e-08
3039 1.35912901264301e-08
3040 1.35886783746153e-08
3041 1.35859685754525e-08
3042 1.35833315761114e-08
3043 1.35807323609904e-08
3044 1.3578099912287e-08
3045 1.35755197012699e-08
3046 1.35729603297774e-08
3047 1.35703795976772e-08
3048 1.35678430204511e-08
3049 1.35653386091616e-08
3050 1.356276539069e-08
3051 1.3560287920461e-08
3052 1.35578190477714e-08
3053 1.35553214846706e-08
3054 1.35528348078895e-08
3055 1.35504068895231e-08
3056 1.35479540343264e-08
3057 1.35454931426693e-08
3058 1.35431087976146e-08
3059 1.35406722125142e-08
3060 1.35382626705594e-08
3061 1.35358608915892e-08
3062 1.35334770813011e-08
3063 1.35311647583014e-08
3064 1.35287924253047e-08
3065 1.35264341488417e-08
3066 1.35241197550817e-08
3067 1.35217920286534e-08
3068 1.35195045407599e-08
3069 1.3517187802653e-08
3070 1.35148745075975e-08
3071 1.35126280545739e-08
3072 1.3510367273567e-08
3073 1.35081066021947e-08
3074 1.35058412111533e-08
3075 1.35035676414597e-08
3076 1.35013284736363e-08
3077 1.34991259134465e-08
3078 1.3496947018965e-08
3079 1.3494661605884e-08
3080 1.34925240146133e-08
3081 1.34903030267752e-08
3082 1.34881376484541e-08
3083 1.3485964810156e-08
3084 1.34837677775446e-08
3085 1.34816217084721e-08
3086 1.34794889308787e-08
3087 1.34772935685978e-08
3088 1.34751753619322e-08
3089 1.34730350245804e-08
3090 1.34709242057274e-08
3091 1.34687482780393e-08
3092 1.34666316770338e-08
3093 1.3464568348498e-08
3094 1.34625115721654e-08
3095 1.34603988615201e-08
3096 1.34582632331792e-08
3097 1.34562123769444e-08
3098 1.34541786389886e-08
3099 1.3452051681323e-08
3100 1.34500156225015e-08
3101 1.34479801383314e-08
3102 1.3445922084715e-08
3103 1.3443888706749e-08
3104 1.34418663104152e-08
3105 1.34398131883817e-08
3106 1.34378254680601e-08
3107 1.34357467332924e-08
3108 1.34337688904584e-08
3109 1.34317780881854e-08
3110 1.34297656947113e-08
3111 1.34277753540413e-08
3112 1.34258020478006e-08
3113 1.34237780707036e-08
3114 1.34218375680839e-08
3115 1.34198414636355e-08
3116 1.34178808747498e-08
3117 1.34159248071086e-08
3118 1.34139556907664e-08
3119 1.34120020324757e-08
3120 1.34100836061968e-08
3121 1.34081564764077e-08
3122 1.34061876618519e-08
3123 1.34042533556922e-08
3124 1.34023538081174e-08
3125 1.34004219041195e-08
3126 1.33984570700185e-08
3127 1.33965566180005e-08
3128 1.33946466140678e-08
3129 1.33927640511289e-08
3130 1.33908218945267e-08
3131 1.33889195260417e-08
3132 1.33870431361649e-08
3133 1.33851724049283e-08
3134 1.33833033301167e-08
3135 1.33813910641878e-08
3136 1.33795057694341e-08
3137 1.33776742889946e-08
3138 1.33757605892126e-08
3139 1.33739482233719e-08
3140 1.33720599421738e-08
3141 1.33701769704508e-08
3142 1.33683208285995e-08
3143 1.33665138114247e-08
3144 1.33646420910905e-08
3145 1.33628183459633e-08
3146 1.33610001571416e-08
3147 1.33591685292644e-08
3148 1.33573461055247e-08
3149 1.33554904554745e-08
3150 1.33536768740228e-08
3151 1.33517879891965e-08
3152 1.33500373910855e-08
3153 1.3348231940491e-08
3154 1.33464426141372e-08
3155 1.33446481680455e-08
3156 1.33428067620733e-08
3157 1.33409910620386e-08
3158 1.33391635765978e-08
3159 1.33374534938002e-08
3160 1.33356226098558e-08
3161 1.3333854647718e-08
3162 1.33320380578394e-08
3163 1.33303030238618e-08
3164 1.33285535723615e-08
3165 1.33267021965866e-08
3166 1.3324985944474e-08
3167 1.33232160598462e-08
3168 1.33213993090686e-08
3169 1.3319716782062e-08
3170 1.33179171174225e-08
3171 1.33161787406189e-08
3172 1.33144703624022e-08
3173 1.33126651679916e-08
3174 1.33109314129909e-08
3175 1.33092038233362e-08
3176 1.33074887076479e-08
3177 1.3305758831017e-08
3178 1.33040168254772e-08
3179 1.33022557150531e-08
3180 1.33005490005889e-08
3181 1.32988591604322e-08
3182 1.32970742637017e-08
3183 1.32953576135741e-08
3184 1.32936233565029e-08
3185 1.32919635073836e-08
3186 1.32901833764798e-08
3187 1.32885268803173e-08
3188 1.32868334013769e-08
3189 1.32851340861051e-08
3190 1.32834494437351e-08
3191 1.32817445896716e-08
3192 1.32800413985279e-08
3193 1.32783332003061e-08
3194 1.32766651631533e-08
3195 1.32749196790349e-08
3196 1.32733426504161e-08
3197 1.32716000288691e-08
3198 1.32699280944448e-08
3199 1.32683059690664e-08
3200 1.32666074228738e-08
3201 1.32649050877676e-08
3202 1.32632975859137e-08
3203 1.32616380224548e-08
3204 1.32599383534382e-08
3205 1.32582543988236e-08
3206 1.32566547363933e-08
3207 1.32549680273153e-08
3208 1.32533176036842e-08
3209 1.32516691588869e-08
3210 1.32499868507052e-08
3211 1.3248359889223e-08
3212 1.32467554647409e-08
3213 1.32450492641434e-08
3214 1.32434163473694e-08
3215 1.32418034020088e-08
3216 1.32401429536466e-08
3217 1.32384791455553e-08
3218 1.32368272241778e-08
3219 1.32352621112497e-08
3220 1.32336093412455e-08
3221 1.32319727930985e-08
3222 1.32303783418053e-08
3223 1.32287370050999e-08
3224 1.32271110603044e-08
3225 1.32255126900904e-08
3226 1.32238823204733e-08
3227 1.32222661641257e-08
3228 1.32206439669158e-08
3229 1.32190462659165e-08
3230 1.32174303130728e-08
3231 1.32158310042763e-08
3232 1.32141718876266e-08
3233 1.32126101046737e-08
3234 1.32110320372192e-08
3235 1.32094256660165e-08
3236 1.32078335870478e-08
3237 1.32062135883848e-08
3238 1.32046767379046e-08
3239 1.32029856866334e-08
3240 1.32014147297688e-08
3241 1.31997898341896e-08
3242 1.31982572260103e-08
3243 1.31966664516925e-08
3244 1.31950960231553e-08
3245 1.31935024014762e-08
3246 1.31919602437969e-08
3247 1.31903139972411e-08
3248 1.31887406132902e-08
3249 1.31872001621347e-08
3250 1.31856156801724e-08
3251 1.31840269541605e-08
3252 1.31825061066848e-08
3253 1.31808964540236e-08
3254 1.31793489733245e-08
3255 1.31778074609068e-08
3256 1.31761943885644e-08
3257 1.31746899576513e-08
3258 1.31730973809419e-08
3259 1.31715512825814e-08
3260 1.31699741217073e-08
3261 1.31684155269263e-08
3262 1.31668808915908e-08
3263 1.3165318739905e-08
3264 1.31637610530644e-08
3265 1.31622095388695e-08
3266 1.31606870367174e-08
3267 1.31591501654482e-08
3268 1.31575901163028e-08
3269 1.31560112944795e-08
3270 1.31544965316366e-08
3271 1.31529451096457e-08
3272 1.31514668532395e-08
3273 1.31499339132424e-08
3274 1.31483455164394e-08
3275 1.31468869081219e-08
3276 1.3145301854367e-08
3277 1.31437778055632e-08
3278 1.31422365249878e-08
3279 1.31407355797586e-08
3280 1.31392044312451e-08
3281 1.31376997097032e-08
3282 1.3136150561921e-08
3283 1.31346218126827e-08
3284 1.3133085100564e-08
3285 1.3131545252093e-08
3286 1.31300448341087e-08
3287 1.31285875874798e-08
3288 1.31270919568049e-08
3289 1.31255358107873e-08
3290 1.31239926440263e-08
3291 1.31225453912365e-08
3292 1.31210279814831e-08
3293 1.31195630261038e-08
3294 1.31180102186934e-08
3295 1.31165634799368e-08
3296 1.31150365431654e-08
3297 1.31134963972934e-08
3298 1.31120361520076e-08
3299 1.31105639502305e-08
3300 1.31090117772292e-08
3301 1.31075666235381e-08
3302 1.31060716933584e-08
3303 1.31045427598231e-08
3304 1.31030983136493e-08
3305 1.31016437570353e-08
3306 1.31001119874075e-08
3307 1.30985717635423e-08
3308 1.3097102072146e-08
3309 1.30956152453177e-08
3310 1.30941447235022e-08
3311 1.30926791206409e-08
3312 1.30911543626822e-08
3313 1.30896839323769e-08
3314 1.30882108251573e-08
3315 1.30867775066379e-08
3316 1.30852952150429e-08
3317 1.3083803141245e-08
3318 1.30823171203831e-08
3319 1.30808638925395e-08
3320 1.30794087543351e-08
3321 1.307791472685e-08
3322 1.30764571059094e-08
3323 1.30750235393939e-08
3324 1.30735606975463e-08
3325 1.30720566453579e-08
3326 1.30706115256951e-08
3327 1.30691132284311e-08
3328 1.30676100492944e-08
3329 1.30661985499603e-08
3330 1.30647950813689e-08
3331 1.30633017595139e-08
3332 1.30618370096647e-08
3333 1.30604093309949e-08
3334 1.30589471145637e-08
3335 1.3057513211262e-08
3336 1.30560842985794e-08
3337 1.30546125603759e-08
3338 1.30531666904521e-08
3339 1.30517176181733e-08
3340 1.30503006723626e-08
3341 1.30488389502581e-08
3342 1.30473829289268e-08
3343 1.3045982061588e-08
3344 1.30445540788282e-08
3345 1.30430891830957e-08
3346 1.30416432388425e-08
3347 1.30401994804796e-08
3348 1.30387540301924e-08
3349 1.30373157004593e-08
3350 1.30359017393167e-08
3351 1.30345362517326e-08
3352 1.30330531416256e-08
3353 1.30316335484326e-08
3354 1.30302379982572e-08
3355 1.30287987792632e-08
3356 1.3027287366546e-08
3357 1.30259048825854e-08
3358 1.30245288017805e-08
3359 1.30230641971762e-08
3360 1.30216315238629e-08
3361 1.30202226024667e-08
3362 1.30188138010023e-08
3363 1.30173629666885e-08
3364 1.30159806298602e-08
3365 1.30145375994151e-08
3366 1.30131588442495e-08
3367 1.30116687968029e-08
3368 1.30103115086044e-08
3369 1.30088993809119e-08
3370 1.30074853940121e-08
3371 1.30060561015499e-08
3372 1.30046972296183e-08
3373 1.30032578443406e-08
3374 1.3001896810777e-08
3375 1.30004542027162e-08
3376 1.29990805139146e-08
3377 1.29976968510359e-08
3378 1.29962802796202e-08
3379 1.2994876835204e-08
3380 1.29934929887221e-08
3381 1.2992040506421e-08
3382 1.29906629558751e-08
3383 1.29892858255765e-08
3384 1.29878882476342e-08
3385 1.29864910680677e-08
3386 1.29850830444256e-08
3387 1.2983689178292e-08
3388 1.29823367469861e-08
3389 1.29809646965129e-08
3390 1.29795235481733e-08
3391 1.29781497504033e-08
3392 1.29767835480021e-08
3393 1.29753607960026e-08
3394 1.29740153151647e-08
3395 1.29726187955148e-08
3396 1.29712413245719e-08
3397 1.29698584318549e-08
3398 1.29684650207462e-08
3399 1.29670673701676e-08
3400 1.29657071460676e-08
3401 1.29642844755862e-08
3402 1.29629668756126e-08
3403 1.29615766542301e-08
3404 1.29601991800954e-08
3405 1.29588435442973e-08
3406 1.29574077057937e-08
3407 1.2956083245752e-08
3408 1.29546561525029e-08
3409 1.29533214965616e-08
3410 1.29519431062153e-08
3411 1.29505320152212e-08
3412 1.29492243665652e-08
3413 1.29478368888214e-08
3414 1.29464653224887e-08
3415 1.29451505128386e-08
3416 1.29437347406447e-08
3417 1.29423925207262e-08
3418 1.29409870167629e-08
3419 1.29396688083039e-08
3420 1.29383106302339e-08
3421 1.2936965748972e-08
3422 1.29355749345084e-08
3423 1.29342849950198e-08
3424 1.29328824129415e-08
3425 1.29315340917091e-08
3426 1.29301584366193e-08
3427 1.29288412133999e-08
3428 1.29275329487366e-08
3429 1.2926120667528e-08
3430 1.29248130584236e-08
3431 1.29234354354912e-08
3432 1.29220880036585e-08
3433 1.29207350672844e-08
3434 1.2919429997732e-08
3435 1.29180887087355e-08
3436 1.29167355710225e-08
3437 1.29153844872221e-08
3438 1.29140354727664e-08
3439 1.29127036470333e-08
3440 1.29113685401749e-08
3441 1.29099941501287e-08
3442 1.29086856451577e-08
3443 1.29073628775678e-08
3444 1.29059998909553e-08
3445 1.29046561373192e-08
3446 1.29033850909399e-08
3447 1.29020439774974e-08
3448 1.29006894510342e-08
3449 1.28993482735595e-08
3450 1.28980413154067e-08
3451 1.28967009370706e-08
3452 1.28954277283377e-08
3453 1.28940386965037e-08
3454 1.28927509413235e-08
3455 1.28914477220132e-08
3456 1.28901061414166e-08
3457 1.28887603537409e-08
3458 1.28874608921192e-08
3459 1.28861654950518e-08
3460 1.28848131401571e-08
3461 1.28834649870679e-08
3462 1.28822358994862e-08
3463 1.28809000951302e-08
3464 1.2879552158479e-08
3465 1.2878230397001e-08
3466 1.28769182180799e-08
3467 1.28756373023453e-08
3468 1.28743280306265e-08
3469 1.2872996977803e-08
3470 1.28716208801505e-08
3471 1.28703553969878e-08
3472 1.28690289716848e-08
3473 1.28677189049076e-08
3474 1.28663771723769e-08
3475 1.28651034538851e-08
3476 1.28638045094609e-08
3477 1.28624895206486e-08
3478 1.28611789375066e-08
3479 1.28598776827638e-08
3480 1.28586053083635e-08
3481 1.28572802880478e-08
3482 1.28559675426632e-08
3483 1.28546880519553e-08
3484 1.2853400871482e-08
3485 1.28520881951255e-08
3486 1.28507592632165e-08
3487 1.28494788983469e-08
3488 1.28482008744657e-08
3489 1.28468846234686e-08
3490 1.28455669149985e-08
3491 1.28443098026121e-08
3492 1.28430363103282e-08
3493 1.28417453069407e-08
3494 1.28404484458777e-08
3495 1.28391651577908e-08
3496 1.28378849831301e-08
3497 1.28365759078652e-08
3498 1.28352897192929e-08
3499 1.28339825253287e-08
3500 1.28326761219266e-08
3501 1.28314085527381e-08
3502 1.28301032847333e-08
3503 1.28288118573794e-08
3504 1.28275513127107e-08
3505 1.28262605761653e-08
3506 1.28249535027991e-08
3507 1.2823712558313e-08
3508 1.28223965372154e-08
3509 1.28210833593989e-08
3510 1.28198304370497e-08
3511 1.28185769289746e-08
3512 1.28172964107554e-08
3513 1.28159780518333e-08
3514 1.28147609039786e-08
3515 1.2813485680746e-08
3516 1.28121968930028e-08
3517 1.28109221875228e-08
3518 1.28096704793412e-08
3519 1.28083573843751e-08
3520 1.28070799527702e-08
3521 1.28057985757379e-08
3522 1.28045666751286e-08
3523 1.28032836604341e-08
3524 1.28020077692359e-08
3525 1.28007554402454e-08
3526 1.27994794980602e-08
3527 1.27982032192553e-08
3528 1.27969187775911e-08
3529 1.27957460746286e-08
3530 1.27944491078169e-08
3531 1.2793197701394e-08
3532 1.27919121395148e-08
3533 1.27906732762528e-08
3534 1.27894224307423e-08
3535 1.27881602193514e-08
3536 1.27868716696955e-08
3537 1.27856297706952e-08
3538 1.27843803516825e-08
3539 1.27831296266034e-08
3540 1.27818781359146e-08
3541 1.278060895199e-08
3542 1.27793373361218e-08
3543 1.27780790281917e-08
3544 1.27768506061887e-08
3545 1.27756366713472e-08
3546 1.27743531287416e-08
3547 1.27730788879732e-08
3548 1.27718460288528e-08
3549 1.27705945150991e-08
3550 1.27693409194274e-08
3551 1.2768091718407e-08
3552 1.27668510467582e-08
3553 1.27656538843057e-08
3554 1.27643884536011e-08
3555 1.27630985239102e-08
3556 1.27618948972952e-08
3557 1.27606895795884e-08
3558 1.27594146610843e-08
3559 1.27582264722037e-08
3560 1.27569718204323e-08
3561 1.27556794549954e-08
3562 1.27545139466945e-08
3563 1.27532342916181e-08
3564 1.2752027737456e-08
3565 1.27508098085316e-08
3566 1.27495544666734e-08
3567 1.27483082421054e-08
3568 1.27470651496708e-08
3569 1.27458542424908e-08
3570 1.27446187639102e-08
3571 1.27433789213205e-08
3572 1.27421794155202e-08
3573 1.27409163222514e-08
3574 1.27397118303008e-08
3575 1.273846850669e-08
3576 1.27372217678945e-08
3577 1.2735985076312e-08
3578 1.27347682664369e-08
3579 1.27336155152646e-08
3580 1.27322975188027e-08
3581 1.27311153548249e-08
3582 1.27298641273976e-08
3583 1.27286114501302e-08
3584 1.2727448289368e-08
3585 1.27262753820134e-08
3586 1.27250279664815e-08
3587 1.27237729762308e-08
3588 1.27225585418445e-08
3589 1.27213678789428e-08
3590 1.27201167054447e-08
3591 1.27188822962032e-08
3592 1.27177080146423e-08
3593 1.27165037163424e-08
3594 1.27152794048013e-08
3595 1.27140759504096e-08
3596 1.27128418126732e-08
3597 1.27116326608945e-08
3598 1.27104419583857e-08
3599 1.27092153444641e-08
3600 1.27080019134973e-08
3601 1.27067961545324e-08
3602 1.27055732146164e-08
3603 1.27043464043242e-08
3604 1.27031522858534e-08
3605 1.27019294077213e-08
3606 1.2700732547416e-08
3607 1.2699515551301e-08
3608 1.26982894594063e-08
3609 1.26971183607438e-08
3610 1.26958992343051e-08
3611 1.2694690679188e-08
3612 1.26935038587961e-08
3613 1.26922849071898e-08
3614 1.26910895127674e-08
3615 1.26898922897523e-08
3616 1.26886483733935e-08
3617 1.26874565062884e-08
3618 1.26863181150183e-08
3619 1.26850775166998e-08
3620 1.26838649629202e-08
3621 1.26827160927545e-08
3622 1.26814880141435e-08
3623 1.26802965607353e-08
3624 1.26790960727208e-08
3625 1.26779391521892e-08
3626 1.26767112025306e-08
3627 1.26755161605208e-08
3628 1.26742947769709e-08
3629 1.26731224115162e-08
3630 1.26719286617782e-08
3631 1.26707305097562e-08
3632 1.2669541710586e-08
3633 1.26683769002545e-08
3634 1.26672099725889e-08
3635 1.26659769711934e-08
3636 1.26648072994451e-08
3637 1.2663629988835e-08
3638 1.26624387358498e-08
3639 1.2661240954781e-08
3640 1.26600704509483e-08
3641 1.26588602193667e-08
3642 1.26577144285711e-08
3643 1.26565115307897e-08
3644 1.26553143420527e-08
3645 1.26541629228982e-08
3646 1.26529393639219e-08
3647 1.26517636011847e-08
3648 1.2650594103214e-08
3649 1.26493944657691e-08
3650 1.26482717373866e-08
3651 1.26471096295044e-08
3652 1.26458885384984e-08
3653 1.26447084211612e-08
3654 1.26435534311409e-08
3655 1.26424000727321e-08
3656 1.26412345299587e-08
3657 1.2640042667017e-08
3658 1.26388482061535e-08
3659 1.26376940131623e-08
3660 1.26365078246815e-08
3661 1.26353440592641e-08
3662 1.26341708073519e-08
3663 1.26330485362425e-08
3664 1.26318100278078e-08
3665 1.26306540825516e-08
3666 1.26294665907523e-08
3667 1.26283548725625e-08
3668 1.26271537743694e-08
3669 1.26260114564347e-08
3670 1.26248478078128e-08
3671 1.26236814797509e-08
3672 1.26225238535893e-08
3673 1.26213701570344e-08
3674 1.26202071092096e-08
3675 1.26190242895063e-08
3676 1.26179398251824e-08
3677 1.26167392202059e-08
3678 1.26155765808322e-08
3679 1.2614438497538e-08
3680 1.26132753287544e-08
3681 1.26121237146171e-08
3682 1.2610945936159e-08
3683 1.26097953878079e-08
3684 1.26086375065448e-08
3685 1.26074652106456e-08
3686 1.26063327085812e-08
3687 1.26051657118875e-08
3688 1.2604004761968e-08
3689 1.26028732078121e-08
3690 1.26017357761632e-08
3691 1.26005633684367e-08
3692 1.25993946550607e-08
3693 1.25983281547004e-08
3694 1.25971685926152e-08
3695 1.25959948180654e-08
3696 1.25948494072436e-08
3697 1.25936763018808e-08
3698 1.25925403511029e-08
3699 1.25913956819657e-08
3700 1.25902638849484e-08
3701 1.25891280421953e-08
3702 1.2587929300506e-08
3703 1.25868227237591e-08
3704 1.25857145497898e-08
3705 1.25845211067233e-08
3706 1.25833771323358e-08
3707 1.25822363224171e-08
3708 1.25811249700181e-08
3709 1.25799563864826e-08
3710 1.25788570274343e-08
3711 1.25776785102338e-08
3712 1.25765671040723e-08
3713 1.25754420769175e-08
3714 1.25742913397731e-08
3715 1.25731605999102e-08
3716 1.25720031057264e-08
3717 1.25708858978169e-08
3718 1.2569763797321e-08
3719 1.25686148518545e-08
3720 1.25674569110279e-08
3721 1.25663341997428e-08
3722 1.25652284762579e-08
3723 1.25641207896765e-08
3724 1.25629618146494e-08
3725 1.25618081810719e-08
3726 1.25607061824129e-08
3727 1.25595590508565e-08
3728 1.25584066842377e-08
3729 1.25572925827322e-08
3730 1.25561671406038e-08
3731 1.25550161998444e-08
3732 1.25539255372287e-08
3733 1.25527792902702e-08
3734 1.25516795496383e-08
3735 1.25505140974314e-08
3736 1.25494319021813e-08
3737 1.25483063846132e-08
3738 1.25471844391045e-08
3739 1.25460546347156e-08
3740 1.25449110864595e-08
3741 1.2543804209203e-08
3742 1.25427040748305e-08
3743 1.25415271223783e-08
3744 1.2540418230067e-08
3745 1.2539303958975e-08
3746 1.25381790761769e-08
3747 1.2537076548913e-08
3748 1.25359982234841e-08
3749 1.25348362661204e-08
3750 1.25337282547433e-08
3751 1.25326438523143e-08
3752 1.25315018179029e-08
3753 1.2530393684429e-08
3754 1.25292708569591e-08
3755 1.25281190212767e-08
3756 1.25270309825898e-08
3757 1.25258932085481e-08
3758 1.25248239680187e-08
3759 1.25237199744277e-08
3760 1.25225586263544e-08
3761 1.25214744718938e-08
3762 1.25204230681519e-08
3763 1.25192631144355e-08
3764 1.25181216549808e-08
3765 1.2517039864518e-08
3766 1.25159714279288e-08
3767 1.25148091393301e-08
3768 1.25137204617931e-08
3769 1.25125795379377e-08
3770 1.25114961234685e-08
3771 1.25103808543414e-08
3772 1.25092821425254e-08
3773 1.25082007489119e-08
3774 1.25070858941478e-08
3775 1.25059524867521e-08
3776 1.25048775233838e-08
3777 1.2503785596335e-08
3778 1.25027097593045e-08
3779 1.2501610074267e-08
3780 1.25004527896933e-08
3781 1.24994301285986e-08
3782 1.24982943756347e-08
3783 1.24971950396513e-08
3784 1.24961231427467e-08
3785 1.24949898892279e-08
3786 1.24938764154148e-08
3787 1.24927853140944e-08
3788 1.24917158075e-08
3789 1.24906092642818e-08
3790 1.2489547537714e-08
3791 1.24884235505329e-08
3792 1.2487300869779e-08
3793 1.24862612219667e-08
3794 1.24851586524033e-08
3795 1.24840365218482e-08
3796 1.24829977291574e-08
3797 1.24819159390277e-08
3798 1.24807833142559e-08
3799 1.24796942463645e-08
3800 1.24786050517134e-08
3801 1.24775371223851e-08
3802 1.24764472557715e-08
3803 1.24753731059191e-08
3804 1.24742719306292e-08
3805 1.2473195128121e-08
3806 1.24721301773545e-08
3807 1.24709952778745e-08
3808 1.24699012354346e-08
3809 1.24688327785838e-08
3810 1.24677426381337e-08
3811 1.2466670897715e-08
3812 1.24655869616075e-08
3813 1.24644920911354e-08
3814 1.24634453534112e-08
3815 1.24623686105219e-08
3816 1.24612900573584e-08
3817 1.24602146460151e-08
3818 1.24591635748683e-08
3819 1.24580418945919e-08
3820 1.24569753618131e-08
3821 1.24559211001074e-08
3822 1.24547840594236e-08
3823 1.2453749459379e-08
3824 1.24526589384832e-08
3825 1.24515795861535e-08
3826 1.24505236728245e-08
3827 1.24494108473816e-08
3828 1.24483689557253e-08
3829 1.24472686384713e-08
3830 1.24462367191158e-08
3831 1.24451778313883e-08
3832 1.24440584234886e-08
3833 1.24430187251334e-08
3834 1.24419284838473e-08
3835 1.24408791853936e-08
3836 1.24398043243878e-08
3837 1.2438709539514e-08
3838 1.24376440054086e-08
3839 1.24366090582251e-08
3840 1.2435551893647e-08
3841 1.24344161415157e-08
3842 1.24334064001697e-08
3843 1.2432302127191e-08
3844 1.24312609623145e-08
3845 1.24301447478048e-08
3846 1.24291340800886e-08
3847 1.24280456050008e-08
3848 1.24270084054801e-08
3849 1.24259747547595e-08
3850 1.24249007144861e-08
3851 1.24237634415714e-08
3852 1.24227718411585e-08
3853 1.24217298469176e-08
3854 1.24206126586313e-08
3855 1.24195976456409e-08
3856 1.24184774800973e-08
3857 1.2417448516211e-08
3858 1.24164018717732e-08
3859 1.24153509557801e-08
3860 1.24142662475402e-08
3861 1.24131835238517e-08
3862 1.24121502768915e-08
3863 1.2411097777526e-08
3864 1.24100670296778e-08
3865 1.24090132943289e-08
3866 1.24079754139916e-08
3867 1.24069350774525e-08
3868 1.24058637372138e-08
3869 1.24048076664829e-08
3870 1.24038249333658e-08
3871 1.24027431039619e-08
3872 1.24017499066598e-08
3873 1.24006554408918e-08
3874 1.23995841921076e-08
3875 1.23985641519997e-08
3876 1.2397507868217e-08
3877 1.2396463719283e-08
3878 1.23954091796608e-08
3879 1.23943721190012e-08
3880 1.23933273117605e-08
3881 1.23922675439592e-08
3882 1.23912656119773e-08
3883 1.23901993610009e-08
3884 1.2389173870947e-08
3885 1.23881535442349e-08
3886 1.23870709137219e-08
3887 1.23860779948359e-08
3888 1.23850241060541e-08
3889 1.23839563056227e-08
3890 1.23828811831317e-08
3891 1.23818716312452e-08
3892 1.23808405074199e-08
3893 1.23797884953869e-08
3894 1.23787095472339e-08
3895 1.23777154480698e-08
3896 1.23766958351201e-08
3897 1.23756256399654e-08
3898 1.23745833589262e-08
3899 1.2373529974935e-08
3900 1.23725386947382e-08
3901 1.23714931907215e-08
3902 1.23704689814208e-08
3903 1.23693803832647e-08
3904 1.23683880787762e-08
3905 1.23673594185636e-08
3906 1.2366308973083e-08
3907 1.23652677704589e-08
3908 1.23642856720285e-08
3909 1.23631828284065e-08
3910 1.23622066220685e-08
3911 1.23611476637031e-08
3912 1.23601213115332e-08
3913 1.23590912185501e-08
3914 1.23580430822778e-08
3915 1.23570685217067e-08
3916 1.23560085474872e-08
3917 1.23550037500753e-08
3918 1.23539922064764e-08
3919 1.23529542633005e-08
3920 1.23519349497503e-08
3921 1.23508606603151e-08
3922 1.23498995227289e-08
3923 1.23488738278943e-08
3924 1.23478478898931e-08
3925 1.2346808118846e-08
3926 1.23457795008775e-08
3927 1.23447529704335e-08
3928 1.23437163349616e-08
3929 1.23426740088195e-08
3930 1.23416726525438e-08
3931 1.2340610436673e-08
3932 1.23396310050539e-08
3933 1.23385997866654e-08
3934 1.23376099888106e-08
3935 1.23365008981857e-08
3936 1.23355426420724e-08
3937 1.23345273777831e-08
3938 1.23334703830413e-08
3939 1.23324441100858e-08
3940 1.23314830265675e-08
3941 1.23304163279492e-08
3942 1.23294181095801e-08
3943 1.23283857583478e-08
3944 1.23273882727537e-08
3945 1.23263870284995e-08
3946 1.23253613061036e-08
3947 1.23243083633151e-08
3948 1.23232799997819e-08
3949 1.23222961903868e-08
3950 1.2321269648452e-08
3951 1.23202806356915e-08
3952 1.23193036490887e-08
3953 1.23182720942749e-08
3954 1.23172318098608e-08
3955 1.23162611510852e-08
3956 1.2315215696751e-08
3957 1.23142478989369e-08
3958 1.2313178243184e-08
3959 1.2312186705582e-08
3960 1.23111835635681e-08
3961 1.23101821355442e-08
3962 1.23091436669276e-08
3963 1.23081530728486e-08
3964 1.23071251929563e-08
3965 1.23061448717987e-08
3966 1.23051859430845e-08
3967 1.23041366386634e-08
3968 1.2303153224924e-08
3969 1.23021069597939e-08
3970 1.23011522483729e-08
3971 1.23001716328119e-08
3972 1.22991306703846e-08
3973 1.22980907001635e-08
3974 1.22971055321053e-08
3975 1.22961023825974e-08
3976 1.22951479428479e-08
3977 1.22941146647459e-08
3978 1.22931123357484e-08
3979 1.2292117489543e-08
3980 1.22910958788025e-08
3981 1.2290128752257e-08
3982 1.22891093579935e-08
3983 1.22881357096649e-08
3984 1.22871238557309e-08
3985 1.22861514296468e-08
3986 1.22850678750397e-08
3987 1.22841077855595e-08
3988 1.22831424350933e-08
3989 1.22821335191703e-08
3990 1.22811353350516e-08
3991 1.2280116060609e-08
3992 1.22791386347743e-08
3993 1.22780750437257e-08
3994 1.22771131616517e-08
3995 1.22761616570821e-08
3996 1.22751688036993e-08
3997 1.22741558361839e-08
3998 1.22731738662341e-08
3999 1.22721944862125e-08
4000 1.22712087133325e-08
4001 1.22702076419956e-08
4002 1.226922451672e-08
4003 1.22682407143193e-08
4004 1.22672499196785e-08
4005 1.22662605124835e-08
4006 1.22652185448602e-08
4007 1.226428755699e-08
4008 1.22633048299514e-08
4009 1.22622992870802e-08
4010 1.22612897895114e-08
4011 1.22603418973299e-08
4012 1.22593903496837e-08
4013 1.22583847003976e-08
4014 1.22574244495466e-08
4015 1.22564430154737e-08
4016 1.22554310850176e-08
4017 1.22544449503437e-08
4018 1.22534704592447e-08
4019 1.22525223436587e-08
4020 1.22515066481588e-08
4021 1.2250482843118e-08
4022 1.2249569490419e-08
4023 1.22485603406275e-08
4024 1.22475932283483e-08
4025 1.22465739995081e-08
4026 1.22455881064465e-08
4027 1.22446247877717e-08
4028 1.22436560512917e-08
4029 1.22426696193279e-08
4030 1.2241663333995e-08
4031 1.2240707825989e-08
4032 1.22396886567677e-08
4033 1.22387415108782e-08
4034 1.22377357434922e-08
4035 1.22367604507845e-08
4036 1.22358126637134e-08
4037 1.22348438302e-08
4038 1.22338204955885e-08
4039 1.22328763989821e-08
4040 1.22318983860892e-08
4041 1.22309677960397e-08
4042 1.22299354826938e-08
4043 1.22290205887832e-08
4044 1.22279690843718e-08
4045 1.22270320064288e-08
4046 1.22260513918948e-08
4047 1.22250810820901e-08
4048 1.22241028619463e-08
4049 1.22231518796256e-08
4050 1.2222178370519e-08
4051 1.2221218713887e-08
4052 1.22202620715661e-08
4053 1.22192537692911e-08
4054 1.22182722062103e-08
4055 1.22173340265652e-08
4056 1.22163583743118e-08
4057 1.2215371507085e-08
4058 1.22144109724054e-08
4059 1.2213476354439e-08
4060 1.22124766974985e-08
4061 1.22115216938667e-08
4062 1.22105300323083e-08
4063 1.22095704516989e-08
4064 1.22086102851415e-08
4065 1.22076227877244e-08
4066 1.22066883316008e-08
4067 1.22057195710845e-08
4068 1.22047485820176e-08
4069 1.22038003429192e-08
4070 1.22028318799705e-08
4071 1.22018407228142e-08
4072 1.22009037364923e-08
4073 1.21999687960062e-08
4074 1.21989325740646e-08
4075 1.21980177842373e-08
4076 1.21970339257427e-08
4077 1.21961166767992e-08
4078 1.21951079214699e-08
4079 1.21941692854122e-08
4080 1.21932061511454e-08
4081 1.21922545707476e-08
4082 1.2191307775411e-08
4083 1.21903268094914e-08
4084 1.2189412296304e-08
4085 1.21884839046904e-08
4086 1.21874892553553e-08
4087 1.21865247840469e-08
4088 1.21855682520822e-08
4089 1.21846106482804e-08
4090 1.21836899402705e-08
4091 1.2182703443081e-08
4092 1.21817850213535e-08
4093 1.21807895452075e-08
4094 1.21798517272731e-08
4095 1.21788939343448e-08
4096 1.21779695148039e-08
4097 1.21770060008686e-08
4098 1.21760559475548e-08
4099 1.2175070862791e-08
4100 1.21741511077966e-08
4101 1.21732106255212e-08
4102 1.21722175639716e-08
4103 1.21712897022119e-08
4104 1.21703395739303e-08
4105 1.21694401119254e-08
4106 1.21684985947279e-08
4107 1.21675180508873e-08
4108 1.21665432023799e-08
4109 1.21656259903513e-08
4110 1.2164664862313e-08
4111 1.21637143271347e-08
4112 1.21627468985253e-08
4113 1.21618059062412e-08
4114 1.21608678903262e-08
4115 1.21599277763951e-08
4116 1.21590215289569e-08
4117 1.21580245439623e-08
4118 1.21570916602476e-08
4119 1.21561093500699e-08
4120 1.21551667065789e-08
4121 1.21542222586701e-08
4122 1.2153295144729e-08
4123 1.21523523636535e-08
4124 1.21514112344234e-08
4125 1.2150455454385e-08
4126 1.21494970015956e-08
4127 1.21485597296966e-08
4128 1.21476256376152e-08
4129 1.21467471115022e-08
4130 1.21457029966521e-08
4131 1.21447756897541e-08
4132 1.21438636824234e-08
4133 1.21429497686176e-08
4134 1.21420009544793e-08
4135 1.21410182957471e-08
4136 1.21401067762761e-08
4137 1.21391897123235e-08
4138 1.21381958063937e-08
4139 1.21373178315343e-08
4140 1.21363306615496e-08
4141 1.21354019148867e-08
4142 1.21344490712216e-08
4143 1.21335202677708e-08
4144 1.21325746066936e-08
4145 1.21316536469684e-08
4146 1.21307247542557e-08
4147 1.21298000878289e-08
4148 1.2128905274883e-08
4149 1.21279341975533e-08
4150 1.212700009115e-08
4151 1.21260632302e-08
4152 1.21251204316664e-08
4153 1.2124208770864e-08
4154 1.21232785585124e-08
4155 1.21223066318621e-08
4156 1.21213974237366e-08
4157 1.21204639624839e-08
4158 1.21195331132806e-08
4159 1.2118586658616e-08
4160 1.21176607167373e-08
4161 1.21167492578289e-08
4162 1.21158153435497e-08
4163 1.21148567363005e-08
4164 1.21139419633765e-08
4165 1.21130086420673e-08
4166 1.2112100202244e-08
4167 1.21111303640897e-08
4168 1.21102207248924e-08
4169 1.2109290057738e-08
4170 1.21083977045189e-08
4171 1.21074725861481e-08
4172 1.21065109112695e-08
4173 1.21055804032655e-08
4174 1.21046757975696e-08
4175 1.21037082497499e-08
4176 1.21027861421374e-08
4177 1.21019197633354e-08
4178 1.21009868510602e-08
4179 1.21000514664016e-08
4180 1.20991303679541e-08
4181 1.20981797195885e-08
4182 1.20972574789158e-08
4183 1.20963446299527e-08
4184 1.20954155042596e-08
4185 1.20945239436288e-08
4186 1.20935787064347e-08
4187 1.2092637542982e-08
4188 1.20917468513781e-08
4189 1.20908230768768e-08
4190 1.20899377994421e-08
4191 1.20890271354535e-08
4192 1.20880554583258e-08
4193 1.20871777569143e-08
4194 1.20861903479819e-08
4195 1.20853247140007e-08
4196 1.20844337463122e-08
4197 1.20835109328199e-08
4198 1.20825833093974e-08
4199 1.20816437058635e-08
4200 1.20807397093192e-08
4201 1.20798336824268e-08
4202 1.20789506768137e-08
4203 1.20780030036516e-08
4204 1.20770941744452e-08
4205 1.2076171860137e-08
4206 1.20752491543641e-08
4207 1.20743583757466e-08
4208 1.2073421936154e-08
4209 1.20725680742007e-08
4210 1.20716385441921e-08
4211 1.20707458643454e-08
4212 1.20697463222852e-08
4213 1.20688808775138e-08
4214 1.20679792924572e-08
4215 1.20670350136631e-08
4216 1.20661573862479e-08
4217 1.2065250445642e-08
4218 1.20643139970011e-08
4219 1.20634585049628e-08
4220 1.20625037347555e-08
4221 1.20615907700794e-08
4222 1.20606971817372e-08
4223 1.20598094746072e-08
4224 1.20588950106859e-08
4225 1.20579271118693e-08
4226 1.20570500755091e-08
4227 1.20561487352566e-08
4228 1.20552545981589e-08
4229 1.20543477687696e-08
4230 1.20534370242342e-08
4231 1.20525272502559e-08
4232 1.20516161167539e-08
4233 1.2050752008913e-08
4234 1.20498089804799e-08
4235 1.20489490220976e-08
4236 1.20480616635776e-08
4237 1.2047126526582e-08
4238 1.20461944055905e-08
4239 1.20452980724162e-08
4240 1.20443862952069e-08
4241 1.20435112796469e-08
4242 1.20426111532013e-08
4243 1.20416886531804e-08
4244 1.20408589432963e-08
4245 1.20399027867812e-08
4246 1.2038995797492e-08
4247 1.20381537033476e-08
4248 1.20372205166253e-08
4249 1.2036306193175e-08
4250 1.2035404820171e-08
4251 1.20344928645488e-08
4252 1.203363164759e-08
4253 1.20327757945626e-08
4254 1.20318330779912e-08
4255 1.20309750204217e-08
4256 1.20300275418261e-08
4257 1.20291898005331e-08
4258 1.20282738962363e-08
4259 1.20273660171311e-08
4260 1.20265288409416e-08
4261 1.20255314211559e-08
4262 1.20247186991274e-08
4263 1.20237719118954e-08
4264 1.20229122037852e-08
4265 1.20220090733814e-08
4266 1.20211201960208e-08
4267 1.20202511907364e-08
4268 1.20193253163314e-08
4269 1.20184500942144e-08
4270 1.2017563348371e-08
4271 1.20166741773009e-08
4272 1.20157613988925e-08
4273 1.2014881422967e-08
4274 1.20139520801144e-08
4275 1.2013068619976e-08
4276 1.20121889395641e-08
4277 1.20113459725069e-08
4278 1.20104108622399e-08
4279 1.20095034907841e-08
4280 1.20086494743432e-08
4281 1.20077397876006e-08
4282 1.20068695335929e-08
4283 1.20059616847967e-08
4284 1.20051373042884e-08
4285 1.20042038078694e-08
4286 1.20033250348706e-08
4287 1.20024034798993e-08
4288 1.20015212287383e-08
4289 1.20006532454697e-08
4290 1.19997895493551e-08
4291 1.19989266374465e-08
4292 1.19980224876914e-08
4293 1.19971461261248e-08
4294 1.19962370065951e-08
4295 1.19953452958066e-08
4296 1.19944500890368e-08
4297 1.1993604321997e-08
4298 1.19926584796481e-08
4299 1.19917890931964e-08
4300 1.19908934857194e-08
4301 1.19900155955144e-08
4302 1.19891388175308e-08
4303 1.19882493410106e-08
4304 1.1987387938367e-08
4305 1.19864890320565e-08
4306 1.19855492583809e-08
4307 1.19847147975471e-08
4308 1.19838541123574e-08
4309 1.19829945051386e-08
4310 1.19821172992474e-08
4311 1.19812660689389e-08
4312 1.19803546970998e-08
4313 1.19794542390583e-08
4314 1.19785880946521e-08
4315 1.19777139019894e-08
4316 1.19768454608649e-08
4317 1.19759313707279e-08
4318 1.19750778622973e-08
4319 1.19742186034943e-08
4320 1.19733392339172e-08
4321 1.19724473188476e-08
4322 1.1971578800507e-08
4323 1.1970721272625e-08
4324 1.19697923191553e-08
4325 1.19689242287779e-08
4326 1.19681082188516e-08
4327 1.19671859285519e-08
4328 1.19663261723413e-08
4329 1.1965414563192e-08
4330 1.19645567617233e-08
4331 1.1963695993239e-08
4332 1.19627950078416e-08
4333 1.1961976482705e-08
4334 1.19610586103436e-08
4335 1.19601709541728e-08
4336 1.1959365276204e-08
4337 1.19584420484875e-08
4338 1.1957601535495e-08
4339 1.19566947112126e-08
4340 1.19557942958592e-08
4341 1.19549561410359e-08
4342 1.19540286169506e-08
4343 1.19532397260258e-08
4344 1.19523300896041e-08
4345 1.1951498044116e-08
4346 1.1950604159594e-08
4347 1.19497759471787e-08
4348 1.19489145112284e-08
4349 1.19480249209936e-08
4350 1.19471810656746e-08
4351 1.19463231841865e-08
4352 1.19454259442098e-08
4353 1.19445725034195e-08
4354 1.19436831811925e-08
4355 1.19428173431801e-08
4356 1.19419576712909e-08
4357 1.19411045976514e-08
4358 1.19402004387092e-08
4359 1.1939387318638e-08
4360 1.19384821447022e-08
4361 1.19376530803017e-08
4362 1.19367862495279e-08
4363 1.19358888329424e-08
4364 1.19350410583974e-08
4365 1.19341713173404e-08
4366 1.19333407227751e-08
4367 1.19324823954214e-08
4368 1.19316318479834e-08
4369 1.19307651398892e-08
4370 1.19299120024119e-08
4371 1.19290277531547e-08
4372 1.19281731881804e-08
4373 1.19272852239782e-08
4374 1.1926396263906e-08
4375 1.19255617514191e-08
4376 1.19246730030387e-08
4377 1.19238148231227e-08
4378 1.19229157719003e-08
4379 1.19220579433144e-08
4380 1.19212099552179e-08
4381 1.19203437357884e-08
4382 1.19194775648757e-08
4383 1.19186572298258e-08
4384 1.19177790340041e-08
4385 1.19169191536428e-08
4386 1.19161037648863e-08
4387 1.19152483933627e-08
4388 1.19143798626709e-08
4389 1.19134766336515e-08
4390 1.19126516736068e-08
4391 1.19117892371234e-08
4392 1.19109327078315e-08
4393 1.19100877069289e-08
4394 1.19092241561147e-08
4395 1.19083597217018e-08
4396 1.19075128631463e-08
4397 1.19066750301211e-08
4398 1.19058475140377e-08
4399 1.19050269820897e-08
4400 1.19041205918491e-08
4401 1.19032490630833e-08
4402 1.19023736836199e-08
4403 1.19015693560987e-08
4404 1.19006873999516e-08
4405 1.18998553854666e-08
4406 1.18989793353175e-08
4407 1.18981301680532e-08
4408 1.18972769569403e-08
4409 1.18964362657292e-08
4410 1.18956213503441e-08
4411 1.18947583840068e-08
4412 1.18939159076681e-08
4413 1.18930703138231e-08
4414 1.18922015808209e-08
4415 1.1891311911677e-08
4416 1.18905169148031e-08
4417 1.18896359518339e-08
4418 1.18888124265581e-08
4419 1.18879688871254e-08
4420 1.18870879016741e-08
4421 1.18862442264334e-08
4422 1.1885421081631e-08
4423 1.18846038187348e-08
4424 1.18837426963669e-08
4425 1.18829063112114e-08
4426 1.18820461354197e-08
4427 1.18811940891528e-08
4428 1.18803520265665e-08
4429 1.18794878342099e-08
4430 1.18786631817247e-08
4431 1.1877808232974e-08
4432 1.18770025133441e-08
4433 1.18761121664923e-08
4434 1.1875301201808e-08
4435 1.18744836610229e-08
4436 1.18736042523104e-08
4437 1.18727604581093e-08
4438 1.18719680549384e-08
4439 1.18711290518214e-08
4440 1.18702628209011e-08
4441 1.18693950161841e-08
4442 1.18685475869462e-08
4443 1.18677489346353e-08
4444 1.1866901613089e-08
4445 1.18660804031001e-08
4446 1.1865192833721e-08
4447 1.18643796612472e-08
4448 1.18635586855154e-08
4449 1.18626693573209e-08
4450 1.18618822188277e-08
4451 1.18610642299011e-08
4452 1.18601896915704e-08
4453 1.18593229897768e-08
4454 1.18585085660539e-08
4455 1.18576837253581e-08
4456 1.18568571032818e-08
4457 1.18559977811683e-08
4458 1.18551493433483e-08
4459 1.18543595208076e-08
4460 1.18535146216292e-08
4461 1.18526764092963e-08
4462 1.18518178348903e-08
4463 1.18510252414827e-08
4464 1.18501624210565e-08
4465 1.18492846286622e-08
4466 1.18484728077461e-08
4467 1.18476763177333e-08
4468 1.18467961706392e-08
4469 1.18459927064551e-08
4470 1.18451337147996e-08
4471 1.18442753908876e-08
4472 1.18434702324477e-08
4473 1.1842631373854e-08
4474 1.18418090971073e-08
4475 1.18409867240488e-08
4476 1.18401409079649e-08
4477 1.18393371720538e-08
4478 1.18385047426228e-08
4479 1.18376146965582e-08
4480 1.18368349167342e-08
4481 1.18359966300441e-08
4482 1.18351686316243e-08
4483 1.18342965453044e-08
4484 1.18335063611918e-08
4485 1.18326582267125e-08
4486 1.18318170170828e-08
4487 1.18309918583359e-08
4488 1.1830170273841e-08
4489 1.18293019553961e-08
4490 1.182849842013e-08
4491 1.18276471277323e-08
4492 1.18268578834513e-08
4493 1.18260024503103e-08
4494 1.18251851132511e-08
4495 1.18243205238511e-08
4496 1.18235154108193e-08
4497 1.18226719587333e-08
4498 1.18218776567758e-08
4499 1.18210518022244e-08
4500 1.18202457749239e-08
4501 1.18194065007182e-08
4502 1.18185732520537e-08
4503 1.18177668626263e-08
4504 1.18169043996641e-08
4505 1.1816105969259e-08
4506 1.18152602925359e-08
4507 1.18144375573781e-08
4508 1.18136183596407e-08
4509 1.1812796434002e-08
4510 1.18120089327156e-08
4511 1.18111232939588e-08
4512 1.18103499718025e-08
4513 1.18095222186865e-08
4514 1.18087139174272e-08
4515 1.18078762202101e-08
4516 1.18070446258889e-08
4517 1.18062292439325e-08
4518 1.18054421680835e-08
4519 1.18045929452804e-08
4520 1.18037223950374e-08
4521 1.1802949174744e-08
4522 1.180207662968e-08
4523 1.18012926929589e-08
4524 1.18004571350028e-08
4525 1.17996761281325e-08
4526 1.17988245164624e-08
4527 1.17980113968075e-08
4528 1.17972335546279e-08
4529 1.17963842664048e-08
4530 1.1795579100915e-08
4531 1.17947549594377e-08
4532 1.17939477114193e-08
4533 1.17931417886186e-08
4534 1.17923050130531e-08
4535 1.17914610446856e-08
4536 1.17906300662607e-08
4537 1.17898051184007e-08
4538 1.1789012648894e-08
4539 1.17881962365951e-08
4540 1.17873353302766e-08
4541 1.17865635497205e-08
4542 1.1785746684978e-08
4543 1.1784952260202e-08
4544 1.17841265094287e-08
4545 1.17833128481237e-08
4546 1.17825273159011e-08
4547 1.1781729805399e-08
4548 1.17808506109318e-08
4549 1.17800470829099e-08
4550 1.17792533364247e-08
4551 1.17784583819058e-08
4552 1.17775931361974e-08
4553 1.17767851870454e-08
4554 1.17760309223391e-08
4555 1.17751763860641e-08
4556 1.17744076422455e-08
4557 1.17735739132985e-08
4558 1.177275229991e-08
4559 1.17719493455681e-08
4560 1.17711268802223e-08
4561 1.17703158548088e-08
4562 1.17694955660275e-08
4563 1.17687200321126e-08
4564 1.17678933797549e-08
4565 1.17671123438245e-08
4566 1.17662590357337e-08
4567 1.17654620402086e-08
4568 1.17646563512486e-08
4569 1.17638375509432e-08
4570 1.17630514818445e-08
4571 1.17622302190923e-08
4572 1.17614225411677e-08
4573 1.17606718054997e-08
4574 1.17598305255096e-08
4575 1.17590603447293e-08
4576 1.17581800407052e-08
4577 1.17574114595897e-08
4578 1.17565838815559e-08
4579 1.17557653389055e-08
4580 1.17549872678535e-08
4581 1.17541932370124e-08
4582 1.17533961372651e-08
4583 1.17525945477204e-08
4584 1.17517591413929e-08
4585 1.17509869341781e-08
4586 1.17501863994007e-08
4587 1.17493945708813e-08
4588 1.17485559575081e-08
4589 1.17477611567551e-08
4590 1.17469877202458e-08
4591 1.17461593606705e-08
4592 1.17453613120433e-08
4593 1.1744544741038e-08
4594 1.17437614890747e-08
4595 1.174294131287e-08
4596 1.17421591685762e-08
4597 1.17413495721586e-08
4598 1.17405398242787e-08
4599 1.17397135061259e-08
4600 1.17389311259652e-08
4601 1.17381535249261e-08
4602 1.17373413009714e-08
4603 1.17365541734693e-08
4604 1.17357936378348e-08
4605 1.17349696029667e-08
4606 1.17341435344476e-08
4607 1.17333636533434e-08
4608 1.17325470471441e-08
4609 1.17317633583358e-08
4610 1.17309791684561e-08
4611 1.17302033879052e-08
4612 1.17293386004691e-08
4613 1.17285985153925e-08
4614 1.17277674369087e-08
4615 1.17269651578877e-08
4616 1.17261635596e-08
4617 1.17254031861691e-08
4618 1.17245896439355e-08
4619 1.17238066705272e-08
4620 1.17229590052004e-08
4621 1.17222076289059e-08
4622 1.17214091870932e-08
4623 1.17206209717946e-08
4624 1.17197656573365e-08
4625 1.17190019702784e-08
4626 1.17182508274638e-08
4627 1.17174115574215e-08
4628 1.17166495526566e-08
4629 1.1715823336339e-08
4630 1.17150449613912e-08
4631 1.17142606587683e-08
4632 1.17134574673938e-08
4633 1.17126617843744e-08
4634 1.17118477550304e-08
4635 1.17111041738616e-08
4636 1.17103026374687e-08
4637 1.17094758102509e-08
4638 1.17087137417871e-08
4639 1.17079228919015e-08
4640 1.17071139935088e-08
4641 1.17063148120655e-08
4642 1.17055514393394e-08
4643 1.1704763887177e-08
4644 1.17039807293329e-08
4645 1.17031832438108e-08
4646 1.17023909667335e-08
4647 1.17015941927257e-08
4648 1.17008307690125e-08
4649 1.17001029623365e-08
4650 1.16992177640052e-08
4651 1.16984343104531e-08
4652 1.16976591156837e-08
4653 1.16969188725113e-08
4654 1.16960856870463e-08
4655 1.16952944075877e-08
4656 1.16945024775938e-08
4657 1.16937400778394e-08
4658 1.16929337783678e-08
4659 1.16921868528186e-08
4660 1.16913784173611e-08
4661 1.16905831498149e-08
4662 1.16898004142718e-08
4663 1.1689041934132e-08
4664 1.16882590263379e-08
4665 1.16874481322637e-08
4666 1.16866797971182e-08
4667 1.16859023465421e-08
4668 1.16851167092646e-08
4669 1.16843387135135e-08
4670 1.16835642837709e-08
4671 1.1682782078748e-08
4672 1.16819824443892e-08
4673 1.16811648832038e-08
4674 1.16803929526565e-08
4675 1.16796130561481e-08
4676 1.16788960650982e-08
4677 1.16780503274239e-08
4678 1.16772817662925e-08
4679 1.16764744508835e-08
4680 1.16757043270022e-08
4681 1.16749248891268e-08
4682 1.16741301992296e-08
4683 1.16734240686589e-08
4684 1.16726226488673e-08
4685 1.16718095610202e-08
4686 1.16710435524581e-08
4687 1.16702572065808e-08
4688 1.16694635455272e-08
4689 1.16687085493505e-08
4690 1.16679359375427e-08
4691 1.16671476046437e-08
4692 1.16664027369762e-08
4693 1.16656104501844e-08
4694 1.16648677468412e-08
4695 1.16640155390924e-08
4696 1.16632692242491e-08
4697 1.16625180181795e-08
4698 1.16617131721009e-08
4699 1.16609231538556e-08
4700 1.16601679703288e-08
4701 1.16593672838394e-08
4702 1.16586220511028e-08
4703 1.16578110442855e-08
4704 1.16570494340384e-08
4705 1.16563196789765e-08
4706 1.16555159153653e-08
4707 1.1654767162611e-08
4708 1.16539696081441e-08
4709 1.16531615532711e-08
4710 1.16524098105475e-08
4711 1.1651679462793e-08
4712 1.16508761302259e-08
4713 1.16501409193615e-08
4714 1.16493258641159e-08
4715 1.16485457745119e-08
4716 1.16477891654032e-08
4717 1.16469801495767e-08
4718 1.16461817192826e-08
4719 1.16454704291447e-08
4720 1.16446978227769e-08
4721 1.16439017248671e-08
4722 1.16431392265526e-08
4723 1.16423949569344e-08
4724 1.16415956146476e-08
4725 1.16408718402183e-08
4726 1.16400671783257e-08
4727 1.16393192072517e-08
4728 1.16385305806987e-08
4729 1.16377863066119e-08
4730 1.16370059693172e-08
4731 1.16362365347733e-08
4732 1.16354435148458e-08
4733 1.16347343730172e-08
4734 1.16339262878351e-08
4735 1.16331436257333e-08
4736 1.16324196996476e-08
4737 1.16316208135792e-08
4738 1.16308713924707e-08
4739 1.16300834673011e-08
4740 1.16293086579733e-08
4741 1.162850057565e-08
4742 1.16278105336542e-08
4743 1.1627003885295e-08
4744 1.16262576690118e-08
4745 1.1625502805146e-08
4746 1.16247855959373e-08
4747 1.16239722717515e-08
4748 1.16231969320146e-08
4749 1.1622457167848e-08
4750 1.1621725598987e-08
4751 1.16209428393799e-08
4752 1.16201613383493e-08
4753 1.16194055181096e-08
4754 1.16186289492726e-08
4755 1.16179059398702e-08
4756 1.16171404327958e-08
4757 1.16163230559352e-08
4758 1.16155911167592e-08
4759 1.161481904291e-08
4760 1.16140434392731e-08
4761 1.16132607707597e-08
4762 1.1612521983978e-08
4763 1.16117972355778e-08
4764 1.16110108694389e-08
4765 1.16102659503958e-08
4766 1.16095197544575e-08
4767 1.16086951087724e-08
4768 1.16079755271836e-08
4769 1.16071842373722e-08
4770 1.16064460801701e-08
4771 1.16056111518492e-08
4772 1.16049151763775e-08
4773 1.16041235952435e-08
4774 1.16033670696791e-08
4775 1.1602586030196e-08
4776 1.1601852082932e-08
4777 1.16011009825834e-08
4778 1.16003530801767e-08
4779 1.15995386347767e-08
4780 1.15988185321048e-08
4781 1.15980652820036e-08
4782 1.15972762304295e-08
4783 1.15965087518488e-08
4784 1.15957514826015e-08
4785 1.15949855414021e-08
4786 1.15942316850082e-08
4787 1.15934453923106e-08
4788 1.15927124112736e-08
4789 1.15919695389821e-08
4790 1.15911916768185e-08
4791 1.15903932791928e-08
4792 1.15896700124962e-08
4793 1.15889396333779e-08
4794 1.15881749304658e-08
4795 1.15874386794335e-08
4796 1.15866690706123e-08
4797 1.15859022791487e-08
4798 1.1585128229602e-08
4799 1.1584377289181e-08
4800 1.15836637398503e-08
4801 1.15828477146029e-08
4802 1.15821478690048e-08
4803 1.1581414560119e-08
4804 1.15806595832046e-08
4805 1.15798864063488e-08
4806 1.15790977900376e-08
4807 1.1578373206006e-08
4808 1.15776159715086e-08
4809 1.15768507998326e-08
4810 1.15761415283577e-08
4811 1.15753325892476e-08
4812 1.15745857151017e-08
4813 1.15738180701264e-08
4814 1.15730805712033e-08
4815 1.15723131735579e-08
4816 1.15716199454163e-08
4817 1.15708227557132e-08
4818 1.15700717823741e-08
4819 1.1569321673871e-08
4820 1.15685294393153e-08
4821 1.15677967718608e-08
4822 1.15671156035169e-08
4823 1.15662822931517e-08
4824 1.15655583797503e-08
4825 1.15647886627379e-08
4826 1.15640533908945e-08
4827 1.15633136536231e-08
4828 1.15625587955859e-08
4829 1.15617525120182e-08
4830 1.15610416771272e-08
4831 1.15602734963027e-08
4832 1.15594945926667e-08
4833 1.15587572863118e-08
4834 1.15580115367664e-08
4835 1.15572897788907e-08
4836 1.15565108935456e-08
4837 1.1555803092117e-08
4838 1.15550431106781e-08
4839 1.15542670842406e-08
4840 1.15535441920223e-08
4841 1.1552813728416e-08
4842 1.15520252579326e-08
4843 1.15512859966971e-08
4844 1.1550521919923e-08
4845 1.15497427418676e-08
4846 1.15490605079316e-08
4847 1.1548347591761e-08
4848 1.15475611830174e-08
4849 1.15467998779983e-08
4850 1.15460451351745e-08
4851 1.1545306575933e-08
4852 1.15445857962748e-08
4853 1.15438112934518e-08
4854 1.15430974944875e-08
4855 1.15423617205734e-08
4856 1.15416003562963e-08
4857 1.15408429993691e-08
4858 1.15401236240875e-08
4859 1.15393498495431e-08
4860 1.15385846015115e-08
4861 1.1537845782672e-08
4862 1.15371215444193e-08
4863 1.15363145294645e-08
4864 1.1535574944932e-08
4865 1.15349050956171e-08
4866 1.15341166233851e-08
4867 1.15333780433546e-08
4868 1.15326376526614e-08
4869 1.15319013334891e-08
4870 1.15311588485267e-08
4871 1.15304166993513e-08
4872 1.15296721446834e-08
4873 1.15288769549637e-08
4874 1.15281807806233e-08
4875 1.15273890952672e-08
4876 1.15266740080278e-08
4877 1.15259295357661e-08
4878 1.15251902331748e-08
4879 1.15244450157592e-08
4880 1.15236829913712e-08
4881 1.15229616017287e-08
4882 1.15222603196963e-08
4883 1.15215148839554e-08
4884 1.15207661920969e-08
4885 1.15200591137843e-08
4886 1.15192695147592e-08
4887 1.15185233610648e-08
4888 1.15178463691301e-08
4889 1.15170731946057e-08
4890 1.15163717590572e-08
4891 1.15156123564608e-08
4892 1.15148351810812e-08
4893 1.15141210898784e-08
4894 1.15134269557116e-08
4895 1.15126126343512e-08
4896 1.15118802510306e-08
4897 1.15112174359222e-08
4898 1.15104286961265e-08
4899 1.15097447508095e-08
4900 1.15090235754678e-08
4901 1.15082371992814e-08
4902 1.15075353701588e-08
4903 1.15067962306592e-08
4904 1.15060437994863e-08
4905 1.1505272090484e-08
4906 1.15046040969624e-08
4907 1.15038629356912e-08
4908 1.1503113923006e-08
4909 1.15023534989489e-08
4910 1.15016063748363e-08
4911 1.15008989324539e-08
4912 1.15001529462921e-08
4913 1.14994436047622e-08
4914 1.1498755431355e-08
4915 1.14980121035668e-08
4916 1.14972960029991e-08
4917 1.14965317858096e-08
4918 1.14957786104541e-08
4919 1.14950566302285e-08
4920 1.14943434059989e-08
4921 1.14936129691212e-08
4922 1.14928745094667e-08
4923 1.14921380755528e-08
4924 1.14914068446159e-08
4925 1.14906747309407e-08
4926 1.14899678510527e-08
4927 1.14892272828349e-08
4928 1.14884777874524e-08
4929 1.14877397916768e-08
4930 1.14870086742325e-08
4931 1.14862674125416e-08
4932 1.14856059176272e-08
4933 1.14848340228013e-08
4934 1.14841214062245e-08
4935 1.14833870284436e-08
4936 1.14826861038475e-08
4937 1.14819502669006e-08
4938 1.14812580809809e-08
4939 1.14804693319426e-08
4940 1.14797227779573e-08
4941 1.14790196956149e-08
4942 1.14782607975317e-08
4943 1.1477538535587e-08
4944 1.14768122498332e-08
4945 1.14760820086879e-08
4946 1.14753345258345e-08
4947 1.14746181010816e-08
4948 1.14738880989673e-08
4949 1.14731927374767e-08
4950 1.14724528665899e-08
4951 1.14717121926511e-08
4952 1.14709582297312e-08
4953 1.14702365494046e-08
4954 1.14694883268374e-08
4955 1.14687694648397e-08
4956 1.14681008004658e-08
4957 1.14673441591329e-08
4958 1.14666208268499e-08
4959 1.14658689515656e-08
4960 1.14651888011019e-08
4961 1.14644236083594e-08
4962 1.14637183303568e-08
4963 1.14629835814006e-08
4964 1.14622347598126e-08
4965 1.14614756243081e-08
4966 1.1460793471807e-08
4967 1.14601091282251e-08
4968 1.14593824783182e-08
4969 1.14586330805799e-08
4970 1.14578755911199e-08
4971 1.14571843244926e-08
4972 1.14564896851466e-08
4973 1.14557300673057e-08
4974 1.1455019736456e-08
4975 1.14543014262392e-08
4976 1.14535633875257e-08
4977 1.14528696666116e-08
4978 1.14521585165006e-08
4979 1.14514167232904e-08
4980 1.1450687036757e-08
4981 1.14499837471083e-08
4982 1.14492559152302e-08
4983 1.14485367692652e-08
4984 1.14477947067426e-08
4985 1.14470755585849e-08
4986 1.14463510366702e-08
4987 1.14456850754674e-08
4988 1.14448980533255e-08
4989 1.14442114147462e-08
4990 1.14434825251308e-08
4991 1.1442753233698e-08
4992 1.14420421129247e-08
4993 1.14413097801191e-08
4994 1.14406073688511e-08
4995 1.14398518222814e-08
4996 1.14391331469399e-08
4997 1.14384380001942e-08
4998 1.14377027568002e-08
4999 1.14370189584767e-08
};
\addlegendentry{Train}
\addplot [semithick, black]
table {%
0 0.000684479717165232
1 0.000197577406652272
2 0.000185595767106861
3 0.000152469423483126
4 6.1389226175379e-05
5 3.15253528242465e-05
6 2.99313342111418e-05
7 2.80424810625846e-05
8 2.46518375206506e-05
9 1.85545304702828e-05
10 1.12300585897174e-05
11 6.89240096107824e-06
12 5.42876614417764e-06
13 4.89565127281821e-06
14 4.47505681222538e-06
15 4.00430280933506e-06
16 3.47203013006947e-06
17 2.91110995931376e-06
18 2.39940413848672e-06
19 1.99631676878198e-06
20 1.73528030700254e-06
21 1.59049614012474e-06
22 1.50865787418297e-06
23 1.45667263495852e-06
24 1.4205007801138e-06
25 1.3950877928437e-06
26 1.37423523938196e-06
27 1.35636935283401e-06
28 1.34057040668267e-06
29 1.32625132209796e-06
30 1.3130863862898e-06
31 1.30077614812762e-06
32 1.28911790397979e-06
33 1.2780197948814e-06
34 1.26735335470585e-06
35 1.25704366382706e-06
36 1.24704547488363e-06
37 1.2373088793538e-06
38 1.22772269151028e-06
39 1.21808886888175e-06
40 1.2081255817975e-06
41 1.19754076877143e-06
42 1.18613331778761e-06
43 1.17401077659451e-06
44 1.16143598916096e-06
45 1.14840258902404e-06
46 1.13456871986273e-06
47 1.11947747427621e-06
48 1.10282860532607e-06
49 1.08467588688654e-06
50 1.06490028883854e-06
51 1.0431631380925e-06
52 1.01913690286892e-06
53 9.92624336504377e-07
54 9.63563820732816e-07
55 9.32127477426548e-07
56 8.98619987310667e-07
57 8.63139121065615e-07
58 8.2585506788746e-07
59 7.88406964602473e-07
60 7.50776678160037e-07
61 7.14263421741634e-07
62 6.79058416608314e-07
63 6.46523403702304e-07
64 6.17151556525641e-07
65 5.90409683809412e-07
66 5.66642086141655e-07
67 5.45612579117005e-07
68 5.28556427070725e-07
69 5.15515125698585e-07
70 5.05598904965154e-07
71 4.97781968533673e-07
72 4.92335288981849e-07
73 4.88673038034904e-07
74 4.85899477098428e-07
75 4.84918416532309e-07
76 4.83473399981449e-07
77 4.81948802644183e-07
78 4.80313531170395e-07
79 4.7908929445839e-07
80 4.78104027479276e-07
81 4.7715673190396e-07
82 4.76216456490874e-07
83 4.75268734589918e-07
84 4.74298815333896e-07
85 4.73322785410346e-07
86 4.72333937295843e-07
87 4.71342730179458e-07
88 4.70344843961357e-07
89 4.69355313725828e-07
90 4.68376725848429e-07
91 4.67401889636676e-07
92 4.66447090730071e-07
93 4.65493855017485e-07
94 4.64553295387304e-07
95 4.63624587609957e-07
96 4.62709067505784e-07
97 4.61801846540766e-07
98 4.60900537291309e-07
99 4.60011449376907e-07
100 4.59122276197377e-07
101 4.58251179225044e-07
102 4.57381645446731e-07
103 4.5652444669031e-07
104 4.55678701882789e-07
105 4.54848532172036e-07
106 4.54019470907951e-07
107 4.53204876293967e-07
108 4.52401224038113e-07
109 4.51611754215264e-07
110 4.5083154986969e-07
111 4.50075702929098e-07
112 4.4932264131603e-07
113 4.48584899004345e-07
114 4.47862589680881e-07
115 4.47150142690589e-07
116 4.46443976898081e-07
117 4.45752505129349e-07
118 4.45069446186608e-07
119 4.44394515852764e-07
120 4.43723479293112e-07
121 4.43065658828345e-07
122 4.424072983511e-07
123 4.41759567593181e-07
124 4.41115389548941e-07
125 4.40470273588289e-07
126 4.39840619037568e-07
127 4.39209031810606e-07
128 4.38579974115783e-07
129 4.37960551380456e-07
130 4.37336581171621e-07
131 4.36725855479381e-07
132 4.36108422263715e-07
133 4.3549840711421e-07
134 4.34888221434448e-07
135 4.342813895164e-07
136 4.33673307043136e-07
137 4.33070766803212e-07
138 4.32472461397992e-07
139 4.31874639161833e-07
140 4.31280255952515e-07
141 4.30697269848679e-07
142 4.30126817718701e-07
143 4.29573418614382e-07
144 4.29018314207497e-07
145 4.28453404310858e-07
146 4.27878035225149e-07
147 4.27286522608483e-07
148 4.26691627808395e-07
149 4.26089854954625e-07
150 4.25487627353505e-07
151 4.24880454374943e-07
152 4.2428530377947e-07
153 4.23684838324334e-07
154 4.23049783648821e-07
155 4.22368316321808e-07
156 4.21657261995279e-07
157 4.20927790401038e-07
158 4.2017921941806e-07
159 4.19422207187381e-07
160 4.18657748468831e-07
161 4.17885104297966e-07
162 4.17101148286747e-07
163 4.1630164560047e-07
164 4.15495918559827e-07
165 4.14693289485513e-07
166 4.13889210904017e-07
167 4.13067795079769e-07
168 4.12230605206787e-07
169 4.1137423067994e-07
170 4.10494862990163e-07
171 4.09595287464981e-07
172 4.08672775620289e-07
173 4.0772965803626e-07
174 4.06759482984853e-07
175 4.05783993073783e-07
176 4.04782241503199e-07
177 4.03759230493961e-07
178 4.02720019110347e-07
179 4.01662731519536e-07
180 4.00586884552467e-07
181 3.99497906755641e-07
182 3.98391875933157e-07
183 3.97267143625868e-07
184 3.96122345591721e-07
185 3.94942617276683e-07
186 3.93717328961429e-07
187 3.92425391737561e-07
188 3.91072745742349e-07
189 3.89666723776827e-07
190 3.88230773751275e-07
191 3.86755971248931e-07
192 3.85237939326544e-07
193 3.83682845495059e-07
194 3.82103451102012e-07
195 3.80481424144818e-07
196 3.78809232870481e-07
197 3.77095318526699e-07
198 3.75349912928868e-07
199 3.73587880631021e-07
200 3.7181305856393e-07
201 3.70020131867932e-07
202 3.68204098322167e-07
203 3.66362485237914e-07
204 3.64489153525938e-07
205 3.62563724820575e-07
206 3.60587534942169e-07
207 3.58565472424743e-07
208 3.56480086338706e-07
209 3.54336066266114e-07
210 3.52118291857551e-07
211 3.49827274703784e-07
212 3.47457529414896e-07
213 3.44999534718227e-07
214 3.42464602454129e-07
215 3.39846792485332e-07
216 3.37151021767568e-07
217 3.34389426370763e-07
218 3.31574455003647e-07
219 3.28708551933232e-07
220 3.25804364820215e-07
221 3.22882044656581e-07
222 3.19941364068654e-07
223 3.1699113378636e-07
224 3.1404016453962e-07
225 3.11083596216122e-07
226 3.0813171747468e-07
227 3.05177678683322e-07
228 3.02244075101044e-07
229 2.99332526765284e-07
230 2.96450508585622e-07
231 2.93610668222755e-07
232 2.90827784965586e-07
233 2.88122407710034e-07
234 2.8551542641253e-07
235 2.83010706425557e-07
236 2.80624703918875e-07
237 2.78395589248248e-07
238 2.76344536587203e-07
239 2.7447651973489e-07
240 2.72751577767849e-07
241 2.71076260105474e-07
242 2.69414982767557e-07
243 2.67742933601767e-07
244 2.65966349388691e-07
245 2.64020911799889e-07
246 2.61956671465668e-07
247 2.59917669609422e-07
248 2.57995964147995e-07
249 2.56243652074772e-07
250 2.5467497266618e-07
251 2.53291062790595e-07
252 2.52101187925291e-07
253 2.51094718350942e-07
254 2.50249456712481e-07
255 2.4953297383945e-07
256 2.48916990130965e-07
257 2.48349806497572e-07
258 2.47719043500183e-07
259 2.4694628564248e-07
260 2.46073426524163e-07
261 2.45191927206179e-07
262 2.44388417058872e-07
263 2.43660451815231e-07
264 2.42996520682937e-07
265 2.42374653680599e-07
266 2.41765889086309e-07
267 2.4115809083014e-07
268 2.40543784002512e-07
269 2.39928937162404e-07
270 2.39309827065881e-07
271 2.38713028011261e-07
272 2.38134177266147e-07
273 2.37595358498766e-07
274 2.37081025034058e-07
275 2.36605302461612e-07
276 2.36160374811334e-07
277 2.35745872601001e-07
278 2.35347130228547e-07
279 2.3495753964653e-07
280 2.34575125546144e-07
281 2.34195979942342e-07
282 2.33822845530085e-07
283 2.33453349096635e-07
284 2.33089807011311e-07
285 2.32735018812491e-07
286 2.32384607556924e-07
287 2.32039951697516e-07
288 2.31704731845639e-07
289 2.3137185678479e-07
290 2.31044822385229e-07
291 2.30721880711826e-07
292 2.30398669032184e-07
293 2.30079422181007e-07
294 2.2976063007718e-07
295 2.2944381328216e-07
296 2.29124481165854e-07
297 2.28806769086987e-07
298 2.28493746590175e-07
299 2.28171657568055e-07
300 2.27847181122343e-07
301 2.27526186336036e-07
302 2.27197091362541e-07
303 2.26873723363497e-07
304 2.26543093617693e-07
305 2.26206765319148e-07
306 2.25869783321286e-07
307 2.25530484954106e-07
308 2.25191428171456e-07
309 2.24848051288973e-07
310 2.24502485934863e-07
311 2.24155016326222e-07
312 2.23804974552877e-07
313 2.23449546865595e-07
314 2.23093493900706e-07
315 2.22733362420513e-07
316 2.22366182356382e-07
317 2.21997098037718e-07
318 2.21624432583667e-07
319 2.21244974341062e-07
320 2.20870489897607e-07
321 2.2048060088764e-07
322 2.20095429881439e-07
323 2.19698577552663e-07
324 2.19304283177735e-07
325 2.18901959669893e-07
326 2.18497362425296e-07
327 2.18087947700951e-07
328 2.17673388647199e-07
329 2.17260080148662e-07
330 2.1683979412046e-07
331 2.16415600107212e-07
332 2.15992116636698e-07
333 2.15561371419426e-07
334 2.15131294112325e-07
335 2.14700079936847e-07
336 2.1426832574889e-07
337 2.13841033769313e-07
338 2.13413386518368e-07
339 2.12985639791441e-07
340 2.12552606626559e-07
341 2.12127062582113e-07
342 2.11697198437832e-07
343 2.11268385896801e-07
344 2.10836788028246e-07
345 2.10405062261998e-07
346 2.09971702247458e-07
347 2.09529588346413e-07
348 2.0909266140734e-07
349 2.08652153332878e-07
350 2.08204298246528e-07
351 2.07759654813344e-07
352 2.07310080213574e-07
353 2.06860079288163e-07
354 2.06406227221123e-07
355 2.05951479870237e-07
356 2.05493904559262e-07
357 2.05041203571454e-07
358 2.04587081498175e-07
359 2.04130174097372e-07
360 2.03671021381524e-07
361 2.03219357786111e-07
362 2.0276679890685e-07
363 2.02317892217252e-07
364 2.0186791971355e-07
365 2.01415375045144e-07
366 2.00970234232045e-07
367 2.00521142801335e-07
368 2.00072790335071e-07
369 1.9962701003351e-07
370 1.99180504978358e-07
371 1.98734568357395e-07
372 1.98286357999677e-07
373 1.97843021965127e-07
374 1.9739631795801e-07
375 1.96951063458073e-07
376 1.96503449956253e-07
377 1.96061563428884e-07
378 1.9560793873552e-07
379 1.9516011207088e-07
380 1.9470905954222e-07
381 1.94260394437151e-07
382 1.93810322457466e-07
383 1.93355901956238e-07
384 1.92907222640315e-07
385 1.92454123748576e-07
386 1.91999433241108e-07
387 1.9154751385031e-07
388 1.91088034284803e-07
389 1.90634438013149e-07
390 1.90175910574908e-07
391 1.89719457921456e-07
392 1.89256809335347e-07
393 1.88799234024373e-07
394 1.88337494932966e-07
395 1.87870597301298e-07
396 1.8740921348126e-07
397 1.86944134838996e-07
398 1.86477933539209e-07
399 1.86005621571894e-07
400 1.85537757602106e-07
401 1.85067079883083e-07
402 1.84592636287562e-07
403 1.84118277957168e-07
404 1.83647600238146e-07
405 1.83169248657578e-07
406 1.82701668904883e-07
407 1.82231858047999e-07
408 1.81766779405734e-07
409 1.81304613988686e-07
410 1.80855479925413e-07
411 1.80404370553333e-07
412 1.79954994905529e-07
413 1.79505477149178e-07
414 1.7904503124555e-07
415 1.7858026524209e-07
416 1.78110937554266e-07
417 1.77641041432253e-07
418 1.77174172222294e-07
419 1.76698620180105e-07
420 1.76230230408692e-07
421 1.75764697019076e-07
422 1.75300257865274e-07
423 1.74839982491903e-07
424 1.74383444573323e-07
425 1.73926864022178e-07
426 1.73478014175998e-07
427 1.73032532302386e-07
428 1.72592024227924e-07
429 1.72153974631328e-07
430 1.71715498709091e-07
431 1.7129165996721e-07
432 1.70860943171647e-07
433 1.70438980262588e-07
434 1.70021678513876e-07
435 1.69605797850636e-07
436 1.69194350974067e-07
437 1.68785547316475e-07
438 1.68386620202909e-07
439 1.67981255572158e-07
440 1.67588666499796e-07
441 1.67196361644528e-07
442 1.66804611012594e-07
443 1.66418118396905e-07
444 1.66031412618395e-07
445 1.65656516060153e-07
446 1.65281107911142e-07
447 1.64904648158881e-07
448 1.64532821145258e-07
449 1.64164418947621e-07
450 1.63797295726908e-07
451 1.63435231570475e-07
452 1.63074886927461e-07
453 1.62715949159065e-07
454 1.62365680012044e-07
455 1.62011204452028e-07
456 1.61661361630649e-07
457 1.61316052071925e-07
458 1.60973542051579e-07
459 1.60631032031233e-07
460 1.60292017881147e-07
461 1.59958119638759e-07
462 1.59625031415089e-07
463 1.59297002255698e-07
464 1.58967139896049e-07
465 1.58642407654952e-07
466 1.5832064548249e-07
467 1.58001228101057e-07
468 1.57688319291083e-07
469 1.57370720899053e-07
470 1.57059915295577e-07
471 1.56751980284753e-07
472 1.5644347683974e-07
473 1.56135229190113e-07
474 1.55835280679639e-07
475 1.55533754764292e-07
476 1.5523097829373e-07
477 1.54934880924884e-07
478 1.54641725202964e-07
479 1.5434828526395e-07
480 1.54057957502118e-07
481 1.53768354493877e-07
482 1.53478111997174e-07
483 1.53193084884151e-07
484 1.5291020361019e-07
485 1.52625659666228e-07
486 1.52349031168342e-07
487 1.52068508896264e-07
488 1.51792391989147e-07
489 1.51519103042119e-07
490 1.51247022017742e-07
491 1.50977626844906e-07
492 1.50707194279676e-07
493 1.50438893342653e-07
494 1.50173988799907e-07
495 1.49909496371947e-07
496 1.49649338254676e-07
497 1.49385343206632e-07
498 1.4912868095962e-07
499 1.48869787608419e-07
500 1.48615910688932e-07
501 1.48362175877992e-07
502 1.48112150100133e-07
503 1.4786328961236e-07
504 1.47607735812016e-07
505 1.47365639691088e-07
506 1.47116125503999e-07
507 1.46872920936403e-07
508 1.46628622132994e-07
509 1.46383882793089e-07
510 1.46142411949768e-07
511 1.45901111636704e-07
512 1.45666462003646e-07
513 1.4543974202752e-07
514 1.45208204571645e-07
515 1.44974052318503e-07
516 1.44739999541343e-07
517 1.44509172628204e-07
518 1.44278715197288e-07
519 1.4405105730475e-07
520 1.43819889331098e-07
521 1.43593993584545e-07
522 1.43371053695773e-07
523 1.43145484798879e-07
524 1.42926836588231e-07
525 1.42703697747493e-07
526 1.42484083198724e-07
527 1.42269229286285e-07
528 1.42053892204785e-07
529 1.41834263445162e-07
530 1.41622251703666e-07
531 1.414089751961e-07
532 1.41196281333578e-07
533 1.40978272611392e-07
534 1.40768733558616e-07
535 1.40563571449093e-07
536 1.40357485634013e-07
537 1.40156075190134e-07
538 1.39950316224713e-07
539 1.39749843697246e-07
540 1.39553364419953e-07
541 1.39352323458297e-07
542 1.39153442546558e-07
543 1.38958569095848e-07
544 1.38761791390607e-07
545 1.38566960572462e-07
546 1.38368648094911e-07
547 1.38176673658563e-07
548 1.37982524961444e-07
549 1.37791246856978e-07
550 1.37598505034475e-07
551 1.37408321165822e-07
552 1.37215764084431e-07
553 1.37026148649966e-07
554 1.36839446440717e-07
555 1.36647770432319e-07
556 1.36462986688457e-07
557 1.36275730255875e-07
558 1.36089042257481e-07
559 1.3590224057225e-07
560 1.35720384264459e-07
561 1.35534861556152e-07
562 1.35354397912124e-07
563 1.35172655291171e-07
564 1.34990003175517e-07
565 1.34808132656872e-07
566 1.34628521664126e-07
567 1.34449351207877e-07
568 1.34270791818381e-07
569 1.34094790382733e-07
570 1.33915563083065e-07
571 1.33737145802115e-07
572 1.33560121184928e-07
573 1.33383821321331e-07
574 1.33207706198846e-07
575 1.33036564875511e-07
576 1.32861046608923e-07
577 1.3268882526063e-07
578 1.32513818584812e-07
579 1.32343402015067e-07
580 1.3217234595686e-07
581 1.31999854602327e-07
582 1.31831825456175e-07
583 1.31664336322501e-07
584 1.31492427613011e-07
585 1.31324085828055e-07
586 1.3115541719344e-07
587 1.30991722357976e-07
588 1.3082286898225e-07
589 1.30656744090629e-07
590 1.30493120309438e-07
591 1.30329979697308e-07
592 1.30164991674064e-07
593 1.30004082166124e-07
594 1.2984189368126e-07
595 1.29678483062889e-07
596 1.29515569824434e-07
597 1.29357275113762e-07
598 1.29197118781121e-07
599 1.2903976198686e-07
600 1.28877161387209e-07
601 1.28718340874912e-07
602 1.28560841972103e-07
603 1.28405986288271e-07
604 1.28247108932555e-07
605 1.28094029605563e-07
606 1.27937497040875e-07
607 1.27782669778753e-07
608 1.27629903090565e-07
609 1.27477079558957e-07
610 1.27324611298718e-07
611 1.27168291896851e-07
612 1.27019831097641e-07
613 1.2687121397903e-07
614 1.26721076298963e-07
615 1.26573823422405e-07
616 1.26422918356184e-07
617 1.26275736533898e-07
618 1.26128881561272e-07
619 1.25980392340352e-07
620 1.25833466313452e-07
621 1.25689538776896e-07
622 1.2554365014239e-07
623 1.25397633610191e-07
624 1.25253535543379e-07
625 1.25112393334348e-07
626 1.24966263115311e-07
627 1.24821553981747e-07
628 1.24682372870666e-07
629 1.24540989077104e-07
630 1.24397416811917e-07
631 1.24256615663398e-07
632 1.2411484817676e-07
633 1.2397335069636e-07
634 1.23833217458014e-07
635 1.23693141063086e-07
636 1.23550663033711e-07
637 1.23413869346223e-07
638 1.23272144492148e-07
639 1.23134128671154e-07
640 1.22995729157083e-07
641 1.22854729056598e-07
642 1.22717736417144e-07
643 1.22579919548116e-07
644 1.22440610539343e-07
645 1.22304882665958e-07
646 1.22168216876162e-07
647 1.22030186844313e-07
648 1.21893805271611e-07
649 1.21757793181132e-07
650 1.21622420579115e-07
651 1.21487047977098e-07
652 1.21353551207903e-07
653 1.21216970683236e-07
654 1.21081797033185e-07
655 1.20946381798603e-07
656 1.20813297144196e-07
657 1.20680425652608e-07
658 1.20545635695635e-07
659 1.20412934734304e-07
660 1.20277718451689e-07
661 1.20147106486002e-07
662 1.20013183391166e-07
663 1.19881377713682e-07
664 1.19748079896453e-07
665 1.19616359484098e-07
666 1.19485861205249e-07
667 1.19352137062378e-07
668 1.19220942451648e-07
669 1.19091872363697e-07
670 1.18960400641299e-07
671 1.18831053441681e-07
672 1.18700057782917e-07
673 1.18572664575822e-07
674 1.1844279157458e-07
675 1.18312975416757e-07
676 1.18184765085516e-07
677 1.18055460518462e-07
678 1.17927669407436e-07
679 1.1780498709868e-07
680 1.17674638033805e-07
681 1.17549575406883e-07
682 1.17424328038851e-07
683 1.17298263546672e-07
684 1.17171836677699e-07
685 1.17046319303427e-07
686 1.16920951143129e-07
687 1.1679577482937e-07
688 1.16672843830656e-07
689 1.16547099082709e-07
690 1.16422654627968e-07
691 1.16300284958015e-07
692 1.16175485231906e-07
693 1.16050777876353e-07
694 1.1592521076409e-07
695 1.15805228517729e-07
696 1.15678957968157e-07
697 1.1555633250282e-07
698 1.15436350256459e-07
699 1.15309227055604e-07
700 1.15186352900309e-07
701 1.15059172856036e-07
702 1.14933861539157e-07
703 1.14807761519842e-07
704 1.14679487239755e-07
705 1.14550559260351e-07
706 1.14421254693298e-07
707 1.14300476639073e-07
708 1.14179329102626e-07
709 1.14059830025326e-07
710 1.13942725477045e-07
711 1.13821997160812e-07
712 1.13711422500273e-07
713 1.13594296635711e-07
714 1.13476666285806e-07
715 1.13360670184193e-07
716 1.1325187898592e-07
717 1.13134667856229e-07
718 1.13021194181329e-07
719 1.12908317362326e-07
720 1.12797557960675e-07
721 1.1268729593894e-07
722 1.12576131527931e-07
723 1.12460185164309e-07
724 1.12349383130095e-07
725 1.12243903060971e-07
726 1.12131118612524e-07
727 1.12024835630109e-07
728 1.11916349965213e-07
729 1.11804588698305e-07
730 1.11698973626062e-07
731 1.11594140150828e-07
732 1.11486841092301e-07
733 1.1138049416104e-07
734 1.11275099357044e-07
735 1.11167778982235e-07
736 1.11060494134563e-07
737 1.10959149424161e-07
738 1.10852951706875e-07
739 1.10745368431253e-07
740 1.10641131811917e-07
741 1.10535950170743e-07
742 1.10433973077306e-07
743 1.10330340419296e-07
744 1.10224483762522e-07
745 1.10122975627291e-07
746 1.10020231147701e-07
747 1.09916797441656e-07
748 1.09811836068729e-07
749 1.09707798401359e-07
750 1.09608045306686e-07
751 1.09505428724788e-07
752 1.09401703696221e-07
753 1.09300700046333e-07
754 1.0919930559794e-07
755 1.09094358435868e-07
756 1.08991834224526e-07
757 1.08895079620197e-07
758 1.08788356101286e-07
759 1.08684432120754e-07
760 1.08582845825822e-07
761 1.08478779736743e-07
762 1.08373129137362e-07
763 1.08268871201744e-07
764 1.0816245321621e-07
765 1.08054372560673e-07
766 1.07952033090442e-07
767 1.0784530957153e-07
768 1.07742366139973e-07
769 1.07644723357225e-07
770 1.07544984473407e-07
771 1.07447668540317e-07
772 1.07349997335859e-07
773 1.07257271508843e-07
774 1.07159316087291e-07
775 1.07067471333266e-07
776 1.06972478874923e-07
777 1.06887341644324e-07
778 1.06815477352029e-07
779 1.06726247395272e-07
780 1.06645025255148e-07
781 1.06560271717626e-07
782 1.06474665528822e-07
783 1.06394281829125e-07
784 1.06309123282244e-07
785 1.06227446394769e-07
786 1.06148235090586e-07
787 1.06068867467002e-07
788 1.05987815857134e-07
789 1.05910743286586e-07
790 1.05834487840184e-07
791 1.05759255575322e-07
792 1.05683120921185e-07
793 1.05612166123592e-07
794 1.05536550165652e-07
795 1.05461651855876e-07
796 1.05392203408883e-07
797 1.05320673071674e-07
798 1.05249974069466e-07
799 1.0518228066303e-07
800 1.0511266168578e-07
801 1.05045778298063e-07
802 1.04980735216031e-07
803 1.04912018628056e-07
804 1.04845923942776e-07
805 1.04777889475827e-07
806 1.04710515813622e-07
807 1.04647703835781e-07
808 1.04584124471785e-07
809 1.04521127752832e-07
810 1.04463055095039e-07
811 1.04400363909463e-07
812 1.04338283790639e-07
813 1.04277901868954e-07
814 1.042162551812e-07
815 1.04156647751097e-07
816 1.04093992092658e-07
817 1.04031713021868e-07
818 1.03963053277312e-07
819 1.03886165447875e-07
820 1.03799933981463e-07
821 1.03702397780125e-07
822 1.03575771959186e-07
823 1.03401930573455e-07
824 1.03203824153297e-07
825 1.03028362730129e-07
826 1.02901474008377e-07
827 1.02806055224391e-07
828 1.02730496109871e-07
829 1.02666426471387e-07
830 1.02606740881583e-07
831 1.02549954306141e-07
832 1.02500358423185e-07
833 1.02450236738605e-07
834 1.02400655066504e-07
835 1.02348607811109e-07
836 1.02296802140245e-07
837 1.02246119126903e-07
838 1.02200651497242e-07
839 1.02146046287999e-07
840 1.02100528920346e-07
841 1.02048502981233e-07
842 1.02004619861873e-07
843 1.01959450660161e-07
844 1.01908483429725e-07
845 1.01860841539292e-07
846 1.01814507047493e-07
847 1.0176925968608e-07
848 1.01723379941632e-07
849 1.01681450814795e-07
850 1.01630320159529e-07
851 1.01587502854272e-07
852 1.01545509778589e-07
853 1.0150030504974e-07
854 1.01456024026447e-07
855 1.01414919129184e-07
856 1.013675756667e-07
857 1.01326662615975e-07
858 1.01281912634477e-07
859 1.01241894867599e-07
860 1.01197997537383e-07
861 1.01154505216527e-07
862 1.01113137418452e-07
863 1.01069588254177e-07
864 1.01027431753664e-07
865 1.00987946893838e-07
866 1.00942521896741e-07
867 1.00904230748711e-07
868 1.00862628471532e-07
869 1.0082055013072e-07
870 1.00782898471152e-07
871 1.00741168296281e-07
872 1.00698272831323e-07
873 1.00660159318977e-07
874 1.00614300890811e-07
875 1.00576613704106e-07
876 1.00541583947233e-07
877 1.00501452493518e-07
878 1.00457633323003e-07
879 1.00422575144421e-07
880 1.00382870016347e-07
881 1.00342326447844e-07
882 1.00305378225585e-07
883 1.00266277058836e-07
884 1.00230494126663e-07
885 1.00190966634273e-07
886 1.00153989990304e-07
887 1.00118761281465e-07
888 1.00083489940062e-07
889 1.00046641193785e-07
890 1.00006801062591e-07
891 9.99737252982413e-08
892 9.99410048052596e-08
893 9.99045894900519e-08
894 9.98721603195918e-08
895 9.98432554411011e-08
896 9.98059448420463e-08
897 9.97768552224443e-08
898 9.97472824337819e-08
899 9.97181928141799e-08
900 9.96838096511965e-08
901 9.96565745481348e-08
902 9.96279041487469e-08
903 9.96002782471805e-08
904 9.95702365003126e-08
905 9.9548486787171e-08
906 9.95198590203472e-08
907 9.94983793134452e-08
908 9.94769706608167e-08
909 9.94519879782274e-08
910 9.94301956325216e-08
911 9.94120057384862e-08
912 9.93895667988909e-08
913 9.93711424257526e-08
914 9.93525191006484e-08
915 9.93361695122985e-08
916 9.9315414558987e-08
917 9.93003794746983e-08
918 9.92793829368566e-08
919 9.92616548955993e-08
920 9.9242335238614e-08
921 9.92213813333365e-08
922 9.9202217995753e-08
923 9.91836301977855e-08
924 9.91567645769464e-08
925 9.91369191183367e-08
926 9.91122064419869e-08
927 9.90909043707688e-08
928 9.90644650755712e-08
929 9.9035233347422e-08
930 9.90141089118879e-08
931 9.89849766597217e-08
932 9.89575923426855e-08
933 9.8927479541544e-08
934 9.89006920804059e-08
935 9.88709132343502e-08
936 9.88434933901772e-08
937 9.88059341011649e-08
938 9.87788055795136e-08
939 9.87475559099948e-08
940 9.87168249366732e-08
941 9.86823422977068e-08
942 9.86520234391719e-08
943 9.8617213950547e-08
944 9.85833850108975e-08
945 9.85504087225308e-08
946 9.85157058153163e-08
947 9.84864172437483e-08
948 9.84470531761872e-08
949 9.84168124773532e-08
950 9.83797008302645e-08
951 9.83441665880491e-08
952 9.83130519216502e-08
953 9.82742491828503e-08
954 9.8237066481488e-08
955 9.82082468681256e-08
956 9.81698491386851e-08
957 9.81330856575369e-08
958 9.81027383772926e-08
959 9.80621379653712e-08
960 9.80330270294871e-08
961 9.79926824129507e-08
962 9.79538938850055e-08
963 9.7919965469373e-08
964 9.78830172471135e-08
965 9.78497922687893e-08
966 9.78123395611874e-08
967 9.77778356059389e-08
968 9.77405107960294e-08
969 9.77065397478327e-08
970 9.7663864551123e-08
971 9.76320322365609e-08
972 9.75926184310083e-08
973 9.75590666030257e-08
974 9.75191269958486e-08
975 9.74828466837607e-08
976 9.74474758663746e-08
977 9.74109894968933e-08
978 9.73715899021954e-08
979 9.73350395838679e-08
980 9.72977360902405e-08
981 9.7259537312766e-08
982 9.72195550730248e-08
983 9.71825642182012e-08
984 9.71434843677343e-08
985 9.71019034068377e-08
986 9.7064813076031e-08
987 9.7023047374023e-08
988 9.69825606489394e-08
989 9.69393738614599e-08
990 9.6904464896852e-08
991 9.68578888205229e-08
992 9.68191926631334e-08
993 9.67728155387704e-08
994 9.67333804169357e-08
995 9.66882467423602e-08
996 9.66465236729164e-08
997 9.6596046716968e-08
998 9.65553326182089e-08
999 9.6500691881829e-08
1000 9.64573700912297e-08
1001 9.64062394359644e-08
1002 9.63565653933074e-08
1003 9.63011288490634e-08
1004 9.6244711755844e-08
1005 9.61816084554812e-08
1006 9.6113133452036e-08
1007 9.60389456849953e-08
1008 9.59506181175129e-08
1009 9.5848747605487e-08
1010 9.57224344233509e-08
1011 9.55626120457964e-08
1012 9.53658840785465e-08
1013 9.51151193362421e-08
1014 9.48406082557085e-08
1015 9.45697067322726e-08
1016 9.43842977108034e-08
1017 9.42745188581284e-08
1018 9.42128721703739e-08
1019 9.41704243473396e-08
1020 9.41312592317445e-08
1021 9.40975013463685e-08
1022 9.40616473599221e-08
1023 9.40253670478342e-08
1024 9.39924262866043e-08
1025 9.39619084761034e-08
1026 9.39227930984998e-08
1027 9.38937319006072e-08
1028 9.38517601412059e-08
1029 9.38205815259607e-08
1030 9.37814661483571e-08
1031 9.37473956241774e-08
1032 9.37079036589239e-08
1033 9.36753181690619e-08
1034 9.36335311507719e-08
1035 9.35970874138548e-08
1036 9.35634503207439e-08
1037 9.35266371016041e-08
1038 9.34865127533158e-08
1039 9.34488610937478e-08
1040 9.34087296400321e-08
1041 9.3370914555635e-08
1042 9.3330122297175e-08
1043 9.32949575371822e-08
1044 9.32544210741071e-08
1045 9.32152417476573e-08
1046 9.31761192646263e-08
1047 9.31365349288171e-08
1048 9.30981158830946e-08
1049 9.30552559452735e-08
1050 9.30164958390378e-08
1051 9.29801018401122e-08
1052 9.29341439359632e-08
1053 9.28989010162695e-08
1054 9.28548047340882e-08
1055 9.28178280901193e-08
1056 9.2776055282684e-08
1057 9.27394836480744e-08
1058 9.26906125187088e-08
1059 9.2654950378801e-08
1060 9.26115433230734e-08
1061 9.25714545019218e-08
1062 9.25321543832069e-08
1063 9.2490189729233e-08
1064 9.24481682318401e-08
1065 9.24083423115007e-08
1066 9.23614251746585e-08
1067 9.23264735774865e-08
1068 9.22796772329093e-08
1069 9.22426650618036e-08
1070 9.21958189792349e-08
1071 9.21587570701377e-08
1072 9.21086851235486e-08
1073 9.20740106380435e-08
1074 9.20253313552166e-08
1075 9.1987587325093e-08
1076 9.19402012300452e-08
1077 9.19029830015461e-08
1078 9.1855234529703e-08
1079 9.18193663324018e-08
1080 9.17672409173065e-08
1081 9.17301221647904e-08
1082 9.16823594820926e-08
1083 9.16405937800846e-08
1084 9.15946642976451e-08
1085 9.15526001676881e-08
1086 9.15080917707201e-08
1087 9.14608691005014e-08
1088 9.1422869274993e-08
1089 9.13675890501509e-08
1090 9.13420663550824e-08
1091 9.12709978706516e-08
1092 9.12653703721844e-08
1093 9.11720192675602e-08
1094 9.1197129847842e-08
1095 9.10654520680509e-08
1096 9.1138851132655e-08
1097 9.09476369770346e-08
1098 9.10907829165808e-08
1099 9.08190642689988e-08
1100 9.10543533905184e-08
1101 9.06651820287152e-08
1102 9.10464095227326e-08
1103 9.04868713291762e-08
1104 9.10356021677217e-08
1105 9.02925947343647e-08
1106 9.10159911882147e-08
1107 9.01061696367833e-08
1108 9.0963879983974e-08
1109 8.99552503597079e-08
1110 9.08916959474482e-08
1111 8.98334491239439e-08
1112 9.08107011809989e-08
1113 8.97280614253759e-08
1114 9.07275961026244e-08
1115 8.96317402521163e-08
1116 9.06396024902278e-08
1117 8.9540392878007e-08
1118 9.05591548416851e-08
1119 8.9450523432788e-08
1120 9.04734989148892e-08
1121 8.93597871254315e-08
1122 9.03855763567663e-08
1123 8.92689158149551e-08
1124 9.02995438423204e-08
1125 8.91803679792247e-08
1126 9.02150816273206e-08
1127 8.9088139532123e-08
1128 9.01276706599674e-08
1129 8.89960745098506e-08
1130 9.00430521255657e-08
1131 8.89089122324549e-08
1132 8.99522589747903e-08
1133 8.88213733674093e-08
1134 8.98666243642765e-08
1135 8.87316744524469e-08
1136 8.97770462415792e-08
1137 8.86424942336816e-08
1138 8.96896423796534e-08
1139 8.85545006212851e-08
1140 8.96071838951684e-08
1141 8.84686741642327e-08
1142 8.9519708978969e-08
1143 8.83801902773484e-08
1144 8.94320351108036e-08
1145 8.82940156543555e-08
1146 8.93458107498191e-08
1147 8.82034498772555e-08
1148 8.92585063638762e-08
1149 8.81104682548539e-08
1150 8.91712801376343e-08
1151 8.8028130562634e-08
1152 8.9084998933231e-08
1153 8.79326691460847e-08
1154 8.90011335741292e-08
1155 8.78431478668062e-08
1156 8.89177584895151e-08
1157 8.77532784215873e-08
1158 8.88377300611864e-08
1159 8.76639276725655e-08
1160 8.87591298237567e-08
1161 8.75732766303372e-08
1162 8.86917632669793e-08
1163 8.7478852606182e-08
1164 8.86285249634966e-08
1165 8.73804353318519e-08
1166 8.85868303157622e-08
1167 8.72944454499702e-08
1168 8.85802435846017e-08
1169 8.72191492362617e-08
1170 8.85681572526664e-08
1171 8.71405632096867e-08
1172 8.85227677827061e-08
1173 8.70695302523927e-08
1174 8.84544846257995e-08
1175 8.69946177317615e-08
1176 8.83792097283731e-08
1177 8.69172609441193e-08
1178 8.83032598153477e-08
1179 8.6840060475879e-08
1180 8.82299246995899e-08
1181 8.67618581423812e-08
1182 8.81501804883555e-08
1183 8.66871516791434e-08
1184 8.80771722222562e-08
1185 8.66056879544885e-08
1186 8.79982593460227e-08
1187 8.65267324456909e-08
1188 8.79166677236753e-08
1189 8.64493543417666e-08
1190 8.78401351656066e-08
1191 8.63693898622842e-08
1192 8.77613288707835e-08
1193 8.62915783272911e-08
1194 8.76833183838244e-08
1195 8.62144347024696e-08
1196 8.76090879842195e-08
1197 8.61342783764485e-08
1198 8.75316530368764e-08
1199 8.60555005033348e-08
1200 8.74553549579105e-08
1201 8.59782005591114e-08
1202 8.73803287504415e-08
1203 8.58992450503138e-08
1204 8.73030998604918e-08
1205 8.58219024735263e-08
1206 8.72241656679762e-08
1207 8.57422719491296e-08
1208 8.71483933906347e-08
1209 8.56697397466633e-08
1210 8.70713208200868e-08
1211 8.55846735703381e-08
1212 8.69923866275712e-08
1213 8.55113455600076e-08
1214 8.69167067207854e-08
1215 8.54332071753561e-08
1216 8.68350369387372e-08
1217 8.53516581855729e-08
1218 8.67556479988707e-08
1219 8.52757935376758e-08
1220 8.66793001819133e-08
1221 8.5197569887896e-08
1222 8.6600550730509e-08
1223 8.51176764626871e-08
1224 8.65238263259016e-08
1225 8.50395096563261e-08
1226 8.64403375544498e-08
1227 8.4965591895525e-08
1228 8.63640039483471e-08
1229 8.48866008595905e-08
1230 8.62810480839471e-08
1231 8.48098622441285e-08
1232 8.62057163431018e-08
1233 8.47335499543078e-08
1234 8.61248850014817e-08
1235 8.46604208959434e-08
1236 8.60407496361404e-08
1237 8.4585643378432e-08
1238 8.59605577829825e-08
1239 8.45052383624534e-08
1240 8.58762660982393e-08
1241 8.44275120925886e-08
1242 8.57953210697815e-08
1243 8.43503400460577e-08
1244 8.5711640451791e-08
1245 8.42747880369643e-08
1246 8.56288266959382e-08
1247 8.41968272879967e-08
1248 8.55445065894855e-08
1249 8.41225329395456e-08
1250 8.54579482734152e-08
1251 8.40449203565186e-08
1252 8.53748858276049e-08
1253 8.39697236187931e-08
1254 8.52882919843978e-08
1255 8.38938944980328e-08
1256 8.5202358945935e-08
1257 8.38178877415885e-08
1258 8.51133066248622e-08
1259 8.37427052147177e-08
1260 8.50288657261444e-08
1261 8.3669192463276e-08
1262 8.49373620326332e-08
1263 8.35933278153789e-08
1264 8.48497734295961e-08
1265 8.35195663739796e-08
1266 8.47595131858725e-08
1267 8.34466575838633e-08
1268 8.46691676770206e-08
1269 8.33732016758404e-08
1270 8.45760723677813e-08
1271 8.33010602718787e-08
1272 8.44865368776482e-08
1273 8.32265527606069e-08
1274 8.43932141947334e-08
1275 8.31556477010054e-08
1276 8.42975254045086e-08
1277 8.30858937206358e-08
1278 8.42028882175327e-08
1279 8.3016374219369e-08
1280 8.41062615108967e-08
1281 8.29506561217386e-08
1282 8.40106366695181e-08
1283 8.28845045930393e-08
1284 8.39192750845541e-08
1285 8.28261690344334e-08
1286 8.38250358015102e-08
1287 8.27655881607825e-08
1288 8.37362037486855e-08
1289 8.27144219783804e-08
1290 8.36494891132133e-08
1291 8.26620478733275e-08
1292 8.35655740161201e-08
1293 8.26233232942286e-08
1294 8.34907964986087e-08
1295 8.25938002435578e-08
1296 8.34291284945721e-08
1297 8.25793051717483e-08
1298 8.33788007525982e-08
1299 8.25802430881595e-08
1300 8.33475937156436e-08
1301 8.25939210358229e-08
1302 8.33303914760108e-08
1303 8.26179302748642e-08
1304 8.330535905543e-08
1305 8.26327593017595e-08
1306 8.32724040833455e-08
1307 8.26263146791462e-08
1308 8.3204639622636e-08
1309 8.25994206365976e-08
1310 8.31257054301204e-08
1311 8.25591826014715e-08
1312 8.30197635082186e-08
1313 8.25111996505257e-08
1314 8.29060837759243e-08
1315 8.24562533807693e-08
1316 8.27810211490032e-08
1317 8.23942869487837e-08
1318 8.26471620030134e-08
1319 8.23354042722713e-08
1320 8.25114625513379e-08
1321 8.22704748770775e-08
1322 8.23702350771782e-08
1323 8.22021206658974e-08
1324 8.22285315393856e-08
1325 8.212577284894e-08
1326 8.20852861238563e-08
1327 8.19852914446528e-08
1328 8.19214847069816e-08
1329 8.19101728666283e-08
1330 8.18216321363252e-08
1331 8.18873857610924e-08
1332 8.16871832398647e-08
1333 8.17460517055224e-08
1334 8.16164629213745e-08
1335 8.16402092596036e-08
1336 8.15309419976984e-08
1337 8.1528291673294e-08
1338 8.14179372810031e-08
1339 8.14314304875552e-08
1340 8.13486167317024e-08
1341 8.13131890708974e-08
1342 8.12006106798435e-08
1343 8.12288263318806e-08
1344 8.11612608231371e-08
1345 8.10965303799094e-08
1346 8.10144342722197e-08
1347 8.10460392131063e-08
1348 8.09064104601021e-08
1349 8.09289346648256e-08
1350 8.08056839218807e-08
1351 8.08282436537411e-08
1352 8.07071884878496e-08
1353 8.07025770654946e-08
1354 8.06186335466919e-08
1355 8.06019002652647e-08
1356 8.05188165031723e-08
1357 8.04828772515975e-08
1358 8.04257140885056e-08
1359 8.03803601456821e-08
1360 8.03218611622469e-08
1361 8.02651243247965e-08
1362 8.02287516421529e-08
1363 8.01567736630204e-08
1364 8.01249626647405e-08
1365 8.00475632445341e-08
1366 8.00271990897272e-08
1367 7.99426516095991e-08
1368 7.99210368995773e-08
1369 7.9837001010219e-08
1370 7.98232093757179e-08
1371 7.9729367996606e-08
1372 7.97192001300573e-08
1373 7.96243639911154e-08
1374 7.96131729430272e-08
1375 7.951797442729e-08
1376 7.95084034166393e-08
1377 7.94169352502649e-08
1378 7.94044652252524e-08
1379 7.9312144407595e-08
1380 7.93004986121559e-08
1381 7.92064298593687e-08
1382 7.9196681213034e-08
1383 7.9104722772172e-08
1384 7.90892684676692e-08
1385 7.90074281553643e-08
1386 7.89841365644861e-08
1387 7.89023886227369e-08
1388 7.88762406500609e-08
1389 7.87977896266057e-08
1390 7.87672576052501e-08
1391 7.86870160141007e-08
1392 7.86547857956066e-08
1393 7.85929898938775e-08
1394 7.85659537427819e-08
1395 7.84896201366792e-08
1396 7.84513147777943e-08
1397 7.84005109721875e-08
1398 7.83428362183258e-08
1399 7.82858222692084e-08
1400 7.82459181891682e-08
1401 7.81755886691826e-08
1402 7.81205002908791e-08
1403 7.80621860485553e-08
1404 7.80086253371337e-08
1405 7.79468720679688e-08
1406 7.78885649310723e-08
1407 7.78269182433178e-08
1408 7.77658044626151e-08
1409 7.76999300455827e-08
1410 7.76476696273676e-08
1411 7.75936825903045e-08
1412 7.75646995521129e-08
1413 7.75582975620637e-08
1414 7.76304887040169e-08
1415 7.78182354110868e-08
1416 7.81970470598026e-08
1417 7.84978908541234e-08
1418 7.83180880148393e-08
1419 7.81900979518468e-08
1420 7.8143585824364e-08
1421 7.80742652750632e-08
1422 7.80142173084641e-08
1423 7.79484352619875e-08
1424 7.78901778630825e-08
1425 7.78237918552804e-08
1426 7.77593740508564e-08
1427 7.76926469825412e-08
1428 7.76327411244893e-08
1429 7.75658293150627e-08
1430 7.75052413359845e-08
1431 7.74377610923693e-08
1432 7.73753825455969e-08
1433 7.73085133687346e-08
1434 7.72461774545263e-08
1435 7.71784272046716e-08
1436 7.71164181401218e-08
1437 7.70513040038168e-08
1438 7.69845414083647e-08
1439 7.69198962302653e-08
1440 7.68536807527198e-08
1441 7.67885097729959e-08
1442 7.67252004152397e-08
1443 7.66539329788429e-08
1444 7.65920376011309e-08
1445 7.65223546750349e-08
1446 7.6455435760181e-08
1447 7.6391017955757e-08
1448 7.63245111556898e-08
1449 7.62533147735667e-08
1450 7.6190474374016e-08
1451 7.61187024522769e-08
1452 7.60519398568249e-08
1453 7.59807363692744e-08
1454 7.59121121518547e-08
1455 7.58412923573815e-08
1456 7.57739613277408e-08
1457 7.57040794496788e-08
1458 7.56384466171767e-08
1459 7.55650688688547e-08
1460 7.54948388248522e-08
1461 7.54194360297333e-08
1462 7.53495399408166e-08
1463 7.52814273141666e-08
1464 7.52105577817019e-08
1465 7.5134259702736e-08
1466 7.50652091596749e-08
1467 7.49948014799884e-08
1468 7.49222053286758e-08
1469 7.48470299072324e-08
1470 7.47776951470769e-08
1471 7.47043173987549e-08
1472 7.46282893260286e-08
1473 7.45576471672393e-08
1474 7.44845181088749e-08
1475 7.44061452451206e-08
1476 7.43345367482107e-08
1477 7.42606545145463e-08
1478 7.41871843956687e-08
1479 7.4109479442086e-08
1480 7.40381551622704e-08
1481 7.39569614438551e-08
1482 7.38791996468535e-08
1483 7.38050403015222e-08
1484 7.37271079742641e-08
1485 7.36473424467476e-08
1486 7.35733038936814e-08
1487 7.34940854840715e-08
1488 7.34126643919808e-08
1489 7.33317548906598e-08
1490 7.32529343849819e-08
1491 7.31707956447281e-08
1492 7.30868592313527e-08
1493 7.29875679894576e-08
1494 7.28648146264277e-08
1495 7.27289659607777e-08
1496 7.26353306390592e-08
1497 7.25445516991385e-08
1498 7.24572259969136e-08
1499 7.23719821849045e-08
1500 7.22897866012318e-08
1501 7.22011463949457e-08
1502 7.21211037557623e-08
1503 7.20346804428118e-08
1504 7.19518737923863e-08
1505 7.18718453640577e-08
1506 7.17917529868828e-08
1507 7.17099766234242e-08
1508 7.16260970534677e-08
1509 7.15551209395926e-08
1510 7.14665517875801e-08
1511 7.14021552994382e-08
1512 7.13096071081054e-08
1513 7.12572969518988e-08
1514 7.11605778747071e-08
1515 7.11178742562879e-08
1516 7.10165082296044e-08
1517 7.0979005784011e-08
1518 7.08842406993426e-08
1519 7.08443792518665e-08
1520 7.07535292576722e-08
1521 7.07159699686599e-08
1522 7.06279763562634e-08
1523 7.05820397683965e-08
1524 7.05159663993982e-08
1525 7.04441447396675e-08
1526 7.03747957686573e-08
1527 7.03151172842809e-08
1528 7.02567177768287e-08
1529 7.01823452686767e-08
1530 7.00976983125656e-08
1531 7.00723390423263e-08
1532 7.00346944881858e-08
1533 6.99501754297671e-08
1534 6.98160178558282e-08
1535 6.98496265272297e-08
1536 6.97335380550612e-08
1537 6.96585829018659e-08
1538 6.96421622592425e-08
1539 6.95674984285688e-08
1540 6.94604338491445e-08
1541 6.9378522482566e-08
1542 6.9299638028042e-08
1543 6.93122643724564e-08
1544 6.92726516149378e-08
1545 6.91131987196059e-08
1546 6.90868233732544e-08
1547 6.89833683509278e-08
1548 6.89711256995906e-08
1549 6.88898893486112e-08
1550 6.87715484559703e-08
1551 6.87223504769463e-08
1552 6.86847769770793e-08
1553 6.85624215179814e-08
1554 6.85364867081262e-08
1555 6.85024232893738e-08
1556 6.83667948919719e-08
1557 6.82977585597655e-08
1558 6.82483261016387e-08
1559 6.81602614349686e-08
1560 6.81756873177619e-08
1561 6.80320724200101e-08
1562 6.8023567223463e-08
1563 6.78908094187136e-08
1564 6.78590836855619e-08
1565 6.77460434417299e-08
1566 6.77130813642179e-08
1567 6.76007019251301e-08
1568 6.75745468470268e-08
1569 6.74580107329348e-08
1570 6.74370070896657e-08
1571 6.73095783554345e-08
1572 6.73101396841957e-08
1573 6.71573801014347e-08
1574 6.71767068638474e-08
1575 6.70104824962436e-08
1576 6.70485036380342e-08
1577 6.6865162295926e-08
1578 6.69197106617503e-08
1579 6.67214550276185e-08
1580 6.67986483904315e-08
1581 6.65806751953824e-08
1582 6.66733654952623e-08
1583 6.64409256501131e-08
1584 6.65515713649256e-08
1585 6.6303208257068e-08
1586 6.6428427203391e-08
1587 6.61707630911224e-08
1588 6.63055814698055e-08
1589 6.60354046999601e-08
1590 6.61813146507484e-08
1591 6.5904501411751e-08
1592 6.60565362409216e-08
1593 6.57760494959803e-08
1594 6.59337189290454e-08
1595 6.56481162764067e-08
1596 6.58109016171693e-08
1597 6.55203677979443e-08
1598 6.56912106933305e-08
1599 6.53965201991014e-08
1600 6.5566915452564e-08
1601 6.52733191941479e-08
1602 6.54464571425706e-08
1603 6.51513616389821e-08
1604 6.53207337109052e-08
1605 6.50305693739028e-08
1606 6.51978595556102e-08
1607 6.49103952810037e-08
1608 6.50742535412974e-08
1609 6.47894466965226e-08
1610 6.49527720497645e-08
1611 6.46719087171732e-08
1612 6.48319797846852e-08
1613 6.45508748675638e-08
1614 6.47084803517828e-08
1615 6.44374509306544e-08
1616 6.45818047928515e-08
1617 6.43200266381427e-08
1618 6.44632436319625e-08
1619 6.42068584966182e-08
1620 6.4344312988851e-08
1621 6.40907700244497e-08
1622 6.42251123394999e-08
1623 6.39772039789932e-08
1624 6.41041637550188e-08
1625 6.3865989829992e-08
1626 6.39839399241282e-08
1627 6.37514219192781e-08
1628 6.38603125935333e-08
1629 6.36401580322854e-08
1630 6.37414885318321e-08
1631 6.35307486618331e-08
1632 6.36175130352967e-08
1633 6.3420628748645e-08
1634 6.3496571556243e-08
1635 6.33135499583659e-08
1636 6.33718713061171e-08
1637 6.3209135703346e-08
1638 6.32484855600524e-08
1639 6.31038616916157e-08
1640 6.31255758776206e-08
1641 6.30024885595049e-08
1642 6.2998516625612e-08
1643 6.28951326575589e-08
1644 6.28799625701504e-08
1645 6.27893470550589e-08
1646 6.27582039669505e-08
1647 6.26803071668292e-08
1648 6.26382714585816e-08
1649 6.25713809654371e-08
1650 6.25197245085474e-08
1651 6.24620639655404e-08
1652 6.24005096483415e-08
1653 6.23513969344458e-08
1654 6.22822682316837e-08
1655 6.22389535465118e-08
1656 6.21635081188288e-08
1657 6.21290965341359e-08
1658 6.20481515056781e-08
1659 6.20195592659911e-08
1660 6.19314377559022e-08
1661 6.19054674189101e-08
1662 6.18142550479206e-08
1663 6.17948856529438e-08
1664 6.1695104136561e-08
1665 6.16798203623148e-08
1666 6.15806783343942e-08
1667 6.15669222270299e-08
1668 6.14669559695358e-08
1669 6.14548341104637e-08
1670 6.13514927749748e-08
1671 6.13392927562018e-08
1672 6.12365624874656e-08
1673 6.12248243214708e-08
1674 6.1120253747049e-08
1675 6.11104837844323e-08
1676 6.10037389492391e-08
1677 6.09947861107685e-08
1678 6.08862720241632e-08
1679 6.08779799904369e-08
1680 6.07712209443889e-08
1681 6.07662471452386e-08
1682 6.06530150548679e-08
1683 6.06493770760608e-08
1684 6.05380776619313e-08
1685 6.05334236070121e-08
1686 6.04226997324986e-08
1687 6.04165109052701e-08
1688 6.03072507487923e-08
1689 6.03000671617338e-08
1690 6.01926686272236e-08
1691 6.01839573732832e-08
1692 6.00750809098827e-08
1693 6.00640319703416e-08
1694 5.99590705974151e-08
1695 5.9952014908049e-08
1696 5.98430318632381e-08
1697 5.98334537471601e-08
1698 5.97271991864545e-08
1699 5.97153402281947e-08
1700 5.96114233530898e-08
1701 5.95998201902148e-08
1702 5.9495338433635e-08
1703 5.94828968303318e-08
1704 5.93807740756347e-08
1705 5.93644884361311e-08
1706 5.92645683639148e-08
1707 5.92448046177196e-08
1708 5.91493538593113e-08
1709 5.91290891804874e-08
1710 5.9033318677848e-08
1711 5.90125104338313e-08
1712 5.89198094758103e-08
1713 5.88923967370647e-08
1714 5.88049680061431e-08
1715 5.87768553828028e-08
1716 5.86891069076501e-08
1717 5.8660980073455e-08
1718 5.85739954317432e-08
1719 5.85422945675873e-08
1720 5.84607207088084e-08
1721 5.84264974179405e-08
1722 5.83451935654011e-08
1723 5.83108921148323e-08
1724 5.82317660757781e-08
1725 5.81934571641796e-08
1726 5.8117862522522e-08
1727 5.8075876552266e-08
1728 5.80050247833697e-08
1729 5.79598555816574e-08
1730 5.78907766168868e-08
1731 5.78409355966869e-08
1732 5.77756615882663e-08
1733 5.77234118281922e-08
1734 5.76607490643255e-08
1735 5.76018805986678e-08
1736 5.75464014218596e-08
1737 5.74846161782716e-08
1738 5.74361145311286e-08
1739 5.73671528059094e-08
1740 5.73255043434528e-08
1741 5.72498670692312e-08
1742 5.72181306779385e-08
1743 5.7136606557151e-08
1744 5.71006033567301e-08
1745 5.70201557081873e-08
1746 5.69409017714406e-08
1747 5.69198661537484e-08
1748 5.68177327409103e-08
1749 5.68453266680535e-08
1750 5.66971856130749e-08
1751 5.67070514989609e-08
1752 5.66317552852524e-08
1753 5.65778357497493e-08
1754 5.64762956400955e-08
1755 5.65056659240781e-08
1756 5.63649003026967e-08
1757 5.63529383157402e-08
1758 5.62674706827693e-08
1759 5.62495650058281e-08
1760 5.6148515170662e-08
1761 5.61192230463803e-08
1762 5.60690551765219e-08
1763 5.60166633079007e-08
1764 5.59371500230554e-08
1765 5.58862289778972e-08
1766 5.58091564073493e-08
1767 5.58025377017657e-08
1768 5.57254900002135e-08
1769 5.566453964434e-08
1770 5.5574361113031e-08
1771 5.55825714343428e-08
1772 5.54884245218545e-08
1773 5.54372334704567e-08
1774 5.53507604195147e-08
1775 5.53144481330037e-08
1776 5.52334071812766e-08
1777 5.5234210094568e-08
1778 5.51156169592559e-08
1779 5.51222356648395e-08
1780 5.50306644697685e-08
1781 5.49826957296773e-08
1782 5.4910472613301e-08
1783 5.4860510800836e-08
1784 5.47880034673653e-08
1785 5.47366845182751e-08
1786 5.46733325279547e-08
1787 5.46207097329443e-08
1788 5.45606333446358e-08
1789 5.45163807430527e-08
1790 5.44636336030635e-08
1791 5.44060299034754e-08
1792 5.43467173486079e-08
1793 5.42898703770334e-08
1794 5.42304974260333e-08
1795 5.41712843471487e-08
1796 5.41155067423915e-08
1797 5.40579847552181e-08
1798 5.40005409277455e-08
1799 5.39469660054692e-08
1800 5.38927729110128e-08
1801 5.38363629232208e-08
1802 5.37829691893421e-08
1803 5.37279909451627e-08
1804 5.3671776356623e-08
1805 5.36175406296024e-08
1806 5.35647224353397e-08
1807 5.35104938137465e-08
1808 5.34566453325169e-08
1809 5.34009956254522e-08
1810 5.33451611772762e-08
1811 5.32893871252327e-08
1812 5.32361568161832e-08
1813 5.31848520779477e-08
1814 5.31273016690648e-08
1815 5.30723482938811e-08
1816 5.30195798376099e-08
1817 5.29693195971959e-08
1818 5.29129096094039e-08
1819 5.2860933408283e-08
1820 5.28034860280968e-08
1821 5.27521883952886e-08
1822 5.26988870319656e-08
1823 5.26465377959084e-08
1824 5.25960111019685e-08
1825 5.25407628515495e-08
1826 5.24893835063267e-08
1827 5.24380645572364e-08
1828 5.24175547411687e-08
1829 5.23679233310759e-08
1830 5.23118259820876e-08
1831 5.22593097684876e-08
1832 5.2206036826874e-08
1833 5.2153243501607e-08
1834 5.21004004383485e-08
1835 5.20471878928674e-08
1836 5.19949345800796e-08
1837 5.19429654843861e-08
1838 5.18905274304871e-08
1839 5.18408782568258e-08
1840 5.17874561012377e-08
1841 5.17350287054796e-08
1842 5.16829850027989e-08
1843 5.16299607511428e-08
1844 5.15787590416039e-08
1845 5.15281612933904e-08
1846 5.14762987791073e-08
1847 5.14253919448038e-08
1848 5.13737319352003e-08
1849 5.13221358744431e-08
1850 5.12731084256757e-08
1851 5.12181692613467e-08
1852 5.11713267314917e-08
1853 5.11189348628704e-08
1854 5.1066798789634e-08
1855 5.10153306265693e-08
1856 5.09667898995758e-08
1857 5.09158901706996e-08
1858 5.08623614337012e-08
1859 5.08120301390136e-08
1860 5.07596631393881e-08
1861 5.07109056968602e-08
1862 5.066232944273e-08
1863 5.06128472466116e-08
1864 5.05610167067516e-08
1865 5.0508926818793e-08
1866 5.04581869620324e-08
1867 5.0408782925615e-08
1868 5.03576202959266e-08
1869 5.03110264560291e-08
1870 5.02590360440536e-08
1871 5.02073405073133e-08
1872 5.01566503885442e-08
1873 5.01071468761438e-08
1874 5.00583539064792e-08
1875 5.00057772967466e-08
1876 4.99578867163564e-08
1877 4.9905860777244e-08
1878 4.98570464912973e-08
1879 4.98093619683004e-08
1880 4.97569097035466e-08
1881 4.97078325167877e-08
1882 4.96572418740016e-08
1883 4.96091487889316e-08
1884 4.95551830681507e-08
1885 4.95083725127188e-08
1886 4.94571565923252e-08
1887 4.94079266388781e-08
1888 4.93545186941446e-08
1889 4.93042122684528e-08
1890 4.92546483599199e-08
1891 4.92058127576911e-08
1892 4.91543019620622e-08
1893 4.91020877291248e-08
1894 4.90529004082418e-08
1895 4.90029954391957e-08
1896 4.89559219829516e-08
1897 4.89034199802063e-08
1898 4.8849546629981e-08
1899 4.88009845867055e-08
1900 4.87490403600077e-08
1901 4.86989506498503e-08
1902 4.86463562765493e-08
1903 4.85946145545313e-08
1904 4.85434341612745e-08
1905 4.84889390861554e-08
1906 4.84392863597805e-08
1907 4.83862621081244e-08
1908 4.83315005794793e-08
1909 4.82780073696176e-08
1910 4.82192881179344e-08
1911 4.81613042779827e-08
1912 4.8101849614568e-08
1913 4.80345541120641e-08
1914 4.79625050786581e-08
1915 4.78973412043615e-08
1916 4.78815920246234e-08
1917 4.78668091830059e-08
1918 4.78229686962095e-08
1919 4.77756323391532e-08
1920 4.77268535803432e-08
1921 4.76791939263421e-08
1922 4.76334776067233e-08
1923 4.75845425285115e-08
1924 4.753715288075e-08
1925 4.74911630021779e-08
1926 4.74458232702091e-08
1927 4.74005368289454e-08
1928 4.73531365230428e-08
1929 4.73074521778472e-08
1930 4.72622865288486e-08
1931 4.7216069276601e-08
1932 4.71695749126866e-08
1933 4.71240575450338e-08
1934 4.70782204331499e-08
1935 4.70326568802193e-08
1936 4.69871537234212e-08
1937 4.6941121212285e-08
1938 4.68974228340358e-08
1939 4.68520440222164e-08
1940 4.68056349234303e-08
1941 4.67606717791114e-08
1942 4.67155736316727e-08
1943 4.66696441492331e-08
1944 4.66259955089754e-08
1945 4.65807019622844e-08
1946 4.65363463320045e-08
1947 4.64886156237299e-08
1948 4.64459759541569e-08
1949 4.63990126320368e-08
1950 4.63545077877825e-08
1951 4.63099958381008e-08
1952 4.62668268141897e-08
1953 4.62207587759167e-08
1954 4.61751277214262e-08
1955 4.61322073874726e-08
1956 4.60880968944366e-08
1957 4.60445441774482e-08
1958 4.60015598946484e-08
1959 4.59579752032369e-08
1960 4.59138718156282e-08
1961 4.58706956862898e-08
1962 4.58288909044313e-08
1963 4.57833060352186e-08
1964 4.57420377131257e-08
1965 4.56995330466725e-08
1966 4.56546302984862e-08
1967 4.5614171995112e-08
1968 4.5571720619364e-08
1969 4.5529180425774e-08
1970 4.54881217137881e-08
1971 4.54456241527623e-08
1972 4.54027073715224e-08
1973 4.53642670095178e-08
1974 4.53245938558666e-08
1975 4.52820927421271e-08
1976 4.52402275641361e-08
1977 4.51996200467875e-08
1978 4.51596875450377e-08
1979 4.51193926664928e-08
1980 4.50783197436522e-08
1981 4.5037186424679e-08
1982 4.49967139104501e-08
1983 4.49599326657335e-08
1984 4.49188668483202e-08
1985 4.48774315486844e-08
1986 4.48388846052694e-08
1987 4.48002808184356e-08
1988 4.47595880359586e-08
1989 4.47215242616039e-08
1990 4.46807000287208e-08
1991 4.464250480396e-08
1992 4.46022276889835e-08
1993 4.4564814061232e-08
1994 4.45252439362775e-08
1995 4.44847287894845e-08
1996 4.4448178471157e-08
1997 4.4406501586991e-08
1998 4.43681500428283e-08
1999 4.43300223196275e-08
2000 4.42897380992235e-08
2001 4.42515677434585e-08
2002 4.42118128773927e-08
2003 4.41746195178894e-08
2004 4.41356711178287e-08
2005 4.40952589997323e-08
2006 4.40586909178364e-08
2007 4.40184955152745e-08
2008 4.398135899919e-08
2009 4.3939603955323e-08
2010 4.39029896881493e-08
2011 4.38642260291999e-08
2012 4.38247553802285e-08
2013 4.37865530500403e-08
2014 4.37494485083789e-08
2015 4.37107487982757e-08
2016 4.36724079122541e-08
2017 4.36336691223005e-08
2018 4.35956373223689e-08
2019 4.35582130364764e-08
2020 4.35206679583189e-08
2021 4.34816307404162e-08
2022 4.34416556061024e-08
2023 4.34046576458513e-08
2024 4.33650697573285e-08
2025 4.33291198476127e-08
2026 4.32909530445613e-08
2027 4.32516493731328e-08
2028 4.32137277073252e-08
2029 4.31763282904285e-08
2030 4.31382964904969e-08
2031 4.3101529456635e-08
2032 4.30624709224503e-08
2033 4.30255227001908e-08
2034 4.2986652459831e-08
2035 4.29487130304551e-08
2036 4.29108517607801e-08
2037 4.28741309121961e-08
2038 4.28363016169442e-08
2039 4.27981419193202e-08
2040 4.27608277675517e-08
2041 4.2722941628881e-08
2042 4.26860147229036e-08
2043 4.2647496201198e-08
2044 4.26104023176777e-08
2045 4.25735535714011e-08
2046 4.25379873547627e-08
2047 4.24996606795958e-08
2048 4.24660342446259e-08
2049 4.24253414621489e-08
2050 4.23885211375818e-08
2051 4.23533172977386e-08
2052 4.23142552108402e-08
2053 4.22771080366147e-08
2054 4.22440713521155e-08
2055 4.22052082171831e-08
2056 4.21696384478309e-08
2057 4.21316421750362e-08
2058 4.2098882602204e-08
2059 4.20606802720158e-08
2060 4.2023657442769e-08
2061 4.19871959422835e-08
2062 4.19497929726731e-08
2063 4.19169481347126e-08
2064 4.18781880284769e-08
2065 4.1843279063869e-08
2066 4.18040499994277e-08
2067 4.17737879843116e-08
2068 4.17358556603631e-08
2069 4.16994261343007e-08
2070 4.16629752919562e-08
2071 4.16242400547162e-08
2072 4.15905923034643e-08
2073 4.15557899202668e-08
2074 4.15215133386937e-08
2075 4.14836200945956e-08
2076 4.14499154999248e-08
2077 4.14126510861479e-08
2078 4.13784988495536e-08
2079 4.13420799816322e-08
2080 4.13066594262546e-08
2081 4.12727629850451e-08
2082 4.12369089985987e-08
2083 4.12048990483527e-08
2084 4.1173262133043e-08
2085 4.11393017429873e-08
2086 4.1110812531997e-08
2087 4.10843874476541e-08
2088 4.10656895155626e-08
2089 4.1056161137476e-08
2090 4.10471052703087e-08
2091 4.10407956508152e-08
2092 4.102483686097e-08
2093 4.10121678839914e-08
2094 4.09890148489467e-08
2095 4.09652223254398e-08
2096 4.09365448206245e-08
2097 4.09043749982629e-08
2098 4.0879598373067e-08
2099 4.08490770098524e-08
2100 4.0821440450145e-08
2101 4.07937399415914e-08
2102 4.0762785147308e-08
2103 4.07392022339081e-08
2104 4.07022042736571e-08
2105 4.06773281724782e-08
2106 4.06443163569747e-08
2107 4.06184810231025e-08
2108 4.05868512132201e-08
2109 4.05548519211152e-08
2110 4.0527964983994e-08
2111 4.04979125789851e-08
2112 4.04688620392335e-08
2113 4.04422095812151e-08
2114 4.04104483209267e-08
2115 4.03758448896951e-08
2116 4.03544682114898e-08
2117 4.03283024752454e-08
2118 4.0296850301047e-08
2119 4.02655295772547e-08
2120 4.02377686725686e-08
2121 4.02114963549138e-08
2122 4.01789819193255e-08
2123 4.01535018568211e-08
2124 4.01232789215555e-08
2125 4.00958128921047e-08
2126 4.00633233255121e-08
2127 4.00347381912525e-08
2128 4.00070518935536e-08
2129 3.99760615721334e-08
2130 3.9950471375505e-08
2131 3.99227282343873e-08
2132 3.98969746129296e-08
2133 3.98654087518935e-08
2134 3.98347026475676e-08
2135 3.98096773324141e-08
2136 3.97785271388784e-08
2137 3.97492030401736e-08
2138 3.97243731242725e-08
2139 3.96961077342439e-08
2140 3.96660873036581e-08
2141 3.96433748051095e-08
2142 3.9612874758177e-08
2143 3.95871637692835e-08
2144 3.95580066481216e-08
2145 3.95303665357005e-08
2146 3.94996888530841e-08
2147 3.94783299384471e-08
2148 3.94515780044458e-08
2149 3.94247514634571e-08
2150 3.93946812948798e-08
2151 3.93659824737824e-08
2152 3.93402892484573e-08
2153 3.93140311416573e-08
2154 3.92848171770765e-08
2155 3.92568999529885e-08
2156 3.92329759790755e-08
2157 3.9204245183555e-08
2158 3.91781185271611e-08
2159 3.91496044471751e-08
2160 3.91225825069341e-08
2161 3.90979941755631e-08
2162 3.90652630244404e-08
2163 3.90389409687941e-08
2164 3.90146333018038e-08
2165 3.89836252168152e-08
2166 3.89623764363023e-08
2167 3.89331589190078e-08
2168 3.89001328926497e-08
2169 3.88779035631615e-08
2170 3.88494427738806e-08
2171 3.88275758211876e-08
2172 3.87970153781225e-08
2173 3.87696346138e-08
2174 3.874786713709e-08
2175 3.87157790271431e-08
2176 3.8687254289016e-08
2177 3.86612093450367e-08
2178 3.86343756986207e-08
2179 3.86096203897068e-08
2180 3.85846306016902e-08
2181 3.85532885616158e-08
2182 3.85296914373612e-08
2183 3.84971592382044e-08
2184 3.84746101644851e-08
2185 3.84464620140079e-08
2186 3.84222005322954e-08
2187 3.8391355872136e-08
2188 3.83649663149299e-08
2189 3.83385447833007e-08
2190 3.83136402604123e-08
2191 3.82860321224143e-08
2192 3.82621259120697e-08
2193 3.82306417634481e-08
2194 3.82075491245359e-08
2195 3.81723523901201e-08
2196 3.81623230794048e-08
2197 3.81088760548209e-08
2198 3.81132458926459e-08
2199 3.80613158768028e-08
2200 3.80572586777816e-08
2201 3.79936189176533e-08
2202 3.80150559919912e-08
2203 3.795155834041e-08
2204 3.79658402493988e-08
2205 3.78819287050192e-08
2206 3.79154698748607e-08
2207 3.78332032369144e-08
2208 3.78624847030551e-08
2209 3.77810422946823e-08
2210 3.78189710659171e-08
2211 3.77177293842124e-08
2212 3.77708921917019e-08
2213 3.76643782828978e-08
2214 3.77159281583772e-08
2215 3.76126152445977e-08
2216 3.76697499859802e-08
2217 3.7560592858199e-08
2218 3.76175357530428e-08
2219 3.75082400694282e-08
2220 3.75664654939101e-08
2221 3.74537769687322e-08
2222 3.75145177144987e-08
2223 3.74043658268874e-08
2224 3.74652806556242e-08
2225 3.73496433780929e-08
2226 3.74156456928176e-08
2227 3.73024278133016e-08
2228 3.7362372751204e-08
2229 3.72515280844254e-08
2230 3.73112918339302e-08
2231 3.72006923043955e-08
2232 3.72634474388178e-08
2233 3.71496859941089e-08
2234 3.72149244753928e-08
2235 3.70906860780451e-08
2236 3.71686148525896e-08
2237 3.70475348177024e-08
2238 3.71172781399309e-08
2239 3.69966244306852e-08
2240 3.70683501671465e-08
2241 3.69440371628116e-08
2242 3.70210173628038e-08
2243 3.68972479236618e-08
2244 3.697137529457e-08
2245 3.68465364886106e-08
2246 3.69206141215273e-08
2247 3.68006531914489e-08
2248 3.68733807931676e-08
2249 3.67456998162652e-08
2250 3.68259520655556e-08
2251 3.66995003275861e-08
2252 3.67755141894577e-08
2253 3.66535282125824e-08
2254 3.6724355112483e-08
2255 3.66042094412933e-08
2256 3.66774095539313e-08
2257 3.65556473980178e-08
2258 3.66299524046099e-08
2259 3.65059804607881e-08
2260 3.65837848903539e-08
2261 3.64581076439663e-08
2262 3.65345549369067e-08
2263 3.64115315676372e-08
2264 3.64866821200849e-08
2265 3.6363463351563e-08
2266 3.64370151828552e-08
2267 3.63172176776061e-08
2268 3.63916434764633e-08
2269 3.62722936131377e-08
2270 3.63447121287663e-08
2271 3.62208858462054e-08
2272 3.62982568447023e-08
2273 3.61752263700055e-08
2274 3.62492329486486e-08
2275 3.61302774365413e-08
2276 3.62052787750144e-08
2277 3.60822589584586e-08
2278 3.61555336780839e-08
2279 3.60365390861261e-08
2280 3.61100518375679e-08
2281 3.599081566108e-08
2282 3.60616780881173e-08
2283 3.59458383059064e-08
2284 3.60192800030745e-08
2285 3.58972052083573e-08
2286 3.597384790055e-08
2287 3.58539544720315e-08
2288 3.59255274418047e-08
2289 3.58057477001239e-08
2290 3.58838789793481e-08
2291 3.57588803012732e-08
2292 3.58350504825466e-08
2293 3.57124889660554e-08
2294 3.57899203606848e-08
2295 3.5671867237852e-08
2296 3.5744491810874e-08
2297 3.56259803879766e-08
2298 3.57009390938856e-08
2299 3.55820297670562e-08
2300 3.56544305191164e-08
2301 3.55341924773711e-08
2302 3.561096661997e-08
2303 3.54921247947004e-08
2304 3.55687532760385e-08
2305 3.54467921681589e-08
2306 3.55206815072506e-08
2307 3.54039499939063e-08
2308 3.54756650722265e-08
2309 3.53574520772781e-08
2310 3.54290037307692e-08
2311 3.53158675636678e-08
2312 3.53893092608359e-08
2313 3.52691564842189e-08
2314 3.53439055800209e-08
2315 3.5224182681759e-08
2316 3.53016922360894e-08
2317 3.51776314744257e-08
2318 3.52570488360016e-08
2319 3.51366686857091e-08
2320 3.52130129499528e-08
2321 3.5097304618148e-08
2322 3.5168358891724e-08
2323 3.50504691937203e-08
2324 3.5127477815422e-08
2325 3.50059110587608e-08
2326 3.50804327808873e-08
2327 3.49635733698506e-08
2328 3.50346383015676e-08
2329 3.49176119129879e-08
2330 3.49923467979352e-08
2331 3.48749082945687e-08
2332 3.49506770191965e-08
2333 3.48323005994189e-08
2334 3.49058026927196e-08
2335 3.47865949379411e-08
2336 3.48598234722886e-08
2337 3.47465594074947e-08
2338 3.48209816536382e-08
2339 3.47028503711044e-08
2340 3.47721034188453e-08
2341 3.46605411039036e-08
2342 3.47349882190429e-08
2343 3.46149029439857e-08
2344 3.46889059699151e-08
2345 3.45701636206286e-08
2346 3.46471118461977e-08
2347 3.45297337389638e-08
2348 3.46065185397038e-08
2349 3.44823760656254e-08
2350 3.45558426317893e-08
2351 3.44428521259488e-08
2352 3.45145139135639e-08
2353 3.43980488537454e-08
2354 3.44693944498431e-08
2355 3.43555761617154e-08
2356 3.44256001483245e-08
2357 3.43134090030617e-08
2358 3.43793047363761e-08
2359 3.42693660115856e-08
2360 3.43405446301404e-08
2361 3.42233228423083e-08
2362 3.42998234259539e-08
2363 3.41809176518382e-08
2364 3.42499539840446e-08
2365 3.41405801407291e-08
2366 3.42065895608812e-08
2367 3.40918298036286e-08
2368 3.41624506461358e-08
2369 3.40534036524787e-08
2370 3.41182584406852e-08
2371 3.40134747034426e-08
2372 3.40747199345515e-08
2373 3.39676979876913e-08
2374 3.40325740921799e-08
2375 3.39218964029442e-08
2376 3.39866517151677e-08
2377 3.3881736527519e-08
2378 3.39456640574554e-08
2379 3.38372672104015e-08
2380 3.39018164652316e-08
2381 3.37945920136917e-08
2382 3.38580825598456e-08
2383 3.37540093653388e-08
2384 3.38165122570899e-08
2385 3.37097922908924e-08
2386 3.37700640784533e-08
2387 3.36639658371496e-08
2388 3.37292114238608e-08
2389 3.36259589062138e-08
2390 3.36831256220194e-08
2391 3.35814327456774e-08
2392 3.36422765201405e-08
2393 3.3539549804118e-08
2394 3.36023227021087e-08
2395 3.34981642424736e-08
2396 3.35587628796929e-08
2397 3.34536132129415e-08
2398 3.35151533192857e-08
2399 3.3415801681258e-08
2400 3.34728440520848e-08
2401 3.33701883903359e-08
2402 3.34284671055229e-08
2403 3.33325900214732e-08
2404 3.33874901059517e-08
2405 3.3285694200913e-08
2406 3.33439160726812e-08
2407 3.32447314121964e-08
2408 3.33028040699901e-08
2409 3.32063478936107e-08
2410 3.32598517616134e-08
2411 3.31626424099341e-08
2412 3.32168461625315e-08
2413 3.31204716985667e-08
2414 3.31763523320205e-08
2415 3.30778640034168e-08
2416 3.31351444060601e-08
2417 3.30364962053409e-08
2418 3.30953611182849e-08
2419 3.29949934041451e-08
2420 3.30507177181971e-08
2421 3.29555867040199e-08
2422 3.30138050230744e-08
2423 3.29129221654512e-08
2424 3.29708278457019e-08
2425 3.28714868658153e-08
2426 3.2932302218569e-08
2427 3.28306519747912e-08
2428 3.28900320312187e-08
2429 3.27901510388529e-08
2430 3.28493570123101e-08
2431 3.27491811447089e-08
2432 3.28098010982103e-08
2433 3.2708598496356e-08
2434 3.27647136089126e-08
2435 3.26644595816106e-08
2436 3.27270406330626e-08
2437 3.26258771110588e-08
2438 3.26900035929611e-08
2439 3.2585923293027e-08
2440 3.26443583276159e-08
2441 3.25428786140947e-08
2442 3.26064686362315e-08
2443 3.25059268391215e-08
2444 3.25664295530714e-08
2445 3.24619975344831e-08
2446 3.25291722447218e-08
2447 3.24223314862593e-08
2448 3.2482869727346e-08
2449 3.23825801729072e-08
2450 3.24487210434654e-08
2451 3.2342796885132e-08
2452 3.2407612593488e-08
2453 3.23026050352837e-08
2454 3.23683408964826e-08
2455 3.22648432415917e-08
2456 3.23282556280446e-08
2457 3.22220650161853e-08
2458 3.2286209261656e-08
2459 3.21850599505069e-08
2460 3.22472146763175e-08
2461 3.2145550221685e-08
2462 3.22134603436552e-08
2463 3.21057704866234e-08
2464 3.21710373896167e-08
2465 3.20632658201703e-08
2466 3.2132387417505e-08
2467 3.20217381499788e-08
2468 3.20962918465284e-08
2469 3.19861364062035e-08
2470 3.20540074483233e-08
2471 3.19464952269755e-08
2472 3.20113429097546e-08
2473 3.19064206166786e-08
2474 3.19744479782003e-08
2475 3.18655253295219e-08
2476 3.19390380809637e-08
2477 3.18254365083703e-08
2478 3.18972261936779e-08
2479 3.17856390097404e-08
2480 3.18582635827624e-08
2481 3.17484705192328e-08
2482 3.18198090099031e-08
2483 3.1707742209619e-08
2484 3.1781230092065e-08
2485 3.16666621813511e-08
2486 3.17459338816661e-08
2487 3.16255004406685e-08
2488 3.16999830829445e-08
2489 3.15878594392416e-08
2490 3.16674189093646e-08
2491 3.154708494435e-08
2492 3.15905595016375e-08
2493 3.15153769747667e-08
2494 3.15881152346265e-08
2495 3.14666692702303e-08
2496 3.14749861729524e-08
2497 3.15217540958201e-08
2498 3.1408781353548e-08
2499 3.14155208513966e-08
2500 3.14690034031173e-08
2501 3.1348765361372e-08
2502 3.13516466121655e-08
2503 3.13233954329917e-08
2504 3.1313238224584e-08
2505 3.13684402897252e-08
2506 3.12533003921089e-08
2507 3.12627577159219e-08
2508 3.13116146344328e-08
2509 3.1195330763012e-08
2510 3.12084544873414e-08
2511 3.12497796528532e-08
2512 3.11363947957943e-08
2513 3.11440508937721e-08
2514 3.11912842221318e-08
2515 3.10820631455044e-08
2516 3.10822834137525e-08
2517 3.11379650952404e-08
2518 3.10177661333455e-08
2519 3.10282608495527e-08
2520 3.10819565640941e-08
2521 3.0961206931579e-08
2522 3.0970088715776e-08
2523 3.10201677677924e-08
2524 3.09071701565244e-08
2525 3.09065271153486e-08
2526 3.09629974992731e-08
2527 3.08456087338982e-08
2528 3.08468131038353e-08
2529 3.090109146342e-08
2530 3.07908969432447e-08
2531 3.07898666562778e-08
2532 3.08488914413374e-08
2533 3.07310550340389e-08
2534 3.07314245162615e-08
2535 3.07950323019668e-08
2536 3.06709289077389e-08
2537 3.06719947218426e-08
2538 3.07304404145725e-08
2539 3.06171052955051e-08
2540 3.06159293472774e-08
2541 3.06768477287278e-08
2542 3.05644647369263e-08
2543 3.05529361810386e-08
2544 3.06218659318347e-08
2545 3.0508942927554e-08
2546 3.04932221695253e-08
2547 3.05652854137861e-08
2548 3.04497156378147e-08
2549 3.04384002447478e-08
2550 3.05079943530018e-08
2551 3.0398258132891e-08
2552 3.03797520473381e-08
2553 3.04519218730093e-08
2554 3.03429423809121e-08
2555 3.03232887688409e-08
2556 3.03963041403676e-08
2557 3.02825746700819e-08
2558 3.02663067941467e-08
2559 3.0339702306037e-08
2560 3.02325098289202e-08
2561 3.02057081569274e-08
2562 3.02868841117743e-08
2563 3.01778300126898e-08
2564 3.01544069714055e-08
2565 3.02329823398395e-08
2566 3.01248448408842e-08
2567 3.00963201027571e-08
2568 3.01744584874086e-08
2569 3.0067607070805e-08
2570 3.00410931686201e-08
2571 3.01247169431917e-08
2572 3.00126714591897e-08
2573 2.99866016462147e-08
2574 3.00706410882867e-08
2575 2.99598141850765e-08
2576 2.99269942161118e-08
2577 3.0018210139815e-08
2578 2.99091880151536e-08
2579 2.9874076545866e-08
2580 2.99610150023e-08
2581 2.98548847865732e-08
2582 2.981861868534e-08
2583 2.99122966396226e-08
2584 2.98001143761439e-08
2585 2.97624911382854e-08
2586 2.985797209476e-08
2587 2.97406650417997e-08
2588 2.97116837799649e-08
2589 2.98065643278278e-08
2590 2.96920870113127e-08
2591 2.96546680544907e-08
2592 2.97543927274546e-08
2593 2.96358155793541e-08
2594 2.96010487232934e-08
2595 2.96987110459668e-08
2596 2.9586088245992e-08
2597 2.95509572367791e-08
2598 2.96455375803362e-08
2599 2.95233864022748e-08
2600 2.95000273098367e-08
2601 2.96005833178015e-08
2602 2.94811108858539e-08
2603 2.94497848329911e-08
2604 2.95331101796137e-08
2605 2.94053137395167e-08
2606 2.93981265997445e-08
2607 2.94921207455445e-08
2608 2.93866015965705e-08
2609 2.93369240011998e-08
2610 2.93701472031671e-08
2611 2.93016544361535e-08
2612 2.93226545267089e-08
2613 2.92687403202763e-08
2614 2.9320728955895e-08
2615 2.92283424130346e-08
2616 2.92398514289971e-08
2617 2.9212875674034e-08
2618 2.91986346212525e-08
2619 2.91772721539019e-08
2620 2.91715878120158e-08
2621 2.91785742234651e-08
2622 2.91124884199689e-08
2623 2.91159842902289e-08
2624 2.91029440546708e-08
2625 2.90920176837517e-08
2626 2.90453758822196e-08
2627 2.90523356483163e-08
2628 2.90278521219989e-08
2629 2.90789508028411e-08
2630 2.89757409177582e-08
2631 2.89906409989271e-08
2632 2.89762489558143e-08
2633 2.89344566084537e-08
2634 2.89399366693033e-08
2635 2.89155970278898e-08
2636 2.89318755619661e-08
2637 2.88661112790578e-08
2638 2.8878961444434e-08
2639 2.88541492921013e-08
2640 2.88379684576512e-08
2641 2.88320904928696e-08
2642 2.88051005270518e-08
2643 2.87965384870859e-08
2644 2.87778529894922e-08
2645 2.87550179223217e-08
2646 2.87455304004425e-08
2647 2.8715517075284e-08
2648 2.87164692025499e-08
2649 2.86926074011262e-08
2650 2.86862853471348e-08
2651 2.8655888328899e-08
2652 2.8654682182605e-08
2653 2.8632129556172e-08
2654 2.86217094469521e-08
2655 2.8596934598113e-08
2656 2.85933197119448e-08
2657 2.85723231741031e-08
2658 2.8562435971935e-08
2659 2.85421357659743e-08
2660 2.85322361293083e-08
2661 2.85125878463077e-08
2662 2.85019527979102e-08
2663 2.84854273502333e-08
2664 2.84738685962793e-08
2665 2.84581975762421e-08
2666 2.84416845630631e-08
2667 2.84288752538941e-08
2668 2.8416135222642e-08
2669 2.83992438454561e-08
2670 2.8388825512593e-08
2671 2.83732983774598e-08
2672 2.83582473059596e-08
2673 2.83463670314177e-08
2674 2.83282535207263e-08
2675 2.83135701550918e-08
2676 2.83014038870988e-08
2677 2.82862249179061e-08
2678 2.82733374490363e-08
2679 2.82578671573219e-08
2680 2.82461449785387e-08
2681 2.823217215564e-08
2682 2.82145737884321e-08
2683 2.82044201327381e-08
2684 2.81870864426992e-08
2685 2.81763075093977e-08
2686 2.8163949394866e-08
2687 2.81445000638314e-08
2688 2.81378866873183e-08
2689 2.81181975481104e-08
2690 2.81036527383094e-08
2691 2.80911649497284e-08
2692 2.80768848170965e-08
2693 2.80627574511527e-08
2694 2.80485465964375e-08
2695 2.80360339388608e-08
2696 2.80229048854608e-08
2697 2.80038054967235e-08
2698 2.79918719314765e-08
2699 2.79803753500119e-08
2700 2.79602918595856e-08
2701 2.79505414368941e-08
2702 2.79353180587805e-08
2703 2.7919860201564e-08
2704 2.79085803356338e-08
2705 2.78925114116646e-08
2706 2.78763394589987e-08
2707 2.7862775198173e-08
2708 2.78493939021018e-08
2709 2.78369132189482e-08
2710 2.78202794135041e-08
2711 2.78084790750199e-08
2712 2.77918363877916e-08
2713 2.77790963565394e-08
2714 2.77614944366178e-08
2715 2.77511809088082e-08
2716 2.77316161145791e-08
2717 2.77189311503889e-08
2718 2.77063278986134e-08
2719 2.76888396655295e-08
2720 2.76752665229196e-08
2721 2.76634217755145e-08
2722 2.76475962124323e-08
2723 2.76322627001946e-08
2724 2.76170553092925e-08
2725 2.76054255010649e-08
2726 2.75913958347473e-08
2727 2.75739306943024e-08
2728 2.75596150345336e-08
2729 2.75435567687055e-08
2730 2.75334475219324e-08
2731 2.75196754273566e-08
2732 2.75073031019701e-08
2733 2.74980180847706e-08
2734 2.74853420023646e-08
2735 2.7473060271177e-08
2736 2.74549449841288e-08
2737 2.74402562894238e-08
2738 2.74281468648496e-08
2739 2.74130549371421e-08
2740 2.74026543678474e-08
2741 2.73861164856726e-08
2742 2.73717617460534e-08
2743 2.73579381371292e-08
2744 2.73479923151854e-08
2745 2.73329714417514e-08
2746 2.73171227860303e-08
2747 2.73054148181018e-08
2748 2.72943712076312e-08
2749 2.7281609860097e-08
2750 2.72649351984455e-08
2751 2.72542024504219e-08
2752 2.72385349830984e-08
2753 2.72310760607297e-08
2754 2.72161368997104e-08
2755 2.72020415081897e-08
2756 2.71883795477379e-08
2757 2.71754672098723e-08
2758 2.71654911898622e-08
2759 2.71489213332643e-08
2760 2.71356430658898e-08
2761 2.71232121207277e-08
2762 2.71088165249012e-08
2763 2.70937263735505e-08
2764 2.70827946735608e-08
2765 2.70723710116272e-08
2766 2.70601354657174e-08
2767 2.70455959849869e-08
2768 2.70306443894697e-08
2769 2.70186095718827e-08
2770 2.70092570531233e-08
2771 2.69930549023911e-08
2772 2.69843170030981e-08
2773 2.6968070443445e-08
2774 2.69554689680263e-08
2775 2.69433524380247e-08
2776 2.69309374800741e-08
2777 2.69171049893657e-08
2778 2.69039244216174e-08
2779 2.68926942936787e-08
2780 2.68780517842515e-08
2781 2.68672675218795e-08
2782 2.68520849999732e-08
2783 2.68402509107091e-08
2784 2.68263722347228e-08
2785 2.68167390515828e-08
2786 2.68046971285685e-08
2787 2.67921365093571e-08
2788 2.67756963268084e-08
2789 2.6763666838292e-08
2790 2.67522750618809e-08
2791 2.67405653175956e-08
2792 2.67268536191523e-08
2793 2.67159236955195e-08
2794 2.67015636268297e-08
2795 2.66880952892734e-08
2796 2.66779398572226e-08
2797 2.66661377423816e-08
2798 2.66529589509901e-08
2799 2.66395829839894e-08
2800 2.66259867487406e-08
2801 2.66166271245538e-08
2802 2.65983022273986e-08
2803 2.6588132584493e-08
2804 2.65760338180598e-08
2805 2.65625903494993e-08
2806 2.65491966189302e-08
2807 2.65356234763203e-08
2808 2.65229651574828e-08
2809 2.65103725638483e-08
2810 2.64982649156309e-08
2811 2.64848178943566e-08
2812 2.64730743992914e-08
2813 2.64574619990299e-08
2814 2.64433843710776e-08
2815 2.64313655407022e-08
2816 2.6419986198789e-08
2817 2.64094630608724e-08
2818 2.63929624821913e-08
2819 2.63826755997343e-08
2820 2.63680259848798e-08
2821 2.63535149258587e-08
2822 2.63426134239353e-08
2823 2.63289621216245e-08
2824 2.63134527500597e-08
2825 2.63018566926121e-08
2826 2.62893706803879e-08
2827 2.6276694597982e-08
2828 2.62632475767077e-08
2829 2.62534580741658e-08
2830 2.6238103245646e-08
2831 2.62262744854524e-08
2832 2.62090562586081e-08
2833 2.61979771210008e-08
2834 2.618329730808e-08
2835 2.61717367777692e-08
2836 2.61559875980311e-08
2837 2.61426809089471e-08
2838 2.61278394475539e-08
2839 2.6114101103758e-08
2840 2.61003147983274e-08
2841 2.60852583977567e-08
2842 2.60729677847849e-08
2843 2.60608690183517e-08
2844 2.60449102285065e-08
2845 2.60313690603198e-08
2846 2.60181511890778e-08
2847 2.60052424039259e-08
2848 2.59910315492107e-08
2849 2.59786556711106e-08
2850 2.59661803170275e-08
2851 2.59507473288068e-08
2852 2.59389327794679e-08
2853 2.59256331958113e-08
2854 2.59087666876212e-08
2855 2.58947707720836e-08
2856 2.58836703181942e-08
2857 2.58699106581162e-08
2858 2.58556713816915e-08
2859 2.58440060463272e-08
2860 2.58316710244344e-08
2861 2.58217482951295e-08
2862 2.58092249794117e-08
2863 2.57995456109938e-08
2864 2.57902961209311e-08
2865 2.5779401724435e-08
2866 2.5769447020707e-08
2867 2.57613876897267e-08
2868 2.57512589030284e-08
2869 2.57419596749742e-08
2870 2.57334704656387e-08
2871 2.57234535894213e-08
2872 2.57153178750968e-08
2873 2.57064378672567e-08
2874 2.56956731448099e-08
2875 2.5685729099223e-08
2876 2.56782133334355e-08
2877 2.56682799459895e-08
2878 2.56600749537483e-08
2879 2.5651539559135e-08
2880 2.56430610079406e-08
2881 2.56363268391624e-08
2882 2.56254892860852e-08
2883 2.56166927670165e-08
2884 2.56109107255043e-08
2885 2.56007730570218e-08
2886 2.5594101060733e-08
2887 2.55841925422828e-08
2888 2.55746428479142e-08
2889 2.55674290627894e-08
2890 2.55613841204649e-08
2891 2.55532164317174e-08
2892 2.55435281815153e-08
2893 2.55354155598297e-08
2894 2.55272354365843e-08
2895 2.55179770647374e-08
2896 2.55120422565369e-08
2897 2.55031427087715e-08
2898 2.54929712895091e-08
2899 2.5485020316296e-08
2900 2.54784868758406e-08
2901 2.54706389313242e-08
2902 2.54640095675995e-08
2903 2.54547956046736e-08
2904 2.54469441074434e-08
2905 2.54364973528709e-08
2906 2.54312126912737e-08
2907 2.54221586004633e-08
2908 2.54131826693538e-08
2909 2.54053684756173e-08
2910 2.539696453141e-08
2911 2.53892515900134e-08
2912 2.53818033257858e-08
2913 2.53739074196346e-08
2914 2.53656224913357e-08
2915 2.53560141771914e-08
2916 2.53470933131439e-08
2917 2.53368064306869e-08
2918 2.53265319827278e-08
2919 2.53098235702964e-08
2920 2.52818068702254e-08
2921 2.52250451637792e-08
2922 2.50945060287222e-08
2923 2.50991352146457e-08
2924 2.51063863032641e-08
2925 2.50953213765115e-08
2926 2.50825493708362e-08
2927 2.50691716274787e-08
2928 2.50572380622316e-08
2929 2.50466047901909e-08
2930 2.50357388154043e-08
2931 2.50238514354351e-08
2932 2.50162948134403e-08
2933 2.50067575535695e-08
2934 2.49967957444142e-08
2935 2.49879903435613e-08
2936 2.49787834860626e-08
2937 2.49682177155819e-08
2938 2.49597427171011e-08
2939 2.49521647788242e-08
2940 2.49437661636875e-08
2941 2.49327190005033e-08
2942 2.49245619698968e-08
2943 2.49150993170133e-08
2944 2.49073632829777e-08
2945 2.4899019734903e-08
2946 2.48904470367961e-08
2947 2.48817482173536e-08
2948 2.48727278773231e-08
2949 2.48641836009256e-08
2950 2.485435679489e-08
2951 2.48475444664109e-08
2952 2.48396325730482e-08
2953 2.48299922844808e-08
2954 2.48236098343568e-08
2955 2.4814383436933e-08
2956 2.48054110585372e-08
2957 2.47971954081549e-08
2958 2.47889726523454e-08
2959 2.47795366448145e-08
2960 2.47720475243796e-08
2961 2.47636631200976e-08
2962 2.47563711752719e-08
2963 2.47478819659364e-08
2964 2.47386111595915e-08
2965 2.47311007228745e-08
2966 2.47232261330055e-08
2967 2.47145415244177e-08
2968 2.47067042380422e-08
2969 2.46990321528529e-08
2970 2.46901574740832e-08
2971 2.46836311390553e-08
2972 2.46747760002108e-08
2973 2.46679032755992e-08
2974 2.46590836638916e-08
2975 2.46506406398339e-08
2976 2.46438975892715e-08
2977 2.46356055555452e-08
2978 2.46271394388486e-08
2979 2.46199522990764e-08
2980 2.4611981785938e-08
2981 2.4605245840803e-08
2982 2.45957458844259e-08
2983 2.45890667827098e-08
2984 2.45811708765586e-08
2985 2.45732838521917e-08
2986 2.45652582719913e-08
2987 2.45581901481273e-08
2988 2.45512286056737e-08
2989 2.45423485978336e-08
2990 2.45356801542584e-08
2991 2.45276865484811e-08
2992 2.45223290562535e-08
2993 2.45143016996963e-08
2994 2.45054661007771e-08
2995 2.44983713315605e-08
2996 2.44898039625241e-08
2997 2.44832847329235e-08
2998 2.44766003021368e-08
2999 2.44692017759007e-08
3000 1.31081145937628e-08
3001 1.32254083240468e-08
3002 1.33498625487505e-08
3003 1.33990161188535e-08
3004 1.33999131790574e-08
3005 1.33954296543948e-08
3006 1.33898829801637e-08
3007 1.33839996863117e-08
3008 1.33784565647943e-08
3009 1.3372940976808e-08
3010 1.33672672930629e-08
3011 1.33619124653706e-08
3012 1.33566189219891e-08
3013 1.33516309119841e-08
3014 1.33466846463648e-08
3015 1.33417614733844e-08
3016 1.3336855175794e-08
3017 1.33321123030328e-08
3018 1.33275630531671e-08
3019 1.33230839693965e-08
3020 1.33186057738044e-08
3021 1.33143949199166e-08
3022 1.33099096188971e-08
3023 1.33059883111741e-08
3024 1.33018049908173e-08
3025 1.32980764178114e-08
3026 1.3293834477679e-08
3027 1.32896156301854e-08
3028 1.32857413959186e-08
3029 1.32818582798677e-08
3030 1.32778534833733e-08
3031 1.32740707314838e-08
3032 1.32703474875484e-08
3033 1.32667379304507e-08
3034 1.32628485971509e-08
3035 1.32593580559615e-08
3036 1.32554047738154e-08
3037 1.32520865392394e-08
3038 1.32485906689794e-08
3039 1.32450166390186e-08
3040 1.32414896825139e-08
3041 1.32379893713619e-08
3042 1.32345361336661e-08
3043 1.32311024358955e-08
3044 1.32278028530664e-08
3045 1.32245459028013e-08
3046 1.32208182179738e-08
3047 1.32176616318702e-08
3048 1.3214378924431e-08
3049 1.32110891115644e-08
3050 1.32077433434574e-08
3051 1.32046205081338e-08
3052 1.32015136600216e-08
3053 1.3198263815184e-08
3054 1.31950095294542e-08
3055 1.31920208090719e-08
3056 1.31889601462376e-08
3057 1.31856756624416e-08
3058 1.31824142712844e-08
3059 1.31796218383329e-08
3060 1.31766846322989e-08
3061 1.31733699504366e-08
3062 1.31705615302735e-08
3063 1.31675461645386e-08
3064 1.31646675782804e-08
3065 1.31614426024385e-08
3066 1.31587611917894e-08
3067 1.31559048099916e-08
3068 1.31529835911692e-08
3069 1.31502000400019e-08
3070 1.31472637221464e-08
3071 1.31444810591574e-08
3072 1.31416406645712e-08
3073 1.31388269153376e-08
3074 1.31360620159171e-08
3075 1.31332758002145e-08
3076 1.31305117889724e-08
3077 1.31278845572069e-08
3078 1.31250397217286e-08
3079 1.31223796273616e-08
3080 1.31199371367074e-08
3081 1.31168285122385e-08
3082 1.31141888459751e-08
3083 1.31113138124306e-08
3084 1.31087762866855e-08
3085 1.3106227214621e-08
3086 1.31036195227807e-08
3087 1.31009265658122e-08
3088 1.30982105162047e-08
3089 1.30957040767044e-08
3090 1.30931594455319e-08
3091 1.30905535300485e-08
3092 1.30881616655643e-08
3093 1.30854553859194e-08
3094 1.30830137834437e-08
3095 1.3080692973233e-08
3096 1.30779138629578e-08
3097 1.30753941007811e-08
3098 1.30729151948117e-08
3099 1.3070494908618e-08
3100 1.30679573828729e-08
3101 1.30654553842646e-08
3102 1.3063162107585e-08
3103 1.30606281345536e-08
3104 1.30583437396581e-08
3105 1.30557831212741e-08
3106 1.30535999787185e-08
3107 1.30508830409326e-08
3108 1.30485711125061e-08
3109 1.30461215164246e-08
3110 1.30438140288902e-08
3111 1.30414843368953e-08
3112 1.3039104906909e-08
3113 1.30368471573661e-08
3114 1.30344171012098e-08
3115 1.30321078373186e-08
3116 1.30298287714936e-08
3117 1.30273987153373e-08
3118 1.30251205376908e-08
3119 1.30229622641309e-08
3120 1.30207622461853e-08
3121 1.30183472890621e-08
3122 1.30161685873986e-08
3123 1.30138024800885e-08
3124 1.30116175611761e-08
3125 1.30095028083588e-08
3126 1.30074049309314e-08
3127 1.30049242486052e-08
3128 1.30027988376469e-08
3129 1.30005837206681e-08
3130 1.29984991659171e-08
3131 1.29962467454448e-08
3132 1.299420926415e-08
3133 1.29919044411508e-08
3134 1.29896502443216e-08
3135 1.29875781240685e-08
3136 1.29854536012886e-08
3137 1.29832100626004e-08
3138 1.29811024152104e-08
3139 1.29788588765223e-08
3140 1.29768782386463e-08
3141 1.29744908150542e-08
3142 1.29725998831987e-08
3143 1.29706201335011e-08
3144 1.29683694893856e-08
3145 1.29662867109914e-08
3146 1.29643424884307e-08
3147 1.29619923683322e-08
3148 1.29600756793025e-08
3149 1.29578614505022e-08
3150 1.29560291384223e-08
3151 1.2954185280023e-08
3152 1.29518458180655e-08
3153 1.29498411993723e-08
3154 1.29477699672975e-08
3155 1.29459403197529e-08
3156 1.29440724805363e-08
3157 1.29417161431888e-08
3158 1.29400783421829e-08
3159 1.29379209568015e-08
3160 1.29355717248814e-08
3161 1.29338904031329e-08
3162 1.29320012476342e-08
3163 1.29298820539248e-08
3164 1.29280230964923e-08
3165 1.29259394299197e-08
3166 1.29239969837158e-08
3167 1.29222845757226e-08
3168 1.29203545640166e-08
3169 1.29183908015307e-08
3170 1.29160762085689e-08
3171 1.29142323501696e-08
3172 1.29124853032181e-08
3173 1.29105490742631e-08
3174 1.29086474842666e-08
3175 1.29068000731536e-08
3176 1.29047341701494e-08
3177 1.29029489315258e-08
3178 1.29010242488903e-08
3179 1.28992025949515e-08
3180 1.28973827173695e-08
3181 1.28953319133984e-08
3182 1.28933770326967e-08
3183 1.28915038644095e-08
3184 1.28896697759728e-08
3185 1.28877966076857e-08
3186 1.28859571901785e-08
3187 1.28842305713306e-08
3188 1.28822499334547e-08
3189 1.28804247268022e-08
3190 1.28785861974734e-08
3191 1.28767174700783e-08
3192 1.28750041739067e-08
3193 1.28731461046527e-08
3194 1.28713946168091e-08
3195 1.28695605283724e-08
3196 1.28680239797063e-08
3197 1.28659909393036e-08
3198 1.28643806718287e-08
3199 1.28625758932799e-08
3200 1.28606156835076e-08
3201 1.2858747844291e-08
3202 1.28568737878254e-08
3203 1.28554038525408e-08
3204 1.28535919685646e-08
3205 1.28518697906088e-08
3206 1.28499628715417e-08
3207 1.28482255945528e-08
3208 1.28462751547431e-08
3209 1.28446071556709e-08
3210 1.28430368562249e-08
3211 1.28411592470457e-08
3212 1.28394290754841e-08
3213 1.28376775876404e-08
3214 1.28358932371953e-08
3215 1.28342447780483e-08
3216 1.28324773029931e-08
3217 1.28308910163355e-08
3218 1.2829038276152e-08
3219 1.28273107691257e-08
3220 1.28256907316882e-08
3221 1.28239836527655e-08
3222 1.28221238071546e-08
3223 1.28203234694979e-08
3224 1.28187336301266e-08
3225 1.28170487556645e-08
3226 1.28152644052193e-08
3227 1.28136390387112e-08
3228 1.28118937681165e-08
3229 1.28103447849526e-08
3230 1.28086261597105e-08
3231 1.28069013172194e-08
3232 1.28051436121268e-08
3233 1.28036621305228e-08
3234 1.28020349876579e-08
3235 1.28002275445738e-08
3236 1.27984236542034e-08
3237 1.27969714824872e-08
3238 1.2795252857245e-08
3239 1.27938513116987e-08
3240 1.27917791914456e-08
3241 1.27904371538534e-08
3242 1.2788561321031e-08
3243 1.27870958266385e-08
3244 1.27853549969359e-08
3245 1.27838761798671e-08
3246 1.27820802831025e-08
3247 1.27804389293829e-08
3248 1.27785275694237e-08
3249 1.27771420110889e-08
3250 1.27754979928341e-08
3251 1.27738690736123e-08
3252 1.27721406784076e-08
3253 1.27704531394102e-08
3254 1.27691368589922e-08
3255 1.2767507051592e-08
3256 1.27658328352709e-08
3257 1.27641390790245e-08
3258 1.27627099999472e-08
3259 1.2760986933813e-08
3260 1.27591803789073e-08
3261 1.27578516639915e-08
3262 1.27563897223126e-08
3263 1.27548247519371e-08
3264 1.27532704397026e-08
3265 1.27517125747545e-08
3266 1.27501404989516e-08
3267 1.27484449663484e-08
3268 1.27470816124742e-08
3269 1.27453834153357e-08
3270 1.27436754482346e-08
3271 1.27420571871539e-08
3272 1.27406414307529e-08
3273 1.27390844539832e-08
3274 1.27375816560971e-08
3275 1.27362280721854e-08
3276 1.27345254341549e-08
3277 1.2733042176194e-08
3278 1.27313919406902e-08
3279 1.27297754559663e-08
3280 1.27283099615738e-08
3281 1.27267876237624e-08
3282 1.27252191006733e-08
3283 1.27238246605543e-08
3284 1.27223183099545e-08
3285 1.27205197486546e-08
3286 1.27191004395399e-08
3287 1.27174288877541e-08
3288 1.27159314189385e-08
3289 1.27146524420141e-08
3290 1.27128361171458e-08
3291 1.27113528591849e-08
3292 1.27099424318544e-08
3293 1.27084289758272e-08
3294 1.27068489064186e-08
3295 1.27052564025121e-08
3296 1.27038539687874e-08
3297 1.27022286022793e-08
3298 1.27008590311561e-08
3299 1.26991883675487e-08
3300 1.26977131031936e-08
3301 1.26962849122947e-08
3302 1.26947963252633e-08
3303 1.26932260258172e-08
3304 1.26919035281503e-08
3305 1.26902079955471e-08
3306 1.26887220730509e-08
3307 1.26873462846788e-08
3308 1.26857768734112e-08
3309 1.26842927272719e-08
3310 1.26827588431411e-08
3311 1.26814132528352e-08
3312 1.26800028255047e-08
3313 1.2678589733639e-08
3314 1.26771082520349e-08
3315 1.26754642337801e-08
3316 1.26741008799058e-08
3317 1.26726247273723e-08
3318 1.26710588688184e-08
3319 1.26696022562101e-08
3320 1.26681074519297e-08
3321 1.26664794208864e-08
3322 1.26651800158584e-08
3323 1.26639418951413e-08
3324 1.26623831420147e-08
3325 1.26606352068848e-08
3326 1.26596333416273e-08
3327 1.26578587611448e-08
3328 1.26563612923292e-08
3329 1.26549952739197e-08
3330 1.2653575964805e-08
3331 1.26522623489222e-08
3332 1.26505144137923e-08
3333 1.26491856988764e-08
3334 1.26478054696122e-08
3335 1.264646343202e-08
3336 1.26448815862545e-08
3337 1.26436781044958e-08
3338 1.26420438562036e-08
3339 1.26405161893217e-08
3340 1.26391510590906e-08
3341 1.26376686893082e-08
3342 1.26364652075495e-08
3343 1.26350263585095e-08
3344 1.26334018801799e-08
3345 1.26323387306115e-08
3346 1.26306254344399e-08
3347 1.26293189239846e-08
3348 1.26277805989616e-08
3349 1.26264554367594e-08
3350 1.26251036292047e-08
3351 1.26235590869328e-08
3352 1.26222383656227e-08
3353 1.26207506667697e-08
3354 1.26194299454596e-08
3355 1.26179209303245e-08
3356 1.26166561642549e-08
3357 1.26152226442855e-08
3358 1.26137198463994e-08
3359 1.26124142241224e-08
3360 1.26109931386509e-08
3361 1.26095764940715e-08
3362 1.26083534723875e-08
3363 1.26066579397843e-08
3364 1.26052315252423e-08
3365 1.260395698921e-08
3366 1.26024772839628e-08
3367 1.26009389589399e-08
3368 1.25998491640189e-08
3369 1.25984565002568e-08
3370 1.259704962564e-08
3371 1.25956152174922e-08
3372 1.25942936080037e-08
3373 1.25928112382212e-08
3374 1.25913643955755e-08
3375 1.25901644665305e-08
3376 1.25889902946597e-08
3377 1.25874493051015e-08
3378 1.25859909161363e-08
3379 1.25845840415195e-08
3380 1.25831940422927e-08
3381 1.25816876916929e-08
3382 1.25805037498594e-08
3383 1.25792265492919e-08
3384 1.25779324733344e-08
3385 1.25765913239206e-08
3386 1.25752395163659e-08
3387 1.25737420475502e-08
3388 1.2572461294269e-08
3389 1.25712640297593e-08
3390 1.2569680407637e-08
3391 1.25682353413481e-08
3392 1.25670256423405e-08
3393 1.25658266014739e-08
3394 1.25642953818783e-08
3395 1.2562893836332e-08
3396 1.25618049295895e-08
3397 1.25604122658274e-08
3398 1.25589609822896e-08
3399 1.25577770404561e-08
3400 1.25563435204867e-08
3401 1.2554957962152e-08
3402 1.25537216177918e-08
3403 1.25522054972294e-08
3404 1.25507426673721e-08
3405 1.25495400737918e-08
3406 1.25482202406602e-08
3407 1.25469146183832e-08
3408 1.25455539290442e-08
3409 1.2544290051153e-08
3410 1.25429133746024e-08
3411 1.25417374263748e-08
3412 1.25404993056577e-08
3413 1.25390107186263e-08
3414 1.25375327897359e-08
3415 1.25363950331803e-08
3416 1.2535004145775e-08
3417 1.25337002998549e-08
3418 1.25325030353451e-08
3419 1.25310624099484e-08
3420 1.25300001485584e-08
3421 1.25284254082203e-08
3422 1.25271917283953e-08
3423 1.25260202210598e-08
3424 1.25247021642849e-08
3425 1.25232837433487e-08
3426 1.25222507918465e-08
3427 1.25206947032552e-08
3428 1.25194485889324e-08
3429 1.2518169612008e-08
3430 1.25168462261627e-08
3431 1.2515680936076e-08
3432 1.25140982021321e-08
3433 1.2513027058958e-08
3434 1.25117116667184e-08
3435 1.25104024917277e-08
3436 1.25092780578484e-08
3437 1.25079697710362e-08
3438 1.25064785194695e-08
3439 1.25052572741424e-08
3440 1.25039001375171e-08
3441 1.25025829689207e-08
3442 1.25014558705061e-08
3443 1.25002612705316e-08
3444 1.24988224214917e-08
3445 1.24974421922275e-08
3446 1.2496258250394e-08
3447 1.24949792734697e-08
3448 1.24937598044994e-08
3449 1.24923840161273e-08
3450 1.24913608345878e-08
3451 1.24899965925351e-08
3452 1.24887042929345e-08
3453 1.24873871243381e-08
3454 1.24862289396788e-08
3455 1.24849899307833e-08
3456 1.24837482573525e-08
3457 1.24824293123993e-08
3458 1.24809869106457e-08
3459 1.24798082978828e-08
3460 1.24785684008089e-08
3461 1.2477455513249e-08
3462 1.24760548558811e-08
3463 1.24749792718148e-08
3464 1.24735715090196e-08
3465 1.24722063787885e-08
3466 1.24708865456569e-08
3467 1.24697034920018e-08
3468 1.24683108282397e-08
3469 1.24670158641038e-08
3470 1.24661134748294e-08
3471 1.24645325172423e-08
3472 1.24633103837368e-08
3473 1.24621930552848e-08
3474 1.24608590112985e-08
3475 1.24596670758592e-08
3476 1.24582717475619e-08
3477 1.24572085979935e-08
3478 1.24559909053801e-08
3479 1.2454807851725e-08
3480 1.24536132517505e-08
3481 1.2452452402556e-08
3482 1.24510437515823e-08
3483 1.24498118481142e-08
3484 1.24486509989197e-08
3485 1.24475336704677e-08
3486 1.24461045913904e-08
3487 1.24448193972171e-08
3488 1.24437606885408e-08
3489 1.24422490088705e-08
3490 1.24409949009419e-08
3491 1.2439755003868e-08
3492 1.24387291577932e-08
3493 1.24373400467448e-08
3494 1.24360903797083e-08
3495 1.24349277541569e-08
3496 1.24337038442945e-08
3497 1.24324133210507e-08
3498 1.24313208615945e-08
3499 1.24298988879445e-08
3500 1.2428871265513e-08
3501 1.24276446911153e-08
3502 1.24264731837798e-08
3503 1.24251622324323e-08
3504 1.24239907250967e-08
3505 1.24226611220024e-08
3506 1.24215295826957e-08
3507 1.24202692575182e-08
3508 1.24191839034893e-08
3509 1.24179901916932e-08
3510 1.24168098025734e-08
3511 1.24155148384375e-08
3512 1.24141559254554e-08
3513 1.2413059025107e-08
3514 1.24118191280331e-08
3515 1.24105143939346e-08
3516 1.24092789377528e-08
3517 1.24081953600808e-08
3518 1.2407053162633e-08
3519 1.24057946138123e-08
3520 1.24047181415676e-08
3521 1.24034462700706e-08
3522 1.24022490055609e-08
3523 1.24011769742083e-08
3524 1.23998340484377e-08
3525 1.23987273781268e-08
3526 1.23974404075966e-08
3527 1.23961765297054e-08
3528 1.23950387731497e-08
3529 1.23938557194947e-08
3530 1.23926175987776e-08
3531 1.23914061234132e-08
3532 1.23901884307998e-08
3533 1.23891004122356e-08
3534 1.2387901371369e-08
3535 1.23866428225483e-08
3536 1.23853975964039e-08
3537 1.23843815202918e-08
3538 1.23832331055951e-08
3539 1.23820775854711e-08
3540 1.23807115670616e-08
3541 1.23795445006181e-08
3542 1.23782086802748e-08
3543 1.23772849747183e-08
3544 1.23759651415867e-08
3545 1.23749401836903e-08
3546 1.23736478840897e-08
3547 1.23724808176462e-08
3548 1.23713697064431e-08
3549 1.237034297219e-08
3550 1.23690870879045e-08
3551 1.23679075869632e-08
3552 1.23667680540507e-08
3553 1.23655405914747e-08
3554 1.23645556016072e-08
3555 1.23632535320439e-08
3556 1.236204738575e-08
3557 1.23606369584195e-08
3558 1.23596812784399e-08
3559 1.23585248701374e-08
3560 1.23574457333575e-08
3561 1.23562369225283e-08
3562 1.23550973896158e-08
3563 1.23539685148444e-08
3564 1.23528378637161e-08
3565 1.23515100369787e-08
3566 1.23502816862242e-08
3567 1.23491750159133e-08
3568 1.23480630165318e-08
3569 1.23470043078555e-08
3570 1.23456800338317e-08
3571 1.23445351718487e-08
3572 1.23436567633917e-08
3573 1.23423067321937e-08
3574 1.23411938446338e-08
3575 1.23400827334308e-08
3576 1.23389982675803e-08
3577 1.23376526772745e-08
3578 1.23366517001955e-08
3579 1.23354793046815e-08
3580 1.23342882574207e-08
3581 1.23331984624997e-08
3582 1.23319896516705e-08
3583 1.23308927513222e-08
3584 1.23297025922398e-08
3585 1.23284786823774e-08
3586 1.23273427021786e-08
3587 1.23262529072576e-08
3588 1.23252874573154e-08
3589 1.23240910809841e-08
3590 1.23229115800427e-08
3591 1.23215802005916e-08
3592 1.23206218560767e-08
3593 1.23195142975874e-08
3594 1.23182912759034e-08
3595 1.23171401966715e-08
3596 1.23159527021244e-08
3597 1.23146000063912e-08
3598 1.23136647545152e-08
3599 1.23124337392255e-08
3600 1.23114212158271e-08
3601 1.23102461557778e-08
3602 1.23090444503759e-08
3603 1.23078267577625e-08
3604 1.23066525858917e-08
3605 1.23056898004847e-08
3606 1.23045635902486e-08
3607 1.23037100507872e-08
3608 1.23022489972868e-08
3609 1.23011547614738e-08
3610 1.22999459506445e-08
3611 1.2298901452823e-08
3612 1.22977974470473e-08
3613 1.22967058757695e-08
3614 1.22954073589199e-08
3615 1.22942127589454e-08
3616 1.22931229640244e-08
3617 1.22920509326718e-08
3618 1.22910224220618e-08
3619 1.2289670614507e-08
3620 1.22886687492496e-08
3621 1.22876038233244e-08
3622 1.22864936002998e-08
3623 1.22853123230016e-08
3624 1.22842340744e-08
3625 1.22830865478818e-08
3626 1.22821894876779e-08
3627 1.22808518909778e-08
3628 1.22799024282472e-08
3629 1.22787255918411e-08
3630 1.22775851707502e-08
3631 1.22763914589541e-08
3632 1.22753318620994e-08
3633 1.22741008468097e-08
3634 1.22731638185769e-08
3635 1.22718661899057e-08
3636 1.22709691297018e-08
3637 1.22698979865277e-08
3638 1.2268793980752e-08
3639 1.2267837412594e-08
3640 1.22666321544784e-08
3641 1.22655254841675e-08
3642 1.22643557531887e-08
3643 1.22632517474131e-08
3644 1.2262212578662e-08
3645 1.22611307773468e-08
3646 1.22602781260639e-08
3647 1.22589440820775e-08
3648 1.22578018846298e-08
3649 1.22566907734267e-08
3650 1.22557279880198e-08
3651 1.22544081548881e-08
3652 1.22531771395984e-08
3653 1.22523369228134e-08
3654 1.22512817668508e-08
3655 1.22500765087352e-08
3656 1.22490186882374e-08
3657 1.22478693853623e-08
3658 1.22468337693249e-08
3659 1.22457297635492e-08
3660 1.22447563200012e-08
3661 1.22435519500641e-08
3662 1.22423431392349e-08
3663 1.2241545555014e-08
3664 1.22403367441848e-08
3665 1.22392229684465e-08
3666 1.22380079403683e-08
3667 1.22370362731772e-08
3668 1.22359917753556e-08
3669 1.22350467535171e-08
3670 1.22337846519827e-08
3671 1.22327810103684e-08
3672 1.22316263784228e-08
3673 1.22305339189666e-08
3674 1.22296119897669e-08
3675 1.22286687442852e-08
3676 1.22274235181408e-08
3677 1.22262688861952e-08
3678 1.22253274170703e-08
3679 1.22242420630414e-08
3680 1.22230146004654e-08
3681 1.22221859299998e-08
3682 1.22211503139624e-08
3683 1.22199699248426e-08
3684 1.22188641427101e-08
3685 1.22177015171587e-08
3686 1.22166818883329e-08
3687 1.22155112691757e-08
3688 1.22146595060713e-08
3689 1.22135883628971e-08
3690 1.22125358714698e-08
3691 1.22113696932047e-08
3692 1.22104406585777e-08
3693 1.22091607934749e-08
3694 1.22080745512676e-08
3695 1.2207063804226e-08
3696 1.2206073485288e-08
3697 1.22049970130433e-08
3698 1.22037331351521e-08
3699 1.22028476212677e-08
3700 1.22017516090978e-08
3701 1.22006520442142e-08
3702 1.21996750479525e-08
3703 1.21985568313221e-08
3704 1.21976571065829e-08
3705 1.21964012222975e-08
3706 1.21954659704215e-08
3707 1.21943815045711e-08
3708 1.21933902974547e-08
3709 1.21922738571811e-08
3710 1.21911725159407e-08
3711 1.21903926952882e-08
3712 1.21892744786578e-08
3713 1.21882983705746e-08
3714 1.21872441027904e-08
3715 1.21861303270521e-08
3716 1.21849410561481e-08
3717 1.21840830757947e-08
3718 1.21830332489026e-08
3719 1.21818972687038e-08
3720 1.21810463937777e-08
3721 1.21798420238406e-08
3722 1.21788854556826e-08
3723 1.21778533923589e-08
3724 1.21766756677744e-08
3725 1.21756489335212e-08
3726 1.21746790426869e-08
3727 1.21737802061261e-08
3728 1.21727561364082e-08
3729 1.21716130507821e-08
3730 1.21704450961602e-08
3731 1.21696555055451e-08
3732 1.21683498832681e-08
3733 1.21676446696029e-08
3734 1.2166309737438e-08
3735 1.21653132012511e-08
3736 1.21641612338408e-08
3737 1.21633192406989e-08
3738 1.2162062468235e-08
3739 1.21615020276522e-08
3740 1.21601360092427e-08
3741 1.21591074986327e-08
3742 1.21581660295078e-08
3743 1.21570993272258e-08
3744 1.21561321009267e-08
3745 1.21550405296489e-08
3746 1.21539356356948e-08
3747 1.21528334062759e-08
3748 1.21519541096404e-08
3749 1.21508936246073e-08
3750 1.21497576444085e-08
3751 1.21488019644289e-08
3752 1.21477494730016e-08
3753 1.21468772817934e-08
3754 1.21458212376524e-08
3755 1.21446444012463e-08
3756 1.214358569257e-08
3757 1.21425367538563e-08
3758 1.2141611271943e-08
3759 1.21406422692871e-08
3760 1.21396235286397e-08
3761 1.21387220275437e-08
3762 1.2137624239017e-08
3763 1.21365379968097e-08
3764 1.21357057736304e-08
3765 1.21344321257766e-08
3766 1.21335341773943e-08
3767 1.21326504398667e-08
3768 1.21315757439788e-08
3769 1.21304450928506e-08
3770 1.21296670485549e-08
3771 1.21284875476135e-08
3772 1.21274696951446e-08
3773 1.21263497021573e-08
3774 1.21253886931072e-08
3775 1.21245022910443e-08
3776 1.21234018379823e-08
3777 1.21225740556952e-08
3778 1.21214256409985e-08
3779 1.21205072645125e-08
3780 1.2119462766691e-08
3781 1.21184680068609e-08
3782 1.21173204803426e-08
3783 1.21166401356732e-08
3784 1.21154872800844e-08
3785 1.21143397535661e-08
3786 1.21131842334421e-08
3787 1.21123751029018e-08
3788 1.21113030715492e-08
3789 1.21102301520182e-08
3790 1.21092842420012e-08
3791 1.21083001403122e-08
3792 1.21073489012247e-08
3793 1.21063132851873e-08
3794 1.21053957968797e-08
3795 1.21043308709545e-08
3796 1.21033592037634e-08
3797 1.21024879007336e-08
3798 1.21014931409036e-08
3799 1.21003091990701e-08
3800 1.20995578001271e-08
3801 1.20984049445383e-08
3802 1.20974199546708e-08
3803 1.20964962491144e-08
3804 1.20956542559725e-08
3805 1.20944374515375e-08
3806 1.20935546021883e-08
3807 1.20925323088272e-08
3808 1.20916157086981e-08
3809 1.20905365719182e-08
3810 1.20898580036055e-08
3811 1.20886998189462e-08
3812 1.20875647269258e-08
3813 1.2086672107614e-08
3814 1.20857670538044e-08
3815 1.2084693246095e-08
3816 1.20836105566013e-08
3817 1.20827197136464e-08
3818 1.20819789728444e-08
3819 1.20807586156957e-08
3820 1.20795986546796e-08
3821 1.20789049873338e-08
3822 1.20776464385131e-08
3823 1.20768142153338e-08
3824 1.20757928101511e-08
3825 1.20748433474205e-08
3826 1.20738992137603e-08
3827 1.20730305752659e-08
3828 1.20720988761036e-08
3829 1.20710508255684e-08
3830 1.20700214267799e-08
3831 1.20690542004809e-08
3832 1.20679386483857e-08
3833 1.20671410641648e-08
3834 1.20660574864928e-08
3835 1.20650085477791e-08
3836 1.20642402734461e-08
3837 1.20631824529482e-08
3838 1.20621885812966e-08
3839 1.20612213549975e-08
3840 1.20601688635702e-08
3841 1.20593908192745e-08
3842 1.20583445450961e-08
3843 1.20574181750044e-08
3844 1.20562866356977e-08
3845 1.20552368088056e-08
3846 1.20544934034683e-08
3847 1.20533867331574e-08
3848 1.20524408231404e-08
3849 1.20516663315584e-08
3850 1.20508412138065e-08
3851 1.2049773623346e-08
3852 1.20487717580886e-08
3853 1.20476189024998e-08
3854 1.20468834907683e-08
3855 1.20458194530215e-08
3856 1.20449819007717e-08
3857 1.20439374029502e-08
3858 1.20432162020734e-08
3859 1.20422241067786e-08
3860 1.20411067783266e-08
3861 1.20401573155959e-08
3862 1.20391359104133e-08
3863 1.2038237962031e-08
3864 1.20372494194498e-08
3865 1.20362004807362e-08
3866 1.20354757271457e-08
3867 1.20345662324439e-08
3868 1.20336594022774e-08
3869 1.20326060226716e-08
3870 1.20318262020191e-08
3871 1.2030771934235e-08
3872 1.20298420114295e-08
3873 1.20287948490727e-08
3874 1.20276224535587e-08
3875 1.20269270098561e-08
3876 1.2025949125416e-08
3877 1.20249028512376e-08
3878 1.20238095036029e-08
3879 1.20231931077797e-08
3880 1.20221068655724e-08
3881 1.20211636200906e-08
3882 1.20200933650949e-08
3883 1.20191483432563e-08
3884 1.20183463181434e-08
3885 1.20174670215079e-08
3886 1.20164616035368e-08
3887 1.20154890481672e-08
3888 1.20146417259548e-08
3889 1.20134799885818e-08
3890 1.20125260849591e-08
3891 1.20117631396965e-08
3892 1.20107879197917e-08
3893 1.20097505273975e-08
3894 1.20089334032514e-08
3895 1.20080470011885e-08
3896 1.20069643116949e-08
3897 1.20061516284409e-08
3898 1.20053762486805e-08
3899 1.20044312268419e-08
3900 1.20033565309541e-08
3901 1.20024141736508e-08
3902 1.20014851390238e-08
3903 1.20007443982217e-08
3904 1.19997141112549e-08
3905 1.19986394153671e-08
3906 1.19977539014826e-08
3907 1.19969678635812e-08
3908 1.19958425415234e-08
3909 1.19950245291989e-08
3910 1.19940812837172e-08
3911 1.19929621789083e-08
3912 1.19920544605634e-08
3913 1.1991289738944e-08
3914 1.19902034967367e-08
3915 1.19893002192839e-08
3916 1.19882992422049e-08
3917 1.19873897475031e-08
3918 1.19864296266314e-08
3919 1.19856364833026e-08
3920 1.19845866564106e-08
3921 1.19836833789577e-08
3922 1.19828165168201e-08
3923 1.19817791244259e-08
3924 1.19811831567063e-08
3925 1.19801493170257e-08
3926 1.19791723207641e-08
3927 1.19780398932789e-08
3928 1.19772165518839e-08
3929 1.19764766992603e-08
3930 1.19753869043393e-08
3931 1.19743077675594e-08
3932 1.19736025538941e-08
3933 1.19726841774082e-08
3934 1.19716991875407e-08
3935 1.1970821667262e-08
3936 1.19698579936767e-08
3937 1.19689405053691e-08
3938 1.196804078063e-08
3939 1.19670771070446e-08
3940 1.19663114972468e-08
3941 1.19651852870106e-08
3942 1.19643219775867e-08
3943 1.19635457096479e-08
3944 1.19626362149461e-08
3945 1.1961453161291e-08
3946 1.19608092319368e-08
3947 1.19598357883888e-08
3948 1.19588756675171e-08
3949 1.19578720259028e-08
3950 1.19570184864415e-08
3951 1.1956084122744e-08
3952 1.19550866983786e-08
3953 1.19542553633778e-08
3954 1.19533147824313e-08
3955 1.19523804187338e-08
3956 1.19513918761527e-08
3957 1.19505774165418e-08
3958 1.19495728867491e-08
3959 1.19488987593286e-08
3960 1.19478658078265e-08
3961 1.19469723003363e-08
3962 1.19460361602819e-08
3963 1.19451444291485e-08
3964 1.19442802315461e-08
3965 1.19432437273304e-08
3966 1.19424221622921e-08
3967 1.19416085908597e-08
3968 1.19405241250092e-08
3969 1.19395568987102e-08
3970 1.1938762867203e-08
3971 1.19378196217212e-08
3972 1.19369358841936e-08
3973 1.19361089900849e-08
3974 1.19352518979099e-08
3975 1.19341043713916e-08
3976 1.19335119563857e-08
3977 1.19325864744724e-08
3978 1.19318039892846e-08
3979 1.19307159707205e-08
3980 1.19299299328191e-08
3981 1.19288774413917e-08
3982 1.19281695631912e-08
3983 1.19271348353323e-08
3984 1.19261436282159e-08
3985 1.19253034114308e-08
3986 1.19243441787376e-08
3987 1.19234293549653e-08
3988 1.19225642691845e-08
3989 1.19217240523994e-08
3990 1.19210019633442e-08
3991 1.19200445070078e-08
3992 1.1919107478775e-08
3993 1.19182832492015e-08
3994 1.19173817481055e-08
3995 1.19163505729603e-08
3996 1.19155476596688e-08
3997 1.19147358645932e-08
3998 1.19138192644641e-08
3999 1.19128849007666e-08
4000 1.19120233676995e-08
4001 1.19110810103962e-08
4002 1.19101599693749e-08
4003 1.19093606087972e-08
4004 1.19082148586358e-08
4005 1.1907665076194e-08
4006 1.19066694281855e-08
4007 1.19057537162348e-08
4008 1.19048486624251e-08
4009 1.19039853530012e-08
4010 1.19030447720547e-08
4011 1.19021823508092e-08
4012 1.19013661148415e-08
4013 1.19004788246002e-08
4014 1.18996652531678e-08
4015 1.18987246722213e-08
4016 1.18978054075569e-08
4017 1.18970389095807e-08
4018 1.18962004691525e-08
4019 1.18952421246377e-08
4020 1.18943432880769e-08
4021 1.18933760617779e-08
4022 1.18925944647685e-08
4023 1.18918190850081e-08
4024 1.18908456414601e-08
4025 1.18899210477252e-08
4026 1.18890985945086e-08
4027 1.18882725885783e-08
4028 1.18873915155859e-08
4029 1.18865779441535e-08
4030 1.18854863728757e-08
4031 1.18846719132648e-08
4032 1.18838840990065e-08
4033 1.18829488471306e-08
4034 1.18821130712377e-08
4035 1.18813767713277e-08
4036 1.18804548421281e-08
4037 1.187954890014e-08
4038 1.18785719038783e-08
4039 1.18777334634501e-08
4040 1.18768239687483e-08
4041 1.18760983269794e-08
4042 1.18751639632819e-08
4043 1.18742438104391e-08
4044 1.18736647181095e-08
4045 1.1872586469508e-08
4046 1.18717666808266e-08
4047 1.18709060359379e-08
4048 1.1870080918186e-08
4049 1.18691332318122e-08
4050 1.1868305449525e-08
4051 1.18672645044171e-08
4052 1.18663701087485e-08
4053 1.18655982817018e-08
4054 1.18647358604562e-08
4055 1.18638228130408e-08
4056 1.18629088774469e-08
4057 1.18619807309983e-08
4058 1.18612613064784e-08
4059 1.18603002974282e-08
4060 1.18594059017596e-08
4061 1.1858539039622e-08
4062 1.18577583307911e-08
4063 1.18569882801012e-08
4064 1.18560281592295e-08
4065 1.18550671501794e-08
4066 1.18544143390409e-08
4067 1.18533582948999e-08
4068 1.18526948256203e-08
4069 1.18516911840061e-08
4070 1.1851011727515e-08
4071 1.18500782519959e-08
4072 1.18491563227963e-08
4073 1.18482876843018e-08
4074 1.18474803301183e-08
4075 1.18466561005448e-08
4076 1.18458753917139e-08
4077 1.18450218522526e-08
4078 1.18440572904888e-08
4079 1.18430785178703e-08
4080 1.18422773809357e-08
4081 1.18416343397598e-08
4082 1.18406724425313e-08
4083 1.18396625836681e-08
4084 1.18388721048746e-08
4085 1.18379395175339e-08
4086 1.1837186342234e-08
4087 1.18361320744498e-08
4088 1.18353540301541e-08
4089 1.1834564439539e-08
4090 1.18337268872892e-08
4091 1.18328475906537e-08
4092 1.18319540831635e-08
4093 1.18311369590174e-08
4094 1.18303660201491e-08
4095 1.18293090878296e-08
4096 1.18286838102222e-08
4097 1.18277876381967e-08
4098 1.18269385396275e-08
4099 1.18260707893114e-08
4100 1.18254153136377e-08
4101 1.18245484515e-08
4102 1.18234400048323e-08
4103 1.18227880818722e-08
4104 1.18219887212945e-08
4105 1.18209433352945e-08
4106 1.18202425625213e-08
4107 1.18192930997907e-08
4108 1.18184395603294e-08
4109 1.18177521102325e-08
4110 1.18170069285384e-08
4111 1.18161711526454e-08
4112 1.18151302075375e-08
4113 1.18143521632419e-08
4114 1.18135465854152e-08
4115 1.18126335379998e-08
4116 1.18118190783889e-08
4117 1.18110392577364e-08
4118 1.18099423573881e-08
4119 1.18092744472165e-08
4120 1.18083764988341e-08
4121 1.18076188826421e-08
4122 1.18068603782717e-08
4123 1.18059046982921e-08
4124 1.18048388841885e-08
4125 1.18040022201171e-08
4126 1.18033511853355e-08
4127 1.18024221507085e-08
4128 1.1801758681429e-08
4129 1.18007426053168e-08
4130 1.17999245929923e-08
4131 1.17990159864689e-08
4132 1.17983125491605e-08
4133 1.17974181534919e-08
4134 1.17965504031758e-08
4135 1.17957368317434e-08
4136 1.17949374711657e-08
4137 1.17940839317043e-08
4138 1.17933343091181e-08
4139 1.17924212617027e-08
4140 1.17915783803824e-08
4141 1.17906786556432e-08
4142 1.17899237039865e-08
4143 1.17890648354546e-08
4144 1.17882370531674e-08
4145 1.1787434139876e-08
4146 1.17866729709704e-08
4147 1.17856195913646e-08
4148 1.17850280645371e-08
4149 1.17841265634411e-08
4150 1.17832570367682e-08
4151 1.1782308462216e-08
4152 1.17816387756875e-08
4153 1.17805640797997e-08
4154 1.17798641952049e-08
4155 1.1779011543922e-08
4156 1.17782965602942e-08
4157 1.17773719665593e-08
4158 1.17765477369858e-08
4159 1.17757847917233e-08
4160 1.17748122363537e-08
4161 1.17741869587462e-08
4162 1.17733387483554e-08
4163 1.17724336945457e-08
4164 1.17716023595449e-08
4165 1.17708180980003e-08
4166 1.17701031143724e-08
4167 1.17693392809315e-08
4168 1.17685123868227e-08
4169 1.17677343425271e-08
4170 1.17668124133274e-08
4171 1.17661027587701e-08
4172 1.17653389253292e-08
4173 1.17644320951626e-08
4174 1.17637153351779e-08
4175 1.176272501624e-08
4176 1.17620704287447e-08
4177 1.17612133365697e-08
4178 1.17605427618628e-08
4179 1.17596696824762e-08
4180 1.17589218362468e-08
4181 1.1758073625856e-08
4182 1.1757017581715e-08
4183 1.17563052626224e-08
4184 1.17555103429368e-08
4185 1.17547767075621e-08
4186 1.17539240562792e-08
4187 1.17532721333191e-08
4188 1.17524781018119e-08
4189 1.17515126518697e-08
4190 1.1750812767275e-08
4191 1.17500951191118e-08
4192 1.1749126116456e-08
4193 1.17481926409368e-08
4194 1.17476224303914e-08
4195 1.17466472104866e-08
4196 1.17460476900533e-08
4197 1.17452589876166e-08
4198 1.17444693970015e-08
4199 1.17433804902589e-08
4200 1.17427649826141e-08
4201 1.17422596090933e-08
4202 1.17412257694127e-08
4203 1.17404557187228e-08
4204 1.17395959620126e-08
4205 1.17389626907993e-08
4206 1.17379741482182e-08
4207 1.17373115671171e-08
4208 1.17365299701078e-08
4209 1.17355352102777e-08
4210 1.17349223671681e-08
4211 1.17340466232463e-08
4212 1.17333822657884e-08
4213 1.17325216208997e-08
4214 1.17316503178699e-08
4215 1.17308056601928e-08
4216 1.1729951232553e-08
4217 1.17293295076593e-08
4218 1.17283995848538e-08
4219 1.1727650850446e-08
4220 1.17269953747723e-08
4221 1.17261382825973e-08
4222 1.17252971776338e-08
4223 1.17243832420399e-08
4224 1.17237393126857e-08
4225 1.17230811724767e-08
4226 1.17220970707876e-08
4227 1.17214895567486e-08
4228 1.17207310523781e-08
4229 1.17198988291989e-08
4230 1.17190115389576e-08
4231 1.17181313541437e-08
4232 1.17173692970596e-08
4233 1.17165726010171e-08
4234 1.17157883394725e-08
4235 1.17150280587452e-08
4236 1.17142571198769e-08
4237 1.17137801680656e-08
4238 1.17126921495014e-08
4239 1.17120029230477e-08
4240 1.17113208020214e-08
4241 1.17103793328965e-08
4242 1.17094502982695e-08
4243 1.17089848927776e-08
4244 1.17080762862543e-08
4245 1.17071712324446e-08
4246 1.17065006577377e-08
4247 1.17057874504667e-08
4248 1.17050173997768e-08
4249 1.17044196557003e-08
4250 1.17034488766876e-08
4251 1.17027498802713e-08
4252 1.17017382450513e-08
4253 1.1701011715104e-08
4254 1.17004885780148e-08
4255 1.16995657606367e-08
4256 1.16987042275696e-08
4257 1.16979670394812e-08
4258 1.16971206054473e-08
4259 1.16962119989239e-08
4260 1.16954508300182e-08
4261 1.16947793671329e-08
4262 1.16938974059622e-08
4263 1.16932978855289e-08
4264 1.16923599691177e-08
4265 1.16915872538925e-08
4266 1.16907177272196e-08
4267 1.1690005408127e-08
4268 1.16890808143921e-08
4269 1.1688164214263e-08
4270 1.16877956202188e-08
4271 1.16868026367456e-08
4272 1.16862732824075e-08
4273 1.16850937814661e-08
4274 1.16844507402902e-08
4275 1.16838494435001e-08
4276 1.16830056740014e-08
4277 1.16822098661373e-08
4278 1.1681328793145e-08
4279 1.16805241034967e-08
4280 1.16799050431382e-08
4281 1.16790594972827e-08
4282 1.16782636894186e-08
4283 1.1677589561998e-08
4284 1.16769554026064e-08
4285 1.16760299206931e-08
4286 1.16751577294849e-08
4287 1.16744018896497e-08
4288 1.1673685129665e-08
4289 1.16728884336226e-08
4290 1.16721450282853e-08
4291 1.16714922171468e-08
4292 1.16705516362003e-08
4293 1.16698002372573e-08
4294 1.16692220331061e-08
4295 1.16683027684417e-08
4296 1.16674900851876e-08
4297 1.16666960536804e-08
4298 1.16660112681188e-08
4299 1.16651319714833e-08
4300 1.16644658376686e-08
4301 1.16636300617756e-08
4302 1.16628191548784e-08
4303 1.16621041712506e-08
4304 1.16614895517841e-08
4305 1.1660678644887e-08
4306 1.16598721788819e-08
4307 1.16591643006814e-08
4308 1.16582761222617e-08
4309 1.16576046593764e-08
4310 1.16567857588734e-08
4311 1.16560538998556e-08
4312 1.165519147861e-08
4313 1.16545102457621e-08
4314 1.16537934857774e-08
4315 1.16530260996228e-08
4316 1.16522853588208e-08
4317 1.16516147841139e-08
4318 1.16507994363246e-08
4319 1.16498481972371e-08
4320 1.16493810153884e-08
4321 1.1648544351317e-08
4322 1.16476037703706e-08
4323 1.16470078026509e-08
4324 1.16461595922601e-08
4325 1.16453913179271e-08
4326 1.16446932096892e-08
4327 1.16438023667342e-08
4328 1.16431646546289e-08
4329 1.16426059904029e-08
4330 1.16416236650707e-08
4331 1.16409761830027e-08
4332 1.16401865923876e-08
4333 1.16394005544862e-08
4334 1.16386607018626e-08
4335 1.16380727277487e-08
4336 1.16370832969892e-08
4337 1.16365148628006e-08
4338 1.16356364543435e-08
4339 1.16346754452934e-08
4340 1.16340226341549e-08
4341 1.16332508071082e-08
4342 1.16327134591643e-08
4343 1.16318785714498e-08
4344 1.16311191789009e-08
4345 1.16303366937132e-08
4346 1.16297460550641e-08
4347 1.16287779405866e-08
4348 1.1628093155025e-08
4349 1.16272840244847e-08
4350 1.16265725935705e-08
4351 1.16257936610964e-08
4352 1.16250200576928e-08
4353 1.16243308312391e-08
4354 1.16235190361635e-08
4355 1.16228546787056e-08
4356 1.16222231838492e-08
4357 1.16213243472885e-08
4358 1.16205063349639e-08
4359 1.1619924578099e-08
4360 1.16191110066666e-08
4361 1.16182494735995e-08
4362 1.16176970266224e-08
4363 1.16167955255264e-08
4364 1.16162155450183e-08
4365 1.1615323813885e-08
4366 1.16147456097337e-08
4367 1.16140359551764e-08
4368 1.16132614635944e-08
4369 1.16125056237593e-08
4370 1.16116742887584e-08
4371 1.16108802572512e-08
4372 1.16101297464866e-08
4373 1.16092051527517e-08
4374 1.16086305013141e-08
4375 1.16078266998443e-08
4376 1.16072191858052e-08
4377 1.16064775568248e-08
4378 1.16056826371391e-08
4379 1.16050653531374e-08
4380 1.1604251781705e-08
4381 1.1603638050417e-08
4382 1.16027951690967e-08
4383 1.16020730800415e-08
4384 1.16013056938868e-08
4385 1.16006058092921e-08
4386 1.15999094774111e-08
4387 1.15990115290288e-08
4388 1.15982663473346e-08
4389 1.15975122838563e-08
4390 1.15969118752446e-08
4391 1.15960849811358e-08
4392 1.15953131540891e-08
4393 1.15947651480042e-08
4394 1.15938476596966e-08
4395 1.15932312638733e-08
4396 1.15924398969014e-08
4397 1.15917178078462e-08
4398 1.15910925302387e-08
4399 1.15903642239346e-08
4400 1.15895701924273e-08
4401 1.15888338925174e-08
4402 1.15879963402676e-08
4403 1.15874421169337e-08
4404 1.1586759107729e-08
4405 1.15860832039516e-08
4406 1.15852927251581e-08
4407 1.15844454029457e-08
4408 1.15839222658565e-08
4409 1.1583108694424e-08
4410 1.15823173274521e-08
4411 1.15816778389899e-08
4412 1.15807283762592e-08
4413 1.15802061273484e-08
4414 1.15795346644632e-08
4415 1.15787122112465e-08
4416 1.15779563714113e-08
4417 1.15771934261488e-08
4418 1.1576499758803e-08
4419 1.1575830960453e-08
4420 1.1574885050436e-08
4421 1.15743086226416e-08
4422 1.15736398242916e-08
4423 1.1572601543719e-08
4424 1.15721068283392e-08
4425 1.15714353654539e-08
4426 1.15707408099297e-08
4427 1.15699423375304e-08
4428 1.15691509705584e-08
4429 1.15684821722084e-08
4430 1.15677831757921e-08
4431 1.15671339173673e-08
4432 1.15663443267522e-08
4433 1.15653975285568e-08
4434 1.15647305065636e-08
4435 1.15643343789884e-08
4436 1.15635261366265e-08
4437 1.1562812041177e-08
4438 1.15621228147234e-08
4439 1.15613802975645e-08
4440 1.15603855377344e-08
4441 1.15598970396036e-08
4442 1.15590044202918e-08
4443 1.15583631554728e-08
4444 1.1557669488127e-08
4445 1.15567910796699e-08
4446 1.15560521152247e-08
4447 1.1555147061415e-08
4448 1.15545466528033e-08
4449 1.15538716372043e-08
4450 1.15531229027965e-08
4451 1.15525553567863e-08
4452 1.15518101750922e-08
4453 1.1550988610054e-08
4454 1.15504361630769e-08
4455 1.15496412433913e-08
4456 1.15489093843735e-08
4457 1.15482068352435e-08
4458 1.15475033979351e-08
4459 1.15468701267218e-08
4460 1.15462297500812e-08
4461 1.15453726579062e-08
4462 1.15448068882529e-08
4463 1.15438094638876e-08
4464 1.15430571767661e-08
4465 1.15424780844364e-08
4466 1.15417426727049e-08
4467 1.1541277267213e-08
4468 1.15403082645571e-08
4469 1.15396705524518e-08
4470 1.15387646104637e-08
4471 1.15382627896565e-08
4472 1.15375344833524e-08
4473 1.15367653208409e-08
4474 1.15360681007814e-08
4475 1.15353655516515e-08
4476 1.15346088236379e-08
4477 1.15338032458112e-08
4478 1.15331397765317e-08
4479 1.1532378607626e-08
4480 1.15315517135173e-08
4481 1.1530924659553e-08
4482 1.15304601422395e-08
4483 1.15296119318486e-08
4484 1.15289688906728e-08
4485 1.15282228208002e-08
4486 1.15275184953134e-08
4487 1.15267582145862e-08
4488 1.15260867517009e-08
4489 1.15254383814545e-08
4490 1.15247065224366e-08
4491 1.15238076858759e-08
4492 1.15233413922056e-08
4493 1.15224283447901e-08
4494 1.1521964715655e-08
4495 1.15212772655582e-08
4496 1.15204086270637e-08
4497 1.15196208128054e-08
4498 1.15191172156415e-08
4499 1.15181553184129e-08
4500 1.15174865200629e-08
4501 1.15168390379949e-08
4502 1.15161311597944e-08
4503 1.15155174285064e-08
4504 1.15146132628752e-08
4505 1.15137730460901e-08
4506 1.15131779665489e-08
4507 1.15125926569704e-08
4508 1.15118927723756e-08
4509 1.15112541720919e-08
4510 1.15103713227427e-08
4511 1.15096563391148e-08
4512 1.15089804353374e-08
4513 1.15084413110367e-08
4514 1.15075966533595e-08
4515 1.15069616057895e-08
4516 1.15062039895975e-08
4517 1.15055804883468e-08
4518 1.15047864568396e-08
4519 1.15043219395261e-08
4520 1.15034755054921e-08
4521 1.15027933844658e-08
4522 1.15021157043316e-08
4523 1.15015108548278e-08
4524 1.15008287338014e-08
4525 1.14999609834854e-08
4526 1.14994431754667e-08
4527 1.149867134842e-08
4528 1.14979545884353e-08
4529 1.14972680265168e-08
4530 1.14965788000632e-08
4531 1.14958602637216e-08
4532 1.14951896890148e-08
4533 1.14944107565407e-08
4534 1.14938094597505e-08
4535 1.1492947038505e-08
4536 1.14924532113037e-08
4537 1.14917009241822e-08
4538 1.14909770587701e-08
4539 1.14902523051796e-08
4540 1.14894431746393e-08
4541 1.14887344082604e-08
4542 1.14881544277523e-08
4543 1.14873852652408e-08
4544 1.14868505818322e-08
4545 1.14862173106189e-08
4546 1.14855041033479e-08
4547 1.14845013499121e-08
4548 1.14839284748314e-08
4549 1.1483368922427e-08
4550 1.14827916064542e-08
4551 1.14819567187396e-08
4552 1.14812817031407e-08
4553 1.14804468154261e-08
4554 1.14797638062214e-08
4555 1.14789786564984e-08
4556 1.14783293980736e-08
4557 1.14777645165987e-08
4558 1.14769660441993e-08
4559 1.14763611946955e-08
4560 1.14756959490592e-08
4561 1.14749605373277e-08
4562 1.1474535988043e-08
4563 1.14736495859802e-08
4564 1.14730740463642e-08
4565 1.14723262001348e-08
4566 1.14716627308553e-08
4567 1.14709823861858e-08
4568 1.14703793130388e-08
4569 1.1469630578631e-08
4570 1.14688587515843e-08
4571 1.14681855123422e-08
4572 1.1467590432801e-08
4573 1.14669287398783e-08
4574 1.14662270789267e-08
4575 1.14655174243694e-08
4576 1.14650537952343e-08
4577 1.14643095017186e-08
4578 1.14635101411409e-08
4579 1.1462801374762e-08
4580 1.14623341929132e-08
4581 1.14616405255674e-08
4582 1.14607141554757e-08
4583 1.14601839129591e-08
4584 1.14595595235301e-08
4585 1.14587299648861e-08
4586 1.14581766297306e-08
4587 1.14575691156915e-08
4588 1.14568345921384e-08
4589 1.1456014803457e-08
4590 1.14555440688946e-08
4591 1.14548681651172e-08
4592 1.14540803508589e-08
4593 1.14532356931818e-08
4594 1.14526992334163e-08
4595 1.14519380645106e-08
4596 1.14514264737409e-08
4597 1.1450722148254e-08
4598 1.14499991710204e-08
4599 1.14493792224835e-08
4600 1.14486571334282e-08
4601 1.1448023862215e-08
4602 1.14475575685447e-08
4603 1.14467679779295e-08
4604 1.14460627642643e-08
4605 1.14452900490392e-08
4606 1.14446807586432e-08
4607 1.14439000498123e-08
4608 1.14434763887061e-08
4609 1.14427285424767e-08
4610 1.14421228047945e-08
4611 1.14412186391633e-08
4612 1.14406262241573e-08
4613 1.14398126527249e-08
4614 1.14393206018804e-08
4615 1.14384954841285e-08
4616 1.14378924109815e-08
4617 1.14371196957563e-08
4618 1.14364961945057e-08
4619 1.14359552938481e-08
4620 1.14351355051667e-08
4621 1.14344880230988e-08
4622 1.14337037615542e-08
4623 1.14330260814199e-08
4624 1.14322746824769e-08
4625 1.14319265165364e-08
4626 1.14310880761082e-08
4627 1.14303793097292e-08
4628 1.14299165687726e-08
4629 1.14290532593486e-08
4630 1.14282991958703e-08
4631 1.14279279372909e-08
4632 1.14270797269e-08
4633 1.1426322110708e-08
4634 1.14258673633572e-08
4635 1.14251381688746e-08
4636 1.14244045334999e-08
4637 1.14237810322493e-08
4638 1.14230500614099e-08
4639 1.14224372183003e-08
4640 1.14218634550411e-08
4641 1.14210214618993e-08
4642 1.14204476986401e-08
4643 1.141984640185e-08
4644 1.14191553990395e-08
4645 1.14185034760794e-08
4646 1.14178897447914e-08
4647 1.14170637388611e-08
4648 1.1416627643257e-08
4649 1.14158442698908e-08
4650 1.14151239571925e-08
4651 1.14144613760914e-08
4652 1.14136557982647e-08
4653 1.14130136452673e-08
4654 1.14124958372486e-08
4655 1.14119371730226e-08
4656 1.14109939275409e-08
4657 1.14101919024279e-08
4658 1.14098943626573e-08
4659 1.14091793790294e-08
4660 1.14084777180778e-08
4661 1.14076827983922e-08
4662 1.14071543322325e-08
4663 1.14064437894967e-08
4664 1.14057696620762e-08
4665 1.14052216559912e-08
4666 1.14043752219573e-08
4667 1.14038289922291e-08
4668 1.14033014142478e-08
4669 1.14025828779063e-08
4670 1.14020064501119e-08
4671 1.14010978435886e-08
4672 1.14006049045656e-08
4673 1.13997709050295e-08
4674 1.13991882599862e-08
4675 1.13985674232708e-08
4676 1.13978764204603e-08
4677 1.13973523951927e-08
4678 1.13966116543907e-08
4679 1.13958797953728e-08
4680 1.13952625113711e-08
4681 1.13945537449922e-08
4682 1.13938138923686e-08
4683 1.13933138479183e-08
4684 1.1392676135813e-08
4685 1.13917479893644e-08
4686 1.1391309229225e-08
4687 1.13907132615054e-08
4688 1.13900853193627e-08
4689 1.13893703357348e-08
4690 1.13886526875717e-08
4691 1.13879776719727e-08
4692 1.13874341067799e-08
4693 1.13864784268003e-08
4694 1.13859446315701e-08
4695 1.13852856031826e-08
4696 1.13849649707731e-08
4697 1.13839906390467e-08
4698 1.13834479620323e-08
4699 1.13827560710433e-08
4700 1.13821538860748e-08
4701 1.13813527491402e-08
4702 1.13806057910892e-08
4703 1.13801261747426e-08
4704 1.13793854339406e-08
4705 1.13787974598267e-08
4706 1.13782574473476e-08
4707 1.13776801313747e-08
4708 1.1376764419424e-08
4709 1.13765095122176e-08
4710 1.1375663966362e-08
4711 1.13752323116501e-08
4712 1.13742499863179e-08
4713 1.13736797757724e-08
4714 1.13729887729619e-08
4715 1.13723528372134e-08
4716 1.13718021665932e-08
4717 1.13712594895787e-08
4718 1.13703508830554e-08
4719 1.13697860015805e-08
4720 1.13691145386952e-08
4721 1.13684555103077e-08
4722 1.1367827568165e-08
4723 1.13671196899645e-08
4724 1.13664020418014e-08
4725 1.13657225853103e-08
4726 1.13651816846527e-08
4727 1.13645635124726e-08
4728 1.13640332699561e-08
4729 1.13633387144318e-08
4730 1.13626716924387e-08
4731 1.13622302677641e-08
4732 1.1361307450386e-08
4733 1.13608615848193e-08
4734 1.13601164031252e-08
4735 1.13594689210572e-08
4736 1.13588374262008e-08
4737 1.13582885319374e-08
4738 1.13575708837743e-08
4739 1.13568594528601e-08
4740 1.13562723669247e-08
4741 1.13554783354175e-08
4742 1.13547145019766e-08
4743 1.13541931412442e-08
4744 1.13537925727769e-08
4745 1.13531140044643e-08
4746 1.135247007511e-08
4747 1.13518341393615e-08
4748 1.13510774113479e-08
4749 1.13505445042961e-08
4750 1.13499130094397e-08
4751 1.13493285880395e-08
4752 1.13484013297693e-08
4753 1.13479261543148e-08
4754 1.13472191642927e-08
4755 1.134657434676e-08
4756 1.13460849604508e-08
4757 1.13453868522129e-08
4758 1.13448024308127e-08
4759 1.13441442906037e-08
4760 1.13435865145561e-08
4761 1.13428235692936e-08
4762 1.13424061254364e-08
4763 1.13415605795808e-08
4764 1.13407798707499e-08
4765 1.13402611745528e-08
4766 1.13395710599207e-08
4767 1.13389031497491e-08
4768 1.1338450178755e-08
4769 1.13378488819649e-08
4770 1.13369944543251e-08
4771 1.13365805631815e-08
4772 1.1335923311151e-08
4773 1.13351195096811e-08
4774 1.13343627816676e-08
4775 1.13338387563999e-08
4776 1.13332205842198e-08
4777 1.13324860606667e-08
4778 1.13319345018681e-08
4779 1.1331326987829e-08
4780 1.13306759530474e-08
4781 1.1330127058784e-08
4782 1.13294325032598e-08
4783 1.1328882720818e-08
4784 1.13280416158545e-08
4785 1.13274740698444e-08
4786 1.1326825699598e-08
4787 1.13261720002811e-08
4788 1.13253859623796e-08
4789 1.13249249977798e-08
4790 1.13240785637458e-08
4791 1.13235021359515e-08
4792 1.13228146858546e-08
4793 1.13224407627399e-08
4794 1.13217346608963e-08
4795 1.13211973129523e-08
4796 1.13205205209965e-08
4797 1.13199680740195e-08
4798 1.13195657291953e-08
4799 1.13185887329337e-08
4800 1.13180025351767e-08
4801 1.13173275195777e-08
4802 1.13165929960246e-08
4803 1.13161560122421e-08
4804 1.13153379999176e-08
4805 1.13146345626092e-08
4806 1.13140474766737e-08
4807 1.13136060519992e-08
4808 1.13129194900807e-08
4809 1.13122862188675e-08
4810 1.13115703470612e-08
4811 1.13109388522048e-08
4812 1.13102389676101e-08
4813 1.13096767506704e-08
4814 1.13089280162626e-08
4815 1.13084341890612e-08
4816 1.13077067709355e-08
4817 1.13072564644767e-08
4818 1.13065778961641e-08
4819 1.1305953506735e-08
4820 1.1305223424074e-08
4821 1.13046114691429e-08
4822 1.13041007665515e-08
4823 1.13033475912516e-08
4824 1.13026779047232e-08
4825 1.13021476622066e-08
4826 1.1301524160956e-08
4827 1.13007230240214e-08
4828 1.1300053337493e-08
4829 1.12995683920758e-08
4830 1.12990390377377e-08
4831 1.12983142841472e-08
4832 1.12976783483987e-08
4833 1.12971534349526e-08
4834 1.1296502400171e-08
4835 1.12958691289577e-08
4836 1.12952669439892e-08
4837 1.12947207142611e-08
4838 1.129402438238e-08
4839 1.12934808171872e-08
4840 1.1292824453335e-08
4841 1.12923492778805e-08
4842 1.1291840351646e-08
4843 1.12909992466825e-08
4844 1.12903197901915e-08
4845 1.12898561610564e-08
4846 1.12892575288015e-08
4847 1.12886002767709e-08
4848 1.12880762515033e-08
4849 1.12873070889918e-08
4850 1.12868416834999e-08
4851 1.12860734091669e-08
4852 1.12857003742306e-08
4853 1.12847686750683e-08
4854 1.12842570842986e-08
4855 1.12836930910021e-08
4856 1.12830234044736e-08
4857 1.12823590470157e-08
4858 1.12820197628594e-08
4859 1.12811981978211e-08
4860 1.12805613738942e-08
4861 1.12800160323445e-08
4862 1.12794396045501e-08
4863 1.12786375794371e-08
4864 1.12780229599707e-08
4865 1.12773541616207e-08
4866 1.12767439830463e-08
4867 1.12762759130192e-08
4868 1.12755786929597e-08
4869 1.12748876901492e-08
4870 1.12742819524669e-08
4871 1.12737463808799e-08
4872 1.1273113997845e-08
4873 1.12724887202376e-08
4874 1.12717879474644e-08
4875 1.12713953726029e-08
4876 1.12708233857006e-08
4877 1.12701803445248e-08
4878 1.12693436804534e-08
4879 1.12689022557788e-08
4880 1.12682387864993e-08
4881 1.12674927166267e-08
4882 1.12669793495002e-08
4883 1.12663069984364e-08
4884 1.12658948836497e-08
4885 1.1265211874445e-08
4886 1.12646381111858e-08
4887 1.12639213512011e-08
4888 1.12634985782734e-08
4889 1.12628528725622e-08
4890 1.12622879910873e-08
4891 1.12614753078333e-08
4892 1.12610258895529e-08
4893 1.12603677493439e-08
4894 1.12597637880185e-08
4895 1.12592468681783e-08
4896 1.12585958333966e-08
4897 1.12579288114034e-08
4898 1.12573452781817e-08
4899 1.12567457577484e-08
4900 1.12559828124859e-08
4901 1.12556612918979e-08
4902 1.12548939057433e-08
4903 1.12542899444179e-08
4904 1.1253619369711e-08
4905 1.12530518237008e-08
4906 1.12525180284706e-08
4907 1.12519868977756e-08
4908 1.12513518502055e-08
4909 1.12506599592166e-08
4910 1.1249964515514e-08
4911 1.12496909565607e-08
4912 1.12489821901818e-08
4913 1.12485079029057e-08
4914 1.12476605806933e-08
4915 1.12471738589193e-08
4916 1.12466533863653e-08
4917 1.12460645240731e-08
4918 1.12453921730093e-08
4919 1.12450013745047e-08
4920 1.12440972088734e-08
4921 1.12436788768377e-08
4922 1.12430358356619e-08
4923 1.12425997400578e-08
4924 1.12416165265472e-08
4925 1.12411182584538e-08
4926 1.12405444951946e-08
4927 1.12401892238267e-08
4928 1.12395799334308e-08
4929 1.12389892947817e-08
4930 1.12379980876653e-08
4931 1.12375353467087e-08
4932 1.12370575067189e-08
4933 1.12364642035345e-08
4934 1.12358335968565e-08
4935 1.12353806258625e-08
4936 1.12346993930146e-08
4937 1.12340909907971e-08
4938 1.12334346269449e-08
4939 1.12328502055448e-08
4940 1.12321050238506e-08
4941 1.12316280720393e-08
4942 1.12309956890044e-08
4943 1.12303712995754e-08
4944 1.12298339516315e-08
4945 1.12292308784845e-08
4946 1.12285842845949e-08
4947 1.12278888408923e-08
4948 1.12274420871472e-08
4949 1.12268274676808e-08
4950 1.12261382412271e-08
4951 1.12256151041379e-08
4952 1.12250688744098e-08
4953 1.12244329386613e-08
4954 1.12237961147343e-08
4955 1.12231930415874e-08
4956 1.12224931569926e-08
4957 1.12217879433274e-08
4958 1.12211919756078e-08
4959 1.12207345637216e-08
4960 1.12202664936945e-08
4961 1.12196216761618e-08
4962 1.12189404433138e-08
4963 1.12185176703861e-08
4964 1.12178790701023e-08
4965 1.12173603739052e-08
4966 1.12167501953309e-08
4967 1.12161471221839e-08
4968 1.12155120746138e-08
4969 1.12148885733632e-08
4970 1.12142348740463e-08
4971 1.12137135133139e-08
4972 1.12130900120633e-08
4973 1.12125881912561e-08
4974 1.12119753481466e-08
4975 1.12112514827345e-08
4976 1.12107940708484e-08
4977 1.12101483651372e-08
4978 1.1209559502845e-08
4979 1.12090239312579e-08
4980 1.12084048708994e-08
4981 1.12077866987192e-08
4982 1.12070166480294e-08
4983 1.12064419965918e-08
4984 1.12058025081296e-08
4985 1.12055067447159e-08
4986 1.12047171541008e-08
4987 1.12040305921823e-08
4988 1.12037845667601e-08
4989 1.12028137877473e-08
4990 1.12023412768281e-08
4991 1.12017941589215e-08
4992 1.12012559227992e-08
4993 1.12005800190218e-08
4994 1.12000870799989e-08
4995 1.11995728246939e-08
4996 1.11988827100618e-08
4997 1.119831605223e-08
4998 1.11975371197559e-08
4999 1.11971587557491e-08
};
\addlegendentry{Test}

\nextgroupplot[
title={Tanh/Tanh $\rare$},
ymin=1.32532392334702e-08, ymax=1e-05,
]
\addplot [semithick, black, dashed]
table {%
0 0.0056003590002947
1 0.000293405021699073
2 0.00018226213007074
3 0.000163994076780909
4 0.000103804942683837
5 3.60921226413211e-05
6 2.74592383703407e-05
7 2.53970518317601e-05
8 2.22701214879635e-05
9 1.7115937798394e-05
10 1.09938357276747e-05
11 7.27995904036049e-06
12 6.12950939925128e-06
13 5.80576325571514e-06
14 5.64085797281422e-06
15 5.51558242707273e-06
16 5.40543063983279e-06
17 5.29822008119396e-06
18 5.1840570003634e-06
19 5.05249488176673e-06
20 4.89006860936314e-06
21 4.67868564808782e-06
22 4.39478471149357e-06
23 4.01123219220523e-06
24 3.51431517265866e-06
25 2.93855689668021e-06
26 2.39216637677586e-06
27 1.99514910889675e-06
28 1.77003067115322e-06
29 1.65911548797837e-06
30 1.60132354282005e-06
31 1.56790162350617e-06
32 1.5447656155807e-06
33 1.52686224128296e-06
34 1.5124131701052e-06
35 1.50010282577018e-06
36 1.48957981392606e-06
37 1.48022453004693e-06
38 1.47207127704796e-06
39 1.46468819730394e-06
40 1.45837071049293e-06
41 1.45248004735521e-06
42 1.44735141210361e-06
43 1.44219879947016e-06
44 1.43772965183153e-06
45 1.43317228066131e-06
46 1.42908903040784e-06
47 1.42477350562764e-06
48 1.42077140270835e-06
49 1.41640343803218e-06
50 1.41219160518169e-06
51 1.40749679526664e-06
52 1.4027783133006e-06
53 1.39744213642423e-06
54 1.39182568518592e-06
55 1.38547751002527e-06
56 1.3784830822221e-06
57 1.37054967677486e-06
58 1.36149844276812e-06
59 1.35109412013801e-06
60 1.3390488909879e-06
61 1.32494706248565e-06
62 1.30841214342237e-06
63 1.28896227174025e-06
64 1.26606953219266e-06
65 1.23917723064881e-06
66 1.20763243426492e-06
67 1.17086242699571e-06
68 1.12842234520105e-06
69 1.08002865271573e-06
70 1.0265577056856e-06
71 9.68558782188822e-07
72 9.08788739961253e-07
73 8.49514614237989e-07
74 7.93846120302533e-07
75 7.44145482880043e-07
76 7.03054883894438e-07
77 6.70366063697969e-07
78 6.46164182604991e-07
79 6.28547826611481e-07
80 6.15877208456794e-07
81 6.06589651230394e-07
82 5.99590714795539e-07
83 5.93914173018462e-07
84 5.89071284510467e-07
85 5.84840443190515e-07
86 5.81065485244281e-07
87 5.77628000684172e-07
88 5.74447832027758e-07
89 5.71470992907663e-07
90 5.68660036627477e-07
91 5.65988677779572e-07
92 5.63437065673966e-07
93 5.60990097181246e-07
94 5.58637622733116e-07
95 5.56371138634049e-07
96 5.54183179142242e-07
97 5.5206952954201e-07
98 5.50024975991548e-07
99 5.48044798113878e-07
100 5.46126572448458e-07
101 5.44266323553089e-07
102 5.42462945556821e-07
103 5.40715548440929e-07
104 5.39025337388921e-07
105 5.37379897521006e-07
106 5.35793632270298e-07
107 5.34233529482009e-07
108 5.32752992576491e-07
109 5.31266577304734e-07
110 5.29878111587223e-07
111 5.28473278533781e-07
112 5.27163962809496e-07
113 5.25839605822753e-07
114 5.24599219982491e-07
115 5.23357406796876e-07
116 5.2217499808549e-07
117 5.21011319964515e-07
118 5.19883958771672e-07
119 5.18787650928942e-07
120 5.17724026545707e-07
121 5.16690795448582e-07
122 5.15689194138247e-07
123 5.1471729496555e-07
124 5.13774646544007e-07
125 5.12861737158588e-07
126 5.11977446109313e-07
127 5.11121885141463e-07
128 5.10293490597391e-07
129 5.09493481937184e-07
130 5.08723703909197e-07
131 5.07982831040366e-07
132 5.07269279259148e-07
133 5.0658155749872e-07
134 5.05919429036439e-07
135 5.0528181362175e-07
136 5.04668016329646e-07
137 5.04077052385199e-07
138 5.03507535059455e-07
139 5.02959002885106e-07
140 5.02429732497589e-07
141 5.01918651048427e-07
142 5.01425744031891e-07
143 5.0094851846616e-07
144 5.00489983391361e-07
145 5.00041446610311e-07
146 4.99612949292683e-07
147 4.99198499039011e-07
148 4.98790613232103e-07
149 4.98399911530001e-07
150 4.98024025567645e-07
151 4.97652089958223e-07
152 4.97295611198823e-07
153 4.96947040453222e-07
154 4.96608266679743e-07
155 4.96272184999924e-07
156 4.95945945399256e-07
157 4.95626801431115e-07
158 4.95321763908763e-07
159 4.95016312671126e-07
160 4.94715821430347e-07
161 4.94422041965592e-07
162 4.94131923181484e-07
163 4.93845131028792e-07
164 4.93565126275186e-07
165 4.93283271978839e-07
166 4.93007054892303e-07
167 4.92740561369942e-07
168 4.92478365874405e-07
169 4.92213516913864e-07
170 4.91956744044231e-07
171 4.9169307453667e-07
172 4.91441274295568e-07
173 4.9119745242443e-07
174 4.90942379832049e-07
175 4.90696520092371e-07
176 4.90457223012442e-07
177 4.902183282951e-07
178 4.89976693421212e-07
179 4.89742173328977e-07
180 4.8949927592723e-07
181 4.89264377804943e-07
182 4.89030167734938e-07
183 4.88798095908294e-07
184 4.88571656431347e-07
185 4.88335712828913e-07
186 4.88110056929614e-07
187 4.8787579895837e-07
188 4.87652679771244e-07
189 4.8742062804763e-07
190 4.87192006863424e-07
191 4.86960414580295e-07
192 4.86735280940565e-07
193 4.86500155602698e-07
194 4.86277774200516e-07
195 4.86043623528332e-07
196 4.85814486422598e-07
197 4.85583028035919e-07
198 4.85352789226567e-07
199 4.85119449102456e-07
200 4.84890374908176e-07
201 4.84653054172668e-07
202 4.84423951927937e-07
203 4.84185212735966e-07
204 4.83953095169909e-07
205 4.83714118592715e-07
206 4.83479133826847e-07
207 4.83239061392027e-07
208 4.83000766658748e-07
209 4.82759711431413e-07
210 4.82517853008346e-07
211 4.82274492418e-07
212 4.82030294945091e-07
213 4.81784743609381e-07
214 4.81536966214335e-07
215 4.81288204706942e-07
216 4.8103759278284e-07
217 4.80785734577438e-07
218 4.80532226466934e-07
219 4.80277183511291e-07
220 4.80019033961554e-07
221 4.79760801146867e-07
222 4.79499718696985e-07
223 4.79238393912951e-07
224 4.78977878950459e-07
225 4.78710688424755e-07
226 4.78438456315899e-07
227 4.78166408296232e-07
228 4.77890734664754e-07
229 4.77613509117347e-07
230 4.77332414181575e-07
231 4.77049814913144e-07
232 4.76763791066759e-07
233 4.7647526794492e-07
234 4.76183560769172e-07
235 4.75889010202124e-07
236 4.75590957638161e-07
237 4.7529009481373e-07
238 4.74986251191467e-07
239 4.74678208703949e-07
240 4.74367282723875e-07
241 4.74052562118743e-07
242 4.737340610248e-07
243 4.73412449043309e-07
244 4.73086287064106e-07
245 4.72756055328816e-07
246 4.72422433102082e-07
247 4.72083428022074e-07
248 4.71741092301414e-07
249 4.71393254313668e-07
250 4.71041504114922e-07
251 4.70684550691658e-07
252 4.70323207101941e-07
253 4.69955617745654e-07
254 4.69583949204377e-07
255 4.69206048537174e-07
256 4.688227522891e-07
257 4.68434338110058e-07
258 4.68039176372059e-07
259 4.67638625897493e-07
260 4.67231008130398e-07
261 4.66817622314153e-07
262 4.66397357419268e-07
263 4.65970722309805e-07
264 4.65536191722293e-07
265 4.65094482319017e-07
266 4.64645421413223e-07
267 4.64188711330138e-07
268 4.6372345271628e-07
269 4.63249874291805e-07
270 4.62767266318664e-07
271 4.62276207876755e-07
272 4.61776091755794e-07
273 4.61266080991507e-07
274 4.60746645721599e-07
275 4.60217639741245e-07
276 4.59679133816948e-07
277 4.59135013413103e-07
278 4.58598790023856e-07
279 4.5804571235486e-07
280 4.57449968752854e-07
281 4.56857978898384e-07
282 4.56242170391974e-07
283 4.55630747655533e-07
284 4.54987079049118e-07
285 4.54355419707397e-07
286 4.53692591348087e-07
287 4.53023155063903e-07
288 4.52338892364068e-07
289 4.51631811886344e-07
290 4.50923662722147e-07
291 4.50179370631076e-07
292 4.49439340615143e-07
293 4.48666899881545e-07
294 4.47887372805766e-07
295 4.47070446737641e-07
296 4.4625324188452e-07
297 4.4539740755134e-07
298 4.44528051362525e-07
299 4.43643969568086e-07
300 4.42709922213069e-07
301 4.41765830608176e-07
302 4.40832938533475e-07
303 4.39844372513321e-07
304 4.38831757898939e-07
305 4.37789324450932e-07
306 4.36720193523144e-07
307 4.35623826554377e-07
308 4.3449861478706e-07
309 4.33344763106192e-07
310 4.32167990172871e-07
311 4.30955454506332e-07
312 4.29694781706402e-07
313 4.28408987637852e-07
314 4.27092066869861e-07
315 4.25742042008181e-07
316 4.24359572516408e-07
317 4.22945098609517e-07
318 4.21500270176622e-07
319 4.20021817566507e-07
320 4.18504624553506e-07
321 4.16941300324325e-07
322 4.15336207847616e-07
323 4.13692863698856e-07
324 4.11985938985282e-07
325 4.10195231728139e-07
326 4.0836256951593e-07
327 4.0648028444501e-07
328 4.04552905008515e-07
329 4.02591393688922e-07
330 4.00594331214421e-07
331 3.98503543408424e-07
332 3.96375974405672e-07
333 3.94210858832622e-07
334 3.91997446341108e-07
335 3.89736224589043e-07
336 3.87430538795641e-07
337 3.85084004477676e-07
338 3.82700146975168e-07
339 3.80283569063167e-07
340 3.77835596479414e-07
341 3.75360973285055e-07
342 3.72858734065673e-07
343 3.70330693746368e-07
344 3.67780204260981e-07
345 3.65210650258518e-07
346 3.62623505909099e-07
347 3.60022148448991e-07
348 3.57412522461686e-07
349 3.54798301865245e-07
350 3.521853007733e-07
351 3.49580439541697e-07
352 3.46986542572481e-07
353 3.44408649118222e-07
354 3.41848546652201e-07
355 3.39305711722915e-07
356 3.36777604882599e-07
357 3.3426763202371e-07
358 3.31781396633701e-07
359 3.29321782599479e-07
360 3.26892112624932e-07
361 3.24492841690116e-07
362 3.22126639531461e-07
363 3.19796789440829e-07
364 3.17505577323018e-07
365 3.15254573973078e-07
366 3.13045874346329e-07
367 3.10880577563566e-07
368 3.08759687511539e-07
369 3.066841192787e-07
370 3.04653478469419e-07
371 3.02665292805315e-07
372 3.00719518248727e-07
373 2.98813860744929e-07
374 2.96949455689699e-07
375 2.95128969117542e-07
376 2.93355645451854e-07
377 2.91634198584134e-07
378 2.89958893070263e-07
379 2.88315547965823e-07
380 2.86629549039574e-07
381 2.85078531110372e-07
382 2.83479719671576e-07
383 2.81981472008752e-07
384 2.80458038639964e-07
385 2.79013775458026e-07
386 2.77561086596378e-07
387 2.761695175586e-07
388 2.74781479378205e-07
389 2.73439429882671e-07
390 2.72110758158917e-07
391 2.70816076506364e-07
392 2.69539522453677e-07
393 2.68290353929501e-07
394 2.67061039572525e-07
395 2.65855124344405e-07
396 2.64670135000422e-07
397 2.63508332835727e-07
398 2.62369377462157e-07
399 2.61252820524582e-07
400 2.60157682553164e-07
401 2.59083302933227e-07
402 2.5802692694743e-07
403 2.56988384960799e-07
404 2.5596667663752e-07
405 2.54961164572798e-07
406 2.53970986672236e-07
407 2.52995969746372e-07
408 2.52034311515992e-07
409 2.51085074911117e-07
410 2.5014471891005e-07
411 2.49216991607959e-07
412 2.4830721231428e-07
413 2.47439776445368e-07
414 2.46582599006118e-07
415 2.45736988342848e-07
416 2.44907288142748e-07
417 2.44091894797016e-07
418 2.43290559520659e-07
419 2.42503552217777e-07
420 2.41730274923135e-07
421 2.40969586771556e-07
422 2.40223017559771e-07
423 2.39488444046643e-07
424 2.38766558746661e-07
425 2.3805692208434e-07
426 2.3735929604296e-07
427 2.36671740511341e-07
428 2.35996278554751e-07
429 2.35331111208659e-07
430 2.34675398183448e-07
431 2.34030277634467e-07
432 2.33393885703315e-07
433 2.32766912893645e-07
434 2.32148403294552e-07
435 2.31537743945687e-07
436 2.30934238035552e-07
437 2.3033904224512e-07
438 2.29750688487407e-07
439 2.2916885667712e-07
440 2.2859479822035e-07
441 2.2802710154135e-07
442 2.27466948190092e-07
443 2.26913643217586e-07
444 2.26368073443339e-07
445 2.25829395552601e-07
446 2.25296803748165e-07
447 2.24771566165849e-07
448 2.24252378212775e-07
449 2.23738281867725e-07
450 2.23230922465234e-07
451 2.22728353859836e-07
452 2.22230677154656e-07
453 2.21737672252864e-07
454 2.21247884735654e-07
455 2.20761555167037e-07
456 2.20278460110634e-07
457 2.19797490327522e-07
458 2.1932073262132e-07
459 2.18848509568659e-07
460 2.18383723818327e-07
461 2.17926261954382e-07
462 2.17476183448362e-07
463 2.17032829860742e-07
464 2.1659377493588e-07
465 2.16160536683496e-07
466 2.15730861612862e-07
467 2.15304919823467e-07
468 2.14883712875924e-07
469 2.14464814106918e-07
470 2.14050432910895e-07
471 2.13639136441124e-07
472 2.13231799228275e-07
473 2.12827915801128e-07
474 2.12427797187331e-07
475 2.12031884442965e-07
476 2.11639218639093e-07
477 2.11250954221143e-07
478 2.10864788503073e-07
479 2.10483448402243e-07
480 2.10104732389027e-07
481 2.09728373055995e-07
482 2.09355624511076e-07
483 2.08985512871962e-07
484 2.08618394697879e-07
485 2.08253161190441e-07
486 2.07890984331272e-07
487 2.07531008594941e-07
488 2.07173050549514e-07
489 2.06817347130794e-07
490 2.06464990130328e-07
491 2.06113179880063e-07
492 2.05764280231335e-07
493 2.05416956594817e-07
494 2.05071861133277e-07
495 2.04728458022885e-07
496 2.04387128077244e-07
497 2.04047593616252e-07
498 2.03709270307328e-07
499 2.03373268846008e-07
500 2.03039044579967e-07
501 2.02705417757798e-07
502 2.02374267360028e-07
503 2.0204438928495e-07
504 2.01715972860228e-07
505 2.01389449729916e-07
506 2.01064484798508e-07
507 2.00740370708274e-07
508 2.00418346986808e-07
509 2.00097478275652e-07
510 1.99777930069089e-07
511 1.99459792179724e-07
512 1.9914228234974e-07
513 1.98827626530118e-07
514 1.9851324013942e-07
515 1.98199982078151e-07
516 1.97888740503416e-07
517 1.97578077932192e-07
518 1.97268647079696e-07
519 1.96960548032976e-07
520 1.96653289785687e-07
521 1.96347228685667e-07
522 1.96041562826821e-07
523 1.95737485160663e-07
524 1.95433324760685e-07
525 1.95130230327578e-07
526 1.94827516605045e-07
527 1.94525682452706e-07
528 1.94224019706368e-07
529 1.93922974487393e-07
530 1.93622507783431e-07
531 1.93322557922926e-07
532 1.93023430247408e-07
533 1.92724475262196e-07
534 1.92426386748323e-07
535 1.92128327306307e-07
536 1.91831783308061e-07
537 1.91535206198523e-07
538 1.91238998834109e-07
539 1.90943359813289e-07
540 1.90648683206618e-07
541 1.90354730458253e-07
542 1.90060699026162e-07
543 1.89767386507e-07
544 1.89475416203067e-07
545 1.89182859200443e-07
546 1.88890611461545e-07
547 1.8859986252906e-07
548 1.88309138959397e-07
549 1.88018922649746e-07
550 1.87729530495062e-07
551 1.87440617361645e-07
552 1.87151756636084e-07
553 1.86863258206493e-07
554 1.86575674436362e-07
555 1.86288299034665e-07
556 1.86001402425084e-07
557 1.85715339438453e-07
558 1.85429653864588e-07
559 1.8514443808737e-07
560 1.84858905841523e-07
561 1.8457494925439e-07
562 1.84291118810798e-07
563 1.8400776027061e-07
564 1.83725160771253e-07
565 1.83442064533246e-07
566 1.83161335008286e-07
567 1.82879586673401e-07
568 1.82599274888062e-07
569 1.82319635817052e-07
570 1.82040075942247e-07
571 1.8176164390038e-07
572 1.81483656370496e-07
573 1.81206657159372e-07
574 1.80929911006267e-07
575 1.8065416025248e-07
576 1.80378606580156e-07
577 1.80104606916309e-07
578 1.79829947820842e-07
579 1.79557341606795e-07
580 1.79285106556293e-07
581 1.79013293583274e-07
582 1.78742019678602e-07
583 1.78472208935965e-07
584 1.78202198033439e-07
585 1.77933460563828e-07
586 1.77665834846685e-07
587 1.77397904286281e-07
588 1.77132080725784e-07
589 1.76865973587681e-07
590 1.76600980738684e-07
591 1.76336322449622e-07
592 1.76072965984098e-07
593 1.75809991400655e-07
594 1.75548061078779e-07
595 1.75285871097941e-07
596 1.7502642248246e-07
597 1.74765715580172e-07
598 1.74506491259052e-07
599 1.74248525652843e-07
600 1.739897476849e-07
601 1.73734007876014e-07
602 1.73476671604611e-07
603 1.73222000779738e-07
604 1.72966672121966e-07
605 1.72712575677636e-07
606 1.72458310198742e-07
607 1.72206108081774e-07
608 1.71953155372329e-07
609 1.71701974138294e-07
610 1.71450428519293e-07
611 1.71200874219402e-07
612 1.70949903026241e-07
613 1.70701718350763e-07
614 1.70452314588232e-07
615 1.70205221687425e-07
616 1.69957627612405e-07
617 1.69711670356065e-07
618 1.69464448143586e-07
619 1.69220134177195e-07
620 1.68975548143102e-07
621 1.68731027354596e-07
622 1.68488446906423e-07
623 1.68245134203104e-07
624 1.68003951111828e-07
625 1.67762349204459e-07
626 1.67522351183891e-07
627 1.67283188130973e-07
628 1.67045184091386e-07
629 1.66806481581183e-07
630 1.66570500541141e-07
631 1.66334396792145e-07
632 1.66099392319374e-07
633 1.65864042370956e-07
634 1.65632005673544e-07
635 1.65398347260037e-07
636 1.65166947435402e-07
637 1.64935936791899e-07
638 1.64705512809782e-07
639 1.64476958772752e-07
640 1.64247298818587e-07
641 1.64020847435609e-07
642 1.63793918206423e-07
643 1.63568602111752e-07
644 1.63344640855101e-07
645 1.6312135282881e-07
646 1.6289895272692e-07
647 1.62679180482073e-07
648 1.62459210341304e-07
649 1.62240949683756e-07
650 1.62023688695889e-07
651 1.61807755760357e-07
652 1.61593120784787e-07
653 1.61378342303564e-07
654 1.61166455979078e-07
655 1.60954189980522e-07
656 1.60742713350537e-07
657 1.60533590844558e-07
658 1.60324340537699e-07
659 1.60116060168747e-07
660 1.59909060066354e-07
661 1.59703046871407e-07
662 1.59497858605029e-07
663 1.59293506680314e-07
664 1.59090343824886e-07
665 1.58887914036399e-07
666 1.58686491428028e-07
667 1.58485647959061e-07
668 1.58285863464336e-07
669 1.58087285997865e-07
670 1.57889118668209e-07
671 1.57691566424312e-07
672 1.57495123553453e-07
673 1.57299829769642e-07
674 1.57104566200772e-07
675 1.56910423994461e-07
676 1.5671723966193e-07
677 1.56524898625499e-07
678 1.56332310632834e-07
679 1.56142101737977e-07
680 1.55951767988327e-07
681 1.55761487429906e-07
682 1.55573558968669e-07
683 1.55384933295366e-07
684 1.55197710316024e-07
685 1.5501119660577e-07
686 1.54824794915243e-07
687 1.5463973254537e-07
688 1.5445493106192e-07
689 1.54270967644976e-07
690 1.54087902455302e-07
691 1.53905319960401e-07
692 1.53723397356309e-07
693 1.53542190475697e-07
694 1.533613663387e-07
695 1.53181142010794e-07
696 1.530027094514e-07
697 1.52823176751582e-07
698 1.52645443979971e-07
699 1.52467726409533e-07
700 1.52290757712059e-07
701 1.52114467458553e-07
702 1.51938779256788e-07
703 1.51763339574451e-07
704 1.51589373365901e-07
705 1.51414858879484e-07
706 1.51241945497205e-07
707 1.51068063426685e-07
708 1.50896780895415e-07
709 1.5072447846709e-07
710 1.50553557149991e-07
711 1.50382931422222e-07
712 1.50212871791844e-07
713 1.5004317785472e-07
714 1.49874476103484e-07
715 1.49705896946628e-07
716 1.49538228014556e-07
717 1.49371549930777e-07
718 1.49204777502909e-07
719 1.49037501174121e-07
720 1.48872666901667e-07
721 1.48707085791067e-07
722 1.48542934913287e-07
723 1.48377916021403e-07
724 1.48214657142098e-07
725 1.48050742307504e-07
726 1.47887119114287e-07
727 1.47725172968016e-07
728 1.47561868107715e-07
729 1.474002305204e-07
730 1.47238244953662e-07
731 1.470762098843e-07
732 1.46916020325794e-07
733 1.46754798024507e-07
734 1.46594628227525e-07
735 1.46434588407018e-07
736 1.46274452301753e-07
737 1.46115802174052e-07
738 1.45956730165153e-07
739 1.45797720641827e-07
740 1.45639235430206e-07
741 1.45481257855629e-07
742 1.45323312900292e-07
743 1.45166420970622e-07
744 1.45008992799678e-07
745 1.44852160133269e-07
746 1.44695424784569e-07
747 1.44539788700992e-07
748 1.44384077359394e-07
749 1.44228551058667e-07
750 1.4407305604669e-07
751 1.4391855849416e-07
752 1.43763350225257e-07
753 1.43609307053705e-07
754 1.43455290579464e-07
755 1.43301952748143e-07
756 1.4314813414229e-07
757 1.42995400647106e-07
758 1.42842733358517e-07
759 1.42690142932889e-07
760 1.42537672974896e-07
761 1.42385830570912e-07
762 1.42234148795328e-07
763 1.42082827518486e-07
764 1.41931955771835e-07
765 1.41781406245212e-07
766 1.41630933760517e-07
767 1.41480376282921e-07
768 1.41330202671153e-07
769 1.41180485360159e-07
770 1.41032176677047e-07
771 1.40882011215204e-07
772 1.40733721964992e-07
773 1.40585235339774e-07
774 1.40436411165457e-07
775 1.40288885912554e-07
776 1.40140653416498e-07
777 1.39993878110811e-07
778 1.3984566257097e-07
779 1.39699554944972e-07
780 1.39552240449259e-07
781 1.39405055183062e-07
782 1.39259898814803e-07
783 1.39112905654937e-07
784 1.38966586074041e-07
785 1.38822028528107e-07
786 1.38675719659975e-07
787 1.38529964420098e-07
788 1.38385481362135e-07
789 1.38240008940471e-07
790 1.38095512120184e-07
791 1.37950984256729e-07
792 1.37805775537903e-07
793 1.37662070291533e-07
794 1.37517536072718e-07
795 1.37374148848846e-07
796 1.37230543523437e-07
797 1.3708670391388e-07
798 1.36943883033869e-07
799 1.36800095205114e-07
800 1.36657338425827e-07
801 1.36514950810795e-07
802 1.36372475865176e-07
803 1.3623008986885e-07
804 1.36087439283283e-07
805 1.35947035122985e-07
806 1.35804173503562e-07
807 1.35663142592168e-07
808 1.35521780439518e-07
809 1.35380656273831e-07
810 1.35239666618769e-07
811 1.35099430879304e-07
812 1.34959086777187e-07
813 1.34818337191422e-07
814 1.34678964720525e-07
815 1.34539064470474e-07
816 1.34399076668235e-07
817 1.34260124152519e-07
818 1.34120996411635e-07
819 1.33981828480678e-07
820 1.33843208828921e-07
821 1.33705022875574e-07
822 1.33566101649496e-07
823 1.33428134389213e-07
824 1.33290688435927e-07
825 1.33152523722924e-07
826 1.33015349473187e-07
827 1.32878059628805e-07
828 1.32741128023639e-07
829 1.32604127137004e-07
830 1.32467916680756e-07
831 1.32330933083846e-07
832 1.32194402038266e-07
833 1.32057935455343e-07
834 1.31923110769261e-07
835 1.31787567386965e-07
836 1.31651487123463e-07
837 1.31516584190639e-07
838 1.31381999690205e-07
839 1.31246732346213e-07
840 1.31112231970487e-07
841 1.30977994739823e-07
842 1.30843804790448e-07
843 1.30709797149908e-07
844 1.30576469576482e-07
845 1.30443053951712e-07
846 1.30308969232829e-07
847 1.30176839216123e-07
848 1.30043787778256e-07
849 1.29911034839392e-07
850 1.29779069840019e-07
851 1.29646001584938e-07
852 1.29514523533825e-07
853 1.29382887736451e-07
854 1.29251277054987e-07
855 1.29120072679179e-07
856 1.28989109203115e-07
857 1.28858314602187e-07
858 1.28726703211512e-07
859 1.28597295881328e-07
860 1.28467312555003e-07
861 1.28336560699438e-07
862 1.28207272957592e-07
863 1.28077123460191e-07
864 1.27949230838809e-07
865 1.27819577701693e-07
866 1.27690792465884e-07
867 1.27562102970824e-07
868 1.2743423946171e-07
869 1.27305305362757e-07
870 1.27178021059215e-07
871 1.27050097018522e-07
872 1.26923232565801e-07
873 1.26795419585157e-07
874 1.26669527672263e-07
875 1.26542195628243e-07
876 1.2641600838359e-07
877 1.26290145987529e-07
878 1.26163779764266e-07
879 1.260385454529e-07
880 1.25913080126772e-07
881 1.25787630974372e-07
882 1.25662415430661e-07
883 1.25537910225404e-07
884 1.25413571411936e-07
885 1.2528889176755e-07
886 1.2516494426329e-07
887 1.25041219753985e-07
888 1.24918128141083e-07
889 1.24794513160165e-07
890 1.24671752996353e-07
891 1.24549005744878e-07
892 1.24426561570079e-07
893 1.24303994978092e-07
894 1.24182316515675e-07
895 1.24060502843637e-07
896 1.2393913379416e-07
897 1.23817356786571e-07
898 1.23696696574527e-07
899 1.23576059490649e-07
900 1.23455183924204e-07
901 1.23335281250636e-07
902 1.23215145730438e-07
903 1.23095454293676e-07
904 1.22975835264771e-07
905 1.22857303822332e-07
906 1.22738134368472e-07
907 1.22618998716018e-07
908 1.22501266234476e-07
909 1.22382709412783e-07
910 1.22264778941616e-07
911 1.22147131553341e-07
912 1.22029609066043e-07
913 1.21912770178145e-07
914 1.21795581549566e-07
915 1.21679203031633e-07
916 1.21563150046633e-07
917 1.21446823241111e-07
918 1.21331362510357e-07
919 1.2121563517864e-07
920 1.21100597847423e-07
921 1.20985012986186e-07
922 1.20871243330711e-07
923 1.20756619204965e-07
924 1.20642920621083e-07
925 1.2052871546997e-07
926 1.20415284587505e-07
927 1.20301969385839e-07
928 1.20189244043445e-07
929 1.20076722998963e-07
930 1.19964066033429e-07
931 1.19852205198612e-07
932 1.19740777826838e-07
933 1.19629433784851e-07
934 1.19518143531838e-07
935 1.19407344110645e-07
936 1.19297139538066e-07
937 1.19186569761176e-07
938 1.1907708417791e-07
939 1.18967401384396e-07
940 1.18858094949381e-07
941 1.18748971015048e-07
942 1.1863956330771e-07
943 1.18530655432103e-07
944 1.18422570333898e-07
945 1.18314192721236e-07
946 1.18205930522919e-07
947 1.18098597635452e-07
948 1.17990305509963e-07
949 1.17882764322452e-07
950 1.17775699126632e-07
951 1.17668430132944e-07
952 1.17561237900965e-07
953 1.17455883802187e-07
954 1.17348753200019e-07
955 1.17242846690679e-07
956 1.17137359750608e-07
957 1.1703113013084e-07
958 1.16926211665813e-07
959 1.16820793235384e-07
960 1.16715749092222e-07
961 1.16611570541991e-07
962 1.16506271197636e-07
963 1.16403024154721e-07
964 1.16298667556691e-07
965 1.16194167596007e-07
966 1.16091202195356e-07
967 1.15987492840475e-07
968 1.15885045687492e-07
969 1.15781945361881e-07
970 1.15679487611153e-07
971 1.15576826312136e-07
972 1.15474106431535e-07
973 1.15372675244529e-07
974 1.1527058768479e-07
975 1.15169371182056e-07
976 1.15067732402885e-07
977 1.14967122376797e-07
978 1.14866250509493e-07
979 1.14765447901455e-07
980 1.14664940365827e-07
981 1.14564862467859e-07
982 1.14464902109646e-07
983 1.14364610802831e-07
984 1.14265458708296e-07
985 1.1416637591033e-07
986 1.14066901139687e-07
987 1.13968971744427e-07
988 1.13870246331693e-07
989 1.13772268571566e-07
990 1.13675196608654e-07
991 1.13578640706713e-07
992 1.13481777705537e-07
993 1.13386418115979e-07
994 1.13290649709441e-07
995 1.13194892812807e-07
996 1.13099063820066e-07
997 1.13003097905739e-07
998 1.1290816497489e-07
999 1.12813617091589e-07
1000 1.12719066293288e-07
1001 1.1262438451265e-07
1002 1.12529670449391e-07
1003 1.12434898368541e-07
1004 1.12339991620303e-07
1005 1.12244749421286e-07
1006 1.12149524105654e-07
1007 1.12054636608683e-07
1008 1.11960012610268e-07
1009 1.11865539979661e-07
1010 1.1177110210614e-07
1011 1.11676528985249e-07
1012 1.11582322306347e-07
1013 1.11487815767752e-07
1014 1.11394170344781e-07
1015 1.113002670321e-07
1016 1.11206759664029e-07
1017 1.11112629102017e-07
1018 1.11019576769067e-07
1019 1.10925796493966e-07
1020 1.10832929369309e-07
1021 1.10739996959985e-07
1022 1.10647224579985e-07
1023 1.10554312923838e-07
1024 1.10461829561537e-07
1025 1.10369591588722e-07
1026 1.10276923525277e-07
1027 1.10184604303676e-07
1028 1.10092046930355e-07
1029 1.10001279539418e-07
1030 1.09908919466939e-07
1031 1.09817420657521e-07
1032 1.09725474319866e-07
1033 1.09634464368291e-07
1034 1.09543236009202e-07
1035 1.09451826962648e-07
1036 1.09361025477206e-07
1037 1.09270002165651e-07
1038 1.09179203175991e-07
1039 1.09088300230376e-07
1040 1.08997940424516e-07
1041 1.08907551588544e-07
1042 1.08817658755811e-07
1043 1.08726831090422e-07
1044 1.08637057401495e-07
1045 1.08547000639447e-07
1046 1.0845715191099e-07
1047 1.08367433119483e-07
1048 1.08277926136324e-07
1049 1.08188237735618e-07
1050 1.08098565586801e-07
1051 1.08008981201557e-07
1052 1.07919912942123e-07
1053 1.07830683617749e-07
1054 1.07741720461796e-07
1055 1.07652622605592e-07
1056 1.0756361955e-07
1057 1.07474700985044e-07
1058 1.07386048857627e-07
1059 1.07297590985489e-07
1060 1.07208703667538e-07
1061 1.07120296266139e-07
1062 1.07031753813658e-07
1063 1.06943736676079e-07
1064 1.06855285808116e-07
1065 1.06767673887731e-07
1066 1.06679345043048e-07
1067 1.06591430266256e-07
1068 1.06503435395311e-07
1069 1.06415877506372e-07
1070 1.06328032321024e-07
1071 1.06240615665776e-07
1072 1.06152782312741e-07
1073 1.06065393471688e-07
1074 1.05978166552667e-07
1075 1.05890758356431e-07
1076 1.05803893569067e-07
1077 1.057164715208e-07
1078 1.05629342257885e-07
1079 1.05542587768959e-07
1080 1.05456305027118e-07
1081 1.05369226661267e-07
1082 1.05282722634215e-07
1083 1.05196495073123e-07
1084 1.05109831621597e-07
1085 1.0502355247155e-07
1086 1.0493751279661e-07
1087 1.04851073976064e-07
1088 1.04765331136569e-07
1089 1.04679068543945e-07
1090 1.04593258689611e-07
1091 1.04507235413998e-07
1092 1.04421168819258e-07
1093 1.04334941374518e-07
1094 1.04248573194354e-07
1095 1.04162610901959e-07
1096 1.0407625225195e-07
1097 1.03989419564243e-07
1098 1.03902782831788e-07
1099 1.03816097391629e-07
1100 1.03729611245384e-07
1101 1.0364277448538e-07
1102 1.03556498811663e-07
1103 1.03469686052016e-07
1104 1.03383449792105e-07
1105 1.03296676684295e-07
1106 1.03210217333505e-07
1107 1.03123962992058e-07
1108 1.03037669664907e-07
1109 1.02951372321414e-07
1110 1.02864950138226e-07
1111 1.02778861862163e-07
1112 1.02692624178502e-07
1113 1.0260662943562e-07
1114 1.02520354930746e-07
1115 1.02434063460777e-07
1116 1.02348184745971e-07
1117 1.0226203376007e-07
1118 1.0217639637311e-07
1119 1.02090016912193e-07
1120 1.02004801471267e-07
1121 1.01918452955374e-07
1122 1.0183264600272e-07
1123 1.01746898833355e-07
1124 1.01661251156848e-07
1125 1.01576000306913e-07
1126 1.01490182567332e-07
1127 1.01404749903722e-07
1128 1.01318829993424e-07
1129 1.01233704970838e-07
1130 1.01147598738471e-07
1131 1.01062485372339e-07
1132 1.00977572373928e-07
1133 1.00892001193564e-07
1134 1.00806653309249e-07
1135 1.00721412915838e-07
1136 1.00636152575717e-07
1137 1.00550932431887e-07
1138 1.00465635178182e-07
1139 1.00380516832921e-07
1140 1.00295031985276e-07
1141 1.00210819848456e-07
1142 1.00125121202943e-07
1143 1.00040059550999e-07
1144 9.99551737059789e-08
1145 9.98702481993519e-08
1146 9.97853421846884e-08
1147 9.97000654674451e-08
1148 9.96157109387497e-08
1149 9.95303091801958e-08
1150 9.94458127543396e-08
1151 9.93607042438782e-08
1152 9.92760601308262e-08
1153 9.91910179726219e-08
1154 9.91064357993565e-08
1155 9.9021675695532e-08
1156 9.89366743384679e-08
1157 9.8852198941124e-08
1158 9.87674396322191e-08
1159 9.86827906581667e-08
1160 9.85978867937831e-08
1161 9.85133141977457e-08
1162 9.84284658347434e-08
1163 9.83437091237604e-08
1164 9.82590640532521e-08
1165 9.8174462561218e-08
1166 9.80897718751983e-08
1167 9.8005418835978e-08
1168 9.79207630500412e-08
1169 9.78360050778449e-08
1170 9.77516011331225e-08
1171 9.76667123828712e-08
1172 9.7582147538855e-08
1173 9.74979199885873e-08
1174 9.74136562641448e-08
1175 9.73284671088592e-08
1176 9.72447470286575e-08
1177 9.71597478365283e-08
1178 9.70752037079414e-08
1179 9.69900040983518e-08
1180 9.69065642402178e-08
1181 9.6821624955723e-08
1182 9.67371761633196e-08
1183 9.66528142458678e-08
1184 9.65685744089839e-08
1185 9.648403295337e-08
1186 9.63994070501961e-08
1187 9.63155164632745e-08
1188 9.62304867657693e-08
1189 9.61466708924164e-08
1190 9.6062129469221e-08
1191 9.5977886843901e-08
1192 9.58940337074665e-08
1193 9.58092097245178e-08
1194 9.57256177920307e-08
1195 9.56411965478132e-08
1196 9.55576744474662e-08
1197 9.54744152901199e-08
1198 9.53908062846232e-08
1199 9.5308644096459e-08
1200 9.522748628088e-08
1201 9.5148079532148e-08
1202 9.50708760063534e-08
1203 9.49936258831663e-08
1204 9.49151736504561e-08
1205 9.48343073039482e-08
1206 9.47503095503066e-08
1207 9.46664389860352e-08
1208 9.45824016080898e-08
1209 9.44975598864417e-08
1210 9.44130015572142e-08
1211 9.43279847773759e-08
1212 9.42436515014577e-08
1213 9.41592772556454e-08
1214 9.40737921051316e-08
1215 9.39898149798779e-08
1216 9.39051523918266e-08
1217 9.38202068314631e-08
1218 9.37354873831175e-08
1219 9.36514767153618e-08
1220 9.35658611638246e-08
1221 9.3481782561522e-08
1222 9.33965813407767e-08
1223 9.33120457276004e-08
1224 9.32274830076629e-08
1225 9.31428794137545e-08
1226 9.30580219971056e-08
1227 9.29731401302369e-08
1228 9.28888982674181e-08
1229 9.280367817599e-08
1230 9.27192858677373e-08
1231 9.26344780536326e-08
1232 9.25497719288515e-08
1233 9.24652832297035e-08
1234 9.23804083217128e-08
1235 9.22962191189924e-08
1236 9.22109695795648e-08
1237 9.2126606795695e-08
1238 9.20420483621065e-08
1239 9.19567946260358e-08
1240 9.18726275327408e-08
1241 9.17880837283391e-08
1242 9.17036616043809e-08
1243 9.16192076001465e-08
1244 9.15344542362639e-08
1245 9.14504085391954e-08
1246 9.13657551682157e-08
1247 9.12806315809256e-08
1248 9.11968528143348e-08
1249 9.11119703648211e-08
1250 9.10281201966789e-08
1251 9.09434647491381e-08
1252 9.0859189711523e-08
1253 9.07745864808796e-08
1254 9.06903891109145e-08
1255 9.06066658337146e-08
1256 9.05218734139623e-08
1257 9.04376676116314e-08
1258 9.0354068245091e-08
1259 9.02698282727599e-08
1260 9.01852340864373e-08
1261 9.01014415664036e-08
1262 9.00180260949845e-08
1263 8.99341719828861e-08
1264 8.98503487856139e-08
1265 8.97661851486653e-08
1266 8.96827493419572e-08
1267 8.95988459301833e-08
1268 8.95152231734642e-08
1269 8.9431907909443e-08
1270 8.93483556909658e-08
1271 8.92647104926425e-08
1272 8.91813366776795e-08
1273 8.90980059491398e-08
1274 8.90148505869881e-08
1275 8.89311401714998e-08
1276 8.88488003729648e-08
1277 8.87649968026594e-08
1278 8.86824777630579e-08
1279 8.85992717072348e-08
1280 8.85164340416189e-08
1281 8.84337440187899e-08
1282 8.83514296945442e-08
1283 8.82678143017124e-08
1284 8.81859042687338e-08
1285 8.81029122941079e-08
1286 8.80210068778808e-08
1287 8.79386398979776e-08
1288 8.78555869765485e-08
1289 8.77738988567245e-08
1290 8.76919545356181e-08
1291 8.76099293609656e-08
1292 8.75276073899478e-08
1293 8.74460283610823e-08
1294 8.73641019203397e-08
1295 8.72825213065198e-08
1296 8.7200818009503e-08
1297 8.71195586245044e-08
1298 8.70376975279363e-08
1299 8.69564964181002e-08
1300 8.68752748486301e-08
1301 8.67945428502104e-08
1302 8.67129235442299e-08
1303 8.66320321968317e-08
1304 8.65511649723594e-08
1305 8.64704022118801e-08
1306 8.63892632865237e-08
1307 8.63089737843659e-08
1308 8.62288970684411e-08
1309 8.61476628943869e-08
1310 8.60676620306755e-08
1311 8.59874969187935e-08
1312 8.59073064907179e-08
1313 8.5827307799935e-08
1314 8.57475086708526e-08
1315 8.56674792850676e-08
1316 8.55880205703485e-08
1317 8.55082058253132e-08
1318 8.54286715097352e-08
1319 8.53496709285473e-08
1320 8.52704064064902e-08
1321 8.51907582162248e-08
1322 8.5112037430779e-08
1323 8.50331027364426e-08
1324 8.49543337633207e-08
1325 8.48755236018128e-08
1326 8.47973063669016e-08
1327 8.47183792673789e-08
1328 8.46404431675829e-08
1329 8.45617981481972e-08
1330 8.44841429818999e-08
1331 8.4405770802487e-08
1332 8.43283170914866e-08
1333 8.42500293205539e-08
1334 8.41731963361347e-08
1335 8.40951695999514e-08
1336 8.4017851878837e-08
1337 8.39406687944688e-08
1338 8.38635525850506e-08
1339 8.37863557756613e-08
1340 8.37099924484441e-08
1341 8.36329503575151e-08
1342 8.35567168824269e-08
1343 8.34799230990946e-08
1344 8.34037117010134e-08
1345 8.3327697437241e-08
1346 8.32513258957057e-08
1347 8.31755666408363e-08
1348 8.30995906944132e-08
1349 8.30240512494917e-08
1350 8.29486143585356e-08
1351 8.28729625363955e-08
1352 8.27979091311271e-08
1353 8.27228494020282e-08
1354 8.26473546542417e-08
1355 8.25725638713415e-08
1356 8.24979710740692e-08
1357 8.24229146254574e-08
1358 8.23485879761776e-08
1359 8.22740770960451e-08
1360 8.22000312892257e-08
1361 8.21260684036318e-08
1362 8.20520229001254e-08
1363 8.19775522962196e-08
1364 8.19042151856131e-08
1365 8.18306849774686e-08
1366 8.17569028277454e-08
1367 8.16839544950376e-08
1368 8.16107022787449e-08
1369 8.15375039491251e-08
1370 8.14639351816027e-08
1371 8.13917293238653e-08
1372 8.1318490112281e-08
1373 8.1246200630769e-08
1374 8.117376926009e-08
1375 8.11016805366549e-08
1376 8.10289664130437e-08
1377 8.09569981790048e-08
1378 8.08854254303881e-08
1379 8.08137099355122e-08
1380 8.07416715220555e-08
1381 8.06708643183995e-08
1382 8.0599326496511e-08
1383 8.05287387155573e-08
1384 8.04584831168853e-08
1385 8.03871097758702e-08
1386 8.03171427721239e-08
1387 8.02464057407803e-08
1388 8.01769236797867e-08
1389 8.01067713886994e-08
1390 8.00371845830483e-08
1391 7.9968033393385e-08
1392 7.9898459690142e-08
1393 7.9829400406517e-08
1394 7.97605471740681e-08
1395 7.96914132008553e-08
1396 7.96232511648398e-08
1397 7.95545777378948e-08
1398 7.94860608710479e-08
1399 7.94184863766034e-08
1400 7.93496807727223e-08
1401 7.9282179446416e-08
1402 7.92146429580143e-08
1403 7.91473615135985e-08
1404 7.90793629903419e-08
1405 7.90122413931727e-08
1406 7.89453125316086e-08
1407 7.88785030652051e-08
1408 7.88120353893973e-08
1409 7.87448586558881e-08
1410 7.86789529527709e-08
1411 7.86127545655546e-08
1412 7.85462332837383e-08
1413 7.84801144844138e-08
1414 7.84148112549232e-08
1415 7.83493182012762e-08
1416 7.82840419351061e-08
1417 7.82182389347241e-08
1418 7.81535737228722e-08
1419 7.80880096611369e-08
1420 7.80236870330242e-08
1421 7.79589648116641e-08
1422 7.78944685917438e-08
1423 7.78301441890505e-08
1424 7.77664706297365e-08
1425 7.77021176228132e-08
1426 7.76386672827201e-08
1427 7.75752733130908e-08
1428 7.75114222610895e-08
1429 7.74486101482452e-08
1430 7.73855234021958e-08
1431 7.73221075198727e-08
1432 7.72598032208194e-08
1433 7.7197555780284e-08
1434 7.71351077726301e-08
1435 7.70725205661016e-08
1436 7.70104514655756e-08
1437 7.69488679330799e-08
1438 7.68872757199723e-08
1439 7.6825845886308e-08
1440 7.67645838006814e-08
1441 7.67033378261672e-08
1442 7.66423249789483e-08
1443 7.65814724652536e-08
1444 7.65217411320407e-08
1445 7.64607187289101e-08
1446 7.64007486999674e-08
1447 7.63403863484058e-08
1448 7.62805139062905e-08
1449 7.62209717555251e-08
1450 7.61611597130951e-08
1451 7.61023731765675e-08
1452 7.6042645479113e-08
1453 7.59837575863287e-08
1454 7.59250329434025e-08
1455 7.5866190035967e-08
1456 7.5807794571503e-08
1457 7.57492728959974e-08
1458 7.56911794317361e-08
1459 7.56332624511913e-08
1460 7.55759858277472e-08
1461 7.5517623839616e-08
1462 7.54605482136128e-08
1463 7.540325461175e-08
1464 7.53459404823076e-08
1465 7.52893404367683e-08
1466 7.52322839714381e-08
1467 7.51756201644582e-08
1468 7.51198111013096e-08
1469 7.50633878698892e-08
1470 7.5007249716208e-08
1471 7.49512016939846e-08
1472 7.48949245501507e-08
1473 7.48400948653583e-08
1474 7.47843234121248e-08
1475 7.47293373715152e-08
1476 7.46741343862212e-08
1477 7.46193131240247e-08
1478 7.45641705419686e-08
1479 7.4510316151688e-08
1480 7.44550983116099e-08
1481 7.44011536761846e-08
1482 7.43470618953523e-08
1483 7.42929704813378e-08
1484 7.42391204306969e-08
1485 7.41859015054302e-08
1486 7.41319993240452e-08
1487 7.40790854036888e-08
1488 7.40260326050901e-08
1489 7.39724295319988e-08
1490 7.39194975531987e-08
1491 7.38676278144723e-08
1492 7.38145750416308e-08
1493 7.37621008299882e-08
1494 7.37100584613515e-08
1495 7.36583694420645e-08
1496 7.36061570885127e-08
1497 7.35542768603992e-08
1498 7.35026032616659e-08
1499 7.34513893583255e-08
1500 7.34001905908777e-08
1501 7.33485121879873e-08
1502 7.32982071678379e-08
1503 7.32465211337185e-08
1504 7.31966126106265e-08
1505 7.31454526996522e-08
1506 7.30955229029107e-08
1507 7.30449877703876e-08
1508 7.29947984825507e-08
1509 7.2944964850219e-08
1510 7.2895271726825e-08
1511 7.28455315202048e-08
1512 7.27960886548473e-08
1513 7.27467832826179e-08
1514 7.26976580907035e-08
1515 7.26481834143478e-08
1516 7.2599194409495e-08
1517 7.25506944734278e-08
1518 7.25015857625522e-08
1519 7.24537286358462e-08
1520 7.24050350009442e-08
1521 7.23564768638774e-08
1522 7.23084703073695e-08
1523 7.22606048748631e-08
1524 7.22128371566377e-08
1525 7.21652909243531e-08
1526 7.21179332550115e-08
1527 7.20701428931214e-08
1528 7.20233929287772e-08
1529 7.19754788076621e-08
1530 7.19288694850384e-08
1531 7.1882347628005e-08
1532 7.18353872684041e-08
1533 7.17887261068029e-08
1534 7.17419807108044e-08
1535 7.16957475188629e-08
1536 7.16500153505884e-08
1537 7.16038283781195e-08
1538 7.15578643664117e-08
1539 7.15122425352099e-08
1540 7.14667553931569e-08
1541 7.14208748648915e-08
1542 7.13752670971068e-08
1543 7.13307042694566e-08
1544 7.12853863586993e-08
1545 7.12401439728616e-08
1546 7.11956104910705e-08
1547 7.11506906747594e-08
1548 7.11063332907713e-08
1549 7.10617275707648e-08
1550 7.10175079867703e-08
1551 7.09731686723281e-08
1552 7.09289310645289e-08
1553 7.08855760516158e-08
1554 7.0841278086764e-08
1555 7.07976356149054e-08
1556 7.07538082962422e-08
1557 7.07109986799459e-08
1558 7.06671040666684e-08
1559 7.06242637535937e-08
1560 7.05808026961741e-08
1561 7.05380721059967e-08
1562 7.04949882486261e-08
1563 7.04520472325498e-08
1564 7.04103812845247e-08
1565 7.03674207533922e-08
1566 7.03249128206629e-08
1567 7.02827379339688e-08
1568 7.02404211163632e-08
1569 7.01987938178661e-08
1570 7.01562741824979e-08
1571 7.01153904891783e-08
1572 7.0073370149526e-08
1573 7.00317109401105e-08
1574 6.99901779501744e-08
1575 6.99486460602472e-08
1576 6.99077917207802e-08
1577 6.98671117698169e-08
1578 6.98256083544457e-08
1579 6.97848091553155e-08
1580 6.97440580967879e-08
1581 6.97039840420466e-08
1582 6.96630703260759e-08
1583 6.96224030072301e-08
1584 6.95825774301717e-08
1585 6.95426798920096e-08
1586 6.95027691755001e-08
1587 6.94624417332257e-08
1588 6.94223416557449e-08
1589 6.93830734102008e-08
1590 6.93437485748127e-08
1591 6.93041729395993e-08
1592 6.92642468620441e-08
1593 6.9225812610707e-08
1594 6.9186335278193e-08
1595 6.91475566796385e-08
1596 6.9108862757794e-08
1597 6.90700214645013e-08
1598 6.90310000561745e-08
1599 6.8992659459699e-08
1600 6.89543180869556e-08
1601 6.89160036282388e-08
1602 6.88777047872513e-08
1603 6.88396823136372e-08
1604 6.88018105852173e-08
1605 6.87630765354719e-08
1606 6.87255141511578e-08
1607 6.86877771101102e-08
1608 6.8649907142948e-08
1609 6.86126494851003e-08
1610 6.85751099207543e-08
1611 6.85379505456218e-08
1612 6.85005167229136e-08
1613 6.84629836888995e-08
1614 6.84261953307619e-08
1615 6.83888546300082e-08
1616 6.83517614308293e-08
1617 6.83152708225165e-08
1618 6.82779521379295e-08
1619 6.82418520243644e-08
1620 6.82051713609688e-08
1621 6.81683629335517e-08
1622 6.81319159792082e-08
1623 6.80959476793141e-08
1624 6.80589959354272e-08
1625 6.80232624206134e-08
1626 6.79871079349859e-08
1627 6.79510112120418e-08
1628 6.7914975515837e-08
1629 6.78790075214764e-08
1630 6.7843255171951e-08
1631 6.78073336497498e-08
1632 6.77720841930629e-08
1633 6.77366064216756e-08
1634 6.77003251898789e-08
1635 6.76653188138587e-08
1636 6.76303690432256e-08
1637 6.75950338444586e-08
1638 6.75592318950535e-08
1639 6.75246455821288e-08
1640 6.74899485666458e-08
1641 6.74547112446966e-08
1642 6.74198718804142e-08
1643 6.73850492005634e-08
1644 6.73504355526156e-08
1645 6.73158122017625e-08
1646 6.72814266380328e-08
1647 6.72467521076747e-08
1648 6.72121529206038e-08
1649 6.71785991928076e-08
1650 6.71438354813425e-08
1651 6.71103054661337e-08
1652 6.70750208513482e-08
1653 6.70417639807575e-08
1654 6.7007488626647e-08
1655 6.69738472693915e-08
1656 6.6940267161808e-08
1657 6.69064888936255e-08
1658 6.68726293069355e-08
1659 6.68394901706115e-08
1660 6.68055016581626e-08
1661 6.67721523379683e-08
1662 6.67388588468931e-08
1663 6.67056246483888e-08
1664 6.66720581063984e-08
1665 6.6639710778027e-08
1666 6.66058972793593e-08
1667 6.65728645432573e-08
1668 6.65405621744775e-08
1669 6.65070905450094e-08
1670 6.64748030074769e-08
1671 6.64424077303494e-08
1672 6.6408974430221e-08
1673 6.637692098721e-08
1674 6.63439681312106e-08
1675 6.63120408423978e-08
1676 6.62792955123948e-08
1677 6.62466742222811e-08
1678 6.62149736072593e-08
1679 6.61822091001518e-08
1680 6.61507505079761e-08
1681 6.61184886672217e-08
1682 6.60868590256491e-08
1683 6.60544000141705e-08
1684 6.60225069921694e-08
1685 6.5991323601633e-08
1686 6.59590941296528e-08
1687 6.59274766570128e-08
1688 6.5895818520012e-08
1689 6.58651185716508e-08
1690 6.58331047107552e-08
1691 6.5801314932834e-08
1692 6.57704238515855e-08
1693 6.57385674389666e-08
1694 6.57080359895978e-08
1695 6.56762500845787e-08
1696 6.56453381657762e-08
1697 6.56148621449581e-08
1698 6.55831790372652e-08
1699 6.55523759331977e-08
1700 6.55213215727812e-08
1701 6.54917919806763e-08
1702 6.5460147948837e-08
1703 6.54295556534734e-08
1704 6.53986961447472e-08
1705 6.53687580616236e-08
1706 6.53376443686682e-08
1707 6.53072792515807e-08
1708 6.52773263083439e-08
1709 6.52467346546892e-08
1710 6.52160741760888e-08
1711 6.51861580012181e-08
1712 6.51563038336356e-08
1713 6.51258534403176e-08
1714 6.50960827246472e-08
1715 6.50656671488115e-08
1716 6.50360792393556e-08
1717 6.50063769671583e-08
1718 6.49759796420568e-08
1719 6.49469522930879e-08
1720 6.49172828648403e-08
1721 6.48874016548895e-08
1722 6.48575185531186e-08
1723 6.48281427721997e-08
1724 6.47988059832017e-08
1725 6.47695114759372e-08
1726 6.47398123847509e-08
1727 6.47108286466746e-08
1728 6.46815603584372e-08
1729 6.46525882674887e-08
1730 6.46228227272338e-08
1731 6.45943255270787e-08
1732 6.4565189399346e-08
1733 6.45366061395869e-08
1734 6.45070632967482e-08
1735 6.44781354997015e-08
1736 6.44497979522818e-08
1737 6.44210169671311e-08
1738 6.43916255773647e-08
1739 6.43634786210434e-08
1740 6.43347197089028e-08
1741 6.43067855867763e-08
1742 6.42774535548618e-08
1743 6.42492453266641e-08
1744 6.42209510877123e-08
1745 6.41930920632916e-08
1746 6.41643096392919e-08
1747 6.41363461322264e-08
1748 6.41079844561077e-08
1749 6.40796893285334e-08
1750 6.40521350727674e-08
1751 6.40235847555459e-08
1752 6.39960932096173e-08
1753 6.39677742939604e-08
1754 6.39398858202078e-08
1755 6.39128005430756e-08
1756 6.38843433629077e-08
1757 6.38568127726558e-08
1758 6.38289017329541e-08
1759 6.38019223861974e-08
1760 6.3773429630487e-08
1761 6.37463818460304e-08
1762 6.37193153494309e-08
1763 6.36912975173764e-08
1764 6.36640122180232e-08
1765 6.36366603150584e-08
1766 6.3609230338546e-08
1767 6.35823507018074e-08
1768 6.35550381953287e-08
1769 6.35281984306957e-08
1770 6.35004729683253e-08
1771 6.34741410538986e-08
1772 6.34469464788623e-08
1773 6.34198802944574e-08
1774 6.33927879585272e-08
1775 6.33666543876643e-08
1776 6.3338700095894e-08
1777 6.33126697033859e-08
1778 6.32859668572294e-08
1779 6.32597675216395e-08
1780 6.32321717306539e-08
1781 6.32063534009575e-08
1782 6.31789648601888e-08
1783 6.31535993456112e-08
1784 6.31267688726567e-08
1785 6.31001258901698e-08
1786 6.30742074880786e-08
1787 6.30472712637165e-08
1788 6.30219847082003e-08
1789 6.29947176200574e-08
1790 6.29693166778189e-08
1791 6.2943249583558e-08
1792 6.29170439325755e-08
1793 6.28907953337254e-08
1794 6.28643540703244e-08
1795 6.28393187898091e-08
1796 6.28131711426505e-08
1797 6.27872493974557e-08
1798 6.27613467778509e-08
1799 6.2736219405668e-08
1800 6.2709552388629e-08
1801 6.26846440217044e-08
1802 6.26585555578707e-08
1803 6.2633102056342e-08
1804 6.2607849227625e-08
1805 6.25819389319382e-08
1806 6.25566672938227e-08
1807 6.25315879072552e-08
1808 6.25057407694918e-08
1809 6.24804165596515e-08
1810 6.24557406476711e-08
1811 6.24299965572561e-08
1812 6.2405395630627e-08
1813 6.23793162182196e-08
1814 6.23543128863702e-08
1815 6.23302886957333e-08
1816 6.23047351222539e-08
1817 6.22790080537783e-08
1818 6.22549913948944e-08
1819 6.22294994787609e-08
1820 6.22049659320112e-08
1821 6.21798630335846e-08
1822 6.21555343007785e-08
1823 6.21302598338147e-08
1824 6.21055899108214e-08
1825 6.20813568028567e-08
1826 6.20564388320588e-08
1827 6.20319732309582e-08
1828 6.20071693031576e-08
1829 6.19833358035393e-08
1830 6.19581714387252e-08
1831 6.19337573342094e-08
1832 6.19099199701267e-08
1833 6.18851811076126e-08
1834 6.18605425914964e-08
1835 6.1836793862291e-08
1836 6.18129708085391e-08
1837 6.17876049311406e-08
1838 6.17648404945825e-08
1839 6.17398159539206e-08
1840 6.1716050138827e-08
1841 6.16920383604835e-08
1842 6.16680947169712e-08
1843 6.16440326046686e-08
1844 6.16203933576642e-08
1845 6.15959892700246e-08
1846 6.15726316519627e-08
1847 6.1548532443112e-08
1848 6.15245275339404e-08
1849 6.15011875577665e-08
1850 6.14773233342802e-08
1851 6.14537358538669e-08
1852 6.14306161699218e-08
1853 6.14060020502905e-08
1854 6.13831976341572e-08
1855 6.13597153877876e-08
1856 6.13358212735449e-08
1857 6.13128426314446e-08
1858 6.12893553109117e-08
1859 6.12658592311632e-08
1860 6.12426981705383e-08
1861 6.12191609210555e-08
1862 6.11963406660365e-08
1863 6.1172423230893e-08
1864 6.1149677238248e-08
1865 6.1126657687538e-08
1866 6.11032207169515e-08
1867 6.1080294790905e-08
1868 6.10573264863135e-08
1869 6.10338625968154e-08
1870 6.10119262622888e-08
1871 6.09880688289266e-08
1872 6.09658708459193e-08
1873 6.09423745290272e-08
1874 6.09200223662398e-08
1875 6.08970004432052e-08
1876 6.08743219174812e-08
1877 6.08518793274726e-08
1878 6.08288234529297e-08
1879 6.0805829152244e-08
1880 6.07842033284456e-08
1881 6.07605180276138e-08
1882 6.07384076545259e-08
1883 6.07156353220617e-08
1884 6.06942749925565e-08
1885 6.06712451070024e-08
1886 6.06484279428443e-08
1887 6.0626716721579e-08
1888 6.06037950623417e-08
1889 6.05820857799699e-08
1890 6.05592924269871e-08
1891 6.05376187383655e-08
1892 6.05151293031625e-08
1893 6.04925295029091e-08
1894 6.04714139376128e-08
1895 6.0448581920447e-08
1896 6.04269165762616e-08
1897 6.04043311609459e-08
1898 6.03828438965337e-08
1899 6.03606960396519e-08
1900 6.03392174749473e-08
1901 6.0316219114398e-08
1902 6.02950301180627e-08
1903 6.02732073176071e-08
1904 6.02517551024917e-08
1905 6.0229233625364e-08
1906 6.02082198266629e-08
1907 6.01864194864632e-08
1908 6.01646043527637e-08
1909 6.01431710371969e-08
1910 6.01217093292306e-08
1911 6.00994802457677e-08
1912 6.00784859416947e-08
1913 6.00566851511886e-08
1914 6.00355574489342e-08
1915 6.00139993247595e-08
1916 5.99926595503675e-08
1917 5.99719489207118e-08
1918 5.99495915070136e-08
1919 5.99285979303588e-08
1920 5.99076660714459e-08
1921 5.98862890308638e-08
1922 5.9864549203148e-08
1923 5.98438506225207e-08
1924 5.98230475823414e-08
1925 5.98018603126249e-08
1926 5.97809140145955e-08
1927 5.97594475957308e-08
1928 5.973856107655e-08
1929 5.97176354730777e-08
1930 5.96971719444284e-08
1931 5.96758768294237e-08
1932 5.96552748923784e-08
1933 5.96338964280463e-08
1934 5.96137872284075e-08
1935 5.95922766262191e-08
1936 5.95717346154601e-08
1937 5.95510557666046e-08
1938 5.95306106401239e-08
1939 5.95099331919258e-08
1940 5.9489221675868e-08
1941 5.94687144985784e-08
1942 5.94486886837942e-08
1943 5.94271699743132e-08
1944 5.94075317330578e-08
1945 5.93869945797465e-08
1946 5.93666277843852e-08
1947 5.93461369646775e-08
1948 5.93257807874892e-08
1949 5.93054252675529e-08
1950 5.92851121670179e-08
1951 5.92651697228774e-08
1952 5.92441223812301e-08
1953 5.92252098785551e-08
1954 5.92039972637792e-08
1955 5.91839491965729e-08
1956 5.91638267213312e-08
1957 5.91438097643504e-08
1958 5.91239237905938e-08
1959 5.9104062909654e-08
1960 5.90840224070632e-08
1961 5.90641623716692e-08
1962 5.90441497845262e-08
1963 5.90241559841331e-08
1964 5.90047041022501e-08
1965 5.89847257765896e-08
1966 5.89650468612035e-08
1967 5.89447873702298e-08
1968 5.89250598155289e-08
1969 5.89058013265031e-08
1970 5.88859452497203e-08
1971 5.88660969098598e-08
1972 5.88466186668413e-08
1973 5.88272968968795e-08
1974 5.88075125067533e-08
1975 5.87881477960295e-08
1976 5.87682328792027e-08
1977 5.87489926422435e-08
1978 5.87301633920845e-08
1979 5.87101681048807e-08
1980 5.86909378474942e-08
1981 5.86713091959012e-08
1982 5.86523088133006e-08
1983 5.86330461009865e-08
1984 5.86137864022618e-08
1985 5.85944417359485e-08
1986 5.85749335115793e-08
1987 5.85562259605155e-08
1988 5.85369323902185e-08
1989 5.85179504808408e-08
1990 5.84987813621041e-08
1991 5.84803830969705e-08
1992 5.84607561675554e-08
1993 5.84415292523843e-08
1994 5.84231245541744e-08
1995 5.84041723890039e-08
1996 5.83855936158173e-08
1997 5.83664151214691e-08
1998 5.8347964469796e-08
1999 5.83293285991715e-08
2000 5.83101749365156e-08
2001 5.82917213050038e-08
2002 5.82736202678902e-08
2003 5.82542314724321e-08
2004 5.82363088934557e-08
2005 5.82174359449184e-08
2006 5.81984738303731e-08
2007 5.8180931370444e-08
2008 5.81621139277289e-08
2009 5.8143735165217e-08
2010 5.81253118889791e-08
2011 5.81071002372369e-08
2012 5.80886407144376e-08
2013 5.80712978930364e-08
2014 5.80527874429748e-08
2015 5.80344350389339e-08
2016 5.80160891745507e-08
2017 5.79978169916728e-08
2018 5.79803797178258e-08
2019 5.79620564629124e-08
2020 5.79446345279067e-08
2021 5.79261514674911e-08
2022 5.79088176646536e-08
2023 5.7890743723199e-08
2024 5.78726500064519e-08
2025 5.78550064220806e-08
2026 5.78374386597247e-08
2027 5.78197670044744e-08
2028 5.78018838979233e-08
2029 5.77839806865654e-08
2030 5.77671358090726e-08
2031 5.77488801445369e-08
2032 5.77321157697597e-08
2033 5.77142406195108e-08
2034 5.76965852960853e-08
2035 5.76798085587527e-08
2036 5.76622496009094e-08
2037 5.7644526233247e-08
2038 5.76282155444474e-08
2039 5.76103020923924e-08
2040 5.75927773409646e-08
2041 5.75763497820958e-08
2042 5.75588169740016e-08
2043 5.75425684488096e-08
2044 5.75242980094259e-08
2045 5.7507582316596e-08
2046 5.74905659158631e-08
2047 5.74739969847471e-08
2048 5.74563476289924e-08
2049 5.74401346113973e-08
2050 5.74225129410522e-08
2051 5.74062081861726e-08
2052 5.7388997131369e-08
2053 5.73729441759596e-08
2054 5.73552374856234e-08
2055 5.73391823968095e-08
2056 5.73223015729596e-08
2057 5.73054069965551e-08
2058 5.7288717427717e-08
2059 5.72726353542663e-08
2060 5.72557167757282e-08
2061 5.72390596005334e-08
2062 5.72230505699878e-08
2063 5.72056256009645e-08
2064 5.7190333740742e-08
2065 5.71729354543749e-08
2066 5.7157065319835e-08
2067 5.71405679394843e-08
2068 5.71240389586336e-08
2069 5.71077252278229e-08
2070 5.70912442823257e-08
2071 5.70753515560796e-08
2072 5.70590647424041e-08
2073 5.70427930508544e-08
2074 5.70265781432155e-08
2075 5.70099018277759e-08
2076 5.69944256429977e-08
2077 5.69785263531131e-08
2078 5.69617643035869e-08
2079 5.69456738315211e-08
2080 5.69304149111716e-08
2081 5.69135394448139e-08
2082 5.68980926249907e-08
2083 5.68821927302565e-08
2084 5.68659073625355e-08
2085 5.68501117630404e-08
2086 5.68341382942705e-08
2087 5.6817921795016e-08
2088 5.68028665748344e-08
2089 5.67866732348321e-08
2090 5.6770680823881e-08
2091 5.67551532695276e-08
2092 5.67396826642863e-08
2093 5.67233620287766e-08
2094 5.67078264284149e-08
2095 5.66922762006428e-08
2096 5.66763457037212e-08
2097 5.66609023469056e-08
2098 5.66452128785144e-08
2099 5.66295343826795e-08
2100 5.6614005464084e-08
2101 5.6598931006846e-08
2102 5.6582829816243e-08
2103 5.6567432544341e-08
2104 5.65524425519826e-08
2105 5.65356814110629e-08
2106 5.65214041037265e-08
2107 5.65060536068529e-08
2108 5.64905290927342e-08
2109 5.64746156879181e-08
2110 5.64601653012531e-08
2111 5.6444104651554e-08
2112 5.64290786133626e-08
2113 5.64135379708119e-08
2114 5.63989958317634e-08
2115 5.63836848206378e-08
2116 5.63679393135175e-08
2117 5.63525519519104e-08
2118 5.6338040806736e-08
2119 5.63223354355102e-08
2120 5.63077975215265e-08
2121 5.62921933981997e-08
2122 5.62778726651558e-08
2123 5.62620943069803e-08
2124 5.62478693062474e-08
2125 5.62320899728519e-08
2126 5.62173058167303e-08
2127 5.62023971948378e-08
2128 5.6187326840984e-08
2129 5.61730793897652e-08
2130 5.61575028310557e-08
2131 5.61433432917369e-08
2132 5.61278041804059e-08
2133 5.61132674476994e-08
2134 5.60985300568362e-08
2135 5.60836413106003e-08
2136 5.60694671305484e-08
2137 5.6054233404268e-08
2138 5.60398279452201e-08
2139 5.60249008536573e-08
2140 5.60107880041016e-08
2141 5.59957573251779e-08
2142 5.59811508971109e-08
2143 5.59667323933866e-08
2144 5.59520311531614e-08
2145 5.5937373753423e-08
2146 5.59228066503437e-08
2147 5.59085059608222e-08
2148 5.58938025951861e-08
2149 5.5878969377865e-08
2150 5.58650145849882e-08
2151 5.58502627736601e-08
2152 5.58361222262604e-08
2153 5.5821440358983e-08
2154 5.58071146805617e-08
2155 5.57923435247076e-08
2156 5.5778105861215e-08
2157 5.57640237710544e-08
2158 5.57490952175499e-08
2159 5.57356905623863e-08
2160 5.57209254647972e-08
2161 5.57060139714238e-08
2162 5.5692126889717e-08
2163 5.56775903923779e-08
2164 5.56636649422515e-08
2165 5.56487647935811e-08
2166 5.56348930391692e-08
2167 5.56205105093355e-08
2168 5.56063010783703e-08
2169 5.55920450544534e-08
2170 5.55779952478019e-08
2171 5.55634180727793e-08
2172 5.55493108658212e-08
2173 5.55353332494235e-08
2174 5.55213161277379e-08
2175 5.55067084970773e-08
2176 5.54928729004089e-08
2177 5.54787433504345e-08
2178 5.54647707238232e-08
2179 5.54505404553041e-08
2180 5.54364791458539e-08
2181 5.54225588960122e-08
2182 5.54085827877415e-08
2183 5.53944717660571e-08
2184 5.53804099281408e-08
2185 5.53667020568582e-08
2186 5.53524749493661e-08
2187 5.53383031318688e-08
2188 5.53247462100615e-08
2189 5.53106009473225e-08
2190 5.52963013511487e-08
2191 5.52828342987866e-08
2192 5.52689923960514e-08
2193 5.52549467682795e-08
2194 5.52406865654831e-08
2195 5.52275332674768e-08
2196 5.52136056564123e-08
2197 5.51996809354804e-08
2198 5.51858086828005e-08
2199 5.51721144241846e-08
2200 5.51580618699532e-08
2201 5.51443961485099e-08
2202 5.51306734823953e-08
2203 5.51168372262545e-08
2204 5.51029858630869e-08
2205 5.50898634941177e-08
2206 5.50759039303905e-08
2207 5.50620853490003e-08
2208 5.50487855104365e-08
2209 5.50349471120093e-08
2210 5.502137929847e-08
2211 5.5007732465473e-08
2212 5.49939171485825e-08
2213 5.49802615599226e-08
2214 5.49671052105793e-08
2215 5.49533898546173e-08
2216 5.49397264526519e-08
2217 5.49261679179125e-08
2218 5.4912406871388e-08
2219 5.48993218196792e-08
2220 5.48855929998204e-08
2221 5.48724082767293e-08
2222 5.48587257758193e-08
2223 5.48450971025183e-08
2224 5.48324733151695e-08
2225 5.48181755331001e-08
2226 5.48054254942976e-08
2227 5.4791268509824e-08
2228 5.4778246290077e-08
2229 5.47650069870897e-08
2230 5.47520594409434e-08
2231 5.47380334006142e-08
2232 5.47250522280329e-08
2233 5.47115873827941e-08
2234 5.46985230687191e-08
2235 5.46848982612147e-08
2236 5.46717104894512e-08
2237 5.46586051259901e-08
2238 5.4645185823432e-08
2239 5.46322463965687e-08
2240 5.46191941359631e-08
2241 5.46052471501746e-08
2242 5.45928378290483e-08
2243 5.45795030331142e-08
2244 5.45664450171124e-08
2245 5.45531386459963e-08
2246 5.45399703488769e-08
2247 5.45266890963347e-08
2248 5.45136860989892e-08
2249 5.45005970815637e-08
2250 5.44875080712437e-08
2251 5.44744660122554e-08
2252 5.44614786486264e-08
2253 5.44489520675207e-08
2254 5.44353293552291e-08
2255 5.44230595407313e-08
2256 5.4409107527853e-08
2257 5.43964099510852e-08
2258 5.43839792284118e-08
2259 5.43709630989042e-08
2260 5.43580715572922e-08
2261 5.43450665992928e-08
2262 5.43316490713153e-08
2263 5.43192097679501e-08
2264 5.43062898352709e-08
2265 5.42933705656168e-08
2266 5.42807788992405e-08
2267 5.42676631760663e-08
2268 5.42550505397976e-08
2269 5.42416944666613e-08
2270 5.42291326879329e-08
2271 5.42167409762762e-08
2272 5.4204134366298e-08
2273 5.41911516664939e-08
2274 5.41785217418322e-08
2275 5.41658885064855e-08
2276 5.41532128037758e-08
2277 5.41401672204955e-08
2278 5.41281675232597e-08
2279 5.4115040144076e-08
2280 5.41024726832262e-08
2281 5.40899771261927e-08
2282 5.40773536576999e-08
2283 5.40647573439301e-08
2284 5.40517986933331e-08
2285 5.40396292025846e-08
2286 5.40268758961737e-08
2287 5.40146578438261e-08
2288 5.40018573857992e-08
2289 5.3989353895556e-08
2290 5.397751413172e-08
2291 5.39646197657007e-08
2292 5.39519888635986e-08
2293 5.39397653427365e-08
2294 5.39275156188346e-08
2295 5.39144402043057e-08
2296 5.39025641210777e-08
2297 5.3890759901698e-08
2298 5.38780226051472e-08
2299 5.38650127559492e-08
2300 5.3853666642123e-08
2301 5.384082620985e-08
2302 5.38283238324944e-08
2303 5.38161603849119e-08
2304 5.38045321345493e-08
2305 5.37912376086247e-08
2306 5.37796753805608e-08
2307 5.37679165648619e-08
2308 5.37548666463827e-08
2309 5.37428559694852e-08
2310 5.37309410884035e-08
2311 5.37183534143892e-08
2312 5.37062191656723e-08
2313 5.36943471445284e-08
2314 5.36816321354827e-08
2315 5.36701197786371e-08
2316 5.36580332810566e-08
2317 5.36459649693732e-08
2318 5.36341540442464e-08
2319 5.36216505611087e-08
2320 5.36096678351861e-08
2321 5.35977823790112e-08
2322 5.35861976604934e-08
2323 5.3573500035764e-08
2324 5.35619389645525e-08
2325 5.35501586469422e-08
2326 5.35371980512345e-08
2327 5.35268798409483e-08
2328 5.35138471899899e-08
2329 5.35022319234635e-08
2330 5.34900704360908e-08
2331 5.34783751939827e-08
2332 5.34664855984524e-08
2333 5.34544134720427e-08
2334 5.34427052993891e-08
2335 5.34311436086732e-08
2336 5.34187568876909e-08
2337 5.3407261539018e-08
2338 5.33952888814859e-08
2339 5.33838175096335e-08
2340 5.33719913544672e-08
2341 5.33601887582336e-08
2342 5.33482077025305e-08
2343 5.33365912316341e-08
2344 5.33250559868748e-08
2345 5.33130521298553e-08
2346 5.33012059875659e-08
2347 5.32899723695301e-08
2348 5.32776720252315e-08
2349 5.32664184920151e-08
2350 5.32547889680046e-08
2351 5.32426941890485e-08
2352 5.32317872323773e-08
2353 5.32194909759198e-08
2354 5.32080617698938e-08
2355 5.31962997110114e-08
2356 5.3184960914443e-08
2357 5.3173591135991e-08
2358 5.3161965357873e-08
2359 5.31501543350465e-08
2360 5.31387352857848e-08
2361 5.31276607889097e-08
2362 5.31157059247001e-08
2363 5.31040784155223e-08
2364 5.30920450461103e-08
2365 5.30810719285846e-08
2366 5.30696133353992e-08
2367 5.30578893687128e-08
2368 5.30465968275884e-08
2369 5.30349697851484e-08
2370 5.3023637902605e-08
2371 5.30116864569941e-08
2372 5.30010882648924e-08
2373 5.29890726048698e-08
2374 5.29777438518231e-08
2375 5.29661127730563e-08
2376 5.29545799161646e-08
2377 5.29434146110397e-08
2378 5.29318763176079e-08
2379 5.29205744244088e-08
2380 5.2909040974658e-08
2381 5.28977846152578e-08
2382 5.28862899469296e-08
2383 5.28752278237121e-08
2384 5.28635700729474e-08
2385 5.2852796804892e-08
2386 5.28412751807927e-08
2387 5.28300094067014e-08
2388 5.28184149231059e-08
2389 5.28070148408943e-08
2390 5.27960676768302e-08
2391 5.2784434748876e-08
2392 5.27733174697786e-08
2393 5.27617769936484e-08
2394 5.27503903269277e-08
2395 5.27397087837223e-08
2396 5.27282371751703e-08
2397 5.2716996909119e-08
2398 5.27056091086386e-08
2399 5.26946892089875e-08
2400 5.26830644602683e-08
2401 5.26717653679398e-08
2402 5.26611133166988e-08
2403 5.26494016752643e-08
2404 5.26381867138603e-08
2405 5.26272105236814e-08
2406 5.26162336358382e-08
2407 5.26045683817422e-08
2408 5.25938897162348e-08
2409 5.25822938088893e-08
2410 5.25715321511022e-08
2411 5.25601696366174e-08
2412 5.25491277865164e-08
2413 5.25381331750729e-08
2414 5.25265963968735e-08
2415 5.25156023964968e-08
2416 5.2504407084264e-08
2417 5.24936214167937e-08
2418 5.24825621450731e-08
2419 5.24714670273596e-08
2420 5.24605147331769e-08
2421 5.24492050315573e-08
2422 5.24383884084045e-08
2423 5.24272127084835e-08
2424 5.24159853956263e-08
2425 5.240512385285e-08
2426 5.23938264236357e-08
2427 5.23830141760939e-08
2428 5.23720458009969e-08
2429 5.23608165745593e-08
2430 5.23498841156211e-08
2431 5.23388411557413e-08
2432 5.23277378703391e-08
2433 5.23167031381e-08
2434 5.23064686888119e-08
2435 5.22945508669714e-08
2436 5.22844241541698e-08
2437 5.22731827756751e-08
2438 5.22623361112196e-08
2439 5.22509060161269e-08
2440 5.2240329960096e-08
2441 5.22294033369342e-08
2442 5.2218077827515e-08
2443 5.22076633235891e-08
2444 5.21965300896987e-08
2445 5.21856765205442e-08
2446 5.21743733590263e-08
2447 5.21638991952678e-08
2448 5.21531708055356e-08
2449 5.2142148958545e-08
2450 5.21311222669851e-08
2451 5.21203316914409e-08
2452 5.2109152736346e-08
2453 5.20987675654006e-08
2454 5.20877918459561e-08
2455 5.20763534148649e-08
2456 5.20663365866092e-08
2457 5.20548942541943e-08
2458 5.20446360741111e-08
2459 5.20332640672194e-08
2460 5.20232144931576e-08
2461 5.20118980378292e-08
2462 5.20006736133283e-08
2463 5.19904842470886e-08
2464 5.19796551050611e-08
2465 5.1968275708969e-08
2466 5.19574826784996e-08
2467 5.19469289672614e-08
2468 5.19362084396846e-08
2469 5.19253479329684e-08
2470 5.19144484822931e-08
2471 5.19039517197228e-08
2472 5.18928146400199e-08
2473 5.18824449229349e-08
2474 5.18714465433945e-08
2475 5.18612937847607e-08
2476 5.18501354695999e-08
2477 5.18395061579469e-08
2478 5.18285695023657e-08
2479 5.18176894175326e-08
2480 5.18069600272675e-08
2481 5.17968096929167e-08
2482 5.17855667392375e-08
2483 5.1774963037321e-08
2484 5.17641849810957e-08
2485 5.17534536905728e-08
2486 5.17427823436556e-08
2487 5.17324163795685e-08
2488 5.1721510351932e-08
2489 5.17106449371418e-08
2490 5.17000826452119e-08
2491 5.16889571384738e-08
2492 5.16785386492913e-08
2493 5.16681516549156e-08
2494 5.16569866122474e-08
2495 5.16463043616078e-08
2496 5.1635985919507e-08
2497 5.16253145619316e-08
2498 5.161485142402e-08
2499 5.16037366042887e-08
2500 5.15934786644578e-08
2501 5.1582801216199e-08
2502 5.15723225320563e-08
2503 5.15615784961732e-08
2504 5.15508904341821e-08
2505 5.15405790282308e-08
2506 5.15295014436035e-08
2507 5.15197736175033e-08
2508 5.15084386649711e-08
2509 5.14983732005092e-08
2510 5.14874493853235e-08
2511 5.14769517772073e-08
2512 5.14669048192751e-08
2513 5.14555833959207e-08
2514 5.14453249356173e-08
2515 5.14347424194206e-08
2516 5.14245945968383e-08
2517 5.14140624976456e-08
2518 5.14037816121693e-08
2519 5.13929656542622e-08
2520 5.13823795151858e-08
2521 5.13721332597861e-08
2522 5.13620594588815e-08
2523 5.13510915367554e-08
2524 5.13406594842003e-08
2525 5.13303205305071e-08
2526 5.13200629201904e-08
2527 5.13096079854947e-08
2528 5.12991689802789e-08
2529 5.1289079183281e-08
2530 5.12784821005141e-08
2531 5.12679276201133e-08
2532 5.12578673848019e-08
2533 5.12478138143813e-08
2534 5.12372843841646e-08
2535 5.12267083232487e-08
2536 5.12168118569356e-08
2537 5.12065886719704e-08
2538 5.11966284437726e-08
2539 5.1185923403807e-08
2540 5.11755389882573e-08
2541 5.11655953063794e-08
2542 5.11554886686305e-08
2543 5.11451659450657e-08
2544 5.1135221363463e-08
2545 5.11249661543367e-08
2546 5.11147617303642e-08
2547 5.11047761970751e-08
2548 5.10945099265747e-08
2549 5.10842889784868e-08
2550 5.10740572545743e-08
2551 5.10640945665664e-08
2552 5.10536336837397e-08
2553 5.10439782384076e-08
2554 5.10337757999579e-08
2555 5.10236878370485e-08
2556 5.10131991355856e-08
2557 5.10037320049506e-08
2558 5.09933559245113e-08
2559 5.09830175472459e-08
2560 5.09728950754962e-08
2561 5.09635258723051e-08
2562 5.0953017222799e-08
2563 5.09430844295444e-08
2564 5.09331271407731e-08
2565 5.09229256091537e-08
2566 5.09129675996256e-08
2567 5.0902960678556e-08
2568 5.08933575518178e-08
2569 5.08827215743857e-08
2570 5.08730351538311e-08
2571 5.08629488624734e-08
2572 5.08529620013576e-08
2573 5.08431730636971e-08
2574 5.08329413833053e-08
2575 5.08228240736486e-08
2576 5.08127371503519e-08
2577 5.08030307861951e-08
2578 5.07929758257752e-08
2579 5.0783174105451e-08
2580 5.07728399776752e-08
2581 5.07634230078047e-08
2582 5.0753447736529e-08
2583 5.07432706102762e-08
2584 5.07329945422796e-08
2585 5.07234969409076e-08
2586 5.07132196490012e-08
2587 5.07033854151295e-08
2588 5.06939935216444e-08
2589 5.06835046953924e-08
2590 5.06745118102891e-08
2591 5.06633872330298e-08
2592 5.06539978166742e-08
2593 5.0644296752278e-08
2594 5.06340181005704e-08
2595 5.06246860356718e-08
2596 5.06147010121971e-08
2597 5.06047436004131e-08
2598 5.05952245588581e-08
2599 5.05849997907148e-08
2600 5.05751387089859e-08
2601 5.05656468963167e-08
2602 5.05558967320852e-08
2603 5.05454984840448e-08
2604 5.05364477261239e-08
2605 5.05266086019418e-08
2606 5.05160816999251e-08
2607 5.0506258455929e-08
2608 5.04970900547796e-08
2609 5.04874504660968e-08
2610 5.04776745806623e-08
2611 5.04674778389891e-08
2612 5.04579845950204e-08
2613 5.04481792793499e-08
2614 5.0438900224492e-08
2615 5.04287196183562e-08
2616 5.04190652144132e-08
2617 5.04093769819747e-08
2618 5.03999058483196e-08
2619 5.03898541128756e-08
2620 5.03801243509905e-08
2621 5.03709683039588e-08
2622 5.03607866186861e-08
2623 5.03513544853895e-08
2624 5.03417930386441e-08
2625 5.03317552973392e-08
2626 5.03221801984033e-08
2627 5.0312648441242e-08
2628 5.03030580643049e-08
2629 5.02936459017e-08
2630 5.0283655455452e-08
2631 5.0273824152125e-08
2632 5.0264379178433e-08
2633 5.02552306826942e-08
2634 5.02449512058689e-08
2635 5.02360462260043e-08
2636 5.02258893355645e-08
2637 5.02168655942548e-08
2638 5.02074949362274e-08
2639 5.01972568289766e-08
2640 5.01878706447023e-08
2641 5.01788406044312e-08
2642 5.01686946225988e-08
2643 5.01595738859883e-08
2644 5.01499919050019e-08
2645 5.01403488035734e-08
2646 5.01313008616222e-08
2647 5.01211588965766e-08
2648 5.01118204727646e-08
2649 5.01021094274634e-08
2650 5.00931736868715e-08
2651 5.00834216428103e-08
2652 5.00745777505252e-08
2653 5.00640287377507e-08
2654 5.00552292739265e-08
2655 5.00457261769505e-08
2656 5.00359083952517e-08
2657 5.0026831518224e-08
2658 5.0017025050586e-08
2659 5.00082448260386e-08
2660 4.99985574089479e-08
2661 4.99895234939984e-08
2662 4.99799196571615e-08
2663 4.99700779403867e-08
2664 4.99608584250133e-08
2665 4.99513193790158e-08
2666 4.99423537982935e-08
2667 4.99325888752011e-08
2668 4.99237645108508e-08
2669 4.99141368899281e-08
2670 4.99047291131482e-08
2671 4.98953040719563e-08
2672 4.98854875203847e-08
2673 4.98771655692565e-08
2674 4.98672022355429e-08
2675 4.98582844539186e-08
2676 4.98488171518652e-08
2677 4.98392431103056e-08
2678 4.98305187148773e-08
2679 4.9820977375159e-08
2680 4.98113722482429e-08
2681 4.9802155916101e-08
2682 4.97933643113235e-08
2683 4.97838558644048e-08
2684 4.97747728882558e-08
2685 4.97652012452221e-08
2686 4.97562244228256e-08
2687 4.97466424076443e-08
2688 4.97377425721091e-08
2689 4.97282886451345e-08
2690 4.97193832567078e-08
2691 4.97092496423157e-08
2692 4.97004891730768e-08
2693 4.96920640409293e-08
2694 4.96824172016019e-08
2695 4.96731009236306e-08
2696 4.96640779412694e-08
2697 4.96545003620952e-08
2698 4.96453415355091e-08
2699 4.96360539483653e-08
2700 4.96273580266049e-08
2701 4.96180026936521e-08
2702 4.96089439816494e-08
2703 4.95994620481888e-08
2704 4.95906923219103e-08
2705 4.95813634429076e-08
2706 4.957192931343e-08
2707 4.95632310446581e-08
2708 4.95538367664139e-08
2709 4.95447999826482e-08
2710 4.9535960257252e-08
2711 4.95266469715538e-08
2712 4.95176187453872e-08
2713 4.95085278719998e-08
2714 4.94990406973983e-08
2715 4.94907468158168e-08
2716 4.94806270410209e-08
2717 4.94725069302149e-08
2718 4.94632808321072e-08
2719 4.94538611657269e-08
2720 4.9445463450315e-08
2721 4.94358368574588e-08
2722 4.94272394853112e-08
2723 4.9417838043464e-08
2724 4.9409269631262e-08
2725 4.94000368220782e-08
2726 4.9390704674579e-08
2727 4.93819114351091e-08
2728 4.93734956248382e-08
2729 4.93639822392034e-08
2730 4.93552918046092e-08
2731 4.93467524567137e-08
2732 4.93373560161991e-08
2733 4.93283679237067e-08
2734 4.9319177921614e-08
2735 4.93110124244467e-08
2736 4.93017533189644e-08
2737 4.92931105271488e-08
2738 4.92840484067614e-08
2739 4.92749252773983e-08
2740 4.92658204871432e-08
2741 4.92576675790168e-08
2742 4.92483313236924e-08
2743 4.92395512172727e-08
2744 4.92308344215431e-08
2745 4.92217969982889e-08
2746 4.92131395919415e-08
2747 4.92045146471831e-08
2748 4.91955615866679e-08
2749 4.91860865805549e-08
2750 4.91780796547658e-08
2751 4.91692452206927e-08
2752 4.91603700045751e-08
2753 4.91519841738786e-08
2754 4.91423922865941e-08
2755 4.91339435351179e-08
2756 4.91248099807606e-08
2757 4.91169959127014e-08
2758 4.91076915811739e-08
2759 4.90990929238322e-08
2760 4.90901526077892e-08
2761 4.90814300824205e-08
2762 4.90729761346564e-08
2763 4.90642092505489e-08
2764 4.90550703560189e-08
2765 4.90470178733382e-08
2766 4.90377963413557e-08
2767 4.90295869428614e-08
2768 4.90205679524181e-08
2769 4.90119479144013e-08
2770 4.90034031952469e-08
2771 4.89943999015807e-08
2772 4.89859372927448e-08
2773 4.89771790102012e-08
2774 4.89686059448324e-08
2775 4.8959915922353e-08
2776 4.8951229450811e-08
2777 4.89423793088584e-08
2778 4.89334420996634e-08
2779 4.89250993163104e-08
2780 4.89159573469067e-08
2781 4.89072071183649e-08
2782 4.88984529152248e-08
2783 4.88893033994131e-08
2784 4.88809011658375e-08
2785 4.88714245974187e-08
2786 4.88628701229565e-08
2787 4.8854593325931e-08
2788 4.88457639251649e-08
2789 4.88373502478545e-08
2790 4.88288066571307e-08
2791 4.88205483764048e-08
2792 4.88120892705446e-08
2793 4.88039056105727e-08
2794 4.87961982940099e-08
2795 4.87872363659214e-08
2796 4.87787400227369e-08
2797 4.87709219560983e-08
2798 4.87631285825962e-08
2799 4.87545283762714e-08
2800 4.87462961626939e-08
2801 4.8738419796468e-08
2802 4.87310882220093e-08
2803 4.87229763805885e-08
2804 4.87150583636797e-08
2805 4.87078670490249e-08
2806 4.87006175031723e-08
2807 4.86935046533077e-08
2808 4.86862758255491e-08
2809 4.86786974192022e-08
2810 4.86709121907403e-08
2811 4.86636623935333e-08
2812 4.86562548989156e-08
2813 4.86486000084163e-08
2814 4.86409527402643e-08
2815 4.86336478031646e-08
2816 4.86261195558058e-08
2817 4.8618258657207e-08
2818 4.86110485486968e-08
2819 4.86034180169348e-08
2820 4.85958848170931e-08
2821 4.8588898541535e-08
2822 4.858096243332e-08
2823 4.85746972906931e-08
2824 4.85669904608521e-08
2825 4.85603401587653e-08
2826 4.85530345599727e-08
2827 4.85463649066986e-08
2828 4.85395248248643e-08
2829 4.85330665092043e-08
2830 4.85274017942849e-08
2831 4.85207428959633e-08
2832 4.85172314683702e-08
2833 4.851477319745e-08
2834 4.8517895840483e-08
2835 4.85327512884659e-08
2836 4.85258381255349e-08
2837 4.85095251359091e-08
2838 4.8505703801105e-08
2839 4.8493684574602e-08
2840 4.84858830263057e-08
2841 4.84777715308304e-08
2842 4.84683994610435e-08
2843 4.84616543618976e-08
2844 4.84522239205809e-08
2845 4.84448575042151e-08
2846 4.84355195040642e-08
2847 4.84282173811579e-08
2848 4.84191261849176e-08
2849 4.84117136463347e-08
2850 4.84030406164848e-08
2851 4.83957399675106e-08
2852 4.83865538014605e-08
2853 4.83799515778038e-08
2854 4.83703483453724e-08
2855 4.83633836090291e-08
2856 4.83544943210568e-08
2857 4.83467877763211e-08
2858 4.83382330158655e-08
2859 4.83315956003594e-08
2860 4.83224420175787e-08
2861 4.83156216488645e-08
2862 4.83063019123264e-08
2863 4.82996076329734e-08
2864 4.82907144898626e-08
2865 4.82837196815744e-08
2866 4.82746770411602e-08
2867 4.82677539208609e-08
2868 4.82586315415645e-08
2869 4.82526668816874e-08
2870 4.82424363839051e-08
2871 4.82367807439488e-08
2872 4.82275800779419e-08
2873 4.82203605915998e-08
2874 4.82111602506663e-08
2875 4.82061801059253e-08
2876 4.81956210078849e-08
2877 4.8189758936168e-08
2878 4.81802465839287e-08
2879 4.81738711934909e-08
2880 4.81644899004152e-08
2881 4.8158243030727e-08
2882 4.81495574891078e-08
2883 4.8142624510028e-08
2884 4.81333050901256e-08
2885 4.81279875543805e-08
2886 4.8117789576807e-08
2887 4.81124019247048e-08
2888 4.81022760392413e-08
2889 4.80974170238291e-08
2890 4.80870637442621e-08
2891 4.80818232069247e-08
2892 4.80716343007614e-08
2893 4.80665415847348e-08
2894 4.80560456592727e-08
2895 4.80516343204229e-08
2896 4.80407959342344e-08
2897 4.80364926729315e-08
2898 4.80259197486177e-08
2899 4.80210590465546e-08
2900 4.8009710083452e-08
2901 4.80067014336782e-08
2902 4.7994448816091e-08
2903 4.79919196818557e-08
2904 4.79794963159641e-08
2905 4.79769559720289e-08
2906 4.79637635604213e-08
2907 4.79619073354698e-08
2908 4.79488375013659e-08
2909 4.79476115651245e-08
2910 4.79339154262348e-08
2911 4.79319667143763e-08
2912 4.79197733631587e-08
2913 4.79175381697949e-08
2914 4.79040887548798e-08
2915 4.79033260889494e-08
2916 4.78892185791224e-08
2917 4.7888992365408e-08
2918 4.78749354768659e-08
2919 4.78744376888329e-08
2920 4.78593440180752e-08
2921 4.78606350520216e-08
2922 4.78450237118011e-08
2923 4.78467302973407e-08
2924 4.78298244597219e-08
2925 4.78331240776342e-08
2926 4.78155823917525e-08
2927 4.78184183987374e-08
2928 4.78014305835828e-08
2929 4.78040811566771e-08
2930 4.77871511752603e-08
2931 4.77905356293107e-08
2932 4.7774162007741e-08
2933 4.77754411232212e-08
2934 4.77610678979445e-08
2935 4.77606157334165e-08
2936 4.77491653727569e-08
2937 4.77441531416822e-08
2938 4.7737422113503e-08
2939 4.77270381300521e-08
2940 4.77266281531108e-08
2941 4.7707916823736e-08
2942 4.77163090000232e-08
2943 4.7682836588514e-08
2944 4.77277188628555e-08
2945 4.7659766625685e-08
2946 4.77394539690756e-08
2947 4.76513633929088e-08
2948 4.77016499131899e-08
2949 4.76240083995272e-08
2950 4.77243829473117e-08
2951 4.76211472246746e-08
2952 4.76723028475234e-08
2953 4.75948762539424e-08
2954 4.76908761988248e-08
2955 4.75906860386566e-08
2956 4.76497656327624e-08
2957 4.75670998438815e-08
2958 4.76550323140756e-08
2959 4.75577942689043e-08
2960 4.76287894621841e-08
2961 4.75394036842935e-08
2962 4.7619938018606e-08
2963 4.75243412418891e-08
2964 4.76038110130972e-08
2965 4.75088391658218e-08
2966 4.75895571261376e-08
2967 4.74940765839982e-08
2968 4.75752108917149e-08
2969 4.74795132556771e-08
2970 4.75607319492255e-08
2971 4.7466432000931e-08
2972 4.75459804971656e-08
2973 4.74528440168598e-08
2974 4.75315476275995e-08
2975 4.74394591130078e-08
2976 4.75172294773785e-08
2977 4.74272999571035e-08
2978 4.75033990809948e-08
2979 4.74141548676421e-08
2980 4.74898522297984e-08
2981 4.74014718720994e-08
2982 4.74766985703035e-08
2983 4.73871409574222e-08
2984 4.74645921682004e-08
2985 4.73737990915879e-08
2986 4.74522756830353e-08
2987 4.73597545234128e-08
2988 4.74408481254684e-08
2989 4.73455576521786e-08
2990 4.7429802624066e-08
2991 4.73316392701761e-08
2992 4.74185972905161e-08
2993 4.7316766088823e-08
2994 4.74076468619522e-08
2995 4.73041044495304e-08
2996 4.73962685267892e-08
2997 4.72906361794756e-08
2998 4.73855904257192e-08
2999 4.72788650629497e-08
3000 2.69082619881178e-08
3001 2.68119144821166e-08
3002 2.7011572029223e-08
3003 2.7109739271336e-08
3004 2.71448910365013e-08
3005 2.71528955149702e-08
3006 2.71520011883575e-08
3007 2.71484613619255e-08
3008 2.71443745739064e-08
3009 2.71403358864664e-08
3010 2.71365058052697e-08
3011 2.71328561550765e-08
3012 2.71293326057909e-08
3013 2.71259729436868e-08
3014 2.71226868782115e-08
3015 2.71196133329044e-08
3016 2.71166109586107e-08
3017 2.71136901725533e-08
3018 2.7110733207325e-08
3019 2.71079996881918e-08
3020 2.71052443618913e-08
3021 2.71025489523291e-08
3022 2.70999249263326e-08
3023 2.70973169497757e-08
3024 2.70947304499281e-08
3025 2.70922196153878e-08
3026 2.70896932970111e-08
3027 2.70873055037701e-08
3028 2.70848000217816e-08
3029 2.70824103821288e-08
3030 2.70800189048348e-08
3031 2.7077615197324e-08
3032 2.70753295866788e-08
3033 2.70729566536065e-08
3034 2.70706677049648e-08
3035 2.70683396906279e-08
3036 2.70660157830616e-08
3037 2.70637386534101e-08
3038 2.70614811612169e-08
3039 2.70592555861038e-08
3040 2.70570167419382e-08
3041 2.70548207687571e-08
3042 2.70526150401573e-08
3043 2.70504420697981e-08
3044 2.70482643789927e-08
3045 2.7046077679338e-08
3046 2.70439035506276e-08
3047 2.70418016590224e-08
3048 2.70396410537166e-08
3049 2.70374816306873e-08
3050 2.70353341455531e-08
3051 2.70331835723892e-08
3052 2.70310437265886e-08
3053 2.70289381132383e-08
3054 2.7026853674561e-08
3055 2.70247153602576e-08
3056 2.70226304887045e-08
3057 2.70205478125063e-08
3058 2.70184378016736e-08
3059 2.70163705654025e-08
3060 2.70142615143021e-08
3061 2.70122697287323e-08
3062 2.70101719175853e-08
3063 2.70080863306599e-08
3064 2.70060449832354e-08
3065 2.7004026985189e-08
3066 2.70019850477921e-08
3067 2.69999722253278e-08
3068 2.69979256492525e-08
3069 2.69959095475225e-08
3070 2.69938792600288e-08
3071 2.69918329202645e-08
3072 2.69898305782501e-08
3073 2.69877831459708e-08
3074 2.69857761947545e-08
3075 2.69837579589538e-08
3076 2.69817781133885e-08
3077 2.69797681303752e-08
3078 2.69778118368036e-08
3079 2.69758382036578e-08
3080 2.69738176667533e-08
3081 2.69718486450299e-08
3082 2.69698449285039e-08
3083 2.6967890037588e-08
3084 2.69659626075391e-08
3085 2.69639685667733e-08
3086 2.69620424172001e-08
3087 2.69600715250284e-08
3088 2.69580856804286e-08
3089 2.69561716935707e-08
3090 2.69542069876172e-08
3091 2.69522734311356e-08
3092 2.69503266800419e-08
3093 2.69483982567875e-08
3094 2.69464541129971e-08
3095 2.69445015443348e-08
3096 2.69425367767639e-08
3097 2.69406471852252e-08
3098 2.69386577716468e-08
3099 2.69367676333232e-08
3100 2.6934813283086e-08
3101 2.69329303120291e-08
3102 2.69310056050798e-08
3103 2.69291031945729e-08
3104 2.69271506177504e-08
3105 2.69252495918026e-08
3106 2.69233439419425e-08
3107 2.69214774545334e-08
3108 2.69195349019702e-08
3109 2.69176187181475e-08
3110 2.69156697333406e-08
3111 2.69138510284872e-08
3112 2.69119343938584e-08
3113 2.69100624107343e-08
3114 2.69081487321854e-08
3115 2.69062715287927e-08
3116 2.69043985125506e-08
3117 2.69025445546744e-08
3118 2.69006192281096e-08
3119 2.68987329806181e-08
3120 2.68969000017383e-08
3121 2.6894984131054e-08
3122 2.68931440439157e-08
3123 2.6891296832976e-08
3124 2.68893576457763e-08
3125 2.68875145374436e-08
3126 2.68856116657501e-08
3127 2.68838041636577e-08
3128 2.68819522252217e-08
3129 2.68801093060156e-08
3130 2.68782037257664e-08
3131 2.6876344272897e-08
3132 2.6874548543887e-08
3133 2.68726907751704e-08
3134 2.6870867931017e-08
3135 2.68689978490944e-08
3136 2.68671475270321e-08
3137 2.68653372706984e-08
3138 2.68634698999959e-08
3139 2.68616798224097e-08
3140 2.68597899171219e-08
3141 2.68579599513319e-08
3142 2.68560992179867e-08
3143 2.68542664613736e-08
3144 2.6852463899163e-08
3145 2.68506258110368e-08
3146 2.68487877408963e-08
3147 2.68469504069446e-08
3148 2.68451693077876e-08
3149 2.68433567401916e-08
3150 2.68415163287572e-08
3151 2.68397029861145e-08
3152 2.68378894870414e-08
3153 2.68360777835874e-08
3154 2.68342469034732e-08
3155 2.68324592260227e-08
3156 2.68306365270865e-08
3157 2.68288626144586e-08
3158 2.68270196500109e-08
3159 2.68252516755219e-08
3160 2.6823451476532e-08
3161 2.68216452630754e-08
3162 2.68198238538297e-08
3163 2.68180647264304e-08
3164 2.68162429117313e-08
3165 2.68144920568258e-08
3166 2.68127157240783e-08
3167 2.68108955832624e-08
3168 2.68090966267232e-08
3169 2.68072827975807e-08
3170 2.68055035461123e-08
3171 2.68037476556793e-08
3172 2.68019551000198e-08
3173 2.68001708855214e-08
3174 2.67983876584554e-08
3175 2.67966239749229e-08
3176 2.67948414140462e-08
3177 2.67930777118619e-08
3178 2.67913105490014e-08
3179 2.67895007703411e-08
3180 2.67876870403971e-08
3181 2.67859484237576e-08
3182 2.67841556816917e-08
3183 2.67824001476957e-08
3184 2.67806502250445e-08
3185 2.67788401086544e-08
3186 2.67771174237708e-08
3187 2.67753568272688e-08
3188 2.67735793809121e-08
3189 2.67718164084219e-08
3190 2.67700626451761e-08
3191 2.6768312720693e-08
3192 2.67665506464287e-08
3193 2.67648037238222e-08
3194 2.67630718538969e-08
3195 2.67612832368647e-08
3196 2.67595508901541e-08
3197 2.67577888401482e-08
3198 2.67560487356433e-08
3199 2.67543293486772e-08
3200 2.67525891524123e-08
3201 2.67508938102679e-08
3202 2.67491389392416e-08
3203 2.67474051928174e-08
3204 2.67456642981112e-08
3205 2.67439318040186e-08
3206 2.6742198476204e-08
3207 2.67404338130661e-08
3208 2.67387521640239e-08
3209 2.67370192074678e-08
3210 2.67352830105039e-08
3211 2.67335414289027e-08
3212 2.67317627490882e-08
3213 2.67300885694044e-08
3214 2.67283385090855e-08
3215 2.67266499215379e-08
3216 2.67249531519242e-08
3217 2.67232492349723e-08
3218 2.67214908812874e-08
3219 2.67197657907725e-08
3220 2.6718029195294e-08
3221 2.67163710179763e-08
3222 2.67146540360308e-08
3223 2.67128654390936e-08
3224 2.67112437243311e-08
3225 2.67095526753636e-08
3226 2.67078164636891e-08
3227 2.67060773820882e-08
3228 2.67043905395892e-08
3229 2.67026733971609e-08
3230 2.67009846240396e-08
3231 2.66993069917287e-08
3232 2.66976143056263e-08
3233 2.66958734925216e-08
3234 2.6694239133207e-08
3235 2.66925279889252e-08
3236 2.66907898068269e-08
3237 2.66891105543121e-08
3238 2.66874186795052e-08
3239 2.66857349722205e-08
3240 2.66840142590374e-08
3241 2.66822970017566e-08
3242 2.66806371806982e-08
3243 2.6678913915501e-08
3244 2.66772290323236e-08
3245 2.66755876373037e-08
3246 2.6673874017058e-08
3247 2.66721625319932e-08
3248 2.66704906869975e-08
3249 2.66687914338704e-08
3250 2.66671154098841e-08
3251 2.66654296581792e-08
3252 2.66637729897101e-08
3253 2.66621152064661e-08
3254 2.66604262187387e-08
3255 2.66587425059028e-08
3256 2.66570978826874e-08
3257 2.66554142871467e-08
3258 2.66537393371902e-08
3259 2.66520942699966e-08
3260 2.66504635139109e-08
3261 2.66487963510587e-08
3262 2.66471448853167e-08
3263 2.66454781113756e-08
3264 2.66438260316804e-08
3265 2.66421730744648e-08
3266 2.66404853461744e-08
3267 2.66387944355961e-08
3268 2.66371924695918e-08
3269 2.66355602878687e-08
3270 2.66338856209636e-08
3271 2.66321954323079e-08
3272 2.66305261751865e-08
3273 2.66289324556412e-08
3274 2.66272273120038e-08
3275 2.66255565536944e-08
3276 2.66239276022207e-08
3277 2.66222938262728e-08
3278 2.66206479548892e-08
3279 2.66189879770118e-08
3280 2.66173413860926e-08
3281 2.66156574513787e-08
3282 2.66139968926882e-08
3283 2.66123671162077e-08
3284 2.66107263858228e-08
3285 2.660909482749e-08
3286 2.66074690938312e-08
3287 2.66057812041698e-08
3288 2.66042027246005e-08
3289 2.66025799640635e-08
3290 2.66009224518804e-08
3291 2.65993223145244e-08
3292 2.65976686341096e-08
3293 2.65960321312875e-08
3294 2.65944189421496e-08
3295 2.65927859399162e-08
3296 2.65911150938436e-08
3297 2.65895503500091e-08
3298 2.6587969362557e-08
3299 2.65863130328725e-08
3300 2.65846912694268e-08
3301 2.65830899227049e-08
3302 2.65814855468505e-08
3303 2.65798756660662e-08
3304 2.65782289209926e-08
3305 2.65765701490395e-08
3306 2.65749863773701e-08
3307 2.65734039918697e-08
3308 2.65717478077354e-08
3309 2.65701685555619e-08
3310 2.65685797014026e-08
3311 2.65669447837569e-08
3312 2.65653156888423e-08
3313 2.6563683239722e-08
3314 2.65621009994388e-08
3315 2.6560479484905e-08
3316 2.65588602108013e-08
3317 2.65572687236371e-08
3318 2.65556544795653e-08
3319 2.65540652067409e-08
3320 2.65524312101362e-08
3321 2.65508078632903e-08
3322 2.65491952223806e-08
3323 2.65475503245516e-08
3324 2.65459972245696e-08
3325 2.65443689101974e-08
3326 2.65428378860566e-08
3327 2.65411576849672e-08
3328 2.65395993251372e-08
3329 2.65380027000273e-08
3330 2.65364076728614e-08
3331 2.65348229137596e-08
3332 2.65332272222918e-08
3333 2.65316313362574e-08
3334 2.65300436161353e-08
3335 2.65284842490554e-08
3336 2.65268832522758e-08
3337 2.65252693468776e-08
3338 2.65236949016923e-08
3339 2.65221680662564e-08
3340 2.65205592900331e-08
3341 2.65189807711619e-08
3342 2.65173689134035e-08
3343 2.65157993116771e-08
3344 2.65142117064077e-08
3345 2.65126337212207e-08
3346 2.65111010515073e-08
3347 2.65095061202647e-08
3348 2.65078836124699e-08
3349 2.65063173525149e-08
3350 2.65047348115832e-08
3351 2.65031076389644e-08
3352 2.65015730531171e-08
3353 2.6500002927865e-08
3354 2.64984571696214e-08
3355 2.64969063416554e-08
3356 2.64953254232592e-08
3357 2.64937743512106e-08
3358 2.64922377470334e-08
3359 2.64906392221653e-08
3360 2.64890569335874e-08
3361 2.64874767101353e-08
3362 2.64859608727908e-08
3363 2.64843609152354e-08
3364 2.64828067695899e-08
3365 2.64812491210242e-08
3366 2.64796856345728e-08
3367 2.647812219575e-08
3368 2.64765702718828e-08
3369 2.64750171785066e-08
3370 2.64734680288092e-08
3371 2.64719269063574e-08
3372 2.64703302301217e-08
3373 2.64687639030536e-08
3374 2.64672107587738e-08
3375 2.64656370137506e-08
3376 2.64640882993716e-08
3377 2.64625812597674e-08
3378 2.6461032062941e-08
3379 2.64595111189592e-08
3380 2.64578804074489e-08
3381 2.64563501768955e-08
3382 2.6454862515124e-08
3383 2.64532416464069e-08
3384 2.64517373265716e-08
3385 2.64501650869553e-08
3386 2.64485976028683e-08
3387 2.64470570610631e-08
3388 2.64455727456148e-08
3389 2.64439865151345e-08
3390 2.64424439688771e-08
3391 2.64409465791093e-08
3392 2.64393780050609e-08
3393 2.64378466699466e-08
3394 2.64362936458484e-08
3395 2.64347852935165e-08
3396 2.64332432871606e-08
3397 2.64317444516049e-08
3398 2.64301636420106e-08
3399 2.6428702483261e-08
3400 2.64271171644959e-08
3401 2.64256159543397e-08
3402 2.64240875023636e-08
3403 2.64225195933943e-08
3404 2.64209845219909e-08
3405 2.64194837963916e-08
3406 2.64179266862841e-08
3407 2.64164338009687e-08
3408 2.64148671825226e-08
3409 2.6413327696484e-08
3410 2.64118641571387e-08
3411 2.64103064883114e-08
3412 2.64087573814686e-08
3413 2.64072834896822e-08
3414 2.64057817902508e-08
3415 2.64042109957563e-08
3416 2.64027427120395e-08
3417 2.64012086323984e-08
3418 2.63996987537318e-08
3419 2.63982012670971e-08
3420 2.63966236797586e-08
3421 2.63951594944856e-08
3422 2.63936257768327e-08
3423 2.63921267195655e-08
3424 2.6390603629467e-08
3425 2.63890868830718e-08
3426 2.6387642638126e-08
3427 2.63861155218592e-08
3428 2.63846078513708e-08
3429 2.63831091510958e-08
3430 2.6381598594527e-08
3431 2.6380074874377e-08
3432 2.63786065483607e-08
3433 2.6377107508746e-08
3434 2.63755919637787e-08
3435 2.63740725288675e-08
3436 2.63725634909728e-08
3437 2.63711030232816e-08
3438 2.63695997482771e-08
3439 2.6368094317275e-08
3440 2.63665832125892e-08
3441 2.63650985397046e-08
3442 2.63635988191346e-08
3443 2.6362102704125e-08
3444 2.63605799710187e-08
3445 2.63591054210366e-08
3446 2.63575903611257e-08
3447 2.63561007064483e-08
3448 2.63545764575546e-08
3449 2.63530892302133e-08
3450 2.6351583639117e-08
3451 2.6350088547844e-08
3452 2.63486116058864e-08
3453 2.63471224523082e-08
3454 2.63455948638636e-08
3455 2.63441759943417e-08
3456 2.63426474469419e-08
3457 2.63411571583827e-08
3458 2.63396573214614e-08
3459 2.63381709035837e-08
3460 2.63366919030505e-08
3461 2.63352238061842e-08
3462 2.63337708837907e-08
3463 2.63322443183056e-08
3464 2.63307825281722e-08
3465 2.63293603467996e-08
3466 2.63277957439079e-08
3467 2.63263697923843e-08
3468 2.63249392384868e-08
3469 2.63234170908566e-08
3470 2.63219423411454e-08
3471 2.63204858698685e-08
3472 2.63189896626548e-08
3473 2.63175295908136e-08
3474 2.63160578277688e-08
3475 2.63145561108513e-08
3476 2.63130619575502e-08
3477 2.6311588972816e-08
3478 2.63101473065852e-08
3479 2.63086820331271e-08
3480 2.6307194821884e-08
3481 2.63057122975585e-08
3482 2.63042721116991e-08
3483 2.63027913267044e-08
3484 2.63013181233118e-08
3485 2.62998095357791e-08
3486 2.62983741030065e-08
3487 2.62969187437845e-08
3488 2.62954357481693e-08
3489 2.629394654724e-08
3490 2.62925127805791e-08
3491 2.62910724779797e-08
3492 2.62895804414853e-08
3493 2.62881536079451e-08
3494 2.6286644139506e-08
3495 2.62852132963953e-08
3496 2.62837379207959e-08
3497 2.62822829080744e-08
3498 2.62808554961635e-08
3499 2.62793783291082e-08
3500 2.62779463277574e-08
3501 2.62765206107129e-08
3502 2.62750475584217e-08
3503 2.62735823680638e-08
3504 2.62721520389864e-08
3505 2.62706921849154e-08
3506 2.62692419392696e-08
3507 2.62678047583953e-08
3508 2.62663654811424e-08
3509 2.62649356491673e-08
3510 2.62634746870383e-08
3511 2.62620601051977e-08
3512 2.62606101077978e-08
3513 2.62591581066673e-08
3514 2.62577068920189e-08
3515 2.62562590775395e-08
3516 2.62548149925212e-08
3517 2.62534071940879e-08
3518 2.6251944520328e-08
3519 2.62504824695142e-08
3520 2.62490928481518e-08
3521 2.62476214866192e-08
3522 2.62461997953545e-08
3523 2.62447341069505e-08
3524 2.62432832323634e-08
3525 2.62418078637583e-08
3526 2.62404254169901e-08
3527 2.62389467184931e-08
3528 2.62375732910103e-08
3529 2.62360972341225e-08
3530 2.62346668091773e-08
3531 2.62332629401008e-08
3532 2.62317914655474e-08
3533 2.62303787184615e-08
3534 2.62289394962201e-08
3535 2.62275396976097e-08
3536 2.62260817693871e-08
3537 2.6224645020223e-08
3538 2.62232709310473e-08
3539 2.62218086468091e-08
3540 2.62203748884748e-08
3541 2.62189582120875e-08
3542 2.62175350573823e-08
3543 2.62161087132284e-08
3544 2.62146997020984e-08
3545 2.62132336396603e-08
3546 2.62118120995525e-08
3547 2.62103585522699e-08
3548 2.6208965744845e-08
3549 2.62075206433066e-08
3550 2.62061203899489e-08
3551 2.62046727683152e-08
3552 2.62032979632676e-08
3553 2.62018587328661e-08
3554 2.62004355043866e-08
3555 2.61990029897796e-08
3556 2.61976193896007e-08
3557 2.61962010725258e-08
3558 2.61947537140705e-08
3559 2.61933649688406e-08
3560 2.61919193615956e-08
3561 2.61905483148195e-08
3562 2.6189089144979e-08
3563 2.61877171096603e-08
3564 2.6186263461403e-08
3565 2.61848503305129e-08
3566 2.61834368182612e-08
3567 2.61820211659991e-08
3568 2.61806633437001e-08
3569 2.61792646077952e-08
3570 2.6177834275054e-08
3571 2.61764303266521e-08
3572 2.61750305198261e-08
3573 2.61736195255602e-08
3574 2.61721853500596e-08
3575 2.61708006503714e-08
3576 2.6169389091113e-08
3577 2.61680383211171e-08
3578 2.61666049974907e-08
3579 2.61652067944929e-08
3580 2.61638119197216e-08
3581 2.61624249638381e-08
3582 2.6161000432845e-08
3583 2.61596320145219e-08
3584 2.61582120245074e-08
3585 2.61568594326356e-08
3586 2.61554390361685e-08
3587 2.61540424100204e-08
3588 2.61526581698535e-08
3589 2.61512517656937e-08
3590 2.61498437749141e-08
3591 2.61485160191754e-08
3592 2.61471162972815e-08
3593 2.6145698785951e-08
3594 2.61443011306817e-08
3595 2.61429293166859e-08
3596 2.6141511536959e-08
3597 2.61401266954397e-08
3598 2.61387085194742e-08
3599 2.61373375153862e-08
3600 2.6135918542558e-08
3601 2.61345808102997e-08
3602 2.61331472053428e-08
3603 2.61318088179419e-08
3604 2.6130369961519e-08
3605 2.61290281871052e-08
3606 2.61276052451187e-08
3607 2.612622384629e-08
3608 2.61248509014766e-08
3609 2.61234693144652e-08
3610 2.61220959432706e-08
3611 2.61207060315849e-08
3612 2.61192953889267e-08
3613 2.61178985233035e-08
3614 2.61165212721015e-08
3615 2.61151529969417e-08
3616 2.61137737988526e-08
3617 2.61123688294895e-08
3618 2.61109672475834e-08
3619 2.61095994162353e-08
3620 2.61082422134962e-08
3621 2.61068342838344e-08
3622 2.61054560127261e-08
3623 2.6104068528543e-08
3624 2.61027247215329e-08
3625 2.61013140976374e-08
3626 2.60999454776423e-08
3627 2.60985613178e-08
3628 2.60972279790272e-08
3629 2.60958482088403e-08
3630 2.60944826601106e-08
3631 2.60931419284738e-08
3632 2.60917268738448e-08
3633 2.60904070065182e-08
3634 2.60890542659875e-08
3635 2.60876388011311e-08
3636 2.60862783088012e-08
3637 2.60848727071106e-08
3638 2.6083530284049e-08
3639 2.60821374267195e-08
3640 2.60807800781526e-08
3641 2.60794196210168e-08
3642 2.60780383846138e-08
3643 2.60766934412904e-08
3644 2.60752991490532e-08
3645 2.60739017377554e-08
3646 2.60725755993341e-08
3647 2.60711683961468e-08
3648 2.60698457578701e-08
3649 2.6068491357778e-08
3650 2.60671204032059e-08
3651 2.60657701693368e-08
3652 2.60644017761047e-08
3653 2.60630033354636e-08
3654 2.60616622020371e-08
3655 2.6060292643737e-08
3656 2.60589753712237e-08
3657 2.60575895983939e-08
3658 2.60562270297804e-08
3659 2.60548770049107e-08
3660 2.60535113278948e-08
3661 2.60521659334323e-08
3662 2.60508266987647e-08
3663 2.60494649411136e-08
3664 2.60481076932439e-08
3665 2.60466673426274e-08
3666 2.60453828850316e-08
3667 2.60440423627273e-08
3668 2.60426692287319e-08
3669 2.60412967627022e-08
3670 2.60399454388716e-08
3671 2.60386241833777e-08
3672 2.60372339427328e-08
3673 2.60359010218481e-08
3674 2.6034557496335e-08
3675 2.60332060197377e-08
3676 2.60318929149461e-08
3677 2.60305043260911e-08
3678 2.60291468859308e-08
3679 2.60278225364674e-08
3680 2.6026508133048e-08
3681 2.60251625134877e-08
3682 2.60237684567843e-08
3683 2.60224365392636e-08
3684 2.60210981176678e-08
3685 2.60197498861969e-08
3686 2.6018373473935e-08
3687 2.60170233682411e-08
3688 2.60157398025429e-08
3689 2.60143831825044e-08
3690 2.60130586021146e-08
3691 2.60117209150978e-08
3692 2.6010380584196e-08
3693 2.6009085903278e-08
3694 2.60077200469611e-08
3695 2.60063514956888e-08
3696 2.60049956684605e-08
3697 2.60036431002364e-08
3698 2.60023224335493e-08
3699 2.60010186777793e-08
3700 2.59996591269185e-08
3701 2.59983306490907e-08
3702 2.59969729224374e-08
3703 2.59956508142367e-08
3704 2.59943504818949e-08
3705 2.59929905629397e-08
3706 2.59916232656088e-08
3707 2.59903413831752e-08
3708 2.5989033529239e-08
3709 2.5987672158112e-08
3710 2.59863304011287e-08
3711 2.59850034201592e-08
3712 2.5983675741692e-08
3713 2.59823373179313e-08
3714 2.59810076060352e-08
3715 2.59796955720537e-08
3716 2.5978359480483e-08
3717 2.59770210487287e-08
3718 2.59757082528567e-08
3719 2.59744030690623e-08
3720 2.59731038161348e-08
3721 2.59717010422378e-08
3722 2.59704054905718e-08
3723 2.59690902079668e-08
3724 2.59677827617599e-08
3725 2.59664667162096e-08
3726 2.59651311619313e-08
3727 2.59638137191098e-08
3728 2.59624782545931e-08
3729 2.59611522279157e-08
3730 2.5959852458568e-08
3731 2.59585350534941e-08
3732 2.59572238044958e-08
3733 2.59559177322455e-08
3734 2.595456683957e-08
3735 2.59532820797159e-08
3736 2.59519534670516e-08
3737 2.59506450338565e-08
3738 2.59493507232533e-08
3739 2.59479939724305e-08
3740 2.59467352130005e-08
3741 2.59454241451906e-08
3742 2.5944121528565e-08
3743 2.59427610379559e-08
3744 2.59414974851935e-08
3745 2.59401678942561e-08
3746 2.59388810589511e-08
3747 2.59375602631451e-08
3748 2.59362757214499e-08
3749 2.59349152352817e-08
3750 2.59335890137602e-08
3751 2.59323099999209e-08
3752 2.59309988477896e-08
3753 2.59296698812417e-08
3754 2.59283758080042e-08
3755 2.59270803908418e-08
3756 2.59257805803603e-08
3757 2.59244806724013e-08
3758 2.59231543389693e-08
3759 2.59218389517257e-08
3760 2.59205373336346e-08
3761 2.59192391007801e-08
3762 2.59179308123025e-08
3763 2.5916645355839e-08
3764 2.59153802180112e-08
3765 2.59140355539089e-08
3766 2.59127199082609e-08
3767 2.59114189737342e-08
3768 2.59101212929935e-08
3769 2.59088239734084e-08
3770 2.59075701009581e-08
3771 2.59062552314671e-08
3772 2.59049201145056e-08
3773 2.59036446519922e-08
3774 2.59023454343144e-08
3775 2.59010341863153e-08
3776 2.58997568812203e-08
3777 2.5898472785113e-08
3778 2.58971424229038e-08
3779 2.58958749678184e-08
3780 2.58945648388131e-08
3781 2.58932813522184e-08
3782 2.58919982176753e-08
3783 2.58906791945623e-08
3784 2.58893990413012e-08
3785 2.58880917733406e-08
3786 2.58868424686809e-08
3787 2.58854795766594e-08
3788 2.58842289317385e-08
3789 2.58829168053309e-08
3790 2.58816087435609e-08
3791 2.58803387584439e-08
3792 2.58790557435828e-08
3793 2.58777993561998e-08
3794 2.58765140756512e-08
3795 2.58752229494452e-08
3796 2.58739544304332e-08
3797 2.58726547489596e-08
3798 2.587138797433e-08
3799 2.58701186715005e-08
3800 2.58688330391221e-08
3801 2.58675244967921e-08
3802 2.58662680746591e-08
3803 2.58649950986567e-08
3804 2.5863701656248e-08
3805 2.58624255756734e-08
3806 2.58611345830828e-08
3807 2.58598561002632e-08
3808 2.58585371922804e-08
3809 2.58572703782933e-08
3810 2.58560056435875e-08
3811 2.58546758566958e-08
3812 2.58533883437662e-08
3813 2.58521289696056e-08
3814 2.58508532736679e-08
3815 2.58495532461933e-08
3816 2.58483083324657e-08
3817 2.58470162723401e-08
3818 2.58457619359276e-08
3819 2.58445022756071e-08
3820 2.58432208061765e-08
3821 2.58419773485063e-08
3822 2.58406655309074e-08
3823 2.58393605599982e-08
3824 2.58380911849487e-08
3825 2.58368438605383e-08
3826 2.58355364481377e-08
3827 2.58342735348083e-08
3828 2.58330003468088e-08
3829 2.58317269437036e-08
3830 2.58304510738494e-08
3831 2.58291822489709e-08
3832 2.58279204405354e-08
3833 2.58266380809835e-08
3834 2.58253404814557e-08
3835 2.5824090470139e-08
3836 2.58228602993404e-08
3837 2.58216011691514e-08
3838 2.58203254874245e-08
3839 2.58190869433239e-08
3840 2.58178271584919e-08
3841 2.58165638838959e-08
3842 2.58152683380586e-08
3843 2.58139865476625e-08
3844 2.58127267978581e-08
3845 2.58114791524267e-08
3846 2.58102464784637e-08
3847 2.58089910199488e-08
3848 2.58077150348535e-08
3849 2.58064384432988e-08
3850 2.5805167130688e-08
3851 2.58039419709921e-08
3852 2.5802657501961e-08
3853 2.58014194433609e-08
3854 2.58001215127091e-08
3855 2.57988779019391e-08
3856 2.57975974485292e-08
3857 2.57963680929674e-08
3858 2.57950901882409e-08
3859 2.57938387839274e-08
3860 2.57925838976769e-08
3861 2.57913434786317e-08
3862 2.57900722279714e-08
3863 2.57888305260634e-08
3864 2.57875572620692e-08
3865 2.57863479112275e-08
3866 2.57850744560528e-08
3867 2.57838392622167e-08
3868 2.57825648257159e-08
3869 2.57813489361047e-08
3870 2.57800416952891e-08
3871 2.57788179439111e-08
3872 2.57775799860638e-08
3873 2.57763158027502e-08
3874 2.57750716094463e-08
3875 2.5773836531684e-08
3876 2.57726023707994e-08
3877 2.57713488296951e-08
3878 2.57701105172425e-08
3879 2.57688618867657e-08
3880 2.57676408086382e-08
3881 2.57663881226278e-08
3882 2.57651644014478e-08
3883 2.57638569563512e-08
3884 2.57626252849752e-08
3885 2.57613944318891e-08
3886 2.57601176399946e-08
3887 2.57588697526456e-08
3888 2.57576797896286e-08
3889 2.57564472478378e-08
3890 2.5755187374632e-08
3891 2.57539376142257e-08
3892 2.57527042012429e-08
3893 2.57514434330308e-08
3894 2.5750179273365e-08
3895 2.5748918319024e-08
3896 2.57476900782372e-08
3897 2.57464554879738e-08
3898 2.57452220399634e-08
3899 2.57439792361591e-08
3900 2.5742740605017e-08
3901 2.57415260838112e-08
3902 2.57402518836769e-08
3903 2.57390141679692e-08
3904 2.57377837439843e-08
3905 2.5736562126677e-08
3906 2.57353375119895e-08
3907 2.57341004388789e-08
3908 2.57328861990036e-08
3909 2.57315998392627e-08
3910 2.57303991865743e-08
3911 2.57291507860247e-08
3912 2.57278912931813e-08
3913 2.57266629156705e-08
3914 2.57254374316229e-08
3915 2.57241837923194e-08
3916 2.57229456064878e-08
3917 2.5721702031356e-08
3918 2.57204848418957e-08
3919 2.57192293910968e-08
3920 2.57180162211434e-08
3921 2.57167431353955e-08
3922 2.57155334363879e-08
3923 2.57143215147693e-08
3924 2.57130580395004e-08
3925 2.57118739951379e-08
3926 2.57106348905412e-08
3927 2.57094041504757e-08
3928 2.5708170840133e-08
3929 2.57069256053843e-08
3930 2.57056886530105e-08
3931 2.57044710885723e-08
3932 2.57032562208659e-08
3933 2.5702042443676e-08
3934 2.57007800883446e-08
3935 2.569956765891e-08
3936 2.56983792950583e-08
3937 2.56971590881228e-08
3938 2.56959182660665e-08
3939 2.56947194232082e-08
3940 2.5693510925795e-08
3941 2.56922796820214e-08
3942 2.5691045119125e-08
3943 2.56898258170768e-08
3944 2.56886169007209e-08
3945 2.56873921102851e-08
3946 2.5686158623861e-08
3947 2.56849225331313e-08
3948 2.56836766973634e-08
3949 2.56824957339252e-08
3950 2.56812783615556e-08
3951 2.56800360892706e-08
3952 2.56788087030779e-08
3953 2.56776082621646e-08
3954 2.56763767191859e-08
3955 2.56751355194318e-08
3956 2.56739011205709e-08
3957 2.56727002000412e-08
3958 2.56714425745375e-08
3959 2.56702484312532e-08
3960 2.56690314213159e-08
3961 2.566785571817e-08
3962 2.56666084451074e-08
3963 2.56654159225822e-08
3964 2.56642002754437e-08
3965 2.56629572109035e-08
3966 2.5661791503484e-08
3967 2.56605465945303e-08
3968 2.56593643671588e-08
3969 2.56581361481878e-08
3970 2.56569310677635e-08
3971 2.56556979444933e-08
3972 2.56545102711447e-08
3973 2.56533035569162e-08
3974 2.56520822787265e-08
3975 2.565089656964e-08
3976 2.56496154139585e-08
3977 2.56484140891411e-08
3978 2.56472398841301e-08
3979 2.56460648784818e-08
3980 2.56448336112269e-08
3981 2.56436062259224e-08
3982 2.56423997703203e-08
3983 2.56412289082464e-08
3984 2.56399698861931e-08
3985 2.56387718156059e-08
3986 2.56375961630861e-08
3987 2.56363447261876e-08
3988 2.56351275739752e-08
3989 2.56339508283854e-08
3990 2.56327535810841e-08
3991 2.56315762579007e-08
3992 2.56303164244409e-08
3993 2.5629112687775e-08
3994 2.56279397170545e-08
3995 2.56267580355241e-08
3996 2.56255345407741e-08
3997 2.5624297542326e-08
3998 2.56231497985371e-08
3999 2.56219192630303e-08
4000 2.5620690290773e-08
4001 2.56195331633236e-08
4002 2.56183197646087e-08
4003 2.56171415257689e-08
4004 2.56159325123795e-08
4005 2.56147577992749e-08
4006 2.5613544773706e-08
4007 2.56123091306182e-08
4008 2.56111421784166e-08
4009 2.56099916025598e-08
4010 2.56087698244367e-08
4011 2.56075779406228e-08
4012 2.56063836920339e-08
4013 2.56051687349546e-08
4014 2.56039890398352e-08
4015 2.56027256109737e-08
4016 2.56015970047652e-08
4017 2.56003564846896e-08
4018 2.55991938570288e-08
4019 2.55979575512488e-08
4020 2.559681273967e-08
4021 2.55955854728263e-08
4022 2.55944061260394e-08
4023 2.55932240845746e-08
4024 2.55920306567736e-08
4025 2.55908024534568e-08
4026 2.55896064080274e-08
4027 2.55884118720573e-08
4028 2.55872566301552e-08
4029 2.55860688077036e-08
4030 2.55848833091155e-08
4031 2.55837131001857e-08
4032 2.55824905145974e-08
4033 2.55813076218692e-08
4034 2.55801144505297e-08
4035 2.55789620994262e-08
4036 2.55777360127496e-08
4037 2.55765660975849e-08
4038 2.5575357465446e-08
4039 2.5574176943266e-08
4040 2.55729697529183e-08
4041 2.55718221986445e-08
4042 2.55706133940881e-08
4043 2.55694345298596e-08
4044 2.55682477686148e-08
4045 2.55670792325136e-08
4046 2.55658571490236e-08
4047 2.55646984282376e-08
4048 2.5563519498728e-08
4049 2.55623071319655e-08
4050 2.5561168861099e-08
4051 2.55599290743258e-08
4052 2.55587421230663e-08
4053 2.55575971817024e-08
4054 2.5556404936844e-08
4055 2.55552159180716e-08
4056 2.55540770731089e-08
4057 2.55528519449433e-08
4058 2.55516540197953e-08
4059 2.55504826005559e-08
4060 2.55493118867522e-08
4061 2.55481504978228e-08
4062 2.55468951165239e-08
4063 2.55457624362965e-08
4064 2.55446096813494e-08
4065 2.55433878572631e-08
4066 2.55422035330133e-08
4067 2.55410444673365e-08
4068 2.55398634189108e-08
4069 2.5538662196678e-08
4070 2.55374497841743e-08
4071 2.55363207323778e-08
4072 2.55351359466083e-08
4073 2.5533948167511e-08
4074 2.55327491401891e-08
4075 2.55316193971122e-08
4076 2.55304254740407e-08
4077 2.55292704983701e-08
4078 2.55280958603721e-08
4079 2.55269234903932e-08
4080 2.55257877052051e-08
4081 2.55245995789966e-08
4082 2.55234412099847e-08
4083 2.55222925893417e-08
4084 2.55211196137362e-08
4085 2.55199331435363e-08
4086 2.55187906985643e-08
4087 2.55176561376746e-08
4088 2.55164198947333e-08
4089 2.55152849605866e-08
4090 2.55141050791718e-08
4091 2.55129539288856e-08
4092 2.5511757829888e-08
4093 2.55105619439977e-08
4094 2.55094259142274e-08
4095 2.5508251400852e-08
4096 2.55070998811391e-08
4097 2.55059415593673e-08
4098 2.5504739909088e-08
4099 2.55035747024901e-08
4100 2.55024468478471e-08
4101 2.55012685810851e-08
4102 2.55001156877488e-08
4103 2.54989378307702e-08
4104 2.54977798613831e-08
4105 2.54966699523851e-08
4106 2.54954782044625e-08
4107 2.54943215046155e-08
4108 2.54931705306327e-08
4109 2.54920066947717e-08
4110 2.54908319464731e-08
4111 2.54896926785708e-08
4112 2.54885221398493e-08
4113 2.54873644071618e-08
4114 2.54862653003562e-08
4115 2.54850707530618e-08
4116 2.54839769698867e-08
4117 2.54828184641509e-08
4118 2.54816054469087e-08
4119 2.54804571960809e-08
4120 2.54792972647633e-08
4121 2.54781661614412e-08
4122 2.54769583757919e-08
4123 2.54758583194126e-08
4124 2.5474671765946e-08
4125 2.54735039201814e-08
4126 2.5472351548983e-08
4127 2.54711809675734e-08
4128 2.5470051684906e-08
4129 2.54688500148648e-08
4130 2.54677117061952e-08
4131 2.54665798886666e-08
4132 2.5465431683469e-08
4133 2.54642353906265e-08
4134 2.54630712197557e-08
4135 2.54619472914164e-08
4136 2.5460804460975e-08
4137 2.54596510831928e-08
4138 2.54585016284947e-08
4139 2.54573218310683e-08
4140 2.54561567446521e-08
4141 2.54550105709961e-08
4142 2.54538699834272e-08
4143 2.54527123419446e-08
4144 2.54515849740233e-08
4145 2.54503954233987e-08
4146 2.54492842536869e-08
4147 2.54481269376661e-08
4148 2.54469362266363e-08
4149 2.54457945006448e-08
4150 2.54446150147469e-08
4151 2.54435183293378e-08
4152 2.54423504540413e-08
4153 2.54411955002976e-08
4154 2.54400508811758e-08
4155 2.5438931731514e-08
4156 2.54377459443234e-08
4157 2.54366093229708e-08
4158 2.54354350325836e-08
4159 2.54343312408589e-08
4160 2.54331429686561e-08
4161 2.5432006964754e-08
4162 2.543088349094e-08
4163 2.5429802307797e-08
4164 2.5428597851318e-08
4165 2.54275281222349e-08
4166 2.54263124957466e-08
4167 2.5425209558505e-08
4168 2.54240607822642e-08
4169 2.54229558860342e-08
4170 2.54217742929885e-08
4171 2.54206497152243e-08
4172 2.54195290032011e-08
4173 2.54183797870899e-08
4174 2.54172773841432e-08
4175 2.54161036331579e-08
4176 2.54149848461505e-08
4177 2.5413839558508e-08
4178 2.54127318249919e-08
4179 2.5411576215939e-08
4180 2.54104291499968e-08
4181 2.54093270024569e-08
4182 2.54081711936194e-08
4183 2.54070389411609e-08
4184 2.54058787356182e-08
4185 2.54047634249965e-08
4186 2.54036361592158e-08
4187 2.54024923993512e-08
4188 2.54013587772106e-08
4189 2.54002249671648e-08
4190 2.53990920090486e-08
4191 2.53979701958507e-08
4192 2.53968161794682e-08
4193 2.53956811158695e-08
4194 2.53945790167354e-08
4195 2.53934525223376e-08
4196 2.53923284053159e-08
4197 2.53912072160634e-08
4198 2.53900963905207e-08
4199 2.53889673382801e-08
4200 2.53878558097442e-08
4201 2.53866925987167e-08
4202 2.53856303505939e-08
4203 2.5384437293996e-08
4204 2.53833118155633e-08
4205 2.53821911640473e-08
4206 2.53810626303363e-08
4207 2.53799204755767e-08
4208 2.5378821505162e-08
4209 2.53776849554188e-08
4210 2.53765810427908e-08
4211 2.53754098911152e-08
4212 2.53743068028833e-08
4213 2.53731897392195e-08
4214 2.53720510633992e-08
4215 2.53709314299022e-08
4216 2.5369820863097e-08
4217 2.53686759551508e-08
4218 2.53675741367365e-08
4219 2.5366439778074e-08
4220 2.53652917996394e-08
4221 2.5364185712029e-08
4222 2.53630517035863e-08
4223 2.53619729054799e-08
4224 2.53608321994503e-08
4225 2.53596985175797e-08
4226 2.53585926418554e-08
4227 2.53574949674595e-08
4228 2.53563314293048e-08
4229 2.53552497103682e-08
4230 2.53541186230333e-08
4231 2.53530040844052e-08
4232 2.53518459106816e-08
4233 2.53507632778094e-08
4234 2.53496394161945e-08
4235 2.53485128640096e-08
4236 2.53473941528859e-08
4237 2.53462623343581e-08
4238 2.53451740212518e-08
4239 2.53440397997573e-08
4240 2.53428962437852e-08
4241 2.53418099091518e-08
4242 2.53406580791316e-08
4243 2.53395740739126e-08
4244 2.53384141979951e-08
4245 2.53372807866303e-08
4246 2.53362064258078e-08
4247 2.53350729232937e-08
4248 2.53339214968951e-08
4249 2.53328331087377e-08
4250 2.53317152651422e-08
4251 2.53305163695483e-08
4252 2.5329450250966e-08
4253 2.53283561125195e-08
4254 2.53272453156761e-08
4255 2.5326106494028e-08
4256 2.53249703028868e-08
4257 2.53238595008809e-08
4258 2.53227529319333e-08
4259 2.53216142196422e-08
4260 2.53205132035861e-08
4261 2.53194014047109e-08
4262 2.53183022010384e-08
4263 2.53171826643528e-08
4264 2.53160620485859e-08
4265 2.5314964299028e-08
4266 2.5313848891817e-08
4267 2.53127261782837e-08
4268 2.53115821252092e-08
4269 2.53105051412628e-08
4270 2.53093562873619e-08
4271 2.53083375362229e-08
4272 2.53071531374216e-08
4273 2.53060482193201e-08
4274 2.53049156246909e-08
4275 2.53038009739304e-08
4276 2.53027015260643e-08
4277 2.5301597863403e-08
4278 2.53005207540014e-08
4279 2.52993671874258e-08
4280 2.52982702665383e-08
4281 2.52971555067538e-08
4282 2.5296083165427e-08
4283 2.5294936028375e-08
4284 2.52938571571604e-08
4285 2.52927547512161e-08
4286 2.52915867570702e-08
4287 2.52905110674217e-08
4288 2.52894212123267e-08
4289 2.52882925991793e-08
4290 2.52871959192102e-08
4291 2.52860656679288e-08
4292 2.52849307910696e-08
4293 2.52838570384828e-08
4294 2.52827049179172e-08
4295 2.5281683694478e-08
4296 2.5280550423834e-08
4297 2.52794291791258e-08
4298 2.52783489745334e-08
4299 2.52772237407939e-08
4300 2.52761550444403e-08
4301 2.52749851246681e-08
4302 2.52738980444089e-08
4303 2.52727911190243e-08
4304 2.52717096232979e-08
4305 2.52705702858402e-08
4306 2.52694822353017e-08
4307 2.52683691656097e-08
4308 2.52672820481026e-08
4309 2.52661510323771e-08
4310 2.52650771892515e-08
4311 2.52639484837891e-08
4312 2.52628871109106e-08
4313 2.52617761393736e-08
4314 2.52606914067921e-08
4315 2.52595943972533e-08
4316 2.52584867049821e-08
4317 2.52573715072701e-08
4318 2.52563070090583e-08
4319 2.52551507290422e-08
4320 2.52540535771728e-08
4321 2.52529659944822e-08
4322 2.52518556475012e-08
4323 2.52507501144472e-08
4324 2.5249698071883e-08
4325 2.52486115761008e-08
4326 2.52474919912871e-08
4327 2.52464067373448e-08
4328 2.5245317933742e-08
4329 2.52441838994866e-08
4330 2.5243089564253e-08
4331 2.52420079695503e-08
4332 2.52409211352611e-08
4333 2.52397921082781e-08
4334 2.52387233884988e-08
4335 2.52376011358191e-08
4336 2.5236550330876e-08
4337 2.52354545690059e-08
4338 2.52343291346491e-08
4339 2.52332691452195e-08
4340 2.52321637483899e-08
4341 2.52310776246434e-08
4342 2.52299839481052e-08
4343 2.5228937436339e-08
4344 2.52278388429561e-08
4345 2.5226746570739e-08
4346 2.5225644450011e-08
4347 2.52245643019844e-08
4348 2.52234532737705e-08
4349 2.52223401532303e-08
4350 2.52212477859226e-08
4351 2.52201491578452e-08
4352 2.52190815203668e-08
4353 2.52179892822335e-08
4354 2.52168606162395e-08
4355 2.52157799320862e-08
4356 2.52146806919984e-08
4357 2.52136044584517e-08
4358 2.52125137177872e-08
4359 2.52114950385351e-08
4360 2.52103645515533e-08
4361 2.52093288427568e-08
4362 2.52082078174287e-08
4363 2.52071166942924e-08
4364 2.52060173685509e-08
4365 2.52049205913818e-08
4366 2.52038438595115e-08
4367 2.52028170112384e-08
4368 2.5201726629509e-08
4369 2.52006153095841e-08
4370 2.51995793048576e-08
4371 2.51984336624433e-08
4372 2.5197390621734e-08
4373 2.51963149474621e-08
4374 2.51952122933274e-08
4375 2.51941319927562e-08
4376 2.51930771587028e-08
4377 2.51919520986021e-08
4378 2.51908821244373e-08
4379 2.51898030179665e-08
4380 2.51887316874977e-08
4381 2.51876569482556e-08
4382 2.51865320468614e-08
4383 2.51854482254399e-08
4384 2.51843651137285e-08
4385 2.51833107851041e-08
4386 2.51822084738063e-08
4387 2.5181114376216e-08
4388 2.518002689883e-08
4389 2.51789787586998e-08
4390 2.5177851036895e-08
4391 2.51768370876926e-08
4392 2.51757277749398e-08
4393 2.51746586656942e-08
4394 2.51735791437779e-08
4395 2.51725220933863e-08
4396 2.51714782132373e-08
4397 2.51703703209039e-08
4398 2.51692781389479e-08
4399 2.51681812792892e-08
4400 2.51671444531087e-08
4401 2.51660843419432e-08
4402 2.51649856984337e-08
4403 2.51639471741116e-08
4404 2.51628627613854e-08
4405 2.51618142785848e-08
4406 2.51607146751764e-08
4407 2.51596411088295e-08
4408 2.51585804706411e-08
4409 2.51575013119343e-08
4410 2.51564303203056e-08
4411 2.51553285718353e-08
4412 2.51542798727633e-08
4413 2.51532192854786e-08
4414 2.51521389094678e-08
4415 2.51510480618888e-08
4416 2.51499873360483e-08
4417 2.51489436058905e-08
4418 2.51478578036646e-08
4419 2.5146773548923e-08
4420 2.51456752180523e-08
4421 2.51446530606936e-08
4422 2.51435908295572e-08
4423 2.51424527842326e-08
4424 2.51414109527226e-08
4425 2.51403220188351e-08
4426 2.5139268727048e-08
4427 2.51382180425086e-08
4428 2.51371382844479e-08
4429 2.51360406827716e-08
4430 2.51349644407872e-08
4431 2.51338829844183e-08
4432 2.51328498233616e-08
4433 2.51317813277363e-08
4434 2.51307518243649e-08
4435 2.51296239609511e-08
4436 2.51285633272036e-08
4437 2.51275087330138e-08
4438 2.51264305181631e-08
4439 2.51253475561097e-08
4440 2.51243121277556e-08
4441 2.51232470186991e-08
4442 2.51221907873744e-08
4443 2.5121065531819e-08
4444 2.51200500870352e-08
4445 2.51189975924437e-08
4446 2.51179268863644e-08
4447 2.51168585369554e-08
4448 2.51158112183347e-08
4449 2.51147260825757e-08
4450 2.51136985576217e-08
4451 2.51126205076391e-08
4452 2.51115759213882e-08
4453 2.51104795030432e-08
4454 2.51094399713603e-08
4455 2.51083691924503e-08
4456 2.51073338572994e-08
4457 2.51062487379161e-08
4458 2.51052140298191e-08
4459 2.51041127652951e-08
4460 2.51030699265353e-08
4461 2.51019892374416e-08
4462 2.51009112088307e-08
4463 2.50999119396078e-08
4464 2.50988470588132e-08
4465 2.50977823759713e-08
4466 2.5096723893725e-08
4467 2.50956402771951e-08
4468 2.50945424429272e-08
4469 2.50935597588264e-08
4470 2.50924682887454e-08
4471 2.50914423324255e-08
4472 2.50903440332517e-08
4473 2.50893006769615e-08
4474 2.50882681447906e-08
4475 2.50871740826164e-08
4476 2.50861350744591e-08
4477 2.50850398854641e-08
4478 2.50839829067928e-08
4479 2.50829627594373e-08
4480 2.50819455631102e-08
4481 2.50808437441963e-08
4482 2.50797942695225e-08
4483 2.50787324755364e-08
4484 2.50776505819061e-08
4485 2.50766305014416e-08
4486 2.50755866723074e-08
4487 2.50745403150843e-08
4488 2.50734893394444e-08
4489 2.5072445198504e-08
4490 2.50713649201373e-08
4491 2.5070371217073e-08
4492 2.50692855454093e-08
4493 2.50682256136558e-08
4494 2.50671677816672e-08
4495 2.50661263366259e-08
4496 2.50650804946573e-08
4497 2.5064045172607e-08
4498 2.50629610784037e-08
4499 2.50619105255923e-08
4500 2.50608936804286e-08
4501 2.50598127127799e-08
4502 2.5058741287054e-08
4503 2.50576983824025e-08
4504 2.50567012100578e-08
4505 2.5055657945805e-08
4506 2.50545862431339e-08
4507 2.50535412000819e-08
4508 2.50525191309303e-08
4509 2.5051430630918e-08
4510 2.50503329268792e-08
4511 2.50492920111922e-08
4512 2.50482670970387e-08
4513 2.50472409694114e-08
4514 2.50461656398082e-08
4515 2.50451186835643e-08
4516 2.50440631262006e-08
4517 2.50430132841539e-08
4518 2.5041995556474e-08
4519 2.50409258495954e-08
4520 2.50398717832057e-08
4521 2.50388516788713e-08
4522 2.50377659638534e-08
4523 2.50367409717622e-08
4524 2.50357112076549e-08
4525 2.50346551060598e-08
4526 2.50336101940696e-08
4527 2.50325312045052e-08
4528 2.50315236100529e-08
4529 2.5030444836982e-08
4530 2.50294161234788e-08
4531 2.50283058980671e-08
4532 2.50272991165756e-08
4533 2.50262422430425e-08
4534 2.50252011443353e-08
4535 2.50241299751819e-08
4536 2.5023107869282e-08
4537 2.50220851302774e-08
4538 2.50210034085652e-08
4539 2.50199639095783e-08
4540 2.50189394040978e-08
4541 2.5017854067777e-08
4542 2.50168364122616e-08
4543 2.50158403835576e-08
4544 2.50147740982198e-08
4545 2.5013718104816e-08
4546 2.50126772428083e-08
4547 2.50116433984648e-08
4548 2.50105817634072e-08
4549 2.50095368797276e-08
4550 2.5008525891157e-08
4551 2.50074430188652e-08
4552 2.50064093462732e-08
4553 2.50053623639945e-08
4554 2.50043519116616e-08
4555 2.50032778282283e-08
4556 2.50022757000701e-08
4557 2.50011846625875e-08
4558 2.50001465117999e-08
4559 2.499910532483e-08
4560 2.49980832500163e-08
4561 2.49970608028338e-08
4562 2.49960014723216e-08
4563 2.49949548323247e-08
4564 2.49939244680308e-08
4565 2.49928954633716e-08
4566 2.49918420580086e-08
4567 2.49908281493849e-08
4568 2.49898022149364e-08
4569 2.49887240387214e-08
4570 2.49876690816553e-08
4571 2.4986664471871e-08
4572 2.49856335625132e-08
4573 2.49845819541572e-08
4574 2.49835315515035e-08
4575 2.49825126746317e-08
4576 2.49814357021205e-08
4577 2.49804420542343e-08
4578 2.49793845173429e-08
4579 2.49783405093518e-08
4580 2.49773011981036e-08
4581 2.4976316749914e-08
4582 2.49752633725842e-08
4583 2.49742147761522e-08
4584 2.4973172096654e-08
4585 2.49720952557597e-08
4586 2.49710894228428e-08
4587 2.49700732281588e-08
4588 2.49689855934543e-08
4589 2.49679608508857e-08
4590 2.49669401238828e-08
4591 2.49659139636704e-08
4592 2.49648531373547e-08
4593 2.49638386342621e-08
4594 2.49627927489393e-08
4595 2.49617590126205e-08
4596 2.49607538808094e-08
4597 2.49597094225118e-08
4598 2.49587062455259e-08
4599 2.49577080901342e-08
4600 2.49565999981272e-08
4601 2.49555825101444e-08
4602 2.49546132097267e-08
4603 2.49535556550162e-08
4604 2.49524877589669e-08
4605 2.49514724582345e-08
4606 2.49504645610799e-08
4607 2.49494079109236e-08
4608 2.49483804442008e-08
4609 2.494733562175e-08
4610 2.49463136795525e-08
4611 2.49453051416881e-08
4612 2.49442695539059e-08
4613 2.49432438159114e-08
4614 2.49422116354037e-08
4615 2.49412248880532e-08
4616 2.49401855246245e-08
4617 2.4939109648292e-08
4618 2.49380843797553e-08
4619 2.49371182613478e-08
4620 2.49360245914709e-08
4621 2.49350189376329e-08
4622 2.49340136419529e-08
4623 2.49329825321443e-08
4624 2.49320031249778e-08
4625 2.49309353656524e-08
4626 2.49299447632745e-08
4627 2.49288644258439e-08
4628 2.49278771499717e-08
4629 2.49268559899818e-08
4630 2.49257916441481e-08
4631 2.49248147271564e-08
4632 2.49237739198271e-08
4633 2.49227585242262e-08
4634 2.49217180520733e-08
4635 2.49206796250068e-08
4636 2.49196834580245e-08
4637 2.49186970709414e-08
4638 2.49176537244766e-08
4639 2.49166010051205e-08
4640 2.49155774977305e-08
4641 2.49145910618531e-08
4642 2.4913592389153e-08
4643 2.49125482789103e-08
4644 2.49114887467816e-08
4645 2.49104964421543e-08
4646 2.4909426936004e-08
4647 2.49084047845294e-08
4648 2.49074484515166e-08
4649 2.49063755279888e-08
4650 2.49053358680751e-08
4651 2.49043576274754e-08
4652 2.49033601407711e-08
4653 2.49023350782918e-08
4654 2.49013113688967e-08
4655 2.4900281958784e-08
4656 2.48992818083216e-08
4657 2.4898247353633e-08
4658 2.48972584717677e-08
4659 2.48962252939466e-08
4660 2.48952621492715e-08
4661 2.48942386482653e-08
4662 2.48932040861627e-08
4663 2.48922028959209e-08
4664 2.48911392761175e-08
4665 2.48901392809198e-08
4666 2.48891488040526e-08
4667 2.48881169367943e-08
4668 2.48871038382448e-08
4669 2.48860791593475e-08
4670 2.48850549779966e-08
4671 2.4884046122553e-08
4672 2.48830342513551e-08
4673 2.48820344909695e-08
4674 2.4881053883874e-08
4675 2.48799940303912e-08
4676 2.48789827174134e-08
4677 2.48779572430968e-08
4678 2.48769607948396e-08
4679 2.48759257107656e-08
4680 2.48749292995343e-08
4681 2.48739271733189e-08
4682 2.48728975960621e-08
4683 2.48718690826766e-08
4684 2.48708881497306e-08
4685 2.48698502581801e-08
4686 2.48688699655553e-08
4687 2.48678426723048e-08
4688 2.48668366886751e-08
4689 2.48657963179966e-08
4690 2.48648204608237e-08
4691 2.48638110753041e-08
4692 2.48627697486681e-08
4693 2.48617777124371e-08
4694 2.48607407750678e-08
4695 2.48597107568305e-08
4696 2.48586884951663e-08
4697 2.48577116804261e-08
4698 2.48566769855962e-08
4699 2.48556728804084e-08
4700 2.485467782809e-08
4701 2.48536799083987e-08
4702 2.48526799175308e-08
4703 2.48516376720742e-08
4704 2.48506160446804e-08
4705 2.48495905692536e-08
4706 2.48485764063333e-08
4707 2.48475973367857e-08
4708 2.48465975904444e-08
4709 2.48455851821761e-08
4710 2.48445746726667e-08
4711 2.4843523441731e-08
4712 2.48425153712151e-08
4713 2.4841566752587e-08
4714 2.4840524582681e-08
4715 2.48394983640154e-08
4716 2.48385076961899e-08
4717 2.48374556472419e-08
4718 2.48364684338198e-08
4719 2.48354823701447e-08
4720 2.48344610757623e-08
4721 2.48334331266475e-08
4722 2.48324753833185e-08
4723 2.48314728139576e-08
4724 2.48304060451732e-08
4725 2.4829455808173e-08
4726 2.48284328239201e-08
4727 2.48274577185348e-08
4728 2.48264534075116e-08
4729 2.4825423171837e-08
4730 2.48244399059239e-08
4731 2.48234083770615e-08
4732 2.48224307304867e-08
4733 2.48213989775814e-08
4734 2.48203682771675e-08
4735 2.48194015706749e-08
4736 2.48184071720559e-08
4737 2.48174199033446e-08
4738 2.48163756834674e-08
4739 2.48153708661825e-08
4740 2.48143644591137e-08
4741 2.48133406681728e-08
4742 2.48123942220846e-08
4743 2.4811332684227e-08
4744 2.4810353065341e-08
4745 2.48093392332671e-08
4746 2.48083126449528e-08
4747 2.48073165541873e-08
4748 2.48063424139189e-08
4749 2.48053293664952e-08
4750 2.48042919733238e-08
4751 2.48032997085534e-08
4752 2.48023120533736e-08
4753 2.48012729899272e-08
4754 2.48003005277053e-08
4755 2.47992883739556e-08
4756 2.47983022218512e-08
4757 2.47972586083223e-08
4758 2.47962950887248e-08
4759 2.4795270101019e-08
4760 2.47943012200436e-08
4761 2.47932942697981e-08
4762 2.47923311484377e-08
4763 2.47912950986917e-08
4764 2.47903036812436e-08
4765 2.47892898668223e-08
4766 2.47882901406316e-08
4767 2.47872763690649e-08
4768 2.47862671013954e-08
4769 2.47852541609972e-08
4770 2.47842890435113e-08
4771 2.47832709644458e-08
4772 2.47822747146409e-08
4773 2.4781297174703e-08
4774 2.47802832741839e-08
4775 2.47792704881067e-08
4776 2.47782608250313e-08
4777 2.47772652877676e-08
4778 2.4776256724035e-08
4779 2.4775274964639e-08
4780 2.47742894685099e-08
4781 2.47732716154858e-08
4782 2.47722994019539e-08
4783 2.477129406947e-08
4784 2.47703021240553e-08
4785 2.47693186294917e-08
4786 2.47683309742008e-08
4787 2.47673447386632e-08
4788 2.47663414200128e-08
4789 2.47653322114072e-08
4790 2.47643460385416e-08
4791 2.47633661100699e-08
4792 2.47623709963007e-08
4793 2.47613962199855e-08
4794 2.47603794905071e-08
4795 2.47593739940433e-08
4796 2.47583478903413e-08
4797 2.47573654912903e-08
4798 2.47563680592644e-08
4799 2.47553468703532e-08
4800 2.47543721910159e-08
4801 2.47534021212692e-08
4802 2.47523857744292e-08
4803 2.47513577818492e-08
4804 2.47503744897348e-08
4805 2.47493620187944e-08
4806 2.47483968429663e-08
4807 2.47473522422825e-08
4808 2.47463677010007e-08
4809 2.47453749014359e-08
4810 2.47443870466935e-08
4811 2.47433885839365e-08
4812 2.47423933882329e-08
4813 2.47414069178831e-08
4814 2.47404165839016e-08
4815 2.47393978757282e-08
4816 2.47384334094991e-08
4817 2.47374396766809e-08
4818 2.47364429896146e-08
4819 2.47354388004384e-08
4820 2.47344591091658e-08
4821 2.47334262509225e-08
4822 2.47324481983946e-08
4823 2.47314461962467e-08
4824 2.47304632151057e-08
4825 2.47294544877819e-08
4826 2.47285103583406e-08
4827 2.47274626756777e-08
4828 2.47264697280647e-08
4829 2.47255139750879e-08
4830 2.47245235253657e-08
4831 2.47235218098774e-08
4832 2.47225086106861e-08
4833 2.47215138376444e-08
4834 2.47205198478651e-08
4835 2.47195283138435e-08
4836 2.47185270234596e-08
4837 2.47175712685399e-08
4838 2.471657143438e-08
4839 2.47155824780743e-08
4840 2.47145959451633e-08
4841 2.47136079560284e-08
4842 2.47125866301157e-08
4843 2.4711624192264e-08
4844 2.47105963866456e-08
4845 2.4709594906358e-08
4846 2.47086146613062e-08
4847 2.47076376870825e-08
4848 2.47066641786975e-08
4849 2.47056650324318e-08
4850 2.47046688881536e-08
4851 2.47036822363378e-08
4852 2.47027125204191e-08
4853 2.47017247511638e-08
4854 2.47007626207885e-08
4855 2.46997246489133e-08
4856 2.46987598389037e-08
4857 2.46977554028138e-08
4858 2.46967658615316e-08
4859 2.46958037792844e-08
4860 2.46947941157649e-08
4861 2.46938458195456e-08
4862 2.46928654749068e-08
4863 2.46918691181319e-08
4864 2.46908675893831e-08
4865 2.4689896569785e-08
4866 2.46889113578175e-08
4867 2.46879334341865e-08
4868 2.46869891980528e-08
4869 2.46859519688059e-08
4870 2.46849745044186e-08
4871 2.4684021027177e-08
4872 2.46830422018229e-08
4873 2.46820512671553e-08
4874 2.46810906286421e-08
4875 2.46800654475909e-08
4876 2.46790914051331e-08
4877 2.46781206891811e-08
4878 2.4677118813432e-08
4879 2.46761532355921e-08
4880 2.46751983049021e-08
4881 2.46742397411737e-08
4882 2.46732500802649e-08
4883 2.46722642188169e-08
4884 2.46712962601037e-08
4885 2.46703194051734e-08
4886 2.46693116932595e-08
4887 2.46683758718391e-08
4888 2.46673634571315e-08
4889 2.46663873533781e-08
4890 2.4665371434085e-08
4891 2.46644070726609e-08
4892 2.46634556034797e-08
4893 2.46624989154176e-08
4894 2.46615035858766e-08
4895 2.46605011015033e-08
4896 2.46595474072686e-08
4897 2.46585754716255e-08
4898 2.46575709547114e-08
4899 2.46565646653818e-08
4900 2.46556266056408e-08
4901 2.46546449166329e-08
4902 2.46536550908005e-08
4903 2.46526796483515e-08
4904 2.46516944735764e-08
4905 2.46507331683743e-08
4906 2.46497260534273e-08
4907 2.4648739886779e-08
4908 2.46477917468235e-08
4909 2.46468183111026e-08
4910 2.46458391310878e-08
4911 2.46448398620314e-08
4912 2.46439057188241e-08
4913 2.46429328862319e-08
4914 2.464194395152e-08
4915 2.46409671297299e-08
4916 2.46399812285913e-08
4917 2.46389714894657e-08
4918 2.46380524009382e-08
4919 2.46370961466957e-08
4920 2.4636065641348e-08
4921 2.46351400157163e-08
4922 2.46341792488058e-08
4923 2.46331645105125e-08
4924 2.46322397199905e-08
4925 2.46312503379698e-08
4926 2.46302696372824e-08
4927 2.46292652458235e-08
4928 2.46282870491887e-08
4929 2.46273661637098e-08
4930 2.46263814077108e-08
4931 2.46253779290773e-08
4932 2.46244273593432e-08
4933 2.46234368282972e-08
4934 2.46224838101328e-08
4935 2.46214819298873e-08
4936 2.46205038521574e-08
4937 2.4619558176675e-08
4938 2.46185784143482e-08
4939 2.46176083890104e-08
4940 2.46166299724959e-08
4941 2.4615668538619e-08
4942 2.46147482682035e-08
4943 2.46137500057864e-08
4944 2.46127634750404e-08
4945 2.46117846212091e-08
4946 2.46108423307412e-08
4947 2.46098651389137e-08
4948 2.46089065597532e-08
4949 2.46079068325633e-08
4950 2.46069474526545e-08
4951 2.46059910710139e-08
4952 2.46050199185222e-08
4953 2.46040243631063e-08
4954 2.46030464772784e-08
4955 2.46020870938723e-08
4956 2.46010875233349e-08
4957 2.46001421346786e-08
4958 2.45991829752601e-08
4959 2.45982313151205e-08
4960 2.4597273141469e-08
4961 2.45963146982553e-08
4962 2.45953426301071e-08
4963 2.45943381045333e-08
4964 2.45934188793373e-08
4965 2.45924553587407e-08
4966 2.45914800297564e-08
4967 2.45904787981588e-08
4968 2.45895339884283e-08
4969 2.45885524990941e-08
4970 2.45875934917206e-08
4971 2.45866449858356e-08
4972 2.45856561758018e-08
4973 2.45847095488028e-08
4974 2.45837442666708e-08
4975 2.45827718038383e-08
4976 2.45818297486267e-08
4977 2.45808353603327e-08
4978 2.45798826855603e-08
4979 2.45789175081224e-08
4980 2.45779654902689e-08
4981 2.45769980330435e-08
4982 2.45760125380245e-08
4983 2.45750780076826e-08
4984 2.45740840831288e-08
4985 2.45731437032437e-08
4986 2.45721549653743e-08
4987 2.45712333594161e-08
4988 2.45702631102018e-08
4989 2.45692598599412e-08
4990 2.45683325497126e-08
4991 2.45673427441417e-08
4992 2.45663905261151e-08
4993 2.45654113487093e-08
4994 2.45644206015028e-08
4995 2.45634719030496e-08
4996 2.45625201734656e-08
4997 2.45615663556076e-08
4998 2.45605882837618e-08
4999 2.45596256991942e-08
};
\addlegendentry{Train}
\addplot [semithick, black]
table {%
0 0.000768683792557567
1 0.00019661447731778
2 0.00018548039952293
3 0.000151924134115689
4 6.01103711233009e-05
5 3.18478596454952e-05
6 2.98793729598401e-05
7 2.73047571681673e-05
8 2.26329957513371e-05
9 1.5272735254257e-05
10 8.95447010407224e-06
11 6.56182783131953e-06
12 5.98574433752219e-06
13 5.77387800149154e-06
14 5.64328229302191e-06
15 5.53762356503285e-06
16 5.44006343261572e-06
17 5.33823367732111e-06
18 5.22067330166465e-06
19 5.07594950249768e-06
20 4.89061494590715e-06
21 4.64755157736363e-06
22 4.32192064181436e-06
23 3.88297894460266e-06
24 3.30719717567263e-06
25 2.65992775894119e-06
26 2.10636130759667e-06
27 1.7518466393085e-06
28 1.56236251314112e-06
29 1.46952561408398e-06
30 1.42019359827827e-06
31 1.39028577450517e-06
32 1.37043343784171e-06
33 1.35587140448479e-06
34 1.34394520046044e-06
35 1.3340581972443e-06
36 1.32579805267596e-06
37 1.31894330479554e-06
38 1.31264528135944e-06
39 1.30712817281164e-06
40 1.30180717405892e-06
41 1.29707564155979e-06
42 1.29227134948451e-06
43 1.28824171952147e-06
44 1.28401916299481e-06
45 1.28024009882211e-06
46 1.27615157907712e-06
47 1.27242105918413e-06
48 1.26836289382481e-06
49 1.26456848192902e-06
50 1.26042937154125e-06
51 1.25635824588244e-06
52 1.25193662370293e-06
53 1.24737209716841e-06
54 1.2426689863787e-06
55 1.23767551940546e-06
56 1.23191421153024e-06
57 1.22456185636111e-06
58 1.21539540032245e-06
59 1.2047048585373e-06
60 1.1928430012631e-06
61 1.17982074243628e-06
62 1.16527951377066e-06
63 1.1486872608657e-06
64 1.12958059617085e-06
65 1.10753376247885e-06
66 1.08205415472185e-06
67 1.05268952665938e-06
68 1.0194722790402e-06
69 9.83421273303975e-07
70 9.43965460464824e-07
71 9.00679367532575e-07
72 8.53818050927657e-07
73 8.0516184652879e-07
74 7.58155294988683e-07
75 7.12667542757117e-07
76 6.73119018301804e-07
77 6.42867348688014e-07
78 6.18960825704562e-07
79 6.0092327203165e-07
80 5.88103375775972e-07
81 5.77115542910178e-07
82 5.68773828035773e-07
83 5.61820286293369e-07
84 5.55869576146506e-07
85 5.50720642422675e-07
86 5.46196361028706e-07
87 5.42159227734373e-07
88 5.38507208602823e-07
89 5.35173342086637e-07
90 5.32099193151225e-07
91 5.2923633120372e-07
92 5.26551559687505e-07
93 5.24018503256229e-07
94 5.21615675097564e-07
95 5.19324316883285e-07
96 5.17125670285168e-07
97 5.15014221491583e-07
98 5.12980420808162e-07
99 5.11011762682756e-07
100 5.09103074364248e-07
101 5.07245147218782e-07
102 5.05442585563287e-07
103 5.03685498642881e-07
104 5.01958652421308e-07
105 5.00280293636024e-07
106 4.98626491207688e-07
107 4.97024416290515e-07
108 4.95417395995901e-07
109 4.93885181640508e-07
110 4.92338415369886e-07
111 4.90870320390968e-07
112 4.89392277813749e-07
113 4.8799608975969e-07
114 4.86588021431089e-07
115 4.85258055960003e-07
116 4.83945598261926e-07
117 4.82680718505435e-07
118 4.81466827295662e-07
119 4.80280675674294e-07
120 4.79125162655691e-07
121 4.78011600080208e-07
122 4.76930381410057e-07
123 4.75891198448153e-07
124 4.74883620427136e-07
125 4.73922682431294e-07
126 4.72997953693266e-07
127 4.7211855758178e-07
128 4.71283357228458e-07
129 4.70485531423037e-07
130 4.69730849772532e-07
131 4.69013428983089e-07
132 4.68339948156427e-07
133 4.6769909545219e-07
134 4.6708885292901e-07
135 4.66516638653047e-07
136 4.65959971052143e-07
137 4.65428769302889e-07
138 4.64913142650403e-07
139 4.644098225981e-07
140 4.63916848048029e-07
141 4.63426033547876e-07
142 4.6294431399474e-07
143 4.62460207018012e-07
144 4.61969705156662e-07
145 4.61492533077035e-07
146 4.61005384977398e-07
147 4.60512524114165e-07
148 4.60030946669576e-07
149 4.59537943697796e-07
150 4.59047356571318e-07
151 4.58567285477329e-07
152 4.58081927945386e-07
153 4.57606887493966e-07
154 4.5712769747297e-07
155 4.56664423609254e-07
156 4.56196943332543e-07
157 4.55739439075842e-07
158 4.55281934819141e-07
159 4.54847537412206e-07
160 4.54408137784412e-07
161 4.53980277370647e-07
162 4.53558044455349e-07
163 4.53150704515792e-07
164 4.52743222467689e-07
165 4.52350263913104e-07
166 4.51968332981778e-07
167 4.51587652605667e-07
168 4.51220842023758e-07
169 4.5086500222169e-07
170 4.5051672259433e-07
171 4.50183421207839e-07
172 4.49852535666651e-07
173 4.49526879719997e-07
174 4.49219754727892e-07
175 4.48916324558013e-07
176 4.48617896608994e-07
177 4.48330069957592e-07
178 4.48048581347393e-07
179 4.47771270728481e-07
180 4.47503566647356e-07
181 4.47249448143339e-07
182 4.46988792646152e-07
183 4.46737260517693e-07
184 4.46486609462227e-07
185 4.46249686092415e-07
186 4.46007305754392e-07
187 4.45775071966636e-07
188 4.4554286660059e-07
189 4.45319585651305e-07
190 4.45104149093822e-07
191 4.44883369254967e-07
192 4.44664351562096e-07
193 4.44455281467526e-07
194 4.44241607056028e-07
195 4.44034384372571e-07
196 4.43826110085865e-07
197 4.43622496959506e-07
198 4.43419367002207e-07
199 4.43219533963202e-07
200 4.4301503976385e-07
201 4.42818191004335e-07
202 4.42619665363964e-07
203 4.42421850266328e-07
204 4.4222664996596e-07
205 4.42027129565759e-07
206 4.41834146158726e-07
207 4.4163633106109e-07
208 4.41440647591662e-07
209 4.41243855675566e-07
210 4.41048598531779e-07
211 4.40850072891408e-07
212 4.40651319877361e-07
213 4.40454613226393e-07
214 4.40256599176791e-07
215 4.40054805039836e-07
216 4.39853351963393e-07
217 4.39646356653611e-07
218 4.39440213995113e-07
219 4.39232223925501e-07
220 4.39021789588878e-07
221 4.38811895264735e-07
222 4.38595606055969e-07
223 4.38378094713698e-07
224 4.38160697058265e-07
225 4.37945033127107e-07
226 4.37723059576456e-07
227 4.37502052363925e-07
228 4.37281585163873e-07
229 4.37055831525868e-07
230 4.36834568517952e-07
231 4.36611344412086e-07
232 4.36384482327412e-07
233 4.36158813954535e-07
234 4.35936897247302e-07
235 4.35710745705364e-07
236 4.35477659266326e-07
237 4.35249233987633e-07
238 4.35019131828085e-07
239 4.34785107472635e-07
240 4.34548411476499e-07
241 4.34313847108569e-07
242 4.3407331418166e-07
243 4.33833918123128e-07
244 4.33594522064595e-07
245 4.3334929955563e-07
246 4.33107402386668e-07
247 4.32861099852744e-07
248 4.32615365753009e-07
249 4.32363947311387e-07
250 4.3211284150857e-07
251 4.31863128369514e-07
252 4.3160684981558e-07
253 4.3135239025105e-07
254 4.31094434816259e-07
255 4.30832187703345e-07
256 4.30565762599144e-07
257 4.30296466902291e-07
258 4.30021430020133e-07
259 4.29746449981394e-07
260 4.29463312912048e-07
261 4.29175400995518e-07
262 4.28879673108895e-07
263 4.28580108291499e-07
264 4.28268066343662e-07
265 4.27953210646592e-07
266 4.27624740950705e-07
267 4.27289364779426e-07
268 4.26944552600617e-07
269 4.26585927471024e-07
270 4.26218292659541e-07
271 4.25840397610955e-07
272 4.25448206442525e-07
273 4.25047090857333e-07
274 4.24627728534688e-07
275 4.24191284764674e-07
276 4.23744495492429e-07
277 4.23272183525114e-07
278 4.22834887103818e-07
279 4.22376842834638e-07
280 4.21863575184034e-07
281 4.21338910427949e-07
282 4.2080193907168e-07
283 4.20221965669043e-07
284 4.19676638330202e-07
285 4.19035870891094e-07
286 4.18534312984775e-07
287 4.17813964759262e-07
288 4.17373229311124e-07
289 4.16506566125463e-07
290 4.16182615481375e-07
291 4.15109354889864e-07
292 4.14934277159773e-07
293 4.13639412499833e-07
294 4.13621847883405e-07
295 4.12077696410051e-07
296 4.12147329598156e-07
297 4.10506402204192e-07
298 4.10519675142496e-07
299 4.08895999726155e-07
300 4.08659133199762e-07
301 4.0751808683126e-07
302 4.06640282335502e-07
303 4.05604026809669e-07
304 4.04752796612229e-07
305 4.03792853376217e-07
306 4.02835979684824e-07
307 4.01873535338382e-07
308 4.0089960862133e-07
309 3.99906497250413e-07
310 3.98874362872448e-07
311 3.97791097839217e-07
312 3.96693536686143e-07
313 3.95575852962793e-07
314 3.94426479033427e-07
315 3.93252037156344e-07
316 3.92038828067598e-07
317 3.90798987837115e-07
318 3.89522597288305e-07
319 3.88203204693127e-07
320 3.86830578236186e-07
321 3.85379678391473e-07
322 3.83844280804624e-07
323 3.82226687634102e-07
324 3.80523971443836e-07
325 3.78678663537357e-07
326 3.76769150989276e-07
327 3.74822150206455e-07
328 3.72816572280499e-07
329 3.70766940704925e-07
330 3.68691019048129e-07
331 3.66437262755426e-07
332 3.64105090966405e-07
333 3.61746742782998e-07
334 3.59384046078048e-07
335 3.57009724893942e-07
336 3.54606669361601e-07
337 3.52172179418631e-07
338 3.49712479419395e-07
339 3.47222680829873e-07
340 3.44727851597781e-07
341 3.42242543638349e-07
342 3.39780740432616e-07
343 3.37339059797159e-07
344 3.34907269916584e-07
345 3.32461638663517e-07
346 3.29997419612482e-07
347 3.27511003206382e-07
348 3.24991617617343e-07
349 3.22422692988766e-07
350 3.1980488301997e-07
351 3.17136112926164e-07
352 3.14418741709233e-07
353 3.11654758888835e-07
354 3.08874945176285e-07
355 3.06123496329747e-07
356 3.03445375493538e-07
357 3.0079769430813e-07
358 2.98120397701496e-07
359 2.95369346758889e-07
360 2.9252910849209e-07
361 2.89607783088286e-07
362 2.86631035351093e-07
363 2.83623393215748e-07
364 2.80601284430304e-07
365 2.77584462082814e-07
366 2.7460023943604e-07
367 2.71662372597348e-07
368 2.68787459845043e-07
369 2.65973824298271e-07
370 2.63225700791736e-07
371 2.60550564235018e-07
372 2.57944549275635e-07
373 2.554468778726e-07
374 2.53076137823882e-07
375 2.50857510764035e-07
376 2.48790172463487e-07
377 2.46872474463089e-07
378 2.45086425820773e-07
379 2.4336958404092e-07
380 2.41804912093357e-07
381 2.40267155504625e-07
382 2.38841039390536e-07
383 2.37452439932895e-07
384 2.36150683008418e-07
385 2.34890507044838e-07
386 2.33698713714148e-07
387 2.32545744438539e-07
388 2.3143445559981e-07
389 2.30359631814281e-07
390 2.29317677735708e-07
391 2.28307058591781e-07
392 2.2732587012797e-07
393 2.2637259178282e-07
394 2.25455394797791e-07
395 2.24566718998176e-07
396 2.23715574065864e-07
397 2.22897398316491e-07
398 2.22115474457496e-07
399 2.21360252794511e-07
400 2.20629203795397e-07
401 2.19924046973574e-07
402 2.19228340370137e-07
403 2.18546290398081e-07
404 2.17872312191503e-07
405 2.17202725139032e-07
406 2.16533038610578e-07
407 2.15861106767079e-07
408 2.15185764318448e-07
409 2.14496921557839e-07
410 2.13787842540114e-07
411 2.13115043834478e-07
412 2.12519282172252e-07
413 2.11964703566991e-07
414 2.11432507057907e-07
415 2.10924824273206e-07
416 2.10442479442463e-07
417 2.09984179377898e-07
418 2.09546186624721e-07
419 2.09115754046252e-07
420 2.08686643077272e-07
421 2.08264353318555e-07
422 2.07832783871709e-07
423 2.07398684892723e-07
424 2.06964330118353e-07
425 2.06536029168092e-07
426 2.06111920419971e-07
427 2.05704608902124e-07
428 2.05308040790442e-07
429 2.04927971481084e-07
430 2.0456467098029e-07
431 2.04219674060369e-07
432 2.03898537165514e-07
433 2.03591852709906e-07
434 2.03302889190127e-07
435 2.03030182888142e-07
436 2.02774444346687e-07
437 2.02533243509606e-07
438 2.02301578156039e-07
439 2.02083384692742e-07
440 2.01873632477145e-07
441 2.01669436705743e-07
442 2.01470896854516e-07
443 2.01276918687654e-07
444 2.0108498688387e-07
445 2.00896252522398e-07
446 2.00712250375545e-07
447 2.00531047767072e-07
448 2.00348651446802e-07
449 2.00173474240728e-07
450 1.99996250671575e-07
451 1.99822878244049e-07
452 1.99651580601312e-07
453 1.99478762397121e-07
454 1.99306242620878e-07
455 1.99129260636255e-07
456 1.98948498564278e-07
457 1.98765462755546e-07
458 1.98575051513217e-07
459 1.98381968630201e-07
460 1.98192452671719e-07
461 1.98013296426325e-07
462 1.97838545545892e-07
463 1.97671795376664e-07
464 1.9751081481445e-07
465 1.97350544794972e-07
466 1.97193898543446e-07
467 1.97041643446028e-07
468 1.96887043557581e-07
469 1.96734859514436e-07
470 1.96583684441975e-07
471 1.9643101722977e-07
472 1.96279870579019e-07
473 1.96129406049295e-07
474 1.95980888406666e-07
475 1.95833266047885e-07
476 1.95688258486371e-07
477 1.95544359371524e-07
478 1.95401867131295e-07
479 1.95258294866107e-07
480 1.95117351609042e-07
481 1.94978383660782e-07
482 1.94835635625168e-07
483 1.94695772393061e-07
484 1.94556477595142e-07
485 1.944128911191e-07
486 1.94268650943741e-07
487 1.94125064467698e-07
488 1.93980085327894e-07
489 1.93834623019029e-07
490 1.9368637538264e-07
491 1.93537474046934e-07
492 1.93386313185329e-07
493 1.93234697576372e-07
494 1.93081291399722e-07
495 1.92924773045888e-07
496 1.92768496276585e-07
497 1.92608368365654e-07
498 1.92448879943186e-07
499 1.92287274103364e-07
500 1.921228260926e-07
501 1.91956729622689e-07
502 1.91787691505851e-07
503 1.91621793987906e-07
504 1.91449899489271e-07
505 1.91277607086704e-07
506 1.91101563018492e-07
507 1.90926328968999e-07
508 1.90748224326853e-07
509 1.90568570701544e-07
510 1.90387325460506e-07
511 1.90206151273742e-07
512 1.90022149126889e-07
513 1.89836569575164e-07
514 1.89649213666598e-07
515 1.89460394039997e-07
516 1.89269911743395e-07
517 1.89081248436196e-07
518 1.88891107200106e-07
519 1.88699033287776e-07
520 1.88503577192023e-07
521 1.88311730653368e-07
522 1.88116089816504e-07
523 1.87920761618443e-07
524 1.87720402777813e-07
525 1.87529650474971e-07
526 1.87331465895113e-07
527 1.87139420404492e-07
528 1.86942031632498e-07
529 1.86747769248541e-07
530 1.86550224157145e-07
531 1.86355165965324e-07
532 1.86157095072303e-07
533 1.85958597853642e-07
534 1.85761493298742e-07
535 1.85564061894183e-07
536 1.85365877314325e-07
537 1.85165987431901e-07
538 1.84967731797769e-07
539 1.84767145583464e-07
540 1.8456876205164e-07
541 1.84366001576564e-07
542 1.84166609074055e-07
543 1.83965738642655e-07
544 1.83763077643562e-07
545 1.83560970867802e-07
546 1.83359091465718e-07
547 1.83155648869615e-07
548 1.82950770977186e-07
549 1.82747598387323e-07
550 1.82542862603441e-07
551 1.8233906473597e-07
552 1.82132239956445e-07
553 1.81928456299829e-07
554 1.81721958369963e-07
555 1.81513726715821e-07
556 1.8130667456262e-07
557 1.8109763288976e-07
558 1.80893891865708e-07
559 1.80685717054985e-07
560 1.80477073286056e-07
561 1.80274682293202e-07
562 1.800667206453e-07
563 1.79860677462784e-07
564 1.79654378484884e-07
565 1.79451390636132e-07
566 1.79242746867203e-07
567 1.79043965431447e-07
568 1.78842327613893e-07
569 1.78640178205569e-07
570 1.78437957742972e-07
571 1.78236732040205e-07
572 1.7803536422889e-07
573 1.77831950054497e-07
574 1.77628990627454e-07
575 1.77425206970838e-07
576 1.77218623775843e-07
577 1.77014513269569e-07
578 1.76806480567393e-07
579 1.76597751533336e-07
580 1.76390358319622e-07
581 1.76178261312998e-07
582 1.7596735801817e-07
583 1.75754138354023e-07
584 1.75540904479021e-07
585 1.75323776829828e-07
586 1.75109462929868e-07
587 1.74888370452209e-07
588 1.74673630226607e-07
589 1.74456417312285e-07
590 1.74241591821556e-07
591 1.74023909949028e-07
592 1.73810704495736e-07
593 1.73595594787912e-07
594 1.73382105117525e-07
595 1.73173916095948e-07
596 1.72964675471121e-07
597 1.72760849181941e-07
598 1.72555715494127e-07
599 1.72357033534354e-07
600 1.72158820532786e-07
601 1.71960266470705e-07
602 1.71765066170337e-07
603 1.71579372931774e-07
604 1.71392386505431e-07
605 1.71210459143367e-07
606 1.71030350770707e-07
607 1.70853553527195e-07
608 1.70677935784624e-07
609 1.70510105590438e-07
610 1.70343469108047e-07
611 1.70178296343693e-07
612 1.7001825369789e-07
613 1.69860456367132e-07
614 1.69704264862958e-07
615 1.69548428630151e-07
616 1.69393743476576e-07
617 1.6924261103668e-07
618 1.69088494317293e-07
619 1.68938569800048e-07
620 1.68782662512967e-07
621 1.68627238394947e-07
622 1.68470336348037e-07
623 1.68311572679158e-07
624 1.68147266776941e-07
625 1.6797692126147e-07
626 1.67810057405404e-07
627 1.67637963954803e-07
628 1.67465088907193e-07
629 1.67285378438464e-07
630 1.67109888593586e-07
631 1.66934668754948e-07
632 1.66758269415368e-07
633 1.66580932159377e-07
634 1.66410430324504e-07
635 1.66236262089114e-07
636 1.66061440154408e-07
637 1.65896494763729e-07
638 1.65729247214585e-07
639 1.65563150744674e-07
640 1.65398532203653e-07
641 1.65235647386908e-07
642 1.65070986213323e-07
643 1.64911241995469e-07
644 1.64749977216161e-07
645 1.64589522455572e-07
646 1.64423255455404e-07
647 1.64264946533876e-07
648 1.64102289090806e-07
649 1.63938125297136e-07
650 1.6377701683723e-07
651 1.63611559855781e-07
652 1.63448902412711e-07
653 1.63280915899122e-07
654 1.63118471618873e-07
655 1.62950030357933e-07
656 1.62782001211781e-07
657 1.62616728971443e-07
658 1.62446767149049e-07
659 1.62275995307937e-07
660 1.62106431389475e-07
661 1.61937421694347e-07
662 1.61763338724086e-07
663 1.61595792746994e-07
664 1.61424281941436e-07
665 1.61251520580663e-07
666 1.61079995564251e-07
667 1.60908811608351e-07
668 1.60733648613132e-07
669 1.60563132567404e-07
670 1.60392460202274e-07
671 1.60215734013036e-07
672 1.60045658503805e-07
673 1.59873991378845e-07
674 1.5970408639987e-07
675 1.59531737153884e-07
676 1.59361860596618e-07
677 1.59188914494734e-07
678 1.59019919010461e-07
679 1.58850951947898e-07
680 1.58685637074996e-07
681 1.58513202563881e-07
682 1.58350246692862e-07
683 1.58179574327733e-07
684 1.58014657358763e-07
685 1.57848461412868e-07
686 1.57687694013475e-07
687 1.57524894461858e-07
688 1.57365803943321e-07
689 1.57205121809056e-07
690 1.57044084403424e-07
691 1.56889299773866e-07
692 1.56731090328321e-07
693 1.56575467258335e-07
694 1.564195457604e-07
695 1.56269862827685e-07
696 1.56117252458898e-07
697 1.5596448577071e-07
698 1.55819449787487e-07
699 1.55669653167934e-07
700 1.55520922362484e-07
701 1.55375303734218e-07
702 1.55234090470913e-07
703 1.55088187625552e-07
704 1.54945198005407e-07
705 1.54806343743985e-07
706 1.54665343643501e-07
707 1.54528279949773e-07
708 1.54390875195531e-07
709 1.54253001483085e-07
710 1.54117472561666e-07
711 1.53982540496145e-07
712 1.53848617401309e-07
713 1.53717053308355e-07
714 1.5358735083737e-07
715 1.53460462115618e-07
716 1.53328088003946e-07
717 1.53201753505527e-07
718 1.5307611533899e-07
719 1.52949311882367e-07
720 1.52826444832499e-07
721 1.52703336198101e-07
722 1.52580625467635e-07
723 1.52459321611786e-07
724 1.52337435110894e-07
725 1.52221076632486e-07
726 1.52101506500912e-07
727 1.51981154772329e-07
728 1.51865535258366e-07
729 1.51747158838589e-07
730 1.51629890865479e-07
731 1.51515948232372e-07
732 1.51401010839436e-07
733 1.51288858774024e-07
734 1.511735092663e-07
735 1.5106175510482e-07
736 1.50952757849154e-07
737 1.50841373169897e-07
738 1.50730016912348e-07
739 1.50618859606766e-07
740 1.50509634977425e-07
741 1.50398662412954e-07
742 1.5029256417165e-07
743 1.50184177982737e-07
744 1.50074328075789e-07
745 1.49968670370981e-07
746 1.49863254250704e-07
747 1.49755393863416e-07
748 1.49651526726302e-07
749 1.49542771055167e-07
750 1.49439756569336e-07
751 1.49338077903849e-07
752 1.49232462831606e-07
753 1.49131039961503e-07
754 1.49027371776356e-07
755 1.48926588394716e-07
756 1.48823275480936e-07
757 1.48721824189124e-07
758 1.4862148134398e-07
759 1.48518779496953e-07
760 1.48416106071636e-07
761 1.48316431136664e-07
762 1.4821732463588e-07
763 1.48117592857488e-07
764 1.48017221590635e-07
765 1.47919152482245e-07
766 1.47818767004537e-07
767 1.47721237908627e-07
768 1.47620866641773e-07
769 1.47524175986291e-07
770 1.4742619214303e-07
771 1.47324584531816e-07
772 1.47227666502658e-07
773 1.4713077689521e-07
774 1.47033503594685e-07
775 1.46936500300399e-07
776 1.46840477555088e-07
777 1.46741697903963e-07
778 1.46645959375746e-07
779 1.46549808732743e-07
780 1.46451796467773e-07
781 1.46355461083658e-07
782 1.46258926747578e-07
783 1.46164580883124e-07
784 1.46069069728583e-07
785 1.45974212273359e-07
786 1.45876668966594e-07
787 1.4578006357624e-07
788 1.45687337749223e-07
789 1.45591002365109e-07
790 1.45495988590483e-07
791 1.45402196949362e-07
792 1.45305691034991e-07
793 1.45210677260366e-07
794 1.45116189287364e-07
795 1.45021104458465e-07
796 1.44924001688196e-07
797 1.44832583259813e-07
798 1.44738336871342e-07
799 1.44641944643809e-07
800 1.44546547176105e-07
801 1.44455910344732e-07
802 1.44358281772838e-07
803 1.44265555945822e-07
804 1.44170783755726e-07
805 1.44076224728451e-07
806 1.43980216193995e-07
807 1.43886325076892e-07
808 1.43794295581756e-07
809 1.43698301258155e-07
810 1.43605049629514e-07
811 1.43510263228563e-07
812 1.43416968967358e-07
813 1.433215572888e-07
814 1.4323101993341e-07
815 1.43138521480068e-07
816 1.43041646083475e-07
817 1.42950185022528e-07
818 1.4285775762346e-07
819 1.42765486543794e-07
820 1.42673030723017e-07
821 1.42577633255314e-07
822 1.42484850584879e-07
823 1.42393815849573e-07
824 1.42300009997598e-07
825 1.42208534725796e-07
826 1.42115084145189e-07
827 1.4202572629074e-07
828 1.41934620501161e-07
829 1.41840416745254e-07
830 1.41750831517129e-07
831 1.41659313612763e-07
832 1.41570254186263e-07
833 1.41477272563861e-07
834 1.41387914709412e-07
835 1.41300063205563e-07
836 1.41208118975555e-07
837 1.41119727459227e-07
838 1.41029389055802e-07
839 1.40937856940582e-07
840 1.4084987753904e-07
841 1.40762224987157e-07
842 1.40674416115871e-07
843 1.40581391860906e-07
844 1.40493810363296e-07
845 1.4040554674466e-07
846 1.40316544161578e-07
847 1.40227740530463e-07
848 1.40138922688493e-07
849 1.40051781727379e-07
850 1.39964072332077e-07
851 1.39877798233101e-07
852 1.39789591457884e-07
853 1.39702976298395e-07
854 1.39615991656683e-07
855 1.39528779641296e-07
856 1.39441638680182e-07
857 1.39356245654199e-07
858 1.39268863108555e-07
859 1.39182873226673e-07
860 1.39095121198807e-07
861 1.39012740874023e-07
862 1.38924193038292e-07
863 1.38838899488292e-07
864 1.38755083867181e-07
865 1.38670074534275e-07
866 1.38583288844529e-07
867 1.38498094770512e-07
868 1.38415558126326e-07
869 1.38329198762221e-07
870 1.38244843128632e-07
871 1.38157503215552e-07
872 1.3807554921641e-07
873 1.37991747806154e-07
874 1.37908116926155e-07
875 1.3782528185402e-07
876 1.37740329364533e-07
877 1.37655305820772e-07
878 1.37574176051203e-07
879 1.37490872020862e-07
880 1.37406757971803e-07
881 1.37324605020694e-07
882 1.37240363073943e-07
883 1.37159304358647e-07
884 1.37074394501724e-07
885 1.36991232579931e-07
886 1.36910600190276e-07
887 1.36829626740109e-07
888 1.36746095336093e-07
889 1.36663800276438e-07
890 1.36581661536184e-07
891 1.36499906489007e-07
892 1.36418265128668e-07
893 1.3633861328799e-07
894 1.36253049731749e-07
895 1.3617291472201e-07
896 1.36091529157056e-07
897 1.36010555706889e-07
898 1.35928075906122e-07
899 1.35849333560145e-07
900 1.35769099074423e-07
901 1.35688623004171e-07
902 1.35606953222123e-07
903 1.35526121880503e-07
904 1.35444821580677e-07
905 1.35363563913415e-07
906 1.35282590463248e-07
907 1.35202697038039e-07
908 1.35124679445653e-07
909 1.3504148910215e-07
910 1.3496490680609e-07
911 1.34883961777632e-07
912 1.34803201490286e-07
913 1.34724828626531e-07
914 1.34644395188843e-07
915 1.34568679754921e-07
916 1.34487805780736e-07
917 1.34407869722963e-07
918 1.34328587364507e-07
919 1.34250512928702e-07
920 1.34171344257084e-07
921 1.34092132952901e-07
922 1.34014783270686e-07
923 1.3393724884736e-07
924 1.33858634399076e-07
925 1.33783359501649e-07
926 1.33705810867468e-07
927 1.3362863171551e-07
928 1.33550429382012e-07
929 1.33472255470224e-07
930 1.33398714297073e-07
931 1.33323140971697e-07
932 1.33244625999396e-07
933 1.33167915805643e-07
934 1.33093749354884e-07
935 1.33019526060707e-07
936 1.32944322217554e-07
937 1.3286917521782e-07
938 1.32797623564329e-07
939 1.32723400270152e-07
940 1.3264738640828e-07
941 1.32570932009912e-07
942 1.32498243488044e-07
943 1.32423181753438e-07
944 1.32350606918408e-07
945 1.32275999931153e-07
946 1.32199886593298e-07
947 1.32127112806302e-07
948 1.32052235812807e-07
949 1.31976676698287e-07
950 1.3190366132676e-07
951 1.31825260041296e-07
952 1.31751846765837e-07
953 1.31675761849692e-07
954 1.31600543795685e-07
955 1.31523819391077e-07
956 1.31449510831771e-07
957 1.31371550082804e-07
958 1.31296602035036e-07
959 1.31221440824447e-07
960 1.31147800175313e-07
961 1.31069583630961e-07
962 1.30993171865157e-07
963 1.30916021134908e-07
964 1.30835957179443e-07
965 1.30760042793554e-07
966 1.30681257815013e-07
967 1.30602600734164e-07
968 1.3052745373443e-07
969 1.30447347146401e-07
970 1.30363090988794e-07
971 1.30285101818117e-07
972 1.3020786582274e-07
973 1.30128142927788e-07
974 1.30044028878729e-07
975 1.29962288042407e-07
976 1.29880831423179e-07
977 1.29798664261216e-07
978 1.29714919694379e-07
979 1.29632226730791e-07
980 1.29548809013613e-07
981 1.29465576037546e-07
982 1.29384872593619e-07
983 1.2929874060319e-07
984 1.29218406641485e-07
985 1.29132843085245e-07
986 1.29051173303196e-07
987 1.28969134038925e-07
988 1.28887435835168e-07
989 1.28807570831668e-07
990 1.28732551729627e-07
991 1.28655230469121e-07
992 1.28585028846828e-07
993 1.28516930431033e-07
994 1.28449670455666e-07
995 1.28381415720469e-07
996 1.28311285152449e-07
997 1.28244906250075e-07
998 1.28179308944709e-07
999 1.28118145426015e-07
1000 1.28063703641601e-07
1001 1.28005169131029e-07
1002 1.27949562056529e-07
1003 1.27884803191591e-07
1004 1.27821905948622e-07
1005 1.27750979572738e-07
1006 1.27682852735234e-07
1007 1.27607080457892e-07
1008 1.2753510247876e-07
1009 1.27463550825269e-07
1010 1.2738938437451e-07
1011 1.27313583675459e-07
1012 1.27236930325125e-07
1013 1.27161882801374e-07
1014 1.27088156887112e-07
1015 1.27011091421991e-07
1016 1.26932846455929e-07
1017 1.26855326243458e-07
1018 1.26776654951755e-07
1019 1.267005416139e-07
1020 1.26620875562367e-07
1021 1.26544549061691e-07
1022 1.26463490346396e-07
1023 1.26384733789564e-07
1024 1.26303405068029e-07
1025 1.262202857788e-07
1026 1.26138644418461e-07
1027 1.26059049421201e-07
1028 1.25978417031547e-07
1029 1.2589693199061e-07
1030 1.25811794760011e-07
1031 1.25726828059669e-07
1032 1.25646508308819e-07
1033 1.25561001595997e-07
1034 1.254760206848e-07
1035 1.25390258176594e-07
1036 1.25306897302835e-07
1037 1.25218633684199e-07
1038 1.2513226010924e-07
1039 1.25046582866162e-07
1040 1.24959711911288e-07
1041 1.24872357787353e-07
1042 1.24786055266668e-07
1043 1.24697109527006e-07
1044 1.24606245321957e-07
1045 1.24521491784435e-07
1046 1.24434066606227e-07
1047 1.24344254004427e-07
1048 1.24255606692714e-07
1049 1.24163875625527e-07
1050 1.24075981489113e-07
1051 1.23986836797485e-07
1052 1.23898161064062e-07
1053 1.23804554164053e-07
1054 1.23713732591568e-07
1055 1.2362509949071e-07
1056 1.23535414786602e-07
1057 1.23442831068132e-07
1058 1.23354254810693e-07
1059 1.23261798989915e-07
1060 1.2316974107307e-07
1061 1.23077313674003e-07
1062 1.22986762107757e-07
1063 1.22896125276384e-07
1064 1.22805658975267e-07
1065 1.22713544215003e-07
1066 1.22624513210212e-07
1067 1.22531361057554e-07
1068 1.22436702554296e-07
1069 1.22347870501471e-07
1070 1.22253752010693e-07
1071 1.22163697824362e-07
1072 1.22068598784608e-07
1073 1.21974863986907e-07
1074 1.21882536063822e-07
1075 1.21792112395269e-07
1076 1.21698690236371e-07
1077 1.216044438479e-07
1078 1.21513835438236e-07
1079 1.21422246479597e-07
1080 1.21328298519074e-07
1081 1.21233568961543e-07
1082 1.21141354725296e-07
1083 1.21045758305627e-07
1084 1.20952137194763e-07
1085 1.20855176533041e-07
1086 1.20761242783374e-07
1087 1.20668289582682e-07
1088 1.20573787398826e-07
1089 1.20474922482572e-07
1090 1.20380164503331e-07
1091 1.20286500759903e-07
1092 1.20188161645274e-07
1093 1.2009505212518e-07
1094 1.1999975413346e-07
1095 1.19903020845413e-07
1096 1.1980765179942e-07
1097 1.19711970114622e-07
1098 1.19616842653159e-07
1099 1.19520777275284e-07
1100 1.19426843525616e-07
1101 1.19332952408513e-07
1102 1.19241761353805e-07
1103 1.19143130916655e-07
1104 1.19050596936177e-07
1105 1.18957395045527e-07
1106 1.18864136311458e-07
1107 1.18770735468843e-07
1108 1.18675949067892e-07
1109 1.18581297670062e-07
1110 1.1849054715185e-07
1111 1.18397224468936e-07
1112 1.18298963514007e-07
1113 1.1820790746242e-07
1114 1.18114471092667e-07
1115 1.18020174966205e-07
1116 1.17925715414913e-07
1117 1.17834211721402e-07
1118 1.17739148208784e-07
1119 1.17646855812836e-07
1120 1.17551813616501e-07
1121 1.17459492798844e-07
1122 1.17365964058536e-07
1123 1.17272740851604e-07
1124 1.171787715748e-07
1125 1.17089093976119e-07
1126 1.16994790744229e-07
1127 1.16896600843575e-07
1128 1.16803953176259e-07
1129 1.1671131261437e-07
1130 1.16615886724958e-07
1131 1.16523303006488e-07
1132 1.16429234253701e-07
1133 1.16337780298181e-07
1134 1.16239078806757e-07
1135 1.16148804352179e-07
1136 1.16053911369818e-07
1137 1.15961164226519e-07
1138 1.15866221506167e-07
1139 1.15771563002909e-07
1140 1.15679398504653e-07
1141 1.15585244486738e-07
1142 1.15493229202457e-07
1143 1.15397483568813e-07
1144 1.15304914061198e-07
1145 1.15210056605974e-07
1146 1.15117330778958e-07
1147 1.15025351021814e-07
1148 1.14929420647059e-07
1149 1.14836382181238e-07
1150 1.14741460777168e-07
1151 1.14647569660065e-07
1152 1.1455575332775e-07
1153 1.14457179734018e-07
1154 1.14364141268197e-07
1155 1.14271941242805e-07
1156 1.1417574086181e-07
1157 1.14084592439667e-07
1158 1.13988150474142e-07
1159 1.13895389119989e-07
1160 1.13800325607372e-07
1161 1.1370621422202e-07
1162 1.13612742325131e-07
1163 1.13517678812514e-07
1164 1.13424668768403e-07
1165 1.13331701356856e-07
1166 1.13236112042614e-07
1167 1.13143698854401e-07
1168 1.13047128991184e-07
1169 1.12957216913401e-07
1170 1.12862338141895e-07
1171 1.12767317261842e-07
1172 1.12673021135379e-07
1173 1.12578845801181e-07
1174 1.12486134185019e-07
1175 1.1239389863249e-07
1176 1.12300291732481e-07
1177 1.12204723734521e-07
1178 1.12110498662332e-07
1179 1.12015754893946e-07
1180 1.11922553003296e-07
1181 1.11829770332861e-07
1182 1.11736028429732e-07
1183 1.11641291766773e-07
1184 1.1154887857856e-07
1185 1.11454610873807e-07
1186 1.11365416444187e-07
1187 1.11269429226013e-07
1188 1.11175083361559e-07
1189 1.11085739717964e-07
1190 1.10994307078727e-07
1191 1.1090096307953e-07
1192 1.10808286990505e-07
1193 1.10719931001313e-07
1194 1.10627766503058e-07
1195 1.1053880655254e-07
1196 1.10447700762961e-07
1197 1.1036270564091e-07
1198 1.10275792053471e-07
1199 1.10190676139155e-07
1200 1.10111471940399e-07
1201 1.10039209744173e-07
1202 1.09969981565428e-07
1203 1.09897740685483e-07
1204 1.09822039462415e-07
1205 1.09741890241821e-07
1206 1.09653726099168e-07
1207 1.09565718275917e-07
1208 1.09479110221855e-07
1209 1.0938880024014e-07
1210 1.09299278960862e-07
1211 1.09209786103293e-07
1212 1.09120847469057e-07
1213 1.09030487749351e-07
1214 1.08942252552424e-07
1215 1.08854557367977e-07
1216 1.08763678952073e-07
1217 1.08678960941688e-07
1218 1.08590938907582e-07
1219 1.08500806561551e-07
1220 1.08414091926079e-07
1221 1.08328165993044e-07
1222 1.08241763996375e-07
1223 1.08150672417651e-07
1224 1.08067538917567e-07
1225 1.07978053165425e-07
1226 1.07892390133202e-07
1227 1.07803316495847e-07
1228 1.07717468722512e-07
1229 1.07630043544304e-07
1230 1.07546469507724e-07
1231 1.07459939613364e-07
1232 1.07373871571781e-07
1233 1.07288265382977e-07
1234 1.07202829724429e-07
1235 1.07120392556226e-07
1236 1.07035418750456e-07
1237 1.06949322287164e-07
1238 1.06864973759002e-07
1239 1.06779047825967e-07
1240 1.06696425916653e-07
1241 1.06612851880072e-07
1242 1.06530400501015e-07
1243 1.06445995129434e-07
1244 1.06367551211406e-07
1245 1.06281312639567e-07
1246 1.06199216531877e-07
1247 1.06118896781027e-07
1248 1.06034967473079e-07
1249 1.0595558563864e-07
1250 1.05872921096761e-07
1251 1.05794185856212e-07
1252 1.05712011588821e-07
1253 1.05631464464295e-07
1254 1.05551407614257e-07
1255 1.0547169893016e-07
1256 1.05392402360849e-07
1257 1.05312082609998e-07
1258 1.05233411318295e-07
1259 1.05154768448301e-07
1260 1.05077390344377e-07
1261 1.04998413519297e-07
1262 1.04918477461524e-07
1263 1.04843870474269e-07
1264 1.04763323349744e-07
1265 1.04689021895865e-07
1266 1.04613512519336e-07
1267 1.04534727540795e-07
1268 1.04460880834267e-07
1269 1.04383616417181e-07
1270 1.04310480253389e-07
1271 1.04234267439551e-07
1272 1.04158772273877e-07
1273 1.04085088992178e-07
1274 1.04010972279411e-07
1275 1.03934993944677e-07
1276 1.03860671174516e-07
1277 1.03791819583421e-07
1278 1.03715237287361e-07
1279 1.036409145172e-07
1280 1.03567053599818e-07
1281 1.03494599557052e-07
1282 1.0342434109134e-07
1283 1.03348618551991e-07
1284 1.03276505569738e-07
1285 1.03203753099024e-07
1286 1.03132705930875e-07
1287 1.03060443734648e-07
1288 1.0298439434564e-07
1289 1.02914221145056e-07
1290 1.02838733084809e-07
1291 1.02766925635933e-07
1292 1.02693157089107e-07
1293 1.0261928196087e-07
1294 1.0254503735041e-07
1295 1.02470352203454e-07
1296 1.02394629664104e-07
1297 1.02318018946335e-07
1298 1.02244499089466e-07
1299 1.02168719706697e-07
1300 1.02091540554738e-07
1301 1.02014780622994e-07
1302 1.01938582020011e-07
1303 1.01862262624763e-07
1304 1.01783804495881e-07
1305 1.01704351607168e-07
1306 1.0162666086444e-07
1307 1.01547932729318e-07
1308 1.01465680302226e-07
1309 1.01385438711077e-07
1310 1.01300905441803e-07
1311 1.012237405007e-07
1312 1.01139427499675e-07
1313 1.01057999302157e-07
1314 1.00974247629892e-07
1315 1.00890339638227e-07
1316 1.00807206138143e-07
1317 1.00718757778395e-07
1318 1.00635368482926e-07
1319 1.00551652337799e-07
1320 1.00464916386045e-07
1321 1.00376531975144e-07
1322 1.00293419791342e-07
1323 1.00203529029841e-07
1324 1.00115443046889e-07
1325 1.00027001792569e-07
1326 9.99424258907311e-08
1327 9.9852989876581e-08
1328 9.9760754324052e-08
1329 9.96692293142587e-08
1330 9.95800846226302e-08
1331 9.94867619397155e-08
1332 9.93959403672307e-08
1333 9.93043229868817e-08
1334 9.92108866171293e-08
1335 9.91227651070403e-08
1336 9.90294353186982e-08
1337 9.8933142567148e-08
1338 9.88435076010319e-08
1339 9.87452679623857e-08
1340 9.86531318858397e-08
1341 9.85597523595061e-08
1342 9.84633885536823e-08
1343 9.83705632506826e-08
1344 9.82708598940008e-08
1345 9.81759527007853e-08
1346 9.80802212779963e-08
1347 9.79845253823441e-08
1348 9.78829604036946e-08
1349 9.77869802909481e-08
1350 9.76864669155475e-08
1351 9.75891367716031e-08
1352 9.7489667894024e-08
1353 9.73926219671739e-08
1354 9.72932809872873e-08
1355 9.71935421034686e-08
1356 9.70928866195209e-08
1357 9.69923661386929e-08
1358 9.6894012813209e-08
1359 9.67899609349843e-08
1360 9.66905133736873e-08
1361 9.65872501978993e-08
1362 9.64840367601028e-08
1363 9.63813420185033e-08
1364 9.62769490797655e-08
1365 9.61784607511618e-08
1366 9.60713890663101e-08
1367 9.59683248424881e-08
1368 9.58641521719983e-08
1369 9.57614574303989e-08
1370 9.5654343112983e-08
1371 9.55489056764236e-08
1372 9.54439087763603e-08
1373 9.533801659245e-08
1374 9.52304404222559e-08
1375 9.51220755496252e-08
1376 9.50159133594752e-08
1377 9.49085787738113e-08
1378 9.48010665524635e-08
1379 9.46907050547452e-08
1380 9.45809688346344e-08
1381 9.44721918472169e-08
1382 9.43631022209956e-08
1383 9.42562436989647e-08
1384 9.41458395686823e-08
1385 9.40348456879292e-08
1386 9.39261965982041e-08
1387 9.38161335284349e-08
1388 9.37069941642221e-08
1389 9.3598025330266e-08
1390 9.34867685487006e-08
1391 9.33748864895279e-08
1392 9.32661805563839e-08
1393 9.31553216787506e-08
1394 9.30431127699194e-08
1395 9.29310246533532e-08
1396 9.28207128936265e-08
1397 9.27093495306508e-08
1398 9.26000822687456e-08
1399 9.24868075458107e-08
1400 9.23756999782199e-08
1401 9.22637539702009e-08
1402 9.21549343502193e-08
1403 9.20443952168171e-08
1404 9.19287401757174e-08
1405 9.18164033691937e-08
1406 9.17047202619869e-08
1407 9.15906781528975e-08
1408 9.1479279262785e-08
1409 9.13661466483973e-08
1410 9.12534616759331e-08
1411 9.1140023528169e-08
1412 9.10233239892477e-08
1413 9.09124437953324e-08
1414 9.07966821728223e-08
1415 9.06836135072808e-08
1416 9.05722856714419e-08
1417 9.04585490957288e-08
1418 9.03452317402298e-08
1419 9.0227686655453e-08
1420 9.01158045962802e-08
1421 8.9999403485308e-08
1422 8.98861003406637e-08
1423 8.97739766969607e-08
1424 8.9660922242274e-08
1425 8.95458214245082e-08
1426 8.94337262025147e-08
1427 8.93216451913759e-08
1428 8.92098626081861e-08
1429 8.90957778665324e-08
1430 8.89845210849671e-08
1431 8.88740885329753e-08
1432 8.87636915081202e-08
1433 8.86524347265549e-08
1434 8.85406663542199e-08
1435 8.84344899532152e-08
1436 8.83218902458793e-08
1437 8.82158701642766e-08
1438 8.81065673752346e-08
1439 8.7999715958631e-08
1440 8.78943922089093e-08
1441 8.77890826700423e-08
1442 8.76819328254896e-08
1443 8.75800054700449e-08
1444 8.7474539611776e-08
1445 8.73720296112879e-08
1446 8.72718857181098e-08
1447 8.7167691731338e-08
1448 8.70683649623061e-08
1449 8.6970167956224e-08
1450 8.68714522539449e-08
1451 8.67753726652154e-08
1452 8.66793996578963e-08
1453 8.65792983972824e-08
1454 8.64836025016302e-08
1455 8.63871036926867e-08
1456 8.62931415213097e-08
1457 8.62012541347212e-08
1458 8.6105238494838e-08
1459 8.60139763858569e-08
1460 8.59234816630305e-08
1461 8.58329514130673e-08
1462 8.57421724731466e-08
1463 8.56512372138241e-08
1464 8.55588737636026e-08
1465 8.54706243558212e-08
1466 8.53808472811579e-08
1467 8.52928963013255e-08
1468 8.52055066502544e-08
1469 8.51177901495248e-08
1470 8.50311110411894e-08
1471 8.49428900551175e-08
1472 8.48551309218237e-08
1473 8.4769617103575e-08
1474 8.46852969971224e-08
1475 8.45944683192101e-08
1476 8.4513118281393e-08
1477 8.44243714936965e-08
1478 8.4337926864464e-08
1479 8.42527541067284e-08
1480 8.41660039441194e-08
1481 8.40793603629209e-08
1482 8.39972926769406e-08
1483 8.3909178272279e-08
1484 8.38223996879606e-08
1485 8.37383780094569e-08
1486 8.36509954638132e-08
1487 8.35655669106927e-08
1488 8.34829450013785e-08
1489 8.33935160926558e-08
1490 8.33045987747028e-08
1491 8.3218090196624e-08
1492 8.31351485430787e-08
1493 8.30489739200857e-08
1494 8.29616695341429e-08
1495 8.28735409186265e-08
1496 8.27871389219581e-08
1497 8.27041688467034e-08
1498 8.26147825705448e-08
1499 8.25275066063114e-08
1500 8.24392643039573e-08
1501 8.23506454139533e-08
1502 8.2266410572629e-08
1503 8.21783459059588e-08
1504 8.20896914888181e-08
1505 8.20008168034292e-08
1506 8.19139103214184e-08
1507 8.1824310882439e-08
1508 8.1733844581322e-08
1509 8.16466680930716e-08
1510 8.15590084357609e-08
1511 8.14713985164417e-08
1512 8.13797385035286e-08
1513 8.12926614912612e-08
1514 8.12050586773694e-08
1515 8.11173421766398e-08
1516 8.10263003359069e-08
1517 8.09380296118434e-08
1518 8.08494178272667e-08
1519 8.07617226428192e-08
1520 8.06737645575595e-08
1521 8.0583248518451e-08
1522 8.04936703957537e-08
1523 8.04050586111771e-08
1524 8.03169015739513e-08
1525 8.02261652665948e-08
1526 8.0137311897488e-08
1527 8.00482240492784e-08
1528 7.99607420276516e-08
1529 7.98709010041421e-08
1530 7.97831916088398e-08
1531 7.96937342784076e-08
1532 7.96074388631496e-08
1533 7.95146135601499e-08
1534 7.9427884713823e-08
1535 7.93391876641181e-08
1536 7.92532759419373e-08
1537 7.91639394037702e-08
1538 7.9078397163812e-08
1539 7.89888190411148e-08
1540 7.89031275871821e-08
1541 7.88140823715366e-08
1542 7.87248524147799e-08
1543 7.86390543794369e-08
1544 7.85524179036656e-08
1545 7.84646374540898e-08
1546 7.83755140787434e-08
1547 7.82886075967326e-08
1548 7.82046996050667e-08
1549 7.81170541586107e-08
1550 7.80316256054903e-08
1551 7.79472841827555e-08
1552 7.78588358230081e-08
1553 7.7777706053439e-08
1554 7.76878366082201e-08
1555 7.76035946614684e-08
1556 7.75176260958688e-08
1557 7.74346560206141e-08
1558 7.73511317220255e-08
1559 7.72631381096289e-08
1560 7.71804309351865e-08
1561 7.70971837482648e-08
1562 7.70141284078818e-08
1563 7.69289485447189e-08
1564 7.68471650758329e-08
1565 7.67652750255365e-08
1566 7.66805428042971e-08
1567 7.65957182125021e-08
1568 7.65142829095566e-08
1569 7.64336505199026e-08
1570 7.63486127652868e-08
1571 7.62681295896073e-08
1572 7.61868150789269e-08
1573 7.61055076736739e-08
1574 7.60247971243189e-08
1575 7.59441149966733e-08
1576 7.58646265808238e-08
1577 7.57813580776201e-08
1578 7.57049036792523e-08
1579 7.5620867789894e-08
1580 7.55420970222076e-08
1581 7.54628075583241e-08
1582 7.53833617750388e-08
1583 7.53063460479098e-08
1584 7.52277173887705e-08
1585 7.51488400396738e-08
1586 7.50719735265193e-08
1587 7.49952064893478e-08
1588 7.4916606251918e-08
1589 7.48381197013259e-08
1590 7.47626671682156e-08
1591 7.4686951734293e-08
1592 7.4607434896734e-08
1593 7.45310870797766e-08
1594 7.44596349022686e-08
1595 7.43820578463783e-08
1596 7.43077848142093e-08
1597 7.42321191182782e-08
1598 7.41600700848721e-08
1599 7.40852073022324e-08
1600 7.40106145258324e-08
1601 7.3936234912253e-08
1602 7.38638661346158e-08
1603 7.37931173944162e-08
1604 7.37204288725479e-08
1605 7.36488772190569e-08
1606 7.3573751535605e-08
1607 7.35038980792524e-08
1608 7.34314085093501e-08
1609 7.33608160885524e-08
1610 7.32900815592075e-08
1611 7.32220968302499e-08
1612 7.31520657382134e-08
1613 7.30821980710061e-08
1614 7.30119324998668e-08
1615 7.2942214046634e-08
1616 7.28753875023358e-08
1617 7.28075022493613e-08
1618 7.27380964349322e-08
1619 7.26689535213154e-08
1620 7.26001587736391e-08
1621 7.25355775443859e-08
1622 7.24701507692771e-08
1623 7.2402869477628e-08
1624 7.23362987287146e-08
1625 7.22709572187341e-08
1626 7.22039956713161e-08
1627 7.21373041301376e-08
1628 7.20739734560993e-08
1629 7.20082553584689e-08
1630 7.19429920081893e-08
1631 7.18822690259913e-08
1632 7.1813495594597e-08
1633 7.17512733672265e-08
1634 7.16881700668637e-08
1635 7.16228711894473e-08
1636 7.15585741772884e-08
1637 7.14984480509884e-08
1638 7.14348047381463e-08
1639 7.13720993417155e-08
1640 7.13093584181479e-08
1641 7.12483654297102e-08
1642 7.11855605572964e-08
1643 7.11216543436421e-08
1644 7.10629493028136e-08
1645 7.10007341808705e-08
1646 7.09386966946113e-08
1647 7.08801977111762e-08
1648 7.08196949972262e-08
1649 7.07570677604963e-08
1650 7.06969416341963e-08
1651 7.06381868553763e-08
1652 7.05756946217662e-08
1653 7.0518566985811e-08
1654 7.04570268794669e-08
1655 7.0398790796844e-08
1656 7.03411657809738e-08
1657 7.02801230545447e-08
1658 7.02238764915819e-08
1659 7.01627413945971e-08
1660 7.01052869089835e-08
1661 7.00471503023437e-08
1662 6.99893547562169e-08
1663 6.99325113373561e-08
1664 6.98718025660128e-08
1665 6.98175668389922e-08
1666 6.97576396646582e-08
1667 6.97015138939605e-08
1668 6.96451962767242e-08
1669 6.95863278110664e-08
1670 6.9532212876311e-08
1671 6.9474552333304e-08
1672 6.94158543979029e-08
1673 6.93637645099443e-08
1674 6.93059618583902e-08
1675 6.92490402798285e-08
1676 6.91939447960976e-08
1677 6.91362060933898e-08
1678 6.90811461367957e-08
1679 6.90252690560555e-08
1680 6.89710120127529e-08
1681 6.8916591544621e-08
1682 6.885888836905e-08
1683 6.88048515939954e-08
1684 6.87535930410377e-08
1685 6.86976804331607e-08
1686 6.86413841322064e-08
1687 6.85870347183481e-08
1688 6.85311505321806e-08
1689 6.84796077621286e-08
1690 6.84247325466458e-08
1691 6.83697152226159e-08
1692 6.83179806060252e-08
1693 6.82640788340905e-08
1694 6.82101699567284e-08
1695 6.81575826888547e-08
1696 6.8106125183931e-08
1697 6.80503973171653e-08
1698 6.79971918771116e-08
1699 6.79444482898361e-08
1700 6.78925147212794e-08
1701 6.78395082331917e-08
1702 6.77859475217701e-08
1703 6.77347102850945e-08
1704 6.76836577895301e-08
1705 6.76310776270839e-08
1706 6.7576159779037e-08
1707 6.75241480507793e-08
1708 6.74748363849176e-08
1709 6.74208280315725e-08
1710 6.73717366339588e-08
1711 6.73202222856162e-08
1712 6.72673365897936e-08
1713 6.72167317361527e-08
1714 6.71628583859274e-08
1715 6.7113653301476e-08
1716 6.70621531639881e-08
1717 6.70135520408621e-08
1718 6.69612987280743e-08
1719 6.69120723273409e-08
1720 6.68601529696389e-08
1721 6.68098820710838e-08
1722 6.67611672611201e-08
1723 6.67109318897019e-08
1724 6.66592399056753e-08
1725 6.66100419266513e-08
1726 6.65601902483104e-08
1727 6.65103243591147e-08
1728 6.64608634792785e-08
1729 6.64090151758501e-08
1730 6.63631922748209e-08
1731 6.63110313325888e-08
1732 6.6261996778394e-08
1733 6.62135590800972e-08
1734 6.61660877199211e-08
1735 6.61139267776889e-08
1736 6.60663275198203e-08
1737 6.60187353673791e-08
1738 6.59695089666457e-08
1739 6.59203251984763e-08
1740 6.58721504009918e-08
1741 6.58251053664571e-08
1742 6.57767742495707e-08
1743 6.5727860487641e-08
1744 6.56811707244742e-08
1745 6.56320651160058e-08
1746 6.55864624832248e-08
1747 6.55392611292882e-08
1748 6.54897860385972e-08
1749 6.54422009915834e-08
1750 6.53943743600394e-08
1751 6.5349958333627e-08
1752 6.52990976846013e-08
1753 6.52534239975466e-08
1754 6.52069260809185e-08
1755 6.51609894930516e-08
1756 6.51127010087293e-08
1757 6.50659544021437e-08
1758 6.50187530482071e-08
1759 6.49728235657676e-08
1760 6.4926574339097e-08
1761 6.4882939909694e-08
1762 6.48334008701568e-08
1763 6.47872724357512e-08
1764 6.47444338142122e-08
1765 6.46972466711304e-08
1766 6.46534132897614e-08
1767 6.46059135078758e-08
1768 6.45604885107787e-08
1769 6.45162074874861e-08
1770 6.44687219164553e-08
1771 6.44253148607277e-08
1772 6.43806217226484e-08
1773 6.4334734872773e-08
1774 6.42888551283249e-08
1775 6.42462865130256e-08
1776 6.42012665252878e-08
1777 6.41568789205849e-08
1778 6.41116670863084e-08
1779 6.40703206045146e-08
1780 6.40265795937012e-08
1781 6.3982128040152e-08
1782 6.3937996230834e-08
1783 6.38928980833953e-08
1784 6.38470822877935e-08
1785 6.3808109018737e-08
1786 6.3762428226255e-08
1787 6.37190780139463e-08
1788 6.36761683381337e-08
1789 6.3632221269927e-08
1790 6.35890202715927e-08
1791 6.35473824672772e-08
1792 6.35030659168478e-08
1793 6.34630055174057e-08
1794 6.34175663094538e-08
1795 6.33753032275308e-08
1796 6.33336298960785e-08
1797 6.32884180618021e-08
1798 6.3249750326122e-08
1799 6.32087164831319e-08
1800 6.31655581173618e-08
1801 6.31239416293283e-08
1802 6.30821332947562e-08
1803 6.30412415603132e-08
1804 6.30008045732211e-08
1805 6.29589891332216e-08
1806 6.29164489396317e-08
1807 6.28760759013858e-08
1808 6.28362499810464e-08
1809 6.27967295940834e-08
1810 6.27541325570746e-08
1811 6.27154506105398e-08
1812 6.26764276034919e-08
1813 6.26333047648586e-08
1814 6.25947151888795e-08
1815 6.25517913022122e-08
1816 6.25142178023452e-08
1817 6.24713365482421e-08
1818 6.24363778456427e-08
1819 6.23941502908565e-08
1820 6.23552054435095e-08
1821 6.231476135099e-08
1822 6.22773228542428e-08
1823 6.22395717186919e-08
1824 6.21977349624103e-08
1825 6.21598204020302e-08
1826 6.21228366526339e-08
1827 6.20820728158833e-08
1828 6.20441795717852e-08
1829 6.20083255853388e-08
1830 6.19683646618796e-08
1831 6.19285955849591e-08
1832 6.18920168449222e-08
1833 6.18529512053101e-08
1834 6.18139210928348e-08
1835 6.17798079360909e-08
1836 6.17425612858824e-08
1837 6.17047248852032e-08
1838 6.16693753840991e-08
1839 6.16304589584615e-08
1840 6.15941360138095e-08
1841 6.15587936181328e-08
1842 6.15212556454026e-08
1843 6.14853092884005e-08
1844 6.14492208228512e-08
1845 6.14109438856758e-08
1846 6.13743722510662e-08
1847 6.13389516956886e-08
1848 6.13040498365081e-08
1849 6.12685298051474e-08
1850 6.12311410463917e-08
1851 6.11969142028101e-08
1852 6.11635400105115e-08
1853 6.11267338967991e-08
1854 6.10915549259516e-08
1855 6.10585644267303e-08
1856 6.10228383379763e-08
1857 6.09891017688824e-08
1858 6.09533117312822e-08
1859 6.09199375389835e-08
1860 6.08850072580935e-08
1861 6.08507093602384e-08
1862 6.08156653925107e-08
1863 6.07827601584177e-08
1864 6.07514749617621e-08
1865 6.07178591849333e-08
1866 6.06840089290017e-08
1867 6.06502297273437e-08
1868 6.06156547178216e-08
1869 6.05827992217201e-08
1870 6.05519261398513e-08
1871 6.05183672064413e-08
1872 6.04867764764094e-08
1873 6.04536936066324e-08
1874 6.04228631573278e-08
1875 6.03903131946026e-08
1876 6.03583885094849e-08
1877 6.03275367438982e-08
1878 6.02958039053192e-08
1879 6.02640355396034e-08
1880 6.02356990953012e-08
1881 6.02012377726169e-08
1882 6.01710823389112e-08
1883 6.01396479282812e-08
1884 6.01104730435509e-08
1885 6.00809570983074e-08
1886 6.00464034050674e-08
1887 6.00203051703829e-08
1888 5.99899436792839e-08
1889 5.99598664052792e-08
1890 5.99279417201615e-08
1891 5.99015734792374e-08
1892 5.9867865331853e-08
1893 5.9840246535714e-08
1894 5.98127911644042e-08
1895 5.97829199477928e-08
1896 5.97530700474636e-08
1897 5.97239591115795e-08
1898 5.96971574395866e-08
1899 5.96696381194306e-08
1900 5.96414793108124e-08
1901 5.96128231222792e-08
1902 5.95857088114826e-08
1903 5.95583102835917e-08
1904 5.95317892759795e-08
1905 5.95041065309942e-08
1906 5.94734288483778e-08
1907 5.94487374883101e-08
1908 5.94210014526197e-08
1909 5.93935354231689e-08
1910 5.93669824411336e-08
1911 5.93413922445052e-08
1912 5.93162390316593e-08
1913 5.92868971693861e-08
1914 5.92624438411349e-08
1915 5.9236128890916e-08
1916 5.92109259400786e-08
1917 5.9185072842638e-08
1918 5.91575464170546e-08
1919 5.91340203470736e-08
1920 5.91069237998454e-08
1921 5.90813584722127e-08
1922 5.9057132517637e-08
1923 5.90316915349831e-08
1924 5.90067372741032e-08
1925 5.89817368279455e-08
1926 5.89580508858489e-08
1927 5.89301514253293e-08
1928 5.89072364221011e-08
1929 5.88824455860504e-08
1930 5.88595696626726e-08
1931 5.883393683348e-08
1932 5.8811274072923e-08
1933 5.87856305855894e-08
1934 5.87624917613994e-08
1935 5.87398041318465e-08
1936 5.87158694997925e-08
1937 5.86932600299406e-08
1938 5.86697694870963e-08
1939 5.86450923378834e-08
1940 5.86200670227299e-08
1941 5.85990740376019e-08
1942 5.8575668759886e-08
1943 5.85510555595192e-08
1944 5.85292667665271e-08
1945 5.85072221781502e-08
1946 5.84849182416747e-08
1947 5.84622590338313e-08
1948 5.84398627268001e-08
1949 5.84166066630587e-08
1950 5.83926613728636e-08
1951 5.83709152124356e-08
1952 5.83503485529491e-08
1953 5.83277106613878e-08
1954 5.83025396849735e-08
1955 5.82819623673458e-08
1956 5.82600137022382e-08
1957 5.82390136116828e-08
1958 5.82177293040331e-08
1959 5.8195233521019e-08
1960 5.81731676163599e-08
1961 5.8152483717322e-08
1962 5.8131135460826e-08
1963 5.81063588356301e-08
1964 5.80844563558003e-08
1965 5.80649199832806e-08
1966 5.80456571697141e-08
1967 5.80209871259285e-08
1968 5.80002357253306e-08
1969 5.79809906753326e-08
1970 5.79586831861434e-08
1971 5.7936897945865e-08
1972 5.79160754909935e-08
1973 5.78930645644959e-08
1974 5.78720680266542e-08
1975 5.7854638413346e-08
1976 5.78310661580872e-08
1977 5.78105492365921e-08
1978 5.77907108834097e-08
1979 5.77698102688373e-08
1980 5.7747161719135e-08
1981 5.77285277358897e-08
1982 5.77080996322366e-08
1983 5.76915653027754e-08
1984 5.76662841922371e-08
1985 5.76448861977497e-08
1986 5.76252183748238e-08
1987 5.76032448407204e-08
1988 5.75819782966391e-08
1989 5.75618237519393e-08
1990 5.75430263438648e-08
1991 5.75219800680316e-08
1992 5.75025396187812e-08
1993 5.7480718851366e-08
1994 5.74605749648072e-08
1995 5.74407401643384e-08
1996 5.74186600488247e-08
1997 5.73988074847875e-08
1998 5.7377278039894e-08
1999 5.73573224471602e-08
2000 5.73349083765606e-08
2001 5.73163951855804e-08
2002 5.72968232859239e-08
2003 5.72751375216285e-08
2004 5.72546277055608e-08
2005 5.72327785164362e-08
2006 5.72124072562019e-08
2007 5.71904443802396e-08
2008 5.7170176148702e-08
2009 5.71485614386802e-08
2010 5.71262752657731e-08
2011 5.71044438402168e-08
2012 5.70875577921015e-08
2013 5.70659999254985e-08
2014 5.70475755523603e-08
2015 5.7025545174838e-08
2016 5.70035147973158e-08
2017 5.69843692233007e-08
2018 5.69629960978091e-08
2019 5.69419746909716e-08
2020 5.69212410539421e-08
2021 5.69009550588362e-08
2022 5.68818094848211e-08
2023 5.68598892414229e-08
2024 5.6837457407255e-08
2025 5.68200704265109e-08
2026 5.67984415056344e-08
2027 5.67781306415327e-08
2028 5.6756757516041e-08
2029 5.67365745496318e-08
2030 5.67170559406804e-08
2031 5.6694961614312e-08
2032 5.66778233235254e-08
2033 5.66565070414526e-08
2034 5.66356703757265e-08
2035 5.66152529302144e-08
2036 5.65965017074177e-08
2037 5.65765603255386e-08
2038 5.65567965793434e-08
2039 5.65371109928492e-08
2040 5.6515887081332e-08
2041 5.64962974181071e-08
2042 5.64754962795178e-08
2043 5.64589477392019e-08
2044 5.64385089774078e-08
2045 5.64171607209119e-08
2046 5.63959581256768e-08
2047 5.63771216377518e-08
2048 5.63546649345881e-08
2049 5.63367201777965e-08
2050 5.63168391920499e-08
2051 5.62949722393569e-08
2052 5.62738797782458e-08
2053 5.6256556746348e-08
2054 5.62374786738928e-08
2055 5.62169049089789e-08
2056 5.61970736612238e-08
2057 5.61779778252003e-08
2058 5.61592408132583e-08
2059 5.61383330932586e-08
2060 5.61178836733234e-08
2061 5.60980915054188e-08
2062 5.60795108128787e-08
2063 5.60589050735416e-08
2064 5.6041248086558e-08
2065 5.60193136323051e-08
2066 5.59997452853622e-08
2067 5.59799673283123e-08
2068 5.59605872751945e-08
2069 5.59425714641293e-08
2070 5.59207364858594e-08
2071 5.59032216074229e-08
2072 5.5882956928599e-08
2073 5.58646142678754e-08
2074 5.58395747418672e-08
2075 5.58237935877059e-08
2076 5.58025128327699e-08
2077 5.57844330728585e-08
2078 5.57616033347585e-08
2079 5.57451578231394e-08
2080 5.57259873801286e-08
2081 5.57059891548306e-08
2082 5.56868045009651e-08
2083 5.56637012039118e-08
2084 5.56450565625255e-08
2085 5.56255379535742e-08
2086 5.56059411849219e-08
2087 5.55840031779553e-08
2088 5.55665593537924e-08
2089 5.55484618303126e-08
2090 5.55259838108668e-08
2091 5.55085328812766e-08
2092 5.54872876534773e-08
2093 5.54682522135863e-08
2094 5.5448058589036e-08
2095 5.542814918158e-08
2096 5.54091315052574e-08
2097 5.53901813304947e-08
2098 5.53706129835518e-08
2099 5.53490764332309e-08
2100 5.53286483295778e-08
2101 5.53097159183835e-08
2102 5.52920340624041e-08
2103 5.52694316979796e-08
2104 5.52514052287734e-08
2105 5.52311476553768e-08
2106 5.52118883945241e-08
2107 5.51955352534605e-08
2108 5.51706555995679e-08
2109 5.51487921995886e-08
2110 5.51339702781206e-08
2111 5.51124408332271e-08
2112 5.50932348630795e-08
2113 5.50718368685921e-08
2114 5.50530145915218e-08
2115 5.50342065253062e-08
2116 5.50118102182751e-08
2117 5.49915100123144e-08
2118 5.49754624046273e-08
2119 5.49559331375349e-08
2120 5.49334870925122e-08
2121 5.49142491479415e-08
2122 5.48957146406792e-08
2123 5.48742740136277e-08
2124 5.48561587265795e-08
2125 5.4836807095171e-08
2126 5.48145457912597e-08
2127 5.47936735983967e-08
2128 5.47746275003647e-08
2129 5.47551408658364e-08
2130 5.47346488133371e-08
2131 5.47171588038964e-08
2132 5.46969403103503e-08
2133 5.46756275809912e-08
2134 5.46591500949489e-08
2135 5.4637414592662e-08
2136 5.46197043149732e-08
2137 5.45997167478163e-08
2138 5.45803153784163e-08
2139 5.45607043989094e-08
2140 5.45426175335706e-08
2141 5.45225624648538e-08
2142 5.45018998820979e-08
2143 5.44831380011601e-08
2144 5.44640172961408e-08
2145 5.44430278637265e-08
2146 5.44245111200325e-08
2147 5.44060405616165e-08
2148 5.43868914348877e-08
2149 5.43650919837546e-08
2150 5.43455627166622e-08
2151 5.43283746878842e-08
2152 5.43093570115616e-08
2153 5.42904672329314e-08
2154 5.42692113469911e-08
2155 5.42505063094723e-08
2156 5.42327356356509e-08
2157 5.42139062531533e-08
2158 5.4193435516936e-08
2159 5.41756186578368e-08
2160 5.41565228218133e-08
2161 5.41351248273259e-08
2162 5.41153823974128e-08
2163 5.40996900610935e-08
2164 5.40799618420351e-08
2165 5.40628803946674e-08
2166 5.40408784388546e-08
2167 5.40248770164453e-08
2168 5.40042108809757e-08
2169 5.39854916326021e-08
2170 5.39679128053194e-08
2171 5.39473212768371e-08
2172 5.39301829860506e-08
2173 5.39115383446642e-08
2174 5.38934763483212e-08
2175 5.38744018285797e-08
2176 5.38527480387074e-08
2177 5.38362776580925e-08
2178 5.38187983067928e-08
2179 5.38005551220522e-08
2180 5.37825144419912e-08
2181 5.37617488305386e-08
2182 5.3743395511674e-08
2183 5.37262714317421e-08
2184 5.37043511883439e-08
2185 5.36873159262541e-08
2186 5.3669818811386e-08
2187 5.36524069616462e-08
2188 5.36321422828223e-08
2189 5.36132596096195e-08
2190 5.35938156076554e-08
2191 5.35783932775757e-08
2192 5.35600896967026e-08
2193 5.35391322387113e-08
2194 5.35199973228373e-08
2195 5.35035731275002e-08
2196 5.34830633114325e-08
2197 5.34685042907768e-08
2198 5.34492627934924e-08
2199 5.34319362088809e-08
2200 5.34110817795863e-08
2201 5.33938759872399e-08
2202 5.33769259902783e-08
2203 5.33574819883142e-08
2204 5.3340627914622e-08
2205 5.33216564235772e-08
2206 5.33043724715299e-08
2207 5.32860191526652e-08
2208 5.32662092211922e-08
2209 5.32503676708984e-08
2210 5.32312967038706e-08
2211 5.32153912047306e-08
2212 5.31955315352661e-08
2213 5.31787520685612e-08
2214 5.31604520404017e-08
2215 5.31435482287179e-08
2216 5.31268362635728e-08
2217 5.31070334375272e-08
2218 5.30884278759913e-08
2219 5.30720498659321e-08
2220 5.30534265408278e-08
2221 5.30359081096776e-08
2222 5.30189119274382e-08
2223 5.30012407295999e-08
2224 5.29839923046893e-08
2225 5.29627079970396e-08
2226 5.29473140886694e-08
2227 5.29309360786101e-08
2228 5.29120924852577e-08
2229 5.28947516897915e-08
2230 5.28777164277017e-08
2231 5.28610968331122e-08
2232 5.28457739790156e-08
2233 5.28266177468595e-08
2234 5.28086410156448e-08
2235 5.27912469294733e-08
2236 5.277407311155e-08
2237 5.27569596897592e-08
2238 5.27399386385241e-08
2239 5.27219228274589e-08
2240 5.27041486009239e-08
2241 5.26873265016548e-08
2242 5.26719361459982e-08
2243 5.26533803224538e-08
2244 5.26370769193818e-08
2245 5.26217895924219e-08
2246 5.26045838000755e-08
2247 5.25842693832601e-08
2248 5.25693657493775e-08
2249 5.25525223338263e-08
2250 5.25353804903261e-08
2251 5.25189491895617e-08
2252 5.25008267970861e-08
2253 5.24850030103607e-08
2254 5.24670866752786e-08
2255 5.2451419207955e-08
2256 5.24332790519111e-08
2257 5.24177075078569e-08
2258 5.24023242576277e-08
2259 5.23838750154937e-08
2260 5.23678984620801e-08
2261 5.23514351868926e-08
2262 5.23329006796303e-08
2263 5.2318956278441e-08
2264 5.23014485054318e-08
2265 5.22855110318687e-08
2266 5.22669552083244e-08
2267 5.22540126723925e-08
2268 5.22374676847903e-08
2269 5.2219597534986e-08
2270 5.2203713352128e-08
2271 5.21879215398258e-08
2272 5.21728260594045e-08
2273 5.21568388478499e-08
2274 5.21403364928119e-08
2275 5.21244167828172e-08
2276 5.21063547864742e-08
2277 5.20917211588312e-08
2278 5.2075275647212e-08
2279 5.20613987475826e-08
2280 5.20438128148726e-08
2281 5.20271683512874e-08
2282 5.20136147486028e-08
2283 5.19967677803379e-08
2284 5.19784855157468e-08
2285 5.19650313890452e-08
2286 5.1950621582364e-08
2287 5.19347906902112e-08
2288 5.19202885129744e-08
2289 5.19000309395778e-08
2290 5.18885450162543e-08
2291 5.1872454776003e-08
2292 5.18553342487849e-08
2293 5.18406508831504e-08
2294 5.18240561575567e-08
2295 5.18100868873717e-08
2296 5.17946929790014e-08
2297 5.17822797974077e-08
2298 5.17657099408098e-08
2299 5.17482661166468e-08
2300 5.17362224172757e-08
2301 5.17193079474509e-08
2302 5.17038429848071e-08
2303 5.16900620084471e-08
2304 5.16776168524302e-08
2305 5.16621945223505e-08
2306 5.16469178535317e-08
2307 5.16330516120433e-08
2308 5.16154301521965e-08
2309 5.16012299556223e-08
2310 5.15881453111433e-08
2311 5.15727904826235e-08
2312 5.1557897506882e-08
2313 5.15433313807989e-08
2314 5.15268681056114e-08
2315 5.1513787013846e-08
2316 5.15028268921469e-08
2317 5.14857489974929e-08
2318 5.14725222444667e-08
2319 5.14610327684295e-08
2320 5.14433367015954e-08
2321 5.14308098331639e-08
2322 5.14167446397096e-08
2323 5.14009705909757e-08
2324 5.13874596208552e-08
2325 5.13739983887262e-08
2326 5.1360274255785e-08
2327 5.13472748764343e-08
2328 5.13315114858415e-08
2329 5.13171798388612e-08
2330 5.13030080639965e-08
2331 5.1292637692768e-08
2332 5.12783984163434e-08
2333 5.12647631012442e-08
2334 5.12508151473412e-08
2335 5.12384268347432e-08
2336 5.12236724148352e-08
2337 5.12109536998651e-08
2338 5.11979791895101e-08
2339 5.11852888962494e-08
2340 5.11710425143974e-08
2341 5.11576629946831e-08
2342 5.11468840613816e-08
2343 5.11324991236961e-08
2344 5.11190556551355e-08
2345 5.11065927355503e-08
2346 5.10964213162879e-08
2347 5.10814928134096e-08
2348 5.1065715211962e-08
2349 5.1053305583082e-08
2350 5.1042199800122e-08
2351 5.10303834744263e-08
2352 5.10173876477893e-08
2353 5.10047328816654e-08
2354 5.09937052584064e-08
2355 5.09814057636504e-08
2356 5.0968125719919e-08
2357 5.09563484740738e-08
2358 5.09451503205582e-08
2359 5.09321651520622e-08
2360 5.09199225007251e-08
2361 5.09094206790905e-08
2362 5.08964888012997e-08
2363 5.08837558754749e-08
2364 5.08718542846509e-08
2365 5.0859576106177e-08
2366 5.08481861061227e-08
2367 5.08362454354483e-08
2368 5.08241448926583e-08
2369 5.08138846555539e-08
2370 5.08012583111395e-08
2371 5.0789306982324e-08
2372 5.07762969448322e-08
2373 5.07658199921934e-08
2374 5.07532824656209e-08
2375 5.07419493089856e-08
2376 5.07302608809823e-08
2377 5.07203843369552e-08
2378 5.07085324841228e-08
2379 5.06985884385358e-08
2380 5.06877313455334e-08
2381 5.06789170628963e-08
2382 5.06647275244632e-08
2383 5.06519555187879e-08
2384 5.06393362798008e-08
2385 5.06314243864381e-08
2386 5.06209580919403e-08
2387 5.06097066477196e-08
2388 5.05973751785405e-08
2389 5.05870474398762e-08
2390 5.05784676363419e-08
2391 5.0566409726116e-08
2392 5.05556378982419e-08
2393 5.05441875020551e-08
2394 5.05335968625786e-08
2395 5.05252870652839e-08
2396 5.05146182661065e-08
2397 5.0504912252336e-08
2398 5.04938810763633e-08
2399 5.04816810575903e-08
2400 5.04722237337774e-08
2401 5.04621553432116e-08
2402 5.04532451373052e-08
2403 5.04435213599663e-08
2404 5.04304757953378e-08
2405 5.04213240049012e-08
2406 5.04104562537577e-08
2407 5.04010984059278e-08
2408 5.03917298999568e-08
2409 5.03808585960996e-08
2410 5.03712200838891e-08
2411 5.03601143009291e-08
2412 5.03537869178672e-08
2413 5.03428729814459e-08
2414 5.03338704049838e-08
2415 5.03218515746084e-08
2416 5.03125683337657e-08
2417 5.03011499120021e-08
2418 5.02944601521449e-08
2419 5.02855570516658e-08
2420 5.02751262843049e-08
2421 5.02648802580552e-08
2422 5.0257206396509e-08
2423 5.02484880371412e-08
2424 5.02375812061473e-08
2425 5.02308630245807e-08
2426 5.02211143782461e-08
2427 5.02106054511842e-08
2428 5.02020043313678e-08
2429 5.01928454355038e-08
2430 5.01814128028855e-08
2431 5.01739485514463e-08
2432 5.01647612338729e-08
2433 5.01563590660226e-08
2434 5.01481558501382e-08
2435 5.01370251981825e-08
2436 5.01301897770645e-08
2437 5.0120274153187e-08
2438 5.01123089691191e-08
2439 5.01057826340912e-08
2440 5.00971601979927e-08
2441 5.00859549390498e-08
2442 5.00788424062648e-08
2443 5.00722912022411e-08
2444 5.00607377773576e-08
2445 5.00537922221156e-08
2446 5.00448038565082e-08
2447 5.00370518352611e-08
2448 5.00271823966614e-08
2449 5.00210752818475e-08
2450 5.0010196872563e-08
2451 5.00041927864459e-08
2452 4.99945187470985e-08
2453 4.99846706247808e-08
2454 4.99786487750953e-08
2455 4.9972324944747e-08
2456 4.99642496265551e-08
2457 4.995415281428e-08
2458 4.99461627612163e-08
2459 4.99387482477687e-08
2460 4.99302998946405e-08
2461 4.99228747230518e-08
2462 4.99141812326798e-08
2463 4.99080350380154e-08
2464 4.98994303654854e-08
2465 4.9891692555093e-08
2466 4.98838481632902e-08
2467 4.98767178669368e-08
2468 4.98693246697712e-08
2469 4.98610894794638e-08
2470 4.98543322180467e-08
2471 4.98474896915013e-08
2472 4.98376628854658e-08
2473 4.98288059702645e-08
2474 4.98263865722492e-08
2475 4.98163785778161e-08
2476 4.9810306990139e-08
2477 4.98017804773099e-08
2478 4.97938223986694e-08
2479 4.9785320754836e-08
2480 4.97775545227341e-08
2481 4.97698273704827e-08
2482 4.97636349905406e-08
2483 4.97556271739086e-08
2484 4.97493495288381e-08
2485 4.9741021967975e-08
2486 4.97330958637576e-08
2487 4.97252123921044e-08
2488 4.97165331125871e-08
2489 4.97105148156152e-08
2490 4.97027343726586e-08
2491 4.96970145036357e-08
2492 4.96891665591193e-08
2493 4.96797660787252e-08
2494 4.96743837175018e-08
2495 4.96642726943719e-08
2496 4.96561334273338e-08
2497 4.96481220579881e-08
2498 4.96443988140527e-08
2499 4.96349024103893e-08
2500 4.96263368177097e-08
2501 4.96176326691966e-08
2502 4.96115504233785e-08
2503 4.96020504670014e-08
2504 4.95954495249862e-08
2505 4.95859886484595e-08
2506 4.95759238106075e-08
2507 4.95712768611156e-08
2508 4.9563176673928e-08
2509 4.9556174275267e-08
2510 4.95470260375441e-08
2511 4.95396186295238e-08
2512 4.95300831460099e-08
2513 4.95214820261936e-08
2514 4.9511644562017e-08
2515 4.95032317360256e-08
2516 4.94962506536467e-08
2517 4.94845693310708e-08
2518 4.94780607596113e-08
2519 4.94703513709283e-08
2520 4.94598069167296e-08
2521 4.94523355598631e-08
2522 4.94444236665004e-08
2523 4.94350835822388e-08
2524 4.94253207250495e-08
2525 4.94147798235645e-08
2526 4.94051946020591e-08
2527 4.93963732139946e-08
2528 4.93900103037959e-08
2529 4.93798317791061e-08
2530 4.93721934446967e-08
2531 4.93618799168871e-08
2532 4.93514278332441e-08
2533 4.93432459336418e-08
2534 4.93316498761942e-08
2535 4.93196488093872e-08
2536 4.93132432666243e-08
2537 4.93030256620841e-08
2538 4.92932521467537e-08
2539 4.92828924336663e-08
2540 4.92735701129732e-08
2541 4.9263761070506e-08
2542 4.92554868003481e-08
2543 4.92433329668529e-08
2544 4.9232923515774e-08
2545 4.92242016036926e-08
2546 4.92128684470572e-08
2547 4.92030807208721e-08
2548 4.91932468094092e-08
2549 4.9181767991513e-08
2550 4.9171127614045e-08
2551 4.91627396570493e-08
2552 4.91531721991123e-08
2553 4.91421410231396e-08
2554 4.91315823580862e-08
2555 4.91217129194865e-08
2556 4.91109872768902e-08
2557 4.91047167372471e-08
2558 4.90953127041394e-08
2559 4.90847789080817e-08
2560 4.90716693946069e-08
2561 4.90614588954941e-08
2562 4.90549751930303e-08
2563 4.90445799528061e-08
2564 4.90358473825836e-08
2565 4.90257185958853e-08
2566 4.90165135147436e-08
2567 4.90085731996714e-08
2568 4.89956768490174e-08
2569 4.89890368271517e-08
2570 4.89801195158179e-08
2571 4.89737352893371e-08
2572 4.89603451114817e-08
2573 4.89538329873085e-08
2574 4.89417004700954e-08
2575 4.89358491506664e-08
2576 4.89264415648449e-08
2577 4.89203912934499e-08
2578 4.89130087544254e-08
2579 4.8904119864801e-08
2580 4.88974940537901e-08
2581 4.8888502135469e-08
2582 4.88787890162712e-08
2583 4.88725895309017e-08
2584 4.88656475283733e-08
2585 4.88536713305621e-08
2586 4.88491629369037e-08
2587 4.88408495868953e-08
2588 4.8834326804581e-08
2589 4.88252815955548e-08
2590 4.88196150172371e-08
2591 4.88093228057096e-08
2592 4.88041216328838e-08
2593 4.87955098549264e-08
2594 4.87894240563946e-08
2595 4.87819526995281e-08
2596 4.87742504162725e-08
2597 4.87694826745155e-08
2598 4.87591549358513e-08
2599 4.875516523839e-08
2600 4.87484221878276e-08
2601 4.87388938097411e-08
2602 4.87308540186859e-08
2603 4.87216631483989e-08
2604 4.87177622687796e-08
2605 4.87101736723616e-08
2606 4.87022084882938e-08
2607 4.86965276991214e-08
2608 4.86870241900306e-08
2609 4.86834075275056e-08
2610 4.86781779329704e-08
2611 4.8668912455696e-08
2612 4.86644857744523e-08
2613 4.86546376521346e-08
2614 4.864975622354e-08
2615 4.86430131729776e-08
2616 4.8634547056281e-08
2617 4.86276263700347e-08
2618 4.86228763918461e-08
2619 4.86161617629932e-08
2620 4.86092872620247e-08
2621 4.86045372838362e-08
2622 4.85962701191056e-08
2623 4.85906710423478e-08
2624 4.8582528222596e-08
2625 4.85743392175664e-08
2626 4.85694293672623e-08
2627 4.85626934221273e-08
2628 4.85557514195989e-08
2629 4.85499818125845e-08
2630 4.85437574582193e-08
2631 4.85376077108413e-08
2632 4.85276743233953e-08
2633 4.852426016555e-08
2634 4.85151829821007e-08
2635 4.85082729539954e-08
2636 4.8502016625207e-08
2637 4.84956821367177e-08
2638 4.84889000063049e-08
2639 4.84821001123237e-08
2640 4.84763020835999e-08
2641 4.84694915314776e-08
2642 4.84594870897581e-08
2643 4.84569682157598e-08
2644 4.8447454048528e-08
2645 4.84431375014083e-08
2646 4.84352078444772e-08
2647 4.84315343385333e-08
2648 4.84229367714306e-08
2649 4.84150781687731e-08
2650 4.84076139173339e-08
2651 4.8401773256046e-08
2652 4.83940603146493e-08
2653 4.83884647906052e-08
2654 4.83838960008143e-08
2655 4.83741899870438e-08
2656 4.83693938235774e-08
2657 4.83627822234212e-08
2658 4.83537405671086e-08
2659 4.83480455670815e-08
2660 4.83391673355982e-08
2661 4.83346909163629e-08
2662 4.83267790230002e-08
2663 4.83209277035712e-08
2664 4.83136695095254e-08
2665 4.83090687453114e-08
2666 4.82990678563056e-08
2667 4.82955648806183e-08
2668 4.82848960814408e-08
2669 4.82795599054953e-08
2670 4.82711932647817e-08
2671 4.82667523726832e-08
2672 4.82584461281022e-08
2673 4.82523958567072e-08
2674 4.82472550800139e-08
2675 4.82383128996844e-08
2676 4.82300315240991e-08
2677 4.82253241784747e-08
2678 4.82180659844289e-08
2679 4.82098521104035e-08
2680 4.82043631677698e-08
2681 4.81988173817172e-08
2682 4.81913566829917e-08
2683 4.81836259780266e-08
2684 4.81778528182986e-08
2685 4.81718878120319e-08
2686 4.81643525063191e-08
2687 4.81595456847117e-08
2688 4.8150450737694e-08
2689 4.81453099610007e-08
2690 4.8138836916678e-08
2691 4.81323141343637e-08
2692 4.81237591998251e-08
2693 4.8119126461188e-08
2694 4.81109090344489e-08
2695 4.81050861367294e-08
2696 4.80984923001415e-08
2697 4.80925841372937e-08
2698 4.80844875028197e-08
2699 4.8078224068604e-08
2700 4.80700315108606e-08
2701 4.8063601099102e-08
2702 4.8059408896961e-08
2703 4.8050814882572e-08
2704 4.80478128395134e-08
2705 4.80404480640573e-08
2706 4.80323372187286e-08
2707 4.80254875867558e-08
2708 4.80187622997619e-08
2709 4.80160764482207e-08
2710 4.80068536035105e-08
2711 4.79998085722855e-08
2712 4.79916053564011e-08
2713 4.79876973713544e-08
2714 4.7981895789917e-08
2715 4.79765240868346e-08
2716 4.79714152845645e-08
2717 4.79639084005612e-08
2718 4.79585260393378e-08
2719 4.79525930074942e-08
2720 4.7946368653129e-08
2721 4.79388511109846e-08
2722 4.7933017555124e-08
2723 4.79273367659516e-08
2724 4.79212296511378e-08
2725 4.79154920185465e-08
2726 4.79085251470224e-08
2727 4.79026134314608e-08
2728 4.78973056772247e-08
2729 4.78918131818773e-08
2730 4.7885478693388e-08
2731 4.78805830539386e-08
2732 4.78756803090619e-08
2733 4.78700954431588e-08
2734 4.78643578105675e-08
2735 4.78585242547069e-08
2736 4.78540371773306e-08
2737 4.78466262165966e-08
2738 4.78405688397743e-08
2739 4.78384194479986e-08
2740 4.78317403462825e-08
2741 4.78287525140786e-08
2742 4.78210360199682e-08
2743 4.78147370586157e-08
2744 4.78110635526718e-08
2745 4.78050949936915e-08
2746 4.78009773985377e-08
2747 4.77931223485939e-08
2748 4.77889052774572e-08
2749 4.77826525013825e-08
2750 4.77765915718464e-08
2751 4.77778776541982e-08
2752 4.77706869617123e-08
2753 4.77673687271363e-08
2754 4.77610804239248e-08
2755 4.77552575262052e-08
2756 4.7750891241094e-08
2757 4.77496051587423e-08
2758 4.77420520894611e-08
2759 4.77382116059744e-08
2760 4.77328647718878e-08
2761 4.77283172983789e-08
2762 4.77266972609414e-08
2763 4.77210129190553e-08
2764 4.77186574698862e-08
2765 4.7714987516656e-08
2766 4.77104080687241e-08
2767 4.77041943724998e-08
2768 4.76982506825152e-08
2769 4.76971280249927e-08
2770 4.76953907480038e-08
2771 4.76910564373156e-08
2772 4.76883919020565e-08
2773 4.76848498465188e-08
2774 4.76826542694653e-08
2775 4.76769912438613e-08
2776 4.76765968926429e-08
2777 4.76739039356744e-08
2778 4.76712465058426e-08
2779 4.76704542506923e-08
2780 4.76717438857577e-08
2781 4.76682835426345e-08
2782 4.76713282182573e-08
2783 4.76749022482181e-08
2784 4.76773429625155e-08
2785 4.7677392700507e-08
2786 4.76787853642691e-08
2787 4.76839900898085e-08
2788 4.76883776912018e-08
2789 4.76864236986785e-08
2790 4.76832191509402e-08
2791 4.76826329531832e-08
2792 4.76816346406395e-08
2793 4.76765507073651e-08
2794 4.76762487267024e-08
2795 4.76699959506277e-08
2796 4.76694452800075e-08
2797 4.76647628033788e-08
2798 4.76626738077357e-08
2799 4.76596184739719e-08
2800 4.76553054795659e-08
2801 4.7650360102125e-08
2802 4.76426080808778e-08
2803 4.76412154171157e-08
2804 4.76387036485448e-08
2805 4.76310795249901e-08
2806 4.76248516179112e-08
2807 4.7625562160647e-08
2808 4.76159307538637e-08
2809 4.761112748497e-08
2810 4.76073793720388e-08
2811 4.76046757569293e-08
2812 4.76008352734425e-08
2813 4.75966714930109e-08
2814 4.75925041598657e-08
2815 4.7591527163604e-08
2816 4.75897081742005e-08
2817 4.75869406102447e-08
2818 4.75838319857758e-08
2819 4.75763997087597e-08
2820 4.75805421729092e-08
2821 4.75732164773035e-08
2822 4.75768509033969e-08
2823 4.75676849021056e-08
2824 4.75702144342449e-08
2825 4.75605759220343e-08
2826 4.75627075502416e-08
2827 4.75553179057897e-08
2828 4.75568420199579e-08
2829 4.75482870854194e-08
2830 4.75479744466156e-08
2831 4.75450470105443e-08
2832 4.7543789349902e-08
2833 4.75310422132225e-08
2834 4.75203378869082e-08
2835 4.74836880925977e-08
2836 4.73614250040555e-08
2837 4.74472798828174e-08
2838 4.74068535538663e-08
2839 4.74545451822905e-08
2840 4.74360675184471e-08
2841 4.74609045397756e-08
2842 4.74433612396297e-08
2843 4.7462403784948e-08
2844 4.7444917328221e-08
2845 4.7470013697648e-08
2846 4.74483314860663e-08
2847 4.74719996645945e-08
2848 4.74539945116703e-08
2849 4.74721630894237e-08
2850 4.74559307406253e-08
2851 4.74706709496786e-08
2852 4.74557140250909e-08
2853 4.74709125342088e-08
2854 4.7456531149237e-08
2855 4.7470678055106e-08
2856 4.74540726713713e-08
2857 4.74712322784399e-08
2858 4.74570249764383e-08
2859 4.74672781081154e-08
2860 4.74545558404316e-08
2861 4.74719819010261e-08
2862 4.74521719695531e-08
2863 4.7466098607174e-08
2864 4.7451354845407e-08
2865 4.74652743776005e-08
2866 4.74490349233747e-08
2867 4.74644537007407e-08
2868 4.74500190250637e-08
2869 4.74610892808869e-08
2870 4.74471697486933e-08
2871 4.746424053792e-08
2872 4.74468286881802e-08
2873 4.7459472796163e-08
2874 4.74457415577945e-08
2875 4.74614019196906e-08
2876 4.74426045116161e-08
2877 4.74586094867391e-08
2878 4.74402312988786e-08
2879 4.74548009776754e-08
2880 4.74380463799662e-08
2881 4.74568508934681e-08
2882 4.74394177274462e-08
2883 4.74575898579133e-08
2884 4.743136727825e-08
2885 4.74584282983415e-08
2886 4.74320955845542e-08
2887 4.745693971131e-08
2888 4.74298573749365e-08
2889 4.7457525909067e-08
2890 4.74275800854684e-08
2891 4.7454097540367e-08
2892 4.74282764173495e-08
2893 4.74550923001971e-08
2894 4.74254591154022e-08
2895 4.7454186358209e-08
2896 4.74220449575569e-08
2897 4.74551278273339e-08
2898 4.74232280112119e-08
2899 4.74487755752762e-08
2900 4.74208015077693e-08
2901 4.7448324380639e-08
2902 4.74191921284728e-08
2903 4.74492836133322e-08
2904 4.74192276556096e-08
2905 4.74499195490807e-08
2906 4.74168828645816e-08
2907 4.74488004442719e-08
2908 4.74151882201568e-08
2909 4.74468073718981e-08
2910 4.74142112238951e-08
2911 4.74400998484725e-08
2912 4.74151171658832e-08
2913 4.74409382889007e-08
2914 4.74157140217812e-08
2915 4.74341383949195e-08
2916 4.74151207185969e-08
2917 4.74306993680784e-08
2918 4.74177710430013e-08
2919 4.74230148483912e-08
2920 4.74213628365305e-08
2921 4.74135148920141e-08
2922 4.74294914454276e-08
2923 4.74009134165954e-08
2924 4.74383128334921e-08
2925 4.73817038937341e-08
2926 4.74530885696822e-08
2927 4.73626187158516e-08
2928 4.74708166109394e-08
2929 4.73353907182172e-08
2930 4.748875070959e-08
2931 4.73121382071895e-08
2932 4.75125254695286e-08
2933 4.72819436936334e-08
2934 4.75317492032445e-08
2935 4.72563890241418e-08
2936 4.75481165551628e-08
2937 4.72450771837885e-08
2938 4.75509835951016e-08
2939 4.7240359180023e-08
2940 4.75380623754518e-08
2941 4.72735877110608e-08
2942 4.74657824156566e-08
2943 4.73753303253943e-08
2944 4.73192578454018e-08
2945 4.7490985366494e-08
2946 4.72115502248016e-08
2947 4.74848285136886e-08
2948 4.71812882096856e-08
2949 4.76227128842766e-08
2950 4.70838266153351e-08
2951 4.76045123321001e-08
2952 4.70747423264584e-08
2953 4.7697124472279e-08
2954 4.70058516555127e-08
2955 4.76877630717354e-08
2956 4.69994105856131e-08
2957 4.7720916995786e-08
2958 4.69671590508369e-08
2959 4.77197232839899e-08
2960 4.6962988164978e-08
2961 4.77188244474291e-08
2962 4.69566749927708e-08
2963 4.77117794162041e-08
2964 4.69593288698889e-08
2965 4.77024642009383e-08
2966 4.69610199616e-08
2967 4.76897348278271e-08
2968 4.6966007971605e-08
2969 4.76717900710355e-08
2970 4.69725094376372e-08
2971 4.76624144596371e-08
2972 4.6980929369056e-08
2973 4.7644519440837e-08
2974 4.69930085955639e-08
2975 4.76298467333436e-08
2976 4.7001723402218e-08
2977 4.76160124662783e-08
2978 4.70151597653512e-08
2979 4.76024872853031e-08
2980 4.70287133680358e-08
2981 4.75820520762227e-08
2982 4.7041741169096e-08
2983 4.75672301547547e-08
2984 4.70554901710329e-08
2985 4.75521559906156e-08
2986 4.70698431342953e-08
2987 4.75358419294025e-08
2988 4.70890277881608e-08
2989 4.75161066049168e-08
2990 4.71051535555489e-08
2991 4.74948755879723e-08
2992 4.71256385026209e-08
2993 4.74662904537126e-08
2994 4.71565506643401e-08
2995 4.74416204099271e-08
2996 4.71852352745827e-08
2997 4.74045869225392e-08
2998 4.72173695698075e-08
2999 4.73652974619654e-08
3000 2.70380038358553e-08
3001 2.72348028573788e-08
3002 2.73558882213365e-08
3003 2.74217928364351e-08
3004 2.74469051930737e-08
3005 2.74535594257941e-08
3006 2.74532450106335e-08
3007 2.74510707498621e-08
3008 2.74480971285129e-08
3009 2.74453899606897e-08
3010 2.74427147672895e-08
3011 2.74401337208019e-08
3012 2.74382223608427e-08
3013 2.74361262597722e-08
3014 2.74338880501546e-08
3015 2.74322982107833e-08
3016 2.74304472469566e-08
3017 2.74287206281087e-08
3018 2.74270455236092e-08
3019 2.74253864063212e-08
3020 2.7423828541373e-08
3021 2.74221712004419e-08
3022 2.74207643258251e-08
3023 2.74193503457809e-08
3024 2.74179647874462e-08
3025 2.74167426539407e-08
3026 2.74151439327852e-08
3027 2.74139875244828e-08
3028 2.74124136723231e-08
3029 2.74112075260291e-08
3030 2.74097509134208e-08
3031 2.74090066199051e-08
3032 2.74076423778524e-08
3033 2.74064273497743e-08
3034 2.74052940341107e-08
3035 2.74040683478916e-08
3036 2.74027165403368e-08
3037 2.74016151990963e-08
3038 2.7400446356296e-08
3039 2.73993538968398e-08
3040 2.73982525555994e-08
3041 2.73969682496045e-08
3042 2.73960161223386e-08
3043 2.7394738921771e-08
3044 2.73937601491525e-08
3045 2.73925309102196e-08
3046 2.73915485848875e-08
3047 2.73906746173225e-08
3048 2.73894027458255e-08
3049 2.7388479040269e-08
3050 2.73873403955349e-08
3051 2.73864912969657e-08
3052 2.73854627863557e-08
3053 2.7384386314111e-08
3054 2.7383357803501e-08
3055 2.73822156060533e-08
3056 2.73812243989369e-08
3057 2.73801958883269e-08
3058 2.73791940230694e-08
3059 2.7378192157812e-08
3060 2.73770126568706e-08
3061 2.73763038904917e-08
3062 2.73751741275419e-08
3063 2.73743694378936e-08
3064 2.73734261924119e-08
3065 2.73723177457441e-08
3066 2.73713922638308e-08
3067 2.73703495423661e-08
3068 2.73695501817883e-08
3069 2.73684559459753e-08
3070 2.73671965089761e-08
3071 2.73661999727892e-08
3072 2.73654698901282e-08
3073 2.73643454562489e-08
3074 2.7363293853e-08
3075 2.73623719238003e-08
3076 2.73615103907332e-08
3077 2.73607145828692e-08
3078 2.73595954780603e-08
3079 2.73587534849185e-08
3080 2.73578599774282e-08
3081 2.73569575881538e-08
3082 2.735602677717e-08
3083 2.73550053719873e-08
3084 2.73541331807792e-08
3085 2.73530726957461e-08
3086 2.73521116866959e-08
3087 2.73513691695371e-08
3088 2.73503459879976e-08
3089 2.73494169533706e-08
3090 2.73484523916068e-08
3091 2.73476725709543e-08
3092 2.7346525044436e-08
3093 2.73455462718175e-08
3094 2.73445266429917e-08
3095 2.73434306308218e-08
3096 2.73425833086094e-08
3097 2.73416613794097e-08
3098 2.73406968176459e-08
3099 2.73400164729765e-08
3100 2.73390980964905e-08
3101 2.73379825443953e-08
3102 2.73371210113282e-08
3103 2.73362577019043e-08
3104 2.7335332219991e-08
3105 2.7334369434584e-08
3106 2.73334528344549e-08
3107 2.73327493971465e-08
3108 2.73319411547845e-08
3109 2.73311471232773e-08
3110 2.73300635456053e-08
3111 2.73293121466622e-08
3112 2.73284967988729e-08
3113 2.73275464479639e-08
3114 2.73268074835187e-08
3115 2.73257345639877e-08
3116 2.73248552673522e-08
3117 2.73241216319775e-08
3118 2.7322943907393e-08
3119 2.73221374413879e-08
3120 2.73209224133097e-08
3121 2.7320167461653e-08
3122 2.73191798072503e-08
3123 2.73181903764907e-08
3124 2.73173856868425e-08
3125 2.73164015851535e-08
3126 2.7315323336552e-08
3127 2.7314227324382e-08
3128 2.73134403983022e-08
3129 2.73126072869445e-08
3130 2.73116587123923e-08
3131 2.73108771153829e-08
3132 2.73098450520592e-08
3133 2.73089248992164e-08
3134 2.73083315960321e-08
3135 2.73072497947169e-08
3136 2.73064539868528e-08
3137 2.73055693611468e-08
3138 2.73048232912743e-08
3139 2.73038072151621e-08
3140 2.73029261421698e-08
3141 2.73020202001817e-08
3142 2.73010716256294e-08
3143 2.73001887762803e-08
3144 2.72994604699761e-08
3145 2.72981637294833e-08
3146 2.7297193838649e-08
3147 2.72964921776975e-08
3148 2.72957478841818e-08
3149 2.72948330604095e-08
3150 2.72939040257825e-08
3151 2.72931011124911e-08
3152 2.7292250237565e-08
3153 2.729137627e-08
3154 2.72904703280119e-08
3155 2.72895572805965e-08
3156 2.72887650254461e-08
3157 2.72879265850179e-08
3158 2.72869158379763e-08
3159 2.72859530525693e-08
3160 2.7285192771842e-08
3161 2.72843845294801e-08
3162 2.72833879932932e-08
3163 2.72825619873629e-08
3164 2.72817182178642e-08
3165 2.72807660905983e-08
3166 2.72796878419967e-08
3167 2.727893289034e-08
3168 2.727821346582e-08
3169 2.7277229364131e-08
3170 2.72764353326238e-08
3171 2.72755276142789e-08
3172 2.7274630554075e-08
3173 2.7273822311713e-08
3174 2.72727564976094e-08
3175 2.72719375971064e-08
3176 2.72710494186867e-08
3177 2.72706337511863e-08
3178 2.72694560266018e-08
3179 2.72685376501158e-08
3180 2.72676796697624e-08
3181 2.72666991207871e-08
3182 2.72659157474209e-08
3183 2.72649902655075e-08
3184 2.72639848475364e-08
3185 2.72630114039885e-08
3186 2.72625229058576e-08
3187 2.72614819607497e-08
3188 2.72606754947446e-08
3189 2.72598850159511e-08
3190 2.72590821026597e-08
3191 2.72581726079579e-08
3192 2.72571263337795e-08
3193 2.72565099379563e-08
3194 2.72556519576028e-08
3195 2.72547406865442e-08
3196 2.72537903356351e-08
3197 2.7253006962269e-08
3198 2.72519820043726e-08
3199 2.72511346821602e-08
3200 2.7250484535557e-08
3201 2.72496976094772e-08
3202 2.72486904151492e-08
3203 2.72477453933107e-08
3204 2.72468874129572e-08
3205 2.72460738415248e-08
3206 2.72453917204984e-08
3207 2.72444715676556e-08
3208 2.72432831849301e-08
3209 2.72424287572903e-08
3210 2.72415636715095e-08
3211 2.72406879275877e-08
3212 2.72397198131102e-08
3213 2.72387961075538e-08
3214 2.72381264210253e-08
3215 2.72373714693686e-08
3216 2.72365969777866e-08
3217 2.72355649144629e-08
3218 2.7234479560434e-08
3219 2.72334190754009e-08
3220 2.72328293249302e-08
3221 2.72319962135725e-08
3222 2.723090730683e-08
3223 2.7230088406327e-08
3224 2.72293370073839e-08
3225 2.72284594871053e-08
3226 2.7227489596271e-08
3227 2.72267506318258e-08
3228 2.72259139677544e-08
3229 2.72252034250187e-08
3230 2.72242566268233e-08
3231 2.72235425313738e-08
3232 2.72225744168964e-08
3233 2.72216755803356e-08
3234 2.72208833251852e-08
3235 2.7220133702599e-08
3236 2.72192099970425e-08
3237 2.72183573457596e-08
3238 2.72177427262932e-08
3239 2.72169646819975e-08
3240 2.72158402481182e-08
3241 2.721496805691e-08
3242 2.72143569901573e-08
3243 2.72134297318871e-08
3244 2.72124864864054e-08
3245 2.72116444932635e-08
3246 2.721082026369e-08
3247 2.72098326092873e-08
3248 2.72090865394148e-08
3249 2.72085678432177e-08
3250 2.72075517671055e-08
3251 2.72065641127028e-08
3252 2.72059015316017e-08
3253 2.72047113725193e-08
3254 2.72040256987793e-08
3255 2.7203233443629e-08
3256 2.72023097380725e-08
3257 2.72014979429969e-08
3258 2.72004143653248e-08
3259 2.72000963974506e-08
3260 2.71991922318193e-08
3261 2.71982543154081e-08
3262 2.71973465970632e-08
3263 2.71966928977463e-08
3264 2.71955862274353e-08
3265 2.71947708796461e-08
3266 2.71938009888117e-08
3267 2.71929998518772e-08
3268 2.71923177308508e-08
3269 2.71913744853691e-08
3270 2.71907651949732e-08
3271 2.71896638537328e-08
3272 2.71888218605909e-08
3273 2.71882107938382e-08
3274 2.71871005708135e-08
3275 2.71864042389325e-08
3276 2.71855675748611e-08
3277 2.71848570321254e-08
3278 2.71839919463446e-08
3279 2.71830060682987e-08
3280 2.71821853914389e-08
3281 2.71813487273675e-08
3282 2.71804605489478e-08
3283 2.71797091500048e-08
3284 2.71788760386471e-08
3285 2.71780891125672e-08
3286 2.7177174288795e-08
3287 2.71765046022665e-08
3288 2.7175481420727e-08
3289 2.71748490376922e-08
3290 2.71739022394968e-08
3291 2.71731153134169e-08
3292 2.71724367451043e-08
3293 2.71715023814068e-08
3294 2.71705093979335e-08
3295 2.71697366827084e-08
3296 2.7169177130304e-08
3297 2.71680011820763e-08
3298 2.71672000451417e-08
3299 2.71664362117008e-08
3300 2.71655498096379e-08
3301 2.71647255800644e-08
3302 2.71638960214204e-08
3303 2.71631606096889e-08
3304 2.71620681502327e-08
3305 2.71615174796125e-08
3306 2.71605795632013e-08
3307 2.7159664739429e-08
3308 2.71587747846525e-08
3309 2.71581104271945e-08
3310 2.71571192200781e-08
3311 2.71563447284962e-08
3312 2.71556910291793e-08
3313 2.71548561414647e-08
3314 2.71540727680986e-08
3315 2.71532503148819e-08
3316 2.71522022643467e-08
3317 2.71512785587902e-08
3318 2.71504738691419e-08
3319 2.71496123360748e-08
3320 2.71486655378794e-08
3321 2.71481805924623e-08
3322 2.71471272128565e-08
3323 2.71464326573323e-08
3324 2.71455391498421e-08
3325 2.71447149202686e-08
3326 2.71439333232593e-08
3327 2.71430362630554e-08
3328 2.71422955222533e-08
3329 2.71414464236841e-08
3330 2.71405884433307e-08
3331 2.71396274342806e-08
3332 2.71390909745151e-08
3333 2.71382027960954e-08
3334 2.71372098126221e-08
3335 2.71364459791812e-08
3336 2.71354654302058e-08
3337 2.71347744273953e-08
3338 2.71340816482279e-08
3339 2.71333373547122e-08
3340 2.71326161538354e-08
3341 2.71314792854582e-08
3342 2.71307314392288e-08
3343 2.71299320786511e-08
3344 2.71291451525713e-08
3345 2.71282836195041e-08
3346 2.71273510321635e-08
3347 2.71266404894277e-08
3348 2.71257576400785e-08
3349 2.71250790717659e-08
3350 2.71240612192969e-08
3351 2.7123263635076e-08
3352 2.71222937442417e-08
3353 2.7121306089839e-08
3354 2.71205511381822e-08
3355 2.71199400714295e-08
3356 2.71190536693666e-08
3357 2.71181885835858e-08
3358 2.7117323497805e-08
3359 2.71165685461483e-08
3360 2.71158899778356e-08
3361 2.71147548858153e-08
3362 2.711404967215e-08
3363 2.71131632700872e-08
3364 2.7112479372704e-08
3365 2.71118878458765e-08
3366 2.71109321658969e-08
3367 2.71101079363234e-08
3368 2.71093636428077e-08
3369 2.71085838221552e-08
3370 2.71075748514704e-08
3371 2.71067488455401e-08
3372 2.71060116574517e-08
3373 2.71051927569488e-08
3374 2.71043596455911e-08
3375 2.71034732435282e-08
3376 2.71024713782708e-08
3377 2.71015885289216e-08
3378 2.7100899302468e-08
3379 2.71003504082046e-08
3380 2.70993982809387e-08
3381 2.70984426009591e-08
3382 2.70978386396337e-08
3383 2.70966911131154e-08
3384 2.70959858994502e-08
3385 2.70952700276439e-08
3386 2.70944333635725e-08
3387 2.70935807122896e-08
3388 2.70929998436031e-08
3389 2.70921489686771e-08
3390 2.7091397569734e-08
3391 2.70905271548827e-08
3392 2.70899072063457e-08
3393 2.70889017883746e-08
3394 2.70881326258632e-08
3395 2.70874220831274e-08
3396 2.70863385054554e-08
3397 2.70858837581045e-08
3398 2.70848143912872e-08
3399 2.70840079252821e-08
3400 2.70832085647044e-08
3401 2.70826028270221e-08
3402 2.70816915559635e-08
3403 2.70809366043068e-08
3404 2.70800679658123e-08
3405 2.70791602474674e-08
3406 2.70784425993043e-08
3407 2.70777427147095e-08
3408 2.70767390730953e-08
3409 2.70760356357869e-08
3410 2.70751279174419e-08
3411 2.70743871766399e-08
3412 2.70735682761369e-08
3413 2.70726356887963e-08
3414 2.70720157402593e-08
3415 2.70711186800554e-08
3416 2.70705804439331e-08
3417 2.70694986426179e-08
3418 2.70685784897751e-08
3419 2.70680136083001e-08
3420 2.70672231295066e-08
3421 2.70664166635015e-08
3422 2.70656279610648e-08
3423 2.70648143896324e-08
3424 2.70638764732212e-08
3425 2.70630859944276e-08
3426 2.70622937392773e-08
3427 2.70615050368406e-08
3428 2.70605955421388e-08
3429 2.70600164498092e-08
3430 2.70592241946588e-08
3431 2.70583377925959e-08
3432 2.70576148153623e-08
3433 2.70567621640794e-08
3434 2.70561493209698e-08
3435 2.70550817305093e-08
3436 2.70543960567693e-08
3437 2.70536499868967e-08
3438 2.70529696422273e-08
3439 2.70520477130276e-08
3440 2.70513513811466e-08
3441 2.70505324806436e-08
3442 2.70496300913692e-08
3443 2.70490563281101e-08
3444 2.70479461050854e-08
3445 2.70472675367728e-08
3446 2.70465356777549e-08
3447 2.70457185536088e-08
3448 2.7044997352732e-08
3449 2.70441873340133e-08
3450 2.70433133664483e-08
3451 2.70425619675052e-08
3452 2.70417999104211e-08
3453 2.70410076552707e-08
3454 2.704007506793e-08
3455 2.70391193879504e-08
3456 2.70380411393489e-08
3457 2.7037486916015e-08
3458 2.70367301880015e-08
3459 2.70357247700304e-08
3460 2.70351918629785e-08
3461 2.70341224961612e-08
3462 2.70333657681476e-08
3463 2.70324473916617e-08
3464 2.70318007977721e-08
3465 2.70310849259658e-08
3466 2.70303033289565e-08
3467 2.70294897575241e-08
3468 2.70288094128546e-08
3469 2.70279958414221e-08
3470 2.70270312796583e-08
3471 2.70265072543907e-08
3472 2.70255977596889e-08
3473 2.70247326739081e-08
3474 2.70239084443347e-08
3475 2.70231250709685e-08
3476 2.70222866305403e-08
3477 2.70214126629753e-08
3478 2.70209667974086e-08
3479 2.70201940821835e-08
3480 2.7019215309565e-08
3481 2.70184639106219e-08
3482 2.70177498151725e-08
3483 2.70169948635157e-08
3484 2.70162150428632e-08
3485 2.70154334458539e-08
3486 2.7014735337616e-08
3487 2.70137299196449e-08
3488 2.70130513513323e-08
3489 2.70123141632439e-08
3490 2.70115947387239e-08
3491 2.70107527455821e-08
3492 2.70098645671624e-08
3493 2.70087774367767e-08
3494 2.70082178843722e-08
3495 2.7007468261786e-08
3496 2.7006683112063e-08
3497 2.70060098728209e-08
3498 2.70052638029483e-08
3499 2.70044360206612e-08
3500 2.70035993565898e-08
3501 2.70027502580206e-08
3502 2.70020787951353e-08
3503 2.70012208147818e-08
3504 2.70003255309348e-08
3505 2.69998974289365e-08
3506 2.69989861578779e-08
3507 2.69983093659221e-08
3508 2.69973448041583e-08
3509 2.69967763699697e-08
3510 2.69958295717743e-08
3511 2.6995268243013e-08
3512 2.69942077579799e-08
3513 2.69935842567293e-08
3514 2.69926641038865e-08
3515 2.69918540851677e-08
3516 2.69909303796112e-08
3517 2.69904436578372e-08
3518 2.69894933069281e-08
3519 2.69888413839681e-08
3520 2.69878501768517e-08
3521 2.69869921964982e-08
3522 2.69862052704184e-08
3523 2.69856954560055e-08
3524 2.69849120826393e-08
3525 2.69841340383437e-08
3526 2.69831481602978e-08
3527 2.69823949849979e-08
3528 2.69816169407022e-08
3529 2.69805386921007e-08
3530 2.69800839447498e-08
3531 2.69791957663301e-08
3532 2.6978478118167e-08
3533 2.69775668471084e-08
3534 2.69769060423641e-08
3535 2.69759876658782e-08
3536 2.69752096215825e-08
3537 2.69745203951288e-08
3538 2.69734439228841e-08
3539 2.69727564727873e-08
3540 2.69718896106497e-08
3541 2.69710636047193e-08
3542 2.69703601674109e-08
3543 2.69695501486922e-08
3544 2.69688058551765e-08
3545 2.69678803732631e-08
3546 2.69671751595979e-08
3547 2.69663704699497e-08
3548 2.69655533458035e-08
3549 2.69648801065614e-08
3550 2.69640487715606e-08
3551 2.69633702032479e-08
3552 2.69626738713669e-08
3553 2.69616808878936e-08
3554 2.69610307412904e-08
3555 2.69602118407875e-08
3556 2.69594924162675e-08
3557 2.69586823975487e-08
3558 2.69577906664153e-08
3559 2.69570392674723e-08
3560 2.69562274723967e-08
3561 2.69552309362098e-08
3562 2.69546855946601e-08
3563 2.69536961639005e-08
3564 2.69530975316457e-08
3565 2.69522324458649e-08
3566 2.69512980821673e-08
3567 2.69508113603933e-08
3568 2.69498929839074e-08
3569 2.69490048054877e-08
3570 2.69483440007434e-08
3571 2.69473474645565e-08
3572 2.69465179059125e-08
3573 2.69458766410935e-08
3574 2.69452140599924e-08
3575 2.6944356079639e-08
3576 2.69434234922983e-08
3577 2.69425139975965e-08
3578 2.69419722087605e-08
3579 2.69410573849882e-08
3580 2.69404889507996e-08
3581 2.69395243890358e-08
3582 2.69387712137359e-08
3583 2.69382436357546e-08
3584 2.69373803263306e-08
3585 2.69365969529645e-08
3586 2.69356235094165e-08
3587 2.69349680337427e-08
3588 2.69341562386671e-08
3589 2.69332325331106e-08
3590 2.6932509555877e-08
3591 2.69319624379705e-08
3592 2.6930994323493e-08
3593 2.69303477296035e-08
3594 2.69294737620385e-08
3595 2.69288857879246e-08
3596 2.69283084719518e-08
3597 2.69272106834251e-08
3598 2.69266156038839e-08
3599 2.69257451890326e-08
3600 2.69247610873435e-08
3601 2.69240665318193e-08
3602 2.69234270433572e-08
3603 2.69224713633776e-08
3604 2.69217039772229e-08
3605 2.69209259329273e-08
3606 2.69202917735356e-08
3607 2.69193716206928e-08
3608 2.69186433143886e-08
3609 2.6917875928234e-08
3610 2.69171387401457e-08
3611 2.69162825361491e-08
3612 2.6915486728285e-08
3613 2.69146784859231e-08
3614 2.69141011699503e-08
3615 2.69131650298959e-08
3616 2.69126037011347e-08
3617 2.69115858486657e-08
3618 2.69109285966351e-08
3619 2.69100546290701e-08
3620 2.69094648785995e-08
3621 2.69085749238229e-08
3622 2.69076210202002e-08
3623 2.69071005476462e-08
3624 2.69063367142053e-08
3625 2.69055497881254e-08
3626 2.69049653667253e-08
3627 2.69041748879317e-08
3628 2.69032920385825e-08
3629 2.69022848442546e-08
3630 2.69016897647134e-08
3631 2.69010165254713e-08
3632 2.69001425579063e-08
3633 2.68993574081833e-08
3634 2.6898453242552e-08
3635 2.68977107253932e-08
3636 2.68967923489072e-08
3637 2.68962487837143e-08
3638 2.68954476467798e-08
3639 2.68946109827084e-08
3640 2.68937334624297e-08
3641 2.68928666002921e-08
3642 2.68921258594901e-08
3643 2.68915343326626e-08
3644 2.68906479305997e-08
3645 2.68898325828104e-08
3646 2.68892765831197e-08
3647 2.68883528775632e-08
3648 2.68875890441223e-08
3649 2.68868589614613e-08
3650 2.68859672303279e-08
3651 2.68851767515343e-08
3652 2.6884405812666e-08
3653 2.68836686245777e-08
3654 2.68827999860832e-08
3655 2.6882039705356e-08
3656 2.68811835013594e-08
3657 2.68806363834528e-08
3658 2.68796913616143e-08
3659 2.68790909530026e-08
3660 2.68780837586746e-08
3661 2.68775828260459e-08
3662 2.68766768840578e-08
3663 2.68760391719525e-08
3664 2.68751421117486e-08
3665 2.6874396041876e-08
3666 2.68737778696959e-08
3667 2.68727458063722e-08
3668 2.6872006841927e-08
3669 2.6871465053091e-08
3670 2.68705147021819e-08
3671 2.68696531691148e-08
3672 2.68689586135906e-08
3673 2.6868102409594e-08
3674 2.68673687742194e-08
3675 2.68665818481395e-08
3676 2.68656901170061e-08
3677 2.68650293122619e-08
3678 2.68642743606051e-08
3679 2.68637379008396e-08
3680 2.68627946553579e-08
3681 2.68618016718847e-08
3682 2.68610289566595e-08
3683 2.68600999220325e-08
3684 2.68594728680682e-08
3685 2.68588156160376e-08
3686 2.68580233608873e-08
3687 2.68572453165916e-08
3688 2.6856389112595e-08
3689 2.68554849469638e-08
3690 2.6854928947273e-08
3691 2.6854332091375e-08
3692 2.6853403056748e-08
3693 2.68525539581788e-08
3694 2.68519055879324e-08
3695 2.68512057033377e-08
3696 2.68506550327174e-08
3697 2.68496549438169e-08
3698 2.684896749372e-08
3699 2.68482303056317e-08
3700 2.68476476605883e-08
3701 2.68468127728738e-08
3702 2.68461715080548e-08
3703 2.68453295149129e-08
3704 2.68443915985017e-08
3705 2.68435940142808e-08
3706 2.68429385386071e-08
3707 2.68420574656147e-08
3708 2.68414090953684e-08
3709 2.68406186165748e-08
3710 2.68400821568093e-08
3711 2.68388422597354e-08
3712 2.68383004708994e-08
3713 2.68373909761976e-08
3714 2.68367763567312e-08
3715 2.68359219290915e-08
3716 2.68351740828621e-08
3717 2.68344191312053e-08
3718 2.68337139175401e-08
3719 2.6833006927518e-08
3720 2.6832166710733e-08
3721 2.68313407048026e-08
3722 2.68304667372377e-08
3723 2.68300297534552e-08
3724 2.68291113769692e-08
3725 2.68282605020431e-08
3726 2.68277329240618e-08
3727 2.68269317871273e-08
3728 2.68261217684085e-08
3729 2.68252833279803e-08
3730 2.68245248236099e-08
3731 2.68237680955963e-08
3732 2.68231659106277e-08
3733 2.68225015531698e-08
3734 2.68218052212887e-08
3735 2.68210005316405e-08
3736 2.68203486086804e-08
3737 2.68193982577714e-08
3738 2.6818716136745e-08
3739 2.68180002649387e-08
3740 2.68170872175233e-08
3741 2.68162487770951e-08
3742 2.68158473204494e-08
3743 2.68149413784613e-08
3744 2.68140443182574e-08
3745 2.6813429698791e-08
3746 2.6812744025051e-08
3747 2.68120512458836e-08
3748 2.68112732015879e-08
3749 2.6810111464215e-08
3750 2.68095980970884e-08
3751 2.68087188004529e-08
3752 2.6808109510057e-08
3753 2.68073723219686e-08
3754 2.68068092168505e-08
3755 2.68059103802898e-08
3756 2.68051199014963e-08
3757 2.68044644258225e-08
3758 2.68034092698599e-08
3759 2.68028674810239e-08
3760 2.68020983185124e-08
3761 2.68012900761505e-08
3762 2.68005866388421e-08
3763 2.67997979364054e-08
3764 2.67989737068319e-08
3765 2.67986326463188e-08
3766 2.67975917012109e-08
3767 2.67967763534216e-08
3768 2.67961208777479e-08
3769 2.67952930954607e-08
3770 2.67946180798617e-08
3771 2.67935948983222e-08
3772 2.67929927133537e-08
3773 2.67922466434811e-08
3774 2.6791296292572e-08
3775 2.67906674622509e-08
3776 2.67899658012993e-08
3777 2.67891344662985e-08
3778 2.67883191185092e-08
3779 2.6787823514951e-08
3780 2.67867861225568e-08
3781 2.67861697267335e-08
3782 2.67853934587947e-08
3783 2.67843631718279e-08
3784 2.67838213829918e-08
3785 2.67830575495509e-08
3786 2.67821924637701e-08
3787 2.67815281063122e-08
3788 2.67806576914609e-08
3789 2.67800164266419e-08
3790 2.67791602226453e-08
3791 2.67784194818432e-08
3792 2.67774939999299e-08
3793 2.67766750994269e-08
3794 2.67758508698535e-08
3795 2.67751794069682e-08
3796 2.67746411708458e-08
3797 2.67737352288577e-08
3798 2.67731437020302e-08
3799 2.67721045332792e-08
3800 2.67715130064516e-08
3801 2.67706639078824e-08
3802 2.67699569178603e-08
3803 2.67689568289597e-08
3804 2.67684683308289e-08
3805 2.67677631171637e-08
3806 2.67666990794169e-08
3807 2.67659334696191e-08
3808 2.67652424668086e-08
3809 2.67646171892011e-08
3810 2.67637947359844e-08
3811 2.6763169458377e-08
3812 2.67623097016667e-08
3813 2.67615352100847e-08
3814 2.67607127568681e-08
3815 2.67600288594849e-08
3816 2.67593254221765e-08
3817 2.67585225088851e-08
3818 2.67578403878588e-08
3819 2.67570108292148e-08
3820 2.67559006061902e-08
3821 2.67555382293949e-08
3822 2.67549093990738e-08
3823 2.67540674059319e-08
3824 2.67532715980678e-08
3825 2.67526782948835e-08
3826 2.6751987292073e-08
3827 2.67512625384825e-08
3828 2.67504400852658e-08
3829 2.67495803285556e-08
3830 2.67489745908733e-08
3831 2.67482587190671e-08
3832 2.6747253301096e-08
3833 2.67466653269821e-08
3834 2.67460116276652e-08
3835 2.67451056856771e-08
3836 2.6744498171638e-08
3837 2.67437165746287e-08
3838 2.67429935973951e-08
3839 2.6742441150418e-08
3840 2.67413220456092e-08
3841 2.67407038734291e-08
3842 2.67398938547103e-08
3843 2.67390554142821e-08
3844 2.67382862517707e-08
3845 2.67376218943127e-08
3846 2.67369717477095e-08
3847 2.67362239014801e-08
3848 2.67354867133918e-08
3849 2.67348276850043e-08
3850 2.67339341775141e-08
3851 2.67332982417656e-08
3852 2.67323745362091e-08
3853 2.67317954438795e-08
3854 2.67306994317096e-08
3855 2.67301896172967e-08
3856 2.67293795985779e-08
3857 2.67286726085558e-08
3858 2.67279478549654e-08
3859 2.67270703346867e-08
3860 2.67264166353698e-08
3861 2.6725695434493e-08
3862 2.67249991026119e-08
3863 2.67240292117776e-08
3864 2.67232529438388e-08
3865 2.67226365480155e-08
3866 2.67218638327904e-08
3867 2.67212296733987e-08
3868 2.67204480763894e-08
3869 2.67195296999034e-08
3870 2.67187196811847e-08
3871 2.67180322310878e-08
3872 2.67170925383198e-08
3873 2.67165027878491e-08
3874 2.67156679001346e-08
3875 2.67150035426766e-08
3876 2.671414733868e-08
3877 2.67132111986257e-08
3878 2.67127830966274e-08
3879 2.67120956465305e-08
3880 2.6711294509596e-08
3881 2.67106923246274e-08
3882 2.67098609896266e-08
3883 2.67092055139528e-08
3884 2.67082889138237e-08
3885 2.67077311377761e-08
3886 2.67069513171236e-08
3887 2.67064148573581e-08
3888 2.67056687874856e-08
3889 2.67048303470574e-08
3890 2.67040309864797e-08
3891 2.6703286692964e-08
3892 2.67024411471084e-08
3893 2.67017004063064e-08
3894 2.67009170329402e-08
3895 2.67003894549589e-08
3896 2.66994995001824e-08
3897 2.66987445485256e-08
3898 2.6698065980213e-08
3899 2.66970658913124e-08
3900 2.66961954764611e-08
3901 2.66956945438324e-08
3902 2.66947637328485e-08
3903 2.66940745063948e-08
3904 2.6693552257484e-08
3905 2.66927635550473e-08
3906 2.66920405778137e-08
3907 2.66911488466803e-08
3908 2.66904880419361e-08
3909 2.66898378953329e-08
3910 2.66890367583983e-08
3911 2.66883635191562e-08
3912 2.66875375132258e-08
3913 2.66867026255113e-08
3914 2.66860737951902e-08
3915 2.66851891694841e-08
3916 2.6684261911214e-08
3917 2.66836437390339e-08
3918 2.66826543082743e-08
3919 2.66819935035301e-08
3920 2.66813984239889e-08
3921 2.66802775428232e-08
3922 2.66797179904188e-08
3923 2.66788049430033e-08
3924 2.6678277365022e-08
3925 2.66775419532905e-08
3926 2.66766999601487e-08
3927 2.66759183631393e-08
3928 2.66749804467281e-08
3929 2.66745203703067e-08
3930 2.66736641663101e-08
3931 2.66729411890765e-08
3932 2.66720672215115e-08
3933 2.6671376218701e-08
3934 2.66704702767129e-08
3935 2.66694719641691e-08
3936 2.66689710315404e-08
3937 2.66682338434521e-08
3938 2.66673882975965e-08
3939 2.66669228921046e-08
3940 2.66660311609712e-08
3941 2.66652619984598e-08
3942 2.66645976410018e-08
3943 2.66636153156696e-08
3944 2.66631268175388e-08
3945 2.66625175271429e-08
3946 2.66614357258277e-08
3947 2.66608513044275e-08
3948 2.66601745124717e-08
3949 2.66594959441591e-08
3950 2.66587480979297e-08
3951 2.66578901175762e-08
3952 2.6657229312832e-08
3953 2.66563873196901e-08
3954 2.66558313199994e-08
3955 2.66550621574879e-08
3956 2.66541420046451e-08
3957 2.66537423243562e-08
3958 2.66528772385755e-08
3959 2.66521578140555e-08
3960 2.66513993096851e-08
3961 2.66505200130496e-08
3962 2.66496407164141e-08
3963 2.66491824163495e-08
3964 2.66481752220216e-08
3965 2.66471822385483e-08
3966 2.66468465071057e-08
3967 2.66459529996155e-08
3968 2.6645183837104e-08
3969 2.6644542572285e-08
3970 2.66439599272417e-08
3971 2.66429864836937e-08
3972 2.66424216022187e-08
3973 2.66418158645365e-08
3974 2.66409241334031e-08
3975 2.66402189197379e-08
3976 2.66396185111262e-08
3977 2.66387534253454e-08
3978 2.66380393298959e-08
3979 2.66371138479826e-08
3980 2.66364423850973e-08
3981 2.66358082257057e-08
3982 2.6634797478664e-08
3983 2.66339821308748e-08
3984 2.66334652110345e-08
3985 2.66326356523905e-08
3986 2.66318984643021e-08
3987 2.66311293017907e-08
3988 2.66305146823242e-08
3989 2.6629964011704e-08
3990 2.66291397821306e-08
3991 2.66282285110719e-08
3992 2.66273953997143e-08
3993 2.66267417003974e-08
3994 2.66256812153642e-08
3995 2.66251625191671e-08
3996 2.66244519764314e-08
3997 2.66235868906506e-08
3998 2.6622885229699e-08
3999 2.66222635048052e-08
4000 2.66215387512148e-08
4001 2.66207642596328e-08
4002 2.6620053716897e-08
4003 2.66192952125266e-08
4004 2.66184621011689e-08
4005 2.66178101782089e-08
4006 2.66167319296073e-08
4007 2.66161368500661e-08
4008 2.66153481476294e-08
4009 2.66146571448189e-08
4010 2.66137423210466e-08
4011 2.66129358550415e-08
4012 2.66124065007034e-08
4013 2.66115893765573e-08
4014 2.6610818437689e-08
4015 2.66102002655089e-08
4016 2.66094488665658e-08
4017 2.66087578637553e-08
4018 2.66079744903891e-08
4019 2.66072817112217e-08
4020 2.66066404464027e-08
4021 2.66056243702906e-08
4022 2.660496711826e-08
4023 2.66042743390926e-08
4024 2.66036224161326e-08
4025 2.66027448958539e-08
4026 2.66022563977231e-08
4027 2.66012847305319e-08
4028 2.66007464944096e-08
4029 2.65996398240986e-08
4030 2.6598971913927e-08
4031 2.65981601188514e-08
4032 2.6597556157526e-08
4033 2.6596781665944e-08
4034 2.65961812573323e-08
4035 2.65954671618829e-08
4036 2.6594769053645e-08
4037 2.65941224597555e-08
4038 2.65933035592525e-08
4039 2.65924704478948e-08
4040 2.65916710873171e-08
4041 2.65909410046561e-08
4042 2.65900759188753e-08
4043 2.65892765582976e-08
4044 2.65885429229229e-08
4045 2.65876725080716e-08
4046 2.6587114732024e-08
4047 2.65863864257199e-08
4048 2.65855035763707e-08
4049 2.65845727653868e-08
4050 2.65836668233987e-08
4051 2.65830184531524e-08
4052 2.65823381084829e-08
4053 2.65815280897641e-08
4054 2.65809134702977e-08
4055 2.65800004228822e-08
4056 2.65795314646766e-08
4057 2.65786077591201e-08
4058 2.6578078404782e-08
4059 2.65773358876231e-08
4060 2.65764583673445e-08
4061 2.65757655881771e-08
4062 2.65750603745118e-08
4063 2.65746002980904e-08
4064 2.65737192250981e-08
4065 2.6572902100952e-08
4066 2.65722395198509e-08
4067 2.65713602232154e-08
4068 2.65708361979478e-08
4069 2.65699302559597e-08
4070 2.65694346524015e-08
4071 2.65684096945051e-08
4072 2.65679656052953e-08
4073 2.65670667687345e-08
4074 2.65663224752188e-08
4075 2.65655923925578e-08
4076 2.65646367125782e-08
4077 2.65640682783896e-08
4078 2.65633666174381e-08
4079 2.65624944262299e-08
4080 2.65618282924152e-08
4081 2.65611639349572e-08
4082 2.65603077309606e-08
4083 2.65597890347635e-08
4084 2.65591353354466e-08
4085 2.65583963710014e-08
4086 2.65576094449216e-08
4087 2.65568989021858e-08
4088 2.65561066470354e-08
4089 2.65554707112869e-08
4090 2.65548241173974e-08
4091 2.65538986354841e-08
4092 2.65531898691052e-08
4093 2.6552516629863e-08
4094 2.65516106878749e-08
4095 2.65507953400856e-08
4096 2.65501078899888e-08
4097 2.65494559670287e-08
4098 2.65486299610984e-08
4099 2.65478785621553e-08
4100 2.65470188054451e-08
4101 2.6546311815423e-08
4102 2.65455870618325e-08
4103 2.6544906717163e-08
4104 2.65441446600789e-08
4105 2.65435993185292e-08
4106 2.65427253509642e-08
4107 2.654206454622e-08
4108 2.65411053135267e-08
4109 2.65404640487077e-08
4110 2.65396487009184e-08
4111 2.65389257236848e-08
4112 2.65383643949235e-08
4113 2.65373802932345e-08
4114 2.65370303509371e-08
4115 2.65362700702099e-08
4116 2.65354156425701e-08
4117 2.65348134576016e-08
4118 2.65338258031989e-08
4119 2.65332253945871e-08
4120 2.65324935355693e-08
4121 2.65315609482286e-08
4122 2.65308326419245e-08
4123 2.65302411150969e-08
4124 2.65294719525855e-08
4125 2.65285908795931e-08
4126 2.6528050867114e-08
4127 2.65272852573162e-08
4128 2.65265427401573e-08
4129 2.65259281206909e-08
4130 2.65250506004122e-08
4131 2.65242725561166e-08
4132 2.65236543839364e-08
4133 2.65229260776323e-08
4134 2.6522155138764e-08
4135 2.6521414397962e-08
4136 2.65204906924055e-08
4137 2.65199116000758e-08
4138 2.65191779647012e-08
4139 2.6518350182414e-08
4140 2.65177142466655e-08
4141 2.65168615953826e-08
4142 2.65162753976256e-08
4143 2.651546360255e-08
4144 2.6514678452827e-08
4145 2.6514003437228e-08
4146 2.65133337506995e-08
4147 2.65124011633588e-08
4148 2.65116106845653e-08
4149 2.65106887553657e-08
4150 2.65101167684634e-08
4151 2.65093049733878e-08
4152 2.65085375872331e-08
4153 2.65077506611533e-08
4154 2.65068891280862e-08
4155 2.65062194415577e-08
4156 2.65053685666317e-08
4157 2.65045692060539e-08
4158 2.65040220881474e-08
4159 2.65034163504652e-08
4160 2.65023327727931e-08
4161 2.6501769667675e-08
4162 2.65011124156445e-08
4163 2.65004569399707e-08
4164 2.64996433685383e-08
4165 2.64989470366572e-08
4166 2.64982134012826e-08
4167 2.64973945007796e-08
4168 2.64966661944754e-08
4169 2.64958153195494e-08
4170 2.64951456330209e-08
4171 2.64943587069411e-08
4172 2.64937156657652e-08
4173 2.64931117044398e-08
4174 2.64922999093642e-08
4175 2.64915964720558e-08
4176 2.64910227087967e-08
4177 2.64903352586998e-08
4178 2.64897117574492e-08
4179 2.64885091638689e-08
4180 2.64876724997976e-08
4181 2.64871165001068e-08
4182 2.64864166155121e-08
4183 2.64857415999131e-08
4184 2.64850044118248e-08
4185 2.64841535368987e-08
4186 2.64834429941629e-08
4187 2.64830237739488e-08
4188 2.64822652695784e-08
4189 2.64813309058809e-08
4190 2.64807997751859e-08
4191 2.6479737513796e-08
4192 2.64794106641375e-08
4193 2.64782915593287e-08
4194 2.6478044645728e-08
4195 2.64769290936329e-08
4196 2.6476662640107e-08
4197 2.64758384105335e-08
4198 2.6475003522819e-08
4199 2.64742432420917e-08
4200 2.64736303989821e-08
4201 2.64728114984791e-08
4202 2.64719819398351e-08
4203 2.64714401509991e-08
4204 2.6470589276073e-08
4205 2.64699284713288e-08
4206 2.64691788487426e-08
4207 2.64682711303976e-08
4208 2.64678163830467e-08
4209 2.64672674887834e-08
4210 2.64663295723722e-08
4211 2.64656421222753e-08
4212 2.64648232217723e-08
4213 2.64639385960663e-08
4214 2.64634607560765e-08
4215 2.64627750823365e-08
4216 2.64618442713527e-08
4217 2.64614126166407e-08
4218 2.64605883870672e-08
4219 2.64596966559338e-08
4220 2.64590465093306e-08
4221 2.64583359665949e-08
4222 2.6457549040515e-08
4223 2.6456996593538e-08
4224 2.64560320317742e-08
4225 2.64554049778098e-08
4226 2.64547228567835e-08
4227 2.64539163907784e-08
4228 2.64532431515363e-08
4229 2.64527422189076e-08
4230 2.64516586412356e-08
4231 2.64510937597606e-08
4232 2.64503938751659e-08
4233 2.64496726742891e-08
4234 2.64486903489569e-08
4235 2.64479211864455e-08
4236 2.64472355127054e-08
4237 2.64466244459527e-08
4238 2.6445537315567e-08
4239 2.64448711817522e-08
4240 2.64443222874888e-08
4241 2.64435620067616e-08
4242 2.64430184415687e-08
4243 2.64424127038865e-08
4244 2.64414392603385e-08
4245 2.64406931904659e-08
4246 2.64399879768007e-08
4247 2.64392774340649e-08
4248 2.64383448467242e-08
4249 2.64378439140955e-08
4250 2.64367301383572e-08
4251 2.64363588797778e-08
4252 2.64355168866359e-08
4253 2.64347033152035e-08
4254 2.64341650790811e-08
4255 2.64333390731508e-08
4256 2.64324686582995e-08
4257 2.64317119302859e-08
4258 2.64307988828705e-08
4259 2.64302144614703e-08
4260 2.64293955609674e-08
4261 2.6428990551608e-08
4262 2.64284132356352e-08
4263 2.64274024885935e-08
4264 2.64267683292019e-08
4265 2.64259423232716e-08
4266 2.64252708603863e-08
4267 2.64246704517745e-08
4268 2.64237005609402e-08
4269 2.64233115387924e-08
4270 2.64223078971781e-08
4271 2.64215884726582e-08
4272 2.64209649714076e-08
4273 2.64201549526888e-08
4274 2.64196611254874e-08
4275 2.64186503784458e-08
4276 2.64180677334025e-08
4277 2.64172523856132e-08
4278 2.64165738173006e-08
4279 2.64157264950882e-08
4280 2.64150301632071e-08
4281 2.64142290262726e-08
4282 2.64134776273295e-08
4283 2.64127582028095e-08
4284 2.64120050275096e-08
4285 2.64113850789727e-08
4286 2.64106567726685e-08
4287 2.64099195845802e-08
4288 2.64093227286821e-08
4289 2.6408550013457e-08
4290 2.64077613110203e-08
4291 2.64069921485088e-08
4292 2.64061998933585e-08
4293 2.64053845455692e-08
4294 2.64048019005259e-08
4295 2.64040735942217e-08
4296 2.64031818630883e-08
4297 2.64024961893483e-08
4298 2.64017518958326e-08
4299 2.64011301709388e-08
4300 2.64003787719957e-08
4301 2.63997286253925e-08
4302 2.63987534054877e-08
4303 2.63980481918225e-08
4304 2.63970569847061e-08
4305 2.63966786206993e-08
4306 2.63959556434656e-08
4307 2.63952024681657e-08
4308 2.63943391587418e-08
4309 2.63936108524376e-08
4310 2.63929731403323e-08
4311 2.63921045018378e-08
4312 2.63913886300315e-08
4313 2.63904205155541e-08
4314 2.63899959662695e-08
4315 2.63892090401896e-08
4316 2.63885500118022e-08
4317 2.63878252582117e-08
4318 2.6386937079792e-08
4319 2.63863171312551e-08
4320 2.63854254001217e-08
4321 2.63849653237003e-08
4322 2.63841979375457e-08
4323 2.63831463342967e-08
4324 2.63824961876935e-08
4325 2.63818762391566e-08
4326 2.6381185236346e-08
4327 2.63803645594862e-08
4328 2.63796735566757e-08
4329 2.63788990650937e-08
4330 2.63780162157445e-08
4331 2.63773554110003e-08
4332 2.63766271046961e-08
4333 2.63759911689476e-08
4334 2.63750674633911e-08
4335 2.63745789652603e-08
4336 2.63737565120437e-08
4337 2.63731543270751e-08
4338 2.63724668769783e-08
4339 2.63717652160267e-08
4340 2.6371175465556e-08
4341 2.63704293956835e-08
4342 2.63695216773385e-08
4343 2.63688662016648e-08
4344 2.63679051926147e-08
4345 2.63677151224329e-08
4346 2.6366695493607e-08
4347 2.63659529764482e-08
4348 2.63653490151228e-08
4349 2.63646260378891e-08
4350 2.63639279296513e-08
4351 2.63631179109325e-08
4352 2.63626525054406e-08
4353 2.6361817617726e-08
4354 2.63610147044346e-08
4355 2.63602508709937e-08
4356 2.63594124305655e-08
4357 2.63587836002444e-08
4358 2.63581831916326e-08
4359 2.63577426551365e-08
4360 2.63566448666097e-08
4361 2.63560266944296e-08
4362 2.63551171997278e-08
4363 2.63545825163192e-08
4364 2.63540922418315e-08
4365 2.63533852518094e-08
4366 2.63526001020864e-08
4367 2.63518398213591e-08
4368 2.63511736875444e-08
4369 2.63503174835478e-08
4370 2.63494843721901e-08
4371 2.63488306728732e-08
4372 2.6348102366569e-08
4373 2.63475019579573e-08
4374 2.63467505590143e-08
4375 2.6346034687208e-08
4376 2.63452388793439e-08
4377 2.63445425474629e-08
4378 2.63439368097806e-08
4379 2.63430095515105e-08
4380 2.63424446700355e-08
4381 2.63416701784536e-08
4382 2.63408477252369e-08
4383 2.63401869204927e-08
4384 2.63391939370194e-08
4385 2.63385970811214e-08
4386 2.63379007492404e-08
4387 2.63373109987697e-08
4388 2.63364530184163e-08
4389 2.6335603919847e-08
4390 2.63348418627629e-08
4391 2.63342236905828e-08
4392 2.63335184769176e-08
4393 2.63325805605064e-08
4394 2.63322945670552e-08
4395 2.63314152704197e-08
4396 2.63304080760918e-08
4397 2.6329995961305e-08
4398 2.63290580448938e-08
4399 2.63283741475107e-08
4400 2.63277453171895e-08
4401 2.63267363465047e-08
4402 2.63261625832456e-08
4403 2.63254307242278e-08
4404 2.6324510571385e-08
4405 2.63238995046322e-08
4406 2.63231729746849e-08
4407 2.63226951346951e-08
4408 2.63219295248973e-08
4409 2.63211088480375e-08
4410 2.6320527979351e-08
4411 2.63195634175872e-08
4412 2.63192152516467e-08
4413 2.63183697057912e-08
4414 2.6317486856442e-08
4415 2.6316657297798e-08
4416 2.631602669112e-08
4417 2.63154671387156e-08
4418 2.63146979762041e-08
4419 2.63139590117589e-08
4420 2.63135806477521e-08
4421 2.63126676003367e-08
4422 2.63120067955924e-08
4423 2.63110813136791e-08
4424 2.63102766240308e-08
4425 2.63095856212203e-08
4426 2.63091219920852e-08
4427 2.63083368423622e-08
4428 2.63077133411116e-08
4429 2.63070347727989e-08
4430 2.63060488947531e-08
4431 2.63054378280003e-08
4432 2.63047841286834e-08
4433 2.63039172665458e-08
4434 2.6303284883511e-08
4435 2.63026667113309e-08
4436 2.63019508395246e-08
4437 2.6301160360731e-08
4438 2.63004942269163e-08
4439 2.62994035438169e-08
4440 2.62988795185493e-08
4441 2.62980464071916e-08
4442 2.62971990849792e-08
4443 2.62964885422434e-08
4444 2.62956802998815e-08
4445 2.62951527219002e-08
4446 2.62941650674975e-08
4447 2.62935664352426e-08
4448 2.62928381289385e-08
4449 2.62924828575706e-08
4450 2.62913815163301e-08
4451 2.62907278170132e-08
4452 2.62900350378459e-08
4453 2.6289326271467e-08
4454 2.6288761389992e-08
4455 2.62880792689657e-08
4456 2.62871768796913e-08
4457 2.62864574551713e-08
4458 2.62855248678306e-08
4459 2.62848658394432e-08
4460 2.62841997056285e-08
4461 2.62836987729997e-08
4462 2.62829615849114e-08
4463 2.62820964991306e-08
4464 2.62814499052411e-08
4465 2.62807127171527e-08
4466 2.62798032224509e-08
4467 2.62794426220125e-08
4468 2.6278497600174e-08
4469 2.6277513498485e-08
4470 2.62770640802046e-08
4471 2.6276417486315e-08
4472 2.62755026625427e-08
4473 2.62750639024034e-08
4474 2.6274459941078e-08
4475 2.62734065614723e-08
4476 2.62728399036405e-08
4477 2.62720618593448e-08
4478 2.62713228948996e-08
4479 2.62707082754332e-08
4480 2.62698449660093e-08
4481 2.62691930430492e-08
4482 2.62685286855913e-08
4483 2.62679691331869e-08
4484 2.62670116768504e-08
4485 2.62661163930034e-08
4486 2.62657255944987e-08
4487 2.62648356397222e-08
4488 2.62640309500739e-08
4489 2.62635477810136e-08
4490 2.62625388103288e-08
4491 2.62620183377749e-08
4492 2.62612278589813e-08
4493 2.6260645213938e-08
4494 2.62598529587876e-08
4495 2.62589843202932e-08
4496 2.62584087806772e-08
4497 2.62577941612108e-08
4498 2.62571209219686e-08
4499 2.62561350439228e-08
4500 2.62556003605141e-08
4501 2.62547956708659e-08
4502 2.62540851281301e-08
4503 2.62536019590698e-08
4504 2.62528843109067e-08
4505 2.62519588289933e-08
4506 2.62514898707877e-08
4507 2.62508166315456e-08
4508 2.62496655523137e-08
4509 2.62492250158175e-08
4510 2.62483279556136e-08
4511 2.62474433299076e-08
4512 2.62467576561676e-08
4513 2.62461306022033e-08
4514 2.62452868327046e-08
4515 2.62445816190393e-08
4516 2.62440043030665e-08
4517 2.62433488273928e-08
4518 2.62425476904582e-08
4519 2.6241842476793e-08
4520 2.62409987072942e-08
4521 2.62404391548898e-08
4522 2.62396167016732e-08
4523 2.62389647787131e-08
4524 2.62381547599944e-08
4525 2.62373269777072e-08
4526 2.62368438086469e-08
4527 2.62358419433895e-08
4528 2.62353285762629e-08
4529 2.62346002699587e-08
4530 2.62340158485586e-08
4531 2.62333355038891e-08
4532 2.62324153510463e-08
4533 2.62317367827336e-08
4534 2.6230825511675e-08
4535 2.62302251030633e-08
4536 2.62293191610752e-08
4537 2.6228569538489e-08
4538 2.62277257689902e-08
4539 2.62272585871415e-08
4540 2.62265302808373e-08
4541 2.62258499361678e-08
4542 2.62249901794576e-08
4543 2.62241712789546e-08
4544 2.62235619885587e-08
4545 2.62229935543701e-08
4546 2.6222386040331e-08
4547 2.62213983859283e-08
4548 2.62209454149342e-08
4549 2.62201744760659e-08
4550 2.62193200484262e-08
4551 2.62186539146114e-08
4552 2.62178136978264e-08
4553 2.62172665799199e-08
4554 2.62164210340643e-08
4555 2.62155364083583e-08
4556 2.62149875140949e-08
4557 2.6214179271733e-08
4558 2.62134509654288e-08
4559 2.62128700967423e-08
4560 2.62119286276175e-08
4561 2.62112731519437e-08
4562 2.62106762960457e-08
4563 2.6209500347818e-08
4564 2.62089905334051e-08
4565 2.62084540736396e-08
4566 2.62077666235427e-08
4567 2.62069068668325e-08
4568 2.62062833655818e-08
4569 2.62055266375683e-08
4570 2.62047858967662e-08
4571 2.62041659482293e-08
4572 2.6203405667502e-08
4573 2.62026773611979e-08
4574 2.6201989911101e-08
4575 2.6201115943536e-08
4576 2.62005439566337e-08
4577 2.61997286088445e-08
4578 2.61990233951792e-08
4579 2.61982826543772e-08
4580 2.61975579007867e-08
4581 2.61970303228054e-08
4582 2.61961154990331e-08
4583 2.61953694291606e-08
4584 2.61947459279099e-08
4585 2.61941170975888e-08
4586 2.61929518075021e-08
4587 2.61924046895956e-08
4588 2.61917989519134e-08
4589 2.61911221599576e-08
4590 2.6190420499006e-08
4591 2.61896886399882e-08
4592 2.61890225061734e-08
4593 2.61883670304996e-08
4594 2.61875428009262e-08
4595 2.61867185713527e-08
4596 2.61860240158285e-08
4597 2.61855053196314e-08
4598 2.61846668792032e-08
4599 2.61840469306662e-08
4600 2.61832102665949e-08
4601 2.61828372316586e-08
4602 2.61821178071386e-08
4603 2.61812633794989e-08
4604 2.61806167856093e-08
4605 2.61797516998286e-08
4606 2.61790038535992e-08
4607 2.61783856814191e-08
4608 2.61776040844097e-08
4609 2.6177103151781e-08
4610 2.6176218526075e-08
4611 2.61754315999951e-08
4612 2.6174802769674e-08
4613 2.61738932749722e-08
4614 2.61731045725355e-08
4615 2.61726178507615e-08
4616 2.61719481642331e-08
4617 2.61713584137624e-08
4618 2.61705466186868e-08
4619 2.61697383763249e-08
4620 2.61691734948499e-08
4621 2.61685446645288e-08
4622 2.61676813551048e-08
4623 2.6166960154228e-08
4624 2.61663384293342e-08
4625 2.61655177524744e-08
4626 2.61646810884031e-08
4627 2.61640256127293e-08
4628 2.61632688847158e-08
4629 2.61624180097897e-08
4630 2.6161584898432e-08
4631 2.61611905472137e-08
4632 2.61603823048517e-08
4633 2.61596859729707e-08
4634 2.61589097050319e-08
4635 2.61581583060888e-08
4636 2.61573074311627e-08
4637 2.61568651183097e-08
4638 2.61561368120056e-08
4639 2.61553090297184e-08
4640 2.61546890811815e-08
4641 2.61537334012019e-08
4642 2.61533106282741e-08
4643 2.61523407374398e-08
4644 2.61516390764882e-08
4645 2.6151198539992e-08
4646 2.61505324061773e-08
4647 2.61495376463472e-08
4648 2.6148832432682e-08
4649 2.61481751806514e-08
4650 2.6147416676281e-08
4651 2.61466244211306e-08
4652 2.61459458528179e-08
4653 2.61451305050286e-08
4654 2.61442600901773e-08
4655 2.6143588627292e-08
4656 2.61428105829964e-08
4657 2.61423416247908e-08
4658 2.61415067370763e-08
4659 2.61407393509216e-08
4660 2.61400359136132e-08
4661 2.61393715561553e-08
4662 2.61385491029387e-08
4663 2.61379344834722e-08
4664 2.61373553911426e-08
4665 2.61368136023066e-08
4666 2.61357779862692e-08
4667 2.61351527086617e-08
4668 2.61341916996116e-08
4669 2.61338843898784e-08
4670 2.61329500261809e-08
4671 2.61323993555607e-08
4672 2.61315520333483e-08
4673 2.61307651072684e-08
4674 2.61299675230475e-08
4675 2.61292143477476e-08
4676 2.61284949232277e-08
4677 2.61278927382591e-08
4678 2.61269406109932e-08
4679 2.6126455665576e-08
4680 2.61258321643254e-08
4681 2.61249990529677e-08
4682 2.61241872578921e-08
4683 2.61235015841521e-08
4684 2.61227306452838e-08
4685 2.61220129971207e-08
4686 2.61213148888828e-08
4687 2.6120583029865e-08
4688 2.61198227491377e-08
4689 2.61191352990409e-08
4690 2.61185633121386e-08
4691 2.61175827631632e-08
4692 2.61171226867418e-08
4693 2.61164423420723e-08
4694 2.6115650086922e-08
4695 2.61147672375728e-08
4696 2.61140424839823e-08
4697 2.61133692447402e-08
4698 2.61125769895898e-08
4699 2.61119179612024e-08
4700 2.61112340638192e-08
4701 2.61106105625686e-08
4702 2.61097419240741e-08
4703 2.6109134410035e-08
4704 2.6108338602171e-08
4705 2.61077985896918e-08
4706 2.61068109352891e-08
4707 2.61063402007267e-08
4708 2.61056349870614e-08
4709 2.61048480609816e-08
4710 2.6104002515126e-08
4711 2.61031409820589e-08
4712 2.61027004455627e-08
4713 2.61018069380725e-08
4714 2.61012385038839e-08
4715 2.61006771751227e-08
4716 2.60998085366282e-08
4717 2.60991317446724e-08
4718 2.60986698918941e-08
4719 2.60976467103546e-08
4720 2.60972594645636e-08
4721 2.6096381944285e-08
4722 2.60957673248186e-08
4723 2.609516513985e-08
4724 2.60942414342935e-08
4725 2.60936143803292e-08
4726 2.6092964233726e-08
4727 2.6092134675082e-08
4728 2.60914578831262e-08
4729 2.60908006310956e-08
4730 2.60901735771313e-08
4731 2.60894719161797e-08
4732 2.60885926195442e-08
4733 2.60879460256547e-08
4734 2.60873971313913e-08
4735 2.60866865886555e-08
4736 2.60858392664431e-08
4737 2.60853525446691e-08
4738 2.6084514104241e-08
4739 2.60840327115375e-08
4740 2.60831782838977e-08
4741 2.60824908338009e-08
4742 2.6081803383704e-08
4743 2.60811745533829e-08
4744 2.60804462470787e-08
4745 2.60797730078366e-08
4746 2.60789487782631e-08
4747 2.60782968553031e-08
4748 2.60777142102597e-08
4749 2.60768491244789e-08
4750 2.60761634507389e-08
4751 2.60755630421272e-08
4752 2.60748347358231e-08
4753 2.60739323465486e-08
4754 2.60736232604586e-08
4755 2.60726391587696e-08
4756 2.60720156575189e-08
4757 2.60712234023686e-08
4758 2.60705679266948e-08
4759 2.60696726428478e-08
4760 2.60691130904434e-08
4761 2.60682835317994e-08
4762 2.60676227270551e-08
4763 2.60666492835071e-08
4764 2.60658445938589e-08
4765 2.60651979999693e-08
4766 2.60644998917314e-08
4767 2.60635850679591e-08
4768 2.60630983461851e-08
4769 2.6062226154977e-08
4770 2.60615973246558e-08
4771 2.6061009350542e-08
4772 2.60600003798572e-08
4773 2.60595260925811e-08
4774 2.60586663358708e-08
4775 2.60579291477825e-08
4776 2.60572363686151e-08
4777 2.60563997045438e-08
4778 2.60557513342974e-08
4779 2.60549342101513e-08
4780 2.60544155139542e-08
4781 2.60537440510689e-08
4782 2.60528629780765e-08
4783 2.60521648698386e-08
4784 2.60514738670281e-08
4785 2.60508414839933e-08
4786 2.60501682447511e-08
4787 2.60494985582227e-08
4788 2.60488111081258e-08
4789 2.60480099711913e-08
4790 2.60475552238404e-08
4791 2.60466297419271e-08
4792 2.60461394674394e-08
4793 2.60453116851522e-08
4794 2.60447912125983e-08
4795 2.60438355326187e-08
4796 2.6043290191069e-08
4797 2.60421373354802e-08
4798 2.60415742303621e-08
4799 2.604090099112e-08
4800 2.6040204659239e-08
4801 2.60393129281056e-08
4802 2.60385846218014e-08
4803 2.60380428329654e-08
4804 2.60372186033919e-08
4805 2.60363499648975e-08
4806 2.60357477799289e-08
4807 2.60349910519153e-08
4808 2.60346055824812e-08
4809 2.60335770718712e-08
4810 2.60330565993172e-08
4811 2.60322252643164e-08
4812 2.60316248557046e-08
4813 2.6030846811409e-08
4814 2.60301984411626e-08
4815 2.60292303266851e-08
4816 2.60285926145798e-08
4817 2.60280561548143e-08
4818 2.60273367302943e-08
4819 2.60265036189367e-08
4820 2.6025880117686e-08
4821 2.60248693706444e-08
4822 2.60242405403233e-08
4823 2.60236365789979e-08
4824 2.60227448478645e-08
4825 2.6022197729958e-08
4826 2.60216150849146e-08
4827 2.60208548041874e-08
4828 2.6020007481975e-08
4829 2.60194195078611e-08
4830 2.601875692676e-08
4831 2.60179771061075e-08
4832 2.60172452470897e-08
4833 2.6016490295433e-08
4834 2.60156198805817e-08
4835 2.60149786157626e-08
4836 2.60143089292342e-08
4837 2.60137849039666e-08
4838 2.60129446871815e-08
4839 2.60123105277899e-08
4840 2.60115520234194e-08
4841 2.60107899663353e-08
4842 2.60101309379479e-08
4843 2.6009367104507e-08
4844 2.60085641912156e-08
4845 2.60079957570269e-08
4846 2.60072354762997e-08
4847 2.60063561796642e-08
4848 2.60055887935096e-08
4849 2.60051500333702e-08
4850 2.60041996824611e-08
4851 2.6003515785078e-08
4852 2.60028016896285e-08
4853 2.60021728593074e-08
4854 2.60013592878749e-08
4855 2.6000749997479e-08
4856 2.59998582663457e-08
4857 2.59990233786311e-08
4858 2.5998305730468e-08
4859 2.59977692707025e-08
4860 2.59969130667059e-08
4861 2.59961883131155e-08
4862 2.59956163262132e-08
4863 2.59947174896524e-08
4864 2.59940620139787e-08
4865 2.59934953561469e-08
4866 2.59928150114774e-08
4867 2.59919410439124e-08
4868 2.59912233957493e-08
4869 2.59907046995522e-08
4870 2.59899355370408e-08
4871 2.59892107834503e-08
4872 2.59883830011631e-08
4873 2.59877577235557e-08
4874 2.59870756025293e-08
4875 2.59864219032124e-08
4876 2.5985618989921e-08
4877 2.59851518080723e-08
4878 2.59842867222915e-08
4879 2.59836312466177e-08
4880 2.59828514259652e-08
4881 2.59822545700672e-08
4882 2.59816737013807e-08
4883 2.59809453950766e-08
4884 2.59801584689967e-08
4885 2.59793981882694e-08
4886 2.59785153389203e-08
4887 2.59778030198277e-08
4888 2.5977268336419e-08
4889 2.59766785859483e-08
4890 2.5975841921877e-08
4891 2.59751047337886e-08
4892 2.59743320185635e-08
4893 2.59736125940435e-08
4894 2.59729890927929e-08
4895 2.59723424989033e-08
4896 2.59714774131226e-08
4897 2.59709089789339e-08
4898 2.59702357396918e-08
4899 2.59695553950223e-08
4900 2.59688235360045e-08
4901 2.59679300285143e-08
4902 2.59673544888983e-08
4903 2.5966638617092e-08
4904 2.59658285983733e-08
4905 2.59651216083512e-08
4906 2.59644590272501e-08
4907 2.5963862171352e-08
4908 2.59631942611804e-08
4909 2.59623291753996e-08
4910 2.59617003450785e-08
4911 2.59608796682187e-08
4912 2.59600110297242e-08
4913 2.59593200269137e-08
4914 2.5958874161347e-08
4915 2.59578136763139e-08
4916 2.59574353123071e-08
4917 2.59567709548492e-08
4918 2.59558010640148e-08
4919 2.59551171666317e-08
4920 2.59544528091737e-08
4921 2.59537813462885e-08
4922 2.59529517876445e-08
4923 2.59522732193318e-08
4924 2.59516905742885e-08
4925 2.59509445044159e-08
4926 2.5950146920195e-08
4927 2.59494754573097e-08
4928 2.59486458986657e-08
4929 2.59480188447014e-08
4930 2.59474202124466e-08
4931 2.59468695418263e-08
4932 2.59460115614729e-08
4933 2.59452512807457e-08
4934 2.59444767891637e-08
4935 2.59438515115562e-08
4936 2.59432830773676e-08
4937 2.59423362791722e-08
4938 2.59418122539046e-08
4939 2.59407997305061e-08
4940 2.59401620184008e-08
4941 2.59395473989343e-08
4942 2.59385490863906e-08
4943 2.59379504541357e-08
4944 2.59370764865707e-08
4945 2.59366927934934e-08
4946 2.59359520526914e-08
4947 2.59351402576158e-08
4948 2.59344989927968e-08
4949 2.59338968078282e-08
4950 2.59334242969089e-08
4951 2.59323638118758e-08
4952 2.59316177420033e-08
4953 2.59309302919064e-08
4954 2.59302836980169e-08
4955 2.59299230975785e-08
4956 2.59290278137314e-08
4957 2.59281573988801e-08
4958 2.59273011948835e-08
4959 2.59266599300645e-08
4960 2.59261589974358e-08
4961 2.5925348978717e-08
4962 2.59245318545709e-08
4963 2.59238124300509e-08
4964 2.59232049160119e-08
4965 2.59225583221223e-08
4966 2.59214623099524e-08
4967 2.59207357800051e-08
4968 2.59203432051436e-08
4969 2.59194408158692e-08
4970 2.59189043561037e-08
4971 2.59181955897247e-08
4972 2.59174584016364e-08
4973 2.5916754964328e-08
4974 2.59158863258335e-08
4975 2.59151864412388e-08
4976 2.59144528058641e-08
4977 2.59139394387375e-08
4978 2.59132022506492e-08
4979 2.59125219059797e-08
4980 2.59117030054767e-08
4981 2.59110564115872e-08
4982 2.59102801436484e-08
4983 2.59090953136365e-08
4984 2.59087951093306e-08
4985 2.59079442344046e-08
4986 2.59072390207393e-08
4987 2.59069050656535e-08
4988 2.59058694496161e-08
4989 2.59053258844233e-08
4990 2.59045425110571e-08
4991 2.59036561089943e-08
4992 2.59032546523486e-08
4993 2.59026151638864e-08
4994 2.5901668365691e-08
4995 2.59011443404233e-08
4996 2.59004693248244e-08
4997 2.58996557533919e-08
4998 2.58990056067887e-08
4999 2.58983483547581e-08
};
\addlegendentry{Test}

\nextgroupplot[
title={SiLU/SiLU $\rare$},
ymin=6.69109266508683e-09, ymax=1e-05,
]
\addplot [semithick, black, dashed]
table {%
0 0.0052208109376661
1 0.000356375322051463
2 0.000180012169860674
3 0.000167791900978955
4 0.000141975468598503
5 8.67083700484272e-05
6 3.87360342181182e-05
7 3.08458749971123e-05
8 2.94005234950987e-05
9 2.83507832310761e-05
10 2.72112455841977e-05
11 2.56166622367573e-05
12 2.29804413716863e-05
13 1.8486761055442e-05
14 1.22981297470659e-05
15 7.38985352349175e-06
16 5.6367741808856e-06
17 5.20786944569807e-06
18 4.97421912105267e-06
19 4.75995320812217e-06
20 4.53732039224519e-06
21 4.297673597776e-06
22 4.03557224204576e-06
23 3.74720948082086e-06
24 3.43215091634619e-06
25 3.09656116743184e-06
26 2.75611381402285e-06
27 2.43638394917411e-06
28 2.16625076828336e-06
29 1.9652623397306e-06
30 1.8336223693538e-06
31 1.75421078288807e-06
32 1.70540041010625e-06
33 1.67204651830133e-06
34 1.64606191576766e-06
35 1.6237817550433e-06
36 1.60364063711782e-06
37 1.58493259985804e-06
38 1.56730181323894e-06
39 1.55054217414374e-06
40 1.534512104854e-06
41 1.51909967589603e-06
42 1.50421244602583e-06
43 1.48977316424137e-06
44 1.47571524871992e-06
45 1.46198292623723e-06
46 1.44852724436717e-06
47 1.43530351698473e-06
48 1.42227152180752e-06
49 1.40939240490212e-06
50 1.39663089720798e-06
51 1.38395576967909e-06
52 1.37133860774696e-06
53 1.35875010566266e-06
54 1.34616515816433e-06
55 1.3335563218746e-06
56 1.32089891868148e-06
57 1.3081675165445e-06
58 1.29533793008463e-06
59 1.28238619861776e-06
60 1.26928812110094e-06
61 1.25601946827558e-06
62 1.24255752964331e-06
63 1.22887941616412e-06
64 1.21496013845857e-06
65 1.20077861872403e-06
66 1.18631492301091e-06
67 1.17155108770106e-06
68 1.15647268024333e-06
69 1.14106544688219e-06
70 1.12531898128054e-06
71 1.10922592001295e-06
72 1.09278607229335e-06
73 1.07600287108234e-06
74 1.05888835839352e-06
75 1.04145849586068e-06
76 1.02373261420041e-06
77 1.00573563385886e-06
78 9.87501615661301e-07
79 9.69073427350864e-07
80 9.50503085892507e-07
81 9.3185249791361e-07
82 9.13190490914317e-07
83 8.94595551708122e-07
84 8.76151731064923e-07
85 8.57950057035595e-07
86 8.40089081274087e-07
87 8.22675061924016e-07
88 8.0580580766032e-07
89 7.8955457812846e-07
90 7.73975800708371e-07
91 7.59113648667764e-07
92 7.44999283210035e-07
93 7.3164968390671e-07
94 7.19045752475722e-07
95 7.07159162438131e-07
96 6.9596228535751e-07
97 6.85385905976332e-07
98 6.75357465999227e-07
99 6.65811664847027e-07
100 6.56685516572608e-07
101 6.47918337781661e-07
102 6.39456988421827e-07
103 6.31255386737806e-07
104 6.23274849754552e-07
105 6.15483139107198e-07
106 6.07851918886126e-07
107 6.00359780646542e-07
108 5.92989382564468e-07
109 5.85725802221049e-07
110 5.78558052247047e-07
111 5.71475086315942e-07
112 5.64469429264136e-07
113 5.57531910756026e-07
114 5.50655809723466e-07
115 5.43835881943622e-07
116 5.37066037026079e-07
117 5.3034266989016e-07
118 5.23660169946538e-07
119 5.17013627380436e-07
120 5.10398153785729e-07
121 5.03808362006808e-07
122 4.97238237237241e-07
123 4.9068284239695e-07
124 4.84137846633814e-07
125 4.77597629368276e-07
126 4.71057172541478e-07
127 4.64513151113621e-07
128 4.57960722178186e-07
129 4.5139786539572e-07
130 4.44820270052659e-07
131 4.38226321753277e-07
132 4.31615149803832e-07
133 4.249849179363e-07
134 4.18336964489896e-07
135 4.11671767450272e-07
136 4.04993426691647e-07
137 3.98304352737355e-07
138 3.91611546572079e-07
139 3.84920643673681e-07
140 3.7823978729179e-07
141 3.71578420899965e-07
142 3.64947551792838e-07
143 3.58359998529956e-07
144 3.51830751440296e-07
145 3.45373837946639e-07
146 3.39005656653235e-07
147 3.3274286681273e-07
148 3.26601511802416e-07
149 3.20596719038058e-07
150 3.14744907516484e-07
151 3.09067372206151e-07
152 3.03572572133426e-07
153 2.9826101307151e-07
154 2.93145306806686e-07
155 2.88243052920301e-07
156 2.83568003255574e-07
157 2.7913128094692e-07
158 2.74964960439661e-07
159 2.71063417497608e-07
160 2.67399560533121e-07
161 2.63955171707408e-07
162 2.60749555098805e-07
163 2.57794805286871e-07
164 2.55030503761411e-07
165 2.5246569316284e-07
166 2.50079490289323e-07
167 2.47858400549639e-07
168 2.45816610103589e-07
169 2.4394604470146e-07
170 2.42208375116348e-07
171 2.40582016224344e-07
172 2.39053052829341e-07
173 2.3760928607075e-07
174 2.36239054271614e-07
175 2.34934086122962e-07
176 2.33685727154764e-07
177 2.32488393936414e-07
178 2.31336614424649e-07
179 2.30224688818836e-07
180 2.29148420803682e-07
181 2.281035868128e-07
182 2.27086605434579e-07
183 2.26094797288745e-07
184 2.2512457186874e-07
185 2.2417454870638e-07
186 2.23241977002075e-07
187 2.22325098049758e-07
188 2.21422976784069e-07
189 2.2053462497329e-07
190 2.19659092238977e-07
191 2.18795364510527e-07
192 2.17942655716286e-07
193 2.17100404486636e-07
194 2.16268140278864e-07
195 2.15444731840542e-07
196 2.1463018855794e-07
197 2.1382321635155e-07
198 2.13024398322048e-07
199 2.1223216815347e-07
200 2.11447023684919e-07
201 2.10667661219688e-07
202 2.09894512747333e-07
203 2.09126952504413e-07
204 2.08364353079915e-07
205 2.07607090570328e-07
206 2.06854442788895e-07
207 2.06106209119206e-07
208 2.05362505438345e-07
209 2.04622850478842e-07
210 2.03886674609244e-07
211 2.03154564930585e-07
212 2.02425710821963e-07
213 2.01700272119076e-07
214 2.00977630149701e-07
215 2.00258020618627e-07
216 1.99541435560668e-07
217 1.98826413897457e-07
218 1.98115088379858e-07
219 1.97405003293127e-07
220 1.96696497031468e-07
221 1.95989686842424e-07
222 1.95283091853682e-07
223 1.94579091651548e-07
224 1.93876532249782e-07
225 1.9317139075703e-07
226 1.92465245654638e-07
227 1.91770096303223e-07
228 1.91080057345694e-07
229 1.90391715586991e-07
230 1.89704610132679e-07
231 1.89019272091251e-07
232 1.8833488736103e-07
233 1.87651305375347e-07
234 1.86969844084395e-07
235 1.86288303666515e-07
236 1.85608119793734e-07
237 1.84929006907808e-07
238 1.84251075952879e-07
239 1.83573496800982e-07
240 1.82897287238859e-07
241 1.82221348632439e-07
242 1.81546554783196e-07
243 1.80871828170837e-07
244 1.80198313590729e-07
245 1.79525537656211e-07
246 1.78853220899278e-07
247 1.78181554636225e-07
248 1.77510500395606e-07
249 1.76840089534558e-07
250 1.76170709528911e-07
251 1.75501737089689e-07
252 1.74833396853202e-07
253 1.74165295099726e-07
254 1.73498659274074e-07
255 1.72832143849888e-07
256 1.72166008947094e-07
257 1.71500879322828e-07
258 1.70836308836186e-07
259 1.7017228715499e-07
260 1.69508454649581e-07
261 1.68845999743539e-07
262 1.68183343646433e-07
263 1.67521855998665e-07
264 1.66860285342629e-07
265 1.66200292646579e-07
266 1.65540191868097e-07
267 1.6488126471792e-07
268 1.64223002045105e-07
269 1.63565782206554e-07
270 1.62909274399325e-07
271 1.62253776993637e-07
272 1.61599398690093e-07
273 1.60946479478952e-07
274 1.6029416635277e-07
275 1.59643872057025e-07
276 1.58994504381837e-07
277 1.58347082983568e-07
278 1.57700947863049e-07
279 1.57056271695488e-07
280 1.56413586555715e-07
281 1.55772544952804e-07
282 1.55133060097512e-07
283 1.54495772124008e-07
284 1.53860381296056e-07
285 1.53227313339777e-07
286 1.52596119172621e-07
287 1.51967722183954e-07
288 1.5134127923222e-07
289 1.5071735338168e-07
290 1.50096043049608e-07
291 1.4947731797843e-07
292 1.4886167352568e-07
293 1.48248319294275e-07
294 1.47637959605262e-07
295 1.47030656710179e-07
296 1.46427115531367e-07
297 1.45825629827456e-07
298 1.45228492504756e-07
299 1.4463432034173e-07
300 1.44043418116802e-07
301 1.43456961928301e-07
302 1.42873276215205e-07
303 1.42294361623385e-07
304 1.41718985739558e-07
305 1.4114749946792e-07
306 1.40580729651951e-07
307 1.40018049068935e-07
308 1.39459721339996e-07
309 1.38905664024414e-07
310 1.38356008508644e-07
311 1.37811242485242e-07
312 1.37270748833629e-07
313 1.36735102104879e-07
314 1.36203813044133e-07
315 1.35677436597703e-07
316 1.35155725066838e-07
317 1.34638744843762e-07
318 1.34126712519667e-07
319 1.3361960224767e-07
320 1.33117125312943e-07
321 1.32619717883919e-07
322 1.32126852990133e-07
323 1.31638989318006e-07
324 1.31155960040097e-07
325 1.30677307239146e-07
326 1.30204122874567e-07
327 1.29735122055941e-07
328 1.29270760005795e-07
329 1.28811402575479e-07
330 1.28356687397968e-07
331 1.27906718420334e-07
332 1.27460922759504e-07
333 1.27019819909258e-07
334 1.26583292868077e-07
335 1.2615075984268e-07
336 1.2572250073184e-07
337 1.25297947896286e-07
338 1.24875501607669e-07
339 1.24454862957535e-07
340 1.24041640455097e-07
341 1.23635459610583e-07
342 1.23234090589897e-07
343 1.22837586113977e-07
344 1.22445562223739e-07
345 1.2205753683503e-07
346 1.21673297858393e-07
347 1.21293497788333e-07
348 1.20917566133638e-07
349 1.20545301201425e-07
350 1.20176992054244e-07
351 1.19812865597702e-07
352 1.1945157821236e-07
353 1.19094752765392e-07
354 1.18741091803987e-07
355 1.18391255437444e-07
356 1.18044936157524e-07
357 1.17702143485054e-07
358 1.17362772660723e-07
359 1.17026772367801e-07
360 1.16694435398745e-07
361 1.16364998689278e-07
362 1.16039127579182e-07
363 1.15716349711636e-07
364 1.15396692455416e-07
365 1.15080136659618e-07
366 1.14766332527161e-07
367 1.14455556903703e-07
368 1.14147990552649e-07
369 1.13842960995036e-07
370 1.13540630631181e-07
371 1.13241063011582e-07
372 1.12944265669768e-07
373 1.12650347172671e-07
374 1.1235819920774e-07
375 1.1206946083453e-07
376 1.11782473663169e-07
377 1.1149810187927e-07
378 1.11216076775555e-07
379 1.10936148015828e-07
380 1.10659010826186e-07
381 1.10383846174233e-07
382 1.10110855492174e-07
383 1.09840018186702e-07
384 1.09571292073785e-07
385 1.09304594180681e-07
386 1.09040010875194e-07
387 1.08777529355564e-07
388 1.08516645248802e-07
389 1.08257983666427e-07
390 1.08000951354992e-07
391 1.0774592820173e-07
392 1.07492907699847e-07
393 1.07241283769266e-07
394 1.06991459000394e-07
395 1.06743550133626e-07
396 1.06497033264485e-07
397 1.06252023081232e-07
398 1.06008451246975e-07
399 1.05766406540031e-07
400 1.055255904161e-07
401 1.05286147948469e-07
402 1.05047585790885e-07
403 1.04809993783928e-07
404 1.04572947737136e-07
405 1.0433642483898e-07
406 1.04100765838222e-07
407 1.03867509848587e-07
408 1.03637002327073e-07
409 1.03409940626964e-07
410 1.03184867675132e-07
411 1.02961257449241e-07
412 1.02739289976839e-07
413 1.02518803445584e-07
414 1.02299306765641e-07
415 1.02080802888427e-07
416 1.01863575278305e-07
417 1.01645509895665e-07
418 1.01424679846751e-07
419 1.01198780404044e-07
420 1.00988588030315e-07
421 1.00814562116636e-07
422 1.00573971165474e-07
423 1.0039196160383e-07
424 1.00174758392413e-07
425 1.00009792805889e-07
426 9.9777263872447e-08
427 9.96055629527426e-08
428 9.93899648733532e-08
429 9.92337490623996e-08
430 9.90046193298788e-08
431 9.88412369054004e-08
432 9.86283515889141e-08
433 9.84755215731603e-08
434 9.82524589310252e-08
435 9.81015571408506e-08
436 9.78856803666162e-08
437 9.77349603226152e-08
438 9.7523471109362e-08
439 9.73702325146597e-08
440 9.71683400421419e-08
441 9.70090942074009e-08
442 9.68176979552204e-08
443 9.66517100855491e-08
444 9.64701803276569e-08
445 9.62989832213879e-08
446 9.61237451133634e-08
447 9.5951612688161e-08
448 9.57797742189292e-08
449 9.56093694752802e-08
450 9.5440403848901e-08
451 9.5273273822194e-08
452 9.51076988182642e-08
453 9.49435090689121e-08
454 9.47811474238947e-08
455 9.46194986819471e-08
456 9.44596659100405e-08
457 9.43004581639428e-08
458 9.41427733636147e-08
459 9.39861254662944e-08
460 9.38301298307742e-08
461 9.36758090723799e-08
462 9.35220249003166e-08
463 9.33693956626236e-08
464 9.32178318300281e-08
465 9.30670547325541e-08
466 9.2917310949403e-08
467 9.2768490505879e-08
468 9.26208911806725e-08
469 9.24738053069696e-08
470 9.23277587356885e-08
471 9.21828848352568e-08
472 9.20384305960198e-08
473 9.18951706645466e-08
474 9.17526440331606e-08
475 9.16107364066221e-08
476 9.146990907416e-08
477 9.1330222542485e-08
478 9.11910923724335e-08
479 9.10527221669177e-08
480 9.09149960111044e-08
481 9.0778346751641e-08
482 9.0642305570654e-08
483 9.05070977328393e-08
484 9.03726095038415e-08
485 9.02389716639362e-08
486 9.01062057550028e-08
487 8.99738218498314e-08
488 8.98422414143951e-08
489 8.97118355620563e-08
490 8.95815538677347e-08
491 8.94520716312464e-08
492 8.93233843695107e-08
493 8.91956898070312e-08
494 8.90682730014802e-08
495 8.89412576303705e-08
496 8.88153338998166e-08
497 8.86899659375828e-08
498 8.8565881967817e-08
499 8.84420896714033e-08
500 8.83192153318113e-08
501 8.81969729822352e-08
502 8.80753334735474e-08
503 8.79545526877301e-08
504 8.78339441738873e-08
505 8.77148611215794e-08
506 8.75954202839146e-08
507 8.74770400089986e-08
508 8.73588407515058e-08
509 8.72417592003849e-08
510 8.71248840095262e-08
511 8.70086857469587e-08
512 8.68935411828531e-08
513 8.67782416822394e-08
514 8.66639999608232e-08
515 8.65495897839352e-08
516 8.64367246404818e-08
517 8.63235965553599e-08
518 8.62117218245473e-08
519 8.60997874565506e-08
520 8.59889489461807e-08
521 8.58779590413583e-08
522 8.57680227710489e-08
523 8.56582846036069e-08
524 8.55490122537717e-08
525 8.54408468207168e-08
526 8.53322556233493e-08
527 8.52249686675854e-08
528 8.51175671132687e-08
529 8.50107409537948e-08
530 8.49046864495051e-08
531 8.47984216778563e-08
532 8.46933190881138e-08
533 8.45878220161467e-08
534 8.44830798922835e-08
535 8.43784959161375e-08
536 8.42741971132099e-08
537 8.41706088205996e-08
538 8.40664318619844e-08
539 8.3962711412866e-08
540 8.38589526801492e-08
541 8.37549257184378e-08
542 8.36507587060886e-08
543 8.35467382040989e-08
544 8.34435511118947e-08
545 8.33418861176938e-08
546 8.3241547649493e-08
547 8.31419173974446e-08
548 8.30435423933018e-08
549 8.29449954693473e-08
550 8.28473218188464e-08
551 8.27498418520811e-08
552 8.26526716868337e-08
553 8.25562364275001e-08
554 8.24598023672074e-08
555 8.23644383793898e-08
556 8.22687523953647e-08
557 8.21739962080237e-08
558 8.20794616847032e-08
559 8.19853258047232e-08
560 8.18915519964492e-08
561 8.17981936207524e-08
562 8.1705386011155e-08
563 8.16133019077014e-08
564 8.1520815012226e-08
565 8.14293279849387e-08
566 8.13376645778519e-08
567 8.12468497541907e-08
568 8.11563336102772e-08
569 8.10660218770742e-08
570 8.09760665458903e-08
571 8.08866651595963e-08
572 8.07976296184343e-08
573 8.07085609397795e-08
574 8.0620297055578e-08
575 8.05320326118242e-08
576 8.04443870636717e-08
577 8.035641287929e-08
578 8.0269759818119e-08
579 8.01825398775158e-08
580 8.00962462514931e-08
581 8.00099834044055e-08
582 7.99243726143928e-08
583 7.98386210814073e-08
584 7.97533437784104e-08
585 7.96683008648102e-08
586 7.9583618359802e-08
587 7.94991671382306e-08
588 7.94150140959182e-08
589 7.93315461824484e-08
590 7.92475860635911e-08
591 7.91649278766293e-08
592 7.90816841775843e-08
593 7.89988653093765e-08
594 7.89166317956003e-08
595 7.88344267053098e-08
596 7.8752895707801e-08
597 7.86712845775028e-08
598 7.85900081639035e-08
599 7.85093511748158e-08
600 7.84288470287997e-08
601 7.83478595947074e-08
602 7.82680342426012e-08
603 7.81882078535467e-08
604 7.81090355270742e-08
605 7.80295114166663e-08
606 7.79508802803974e-08
607 7.78720065923721e-08
608 7.77934305089545e-08
609 7.77156936457857e-08
610 7.76374645305999e-08
611 7.7560174494451e-08
612 7.74826843574772e-08
613 7.74057250434446e-08
614 7.73289302955504e-08
615 7.72525184027373e-08
616 7.7176254570599e-08
617 7.71001570836205e-08
618 7.70242260301757e-08
619 7.69484753360139e-08
620 7.6873669048183e-08
621 7.67980854643469e-08
622 7.67235554972601e-08
623 7.66487052707987e-08
624 7.65747166160402e-08
625 7.6499978217015e-08
626 7.6426426680154e-08
627 7.63529506482286e-08
628 7.6279364700671e-08
629 7.62061462076247e-08
630 7.6133264267142e-08
631 7.60602367306795e-08
632 7.5987501483521e-08
633 7.59150553091281e-08
634 7.58427346401191e-08
635 7.57697307709471e-08
636 7.56966023254257e-08
637 7.56232796943479e-08
638 7.55488779891067e-08
639 7.54744831241716e-08
640 7.54037644492556e-08
641 7.533113699143e-08
642 7.52628879094885e-08
643 7.51880915639269e-08
644 7.51176810669385e-08
645 7.5041685012156e-08
646 7.49699696740436e-08
647 7.48925832341207e-08
648 7.48315271135169e-08
649 7.47603869801772e-08
650 7.4698079978841e-08
651 7.46265799769219e-08
652 7.45657954071e-08
653 7.44934722671786e-08
654 7.4433972498511e-08
655 7.43612648257397e-08
656 7.43038429122933e-08
657 7.42296500804151e-08
658 7.41740881156083e-08
659 7.40992671381946e-08
660 7.4045629655739e-08
661 7.39694333673491e-08
662 7.39173170556384e-08
663 7.38402312050468e-08
664 7.37915450272908e-08
665 7.37116364653012e-08
666 7.36649531343225e-08
667 7.35840329308957e-08
668 7.35403931364509e-08
669 7.34570039964133e-08
670 7.3415363699425e-08
671 7.33309451566022e-08
672 7.32926047266602e-08
673 7.32055826278e-08
674 7.31687516948298e-08
675 7.30814119087597e-08
676 7.30487419615677e-08
677 7.29571157087605e-08
678 7.29234904373044e-08
679 7.28363668729415e-08
680 7.28071701558974e-08
681 7.27121581163281e-08
682 7.26809145632501e-08
683 7.25948637887441e-08
684 7.25650545425616e-08
685 7.2473766221659e-08
686 7.24389180404295e-08
687 7.23578080901177e-08
688 7.23245252554072e-08
689 7.22390790937055e-08
690 7.21982975870894e-08
691 7.21253288609169e-08
692 7.20802061384695e-08
693 7.20085583560959e-08
694 7.19620949918287e-08
695 7.19017336634487e-08
696 7.18422257484086e-08
697 7.17752681596018e-08
698 7.17306831363729e-08
699 7.16680895598287e-08
700 7.16126099700176e-08
701 7.15433641329533e-08
702 7.15052975710506e-08
703 7.1428250091099e-08
704 7.13723947245448e-08
705 7.13122324560444e-08
706 7.12420132424896e-08
707 7.12005369805091e-08
708 7.11406704740902e-08
709 7.10914192643131e-08
710 7.10248450066508e-08
711 7.09838852595013e-08
712 7.09157134979321e-08
713 7.08731722922629e-08
714 7.08034741778718e-08
715 7.0761966009858e-08
716 7.06986545653265e-08
717 7.06574111482006e-08
718 7.05843869033274e-08
719 7.0543081423402e-08
720 7.04830143174462e-08
721 7.04366027202141e-08
722 7.03746958334861e-08
723 7.03311494341108e-08
724 7.02665299456839e-08
725 7.02272418804384e-08
726 7.01581094966208e-08
727 7.01207923468061e-08
728 7.00530143502576e-08
729 7.00208977288952e-08
730 6.99424716548158e-08
731 6.991601797246e-08
732 6.98387062860739e-08
733 6.98142075110653e-08
734 6.97323836527097e-08
735 6.97119703545468e-08
736 6.96278095135838e-08
737 6.96033861027345e-08
738 6.95300297408608e-08
739 6.9507385206613e-08
740 6.94217804912967e-08
741 6.93797862214574e-08
742 6.93380811234157e-08
743 6.92787177638543e-08
744 6.92368873025373e-08
745 6.91879856131727e-08
746 6.91295670738334e-08
747 6.90787675194926e-08
748 6.90335053121771e-08
749 6.897815320972e-08
750 6.89349836133779e-08
751 6.8882698908368e-08
752 6.88315276047291e-08
753 6.87822476410638e-08
754 6.87348560104439e-08
755 6.86846034012589e-08
756 6.86370034719275e-08
757 6.85859495450636e-08
758 6.85388474326309e-08
759 6.84841521780832e-08
760 6.84447249379438e-08
761 6.83890874491055e-08
762 6.834563746283e-08
763 6.82958586701865e-08
764 6.82475077415923e-08
765 6.81993396218772e-08
766 6.81515184286496e-08
767 6.81035348346626e-08
768 6.80557332284337e-08
769 6.80080127675176e-08
770 6.79605712083919e-08
771 6.79129406875312e-08
772 6.78657989543474e-08
773 6.78185885436555e-08
774 6.77714250887362e-08
775 6.77241539213114e-08
776 6.76776121797396e-08
777 6.76305017215384e-08
778 6.75838628279024e-08
779 6.7537214060831e-08
780 6.74908941116215e-08
781 6.74442941166475e-08
782 6.73980251133521e-08
783 6.73517907676668e-08
784 6.73057860480775e-08
785 6.72598889552845e-08
786 6.72138475907857e-08
787 6.71680438490974e-08
788 6.7122217686677e-08
789 6.70766865598083e-08
790 6.70312036272769e-08
791 6.69860294579827e-08
792 6.69405901772002e-08
793 6.68953538092154e-08
794 6.68499670308798e-08
795 6.6805349357324e-08
796 6.67598032997319e-08
797 6.67152957074535e-08
798 6.66703377167899e-08
799 6.66258182921986e-08
800 6.65808671289625e-08
801 6.65367031515984e-08
802 6.64913503167774e-08
803 6.64478946430513e-08
804 6.64023250536161e-08
805 6.63594382595001e-08
806 6.63115264645242e-08
807 6.62683590344493e-08
808 6.62237007493793e-08
809 6.61792710556775e-08
810 6.61373215553596e-08
811 6.60872354751874e-08
812 6.61027985056073e-08
813 6.59652264407562e-08
814 6.59755235146164e-08
815 6.59055418905652e-08
816 6.5869027191745e-08
817 6.58595750615554e-08
818 6.57667626908776e-08
819 6.57913892614381e-08
820 6.56670576253404e-08
821 6.56652577237793e-08
822 6.5607701304149e-08
823 6.55714288666509e-08
824 6.55749079561119e-08
825 6.54545850049537e-08
826 6.54462740605766e-08
827 6.53985459311279e-08
828 6.53726578043745e-08
829 6.53032553818988e-08
830 6.53297884141146e-08
831 6.52037905459757e-08
832 6.51851651798374e-08
833 6.51644517466465e-08
834 6.50867947507017e-08
835 6.51191259239958e-08
836 6.49893803998935e-08
837 6.49786882003234e-08
838 6.49352351977939e-08
839 6.48921308670225e-08
840 6.48458460368317e-08
841 6.48105491594109e-08
842 6.47662587107689e-08
843 6.47285134731668e-08
844 6.46846166492132e-08
845 6.46443298606414e-08
846 6.46047225179025e-08
847 6.45637080545214e-08
848 6.45223989534927e-08
849 6.4483901956347e-08
850 6.44382793755227e-08
851 6.43981222774315e-08
852 6.4357928754255e-08
853 6.43208974184795e-08
854 6.42765828224867e-08
855 6.42374756094632e-08
856 6.41920311381661e-08
857 6.41625586914607e-08
858 6.41452404765985e-08
859 6.40609281967741e-08
860 6.40804792189087e-08
861 6.39797337607106e-08
862 6.39675860107936e-08
863 6.39458906963597e-08
864 6.38678856965669e-08
865 6.38286147931488e-08
866 6.37939692538225e-08
867 6.38114680735491e-08
868 6.3698943812085e-08
869 6.3714711579177e-08
870 6.36250060863652e-08
871 6.3624145737684e-08
872 6.35466221776682e-08
873 6.35589849049545e-08
874 6.34697463510747e-08
875 6.34770080476521e-08
876 6.33934724691976e-08
877 6.33983498112833e-08
878 6.33146402031493e-08
879 6.33238490994614e-08
880 6.32459306491029e-08
881 6.32282947634621e-08
882 6.31808026834157e-08
883 6.31369530306181e-08
884 6.31040411418482e-08
885 6.30591789723134e-08
886 6.3026379972797e-08
887 6.29835267065282e-08
888 6.29478823235274e-08
889 6.29072266025155e-08
890 6.28707481840074e-08
891 6.28308622236951e-08
892 6.27932255508057e-08
893 6.27544300777494e-08
894 6.27160286517103e-08
895 6.26778334780198e-08
896 6.26396136937934e-08
897 6.26010603399507e-08
898 6.25625086896342e-08
899 6.25247022796316e-08
900 6.24858146194285e-08
901 6.24476479247349e-08
902 6.2408814033077e-08
903 6.23693288770433e-08
904 6.23321022210632e-08
905 6.22941738783744e-08
906 6.22562861876119e-08
907 6.22179445373128e-08
908 6.21799996594063e-08
909 6.21415076822451e-08
910 6.21031125005445e-08
911 6.20648335885043e-08
912 6.20266387101331e-08
913 6.198850362793e-08
914 6.19501082614882e-08
915 6.19122804388361e-08
916 6.18748923724155e-08
917 6.18371011968932e-08
918 6.1799099714932e-08
919 6.17617911879975e-08
920 6.172417196737e-08
921 6.16867883800332e-08
922 6.16490429274918e-08
923 6.16114477245411e-08
924 6.15748448584696e-08
925 6.15366561236286e-08
926 6.14995005339125e-08
927 6.14618292695823e-08
928 6.14252444224306e-08
929 6.13881353470624e-08
930 6.13514258822612e-08
931 6.13138100691302e-08
932 6.12742467867022e-08
933 6.1240612982516e-08
934 6.11980431619585e-08
935 6.1167718695021e-08
936 6.11234822751605e-08
937 6.1094500591885e-08
938 6.1045387433456e-08
939 6.10213242331348e-08
940 6.0979170116493e-08
941 6.09364174346538e-08
942 6.09089153860154e-08
943 6.08666958865633e-08
944 6.08323339870154e-08
945 6.07867475372892e-08
946 6.07624065569468e-08
947 6.07162296089037e-08
948 6.06847827926416e-08
949 6.06453549201191e-08
950 6.06091939143738e-08
951 6.05727264648692e-08
952 6.05353096401196e-08
953 6.05005649205737e-08
954 6.04604670604303e-08
955 6.04271930035871e-08
956 6.03874311284613e-08
957 6.03523400251049e-08
958 6.03147752351596e-08
959 6.02779979801404e-08
960 6.02414941290874e-08
961 6.02042133555081e-08
962 6.0168958596396e-08
963 6.01300929301551e-08
964 6.00955957921201e-08
965 6.00557393970469e-08
966 6.00227520655139e-08
967 5.99820815390117e-08
968 5.99492789326028e-08
969 5.99089461386093e-08
970 5.98753785445716e-08
971 5.98355267737993e-08
972 5.98019734967536e-08
973 5.97627206380125e-08
974 5.972777912433e-08
975 5.96902598570814e-08
976 5.96533915779851e-08
977 5.96175077274275e-08
978 5.95793020146118e-08
979 5.95457600955918e-08
980 5.95045899602908e-08
981 5.94744897557398e-08
982 5.94293950766378e-08
983 5.94037108694323e-08
984 5.93539534454735e-08
985 5.93333760980919e-08
986 5.92786023871916e-08
987 5.92624314506018e-08
988 5.92041329663395e-08
989 5.91908514353534e-08
990 5.91300231120861e-08
991 5.91186529472765e-08
992 5.90569033396271e-08
993 5.90466205458995e-08
994 5.89830926567281e-08
995 5.8974761198094e-08
996 5.89098128931376e-08
997 5.89024700548002e-08
998 5.88370771756885e-08
999 5.88299121879743e-08
1000 5.87645602179165e-08
1001 5.87582024595967e-08
1002 5.86917908846907e-08
1003 5.86856381725731e-08
1004 5.8618887524009e-08
1005 5.86134358813162e-08
1006 5.85464602038677e-08
1007 5.85412067732882e-08
1008 5.84743431621959e-08
1009 5.84690416154388e-08
1010 5.84020687015574e-08
1011 5.83969849756727e-08
1012 5.83300470937687e-08
1013 5.83247734193826e-08
1014 5.82579794001781e-08
1015 5.82524512271476e-08
1016 5.81859537787111e-08
1017 5.81807830117853e-08
1018 5.81139598332392e-08
1019 5.81087374076361e-08
1020 5.8042053175722e-08
1021 5.80369549521009e-08
1022 5.79700561944563e-08
1023 5.79650921062047e-08
1024 5.78981551258018e-08
1025 5.78933429724593e-08
1026 5.78265900625929e-08
1027 5.78216454303337e-08
1028 5.7754964684964e-08
1029 5.77501489860133e-08
1030 5.76829398295509e-08
1031 5.7678958467644e-08
1032 5.76115103219088e-08
1033 5.76073213447437e-08
1034 5.75395872774997e-08
1035 5.75358477563626e-08
1036 5.74678187033051e-08
1037 5.74648496081132e-08
1038 5.73963951118195e-08
1039 5.73936416654597e-08
1040 5.73245345898421e-08
1041 5.73226527280646e-08
1042 5.72532475535681e-08
1043 5.72515849279753e-08
1044 5.71819599248791e-08
1045 5.71805378077883e-08
1046 5.71108603080184e-08
1047 5.7109570692937e-08
1048 5.70392999725833e-08
1049 5.70391089862454e-08
1050 5.69680744697543e-08
1051 5.6968563046933e-08
1052 5.68971139935925e-08
1053 5.68976937760368e-08
1054 5.68259544273531e-08
1055 5.68271377705543e-08
1056 5.67551737948868e-08
1057 5.67569993750183e-08
1058 5.66840727085349e-08
1059 5.66865453559551e-08
1060 5.66137377000508e-08
1061 5.66161331327919e-08
1062 5.65431409795814e-08
1063 5.65459416046465e-08
1064 5.64724763369995e-08
1065 5.64762339481106e-08
1066 5.64019379454272e-08
1067 5.64060769456809e-08
1068 5.63314579036245e-08
1069 5.63365290959794e-08
1070 5.62614540990936e-08
1071 5.62664313887851e-08
1072 5.61915691905668e-08
1073 5.61972859434157e-08
1074 5.61217062919894e-08
1075 5.61281923703305e-08
1076 5.60535383549166e-08
1077 5.6070229374594e-08
1078 5.59767223808905e-08
1079 5.59938432576068e-08
1080 5.59174644796379e-08
1081 5.59243831754763e-08
1082 5.58465882138925e-08
1083 5.58567388617703e-08
1084 5.57706676338476e-08
1085 5.5786829757043e-08
1086 5.57038722144299e-08
1087 5.57211145504333e-08
1088 5.5628176666378e-08
1089 5.56545130785402e-08
1090 5.556337536472e-08
1091 5.55852866388484e-08
1092 5.54909704701956e-08
1093 5.55169684730572e-08
1094 5.54271509587423e-08
1095 5.5445987270808e-08
1096 5.53596022192515e-08
1097 5.5376890836456e-08
1098 5.52900805019796e-08
1099 5.53094726152281e-08
1100 5.52195216898532e-08
1101 5.52440447387781e-08
1102 5.51524999226061e-08
1103 5.51725561144245e-08
1104 5.50891129229569e-08
1105 5.51103777759465e-08
1106 5.50114762338794e-08
1107 5.50453992409494e-08
1108 5.49440219232977e-08
1109 5.49771538915778e-08
1110 5.48702744151974e-08
1111 5.49169135850569e-08
1112 5.48112369211928e-08
1113 5.48363159640353e-08
1114 5.47398344736827e-08
1115 5.4766154692576e-08
1116 5.46784471735506e-08
1117 5.47105243144941e-08
1118 5.46043966664911e-08
1119 5.46315956069243e-08
1120 5.45438212822269e-08
1121 5.45783630596652e-08
1122 5.44617726694341e-08
1123 5.45121840489493e-08
1124 5.44097313541769e-08
1125 5.44311277050902e-08
1126 5.43388647940013e-08
1127 5.43783082576077e-08
1128 5.42590946741583e-08
1129 5.43137762361035e-08
1130 5.42092507922476e-08
1131 5.42310121596223e-08
1132 5.41406553180579e-08
1133 5.4183187906176e-08
1134 5.40577337249104e-08
1135 5.41115474632647e-08
1136 5.39925898954685e-08
1137 5.40465514435873e-08
1138 5.39361299627394e-08
1139 5.39769885832264e-08
1140 5.38685025137653e-08
1141 5.39133870502972e-08
1142 5.38099926172819e-08
1143 5.38396237508287e-08
1144 5.37427148148417e-08
1145 5.37944549656721e-08
1146 5.36627753047192e-08
1147 5.37207462478762e-08
1148 5.35977773776786e-08
1149 5.36563414987867e-08
1150 5.35396974150615e-08
1151 5.35941722543676e-08
1152 5.34799355120796e-08
1153 5.35304893722888e-08
1154 5.34255399480266e-08
1155 5.34547545765207e-08
1156 5.3347649563662e-08
1157 5.33946388179274e-08
1158 5.32803531534398e-08
1159 5.33400883839796e-08
1160 5.32027084956255e-08
1161 5.32749348529649e-08
1162 5.31508477550346e-08
1163 5.31976099229503e-08
1164 5.30846993012091e-08
1165 5.31584396608054e-08
1166 5.30122576889447e-08
1167 5.30831198712889e-08
1168 5.29630248866653e-08
1169 5.30166753458516e-08
1170 5.28821673260183e-08
1171 5.29749368349819e-08
1172 5.28118708311709e-08
1173 5.28879115333503e-08
1174 5.2750636601484e-08
1175 5.28518264846234e-08
1176 5.26845720445479e-08
1177 5.27607147620124e-08
1178 5.2617867742466e-08
1179 5.27013165498857e-08
1180 5.25609268358096e-08
1181 5.26582740287829e-08
1182 5.24958777439188e-08
1183 5.25672593600923e-08
1184 5.24270437227692e-08
1185 5.25066842631539e-08
1186 5.23645310317455e-08
1187 5.24466492084485e-08
1188 5.23062013177977e-08
1189 5.23780213286251e-08
1190 5.22463370082882e-08
1191 5.23174692910189e-08
1192 5.21781851281489e-08
1193 5.22554420374632e-08
1194 5.21082111695215e-08
1195 5.21935115513905e-08
1196 5.2049870953752e-08
1197 5.21299624041127e-08
1198 5.19871158952689e-08
1199 5.20684927276527e-08
1200 5.19244648664596e-08
1201 5.20002960509025e-08
1202 5.18682196091191e-08
1203 5.19438167776443e-08
1204 5.1800323665141e-08
1205 5.18895714618495e-08
1206 5.17489269249438e-08
1207 5.18005447895931e-08
1208 5.17092617986492e-08
1209 5.17570763998165e-08
1210 5.16383955582178e-08
1211 5.16946701427656e-08
1212 5.15765192479201e-08
1213 5.16323268469066e-08
1214 5.15168721042514e-08
1215 5.1570681826707e-08
1216 5.14568660863191e-08
1217 5.15082231666852e-08
1218 5.13972490954195e-08
1219 5.14467116814465e-08
1220 5.13375691966189e-08
1221 5.13848014231932e-08
1222 5.12780233670185e-08
1223 5.1323428518657e-08
1224 5.1218356854843e-08
1225 5.12621326738127e-08
1226 5.11579171238452e-08
1227 5.12019407685749e-08
1228 5.1096053654387e-08
1229 5.11420775075599e-08
1230 5.10328080767231e-08
1231 5.10860309548278e-08
1232 5.0969875536655e-08
1233 5.10290262454305e-08
1234 5.091262077217e-08
1235 5.09640543451262e-08
1236 5.08513024235491e-08
1237 5.09066783807022e-08
1238 5.07963161573421e-08
1239 5.08438564006397e-08
1240 5.073669806821e-08
1241 5.07803693308162e-08
1242 5.06744911970003e-08
1243 5.07342733953031e-08
1244 5.06082097375682e-08
1245 5.06773020205742e-08
1246 5.0555776183181e-08
1247 5.06085169740267e-08
1248 5.04966653203098e-08
1249 5.05488799316112e-08
1250 5.04483784276033e-08
1251 5.04767628890512e-08
1252 5.03873718256109e-08
1253 5.04317036083002e-08
1254 5.03304581092046e-08
1255 5.03605652673045e-08
1256 5.02675065363611e-08
1257 5.03174303188025e-08
1258 5.02189252014063e-08
1259 5.0239762362736e-08
1260 5.01593164283776e-08
1261 5.01951979900994e-08
1262 5.01017018970096e-08
1263 5.01270133566045e-08
1264 5.00408010610798e-08
1265 5.00823755373858e-08
1266 4.99907672786293e-08
1267 5.00079829821587e-08
1268 4.99272790484007e-08
1269 4.99674307588016e-08
1270 4.9879504105288e-08
1271 4.9890937024788e-08
1272 4.98195487264042e-08
1273 4.98474386487757e-08
1274 4.97670370798176e-08
1275 4.97779010188992e-08
1276 4.97026736758777e-08
1277 4.97378118868852e-08
1278 4.96564999523663e-08
1279 4.96602940933677e-08
1280 4.96039633408607e-08
1281 4.96038479917971e-08
1282 4.95434084366586e-08
1283 4.95583590267579e-08
1284 4.94901370635681e-08
1285 4.94940458746207e-08
1286 4.94411506797299e-08
1287 4.94259244714357e-08
1288 4.93926772051623e-08
1289 4.93695476402323e-08
1290 4.93434225474765e-08
1291 4.93115294228907e-08
1292 4.92868322501394e-08
1293 4.92570482393262e-08
1294 4.92347072365007e-08
1295 4.92015931485845e-08
1296 4.91775509816605e-08
1297 4.91447255410726e-08
1298 4.91285170545197e-08
1299 4.90927832403898e-08
1300 4.90610804879665e-08
1301 4.90555934042902e-08
1302 4.90008965980948e-08
1303 4.89940382353282e-08
1304 4.89435013735218e-08
1305 4.89539181045551e-08
1306 4.88929110513681e-08
1307 4.88764237536188e-08
1308 4.88648026104599e-08
1309 4.88193769001555e-08
1310 4.87886112323643e-08
1311 4.87934032444848e-08
1312 4.87322528788248e-08
1313 4.87155514279358e-08
1314 4.87058304021737e-08
1315 4.86579752041472e-08
1316 4.86316570236234e-08
1317 4.86319261514545e-08
1318 4.85740550093539e-08
1319 4.85555944860216e-08
1320 4.85490195227811e-08
1321 4.84976407197912e-08
1322 4.84756604253178e-08
1323 4.84725853366896e-08
1324 4.84179950022146e-08
1325 4.83972310361125e-08
1326 4.83934000037323e-08
1327 4.83396440738026e-08
1328 4.83220607532253e-08
1329 4.83150548098443e-08
1330 4.82624568540757e-08
1331 4.8240981639136e-08
1332 4.82386104203592e-08
1333 4.81836864798524e-08
1334 4.8169548467758e-08
1335 4.81577545095035e-08
1336 4.8109395061946e-08
1337 4.80866232090982e-08
1338 4.80848639292653e-08
1339 4.80296532492375e-08
1340 4.80174250667709e-08
1341 4.80025773050841e-08
1342 4.79577100755968e-08
1343 4.79334966478895e-08
1344 4.79319022912428e-08
1345 4.78773549428269e-08
1346 4.78669815597854e-08
1347 4.78486043120618e-08
1348 4.78073012413205e-08
1349 4.77822006850204e-08
1350 4.77790863095962e-08
1351 4.77270329182211e-08
1352 4.77158975042968e-08
1353 4.76977082590757e-08
1354 4.76574325527501e-08
1355 4.76328590615083e-08
1356 4.76248861840389e-08
1357 4.75764593939054e-08
1358 4.7562676812607e-08
1359 4.75532887707075e-08
1360 4.75038740987976e-08
1361 4.74878063645434e-08
1362 4.74684187490304e-08
1363 4.7439548862549e-08
1364 4.74148864757495e-08
1365 4.737942744093e-08
1366 4.7388177849772e-08
1367 4.73273525574136e-08
1368 4.73302842265433e-08
1369 4.72784394163206e-08
1370 4.72908410529271e-08
1371 4.72282933543511e-08
1372 4.7235427123038e-08
1373 4.71807699882199e-08
1374 4.71965375807848e-08
1375 4.71281740952456e-08
1376 4.71416679435066e-08
1377 4.70842722943843e-08
1378 4.70977987929722e-08
1379 4.70338864864139e-08
1380 4.70457022805348e-08
1381 4.69898019552772e-08
1382 4.69970648069307e-08
1383 4.69423748410414e-08
1384 4.69481742886302e-08
1385 4.6894731449143e-08
1386 4.68995105076253e-08
1387 4.68486146454339e-08
1388 4.68526011943027e-08
1389 4.68012424903463e-08
1390 4.68000211535369e-08
1391 4.67539819744012e-08
1392 4.67531910777197e-08
1393 4.67083149611192e-08
1394 4.66956140585673e-08
1395 4.66662699780684e-08
1396 4.66561739020932e-08
1397 4.66159025966917e-08
1398 4.66077387093478e-08
1399 4.65835048082397e-08
1400 4.65488089287369e-08
1401 4.65334044146282e-08
1402 4.65197272538909e-08
1403 4.64721301991489e-08
1404 4.64739442014306e-08
1405 4.64264724979735e-08
1406 4.6423897196668e-08
1407 4.63862186301789e-08
1408 4.63742450884652e-08
1409 4.63459946500322e-08
1410 4.6334455602981e-08
1411 4.6286139511853e-08
1412 4.62913743639604e-08
1413 4.62537276000319e-08
1414 4.62382497969926e-08
1415 4.62039374711232e-08
1416 4.62085275629853e-08
1417 4.61575937520387e-08
1418 4.61446136288401e-08
1419 4.61057109428786e-08
1420 4.61127371309544e-08
1421 4.60751319812935e-08
1422 4.60393927053104e-08
1423 4.60280409666503e-08
1424 4.6021515506478e-08
1425 4.59790559310669e-08
1426 4.59715333440691e-08
1427 4.59262255669834e-08
1428 4.59398960233059e-08
1429 4.58793318531825e-08
1430 4.58876425044608e-08
1431 4.58393545232916e-08
1432 4.58321574048348e-08
1433 4.58028922416531e-08
1434 4.57861674187932e-08
1435 4.57514883165366e-08
1436 4.57593955864866e-08
1437 4.57042808412744e-08
1438 4.56922170659624e-08
1439 4.56698973771985e-08
1440 4.56605738676785e-08
1441 4.56218701940792e-08
1442 4.56226429705886e-08
1443 4.55734480975245e-08
1444 4.55715162996917e-08
1445 4.5535947587716e-08
1446 4.55208256386896e-08
1447 4.54883713851828e-08
1448 4.5490213570254e-08
1449 4.54431723331261e-08
1450 4.54394741868924e-08
1451 4.54019388129723e-08
1452 4.54037914239258e-08
1453 4.53555024759744e-08
1454 4.53608355317492e-08
1455 4.53124948518457e-08
1456 4.53091551149143e-08
1457 4.52706403426539e-08
1458 4.52598765598999e-08
1459 4.52285361118854e-08
1460 4.52283563912026e-08
1461 4.51837262698263e-08
1462 4.51874692966037e-08
1463 4.51385627391332e-08
1464 4.51312560816319e-08
1465 4.51044713503101e-08
1466 4.50601300769904e-08
1467 4.50987139308801e-08
1468 4.50109930349996e-08
1469 4.50567808294444e-08
1470 4.4961595603521e-08
1471 4.49719170445562e-08
1472 4.49417206187519e-08
1473 4.49394010115789e-08
1474 4.48913157851116e-08
1475 4.48987850942828e-08
1476 4.48604119003271e-08
1477 4.48530056593732e-08
1478 4.48087651037099e-08
1479 4.48473208565225e-08
1480 4.47459067851597e-08
1481 4.47720561558285e-08
1482 4.47252705000345e-08
1483 4.47284181142038e-08
1484 4.46796268458449e-08
1485 4.46827489613e-08
1486 4.46761261749806e-08
1487 4.46259297266494e-08
1488 4.46373077895856e-08
1489 4.45847864694038e-08
1490 4.45767578995948e-08
1491 4.45503597523711e-08
1492 4.45119900391866e-08
1493 4.45586755715155e-08
1494 4.44540736093124e-08
1495 4.44983884042571e-08
1496 4.44231799145989e-08
1497 4.44598700548582e-08
1498 4.43748224285834e-08
1499 4.43948684054618e-08
1500 4.43498825939415e-08
1501 4.43551247233387e-08
1502 4.43099267606328e-08
1503 4.43147311539605e-08
1504 4.4263838398173e-08
1505 4.42754513607646e-08
1506 4.42257402895052e-08
1507 4.42323222542562e-08
1508 4.41838933444849e-08
1509 4.41904478685196e-08
1510 4.41452086219485e-08
1511 4.41526004557602e-08
1512 4.4103404551965e-08
1513 4.41091524745474e-08
1514 4.40613905392517e-08
1515 4.40679554181145e-08
1516 4.40218367749878e-08
1517 4.40304525728408e-08
1518 4.39848982671798e-08
1519 4.39844487560848e-08
1520 4.39414153423634e-08
1521 4.39476620459622e-08
1522 4.39034603409993e-08
1523 4.39044942051048e-08
1524 4.38643482305601e-08
1525 4.38640606774676e-08
1526 4.3825097781891e-08
1527 4.38236729825015e-08
1528 4.37854312114183e-08
1529 4.37828039050459e-08
1530 4.37460937763312e-08
1531 4.3741063830538e-08
1532 4.3706794567111e-08
1533 4.37008713043419e-08
1534 4.36671925534426e-08
1535 4.36601128543934e-08
1536 4.36275253208329e-08
1537 4.36198041784408e-08
1538 4.35876302211646e-08
1539 4.35798433353618e-08
1540 4.3548364163648e-08
1541 4.35395161582086e-08
1542 4.35088213297519e-08
1543 4.34994391174115e-08
1544 4.34697016635788e-08
1545 4.34588642055367e-08
1546 4.34302315635016e-08
1547 4.34196129655007e-08
1548 4.33902316960832e-08
1549 4.33797605481345e-08
1550 4.33515437543441e-08
1551 4.33394765115835e-08
1552 4.33123004719427e-08
1553 4.32996165677935e-08
1554 4.32736100766107e-08
1555 4.32596374735361e-08
1556 4.32346572121212e-08
1557 4.32197395885403e-08
1558 4.31956890358798e-08
1559 4.31801758171702e-08
1560 4.31563442058191e-08
1561 4.31411037062901e-08
1562 4.31178914270802e-08
1563 4.31010839783141e-08
1564 4.30791913146322e-08
1565 4.30614966249188e-08
1566 4.30400717648105e-08
1567 4.30221928486851e-08
1568 4.30020127804021e-08
1569 4.29827991905718e-08
1570 4.29628789300196e-08
1571 4.29437234688024e-08
1572 4.29243724702211e-08
1573 4.29051744186282e-08
1574 4.28853788854155e-08
1575 4.28662882798747e-08
1576 4.28469051545477e-08
1577 4.2826931243134e-08
1578 4.28083829140391e-08
1579 4.27884617661967e-08
1580 4.2769894172956e-08
1581 4.27495791512911e-08
1582 4.27316883122586e-08
1583 4.27105937759187e-08
1584 4.26932250956469e-08
1585 4.26712179910638e-08
1586 4.2656577594169e-08
1587 4.26311184802408e-08
1588 4.26183027526505e-08
1589 4.2592826607013e-08
1590 4.25802949934884e-08
1591 4.25534787993342e-08
1592 4.25427038117654e-08
1593 4.25157249486219e-08
1594 4.25045805827473e-08
1595 4.24759598915969e-08
1596 4.24671269105836e-08
1597 4.24387768815926e-08
1598 4.24286042899347e-08
1599 4.23985628454915e-08
1600 4.23917925491679e-08
1601 4.23618939939807e-08
1602 4.23531885207495e-08
1603 4.23226645116515e-08
1604 4.23162714673353e-08
1605 4.22845415233297e-08
1606 4.22781856097565e-08
1607 4.2245756138648e-08
1608 4.22413570184688e-08
1609 4.2206611871709e-08
1610 4.22047354668997e-08
1611 4.21663620620016e-08
1612 4.21688720444457e-08
1613 4.21265666292747e-08
1614 4.21311637173183e-08
1615 4.20929299256301e-08
1616 4.20928477273819e-08
1617 4.20509480467679e-08
1618 4.20558790841241e-08
1619 4.20163859047307e-08
1620 4.20185869751677e-08
1621 4.19747877327126e-08
1622 4.19777281890354e-08
1623 4.19568804832515e-08
1624 4.19188970686157e-08
1625 4.19250206062927e-08
1626 4.18884188202462e-08
1627 4.18852875481157e-08
1628 4.18440960441302e-08
1629 4.18502291115175e-08
1630 4.18124647199036e-08
1631 4.18123278631555e-08
1632 4.17690743561572e-08
1633 4.17743561076378e-08
1634 4.17457323909076e-08
1635 4.1730280924579e-08
1636 4.17015818805488e-08
1637 4.17001293695485e-08
1638 4.16583810289861e-08
1639 4.16649613268483e-08
1640 4.16232812494588e-08
1641 4.16286176907477e-08
1642 4.1586054226217e-08
1643 4.15905581352405e-08
1644 4.15476034221207e-08
1645 4.15540107894241e-08
1646 4.15117948542321e-08
1647 4.15176615145008e-08
1648 4.14756252056225e-08
1649 4.14798804164729e-08
1650 4.14378211079924e-08
1651 4.14434871349734e-08
1652 4.14014001706153e-08
1653 4.14064905875922e-08
1654 4.13652155428768e-08
1655 4.1369052141027e-08
1656 4.13280423521289e-08
1657 4.13324841903595e-08
1658 4.12918703291965e-08
1659 4.12952122923471e-08
1660 4.12554576207924e-08
1661 4.12579555548742e-08
1662 4.12192406529233e-08
1663 4.12208942148773e-08
1664 4.11821776327148e-08
1665 4.1184160731067e-08
1666 4.11458827700439e-08
1667 4.11465331739969e-08
1668 4.11088548759597e-08
1669 4.1110508399278e-08
1670 4.10737002767281e-08
1671 4.10720257106867e-08
1672 4.10357093916236e-08
1673 4.10376458903627e-08
1674 4.10027840602734e-08
1675 4.09977504922665e-08
1676 4.09629081745688e-08
1677 4.09661820601759e-08
1678 4.0933237853702e-08
1679 4.09222057471403e-08
1680 4.08926474289029e-08
1681 4.08933060371819e-08
1682 4.08632897705985e-08
1683 4.08435057142054e-08
1684 4.08355563910057e-08
1685 4.08012523671619e-08
1686 4.08038901389407e-08
1687 4.07720326331962e-08
1688 4.07564523734472e-08
1689 4.07413470828732e-08
1690 4.07141223772722e-08
1691 4.0713989158947e-08
1692 4.06851245060569e-08
1693 4.06574113689384e-08
1694 4.06625090290103e-08
1695 4.06300304993668e-08
1696 4.06071078826731e-08
1697 4.06063812181667e-08
1698 4.05771066842586e-08
1699 4.05511872338948e-08
1700 4.0554449172836e-08
1701 4.05227134796515e-08
1702 4.05008946540164e-08
1703 4.04983664974434e-08
1704 4.04697954379252e-08
1705 4.04452880629069e-08
1706 4.04471098154335e-08
1707 4.04149257644537e-08
1708 4.03968679407729e-08
1709 4.03893447660231e-08
1710 4.03632271734899e-08
1711 4.03396298600534e-08
1712 4.03398388952869e-08
1713 4.03082994253623e-08
1714 4.02906612066989e-08
1715 4.0282912954881e-08
1716 4.0257237396979e-08
1717 4.02336944214099e-08
1718 4.02332800282235e-08
1719 4.02019681911003e-08
1720 4.01857420309071e-08
1721 4.01745549125643e-08
1722 4.01510676952821e-08
1723 4.0130676782546e-08
1724 4.01252067339097e-08
1725 4.00979469390439e-08
1726 4.00765811525705e-08
1727 4.00728157698982e-08
1728 4.0044109004933e-08
1729 4.00252577050786e-08
1730 4.001888763705e-08
1731 3.99928273302486e-08
1732 3.99715079661966e-08
1733 3.99667410437843e-08
1734 3.99402377886382e-08
1735 3.991960116001e-08
1736 3.99133559778608e-08
1737 3.98886860704106e-08
1738 3.98682918762994e-08
1739 3.98576766302838e-08
1740 3.98347916374942e-08
1741 3.98166775437136e-08
1742 3.98064596069947e-08
1743 3.97848028952374e-08
1744 3.97656507005184e-08
1745 3.97472025537304e-08
1746 3.9735143619879e-08
1747 3.97140970531673e-08
1748 3.9694408041635e-08
1749 3.96829870619175e-08
1750 3.96624027882986e-08
1751 3.96434932072864e-08
1752 3.96280997576603e-08
1753 3.96094752499465e-08
1754 3.95930653036558e-08
1755 3.95753414641487e-08
1756 3.95583665033783e-08
1757 3.95405004554039e-08
1758 3.95238680892529e-08
1759 3.95058713538177e-08
1760 3.94895324482558e-08
1761 3.94720184679898e-08
1762 3.9454681423079e-08
1763 3.94373476784171e-08
1764 3.94206619807047e-08
1765 3.94033484063527e-08
1766 3.93869600137098e-08
1767 3.93695454277143e-08
1768 3.93535043086146e-08
1769 3.93336150199541e-08
1770 3.93193035079786e-08
1771 3.92989518176456e-08
1772 3.92848253025768e-08
1773 3.92642350912631e-08
1774 3.92509789644002e-08
1775 3.92292136746075e-08
1776 3.92171191236912e-08
1777 3.91947741560372e-08
1778 3.91830445272134e-08
1779 3.91590588764146e-08
1780 3.91498362251053e-08
1781 3.91250522262521e-08
1782 3.91160127091172e-08
1783 3.90889145556184e-08
1784 3.90825538068018e-08
1785 3.90554933034082e-08
1786 3.90492166513212e-08
1787 3.90191408552454e-08
1788 3.90167034594047e-08
1789 3.89845368162778e-08
1790 3.89834487666274e-08
1791 3.89536212384023e-08
1792 3.8947737520445e-08
1793 3.8928681919348e-08
1794 3.89208745312786e-08
1795 3.89015057242759e-08
1796 3.88868947858168e-08
1797 3.88682985821287e-08
1798 3.88522851275663e-08
1799 3.88344444808286e-08
1800 3.88182609778465e-08
1801 3.88005797369306e-08
1802 3.87844535945092e-08
1803 3.87668660366547e-08
1804 3.87502098952996e-08
1805 3.87328156192801e-08
1806 3.87163573825067e-08
1807 3.86993483163511e-08
1808 3.86823404971981e-08
1809 3.86652157600142e-08
1810 3.86487942187763e-08
1811 3.8631492614849e-08
1812 3.86148685480592e-08
1813 3.8598168773607e-08
1814 3.85807940657124e-08
1815 3.85643990894469e-08
1816 3.85470568635693e-08
1817 3.85302743184379e-08
1818 3.85136053056012e-08
1819 3.84967236932709e-08
1820 3.84797307326767e-08
1821 3.84633817114288e-08
1822 3.84461818578874e-08
1823 3.84296327704536e-08
1824 3.84125464008012e-08
1825 3.83960135872385e-08
1826 3.83790956668406e-08
1827 3.83622963013863e-08
1828 3.83454888954748e-08
1829 3.83287836658308e-08
1830 3.83118439148955e-08
1831 3.82952579387474e-08
1832 3.8278130610303e-08
1833 3.82619174219556e-08
1834 3.8245159183159e-08
1835 3.82284322524296e-08
1836 3.82114954811108e-08
1837 3.81947927581283e-08
1838 3.81778827849288e-08
1839 3.81618029390296e-08
1840 3.81448160478026e-08
1841 3.81279438477211e-08
1842 3.8111576277533e-08
1843 3.80948794627134e-08
1844 3.80784201576834e-08
1845 3.80613215786862e-08
1846 3.80451492363321e-08
1847 3.80280592280346e-08
1848 3.80121673020373e-08
1849 3.79949345057362e-08
1850 3.79786538469151e-08
1851 3.79622068418239e-08
1852 3.79456687016333e-08
1853 3.79292374725892e-08
1854 3.79127673129087e-08
1855 3.78964252625291e-08
1856 3.7879782486705e-08
1857 3.78633452771115e-08
1858 3.78470260702368e-08
1859 3.78305189758787e-08
1860 3.78145890818082e-08
1861 3.77974003906711e-08
1862 3.7781445010765e-08
1863 3.77650300726895e-08
1864 3.77484397808825e-08
1865 3.77320044164797e-08
1866 3.77163332128116e-08
1867 3.76991775892233e-08
1868 3.7683182643633e-08
1869 3.76668250330336e-08
1870 3.76504195791494e-08
1871 3.76338224685746e-08
1872 3.76175744341012e-08
1873 3.76014477045938e-08
1874 3.75846593680951e-08
1875 3.75688516678352e-08
1876 3.75519505562139e-08
1877 3.75360267772518e-08
1878 3.75194820028124e-08
1879 3.75036855844169e-08
1880 3.7486656027319e-08
1881 3.74706033714478e-08
1882 3.74545708130558e-08
1883 3.74383280614676e-08
1884 3.7421821546646e-08
1885 3.74058272591959e-08
1886 3.73892847229662e-08
1887 3.73733158005596e-08
1888 3.73568829008519e-08
1889 3.73404285571866e-08
1890 3.73244373772508e-08
1891 3.73084079499098e-08
1892 3.72917068902634e-08
1893 3.72758579088028e-08
1894 3.72595707274215e-08
1895 3.72433111324177e-08
1896 3.72267279062921e-08
1897 3.72111371784722e-08
1898 3.7194622541703e-08
1899 3.71785681718695e-08
1900 3.71627362365512e-08
1901 3.7145860642962e-08
1902 3.71301255774892e-08
1903 3.71138372476931e-08
1904 3.70975485568525e-08
1905 3.70813459531494e-08
1906 3.70651815631007e-08
1907 3.70488414471737e-08
1908 3.70329278926995e-08
1909 3.70166986769505e-08
1910 3.70004804377544e-08
1911 3.69842407705878e-08
1912 3.69682826388829e-08
1913 3.69516903171441e-08
1914 3.69361846983551e-08
1915 3.69193670419765e-08
1916 3.69038788292642e-08
1917 3.68872501603779e-08
1918 3.68714058960329e-08
1919 3.68552365483943e-08
1920 3.68392817629015e-08
1921 3.68229349667626e-08
1922 3.68069734062448e-08
1923 3.67904952678355e-08
1924 3.67747177913991e-08
1925 3.67586305993761e-08
1926 3.67425796905518e-08
1927 3.67261572047362e-08
1928 3.67103039431438e-08
1929 3.66941772353968e-08
1930 3.66784968170997e-08
1931 3.66619572726989e-08
1932 3.66463854883925e-08
1933 3.66297878144906e-08
1934 3.66136666924977e-08
1935 3.65981548420269e-08
1936 3.65818263188267e-08
1937 3.65656976035744e-08
1938 3.65496201752968e-08
1939 3.65339811219112e-08
1940 3.65177402579242e-08
1941 3.65020736698973e-08
1942 3.64857191417212e-08
1943 3.64702653996574e-08
1944 3.64535223669193e-08
1945 3.64388200762011e-08
1946 3.64216691985941e-08
1947 3.64067281262148e-08
1948 3.63900266009409e-08
1949 3.6374488993296e-08
1950 3.63588484442179e-08
1951 3.63426198621841e-08
1952 3.6326849724766e-08
1953 3.63116289729959e-08
1954 3.62951751389229e-08
1955 3.6280310764969e-08
1956 3.62640715592111e-08
1957 3.62500485227013e-08
1958 3.62332914225494e-08
1959 3.6219743266841e-08
1960 3.62028614697696e-08
1961 3.61911993278508e-08
1962 3.61727323581196e-08
1963 3.61623019273782e-08
1964 3.61428036916855e-08
1965 3.61332685543392e-08
1966 3.61116788178961e-08
1967 3.61042313992144e-08
1968 3.6079588566551e-08
1969 3.60746039482773e-08
1970 3.60464723752063e-08
1971 3.60456899415329e-08
1972 3.60129485099669e-08
1973 3.60159398702375e-08
1974 3.59801722880082e-08
1975 3.59855670297726e-08
1976 3.59476125604274e-08
1977 3.5954862937615e-08
1978 3.59160183307505e-08
1979 3.59234726006186e-08
1980 3.58844838659333e-08
1981 3.58923190213467e-08
1982 3.58527880881532e-08
1983 3.58612568054362e-08
1984 3.58215124918182e-08
1985 3.58297006779473e-08
1986 3.57901511116587e-08
1987 3.57987601433685e-08
1988 3.57581078742619e-08
1989 3.57674359257043e-08
1990 3.57263539312314e-08
1991 3.57369617740577e-08
1992 3.56948925408496e-08
1993 3.57054097530618e-08
1994 3.56634064062611e-08
1995 3.5674562960164e-08
1996 3.56317812213458e-08
1997 3.56431469914487e-08
1998 3.56006997048741e-08
1999 3.56118450146603e-08
2000 3.55693559273007e-08
2001 3.55808882339659e-08
2002 3.55376354428394e-08
2003 3.55500781212825e-08
2004 3.55063143693268e-08
2005 3.55189260059507e-08
2006 3.54746834538844e-08
2007 3.54879751300885e-08
2008 3.54436646983203e-08
2009 3.54566523816935e-08
2010 3.54121087184911e-08
2011 3.54259066495111e-08
2012 3.5381434308146e-08
2013 3.53947386730891e-08
2014 3.53497508982503e-08
2015 3.53638507866183e-08
2016 3.53186744388445e-08
2017 3.53328160151989e-08
2018 3.52874139954995e-08
2019 3.53020898189449e-08
2020 3.52564252035226e-08
2021 3.52704896182665e-08
2022 3.52255868665274e-08
2023 3.52397508172064e-08
2024 3.51944843259844e-08
2025 3.52085694750315e-08
2026 3.51637910085412e-08
2027 3.51775470845972e-08
2028 3.51326006826635e-08
2029 3.51467499573044e-08
2030 3.51018323889729e-08
2031 3.51156686757559e-08
2032 3.50713009400483e-08
2033 3.50846368570856e-08
2034 3.5040151943555e-08
2035 3.50539678131145e-08
2036 3.50093346663805e-08
2037 3.50229760774923e-08
2038 3.49784779072326e-08
2039 3.49923832432264e-08
2040 3.49477809356014e-08
2041 3.49610436198855e-08
2042 3.49172474363169e-08
2043 3.49301566575644e-08
2044 3.48869999902135e-08
2045 3.48991961691691e-08
2046 3.48564117986783e-08
2047 3.48683807798178e-08
2048 3.48256320488183e-08
2049 3.48378952044204e-08
2050 3.47951741499486e-08
2051 3.48069722888589e-08
2052 3.4764378432417e-08
2053 3.4776031978323e-08
2054 3.47344008877482e-08
2055 3.47449087569895e-08
2056 3.47042907424822e-08
2057 3.47140476331109e-08
2058 3.46738170919902e-08
2059 3.46839893872186e-08
2060 3.46431790994028e-08
2061 3.46529749608582e-08
2062 3.46133492044043e-08
2063 3.46224374463411e-08
2064 3.45828307146689e-08
2065 3.45916502226817e-08
2066 3.45525609424247e-08
2067 3.4560693410679e-08
2068 3.45227402505088e-08
2069 3.45302625481825e-08
2070 3.44920990615893e-08
2071 3.45000008901142e-08
2072 3.44619279422087e-08
2073 3.44697708936081e-08
2074 3.44321818277482e-08
2075 3.44385290826921e-08
2076 3.4402404791356e-08
2077 3.44077485217031e-08
2078 3.43726865865701e-08
2079 3.43769765027702e-08
2080 3.43427721074541e-08
2081 3.43470112409427e-08
2082 3.43125592123794e-08
2083 3.43165333009932e-08
2084 3.42830690915896e-08
2085 3.42860241813181e-08
2086 3.42535067143768e-08
2087 3.42554424350716e-08
2088 3.42239714523629e-08
2089 3.42248964666503e-08
2090 3.41941135946211e-08
2091 3.41951129292362e-08
2092 3.41644125503304e-08
2093 3.41643571193373e-08
2094 3.4135214746378e-08
2095 3.41336461802122e-08
2096 3.4105874438195e-08
2097 3.41037915898568e-08
2098 3.40763509651953e-08
2099 3.40735833743722e-08
2100 3.40469353450334e-08
2101 3.4043416772489e-08
2102 3.40173577548786e-08
2103 3.40132596816645e-08
2104 3.39888932325838e-08
2105 3.3982560563528e-08
2106 3.39593389591109e-08
2107 3.39529249777648e-08
2108 3.39298234262042e-08
2109 3.39228553065674e-08
2110 3.3900805862519e-08
2111 3.38930145595828e-08
2112 3.38715653096511e-08
2113 3.38630495051451e-08
2114 3.38423220500594e-08
2115 3.38333308322891e-08
2116 3.38130444497153e-08
2117 3.38036104190476e-08
2118 3.37838740589458e-08
2119 3.37738081175321e-08
2120 3.37547684552408e-08
2121 3.37440625686192e-08
2122 3.37258531202256e-08
2123 3.37149726312713e-08
2124 3.36967454044323e-08
2125 3.36851997897458e-08
2126 3.36677204322289e-08
2127 3.3655899406515e-08
2128 3.36384809935808e-08
2129 3.36265481013509e-08
2130 3.36099103928067e-08
2131 3.35972670779672e-08
2132 3.35810565932349e-08
2133 3.35680711429642e-08
2134 3.35520081589102e-08
2135 3.35389505754513e-08
2136 3.35228709644753e-08
2137 3.35103684141469e-08
2138 3.34942159299079e-08
2139 3.34810607267588e-08
2140 3.34654229698916e-08
2141 3.34530130265964e-08
2142 3.34372817549777e-08
2143 3.34218478925674e-08
2144 3.34062541238467e-08
2145 3.33970202510692e-08
2146 3.33795632183609e-08
2147 3.33645402683658e-08
2148 3.33489775659057e-08
2149 3.33399654359745e-08
2150 3.33230237090643e-08
2151 3.33053901429015e-08
2152 3.32911106624145e-08
2153 3.32819394095463e-08
2154 3.32648897376497e-08
2155 3.3250377231564e-08
2156 3.32347995497528e-08
2157 3.32260529110062e-08
2158 3.32105739122568e-08
2159 3.31890011731062e-08
2160 3.31837678517743e-08
2161 3.31647124058865e-08
2162 3.31503046646642e-08
2163 3.31333611798268e-08
2164 3.31280486320917e-08
2165 3.31096807069109e-08
2166 3.30901966110186e-08
2167 3.30831826158562e-08
2168 3.30659205560746e-08
2169 3.30505612902154e-08
2170 3.30346000660953e-08
2171 3.30285388261409e-08
2172 3.30108231052506e-08
2173 3.2990456208859e-08
2174 3.29859512395725e-08
2175 3.29679712678388e-08
2176 3.29482877190479e-08
2177 3.2943279469011e-08
2178 3.2925212474666e-08
2179 3.2905712386011e-08
2180 3.28995003477939e-08
2181 3.28819136248271e-08
2182 3.28660424120386e-08
2183 3.28508756259893e-08
2184 3.2845025778272e-08
2185 3.28270029674726e-08
2186 3.28065690478052e-08
2187 3.28030986513905e-08
2188 3.27849934076419e-08
2189 3.27641353246033e-08
2190 3.27615244808399e-08
2191 3.27422640153951e-08
2192 3.27227948211384e-08
2193 3.27185102173555e-08
2194 3.26998730948436e-08
2195 3.26802935290971e-08
2196 3.26768325755733e-08
2197 3.26577650462401e-08
2198 3.26384663293933e-08
2199 3.26346206427353e-08
2200 3.26155147343243e-08
2201 3.25966640892794e-08
2202 3.25930105367078e-08
2203 3.25732587866945e-08
2204 3.25550940571961e-08
2205 3.25507318579277e-08
2206 3.25315637617063e-08
2207 3.25134406384819e-08
2208 3.25087951782432e-08
2209 3.2489991394602e-08
2210 3.24717420243648e-08
2211 3.24671819933453e-08
2212 3.24481009568167e-08
2213 3.24303908554313e-08
2214 3.24251559817856e-08
2215 3.24062685335136e-08
2216 3.23889834945579e-08
2217 3.23825922146082e-08
2218 3.23634689198826e-08
2219 3.23532619146416e-08
2220 3.23367161871868e-08
2221 3.23232828016717e-08
2222 3.23065802647626e-08
2223 3.23003056947879e-08
2224 3.22799408847274e-08
2225 3.2271743841239e-08
2226 3.22538460495458e-08
2227 3.22388235864945e-08
2228 3.22270946628844e-08
2229 3.22101946299558e-08
2230 3.22036626403399e-08
2231 3.21844206290223e-08
2232 3.21736626665015e-08
2233 3.2157154176371e-08
2234 3.21482273573093e-08
2235 3.21316884508427e-08
2236 3.21141302939143e-08
2237 3.21081987721961e-08
2238 3.20878449573403e-08
2239 3.20800125526155e-08
2240 3.20626112930711e-08
2241 3.20476315918139e-08
2242 3.20336791561626e-08
2243 3.20253345071908e-08
2244 3.20054282183513e-08
2245 3.1998932426669e-08
2246 3.19797079821882e-08
2247 3.19693723076497e-08
2248 3.19512807314126e-08
2249 3.1943665359746e-08
2250 3.19228438323549e-08
2251 3.19146810108251e-08
2252 3.18998200692366e-08
2253 3.18835068031653e-08
2254 3.1874639329077e-08
2255 3.18561400776129e-08
2256 3.18493445028434e-08
2257 3.18291122267667e-08
2258 3.18221203225999e-08
2259 3.18022296623699e-08
2260 3.17950340755768e-08
2261 3.17755431176181e-08
2262 3.17665466849082e-08
2263 3.17496677766371e-08
2264 3.17413360033658e-08
2265 3.17222188688238e-08
2266 3.171446530148e-08
2267 3.16949288996504e-08
2268 3.16855054052212e-08
2269 3.16722664097657e-08
2270 3.16552279537863e-08
2271 3.16451009916285e-08
2272 3.16331953764681e-08
2273 3.16148720147513e-08
2274 3.16053620319501e-08
2275 3.1593233288385e-08
2276 3.15750751633814e-08
2277 3.15651881623857e-08
2278 3.15533302639448e-08
2279 3.15350864037445e-08
2280 3.15254000753384e-08
2281 3.1513251830706e-08
2282 3.14956747657913e-08
2283 3.14856823679932e-08
2284 3.14737020401523e-08
2285 3.14567897501394e-08
2286 3.14462161252749e-08
2287 3.14340580103156e-08
2288 3.14170340733888e-08
2289 3.14067897417836e-08
2290 3.13943662102467e-08
2291 3.13780060134938e-08
2292 3.13677270202817e-08
2293 3.13543557470819e-08
2294 3.13395439004438e-08
2295 3.13288641642373e-08
2296 3.13136270460035e-08
2297 3.13026147993334e-08
2298 3.12879404547051e-08
2299 3.12763640115676e-08
2300 3.12620623001969e-08
2301 3.12499816375045e-08
2302 3.12364062056147e-08
2303 3.12237519710656e-08
2304 3.12107353370728e-08
2305 3.11978459117679e-08
2306 3.11849112923923e-08
2307 3.11715754182096e-08
2308 3.11590693826691e-08
2309 3.1145688470291e-08
2310 3.11334660560991e-08
2311 3.11198401417023e-08
2312 3.11069793563767e-08
2313 3.10933950997683e-08
2314 3.10803555347849e-08
2315 3.10676476662497e-08
2316 3.10547864077471e-08
2317 3.10417844218147e-08
2318 3.10292583156624e-08
2319 3.10162490333443e-08
2320 3.10038045201466e-08
2321 3.09910012337156e-08
2322 3.09780332106158e-08
2323 3.09658474189067e-08
2324 3.09524554498175e-08
2325 3.09400553142325e-08
2326 3.09273405219024e-08
2327 3.09145923316212e-08
2328 3.09018812507667e-08
2329 3.08890034210751e-08
2330 3.08766463057442e-08
2331 3.08636793522332e-08
2332 3.08516508895629e-08
2333 3.08382542422159e-08
2334 3.08260629731105e-08
2335 3.08134589586118e-08
2336 3.08008528895343e-08
2337 3.07884211814269e-08
2338 3.07753769979158e-08
2339 3.07633883103176e-08
2340 3.07507580468158e-08
2341 3.07383330087063e-08
2342 3.07250794164826e-08
2343 3.07133686514582e-08
2344 3.07000706962768e-08
2345 3.06885765701814e-08
2346 3.06750166156888e-08
2347 3.0663565303346e-08
2348 3.06502792950525e-08
2349 3.06389788962136e-08
2350 3.06250134982644e-08
2351 3.06142167667556e-08
2352 3.06000374277193e-08
2353 3.05895534340461e-08
2354 3.05758392395994e-08
2355 3.05649453895551e-08
2356 3.05503165427545e-08
2357 3.05406423417587e-08
2358 3.05254428576429e-08
2359 3.05166652099231e-08
2360 3.05005429175331e-08
2361 3.04923813949642e-08
2362 3.04754499216298e-08
2363 3.04687885348542e-08
2364 3.04508063495579e-08
2365 3.04455859345687e-08
2366 3.04250406630846e-08
2367 3.04242425850365e-08
2368 3.039925810433e-08
2369 3.04011764620249e-08
2370 3.03742131217977e-08
2371 3.03762130591245e-08
2372 3.03504249778985e-08
2373 3.03542661315159e-08
2374 3.03251977216146e-08
2375 3.0325254677166e-08
2376 3.0304202558451e-08
2377 3.02993753962255e-08
2378 3.02802506360056e-08
2379 3.0275746686792e-08
2380 3.02559662967994e-08
2381 3.02558762479421e-08
2382 3.0229441674523e-08
2383 3.02342778251141e-08
2384 3.02052885303716e-08
2385 3.02076502605164e-08
2386 3.01824272750384e-08
2387 3.01871797017128e-08
2388 3.01578109096479e-08
2389 3.01564094427054e-08
2390 3.01423357975672e-08
2391 3.01265254980532e-08
2392 3.01227761956291e-08
2393 3.01016485411143e-08
2394 3.01004361582535e-08
2395 3.00767906140642e-08
2396 3.00781763742375e-08
2397 3.00521989502478e-08
2398 3.00545101130645e-08
2399 3.00282750316239e-08
2400 3.00327840023762e-08
2401 3.00041661669415e-08
2402 3.00057584026181e-08
2403 2.99816900615202e-08
2404 2.99838492203719e-08
2405 2.99570902129176e-08
2406 2.99632653737447e-08
2407 2.99336672571204e-08
2408 2.99369958547402e-08
2409 2.99113363986248e-08
2410 2.99150118854286e-08
2411 2.98875789799258e-08
2412 2.9892383431207e-08
2413 2.98643617175909e-08
2414 2.98689038802369e-08
2415 2.98410891084178e-08
2416 2.98447397444335e-08
2417 2.9819002103082e-08
2418 2.9820616746612e-08
2419 2.97965642113152e-08
2420 2.97966872653266e-08
2421 2.97744787161047e-08
2422 2.97729335934083e-08
2423 2.97522320715782e-08
2424 2.97495823038396e-08
2425 2.97296332107955e-08
2426 2.97261283335182e-08
2427 2.97071156740714e-08
2428 2.97028718549974e-08
2429 2.96850311762853e-08
2430 2.96793925771421e-08
2431 2.96632285714526e-08
2432 2.96556139940396e-08
2433 2.9641514326384e-08
2434 2.96326019848525e-08
2435 2.9619294367178e-08
2436 2.96095798189011e-08
2437 2.95975224864353e-08
2438 2.95871335580511e-08
2439 2.95750282204299e-08
2440 2.95644346557378e-08
2441 2.95525122711027e-08
2442 2.954243192832e-08
2443 2.95305721349504e-08
2444 2.95203990205994e-08
2445 2.95093045683537e-08
2446 2.95106626355679e-08
2447 2.95095946307722e-08
2448 2.94882599769508e-08
2449 2.94870065196129e-08
2450 2.94619954330777e-08
2451 2.9470801587772e-08
2452 2.94365206134906e-08
2453 2.94467053767455e-08
2454 2.94115896828373e-08
2455 2.94300684691162e-08
2456 2.93891363345455e-08
2457 2.94037641843659e-08
2458 2.93665508019902e-08
2459 2.93867510765544e-08
2460 2.93454845885499e-08
2461 2.9359192158962e-08
2462 2.93238427462317e-08
2463 2.93427507744859e-08
2464 2.9303017617055e-08
2465 2.93147320278297e-08
2466 2.92813446225448e-08
2467 2.92988803387573e-08
2468 2.92605281260183e-08
2469 2.92701988369171e-08
2470 2.92386730911254e-08
2471 2.92554624261587e-08
2472 2.92180287579935e-08
2473 2.92271668269617e-08
2474 2.91962136809154e-08
2475 2.92128701582239e-08
2476 2.91751035290666e-08
2477 2.91845822220083e-08
2478 2.91538187535689e-08
2479 2.91704046255781e-08
2480 2.91330063029172e-08
2481 2.91411560366939e-08
2482 2.91119262634254e-08
2483 2.91273732069275e-08
2484 2.90910646343345e-08
2485 2.90989493796356e-08
2486 2.90702222365269e-08
2487 2.90856147249219e-08
2488 2.90492882943916e-08
2489 2.90561669635814e-08
2490 2.90290564755047e-08
2491 2.90424930766697e-08
2492 2.90074366227078e-08
2493 2.9015482328365e-08
2494 2.89867436984448e-08
2495 2.90021766735649e-08
2496 2.89668199140447e-08
2497 2.89718012262963e-08
2498 2.89471604220104e-08
2499 2.89571174574288e-08
2500 2.89238788506729e-08
2501 2.89360629830426e-08
2502 2.89035532368143e-08
2503 2.89182695762946e-08
2504 2.88843094009561e-08
2505 2.88917168347336e-08
2506 2.88632055620575e-08
2507 2.8879347102384e-08
2508 2.88450664742701e-08
2509 2.88466401057175e-08
2510 2.88272329336259e-08
2511 2.88281516283018e-08
2512 2.88032933348781e-08
2513 2.88177220419961e-08
2514 2.87824326696828e-08
2515 2.8788142902636e-08
2516 2.87636761979737e-08
2517 2.87755899084363e-08
2518 2.87419999334126e-08
2519 2.87506601581722e-08
2520 2.87217327414879e-08
2521 2.87389827064999e-08
2522 2.87047219078573e-08
2523 2.87039155610902e-08
2524 2.86900092152376e-08
2525 2.86839653225179e-08
2526 2.86712981314441e-08
2527 2.866354936093e-08
2528 2.86556659259141e-08
2529 2.86413070604841e-08
2530 2.86406665843675e-08
2531 2.86229101416069e-08
2532 2.86100364297326e-08
2533 2.86058724359162e-08
2534 2.85886503810229e-08
2535 2.85902865682086e-08
2536 2.85635893393366e-08
2537 2.8579951644403e-08
2538 2.85451358006394e-08
2539 2.85494591869551e-08
2540 2.85283220839982e-08
2541 2.85366786605401e-08
2542 2.85030883264703e-08
2543 2.85196657687781e-08
2544 2.84877095040326e-08
2545 2.84919809085959e-08
2546 2.84671747783349e-08
2547 2.84850506484524e-08
2548 2.8451194808099e-08
2549 2.84462714887912e-08
2550 2.84415724283171e-08
2551 2.8429063915425e-08
2552 2.84193436448366e-08
2553 2.84070988438856e-08
2554 2.84052722447914e-08
2555 2.8390562585523e-08
2556 2.83627528658847e-08
2557 2.83731227956885e-08
2558 2.83515730763462e-08
2559 2.83645390382059e-08
2560 2.83284879740719e-08
2561 2.83360818953415e-08
2562 2.83115391361832e-08
2563 2.83243800627275e-08
2564 2.82876422892375e-08
2565 2.830859438685e-08
2566 2.8270823324128e-08
2567 2.82847703036992e-08
2568 2.82519016430172e-08
2569 2.82702548535241e-08
2570 2.82326410390166e-08
2571 2.82508506634915e-08
2572 2.82103303832493e-08
2573 2.82330198682068e-08
2574 2.81880157915193e-08
2575 2.81914245006121e-08
2576 2.81711770413473e-08
2577 2.81746571633157e-08
2578 2.81689584002987e-08
2579 2.81568193638559e-08
2580 2.8155759959958e-08
2581 2.81297705457817e-08
2582 2.81409220601603e-08
2583 2.8109905812812e-08
2584 2.81245213082926e-08
2585 2.80885653138885e-08
2586 2.8111706743994e-08
2587 2.80621019594918e-08
2588 2.81060821250012e-08
2589 2.80327536075209e-08
2590 2.80907441680966e-08
2591 2.80258406619716e-08
2592 2.80669846706161e-08
2593 2.80065812963137e-08
2594 2.80462563235773e-08
2595 2.7989659384664e-08
2596 2.80379207964199e-08
2597 2.79631818920212e-08
2598 2.80217941763805e-08
2599 2.79568130423513e-08
2600 2.79995169290004e-08
2601 2.79330879968676e-08
2602 2.79827456686199e-08
2603 2.79195560168866e-08
2604 2.79686089934561e-08
2605 2.79002646403903e-08
2606 2.79406110670344e-08
2607 2.78949097241021e-08
2608 2.79195338581673e-08
2609 2.78782396383459e-08
2610 2.79053603440271e-08
2611 2.78609278099307e-08
2612 2.78873845793726e-08
2613 2.7845209013222e-08
2614 2.78701124758474e-08
2615 2.78165119873552e-08
2616 2.78608173671646e-08
2617 2.78072805044438e-08
2618 2.78366960435594e-08
2619 2.77927355010199e-08
2620 2.78163095077666e-08
2621 2.77747712433829e-08
2622 2.77977494906878e-08
2623 2.77572436560014e-08
2624 2.7795520185947e-08
2625 2.77320524353364e-08
2626 2.77704720874628e-08
2627 2.77219604614132e-08
2628 2.77644441313285e-08
2629 2.76962895400246e-08
2630 2.77277232314521e-08
2631 2.77024371038159e-08
2632 2.77191024582457e-08
2633 2.76712065552243e-08
2634 2.77087101130391e-08
2635 2.764685220491e-08
2636 2.76888098806882e-08
2637 2.76393169751366e-08
2638 2.76818706477222e-08
2639 2.76162161043647e-08
2640 2.76439393498418e-08
2641 2.76128110072715e-08
2642 2.76294893293283e-08
2643 2.76005925530232e-08
2644 2.76199600504068e-08
2645 2.75789261083847e-08
2646 2.76069790363653e-08
2647 2.75480056561861e-08
2648 2.75924353869694e-08
2649 2.75473637174617e-08
2650 2.7572702911316e-08
2651 2.75320359630626e-08
2652 2.75524800819049e-08
2653 2.75144610981659e-08
2654 2.75390300612521e-08
2655 2.74900510808251e-08
2656 2.75340161646209e-08
2657 2.7481480875835e-08
2658 2.75030607328475e-08
2659 2.74657622361119e-08
2660 2.74868418614904e-08
2661 2.74419146157889e-08
2662 2.74811924705354e-08
2663 2.74349145050756e-08
2664 2.74575857404091e-08
2665 2.74109446518622e-08
2666 2.74349779243455e-08
2667 2.74165971043683e-08
2668 2.7424891937633e-08
2669 2.73912951129773e-08
2670 2.74194515768045e-08
2671 2.7356962711611e-08
2672 2.73948791686962e-08
2673 2.73449619299093e-08
2674 2.73641886632259e-08
2675 2.73693377037532e-08
2676 2.73483167030353e-08
2677 2.73206691343564e-08
2678 2.73403718531462e-08
2679 2.72921300932261e-08
2680 2.73810244546535e-08
2681 2.72680283517346e-08
2682 2.72864758690261e-08
2683 2.72891508685813e-08
2684 2.73042431866433e-08
2685 2.72370780656228e-08
2686 2.7340074350013e-08
2687 2.72258999418629e-08
2688 2.72773066050114e-08
2689 2.72202389666187e-08
2690 2.72597220229986e-08
2691 2.72309887743294e-08
2692 2.72464012061047e-08
2693 2.72018552591735e-08
2694 2.72358079642654e-08
2695 2.71767161075243e-08
2696 2.72144427391208e-08
2697 2.71676917060759e-08
2698 2.71768452342336e-08
2699 2.71946248144861e-08
2700 2.71733112684469e-08
2701 2.71485158602758e-08
2702 2.71701988912021e-08
2703 2.71399749716128e-08
2704 2.71437979926237e-08
2705 2.71221430308e-08
2706 2.71532531634922e-08
2707 2.70903686281354e-08
2708 2.71231008319717e-08
2709 2.70895580885e-08
2710 2.71066962176381e-08
2711 2.70731315004102e-08
2712 2.70919961402605e-08
2713 2.70472744741124e-08
2714 2.70744868056116e-08
2715 2.70289250885458e-08
2716 2.70859968449777e-08
2717 2.70038163445729e-08
2718 2.70401695869893e-08
2719 2.70320731283746e-08
2720 2.70395778461108e-08
2721 2.6998334793582e-08
2722 2.70226012426544e-08
2723 2.70011053056241e-08
2724 2.69928914740092e-08
2725 2.69628510376485e-08
2726 2.6978255645016e-08
2727 2.69399064467546e-08
2728 2.7014498000355e-08
2729 2.69111756734386e-08
2730 2.69910485306557e-08
2731 2.6901103798771e-08
2732 2.6945255196642e-08
2733 2.68917187655759e-08
2734 2.69200313305351e-08
2735 2.68963115124254e-08
2736 2.69146292435352e-08
2737 2.68910842984305e-08
2738 2.68967084156024e-08
2739 2.68880503773161e-08
2740 2.68895276389625e-08
2741 2.68299167123232e-08
2742 2.6868352241749e-08
2743 2.6846075069864e-08
2744 2.68434144943264e-08
2745 2.68582939582096e-08
2746 2.68387445419016e-08
2747 2.682720934577e-08
2748 2.68209958673715e-08
2749 2.67857257303383e-08
2750 2.68002182339799e-08
2751 2.68075210381191e-08
2752 2.67904865780544e-08
2753 2.67354692224941e-08
2754 2.68144352426614e-08
2755 2.67392493571439e-08
2756 2.67786235890188e-08
2757 2.6717358385886e-08
2758 2.67637718189828e-08
2759 2.6732846851063e-08
2760 2.6749284720351e-08
2761 2.67164866625258e-08
2762 2.67610778921235e-08
2763 2.66746672639151e-08
2764 2.6727749354416e-08
2765 2.66504539707668e-08
2766 2.66943705473643e-08
2767 2.66910016686328e-08
2768 2.66689821788457e-08
2769 2.66999867604145e-08
2770 2.66399081239044e-08
2771 2.66518176166297e-08
2772 2.66446495598149e-08
2773 2.66601869336469e-08
2774 2.6613958907129e-08
2775 2.6640263583344e-08
2776 2.66107942541449e-08
2777 2.6614779510048e-08
2778 2.66244708486685e-08
2779 2.65935957459718e-08
2780 2.66276527189913e-08
2781 2.65355115791621e-08
2782 2.65875501430646e-08
2783 2.65410720525772e-08
2784 2.66283510019782e-08
2785 2.65074410452648e-08
2786 2.65502698975073e-08
2787 2.65530057352859e-08
2788 2.65240950043655e-08
2789 2.65563953101289e-08
2790 2.65160012529186e-08
2791 2.65121109919164e-08
2792 2.6549442507795e-08
2793 2.64376180674208e-08
2794 2.65578554903279e-08
2795 2.64667890366166e-08
2796 2.64641274043687e-08
2797 2.64548012578469e-08
2798 2.65041939566668e-08
2799 2.64109524139577e-08
2800 2.65091977664778e-08
2801 2.64325551075917e-08
2802 2.64798549496348e-08
2803 2.63583652133637e-08
2804 2.64673452048392e-08
2805 2.64240737615218e-08
2806 2.64071084479234e-08
2807 2.64190161962663e-08
2808 2.63882538600324e-08
2809 2.64105350615829e-08
2810 2.63904714525864e-08
2811 2.6369901518164e-08
2812 2.6402868026576e-08
2813 2.63208153823413e-08
2814 2.638830605739e-08
2815 2.63018834711914e-08
2816 2.63851509376689e-08
2817 2.63312589798836e-08
2818 2.63587171429602e-08
2819 2.63071194182007e-08
2820 2.63521659902288e-08
2821 2.62933044179459e-08
2822 2.6338951156335e-08
2823 2.62451039239053e-08
2824 2.63191453226952e-08
2825 2.62831262800578e-08
2826 2.63216311906511e-08
2827 2.62481073682874e-08
2828 2.63062643264256e-08
2829 2.62338403942408e-08
2830 2.62941802062766e-08
2831 2.61842674973423e-08
2832 2.62905284154069e-08
2833 2.6201446310381e-08
2834 2.62652558804533e-08
2835 2.61838494790556e-08
2836 2.62513533741959e-08
2837 2.61932154219657e-08
2838 2.61941601409355e-08
2839 2.62123693763883e-08
2840 2.61882787178092e-08
2841 2.61632708407067e-08
2842 2.61865618833568e-08
2843 2.61678777615515e-08
2844 2.61767155878001e-08
2845 2.61551040234842e-08
2846 2.61690798488701e-08
2847 2.61236061012138e-08
2848 2.6168265264026e-08
2849 2.60579722362042e-08
2850 2.61701037624906e-08
2851 2.60478664755315e-08
2852 2.61422043599246e-08
2853 2.60555855229327e-08
2854 2.61633684894846e-08
2855 2.60402685006156e-08
2856 2.60672018401742e-08
2857 2.6080699242037e-08
2858 2.61017600329261e-08
2859 2.60310256934293e-08
2860 2.60771539390969e-08
2861 2.60367783160298e-08
2862 2.60487056100711e-08
2863 2.60398625002711e-08
2864 2.6032119051278e-08
2865 2.60386819643799e-08
2866 2.59961738418024e-08
2867 2.60475608229171e-08
2868 2.59752500022525e-08
2869 2.60266574285684e-08
2870 2.59539513307594e-08
2871 2.60337091391172e-08
2872 2.59182699091642e-08
2873 2.6012414494847e-08
2874 2.59616648436989e-08
2875 2.59958507948799e-08
2876 2.59076587947327e-08
2877 2.59719204334097e-08
2878 2.59358773877505e-08
2879 2.59535267108646e-08
2880 2.59300981011457e-08
2881 2.59489435920912e-08
2882 2.58752475565949e-08
2883 2.59199746530658e-08
2884 2.5889795428391e-08
2885 2.58852657819553e-08
2886 2.58608832288676e-08
2887 2.5899636125537e-08
2888 2.58487552966535e-08
2889 2.58847026028963e-08
2890 2.58135142070515e-08
2891 2.58670579469378e-08
2892 2.58660977883185e-08
2893 2.58311297824942e-08
2894 2.58393206036267e-08
2895 2.58148613243581e-08
2896 2.58493485145728e-08
2897 2.58021241736905e-08
2898 2.58457694553016e-08
2899 2.5795924509131e-08
2900 2.57993852539329e-08
2901 2.57726713013273e-08
2902 2.58100025274377e-08
2903 2.57639811516164e-08
2904 2.57734624744543e-08
2905 2.57768455895757e-08
2906 2.57368693223903e-08
2907 2.57956913087831e-08
2908 2.57271921184632e-08
2909 2.57148811404484e-08
2910 2.57176099522916e-08
2911 2.57307479702984e-08
2912 2.57016605664795e-08
2913 2.57015423250628e-08
2914 2.57349801802143e-08
2915 2.56964576772489e-08
2916 2.57062976884992e-08
2917 2.56543325902747e-08
2918 2.57252277000575e-08
2919 2.56395517801433e-08
2920 2.56666133175987e-08
2921 2.56812020913433e-08
2922 2.56426921079189e-08
2923 2.56153747062449e-08
2924 2.56851154518678e-08
2925 2.56259963444805e-08
2926 2.55965148139303e-08
2927 2.56129906679448e-08
2928 2.5674651631169e-08
2929 2.55574845644535e-08
2930 2.55750919142539e-08
2931 2.55763007981358e-08
2932 2.55990490762859e-08
2933 2.55593217517447e-08
2934 2.55691900290511e-08
2935 2.56304152095321e-08
2936 2.55362070524789e-08
2937 2.5558577821494e-08
2938 2.55303288654307e-08
2939 2.5610982378188e-08
2940 2.54756700706071e-08
2941 2.55382610376831e-08
2942 2.55182794153264e-08
2943 2.55542738605286e-08
2944 2.54955339080709e-08
2945 2.55204245966922e-08
2946 2.55186219035863e-08
2947 2.54568523405663e-08
2948 2.55481498026011e-08
2949 2.54551971525796e-08
2950 2.54415585776435e-08
2951 2.54449046779293e-08
2952 2.55190894997703e-08
2953 2.54191804811565e-08
2954 2.54563595372126e-08
2955 2.54266225812039e-08
2956 2.55045970412038e-08
2957 2.53635782474504e-08
2958 2.54437274707087e-08
2959 2.54176584941757e-08
2960 2.54114311237785e-08
2961 2.53508275289693e-08
2962 2.54410068656341e-08
2963 2.53570531096869e-08
2964 2.5415842356491e-08
2965 2.53353185377669e-08
2966 2.54094237368552e-08
2967 2.53115843142915e-08
2968 2.53460039127962e-08
2969 2.53153158418229e-08
2970 2.53662732863091e-08
2971 2.52611973070582e-08
2972 2.52805519100896e-08
2973 2.52643311424716e-08
2974 2.53221309562957e-08
2975 2.52853627613625e-08
2976 2.52767738500026e-08
2977 2.52684730464026e-08
2978 2.53130692913217e-08
2979 2.52244910976529e-08
2980 2.53009916939551e-08
2981 2.5210830142619e-08
2982 2.52774836115854e-08
2983 2.52036740591155e-08
2984 2.53160995591362e-08
2985 2.51650550806559e-08
2986 2.5237202848194e-08
2987 2.52290208917483e-08
2988 2.51919243259291e-08
2989 2.51963798432175e-08
2990 2.52183727758037e-08
2991 2.5211726386587e-08
2992 2.52094121988566e-08
2993 2.51415432370727e-08
2994 2.52401797922186e-08
2995 2.51347146908998e-08
2996 2.51699101088132e-08
2997 2.51517831555415e-08
2998 2.52255123784906e-08
2999 2.50713985892048e-08
3000 1.42652812678834e-08
3001 1.42493403834931e-08
3002 1.43494691830504e-08
3003 1.44023437193413e-08
3004 1.44244363923796e-08
3005 1.44301384256279e-08
3006 1.44299896961964e-08
3007 1.44282206694102e-08
3008 1.44260869582857e-08
3009 1.44239917330979e-08
3010 1.44219172331789e-08
3011 1.44199225644709e-08
3012 1.4418033570246e-08
3013 1.44162485058857e-08
3014 1.44144342648506e-08
3015 1.44127135805056e-08
3016 1.44110217542431e-08
3017 1.4409390590886e-08
3018 1.44078045587609e-08
3019 1.44062084260765e-08
3020 1.44046741930443e-08
3021 1.44031491633168e-08
3022 1.44016493604376e-08
3023 1.44001593640752e-08
3024 1.4398699338794e-08
3025 1.43972336830861e-08
3026 1.43957911149101e-08
3027 1.43943817097203e-08
3028 1.43929731884623e-08
3029 1.43915761993685e-08
3030 1.43901805638447e-08
3031 1.43887937503473e-08
3032 1.43874555301737e-08
3033 1.4386084105325e-08
3034 1.43847693491739e-08
3035 1.43833907327168e-08
3036 1.438208444976e-08
3037 1.43807495664588e-08
3038 1.43794511204831e-08
3039 1.43781136038718e-08
3040 1.43768171646796e-08
3041 1.43755055039135e-08
3042 1.43742354486442e-08
3043 1.43729234128032e-08
3044 1.43716399799987e-08
3045 1.43703513496851e-08
3046 1.43690913944061e-08
3047 1.43678074983056e-08
3048 1.43665515945079e-08
3049 1.43652897563462e-08
3050 1.43640109326465e-08
3051 1.43627475855251e-08
3052 1.43615139549802e-08
3053 1.43602680064137e-08
3054 1.43590202764665e-08
3055 1.43577781313575e-08
3056 1.43565465551276e-08
3057 1.43553438260585e-08
3058 1.43540909531975e-08
3059 1.43528610337118e-08
3060 1.43516257723464e-08
3061 1.43504285173013e-08
3062 1.43491876424262e-08
3063 1.43479817843184e-08
3064 1.43467872861652e-08
3065 1.43455745039295e-08
3066 1.43443625355705e-08
3067 1.43431941131561e-08
3068 1.43419632951253e-08
3069 1.43407709556481e-08
3070 1.43396045917815e-08
3071 1.43383689381743e-08
3072 1.43372008422071e-08
3073 1.43360094838063e-08
3074 1.43348295646162e-08
3075 1.43336562424407e-08
3076 1.4332480790108e-08
3077 1.43312713613009e-08
3078 1.43301011395203e-08
3079 1.43289141757569e-08
3080 1.43277360638017e-08
3081 1.43266052321372e-08
3082 1.43254158378597e-08
3083 1.43242813488154e-08
3084 1.43231055189097e-08
3085 1.43219310988624e-08
3086 1.43207629042519e-08
3087 1.43196004309815e-08
3088 1.43184448242739e-08
3089 1.43172847032608e-08
3090 1.43161615029086e-08
3091 1.4314988260461e-08
3092 1.4313846671235e-08
3093 1.43126607791699e-08
3094 1.43115228101592e-08
3095 1.43103977400111e-08
3096 1.43092537471246e-08
3097 1.43081120133198e-08
3098 1.43069633681825e-08
3099 1.43058259113871e-08
3100 1.43046974083183e-08
3101 1.43035732558527e-08
3102 1.43024149713011e-08
3103 1.43013008545578e-08
3104 1.43001688681227e-08
3105 1.42990473576826e-08
3106 1.4297903830604e-08
3107 1.42967926129306e-08
3108 1.42956563309732e-08
3109 1.42945185707261e-08
3110 1.4293445976199e-08
3111 1.4292277686484e-08
3112 1.42911800181666e-08
3113 1.42900590218847e-08
3114 1.42889373299787e-08
3115 1.42878546673664e-08
3116 1.42867094307664e-08
3117 1.42856288975063e-08
3118 1.42845000523639e-08
3119 1.428338117343e-08
3120 1.42822846833923e-08
3121 1.42811713548519e-08
3122 1.42800558088074e-08
3123 1.42789455942344e-08
3124 1.42778759918694e-08
3125 1.42767721750536e-08
3126 1.42756409265699e-08
3127 1.42745661722565e-08
3128 1.42734448027593e-08
3129 1.42723686775009e-08
3130 1.42712388185029e-08
3131 1.42701797878619e-08
3132 1.42690841770377e-08
3133 1.42679859283373e-08
3134 1.42669184549776e-08
3135 1.42658083172043e-08
3136 1.42647220040398e-08
3137 1.42636350805442e-08
3138 1.42625679202535e-08
3139 1.42615012596603e-08
3140 1.42603936634095e-08
3141 1.42593099899391e-08
3142 1.42582341663283e-08
3143 1.42571475217068e-08
3144 1.42560315004314e-08
3145 1.42550169363875e-08
3146 1.42538835698203e-08
3147 1.42528283166993e-08
3148 1.42517703177192e-08
3149 1.42506803747916e-08
3150 1.42495992095093e-08
3151 1.42485638949114e-08
3152 1.42474556613093e-08
3153 1.42463676274657e-08
3154 1.42453255779912e-08
3155 1.42442380735852e-08
3156 1.42431927574876e-08
3157 1.42421174815499e-08
3158 1.42410545959093e-08
3159 1.42399682076105e-08
3160 1.42389357454531e-08
3161 1.42378821794548e-08
3162 1.42367927011278e-08
3163 1.42357431531653e-08
3164 1.42346789208936e-08
3165 1.42336061074166e-08
3166 1.42325694621331e-08
3167 1.42315107202334e-08
3168 1.42304640530916e-08
3169 1.42293995483572e-08
3170 1.42283527774373e-08
3171 1.42272675963118e-08
3172 1.42262175472363e-08
3173 1.42251620541545e-08
3174 1.42241104646029e-08
3175 1.42230586822611e-08
3176 1.42220146739508e-08
3177 1.4220942512036e-08
3178 1.42199069557564e-08
3179 1.4218874949512e-08
3180 1.42178245611524e-08
3181 1.42167575907515e-08
3182 1.42157059406789e-08
3183 1.42146689332961e-08
3184 1.42136238234919e-08
3185 1.42125717105812e-08
3186 1.42115308606749e-08
3187 1.42104997113285e-08
3188 1.42094423763311e-08
3189 1.4208415670014e-08
3190 1.42073718724239e-08
3191 1.42063345532489e-08
3192 1.42052817860283e-08
3193 1.42042631706141e-08
3194 1.42032426349858e-08
3195 1.42022132986752e-08
3196 1.42011752600896e-08
3197 1.4200113536561e-08
3198 1.41990965350086e-08
3199 1.41980816464882e-08
3200 1.41970427474519e-08
3201 1.41960015213188e-08
3202 1.41949902277005e-08
3203 1.41939488800813e-08
3204 1.41929231181753e-08
3205 1.41919077293051e-08
3206 1.41908736124019e-08
3207 1.41898195805262e-08
3208 1.41888086458986e-08
3209 1.4187782681585e-08
3210 1.41867425202946e-08
3211 1.41857425700054e-08
3212 1.41847077311935e-08
3213 1.41837123210503e-08
3214 1.41826737422718e-08
3215 1.41816345976264e-08
3216 1.41806257649979e-08
3217 1.41796208414507e-08
3218 1.417856463172e-08
3219 1.41775480213824e-08
3220 1.41765742098648e-08
3221 1.41755252004994e-08
3222 1.41745148082295e-08
3223 1.41734654276748e-08
3224 1.41724812984523e-08
3225 1.41714633395129e-08
3226 1.41704313082608e-08
3227 1.41694286980992e-08
3228 1.41683925586611e-08
3229 1.41673864873099e-08
3230 1.41664072885878e-08
3231 1.41653621164861e-08
3232 1.4164362436217e-08
3233 1.4163338975165e-08
3234 1.41623318813955e-08
3235 1.41613387045492e-08
3236 1.4160317196657e-08
3237 1.41593048189198e-08
3238 1.41583110536553e-08
3239 1.4157286501823e-08
3240 1.41562931445377e-08
3241 1.4155273758934e-08
3242 1.41542826358337e-08
3243 1.41532731307015e-08
3244 1.41522566428909e-08
3245 1.41512620063372e-08
3246 1.41502631582635e-08
3247 1.41492640044066e-08
3248 1.41482520688851e-08
3249 1.4147233747569e-08
3250 1.41462442129064e-08
3251 1.41452418040144e-08
3252 1.41442582159007e-08
3253 1.41432373400446e-08
3254 1.41422280086345e-08
3255 1.41412303509836e-08
3256 1.41402605658009e-08
3257 1.41392391843076e-08
3258 1.41382447510913e-08
3259 1.41372742045454e-08
3260 1.4136243307733e-08
3261 1.41352529884897e-08
3262 1.41342624983137e-08
3263 1.41332603913469e-08
3264 1.41322828690338e-08
3265 1.41312840386959e-08
3266 1.41303079101429e-08
3267 1.41292979805585e-08
3268 1.41283192555547e-08
3269 1.4127347427853e-08
3270 1.41263125729429e-08
3271 1.41253527448648e-08
3272 1.41243605809166e-08
3273 1.41233877543057e-08
3274 1.41224167817117e-08
3275 1.41214104743964e-08
3276 1.41204174788911e-08
3277 1.41194213900547e-08
3278 1.41184503383573e-08
3279 1.41174841008229e-08
3280 1.41164933530197e-08
3281 1.41154948567479e-08
3282 1.4114503994786e-08
3283 1.41135471152104e-08
3284 1.41125601097331e-08
3285 1.41115745778686e-08
3286 1.41105902981553e-08
3287 1.41096299420135e-08
3288 1.41086384432554e-08
3289 1.41076691895503e-08
3290 1.41066993962075e-08
3291 1.41056994100025e-08
3292 1.41047215239942e-08
3293 1.41037813279066e-08
3294 1.41027743421202e-08
3295 1.41018002175336e-08
3296 1.41008056104425e-08
3297 1.4099850199692e-08
3298 1.40988701519823e-08
3299 1.40979166117911e-08
3300 1.40969162695653e-08
3301 1.4095947247994e-08
3302 1.40950026321085e-08
3303 1.40940007282847e-08
3304 1.40930248105631e-08
3305 1.40920595621541e-08
3306 1.40910982097536e-08
3307 1.40901342383926e-08
3308 1.40891528841586e-08
3309 1.4088179913288e-08
3310 1.40872281001875e-08
3311 1.40862666184183e-08
3312 1.40852817162862e-08
3313 1.40843092710369e-08
3314 1.40833394058765e-08
3315 1.40823829489073e-08
3316 1.4081416956066e-08
3317 1.40804366015324e-08
3318 1.4079477070425e-08
3319 1.4078506273052e-08
3320 1.40775439141094e-08
3321 1.40765711169888e-08
3322 1.40755836713774e-08
3323 1.40746473148434e-08
3324 1.40736778491135e-08
3325 1.40727072316382e-08
3326 1.40717578000216e-08
3327 1.40707642511811e-08
3328 1.40698035221709e-08
3329 1.4068853048374e-08
3330 1.40679086178541e-08
3331 1.40669389577242e-08
3332 1.40659465843879e-08
3333 1.40650071055876e-08
3334 1.40640465316894e-08
3335 1.40631019178161e-08
3336 1.40621364371207e-08
3337 1.40611685493369e-08
3338 1.40602246576638e-08
3339 1.4059248304582e-08
3340 1.40582800557398e-08
3341 1.40572946743661e-08
3342 1.40563814236411e-08
3343 1.40554068570886e-08
3344 1.40544265537779e-08
3345 1.40535132129166e-08
3346 1.40525232462108e-08
3347 1.40515861599411e-08
3348 1.40506276200686e-08
3349 1.40496903690696e-08
3350 1.40487521023969e-08
3351 1.40477801069683e-08
3352 1.40468418059481e-08
3353 1.40458855123621e-08
3354 1.40449276795906e-08
3355 1.40439812022081e-08
3356 1.40430101308042e-08
3357 1.40420834820604e-08
3358 1.40411291199155e-08
3359 1.40401723506406e-08
3360 1.40392107558368e-08
3361 1.40382751900314e-08
3362 1.40373119242448e-08
3363 1.40363643047897e-08
3364 1.40354207089632e-08
3365 1.40344803031406e-08
3366 1.40335035643535e-08
3367 1.40325778884703e-08
3368 1.40316562426135e-08
3369 1.40306915022842e-08
3370 1.40297483506441e-08
3371 1.40288087258633e-08
3372 1.40278678825018e-08
3373 1.40268908707802e-08
3374 1.40259596501241e-08
3375 1.40250257180397e-08
3376 1.40240793264773e-08
3377 1.40231367667387e-08
3378 1.40222062985224e-08
3379 1.40212341105256e-08
3380 1.40203332216066e-08
3381 1.40193669760091e-08
3382 1.40184264565274e-08
3383 1.40174771997154e-08
3384 1.40165380111551e-08
3385 1.40156174682354e-08
3386 1.40146785733014e-08
3387 1.40137229389242e-08
3388 1.4012785430853e-08
3389 1.40118647397047e-08
3390 1.40109403251598e-08
3391 1.40099752270334e-08
3392 1.40090653996117e-08
3393 1.40081136223852e-08
3394 1.40071963257715e-08
3395 1.4006233771055e-08
3396 1.40053123782596e-08
3397 1.40043752461377e-08
3398 1.40034451216603e-08
3399 1.40025330694488e-08
3400 1.40015665234944e-08
3401 1.4000679655346e-08
3402 1.39997021079974e-08
3403 1.399876161369e-08
3404 1.3997853250694e-08
3405 1.399689765548e-08
3406 1.3996008237746e-08
3407 1.39950479153411e-08
3408 1.39941379868752e-08
3409 1.39931737101195e-08
3410 1.39922525248665e-08
3411 1.39913604283726e-08
3412 1.39904222728054e-08
3413 1.3989488340041e-08
3414 1.39885701770787e-08
3415 1.39876239709097e-08
3416 1.39866961275104e-08
3417 1.39857849109776e-08
3418 1.39848510544993e-08
3419 1.39839293370053e-08
3420 1.39829855211049e-08
3421 1.3982063403431e-08
3422 1.39811154562103e-08
3423 1.39802010887674e-08
3424 1.39793034843461e-08
3425 1.39783591081438e-08
3426 1.39774474133686e-08
3427 1.3976522494602e-08
3428 1.39755741849767e-08
3429 1.39746660546558e-08
3430 1.39737601198842e-08
3431 1.39728047352378e-08
3432 1.39718837509212e-08
3433 1.3970959192755e-08
3434 1.39700460848602e-08
3435 1.39691055850849e-08
3436 1.39681909139405e-08
3437 1.3967268379253e-08
3438 1.39663340798651e-08
3439 1.39654225479041e-08
3440 1.39644829591362e-08
3441 1.39635771526925e-08
3442 1.39626525596237e-08
3443 1.39617296320144e-08
3444 1.39608235464189e-08
3445 1.39598769094001e-08
3446 1.39589914632948e-08
3447 1.39580678217427e-08
3448 1.39571442241554e-08
3449 1.39562296030543e-08
3450 1.39553053483676e-08
3451 1.39544010311771e-08
3452 1.39534837222816e-08
3453 1.39525525032352e-08
3454 1.39516526664052e-08
3455 1.39507269799194e-08
3456 1.39498328914495e-08
3457 1.39489030898388e-08
3458 1.39480022863098e-08
3459 1.39470806826136e-08
3460 1.39461461179102e-08
3461 1.3945266089832e-08
3462 1.39443436708159e-08
3463 1.39434133630822e-08
3464 1.39424979024166e-08
3465 1.39415962952111e-08
3466 1.39406662647418e-08
3467 1.39397442468631e-08
3468 1.3938852372275e-08
3469 1.39379262557165e-08
3470 1.39370015601742e-08
3471 1.39361286816131e-08
3472 1.39351960186246e-08
3473 1.393427352675e-08
3474 1.39333625498034e-08
3475 1.39324669135577e-08
3476 1.3931567241332e-08
3477 1.39306603626765e-08
3478 1.39297315211451e-08
3479 1.39288603149129e-08
3480 1.39279432559841e-08
3481 1.39270259498364e-08
3482 1.39261012318398e-08
3483 1.39252272031709e-08
3484 1.39243134252148e-08
3485 1.39233871750133e-08
3486 1.39225082187722e-08
3487 1.39215920046398e-08
3488 1.39206703541617e-08
3489 1.39198026328086e-08
3490 1.39188805318297e-08
3491 1.3917964595267e-08
3492 1.39170701683317e-08
3493 1.39161627974033e-08
3494 1.3915242695714e-08
3495 1.39143419939508e-08
3496 1.39134251944534e-08
3497 1.39125386250938e-08
3498 1.39116158092839e-08
3499 1.39107309410358e-08
3500 1.39098335904947e-08
3501 1.39089085541971e-08
3502 1.39080203003378e-08
3503 1.39071332030949e-08
3504 1.3906220692958e-08
3505 1.39053201592132e-08
3506 1.39044412095918e-08
3507 1.39035108620289e-08
3508 1.39026124538616e-08
3509 1.3901695418192e-08
3510 1.39008365571264e-08
3511 1.38999341375984e-08
3512 1.38990160341718e-08
3513 1.38981209081707e-08
3514 1.38972373067564e-08
3515 1.38963045422102e-08
3516 1.38954279128439e-08
3517 1.38945369382998e-08
3518 1.38936363014985e-08
3519 1.38927624567381e-08
3520 1.38918643765584e-08
3521 1.38909331389298e-08
3522 1.38900562125926e-08
3523 1.38891568453731e-08
3524 1.38882783654315e-08
3525 1.38873457234678e-08
3526 1.38864850247333e-08
3527 1.38855665642174e-08
3528 1.38847005913822e-08
3529 1.38837880394316e-08
3530 1.38829048795114e-08
3531 1.38820254287492e-08
3532 1.38811290764651e-08
3533 1.38802142096023e-08
3534 1.38793564039841e-08
3535 1.38784463451364e-08
3536 1.3877552985625e-08
3537 1.38766667120843e-08
3538 1.38757774408443e-08
3539 1.38748878549533e-08
3540 1.38739865772342e-08
3541 1.38731004717674e-08
3542 1.38722051891277e-08
3543 1.38713227834791e-08
3544 1.38704421232122e-08
3545 1.38695660381605e-08
3546 1.38686598983168e-08
3547 1.3867799566844e-08
3548 1.38669021331334e-08
3549 1.38660282432423e-08
3550 1.38651262309303e-08
3551 1.38642523693083e-08
3552 1.38633508914715e-08
3553 1.3862460865377e-08
3554 1.38615884956728e-08
3555 1.38607152125464e-08
3556 1.38598201023521e-08
3557 1.38589382732979e-08
3558 1.38580507506036e-08
3559 1.38571694924261e-08
3560 1.38562980802753e-08
3561 1.38554234237198e-08
3562 1.38545243782151e-08
3563 1.3853652730586e-08
3564 1.38527822875489e-08
3565 1.38518785613717e-08
3566 1.38509921515512e-08
3567 1.38501331309326e-08
3568 1.38492482429642e-08
3569 1.38483655853644e-08
3570 1.38474799197125e-08
3571 1.38466238666229e-08
3572 1.38457085967075e-08
3573 1.38448607774483e-08
3574 1.38439918111605e-08
3575 1.38430856114341e-08
3576 1.38422007657374e-08
3577 1.38413233105039e-08
3578 1.38404560438982e-08
3579 1.38395931807977e-08
3580 1.38386790274891e-08
3581 1.38378340575757e-08
3582 1.38369195659821e-08
3583 1.38360695022127e-08
3584 1.38351854263169e-08
3585 1.38343246480349e-08
3586 1.38334638106336e-08
3587 1.3832569974101e-08
3588 1.38317283996103e-08
3589 1.38308129769421e-08
3590 1.38299466352215e-08
3591 1.38290745142072e-08
3592 1.38281827519598e-08
3593 1.38273264983918e-08
3594 1.38264517746817e-08
3595 1.38255738924287e-08
3596 1.38247118613433e-08
3597 1.38238356617443e-08
3598 1.38229680280988e-08
3599 1.38220906445302e-08
3600 1.38212253869979e-08
3601 1.38203346247007e-08
3602 1.38194589713869e-08
3603 1.38186068131402e-08
3604 1.38177201266937e-08
3605 1.38168666199007e-08
3606 1.38159937612126e-08
3607 1.38151089589672e-08
3608 1.38142426536064e-08
3609 1.38133827854298e-08
3610 1.38125137565948e-08
3611 1.38116261430982e-08
3612 1.38107433973883e-08
3613 1.38098861918595e-08
3614 1.38089931175228e-08
3615 1.38081460614586e-08
3616 1.38072779380105e-08
3617 1.38064234016383e-08
3618 1.38055260313213e-08
3619 1.38046527632107e-08
3620 1.38037880366843e-08
3621 1.38029326184202e-08
3622 1.38020669250977e-08
3623 1.38012124012571e-08
3624 1.38003148325433e-08
3625 1.37994605339808e-08
3626 1.3798600680362e-08
3627 1.37977316701371e-08
3628 1.37968550732032e-08
3629 1.37959687283307e-08
3630 1.37951318666535e-08
3631 1.37942683482578e-08
3632 1.37933867633971e-08
3633 1.3792508292268e-08
3634 1.37916577202107e-08
3635 1.37908180975616e-08
3636 1.37899056397994e-08
3637 1.37890463347279e-08
3638 1.37881739421114e-08
3639 1.37873233230085e-08
3640 1.37864665892967e-08
3641 1.37855786855323e-08
3642 1.37847405030089e-08
3643 1.37838618577274e-08
3644 1.378300177797e-08
3645 1.37821200854732e-08
3646 1.37812636248902e-08
3647 1.37804017356219e-08
3648 1.37795412426117e-08
3649 1.37786931527356e-08
3650 1.37778015095597e-08
3651 1.37769800380716e-08
3652 1.37760765068634e-08
3653 1.37752350544695e-08
3654 1.37743628507297e-08
3655 1.37735319661048e-08
3656 1.37726308759994e-08
3657 1.37717881472238e-08
3658 1.37709225023069e-08
3659 1.37700667623281e-08
3660 1.3769208405609e-08
3661 1.37683527061255e-08
3662 1.37674595178661e-08
3663 1.37666206491555e-08
3664 1.37657780366618e-08
3665 1.37649226937403e-08
3666 1.37640228627528e-08
3667 1.37632025929285e-08
3668 1.37623416378296e-08
3669 1.37614503121081e-08
3670 1.3760619478137e-08
3671 1.37597666139133e-08
3672 1.37588960663348e-08
3673 1.37580360893702e-08
3674 1.3757204768497e-08
3675 1.37563522233791e-08
3676 1.37554775049009e-08
3677 1.37546219404622e-08
3678 1.37538010187843e-08
3679 1.37528902133949e-08
3680 1.37520935634267e-08
3681 1.37512107943882e-08
3682 1.3750382774555e-08
3683 1.37495116201009e-08
3684 1.37486639332218e-08
3685 1.37478044546779e-08
3686 1.37469580611393e-08
3687 1.37461184100823e-08
3688 1.37452463628146e-08
3689 1.37443732879799e-08
3690 1.37435249521062e-08
3691 1.37426824841358e-08
3692 1.37418278551849e-08
3693 1.37409967762986e-08
3694 1.3740148014571e-08
3695 1.3739295202389e-08
3696 1.37384605464475e-08
3697 1.37376028850472e-08
3698 1.37367686716267e-08
3699 1.37358972494839e-08
3700 1.37350560764499e-08
3701 1.37342235967813e-08
3702 1.37333538762358e-08
3703 1.37325090566603e-08
3704 1.37316511086005e-08
3705 1.37308398278219e-08
3706 1.37299695012194e-08
3707 1.37291472025042e-08
3708 1.37282966768681e-08
3709 1.37274391964343e-08
3710 1.37265952144389e-08
3711 1.3725766245129e-08
3712 1.37249124994299e-08
3713 1.37240764322005e-08
3714 1.37232008252247e-08
3715 1.37223652688229e-08
3716 1.37215073338914e-08
3717 1.37206746347179e-08
3718 1.37198394282861e-08
3719 1.37189682239275e-08
3720 1.37181334032149e-08
3721 1.37173018237846e-08
3722 1.37164465479483e-08
3723 1.37156232655061e-08
3724 1.37147726636533e-08
3725 1.37139343060061e-08
3726 1.37130910563416e-08
3727 1.37122369738424e-08
3728 1.37114170057073e-08
3729 1.37105688765987e-08
3730 1.37096984660218e-08
3731 1.37088892704501e-08
3732 1.37080194950595e-08
3733 1.37071806669692e-08
3734 1.37063539628335e-08
3735 1.37054836126665e-08
3736 1.37046489776527e-08
3737 1.37038157283359e-08
3738 1.37029661165661e-08
3739 1.37021510616814e-08
3740 1.37013197209906e-08
3741 1.37004725821038e-08
3742 1.36996109519061e-08
3743 1.36987942802452e-08
3744 1.36979547620403e-08
3745 1.36971126440677e-08
3746 1.36962800788842e-08
3747 1.36954397090133e-08
3748 1.36946075998956e-08
3749 1.36937748942412e-08
3750 1.36929204111597e-08
3751 1.36920745255897e-08
3752 1.369124625554e-08
3753 1.36903937743432e-08
3754 1.36895943209925e-08
3755 1.36887490128218e-08
3756 1.36879200202528e-08
3757 1.36870691857388e-08
3758 1.36862278770911e-08
3759 1.36854118318319e-08
3760 1.36845572104027e-08
3761 1.3683738814288e-08
3762 1.36829064158739e-08
3763 1.36820705991664e-08
3764 1.36812610629644e-08
3765 1.36804067484148e-08
3766 1.36795666910855e-08
3767 1.36787501162777e-08
3768 1.36778982885583e-08
3769 1.36770581880413e-08
3770 1.36762340842422e-08
3771 1.36754130656003e-08
3772 1.36745610838096e-08
3773 1.36737235181122e-08
3774 1.36729034216493e-08
3775 1.36720637278487e-08
3776 1.36712216682461e-08
3777 1.36704271056359e-08
3778 1.36695907654166e-08
3779 1.36687322130069e-08
3780 1.36679402842205e-08
3781 1.3667106276663e-08
3782 1.36663111941077e-08
3783 1.36654582229695e-08
3784 1.36645931174684e-08
3785 1.36638025998032e-08
3786 1.36629664227589e-08
3787 1.366213140018e-08
3788 1.36613023343224e-08
3789 1.36604740045565e-08
3790 1.36596437298048e-08
3791 1.36588383576469e-08
3792 1.36579946513199e-08
3793 1.3657175254575e-08
3794 1.36563301539189e-08
3795 1.36555005837147e-08
3796 1.36546861029957e-08
3797 1.36538515756873e-08
3798 1.36530497991255e-08
3799 1.36521867036693e-08
3800 1.36513798813104e-08
3801 1.36505395220698e-08
3802 1.36497406197922e-08
3803 1.3648895490076e-08
3804 1.36480619127438e-08
3805 1.36472419269423e-08
3806 1.36464154257832e-08
3807 1.36455851334205e-08
3808 1.36447623785424e-08
3809 1.36439488239021e-08
3810 1.3643099475949e-08
3811 1.36423026092097e-08
3812 1.36414722157752e-08
3813 1.3640643241733e-08
3814 1.36398245919184e-08
3815 1.36390062901171e-08
3816 1.36381711702133e-08
3817 1.36373615464147e-08
3818 1.36365120029375e-08
3819 1.36356941809329e-08
3820 1.36348563185285e-08
3821 1.36340467744023e-08
3822 1.36332055522964e-08
3823 1.36323921560016e-08
3824 1.36315559942368e-08
3825 1.36307720237649e-08
3826 1.36299278791219e-08
3827 1.36291257471638e-08
3828 1.36283030586076e-08
3829 1.36274824174693e-08
3830 1.36266755247361e-08
3831 1.36258405639827e-08
3832 1.36250191029724e-08
3833 1.36242217154831e-08
3834 1.36233816454001e-08
3835 1.3622555186818e-08
3836 1.36217573213082e-08
3837 1.3620915291182e-08
3838 1.36201027409466e-08
3839 1.36192883690711e-08
3840 1.3618483491934e-08
3841 1.36176599609544e-08
3842 1.36168283071947e-08
3843 1.36160251632406e-08
3844 1.36152220704955e-08
3845 1.36143905224984e-08
3846 1.36135846845603e-08
3847 1.36127462786045e-08
3848 1.36119441971894e-08
3849 1.36111213525775e-08
3850 1.36102816062772e-08
3851 1.36094903067097e-08
3852 1.3608678959387e-08
3853 1.36078678910495e-08
3854 1.3607038525959e-08
3855 1.36062140902726e-08
3856 1.36054037269268e-08
3857 1.36046107385573e-08
3858 1.36037596060318e-08
3859 1.36029615476346e-08
3860 1.36021261706448e-08
3861 1.36013633693088e-08
3862 1.36005122237243e-08
3863 1.35997125843695e-08
3864 1.35988838730339e-08
3865 1.35980939559299e-08
3866 1.35972750351099e-08
3867 1.35964275261441e-08
3868 1.35956187678199e-08
3869 1.35948253644075e-08
3870 1.35940078300201e-08
3871 1.35932080985862e-08
3872 1.35924032070717e-08
3873 1.35915627254429e-08
3874 1.35907845928013e-08
3875 1.35899656334287e-08
3876 1.35891468255461e-08
3877 1.35883331322112e-08
3878 1.35875444703809e-08
3879 1.35867260784017e-08
3880 1.35859178575504e-08
3881 1.35850876273463e-08
3882 1.35842978942896e-08
3883 1.3583471091233e-08
3884 1.35826843161713e-08
3885 1.3581840805757e-08
3886 1.35810658663943e-08
3887 1.35802149376502e-08
3888 1.3579447844983e-08
3889 1.35786236318131e-08
3890 1.35778184917473e-08
3891 1.35769801916791e-08
3892 1.3576195749293e-08
3893 1.35753826862456e-08
3894 1.35745680873017e-08
3895 1.35737493481419e-08
3896 1.35729462195783e-08
3897 1.35721411602258e-08
3898 1.35713348319022e-08
3899 1.3570503537258e-08
3900 1.35697090104109e-08
3901 1.35689181289395e-08
3902 1.35681201289817e-08
3903 1.35672812702076e-08
3904 1.35664723575346e-08
3905 1.35656637148401e-08
3906 1.35648499607899e-08
3907 1.35640625222172e-08
3908 1.35632444967088e-08
3909 1.35624658517131e-08
3910 1.3561649195859e-08
3911 1.3560818361763e-08
3912 1.3560021256645e-08
3913 1.3559209514874e-08
3914 1.35583917964394e-08
3915 1.35575912877522e-08
3916 1.35568130381625e-08
3917 1.35560224636261e-08
3918 1.35551926748606e-08
3919 1.35543827141094e-08
3920 1.35536125597663e-08
3921 1.35527948338793e-08
3922 1.35519772614806e-08
3923 1.35511747146599e-08
3924 1.3550360999065e-08
3925 1.35495637962613e-08
3926 1.35487591455402e-08
3927 1.35479651032083e-08
3928 1.35471690482364e-08
3929 1.35463656529555e-08
3930 1.3545563721476e-08
3931 1.35447720982784e-08
3932 1.35439512456426e-08
3933 1.3543161197338e-08
3934 1.35423941104162e-08
3935 1.35415577352943e-08
3936 1.35407733110604e-08
3937 1.35399671882114e-08
3938 1.35391647678729e-08
3939 1.35383406549422e-08
3940 1.3537567129504e-08
3941 1.3536751329063e-08
3942 1.35359729654255e-08
3943 1.35351373332232e-08
3944 1.35343558027717e-08
3945 1.35335719672891e-08
3946 1.35327253057843e-08
3947 1.35319421635666e-08
3948 1.3531144585438e-08
3949 1.35303641512624e-08
3950 1.35295606308178e-08
3951 1.35287713526056e-08
3952 1.35279720923781e-08
3953 1.35271744787779e-08
3954 1.35263437354149e-08
3955 1.35255651843025e-08
3956 1.3524770480694e-08
3957 1.35239639011131e-08
3958 1.35231785439865e-08
3959 1.35223599547624e-08
3960 1.35215454866305e-08
3961 1.35207525503861e-08
3962 1.35199587139451e-08
3963 1.35191371597454e-08
3964 1.35183501320385e-08
3965 1.3517558036677e-08
3966 1.35167682567411e-08
3967 1.35159747456787e-08
3968 1.35151585650695e-08
3969 1.35143773074831e-08
3970 1.35135607445408e-08
3971 1.35127612847508e-08
3972 1.35119627819591e-08
3973 1.35111814568295e-08
3974 1.35104072914588e-08
3975 1.35096069034946e-08
3976 1.35087863380595e-08
3977 1.35079725355758e-08
3978 1.35072190054764e-08
3979 1.35064121821044e-08
3980 1.35056250495369e-08
3981 1.35048581852565e-08
3982 1.35040237002337e-08
3983 1.35032462796059e-08
3984 1.35024238619025e-08
3985 1.35016604997235e-08
3986 1.35008595217867e-08
3987 1.35000770879051e-08
3988 1.34993077115092e-08
3989 1.34984909497954e-08
3990 1.34977170052203e-08
3991 1.34969017020065e-08
3992 1.34961225831537e-08
3993 1.34953366736495e-08
3994 1.34945248680129e-08
3995 1.34937509221333e-08
3996 1.34929454437688e-08
3997 1.34921738261795e-08
3998 1.34913666861997e-08
3999 1.34905855186246e-08
4000 1.34897955288288e-08
4001 1.34889947856071e-08
4002 1.34882248785523e-08
4003 1.34874407594909e-08
4004 1.34866411475726e-08
4005 1.34858566218504e-08
4006 1.34850646062862e-08
4007 1.34842744647229e-08
4008 1.34835011534884e-08
4009 1.34826935271198e-08
4010 1.34819353711929e-08
4011 1.34811107444094e-08
4012 1.34803254017019e-08
4013 1.34795187032993e-08
4014 1.34787550256504e-08
4015 1.3477959812408e-08
4016 1.34771756270524e-08
4017 1.34763965424084e-08
4018 1.3475603117083e-08
4019 1.34748110627581e-08
4020 1.34740082092799e-08
4021 1.34732353378603e-08
4022 1.34724559192057e-08
4023 1.34716585874911e-08
4024 1.34708974568742e-08
4025 1.34701041279439e-08
4026 1.34692781455364e-08
4027 1.34685041572741e-08
4028 1.34677501351793e-08
4029 1.34669597584569e-08
4030 1.34661757816845e-08
4031 1.34653778142702e-08
4032 1.34645897059693e-08
4033 1.34638410171778e-08
4034 1.34630335607289e-08
4035 1.34622277391666e-08
4036 1.34614611085465e-08
4037 1.34606823700562e-08
4038 1.34599025102128e-08
4039 1.34591002724366e-08
4040 1.34583357973006e-08
4041 1.34575399319131e-08
4042 1.34567699164728e-08
4043 1.34559872174145e-08
4044 1.34551974569075e-08
4045 1.34544160781264e-08
4046 1.34536199239144e-08
4047 1.34528375710513e-08
4048 1.34520353468198e-08
4049 1.34512898910671e-08
4050 1.34504826527909e-08
4051 1.34496922646332e-08
4052 1.34489128559984e-08
4053 1.34481417606996e-08
4054 1.34473454038997e-08
4055 1.34465829486757e-08
4056 1.34458002649662e-08
4057 1.34450020483762e-08
4058 1.34442207255087e-08
4059 1.34434384113091e-08
4060 1.34426626693246e-08
4061 1.34418851947538e-08
4062 1.34410939889862e-08
4063 1.34403330657312e-08
4064 1.34395203702092e-08
4065 1.34387635758321e-08
4066 1.34379739201856e-08
4067 1.34371992731724e-08
4068 1.34364168098283e-08
4069 1.34356407524711e-08
4070 1.34348534677614e-08
4071 1.34340768629532e-08
4072 1.34332865334846e-08
4073 1.34325100999977e-08
4074 1.34317463417605e-08
4075 1.34309802717197e-08
4076 1.34301831549166e-08
4077 1.34294137534158e-08
4078 1.34286540150741e-08
4079 1.34278612188704e-08
4080 1.3427101693983e-08
4081 1.34262969155863e-08
4082 1.34255092641838e-08
4083 1.34247413890592e-08
4084 1.3423959529954e-08
4085 1.34231945012192e-08
4086 1.34224276245598e-08
4087 1.34216408555349e-08
4088 1.34208633097849e-08
4089 1.34200994850037e-08
4090 1.34193139922495e-08
4091 1.34185746693272e-08
4092 1.34177849093198e-08
4093 1.34169946445356e-08
4094 1.34162448789527e-08
4095 1.3415466116079e-08
4096 1.34146869090329e-08
4097 1.34138876962675e-08
4098 1.34131326145065e-08
4099 1.34123618791004e-08
4100 1.34115854337896e-08
4101 1.3410860347382e-08
4102 1.34100423311717e-08
4103 1.34092735547542e-08
4104 1.34085166833969e-08
4105 1.34077329658339e-08
4106 1.34069556940175e-08
4107 1.34061855236872e-08
4108 1.34053996517086e-08
4109 1.34046365144053e-08
4110 1.34038582498969e-08
4111 1.34030876542818e-08
4112 1.34023055956278e-08
4113 1.3401531343174e-08
4114 1.34007975360895e-08
4115 1.34000010163365e-08
4116 1.33992256596688e-08
4117 1.33984555753947e-08
4118 1.33976769688127e-08
4119 1.33969363185354e-08
4120 1.33961232224727e-08
4121 1.33954022474569e-08
4122 1.33946157809539e-08
4123 1.33938456660099e-08
4124 1.33930652784914e-08
4125 1.33922894907934e-08
4126 1.33915338707963e-08
4127 1.33907772035258e-08
4128 1.33900101721846e-08
4129 1.33892153737908e-08
4130 1.3388456035851e-08
4131 1.33877036109509e-08
4132 1.33869133068787e-08
4133 1.33861351206827e-08
4134 1.33853727796313e-08
4135 1.33846036662472e-08
4136 1.33838411999071e-08
4137 1.33830801123946e-08
4138 1.33823216152823e-08
4139 1.33815281271182e-08
4140 1.33807870178193e-08
4141 1.33800093611858e-08
4142 1.33792505274954e-08
4143 1.33784842737961e-08
4144 1.33777145432806e-08
4145 1.33769281069618e-08
4146 1.33761918346964e-08
4147 1.337540895055e-08
4148 1.33746300691601e-08
4149 1.33738747327e-08
4150 1.33731003639642e-08
4151 1.33723443197231e-08
4152 1.33715697988868e-08
4153 1.33708306546271e-08
4154 1.33700184815899e-08
4155 1.33692671513835e-08
4156 1.33685114690335e-08
4157 1.33677455612935e-08
4158 1.33669705995598e-08
4159 1.33661947521874e-08
4160 1.33654677878653e-08
4161 1.33646733235931e-08
4162 1.33639203655778e-08
4163 1.33631556809427e-08
4164 1.3362408111825e-08
4165 1.33616268209874e-08
4166 1.33608713560746e-08
4167 1.33601049202431e-08
4168 1.33593356104883e-08
4169 1.33585799753227e-08
4170 1.33577781375183e-08
4171 1.33570402093275e-08
4172 1.33562732102938e-08
4173 1.33554901243227e-08
4174 1.33547469094025e-08
4175 1.33539751984157e-08
4176 1.3353230016791e-08
4177 1.33524431598364e-08
4178 1.33516967456698e-08
4179 1.33509288291894e-08
4180 1.33501674644265e-08
4181 1.33494197825101e-08
4182 1.33486256292253e-08
4183 1.33479145989629e-08
4184 1.3347147843068e-08
4185 1.33463646817961e-08
4186 1.33456201792531e-08
4187 1.33448608960474e-08
4188 1.33440886495306e-08
4189 1.33433418346013e-08
4190 1.33425801307901e-08
4191 1.33417953475518e-08
4192 1.33410261430461e-08
4193 1.33402565094393e-08
4194 1.33395434928768e-08
4195 1.33387522536638e-08
4196 1.33379936263084e-08
4197 1.33372238086543e-08
4198 1.33364782751583e-08
4199 1.33357215949398e-08
4200 1.33349404450728e-08
4201 1.33341824073152e-08
4202 1.33334012307196e-08
4203 1.33326628846409e-08
4204 1.33318781088548e-08
4205 1.33311256203666e-08
4206 1.33303826627545e-08
4207 1.33296319734799e-08
4208 1.33288455221592e-08
4209 1.33280661472335e-08
4210 1.33273111665722e-08
4211 1.33265745082406e-08
4212 1.33258217032139e-08
4213 1.33250362602672e-08
4214 1.33242881295986e-08
4215 1.33235365962908e-08
4216 1.33227594419083e-08
4217 1.33220254363575e-08
4218 1.33212715657111e-08
4219 1.33205023539751e-08
4220 1.33197677867208e-08
4221 1.33190018686558e-08
4222 1.33182483756239e-08
4223 1.33174601725655e-08
4224 1.3316728721291e-08
4225 1.33159767143481e-08
4226 1.33152199934816e-08
4227 1.33144580030109e-08
4228 1.33137063951239e-08
4229 1.33129565354911e-08
4230 1.33121904688366e-08
4231 1.33114595673861e-08
4232 1.33107069033717e-08
4233 1.33099477224313e-08
4234 1.33091662216506e-08
4235 1.33084300914105e-08
4236 1.33076530910159e-08
4237 1.33069201057878e-08
4238 1.33061375503507e-08
4239 1.33054130597582e-08
4240 1.33046499710271e-08
4241 1.33038848623424e-08
4242 1.33031253260613e-08
4243 1.33023857870757e-08
4244 1.33016154260091e-08
4245 1.33008751285607e-08
4246 1.33001244037589e-08
4247 1.32993574673695e-08
4248 1.32986127691637e-08
4249 1.32978551425772e-08
4250 1.32970858406772e-08
4251 1.32963466688008e-08
4252 1.32955780952487e-08
4253 1.32948317623227e-08
4254 1.32941117192281e-08
4255 1.32933379540651e-08
4256 1.32925663956923e-08
4257 1.32918240166591e-08
4258 1.32910847923245e-08
4259 1.32903268922763e-08
4260 1.32895732899985e-08
4261 1.32888164590117e-08
4262 1.32880603460339e-08
4263 1.32873252026294e-08
4264 1.32865630312479e-08
4265 1.32858157684129e-08
4266 1.32850603755258e-08
4267 1.32843243885322e-08
4268 1.32835765059985e-08
4269 1.3282819042143e-08
4270 1.32820896158042e-08
4271 1.32813424576356e-08
4272 1.32805808225334e-08
4273 1.32798333146306e-08
4274 1.32790766866619e-08
4275 1.32783478527104e-08
4276 1.32775591384915e-08
4277 1.32768515151843e-08
4278 1.32761222563088e-08
4279 1.32753665328528e-08
4280 1.32746028223413e-08
4281 1.32738664044701e-08
4282 1.32730786159274e-08
4283 1.32723477862806e-08
4284 1.32715810224476e-08
4285 1.32708667875681e-08
4286 1.32701020006537e-08
4287 1.32693602826611e-08
4288 1.32685993609327e-08
4289 1.32678633986416e-08
4290 1.32670950049041e-08
4291 1.32663779068426e-08
4292 1.32656190121733e-08
4293 1.32648729016804e-08
4294 1.32641346359402e-08
4295 1.32633548045741e-08
4296 1.32626077597175e-08
4297 1.32618806889112e-08
4298 1.32611551671019e-08
4299 1.32603761762579e-08
4300 1.32596507986249e-08
4301 1.3258897148774e-08
4302 1.32581549575211e-08
4303 1.32574060132673e-08
4304 1.32566720974781e-08
4305 1.32559069929844e-08
4306 1.32551953442028e-08
4307 1.32544374213811e-08
4308 1.32536956371082e-08
4309 1.32529410830362e-08
4310 1.32522027421061e-08
4311 1.32514743299422e-08
4312 1.32507260837272e-08
4313 1.32499863982616e-08
4314 1.32492222281871e-08
4315 1.3248477667413e-08
4316 1.32477525557478e-08
4317 1.32469745292435e-08
4318 1.32462496722358e-08
4319 1.32455217093236e-08
4320 1.32447490737708e-08
4321 1.32440446167364e-08
4322 1.32433025955697e-08
4323 1.32425615858439e-08
4324 1.32418209675272e-08
4325 1.32410406006173e-08
4326 1.3240318271393e-08
4327 1.32395609223207e-08
4328 1.32388235178083e-08
4329 1.32381211411653e-08
4330 1.32373512064243e-08
4331 1.3236610323375e-08
4332 1.32358611577149e-08
4333 1.32351169766093e-08
4334 1.32343679010855e-08
4335 1.32336503338021e-08
4336 1.32328688316752e-08
4337 1.32321529437845e-08
4338 1.32314053173932e-08
4339 1.32306661521087e-08
4340 1.32299166831634e-08
4341 1.32292138180501e-08
4342 1.32284694823598e-08
4343 1.32277355282817e-08
4344 1.3226947021841e-08
4345 1.32262422364027e-08
4346 1.32254855825104e-08
4347 1.32247383441003e-08
4348 1.32240044546789e-08
4349 1.3223273019905e-08
4350 1.32225087901977e-08
4351 1.32217955768205e-08
4352 1.32210286607898e-08
4353 1.3220279442186e-08
4354 1.32195770715521e-08
4355 1.32188330546901e-08
4356 1.32180750824912e-08
4357 1.32173611497727e-08
4358 1.32166174145604e-08
4359 1.3215898302818e-08
4360 1.32151479211723e-08
4361 1.32143911538291e-08
4362 1.32136598290783e-08
4363 1.32129076613335e-08
4364 1.32121563218429e-08
4365 1.32114502206238e-08
4366 1.32107159520195e-08
4367 1.3209948097212e-08
4368 1.32092274777035e-08
4369 1.32084984983266e-08
4370 1.32077581398715e-08
4371 1.32070375642446e-08
4372 1.32062761849239e-08
4373 1.32055288329519e-08
4374 1.32048076589286e-08
4375 1.32040889797014e-08
4376 1.32033432036216e-08
4377 1.32026274866359e-08
4378 1.32018759487484e-08
4379 1.3201148389666e-08
4380 1.32003997078689e-08
4381 1.31996772304993e-08
4382 1.319892472397e-08
4383 1.31982202560277e-08
4384 1.31974660627088e-08
4385 1.31967468950805e-08
4386 1.31959994289499e-08
4387 1.31952642567351e-08
4388 1.31945009713141e-08
4389 1.31937936688614e-08
4390 1.31930480616049e-08
4391 1.31923098639214e-08
4392 1.31915732819593e-08
4393 1.31908686726717e-08
4394 1.31901044748556e-08
4395 1.3189395446811e-08
4396 1.31886337584597e-08
4397 1.31879067128832e-08
4398 1.31871632667313e-08
4399 1.31864491048489e-08
4400 1.31857337983959e-08
4401 1.3185003476246e-08
4402 1.31842628644935e-08
4403 1.31835379384715e-08
4404 1.31827858616956e-08
4405 1.31820569031493e-08
4406 1.31813206820375e-08
4407 1.31805950533692e-08
4408 1.31798456261267e-08
4409 1.31791170720902e-08
4410 1.31783897002885e-08
4411 1.31776532751593e-08
4412 1.31769173957186e-08
4413 1.31761764135818e-08
4414 1.31754474172463e-08
4415 1.31747284935063e-08
4416 1.31740047772805e-08
4417 1.31732743057361e-08
4418 1.31725533663585e-08
4419 1.31718029518918e-08
4420 1.31710480905517e-08
4421 1.317035774695e-08
4422 1.31696092613154e-08
4423 1.31689089766102e-08
4424 1.31681586616889e-08
4425 1.31674349647393e-08
4426 1.31667136861885e-08
4427 1.31659646087634e-08
4428 1.31652459731402e-08
4429 1.31645063149027e-08
4430 1.31637889975167e-08
4431 1.31630408498479e-08
4432 1.31623156627986e-08
4433 1.31616072825275e-08
4434 1.31608530460486e-08
4435 1.31601141121346e-08
4436 1.31594026317322e-08
4437 1.31586747437601e-08
4438 1.3157940606856e-08
4439 1.31572093931692e-08
4440 1.31564921081045e-08
4441 1.31557673527932e-08
4442 1.31550380703532e-08
4443 1.31543032820397e-08
4444 1.31535924858123e-08
4445 1.31528647519669e-08
4446 1.31521158517472e-08
4447 1.31513906085207e-08
4448 1.31506628644612e-08
4449 1.31499435864352e-08
4450 1.31492249552806e-08
4451 1.31484821883077e-08
4452 1.31477743992997e-08
4453 1.31470326975164e-08
4454 1.31463142282601e-08
4455 1.31455795881474e-08
4456 1.31448613383128e-08
4457 1.31441377549391e-08
4458 1.3143399516663e-08
4459 1.31426516714328e-08
4460 1.31419385671766e-08
4461 1.3141194614763e-08
4462 1.3140492741523e-08
4463 1.31397527603216e-08
4464 1.3139034192991e-08
4465 1.31382918133055e-08
4466 1.31375667773853e-08
4467 1.31368616131111e-08
4468 1.31361138806518e-08
4469 1.31354089046715e-08
4470 1.31346718900555e-08
4471 1.31339663345942e-08
4472 1.31332343805829e-08
4473 1.3132501371832e-08
4474 1.31317740253295e-08
4475 1.31310157324149e-08
4476 1.3130311477344e-08
4477 1.31295995664665e-08
4478 1.31288791415107e-08
4479 1.31281617740259e-08
4480 1.31274169127005e-08
4481 1.31267188903522e-08
4482 1.31259700494046e-08
4483 1.31252385775771e-08
4484 1.31245432032356e-08
4485 1.31237981461224e-08
4486 1.31230999116383e-08
4487 1.31223632452299e-08
4488 1.31216300355563e-08
4489 1.31209205969929e-08
4490 1.31202061233321e-08
4491 1.31194635912962e-08
4492 1.31187743874633e-08
4493 1.31180429892713e-08
4494 1.31173180768912e-08
4495 1.31165863904575e-08
4496 1.31158766162598e-08
4497 1.31151662250001e-08
4498 1.31144466713334e-08
4499 1.31137163803391e-08
4500 1.31129932377239e-08
4501 1.31122674796452e-08
4502 1.31115573460544e-08
4503 1.31108331675317e-08
4504 1.31101155260993e-08
4505 1.31093933845305e-08
4506 1.31086802161173e-08
4507 1.31079520319793e-08
4508 1.31072520693498e-08
4509 1.31065061691193e-08
4510 1.31057738418511e-08
4511 1.31050507902325e-08
4512 1.31043480925824e-08
4513 1.3103595595032e-08
4514 1.31028849179454e-08
4515 1.31021898368139e-08
4516 1.31014625084774e-08
4517 1.3100740798036e-08
4518 1.31000130514924e-08
4519 1.30993163743043e-08
4520 1.30985871881484e-08
4521 1.30978474204568e-08
4522 1.30971422769993e-08
4523 1.30964394003258e-08
4524 1.30957176586038e-08
4525 1.30950044352485e-08
4526 1.30942972487863e-08
4527 1.3093553796778e-08
4528 1.30928516652584e-08
4529 1.30921257945893e-08
4530 1.30914245445313e-08
4531 1.30906873979098e-08
4532 1.3089982514522e-08
4533 1.30892308377734e-08
4534 1.30885472279657e-08
4535 1.30878295133419e-08
4536 1.30870949773682e-08
4537 1.30864094534666e-08
4538 1.308565799997e-08
4539 1.30849790945659e-08
4540 1.30842479446058e-08
4541 1.30835357028541e-08
4542 1.30828126793242e-08
4543 1.30821133633857e-08
4544 1.30813930457191e-08
4545 1.30806797744853e-08
4546 1.30799495150907e-08
4547 1.3079250662143e-08
4548 1.30785182722998e-08
4549 1.3077793381347e-08
4550 1.30771019463821e-08
4551 1.3076378277993e-08
4552 1.30756820401479e-08
4553 1.30749395056695e-08
4554 1.30742408758627e-08
4555 1.30735174642127e-08
4556 1.30728188690726e-08
4557 1.30721054449751e-08
4558 1.3071399949896e-08
4559 1.30706921682017e-08
4560 1.30699748972091e-08
4561 1.30692479188149e-08
4562 1.30685329120389e-08
4563 1.30678328396083e-08
4564 1.30671085376149e-08
4565 1.30664252862706e-08
4566 1.30656957665898e-08
4567 1.30650057863363e-08
4568 1.30642596770369e-08
4569 1.3063558531326e-08
4570 1.30628459784482e-08
4571 1.30621470402492e-08
4572 1.30614057518852e-08
4573 1.30607026089941e-08
4574 1.30599587067765e-08
4575 1.30592915336264e-08
4576 1.3058568744187e-08
4577 1.30578515705887e-08
4578 1.3057140028791e-08
4579 1.30564252720927e-08
4580 1.30557313332696e-08
4581 1.30549841423494e-08
4582 1.3054286385622e-08
4583 1.30536013873972e-08
4584 1.30528806871061e-08
4585 1.30521743321316e-08
4586 1.30514585311992e-08
4587 1.30507545634123e-08
4588 1.30500356513435e-08
4589 1.30493480066801e-08
4590 1.30486292770765e-08
4591 1.30479350071849e-08
4592 1.30472022403377e-08
4593 1.30465243270567e-08
4594 1.30457935359346e-08
4595 1.3045111954782e-08
4596 1.30444015332132e-08
4597 1.30436736122119e-08
4598 1.30429752696198e-08
4599 1.30422662101559e-08
4600 1.30415844936255e-08
4601 1.30408550293587e-08
4602 1.30401358518356e-08
4603 1.30394503160824e-08
4604 1.30387377721419e-08
4605 1.30380124317014e-08
4606 1.30373247164833e-08
4607 1.3036623992338e-08
4608 1.30358864424418e-08
4609 1.30351918084109e-08
4610 1.30345022673756e-08
4611 1.30337732790614e-08
4612 1.30331128372213e-08
4613 1.30323678862315e-08
4614 1.30316749432646e-08
4615 1.30309292973724e-08
4616 1.30302595748172e-08
4617 1.30295414171033e-08
4618 1.30288181286881e-08
4619 1.30280968705793e-08
4620 1.30274098706501e-08
4621 1.30267444399257e-08
4622 1.30260183566627e-08
4623 1.30253022912474e-08
4624 1.3024598666575e-08
4625 1.30238829210982e-08
4626 1.30231945103393e-08
4627 1.30224897303827e-08
4628 1.30217746685118e-08
4629 1.30210743997244e-08
4630 1.30203548747432e-08
4631 1.30196657982112e-08
4632 1.30189479792264e-08
4633 1.30182265481171e-08
4634 1.30175171619423e-08
4635 1.30168090943378e-08
4636 1.3016136377858e-08
4637 1.30154081176975e-08
4638 1.30146977919826e-08
4639 1.30139985209249e-08
4640 1.3013293232278e-08
4641 1.30125940347309e-08
4642 1.30118793008754e-08
4643 1.30111834630159e-08
4644 1.3010460966037e-08
4645 1.30097878704716e-08
4646 1.30090948197992e-08
4647 1.30083982721324e-08
4648 1.30076579043376e-08
4649 1.30069871385752e-08
4650 1.30062748913695e-08
4651 1.30055570472659e-08
4652 1.30048860931126e-08
4653 1.30041666063646e-08
4654 1.30034841234333e-08
4655 1.30027688277495e-08
4656 1.30020582492923e-08
4657 1.30013743181862e-08
4658 1.30006425575063e-08
4659 1.29999660069685e-08
4660 1.2999240197667e-08
4661 1.29985538264571e-08
4662 1.29978374700107e-08
4663 1.2997131618106e-08
4664 1.29964406344885e-08
4665 1.29957497876226e-08
4666 1.29950373231602e-08
4667 1.29943317326298e-08
4668 1.2993656507046e-08
4669 1.29929354428099e-08
4670 1.29922455775616e-08
4671 1.29915347465287e-08
4672 1.29908125225536e-08
4673 1.29901425431622e-08
4674 1.29894236250844e-08
4675 1.29886993835288e-08
4676 1.29880222806272e-08
4677 1.29873098069222e-08
4678 1.29865888935377e-08
4679 1.29859427581841e-08
4680 1.29851869666853e-08
4681 1.29844981015259e-08
4682 1.29838099208052e-08
4683 1.29831090394661e-08
4684 1.29824226749592e-08
4685 1.29816885032635e-08
4686 1.29809777601603e-08
4687 1.29803144739565e-08
4688 1.2979584908257e-08
4689 1.29788884366128e-08
4690 1.2978227977925e-08
4691 1.29775317758013e-08
4692 1.29768357330501e-08
4693 1.29761192689259e-08
4694 1.29754042845209e-08
4695 1.29746998308033e-08
4696 1.29740174530102e-08
4697 1.29733200353171e-08
4698 1.29726390862145e-08
4699 1.29719180696763e-08
4700 1.29712175966218e-08
4701 1.29705264606678e-08
4702 1.29698044911697e-08
4703 1.2969134609353e-08
4704 1.29684147688464e-08
4705 1.29677381745658e-08
4706 1.2967065156716e-08
4707 1.29663278527065e-08
4708 1.29656470168327e-08
4709 1.29649560530326e-08
4710 1.29642644668276e-08
4711 1.29635701592579e-08
4712 1.29628756897759e-08
4713 1.29621678595093e-08
4714 1.29614819435325e-08
4715 1.29608013646754e-08
4716 1.29600884900827e-08
4717 1.2959385594119e-08
4718 1.29587070136356e-08
4719 1.29579905326221e-08
4720 1.29573034666902e-08
4721 1.29565999601594e-08
4722 1.29559122242079e-08
4723 1.29552245568126e-08
4724 1.29545056010288e-08
4725 1.29538296596426e-08
4726 1.29531231415486e-08
4727 1.29524216661431e-08
4728 1.29517389586137e-08
4729 1.29510241786146e-08
4730 1.29503629842792e-08
4731 1.2949651065311e-08
4732 1.2948942662197e-08
4733 1.29482734973207e-08
4734 1.29475516021521e-08
4735 1.2946874519873e-08
4736 1.29461662592839e-08
4737 1.29454662667478e-08
4738 1.29447763222573e-08
4739 1.29440700257638e-08
4740 1.29433954324382e-08
4741 1.29426955118583e-08
4742 1.29419984591511e-08
4743 1.29413187271304e-08
4744 1.29406129651261e-08
4745 1.29399087085563e-08
4746 1.2939235527476e-08
4747 1.29385512117192e-08
4748 1.29378152744636e-08
4749 1.29371311078097e-08
4750 1.29364369838292e-08
4751 1.29357626088428e-08
4752 1.29350409980855e-08
4753 1.29343511716812e-08
4754 1.29336487647014e-08
4755 1.29329715379406e-08
4756 1.29322783183755e-08
4757 1.29315985861744e-08
4758 1.29308884097429e-08
4759 1.29301869251502e-08
4760 1.29295192869139e-08
4761 1.2928821109412e-08
4762 1.29281165038575e-08
4763 1.29274013645347e-08
4764 1.29267433761487e-08
4765 1.29260455945801e-08
4766 1.29253460257744e-08
4767 1.29246633203683e-08
4768 1.29239606389969e-08
4769 1.2923280271665e-08
4770 1.29225757280055e-08
4771 1.2921891992479e-08
4772 1.2921177582309e-08
4773 1.29204922549586e-08
4774 1.29197990243607e-08
4775 1.29191293038594e-08
4776 1.29184270730026e-08
4777 1.29177451803769e-08
4778 1.29170549244412e-08
4779 1.29163183008041e-08
4780 1.29156652444035e-08
4781 1.29149693556541e-08
4782 1.29142905621604e-08
4783 1.29135791561708e-08
4784 1.29128492013103e-08
4785 1.29121933141602e-08
4786 1.29115022194232e-08
4787 1.29108134232087e-08
4788 1.29101163352796e-08
4789 1.29094360079157e-08
4790 1.29087318404553e-08
4791 1.29080387750585e-08
4792 1.2907354253619e-08
4793 1.29066714153192e-08
4794 1.29059834713535e-08
4795 1.29053072271401e-08
4796 1.29045902373115e-08
4797 1.2903923833546e-08
4798 1.29032256957068e-08
4799 1.29025346133488e-08
4800 1.29018476544146e-08
4801 1.2901166433224e-08
4802 1.29004778220143e-08
4803 1.28997904332712e-08
4804 1.28991092346875e-08
4805 1.2898421191565e-08
4806 1.28977138936781e-08
4807 1.28970423796809e-08
4808 1.2896350626429e-08
4809 1.28956784752332e-08
4810 1.28949648161431e-08
4811 1.28942788680114e-08
4812 1.28935953909587e-08
4813 1.2892915104909e-08
4814 1.28922025484368e-08
4815 1.28915188949003e-08
4816 1.28908561828206e-08
4817 1.28901442638663e-08
4818 1.28894526604806e-08
4819 1.28887724250987e-08
4820 1.2888070311981e-08
4821 1.28873841177407e-08
4822 1.28867312038927e-08
4823 1.28860286348204e-08
4824 1.28853538617496e-08
4825 1.28846843128611e-08
4826 1.28839751313697e-08
4827 1.28833082082142e-08
4828 1.28826119266401e-08
4829 1.28819179263939e-08
4830 1.28812295512176e-08
4831 1.28805579102526e-08
4832 1.2879883335315e-08
4833 1.28791985186116e-08
4834 1.28785183250019e-08
4835 1.28778324060552e-08
4836 1.28771401848721e-08
4837 1.28764549207766e-08
4838 1.28757863254308e-08
4839 1.28750955639967e-08
4840 1.28744050521684e-08
4841 1.28737242346272e-08
4842 1.28730463736793e-08
4843 1.28723755858928e-08
4844 1.28716532696582e-08
4845 1.28709881104666e-08
4846 1.28702922290586e-08
4847 1.28696189138355e-08
4848 1.28689411525024e-08
4849 1.28682631869298e-08
4850 1.28675698339298e-08
4851 1.28668976448199e-08
4852 1.2866212144122e-08
4853 1.28655161648478e-08
4854 1.28648704637718e-08
4855 1.28641738296326e-08
4856 1.28634933299482e-08
4857 1.28628096989353e-08
4858 1.28621352168956e-08
4859 1.28614573869645e-08
4860 1.28607395715324e-08
4861 1.28600645667165e-08
4862 1.28593891492723e-08
4863 1.28587162696869e-08
4864 1.2858031439078e-08
4865 1.28573368955442e-08
4866 1.28566618423503e-08
4867 1.28559559193359e-08
4868 1.28553029373063e-08
4869 1.28546269283769e-08
4870 1.28539262344296e-08
4871 1.28532757985994e-08
4872 1.28525898755727e-08
4873 1.28518940459565e-08
4874 1.2851224323096e-08
4875 1.28505317089772e-08
4876 1.28498351497919e-08
4877 1.28491995391167e-08
4878 1.28484759633479e-08
4879 1.28478260250503e-08
4880 1.2847137136146e-08
4881 1.28464667370765e-08
4882 1.28457700753343e-08
4883 1.28451120712803e-08
4884 1.28444006749634e-08
4885 1.28437367739459e-08
4886 1.28430830721032e-08
4887 1.28423587959642e-08
4888 1.28416854384972e-08
4889 1.28409908399102e-08
4890 1.28403405002531e-08
4891 1.2839663413991e-08
4892 1.28389797273976e-08
4893 1.28383053012854e-08
4894 1.28376299378813e-08
4895 1.2836919760173e-08
4896 1.28362687044498e-08
4897 1.28355824942222e-08
4898 1.28348812497014e-08
4899 1.28342120455788e-08
4900 1.28335349061925e-08
4901 1.28328638620973e-08
4902 1.28322074637727e-08
4903 1.28315106428939e-08
4904 1.28308321281495e-08
4905 1.28301324503477e-08
4906 1.28294508241339e-08
4907 1.28287916198594e-08
4908 1.28280937224251e-08
4909 1.28274182242399e-08
4910 1.28267460032527e-08
4911 1.28260677246667e-08
4912 1.282539315553e-08
4913 1.28247298453732e-08
4914 1.28240359416193e-08
4915 1.2823371189033e-08
4916 1.28226774411683e-08
4917 1.28219651416711e-08
4918 1.28213516071979e-08
4919 1.28206630909677e-08
4920 1.28199736275786e-08
4921 1.28193064365817e-08
4922 1.2818621137764e-08
4923 1.28179311303517e-08
4924 1.28172524919562e-08
4925 1.28165974282307e-08
4926 1.28159310424369e-08
4927 1.28152421537131e-08
4928 1.28145876113067e-08
4929 1.28138968665537e-08
4930 1.28131963396394e-08
4931 1.28125395821854e-08
4932 1.28118594540927e-08
4933 1.28112041528211e-08
4934 1.2810509110936e-08
4935 1.2809849435691e-08
4936 1.28091465644298e-08
4937 1.28084819564361e-08
4938 1.28078074861926e-08
4939 1.28071298815674e-08
4940 1.28064267541222e-08
4941 1.28057572757745e-08
4942 1.28050850543016e-08
4943 1.28044250417708e-08
4944 1.28037245105544e-08
4945 1.28030584445049e-08
4946 1.2802384947927e-08
4947 1.28016894634098e-08
4948 1.28010245140087e-08
4949 1.28003512997465e-08
4950 1.27996857271634e-08
4951 1.2798993629784e-08
4952 1.27983393756054e-08
4953 1.27976614936742e-08
4954 1.27969886165313e-08
4955 1.27962896423195e-08
4956 1.27956227051751e-08
4957 1.27949592011317e-08
4958 1.27942668859821e-08
4959 1.27936081236041e-08
4960 1.27929335539262e-08
4961 1.27922534024355e-08
4962 1.27915677059082e-08
4963 1.27908943286098e-08
4964 1.27902330438329e-08
4965 1.27895636761327e-08
4966 1.27888820869587e-08
4967 1.27882206797658e-08
4968 1.27875366729563e-08
4969 1.27868538185444e-08
4970 1.27861433605603e-08
4971 1.27854977496622e-08
4972 1.27848433188749e-08
4973 1.27841600630618e-08
4974 1.27835053473219e-08
4975 1.27828191391205e-08
4976 1.27821578499443e-08
4977 1.27814564516993e-08
4978 1.27808054909556e-08
4979 1.27801405639394e-08
4980 1.27794381688945e-08
4981 1.27787803124585e-08
4982 1.27781225233298e-08
4983 1.27774449730639e-08
4984 1.27767763914988e-08
4985 1.27761053778791e-08
4986 1.27754260744878e-08
4987 1.2774769271931e-08
4988 1.27740786473179e-08
4989 1.27734049500255e-08
4990 1.27727437678332e-08
4991 1.27720855127716e-08
4992 1.27714021166125e-08
4993 1.27707402081539e-08
4994 1.27700709128403e-08
4995 1.27694118653848e-08
4996 1.27687282028416e-08
4997 1.27680645510414e-08
4998 1.27673869165096e-08
4999 1.27667379901664e-08
};
\addlegendentry{Train}
\addplot [semithick, black]
table {%
0 0.000918746809475124
1 0.000196227410924621
2 0.00018483295571059
3 0.000167378282640129
4 0.000127059305668809
5 5.82199209020473e-05
6 3.59959849447478e-05
7 3.3619136956986e-05
8 3.2382315112045e-05
9 3.12153060804121e-05
10 2.97666229016613e-05
11 2.753862827376e-05
12 2.36605119425803e-05
13 1.72897016454954e-05
14 1.00806873888359e-05
15 6.34127400189755e-06
16 5.49731930732378e-06
17 5.20230059919413e-06
18 4.96984557685209e-06
19 4.73673526357743e-06
20 4.48806258646073e-06
21 4.2167766878265e-06
22 3.91517096431926e-06
23 3.58283432433382e-06
24 3.22358982884907e-06
25 2.85011356027098e-06
26 2.48535570790409e-06
27 2.1606897462334e-06
28 1.90699824997864e-06
29 1.73121213720151e-06
30 1.62117976287846e-06
31 1.55483064645523e-06
32 1.51446886320628e-06
33 1.48693152368651e-06
34 1.46548643442657e-06
35 1.44708030802576e-06
36 1.43036470490188e-06
37 1.41473174153361e-06
38 1.39985183977842e-06
39 1.38548398354033e-06
40 1.37152392198914e-06
41 1.35797199618537e-06
42 1.34481751956628e-06
43 1.33200751406548e-06
44 1.31948843318241e-06
45 1.30720445667976e-06
46 1.29509021462582e-06
47 1.28308249713882e-06
48 1.27112241443683e-06
49 1.25917483728699e-06
50 1.2472240769057e-06
51 1.23525114759104e-06
52 1.22324399853824e-06
53 1.21118512197427e-06
54 1.19906690088101e-06
55 1.18686705263826e-06
56 1.17458716886176e-06
57 1.16220985546533e-06
58 1.14972226583632e-06
59 1.13710586902016e-06
60 1.12432439891563e-06
61 1.1113434084109e-06
62 1.09813277049398e-06
63 1.08466247183969e-06
64 1.0709139814935e-06
65 1.05686444840103e-06
66 1.04250636923098e-06
67 1.02782280464453e-06
68 1.01280681974458e-06
69 9.97467168417643e-07
70 9.81824655355013e-07
71 9.65907133831934e-07
72 9.49737284372532e-07
73 9.33279579840018e-07
74 9.16482406410069e-07
75 8.99330359516171e-07
76 8.81863684298878e-07
77 8.64150251800311e-07
78 8.46255545638996e-07
79 8.28252268547658e-07
80 8.10205619927729e-07
81 7.92193247889372e-07
82 7.74307181927725e-07
83 7.56641213683906e-07
84 7.39302890906401e-07
85 7.22407833109173e-07
86 7.0607887892038e-07
87 6.90434035277576e-07
88 6.7550735138866e-07
89 6.61306046367827e-07
90 6.47884746740601e-07
91 6.3528653981848e-07
92 6.2358498098547e-07
93 6.12663029642135e-07
94 6.02273871663783e-07
95 5.92621518080705e-07
96 5.83868029480072e-07
97 5.75792796553287e-07
98 5.6828997685443e-07
99 5.61289027700695e-07
100 5.54719349565858e-07
101 5.48514208276174e-07
102 5.4260846127363e-07
103 5.36958452812542e-07
104 5.31515411239525e-07
105 5.26242160958645e-07
106 5.21106983342179e-07
107 5.1607622708616e-07
108 5.11141536208015e-07
109 5.06286596646532e-07
110 5.01498107041698e-07
111 4.96776124236931e-07
112 4.92107290028798e-07
113 4.87488989620033e-07
114 4.82908546928229e-07
115 4.78368406220397e-07
116 4.73862712624395e-07
117 4.69391551405351e-07
118 4.64955093093522e-07
119 4.60545635405651e-07
120 4.56156953987374e-07
121 4.51783080279711e-07
122 4.47413157189658e-07
123 4.43032860175663e-07
124 4.38640057609518e-07
125 4.34217724887276e-07
126 4.29761087161751e-07
127 4.25258491532077e-07
128 4.20709255877227e-07
129 4.1610056200625e-07
130 4.11431756219827e-07
131 4.06710910283437e-07
132 4.01927962911941e-07
133 3.97083965708589e-07
134 3.92186848330311e-07
135 3.87228794807015e-07
136 3.8221477893785e-07
137 3.77138945850675e-07
138 3.72011953686524e-07
139 3.66838605714292e-07
140 3.61644680424433e-07
141 3.56442797055934e-07
142 3.51269392240283e-07
143 3.46132509321251e-07
144 3.41072080800586e-07
145 3.36086230845467e-07
146 3.31180018520172e-07
147 3.26362197711205e-07
148 3.21669460845442e-07
149 3.17125454785128e-07
150 3.12742940877797e-07
151 3.0851691690259e-07
152 3.04376783333282e-07
153 3.002228652349e-07
154 2.96008380473722e-07
155 2.91697375587319e-07
156 2.87206376015092e-07
157 2.82728422007494e-07
158 2.7868293273059e-07
159 2.7490960974319e-07
160 2.71322107892047e-07
161 2.68084107801769e-07
162 2.65281045130905e-07
163 2.62602156908542e-07
164 2.60056253864605e-07
165 2.57524902735895e-07
166 2.5491050337223e-07
167 2.52443356885124e-07
168 2.50324234229993e-07
169 2.48529119062368e-07
170 2.46981244345079e-07
171 2.4561100531173e-07
172 2.44361984869101e-07
173 2.43209200334604e-07
174 2.42139662987029e-07
175 2.4113913354995e-07
176 2.40201472934132e-07
177 2.39313067140756e-07
178 2.38466924429304e-07
179 2.37650979784121e-07
180 2.3686141048529e-07
181 2.36092247973829e-07
182 2.35337964227256e-07
183 2.34597195003516e-07
184 2.33863019616365e-07
185 2.33142529282304e-07
186 2.3242368740739e-07
187 2.31712817821972e-07
188 2.30999717132363e-07
189 2.30288435432158e-07
190 2.29575121579728e-07
191 2.28862177209521e-07
192 2.28148252290339e-07
193 2.2743093097688e-07
194 2.26713936513079e-07
195 2.25992323521496e-07
196 2.2527106580128e-07
197 2.24548358573884e-07
198 2.2382376130281e-07
199 2.23101309870799e-07
200 2.22377181557931e-07
201 2.21653380094722e-07
202 2.20928782823648e-07
203 2.20200618628041e-07
204 2.19476817164832e-07
205 2.18752262526323e-07
206 2.18025434151059e-07
207 2.17301462157593e-07
208 2.16574960631988e-07
209 2.15849780715871e-07
210 2.15124373426079e-07
211 2.14397260833721e-07
212 2.136725214541e-07
213 2.12945508337725e-07
214 2.12220655271267e-07
215 2.11496697488656e-07
216 2.10775596087842e-07
217 2.10050046689503e-07
218 2.09330124789631e-07
219 2.0861035920916e-07
220 2.07892497883222e-07
221 2.07175219202327e-07
222 2.06461464813401e-07
223 2.05748435178066e-07
224 2.05033643396746e-07
225 2.04323313823807e-07
226 2.03618711225317e-07
227 2.02909731683576e-07
228 2.02204645916026e-07
229 2.014976701048e-07
230 2.00793024873747e-07
231 2.00090653379448e-07
232 1.99387173438481e-07
233 1.98686024077688e-07
234 1.97982075178516e-07
235 1.97285487502086e-07
236 1.96585432377105e-07
237 1.95885576204091e-07
238 1.95187794815865e-07
239 1.9449296928542e-07
240 1.93795827385657e-07
241 1.93106515666841e-07
242 1.92417232369735e-07
243 1.91730663345879e-07
244 1.91050304465534e-07
245 1.90368098174076e-07
246 1.89692073604419e-07
247 1.89018081186987e-07
248 1.88350654184433e-07
249 1.87687163588635e-07
250 1.87024625120102e-07
251 1.86366477805677e-07
252 1.85712650591086e-07
253 1.85062646096412e-07
254 1.84413238457637e-07
255 1.8377262733793e-07
256 1.83134233111559e-07
257 1.82493550937579e-07
258 1.81858354153519e-07
259 1.8122933909126e-07
260 1.80603421995329e-07
261 1.79973980607429e-07
262 1.79353449425435e-07
263 1.7873345825592e-07
264 1.78117417704016e-07
265 1.77500638187666e-07
266 1.76887908764911e-07
267 1.76276799379593e-07
268 1.75669086388552e-07
269 1.75056527496054e-07
270 1.74451514567409e-07
271 1.7384890327321e-07
272 1.7324239820482e-07
273 1.72637683704124e-07
274 1.72034205547789e-07
275 1.71431466355898e-07
276 1.70829650869564e-07
277 1.70228659612803e-07
278 1.69626488855101e-07
279 1.69021575402439e-07
280 1.68424648450127e-07
281 1.67825305652514e-07
282 1.67229089242937e-07
283 1.6663516078097e-07
284 1.66042440241654e-07
285 1.65448867051055e-07
286 1.64862456131232e-07
287 1.64275490988075e-07
288 1.63694735988429e-07
289 1.63114279416732e-07
290 1.62534604442044e-07
291 1.61963541245314e-07
292 1.61392847530806e-07
293 1.60830524009725e-07
294 1.60268768922833e-07
295 1.59712527647571e-07
296 1.5915794904231e-07
297 1.58610745870646e-07
298 1.58066470135054e-07
299 1.57528347699554e-07
300 1.56993905875424e-07
301 1.56463542566598e-07
302 1.55940441004532e-07
303 1.55421176373238e-07
304 1.5490388705075e-07
305 1.54394953710835e-07
306 1.53890383103317e-07
307 1.53391880530762e-07
308 1.52896092231458e-07
309 1.5240574668951e-07
310 1.51923330804493e-07
311 1.51444098150932e-07
312 1.50969086121222e-07
313 1.50500312656732e-07
314 1.50033187651388e-07
315 1.49574489682891e-07
316 1.49118662307046e-07
317 1.48671489341723e-07
318 1.48224373219819e-07
319 1.47786622051171e-07
320 1.47351428836373e-07
321 1.46919404642176e-07
322 1.46494670616448e-07
323 1.46074825124742e-07
324 1.45659839745349e-07
325 1.45249572369721e-07
326 1.44842758231789e-07
327 1.44445962746431e-07
328 1.44051711004067e-07
329 1.43659534046492e-07
330 1.43278242603628e-07
331 1.42898187505125e-07
332 1.42526062063553e-07
333 1.4215643773241e-07
334 1.41790906127426e-07
335 1.41434824740827e-07
336 1.41080150228845e-07
337 1.407301084555e-07
338 1.40386703151307e-07
339 1.40046068963784e-07
340 1.39714074975927e-07
341 1.39385008424142e-07
342 1.39061370418858e-07
343 1.387391677099e-07
344 1.38421853534965e-07
345 1.38108816827298e-07
346 1.37800029165192e-07
347 1.37494055252319e-07
348 1.37192927240903e-07
349 1.36895778268808e-07
350 1.36598401923038e-07
351 1.36309878939755e-07
352 1.36020730678865e-07
353 1.35736627271399e-07
354 1.35457895567015e-07
355 1.35177899096561e-07
356 1.34904553306114e-07
357 1.34634703385927e-07
358 1.34366402448904e-07
359 1.34105874849411e-07
360 1.33847649408381e-07
361 1.33588557105213e-07
362 1.33338119212567e-07
363 1.33088676079751e-07
364 1.32841762479075e-07
365 1.32600106894643e-07
366 1.32355623350122e-07
367 1.32119566842448e-07
368 1.3188467562486e-07
369 1.31650480739154e-07
370 1.31420023308237e-07
371 1.3119034747433e-07
372 1.30962234834442e-07
373 1.30736253822761e-07
374 1.30514422380656e-07
375 1.30289834032737e-07
376 1.30070318959952e-07
377 1.29850278085542e-07
378 1.29629199818737e-07
379 1.29414644334247e-07
380 1.29195768749923e-07
381 1.28980587987826e-07
382 1.28764270357351e-07
383 1.28550482259016e-07
384 1.28337191540595e-07
385 1.28121953935079e-07
386 1.27907014757511e-07
387 1.27693837725928e-07
388 1.27476326383658e-07
389 1.2726536624541e-07
390 1.27048053855106e-07
391 1.26834734714976e-07
392 1.26616569673388e-07
393 1.26401005218213e-07
394 1.26182726489787e-07
395 1.25959275010246e-07
396 1.25739205714126e-07
397 1.25515612126037e-07
398 1.25290711139314e-07
399 1.25060523714637e-07
400 1.2482934153013e-07
401 1.24591949202113e-07
402 1.243495972858e-07
403 1.24099017284607e-07
404 1.23838034937762e-07
405 1.2357193668322e-07
406 1.23302285714999e-07
407 1.23040123867213e-07
408 1.22786545375675e-07
409 1.22546481406971e-07
410 1.22307085348439e-07
411 1.22069209851361e-07
412 1.21833338084798e-07
413 1.21594439406181e-07
414 1.21357999205429e-07
415 1.21126760177503e-07
416 1.2089164158624e-07
417 1.20657986713013e-07
418 1.20431352002015e-07
419 1.20220434496332e-07
420 1.20013027071764e-07
421 1.1976672453784e-07
422 1.19565157774559e-07
423 1.19338380955014e-07
424 1.19137574472461e-07
425 1.18900572942948e-07
426 1.18702260465398e-07
427 1.18478794775001e-07
428 1.18280894412237e-07
429 1.18045349495333e-07
430 1.17854469294798e-07
431 1.17633291552011e-07
432 1.17439590496815e-07
433 1.17209651762096e-07
434 1.17023247980796e-07
435 1.16798339888646e-07
436 1.16610003431106e-07
437 1.16388562787506e-07
438 1.16199046829024e-07
439 1.15981961812395e-07
440 1.15791273458399e-07
441 1.15576831660746e-07
442 1.15378725240589e-07
443 1.15169726200293e-07
444 1.14968578657226e-07
445 1.14758051950048e-07
446 1.14550417151804e-07
447 1.14338057244368e-07
448 1.14122862271415e-07
449 1.13905294085725e-07
450 1.13689779368542e-07
451 1.13474897034394e-07
452 1.13260469447596e-07
453 1.13051257244479e-07
454 1.12845022215424e-07
455 1.1264190646898e-07
456 1.12439757060656e-07
457 1.12239959548788e-07
458 1.12041135480467e-07
459 1.11844080663559e-07
460 1.11650081180414e-07
461 1.11455548790218e-07
462 1.11260582968953e-07
463 1.11071379649275e-07
464 1.10881245518613e-07
465 1.10692432997439e-07
466 1.10503080463786e-07
467 1.10316577206504e-07
468 1.1013029421747e-07
469 1.09945844428694e-07
470 1.09760726729746e-07
471 1.09578898843665e-07
472 1.0939665173737e-07
473 1.09215307020349e-07
474 1.09033251760593e-07
475 1.08854877112208e-07
476 1.08678889887415e-07
477 1.08499442319498e-07
478 1.08323057190773e-07
479 1.0814768813816e-07
480 1.07972731200334e-07
481 1.07797440307422e-07
482 1.07625837131309e-07
483 1.07455363718145e-07
484 1.07283725014895e-07
485 1.07114111358442e-07
486 1.06944078481774e-07
487 1.06776738562075e-07
488 1.06607409122716e-07
489 1.0643947234712e-07
490 1.06274647748705e-07
491 1.06110746855848e-07
492 1.05945566986065e-07
493 1.05782490322781e-07
494 1.05618305212829e-07
495 1.0545720385835e-07
496 1.05296784624898e-07
497 1.05136834349651e-07
498 1.04978347792439e-07
499 1.04820564672536e-07
500 1.04664316324943e-07
501 1.04508892206923e-07
502 1.04354484165015e-07
503 1.04198889516738e-07
504 1.04047536808594e-07
505 1.03894180369934e-07
506 1.03741768953114e-07
507 1.03592498135185e-07
508 1.03441763599221e-07
509 1.03291597497446e-07
510 1.03141118756866e-07
511 1.02993681139196e-07
512 1.02847053540245e-07
513 1.02699246440352e-07
514 1.02553720182641e-07
515 1.0240781733728e-07
516 1.02260187873071e-07
517 1.02117361677756e-07
518 1.01973448352055e-07
519 1.01828959486738e-07
520 1.01685763809201e-07
521 1.01542362074269e-07
522 1.01400019048015e-07
523 1.01259608698001e-07
524 1.01114949302428e-07
525 1.00974595795833e-07
526 1.008325440921e-07
527 1.00692986393369e-07
528 1.00551616810662e-07
529 1.00410197489964e-07
530 1.00270419522985e-07
531 1.00131344993315e-07
532 9.99922278310805e-08
533 9.98513201011519e-08
534 9.97115989775921e-08
535 9.95699096506542e-08
536 9.94310624946593e-08
537 9.92905029306712e-08
538 9.91472361988599e-08
539 9.9004076048459e-08
540 9.88578392480122e-08
541 9.87065149615773e-08
542 9.8550849259027e-08
543 9.83896413231378e-08
544 9.82253069992112e-08
545 9.80654135673831e-08
546 9.79130589939814e-08
547 9.77636531729331e-08
548 9.76194911572748e-08
549 9.74785336893547e-08
550 9.73371356849384e-08
551 9.719609295189e-08
552 9.70570752656386e-08
553 9.69195284028501e-08
554 9.67782156635622e-08
555 9.66395390378239e-08
556 9.65010400477695e-08
557 9.63654258612223e-08
558 9.62269623983047e-08
559 9.60894865897899e-08
560 9.59540074063625e-08
561 9.58175832010966e-08
562 9.56825942921569e-08
563 9.55487635678764e-08
564 9.54133838604321e-08
565 9.5278664957732e-08
566 9.51457863607175e-08
567 9.50110106145985e-08
568 9.48781035958746e-08
569 9.47465252920665e-08
570 9.46150748859509e-08
571 9.44830063076552e-08
572 9.43510514161972e-08
573 9.42196720643551e-08
574 9.40907867175156e-08
575 9.39598052696056e-08
576 9.3830323066868e-08
577 9.3699838998873e-08
578 9.357012942246e-08
579 9.34410522290818e-08
580 9.33135169134403e-08
581 9.31852213170714e-08
582 9.30564212353602e-08
583 9.29290351336931e-08
584 9.2800377160529e-08
585 9.26726784200582e-08
586 9.25472818380513e-08
587 9.24179204275788e-08
588 9.22920619927936e-08
589 9.21658340757858e-08
590 9.20368137258265e-08
591 9.19104294894169e-08
592 9.17839031444601e-08
593 9.16571210041184e-08
594 9.15301754389475e-08
595 9.1404906754633e-08
596 9.12772577521537e-08
597 9.11518824864288e-08
598 9.10239990048467e-08
599 9.08978137204031e-08
600 9.07715929088226e-08
601 9.06445620785234e-08
602 9.05185615351911e-08
603 9.03928807360899e-08
604 9.02673775726726e-08
605 9.01415546650242e-08
606 9.00149288440844e-08
607 8.98870879950664e-08
608 8.97625440643424e-08
609 8.96361527225054e-08
610 8.95110900955842e-08
611 8.93856508810131e-08
612 8.92591458523384e-08
613 8.91345166564861e-08
614 8.90099869366168e-08
615 8.8886601190552e-08
616 8.87619151512808e-08
617 8.86375204345313e-08
618 8.851396415821e-08
619 8.83922908201384e-08
620 8.82675266211663e-08
621 8.81484254477982e-08
622 8.80255726087853e-08
623 8.79059456337927e-08
624 8.77861054959794e-08
625 8.76653274417549e-08
626 8.75474910344565e-08
627 8.74299601605344e-08
628 8.73123155997746e-08
629 8.71951542080751e-08
630 8.70791012630434e-08
631 8.69643272949361e-08
632 8.68502638695645e-08
633 8.67357741185515e-08
634 8.66224851847619e-08
635 8.65078320089196e-08
636 8.63984439547494e-08
637 8.6283549194377e-08
638 8.61704023691345e-08
639 8.60554294490612e-08
640 8.59423323618103e-08
641 8.58294413319527e-08
642 8.57221209571435e-08
643 8.56117665648526e-08
644 8.55002397770477e-08
645 8.53901198638596e-08
646 8.52895709613222e-08
647 8.51840979976259e-08
648 8.50846078037648e-08
649 8.49808827751986e-08
650 8.48832826250145e-08
651 8.47776533419164e-08
652 8.4684316448147e-08
653 8.45809964289401e-08
654 8.44908001340627e-08
655 8.43830818553215e-08
656 8.42963316927126e-08
657 8.4189878180041e-08
658 8.41057357092723e-08
659 8.40003195889949e-08
660 8.39179463696382e-08
661 8.38108320522224e-08
662 8.37313294255182e-08
663 8.3622559543528e-08
664 8.35484357253335e-08
665 8.34371789437682e-08
666 8.33665652066884e-08
667 8.32545836715326e-08
668 8.31859949812497e-08
669 8.30740702895127e-08
670 8.30081319236342e-08
671 8.289544695117e-08
672 8.28324928647817e-08
673 8.27193673558213e-08
674 8.26589925395638e-08
675 8.25437069806867e-08
676 8.24842985025498e-08
677 8.23696097995708e-08
678 8.23133916583174e-08
679 8.2196848438798e-08
680 8.21379231297215e-08
681 8.20283077018757e-08
682 8.19754362169078e-08
683 8.18581185058065e-08
684 8.18025824855795e-08
685 8.16914251799972e-08
686 8.16426108940504e-08
687 8.1525847406283e-08
688 8.14751714983686e-08
689 8.13623373119299e-08
690 8.13075473615754e-08
691 8.11982943105249e-08
692 8.11432059322215e-08
693 8.10349334301463e-08
694 8.09733151641012e-08
695 8.08922493433784e-08
696 8.07985287565316e-08
697 8.07424598292528e-08
698 8.06364610639321e-08
699 8.05859130537101e-08
700 8.04762194661635e-08
701 8.03942512561662e-08
702 8.03494089041124e-08
703 8.02328088411741e-08
704 8.0152950943102e-08
705 8.00657247168601e-08
706 7.99758410607865e-08
707 7.9931680829759e-08
708 7.98233301679829e-08
709 7.97676875663456e-08
710 7.9667493935176e-08
711 7.96192622942726e-08
712 7.95154875277149e-08
713 7.9459397284154e-08
714 7.93601699911051e-08
715 7.9323150714572e-08
716 7.92094709822777e-08
717 7.91500553987134e-08
718 7.90583953858004e-08
719 7.90242822290566e-08
720 7.89073268947504e-08
721 7.88704923593286e-08
722 7.87574307992145e-08
723 7.87171572369516e-08
724 7.86093607985094e-08
725 7.85568232686273e-08
726 7.84612481652402e-08
727 7.84115314900191e-08
728 7.83114160185505e-08
729 7.82461029302794e-08
730 7.81630902224606e-08
731 7.81013369532957e-08
732 7.80179973958184e-08
733 7.79487479007912e-08
734 7.78704603021652e-08
735 7.78010758040182e-08
736 7.77262698647974e-08
737 7.76717570261098e-08
738 7.75799975372138e-08
739 7.75085524651331e-08
740 7.7435629464162e-08
741 7.73905739492875e-08
742 7.72950627947466e-08
743 7.72454171737991e-08
744 7.71482220329744e-08
745 7.71244259567538e-08
746 7.70081740597561e-08
747 7.69618324625299e-08
748 7.68693837471801e-08
749 7.68094210457093e-08
750 7.67234453746823e-08
751 7.66534284935005e-08
752 7.65876748687333e-08
753 7.65275416370059e-08
754 7.64542988918038e-08
755 7.63773186918115e-08
756 7.63045804319518e-08
757 7.62385852226544e-08
758 7.61697549478413e-08
759 7.61285434691672e-08
760 7.60311280600945e-08
761 7.59704974484521e-08
762 7.5891314565979e-08
763 7.58241043286034e-08
764 7.57563185516119e-08
765 7.56884617203468e-08
766 7.56224736164768e-08
767 7.55556399667512e-08
768 7.54871933850154e-08
769 7.54232729605064e-08
770 7.53578390799703e-08
771 7.52911972767834e-08
772 7.52270210568895e-08
773 7.51608340010534e-08
774 7.50964730400483e-08
775 7.50318207565215e-08
776 7.49675663769267e-08
777 7.4901585378484e-08
778 7.48400310612851e-08
779 7.47764019592978e-08
780 7.47135331380377e-08
781 7.46497974546401e-08
782 7.45879304986374e-08
783 7.45253458944717e-08
784 7.44625197057758e-08
785 7.43997645713534e-08
786 7.43376347145386e-08
787 7.42758317073822e-08
788 7.42122878705231e-08
789 7.41489500910575e-08
790 7.40863868031738e-08
791 7.40244630037523e-08
792 7.39619778755696e-08
793 7.38991658977284e-08
794 7.38374978936918e-08
795 7.37725400767886e-08
796 7.37120586791207e-08
797 7.36504830456397e-08
798 7.35887724090389e-08
799 7.35261664885911e-08
800 7.3466019046009e-08
801 7.34026386339792e-08
802 7.33437488520394e-08
803 7.32787768242815e-08
804 7.32281435489313e-08
805 7.31521225816323e-08
806 7.31230755945944e-08
807 7.30545508531577e-08
808 7.29827718259912e-08
809 7.29225462237082e-08
810 7.28540641148356e-08
811 7.28293798601953e-08
812 7.27222300156427e-08
813 7.26844007203908e-08
814 7.26346485180329e-08
815 7.25803346313114e-08
816 7.24905504512208e-08
817 7.24297422038944e-08
818 7.24244770822224e-08
819 7.23095112675765e-08
820 7.22842656841749e-08
821 7.21942470249815e-08
822 7.21973236750273e-08
823 7.21144601811829e-08
824 7.20232833373302e-08
825 7.20400024079026e-08
826 7.1942011459214e-08
827 7.18877970484755e-08
828 7.18291133239291e-08
829 7.17388601856328e-08
830 7.17020043339289e-08
831 7.17147656814632e-08
832 7.16085750696038e-08
833 7.15472836532172e-08
834 7.14628356490721e-08
835 7.14233792109553e-08
836 7.14260295353597e-08
837 7.1297300507922e-08
838 7.12697314497746e-08
839 7.12638694722045e-08
840 7.1216668118268e-08
841 7.11112235762812e-08
842 7.10468768261308e-08
843 7.0969171872548e-08
844 7.09714029767383e-08
845 7.09840719537169e-08
846 7.0936962970336e-08
847 7.0889036862809e-08
848 7.08463829823813e-08
849 7.07710370306813e-08
850 7.07435887647989e-08
851 7.06249352333543e-08
852 7.06427769614493e-08
853 7.04410894059038e-08
854 7.03882818697821e-08
855 7.03319926742552e-08
856 7.0263816098759e-08
857 7.02884364045531e-08
858 7.01548543702302e-08
859 7.02981353128962e-08
860 7.00458073765731e-08
861 7.0093889803502e-08
862 6.99519659974612e-08
863 6.98985758162962e-08
864 6.99883315746774e-08
865 6.97907296398625e-08
866 6.97488218293074e-08
867 6.97131099514081e-08
868 6.96382116416316e-08
869 6.95725717037021e-08
870 6.96292303814516e-08
871 6.94752984031766e-08
872 6.94340513973657e-08
873 6.93719002242688e-08
874 6.93303476850815e-08
875 6.92676351832233e-08
876 6.92225796683488e-08
877 6.91629935545279e-08
878 6.91252068918402e-08
879 6.90647610213091e-08
880 6.90578687567722e-08
881 6.89667913889025e-08
882 6.89233701223202e-08
883 6.88633932099947e-08
884 6.88257202341447e-08
885 6.87643648689118e-08
886 6.8724808954812e-08
887 6.86689460849266e-08
888 6.86246011127878e-08
889 6.8571175404486e-08
890 6.85236400954636e-08
891 6.84723104882323e-08
892 6.84248675497656e-08
893 6.83741774309965e-08
894 6.83255478861611e-08
895 6.82752414604693e-08
896 6.82267469187536e-08
897 6.81781955336191e-08
898 6.81292462445526e-08
899 6.80795508856136e-08
900 6.80294789390246e-08
901 6.79790161939309e-08
902 6.79262299740913e-08
903 6.78715110780104e-08
904 6.78234499673636e-08
905 6.77754883326998e-08
906 6.77274059057709e-08
907 6.76786910958072e-08
908 6.76315679015715e-08
909 6.75824480822484e-08
910 6.7535765424509e-08
911 6.74880666906574e-08
912 6.74416540391576e-08
913 6.73940121487249e-08
914 6.73472158041477e-08
915 6.73011513185884e-08
916 6.72531967893519e-08
917 6.72063933393474e-08
918 6.71579769573327e-08
919 6.71108750793792e-08
920 6.70637092525794e-08
921 6.70174102879173e-08
922 6.69703439371006e-08
923 6.69230644234631e-08
924 6.68757209609794e-08
925 6.68280080162731e-08
926 6.67822348532354e-08
927 6.67357440420346e-08
928 6.66897008727574e-08
929 6.66437500740358e-08
930 6.66048833863897e-08
931 6.65635866425873e-08
932 6.65108927933034e-08
933 6.64751098611305e-08
934 6.64144437223513e-08
935 6.63842101289447e-08
936 6.63235226738834e-08
937 6.62986892052686e-08
938 6.62311947507987e-08
939 6.61962005210626e-08
940 6.61597923112822e-08
941 6.60935697283094e-08
942 6.60657946127685e-08
943 6.60112178252348e-08
944 6.5987656228117e-08
945 6.59201333519377e-08
946 6.58917613804988e-08
947 6.58218652915821e-08
948 6.580213351981e-08
949 6.57301981732417e-08
950 6.57195471376326e-08
951 6.56446843549929e-08
952 6.56301892831834e-08
953 6.55541754213118e-08
954 6.55404619465116e-08
955 6.54629488394676e-08
956 6.54547562817243e-08
957 6.53752465495927e-08
958 6.5368197965654e-08
959 6.52878071605301e-08
960 6.52826273039864e-08
961 6.52015330615541e-08
962 6.51964029430019e-08
963 6.51150955377489e-08
964 6.511017147659e-08
965 6.50286224868069e-08
966 6.50246079203498e-08
967 6.49418652187705e-08
968 6.49382698725276e-08
969 6.48547739956484e-08
970 6.48525642077402e-08
971 6.47688622734677e-08
972 6.47666311692774e-08
973 6.46817142069267e-08
974 6.46804139137203e-08
975 6.45956603761988e-08
976 6.45942037635905e-08
977 6.45105799890189e-08
978 6.45078657157683e-08
979 6.44251159087617e-08
980 6.44217692524762e-08
981 6.43404050038043e-08
982 6.43361985908086e-08
983 6.42571933440195e-08
984 6.42503508174741e-08
985 6.41733564066271e-08
986 6.41642046161905e-08
987 6.40905639670564e-08
988 6.40787831684975e-08
989 6.40067767676555e-08
990 6.39937596247364e-08
991 6.39233235233405e-08
992 6.39093116205913e-08
993 6.38382573470153e-08
994 6.38245296613604e-08
995 6.37541290871013e-08
996 6.37396624370012e-08
997 6.36702210954354e-08
998 6.36553494359759e-08
999 6.35861212572308e-08
1000 6.35723083064477e-08
1001 6.35025898532149e-08
1002 6.3486993440165e-08
1003 6.34188950243697e-08
1004 6.34035473012773e-08
1005 6.33353636203537e-08
1006 6.33214511935876e-08
1007 6.32537648925791e-08
1008 6.32354968388427e-08
1009 6.31707237630508e-08
1010 6.31527683481181e-08
1011 6.30880165886083e-08
1012 6.30691232572644e-08
1013 6.30061052220299e-08
1014 6.29858192269239e-08
1015 6.2924442545409e-08
1016 6.29036378541059e-08
1017 6.28427514470786e-08
1018 6.28195806484655e-08
1019 6.27614582526803e-08
1020 6.2737072425989e-08
1021 6.26796534675123e-08
1022 6.26541307724438e-08
1023 6.25978344714895e-08
1024 6.25704004164618e-08
1025 6.25180334168363e-08
1026 6.24887377398409e-08
1027 6.24366478518823e-08
1028 6.24052063358249e-08
1029 6.23558307211169e-08
1030 6.23237497165974e-08
1031 6.22747293732573e-08
1032 6.22419946694208e-08
1033 6.21938553990731e-08
1034 6.21595219740811e-08
1035 6.21140543444199e-08
1036 6.20787474758799e-08
1037 6.20323774569442e-08
1038 6.19958626657535e-08
1039 6.19525266642995e-08
1040 6.19156494963136e-08
1041 6.18717805878077e-08
1042 6.18334468072135e-08
1043 6.17922424339667e-08
1044 6.17519191337124e-08
1045 6.17122708490569e-08
1046 6.16697164446123e-08
1047 6.1632555059532e-08
1048 6.1589318534061e-08
1049 6.15518658264591e-08
1050 6.15074213783373e-08
1051 6.14719652958229e-08
1052 6.14259434428277e-08
1053 6.13935497995044e-08
1054 6.13453821074472e-08
1055 6.13130026749786e-08
1056 6.12653465736912e-08
1057 6.12339832173348e-08
1058 6.11839183761731e-08
1059 6.11547719131522e-08
1060 6.11030372965615e-08
1061 6.10752906027301e-08
1062 6.10230728170791e-08
1063 6.09962782505136e-08
1064 6.09406001217394e-08
1065 6.09184738209478e-08
1066 6.08606285368296e-08
1067 6.08401364843303e-08
1068 6.07805361596547e-08
1069 6.07617138825844e-08
1070 6.06999606134195e-08
1071 6.06818915116492e-08
1072 6.06196337571419e-08
1073 6.06048260465286e-08
1074 6.05401879738565e-08
1075 6.05300911615814e-08
1076 6.04601879672373e-08
1077 6.04804597514885e-08
1078 6.03797190024125e-08
1079 6.03809340304906e-08
1080 6.02881868871918e-08
1081 6.0330791029628e-08
1082 6.0215924690965e-08
1083 6.02457035370207e-08
1084 6.01343117523356e-08
1085 6.01513931997033e-08
1086 6.00582907850367e-08
1087 6.00848579779267e-08
1088 5.99689613522969e-08
1089 6.0004310853401e-08
1090 5.99006853008177e-08
1091 5.99285598923416e-08
1092 5.98140204033371e-08
1093 5.98479843461064e-08
1094 5.97393992052275e-08
1095 5.9774002636459e-08
1096 5.96582481193764e-08
1097 5.96907128169732e-08
1098 5.95791362911768e-08
1099 5.961091176232e-08
1100 5.94992179969722e-08
1101 5.95368625511128e-08
1102 5.94237477002935e-08
1103 5.94645683804629e-08
1104 5.93371183299496e-08
1105 5.93967044437704e-08
1106 5.92617048766897e-08
1107 5.93002162929679e-08
1108 5.92055577897099e-08
1109 5.92245825714599e-08
1110 5.91143489714341e-08
1111 5.91336828392741e-08
1112 5.90417883472583e-08
1113 5.907138955763e-08
1114 5.89534749906306e-08
1115 5.89910520432113e-08
1116 5.88637938392367e-08
1117 5.89212199031408e-08
1118 5.87999977597065e-08
1119 5.8837642313847e-08
1120 5.87067674473474e-08
1121 5.87619695124886e-08
1122 5.86296344806669e-08
1123 5.86743240660326e-08
1124 5.85631063643177e-08
1125 5.86036605909612e-08
1126 5.84751127519212e-08
1127 5.85354769100377e-08
1128 5.83950914290199e-08
1129 5.84433337280643e-08
1130 5.83304853307709e-08
1131 5.83734696135707e-08
1132 5.82424171113871e-08
1133 5.83022270461697e-08
1134 5.8174052242066e-08
1135 5.82265329285292e-08
1136 5.81017793876981e-08
1137 5.81518300180051e-08
1138 5.8041930373065e-08
1139 5.80806052141725e-08
1140 5.79679202417083e-08
1141 5.80024703822346e-08
1142 5.78820937846558e-08
1143 5.7920992446725e-08
1144 5.77815342239774e-08
1145 5.78489860458831e-08
1146 5.77253125300103e-08
1147 5.77741872120896e-08
1148 5.76361074422493e-08
1149 5.77010261793021e-08
1150 5.75829552929008e-08
1151 5.76245824390753e-08
1152 5.75102774291736e-08
1153 5.75502738797695e-08
1154 5.74103040662521e-08
1155 5.75023229032467e-08
1156 5.73343612586541e-08
1157 5.74275311748806e-08
1158 5.7258507268898e-08
1159 5.73373775125674e-08
1160 5.71822980077741e-08
1161 5.72513556562626e-08
1162 5.71310181385343e-08
1163 5.71615359490352e-08
1164 5.70279965472764e-08
1165 5.70915190678534e-08
1166 5.69898439550798e-08
1167 5.70222873363946e-08
1168 5.68822144941805e-08
1169 5.69748337397868e-08
1170 5.68102009879112e-08
1171 5.68607028128554e-08
1172 5.67632127967954e-08
1173 5.67742155510587e-08
1174 5.6658869596049e-08
1175 5.67104123661011e-08
1176 5.66111566513428e-08
1177 5.66386368916483e-08
1178 5.64946880388106e-08
1179 5.65499256310886e-08
1180 5.6437787776531e-08
1181 5.6490115696306e-08
1182 5.63865079072912e-08
1183 5.64429143423695e-08
1184 5.6267882797556e-08
1185 5.63336541858916e-08
1186 5.61992976599868e-08
1187 5.62838948781064e-08
1188 5.61546507071853e-08
1189 5.61910233898288e-08
1190 5.60902542190433e-08
1191 5.61310855573538e-08
1192 5.59943487132841e-08
1193 5.60864670262617e-08
1194 5.5917887209489e-08
1195 5.60148940564886e-08
1196 5.58204718004163e-08
1197 5.5941512755453e-08
1198 5.57460779759822e-08
1199 5.58722810239942e-08
1200 5.5669119092272e-08
1201 5.57936381540003e-08
1202 5.56278543228927e-08
1203 5.57283073021608e-08
1204 5.55286909786901e-08
1205 5.56444845756232e-08
1206 5.5501050866269e-08
1207 5.55501173948869e-08
1208 5.54109647055157e-08
1209 5.54907337857458e-08
1210 5.53261649827164e-08
1211 5.54136398989158e-08
1212 5.52510428519781e-08
1213 5.53394841062982e-08
1214 5.51750360955339e-08
1215 5.52641132856024e-08
1216 5.50991394732137e-08
1217 5.51907675117036e-08
1218 5.50236727292486e-08
1219 5.5116711195069e-08
1220 5.49481704581467e-08
1221 5.50436745072602e-08
1222 5.48741851957857e-08
1223 5.49710570396655e-08
1224 5.4798050541649e-08
1225 5.48972707292705e-08
1226 5.47252447802293e-08
1227 5.4824905504347e-08
1228 5.46513767574197e-08
1229 5.47547109874813e-08
1230 5.45790683759151e-08
1231 5.46789244992851e-08
1232 5.45069767099449e-08
1233 5.46068399387423e-08
1234 5.44299432192474e-08
1235 5.45354374992257e-08
1236 5.43601181846043e-08
1237 5.44663869561646e-08
1238 5.42828040295262e-08
1239 5.4384145187214e-08
1240 5.42094831246231e-08
1241 5.43244595974102e-08
1242 5.41417826127599e-08
1243 5.42302522887894e-08
1244 5.40709770291414e-08
1245 5.41575602142075e-08
1246 5.39897655471577e-08
1247 5.40877138632823e-08
1248 5.39197912985401e-08
1249 5.40270690407851e-08
1250 5.38396847105105e-08
1251 5.39602247329185e-08
1252 5.37742934625385e-08
1253 5.38754854062518e-08
1254 5.36931921146788e-08
1255 5.38148192674726e-08
1256 5.36320392541256e-08
1257 5.3733838711878e-08
1258 5.35455697558973e-08
1259 5.3668458122047e-08
1260 5.34844026844894e-08
1261 5.35873709850421e-08
1262 5.3404438205007e-08
1263 5.3525376131347e-08
1264 5.33399173718863e-08
1265 5.34459836387668e-08
1266 5.32582156154149e-08
1267 5.33799102697685e-08
1268 5.31975246076399e-08
1269 5.33036264016573e-08
1270 5.3112632514285e-08
1271 5.32347108617159e-08
1272 5.30537000997811e-08
1273 5.3160935209462e-08
1274 5.29719770270276e-08
1275 5.30914299190499e-08
1276 5.29113570735262e-08
1277 5.30207699966923e-08
1278 5.28262091847864e-08
1279 5.29470156607204e-08
1280 5.27694439256265e-08
1281 5.2872845657248e-08
1282 5.2697366470511e-08
1283 5.28070280836346e-08
1284 5.26186525462435e-08
1285 5.27330179522778e-08
1286 5.2556409002591e-08
1287 5.26615124840646e-08
1288 5.24934513634889e-08
1289 5.25829193520622e-08
1290 5.24242622645943e-08
1291 5.25118970529093e-08
1292 5.2355954238692e-08
1293 5.24422887338005e-08
1294 5.2286083018771e-08
1295 5.23707655020189e-08
1296 5.22156398119478e-08
1297 5.23047631872942e-08
1298 5.21510585826945e-08
1299 5.22279677284132e-08
1300 5.20743554943692e-08
1301 5.21739558223544e-08
1302 5.19937159992878e-08
1303 5.21032781364283e-08
1304 5.19335152660005e-08
1305 5.20407823501046e-08
1306 5.18516181102768e-08
1307 5.19684419941768e-08
1308 5.18095504276062e-08
1309 5.18798408677412e-08
1310 5.17333944571874e-08
1311 5.18350340428242e-08
1312 5.1648175514174e-08
1313 5.17653475640145e-08
1314 5.16062605981915e-08
1315 5.16764053770657e-08
1316 5.15295752734346e-08
1317 5.16317797405463e-08
1318 5.14452871414051e-08
1319 5.15640934395378e-08
1320 5.14051947675398e-08
1321 5.14741600454727e-08
1322 5.13292626180828e-08
1323 5.14315132704724e-08
1324 5.12442355216081e-08
1325 5.13633615639719e-08
1326 5.12047719780639e-08
1327 5.12741280545015e-08
1328 5.11279942827514e-08
1329 5.12312716693941e-08
1330 5.10442745849105e-08
1331 5.11655002810585e-08
1332 5.10059123826068e-08
1333 5.10773787709695e-08
1334 5.09294011408201e-08
1335 5.10340072423787e-08
1336 5.0847088317596e-08
1337 5.09685840199836e-08
1338 5.080746845465e-08
1339 5.08814821387205e-08
1340 5.07332664767546e-08
1341 5.08373574348298e-08
1342 5.06507582542781e-08
1343 5.07747657252366e-08
1344 5.06114865572727e-08
1345 5.0688100827756e-08
1346 5.05369044390136e-08
1347 5.06431021563003e-08
1348 5.04585244698319e-08
1349 5.05792527860649e-08
1350 5.04151884683779e-08
1351 5.04987873739537e-08
1352 5.03446386801443e-08
1353 5.04506125764692e-08
1354 5.0265690276774e-08
1355 5.03892110259585e-08
1356 5.02183432615766e-08
1357 5.03174071297963e-08
1358 5.01543588882214e-08
1359 5.0264329587435e-08
1360 5.00734564923278e-08
1361 5.0199346901536e-08
1362 5.00249157653343e-08
1363 5.01358705662369e-08
1364 4.99546288779129e-08
1365 5.00719892215784e-08
1366 4.99091896699611e-08
1367 4.99981069879141e-08
1368 4.98342416221931e-08
1369 4.99454415603395e-08
1370 4.97859851122939e-08
1371 4.98715273522521e-08
1372 4.97092962348233e-08
1373 4.98219776545739e-08
1374 4.96638996594356e-08
1375 4.97457470771678e-08
1376 4.95855623228181e-08
1377 4.96994658760741e-08
1378 4.95375509501628e-08
1379 4.96206489231099e-08
1380 4.94613026091884e-08
1381 4.95783147869133e-08
1382 4.94088752134303e-08
1383 4.95028125158115e-08
1384 4.93366378861992e-08
1385 4.9457678841236e-08
1386 4.92864948853367e-08
1387 4.93797429612641e-08
1388 4.92148295450079e-08
1389 4.9336552621071e-08
1390 4.91604019714487e-08
1391 4.92657576955935e-08
1392 4.90959877197383e-08
1393 4.92225531445456e-08
1394 4.90376308448504e-08
1395 4.91496408017156e-08
1396 4.89740976661324e-08
1397 4.90991176604894e-08
1398 4.89118328061977e-08
1399 4.90263190044971e-08
1400 4.88547584609478e-08
1401 4.89753446686336e-08
1402 4.87863474063488e-08
1403 4.89027627281757e-08
1404 4.87436366825023e-08
1405 4.88726712433163e-08
1406 4.86716977832202e-08
1407 4.87682889627195e-08
1408 4.86347424555333e-08
1409 4.87425921846807e-08
1410 4.85451856491181e-08
1411 4.8645475203557e-08
1412 4.85193432098185e-08
1413 4.86244466912922e-08
1414 4.84215014751044e-08
1415 4.8540364616656e-08
1416 4.83729429845425e-08
1417 4.84968651903728e-08
1418 4.83178759225211e-08
1419 4.84249760290822e-08
1420 4.82706177251657e-08
1421 4.83904720738337e-08
1422 4.82068784890544e-08
1423 4.82860720296685e-08
1424 4.81610449298842e-08
1425 4.82604924911811e-08
1426 4.80750763642845e-08
1427 4.81737423285722e-08
1428 4.80393183011074e-08
1429 4.81387090189855e-08
1430 4.79611266257507e-08
1431 4.80548365544564e-08
1432 4.79300013012107e-08
1433 4.8025441401478e-08
1434 4.78513868529262e-08
1435 4.79350283910662e-08
1436 4.78101043199786e-08
1437 4.78989718999401e-08
1438 4.77507100526964e-08
1439 4.78281414473258e-08
1440 4.76835886331628e-08
1441 4.7789647794616e-08
1442 4.76120689540949e-08
1443 4.77023540668142e-08
1444 4.7581451667611e-08
1445 4.76728096998613e-08
1446 4.75058605786671e-08
1447 4.758205918165e-08
1448 4.74726817856208e-08
1449 4.75427128776573e-08
1450 4.73937191713958e-08
1451 4.74906904912586e-08
1452 4.73292942615444e-08
1453 4.74329659994055e-08
1454 4.72712216037507e-08
1455 4.73686938562423e-08
1456 4.7219476329019e-08
1457 4.73209453843992e-08
1458 4.71670134061242e-08
1459 4.72533123740959e-08
1460 4.71063010820671e-08
1461 4.72006966845129e-08
1462 4.70414285302923e-08
1463 4.71224588238783e-08
1464 4.70144492226154e-08
1465 4.70763303894728e-08
1466 4.69911469735962e-08
1467 4.69551260096068e-08
1468 4.69541205916357e-08
1469 4.69040486450467e-08
1470 4.68839651546205e-08
1471 4.68998102576279e-08
1472 4.67859599950771e-08
1473 4.68283758436883e-08
1474 4.6765642025548e-08
1475 4.67925502789512e-08
1476 4.6690484367673e-08
1477 4.67135166104526e-08
1478 4.66674983101711e-08
1479 4.6616133175803e-08
1480 4.66050735781209e-08
1481 4.65748932754195e-08
1482 4.65277558703292e-08
1483 4.6548397136803e-08
1484 4.64764973173715e-08
1485 4.6515701512817e-08
1486 4.63699123542938e-08
1487 4.6446881896145e-08
1488 4.63111966553242e-08
1489 4.63864466837549e-08
1490 4.62689939695338e-08
1491 4.63207925349707e-08
1492 4.62693208191922e-08
1493 4.62191671601886e-08
1494 4.61964972942042e-08
1495 4.61625369041485e-08
1496 4.61553426589489e-08
1497 4.61036648857771e-08
1498 4.6102570649964e-08
1499 4.60660842804828e-08
1500 4.60235334287518e-08
1501 4.60337581387193e-08
1502 4.59728717316921e-08
1503 4.59830218346724e-08
1504 4.59190019341804e-08
1505 4.59256632723282e-08
1506 4.58603643949118e-08
1507 4.58695836869083e-08
1508 4.58068143416313e-08
1509 4.58133904146507e-08
1510 4.57526212471748e-08
1511 4.5756767264038e-08
1512 4.57006628096224e-08
1513 4.57039952550531e-08
1514 4.56456277220241e-08
1515 4.56497168954684e-08
1516 4.55902728901947e-08
1517 4.55892319450868e-08
1518 4.55380444464026e-08
1519 4.55399806753576e-08
1520 4.54831692309199e-08
1521 4.54817552508757e-08
1522 4.54288020534932e-08
1523 4.54283153317192e-08
1524 4.53754545048923e-08
1525 4.53716495485423e-08
1526 4.5318362396074e-08
1527 4.53136337341675e-08
1528 4.52615900314868e-08
1529 4.52589645760781e-08
1530 4.52059794042725e-08
1531 4.52039756737577e-08
1532 4.5150780891845e-08
1533 4.51492461195357e-08
1534 4.50955823794175e-08
1535 4.5094868283968e-08
1536 4.50434214371853e-08
1537 4.50404975538277e-08
1538 4.49900134924519e-08
1539 4.49854979933662e-08
1540 4.49360761933804e-08
1541 4.49309496275418e-08
1542 4.48837056410412e-08
1543 4.48770478556071e-08
1544 4.48297683419696e-08
1545 4.48229577898474e-08
1546 4.47794228364273e-08
1547 4.47687078519721e-08
1548 4.47270593895155e-08
1549 4.47154100413627e-08
1550 4.46737082882009e-08
1551 4.46615899818426e-08
1552 4.46227801376153e-08
1553 4.46085657301865e-08
1554 4.45704273488445e-08
1555 4.45543548721616e-08
1556 4.45191687958868e-08
1557 4.45027446005497e-08
1558 4.44675514188475e-08
1559 4.44504486551978e-08
1560 4.44168790636468e-08
1561 4.43972716368535e-08
1562 4.43646790415642e-08
1563 4.43451178000487e-08
1564 4.43146390693983e-08
1565 4.42924488197605e-08
1566 4.42633734110132e-08
1567 4.42407284140245e-08
1568 4.42121255161965e-08
1569 4.41889937974338e-08
1570 4.41618652757825e-08
1571 4.41384671034939e-08
1572 4.411059251197e-08
1573 4.40866472217749e-08
1574 4.40599308149103e-08
1575 4.4034859314479e-08
1576 4.40087681852219e-08
1577 4.39853344857966e-08
1578 4.39571898880331e-08
1579 4.39335039459365e-08
1580 4.39064535839861e-08
1581 4.38828884341547e-08
1582 4.38563105831236e-08
1583 4.38337721675452e-08
1584 4.38047855766399e-08
1585 4.37857430313215e-08
1586 4.37522658103262e-08
1587 4.37396820984759e-08
1588 4.37010179155095e-08
1589 4.36890985611171e-08
1590 4.36511307100318e-08
1591 4.36421530025655e-08
1592 4.36000959780358e-08
1593 4.35892246741787e-08
1594 4.35494058592667e-08
1595 4.35442224500093e-08
1596 4.35007976307134e-08
1597 4.34900400136939e-08
1598 4.34480043054464e-08
1599 4.34470166510437e-08
1600 4.33991615977902e-08
1601 4.33943831978922e-08
1602 4.33487237216923e-08
1603 4.33475975114561e-08
1604 4.32990461263216e-08
1605 4.32973870090336e-08
1606 4.32485762758006e-08
1607 4.3251148440504e-08
1608 4.31983941950875e-08
1609 4.32050626386626e-08
1610 4.3147387884801e-08
1611 4.31596696159886e-08
1612 4.30971311971007e-08
1613 4.31123226007912e-08
1614 4.30511271076739e-08
1615 4.30585771482583e-08
1616 4.29973248117221e-08
1617 4.30144062590898e-08
1618 4.29511892718892e-08
1619 4.29630340192944e-08
1620 4.28991384637811e-08
1621 4.29165467608073e-08
1622 4.28654232109693e-08
1623 4.28257003193266e-08
1624 4.28433537535966e-08
1625 4.27818029891114e-08
1626 4.27846629236228e-08
1627 4.2728117932711e-08
1628 4.27448263451424e-08
1629 4.2683421241918e-08
1630 4.26901145544889e-08
1631 4.26295088118422e-08
1632 4.26481321369465e-08
1633 4.25898853961826e-08
1634 4.25712229912278e-08
1635 4.2538381705981e-08
1636 4.2543447875687e-08
1637 4.24837871548789e-08
1638 4.25019273109228e-08
1639 4.24373070018191e-08
1640 4.24521431341418e-08
1641 4.23851140851639e-08
1642 4.24028776535579e-08
1643 4.23376640412698e-08
1644 4.23560280182755e-08
1645 4.22918269293859e-08
1646 4.23085566580994e-08
1647 4.22402735011929e-08
1648 4.22588790627287e-08
1649 4.21931005689657e-08
1650 4.22107611086631e-08
1651 4.21447161613742e-08
1652 4.21621457746824e-08
1653 4.20958379265812e-08
1654 4.21133989902955e-08
1655 4.20490273711494e-08
1656 4.20654657773412e-08
1657 4.19996872835782e-08
1658 4.20167580728048e-08
1659 4.19525747474836e-08
1660 4.19678727325845e-08
1661 4.19043111321571e-08
1662 4.19195451684118e-08
1663 4.18562144943735e-08
1664 4.18717860384277e-08
1665 4.18095140730657e-08
1666 4.18223997655787e-08
1667 4.17628420734673e-08
1668 4.17753867054671e-08
1669 4.17120524787151e-08
1670 4.17234495841967e-08
1671 4.16674694747599e-08
1672 4.16801384517385e-08
1673 4.16185308438344e-08
1674 4.16244425593959e-08
1675 4.15741325809904e-08
1676 4.15868335323921e-08
1677 4.15237195738882e-08
1678 4.15214067572833e-08
1679 4.14842666884851e-08
1680 4.14911802693041e-08
1681 4.14297787187934e-08
1682 4.14192449227357e-08
1683 4.14060181697096e-08
1684 4.13633785001366e-08
1685 4.13714218439054e-08
1686 4.13138359078857e-08
1687 4.13046024050345e-08
1688 4.12839398222786e-08
1689 4.12521430348534e-08
1690 4.12525444914991e-08
1691 4.11957472579161e-08
1692 4.11790175292026e-08
1693 4.11894802709867e-08
1694 4.11255349774819e-08
1695 4.11118143972544e-08
1696 4.11151113155483e-08
1697 4.10568290476476e-08
1698 4.10418117269273e-08
1699 4.10497627001405e-08
1700 4.09865421602262e-08
1701 4.09743421414532e-08
1702 4.09740543716453e-08
1703 4.09187954630852e-08
1704 4.09024778491585e-08
1705 4.09095477493793e-08
1706 4.08495175463486e-08
1707 4.08378397764864e-08
1708 4.08295157683369e-08
1709 4.07824458648065e-08
1710 4.07668210300471e-08
1711 4.07694393800284e-08
1712 4.07125924084539e-08
1713 4.07005451563691e-08
1714 4.06920506179631e-08
1715 4.06446822864837e-08
1716 4.0628286512856e-08
1717 4.06318321211074e-08
1718 4.05746050091693e-08
1719 4.0564366088347e-08
1720 4.05525462099376e-08
1721 4.05107307699382e-08
1722 4.04954292321236e-08
1723 4.04886186800013e-08
1724 4.04399571607428e-08
1725 4.04227904482468e-08
1726 4.04240800833122e-08
1727 4.03713471541778e-08
1728 4.03574560436937e-08
1729 4.035364753463e-08
1730 4.03050925967818e-08
1731 4.02875421912086e-08
1732 4.02869346771695e-08
1733 4.02382980269067e-08
1734 4.0219845232059e-08
1735 4.02173405689155e-08
1736 4.01711126585269e-08
1737 4.01510398262417e-08
1738 4.01452382448042e-08
1739 4.01076505340825e-08
1740 4.0090711195262e-08
1741 4.0079495278178e-08
1742 4.00388877608293e-08
1743 4.00168502778797e-08
1744 4.00035702341484e-08
1745 3.99876327605853e-08
1746 3.99527202432637e-08
1747 3.99329422862138e-08
1748 3.99216730784246e-08
1749 3.98859079098202e-08
1750 3.98648118959954e-08
1751 3.98502955079039e-08
1752 3.98256716493961e-08
1753 3.98039539106776e-08
1754 3.97811383834323e-08
1755 3.97595876222567e-08
1756 3.97368538074261e-08
1757 3.97157151610372e-08
1758 3.96919332956713e-08
1759 3.9670645435308e-08
1760 3.9647826355349e-08
1761 3.96250356970995e-08
1762 3.96032646676758e-08
1763 3.95805948016914e-08
1764 3.95577615108778e-08
1765 3.95348891402136e-08
1766 3.95119776896991e-08
1767 3.94872543552083e-08
1768 3.946403026589e-08
1769 3.94438472994807e-08
1770 3.94201329356747e-08
1771 3.9400617879437e-08
1772 3.93774826079607e-08
1773 3.93563333034308e-08
1774 3.93334858017624e-08
1775 3.93149406363591e-08
1776 3.92907537616338e-08
1777 3.92713985775117e-08
1778 3.92474177601798e-08
1779 3.9229995252299e-08
1780 3.92032966090028e-08
1781 3.91852701397966e-08
1782 3.91605254890237e-08
1783 3.91462471327486e-08
1784 3.9118944528127e-08
1785 3.90990848586625e-08
1786 3.90726349053239e-08
1787 3.90577703512918e-08
1788 3.90302297148537e-08
1789 3.90111516423985e-08
1790 3.89853695992315e-08
1791 3.8958297920999e-08
1792 3.89415646395719e-08
1793 3.89117857935162e-08
1794 3.88871583822947e-08
1795 3.88671494988557e-08
1796 3.88436163234473e-08
1797 3.88236820469956e-08
1798 3.88013958740885e-08
1799 3.87806267099222e-08
1800 3.87589622619089e-08
1801 3.87375429511394e-08
1802 3.87170437932127e-08
1803 3.86955392173149e-08
1804 3.86730469870145e-08
1805 3.86533933749433e-08
1806 3.8631927878896e-08
1807 3.86103309324426e-08
1808 3.8589476503148e-08
1809 3.85680927195153e-08
1810 3.85471992103703e-08
1811 3.85252079126985e-08
1812 3.85049290230199e-08
1813 3.84836873479344e-08
1814 3.84625167271224e-08
1815 3.84411400489171e-08
1816 3.84203318049003e-08
1817 3.83998326469737e-08
1818 3.83794755975941e-08
1819 3.83580882612478e-08
1820 3.83370206691325e-08
1821 3.83152496397088e-08
1822 3.82950950950089e-08
1823 3.82741589532998e-08
1824 3.8253141099176e-08
1825 3.82313629643249e-08
1826 3.82117377739633e-08
1827 3.8190890450096e-08
1828 3.8170561822426e-08
1829 3.8149114089947e-08
1830 3.81286291428751e-08
1831 3.8108012745397e-08
1832 3.80874034533463e-08
1833 3.80662790178121e-08
1834 3.8045463668368e-08
1835 3.80251137244159e-08
1836 3.80051794479641e-08
1837 3.7984275280678e-08
1838 3.79629661040326e-08
1839 3.79421791762979e-08
1840 3.79220708168759e-08
1841 3.79014259976884e-08
1842 3.78814597468136e-08
1843 3.7859344104163e-08
1844 3.78400528688871e-08
1845 3.78191025163233e-08
1846 3.7799129160021e-08
1847 3.77787010563679e-08
1848 3.77573883270088e-08
1849 3.77370419357703e-08
1850 3.77164646181427e-08
1851 3.76958624315193e-08
1852 3.7675160768913e-08
1853 3.76549991187858e-08
1854 3.76340736352176e-08
1855 3.76145159464158e-08
1856 3.75934483543006e-08
1857 3.75725157653051e-08
1858 3.7552883469516e-08
1859 3.75327147139615e-08
1860 3.75120343676372e-08
1861 3.74914215228728e-08
1862 3.74710928952027e-08
1863 3.74509880884943e-08
1864 3.74304001127257e-08
1865 3.74100181943504e-08
1866 3.73900093109114e-08
1867 3.73700359546092e-08
1868 3.73490927074727e-08
1869 3.73301389799963e-08
1870 3.73091140204451e-08
1871 3.72888457889076e-08
1872 3.72701016715382e-08
1873 3.72497694911544e-08
1874 3.72284318927996e-08
1875 3.72090021016902e-08
1876 3.71891424322257e-08
1877 3.71690056510943e-08
1878 3.71488795281039e-08
1879 3.71290482803488e-08
1880 3.71084603045801e-08
1881 3.7088984328193e-08
1882 3.70698032270411e-08
1883 3.70495065737941e-08
1884 3.70292951856754e-08
1885 3.70094994650572e-08
1886 3.69900057251016e-08
1887 3.69702419789064e-08
1888 3.69515618103833e-08
1889 3.69309631764736e-08
1890 3.69112314047015e-08
1891 3.68912544956856e-08
1892 3.68708370501736e-08
1893 3.68516879234448e-08
1894 3.68311745546634e-08
1895 3.68116666038532e-08
1896 3.67925920841117e-08
1897 3.67726151750958e-08
1898 3.67535299972133e-08
1899 3.67339403339884e-08
1900 3.67135761791815e-08
1901 3.66940895446533e-08
1902 3.66751606861726e-08
1903 3.6655368518268e-08
1904 3.66363259729496e-08
1905 3.6617020526819e-08
1906 3.65966172921617e-08
1907 3.65782177880192e-08
1908 3.6559040239581e-08
1909 3.65396886081726e-08
1910 3.65200989449477e-08
1911 3.6501059952343e-08
1912 3.64815768705284e-08
1913 3.64625520887785e-08
1914 3.64429979526903e-08
1915 3.64239305383762e-08
1916 3.64056766954945e-08
1917 3.63857459717565e-08
1918 3.63666181613098e-08
1919 3.63475862741325e-08
1920 3.63282524062924e-08
1921 3.63094478927906e-08
1922 3.62912544460414e-08
1923 3.62716896518123e-08
1924 3.62527288189085e-08
1925 3.62333310022223e-08
1926 3.62150238686354e-08
1927 3.619577171321e-08
1928 3.6176636797336e-08
1929 3.61578287311204e-08
1930 3.61393475145633e-08
1931 3.61201735188388e-08
1932 3.61021932349104e-08
1933 3.60826852841001e-08
1934 3.60648790831419e-08
1935 3.60453533687632e-08
1936 3.6026342797868e-08
1937 3.6008191983683e-08
1938 3.59899452462287e-08
1939 3.59703697938585e-08
1940 3.59525706983277e-08
1941 3.59326080001665e-08
1942 3.59146170580971e-08
1943 3.58972229719257e-08
1944 3.58785356979752e-08
1945 3.58599372418666e-08
1946 3.58418219548184e-08
1947 3.58238594344584e-08
1948 3.58052609783499e-08
1949 3.57872949052762e-08
1950 3.57682630180989e-08
1951 3.57508085357949e-08
1952 3.57327287758835e-08
1953 3.57152494245838e-08
1954 3.569735085307e-08
1955 3.56794522815562e-08
1956 3.56623885977569e-08
1957 3.56451579364148e-08
1958 3.56283678115688e-08
1959 3.56132865420022e-08
1960 3.55967237908317e-08
1961 3.55811202723544e-08
1962 3.55663090090275e-08
1963 3.55508014138195e-08
1964 3.55349705216668e-08
1965 3.55201592583398e-08
1966 3.55039802002466e-08
1967 3.54885116848891e-08
1968 3.54702081040159e-08
1969 3.54545157676966e-08
1970 3.54357077014811e-08
1971 3.54201610264226e-08
1972 3.54000952995648e-08
1973 3.53843745415361e-08
1974 3.5363747485917e-08
1975 3.53479165937642e-08
1976 3.53278188924833e-08
1977 3.53129436803101e-08
1978 3.52922704394132e-08
1979 3.52775195722188e-08
1980 3.52575568740576e-08
1981 3.52408768833357e-08
1982 3.52212694565424e-08
1983 3.52050122387482e-08
1984 3.51857174507586e-08
1985 3.51696378686484e-08
1986 3.51499167550173e-08
1987 3.51335600612401e-08
1988 3.51145068577807e-08
1989 3.50980258190248e-08
1990 3.50796831583011e-08
1991 3.50617597177916e-08
1992 3.50433992934995e-08
1993 3.50267121973502e-08
1994 3.50070905597022e-08
1995 3.49911424279981e-08
1996 3.49726256843041e-08
1997 3.49558604284539e-08
1998 3.49369813079647e-08
1999 3.49211752848078e-08
2000 3.49016247014333e-08
2001 3.48850797138311e-08
2002 3.48654012327643e-08
2003 3.48494317847781e-08
2004 3.48309008302294e-08
2005 3.48136133254684e-08
2006 3.47948549972443e-08
2007 3.47795179322929e-08
2008 3.47597399752431e-08
2009 3.47425768154608e-08
2010 3.47247208765111e-08
2011 3.47085133967084e-08
2012 3.46888064939321e-08
2013 3.46721868993427e-08
2014 3.46551196628297e-08
2015 3.46385569116592e-08
2016 3.46190667244173e-08
2017 3.46028024011957e-08
2018 3.45845201366046e-08
2019 3.45679964652845e-08
2020 3.45491777409279e-08
2021 3.45322632711031e-08
2022 3.45148656322181e-08
2023 3.44981323507909e-08
2024 3.44795978435286e-08
2025 3.44629391690887e-08
2026 3.4444962437874e-08
2027 3.44278205943738e-08
2028 3.44103519012151e-08
2029 3.43932633484201e-08
2030 3.43758728149623e-08
2031 3.43590258466975e-08
2032 3.43397452695626e-08
2033 3.43239285882646e-08
2034 3.4307056751004e-08
2035 3.42888277771181e-08
2036 3.42709007838948e-08
2037 3.42538690745187e-08
2038 3.42371926365104e-08
2039 3.42205197512158e-08
2040 3.42012356213672e-08
2041 3.41844952345127e-08
2042 3.41684796012487e-08
2043 3.41503820777689e-08
2044 3.41328636466187e-08
2045 3.41161907613241e-08
2046 3.40994112946191e-08
2047 3.40820989208623e-08
2048 3.40644170648829e-08
2049 3.40469767934337e-08
2050 3.4030371409699e-08
2051 3.40132579879082e-08
2052 3.39972032747937e-08
2053 3.39791235148823e-08
2054 3.39628876133702e-08
2055 3.39448149588861e-08
2056 3.39286323480792e-08
2057 3.39106946967149e-08
2058 3.38940218114203e-08
2059 3.38761374507612e-08
2060 3.38600898430741e-08
2061 3.38415162559613e-08
2062 3.38263177468434e-08
2063 3.38084156226159e-08
2064 3.37922472226637e-08
2065 3.37743344402952e-08
2066 3.37591110621815e-08
2067 3.37403776029532e-08
2068 3.37247882953307e-08
2069 3.3706257340782e-08
2070 3.36907817199972e-08
2071 3.36717178583967e-08
2072 3.3657745035498e-08
2073 3.36379670784481e-08
2074 3.36238450415749e-08
2075 3.36049730265131e-08
2076 3.35907905935073e-08
2077 3.35707568410726e-08
2078 3.35571286314007e-08
2079 3.35377627891376e-08
2080 3.35239462856407e-08
2081 3.35034862075645e-08
2082 3.34900711607133e-08
2083 3.34708154525742e-08
2084 3.34563701187562e-08
2085 3.34360272802314e-08
2086 3.3423610545924e-08
2087 3.34029337523134e-08
2088 3.33906768901215e-08
2089 3.33692220522153e-08
2090 3.33568195287626e-08
2091 3.33357057513695e-08
2092 3.33238610039643e-08
2093 3.33021112908227e-08
2094 3.32909984024354e-08
2095 3.32679697123694e-08
2096 3.32585265994112e-08
2097 3.3236418062188e-08
2098 3.32252660939503e-08
2099 3.32018998960848e-08
2100 3.31923004637247e-08
2101 3.31685150456451e-08
2102 3.3160077350658e-08
2103 3.31362599581553e-08
2104 3.31268807940432e-08
2105 3.31035430178872e-08
2106 3.3093524365313e-08
2107 3.30697957906523e-08
2108 3.30619762678452e-08
2109 3.3037590441154e-08
2110 3.30286624716791e-08
2111 3.30045004659496e-08
2112 3.2996322119061e-08
2113 3.29724620939942e-08
2114 3.29628342399246e-08
2115 3.29392655373795e-08
2116 3.29316343083974e-08
2117 3.29063460924317e-08
2118 3.2898245905244e-08
2119 3.28735261234669e-08
2120 3.28671738714092e-08
2121 3.28414806460842e-08
2122 3.28343610078718e-08
2123 3.28093996415646e-08
2124 3.28028164631178e-08
2125 3.2776920733113e-08
2126 3.27699929414393e-08
2127 3.27454117154957e-08
2128 3.27368283592477e-08
2129 3.27134905830917e-08
2130 3.2705855801396e-08
2131 3.2680890882375e-08
2132 3.26730784649953e-08
2133 3.26484688173423e-08
2134 3.26414806295361e-08
2135 3.26171729625457e-08
2136 3.26099751646325e-08
2137 3.25848148463592e-08
2138 3.25774109910526e-08
2139 3.25546345436578e-08
2140 3.25458309191617e-08
2141 3.25214024599063e-08
2142 3.25143076906897e-08
2143 3.24912505789143e-08
2144 3.24833422382653e-08
2145 3.24589528588604e-08
2146 3.24505364801553e-08
2147 3.24283320196628e-08
2148 3.2420210516193e-08
2149 3.23955511305485e-08
2150 3.23887192621441e-08
2151 3.23673816637893e-08
2152 3.23552242775804e-08
2153 3.23339826024949e-08
2154 3.23257616230421e-08
2155 3.23050706185768e-08
2156 3.2295467633503e-08
2157 3.22725810519842e-08
2158 3.22646975803309e-08
2159 3.22475308678349e-08
2160 3.2226967761062e-08
2161 3.22183133505405e-08
2162 3.21987023710335e-08
2163 3.21861044483285e-08
2164 3.21658220059362e-08
2165 3.21565032379567e-08
2166 3.21386828261438e-08
2167 3.21197042296717e-08
2168 3.21107371803464e-08
2169 3.20900177541716e-08
2170 3.20790825014683e-08
2171 3.20573256828993e-08
2172 3.20487743010744e-08
2173 3.20326840608232e-08
2174 3.20113571206093e-08
2175 3.20032462752806e-08
2176 3.19867048403921e-08
2177 3.19661239700508e-08
2178 3.19578461471792e-08
2179 3.19402637671828e-08
2180 3.19203223853037e-08
2181 3.19115471825171e-08
2182 3.18924975317714e-08
2183 3.18811324007129e-08
2184 3.18594359782765e-08
2185 3.18503303731177e-08
2186 3.1835693192761e-08
2187 3.18136841315209e-08
2188 3.18048911651658e-08
2189 3.17900266111337e-08
2190 3.1767857677778e-08
2191 3.17591606346923e-08
2192 3.17433013208301e-08
2193 3.17224007062578e-08
2194 3.17134016825094e-08
2195 3.16978372438825e-08
2196 3.16766524122158e-08
2197 3.16674828582109e-08
2198 3.16530055499697e-08
2199 3.16318065074483e-08
2200 3.16227897201315e-08
2201 3.16060848604138e-08
2202 3.15866692801592e-08
2203 3.15774215664533e-08
2204 3.15618002844076e-08
2205 3.15402033379542e-08
2206 3.15318047228175e-08
2207 3.15153094732068e-08
2208 3.14960004743625e-08
2209 3.14868380257849e-08
2210 3.14701829040587e-08
2211 3.14499146725211e-08
2212 3.14408481472128e-08
2213 3.14251380473252e-08
2214 3.14047206018131e-08
2215 3.13951922237266e-08
2216 3.13792902773002e-08
2217 3.13602583901229e-08
2218 3.13501047344289e-08
2219 3.13313464062048e-08
2220 3.13203258883732e-08
2221 3.13025942944023e-08
2222 3.1289911106569e-08
2223 3.12702503890705e-08
2224 3.1258675647905e-08
2225 3.12391073009621e-08
2226 3.12283852110795e-08
2227 3.12118615397594e-08
2228 3.11958068266449e-08
2229 3.11831769295168e-08
2230 3.11632071259282e-08
2231 3.11536147989955e-08
2232 3.11341565861767e-08
2233 3.11232852823196e-08
2234 3.11041823408686e-08
2235 3.10926360214125e-08
2236 3.10774197487262e-08
2237 3.10579046924886e-08
2238 3.10487457966246e-08
2239 3.10278309711975e-08
2240 3.10181782481322e-08
2241 3.10011962767476e-08
2242 3.09864560676942e-08
2243 3.09694314637454e-08
2244 3.09567091960616e-08
2245 3.09376879670253e-08
2246 3.09277901067162e-08
2247 3.09096037653944e-08
2248 3.08971408458092e-08
2249 3.08796188619453e-08
2250 3.08665129011843e-08
2251 3.0849758303475e-08
2252 3.08335827980954e-08
2253 3.08213969901772e-08
2254 3.08044043606515e-08
2255 3.0790705096706e-08
2256 3.07731617965601e-08
2257 3.07618002182153e-08
2258 3.07425125356531e-08
2259 3.07315346503856e-08
2260 3.07138279254104e-08
2261 3.07010203925984e-08
2262 3.06845251429877e-08
2263 3.06691987361773e-08
2264 3.06527532245582e-08
2265 3.06409582151446e-08
2266 3.06224805513011e-08
2267 3.06104581682121e-08
2268 3.05949967582819e-08
2269 3.05785512466628e-08
2270 3.05651006726748e-08
2271 3.05500975628092e-08
2272 3.05336840256132e-08
2273 3.05210683393398e-08
2274 3.05050633642168e-08
2275 3.04876657253317e-08
2276 3.04762863834185e-08
2277 3.04592369104739e-08
2278 3.04435197051589e-08
2279 3.0430555852945e-08
2280 3.04149878616045e-08
2281 3.03987057748145e-08
2282 3.03856992900364e-08
2283 3.03697262893365e-08
2284 3.03541050072909e-08
2285 3.03405727208883e-08
2286 3.03259071188222e-08
2287 3.03089287001512e-08
2288 3.02961389309075e-08
2289 3.02803115914685e-08
2290 3.02642213512172e-08
2291 3.02517513262046e-08
2292 3.02346201408454e-08
2293 3.02207290303613e-08
2294 3.02060811918636e-08
2295 3.01901756927236e-08
2296 3.01766078791843e-08
2297 3.01609794917113e-08
2298 3.01468894292611e-08
2299 3.01314138084763e-08
2300 3.01169187366668e-08
2301 3.01013898251767e-08
2302 3.00855340640283e-08
2303 3.00711562317701e-08
2304 3.00569631406233e-08
2305 3.00413276477229e-08
2306 3.00265128316823e-08
2307 3.00119573637403e-08
2308 2.99984854734703e-08
2309 2.99818161408894e-08
2310 2.99669231651478e-08
2311 2.9952740732142e-08
2312 2.99381639479179e-08
2313 2.9922567534868e-08
2314 2.99079090382293e-08
2315 2.98918649832558e-08
2316 2.9877661233968e-08
2317 2.9862452066709e-08
2318 2.98484295058188e-08
2319 2.98339131177272e-08
2320 2.98179720914504e-08
2321 2.98040028212654e-08
2322 2.97888345102137e-08
2323 2.977367863366e-08
2324 2.97589579645319e-08
2325 2.9745130802894e-08
2326 2.97290245754311e-08
2327 2.97147568772971e-08
2328 2.97002529237034e-08
2329 2.96850970471496e-08
2330 2.96701863078397e-08
2331 2.96557516321627e-08
2332 2.96418107836871e-08
2333 2.96272499866745e-08
2334 2.9612014174063e-08
2335 2.95962649943249e-08
2336 2.95828979091084e-08
2337 2.9567074122383e-08
2338 2.95537248007349e-08
2339 2.95377926562423e-08
2340 2.95243527403954e-08
2341 2.95076016953999e-08
2342 2.94955402324604e-08
2343 2.94778743636925e-08
2344 2.94660758015652e-08
2345 2.94492021879478e-08
2346 2.9436767690072e-08
2347 2.94193398531206e-08
2348 2.94072268758327e-08
2349 2.93895983105585e-08
2350 2.93780004767541e-08
2351 2.93609172530296e-08
2352 2.9348314001254e-08
2353 2.93308577425933e-08
2354 2.93192101707973e-08
2355 2.93009936314093e-08
2356 2.92907262888775e-08
2357 2.92730408801845e-08
2358 2.92602635454386e-08
2359 2.92423454339996e-08
2360 2.92324227046947e-08
2361 2.92126252077196e-08
2362 2.9203285123458e-08
2363 2.91837185528721e-08
2364 2.91743909031084e-08
2365 2.91541777386328e-08
2366 2.91457826762098e-08
2367 2.91251449624497e-08
2368 2.91173591904226e-08
2369 2.90952311132742e-08
2370 2.90882038456175e-08
2371 2.90672534930536e-08
2372 2.90602280017538e-08
2373 2.90373893818696e-08
2374 2.90305557371084e-08
2375 2.90112627254757e-08
2376 2.8997588330526e-08
2377 2.89807786657548e-08
2378 2.8969317611427e-08
2379 2.89524564323074e-08
2380 2.89413790710569e-08
2381 2.89218569093919e-08
2382 2.89144868048652e-08
2383 2.88932771042028e-08
2384 2.88872712417287e-08
2385 2.88654113944631e-08
2386 2.88555206395813e-08
2387 2.88347230537056e-08
2388 2.88281256644041e-08
2389 2.88133978898486e-08
2390 2.87923995756501e-08
2391 2.87844166280138e-08
2392 2.8765013482257e-08
2393 2.87554176026106e-08
2394 2.87356165529218e-08
2395 2.87272978738429e-08
2396 2.87069585880317e-08
2397 2.86979329189307e-08
2398 2.86793984116684e-08
2399 2.86704899821189e-08
2400 2.8650081418391e-08
2401 2.86404180371846e-08
2402 2.86234769220073e-08
2403 2.86093921886277e-08
2404 2.85924279808114e-08
2405 2.85838535063476e-08
2406 2.85648233955271e-08
2407 2.85556094326012e-08
2408 2.85368439989497e-08
2409 2.8524906880989e-08
2410 2.8507020743973e-08
2411 2.84981211962076e-08
2412 2.84792776028553e-08
2413 2.84699961383694e-08
2414 2.84513976822609e-08
2415 2.84409509276884e-08
2416 2.84215779799979e-08
2417 2.84110370785129e-08
2418 2.83939716183568e-08
2419 2.83831713687732e-08
2420 2.83655960942042e-08
2421 2.83548633461805e-08
2422 2.83377730170287e-08
2423 2.83252514776677e-08
2424 2.83089267583136e-08
2425 2.82953962482679e-08
2426 2.82805654450158e-08
2427 2.82684080588069e-08
2428 2.82526553263551e-08
2429 2.82392740302839e-08
2430 2.82251804151201e-08
2431 2.82105094839835e-08
2432 2.81972010185427e-08
2433 2.81816667779822e-08
2434 2.81680438973808e-08
2435 2.81537548829647e-08
2436 2.8140036079094e-08
2437 2.81263030643686e-08
2438 2.81123533341088e-08
2439 2.80988601275567e-08
2440 2.80852781031626e-08
2441 2.80710370503812e-08
2442 2.80565704002811e-08
2443 2.80445178191258e-08
2444 2.80313123823817e-08
2445 2.80192722357242e-08
2446 2.80037575350889e-08
2447 2.80214074166452e-08
2448 2.79836207539574e-08
2449 2.80185723511295e-08
2450 2.79545808723469e-08
2451 2.80120087126079e-08
2452 2.79332148522826e-08
2453 2.79677330183858e-08
2454 2.78986966861794e-08
2455 2.79623897370129e-08
2456 2.78803859998789e-08
2457 2.7912184563661e-08
2458 2.78442957579728e-08
2459 2.79088379073755e-08
2460 2.78278822207767e-08
2461 2.78554210808579e-08
2462 2.77898326572767e-08
2463 2.78575278400695e-08
2464 2.777354879413e-08
2465 2.77986060837065e-08
2466 2.77357123934507e-08
2467 2.78037202150472e-08
2468 2.77216258837143e-08
2469 2.77419225369613e-08
2470 2.76816578548278e-08
2471 2.77506106982628e-08
2472 2.76683262967481e-08
2473 2.76862088810503e-08
2474 2.76287348555115e-08
2475 2.76967515588922e-08
2476 2.76149201283715e-08
2477 2.76326623804835e-08
2478 2.7575328687135e-08
2479 2.76438960611358e-08
2480 2.75626437229448e-08
2481 2.75788476500338e-08
2482 2.75221960777117e-08
2483 2.75900404744789e-08
2484 2.75097278290559e-08
2485 2.75256688553327e-08
2486 2.74693192636732e-08
2487 2.75384230974396e-08
2488 2.74582312442817e-08
2489 2.74695999280539e-08
2490 2.74167177849449e-08
2491 2.7482849773719e-08
2492 2.74039049230623e-08
2493 2.74214695394903e-08
2494 2.73627041025293e-08
2495 2.74357123686286e-08
2496 2.73563394159737e-08
2497 2.73592171140535e-08
2498 2.73166804731773e-08
2499 2.73713549603372e-08
2500 2.72929554512302e-08
2501 2.73391336236273e-08
2502 2.7262982982279e-08
2503 2.73266262951211e-08
2504 2.72480864538238e-08
2505 2.72710636295415e-08
2506 2.72079851981744e-08
2507 2.7282295533837e-08
2508 2.72085411978651e-08
2509 2.71942752760879e-08
2510 2.71742148783005e-08
2511 2.71825086883837e-08
2512 2.71341953350657e-08
2513 2.72022404601557e-08
2514 2.71253153272255e-08
2515 2.7130287350019e-08
2516 2.70851785444393e-08
2517 2.71480296021309e-08
2518 2.70696212112398e-08
2519 2.70946181046838e-08
2520 2.70308078142989e-08
2521 2.71076228131051e-08
2522 2.7037586391998e-08
2523 2.70124562717911e-08
2524 2.70141224945064e-08
2525 2.69833417831933e-08
2526 2.69984550271829e-08
2527 2.695228573657e-08
2528 2.69985598322364e-08
2529 2.69244750938924e-08
2530 2.69957460830028e-08
2531 2.69176467782017e-08
2532 2.69122253371279e-08
2533 2.68950746118435e-08
2534 2.6882139181339e-08
2535 2.68841731099201e-08
2536 2.68415423221313e-08
2537 2.69060098645468e-08
2538 2.68386752821925e-08
2539 2.68290936134008e-08
2540 2.68039528350528e-08
2541 2.68405138115213e-08
2542 2.67705537737584e-08
2543 2.68314526152835e-08
2544 2.67626756311756e-08
2545 2.67717918944754e-08
2546 2.67217501459527e-08
2547 2.67912145801574e-08
2548 2.67441819801206e-08
2549 2.66892197231527e-08
2550 2.67490651850721e-08
2551 2.66692765649168e-08
2552 2.6710420542031e-08
2553 2.66369681867218e-08
2554 2.67131845532731e-08
2555 2.66569291085261e-08
2556 2.6603363068034e-08
2557 2.66547797167505e-08
2558 2.6579826339912e-08
2559 2.66370729917753e-08
2560 2.65718558267736e-08
2561 2.65647468467023e-08
2562 2.65376645103288e-08
2563 2.6567205324568e-08
2564 2.65003929911245e-08
2565 2.65547477340533e-08
2566 2.64873936117738e-08
2567 2.65218460526739e-08
2568 2.64571937691471e-08
2569 2.65152380052314e-08
2570 2.64423007934056e-08
2571 2.64829349561069e-08
2572 2.64134545346906e-08
2573 2.64676316419354e-08
2574 2.63962540714147e-08
2575 2.64014854423067e-08
2576 2.63675463685331e-08
2577 2.63899639918463e-08
2578 2.63610608897125e-08
2579 2.63702322200743e-08
2580 2.63404782430143e-08
2581 2.63508876940932e-08
2582 2.63157478030962e-08
2583 2.63219579466067e-08
2584 2.62976378451185e-08
2585 2.62976449505459e-08
2586 2.62935575534584e-08
2587 2.62553729868387e-08
2588 2.62831072461722e-08
2589 2.62300670073046e-08
2590 2.62350425828117e-08
2591 2.62162238584551e-08
2592 2.62410075890784e-08
2593 2.6188230251023e-08
2594 2.62068606815546e-08
2595 2.61733639206341e-08
2596 2.62064645539795e-08
2597 2.6148113008162e-08
2598 2.61496371223302e-08
2599 2.61362718134706e-08
2600 2.61714259153223e-08
2601 2.61086672281863e-08
2602 2.61182311334096e-08
2603 2.60902197624091e-08
2604 2.61112109711803e-08
2605 2.60627803783109e-08
2606 2.60753338920949e-08
2607 2.60598547185964e-08
2608 2.6075886339072e-08
2609 2.60218477876606e-08
2610 2.6077158210569e-08
2611 2.60130210705256e-08
2612 2.60452068800987e-08
2613 2.59804409097342e-08
2614 2.601913351441e-08
2615 2.59614942876851e-08
2616 2.59730441598549e-08
2617 2.5951750970421e-08
2618 2.59814285641369e-08
2619 2.59285304338164e-08
2620 2.59653187839604e-08
2621 2.59096282206883e-08
2622 2.59326338181154e-08
2623 2.58874166547685e-08
2624 2.59213628339694e-08
2625 2.58619383686209e-08
2626 2.58844696787719e-08
2627 2.5861185193321e-08
2628 2.58839207845085e-08
2629 2.58257522034455e-08
2630 2.58324490687301e-08
2631 2.58263899155509e-08
2632 2.58302286226808e-08
2633 2.57994248187288e-08
2634 2.58378509698787e-08
2635 2.57740495612779e-08
2636 2.57864023467391e-08
2637 2.57658392399662e-08
2638 2.57855976570909e-08
2639 2.57352379406939e-08
2640 2.57433079298153e-08
2641 2.5739671727365e-08
2642 2.57324472840992e-08
2643 2.57149785909405e-08
2644 2.5737913134094e-08
2645 2.56819134847319e-08
2646 2.57315004859038e-08
2647 2.56640433349276e-08
2648 2.56640841911349e-08
2649 2.56637875395427e-08
2650 2.56867398462646e-08
2651 2.56345806803893e-08
2652 2.56618211125215e-08
2653 2.56091396977354e-08
2654 2.56447432178675e-08
2655 2.5597707065117e-08
2656 2.55978651608757e-08
2657 2.55911931645869e-08
2658 2.56078109828195e-08
2659 2.55614569510954e-08
2660 2.55819028183168e-08
2661 2.55499816859128e-08
2662 2.55504239987658e-08
2663 2.55339056565163e-08
2664 2.5553640981002e-08
2665 2.55194656517688e-08
2666 2.55196042076022e-08
2667 2.55253738146166e-08
2668 2.55145380378963e-08
2669 2.54792649201363e-08
2670 2.55347956112928e-08
2671 2.54650185382843e-08
2672 2.54534384680483e-08
2673 2.54405989608131e-08
2674 2.54284735490273e-08
2675 2.54506158370305e-08
2676 2.54530512222573e-08
2677 2.54135166244396e-08
2678 2.54099692398313e-08
2679 2.53834695485011e-08
2680 2.5397207892297e-08
2681 2.54197054516681e-08
2682 2.53418175333309e-08
2683 2.53819987250381e-08
2684 2.5350974652838e-08
2685 2.53355025847668e-08
2686 2.52857983440435e-08
2687 2.53754226520186e-08
2688 2.53416700957132e-08
2689 2.53310030728926e-08
2690 2.5308633411214e-08
2691 2.53282479434347e-08
2692 2.5342844267584e-08
2693 2.52839864600674e-08
2694 2.53292711249742e-08
2695 2.5258119151772e-08
2696 2.52662566424533e-08
2697 2.52561758173897e-08
2698 2.52248337773153e-08
2699 2.52914063025855e-08
2700 2.52443879134034e-08
2701 2.5260995073495e-08
2702 2.52125165189909e-08
2703 2.52594158922648e-08
2704 2.52102339004523e-08
2705 2.52083687257709e-08
2706 2.52382932330875e-08
2707 2.51777088067229e-08
2708 2.51674840967553e-08
2709 2.51860274858018e-08
2710 2.51582914501114e-08
2711 2.51686742558377e-08
2712 2.51670986273211e-08
2713 2.5120202806761e-08
2714 2.5118598756535e-08
2715 2.513153951611e-08
2716 2.51285854346861e-08
2717 2.50938292367664e-08
2718 2.50886849073595e-08
2719 2.51218850166879e-08
2720 2.50993998918148e-08
2721 2.5111708268355e-08
2722 2.50544331947822e-08
2723 2.51289034025604e-08
2724 2.50442742100176e-08
2725 2.50580161065272e-08
2726 2.50015101954659e-08
2727 2.50310954186261e-08
2728 2.49115714723303e-08
2729 2.50235263621335e-08
2730 2.49931577656071e-08
2731 2.50021798819944e-08
2732 2.50000038448661e-08
2733 2.49771776594798e-08
2734 2.49993217238398e-08
2735 2.494907036521e-08
2736 2.49794833706574e-08
2737 2.49609133362583e-08
2738 2.49354386028244e-08
2739 2.49922251782664e-08
2740 2.49335201374379e-08
2741 2.49200056146037e-08
2742 2.49135929664135e-08
2743 2.49006255614859e-08
2744 2.4895745909248e-08
2745 2.49402472007887e-08
2746 2.48760834153927e-08
2747 2.49256189022162e-08
2748 2.48555096504788e-08
2749 2.48763143417818e-08
2750 2.48633522659247e-08
2751 2.48789717716136e-08
2752 2.48291893711894e-08
2753 2.48309124373236e-08
2754 2.48446436756922e-08
2755 2.48132376867716e-08
2756 2.48943212710628e-08
2757 2.47859759383573e-08
2758 2.48103724231896e-08
2759 2.47899443195365e-08
2760 2.47803590980311e-08
2761 2.47875107106665e-08
2762 2.4802083942177e-08
2763 2.47547973231121e-08
2764 2.47566163125157e-08
2765 2.47304736689102e-08
2766 2.47753479953872e-08
2767 2.47582114809575e-08
2768 2.46885907273509e-08
2769 2.45883544636172e-08
2770 2.47308147294234e-08
2771 2.47118183693829e-08
2772 2.4666810816143e-08
2773 2.45693669853608e-08
2774 2.46805935688599e-08
2775 2.46727509534139e-08
2776 2.46435707396131e-08
2777 2.45992559655406e-08
2778 2.46583056195959e-08
2779 2.4664709386002e-08
2780 2.46388545122045e-08
2781 2.46128468717188e-08
2782 2.46123228464512e-08
2783 2.46102427325923e-08
2784 2.45354669914377e-08
2785 2.46068427856017e-08
2786 2.46180675844698e-08
2787 2.45461855286067e-08
2788 2.45811655474881e-08
2789 2.45286884137386e-08
2790 2.45439508717027e-08
2791 2.45525644260169e-08
2792 2.4445894197811e-08
2793 2.45527438380577e-08
2794 2.44242830405028e-08
2795 2.45383073860239e-08
2796 2.44976678942521e-08
2797 2.45358346973035e-08
2798 2.45059990078289e-08
2799 2.44988811459734e-08
2800 2.44853115560772e-08
2801 2.44916158465003e-08
2802 2.45398741327563e-08
2803 2.4458923775228e-08
2804 2.44685960382185e-08
2805 2.43575470904034e-08
2806 2.44445015340489e-08
2807 2.4321481717493e-08
2808 2.44165914153882e-08
2809 2.43094184781967e-08
2810 2.44035156526934e-08
2811 2.44074183086695e-08
2812 2.44109266134274e-08
2813 2.43687150458527e-08
2814 2.44025546436433e-08
2815 2.43824747059307e-08
2816 2.43797888543895e-08
2817 2.43851214776214e-08
2818 2.43893740758949e-08
2819 2.43491182772004e-08
2820 2.43744260330914e-08
2821 2.43438638136695e-08
2822 2.43946089995006e-08
2823 2.42984530274271e-08
2824 2.43386075737817e-08
2825 2.42925892735002e-08
2826 2.42420412632782e-08
2827 2.42945539241646e-08
2828 2.42507418590776e-08
2829 2.4271830767475e-08
2830 2.4219867000852e-08
2831 2.42548452433766e-08
2832 2.41753550511703e-08
2833 2.42391049454227e-08
2834 2.41741915374405e-08
2835 2.4200632608995e-08
2836 2.42422473206716e-08
2837 2.41845512505279e-08
2838 2.42284965423778e-08
2839 2.40948185847856e-08
2840 2.42084539081588e-08
2841 2.41894948516119e-08
2842 2.42050965937324e-08
2843 2.42163196162437e-08
2844 2.4142464027932e-08
2845 2.41604229955783e-08
2846 2.41442297266303e-08
2847 2.41778845833096e-08
2848 2.4168832268856e-08
2849 2.41391937549906e-08
2850 2.41513991028341e-08
2851 2.41124986644081e-08
2852 2.40966873121806e-08
2853 2.41275248669126e-08
2854 2.39896049691879e-08
2855 2.41082940277693e-08
2856 2.40925093208944e-08
2857 2.39693118686546e-08
2858 2.4085345273761e-08
2859 2.40505979576255e-08
2860 2.40320670030769e-08
2861 2.40573427845447e-08
2862 2.40084006009056e-08
2863 2.39232882393026e-08
2864 2.40200836998383e-08
2865 2.40300384035663e-08
2866 2.3995577080882e-08
2867 2.40323778655238e-08
2868 2.40054394140543e-08
2869 2.40120119343601e-08
2870 2.39765860499119e-08
2871 2.39994317752235e-08
2872 2.39194033468948e-08
2873 2.39550566050184e-08
2874 2.38486634884794e-08
2875 2.39740103324948e-08
2876 2.39467858875742e-08
2877 2.39232846865889e-08
2878 2.39362396570186e-08
2879 2.392322784317e-08
2880 2.38192718882146e-08
2881 2.39209736463408e-08
2882 2.39150441672109e-08
2883 2.39326620743441e-08
2884 2.39231550125396e-08
2885 2.38393447204999e-08
2886 2.38919160011619e-08
2887 2.38564901167138e-08
2888 2.38540032171386e-08
2889 2.38619861647749e-08
2890 2.38045672062981e-08
2891 2.38657609230586e-08
2892 2.37299389027612e-08
2893 2.38168755828383e-08
2894 2.37878516884393e-08
2895 2.37922161971937e-08
2896 2.3809512583739e-08
2897 2.37474342412725e-08
2898 2.37830803939687e-08
2899 2.36993464852731e-08
2900 2.37632544752842e-08
2901 2.38157227272495e-08
2902 2.37301325256567e-08
2903 2.37933104330068e-08
2904 2.37364758959302e-08
2905 2.37532624680625e-08
2906 2.37173054529194e-08
2907 2.37606840869375e-08
2908 2.37117028234479e-08
2909 2.36725874458443e-08
2910 2.3698584428189e-08
2911 2.37414727877194e-08
2912 2.37063755292866e-08
2913 2.36655051111256e-08
2914 2.36554065224936e-08
2915 2.36624870808555e-08
2916 2.36769004402504e-08
2917 2.35620962740768e-08
2918 2.36678729947926e-08
2919 2.36446329182627e-08
2920 2.36231798567133e-08
2921 2.35832047223994e-08
2922 2.36187904789631e-08
2923 2.35795560854513e-08
2924 2.36066508563226e-08
2925 2.35270967152701e-08
2926 2.35664501246902e-08
2927 2.3591571363113e-08
2928 2.36137491782529e-08
2929 2.35743531362687e-08
2930 2.35574191265187e-08
2931 2.35646950841328e-08
2932 2.36055335278706e-08
2933 2.36057786651145e-08
2934 2.35369288503762e-08
2935 2.34804282683854e-08
2936 2.3547848115868e-08
2937 2.35228512224239e-08
2938 2.35438388784814e-08
2939 2.3551043781822e-08
2940 2.35006147875083e-08
2941 2.35185435570884e-08
2942 2.34206432025985e-08
2943 2.351715444604e-08
2944 2.34837163048951e-08
2945 2.34677752786183e-08
2946 2.34248478392374e-08
2947 2.3470743570897e-08
2948 2.34119958975043e-08
2949 2.34603056981086e-08
2950 2.34405934662618e-08
2951 2.34343655591829e-08
2952 2.33670132132602e-08
2953 2.34483721328616e-08
2954 2.34112054187108e-08
2955 2.34161454670812e-08
2956 2.34281785083112e-08
2957 2.33907258007093e-08
2958 2.34187531589214e-08
2959 2.33348451672555e-08
2960 2.34027819345783e-08
2961 2.33802346372158e-08
2962 2.34425332479304e-08
2963 2.33657946324684e-08
2964 2.33944934535657e-08
2965 2.33252119841154e-08
2966 2.33615722322611e-08
2967 2.33399397586709e-08
2968 2.34065886672852e-08
2969 2.32869474814379e-08
2970 2.33786234815625e-08
2971 2.33313208752861e-08
2972 2.3386979464135e-08
2973 2.33416006523157e-08
2974 2.33218298006932e-08
2975 2.33415544670379e-08
2976 2.33126318249788e-08
2977 2.33049934905694e-08
2978 2.3214111521952e-08
2979 2.32815846601397e-08
2980 2.32731984795009e-08
2981 2.3265664950145e-08
2982 2.33376091784976e-08
2983 2.3253663883338e-08
2984 2.32387460386008e-08
2985 2.3246807145938e-08
2986 2.32449846038207e-08
2987 2.32329000482423e-08
2988 2.32480825701487e-08
2989 2.32772254804559e-08
2990 2.32384493870086e-08
2991 2.31820056484366e-08
2992 2.31786074778029e-08
2993 2.31418475493683e-08
2994 2.3188947650965e-08
2995 2.31989076837635e-08
2996 2.31622898638761e-08
2997 2.31642953707478e-08
2998 2.31894450308801e-08
2999 2.31232526459735e-08
3000 1.51635362044544e-08
3001 1.52857602131462e-08
3002 1.53374397626749e-08
3003 1.53482915266068e-08
3004 1.53508956657333e-08
3005 1.53501567012881e-08
3006 1.53488795007206e-08
3007 1.5347580983871e-08
3008 1.53461847673952e-08
3009 1.53450443463043e-08
3010 1.53436392480444e-08
3011 1.53425023796672e-08
3012 1.53411701120376e-08
3013 1.53400172564488e-08
3014 1.5338997627623e-08
3015 1.53377399669807e-08
3016 1.53366208621719e-08
3017 1.53354644538695e-08
3018 1.53342583075755e-08
3019 1.53332635477454e-08
3020 1.53322883278406e-08
3021 1.53312118555959e-08
3022 1.53301176197829e-08
3023 1.53289239079868e-08
3024 1.53278527648126e-08
3025 1.53268349123437e-08
3026 1.53257069257506e-08
3027 1.53245238720956e-08
3028 1.53238080002893e-08
3029 1.53227563970404e-08
3030 1.53217296627872e-08
3031 1.53207349029572e-08
3032 1.53196175745052e-08
3033 1.53186086038204e-08
3034 1.53176458184134e-08
3035 1.53166936911475e-08
3036 1.53156243243302e-08
3037 1.53147183823421e-08
3038 1.5313602830247e-08
3039 1.53126578084084e-08
3040 1.53116896939309e-08
3041 1.53105652600516e-08
3042 1.53097747812581e-08
3043 1.53089096954773e-08
3044 1.53077710507432e-08
3045 1.5306788725411e-08
3046 1.53057779783694e-08
3047 1.53047050588384e-08
3048 1.53039838579616e-08
3049 1.53029517946379e-08
3050 1.53018984150322e-08
3051 1.53010226711103e-08
3052 1.53001220581928e-08
3053 1.5298924793683e-08
3054 1.52980828005411e-08
3055 1.52972230438309e-08
3056 1.52961305843746e-08
3057 1.52952512877391e-08
3058 1.52942405406975e-08
3059 1.52934269692651e-08
3060 1.52924730656423e-08
3061 1.52915369255879e-08
3062 1.52907020378734e-08
3063 1.52897428051801e-08
3064 1.5288760479848e-08
3065 1.52878474324325e-08
3066 1.52869894520791e-08
3067 1.52859964686058e-08
3068 1.52850265777715e-08
3069 1.52841046485719e-08
3070 1.52829731092652e-08
3071 1.52822590138157e-08
3072 1.52812731357699e-08
3073 1.52802748232261e-08
3074 1.52794239483001e-08
3075 1.52784931373162e-08
3076 1.52776138406807e-08
3077 1.52767078986926e-08
3078 1.52756332028048e-08
3079 1.52748551585091e-08
3080 1.52737982261897e-08
3081 1.52729437985499e-08
3082 1.52720129875661e-08
3083 1.52711798762084e-08
3084 1.52700909694659e-08
3085 1.52692614108219e-08
3086 1.52680499354574e-08
3087 1.52672985365143e-08
3088 1.52661971952739e-08
3089 1.52652805951448e-08
3090 1.5264371100443e-08
3091 1.52634740402391e-08
3092 1.52625663218942e-08
3093 1.52615129422884e-08
3094 1.52607064762833e-08
3095 1.52594896718483e-08
3096 1.52587880108968e-08
3097 1.52579033851907e-08
3098 1.52569334943564e-08
3099 1.5255931629099e-08
3100 1.52550860832434e-08
3101 1.52540469144924e-08
3102 1.52532226849189e-08
3103 1.52522545704414e-08
3104 1.52513184303871e-08
3105 1.52504302519674e-08
3106 1.52493697669343e-08
3107 1.52486467897006e-08
3108 1.52474743941866e-08
3109 1.52466288483311e-08
3110 1.52457211299861e-08
3111 1.52448276224959e-08
3112 1.52438897060847e-08
3113 1.52429411315325e-08
3114 1.52420707166812e-08
3115 1.52410351006438e-08
3116 1.52399923791791e-08
3117 1.52392622965181e-08
3118 1.52381378626387e-08
3119 1.52372159334391e-08
3120 1.52364343364297e-08
3121 1.52356278704247e-08
3122 1.52345585036073e-08
3123 1.52335744019183e-08
3124 1.52327555014153e-08
3125 1.52317607415853e-08
3126 1.52307961798215e-08
3127 1.52299097777586e-08
3128 1.52290038357705e-08
3129 1.52279948650857e-08
3130 1.52270054343262e-08
3131 1.52261137031928e-08
3132 1.5225127825147e-08
3133 1.52241810269516e-08
3134 1.52232981776024e-08
3135 1.52223105231997e-08
3136 1.52213797122158e-08
3137 1.5220370741531e-08
3138 1.52194949976092e-08
3139 1.52184860269244e-08
3140 1.52174930434512e-08
3141 1.52166528266662e-08
3142 1.52157451083212e-08
3143 1.52147627829891e-08
3144 1.52138159847937e-08
3145 1.52131018893442e-08
3146 1.52121231167257e-08
3147 1.52111301332525e-08
3148 1.52102330730486e-08
3149 1.52093448946289e-08
3150 1.52081955917538e-08
3151 1.52073518222551e-08
3152 1.52064476566238e-08
3153 1.52053480917402e-08
3154 1.52044794532458e-08
3155 1.52035752876145e-08
3156 1.520270664912e-08
3157 1.52015999788091e-08
3158 1.52008663434344e-08
3159 1.51998289510402e-08
3160 1.51988714947038e-08
3161 1.51979548945747e-08
3162 1.51969477002467e-08
3163 1.51959955729808e-08
3164 1.5195285030245e-08
3165 1.51941446091541e-08
3166 1.51933541303606e-08
3167 1.51922083801992e-08
3168 1.51913646107005e-08
3169 1.51903911671525e-08
3170 1.51894496980276e-08
3171 1.51885259924711e-08
3172 1.51874850473632e-08
3173 1.51866288433666e-08
3174 1.51856553998186e-08
3175 1.51848453810999e-08
3176 1.51838737139087e-08
3177 1.51829766537048e-08
3178 1.51819801175179e-08
3179 1.51811505588739e-08
3180 1.51799621761484e-08
3181 1.51791024194381e-08
3182 1.51782018065205e-08
3183 1.51771768486242e-08
3184 1.51761234690184e-08
3185 1.51752068688893e-08
3186 1.51742014509182e-08
3187 1.51732351127976e-08
3188 1.51722971963864e-08
3189 1.51713983598256e-08
3190 1.51703840600703e-08
3191 1.51694798944391e-08
3192 1.5168595268733e-08
3193 1.51677266302386e-08
3194 1.51667567394043e-08
3195 1.51658685609846e-08
3196 1.51647761015283e-08
3197 1.51638914758223e-08
3198 1.51629642175521e-08
3199 1.5161958799581e-08
3200 1.51611878607127e-08
3201 1.51601540210322e-08
3202 1.51591574848453e-08
3203 1.51582302265751e-08
3204 1.51572638884545e-08
3205 1.51562211669898e-08
3206 1.51553951610595e-08
3207 1.51542565163254e-08
3208 1.51536383441453e-08
3209 1.51524517377766e-08
3210 1.51516967861198e-08
3211 1.5150666499153e-08
3212 1.51496646338956e-08
3213 1.51487498101233e-08
3214 1.51480303856033e-08
3215 1.51468366738072e-08
3216 1.51459964570222e-08
3217 1.51448933394249e-08
3218 1.5144074438922e-08
3219 1.51433052764105e-08
3220 1.51423602545719e-08
3221 1.51412287152652e-08
3222 1.51404737636085e-08
3223 1.51393564351565e-08
3224 1.51385179947283e-08
3225 1.51373313883596e-08
3226 1.51365782130597e-08
3227 1.51355514788065e-08
3228 1.51346224441795e-08
3229 1.51336241316358e-08
3230 1.51326489117309e-08
3231 1.51317891550207e-08
3232 1.51308832130326e-08
3233 1.51298582551362e-08
3234 1.51288865879451e-08
3235 1.51280872273674e-08
3236 1.51270374004753e-08
3237 1.51260159952926e-08
3238 1.51252539382085e-08
3239 1.51243320090089e-08
3240 1.51234100798092e-08
3241 1.51222554478636e-08
3242 1.5121443652788e-08
3243 1.51205465925841e-08
3244 1.51195820308203e-08
3245 1.5118601481845e-08
3246 1.51177772522715e-08
3247 1.51168322304329e-08
3248 1.51161074768424e-08
3249 1.51149190941169e-08
3250 1.51138834780795e-08
3251 1.51130219450124e-08
3252 1.51122243607915e-08
3253 1.51112100610362e-08
3254 1.51102632628408e-08
3255 1.51092187650192e-08
3256 1.51084744715035e-08
3257 1.51073873411178e-08
3258 1.51062593545248e-08
3259 1.51055168373659e-08
3260 1.51045274066064e-08
3261 1.51035308704195e-08
3262 1.51027279571281e-08
3263 1.51017012228749e-08
3264 1.51007615301069e-08
3265 1.50998555881188e-08
3266 1.50990029368359e-08
3267 1.50982781832454e-08
3268 1.50972194745691e-08
3269 1.50962495837348e-08
3270 1.50952779165436e-08
3271 1.50943382237756e-08
3272 1.50933541220866e-08
3273 1.50923558095428e-08
3274 1.50914996055462e-08
3275 1.50904782003636e-08
3276 1.50896930506406e-08
3277 1.50888315175735e-08
3278 1.5087874061237e-08
3279 1.50867212056482e-08
3280 1.50859680303483e-08
3281 1.50851011682107e-08
3282 1.50841099610943e-08
3283 1.50830370415633e-08
3284 1.508214531043e-08
3285 1.50813743715617e-08
3286 1.508036362452e-08
3287 1.50794612352456e-08
3288 1.50784309482788e-08
3289 1.50775196772202e-08
3290 1.50765355755311e-08
3291 1.50754484451454e-08
3292 1.50747663241191e-08
3293 1.50736401138829e-08
3294 1.50727998970979e-08
3295 1.50717749392015e-08
3296 1.50710075530469e-08
3297 1.50700092405032e-08
3298 1.50691121802993e-08
3299 1.50680463661956e-08
3300 1.50673429288872e-08
3301 1.50663037601362e-08
3302 1.50654546615669e-08
3303 1.50645647067904e-08
3304 1.50637760043537e-08
3305 1.50626870976112e-08
3306 1.50618273409009e-08
3307 1.50607988302909e-08
3308 1.50599337445101e-08
3309 1.50588661540496e-08
3310 1.50580312663351e-08
3311 1.50570471646461e-08
3312 1.50561607625832e-08
3313 1.50551962008194e-08
3314 1.50542351917693e-08
3315 1.50535246490335e-08
3316 1.5052600943477e-08
3317 1.50515866437217e-08
3318 1.50505687912528e-08
3319 1.5049758772534e-08
3320 1.50487871053429e-08
3321 1.50477763583012e-08
3322 1.504718127876e-08
3323 1.50460834902333e-08
3324 1.50452414970914e-08
3325 1.50441703539173e-08
3326 1.50433816514806e-08
3327 1.50423655753684e-08
3328 1.50415591093633e-08
3329 1.50407526433582e-08
3330 1.5039749001744e-08
3331 1.50388022035486e-08
3332 1.50378003382912e-08
3333 1.50369316997967e-08
3334 1.50360790485138e-08
3335 1.50350860650406e-08
3336 1.50343701932343e-08
3337 1.50332937209896e-08
3338 1.50324250824951e-08
3339 1.50314445335198e-08
3340 1.50306433965852e-08
3341 1.50295154099922e-08
3342 1.50285952571494e-08
3343 1.50277994492853e-08
3344 1.50269130472225e-08
3345 1.50260799358648e-08
3346 1.50248808949982e-08
3347 1.50240673235658e-08
3348 1.50231773687892e-08
3349 1.50222909667264e-08
3350 1.50214081173772e-08
3351 1.50204204629745e-08
3352 1.50195464954095e-08
3353 1.50185819336457e-08
3354 1.50177008606533e-08
3355 1.50166243884087e-08
3356 1.50158498968267e-08
3357 1.50148053990051e-08
3358 1.5013977616718e-08
3359 1.50129793041742e-08
3360 1.50123913300604e-08
3361 1.5011211829119e-08
3362 1.50101211460196e-08
3363 1.50094230377817e-08
3364 1.5008435383379e-08
3365 1.50073660165617e-08
3366 1.50064956017104e-08
3367 1.50056411740707e-08
3368 1.50048062863561e-08
3369 1.50036925106178e-08
3370 1.50028967027538e-08
3371 1.50018255595796e-08
3372 1.50008823140979e-08
3373 1.49999834775372e-08
3374 1.49992089859552e-08
3375 1.49982728459008e-08
3376 1.49972141372245e-08
3377 1.49962904316681e-08
3378 1.49954697548083e-08
3379 1.499469881594e-08
3380 1.4993725372392e-08
3381 1.49927608106282e-08
3382 1.49915777569731e-08
3383 1.49905527990768e-08
3384 1.49897338985738e-08
3385 1.49890837519706e-08
3386 1.49881405064889e-08
3387 1.49871794974388e-08
3388 1.49863339515832e-08
3389 1.49853924824583e-08
3390 1.49845895691669e-08
3391 1.49834527007897e-08
3392 1.49825059025943e-08
3393 1.49817509509376e-08
3394 1.49808307980948e-08
3395 1.49799106452519e-08
3396 1.49791148373879e-08
3397 1.49780969849189e-08
3398 1.49772692026318e-08
3399 1.49762460210923e-08
3400 1.49755976508459e-08
3401 1.49743453192741e-08
3402 1.49734766807796e-08
3403 1.49726915310566e-08
3404 1.49718832886947e-08
3405 1.49708316854458e-08
3406 1.49700483120796e-08
3407 1.49689185491297e-08
3408 1.49681760319709e-08
3409 1.49672558791281e-08
3410 1.49663623716378e-08
3411 1.49654777459318e-08
3412 1.49646659508562e-08
3413 1.49636605328851e-08
3414 1.49627599199675e-08
3415 1.49619463485351e-08
3416 1.49608538890789e-08
3417 1.49601468990568e-08
3418 1.49592302989277e-08
3419 1.49583758712879e-08
3420 1.4957489469225e-08
3421 1.49566776741494e-08
3422 1.49554804096397e-08
3423 1.49547183525556e-08
3424 1.4953828397779e-08
3425 1.49528798232268e-08
3426 1.49520307246576e-08
3427 1.49509258307035e-08
3428 1.49502579205318e-08
3429 1.49493590839711e-08
3430 1.49483518896432e-08
3431 1.49475276600697e-08
3432 1.49463978971198e-08
3433 1.49456003128989e-08
3434 1.49448453612422e-08
3435 1.4943962511893e-08
3436 1.49430405826934e-08
3437 1.49422003659083e-08
3438 1.49410919192405e-08
3439 1.49403422966543e-08
3440 1.4939427472882e-08
3441 1.49384593584045e-08
3442 1.49376084834785e-08
3443 1.49366545798557e-08
3444 1.49356864653782e-08
3445 1.49348498013069e-08
3446 1.49340895205796e-08
3447 1.49331071952474e-08
3448 1.49322119114004e-08
3449 1.49311514263673e-08
3450 1.49304852925525e-08
3451 1.49295988904896e-08
3452 1.49285526163112e-08
3453 1.49276218053274e-08
3454 1.49268135629654e-08
3455 1.49258259085627e-08
3456 1.49251402348227e-08
3457 1.49239873792339e-08
3458 1.49233052582076e-08
3459 1.49221790479714e-08
3460 1.49213015276928e-08
3461 1.49203227550743e-08
3462 1.49196424104048e-08
3463 1.49187595610556e-08
3464 1.49177967756486e-08
3465 1.49169885332867e-08
3466 1.49161696327837e-08
3467 1.49150505279749e-08
3468 1.49143062344592e-08
3469 1.49134500304626e-08
3470 1.49124339543505e-08
3471 1.49116186065612e-08
3472 1.4910725099071e-08
3473 1.49097481028093e-08
3474 1.49090162437915e-08
3475 1.49080854328076e-08
3476 1.49071457400396e-08
3477 1.49062024945579e-08
3478 1.49054120157643e-08
3479 1.49047014730286e-08
3480 1.49035717100787e-08
3481 1.49027883367125e-08
3482 1.49019268036454e-08
3483 1.49012073791255e-08
3484 1.49002206129012e-08
3485 1.4899337763552e-08
3486 1.4898425604315e-08
3487 1.4897496569688e-08
3488 1.48965142443558e-08
3489 1.48957077783507e-08
3490 1.48947778555453e-08
3491 1.48939394151171e-08
3492 1.48933114729743e-08
3493 1.48921852627382e-08
3494 1.48914338637951e-08
3495 1.48904781838155e-08
3496 1.48896655005615e-08
3497 1.48888332773822e-08
3498 1.48879788497425e-08
3499 1.48869352400993e-08
3500 1.48860950233143e-08
3501 1.4885190857683e-08
3502 1.4884217414135e-08
3503 1.48832528523712e-08
3504 1.48825236578887e-08
3505 1.48816914347094e-08
3506 1.48808849687043e-08
3507 1.48800065602472e-08
3508 1.48792471676984e-08
3509 1.48782062225905e-08
3510 1.48775720631988e-08
3511 1.48766430285718e-08
3512 1.48758152462847e-08
3513 1.48747840711394e-08
3514 1.48740300076611e-08
3515 1.48733567684189e-08
3516 1.48722234527554e-08
3517 1.48714560666008e-08
3518 1.48704897284802e-08
3519 1.48695464829984e-08
3520 1.4868616560193e-08
3521 1.48677541389475e-08
3522 1.48669094812703e-08
3523 1.48660648235932e-08
3524 1.48650798337258e-08
3525 1.48642378405839e-08
3526 1.48634704544293e-08
3527 1.48626124740758e-08
3528 1.4861709196623e-08
3529 1.48607242067555e-08
3530 1.48600545202271e-08
3531 1.48590126869408e-08
3532 1.485813072577e-08
3533 1.48573615632586e-08
3534 1.48564929247641e-08
3535 1.48556456025517e-08
3536 1.4854801833053e-08
3537 1.4853872798426e-08
3538 1.48529748500437e-08
3539 1.48522802945195e-08
3540 1.48513974451703e-08
3541 1.48505572283852e-08
3542 1.48497099061728e-08
3543 1.48488075168984e-08
3544 1.48477692363258e-08
3545 1.48470595817685e-08
3546 1.4846108342681e-08
3547 1.48454040171941e-08
3548 1.48444190273267e-08
3549 1.48435912450395e-08
3550 1.48427616863955e-08
3551 1.48418468626232e-08
3552 1.48410599365434e-08
3553 1.484015133002e-08
3554 1.48391192666963e-08
3555 1.48383580977907e-08
3556 1.48375054465077e-08
3557 1.48367078622869e-08
3558 1.48358170193319e-08
3559 1.48348826556344e-08
3560 1.4834117934015e-08
3561 1.48330672189445e-08
3562 1.48322936155409e-08
3563 1.48315040249258e-08
3564 1.48305270286642e-08
3565 1.48297196744807e-08
3566 1.48287933043889e-08
3567 1.48280721035121e-08
3568 1.48272700783991e-08
3569 1.48261936061544e-08
3570 1.48253818110788e-08
3571 1.48244874154102e-08
3572 1.48236285468784e-08
3573 1.48227616847407e-08
3574 1.48218761708563e-08
3575 1.48211354300543e-08
3576 1.48202534688835e-08
3577 1.48194372329158e-08
3578 1.48185135273593e-08
3579 1.48176679815037e-08
3580 1.48169032598844e-08
3581 1.48160319568547e-08
3582 1.48152086154596e-08
3583 1.48143204370399e-08
3584 1.48134562394375e-08
3585 1.4812477466819e-08
3586 1.48118832754562e-08
3587 1.48107730524316e-08
3588 1.48099488228581e-08
3589 1.48091112706084e-08
3590 1.48082417439355e-08
3591 1.4807417514362e-08
3592 1.48066687799542e-08
3593 1.48056082949211e-08
3594 1.4805066506085e-08
3595 1.48040841807529e-08
3596 1.48031054081343e-08
3597 1.48023309165524e-08
3598 1.48015457668293e-08
3599 1.4800736636289e-08
3600 1.4799862668724e-08
3601 1.47990766308226e-08
3602 1.4798120950843e-08
3603 1.4797401526323e-08
3604 1.47963703511778e-08
3605 1.47955727669569e-08
3606 1.47945939943384e-08
3607 1.47937946337606e-08
3608 1.47929588578677e-08
3609 1.47921772608584e-08
3610 1.47912224690572e-08
3611 1.47904444247615e-08
3612 1.47896281887938e-08
3613 1.47888536972118e-08
3614 1.47878251866018e-08
3615 1.47870427014141e-08
3616 1.47861065613597e-08
3617 1.47854040122297e-08
3618 1.47844723130675e-08
3619 1.47835308439426e-08
3620 1.47826311192034e-08
3621 1.47818877138661e-08
3622 1.47809409156707e-08
3623 1.47801779704082e-08
3624 1.4779383938901e-08
3625 1.47786689552731e-08
3626 1.47776546555178e-08
3627 1.47769174674295e-08
3628 1.4776154522167e-08
3629 1.47752050594363e-08
3630 1.47744430023522e-08
3631 1.47734011690659e-08
3632 1.47726293420192e-08
3633 1.4771928569246e-08
3634 1.47710856879257e-08
3635 1.4770296985489e-08
3636 1.47693262064763e-08
3637 1.47684442453055e-08
3638 1.47675951467363e-08
3639 1.47667913452665e-08
3640 1.47660808025307e-08
3641 1.47653036464135e-08
3642 1.47641987524594e-08
3643 1.47633505420686e-08
3644 1.47626790791833e-08
3645 1.47618646195724e-08
3646 1.47608885114892e-08
3647 1.47599710231816e-08
3648 1.4759347521931e-08
3649 1.47583953946651e-08
3650 1.47574690245733e-08
3651 1.4756708743846e-08
3652 1.47559040541978e-08
3653 1.47551180162964e-08
3654 1.47542857931171e-08
3655 1.47532723815402e-08
3656 1.47524552573941e-08
3657 1.47517713600109e-08
3658 1.4750819232745e-08
3659 1.47500660574451e-08
3660 1.47491689972412e-08
3661 1.47483429913109e-08
3662 1.47474032985428e-08
3663 1.47467726918649e-08
3664 1.47458711907689e-08
3665 1.47450434084817e-08
3666 1.47440788467179e-08
3667 1.47434029429405e-08
3668 1.47424534802099e-08
3669 1.47415937234996e-08
3670 1.47408831807638e-08
3671 1.4740041187622e-08
3672 1.47393626193093e-08
3673 1.47383163451309e-08
3674 1.47375969206109e-08
3675 1.47364502822711e-08
3676 1.47356526980502e-08
3677 1.47350185386586e-08
3678 1.4734156117413e-08
3679 1.47333203415201e-08
3680 1.47324215049593e-08
3681 1.47316425724853e-08
3682 1.47307570586008e-08
3683 1.4729870656538e-08
3684 1.47291689955864e-08
3685 1.4728255948171e-08
3686 1.47274690220911e-08
3687 1.47265675209951e-08
3688 1.47257530613842e-08
3689 1.472505850586e-08
3690 1.47241534520504e-08
3691 1.47232492864191e-08
3692 1.47223229163274e-08
3693 1.47214631596171e-08
3694 1.47207490641676e-08
3695 1.47197152244871e-08
3696 1.4718864349561e-08
3697 1.47181919984973e-08
3698 1.47173304654302e-08
3699 1.47166039354829e-08
3700 1.4715634932827e-08
3701 1.47149021856308e-08
3702 1.47141170359077e-08
3703 1.47131649086418e-08
3704 1.47124072924498e-08
3705 1.47117251714235e-08
3706 1.47108325521117e-08
3707 1.47101104630565e-08
3708 1.47093235369766e-08
3709 1.47085108537226e-08
3710 1.47074965539673e-08
3711 1.47067398259537e-08
3712 1.47059386890191e-08
3713 1.47051633092588e-08
3714 1.47041347986487e-08
3715 1.47034606712282e-08
3716 1.4702704831393e-08
3717 1.47018832663548e-08
3718 1.47009684425825e-08
3719 1.47003449413319e-08
3720 1.46993661687134e-08
3721 1.46983989424143e-08
3722 1.46975702719487e-08
3723 1.46969583170176e-08
3724 1.46959227009802e-08
3725 1.46953080815138e-08
3726 1.46942999990074e-08
3727 1.46934242550856e-08
3728 1.46926435462547e-08
3729 1.46917695786897e-08
3730 1.46909409082241e-08
3731 1.46902774389446e-08
3732 1.46893137653592e-08
3733 1.46884673313252e-08
3734 1.46876750761749e-08
3735 1.46868535111366e-08
3736 1.46860328342768e-08
3737 1.46852237037365e-08
3738 1.46843985859846e-08
3739 1.4683542381988e-08
3740 1.46825405167306e-08
3741 1.46820022806082e-08
3742 1.46811487411469e-08
3743 1.46802463518725e-08
3744 1.46795136046762e-08
3745 1.46786396371112e-08
3746 1.46778420528904e-08
3747 1.46770897657689e-08
3748 1.46761918173866e-08
3749 1.46753089680374e-08
3750 1.46744474349703e-08
3751 1.46736356398947e-08
3752 1.46728798000595e-08
3753 1.46720351423824e-08
3754 1.46710457116228e-08
3755 1.46703813541649e-08
3756 1.46695890990145e-08
3757 1.46687693103331e-08
3758 1.46680170232116e-08
3759 1.46672265444181e-08
3760 1.46664458355872e-08
3761 1.46655283472796e-08
3762 1.46648346799338e-08
3763 1.46639198561616e-08
3764 1.46631080610859e-08
3765 1.46622838315125e-08
3766 1.46615688478846e-08
3767 1.46604559603247e-08
3768 1.46597960437589e-08
3769 1.4659197411504e-08
3770 1.46582026516739e-08
3771 1.46572602943706e-08
3772 1.46565346526017e-08
3773 1.46556340396842e-08
3774 1.46548524426748e-08
3775 1.46541134782296e-08
3776 1.46532412870215e-08
3777 1.46524792299374e-08
3778 1.46516274668329e-08
3779 1.46508387643962e-08
3780 1.46500775954905e-08
3781 1.4649204516104e-08
3782 1.4648430024522e-08
3783 1.46476040185917e-08
3784 1.46467309392051e-08
3785 1.46459546712663e-08
3786 1.46452139304643e-08
3787 1.46442014070658e-08
3788 1.46433603021023e-08
3789 1.4642633772155e-08
3790 1.46416736512833e-08
3791 1.46410270573938e-08
3792 1.46401299971899e-08
3793 1.46392657995875e-08
3794 1.46386165411627e-08
3795 1.46377230336725e-08
3796 1.46369307785221e-08
3797 1.46361731623301e-08
3798 1.46352094887447e-08
3799 1.46344181217728e-08
3800 1.4633696920896e-08
3801 1.46328638095383e-08
3802 1.46319480975876e-08
3803 1.46312224558187e-08
3804 1.46304754977677e-08
3805 1.46297640668536e-08
3806 1.46289416136369e-08
3807 1.46282017610133e-08
3808 1.46274627965681e-08
3809 1.46267042921977e-08
3810 1.46257974620312e-08
3811 1.46250203059139e-08
3812 1.46242991050372e-08
3813 1.46234224729369e-08
3814 1.46225858088656e-08
3815 1.46217429275453e-08
3816 1.46211034390831e-08
3817 1.46202481232649e-08
3818 1.46195358041723e-08
3819 1.46185890059769e-08
3820 1.46176155624289e-08
3821 1.46169600867552e-08
3822 1.46160576974808e-08
3823 1.4615336496604e-08
3824 1.46145673340925e-08
3825 1.46136622802828e-08
3826 1.46128993350203e-08
3827 1.46120999744426e-08
3828 1.46112677512633e-08
3829 1.46105048060008e-08
3830 1.46097240971699e-08
3831 1.46090650687825e-08
3832 1.4608113829695e-08
3833 1.46074992102285e-08
3834 1.46064431660875e-08
3835 1.46056455818666e-08
3836 1.46050549432175e-08
3837 1.46042244963951e-08
3838 1.46034047077137e-08
3839 1.46026675196254e-08
3840 1.46015999291649e-08
3841 1.46010972201793e-08
3842 1.46002028245107e-08
3843 1.45993235278752e-08
3844 1.45986795985209e-08
3845 1.45978393817359e-08
3846 1.45971004172907e-08
3847 1.45961802644479e-08
3848 1.45952796515303e-08
3849 1.45946419394249e-08
3850 1.45936729367691e-08
3851 1.45929668349254e-08
3852 1.45922705030443e-08
3853 1.45914942351055e-08
3854 1.45907055326688e-08
3855 1.45899514691905e-08
3856 1.45890481917377e-08
3857 1.45882612656578e-08
3858 1.45875009849306e-08
3859 1.45868082057632e-08
3860 1.45860150624344e-08
3861 1.45851881683257e-08
3862 1.45842733445534e-08
3863 1.45835405973571e-08
3864 1.45828513709034e-08
3865 1.45819747388032e-08
3866 1.45813396912331e-08
3867 1.45804222029255e-08
3868 1.45796326123104e-08
3869 1.45788456862306e-08
3870 1.4577928197923e-08
3871 1.45772229842578e-08
3872 1.45765151060573e-08
3873 1.45757841352179e-08
3874 1.45749359248271e-08
3875 1.4574123241573e-08
3876 1.45732057532655e-08
3877 1.45723522138042e-08
3878 1.45717660160471e-08
3879 1.45708023424618e-08
3880 1.45701379850038e-08
3881 1.45692942155051e-08
3882 1.4568576567342e-08
3883 1.45678233920421e-08
3884 1.45669334372656e-08
3885 1.45662282236003e-08
3886 1.45653906713505e-08
3887 1.45645211446777e-08
3888 1.45636676052163e-08
3889 1.45629188708085e-08
3890 1.45623006986284e-08
3891 1.45613352486862e-08
3892 1.45604879264738e-08
3893 1.45596237288714e-08
3894 1.45591112499233e-08
3895 1.45582603749972e-08
3896 1.45573020304823e-08
3897 1.45565266507219e-08
3898 1.45558569641935e-08
3899 1.45549643448817e-08
3900 1.45543044283158e-08
3901 1.45536045437211e-08
3902 1.45527510042598e-08
3903 1.45520377969888e-08
3904 1.4551111426897e-08
3905 1.45503271653524e-08
3906 1.45495278047747e-08
3907 1.45488270320016e-08
3908 1.45480765212369e-08
3909 1.45471696910704e-08
3910 1.45463827649905e-08
3911 1.45456224842633e-08
3912 1.4544889737067e-08
3913 1.45441951815428e-08
3914 1.45433922682514e-08
3915 1.45426728437315e-08
3916 1.45418495023364e-08
3917 1.45410030683024e-08
3918 1.45402401230399e-08
3919 1.45394523087816e-08
3920 1.45388865391283e-08
3921 1.45379210891861e-08
3922 1.45372291981971e-08
3923 1.45364307257978e-08
3924 1.4535654457859e-08
3925 1.453496256687e-08
3926 1.453393405626e-08
3927 1.45333416412541e-08
3928 1.45323904021666e-08
3929 1.45317988753391e-08
3930 1.45309568821972e-08
3931 1.4530107783628e-08
3932 1.45293590492201e-08
3933 1.45286289665592e-08
3934 1.45278589158693e-08
3935 1.45270036000511e-08
3936 1.45262486483944e-08
3937 1.45256269235006e-08
3938 1.45248044702839e-08
3939 1.4523951819001e-08
3940 1.45232759152236e-08
3941 1.45224996472848e-08
3942 1.45217073921344e-08
3943 1.45209435586935e-08
3944 1.45200553802738e-08
3945 1.45195357958983e-08
3946 1.45186191957691e-08
3947 1.4517763879951e-08
3948 1.45170462317878e-08
3949 1.45163179254837e-08
3950 1.45155283348686e-08
3951 1.45146952235109e-08
3952 1.45139535945304e-08
3953 1.45131977546953e-08
3954 1.45124898764948e-08
3955 1.45115706118304e-08
3956 1.45109106952646e-08
3957 1.45100216286664e-08
3958 1.45094913861499e-08
3959 1.45084690927888e-08
3960 1.45077319047004e-08
3961 1.45068890233802e-08
3962 1.45061127554413e-08
3963 1.45052547750879e-08
3964 1.45046525901193e-08
3965 1.45037324372765e-08
3966 1.4503024559076e-08
3967 1.45022074349299e-08
3968 1.45013521191117e-08
3969 1.45006291418781e-08
3970 1.44997756024168e-08
3971 1.4499026868009e-08
3972 1.44982132965765e-08
3973 1.44976191052137e-08
3974 1.44968073101381e-08
3975 1.44959511061415e-08
3976 1.44950433877966e-08
3977 1.44945353497405e-08
3978 1.44935414780889e-08
3979 1.44927261302996e-08
3980 1.44920351274891e-08
3981 1.4491317479326e-08
3982 1.4490680655399e-08
3983 1.44896405984696e-08
3984 1.44888572251034e-08
3985 1.44882204011765e-08
3986 1.44872647211969e-08
3987 1.44866687534773e-08
3988 1.4485637578332e-08
3989 1.44852210226531e-08
3990 1.44842280391799e-08
3991 1.44835112791952e-08
3992 1.44826541870202e-08
3993 1.4481987165027e-08
3994 1.44811496127772e-08
3995 1.44806033830491e-08
3996 1.44796272749659e-08
3997 1.44788145917119e-08
3998 1.44780427646651e-08
3999 1.4477430809734e-08
4000 1.44765106568912e-08
4001 1.44757947850849e-08
4002 1.44751597375148e-08
4003 1.44741942875726e-08
4004 1.44734659812684e-08
4005 1.44728105055947e-08
4006 1.44719685124528e-08
4007 1.44711291838462e-08
4008 1.44703724558326e-08
4009 1.4469649478599e-08
4010 1.44688367953449e-08
4011 1.44680925018292e-08
4012 1.44673464319567e-08
4013 1.44665355250595e-08
4014 1.44658232059669e-08
4015 1.44651108868743e-08
4016 1.44643301780434e-08
4017 1.44636977950086e-08
4018 1.44627918530205e-08
4019 1.44620866393552e-08
4020 1.44614658026399e-08
4021 1.44605101226603e-08
4022 1.44598200080281e-08
4023 1.44590552864088e-08
4024 1.44584015870919e-08
4025 1.44574343607928e-08
4026 1.44567060544887e-08
4027 1.44560416970307e-08
4028 1.44551082215116e-08
4029 1.445442343595e-08
4030 1.44536436152976e-08
4031 1.4452997021408e-08
4032 1.44521434819467e-08
4033 1.44514062938583e-08
4034 1.44505003518702e-08
4035 1.44499239240758e-08
4036 1.44491956177717e-08
4037 1.44483216502067e-08
4038 1.44476395291804e-08
4039 1.44468401686026e-08
4040 1.44459786355355e-08
4041 1.44451854922067e-08
4042 1.44445104766078e-08
4043 1.44438789817514e-08
4044 1.44430396531448e-08
4045 1.44423406567284e-08
4046 1.44416354430632e-08
4047 1.44407907853861e-08
4048 1.4440094453505e-08
4049 1.44392524603632e-08
4050 1.44385037259553e-08
4051 1.44377159116971e-08
4052 1.44371039567659e-08
4053 1.44363827558891e-08
4054 1.44354634912247e-08
4055 1.44345149166725e-08
4056 1.44338434537872e-08
4057 1.44331693263666e-08
4058 1.44323939466062e-08
4059 1.44315972505638e-08
4060 1.44309391103548e-08
4061 1.44300829063582e-08
4062 1.44294922677091e-08
4063 1.44286067538246e-08
4064 1.44277532143633e-08
4065 1.44271172786148e-08
4066 1.44263303525349e-08
4067 1.44255203338162e-08
4068 1.44248106792588e-08
4069 1.44239722388306e-08
4070 1.44231764309666e-08
4071 1.44223681886047e-08
4072 1.4421778438134e-08
4073 1.4420959537631e-08
4074 1.4420335148202e-08
4075 1.44194185480728e-08
4076 1.44188057049632e-08
4077 1.44180170025265e-08
4078 1.44173206706455e-08
4079 1.44166829585402e-08
4080 1.44157095149922e-08
4081 1.44150176240032e-08
4082 1.44143177394085e-08
4083 1.44136240720627e-08
4084 1.44127270118588e-08
4085 1.44119693956668e-08
4086 1.44113165845283e-08
4087 1.44104816968138e-08
4088 1.44097986876091e-08
4089 1.4408930937293e-08
4090 1.4408406023847e-08
4091 1.44075746888461e-08
4092 1.4406928983135e-08
4093 1.44061349516278e-08
4094 1.44054315143194e-08
4095 1.4404645476418e-08
4096 1.44038727611928e-08
4097 1.44030920523619e-08
4098 1.44022855863568e-08
4099 1.44015217529159e-08
4100 1.44009133506984e-08
4101 1.44001095492285e-08
4102 1.43992116008462e-08
4103 1.43984184575174e-08
4104 1.43977336719558e-08
4105 1.43969511867681e-08
4106 1.43963019283433e-08
4107 1.43953604592184e-08
4108 1.43947342934325e-08
4109 1.43938807539712e-08
4110 1.43932625817911e-08
4111 1.43923974960103e-08
4112 1.43915874772915e-08
4113 1.43911389471896e-08
4114 1.43903644556076e-08
4115 1.43894451909432e-08
4116 1.43886991210707e-08
4117 1.43879521630197e-08
4118 1.43872807001344e-08
4119 1.43864724577725e-08
4120 1.43856784262653e-08
4121 1.43850575895499e-08
4122 1.43842520117232e-08
4123 1.4383529922668e-08
4124 1.4382761648335e-08
4125 1.4381980939504e-08
4126 1.43813858599628e-08
4127 1.43805216623605e-08
4128 1.43798644103299e-08
4129 1.43789353757029e-08
4130 1.43783465134106e-08
4131 1.43775080729824e-08
4132 1.43768250637777e-08
4133 1.43760336968057e-08
4134 1.43754537162977e-08
4135 1.43746659020394e-08
4136 1.43738105862212e-08
4137 1.43730654045271e-08
4138 1.43723974943555e-08
4139 1.43715936928857e-08
4140 1.4370970191635e-08
4141 1.43701255339579e-08
4142 1.43693705823011e-08
4143 1.43686209597149e-08
4144 1.43679796948959e-08
4145 1.4367209644206e-08
4146 1.43665621621381e-08
4147 1.43657814533071e-08
4148 1.43648364314686e-08
4149 1.4364162304048e-08
4150 1.43634748539512e-08
4151 1.43626763815519e-08
4152 1.43620138004508e-08
4153 1.43612810532545e-08
4154 1.43605207725273e-08
4155 1.43598564150693e-08
4156 1.43590446199937e-08
4157 1.43583163136896e-08
4158 1.43575382693939e-08
4159 1.4356945854388e-08
4160 1.4356016819761e-08
4161 1.43553213760583e-08
4162 1.43544900410575e-08
4163 1.43537395302928e-08
4164 1.43530103358103e-08
4165 1.43523628537423e-08
4166 1.4351625665654e-08
4167 1.43507694616574e-08
4168 1.43500091809301e-08
4169 1.43494007787126e-08
4170 1.43485410220023e-08
4171 1.43478429137645e-08
4172 1.43472371760822e-08
4173 1.43464404800397e-08
4174 1.43457734580466e-08
4175 1.43448399825274e-08
4176 1.43442440148078e-08
4177 1.43434633059769e-08
4178 1.43426301946192e-08
4179 1.43420182396881e-08
4180 1.43412783870644e-08
4181 1.43404914609846e-08
4182 1.4339817333564e-08
4183 1.43390437301605e-08
4184 1.43383136474995e-08
4185 1.43374547789676e-08
4186 1.43366634119957e-08
4187 1.43360967541639e-08
4188 1.43352636428062e-08
4189 1.43345717518173e-08
4190 1.43338816371852e-08
4191 1.43329623725208e-08
4192 1.4332296238706e-08
4193 1.43315270761946e-08
4194 1.43308049871393e-08
4195 1.43300651345157e-08
4196 1.43292355758717e-08
4197 1.43287612885956e-08
4198 1.43277487651972e-08
4199 1.43271003949508e-08
4200 1.43263285679041e-08
4201 1.4325759245537e-08
4202 1.43250202810918e-08
4203 1.43241596362031e-08
4204 1.4323469521571e-08
4205 1.43227190108064e-08
4206 1.43220848514147e-08
4207 1.43213041425838e-08
4208 1.43206433378396e-08
4209 1.43197222968183e-08
4210 1.4319200936086e-08
4211 1.43183704892635e-08
4212 1.43175613587232e-08
4213 1.43167957489254e-08
4214 1.43161571486417e-08
4215 1.431548923847e-08
4216 1.43146188236187e-08
4217 1.43139002872772e-08
4218 1.43132972141302e-08
4219 1.43125866713945e-08
4220 1.43117748763189e-08
4221 1.43110394645873e-08
4222 1.43103795480215e-08
4223 1.43095526539128e-08
4224 1.43087834914013e-08
4225 1.43080081116409e-08
4226 1.43072389491294e-08
4227 1.43065745916715e-08
4228 1.43058729307199e-08
4229 1.43052103496188e-08
4230 1.43044323053232e-08
4231 1.43038221267489e-08
4232 1.43029366128644e-08
4233 1.43022118592739e-08
4234 1.43014968756461e-08
4235 1.43007596875577e-08
4236 1.43001672725518e-08
4237 1.42993821228288e-08
4238 1.42985081552638e-08
4239 1.42978890949053e-08
4240 1.4297029338195e-08
4241 1.42964626803632e-08
4242 1.42957201632044e-08
4243 1.42949714287965e-08
4244 1.42941294356547e-08
4245 1.42934881708356e-08
4246 1.42925795643123e-08
4247 1.42920608681152e-08
4248 1.42913485490226e-08
4249 1.42904559297108e-08
4250 1.42899576616173e-08
4251 1.42891911636411e-08
4252 1.4288308314292e-08
4253 1.42877665254559e-08
4254 1.4287013350156e-08
4255 1.42862823793166e-08
4256 1.42854013063243e-08
4257 1.42846348083481e-08
4258 1.42841072303668e-08
4259 1.42832572436191e-08
4260 1.42825635762733e-08
4261 1.42820013593337e-08
4262 1.4281162030727e-08
4263 1.42804950087339e-08
4264 1.42797258462224e-08
4265 1.42788492141221e-08
4266 1.42782692336141e-08
4267 1.42774387867917e-08
4268 1.42768286082173e-08
4269 1.42760097077144e-08
4270 1.42752147880287e-08
4271 1.42745841813507e-08
4272 1.42738558750466e-08
4273 1.42730689489667e-08
4274 1.42721772178334e-08
4275 1.42715252948733e-08
4276 1.42707614614324e-08
4277 1.42701059857586e-08
4278 1.42692817561851e-08
4279 1.42685676607357e-08
4280 1.42678029391163e-08
4281 1.42672487157824e-08
4282 1.42664395852421e-08
4283 1.42657885504605e-08
4284 1.42649616563517e-08
4285 1.42641773948071e-08
4286 1.42634917210671e-08
4287 1.42626976895599e-08
4288 1.42621621179728e-08
4289 1.42613574283246e-08
4290 1.42606122466304e-08
4291 1.42599079211436e-08
4292 1.42591174423501e-08
4293 1.42584593021411e-08
4294 1.42577709638658e-08
4295 1.42570604211301e-08
4296 1.42562361915566e-08
4297 1.42554741344725e-08
4298 1.42548328696535e-08
4299 1.42539944292253e-08
4300 1.42534783975634e-08
4301 1.42526426216705e-08
4302 1.42519915868888e-08
4303 1.42512961431862e-08
4304 1.42505056643927e-08
4305 1.42499114730299e-08
4306 1.42489540166935e-08
4307 1.4248466406741e-08
4308 1.42476679343417e-08
4309 1.42468463693035e-08
4310 1.42460772067921e-08
4311 1.42453924212305e-08
4312 1.4244595725188e-08
4313 1.42439109396264e-08
4314 1.42432048377827e-08
4315 1.42425671256774e-08
4316 1.42417801995975e-08
4317 1.42411105130691e-08
4318 1.42403351333087e-08
4319 1.42395810698304e-08
4320 1.42387417412237e-08
4321 1.42382932111218e-08
4322 1.42374769751541e-08
4323 1.42369005473597e-08
4324 1.42360194743674e-08
4325 1.42353897558678e-08
4326 1.42347413856214e-08
4327 1.42339402486868e-08
4328 1.42332519104116e-08
4329 1.42324623197965e-08
4330 1.42318983265e-08
4331 1.42310154771508e-08
4332 1.42303093753071e-08
4333 1.42295242255841e-08
4334 1.42288421045578e-08
4335 1.42281768589214e-08
4336 1.42274174663726e-08
4337 1.42267291280973e-08
4338 1.42260061508637e-08
4339 1.42252680745969e-08
4340 1.42247023049435e-08
4341 1.42238603118017e-08
4342 1.42231559863149e-08
4343 1.42223894883386e-08
4344 1.42218068432953e-08
4345 1.42208973485936e-08
4346 1.42203235853344e-08
4347 1.42195579755366e-08
4348 1.42187532858884e-08
4349 1.4217961030738e-08
4350 1.42174831907482e-08
4351 1.42165887950796e-08
4352 1.42158143034976e-08
4353 1.42152867255163e-08
4354 1.42145237802538e-08
4355 1.42137546177423e-08
4356 1.42132252634042e-08
4357 1.42124028101875e-08
4358 1.4211656740315e-08
4359 1.42109026768367e-08
4360 1.42103226963286e-08
4361 1.42094922495062e-08
4362 1.42088660837203e-08
4363 1.42081315601672e-08
4364 1.42073188769132e-08
4365 1.42066545194552e-08
4366 1.42059644048231e-08
4367 1.42052432039463e-08
4368 1.42045077922148e-08
4369 1.42036684636082e-08
4370 1.42029783489761e-08
4371 1.42024711990985e-08
4372 1.42013627524307e-08
4373 1.42007330339311e-08
4374 1.42001344016762e-08
4375 1.41993776736626e-08
4376 1.41987506196983e-08
4377 1.41979006329507e-08
4378 1.41971234768334e-08
4379 1.41966296496321e-08
4380 1.41957210431087e-08
4381 1.41950087240161e-08
4382 1.41945450948811e-08
4383 1.41937066544529e-08
4384 1.41929357155846e-08
4385 1.4192200303853e-08
4386 1.41915519336067e-08
4387 1.41908707007588e-08
4388 1.41901272954215e-08
4389 1.41893323757358e-08
4390 1.41886591364937e-08
4391 1.41880445170273e-08
4392 1.4187256702769e-08
4393 1.41864902047928e-08
4394 1.41858569335795e-08
4395 1.4184983854193e-08
4396 1.4184391439187e-08
4397 1.41836107303561e-08
4398 1.4183020091707e-08
4399 1.41822189547725e-08
4400 1.41815608145635e-08
4401 1.41809533005244e-08
4402 1.418034312195e-08
4403 1.41795561958702e-08
4404 1.4178745288973e-08
4405 1.41782443563443e-08
4406 1.41773384143562e-08
4407 1.41766944850019e-08
4408 1.41759075589221e-08
4409 1.41751970161863e-08
4410 1.41743772275049e-08
4411 1.41738745185194e-08
4412 1.41729481484276e-08
4413 1.41724711966162e-08
4414 1.41716736123954e-08
4415 1.41709435297344e-08
4416 1.41700722267046e-08
4417 1.41694229682798e-08
4418 1.41687888088882e-08
4419 1.41680098764141e-08
4420 1.41673295317446e-08
4421 1.41665514874489e-08
4422 1.41658800245636e-08
4423 1.41653355711924e-08
4424 1.41643852202833e-08
4425 1.41638878403683e-08
4426 1.41630804861848e-08
4427 1.41622935601049e-08
4428 1.41616096627217e-08
4429 1.41610749793131e-08
4430 1.41601121939061e-08
4431 1.41596601110905e-08
4432 1.41588225588407e-08
4433 1.41580462909019e-08
4434 1.41573881506929e-08
4435 1.41566829370277e-08
4436 1.41558755828441e-08
4437 1.41552032317804e-08
4438 1.4154527328003e-08
4439 1.41538292197652e-08
4440 1.41532030539793e-08
4441 1.41523246455222e-08
4442 1.41518352592129e-08
4443 1.41512002116428e-08
4444 1.4150423943704e-08
4445 1.41496867556157e-08
4446 1.41490366090125e-08
4447 1.41481191207049e-08
4448 1.41474476578196e-08
4449 1.41466811598434e-08
4450 1.41460887448375e-08
4451 1.41452929369734e-08
4452 1.41446827583991e-08
4453 1.41439171486013e-08
4454 1.41432581202139e-08
4455 1.41426230726438e-08
4456 1.41417038079794e-08
4457 1.41410323450941e-08
4458 1.41402836106863e-08
4459 1.41397773489871e-08
4460 1.41390765762139e-08
4461 1.41382550111757e-08
4462 1.41375293694068e-08
4463 1.41369529416124e-08
4464 1.41360612104791e-08
4465 1.413545369644e-08
4466 1.41347076265674e-08
4467 1.41341018888852e-08
4468 1.41333673653321e-08
4469 1.41326834679489e-08
4470 1.41318530211265e-08
4471 1.41312677115479e-08
4472 1.41305633860611e-08
4473 1.41299389966321e-08
4474 1.4129187597689e-08
4475 1.41285143584469e-08
4476 1.41277407550433e-08
4477 1.41270488640544e-08
4478 1.41262654906882e-08
4479 1.41255940278029e-08
4480 1.41249394403076e-08
4481 1.41241693896177e-08
4482 1.41234481887409e-08
4483 1.41228246874903e-08
4484 1.41220324323399e-08
4485 1.412124550626e-08
4486 1.41206673021088e-08
4487 1.41198732706016e-08
4488 1.41192524338862e-08
4489 1.41186093927104e-08
4490 1.41179574697503e-08
4491 1.41171723200273e-08
4492 1.41167388889585e-08
4493 1.41159066657792e-08
4494 1.41152023402924e-08
4495 1.41144225196399e-08
4496 1.41138398745966e-08
4497 1.41130680475499e-08
4498 1.41123166486068e-08
4499 1.41116434093647e-08
4500 1.41108422724301e-08
4501 1.41102276529637e-08
4502 1.41094567140954e-08
4503 1.4108901602583e-08
4504 1.41082869831166e-08
4505 1.41073597248464e-08
4506 1.41068046133341e-08
4507 1.41061899938677e-08
4508 1.41054643520988e-08
4509 1.41046081481022e-08
4510 1.41039508960716e-08
4511 1.41034135481277e-08
4512 1.4102508494318e-08
4513 1.41019098620632e-08
4514 1.41011406995517e-08
4515 1.41004710130233e-08
4516 1.40997391540054e-08
4517 1.40991378572153e-08
4518 1.40984317553716e-08
4519 1.40977283180632e-08
4520 1.40970053408296e-08
4521 1.40963427597285e-08
4522 1.40955034311219e-08
4523 1.40948888116554e-08
4524 1.40941240900361e-08
4525 1.409361605198e-08
4526 1.40928353431491e-08
4527 1.40920866087413e-08
4528 1.40914728774533e-08
4529 1.40909062196215e-08
4530 1.40898954725799e-08
4531 1.40894043099138e-08
4532 1.40886600163981e-08
4533 1.40878242405051e-08
4534 1.40872478127108e-08
4535 1.40865843434312e-08
4536 1.40860532127363e-08
4537 1.40850744401178e-08
4538 1.40845406448875e-08
4539 1.40838780637864e-08
4540 1.40830378470014e-08
4541 1.40825511252274e-08
4542 1.40818885441263e-08
4543 1.40810358928434e-08
4544 1.40803280146429e-08
4545 1.40797045133922e-08
4546 1.40791902580872e-08
4547 1.40782878688128e-08
4548 1.40778055879309e-08
4549 1.40769573775401e-08
4550 1.4076267262908e-08
4551 1.40756117872343e-08
4552 1.40747875576608e-08
4553 1.40742217880074e-08
4554 1.40734792708486e-08
4555 1.40728015907143e-08
4556 1.40720706198749e-08
4557 1.40714977447942e-08
4558 1.40707046014654e-08
4559 1.40701406081689e-08
4560 1.40694034200806e-08
4561 1.40687728134026e-08
4562 1.40679601301485e-08
4563 1.40674352167025e-08
4564 1.40666474024442e-08
4565 1.40658400482607e-08
4566 1.4065195230728e-08
4567 1.40645166624154e-08
4568 1.40639278001231e-08
4569 1.40631648548606e-08
4570 1.40624329958428e-08
4571 1.40618068300569e-08
4572 1.40611646770594e-08
4573 1.40604985432446e-08
4574 1.40597355979821e-08
4575 1.40590481478853e-08
4576 1.40583340524358e-08
4577 1.40577034457579e-08
4578 1.4057015995661e-08
4579 1.40562192996185e-08
4580 1.40554723415676e-08
4581 1.40549101246279e-08
4582 1.40543301441198e-08
4583 1.40535316717205e-08
4584 1.40526914549355e-08
4585 1.40521159153195e-08
4586 1.40514702096084e-08
4587 1.40506930534912e-08
4588 1.40500908685226e-08
4589 1.40495099998361e-08
4590 1.40487426136815e-08
4591 1.40479556876016e-08
4592 1.40472824483595e-08
4593 1.40465354903085e-08
4594 1.40459253117342e-08
4595 1.40451970054301e-08
4596 1.40445122198685e-08
4597 1.40439206930409e-08
4598 1.4043256335583e-08
4599 1.40423503935949e-08
4600 1.4041929397024e-08
4601 1.4041127371911e-08
4602 1.40404017301421e-08
4603 1.40397959924599e-08
4604 1.4039036599911e-08
4605 1.40384885938261e-08
4606 1.40378597635049e-08
4607 1.40371705370512e-08
4608 1.40363800582577e-08
4609 1.40356890554472e-08
4610 1.40349740718193e-08
4611 1.40342395482662e-08
4612 1.40334766030037e-08
4613 1.40328166864379e-08
4614 1.4032218054183e-08
4615 1.40315483676545e-08
4616 1.40309133200844e-08
4617 1.40301352757888e-08
4618 1.40294709183308e-08
4619 1.40287452765619e-08
4620 1.40281279925603e-08
4621 1.40275542293011e-08
4622 1.40268694437395e-08
4623 1.40260087988509e-08
4624 1.40254856617616e-08
4625 1.4024768901777e-08
4626 1.40240299373318e-08
4627 1.40233700207659e-08
4628 1.40227118805569e-08
4629 1.40220466349206e-08
4630 1.40213751720353e-08
4631 1.40206362075901e-08
4632 1.40200242526589e-08
4633 1.40192684128237e-08
4634 1.40185667518722e-08
4635 1.40177212060166e-08
4636 1.40173250784414e-08
4637 1.40165150597227e-08
4638 1.40159732708867e-08
4639 1.40152449645825e-08
4640 1.40144846838552e-08
4641 1.40139837512265e-08
4642 1.40131000136989e-08
4643 1.4012617732817e-08
4644 1.40119009728323e-08
4645 1.40113094460048e-08
4646 1.40103768586641e-08
4647 1.40097125012062e-08
4648 1.40091334088766e-08
4649 1.40083411537262e-08
4650 1.40076856780524e-08
4651 1.40069920107067e-08
4652 1.4006429793767e-08
4653 1.40056419795087e-08
4654 1.40050664398927e-08
4655 1.40044456031774e-08
4656 1.40037670348647e-08
4657 1.40032252460287e-08
4658 1.40024525308036e-08
4659 1.40017739624909e-08
4660 1.40011495730619e-08
4661 1.40005314008818e-08
4662 1.399963878157e-08
4663 1.39991245262649e-08
4664 1.39983917790687e-08
4665 1.39976519264451e-08
4666 1.39970124379829e-08
4667 1.39961970901936e-08
4668 1.39956508604655e-08
4669 1.39950779853848e-08
4670 1.39943132637654e-08
4671 1.3993516567723e-08
4672 1.39929676734596e-08
4673 1.39923876929515e-08
4674 1.3991734881813e-08
4675 1.39910536489651e-08
4676 1.39902489593169e-08
4677 1.3989540192938e-08
4678 1.39890854455871e-08
4679 1.3988323388503e-08
4680 1.39875675486678e-08
4681 1.39869742454835e-08
4682 1.39864342330043e-08
4683 1.39858427061768e-08
4684 1.39850895308768e-08
4685 1.39843105984028e-08
4686 1.39838078894172e-08
4687 1.39830769185778e-08
4688 1.3982239366328e-08
4689 1.39817135647036e-08
4690 1.39809719357231e-08
4691 1.39803990606424e-08
4692 1.39796583198404e-08
4693 1.39789397834988e-08
4694 1.39782780905762e-08
4695 1.39776465957198e-08
4696 1.39767806217606e-08
4697 1.39762104112151e-08
4698 1.39756837214122e-08
4699 1.39749998240291e-08
4700 1.39741995752729e-08
4701 1.39734712689688e-08
4702 1.39729383619169e-08
4703 1.39722038383638e-08
4704 1.39715226055159e-08
4705 1.39708289381701e-08
4706 1.39701521462143e-08
4707 1.39694060763418e-08
4708 1.39685454314531e-08
4709 1.39680658151065e-08
4710 1.39674654064947e-08
4711 1.39667681864353e-08
4712 1.39661997522467e-08
4713 1.39654252606647e-08
4714 1.39647724495262e-08
4715 1.39640734531099e-08
4716 1.39633540285899e-08
4717 1.39627829298661e-08
4718 1.39620750516656e-08
4719 1.39614044769587e-08
4720 1.39607845284218e-08
4721 1.39601725734906e-08
4722 1.39595544013105e-08
4723 1.39589149128483e-08
4724 1.39582043701125e-08
4725 1.39575098145883e-08
4726 1.39568667734125e-08
4727 1.39562539303029e-08
4728 1.39554146016962e-08
4729 1.39549367617064e-08
4730 1.39540095034363e-08
4731 1.39533913312562e-08
4732 1.39527580600429e-08
4733 1.39521079134397e-08
4734 1.39515545782842e-08
4735 1.39509017671458e-08
4736 1.39501317164559e-08
4737 1.39493883111186e-08
4738 1.39487692507601e-08
4739 1.39479814365018e-08
4740 1.39473419480396e-08
4741 1.3946839239054e-08
4742 1.39461375781025e-08
4743 1.39454412462214e-08
4744 1.39447697833361e-08
4745 1.39440388124967e-08
4746 1.39434730428434e-08
4747 1.3942742072004e-08
4748 1.39419835676335e-08
4749 1.39414080280176e-08
4750 1.39408848909284e-08
4751 1.39401459264832e-08
4752 1.39393891984696e-08
4753 1.39387523745427e-08
4754 1.39380178509896e-08
4755 1.39373872443116e-08
4756 1.39367628548825e-08
4757 1.39360825102131e-08
4758 1.3935502529705e-08
4759 1.39347973160397e-08
4760 1.39340938787313e-08
4761 1.39332447801621e-08
4762 1.3932687892293e-08
4763 1.39320155412292e-08
4764 1.39314533242896e-08
4765 1.39307028135249e-08
4766 1.39300411206023e-08
4767 1.39293776513227e-08
4768 1.3928650233197e-08
4769 1.39280764699379e-08
4770 1.392736148631e-08
4771 1.39267797294451e-08
4772 1.3925935071768e-08
4773 1.39252565034553e-08
4774 1.39246445485242e-08
4775 1.39239748619957e-08
4776 1.39233318208198e-08
4777 1.39225253548148e-08
4778 1.39219791250866e-08
4779 1.39211984162557e-08
4780 1.39206042248929e-08
4781 1.39199354265429e-08
4782 1.39190774461895e-08
4783 1.39184637149015e-08
4784 1.3917821561904e-08
4785 1.39171367763424e-08
4786 1.39164404444614e-08
4787 1.39159261891564e-08
4788 1.39152156464206e-08
4789 1.39145486244274e-08
4790 1.39138753851853e-08
4791 1.39130289511513e-08
4792 1.39125573284105e-08
4793 1.39118725428489e-08
4794 1.39111442365447e-08
4795 1.39105997831734e-08
4796 1.39098164098073e-08
4797 1.39091369533162e-08
4798 1.39084965766756e-08
4799 1.3907873075425e-08
4800 1.3907236251498e-08
4801 1.39064715298787e-08
4802 1.39057636516782e-08
4803 1.39052769299042e-08
4804 1.3904466023007e-08
4805 1.39037137358855e-08
4806 1.39032287904683e-08
4807 1.39023912382186e-08
4808 1.39018307976357e-08
4809 1.39012614752687e-08
4810 1.39005837951345e-08
4811 1.38998350607267e-08
4812 1.38990801090699e-08
4813 1.38985178921303e-08
4814 1.38977664931872e-08
4815 1.38971927299281e-08
4816 1.3896502615296e-08
4817 1.38958187179128e-08
4818 1.38950824180029e-08
4819 1.38943727634455e-08
4820 1.38937226168423e-08
4821 1.38931328663716e-08
4822 1.38925848602867e-08
4823 1.38919054037956e-08
4824 1.3891281902545e-08
4825 1.38905287272451e-08
4826 1.38899984847285e-08
4827 1.38891813605824e-08
4828 1.38884734823819e-08
4829 1.38878979427659e-08
4830 1.38871989463496e-08
4831 1.38865257071075e-08
4832 1.38859412857073e-08
4833 1.38852103148679e-08
4834 1.38845317465552e-08
4835 1.38839384433709e-08
4836 1.38832669804856e-08
4837 1.38824365336632e-08
4838 1.38818307959809e-08
4839 1.38810909433573e-08
4840 1.38805305027745e-08
4841 1.38799673976564e-08
4842 1.38791245163361e-08
4843 1.38785285486165e-08
4844 1.38777913605281e-08
4845 1.38771474311739e-08
4846 1.38766598212214e-08
4847 1.38759705947678e-08
4848 1.38752787037788e-08
4849 1.38746774069887e-08
4850 1.38739295607593e-08
4851 1.38733007304381e-08
4852 1.38725653187066e-08
4853 1.38719151721034e-08
4854 1.38713884823005e-08
4855 1.3870492310275e-08
4856 1.38698625917755e-08
4857 1.38692772821969e-08
4858 1.38685312123243e-08
4859 1.38678295513728e-08
4860 1.38672167082632e-08
4861 1.3866450210287e-08
4862 1.38659856929735e-08
4863 1.38652147541052e-08
4864 1.38646569780576e-08
4865 1.38639579816413e-08
4866 1.38632509916192e-08
4867 1.38624169920831e-08
4868 1.38618920786371e-08
4869 1.38612916700254e-08
4870 1.38605784627543e-08
4871 1.38599594023958e-08
4872 1.38592204379506e-08
4873 1.38585365405675e-08
4874 1.38579476782752e-08
4875 1.38573854613355e-08
4876 1.38566438323551e-08
4877 1.3856036318316e-08
4878 1.38552866957298e-08
4879 1.38546791816907e-08
4880 1.38539562044571e-08
4881 1.38533184923517e-08
4882 1.38527775916941e-08
4883 1.38519640202617e-08
4884 1.38513698288989e-08
4885 1.38506424107732e-08
4886 1.38498981172575e-08
4887 1.38494513635123e-08
4888 1.3848769242486e-08
4889 1.38480880096381e-08
4890 1.38474014477197e-08
4891 1.3846811697249e-08
4892 1.38463001064792e-08
4893 1.3845457225159e-08
4894 1.3844875468294e-08
4895 1.38442040054088e-08
4896 1.38434685936772e-08
4897 1.38428513096756e-08
4898 1.38422722173459e-08
4899 1.38415732209296e-08
4900 1.38408244865218e-08
4901 1.38404052663077e-08
4902 1.38397426852066e-08
4903 1.38389211201684e-08
4904 1.38383962067223e-08
4905 1.38376403668872e-08
4906 1.38370763735907e-08
4907 1.3836497281261e-08
4908 1.38358586809773e-08
4909 1.38351028411421e-08
4910 1.38345797040529e-08
4911 1.38338531741056e-08
4912 1.38332110211081e-08
4913 1.38324987020155e-08
4914 1.38319373732543e-08
4915 1.38311975206307e-08
4916 1.38306708308278e-08
4917 1.38300704222161e-08
4918 1.38292870488499e-08
4919 1.38287870043996e-08
4920 1.38278677397352e-08
4921 1.38273534844302e-08
4922 1.38267370886069e-08
4923 1.38260416449043e-08
4924 1.38253231085628e-08
4925 1.38245379588398e-08
4926 1.38240574543147e-08
4927 1.38235325408687e-08
4928 1.38227580492867e-08
4929 1.38221771806002e-08
4930 1.38214728551134e-08
4931 1.38208235966886e-08
4932 1.38202089772221e-08
4933 1.3819497546308e-08
4934 1.38189193421567e-08
4935 1.38182159048483e-08
4936 1.38175400010709e-08
4937 1.38169848895586e-08
4938 1.38162361551508e-08
4939 1.38156526219291e-08
4940 1.38150815232052e-08
4941 1.3814585919647e-08
4942 1.38137519201109e-08
4943 1.38131017735077e-08
4944 1.38123432691373e-08
4945 1.38118245729402e-08
4946 1.38111619918391e-08
4947 1.38103439795145e-08
4948 1.38098830149147e-08
4949 1.38092755008756e-08
4950 1.38083766643149e-08
4951 1.38079201406072e-08
4952 1.38071429844899e-08
4953 1.3806688237139e-08
4954 1.38059190746276e-08
4955 1.3805333765049e-08
4956 1.38045734843217e-08
4957 1.38040139319173e-08
4958 1.38033264818205e-08
4959 1.38027003160346e-08
4960 1.38020519457882e-08
4961 1.38015314732343e-08
4962 1.38008990901994e-08
4963 1.38002178573515e-08
4964 1.37994948801179e-08
4965 1.37989610848877e-08
4966 1.37983011683218e-08
4967 1.37974502933957e-08
4968 1.37970301850032e-08
4969 1.37963898083626e-08
4970 1.37957689716472e-08
4971 1.37950149081689e-08
4972 1.37944216049846e-08
4973 1.37939393241027e-08
4974 1.37931834842675e-08
4975 1.37924711651749e-08
4976 1.37918583220653e-08
4977 1.37913218622998e-08
4978 1.37905686869999e-08
4979 1.37897853136337e-08
4980 1.3789339448067e-08
4981 1.37886519979702e-08
4982 1.37879423434129e-08
4983 1.37874485162115e-08
4984 1.37868632066329e-08
4985 1.37861491111835e-08
4986 1.37854954118666e-08
4987 1.37847786518819e-08
4988 1.3784296371e-08
4989 1.37836453362183e-08
4990 1.37830644675319e-08
4991 1.37823930046466e-08
4992 1.37818609857732e-08
4993 1.37811504430374e-08
4994 1.37805038491479e-08
4995 1.37799007760009e-08
4996 1.37793350063475e-08
4997 1.37786946297069e-08
4998 1.37779050390918e-08
4999 1.37772975250527e-08
};
\addlegendentry{Test}

\nextgroupplot[
title={Leaky/Leaky $\rare$},
ymin=3.54436308965785e-09, ymax=1e-05,
]
\addplot [semithick, black, dashed]
table {%
0 0.00639266828284599
1 0.000550484143097492
2 0.000139280548670286
3 9.76271857243773e-05
4 5.91572264731894e-05
5 3.66297950477019e-05
6 2.79266912817775e-05
7 2.4105634819648e-05
8 2.13873921317713e-05
9 1.86441825729844e-05
10 1.57835531190358e-05
11 1.28448787328352e-05
12 1.01101008984301e-05
13 7.86834085761256e-06
14 6.29145947902998e-06
15 5.32018328263462e-06
16 4.72851236328609e-06
17 4.30687789203432e-06
18 3.93565061516199e-06
19 3.60110685292625e-06
20 3.28769201410495e-06
21 2.99319917957774e-06
22 2.71676886844574e-06
23 2.47341393298939e-06
24 2.28556317741635e-06
25 2.15171956879345e-06
26 2.05269436766997e-06
27 1.97562918086902e-06
28 1.91248048273707e-06
29 1.85703912244151e-06
30 1.80616896459895e-06
31 1.75927270458942e-06
32 1.71553983225436e-06
33 1.67466619566881e-06
34 1.63554831715373e-06
35 1.59726490868195e-06
36 1.56070223003013e-06
37 1.52556825411665e-06
38 1.49200708087704e-06
39 1.45840282913667e-06
40 1.42623564863698e-06
41 1.3951152364271e-06
42 1.36575776791403e-06
43 1.33648801715225e-06
44 1.30901010740558e-06
45 1.28174571659656e-06
46 1.25574044503907e-06
47 1.22990649933996e-06
48 1.20553109295685e-06
49 1.1818934190444e-06
50 1.15904055029858e-06
51 1.13676675316299e-06
52 1.11559017400253e-06
53 1.09552593407614e-06
54 1.0759202085886e-06
55 1.05715179683585e-06
56 1.03901692089536e-06
57 1.02172207449769e-06
58 1.00500771369738e-06
59 9.89071536878328e-07
60 9.73289230401519e-07
61 9.58848112205146e-07
62 9.44659631947786e-07
63 9.30893642294706e-07
64 9.18217927114995e-07
65 9.05228803262048e-07
66 8.93603115514452e-07
67 8.81428693936925e-07
68 8.7053143176874e-07
69 8.58725870440224e-07
70 8.48147544200373e-07
71 8.38026904425249e-07
72 8.28095078173163e-07
73 8.17387960996285e-07
74 8.07515973910711e-07
75 7.98049989288785e-07
76 7.88176294520326e-07
77 7.78215239698454e-07
78 7.69547374225965e-07
79 7.6034185921614e-07
80 7.52087187105843e-07
81 7.43188686303853e-07
82 7.34882657997815e-07
83 7.26374653112316e-07
84 7.18565597129128e-07
85 7.10393960229538e-07
86 7.02599472784726e-07
87 6.9536597499642e-07
88 6.87848276490044e-07
89 6.80959642259893e-07
90 6.74567784752611e-07
91 6.66539876512218e-07
92 6.59243315094926e-07
93 6.52933807748113e-07
94 6.45520796547672e-07
95 6.39192373322217e-07
96 6.32494880111523e-07
97 6.26014665492391e-07
98 6.20275582019048e-07
99 6.14403231113059e-07
100 6.08651037525831e-07
101 6.02731909388865e-07
102 5.97103928322795e-07
103 5.91832842529172e-07
104 5.86249168275543e-07
105 5.81187631345514e-07
106 5.75531961210629e-07
107 5.70267463674057e-07
108 5.65167866342975e-07
109 5.59608503246878e-07
110 5.53636723161333e-07
111 5.48459019594105e-07
112 5.4399276892525e-07
113 5.39935615154974e-07
114 5.35849832058233e-07
115 5.31509946972264e-07
116 5.27689189327773e-07
117 5.23522804478205e-07
118 5.19741319036981e-07
119 5.15632754794382e-07
120 5.11579942109819e-07
121 5.097698261185e-07
122 5.05607248886264e-07
123 5.00374048890251e-07
124 4.969684831444e-07
125 4.93887920852032e-07
126 4.90803693431019e-07
127 4.87908372427626e-07
128 4.85017656869147e-07
129 4.82408346739405e-07
130 4.79665301329035e-07
131 4.76800462315907e-07
132 4.73793770964903e-07
133 4.71099301424971e-07
134 4.68210622464937e-07
135 4.6545888415217e-07
136 4.62900089797103e-07
137 4.60184821424292e-07
138 4.57714165392531e-07
139 4.55314537511242e-07
140 4.529190976168e-07
141 4.50723747819737e-07
142 4.48456908799244e-07
143 4.46228324955555e-07
144 4.43817241965405e-07
145 4.41667952225444e-07
146 4.39615825655437e-07
147 4.37487569488582e-07
148 4.35459342071454e-07
149 4.33417439005268e-07
150 4.31447709173582e-07
151 4.2943793712702e-07
152 4.27529491338063e-07
153 4.2566227679508e-07
154 4.23720459480137e-07
155 4.21809083054114e-07
156 4.20062744668215e-07
157 4.18244490321129e-07
158 4.16348084902651e-07
159 4.14541437665505e-07
160 4.1273910322559e-07
161 4.1101460574744e-07
162 4.09297577496304e-07
163 4.07387719393526e-07
164 4.0541933482352e-07
165 4.03300040149546e-07
166 4.0148119772887e-07
167 3.99702004649782e-07
168 3.98207529288541e-07
169 3.96167296864647e-07
170 3.94664589224547e-07
171 3.93199072338035e-07
172 3.91306652840484e-07
173 3.8990161741026e-07
174 3.88348650725234e-07
175 3.86945826893026e-07
176 3.85168759237331e-07
177 3.83812769202407e-07
178 3.82544949319552e-07
179 3.80890954795987e-07
180 3.79380347723668e-07
181 3.78229289029974e-07
182 3.76503710540987e-07
183 3.75110867571138e-07
184 3.74077657580685e-07
185 3.72315717967808e-07
186 3.71086092208373e-07
187 3.69928999839786e-07
188 3.68480127754367e-07
189 3.67271146096471e-07
190 3.6613841716715e-07
191 3.64712724147509e-07
192 3.63274615185816e-07
193 3.61948218619901e-07
194 3.61037333634684e-07
195 3.59528785503827e-07
196 3.58365367260305e-07
197 3.5757094689437e-07
198 3.5604606193651e-07
199 3.55029425117337e-07
200 3.54219302554526e-07
201 3.52731874434298e-07
202 3.51712189672782e-07
203 3.50805244201169e-07
204 3.48323808504247e-07
205 3.47417719396503e-07
206 3.46008032966694e-07
207 3.44858239255785e-07
208 3.44161284013822e-07
209 3.42750959697824e-07
210 3.41800891158428e-07
211 3.4077912106234e-07
212 3.39803496139268e-07
213 3.38826958772742e-07
214 3.38246348354332e-07
215 3.36939917174206e-07
216 3.35871940272625e-07
217 3.35113444561941e-07
218 3.34144105218925e-07
219 3.32918343397104e-07
220 3.32032139810146e-07
221 3.31068037761284e-07
222 3.30231905428136e-07
223 3.2928413386113e-07
224 3.28367355529657e-07
225 3.27482429391068e-07
226 3.26571306003665e-07
227 3.2611526450399e-07
228 3.24740405420521e-07
229 3.23915689760312e-07
230 3.23072035519623e-07
231 3.22634463531202e-07
232 3.21270535970974e-07
233 3.20493456518633e-07
234 3.19565409134803e-07
235 3.19234440561367e-07
236 3.17880037924212e-07
237 3.17101765260119e-07
238 3.16386181903106e-07
239 3.16021215751583e-07
240 3.14719329221447e-07
241 3.13926688765953e-07
242 3.13138673966762e-07
243 3.12557677386138e-07
244 3.11452918468014e-07
245 3.10778897103958e-07
246 3.09969993536718e-07
247 3.09379312867897e-07
248 3.08262465249065e-07
249 3.07577546602822e-07
250 3.06823512345566e-07
251 3.0623569871846e-07
252 3.05144289681536e-07
253 3.04493773644943e-07
254 3.0371864092249e-07
255 3.03211076687404e-07
256 3.02197543895488e-07
257 3.01516143703218e-07
258 3.00772982439135e-07
259 3.00131844998219e-07
260 2.99358128185823e-07
261 2.98539655283037e-07
262 2.97847587029665e-07
263 2.97176883103489e-07
264 2.96585268377392e-07
265 2.95501964722611e-07
266 2.94777032006088e-07
267 2.94011010588235e-07
268 2.93278332047286e-07
269 2.92549905302586e-07
270 2.91827586548443e-07
271 2.91127172943106e-07
272 2.90484430070848e-07
273 2.89672701506483e-07
274 2.89009392421491e-07
275 2.88311981306855e-07
276 2.87820607334233e-07
277 2.86807200063244e-07
278 2.8620831178916e-07
279 2.855048228394e-07
280 2.84365228380601e-07
281 2.83356528949952e-07
282 2.8261014372255e-07
283 2.81880397371204e-07
284 2.80927941159526e-07
285 2.80157972210482e-07
286 2.79369886335701e-07
287 2.78682715675771e-07
288 2.78022430795666e-07
289 2.77286668623589e-07
290 2.76599827557789e-07
291 2.75949442873902e-07
292 2.75327586622964e-07
293 2.74583973955522e-07
294 2.73731074611483e-07
295 2.7295761774937e-07
296 2.72432128429756e-07
297 2.71724884656877e-07
298 2.71092695834696e-07
299 2.70465790774388e-07
300 2.69550825279907e-07
301 2.69153848213577e-07
302 2.68444039092763e-07
303 2.6754663281281e-07
304 2.67123240736744e-07
305 2.66451525225087e-07
306 2.65712347485625e-07
307 2.64988396504684e-07
308 2.64265525875551e-07
309 2.63441425067512e-07
310 2.63036985241882e-07
311 2.62337176765826e-07
312 2.61437067557324e-07
313 2.61046195871195e-07
314 2.60173541341224e-07
315 2.59787083888519e-07
316 2.58918145581433e-07
317 2.58431877274035e-07
318 2.57577976881862e-07
319 2.56980176976285e-07
320 2.56370456942712e-07
321 2.55992434674113e-07
322 2.55067079162785e-07
323 2.54693972665088e-07
324 2.54037899380499e-07
325 2.53142398745077e-07
326 2.52556213134802e-07
327 2.51863898864002e-07
328 2.5145540669147e-07
329 2.51016082048139e-07
330 2.50073499947767e-07
331 2.49693418061803e-07
332 2.48820782408288e-07
333 2.48203934599012e-07
334 2.47805984544769e-07
335 2.47106871707814e-07
336 2.46556782892426e-07
337 2.45955621744898e-07
338 2.45183384148362e-07
339 2.43222710722968e-07
340 2.42149397440272e-07
341 2.41287993839379e-07
342 2.40675140934954e-07
343 2.39919426649138e-07
344 2.39037776588624e-07
345 2.38673182016313e-07
346 2.37972993744506e-07
347 2.37054098859879e-07
348 2.36393006106006e-07
349 2.35746497112999e-07
350 2.34760016948776e-07
351 2.34167193378099e-07
352 2.32899785911123e-07
353 2.30383016268654e-07
354 2.28629650210976e-07
355 2.26523812528079e-07
356 2.24479767856778e-07
357 2.23191292945124e-07
358 2.22055417112621e-07
359 2.21283264372651e-07
360 2.20351121203777e-07
361 2.19534076190797e-07
362 2.18811891699922e-07
363 2.18017246713664e-07
364 2.17352012238159e-07
365 2.16728562461022e-07
366 2.15882518147303e-07
367 2.15482010858636e-07
368 2.14849677078632e-07
369 2.14163424147706e-07
370 2.13550072605706e-07
371 2.12959220006859e-07
372 2.12265067345285e-07
373 2.1183669877356e-07
374 2.11095539876638e-07
375 2.10636580364465e-07
376 2.10178486359958e-07
377 2.09393849498696e-07
378 2.0884023376766e-07
379 2.08000517252671e-07
380 2.07460437612994e-07
381 2.06915931550533e-07
382 2.06699339036831e-07
383 2.0610777192509e-07
384 2.05478105953283e-07
385 2.04607838032445e-07
386 2.03685751154481e-07
387 2.0307691839605e-07
388 2.02171794194683e-07
389 2.01462209303394e-07
390 2.01320428987906e-07
391 2.00744771125194e-07
392 1.99739587275261e-07
393 1.99231443695602e-07
394 1.98119296363153e-07
395 1.97354221706192e-07
396 1.96401175161398e-07
397 1.9573746484447e-07
398 1.94799124966671e-07
399 1.94299017182153e-07
400 1.93561698054623e-07
401 1.92794878654645e-07
402 1.92298780907763e-07
403 1.91859266251093e-07
404 1.90754160298567e-07
405 1.90098232115687e-07
406 1.89346206245666e-07
407 1.88667174178647e-07
408 1.88066287283739e-07
409 1.87050825680046e-07
410 1.86428051457632e-07
411 1.85629252388964e-07
412 1.84830423600957e-07
413 1.84100773086726e-07
414 1.8355542247761e-07
415 1.82535937539363e-07
416 1.81736507715691e-07
417 1.810951779877e-07
418 1.803856835636e-07
419 1.79530575731057e-07
420 1.78821941467788e-07
421 1.78296238448894e-07
422 1.77264238649499e-07
423 1.76659019901493e-07
424 1.76114725994658e-07
425 1.75301740870104e-07
426 1.74645406433527e-07
427 1.740794500531e-07
428 1.73318679891565e-07
429 1.72804967895068e-07
430 1.72148198961608e-07
431 1.71477135649134e-07
432 1.70962196311741e-07
433 1.70726072870941e-07
434 1.69601806948805e-07
435 1.69135923379748e-07
436 1.68780365859433e-07
437 1.68058240846314e-07
438 1.67818931887886e-07
439 1.66614794783193e-07
440 1.66135844594351e-07
441 1.65436588572376e-07
442 1.65015317077355e-07
443 1.64467081849473e-07
444 1.63893556107908e-07
445 1.63440330489095e-07
446 1.62815759337853e-07
447 1.62238967340578e-07
448 1.6173522721985e-07
449 1.61210235454234e-07
450 1.60674047476661e-07
451 1.601908962785e-07
452 1.59717454713437e-07
453 1.59231425181261e-07
454 1.58674240352319e-07
455 1.58257703321674e-07
456 1.57676286934638e-07
457 1.57195237459717e-07
458 1.56759729592615e-07
459 1.56226428317474e-07
460 1.55684231712527e-07
461 1.55229752881958e-07
462 1.54720608962933e-07
463 1.54282645405956e-07
464 1.53757618481798e-07
465 1.53284352541583e-07
466 1.52776395819565e-07
467 1.52345512419494e-07
468 1.51809290324323e-07
469 1.51341671693395e-07
470 1.50879971632101e-07
471 1.50409433211429e-07
472 1.49922279386594e-07
473 1.49521513068684e-07
474 1.49094008819795e-07
475 1.48557685181849e-07
476 1.48114264042043e-07
477 1.47721014594815e-07
478 1.47307928648921e-07
479 1.46881154341472e-07
480 1.46491371255664e-07
481 1.46058332376242e-07
482 1.45599762677584e-07
483 1.45371029227626e-07
484 1.44677807988103e-07
485 1.44273721176091e-07
486 1.43924962589637e-07
487 1.4347951817939e-07
488 1.43167756522011e-07
489 1.43011684090588e-07
490 1.42230218976636e-07
491 1.41889404762452e-07
492 1.41453020449767e-07
493 1.41031133686909e-07
494 1.40613045220839e-07
495 1.4038257829796e-07
496 1.39651008113972e-07
497 1.3939356389514e-07
498 1.38959430779906e-07
499 1.38570890625189e-07
500 1.38187967894332e-07
501 1.3768266560632e-07
502 1.37484050301673e-07
503 1.36799258600995e-07
504 1.36478701064924e-07
505 1.36077327974604e-07
506 1.35698036341836e-07
507 1.35314573285861e-07
508 1.34912982874269e-07
509 1.34518258847471e-07
510 1.34295456230227e-07
511 1.33714364198845e-07
512 1.3337756936771e-07
513 1.33017492288445e-07
514 1.32613554343131e-07
515 1.32181782611873e-07
516 1.31898060049096e-07
517 1.3171227176656e-07
518 1.31070618216711e-07
519 1.30787701954738e-07
520 1.30362841798792e-07
521 1.30029100915863e-07
522 1.2963731370963e-07
523 1.29284251411121e-07
524 1.28961053561305e-07
525 1.28513321792312e-07
526 1.28445204200744e-07
527 1.2777527247021e-07
528 1.27499088447003e-07
529 1.27132933458007e-07
530 1.26801699692258e-07
531 1.26408716983661e-07
532 1.25994420977271e-07
533 1.25662349952371e-07
534 1.25490547055129e-07
535 1.24911517137782e-07
536 1.24693449013336e-07
537 1.24309628494856e-07
538 1.23957449690515e-07
539 1.23616339928745e-07
540 1.2330904651936e-07
541 1.22943449848201e-07
542 1.22714161134851e-07
543 1.22200308910969e-07
544 1.21925734332784e-07
545 1.21637930655005e-07
546 1.21313346086005e-07
547 1.20918536381076e-07
548 1.20580077287435e-07
549 1.20310829062831e-07
550 1.20151976845229e-07
551 1.19568852440466e-07
552 1.19436466310852e-07
553 1.19100902233882e-07
554 1.18768483517329e-07
555 1.18567019522509e-07
556 1.17995334906418e-07
557 1.17761895428004e-07
558 1.17493832692439e-07
559 1.17220456838929e-07
560 1.16832492126129e-07
561 1.16598884667418e-07
562 1.16351418857796e-07
563 1.15820931883448e-07
564 1.15660687507457e-07
565 1.15346312004938e-07
566 1.15025297367843e-07
567 1.1470422816906e-07
568 1.1437820147453e-07
569 1.14149198364721e-07
570 1.13728518519096e-07
571 1.13522993068216e-07
572 1.1317570943703e-07
573 1.12885249082595e-07
574 1.12700490676687e-07
575 1.12163502489437e-07
576 1.11953952097288e-07
577 1.11866680978068e-07
578 1.11346138256252e-07
579 1.11115921255234e-07
580 1.10787572418225e-07
581 1.10650568527326e-07
582 1.10059551938768e-07
583 1.10003370356715e-07
584 1.09566285744656e-07
585 1.09479653084499e-07
586 1.08992644560324e-07
587 1.08639524459164e-07
588 1.08340583902411e-07
589 1.08272754621552e-07
590 1.07871493225886e-07
591 1.07531928742688e-07
592 1.07263991115047e-07
593 1.06961966290342e-07
594 1.0668918109058e-07
595 1.06375896678834e-07
596 1.06258118646174e-07
597 1.05710362820588e-07
598 1.05334225374065e-07
599 1.05349724922377e-07
600 1.04763795135909e-07
601 1.04710226897176e-07
602 1.04213330296865e-07
603 1.03921646581462e-07
604 1.03666805301827e-07
605 1.03177568011148e-07
606 1.02958535820541e-07
607 1.02431287974269e-07
608 1.02281746556798e-07
609 1.01868778247471e-07
610 1.01588473427228e-07
611 1.01220186264683e-07
612 1.00913151666937e-07
613 1.00397030124988e-07
614 1.00086885494122e-07
615 9.98014326558661e-08
616 9.96305691609578e-08
617 9.91837127131134e-08
618 9.89728153921732e-08
619 9.86428288687336e-08
620 9.85598802616394e-08
621 9.82172557373317e-08
622 9.78156585826895e-08
623 9.76401403720573e-08
624 9.74952638443938e-08
625 9.7144359214596e-08
626 9.69942510078425e-08
627 9.66408206499736e-08
628 9.6463491161991e-08
629 9.62121907397417e-08
630 9.59560282023375e-08
631 9.60013763267042e-08
632 9.5440243198297e-08
633 9.53876790887875e-08
634 9.50906109808969e-08
635 9.49816199433684e-08
636 9.46374125607719e-08
637 9.4591817316747e-08
638 9.42502145768742e-08
639 9.40521457648558e-08
640 9.3909074960763e-08
641 9.3715416610074e-08
642 9.34184895786672e-08
643 9.32966520954537e-08
644 9.29993233524939e-08
645 9.29410447882972e-08
646 9.25719028792571e-08
647 9.24329296294601e-08
648 9.23046373522141e-08
649 9.20439581051369e-08
650 9.18345667608733e-08
651 9.16551617038408e-08
652 9.13158274755688e-08
653 9.12690586503118e-08
654 9.10325471092932e-08
655 9.0793372364395e-08
656 9.05979184517847e-08
657 9.05110519004992e-08
658 9.01878656485167e-08
659 9.01082561020061e-08
660 8.99495098591885e-08
661 8.9680228452238e-08
662 8.96020867644509e-08
663 8.93822388405852e-08
664 8.93870355609394e-08
665 8.89568611661673e-08
666 8.89821428575743e-08
667 8.86676558344846e-08
668 8.86080061937022e-08
669 8.84741729096916e-08
670 8.82004163211825e-08
671 8.80440109742864e-08
672 8.80007172532338e-08
673 8.76124529507472e-08
674 8.75644833251421e-08
675 8.74722494650371e-08
676 8.71908629904716e-08
677 8.69214508951721e-08
678 8.70200007043742e-08
679 8.65235395219877e-08
680 8.6523001308958e-08
681 8.6250528132048e-08
682 8.6168215916782e-08
683 8.59233299572715e-08
684 8.59103886239332e-08
685 8.55933081909122e-08
686 8.55227251945756e-08
687 8.53084918697178e-08
688 8.52974141229978e-08
689 8.50078733667381e-08
690 8.48041216623407e-08
691 8.4769902245263e-08
692 8.45768203752151e-08
693 8.4564075769844e-08
694 8.42649382861715e-08
695 8.41369069322084e-08
696 8.39136060979406e-08
697 8.38521572967821e-08
698 8.37725264450917e-08
699 8.34479953493172e-08
700 8.35067714444193e-08
701 8.34086140035772e-08
702 8.32103908550153e-08
703 8.3156828517339e-08
704 8.28314822234333e-08
705 8.28322059485131e-08
706 8.27181241471742e-08
707 8.25260276928752e-08
708 8.22034059835275e-08
709 8.20999796613542e-08
710 8.19482947060912e-08
711 8.1884529027576e-08
712 8.15492769152826e-08
713 8.15451220099739e-08
714 8.13044541372321e-08
715 8.12385842139385e-08
716 8.10199701750669e-08
717 8.08309639124261e-08
718 8.09825375074347e-08
719 8.05513112318579e-08
720 8.05199841629189e-08
721 8.0394369264436e-08
722 8.03571097400813e-08
723 8.00366721120938e-08
724 8.00833524530198e-08
725 7.98697074748667e-08
726 7.98909921089219e-08
727 7.95227193246539e-08
728 7.97349302703587e-08
729 7.92247824428216e-08
730 7.93397929763628e-08
731 7.91927269347958e-08
732 7.89648818630972e-08
733 7.8853875282725e-08
734 7.87918342655125e-08
735 7.86087563886539e-08
736 7.84242504026089e-08
737 7.82154593816919e-08
738 7.82615463439385e-08
739 7.7925336434248e-08
740 7.79033298754506e-08
741 7.76774277957415e-08
742 7.77174939425862e-08
743 7.74018624047912e-08
744 7.73378969762817e-08
745 7.72905014443381e-08
746 7.70500589037937e-08
747 7.70846444471118e-08
748 7.68797037311586e-08
749 7.67245534412098e-08
750 7.66305869612793e-08
751 7.64830234407654e-08
752 7.63330782911353e-08
753 7.62728006211688e-08
754 7.63083099162287e-08
755 7.59728037476037e-08
756 7.60412450659942e-08
757 7.57888530049655e-08
758 7.56857253092491e-08
759 7.55867927013654e-08
760 7.54951987120833e-08
761 7.54636184314705e-08
762 7.52176993366049e-08
763 7.51728367269777e-08
764 7.49781758004886e-08
765 7.49103074353386e-08
766 7.4865017335668e-08
767 7.46729902942889e-08
768 7.46439937451271e-08
769 7.44874382565541e-08
770 7.44498930185244e-08
771 7.42747585595893e-08
772 7.41090894855745e-08
773 7.41100729397814e-08
774 7.40100159406509e-08
775 7.40762776105797e-08
776 7.36763699258347e-08
777 7.36514981758418e-08
778 7.34663481578579e-08
779 7.36695042737878e-08
780 7.32449630707066e-08
781 7.3309791902787e-08
782 7.3056063159882e-08
783 7.31697232287942e-08
784 7.28157761464843e-08
785 7.29046461680483e-08
786 7.26845562208922e-08
787 7.27021421749185e-08
788 7.25018673009004e-08
789 7.23947760228327e-08
790 7.2230592806477e-08
791 7.23684520487211e-08
792 7.20388794155369e-08
793 7.21131778131934e-08
794 7.19944008231721e-08
795 7.19291317308191e-08
796 7.20022299613987e-08
797 7.21237162677468e-08
798 7.15025293720473e-08
799 7.17985893601281e-08
800 7.1166614231899e-08
801 7.12894184422552e-08
802 7.13454060958796e-08
803 7.10530151044608e-08
804 7.11774663084697e-08
805 7.10712177367512e-08
806 7.10230988429927e-08
807 7.05901208881699e-08
808 7.06407780981522e-08
809 7.08540766352073e-08
810 7.05574098001094e-08
811 7.02210417173887e-08
812 7.05696162119907e-08
813 7.0014988100553e-08
814 7.00127900534753e-08
815 7.02365244169556e-08
816 6.97253932893105e-08
817 6.97972790710821e-08
818 6.98515553141377e-08
819 6.99035274758231e-08
820 6.93908340227978e-08
821 6.93654334105176e-08
822 6.95980193201606e-08
823 6.94594593682041e-08
824 6.92610824253315e-08
825 6.89076859128335e-08
826 6.91197988444969e-08
827 6.90255627562308e-08
828 6.86033866985625e-08
829 6.88852499122206e-08
830 6.88420915677668e-08
831 6.82408920278377e-08
832 6.88615147508287e-08
833 6.81375090128e-08
834 6.85287121848788e-08
835 6.83791471560724e-08
836 6.77617594648794e-08
837 6.83974003710119e-08
838 6.76211419046169e-08
839 6.79390781090916e-08
840 6.75045344307179e-08
841 6.77402744200251e-08
842 6.73288444676956e-08
843 6.76186373365084e-08
844 6.70971050533709e-08
845 6.74121427568508e-08
846 6.69207149703865e-08
847 6.69160437967164e-08
848 6.71367448967786e-08
849 6.67973792314314e-08
850 6.70078055318157e-08
851 6.65126296919283e-08
852 6.67566807539544e-08
853 6.63872701132107e-08
854 6.6380895963114e-08
855 6.63429631524437e-08
856 6.61862777580069e-08
857 6.62020221433579e-08
858 6.60141371202627e-08
859 6.60541056554109e-08
860 6.60906045624188e-08
861 6.57860123958187e-08
862 6.57433274309227e-08
863 6.59657311707917e-08
864 6.55916331036011e-08
865 6.54210960790991e-08
866 6.55087054877157e-08
867 6.57979669913544e-08
868 6.51728542839791e-08
869 6.56889574970343e-08
870 6.49679104882672e-08
871 6.51087673531059e-08
872 6.52929258886381e-08
873 6.47216486417967e-08
874 6.48872222637209e-08
875 6.51326053704437e-08
876 6.45543881474531e-08
877 6.46735512574992e-08
878 6.45487430102065e-08
879 6.49269926658036e-08
880 6.45228481550575e-08
881 6.42986779766375e-08
882 6.41528875870989e-08
883 6.47437260203887e-08
884 6.39288591925968e-08
885 6.40694320099833e-08
886 6.4004796873629e-08
887 6.42558669206039e-08
888 6.36711689705471e-08
889 6.42315322676446e-08
890 6.35124238210771e-08
891 6.36675860925528e-08
892 6.38545417759317e-08
893 6.34366355138916e-08
894 6.33727662537886e-08
895 6.33993925558407e-08
896 6.32870434502664e-08
897 6.3970659358592e-08
898 6.28898843699766e-08
899 6.31303887836943e-08
900 6.33301789378926e-08
901 6.28163022922124e-08
902 6.27179480954254e-08
903 6.2555188686364e-08
904 6.23509410586642e-08
905 6.21638449609385e-08
906 6.18482157759281e-08
907 6.14473152080741e-08
908 6.0907075688732e-08
909 6.03939177366008e-08
910 5.99855286029793e-08
911 5.92622134174547e-08
912 5.85745221988176e-08
913 5.80185986036597e-08
914 5.74672810302879e-08
915 5.68260388700281e-08
916 5.60614585389807e-08
917 5.56028848932577e-08
918 5.51802866812778e-08
919 5.46470639815055e-08
920 5.44134238391614e-08
921 5.40319474162665e-08
922 5.37277944370196e-08
923 5.33875942703155e-08
924 5.31915242834025e-08
925 5.29505831989852e-08
926 5.2819832591311e-08
927 5.25884931947829e-08
928 5.2460429101675e-08
929 5.21878354682315e-08
930 5.2088180290788e-08
931 5.18662262738445e-08
932 5.18350611908858e-08
933 5.1520094464852e-08
934 5.16080418702103e-08
935 5.10188741258411e-08
936 5.13929307999206e-08
937 5.10003965321637e-08
938 5.11437767234391e-08
939 5.0803864177329e-08
940 5.05234490120188e-08
941 5.0387687428266e-08
942 5.04491283193076e-08
943 5.03262264675008e-08
944 5.01467952869206e-08
945 5.01680322830822e-08
946 4.99580569441882e-08
947 4.9957825519531e-08
948 4.97833695858318e-08
949 4.97367274405747e-08
950 4.96082163841116e-08
951 4.96056110255161e-08
952 4.9459923125994e-08
953 4.94451457817568e-08
954 4.92972605536934e-08
955 4.92692499785008e-08
956 4.9126998831639e-08
957 4.91569321741103e-08
958 4.89774383525265e-08
959 4.89677165567137e-08
960 4.86751083816905e-08
961 4.88409458658801e-08
962 4.87628998846468e-08
963 4.86027700894454e-08
964 4.8523905650466e-08
965 4.84444045580901e-08
966 4.83840236484845e-08
967 4.82988045877875e-08
968 4.81517248349661e-08
969 4.81227854063349e-08
970 4.79404527005123e-08
971 4.79753831910124e-08
972 4.79709735197176e-08
973 4.77934865310559e-08
974 4.77378268413098e-08
975 4.77560899319052e-08
976 4.75934321086946e-08
977 4.75786179916504e-08
978 4.74769828637811e-08
979 4.75447861090217e-08
980 4.73110497201112e-08
981 4.74504124690078e-08
982 4.72479968425255e-08
983 4.72701040012957e-08
984 4.70883361609609e-08
985 4.71542460225294e-08
986 4.6984101374381e-08
987 4.70344780785226e-08
988 4.69214344511748e-08
989 4.69043241499989e-08
990 4.6784776199793e-08
991 4.68594284313006e-08
992 4.66488987713731e-08
993 4.6744572354207e-08
994 4.66712565283522e-08
995 4.6636065541783e-08
996 4.65599816492635e-08
997 4.65498708157597e-08
998 4.63449930108162e-08
999 4.64048814872164e-08
1000 4.63387282976235e-08
1001 4.63156972365653e-08
1002 4.61766843238287e-08
1003 4.6256772043396e-08
1004 4.61824508373176e-08
1005 4.61593465628241e-08
1006 4.61346036915167e-08
1007 4.60946311724442e-08
1008 4.60710525764796e-08
1009 4.59511968622373e-08
1010 4.59252619307016e-08
1011 4.60164391542506e-08
1012 4.58993651175632e-08
1013 4.58681477608636e-08
1014 4.58173986190857e-08
1015 4.57677186518346e-08
1016 4.57221172989186e-08
1017 4.56729917068799e-08
1018 4.56570479410168e-08
1019 4.5439231595612e-08
1020 4.54046487652349e-08
1021 4.54871496917697e-08
1022 4.55347814325435e-08
1023 4.55357997344308e-08
1024 4.53451553181772e-08
1025 4.53023498314664e-08
1026 4.52575334821681e-08
1027 4.53500591812706e-08
1028 4.50900151562905e-08
1029 4.50927455037942e-08
1030 4.50376759761895e-08
1031 4.50706656636157e-08
1032 4.48500709069322e-08
1033 4.48907290770961e-08
1034 4.48403222685911e-08
1035 4.47787387849452e-08
1036 4.48052750923189e-08
1037 4.49357175398291e-08
1038 4.46282186916669e-08
1039 4.4655669176219e-08
1040 4.45890279858752e-08
1041 4.45642145781022e-08
1042 4.44473021956604e-08
1043 4.44793546536815e-08
1044 4.44164771740141e-08
1045 4.44343633385635e-08
1046 4.43812920729592e-08
1047 4.43722223311127e-08
1048 4.43128193028386e-08
1049 4.42304761891776e-08
1050 4.41328822629039e-08
1051 4.41584022197183e-08
1052 4.40753578501152e-08
1053 4.40745483598626e-08
1054 4.40225693267848e-08
1055 4.39406731822523e-08
1056 4.39662168414401e-08
1057 4.38952535244397e-08
1058 4.38203851080843e-08
1059 4.37571059803687e-08
1060 4.40333969766726e-08
1061 4.36498814209152e-08
1062 4.36520391455808e-08
1063 4.36106878640707e-08
1064 4.35962172651294e-08
1065 4.35759118904144e-08
1066 4.35426203089229e-08
1067 4.35288793201316e-08
1068 4.33480123684937e-08
1069 4.33791592149291e-08
1070 4.34585542348209e-08
1071 4.3248705841048e-08
1072 4.3296340833443e-08
1073 4.32452687335605e-08
1074 4.3214090719168e-08
1075 4.31175834330055e-08
1076 4.32354491000098e-08
1077 4.31663599198551e-08
1078 4.30565439613062e-08
1079 4.30558684541005e-08
1080 4.29333225735462e-08
1081 4.29826798300503e-08
1082 4.29573251579107e-08
1083 4.27749555074186e-08
1084 4.29598080442517e-08
1085 4.27896117067839e-08
1086 4.26279881757807e-08
1087 4.27399929603389e-08
1088 4.25141664721629e-08
1089 4.25838033368819e-08
1090 4.26017397758827e-08
1091 4.25777038302044e-08
1092 4.24766256315046e-08
1093 4.25206783738474e-08
1094 4.23628226742245e-08
1095 4.24051594349883e-08
1096 4.23856514717436e-08
1097 4.23835907024106e-08
1098 4.22450804467012e-08
1099 4.22132893582727e-08
1100 4.22120134366821e-08
1101 4.23378308251898e-08
1102 4.21375162189541e-08
1103 4.21238110543953e-08
1104 4.22320239610308e-08
1105 4.20084211132377e-08
1106 4.20322475203072e-08
1107 4.19322954505752e-08
1108 4.21262123042609e-08
1109 4.18944169036095e-08
1110 4.18889404811829e-08
1111 4.18677106130971e-08
1112 4.17829334606878e-08
1113 4.17890989621306e-08
1114 4.16482059302226e-08
1115 4.16191958461098e-08
1116 4.16126598006272e-08
1117 4.16599009334107e-08
1118 4.15678648493234e-08
1119 4.1568845344564e-08
1120 4.15357823926321e-08
1121 4.1422018507653e-08
1122 4.14116873366233e-08
1123 4.13583534726314e-08
1124 4.14684939435972e-08
1125 4.13291089387258e-08
1126 4.13047057072902e-08
1127 4.15829793558586e-08
1128 4.13521140036899e-08
1129 4.13341457958794e-08
1130 4.13573696227409e-08
1131 4.11848835928375e-08
1132 4.13006091815049e-08
1133 4.11322232105604e-08
1134 4.10140667619352e-08
1135 4.11248679301757e-08
1136 4.0933230048612e-08
1137 4.1319673813156e-08
1138 4.10346303567621e-08
1139 4.10384597975266e-08
1140 4.09156260396948e-08
1141 4.09905457003923e-08
1142 4.0810462786478e-08
1143 4.07560918462835e-08
1144 4.08293160134576e-08
1145 4.07832992666002e-08
1146 4.07475838319904e-08
1147 4.09660192350891e-08
1148 4.07263909814048e-08
1149 4.06752636550323e-08
1150 4.05794020794659e-08
1151 4.04703290928232e-08
1152 4.06187871888619e-08
1153 4.05392382503145e-08
1154 4.06321918706709e-08
1155 4.04667961970695e-08
1156 4.04539964340422e-08
1157 4.03856908999245e-08
1158 4.04976351364716e-08
1159 4.02848259013489e-08
1160 4.03060009661616e-08
1161 4.02477352507979e-08
1162 4.01400819542985e-08
1163 4.0190456706668e-08
1164 4.0114540063918e-08
1165 4.00608516959444e-08
1166 4.0117134891382e-08
1167 3.99593462425685e-08
1168 3.99921360014766e-08
1169 3.99323362714199e-08
1170 3.98582817919824e-08
1171 3.99058444879863e-08
1172 3.98500721119355e-08
1173 3.98250477031681e-08
1174 3.9682609734637e-08
1175 3.97361764954418e-08
1176 3.97374674871998e-08
1177 3.96576265395687e-08
1178 3.9686726484689e-08
1179 3.95573232578883e-08
1180 3.9566657996204e-08
1181 3.95528471379691e-08
1182 3.95480813448756e-08
1183 3.96847883554763e-08
1184 3.94569686705815e-08
1185 3.94353054491425e-08
1186 3.93713205011359e-08
1187 3.92887176001722e-08
1188 3.93929057405096e-08
1189 3.9267367863971e-08
1190 3.91870350084922e-08
1191 3.94059722008677e-08
1192 3.92131949062957e-08
1193 3.92178153085787e-08
1194 3.90295574081989e-08
1195 3.93024120448615e-08
1196 3.91030192723463e-08
1197 3.90716983740269e-08
1198 3.9020508397325e-08
1199 3.89942824763345e-08
1200 3.91714583090064e-08
1201 3.88781101623437e-08
1202 3.89200965167369e-08
1203 3.88407974889482e-08
1204 3.88562908590906e-08
1205 3.88229454371114e-08
1206 3.8956119455591e-08
1207 3.86508901035754e-08
1208 3.87421535568144e-08
1209 3.85537082210341e-08
1210 3.88021263404426e-08
1211 3.85541123391064e-08
1212 3.86260631803914e-08
1213 3.84725348485659e-08
1214 3.87271141795154e-08
1215 3.84948107106098e-08
1216 3.84982986196825e-08
1217 3.8409302138831e-08
1218 3.84087779847775e-08
1219 3.83520728661768e-08
1220 3.85375156279899e-08
1221 3.82814989268176e-08
1222 3.83027617192333e-08
1223 3.82354802637153e-08
1224 3.82441866069172e-08
1225 3.82957357900615e-08
1226 3.81817408867668e-08
1227 3.81227810355433e-08
1228 3.81452144879724e-08
1229 3.81909170803496e-08
1230 3.80572051921213e-08
1231 3.79955674625343e-08
1232 3.79426776855318e-08
1233 3.81580526918057e-08
1234 3.80025603696765e-08
1235 3.78542714116747e-08
1236 3.77667793087788e-08
1237 3.77548642798153e-08
1238 3.78945893166893e-08
1239 3.77150857677755e-08
1240 3.77170766969748e-08
1241 3.77188020590502e-08
1242 3.76646035338268e-08
1243 3.76876409275972e-08
1244 3.75885874630555e-08
1245 3.76562517372392e-08
1246 3.75484005181725e-08
1247 3.7594999077406e-08
1248 3.73997783906788e-08
1249 3.77198778820365e-08
1250 3.73777261657438e-08
1251 3.74015631581237e-08
1252 3.74792809068047e-08
1253 3.73645127482725e-08
1254 3.71016707476635e-08
1255 3.74975929613441e-08
1256 3.72620466086993e-08
1257 3.72752694763889e-08
1258 3.71979972864089e-08
1259 3.71732924855372e-08
1260 3.70620368315144e-08
1261 3.72312924530327e-08
1262 3.70417015278512e-08
1263 3.70822132240711e-08
1264 3.70075613300713e-08
1265 3.6981717692619e-08
1266 3.69555602421912e-08
1267 3.70738293735684e-08
1268 3.69265524264861e-08
1269 3.69277033356319e-08
1270 3.68192815249913e-08
1271 3.67990939844631e-08
1272 3.674601015069e-08
1273 3.69042122736474e-08
1274 3.67706086232822e-08
1275 3.66758063199413e-08
1276 3.67052685539271e-08
1277 3.66670316700457e-08
1278 3.66532941691311e-08
1279 3.65664958352774e-08
1280 3.64895526048237e-08
1281 3.65217291005138e-08
1282 3.64578798670578e-08
1283 3.64071254961296e-08
1284 3.63790040602829e-08
1285 3.63837145425094e-08
1286 3.6380524310875e-08
1287 3.6282704102053e-08
1288 3.62392835957515e-08
1289 3.64692903951358e-08
1290 3.61691976653589e-08
1291 3.61854716368626e-08
1292 3.61273909761373e-08
1293 3.61941897826235e-08
1294 3.61768891012915e-08
1295 3.62208859603363e-08
1296 3.612770713457e-08
1297 3.6104094118139e-08
1298 3.60890376980283e-08
1299 3.6058858152499e-08
1300 3.63187192990999e-08
1301 3.59064247594798e-08
1302 3.60688944571752e-08
1303 3.59934266438433e-08
1304 3.60915646304694e-08
1305 3.59702184087318e-08
1306 3.59724761975766e-08
1307 3.58357881105054e-08
1308 3.59688163780181e-08
1309 3.58330693994713e-08
1310 3.57575529015275e-08
1311 3.58741615487101e-08
1312 3.59660988635824e-08
1313 3.5769270317676e-08
1314 3.57458835327229e-08
1315 3.5615976299086e-08
1316 3.57423017327552e-08
1317 3.57890865039234e-08
1318 3.56824700489256e-08
1319 3.56553104794433e-08
1320 3.5681581980862e-08
1321 3.56421057630119e-08
1322 3.56427627166145e-08
1323 3.5604705444392e-08
1324 3.55177646327309e-08
1325 3.55654241503967e-08
1326 3.557548764066e-08
1327 3.53744084280461e-08
1328 3.54964436293237e-08
1329 3.5448479456468e-08
1330 3.53106293173866e-08
1331 3.53805912911564e-08
1332 3.53346911730057e-08
1333 3.53091836706731e-08
1334 3.51395846698299e-08
1335 3.52430520809133e-08
1336 3.52491265547261e-08
1337 3.52396602139038e-08
1338 3.52366243616675e-08
1339 3.50422454926669e-08
1340 3.5220697742977e-08
1341 3.52045141280843e-08
1342 3.51082574434614e-08
1343 3.50961475645839e-08
1344 3.50289765440603e-08
1345 3.499926935846e-08
1346 3.50342812278637e-08
1347 3.51355330958647e-08
1348 3.50768839203486e-08
1349 3.50623846221421e-08
1350 3.50292792277163e-08
1351 3.5051213586268e-08
1352 3.49438466549223e-08
1353 3.5050651661761e-08
1354 3.48456674101882e-08
1355 3.50292955817455e-08
1356 3.49066233216178e-08
1357 3.48448146663127e-08
1358 3.48367268137029e-08
1359 3.49606110348066e-08
1360 3.49719955106131e-08
1361 3.46467201421952e-08
1362 3.49423424519379e-08
1363 3.46741304202425e-08
1364 3.47886638429884e-08
1365 3.46033498817011e-08
1366 3.47059669645322e-08
1367 3.4609285269438e-08
1368 3.46863289470978e-08
1369 3.46012855516609e-08
1370 3.44541630969974e-08
1371 3.44734615265185e-08
1372 3.44783536607718e-08
1373 3.4630256744661e-08
1374 3.44150285220834e-08
1375 3.47468691894726e-08
1376 3.42983796968177e-08
1377 3.44549150124163e-08
1378 3.43680293666271e-08
1379 3.429672751043e-08
1380 3.42581789196661e-08
1381 3.40411411796637e-08
1382 3.4130696219048e-08
1383 3.43402810052407e-08
1384 3.41391361433629e-08
1385 3.41822274956272e-08
1386 3.40653693431392e-08
1387 3.41478332670508e-08
1388 3.40814091381869e-08
1389 3.43203988248941e-08
1390 3.39625150407308e-08
1391 3.37426882806025e-08
1392 3.37376575216819e-08
1393 3.3548095496716e-08
1394 3.36228592059396e-08
1395 3.33996923804758e-08
1396 3.36945281109724e-08
1397 3.34397022458965e-08
1398 3.39078665312353e-08
1399 3.37377990542453e-08
1400 3.36806030221126e-08
1401 3.39746803534879e-08
1402 3.38620829467562e-08
1403 3.39268961853101e-08
1404 3.35322298405938e-08
1405 3.37501213796276e-08
1406 3.3286528837051e-08
1407 3.33608471405622e-08
1408 3.31617015407737e-08
1409 3.32187424110231e-08
1410 3.30049204864125e-08
1411 3.33138679582401e-08
1412 3.30135336081838e-08
1413 3.32318145719324e-08
1414 3.31274270481963e-08
1415 3.32249078855984e-08
1416 3.33866609796285e-08
1417 3.32927924990845e-08
1418 3.32172163877154e-08
1419 3.32476972011442e-08
1420 3.33199325934785e-08
1421 3.31282581704784e-08
1422 3.32634028719081e-08
1423 3.31277496901095e-08
1424 3.30957700249712e-08
1425 3.30378593271785e-08
1426 3.30369815046971e-08
1427 3.29672449885798e-08
1428 3.29855168847981e-08
1429 3.30111134227984e-08
1430 3.2663245885356e-08
1431 3.25891274097767e-08
1432 3.30090688209772e-08
1433 3.27325771825038e-08
1434 3.29008018897792e-08
1435 3.29373322505155e-08
1436 3.26931847345335e-08
1437 3.29599990134266e-08
1438 3.2749561331924e-08
1439 3.29031246242284e-08
1440 3.25891354924224e-08
1441 3.28564242644269e-08
1442 3.25853394864684e-08
1443 3.28481211535614e-08
1444 3.25563175784804e-08
1445 3.28328275209788e-08
1446 3.23674079547587e-08
1447 3.28233372821618e-08
1448 3.25789287396461e-08
1449 3.27593740816479e-08
1450 3.23845026624081e-08
1451 3.25203114248662e-08
1452 3.26090586562078e-08
1453 3.26427584764755e-08
1454 3.23536438895378e-08
1455 3.25229951094741e-08
1456 3.23102900112726e-08
1457 3.2672745353679e-08
1458 3.22717890692825e-08
1459 3.24415854211502e-08
1460 3.2139215565774e-08
1461 3.22936504792981e-08
1462 3.23310816514599e-08
1463 3.22062169677562e-08
1464 3.26094380467268e-08
1465 3.20894121361714e-08
1466 3.22084522803578e-08
1467 3.21908273746008e-08
1468 3.22994952210998e-08
1469 3.20596196972112e-08
1470 3.23849861703085e-08
1471 3.20516775187407e-08
1472 3.22200081293023e-08
1473 3.21878954350208e-08
1474 3.19157378947388e-08
1475 3.22873946407842e-08
1476 3.19907962975652e-08
1477 3.22008729389811e-08
1478 3.19338547616344e-08
1479 3.19142678496531e-08
1480 3.19380239763856e-08
1481 3.18116881925512e-08
1482 3.18849455818171e-08
1483 3.18762919120363e-08
1484 3.18187397876368e-08
1485 3.19069196108579e-08
1486 3.15959868466553e-08
1487 3.19013813128155e-08
1488 3.18728866091078e-08
1489 3.18309048294996e-08
1490 3.16326883009754e-08
1491 3.16239017537079e-08
1492 3.1759811082388e-08
1493 3.17256786888098e-08
1494 3.16533028039689e-08
1495 3.16961240465208e-08
1496 3.1743765402048e-08
1497 3.16434377582997e-08
1498 3.13523172383867e-08
1499 3.17003001878424e-08
1500 3.14640299392543e-08
1501 3.13509718219418e-08
1502 3.16191568601187e-08
1503 3.15610303067704e-08
1504 3.14460529042826e-08
1505 3.13420410091858e-08
1506 3.13002594707346e-08
1507 3.15244424146099e-08
1508 3.15261064514072e-08
1509 3.11401516353005e-08
1510 3.11578168362692e-08
1511 3.09322660664302e-08
1512 3.13200672452663e-08
1513 3.08947814224858e-08
1514 3.08836487508124e-08
1515 3.08906631389938e-08
1516 3.09200834837098e-08
1517 3.12947964440369e-08
1518 3.13176236356183e-08
1519 3.1297475206804e-08
1520 3.12362852837866e-08
1521 3.10753392123164e-08
1522 3.11621864617084e-08
1523 3.1305982508556e-08
1524 3.12275181209021e-08
1525 3.11288718113367e-08
1526 3.12782656008981e-08
1527 3.10281218108832e-08
1528 3.08859019300112e-08
1529 3.10181556635314e-08
1530 3.06253198388839e-08
1531 3.06205754128097e-08
1532 3.05454168046948e-08
1533 3.09590752445299e-08
1534 3.08275637519451e-08
1535 3.0977128390286e-08
1536 3.07058769287716e-08
1537 3.08098103718457e-08
1538 3.04403921170815e-08
1539 3.0332626744678e-08
1540 3.08719542866376e-08
1541 3.04892000088097e-08
1542 3.06898614944595e-08
1543 3.05443414081452e-08
1544 3.07060035448226e-08
1545 3.06416953193001e-08
1546 3.07680730473603e-08
1547 3.04729701636752e-08
1548 3.06781224302943e-08
1549 3.04202770883277e-08
1550 3.03999615706152e-08
1551 3.06561781556747e-08
1552 3.03625995494716e-08
1553 3.02000256786616e-08
1554 3.04677741064241e-08
1555 3.02973430702513e-08
1556 3.04021386171582e-08
1557 3.03990550446498e-08
1558 3.02341169018394e-08
1559 3.03277893632936e-08
1560 3.03182047802775e-08
1561 3.02733629885532e-08
1562 3.01512492980605e-08
1563 3.02340561892933e-08
1564 3.02633301459965e-08
1565 3.01362880655853e-08
1566 3.01448226119883e-08
1567 3.00670022481686e-08
1568 3.00977472719177e-08
1569 3.01355102593215e-08
1570 3.00983004983824e-08
1571 3.01050681790205e-08
1572 2.98478572560157e-08
1573 2.99916708769166e-08
1574 2.97553273904683e-08
1575 2.99156533544176e-08
1576 2.98853319212755e-08
1577 2.93450022466057e-08
1578 2.98060234690256e-08
1579 2.96049233891038e-08
1580 2.98665963913702e-08
1581 2.96021957495451e-08
1582 2.96946968959544e-08
1583 2.96843666296454e-08
1584 2.96868646014747e-08
1585 2.94973314072244e-08
1586 2.95995248985603e-08
1587 2.93916899857249e-08
1588 2.94762092669654e-08
1589 2.93966270618062e-08
1590 2.96836639983589e-08
1591 2.95226158545381e-08
1592 2.94334557077169e-08
1593 2.94427078301185e-08
1594 2.93381085827438e-08
1595 2.93158368511737e-08
1596 2.92878507491823e-08
1597 2.9251059141977e-08
1598 2.91064381696682e-08
1599 2.9285782635724e-08
1600 2.90732876053745e-08
1601 2.9159231105913e-08
1602 2.92045611159919e-08
1603 2.89445787630127e-08
1604 2.91978859419117e-08
1605 2.90564195284393e-08
1606 2.93973981386753e-08
1607 2.89381930623778e-08
1608 2.91224631346898e-08
1609 2.9097963642144e-08
1610 2.90793898329866e-08
1611 2.90082033336248e-08
1612 2.86929799684366e-08
1613 2.88084534817434e-08
1614 2.88467914446588e-08
1615 2.91303532875542e-08
1616 2.87767091675661e-08
1617 2.89022039634546e-08
1618 2.88018204208651e-08
1619 2.87355058496974e-08
1620 2.88769270928579e-08
1621 2.86832303697526e-08
1622 2.88162975234929e-08
1623 2.87382800380209e-08
1624 2.87411438115726e-08
1625 2.85958220420657e-08
1626 2.87125149687206e-08
1627 2.86978516395031e-08
1628 2.86231476263055e-08
1629 2.8606763228689e-08
1630 2.84917393582607e-08
1631 2.8626619281269e-08
1632 2.85001897744053e-08
1633 2.85990815509507e-08
1634 2.83304278687613e-08
1635 2.85175816796546e-08
1636 2.84905214699149e-08
1637 2.83933514477752e-08
1638 2.84283543807717e-08
1639 2.8434385652254e-08
1640 2.83303585477679e-08
1641 2.82702170479876e-08
1642 2.82813551275574e-08
1643 2.82564017748399e-08
1644 2.83993728059651e-08
1645 2.81689663723661e-08
1646 2.84954706953888e-08
1647 2.81906312450397e-08
1648 2.82125116868315e-08
1649 2.82194713823181e-08
1650 2.81256498424476e-08
1651 2.82084918952208e-08
1652 2.81440984261083e-08
1653 2.80632214066934e-08
1654 2.80641867798082e-08
1655 2.81498658785129e-08
1656 2.81229695745511e-08
1657 2.78262532681373e-08
1658 2.81152595844603e-08
1659 2.80978226059325e-08
1660 2.79516244852118e-08
1661 2.78943760385619e-08
1662 2.79658791044746e-08
1663 2.78277536485128e-08
1664 2.80023152290365e-08
1665 2.79128439117926e-08
1666 2.7897136156696e-08
1667 2.77075219505818e-08
1668 2.79106471205992e-08
1669 2.76841531492433e-08
1670 2.7678924863217e-08
1671 2.79674940422137e-08
1672 2.77012584739555e-08
1673 2.7686614389788e-08
1674 2.76578665996396e-08
1675 2.7500948801995e-08
1676 2.78345738775609e-08
1677 2.74509543514156e-08
1678 2.75210666080827e-08
1679 2.75053529441571e-08
1680 2.74475394685947e-08
1681 2.75807887444302e-08
1682 2.76947286979023e-08
1683 2.74343467587812e-08
1684 2.74334480561134e-08
1685 2.74315980763884e-08
1686 2.74044391626038e-08
1687 2.7445086620892e-08
1688 2.73404861366178e-08
1689 2.72715645792676e-08
1690 2.73514556468069e-08
1691 2.71716442334924e-08
1692 2.72701849435286e-08
1693 2.71976452839162e-08
1694 2.7377970337028e-08
1695 2.72634717259912e-08
1696 2.73470870393311e-08
1697 2.73837353930162e-08
1698 2.70808842524062e-08
1699 2.70746259296573e-08
1700 2.72501972473638e-08
1701 2.72303381951833e-08
1702 2.71665033899637e-08
1703 2.71494568817587e-08
1704 2.70043787722329e-08
1705 2.70434801179498e-08
1706 2.71050752125301e-08
1707 2.6744345520191e-08
1708 2.69874022171823e-08
1709 2.67261778263972e-08
1710 2.69710650022681e-08
1711 2.69675744470899e-08
1712 2.70572562841576e-08
1713 2.70181432391325e-08
1714 2.69291252357196e-08
1715 2.67554949959781e-08
1716 2.67099414472671e-08
1717 2.69469694398339e-08
1718 2.67576639836342e-08
1719 2.6812395259812e-08
1720 2.68022366998188e-08
1721 2.67500768504902e-08
1722 2.66494636140813e-08
1723 2.68118679433993e-08
1724 2.67029782461403e-08
1725 2.66949636853475e-08
1726 2.66502388024392e-08
1727 2.65341316854073e-08
1728 2.67588061044766e-08
1729 2.6614885338283e-08
1730 2.66600372125225e-08
1731 2.66130355506267e-08
1732 2.66703143800839e-08
1733 2.64110081102942e-08
1734 2.64938039853391e-08
1735 2.66615982285057e-08
1736 2.64054320213702e-08
1737 2.65587347143548e-08
1738 2.65222390625208e-08
1739 2.63894420862165e-08
1740 2.6360259772984e-08
1741 2.64929267144165e-08
1742 2.6412281394661e-08
1743 2.6407221850322e-08
1744 2.63294615512333e-08
1745 2.63403224807224e-08
1746 2.6231389064435e-08
1747 2.63035168626136e-08
1748 2.63920992353839e-08
1749 2.61704505164495e-08
1750 2.62700999285048e-08
1751 2.61262359129155e-08
1752 2.61894954578512e-08
1753 2.61557138292456e-08
1754 2.63159896440879e-08
1755 2.60753099894373e-08
1756 2.62050543389236e-08
1757 2.60775467004759e-08
1758 2.59859599829326e-08
1759 2.6036633839599e-08
1760 2.61388372055915e-08
1761 2.59640781588022e-08
1762 2.61055038590641e-08
1763 2.60683015800289e-08
1764 2.60145808088641e-08
1765 2.60897448483011e-08
1766 2.59506690316602e-08
1767 2.61972732054083e-08
1768 2.58460445594633e-08
1769 2.58610650405444e-08
1770 2.57995828321089e-08
1771 2.56075826021718e-08
1772 2.58178713961055e-08
1773 2.5774479145424e-08
1774 2.55945550751147e-08
1775 2.58654844456085e-08
1776 2.56929315549659e-08
1777 2.57321764469687e-08
1778 2.55612184734133e-08
1779 2.57684632134492e-08
1780 2.56617219460686e-08
1781 2.57087004098544e-08
1782 2.56135626492959e-08
1783 2.55158560598101e-08
1784 2.54185498531623e-08
1785 2.56052989009437e-08
1786 2.56310354231903e-08
1787 2.55031264424499e-08
1788 2.5316933796482e-08
1789 2.53860044070464e-08
1790 2.56797648408025e-08
1791 2.48707858949704e-08
1792 2.54236516012973e-08
1793 2.49762603627968e-08
1794 2.55016684791443e-08
1795 2.50913213939441e-08
1796 2.54429933099765e-08
1797 2.55209315955796e-08
1798 2.55042124885918e-08
1799 2.50115643598559e-08
1800 2.53265058174357e-08
1801 2.50815077866839e-08
1802 2.49071452327332e-08
1803 2.49964702279115e-08
1804 2.51852548602827e-08
1805 2.52614290421338e-08
1806 2.50941749568856e-08
1807 2.57214992747112e-08
1808 2.493479761978e-08
1809 2.53148009587978e-08
1810 2.51504016126614e-08
1811 2.52236170983355e-08
1812 2.49849139715153e-08
1813 2.530442585158e-08
1814 2.50434625774787e-08
1815 2.48893271959094e-08
1816 2.51477175960968e-08
1817 2.50058823516586e-08
1818 2.50295362329567e-08
1819 2.49894153807606e-08
1820 2.51651203055925e-08
1821 2.46673818224963e-08
1822 2.46630429459049e-08
1823 2.50481074250963e-08
1824 2.49097898838979e-08
1825 2.46014107589954e-08
1826 2.46993615939939e-08
1827 2.45339520978938e-08
1828 2.49525591715649e-08
1829 2.47696219430882e-08
1830 2.44621726206429e-08
1831 2.45611794844613e-08
1832 2.46466997708694e-08
1833 2.44954276324982e-08
1834 2.46191131891837e-08
1835 2.45088469990051e-08
1836 2.46523927169839e-08
1837 2.51540131090966e-08
1838 2.42904432734559e-08
1839 2.44662644071081e-08
1840 2.43234662393732e-08
1841 2.46560865264644e-08
1842 2.45945296670769e-08
1843 2.45644592509198e-08
1844 2.45259563544931e-08
1845 2.45034537578181e-08
1846 2.4272482316734e-08
1847 2.42850607017342e-08
1848 2.43279326630486e-08
1849 2.44551986623431e-08
1850 2.41824358659937e-08
1851 2.42782829682486e-08
1852 2.43085409401544e-08
1853 2.42297266825897e-08
1854 2.45176271522762e-08
1855 2.44734540038039e-08
1856 2.41164308609765e-08
1857 2.4421313831624e-08
1858 2.39973656888104e-08
1859 2.44747148088198e-08
1860 2.40032483276309e-08
1861 2.43450992958305e-08
1862 2.43010490765805e-08
1863 2.40007947220899e-08
1864 2.41608375428637e-08
1865 2.4229734788328e-08
1866 2.44724201283741e-08
1867 2.38308878710569e-08
1868 2.42633171996243e-08
1869 2.39411463001105e-08
1870 2.41847772872728e-08
1871 2.42143409665552e-08
1872 2.38694977880982e-08
1873 2.42167625599254e-08
1874 2.3863186029649e-08
1875 2.41285378916878e-08
1876 2.43865005382649e-08
1877 2.36942367346504e-08
1878 2.37669945895025e-08
1879 2.44419815864649e-08
1880 2.4345149153282e-08
1881 2.37244264751357e-08
1882 2.40047764989981e-08
1883 2.37200104897806e-08
1884 2.36512128066746e-08
1885 2.37650904628683e-08
1886 2.40899134211769e-08
1887 2.42278880209224e-08
1888 2.38468219899701e-08
1889 2.39388029488197e-08
1890 2.3890235657742e-08
1891 2.38744797604884e-08
1892 2.40933459849657e-08
1893 2.35299260729072e-08
1894 2.39192101585406e-08
1895 2.33616429412553e-08
1896 2.43039577669801e-08
1897 2.33248757117721e-08
1898 2.38793327469455e-08
1899 2.3452037661853e-08
1900 2.35405627295737e-08
1901 2.35744488044087e-08
1902 2.38533790353213e-08
1903 2.37158306313212e-08
1904 2.42975806892254e-08
1905 2.33427496685312e-08
1906 2.36155815491745e-08
1907 2.3639630454042e-08
1908 2.38047376226458e-08
1909 2.32812016978112e-08
1910 2.38944234589589e-08
1911 2.35866443016874e-08
1912 2.34638540899113e-08
1913 2.34778643337652e-08
1914 2.37370965949779e-08
1915 2.33362868733433e-08
1916 2.34993911598647e-08
1917 2.32794279590998e-08
1918 2.33390705024483e-08
1919 2.34632395423873e-08
1920 2.32336091614549e-08
1921 2.34295352576908e-08
1922 2.32223738432236e-08
1923 2.33132657769808e-08
1924 2.3255931385302e-08
1925 2.31236226575504e-08
1926 2.33470837687211e-08
1927 2.30042482134429e-08
1928 2.3194602268628e-08
1929 2.33100002793307e-08
1930 2.3435807366079e-08
1931 2.32626270719738e-08
1932 2.33190908354164e-08
1933 2.3117477975898e-08
1934 2.37152451603162e-08
1935 2.28862426983056e-08
1936 2.31845497471461e-08
1937 2.31226390734474e-08
1938 2.30709065893731e-08
1939 2.39123361041038e-08
1940 2.28171227676377e-08
1941 2.32530431671929e-08
1942 2.3131472768112e-08
1943 2.30621939898423e-08
1944 2.29314333273223e-08
1945 2.32168188669668e-08
1946 2.3098095009555e-08
1947 2.32751532329711e-08
1948 2.30866664168161e-08
1949 2.2880534887193e-08
1950 2.29379246698969e-08
1951 2.31533310497412e-08
1952 2.28266297404023e-08
1953 2.27963624148675e-08
1954 2.28977941485464e-08
1955 2.27577228928411e-08
1956 2.30468410924445e-08
1957 2.27781810546723e-08
1958 2.28757487943376e-08
1959 2.2992807299671e-08
1960 2.29219160849947e-08
1961 2.25344397821736e-08
1962 2.29710848955733e-08
1963 2.29069330204279e-08
1964 2.27208864127526e-08
1965 2.27095788860598e-08
1966 2.25576825658713e-08
1967 2.27239538534985e-08
1968 2.26483780958464e-08
1969 2.26724076397566e-08
1970 2.25678247596228e-08
1971 2.27302247266525e-08
1972 2.25470728010269e-08
1973 2.26035787846968e-08
1974 2.28273870743756e-08
1975 2.26087536592257e-08
1976 2.25747645283825e-08
1977 2.24962888797364e-08
1978 2.27287393981257e-08
1979 2.25441221823885e-08
1980 2.25790164947171e-08
1981 2.242462721469e-08
1982 2.24742255476862e-08
1983 2.25754250799959e-08
1984 2.25259860182625e-08
1985 2.24557539014736e-08
1986 2.25188727900338e-08
1987 2.2686083546164e-08
1988 2.23732329931448e-08
1989 2.24533313288866e-08
1990 2.24434592781542e-08
1991 2.24497162739645e-08
1992 2.23874790501455e-08
1993 2.2536742137369e-08
1994 2.22100328213237e-08
1995 2.23196954975968e-08
1996 2.25118213623698e-08
1997 2.21823936323862e-08
1998 2.21988227713243e-08
1999 2.22187915954741e-08
2000 2.21332472034597e-08
2001 2.23908839440679e-08
2002 2.21010236771679e-08
2003 2.21985469863739e-08
2004 2.22542241703483e-08
2005 2.20859244088878e-08
2006 2.23788570381522e-08
2007 2.21882864908096e-08
2008 2.2117744433503e-08
2009 2.22524815014502e-08
2010 2.24436834201924e-08
2011 2.21177146058071e-08
2012 2.22553940352199e-08
2013 2.2530287963507e-08
2014 2.19837373407472e-08
2015 2.23302817783377e-08
2016 2.20564601443041e-08
2017 2.19482824952433e-08
2018 2.22820709814275e-08
2019 2.22843829991159e-08
2020 2.18305001886687e-08
2021 2.19540335997248e-08
2022 2.21178324530946e-08
2023 2.19199797091285e-08
2024 2.2231762760061e-08
2025 2.23676870811929e-08
2026 2.18616234262115e-08
2027 2.19298957728764e-08
2028 2.19169428108401e-08
2029 2.18028709333407e-08
2030 2.19173556310626e-08
2031 2.19906482450405e-08
2032 2.17367337167484e-08
2033 2.19169118433893e-08
2034 2.18894617132204e-08
2035 2.17210407837953e-08
2036 2.17937424065173e-08
2037 2.20567556084017e-08
2038 2.189161369337e-08
2039 2.17621018663294e-08
2040 2.16101795409429e-08
2041 2.17156638513405e-08
2042 2.19005479265011e-08
2043 2.1729553480565e-08
2044 2.15771030651624e-08
2045 2.17384802687626e-08
2046 2.19873718596553e-08
2047 2.13370392725398e-08
2048 2.15733347692648e-08
2049 2.20031073150384e-08
2050 2.15017191687039e-08
2051 2.19424855396611e-08
2052 2.18187407137194e-08
2053 2.16646034096879e-08
2054 2.15932464886492e-08
2055 2.14310003832097e-08
2056 2.15815487010218e-08
2057 2.13446278904961e-08
2058 2.15179588347603e-08
2059 2.13864460432145e-08
2060 2.13749149935438e-08
2061 2.14256360802079e-08
2062 2.1648634122462e-08
2063 2.12002782089549e-08
2064 2.12036070714738e-08
2065 2.1239837395326e-08
2066 2.12868816207301e-08
2067 2.15947479906831e-08
2068 2.1101051398853e-08
2069 2.1370955280986e-08
2070 2.10768452639787e-08
2071 2.11820005475705e-08
2072 2.13040647638607e-08
2073 2.16485227089169e-08
2074 2.12624537172523e-08
2075 2.11493394284279e-08
2076 2.12736924660728e-08
2077 2.12909563688246e-08
2078 2.10832268814354e-08
2079 2.08918191997309e-08
2080 2.1037937305568e-08
2081 2.12358856903627e-08
2082 2.10105512701286e-08
2083 2.09346251971443e-08
2084 2.08090945263972e-08
2085 2.13037659657633e-08
2086 2.08495705549261e-08
2087 2.10626978331963e-08
2088 2.08959614009796e-08
2089 2.11670114853568e-08
2090 2.08808468453725e-08
2091 2.1377395625688e-08
2092 2.09032860836178e-08
2093 2.07949947699237e-08
2094 2.13837006239892e-08
2095 2.10224441157258e-08
2096 2.14489621639391e-08
2097 2.09005829110431e-08
2098 2.07459174539348e-08
2099 2.07949398498553e-08
2100 2.12638737637771e-08
2101 2.06582394202126e-08
2102 2.07968316392471e-08
2103 2.06335235590771e-08
2104 2.09524346823642e-08
2105 2.12689093408436e-08
2106 2.04819985196725e-08
2107 2.07256096698138e-08
2108 2.11357777428756e-08
2109 2.0571573642103e-08
2110 2.05780526745247e-08
2111 2.06272095109039e-08
2112 2.11973604757887e-08
2113 2.08360683278119e-08
2114 2.0730334511887e-08
2115 2.04650406465667e-08
2116 2.05816970271222e-08
2117 2.05755886042436e-08
2118 2.0690599925155e-08
2119 2.09416671561602e-08
2120 2.04880210450398e-08
2121 2.04747215810652e-08
2122 2.075157809589e-08
2123 2.06296829026176e-08
2124 2.0593225479093e-08
2125 2.07324723200397e-08
2126 2.02645847782179e-08
2127 2.05422752490581e-08
2128 2.03945892023327e-08
2129 2.03805121012923e-08
2130 2.07005650920689e-08
2131 2.07021155484988e-08
2132 2.04187257299981e-08
2133 2.02694781255008e-08
2134 2.04208749405854e-08
2135 2.02069237444924e-08
2136 2.04020739054123e-08
2137 2.04576490969721e-08
2138 2.08270429609136e-08
2139 2.05362811107612e-08
2140 2.014992423649e-08
2141 2.03573385348932e-08
2142 2.07267850880211e-08
2143 2.01660785403845e-08
2144 2.00499002269705e-08
2145 2.03590918947949e-08
2146 2.04229112745757e-08
2147 2.03734770141128e-08
2148 2.07319408107587e-08
2149 2.04063193200987e-08
2150 2.01375142765414e-08
2151 1.99445387309982e-08
2152 2.06006851979357e-08
2153 2.00142558879346e-08
2154 2.01612419568065e-08
2155 2.06378325011691e-08
2156 2.00262113765337e-08
2157 2.02984457644551e-08
2158 2.0390784864821e-08
2159 2.00895876121265e-08
2160 2.04201354763178e-08
2161 1.99852220399421e-08
2162 2.04412891371408e-08
2163 2.03172153596665e-08
2164 2.01257913319264e-08
2165 2.04093899558444e-08
2166 1.98856412638726e-08
2167 1.99936444444937e-08
2168 2.00312219200693e-08
2169 2.02891134699623e-08
2170 2.01223698634845e-08
2171 2.01661116430163e-08
2172 2.01733477966926e-08
2173 2.00352967949513e-08
2174 1.98352333318219e-08
2175 2.01847917260523e-08
2176 1.99118999595349e-08
2177 1.99749063114485e-08
2178 2.03200145054705e-08
2179 1.98196263581085e-08
2180 1.98023723008145e-08
2181 2.01643725785683e-08
2182 1.98059063907241e-08
2183 1.97974697131453e-08
2184 1.97923357754259e-08
2185 1.9798583585251e-08
2186 1.99775578333394e-08
2187 1.97913241262171e-08
2188 2.00028937975549e-08
2189 1.96899198359191e-08
2190 1.96931721088145e-08
2191 1.9809620651623e-08
2192 2.00088359754158e-08
2193 2.00936854064526e-08
2194 1.97816085580893e-08
2195 1.97290626673396e-08
2196 1.97769645617907e-08
2197 2.00225301170143e-08
2198 1.96760742068314e-08
2199 1.98215268112367e-08
2200 1.97091214175771e-08
2201 1.97403981774258e-08
2202 1.97043374037253e-08
2203 1.96908882819091e-08
2204 1.98024967703603e-08
2205 1.95832449558786e-08
2206 1.95473057789108e-08
2207 1.99042813022743e-08
2208 2.00339110434378e-08
2209 1.94843420033841e-08
2210 2.03225907988713e-08
2211 1.93643077077699e-08
2212 2.01052552488701e-08
2213 1.96875866786961e-08
2214 1.95392798638672e-08
2215 1.97346832930823e-08
2216 2.01251853657602e-08
2217 1.96396726830361e-08
2218 1.97976378975007e-08
2219 1.93840411266688e-08
2220 1.98175591048511e-08
2221 1.97619175867914e-08
2222 1.95152470952031e-08
2223 1.96167355444032e-08
2224 1.96758683990161e-08
2225 1.95586814828452e-08
2226 1.92195278152507e-08
2227 1.93835202200177e-08
2228 1.95243636993414e-08
2229 1.94923662479862e-08
2230 1.95959563731307e-08
2231 1.9480695258256e-08
2232 1.95381839063202e-08
2233 1.98101065904677e-08
2234 1.95228535557757e-08
2235 1.93792279290506e-08
2236 1.92267371361687e-08
2237 1.98558587516384e-08
2238 1.94638683266302e-08
2239 1.91170541774532e-08
2240 1.91112744496458e-08
2241 1.95661397115465e-08
2242 1.91930601478596e-08
2243 1.92763846245114e-08
2244 1.96587544296634e-08
2245 1.94492178189343e-08
2246 1.92604794257978e-08
2247 1.92512262626732e-08
2248 1.91991181739404e-08
2249 1.92415494570941e-08
2250 1.91691796955773e-08
2251 1.91044159740805e-08
2252 1.94094954781843e-08
2253 1.90144315450613e-08
2254 1.91286069544105e-08
2255 1.95506282947289e-08
2256 1.90811945746105e-08
2257 1.90553437449115e-08
2258 1.92911268241947e-08
2259 1.95897083323793e-08
2260 1.94867551028821e-08
2261 1.93088127387053e-08
2262 1.9166587465147e-08
2263 1.88763530466485e-08
2264 1.89933752581251e-08
2265 1.89515436213039e-08
2266 1.9200197488578e-08
2267 1.89558797321077e-08
2268 1.95598275267717e-08
2269 1.91398654989516e-08
2270 1.93169506856883e-08
2271 1.92537570506968e-08
2272 1.90055643853881e-08
2273 1.94923393925794e-08
2274 1.88910285041644e-08
2275 1.93004324711143e-08
2276 1.90596130733578e-08
2277 1.89436340833904e-08
2278 1.90903537675702e-08
2279 1.90479876649441e-08
2280 1.96236722667109e-08
2281 1.88406647971817e-08
2282 1.9374209436096e-08
2283 1.89678513167202e-08
2284 1.89202677687295e-08
2285 1.89874890019759e-08
2286 1.89248890587468e-08
2287 1.87985007535918e-08
2288 1.89759610682572e-08
2289 1.91566846241553e-08
2290 1.93767623724028e-08
2291 1.87469946262286e-08
2292 1.91795490582791e-08
2293 1.88255360056289e-08
2294 1.92845159743271e-08
2295 1.89576628355503e-08
2296 1.89444643412529e-08
2297 1.86804328767831e-08
2298 1.87705313055009e-08
2299 1.87409571332786e-08
2300 1.88614940321852e-08
2301 1.88212138370059e-08
2302 1.89356390609685e-08
2303 1.91064999428558e-08
2304 1.88877678766186e-08
2305 1.89666915664244e-08
2306 1.87139388183599e-08
2307 1.88976748967118e-08
2308 1.90677952998097e-08
2309 1.88674909940012e-08
2310 1.90908141439738e-08
2311 1.86621517175301e-08
2312 1.900495015561e-08
2313 1.87840746992407e-08
2314 1.85797852312319e-08
2315 1.87178941919441e-08
2316 1.86652538718413e-08
2317 1.90406257360998e-08
2318 1.87590522822134e-08
2319 1.90529806531892e-08
2320 1.91897628507576e-08
2321 1.87160900819716e-08
2322 1.86835460449508e-08
2323 1.87978157284441e-08
2324 1.89564070471882e-08
2325 1.89373082180033e-08
2326 1.89218414849979e-08
2327 1.84755336423148e-08
2328 1.85475461005336e-08
2329 1.8800248172246e-08
2330 1.86165314632891e-08
2331 1.88761392758696e-08
2332 1.87689149531156e-08
2333 1.84021471820728e-08
2334 1.87538754952143e-08
2335 1.8810093822319e-08
2336 1.89122490215077e-08
2337 1.86171595073503e-08
2338 1.84151250313125e-08
2339 1.86423327324103e-08
2340 1.85609352114646e-08
2341 1.898263896849e-08
2342 1.89943804642656e-08
2343 1.87395865007822e-08
2344 1.86339028922511e-08
2345 1.89798085914461e-08
2346 1.83525732528889e-08
2347 1.8701470245519e-08
2348 1.88887670218296e-08
2349 1.8521999645138e-08
2350 1.8312542867216e-08
2351 1.89870226046107e-08
2352 1.8341594698823e-08
2353 1.8692933553055e-08
2354 1.85081066033099e-08
2355 1.8401296452808e-08
2356 1.86466907099447e-08
2357 1.85339731446632e-08
2358 1.89263475007806e-08
2359 1.87447523924966e-08
2360 1.84979867114965e-08
2361 1.83079731790325e-08
2362 1.89897032958353e-08
2363 1.88532550939868e-08
2364 1.87200249428532e-08
2365 1.88425983627116e-08
2366 1.85156413645693e-08
2367 1.88243121008735e-08
2368 1.81906045844116e-08
2369 1.87005590077671e-08
2370 1.80933503142189e-08
2371 1.90262213093373e-08
2372 1.82523456995209e-08
2373 1.90345922241875e-08
2374 1.79940517612831e-08
2375 1.81962712433315e-08
2376 1.90402639432818e-08
2377 1.84648256285058e-08
2378 1.88436246364443e-08
2379 1.84876005868695e-08
2380 1.85066920197485e-08
2381 1.86399244987978e-08
2382 1.85425433567588e-08
2383 1.89970107729032e-08
2384 1.86736731888626e-08
2385 1.82198056934979e-08
2386 1.84451713769462e-08
2387 1.88117157129408e-08
2388 1.82903140746316e-08
2389 1.84393486635237e-08
2390 1.83462221823305e-08
2391 1.82706276210531e-08
2392 1.84474338593699e-08
2393 1.80577823833961e-08
2394 1.823673548218e-08
2395 1.8061670024494e-08
2396 1.86410614033683e-08
2397 1.8271929394853e-08
2398 1.83929333232857e-08
2399 1.8313447395224e-08
2400 1.82963550126036e-08
2401 1.85258290243961e-08
2402 1.85121435916091e-08
2403 1.79782782310234e-08
2404 1.87206487591851e-08
2405 1.8342616085798e-08
2406 1.82142199087743e-08
2407 1.80135358931466e-08
2408 1.83073096158282e-08
2409 1.82044916670065e-08
2410 1.8039286078686e-08
2411 1.82480505557248e-08
2412 1.80648150029938e-08
2413 1.88210137521683e-08
2414 1.78913182853346e-08
2415 1.81002876669112e-08
2416 1.85645720660599e-08
2417 1.80310507582604e-08
2418 1.81843009574578e-08
2419 1.8250714355128e-08
2420 1.83566558964454e-08
2421 1.80583868174544e-08
2422 1.83721807049952e-08
2423 1.83853011042068e-08
2424 1.78916268493978e-08
2425 1.80220701058165e-08
2426 1.80955656792214e-08
2427 1.78619698929516e-08
2428 1.79940330620187e-08
2429 1.80393148674352e-08
2430 1.79071599211156e-08
2431 1.79361884304896e-08
2432 1.82428640091992e-08
2433 1.79611105994848e-08
2434 1.81507056862262e-08
2435 1.83561203506155e-08
2436 1.78523512166517e-08
2437 1.78939581703119e-08
2438 1.7932268956633e-08
2439 1.79626036111991e-08
2440 1.79656325793864e-08
2441 1.82202988303626e-08
2442 1.77025262377839e-08
2443 1.78247507978124e-08
2444 1.79070515742286e-08
2445 1.78804563393253e-08
2446 1.78709021330103e-08
2447 1.78222960889318e-08
2448 1.82677588318469e-08
2449 1.78072352761127e-08
2450 1.78083397717188e-08
2451 1.78357737936619e-08
2452 1.7733662584396e-08
2453 1.78369621151031e-08
2454 1.81921874775615e-08
2455 1.80129046103428e-08
2456 1.73870041422219e-08
2457 1.78152568095502e-08
2458 1.7719189612686e-08
2459 1.77993092422835e-08
2460 1.77894833721659e-08
2461 1.74286095966369e-08
2462 1.75336487653333e-08
2463 1.76273521805825e-08
2464 1.76909552538351e-08
2465 1.76521678632025e-08
2466 1.75233280914622e-08
2467 1.76260764033209e-08
2468 1.75034569027677e-08
2469 1.76619374403497e-08
2470 1.76306363419698e-08
2471 1.74895572742084e-08
2472 1.7478824213546e-08
2473 1.80192008043534e-08
2474 1.74286154823733e-08
2475 1.74435842517084e-08
2476 1.75984178352184e-08
2477 1.78268733683762e-08
2478 1.74219431605671e-08
2479 1.75677423450704e-08
2480 1.75285336416753e-08
2481 1.7518212588552e-08
2482 1.76798646742693e-08
2483 1.73656149724533e-08
2484 1.74868674573947e-08
2485 1.74401413373104e-08
2486 1.76521084396253e-08
2487 1.73791941038282e-08
2488 1.75401213744486e-08
2489 1.75498402747998e-08
2490 1.73503958229571e-08
2491 1.74706390072554e-08
2492 1.73547720843104e-08
2493 1.7359947953377e-08
2494 1.74450575780583e-08
2495 1.7351704322266e-08
2496 1.77361788160635e-08
2497 1.73896305069032e-08
2498 1.75352552869157e-08
2499 1.73314632476718e-08
2500 1.74157360051108e-08
2501 1.73333573072476e-08
2502 1.74170291968867e-08
2503 1.72920759300421e-08
2504 1.73116748434232e-08
2505 1.73558632794446e-08
2506 1.72506325779587e-08
2507 1.72286062021243e-08
2508 1.74141592583776e-08
2509 1.71609552523933e-08
2510 1.7412173514586e-08
2511 1.72918832761493e-08
2512 1.7368417840613e-08
2513 1.72781721663462e-08
2514 1.75275035052547e-08
2515 1.72547462773398e-08
2516 1.72112954108794e-08
2517 1.76763442334416e-08
2518 1.7161193619053e-08
2519 1.7149771910141e-08
2520 1.7321521503133e-08
2521 1.7224095180568e-08
2522 1.76499992037282e-08
2523 1.73628966317896e-08
2524 1.7212159211466e-08
2525 1.70812718467417e-08
2526 1.72802001452421e-08
2527 1.75373192798922e-08
2528 1.73448334057635e-08
2529 1.75325033355822e-08
2530 1.69154154687767e-08
2531 1.73052489262915e-08
2532 1.72127717164017e-08
2533 1.72762865757736e-08
2534 1.71234992560887e-08
2535 1.72037051866525e-08
2536 1.72227574992689e-08
2537 1.71784154612276e-08
2538 1.72287582222985e-08
2539 1.71540684346105e-08
2540 1.71580401702176e-08
2541 1.71026515916051e-08
2542 1.71716151454726e-08
2543 1.72028925207179e-08
2544 1.75040566019469e-08
2545 1.70064121174374e-08
2546 1.71672216304763e-08
2547 1.72303121921402e-08
2548 1.70829131209693e-08
2549 1.7119413356026e-08
2550 1.72043918726938e-08
2551 1.70479546146574e-08
2552 1.71217061994966e-08
2553 1.75199146059679e-08
2554 1.70003770039173e-08
2555 1.70869566660237e-08
2556 1.71090864875545e-08
2557 1.70490209885354e-08
2558 1.75434095484039e-08
2559 1.70204277097907e-08
2560 1.70693198191518e-08
2561 1.70732893920444e-08
2562 1.69848094775382e-08
2563 1.69170926898055e-08
2564 1.72599553798225e-08
2565 1.69035269852458e-08
2566 1.72974700149986e-08
2567 1.72377045759564e-08
2568 1.74020242407202e-08
2569 1.70859814949686e-08
2570 1.70906438781149e-08
2571 1.69714077584704e-08
2572 1.70650182196308e-08
2573 1.69858550362889e-08
2574 1.75580012817811e-08
2575 1.68284381851791e-08
2576 1.72067583197322e-08
2577 1.7206748492038e-08
2578 1.70497603526609e-08
2579 1.71148560015322e-08
2580 1.69814785013767e-08
2581 1.73402780694332e-08
2582 1.67254618170798e-08
2583 1.70512141033363e-08
2584 1.75246798057582e-08
2585 1.67362916809743e-08
2586 1.73252246638e-08
2587 1.69302387342807e-08
2588 1.68468214722939e-08
2589 1.70929645135764e-08
2590 1.69451155311862e-08
2591 1.70608463043731e-08
2592 1.68739537338425e-08
2593 1.69871587909487e-08
2594 1.73208850542395e-08
2595 1.6887631194562e-08
2596 1.68908818467539e-08
2597 1.68425615654222e-08
2598 1.70171483280246e-08
2599 1.70185259145139e-08
2600 1.68838278173933e-08
2601 1.6819163873194e-08
2602 1.69929850781791e-08
2603 1.68623744782881e-08
2604 1.68872854247049e-08
2605 1.69810452117503e-08
2606 1.6990869231126e-08
2607 1.69876719635642e-08
2608 1.69019583431407e-08
2609 1.73598393666818e-08
2610 1.68885702451771e-08
2611 1.70093909950264e-08
2612 1.71500837422567e-08
2613 1.67884524904238e-08
2614 1.683694853849e-08
2615 1.6827801683883e-08
2616 1.68408195750924e-08
2617 1.68605274735167e-08
2618 1.68054821281238e-08
2619 1.6856910155072e-08
2620 1.69486181542666e-08
2621 1.672277191056e-08
2622 1.731093763091e-08
2623 1.67461745934627e-08
2624 1.66941445294277e-08
2625 1.68025309219555e-08
2626 1.67977005232078e-08
2627 1.68062682561754e-08
2628 1.68773278637069e-08
2629 1.69378903536188e-08
2630 1.68459291522982e-08
2631 1.68067632879687e-08
2632 1.68344030702094e-08
2633 1.72988242670424e-08
2634 1.69735068782373e-08
2635 1.67832226320996e-08
2636 1.72552780295376e-08
2637 1.65682832979908e-08
2638 1.68911512858916e-08
2639 1.68414867653954e-08
2640 1.6674845026099e-08
2641 1.68149085228997e-08
2642 1.68314408965919e-08
2643 1.67882118993212e-08
2644 1.66867251909508e-08
2645 1.67552526928461e-08
2646 1.67503898071963e-08
2647 1.68196742389437e-08
2648 1.67809109448136e-08
2649 1.69369016784771e-08
2650 1.66673947783469e-08
2651 1.71070229235681e-08
2652 1.66858462933295e-08
2653 1.6721149785015e-08
2654 1.67169577067749e-08
2655 1.66663827523283e-08
2656 1.67117415914575e-08
2657 1.67548881475632e-08
2658 1.68027151330463e-08
2659 1.67960257477784e-08
2660 1.65746527041044e-08
2661 1.67118116343179e-08
2662 1.66733306299349e-08
2663 1.66785086628263e-08
2664 1.67168115161598e-08
2665 1.66662657374861e-08
2666 1.67615080761685e-08
2667 1.70961662597779e-08
2668 1.65105955167721e-08
2669 1.66883268448714e-08
2670 1.663810821384e-08
2671 1.67183459987008e-08
2672 1.67091530125507e-08
2673 1.66060111916888e-08
2674 1.66705349080587e-08
2675 1.69058925598708e-08
2676 1.65161774376976e-08
2677 1.66650983528527e-08
2678 1.66491732076501e-08
2679 1.67095460850142e-08
2680 1.65650944699003e-08
2681 1.6666949997024e-08
2682 1.67247232145673e-08
2683 1.65604009942921e-08
2684 1.7147898463854e-08
2685 1.63923120601872e-08
2686 1.65876805990894e-08
2687 1.66852911982485e-08
2688 1.65463786934161e-08
2689 1.65446657376389e-08
2690 1.66621663029165e-08
2691 1.66914951584829e-08
2692 1.66785152104776e-08
2693 1.71033724627456e-08
2694 1.68671790361508e-08
2695 1.63827613044454e-08
2696 1.65778723681953e-08
2697 1.65245760084254e-08
2698 1.65698588412422e-08
2699 1.64990955322519e-08
2700 1.69735710531249e-08
2701 1.63894744293902e-08
2702 1.6507133953958e-08
2703 1.65099004765246e-08
2704 1.64956992618759e-08
2705 1.65443371378249e-08
2706 1.68203405268574e-08
2707 1.65467062009927e-08
2708 1.6462312563803e-08
2709 1.68542823424378e-08
2710 1.64305919996277e-08
2711 1.65970827923356e-08
2712 1.70403061909852e-08
2713 1.63269882249573e-08
2714 1.64537319184976e-08
2715 1.64763636529841e-08
2716 1.64450958344275e-08
2717 1.70289317951156e-08
2718 1.67212488095814e-08
2719 1.63531601624634e-08
2720 1.64670149687129e-08
2721 1.64604117791622e-08
2722 1.64695932465264e-08
2723 1.64695758846367e-08
2724 1.65022150506733e-08
2725 1.65006637964371e-08
2726 1.65116734081039e-08
2727 1.68173986347941e-08
2728 1.67512183617546e-08
2729 1.6714788072747e-08
2730 1.64405862217443e-08
2731 1.64672568381263e-08
2732 1.65219694927554e-08
2733 1.65356496335534e-08
2734 1.64665856661195e-08
2735 1.6723493714732e-08
2736 1.67475212968782e-08
2737 1.64081392071136e-08
2738 1.63759429687715e-08
2739 1.64095856840518e-08
2740 1.63836800084471e-08
2741 1.63162998609856e-08
2742 1.64880010016244e-08
2743 1.66152194864821e-08
2744 1.65232725213293e-08
2745 1.64040149626121e-08
2746 1.64502719726123e-08
2747 1.64550710004541e-08
2748 1.67369591093092e-08
2749 1.66635051821462e-08
2750 1.63759238489547e-08
2751 1.63353409787792e-08
2752 1.64622076503917e-08
2753 1.64269505029679e-08
2754 1.65170474906162e-08
2755 1.64545242415937e-08
2756 1.6375801739521e-08
2757 1.63794464973499e-08
2758 1.68233646689053e-08
2759 1.63204796594929e-08
2760 1.63167278801613e-08
2761 1.63658056118177e-08
2762 1.64068236012671e-08
2763 1.68336601364949e-08
2764 1.62572024573482e-08
2765 1.64061647953684e-08
2766 1.63285533236746e-08
2767 1.64252942165266e-08
2768 1.65312766231551e-08
2769 1.67226862115566e-08
2770 1.66404445758328e-08
2771 1.63058135347338e-08
2772 1.62986680944943e-08
2773 1.67907081451979e-08
2774 1.67660038292716e-08
2775 1.67081080424403e-08
2776 1.62921366329005e-08
2777 1.64469860850502e-08
2778 1.61987302325084e-08
2779 1.6682803938739e-08
2780 1.61897289872037e-08
2781 1.63668709594056e-08
2782 1.62048352561062e-08
2783 1.63031299336147e-08
2784 1.62761777040554e-08
2785 1.62710747249051e-08
2786 1.66025361567623e-08
2787 1.62378110668548e-08
2788 1.63341808765427e-08
2789 1.62428789935998e-08
2790 1.63601794358481e-08
2791 1.62635614389117e-08
2792 1.65257646984607e-08
2793 1.67559817632057e-08
2794 1.60796645938266e-08
2795 1.66913931529677e-08
2796 1.61397482565384e-08
2797 1.6294754620727e-08
2798 1.62314662399687e-08
2799 1.62079695831263e-08
2800 1.63577023988282e-08
2801 1.63867749447544e-08
2802 1.65912921914479e-08
2803 1.62043785241206e-08
2804 1.63420889218724e-08
2805 1.62551796847321e-08
2806 1.62922866198123e-08
2807 1.62526674387653e-08
2808 1.65788964388014e-08
2809 1.61684335109324e-08
2810 1.62936598855357e-08
2811 1.62561772394376e-08
2812 1.62948018886944e-08
2813 1.62742659144399e-08
2814 1.62043981550841e-08
2815 1.63017524179576e-08
2816 1.65938810001709e-08
2817 1.61296454650461e-08
2818 1.61996362726402e-08
2819 1.6352634137462e-08
2820 1.61667457323489e-08
2821 1.62759135631241e-08
2822 1.64319076132458e-08
2823 1.63715803389586e-08
2824 1.62984420164491e-08
2825 1.63146432963224e-08
2826 1.6675664419985e-08
2827 1.6222710097713e-08
2828 1.64883085518319e-08
2829 1.61625859591563e-08
2830 1.61906930755684e-08
2831 1.61951319204157e-08
2832 1.61871309998229e-08
2833 1.64584163158121e-08
2834 1.61208710571792e-08
2835 1.65817516888378e-08
2836 1.64368880342103e-08
2837 1.59675175954721e-08
2838 1.62429322243529e-08
2839 1.61351854885972e-08
2840 1.6184824281007e-08
2841 1.61502991944396e-08
2842 1.62238586121077e-08
2843 1.6115593479693e-08
2844 1.66895828233038e-08
2845 1.62201648914451e-08
2846 1.59936510961689e-08
2847 1.61551334407761e-08
2848 1.61465573040864e-08
2849 1.63188597583908e-08
2850 1.64403541191849e-08
2851 1.63894670763831e-08
2852 1.64526909707252e-08
2853 1.61493304828841e-08
2854 1.61985732356484e-08
2855 1.65140238690409e-08
2856 1.62571888628893e-08
2857 1.61545192456369e-08
2858 1.6058546179698e-08
2859 1.61292957276959e-08
2860 1.61287508473329e-08
2861 1.61968615404184e-08
2862 1.62989636987021e-08
2863 1.60065487253558e-08
2864 1.61321462107677e-08
2865 1.61245118879272e-08
2866 1.61241435947534e-08
2867 1.60415255230362e-08
2868 1.61545855887901e-08
2869 1.65095099253776e-08
2870 1.60356169489617e-08
2871 1.65251938233357e-08
2872 1.57981769128845e-08
2873 1.61310511572754e-08
2874 1.61140599335141e-08
2875 1.6272545587448e-08
2876 1.63453015169512e-08
2877 1.5979745824346e-08
2878 1.61043688435836e-08
2879 1.61563174130297e-08
2880 1.60789677996487e-08
2881 1.64468859127354e-08
2882 1.5948309751046e-08
2883 1.59841528664106e-08
2884 1.60580976187319e-08
2885 1.60531734598734e-08
2886 1.60167595983918e-08
2887 1.63654115294953e-08
2888 1.61044359441309e-08
2889 1.60251752794327e-08
2890 1.59791769185347e-08
2891 1.60459645508482e-08
2892 1.59271053010723e-08
2893 1.62632089273362e-08
2894 1.647987110176e-08
2895 1.57652295380561e-08
2896 1.59807437642989e-08
2897 1.5984763726884e-08
2898 1.60327299534302e-08
2899 1.63852777605999e-08
2900 1.58339510802907e-08
2901 1.59709048703771e-08
2902 1.59432618103139e-08
2903 1.60393850441309e-08
2904 1.60587796880218e-08
2905 1.59228329694727e-08
2906 1.63188239059586e-08
2907 1.62630726194823e-08
2908 1.58265761496246e-08
2909 1.59330876248198e-08
2910 1.59828097632264e-08
2911 1.61668382592239e-08
2912 1.63233623444725e-08
2913 1.61357690111608e-08
2914 1.61927661994632e-08
2915 1.62177980407296e-08
2916 1.61902408239989e-08
2917 1.6101105394517e-08
2918 1.62416744076133e-08
2919 1.62248070365578e-08
2920 1.62093084614678e-08
2921 1.60120944037168e-08
2922 1.60859587796658e-08
2923 1.59976049016741e-08
2924 1.60300163678517e-08
2925 1.58852414260391e-08
2926 1.62259765785766e-08
2927 1.60445601999903e-08
2928 1.61802237181874e-08
2929 1.61204703092999e-08
2930 1.61771609235029e-08
2931 1.6161047851071e-08
2932 1.62064198085954e-08
2933 1.61019860847045e-08
2934 1.62762442110775e-08
2935 1.6169142225797e-08
2936 1.61793063422344e-08
2937 1.60064247416436e-08
2938 1.60779539086775e-08
2939 1.63141051341409e-08
2940 1.60995204578906e-08
2941 1.60642721656057e-08
2942 1.61247602430414e-08
2943 1.59762153744136e-08
2944 1.60272824409891e-08
2945 1.6156383816357e-08
2946 1.60545123788491e-08
2947 1.62255713778148e-08
2948 1.60447627111093e-08
2949 1.61789163006798e-08
2950 1.6132468235952e-08
2951 1.61870987367418e-08
2952 1.61145345365377e-08
2953 1.62262168830196e-08
2954 1.56273975855381e-08
2955 1.59348107149349e-08
2956 1.58872274540478e-08
2957 1.59579722571745e-08
2958 1.58292455805231e-08
2959 1.61836966694473e-08
2960 1.61496402517614e-08
2961 1.61543197709779e-08
2962 1.59989006229821e-08
2963 1.58564874519485e-08
2964 1.62623134241091e-08
2965 1.60143048293371e-08
2966 1.58322105676678e-08
2967 1.61724398528573e-08
2968 1.60438989882383e-08
2969 1.59174570211196e-08
2970 1.60543592728768e-08
2971 1.59587963830532e-08
2972 1.61308638542135e-08
2973 1.59387769975705e-08
2974 1.61467450487951e-08
2975 1.61235196554088e-08
2976 1.59650468503081e-08
2977 1.60363566321653e-08
2978 1.61807352248022e-08
2979 1.60202671475318e-08
2980 1.62128230676295e-08
2981 1.58526126872172e-08
2982 1.60156271813428e-08
2983 1.60346230466679e-08
2984 1.60494697203362e-08
2985 1.58844641646727e-08
2986 1.6053013404127e-08
2987 1.59380126578679e-08
2988 1.57769568613908e-08
2989 1.60682179735083e-08
2990 1.59533573671489e-08
2991 1.60167636162889e-08
2992 1.57847443045256e-08
2993 1.56536114455097e-08
2994 1.59060098128272e-08
2995 1.58694335044007e-08
2996 1.61236617626237e-08
2997 1.59763246569966e-08
2998 1.58449872647015e-08
2999 1.60266348518956e-08
3000 7.56874798681084e-09
3001 7.58259784133797e-09
3002 7.65887761330908e-09
3003 7.69577613995986e-09
3004 7.70949948816768e-09
3005 7.71063406970751e-09
3006 7.70975328706625e-09
3007 7.70761984905666e-09
3008 7.7054330748505e-09
3009 7.7034081082672e-09
3010 7.70247467726803e-09
3011 7.70036605590718e-09
3012 7.6990481123751e-09
3013 7.6983339539205e-09
3014 7.69639954623091e-09
3015 7.69527901739453e-09
3016 7.6941578217582e-09
3017 7.69380450352242e-09
3018 7.69257875424256e-09
3019 7.69134871400623e-09
3020 7.69365212185957e-09
3021 7.69207675849204e-09
3022 7.69115444818058e-09
3023 7.69096364058863e-09
3024 7.68888112190336e-09
3025 7.68822855987117e-09
3026 7.68699132366879e-09
3027 7.68641221807198e-09
3028 7.68395863617877e-09
3029 7.68356780345147e-09
3030 7.68391997077455e-09
3031 7.68319718638666e-09
3032 7.68273442550149e-09
3033 7.68191937006168e-09
3034 7.68127385533668e-09
3035 7.68031830933324e-09
3036 7.67922716995995e-09
3037 7.67839355672595e-09
3038 7.67750602370665e-09
3039 7.67715991836781e-09
3040 7.6763802721036e-09
3041 7.67513259497532e-09
3042 7.67540672860778e-09
3043 7.67321697961454e-09
3044 7.67326774323007e-09
3045 7.67263380110905e-09
3046 7.67148432201381e-09
3047 7.67090708014839e-09
3048 7.67037067603282e-09
3049 7.67000959217556e-09
3050 7.66932400311782e-09
3051 7.66847319877417e-09
3052 7.66753514941376e-09
3053 7.66744000879616e-09
3054 7.66724027093102e-09
3055 7.66588964559722e-09
3056 7.665069369861e-09
3057 7.66525919185246e-09
3058 7.66392047946152e-09
3059 7.66302652807638e-09
3060 7.66248204006614e-09
3061 7.66137663907318e-09
3062 7.66096527546334e-09
3063 7.66104683425084e-09
3064 7.66028289458931e-09
3065 7.65084065409982e-09
3066 7.65269974609129e-09
3067 7.6530174974998e-09
3068 7.65174382855172e-09
3069 7.65176212608187e-09
3070 7.65111080394676e-09
3071 7.65050221152031e-09
3072 7.6499444788547e-09
3073 7.64989166301966e-09
3074 7.65062220306478e-09
3075 7.64656466148428e-09
3076 7.6496293230921e-09
3077 7.64622039597929e-09
3078 7.64913732767547e-09
3079 7.64498537800251e-09
3080 7.64875576791946e-09
3081 7.64437923453376e-09
3082 7.64341647332967e-09
3083 7.6452907474267e-09
3084 7.64120997823192e-09
3085 7.64263210989435e-09
3086 7.64321696514192e-09
3087 7.64033596309388e-09
3088 7.6382570942668e-09
3089 7.64144867948513e-09
3090 7.63983245208322e-09
3091 7.63995968289244e-09
3092 7.63817724755422e-09
3093 7.64045856327389e-09
3094 7.63509034235055e-09
3095 7.63723293459306e-09
3096 7.63925697133683e-09
3097 7.63350449489719e-09
3098 7.63692973254626e-09
3099 7.63239868029064e-09
3100 7.63064464107122e-09
3101 7.63011478085995e-09
3102 7.6296165428813e-09
3103 7.62901366682667e-09
3104 7.62796893083451e-09
3105 7.62805627851915e-09
3106 7.62779935306601e-09
3107 7.6277533811453e-09
3108 7.62663840053746e-09
3109 7.62504685047904e-09
3110 7.62585120947512e-09
3111 7.62482948252208e-09
3112 7.62368400003544e-09
3113 7.6239536382916e-09
3114 7.62303189946745e-09
3115 7.62180441335469e-09
3116 7.62235719958815e-09
3117 7.62130553466633e-09
3118 7.6205621716563e-09
3119 7.62023176359428e-09
3120 7.62004524673676e-09
3121 7.61808190433144e-09
3122 7.61900685863903e-09
3123 7.61818540853665e-09
3124 7.61689606912364e-09
3125 7.61723328740471e-09
3126 7.61668485829703e-09
3127 7.61649655381391e-09
3128 7.61453374256704e-09
3129 7.61509401586702e-09
3130 7.6144692597313e-09
3131 7.61437102267393e-09
3132 7.61287654679754e-09
3133 7.61285795169986e-09
3134 7.61236814683652e-09
3135 7.61253892178626e-09
3136 7.61018383388379e-09
3137 7.61085744524492e-09
3138 7.61034685076156e-09
3139 7.60924798245966e-09
3140 7.60922174597445e-09
3141 7.60853343015344e-09
3142 7.60844511807357e-09
3143 7.6059973614373e-09
3144 7.60705234525982e-09
3145 7.60541181640817e-09
3146 7.60558413334666e-09
3147 7.60455759307721e-09
3148 7.60559809029337e-09
3149 7.60316511508319e-09
3150 7.60367125629546e-09
3151 7.60247144385162e-09
3152 7.60439652980338e-09
3153 7.60018389470152e-09
3154 7.60160376855889e-09
3155 7.60142509265149e-09
3156 7.59973766911726e-09
3157 7.59994205851156e-09
3158 7.59946425321623e-09
3159 7.5998374019226e-09
3160 7.59770813277272e-09
3161 7.5985518426247e-09
3162 7.5966420298168e-09
3163 7.59802926400543e-09
3164 7.59498000862946e-09
3165 7.59754707868599e-09
3166 7.59425842195016e-09
3167 7.59601434896506e-09
3168 7.59333133132367e-09
3169 7.59466569738576e-09
3170 7.59392203961151e-09
3171 7.59432086916423e-09
3172 7.59198614530554e-09
3173 7.59233342673138e-09
3174 7.59185997414868e-09
3175 7.59210559525547e-09
3176 7.58937660941261e-09
3177 7.5914958299339e-09
3178 7.58838437470155e-09
3179 7.58989969573687e-09
3180 7.58801696962297e-09
3181 7.58946771087232e-09
3182 7.58633385580798e-09
3183 7.58810920006847e-09
3184 7.58648479834423e-09
3185 7.58620030841262e-09
3186 7.58618171942116e-09
3187 7.5866613076514e-09
3188 7.58447545848306e-09
3189 7.58591901056649e-09
3190 7.58355120017429e-09
3191 7.58489952229491e-09
3192 7.58264507041973e-09
3193 7.58405894829561e-09
3194 7.58194028413484e-09
3195 7.58275425000177e-09
3196 7.58073622600386e-09
3197 7.58197097222579e-09
3198 7.57876215809472e-09
3199 7.58075703147232e-09
3200 7.58011831325511e-09
3201 7.57962537414891e-09
3202 7.58048001589518e-09
3203 7.57770259707247e-09
3204 7.57976444512587e-09
3205 7.57722249128401e-09
3206 7.57972324874623e-09
3207 7.57634389436657e-09
3208 7.57751614330338e-09
3209 7.57546726348757e-09
3210 7.5775587550786e-09
3211 7.5740074243491e-09
3212 7.57605224557811e-09
3213 7.57623232772175e-09
3214 7.57663206674142e-09
3215 7.57360337610979e-09
3216 7.57418419364275e-09
3217 7.57549017521564e-09
3218 7.57267236592418e-09
3219 7.5734795775273e-09
3220 7.57019303990192e-09
3221 7.57314867053105e-09
3222 7.57124711178725e-09
3223 7.57048883795464e-09
3224 7.56849231664525e-09
3225 7.57293176031904e-09
3226 7.56963556566847e-09
3227 7.56925948922071e-09
3228 7.57091238565155e-09
3229 7.56859215012007e-09
3230 7.56824084147123e-09
3231 7.56732232234936e-09
3232 7.5696173043871e-09
3233 7.56752769304203e-09
3234 7.56764210732563e-09
3235 7.56648988808895e-09
3236 7.56610901456178e-09
3237 7.56374013333172e-09
3238 7.56489328349041e-09
3239 7.56367195420271e-09
3240 7.56369481780261e-09
3241 7.56442294227999e-09
3242 7.56200032181464e-09
3243 7.56453291328452e-09
3244 7.56176956229204e-09
3245 7.56280142311105e-09
3246 7.56208267169156e-09
3247 7.56241733193552e-09
3248 7.56056330958321e-09
3249 7.56123619435889e-09
3250 7.55961792728299e-09
3251 7.5582788890971e-09
3252 7.55978982705519e-09
3253 7.55909953051304e-09
3254 7.55958800843781e-09
3255 7.55753193754094e-09
3256 7.55898408932865e-09
3257 7.55664166302017e-09
3258 7.55890305112472e-09
3259 7.5559961200955e-09
3260 7.55637574578749e-09
3261 7.55489950579613e-09
3262 7.55648115069452e-09
3263 7.5551852460054e-09
3264 7.55541625518941e-09
3265 7.55279512315576e-09
3266 7.55420544418239e-09
3267 7.5552925471456e-09
3268 7.55388990714145e-09
3269 7.55214343656219e-09
3270 7.55407582711531e-09
3271 7.55269287219806e-09
3272 7.5497252301493e-09
3273 7.55191356741069e-09
3274 7.55016526740593e-09
3275 7.54909538322468e-09
3276 7.55179215136681e-09
3277 7.55062515570004e-09
3278 7.54820174134196e-09
3279 7.54895185525739e-09
3280 7.54778718914384e-09
3281 7.54937606350925e-09
3282 7.54780787157183e-09
3283 7.54806158476118e-09
3284 7.54858313603557e-09
3285 7.5463890729377e-09
3286 7.54613556605555e-09
3287 7.54550700762802e-09
3288 7.54569412073081e-09
3289 7.5462580186314e-09
3290 7.54386399715035e-09
3291 7.54458489163334e-09
3292 7.54361064564391e-09
3293 7.54422783003084e-09
3294 7.54253959864282e-09
3295 7.54074924105841e-09
3296 7.54103675554285e-09
3297 7.54191800036996e-09
3298 7.53943437548088e-09
3299 7.53920541471387e-09
3300 7.53918417481514e-09
3301 7.53895374100422e-09
3302 7.53727096045043e-09
3303 7.53826967395121e-09
3304 7.53631331956717e-09
3305 7.53562821231846e-09
3306 7.53496051800195e-09
3307 7.53513991849641e-09
3308 7.53379923895681e-09
3309 7.5356222236922e-09
3310 7.5343025510155e-09
3311 7.53133162267883e-09
3312 7.53313587398274e-09
3313 7.53264259201192e-09
3314 7.53082909110692e-09
3315 7.53003459433299e-09
3316 7.53196850539206e-09
3317 7.53000952496974e-09
3318 7.52847550872238e-09
3319 7.52873911177843e-09
3320 7.52790506017287e-09
3321 7.52882170276803e-09
3322 7.52768105780066e-09
3323 7.52766954187356e-09
3324 7.52816058147143e-09
3325 7.52772685364e-09
3326 7.52370439768835e-09
3327 7.52682936455695e-09
3328 7.52550871693791e-09
3329 7.52506504256289e-09
3330 7.52544599513794e-09
3331 7.52387500563279e-09
3332 7.52473249995833e-09
3333 7.52250654517672e-09
3334 7.52213909865906e-09
3335 7.52257779518262e-09
3336 7.52108948082197e-09
3337 7.51926027089955e-09
3338 7.52065805861846e-09
3339 7.52252266078557e-09
3340 7.52070126375237e-09
3341 7.51994898198771e-09
3342 7.5205137323453e-09
3343 7.51698623349672e-09
3344 7.51859739619998e-09
3345 7.5196593888327e-09
3346 7.51527374906469e-09
3347 7.51722499228724e-09
3348 7.51866748202601e-09
3349 7.51644504448645e-09
3350 7.51691613865013e-09
3351 7.51708941532092e-09
3352 7.5141473344642e-09
3353 7.51445852401056e-09
3354 7.51581073310526e-09
3355 7.51532227252771e-09
3356 7.51487209721402e-09
3357 7.51388272984821e-09
3358 7.51322929434806e-09
3359 7.51339263335704e-09
3360 7.51312840002738e-09
3361 7.51223082262609e-09
3362 7.51131426141027e-09
3363 7.51004129492805e-09
3364 7.51105219512493e-09
3365 7.51128095488607e-09
3366 7.51119053862825e-09
3367 7.50926087320081e-09
3368 7.50876686450574e-09
3369 7.50702051802166e-09
3370 7.50748582245553e-09
3371 7.50963745330124e-09
3372 7.50650766681571e-09
3373 7.50842974597821e-09
3374 7.50712441402457e-09
3375 7.5072698886014e-09
3376 7.50607326344133e-09
3377 7.50676557215169e-09
3378 7.50597947338227e-09
3379 7.50360415371909e-09
3380 7.50511691066658e-09
3381 7.50513872163627e-09
3382 7.50344282920978e-09
3383 7.50306988225868e-09
3384 7.50342606470333e-09
3385 7.50374763253125e-09
3386 7.50380705058507e-09
3387 7.49906965824554e-09
3388 7.50297723633997e-09
3389 7.50144743222325e-09
3390 7.50223783033133e-09
3391 7.49931628934997e-09
3392 7.50140069410987e-09
3393 7.5003124817008e-09
3394 7.50081255188051e-09
3395 7.49969365085312e-09
3396 7.4997281141187e-09
3397 7.49786316919798e-09
3398 7.49914773159888e-09
3399 7.49751152120948e-09
3400 7.49494441240617e-09
3401 7.49616572132128e-09
3402 7.49755527251761e-09
3403 7.49606648040024e-09
3404 7.49417755793735e-09
3405 7.49643968706026e-09
3406 7.49461127719231e-09
3407 7.49541838618306e-09
3408 7.49452998130562e-09
3409 7.49412929018645e-09
3410 7.49335301417076e-09
3411 7.49351118461972e-09
3412 7.4922075997963e-09
3413 7.49105071054523e-09
3414 7.49098988409025e-09
3415 7.4917520964668e-09
3416 7.49119624537942e-09
3417 7.4890762877533e-09
3418 7.49041021622099e-09
3419 7.48840876410584e-09
3420 7.48931639013573e-09
3421 7.48852853016402e-09
3422 7.48967642669185e-09
3423 7.48856118970576e-09
3424 7.48760873756082e-09
3425 7.48730732616498e-09
3426 7.48680725681794e-09
3427 7.48512419021519e-09
3428 7.48721854179624e-09
3429 7.48612795847814e-09
3430 7.48607469483464e-09
3431 7.48434688008226e-09
3432 7.48585684637315e-09
3433 7.48427156752052e-09
3434 7.48427748389902e-09
3435 7.48469960945575e-09
3436 7.4837889393331e-09
3437 7.48199928649051e-09
3438 7.48338756165201e-09
3439 7.48209017317758e-09
3440 7.48183137042058e-09
3441 7.48169093178763e-09
3442 7.48016510038196e-09
3443 7.48205135345237e-09
3444 7.48020738067234e-09
3445 7.48083229551444e-09
3446 7.47954406724038e-09
3447 7.48044242346313e-09
3448 7.47877213450043e-09
3449 7.47931415348568e-09
3450 7.47891054850291e-09
3451 7.47800139624943e-09
3452 7.47829385058796e-09
3453 7.47628885505725e-09
3454 7.47500950767921e-09
3455 7.476642914922e-09
3456 7.47603688222576e-09
3457 7.47670510134468e-09
3458 7.47578118212577e-09
3459 7.4748625868426e-09
3460 7.47445516147138e-09
3461 7.47432891068378e-09
3462 7.47449799737288e-09
3463 7.47396727393546e-09
3464 7.47162815836755e-09
3465 7.47310459264794e-09
3466 7.473164692684e-09
3467 7.47139844875511e-09
3468 7.4705768068617e-09
3469 7.47119827315679e-09
3470 7.47115397928266e-09
3471 7.46938192747182e-09
3472 7.46983256108558e-09
3473 7.46876549551079e-09
3474 7.47086463073754e-09
3475 7.4690439319236e-09
3476 7.4684744737763e-09
3477 7.46955171787578e-09
3478 7.46722291078417e-09
3479 7.46742795887379e-09
3480 7.46805098664205e-09
3481 7.46615131011485e-09
3482 7.46592322642403e-09
3483 7.46630487524769e-09
3484 7.46497083933817e-09
3485 7.46590656841573e-09
3486 7.46406709623004e-09
3487 7.46405341567336e-09
3488 7.46490952019396e-09
3489 7.46175327728804e-09
3490 7.46158196390789e-09
3491 7.46204393284322e-09
3492 7.46070386439812e-09
3493 7.462116501461e-09
3494 7.46144166513529e-09
3495 7.46085402944519e-09
3496 7.46102403523241e-09
3497 7.45773160118479e-09
3498 7.45909856150751e-09
3499 7.460837569917e-09
3500 7.46235300946862e-09
3501 7.4577770277362e-09
3502 7.45728757894915e-09
3503 7.45732725201953e-09
3504 7.45623910328175e-09
3505 7.45779666899704e-09
3506 7.45767296425615e-09
3507 7.45660285098038e-09
3508 7.45456477135864e-09
3509 7.45675796010037e-09
3510 7.45338006405416e-09
3511 7.45402734142475e-09
3512 7.45642547478331e-09
3513 7.45473530658347e-09
3514 7.45433303100951e-09
3515 7.45255399206557e-09
3516 7.45208224567362e-09
3517 7.45412646313559e-09
3518 7.45294166371036e-09
3519 7.45286366080067e-09
3520 7.45163268500715e-09
3521 7.45122564874134e-09
3522 7.45176170083295e-09
3523 7.44968391641621e-09
3524 7.45219053174817e-09
3525 7.44872843527755e-09
3526 7.44908379129994e-09
3527 7.44987508780048e-09
3528 7.44685480166818e-09
3529 7.44789288820158e-09
3530 7.44966668372871e-09
3531 7.44755323395241e-09
3532 7.44793687140755e-09
3533 7.44750489994894e-09
3534 7.44661577103956e-09
3535 7.4475048499334e-09
3536 7.44715139389118e-09
3537 7.44581098283126e-09
3538 7.44684189005196e-09
3539 7.44586240761236e-09
3540 7.44574895159e-09
3541 7.44466748736627e-09
3542 7.44434815297401e-09
3543 7.44456000711935e-09
3544 7.44352059012243e-09
3545 7.44336232871845e-09
3546 7.44344286082721e-09
3547 7.44339781660885e-09
3548 7.44253558204733e-09
3549 7.44287056100856e-09
3550 7.44330332011489e-09
3551 7.43919997456266e-09
3552 7.44179130166001e-09
3553 7.44068119612473e-09
3554 7.44146727638117e-09
3555 7.4406159822904e-09
3556 7.44006526337548e-09
3557 7.43972915334679e-09
3558 7.440060869085e-09
3559 7.43922501697525e-09
3560 7.43809134465256e-09
3561 7.43684899812136e-09
3562 7.43765582988942e-09
3563 7.43685971596464e-09
3564 7.43795882640619e-09
3565 7.43702185476702e-09
3566 7.43433809724525e-09
3567 7.43500599845182e-09
3568 7.4357280278603e-09
3569 7.43255043525459e-09
3570 7.4360906772708e-09
3571 7.43267350331034e-09
3572 7.43426178431705e-09
3573 7.43278535064729e-09
3574 7.43512051504247e-09
3575 7.43208367617876e-09
3576 7.433472587276e-09
3577 7.43213322335068e-09
3578 7.43279501230765e-09
3579 7.43213337867088e-09
3580 7.42867072214826e-09
3581 7.4313701184936e-09
3582 7.43197149322139e-09
3583 7.43094944327671e-09
3584 7.42983203960712e-09
3585 7.43104953332407e-09
3586 7.42988888455254e-09
3587 7.42833814726396e-09
3588 7.42881229073844e-09
3589 7.42741697157268e-09
3590 7.4277459494454e-09
3591 7.42849388180034e-09
3592 7.42633052430719e-09
3593 7.42629949590512e-09
3594 7.4275171278726e-09
3595 7.42624332664143e-09
3596 7.42746508372583e-09
3597 7.42426589145362e-09
3598 7.42690296115511e-09
3599 7.42503865683308e-09
3600 7.42324940805616e-09
3601 7.42472251016868e-09
3602 7.42278848550604e-09
3603 7.42539701850675e-09
3604 7.42356936064836e-09
3605 7.42270844525383e-09
3606 7.42265267933417e-09
3607 7.42185751317903e-09
3608 7.42210776791263e-09
3609 7.42216670063245e-09
3610 7.42189638819335e-09
3611 7.42001604625564e-09
3612 7.42173850509786e-09
3613 7.41882579213815e-09
3614 7.41992479966203e-09
3615 7.42099863121348e-09
3616 7.42015276883334e-09
3617 7.41854871186276e-09
3618 7.42009805271282e-09
3619 7.41721839384035e-09
3620 7.41829361003421e-09
3621 7.41751913987954e-09
3622 7.41747225865841e-09
3623 7.41527956302224e-09
3624 7.41761592484846e-09
3625 7.41718516383827e-09
3626 7.41590295261219e-09
3627 7.41575999632649e-09
3628 7.41503607804006e-09
3629 7.41646095669957e-09
3630 7.41345151991646e-09
3631 7.41436156018183e-09
3632 7.41277319404876e-09
3633 7.41264785406037e-09
3634 7.41111919702608e-09
3635 7.41208786556036e-09
3636 7.41323846101261e-09
3637 7.41136184698177e-09
3638 7.41329670989055e-09
3639 7.40975028537538e-09
3640 7.41031010609294e-09
3641 7.4101762150558e-09
3642 7.40865721418627e-09
3643 7.40989014075266e-09
3644 7.41012761934501e-09
3645 7.40939493787396e-09
3646 7.4077789131155e-09
3647 7.4084640588612e-09
3648 7.40988923389474e-09
3649 7.40669414561745e-09
3650 7.4061340939735e-09
3651 7.40851312766577e-09
3652 7.40849324876169e-09
3653 7.40859962269913e-09
3654 7.40553426775703e-09
3655 7.40654467820678e-09
3656 7.40661663461983e-09
3657 7.4069448936509e-09
3658 7.40505321086826e-09
3659 7.40542280744383e-09
3660 7.40440003971221e-09
3661 7.40389190989621e-09
3662 7.40409700472622e-09
3663 7.40386921593839e-09
3664 7.40402595880707e-09
3665 7.40090968537332e-09
3666 7.40377624783295e-09
3667 7.40152142772454e-09
3668 7.40288603323647e-09
3669 7.40197794646402e-09
3670 7.40063523846368e-09
3671 7.40071580962454e-09
3672 7.3996534963594e-09
3673 7.40100997551063e-09
3674 7.39908795535649e-09
3675 7.4007572462842e-09
3676 7.39657252016301e-09
3677 7.39846517933129e-09
3678 7.39947813369057e-09
3679 7.39871594115926e-09
3680 7.39837801402854e-09
3681 7.39784535011556e-09
3682 7.39728157619912e-09
3683 7.39716107700517e-09
3684 7.39467955981898e-09
3685 7.39612572739357e-09
3686 7.39724432877176e-09
3687 7.39558728771184e-09
3688 7.39595891077749e-09
3689 7.39595034560114e-09
3690 7.39374121397352e-09
3691 7.394389761467e-09
3692 7.39237308311491e-09
3693 7.39291878720794e-09
3694 7.39300846816504e-09
3695 7.39288700668528e-09
3696 7.39332778360025e-09
3697 7.39209548145103e-09
3698 7.39331393034837e-09
3699 7.39097832858082e-09
3700 7.39113950548598e-09
3701 7.39161184651338e-09
3702 7.39176664088514e-09
3703 7.38990088197666e-09
3704 7.38969897870478e-09
3705 7.39068665980835e-09
3706 7.38970765631897e-09
3707 7.38917070744471e-09
3708 7.38962961177592e-09
3709 7.38701340771697e-09
3710 7.38842221273406e-09
3711 7.38859609206166e-09
3712 7.38933902480476e-09
3713 7.38604224564954e-09
3714 7.38656351859102e-09
3715 7.38723986537515e-09
3716 7.38550587625619e-09
3717 7.38541450939767e-09
3718 7.38584328918712e-09
3719 7.38605045086382e-09
3720 7.3851034018968e-09
3721 7.38507626144025e-09
3722 7.383951279194e-09
3723 7.38525906721077e-09
3724 7.38303121547434e-09
3725 7.3837763278628e-09
3726 7.38387862334044e-09
3727 7.382074005996e-09
3728 7.38254783649861e-09
3729 7.38272518460703e-09
3730 7.38263689076257e-09
3731 7.38122637999883e-09
3732 7.38055201515708e-09
3733 7.38129115698816e-09
3734 7.38153905197048e-09
3735 7.37964315794537e-09
3736 7.38063531643962e-09
3737 7.38046578202534e-09
3738 7.37933822747561e-09
3739 7.37877749434901e-09
3740 7.37868167804967e-09
3741 7.37644999396214e-09
3742 7.37666883038335e-09
3743 7.37883217871715e-09
3744 7.37714470649964e-09
3745 7.3774133000637e-09
3746 7.37561824445954e-09
3747 7.3772181372056e-09
3748 7.37671624717606e-09
3749 7.37566393183009e-09
3750 7.37633035680085e-09
3751 7.37475190437653e-09
3752 7.37576846682186e-09
3753 7.37506904710528e-09
3754 7.37487375226942e-09
3755 7.37448066401303e-09
3756 7.37261295472025e-09
3757 7.37315913942882e-09
3758 7.37311225787463e-09
3759 7.37260925473548e-09
3760 7.37272376619136e-09
3761 7.3726597105972e-09
3762 7.37189543156824e-09
3763 7.36998626391094e-09
3764 7.37141624623794e-09
3765 7.37080966714543e-09
3766 7.37079119367268e-09
3767 7.36894646788389e-09
3768 7.3688142840922e-09
3769 7.36868244058386e-09
3770 7.36921930738488e-09
3771 7.3682983517398e-09
3772 7.36894662281551e-09
3773 7.36830514896925e-09
3774 7.36826507655164e-09
3775 7.36783833774068e-09
3776 7.3660224770844e-09
3777 7.36674853443753e-09
3778 7.36680367338649e-09
3779 7.36675172735568e-09
3780 7.36607854290305e-09
3781 7.36441976589841e-09
3782 7.36498746831127e-09
3783 7.36580073534832e-09
3784 7.36400560263917e-09
3785 7.36439012435919e-09
3786 7.36315210972394e-09
3787 7.36483717891923e-09
3788 7.36248584226606e-09
3789 7.36315493271578e-09
3790 7.36289558189718e-09
3791 7.36284469324278e-09
3792 7.36266588377554e-09
3793 7.36143488361263e-09
3794 7.36057600644879e-09
3795 7.3620407894659e-09
3796 7.36043186230773e-09
3797 7.36099524106804e-09
3798 7.36110832791348e-09
3799 7.36046174620864e-09
3800 7.3595973675844e-09
3801 7.35878676960677e-09
3802 7.35930670497931e-09
3803 7.35777801241788e-09
3804 7.35609877150489e-09
3805 7.3576020904742e-09
3806 7.35672899176976e-09
3807 7.35715466843034e-09
3808 7.35690177597337e-09
3809 7.35665048740009e-09
3810 7.3577743909814e-09
3811 7.35499266688033e-09
3812 7.35611192306806e-09
3813 7.35453865041702e-09
3814 7.35461083467559e-09
3815 7.35319517486399e-09
3816 7.35377318489272e-09
3817 7.35563362933012e-09
3818 7.3528012323465e-09
3819 7.35397481579914e-09
3820 7.35409928651065e-09
3821 7.35310595886829e-09
3822 7.3525010413078e-09
3823 7.35271736587717e-09
3824 7.35203137117169e-09
3825 7.35045210645269e-09
3826 7.35138051219386e-09
3827 7.3513904269018e-09
3828 7.35039457294695e-09
3829 7.35072414786986e-09
3830 7.35012466979157e-09
3831 7.34981960964776e-09
3832 7.34955781403479e-09
3833 7.34916198849578e-09
3834 7.34820663061964e-09
3835 7.3487411876294e-09
3836 7.34836098262304e-09
3837 7.34736400032077e-09
3838 7.34691171946911e-09
3839 7.34569398175289e-09
3840 7.34739889543534e-09
3841 7.34604284899354e-09
3842 7.34542873576105e-09
3843 7.34560953583041e-09
3844 7.345312498519e-09
3845 7.34517469760942e-09
3846 7.34477190894589e-09
3847 7.34432355570247e-09
3848 7.343488578615e-09
3849 7.34408765762362e-09
3850 7.34266129082117e-09
3851 7.34386622455729e-09
3852 7.34253047504629e-09
3853 7.34198881710557e-09
3854 7.34174211669547e-09
3855 7.34225714726922e-09
3856 7.34223487569574e-09
3857 7.34068049143555e-09
3858 7.340890266605e-09
3859 7.33957314327793e-09
3860 7.33988350495873e-09
3861 7.34052720330336e-09
3862 7.33814907230568e-09
3863 7.34009259928392e-09
3864 7.33870963715022e-09
3865 7.33908639358183e-09
3866 7.33887085885288e-09
3867 7.33741632694129e-09
3868 7.33888905149471e-09
3869 7.33582094353258e-09
3870 7.3376523619384e-09
3871 7.33695448520977e-09
3872 7.33584994272452e-09
3873 7.33688409762512e-09
3874 7.33638062677677e-09
3875 7.33494733179407e-09
3876 7.33588181475731e-09
3877 7.33611351422248e-09
3878 7.33524515089678e-09
3879 7.33376877615211e-09
3880 7.3334222012722e-09
3881 7.33441405442736e-09
3882 7.33178846734628e-09
3883 7.33346881950925e-09
3884 7.33430296523396e-09
3885 7.33200362432185e-09
3886 7.33287752310852e-09
3887 7.33290477153425e-09
3888 7.33096593075189e-09
3889 7.33161160282325e-09
3890 7.33204076050487e-09
3891 7.33093697044551e-09
3892 7.33169985028814e-09
3893 7.32997350189057e-09
3894 7.3304896586468e-09
3895 7.33009475892277e-09
3896 7.32991579591169e-09
3897 7.32846142620369e-09
3898 7.32909292652795e-09
3899 7.32935387118272e-09
3900 7.3279182550845e-09
3901 7.32842969150549e-09
3902 7.32805335543207e-09
3903 7.32658890664761e-09
3904 7.32636434375156e-09
3905 7.32599403649448e-09
3906 7.32782520126984e-09
3907 7.32522411653336e-09
3908 7.32653361382174e-09
3909 7.32593523525282e-09
3910 7.32604135189607e-09
3911 7.32477475909321e-09
3912 7.32511555867621e-09
3913 7.32475608658523e-09
3914 7.32483263568562e-09
3915 7.32320015028876e-09
3916 7.32396058655826e-09
3917 7.32355220456338e-09
3918 7.32174192125124e-09
3919 7.32290906788058e-09
3920 7.32216949406173e-09
3921 7.32219278704038e-09
3922 7.32161583769098e-09
3923 7.31943296677917e-09
3924 7.32214452206503e-09
3925 7.32188660879096e-09
3926 7.32166114977839e-09
3927 7.32000444259029e-09
3928 7.32053612831351e-09
3929 7.32055418931066e-09
3930 7.31950788676605e-09
3931 7.31983804677649e-09
3932 7.31844398335624e-09
3933 7.31939502679468e-09
3934 7.31900583317291e-09
3935 7.31820540703088e-09
3936 7.31700019343529e-09
3937 7.31731027955873e-09
3938 7.31705545989336e-09
3939 7.31739967688161e-09
3940 7.31676642415358e-09
3941 7.31631630729312e-09
3942 7.3158065395651e-09
3943 7.31565589454086e-09
3944 7.31600488804163e-09
3945 7.31364362233089e-09
3946 7.31361403141784e-09
3947 7.31269145190505e-09
3948 7.31501926309885e-09
3949 7.31459300953863e-09
3950 7.31403536816111e-09
3951 7.31372116377638e-09
3952 7.31268326345513e-09
3953 7.31191193650838e-09
3954 7.31311840462823e-09
3955 7.31288447963774e-09
3956 7.31153014094099e-09
3957 7.31156381225673e-09
3958 7.31130599168628e-09
3959 7.31048482974228e-09
3960 7.31121182054317e-09
3961 7.31021979052948e-09
3962 7.31051050703591e-09
3963 7.31123318875593e-09
3964 7.30879928204087e-09
3965 7.3097549753709e-09
3966 7.30964220202468e-09
3967 7.30773897039683e-09
3968 7.3079087997141e-09
3969 7.3086592313476e-09
3970 7.30591047837237e-09
3971 7.30658643072202e-09
3972 7.30615470379004e-09
3973 7.30755198696809e-09
3974 7.30662749665045e-09
3975 7.30602322593366e-09
3976 7.30579989124958e-09
3977 7.30523033248831e-09
3978 7.30489961478509e-09
3979 7.30559366415862e-09
3980 7.30498429038495e-09
3981 7.30433144482956e-09
3982 7.30349105787509e-09
3983 7.30446914212757e-09
3984 7.30533804049749e-09
3985 7.30372740770369e-09
3986 7.30380382624185e-09
3987 7.30155788680742e-09
3988 7.30167678950067e-09
3989 7.3024492831808e-09
3990 7.30235429086124e-09
3991 7.30105461108077e-09
3992 7.30206585061732e-09
3993 7.300731962534e-09
3994 7.2991488861418e-09
3995 7.30116163930039e-09
3996 7.29986787301184e-09
3997 7.30034212359509e-09
3998 7.2983850070707e-09
3999 7.29886765168941e-09
4000 7.29925255613106e-09
4001 7.30010772315159e-09
4002 7.29706386728979e-09
4003 7.29804563898151e-09
4004 7.29760017739722e-09
4005 7.2974491901745e-09
4006 7.29701049850817e-09
4007 7.29653288059073e-09
4008 7.295979244204e-09
4009 7.29587124123632e-09
4010 7.29644809902008e-09
4011 7.2963442434848e-09
4012 7.29514220781935e-09
4013 7.29411611988251e-09
4014 7.29451981598683e-09
4015 7.29366462731695e-09
4016 7.29492344894722e-09
4017 7.29327995488416e-09
4018 7.29354690193196e-09
4019 7.29345421129901e-09
4020 7.29260229731521e-09
4021 7.29196239149243e-09
4022 7.29343216088196e-09
4023 7.29138527610917e-09
4024 7.29169203173008e-09
4025 7.29221176659633e-09
4026 7.29179098316046e-09
4027 7.28996143495309e-09
4028 7.29160305210086e-09
4029 7.29071109098478e-09
4030 7.28999064295022e-09
4031 7.28979260811169e-09
4032 7.28895392146378e-09
4033 7.28938311869709e-09
4034 7.28925704188144e-09
4035 7.28817220480016e-09
4036 7.28764903465406e-09
4037 7.28895845186761e-09
4038 7.28896062468509e-09
4039 7.28597288890409e-09
4040 7.28695265644896e-09
4041 7.28712959746636e-09
4042 7.28615460365845e-09
4043 7.28743660749487e-09
4044 7.28450977027784e-09
4045 7.28565775826029e-09
4046 7.28522304502266e-09
4047 7.28605995217735e-09
4048 7.28458319199099e-09
4049 7.28433081809143e-09
4050 7.28378166153787e-09
4051 7.28373221134393e-09
4052 7.28416854331448e-09
4053 7.28367487648485e-09
4054 7.28268892091766e-09
4055 7.28192064888833e-09
4056 7.2835138534566e-09
4057 7.28044760711644e-09
4058 7.28182197384797e-09
4059 7.28131914909391e-09
4060 7.28028076371623e-09
4061 7.2813688088702e-09
4062 7.28126508300897e-09
4063 7.27870288577681e-09
4064 7.27972521802345e-09
4065 7.28121817000771e-09
4066 7.27844127704214e-09
4067 7.28008373049316e-09
4068 7.27879327125369e-09
4069 7.2785735498182e-09
4070 7.27937813629897e-09
4071 7.27716194817707e-09
4072 7.27794598581744e-09
4073 7.27780019188495e-09
4074 7.27635802338922e-09
4075 7.27627503971373e-09
4076 7.27756723348239e-09
4077 7.27467129815751e-09
4078 7.27519316653935e-09
4079 7.27566339031593e-09
4080 7.2753660619318e-09
4081 7.27510859477309e-09
4082 7.27533049407758e-09
4083 7.27493332070006e-09
4084 7.27394390021008e-09
4085 7.27433322758597e-09
4086 7.27209339651536e-09
4087 7.27291093846616e-09
4088 7.27511034848138e-09
4089 7.27162789404545e-09
4090 7.27171299974572e-09
4091 7.27226789662416e-09
4092 7.27250963084014e-09
4093 7.27317467749677e-09
4094 7.27304732434098e-09
4095 7.26992723579234e-09
4096 7.2698638100277e-09
4097 7.27057567623257e-09
4098 7.27156210120206e-09
4099 7.26937286890661e-09
4100 7.26995467817404e-09
4101 7.26955884855496e-09
4102 7.26808378909149e-09
4103 7.26707745096755e-09
4104 7.26867828679256e-09
4105 7.26909960416244e-09
4106 7.26825075203497e-09
4107 7.26815397200653e-09
4108 7.26706701228985e-09
4109 7.26675507289887e-09
4110 7.26780602511301e-09
4111 7.26616037602379e-09
4112 7.26396520722705e-09
4113 7.26586292637554e-09
4114 7.26675084020134e-09
4115 7.26480498614013e-09
4116 7.26546880289169e-09
4117 7.26343881382596e-09
4118 7.26611627080431e-09
4119 7.26369831724472e-09
4120 7.26428574096549e-09
4121 7.26359996008896e-09
4122 7.26295551539691e-09
4123 7.26357473529449e-09
4124 7.26374960538512e-09
4125 7.26235119080654e-09
4126 7.26161158079441e-09
4127 7.26249572433058e-09
4128 7.26274071013999e-09
4129 7.26150579635765e-09
4130 7.25972558518473e-09
4131 7.25968884288108e-09
4132 7.25959394726194e-09
4133 7.26033315126551e-09
4134 7.2603934262172e-09
4135 7.25990988398317e-09
4136 7.25927021552608e-09
4137 7.26045847684875e-09
4138 7.25931505918265e-09
4139 7.25801813469817e-09
4140 7.25780864965775e-09
4141 7.26092689426472e-09
4142 7.25822621924777e-09
4143 7.25686294447692e-09
4144 7.25684619520828e-09
4145 7.25791735217696e-09
4146 7.25546602653448e-09
4147 7.25357868106014e-09
4148 7.25542514004585e-09
4149 7.25690473257767e-09
4150 7.25485153216243e-09
4151 7.25519448080414e-09
4152 7.25423695513427e-09
4153 7.25600924439407e-09
4154 7.25413872915137e-09
4155 7.25217280456158e-09
4156 7.25380067351233e-09
4157 7.25336812892885e-09
4158 7.25354561886826e-09
4159 7.25219882519168e-09
4160 7.25333856785304e-09
4161 7.24979150834581e-09
4162 7.25147088578848e-09
4163 7.25131226650411e-09
4164 7.2514398625767e-09
4165 7.25217820757296e-09
4166 7.25104352322647e-09
4167 7.24860210574163e-09
4168 7.25039946375405e-09
4169 7.24975256460869e-09
4170 7.25004818311059e-09
4171 7.24751416286695e-09
4172 7.24894241352358e-09
4173 7.24945044380809e-09
4174 7.2493532946305e-09
4175 7.24830410578359e-09
4176 7.24609183169744e-09
4177 7.24725743181387e-09
4178 7.24686776024108e-09
4179 7.24819611622185e-09
4180 7.24778175884011e-09
4181 7.24655177178346e-09
4182 7.24613422911635e-09
4183 7.24643555585769e-09
4184 7.2446328620579e-09
4185 7.24467962648356e-09
4186 7.2460182223566e-09
4187 7.24269901908325e-09
4188 7.24511471847378e-09
4189 7.2444747666045e-09
4190 7.24317698005961e-09
4191 7.24305048821483e-09
4192 7.24280148933776e-09
4193 7.24188147713245e-09
4194 7.24148873573749e-09
4195 7.24376842439756e-09
4196 7.24160619192116e-09
4197 7.24103931479259e-09
4198 7.24075036437344e-09
4199 7.24131083834623e-09
4200 7.24177929981029e-09
4201 7.24200097518279e-09
4202 7.23928311102684e-09
4203 7.2393431826967e-09
4204 7.24047087244384e-09
4205 7.24011862859864e-09
4206 7.2406185497309e-09
4207 7.23840138766585e-09
4208 7.23630597182501e-09
4209 7.23928549575814e-09
4210 7.23757712661754e-09
4211 7.23707127583961e-09
4212 7.23649404354987e-09
4213 7.23710229322272e-09
4214 7.23796385865283e-09
4215 7.23604737554173e-09
4216 7.23630892640603e-09
4217 7.23611460054507e-09
4218 7.23703253394103e-09
4219 7.23619412543175e-09
4220 7.23559853163525e-09
4221 7.23691234438828e-09
4222 7.23432550547853e-09
4223 7.2308304912283e-09
4224 7.23371155120223e-09
4225 7.23451028833422e-09
4226 7.23345704450873e-09
4227 7.23519853260135e-09
4228 7.23309125238991e-09
4229 7.23392586693405e-09
4230 7.23385362166873e-09
4231 7.23296555821262e-09
4232 7.23197673563325e-09
4233 7.23028528792913e-09
4234 7.23251378020873e-09
4235 7.23186305878754e-09
4236 7.23178188816176e-09
4237 7.23122971668655e-09
4238 7.23090540066806e-09
4239 7.23076466019323e-09
4240 7.23055973012032e-09
4241 7.22864971414161e-09
4242 7.2295535573641e-09
4243 7.23083302120453e-09
4244 7.22894137411556e-09
4245 7.22811459172301e-09
4246 7.22800724772821e-09
4247 7.22534179128931e-09
4248 7.22915339174124e-09
4249 7.22770463582045e-09
4250 7.22837303304691e-09
4251 7.22760376214993e-09
4252 7.22600668570661e-09
4253 7.22672030678551e-09
4254 7.22541884995431e-09
4255 7.22738500122389e-09
4256 7.22535403466229e-09
4257 7.2251463341666e-09
4258 7.22415951506794e-09
4259 7.22599814259595e-09
4260 7.22431221059083e-09
4261 7.22591410270979e-09
4262 7.22494685886921e-09
4263 7.22408396205965e-09
4264 7.22446479287098e-09
4265 7.22440420636294e-09
4266 7.22256795895881e-09
4267 7.2225374052326e-09
4268 7.22086443166736e-09
4269 7.22284742504797e-09
4270 7.2203508829638e-09
4271 7.22217532347913e-09
4272 7.22234858202553e-09
4273 7.22220195664702e-09
4274 7.22234872815863e-09
4275 7.21995440428058e-09
4276 7.21910215795152e-09
4277 7.21936873451789e-09
4278 7.21961722507936e-09
4279 7.22115782225696e-09
4280 7.21970039915809e-09
4281 7.21791973623542e-09
4282 7.21782337134713e-09
4283 7.21835922248837e-09
4284 7.21830609026752e-09
4285 7.21687687310624e-09
4286 7.21690838287326e-09
4287 7.21736257758288e-09
4288 7.21572837186768e-09
4289 7.21712531720486e-09
4290 7.21817414806036e-09
4291 7.21545360976572e-09
4292 7.21667005582072e-09
4293 7.2144725872525e-09
4294 7.215532217969e-09
4295 7.21498702355161e-09
4296 7.21127645716968e-09
4297 7.21407810599928e-09
4298 7.21407654935557e-09
4299 7.21406274678538e-09
4300 7.21575820267217e-09
4301 7.21479512461043e-09
4302 7.21503973466486e-09
4303 7.21472339959561e-09
4304 7.21166642189686e-09
4305 7.21067712133872e-09
4306 7.21372317827318e-09
4307 7.21209560700098e-09
4308 7.21236194151098e-09
4309 7.21289106866974e-09
4310 7.21070683848413e-09
4311 7.20754595376349e-09
4312 7.21100863304569e-09
4313 7.21097756781197e-09
4314 7.21160956296263e-09
4315 7.21114353482988e-09
4316 7.20928062303283e-09
4317 7.20780608079452e-09
4318 7.21113491677916e-09
4319 7.21047174900846e-09
4320 7.21037946860292e-09
4321 7.20864496728257e-09
4322 7.20868313530687e-09
4323 7.20674638016172e-09
4324 7.20752146390891e-09
4325 7.20695994316234e-09
4326 7.20659279399016e-09
4327 7.20670575171667e-09
4328 7.20560935177361e-09
4329 7.20550802676967e-09
4330 7.20691055736133e-09
4331 7.20599891460005e-09
4332 7.20565664830142e-09
4333 7.20756813613055e-09
4334 7.20543744983648e-09
4335 7.20551431021565e-09
4336 7.20588756195073e-09
4337 7.20420795286003e-09
4338 7.20405547771641e-09
4339 7.20479805235708e-09
4340 7.20246613436504e-09
4341 7.20118708461004e-09
4342 7.20271141879669e-09
4343 7.20432108189395e-09
4344 7.20065976322903e-09
4345 7.20183344310454e-09
4346 7.20280040181209e-09
4347 7.20192252517959e-09
4348 7.20027960179892e-09
4349 7.19940798551977e-09
4350 7.20112533361683e-09
4351 7.1995636508615e-09
4352 7.19949871824666e-09
4353 7.20096856737795e-09
4354 7.19897066739361e-09
4355 7.20007598584149e-09
4356 7.19813318039719e-09
4357 7.19740550855774e-09
4358 7.19923450781534e-09
4359 7.19921316780225e-09
4360 7.19688774974969e-09
4361 7.19650827191165e-09
4362 7.19777926619369e-09
4363 7.19911143368113e-09
4364 7.19762183076789e-09
4365 7.19539650120837e-09
4366 7.19630158357853e-09
4367 7.19660219183904e-09
4368 7.19659444814447e-09
4369 7.1946055831551e-09
4370 7.19497166648542e-09
4371 7.19469712906484e-09
4372 7.1934964872844e-09
4373 7.19420025380169e-09
4374 7.19398114579217e-09
4375 7.19580389432761e-09
4376 7.19295459836178e-09
4377 7.19396190954069e-09
4378 7.19268868648593e-09
4379 7.19181028852045e-09
4380 7.19245813354807e-09
4381 7.1923251505901e-09
4382 7.19062156021022e-09
4383 7.19101603352534e-09
4384 7.19194006060242e-09
4385 7.19167925766762e-09
4386 7.19172932442391e-09
4387 7.19239499569202e-09
4388 7.19081891473183e-09
4389 7.18884268455233e-09
4390 7.18983870842682e-09
4391 7.19021029790823e-09
4392 7.19019408304544e-09
4393 7.1872985013266e-09
4394 7.18978248309687e-09
4395 7.19014436995069e-09
4396 7.18751610528923e-09
4397 7.18725593443215e-09
4398 7.18716051154034e-09
4399 7.18760850498823e-09
4400 7.18650828299139e-09
4401 7.18471326385806e-09
4402 7.18798060811432e-09
4403 7.18678789202176e-09
4404 7.18586567410862e-09
4405 7.18544941810673e-09
4406 7.18520368547804e-09
4407 7.18638696964313e-09
4408 7.18589527221036e-09
4409 7.183539165595e-09
4410 7.18573826352653e-09
4411 7.18573294691827e-09
4412 7.18352571568714e-09
4413 7.18260804108417e-09
4414 7.18501213428646e-09
4415 7.18317617459663e-09
4416 7.18165486249256e-09
4417 7.18245819933405e-09
4418 7.18336668922848e-09
4419 7.18320932491245e-09
4420 7.18216910589042e-09
4421 7.1800497796759e-09
4422 7.18303967661704e-09
4423 7.18139890309999e-09
4424 7.18137725447265e-09
4425 7.18196912780078e-09
4426 7.18116068734731e-09
4427 7.17860798568615e-09
4428 7.17952573664471e-09
4429 7.17885817752562e-09
4430 7.17959257323653e-09
4431 7.18004251418214e-09
4432 7.17987631118633e-09
4433 7.18039349384414e-09
4434 7.17661025692595e-09
4435 7.17825451743148e-09
4436 7.17738940939072e-09
4437 7.17848213094285e-09
4438 7.17686411658147e-09
4439 7.17654638590637e-09
4440 7.17623917909083e-09
4441 7.17663705310234e-09
4442 7.17638131464415e-09
4443 7.17703902297639e-09
4444 7.17635091693203e-09
4445 7.17714853368245e-09
4446 7.17624470650269e-09
4447 7.17586100423828e-09
4448 7.17601447042249e-09
4449 7.17565328875458e-09
4450 7.17543813658073e-09
4451 7.17281116754953e-09
4452 7.17387202614494e-09
4453 7.17302841374279e-09
4454 7.17337427758036e-09
4455 7.17425380358794e-09
4456 7.17369394742651e-09
4457 7.17355941168285e-09
4458 7.17216920245734e-09
4459 7.1712831012094e-09
4460 7.17101385763752e-09
4461 7.1725857761773e-09
4462 7.17169023045638e-09
4463 7.17149030257658e-09
4464 7.16904629394222e-09
4465 7.16880344700521e-09
4466 7.17197825647609e-09
4467 7.1709240840323e-09
4468 7.17025745358701e-09
4469 7.17093933974566e-09
4470 7.16943795839309e-09
4471 7.16989782290067e-09
4472 7.16843714793081e-09
4473 7.16957418658848e-09
4474 7.16822696961161e-09
4475 7.16802064365529e-09
4476 7.16860361407723e-09
4477 7.16741683956523e-09
4478 7.16679712298163e-09
4479 7.16847304005341e-09
4480 7.16692645907901e-09
4481 7.16676347289891e-09
4482 7.16781509632614e-09
4483 7.16618131807456e-09
4484 7.16462661384809e-09
4485 7.16491353222604e-09
4486 7.16232549699547e-09
4487 7.16540945669419e-09
4488 7.16622504101649e-09
4489 7.16451383622752e-09
4490 7.16352922139341e-09
4491 7.16287175550612e-09
4492 7.16329085001521e-09
4493 7.16103876474694e-09
4494 7.1628499598575e-09
4495 7.16151231136553e-09
4496 7.16300839026518e-09
4497 7.16274998299737e-09
4498 7.16351408797111e-09
4499 7.16150387822245e-09
4500 7.15917554555689e-09
4501 7.16232753722457e-09
4502 7.15888542041077e-09
4503 7.16323417915277e-09
4504 7.16202157463597e-09
4505 7.16014226553874e-09
4506 7.15826772204609e-09
4507 7.16116899487917e-09
4508 7.16033624439949e-09
4509 7.15808231177562e-09
4510 7.1601068347138e-09
4511 7.15950138918675e-09
4512 7.15941593976144e-09
4513 7.15737060455468e-09
4514 7.15664627004564e-09
4515 7.15532134037966e-09
4516 7.15815694485311e-09
4517 7.15900567929695e-09
4518 7.15718427296563e-09
4519 7.15629268488449e-09
4520 7.1577345311935e-09
4521 7.15388993241528e-09
4522 7.15937732517768e-09
4523 7.15700068548708e-09
4524 7.15439372733773e-09
4525 7.15223655975428e-09
4526 7.15644507273683e-09
4527 7.15554831878795e-09
4528 7.15466161799183e-09
4529 7.15369173540092e-09
4530 7.15499648007434e-09
4531 7.15486681296396e-09
4532 7.15413503948881e-09
4533 7.15479007470932e-09
4534 7.15296614223204e-09
4535 7.15201665885123e-09
4536 7.15283938215738e-09
4537 7.15015062546787e-09
4538 7.15461657507799e-09
4539 7.15248944158087e-09
4540 7.15089344432984e-09
4541 7.15197502076936e-09
4542 7.15191777472812e-09
4543 7.15019297206632e-09
4544 7.14991861588965e-09
4545 7.15131463410734e-09
4546 7.1484747575512e-09
4547 7.15103701695585e-09
4548 7.14990554467887e-09
4549 7.14716259572756e-09
4550 7.14933977080667e-09
4551 7.14951928224017e-09
4552 7.14817166258541e-09
4553 7.14621251310388e-09
4554 7.14700959381087e-09
4555 7.14686669051057e-09
4556 7.14710544669206e-09
4557 7.14695244979913e-09
4558 7.14684110098007e-09
4559 7.14640433144287e-09
4560 7.1461199368239e-09
4561 7.14692549011464e-09
4562 7.14592578776596e-09
4563 7.14544183472166e-09
4564 7.14725738626409e-09
4565 7.14570558196792e-09
4566 7.14596237522125e-09
4567 7.14447228714055e-09
4568 7.14318171576855e-09
4569 7.14271101112662e-09
4570 7.1436429993077e-09
4571 7.14318426173199e-09
4572 7.14277883093195e-09
4573 7.14376784263782e-09
4574 7.14231059650849e-09
4575 7.14253451983282e-09
4576 7.14211026947575e-09
4577 7.14096329448855e-09
4578 7.14140427288124e-09
4579 7.14138692364807e-09
4580 7.14028784426501e-09
4581 7.14147967384449e-09
4582 7.14049981673237e-09
4583 7.13979145366972e-09
4584 7.14115475086707e-09
4585 7.14077288727077e-09
4586 7.14012039981782e-09
4587 7.13844128205632e-09
4588 7.13856261783108e-09
4589 7.13958593667741e-09
4590 7.13972566951382e-09
4591 7.13841320326702e-09
4592 7.13708647365197e-09
4593 7.1372788039703e-09
4594 7.13719545611391e-09
4595 7.13660493309454e-09
4596 7.13898272275415e-09
4597 7.13806666904904e-09
4598 7.13577613134508e-09
4599 7.13482793920917e-09
4600 7.13583959713326e-09
4601 7.13428911452985e-09
4602 7.13541708913246e-09
4603 7.13463088083954e-09
4604 7.13456324805128e-09
4605 7.13476558830695e-09
4606 7.13265750221237e-09
4607 7.13395023391183e-09
4608 7.13308403063606e-09
4609 7.13273628316635e-09
4610 7.1320926829932e-09
4611 7.13231775018031e-09
4612 7.13200858004637e-09
4613 7.13079309408449e-09
4614 7.13140758717978e-09
4615 7.12995629556512e-09
4616 7.12963087853846e-09
4617 7.12964564802965e-09
4618 7.13032482765885e-09
4619 7.13104565475131e-09
4620 7.13175938990562e-09
4621 7.13018112302732e-09
4622 7.12862512444734e-09
4623 7.12855408932511e-09
4624 7.12912551217859e-09
4625 7.12992186863159e-09
4626 7.12755724222025e-09
4627 7.12867636704084e-09
4628 7.128065578621e-09
4629 7.12658789484011e-09
4630 7.1279569893723e-09
4631 7.12669298269519e-09
4632 7.12648394507465e-09
4633 7.12841785699414e-09
4634 7.12729377519428e-09
4635 7.12542516295711e-09
4636 7.12615440465192e-09
4637 7.12755171389245e-09
4638 7.12445853470745e-09
4639 7.12643800868107e-09
4640 7.12643559955262e-09
4641 7.125446342654e-09
4642 7.12363859842191e-09
4643 7.12381199935441e-09
4644 7.12455875975793e-09
4645 7.1229914543891e-09
4646 7.12375010250565e-09
4647 7.12456251886207e-09
4648 7.12274129338608e-09
4649 7.12288963614083e-09
4650 7.12101844516089e-09
4651 7.12266342739465e-09
4652 7.12273160180521e-09
4653 7.12125703566935e-09
4654 7.12246442927111e-09
4655 7.12038893155298e-09
4656 7.12046744333339e-09
4657 7.12012089332248e-09
4658 7.11971024441871e-09
4659 7.11972746150757e-09
4660 7.12159404142598e-09
4661 7.11886009380214e-09
4662 7.11889613957961e-09
4663 7.12069282277072e-09
4664 7.11834263866784e-09
4665 7.11848545625893e-09
4666 7.11754898488626e-09
4667 7.11901959232764e-09
4668 7.11735783151535e-09
4669 7.11829497648808e-09
4670 7.1166429559677e-09
4671 7.11785857995806e-09
4672 7.11601052094668e-09
4673 7.11773146266914e-09
4674 7.11673568432047e-09
4675 7.11463921762578e-09
4676 7.11598072625219e-09
4677 7.11542770809315e-09
4678 7.1161530275643e-09
4679 7.11565964089944e-09
4680 7.11681130186048e-09
4681 7.11424516944281e-09
4682 7.11561164382091e-09
4683 7.11367180580846e-09
4684 7.11464024732988e-09
4685 7.11495496433234e-09
4686 7.1132207024871e-09
4687 7.11227727942521e-09
4688 7.11286813290246e-09
4689 7.11354784621587e-09
4690 7.11290261329323e-09
4691 7.11138014403145e-09
4692 7.11155734703373e-09
4693 7.11146291162978e-09
4694 7.1107904882628e-09
4695 7.11163558969607e-09
4696 7.11074247902732e-09
4697 7.10989680494101e-09
4698 7.11059359642663e-09
4699 7.11034358638618e-09
4700 7.1085317293329e-09
4701 7.11033602615596e-09
4702 7.10878417473082e-09
4703 7.10860417418857e-09
4704 7.10880951088066e-09
4705 7.10838236664402e-09
4706 7.10827888342203e-09
4707 7.10793527711195e-09
4708 7.10800488773478e-09
4709 7.10703906758869e-09
4710 7.10732331701824e-09
4711 7.10904943124202e-09
4712 7.10589712343834e-09
4713 7.10749648563658e-09
4714 7.10915994953099e-09
4715 7.10567379486049e-09
4716 7.10537109527287e-09
4717 7.10563915309881e-09
4718 7.10505692219643e-09
4719 7.1037725731804e-09
4720 7.1057461512869e-09
4721 7.10507588308484e-09
4722 7.10603280137945e-09
4723 7.10526640390619e-09
4724 7.10363902550748e-09
4725 7.10300489140114e-09
4726 7.10245377391616e-09
4727 7.1047215878528e-09
4728 7.10488535210496e-09
4729 7.10274373202924e-09
4730 7.10059416902942e-09
4731 7.10231154021912e-09
4732 7.09977513993487e-09
4733 7.10210351143048e-09
4734 7.10041085494328e-09
4735 7.10129430123163e-09
4736 7.10124008479474e-09
4737 7.09973880647086e-09
4738 7.09952925512236e-09
4739 7.10018661123635e-09
4740 7.09891178837796e-09
4741 7.10115855800941e-09
4742 7.09934168621751e-09
4743 7.09837334192387e-09
4744 7.09769071496891e-09
4745 7.09742129048618e-09
4746 7.09809142265816e-09
4747 7.09785414831909e-09
4748 7.09786685959557e-09
4749 7.09848259222712e-09
4750 7.09757316968984e-09
4751 7.09561231632128e-09
4752 7.09567990567206e-09
4753 7.09648232374849e-09
4754 7.09619840727482e-09
4755 7.09785967475951e-09
4756 7.09605653209655e-09
4757 7.09440838811415e-09
4758 7.09596882911279e-09
4759 7.09503206189344e-09
4760 7.09540928198193e-09
4761 7.09385551125874e-09
4762 7.09578136870448e-09
4763 7.0940855741719e-09
4764 7.09396678971741e-09
4765 7.09124694761587e-09
4766 7.09140250373941e-09
4767 7.09423137865151e-09
4768 7.09263193068854e-09
4769 7.09355659650468e-09
4770 7.09252190850274e-09
4771 7.09220604150351e-09
4772 7.09186897360214e-09
4773 7.09174472321439e-09
4774 7.09141317006834e-09
4775 7.09156014444545e-09
4776 7.09118060898684e-09
4777 7.0883671769717e-09
4778 7.09081715991045e-09
4779 7.09002013465909e-09
4780 7.08962429940563e-09
4781 7.09081856900551e-09
4782 7.08910627900128e-09
4783 7.0894558069079e-09
4784 7.08847211228214e-09
4785 7.08929699208549e-09
4786 7.08748346278654e-09
4787 7.08799118764869e-09
4788 7.08778637892316e-09
4789 7.08929505882638e-09
4790 7.08870989291066e-09
4791 7.0878574820743e-09
4792 7.08694994014381e-09
4793 7.08612196939629e-09
4794 7.08574329558176e-09
4795 7.08765656554089e-09
4796 7.08577559116463e-09
4797 7.0844723868424e-09
4798 7.08562288703751e-09
4799 7.08417490211111e-09
4800 7.08432167659256e-09
4801 7.08647996783274e-09
4802 7.08716717423563e-09
4803 7.0844018103533e-09
4804 7.08227433465769e-09
4805 7.08385063849515e-09
4806 7.0837077210395e-09
4807 7.08237422697433e-09
4808 7.08299391902201e-09
4809 7.0838074986701e-09
4810 7.08289727019418e-09
4811 7.08117538525399e-09
4812 7.0840604246003e-09
4813 7.08161814116925e-09
4814 7.08150673656149e-09
4815 7.0814734057234e-09
4816 7.08212655448071e-09
4817 7.08209818348071e-09
4818 7.08168303631407e-09
4819 7.0809441182762e-09
4820 7.08125949741567e-09
4821 7.07937889934951e-09
4822 7.08070159929908e-09
4823 7.08184398184875e-09
4824 7.08100709906345e-09
4825 7.07988984507968e-09
4826 7.07839407901312e-09
4827 7.07897627344467e-09
4828 7.07789188875152e-09
4829 7.07828909266017e-09
4830 7.07572922167832e-09
4831 7.07741577182786e-09
4832 7.07700314303561e-09
4833 7.07704043412249e-09
4834 7.07776838834717e-09
4835 7.07631493512828e-09
4836 7.07663144364212e-09
4837 7.07796817242534e-09
4838 7.07511535655292e-09
4839 7.07694447132923e-09
4840 7.07569061836333e-09
4841 7.07547659531405e-09
4842 7.07427921167736e-09
4843 7.07472509325902e-09
4844 7.07673488212213e-09
4845 7.07518762979587e-09
4846 7.07240396913278e-09
4847 7.07329321958383e-09
4848 7.07400198460273e-09
4849 7.07195203314415e-09
4850 7.0730224865645e-09
4851 7.07292898288747e-09
4852 7.07228692073403e-09
4853 7.07218770779061e-09
4854 7.07128794996681e-09
4855 7.0716145613714e-09
4856 7.0709118466239e-09
4857 7.07120071977152e-09
4858 7.07024691121161e-09
4859 7.07036486594093e-09
4860 7.07078611428269e-09
4861 7.07046449860416e-09
4862 7.06942454298254e-09
4863 7.06985253798309e-09
4864 7.06691118090852e-09
4865 7.06983150428053e-09
4866 7.07020790166601e-09
4867 7.07080530870652e-09
4868 7.06971923222777e-09
4869 7.06717126611189e-09
4870 7.06674985248568e-09
4871 7.06747333956148e-09
4872 7.06823557328207e-09
4873 7.06697063032613e-09
4874 7.06721921586717e-09
4875 7.06626527113841e-09
4876 7.06739009898039e-09
4877 7.06612054607314e-09
4878 7.06625177923637e-09
4879 7.06557725876911e-09
4880 7.06702294153705e-09
4881 7.06571989469995e-09
4882 7.06650474649462e-09
4883 7.06540532530164e-09
4884 7.06201518096972e-09
4885 7.06419910673217e-09
4886 7.06409037309896e-09
4887 7.06457161037122e-09
4888 7.06481859802377e-09
4889 7.06463515923184e-09
4890 7.06376408421416e-09
4891 7.0625180282613e-09
4892 7.06133480529725e-09
4893 7.06378663994922e-09
4894 7.06302222641675e-09
4895 7.06309710776787e-09
4896 7.06132176145346e-09
4897 7.06039062162156e-09
4898 7.06229539812475e-09
4899 7.06139767375769e-09
4900 7.06285511162252e-09
4901 7.06087183788284e-09
4902 7.05966994904439e-09
4903 7.06023146143653e-09
4904 7.06131977273872e-09
4905 7.05973718617692e-09
4906 7.06015995333153e-09
4907 7.05979151907621e-09
4908 7.05900473282894e-09
4909 7.06029317729118e-09
4910 7.05881399162833e-09
4911 7.05819745500103e-09
4912 7.05865823324992e-09
4913 7.05947359155856e-09
4914 7.05809106490984e-09
4915 7.0579114113678e-09
4916 7.05648632548517e-09
4917 7.05572879125538e-09
4918 7.05789843366555e-09
4919 7.05617390237001e-09
4920 7.05462288705383e-09
4921 7.05661500155497e-09
4922 7.0552395621537e-09
4923 7.05568231160192e-09
4924 7.05551175461672e-09
4925 7.05556741326108e-09
4926 7.05461164474119e-09
4927 7.05490672484044e-09
4928 7.05335021755626e-09
4929 7.05433720449289e-09
4930 7.05537803383227e-09
4931 7.05328800737481e-09
4932 7.05349308813275e-09
4933 7.05230004544033e-09
4934 7.05348793475502e-09
4935 7.05487171515085e-09
4936 7.05183321106362e-09
4937 7.05221101324205e-09
4938 7.05217167279448e-09
4939 7.05310506182721e-09
4940 7.05228919109513e-09
4941 7.05082450727645e-09
4942 7.05024682043365e-09
4943 7.05133106890243e-09
4944 7.05117671634392e-09
4945 7.0487996736146e-09
4946 7.05132061831759e-09
4947 7.04902463344315e-09
4948 7.04843236404296e-09
4949 7.05073966134062e-09
4950 7.04954451685724e-09
4951 7.04918391974951e-09
4952 7.04932736919628e-09
4953 7.04771221121669e-09
4954 7.04922770783378e-09
4955 7.04851652746918e-09
4956 7.04795834835514e-09
4957 7.04795117070778e-09
4958 7.04886027155793e-09
4959 7.04760461994747e-09
4960 7.04676441262708e-09
4961 7.04621105826475e-09
4962 7.04616828564597e-09
4963 7.04665426351503e-09
4964 7.04609520940647e-09
4965 7.0465585251811e-09
4966 7.0460578143472e-09
4967 7.04387479441571e-09
4968 7.04525119532562e-09
4969 7.04292757139346e-09
4970 7.04334177581423e-09
4971 7.04415254745849e-09
4972 7.04343578117328e-09
4973 7.04181623562006e-09
4974 7.04470920379063e-09
4975 7.0424041808681e-09
4976 7.04321878486347e-09
4977 7.04447046567802e-09
4978 7.04247998051244e-09
4979 7.04518417940503e-09
4980 7.04294679568229e-09
4981 7.04270406656771e-09
4982 7.041855974943e-09
4983 7.04411731999865e-09
4984 7.04247539889957e-09
4985 7.04164153542131e-09
4986 7.04149592328784e-09
4987 7.03964702036819e-09
4988 7.04059142075941e-09
4989 7.0402792713975e-09
4990 7.04100030268773e-09
4991 7.03985363856297e-09
4992 7.03994090825444e-09
4993 7.03820340031136e-09
4994 7.03829947285017e-09
4995 7.0386337871764e-09
4996 7.03862819767531e-09
4997 7.03811985638958e-09
4998 7.03800907653207e-09
4999 7.03963395087825e-09
};
\addlegendentry{Train}
\addplot [semithick, black]
table {%
0 0.00136148638557643
1 0.000174645829247311
2 0.000125022401334718
3 8.19502311060205e-05
4 4.86851240566466e-05
5 3.40955848514568e-05
6 2.85796868411126e-05
7 2.53252674156101e-05
8 2.22924445552053e-05
9 1.90931132237893e-05
10 1.57168033183552e-05
11 1.2370219337754e-05
12 9.40816971706226e-06
13 7.15702526576933e-06
14 5.74189198232489e-06
15 4.95481162943179e-06
16 4.44743955085869e-06
17 4.00834096581093e-06
18 3.6278022434999e-06
19 3.26697818309185e-06
20 2.9371879008977e-06
21 2.64080085798923e-06
22 2.35962352235219e-06
23 2.13428415918315e-06
24 1.9787455585174e-06
25 1.86505747024057e-06
26 1.78028494701721e-06
27 1.71838394180668e-06
28 1.66220593200705e-06
29 1.61484376803855e-06
30 1.57025681346568e-06
31 1.53031783156621e-06
32 1.49477909872076e-06
33 1.4602084092985e-06
34 1.42857550144981e-06
35 1.39839460189251e-06
36 1.36954429308389e-06
37 1.34058484491106e-06
38 1.30695025291061e-06
39 1.27698444885027e-06
40 1.24700352444052e-06
41 1.21964490062965e-06
42 1.19234448447969e-06
43 1.16719047582592e-06
44 1.14181273147551e-06
45 1.11840040517563e-06
46 1.09460358999058e-06
47 1.0725329957495e-06
48 1.05231447378173e-06
49 1.03250840766123e-06
50 1.01203067970346e-06
51 9.92464833871054e-07
52 9.7433610335429e-07
53 9.5723453341634e-07
54 9.40036613883422e-07
55 9.2456082256831e-07
56 9.08156323475851e-07
57 8.93365722731687e-07
58 8.79353478921985e-07
59 8.65098911617679e-07
60 8.52188691169431e-07
61 8.39711958633416e-07
62 8.28091572202538e-07
63 8.16681620108284e-07
64 8.06520290552726e-07
65 7.95934397501696e-07
66 7.86602697644412e-07
67 7.76862577822612e-07
68 7.67443680160795e-07
69 7.57979989884916e-07
70 7.49821651879756e-07
71 7.42751353755011e-07
72 7.32957687432645e-07
73 7.25357551800698e-07
74 7.17382874881878e-07
75 7.08485401901271e-07
76 7.01069893693784e-07
77 6.94097082032386e-07
78 6.89414378030051e-07
79 6.82742779645196e-07
80 6.75414014494891e-07
81 6.67312008317822e-07
82 6.58823637422756e-07
83 6.52618723506748e-07
84 6.47145157017803e-07
85 6.404779355762e-07
86 6.34341574823338e-07
87 6.27735118996497e-07
88 6.23523987997032e-07
89 6.1573950915772e-07
90 6.08553932579525e-07
91 6.03109469921037e-07
92 5.98507767790579e-07
93 5.90251545418141e-07
94 5.86017904424807e-07
95 5.77869798235042e-07
96 5.72767021367326e-07
97 5.69150870433077e-07
98 5.62827892736095e-07
99 5.59147395051696e-07
100 5.5337545745715e-07
101 5.48994023574778e-07
102 5.45449722721969e-07
103 5.40849896424334e-07
104 5.38018014140107e-07
105 5.33425065896154e-07
106 5.30555894329154e-07
107 5.28410168953997e-07
108 5.23771916505211e-07
109 5.18876163368986e-07
110 5.11341966102918e-07
111 5.06500100527774e-07
112 5.02330067320145e-07
113 4.99108580243046e-07
114 4.96250265769049e-07
115 4.92487743031234e-07
116 4.88403372855828e-07
117 4.84025576952263e-07
118 4.80388280266197e-07
119 4.77377682273072e-07
120 4.74243989856404e-07
121 4.7107619138842e-07
122 4.68175471723953e-07
123 4.63830104990848e-07
124 4.61086472114403e-07
125 4.57979041357248e-07
126 4.5526050485023e-07
127 4.51638953791189e-07
128 4.52589205224285e-07
129 4.49263012569645e-07
130 4.45643451030264e-07
131 4.43057643906286e-07
132 4.39509307170738e-07
133 4.36320931385126e-07
134 4.33539241839753e-07
135 4.30780261240216e-07
136 4.28169073529716e-07
137 4.25784975277566e-07
138 4.2344203166067e-07
139 4.209771020669e-07
140 4.18607811525362e-07
141 4.15794033870043e-07
142 4.13120005759993e-07
143 4.10587347232649e-07
144 4.0880297547119e-07
145 4.06738365654746e-07
146 4.04743474291536e-07
147 4.03344330379696e-07
148 4.02014165956643e-07
149 4.00731522631759e-07
150 3.98351943431408e-07
151 3.96680036374164e-07
152 3.94704215977981e-07
153 3.92920696867805e-07
154 3.91250864595349e-07
155 3.89844757364699e-07
156 3.87699401471764e-07
157 3.86532150287167e-07
158 3.83956802352259e-07
159 3.82341994509261e-07
160 3.80651897557982e-07
161 3.78830577574263e-07
162 3.77406394136415e-07
163 3.76160869564046e-07
164 3.75171765654159e-07
165 3.73399473119207e-07
166 3.71882464378359e-07
167 3.70316826092676e-07
168 3.68432296227184e-07
169 3.67185350569343e-07
170 3.65897818710437e-07
171 3.64746796321924e-07
172 3.63810386261321e-07
173 3.62791126917728e-07
174 3.61209856691858e-07
175 3.59746792355509e-07
176 3.58756494733825e-07
177 3.57792572458493e-07
178 3.56761887587709e-07
179 3.55947150865177e-07
180 3.54983882289162e-07
181 3.51664454001366e-07
182 3.50674326909939e-07
183 3.50574339336163e-07
184 3.49594273529874e-07
185 3.49080721662176e-07
186 3.48438732089562e-07
187 3.47679161905035e-07
188 3.47126473343451e-07
189 3.46677779816673e-07
190 3.46543743034999e-07
191 3.46048324217918e-07
192 3.45538978763216e-07
193 3.4439310070411e-07
194 3.44752777436952e-07
195 3.44377440342214e-07
196 3.439310489739e-07
197 3.4303738516428e-07
198 3.42554329790801e-07
199 3.42812569442685e-07
200 3.4231237577842e-07
201 3.42112883799928e-07
202 3.41369030820715e-07
203 3.40956773925427e-07
204 3.40979369184424e-07
205 3.3922768238881e-07
206 3.38772395025444e-07
207 3.39076336786093e-07
208 3.38171673774923e-07
209 3.37945948558627e-07
210 3.37574846298594e-07
211 3.37093212010586e-07
212 3.3665344290057e-07
213 3.36137134127057e-07
214 3.35107756654907e-07
215 3.34780395405687e-07
216 3.34897720222216e-07
217 3.33878546143751e-07
218 3.33465891344531e-07
219 3.3327134474348e-07
220 3.33170362409874e-07
221 3.32828818727648e-07
222 3.32369495481544e-07
223 3.33952016262629e-07
224 3.33505226990383e-07
225 3.32874463992994e-07
226 3.30582423657688e-07
227 3.31355948901546e-07
228 3.31058117808425e-07
229 3.30450376395675e-07
230 3.27957053514183e-07
231 3.29149969502396e-07
232 3.28805072058458e-07
233 3.28188434650656e-07
234 3.26023354091376e-07
235 3.27479256156948e-07
236 3.26982615206362e-07
237 3.26501520930833e-07
238 3.23785258160569e-07
239 3.24820888408794e-07
240 3.24534340734317e-07
241 3.23901929277781e-07
242 3.23388889000853e-07
243 3.22468679314625e-07
244 3.22109769967938e-07
245 3.21551397064468e-07
246 3.2102838076753e-07
247 3.20208755510976e-07
248 3.20140600251761e-07
249 3.19565572226566e-07
250 3.18973121693489e-07
251 3.17854926379368e-07
252 3.17700454388614e-07
253 3.16869460448288e-07
254 3.16658031351835e-07
255 3.15559475438931e-07
256 3.15735377398596e-07
257 3.15178851906239e-07
258 3.1459799743061e-07
259 3.13673155005745e-07
260 3.12300670657351e-07
261 3.12620869635793e-07
262 3.10427992644691e-07
263 3.10965077687797e-07
264 3.10691916638461e-07
265 3.11155815779784e-07
266 3.10961326022152e-07
267 3.10431090611019e-07
268 3.09638863882356e-07
269 3.09170019363592e-07
270 3.08542354332531e-07
271 3.08289486383728e-07
272 3.0738576128897e-07
273 3.08009390437292e-07
274 3.07480632955048e-07
275 3.08015586369947e-07
276 3.06231157765069e-07
277 3.05770839759134e-07
278 3.05050690485587e-07
279 3.05443762727009e-07
280 3.03807411228263e-07
281 3.03110851973543e-07
282 3.02042508337763e-07
283 3.01502439015167e-07
284 3.00294175303861e-07
285 3.01203527897087e-07
286 2.98857372627026e-07
287 2.9839063131476e-07
288 2.97727325460073e-07
289 2.97070130272914e-07
290 2.96588524406616e-07
291 2.97904932722304e-07
292 2.9550301405834e-07
293 2.9493222086785e-07
294 2.93695876507627e-07
295 2.93054853273134e-07
296 2.92440404336958e-07
297 2.9182206162659e-07
298 2.91077725478317e-07
299 2.90409531089608e-07
300 2.89891033844469e-07
301 2.89184896473671e-07
302 2.89018515786665e-07
303 2.88391277081246e-07
304 2.87821563915713e-07
305 2.87228914430671e-07
306 2.86573225594111e-07
307 2.79884261544794e-07
308 2.79258557611683e-07
309 2.78537555686853e-07
310 2.77909265378185e-07
311 2.7713306849364e-07
312 2.76531352483289e-07
313 2.75881717470838e-07
314 2.75241490044209e-07
315 2.74656770216097e-07
316 2.73987154741917e-07
317 2.73366197234282e-07
318 2.72836672365884e-07
319 2.72133831913379e-07
320 2.71721006583903e-07
321 2.7106446509606e-07
322 2.70662468437877e-07
323 2.69776649020059e-07
324 2.69214950776586e-07
325 2.68868717512305e-07
326 2.68248669499371e-07
327 2.67266557330004e-07
328 2.66842562268721e-07
329 2.66188720843274e-07
330 2.65405361687954e-07
331 2.64662446625152e-07
332 2.63941473122031e-07
333 2.63331855876459e-07
334 2.62413692553309e-07
335 2.61833150716484e-07
336 2.61128349166029e-07
337 2.60791239270475e-07
338 2.60418858033518e-07
339 2.57642852830031e-07
340 2.57227299016449e-07
341 2.56626549344219e-07
342 2.56249080621274e-07
343 2.55625110412439e-07
344 2.55015237371481e-07
345 2.54397974686071e-07
346 2.53642525649411e-07
347 2.5304831297035e-07
348 2.53040383313419e-07
349 2.53446614806307e-07
350 2.52496050734408e-07
351 2.51605683843081e-07
352 2.52540644396504e-07
353 2.50090096187705e-07
354 2.4760643668742e-07
355 2.44500029111805e-07
356 2.41812671220032e-07
357 2.41228491404399e-07
358 2.40020057162837e-07
359 2.38992186041287e-07
360 2.37955291026992e-07
361 2.3689131012361e-07
362 2.37020898907758e-07
363 2.35932631653668e-07
364 2.35331327758104e-07
365 2.34541076338246e-07
366 2.33597191368062e-07
367 2.32902337415908e-07
368 2.32194850013911e-07
369 2.31077351031672e-07
370 2.30492616992706e-07
371 2.29717841193633e-07
372 2.29299175202868e-07
373 2.28559500214942e-07
374 2.28144386937856e-07
375 2.27389719498206e-07
376 2.26405489911485e-07
377 2.25575206513895e-07
378 2.25790955710181e-07
379 2.24893454969788e-07
380 2.24211348154313e-07
381 2.23532794052517e-07
382 2.2290439005701e-07
383 2.22239805225399e-07
384 2.21924423726705e-07
385 2.21620226170671e-07
386 2.20787725879745e-07
387 2.19807375856362e-07
388 2.18758486880688e-07
389 2.18864514067718e-07
390 2.20175792264854e-07
391 2.20314092302942e-07
392 2.1954130602353e-07
393 2.18121812167738e-07
394 2.17052416928709e-07
395 2.16180382039965e-07
396 2.15146357618323e-07
397 2.13375713542518e-07
398 2.12200816918084e-07
399 2.11217042078715e-07
400 2.1125123339516e-07
401 2.10354428986648e-07
402 2.09169130016562e-07
403 2.07134945640064e-07
404 2.07240034910683e-07
405 2.06569580996074e-07
406 2.0564965552694e-07
407 2.04594698516303e-07
408 2.03824939148944e-07
409 2.03052579195173e-07
410 2.02473771082623e-07
411 2.0109447973482e-07
412 2.00613797574078e-07
413 1.9972502229848e-07
414 1.983301274322e-07
415 1.9738806145142e-07
416 1.96938046315154e-07
417 1.95923377077634e-07
418 1.94958658994437e-07
419 1.9422226671395e-07
420 1.92566702139629e-07
421 1.92616965932757e-07
422 1.92384860042694e-07
423 1.92421282463329e-07
424 1.90833887359076e-07
425 1.90138379707605e-07
426 1.88967732128731e-07
427 1.88125440558906e-07
428 1.87753840918958e-07
429 1.87268057061374e-07
430 1.86658525080929e-07
431 1.85802377927757e-07
432 1.85693664889186e-07
433 1.84890552645811e-07
434 1.84071808462249e-07
435 1.83532335995551e-07
436 1.82603727694186e-07
437 1.82715979235581e-07
438 1.80965670892874e-07
439 1.81656389486307e-07
440 1.8268535484367e-07
441 1.83164416966974e-07
442 1.82222024136536e-07
443 1.81486967676392e-07
444 1.80804320848438e-07
445 1.80006495043017e-07
446 1.79515353693205e-07
447 1.78659689709093e-07
448 1.78050470367452e-07
449 1.77704251314026e-07
450 1.7700870102999e-07
451 1.76285652742081e-07
452 1.75690246351223e-07
453 1.75131688706642e-07
454 1.74922277551559e-07
455 1.74132068764266e-07
456 1.73068571029944e-07
457 1.72307039747466e-07
458 1.71965027107035e-07
459 1.70918852404611e-07
460 1.7024555631906e-07
461 1.69432283314563e-07
462 1.68858335314326e-07
463 1.68362205954509e-07
464 1.67776576631695e-07
465 1.67328352063123e-07
466 1.67552371976853e-07
467 1.66577578397664e-07
468 1.65917541039562e-07
469 1.65312442845789e-07
470 1.64735951102557e-07
471 1.64255965273696e-07
472 1.63616519444076e-07
473 1.63000933639523e-07
474 1.6230876553891e-07
475 1.61806909204643e-07
476 1.61228896899956e-07
477 1.60785731395663e-07
478 1.6055435025919e-07
479 1.59862779014475e-07
480 1.59529264465164e-07
481 1.59130649990402e-07
482 1.58843945996523e-07
483 1.58119362936304e-07
484 1.57720066340516e-07
485 1.574484684852e-07
486 1.57220156893345e-07
487 1.56765224801347e-07
488 1.56538277451546e-07
489 1.55585851757678e-07
490 1.5572572920064e-07
491 1.55328805817589e-07
492 1.54752413550341e-07
493 1.54256682094456e-07
494 1.54016774445154e-07
495 1.53042222450495e-07
496 1.52959557908616e-07
497 1.52577882772675e-07
498 1.52208826875722e-07
499 1.51721366137281e-07
500 1.51260763914252e-07
501 1.50918765484676e-07
502 1.49405749994003e-07
503 1.48957511214576e-07
504 1.48276768641153e-07
505 1.47758910884477e-07
506 1.47283998330749e-07
507 1.46836882208845e-07
508 1.46353258401177e-07
509 1.45882097513095e-07
510 1.45087895475626e-07
511 1.44687817282829e-07
512 1.44404864954595e-07
513 1.43701697652432e-07
514 1.43230636240332e-07
515 1.42721830798109e-07
516 1.42196043384502e-07
517 1.41091575756036e-07
518 1.40922011837574e-07
519 1.40520057811955e-07
520 1.39486814987322e-07
521 1.38842324304278e-07
522 1.38475797939464e-07
523 1.37989587756238e-07
524 1.37701519520306e-07
525 1.37281233492104e-07
526 1.36494904268147e-07
527 1.36546844942131e-07
528 1.36206068646061e-07
529 1.35813920110195e-07
530 1.35413586122013e-07
531 1.35046434479591e-07
532 1.34898002102091e-07
533 1.34452491806769e-07
534 1.33664059376315e-07
535 1.33634060262011e-07
536 1.33267249680102e-07
537 1.33209141495172e-07
538 1.32917122641629e-07
539 1.32570235678031e-07
540 1.32208143099888e-07
541 1.31788127077925e-07
542 1.31218271803846e-07
543 1.31111619339208e-07
544 1.30861820935024e-07
545 1.30524796304599e-07
546 1.30265576103739e-07
547 1.29839847318181e-07
548 1.29591015252117e-07
549 1.29253407976648e-07
550 1.2874413357622e-07
551 1.27584584674878e-07
552 1.27410260120087e-07
553 1.27199228927566e-07
554 1.26889460716484e-07
555 1.26164522384897e-07
556 1.26177951642603e-07
557 1.25967389408288e-07
558 1.2529741866274e-07
559 1.24875114693168e-07
560 1.24544001778304e-07
561 1.2408661120844e-07
562 1.23640845117734e-07
563 1.23565513376889e-07
564 1.23678333352473e-07
565 1.25386875993172e-07
566 1.24969929515828e-07
567 1.24711064586336e-07
568 1.24282038882484e-07
569 1.23280813113524e-07
570 1.23175993849145e-07
571 1.22383951861593e-07
572 1.21899248028967e-07
573 1.21317697221457e-07
574 1.20628129707256e-07
575 1.2056156606377e-07
576 1.20279906923315e-07
577 1.19950399835034e-07
578 1.19459656389154e-07
579 1.19156638334061e-07
580 1.18501006340921e-07
581 1.17045736658383e-07
582 1.17804937360688e-07
583 1.17474151295482e-07
584 1.17012945111128e-07
585 1.1690508472384e-07
586 1.16229095681319e-07
587 1.16668594785097e-07
588 1.15950570034329e-07
589 1.156475164521e-07
590 1.14119039551497e-07
591 1.1495867369149e-07
592 1.14473536427795e-07
593 1.14228249969983e-07
594 1.13548921376605e-07
595 1.12295886367519e-07
596 1.12252280359826e-07
597 1.11468217767197e-07
598 1.11384991896557e-07
599 1.09493342392852e-07
600 1.09992527086433e-07
601 1.09089910438342e-07
602 1.09404716397421e-07
603 1.08371544627062e-07
604 1.08549478738951e-07
605 1.08382451458056e-07
606 1.08467453685535e-07
607 1.08266341669605e-07
608 1.08017310651576e-07
609 1.0742195399871e-07
610 1.0756901502873e-07
611 1.07559941397994e-07
612 1.07179644714961e-07
613 1.0725950971846e-07
614 1.07300898832818e-07
615 1.06665424937091e-07
616 1.06411491174185e-07
617 1.06102625352378e-07
618 1.05813882100847e-07
619 1.05390533633454e-07
620 1.05199035260739e-07
621 1.04175292392483e-07
622 1.04344692886116e-07
623 1.04086517183077e-07
624 1.03774787874045e-07
625 1.03516669014425e-07
626 1.03204733647999e-07
627 1.0283346085771e-07
628 1.02324520412367e-07
629 1.02011298963589e-07
630 1.01840797128716e-07
631 1.01031950805464e-07
632 1.01204257418885e-07
633 1.00907463718158e-07
634 1.00670355607235e-07
635 1.00169195604849e-07
636 1.00067744313037e-07
637 9.93434383644853e-08
638 9.92485453821246e-08
639 9.92072699546043e-08
640 9.89706876453056e-08
641 9.86921548928876e-08
642 9.84364874057064e-08
643 9.79761125563527e-08
644 9.80364518454735e-08
645 9.76768035343412e-08
646 9.74167875256171e-08
647 9.71861524590167e-08
648 9.70015392454115e-08
649 9.68008180279867e-08
650 9.66411732861161e-08
651 9.63869837278253e-08
652 9.60244719294678e-08
653 9.59153254598277e-08
654 9.55129593194215e-08
655 9.54557890509022e-08
656 9.51322149944644e-08
657 9.49956486806514e-08
658 9.46392120226847e-08
659 9.44405869063303e-08
660 9.42577429441371e-08
661 9.38245818815631e-08
662 9.33538402136946e-08
663 9.3194948647124e-08
664 9.27720762433637e-08
665 9.25576273402839e-08
666 9.24166840832186e-08
667 9.20740106380435e-08
668 9.1048384831538e-08
669 9.08191211124176e-08
670 9.05010466567546e-08
671 9.03077932434826e-08
672 8.99565293366322e-08
673 8.97840308766717e-08
674 8.94979876875368e-08
675 8.91956233317615e-08
676 8.88862530246115e-08
677 8.87258835291505e-08
678 8.83567139453589e-08
679 8.812787655188e-08
680 8.79663204500503e-08
681 8.78692034689266e-08
682 8.75406414024837e-08
683 8.73777281640287e-08
684 8.70044871703612e-08
685 8.6873718885272e-08
686 8.64410552026129e-08
687 8.63334861378462e-08
688 8.61724132050767e-08
689 8.59448476830948e-08
690 8.6001698207383e-08
691 8.56864801335178e-08
692 8.53708712611478e-08
693 8.51203907359377e-08
694 8.48505763428875e-08
695 8.44152268086873e-08
696 8.41664373751883e-08
697 8.3782033755142e-08
698 8.36361238043537e-08
699 8.33255953125445e-08
700 8.30082029779078e-08
701 8.29248563150031e-08
702 8.27571184913722e-08
703 8.2507789045394e-08
704 8.22632060248907e-08
705 8.19830674458899e-08
706 8.20342975771382e-08
707 8.17104464090335e-08
708 8.14854388409003e-08
709 8.10739848589037e-08
710 8.08081779268832e-08
711 8.05232431844161e-08
712 8.03405697524795e-08
713 7.99363064629688e-08
714 7.97669770236098e-08
715 7.9419677945225e-08
716 7.92580507891216e-08
717 7.91133487609841e-08
718 7.88488137004606e-08
719 7.84472788950552e-08
720 7.82591698111901e-08
721 7.80946436407248e-08
722 7.78324746875114e-08
723 7.76448629835613e-08
724 7.74847350726304e-08
725 7.7757874805684e-08
726 7.76779529587657e-08
727 7.75636337380092e-08
728 7.71931496501566e-08
729 7.71343451333451e-08
730 7.69410632983636e-08
731 7.6599789622378e-08
732 7.64588463653126e-08
733 7.62618626026779e-08
734 7.5937869326026e-08
735 7.58865894567862e-08
736 7.56563807158273e-08
737 7.5216419759272e-08
738 7.54321902718402e-08
739 7.49776276620651e-08
740 7.48630952784879e-08
741 7.43051202789502e-08
742 7.44147001796591e-08
743 7.4222498369636e-08
744 7.37745082801666e-08
745 7.37783594217944e-08
746 7.37504208814244e-08
747 7.36881986540538e-08
748 7.33801002184009e-08
749 7.30818356942109e-08
750 7.28282074646813e-08
751 7.28849514075591e-08
752 7.24205904134578e-08
753 7.24622069014913e-08
754 7.22536839248278e-08
755 7.22516659834582e-08
756 7.17662373972416e-08
757 7.14856298600353e-08
758 7.16111756560167e-08
759 7.15761458991437e-08
760 7.09160303813405e-08
761 7.12179755169018e-08
762 7.09473297888508e-08
763 7.10043579488229e-08
764 7.08336784782659e-08
765 7.06160889762941e-08
766 7.04364353509845e-08
767 7.03188121065068e-08
768 6.99257967085032e-08
769 7.00046030033263e-08
770 6.95650044235663e-08
771 6.98014943623093e-08
772 6.96330957339342e-08
773 6.95633133318552e-08
774 6.82396574802624e-08
775 6.78026026434964e-08
776 6.77512019819915e-08
777 6.76893208151341e-08
778 6.71915003636059e-08
779 6.69992985535828e-08
780 6.6627599437652e-08
781 6.64431851760128e-08
782 6.60455015122352e-08
783 6.59372929590063e-08
784 6.59426646620886e-08
785 6.57581935570306e-08
786 6.56210872307383e-08
787 6.53365432867758e-08
788 6.5378074509681e-08
789 6.52690275160239e-08
790 6.49979341460494e-08
791 6.49871836344573e-08
792 6.45643538632612e-08
793 6.5107904845263e-08
794 6.47515747687066e-08
795 6.48519602464148e-08
796 6.78370355444713e-08
797 6.69031408051524e-08
798 6.41622079911031e-08
799 6.84245833326713e-08
800 6.34761221363078e-08
801 6.34407015809302e-08
802 6.64673578398833e-08
803 6.32759480367895e-08
804 6.76056615134257e-08
805 6.55350405054378e-08
806 6.70956410431245e-08
807 6.28979321959378e-08
808 6.27721590262809e-08
809 6.7199550812802e-08
810 6.57310863516614e-08
811 6.25623854944024e-08
812 6.69279955900492e-08
813 6.24787048764119e-08
814 6.22793834281765e-08
815 6.52448690630081e-08
816 6.21651636834031e-08
817 6.20535516304699e-08
818 6.60655956608025e-08
819 6.61083916497773e-08
820 6.18855935385909e-08
821 6.17400175428884e-08
822 6.46284661343088e-08
823 6.560653531551e-08
824 6.44003392835657e-08
825 6.13917805480924e-08
826 6.12768857877199e-08
827 6.47217106575226e-08
828 6.11794703786472e-08
829 6.41185309291359e-08
830 6.45065796334165e-08
831 6.12172854630444e-08
832 6.49782236905594e-08
833 6.11644210835038e-08
834 6.43085797946696e-08
835 6.49600124802419e-08
836 6.08613106578559e-08
837 6.3165124686293e-08
838 6.03699206180863e-08
839 6.27493363936082e-08
840 6.02378875669274e-08
841 6.30539318535739e-08
842 6.00635488012813e-08
843 6.34949302025234e-08
844 6.00783209847577e-08
845 6.12032167168763e-08
846 5.93162745587961e-08
847 5.89594009170469e-08
848 6.13309936170481e-08
849 5.87782835737016e-08
850 6.19584099581516e-08
851 5.88988449123917e-08
852 6.17945090652938e-08
853 5.86939847835311e-08
854 6.02380154646198e-08
855 5.8235837485654e-08
856 5.82842467622413e-08
857 5.8214478571017e-08
858 5.81168890789741e-08
859 5.80117642812183e-08
860 5.81306522917657e-08
861 5.81014525380397e-08
862 5.8009423042904e-08
863 6.13725958942268e-08
864 5.79039394210668e-08
865 5.78260177519496e-08
866 5.76001113472557e-08
867 5.91892543866379e-08
868 5.77594434503226e-08
869 5.98831420006718e-08
870 5.72666607467909e-08
871 5.73105651824335e-08
872 5.95905227385174e-08
873 5.72624436756541e-08
874 5.71640477176061e-08
875 5.95054068242007e-08
876 5.71341978172768e-08
877 5.689359028338e-08
878 5.81068206884083e-08
879 5.9151791020895e-08
880 5.70120910481364e-08
881 5.68377167553535e-08
882 5.66297266857418e-08
883 5.87366564275271e-08
884 5.65607329860995e-08
885 5.65165372279353e-08
886 5.75126080093469e-08
887 5.64772548727888e-08
888 5.63068986991766e-08
889 5.83090908889972e-08
890 5.63318600654839e-08
891 5.60990329745437e-08
892 5.61035164992063e-08
893 5.65088953408122e-08
894 5.5963955247762e-08
895 5.58865451694146e-08
896 5.57880639462383e-08
897 5.75222429688438e-08
898 5.59429800262023e-08
899 5.5378873042855e-08
900 5.4360320689284e-08
901 5.54458559065552e-08
902 5.36224114000561e-08
903 5.33165582794481e-08
904 5.32717265855354e-08
905 5.47580718546214e-08
906 5.29561745565843e-08
907 5.2679006046219e-08
908 5.24970005244541e-08
909 5.3895458762554e-08
910 4.99618018068304e-08
911 4.83667861317372e-08
912 4.71864041173831e-08
913 4.8966580123988e-08
914 4.78935824332893e-08
915 4.49338664054721e-08
916 4.45550760730384e-08
917 4.59911717598516e-08
918 4.41066028145087e-08
919 4.477112014456e-08
920 4.39838458987651e-08
921 4.43942305139444e-08
922 4.42385790222488e-08
923 4.36970033490525e-08
924 4.40568648230055e-08
925 4.37786233931092e-08
926 4.39153104991874e-08
927 4.37811529252485e-08
928 4.3505490765483e-08
929 4.3049301012843e-08
930 4.36014531146611e-08
931 4.24110346841644e-08
932 4.32864446509029e-08
933 4.18764045662101e-08
934 4.29861870543391e-08
935 4.14358929390346e-08
936 4.23705692753629e-08
937 4.11932248312041e-08
938 4.22751611495187e-08
939 4.07466203000695e-08
940 4.19564898379576e-08
941 4.02447817293705e-08
942 4.20842525272747e-08
943 4.13472207583254e-08
944 4.16331360497679e-08
945 3.95084960302938e-08
946 4.14818721594656e-08
947 4.12313561071187e-08
948 4.11114484677455e-08
949 4.09927132238863e-08
950 3.89797811806147e-08
951 4.09612255225511e-08
952 4.09030285197787e-08
953 4.08021776365786e-08
954 4.06820959142351e-08
955 4.09024067948849e-08
956 4.05662419211694e-08
957 4.02671922472564e-08
958 4.02328979021149e-08
959 3.98572659321417e-08
960 3.97155659470627e-08
961 3.97582553546272e-08
962 3.94369408240891e-08
963 3.93362604711456e-08
964 3.92311712005267e-08
965 3.91274852518109e-08
966 3.90293166674383e-08
967 3.90261440941231e-08
968 3.88261511830024e-08
969 3.86048810696593e-08
970 3.9263863271799e-08
971 3.8952435943429e-08
972 3.91149264089563e-08
973 3.88068777112949e-08
974 3.87604828233634e-08
975 3.87025522741169e-08
976 3.87392624645599e-08
977 3.86571166188787e-08
978 3.86100111882115e-08
979 3.84856058133209e-08
980 3.84760916460891e-08
981 3.83792837510555e-08
982 3.83480660559599e-08
983 3.84187792690227e-08
984 3.83344520571427e-08
985 3.83435683204425e-08
986 3.81787579328829e-08
987 3.82189071501671e-08
988 3.81478741928731e-08
989 3.82499187878693e-08
990 3.80123630350226e-08
991 3.81897429235778e-08
992 3.80105902308969e-08
993 3.80851226111645e-08
994 3.81003637528465e-08
995 3.79574913722536e-08
996 3.74414170778437e-08
997 3.47378836806911e-08
998 3.74235575861803e-08
999 3.72861599373664e-08
1000 3.74078190645832e-08
1001 3.74236961420138e-08
1002 3.72616355548416e-08
1003 3.72498512035691e-08
1004 3.7125097662738e-08
1005 3.70972372820688e-08
1006 3.728623099164e-08
1007 3.41697017347542e-08
1008 3.45321247152697e-08
1009 3.41539063697383e-08
1010 3.73815787213516e-08
1011 3.64440246869435e-08
1012 3.62941925402538e-08
1013 3.62866323655453e-08
1014 3.62279983789904e-08
1015 3.61694780792732e-08
1016 3.61040370933097e-08
1017 3.66384647065843e-08
1018 3.40248149655054e-08
1019 3.39311334585091e-08
1020 3.3953575950818e-08
1021 3.67385268873477e-08
1022 3.66050620925762e-08
1023 3.34928635936649e-08
1024 3.62765177897018e-08
1025 3.61276129012822e-08
1026 3.60683216626967e-08
1027 3.36633014796917e-08
1028 3.37016601292817e-08
1029 3.39055965525858e-08
1030 3.35593348665952e-08
1031 3.3773893903799e-08
1032 3.3366326590567e-08
1033 3.33532064189512e-08
1034 3.33011627162705e-08
1035 3.60361482876215e-08
1036 3.60233514129504e-08
1037 3.36045111737349e-08
1038 3.31802532116399e-08
1039 3.31357519200992e-08
1040 3.30839817763717e-08
1041 3.3058483950299e-08
1042 3.30017790872716e-08
1043 3.29818412581062e-08
1044 3.30319345209773e-08
1045 3.28745741740022e-08
1046 3.57471208189963e-08
1047 3.28612266287109e-08
1048 3.26550271267934e-08
1049 3.26944551432007e-08
1050 3.263267700504e-08
1051 3.25534941225669e-08
1052 3.25606777096255e-08
1053 3.24827063025168e-08
1054 3.25206173101833e-08
1055 3.24286268948981e-08
1056 3.23917568323395e-08
1057 3.23682520786406e-08
1058 3.23066018381724e-08
1059 3.56208822438475e-08
1060 3.21553237370154e-08
1061 3.21964996885526e-08
1062 3.21421822491175e-08
1063 3.21192743513166e-08
1064 3.56390970068787e-08
1065 3.21169224548612e-08
1066 3.18065893623043e-08
1067 3.15145456397659e-08
1068 3.1783077503178e-08
1069 3.17438200170272e-08
1070 3.134487158718e-08
1071 3.14037222892694e-08
1072 3.13794501494158e-08
1073 3.13514725291952e-08
1074 3.13316661504359e-08
1075 3.152014116381e-08
1076 3.1257613386515e-08
1077 3.12611270203433e-08
1078 3.1264061561842e-08
1079 3.11955545839737e-08
1080 3.15075787682417e-08
1081 3.14932506739751e-08
1082 3.11032550825985e-08
1083 3.14845927107399e-08
1084 3.0826321051336e-08
1085 3.08867136311619e-08
1086 3.13642871674347e-08
1087 3.0881459167631e-08
1088 3.5681157584122e-08
1089 3.11701491284566e-08
1090 3.06195460098024e-08
1091 3.0710868514916e-08
1092 3.11846442002661e-08
1093 3.06774410319122e-08
1094 3.11901793281777e-08
1095 3.52862947750054e-08
1096 3.52285454141565e-08
1097 3.52598945596583e-08
1098 3.11130889940614e-08
1099 3.09283194610543e-08
1100 3.10481809151497e-08
1101 3.52623246158146e-08
1102 3.10076657683567e-08
1103 3.09726360114837e-08
1104 3.53055966684224e-08
1105 3.52295934646918e-08
1106 3.50709647989333e-08
1107 3.0803580131078e-08
1108 3.45685968738962e-08
1109 3.48123663229671e-08
1110 3.47189477167831e-08
1111 3.08860990116955e-08
1112 3.46617916591185e-08
1113 3.45677122481902e-08
1114 3.0617947288647e-08
1115 3.47170470149649e-08
1116 3.05468219607974e-08
1117 3.43993491469519e-08
1118 3.4449662678071e-08
1119 3.4321313790997e-08
1120 3.44519186512571e-08
1121 3.01061398033653e-08
1122 3.03405229828968e-08
1123 3.42443051692953e-08
1124 3.41052128760566e-08
1125 3.00226226102041e-08
1126 3.03986205096862e-08
1127 3.3975421587229e-08
1128 3.39081438482935e-08
1129 3.39534089732751e-08
1130 3.00558014032504e-08
1131 3.39495755952157e-08
1132 3.37111707437998e-08
1133 2.98605371540361e-08
1134 3.38716290571028e-08
1135 2.98334157378122e-08
1136 2.98447204727381e-08
1137 3.36047989435428e-08
1138 3.37287922036467e-08
1139 2.97946112226555e-08
1140 3.35855858679679e-08
1141 3.34251986089384e-08
1142 2.9604835916075e-08
1143 2.97560447393153e-08
1144 3.33675664876409e-08
1145 3.33504139859997e-08
1146 2.94339130846311e-08
1147 3.28701936780362e-08
1148 3.27841114255989e-08
1149 3.26973399467079e-08
1150 2.90594144303213e-08
1151 3.2648969749971e-08
1152 3.24451789879276e-08
1153 2.90188406637526e-08
1154 3.25916005294857e-08
1155 3.24300337695149e-08
1156 3.22516484629887e-08
1157 2.89921402440996e-08
1158 3.26091615932e-08
1159 3.2298597574254e-08
1160 3.23174162986106e-08
1161 2.89254380447801e-08
1162 3.2245207393089e-08
1163 3.2106040492863e-08
1164 2.89008816878322e-08
1165 3.2137275951527e-08
1166 3.21820650128757e-08
1167 3.20638733342093e-08
1168 3.18783008879109e-08
1169 2.84132397609937e-08
1170 3.20150306265532e-08
1171 3.1980036396817e-08
1172 3.16493888874447e-08
1173 3.1618725415683e-08
1174 2.84470242917223e-08
1175 3.17171782171499e-08
1176 3.17098560742579e-08
1177 3.15213668500292e-08
1178 2.83448127191832e-08
1179 3.16249071374841e-08
1180 3.16940322875325e-08
1181 3.16247366072275e-08
1182 2.83915344567731e-08
1183 3.14497228259825e-08
1184 3.15534975925402e-08
1185 3.14243955301663e-08
1186 2.80660756857287e-08
1187 3.14208286056328e-08
1188 3.14450439020675e-08
1189 3.13704404675264e-08
1190 2.84656902493907e-08
1191 3.14816155366771e-08
1192 3.11396348706694e-08
1193 3.1294600688625e-08
1194 2.7874998309585e-08
1195 3.13972243759508e-08
1196 3.10664347580314e-08
1197 3.11570467204092e-08
1198 3.09603827020055e-08
1199 2.78576788304008e-08
1200 2.98580253854652e-08
1201 2.97131848014942e-08
1202 2.96151139167478e-08
1203 2.98413098676065e-08
1204 2.95829920560209e-08
1205 2.76515965680346e-08
1206 2.96276407851792e-08
1207 2.94573272441312e-08
1208 2.93820932029121e-08
1209 2.76981424462974e-08
1210 2.96426012624806e-08
1211 2.93290458586171e-08
1212 2.92092998677163e-08
1213 2.77460134867624e-08
1214 2.9317742900048e-08
1215 2.9033934367817e-08
1216 2.90663599855634e-08
1217 2.92036457238964e-08
1218 2.92088611075769e-08
1219 2.73511329140774e-08
1220 2.91602404445257e-08
1221 2.90244628331493e-08
1222 2.90941475356021e-08
1223 2.88239210277652e-08
1224 2.7232532673338e-08
1225 2.8940105423203e-08
1226 2.87344850136151e-08
1227 2.87841075419237e-08
1228 2.70209916664044e-08
1229 2.87071788562798e-08
1230 2.86208052813208e-08
1231 2.87864008186034e-08
1232 2.75821108175478e-08
1233 2.86690848838589e-08
1234 2.86963128814932e-08
1235 2.88117778524111e-08
1236 2.87015851085926e-08
1237 2.73573288467333e-08
1238 2.8536055296513e-08
1239 2.87464771986379e-08
1240 2.84303691699961e-08
1241 2.8592255674198e-08
1242 2.71098006265902e-08
1243 2.83471539574975e-08
1244 2.85924883769439e-08
1245 2.82804606399623e-08
1246 2.82517085281597e-08
1247 2.82003824736421e-08
1248 2.70259157275632e-08
1249 2.80145311393198e-08
1250 2.82711845045469e-08
1251 2.81463670148696e-08
1252 2.8103768201504e-08
1253 2.79886442910993e-08
1254 2.65553889988723e-08
1255 2.78211125248617e-08
1256 2.79237291067602e-08
1257 2.76294134238242e-08
1258 2.79080740739346e-08
1259 2.76695804046767e-08
1260 2.6985933487822e-08
1261 2.74877525185957e-08
1262 2.76697740275722e-08
1263 2.73821427754228e-08
1264 2.74791922549866e-08
1265 2.74911577946568e-08
1266 2.64557264983978e-08
1267 2.73450186938362e-08
1268 2.73758882229913e-08
1269 2.73922537985527e-08
1270 2.69852815648619e-08
1271 2.70048463590911e-08
1272 2.65193058623936e-08
1273 2.67762061412213e-08
1274 2.6714568335251e-08
1275 2.68318611773566e-08
1276 2.69272018016409e-08
1277 2.66941206916727e-08
1278 2.68413273829538e-08
1279 2.67144262267038e-08
1280 2.67936570708116e-08
1281 2.67348632121411e-08
1282 2.66148383332165e-08
1283 2.65790571774005e-08
1284 2.66015760530536e-08
1285 2.64489923296196e-08
1286 2.66536321902322e-08
1287 2.64235744396046e-08
1288 2.6330681635045e-08
1289 2.63341188855293e-08
1290 2.63460169236396e-08
1291 2.64694595131232e-08
1292 2.63580783865791e-08
1293 2.66545647775729e-08
1294 2.65813646649349e-08
1295 2.63258854715787e-08
1296 2.60921808603598e-08
1297 2.65996540349533e-08
1298 2.63351989104876e-08
1299 2.68192348329421e-08
1300 2.60955310693589e-08
1301 2.6001414354937e-08
1302 2.63300599101512e-08
1303 2.66582631525125e-08
1304 2.61208601415319e-08
1305 2.6644398687381e-08
1306 2.65943569388583e-08
1307 2.64475925604302e-08
1308 2.65048605285756e-08
1309 2.65620805350864e-08
1310 2.60937547125195e-08
1311 2.63097934549705e-08
1312 2.57708396844691e-08
1313 2.62764583425223e-08
1314 2.57828993710518e-08
1315 2.61846899718421e-08
1316 2.63804977862492e-08
1317 2.6176770973052e-08
1318 2.63861394955711e-08
1319 2.58896015736809e-08
1320 2.60771368942869e-08
1321 2.59546943937039e-08
1322 2.63090349506001e-08
1323 2.63131259004012e-08
1324 2.56863934566809e-08
1325 2.59335575236719e-08
1326 2.62323780475526e-08
1327 2.60355115244693e-08
1328 2.61199559759007e-08
1329 2.57222332322726e-08
1330 2.60919748029664e-08
1331 2.59334651531162e-08
1332 2.53243417347448e-08
1333 2.59335255492488e-08
1334 2.59198014163076e-08
1335 2.55963730211306e-08
1336 2.58746695180889e-08
1337 2.57408085957422e-08
1338 2.57675552006731e-08
1339 2.53098715319311e-08
1340 2.603850113303e-08
1341 2.58764583094262e-08
1342 2.54342431560417e-08
1343 2.57238124135029e-08
1344 2.57425689653701e-08
1345 2.56124224051746e-08
1346 2.57266741243711e-08
1347 2.57360159849895e-08
1348 2.56346535110197e-08
1349 2.56162540068772e-08
1350 2.56006416066157e-08
1351 2.56965329015202e-08
1352 2.55422882844414e-08
1353 2.56223415817658e-08
1354 2.55401459980931e-08
1355 2.55414658312247e-08
1356 2.57542360770913e-08
1357 2.58607251168996e-08
1358 2.57125378766432e-08
1359 2.58230503646928e-08
1360 2.55883438882165e-08
1361 2.56294274691982e-08
1362 2.55955896477644e-08
1363 2.5663585034863e-08
1364 2.57113441648471e-08
1365 2.55343959310039e-08
1366 2.59032244542823e-08
1367 2.55661820602882e-08
1368 2.54625103224271e-08
1369 2.57425742944406e-08
1370 2.56218406491371e-08
1371 2.56199186310369e-08
1372 2.55981156271901e-08
1373 2.5719419483039e-08
1374 2.57129926239941e-08
1375 2.55593803899501e-08
1376 2.54853933512322e-08
1377 2.56765666506453e-08
1378 2.59434287386284e-08
1379 2.5864990149671e-08
1380 2.57729073638302e-08
1381 2.58321133372874e-08
1382 2.5726830443773e-08
1383 2.56026577716284e-08
1384 2.5689068650081e-08
1385 2.57855621299541e-08
1386 2.57319587859683e-08
1387 2.65115858155696e-08
1388 2.55863454867722e-08
1389 2.53480614276214e-08
1390 2.70565791993249e-08
1391 2.61183430438905e-08
1392 2.60024695108996e-08
1393 2.6261563590424e-08
1394 2.58398458186093e-08
1395 2.57211354437459e-08
1396 2.60717349931383e-08
1397 2.63233452812983e-08
1398 2.55475995913912e-08
1399 2.55351189082376e-08
1400 2.53991654375341e-08
1401 2.56673473586488e-08
1402 2.7240961486541e-08
1403 2.54864200854854e-08
1404 2.63542201395239e-08
1405 2.72204054851954e-08
1406 2.57815244708581e-08
1407 2.55295571349734e-08
1408 2.5830470207211e-08
1409 2.55197818432862e-08
1410 2.61444572657865e-08
1411 2.55043151042855e-08
1412 2.58549111009643e-08
1413 2.54515466480143e-08
1414 2.50515110877814e-08
1415 2.56055390224219e-08
1416 2.54836720614549e-08
1417 2.5255243230049e-08
1418 2.53403111827311e-08
1419 2.52925573818175e-08
1420 2.53555683116247e-08
1421 2.52544101186913e-08
1422 2.52540726108919e-08
1423 2.51818423890882e-08
1424 2.50661233991423e-08
1425 2.50219649444716e-08
1426 2.52161242997317e-08
1427 2.53140886030678e-08
1428 2.53359484503335e-08
1429 2.53746197387272e-08
1430 2.61945665158692e-08
1431 2.55860221898274e-08
1432 2.52209577666918e-08
1433 2.55178971286796e-08
1434 2.52400589317858e-08
1435 2.51388279082221e-08
1436 2.54411034461555e-08
1437 2.50103866505924e-08
1438 2.5321165608716e-08
1439 2.52264946709602e-08
1440 2.542047994325e-08
1441 2.49992702094914e-08
1442 2.53231000613141e-08
1443 2.51322909150531e-08
1444 2.52529304134441e-08
1445 2.47604088343678e-08
1446 2.56593590819421e-08
1447 2.49081182346345e-08
1448 2.54851606484863e-08
1449 2.51885960977916e-08
1450 2.54816647782263e-08
1451 2.533101017832e-08
1452 2.53156819951528e-08
1453 2.47072051706709e-08
1454 2.52309462211997e-08
1455 2.53875391820202e-08
1456 2.53530956229042e-08
1457 2.50903955389958e-08
1458 2.57079637577817e-08
1459 2.49590659251453e-08
1460 2.60142396513174e-08
1461 2.53705056962872e-08
1462 2.55257219805571e-08
1463 2.52809968515066e-08
1464 2.57322909646973e-08
1465 2.53826772933508e-08
1466 2.58475338910102e-08
1467 2.52734579930802e-08
1468 2.51198795098162e-08
1469 2.51090774838758e-08
1470 2.55022882811318e-08
1471 2.56909906681813e-08
1472 2.52616292328867e-08
1473 2.52765506303376e-08
1474 2.51370799730921e-08
1475 2.52111682641498e-08
1476 2.53724312671011e-08
1477 2.51197054268459e-08
1478 2.47073135284381e-08
1479 2.53053471510611e-08
1480 2.50434357695895e-08
1481 2.49326834733665e-08
1482 2.56119125907617e-08
1483 2.47882372406139e-08
1484 2.49924241302324e-08
1485 2.47013591803125e-08
1486 2.51528913253196e-08
1487 2.498293838471e-08
1488 2.54467931171121e-08
1489 2.50908520627036e-08
1490 2.47185685253726e-08
1491 2.46890845545522e-08
1492 2.49105429617202e-08
1493 2.48890597021045e-08
1494 2.48215243914274e-08
1495 2.51485090529968e-08
1496 2.46686564508991e-08
1497 2.46647680057777e-08
1498 2.45441960089465e-08
1499 2.46981866069973e-08
1500 2.45508786633764e-08
1501 2.49608991254036e-08
1502 2.45829134826181e-08
1503 2.45430538114988e-08
1504 2.44338007604483e-08
1505 2.45182523173071e-08
1506 2.45946356614013e-08
1507 2.546857302832e-08
1508 2.46913600676635e-08
1509 2.45183251479375e-08
1510 2.46048799112941e-08
1511 2.48207978614801e-08
1512 2.46363835998409e-08
1513 2.47121008101203e-08
1514 2.45041675839275e-08
1515 2.46929019454001e-08
1516 2.46993607788681e-08
1517 2.49200517998815e-08
1518 2.48851321771326e-08
1519 2.46845921481054e-08
1520 2.47943283682162e-08
1521 2.48397462598859e-08
1522 2.45314026869892e-08
1523 2.49307028354906e-08
1524 2.4545540711074e-08
1525 2.50061269468915e-08
1526 2.45479530036619e-08
1527 2.48368596800219e-08
1528 2.4896232631022e-08
1529 2.45811246912808e-08
1530 2.49795810702835e-08
1531 2.47608511472208e-08
1532 2.47204905434728e-08
1533 2.42722446586185e-08
1534 2.48119302881378e-08
1535 2.4353614236361e-08
1536 2.44424160911194e-08
1537 2.40906707915656e-08
1538 2.46377229728978e-08
1539 2.48697684668286e-08
1540 2.42076136913738e-08
1541 2.4475525606249e-08
1542 2.41083597529723e-08
1543 2.42259812210932e-08
1544 2.43541506961265e-08
1545 2.42705429087664e-08
1546 2.43124986809562e-08
1547 2.4258712372216e-08
1548 2.40463808864888e-08
1549 2.42224356128418e-08
1550 2.42507383063639e-08
1551 2.41617890139878e-08
1552 2.40135875628766e-08
1553 2.43910971420291e-08
1554 2.42709390363416e-08
1555 2.42483757517675e-08
1556 2.39598119122775e-08
1557 2.43613218486871e-08
1558 2.40586590649627e-08
1559 2.42898217095444e-08
1560 2.41341844287035e-08
1561 2.43936479904505e-08
1562 2.40214390601068e-08
1563 2.4280225829898e-08
1564 2.43536977251324e-08
1565 2.42750850532047e-08
1566 2.4040936352776e-08
1567 2.4191628256176e-08
1568 2.42948967610346e-08
1569 2.39723743078457e-08
1570 2.39305215643526e-08
1571 2.42796733829209e-08
1572 2.43535822619378e-08
1573 2.4397094122719e-08
1574 2.40865816181213e-08
1575 2.43002471478349e-08
1576 2.59492907161984e-08
1577 2.37546657899657e-08
1578 2.42106263925734e-08
1579 2.42341755551934e-08
1580 2.40797728423559e-08
1581 2.3847926300391e-08
1582 2.39707702576197e-08
1583 2.37093402688515e-08
1584 2.38625794679592e-08
1585 2.38414727959935e-08
1586 2.29700365395047e-08
1587 2.32882335637896e-08
1588 2.35641053336622e-08
1589 2.33900117052599e-08
1590 2.32109407249936e-08
1591 2.34166659396351e-08
1592 2.3263066140089e-08
1593 2.32700063662605e-08
1594 2.33685053530053e-08
1595 2.31513723747412e-08
1596 2.30429382241937e-08
1597 2.3306490959385e-08
1598 2.31462333744048e-08
1599 2.31334169598085e-08
1600 2.28609007280056e-08
1601 2.27910295080846e-08
1602 2.32147936429783e-08
1603 2.28036505234286e-08
1604 2.29760548364766e-08
1605 2.29980621213599e-08
1606 2.26682050907812e-08
1607 2.2715923364558e-08
1608 2.30856862515338e-08
1609 2.29561525344479e-08
1610 2.29339764956649e-08
1611 2.29911165661179e-08
1612 2.28168701710274e-08
1613 2.30428476299949e-08
1614 2.27185861234602e-08
1615 2.24251852642965e-08
1616 2.18973230659003e-08
1617 2.27526353313579e-08
1618 2.24669260973087e-08
1619 2.2406840827216e-08
1620 2.3151860872872e-08
1621 2.23546603450586e-08
1622 2.25557368338514e-08
1623 2.23989484737785e-08
1624 2.23094431817117e-08
1625 2.23661285048138e-08
1626 2.26707683737004e-08
1627 2.24927330094715e-08
1628 2.25503811179806e-08
1629 2.25777778695146e-08
1630 2.20968363606744e-08
1631 2.25736513925767e-08
1632 2.24333813747535e-08
1633 2.26371188460917e-08
1634 2.20818225926678e-08
1635 2.23888729777855e-08
1636 2.20105746961963e-08
1637 2.25430092370971e-08
1638 2.23423732848005e-08
1639 2.22880451872243e-08
1640 2.20726938948701e-08
1641 2.20138165474282e-08
1642 2.20715961063433e-08
1643 2.22245351011452e-08
1644 2.23838956259215e-08
1645 2.22368470303991e-08
1646 2.21138538591958e-08
1647 2.22540670336002e-08
1648 2.21200213701422e-08
1649 2.20361151548332e-08
1650 2.22381952852402e-08
1651 2.2417392386842e-08
1652 2.19625100328358e-08
1653 2.1916317649584e-08
1654 2.22276845818214e-08
1655 2.18489848435865e-08
1656 2.19335127837894e-08
1657 2.20848654919337e-08
1658 2.20763389791045e-08
1659 2.19311804272593e-08
1660 2.1898365787365e-08
1661 2.16915765349768e-08
1662 2.21364793162593e-08
1663 2.18330402645961e-08
1664 2.18532765217105e-08
1665 2.20053077981675e-08
1666 2.17106261857225e-08
1667 2.18242899308052e-08
1668 2.23769731633183e-08
1669 2.18172342414391e-08
1670 2.18616289515694e-08
1671 2.21947420442348e-08
1672 2.18137721219591e-08
1673 2.19750706520472e-08
1674 2.19319957750486e-08
1675 2.16943512043599e-08
1676 2.20480025348024e-08
1677 2.1894495105812e-08
1678 2.17160742721489e-08
1679 2.16967421806658e-08
1680 2.14072581883329e-08
1681 2.16136548658596e-08
1682 2.21221494456358e-08
1683 2.16296758281942e-08
1684 2.13140225469033e-08
1685 2.15570565842427e-08
1686 2.18334044177482e-08
1687 2.16713154088666e-08
1688 2.21407017164665e-08
1689 2.16495390503724e-08
1690 2.17160867066468e-08
1691 2.16039985900807e-08
1692 2.15677999904074e-08
1693 2.1912502035093e-08
1694 2.14922302177456e-08
1695 2.13948876393033e-08
1696 2.13430659812275e-08
1697 2.16014406362319e-08
1698 2.15848334761404e-08
1699 2.17870930185882e-08
1700 2.12642614627612e-08
1701 2.16483471149331e-08
1702 2.16043964940127e-08
1703 2.17418705261707e-08
1704 2.17356568299465e-08
1705 2.15363851197026e-08
1706 2.28279031233569e-08
1707 2.19900524456307e-08
1708 2.21396376787197e-08
1709 2.14696367351053e-08
1710 2.16042881362455e-08
1711 2.12749551309344e-08
1712 2.12057944537491e-08
1713 2.12551078959677e-08
1714 2.14380886376375e-08
1715 2.15892512756e-08
1716 2.11634603175526e-08
1717 2.20999307742886e-08
1718 2.11506492320268e-08
1719 2.14071391724246e-08
1720 2.12547455191725e-08
1721 2.14628048667009e-08
1722 2.13067803400691e-08
1723 2.11453574650022e-08
1724 2.12128661303268e-08
1725 2.13223145806296e-08
1726 2.16526672147666e-08
1727 2.10841513137439e-08
1728 2.12413642231013e-08
1729 2.11749657808014e-08
1730 2.08169268489655e-08
1731 2.14614406246483e-08
1732 2.10952251222807e-08
1733 2.13800657178354e-08
1734 2.11874482403118e-08
1735 2.19967439818447e-08
1736 2.12224176010523e-08
1737 2.19637197318434e-08
1738 2.19286278024811e-08
1739 2.12281978662077e-08
1740 2.11030535268719e-08
1741 2.17998987750434e-08
1742 2.09532799999579e-08
1743 2.07956798448095e-08
1744 2.14478852456068e-08
1745 2.17882494268906e-08
1746 2.1556854079563e-08
1747 2.17651781042605e-08
1748 2.21045901582784e-08
1749 2.14817088561858e-08
1750 2.20294200659055e-08
1751 2.16528164287411e-08
1752 2.15088107324846e-08
1753 2.16859099566591e-08
1754 2.14850377489029e-08
1755 2.13545678917626e-08
1756 2.15499689204535e-08
1757 2.10409645262644e-08
1758 2.22548131034728e-08
1759 2.07353068049088e-08
1760 2.13982040975225e-08
1761 2.06740065067379e-08
1762 2.12549693401343e-08
1763 2.1000740701993e-08
1764 2.14107362950244e-08
1765 2.07268548990669e-08
1766 2.20174154463848e-08
1767 2.10459116800621e-08
1768 2.0730016814241e-08
1769 2.09582751153903e-08
1770 2.07426840148628e-08
1771 2.04800443270869e-08
1772 2.08169286253224e-08
1773 2.07384829309376e-08
1774 2.15231832356721e-08
1775 2.06124539658958e-08
1776 2.0729142846676e-08
1777 2.07029682286475e-08
1778 2.08160884085373e-08
1779 2.06513384171103e-08
1780 2.070025217904e-08
1781 2.04215861998591e-08
1782 2.03437231505177e-08
1783 2.06590868856438e-08
1784 2.07424744047557e-08
1785 2.04917878221522e-08
1786 2.03518180086348e-08
1787 2.06708214989249e-08
1788 2.01957064405178e-08
1789 2.04332142317298e-08
1790 2.08476862439966e-08
1791 2.03859649161586e-08
1792 2.04044159346495e-08
1793 2.01875884897618e-08
1794 2.05840358091791e-08
1795 2.09552872831864e-08
1796 2.0567600955701e-08
1797 2.04460270936124e-08
1798 2.08882671159927e-08
1799 2.07208437075224e-08
1800 2.0298392300333e-08
1801 2.06884411824149e-08
1802 2.05553618570775e-08
1803 2.07514361250105e-08
1804 2.0531741640184e-08
1805 2.03301322443394e-08
1806 2.03061585324349e-08
1807 2.05678798437248e-08
1808 2.03082706207169e-08
1809 2.03676808752107e-08
1810 2.00176195619406e-08
1811 2.01562926349652e-08
1812 2.01127718923999e-08
1813 2.0059026439867e-08
1814 2.0176226911417e-08
1815 2.00015222162619e-08
1816 1.98184082478292e-08
1817 2.00073486666952e-08
1818 1.98133154327707e-08
1819 2.02227905532482e-08
1820 1.98359728642572e-08
1821 2.02177936614589e-08
1822 2.00119902871165e-08
1823 2.0058916305743e-08
1824 2.09313721910576e-08
1825 2.04968984007792e-08
1826 2.02634815593683e-08
1827 2.00003089645406e-08
1828 1.98559231279205e-08
1829 1.95714608963726e-08
1830 1.98936138673389e-08
1831 1.95078584397379e-08
1832 2.03455492453486e-08
1833 1.9775693971269e-08
1834 2.00045793263826e-08
1835 1.98977847531978e-08
1836 1.96903293669948e-08
1837 1.97785574584941e-08
1838 1.99427070413094e-08
1839 2.00134611105796e-08
1840 1.9273059592706e-08
1841 1.96329601465095e-08
1842 1.94045224333195e-08
1843 1.93518019386829e-08
1844 1.97051726047448e-08
1845 1.95925071722058e-08
1846 1.96169374078181e-08
1847 1.92487785710682e-08
1848 1.97206500018865e-08
1849 1.89598949873471e-08
1850 1.90500504260172e-08
1851 1.90786071385673e-08
1852 1.9366213521721e-08
1853 1.93399714021325e-08
1854 1.90757969420474e-08
1855 1.94914786533218e-08
1856 1.95974525496467e-08
1857 1.92288460709733e-08
1858 1.91267783833382e-08
1859 1.94217122384543e-08
1860 1.92169942181408e-08
1861 1.88786692945087e-08
1862 1.93731324316104e-08
1863 1.90090627683048e-08
1864 1.89796605098991e-08
1865 1.86317290484794e-08
1866 1.94380334050948e-08
1867 1.88938162892782e-08
1868 1.9184540178685e-08
1869 1.88766957620601e-08
1870 1.87320026157067e-08
1871 1.92725266856542e-08
1872 1.87225026593296e-08
1873 1.86767117327236e-08
1874 1.88640250087246e-08
1875 1.91713755981482e-08
1876 1.9002822426728e-08
1877 1.89930542404682e-08
1878 1.87305779775215e-08
1879 1.89523028382155e-08
1880 1.88021882507883e-08
1881 1.89765305691481e-08
1882 1.88841884352087e-08
1883 1.87070341439721e-08
1884 1.89133260164454e-08
1885 1.87384383565359e-08
1886 1.89122140170639e-08
1887 1.91560314277694e-08
1888 1.86353226183655e-08
1889 1.85981043898664e-08
1890 1.85829094334622e-08
1891 1.87190192235676e-08
1892 1.86730364504228e-08
1893 1.90795717003311e-08
1894 1.87494215708739e-08
1895 1.86046928973838e-08
1896 1.89706916842169e-08
1897 1.89176159182125e-08
1898 1.8775402566007e-08
1899 1.88127735611943e-08
1900 1.88738198403371e-08
1901 1.88554398761198e-08
1902 1.91236164681641e-08
1903 1.8838182569425e-08
1904 1.88409661205924e-08
1905 1.83658972474632e-08
1906 1.89187652210876e-08
1907 1.82784152258364e-08
1908 1.88139086532146e-08
1909 1.868390775428e-08
1910 1.87244104665751e-08
1911 1.86485742403875e-08
1912 1.84438437855761e-08
1913 1.87113418093077e-08
1914 1.81547203936816e-08
1915 1.83212502946617e-08
1916 1.87725319733545e-08
1917 1.85156387999541e-08
1918 1.85902866434162e-08
1919 1.84315531726043e-08
1920 1.82133614856639e-08
1921 1.83302368839122e-08
1922 1.82415007543568e-08
1923 1.85190316415174e-08
1924 1.80887340661684e-08
1925 1.80877037792015e-08
1926 1.80874675237419e-08
1927 1.83327788505494e-08
1928 1.8124586276258e-08
1929 1.84027051375324e-08
1930 1.86218418463113e-08
1931 1.86859701045705e-08
1932 1.85502120331194e-08
1933 1.87525106554176e-08
1934 1.85649202677496e-08
1935 1.82277553051335e-08
1936 1.85315798262309e-08
1937 1.84675954528757e-08
1938 1.81015646916194e-08
1939 1.86922726186367e-08
1940 1.84373316614028e-08
1941 1.85446804579215e-08
1942 1.87122068950885e-08
1943 1.82570953910499e-08
1944 1.8627220654821e-08
1945 1.87827069453306e-08
1946 1.80086843215577e-08
1947 1.87112743077478e-08
1948 1.86977011651379e-08
1949 1.85578663547403e-08
1950 1.87150082098242e-08
1951 1.86043731531527e-08
1952 1.81767436657765e-08
1953 1.82119208602671e-08
1954 1.84490858146091e-08
1955 1.89618560852978e-08
1956 1.8208261565178e-08
1957 1.85106099337418e-08
1958 1.86958111214608e-08
1959 1.86123720880005e-08
1960 1.88137807555222e-08
1961 1.82411739046984e-08
1962 1.89133757544369e-08
1963 1.86394668588719e-08
1964 1.89282740592489e-08
1965 1.8791606493096e-08
1966 1.83290360666888e-08
1967 1.86864781426266e-08
1968 1.86773405630447e-08
1969 1.86407316249415e-08
1970 1.81765038576032e-08
1971 1.84658421886752e-08
1972 1.83697004274563e-08
1973 1.83343704662775e-08
1974 1.86463875451182e-08
1975 1.83037798251462e-08
1976 1.8549419777969e-08
1977 1.82732247111517e-08
1978 1.95831830751558e-08
1979 1.86300148641294e-08
1980 1.86204118790556e-08
1981 1.88864515138221e-08
1982 1.85951094522352e-08
1983 1.8403904178399e-08
1984 1.84855242224558e-08
1985 1.8623261155426e-08
1986 1.84326456320605e-08
1987 1.96151326292693e-08
1988 1.84310096074114e-08
1989 1.81845329905173e-08
1990 1.85508373107268e-08
1991 1.85571114030836e-08
1992 1.82371504564571e-08
1993 1.99235223874439e-08
1994 1.82261086223434e-08
1995 1.81901356199887e-08
1996 1.99807566048094e-08
1997 1.87807369655957e-08
1998 1.88767472764084e-08
1999 1.86168964688704e-08
2000 1.85268866914612e-08
2001 1.86974666860351e-08
2002 1.84424973070918e-08
2003 1.84845543316214e-08
2004 1.87280697616643e-08
2005 1.87467801282537e-08
2006 1.84402626501878e-08
2007 1.8978123961233e-08
2008 1.8903165255324e-08
2009 1.93519547053711e-08
2010 1.95249487688898e-08
2011 1.87039681520673e-08
2012 1.85694908338974e-08
2013 1.94231812855605e-08
2014 1.87842594812082e-08
2015 1.90684659173712e-08
2016 1.82842256890581e-08
2017 1.896972889881e-08
2018 1.84788682133785e-08
2019 1.92695566170187e-08
2020 1.83072419446262e-08
2021 1.8558987235906e-08
2022 1.84710913231356e-08
2023 1.77001755474748e-08
2024 1.87115460903442e-08
2025 1.92026607948037e-08
2026 1.83139210463423e-08
2027 1.89215327850434e-08
2028 1.85769000182745e-08
2029 1.87648225846715e-08
2030 1.87070163804037e-08
2031 1.85570385724532e-08
2032 1.80498069823898e-08
2033 1.86872011198602e-08
2034 1.81971913093548e-08
2035 1.87257160888521e-08
2036 1.84668671465715e-08
2037 1.83502990580564e-08
2038 1.85317610146285e-08
2039 1.86010833402861e-08
2040 1.80303292296458e-08
2041 1.80185697473689e-08
2042 1.89614919321457e-08
2043 1.90040552183746e-08
2044 1.80455206333363e-08
2045 1.81706969470952e-08
2046 1.90307609670981e-08
2047 1.72715104440613e-08
2048 1.79528374388838e-08
2049 1.83231172456999e-08
2050 1.83073609605344e-08
2051 1.91066593657752e-08
2052 1.87412929619768e-08
2053 1.84617192644509e-08
2054 1.79759176432981e-08
2055 1.78376140524961e-08
2056 1.73343295273298e-08
2057 1.73679790549386e-08
2058 1.81363848383853e-08
2059 1.75686594161562e-08
2060 1.78340382461784e-08
2061 1.77548713509168e-08
2062 1.86867197271567e-08
2063 1.78360419766932e-08
2064 1.74880021575063e-08
2065 1.75609606856142e-08
2066 1.83092279115726e-08
2067 1.9215919522253e-08
2068 1.90820479417653e-08
2069 1.77780066223931e-08
2070 1.78804349104666e-08
2071 1.79271779643386e-08
2072 1.79797865484943e-08
2073 1.92865936554654e-08
2074 1.81017743017264e-08
2075 1.78963190933246e-08
2076 1.85734272406535e-08
2077 1.82965820272329e-08
2078 1.73764647115604e-08
2079 1.76785288630299e-08
2080 1.84176034423444e-08
2081 1.89487696644619e-08
2082 1.76784098471217e-08
2083 1.77220638164499e-08
2084 1.78022450114668e-08
2085 1.89057285382432e-08
2086 1.7666470952804e-08
2087 1.80645223224474e-08
2088 1.85984223577407e-08
2089 1.79462951166443e-08
2090 1.8036974580582e-08
2091 1.82959354333434e-08
2092 1.74329173319165e-08
2093 1.77242700516445e-08
2094 1.9287858421535e-08
2095 1.80421135809183e-08
2096 1.79233765607023e-08
2097 1.74053411683417e-08
2098 1.76955303743398e-08
2099 1.73471939035608e-08
2100 1.82935551151786e-08
2101 1.79318373483284e-08
2102 1.75260055357285e-08
2103 1.81317716396734e-08
2104 1.82153527816808e-08
2105 1.85383566275732e-08
2106 1.79466681515805e-08
2107 1.7458960499539e-08
2108 1.7877233915442e-08
2109 1.75940026991839e-08
2110 1.72238223683507e-08
2111 1.76924856987171e-08
2112 1.92045330749124e-08
2113 1.79644459308292e-08
2114 1.73380740875473e-08
2115 1.74723293611123e-08
2116 1.77030798909072e-08
2117 1.77258705491568e-08
2118 1.76161734088964e-08
2119 1.87871638246406e-08
2120 1.72521499308687e-08
2121 1.75662844270619e-08
2122 1.80832433471778e-08
2123 1.73681744541909e-08
2124 1.7305492150399e-08
2125 1.87864674927596e-08
2126 1.7342733471537e-08
2127 1.70264602417092e-08
2128 1.71675900162427e-08
2129 1.68016534018989e-08
2130 1.81235328966523e-08
2131 1.82982358154504e-08
2132 1.71629039869003e-08
2133 1.68169353997882e-08
2134 1.7109064387455e-08
2135 1.73392784574844e-08
2136 1.73851955054261e-08
2137 1.74980989697815e-08
2138 1.78642221015934e-08
2139 1.78755001911668e-08
2140 1.66722866623559e-08
2141 1.703143936993e-08
2142 1.7457843171087e-08
2143 1.7160308729558e-08
2144 1.63582321022204e-08
2145 1.66717715188724e-08
2146 1.75362142584845e-08
2147 1.71706240337244e-08
2148 1.6893002552365e-08
2149 1.75115992817609e-08
2150 1.72110521390323e-08
2151 1.66056430828121e-08
2152 1.7273615426916e-08
2153 1.70613851935286e-08
2154 1.71647744906522e-08
2155 1.78891017554861e-08
2156 1.70943774691068e-08
2157 1.7097107729569e-08
2158 1.72521428254413e-08
2159 1.75060623774925e-08
2160 1.74615415460266e-08
2161 1.72938481313167e-08
2162 1.72025771405515e-08
2163 1.69225167212517e-08
2164 1.6772760957906e-08
2165 1.7522006956483e-08
2166 1.73263945413282e-08
2167 1.62352975507929e-08
2168 1.69730949295399e-08
2169 1.68697766866899e-08
2170 1.73090128896547e-08
2171 1.75706951210941e-08
2172 1.69309704034504e-08
2173 1.66981681815059e-08
2174 1.63292384058877e-08
2175 1.6233293820278e-08
2176 1.66649556376797e-08
2177 1.69050267118109e-08
2178 1.71690164307847e-08
2179 1.68849680903804e-08
2180 1.68293752267346e-08
2181 1.69720966169962e-08
2182 1.70382250530565e-08
2183 1.64128639568162e-08
2184 1.65145266350919e-08
2185 1.66499685150256e-08
2186 1.63697695398923e-08
2187 1.66089506592471e-08
2188 1.64378466394055e-08
2189 1.70190528336889e-08
2190 1.60481121724843e-08
2191 1.60899862322594e-08
2192 1.64381859235618e-08
2193 1.68975908820812e-08
2194 1.63727520430257e-08
2195 1.65018576581133e-08
2196 1.67876130774403e-08
2197 1.62679327786464e-08
2198 1.64033195915181e-08
2199 1.70644938179976e-08
2200 1.65583191602536e-08
2201 1.59100554952829e-08
2202 1.6516549905532e-08
2203 1.66233817822103e-08
2204 1.67710449971992e-08
2205 1.64548179526491e-08
2206 1.58899151614378e-08
2207 1.62638684741978e-08
2208 1.70389284903649e-08
2209 1.67219695867971e-08
2210 1.71096576906393e-08
2211 1.66319367167489e-08
2212 1.69760845381006e-08
2213 1.69876095412747e-08
2214 1.62630833244748e-08
2215 1.63045470458201e-08
2216 1.68191700566922e-08
2217 1.60938391502441e-08
2218 1.71913345781149e-08
2219 1.64724482942802e-08
2220 1.73403709169406e-08
2221 1.65792783946017e-08
2222 1.66514304567045e-08
2223 1.6981799078053e-08
2224 1.70825469325564e-08
2225 1.66127378520287e-08
2226 1.67849911747453e-08
2227 1.58788058257642e-08
2228 1.6766302124438e-08
2229 1.60475224220136e-08
2230 1.63253819351894e-08
2231 1.62707660678052e-08
2232 1.64320876905322e-08
2233 1.59455275650089e-08
2234 1.66848845850609e-08
2235 1.66021525416227e-08
2236 1.63521267637634e-08
2237 1.70135070476363e-08
2238 1.74035719169296e-08
2239 1.647189762366e-08
2240 1.65102278515405e-08
2241 1.72832965716907e-08
2242 1.64185927076232e-08
2243 1.65437192833906e-08
2244 1.70063589877145e-08
2245 1.63671547426247e-08
2246 1.58761981339239e-08
2247 1.64103628463863e-08
2248 1.65063731571991e-08
2249 1.64767275379063e-08
2250 1.65584648215145e-08
2251 1.63675295539178e-08
2252 1.64223834531185e-08
2253 1.60982338570648e-08
2254 1.63494515703633e-08
2255 1.75731500462462e-08
2256 1.60939155335882e-08
2257 1.65923008665914e-08
2258 1.59186122061783e-08
2259 1.63875562009252e-08
2260 1.68383973431219e-08
2261 1.66844102977848e-08
2262 1.62891815591593e-08
2263 1.6651037881843e-08
2264 1.63841971101419e-08
2265 1.65460907197712e-08
2266 1.66832201387024e-08
2267 1.64967506322e-08
2268 1.64178057815434e-08
2269 1.58334181321607e-08
2270 1.60042930019699e-08
2271 1.66154361380677e-08
2272 1.61915849616889e-08
2273 1.6211073372574e-08
2274 1.63871014535744e-08
2275 1.73135124015289e-08
2276 1.69877871769586e-08
2277 1.60001025761858e-08
2278 1.63440869727083e-08
2279 1.58308601783119e-08
2280 1.6256301194062e-08
2281 1.60947006833112e-08
2282 1.67061990907769e-08
2283 1.62442628237613e-08
2284 1.63661706409357e-08
2285 1.62514997015251e-08
2286 1.61250888197628e-08
2287 1.62979354456638e-08
2288 1.62930700042807e-08
2289 1.67023355146512e-08
2290 1.71196230525084e-08
2291 1.64101408017814e-08
2292 1.68245755105545e-08
2293 1.62800457559342e-08
2294 1.62903628364575e-08
2295 1.66994240657914e-08
2296 1.66872755613667e-08
2297 1.63815432330239e-08
2298 1.65981823840866e-08
2299 1.65345923619498e-08
2300 1.63784239504139e-08
2301 1.63694924282254e-08
2302 1.6330639951434e-08
2303 1.70585234826603e-08
2304 1.62829429939393e-08
2305 1.69631508839529e-08
2306 1.6821063653083e-08
2307 1.61724997838064e-08
2308 1.70025415968666e-08
2309 1.64230300470081e-08
2310 1.6582925255193e-08
2311 1.6791044998854e-08
2312 1.63403583997024e-08
2313 1.63790065954572e-08
2314 1.65284355091444e-08
2315 1.66835629755724e-08
2316 1.6559454252274e-08
2317 1.71569034534969e-08
2318 1.6212400311133e-08
2319 1.62009605730873e-08
2320 1.68260481103744e-08
2321 1.64202482721976e-08
2322 1.62205679998806e-08
2323 1.61819251331963e-08
2324 1.68345604123488e-08
2325 1.64506452904334e-08
2326 1.6897770294122e-08
2327 1.61960258537874e-08
2328 1.62211843957039e-08
2329 1.67321534405573e-08
2330 1.6205012443038e-08
2331 1.65887961145472e-08
2332 1.62388413826875e-08
2333 1.61522883956877e-08
2334 1.60269717497385e-08
2335 1.68077445295012e-08
2336 1.67979266052498e-08
2337 1.64350382192424e-08
2338 1.63144093789924e-08
2339 1.61120201624954e-08
2340 1.60836606255543e-08
2341 1.72111036533806e-08
2342 1.67115654647887e-08
2343 1.65538924790098e-08
2344 1.61819802002583e-08
2345 1.660843196305e-08
2346 1.64033711058664e-08
2347 1.65682454422722e-08
2348 1.6043557593548e-08
2349 1.61205147009014e-08
2350 1.61084798833144e-08
2351 1.721534736987e-08
2352 1.58786335191508e-08
2353 1.66041882465606e-08
2354 1.61773368034801e-08
2355 1.59817865608147e-08
2356 1.63203885961138e-08
2357 1.62540025883118e-08
2358 1.59434652147183e-08
2359 1.61712279123094e-08
2360 1.62471778253348e-08
2361 1.58150434970139e-08
2362 1.64192535123675e-08
2363 1.58975641539882e-08
2364 1.60832200890582e-08
2365 1.58650461656862e-08
2366 1.58137254402391e-08
2367 1.65428328813277e-08
2368 1.57244084419972e-08
2369 1.63898796756712e-08
2370 1.62198539044311e-08
2371 1.58697339713854e-08
2372 1.58503361546991e-08
2373 1.62171183148985e-08
2374 1.58020725393726e-08
2375 1.58047814835527e-08
2376 1.65818310193799e-08
2377 1.68973031122732e-08
2378 1.62962319194548e-08
2379 1.57734820760425e-08
2380 1.54475223723693e-08
2381 1.68594098681751e-08
2382 1.58053143906045e-08
2383 1.66564451120621e-08
2384 1.62516400337154e-08
2385 1.63385553975104e-08
2386 1.64335336449994e-08
2387 1.625031664787e-08
2388 1.54051171818992e-08
2389 1.55286823400047e-08
2390 1.54633550408789e-08
2391 1.59033284319321e-08
2392 1.51886432320225e-08
2393 1.53767238941782e-08
2394 1.52664032526673e-08
2395 1.53394701385423e-08
2396 1.5933304453597e-08
2397 1.53063979269064e-08
2398 1.55559334302779e-08
2399 1.53531978241972e-08
2400 1.50487071692851e-08
2401 1.53618486820051e-08
2402 1.53682293557722e-08
2403 1.5255112728596e-08
2404 1.52713734991039e-08
2405 1.4960553684773e-08
2406 1.55944732682656e-08
2407 1.53753205722751e-08
2408 1.5439905354242e-08
2409 1.52407402254084e-08
2410 1.5384811646868e-08
2411 1.56008379548211e-08
2412 1.54258685824971e-08
2413 1.67074425405644e-08
2414 1.56540771456548e-08
2415 1.51230814537939e-08
2416 1.53190349294618e-08
2417 1.52475436721033e-08
2418 1.53706949390653e-08
2419 1.53382977430283e-08
2420 1.52917589701929e-08
2421 1.52617101178976e-08
2422 1.51793386748977e-08
2423 1.52544377129971e-08
2424 1.57534270073256e-08
2425 1.53308228334481e-08
2426 1.55398574008814e-08
2427 1.55028203607799e-08
2428 1.5146563114854e-08
2429 1.53750363551808e-08
2430 1.51467993703136e-08
2431 1.5311224288439e-08
2432 1.5732760871856e-08
2433 1.55178820904212e-08
2434 1.52719419332925e-08
2435 1.54034704991091e-08
2436 1.54597366019971e-08
2437 1.52849608525685e-08
2438 1.52215751114682e-08
2439 1.53446571005134e-08
2440 1.53166261895876e-08
2441 1.58340629496934e-08
2442 1.53528194601904e-08
2443 1.51949812732255e-08
2444 1.53728638707662e-08
2445 1.5438599731965e-08
2446 1.56300234976925e-08
2447 1.50785570696144e-08
2448 1.51175889584465e-08
2449 1.50900145712285e-08
2450 1.59305102442886e-08
2451 1.52708032885585e-08
2452 1.51318211294438e-08
2453 1.51632058020823e-08
2454 1.52341819159574e-08
2455 1.53615538067697e-08
2456 1.55337680496359e-08
2457 1.6067957631094e-08
2458 1.52726737923103e-08
2459 1.53072168274093e-08
2460 1.50969352574748e-08
2461 1.55286112857311e-08
2462 1.52157930699559e-08
2463 1.53817101278264e-08
2464 1.50005341481574e-08
2465 1.51238133128118e-08
2466 1.50925441033678e-08
2467 1.53160844007516e-08
2468 1.5091147886892e-08
2469 1.52214489901326e-08
2470 1.5285209542526e-08
2471 1.5350854809526e-08
2472 1.52852894785838e-08
2473 1.4986524021765e-08
2474 1.53055967899718e-08
2475 1.52943098186142e-08
2476 1.52099239869585e-08
2477 1.51365231459977e-08
2478 1.55217705355426e-08
2479 1.5033609912507e-08
2480 1.54259272022728e-08
2481 1.52879948700502e-08
2482 1.50564467560343e-08
2483 1.53423940219e-08
2484 1.51675187964884e-08
2485 1.50033443446773e-08
2486 1.47668535177559e-08
2487 1.51705012996217e-08
2488 1.50879948535021e-08
2489 1.50336170179344e-08
2490 1.52784576101794e-08
2491 1.50788057595719e-08
2492 1.52515085005689e-08
2493 1.51225805211652e-08
2494 1.50414951605171e-08
2495 1.5105493744727e-08
2496 1.50087533512533e-08
2497 1.50659253961294e-08
2498 1.50864298831266e-08
2499 1.50288688161027e-08
2500 1.51038950235716e-08
2501 1.4972842521388e-08
2502 1.49985233122152e-08
2503 1.52122598962023e-08
2504 1.52323664792675e-08
2505 1.51778127843727e-08
2506 1.51027563788375e-08
2507 1.52391770313898e-08
2508 1.51205110654473e-08
2509 1.51402197445805e-08
2510 1.53523807000511e-08
2511 1.48624001994335e-08
2512 1.51014365457058e-08
2513 1.51498831257868e-08
2514 1.5012727061503e-08
2515 1.5001312192453e-08
2516 1.50969210466201e-08
2517 1.47896210833665e-08
2518 1.49474335131572e-08
2519 1.50154058076168e-08
2520 1.50012944288846e-08
2521 1.48991654569386e-08
2522 1.48689514034572e-08
2523 1.47674281691934e-08
2524 1.50158410150425e-08
2525 1.49635255297653e-08
2526 1.50033017121132e-08
2527 1.49545904548631e-08
2528 1.50254706454689e-08
2529 1.48185241855003e-08
2530 1.48718406478565e-08
2531 1.49072896249436e-08
2532 1.48294301283158e-08
2533 1.49932883886095e-08
2534 1.49309151709076e-08
2535 1.48429331048305e-08
2536 1.5295606559107e-08
2537 1.50775587570706e-08
2538 1.47870720113019e-08
2539 1.50324748204866e-08
2540 1.51226000610905e-08
2541 1.52926613594673e-08
2542 1.49104568691882e-08
2543 1.49071102129028e-08
2544 1.48904737429234e-08
2545 1.49557806139455e-08
2546 1.50099772611156e-08
2547 1.48602845584378e-08
2548 1.48500909347149e-08
2549 1.48603840344208e-08
2550 1.50750611993544e-08
2551 1.50054813019551e-08
2552 1.49494123746763e-08
2553 1.49503414093033e-08
2554 1.50958339162344e-08
2555 1.49247796343843e-08
2556 1.4913855039822e-08
2557 1.49259040682637e-08
2558 1.49990828646196e-08
2559 1.47663721250524e-08
2560 1.49674210803141e-08
2561 1.50372017060363e-08
2562 1.49501779844741e-08
2563 1.4983925211709e-08
2564 1.48755967543934e-08
2565 1.50057548609084e-08
2566 1.46274263812529e-08
2567 1.51017438554391e-08
2568 1.4905271683574e-08
2569 1.46851721893881e-08
2570 1.4938885684046e-08
2571 1.47704621866751e-08
2572 1.47666368022215e-08
2573 1.47658330007516e-08
2574 1.46012313351207e-08
2575 1.4841158524348e-08
2576 1.50522190267566e-08
2577 1.49948657934829e-08
2578 1.49916807856698e-08
2579 1.48987027159819e-08
2580 1.49773473623327e-08
2581 1.50218379957323e-08
2582 1.48883598782845e-08
2583 1.49706753660439e-08
2584 1.4589033980883e-08
2585 1.48593697346655e-08
2586 1.49699275198145e-08
2587 1.47334073830052e-08
2588 1.49214240963147e-08
2589 1.49641117275223e-08
2590 1.4989190333381e-08
2591 1.4561840622207e-08
2592 1.47697951646819e-08
2593 1.47995793398081e-08
2594 1.48007286426832e-08
2595 1.50692311962075e-08
2596 1.5015171328514e-08
2597 1.48072727412796e-08
2598 1.47542671413703e-08
2599 1.49423069473187e-08
2600 1.47828096430658e-08
2601 1.49199017585033e-08
2602 1.50107481999839e-08
2603 1.47761225477439e-08
2604 1.47274619166637e-08
2605 1.4728006370035e-08
2606 1.46030174619227e-08
2607 1.48931640353567e-08
2608 1.49239820501634e-08
2609 1.44237013444126e-08
2610 1.4938487780114e-08
2611 1.47310013076662e-08
2612 1.49024206308468e-08
2613 1.46481689000666e-08
2614 1.46424987690352e-08
2615 1.46788270427578e-08
2616 1.47310128539857e-08
2617 1.48317749193438e-08
2618 1.48125911536567e-08
2619 1.49741126165281e-08
2620 1.47411585160739e-08
2621 1.46183269933431e-08
2622 1.4394601954848e-08
2623 1.48945016320567e-08
2624 1.47733283384355e-08
2625 1.4581208240827e-08
2626 1.4805089598724e-08
2627 1.46083412033704e-08
2628 1.50799905895838e-08
2629 1.45465799405997e-08
2630 1.45231844328464e-08
2631 1.49520769099354e-08
2632 1.47380809778497e-08
2633 1.46540903855907e-08
2634 1.45341543245081e-08
2635 1.46587151306221e-08
2636 1.46319028004882e-08
2637 1.47587755350287e-08
2638 1.47888377100003e-08
2639 1.47732190924899e-08
2640 1.46173535497951e-08
2641 1.48600571847624e-08
2642 1.48299790225792e-08
2643 1.49077283850829e-08
2644 1.46803040834698e-08
2645 1.48686369882967e-08
2646 1.47929508642619e-08
2647 1.44505589716459e-08
2648 1.48286494194849e-08
2649 1.45085445879545e-08
2650 1.46207783657815e-08
2651 1.48080072648327e-08
2652 1.46360674690982e-08
2653 1.48360639329326e-08
2654 1.47795429228381e-08
2655 1.48613672479314e-08
2656 1.50167256407485e-08
2657 1.47612135847908e-08
2658 1.47839100961278e-08
2659 1.49484780109788e-08
2660 1.46987879645621e-08
2661 1.49683820893642e-08
2662 1.45646055216275e-08
2663 1.45838008336341e-08
2664 1.46134748746363e-08
2665 1.44662406498242e-08
2666 1.4499369704879e-08
2667 1.46495802155755e-08
2668 1.45527918604671e-08
2669 1.48386201104245e-08
2670 1.48070498084962e-08
2671 1.47314951348676e-08
2672 1.45773917381575e-08
2673 1.46962788605265e-08
2674 1.48014240863859e-08
2675 1.44992693407175e-08
2676 1.45538701090686e-08
2677 1.45530831829888e-08
2678 1.47630094815554e-08
2679 1.4614993659734e-08
2680 1.48835619384613e-08
2681 1.46300074277406e-08
2682 1.44706389093585e-08
2683 1.44983909322605e-08
2684 1.44638985233314e-08
2685 1.46882577212182e-08
2686 1.48130743227171e-08
2687 1.46102232534417e-08
2688 1.44601495222219e-08
2689 1.45736311907285e-08
2690 1.46069485396083e-08
2691 1.47770995440055e-08
2692 1.48767416163764e-08
2693 1.46007774759482e-08
2694 1.4285470584241e-08
2695 1.45115457428346e-08
2696 1.47335770250834e-08
2697 1.45711025467676e-08
2698 1.47674805717202e-08
2699 1.47047973797498e-08
2700 1.45643710425247e-08
2701 1.44013112546304e-08
2702 1.47252530169339e-08
2703 1.468219235079e-08
2704 1.46882745966082e-08
2705 1.47494381153024e-08
2706 1.45240797166934e-08
2707 1.46924268307203e-08
2708 1.44011975677927e-08
2709 1.45762557579587e-08
2710 1.44624507925073e-08
2711 1.46742973328173e-08
2712 1.43797684870606e-08
2713 1.45706522403088e-08
2714 1.47835788055772e-08
2715 1.46591503380478e-08
2716 1.46040024517902e-08
2717 1.46489949059969e-08
2718 1.46911647291859e-08
2719 1.45521683592165e-08
2720 1.46883882834459e-08
2721 1.45988954258769e-08
2722 1.47433203423475e-08
2723 1.47249998860843e-08
2724 1.43597196355927e-08
2725 1.45588110456174e-08
2726 1.45207721402585e-08
2727 1.4591865493685e-08
2728 1.45738132673046e-08
2729 1.46712784143688e-08
2730 1.44114729039302e-08
2731 1.47798342453598e-08
2732 1.46721426119711e-08
2733 1.44603520269015e-08
2734 1.47903449487785e-08
2735 1.46758347696618e-08
2736 1.44509169075491e-08
2737 1.46377034937473e-08
2738 1.46618104324148e-08
2739 1.47728753674414e-08
2740 1.46794709721121e-08
2741 1.47846552778219e-08
2742 1.46107037579668e-08
2743 1.48645149522508e-08
2744 1.46956393720643e-08
2745 1.45947174345906e-08
2746 1.47151615337293e-08
2747 1.46286023294806e-08
2748 1.45338470147749e-08
2749 1.46093714903373e-08
2750 1.46688110547188e-08
2751 1.48128851407137e-08
2752 1.4706200701653e-08
2753 1.4789554469985e-08
2754 1.47206646872178e-08
2755 1.47999674737775e-08
2756 1.47200385214319e-08
2757 1.47945433681684e-08
2758 1.45282443853034e-08
2759 1.46582888049807e-08
2760 1.47665408789521e-08
2761 1.47433505404138e-08
2762 1.47500047731342e-08
2763 1.44123131207152e-08
2764 1.47000651651297e-08
2765 1.48554342160878e-08
2766 1.46673286849364e-08
2767 1.4566458261811e-08
2768 1.46941978584891e-08
2769 1.45706939846946e-08
2770 1.4631955203015e-08
2771 1.46215164420482e-08
2772 1.46200065387347e-08
2773 1.45449856603364e-08
2774 1.43907739058591e-08
2775 1.43236862371054e-08
2776 1.4524743185973e-08
2777 1.45582141897194e-08
2778 1.4648125379324e-08
2779 1.44884255703914e-08
2780 1.46222767227755e-08
2781 1.44919596323234e-08
2782 1.46277363555214e-08
2783 1.45832812492586e-08
2784 1.46997791716785e-08
2785 1.46036098769287e-08
2786 1.46952672253065e-08
2787 1.45123744133002e-08
2788 1.46046108540077e-08
2789 1.45766101411482e-08
2790 1.45772860449256e-08
2791 1.45427474507187e-08
2792 1.45832581566196e-08
2793 1.42819782666948e-08
2794 1.45251801697555e-08
2795 1.44041933936023e-08
2796 1.45171599186256e-08
2797 1.45389842387544e-08
2798 1.45877603330291e-08
2799 1.46011860380213e-08
2800 1.47102383607489e-08
2801 1.45081893165866e-08
2802 1.45331240375413e-08
2803 1.46479859353121e-08
2804 1.45621816827202e-08
2805 1.47270275974165e-08
2806 1.45345548929754e-08
2807 1.46091103658819e-08
2808 1.47130148064889e-08
2809 1.43030831623037e-08
2810 1.46153098512514e-08
2811 1.45657770289631e-08
2812 1.45905394433044e-08
2813 1.4603716458339e-08
2814 1.45121017425254e-08
2815 1.47677656769929e-08
2816 1.45291547681836e-08
2817 1.4576529316912e-08
2818 1.46425795932714e-08
2819 1.44846428185019e-08
2820 1.45928131800588e-08
2821 1.46491565544693e-08
2822 1.44166945048596e-08
2823 1.44785552436133e-08
2824 1.45986049915336e-08
2825 1.43365674887264e-08
2826 1.43353746651087e-08
2827 1.43822012077521e-08
2828 1.4498252376427e-08
2829 1.43779255168397e-08
2830 1.44607987806467e-08
2831 1.44896343812206e-08
2832 1.45685854491262e-08
2833 1.44526506318243e-08
2834 1.44185738903957e-08
2835 1.4597500985758e-08
2836 1.42823024518179e-08
2837 1.43820333420308e-08
2838 1.43196006163748e-08
2839 1.44387186651329e-08
2840 1.44107978883312e-08
2841 1.44620706521437e-08
2842 1.45085259362077e-08
2843 1.45373038051844e-08
2844 1.42692382354426e-08
2845 1.44524694434267e-08
2846 1.43529863549929e-08
2847 1.45280205643417e-08
2848 1.45247192051556e-08
2849 1.4477136822677e-08
2850 1.433763330283e-08
2851 1.42869929220524e-08
2852 1.45732128586928e-08
2853 1.44514906708082e-08
2854 1.43882346037572e-08
2855 1.43721736733937e-08
2856 1.44161464987747e-08
2857 1.43134908370257e-08
2858 1.43660408014057e-08
2859 1.43907170624402e-08
2860 1.43942617825132e-08
2861 1.43964209442515e-08
2862 1.44073952768053e-08
2863 1.44110021693677e-08
2864 1.45082461600055e-08
2865 1.44471323793027e-08
2866 1.44120519962598e-08
2867 1.42866438679334e-08
2868 1.43986103040561e-08
2869 1.4212829135829e-08
2870 1.42571465744368e-08
2871 1.41003893006086e-08
2872 1.43111282824293e-08
2873 1.43747849179476e-08
2874 1.43575906719207e-08
2875 1.45924596850477e-08
2876 1.43021274823241e-08
2877 1.44096592435972e-08
2878 1.44345104757804e-08
2879 1.46319916183302e-08
2880 1.43363489968351e-08
2881 1.4234215583997e-08
2882 1.42808112002513e-08
2883 1.43818024156417e-08
2884 1.45255674155464e-08
2885 1.42134934932869e-08
2886 1.44656340239635e-08
2887 1.43834641974649e-08
2888 1.43751845982365e-08
2889 1.43692862053513e-08
2890 1.4254345259701e-08
2891 1.43283855891241e-08
2892 1.42661598090399e-08
2893 1.42727953900135e-08
2894 1.40794806924305e-08
2895 1.42915936862664e-08
2896 1.43366900573483e-08
2897 1.42529730240426e-08
2898 1.43091032356324e-08
2899 1.41402054509854e-08
2900 1.4123905600627e-08
2901 1.42811913406149e-08
2902 1.4180791652052e-08
2903 1.4282971250168e-08
2904 1.43068445979111e-08
2905 1.42190712537626e-08
2906 1.42021914228962e-08
2907 1.41768632389017e-08
2908 1.41180249713102e-08
2909 1.41680924770071e-08
2910 1.42327065688619e-08
2911 1.42116629575639e-08
2912 1.41200846570655e-08
2913 1.42992098162154e-08
2914 1.44395446710632e-08
2915 1.41723495161727e-08
2916 1.43264768937001e-08
2917 1.42764982058452e-08
2918 1.43446863276608e-08
2919 1.41244820284214e-08
2920 1.42581821904741e-08
2921 1.41576164125468e-08
2922 1.43205935998481e-08
2923 1.4222307775924e-08
2924 1.41886493665311e-08
2925 1.41817517729237e-08
2926 1.40286875449647e-08
2927 1.42065408326175e-08
2928 1.41696903099842e-08
2929 1.41520475338552e-08
2930 1.41335103620577e-08
2931 1.41662690467115e-08
2932 1.41177585177843e-08
2933 1.41606086856427e-08
2934 1.40551001948097e-08
2935 1.39338887095164e-08
2936 1.41845575285515e-08
2937 1.41826257404887e-08
2938 1.43530334284492e-08
2939 1.3971400036894e-08
2940 1.41653648810802e-08
2941 1.4162513828353e-08
2942 1.4173353157787e-08
2943 1.41954519250476e-08
2944 1.40452174335337e-08
2945 1.4200971953926e-08
2946 1.40907827628212e-08
2947 1.40446285712414e-08
2948 1.41185276802958e-08
2949 1.41183935653544e-08
2950 1.43457965506855e-08
2951 1.41319960178521e-08
2952 1.40787301816658e-08
2953 1.40034117634968e-08
2954 1.4149367899563e-08
2955 1.43096627880368e-08
2956 1.40443550122882e-08
2957 1.42356721966053e-08
2958 1.42514293699492e-08
2959 1.42186689089385e-08
2960 1.41660363439655e-08
2961 1.4165434158997e-08
2962 1.41752627413894e-08
2963 1.41607729986504e-08
2964 1.42260718760667e-08
2965 1.41429019606676e-08
2966 1.41744482817785e-08
2967 1.4106463552821e-08
2968 1.42185525575655e-08
2969 1.41890303950731e-08
2970 1.40936915471457e-08
2971 1.41203262415956e-08
2972 1.41106584194972e-08
2973 1.42491520804811e-08
2974 1.39680658151065e-08
2975 1.40552156580043e-08
2976 1.41819791465991e-08
2977 1.41862050995201e-08
2978 1.39782434516178e-08
2979 1.44055158912693e-08
2980 1.40750247012988e-08
2981 1.4306171358669e-08
2982 1.40515012958531e-08
2983 1.42615146359049e-08
2984 1.41408378340202e-08
2985 1.4118441526989e-08
2986 1.41609772796869e-08
2987 1.40972353790403e-08
2988 1.41159537392355e-08
2989 1.41044012025304e-08
2990 1.42190099694517e-08
2991 1.41334393077841e-08
2992 1.41586840030072e-08
2993 1.41260638741869e-08
2994 1.41181564217163e-08
2995 1.42468365993409e-08
2996 1.42157192684067e-08
2997 1.42941161129784e-08
2998 1.45517375926829e-08
2999 1.41141898168939e-08
3000 1.02227639686703e-08
3001 1.02083799191632e-08
3002 1.02394421830354e-08
3003 1.02710666638473e-08
3004 1.02892823150569e-08
3005 1.02968211734833e-08
3006 1.03013242380712e-08
3007 1.03039727861187e-08
3008 1.0305607034411e-08
3009 1.03064596856939e-08
3010 1.03061816858485e-08
3011 1.03061728040643e-08
3012 1.0305374331665e-08
3013 1.03045527666268e-08
3014 1.03039070609157e-08
3015 1.03030943776616e-08
3016 1.03021049469021e-08
3017 1.03014832220083e-08
3018 1.03003889861952e-08
3019 1.02996402517874e-08
3020 1.03114468075205e-08
3021 1.03168433795986e-08
3022 1.03198036782715e-08
3023 1.03208730450888e-08
3024 1.03219939262544e-08
3025 1.03231423409511e-08
3026 1.03434922849033e-08
3027 1.03537844964308e-08
3028 1.03539212759074e-08
3029 1.03542099338938e-08
3030 1.0356237645226e-08
3031 1.03580823918037e-08
3032 1.03590611644222e-08
3033 1.0358538027333e-08
3034 1.03591739630815e-08
3035 1.03581472288283e-08
3036 1.03585922062166e-08
3037 1.03580459764885e-08
3038 1.03590460653891e-08
3039 1.03591100142353e-08
3040 1.03595052536321e-08
3041 1.03587654010084e-08
3042 1.0356525415034e-08
3043 1.03605604095947e-08
3044 1.03629975711783e-08
3045 1.03618660318716e-08
3046 1.0361352664745e-08
3047 1.03601500711648e-08
3048 1.03604156365122e-08
3049 1.03578479127009e-08
3050 1.03582342703135e-08
3051 1.03571879961351e-08
3052 1.03573949417068e-08
3053 1.03546815566347e-08
3054 1.03545518825854e-08
3055 1.03523944972039e-08
3056 1.0352426471627e-08
3057 1.03511847981963e-08
3058 1.03478479118735e-08
3059 1.03478186019856e-08
3060 1.03474135926263e-08
3061 1.03473807300247e-08
3062 1.0349042511848e-08
3063 1.03472430623697e-08
3064 1.0347609880057e-08
3065 1.03391943895303e-08
3066 1.03402628681692e-08
3067 1.03394448558447e-08
3068 1.03372359561149e-08
3069 1.0337456224363e-08
3070 1.03353743341472e-08
3071 1.03331485590274e-08
3072 1.0330919231194e-08
3073 1.03320427768949e-08
3074 1.03304094167811e-08
3075 1.03292761011176e-08
3076 1.03280068941558e-08
3077 1.0326616894929e-08
3078 1.03261941220012e-08
3079 1.03234381043649e-08
3080 1.03213251279044e-08
3081 1.03217976388237e-08
3082 1.03198978251839e-08
3083 1.03159720765689e-08
3084 1.03167474563293e-08
3085 1.03156008179894e-08
3086 1.03147446139928e-08
3087 1.03104254023378e-08
3088 1.03112762772639e-08
3089 1.03100070703022e-08
3090 1.03075210589054e-08
3091 1.03045367794152e-08
3092 1.03054418332249e-08
3093 1.03046682298213e-08
3094 1.03013952923448e-08
3095 1.03003259255274e-08
3096 1.02994004436141e-08
3097 1.02996340345385e-08
3098 1.02961408288138e-08
3099 1.0294590069293e-08
3100 1.02909920585148e-08
3101 1.02889972097842e-08
3102 1.02873460861019e-08
3103 1.02850172822855e-08
3104 1.02851425154427e-08
3105 1.02828359160867e-08
3106 1.02814192715073e-08
3107 1.02778097144096e-08
3108 1.02786223976636e-08
3109 1.02795256751165e-08
3110 1.02778656696501e-08
3111 1.02757917730401e-08
3112 1.02770565391097e-08
3113 1.02738795249024e-08
3114 1.02757855557911e-08
3115 1.02733510587427e-08
3116 1.02715986827207e-08
3117 1.02718091810061e-08
3118 1.02703436866136e-08
3119 1.02705710602891e-08
3120 1.02685424607785e-08
3121 1.0269515016148e-08
3122 1.02669917012577e-08
3123 1.02678212599017e-08
3124 1.02649355682161e-08
3125 1.02655146605457e-08
3126 1.0263752514561e-08
3127 1.02637631727021e-08
3128 1.02621573461192e-08
3129 1.02618349373529e-08
3130 1.02594599482586e-08
3131 1.02610213659204e-08
3132 1.02665058676621e-08
3133 1.02673194390945e-08
3134 1.02678523461464e-08
3135 1.02671782187258e-08
3136 1.02669730495109e-08
3137 1.0265682526267e-08
3138 1.0265688743516e-08
3139 1.02634993837114e-08
3140 1.0266086647448e-08
3141 1.02637764953784e-08
3142 1.02623882725084e-08
3143 1.026442753016e-08
3144 1.02603712193172e-08
3145 1.02617709885067e-08
3146 1.02582564664999e-08
3147 1.0260749583324e-08
3148 1.02565476112204e-08
3149 1.02616892760921e-08
3150 1.02581383387701e-08
3151 1.02549586600276e-08
3152 1.02606696472662e-08
3153 1.02540518298611e-08
3154 1.02528261436419e-08
3155 1.025542939459e-08
3156 1.025335549798e-08
3157 1.02533679324779e-08
3158 1.02512913713326e-08
3159 1.02530171020021e-08
3160 1.02512718314074e-08
3161 1.02486721331729e-08
3162 1.02502761833989e-08
3163 1.02468700191594e-08
3164 1.02485078201653e-08
3165 1.02467057061517e-08
3166 1.02485948616504e-08
3167 1.02456159112307e-08
3168 1.02473984853191e-08
3169 1.02449551064865e-08
3170 1.02412398561569e-08
3171 1.02429682513616e-08
3172 1.02416848335452e-08
3173 1.02435437909776e-08
3174 1.02406989554993e-08
3175 1.02376951360839e-08
3176 1.02396322532172e-08
3177 1.02378416855231e-08
3178 1.02386756850592e-08
3179 1.02365627085987e-08
3180 1.02363273413175e-08
3181 1.02340234064968e-08
3182 1.02350128372564e-08
3183 1.02326236373074e-08
3184 1.02331503271103e-08
3185 1.02330153239905e-08
3186 1.02316404237968e-08
3187 1.02311545902012e-08
3188 1.02310364624714e-08
3189 1.02298169935011e-08
3190 1.02296251469625e-08
3191 1.02284181124901e-08
3192 1.02291313197611e-08
3193 1.02245323319039e-08
3194 1.02291446424374e-08
3195 1.02265031998172e-08
3196 1.02276365154808e-08
3197 1.02258903567076e-08
3198 1.02258725931392e-08
3199 1.02241513033619e-08
3200 1.02220685249677e-08
3201 1.0229232572101e-08
3202 1.0226339774988e-08
3203 1.02232791121537e-08
3204 1.02271950908062e-08
3205 1.02267687651647e-08
3206 1.02228794318648e-08
3207 1.02273594038138e-08
3208 1.02168149496151e-08
3209 1.02248911559855e-08
3210 1.02212203145768e-08
3211 1.02162047710408e-08
3212 1.02162589499244e-08
3213 1.02118509204274e-08
3214 1.02116732847435e-08
3215 1.02124655398939e-08
3216 1.02085522257767e-08
3217 1.02088515419041e-08
3218 1.02083426156696e-08
3219 1.02096642251581e-08
3220 1.0207431344611e-08
3221 1.02100328192023e-08
3222 1.02134691815081e-08
3223 1.02120472078582e-08
3224 1.02087183151411e-08
3225 1.02102477583799e-08
3226 1.02103081545124e-08
3227 1.02066648466348e-08
3228 1.02107691191122e-08
3229 1.0210057688198e-08
3230 1.02122994505294e-08
3231 1.02117772016186e-08
3232 1.02104431576322e-08
3233 1.02055830453196e-08
3234 1.02062394091718e-08
3235 1.02104777965906e-08
3236 1.02001171953248e-08
3237 1.01957784437445e-08
3238 1.01975397015508e-08
3239 1.0199893374363e-08
3240 1.02015302871905e-08
3241 1.02024788617427e-08
3242 1.02024309001081e-08
3243 1.02067598817257e-08
3244 1.02008126390274e-08
3245 1.01953219200368e-08
3246 1.01993729018091e-08
3247 1.02020827341676e-08
3248 1.02000550228354e-08
3249 1.01894386261847e-08
3250 1.01983133049544e-08
3251 1.01897601467726e-08
3252 1.01918367079179e-08
3253 1.01920862860538e-08
3254 1.01989909850886e-08
3255 1.01957375875372e-08
3256 1.01920143436018e-08
3257 1.01931663110122e-08
3258 1.01935846430479e-08
3259 1.01934771734591e-08
3260 1.0184303178562e-08
3261 1.01886259429307e-08
3262 1.01881152403394e-08
3263 1.01918100625653e-08
3264 1.01946939778941e-08
3265 1.01876782565569e-08
3266 1.01886374892501e-08
3267 1.01859285450701e-08
3268 1.01885273551261e-08
3269 1.01927666307233e-08
3270 1.01902966065381e-08
3271 1.01884518599604e-08
3272 1.01857384748882e-08
3273 1.01818633524431e-08
3274 1.01853334655289e-08
3275 1.0181317122715e-08
3276 1.01860271328746e-08
3277 1.0179928899845e-08
3278 1.01829309429036e-08
3279 1.01754817904975e-08
3280 1.01778976357991e-08
3281 1.01850385902935e-08
3282 1.01817496656054e-08
3283 1.0177628517738e-08
3284 1.01877084546231e-08
3285 1.01771595595324e-08
3286 1.01816111097719e-08
3287 1.01815640363156e-08
3288 1.01751478354117e-08
3289 1.0175116749167e-08
3290 1.01745065705927e-08
3291 1.01756558734678e-08
3292 1.01780068817447e-08
3293 1.0172263920083e-08
3294 1.01745847302936e-08
3295 1.01684767273014e-08
3296 1.01730828205859e-08
3297 1.01684340947372e-08
3298 1.01697699150805e-08
3299 1.01674144659114e-08
3300 1.01616253189718e-08
3301 1.01542925179388e-08
3302 1.01509138872302e-08
3303 1.0143771156379e-08
3304 1.01478487835038e-08
3305 1.01407024999389e-08
3306 1.01384989292796e-08
3307 1.01413153430485e-08
3308 1.01414805442346e-08
3309 1.01378425654275e-08
3310 1.01409156627597e-08
3311 1.01434247667953e-08
3312 1.01414396880273e-08
3313 1.01380699391029e-08
3314 1.01375166039475e-08
3315 1.01413641928616e-08
3316 1.01412087616382e-08
3317 1.01378745398506e-08
3318 1.01353263559645e-08
3319 1.01328652135635e-08
3320 1.01408073049924e-08
3321 1.01325543511166e-08
3322 1.01370147831403e-08
3323 1.01321617762551e-08
3324 1.01355341897147e-08
3325 1.01308268440903e-08
3326 1.01323998080716e-08
3327 1.01284713949212e-08
3328 1.01340340563638e-08
3329 1.01314130418473e-08
3330 1.01344381775448e-08
3331 1.01334984847767e-08
3332 1.013272665773e-08
3333 1.01218660120139e-08
3334 1.01326795842738e-08
3335 1.0128999861081e-08
3336 1.01204973290692e-08
3337 1.01228296855993e-08
3338 1.01309778344216e-08
3339 1.01233608162943e-08
3340 1.01281187880886e-08
3341 1.01243777805848e-08
3342 1.0123502036663e-08
3343 1.01194590484965e-08
3344 1.0127694238804e-08
3345 1.01191419688007e-08
3346 1.01184820522349e-08
3347 1.01241734995483e-08
3348 1.01157890952663e-08
3349 1.01230668292374e-08
3350 1.01202761726427e-08
3351 1.01186978795909e-08
3352 1.01152872744592e-08
3353 1.01229220561549e-08
3354 1.01178132538848e-08
3355 1.01228012638899e-08
3356 1.01142862973802e-08
3357 1.01183159628704e-08
3358 1.01146806485986e-08
3359 1.01198862623164e-08
3360 1.01168646793326e-08
3361 1.01153716514091e-08
3362 1.0112745307822e-08
3363 1.01170716249044e-08
3364 1.01137977992494e-08
3365 1.01168806665441e-08
3366 1.0112042758692e-08
3367 1.01113659667362e-08
3368 1.01062447299682e-08
3369 1.01033856836352e-08
3370 1.01101882421517e-08
3371 1.00995007912275e-08
3372 1.01065440460957e-08
3373 1.01036148336675e-08
3374 1.010991379502e-08
3375 1.01069925761976e-08
3376 1.01065049662452e-08
3377 1.01075920966309e-08
3378 1.00963388760533e-08
3379 1.01028643229029e-08
3380 1.01024584253651e-08
3381 1.01034833832614e-08
3382 1.01014165920787e-08
3383 1.01015924514058e-08
3384 1.0100158043258e-08
3385 1.01028998500396e-08
3386 1.01022807896811e-08
3387 1.00974393291153e-08
3388 1.00916404122131e-08
3389 1.01013126752036e-08
3390 1.00919708145852e-08
3391 1.00957473492258e-08
3392 1.00903019273346e-08
3393 1.0100346337083e-08
3394 1.00908721378801e-08
3395 1.00994501650575e-08
3396 1.00927541879514e-08
3397 1.00976151884424e-08
3398 1.00912114220364e-08
3399 1.00909041123032e-08
3400 1.00854791185157e-08
3401 1.0093942570677e-08
3402 1.0085259738446e-08
3403 1.00885966247688e-08
3404 1.00891934806668e-08
3405 1.00817478809745e-08
3406 1.00902530775215e-08
3407 1.00798374091937e-08
3408 1.00916750511715e-08
3409 1.00809582903594e-08
3410 1.00893311483219e-08
3411 1.00828847493517e-08
3412 1.0087720880847e-08
3413 1.0080984935712e-08
3414 1.00795869428794e-08
3415 1.00800594537986e-08
3416 1.00810337855251e-08
3417 1.00839194772107e-08
3418 1.00752908238633e-08
3419 1.00735899621895e-08
3420 1.00751842424529e-08
3421 1.00801988978105e-08
3422 1.0073906153707e-08
3423 1.00807113767587e-08
3424 1.00732053809338e-08
3425 1.00736636809984e-08
3426 1.00780050971139e-08
3427 1.00749408815659e-08
3428 1.00682155945719e-08
3429 1.00746158082643e-08
3430 1.00700257021913e-08
3431 1.00652215451191e-08
3432 1.0072799483396e-08
3433 1.00702752803272e-08
3434 1.00727950425039e-08
3435 1.0067623179566e-08
3436 1.00668717806229e-08
3437 1.00697050697818e-08
3438 1.00653387846705e-08
3439 1.00702504113315e-08
3440 1.00630455079909e-08
3441 1.00615276110716e-08
3442 1.00678132497478e-08
3443 1.00568708916171e-08
3444 1.00680539460996e-08
3445 1.0062529476329e-08
3446 1.00665404900724e-08
3447 1.00634691690971e-08
3448 1.00577608463936e-08
3449 1.0057065402691e-08
3450 1.00604742314658e-08
3451 1.00547890014013e-08
3452 1.00588923857003e-08
3453 1.00548316339655e-08
3454 1.00559764959485e-08
3455 1.00535979541405e-08
3456 1.00520756163291e-08
3457 1.00533137370462e-08
3458 1.00474402131567e-08
3459 1.00533776858924e-08
3460 1.00481614140335e-08
3461 1.00497938859689e-08
3462 1.00470689545773e-08
3463 1.00434336403055e-08
3464 1.00477990372383e-08
3465 1.00468096064787e-08
3466 1.0045153153726e-08
3467 1.00483585896427e-08
3468 1.00447961060013e-08
3469 1.00463513064142e-08
3470 1.00414432324669e-08
3471 1.00374206724041e-08
3472 1.00416333026487e-08
3473 1.00324877294611e-08
3474 1.00358850119164e-08
3475 1.00394492719147e-08
3476 1.00380823653268e-08
3477 1.00370893818535e-08
3478 1.00417061332791e-08
3479 1.00361150501271e-08
3480 1.00369064170991e-08
3481 1.00333439334577e-08
3482 1.00330721508612e-08
3483 1.00294537119794e-08
3484 1.00317771867253e-08
3485 1.00291854820966e-08
3486 1.00318491291773e-08
3487 1.00269410552301e-08
3488 1.00275938663685e-08
3489 1.00289065940729e-08
3490 1.002286076357e-08
3491 1.00270609593167e-08
3492 1.0021836693852e-08
3493 1.00265102886965e-08
3494 1.00179695650127e-08
3495 1.00221031473779e-08
3496 1.00217469878316e-08
3497 1.00230330701834e-08
3498 1.00102610645081e-08
3499 1.00209698317144e-08
3500 1.00206438702344e-08
3501 1.00150918669328e-08
3502 1.00169676997552e-08
3503 1.00169108563364e-08
3504 1.00166719363415e-08
3505 1.00148929149668e-08
3506 1.00036698924555e-08
3507 1.00152002247e-08
3508 1.00150661097587e-08
3509 1.00178256801087e-08
3510 1.00153538795666e-08
3511 1.0005395623125e-08
3512 1.00126706925607e-08
3513 1.00090824517451e-08
3514 1.00172004025012e-08
3515 1.00051975593374e-08
3516 1.00002521818965e-08
3517 1.00090282728615e-08
3518 1.00099626365591e-08
3519 1.00066408492694e-08
3520 1.0009050477322e-08
3521 1.00049515339151e-08
3522 1.00105319589261e-08
3523 9.99412907987107e-09
3524 9.99843052795768e-09
3525 1.00001331659882e-08
3526 1.00023518356807e-08
3527 1.00040828954207e-08
3528 1.00037480521564e-08
3529 9.99698457349041e-09
3530 9.98938443075303e-09
3531 9.99808147383874e-09
3532 1.00036983141649e-08
3533 1.00033812344691e-08
3534 9.98590810041833e-09
3535 1.00004493575057e-08
3536 1.00001793512661e-08
3537 9.98664795304194e-09
3538 9.99874227858299e-09
3539 9.99706184501292e-09
3540 9.99361304820923e-09
3541 9.9813259879511e-09
3542 9.99770044529669e-09
3543 9.99628735343094e-09
3544 9.99097959919482e-09
3545 9.97941196345664e-09
3546 9.99382088195944e-09
3547 9.99266625001383e-09
3548 9.97972282590354e-09
3549 9.99082327979295e-09
3550 9.98978322286348e-09
3551 9.97886306919327e-09
3552 9.98926985573689e-09
3553 9.98557680986778e-09
3554 9.97158888793592e-09
3555 9.99150984171138e-09
3556 9.98791449546843e-09
3557 9.97500748667335e-09
3558 9.98667459839453e-09
3559 9.9882422333053e-09
3560 9.98940752339195e-09
3561 9.97950699854755e-09
3562 9.98481475278368e-09
3563 9.98584059885843e-09
3564 9.98891014347691e-09
3565 9.96572424583064e-09
3566 9.96319826640502e-09
3567 9.97792781731732e-09
3568 9.97337856745162e-09
3569 9.97790738921367e-09
3570 9.97575266836748e-09
3571 9.96569404776437e-09
3572 9.97711602224172e-09
3573 9.98116433947871e-09
3574 9.97994842322214e-09
3575 9.96594984314925e-09
3576 9.97613636144479e-09
3577 9.96099913663784e-09
3578 9.97474547403954e-09
3579 9.97731852692141e-09
3580 9.95916060730906e-09
3581 9.96707250067175e-09
3582 9.97580507089424e-09
3583 9.97380222855782e-09
3584 9.95926008329207e-09
3585 9.97006655012456e-09
3586 9.96776883255279e-09
3587 9.9562997846192e-09
3588 9.96721638557574e-09
3589 9.97101956556889e-09
3590 9.95490001542976e-09
3591 9.96275151265991e-09
3592 9.96852378420954e-09
3593 9.97107907352301e-09
3594 9.96866145186459e-09
3595 9.95391680191915e-09
3596 9.96162974331583e-09
3597 9.96908777750605e-09
3598 9.96756988058678e-09
3599 9.94899984618769e-09
3600 9.94730608994132e-09
3601 9.95543736337368e-09
3602 9.9619894555758e-09
3603 9.95919879898111e-09
3604 9.94883908589372e-09
3605 9.96178695089611e-09
3606 9.96224081006858e-09
3607 9.94375159990568e-09
3608 9.95377735790726e-09
3609 9.95939153369818e-09
3610 9.95504922940427e-09
3611 9.94244242491504e-09
3612 9.95908422396496e-09
3613 9.95763649314085e-09
3614 9.94253390729227e-09
3615 9.95337856579681e-09
3616 9.93543647354045e-09
3617 9.95088633715113e-09
3618 9.95194948671951e-09
3619 9.93689130979192e-09
3620 9.95247884105765e-09
3621 9.95528548486391e-09
3622 9.95246995927346e-09
3623 9.93853532804678e-09
3624 9.94617543881304e-09
3625 9.95280213800243e-09
3626 9.95057547470424e-09
3627 9.92841187041904e-09
3628 9.94778570628796e-09
3629 9.94581039748255e-09
3630 9.93278970184974e-09
3631 9.94451099245452e-09
3632 9.9445953694044e-09
3633 9.94861171221828e-09
3634 9.92237136898666e-09
3635 9.94508386753523e-09
3636 9.94382354235768e-09
3637 9.92948478995004e-09
3638 9.93668702875539e-09
3639 9.9416919141504e-09
3640 9.94355264793967e-09
3641 9.94210402893714e-09
3642 9.92327464643949e-09
3643 9.94220972216908e-09
3644 9.93984983210794e-09
3645 9.91917037396206e-09
3646 9.9389643182235e-09
3647 9.93759119438664e-09
3648 9.93522331071972e-09
3649 9.92150450684903e-09
3650 9.92040405378702e-09
3651 9.92868187665863e-09
3652 9.93365301127369e-09
3653 9.9303125722372e-09
3654 9.91812054706998e-09
3655 9.93334925425415e-09
3656 9.93195570231364e-09
3657 9.93389814851753e-09
3658 9.91635840108529e-09
3659 9.92938975485913e-09
3660 9.93695525863814e-09
3661 9.90905402176168e-09
3662 9.92993953730092e-09
3663 9.92982851499846e-09
3664 9.92526949517014e-09
3665 9.91446125198081e-09
3666 9.92451276715656e-09
3667 9.91036230857389e-09
3668 9.91753523749139e-09
3669 9.92975213165437e-09
3670 9.92506343777677e-09
3671 9.92354820539276e-09
3672 9.92634774377166e-09
3673 9.92425608359326e-09
3674 9.90991555482879e-09
3675 9.92221238504953e-09
3676 9.90337056805402e-09
3677 9.91856374810141e-09
3678 9.91493465107851e-09
3679 9.90303306025453e-09
3680 9.91815252149308e-09
3681 9.91853088549988e-09
3682 9.8980361684653e-09
3683 9.91731496924331e-09
3684 9.91703785757636e-09
3685 9.91691262441918e-09
3686 9.91987647580572e-09
3687 9.89632820136421e-09
3688 9.91557413954069e-09
3689 9.91262893990097e-09
3690 9.89868897960378e-09
3691 9.90741177986365e-09
3692 9.91351356560699e-09
3693 9.91531923233424e-09
3694 9.91417614670809e-09
3695 9.91493109836483e-09
3696 9.89484671976015e-09
3697 9.91121940074891e-09
3698 9.91442661302244e-09
3699 9.89567805476099e-09
3700 9.90806992007265e-09
3701 9.90457316163429e-09
3702 9.89348780677801e-09
3703 9.90853887827825e-09
3704 9.90712223369883e-09
3705 9.90621451535389e-09
3706 9.89173187804226e-09
3707 9.90487070140489e-09
3708 9.90582549320607e-09
3709 9.88893855691231e-09
3710 9.90393100863685e-09
3711 9.90161108660459e-09
3712 9.90462822869631e-09
3713 9.88672965718251e-09
3714 9.89948034657573e-09
3715 9.89993864664029e-09
3716 9.90084636498523e-09
3717 9.88567805393359e-09
3718 9.89686288477287e-09
3719 9.8947809945571e-09
3720 9.88205606233805e-09
3721 9.89521176109065e-09
3722 9.8982164686845e-09
3723 9.89516824034808e-09
3724 9.87981518818515e-09
3725 9.89297976872194e-09
3726 9.89087745040251e-09
3727 9.87932935458957e-09
3728 9.88888082531503e-09
3729 9.88691173375855e-09
3730 9.89567983111783e-09
3731 9.8933385928035e-09
3732 9.87733539403735e-09
3733 9.88957626901765e-09
3734 9.88504655907718e-09
3735 9.87404291663552e-09
3736 9.88946080582309e-09
3737 9.88869164331163e-09
3738 9.86725723350901e-09
3739 9.88681048141871e-09
3740 9.88863213535751e-09
3741 9.88730430862006e-09
3742 9.88875736851469e-09
3743 9.88441684057761e-09
3744 9.87229675786239e-09
3745 9.88233228582658e-09
3746 9.8811971938062e-09
3747 9.88453052741534e-09
3748 9.86549775205958e-09
3749 9.88334925011713e-09
3750 9.8813730531333e-09
3751 9.86666393032465e-09
3752 9.87940840246893e-09
3753 9.87660797591161e-09
3754 9.88490000963793e-09
3755 9.8817203308954e-09
3756 9.86605108721506e-09
3757 9.87756276771279e-09
3758 9.87356507664572e-09
3759 9.86147341563992e-09
3760 9.87471882041291e-09
3761 9.8701198325557e-09
3762 9.87846782152246e-09
3763 9.87724657619538e-09
3764 9.87882042835508e-09
3765 9.8570449580393e-09
3766 9.87444970235174e-09
3767 9.87051151923879e-09
3768 9.85071757497735e-09
3769 9.87152581899409e-09
3770 9.86789405743593e-09
3771 9.85494175154145e-09
3772 9.8683523575005e-09
3773 9.86482806553113e-09
3774 9.87319737077996e-09
3775 9.86863035734586e-09
3776 9.85507231376914e-09
3777 9.86691972570952e-09
3778 9.86363080102137e-09
3779 9.87142900754634e-09
3780 9.86720127826857e-09
3781 9.85269288378277e-09
3782 9.86456871743258e-09
3783 9.85945458609194e-09
3784 9.85111547890938e-09
3785 9.85877690595771e-09
3786 9.86703252436882e-09
3787 9.86466641705874e-09
3788 9.84117765057135e-09
3789 9.86146453385572e-09
3790 9.85803527697726e-09
3791 9.86600312558039e-09
3792 9.86276127434849e-09
3793 9.83882841865125e-09
3794 9.85986670087868e-09
3795 9.85710268963658e-09
3796 9.84474368692645e-09
3797 9.85571801948026e-09
3798 9.85171499934268e-09
3799 9.86017489879032e-09
3800 9.85770309824829e-09
3801 9.84246817381518e-09
3802 9.85439818634859e-09
3803 9.85453763036048e-09
3804 9.85812409481923e-09
3805 9.85498971317611e-09
3806 9.85258274965872e-09
3807 9.85679893261704e-09
3808 9.83479697680423e-09
3809 9.85322046176407e-09
3810 9.84987558183548e-09
3811 9.83614167893165e-09
3812 9.85018822063921e-09
3813 9.84883197219233e-09
3814 9.84407755311167e-09
3815 9.8308463591934e-09
3816 9.84738424136822e-09
3817 9.847148874087e-09
3818 9.83322134828768e-09
3819 9.84395143177608e-09
3820 9.84160664074807e-09
3821 9.85494175154145e-09
3822 9.84868275821782e-09
3823 9.82621095602099e-09
3824 9.85032322375901e-09
3825 9.85054526836393e-09
3826 9.846662152313e-09
3827 9.82240333513573e-09
3828 9.84630954548038e-09
3829 9.84385106761465e-09
3830 9.8518695423877e-09
3831 9.85035075729002e-09
3832 9.82744197131069e-09
3833 9.84695613936992e-09
3834 9.84475789778116e-09
3835 9.8497592304625e-09
3836 9.84822534633167e-09
3837 9.8329202558034e-09
3838 9.84711867602073e-09
3839 9.84587789076841e-09
3840 9.84260228875655e-09
3841 9.82648717950951e-09
3842 9.84330039699444e-09
3843 9.84438930373699e-09
3844 9.82580949937528e-09
3845 9.84329684428076e-09
3846 9.84477921406324e-09
3847 9.84339454390692e-09
3848 9.8280779070592e-09
3849 9.84049997043712e-09
3850 9.83968817536152e-09
3851 9.83592673975409e-09
3852 9.84474368692645e-09
3853 9.84140680060364e-09
3854 9.82640901980858e-09
3855 9.83699877110666e-09
3856 9.83405801235904e-09
3857 9.82210757172197e-09
3858 9.83988446279227e-09
3859 9.83808856602764e-09
3860 9.8362340494873e-09
3861 9.83762760142781e-09
3862 9.83900605433519e-09
3863 9.83770664930717e-09
3864 9.82238557156734e-09
3865 9.83168213508634e-09
3866 9.83501813323073e-09
3867 9.83607328919334e-09
3868 9.83598180681611e-09
3869 9.82289805051551e-09
3870 9.83077796945508e-09
3871 9.82835679508298e-09
3872 9.8157340033822e-09
3873 9.82832215612461e-09
3874 9.82508208124955e-09
3875 9.81285452894554e-09
3876 9.82447279085363e-09
3877 9.8236734302759e-09
3878 9.83039960544829e-09
3879 9.82882486511016e-09
3880 9.81428627255809e-09
3881 9.82728387555198e-09
3882 9.82456871412296e-09
3883 9.82836301233192e-09
3884 9.82576153774062e-09
3885 9.80948389184277e-09
3886 9.82321424203292e-09
3887 9.82325332188339e-09
3888 9.80797221217244e-09
3889 9.81930714516466e-09
3890 9.81872805283501e-09
3891 9.82614167810425e-09
3892 9.82370806923427e-09
3893 9.80885683787847e-09
3894 9.81748460304743e-09
3895 9.82092096535325e-09
3896 9.82134995552997e-09
3897 9.80354997182076e-09
3898 9.81669501243232e-09
3899 9.81597203519868e-09
3900 9.82192194243225e-09
3901 9.82035874841358e-09
3902 9.81411130140941e-09
3903 9.8004138138208e-09
3904 9.81630510210607e-09
3905 9.81769687768974e-09
3906 9.8174739449064e-09
3907 9.80141567907822e-09
3908 9.80813918971535e-09
3909 9.8145660487603e-09
3910 9.80961623042731e-09
3911 9.79932490707824e-09
3912 9.80868009037295e-09
3913 9.81299042024375e-09
3914 9.81376047093363e-09
3915 9.80986136767115e-09
3916 9.81583170300837e-09
3917 9.81379688624884e-09
3918 9.79661152200606e-09
3919 9.80836478703395e-09
3920 9.80618786172727e-09
3921 9.81400738453431e-09
3922 9.78738512458222e-09
3923 9.80691883256668e-09
3924 9.80491599023026e-09
3925 9.80831682539929e-09
3926 9.8031414097477e-09
3927 9.7926458053621e-09
3928 9.80355085999918e-09
3929 9.80007541784289e-09
3930 9.80797132399402e-09
3931 9.80665504357603e-09
3932 9.79051151261956e-09
3933 9.80165459907312e-09
3934 9.79847314397375e-09
3935 9.80660086469243e-09
3936 9.80350822743503e-09
3937 9.78103553705978e-09
3938 9.8053822839006e-09
3939 9.80232695013683e-09
3940 9.79915082410798e-09
3941 9.80319825316656e-09
3942 9.78113501304279e-09
3943 9.79921654931104e-09
3944 9.79974590364918e-09
3945 9.79864100969507e-09
3946 9.77911529531639e-09
3947 9.79667458267386e-09
3948 9.79627667874183e-09
3949 9.79740644169169e-09
3950 9.79272840595513e-09
3951 9.79960024238835e-09
3952 9.79768266518022e-09
3953 9.78090763936734e-09
3954 9.79207470663823e-09
3955 9.784760557352e-09
3956 9.79614078744362e-09
3957 9.79365744058214e-09
3958 9.77147429637171e-09
3959 9.79126824063314e-09
3960 9.78453584821182e-09
3961 9.77700675974802e-09
3962 9.78651026883881e-09
3963 9.78271952334353e-09
3964 9.79446479476564e-09
3965 9.79056657968158e-09
3966 9.78994485478779e-09
3967 9.78422054487282e-09
3968 9.77307035299191e-09
3969 9.78117942196377e-09
3970 9.78912595428483e-09
3971 9.78726610867398e-09
3972 9.77333236562572e-09
3973 9.78475078738938e-09
3974 9.78488312597392e-09
3975 9.78191039280318e-09
3976 9.78265113360521e-09
3977 9.76419567422226e-09
3978 9.78143432917022e-09
3979 9.78091652115154e-09
3980 9.78688952102402e-09
3981 9.75746239362252e-09
3982 9.77880887376159e-09
3983 9.77420189229861e-09
3984 9.78486625058395e-09
3985 9.78161907028152e-09
3986 9.78115455296802e-09
3987 9.77710534755261e-09
3988 9.76590719403703e-09
3989 9.77506076083046e-09
3990 9.77184644312956e-09
3991 9.76150626996741e-09
3992 9.77399494672682e-09
3993 9.77475700381092e-09
3994 9.7773709129001e-09
3995 9.77292025083898e-09
3996 9.77803527035803e-09
3997 9.77017666770053e-09
3998 9.76067138225289e-09
3999 9.77073089103442e-09
4000 9.77275771418817e-09
4001 9.75880443121468e-09
4002 9.75432534744414e-09
4003 9.76769776173114e-09
4004 9.76457847912116e-09
4005 9.77212089026125e-09
4006 9.76906822103274e-09
4007 9.77875913577009e-09
4008 9.75030012284606e-09
4009 9.76883463010836e-09
4010 9.76393277341003e-09
4011 9.76988534517886e-09
4012 9.76890657256035e-09
4013 9.7701891021984e-09
4014 9.76562297694272e-09
4015 9.75350822329801e-09
4016 9.75897318511443e-09
4017 9.76786207473879e-09
4018 9.76614433767509e-09
4019 9.76621805648392e-09
4020 9.76168568200819e-09
4021 9.7496259954255e-09
4022 9.75849800965989e-09
4023 9.76559455523329e-09
4024 9.76333858204725e-09
4025 9.76349312509228e-09
4026 9.75855218854349e-09
4027 9.74564962064051e-09
4028 9.75632286071004e-09
4029 9.75411040826657e-09
4030 9.76249214801328e-09
4031 9.75742153741521e-09
4032 9.76340608360715e-09
4033 9.75985603446361e-09
4034 9.73918101720983e-09
4035 9.75540892511617e-09
4036 9.75698899452482e-09
4037 9.75747749265565e-09
4038 9.74538050257934e-09
4039 9.73769154199999e-09
4040 9.75970060324016e-09
4041 9.75630065624955e-09
4042 9.75348335430226e-09
4043 9.74448965962438e-09
4044 9.74239000584021e-09
4045 9.75439640171771e-09
4046 9.74973968226323e-09
4047 9.75258451774152e-09
4048 9.74794467367701e-09
4049 9.75831504490543e-09
4050 9.73279323801535e-09
4051 9.74993330515872e-09
4052 9.74829017508227e-09
4053 9.74974856404742e-09
4054 9.74328884240094e-09
4055 9.73289449035519e-09
4056 9.74323377533892e-09
4057 9.74917835350197e-09
4058 9.74589120517066e-09
4059 9.72988534186925e-09
4060 9.7424752709685e-09
4061 9.73934444203906e-09
4062 9.75114922141529e-09
4063 9.74652181184865e-09
4064 9.74423297606108e-09
4065 9.73960823102971e-09
4066 9.7298560319814e-09
4067 9.73342917376385e-09
4068 9.74451541679855e-09
4069 9.74161284972297e-09
4070 9.7427008682871e-09
4071 9.73915970092776e-09
4072 9.74167413403393e-09
4073 9.74250458085635e-09
4074 9.7370200791147e-09
4075 9.72614966343599e-09
4076 9.7329309056704e-09
4077 9.74008429466267e-09
4078 9.73925029512657e-09
4079 9.71845803832139e-09
4080 9.7354870831623e-09
4081 9.7320427272507e-09
4082 9.74009140009002e-09
4083 9.73549596494649e-09
4084 9.7133279197692e-09
4085 9.73681046900765e-09
4086 9.73407221493972e-09
4087 9.73310854135434e-09
4088 9.72833102963477e-09
4089 9.71960023576912e-09
4090 9.72185443259832e-09
4091 9.73272840099071e-09
4092 9.7302663704113e-09
4093 9.73182867625155e-09
4094 9.72253921815991e-09
4095 9.71686109352277e-09
4096 9.72058789017183e-09
4097 9.73041114349371e-09
4098 9.72579972113863e-09
4099 9.73241132129488e-09
4100 9.72881508687351e-09
4101 9.72917479913349e-09
4102 9.72512914643175e-09
4103 9.73276748084118e-09
4104 9.71149916040304e-09
4105 9.72528724219046e-09
4106 9.72383151776057e-09
4107 9.7240402396892e-09
4108 9.71859037690592e-09
4109 9.70968283553475e-09
4110 9.71570113250664e-09
4111 9.71702984742251e-09
4112 9.7232648599288e-09
4113 9.72016955813615e-09
4114 9.71394698012773e-09
4115 9.70365654495708e-09
4116 9.72402069976397e-09
4117 9.72025127055076e-09
4118 9.71548530515065e-09
4119 9.7212495830945e-09
4120 9.72284208700103e-09
4121 9.69506608328174e-09
4122 9.71800506732734e-09
4123 9.71476232791701e-09
4124 9.71510161207334e-09
4125 9.71039071373525e-09
4126 9.70159153013128e-09
4127 9.71037739105896e-09
4128 9.70303926095539e-09
4129 9.71553237860689e-09
4130 9.71441149744123e-09
4131 9.71138724992215e-09
4132 9.71499858337666e-09
4133 9.69413171958422e-09
4134 9.71124691773184e-09
4135 9.70869873384572e-09
4136 9.71472147170971e-09
4137 9.71178870656786e-09
4138 9.71407487782017e-09
4139 9.69072377898783e-09
4140 9.70819868939543e-09
4141 9.7168895152322e-09
4142 9.68461311146029e-09
4143 9.70472058270389e-09
4144 9.70434577141077e-09
4145 9.70572422431815e-09
4146 9.70413882583898e-09
4147 9.70806102174038e-09
4148 9.70548885703693e-09
4149 9.69664526451197e-09
4150 9.68713464999382e-09
4151 9.70276303746687e-09
4152 9.70293267954503e-09
4153 9.70332081351444e-09
4154 9.69802194106251e-09
4155 9.70737357164353e-09
4156 9.68301172576957e-09
4157 9.7024983602978e-09
4158 9.70168301250851e-09
4159 9.70011004852722e-09
4160 9.69695346242361e-09
4161 9.70317159953993e-09
4162 9.70459002047619e-09
4163 9.68156665948072e-09
4164 9.69619762258844e-09
4165 9.69692504071418e-09
4166 9.69084901214501e-09
4167 9.70203117844903e-09
4168 9.68055857697436e-09
4169 9.6952490480362e-09
4170 9.69652536042531e-09
4171 9.69676872131231e-09
4172 9.6927106341127e-09
4173 9.6966825680056e-09
4174 9.67446212030154e-09
4175 9.69577396148225e-09
4176 9.69479430068532e-09
4177 9.69172386788841e-09
4178 9.68937463596831e-09
4179 9.69371072301328e-09
4180 9.69402869088754e-09
4181 9.68800328848829e-09
4182 9.67528723805344e-09
4183 9.68895363939737e-09
4184 9.6915337977066e-09
4185 9.69018376650865e-09
4186 9.68523305999724e-09
4187 9.69751035029276e-09
4188 9.67061541956582e-09
4189 9.68752722485533e-09
4190 9.68699254144667e-09
4191 9.68592228645093e-09
4192 9.68600577522238e-09
4193 9.66855040474002e-09
4194 9.68251345767612e-09
4195 9.68322755312556e-09
4196 9.66683710856842e-09
4197 9.68291846703551e-09
4198 9.68199209694376e-09
4199 9.67793578610099e-09
4200 9.6828101092683e-09
4201 9.68204982854104e-09
4202 9.68439195503379e-09
4203 9.68711511006859e-09
4204 9.66559809967293e-09
4205 9.67865965151304e-09
4206 9.68015267943656e-09
4207 9.67581037514265e-09
4208 9.68542401835748e-09
4209 9.6795549353601e-09
4210 9.67947411112391e-09
4211 9.68478719443056e-09
4212 9.68166435910689e-09
4213 9.67798374773565e-09
4214 9.68210400742464e-09
4215 9.68088986752491e-09
4216 9.67840740884185e-09
4217 9.66738955554547e-09
4218 9.67003188634408e-09
4219 9.67830082743149e-09
4220 9.67575086718853e-09
4221 9.67657243222675e-09
4222 9.67310498367624e-09
4223 9.6669987570408e-09
4224 9.67146274177821e-09
4225 9.67019708753014e-09
4226 9.67297975051906e-09
4227 9.67429514275864e-09
4228 9.67124691442223e-09
4229 9.67313429356409e-09
4230 9.65223989624064e-09
4231 9.66914281974596e-09
4232 9.67053281897279e-09
4233 9.67086943859385e-09
4234 9.6648165026636e-09
4235 9.67297175691328e-09
4236 9.66792690348939e-09
4237 9.67402602469747e-09
4238 9.67006563712403e-09
4239 9.64877600040381e-09
4240 9.66958513259897e-09
4241 9.6687298167808e-09
4242 9.6670120797171e-09
4243 9.66154267700858e-09
4244 9.66869340146559e-09
4245 9.66717905726e-09
4246 9.66412638803149e-09
4247 9.65311386380563e-09
4248 9.6562819962287e-09
4249 9.66492130771712e-09
4250 9.66190505380382e-09
4251 9.66711510841378e-09
4252 9.66592672568822e-09
4253 9.64202939712777e-09
4254 9.66070512475881e-09
4255 9.66105773159143e-09
4256 9.65986313161693e-09
4257 9.65887636539264e-09
4258 9.66307300842573e-09
4259 9.65772795069597e-09
4260 9.64453139573607e-09
4261 9.65595514657025e-09
4262 9.65286606202653e-09
4263 9.66031876714624e-09
4264 9.65813562459061e-09
4265 9.65787450013522e-09
4266 9.65467616964588e-09
4267 9.64691881932822e-09
4268 9.65355351212338e-09
4269 9.64861790464511e-09
4270 9.65717639189734e-09
4271 9.65336344194156e-09
4272 9.65884350279111e-09
4273 9.65517887863143e-09
4274 9.6552001949135e-09
4275 9.65136148778356e-09
4276 9.65547464204519e-09
4277 9.65422497500867e-09
4278 9.65205515512935e-09
4279 9.65393809337911e-09
4280 9.63227542172262e-09
4281 9.65163682309367e-09
4282 9.65246993445135e-09
4283 9.65021573762215e-09
4284 9.64746504905634e-09
4285 9.65405710928735e-09
4286 9.65113056139444e-09
4287 9.64816759818632e-09
4288 9.65211022219137e-09
4289 9.65213065029502e-09
4290 9.64463175989749e-09
4291 9.6331174148645e-09
4292 9.64820312532311e-09
4293 9.64832480576661e-09
4294 9.64581481355253e-09
4295 9.642192821957e-09
4296 9.63646762386361e-09
4297 9.64083923804537e-09
4298 9.63889057459255e-09
4299 9.6456194143002e-09
4300 9.64014024162907e-09
4301 9.64653157353723e-09
4302 9.64285540305809e-09
4303 9.64351087873183e-09
4304 9.64302149242258e-09
4305 9.62941459903277e-09
4306 9.63409174659091e-09
4307 9.64079660548123e-09
4308 9.63937107911761e-09
4309 9.63959490007937e-09
4310 9.63573487666736e-09
4311 9.63009494370226e-09
4312 9.63348600890868e-09
4313 9.63955315569365e-09
4314 9.63723501001823e-09
4315 9.63622603933345e-09
4316 9.6361514323462e-09
4317 9.63957624833256e-09
4318 9.63693924660447e-09
4319 9.63604041004373e-09
4320 9.62899360246183e-09
4321 9.61772617102952e-09
4322 9.63644186668944e-09
4323 9.6351824296903e-09
4324 9.63489732441758e-09
4325 9.63061008718569e-09
4326 9.63457136293755e-09
4327 9.63438839818309e-09
4328 9.63445856427825e-09
4329 9.62916146818316e-09
4330 9.63300106349152e-09
4331 9.61084101192e-09
4332 9.6285006634389e-09
4333 9.62354640421381e-09
4334 9.63380042406925e-09
4335 9.62946522520269e-09
4336 9.62945900795376e-09
4337 9.62329504972104e-09
4338 9.61593649151382e-09
4339 9.62350377164967e-09
4340 9.62446033980768e-09
4341 9.63008428556122e-09
4342 9.62775548174477e-09
4343 9.62421697892069e-09
4344 9.62903445866914e-09
4345 9.62634683077113e-09
4346 9.62128776649251e-09
4347 9.6281533856768e-09
4348 9.62954516126047e-09
4349 9.62638413426475e-09
4350 9.62192991948996e-09
4351 9.62557500372441e-09
4352 9.62433865936418e-09
4353 9.61848645175678e-09
4354 9.6064676213814e-09
4355 9.62186330610848e-09
4356 9.62450119601499e-09
4357 9.6223375933846e-09
4358 9.61750146188933e-09
4359 9.6235321933591e-09
4360 9.62448254426818e-09
4361 9.62101509571767e-09
4362 9.61863477755287e-09
4363 9.62157020722998e-09
4364 9.59829637992016e-09
4365 9.61663371157329e-09
4366 9.61570734148154e-09
4367 9.61413171296499e-09
4368 9.62119361958003e-09
4369 9.61924317977036e-09
4370 9.61501278595733e-09
4371 9.61148938216638e-09
4372 9.61714796687829e-09
4373 9.61430846047051e-09
4374 9.6148760064807e-09
4375 9.60831414431595e-09
4376 9.59602264316572e-09
4377 9.61214485784012e-09
4378 9.61325863357843e-09
4379 9.61278345812389e-09
4380 9.61077528671694e-09
4381 9.60620294421233e-09
4382 9.61210488981123e-09
4383 9.61160129264726e-09
4384 9.60592849708064e-09
4385 9.60095203339506e-09
4386 9.60696500129643e-09
4387 9.60386792314694e-09
4388 9.60989243736776e-09
4389 9.60739754418682e-09
4390 9.60380930337124e-09
4391 9.61397095267102e-09
4392 9.59504387054722e-09
4393 9.60764978685802e-09
4394 9.60501900237887e-09
4395 9.60551282958022e-09
4396 9.60329060717413e-09
4397 9.60184465270686e-09
4398 9.60824486639922e-09
4399 9.60527035687164e-09
4400 9.60415569295492e-09
4401 9.59454560245376e-09
4402 9.59814983048091e-09
4403 9.60476409517241e-09
4404 9.60363522040097e-09
4405 9.60331814070514e-09
4406 9.59881774065252e-09
4407 9.60712220887672e-09
4408 9.58808499262886e-09
4409 9.6006020910977e-09
4410 9.59920321008667e-09
4411 9.59871560013426e-09
4412 9.59667367794736e-09
4413 9.60103019309599e-09
4414 9.59641166531355e-09
4415 9.59049195614625e-09
4416 9.59637080910625e-09
4417 9.59746326856248e-09
4418 9.58872625744789e-09
4419 9.59950252621411e-09
4420 9.58282697638424e-09
4421 9.59700141578423e-09
4422 9.59359169883101e-09
4423 9.59295753943934e-09
4424 9.59799884014956e-09
4425 9.59376134090917e-09
4426 9.58017132290934e-09
4427 9.59462731486838e-09
4428 9.59399937272565e-09
4429 9.59311741155489e-09
4430 9.59152135493468e-09
4431 9.58799706296531e-09
4432 9.59081969398312e-09
4433 9.59112878007318e-09
4434 9.5951602219202e-09
4435 9.5925543064368e-09
4436 9.59065715733232e-09
4437 9.58410151241651e-09
4438 9.58092094549556e-09
4439 9.58947588003412e-09
4440 9.58888080049292e-09
4441 9.58084456215147e-09
4442 9.59345847206805e-09
4443 9.5747054729145e-09
4444 9.59408819056762e-09
4445 9.59369295117085e-09
4446 9.58622159430433e-09
4447 9.59464951932887e-09
4448 9.59531298860838e-09
4449 9.58551282792541e-09
4450 9.59260049171462e-09
4451 9.59497459263048e-09
4452 9.59585300108756e-09
4453 9.5759675744489e-09
4454 9.59473744899242e-09
4455 9.59338475325922e-09
4456 9.58329149369774e-09
4457 9.59005586054218e-09
4458 9.59059942573504e-09
4459 9.58230383929504e-09
4460 9.58671808604095e-09
4461 9.59204804473757e-09
4462 9.57669499257463e-09
4463 9.59267953959397e-09
4464 9.57442392035546e-09
4465 9.58723322952437e-09
4466 9.5826493407003e-09
4467 9.57851664651344e-09
4468 9.58724122313015e-09
4469 9.58278256746325e-09
4470 9.57764623166213e-09
4471 9.58632551117944e-09
4472 9.58498791447937e-09
4473 9.56772261417882e-09
4474 9.58482271329331e-09
4475 9.58572510256772e-09
4476 9.56942702856622e-09
4477 9.58492307745473e-09
4478 9.58350909741057e-09
4479 9.56884349534448e-09
4480 9.58524815075634e-09
4481 9.58262713623981e-09
4482 9.56833456910999e-09
4483 9.58333767897557e-09
4484 9.58406332074446e-09
4485 9.56951673458661e-09
4486 9.58177892584899e-09
4487 9.58072110535113e-09
4488 9.56503232174555e-09
4489 9.57990131666975e-09
4490 9.5808863065372e-09
4491 9.58057810862556e-09
4492 9.58176205045902e-09
4493 9.56527390627571e-09
4494 9.57694368253215e-09
4495 9.57895185393909e-09
4496 9.57485468688901e-09
4497 9.56784962369284e-09
4498 9.5762606733274e-09
4499 9.57361745435037e-09
4500 9.58392298855415e-09
4501 9.56419921038787e-09
4502 9.57881685081929e-09
4503 9.57650936328491e-09
4504 9.56098489268697e-09
4505 9.56408996444225e-09
4506 9.57192636263926e-09
4507 9.57019352654243e-09
4508 9.57604218143615e-09
4509 9.57524282085842e-09
4510 9.57294421510824e-09
4511 9.55982404349243e-09
4512 9.57322754402412e-09
4513 9.57325951844723e-09
4514 9.55287049464459e-09
4515 9.57127799239288e-09
4516 9.57080459329518e-09
4517 9.55564960491984e-09
4518 9.5704404401431e-09
4519 9.57069890006323e-09
4520 9.55717904815856e-09
4521 9.57065626749909e-09
4522 9.56697210341417e-09
4523 9.5687857637472e-09
4524 9.5684296042009e-09
4525 9.56149648345672e-09
4526 9.55916856781869e-09
4527 9.55580681250012e-09
4528 9.56532186791037e-09
4529 9.56603773971665e-09
4530 9.5607921579699e-09
4531 9.55678203240495e-09
4532 9.5644709929843e-09
4533 9.56124157625027e-09
4534 9.56749790503864e-09
4535 9.56538137586449e-09
4536 9.55056478346705e-09
4537 9.56474721647282e-09
4538 9.56193435541763e-09
4539 9.54773060612979e-09
4540 9.56408285901489e-09
4541 9.56122292450345e-09
4542 9.561725633489e-09
4543 9.55775014688243e-09
4544 9.55183665496406e-09
4545 9.56230206128339e-09
4546 9.56174162070056e-09
4547 9.56538404039975e-09
4548 9.54819867615697e-09
4549 9.56007717434204e-09
4550 9.55990753226388e-09
4551 9.54461309987664e-09
4552 9.56298151777446e-09
4553 9.56111279037941e-09
4554 9.55860279816534e-09
4555 9.55844736694189e-09
4556 9.54996792756901e-09
4557 9.54862855451211e-09
4558 9.55523926648993e-09
4559 9.5498995378307e-09
4560 9.54678647246965e-09
4561 9.55389811707619e-09
4562 9.54989154422492e-09
4563 9.54464240976449e-09
4564 9.55483159259529e-09
4565 9.5526813126412e-09
4566 9.54711421030652e-09
4567 9.55248413703202e-09
4568 9.55317691619939e-09
4569 9.54971124400572e-09
4570 9.55541956670913e-09
4571 9.55441326055961e-09
4572 9.55397805313396e-09
4573 9.54499412841869e-09
4574 9.5472980632394e-09
4575 9.55139345393263e-09
4576 9.54684509224535e-09
4577 9.54323375879085e-09
4578 9.54864720625892e-09
4579 9.549354196281e-09
4580 9.54886480997175e-09
4581 9.55433776539394e-09
4582 9.54678647246965e-09
4583 9.54100354277898e-09
4584 9.54693479826574e-09
4585 9.54081347259716e-09
4586 9.54816226084176e-09
4587 9.54611145687068e-09
4588 9.54836121280778e-09
4589 9.54008072540091e-09
4590 9.53981160733974e-09
4591 9.54513001971691e-09
4592 9.54020773491493e-09
4593 9.53589385233045e-09
4594 9.54391765617402e-09
4595 9.53895629152157e-09
4596 9.55630508059357e-09
4597 9.55050794004819e-09
4598 9.54617185300322e-09
4599 9.53915435530917e-09
4600 9.53282075499828e-09
4601 9.54212531212306e-09
4602 9.53505274736699e-09
4603 9.54314760548414e-09
4604 9.5455536808231e-09
4605 9.53978229745189e-09
4606 9.54568335487238e-09
4607 9.54164747213326e-09
4608 9.54279766318678e-09
4609 9.53652001811633e-09
4610 9.53274703618945e-09
4611 9.53933732006362e-09
4612 9.5388266174723e-09
4613 9.53036849438149e-09
4614 9.53569134765075e-09
4615 9.53451539942307e-09
4616 9.54086498694551e-09
4617 9.53849355056491e-09
4618 9.53883727561333e-09
4619 9.53028944650214e-09
4620 9.53235534950636e-09
4621 9.53522150126673e-09
4622 9.53960821448163e-09
4623 9.53550571836104e-09
4624 9.53504830647489e-09
4625 9.53298773254119e-09
4626 9.53829548677732e-09
4627 9.53684420323953e-09
4628 9.51828571515989e-09
4629 9.5346566197918e-09
4630 9.53173273643415e-09
4631 9.52397183340281e-09
4632 9.53172119011469e-09
4633 9.52334389126008e-09
4634 9.52664169773243e-09
4635 9.53059675623535e-09
4636 9.52942169618609e-09
4637 9.53170697925998e-09
4638 9.53095202760323e-09
4639 9.524124600091e-09
4640 9.53557144356409e-09
4641 9.53099732470264e-09
4642 9.52059675540795e-09
4643 9.52809742216232e-09
4644 9.52323819802814e-09
4645 9.53219281285556e-09
4646 9.52818801636113e-09
4647 9.51683354344368e-09
4648 9.52831324951831e-09
4649 9.52815604193802e-09
4650 9.52628020911561e-09
4651 9.52433598655489e-09
4652 9.53030543371369e-09
4653 9.52782652774431e-09
4654 9.52261292042067e-09
4655 9.51803791338079e-09
4656 9.52372847251581e-09
4657 9.52185175151499e-09
4658 9.5274925726585e-09
4659 9.51883549760169e-09
4660 9.51260048509539e-09
4661 9.52335987847164e-09
4662 9.52598089298817e-09
4663 9.51111100988555e-09
4664 9.51675982463485e-09
4665 9.51546663685576e-09
4666 9.52298773171378e-09
4667 9.52054612923803e-09
4668 9.52027701117686e-09
4669 9.51316092567822e-09
4670 9.52278700339093e-09
4671 9.51709289154223e-09
4672 9.52400469600434e-09
4673 9.51865342102565e-09
4674 9.51525702674871e-09
4675 9.52143430765773e-09
4676 9.51816137018113e-09
4677 9.51832745954562e-09
4678 9.51153822370543e-09
4679 9.52746948001959e-09
4680 9.50985423742168e-09
4681 9.5145891165771e-09
4682 9.51046796870969e-09
4683 9.51804324245131e-09
4684 9.51187217879124e-09
4685 9.52224077366282e-09
4686 9.51756717881835e-09
4687 9.51547640681838e-09
4688 9.51504741664166e-09
4689 9.50802192534184e-09
4690 9.5078354078737e-09
4691 9.5111536424497e-09
4692 9.50870671090343e-09
4693 9.51362810752698e-09
4694 9.51322665088128e-09
4695 9.50776257724328e-09
4696 9.50334744231895e-09
4697 9.50830703061456e-09
4698 9.50518952436141e-09
4699 9.51216616584816e-09
4700 9.51011980276917e-09
4701 9.50758494155934e-09
4702 9.51205425536727e-09
4703 9.50965617363408e-09
4704 9.50985867831378e-09
4705 9.50568335156277e-09
4706 9.50404555055684e-09
4707 9.50780787434269e-09
4708 9.50335099503263e-09
4709 9.50886303030529e-09
4710 9.50142631239714e-09
4711 9.49392564564278e-09
4712 9.50597822679811e-09
4713 9.50252410092389e-09
4714 9.50876444250071e-09
4715 9.50474365879472e-09
4716 9.49849177089845e-09
4717 9.50861434034778e-09
4718 9.49208533995716e-09
4719 9.50331102700375e-09
4720 9.50034806379563e-09
4721 9.50248768560868e-09
4722 9.50183398629179e-09
4723 9.49539913364106e-09
4724 9.49843848019327e-09
4725 9.49590450716187e-09
4726 9.50225764739798e-09
4727 9.50020861978373e-09
4728 9.50103729024931e-09
4729 9.49739309419328e-09
4730 9.50073175687294e-09
4731 9.50001499688824e-09
4732 9.49997236432409e-09
4733 9.50087741813377e-09
4734 9.49668699234962e-09
4735 9.49342471301406e-09
4736 9.47941281026488e-09
4737 9.4955678875408e-09
4738 9.49325684729274e-09
4739 9.48847045378898e-09
4740 9.49582368292567e-09
4741 9.49182688003702e-09
4742 9.49327727539639e-09
4743 9.48864364858082e-09
4744 9.49416278928084e-09
4745 9.49168743602513e-09
4746 9.49187572985011e-09
4747 9.48620737517558e-09
4748 9.49625889035133e-09
4749 9.49220524404382e-09
4750 9.47365119685628e-09
4751 9.4888772395052e-09
4752 9.48713996251627e-09
4753 9.48795975119765e-09
4754 9.48237044440248e-09
4755 9.49449674436664e-09
4756 9.47489731117912e-09
4757 9.48690814794872e-09
4758 9.48378175991138e-09
4759 9.48617628893089e-09
4760 9.47965084208136e-09
4761 9.48847045378898e-09
4762 9.4828491725707e-09
4763 9.48284384350018e-09
4764 9.48849976367683e-09
4765 9.48364498043475e-09
4766 9.48458556138121e-09
4767 9.48108436205075e-09
4768 9.47299927389622e-09
4769 9.47961531494457e-09
4770 9.47910461235324e-09
4771 9.47146183705172e-09
4772 9.480987550603e-09
4773 9.48537515199632e-09
4774 9.47577483145778e-09
4775 9.4726466670636e-09
4776 9.48922096455362e-09
4777 9.48050171700743e-09
4778 9.47854061905673e-09
4779 9.48822176383146e-09
4780 9.48045819626486e-09
4781 9.47210221369232e-09
4782 9.48770306763436e-09
4783 9.46659017841966e-09
4784 9.47273282037031e-09
4785 9.46802369838906e-09
4786 9.47851930277466e-09
4787 9.47431821884948e-09
4788 9.47537248663366e-09
4789 9.46825640113502e-09
4790 9.47260314632103e-09
4791 9.47290423880531e-09
4792 9.47272216222927e-09
4793 9.46708222926418e-09
4794 9.47566647369058e-09
4795 9.47116607363796e-09
4796 9.47685929730824e-09
4797 9.47284828356487e-09
4798 9.47292289055213e-09
4799 9.46784339816986e-09
4800 9.47548883800664e-09
4801 9.46908773613586e-09
4802 9.46661238288016e-09
4803 9.4779259995903e-09
4804 9.46991285388776e-09
4805 9.46841716142899e-09
4806 9.4664924787935e-09
4807 9.47191569622419e-09
4808 9.46831324455388e-09
4809 9.46609013396937e-09
4810 9.47290335062689e-09
4811 9.47400202733206e-09
4812 9.46297529225149e-09
4813 9.46878664365158e-09
4814 9.46806721913163e-09
4815 9.46573130988781e-09
4816 9.45803080298901e-09
4817 9.47242018156658e-09
4818 9.46624201247914e-09
4819 9.46676337321151e-09
4820 9.45895362036708e-09
4821 9.46884526342728e-09
4822 9.46080991326426e-09
4823 9.47052480881894e-09
4824 9.46292111336788e-09
4825 9.46973699456066e-09
4826 9.46283318370433e-09
4827 9.46096800902296e-09
4828 9.47082856583847e-09
4829 9.45891542869504e-09
4830 9.46337852525403e-09
4831 9.45849709665936e-09
4832 9.46262801448938e-09
4833 9.46101064158711e-09
4834 9.45514511130341e-09
4835 9.46159595116569e-09
4836 9.46523392997278e-09
4837 9.44935951707748e-09
4838 9.4597298883059e-09
4839 9.46178335681225e-09
4840 9.4602077282957e-09
4841 9.4543510797962e-09
4842 9.46071576635177e-09
4843 9.45325151491261e-09
4844 9.45492040216322e-09
4845 9.44294775706567e-09
4846 9.45693390264069e-09
4847 9.45893852133395e-09
4848 9.45364408977412e-09
4849 9.45802636209692e-09
4850 9.45332523372144e-09
4851 9.43703337696888e-09
4852 9.45846334587941e-09
4853 9.45263245455408e-09
4854 9.43670830366727e-09
4855 9.45724654144442e-09
4856 9.4417362817012e-09
4857 9.45416811504174e-09
4858 9.43345401793749e-09
4859 9.45574551991513e-09
4860 9.43790467999861e-09
4861 9.44818889792032e-09
4862 9.44618872011915e-09
4863 9.45030365073762e-09
4864 9.44551636905544e-09
4865 9.43998657021439e-09
4866 9.43580147350076e-09
4867 9.45545064467979e-09
4868 9.43191835744983e-09
4869 9.45007183617008e-09
4870 9.44444789041654e-09
4871 9.4421919172305e-09
4872 9.43988087698244e-09
4873 9.44226208332566e-09
4874 9.447877147295e-09
4875 9.43778655226879e-09
4876 9.45063760582343e-09
4877 9.42966771333431e-09
4878 9.44938616243007e-09
4879 9.43085343152461e-09
4880 9.44572065009197e-09
4881 9.42910460821622e-09
4882 9.45335276725245e-09
4883 9.43938793795951e-09
4884 9.44031430805126e-09
4885 9.442313597674e-09
4886 9.43342381987122e-09
4887 9.44029654448286e-09
4888 9.422890911992e-09
4889 9.4465741895533e-09
4890 9.42903888301316e-09
4891 9.43535560793407e-09
4892 9.44279054948538e-09
4893 9.44006295355848e-09
4894 9.43767020089581e-09
4895 9.43988442969612e-09
4896 9.43771993888731e-09
4897 9.43630773519999e-09
4898 9.43673406084145e-09
4899 9.42651823265805e-09
4900 9.4428633801158e-09
4901 9.424378610845e-09
4902 9.43557942889584e-09
4903 9.43240507922383e-09
4904 9.4210426127006e-09
4905 9.44117228840469e-09
4906 9.43545863663076e-09
4907 9.42227273981189e-09
4908 9.44080902343103e-09
4909 9.43479694370808e-09
4910 9.42203559617383e-09
4911 9.43963307520335e-09
4912 9.43814271181509e-09
4913 9.42998923392224e-09
4914 9.42895894695539e-09
4915 9.43190503477354e-09
4916 9.43032851807857e-09
4917 9.42952027571664e-09
4918 9.43118472207516e-09
4919 9.42883726651189e-09
4920 9.42788780378123e-09
4921 9.43154088162146e-09
4922 9.42384392743634e-09
4923 9.43277100873274e-09
4924 9.41923694597335e-09
4925 9.42639299950088e-09
4926 9.42143607574053e-09
4927 9.42819422533603e-09
4928 9.42244238189005e-09
4929 9.42849887053399e-09
4930 9.41460687187146e-09
4931 9.43273104070386e-09
4932 9.41921918240496e-09
4933 9.42803701775574e-09
4934 9.42925471036915e-09
4935 9.41531830278564e-09
4936 9.42813738191717e-09
4937 9.42041644691471e-09
4938 9.42857347752124e-09
4939 9.41420008615523e-09
4940 9.4256815685867e-09
4941 9.42340339094017e-09
4942 9.42096978207019e-09
4943 9.42549949201066e-09
4944 9.42208799870059e-09
4945 9.43043598766735e-09
4946 9.42002298387479e-09
4947 9.41765421202945e-09
4948 9.42541422688237e-09
4949 9.40917210812131e-09
4950 9.42707778506247e-09
4951 9.41054345560133e-09
4952 9.42546218851703e-09
4953 9.41430844392244e-09
4954 9.42116340496568e-09
4955 9.40525524129043e-09
4956 9.42541067416869e-09
4957 9.40782296510179e-09
4958 9.42065803144487e-09
4959 9.40232158797016e-09
4960 9.42374267509649e-09
4961 9.41458999648148e-09
4962 9.4147845075554e-09
4963 9.41471434146024e-09
4964 9.40702182816722e-09
4965 9.42051769925456e-09
4966 9.41579969548911e-09
4967 9.421817992461e-09
4968 9.40886213385284e-09
4969 9.41797573261738e-09
4970 9.41870315074311e-09
4971 9.40295841189709e-09
4972 9.40990751985282e-09
4973 9.41951228128346e-09
4974 9.41760003314585e-09
4975 9.40812672212132e-09
4976 9.41322575442882e-09
4977 9.39991195991752e-09
4978 9.42172384554851e-09
4979 9.40333855226072e-09
4980 9.41584232805326e-09
4981 9.39762756502205e-09
4982 9.42014466431829e-09
4983 9.40225586276711e-09
4984 9.41238820217905e-09
4985 9.40893585266167e-09
4986 9.41570110768453e-09
4987 9.41362809925295e-09
4988 9.40146627215199e-09
4989 9.41377553687062e-09
4990 9.39578104208749e-09
4991 9.41468059068029e-09
4992 9.40070687960315e-09
4993 9.40434574658866e-09
4994 9.41224609363189e-09
4995 9.41460154280094e-09
4996 9.39340694117163e-09
4997 9.41293798462084e-09
4998 9.39584854364739e-09
4999 9.40809830041189e-09
};
\addlegendentry{Test}

\nextgroupplot[
title={ELU/Tanh $\rare$},
ymin=7.68070578300272e-09, ymax=1e-05,
]
\addplot [semithick, black, dashed]
table {%
0 0.00496433921059361
1 0.000251943005154317
2 0.000166717545652318
3 0.000123879264922834
4 5.53700450734596e-05
5 3.07928690497761e-05
6 2.76195705380928e-05
7 2.53265211790961e-05
8 2.11848385095834e-05
9 1.43151888045026e-05
10 8.12326572395961e-06
11 5.99091211361724e-06
12 5.58459134953182e-06
13 5.43256214022847e-06
14 5.3074720219044e-06
15 5.18181798737771e-06
16 5.04638739734276e-06
17 4.8928834311468e-06
18 4.7120671408365e-06
19 4.49328674242366e-06
20 4.22380738304184e-06
21 3.88938836183428e-06
22 3.48027813654994e-06
23 3.00537395322209e-06
24 2.51080811458593e-06
25 2.07969259896856e-06
26 1.78326463436917e-06
27 1.62116445736515e-06
28 1.54417544135299e-06
29 1.50760969331287e-06
30 1.48736334873156e-06
31 1.47252671288989e-06
32 1.45981917999904e-06
33 1.44819551519504e-06
34 1.43731811046344e-06
35 1.4270398249856e-06
36 1.41730437305831e-06
37 1.40805910676711e-06
38 1.39925491063586e-06
39 1.39084785916488e-06
40 1.3827861316642e-06
41 1.37503995911104e-06
42 1.36756879167166e-06
43 1.36033585891404e-06
44 1.35331133430761e-06
45 1.34647246148845e-06
46 1.33979780780535e-06
47 1.33325682874919e-06
48 1.32681472111607e-06
49 1.32044056453573e-06
50 1.31410525488818e-06
51 1.30778237514306e-06
52 1.30145137374171e-06
53 1.29510919818543e-06
54 1.28875247180105e-06
55 1.2822265382546e-06
56 1.27557188113769e-06
57 1.26881068195672e-06
58 1.26190474984611e-06
59 1.25483589610198e-06
60 1.24759061028357e-06
61 1.24015110931452e-06
62 1.23248881821425e-06
63 1.22458044946683e-06
64 1.21640894419528e-06
65 1.20796421181169e-06
66 1.19922575369813e-06
67 1.19016918611692e-06
68 1.18077157481622e-06
69 1.17101282600274e-06
70 1.16086985305941e-06
71 1.15031567301216e-06
72 1.13932128540029e-06
73 1.12785879776567e-06
74 1.1158988772948e-06
75 1.10341066983821e-06
76 1.09037156694569e-06
77 1.07675558267317e-06
78 1.06254026292163e-06
79 1.0477074172357e-06
80 1.03224722569095e-06
81 1.01614941953443e-06
82 9.99407170588995e-07
83 9.81992707330903e-07
84 9.6391538676599e-07
85 9.45225028520014e-07
86 9.25980061396103e-07
87 9.06247126494009e-07
88 8.86109292331128e-07
89 8.65684644399067e-07
90 8.45114016055248e-07
91 8.24558397198416e-07
92 8.04199635279446e-07
93 7.84247526475212e-07
94 7.64872490346846e-07
95 7.46226036863362e-07
96 7.28426828178641e-07
97 7.115826259394e-07
98 6.95777556153132e-07
99 6.81055349868842e-07
100 6.67422909099713e-07
101 6.54845880575294e-07
102 6.43253672286903e-07
103 6.3255554563213e-07
104 6.22656146145317e-07
105 6.13463013515059e-07
106 6.04879987886164e-07
107 5.96810411533255e-07
108 5.89162452413916e-07
109 5.81861323517785e-07
110 5.74841026608652e-07
111 5.68048364311125e-07
112 5.61438667116221e-07
113 5.54977051844929e-07
114 5.48634759006461e-07
115 5.42388897187607e-07
116 5.3622194157299e-07
117 5.30118966413085e-07
118 5.24067960483876e-07
119 5.18060058666947e-07
120 5.12090595680803e-07
121 5.06153441953927e-07
122 5.00245497473628e-07
123 4.9436369767264e-07
124 4.8850534913214e-07
125 4.82671034340854e-07
126 4.76859477853253e-07
127 4.71071916333443e-07
128 4.65309668200575e-07
129 4.5957617206227e-07
130 4.53870374673926e-07
131 4.48196460283157e-07
132 4.42556790192228e-07
133 4.3695614058592e-07
134 4.31401063766046e-07
135 4.25893971822688e-07
136 4.20440599253169e-07
137 4.150449414837e-07
138 4.09711449922412e-07
139 4.04444479183752e-07
140 3.99249968534576e-07
141 3.94133403748498e-07
142 3.89102286447951e-07
143 3.84160463541505e-07
144 3.79312031805412e-07
145 3.74559993462142e-07
146 3.699073471779e-07
147 3.65360534649639e-07
148 3.6092560200629e-07
149 3.56610394458556e-07
150 3.52417352088352e-07
151 3.48351612721842e-07
152 3.44414643162594e-07
153 3.40610703341682e-07
154 3.36941219982734e-07
155 3.3341059077685e-07
156 3.30022310350842e-07
157 3.26780262754234e-07
158 3.23682869920461e-07
159 3.20726581987785e-07
160 3.17902504300527e-07
161 3.15196158667597e-07
162 3.12595441699592e-07
163 3.10093865002692e-07
164 3.07702629599227e-07
165 3.05438054481577e-07
166 3.03295566884465e-07
167 3.01258462674703e-07
168 2.99315673298217e-07
169 2.97467216079816e-07
170 2.95719429086461e-07
171 2.94073857213917e-07
172 2.9252428539106e-07
173 2.91060446124369e-07
174 2.8967253627421e-07
175 2.8835318878162e-07
176 2.87094246398389e-07
177 2.85895834049477e-07
178 2.84751816944429e-07
179 2.83658270737064e-07
180 2.82612631588464e-07
181 2.81607796760674e-07
182 2.80641176765961e-07
183 2.79707522277128e-07
184 2.78806086910777e-07
185 2.77932871725284e-07
186 2.77085957304024e-07
187 2.76262710603881e-07
188 2.75462787072023e-07
189 2.74682546121063e-07
190 2.73920876106004e-07
191 2.73175470884013e-07
192 2.72445650775666e-07
193 2.71729725661807e-07
194 2.71026956706777e-07
195 2.70336783932734e-07
196 2.69657007880397e-07
197 2.68987926659747e-07
198 2.68328719349675e-07
199 2.67678464468624e-07
200 2.67036849071367e-07
201 2.66403672788407e-07
202 2.65777688277069e-07
203 2.6515931761395e-07
204 2.6454835578793e-07
205 2.6394297781529e-07
206 2.63344534674026e-07
207 2.62751846500819e-07
208 2.62165530802605e-07
209 2.61584046747032e-07
210 2.61008147247388e-07
211 2.60437575227535e-07
212 2.59871563990544e-07
213 2.59309879869996e-07
214 2.58753554254199e-07
215 2.5820038069746e-07
216 2.57652677351672e-07
217 2.57106854462563e-07
218 2.56566625982835e-07
219 2.56028837173794e-07
220 2.55495117424509e-07
221 2.54963806447073e-07
222 2.54436571040628e-07
223 2.53911683069319e-07
224 2.53390267777043e-07
225 2.52870927418769e-07
226 2.52354373571073e-07
227 2.5183841006271e-07
228 2.51324618405491e-07
229 2.50811091145664e-07
230 2.50298826554918e-07
231 2.49787370606214e-07
232 2.49279032019878e-07
233 2.48774151354603e-07
234 2.48271922671961e-07
235 2.47773354793424e-07
236 2.47278231856996e-07
237 2.46786588144587e-07
238 2.46298280145218e-07
239 2.45812203027107e-07
240 2.45329055136523e-07
241 2.44848242800799e-07
242 2.443710301403e-07
243 2.43895031066366e-07
244 2.43422050480113e-07
245 2.42950807273878e-07
246 2.42481707336317e-07
247 2.42013590788126e-07
248 2.41548941620096e-07
249 2.41084532551561e-07
250 2.40622732373019e-07
251 2.40162203170691e-07
252 2.39703217238763e-07
253 2.39246121003234e-07
254 2.38790061139582e-07
255 2.38335326842254e-07
256 2.37882133762746e-07
257 2.374306508095e-07
258 2.36979619517186e-07
259 2.36531077395519e-07
260 2.3608259586716e-07
261 2.35636748361578e-07
262 2.3519098604563e-07
263 2.34746242285588e-07
264 2.34302819648313e-07
265 2.33860186711787e-07
266 2.33419250944777e-07
267 2.3297841336678e-07
268 2.325382395858e-07
269 2.32097174350798e-07
270 2.31654034640982e-07
271 2.31205148804925e-07
272 2.30740171174304e-07
273 2.30249134420824e-07
274 2.29799562830202e-07
275 2.29391349239272e-07
276 2.28964587630998e-07
277 2.28534516617529e-07
278 2.28106942517847e-07
279 2.27679335695008e-07
280 2.27253402915295e-07
281 2.26827444779865e-07
282 2.26402183145957e-07
283 2.25979084750527e-07
284 2.25555710653857e-07
285 2.2513326936302e-07
286 2.24711161796343e-07
287 2.2429072389496e-07
288 2.23870648819791e-07
289 2.23451706079736e-07
290 2.23032696956338e-07
291 2.22614630406959e-07
292 2.22197514381683e-07
293 2.21781127723197e-07
294 2.21366028799075e-07
295 2.20949520992164e-07
296 2.20534756910418e-07
297 2.20120638583232e-07
298 2.19706842625733e-07
299 2.19294612895382e-07
300 2.18882175387236e-07
301 2.1847069036518e-07
302 2.18060111075502e-07
303 2.17651168863853e-07
304 2.17241879251873e-07
305 2.16833991739662e-07
306 2.16425652399543e-07
307 2.1601790931669e-07
308 2.15612276291743e-07
309 2.15205686578734e-07
310 2.14799489490325e-07
311 2.14394384894234e-07
312 2.13989285451355e-07
313 2.13584765631047e-07
314 2.13179362988036e-07
315 2.12776523381031e-07
316 2.12372889648371e-07
317 2.11969088775632e-07
318 2.11566103145877e-07
319 2.11163533440306e-07
320 2.10760748071692e-07
321 2.10358902456242e-07
322 2.09957576140063e-07
323 2.0955639729614e-07
324 2.09154714697846e-07
325 2.08754953536783e-07
326 2.08354080701412e-07
327 2.07954585491166e-07
328 2.07555117087921e-07
329 2.07156305460643e-07
330 2.06758999278023e-07
331 2.06361065053784e-07
332 2.05965256014906e-07
333 2.05569374282177e-07
334 2.05174662202268e-07
335 2.04779446839609e-07
336 2.04386427268943e-07
337 2.03993140303282e-07
338 2.03600182407726e-07
339 2.03208166009716e-07
340 2.02816687682628e-07
341 2.02425827962927e-07
342 2.02035635004094e-07
343 2.01646041286807e-07
344 2.01257257064213e-07
345 2.00869034037154e-07
346 2.00481492662163e-07
347 2.00093997158035e-07
348 1.9970769536215e-07
349 1.99322147540926e-07
350 1.98936138460226e-07
351 1.98550612413584e-07
352 1.98166680497991e-07
353 1.97782477012964e-07
354 1.97399483845118e-07
355 1.97018149554751e-07
356 1.96637649475306e-07
357 1.96260030277173e-07
358 1.95881985478152e-07
359 1.95506477721707e-07
360 1.95131217703093e-07
361 1.94757918205113e-07
362 1.94384770493272e-07
363 1.94012779958896e-07
364 1.936427767113e-07
365 1.93273128761717e-07
366 1.92905349368999e-07
367 1.92538999057135e-07
368 1.92172236930688e-07
369 1.91809071784732e-07
370 1.91446009639051e-07
371 1.91084443859069e-07
372 1.90725299914263e-07
373 1.90366909754047e-07
374 1.90010524498518e-07
375 1.8965514585112e-07
376 1.89297855694548e-07
377 1.88942216256649e-07
378 1.88588497973541e-07
379 1.88235302204021e-07
380 1.87882517559856e-07
381 1.87532610237184e-07
382 1.87182983996337e-07
383 1.86835277505182e-07
384 1.864887640739e-07
385 1.86142913996434e-07
386 1.85800252864965e-07
387 1.85457620514029e-07
388 1.85117948694469e-07
389 1.8478003347866e-07
390 1.84444714202314e-07
391 1.8411586465561e-07
392 1.83801972175601e-07
393 1.83487705470498e-07
394 1.83167671703899e-07
395 1.82824928810277e-07
396 1.82498366928741e-07
397 1.82172291536276e-07
398 1.81846743709357e-07
399 1.81523491850299e-07
400 1.8120162107671e-07
401 1.80881289201906e-07
402 1.80562993127964e-07
403 1.80245953862013e-07
404 1.79930555102459e-07
405 1.79617270230636e-07
406 1.79304753391563e-07
407 1.78994413403188e-07
408 1.78685957409996e-07
409 1.78379170483467e-07
410 1.78073425788128e-07
411 1.77770023100265e-07
412 1.77468305082584e-07
413 1.77167277586321e-07
414 1.76869332392471e-07
415 1.76571998412811e-07
416 1.76276961983746e-07
417 1.75983433885918e-07
418 1.75691385081222e-07
419 1.75401675027942e-07
420 1.75112257052135e-07
421 1.74825446096705e-07
422 1.74538850483685e-07
423 1.74254401089868e-07
424 1.73970691939473e-07
425 1.73688987860743e-07
426 1.73407984837226e-07
427 1.73128382845888e-07
428 1.72849897010607e-07
429 1.72572530308557e-07
430 1.72296651230752e-07
431 1.72021912169029e-07
432 1.71749109842523e-07
433 1.71476613248345e-07
434 1.71205102443395e-07
435 1.70935221623481e-07
436 1.70666735243685e-07
437 1.70399798340881e-07
438 1.70133661317351e-07
439 1.69869055904215e-07
440 1.69607668581762e-07
441 1.69346676113236e-07
442 1.69088305717935e-07
443 1.6883230099296e-07
444 1.68577276486559e-07
445 1.68323087274658e-07
446 1.68071813407167e-07
447 1.6782088210654e-07
448 1.67571009734857e-07
449 1.67323039869771e-07
450 1.67075642259817e-07
451 1.66829878578412e-07
452 1.66584979693241e-07
453 1.6634298031093e-07
454 1.66101174038147e-07
455 1.65861935547795e-07
456 1.65623393732162e-07
457 1.65388081006412e-07
458 1.65153117713146e-07
459 1.64919591258084e-07
460 1.64687136736674e-07
461 1.64456808565916e-07
462 1.64227094346714e-07
463 1.63998439900226e-07
464 1.63771365157217e-07
465 1.63544732481924e-07
466 1.63320085819763e-07
467 1.63095889668341e-07
468 1.62871852690927e-07
469 1.62650417161458e-07
470 1.62428568787121e-07
471 1.62208044088175e-07
472 1.61988294209436e-07
473 1.61770249318849e-07
474 1.61552080205851e-07
475 1.61335368762394e-07
476 1.61119726909575e-07
477 1.60903801628365e-07
478 1.60690368744554e-07
479 1.60476490226102e-07
480 1.60263969002372e-07
481 1.60051876118672e-07
482 1.59842231429508e-07
483 1.59631572293151e-07
484 1.59421854409203e-07
485 1.59212471334769e-07
486 1.59004371081295e-07
487 1.58796134829942e-07
488 1.5858817891079e-07
489 1.58379880868864e-07
490 1.58172457476624e-07
491 1.57963898757885e-07
492 1.5775687779751e-07
493 1.57548046674449e-07
494 1.57340963694352e-07
495 1.57132579916563e-07
496 1.56926030522087e-07
497 1.56718928025157e-07
498 1.56513324064633e-07
499 1.56307900882169e-07
500 1.56102890462861e-07
501 1.55898548646505e-07
502 1.55695554674651e-07
503 1.55491685793763e-07
504 1.55288840698375e-07
505 1.55087109659391e-07
506 1.54885483263811e-07
507 1.5468405521446e-07
508 1.54484644403041e-07
509 1.54285349344363e-07
510 1.54086386464414e-07
511 1.53888268130231e-07
512 1.53691428110392e-07
513 1.53494223008366e-07
514 1.53298567566296e-07
515 1.5310329782281e-07
516 1.52908127276419e-07
517 1.52714105867702e-07
518 1.52520469468342e-07
519 1.5232730756054e-07
520 1.52134380252633e-07
521 1.51942577065967e-07
522 1.51750793195404e-07
523 1.51559949367552e-07
524 1.51369183132743e-07
525 1.51178577528555e-07
526 1.50989277690527e-07
527 1.5080005743151e-07
528 1.50610887670766e-07
529 1.50422820585305e-07
530 1.50234687191997e-07
531 1.50047102385642e-07
532 1.49860123078938e-07
533 1.49673463711864e-07
534 1.49487686048388e-07
535 1.49301606508168e-07
536 1.49116779200931e-07
537 1.48932486120579e-07
538 1.48748688672207e-07
539 1.48565128514022e-07
540 1.48382163411398e-07
541 1.48199507619573e-07
542 1.48018307437603e-07
543 1.47836581717797e-07
544 1.4765543901607e-07
545 1.47474820402671e-07
546 1.47294457118363e-07
547 1.4711467188544e-07
548 1.46934576139124e-07
549 1.46754922833559e-07
550 1.46576109919749e-07
551 1.46396280578998e-07
552 1.46217592702946e-07
553 1.46037755579087e-07
554 1.45856742668649e-07
555 1.45676058838973e-07
556 1.45493761108284e-07
557 1.45310812355248e-07
558 1.45125596625384e-07
559 1.44940814936767e-07
560 1.44759174473208e-07
561 1.44580633556934e-07
562 1.44406760825611e-07
563 1.44236685029497e-07
564 1.44065841452878e-07
565 1.43894216916962e-07
566 1.43721901436855e-07
567 1.4355167284652e-07
568 1.43381390164343e-07
569 1.4321280193208e-07
570 1.4304328589354e-07
571 1.42875119684582e-07
572 1.42707056951075e-07
573 1.42539817169052e-07
574 1.42372630374865e-07
575 1.42206242648157e-07
576 1.42039759817969e-07
577 1.41874659758123e-07
578 1.41708116822237e-07
579 1.41543768648056e-07
580 1.41378939380488e-07
581 1.41214901213438e-07
582 1.41050432207912e-07
583 1.40886619050207e-07
584 1.40723415904098e-07
585 1.40560349514374e-07
586 1.40397347029086e-07
587 1.40234578685394e-07
588 1.40073185360023e-07
589 1.39910756710826e-07
590 1.39749366807607e-07
591 1.39588461963491e-07
592 1.39427499672884e-07
593 1.39267084956707e-07
594 1.39106678180845e-07
595 1.38946575327203e-07
596 1.38786891626275e-07
597 1.38627969658245e-07
598 1.38468772098221e-07
599 1.3831043348933e-07
600 1.38151644971884e-07
601 1.37993857928898e-07
602 1.37835943966991e-07
603 1.3767797245734e-07
604 1.37521476163194e-07
605 1.37365093153541e-07
606 1.37207864278288e-07
607 1.37051939391064e-07
608 1.36896282852739e-07
609 1.36740107667954e-07
610 1.36584907706627e-07
611 1.36430352098316e-07
612 1.36275464691415e-07
613 1.36121003947665e-07
614 1.35967370571422e-07
615 1.3581351165648e-07
616 1.35660106827196e-07
617 1.3550772112314e-07
618 1.35354700097956e-07
619 1.35202530485046e-07
620 1.35050191024888e-07
621 1.34899176728709e-07
622 1.34747624911391e-07
623 1.34597471433295e-07
624 1.34446898085727e-07
625 1.34296857726035e-07
626 1.34147579093913e-07
627 1.33998172611172e-07
628 1.33848777106316e-07
629 1.33699842847612e-07
630 1.33550592190446e-07
631 1.33402053153375e-07
632 1.3325291966737e-07
633 1.33104376471849e-07
634 1.32956018986441e-07
635 1.32807328280471e-07
636 1.32659375417177e-07
637 1.32510779701889e-07
638 1.32363638477351e-07
639 1.32215149132975e-07
640 1.32068159343746e-07
641 1.3192121683403e-07
642 1.3177286332855e-07
643 1.31626606252766e-07
644 1.31479897266473e-07
645 1.31333654683985e-07
646 1.3118673753354e-07
647 1.31040436283314e-07
648 1.30894193350883e-07
649 1.30748249091539e-07
650 1.30603003881902e-07
651 1.30457460132405e-07
652 1.30312093057583e-07
653 1.30167048093099e-07
654 1.30021768718791e-07
655 1.29878059553334e-07
656 1.29733187259617e-07
657 1.29588849539175e-07
658 1.29444899760855e-07
659 1.29300996256632e-07
660 1.29157061030227e-07
661 1.29013910177278e-07
662 1.28870509438883e-07
663 1.28727359161473e-07
664 1.28585040563678e-07
665 1.28442655576322e-07
666 1.28299197362125e-07
667 1.28157414771124e-07
668 1.28015189156727e-07
669 1.27873845015003e-07
670 1.27731557954647e-07
671 1.27590868384431e-07
672 1.27449491724718e-07
673 1.27308344611166e-07
674 1.27167503608305e-07
675 1.27026282761733e-07
676 1.26885639454954e-07
677 1.26745227898795e-07
678 1.26604558001731e-07
679 1.26465136535359e-07
680 1.26324576982029e-07
681 1.26184385212724e-07
682 1.26044677127624e-07
683 1.2590422291936e-07
684 1.2576441315737e-07
685 1.25624402297575e-07
686 1.25484293270972e-07
687 1.25342881565516e-07
688 1.25199912153207e-07
689 1.25055692982379e-07
690 1.24906464574082e-07
691 1.24750857252387e-07
692 1.24609396687347e-07
693 1.24476373523308e-07
694 1.24339010380226e-07
695 1.24202785879213e-07
696 1.24067094633418e-07
697 1.23931600019667e-07
698 1.2379607260371e-07
699 1.23660539137482e-07
700 1.23526029139676e-07
701 1.23390724744432e-07
702 1.232561443123e-07
703 1.23122188367297e-07
704 1.22987951973741e-07
705 1.22853858374405e-07
706 1.22721235219281e-07
707 1.22587376202965e-07
708 1.22454009705564e-07
709 1.2232140922297e-07
710 1.22188948284752e-07
711 1.22056577422036e-07
712 1.21924732252587e-07
713 1.21791463358889e-07
714 1.21661276911489e-07
715 1.2152963020462e-07
716 1.21398197728162e-07
717 1.21266955301458e-07
718 1.21136680438561e-07
719 1.21006465143125e-07
720 1.2087568533925e-07
721 1.20745879335082e-07
722 1.20616397843776e-07
723 1.20486935208497e-07
724 1.20357130209747e-07
725 1.20229339528599e-07
726 1.20099577137545e-07
727 1.19971939303909e-07
728 1.19843738607628e-07
729 1.19715242534291e-07
730 1.19587950353939e-07
731 1.19460293879214e-07
732 1.19332971019404e-07
733 1.19205874566042e-07
734 1.19079337902228e-07
735 1.18952591344978e-07
736 1.18826603253019e-07
737 1.18699966352942e-07
738 1.18574041816366e-07
739 1.18448274813332e-07
740 1.18322696585516e-07
741 1.18197499502593e-07
742 1.18073106210481e-07
743 1.17947499751914e-07
744 1.17822915775889e-07
745 1.17698489787266e-07
746 1.17574076211824e-07
747 1.1745056132284e-07
748 1.17325858123962e-07
749 1.17203655034537e-07
750 1.17078778934854e-07
751 1.1695654723809e-07
752 1.16832843026771e-07
753 1.1671004839453e-07
754 1.16587823381309e-07
755 1.1646509931218e-07
756 1.16343159221266e-07
757 1.16220684392765e-07
758 1.16099171735051e-07
759 1.15977469648421e-07
760 1.15856925877011e-07
761 1.15735877296075e-07
762 1.15614858878565e-07
763 1.15494782035874e-07
764 1.15374444788152e-07
765 1.15254518028074e-07
766 1.15134890487312e-07
767 1.1501565256733e-07
768 1.14895491139855e-07
769 1.14777218291451e-07
770 1.1465820022849e-07
771 1.14538818078103e-07
772 1.14421164461476e-07
773 1.14302639917518e-07
774 1.1418533184937e-07
775 1.14066830035675e-07
776 1.13948576492717e-07
777 1.13831441833412e-07
778 1.13714383237706e-07
779 1.13596503305224e-07
780 1.13481414162564e-07
781 1.13363904693742e-07
782 1.13248537263466e-07
783 1.13132184584863e-07
784 1.13017470954269e-07
785 1.12901222447448e-07
786 1.12785738955345e-07
787 1.12671795272412e-07
788 1.1255763932283e-07
789 1.12443240949389e-07
790 1.12329506764652e-07
791 1.12214316407844e-07
792 1.12102413163839e-07
793 1.11987714524808e-07
794 1.11876301904346e-07
795 1.1176087332565e-07
796 1.11652352607017e-07
797 1.11535196176682e-07
798 1.11428976238948e-07
799 1.11309412831773e-07
800 1.11207045890183e-07
801 1.11085391174548e-07
802 1.10988088110275e-07
803 1.10860372451782e-07
804 1.10769595208282e-07
805 1.10638112078121e-07
806 1.10549823562955e-07
807 1.10418033679949e-07
808 1.10329956816813e-07
809 1.1020023109154e-07
810 1.10110454710544e-07
811 1.09983783046275e-07
812 1.09890725441097e-07
813 1.09768799254617e-07
814 1.09672938174654e-07
815 1.09554056320071e-07
816 1.09455254503565e-07
817 1.09340149509585e-07
818 1.09238882392226e-07
819 1.09127216986415e-07
820 1.09023154180043e-07
821 1.08913611423844e-07
822 1.08808607066635e-07
823 1.08701001370726e-07
824 1.0859589765122e-07
825 1.08488047403377e-07
826 1.08383932463596e-07
827 1.08277453783501e-07
828 1.08172062978085e-07
829 1.08066050190203e-07
830 1.07961049880423e-07
831 1.07855349022579e-07
832 1.07750790188099e-07
833 1.0764522253659e-07
834 1.07541068068429e-07
835 1.07434605039813e-07
836 1.07329437559756e-07
837 1.07223478202023e-07
838 1.07116595921219e-07
839 1.07014337972444e-07
840 1.06904633938143e-07
841 1.0680639815952e-07
842 1.06697547982648e-07
843 1.06603033898978e-07
844 1.06490700886752e-07
845 1.06392920901754e-07
846 1.06293446641814e-07
847 1.06185374085577e-07
848 1.0608221666164e-07
849 1.05977590513895e-07
850 1.0587885162483e-07
851 1.05784740600967e-07
852 1.05682198386248e-07
853 1.05591701456298e-07
854 1.05488800427267e-07
855 1.05370284878781e-07
856 1.05293236104309e-07
857 1.05190053594661e-07
858 1.05083510076653e-07
859 1.04981839307072e-07
860 1.04898549034615e-07
861 1.04790166161273e-07
862 1.04697629046413e-07
863 1.04593150495269e-07
864 1.04483438651215e-07
865 1.04393516902945e-07
866 1.04300461148732e-07
867 1.0419264609407e-07
868 1.04109240615102e-07
869 1.0399474818179e-07
870 1.03915938026944e-07
871 1.03798619614182e-07
872 1.03724449440179e-07
873 1.036082668886e-07
874 1.03537417466271e-07
875 1.0342157583576e-07
876 1.03318770952932e-07
877 1.03233115651413e-07
878 1.03126775987228e-07
879 1.03042666383502e-07
880 1.02933508186354e-07
881 1.02853367760503e-07
882 1.02740709394311e-07
883 1.02663089061039e-07
884 1.02549943478358e-07
885 1.02474228662608e-07
886 1.02358621164811e-07
887 1.02285860888784e-07
888 1.02169531583485e-07
889 1.02097651760502e-07
890 1.01980430257775e-07
891 1.01909528702038e-07
892 1.01792784111154e-07
893 1.01722474658672e-07
894 1.01604432791724e-07
895 1.01536334674357e-07
896 1.01418114586416e-07
897 1.01349384635796e-07
898 1.0123174251131e-07
899 1.01163814827387e-07
900 1.01045964103541e-07
901 1.00978746801594e-07
902 1.00861262807506e-07
903 1.00794960895101e-07
904 1.00676476512795e-07
905 1.00610782425292e-07
906 1.00492978081235e-07
907 1.00426354522121e-07
908 1.00310019501393e-07
909 1.00241600824447e-07
910 1.0012807230364e-07
911 1.00056760715006e-07
912 9.99467257303266e-08
913 9.98760222659101e-08
914 9.97654415950677e-08
915 9.96946342519678e-08
916 9.95848007159239e-08
917 9.95150855978011e-08
918 9.94051787346351e-08
919 9.93339590991127e-08
920 9.92262674994748e-08
921 9.91562607151053e-08
922 9.9047950060438e-08
923 9.89748166189131e-08
924 9.88721084871358e-08
925 9.88024268204768e-08
926 9.86937796350063e-08
927 9.86154155340202e-08
928 9.8529826251692e-08
929 9.84335171301254e-08
930 9.83577517645884e-08
931 9.82553836657019e-08
932 9.81754450517513e-08
933 9.80902439726705e-08
934 9.79998555097339e-08
935 9.79169083734632e-08
936 9.78246322116405e-08
937 9.77469015515098e-08
938 9.76523361062576e-08
939 9.75765255137873e-08
940 9.74894369871748e-08
941 9.74324538933757e-08
942 9.73629018243827e-08
943 9.72788446240713e-08
944 9.71754217369281e-08
945 9.71073610500639e-08
946 9.70071589287969e-08
947 9.69350912809475e-08
948 9.68370209015745e-08
949 9.6763989815507e-08
950 9.66677008005235e-08
951 9.65927550122814e-08
952 9.64983878581904e-08
953 9.64223927368835e-08
954 9.63291862889903e-08
955 9.62524348562255e-08
956 9.61617075780907e-08
957 9.60828179108475e-08
958 9.59947679453066e-08
959 9.59130502202754e-08
960 9.5827086708411e-08
961 9.57450320093756e-08
962 9.56597191255071e-08
963 9.55776545090714e-08
964 9.54925117033767e-08
965 9.54097284004973e-08
966 9.53266565346667e-08
967 9.52431270366816e-08
968 9.5160483839507e-08
969 9.50766411156323e-08
970 9.49946821799941e-08
971 9.4911847674517e-08
972 9.48291233315857e-08
973 9.47474473438348e-08
974 9.46644233801308e-08
975 9.45825344640383e-08
976 9.45003170285119e-08
977 9.44183966948842e-08
978 9.43372159767364e-08
979 9.42547929367876e-08
980 9.41733534594036e-08
981 9.409208658262e-08
982 9.40104454585366e-08
983 9.39293562751686e-08
984 9.38489509110241e-08
985 9.37673095045e-08
986 9.36864239173829e-08
987 9.36054649427476e-08
988 9.35250407296806e-08
989 9.34442968221205e-08
990 9.33642043028371e-08
991 9.32840780212274e-08
992 9.32034687055605e-08
993 9.31234453291907e-08
994 9.30437498967152e-08
995 9.29641213547328e-08
996 9.28839968707962e-08
997 9.28047555150968e-08
998 9.27251743068069e-08
999 9.26458719741063e-08
1000 9.25664143691662e-08
1001 9.24877380183986e-08
1002 9.24085709819167e-08
1003 9.23292170424972e-08
1004 9.22506053449013e-08
1005 9.21714955683939e-08
1006 9.20938646409297e-08
1007 9.20148721927916e-08
1008 9.19371351324472e-08
1009 9.1858465867567e-08
1010 9.1780611507275e-08
1011 9.17016744184096e-08
1012 9.16251159965853e-08
1013 9.15469880728992e-08
1014 9.14697354277294e-08
1015 9.13913139806155e-08
1016 9.13143257799121e-08
1017 9.12373784078824e-08
1018 9.11599556392417e-08
1019 9.10824849684744e-08
1020 9.10061050269917e-08
1021 9.09289370598643e-08
1022 9.08523143365869e-08
1023 9.07762824535752e-08
1024 9.06996457441522e-08
1025 9.06241392666374e-08
1026 9.05502484913612e-08
1027 9.04845776759089e-08
1028 9.04597340092295e-08
1029 9.03578933293403e-08
1030 9.02859841458437e-08
1031 9.02534597511107e-08
1032 9.01144258742548e-08
1033 9.01129648278598e-08
1034 9.00034097321978e-08
1035 8.99212435196972e-08
1036 8.98547684160533e-08
1037 8.97563912714006e-08
1038 8.97319793642382e-08
1039 8.96145767343981e-08
1040 8.95708235146841e-08
1041 8.94513445599543e-08
1042 8.94397956674453e-08
1043 8.93368770586989e-08
1044 8.92616337653251e-08
1045 8.91744937847605e-08
1046 8.91189048184415e-08
1047 8.89980740330643e-08
1048 8.90036262184424e-08
1049 8.88933575859952e-08
1050 8.88197643611477e-08
1051 8.87263363562596e-08
1052 8.86867209732856e-08
1053 8.85559144263581e-08
1054 8.85614867689455e-08
1055 8.84528476756685e-08
1056 8.83707941783385e-08
1057 8.82826710819629e-08
1058 8.82543282179071e-08
1059 8.81207235554626e-08
1060 8.81194117177131e-08
1061 8.80089120682825e-08
1062 8.79222666663182e-08
1063 8.78803294437347e-08
1064 8.77685420048735e-08
1065 8.77393020237704e-08
1066 8.76253702175589e-08
1067 8.75868956207171e-08
1068 8.74831936865661e-08
1069 8.74423340464503e-08
1070 8.73413074593543e-08
1071 8.729045783884e-08
1072 8.71996470861447e-08
1073 8.71584861394936e-08
1074 8.70574415214875e-08
1075 8.69889208185981e-08
1076 8.69428776528736e-08
1077 8.68451494566713e-08
1078 8.67757386098234e-08
1079 8.67229136600756e-08
1080 8.66352509181212e-08
1081 8.65659138487018e-08
1082 8.65024334606446e-08
1083 8.64263327793324e-08
1084 8.63578315026814e-08
1085 8.62874877869402e-08
1086 8.62177321270252e-08
1087 8.61473722091333e-08
1088 8.60777596232154e-08
1089 8.60072756534436e-08
1090 8.59385540978863e-08
1091 8.58688497018534e-08
1092 8.58000618055854e-08
1093 8.57304077808152e-08
1094 8.56614752127172e-08
1095 8.55925388938417e-08
1096 8.5524014052929e-08
1097 8.5455036005655e-08
1098 8.53852761366625e-08
1099 8.53176815684975e-08
1100 8.52484491122851e-08
1101 8.51805265291716e-08
1102 8.51116365421234e-08
1103 8.50435918549053e-08
1104 8.49749687370505e-08
1105 8.49075406188504e-08
1106 8.48384480818254e-08
1107 8.47722898136283e-08
1108 8.47034591062723e-08
1109 8.4635441254477e-08
1110 8.45693521398516e-08
1111 8.45013398329542e-08
1112 8.44335206640068e-08
1113 8.43668009649079e-08
1114 8.42998500232284e-08
1115 8.42326281800254e-08
1116 8.41657276282604e-08
1117 8.40991533204161e-08
1118 8.4032679161794e-08
1119 8.39654433093528e-08
1120 8.38994693008743e-08
1121 8.38330783770047e-08
1122 8.37673009783657e-08
1123 8.37010798200311e-08
1124 8.36351164323901e-08
1125 8.35684129461001e-08
1126 8.35037545154904e-08
1127 8.34388591028912e-08
1128 8.33730164870872e-08
1129 8.33092982279737e-08
1130 8.32439687572517e-08
1131 8.31813311386043e-08
1132 8.31216096370824e-08
1133 8.30635869961682e-08
1134 8.29942485891522e-08
1135 8.29317003807617e-08
1136 8.28651203708119e-08
1137 8.2802502562096e-08
1138 8.27361158668438e-08
1139 8.26728471405858e-08
1140 8.26077631437272e-08
1141 8.25443163519068e-08
1142 8.24793579115024e-08
1143 8.24163989090465e-08
1144 8.2351572550543e-08
1145 8.22882015292237e-08
1146 8.22235499500223e-08
1147 8.21604170599954e-08
1148 8.20960932701453e-08
1149 8.20338488871641e-08
1150 8.19696036593598e-08
1151 8.19063662733655e-08
1152 8.18428475177768e-08
1153 8.17799636516625e-08
1154 8.17169866849099e-08
1155 8.16546616926317e-08
1156 8.15903164834708e-08
1157 8.15291326965451e-08
1158 8.14651394120958e-08
1159 8.14041550816214e-08
1160 8.13397367682711e-08
1161 8.12796981248809e-08
1162 8.12144708790541e-08
1163 8.11550529533633e-08
1164 8.10898772547475e-08
1165 8.10319019439731e-08
1166 8.09648683857134e-08
1167 8.09088143975956e-08
1168 8.08411616119997e-08
1169 8.07864508001188e-08
1170 8.07164465808086e-08
1171 8.06654223159953e-08
1172 8.05931365608359e-08
1173 8.05434823343276e-08
1174 8.0470003523736e-08
1175 8.04209031812775e-08
1176 8.0348191256796e-08
1177 8.02990021790961e-08
1178 8.02262840613466e-08
1179 8.01771075851221e-08
1180 8.01054049901495e-08
1181 8.00561042755277e-08
1182 7.99843649446075e-08
1183 7.99356215557268e-08
1184 7.98641790051846e-08
1185 7.98141245139661e-08
1186 7.97441698408008e-08
1187 7.96938042446627e-08
1188 7.96255736110751e-08
1189 7.95735959036037e-08
1190 7.95065679470852e-08
1191 7.94534323382834e-08
1192 7.93874667239791e-08
1193 7.93348512519998e-08
1194 7.92688390243157e-08
1195 7.92160816924792e-08
1196 7.91509083910569e-08
1197 7.90966263100046e-08
1198 7.90332429989604e-08
1199 7.89781853063332e-08
1200 7.89156177685157e-08
1201 7.88602398333893e-08
1202 7.87980182614945e-08
1203 7.87428144688462e-08
1204 7.86806867099088e-08
1205 7.86246273341007e-08
1206 7.85636061140238e-08
1207 7.850852543001e-08
1208 7.84466592840261e-08
1209 7.83910999393456e-08
1210 7.83296182360615e-08
1211 7.82753790433688e-08
1212 7.82131802505859e-08
1213 7.81589895071377e-08
1214 7.80957292265683e-08
1215 7.80441435184187e-08
1216 7.79787111078178e-08
1217 7.79280615210709e-08
1218 7.78617829908512e-08
1219 7.78145261906005e-08
1220 7.77460494951754e-08
1221 7.76984388952684e-08
1222 7.76297216420474e-08
1223 7.75837404329849e-08
1224 7.75151594272216e-08
1225 7.74676360517645e-08
1226 7.74011720574563e-08
1227 7.73533746896149e-08
1228 7.72873186001277e-08
1229 7.72378892284209e-08
1230 7.71735875630952e-08
1231 7.71242701933517e-08
1232 7.70617304208798e-08
1233 7.70080703942355e-08
1234 7.6949215534583e-08
1235 7.68945395517306e-08
1236 7.68360364045151e-08
1237 7.67811714954547e-08
1238 7.672433977568e-08
1239 7.66684182327282e-08
1240 7.6612008408361e-08
1241 7.65561864728426e-08
1242 7.65007564353937e-08
1243 7.64445714374062e-08
1244 7.63892977575509e-08
1245 7.63337494342764e-08
1246 7.62785937871158e-08
1247 7.62236783131698e-08
1248 7.61678620646578e-08
1249 7.61129021187301e-08
1250 7.60583642707147e-08
1251 7.60041803484768e-08
1252 7.59492239206239e-08
1253 7.58951337971325e-08
1254 7.58408003971311e-08
1255 7.57861930642534e-08
1256 7.57328603970819e-08
1257 7.56776331858333e-08
1258 7.56255004192852e-08
1259 7.55677695618573e-08
1260 7.55210612783941e-08
1261 7.5455491955978e-08
1262 7.54228841000071e-08
1263 7.53280163072745e-08
1264 7.53523716472415e-08
1265 7.52081915313951e-08
1266 7.52358802147413e-08
1267 7.50948560162001e-08
1268 7.51530399139e-08
1269 7.49801304928965e-08
1270 7.50332321954872e-08
1271 7.48825844487655e-08
1272 7.49354434930183e-08
1273 7.47623009340614e-08
1274 7.48307505409329e-08
1275 7.46532599915639e-08
1276 7.47269774254988e-08
1277 7.45409971179001e-08
1278 7.4622333429808e-08
1279 7.44333799671537e-08
1280 7.45173239833363e-08
1281 7.43279868364866e-08
1282 7.44090140996079e-08
1283 7.42261735675598e-08
1284 7.4300600386934e-08
1285 7.41261744154897e-08
1286 7.41922396292338e-08
1287 7.40282605793396e-08
1288 7.40803511787291e-08
1289 7.39368186399858e-08
1290 7.39667582019266e-08
1291 7.38315857571337e-08
1292 7.3855449588045e-08
1293 7.37347553423362e-08
1294 7.37538957906736e-08
1295 7.36215751642177e-08
1296 7.36398801368665e-08
1297 7.36225339750263e-08
1298 7.34372526567384e-08
1299 7.34737777472461e-08
1300 7.34800244037714e-08
1301 7.32718593940973e-08
1302 7.33209410039848e-08
1303 7.33246518462849e-08
1304 7.31204503425786e-08
1305 7.31618519926869e-08
1306 7.31784478054109e-08
1307 7.29641391004065e-08
1308 7.30124324466175e-08
1309 7.30247145153129e-08
1310 7.28139210073309e-08
1311 7.28618485217325e-08
1312 7.28730810335421e-08
1313 7.26632894076928e-08
1314 7.27144560928039e-08
1315 7.27111706702033e-08
1316 7.25178608158572e-08
1317 7.25502474230311e-08
1318 7.25715938658666e-08
1319 7.23639550237465e-08
1320 7.24011953421311e-08
1321 7.24170248211919e-08
1322 7.22178113212735e-08
1323 7.22509875661359e-08
1324 7.22740361949192e-08
1325 7.20658986717027e-08
1326 7.21233574596525e-08
1327 7.20371010061172e-08
1328 7.20731466259394e-08
1329 7.18726206576648e-08
1330 7.19724243216646e-08
1331 7.17850481883886e-08
1332 7.18386386679981e-08
1333 7.17254430488978e-08
1334 7.17093735635999e-08
1335 7.1704530096639e-08
1336 7.15505172914632e-08
1337 7.16615894456041e-08
1338 7.14126749583599e-08
1339 7.15567096678527e-08
1340 7.13268816339152e-08
1341 7.14495686873207e-08
1342 7.12398165418904e-08
1343 7.1338264154619e-08
1344 7.11540355826656e-08
1345 7.12259265842619e-08
1346 7.10680478883674e-08
1347 7.11206207917314e-08
1348 7.09787049038013e-08
1349 7.10166975892435e-08
1350 7.08872486274714e-08
1351 7.09181640248246e-08
1352 7.07944810267591e-08
1353 7.08146748920058e-08
1354 7.0702551249191e-08
1355 7.07318490205111e-08
1356 7.06043237785892e-08
1357 7.0599649050429e-08
1358 7.05548622379482e-08
1359 7.04895554992646e-08
1360 7.04421849899717e-08
1361 7.03852614112321e-08
1362 7.04659668011587e-08
1363 7.02192386787814e-08
1364 7.02953338729273e-08
1365 7.01765344999217e-08
1366 7.02074535050556e-08
1367 7.00955670449588e-08
1368 7.00698842566183e-08
1369 7.00576100767236e-08
1370 6.99834112314335e-08
1371 6.99257318119706e-08
1372 6.99766848875072e-08
1373 6.97767572788166e-08
1374 6.98637101157473e-08
1375 6.97055243445988e-08
1376 6.97321534754991e-08
1377 6.96499001930917e-08
1378 6.97050232592034e-08
1379 6.95197717623941e-08
1380 6.96231528802826e-08
1381 6.94327010961615e-08
1382 6.95279203828392e-08
1383 6.9347483287352e-08
1384 6.94357787454081e-08
1385 6.92606526806472e-08
1386 6.93468427028776e-08
1387 6.9172500962722e-08
1388 6.92586657669025e-08
1389 6.90824182223437e-08
1390 6.91721592600558e-08
1391 6.89933672974874e-08
1392 6.90854467784163e-08
1393 6.89044460822075e-08
1394 6.89978678503067e-08
1395 6.88166489313247e-08
1396 6.89117653678295e-08
1397 6.87285123053272e-08
1398 6.88238489603421e-08
1399 6.86421015378968e-08
1400 6.87369566829688e-08
1401 6.8555764372924e-08
1402 6.86485543948123e-08
1403 6.84707140896634e-08
1404 6.85611143467213e-08
1405 6.83850294027621e-08
1406 6.84752754320783e-08
1407 6.82996976939521e-08
1408 6.83884314502947e-08
1409 6.82147897723695e-08
1410 6.83025287271377e-08
1411 6.81300831839593e-08
1412 6.8218321352731e-08
1413 6.80455078416742e-08
1414 6.81320976667621e-08
1415 6.79605311715292e-08
1416 6.80539095263555e-08
1417 6.78768733255453e-08
1418 6.79609684146598e-08
1419 6.7792407812739e-08
1420 6.78984598359023e-08
1421 6.7708616446005e-08
1422 6.77754119546847e-08
1423 6.76567749042789e-08
1424 6.77335495975484e-08
1425 6.75458821719488e-08
1426 6.75883575649294e-08
1427 6.74902189965465e-08
1428 6.74960374142941e-08
1429 6.74279072603312e-08
1430 6.73895101561683e-08
1431 6.73579511989431e-08
1432 6.73081845414814e-08
1433 6.72766757272569e-08
1434 6.72304502575827e-08
1435 6.71961767313434e-08
1436 6.71530776017448e-08
1437 6.71162084460164e-08
1438 6.70759971557189e-08
1439 6.70382803615865e-08
1440 6.6997662630186e-08
1441 6.696010528362e-08
1442 6.69207359118573e-08
1443 6.68832151253795e-08
1444 6.68439764561768e-08
1445 6.68069567533181e-08
1446 6.67670682297228e-08
1447 6.67288348319417e-08
1448 6.66912255278262e-08
1449 6.66537583295934e-08
1450 6.66147574266418e-08
1451 6.65772448193991e-08
1452 6.65377026898284e-08
1453 6.65002659374636e-08
1454 6.64631858828812e-08
1455 6.64247961754683e-08
1456 6.63874423381117e-08
1457 6.63494361727857e-08
1458 6.63118943258212e-08
1459 6.62756386304864e-08
1460 6.62369524544459e-08
1461 6.62008998668639e-08
1462 6.61629059894864e-08
1463 6.61261443379857e-08
1464 6.60886736278954e-08
1465 6.60528654634618e-08
1466 6.60146763564739e-08
1467 6.59791804409338e-08
1468 6.59419342561307e-08
1469 6.59063216641442e-08
1470 6.5868114340617e-08
1471 6.58340463433049e-08
1472 6.57964270445177e-08
1473 6.57605988063636e-08
1474 6.57248836226643e-08
1475 6.56888709267278e-08
1476 6.56529720197696e-08
1477 6.56181537070921e-08
1478 6.55813331986721e-08
1479 6.55482322349243e-08
1480 6.55124400497087e-08
1481 6.54776371424859e-08
1482 6.54428391664297e-08
1483 6.54077968400557e-08
1484 6.537483979141e-08
1485 6.53383960056431e-08
1486 6.53067952001152e-08
1487 6.52705282693233e-08
1488 6.52366494415801e-08
1489 6.5202091491301e-08
1490 6.51681293479811e-08
1491 6.51333276033839e-08
1492 6.50992397179806e-08
1493 6.50649764146749e-08
1494 6.5032734797299e-08
1495 6.49969062438416e-08
1496 6.49650426165493e-08
1497 6.49304395263783e-08
1498 6.4896606446041e-08
1499 6.48633971040979e-08
1500 6.48306480703908e-08
1501 6.47973204728203e-08
1502 6.47652361651652e-08
1503 6.47309455059641e-08
1504 6.47019088049206e-08
1505 6.46711419447499e-08
1506 6.46505755224069e-08
1507 6.46041815093312e-08
1508 6.45728626862407e-08
1509 6.45431333410329e-08
1510 6.44921404342469e-08
1511 6.44867648125214e-08
1512 6.44198978960731e-08
1513 6.44157079543461e-08
1514 6.43754620393011e-08
1515 6.43452805411115e-08
1516 6.43154332475859e-08
1517 6.42615645327638e-08
1518 6.42503821817186e-08
1519 6.42018589438464e-08
1520 6.41944429764507e-08
1521 6.4131264493561e-08
1522 6.41308655264794e-08
1523 6.40639534683629e-08
1524 6.40717424902348e-08
1525 6.39957517760692e-08
1526 6.40127106077415e-08
1527 6.39280037217915e-08
1528 6.3955404209004e-08
1529 6.38594792192748e-08
1530 6.38980141545176e-08
1531 6.3792610010438e-08
1532 6.38406042252626e-08
1533 6.37255068935971e-08
1534 6.37850589084721e-08
1535 6.36587020661494e-08
1536 6.37283075501926e-08
1537 6.35923754739309e-08
1538 6.36720069406849e-08
1539 6.35274079270331e-08
1540 6.36152340707241e-08
1541 6.3462438043338e-08
1542 6.35587107717939e-08
1543 6.3398386163982e-08
1544 6.3501748437389e-08
1545 6.33358558372876e-08
1546 6.34433885480234e-08
1547 6.32730789291713e-08
1548 6.33852629894349e-08
1549 6.32119350223448e-08
1550 6.33271415111381e-08
1551 6.31505498169815e-08
1552 6.32685816190914e-08
1553 6.30902852636339e-08
1554 6.32091498591336e-08
1555 6.30301150597035e-08
1556 6.31503686214785e-08
1557 6.2971738673312e-08
1558 6.30898753835041e-08
1559 6.29131850393705e-08
1560 6.3029729454378e-08
1561 6.28555440753686e-08
1562 6.29694686571369e-08
1563 6.27990778703236e-08
1564 6.29065149597707e-08
1565 6.27447214451848e-08
1566 6.28420033770283e-08
1567 6.26918464998738e-08
1568 6.27799028842801e-08
1569 6.26371824683503e-08
1570 6.27212875761174e-08
1571 6.25809394358967e-08
1572 6.26630062168232e-08
1573 6.25246500307597e-08
1574 6.26050052154881e-08
1575 6.2469886572103e-08
1576 6.25469072570439e-08
1577 6.2413147606577e-08
1578 6.24908495128196e-08
1579 6.23582953860336e-08
1580 6.24331189813532e-08
1581 6.23030064108576e-08
1582 6.23769484215941e-08
1583 6.22470348545789e-08
1584 6.23210739547631e-08
1585 6.21920281256294e-08
1586 6.22655492437119e-08
1587 6.21360176698005e-08
1588 6.22109870143106e-08
1589 6.20811379326014e-08
1590 6.21554448558825e-08
1591 6.20259650681021e-08
1592 6.21007030821019e-08
1593 6.19714301075547e-08
1594 6.20469061889395e-08
1595 6.19164331174815e-08
1596 6.19929166667532e-08
1597 6.18624312274107e-08
1598 6.19394588650124e-08
1599 6.18086175983024e-08
1600 6.1884963166392e-08
1601 6.17553180841668e-08
1602 6.18320693757113e-08
1603 6.17017701438627e-08
1604 6.17791475576723e-08
1605 6.16485005666689e-08
1606 6.17260856285995e-08
1607 6.15964604957497e-08
1608 6.1673425169495e-08
1609 6.15442870302019e-08
1610 6.16202349448258e-08
1611 6.14925792250531e-08
1612 6.15680774531668e-08
1613 6.14411003541093e-08
1614 6.15151567542327e-08
1615 6.13906273887466e-08
1616 6.14621534813509e-08
1617 6.13403638602961e-08
1618 6.14100292510855e-08
1619 6.12898001808304e-08
1620 6.13575466976712e-08
1621 6.12393262429123e-08
1622 6.13070291421991e-08
1623 6.1189900965708e-08
1624 6.12540385658278e-08
1625 6.11404510602043e-08
1626 6.12019439651945e-08
1627 6.10909270450932e-08
1628 6.11502214606929e-08
1629 6.10424014952926e-08
1630 6.10981925497356e-08
1631 6.09945026965875e-08
1632 6.10467749169175e-08
1633 6.09458855898026e-08
1634 6.09944319007738e-08
1635 6.08987276358164e-08
1636 6.09431268214777e-08
1637 6.08515535480691e-08
1638 6.08915684532363e-08
1639 6.08052450132845e-08
1640 6.08388831730977e-08
1641 6.07591704371302e-08
1642 6.07873437044049e-08
1643 6.07126588079154e-08
1644 6.07353147072942e-08
1645 6.06674851777456e-08
1646 6.06838016334876e-08
1647 6.06220882692909e-08
1648 6.06317294087333e-08
1649 6.05771087895946e-08
1650 6.05799531001594e-08
1651 6.0531417844345e-08
1652 6.05295662916561e-08
1653 6.04859146822179e-08
1654 6.0477020856986e-08
1655 6.0440976328735e-08
1656 6.04266873969195e-08
1657 6.03952213795722e-08
1658 6.03767761928609e-08
1659 6.03484836645407e-08
1660 6.03282398250471e-08
1661 6.03013106577066e-08
1662 6.02794121924433e-08
1663 6.02543891217167e-08
1664 6.02309656478894e-08
1665 6.02064497421395e-08
1666 6.0183124652724e-08
1667 6.01588455486635e-08
1668 6.01356863505487e-08
1669 6.01122579482194e-08
1670 6.00876062248901e-08
1671 6.00647682285071e-08
1672 6.00405452733099e-08
1673 6.00170277902379e-08
1674 5.9993355147725e-08
1675 5.9970142240573e-08
1676 5.99462952575181e-08
1677 5.99226709185885e-08
1678 5.98994840084188e-08
1679 5.98757753937917e-08
1680 5.9852795669002e-08
1681 5.98286815582938e-08
1682 5.98061069707612e-08
1683 5.97823754864279e-08
1684 5.97585325774474e-08
1685 5.97354517344328e-08
1686 5.97122143553008e-08
1687 5.96883995447328e-08
1688 5.96662149900951e-08
1689 5.96421201786157e-08
1690 5.96190365964588e-08
1691 5.95967197796199e-08
1692 5.95727504002497e-08
1693 5.95493983004403e-08
1694 5.95266001059969e-08
1695 5.95038887007959e-08
1696 5.94798065769453e-08
1697 5.94579225801084e-08
1698 5.94352658831454e-08
1699 5.94108297979901e-08
1700 5.9388099769464e-08
1701 5.93657632705913e-08
1702 5.93426628725524e-08
1703 5.93202588303754e-08
1704 5.92967226005214e-08
1705 5.92744870662187e-08
1706 5.92513805299788e-08
1707 5.92285388192337e-08
1708 5.9206233532727e-08
1709 5.91831760541339e-08
1710 5.91610016380528e-08
1711 5.9138223751809e-08
1712 5.91159120277851e-08
1713 5.90942421059282e-08
1714 5.90707423713255e-08
1715 5.90492907246443e-08
1716 5.90273807938857e-08
1717 5.90048680937372e-08
1718 5.89836922388898e-08
1719 5.89631255722978e-08
1720 5.89405285893463e-08
1721 5.89189200450591e-08
1722 5.88979985582938e-08
1723 5.8876236702865e-08
1724 5.88542974169215e-08
1725 5.88339113463121e-08
1726 5.8812706366318e-08
1727 5.87922723527257e-08
1728 5.87708972945578e-08
1729 5.87505158931023e-08
1730 5.87305361214874e-08
1731 5.87094625164042e-08
1732 5.8689482046681e-08
1733 5.86691909250092e-08
1734 5.86499384978012e-08
1735 5.86298100895277e-08
1736 5.86107854791962e-08
1737 5.85909612516033e-08
1738 5.8571320043832e-08
1739 5.85528331162877e-08
1740 5.85326352746662e-08
1741 5.85148031646554e-08
1742 5.8495415840909e-08
1743 5.84764741544674e-08
1744 5.84573889002016e-08
1745 5.84377639514244e-08
1746 5.8419483123906e-08
1747 5.84001204142481e-08
1748 5.83817638917239e-08
1749 5.83631220125724e-08
1750 5.83446247945929e-08
1751 5.83264054325738e-08
1752 5.83081179952316e-08
1753 5.82892835758742e-08
1754 5.82714264760753e-08
1755 5.82543330729379e-08
1756 5.82357159215618e-08
1757 5.82180610164684e-08
1758 5.81998465696287e-08
1759 5.81811152668976e-08
1760 5.81629359421854e-08
1761 5.8145001605503e-08
1762 5.81263760812689e-08
1763 5.81074226992939e-08
1764 5.80887066865543e-08
1765 5.80693116694064e-08
1766 5.80510249923449e-08
1767 5.80313498819152e-08
1768 5.80124631959222e-08
1769 5.79931207278506e-08
1770 5.79742726394272e-08
1771 5.79543275245342e-08
1772 5.79355409531246e-08
1773 5.79163670106908e-08
1774 5.7897227854653e-08
1775 5.78765630994127e-08
1776 5.78584502006763e-08
1777 5.78384773790574e-08
1778 5.78191393811878e-08
1779 5.77999186370803e-08
1780 5.77800156325026e-08
1781 5.77613507237729e-08
1782 5.7741145567114e-08
1783 5.77214916246405e-08
1784 5.77025890766691e-08
1785 5.76835768022477e-08
1786 5.76640775395987e-08
1787 5.76437491028869e-08
1788 5.76251647244064e-08
1789 5.76054239473933e-08
1790 5.75859296043646e-08
1791 5.75671797440336e-08
1792 5.75468505878973e-08
1793 5.75277964225407e-08
1794 5.75081977514102e-08
1795 5.74887033586435e-08
1796 5.74702335587318e-08
1797 5.74501724308618e-08
1798 5.74308302283555e-08
1799 5.74122601841864e-08
1800 5.73926742832853e-08
1801 5.73729897874742e-08
1802 5.73539907131604e-08
1803 5.73345957848304e-08
1804 5.73154474583504e-08
1805 5.72966828764621e-08
1806 5.72769730222689e-08
1807 5.72580499884623e-08
1808 5.72382438761565e-08
1809 5.72198936632518e-08
1810 5.72001766023789e-08
1811 5.71814000673854e-08
1812 5.71623104441699e-08
1813 5.71431385836263e-08
1814 5.71242922262627e-08
1815 5.71048046786871e-08
1816 5.70852544976574e-08
1817 5.70673137616495e-08
1818 5.70469871536972e-08
1819 5.70289116259559e-08
1820 5.70089613418645e-08
1821 5.69905805871684e-08
1822 5.69728311994311e-08
1823 5.69512487214041e-08
1824 5.69343273797429e-08
1825 5.69151540155133e-08
1826 5.68953819026774e-08
1827 5.6877065681249e-08
1828 5.68577601960385e-08
1829 5.68392555164721e-08
1830 5.68205959599055e-08
1831 5.68013444954829e-08
1832 5.67826246369307e-08
1833 5.6763574812102e-08
1834 5.67444881651724e-08
1835 5.67261865871416e-08
1836 5.67074217707741e-08
1837 5.66885960520125e-08
1838 5.66693980914579e-08
1839 5.66512771031924e-08
1840 5.66328886870693e-08
1841 5.66132394057561e-08
1842 5.65950292381601e-08
1843 5.65760827910822e-08
1844 5.65573231297023e-08
1845 5.65385487742986e-08
1846 5.65207770160114e-08
1847 5.65012491051675e-08
1848 5.6482528084878e-08
1849 5.6464579886395e-08
1850 5.64461673482342e-08
1851 5.64267131721863e-08
1852 5.64097615698422e-08
1853 5.63900969203779e-08
1854 5.6372010118988e-08
1855 5.63528444827988e-08
1856 5.63356055267583e-08
1857 5.6316821495983e-08
1858 5.62977132037012e-08
1859 5.62796416025968e-08
1860 5.62615527681665e-08
1861 5.62430268566949e-08
1862 5.62246701538704e-08
1863 5.62059055897457e-08
1864 5.61879974347868e-08
1865 5.61703932500102e-08
1866 5.6151459226772e-08
1867 5.61327616739327e-08
1868 5.61155488507659e-08
1869 5.60972822873929e-08
1870 5.60788891279529e-08
1871 5.60609836699477e-08
1872 5.60436175560675e-08
1873 5.60251666863465e-08
1874 5.60072969113534e-08
1875 5.59889865949792e-08
1876 5.5971328619453e-08
1877 5.59526011820743e-08
1878 5.59345373241094e-08
1879 5.59160885895693e-08
1880 5.5898036368518e-08
1881 5.58809291302254e-08
1882 5.58619771440227e-08
1883 5.58439185249782e-08
1884 5.58255922165074e-08
1885 5.58079148178514e-08
1886 5.57891893020468e-08
1887 5.57716291451626e-08
1888 5.57537026151245e-08
1889 5.57349764216397e-08
1890 5.57171787813893e-08
1891 5.56983100019615e-08
1892 5.56797200217396e-08
1893 5.56606425576867e-08
1894 5.56430652358664e-08
1895 5.56231454842404e-08
1896 5.56053681863844e-08
1897 5.55859316166973e-08
1898 5.55671858348816e-08
1899 5.55491317628665e-08
1900 5.55313481513942e-08
1901 5.55143963518745e-08
1902 5.54963827119614e-08
1903 5.54791836564483e-08
1904 5.54618730004286e-08
1905 5.54442000613165e-08
1906 5.54268987387196e-08
1907 5.54102715555338e-08
1908 5.5392104524099e-08
1909 5.53753632170917e-08
1910 5.53582435500743e-08
1911 5.53415962643022e-08
1912 5.53235532021468e-08
1913 5.53071848847786e-08
1914 5.52884528537412e-08
1915 5.52716375512574e-08
1916 5.52548810150988e-08
1917 5.52382607423851e-08
1918 5.5220644906484e-08
1919 5.52038524639364e-08
1920 5.51860427275841e-08
1921 5.51701262949678e-08
1922 5.51513970954431e-08
1923 5.51357022784416e-08
1924 5.51175030674145e-08
1925 5.5101480945563e-08
1926 5.50837002375459e-08
1927 5.50672033705624e-08
1928 5.50498719351644e-08
1929 5.50337565079495e-08
1930 5.50152020268868e-08
1931 5.50003928703191e-08
1932 5.49802874543204e-08
1933 5.49661936624091e-08
1934 5.49464670016597e-08
1935 5.4932216977388e-08
1936 5.49125005964157e-08
1937 5.48992011224492e-08
1938 5.48782435245698e-08
1939 5.48665612538635e-08
1940 5.48430017248691e-08
1941 5.48334758914137e-08
1942 5.48088538234737e-08
1943 5.48008176073367e-08
1944 5.47737554863481e-08
1945 5.47680830726982e-08
1946 5.47391349234871e-08
1947 5.47346843648988e-08
1948 5.4705034578717e-08
1949 5.4702078505553e-08
1950 5.46711488764728e-08
1951 5.46695421062005e-08
1952 5.46370957175135e-08
1953 5.46352627970315e-08
1954 5.4603947877041e-08
1955 5.46025953447327e-08
1956 5.45700486940248e-08
1957 5.45703622427673e-08
1958 5.45361909276565e-08
1959 5.45362799955207e-08
1960 5.4504054030069e-08
1961 5.45035702579355e-08
1962 5.44711859133962e-08
1963 5.44690828898631e-08
1964 5.44383018556971e-08
1965 5.44358620269136e-08
1966 5.44059637075378e-08
1967 5.44024353499317e-08
1968 5.4373112740258e-08
1969 5.43685958196427e-08
1970 5.43419316705318e-08
1971 5.43346549548573e-08
1972 5.43100012606601e-08
1973 5.43021985519587e-08
1974 5.42779020249462e-08
1975 5.42686266706838e-08
1976 5.42453582386138e-08
1977 5.42355432915365e-08
1978 5.42133202463546e-08
1979 5.42022247045359e-08
1980 5.41808251730558e-08
1981 5.41691203217454e-08
1982 5.41493185766129e-08
1983 5.41355472005733e-08
1984 5.41172847614568e-08
1985 5.41032373635453e-08
1986 5.40844813334829e-08
1987 5.40694860160507e-08
1988 5.40529907393505e-08
1989 5.40373050359477e-08
1990 5.40196885863153e-08
1991 5.4004237181271e-08
1992 5.39879775063312e-08
1993 5.39711812548305e-08
1994 5.39554071199433e-08
1995 5.39387527704882e-08
1996 5.39220325008749e-08
1997 5.39056316655184e-08
1998 5.38902859474888e-08
1999 5.38735973427684e-08
2000 5.38574509367784e-08
2001 5.3841490279094e-08
2002 5.38249112187472e-08
2003 5.38086973658203e-08
2004 5.37923478534097e-08
2005 5.37761217787036e-08
2006 5.37590721654269e-08
2007 5.37433396106124e-08
2008 5.37270008620361e-08
2009 5.37110045280009e-08
2010 5.36947665445986e-08
2011 5.36785844826859e-08
2012 5.36628336225142e-08
2013 5.36463803784137e-08
2014 5.36299037210419e-08
2015 5.36136628270789e-08
2016 5.35974900071068e-08
2017 5.35820119256236e-08
2018 5.35650427719858e-08
2019 5.35495899800509e-08
2020 5.35332151390122e-08
2021 5.35167506763301e-08
2022 5.35010055249252e-08
2023 5.34849738187404e-08
2024 5.3468723325345e-08
2025 5.34522699635609e-08
2026 5.34359765684833e-08
2027 5.34207851345947e-08
2028 5.34042487756459e-08
2029 5.33881986393148e-08
2030 5.3373055232786e-08
2031 5.33558566986336e-08
2032 5.33395082080723e-08
2033 5.33236671746984e-08
2034 5.33084114859861e-08
2035 5.32918120614845e-08
2036 5.32760798295229e-08
2037 5.32604290732763e-08
2038 5.32433237676599e-08
2039 5.32280773692939e-08
2040 5.32123427547937e-08
2041 5.31964299463894e-08
2042 5.3180118048779e-08
2043 5.31648555379682e-08
2044 5.31481859309402e-08
2045 5.31323932886352e-08
2046 5.31163554704506e-08
2047 5.31012394002772e-08
2048 5.30838660415256e-08
2049 5.30688552173864e-08
2050 5.30524296165069e-08
2051 5.30370516402812e-08
2052 5.30216684024865e-08
2053 5.30053499248062e-08
2054 5.29890869840344e-08
2055 5.29735669569931e-08
2056 5.29574765577578e-08
2057 5.29420564516769e-08
2058 5.29265590647476e-08
2059 5.29105634514693e-08
2060 5.28942903388341e-08
2061 5.28794885967798e-08
2062 5.28627092042377e-08
2063 5.28472988818862e-08
2064 5.28311889751443e-08
2065 5.2815394230965e-08
2066 5.27999636830145e-08
2067 5.2784659512195e-08
2068 5.27685152009738e-08
2069 5.27525059101919e-08
2070 5.27363165550021e-08
2071 5.27213012349037e-08
2072 5.27056552028782e-08
2073 5.26899487982568e-08
2074 5.26746382396581e-08
2075 5.26577436730236e-08
2076 5.26428766716158e-08
2077 5.2627005566741e-08
2078 5.2611288784199e-08
2079 5.25958590666953e-08
2080 5.25794817640701e-08
2081 5.25642070177135e-08
2082 5.25480767428199e-08
2083 5.25333846472797e-08
2084 5.25179798014364e-08
2085 5.25009789096309e-08
2086 5.24871352420853e-08
2087 5.2470230914814e-08
2088 5.24547525331265e-08
2089 5.24394768626202e-08
2090 5.24235538970075e-08
2091 5.24077324479677e-08
2092 5.23930509150894e-08
2093 5.23767071269887e-08
2094 5.236140310938e-08
2095 5.2345650346286e-08
2096 5.23298807659778e-08
2097 5.23147523963097e-08
2098 5.22999654086753e-08
2099 5.22825709490249e-08
2100 5.22679518821789e-08
2101 5.22520994228337e-08
2102 5.22368644042537e-08
2103 5.22215261535841e-08
2104 5.22064556323087e-08
2105 5.21900995891222e-08
2106 5.21755490887621e-08
2107 5.21590642730274e-08
2108 5.21440448264521e-08
2109 5.21280195329155e-08
2110 5.2113028216727e-08
2111 5.20967017911822e-08
2112 5.20826217762504e-08
2113 5.20660222673719e-08
2114 5.20506047476665e-08
2115 5.20361636620059e-08
2116 5.2019827120553e-08
2117 5.20051451688985e-08
2118 5.19891484365154e-08
2119 5.19736309780861e-08
2120 5.19586422953466e-08
2121 5.19428708982694e-08
2122 5.1928139709112e-08
2123 5.19131968208519e-08
2124 5.18973478667029e-08
2125 5.18824046618072e-08
2126 5.18666180231442e-08
2127 5.18516056047247e-08
2128 5.18357259395863e-08
2129 5.18211192730433e-08
2130 5.18061808918446e-08
2131 5.17911439268381e-08
2132 5.17758504003929e-08
2133 5.17605265968335e-08
2134 5.17452187320799e-08
2135 5.17311250991526e-08
2136 5.17145150693565e-08
2137 5.17008021221343e-08
2138 5.16851284064757e-08
2139 5.16704830193504e-08
2140 5.16560970007518e-08
2141 5.16409240538529e-08
2142 5.16252264093353e-08
2143 5.16105470360628e-08
2144 5.15966800982426e-08
2145 5.15808888410518e-08
2146 5.15667403213627e-08
2147 5.15526945350508e-08
2148 5.1536773166827e-08
2149 5.15230966708913e-08
2150 5.1507456713562e-08
2151 5.14927491184203e-08
2152 5.14792680035292e-08
2153 5.146388701327e-08
2154 5.14501407384849e-08
2155 5.14354898961678e-08
2156 5.14204847985589e-08
2157 5.14066565129312e-08
2158 5.13920149827207e-08
2159 5.13770425434323e-08
2160 5.13619147866073e-08
2161 5.13474342431763e-08
2162 5.13331437512754e-08
2163 5.13186382025133e-08
2164 5.1303256863644e-08
2165 5.12894900079885e-08
2166 5.12745345302434e-08
2167 5.1259541715698e-08
2168 5.12441179583156e-08
2169 5.12303523136914e-08
2170 5.1214996202642e-08
2171 5.12001789076955e-08
2172 5.11855161566821e-08
2173 5.11703557002363e-08
2174 5.11546663037876e-08
2175 5.11398953104702e-08
2176 5.11254144237583e-08
2177 5.11101113707113e-08
2178 5.10951220551448e-08
2179 5.10799520565541e-08
2180 5.10646113451862e-08
2181 5.10494274461593e-08
2182 5.10350364675283e-08
2183 5.10189752218615e-08
2184 5.10046316439805e-08
2185 5.09891313358324e-08
2186 5.09746217094431e-08
2187 5.09583893926191e-08
2188 5.09443662428666e-08
2189 5.09281157641261e-08
2190 5.09133115156324e-08
2191 5.08983697113941e-08
2192 5.0883263300161e-08
2193 5.08670555370294e-08
2194 5.08530754936842e-08
2195 5.08373071692603e-08
2196 5.08218201344945e-08
2197 5.08063077977461e-08
2198 5.07916149135035e-08
2199 5.07760835617432e-08
2200 5.07609704487599e-08
2201 5.0745552010234e-08
2202 5.07301333945165e-08
2203 5.071543569235e-08
2204 5.06997234639428e-08
2205 5.06850236390299e-08
2206 5.06694377571115e-08
2207 5.06542075036087e-08
2208 5.0638271316128e-08
2209 5.06243923421579e-08
2210 5.06078359649997e-08
2211 5.05934142736031e-08
2212 5.05777627739512e-08
2213 5.05623313999948e-08
2214 5.05476018193285e-08
2215 5.05317194066102e-08
2216 5.05168718452076e-08
2217 5.05016037424255e-08
2218 5.04865325585691e-08
2219 5.04707443749197e-08
2220 5.04569406594158e-08
2221 5.04401778096408e-08
2222 5.0425820632416e-08
2223 5.04103646083998e-08
2224 5.03954962236541e-08
2225 5.03799325279886e-08
2226 5.03644577025675e-08
2227 5.03496542383353e-08
2228 5.03341107984667e-08
2229 5.0319249282893e-08
2230 5.03043145112514e-08
2231 5.02890569089587e-08
2232 5.027373089872e-08
2233 5.02586183475096e-08
2234 5.02434387974482e-08
2235 5.02287116206368e-08
2236 5.02134800162146e-08
2237 5.01982212304242e-08
2238 5.01832854729045e-08
2239 5.01686143410396e-08
2240 5.01522394436016e-08
2241 5.0138274546363e-08
2242 5.01231395095836e-08
2243 5.01066854701193e-08
2244 5.00923842010614e-08
2245 5.00783431269802e-08
2246 5.00624705539465e-08
2247 5.00473073712371e-08
2248 5.00331668549237e-08
2249 5.0016907867878e-08
2250 5.00032091066416e-08
2251 4.99872526478207e-08
2252 4.99718779707337e-08
2253 4.99572690864092e-08
2254 4.9941775329021e-08
2255 4.99269315223927e-08
2256 4.99122442194633e-08
2257 4.98964078001762e-08
2258 4.98816323006856e-08
2259 4.98668606643271e-08
2260 4.98513457629635e-08
2261 4.98359945924065e-08
2262 4.98206933463585e-08
2263 4.98066174925427e-08
2264 4.97910383079336e-08
2265 4.97752911035043e-08
2266 4.9760480052452e-08
2267 4.97454657764074e-08
2268 4.97298502484256e-08
2269 4.97147382279017e-08
2270 4.96998956420747e-08
2271 4.96845890105568e-08
2272 4.96689329425592e-08
2273 4.9653185898002e-08
2274 4.96389997906022e-08
2275 4.96229962507755e-08
2276 4.9607367049731e-08
2277 4.95924011874038e-08
2278 4.95769851291961e-08
2279 4.95609430011257e-08
2280 4.95461890852589e-08
2281 4.95309903545404e-08
2282 4.95142451213404e-08
2283 4.95002098541697e-08
2284 4.94837161180151e-08
2285 4.94685771195158e-08
2286 4.94523559484428e-08
2287 4.94374227955063e-08
2288 4.94222426370428e-08
2289 4.94053587565269e-08
2290 4.93914133774531e-08
2291 4.93741681908411e-08
2292 4.9359302462193e-08
2293 4.93433709500835e-08
2294 4.93279536728508e-08
2295 4.93113471349282e-08
2296 4.92964546277008e-08
2297 4.92800357863032e-08
2298 4.92645625560506e-08
2299 4.92483359133544e-08
2300 4.92324179366399e-08
2301 4.92167483274741e-08
2302 4.92005577084065e-08
2303 4.91853479815951e-08
2304 4.91690530868283e-08
2305 4.91531950723711e-08
2306 4.91374401745404e-08
2307 4.91203104413529e-08
2308 4.91046483714896e-08
2309 4.90894254174812e-08
2310 4.90732068119115e-08
2311 4.90571582445476e-08
2312 4.90407141038318e-08
2313 4.90248947300209e-08
2314 4.90083348774206e-08
2315 4.8993048915591e-08
2316 4.89765785838259e-08
2317 4.89596870751896e-08
2318 4.89439479816056e-08
2319 4.89281792028784e-08
2320 4.89110591828101e-08
2321 4.88957807665003e-08
2322 4.88782964058743e-08
2323 4.88630149280134e-08
2324 4.884654866677e-08
2325 4.88301174450534e-08
2326 4.88132790921192e-08
2327 4.87979366989855e-08
2328 4.87804758435573e-08
2329 4.87653840730573e-08
2330 4.87480313466904e-08
2331 4.87325630578184e-08
2332 4.87150851786744e-08
2333 4.86987602843492e-08
2334 4.8683182308551e-08
2335 4.86661788672293e-08
2336 4.86496605938136e-08
2337 4.86331309228483e-08
2338 4.86171592188889e-08
2339 4.86003119175571e-08
2340 4.85837368655595e-08
2341 4.85675078207848e-08
2342 4.85499101872122e-08
2343 4.85346524889962e-08
2344 4.85174489224249e-08
2345 4.85005016219731e-08
2346 4.84846809816197e-08
2347 4.84674054770373e-08
2348 4.84506326468015e-08
2349 4.84342622355527e-08
2350 4.8418277323492e-08
2351 4.84011265942108e-08
2352 4.83842149985314e-08
2353 4.83675764373359e-08
2354 4.83512505455863e-08
2355 4.83341218506794e-08
2356 4.83171801386462e-08
2357 4.83004360352091e-08
2358 4.82839990159079e-08
2359 4.82673947090895e-08
2360 4.82504385086457e-08
2361 4.82337457397009e-08
2362 4.82163173747274e-08
2363 4.81993428729233e-08
2364 4.81827863167972e-08
2365 4.81655092570143e-08
2366 4.81490286299291e-08
2367 4.81315755584966e-08
2368 4.81148758728622e-08
2369 4.80976307075665e-08
2370 4.80804943978619e-08
2371 4.80642235438644e-08
2372 4.80474682653664e-08
2373 4.80301821621509e-08
2374 4.80131965394648e-08
2375 4.79959849917222e-08
2376 4.79790851137807e-08
2377 4.79617747708438e-08
2378 4.79449113117703e-08
2379 4.7927463050712e-08
2380 4.79108683557605e-08
2381 4.78931239711322e-08
2382 4.78763931597292e-08
2383 4.78594783670516e-08
2384 4.78416972153894e-08
2385 4.78246429720386e-08
2386 4.78078655357095e-08
2387 4.77904213660452e-08
2388 4.77737104986886e-08
2389 4.77556369635757e-08
2390 4.77389639379489e-08
2391 4.77207714904004e-08
2392 4.77049979341615e-08
2393 4.76876063939535e-08
2394 4.76700633509353e-08
2395 4.76518391079317e-08
2396 4.76351010378906e-08
2397 4.7617763561103e-08
2398 4.76005620475561e-08
2399 4.75830095041374e-08
2400 4.75658980865212e-08
2401 4.7548359409344e-08
2402 4.75320043187288e-08
2403 4.75131608790313e-08
2404 4.74966657759701e-08
2405 4.74786882240785e-08
2406 4.74618147801031e-08
2407 4.74439634654544e-08
2408 4.7426584562249e-08
2409 4.74094586553342e-08
2410 4.73916236449767e-08
2411 4.73743547724226e-08
2412 4.73560230385139e-08
2413 4.73392839794862e-08
2414 4.73218241872075e-08
2415 4.73038720629759e-08
2416 4.72870073853215e-08
2417 4.72684454804195e-08
2418 4.72511459030933e-08
2419 4.72348306672643e-08
2420 4.72164840781275e-08
2421 4.71993439461471e-08
2422 4.71817097325022e-08
2423 4.7163562879593e-08
2424 4.71461534683471e-08
2425 4.71298414788102e-08
2426 4.71110717730205e-08
2427 4.70951611415416e-08
2428 4.70768696869683e-08
2429 4.70589705012792e-08
2430 4.7041218944166e-08
2431 4.70240990058102e-08
2432 4.70064734034992e-08
2433 4.69891306384973e-08
2434 4.69716096493933e-08
2435 4.69543907972714e-08
2436 4.69370344604592e-08
2437 4.69189707783535e-08
2438 4.6902353692424e-08
2439 4.68843884053882e-08
2440 4.68672643685331e-08
2441 4.6849198304777e-08
2442 4.68323191320508e-08
2443 4.68139826352854e-08
2444 4.67975790918729e-08
2445 4.67787064781788e-08
2446 4.67622242146248e-08
2447 4.67443286904512e-08
2448 4.67274879949464e-08
2449 4.67095359404368e-08
2450 4.669114586342e-08
2451 4.66747674874313e-08
2452 4.66568396135791e-08
2453 4.66400262175704e-08
2454 4.66219168377968e-08
2455 4.66037381152695e-08
2456 4.65877039843576e-08
2457 4.65695345996941e-08
2458 4.65524797088612e-08
2459 4.65345869233857e-08
2460 4.65172308881101e-08
2461 4.64997305944514e-08
2462 4.64818161183267e-08
2463 4.64648451128369e-08
2464 4.64466776408656e-08
2465 4.64301188389804e-08
2466 4.64120295684545e-08
2467 4.63944343622735e-08
2468 4.63767927887382e-08
2469 4.63594521065147e-08
2470 4.63415067804007e-08
2471 4.63246847317578e-08
2472 4.63068260021515e-08
2473 4.62890534285165e-08
2474 4.62709411181983e-08
2475 4.62539781862503e-08
2476 4.62364889917133e-08
2477 4.62186357954586e-08
2478 4.62007319561586e-08
2479 4.61831687093017e-08
2480 4.61663286595027e-08
2481 4.61478658215775e-08
2482 4.61309596970771e-08
2483 4.61126844144566e-08
2484 4.6094710665745e-08
2485 4.60776771955551e-08
2486 4.60605329464236e-08
2487 4.60422907693214e-08
2488 4.60248354392512e-08
2489 4.6007311860663e-08
2490 4.59899816256382e-08
2491 4.59722943668694e-08
2492 4.59540419153193e-08
2493 4.59366818987839e-08
2494 4.59195528894618e-08
2495 4.59010801980853e-08
2496 4.58845287378828e-08
2497 4.58655004749176e-08
2498 4.58494153745548e-08
2499 4.58310705293563e-08
2500 4.58136174619206e-08
2501 4.57959994939472e-08
2502 4.57787900773887e-08
2503 4.57607184376485e-08
2504 4.57430257032598e-08
2505 4.57257158124058e-08
2506 4.57087001120016e-08
2507 4.56908969179715e-08
2508 4.56736003546787e-08
2509 4.565639717935e-08
2510 4.56391271934642e-08
2511 4.56208742809494e-08
2512 4.5603155283569e-08
2513 4.55866312250031e-08
2514 4.55687715339437e-08
2515 4.55514658765921e-08
2516 4.55339894109841e-08
2517 4.55166262671725e-08
2518 4.54996271637143e-08
2519 4.54821383897297e-08
2520 4.54644372416269e-08
2521 4.54469492963128e-08
2522 4.54303013786017e-08
2523 4.54123019109076e-08
2524 4.53956193697813e-08
2525 4.53774400765994e-08
2526 4.53605515220445e-08
2527 4.53426916822153e-08
2528 4.53256731201002e-08
2529 4.53080326785482e-08
2530 4.52913382051889e-08
2531 4.52738045466639e-08
2532 4.52560046109163e-08
2533 4.52387394704878e-08
2534 4.522116781569e-08
2535 4.52038490763584e-08
2536 4.51856571723752e-08
2537 4.51691425067402e-08
2538 4.51514513883922e-08
2539 4.51338963625147e-08
2540 4.5116300111836e-08
2541 4.50991016109903e-08
2542 4.50811129915074e-08
2543 4.50635180113679e-08
2544 4.50467566457391e-08
2545 4.50287946889283e-08
2546 4.50114465340157e-08
2547 4.49925835233067e-08
2548 4.49760138137023e-08
2549 4.49583550645727e-08
2550 4.49408828919751e-08
2551 4.49233158694717e-08
2552 4.49055608373605e-08
2553 4.48877642482692e-08
2554 4.48704956186319e-08
2555 4.48523810931967e-08
2556 4.48351502688737e-08
2557 4.48168728879317e-08
2558 4.47993562131543e-08
2559 4.47820296480828e-08
2560 4.4764258693597e-08
2561 4.47469867479455e-08
2562 4.47292358121132e-08
2563 4.4711400564168e-08
2564 4.46931210222878e-08
2565 4.4676032779023e-08
2566 4.46578585155954e-08
2567 4.46408985159685e-08
2568 4.46227998343041e-08
2569 4.4604542524862e-08
2570 4.4587989229683e-08
2571 4.45696402744389e-08
2572 4.45520502916352e-08
2573 4.45339131971423e-08
2574 4.45159374988791e-08
2575 4.44991207233514e-08
2576 4.44803586412412e-08
2577 4.44638556125199e-08
2578 4.44456252659542e-08
2579 4.44278694620159e-08
2580 4.44102061423202e-08
2581 4.43923167576799e-08
2582 4.4374786773993e-08
2583 4.43565917271904e-08
2584 4.43391494928669e-08
2585 4.4321554690363e-08
2586 4.43037276331992e-08
2587 4.4285714341008e-08
2588 4.42681196628492e-08
2589 4.42504102466934e-08
2590 4.42326403011783e-08
2591 4.42147884549549e-08
2592 4.41968708040363e-08
2593 4.41794936349993e-08
2594 4.41619528754877e-08
2595 4.41437870968286e-08
2596 4.41260942691812e-08
2597 4.41087374336568e-08
2598 4.40907131689094e-08
2599 4.40737289517656e-08
2600 4.40549360711806e-08
2601 4.40384204285493e-08
2602 4.4019955409702e-08
2603 4.40023948069523e-08
2604 4.39851797437996e-08
2605 4.39663204070406e-08
2606 4.39498351094692e-08
2607 4.39317445612986e-08
2608 4.39137228882558e-08
2609 4.38964009901177e-08
2610 4.38791808132777e-08
2611 4.38606000470187e-08
2612 4.38434326168746e-08
2613 4.38256937638215e-08
2614 4.38073965289831e-08
2615 4.37904408077117e-08
2616 4.37727229578577e-08
2617 4.37554614465263e-08
2618 4.37366492023195e-08
2619 4.37199036640301e-08
2620 4.37025007080649e-08
2621 4.36846425531101e-08
2622 4.36668481231806e-08
2623 4.36490708510817e-08
2624 4.36323070487354e-08
2625 4.36138909591932e-08
2626 4.35967734859766e-08
2627 4.35788248411839e-08
2628 4.35609863673747e-08
2629 4.35444158823906e-08
2630 4.35263039704203e-08
2631 4.35084783285689e-08
2632 4.34915255866919e-08
2633 4.34738336196894e-08
2634 4.3455982336571e-08
2635 4.3439071524265e-08
2636 4.34214349041007e-08
2637 4.34034273748551e-08
2638 4.33866414626394e-08
2639 4.33689395960002e-08
2640 4.33514640127974e-08
2641 4.33341886418859e-08
2642 4.33167285769365e-08
2643 4.32987319172184e-08
2644 4.32820891562713e-08
2645 4.32648532133761e-08
2646 4.32466318787128e-08
2647 4.32298974977208e-08
2648 4.32119841797807e-08
2649 4.31948717531938e-08
2650 4.3177260717453e-08
2651 4.31603731323449e-08
2652 4.31429276277484e-08
2653 4.31250558214913e-08
2654 4.31087049221901e-08
2655 4.30914474005561e-08
2656 4.30740545733776e-08
2657 4.30566560325474e-08
2658 4.30395076698176e-08
2659 4.30228155190449e-08
2660 4.30050253465986e-08
2661 4.29876040719535e-08
2662 4.29710102678449e-08
2663 4.2953459167272e-08
2664 4.29372153938345e-08
2665 4.29187247985574e-08
2666 4.29033120887823e-08
2667 4.28853386642558e-08
2668 4.28682037512118e-08
2669 4.28513804839881e-08
2670 4.28347432914755e-08
2671 4.28169458754901e-08
2672 4.28004124972681e-08
2673 4.27832640119696e-08
2674 4.27665379465481e-08
2675 4.27493322470163e-08
2676 4.27323239002852e-08
2677 4.27152897617411e-08
2678 4.26995023712351e-08
2679 4.26819578227544e-08
2680 4.26649711542382e-08
2681 4.26484816973272e-08
2682 4.26317604254045e-08
2683 4.26143120422218e-08
2684 4.25983773078009e-08
2685 4.25807313226834e-08
2686 4.25646285089343e-08
2687 4.25474135954396e-08
2688 4.25314105751973e-08
2689 4.25138229722677e-08
2690 4.24973554622454e-08
2691 4.24812228745353e-08
2692 4.24640932159548e-08
2693 4.24475616651598e-08
2694 4.24311596325389e-08
2695 4.24145045916369e-08
2696 4.23979350610004e-08
2697 4.23803965206027e-08
2698 4.23647528959847e-08
2699 4.23479198352617e-08
2700 4.23317355524588e-08
2701 4.23144573220568e-08
2702 4.22984339576438e-08
2703 4.22822793031408e-08
2704 4.22653907752313e-08
2705 4.22491576004269e-08
2706 4.22330214351341e-08
2707 4.22158789010751e-08
2708 4.22005222842081e-08
2709 4.21835091715117e-08
2710 4.21672622592517e-08
2711 4.21509410730714e-08
2712 4.21345872090306e-08
2713 4.21185714252204e-08
2714 4.21023566183898e-08
2715 4.20859336771606e-08
2716 4.20696428489187e-08
2717 4.20537229124385e-08
2718 4.20373289586884e-08
2719 4.20209523146475e-08
2720 4.20054475247511e-08
2721 4.19889658243022e-08
2722 4.19730670735419e-08
2723 4.19559195385943e-08
2724 4.19403918097139e-08
2725 4.19241036730966e-08
2726 4.19081801092958e-08
2727 4.18924514442587e-08
2728 4.18768804135716e-08
2729 4.18597667022347e-08
2730 4.18445824852398e-08
2731 4.18282382277368e-08
2732 4.18125972205807e-08
2733 4.17967769728023e-08
2734 4.17809930226554e-08
2735 4.17645650716558e-08
2736 4.17488865660509e-08
2737 4.173335099944e-08
2738 4.1717285864884e-08
2739 4.17014849793951e-08
2740 4.16855810572159e-08
2741 4.16698557801354e-08
2742 4.16546160439957e-08
2743 4.16383900945227e-08
2744 4.1622976968192e-08
2745 4.16065975117341e-08
2746 4.1591645354444e-08
2747 4.15756599241313e-08
2748 4.15603795915764e-08
2749 4.15449061508255e-08
2750 4.1529051529654e-08
2751 4.15137031080093e-08
2752 4.14974085658493e-08
2753 4.14829860040378e-08
2754 4.14668906434379e-08
2755 4.14520815152919e-08
2756 4.14356879083755e-08
2757 4.14205714922566e-08
2758 4.14053636514922e-08
2759 4.13896847417661e-08
2760 4.13748230396749e-08
2761 4.13588742715909e-08
2762 4.13433598076551e-08
2763 4.13290579173164e-08
2764 4.13127002785174e-08
2765 4.12974445156422e-08
2766 4.12827378801772e-08
2767 4.12670445077978e-08
2768 4.12514431094024e-08
2769 4.12368310964695e-08
2770 4.1221627705923e-08
2771 4.1206322383136e-08
2772 4.11909861366411e-08
2773 4.11770459587402e-08
2774 4.1160610537716e-08
2775 4.11455400892713e-08
2776 4.11306666330269e-08
2777 4.11164224591865e-08
2778 4.11005631066885e-08
2779 4.10857754835625e-08
2780 4.10704720552602e-08
2781 4.10559003936051e-08
2782 4.10405232189603e-08
2783 4.10258183314305e-08
2784 4.10112834727805e-08
2785 4.09955531699424e-08
2786 4.09812820043598e-08
2787 4.09668569174571e-08
2788 4.09514311270343e-08
2789 4.09361599165159e-08
2790 4.09223989845664e-08
2791 4.0907241765975e-08
2792 4.08924352215401e-08
2793 4.08778033875734e-08
2794 4.08629039903019e-08
2795 4.08478207658547e-08
2796 4.08340523456729e-08
2797 4.08182989621864e-08
2798 4.08043450272899e-08
2799 4.07898911731763e-08
2800 4.07749129109902e-08
2801 4.07605473919936e-08
2802 4.0745784974483e-08
2803 4.07313968242562e-08
2804 4.0716961649867e-08
2805 4.07025457875854e-08
2806 4.0688028638769e-08
2807 4.06735709388428e-08
2808 4.06588852457368e-08
2809 4.0644667195e-08
2810 4.06303127862273e-08
2811 4.0615969988167e-08
2812 4.06020792769191e-08
2813 4.05870835633593e-08
2814 4.05728632189017e-08
2815 4.0558966161619e-08
2816 4.05442104502995e-08
2817 4.05299530923386e-08
2818 4.05163754382265e-08
2819 4.0501403839599e-08
2820 4.04869767529625e-08
2821 4.04734266976625e-08
2822 4.04591709122215e-08
2823 4.04447905308913e-08
2824 4.04312017119324e-08
2825 4.04171258585606e-08
2826 4.04022812783289e-08
2827 4.03887819482307e-08
2828 4.03749526558528e-08
2829 4.0360773821746e-08
2830 4.03468258407536e-08
2831 4.03330985809802e-08
2832 4.03184236605902e-08
2833 4.03044979742084e-08
2834 4.02910452597105e-08
2835 4.02771515419786e-08
2836 4.02638455598847e-08
2837 4.024903098232e-08
2838 4.02359074773706e-08
2839 4.02224823901065e-08
2840 4.02084400286107e-08
2841 4.01952757260737e-08
2842 4.01818182380609e-08
2843 4.01690377640485e-08
2844 4.01552323867627e-08
2845 4.01424566249808e-08
2846 4.0129659761412e-08
2847 4.0116633842846e-08
2848 4.01039840447481e-08
2849 4.00916271781071e-08
2850 4.0079798012016e-08
2851 4.00665032560532e-08
2852 4.00547543106811e-08
2853 4.0042121964845e-08
2854 4.00294679026025e-08
2855 4.00160069085054e-08
2856 4.00034523213577e-08
2857 3.99909407948762e-08
2858 3.99769030199337e-08
2859 3.9963816455213e-08
2860 3.99506913031367e-08
2861 3.99383587481594e-08
2862 3.99237873094371e-08
2863 3.99114385438715e-08
2864 3.98982536653492e-08
2865 3.98845592899377e-08
2866 3.9872084370618e-08
2867 3.98589627135237e-08
2868 3.98451991907578e-08
2869 3.9832532213957e-08
2870 3.98190544288468e-08
2871 3.98060471122896e-08
2872 3.97927878341697e-08
2873 3.97799286675493e-08
2874 3.97666864095925e-08
2875 3.97531164466614e-08
2876 3.97405583463239e-08
2877 3.97279324113597e-08
2878 3.97145916233299e-08
2879 3.9702115727458e-08
2880 3.96889232785469e-08
2881 3.96754538303234e-08
2882 3.96633077865971e-08
2883 3.96501323067788e-08
2884 3.9636917507746e-08
2885 3.96243092231074e-08
2886 3.96120446546355e-08
2887 3.95990831663084e-08
2888 3.95859925803599e-08
2889 3.95726596780221e-08
2890 3.95608842498341e-08
2891 3.95477174062187e-08
2892 3.95345471679853e-08
2893 3.95221442621718e-08
2894 3.95098063892263e-08
2895 3.9496626772717e-08
2896 3.94839163275762e-08
2897 3.9472242316485e-08
2898 3.94594499355527e-08
2899 3.94470686591042e-08
2900 3.94334809579178e-08
2901 3.94218270978186e-08
2902 3.94088244699553e-08
2903 3.93962323337327e-08
2904 3.93842599826222e-08
2905 3.93715220723401e-08
2906 3.93591980278885e-08
2907 3.93467086166055e-08
2908 3.93347201326222e-08
2909 3.9321834414352e-08
2910 3.93091443977589e-08
2911 3.92968326399234e-08
2912 3.92855083726218e-08
2913 3.92726411067024e-08
2914 3.92599962042439e-08
2915 3.92480011881524e-08
2916 3.92360764069899e-08
2917 3.92235961266252e-08
2918 3.92113068201638e-08
2919 3.91990739219139e-08
2920 3.91873834693079e-08
2921 3.91745643875119e-08
2922 3.91629900660106e-08
2923 3.91506387584784e-08
2924 3.91383206017615e-08
2925 3.91267668344852e-08
2926 3.9114469409629e-08
2927 3.91029537221144e-08
2928 3.90894562207755e-08
2929 3.90794782312298e-08
2930 3.90660060967107e-08
2931 3.90550435582782e-08
2932 3.90436968644714e-08
2933 3.90307693334258e-08
2934 3.9018978649441e-08
2935 3.90073374685329e-08
2936 3.89960899953579e-08
2937 3.89832601448425e-08
2938 3.89720682449912e-08
2939 3.89600258219325e-08
2940 3.89478246085595e-08
2941 3.89373140845528e-08
2942 3.89246667515941e-08
2943 3.89127779456544e-08
2944 3.89017057362828e-08
2945 3.8889591594149e-08
2946 3.88780600109051e-08
2947 3.88667775781393e-08
2948 3.88550786656339e-08
2949 3.88437542317988e-08
2950 3.88314764445674e-08
2951 3.88204115302493e-08
2952 3.88087936000758e-08
2953 3.87971535737996e-08
2954 3.87866198248155e-08
2955 3.87745887642232e-08
2956 3.87631979683611e-08
2957 3.87513733253186e-08
2958 3.87401247059493e-08
2959 3.87288517740281e-08
2960 3.87170499958067e-08
2961 3.87054061468106e-08
2962 3.86941962675635e-08
2963 3.86830295990848e-08
2964 3.86713898796742e-08
2965 3.86601311372914e-08
2966 3.86494937369974e-08
2967 3.86368171749751e-08
2968 3.8626396920094e-08
2969 3.8614512896995e-08
2970 3.8603937711823e-08
2971 3.85922748495915e-08
2972 3.85806994609439e-08
2973 3.8569122328358e-08
2974 3.8558971639624e-08
2975 3.854729837105e-08
2976 3.8535832925124e-08
2977 3.85249392422793e-08
2978 3.85136150895526e-08
2979 3.85022478184283e-08
2980 3.84913686972688e-08
2981 3.84799371566658e-08
2982 3.84686826326863e-08
2983 3.84579271099916e-08
2984 3.84466035603381e-08
2985 3.8435898582545e-08
2986 3.84239570934142e-08
2987 3.84132292650108e-08
2988 3.8402469744625e-08
2989 3.83907100003356e-08
2990 3.83800624943476e-08
2991 3.8368330347982e-08
2992 3.8357578681314e-08
2993 3.83467511828606e-08
2994 3.83358458262428e-08
2995 3.83241408692392e-08
2996 3.83131968773043e-08
2997 3.83026178916168e-08
2998 3.82919769927881e-08
2999 3.82801573515223e-08
3000 1.72346820218605e-08
3001 1.72715792701172e-08
3002 1.74690218491258e-08
3003 1.75760074368203e-08
3004 1.76215039660588e-08
3005 1.76352430731608e-08
3006 1.76361427334237e-08
3007 1.76330860605922e-08
3008 1.76290027803505e-08
3009 1.76246608472319e-08
3010 1.76204869960783e-08
3011 1.76162411891778e-08
3012 1.76121995340561e-08
3013 1.76081621785507e-08
3014 1.76041967002161e-08
3015 1.76003226106947e-08
3016 1.75964725692146e-08
3017 1.75926745297761e-08
3018 1.75888689112058e-08
3019 1.75851105180258e-08
3020 1.75814191521184e-08
3021 1.75777287703127e-08
3022 1.7574051825564e-08
3023 1.75704422812339e-08
3024 1.75667919896649e-08
3025 1.75631752260264e-08
3026 1.75596928514987e-08
3027 1.75561706552974e-08
3028 1.75525520588471e-08
3029 1.75491172152986e-08
3030 1.75456673452146e-08
3031 1.75421247940732e-08
3032 1.75386711741554e-08
3033 1.75352775981796e-08
3034 1.75318608731934e-08
3035 1.75284342274484e-08
3036 1.75250827404716e-08
3037 1.75217288882479e-08
3038 1.7518369762326e-08
3039 1.75150017433789e-08
3040 1.75116953224641e-08
3041 1.75084024172101e-08
3042 1.75051133643189e-08
3043 1.75018442496733e-08
3044 1.7498517376896e-08
3045 1.74952922601668e-08
3046 1.74921078390511e-08
3047 1.74888663832251e-08
3048 1.74856785043198e-08
3049 1.74824120629247e-08
3050 1.74792578518379e-08
3051 1.74760465400658e-08
3052 1.74729202478963e-08
3053 1.74696852918432e-08
3054 1.74666408799817e-08
3055 1.74635250302091e-08
3056 1.74603884872671e-08
3057 1.74573004062262e-08
3058 1.74541899240765e-08
3059 1.7451064177193e-08
3060 1.7448063025699e-08
3061 1.7445003008737e-08
3062 1.74419538876425e-08
3063 1.74389013520848e-08
3064 1.74359404229718e-08
3065 1.74328527859369e-08
3066 1.74298378021187e-08
3067 1.74268059065785e-08
3068 1.74237661755006e-08
3069 1.74208594583558e-08
3070 1.74178515621459e-08
3071 1.74149208533314e-08
3072 1.74119453315857e-08
3073 1.74089984012749e-08
3074 1.74061072907294e-08
3075 1.74031237039907e-08
3076 1.7400230975545e-08
3077 1.73973360723945e-08
3078 1.73943614805161e-08
3079 1.73914597335068e-08
3080 1.73886191245642e-08
3081 1.73857314078874e-08
3082 1.73828527332831e-08
3083 1.73800420409653e-08
3084 1.73771701793002e-08
3085 1.73743206285382e-08
3086 1.73714486575993e-08
3087 1.73686740529144e-08
3088 1.73658242899333e-08
3089 1.73629708190848e-08
3090 1.73601623124631e-08
3091 1.73574121485331e-08
3092 1.73545848776713e-08
3093 1.73517857082195e-08
3094 1.73490028517187e-08
3095 1.73462182110895e-08
3096 1.73434734364875e-08
3097 1.73407628271394e-08
3098 1.73379601859924e-08
3099 1.73352197981036e-08
3100 1.73325073337394e-08
3101 1.73297092687374e-08
3102 1.73270037209794e-08
3103 1.7324291102766e-08
3104 1.73215526596826e-08
3105 1.73188657178192e-08
3106 1.73161558524315e-08
3107 1.73134947810127e-08
3108 1.73107819355089e-08
3109 1.73081144469456e-08
3110 1.73053488631281e-08
3111 1.73027868599074e-08
3112 1.73001215934832e-08
3113 1.72974417131655e-08
3114 1.72947992642947e-08
3115 1.72921389072489e-08
3116 1.72895175733212e-08
3117 1.72868907332424e-08
3118 1.72842405919693e-08
3119 1.72817186075425e-08
3120 1.72790024811353e-08
3121 1.72763823163835e-08
3122 1.72738177040277e-08
3123 1.7271233498678e-08
3124 1.72686197940364e-08
3125 1.72660521029211e-08
3126 1.72634679615202e-08
3127 1.72608955434361e-08
3128 1.72583349268562e-08
3129 1.72558062516148e-08
3130 1.72532426933325e-08
3131 1.72506489688196e-08
3132 1.72481034932392e-08
3133 1.72456056917181e-08
3134 1.72430372446797e-08
3135 1.72404892029854e-08
3136 1.72379708276438e-08
3137 1.7235405410515e-08
3138 1.72329294889362e-08
3139 1.72304389592925e-08
3140 1.72279060879355e-08
3141 1.72254343725198e-08
3142 1.72228781665784e-08
3143 1.72203762234258e-08
3144 1.72179293247698e-08
3145 1.72154114259637e-08
3146 1.72129067019244e-08
3147 1.72105239020226e-08
3148 1.72080236502115e-08
3149 1.720554233961e-08
3150 1.72030261188227e-08
3151 1.72006055292884e-08
3152 1.71981728072923e-08
3153 1.71957043250126e-08
3154 1.71932974205591e-08
3155 1.71908167605206e-08
3156 1.71883581724375e-08
3157 1.71859626132698e-08
3158 1.71835761177686e-08
3159 1.71811287840162e-08
3160 1.71786961614406e-08
3161 1.71762277278997e-08
3162 1.71738213854411e-08
3163 1.71714375255538e-08
3164 1.71690555653969e-08
3165 1.71666630589817e-08
3166 1.71642711463416e-08
3167 1.71618882344182e-08
3168 1.71595496449573e-08
3169 1.71571396866133e-08
3170 1.71547238565772e-08
3171 1.71523925695527e-08
3172 1.71500316267226e-08
3173 1.71476836104967e-08
3174 1.71452960873453e-08
3175 1.71429228537356e-08
3176 1.71405549714843e-08
3177 1.71382393644726e-08
3178 1.71358683257461e-08
3179 1.7133575982764e-08
3180 1.71311629499349e-08
3181 1.71288667087655e-08
3182 1.71264995506293e-08
3183 1.71241876437966e-08
3184 1.71219179234405e-08
3185 1.71195695957138e-08
3186 1.71172433211242e-08
3187 1.711495795928e-08
3188 1.711258910575e-08
3189 1.71103697161334e-08
3190 1.71080528856282e-08
3191 1.7105697904779e-08
3192 1.71033909863172e-08
3193 1.71011166379409e-08
3194 1.70988432160457e-08
3195 1.70965433818338e-08
3196 1.70942305664779e-08
3197 1.70919892679144e-08
3198 1.70897055268293e-08
3199 1.70874354936401e-08
3200 1.70851616904943e-08
3201 1.70829348128676e-08
3202 1.7080614410081e-08
3203 1.70783923913731e-08
3204 1.70761167465894e-08
3205 1.70738629786116e-08
3206 1.70716109203772e-08
3207 1.7069376993778e-08
3208 1.70671447216608e-08
3209 1.70649121001565e-08
3210 1.70626598568202e-08
3211 1.70604220838533e-08
3212 1.70581987900542e-08
3213 1.70560042644929e-08
3214 1.70537454204089e-08
3215 1.70514773618624e-08
3216 1.70492868779848e-08
3217 1.70470615150076e-08
3218 1.70448488860386e-08
3219 1.70426870200746e-08
3220 1.70404475799413e-08
3221 1.70382350834775e-08
3222 1.70359928890473e-08
3223 1.70338089839905e-08
3224 1.70316017478467e-08
3225 1.70293988380754e-08
3226 1.70272208245281e-08
3227 1.7025024288575e-08
3228 1.70227844207838e-08
3229 1.7020661307765e-08
3230 1.70184555246533e-08
3231 1.70163006634472e-08
3232 1.70140995648937e-08
3233 1.70119190950613e-08
3234 1.70097522315948e-08
3235 1.70075458303665e-08
3236 1.70054110318618e-08
3237 1.70032242809703e-08
3238 1.70011007173954e-08
3239 1.69989307469143e-08
3240 1.69967753539391e-08
3241 1.6994654775393e-08
3242 1.69924301970104e-08
3243 1.69902985341086e-08
3244 1.69881833377294e-08
3245 1.69860310549885e-08
3246 1.69839597517485e-08
3247 1.69817031164532e-08
3248 1.69796119198484e-08
3249 1.69774779234799e-08
3250 1.69753160935426e-08
3251 1.69731586213973e-08
3252 1.69710289689984e-08
3253 1.69689765020242e-08
3254 1.69668457184746e-08
3255 1.69646941133028e-08
3256 1.6962529997111e-08
3257 1.69604344539831e-08
3258 1.69583344877267e-08
3259 1.69561983980049e-08
3260 1.69541557595299e-08
3261 1.69520206352025e-08
3262 1.69499018171093e-08
3263 1.69478224358033e-08
3264 1.69456576976923e-08
3265 1.69436129734413e-08
3266 1.69415396953643e-08
3267 1.69394134680312e-08
3268 1.69373366008252e-08
3269 1.69352522525756e-08
3270 1.69331258094985e-08
3271 1.69310423173974e-08
3272 1.69289631829217e-08
3273 1.69268634293007e-08
3274 1.69248159504198e-08
3275 1.69227098235081e-08
3276 1.69206765695817e-08
3277 1.69185671503314e-08
3278 1.69164804800226e-08
3279 1.6914424170289e-08
3280 1.69124078327476e-08
3281 1.69102941317834e-08
3282 1.69082399820719e-08
3283 1.6906210505846e-08
3284 1.69040646880725e-08
3285 1.69020413598453e-08
3286 1.6899945534915e-08
3287 1.68979003373482e-08
3288 1.68958419786425e-08
3289 1.68937636362276e-08
3290 1.68917502135213e-08
3291 1.68897080479458e-08
3292 1.68876543489571e-08
3293 1.68856298300712e-08
3294 1.68835487742991e-08
3295 1.68815826093738e-08
3296 1.68794464416588e-08
3297 1.68774865163535e-08
3298 1.68754097401025e-08
3299 1.68734019180272e-08
3300 1.68713632677508e-08
3301 1.68693640233419e-08
3302 1.6867296475237e-08
3303 1.6865298695351e-08
3304 1.6863292947672e-08
3305 1.68612632252818e-08
3306 1.68592033619741e-08
3307 1.68572236959874e-08
3308 1.68551575027154e-08
3309 1.68531986616816e-08
3310 1.68511724450426e-08
3311 1.68491073291033e-08
3312 1.68471137541593e-08
3313 1.68451065280517e-08
3314 1.68431329027885e-08
3315 1.68411123180889e-08
3316 1.68391050586192e-08
3317 1.68371505745446e-08
3318 1.68351195199978e-08
3319 1.68330860834398e-08
3320 1.68311236128926e-08
3321 1.68290918713954e-08
3322 1.68270941168502e-08
3323 1.68251153377652e-08
3324 1.68231314178757e-08
3325 1.682110301926e-08
3326 1.6819146579472e-08
3327 1.68172015250767e-08
3328 1.68152184801262e-08
3329 1.68132577080538e-08
3330 1.6811244190118e-08
3331 1.68092441850953e-08
3332 1.68073352399822e-08
3333 1.68053274484103e-08
3334 1.68033879941465e-08
3335 1.68014131510519e-08
3336 1.67994673348215e-08
3337 1.67975283526522e-08
3338 1.67955335339809e-08
3339 1.67935950990405e-08
3340 1.67916526262191e-08
3341 1.67896368882814e-08
3342 1.67877023857232e-08
3343 1.67857518055259e-08
3344 1.67838073713011e-08
3345 1.6781843118957e-08
3346 1.67799072894326e-08
3347 1.6777975537785e-08
3348 1.67760364273017e-08
3349 1.67740853498077e-08
3350 1.67721704731338e-08
3351 1.67702028312511e-08
3352 1.6768264647693e-08
3353 1.67663556318587e-08
3354 1.67644034775316e-08
3355 1.67624678278633e-08
3356 1.67605545638994e-08
3357 1.67585617395771e-08
3358 1.67566841432765e-08
3359 1.6754719214529e-08
3360 1.67528464395383e-08
3361 1.67509068815797e-08
3362 1.67489257277276e-08
3363 1.67470434055961e-08
3364 1.67451241909755e-08
3365 1.67431600754664e-08
3366 1.67412647935072e-08
3367 1.67393289902118e-08
3368 1.67374683666122e-08
3369 1.67354992740287e-08
3370 1.67336255933737e-08
3371 1.6731699584549e-08
3372 1.67297471107553e-08
3373 1.67278942192761e-08
3374 1.67259646211559e-08
3375 1.67240498694654e-08
3376 1.67221534016493e-08
3377 1.6720230936712e-08
3378 1.67183376079683e-08
3379 1.67164457725022e-08
3380 1.67145577572581e-08
3381 1.67126351114932e-08
3382 1.67107186749282e-08
3383 1.67087877072647e-08
3384 1.67069356639127e-08
3385 1.67050134448621e-08
3386 1.67031462816325e-08
3387 1.67012140414868e-08
3388 1.66993083401956e-08
3389 1.66974745127446e-08
3390 1.66955885373132e-08
3391 1.66936796626715e-08
3392 1.66918231522262e-08
3393 1.66898982789099e-08
3394 1.66880024764227e-08
3395 1.66861577209909e-08
3396 1.66842337990802e-08
3397 1.66823980966568e-08
3398 1.66805515657009e-08
3399 1.66786367053751e-08
3400 1.66767599303619e-08
3401 1.66749092365137e-08
3402 1.66730780981894e-08
3403 1.66711726680702e-08
3404 1.66693252645622e-08
3405 1.66674170229419e-08
3406 1.66655806055904e-08
3407 1.66636881965554e-08
3408 1.66617709187744e-08
3409 1.66599357320774e-08
3410 1.66580963731344e-08
3411 1.665621775368e-08
3412 1.66544206364749e-08
3413 1.66525734984213e-08
3414 1.66506497604468e-08
3415 1.66488045002799e-08
3416 1.66469794708191e-08
3417 1.66450931789186e-08
3418 1.66432521228055e-08
3419 1.66414115160551e-08
3420 1.6639639660726e-08
3421 1.66377279340046e-08
3422 1.66358881444617e-08
3423 1.66339995252285e-08
3424 1.66321683331694e-08
3425 1.66303242609689e-08
3426 1.6628493387183e-08
3427 1.66266367390699e-08
3428 1.66248131298063e-08
3429 1.66229913331761e-08
3430 1.66211262417337e-08
3431 1.6619367216586e-08
3432 1.66174832098298e-08
3433 1.66156631018766e-08
3434 1.66138139121308e-08
3435 1.66120201905151e-08
3436 1.6610158994762e-08
3437 1.66083507542603e-08
3438 1.66065133989368e-08
3439 1.66047082164056e-08
3440 1.66028697013154e-08
3441 1.6601064384919e-08
3442 1.6599234132525e-08
3443 1.65974234862809e-08
3444 1.65955774678872e-08
3445 1.65937646800784e-08
3446 1.65919345558319e-08
3447 1.65901298348203e-08
3448 1.6588322151484e-08
3449 1.65865066399651e-08
3450 1.65846903960598e-08
3451 1.65828861773964e-08
3452 1.65810925653875e-08
3453 1.65793153383553e-08
3454 1.6577469444029e-08
3455 1.65756671405837e-08
3456 1.65738681092431e-08
3457 1.65720814885029e-08
3458 1.6570220056078e-08
3459 1.65684449965342e-08
3460 1.6566618282754e-08
3461 1.6564843336786e-08
3462 1.65630639148706e-08
3463 1.6561249475272e-08
3464 1.65594568609373e-08
3465 1.65576320990957e-08
3466 1.65558925145082e-08
3467 1.65540782062767e-08
3468 1.65522841912846e-08
3469 1.65504924241611e-08
3470 1.65487114417717e-08
3471 1.65469375330574e-08
3472 1.65451423528029e-08
3473 1.65433406993654e-08
3474 1.65415694556859e-08
3475 1.65397586566751e-08
3476 1.65379385908826e-08
3477 1.65361925277829e-08
3478 1.65343641280458e-08
3479 1.65326638524865e-08
3480 1.65308656848995e-08
3481 1.65290570446619e-08
3482 1.6527235474656e-08
3483 1.65255511563323e-08
3484 1.65237654677353e-08
3485 1.65219477567813e-08
3486 1.65201627607414e-08
3487 1.65184653717898e-08
3488 1.65166396788874e-08
3489 1.65148869131215e-08
3490 1.65131341771652e-08
3491 1.65113658911198e-08
3492 1.6509593784686e-08
3493 1.6507844981889e-08
3494 1.65060994069266e-08
3495 1.65043089133399e-08
3496 1.6502583282979e-08
3497 1.650080346427e-08
3498 1.64990640069973e-08
3499 1.64972886112502e-08
3500 1.6495579402781e-08
3501 1.64937971983692e-08
3502 1.64920773545185e-08
3503 1.64903350336754e-08
3504 1.64885609458543e-08
3505 1.64867909189792e-08
3506 1.64850961864305e-08
3507 1.64833124248753e-08
3508 1.64815652041184e-08
3509 1.64798715170944e-08
3510 1.64780843325829e-08
3511 1.64763590641548e-08
3512 1.64746356631218e-08
3513 1.64728851927698e-08
3514 1.64711374706084e-08
3515 1.64694091989159e-08
3516 1.64677312084471e-08
3517 1.64659701028247e-08
3518 1.64641751149996e-08
3519 1.64624672336355e-08
3520 1.64606992575644e-08
3521 1.64590142462118e-08
3522 1.64572006509933e-08
3523 1.64555328652782e-08
3524 1.64537602772574e-08
3525 1.6452098113906e-08
3526 1.64503527549931e-08
3527 1.64486019875454e-08
3528 1.64468977499055e-08
3529 1.64451729096626e-08
3530 1.64434680249292e-08
3531 1.64417196172606e-08
3532 1.64400100477191e-08
3533 1.64382583366651e-08
3534 1.64365156001267e-08
3535 1.64348151464322e-08
3536 1.64331081795033e-08
3537 1.64314033495594e-08
3538 1.64296779083251e-08
3539 1.64279287070968e-08
3540 1.64262648461033e-08
3541 1.64245321394307e-08
3542 1.64228044528536e-08
3543 1.64211121855273e-08
3544 1.64193635656118e-08
3545 1.64176930626814e-08
3546 1.64159533196928e-08
3547 1.64143012817697e-08
3548 1.64125480164035e-08
3549 1.64108751404268e-08
3550 1.64091404541156e-08
3551 1.64074417980387e-08
3552 1.64056804778379e-08
3553 1.64039684844852e-08
3554 1.64023426545701e-08
3555 1.64005404542367e-08
3556 1.63988975057372e-08
3557 1.63972362459408e-08
3558 1.63954616885509e-08
3559 1.63937704417416e-08
3560 1.63920925377037e-08
3561 1.63903751321803e-08
3562 1.63886843570493e-08
3563 1.63869838775421e-08
3564 1.6385306811445e-08
3565 1.63835892020292e-08
3566 1.63818987101161e-08
3567 1.63802223637488e-08
3568 1.63785175694431e-08
3569 1.63768246793372e-08
3570 1.63751419298697e-08
3571 1.63734207642152e-08
3572 1.63717668431318e-08
3573 1.63700984752435e-08
3574 1.63683701186468e-08
3575 1.63666900342419e-08
3576 1.63649706546587e-08
3577 1.63633285459042e-08
3578 1.63616204701678e-08
3579 1.63599188768293e-08
3580 1.63582770767723e-08
3581 1.63566094147638e-08
3582 1.63548940136649e-08
3583 1.63532106733366e-08
3584 1.63515744780185e-08
3585 1.63498875067059e-08
3586 1.63481862216763e-08
3587 1.63464875531927e-08
3588 1.6344811299196e-08
3589 1.63431544203119e-08
3590 1.6341499838618e-08
3591 1.6339821595851e-08
3592 1.63381536321394e-08
3593 1.63364737358618e-08
3594 1.63347928918978e-08
3595 1.63331545338374e-08
3596 1.63314544880666e-08
3597 1.63297979131061e-08
3598 1.63281020742756e-08
3599 1.63264495362248e-08
3600 1.6324833125192e-08
3601 1.63231874781289e-08
3602 1.63215220830848e-08
3603 1.63198051000291e-08
3604 1.63181413226632e-08
3605 1.63165214843719e-08
3606 1.63148394305979e-08
3607 1.63131877979616e-08
3608 1.63115234273759e-08
3609 1.63098675780016e-08
3610 1.63082398915992e-08
3611 1.63065814265118e-08
3612 1.6304905845782e-08
3613 1.63032176770661e-08
3614 1.63016115771186e-08
3615 1.62999484205617e-08
3616 1.62982532193601e-08
3617 1.62966085693883e-08
3618 1.62950251424987e-08
3619 1.62933708636459e-08
3620 1.62917047372424e-08
3621 1.62900645004349e-08
3622 1.6288468061898e-08
3623 1.6286762107437e-08
3624 1.62851629908756e-08
3625 1.6283471223788e-08
3626 1.62818372038409e-08
3627 1.62802018885688e-08
3628 1.62785739965532e-08
3629 1.62768751218456e-08
3630 1.62752344557982e-08
3631 1.62736110382311e-08
3632 1.62719251098065e-08
3633 1.62703317170276e-08
3634 1.62686966309888e-08
3635 1.62670687871291e-08
3636 1.62654467509293e-08
3637 1.62638055327402e-08
3638 1.62621523935591e-08
3639 1.62605552998241e-08
3640 1.6258918792228e-08
3641 1.62573537875466e-08
3642 1.62556877777165e-08
3643 1.62540659185972e-08
3644 1.62524148267229e-08
3645 1.62508304895337e-08
3646 1.62492416571902e-08
3647 1.62475534683237e-08
3648 1.62459278227611e-08
3649 1.62443062721451e-08
3650 1.62427039016311e-08
3651 1.62410902535826e-08
3652 1.62394251134734e-08
3653 1.62378399504448e-08
3654 1.62362044635878e-08
3655 1.62346233110455e-08
3656 1.6232984166753e-08
3657 1.62313428352101e-08
3658 1.62297827696334e-08
3659 1.62281406979881e-08
3660 1.62265330446443e-08
3661 1.62249309375306e-08
3662 1.62233422799085e-08
3663 1.62216755214273e-08
3664 1.62201017897545e-08
3665 1.62184887316508e-08
3666 1.62168252181849e-08
3667 1.62152129862536e-08
3668 1.6213670201437e-08
3669 1.62120673041499e-08
3670 1.62104234206761e-08
3671 1.62088255609705e-08
3672 1.62071847853995e-08
3673 1.62056093135632e-08
3674 1.62040485502113e-08
3675 1.62024257255866e-08
3676 1.62008491859378e-08
3677 1.61991916621251e-08
3678 1.61976238809425e-08
3679 1.61960839605013e-08
3680 1.61943889048222e-08
3681 1.61928623886032e-08
3682 1.61912465717617e-08
3683 1.61895974566673e-08
3684 1.61880688566707e-08
3685 1.61863907207627e-08
3686 1.61848490712602e-08
3687 1.61832605956869e-08
3688 1.61816243678392e-08
3689 1.61800650960164e-08
3690 1.61784594589487e-08
3691 1.61768883565061e-08
3692 1.61752861020104e-08
3693 1.61736613884245e-08
3694 1.61720930207943e-08
3695 1.61705286081393e-08
3696 1.61689151278177e-08
3697 1.61673334835855e-08
3698 1.61657951125438e-08
3699 1.61641730985485e-08
3700 1.61626139547344e-08
3701 1.61609683003716e-08
3702 1.61594324373793e-08
3703 1.61578522811789e-08
3704 1.61562811129279e-08
3705 1.61546908625798e-08
3706 1.61530992533188e-08
3707 1.61515612736862e-08
3708 1.6149973393248e-08
3709 1.61483721066447e-08
3710 1.61467830916973e-08
3711 1.61452278427399e-08
3712 1.61436693845718e-08
3713 1.61420968173287e-08
3714 1.61405415241844e-08
3715 1.61389668391632e-08
3716 1.61373517548469e-08
3717 1.61357526786421e-08
3718 1.61342117785124e-08
3719 1.61326184126565e-08
3720 1.61310261222181e-08
3721 1.61295499716552e-08
3722 1.61279848506224e-08
3723 1.61263886452601e-08
3724 1.61248312782469e-08
3725 1.61232065830075e-08
3726 1.61216310801682e-08
3727 1.61201017574442e-08
3728 1.6118540855925e-08
3729 1.61169434875486e-08
3730 1.61154050084544e-08
3731 1.61138329624888e-08
3732 1.61123158073928e-08
3733 1.61107721775855e-08
3734 1.61091802337032e-08
3735 1.61076111190872e-08
3736 1.61060423221615e-08
3737 1.61044652800257e-08
3738 1.61029292203019e-08
3739 1.61013743773253e-08
3740 1.6099844554196e-08
3741 1.60982240311747e-08
3742 1.60966849122313e-08
3743 1.60951540054133e-08
3744 1.60936176319959e-08
3745 1.60920640593087e-08
3746 1.60904715151844e-08
3747 1.60889276118736e-08
3748 1.60874025114666e-08
3749 1.60858393787322e-08
3750 1.60843006445643e-08
3751 1.60827088515891e-08
3752 1.60812337201555e-08
3753 1.60796409546249e-08
3754 1.60781230378604e-08
3755 1.60765427610066e-08
3756 1.60749869701493e-08
3757 1.60734134537766e-08
3758 1.60719023203915e-08
3759 1.60704021893165e-08
3760 1.60688063095271e-08
3761 1.60672622080693e-08
3762 1.60657190537705e-08
3763 1.60642026709734e-08
3764 1.60626510939676e-08
3765 1.60611090874452e-08
3766 1.60596506164346e-08
3767 1.60580710328317e-08
3768 1.60565210085284e-08
3769 1.60550229916789e-08
3770 1.60534736511897e-08
3771 1.605193127166e-08
3772 1.60504170202691e-08
3773 1.60488395532787e-08
3774 1.60472415822732e-08
3775 1.60457187091678e-08
3776 1.604421604956e-08
3777 1.60426753758602e-08
3778 1.60411447828745e-08
3779 1.60395981599892e-08
3780 1.60380948388827e-08
3781 1.60365890818637e-08
3782 1.60350561940747e-08
3783 1.60334910355997e-08
3784 1.60319662199371e-08
3785 1.60303978602727e-08
3786 1.60289025588067e-08
3787 1.60273785088094e-08
3788 1.60258499581067e-08
3789 1.60243091310297e-08
3790 1.602278222404e-08
3791 1.60212384787695e-08
3792 1.60197677708251e-08
3793 1.60182603635428e-08
3794 1.60167257616528e-08
3795 1.6015220972887e-08
3796 1.60137031282592e-08
3797 1.60121624920295e-08
3798 1.60106455650011e-08
3799 1.600914387348e-08
3800 1.60076171165646e-08
3801 1.60060899815906e-08
3802 1.60045815476018e-08
3803 1.6003069576026e-08
3804 1.60015502072008e-08
3805 1.60000545812722e-08
3806 1.59985085955439e-08
3807 1.59970245924013e-08
3808 1.59955167458314e-08
3809 1.59939537181519e-08
3810 1.59924729524474e-08
3811 1.59909155772464e-08
3812 1.59894654323534e-08
3813 1.59879205624824e-08
3814 1.59863927794157e-08
3815 1.59849082373431e-08
3816 1.59833626678652e-08
3817 1.59818891721231e-08
3818 1.59803516759927e-08
3819 1.59788784916126e-08
3820 1.59773740289748e-08
3821 1.59758638012542e-08
3822 1.59743695645753e-08
3823 1.59728512210688e-08
3824 1.59713470510026e-08
3825 1.59697848577112e-08
3826 1.59683395828947e-08
3827 1.59667637134042e-08
3828 1.59652958667267e-08
3829 1.59638022923514e-08
3830 1.59622876158005e-08
3831 1.59607599803935e-08
3832 1.59592803331832e-08
3833 1.59578016360462e-08
3834 1.5956314405513e-08
3835 1.59548325347758e-08
3836 1.59532942646257e-08
3837 1.59518127398339e-08
3838 1.59503218956081e-08
3839 1.5948803455984e-08
3840 1.59473123916842e-08
3841 1.59457847883349e-08
3842 1.59443550177552e-08
3843 1.59428732978417e-08
3844 1.5941326019453e-08
3845 1.59398521375198e-08
3846 1.5938380888092e-08
3847 1.59368639717772e-08
3848 1.59353765146752e-08
3849 1.59339125581359e-08
3850 1.59324510942416e-08
3851 1.59310238049271e-08
3852 1.59294700854129e-08
3853 1.59279844582083e-08
3854 1.59264716210195e-08
3855 1.5925017038676e-08
3856 1.5923505548826e-08
3857 1.59220040041319e-08
3858 1.59205639919113e-08
3859 1.59190725902425e-08
3860 1.59176113955772e-08
3861 1.59161213869885e-08
3862 1.5914612851109e-08
3863 1.59131540148072e-08
3864 1.59116911744861e-08
3865 1.5910190558327e-08
3866 1.59086661971342e-08
3867 1.59072209831301e-08
3868 1.59057405298424e-08
3869 1.59042861413994e-08
3870 1.59027987612359e-08
3871 1.59013269624697e-08
3872 1.5899844968581e-08
3873 1.58983827040493e-08
3874 1.5896912627239e-08
3875 1.58954169194037e-08
3876 1.58939587100726e-08
3877 1.58924973113761e-08
3878 1.58909822416675e-08
3879 1.58895015958393e-08
3880 1.58880425827901e-08
3881 1.58866320203732e-08
3882 1.58851063769283e-08
3883 1.58836355811931e-08
3884 1.58821939789666e-08
3885 1.58807225437152e-08
3886 1.58792696564047e-08
3887 1.5877778862361e-08
3888 1.58763017828745e-08
3889 1.58748142091991e-08
3890 1.58733203759198e-08
3891 1.58719184736589e-08
3892 1.58704187984415e-08
3893 1.58690024224806e-08
3894 1.58675059747648e-08
3895 1.58660445176428e-08
3896 1.58645937117785e-08
3897 1.58631008824184e-08
3898 1.58616321819793e-08
3899 1.58601857145613e-08
3900 1.585873485907e-08
3901 1.58572677863289e-08
3902 1.58557809847859e-08
3903 1.5854352106548e-08
3904 1.58528969839422e-08
3905 1.58514367824825e-08
3906 1.58499717412275e-08
3907 1.5848521752071e-08
3908 1.58470441624647e-08
3909 1.58455904042676e-08
3910 1.58441015991329e-08
3911 1.58427185616983e-08
3912 1.58412593999901e-08
3913 1.5839794522382e-08
3914 1.5838343435326e-08
3915 1.58368862152036e-08
3916 1.58354427277074e-08
3917 1.58339626192827e-08
3918 1.58324744606309e-08
3919 1.58310343146006e-08
3920 1.58295986071033e-08
3921 1.58281091707513e-08
3922 1.58266635825466e-08
3923 1.58252109920543e-08
3924 1.58237848120191e-08
3925 1.58223228653998e-08
3926 1.58208791838088e-08
3927 1.58194355330266e-08
3928 1.58179843206818e-08
3929 1.5816552838277e-08
3930 1.5815118534046e-08
3931 1.58136647642193e-08
3932 1.58122608695244e-08
3933 1.58107861898127e-08
3934 1.5809349824758e-08
3935 1.58079292349733e-08
3936 1.58064684602777e-08
3937 1.58050265104681e-08
3938 1.5803538533754e-08
3939 1.58021508675499e-08
3940 1.58006916707032e-08
3941 1.57992746332158e-08
3942 1.57977586358327e-08
3943 1.57963709479236e-08
3944 1.5794943025671e-08
3945 1.57935031284417e-08
3946 1.57921035441044e-08
3947 1.57905987833162e-08
3948 1.57891765735774e-08
3949 1.57877997847e-08
3950 1.57863419795734e-08
3951 1.57849373086938e-08
3952 1.57834596652695e-08
3953 1.57820311411927e-08
3954 1.57806440565789e-08
3955 1.57792133505252e-08
3956 1.57777420262961e-08
3957 1.57763394569299e-08
3958 1.57748750365117e-08
3959 1.57735012933369e-08
3960 1.57720407784889e-08
3961 1.57706571373983e-08
3962 1.57692011983068e-08
3963 1.57678149263762e-08
3964 1.57663810577136e-08
3965 1.57649621642109e-08
3966 1.57634931541584e-08
3967 1.57621368819527e-08
3968 1.57606823161238e-08
3969 1.5759262902898e-08
3970 1.57578497628208e-08
3971 1.57564547197953e-08
3972 1.57550421472086e-08
3973 1.57536215545651e-08
3974 1.57521465128652e-08
3975 1.57508048529154e-08
3976 1.57493636289696e-08
3977 1.57479407383587e-08
3978 1.5746506722758e-08
3979 1.57451269929276e-08
3980 1.57436989800808e-08
3981 1.57423111157018e-08
3982 1.57409111822271e-08
3983 1.57394461144655e-08
3984 1.57380055279821e-08
3985 1.5736588549975e-08
3986 1.57352366276231e-08
3987 1.57338130588325e-08
3988 1.57323946639587e-08
3989 1.57309749719836e-08
3990 1.57295601282692e-08
3991 1.57281382600627e-08
3992 1.57267579998233e-08
3993 1.57253127070767e-08
3994 1.57239394185105e-08
3995 1.57225268885008e-08
3996 1.57211034164939e-08
3997 1.57197269488318e-08
3998 1.57182781993503e-08
3999 1.5716898391388e-08
4000 1.57155152866739e-08
4001 1.57140721709914e-08
4002 1.57126815291952e-08
4003 1.57112744690047e-08
4004 1.5709939345715e-08
4005 1.57084679119623e-08
4006 1.57071076212489e-08
4007 1.57056696561686e-08
4008 1.5704312134518e-08
4009 1.5702891917907e-08
4010 1.57014593356875e-08
4011 1.57000945679719e-08
4012 1.56987190298441e-08
4013 1.56973143061179e-08
4014 1.56959240348031e-08
4015 1.56945078662041e-08
4016 1.56931186315878e-08
4017 1.56917199502782e-08
4018 1.56903852340518e-08
4019 1.5688998350083e-08
4020 1.56875493753927e-08
4021 1.56861424930321e-08
4022 1.56847518447134e-08
4023 1.56834014254092e-08
4024 1.56819943926412e-08
4025 1.56805998896559e-08
4026 1.56792298515174e-08
4027 1.56777993476354e-08
4028 1.56764210362537e-08
4029 1.56750729166655e-08
4030 1.56736587899609e-08
4031 1.5672283425916e-08
4032 1.56708699835251e-08
4033 1.56694997132112e-08
4034 1.56680918481367e-08
4035 1.56667122412912e-08
4036 1.56653664857565e-08
4037 1.56640199746871e-08
4038 1.56625838144964e-08
4039 1.5661218920493e-08
4040 1.56598636457672e-08
4041 1.56584416225181e-08
4042 1.56570395326294e-08
4043 1.56556824137954e-08
4044 1.56542989126207e-08
4045 1.56529133217564e-08
4046 1.56514988032264e-08
4047 1.56501450853108e-08
4048 1.56487486419055e-08
4049 1.56473824767245e-08
4050 1.56460397007507e-08
4051 1.56445631028235e-08
4052 1.56432578345689e-08
4053 1.56418927771129e-08
4054 1.56404616557526e-08
4055 1.56390893177871e-08
4056 1.56378122917744e-08
4057 1.56363882397592e-08
4058 1.56349569403191e-08
4059 1.56335976332633e-08
4060 1.56322323253411e-08
4061 1.56308911050396e-08
4062 1.56294593358086e-08
4063 1.56281087030641e-08
4064 1.56267272023447e-08
4065 1.56253535891493e-08
4066 1.56239820132409e-08
4067 1.5622617092953e-08
4068 1.56212780451359e-08
4069 1.56198771290583e-08
4070 1.5618517110072e-08
4071 1.56172254028308e-08
4072 1.56157986263283e-08
4073 1.56144432315042e-08
4074 1.56130787200004e-08
4075 1.56116940622231e-08
4076 1.56102818399118e-08
4077 1.56089374303836e-08
4078 1.56076087040602e-08
4079 1.56061754048586e-08
4080 1.56048116054519e-08
4081 1.5603473822956e-08
4082 1.56021158314312e-08
4083 1.56007338174002e-08
4084 1.55993681150435e-08
4085 1.55980738751615e-08
4086 1.55966523085749e-08
4087 1.55953277781729e-08
4088 1.55938969813862e-08
4089 1.55925595564099e-08
4090 1.5591241678381e-08
4091 1.5589833607832e-08
4092 1.55884766356029e-08
4093 1.55871609791858e-08
4094 1.55857715558871e-08
4095 1.55844102203706e-08
4096 1.55830683017388e-08
4097 1.55816934998831e-08
4098 1.55803260627196e-08
4099 1.55789924234928e-08
4100 1.55776313619516e-08
4101 1.55763131556297e-08
4102 1.55749819273909e-08
4103 1.55736048505217e-08
4104 1.55722123619528e-08
4105 1.5570852374136e-08
4106 1.55694816370566e-08
4107 1.55681519020956e-08
4108 1.55668437273049e-08
4109 1.55654414727424e-08
4110 1.55640922794853e-08
4111 1.55627430402372e-08
4112 1.55614133943438e-08
4113 1.55601058424715e-08
4114 1.55587426139969e-08
4115 1.55573752223526e-08
4116 1.55560388846454e-08
4117 1.55546654957428e-08
4118 1.55532912226308e-08
4119 1.55519477710586e-08
4120 1.55506316141529e-08
4121 1.55492317123473e-08
4122 1.55479146754511e-08
4123 1.55465507691854e-08
4124 1.55452260277855e-08
4125 1.55438627953974e-08
4126 1.55425524485386e-08
4127 1.55411890779278e-08
4128 1.55398568452148e-08
4129 1.55384861794394e-08
4130 1.55371610960908e-08
4131 1.55357803661993e-08
4132 1.55344315805606e-08
4133 1.55331331646857e-08
4134 1.55317571834401e-08
4135 1.55304214496665e-08
4136 1.55290733083457e-08
4137 1.55276738347809e-08
4138 1.55263941597483e-08
4139 1.55250362680881e-08
4140 1.55237453343948e-08
4141 1.55223923971604e-08
4142 1.55210376973081e-08
4143 1.55197099337701e-08
4144 1.55183745091381e-08
4145 1.55170193026077e-08
4146 1.55157364885028e-08
4147 1.55143798212798e-08
4148 1.55130524624181e-08
4149 1.55117267586768e-08
4150 1.55104160246e-08
4151 1.55090425608684e-08
4152 1.55077060380315e-08
4153 1.55064140856531e-08
4154 1.55050394642087e-08
4155 1.5503776313458e-08
4156 1.55023785445574e-08
4157 1.55010780883424e-08
4158 1.54997179993288e-08
4159 1.54984343547493e-08
4160 1.54970614247296e-08
4161 1.54957743828121e-08
4162 1.54943550031428e-08
4163 1.54931242813683e-08
4164 1.5491806091783e-08
4165 1.54905244772741e-08
4166 1.54891261862355e-08
4167 1.54877983436774e-08
4168 1.548652714084e-08
4169 1.54851734952832e-08
4170 1.54838595364526e-08
4171 1.54825455485896e-08
4172 1.54812124109338e-08
4173 1.54799083568746e-08
4174 1.5478593000301e-08
4175 1.54772271555859e-08
4176 1.54759165327534e-08
4177 1.54745882005869e-08
4178 1.54733194270285e-08
4179 1.54719911290291e-08
4180 1.54706360929457e-08
4181 1.54693223334557e-08
4182 1.54680521283479e-08
4183 1.54667352860183e-08
4184 1.54653874475941e-08
4185 1.54640795138328e-08
4186 1.54627810306229e-08
4187 1.54614626540206e-08
4188 1.54601686625233e-08
4189 1.54588363071861e-08
4190 1.54575418263303e-08
4191 1.54562373499145e-08
4192 1.54549463086129e-08
4193 1.54535872011197e-08
4194 1.54523085057756e-08
4195 1.54510581658329e-08
4196 1.54497186721225e-08
4197 1.54484065878202e-08
4198 1.54470838344967e-08
4199 1.54457791207707e-08
4200 1.54444143320165e-08
4201 1.54431356001461e-08
4202 1.54418501809306e-08
4203 1.54405440068173e-08
4204 1.54392040399853e-08
4205 1.54379014412342e-08
4206 1.54365568676151e-08
4207 1.54353134445007e-08
4208 1.54339733292597e-08
4209 1.54327080510441e-08
4210 1.54313879972834e-08
4211 1.54300700552779e-08
4212 1.54288087721721e-08
4213 1.5427493185588e-08
4214 1.54261467002759e-08
4215 1.54248789236977e-08
4216 1.542357137993e-08
4217 1.54222382101887e-08
4218 1.54209632385327e-08
4219 1.54196920878202e-08
4220 1.54183701011612e-08
4221 1.54170913540252e-08
4222 1.54157633033836e-08
4223 1.54145122637228e-08
4224 1.54131628740672e-08
4225 1.54118586085938e-08
4226 1.54105906022273e-08
4227 1.54092962667829e-08
4228 1.5407974289422e-08
4229 1.54066772782013e-08
4230 1.54053966487655e-08
4231 1.54041521592541e-08
4232 1.5402804976139e-08
4233 1.54014866502183e-08
4234 1.54002111716622e-08
4235 1.53988858706544e-08
4236 1.53976321838611e-08
4237 1.53963327628737e-08
4238 1.53950739758546e-08
4239 1.53937225446377e-08
4240 1.53924589046672e-08
4241 1.53911344896729e-08
4242 1.53898643456274e-08
4243 1.53885841894519e-08
4244 1.53872928188026e-08
4245 1.53860298134911e-08
4246 1.5384722575229e-08
4247 1.53834161202016e-08
4248 1.53821383299391e-08
4249 1.53808549061274e-08
4250 1.53795235151022e-08
4251 1.53782415789894e-08
4252 1.53770103158424e-08
4253 1.53756774571578e-08
4254 1.53744089611552e-08
4255 1.53731437009808e-08
4256 1.53717961108024e-08
4257 1.53705413395988e-08
4258 1.53692575662889e-08
4259 1.53680148742563e-08
4260 1.53666704161282e-08
4261 1.5365410419993e-08
4262 1.53641567082197e-08
4263 1.53629034754521e-08
4264 1.53616219009389e-08
4265 1.5360327190822e-08
4266 1.53590284515948e-08
4267 1.53577356744039e-08
4268 1.53564649057192e-08
4269 1.53551819106479e-08
4270 1.53539016226334e-08
4271 1.53526043038255e-08
4272 1.53513830496443e-08
4273 1.53500342420787e-08
4274 1.53488133934898e-08
4275 1.5347510524899e-08
4276 1.53462321906273e-08
4277 1.5344977243148e-08
4278 1.53436357022974e-08
4279 1.53424014372738e-08
4280 1.53411107559065e-08
4281 1.53398731041476e-08
4282 1.53385585107968e-08
4283 1.53372882825409e-08
4284 1.53359910693429e-08
4285 1.53347668112591e-08
4286 1.53334181552389e-08
4287 1.53321727319466e-08
4288 1.53309206573637e-08
4289 1.53296508526024e-08
4290 1.53283853809028e-08
4291 1.53271005526867e-08
4292 1.53258153499924e-08
4293 1.53245238690702e-08
4294 1.53232758232369e-08
4295 1.53219626791157e-08
4296 1.53206915797233e-08
4297 1.5319480146242e-08
4298 1.53181900338084e-08
4299 1.53169112050433e-08
4300 1.53156321627546e-08
4301 1.5314403354505e-08
4302 1.53131231118431e-08
4303 1.53118265444896e-08
4304 1.53105987202862e-08
4305 1.53093134363791e-08
4306 1.53080392096272e-08
4307 1.53068117764166e-08
4308 1.5305524661724e-08
4309 1.5304320220566e-08
4310 1.53030505969653e-08
4311 1.53017412069245e-08
4312 1.53004970411541e-08
4313 1.52992072229574e-08
4314 1.52979774263318e-08
4315 1.52967533605108e-08
4316 1.52954585874443e-08
4317 1.52942420717783e-08
4318 1.52929404562685e-08
4319 1.52916749176502e-08
4320 1.52904093398409e-08
4321 1.52891635131214e-08
4322 1.52878926597544e-08
4323 1.52866172347665e-08
4324 1.52853730646108e-08
4325 1.52840761156459e-08
4326 1.52828186245624e-08
4327 1.52815952327301e-08
4328 1.52803353230813e-08
4329 1.52790953354409e-08
4330 1.52778324687686e-08
4331 1.52765972444291e-08
4332 1.52753318415355e-08
4333 1.52740735337165e-08
4334 1.52728246390621e-08
4335 1.52715476836318e-08
4336 1.52703081788552e-08
4337 1.52690515642928e-08
4338 1.52677774490628e-08
4339 1.52665744417857e-08
4340 1.52653273903791e-08
4341 1.5264035078455e-08
4342 1.52628177539638e-08
4343 1.5261535447425e-08
4344 1.5260303427661e-08
4345 1.52590740300496e-08
4346 1.52578196635778e-08
4347 1.52565194204701e-08
4348 1.52553151210599e-08
4349 1.52540379630417e-08
4350 1.52527927279322e-08
4351 1.525157232668e-08
4352 1.52502887771633e-08
4353 1.52490534786887e-08
4354 1.524776097242e-08
4355 1.52465512720246e-08
4356 1.52453241998862e-08
4357 1.5244088407973e-08
4358 1.52428335325194e-08
4359 1.52415440563547e-08
4360 1.52403338534446e-08
4361 1.52390877028452e-08
4362 1.52378107354523e-08
4363 1.5236543388919e-08
4364 1.52352971067859e-08
4365 1.5234138867587e-08
4366 1.52328697995974e-08
4367 1.52316001272579e-08
4368 1.52304005813786e-08
4369 1.52291214952638e-08
4370 1.52278822706797e-08
4371 1.5226672883506e-08
4372 1.52254285389064e-08
4373 1.52241381630169e-08
4374 1.52229393660663e-08
4375 1.52216192346455e-08
4376 1.52204534135947e-08
4377 1.52192538602214e-08
4378 1.52180040787764e-08
4379 1.52167114201851e-08
4380 1.52155189145076e-08
4381 1.52142851036485e-08
4382 1.5213008150744e-08
4383 1.52118168993687e-08
4384 1.52105625614574e-08
4385 1.52093175472823e-08
4386 1.52081407471527e-08
4387 1.5206871989859e-08
4388 1.52056591110339e-08
4389 1.52044091532577e-08
4390 1.52031852252432e-08
4391 1.52019278726878e-08
4392 1.52006712607899e-08
4393 1.51994614682738e-08
4394 1.51982586672761e-08
4395 1.51970088407005e-08
4396 1.51958104269989e-08
4397 1.51945819640842e-08
4398 1.51932921119979e-08
4399 1.519210794107e-08
4400 1.51909043325793e-08
4401 1.51896576740251e-08
4402 1.51884436992433e-08
4403 1.51871682733673e-08
4404 1.51859407482025e-08
4405 1.51847128878058e-08
4406 1.51834992605793e-08
4407 1.51822933121271e-08
4408 1.51811138586755e-08
4409 1.51798321954e-08
4410 1.5178603797017e-08
4411 1.5177373619335e-08
4412 1.51761314765575e-08
4413 1.51748897333492e-08
4414 1.51736551468329e-08
4415 1.51724772133044e-08
4416 1.51712423787642e-08
4417 1.51700558181367e-08
4418 1.51687963364233e-08
4419 1.516754633063e-08
4420 1.51663375189959e-08
4421 1.51651117408502e-08
4422 1.51638989189795e-08
4423 1.51626961279738e-08
4424 1.51614365510033e-08
4425 1.51602602992684e-08
4426 1.51590408247748e-08
4427 1.51577960719462e-08
4428 1.51566066350917e-08
4429 1.51553674535287e-08
4430 1.51542049016962e-08
4431 1.51529228659408e-08
4432 1.515172805161e-08
4433 1.5150478413356e-08
4434 1.5149273790982e-08
4435 1.51480824399364e-08
4436 1.51468527448684e-08
4437 1.51456538608485e-08
4438 1.5144413160556e-08
4439 1.5143226333364e-08
4440 1.51419875837888e-08
4441 1.51407910085888e-08
4442 1.5139587659585e-08
4443 1.51383415965545e-08
4444 1.51371532798594e-08
4445 1.51359696372311e-08
4446 1.51347555809589e-08
4447 1.51335376599449e-08
4448 1.51323170881623e-08
4449 1.5131116638728e-08
4450 1.51299187062248e-08
4451 1.51287177308501e-08
4452 1.5127508060625e-08
4453 1.51262955782327e-08
4454 1.51251240586292e-08
4455 1.51239516018031e-08
4456 1.51226688257233e-08
4457 1.51215150692441e-08
4458 1.51202789655802e-08
4459 1.511912201696e-08
4460 1.51179021064263e-08
4461 1.51166857968454e-08
4462 1.511545256766e-08
4463 1.51142837885865e-08
4464 1.51130710357439e-08
4465 1.51118739763478e-08
4466 1.51106264786593e-08
4467 1.51094460863477e-08
4468 1.51082491166299e-08
4469 1.51069951025706e-08
4470 1.51058451071939e-08
4471 1.51045553417606e-08
4472 1.51034213855816e-08
4473 1.51022233129128e-08
4474 1.5101030663961e-08
4475 1.50998494593824e-08
4476 1.50986809727416e-08
4477 1.50974489808164e-08
4478 1.50962144358779e-08
4479 1.50950115133941e-08
4480 1.50938131210643e-08
4481 1.50926424174469e-08
4482 1.50914301451643e-08
4483 1.50902507086437e-08
4484 1.50890659623981e-08
4485 1.50878516942954e-08
4486 1.50866351737167e-08
4487 1.50854372654996e-08
4488 1.50842720106903e-08
4489 1.50830403290725e-08
4490 1.50818191132773e-08
4491 1.50806410342819e-08
4492 1.50794505522356e-08
4493 1.50782425069607e-08
4494 1.50770525992883e-08
4495 1.50758979127752e-08
4496 1.50746285994263e-08
4497 1.50734647147432e-08
4498 1.50722692268679e-08
4499 1.50710797900411e-08
4500 1.50698659407977e-08
4501 1.5068684891012e-08
4502 1.50675122831401e-08
4503 1.50663381340566e-08
4504 1.50650881208247e-08
4505 1.50639465539004e-08
4506 1.50627534008241e-08
4507 1.50615479415361e-08
4508 1.50603605257593e-08
4509 1.50591832096814e-08
4510 1.5058001208379e-08
4511 1.50568340512303e-08
4512 1.50555913508155e-08
4513 1.50543782401757e-08
4514 1.50532040701923e-08
4515 1.50520166674606e-08
4516 1.5050839461489e-08
4517 1.5049624036978e-08
4518 1.50484461228506e-08
4519 1.50472846519312e-08
4520 1.50460856595813e-08
4521 1.50449040390022e-08
4522 1.5043710742374e-08
4523 1.50425819547273e-08
4524 1.50413839402064e-08
4525 1.50401686347945e-08
4526 1.50390231462849e-08
4527 1.50377927325418e-08
4528 1.50366395842705e-08
4529 1.50354509097506e-08
4530 1.50343067113756e-08
4531 1.50331251776159e-08
4532 1.50319114372022e-08
4533 1.50307714900932e-08
4534 1.50295552664992e-08
4535 1.50283643470805e-08
4536 1.50271762177911e-08
4537 1.50260359321752e-08
4538 1.50248563787758e-08
4539 1.50237111296858e-08
4540 1.50225378457436e-08
4541 1.50212897947199e-08
4542 1.50201613913215e-08
4543 1.50189684894886e-08
4544 1.5017792096006e-08
4545 1.50166289840659e-08
4546 1.50154972019534e-08
4547 1.50142436486367e-08
4548 1.50130848136643e-08
4549 1.50118910254571e-08
4550 1.50106925837779e-08
4551 1.50095327079713e-08
4552 1.50083824745628e-08
4553 1.50071992428835e-08
4554 1.50060609462876e-08
4555 1.50049045489486e-08
4556 1.50037548632687e-08
4557 1.50025474749682e-08
4558 1.50013620027201e-08
4559 1.50001814397116e-08
4560 1.49990340386763e-08
4561 1.49978232077685e-08
4562 1.49966530989809e-08
4563 1.49954964677734e-08
4564 1.49943524563045e-08
4565 1.49931741101617e-08
4566 1.49919751274985e-08
4567 1.49908355207007e-08
4568 1.4989671204696e-08
4569 1.49885320480103e-08
4570 1.49872926467898e-08
4571 1.49861717527455e-08
4572 1.49850074591673e-08
4573 1.4983815532027e-08
4574 1.49826866400748e-08
4575 1.4981495643801e-08
4576 1.49803597428999e-08
4577 1.49791972626212e-08
4578 1.49780322382942e-08
4579 1.49768258655147e-08
4580 1.49756943679524e-08
4581 1.4974515019861e-08
4582 1.49733516708606e-08
4583 1.49721991133944e-08
4584 1.49710844423168e-08
4585 1.49699500409939e-08
4586 1.4968722577946e-08
4587 1.49675799080984e-08
4588 1.49664468905575e-08
4589 1.49653257892624e-08
4590 1.49641477969475e-08
4591 1.49630254179245e-08
4592 1.49618360278381e-08
4593 1.49606623684462e-08
4594 1.49595480760101e-08
4595 1.49583854637536e-08
4596 1.49572012243526e-08
4597 1.49560368778168e-08
4598 1.49549302000396e-08
4599 1.49537128482646e-08
4600 1.49525794869432e-08
4601 1.49514671468065e-08
4602 1.49503242777849e-08
4603 1.49491485338937e-08
4604 1.49479714201817e-08
4605 1.49468290559784e-08
4606 1.49456813428594e-08
4607 1.49445098854839e-08
4608 1.49433411646693e-08
4609 1.49422091984541e-08
4610 1.49410873549194e-08
4611 1.4939978871531e-08
4612 1.49387823589198e-08
4613 1.49376357502229e-08
4614 1.49365160085901e-08
4615 1.49353107228856e-08
4616 1.49341711558337e-08
4617 1.49330595241859e-08
4618 1.4931926876488e-08
4619 1.49307609033644e-08
4620 1.49295944938399e-08
4621 1.49284686453144e-08
4622 1.49273016288309e-08
4623 1.49261612477358e-08
4624 1.49250691819092e-08
4625 1.49239451827099e-08
4626 1.49227430522869e-08
4627 1.4921602761564e-08
4628 1.4920483975861e-08
4629 1.49193460431407e-08
4630 1.49182313972096e-08
4631 1.4917086197247e-08
4632 1.49159221728978e-08
4633 1.49148214262373e-08
4634 1.49136621545032e-08
4635 1.4912528565808e-08
4636 1.49113516196564e-08
4637 1.49102199619167e-08
4638 1.49091398574108e-08
4639 1.49080019811731e-08
4640 1.49068613621017e-08
4641 1.49057280971199e-08
4642 1.49046113921691e-08
4643 1.49034439692608e-08
4644 1.49023149040028e-08
4645 1.4901172802978e-08
4646 1.49000665527754e-08
4647 1.48989472227301e-08
4648 1.48978528729526e-08
4649 1.48966868419032e-08
4650 1.48956035969094e-08
4651 1.48944211293411e-08
4652 1.4893293465601e-08
4653 1.48921596139495e-08
4654 1.48910673448688e-08
4655 1.48899109901901e-08
4656 1.4888739532759e-08
4657 1.48876561859301e-08
4658 1.48865486893635e-08
4659 1.48853659796833e-08
4660 1.48842968566987e-08
4661 1.48830792281729e-08
4662 1.48820253381698e-08
4663 1.48808642742859e-08
4664 1.48797270556611e-08
4665 1.4878652739192e-08
4666 1.48775004118196e-08
4667 1.48763707921995e-08
4668 1.4875216108462e-08
4669 1.48740966939842e-08
4670 1.48729962141936e-08
4671 1.48718395447112e-08
4672 1.48707314818197e-08
4673 1.48695883914751e-08
4674 1.48684709582458e-08
4675 1.48673646619413e-08
4676 1.48662123184706e-08
4677 1.48651260665489e-08
4678 1.48639840286402e-08
4679 1.48628566749853e-08
4680 1.48618083544716e-08
4681 1.48606435206866e-08
4682 1.48595117653028e-08
4683 1.48583956059711e-08
4684 1.48573527516127e-08
4685 1.4856154504278e-08
4686 1.48550539815495e-08
4687 1.48539042427454e-08
4688 1.4852824608419e-08
4689 1.48517104382462e-08
4690 1.48506337447896e-08
4691 1.48494667156496e-08
4692 1.48483412583389e-08
4693 1.48472309422498e-08
4694 1.48461148535561e-08
4695 1.48450431687042e-08
4696 1.48439248180976e-08
4697 1.48427836291487e-08
4698 1.484172243707e-08
4699 1.48405577612698e-08
4700 1.48394572054844e-08
4701 1.48384060132956e-08
4702 1.48372604746316e-08
4703 1.4836156121828e-08
4704 1.48350493191784e-08
4705 1.48338646899782e-08
4706 1.48328029528078e-08
4707 1.48317023055677e-08
4708 1.48306263376419e-08
4709 1.48294901871626e-08
4710 1.48283938600502e-08
4711 1.48272742097055e-08
4712 1.48261766968472e-08
4713 1.48250925754989e-08
4714 1.48239815277174e-08
4715 1.4822858930591e-08
4716 1.48217817235174e-08
4717 1.48206536057793e-08
4718 1.48195662187933e-08
4719 1.48184588581457e-08
4720 1.48173440910615e-08
4721 1.48162869863522e-08
4722 1.48151401513918e-08
4723 1.48140516931239e-08
4724 1.48129595716751e-08
4725 1.48118609710479e-08
4726 1.48107111527795e-08
4727 1.48096204534154e-08
4728 1.48085399075681e-08
4729 1.48074133187182e-08
4730 1.48062895851397e-08
4731 1.48052349347727e-08
4732 1.48041295721102e-08
4733 1.48030530723597e-08
4734 1.48019386080611e-08
4735 1.48008032862523e-08
4736 1.47997282409495e-08
4737 1.47986202112538e-08
4738 1.47975263398992e-08
4739 1.47964409815682e-08
4740 1.47953152916103e-08
4741 1.47941976865462e-08
4742 1.47931246447797e-08
4743 1.47920301049043e-08
4744 1.47908866290625e-08
4745 1.47898636129185e-08
4746 1.47887286574e-08
4747 1.47876496930932e-08
4748 1.47865666521585e-08
4749 1.47854662201907e-08
4750 1.47843932059022e-08
4751 1.4783249140643e-08
4752 1.47822034868572e-08
4753 1.47811480767646e-08
4754 1.47800650710794e-08
4755 1.47789539726162e-08
4756 1.47778656639508e-08
4757 1.47767256664932e-08
4758 1.47756613786965e-08
4759 1.47745962332246e-08
4760 1.47735140025584e-08
4761 1.47724423463225e-08
4762 1.47713766538438e-08
4763 1.4770226843458e-08
4764 1.47691551741491e-08
4765 1.4768055375064e-08
4766 1.47669637216852e-08
4767 1.4765915171383e-08
4768 1.476478480128e-08
4769 1.47637131706624e-08
4770 1.4762654572259e-08
4771 1.47615709968074e-08
4772 1.47604587652506e-08
4773 1.47593312005978e-08
4774 1.47583321369882e-08
4775 1.47571876767949e-08
4776 1.47561314442768e-08
4777 1.47550905606475e-08
4778 1.47539864809865e-08
4779 1.47528904025085e-08
4780 1.4751793582457e-08
4781 1.47507323941531e-08
4782 1.47496426499139e-08
4783 1.47485604387043e-08
4784 1.47474882429832e-08
4785 1.47464327628355e-08
4786 1.47453379658602e-08
4787 1.47442852160562e-08
4788 1.47432162458661e-08
4789 1.47420774150309e-08
4790 1.4741046478528e-08
4791 1.47399139381332e-08
4792 1.47388957413286e-08
4793 1.47378156979683e-08
4794 1.47367247363139e-08
4795 1.47356768674389e-08
4796 1.47346032367823e-08
4797 1.47334677831623e-08
4798 1.47324692008899e-08
4799 1.47313925726311e-08
4800 1.47303237772178e-08
4801 1.47292095455664e-08
4802 1.47281325617588e-08
4803 1.47270861318993e-08
4804 1.47259943721056e-08
4805 1.47249035142849e-08
4806 1.47237985174131e-08
4807 1.4722767467501e-08
4808 1.47217063320715e-08
4809 1.47206041296033e-08
4810 1.4719570495203e-08
4811 1.47184365383024e-08
4812 1.47174365393021e-08
4813 1.47163537557726e-08
4814 1.47152558291341e-08
4815 1.47142381833887e-08
4816 1.47131836471803e-08
4817 1.47120853997706e-08
4818 1.47110565190678e-08
4819 1.47099737149714e-08
4820 1.47089106385112e-08
4821 1.47078317617733e-08
4822 1.47067536871714e-08
4823 1.47057187943334e-08
4824 1.47046603839185e-08
4825 1.47035798545303e-08
4826 1.47025002584789e-08
4827 1.47014184650462e-08
4828 1.47003848645078e-08
4829 1.46993055112343e-08
4830 1.46982986579114e-08
4831 1.46972333598117e-08
4832 1.46961707531146e-08
4833 1.46951154617325e-08
4834 1.46940598435008e-08
4835 1.46930281794133e-08
4836 1.46919387193356e-08
4837 1.46908443417471e-08
4838 1.46897900485665e-08
4839 1.46887281235464e-08
4840 1.46877277127366e-08
4841 1.4686610761927e-08
4842 1.46855786837263e-08
4843 1.46844865785423e-08
4844 1.46834619867142e-08
4845 1.4682369307184e-08
4846 1.46813682508073e-08
4847 1.46803083875269e-08
4848 1.46792265811324e-08
4849 1.46781901882442e-08
4850 1.46771232353571e-08
4851 1.46760284642222e-08
4852 1.46749923395639e-08
4853 1.46739603089918e-08
4854 1.46729109512245e-08
4855 1.46718449546002e-08
4856 1.46707779810906e-08
4857 1.46697343157975e-08
4858 1.46686749936953e-08
4859 1.46675924780071e-08
4860 1.46665877906182e-08
4861 1.4665533882241e-08
4862 1.46644933246842e-08
4863 1.46634751291286e-08
4864 1.46623801186019e-08
4865 1.46612934408541e-08
4866 1.46603191578665e-08
4867 1.46592607814244e-08
4868 1.46581753751318e-08
4869 1.46571540668439e-08
4870 1.4656061966406e-08
4871 1.46550228629916e-08
4872 1.46540095757319e-08
4873 1.46528698543313e-08
4874 1.46518992787115e-08
4875 1.46508641142018e-08
4876 1.46498277516505e-08
4877 1.46487456915423e-08
4878 1.46477207391416e-08
4879 1.46466814171242e-08
4880 1.46456309149945e-08
4881 1.46446560459201e-08
4882 1.46435609833517e-08
4883 1.46425103162706e-08
4884 1.46414829557129e-08
4885 1.46404732598859e-08
4886 1.46393851343241e-08
4887 1.46383523009219e-08
4888 1.46373452537818e-08
4889 1.46362804723521e-08
4890 1.46352682129924e-08
4891 1.46341985736431e-08
4892 1.46331987044834e-08
4893 1.46321734671162e-08
4894 1.46311036395008e-08
4895 1.46300737825511e-08
4896 1.46290424999362e-08
4897 1.46279880754441e-08
4898 1.46269947249866e-08
4899 1.46259359221357e-08
4900 1.46249214258709e-08
4901 1.46238570936097e-08
4902 1.46228181456542e-08
4903 1.46218179661595e-08
4904 1.46207752567129e-08
4905 1.46197109358592e-08
4906 1.46187193814373e-08
4907 1.46176502281992e-08
4908 1.46166314620166e-08
4909 1.46156127301678e-08
4910 1.46145776399043e-08
4911 1.46135402167846e-08
4912 1.46125323166046e-08
4913 1.46114714300039e-08
4914 1.46104620847576e-08
4915 1.46094188941959e-08
4916 1.46084282707237e-08
4917 1.46073869423946e-08
4918 1.46063706454869e-08
4919 1.46053403507762e-08
4920 1.46042986055306e-08
4921 1.46033056501182e-08
4922 1.46022370933474e-08
4923 1.46012180017585e-08
4924 1.46001623060332e-08
4925 1.4599198120635e-08
4926 1.45981406799722e-08
4927 1.45970690752506e-08
4928 1.45960825578551e-08
4929 1.45950809298245e-08
4930 1.45940435090364e-08
4931 1.45930172553432e-08
4932 1.45919697404073e-08
4933 1.45910093293233e-08
4934 1.45899634126645e-08
4935 1.45889696286372e-08
4936 1.45879224138223e-08
4937 1.4586900669078e-08
4938 1.4585901166847e-08
4939 1.45848118524861e-08
4940 1.45838352013372e-08
4941 1.45828548417937e-08
4942 1.45818152875177e-08
4943 1.45808171496398e-08
4944 1.45797304487161e-08
4945 1.45787479395032e-08
4946 1.45777610900399e-08
4947 1.45766987431628e-08
4948 1.45757136856273e-08
4949 1.45746158214666e-08
4950 1.45736809001595e-08
4951 1.45726387540124e-08
4952 1.45716627821257e-08
4953 1.45705826685161e-08
4954 1.45695917827815e-08
4955 1.4568522847147e-08
4956 1.45675141897117e-08
4957 1.45665456310617e-08
4958 1.45655180115167e-08
4959 1.4564475562695e-08
4960 1.45635086585272e-08
4961 1.45624498072983e-08
4962 1.45614076886014e-08
4963 1.45604106464303e-08
4964 1.4559402933656e-08
4965 1.4558378033408e-08
4966 1.45573196892734e-08
4967 1.45563383361635e-08
4968 1.45553440069057e-08
4969 1.45543000951986e-08
4970 1.45533452053925e-08
4971 1.45522757899474e-08
4972 1.4551303654603e-08
4973 1.4550290099502e-08
4974 1.45492509883993e-08
4975 1.4548300217132e-08
4976 1.45472795794188e-08
4977 1.45462211637026e-08
4978 1.45451992795975e-08
4979 1.45442531004902e-08
4980 1.45432306489501e-08
4981 1.45422568342268e-08
4982 1.4541165015286e-08
4983 1.4540178752076e-08
4984 1.4539166119043e-08
4985 1.45382009981432e-08
4986 1.45371743118239e-08
4987 1.45361987010928e-08
4988 1.4535172975505e-08
4989 1.45341390836995e-08
4990 1.45331412014782e-08
4991 1.45321084536465e-08
4992 1.45311828428085e-08
4993 1.4530110367672e-08
4994 1.45291337905196e-08
4995 1.4528186106838e-08
4996 1.45271405275482e-08
4997 1.45261026538723e-08
4998 1.45251269340618e-08
4999 1.45241162393672e-08
};
\addlegendentry{Train}
\addplot [semithick, black]
table {%
0 0.000603360414970666
1 0.000187876590644009
2 0.00016039599722717
3 9.27206056076102e-05
4 3.90354289265815e-05
5 3.20424951496534e-05
6 3.00789197353879e-05
7 2.68109361059032e-05
8 2.05049218493514e-05
9 1.17782010420342e-05
10 7.02660918250331e-06
11 6.01097326580202e-06
12 5.75443664274644e-06
13 5.58837518838118e-06
14 5.43746727998951e-06
15 5.28328064319794e-06
16 5.11388179802452e-06
17 4.91868013341445e-06
18 4.68751704829629e-06
19 4.40844814875163e-06
20 4.06826075050049e-06
21 3.66007793672907e-06
22 3.17851277031878e-06
23 2.64586492448871e-06
24 2.13015391636873e-06
25 1.73643684320268e-06
26 1.50315486280306e-06
27 1.39020494316355e-06
28 1.34265428641811e-06
29 1.32207890146674e-06
30 1.30765988615167e-06
31 1.29580826069287e-06
32 1.28411329569644e-06
33 1.27337659705518e-06
34 1.26330667171715e-06
35 1.25386043237086e-06
36 1.24499592857319e-06
37 1.23667393836513e-06
38 1.22884694064851e-06
39 1.22145934255968e-06
40 1.21441235023667e-06
41 1.20765889732866e-06
42 1.20115100799012e-06
43 1.19484718652529e-06
44 1.18871048471192e-06
45 1.18272475901904e-06
46 1.17686283829244e-06
47 1.17108959329926e-06
48 1.16536841687775e-06
49 1.15967407054995e-06
50 1.15396881028573e-06
51 1.14823797048302e-06
52 1.14245347049291e-06
53 1.13666305878724e-06
54 1.1308763987472e-06
55 1.124912614614e-06
56 1.11888596165954e-06
57 1.11275619474327e-06
58 1.10644089090783e-06
59 1.09992356556177e-06
60 1.09321433683363e-06
61 1.0863207080547e-06
62 1.0792990678965e-06
63 1.07215214484313e-06
64 1.06485038031678e-06
65 1.05735841771093e-06
66 1.04964806268981e-06
67 1.04167986592074e-06
68 1.03337652035407e-06
69 1.02462104223378e-06
70 1.01530213214573e-06
71 1.00537647540477e-06
72 9.94851689029019e-07
73 9.83778704721772e-07
74 9.72263137555274e-07
75 9.60384227255417e-07
76 9.48167496517272e-07
77 9.35573837068659e-07
78 9.22539243219944e-07
79 9.09042512375891e-07
80 8.95084326657525e-07
81 8.80682421211532e-07
82 8.65827757934312e-07
83 8.50490096127032e-07
84 8.34629474866233e-07
85 8.18237481325923e-07
86 8.01338558176212e-07
87 7.83988014063652e-07
88 7.66315679356921e-07
89 7.48464628941292e-07
90 7.30559236217232e-07
91 7.12761675458751e-07
92 6.95319045007636e-07
93 6.78452295233001e-07
94 6.62395677863969e-07
95 6.47333251890814e-07
96 6.33491367807437e-07
97 6.20999969669356e-07
98 6.09920164151845e-07
99 6.00264797867567e-07
100 5.91989248732716e-07
101 5.8480372899794e-07
102 5.78345407120651e-07
103 5.72396629650029e-07
104 5.66834046367148e-07
105 5.61546414701297e-07
106 5.56443978894094e-07
107 5.51440791696223e-07
108 5.46483192920277e-07
109 5.4151473705133e-07
110 5.36512004600809e-07
111 5.31464138475712e-07
112 5.26358235219959e-07
113 5.21187587310123e-07
114 5.1595543482108e-07
115 5.10651091190084e-07
116 5.05284049268084e-07
117 4.99861982916627e-07
118 4.94380913096393e-07
119 4.88863918235438e-07
120 4.83303949749825e-07
121 4.77718799629656e-07
122 4.72104460413902e-07
123 4.66467497517442e-07
124 4.608221502167e-07
125 4.55160545698163e-07
126 4.4950130018151e-07
127 4.43850012743496e-07
128 4.38223253240722e-07
129 4.3264770965834e-07
130 4.27132221147986e-07
131 4.21678805651027e-07
132 4.16319551277411e-07
133 4.11062302418941e-07
134 4.05930791202991e-07
135 4.00946078116249e-07
136 3.96099267163663e-07
137 3.91350880590835e-07
138 3.86648963512926e-07
139 3.81937439897229e-07
140 3.77185955358073e-07
141 3.72397920500589e-07
142 3.67631940889623e-07
143 3.62903335826559e-07
144 3.58215572759946e-07
145 3.5359730077289e-07
146 3.49082625916708e-07
147 3.4470897958272e-07
148 3.4048957786581e-07
149 3.36424704983074e-07
150 3.32509927147839e-07
151 3.28743453792413e-07
152 3.25154758229473e-07
153 3.21767743116652e-07
154 3.18590082315495e-07
155 3.15586589749728e-07
156 3.12739160790443e-07
157 3.1001738420855e-07
158 3.07424244283538e-07
159 3.04991914390484e-07
160 3.02725169376572e-07
161 3.00645808692934e-07
162 2.9875937457291e-07
163 2.9709377713516e-07
164 2.95679569717322e-07
165 2.94490348551335e-07
166 2.93427632414023e-07
167 2.92417240643772e-07
168 2.91413840614041e-07
169 2.90418739723464e-07
170 2.89392261265675e-07
171 2.88386956981412e-07
172 2.87450916403031e-07
173 2.86619922462705e-07
174 2.85916286202337e-07
175 2.85331566374225e-07
176 2.84840467656977e-07
177 2.84404990225084e-07
178 2.83987844795774e-07
179 2.83577975324079e-07
180 2.83169015347084e-07
181 2.8275508157094e-07
182 2.82342739410524e-07
183 2.81928834056089e-07
184 2.81509528576862e-07
185 2.81095282161914e-07
186 2.8067236712559e-07
187 2.80257040685683e-07
188 2.79843163752957e-07
189 2.79430224736643e-07
190 2.79027517535724e-07
191 2.78628590422159e-07
192 2.78231937045348e-07
193 2.77846993412822e-07
194 2.77463101383546e-07
195 2.77086883215816e-07
196 2.76716860980741e-07
197 2.76350590411312e-07
198 2.75991766329753e-07
199 2.75640587688031e-07
200 2.75287987960837e-07
201 2.74943118938609e-07
202 2.74598789928859e-07
203 2.7426443693912e-07
204 2.73923035365442e-07
205 2.7358834131519e-07
206 2.73248105031598e-07
207 2.72908323495358e-07
208 2.72573032589207e-07
209 2.72235297416046e-07
210 2.71898869641518e-07
211 2.7155596171724e-07
212 2.71210865321336e-07
213 2.70869492169368e-07
214 2.70524367351754e-07
215 2.70173472927127e-07
216 2.69822805876174e-07
217 2.69475492586935e-07
218 2.69127809815473e-07
219 2.68779473344694e-07
220 2.68433069550156e-07
221 2.68083681476128e-07
222 2.6773977879202e-07
223 2.67392181285686e-07
224 2.67044669044481e-07
225 2.66693461981049e-07
226 2.66354021505322e-07
227 2.66010630411984e-07
228 2.65673236299335e-07
229 2.65336097982072e-07
230 2.65003336608061e-07
231 2.64683393425003e-07
232 2.64353047896293e-07
233 2.64019035967067e-07
234 2.63680846046555e-07
235 2.63343309825359e-07
236 2.62997673416976e-07
237 2.62644334725337e-07
238 2.62288580188397e-07
239 2.61935042544792e-07
240 2.61580169080844e-07
241 2.61230525211431e-07
242 2.60877385471758e-07
243 2.60528480566791e-07
244 2.60182901001826e-07
245 2.59841812066952e-07
246 2.59501319987976e-07
247 2.59154859350019e-07
248 2.58811240883006e-07
249 2.58471317238218e-07
250 2.58129944086249e-07
251 2.57793232094627e-07
252 2.5745677589839e-07
253 2.57115544854969e-07
254 2.5678315296318e-07
255 2.5644487777754e-07
256 2.56109757401646e-07
257 2.55772675927801e-07
258 2.55439687180115e-07
259 2.55104907864734e-07
260 2.54773311780809e-07
261 2.54439811442353e-07
262 2.54105543717742e-07
263 2.53780001457926e-07
264 2.53452213883065e-07
265 2.53120504112303e-07
266 2.52796894528728e-07
267 2.52462029948219e-07
268 2.52138306677807e-07
269 2.51810575946365e-07
270 2.51487847435783e-07
271 2.51163839948276e-07
272 2.50834489179397e-07
273 2.50449232908068e-07
274 2.5009535420395e-07
275 2.4975702217489e-07
276 2.49423464993015e-07
277 2.49088770942762e-07
278 2.48753195819518e-07
279 2.4841625645422e-07
280 2.48080482379009e-07
281 2.47739222913879e-07
282 2.47396656050114e-07
283 2.47056192392847e-07
284 2.46710044393694e-07
285 2.46364322720183e-07
286 2.46024512762233e-07
287 2.45676091026326e-07
288 2.45330255665976e-07
289 2.4498328343725e-07
290 2.44634634327667e-07
291 2.44283427264236e-07
292 2.43927843257552e-07
293 2.435771762066e-07
294 2.43224690166244e-07
295 2.42872914668624e-07
296 2.42515994841597e-07
297 2.42158449736962e-07
298 2.41804514189425e-07
299 2.41443473214531e-07
300 2.41086098640153e-07
301 2.40724773448164e-07
302 2.40360122916172e-07
303 2.39996609252557e-07
304 2.39632782950139e-07
305 2.3927097458909e-07
306 2.38904561911113e-07
307 2.3854644837229e-07
308 2.38178017752944e-07
309 2.37804314906498e-07
310 2.37435472172365e-07
311 2.37064341490623e-07
312 2.36687384358447e-07
313 2.36315599977388e-07
314 2.35939452863931e-07
315 2.35564854733639e-07
316 2.3518855130078e-07
317 2.34812134181084e-07
318 2.34432462775658e-07
319 2.3404814442074e-07
320 2.33671372029676e-07
321 2.33287636319801e-07
322 2.32903403230011e-07
323 2.32517493259365e-07
324 2.32133970712312e-07
325 2.31742475875762e-07
326 2.31353283197677e-07
327 2.30955365054797e-07
328 2.30552771540715e-07
329 2.30150902780224e-07
330 2.29753709390934e-07
331 2.29342447255476e-07
332 2.28932933055148e-07
333 2.28519823508577e-07
334 2.28104624966363e-07
335 2.27686271614402e-07
336 2.27265644525687e-07
337 2.26840654704574e-07
338 2.26412808501664e-07
339 2.25988785018671e-07
340 2.25558594024733e-07
341 2.25125958763783e-07
342 2.24691959260781e-07
343 2.24254662839485e-07
344 2.23820208589132e-07
345 2.23380567376807e-07
346 2.22941309857561e-07
347 2.22495756929675e-07
348 2.22050559273157e-07
349 2.21606015315956e-07
350 2.21156597035588e-07
351 2.20712820464541e-07
352 2.20267992290246e-07
353 2.19819867197657e-07
354 2.19372637388915e-07
355 2.18924085970684e-07
356 2.18476131408352e-07
357 2.18027466303283e-07
358 2.17578417505138e-07
359 2.17130093460582e-07
360 2.16678770925682e-07
361 2.16235150674038e-07
362 2.15779309087338e-07
363 2.15333344044666e-07
364 2.14882916793613e-07
365 2.14437022805214e-07
366 2.13985273944672e-07
367 2.13538100979349e-07
368 2.13092917533686e-07
369 2.12646511954517e-07
370 2.12202593274924e-07
371 2.11755747159259e-07
372 2.11305945185813e-07
373 2.10857066917924e-07
374 2.10401296385498e-07
375 2.09940367312811e-07
376 2.094873536862e-07
377 2.09034467957281e-07
378 2.08588758709993e-07
379 2.08141457846978e-07
380 2.07698846566018e-07
381 2.07261663831559e-07
382 2.06823116855048e-07
383 2.06381088219132e-07
384 2.05948765596986e-07
385 2.05519867790827e-07
386 2.05091780003386e-07
387 2.04664289071843e-07
388 2.04236457079787e-07
389 2.03813073085257e-07
390 2.03390740693976e-07
391 2.02975925844839e-07
392 2.02544200078592e-07
393 2.02115714387219e-07
394 2.01718918901861e-07
395 2.01314648506923e-07
396 2.00910477587968e-07
397 2.00508480929784e-07
398 2.00109624870493e-07
399 1.9970856612872e-07
400 1.99313774373877e-07
401 1.9892020475254e-07
402 1.98526876715732e-07
403 1.98129271211656e-07
404 1.97743275975881e-07
405 1.97349208974629e-07
406 1.96956307263463e-07
407 1.96565693499906e-07
408 1.96179414047037e-07
409 1.95785744949717e-07
410 1.95402080294116e-07
411 1.95009548065173e-07
412 1.94619090621018e-07
413 1.94235170170032e-07
414 1.9384981442272e-07
415 1.93458632224974e-07
416 1.93072565934926e-07
417 1.92689540767788e-07
418 1.92310537272533e-07
419 1.91933139603861e-07
420 1.91555017181599e-07
421 1.91175061559079e-07
422 1.9079529067767e-07
423 1.90421886259173e-07
424 1.90048709214352e-07
425 1.89680690709793e-07
426 1.89299058206416e-07
427 1.8892713171681e-07
428 1.88549236668223e-07
429 1.88171995318953e-07
430 1.87797354556096e-07
431 1.87421704822555e-07
432 1.87039816523793e-07
433 1.86668728474615e-07
434 1.86295366688682e-07
435 1.85919674322577e-07
436 1.8554750624844e-07
437 1.85176219247296e-07
438 1.84813742976075e-07
439 1.84451565132804e-07
440 1.84092613153553e-07
441 1.83737071779433e-07
442 1.83388465302414e-07
443 1.83043098900271e-07
444 1.82695728767612e-07
445 1.82355080369234e-07
446 1.82015924110601e-07
447 1.8167541782077e-07
448 1.81348497108047e-07
449 1.81018833700364e-07
450 1.80691372975161e-07
451 1.80366001245602e-07
452 1.80048544962119e-07
453 1.79725603288716e-07
454 1.79410946543612e-07
455 1.79091586005597e-07
456 1.78777796122631e-07
457 1.78461291966414e-07
458 1.78155332264396e-07
459 1.77846672499982e-07
460 1.7754712189344e-07
461 1.7725420775605e-07
462 1.76952397623609e-07
463 1.76658886630321e-07
464 1.76372481064391e-07
465 1.76084384406749e-07
466 1.75797737256289e-07
467 1.75511885913693e-07
468 1.75240202793248e-07
469 1.74959211562964e-07
470 1.74678760345159e-07
471 1.74409365172323e-07
472 1.74137468889057e-07
473 1.73867348962631e-07
474 1.73600838593302e-07
475 1.73335379827222e-07
476 1.73076401210892e-07
477 1.72819127897128e-07
478 1.72560518763021e-07
479 1.72305533396866e-07
480 1.72049396951479e-07
481 1.71798717474303e-07
482 1.71550979644053e-07
483 1.71302303897392e-07
484 1.71055859254921e-07
485 1.70814274724762e-07
486 1.70570260138447e-07
487 1.703288603494e-07
488 1.70094367035745e-07
489 1.69855624676529e-07
490 1.69622211387832e-07
491 1.69390816040504e-07
492 1.69160387031297e-07
493 1.68931634902947e-07
494 1.68699571645448e-07
495 1.68473945905134e-07
496 1.68245804843536e-07
497 1.68023760238611e-07
498 1.6779581812898e-07
499 1.67566980735501e-07
500 1.6734522034767e-07
501 1.67124170502575e-07
502 1.66899198461579e-07
503 1.66678844948365e-07
504 1.66452608141299e-07
505 1.66230719855776e-07
506 1.66010352131707e-07
507 1.65787028549857e-07
508 1.65567414001089e-07
509 1.65346861535909e-07
510 1.65121889494912e-07
511 1.6490021437221e-07
512 1.64681736691819e-07
513 1.6446260531211e-07
514 1.6423783222308e-07
515 1.64017635029268e-07
516 1.63793657748101e-07
517 1.63573076861212e-07
518 1.63354030746632e-07
519 1.63134188824188e-07
520 1.62910581025244e-07
521 1.62690184879466e-07
522 1.62472346687537e-07
523 1.62250927360219e-07
524 1.62027049555036e-07
525 1.61807108156609e-07
526 1.61591870551092e-07
527 1.6136766589625e-07
528 1.61149472432953e-07
529 1.60926063585975e-07
530 1.60702199991647e-07
531 1.60481036459714e-07
532 1.60259872927782e-07
533 1.60039292040892e-07
534 1.59816934797163e-07
535 1.59591834858475e-07
536 1.59372902430732e-07
537 1.59146452460845e-07
538 1.58923910476005e-07
539 1.58700586894156e-07
540 1.58478798084616e-07
541 1.58253129711738e-07
542 1.58027987140485e-07
543 1.57797757083245e-07
544 1.5757245819259e-07
545 1.57345766638173e-07
546 1.57115351839821e-07
547 1.56882279611636e-07
548 1.56648965798922e-07
549 1.56416191998687e-07
550 1.56180774979475e-07
551 1.55940156787437e-07
552 1.55700362824973e-07
553 1.55451715500021e-07
554 1.55205611918063e-07
555 1.54954406639263e-07
556 1.54696479626182e-07
557 1.54442204802763e-07
558 1.54178977140873e-07
559 1.53913603639921e-07
560 1.53650319134613e-07
561 1.53391766843924e-07
562 1.53145833792223e-07
563 1.52907446704376e-07
564 1.52676093989612e-07
565 1.5245191775648e-07
566 1.52230640537709e-07
567 1.52008510667656e-07
568 1.51788810853759e-07
569 1.51571029505249e-07
570 1.51357241406913e-07
571 1.51136845261135e-07
572 1.50921664499037e-07
573 1.50707265333949e-07
574 1.50493946193819e-07
575 1.50280683897108e-07
576 1.50062689385777e-07
577 1.49854272990524e-07
578 1.49638580637657e-07
579 1.49427179962913e-07
580 1.49218948308771e-07
581 1.49001252225389e-07
582 1.48792665299879e-07
583 1.48581122516589e-07
584 1.48367050201159e-07
585 1.48159301716078e-07
586 1.47947815776206e-07
587 1.47733288713425e-07
588 1.4752748711544e-07
589 1.47315191156849e-07
590 1.47104955772193e-07
591 1.46897605191043e-07
592 1.46684897117666e-07
593 1.46477418638824e-07
594 1.46266515343996e-07
595 1.46058070527033e-07
596 1.45848346733146e-07
597 1.45638523463276e-07
598 1.45431712894606e-07
599 1.45224106518071e-07
600 1.4501087264307e-07
601 1.4480328047739e-07
602 1.44592419815126e-07
603 1.44386802958252e-07
604 1.44181214523087e-07
605 1.43970822819028e-07
606 1.43764296467452e-07
607 1.43555311638011e-07
608 1.43342901992582e-07
609 1.43140027830668e-07
610 1.42929593494046e-07
611 1.42723905582898e-07
612 1.42517691870125e-07
613 1.42307484907178e-07
614 1.42101839628594e-07
615 1.41894119565222e-07
616 1.41689469046469e-07
617 1.41481535820276e-07
618 1.4127277836451e-07
619 1.41069392611826e-07
620 1.40862056241531e-07
621 1.40654066171919e-07
622 1.40444328167177e-07
623 1.4024084293851e-07
624 1.4003354920078e-07
625 1.39823185918431e-07
626 1.39621150196945e-07
627 1.3941333065759e-07
628 1.39209305416443e-07
629 1.39005024379912e-07
630 1.38798881721414e-07
631 1.38593776455309e-07
632 1.38394156579125e-07
633 1.38191296628065e-07
634 1.37987797188543e-07
635 1.37784482490133e-07
636 1.37579860393089e-07
637 1.37378364684082e-07
638 1.37178659542769e-07
639 1.36977106990344e-07
640 1.36776819203988e-07
641 1.3657734143635e-07
642 1.36374424641872e-07
643 1.36173881060131e-07
644 1.35975170678648e-07
645 1.35772566522974e-07
646 1.35574410364825e-07
647 1.35375472609667e-07
648 1.35175525883824e-07
649 1.34980624011405e-07
650 1.34779597260604e-07
651 1.34580375288351e-07
652 1.34384777084051e-07
653 1.34184247713165e-07
654 1.33988308448352e-07
655 1.33790521772426e-07
656 1.33599044715993e-07
657 1.33395644752454e-07
658 1.33200657614907e-07
659 1.3300389412052e-07
660 1.32808708031007e-07
661 1.32613848791152e-07
662 1.32415408415909e-07
663 1.32220449700071e-07
664 1.32026485744063e-07
665 1.31833843397544e-07
666 1.31638387301791e-07
667 1.31446142859204e-07
668 1.31249564105929e-07
669 1.31053951690774e-07
670 1.30861380398528e-07
671 1.30666393260981e-07
672 1.30475001469676e-07
673 1.30283083876748e-07
674 1.30091422079204e-07
675 1.29895099121313e-07
676 1.29706108964456e-07
677 1.29511263935456e-07
678 1.29322089037487e-07
679 1.29130043546866e-07
680 1.28939163346331e-07
681 1.28748965266823e-07
682 1.28558312439964e-07
683 1.28369990193278e-07
684 1.28180076330864e-07
685 1.27989551401697e-07
686 1.27798799098855e-07
687 1.27606838873362e-07
688 1.27423717799502e-07
689 1.27240042502308e-07
690 1.2707127439171e-07
691 1.2686970762843e-07
692 1.26615987028345e-07
693 1.26426726865247e-07
694 1.26240223607965e-07
695 1.26054359839145e-07
696 1.25872773537594e-07
697 1.25688330854246e-07
698 1.25502410242007e-07
699 1.25321250266097e-07
700 1.25138399198477e-07
701 1.24957182379148e-07
702 1.24775439758196e-07
703 1.24599580431095e-07
704 1.24415024060909e-07
705 1.24242319543555e-07
706 1.24061557471578e-07
707 1.23881463309772e-07
708 1.23707948773699e-07
709 1.23528863582578e-07
710 1.23354183756419e-07
711 1.23178210742481e-07
712 1.23002124041705e-07
713 1.22826008919219e-07
714 1.22652252798616e-07
715 1.22480088293742e-07
716 1.2230552215442e-07
717 1.22133059221596e-07
718 1.21956787779709e-07
719 1.21784879070219e-07
720 1.21614291970218e-07
721 1.21443733291926e-07
722 1.21270346653546e-07
723 1.21098764793715e-07
724 1.20928760338757e-07
725 1.20758741672944e-07
726 1.20588609320293e-07
727 1.20420097005081e-07
728 1.20249112001147e-07
729 1.20084138188759e-07
730 1.19911049978327e-07
731 1.19743688742346e-07
732 1.19577748591837e-07
733 1.19408824161837e-07
734 1.19243836138594e-07
735 1.19078379157145e-07
736 1.18910548962958e-07
737 1.187458167351e-07
738 1.18579833952026e-07
739 1.1841910207977e-07
740 1.18253751679731e-07
741 1.18090767387002e-07
742 1.17929999987609e-07
743 1.17768301777232e-07
744 1.17604784577452e-07
745 1.17443178737631e-07
746 1.17290113621493e-07
747 1.17126212728635e-07
748 1.1696504742531e-07
749 1.16807719052758e-07
750 1.16650404891061e-07
751 1.16494504709408e-07
752 1.16333843891425e-07
753 1.16177758968661e-07
754 1.16022654594872e-07
755 1.15869752903563e-07
756 1.15714676951484e-07
757 1.15560560232097e-07
758 1.15407402745404e-07
759 1.15252404953026e-07
760 1.15103013342832e-07
761 1.14948534246651e-07
762 1.14801288475519e-07
763 1.14651726335069e-07
764 1.14503805548338e-07
765 1.14355110270026e-07
766 1.14209932178255e-07
767 1.14058046563059e-07
768 1.1391377796599e-07
769 1.13767761433792e-07
770 1.13621609898473e-07
771 1.1347781025961e-07
772 1.13331658724292e-07
773 1.13184917438502e-07
774 1.13042155192034e-07
775 1.12897851067828e-07
776 1.12758812065294e-07
777 1.12614252145704e-07
778 1.12471788327184e-07
779 1.12328947921014e-07
780 1.12187642287154e-07
781 1.12046784295217e-07
782 1.1190390836191e-07
783 1.1176383907241e-07
784 1.11623855048038e-07
785 1.11483927867084e-07
786 1.11342828290617e-07
787 1.11205885389154e-07
788 1.11062334440248e-07
789 1.10922613316689e-07
790 1.10783943796378e-07
791 1.10642147888029e-07
792 1.10503776795667e-07
793 1.10367210481854e-07
794 1.10229485983382e-07
795 1.10084691584689e-07
796 1.09951407978315e-07
797 1.09812397397491e-07
798 1.09676541626413e-07
799 1.09533559111696e-07
800 1.09400197345622e-07
801 1.09257470626289e-07
802 1.09127242353679e-07
803 1.08982057156481e-07
804 1.08854969482763e-07
805 1.0870221700543e-07
806 1.08577786761543e-07
807 1.08422824496301e-07
808 1.08298863210621e-07
809 1.08150558730813e-07
810 1.08022398137564e-07
811 1.07870540944077e-07
812 1.07744000388266e-07
813 1.07598758347649e-07
814 1.07472153842991e-07
815 1.0733101163396e-07
816 1.07203895538532e-07
817 1.07063883092451e-07
818 1.06938138344503e-07
819 1.0680557949172e-07
820 1.06672978006372e-07
821 1.0654213866701e-07
822 1.06416003120557e-07
823 1.0628490798581e-07
824 1.06156335277774e-07
825 1.06026860180464e-07
826 1.05902749680808e-07
827 1.05778070746965e-07
828 1.05655935556115e-07
829 1.05529146310346e-07
830 1.05404787120733e-07
831 1.05282616402746e-07
832 1.05158996177579e-07
833 1.05039497100279e-07
834 1.04920303556355e-07
835 1.04797109656829e-07
836 1.04676509238288e-07
837 1.04559568114837e-07
838 1.04435336822917e-07
839 1.04333182093796e-07
840 1.04213192742009e-07
841 1.04114732835114e-07
842 1.0397859995237e-07
843 1.0389351956519e-07
844 1.03751645497141e-07
845 1.03651899507895e-07
846 1.03555436226088e-07
847 1.03423651864887e-07
848 1.03308799737079e-07
849 1.03193173117688e-07
850 1.03093817926947e-07
851 1.02994491157915e-07
852 1.02877073970831e-07
853 1.02796107626091e-07
854 1.02662419010358e-07
855 1.02552554892554e-07
856 1.02495185672069e-07
857 1.02336791485413e-07
858 1.022724802624e-07
859 1.02133817847516e-07
860 1.02092819531663e-07
861 1.01924641171536e-07
862 1.01873581570544e-07
863 1.01724396017744e-07
864 1.01654208606305e-07
865 1.01551535180988e-07
866 1.01462923396411e-07
867 1.01338770264192e-07
868 1.01268881280703e-07
869 1.01132684449112e-07
870 1.01077631597946e-07
871 1.00932247448782e-07
872 1.00894396837248e-07
873 1.00739839581365e-07
874 1.00731192276271e-07
875 1.00553656068314e-07
876 1.00480825437899e-07
877 1.00398601432516e-07
878 1.00291948967879e-07
879 1.00216837495282e-07
880 1.001042519988e-07
881 1.0003901707023e-07
882 9.9918580076519e-08
883 9.98578926214577e-08
884 9.97304638872265e-08
885 9.96776918782416e-08
886 9.95466038489212e-08
887 9.95010367432769e-08
888 9.93662112591664e-08
889 9.93242323943377e-08
890 9.91896982327489e-08
891 9.9149204402238e-08
892 9.90099735531658e-08
893 9.89702257925273e-08
894 9.88328352491408e-08
895 9.88027935022728e-08
896 9.86593491347776e-08
897 9.86294992344483e-08
898 9.84885488719556e-08
899 9.84606032261581e-08
900 9.83184023084505e-08
901 9.829172853415e-08
902 9.81471828254143e-08
903 9.81238983399635e-08
904 9.79776970666535e-08
905 9.79522241095765e-08
906 9.78121974526402e-08
907 9.77888490183432e-08
908 9.76502008143143e-08
909 9.7623370720612e-08
910 9.74920340013341e-08
911 9.74581126911289e-08
912 9.73308971197184e-08
913 9.72996829773365e-08
914 9.71723821407977e-08
915 9.71360023527268e-08
916 9.70111102560622e-08
917 9.6976094710044e-08
918 9.68518207855595e-08
919 9.68168905046696e-08
920 9.66960556070262e-08
921 9.66634488008822e-08
922 9.65407238595617e-08
923 9.65021342835826e-08
924 9.63908632911625e-08
925 9.63573825174535e-08
926 9.62312043384372e-08
927 9.61833990231753e-08
928 9.61112505137862e-08
929 9.60165138508273e-08
930 9.59702717295841e-08
931 9.58626529268258e-08
932 9.58105275117305e-08
933 9.57383647914867e-08
934 9.56599990331597e-08
935 9.55944088332217e-08
936 9.5512412201515e-08
937 9.54624539417637e-08
938 9.53769969669338e-08
939 9.53336822817619e-08
940 9.52562828615555e-08
941 9.52275343024667e-08
942 9.5155613166753e-08
943 9.50814538214217e-08
944 9.49638163660893e-08
945 9.49156699903142e-08
946 9.48029210690038e-08
947 9.47575173881887e-08
948 9.46494367326522e-08
949 9.45974392152493e-08
950 9.44973947980543e-08
951 9.44473796948841e-08
952 9.43463902558506e-08
953 9.42919271551546e-08
954 9.41983770985644e-08
955 9.41436653079109e-08
956 9.40517423941856e-08
957 9.3992341021476e-08
958 9.39068200977999e-08
959 9.38394535410225e-08
960 9.37661326361194e-08
961 9.3693202529721e-08
962 9.36189863409709e-08
963 9.35510726662869e-08
964 9.34763519921944e-08
965 9.3406562484688e-08
966 9.33343926590169e-08
967 9.32640347173219e-08
968 9.31932504499855e-08
969 9.31224235500849e-08
970 9.30505308360807e-08
971 9.29809260696857e-08
972 9.29098007418361e-08
973 9.28409917833051e-08
974 9.27706267361827e-08
975 9.26990679772643e-08
976 9.26318577398888e-08
977 9.25616561175957e-08
978 9.24924066225685e-08
979 9.24260845636127e-08
980 9.23547602837971e-08
981 9.22870952990706e-08
982 9.22218745813552e-08
983 9.21497260719661e-08
984 9.20818195027095e-08
985 9.20127263270842e-08
986 9.1944031055391e-08
987 9.18773821467767e-08
988 9.18119127391037e-08
989 9.17396079103128e-08
990 9.16765614533688e-08
991 9.16055000743654e-08
992 9.15393982836576e-08
993 9.14726285827783e-08
994 9.14018656317239e-08
995 9.13374549327273e-08
996 9.12734208213806e-08
997 9.12059192614834e-08
998 9.11400732661605e-08
999 9.10736304149395e-08
1000 9.10052619929047e-08
1001 9.09383643943329e-08
1002 9.08706638824697e-08
1003 9.08070489913371e-08
1004 9.07424890783659e-08
1005 9.06744475059895e-08
1006 9.06105910303268e-08
1007 9.05438355403021e-08
1008 9.04779682286971e-08
1009 9.04105093013641e-08
1010 9.03481449654464e-08
1011 9.02810484149086e-08
1012 9.02145060877046e-08
1013 9.01487950955016e-08
1014 9.00877807907818e-08
1015 9.00202223874658e-08
1016 8.99537369036807e-08
1017 8.98920404779346e-08
1018 8.98253560421836e-08
1019 8.97605616501096e-08
1020 8.96946588113678e-08
1021 8.96326213251086e-08
1022 8.9567336658547e-08
1023 8.95045602078426e-08
1024 8.94403981988035e-08
1025 8.93770319976284e-08
1026 8.93137155344448e-08
1027 8.9262734093154e-08
1028 8.92395135565494e-08
1029 8.9160344884931e-08
1030 8.90965381472597e-08
1031 8.90158773358962e-08
1032 8.89685409788399e-08
1033 8.88879156946132e-08
1034 8.88203430804424e-08
1035 8.87659226123105e-08
1036 8.86997213456198e-08
1037 8.86444055936408e-08
1038 8.85709354747632e-08
1039 8.85096937963681e-08
1040 8.84420927604879e-08
1041 8.83926247752242e-08
1042 8.83177548871572e-08
1043 8.82506583366194e-08
1044 8.81907737948495e-08
1045 8.8128551567479e-08
1046 8.80712320849852e-08
1047 8.80153550042451e-08
1048 8.79408403875459e-08
1049 8.78702124396114e-08
1050 8.78140795634863e-08
1051 8.77531505238949e-08
1052 8.76929817650307e-08
1053 8.76374528502311e-08
1054 8.75629240226772e-08
1055 8.74932197802991e-08
1056 8.74378329740466e-08
1057 8.73807692869377e-08
1058 8.73160459491373e-08
1059 8.72555432351874e-08
1060 8.71853913508858e-08
1061 8.71212435527013e-08
1062 8.70633485305916e-08
1063 8.70072938141675e-08
1064 8.69413128157248e-08
1065 8.68846328216932e-08
1066 8.68146727839303e-08
1067 8.67576943619497e-08
1068 8.66896101570092e-08
1069 8.66337899196878e-08
1070 8.65654854464992e-08
1071 8.65064535560123e-08
1072 8.64397335931244e-08
1073 8.63860876165745e-08
1074 8.63176339294114e-08
1075 8.62533724443892e-08
1076 8.61993569856168e-08
1077 8.61303917076839e-08
1078 8.60673949887314e-08
1079 8.60128963608986e-08
1080 8.59499422745102e-08
1081 8.58823909766215e-08
1082 8.58243112134005e-08
1083 8.57616058169697e-08
1084 8.56997033338303e-08
1085 8.56380921732125e-08
1086 8.55772555041767e-08
1087 8.55151753853534e-08
1088 8.54565413987984e-08
1089 8.53949941870269e-08
1090 8.5332864330212e-08
1091 8.52709121090811e-08
1092 8.52117096883376e-08
1093 8.51508943355839e-08
1094 8.50865404800061e-08
1095 8.50313242040102e-08
1096 8.49703454264272e-08
1097 8.49087413712368e-08
1098 8.48497236916046e-08
1099 8.4785611420557e-08
1100 8.47295851258423e-08
1101 8.46683718691565e-08
1102 8.4607520989266e-08
1103 8.45467482690765e-08
1104 8.44871621552556e-08
1105 8.4426766022716e-08
1106 8.43667748995358e-08
1107 8.43075511625102e-08
1108 8.42474037199281e-08
1109 8.41868015299951e-08
1110 8.41249701011293e-08
1111 8.4067700356627e-08
1112 8.40070413232752e-08
1113 8.39492315662937e-08
1114 8.38917415535434e-08
1115 8.38323543916886e-08
1116 8.37757809790673e-08
1117 8.3713302956312e-08
1118 8.3654946081424e-08
1119 8.35962268297408e-08
1120 8.35376212648953e-08
1121 8.3481801027574e-08
1122 8.34227407153776e-08
1123 8.33671904842959e-08
1124 8.33099207397936e-08
1125 8.32517415005896e-08
1126 8.31926527666838e-08
1127 8.31375217558161e-08
1128 8.30823481123844e-08
1129 8.30227904202729e-08
1130 8.29669133395328e-08
1131 8.2912215759734e-08
1132 8.28534112429224e-08
1133 8.27914448109368e-08
1134 8.27365340683173e-08
1135 8.26808062015516e-08
1136 8.26259594077783e-08
1137 8.25688815098147e-08
1138 8.25132602244594e-08
1139 8.24550454581185e-08
1140 8.24026926693477e-08
1141 8.23453092380078e-08
1142 8.2287890279531e-08
1143 8.22360064489658e-08
1144 8.21826020569461e-08
1145 8.21240035975279e-08
1146 8.2072702412006e-08
1147 8.2012974189638e-08
1148 8.19580066035996e-08
1149 8.19051493294864e-08
1150 8.18531518120835e-08
1151 8.17944396658277e-08
1152 8.17422929344502e-08
1153 8.16864513808468e-08
1154 8.16360739008815e-08
1155 8.15792446928754e-08
1156 8.15249592278633e-08
1157 8.14710574559285e-08
1158 8.14200689092104e-08
1159 8.13671334753963e-08
1160 8.13131109111964e-08
1161 8.12606586464426e-08
1162 8.12076379475002e-08
1163 8.11524074606496e-08
1164 8.11047868864989e-08
1165 8.10477658319542e-08
1166 8.09965925441247e-08
1167 8.09418665426165e-08
1168 8.08961218012882e-08
1169 8.08366493743051e-08
1170 8.07925815138333e-08
1171 8.07343880637745e-08
1172 8.06915423368082e-08
1173 8.06320699098251e-08
1174 8.0590481843501e-08
1175 8.0524294787665e-08
1176 8.04889452865609e-08
1177 8.04237956231191e-08
1178 8.0383181000343e-08
1179 8.03221098522044e-08
1180 8.02884230211021e-08
1181 8.02216177930859e-08
1182 8.01871635758289e-08
1183 8.01218220658484e-08
1184 8.0087538378848e-08
1185 8.00199799755319e-08
1186 7.99860799816088e-08
1187 7.99225290393224e-08
1188 7.98884727259974e-08
1189 7.98240620270008e-08
1190 7.9791242058036e-08
1191 7.97231649585228e-08
1192 7.96925476720389e-08
1193 7.96289043591969e-08
1194 7.95955088506162e-08
1195 7.95301318134989e-08
1196 7.95011061427431e-08
1197 7.9431977439981e-08
1198 7.93988661484946e-08
1199 7.93369281382184e-08
1200 7.93036534219027e-08
1201 7.92377434777336e-08
1202 7.92070622424035e-08
1203 7.91434615621256e-08
1204 7.91101655295279e-08
1205 7.90468988043358e-08
1206 7.90123166893864e-08
1207 7.89529508438136e-08
1208 7.89147662771938e-08
1209 7.88581218102991e-08
1210 7.88214578051338e-08
1211 7.87640104249476e-08
1212 7.87266145607646e-08
1213 7.86715261824611e-08
1214 7.86289291454523e-08
1215 7.85771732125795e-08
1216 7.85337945785614e-08
1217 7.84801130748747e-08
1218 7.84399318831674e-08
1219 7.83899167799973e-08
1220 7.83450531116614e-08
1221 7.82952014333205e-08
1222 7.82501885510101e-08
1223 7.82044224933998e-08
1224 7.81566455998473e-08
1225 7.81132314386923e-08
1226 7.80661153498841e-08
1227 7.80198945449229e-08
1228 7.79714497411987e-08
1229 7.79259323735459e-08
1230 7.78762299091795e-08
1231 7.7831771250203e-08
1232 7.77837527721204e-08
1233 7.77411415242568e-08
1234 7.76941320168589e-08
1235 7.76481456910005e-08
1236 7.759802400642e-08
1237 7.75542616793246e-08
1238 7.7505490025942e-08
1239 7.74590560581601e-08
1240 7.74122881352923e-08
1241 7.73687389710176e-08
1242 7.73229160699884e-08
1243 7.72754447098123e-08
1244 7.72305455143396e-08
1245 7.71866766058338e-08
1246 7.71412445033093e-08
1247 7.70954571294169e-08
1248 7.70483126188992e-08
1249 7.70035555319737e-08
1250 7.69529648891876e-08
1251 7.69074475215348e-08
1252 7.68606724932397e-08
1253 7.68137340401154e-08
1254 7.67687495795144e-08
1255 7.67238432786144e-08
1256 7.66742971336498e-08
1257 7.66270034091576e-08
1258 7.65779546441081e-08
1259 7.65297443194868e-08
1260 7.6481605049139e-08
1261 7.64335936764837e-08
1262 7.63822853855345e-08
1263 7.63264296210764e-08
1264 7.62627649919523e-08
1265 7.62287442057641e-08
1266 7.61759650913518e-08
1267 7.61315916975036e-08
1268 7.60711316161178e-08
1269 7.60457368187417e-08
1270 7.59716698439661e-08
1271 7.59454437115892e-08
1272 7.58758815777583e-08
1273 7.58603633244093e-08
1274 7.577275340509e-08
1275 7.57630473913196e-08
1276 7.56764606535398e-08
1277 7.56723963490913e-08
1278 7.55771623062174e-08
1279 7.55809210772895e-08
1280 7.54828093363358e-08
1281 7.54843014760809e-08
1282 7.53860192048705e-08
1283 7.53887618998306e-08
1284 7.52887103772082e-08
1285 7.52893214439609e-08
1286 7.51958140199349e-08
1287 7.51925952613419e-08
1288 7.5104097163603e-08
1289 7.5089353401836e-08
1290 7.50180788600119e-08
1291 7.49907513863945e-08
1292 7.49412620848489e-08
1293 7.48993329580117e-08
1294 7.4835220686964e-08
1295 7.47968798009424e-08
1296 7.47625463759505e-08
1297 7.46593897815728e-08
1298 7.46770396631291e-08
1299 7.46135313534069e-08
1300 7.45139772106995e-08
1301 7.4533538452215e-08
1302 7.4470591471254e-08
1303 7.43708525874354e-08
1304 7.43914085887809e-08
1305 7.43269410463654e-08
1306 7.42233012829274e-08
1307 7.42438288625635e-08
1308 7.41785513014293e-08
1309 7.40748120620083e-08
1310 7.40987431413487e-08
1311 7.40262535714464e-08
1312 7.39211785116822e-08
1313 7.39459409260235e-08
1314 7.3873763994925e-08
1315 7.37693213181956e-08
1316 7.3783894549706e-08
1317 7.37321954602521e-08
1318 7.36245837629212e-08
1319 7.36382261834478e-08
1320 7.3584700999163e-08
1321 7.3473806594393e-08
1322 7.34819565195721e-08
1323 7.34313374550766e-08
1324 7.33177145662012e-08
1325 7.33348599624151e-08
1326 7.32629175104194e-08
1327 7.32156948402007e-08
1328 7.31098381834272e-08
1329 7.31413720700402e-08
1330 7.30125364611922e-08
1331 7.3032957459418e-08
1332 7.29269871158067e-08
1333 7.28844966602082e-08
1334 7.28569418129155e-08
1335 7.27561371149932e-08
1336 7.27266851185959e-08
1337 7.26393949435078e-08
1338 7.26736715250809e-08
1339 7.25438766835396e-08
1340 7.25728241945944e-08
1341 7.24416935327099e-08
1342 7.24717992284241e-08
1343 7.23410309433348e-08
1344 7.23673281299853e-08
1345 7.22394517538305e-08
1346 7.22535347108533e-08
1347 7.21363520028717e-08
1348 7.21457453778385e-08
1349 7.20281292387881e-08
1350 7.20345596505467e-08
1351 7.19231252332975e-08
1352 7.19271611160366e-08
1353 7.18186257131492e-08
1354 7.18059922633074e-08
1355 7.17058270538473e-08
1356 7.17066725997029e-08
1357 7.16285768476155e-08
1358 7.15485484192868e-08
1359 7.15328454248265e-08
1360 7.1454607564192e-08
1361 7.14381300781497e-08
1362 7.13045977818183e-08
1363 7.13510885930191e-08
1364 7.12619794285274e-08
1365 7.11996506197465e-08
1366 7.11488610249944e-08
1367 7.10688610183752e-08
1368 7.10338596832116e-08
1369 7.09795173747807e-08
1370 7.08982454966645e-08
1371 7.08869691834479e-08
1372 7.07716694137162e-08
1373 7.08045462260998e-08
1374 7.06752558699009e-08
1375 7.06760943103291e-08
1376 7.06158900243281e-08
1377 7.05399614275848e-08
1378 7.04583200672459e-08
1379 7.04850933175294e-08
1380 7.03428142401208e-08
1381 7.03827538472979e-08
1382 7.0239117633264e-08
1383 7.02763429671904e-08
1384 7.0133332030764e-08
1385 7.01698184002453e-08
1386 7.00309499279683e-08
1387 7.00596061165015e-08
1388 6.99208868581991e-08
1389 6.99562008321664e-08
1390 6.98154138945029e-08
1391 6.98522910624888e-08
1392 6.97102535696104e-08
1393 6.97443454100721e-08
1394 6.96069548666856e-08
1395 6.96448125836469e-08
1396 6.95024908736741e-08
1397 6.95384230198215e-08
1398 6.93997890266473e-08
1399 6.9436879357454e-08
1400 6.92973216587234e-08
1401 6.93320245659379e-08
1402 6.91937955821231e-08
1403 6.92273118829689e-08
1404 6.90917332235585e-08
1405 6.91287951326558e-08
1406 6.89877666104621e-08
1407 6.90279264858873e-08
1408 6.88868411202748e-08
1409 6.8923704077406e-08
1410 6.87907615315453e-08
1411 6.88214498723028e-08
1412 6.86876475697318e-08
1413 6.87201477944654e-08
1414 6.85900332086931e-08
1415 6.86195704702186e-08
1416 6.84893493030359e-08
1417 6.85211958284526e-08
1418 6.83954439750778e-08
1419 6.84155097019357e-08
1420 6.82936658336075e-08
1421 6.8328844804455e-08
1422 6.82285659081572e-08
1423 6.8200506575522e-08
1424 6.80966820709727e-08
1425 6.81304825889129e-08
1426 6.80585614531992e-08
1427 6.80062299807105e-08
1428 6.79700917771697e-08
1429 6.79058018704382e-08
1430 6.78693083955295e-08
1431 6.78188314395811e-08
1432 6.77754599109903e-08
1433 6.77239881952119e-08
1434 6.76794158493976e-08
1435 6.76293439028086e-08
1436 6.75828246698984e-08
1437 6.7536340964125e-08
1438 6.74876687867254e-08
1439 6.74460238769825e-08
1440 6.73991777944138e-08
1441 6.73541791229582e-08
1442 6.73107081183844e-08
1443 6.72643167831666e-08
1444 6.72211939445333e-08
1445 6.71731541501686e-08
1446 6.71303794774758e-08
1447 6.70866668883718e-08
1448 6.70440414296536e-08
1449 6.69969040245633e-08
1450 6.69498518846012e-08
1451 6.6906451934301e-08
1452 6.68646293888742e-08
1453 6.68183020025026e-08
1454 6.67772539486577e-08
1455 6.67333210913057e-08
1456 6.66875976662595e-08
1457 6.66448656261309e-08
1458 6.66027304419003e-08
1459 6.65571135982646e-08
1460 6.65156250079235e-08
1461 6.64702071162537e-08
1462 6.64280719320232e-08
1463 6.63818227053525e-08
1464 6.63431833913819e-08
1465 6.62954846575303e-08
1466 6.62538255369327e-08
1467 6.62127419559511e-08
1468 6.61679635527435e-08
1469 6.6125174669196e-08
1470 6.60832597532135e-08
1471 6.60378063344069e-08
1472 6.59939587421832e-08
1473 6.59517880308158e-08
1474 6.590891388214e-08
1475 6.58636665207268e-08
1476 6.58214034388038e-08
1477 6.57796874747874e-08
1478 6.57388454783359e-08
1479 6.56938681231622e-08
1480 6.5655235914619e-08
1481 6.56116725394895e-08
1482 6.55680310046591e-08
1483 6.55270753213699e-08
1484 6.54854730441912e-08
1485 6.54428262691908e-08
1486 6.54010818834649e-08
1487 6.53583782650458e-08
1488 6.53189005106469e-08
1489 6.52786695809482e-08
1490 6.52334009032529e-08
1491 6.51939231488541e-08
1492 6.51533724749243e-08
1493 6.51121467853955e-08
1494 6.50734222062965e-08
1495 6.50303704219368e-08
1496 6.49921574336076e-08
1497 6.4952047296174e-08
1498 6.49124487495101e-08
1499 6.48745839271214e-08
1500 6.48311768713938e-08
1501 6.47937383746466e-08
1502 6.47538414000337e-08
1503 6.47173763468345e-08
1504 6.46776001644866e-08
1505 6.4638285834917e-08
1506 6.45932374254699e-08
1507 6.45549533828671e-08
1508 6.45165556534266e-08
1509 6.4476843419925e-08
1510 6.44455440124148e-08
1511 6.44037285724153e-08
1512 6.43735731387096e-08
1513 6.43357935814493e-08
1514 6.42925215288415e-08
1515 6.42614637058614e-08
1516 6.42214246227013e-08
1517 6.41944737367339e-08
1518 6.41537454271202e-08
1519 6.41216502117459e-08
1520 6.40767439108458e-08
1521 6.4052386505864e-08
1522 6.40087307601789e-08
1523 6.39879544905853e-08
1524 6.3936148819721e-08
1525 6.39179233985487e-08
1526 6.38675174968739e-08
1527 6.38524895180126e-08
1528 6.37967119132554e-08
1529 6.37874961739726e-08
1530 6.37280805904084e-08
1531 6.37227728361722e-08
1532 6.36570973711059e-08
1533 6.36587742519623e-08
1534 6.35931129977507e-08
1535 6.35960901718136e-08
1536 6.35244603586216e-08
1537 6.35330010823054e-08
1538 6.3456177201715e-08
1539 6.34695638268568e-08
1540 6.33927896842579e-08
1541 6.34104537766689e-08
1542 6.33271639571831e-08
1543 6.33469454669466e-08
1544 6.32616306006639e-08
1545 6.32854337823119e-08
1546 6.32030250358184e-08
1547 6.32226004881886e-08
1548 6.3141797568278e-08
1549 6.31642649295827e-08
1550 6.30760510489381e-08
1551 6.31056096267457e-08
1552 6.30213889962761e-08
1553 6.30465208928399e-08
1554 6.29610923397195e-08
1555 6.29835241738874e-08
1556 6.29006251529063e-08
1557 6.29281089459255e-08
1558 6.28421545911806e-08
1559 6.28703347160808e-08
1560 6.27869525260394e-08
1561 6.28108267619609e-08
1562 6.27344576287214e-08
1563 6.27552907417339e-08
1564 6.26798311031962e-08
1565 6.26962233241102e-08
1566 6.26269240910915e-08
1567 6.26411846837982e-08
1568 6.25747489380046e-08
1569 6.25826430677989e-08
1570 6.25204847892746e-08
1571 6.25265670350927e-08
1572 6.24629734602422e-08
1573 6.24701499418734e-08
1574 6.24110327862581e-08
1575 6.24154878892114e-08
1576 6.23549212264152e-08
1577 6.2357223384879e-08
1578 6.23005291799927e-08
1579 6.23020568468746e-08
1580 6.22497822178047e-08
1581 6.22517219994734e-08
1582 6.21926190547128e-08
1583 6.21928251121062e-08
1584 6.21442666215444e-08
1585 6.21392288735478e-08
1586 6.20911109194822e-08
1587 6.20876861034958e-08
1588 6.20381186422492e-08
1589 6.20314466459604e-08
1590 6.19867677187358e-08
1591 6.19809839008667e-08
1592 6.19328233142369e-08
1593 6.1927877936796e-08
1594 6.1882118984613e-08
1595 6.18770172877703e-08
1596 6.18321394085797e-08
1597 6.18263129581464e-08
1598 6.17796871438259e-08
1599 6.17751609865991e-08
1600 6.17307094330499e-08
1601 6.17237674305215e-08
1602 6.16790600815875e-08
1603 6.16755500004729e-08
1604 6.16277517906383e-08
1605 6.1622500879821e-08
1606 6.15803443793084e-08
1607 6.15718178664793e-08
1608 6.15333064502011e-08
1609 6.15223072486515e-08
1610 6.14840729440402e-08
1611 6.14715389701814e-08
1612 6.14372979157451e-08
1613 6.14269595189398e-08
1614 6.13885760003541e-08
1615 6.13797084270118e-08
1616 6.13400104043649e-08
1617 6.13310859876037e-08
1618 6.12946351452592e-08
1619 6.12810140410147e-08
1620 6.12475830052972e-08
1621 6.12339050576338e-08
1622 6.12007227118738e-08
1623 6.11881105783141e-08
1624 6.11552906093493e-08
1625 6.11405113204455e-08
1626 6.11116846016557e-08
1627 6.10980563919838e-08
1628 6.10648811516512e-08
1629 6.10483894547542e-08
1630 6.10210335594275e-08
1631 6.10039450066324e-08
1632 6.09758856739973e-08
1633 6.09607724300076e-08
1634 6.09295796039078e-08
1635 6.09163066656038e-08
1636 6.08876078445064e-08
1637 6.08716987926528e-08
1638 6.08421757419819e-08
1639 6.08265651180773e-08
1640 6.08018666525822e-08
1641 6.078364123141e-08
1642 6.07554895282192e-08
1643 6.07390404638863e-08
1644 6.0715308336512e-08
1645 6.06965215865785e-08
1646 6.06714039008693e-08
1647 6.06544929837582e-08
1648 6.06297732019812e-08
1649 6.06107306566628e-08
1650 6.05879506565543e-08
1651 6.05707981549131e-08
1652 6.05469310244189e-08
1653 6.05264816044837e-08
1654 6.05044192525384e-08
1655 6.04864851538878e-08
1656 6.04629804001888e-08
1657 6.04413727955944e-08
1658 6.04234813295079e-08
1659 6.04042966756424e-08
1660 6.03835488277582e-08
1661 6.03597385406829e-08
1662 6.03420104994257e-08
1663 6.03229466378252e-08
1664 6.0299932158614e-08
1665 6.0279646163508e-08
1666 6.02614491640452e-08
1667 6.02439484964634e-08
1668 6.02205716404569e-08
1669 6.02032841356959e-08
1670 6.01843836989246e-08
1671 6.01642682340753e-08
1672 6.01438401304222e-08
1673 6.01254157572839e-08
1674 6.01069061190174e-08
1675 6.00883467427593e-08
1676 6.00688636609448e-08
1677 6.00493663682755e-08
1678 6.0030082238427e-08
1679 6.00128089445207e-08
1680 5.99933613898429e-08
1681 5.99736367234982e-08
1682 5.99579124127558e-08
1683 5.99373350951282e-08
1684 5.9917660166775e-08
1685 5.99006568791083e-08
1686 5.98835683263133e-08
1687 5.98635168103101e-08
1688 5.98469398482848e-08
1689 5.98281602037787e-08
1690 5.98096363546574e-08
1691 5.97894072029703e-08
1692 5.97733844642789e-08
1693 5.97566511828518e-08
1694 5.97378502220636e-08
1695 5.9719141631831e-08
1696 5.9701875443352e-08
1697 5.96836713384619e-08
1698 5.9669169161225e-08
1699 5.96474691860749e-08
1700 5.96307856426392e-08
1701 5.96157292420685e-08
1702 5.95950311321758e-08
1703 5.95781379786331e-08
1704 5.95606834963291e-08
1705 5.95454636709292e-08
1706 5.95223283994528e-08
1707 5.95071760756127e-08
1708 5.9490460557754e-08
1709 5.94741749182504e-08
1710 5.94559672606465e-08
1711 5.94377596030427e-08
1712 5.94231579498228e-08
1713 5.94017990351858e-08
1714 5.93871334331197e-08
1715 5.93683573413273e-08
1716 5.93503592938305e-08
1717 5.93330256037916e-08
1718 5.93134679149898e-08
1719 5.92996585169203e-08
1720 5.92756634887337e-08
1721 5.92610795990822e-08
1722 5.92416498079729e-08
1723 5.92212145988924e-08
1724 5.92025521939377e-08
1725 5.91865756405241e-08
1726 5.91629820689832e-08
1727 5.91457194332179e-08
1728 5.91306346109377e-08
1729 5.91134394767323e-08
1730 5.90930859800665e-08
1731 5.90741464634448e-08
1732 5.90537112543643e-08
1733 5.90362532193467e-08
1734 5.90148765411413e-08
1735 5.89985091892231e-08
1736 5.89781627979846e-08
1737 5.89587898502941e-08
1738 5.89413282625628e-08
1739 5.89224669056421e-08
1740 5.89023230190833e-08
1741 5.88845487925482e-08
1742 5.88660427069954e-08
1743 5.88473270113354e-08
1744 5.88323345596109e-08
1745 5.88178785676519e-08
1746 5.88017385894091e-08
1747 5.87855204514653e-08
1748 5.87734803048079e-08
1749 5.8760825538684e-08
1750 5.87456518985618e-08
1751 5.87327697587625e-08
1752 5.8723152562834e-08
1753 5.8709691330705e-08
1754 5.86996691254171e-08
1755 5.86879878028412e-08
1756 5.86768997834497e-08
1757 5.86660284795926e-08
1758 5.86533950297508e-08
1759 5.86415609404867e-08
1760 5.86272470570748e-08
1761 5.86163153570851e-08
1762 5.85998840563207e-08
1763 5.85871582359232e-08
1764 5.85741553038588e-08
1765 5.8559773918887e-08
1766 5.85451367385303e-08
1767 5.85298174371474e-08
1768 5.85155497390133e-08
1769 5.85014312548537e-08
1770 5.84853800944529e-08
1771 5.84704977768524e-08
1772 5.84536188341644e-08
1773 5.84379087342768e-08
1774 5.84258543767646e-08
1775 5.84085277921531e-08
1776 5.83913148943793e-08
1777 5.83739634407721e-08
1778 5.83616710514434e-08
1779 5.83494461636747e-08
1780 5.83311603463699e-08
1781 5.8315684725585e-08
1782 5.82956296568682e-08
1783 5.82782568869789e-08
1784 5.82643657764947e-08
1785 5.82506842761177e-08
1786 5.82333434806515e-08
1787 5.82158179440739e-08
1788 5.82012873451276e-08
1789 5.81869947779978e-08
1790 5.8171828243303e-08
1791 5.81526222731554e-08
1792 5.81359245188651e-08
1793 5.81210493066919e-08
1794 5.81055665804797e-08
1795 5.80892169921299e-08
1796 5.80724943688438e-08
1797 5.80564432084429e-08
1798 5.80419765583429e-08
1799 5.80255203885827e-08
1800 5.80087942125829e-08
1801 5.7992544100216e-08
1802 5.7977988632274e-08
1803 5.79585979210151e-08
1804 5.79446535198258e-08
1805 5.79304213488285e-08
1806 5.79153862645398e-08
1807 5.7900507499653e-08
1808 5.78817100915785e-08
1809 5.78637333603638e-08
1810 5.7848566825669e-08
1811 5.78325973776828e-08
1812 5.78204257806192e-08
1813 5.7803969610859e-08
1814 5.77846961391515e-08
1815 5.77690073555459e-08
1816 5.77542351720695e-08
1817 5.77409089430603e-08
1818 5.77235290677436e-08
1819 5.77061243234311e-08
1820 5.76933771867516e-08
1821 5.76787684281044e-08
1822 5.7663829267085e-08
1823 5.76470036151022e-08
1824 5.76301530941237e-08
1825 5.76144145725266e-08
1826 5.76005199093288e-08
1827 5.75826781812339e-08
1828 5.75686982529078e-08
1829 5.75531124979989e-08
1830 5.75375693756541e-08
1831 5.75227439014725e-08
1832 5.75082950149408e-08
1833 5.74882470516513e-08
1834 5.74747325288172e-08
1835 5.74624294813475e-08
1836 5.74474299241956e-08
1837 5.74333718361686e-08
1838 5.74145460063846e-08
1839 5.73992267050016e-08
1840 5.73856517860349e-08
1841 5.73681582238805e-08
1842 5.73548035731619e-08
1843 5.7337448566841e-08
1844 5.73207259435549e-08
1845 5.7307296685849e-08
1846 5.72951570632085e-08
1847 5.72773295459683e-08
1848 5.72626177586244e-08
1849 5.72483997984818e-08
1850 5.72286893429919e-08
1851 5.72168836754372e-08
1852 5.7201500425208e-08
1853 5.7186667845599e-08
1854 5.71713911767802e-08
1855 5.71552725148194e-08
1856 5.7142070630789e-08
1857 5.71284530792582e-08
1858 5.71094354029356e-08
1859 5.7096389838307e-08
1860 5.70836711233369e-08
1861 5.706505490366e-08
1862 5.70517109110824e-08
1863 5.70354394824335e-08
1864 5.70226603713309e-08
1865 5.70122544729657e-08
1866 5.69951517093159e-08
1867 5.69841596131937e-08
1868 5.6973018303097e-08
1869 5.69577167652824e-08
1870 5.69460638644159e-08
1871 5.69364715374832e-08
1872 5.69263391980712e-08
1873 5.69164058106253e-08
1874 5.69053248966611e-08
1875 5.68909896969672e-08
1876 5.68813298684745e-08
1877 5.68632430031357e-08
1878 5.68491920205361e-08
1879 5.68381786081318e-08
1880 5.68205607009986e-08
1881 5.68080267271398e-08
1882 5.67944837825962e-08
1883 5.67738638324045e-08
1884 5.67597062683944e-08
1885 5.67459856881669e-08
1886 5.67299771603302e-08
1887 5.67141782426006e-08
1888 5.670042568795e-08
1889 5.66809426061354e-08
1890 5.66651294775511e-08
1891 5.66476607843924e-08
1892 5.66336311180748e-08
1893 5.66152635883554e-08
1894 5.6597958320026e-08
1895 5.65814488595606e-08
1896 5.65656321782626e-08
1897 5.65504336691447e-08
1898 5.65332278767983e-08
1899 5.65192550538995e-08
1900 5.65021665011045e-08
1901 5.64863817942296e-08
1902 5.64748035003504e-08
1903 5.64563720217848e-08
1904 5.64459021745733e-08
1905 5.64329631913552e-08
1906 5.64154234439229e-08
1907 5.64022926141661e-08
1908 5.63858542079743e-08
1909 5.63736008984961e-08
1910 5.63575035528174e-08
1911 5.63458826263741e-08
1912 5.63323609981126e-08
1913 5.63141000498035e-08
1914 5.63006530285293e-08
1915 5.62869466591565e-08
1916 5.62725155361932e-08
1917 5.62580027008153e-08
1918 5.62451241137296e-08
1919 5.62311512908309e-08
1920 5.62143718241259e-08
1921 5.62031488016146e-08
1922 5.61870585613633e-08
1923 5.61757431682963e-08
1924 5.61584840852447e-08
1925 5.6147342775148e-08
1926 5.61280799615815e-08
1927 5.61197879278552e-08
1928 5.6102535950231e-08
1929 5.6091618461096e-08
1930 5.60716628683622e-08
1931 5.60658506287837e-08
1932 5.60446942188264e-08
1933 5.60375070790542e-08
1934 5.60160273721522e-08
1935 5.60100872348812e-08
1936 5.5985236002698e-08
1937 5.59860637849852e-08
1938 5.59587185477994e-08
1939 5.59583384074358e-08
1940 5.592413643285e-08
1941 5.5936634879572e-08
1942 5.58936079642081e-08
1943 5.59094956997797e-08
1944 5.58678756590325e-08
1945 5.58860300259312e-08
1946 5.58326149757704e-08
1947 5.58634312142203e-08
1948 5.58036852282839e-08
1949 5.58344197543192e-08
1950 5.57729507022486e-08
1951 5.58102790648718e-08
1952 5.57454349348063e-08
1953 5.57840209580718e-08
1954 5.57188499783479e-08
1955 5.57584307614434e-08
1956 5.56908581472726e-08
1957 5.57322152872075e-08
1958 5.56613919400206e-08
1959 5.57061241579504e-08
1960 5.56381571925613e-08
1961 5.5680235533373e-08
1962 5.56124390982404e-08
1963 5.56529506923198e-08
1964 5.55867245566333e-08
1965 5.56280426167177e-08
1966 5.55638486332555e-08
1967 5.55997594631208e-08
1968 5.55412853486814e-08
1969 5.55711174854423e-08
1970 5.55187682493852e-08
1971 5.55436834304146e-08
1972 5.5494812301049e-08
1973 5.5515705810194e-08
1974 5.54763417426329e-08
1975 5.54891776971544e-08
1976 5.54498633675848e-08
1977 5.54616974568489e-08
1978 5.54297052701713e-08
1979 5.54370167549223e-08
1980 5.54084493842311e-08
1981 5.54086518889108e-08
1982 5.53854810902976e-08
1983 5.53848984452543e-08
1984 5.53635892686088e-08
1985 5.53603811681569e-08
1986 5.53380985479635e-08
1987 5.53343113551819e-08
1988 5.53156240812314e-08
1989 5.53073462583598e-08
1990 5.52892238658842e-08
1991 5.52797629893576e-08
1992 5.52658150354546e-08
1993 5.52573666823264e-08
1994 5.52430563516282e-08
1995 5.52291758992851e-08
1996 5.52171286472003e-08
1997 5.52034471468232e-08
1998 5.51921068847605e-08
1999 5.517965817603e-08
2000 5.5169792290144e-08
2001 5.51570487061781e-08
2002 5.51418182226371e-08
2003 5.51313377172846e-08
2004 5.51166756679322e-08
2005 5.51057297570878e-08
2006 5.5091923911732e-08
2007 5.50773719965036e-08
2008 5.50669305710016e-08
2009 5.50530607767996e-08
2010 5.50420864442458e-08
2011 5.5030525913935e-08
2012 5.50201413318518e-08
2013 5.50061116655343e-08
2014 5.49943699468258e-08
2015 5.49819674233731e-08
2016 5.49682077632951e-08
2017 5.49579723951865e-08
2018 5.49433067931204e-08
2019 5.49321264031732e-08
2020 5.49215037892736e-08
2021 5.49082024292602e-08
2022 5.48940732869596e-08
2023 5.48819940604517e-08
2024 5.48716556636464e-08
2025 5.48601768457502e-08
2026 5.48441967396229e-08
2027 5.48313003889689e-08
2028 5.48206635642146e-08
2029 5.48091492191816e-08
2030 5.47983241006023e-08
2031 5.47843264087078e-08
2032 5.47718173038447e-08
2033 5.47600293998585e-08
2034 5.47490941471551e-08
2035 5.47349650048545e-08
2036 5.4722704589949e-08
2037 5.47116307814122e-08
2038 5.47004823658881e-08
2039 5.46862821693139e-08
2040 5.46785408062078e-08
2041 5.46611431673227e-08
2042 5.46506022658377e-08
2043 5.46418554847605e-08
2044 5.46259251166248e-08
2045 5.46168941184533e-08
2046 5.46028466885673e-08
2047 5.45919931482786e-08
2048 5.45814593522209e-08
2049 5.4565276741414e-08
2050 5.45534319940089e-08
2051 5.45443086252817e-08
2052 5.45297638154807e-08
2053 5.4518721981367e-08
2054 5.45076623836849e-08
2055 5.4495107093544e-08
2056 5.44825908832536e-08
2057 5.44728600004873e-08
2058 5.44619176423566e-08
2059 5.4447689024073e-08
2060 5.44372724675668e-08
2061 5.44252465317641e-08
2062 5.44103322397405e-08
2063 5.44010667624661e-08
2064 5.43906146788231e-08
2065 5.43770362071427e-08
2066 5.43638591921081e-08
2067 5.43515419337837e-08
2068 5.43441629474728e-08
2069 5.43317639767338e-08
2070 5.43174785150313e-08
2071 5.43076943415599e-08
2072 5.4293657569815e-08
2073 5.42831486427531e-08
2074 5.42744480469537e-08
2075 5.42623688204458e-08
2076 5.42504530187671e-08
2077 5.42386686674945e-08
2078 5.4227889734193e-08
2079 5.42159952487964e-08
2080 5.42062004171839e-08
2081 5.41902025474883e-08
2082 5.41799423103839e-08
2083 5.41703677470196e-08
2084 5.41594538105983e-08
2085 5.41481739446681e-08
2086 5.41369225004473e-08
2087 5.4125411708128e-08
2088 5.41156666145071e-08
2089 5.41033386980416e-08
2090 5.40915046087775e-08
2091 5.40751337041456e-08
2092 5.40690479056138e-08
2093 5.40561728712419e-08
2094 5.40460973752488e-08
2095 5.40353966016482e-08
2096 5.40214770694547e-08
2097 5.4014062556007e-08
2098 5.40030740125985e-08
2099 5.39905933294449e-08
2100 5.39809335009522e-08
2101 5.39707478708351e-08
2102 5.39563167478718e-08
2103 5.39471862737173e-08
2104 5.3936837218771e-08
2105 5.39220579298672e-08
2106 5.39119362485962e-08
2107 5.39026174806168e-08
2108 5.38921440806917e-08
2109 5.3877496242194e-08
2110 5.38666178329095e-08
2111 5.38544497885596e-08
2112 5.38458486687432e-08
2113 5.38337481259532e-08
2114 5.38239994796186e-08
2115 5.38099520497326e-08
2116 5.3796895826963e-08
2117 5.37870903372095e-08
2118 5.37760236341001e-08
2119 5.37625695073984e-08
2120 5.37522346633068e-08
2121 5.37365316688465e-08
2122 5.37287121460395e-08
2123 5.37181712445545e-08
2124 5.37026316749234e-08
2125 5.3690811796514e-08
2126 5.3681620926227e-08
2127 5.36698863129459e-08
2128 5.36546842511143e-08
2129 5.36422284369564e-08
2130 5.36299502584825e-08
2131 5.36169579845591e-08
2132 5.36041007137555e-08
2133 5.35913784460718e-08
2134 5.35786064403965e-08
2135 5.35697175507721e-08
2136 5.35537445500722e-08
2137 5.35415232150172e-08
2138 5.35313020577632e-08
2139 5.35162989478977e-08
2140 5.35006030588647e-08
2141 5.34920197026167e-08
2142 5.3477162254012e-08
2143 5.34686250830418e-08
2144 5.34513624472766e-08
2145 5.34407753605137e-08
2146 5.34279429587059e-08
2147 5.34154409592702e-08
2148 5.34037063459891e-08
2149 5.33929842561065e-08
2150 5.3379086040195e-08
2151 5.33653192746897e-08
2152 5.33525366108734e-08
2153 5.33420490000935e-08
2154 5.3325209137256e-08
2155 5.331465757763e-08
2156 5.33009405501161e-08
2157 5.32884563142488e-08
2158 5.32752864046415e-08
2159 5.32620205717649e-08
2160 5.32448964918331e-08
2161 5.32343271686386e-08
2162 5.32172421685573e-08
2163 5.32022745858285e-08
2164 5.31881383381005e-08
2165 5.31720871776997e-08
2166 5.31575210516166e-08
2167 5.31444577234197e-08
2168 5.31300763384479e-08
2169 5.31150554650139e-08
2170 5.31012531723718e-08
2171 5.30837773737858e-08
2172 5.30682982002872e-08
2173 5.30518775576638e-08
2174 5.30371657703199e-08
2175 5.30225001682538e-08
2176 5.30082324701198e-08
2177 5.29906358792687e-08
2178 5.29754906608559e-08
2179 5.29604129440031e-08
2180 5.29454133868512e-08
2181 5.29314938546577e-08
2182 5.29136663374175e-08
2183 5.28995656168263e-08
2184 5.28821679779412e-08
2185 5.2865900102006e-08
2186 5.28500727625669e-08
2187 5.28337480432128e-08
2188 5.28212780182002e-08
2189 5.28063281990399e-08
2190 5.2790454674323e-08
2191 5.27734655975109e-08
2192 5.27553112306123e-08
2193 5.27398817951052e-08
2194 5.2725273036458e-08
2195 5.27138546146944e-08
2196 5.26953733981372e-08
2197 5.26799261990618e-08
2198 5.2664660188384e-08
2199 5.26483354690299e-08
2200 5.26315062643334e-08
2201 5.26187946547907e-08
2202 5.26006651568878e-08
2203 5.25868877332414e-08
2204 5.25716323807046e-08
2205 5.25559009645349e-08
2206 5.25400984940916e-08
2207 5.25241503623874e-08
2208 5.2508781323013e-08
2209 5.24942080915025e-08
2210 5.24774286247975e-08
2211 5.24605212604001e-08
2212 5.24459125017529e-08
2213 5.24308738647505e-08
2214 5.24139771584942e-08
2215 5.2395773053604e-08
2216 5.23820986586543e-08
2217 5.23670422580835e-08
2218 5.23502983185153e-08
2219 5.23393026696795e-08
2220 5.2320050514254e-08
2221 5.23047560818668e-08
2222 5.22861896001814e-08
2223 5.22705363437126e-08
2224 5.22532204172421e-08
2225 5.22419973947308e-08
2226 5.22208019049231e-08
2227 5.22064098618102e-08
2228 5.21882412840569e-08
2229 5.21735756819908e-08
2230 5.21592617985789e-08
2231 5.21375511652877e-08
2232 5.21254897023482e-08
2233 5.21063689973289e-08
2234 5.20925453884047e-08
2235 5.20727709840685e-08
2236 5.20568868012106e-08
2237 5.20400043058089e-08
2238 5.20226208777785e-08
2239 5.20058343056462e-08
2240 5.1987917970564e-08
2241 5.1969557546272e-08
2242 5.19521847763826e-08
2243 5.19330249915129e-08
2244 5.19159542022862e-08
2245 5.19013596544937e-08
2246 5.18810772121014e-08
2247 5.18635374646692e-08
2248 5.18463494358912e-08
2249 5.18295912854683e-08
2250 5.18097778012816e-08
2251 5.1787740318332e-08
2252 5.17707903213704e-08
2253 5.17524192389374e-08
2254 5.1733266559495e-08
2255 5.17118614595802e-08
2256 5.169297168095e-08
2257 5.16773610570453e-08
2258 5.16535010319785e-08
2259 5.16359186519821e-08
2260 5.16134903705279e-08
2261 5.15955953517278e-08
2262 5.15766593878197e-08
2263 5.15531723976892e-08
2264 5.15344886764524e-08
2265 5.151153104066e-08
2266 5.14906837167928e-08
2267 5.14694171727115e-08
2268 5.14517601857278e-08
2269 5.14288451824996e-08
2270 5.1409351442544e-08
2271 5.13872677743166e-08
2272 5.13640365795709e-08
2273 5.13424076586944e-08
2274 5.13224271969648e-08
2275 5.12991178425182e-08
2276 5.12784659179033e-08
2277 5.12578282041432e-08
2278 5.12347497760857e-08
2279 5.12119200379857e-08
2280 5.11883051501627e-08
2281 5.1165361725225e-08
2282 5.11420559234921e-08
2283 5.11238660294566e-08
2284 5.11005104897322e-08
2285 5.10769595507554e-08
2286 5.10547195631261e-08
2287 5.10331794600916e-08
2288 5.10099091854954e-08
2289 5.09864399589333e-08
2290 5.09617983368571e-08
2291 5.09396329562151e-08
2292 5.09207325194438e-08
2293 5.08953519329225e-08
2294 5.0870703205419e-08
2295 5.08502360219154e-08
2296 5.08280244559955e-08
2297 5.08043811464631e-08
2298 5.07795832049851e-08
2299 5.07589348330839e-08
2300 5.07336750388276e-08
2301 5.07095414548075e-08
2302 5.06897173124798e-08
2303 5.0666621120854e-08
2304 5.06420860801882e-08
2305 5.06185600102071e-08
2306 5.05945649820205e-08
2307 5.05697101971236e-08
2308 5.05500139524884e-08
2309 5.05277348850086e-08
2310 5.05006632067762e-08
2311 5.04781141330568e-08
2312 5.04557569058761e-08
2313 5.04332149375841e-08
2314 5.04052586336456e-08
2315 5.03850934308048e-08
2316 5.03630843695646e-08
2317 5.03345809477196e-08
2318 5.03104367055585e-08
2319 5.02879053954075e-08
2320 5.0265690276774e-08
2321 5.02419510439722e-08
2322 5.02198460594627e-08
2323 5.01964265708921e-08
2324 5.01677774877862e-08
2325 5.01473920166973e-08
2326 5.01229955318649e-08
2327 5.00983148299383e-08
2328 5.00726891061731e-08
2329 5.00498948952099e-08
2330 5.00238321876623e-08
2331 5.00013008775113e-08
2332 4.99793344488353e-08
2333 4.99571690681933e-08
2334 4.9929376189084e-08
2335 4.99030576861514e-08
2336 4.98834111795077e-08
2337 4.98572454432633e-08
2338 4.98321384156952e-08
2339 4.98107617374899e-08
2340 4.97865784154783e-08
2341 4.97608425575891e-08
2342 4.97399668120124e-08
2343 4.97132681687162e-08
2344 4.96879017930496e-08
2345 4.96609828815053e-08
2346 4.96393752769109e-08
2347 4.96151741913309e-08
2348 4.95901488761774e-08
2349 4.95668288635898e-08
2350 4.95397145527932e-08
2351 4.95170127123856e-08
2352 4.94912164583639e-08
2353 4.94668732642367e-08
2354 4.94436669384868e-08
2355 4.94166130238227e-08
2356 4.93922485134135e-08
2357 4.93701648451861e-08
2358 4.93437823934073e-08
2359 4.93199294737678e-08
2360 4.92969824961165e-08
2361 4.92692002751483e-08
2362 4.92448108957433e-08
2363 4.92212670621939e-08
2364 4.91954068593259e-08
2365 4.9170463256587e-08
2366 4.91440630412399e-08
2367 4.91220326637176e-08
2368 4.90966129973458e-08
2369 4.90708771394566e-08
2370 4.90475464687279e-08
2371 4.90199738578667e-08
2372 4.89977871609426e-08
2373 4.89687934646099e-08
2374 4.89454805574496e-08
2375 4.89217804044983e-08
2376 4.88981974910985e-08
2377 4.88707136980793e-08
2378 4.88454894309598e-08
2379 4.8817625497577e-08
2380 4.87927671599664e-08
2381 4.87674221005818e-08
2382 4.87440203755796e-08
2383 4.87169202756377e-08
2384 4.86931206467034e-08
2385 4.86687774525763e-08
2386 4.8641183525433e-08
2387 4.86189541959448e-08
2388 4.85908948633096e-08
2389 4.85693476548477e-08
2390 4.8540364616656e-08
2391 4.85145257300701e-08
2392 4.84909925546617e-08
2393 4.84613806861489e-08
2394 4.84422955082664e-08
2395 4.84135824763143e-08
2396 4.83859317057522e-08
2397 4.83630522296608e-08
2398 4.83372062376475e-08
2399 4.83122732930497e-08
2400 4.82860613715275e-08
2401 4.82627093845167e-08
2402 4.8234589655749e-08
2403 4.82117741285037e-08
2404 4.81832067578125e-08
2405 4.81639617078145e-08
2406 4.81336925872711e-08
2407 4.81076192215824e-08
2408 4.80834074778613e-08
2409 4.80590429674521e-08
2410 4.80331934227252e-08
2411 4.80056350227187e-08
2412 4.79788688778626e-08
2413 4.79543409426242e-08
2414 4.79284096854826e-08
2415 4.79073314352263e-08
2416 4.78811728044093e-08
2417 4.78540940207495e-08
2418 4.78300989925629e-08
2419 4.78038586493312e-08
2420 4.77770960571888e-08
2421 4.7754198817529e-08
2422 4.77281041355582e-08
2423 4.77039669988244e-08
2424 4.76780925851017e-08
2425 4.76493724477223e-08
2426 4.76255159753691e-08
2427 4.76034784924195e-08
2428 4.7577646711261e-08
2429 4.75537476063437e-08
2430 4.75289780865751e-08
2431 4.75026240565057e-08
2432 4.74802384076156e-08
2433 4.74555506002616e-08
2434 4.74326355970334e-08
2435 4.74089674185052e-08
2436 4.73800447764461e-08
2437 4.73566679204396e-08
2438 4.73342964824042e-08
2439 4.73105856713119e-08
2440 4.72857912825475e-08
2441 4.72570356180313e-08
2442 4.72371937121352e-08
2443 4.72123602435204e-08
2444 4.71886067998639e-08
2445 4.71632191079152e-08
2446 4.71369645538289e-08
2447 4.71152894476745e-08
2448 4.70914294226077e-08
2449 4.70665959539929e-08
2450 4.70453187517705e-08
2451 4.70209080560835e-08
2452 4.69939642755435e-08
2453 4.69736427533007e-08
2454 4.69475303077616e-08
2455 4.69254075596837e-08
2456 4.69017713555786e-08
2457 4.68776697459816e-08
2458 4.6855621604891e-08
2459 4.68298573252923e-08
2460 4.68028211741967e-08
2461 4.6781856610778e-08
2462 4.67584762020579e-08
2463 4.6736641223788e-08
2464 4.67109444457492e-08
2465 4.66889034100859e-08
2466 4.66662690712383e-08
2467 4.66409844079863e-08
2468 4.66220200223688e-08
2469 4.65942910921058e-08
2470 4.65709000252446e-08
2471 4.65464005117155e-08
2472 4.65257485871007e-08
2473 4.65051144260542e-08
2474 4.64785792075872e-08
2475 4.64557174950642e-08
2476 4.64338860695079e-08
2477 4.64122820176271e-08
2478 4.63861411503785e-08
2479 4.63646578907628e-08
2480 4.63415332774275e-08
2481 4.63184051113785e-08
2482 4.62933869016524e-08
2483 4.62700668890648e-08
2484 4.62470950424176e-08
2485 4.62237181864111e-08
2486 4.62025546710265e-08
2487 4.61808227214533e-08
2488 4.61547813301877e-08
2489 4.61319693556561e-08
2490 4.61092781733896e-08
2491 4.60904061583278e-08
2492 4.60626488063554e-08
2493 4.60420750414414e-08
2494 4.60183180450713e-08
2495 4.59937083974182e-08
2496 4.5972662121585e-08
2497 4.5949434479553e-08
2498 4.59274538400223e-08
2499 4.59056757051712e-08
2500 4.58781599377289e-08
2501 4.58570745820452e-08
2502 4.58339677322783e-08
2503 4.58105375855666e-08
2504 4.57891502492203e-08
2505 4.57680364718271e-08
2506 4.57454447655437e-08
2507 4.57205970860741e-08
2508 4.5698886452783e-08
2509 4.5674486415237e-08
2510 4.5650484281623e-08
2511 4.56321913588909e-08
2512 4.56049065178377e-08
2513 4.55828299550376e-08
2514 4.55646755881389e-08
2515 4.55408653010636e-08
2516 4.55180853009551e-08
2517 4.54932127524899e-08
2518 4.54719426556949e-08
2519 4.54549926587333e-08
2520 4.54300952412723e-08
2521 4.54054251974867e-08
2522 4.53863364668905e-08
2523 4.53614283912884e-08
2524 4.53444712888995e-08
2525 4.53217907647741e-08
2526 4.53002328981711e-08
2527 4.52756445668001e-08
2528 4.52522712635073e-08
2529 4.52275017437387e-08
2530 4.52083597224373e-08
2531 4.51864821116033e-08
2532 4.51656418931634e-08
2533 4.51438921800218e-08
2534 4.51184618555089e-08
2535 4.50967299059357e-08
2536 4.50725643474925e-08
2537 4.50524275663611e-08
2538 4.50304860066808e-08
2539 4.50038690757992e-08
2540 4.49816148773152e-08
2541 4.49546462277794e-08
2542 4.4933624820942e-08
2543 4.49122339318819e-08
2544 4.488725835472e-08
2545 4.48640804506795e-08
2546 4.48434640532014e-08
2547 4.4818172284522e-08
2548 4.47921877366753e-08
2549 4.47687220628268e-08
2550 4.47441195206011e-08
2551 4.47223911237415e-08
2552 4.46981012203196e-08
2553 4.46745325177744e-08
2554 4.46497736561469e-08
2555 4.46223076266961e-08
2556 4.45998509235324e-08
2557 4.45777423863092e-08
2558 4.45496191048278e-08
2559 4.45264127790779e-08
2560 4.44974617153093e-08
2561 4.44744827632348e-08
2562 4.44521219833405e-08
2563 4.44285639389363e-08
2564 4.44004726318781e-08
2565 4.43756178469812e-08
2566 4.43504255542848e-08
2567 4.43239294156683e-08
2568 4.42991812121818e-08
2569 4.42709584547174e-08
2570 4.42452048332598e-08
2571 4.42247340970425e-08
2572 4.41892922253828e-08
2573 4.41698446707051e-08
2574 4.41435332731999e-08
2575 4.41175735943489e-08
2576 4.40900542741929e-08
2577 4.40659704281643e-08
2578 4.4036905677558e-08
2579 4.40119656275328e-08
2580 4.39838707677609e-08
2581 4.39569944887808e-08
2582 4.39296421461677e-08
2583 4.39037926014407e-08
2584 4.38741203367954e-08
2585 4.38492975263216e-08
2586 4.38222258480891e-08
2587 4.37958505017377e-08
2588 4.37667537767084e-08
2589 4.37422293941836e-08
2590 4.37111467022078e-08
2591 4.36856986141265e-08
2592 4.36592380026468e-08
2593 4.36276543780423e-08
2594 4.3602096155837e-08
2595 4.35744702542706e-08
2596 4.35433022971665e-08
2597 4.3520412162934e-08
2598 4.34920117697857e-08
2599 4.34620872624691e-08
2600 4.34325393428026e-08
2601 4.34058939902116e-08
2602 4.33779199227047e-08
2603 4.33458495763261e-08
2604 4.33207993921769e-08
2605 4.32918980663999e-08
2606 4.32652065285311e-08
2607 4.32323865595663e-08
2608 4.32054392263126e-08
2609 4.31784492604947e-08
2610 4.31496864905512e-08
2611 4.31192468397512e-08
2612 4.30923883243395e-08
2613 4.30609965462736e-08
2614 4.30312567800684e-08
2615 4.30046185329047e-08
2616 4.29784030586688e-08
2617 4.29500879306488e-08
2618 4.29144684233052e-08
2619 4.28848920819291e-08
2620 4.28563033949558e-08
2621 4.28283541964447e-08
2622 4.27998827490228e-08
2623 4.27649702317012e-08
2624 4.27392095048162e-08
2625 4.27100594890817e-08
2626 4.26771151751382e-08
2627 4.26450981194648e-08
2628 4.26205808423674e-08
2629 4.25895088085326e-08
2630 4.25607673548711e-08
2631 4.25308108731315e-08
2632 4.25019450744912e-08
2633 4.24669579501824e-08
2634 4.24376871421828e-08
2635 4.24081143535204e-08
2636 4.23798951487697e-08
2637 4.23507771074583e-08
2638 4.23153565520806e-08
2639 4.22870094496375e-08
2640 4.225769245636e-08
2641 4.22239203601293e-08
2642 4.21983301635009e-08
2643 4.2165567037955e-08
2644 4.21371098013879e-08
2645 4.21022789964809e-08
2646 4.20739674211745e-08
2647 4.2044330683666e-08
2648 4.20124735001082e-08
2649 4.19795114225963e-08
2650 4.19527559358812e-08
2651 4.19194883249929e-08
2652 4.18886685338293e-08
2653 4.18548538050345e-08
2654 4.18242187549822e-08
2655 4.17962660037574e-08
2656 4.17646681682982e-08
2657 4.17337986391431e-08
2658 4.16995504792794e-08
2659 4.16724645901922e-08
2660 4.16414600579174e-08
2661 4.16072758468999e-08
2662 4.15768113271042e-08
2663 4.15495797767562e-08
2664 4.15185645863403e-08
2665 4.14846930141266e-08
2666 4.14551948324515e-08
2667 4.142508203131e-08
2668 4.13940846044625e-08
2669 4.13621776829132e-08
2670 4.1328942046448e-08
2671 4.13032239521272e-08
2672 4.12707308328208e-08
2673 4.12362801682775e-08
2674 4.12068743571581e-08
2675 4.11770706421066e-08
2676 4.11450287174375e-08
2677 4.1115679749737e-08
2678 4.10849771981248e-08
2679 4.10543101736494e-08
2680 4.10239309189819e-08
2681 4.0996045669317e-08
2682 4.09666611744797e-08
2683 4.09359444120128e-08
2684 4.09027229864023e-08
2685 4.08718463518198e-08
2686 4.0839690740313e-08
2687 4.08145659491765e-08
2688 4.07829290338668e-08
2689 4.07488336406914e-08
2690 4.07244868938506e-08
2691 4.06923916784763e-08
2692 4.06642683969949e-08
2693 4.06342017811312e-08
2694 4.06021207766116e-08
2695 4.05768467715006e-08
2696 4.05451245910626e-08
2697 4.05170190731496e-08
2698 4.04873681247864e-08
2699 4.04582038981971e-08
2700 4.04297431089162e-08
2701 4.04010833676693e-08
2702 4.037061529516e-08
2703 4.03430426842988e-08
2704 4.03139104321326e-08
2705 4.02858653103522e-08
2706 4.02574862334859e-08
2707 4.02285955658499e-08
2708 4.02014386224891e-08
2709 4.01749460365863e-08
2710 4.01474409272851e-08
2711 4.01174737874044e-08
2712 4.0090899489087e-08
2713 4.00609359019199e-08
2714 4.00355482099712e-08
2715 4.00080786278068e-08
2716 3.99801223238683e-08
2717 3.99506774328984e-08
2718 3.99244548532351e-08
2719 3.99008506235532e-08
2720 3.98730755080123e-08
2721 3.984487406683e-08
2722 3.98188220174234e-08
2723 3.97925141726319e-08
2724 3.97685866460051e-08
2725 3.97370598648195e-08
2726 3.97121553419311e-08
2727 3.968626316464e-08
2728 3.96595609686301e-08
2729 3.96325283702481e-08
2730 3.96078938535993e-08
2731 3.95830852539802e-08
2732 3.95582056000876e-08
2733 3.95293291433063e-08
2734 3.95030177458011e-08
2735 3.9481207636527e-08
2736 3.9454782552184e-08
2737 3.94280519344647e-08
2738 3.94034778139485e-08
2739 3.93807688681136e-08
2740 3.93543651000527e-08
2741 3.932643366511e-08
2742 3.93036678758563e-08
2743 3.92802874671361e-08
2744 3.92560295381372e-08
2745 3.92290750994562e-08
2746 3.92056982434497e-08
2747 3.91792447373973e-08
2748 3.91560277535064e-08
2749 3.9132505236239e-08
2750 3.91102190633319e-08
2751 3.90837762154206e-08
2752 3.90590244592204e-08
2753 3.90386745152682e-08
2754 3.9012462593746e-08
2755 3.89929866173588e-08
2756 3.8964536486219e-08
2757 3.89425274249788e-08
2758 3.89192784666648e-08
2759 3.88949885632428e-08
2760 3.88736971501658e-08
2761 3.88504908244158e-08
2762 3.88260836814425e-08
2763 3.88024226083417e-08
2764 3.87802465695586e-08
2765 3.875996767988e-08
2766 3.87362995013518e-08
2767 3.87137575330598e-08
2768 3.86914003058791e-08
2769 3.86654832595923e-08
2770 3.86449094946784e-08
2771 3.8624008880106e-08
2772 3.85976832717461e-08
2773 3.85770526634133e-08
2774 3.85559673077296e-08
2775 3.8534903268328e-08
2776 3.85130967117675e-08
2777 3.84888032556319e-08
2778 3.84649609941334e-08
2779 3.84481992909969e-08
2780 3.84248650675545e-08
2781 3.84034812839218e-08
2782 3.8381291034284e-08
2783 3.83616729493497e-08
2784 3.83371840939617e-08
2785 3.83175091656085e-08
2786 3.82971343526606e-08
2787 3.82730824810551e-08
2788 3.82546616606305e-08
2789 3.82303078083623e-08
2790 3.82113789498817e-08
2791 3.81900342460995e-08
2792 3.81683697980861e-08
2793 3.81489790868272e-08
2794 3.8127545565203e-08
2795 3.81066449506307e-08
2796 3.80868172555893e-08
2797 3.80690963197594e-08
2798 3.80437228386654e-08
2799 3.80263820431992e-08
2800 3.80055418247593e-08
2801 3.79858917654019e-08
2802 3.79616267309757e-08
2803 3.79436961850388e-08
2804 3.79240603365361e-08
2805 3.79047371268371e-08
2806 3.78818150181814e-08
2807 3.78637743381205e-08
2808 3.78433107073306e-08
2809 3.7824094079042e-08
2810 3.78024687108791e-08
2811 3.77829643127825e-08
2812 3.77635274162458e-08
2813 3.7746612946421e-08
2814 3.77268420947985e-08
2815 3.77075828339457e-08
2816 3.76893751763419e-08
2817 3.76693876091849e-08
2818 3.76465649765123e-08
2819 3.76305990812398e-08
2820 3.76132121004957e-08
2821 3.7592126744812e-08
2822 3.75710733635515e-08
2823 3.7554620746505e-08
2824 3.7534810815032e-08
2825 3.75185997825156e-08
2826 3.74974078454215e-08
2827 3.74797508584379e-08
2828 3.74619908427576e-08
2829 3.74405253467103e-08
2830 3.74240229916722e-08
2831 3.74063766628296e-08
2832 3.73861333002878e-08
2833 3.73673145759312e-08
2834 3.73512705209578e-08
2835 3.73332689207473e-08
2836 3.73144715126728e-08
2837 3.72959227945557e-08
2838 3.72809978443911e-08
2839 3.72593547126598e-08
2840 3.72425823513822e-08
2841 3.72250106295269e-08
2842 3.7206216774166e-08
2843 3.71906843099623e-08
2844 3.71720751957128e-08
2845 3.71581521108055e-08
2846 3.71402180121549e-08
2847 3.71270090226972e-08
2848 3.71085882022726e-08
2849 3.70936099614028e-08
2850 3.70792605508541e-08
2851 3.70664174909052e-08
2852 3.70504160684959e-08
2853 3.70347628120271e-08
2854 3.70210813116501e-08
2855 3.70051900233648e-08
2856 3.69880304162962e-08
2857 3.69712331860228e-08
2858 3.69544252976084e-08
2859 3.69371875308389e-08
2860 3.6922795487726e-08
2861 3.69034864888818e-08
2862 3.6887058740831e-08
2863 3.68731853939153e-08
2864 3.6856409479924e-08
2865 3.683652138875e-08
2866 3.68198946887333e-08
2867 3.6803616154657e-08
2868 3.67853907334847e-08
2869 3.67719756866336e-08
2870 3.67537857925981e-08
2871 3.6737638708928e-08
2872 3.67219605834634e-08
2873 3.67035752901756e-08
2874 3.66866181877867e-08
2875 3.66699381970648e-08
2876 3.66522421302307e-08
2877 3.66367984838689e-08
2878 3.66173864563279e-08
2879 3.66041668087291e-08
2880 3.65860195472578e-08
2881 3.65700145721348e-08
2882 3.65569263749421e-08
2883 3.65366403798362e-08
2884 3.65198609131312e-08
2885 3.65058170359589e-08
2886 3.6488472687779e-08
2887 3.64719490164589e-08
2888 3.64547751985356e-08
2889 3.64402801267261e-08
2890 3.64209711278818e-08
2891 3.64090553262031e-08
2892 3.63893093435763e-08
2893 3.63779548706589e-08
2894 3.63600314301493e-08
2895 3.63440229023126e-08
2896 3.63293182203961e-08
2897 3.63110146395229e-08
2898 3.62953436194857e-08
2899 3.62793954877816e-08
2900 3.62630423467181e-08
2901 3.62479504190105e-08
2902 3.62332741588034e-08
2903 3.62164698231027e-08
2904 3.62006247200952e-08
2905 3.61846090868312e-08
2906 3.61710164042961e-08
2907 3.61533842863082e-08
2908 3.61362992862269e-08
2909 3.61223619904649e-08
2910 3.61093555056868e-08
2911 3.60933185561407e-08
2912 3.60775942453984e-08
2913 3.60618059858098e-08
2914 3.60458436432509e-08
2915 3.60319489800531e-08
2916 3.60178233904662e-08
2917 3.60025786960705e-08
2918 3.59862717402848e-08
2919 3.59702809760165e-08
2920 3.59547840389496e-08
2921 3.59436569397076e-08
2922 3.59275489358879e-08
2923 3.59127412252747e-08
2924 3.589781627511e-08
2925 3.58814098433413e-08
2926 3.58677354483916e-08
2927 3.5852036006645e-08
2928 3.58396690103291e-08
2929 3.5826175803777e-08
2930 3.58113787513048e-08
2931 3.57960097119303e-08
2932 3.57813014773001e-08
2933 3.57686715801719e-08
2934 3.57528300298782e-08
2935 3.57383136417866e-08
2936 3.5724582403418e-08
2937 3.57096823222491e-08
2938 3.56966829428984e-08
2939 3.56826888037176e-08
2940 3.5670318254688e-08
2941 3.56565124093322e-08
2942 3.56406673063248e-08
2943 3.56280409619103e-08
2944 3.56144376212342e-08
2945 3.55992675338257e-08
2946 3.55878029267842e-08
2947 3.55740468194199e-08
2948 3.55626887937888e-08
2949 3.55452343114848e-08
2950 3.55342102409395e-08
2951 3.55195837187239e-08
2952 3.55039055932593e-08
2953 3.54930627111116e-08
2954 3.54813636249673e-08
2955 3.54661651158494e-08
2956 3.54549385406244e-08
2957 3.54397649005023e-08
2958 3.54271989522204e-08
2959 3.54159226390038e-08
2960 3.54001237212742e-08
2961 3.53900233562854e-08
2962 3.53764839644555e-08
2963 3.53634916905321e-08
2964 3.53486342419274e-08
2965 3.53380507078782e-08
2966 3.53251436990831e-08
2967 3.53100979566534e-08
2968 3.52992763907878e-08
2969 3.5285680155539e-08
2970 3.52716398310804e-08
2971 3.52614470955359e-08
2972 3.52451081653271e-08
2973 3.52341871234785e-08
2974 3.52204310161142e-08
2975 3.52088278532392e-08
2976 3.51956153110677e-08
2977 3.51828219891104e-08
2978 3.51716629154453e-08
2979 3.51591751268643e-08
2980 3.51441933332808e-08
2981 3.5131659359422e-08
2982 3.51195197367815e-08
2983 3.51075506443976e-08
2984 3.50988393904572e-08
2985 3.5082184268731e-08
2986 3.50692808126496e-08
2987 3.50590170228315e-08
2988 3.5045097490638e-08
2989 3.50316540220774e-08
2990 3.5021074040742e-08
2991 3.50086786227166e-08
2992 3.4996020303879e-08
2993 3.49832660617722e-08
2994 3.49729560866763e-08
2995 3.49584396985847e-08
2996 3.49482647266086e-08
2997 3.49335351756963e-08
2998 3.49208804095724e-08
2999 3.49094975149455e-08
3000 2.13769908441463e-08
3001 2.14176552049139e-08
3002 2.15042401663368e-08
3003 2.15619095911279e-08
3004 2.15762376853945e-08
3005 2.15779980550224e-08
3006 2.15764348610037e-08
3007 2.15741771114608e-08
3008 2.15716546847489e-08
3009 2.15688924498636e-08
3010 2.15658104707472e-08
3011 2.15628297439707e-08
3012 2.15597868447048e-08
3013 2.15566746675222e-08
3014 2.15533901837262e-08
3015 2.1549929840603e-08
3016 2.15464375230567e-08
3017 2.15433200168036e-08
3018 2.1539841910112e-08
3019 2.15362909727901e-08
3020 2.15327702335344e-08
3021 2.15292814687018e-08
3022 2.15255564484096e-08
3023 2.15222275556926e-08
3024 2.1518793857922e-08
3025 2.15152624605253e-08
3026 2.1511825210041e-08
3027 2.15081232823877e-08
3028 2.15045883322773e-08
3029 2.15012594395603e-08
3030 2.14976694223878e-08
3031 2.14942232901194e-08
3032 2.14908517648382e-08
3033 2.14873505655078e-08
3034 2.14838600243183e-08
3035 2.14806838982895e-08
3036 2.14771862516727e-08
3037 2.14737827519684e-08
3038 2.1470436095683e-08
3039 2.14670023979124e-08
3040 2.14637392303985e-08
3041 2.1460625276859e-08
3042 2.14572022372295e-08
3043 2.14540918364037e-08
3044 2.14508553142423e-08
3045 2.14474802362474e-08
3046 2.14442810175797e-08
3047 2.1441058706273e-08
3048 2.14380531105007e-08
3049 2.14348396809783e-08
3050 2.1431789676285e-08
3051 2.14284199273607e-08
3052 2.14253876862358e-08
3053 2.14224833428034e-08
3054 2.14193534020524e-08
3055 2.14162234613013e-08
3056 2.14130171372062e-08
3057 2.14101305573422e-08
3058 2.14069775239523e-08
3059 2.14038919921222e-08
3060 2.14009752141919e-08
3061 2.13980140273407e-08
3062 2.13949693517179e-08
3063 2.1392070337356e-08
3064 2.1389084281509e-08
3065 2.13860875675209e-08
3066 2.13831530260222e-08
3067 2.13801989445983e-08
3068 2.13773940771489e-08
3069 2.13744257848703e-08
3070 2.13714841379442e-08
3071 2.13686490724285e-08
3072 2.1365757163494e-08
3073 2.13627107115144e-08
3074 2.13599360421313e-08
3075 2.13571311746819e-08
3076 2.13542197258221e-08
3077 2.13513882130201e-08
3078 2.13485922273549e-08
3079 2.13459507847347e-08
3080 2.13432205242725e-08
3081 2.13401349924425e-08
3082 2.13377688851324e-08
3083 2.13349498068283e-08
3084 2.13319815145496e-08
3085 2.13294395479124e-08
3086 2.13265316517663e-08
3087 2.13239186308556e-08
3088 2.13212061339618e-08
3089 2.13185682440553e-08
3090 2.131583975995e-08
3091 2.13131539084088e-08
3092 2.13102708812585e-08
3093 2.130754417351e-08
3094 2.13050430630801e-08
3095 2.13023181316885e-08
3096 2.12996216220063e-08
3097 2.12971755786384e-08
3098 2.12943742639027e-08
3099 2.12917523612077e-08
3100 2.12893258577651e-08
3101 2.12867448112775e-08
3102 2.12840483015952e-08
3103 2.12815898237295e-08
3104 2.12789181830431e-08
3105 2.12762554241408e-08
3106 2.12735535853881e-08
3107 2.12710400404603e-08
3108 2.12685833389514e-08
3109 2.12659845288954e-08
3110 2.12634159169056e-08
3111 2.12607851324265e-08
3112 2.1258058424678e-08
3113 2.12558983747613e-08
3114 2.12532871302074e-08
3115 2.12507398344997e-08
3116 2.12481197081615e-08
3117 2.12454924763961e-08
3118 2.12430908419492e-08
3119 2.12406394695108e-08
3120 2.12384101416774e-08
3121 2.12357793571982e-08
3122 2.12331556781464e-08
3123 2.12309156921719e-08
3124 2.12283204348296e-08
3125 2.12259720910879e-08
3126 2.12236734853377e-08
3127 2.12210888861364e-08
3128 2.12185256032171e-08
3129 2.12160120582894e-08
3130 2.12137347688213e-08
3131 2.12113029363081e-08
3132 2.1208926170857e-08
3133 2.12061266324781e-08
3134 2.12039648062046e-08
3135 2.12015525136167e-08
3136 2.11991331156014e-08
3137 2.11967847718597e-08
3138 2.11944080064086e-08
3139 2.11920188064596e-08
3140 2.11897948076967e-08
3141 2.11870503363798e-08
3142 2.11849755515914e-08
3143 2.1182612996995e-08
3144 2.11803818928047e-08
3145 2.11779696002168e-08
3146 2.11755164514216e-08
3147 2.11732018584598e-08
3148 2.11706421282543e-08
3149 2.11685176054743e-08
3150 2.11661408400232e-08
3151 2.11640145408865e-08
3152 2.1161774554912e-08
3153 2.11593498278262e-08
3154 2.11570210240097e-08
3155 2.11548680795204e-08
3156 2.11524486815051e-08
3157 2.1149917373009e-08
3158 2.11480024603361e-08
3159 2.11455581933251e-08
3160 2.11431494534509e-08
3161 2.11410036143889e-08
3162 2.11384101334033e-08
3163 2.11365041025147e-08
3164 2.11341912859098e-08
3165 2.11317381371146e-08
3166 2.11296189434051e-08
3167 2.11273274430823e-08
3168 2.11251123261036e-08
3169 2.11227639823619e-08
3170 2.11204369549023e-08
3171 2.11183373011181e-08
3172 2.11160227081564e-08
3173 2.11139177253017e-08
3174 2.11116120141241e-08
3175 2.11094501878506e-08
3176 2.1107167569312e-08
3177 2.11051283116603e-08
3178 2.11027941787734e-08
3179 2.1100575509081e-08
3180 2.1098484737081e-08
3181 2.10962252111813e-08
3182 2.10938733147259e-08
3183 2.10918411625016e-08
3184 2.10894803842621e-08
3185 2.10873309924864e-08
3186 2.1085178047997e-08
3187 2.10833519531661e-08
3188 2.10810586764865e-08
3189 2.10788986265698e-08
3190 2.10766426533837e-08
3191 2.10745447759564e-08
3192 2.10722834736998e-08
3193 2.1070389877309e-08
3194 2.10682049583966e-08
3195 2.10660253685546e-08
3196 2.10640482833924e-08
3197 2.10618154028452e-08
3198 2.10596624583559e-08
3199 2.10575397119328e-08
3200 2.1055436505435e-08
3201 2.10533546152192e-08
3202 2.10515107568199e-08
3203 2.10495159080892e-08
3204 2.10471284844971e-08
3205 2.10450998849865e-08
3206 2.10429629277087e-08
3207 2.10408312995014e-08
3208 2.10386552623731e-08
3209 2.10365591613026e-08
3210 2.10343351625397e-08
3211 2.10324433425058e-08
3212 2.1030217567386e-08
3213 2.10279988976936e-08
3214 2.10260555633113e-08
3215 2.10239399223155e-08
3216 2.10217550034031e-08
3217 2.10196535732621e-08
3218 2.1017729778805e-08
3219 2.10156017033114e-08
3220 2.10133190847728e-08
3221 2.10113331178263e-08
3222 2.10094484032197e-08
3223 2.10071711137516e-08
3224 2.10052242266556e-08
3225 2.10030393077432e-08
3226 2.10010444590125e-08
3227 2.0999003425004e-08
3228 2.09969250875019e-08
3229 2.0994756155801e-08
3230 2.099268847644e-08
3231 2.09908286308291e-08
3232 2.09886739099829e-08
3233 2.09867838663058e-08
3234 2.09846273691028e-08
3235 2.09825330443891e-08
3236 2.09806714224214e-08
3237 2.09785824267783e-08
3238 2.09764898784215e-08
3239 2.09745980583875e-08
3240 2.09725534716654e-08
3241 2.0970500003159e-08
3242 2.09683754803791e-08
3243 2.0966441027781e-08
3244 2.09644515081209e-08
3245 2.09624637648176e-08
3246 2.09604849032985e-08
3247 2.09585664379119e-08
3248 2.09565822473223e-08
3249 2.09545927276622e-08
3250 2.09527470929061e-08
3251 2.09503827619528e-08
3252 2.09486348268229e-08
3253 2.09468282719172e-08
3254 2.09447890142656e-08
3255 2.09427515329708e-08
3256 2.09407104989623e-08
3257 2.09388577587788e-08
3258 2.09369019898986e-08
3259 2.0935116751275e-08
3260 2.09329904521383e-08
3261 2.09309511944866e-08
3262 2.09291357577968e-08
3263 2.09271444617798e-08
3264 2.09251975746838e-08
3265 2.09231370007501e-08
3266 2.09212007717952e-08
3267 2.09192574374129e-08
3268 2.09172803522506e-08
3269 2.09153299124409e-08
3270 2.09135322393195e-08
3271 2.09116031157919e-08
3272 2.09097503756084e-08
3273 2.09077555268777e-08
3274 2.09056807420893e-08
3275 2.09039434651004e-08
3276 2.09019290764445e-08
3277 2.08999715312075e-08
3278 2.08980139859705e-08
3279 2.08961683512143e-08
3280 2.08942214641183e-08
3281 2.08924060274285e-08
3282 2.08904857856851e-08
3283 2.08885015950955e-08
3284 2.08865262862901e-08
3285 2.08845989391193e-08
3286 2.0882611195816e-08
3287 2.08808277335493e-08
3288 2.08790158495731e-08
3289 2.08770654097634e-08
3290 2.08750652319623e-08
3291 2.08733190731891e-08
3292 2.087137396245e-08
3293 2.08695407621917e-08
3294 2.0867579664241e-08
3295 2.08657162659165e-08
3296 2.0863874183874e-08
3297 2.08618935459981e-08
3298 2.08600567930262e-08
3299 2.08580495097976e-08
3300 2.08560191339302e-08
3301 2.08544399527e-08
3302 2.0852338522559e-08
3303 2.08503880827493e-08
3304 2.08485211317111e-08
3305 2.08465138484826e-08
3306 2.08446966354359e-08
3307 2.08428723169618e-08
3308 2.08412682667358e-08
3309 2.08393480249924e-08
3310 2.08372359367104e-08
3311 2.08354808961531e-08
3312 2.08337009866e-08
3313 2.08320489747393e-08
3314 2.08300772186476e-08
3315 2.08281658586884e-08
3316 2.08264250289858e-08
3317 2.08245687360886e-08
3318 2.08228492226681e-08
3319 2.08208046359459e-08
3320 2.08191011097369e-08
3321 2.0817054746658e-08
3322 2.08153760894447e-08
3323 2.08134576240582e-08
3324 2.08117238997829e-08
3325 2.08096473386377e-08
3326 2.08079438124287e-08
3327 2.08061123885273e-08
3328 2.08044088623183e-08
3329 2.08026236236947e-08
3330 2.08006323276777e-08
3331 2.07988435363404e-08
3332 2.07971844190524e-08
3333 2.07952624009522e-08
3334 2.07933723572751e-08
3335 2.07917771888333e-08
3336 2.07898533943762e-08
3337 2.07880415104e-08
3338 2.07862491663491e-08
3339 2.07844532695844e-08
3340 2.07826431619651e-08
3341 2.07806323260229e-08
3342 2.07788435346856e-08
3343 2.07771115867672e-08
3344 2.07753512171394e-08
3345 2.07734540680349e-08
3346 2.0771738107328e-08
3347 2.07698711562898e-08
3348 2.07680415087452e-08
3349 2.07663966023119e-08
3350 2.07647943284428e-08
3351 2.0762756847148e-08
3352 2.07610124647317e-08
3353 2.07592094625397e-08
3354 2.07574135657751e-08
3355 2.07555963527284e-08
3356 2.07537809160385e-08
3357 2.07518269235152e-08
3358 2.0750103857381e-08
3359 2.07482724334795e-08
3360 2.07467802937344e-08
3361 2.07446486655272e-08
3362 2.07429522447455e-08
3363 2.07413854980132e-08
3364 2.07396322338127e-08
3365 2.07379642347405e-08
3366 2.07358183956785e-08
3367 2.07340633551212e-08
3368 2.07322923273523e-08
3369 2.07306154464959e-08
3370 2.07290611342614e-08
3371 2.07270289820372e-08
3372 2.07253467721102e-08
3373 2.07236396931876e-08
3374 2.07219734704722e-08
3375 2.07202486279812e-08
3376 2.07184456257892e-08
3377 2.07166586108087e-08
3378 2.07149071229651e-08
3379 2.07130046447901e-08
3380 2.07113490802158e-08
3381 2.07098516114002e-08
3382 2.07080894654155e-08
3383 2.07060697476891e-08
3384 2.07042809563518e-08
3385 2.07025401266492e-08
3386 2.07009662744895e-08
3387 2.06992059048616e-08
3388 2.06974810623706e-08
3389 2.06957899706595e-08
3390 2.06939017033392e-08
3391 2.06920578449399e-08
3392 2.06904893218507e-08
3393 2.06887484921481e-08
3394 2.06869845698066e-08
3395 2.06851265005525e-08
3396 2.06833856708499e-08
3397 2.06818633330386e-08
3398 2.06799182222994e-08
3399 2.06782022615926e-08
3400 2.06766568311423e-08
3401 2.06748609343776e-08
3402 2.06731414209571e-08
3403 2.06714716455281e-08
3404 2.06697574611781e-08
3405 2.06681249892426e-08
3406 2.06664090285358e-08
3407 2.06647161604678e-08
3408 2.06631387555944e-08
3409 2.06613801623234e-08
3410 2.06596997287534e-08
3411 2.06583123940618e-08
3412 2.06564987337288e-08
3413 2.06544434888656e-08
3414 2.06528625312785e-08
3415 2.06511252542896e-08
3416 2.06493186993839e-08
3417 2.06479491282607e-08
3418 2.06463059981843e-08
3419 2.06443679928725e-08
3420 2.0643012632604e-08
3421 2.06413677261708e-08
3422 2.06396464363934e-08
3423 2.06377688272141e-08
3424 2.06362393839754e-08
3425 2.0634507436057e-08
3426 2.06325623253178e-08
3427 2.06310826200706e-08
3428 2.06293258031565e-08
3429 2.06275529990307e-08
3430 2.06260075685805e-08
3431 2.06242063427453e-08
3432 2.06225259091752e-08
3433 2.06208596864599e-08
3434 2.06194261664905e-08
3435 2.06174313177598e-08
3436 2.06158485838159e-08
3437 2.06142782843699e-08
3438 2.06126564705755e-08
3439 2.06106509637038e-08
3440 2.06091534948882e-08
3441 2.06074357578245e-08
3442 2.06057890750344e-08
3443 2.06039647565603e-08
3444 2.06023607063344e-08
3445 2.06009413972197e-08
3446 2.05991668167371e-08
3447 2.05975485556564e-08
3448 2.05957242371824e-08
3449 2.05941450559521e-08
3450 2.05924290952453e-08
3451 2.05908516903719e-08
3452 2.05890717808188e-08
3453 2.05874677305928e-08
3454 2.05857126900355e-08
3455 2.05844976619574e-08
3456 2.05825116950109e-08
3457 2.05809023157144e-08
3458 2.05791881313644e-08
3459 2.05775165795785e-08
3460 2.0575896542141e-08
3461 2.05743884151843e-08
3462 2.05726156110586e-08
3463 2.05710612988241e-08
3464 2.05692334276364e-08
3465 2.05677146425387e-08
3466 2.05660128926866e-08
3467 2.05644639095226e-08
3468 2.05627443961021e-08
3469 2.05612309400749e-08
3470 2.05595913627121e-08
3471 2.05578700729347e-08
3472 2.05562447064267e-08
3473 2.05545109821514e-08
3474 2.05530081842653e-08
3475 2.05512051820733e-08
3476 2.05498285055228e-08
3477 2.05479135928499e-08
3478 2.054609815616e-08
3479 2.05447427958916e-08
3480 2.05431724964455e-08
3481 2.05412629128432e-08
3482 2.05396606389741e-08
3483 2.05379695472629e-08
3484 2.05365537908619e-08
3485 2.05348662518645e-08
3486 2.05331236458051e-08
3487 2.05316332824168e-08
3488 2.0530016797693e-08
3489 2.05284553800311e-08
3490 2.05268442243778e-08
3491 2.0525293464857e-08
3492 2.05238546158171e-08
3493 2.0522074706264e-08
3494 2.05204937486769e-08
3495 2.05188737112394e-08
3496 2.05174597311952e-08
3497 2.05158556809693e-08
3498 2.05141503784034e-08
3499 2.05125196828249e-08
3500 2.05109547124493e-08
3501 2.05091392757595e-08
3502 2.05076720050101e-08
3503 2.05062384850407e-08
3504 2.05048173995692e-08
3505 2.050328262726e-08
3506 2.0501589759192e-08
3507 2.04999395236882e-08
3508 2.04982040230561e-08
3509 2.04964862859924e-08
3510 2.0495175334645e-08
3511 2.04933705560961e-08
3512 2.0491912167131e-08
3513 2.04902441680588e-08
3514 2.04886649868286e-08
3515 2.04871675180129e-08
3516 2.04853787266757e-08
3517 2.04839452067063e-08
3518 2.04822949712025e-08
3519 2.04809698090003e-08
3520 2.04792929281439e-08
3521 2.04779482260165e-08
3522 2.0476106143974e-08
3523 2.04744932119638e-08
3524 2.04730259412145e-08
3525 2.04715622231788e-08
3526 2.04696846139996e-08
3527 2.04684464932825e-08
3528 2.04667411907167e-08
3529 2.04650554280761e-08
3530 2.04636005918246e-08
3531 2.04618082477737e-08
3532 2.04603480824517e-08
3533 2.04588133101424e-08
3534 2.04572625506216e-08
3535 2.04556513949683e-08
3536 2.04540544501697e-08
3537 2.04523971092385e-08
3538 2.0451119908671e-08
3539 2.04495194111587e-08
3540 2.04479064791485e-08
3541 2.04462047292964e-08
3542 2.04446202189956e-08
3543 2.04432755168682e-08
3544 2.04417194282769e-08
3545 2.04399093206575e-08
3546 2.04385912638827e-08
3547 2.0436962344661e-08
3548 2.04355075084095e-08
3549 2.04338199694121e-08
3550 2.04322461172524e-08
3551 2.04306740414495e-08
3552 2.04289172245353e-08
3553 2.04275831805489e-08
3554 2.0425900970622e-08
3555 2.04243733037401e-08
3556 2.04228083333646e-08
3557 2.04215222510129e-08
3558 2.0419740565103e-08
3559 2.04184704699628e-08
3560 2.04169605666493e-08
3561 2.04152517113698e-08
3562 2.04136743064964e-08
3563 2.04123367097964e-08
3564 2.04107362122841e-08
3565 2.04090717659255e-08
3566 2.04075103482637e-08
3567 2.04059311670335e-08
3568 2.0404565148624e-08
3569 2.04028083317098e-08
3570 2.04014085625204e-08
3571 2.03998098413649e-08
3572 2.03982661872715e-08
3573 2.03969126033599e-08
3574 2.03954240163284e-08
3575 2.03937275955468e-08
3576 2.03923704589215e-08
3577 2.03904448881076e-08
3578 2.03892085437474e-08
3579 2.03878389726242e-08
3580 2.03862136061161e-08
3581 2.03848777857729e-08
3582 2.03832151157712e-08
3583 2.03815968546905e-08
3584 2.03803107723388e-08
3585 2.0378838172519e-08
3586 2.03771932660857e-08
3587 2.03755536887229e-08
3588 2.03739425330696e-08
3589 2.03723047320636e-08
3590 2.03711341129065e-08
3591 2.03697112510781e-08
3592 2.03677803511937e-08
3593 2.0366316633158e-08
3594 2.03650962760094e-08
3595 2.03637942064461e-08
3596 2.03620746930255e-08
3597 2.03607690707486e-08
3598 2.03590211356186e-08
3599 2.03573478074759e-08
3600 2.03558983002949e-08
3601 2.03544434640435e-08
3602 2.03530188258583e-08
3603 2.03513437213587e-08
3604 2.03500452045091e-08
3605 2.03484749050631e-08
3606 2.03471639537156e-08
3607 2.03456860248252e-08
3608 2.03439451951226e-08
3609 2.03426218092773e-08
3610 2.03410550625449e-08
3611 2.0339616213505e-08
3612 2.03381755881082e-08
3613 2.03365573270275e-08
3614 2.03350296601457e-08
3615 2.03336227855289e-08
3616 2.03322034764142e-08
3617 2.03304413304295e-08
3618 2.03291801170735e-08
3619 2.03276524501916e-08
3620 2.03263397224873e-08
3621 2.03245935637142e-08
3622 2.03231866890974e-08
3623 2.03216323768629e-08
3624 2.03202894510923e-08
3625 2.03184207236973e-08
3626 2.03172358936854e-08
3627 2.0315813031857e-08
3628 2.03142285215563e-08
3629 2.03126635511808e-08
3630 2.0311167858722e-08
3631 2.0309498083293e-08
3632 2.0308148052095e-08
3633 2.03065955162174e-08
3634 2.03052632485878e-08
3635 2.03034851153916e-08
3636 2.03020888989158e-08
3637 2.03007690657842e-08
3638 2.02992200826202e-08
3639 2.02978238661444e-08
3640 2.02963885698182e-08
3641 2.02950243277655e-08
3642 2.02935499515888e-08
3643 2.02918517544504e-08
3644 2.02904768542567e-08
3645 2.02890113598642e-08
3646 2.02876453414547e-08
3647 2.0286092805577e-08
3648 2.02845864549772e-08
3649 2.02832843854139e-08
3650 2.02818704053698e-08
3651 2.02801544446629e-08
3652 2.02787528991166e-08
3653 2.0277299839222e-08
3654 2.02755785494446e-08
3655 2.0274512735341e-08
3656 2.02729282250402e-08
3657 2.027138279459e-08
3658 2.02700860540972e-08
3659 2.02688283934549e-08
3660 2.02670857873954e-08
3661 2.0265652267426e-08
3662 2.02645402680446e-08
3663 2.02630996426478e-08
3664 2.02614796052103e-08
3665 2.02602201682112e-08
3666 2.0258617894342e-08
3667 2.02571044383149e-08
3668 2.02557437489759e-08
3669 2.02540775262605e-08
3670 2.0252944210597e-08
3671 2.02513223968026e-08
3672 2.02497503209997e-08
3673 2.02483914080176e-08
3674 2.02470893384543e-08
3675 2.02454621955894e-08
3676 2.02441441388146e-08
3677 2.0242591602937e-08
3678 2.02412895333737e-08
3679 2.02397831827739e-08
3680 2.02383283465224e-08
3681 2.02371754909336e-08
3682 2.02355359135709e-08
3683 2.023401890483e-08
3684 2.02328198639634e-08
3685 2.02312833152973e-08
3686 2.02297982809796e-08
3687 2.02284589079227e-08
3688 2.02268441995557e-08
3689 2.02256398296186e-08
3690 2.02239487379074e-08
3691 2.02225187706517e-08
3692 2.02209484712057e-08
3693 2.02196197562898e-08
3694 2.02181116293332e-08
3695 2.0216587515165e-08
3696 2.02152623529628e-08
3697 2.02138181748523e-08
3698 2.02123171533231e-08
3699 2.02108392244327e-08
3700 2.0209618867284e-08
3701 2.0208346995787e-08
3702 2.02065422172382e-08
3703 2.02052632403138e-08
3704 2.02037035990088e-08
3705 2.02025596252042e-08
3706 2.02008880734184e-08
3707 2.0199697914336e-08
3708 2.01981009695373e-08
3709 2.01967598201236e-08
3710 2.01954559742035e-08
3711 2.01938874511143e-08
3712 2.01924734710701e-08
3713 2.01910275166028e-08
3714 2.01893968210243e-08
3715 2.01881569239504e-08
3716 2.01866878768442e-08
3717 2.01853023185095e-08
3718 2.01838510349717e-08
3719 2.01825152146284e-08
3720 2.01812007105673e-08
3721 2.01798151522325e-08
3722 2.01781098496667e-08
3723 2.01769161378706e-08
3724 2.01753280748562e-08
3725 2.01738092897585e-08
3726 2.01725232074068e-08
3727 2.01710541603006e-08
3728 2.01697183399574e-08
3729 2.0168151593225e-08
3730 2.0166979197711e-08
3731 2.01655261378164e-08
3732 2.01641103814154e-08
3733 2.01626928486576e-08
3734 2.01610532712948e-08
3735 2.01597174509516e-08
3736 2.01582714964843e-08
3737 2.01568273183739e-08
3738 2.0155454194537e-08
3739 2.01540650834886e-08
3740 2.01526919596517e-08
3741 2.01512335706866e-08
3742 2.01498693286339e-08
3743 2.01484997575108e-08
3744 2.01471799243791e-08
3745 2.01457961424012e-08
3746 2.0144128143329e-08
3747 2.01429362078898e-08
3748 2.01413765665848e-08
3749 2.01401970656434e-08
3750 2.01388044018813e-08
3751 2.01373424602025e-08
3752 2.01356940010555e-08
3753 2.01345518036078e-08
3754 2.01332674976129e-08
3755 2.01317167380921e-08
3756 2.0130423550313e-08
3757 2.01288763435059e-08
3758 2.01277234879171e-08
3759 2.01259720000735e-08
3760 2.01247303266427e-08
3761 2.01233323338101e-08
3762 2.01219823026122e-08
3763 2.01206518113395e-08
3764 2.01192627002911e-08
3765 2.01177972058986e-08
3766 2.01164969126921e-08
3767 2.01149106260345e-08
3768 2.01137240196658e-08
3769 2.01122389853481e-08
3770 2.01107521746735e-08
3771 2.01095691210185e-08
3772 2.01082048789658e-08
3773 2.01069561001077e-08
3774 2.01055101456404e-08
3775 2.0103966491547e-08
3776 2.01024441537356e-08
3777 2.01011918221639e-08
3778 2.00997831711902e-08
3779 2.00983070186567e-08
3780 2.00970688979396e-08
3781 2.00955945217629e-08
3782 2.00942427142081e-08
3783 2.00929708427111e-08
3784 2.00915444281691e-08
3785 2.00902743330289e-08
3786 2.00888496948437e-08
3787 2.00873007116797e-08
3788 2.00861656196594e-08
3789 2.008473209969e-08
3790 2.0083371410351e-08
3791 2.00818988105311e-08
3792 2.00805381211921e-08
3793 2.00792253934878e-08
3794 2.00777350300996e-08
3795 2.00764986857394e-08
3796 2.00749141754386e-08
3797 2.00736067768048e-08
3798 2.00723473398057e-08
3799 2.00708498709901e-08
3800 2.00694660890122e-08
3801 2.00679579620555e-08
3802 2.00669507677276e-08
3803 2.00654284299162e-08
3804 2.00641245839961e-08
3805 2.00627816582255e-08
3806 2.00612912948372e-08
3807 2.00601490973895e-08
3808 2.00584704401763e-08
3809 2.00571985686793e-08
3810 2.00557881413488e-08
3811 2.00546388384737e-08
3812 2.00531076188781e-08
3813 2.0051848181879e-08
3814 2.00504750580421e-08
3815 2.00490433144296e-08
3816 2.00477963119283e-08
3817 2.00462526578349e-08
3818 2.00452152654407e-08
3819 2.00436236497126e-08
3820 2.00423517782156e-08
3821 2.00409076001051e-08
3822 2.00392999971655e-08
3823 2.00381133907968e-08
3824 2.00366692126863e-08
3825 2.00355501078775e-08
3826 2.0033951386722e-08
3827 2.00325214194663e-08
3828 2.00313472475955e-08
3829 2.00300771524553e-08
3830 2.00285956708512e-08
3831 2.00273202466406e-08
3832 2.00260501515004e-08
3833 2.00248333470654e-08
3834 2.00233216673951e-08
3835 2.00219822943382e-08
3836 2.00205665379372e-08
3837 2.00194243404894e-08
3838 2.00176941689278e-08
3839 2.00165128916296e-08
3840 2.00150243045982e-08
3841 2.00137293404623e-08
3842 2.00125889193714e-08
3843 2.00110612524895e-08
3844 2.00099083969008e-08
3845 2.00083327683842e-08
3846 2.00072385325711e-08
3847 2.00058121180291e-08
3848 2.00043217546408e-08
3849 2.00032506114667e-08
3850 2.0001676759307e-08
3851 2.00004848238677e-08
3852 1.99990353166868e-08
3853 1.99979002246664e-08
3854 1.99963317015772e-08
3855 1.99950900281465e-08
3856 1.99938270384337e-08
3857 1.99923384514022e-08
3858 1.99909866438475e-08
3859 1.99897698394125e-08
3860 1.99884304663556e-08
3861 1.99870680006597e-08
3862 1.99858103400175e-08
3863 1.99841938552936e-08
3864 1.99830800795553e-08
3865 1.99815097801093e-08
3866 1.99803480427363e-08
3867 1.99791418964423e-08
3868 1.99778007470286e-08
3869 1.99764080832665e-08
3870 1.99751877261178e-08
3871 1.99737222317253e-08
3872 1.99725054272903e-08
3873 1.99711074344577e-08
3874 1.9969821352106e-08
3875 1.99683274360041e-08
3876 1.99672065548384e-08
3877 1.99656717825292e-08
3878 1.99645189269404e-08
3879 1.99633820585632e-08
3880 1.99617993246193e-08
3881 1.99601934980365e-08
3882 1.99592502525547e-08
3883 1.9957964170203e-08
3884 1.99564578196032e-08
3885 1.99552587787366e-08
3886 1.99540242107332e-08
3887 1.99526191124733e-08
3888 1.9951416518893e-08
3889 1.99498604303017e-08
3890 1.99486809293603e-08
3891 1.99473113582371e-08
3892 1.99459613270392e-08
3893 1.99445064907877e-08
3894 1.99433340952737e-08
3895 1.99420444602083e-08
3896 1.99408471956986e-08
3897 1.99393497268829e-08
3898 1.99380707499586e-08
3899 1.99367988784616e-08
3900 1.99353760166332e-08
3901 1.99341094742067e-08
3902 1.99329370786927e-08
3903 1.99314751370139e-08
3904 1.99303809012008e-08
3905 1.992886389246e-08
3906 1.99275440593283e-08
3907 1.99264018618805e-08
3908 1.99251903865161e-08
3909 1.99236396269953e-08
3910 1.992259157646e-08
3911 1.992122022898e-08
3912 1.99199341466283e-08
3913 1.99183070037634e-08
3914 1.99170866466147e-08
3915 1.99160314906521e-08
3916 1.99146423796037e-08
3917 1.99130543165893e-08
3918 1.9912084425755e-08
3919 1.99106207077193e-08
3920 1.99094820629853e-08
3921 1.99079384088918e-08
3922 1.99069702944144e-08
3923 1.99054799310261e-08
3924 1.990428621923e-08
3925 1.99031884307033e-08
3926 1.99016589874645e-08
3927 1.99006215950703e-08
3928 1.98990370847696e-08
3929 1.98977794241273e-08
3930 1.98965874886881e-08
3931 1.98952818664111e-08
3932 1.98941680906728e-08
3933 1.98927008199234e-08
3934 1.98911163096227e-08
3935 1.98900362846643e-08
3936 1.98884997359983e-08
3937 1.98875245160934e-08
3938 1.98860714561988e-08
3939 1.988495235139e-08
3940 1.98835277132048e-08
3941 1.9882309132413e-08
3942 1.9880760149249e-08
3943 1.98796197281581e-08
3944 1.98785077287766e-08
3945 1.98772553972049e-08
3946 1.98756318070537e-08
3947 1.98744007917639e-08
3948 1.98730720768481e-08
3949 1.9871833956131e-08
3950 1.98704253051574e-08
3951 1.98694145581157e-08
3952 1.98681533447598e-08
3953 1.98668157480597e-08
3954 1.98654728222891e-08
3955 1.98642666759952e-08
3956 1.98631209258338e-08
3957 1.98616429969434e-08
3958 1.98603569145916e-08
3959 1.98591525446545e-08
3960 1.9857786526245e-08
3961 1.98564187314787e-08
3962 1.98552072561142e-08
3963 1.98538412377047e-08
3964 1.98526688421907e-08
3965 1.9851437826901e-08
3966 1.98499598980106e-08
3967 1.98488407932018e-08
3968 1.98473610879546e-08
3969 1.98462899447804e-08
3970 1.98447782651101e-08
3971 1.98436413967329e-08
3972 1.98420710972869e-08
3973 1.98408187657151e-08
3974 1.98392680061943e-08
3975 1.98382714700074e-08
3976 1.9837081310925e-08
3977 1.98355269986905e-08
3978 1.98343919066701e-08
3979 1.98332035239446e-08
3980 1.98320346811443e-08
3981 1.98307112952989e-08
3982 1.98296454811953e-08
3983 1.98281693286617e-08
3984 1.98269560769404e-08
3985 1.98255456496099e-08
3986 1.98243981230917e-08
3987 1.98232648074281e-08
3988 1.98220035940722e-08
3989 1.98204723744766e-08
3990 1.98195237999244e-08
3991 1.98179499477646e-08
3992 1.98168326193127e-08
3993 1.98155873931682e-08
3994 1.98140881479958e-08
3995 1.98127665385073e-08
3996 1.98115301941471e-08
3997 1.98101623993807e-08
3998 1.98087288794113e-08
3999 1.98077945157138e-08
4000 1.98066505419092e-08
4001 1.98052418909356e-08
4002 1.98041369969815e-08
4003 1.98030409848116e-08
4004 1.980161989934e-08
4005 1.9800204142939e-08
4006 1.97989802330767e-08
4007 1.97979694860351e-08
4008 1.97966123494098e-08
4009 1.97955394298788e-08
4010 1.97939513668643e-08
4011 1.97927203515746e-08
4012 1.9791466243646e-08
4013 1.97902991772025e-08
4014 1.97889491460046e-08
4015 1.9787760763279e-08
4016 1.97866398821134e-08
4017 1.97852827454881e-08
4018 1.97840375193437e-08
4019 1.97827034753573e-08
4020 1.97815541724822e-08
4021 1.97806073742868e-08
4022 1.97789642442103e-08
4023 1.97778522448289e-08
4024 1.97767064946675e-08
4025 1.97753760033947e-08
4026 1.97741467644619e-08
4027 1.97729477235953e-08
4028 1.97716847338825e-08
4029 1.97704004278876e-08
4030 1.97690894765401e-08
4031 1.9767874448462e-08
4032 1.9766822845213e-08
4033 1.97656326861306e-08
4034 1.9764224035157e-08
4035 1.9763161773767e-08
4036 1.97619822728257e-08
4037 1.97607299412539e-08
4038 1.97594260953338e-08
4039 1.97580778404927e-08
4040 1.97565324100424e-08
4041 1.97555696246354e-08
4042 1.97540650503925e-08
4043 1.97527363354766e-08
4044 1.97516065725267e-08
4045 1.97503240428887e-08
4046 1.97490734876737e-08
4047 1.9747876223164e-08
4048 1.97466611950858e-08
4049 1.97452720840374e-08
4050 1.97443430494104e-08
4051 1.97428988713e-08
4052 1.97416110125914e-08
4053 1.97405221058489e-08
4054 1.97390637168837e-08
4055 1.97379410593612e-08
4056 1.97366993859305e-08
4057 1.97355056741344e-08
4058 1.97341041285881e-08
4059 1.97331111451149e-08
4060 1.97317291394938e-08
4061 1.9730665101747e-08
4062 1.97292422399187e-08
4063 1.97281568858898e-08
4064 1.97270235702263e-08
4065 1.97255456413359e-08
4066 1.97242577826273e-08
4067 1.97232346010878e-08
4068 1.97219041098151e-08
4069 1.97208187557862e-08
4070 1.97194474083062e-08
4071 1.97183318562111e-08
4072 1.97170777482825e-08
4073 1.97157916659307e-08
4074 1.97148182223827e-08
4075 1.97133616097744e-08
4076 1.97123810607991e-08
4077 1.97111535982231e-08
4078 1.97098035670251e-08
4079 1.97087821618425e-08
4080 1.97072900220974e-08
4081 1.97061496010065e-08
4082 1.97049967454177e-08
4083 1.97036698068587e-08
4084 1.97025880055435e-08
4085 1.97014848879462e-08
4086 1.9700060249761e-08
4087 1.96988452216829e-08
4088 1.9697472097846e-08
4089 1.96963299003983e-08
4090 1.96952001374484e-08
4091 1.96938572116778e-08
4092 1.96925569184714e-08
4093 1.96915941330644e-08
4094 1.96905460825292e-08
4095 1.96891285497713e-08
4096 1.968794904883e-08
4097 1.96866860591172e-08
4098 1.96856237977272e-08
4099 1.96842400157493e-08
4100 1.96830622911648e-08
4101 1.96819254227876e-08
4102 1.96807281582778e-08
4103 1.96796108298258e-08
4104 1.96782803385531e-08
4105 1.9677408147345e-08
4106 1.96760225890102e-08
4107 1.96746263725345e-08
4108 1.96737435231853e-08
4109 1.96724503354062e-08
4110 1.96712370836849e-08
4111 1.96700629118141e-08
4112 1.96689544651463e-08
4113 1.96675742358821e-08
4114 1.96663663132313e-08
4115 1.96649043715524e-08
4116 1.96640321803443e-08
4117 1.96625808968065e-08
4118 1.96614209357904e-08
4119 1.96600034030325e-08
4120 1.9659058381194e-08
4121 1.96578486821863e-08
4122 1.96567579990869e-08
4123 1.96555394182951e-08
4124 1.9654379457279e-08
4125 1.96528429086129e-08
4126 1.96518303852145e-08
4127 1.96506260152773e-08
4128 1.96495140158959e-08
4129 1.9648174642839e-08
4130 1.96471283686606e-08
4131 1.96457552448237e-08
4132 1.96445864020234e-08
4133 1.96433926902273e-08
4134 1.96422647036343e-08
4135 1.96407992092418e-08
4136 1.96399021490379e-08
4137 1.96387386353081e-08
4138 1.96374490002427e-08
4139 1.9636228643094e-08
4140 1.96351024328578e-08
4141 1.96339460245554e-08
4142 1.96327505364025e-08
4143 1.96313472144993e-08
4144 1.96303435728851e-08
4145 1.96293310494866e-08
4146 1.96280112163549e-08
4147 1.9626915204185e-08
4148 1.96254195117262e-08
4149 1.96244354100372e-08
4150 1.96233820304315e-08
4151 1.96221776604943e-08
4152 1.96210390157603e-08
4153 1.96200158342208e-08
4154 1.96185485634714e-08
4155 1.96175662381393e-08
4156 1.96162588395055e-08
4157 1.96150296005726e-08
4158 1.96138536523449e-08
4159 1.96125267137859e-08
4160 1.96114484651844e-08
4161 1.96105514049805e-08
4162 1.96093186133339e-08
4163 1.96081479941768e-08
4164 1.96067837521241e-08
4165 1.96057481360867e-08
4166 1.96043981048888e-08
4167 1.96033038690757e-08
4168 1.9601984035944e-08
4169 1.96007601260817e-08
4170 1.9599518452651e-08
4171 1.95983425044233e-08
4172 1.95972447158965e-08
4173 1.95961362692287e-08
4174 1.9594661893052e-08
4175 1.95935836444505e-08
4176 1.95926528334667e-08
4177 1.9591277933273e-08
4178 1.95900096144896e-08
4179 1.95886311615823e-08
4180 1.95878850917097e-08
4181 1.95866753927021e-08
4182 1.95852667417284e-08
4183 1.95841636241312e-08
4184 1.95829805704761e-08
4185 1.95819520598661e-08
4186 1.95808400604847e-08
4187 1.95797404956011e-08
4188 1.95783709244779e-08
4189 1.95771949762502e-08
4190 1.95759799481721e-08
4191 1.95745890607668e-08
4192 1.95736404862146e-08
4193 1.95724769724848e-08
4194 1.95713028006139e-08
4195 1.95700469163285e-08
4196 1.9569045051071e-08
4197 1.95677234415825e-08
4198 1.95667144708978e-08
4199 1.95654035195503e-08
4200 1.95642400058205e-08
4201 1.95630622812359e-08
4202 1.9561921860145e-08
4203 1.95606464359344e-08
4204 1.95594189733583e-08
4205 1.95584579643082e-08
4206 1.95574134664867e-08
4207 1.95559870519446e-08
4208 1.9554743602157e-08
4209 1.95537790403932e-08
4210 1.95526350665887e-08
4211 1.95514395784357e-08
4212 1.95502991573449e-08
4213 1.9549080576553e-08
4214 1.95477678488487e-08
4215 1.95465048591359e-08
4216 1.95453253581945e-08
4217 1.95443679018581e-08
4218 1.9542930829175e-08
4219 1.95417904080841e-08
4220 1.95406286707112e-08
4221 1.95394633806245e-08
4222 1.95385236878565e-08
4223 1.95371328004512e-08
4224 1.95361753441148e-08
4225 1.953516637343e-08
4226 1.95339584507792e-08
4227 1.9532688355639e-08
4228 1.9531515960125e-08
4229 1.95303737626773e-08
4230 1.95292759741506e-08
4231 1.95283380577393e-08
4232 1.95267482183681e-08
4233 1.95256770751939e-08
4234 1.95247018552891e-08
4235 1.95232789934607e-08
4236 1.95221101506604e-08
4237 1.95211047326893e-08
4238 1.95197760177734e-08
4239 1.95184917117786e-08
4240 1.95177776163291e-08
4241 1.95165110739026e-08
4242 1.95151912407709e-08
4243 1.95141076630989e-08
4244 1.95132479063886e-08
4245 1.9511706028652e-08
4246 1.95107396905314e-08
4247 1.95095726240879e-08
4248 1.95083451615119e-08
4249 1.95072438202715e-08
4250 1.95060376739775e-08
4251 1.95048919238161e-08
4252 1.95036733430243e-08
4253 1.95023215354695e-08
4254 1.95014155934814e-08
4255 1.95004421499334e-08
4256 1.94991116586607e-08
4257 1.94982199275273e-08
4258 1.94969942413081e-08
4259 1.94958911237109e-08
4260 1.94946743192759e-08
4261 1.94935481090397e-08
4262 1.94925355856412e-08
4263 1.94915337203838e-08
4264 1.94900540151366e-08
4265 1.9489064584377e-08
4266 1.94878637671536e-08
4267 1.94867748604111e-08
4268 1.94856291102496e-08
4269 1.94845402035071e-08
4270 1.94832452393712e-08
4271 1.9482184754338e-08
4272 1.9481046109604e-08
4273 1.9480072666056e-08
4274 1.94788238871979e-08
4275 1.9477573331983e-08
4276 1.94764187000374e-08
4277 1.94751592630382e-08
4278 1.94743279280374e-08
4279 1.94729032898522e-08
4280 1.94720932711334e-08
4281 1.94706633038777e-08
4282 1.94697395983212e-08
4283 1.94682634457877e-08
4284 1.94673184239491e-08
4285 1.94662170827087e-08
4286 1.94651335050366e-08
4287 1.94638118955481e-08
4288 1.94625879856858e-08
4289 1.94617832960375e-08
4290 1.94604705683332e-08
4291 1.94594189650843e-08
4292 1.94581364354462e-08
4293 1.94571061484794e-08
4294 1.94557792099204e-08
4295 1.94546281306884e-08
4296 1.94536529107836e-08
4297 1.94526439400988e-08
4298 1.94516012186341e-08
4299 1.94502476347225e-08
4300 1.94491409644115e-08
4301 1.944809646659e-08
4302 1.94470164416316e-08
4303 1.94458049662671e-08
4304 1.94446627688194e-08
4305 1.94435667566495e-08
4306 1.94426732491593e-08
4307 1.94413800613802e-08
4308 1.94401561515178e-08
4309 1.94394296215705e-08
4310 1.94381488682893e-08
4311 1.94370883832562e-08
4312 1.94356832849962e-08
4313 1.94345126658391e-08
4314 1.94336138292783e-08
4315 1.94323135360719e-08
4316 1.94314395685069e-08
4317 1.94301623679394e-08
4318 1.94289846433549e-08
4319 1.94277536280651e-08
4320 1.94267659736624e-08
4321 1.94254177188213e-08
4322 1.94244123008502e-08
4323 1.94232807615435e-08
4324 1.94224138994059e-08
4325 1.94209590631544e-08
4326 1.94200033831748e-08
4327 1.94186444701927e-08
4328 1.94177065537815e-08
4329 1.94166513978189e-08
4330 1.94157188104782e-08
4331 1.94144522680517e-08
4332 1.94133971120891e-08
4333 1.94121572150152e-08
4334 1.94110700846295e-08
4335 1.94100984174383e-08
4336 1.94089047056423e-08
4337 1.94075653325854e-08
4338 1.94064710967723e-08
4339 1.9405412388096e-08
4340 1.94044140755523e-08
4341 1.94032381273246e-08
4342 1.94019591504002e-08
4343 1.9400921758006e-08
4344 1.93997031772142e-08
4345 1.93984295293603e-08
4346 1.93974667439534e-08
4347 1.93965341566127e-08
4348 1.93953280103187e-08
4349 1.93942018000826e-08
4350 1.93931963821115e-08
4351 1.93920364210953e-08
4352 1.93907254697478e-08
4353 1.93895122180265e-08
4354 1.93886062760384e-08
4355 1.93874072351718e-08
4356 1.93866149800215e-08
4357 1.93853768593044e-08
4358 1.93842932816324e-08
4359 1.93833233907981e-08
4360 1.93819396088202e-08
4361 1.9380923532708e-08
4362 1.9379873705816e-08
4363 1.93786551250241e-08
4364 1.93776372725551e-08
4365 1.9376511062319e-08
4366 1.93753812993691e-08
4367 1.9374310156195e-08
4368 1.937309335176e-08
4369 1.93720079977311e-08
4370 1.93708213913624e-08
4371 1.93697040629104e-08
4372 1.93687252902919e-08
4373 1.93675564474916e-08
4374 1.93664586589648e-08
4375 1.93654372537821e-08
4376 1.93642346602019e-08
4377 1.93630906863973e-08
4378 1.93619289490243e-08
4379 1.93609182019827e-08
4380 1.93598186370991e-08
4381 1.93585094621085e-08
4382 1.93575822038383e-08
4383 1.93564844153116e-08
4384 1.93552676108766e-08
4385 1.93542621929055e-08
4386 1.93530027559063e-08
4387 1.93521625391213e-08
4388 1.93510949486608e-08
4389 1.93498124190228e-08
4390 1.9348814106479e-08
4391 1.93478211230058e-08
4392 1.93466043185708e-08
4393 1.93455527153219e-08
4394 1.93445934826286e-08
4395 1.93431546335887e-08
4396 1.93421545446881e-08
4397 1.93412770244095e-08
4398 1.93401916703806e-08
4399 1.9338981971373e-08
4400 1.93380351731776e-08
4401 1.93367650780374e-08
4402 1.93357418964979e-08
4403 1.93344984467103e-08
4404 1.93334894760255e-08
4405 1.93325035979797e-08
4406 1.93314662055855e-08
4407 1.93304448004028e-08
4408 1.93293985262244e-08
4409 1.93283486993323e-08
4410 1.93271958437435e-08
4411 1.93260554226526e-08
4412 1.93251192825983e-08
4413 1.93237852386119e-08
4414 1.93227087663672e-08
4415 1.93217939425949e-08
4416 1.932085247347e-08
4417 1.93194917841311e-08
4418 1.93184490626663e-08
4419 1.93172269291608e-08
4420 1.93163813833053e-08
4421 1.93153049110606e-08
4422 1.93143883109315e-08
4423 1.93131235448618e-08
4424 1.93120524016877e-08
4425 1.93109137569536e-08
4426 1.93100220258202e-08
4427 1.9308675547336e-08
4428 1.93076203913733e-08
4429 1.93065634590539e-08
4430 1.9305433696104e-08
4431 1.9304648546381e-08
4432 1.93032807516147e-08
4433 1.93023801386971e-08
4434 1.93011473470506e-08
4435 1.93001081782995e-08
4436 1.92988256486615e-08
4437 1.92977953616946e-08
4438 1.92965767809028e-08
4439 1.92956903788399e-08
4440 1.92944611399071e-08
4441 1.92933455878119e-08
4442 1.92923312880566e-08
4443 1.92914004770728e-08
4444 1.92904110463132e-08
4445 1.92889135774976e-08
4446 1.92881817184798e-08
4447 1.92871976167908e-08
4448 1.92861602243966e-08
4449 1.92851832281349e-08
4450 1.92841262958154e-08
4451 1.92830782452802e-08
4452 1.92819680222556e-08
4453 1.92809732624255e-08
4454 1.92798133014094e-08
4455 1.92788380815045e-08
4456 1.92778291108198e-08
4457 1.92765892137459e-08
4458 1.9275470108937e-08
4459 1.92743581095556e-08
4460 1.92733438098003e-08
4461 1.92726261616372e-08
4462 1.92711215873942e-08
4463 1.92699740608759e-08
4464 1.92689988409711e-08
4465 1.92679205923696e-08
4466 1.92668050402744e-08
4467 1.92656308684036e-08
4468 1.92646840702082e-08
4469 1.92637639173654e-08
4470 1.9262563100142e-08
4471 1.92615434713161e-08
4472 1.92603373250222e-08
4473 1.92594225012499e-08
4474 1.92584028724241e-08
4475 1.92573157420384e-08
4476 1.92564471035439e-08
4477 1.92553013533825e-08
4478 1.92542515264904e-08
4479 1.9252961891425e-08
4480 1.9251755745131e-08
4481 1.92509030938481e-08
4482 1.9249776883612e-08
4483 1.92487004113673e-08
4484 1.92477926930223e-08
4485 1.92465758885874e-08
4486 1.92454141512144e-08
4487 1.92444584712348e-08
4488 1.92431564016715e-08
4489 1.92423375011685e-08
4490 1.92412414889986e-08
4491 1.92401365950445e-08
4492 1.92391969022765e-08
4493 1.92379587815594e-08
4494 1.92370848139944e-08
4495 1.92358644568458e-08
4496 1.92349904892808e-08
4497 1.92338323046215e-08
4498 1.92329299153471e-08
4499 1.9231825021393e-08
4500 1.92309741464669e-08
4501 1.92297111567541e-08
4502 1.92288833744669e-08
4503 1.92277145316666e-08
4504 1.92266806919861e-08
4505 1.92255189546131e-08
4506 1.92247586738858e-08
4507 1.92238154284041e-08
4508 1.92227762596531e-08
4509 1.92214741900898e-08
4510 1.92206002225248e-08
4511 1.92192608494679e-08
4512 1.92182714187084e-08
4513 1.92173299495835e-08
4514 1.92162428191978e-08
4515 1.92150384492606e-08
4516 1.92140596766421e-08
4517 1.92131448528698e-08
4518 1.92118907449412e-08
4519 1.92109741448121e-08
4520 1.92098301710075e-08
4521 1.92087270534103e-08
4522 1.92075546578963e-08
4523 1.92066469395513e-08
4524 1.92056521797213e-08
4525 1.92045117586304e-08
4526 1.92036218038538e-08
4527 1.92023943412778e-08
4528 1.92014635302939e-08
4529 1.92004812049618e-08
4530 1.91993567710824e-08
4531 1.91980991104401e-08
4532 1.91969267149261e-08
4533 1.91961468942736e-08
4534 1.91951947670077e-08
4535 1.919405256956e-08
4536 1.91927984616314e-08
4537 1.9192006206481e-08
4538 1.91909457214479e-08
4539 1.91900468848871e-08
4540 1.91887785661038e-08
4541 1.91878495314768e-08
4542 1.91869702348413e-08
4543 1.91855011877351e-08
4544 1.91845135333324e-08
4545 1.91835578533528e-08
4546 1.91824547357555e-08
4547 1.91814262251455e-08
4548 1.91804296889586e-08
4549 1.91793976256349e-08
4550 1.91784419456553e-08
4551 1.91773246172033e-08
4552 1.91762570267429e-08
4553 1.9175171672714e-08
4554 1.91740774369009e-08
4555 1.91731288623487e-08
4556 1.91721500897302e-08
4557 1.91710363139919e-08
4558 1.9170105503008e-08
4559 1.91690041617676e-08
4560 1.91677660410505e-08
4561 1.91669489169044e-08
4562 1.91658795500871e-08
4563 1.91647551162077e-08
4564 1.91637390400956e-08
4565 1.91629094814516e-08
4566 1.91618489964185e-08
4567 1.91609572652851e-08
4568 1.91597546717048e-08
4569 1.9158646225037e-08
4570 1.91575804109334e-08
4571 1.9156733088721e-08
4572 1.9155715236252e-08
4573 1.91543634286973e-08
4574 1.91535853844016e-08
4575 1.91527789183965e-08
4576 1.91515923120278e-08
4577 1.91504749835758e-08
4578 1.91495281853804e-08
4579 1.91483415790117e-08
4580 1.91474338606668e-08
4581 1.91463769283473e-08
4582 1.91454443410066e-08
4583 1.9144234641999e-08
4584 1.914325054031e-08
4585 1.91421634099243e-08
4586 1.91412077299447e-08
4587 1.9140385276728e-08
4588 1.9139267948276e-08
4589 1.91383140446533e-08
4590 1.91371878344171e-08
4591 1.91361113621724e-08
4592 1.91351947620433e-08
4593 1.91340969735165e-08
4594 1.91332283350221e-08
4595 1.91320506104375e-08
4596 1.91312334862914e-08
4597 1.91302085283951e-08
4598 1.91289899476033e-08
4599 1.91279614369932e-08
4600 1.9126924044599e-08
4601 1.91258440196407e-08
4602 1.91249309722252e-08
4603 1.91239433178225e-08
4604 1.91226696699687e-08
4605 1.91219022838141e-08
4606 1.91205717925413e-08
4607 1.91197155885448e-08
4608 1.9118717276001e-08
4609 1.91175537622712e-08
4610 1.91165501206569e-08
4611 1.91153848305703e-08
4612 1.91143900707402e-08
4613 1.91134148508354e-08
4614 1.91123117332381e-08
4615 1.91113578296154e-08
4616 1.91104003732789e-08
4617 1.91093469936732e-08
4618 1.91079809752637e-08
4619 1.91071638511175e-08
4620 1.91062152765653e-08
4621 1.91050091302714e-08
4622 1.91039255525993e-08
4623 1.91032771823529e-08
4624 1.91018862949477e-08
4625 1.91009661421049e-08
4626 1.91001543470293e-08
4627 1.90990707693572e-08
4628 1.90979392300505e-08
4629 1.90968183488849e-08
4630 1.90957454293539e-08
4631 1.90946476408271e-08
4632 1.90937328170548e-08
4633 1.90928162169257e-08
4634 1.90916935594032e-08
4635 1.90906135344449e-08
4636 1.90895583784823e-08
4637 1.90882936124126e-08
4638 1.90875795169632e-08
4639 1.9086408897806e-08
4640 1.90854585468969e-08
4641 1.90845650394067e-08
4642 1.90836697555596e-08
4643 1.90824671619794e-08
4644 1.90814191114441e-08
4645 1.90806144217959e-08
4646 1.90794615662071e-08
4647 1.90785378606506e-08
4648 1.90774560593354e-08
4649 1.90766478169735e-08
4650 1.90753866036175e-08
4651 1.90744664507747e-08
4652 1.90735427452182e-08
4653 1.90722708737212e-08
4654 1.90713489445216e-08
4655 1.90702440505675e-08
4656 1.90692031054596e-08
4657 1.90679667610993e-08
4658 1.90674445121886e-08
4659 1.90662525767493e-08
4660 1.90651086029447e-08
4661 1.90641564756788e-08
4662 1.90630569107952e-08
4663 1.90617885920119e-08
4664 1.90609483752269e-08
4665 1.90601188165829e-08
4666 1.90590974114002e-08
4667 1.9057905475961e-08
4668 1.90571327607358e-08
4669 1.9055903521803e-08
4670 1.90550331069517e-08
4671 1.90539140021428e-08
4672 1.90528215426866e-08
4673 1.90519813259016e-08
4674 1.90508107067444e-08
4675 1.90498070651302e-08
4676 1.90489757301293e-08
4677 1.90480076156518e-08
4678 1.90470039740376e-08
4679 1.90458582238762e-08
4680 1.90447995151999e-08
4681 1.90439273239917e-08
4682 1.90427353885525e-08
4683 1.90418241174939e-08
4684 1.9040871990228e-08
4685 1.9039909204821e-08
4686 1.90388469434311e-08
4687 1.90375502029383e-08
4688 1.90366868935143e-08
4689 1.90357010154685e-08
4690 1.903472224285e-08
4691 1.90339175532017e-08
4692 1.90325444293649e-08
4693 1.90316669090862e-08
4694 1.90304216829418e-08
4695 1.90296844948534e-08
4696 1.90288744761347e-08
4697 1.90277464895416e-08
4698 1.9026725084359e-08
4699 1.90256486121143e-08
4700 1.90248297116113e-08
4701 1.90238225172834e-08
4702 1.90226927543335e-08
4703 1.90219697770999e-08
4704 1.90209270556352e-08
4705 1.9019955388444e-08
4706 1.90188842452699e-08
4707 1.90178717218714e-08
4708 1.90168503166888e-08
4709 1.90158395696471e-08
4710 1.90150064582895e-08
4711 1.90139921585342e-08
4712 1.90126723254025e-08
4713 1.90117663834144e-08
4714 1.90106739239582e-08
4715 1.90095974517135e-08
4716 1.900877322214e-08
4717 1.90076914208248e-08
4718 1.90067375172021e-08
4719 1.90056059778954e-08
4720 1.90047817483219e-08
4721 1.90035489566753e-08
4722 1.9002634132903e-08
4723 1.90015825296541e-08
4724 1.9000809814429e-08
4725 1.89997493293959e-08
4726 1.89987563459226e-08
4727 1.89979108000671e-08
4728 1.89965234653755e-08
4729 1.89959834528963e-08
4730 1.89948607953738e-08
4731 1.89938376138343e-08
4732 1.89927700233739e-08
4733 1.8991828554249e-08
4734 1.89908035963526e-08
4735 1.8989970484995e-08
4736 1.89888460511156e-08
4737 1.89878370804308e-08
4738 1.89869666655795e-08
4739 1.89859843402473e-08
4740 1.89847071396798e-08
4741 1.89838846864632e-08
4742 1.89828082142185e-08
4743 1.89819147067283e-08
4744 1.89808577744088e-08
4745 1.89799589378481e-08
4746 1.89790672067147e-08
4747 1.89778930348439e-08
4748 1.89769053804412e-08
4749 1.89758964097564e-08
4750 1.89751290236018e-08
4751 1.89739512990172e-08
4752 1.8972659887595e-08
4753 1.89721216514727e-08
4754 1.89709314923903e-08
4755 1.89701871988746e-08
4756 1.89690929630615e-08
4757 1.89680271489578e-08
4758 1.89670341654846e-08
4759 1.89661211180692e-08
4760 1.89651672144464e-08
4761 1.89640569914218e-08
4762 1.89629663083224e-08
4763 1.89621047752553e-08
4764 1.89610851464295e-08
4765 1.89600921629562e-08
4766 1.89590760868441e-08
4767 1.89582092247065e-08
4768 1.89569995256988e-08
4769 1.8956201941478e-08
4770 1.89551041529512e-08
4771 1.89542532780251e-08
4772 1.8953203451133e-08
4773 1.89522957327881e-08
4774 1.89515283466335e-08
4775 1.8950450098032e-08
4776 1.89493238877958e-08
4777 1.89483735368867e-08
4778 1.89473343681357e-08
4779 1.89463875699403e-08
4780 1.89451654364348e-08
4781 1.89441617948205e-08
4782 1.89432913799692e-08
4783 1.89421118790278e-08
4784 1.89414830487067e-08
4785 1.89405113815155e-08
4786 1.89394739891213e-08
4787 1.89382554083295e-08
4788 1.8937475587677e-08
4789 1.89364932623448e-08
4790 1.89355411350789e-08
4791 1.89343936085606e-08
4792 1.89335054301409e-08
4793 1.89327273858453e-08
4794 1.89317042043058e-08
4795 1.89308853038028e-08
4796 1.8929611655949e-08
4797 1.89288211771554e-08
4798 1.89278992479558e-08
4799 1.89269595551878e-08
4800 1.89259914407103e-08
4801 1.89249131921088e-08
4802 1.89240960679626e-08
4803 1.89229041325234e-08
4804 1.89220479285268e-08
4805 1.89211135648293e-08
4806 1.8920044198012e-08
4807 1.89191329269534e-08
4808 1.8918019151215e-08
4809 1.89171700526458e-08
4810 1.89161859509568e-08
4811 1.89149922391607e-08
4812 1.89142248530061e-08
4813 1.89131377226204e-08
4814 1.89121145410809e-08
4815 1.89113258386442e-08
4816 1.89102262737606e-08
4817 1.89091586833001e-08
4818 1.89081443835448e-08
4819 1.89074533807343e-08
4820 1.89062188127309e-08
4821 1.89053999122279e-08
4822 1.89044158105389e-08
4823 1.89032238750997e-08
4824 1.8902424514522e-08
4825 1.89015665341685e-08
4826 1.89004580875007e-08
4827 1.88994402350318e-08
4828 1.88985112004048e-08
4829 1.88977651305322e-08
4830 1.88967810288432e-08
4831 1.88959319302739e-08
4832 1.88946138734991e-08
4833 1.88938109602077e-08
4834 1.88927735678135e-08
4835 1.88918338750454e-08
4836 1.88910842524592e-08
4837 1.8889991793003e-08
4838 1.88889011099036e-08
4839 1.8887890362862e-08
4840 1.8886717967348e-08
4841 1.88860020955417e-08
4842 1.88851441151883e-08
4843 1.88840605375162e-08
4844 1.88831847935944e-08
4845 1.88819111457406e-08
4846 1.88808240153548e-08
4847 1.88801596578969e-08
4848 1.88789996968808e-08
4849 1.88782145471578e-08
4850 1.88770314935027e-08
4851 1.88763333852648e-08
4852 1.88749424978596e-08
4853 1.88742248496965e-08
4854 1.88730879813193e-08
4855 1.88722424354637e-08
4856 1.88713631388282e-08
4857 1.8870370155355e-08
4858 1.88695246094994e-08
4859 1.88686311020092e-08
4860 1.88677606871579e-08
4861 1.88666966494111e-08
4862 1.88656876787263e-08
4863 1.88644584397935e-08
4864 1.88635187470254e-08
4865 1.88628455077833e-08
4866 1.88614279750254e-08
4867 1.88608080264885e-08
4868 1.88596889216797e-08
4869 1.88587687688369e-08
4870 1.88578521687077e-08
4871 1.8856654904198e-08
4872 1.88559514668896e-08
4873 1.88549478252753e-08
4874 1.88540099088641e-08
4875 1.88529423184036e-08
4876 1.88520257182745e-08
4877 1.8851135763498e-08
4878 1.88500433040417e-08
4879 1.88494038155795e-08
4880 1.88482793817002e-08
4881 1.88472544238039e-08
4882 1.88463467054589e-08
4883 1.88454087890477e-08
4884 1.88442150772516e-08
4885 1.88432416337037e-08
4886 1.88425062219721e-08
4887 1.88415771873451e-08
4888 1.88404190026858e-08
4889 1.88393034505907e-08
4890 1.88386088950665e-08
4891 1.88374116305567e-08
4892 1.88367366149578e-08
4893 1.88357223152025e-08
4894 1.88348181495712e-08
4895 1.88339548401473e-08
4896 1.88327504702102e-08
4897 1.8831878279002e-08
4898 1.88310291804328e-08
4899 1.88298194814251e-08
4900 1.88291409131125e-08
4901 1.88281656932077e-08
4902 1.88271229717429e-08
4903 1.88263058475968e-08
4904 1.88252169408543e-08
4905 1.88242257337379e-08
4906 1.8823383740596e-08
4907 1.88223161501355e-08
4908 1.8821426195359e-08
4909 1.88203923556785e-08
4910 1.88196498385196e-08
4911 1.88185858007728e-08
4912 1.88175075521713e-08
4913 1.88168005621492e-08
4914 1.88158324476717e-08
4915 1.88147932789207e-08
4916 1.88136688450413e-08
4917 1.88126598743565e-08
4918 1.88117930122189e-08
4919 1.88106454857007e-08
4920 1.88100273135205e-08
4921 1.88091977548765e-08
4922 1.88080395702173e-08
4923 1.88071638262954e-08
4924 1.88061850536769e-08
4925 1.88050517380134e-08
4926 1.8804314549925e-08
4927 1.88036946013881e-08
4928 1.88026092473592e-08
4929 1.88016375801681e-08
4930 1.88007067691842e-08
4931 1.87996622713626e-08
4932 1.87986888278147e-08
4933 1.87980848664893e-08
4934 1.87967845732828e-08
4935 1.87959212638589e-08
4936 1.87948199226184e-08
4937 1.87941129325964e-08
4938 1.87932602813135e-08
4939 1.87919564353933e-08
4940 1.87911251003925e-08
4941 1.87902635673254e-08
4942 1.87893043346321e-08
4943 1.87882367441716e-08
4944 1.87872313262005e-08
4945 1.87864870326848e-08
4946 1.87853697042328e-08
4947 1.87844300114648e-08
4948 1.87836448617418e-08
4949 1.87826199038454e-08
4950 1.87816837637911e-08
4951 1.8780578869837e-08
4952 1.87796107553595e-08
4953 1.87788025129976e-08
4954 1.87778450566611e-08
4955 1.87767792425575e-08
4956 1.87759248149177e-08
4957 1.87747506430469e-08
4958 1.87741484580783e-08
4959 1.87729050082908e-08
4960 1.87723951938779e-08
4961 1.87711375332356e-08
4962 1.87702866583095e-08
4963 1.87691213682228e-08
4964 1.87683664165661e-08
4965 1.87673130369603e-08
4966 1.87665527562331e-08
4967 1.87655331274073e-08
4968 1.87646485017012e-08
4969 1.876371058529e-08
4970 1.8762859710364e-08
4971 1.876168731485e-08
4972 1.87608080182144e-08
4973 1.87599162870811e-08
4974 1.87588877764711e-08
4975 1.87577917643011e-08
4976 1.87569462184456e-08
4977 1.87558626407736e-08
4978 1.8755116570901e-08
4979 1.87539406226733e-08
4980 1.87533597539868e-08
4981 1.87522104511118e-08
4982 1.87511837168586e-08
4983 1.87502458004474e-08
4984 1.87494286763012e-08
4985 1.87484960889606e-08
4986 1.87474373802843e-08
4987 1.87467961154653e-08
4988 1.87456716815859e-08
4989 1.87446147492665e-08
4990 1.87439415100243e-08
4991 1.87429041176301e-08
4992 1.8741888041518e-08
4993 1.8741168616998e-08
4994 1.87401543172427e-08
4995 1.87392181771884e-08
4996 1.87382944716319e-08
4997 1.87374560312037e-08
4998 1.87364168624526e-08
4999 1.87357791503473e-08
};
\addlegendentry{Test}

\end{groupplot}

\end{tikzpicture}
	\end{figure}
\end{center}
\begin{figure}[H]
	% This file was created by tikzplotlib v0.9.6.
\begin{tikzpicture}

\begin{groupplot}[group style={group size=1 by 2},
legend cell align={left},
legend style={fill opacity=1, draw opacity=1, text opacity=1, draw=white},
log basis y={10},
tick align=outside,
tick pos=left,
title style={at={(0.3,0.85)},anchor=north},
x grid style={white!69.0196078431373!black},
xlabel={Epoch},
x label style={yshift=13pt},
xmin=-99.95, xmax=5098.95,
xtick style={color=black},
xtick = {0,1000,4000,5000},
y grid style={white!69.0196078431373!black},
ylabel={MSE Loss},
ymode=log,
ytick style={color=black},
width=.45\textwidth,
height=.25\textwidth
]

\nextgroupplot[
title={Leaky/Tanh $\hy$},
ymin=4.23041685977616e-09, ymax=1e-05,
]
\addplot [semithick, black, dashed]
table {%
0 0.00794568543741479
1 0.00102331811844488
2 0.000231695338865393
3 0.000205259474405466
4 0.000198213847354054
5 0.000184335938559343
6 0.000158458173230429
7 0.000114654033269289
8 5.73862060190322e-05
9 2.79062188526495e-05
10 1.86789795541245e-05
11 1.65733219109541e-05
12 1.59125527101622e-05
13 1.54527024417206e-05
14 1.49612416946994e-05
15 1.44022566942326e-05
16 1.37646732448147e-05
17 1.30402747662615e-05
18 1.22128500799903e-05
19 1.13149585366017e-05
20 1.03752550253802e-05
21 9.40122402498389e-06
22 8.4121798936394e-06
23 7.42973440005556e-06
24 6.47345199269722e-06
25 5.57745241931684e-06
26 4.749642853632e-06
27 4.03722304712062e-06
28 3.28957287841547e-06
29 2.48021327116277e-06
30 1.91230281020083e-06
31 1.56647904006846e-06
32 1.35910705624198e-06
33 1.22576810794506e-06
34 1.13419015631777e-06
35 1.05617603212593e-06
36 9.89481183532703e-07
37 9.32924012854386e-07
38 8.79053224370097e-07
39 8.3312635247168e-07
40 7.92099386185186e-07
41 7.59245137619047e-07
42 7.30102056662574e-07
43 7.13138552475456e-07
44 6.99631840276993e-07
45 6.86388795125836e-07
46 6.78258965457701e-07
47 6.73138193102218e-07
48 6.68665048165096e-07
49 6.6468104489914e-07
50 6.60666006263355e-07
51 6.5702475482432e-07
52 6.53054142969012e-07
53 6.48885168150315e-07
54 6.4416423826863e-07
55 6.39948975155846e-07
56 6.35384212225532e-07
57 6.30731239201765e-07
58 6.2611564798587e-07
59 6.21080330013513e-07
60 6.15937625411789e-07
61 6.10699282834304e-07
62 6.05542475028642e-07
63 6.00304609022828e-07
64 5.95061167800281e-07
65 5.89520126467846e-07
66 5.83972025797408e-07
67 5.78355382451434e-07
68 5.72110915772583e-07
69 5.65697318398506e-07
70 5.5815196507325e-07
71 5.50607108975143e-07
72 5.42231041972485e-07
73 5.32862973017245e-07
74 5.19747605595811e-07
75 5.07553654095716e-07
76 4.95860261413128e-07
77 4.83928439086512e-07
78 4.73421843297572e-07
79 4.61828229255445e-07
80 4.51616399583799e-07
81 4.41696106635803e-07
82 4.3239411398055e-07
83 4.2444700762978e-07
84 4.1720406791157e-07
85 4.11043423776292e-07
86 4.03316087359329e-07
87 3.96306216206455e-07
88 3.8930630104872e-07
89 3.82755214729968e-07
90 3.76589114944537e-07
91 3.70817245098465e-07
92 3.64933429635173e-07
93 3.58974201049023e-07
94 3.53197651705273e-07
95 3.47528136998676e-07
96 3.41983764345244e-07
97 3.36495823011873e-07
98 3.31690676816265e-07
99 3.26891462847811e-07
100 3.22268556123717e-07
101 3.18459978798202e-07
102 3.14507870109892e-07
103 3.10581897792872e-07
104 3.06597594903479e-07
105 3.03078295595505e-07
106 2.98512048535038e-07
107 2.94755100181732e-07
108 2.90629996152703e-07
109 2.86909880781394e-07
110 2.83067085862676e-07
111 2.79760282497321e-07
112 2.76244385262636e-07
113 2.72879685237371e-07
114 2.69447326574301e-07
115 2.65836300024525e-07
116 2.62738450925504e-07
117 2.59538459495445e-07
118 2.56513780960255e-07
119 2.53483186070547e-07
120 2.50896118551225e-07
121 2.47866121655171e-07
122 2.45793329182398e-07
123 2.42697760764798e-07
124 2.40340267551709e-07
125 2.3782838816544e-07
126 2.34948382021649e-07
127 2.32441133060313e-07
128 2.2981944173317e-07
129 2.2776015364645e-07
130 2.25448820319052e-07
131 2.23387603118574e-07
132 2.21390370486851e-07
133 2.19304481103855e-07
134 2.17318953100509e-07
135 2.15335876403877e-07
136 2.13437060942567e-07
137 2.11633837122527e-07
138 2.10125350789792e-07
139 2.08310994661787e-07
140 2.0667978015565e-07
141 2.0512745442991e-07
142 2.03361514028888e-07
143 2.02098720256672e-07
144 2.00558651278548e-07
145 1.99074377505326e-07
146 1.97697967919908e-07
147 1.96408409788518e-07
148 1.95134635089911e-07
149 1.93861458606115e-07
150 1.9271517006203e-07
151 1.91588951123478e-07
152 1.90496774294946e-07
153 1.89530322090903e-07
154 1.88512227004445e-07
155 1.87808134447298e-07
156 1.86294901865836e-07
157 1.8480061755044e-07
158 1.8394615336037e-07
159 1.82632639075031e-07
160 1.81538218770427e-07
161 1.80771613327302e-07
162 1.80308249003502e-07
163 1.78528085275964e-07
164 1.77086367994406e-07
165 1.76280544737395e-07
166 1.75060842734887e-07
167 1.74141588151322e-07
168 1.73004102527408e-07
169 1.72374263804009e-07
170 1.70666949444787e-07
171 1.69471571480173e-07
172 1.68660518092167e-07
173 1.67634022155916e-07
174 1.66829628075682e-07
175 1.66213814071448e-07
176 1.6548161361829e-07
177 1.64968255075948e-07
178 1.64227632525105e-07
179 1.63670962634477e-07
180 1.63105067970903e-07
181 1.62690604531335e-07
182 1.6214624147004e-07
183 1.61638914007334e-07
184 1.61177961224723e-07
185 1.60711160710925e-07
186 1.6026016026327e-07
187 1.59854491594835e-07
188 1.59376817404855e-07
189 1.58976064857264e-07
190 1.58538541019659e-07
191 1.57995181124804e-07
192 1.57775386407266e-07
193 1.57524764477834e-07
194 1.56784295755052e-07
195 1.56665711503301e-07
196 1.56257626449552e-07
197 1.55871569520905e-07
198 1.55443225181351e-07
199 1.55056444858026e-07
200 1.54669409052843e-07
201 1.54465440444618e-07
202 1.54011377289542e-07
203 1.53511707838838e-07
204 1.53388913460617e-07
205 1.52903520233227e-07
206 1.52697517470823e-07
207 1.52365967927892e-07
208 1.51847101808933e-07
209 1.51714985085594e-07
210 1.51207038144197e-07
211 1.51162994936449e-07
212 1.50808201039965e-07
213 1.50310371982432e-07
214 1.50045077718097e-07
215 1.49704269630124e-07
216 1.49614008678967e-07
217 1.48749607533638e-07
218 1.48624030259725e-07
219 1.48510414353265e-07
220 1.47928337462577e-07
221 1.47601757363613e-07
222 1.47316085735927e-07
223 1.47225212897784e-07
224 1.46676952939462e-07
225 1.46452222482996e-07
226 1.46230315831719e-07
227 1.45946883012282e-07
228 1.45616477571053e-07
229 1.45339503910602e-07
230 1.44990892140928e-07
231 1.44716128456679e-07
232 1.44431743191564e-07
233 1.44154291835896e-07
234 1.43904947873086e-07
235 1.43603759144906e-07
236 1.43297091307115e-07
237 1.42845979264727e-07
238 1.4259458930832e-07
239 1.42260254408111e-07
240 1.42153359246056e-07
241 1.41841441373192e-07
242 1.41578123958208e-07
243 1.41282237509976e-07
244 1.4103928634901e-07
245 1.40765397589337e-07
246 1.40101741454846e-07
247 1.39922459297281e-07
248 1.38854026847213e-07
249 1.37727860150871e-07
250 1.37158081194411e-07
251 1.36559418883575e-07
252 1.35914771680845e-07
253 1.35714135540255e-07
254 1.35239195287795e-07
255 1.34861111250828e-07
256 1.34343245598689e-07
257 1.34143835933465e-07
258 1.34082028610649e-07
259 1.3366321294761e-07
260 1.33411873361133e-07
261 1.33102934797957e-07
262 1.32881773758253e-07
263 1.32532037399713e-07
264 1.32347123007026e-07
265 1.31984005211638e-07
266 1.31825840690158e-07
267 1.3149712662397e-07
268 1.31449851564991e-07
269 1.30694631254702e-07
270 1.30307811495811e-07
271 1.30148085485615e-07
272 1.29811973918592e-07
273 1.29302506774565e-07
274 1.29299348937462e-07
275 1.28729770509484e-07
276 1.28691763030542e-07
277 1.28390351274987e-07
278 1.27908679069755e-07
279 1.27665645340169e-07
280 1.27575992678874e-07
281 1.27336271870782e-07
282 1.2710176189934e-07
283 1.26849985262467e-07
284 1.26579940796123e-07
285 1.26056278573206e-07
286 1.25947969084272e-07
287 1.25673626441003e-07
288 1.25458584328442e-07
289 1.2516743278379e-07
290 1.24974937561362e-07
291 1.2469764292522e-07
292 1.24445488683289e-07
293 1.24216261415899e-07
294 1.23977299957279e-07
295 1.23726842932381e-07
296 1.23529334447259e-07
297 1.23323205063031e-07
298 1.22990055271721e-07
299 1.22744904169192e-07
300 1.22568982837468e-07
301 1.22338203397909e-07
302 1.22048109410855e-07
303 1.21795057999474e-07
304 1.21516071463379e-07
305 1.21299346888204e-07
306 1.21039762889907e-07
307 1.2082518497758e-07
308 1.20675637379275e-07
309 1.20415436026811e-07
310 1.20170617561755e-07
311 1.19975395610705e-07
312 1.19733801687616e-07
313 1.19517681161696e-07
314 1.1929238186914e-07
315 1.19118462066137e-07
316 1.18879637125868e-07
317 1.18688216524276e-07
318 1.18309020268548e-07
319 1.18194290767093e-07
320 1.17977779705125e-07
321 1.17890672531651e-07
322 1.17519434591262e-07
323 1.17314172173977e-07
324 1.17277717745168e-07
325 1.16989938850676e-07
326 1.16842909902548e-07
327 1.16534898951581e-07
328 1.16372703114376e-07
329 1.15959833028256e-07
330 1.15991872969623e-07
331 1.15696127217735e-07
332 1.15510579129285e-07
333 1.15309208962966e-07
334 1.15155130634648e-07
335 1.14917496250921e-07
336 1.14782299907468e-07
337 1.14531395762896e-07
338 1.14363170404808e-07
339 1.14164258930405e-07
340 1.13844137917596e-07
341 1.13264884664455e-07
342 1.13061592880159e-07
343 1.12897521395539e-07
344 1.12679851878683e-07
345 1.12524158725869e-07
346 1.12255588956556e-07
347 1.12160895423763e-07
348 1.11859808155534e-07
349 1.11733374887013e-07
350 1.1155962729692e-07
351 1.11332787246621e-07
352 1.11200354580232e-07
353 1.10835296596168e-07
354 1.10712161368376e-07
355 1.10484817950063e-07
356 1.10416741956776e-07
357 1.10198207398993e-07
358 1.09938049783764e-07
359 1.09721718787981e-07
360 1.09669218930186e-07
361 1.09510970333737e-07
362 1.09233400945374e-07
363 1.09070747711382e-07
364 1.08936695108142e-07
365 1.08831907112084e-07
366 1.08617934035049e-07
367 1.08491789718723e-07
368 1.08352589228478e-07
369 1.08184045610749e-07
370 1.08070605122856e-07
371 1.07784159417035e-07
372 1.07708497876668e-07
373 1.07581510496058e-07
374 1.07366691445288e-07
375 1.0702675326324e-07
376 1.07076396042238e-07
377 1.06727988268052e-07
378 1.06694036032362e-07
379 1.06524324367641e-07
380 1.0630654584487e-07
381 1.06263600313738e-07
382 1.06092802780289e-07
383 1.05974796882347e-07
384 1.05811536677658e-07
385 1.05824837111523e-07
386 1.05559920295706e-07
387 1.05546971831849e-07
388 1.05260859960099e-07
389 1.05236281001009e-07
390 1.04957210878442e-07
391 1.04916747779127e-07
392 1.04603103868506e-07
393 1.0428606344659e-07
394 1.04176516535048e-07
395 1.03936824072726e-07
396 1.03848748933544e-07
397 1.03579781016716e-07
398 1.03526600411463e-07
399 1.03370667528413e-07
400 1.03237542176782e-07
401 1.03078502166287e-07
402 1.0294821336565e-07
403 1.02801807323694e-07
404 1.02649420896217e-07
405 1.02498145305052e-07
406 1.02366849870084e-07
407 1.02219407055681e-07
408 1.02210421573101e-07
409 1.0202031108042e-07
410 1.01868891773105e-07
411 1.01703112118656e-07
412 1.01569944822799e-07
413 1.01460772663042e-07
414 1.01302248802337e-07
415 1.01146613320946e-07
416 1.01012251769728e-07
417 1.00855655419174e-07
418 1.00713402167774e-07
419 1.00584572194418e-07
420 1.0045392963276e-07
421 1.00256124791365e-07
422 1.00186672598923e-07
423 1.00008953692665e-07
424 9.98695246194892e-08
425 9.96865433404714e-08
426 9.96088602769341e-08
427 9.94391052970833e-08
428 9.93310319943319e-08
429 9.91857727656864e-08
430 9.90627221106877e-08
431 9.88924347726794e-08
432 9.875219808686e-08
433 9.86131867133366e-08
434 9.85010500791361e-08
435 9.83414929951465e-08
436 9.8184784080857e-08
437 9.80886653958457e-08
438 9.79324098624446e-08
439 9.77546733027523e-08
440 9.76791751869932e-08
441 9.75220120067455e-08
442 9.73826180943327e-08
443 9.72257439442181e-08
444 9.71487197647392e-08
445 9.69906163712508e-08
446 9.68413031987581e-08
447 9.67297738774242e-08
448 9.65729842636875e-08
449 9.64094186781672e-08
450 9.63227392780652e-08
451 9.61686416891538e-08
452 9.601742798937e-08
453 9.59137380540653e-08
454 9.57427591758986e-08
455 9.56391392903377e-08
456 9.54915149349311e-08
457 9.53319750380821e-08
458 9.52488296372955e-08
459 9.50658255312042e-08
460 9.49733298836186e-08
461 9.48150910979884e-08
462 9.46901375526998e-08
463 9.45499600280009e-08
464 9.43980081924423e-08
465 9.43037853877726e-08
466 9.41570784114276e-08
467 9.40007041418944e-08
468 9.38453483576573e-08
469 9.37677034871243e-08
470 9.36781115945529e-08
471 9.34864044892514e-08
472 9.33762755530587e-08
473 9.32032739004995e-08
474 9.31042485015077e-08
475 9.28996597382437e-08
476 9.28339232539877e-08
477 9.26355300192583e-08
478 9.25505377167291e-08
479 9.23486980402011e-08
480 9.22584395608439e-08
481 9.20756416675772e-08
482 9.19618572861047e-08
483 9.17619974081241e-08
484 9.17242062312695e-08
485 9.14829561975416e-08
486 9.1430295010575e-08
487 9.1224090732922e-08
488 9.11049100742467e-08
489 9.09463287634971e-08
490 9.08229395606419e-08
491 9.06129471571759e-08
492 9.05584134165416e-08
493 9.02925261865306e-08
494 9.02189054490243e-08
495 8.99691032576477e-08
496 8.99347350240554e-08
497 8.97259736833966e-08
498 8.95803507070525e-08
499 8.94547874916718e-08
500 8.92544241275495e-08
501 8.91423505904321e-08
502 8.89970741195789e-08
503 8.88670178924578e-08
504 8.86892602274258e-08
505 8.85600926650021e-08
506 8.84193749706164e-08
507 8.82899315186592e-08
508 8.81499345060099e-08
509 8.80352698184872e-08
510 8.78750940238593e-08
511 8.77766673375025e-08
512 8.76112112648819e-08
513 8.75034119529605e-08
514 8.73116611450619e-08
515 8.7235505273453e-08
516 8.70563547070713e-08
517 8.69286506484102e-08
518 8.67839788893931e-08
519 8.66525991314315e-08
520 8.65234168008211e-08
521 8.63588868369458e-08
522 8.62986985665781e-08
523 8.60697670619359e-08
524 8.5793084712904e-08
525 8.5456364492531e-08
526 8.52212053230605e-08
527 8.50948178927613e-08
528 8.4905073421071e-08
529 8.4785044029756e-08
530 8.46806824217339e-08
531 8.4497116975335e-08
532 8.43964217713822e-08
533 8.42109975485172e-08
534 8.41761933374485e-08
535 8.39115933046752e-08
536 8.38247673489167e-08
537 8.36846800884583e-08
538 8.35699720653782e-08
539 8.34596785501684e-08
540 8.33360167751263e-08
541 8.31336754600898e-08
542 8.31199375384006e-08
543 8.29036556877494e-08
544 8.28108102561842e-08
545 8.27439334993851e-08
546 8.25452877544208e-08
547 8.24568275219484e-08
548 8.23535928577979e-08
549 8.22498794981197e-08
550 8.20709292861377e-08
551 8.1963077787961e-08
552 8.18345720705604e-08
553 8.17132968180534e-08
554 8.15780909111385e-08
555 8.15054891156741e-08
556 8.13435118796058e-08
557 8.1271173041797e-08
558 8.11571885117601e-08
559 8.09759006348276e-08
560 8.09043822824762e-08
561 8.07560934428242e-08
562 8.06210357504611e-08
563 8.04820442394938e-08
564 8.03952613117609e-08
565 8.02291230548313e-08
566 8.01306569526705e-08
567 7.99114395637623e-08
568 7.98915102513398e-08
569 7.97174052733496e-08
570 7.95916435878397e-08
571 7.94398996109003e-08
572 7.93513262671297e-08
573 7.92309614472408e-08
574 7.90078572592279e-08
575 7.89942876728844e-08
576 7.87827023218668e-08
577 7.86949877400467e-08
578 7.86798491159146e-08
579 7.84538429821069e-08
580 7.82942934902664e-08
581 7.82388830300462e-08
582 7.82155599217127e-08
583 7.79506165184962e-08
584 7.78019187119838e-08
585 7.77087196257575e-08
586 7.75878199932301e-08
587 7.74998141475081e-08
588 7.72741267627275e-08
589 7.72778909641225e-08
590 7.7143164418203e-08
591 7.7158170795677e-08
592 7.68037878264849e-08
593 7.68501598598093e-08
594 7.6726880133382e-08
595 7.65300032878891e-08
596 7.6563249785444e-08
597 7.6418150637636e-08
598 7.62791394723905e-08
599 7.62251578869666e-08
600 7.5987531277466e-08
601 7.59991381826808e-08
602 7.58444131316871e-08
603 7.5768074699667e-08
604 7.56016185361474e-08
605 7.55222666795063e-08
606 7.54596483187875e-08
607 7.53319190729407e-08
608 7.5133784768866e-08
609 7.5098841868293e-08
610 7.49683751473107e-08
611 7.48975617579362e-08
612 7.47147350179667e-08
613 7.45629098402034e-08
614 7.44122327174246e-08
615 7.4319666660827e-08
616 7.4136438385608e-08
617 7.41657642842064e-08
618 7.39727487828468e-08
619 7.39740888175966e-08
620 7.39247474030869e-08
621 7.36282212105976e-08
622 7.3569567278664e-08
623 7.35029796934406e-08
624 7.33613072725348e-08
625 7.32610352640606e-08
626 7.31104972500063e-08
627 7.31040226140678e-08
628 7.2916862269512e-08
629 7.30557970292089e-08
630 7.29317733814128e-08
631 7.2890056644237e-08
632 7.26679404587927e-08
633 7.26951633911455e-08
634 7.24581144764258e-08
635 7.24297837413346e-08
636 7.22633186382993e-08
637 7.2198914352839e-08
638 7.21184844834077e-08
639 7.20044945814458e-08
640 7.19441101795226e-08
641 7.1838107322808e-08
642 7.14238396253286e-08
643 7.13422878466652e-08
644 7.12704949159537e-08
645 7.12387912473744e-08
646 7.09728828751643e-08
647 7.09286167106526e-08
648 7.08350788150014e-08
649 7.07908577708416e-08
650 7.06524646028051e-08
651 7.0599759010026e-08
652 7.04199701910824e-08
653 7.02408808392896e-08
654 7.03526347765049e-08
655 7.02308281623765e-08
656 7.00794933368165e-08
657 6.9983330345913e-08
658 6.98604446900397e-08
659 6.98658522169104e-08
660 6.96511450777315e-08
661 6.95996926407538e-08
662 6.95064553672964e-08
663 6.9433598990809e-08
664 6.92832644357821e-08
665 6.92644342463744e-08
666 6.91688321361461e-08
667 6.91297086805598e-08
668 6.8929668368245e-08
669 6.87781577215496e-08
670 6.86969533991721e-08
671 6.8633791127759e-08
672 6.84565769892842e-08
673 6.85380985907535e-08
674 6.84432057433959e-08
675 6.81999745539841e-08
676 6.80647554411884e-08
677 6.79797185183517e-08
678 6.80334999527688e-08
679 6.77525623511421e-08
680 6.7717123290123e-08
681 6.76298922521745e-08
682 6.75620716292791e-08
683 6.7553782787666e-08
684 6.72643572396936e-08
685 6.72949708921955e-08
686 6.71492590336165e-08
687 6.72614362313695e-08
688 6.69533604225059e-08
689 6.70805835025767e-08
690 6.69277748026609e-08
691 6.65966861004286e-08
692 6.67193747547756e-08
693 6.6482039768001e-08
694 6.65329843845264e-08
695 6.64114940729377e-08
696 6.63580887763082e-08
697 6.63527502244854e-08
698 6.60004379788859e-08
699 6.60402638863466e-08
700 6.60129189662406e-08
701 6.58061428895351e-08
702 6.59658102826199e-08
703 6.55877179762676e-08
704 6.5686802895204e-08
705 6.54901881040892e-08
706 6.53923646116716e-08
707 6.53386587812221e-08
708 6.51765887047517e-08
709 6.52695351588406e-08
710 6.52628477957329e-08
711 6.49512369630401e-08
712 6.50557932799778e-08
713 6.48373762568433e-08
714 6.48339608324555e-08
715 6.49046798257835e-08
716 6.4629667716698e-08
717 6.47618304685515e-08
718 6.43923338969898e-08
719 6.44073707882775e-08
720 6.42931987466788e-08
721 6.42379481399225e-08
722 6.41274940864633e-08
723 6.41176599511795e-08
724 6.39493898866306e-08
725 6.39796404455772e-08
726 6.40033480201474e-08
727 6.37313759916935e-08
728 6.36448601647466e-08
729 6.36585264697231e-08
730 6.35304713094698e-08
731 6.35300619813428e-08
732 6.35604960890213e-08
733 6.34791228093512e-08
734 6.32866855072578e-08
735 6.33345643601402e-08
736 6.31458388022921e-08
737 6.31520294813726e-08
738 6.30077312941957e-08
739 6.29781523868722e-08
740 6.28330880925176e-08
741 6.29647436372061e-08
742 6.27796720631402e-08
743 6.27698866733084e-08
744 6.26317218528349e-08
745 6.2381085854657e-08
746 6.23371542545215e-08
747 6.23082316271173e-08
748 6.21841150221236e-08
749 6.21632486654811e-08
750 6.20409027707325e-08
751 6.20666944248782e-08
752 6.18827911154085e-08
753 6.19236120518174e-08
754 6.1739520604398e-08
755 6.17655011052598e-08
756 6.1659822541138e-08
757 6.16530421444494e-08
758 6.15669346517578e-08
759 6.14949810371357e-08
760 6.13523400980753e-08
761 6.13645284661679e-08
762 6.1245416989042e-08
763 6.12181622057229e-08
764 6.10567475445301e-08
765 6.10965967839938e-08
766 6.09497801731251e-08
767 6.09287398969371e-08
768 6.07912189782844e-08
769 6.07894696407207e-08
770 6.06270246068519e-08
771 6.06647108560843e-08
772 6.05039257361994e-08
773 6.05095258499055e-08
774 6.03543323265399e-08
775 6.03809608028527e-08
776 6.02259376778491e-08
777 6.02322697829116e-08
778 6.00170760409746e-08
779 6.0083001498068e-08
780 5.99593888788164e-08
781 5.99178219604646e-08
782 5.97929021104449e-08
783 5.98891610712471e-08
784 5.97962502890681e-08
785 5.95742906965846e-08
786 5.95660477511473e-08
787 5.9517973416412e-08
788 5.94160985394332e-08
789 5.92766604263772e-08
790 5.92630662481497e-08
791 5.91719253328904e-08
792 5.90818852743702e-08
793 5.90219840219675e-08
794 5.90385470586874e-08
795 5.89544604783221e-08
796 5.88285899576491e-08
797 5.88017516118811e-08
798 5.8715584306146e-08
799 5.87058844860877e-08
800 5.84910789958037e-08
801 5.85939112740519e-08
802 5.8350257245543e-08
803 5.84012359978736e-08
804 5.81844480804783e-08
805 5.82448984460804e-08
806 5.82240810005352e-08
807 5.80150396207735e-08
808 5.80121570701309e-08
809 5.78166312554806e-08
810 5.78317478039381e-08
811 5.77172785902746e-08
812 5.76870404631613e-08
813 5.76398069482842e-08
814 5.76035658945706e-08
815 5.74311670260563e-08
816 5.73927107345718e-08
817 5.73324522212815e-08
818 5.72714455222556e-08
819 5.7182059992833e-08
820 5.71623152174627e-08
821 5.70656274030412e-08
822 5.69464087880611e-08
823 5.69055118684325e-08
824 5.68733918591136e-08
825 5.68452006333153e-08
826 5.67426192699116e-08
827 5.66590530777233e-08
828 5.66330419111694e-08
829 5.66095116747434e-08
830 5.64895180019676e-08
831 5.64069942661227e-08
832 5.64104309948021e-08
833 5.62924435651979e-08
834 5.62528343639457e-08
835 5.61568358579123e-08
836 5.61676866097649e-08
837 5.60138821068534e-08
838 5.5985644932921e-08
839 5.59197230249886e-08
840 5.5872112282529e-08
841 5.57960523233092e-08
842 5.57435685788477e-08
843 5.56558899216242e-08
844 5.5623745354616e-08
845 5.55037517551149e-08
846 5.55019634318477e-08
847 5.54369799492527e-08
848 5.53882356633117e-08
849 5.53363827848763e-08
850 5.53374030265452e-08
851 5.52176614645461e-08
852 5.53646256777895e-08
853 5.50983966545182e-08
854 5.50457994554776e-08
855 5.4965207738622e-08
856 5.49263892941632e-08
857 5.48655416152677e-08
858 5.48074885342587e-08
859 5.47671677568751e-08
860 5.46897103563815e-08
861 5.46291053935555e-08
862 5.46609677272159e-08
863 5.44925757997028e-08
864 5.44686026167085e-08
865 5.44053410536094e-08
866 5.43743589074275e-08
867 5.43244479862537e-08
868 5.42442873974203e-08
869 5.41942714966659e-08
870 5.41718288409321e-08
871 5.40494901875022e-08
872 5.40537592201851e-08
873 5.39907330834311e-08
874 5.38934417915193e-08
875 5.37978640182679e-08
876 5.38003297267942e-08
877 5.37382146070797e-08
878 5.37432678187066e-08
879 5.36181763193255e-08
880 5.37634169317514e-08
881 5.35841438669138e-08
882 5.35899381315375e-08
883 5.34512068486315e-08
884 5.34215597944865e-08
885 5.32719450219155e-08
886 5.3342942837542e-08
887 5.32581287009393e-08
888 5.31758494948242e-08
889 5.31686451838986e-08
890 5.30824056750312e-08
891 5.30078076934615e-08
892 5.29345417512594e-08
893 5.294000527023e-08
894 5.27808063801594e-08
895 5.27100915443945e-08
896 5.27206079885545e-08
897 5.25826992752165e-08
898 5.25297585203255e-08
899 5.2454098940391e-08
900 5.24698406003665e-08
901 5.23051918315254e-08
902 5.22859798501862e-08
903 5.22296319815929e-08
904 5.22219366545507e-08
905 5.21500489634175e-08
906 5.20853291363643e-08
907 5.20233889067523e-08
908 5.19489347965418e-08
909 5.18734846746138e-08
910 5.17975898592926e-08
911 5.17308307765862e-08
912 5.16982749825257e-08
913 5.16018047913569e-08
914 5.15841654316063e-08
915 5.14917832665773e-08
916 5.14540376732597e-08
917 5.13712947145528e-08
918 5.13701121989119e-08
919 5.12454181524902e-08
920 5.12462436950134e-08
921 5.11366703446203e-08
922 5.1112236977513e-08
923 5.10232585624326e-08
924 5.10065382228753e-08
925 5.08750842915795e-08
926 5.08852614513611e-08
927 5.07490582801751e-08
928 5.06928063821466e-08
929 5.06550340959588e-08
930 5.05776660080226e-08
931 5.0540473354177e-08
932 5.04841183612825e-08
933 5.03610421080936e-08
934 5.02896475067516e-08
935 5.03446667046159e-08
936 5.019682149765e-08
937 5.0136225609565e-08
938 5.01627822697692e-08
939 5.00517750958718e-08
940 5.00258755413707e-08
941 4.98699628177501e-08
942 4.99576390746714e-08
943 4.98549983387608e-08
944 4.97750853019063e-08
945 4.97693665666432e-08
946 4.96251836352712e-08
947 4.97671235142327e-08
948 4.94212258632665e-08
949 4.95184026327866e-08
950 4.94360963576579e-08
951 4.94316385748483e-08
952 4.93038858435657e-08
953 4.93857151255828e-08
954 4.93289050391699e-08
955 4.91149649888278e-08
956 4.91392406614111e-08
957 4.90374077650735e-08
958 4.91441868177844e-08
959 4.88603343675731e-08
960 4.90036385907278e-08
961 4.87978004992673e-08
962 4.88780259604304e-08
963 4.86986883494467e-08
964 4.87643424449402e-08
965 4.84848716149067e-08
966 4.86476226211341e-08
967 4.84980483761444e-08
968 4.84527087984965e-08
969 4.8404616661557e-08
970 4.82686994789372e-08
971 4.82199074698375e-08
972 4.8353743769658e-08
973 4.79347550630393e-08
974 4.82046215171117e-08
975 4.81618941541306e-08
976 4.8059621941432e-08
977 4.78843916205118e-08
978 4.79967020139416e-08
979 4.78565692113575e-08
980 4.7828994318766e-08
981 4.77244534596855e-08
982 4.772541578002e-08
983 4.76455796436515e-08
984 4.75351868962637e-08
985 4.7470093858748e-08
986 4.74480650887177e-08
987 4.73810392741525e-08
988 4.73225984927161e-08
989 4.72211725979221e-08
990 4.72650578859923e-08
991 4.71040431507808e-08
992 4.71181832497614e-08
993 4.70397625641494e-08
994 4.70913090543368e-08
995 4.69494450906494e-08
996 4.69638006763695e-08
997 4.70106211242216e-08
998 4.68388962420185e-08
999 4.68581770302068e-08
1000 4.67692139431985e-08
1001 4.67093106611971e-08
1002 4.66017025202436e-08
1003 4.66765841384831e-08
1004 4.65095056320086e-08
1005 4.65636812662096e-08
1006 4.64298324445167e-08
1007 4.64824471706482e-08
1008 4.6400193625673e-08
1009 4.63949703356503e-08
1010 4.63757960871281e-08
1011 4.62870417232697e-08
1012 4.61327742962148e-08
1013 4.61497664994148e-08
1014 4.60130083211885e-08
1015 4.60213039545909e-08
1016 4.59218881152523e-08
1017 4.58185185499005e-08
1018 4.58103388292219e-08
1019 4.57545163952133e-08
1020 4.56679376716895e-08
1021 4.56855254860056e-08
1022 4.55714557315856e-08
1023 4.54344139418916e-08
1024 4.55023509646768e-08
1025 4.53356538167027e-08
1026 4.52748885638732e-08
1027 4.52277892211139e-08
1028 4.52505855068619e-08
1029 4.5102539252051e-08
1030 4.50575646024909e-08
1031 4.49357944083406e-08
1032 4.51926428775007e-08
1033 4.49365843144811e-08
1034 4.48962997234847e-08
1035 4.47718079668036e-08
1036 4.49299302041073e-08
1037 4.47387534980859e-08
1038 4.46133625540135e-08
1039 4.47629184974208e-08
1040 4.46894351965321e-08
1041 4.44844560909541e-08
1042 4.45735258177704e-08
1043 4.43984425306798e-08
1044 4.43452976992553e-08
1045 4.44500458961494e-08
1046 4.44758498066022e-08
1047 4.417272756152e-08
1048 4.42428180563725e-08
1049 4.42753176881361e-08
1050 4.42056563937365e-08
1051 4.41739072364333e-08
1052 4.40338045225586e-08
1053 4.41102051252917e-08
1054 4.39684787917338e-08
1055 4.39949409626328e-08
1056 4.38770347821471e-08
1057 4.3902137746521e-08
1058 4.37714434596836e-08
1059 4.38044262085491e-08
1060 4.37706782500102e-08
1061 4.37306418453742e-08
1062 4.35871498448703e-08
1063 4.3647764617849e-08
1064 4.35009683815712e-08
1065 4.35186393764608e-08
1066 4.35361898970532e-08
1067 4.34330154820417e-08
1068 4.34342680861732e-08
1069 4.33684588146965e-08
1070 4.3268653952544e-08
1071 4.33180844536807e-08
1072 4.3192712240181e-08
1073 4.31661250485105e-08
1074 4.32529074105847e-08
1075 4.30172799478212e-08
1076 4.29315806768082e-08
1077 4.30868601418721e-08
1078 4.2857005984609e-08
1079 4.30174460940291e-08
1080 4.2766908137315e-08
1081 4.34217721714258e-08
1082 4.27024961997535e-08
1083 4.27033249986719e-08
1084 4.29681213199862e-08
1085 4.2701403305867e-08
1086 4.24670729571863e-08
1087 4.24562446216248e-08
1088 4.26864282182526e-08
1089 4.25117204343461e-08
1090 4.24412632952231e-08
1091 4.23086888482471e-08
1092 4.23818010510946e-08
1093 4.22055278141986e-08
1094 4.22466337763883e-08
1095 4.23828086778544e-08
1096 4.21933629191074e-08
1097 4.21282059462902e-08
1098 4.22543190374824e-08
1099 4.20246580598027e-08
1100 4.21888610890875e-08
1101 4.18286222451103e-08
1102 4.19437754655139e-08
1103 4.20589287205564e-08
1104 4.1820412876481e-08
1105 4.17124947285075e-08
1106 4.16822401877459e-08
1107 4.16308374319163e-08
1108 4.17298166375257e-08
1109 4.15548989806114e-08
1110 4.17155008526349e-08
1111 4.15554928290263e-08
1112 4.1460854948161e-08
1113 4.146618614842e-08
1114 4.14147049999647e-08
1115 4.13187027525908e-08
1116 4.14810469638871e-08
1117 4.13753821600915e-08
1118 4.11272228525439e-08
1119 4.1099514128673e-08
1120 4.13096610278885e-08
1121 4.10663211634077e-08
1122 4.11138502426356e-08
1123 4.10326773891967e-08
1124 4.10559255206167e-08
1125 4.08572888920267e-08
1126 4.09016128422035e-08
1127 4.0913284530264e-08
1128 4.09567067519712e-08
1129 4.07748379366302e-08
1130 4.08647989595945e-08
1131 4.07295916114991e-08
1132 4.07240569220146e-08
1133 4.05579979170234e-08
1134 4.06503771802624e-08
1135 4.04893964398578e-08
1136 4.07393466530515e-08
1137 4.0443260064249e-08
1138 4.04836339278347e-08
1139 4.0354363449735e-08
1140 4.06209951967496e-08
1141 4.01947038906059e-08
1142 4.06357141728986e-08
1143 4.01344372829016e-08
1144 4.0374054605552e-08
1145 4.02898307692023e-08
1146 4.0255089571728e-08
1147 4.0251150462689e-08
1148 4.01913959999778e-08
1149 4.00401670632711e-08
1150 4.01256294007002e-08
1151 4.00539368210495e-08
1152 3.99986447381373e-08
1153 3.99961875641175e-08
1154 3.99638461119345e-08
1155 3.99164493891657e-08
1156 3.98446465785796e-08
1157 3.98833305734758e-08
1158 3.97953280875107e-08
1159 3.98053880139582e-08
1160 3.98635876275311e-08
1161 3.9591145264195e-08
1162 4.02836325659184e-08
1163 3.96165170303275e-08
1164 3.9643986786464e-08
1165 3.96154087753953e-08
1166 3.95709163836244e-08
1167 3.95137162880754e-08
1168 3.9537806122647e-08
1169 3.94030986445859e-08
1170 3.94586754497439e-08
1171 3.93726880054679e-08
1172 3.93833078522476e-08
1173 3.93159714702218e-08
1174 3.93873224540098e-08
1175 3.9240586163336e-08
1176 3.92284700245771e-08
1177 3.93540612000365e-08
1178 3.91235871616269e-08
1179 3.91117245306383e-08
1180 3.90295222201242e-08
1181 3.90725325555286e-08
1182 3.90026524398346e-08
1183 3.90199168861516e-08
1184 3.89809032850241e-08
1185 3.8840325481404e-08
1186 3.88699060436259e-08
1187 3.88649589591061e-08
1188 3.88303029490444e-08
1189 3.87928692513473e-08
1190 3.8783684522814e-08
1191 3.87117241191781e-08
1192 3.87510602877228e-08
1193 3.86724546117545e-08
1194 3.86308010249525e-08
1195 3.87463515088005e-08
1196 3.85876167038335e-08
1197 3.85776997349208e-08
1198 3.85226528487603e-08
1199 3.85753050471393e-08
1200 3.84918937983425e-08
1201 3.8536581984383e-08
1202 3.84686947544344e-08
1203 3.84393847887576e-08
1204 3.84577160628119e-08
1205 3.8417299481619e-08
1206 3.84333000252868e-08
1207 3.83081275109332e-08
1208 3.84138621850605e-08
1209 3.82654067390131e-08
1210 3.83051051767724e-08
1211 3.81793987692669e-08
1212 3.83176199434398e-08
1213 3.81734328328553e-08
1214 3.82583073491594e-08
1215 3.81698981388734e-08
1216 3.81097018199439e-08
1217 3.81726150761041e-08
1218 3.80676687758985e-08
1219 3.8181216277966e-08
1220 3.80089086369217e-08
1221 3.80933305978992e-08
1222 3.79861792823011e-08
1223 3.79695252218371e-08
1224 3.79786195870979e-08
1225 3.8043422323164e-08
1226 3.7887334151443e-08
1227 3.79814909887699e-08
1228 3.78706210977242e-08
1229 3.78932129190268e-08
1230 3.77942585984004e-08
1231 3.78498578423869e-08
1232 3.77762575746177e-08
1233 3.78372744254118e-08
1234 3.77690566204514e-08
1235 3.77575291580223e-08
1236 3.77537780684722e-08
1237 3.76938131921856e-08
1238 3.76654651840225e-08
1239 3.76961793435715e-08
1240 3.76337180836295e-08
1241 3.76070475881507e-08
1242 3.75639840538566e-08
1243 3.75797261223942e-08
1244 3.76022043844237e-08
1245 3.74655505630539e-08
1246 3.76307346243721e-08
1247 3.74849760051976e-08
1248 3.75604665268092e-08
1249 3.74397827407336e-08
1250 3.74413531984974e-08
1251 3.74117008580255e-08
1252 3.74106085727632e-08
1253 3.7445114545176e-08
1254 3.73107396595129e-08
1255 3.73720741815298e-08
1256 3.74028268151916e-08
1257 3.7283399414334e-08
1258 3.7452707423391e-08
1259 3.72332391411767e-08
1260 3.71696658579834e-08
1261 3.72691330932096e-08
1262 3.71566481252028e-08
1263 3.72341500316509e-08
1264 3.72069684041465e-08
1265 3.71116631373569e-08
1266 3.7140603963981e-08
1267 3.70946616294887e-08
1268 3.71655261539594e-08
1269 3.70214982201578e-08
1270 3.70213142024678e-08
1271 3.70546212069556e-08
1272 3.70175100532499e-08
1273 3.69283056730341e-08
1274 3.69811304927747e-08
1275 3.70122460545685e-08
1276 3.68977255537084e-08
1277 3.69258742725087e-08
1278 3.69270647223585e-08
1279 3.68608744819587e-08
1280 3.70315203005456e-08
1281 3.67336432490761e-08
1282 3.68628941076476e-08
1283 3.68488603205153e-08
1284 3.67841719421946e-08
1285 3.68479282689682e-08
1286 3.67288222086204e-08
1287 3.67679208187477e-08
1288 3.67115062361867e-08
1289 3.67626327085757e-08
1290 3.67250972230782e-08
1291 3.65994632238742e-08
1292 3.65741138472764e-08
1293 3.66033104501895e-08
1294 3.65969761814133e-08
1295 3.65383472373804e-08
1296 3.65872019506552e-08
1297 3.64746961354356e-08
1298 3.66230808606094e-08
1299 3.64264712977569e-08
1300 3.651689937334e-08
1301 3.64566747725892e-08
1302 3.64705124005216e-08
1303 3.64190986223978e-08
1304 3.63917708147143e-08
1305 3.65048590741379e-08
1306 3.62673468153885e-08
1307 3.64216743863333e-08
1308 3.6408958244416e-08
1309 3.63529083634573e-08
1310 3.64245427467713e-08
1311 3.63020139985126e-08
1312 3.63399036626966e-08
1313 3.63546428853168e-08
1314 3.62900266053234e-08
1315 3.61862240652044e-08
1316 3.62872483773424e-08
1317 3.62806942136862e-08
1318 3.60895579678555e-08
1319 3.6201719258111e-08
1320 3.61391673208145e-08
1321 3.60825098795203e-08
1322 3.61499899275142e-08
1323 3.61710267726689e-08
1324 3.58837346515761e-08
1325 3.60685211484579e-08
1326 3.59939650067531e-08
1327 3.59188288757695e-08
1328 3.59924587035598e-08
1329 3.60107811478993e-08
1330 3.58654060294006e-08
1331 3.59164740654228e-08
1332 3.59982487653232e-08
1333 3.59153455750105e-08
1334 3.58685138335257e-08
1335 3.58465194876256e-08
1336 3.58061490658423e-08
1337 3.5853859842061e-08
1338 3.5843375500666e-08
1339 3.5723664718823e-08
1340 3.58304206751203e-08
1341 3.58712001522488e-08
1342 3.57788168414697e-08
1343 3.57708656392175e-08
1344 3.56819430327171e-08
1345 3.58111306434372e-08
1346 3.5581417613173e-08
1347 3.56700738538551e-08
1348 3.57247733435706e-08
1349 3.56779007620878e-08
1350 3.56377216006454e-08
1351 3.5618182825492e-08
1352 3.5774959309065e-08
1353 3.57338098156967e-08
1354 3.56415802854615e-08
1355 3.55604559602041e-08
1356 3.56681283610882e-08
1357 3.55967940153246e-08
1358 3.54804515663165e-08
1359 3.57632505043615e-08
1360 3.54188348801188e-08
1361 3.55569662161548e-08
1362 3.54839288400077e-08
1363 3.55653801519251e-08
1364 3.54479293739329e-08
1365 3.54039772982873e-08
1366 3.55500885415028e-08
1367 3.53771195461539e-08
1368 3.54609469089828e-08
1369 3.53989638967045e-08
1370 3.54848395746066e-08
1371 3.53777685413448e-08
1372 3.53352914831406e-08
1373 3.53682853405646e-08
1374 3.52999482809979e-08
1375 3.53396566759034e-08
1376 3.53158754372585e-08
1377 3.53140629840709e-08
1378 3.53081633920338e-08
1379 3.5336909767536e-08
1380 3.53669938449874e-08
1381 3.52742092085689e-08
1382 3.52576313412678e-08
1383 3.51894271649611e-08
1384 3.52072720755103e-08
1385 3.51311760483641e-08
1386 3.52701317877235e-08
1387 3.51124235963285e-08
1388 3.51777536832243e-08
1389 3.51995129614924e-08
1390 3.5047716601011e-08
1391 3.51562016737139e-08
1392 3.50379724448624e-08
1393 3.5062653151674e-08
1394 3.50574623558808e-08
1395 3.49871297903315e-08
1396 3.50137007579798e-08
1397 3.49466606046622e-08
1398 3.47768717776642e-08
1399 3.46313665487985e-08
1400 3.39132217052063e-08
1401 3.33187329416562e-08
1402 3.32573683244908e-08
1403 3.32004694778565e-08
1404 3.30662505327872e-08
1405 3.29737982827094e-08
1406 3.28913411138387e-08
1407 3.29056405851125e-08
1408 3.28166240692873e-08
1409 3.28563647845614e-08
1410 3.27088583996771e-08
1411 3.26580964974399e-08
1412 3.27044688684941e-08
1413 3.25946619176287e-08
1414 3.25973549694103e-08
1415 3.25918338838216e-08
1416 3.25818939889144e-08
1417 3.25405744276974e-08
1418 3.2536131251093e-08
1419 3.24491043134367e-08
1420 3.24368636973604e-08
1421 3.24490578138548e-08
1422 3.23545586237151e-08
1423 3.23702316661345e-08
1424 3.23338902297188e-08
1425 3.23093400919072e-08
1426 3.22733117849028e-08
1427 3.21280135929802e-08
1428 3.22149724748533e-08
1429 3.21502204476687e-08
1430 3.21392878342985e-08
1431 3.20784159031362e-08
1432 3.2054536342363e-08
1433 3.20281236017017e-08
1434 3.1824389950863e-08
1435 3.18893226567418e-08
1436 3.1973949053854e-08
1437 3.17285103603737e-08
1438 3.17229476622938e-08
1439 3.17210362223985e-08
1440 3.16258785248991e-08
1441 3.17140828851592e-08
1442 3.16081039790639e-08
1443 3.15768830720931e-08
1444 3.15220913507108e-08
1445 3.15542817874714e-08
1446 3.14438024708874e-08
1447 3.15426610786318e-08
1448 3.13882286993028e-08
1449 3.15105469779198e-08
1450 3.13633077504427e-08
1451 3.1377080524253e-08
1452 3.13722165943275e-08
1453 3.13126566737765e-08
1454 3.13328462293594e-08
1455 3.14506938046133e-08
1456 3.14114118923348e-08
1457 3.12214213120665e-08
1458 3.12150469441441e-08
1459 3.14047482077306e-08
1460 3.13709489111469e-08
1461 3.12726853808343e-08
1462 3.11878285075284e-08
1463 3.11863979061133e-08
1464 3.11696561170471e-08
1465 3.1191530809993e-08
1466 3.11196009232795e-08
1467 3.11599132840623e-08
1468 3.10843117790061e-08
1469 3.10389316733328e-08
1470 3.11998927430279e-08
1471 3.10619579967364e-08
1472 3.11231616306751e-08
1473 3.11169309195058e-08
1474 3.09477956921267e-08
1475 3.09250213474943e-08
1476 3.09453195863618e-08
1477 3.10387257758116e-08
1478 3.09170882208942e-08
1479 3.08925513494707e-08
1480 3.09541761223109e-08
1481 3.08504173189839e-08
1482 3.08767523899967e-08
1483 3.07879179854975e-08
1484 3.08490212231893e-08
1485 3.07083730679558e-08
1486 3.09078635883919e-08
1487 3.07270835919748e-08
1488 3.08024448725241e-08
1489 3.06531330196425e-08
1490 3.07041040550349e-08
1491 3.06476476019668e-08
1492 3.0724029298268e-08
1493 3.05886550893142e-08
1494 3.05943502765249e-08
1495 3.05540177520935e-08
1496 3.0591332292218e-08
1497 3.06227187035768e-08
1498 3.05603502854801e-08
1499 3.05394033839335e-08
1500 3.04004892062171e-08
1501 3.04104032271546e-08
1502 3.04973887461646e-08
1503 3.04259226544534e-08
1504 3.0465511522082e-08
1505 3.03429197292537e-08
1506 3.0394893886565e-08
1507 3.0453907239103e-08
1508 3.03791931409725e-08
1509 3.03992121064578e-08
1510 3.03739727232077e-08
1511 3.02395998457161e-08
1512 3.04022563596407e-08
1513 3.0237597856364e-08
1514 3.03013788084394e-08
1515 3.02346872064208e-08
1516 3.02637078123391e-08
1517 3.02249627133211e-08
1518 3.01486655418515e-08
1519 3.01041433851124e-08
1520 3.02426698280955e-08
1521 3.01650382000807e-08
1522 3.00876842683762e-08
1523 3.01888465976674e-08
1524 3.00026594417524e-08
1525 3.01514460312458e-08
1526 2.99987949576863e-08
1527 3.01394372126396e-08
1528 2.99845863391823e-08
1529 3.00458866161479e-08
1530 2.99421803043876e-08
1531 3.01126360825466e-08
1532 2.9783408594608e-08
1533 2.98837041297073e-08
1534 2.99974938220426e-08
1535 2.9840435314954e-08
1536 2.9843804778773e-08
1537 2.9901690169476e-08
1538 2.98934136446771e-08
1539 2.98357502565017e-08
1540 2.98246820871206e-08
1541 2.96993106743138e-08
1542 2.98783997120244e-08
1543 2.97421740307335e-08
1544 2.97853337395404e-08
1545 2.96676535621598e-08
1546 2.96789320786139e-08
1547 2.96590812255415e-08
1548 2.96679038809256e-08
1549 2.96906650913975e-08
1550 2.95967904958561e-08
1551 2.95980398630213e-08
1552 2.97375201542271e-08
1553 2.94598671474189e-08
1554 2.96500224485907e-08
1555 2.95445745953637e-08
1556 2.95780501189391e-08
1557 2.94704894441278e-08
1558 2.95750755544555e-08
1559 2.95274538887336e-08
1560 2.94567925741562e-08
1561 2.94401273907008e-08
1562 2.94195945252929e-08
1563 2.94070479639474e-08
1564 2.94745660818219e-08
1565 2.93773832079625e-08
1566 2.94007242556127e-08
1567 2.93765563784687e-08
1568 2.93040811226719e-08
1569 2.94496700566915e-08
1570 2.92820405531913e-08
1571 2.93901634671467e-08
1572 2.92199964948558e-08
1573 2.92826120866785e-08
1574 2.92129500070182e-08
1575 2.92115747966282e-08
1576 2.92131065756651e-08
1577 2.91819960558382e-08
1578 2.91984244544796e-08
1579 2.91711532648398e-08
1580 2.92079244408328e-08
1581 2.92684108870178e-08
1582 2.9168705728333e-08
1583 2.91326566208561e-08
1584 2.91002125625273e-08
1585 2.91093476503246e-08
1586 2.90737113013417e-08
1587 2.90634909021481e-08
1588 2.90575134731519e-08
1589 2.90701802895255e-08
1590 2.90057880851791e-08
1591 2.89552724260789e-08
1592 2.89752651803088e-08
1593 2.90348840046306e-08
1594 2.89842204060919e-08
1595 2.89792475122175e-08
1596 2.89326148052993e-08
1597 2.89017571294403e-08
1598 2.89136039827165e-08
1599 2.89153672263787e-08
1600 2.89592383714288e-08
1601 2.8808436922767e-08
1602 2.89315974482118e-08
1603 2.88342938059571e-08
1604 2.88089255593427e-08
1605 2.87937982000885e-08
1606 2.88040838094528e-08
1607 2.89280015881577e-08
1608 2.87529113442986e-08
1609 2.88157950845136e-08
1610 2.86904911519503e-08
1611 2.86631445463037e-08
1612 2.88473591394389e-08
1613 2.87480092764358e-08
1614 2.87707806456705e-08
1615 2.8673945859925e-08
1616 2.87527373348251e-08
1617 2.87034537267417e-08
1618 2.86006217621315e-08
1619 2.86361704797677e-08
1620 2.84876225098429e-08
1621 2.85351026912961e-08
1622 2.86905114585956e-08
1623 2.86167296035256e-08
1624 2.86602197845376e-08
1625 2.85203700026715e-08
1626 2.86146593924119e-08
1627 2.86084414644616e-08
1628 2.85127599739532e-08
1629 2.85825063895961e-08
1630 2.85490549152279e-08
1631 2.85555556831518e-08
1632 2.85121921110854e-08
1633 2.84694111180039e-08
1634 2.83670200859065e-08
1635 2.84221892729031e-08
1636 2.84696867072221e-08
1637 2.84079124619474e-08
1638 2.83740615667494e-08
1639 2.84758945294783e-08
1640 2.83011528651178e-08
1641 2.83832370414627e-08
1642 2.83000478306095e-08
1643 2.83675922051474e-08
1644 2.82939737304977e-08
1645 2.83485724359434e-08
1646 2.83073885463114e-08
1647 2.83283021170755e-08
1648 2.83047664424441e-08
1649 2.82728538676391e-08
1650 2.8246726632708e-08
1651 2.82885210964867e-08
1652 2.81601106001395e-08
1653 2.82793208219445e-08
1654 2.81974994001954e-08
1655 2.81980420059336e-08
1656 2.81876603476228e-08
1657 2.81766876975009e-08
1658 2.81349766688699e-08
1659 2.81793985885681e-08
1660 2.82172866723496e-08
1661 2.809811797988e-08
1662 2.8100318318458e-08
1663 2.80538043018197e-08
1664 2.80813929289803e-08
1665 2.80383293747022e-08
1666 2.80996741404138e-08
1667 2.80030078911864e-08
1668 2.80731984045879e-08
1669 2.79909063588546e-08
1670 2.80302330999405e-08
1671 2.79628745585292e-08
1672 2.79451194298286e-08
1673 2.80131074252843e-08
1674 2.79871303558998e-08
1675 2.78837553924216e-08
1676 2.79983959107222e-08
1677 2.79526074514758e-08
1678 2.7901580441414e-08
1679 2.78893249306966e-08
1680 2.80050802139442e-08
1681 2.78897582979276e-08
1682 2.78790461897271e-08
1683 2.78179148921476e-08
1684 2.7970838607283e-08
1685 2.7876004240257e-08
1686 2.78092650798367e-08
1687 2.78672511206945e-08
1688 2.7845357413292e-08
1689 2.78192773104369e-08
1690 2.78149342702871e-08
1691 2.77204040227685e-08
1692 2.79658695931939e-08
1693 2.77662921099875e-08
1694 2.78502889051735e-08
1695 2.76965765929615e-08
1696 2.77231101707409e-08
1697 2.78821867512047e-08
1698 2.76745168741321e-08
1699 2.78063297818321e-08
1700 2.76261739513561e-08
1701 2.77471253113948e-08
1702 2.77251972373715e-08
1703 2.76314246802079e-08
1704 2.76331197737178e-08
1705 2.7720556686317e-08
1706 2.7563151950516e-08
1707 2.76801677913108e-08
1708 2.75843394358377e-08
1709 2.76719048837304e-08
1710 2.76457563410659e-08
1711 2.7603746561855e-08
1712 2.75966145262263e-08
1713 2.74950339120839e-08
1714 2.75970475639431e-08
1715 2.75079881636442e-08
1716 2.75323636316216e-08
1717 2.76108679435616e-08
1718 2.74620871938414e-08
1719 2.7511625891985e-08
1720 2.75931974398347e-08
1721 2.74758412405207e-08
1722 2.7481672808749e-08
1723 2.75089782657556e-08
1724 2.74031090334859e-08
1725 2.74734573348923e-08
1726 2.74092108040191e-08
1727 2.75109828001829e-08
1728 2.74171988048355e-08
1729 2.73650285514959e-08
1730 2.74168205377512e-08
1731 2.74621644440476e-08
1732 2.73283424858661e-08
1733 2.7423143071803e-08
1734 2.73455801463873e-08
1735 2.73244164360475e-08
1736 2.73742878714733e-08
1737 2.73562597695731e-08
1738 2.7318721627978e-08
1739 2.73480717782881e-08
1740 2.72722426418737e-08
1741 2.72877752409695e-08
1742 2.7316397082755e-08
1743 2.72115129609229e-08
1744 2.74079705787633e-08
1745 2.7205366882499e-08
1746 2.73370375116766e-08
1747 2.71611378510617e-08
1748 2.73181977870074e-08
1749 2.71880475869457e-08
1750 2.71608168963544e-08
1751 2.71772536101222e-08
1752 2.71090691428277e-08
1753 2.7151348603649e-08
1754 2.73208308972928e-08
1755 2.70616769639798e-08
1756 2.71418888126984e-08
1757 2.71219905990661e-08
1758 2.71394188696705e-08
1759 2.71144318325645e-08
1760 2.70578096807084e-08
1761 2.70673444157099e-08
1762 2.70375775034415e-08
1763 2.71769372635067e-08
1764 2.69691395329286e-08
1765 2.70962254906859e-08
1766 2.70060038761732e-08
1767 2.70263869783793e-08
1768 2.69539110611117e-08
1769 2.71268098877009e-08
1770 2.69327150913545e-08
1771 2.70256423521298e-08
1772 2.69499816561991e-08
1773 2.69329060170742e-08
1774 2.69669623302882e-08
1775 2.68852474194503e-08
1776 2.70618526434507e-08
1777 2.685336206254e-08
1778 2.69398807525301e-08
1779 2.69127320605689e-08
1780 2.68872400822628e-08
1781 2.68165232634177e-08
1782 2.69186256347531e-08
1783 2.68656174863269e-08
1784 2.68987483442729e-08
1785 2.68420857436613e-08
1786 2.68000151908421e-08
1787 2.68659831489382e-08
1788 2.68264978656729e-08
1789 2.68769687590265e-08
1790 2.67683605529312e-08
1791 2.67714183961321e-08
1792 2.67381539144074e-08
1793 2.67335598145291e-08
1794 2.67140545795463e-08
1795 2.67347460995992e-08
1796 2.66856234739654e-08
1797 2.6700490370235e-08
1798 2.66834899173007e-08
1799 2.66838368883082e-08
1800 2.66622081490642e-08
1801 2.6602108589513e-08
1802 2.66751074277094e-08
1803 2.65791938106608e-08
1804 2.66086965872159e-08
1805 2.65462414384965e-08
1806 2.65944747771529e-08
1807 2.66399718299448e-08
1808 2.65134165795544e-08
1809 2.65837048430972e-08
1810 2.64994846297384e-08
1811 2.66075213571915e-08
1812 2.64185296118269e-08
1813 2.64735282314854e-08
1814 2.65128966711048e-08
1815 2.64646368680621e-08
1816 2.64209887217426e-08
1817 2.64931831831516e-08
1818 2.63758233297606e-08
1819 2.64752091916343e-08
1820 2.64550859156998e-08
1821 2.64392702851168e-08
1822 2.63899981314264e-08
1823 2.64430971985696e-08
1824 2.63860791170911e-08
1825 2.64019898543744e-08
1826 2.63853043479534e-08
1827 2.63168463557895e-08
1828 2.63949210578263e-08
1829 2.63558283489784e-08
1830 2.63375569405921e-08
1831 2.63076673167051e-08
1832 2.6316840673335e-08
1833 2.63554111223918e-08
1834 2.63148418636616e-08
1835 2.62963079773471e-08
1836 2.62620244482203e-08
1837 2.62813127231976e-08
1838 2.62465429045067e-08
1839 2.62707925984262e-08
1840 2.62209588818285e-08
1841 2.62131524354503e-08
1842 2.62264692615366e-08
1843 2.62343072656712e-08
1844 2.62044696073271e-08
1845 2.61550149126544e-08
1846 2.61967055701362e-08
1847 2.62036169065283e-08
1848 2.6172117356893e-08
1849 2.61192933915799e-08
1850 2.61595463170172e-08
1851 2.61459946501175e-08
1852 2.6097666151248e-08
1853 2.61210578147431e-08
1854 2.61372417921235e-08
1855 2.60747308056253e-08
1856 2.60877143662963e-08
1857 2.57240115194568e-08
1858 2.59864037683855e-08
1859 2.59994728230017e-08
1860 2.60516702621505e-08
1861 2.58887132924546e-08
1862 2.60948995040033e-08
1863 2.59035001621832e-08
1864 2.59241300382129e-08
1865 2.58808336580651e-08
1866 2.5974423288333e-08
1867 2.5857274757346e-08
1868 2.58915656689007e-08
1869 2.59122968218062e-08
1870 2.58754490320978e-08
1871 2.58800241065282e-08
1872 2.57949883417652e-08
1873 2.58326914367446e-08
1874 2.58190995345853e-08
1875 2.57837196024902e-08
1876 2.57824186246092e-08
1877 2.57951011973789e-08
1878 2.56523664509123e-08
1879 2.58909142870634e-08
1880 2.56986044355756e-08
1881 2.57612337787316e-08
1882 2.57103384212476e-08
1883 2.5708068785435e-08
1884 2.5684205583576e-08
1885 2.5619217177808e-08
1886 2.5848145291274e-08
1887 2.56461329090074e-08
1888 2.5601013377452e-08
1889 2.57015188003251e-08
1890 2.56984757347478e-08
1891 2.53378971799556e-08
1892 2.56226662913583e-08
1893 2.54687687475386e-08
1894 2.55176148986624e-08
1895 2.54757199743549e-08
1896 2.54544361303344e-08
1897 2.54478370516065e-08
1898 2.53149195054148e-08
1899 2.55404793905178e-08
1900 2.54396925439826e-08
1901 2.54089869048402e-08
1902 2.54397395605954e-08
1903 2.52210019855426e-08
1904 2.54790694110829e-08
1905 2.53907945677589e-08
1906 2.53535623911949e-08
1907 2.54180226058054e-08
1908 2.53561777823208e-08
1909 2.51977715298723e-08
1910 2.53784114537803e-08
1911 2.53169068665793e-08
1912 2.54560329960851e-08
1913 2.52498138468837e-08
1914 2.54889566027039e-08
1915 2.51589794214579e-08
1916 2.51146379861567e-08
1917 2.53932118388089e-08
1918 2.5273226772704e-08
1919 2.53042176022733e-08
1920 2.52693189528586e-08
1921 2.50950923456061e-08
1922 2.51232760457576e-08
1923 2.52362870815093e-08
1924 2.52123860761122e-08
1925 2.51697502656745e-08
1926 2.52075835237564e-08
1927 2.50954337217557e-08
1928 2.50590895883862e-08
1929 2.50737000956347e-08
1930 2.51674598957807e-08
1931 2.51914771826067e-08
1932 2.49612353301343e-08
1933 2.51242263048512e-08
1934 2.51536036454114e-08
1935 2.51037180233338e-08
1936 2.51666976445186e-08
1937 2.51045684952489e-08
1938 2.51820439570727e-08
1939 2.50664743417506e-08
1940 2.51483688067378e-08
1941 2.50710044403801e-08
1942 2.51034105561709e-08
1943 2.58102133451343e-08
1944 2.52911483353824e-08
1945 2.53940795311713e-08
1946 2.53515345101096e-08
1947 2.54020264713883e-08
1948 2.53403210155323e-08
1949 2.53440131758875e-08
1950 2.52705895775751e-08
1951 2.52227434445196e-08
1952 2.52538577545369e-08
1953 2.53162414776131e-08
1954 2.53707582791307e-08
1955 2.51865219436098e-08
1956 2.53394388862738e-08
1957 2.52670234450747e-08
1958 2.51701773190671e-08
1959 2.52581422038878e-08
1960 2.51909043945675e-08
1961 2.52008567780404e-08
1962 2.51402299912273e-08
1963 2.51661491645905e-08
1964 2.51660815621113e-08
1965 2.51508990678495e-08
1966 2.52350970364468e-08
1967 2.49694505027875e-08
1968 2.519936435319e-08
1969 2.50357924549416e-08
1970 2.50742936698245e-08
1971 2.49434256611725e-08
1972 2.52762901535863e-08
1973 2.50624780135933e-08
1974 2.50143391065105e-08
1975 2.51228086325384e-08
1976 2.50761854408976e-08
1977 2.49874236721848e-08
1978 2.49414932811387e-08
1979 2.49946196269057e-08
1980 2.50397628290822e-08
1981 2.50981169698239e-08
1982 2.49452193417099e-08
1983 2.49050464231626e-08
1984 2.49305909528763e-08
1985 2.48424742494979e-08
1986 2.49196172383836e-08
1987 2.4909941031992e-08
1988 2.48592216413046e-08
1989 2.49469235957811e-08
1990 2.49219211023721e-08
1991 2.47946729592385e-08
1992 2.4845945459484e-08
1993 2.5012985712225e-08
1994 2.51045369592529e-08
1995 2.45852555381676e-08
1996 2.48206830761877e-08
1997 2.49265517323627e-08
1998 2.47583248087491e-08
1999 2.48638648936428e-08
2000 2.47071612953231e-08
2001 2.49206896012488e-08
2002 2.47467111620381e-08
2003 2.49181665734621e-08
2004 2.46457301747993e-08
2005 2.47275049682916e-08
2006 2.48525452339887e-08
2007 2.4725760315647e-08
2008 2.4754327635268e-08
2009 2.46262518011076e-08
2010 2.47634545613717e-08
2011 2.47231794727742e-08
2012 2.47396988293236e-08
2013 2.45532180129793e-08
2014 2.46310307263631e-08
2015 2.46976782148911e-08
2016 2.45660618135579e-08
2017 2.48449987645394e-08
2018 2.45830133344116e-08
2019 2.45749460219269e-08
2020 2.45206955673538e-08
2021 2.46130277237189e-08
2022 2.46214898435015e-08
2023 2.45436529897614e-08
2024 2.44689583742685e-08
2025 2.45671922385338e-08
2026 2.45915170927757e-08
2027 2.4513764200873e-08
2028 2.45467057427007e-08
2029 2.44595578322571e-08
2030 2.47158791755187e-08
2031 2.43239314170474e-08
2032 2.45186939018538e-08
2033 2.44924343928421e-08
2034 2.45200221876685e-08
2035 2.44200220930191e-08
2036 2.44677613682187e-08
2037 2.4424860816552e-08
2038 2.44730707685825e-08
2039 2.43609914865051e-08
2040 2.44203907469043e-08
2041 2.43805499415828e-08
2042 2.43143605437313e-08
2043 2.43295797673904e-08
2044 2.43403190383074e-08
2045 2.42585150256325e-08
2046 2.43119513867462e-08
2047 2.43789773703984e-08
2048 2.43109985537115e-08
2049 2.43598907084808e-08
2050 2.43273152371604e-08
2051 2.43429077901869e-08
2052 2.4242503462113e-08
2053 2.42984383631573e-08
2054 2.42242315957597e-08
2055 2.44566516528488e-08
2056 2.4071652523161e-08
2057 2.42158801007086e-08
2058 2.41450027219647e-08
2059 2.44175603779651e-08
2060 2.42515352177897e-08
2061 2.4041250436535e-08
2062 2.41901985856829e-08
2063 2.41378463061714e-08
2064 2.43459166417992e-08
2065 2.40502100683493e-08
2066 2.41888429376447e-08
2067 2.42594496915327e-08
2068 2.41751972815951e-08
2069 2.40491607060855e-08
2070 2.41906308284801e-08
2071 2.41195388158699e-08
2072 2.41033601315888e-08
2073 2.40753223590406e-08
2074 2.40913383550145e-08
2075 2.40820942356557e-08
2076 2.39694732753204e-08
2077 2.41017538135102e-08
2078 2.3967790335977e-08
2079 2.40663380037809e-08
2080 2.38426911634004e-08
2081 2.40054325344463e-08
2082 2.39081047888057e-08
2083 2.40024264531735e-08
2084 2.39361674738703e-08
2085 2.39542178266383e-08
2086 2.39288414540795e-08
2087 2.37301431496029e-08
2088 2.39815404351473e-08
2089 2.39640352426296e-08
2090 2.3738361950576e-08
2091 2.38782566557294e-08
2092 2.38110474963804e-08
2093 2.37583646466311e-08
2094 2.39057104071128e-08
2095 2.37696562825906e-08
2096 2.3864782264793e-08
2097 2.38113229140691e-08
2098 2.38117428774665e-08
2099 2.38072921130428e-08
2100 2.38290457306167e-08
2101 2.36920006677632e-08
2102 2.36592893340104e-08
2103 2.39054354953527e-08
2104 2.36484520924618e-08
2105 2.36586221160628e-08
2106 2.36391644587997e-08
2107 2.38073261235039e-08
2108 2.35314789703844e-08
2109 2.36147245135099e-08
2110 2.36074537516284e-08
2111 2.36903281211065e-08
2112 2.35676246385097e-08
2113 2.35301859788928e-08
2114 2.36025219259028e-08
2115 2.35456383728128e-08
2116 2.33316793241123e-08
2117 2.33798693388687e-08
2118 2.33869954225918e-08
2119 2.34151225798618e-08
2120 2.33554494567523e-08
2121 2.33534876380448e-08
2122 2.34102730926056e-08
2123 2.33911362786943e-08
2124 2.31922103909321e-08
2125 2.3393471261679e-08
2126 2.32585332380353e-08
2127 2.34044739475525e-08
2128 2.33650292270626e-08
2129 2.32330039320328e-08
2130 2.33618849748707e-08
2131 2.32533871762364e-08
2132 2.34044368380149e-08
2133 2.32308251536528e-08
2134 2.32197406185852e-08
2135 2.33023452730352e-08
2136 2.29227924443132e-08
2137 2.33377220776321e-08
2138 2.31620022653756e-08
2139 2.32733366434257e-08
2140 2.31182437882005e-08
2141 2.33378190074296e-08
2142 2.32424730414182e-08
2143 2.3252518339234e-08
2144 2.31631970899482e-08
2145 2.28521522743774e-08
2146 2.32426607896796e-08
2147 2.32756810345514e-08
2148 2.30110299386244e-08
2149 2.32607778535288e-08
2150 2.3074557829128e-08
2151 2.31527123756292e-08
2152 2.31843327168635e-08
2153 2.32706144109862e-08
2154 2.28209440471527e-08
2155 2.28320511587166e-08
2156 2.27979996501038e-08
2157 2.32738846906999e-08
2158 2.29957853890017e-08
2159 2.31920345294956e-08
2160 2.268050999088e-08
2161 2.31896760939376e-08
2162 2.30979157689326e-08
2163 2.30905032483353e-08
2164 2.31511444571675e-08
2165 2.2800232655551e-08
2166 2.31259868981315e-08
2167 2.26905504169261e-08
2168 2.31336855733844e-08
2169 2.29956575702461e-08
2170 2.29702112443109e-08
2171 2.29781201787072e-08
2172 2.29873486413679e-08
2173 2.29695914193417e-08
2174 2.30849604514471e-08
2175 2.2950243355413e-08
2176 2.281856646702e-08
2177 2.27901420890575e-08
2178 2.28884469601898e-08
2179 2.29432690950215e-08
2180 2.25963599512768e-08
2181 2.29758560812421e-08
2182 2.28034672239419e-08
2183 2.29751630836939e-08
2184 2.2772058634879e-08
2185 2.29577047227147e-08
2186 2.271062314807e-08
2187 2.25769137190968e-08
2188 2.29825961387542e-08
2189 2.26603271920034e-08
2190 2.31640487563523e-08
2191 2.24503959440625e-08
2192 2.29976460811576e-08
2193 2.28705544839114e-08
2194 2.26525156516999e-08
2195 2.29265044885407e-08
2196 2.27423208354338e-08
2197 2.2926685495861e-08
2198 2.27221939429478e-08
2199 2.28339440200287e-08
2200 2.26842103396852e-08
2201 2.26750657991159e-08
2202 2.27688065469467e-08
2203 2.26295594278891e-08
2204 2.26848757894915e-08
2205 2.2633556809537e-08
2206 2.26090605000007e-08
2207 2.26070471242013e-08
2208 2.27840645330435e-08
2209 2.26766932607259e-08
2210 2.25492467613719e-08
2211 2.26908276863558e-08
2212 2.25790768662026e-08
2213 2.25602092702726e-08
2214 2.26654767087053e-08
2215 2.24826217097851e-08
2216 2.25414738193042e-08
2217 2.25461178980924e-08
2218 2.25564731282102e-08
2219 2.26297389511743e-08
2220 2.26095518490732e-08
2221 2.25900679614588e-08
2222 2.25199208632176e-08
2223 2.25930897941318e-08
2224 2.24635001754292e-08
2225 2.25470356461921e-08
2226 2.25237051666438e-08
2227 2.25525974956176e-08
2228 2.2499577527979e-08
2229 2.25432196533371e-08
2230 2.264228486204e-08
2231 2.23562240294628e-08
2232 2.24168906937106e-08
2233 2.27338385172526e-08
2234 2.21835532824288e-08
2235 2.26978516539855e-08
2236 2.21947666376732e-08
2237 2.27106515501285e-08
2238 2.21301795022555e-08
2239 2.26740774004197e-08
2240 2.23054850168047e-08
2241 2.24572431128944e-08
2242 2.25599461463055e-08
2243 2.22906346507568e-08
2244 2.24287162921e-08
2245 2.24053413014902e-08
2246 2.23137759868264e-08
2247 2.22813497089414e-08
2248 2.25398445911074e-08
2249 2.22220789084515e-08
2250 2.25035712636545e-08
2251 2.22622318031274e-08
2252 2.2374963452032e-08
2253 2.22298144686439e-08
2254 2.21865667763277e-08
2255 2.25055444451616e-08
2256 2.20688575899297e-08
2257 2.23767496416727e-08
2258 2.23848441408547e-08
2259 2.22085651223614e-08
2260 2.22279872462705e-08
2261 2.2346812867502e-08
2262 2.22378017105118e-08
2263 2.21692857261813e-08
2264 2.21428969413573e-08
2265 2.23869043257663e-08
2266 2.2115686491575e-08
2267 2.22316054057092e-08
2268 2.2318448094083e-08
2269 2.20784431584908e-08
2270 2.21532294475413e-08
2271 2.22695829579234e-08
2272 2.21109719692469e-08
2273 2.2248503716793e-08
2274 2.22377433797272e-08
2275 2.20805866474327e-08
2276 2.21262999522409e-08
2277 2.22007632928456e-08
2278 2.22059215817527e-08
2279 2.21104366052716e-08
2280 2.20069039786885e-08
2281 2.24120538900863e-08
2282 2.19546399174986e-08
2283 2.22407897325638e-08
2284 2.21716997386157e-08
2285 2.2047450747964e-08
2286 2.21402644499102e-08
2287 2.20850487483437e-08
2288 2.21686084604933e-08
2289 2.21128769062329e-08
2290 2.20963720123368e-08
2291 2.20104605502813e-08
2292 2.21697087570139e-08
2293 2.20224819923454e-08
2294 2.21113781582183e-08
2295 2.19930331883944e-08
2296 2.20990513455366e-08
2297 2.20427643140564e-08
2298 2.21831256714822e-08
2299 2.19076811392993e-08
2300 2.21828085648079e-08
2301 2.19908346640318e-08
2302 2.19469906422853e-08
2303 2.20500807330826e-08
2304 2.20294306437774e-08
2305 2.21453278875794e-08
2306 2.19670198707833e-08
2307 2.21523369584586e-08
2308 2.18648462011473e-08
2309 2.2079294830557e-08
2310 2.17917698721592e-08
2311 2.20544981904824e-08
2312 2.18967473029075e-08
2313 2.20391960850108e-08
2314 2.19321937140471e-08
2315 2.18952825624674e-08
2316 2.1954505602606e-08
2317 2.20573140355951e-08
2318 2.17397521186102e-08
2319 2.19834051170542e-08
2320 2.18062105549199e-08
2321 2.19743867443389e-08
2322 2.19011796398494e-08
2323 2.19169271550301e-08
2324 2.18119264691063e-08
2325 2.18457558511798e-08
2326 2.18719893513297e-08
2327 2.18777837901474e-08
2328 2.1908775926005e-08
2329 2.17160988129628e-08
2330 2.17369138442169e-08
2331 2.19647532306855e-08
2332 2.18575291253131e-08
2333 2.17752706301866e-08
2334 2.18372394368282e-08
2335 2.18380493577364e-08
2336 2.17112185945112e-08
2337 2.17760684323443e-08
2338 2.17429765194677e-08
2339 2.16896382828624e-08
2340 2.19030442323298e-08
2341 2.15904310808535e-08
2342 2.18198596606545e-08
2343 2.17004781549734e-08
2344 2.17119146529443e-08
2345 2.17094735408541e-08
2346 2.1750609382587e-08
2347 2.17503647036432e-08
2348 2.17334963216187e-08
2349 2.17648388861624e-08
2350 2.17578098604676e-08
2351 2.17628550595039e-08
2352 2.15584470794061e-08
2353 2.16992440053021e-08
2354 2.18102226267058e-08
2355 2.1568104654146e-08
2356 2.17327649782995e-08
2357 2.16823105483321e-08
2358 2.17765418977223e-08
2359 2.1594379864931e-08
2360 2.17066355270301e-08
2361 2.15888153776156e-08
2362 2.17227205208115e-08
2363 2.16274665589822e-08
2364 2.15568817598655e-08
2365 2.16668312137402e-08
2366 2.15328684594063e-08
2367 2.17617507296541e-08
2368 2.15640124966443e-08
2369 2.15852008691453e-08
2370 2.16984697395395e-08
2371 2.15511333276908e-08
2372 2.15938842115238e-08
2373 2.15434733900066e-08
2374 2.14856657111406e-08
2375 2.16481922662437e-08
2376 2.14929453122847e-08
2377 2.1583274176562e-08
2378 2.15610247245035e-08
2379 2.14410599799209e-08
2380 2.15857514445084e-08
2381 2.14557698835494e-08
2382 2.15226624988851e-08
2383 2.14155501269131e-08
2384 2.15662096854086e-08
2385 2.14940000412556e-08
2386 2.14336318262731e-08
2387 2.18201444235344e-08
2388 2.13078472319772e-08
2389 2.15467476926134e-08
2390 2.14310243102922e-08
2391 2.14064769454847e-08
2392 2.14540175750289e-08
2393 2.13881680051209e-08
2394 2.12969128836615e-08
2395 2.15069165963033e-08
2396 2.1455959494765e-08
2397 2.12736919497081e-08
2398 2.15070517419758e-08
2399 2.12898874047962e-08
2400 2.13932145649576e-08
2401 2.15214739334169e-08
2402 2.12945741130399e-08
2403 2.15236717565626e-08
2404 2.12099273542155e-08
2405 2.13775115931458e-08
2406 2.13143054560438e-08
2407 2.14387150372364e-08
2408 2.11560462980431e-08
2409 2.15269126426776e-08
2410 2.11205199376341e-08
2411 2.14131032771903e-08
2412 2.13409155878086e-08
2413 2.13251863508956e-08
2414 2.13161267874629e-08
2415 2.13196255788306e-08
2416 2.12608759135868e-08
2417 2.1248534280649e-08
2418 2.12386319933255e-08
2419 2.12860923145541e-08
2420 2.13198037598561e-08
2421 2.12190338210183e-08
2422 2.12402030588255e-08
2423 2.13200642112943e-08
2424 2.12193047794962e-08
2425 2.12726106276762e-08
2426 2.11245851255981e-08
2427 2.08213889449915e-08
2428 2.10125790082172e-08
2429 2.09068196126427e-08
2430 2.08060987745284e-08
2431 2.08617430204616e-08
2432 2.08633993686869e-08
2433 2.08897001320318e-08
2434 2.0886002325915e-08
2435 2.08901999136923e-08
2436 2.08943539464768e-08
2437 2.08667410724339e-08
2438 2.08638879813927e-08
2439 2.09310184637879e-08
2440 2.07738088892073e-08
2441 2.07857277731982e-08
2442 2.08434749660036e-08
2443 2.07024549834234e-08
2444 2.09494315919212e-08
2445 2.0658083405567e-08
2446 2.07118047295118e-08
2447 2.07069065749077e-08
2448 2.07205709562563e-08
2449 2.08657860757411e-08
2450 2.04368894516915e-08
2451 2.04970988896824e-08
2452 2.05081843344668e-08
2453 2.06433286558116e-08
2454 2.10338327398252e-08
2455 2.05904643522103e-08
2456 2.03482261527665e-08
2457 2.0668495122389e-08
2458 2.02742229106212e-08
2459 2.04981588924347e-08
2460 2.05222714955711e-08
2461 2.06505445914384e-08
2462 2.01999604653169e-08
2463 2.048607104288e-08
2464 2.04020349079959e-08
2465 2.04765038726573e-08
2466 2.04724728595784e-08
2467 2.10407339744645e-08
2468 2.02918204808555e-08
2469 2.03004520183403e-08
2470 2.04443101700646e-08
2471 2.03979907440477e-08
2472 2.05285332216532e-08
2473 2.01321762259843e-08
2474 2.03761048480633e-08
2475 2.05209973483389e-08
2476 2.02069354951484e-08
2477 2.0323249246168e-08
2478 2.02955936642324e-08
2479 2.03025064251516e-08
2480 2.04662344233109e-08
2481 2.03506355540561e-08
2482 2.02061467137749e-08
2483 2.02635078647151e-08
2484 2.02612237011346e-08
2485 2.02381248013972e-08
2486 2.02876118473028e-08
2487 2.03864892605043e-08
2488 2.0000793442998e-08
2489 2.02690104679215e-08
2490 2.02344825934175e-08
2491 2.07745760114619e-08
2492 2.00638134529307e-08
2493 2.01165921711044e-08
2494 2.02262514313323e-08
2495 2.01433277421947e-08
2496 2.02361825112352e-08
2497 2.01918909535914e-08
2498 2.02222592362578e-08
2499 2.0131004426327e-08
2500 2.01208568356148e-08
2501 2.015874543565e-08
2502 2.09276818575743e-08
2503 1.98055010500187e-08
2504 2.00667499650753e-08
2505 2.01432601588669e-08
2506 2.01592163507347e-08
2507 2.01154172370654e-08
2508 1.99974550702531e-08
2509 2.00929141001005e-08
2510 2.01315888740949e-08
2511 2.01144796066477e-08
2512 1.99053385298575e-08
2513 2.0186700604663e-08
2514 1.99816089170302e-08
2515 2.00627191386804e-08
2516 2.04157448665532e-08
2517 1.99533496608728e-08
2518 2.00314330829898e-08
2519 1.9998506305352e-08
2520 1.99755867508755e-08
2521 2.01432164509407e-08
2522 1.99784147610904e-08
2523 1.98166220216622e-08
2524 1.99987199640539e-08
2525 2.0005056134087e-08
2526 1.99702761436438e-08
2527 1.99656008612603e-08
2528 1.9922454471466e-08
2529 1.98958528292437e-08
2530 1.99863933469935e-08
2531 1.99466429491313e-08
2532 1.99083470343586e-08
2533 1.98685982142344e-08
2534 2.03920536456792e-08
2535 1.97599218072497e-08
2536 1.98995111202471e-08
2537 1.98793323077373e-08
2538 1.99216545772041e-08
2539 1.98001949032189e-08
2540 1.99208106312909e-08
2541 1.99106580665576e-08
2542 1.99451391472705e-08
2543 1.98567684187578e-08
2544 1.99608882358593e-08
2545 1.99315106195197e-08
2546 1.98616408514374e-08
2547 1.98597469199813e-08
2548 1.98311341675916e-08
2549 1.98476825487015e-08
2550 1.98933273833912e-08
2551 1.97306953411136e-08
2552 1.98655209275866e-08
2553 1.97758530271486e-08
2554 1.97554123497756e-08
2555 1.97998601619842e-08
2556 2.01607895131684e-08
2557 1.97439085110052e-08
2558 1.97742979906046e-08
2559 1.96311223340517e-08
2560 1.98198448234654e-08
2561 1.96732126074295e-08
2562 1.9716827719396e-08
2563 1.9763275665996e-08
2564 1.96805522800769e-08
2565 1.95860077125687e-08
2566 1.96956596676401e-08
2567 1.97000534551961e-08
2568 1.97499770665521e-08
2569 1.96660286408967e-08
2570 1.98068518484362e-08
2571 1.95804275517908e-08
2572 1.96581907678794e-08
2573 1.97544446006725e-08
2574 1.96194642204661e-08
2575 1.95641324848284e-08
2576 1.96216113241832e-08
2577 1.95727113809774e-08
2578 1.96881927513326e-08
2579 1.95461651446571e-08
2580 1.96055612615709e-08
2581 1.96243576615629e-08
2582 1.96974470767497e-08
2583 1.96390034544747e-08
2584 1.96113265278353e-08
2585 1.96925996764907e-08
2586 1.95939152466407e-08
2587 1.94089608722692e-08
2588 1.96351762170588e-08
2589 1.9605203660622e-08
2590 1.95498016918316e-08
2591 1.95793073565209e-08
2592 1.95811695923309e-08
2593 1.94903096499832e-08
2594 1.96903225319067e-08
2595 1.93658244262984e-08
2596 1.96436118603538e-08
2597 1.94279856071677e-08
2598 1.95597185104202e-08
2599 1.95120444491659e-08
2600 1.9444577483041e-08
2601 1.94074010555489e-08
2602 1.93228464195894e-08
2603 1.95203017379075e-08
2604 1.93581371186191e-08
2605 1.92119606029495e-08
2606 1.92112832309022e-08
2607 1.92342130685974e-08
2608 1.9179541520864e-08
2609 1.92066710292815e-08
2610 1.95132867551462e-08
2611 1.90550229544062e-08
2612 1.92515144921135e-08
2613 1.93342899350579e-08
2614 1.91344268313243e-08
2615 1.94778805695517e-08
2616 1.93107307495666e-08
2617 1.90801379892358e-08
2618 1.9159296718696e-08
2619 1.91334044729041e-08
2620 1.92235665138485e-08
2621 1.90391230168174e-08
2622 1.93305260665078e-08
2623 1.8987301893203e-08
2624 1.93365126492928e-08
2625 1.9101150384504e-08
2626 1.91745892048623e-08
2627 1.90752402314809e-08
2628 1.90587135376896e-08
2629 1.9111347384082e-08
2630 1.93978446301202e-08
2631 1.90568116965606e-08
2632 1.91017196699006e-08
2633 1.8941842229947e-08
2634 1.90662043114687e-08
2635 1.90625579007397e-08
2636 1.90848350996031e-08
2637 1.91125639051881e-08
2638 1.91290660588317e-08
2639 1.90673967392918e-08
2640 1.8902135291099e-08
2641 1.9000519127732e-08
2642 1.88966359317977e-08
2643 1.90761747590473e-08
2644 1.88374685995418e-08
2645 1.90245563536084e-08
2646 1.90149415086882e-08
2647 1.89111372507211e-08
2648 1.88575744701769e-08
2649 1.88967629732861e-08
2650 1.90014933435512e-08
2651 1.92713520216214e-08
2652 1.92685311475316e-08
2653 1.90093085976617e-08
2654 1.87174602885865e-08
2655 1.89794315268443e-08
2656 1.89407440449596e-08
2657 1.89513028719945e-08
2658 1.88499722735269e-08
2659 1.87083594359594e-08
2660 1.89886163461939e-08
2661 1.87491500633907e-08
2662 1.89829291742383e-08
2663 1.86960806317638e-08
2664 1.90566075113363e-08
2665 1.88328128573056e-08
2666 1.88186299773241e-08
2667 1.89927087996811e-08
2668 1.88031749855933e-08
2669 1.86765826253321e-08
2670 1.88647376199169e-08
2671 1.88150159614597e-08
2672 1.87271429027414e-08
2673 1.87247019819425e-08
2674 1.92926374565916e-08
2675 1.89138437315384e-08
2676 1.88190053715953e-08
2677 1.86443719683016e-08
2678 1.88437251404938e-08
2679 1.86421772365719e-08
2680 1.90035773434127e-08
2681 1.85021778668082e-08
2682 1.88883846048427e-08
2683 1.86650477597139e-08
2684 1.88100361649957e-08
2685 1.87668288463838e-08
2686 1.87371392516011e-08
2687 1.88476721025843e-08
2688 1.86113990243708e-08
2689 1.90068123980813e-08
2690 1.84176347393095e-08
2691 1.90271859858981e-08
2692 1.84399001423863e-08
2693 1.86589767274814e-08
2694 1.88431751670937e-08
2695 1.85811692718918e-08
2696 1.89959965838371e-08
2697 1.83170945250444e-08
2698 1.86496275026427e-08
2699 1.91326875240705e-08
2700 1.83727058699068e-08
2701 1.87052181902514e-08
2702 1.8526081952297e-08
2703 1.86106225882332e-08
2704 1.86268383116639e-08
2705 1.86307527052509e-08
2706 1.84957049531587e-08
2707 1.84501667328529e-08
2708 1.84317739430062e-08
2709 1.8646612854778e-08
2710 1.85459034494073e-08
2711 1.83818092501786e-08
2712 1.84005966661349e-08
2713 1.86225625095071e-08
2714 1.84359875826656e-08
2715 1.84010482661146e-08
2716 1.86161009551045e-08
2717 1.82597379478899e-08
2718 1.83733403292807e-08
2719 1.850746145593e-08
2720 1.84361553552392e-08
2721 1.8532007549199e-08
2722 1.82784722926321e-08
2723 1.82400859058962e-08
2724 1.86413083076431e-08
2725 1.82160578972579e-08
2726 1.84021301636861e-08
2727 1.82328495917927e-08
2728 1.83317809990813e-08
2729 1.82839354228026e-08
2730 1.83446202233206e-08
2731 1.82986089201087e-08
2732 1.83244505231883e-08
2733 1.81600955451722e-08
2734 1.82160819484634e-08
2735 1.84010300757764e-08
2736 1.82329727606012e-08
2737 1.83156968858178e-08
2738 1.81651957198437e-08
2739 1.8249570946427e-08
2740 1.81667892957815e-08
2741 1.81463594107756e-08
2742 1.82227891579734e-08
2743 1.81516332875553e-08
2744 1.83376009890157e-08
2745 1.81446385987272e-08
2746 1.84189249421429e-08
2747 1.83623104813568e-08
2748 1.83601002842204e-08
2749 1.84842321435674e-08
2750 1.82704322557736e-08
2751 1.8084406076091e-08
2752 1.80577350704647e-08
2753 1.83418514294598e-08
2754 1.82299984134993e-08
2755 1.81770460748742e-08
2756 1.8014561847246e-08
2757 1.84240521369228e-08
2758 1.81170466777569e-08
2759 1.80629083026407e-08
2760 1.81485708332652e-08
2761 1.81530043238398e-08
2762 1.8049079867466e-08
2763 1.80000090634902e-08
2764 1.80754437415498e-08
2765 1.8128276787821e-08
2766 1.8107759722108e-08
2767 1.81409321824422e-08
2768 1.80237405831285e-08
2769 1.80512592480309e-08
2770 1.80627263306476e-08
2771 1.83016720060047e-08
2772 1.8144350753091e-08
2773 1.79111150535594e-08
2774 1.81605635022342e-08
2775 1.7936384890227e-08
2776 1.80579742969922e-08
2777 1.82815103143552e-08
2778 1.77961309513197e-08
2779 1.80804229721332e-08
2780 1.81446106483074e-08
2781 1.78475874456074e-08
2782 1.8117325592093e-08
2783 1.79019751045395e-08
2784 1.78357111416672e-08
2785 1.81113978117153e-08
2786 1.78221597046946e-08
2787 1.80310579577236e-08
2788 1.81295730760089e-08
2789 1.78622311184373e-08
2790 1.79028475687515e-08
2791 1.78960524654936e-08
2792 1.82828313251004e-08
2793 1.77441212674267e-08
2794 1.80880363417346e-08
2795 1.78559345157536e-08
2796 1.80699891512148e-08
2797 1.7814480922973e-08
2798 1.78832119719363e-08
2799 1.77678321375829e-08
2800 1.79514929211644e-08
2801 1.8011184986122e-08
2802 1.79266100226449e-08
2803 1.77886618974998e-08
2804 1.78481825282573e-08
2805 1.80516093650773e-08
2806 1.77060778346894e-08
2807 1.82190018609418e-08
2808 1.78394931747983e-08
2809 1.76855940975384e-08
2810 1.79047069103211e-08
2811 1.78421883924029e-08
2812 1.7934560736288e-08
2813 1.77306221801343e-08
2814 1.80753005323275e-08
2815 1.76952609607373e-08
2816 1.77591152354939e-08
2817 1.77267610339538e-08
2818 1.79868533524141e-08
2819 1.77111225032611e-08
2820 1.78432362831771e-08
2821 1.79086208194068e-08
2822 1.77865482534623e-08
2823 1.76658095425442e-08
2824 1.77725616317126e-08
2825 1.76546143676459e-08
2826 1.80294554494864e-08
2827 1.76389903538965e-08
2828 1.76691907979309e-08
2829 1.77404663631586e-08
2830 1.78027505217671e-08
2831 1.76839150256569e-08
2832 1.78978269337327e-08
2833 1.75781270593811e-08
2834 1.78947621198855e-08
2835 1.76289153378528e-08
2836 1.78755664643759e-08
2837 1.77063691425561e-08
2838 1.76333265924367e-08
2839 1.77785917950812e-08
2840 1.76725562038849e-08
2841 1.76831539422384e-08
2842 1.77803488656814e-08
2843 1.76153107159793e-08
2844 1.77657705058287e-08
2845 1.77220362111985e-08
2846 1.76131512242828e-08
2847 1.7782364686747e-08
2848 1.776473123416e-08
2849 1.76572156437294e-08
2850 1.76240587352883e-08
2851 1.77275149783052e-08
2852 1.75553081460666e-08
2853 1.76531473762287e-08
2854 1.76769245912034e-08
2855 1.77190630890056e-08
2856 1.7685791523836e-08
2857 1.75977734964139e-08
2858 1.76272406411382e-08
2859 1.7657421259587e-08
2860 1.76254330701564e-08
2861 1.76014212794895e-08
2862 1.75293701150103e-08
2863 1.76752618794573e-08
2864 1.76152989567191e-08
2865 1.77318253601344e-08
2866 1.75251317706682e-08
2867 1.76691266254858e-08
2868 1.75327348086451e-08
2869 1.77345437564203e-08
2870 1.75575699917774e-08
2871 1.74941877817414e-08
2872 1.75421613711757e-08
2873 1.75651027369828e-08
2874 1.76619395604316e-08
2875 1.75147636195527e-08
2876 1.75903304354685e-08
2877 1.75671024724711e-08
2878 1.74513162228829e-08
2879 1.75873247301173e-08
2880 1.75651778848707e-08
2881 1.75556155850298e-08
2882 1.75664862556157e-08
2883 1.76001152580874e-08
2884 1.74666724491734e-08
2885 1.75105849749002e-08
2886 1.74983004027629e-08
2887 1.75867821678999e-08
2888 1.74740876618396e-08
2889 1.76672877466588e-08
2890 1.74669063530697e-08
2891 1.75906089356914e-08
2892 1.74696096906235e-08
2893 1.75187282066558e-08
2894 1.75855159636473e-08
2895 1.74778197531422e-08
2896 1.748425181336e-08
2897 1.7470943366793e-08
2898 1.75451077903954e-08
2899 1.75702201833383e-08
2900 1.74250058152525e-08
2901 1.74686581289052e-08
2902 1.74521954070528e-08
2903 1.73430729207835e-08
2904 1.7533466608155e-08
2905 1.75484603992526e-08
2906 1.73883311688217e-08
2907 1.74219625652672e-08
2908 1.75328340065173e-08
2909 1.74860246928699e-08
2910 1.75075241587996e-08
2911 1.74214344327295e-08
2912 1.73739873058354e-08
2913 1.75549594396696e-08
2914 1.74664488654663e-08
2915 1.76452066223964e-08
2916 1.75301779637982e-08
2917 1.74598234331524e-08
2918 1.73673628396376e-08
2919 1.75810394301701e-08
2920 1.73602462952838e-08
2921 1.75997955536022e-08
2922 1.73124308858741e-08
2923 1.73808899343042e-08
2924 1.73899956067425e-08
2925 1.7606345891652e-08
2926 1.73487861535038e-08
2927 1.73942090809787e-08
2928 1.73651458404978e-08
2929 1.75237840619458e-08
2930 1.74592701169818e-08
2931 1.74113011914834e-08
2932 1.73865164767584e-08
2933 1.73948245151268e-08
2934 1.74104766758609e-08
2935 1.73074007224772e-08
2936 1.74017040235341e-08
2937 1.73975599895293e-08
2938 1.72487392737786e-08
2939 1.74409925561836e-08
2940 1.73319370251335e-08
2941 1.74296224495496e-08
2942 1.73106968545111e-08
2943 1.72749929694405e-08
2944 1.72979128663098e-08
2945 1.74171406175372e-08
2946 1.72489845940982e-08
2947 1.72971911216457e-08
2948 1.72813292668161e-08
2949 1.72499245812974e-08
2950 1.74165435875562e-08
2951 1.73128184225435e-08
2952 1.70245228784394e-08
2953 1.72323389987516e-08
2954 1.73760227867303e-08
2955 1.72511285236876e-08
2956 1.73112391271268e-08
2957 1.71397158506004e-08
2958 1.72228345997061e-08
2959 1.72220616365681e-08
2960 1.72001457963145e-08
2961 1.72133055428292e-08
2962 1.7270889961174e-08
2963 1.7083939140683e-08
2964 1.73349676869661e-08
2965 1.71614185895397e-08
2966 1.72865383560872e-08
2967 1.71327441302327e-08
2968 1.71904239251175e-08
2969 1.71967320993227e-08
2970 1.71957995791505e-08
2971 1.71502256436362e-08
2972 1.70434830579769e-08
2973 1.72140221780248e-08
2974 1.70802655773405e-08
2975 1.71857398250097e-08
2976 1.70318341334852e-08
2977 1.72169389216492e-08
2978 1.73121866010595e-08
2979 1.71269081430392e-08
2980 1.71141348728199e-08
2981 1.71408396920647e-08
2982 1.71372246002832e-08
2983 1.70663183900466e-08
2984 1.70399162434665e-08
2985 1.70968779011904e-08
2986 1.70231995453296e-08
2987 1.70966050109245e-08
2988 1.7086017507828e-08
2989 1.70501820813129e-08
2990 1.72464146303009e-08
2991 1.69551769919618e-08
2992 1.71957732552075e-08
2993 1.70630725190302e-08
2994 1.69144160540036e-08
2995 1.71051172273007e-08
2996 1.70577676456007e-08
2997 1.70264282207677e-08
2998 1.70097365752575e-08
2999 1.7079408357934e-08
3000 8.85918727927884e-09
3001 8.89458767358575e-09
3002 8.97866005974968e-09
3003 9.01541722428401e-09
3004 9.02799919174496e-09
3005 9.03136684778788e-09
3006 9.03151724768181e-09
3007 9.03083649935604e-09
3008 9.02927514391527e-09
3009 9.02852586297936e-09
3010 9.02730972449084e-09
3011 9.02626305989557e-09
3012 9.02549693373306e-09
3013 9.02436839431836e-09
3014 9.02356643860819e-09
3015 9.02256233333704e-09
3016 9.02172557580572e-09
3017 9.02114874649918e-09
3018 9.01928419044795e-09
3019 9.01936471090631e-09
3020 9.01849573883839e-09
3021 9.01747078171922e-09
3022 9.01681393967318e-09
3023 9.01632573570288e-09
3024 9.01596996510934e-09
3025 9.01468794196891e-09
3026 9.01410956567067e-09
3027 9.01341475469802e-09
3028 9.0124008858411e-09
3029 9.0117221283742e-09
3030 9.01074387794215e-09
3031 9.01017569956281e-09
3032 9.00932615613415e-09
3033 9.00882139146658e-09
3034 9.00833992444061e-09
3035 9.0077916707898e-09
3036 9.00702531740627e-09
3037 9.0062380040043e-09
3038 9.00592120700594e-09
3039 9.00534763472599e-09
3040 9.0049297144304e-09
3041 9.00418733898456e-09
3042 9.00390723412858e-09
3043 9.00304279575354e-09
3044 9.00274450810895e-09
3045 9.00246813612116e-09
3046 9.00225260169057e-09
3047 9.00185837266615e-09
3048 9.00098139092198e-09
3049 9.00038797152503e-09
3050 9.00002988382664e-09
3051 8.99926217507591e-09
3052 8.99898780300223e-09
3053 8.99796984430573e-09
3054 8.99794877173149e-09
3055 8.9974642113605e-09
3056 8.99672631347187e-09
3057 8.99639672033436e-09
3058 8.99562847157115e-09
3059 8.9949504955833e-09
3060 8.99473030236547e-09
3061 8.99408401928287e-09
3062 8.99389807629269e-09
3063 8.99323348658998e-09
3064 8.99272760669645e-09
3065 8.99239685170361e-09
3066 8.99177886939245e-09
3067 8.99151923274694e-09
3068 8.99101682718534e-09
3069 8.99030636017373e-09
3070 8.98997175661359e-09
3071 8.98950167495249e-09
3072 8.98909936884046e-09
3073 8.98851320879635e-09
3074 8.98813663870873e-09
3075 8.98771908393325e-09
3076 8.98707372389701e-09
3077 8.98671158583442e-09
3078 8.98649960334036e-09
3079 8.98602807098553e-09
3080 8.98514102861542e-09
3081 8.98481271749407e-09
3082 8.98454535693677e-09
3083 8.98420171003417e-09
3084 8.98365905920018e-09
3085 8.98332493148163e-09
3086 8.98292037315729e-09
3087 8.98233931804354e-09
3088 8.98168439267266e-09
3089 8.98148032784513e-09
3090 8.9811818140273e-09
3091 8.98058656024353e-09
3092 8.97995302647642e-09
3093 8.97943749548352e-09
3094 8.97951172623007e-09
3095 8.97914091139984e-09
3096 8.97855954171828e-09
3097 8.97823424401978e-09
3098 8.97790764952527e-09
3099 8.97730552202464e-09
3100 8.9767326530224e-09
3101 8.97641855469172e-09
3102 8.97652064701937e-09
3103 8.97637114921634e-09
3104 8.97574565524029e-09
3105 8.97515294820406e-09
3106 8.97462614741357e-09
3107 8.97421229332368e-09
3108 8.97362326199996e-09
3109 8.97368054263853e-09
3110 8.97329766540167e-09
3111 8.97266676320313e-09
3112 8.97233288674887e-09
3113 8.9718073026171e-09
3114 8.97144578355935e-09
3115 8.97138022838001e-09
3116 8.97065848471518e-09
3117 8.97027619439078e-09
3118 8.9701831620867e-09
3119 8.96946811591165e-09
3120 8.96915809511789e-09
3121 8.96901080122553e-09
3122 8.9683074930777e-09
3123 8.96798096867296e-09
3124 8.96787568573781e-09
3125 8.96714850297242e-09
3126 8.9668207580787e-09
3127 8.96671982770353e-09
3128 8.96601091915361e-09
3129 8.96567703555229e-09
3130 8.96564295202462e-09
3131 8.96463223658966e-09
3132 8.96428761795048e-09
3133 8.96427281663753e-09
3134 8.96361795859474e-09
3135 8.96317355329746e-09
3136 8.96295025186455e-09
3137 8.96233328568502e-09
3138 8.96205027040714e-09
3139 8.96169065468799e-09
3140 8.9615493507611e-09
3141 8.96098999752376e-09
3142 8.96075691566439e-09
3143 8.96017034047319e-09
3144 8.95981092997183e-09
3145 8.95972428503061e-09
3146 8.9593556613296e-09
3147 8.95875439618776e-09
3148 8.95851923936319e-09
3149 8.95833833648024e-09
3150 8.95786938431148e-09
3151 8.95737787003775e-09
3152 8.95718184210081e-09
3153 8.95683465799296e-09
3154 8.95588087906213e-09
3155 8.95573829948898e-09
3156 8.95547505802563e-09
3157 8.95480059717041e-09
3158 8.95435833300068e-09
3159 8.95449249214281e-09
3160 8.95380795416789e-09
3161 8.95359756164504e-09
3162 8.95325176500372e-09
3163 8.95274116281125e-09
3164 8.95252905654814e-09
3165 8.95216634791418e-09
3166 8.95163733700272e-09
3167 8.9513850957193e-09
3168 8.9508030999344e-09
3169 8.95055699220299e-09
3170 8.95024842877817e-09
3171 8.94966089935029e-09
3172 8.94975535999098e-09
3173 8.9490998154626e-09
3174 8.94881626441885e-09
3175 8.94869453272468e-09
3176 8.94802348636037e-09
3177 8.94772929203175e-09
3178 8.94726550341313e-09
3179 8.94715715842043e-09
3180 8.94660912933998e-09
3181 8.94615680448879e-09
3182 8.94610075071606e-09
3183 8.94546421163073e-09
3184 8.94529055781024e-09
3185 8.94522497411898e-09
3186 8.94430194258583e-09
3187 8.94427830538946e-09
3188 8.94375885805016e-09
3189 8.9437456822633e-09
3190 8.94288023303019e-09
3191 8.94244641338332e-09
3192 8.94196012448528e-09
3193 8.94167885619884e-09
3194 8.94117294541336e-09
3195 8.94101576653233e-09
3196 8.94039495352022e-09
3197 8.94046706608614e-09
3198 8.93982682664135e-09
3199 8.93952649933666e-09
3200 8.93922946590409e-09
3201 8.93885492141144e-09
3202 8.93842333708761e-09
3203 8.93809272970586e-09
3204 8.9375758416041e-09
3205 8.93738017607171e-09
3206 8.93696349377193e-09
3207 8.93662611459178e-09
3208 8.93639444204952e-09
3209 8.93579639912878e-09
3210 8.93570449698317e-09
3211 8.93510622713284e-09
3212 8.93486772473445e-09
3213 8.93443215677081e-09
3214 8.93434640930613e-09
3215 8.93359756474338e-09
3216 8.93358887493062e-09
3217 8.93315827102392e-09
3218 8.93280833819121e-09
3219 8.93252669177669e-09
3220 8.93206954100806e-09
3221 8.93180786923353e-09
3222 8.9314844016683e-09
3223 8.93107700253998e-09
3224 8.93078313515816e-09
3225 8.93048377494793e-09
3226 8.93008855850502e-09
3227 8.92983930488728e-09
3228 8.92941207881331e-09
3229 8.92912709178628e-09
3230 8.92871603945522e-09
3231 8.92844601987908e-09
3232 8.92788539422207e-09
3233 8.92786709075222e-09
3234 8.92749270785947e-09
3235 8.92709928936031e-09
3236 8.92344753539165e-09
3237 8.92323808809187e-09
3238 8.92280661889122e-09
3239 8.92253261720183e-09
3240 8.9223657248616e-09
3241 8.92190451256408e-09
3242 8.92158012082839e-09
3243 8.921328779464e-09
3244 8.92092271213146e-09
3245 8.92060619970408e-09
3246 8.92028616907015e-09
3247 8.9198195529841e-09
3248 8.91957557584005e-09
3249 8.91934766893143e-09
3250 8.91882771261732e-09
3251 8.91865576104628e-09
3252 8.9184012120741e-09
3253 8.91779659470404e-09
3254 8.91768529737852e-09
3255 8.91715668196319e-09
3256 8.9168953246524e-09
3257 8.91886691271188e-09
3258 8.9159067082345e-09
3259 8.91586446250675e-09
3260 8.91547596420378e-09
3261 8.9152106614171e-09
3262 8.91701230266301e-09
3263 8.91437178870969e-09
3264 8.91424239667093e-09
3265 8.91610977390295e-09
3266 8.91332154663582e-09
3267 8.91327174483425e-09
3268 8.91515918727964e-09
3269 8.91230268289939e-09
3270 8.91215400370543e-09
3271 8.91407351302775e-09
3272 8.9113142201433e-09
3273 8.91126589497998e-09
3274 8.91300796045103e-09
3275 8.910619122518e-09
3276 8.91035113562838e-09
3277 8.91025654813776e-09
3278 8.91224439551752e-09
3279 8.9093387993025e-09
3280 8.90909525090577e-09
3281 8.90871025199108e-09
3282 8.91089582442939e-09
3283 8.90791331076973e-09
3284 8.90771017250658e-09
3285 8.90756975398255e-09
3286 8.90703433270579e-09
3287 8.9091812106798e-09
3288 8.90638051332521e-09
3289 8.90606571109925e-09
3290 8.90579516307083e-09
3291 8.90552272387463e-09
3292 8.90797498524776e-09
3293 8.90471236982687e-09
3294 8.90450542019583e-09
3295 8.90642658933627e-09
3296 8.90325684334975e-09
3297 8.90340077761009e-09
3298 8.90499262581196e-09
3299 8.90316085275233e-09
3300 8.90216801197746e-09
3301 8.90474748097003e-09
3302 8.90171221181596e-09
3303 8.90140638516029e-09
3304 8.90127131138174e-09
3305 8.90374341640243e-09
3306 8.8998614723515e-09
3307 8.90057768131863e-09
3308 8.89976124252706e-09
3309 8.90229361958511e-09
3310 8.89884850115152e-09
3311 8.89908570201464e-09
3312 8.89877745273437e-09
3313 8.90078147807588e-09
3314 8.89740241245135e-09
3315 8.89783255991067e-09
3316 8.89767328211832e-09
3317 8.89905693073068e-09
3318 8.89656794275662e-09
3319 8.89572123183996e-09
3320 8.8988283959493e-09
3321 8.89567921825984e-09
3322 8.89557767465993e-09
3323 8.8955194040008e-09
3324 8.89464365051024e-09
3325 8.89728290056485e-09
3326 8.89357880874686e-09
3327 8.89395562160555e-09
3328 8.89589752952474e-09
3329 8.89257177556085e-09
3330 8.89285960957442e-09
3331 8.8920330267514e-09
3332 8.89416806845733e-09
3333 8.8920940272616e-09
3334 8.89152384105779e-09
3335 8.89355690142368e-09
3336 8.8899002250592e-09
3337 8.89094180042682e-09
3338 8.89260681230519e-09
3339 8.88911570291057e-09
3340 8.89001654707372e-09
3341 8.89163484212724e-09
3342 8.88877061002574e-09
3343 8.88791679033024e-09
3344 8.88881876836417e-09
3345 8.88974873984594e-09
3346 8.88750931701127e-09
3347 8.88745925026191e-09
3348 8.88924476299263e-09
3349 8.88660438857353e-09
3350 8.88605482605304e-09
3351 8.88865029873009e-09
3352 8.88556697592469e-09
3353 8.88566195943186e-09
3354 8.88671168294136e-09
3355 8.88474332260553e-09
3356 8.8847381455523e-09
3357 8.88641906863358e-09
3358 8.88343032717787e-09
3359 8.88364916636075e-09
3360 8.88493515992239e-09
3361 8.88322543728809e-09
3362 8.88243475490741e-09
3363 8.88421516751642e-09
3364 8.88219189570244e-09
3365 8.88204397793413e-09
3366 8.88349959214663e-09
3367 8.87950860330167e-09
3368 8.881579568569e-09
3369 8.88283862435918e-09
3370 8.87950416147282e-09
3371 8.88033004096955e-09
3372 8.87905858464738e-09
3373 8.88170540092742e-09
3374 8.87847967707273e-09
3375 8.87852167377745e-09
3376 8.88005124508734e-09
3377 8.87754264233126e-09
3378 8.87772857455227e-09
3379 8.87938488314949e-09
3380 8.87674417051482e-09
3381 8.87660924905886e-09
3382 8.87684584228304e-09
3383 8.87687339348742e-09
3384 8.87585895262971e-09
3385 8.87527649209158e-09
3386 8.87670205546726e-09
3387 8.8750379281452e-09
3388 8.87466333288761e-09
3389 8.87426942798586e-09
3390 8.87525422305774e-09
3391 8.87371313748608e-09
3392 8.873270283552e-09
3393 8.87501732677043e-09
3394 8.87155224813668e-09
3395 8.87223670971438e-09
3396 8.873948259977e-09
3397 8.87138456802383e-09
3398 8.87105786129966e-09
3399 8.87063657769443e-09
3400 8.87266756925686e-09
3401 8.87022400296311e-09
3402 8.86966000025052e-09
3403 8.87162024267724e-09
3404 8.86930018078463e-09
3405 8.86950437192779e-09
3406 8.86897664714031e-09
3407 8.86934979715814e-09
3408 8.8676145930458e-09
3409 8.86960443244322e-09
3410 8.86701518153232e-09
3411 8.86726813925653e-09
3412 8.86917101161622e-09
3413 8.86630584624043e-09
3414 8.86609304747571e-09
3415 8.86783982016054e-09
3416 8.86522728990591e-09
3417 8.86472515217174e-09
3418 8.86731137090396e-09
3419 8.86415187208167e-09
3420 8.86422673857662e-09
3421 8.86618235581421e-09
3422 8.86233641598372e-09
3423 8.86557260658394e-09
3424 8.86231021270806e-09
3425 8.86289798853607e-09
3426 8.86447118062655e-09
3427 8.86098561248688e-09
3428 8.86219131160382e-09
3429 8.86302779415371e-09
3430 8.86113951862472e-09
3431 8.86016216494062e-09
3432 8.86104768394597e-09
3433 8.86207975623471e-09
3434 8.85889132358486e-09
3435 8.86228124796079e-09
3436 8.85820145028054e-09
3437 8.85932949365453e-09
3438 8.86001417584742e-09
3439 8.85850701604074e-09
3440 8.85829731429172e-09
3441 8.85708106020122e-09
3442 8.85944958912283e-09
3443 8.85672826009265e-09
3444 8.85842888760646e-09
3445 8.85589963909444e-09
3446 8.85794838533655e-09
3447 8.85523759260737e-09
3448 8.85789200681664e-09
3449 8.85444646332029e-09
3450 8.85672526376724e-09
3451 8.85295096888367e-09
3452 8.85362838902448e-09
3453 8.85548516118412e-09
3454 8.85359464956881e-09
3455 8.85473234357032e-09
3456 8.85268974864378e-09
3457 8.85213014484299e-09
3458 8.85460563112422e-09
3459 8.85194448133758e-09
3460 8.85376555591344e-09
3461 8.85085480879094e-09
3462 8.85147743043963e-09
3463 8.8525005621648e-09
3464 8.85073684384757e-09
3465 8.85008336338339e-09
3466 8.85174267164363e-09
3467 8.84821324539237e-09
3468 8.84872432243417e-09
3469 8.85065713685656e-09
3470 8.84863022345556e-09
3471 8.84818699264239e-09
3472 8.84949568814158e-09
3473 8.84756886941618e-09
3474 8.84786178829283e-09
3475 8.84859961756274e-09
3476 8.84548812915553e-09
3477 8.84896988950085e-09
3478 8.84570948573443e-09
3479 8.84783954484269e-09
3480 8.84496368317583e-09
3481 8.84758204716674e-09
3482 8.84404331234073e-09
3483 8.8445487601424e-09
3484 8.84565666152415e-09
3485 8.84385127337961e-09
3486 8.84484207855363e-09
3487 8.84267412597922e-09
3488 8.84193009882683e-09
3489 8.84509545327067e-09
3490 8.84199930294843e-09
3491 8.84116259064482e-09
3492 8.84504383853735e-09
3493 8.84083178301354e-09
3494 8.84072470071912e-09
3495 8.83993881381495e-09
3496 8.84367557758836e-09
3497 8.83915180725087e-09
3498 8.83949695993508e-09
3499 8.83931679382388e-09
3500 8.84212061039863e-09
3501 8.83795074523713e-09
3502 8.83746114559852e-09
3503 8.84176147948612e-09
3504 8.83673159167431e-09
3505 8.83757463285978e-09
3506 8.83684527895612e-09
3507 8.83639345489878e-09
3508 8.83910235043028e-09
3509 8.8359808531821e-09
3510 8.83969470208412e-09
3511 8.83495898360703e-09
3512 8.838889389233e-09
3513 8.8337368182867e-09
3514 8.83871374345036e-09
3515 8.83376098894523e-09
3516 8.83351732239301e-09
3517 8.8375900481677e-09
3518 8.83303856480389e-09
3519 8.83265584342846e-09
3520 8.83222354040175e-09
3521 8.83267992722592e-09
3522 8.83447796804632e-09
3523 8.83161998057463e-09
3524 8.83155564297688e-09
3525 8.83329211939093e-09
3526 8.83087406915867e-09
3527 8.8336029901992e-09
3528 8.83030560280829e-09
3529 8.83359320546367e-09
3530 8.82874079582985e-09
3531 8.83279206322779e-09
3532 8.82800853845012e-09
3533 8.83017635811695e-09
3534 8.82728293617741e-09
3535 8.8321444208922e-09
3536 8.82651818611552e-09
3537 8.83124401274138e-09
3538 8.82663872896899e-09
3539 8.82667836844819e-09
3540 8.82642474111378e-09
3541 8.83127166685649e-09
3542 8.82602654401349e-09
3543 8.82905742560086e-09
3544 8.82634081208339e-09
3545 8.82544207874564e-09
3546 8.82879180334056e-09
3547 8.82478586800711e-09
3548 8.82436331429287e-09
3549 8.82802566836566e-09
3550 8.82384148407495e-09
3551 8.82688503619361e-09
3552 8.8264762481069e-09
3553 8.82341205479808e-09
3554 8.822387527474e-09
3555 8.82610897930125e-09
3556 8.82129560363792e-09
3557 8.82181865197867e-09
3558 8.82521638793238e-09
3559 8.82017090049481e-09
3560 8.82163820892234e-09
3561 8.82360514843639e-09
3562 8.81972205951653e-09
3563 8.82404107938744e-09
3564 8.81871319170086e-09
3565 8.8239299666365e-09
3566 8.8183156162075e-09
3567 8.82293816159563e-09
3568 8.81817001092966e-09
3569 8.8173006425607e-09
3570 8.82172065427844e-09
3571 8.81722840575389e-09
3572 8.82037754703496e-09
3573 8.8160907549123e-09
3574 8.82087495891748e-09
3575 8.81541020538584e-09
3576 8.81592390670344e-09
3577 8.81890690966614e-09
3578 8.81488656657992e-09
3579 8.81370778207241e-09
3580 8.81562240569872e-09
3581 8.81806470665047e-09
3582 8.81343018953318e-09
3583 8.81401936221604e-09
3584 8.81694155549506e-09
3585 8.81260042991106e-09
3586 8.81728490932654e-09
3587 8.81202477922422e-09
3588 8.81256574340045e-09
3589 8.81620848627862e-09
3590 8.81082341573636e-09
3591 8.81109859370011e-09
3592 8.81034887932619e-09
3593 8.81204867719115e-09
3594 8.81108737173231e-09
3595 8.8138456010925e-09
3596 8.80956818901518e-09
3597 8.8128305176377e-09
3598 8.81011954849409e-09
3599 8.80884971148771e-09
3600 8.81249798711375e-09
3601 8.80692478189005e-09
3602 8.81161683993076e-09
3603 8.80749514745038e-09
3604 8.80858425042486e-09
3605 8.80729801049779e-09
3606 8.8111752124323e-09
3607 8.80613861065083e-09
3608 8.81028516558358e-09
3609 8.80611837868889e-09
3610 8.80955796102312e-09
3611 8.80464134734443e-09
3612 8.80546162829177e-09
3613 8.80763556695974e-09
3614 8.80441107440483e-09
3615 8.80436897922332e-09
3616 8.80762178212535e-09
3617 8.80454206864112e-09
3618 8.80321946252621e-09
3619 8.80379438463613e-09
3620 8.80705125518716e-09
3621 8.80195798046157e-09
3622 8.80216672386608e-09
3623 8.80186300319941e-09
3624 8.8049643021601e-09
3625 8.80084252378316e-09
3626 8.80157081994959e-09
3627 8.80407044042547e-09
3628 8.80066197034984e-09
3629 8.79980256245083e-09
3630 8.80083021242983e-09
3631 8.79940886847064e-09
3632 8.80039846145458e-09
3633 8.79848078753909e-09
3634 8.80238547182333e-09
3635 8.79788473510251e-09
3636 8.79828346347228e-09
3637 8.79837531962691e-09
3638 8.79746278670962e-09
3639 8.79867755937402e-09
3640 8.79818441699293e-09
3641 8.79746604374315e-09
3642 8.79672103607648e-09
3643 8.79968807265125e-09
3644 8.79564017396833e-09
3645 8.79496068958985e-09
3646 8.79425468693074e-09
3647 8.79881980947461e-09
3648 8.79351534832651e-09
3649 8.79482088826655e-09
3650 8.79400820459619e-09
3651 8.79757839331724e-09
3652 8.79284511781653e-09
3653 8.79313186193803e-09
3654 8.79379574264538e-09
3655 8.79281960784739e-09
3656 8.79532041023895e-09
3657 8.79222238304306e-09
3658 8.79095888906695e-09
3659 8.79434726129558e-09
3660 8.79092318130381e-09
3661 8.79075135445245e-09
3662 8.79216933225013e-09
3663 8.79200145827152e-09
3664 8.79302477265614e-09
3665 8.78835580189752e-09
3666 8.79266341213131e-09
3667 8.79120414514628e-09
3668 8.78862141331926e-09
3669 8.79074434910476e-09
3670 8.78806435278962e-09
3671 8.78789238360767e-09
3672 8.79122840261731e-09
3673 8.78848844960284e-09
3674 8.78709965620311e-09
3675 8.78756924403701e-09
3676 8.78627506677165e-09
3677 8.78616720097897e-09
3678 8.79010079825326e-09
3679 8.78478217901379e-09
3680 8.78722800074644e-09
3681 8.78459860561426e-09
3682 8.78654380497002e-09
3683 8.78652079931708e-09
3684 8.78364131896153e-09
3685 8.78379414072372e-09
3686 8.78347009933278e-09
3687 8.78721326495052e-09
3688 8.78226873146648e-09
3689 8.78249635474782e-09
3690 8.7855652142893e-09
3691 8.78436992533122e-09
3692 8.78167556564446e-09
3693 8.78035352604167e-09
3694 8.78031780734284e-09
3695 8.78199970971022e-09
3696 8.78092930554214e-09
3697 8.77974688273664e-09
3698 8.78219454313045e-09
3699 8.77907066860334e-09
3700 8.78171998268396e-09
3701 8.77849359632449e-09
3702 8.77764537945275e-09
3703 8.77841337568186e-09
3704 8.77794099847506e-09
3705 8.77811426706204e-09
3706 8.77867687393979e-09
3707 8.77676666738436e-09
3708 8.7769912146124e-09
3709 8.78022701194386e-09
3710 8.77571434342722e-09
3711 8.77513651154765e-09
3712 8.77983283011086e-09
3713 8.77482285585596e-09
3714 8.77390431834601e-09
3715 8.77764228533057e-09
3716 8.77671075467934e-09
3717 8.77458213126986e-09
3718 8.77594032411277e-09
3719 8.77374516569662e-09
3720 8.77378344698448e-09
3721 8.7728664245057e-09
3722 8.77623825502188e-09
3723 8.77236040303098e-09
3724 8.77184267361608e-09
3725 8.77160848111042e-09
3726 8.77292612200958e-09
3727 8.7716043912292e-09
3728 8.77100662083352e-09
3729 8.77216217048948e-09
3730 8.77089087759214e-09
3731 8.77302208558695e-09
3732 8.76925485837621e-09
3733 8.76857468853909e-09
3734 8.76903724360278e-09
3735 8.77122133755015e-09
3736 8.76837648480788e-09
3737 8.76858994341284e-09
3738 8.76809292422315e-09
3739 8.76799710199105e-09
3740 8.76736227107405e-09
3741 8.76751059879916e-09
3742 8.76718278952637e-09
3743 8.76650781726079e-09
3744 8.76716956414725e-09
3745 8.7659580044197e-09
3746 8.76560294755385e-09
3747 8.76557576630355e-09
3748 8.76551865639647e-09
3749 8.76512906441973e-09
3750 8.7651253056209e-09
3751 8.7638142215879e-09
3752 8.76427220994347e-09
3753 8.76682376086541e-09
3754 8.76293611680456e-09
3755 8.76247626386412e-09
3756 8.7624465147304e-09
3757 8.76205255285339e-09
3758 8.76189728874627e-09
3759 8.76244174757151e-09
3760 8.76175200656415e-09
3761 8.76108625395833e-09
3762 8.76176132604683e-09
3763 8.76087476097548e-09
3764 8.76064601843668e-09
3765 8.76065752538485e-09
3766 8.76265557978734e-09
3767 8.75939219384392e-09
3768 8.75885482327921e-09
3769 8.75854123359721e-09
3770 8.75821284662681e-09
3771 8.75859066862966e-09
3772 8.75840210642215e-09
3773 8.75736686442585e-09
3774 8.7576066852349e-09
3775 8.756786291364e-09
3776 8.75770828899502e-09
3777 8.75636491209919e-09
3778 8.75708951417198e-09
3779 8.75901372214549e-09
3780 8.75615066252972e-09
3781 8.75505183314112e-09
3782 8.75565303883052e-09
3783 8.75546035740415e-09
3784 8.75479268517415e-09
3785 8.75401599018111e-09
3786 8.75441099487673e-09
3787 8.75329295199623e-09
3788 8.75607044491938e-09
3789 8.75366321956977e-09
3790 8.75288607238983e-09
3791 8.75259527496897e-09
3792 8.75120020568665e-09
3793 8.75229479238571e-09
3794 8.75262854561909e-09
3795 8.75118241264877e-09
3796 8.75102627401897e-09
3797 8.75477678038278e-09
3798 8.75097045242162e-09
3799 8.75001803366665e-09
3800 8.75006903517522e-09
3801 8.74927398390041e-09
3802 8.75009203100963e-09
3803 8.74930988581379e-09
3804 8.74935528997339e-09
3805 8.74895303104584e-09
3806 8.74812150840343e-09
3807 8.75115287658074e-09
3808 8.74895940655013e-09
3809 8.74724570802676e-09
3810 8.74644131051983e-09
3811 8.74683203273135e-09
3812 8.74766758515028e-09
3813 8.74616098476416e-09
3814 8.7460453346011e-09
3815 8.74536433847356e-09
3816 8.74471063395943e-09
3817 8.74524802037258e-09
3818 8.743819932093e-09
3819 8.74834312556555e-09
3820 8.74274771421602e-09
3821 8.74546454716907e-09
3822 8.74310938353551e-09
3823 8.74348911636402e-09
3824 8.74647510761689e-09
3825 8.74668061096734e-09
3826 8.74227482049184e-09
3827 8.74194522924865e-09
3828 8.74127474634334e-09
3829 8.74504225023554e-09
3830 8.74425325447792e-09
3831 8.74199332624032e-09
3832 8.74022325943191e-09
3833 8.73950314592836e-09
3834 8.73949867335327e-09
3835 8.7408074602377e-09
3836 8.73847853309628e-09
3837 8.73928136351648e-09
3838 8.74199359750249e-09
3839 8.73803128347006e-09
3840 8.736888015351e-09
3841 8.73861417634292e-09
3842 8.73846424942482e-09
3843 8.74118158847303e-09
3844 8.73787737808163e-09
3845 8.73659390924347e-09
3846 8.73676463331724e-09
3847 8.73783754291174e-09
3848 8.73653484238096e-09
3849 8.73613081876085e-09
3850 8.73838534026783e-09
3851 8.73422595543127e-09
3852 8.73553386862569e-09
3853 8.73407627545453e-09
3854 8.73457882930723e-09
3855 8.73289813946992e-09
3856 8.73712053944819e-09
3857 8.73312041328733e-09
3858 8.73293228354638e-09
3859 8.73657321742716e-09
3860 8.7313764144531e-09
3861 8.73246043397308e-09
3862 8.73209076900083e-09
3863 8.73193984800291e-09
3864 8.7305890259029e-09
3865 8.73501320802078e-09
3866 8.73045593489247e-09
3867 8.73076377420207e-09
3868 8.72997551332882e-09
3869 8.72890542592819e-09
3870 8.73200703978283e-09
3871 8.72972303644221e-09
3872 8.728633471837e-09
3873 8.73200560509019e-09
3874 8.72795575392049e-09
3875 8.72783633672214e-09
3876 8.72851359012827e-09
3877 8.7264937175674e-09
3878 8.72783681245964e-09
3879 8.7312539003917e-09
3880 8.72620036580785e-09
3881 8.72736583280159e-09
3882 8.72657072037431e-09
3883 8.7294159179796e-09
3884 8.72540232921271e-09
3885 8.72500242217467e-09
3886 8.72419798998714e-09
3887 8.72586592268187e-09
3888 8.72454557501734e-09
3889 8.72408306853978e-09
3890 8.72359920308513e-09
3891 8.72450815370129e-09
3892 8.7225885370204e-09
3893 8.72591146841573e-09
3894 8.72285577951937e-09
3895 8.72245120586007e-09
3896 8.72214310472519e-09
3897 8.72184593479069e-09
3898 8.72392130073924e-09
3899 8.72202403687933e-09
3900 8.72111401103992e-09
3901 8.72477620899453e-09
3902 8.71969519224136e-09
3903 8.72055172292108e-09
3904 8.71936005802237e-09
3905 8.7196824244476e-09
3906 8.72280407659964e-09
3907 8.7182133898453e-09
3908 8.71824091398798e-09
3909 8.7183414340733e-09
3910 8.72202252272941e-09
3911 8.71698621700973e-09
3912 8.71744043792061e-09
3913 8.71770071383948e-09
3914 8.71678670731624e-09
3915 8.71977684791664e-09
3916 8.7156589903617e-09
3917 8.71640681759933e-09
3918 8.7160105053663e-09
3919 8.71627747087156e-09
3920 8.71576672799995e-09
3921 8.71527742390271e-09
3922 8.71468216732257e-09
3923 8.71504859595307e-09
3924 8.71378848753962e-09
3925 8.71729614058958e-09
3926 8.71285490174761e-09
3927 8.71295320099136e-09
3928 8.71318448054242e-09
3929 8.71326365367575e-09
3930 8.71661158345638e-09
3931 8.7116631105888e-09
3932 8.71245915603364e-09
3933 8.71117349240946e-09
3934 8.71494968705117e-09
3935 8.71043357607293e-09
3936 8.71025051825403e-09
3937 8.71198263726475e-09
3938 8.7092368173669e-09
3939 8.71124652945116e-09
3940 8.71316998725391e-09
3941 8.71109503602918e-09
3942 8.70880256081258e-09
3943 8.70872290627267e-09
3944 8.71284619396312e-09
3945 8.70743740079977e-09
3946 8.70788236796705e-09
3947 8.70852488478457e-09
3948 8.71111466026198e-09
3949 8.70674639338181e-09
3950 8.70657649488377e-09
3951 8.71063703437869e-09
3952 8.70595062426455e-09
3953 8.70651323105209e-09
3954 8.70491676892293e-09
3955 8.70602022493006e-09
3956 8.70514238470593e-09
3957 8.70531998412721e-09
3958 8.7053691064376e-09
3959 8.70359290752742e-09
3960 8.7051892341955e-09
3961 8.70421269828087e-09
3962 8.70325378552611e-09
3963 8.70700038086564e-09
3964 8.70224547544485e-09
3965 8.70268173092104e-09
3966 8.70265323342501e-09
3967 8.70273316463249e-09
3968 8.7051179999062e-09
3969 8.70167415490536e-09
3970 8.70033914626006e-09
3971 8.70538517525749e-09
3972 8.70026662216916e-09
3973 8.70040830173308e-09
3974 8.69925574603636e-09
3975 8.70164719653233e-09
3976 8.69981191779084e-09
3977 8.70277107283685e-09
3978 8.69982107181561e-09
3979 8.69838509973131e-09
3980 8.69923692535524e-09
3981 8.70242004548383e-09
3982 8.69821650126279e-09
3983 8.69764937074663e-09
3984 8.69803836985039e-09
3985 8.69671166447983e-09
3986 8.69764181096744e-09
3987 8.69697397065661e-09
3988 8.69676466507679e-09
3989 8.70055167027234e-09
3990 8.6958121281247e-09
3991 8.69516712344309e-09
3992 8.69609871776561e-09
3993 8.69551695278914e-09
3994 8.6976905636707e-09
3995 8.69338805764563e-09
3996 8.69438620133833e-09
3997 8.69366319214415e-09
3998 8.69699877591862e-09
3999 8.69540693553689e-09
4000 8.69313662069104e-09
4001 8.6928385181137e-09
4002 8.69669468065576e-09
4003 8.69238935678057e-09
4004 8.6922831403699e-09
4005 8.6957527394918e-09
4006 8.69072382563007e-09
4007 8.69174174571163e-09
4008 8.69301259470712e-09
4009 8.6905751518207e-09
4010 8.69370290769445e-09
4011 8.69262839248847e-09
4012 8.68798385222913e-09
4013 8.68849330080274e-09
4014 8.68766615070432e-09
4015 8.69058949815732e-09
4016 8.6885608823542e-09
4017 8.69127001728448e-09
4018 8.68739593243989e-09
4019 8.68751191209632e-09
4020 8.68695384513563e-09
4021 8.68722464794636e-09
4022 8.68610560384592e-09
4023 8.68673569934947e-09
4024 8.68914462677506e-09
4025 8.68443856991835e-09
4026 8.68628110015784e-09
4027 8.68529712338972e-09
4028 8.68508608785346e-09
4029 8.68878647198451e-09
4030 8.68359158222309e-09
4031 8.6831998801426e-09
4032 8.68335959383626e-09
4033 8.68286610422597e-09
4034 8.68269111049119e-09
4035 8.68712374506309e-09
4036 8.68134246555935e-09
4037 8.68337324503449e-09
4038 8.68155625654427e-09
4039 8.68613779418659e-09
4040 8.68070566265727e-09
4041 8.68101307726493e-09
4042 8.68380166128141e-09
4043 8.67974223943413e-09
4044 8.67969486517684e-09
4045 8.67968608929404e-09
4046 8.68305296859717e-09
4047 8.6825724853537e-09
4048 8.67860109499824e-09
4049 8.67902240398594e-09
4050 8.68063074096342e-09
4051 8.67869071419225e-09
4052 8.67726670936847e-09
4053 8.67981084322345e-09
4054 8.67761518962007e-09
4055 8.67738732707179e-09
4056 8.67699820134016e-09
4057 8.67632514571504e-09
4058 8.68060539009618e-09
4059 8.67645847755366e-09
4060 8.67539268215728e-09
4061 8.67873413111508e-09
4062 8.6746124521378e-09
4063 8.67475287756603e-09
4064 8.67467018770412e-09
4065 8.67396789705488e-09
4066 8.67354138672782e-09
4067 8.67342723841896e-09
4068 8.67328546072521e-09
4069 8.67610724294976e-09
4070 8.67253749568808e-09
4071 8.67144451847934e-09
4072 8.6714929201509e-09
4073 8.67178641547617e-09
4074 8.6734771689162e-09
4075 8.67128727090327e-09
4076 8.67150775748365e-09
4077 8.67444959989222e-09
4078 8.67059607483761e-09
4079 8.66972719859582e-09
4080 8.67134485771359e-09
4081 8.67199815792613e-09
4082 8.66956313738676e-09
4083 8.66959851122256e-09
4084 8.6726838603654e-09
4085 8.66795262771669e-09
4086 8.66873286502895e-09
4087 8.66796512190293e-09
4088 8.66838474909704e-09
4089 8.66672105315908e-09
4090 8.67112339424797e-09
4091 8.66657591125364e-09
4092 8.66684654515087e-09
4093 8.66994580930869e-09
4094 8.66585436470196e-09
4095 8.66651776918609e-09
4096 8.6659271400294e-09
4097 8.66855908490671e-09
4098 8.66449789109491e-09
4099 8.66413345129713e-09
4100 8.66437325896391e-09
4101 8.6682278410341e-09
4102 8.66525853092381e-09
4103 8.66377885245379e-09
4104 8.66651545843727e-09
4105 8.66235487359535e-09
4106 8.66213745696204e-09
4107 8.6621828895711e-09
4108 8.66221343466533e-09
4109 8.66206213763487e-09
4110 8.66102749089853e-09
4111 8.66058533659925e-09
4112 8.66101057245966e-09
4113 8.66404886644412e-09
4114 8.66113588140344e-09
4115 8.65982092049938e-09
4116 8.66386045323547e-09
4117 8.65923274853686e-09
4118 8.65898005428245e-09
4119 8.65894508347148e-09
4120 8.65964521391122e-09
4121 8.65836001370801e-09
4122 8.65802628272072e-09
4123 8.65847244607698e-09
4124 8.65678409815718e-09
4125 8.65715003212086e-09
4126 8.65730441595297e-09
4127 8.65856921637015e-09
4128 8.65606193691665e-09
4129 8.65978196427919e-09
4130 8.65471805538975e-09
4131 8.65483837991787e-09
4132 8.65653877941963e-09
4133 8.65730052825031e-09
4134 8.65489033820971e-09
4135 8.65385310713784e-09
4136 8.65442922304283e-09
4137 8.65731609991926e-09
4138 8.65268493025623e-09
4139 8.65340759188815e-09
4140 8.65342746118186e-09
4141 8.65368702595432e-09
4142 8.65215989682283e-09
4143 8.65250004205426e-09
4144 8.6513183493106e-09
4145 8.65519702028528e-09
4146 8.65251625659091e-09
4147 8.65059350609737e-09
4148 8.64967045766801e-09
4149 8.65439441209703e-09
4150 8.64951611229126e-09
4151 8.6500292497127e-09
4152 8.65360329532078e-09
4153 8.64856520681884e-09
4154 8.64877072598302e-09
4155 8.65224050836111e-09
4156 8.64829495208358e-09
4157 8.64860044408333e-09
4158 8.64786486050317e-09
4159 8.64780326426118e-09
4160 8.65128271263643e-09
4161 8.64694800194188e-09
4162 8.6463265215192e-09
4163 8.64995095887361e-09
4164 8.64621774190194e-09
4165 8.64668871505825e-09
4166 8.64612663550868e-09
4167 8.64472759094098e-09
4168 8.64971658874614e-09
4169 8.64473060819409e-09
4170 8.64531949407166e-09
4171 8.64452878913957e-09
4172 8.64462680850997e-09
4173 8.64374496577919e-09
4174 8.64347991102327e-09
4175 8.64277257584673e-09
4176 8.64701167693077e-09
4177 8.64301022737712e-09
4178 8.64333794187849e-09
4179 8.64209016388645e-09
4180 8.64598236231112e-09
4181 8.64084035959167e-09
4182 8.64122566220093e-09
4183 8.64450500492192e-09
4184 8.63975494210661e-09
4185 8.64049547510959e-09
4186 8.6401725951854e-09
4187 8.64007754454443e-09
4188 8.64379137802129e-09
4189 8.6382829427023e-09
4190 8.63945292259305e-09
4191 8.63845093086735e-09
4192 8.63857408364005e-09
4193 8.64239514634241e-09
4194 8.63766188421988e-09
4195 8.63758479908994e-09
4196 8.64102031703645e-09
4197 8.63569601235875e-09
4198 8.63685952705306e-09
4199 8.63979756637878e-09
4200 8.63691420929097e-09
4201 8.63567768651097e-09
4202 8.63713831783519e-09
4203 8.63672693733608e-09
4204 8.63930618161951e-09
4205 8.63424158462078e-09
4206 8.63473368523798e-09
4207 8.63803889094433e-09
4208 8.63408558455375e-09
4209 8.63295453757545e-09
4210 8.63568675110032e-09
4211 8.63306967194277e-09
4212 8.6367179642638e-09
4213 8.63253685241816e-09
4214 8.63248713529885e-09
4215 8.63269804127903e-09
4216 8.63201330665586e-09
4217 8.63493786795733e-09
4218 8.63105903715911e-09
4219 8.63057710631665e-09
4220 8.6348836198652e-09
4221 8.63062550284649e-09
4222 8.63361087682229e-09
4223 8.62947082062859e-09
4224 8.62954729029253e-09
4225 8.63300688468821e-09
4226 8.62858502072988e-09
4227 8.63320465206707e-09
4228 8.62864083497894e-09
4229 8.62785022223006e-09
4230 8.63153695494084e-09
4231 8.62671260775522e-09
4232 8.62710165627084e-09
4233 8.63030429647249e-09
4234 8.62605993710158e-09
4235 8.62913987334513e-09
4236 8.6272176206964e-09
4237 8.62947249768292e-09
4238 8.6257369975376e-09
4239 8.6258447714177e-09
4240 8.62872416280275e-09
4241 8.62418075987803e-09
4242 8.62793660046796e-09
4243 8.62317784857541e-09
4244 8.62362425984448e-09
4245 8.62833625960013e-09
4246 8.62332717455061e-09
4247 8.62680990109527e-09
4248 8.623257270364e-09
4249 8.62488243904419e-09
4250 8.62161631431935e-09
4251 8.62630540634374e-09
4252 8.62053046295913e-09
4253 8.62507689168568e-09
4254 8.62150355252639e-09
4255 8.62442183371254e-09
4256 8.62171824286828e-09
4257 8.62037011663264e-09
4258 8.62336435193678e-09
4259 8.61982992535898e-09
4260 8.62300410933714e-09
4261 8.61899544810085e-09
4262 8.61909142462619e-09
4263 8.6181476339417e-09
4264 8.62288483991641e-09
4265 8.61814285837981e-09
4266 8.62191254626804e-09
4267 8.61698927335369e-09
4268 8.61965977137086e-09
4269 8.61694633118998e-09
4270 8.61867621406581e-09
4271 8.62033338334955e-09
4272 8.61528966070862e-09
4273 8.61888034276587e-09
4274 8.61516761173353e-09
4275 8.61541635841651e-09
4276 8.61860270285481e-09
4277 8.61403843083197e-09
4278 8.61856495645852e-09
4279 8.61402453792431e-09
4280 8.61385295523182e-09
4281 8.61683448139666e-09
4282 8.6133872777075e-09
4283 8.61686856055977e-09
4284 8.61216293172179e-09
4285 8.6155679001193e-09
4286 8.61221479410423e-09
4287 8.61453897608305e-09
4288 8.6124796432191e-09
4289 8.61251231891458e-09
4290 8.61493907598071e-09
4291 8.61050512417777e-09
4292 8.61470915321239e-09
4293 8.61024601463839e-09
4294 8.61443392446287e-09
4295 8.60852621692443e-09
4296 8.61374436609413e-09
4297 8.60857786529751e-09
4298 8.61177859146772e-09
4299 8.60904496151338e-09
4300 8.60837828044886e-09
4301 8.61181609133899e-09
4302 8.60847426968625e-09
4303 8.61123341463349e-09
4304 8.60734419112774e-09
4305 8.61089337824555e-09
4306 8.60699348826516e-09
4307 8.61069587203972e-09
4308 8.60584665105662e-09
4309 8.60967735911294e-09
4310 8.60544871252728e-09
4311 8.60896020288965e-09
4312 8.60545833782783e-09
4313 8.60485511359338e-09
4314 8.60488750734773e-09
4315 8.60812863404808e-09
4316 8.60684453368732e-09
4317 8.60336584367766e-09
4318 8.60721571353806e-09
4319 8.60292657330064e-09
4320 8.60673767278525e-09
4321 8.60226091502908e-09
4322 8.60611979320441e-09
4323 8.60123169705435e-09
4324 8.60254425780887e-09
4325 8.60463559616831e-09
4326 8.60045029082296e-09
4327 8.60430007277452e-09
4328 8.60046644762108e-09
4329 8.60032356375662e-09
4330 8.6032558337043e-09
4331 8.59923742422414e-09
4332 8.602970677625e-09
4333 8.59830929183375e-09
4334 8.59956294642361e-09
4335 8.60053622234874e-09
4336 8.60260450138289e-09
4337 8.59814197185643e-09
4338 8.59705901848362e-09
4339 8.60141792018848e-09
4340 8.5967582161145e-09
4341 8.6005437331671e-09
4342 8.59609236496944e-09
4343 8.59978724711968e-09
4344 8.59658195778018e-09
4345 8.59902527918649e-09
4346 8.59543935611412e-09
4347 8.59586566460957e-09
4348 8.59831454508991e-09
4349 8.59462568093827e-09
4350 8.59794953661508e-09
4351 8.59405141792613e-09
4352 8.59767018067392e-09
4353 8.593031899852e-09
4354 8.59632297865037e-09
4355 8.5928231957702e-09
4356 8.59629697230746e-09
4357 8.59207463500822e-09
4358 8.59518488576655e-09
4359 8.59221524176362e-09
4360 8.59568719356607e-09
4361 8.59152736665952e-09
4362 8.59508671958303e-09
4363 8.59056361956351e-09
4364 8.59287578094947e-09
4365 8.59036451484468e-09
4366 8.59456535141911e-09
4367 8.58979415553629e-09
4368 8.59165625961833e-09
4369 8.592921751642e-09
4370 8.58815654388584e-09
4371 8.59220392448318e-09
4372 8.58883056495396e-09
4373 8.59152018335696e-09
4374 8.58764670680356e-09
4375 8.5913238001209e-09
4376 8.58733997391448e-09
4377 8.59047078432401e-09
4378 8.58561742065472e-09
4379 8.59035258691382e-09
4380 8.58542069431118e-09
4381 8.58870942877649e-09
4382 8.58543419100105e-09
4383 8.58924036166442e-09
4384 8.58433712292578e-09
4385 8.58488347155878e-09
4386 8.58492643716902e-09
4387 8.58792220658627e-09
4388 8.58379423807759e-09
4389 8.58694012381006e-09
4390 8.58341586083139e-09
4391 8.58591329389036e-09
4392 8.58290203985951e-09
4393 8.58561452603268e-09
4394 8.58120999712442e-09
4395 8.58587723159138e-09
4396 8.58076469775759e-09
4397 8.58280213588552e-09
4398 8.58430982650371e-09
4399 8.5798886091254e-09
4400 8.58427487454572e-09
4401 8.57985669022038e-09
4402 8.58328355149429e-09
4403 8.57874274474979e-09
4404 8.58252760461975e-09
4405 8.57866864562712e-09
4406 8.5816844047279e-09
4407 8.57828211264788e-09
4408 8.58147732862163e-09
4409 8.57699521022298e-09
4410 8.5816863673871e-09
4411 8.57632112891138e-09
4412 8.58074947317905e-09
4413 8.5773501822331e-09
4414 8.57982005560143e-09
4415 8.57585253247195e-09
4416 8.57938471662823e-09
4417 8.57607946522926e-09
4418 8.57866465997503e-09
4419 8.57500191641519e-09
4420 8.57881360027263e-09
4421 8.57526684190646e-09
4422 8.57725596208753e-09
4423 8.57432642563383e-09
4424 8.5768471257755e-09
4425 8.5735281262142e-09
4426 8.57679703094444e-09
4427 8.57322377007669e-09
4428 8.57657687201169e-09
4429 8.57304810313736e-09
4430 8.57583654217259e-09
4431 8.57172322234895e-09
4432 8.57527477737818e-09
4433 8.57127157319337e-09
4434 8.57480242973802e-09
4435 8.57016667604044e-09
4436 8.57385298470154e-09
4437 8.56933224741008e-09
4438 8.57266128510142e-09
4439 8.56936391618879e-09
4440 8.57310178371123e-09
4441 8.5692881570329e-09
4442 8.57092405746884e-09
4443 8.56822783381478e-09
4444 8.57130193681471e-09
4445 8.56764280004335e-09
4446 8.57161055771438e-09
4447 8.56789069598324e-09
4448 8.5713885223937e-09
4449 8.5674116520329e-09
4450 8.57007017512901e-09
4451 8.56574195613052e-09
4452 8.57006846955582e-09
4453 8.56499095888696e-09
4454 8.56933535346716e-09
4455 8.56467237355524e-09
4456 8.56857546615436e-09
4457 8.56460048954261e-09
4458 8.56799250954082e-09
4459 8.56322098369972e-09
4460 8.56725794308455e-09
4461 8.56368652069062e-09
4462 8.56760785344912e-09
4463 8.56322319219005e-09
4464 8.56289832132373e-09
4465 8.56663631829124e-09
4466 8.56188968652999e-09
4467 8.56581883767332e-09
4468 8.56140923594095e-09
4469 8.56355199902598e-09
4470 8.56094610954955e-09
4471 8.56398239509198e-09
4472 8.56141487231427e-09
4473 8.56416606510868e-09
4474 8.559278633892e-09
4475 8.56020689703196e-09
4476 8.56308347880336e-09
4477 8.55863531783879e-09
4478 8.56246670696836e-09
4479 8.55837967296547e-09
4480 8.5615495848887e-09
4481 8.55711063280085e-09
4482 8.5586891394665e-09
4483 8.5611238555966e-09
4484 8.55678175621516e-09
4485 8.56000566988591e-09
4486 8.55580417818275e-09
4487 8.56081170979434e-09
4488 8.55547423533193e-09
4489 8.55910725543196e-09
4490 8.555893563085e-09
4491 8.55803816540018e-09
4492 8.55547590111055e-09
4493 8.5588385565355e-09
4494 8.55402081263029e-09
4495 8.55792920118753e-09
4496 8.5542098215638e-09
4497 8.55816263696518e-09
4498 8.55247529135683e-09
4499 8.55364663483044e-09
4500 8.55257012438715e-09
4501 8.55638299387257e-09
4502 8.55221093579855e-09
4503 8.55534590640805e-09
4504 8.55115607917983e-09
4505 8.55220501078807e-09
4506 8.55542685467003e-09
4507 8.55088388800052e-09
4508 8.55387495912741e-09
4509 8.55077632701956e-09
4510 8.5536411718462e-09
4511 8.54917086186985e-09
4512 8.5535768127587e-09
4513 8.54981376371272e-09
4514 8.55301949262421e-09
4515 8.54804922894442e-09
4516 8.54874141113099e-09
4517 8.55213193101667e-09
4518 8.54743320787693e-09
4519 8.55167634926379e-09
4520 8.54688223978633e-09
4521 8.55000937862588e-09
4522 8.54672257374206e-09
4523 8.55041028512138e-09
4524 8.54600710500919e-09
4525 8.5498704405565e-09
4526 8.54519976730556e-09
4527 8.54911692450833e-09
4528 8.54551294775241e-09
4529 8.54783811682358e-09
4530 8.54464292057711e-09
4531 8.54830039086207e-09
4532 8.54423358975787e-09
4533 8.54562047793161e-09
4534 8.5478522287713e-09
4535 8.54354302549931e-09
4536 8.5437985435019e-09
4537 8.54616480316928e-09
4538 8.54216210923603e-09
4539 8.54539318819597e-09
4540 8.5409258657812e-09
4541 8.54522997371238e-09
4542 8.541242781962e-09
4543 8.5445951969812e-09
4544 8.5399420188953e-09
4545 8.54342215639986e-09
4546 8.53985474363189e-09
4547 8.54402819146921e-09
4548 8.53929362006067e-09
4549 8.54364760554532e-09
4550 8.53895941755473e-09
4551 8.54218900880888e-09
4552 8.53799392348808e-09
4553 8.54199510069015e-09
4554 8.53791015485023e-09
4555 8.54160937551612e-09
4556 8.53654929457853e-09
4557 8.54025414144532e-09
4558 8.53682963002783e-09
4559 8.53812277060856e-09
4560 8.5369252713699e-09
4561 8.53977886117879e-09
4562 8.53589104541502e-09
4563 8.53952184343143e-09
4564 8.53655382088842e-09
4565 8.53578075492995e-09
4566 8.53772581185797e-09
4567 8.53416323790684e-09
4568 8.53752807904867e-09
4569 8.5369182672157e-09
4570 8.53343774859933e-09
4571 8.53576752414542e-09
4572 8.53328347798915e-09
4573 8.5365433707546e-09
4574 8.53244098653061e-09
4575 8.53582477195608e-09
4576 8.53541025688798e-09
4577 8.53530878849179e-09
4578 8.53384583197198e-09
4579 8.53056147558789e-09
4580 8.53432786090264e-09
4581 8.53401745010834e-09
4582 8.53333690591096e-09
4583 8.53319708536693e-09
4584 8.52888943784352e-09
4585 8.53276624362964e-09
4586 8.52972509821082e-09
4587 8.53270725430233e-09
4588 8.5278577809586e-09
4589 8.5324733925668e-09
4590 8.52857606106067e-09
4591 8.52837018458508e-09
4592 8.5309317519669e-09
4593 8.52740907181931e-09
4594 8.52988172504038e-09
4595 8.5276854253219e-09
4596 8.53045830025156e-09
4597 8.52567195482046e-09
4598 8.5279146784592e-09
4599 8.5255656333133e-09
4600 8.52888749090092e-09
4601 8.52419100922514e-09
4602 8.52839469355632e-09
4603 8.52560489816856e-09
4604 8.52331796715594e-09
4605 8.52757067898446e-09
4606 8.52283464795878e-09
4607 8.52671928799203e-09
4608 8.52271561933848e-09
4609 8.52336492497552e-09
4610 8.5265535544124e-09
4611 8.52579132635395e-09
4612 8.52517545279347e-09
4613 8.52452012512633e-09
4614 8.52422089079458e-09
4615 8.52112940435273e-09
4616 8.52444313584333e-09
4617 8.52108143905433e-09
4618 8.52043228335292e-09
4619 8.52407413141215e-09
4620 8.51940984932659e-09
4621 8.52033654200751e-09
4622 8.52295954906213e-09
4623 8.51837473003075e-09
4624 8.518362609157e-09
4625 8.52208941463928e-09
4626 8.51809637303358e-09
4627 8.52190451312851e-09
4628 8.51765434948387e-09
4629 8.51708980462163e-09
4630 8.5205237193367e-09
4631 8.51609696896072e-09
4632 8.51963538583966e-09
4633 8.51593377066556e-09
4634 8.5200808059363e-09
4635 8.51554925232478e-09
4636 8.51942105154629e-09
4637 8.51503438029089e-09
4638 8.51531865452698e-09
4639 8.51830900411993e-09
4640 8.51274311463451e-09
4641 8.51549898792103e-09
4642 8.51703879856808e-09
4643 8.51307623547098e-09
4644 8.51711546203532e-09
4645 8.51288075170986e-09
4646 8.51629426010247e-09
4647 8.5125754367768e-09
4648 8.51438602294391e-09
4649 8.51229978594131e-09
4650 8.5146007909459e-09
4651 8.51177778054407e-09
4652 8.51324196169539e-09
4653 8.51038264326059e-09
4654 8.51362821558538e-09
4655 8.50965736476361e-09
4656 8.51024751591378e-09
4657 8.513445660982e-09
4658 8.51288528187083e-09
4659 8.50832811318675e-09
4660 8.51243670396684e-09
4661 8.50827229942341e-09
4662 8.5108806077705e-09
4663 8.50847852795073e-09
4664 8.51181083221658e-09
4665 8.50690651339053e-09
4666 8.51165173992863e-09
4667 8.5070776823376e-09
4668 8.51097609430584e-09
4669 8.50743414007593e-09
4670 8.50649497006334e-09
4671 8.50997242474866e-09
4672 8.50521530068593e-09
4673 8.50533716233864e-09
4674 8.50457155193024e-09
4675 8.50852490707349e-09
4676 8.50431685418124e-09
4677 8.50843622859226e-09
4678 8.50388048306838e-09
4679 8.50688879867889e-09
4680 8.50368573813293e-09
4681 8.507558297427e-09
4682 8.50229568134797e-09
4683 8.50718017617047e-09
4684 8.50235315746906e-09
4685 8.50613144615098e-09
4686 8.50124161344556e-09
4687 8.5015832758592e-09
4688 8.50142582125429e-09
4689 8.50578882986147e-09
4690 8.50009985264161e-09
4691 8.50491738149661e-09
4692 8.50004968353663e-09
4693 8.50310663895459e-09
4694 8.49962490551831e-09
4695 8.50280791794139e-09
4696 8.49937610408052e-09
4697 8.50211535740469e-09
4698 8.49855036744868e-09
4699 8.50212796869865e-09
4700 8.4971394741229e-09
4701 8.50230923880174e-09
4702 8.49757109775556e-09
4703 8.50085108018012e-09
4704 8.49718366636304e-09
4705 8.50105818188396e-09
4706 8.49595207964154e-09
4707 8.49937432854941e-09
4708 8.49561376092128e-09
4709 8.49552007336357e-09
4710 8.49571824338563e-09
4711 8.49655847465214e-09
4712 8.49626076487864e-09
4713 8.49383563217981e-09
4714 8.49759999147631e-09
4715 8.4934248348531e-09
4716 8.49692993654116e-09
4717 8.49273893945013e-09
4718 8.49345380465882e-09
4719 8.49438430322591e-09
4720 8.49248308219114e-09
4721 8.49185113727807e-09
4722 8.49555466168189e-09
4723 8.49127180350268e-09
4724 8.49521449762525e-09
4725 8.49037875636643e-09
4726 8.49453559519792e-09
4727 8.4903997504826e-09
4728 8.49032209356698e-09
4729 8.49462190207623e-09
4730 8.48853522822379e-09
4731 8.49302697048665e-09
4732 8.48793116237784e-09
4733 8.49424991238718e-09
4734 8.48782250540553e-09
4735 8.49235856778541e-09
4736 8.48804906381645e-09
4737 8.49131018935967e-09
4738 8.48687010378962e-09
4739 8.49118633092227e-09
4740 8.48626564414062e-09
4741 8.49172814406524e-09
4742 8.48607991187772e-09
4743 8.48965018254017e-09
4744 8.48474400608895e-09
4745 8.48968545556078e-09
4746 8.48479901107563e-09
4747 8.48704702333808e-09
4748 8.48438879159225e-09
4749 8.48835495950023e-09
4750 8.48357901436086e-09
4751 8.48768826757634e-09
4752 8.48279835337362e-09
4753 8.48432000833538e-09
4754 8.48796941071289e-09
4755 8.48324620137536e-09
4756 8.48797513667426e-09
4757 8.48204082590925e-09
4758 8.48723096617821e-09
4759 8.48177713540232e-09
4760 8.48567569836783e-09
4761 8.48169226983636e-09
4762 8.48276458031982e-09
4763 8.48037494842468e-09
4764 8.48466629197603e-09
4765 8.48046184710044e-09
4766 8.48367006712342e-09
4767 8.47928966163369e-09
4768 8.48267491498217e-09
4769 8.47943523041989e-09
4770 8.48349757302802e-09
4771 8.47842451274366e-09
4772 8.48286278191251e-09
4773 8.47762442010569e-09
4774 8.48131342025271e-09
4775 8.47797057615396e-09
4776 8.48120888176374e-09
4777 8.4764198410997e-09
4778 8.48070557615821e-09
4779 8.47667898137838e-09
4780 8.4781507751e-09
4781 8.48002483996857e-09
4782 8.47613429906524e-09
4783 8.48018111889587e-09
4784 8.47445295461963e-09
4785 8.47848394375239e-09
4786 8.47459728629124e-09
4787 8.47794800794277e-09
4788 8.47407003775474e-09
4789 8.47761703078736e-09
4790 8.4728304537221e-09
4791 8.47717210682858e-09
4792 8.47213097449756e-09
4793 8.47696272223558e-09
4794 8.47283464941451e-09
4795 8.4763804358845e-09
4796 8.47140173263622e-09
4797 8.47532883176821e-09
4798 8.47117569753586e-09
4799 8.47601356004923e-09
4800 8.47024162384941e-09
4801 8.47422584004942e-09
4802 8.4696097269743e-09
4803 8.47114103427055e-09
4804 8.46940208484859e-09
4805 8.47337884601201e-09
4806 8.4683090540022e-09
4807 8.47026728875017e-09
4808 8.46902574559022e-09
4809 8.47228322570226e-09
4810 8.4681838187009e-09
4811 8.47150577497346e-09
4812 8.46716996436014e-09
4813 8.4706933771439e-09
4814 8.46745760709616e-09
4815 8.47096189030005e-09
4816 8.46590431490685e-09
4817 8.47130929886725e-09
4818 8.46681506936481e-09
4819 8.47034045204503e-09
4820 8.46473334934622e-09
4821 8.46893154060613e-09
4822 8.46426394000149e-09
4823 8.46829638382479e-09
4824 8.46384061485572e-09
4825 8.46828448073517e-09
4826 8.46318686145015e-09
4827 8.46829152723744e-09
4828 8.46255951168773e-09
4829 8.4676355482094e-09
4830 8.46245495972342e-09
4831 8.4666044069398e-09
4832 8.46156343101839e-09
4833 8.46644759881082e-09
4834 8.46095045323181e-09
4835 8.46535641458451e-09
4836 8.46006914921593e-09
4837 8.46520794769295e-09
4838 8.46023484557534e-09
4839 8.46402706217164e-09
4840 8.45942452189913e-09
4841 8.46302449712677e-09
4842 8.45916313690909e-09
4843 8.46263582419959e-09
4844 8.45986415103084e-09
4845 8.46357332912745e-09
4846 8.45817432181678e-09
4847 8.45847044345094e-09
4848 8.46164757149076e-09
4849 8.45749821590941e-09
4850 8.4609699266755e-09
4851 8.45582075385126e-09
4852 8.4609431137625e-09
4853 8.45615012868423e-09
4854 8.45951847325277e-09
4855 8.45612156099435e-09
4856 8.45574705093943e-09
4857 8.45949087537046e-09
4858 8.45445256043159e-09
4859 8.45858695195517e-09
4860 8.45442363173882e-09
4861 8.45466824177243e-09
4862 8.45447428370755e-09
4863 8.45766692073036e-09
4864 8.4524638711489e-09
4865 8.45721335647476e-09
4866 8.45288116087944e-09
4867 8.45335624576754e-09
4868 8.45640101882622e-09
4869 8.45163948128003e-09
4870 8.45529006741203e-09
4871 8.45079568434493e-09
4872 8.45512215125188e-09
4873 8.45094721584755e-09
4874 8.45455933674161e-09
4875 8.45036716292519e-09
4876 8.45365695488487e-09
4877 8.44950936911698e-09
4878 8.45383114177201e-09
4879 8.44825907426322e-09
4880 8.45356279528808e-09
4881 8.44972919170767e-09
4882 8.45286207541707e-09
4883 8.44812096646025e-09
4884 8.45222351759239e-09
4885 8.44686732869721e-09
4886 8.45143619179756e-09
4887 8.44636740583021e-09
4888 8.4508407045894e-09
4889 8.44578130002049e-09
4890 8.4503478112799e-09
4891 8.44541734698612e-09
4892 8.44969917360855e-09
4893 8.44566621788223e-09
4894 8.44922762568284e-09
4895 8.44456954715578e-09
4896 8.44900212883248e-09
4897 8.4438775412865e-09
4898 8.44799959085624e-09
4899 8.44363970787382e-09
4900 8.44745847024264e-09
4901 8.44284318379101e-09
4902 8.44689379033564e-09
4903 8.44230047358091e-09
4904 8.44639286182863e-09
4905 8.44136405363238e-09
4906 8.44611814231067e-09
4907 8.44166703454358e-09
4908 8.44476105738012e-09
4909 8.44208067329877e-09
4910 8.44576362312582e-09
4911 8.44001839179348e-09
4912 8.44419194577506e-09
4913 8.43902583160666e-09
4914 8.44411754491459e-09
4915 8.44024125082332e-09
4916 8.44441979374472e-09
4917 8.43898218721995e-09
4918 8.43926213292545e-09
4919 8.44271229100829e-09
4920 8.43770681008271e-09
4921 8.4417203870743e-09
4922 8.43760652124992e-09
4923 8.44108506815183e-09
4924 8.4371452345397e-09
4925 8.44152644522561e-09
4926 8.4364407441917e-09
4927 8.44004010348148e-09
4928 8.43599620536234e-09
4929 8.4407081884022e-09
4930 8.43449664580942e-09
4931 8.43878026816208e-09
4932 8.43546169951997e-09
4933 8.4386892859542e-09
4934 8.43420997902883e-09
4935 8.4351061759233e-09
4936 8.43450087883307e-09
4937 8.43824860366493e-09
4938 8.43435841753337e-09
4939 8.4333636104772e-09
4940 8.43725916189336e-09
4941 8.43216413278641e-09
4942 8.43679368899702e-09
4943 8.43165469103374e-09
4944 8.43606578289358e-09
4945 8.43145263701439e-09
4946 8.43516572929248e-09
4947 8.43051286326252e-09
4948 8.43434346724237e-09
4949 8.42965171864857e-09
4950 8.43387410003754e-09
4951 8.4294916931904e-09
4952 8.43425607510023e-09
4953 8.4295765005668e-09
4954 8.42969797536924e-09
4955 8.43227044256684e-09
4956 8.4287869774255e-09
4957 8.43205173516742e-09
4958 8.4278426022294e-09
4959 8.42824886237298e-09
4960 8.43093034600534e-09
4961 8.42782482033538e-09
4962 8.43091510194932e-09
4963 8.42609078410134e-09
4964 8.43023809881521e-09
4965 8.42572809622855e-09
4966 8.43009355049745e-09
4967 8.42483240980074e-09
4968 8.42953945026648e-09
4969 8.4295475661425e-09
4970 8.42489091265608e-09
4971 8.42916294905432e-09
4972 8.42448044780647e-09
4973 8.42803407168774e-09
4974 8.42306626153144e-09
4975 8.42758078619527e-09
4976 8.42322469579021e-09
4977 8.42687907661593e-09
4978 8.42255504734934e-09
4979 8.42730071747705e-09
4980 8.42612221738093e-09
4981 8.4215228456988e-09
4982 8.42562901929439e-09
4983 8.42010034970975e-09
4984 8.42522844088367e-09
4985 8.41981803808428e-09
4986 8.42438691977393e-09
4987 8.4204275569294e-09
4988 8.42382513773637e-09
4989 8.41962097223453e-09
4990 8.42290082361807e-09
4991 8.41857381109895e-09
4992 8.42262546531941e-09
4993 8.41778453969377e-09
4994 8.42171181616308e-09
4995 8.41723115189291e-09
4996 8.42147971568535e-09
4997 8.41743504914616e-09
4998 8.42100518088501e-09
4999 8.41653005773102e-09
};
\addlegendentry{Train}
\addplot [semithick, black]
table {%
0 0.00200257077813148
1 0.000329827074892819
2 0.000233021681196988
3 0.00022674618230667
4 0.000215797830605879
5 0.000194689724594355
6 0.000156720605446026
7 9.24669075175188e-05
8 4.1404757212149e-05
9 2.2587992134504e-05
10 1.76786452357192e-05
11 1.64886769198347e-05
12 1.59521769091953e-05
13 1.54804438352585e-05
14 1.49452725963783e-05
15 1.43474717333447e-05
16 1.36568160087336e-05
17 1.28745377878658e-05
18 1.20107197290054e-05
19 1.10851688077673e-05
20 1.00969700724818e-05
21 9.07779940462206e-06
22 8.05929903435754e-06
23 7.05589809513185e-06
24 6.09440030530095e-06
25 5.19735340276384e-06
26 4.38687266068882e-06
27 3.69510303244169e-06
28 2.81860366158071e-06
29 2.09936933970312e-06
30 1.66399149748031e-06
31 1.41293185151881e-06
32 1.26940096834005e-06
33 1.18220179956552e-06
34 1.06702498214872e-06
35 1.01800162610743e-06
36 9.7025690593e-07
37 9.14683710107056e-07
38 8.62043179949978e-07
39 8.24828475742834e-07
40 7.8544667303504e-07
41 7.50016170059098e-07
42 7.25975780824228e-07
43 7.1362546805176e-07
44 7.0463482870764e-07
45 7.00729515301646e-07
46 6.97685379691393e-07
47 6.94486004704231e-07
48 6.91532250129967e-07
49 6.88702129991725e-07
50 6.85884742779308e-07
51 6.83170583215542e-07
52 6.79702452544007e-07
53 6.76312652103661e-07
54 6.72789667532925e-07
55 6.67564393097564e-07
56 6.63617868212896e-07
57 6.59280260606465e-07
58 6.5490445422256e-07
59 6.4949551870086e-07
60 6.44156102680427e-07
61 6.39097322618909e-07
62 6.33599881894042e-07
63 6.27684244136617e-07
64 6.21539186340669e-07
65 6.15640601608902e-07
66 6.09327287293127e-07
67 6.01860620008665e-07
68 5.94271739373653e-07
69 5.83113603624952e-07
70 5.7408362863498e-07
71 5.64018762361229e-07
72 5.54059965907072e-07
73 5.47721981547511e-07
74 5.34792548023688e-07
75 5.23190465173684e-07
76 5.0937092055392e-07
77 4.98050326314114e-07
78 4.855934889747e-07
79 4.7499315769528e-07
80 4.64406753053481e-07
81 4.54940021654693e-07
82 4.47077951548636e-07
83 4.40017856817576e-07
84 4.33185675774439e-07
85 4.25836617523601e-07
86 4.22119200038651e-07
87 4.1518191551404e-07
88 4.05760545163503e-07
89 4.00196114469509e-07
90 3.93812001675542e-07
91 3.87314315730691e-07
92 3.81730018261806e-07
93 3.7622103832291e-07
94 3.71625731077074e-07
95 3.65713418659652e-07
96 3.60303772595216e-07
97 3.54408427938324e-07
98 3.48435264641012e-07
99 3.43060577279175e-07
100 3.37579621145778e-07
101 3.33143304942496e-07
102 3.2846466524461e-07
103 3.2417193551737e-07
104 3.1927444865687e-07
105 3.133520465326e-07
106 3.06900801660959e-07
107 3.02249986816605e-07
108 2.97504271884463e-07
109 2.93587277155893e-07
110 2.87494231088203e-07
111 2.84350647916654e-07
112 2.8508682703432e-07
113 2.78951546306416e-07
114 2.76261403087119e-07
115 2.67683077481706e-07
116 2.65294033852115e-07
117 2.59771297805855e-07
118 2.57401239878163e-07
119 2.52728654004386e-07
120 2.50040329774492e-07
121 2.45798418063714e-07
122 2.44649811520503e-07
123 2.42158449736962e-07
124 2.37448247730754e-07
125 2.35135871662351e-07
126 2.31503364034324e-07
127 2.28386326739383e-07
128 2.22633033786224e-07
129 2.21650111598137e-07
130 2.18977518784413e-07
131 2.16372384898023e-07
132 2.13919136626828e-07
133 2.11531741456383e-07
134 2.09422609032117e-07
135 2.04827486527392e-07
136 2.01548687073227e-07
137 2.02775098046004e-07
138 2.0029737868299e-07
139 1.98013509589146e-07
140 1.95813925074617e-07
141 1.94008336507068e-07
142 1.91743765753927e-07
143 1.89979090237102e-07
144 1.87888403502257e-07
145 1.85708117328431e-07
146 1.84175945605602e-07
147 1.82413387506131e-07
148 1.80855550979686e-07
149 1.7883824909859e-07
150 1.7757379566774e-07
151 1.75909349309222e-07
152 1.74050100554268e-07
153 1.72315949953372e-07
154 1.70560994661173e-07
155 1.69432951224735e-07
156 1.66970110626607e-07
157 1.65938430995993e-07
158 1.63861059832016e-07
159 1.61597554892978e-07
160 1.615026832269e-07
161 1.59777314934217e-07
162 1.58078819367802e-07
163 1.56380082216856e-07
164 1.52140671616507e-07
165 1.53726432472467e-07
166 1.50231102225007e-07
167 1.49899264556552e-07
168 1.48529863963631e-07
169 1.46874697293242e-07
170 1.46056009953099e-07
171 1.44032838989006e-07
172 1.43841930366762e-07
173 1.42554242188453e-07
174 1.41668664355166e-07
175 1.40667907544412e-07
176 1.3808426047035e-07
177 1.38191296628065e-07
178 1.38368505986364e-07
179 1.37547530698612e-07
180 1.37332946792412e-07
181 1.36854339416459e-07
182 1.36436213438174e-07
183 1.36050076093852e-07
184 1.35632376441208e-07
185 1.35062776962513e-07
186 1.34534815288134e-07
187 1.34052143607732e-07
188 1.33613355046691e-07
189 1.34051674649527e-07
190 1.31162522620798e-07
191 1.33304453697747e-07
192 1.32911694095128e-07
193 1.30407585174908e-07
194 1.2997932685721e-07
195 1.29821302152777e-07
196 1.29520842051534e-07
197 1.29165869111603e-07
198 1.29773326307259e-07
199 1.29579973418004e-07
200 1.280712353946e-07
201 1.28711974411999e-07
202 1.28558923506716e-07
203 1.2691934614395e-07
204 1.26976587466743e-07
205 1.28807045030044e-07
206 1.28403215171602e-07
207 1.28155704715027e-07
208 1.27418729789497e-07
209 1.27672223015907e-07
210 1.27061767329906e-07
211 1.26125570432123e-07
212 1.26492253116339e-07
213 1.26338818517979e-07
214 1.26038841585796e-07
215 1.2583629427354e-07
216 1.24678280144508e-07
217 1.24519189625971e-07
218 1.24151853242438e-07
219 1.23676940688711e-07
220 1.23020654996253e-07
221 1.22902108046219e-07
222 1.22786275369435e-07
223 1.2268249349745e-07
224 1.22245708666924e-07
225 1.21955110898853e-07
226 1.21621994253474e-07
227 1.216196636733e-07
228 1.20985447438215e-07
229 1.20848582696453e-07
230 1.22113462452944e-07
231 1.2066213628259e-07
232 1.20414838988836e-07
233 1.20520340374242e-07
234 1.19976746759676e-07
235 1.19739894444137e-07
236 1.19542207244194e-07
237 1.18278855154585e-07
238 1.19144196730758e-07
239 1.19277956400765e-07
240 1.18904701196243e-07
241 1.18814156735425e-07
242 1.18643853852518e-07
243 1.18212753363878e-07
244 1.18313529640091e-07
245 1.15825507407408e-07
246 1.18944598170856e-07
247 1.1891258822061e-07
248 1.18174419583283e-07
249 1.15145240897618e-07
250 1.17741265626137e-07
251 1.18530678605566e-07
252 1.19173748203139e-07
253 1.17770653673688e-07
254 1.19988811775329e-07
255 1.19268577236653e-07
256 1.18849797559051e-07
257 1.19493620331923e-07
258 1.18801331439045e-07
259 1.18799562187633e-07
260 1.19135364684553e-07
261 1.1869765614847e-07
262 1.18605946397565e-07
263 1.18526038761502e-07
264 1.14487512803407e-07
265 1.18220199851748e-07
266 1.14192275191272e-07
267 1.18037931429171e-07
268 1.17788346187808e-07
269 1.19944587595455e-07
270 1.20240102319258e-07
271 1.19267610898532e-07
272 1.15533779876387e-07
273 1.19424427680315e-07
274 1.19194702108416e-07
275 1.18980956642645e-07
276 1.20147518600788e-07
277 1.14838762499403e-07
278 1.19593934755358e-07
279 1.20266832936977e-07
280 1.20127708669315e-07
281 1.20000791525854e-07
282 1.19679796739547e-07
283 1.20037114470506e-07
284 1.14149614205417e-07
285 1.18978896068711e-07
286 1.18984630148589e-07
287 1.18598102005762e-07
288 1.18740551613428e-07
289 1.18144342309279e-07
290 1.17487836348573e-07
291 1.17213332373467e-07
292 1.17074918648541e-07
293 1.16905411573498e-07
294 1.16986889509008e-07
295 1.1659606258263e-07
296 1.16269333716446e-07
297 1.1612704753361e-07
298 1.15950548718047e-07
299 1.1587016501835e-07
300 1.15677117662472e-07
301 1.15549347867727e-07
302 1.15549525503411e-07
303 1.15186040261506e-07
304 1.14852475974203e-07
305 1.14825830621612e-07
306 1.14391397687541e-07
307 1.14147191254688e-07
308 1.14058536837547e-07
309 1.13623251252193e-07
310 1.13547031332928e-07
311 1.13179837057942e-07
312 1.12937115659406e-07
313 1.12716577405081e-07
314 1.12794978690545e-07
315 1.12299254340087e-07
316 1.12139090902019e-07
317 1.13729143436103e-07
318 1.13833252157747e-07
319 1.13364258425008e-07
320 1.13030338866338e-07
321 1.13129217993446e-07
322 1.13085683040026e-07
323 1.12994456458182e-07
324 1.08607451920761e-07
325 1.12799689588883e-07
326 1.0760257396214e-07
327 1.11872978436622e-07
328 1.12384626049788e-07
329 1.12328386592253e-07
330 1.12159831644476e-07
331 1.07091750578547e-07
332 1.06787474862813e-07
333 1.06784909803537e-07
334 1.06598960769588e-07
335 1.06472150207537e-07
336 1.06470061211894e-07
337 1.06387943787922e-07
338 1.06337111560606e-07
339 1.05991148302564e-07
340 1.07576838104251e-07
341 1.07502472701526e-07
342 1.07069489274636e-07
343 1.09604293641041e-07
344 1.0671665506834e-07
345 1.06131167854073e-07
346 1.06385563469757e-07
347 1.05770602942812e-07
348 1.06241543562646e-07
349 1.0922737914143e-07
350 1.0870380151573e-07
351 1.07865751886038e-07
352 1.05006712658451e-07
353 1.03751197855217e-07
354 1.04487099861217e-07
355 1.03388437366903e-07
356 1.02798971113316e-07
357 1.04448361071263e-07
358 1.01764229043511e-07
359 1.04289689772941e-07
360 1.04457988925333e-07
361 1.03943349927249e-07
362 1.03804900675186e-07
363 1.03843085241806e-07
364 1.03487323599438e-07
365 1.01931078688722e-07
366 1.01650876160875e-07
367 1.03588334354754e-07
368 1.01613352398999e-07
369 1.01771441052279e-07
370 1.03501484716162e-07
371 1.01533792928876e-07
372 1.01578471856101e-07
373 1.01238903482681e-07
374 1.01400843277588e-07
375 1.01513975891976e-07
376 1.0112045600863e-07
377 1.01355489334765e-07
378 1.00763251964509e-07
379 1.01136016894543e-07
380 1.01045017686374e-07
381 1.00930336088823e-07
382 1.00851792694812e-07
383 1.00496997390565e-07
384 1.02995372230907e-07
385 1.01201507618498e-07
386 1.02915826971639e-07
387 1.00982170181396e-07
388 1.03150043173628e-07
389 1.00834220972956e-07
390 1.02990313166629e-07
391 1.03705609433291e-07
392 1.02971867477208e-07
393 1.03658344130508e-07
394 1.04371146392168e-07
395 1.0486591861536e-07
396 1.04924275490248e-07
397 1.05225140600851e-07
398 1.05328190613818e-07
399 1.05202616396127e-07
400 1.05285387519416e-07
401 1.05138042272301e-07
402 1.04939729794751e-07
403 1.05154427387788e-07
404 1.05399195149403e-07
405 1.04754576568666e-07
406 1.05041486619939e-07
407 1.05028760799541e-07
408 1.05401532835003e-07
409 1.05528449978465e-07
410 1.0555051233041e-07
411 1.05342607525927e-07
412 1.05487522716885e-07
413 1.0554989415823e-07
414 1.05503701774978e-07
415 1.05631606572842e-07
416 1.05276775741459e-07
417 1.05333000988139e-07
418 1.04984167137445e-07
419 1.04862401428818e-07
420 1.05170578024172e-07
421 1.04848929538548e-07
422 1.04698116842883e-07
423 1.04902895259329e-07
424 1.04888023599869e-07
425 1.04587947191703e-07
426 1.0439251951766e-07
427 1.04602861483727e-07
428 1.0429197772055e-07
429 1.04162090508453e-07
430 1.04363458319767e-07
431 1.04307709136719e-07
432 1.04009920676162e-07
433 1.03906906190332e-07
434 1.04070075224172e-07
435 1.03994281630548e-07
436 1.03682467056387e-07
437 1.03905414050587e-07
438 1.03752363145304e-07
439 1.03411402108122e-07
440 1.03304508058955e-07
441 1.03446701871235e-07
442 1.03462966194456e-07
443 1.03322825850682e-07
444 1.02926406952975e-07
445 1.03157958619704e-07
446 1.02674036384087e-07
447 1.0260605165513e-07
448 1.02863459972014e-07
449 1.02419505765283e-07
450 1.02279749114587e-07
451 1.02551027225672e-07
452 1.02308717941924e-07
453 1.02199145146642e-07
454 1.02067140517192e-07
455 1.01954420017591e-07
456 1.02129916967897e-07
457 1.01858731227367e-07
458 1.01798931950725e-07
459 1.01529181506521e-07
460 1.0141183537371e-07
461 1.01421910869703e-07
462 1.01255253071031e-07
463 1.01525301943184e-07
464 1.01017491260791e-07
465 1.00908359002005e-07
466 1.01152821230244e-07
467 1.00758548171598e-07
468 1.00697491234314e-07
469 1.00177679485114e-07
470 1.00244875511635e-07
471 1.0036672648539e-07
472 9.99526719169808e-08
473 9.97381448542001e-08
474 9.98533238316668e-08
475 9.94217828065302e-08
476 9.93969635487701e-08
477 9.90936044331647e-08
478 9.93268827187421e-08
479 9.88238042509693e-08
480 9.87966117804717e-08
481 9.88584147876281e-08
482 9.87693837828374e-08
483 9.8247795676798e-08
484 9.82305934371652e-08
485 9.82093411039386e-08
486 9.81403047717322e-08
487 9.81784253895057e-08
488 9.80150574036998e-08
489 9.79990915084272e-08
490 9.76317480194666e-08
491 9.72916964769865e-08
492 9.70429425706243e-08
493 9.67480602298565e-08
494 9.66250226497323e-08
495 9.58760821845317e-08
496 9.59879500328498e-08
497 9.56344550218091e-08
498 9.54208374537302e-08
499 9.5384095288864e-08
500 9.49067455735531e-08
501 9.45246725336801e-08
502 9.44734424024318e-08
503 9.41607467552785e-08
504 9.39376008091131e-08
505 9.35443225102972e-08
506 9.35380839450772e-08
507 9.30811481225646e-08
508 9.29424714968263e-08
509 9.27417644902562e-08
510 9.26090066855068e-08
511 9.2232362192135e-08
512 9.21860490166182e-08
513 9.1742016650187e-08
514 9.16621942792517e-08
515 9.12209472403447e-08
516 9.11547459736539e-08
517 9.08831552237643e-08
518 9.06421249169398e-08
519 9.02852335116222e-08
520 9.00592169728043e-08
521 8.96739607014752e-08
522 8.95849936455306e-08
523 8.85301716380127e-08
524 8.79042545420816e-08
525 8.74042456189272e-08
526 8.68623786232092e-08
527 8.6453077585702e-08
528 8.59852278267681e-08
529 8.56297859286315e-08
530 8.54134185601652e-08
531 8.5114912451445e-08
532 8.48548609155841e-08
533 8.46577918878211e-08
534 8.43625329594033e-08
535 8.42287022351229e-08
536 8.38576070805175e-08
537 8.35639752949646e-08
538 8.33721429671641e-08
539 8.30898585491013e-08
540 8.28285777743076e-08
541 8.25603621024129e-08
542 8.24036590074684e-08
543 8.2166764059366e-08
544 8.19221241954438e-08
545 8.18092615872956e-08
546 8.16151910498775e-08
547 8.1467788959344e-08
548 8.13037743796485e-08
549 8.1058651346666e-08
550 8.08897695492306e-08
551 8.07120557055896e-08
552 8.04978412816126e-08
553 8.03225788104101e-08
554 8.01561625962677e-08
555 7.99150257080328e-08
556 7.97759867054992e-08
557 7.96150416704222e-08
558 7.94882168975164e-08
559 7.93016141642511e-08
560 7.92342405020463e-08
561 7.89425342873074e-08
562 7.87247103062327e-08
563 7.85286715654365e-08
564 7.82979654445626e-08
565 7.81417313078236e-08
566 7.80132509703435e-08
567 7.78701050307973e-08
568 7.78319133587502e-08
569 7.77327358036928e-08
570 7.75710162770338e-08
571 7.75132491526165e-08
572 7.74763080357843e-08
573 7.73246995322552e-08
574 7.71558248402471e-08
575 7.71228769735899e-08
576 7.6944687066316e-08
577 7.88885898828084e-08
578 7.676747060259e-08
579 7.66957981568339e-08
580 7.6446852403933e-08
581 7.84584486268614e-08
582 7.63758478683485e-08
583 7.66023902087909e-08
584 7.60338849659092e-08
585 7.59606351152797e-08
586 7.58670637424075e-08
587 7.58583951210312e-08
588 7.55152598230779e-08
589 7.54758175958159e-08
590 7.72376580471246e-08
591 7.52314761598427e-08
592 7.50770681179347e-08
593 7.69018768664864e-08
594 7.49164854596529e-08
595 7.47489750096975e-08
596 7.46609885027283e-08
597 7.46051469491249e-08
598 7.46933750406242e-08
599 7.45102539667641e-08
600 7.45100336985161e-08
601 7.45339505670017e-08
602 7.42182351132215e-08
603 7.40738599347424e-08
604 7.41384624802777e-08
605 7.38482981432753e-08
606 7.36935632517088e-08
607 7.35562082354591e-08
608 7.32787555079994e-08
609 7.34205798380572e-08
610 7.53503499595354e-08
611 7.55232818505647e-08
612 7.39071381872236e-08
613 7.32925897750647e-08
614 7.3214408757849e-08
615 7.25872126849936e-08
616 7.23880120290232e-08
617 7.23721527151611e-08
618 7.24842195154451e-08
619 7.43671009217906e-08
620 7.24094704196432e-08
621 7.26592901401091e-08
622 7.23894117982127e-08
623 7.23890067888533e-08
624 7.20789259389676e-08
625 7.17407218076005e-08
626 7.16592936100824e-08
627 7.14588921368886e-08
628 7.30073494992212e-08
629 7.14420451686237e-08
630 7.32847880158261e-08
631 7.14086354491883e-08
632 7.13943819619089e-08
633 7.10237273437997e-08
634 7.13783805394996e-08
635 7.1014056857166e-08
636 7.10083227772884e-08
637 7.06136304984284e-08
638 7.0634108340073e-08
639 7.05164779901679e-08
640 7.26697564346068e-08
641 7.12777250555519e-08
642 7.14936163603852e-08
643 7.11316019419428e-08
644 7.14737495854934e-08
645 7.16855836913055e-08
646 7.14284169589519e-08
647 7.15611960799833e-08
648 7.14551759983806e-08
649 7.14000378820856e-08
650 7.10787233515475e-08
651 7.13208976321766e-08
652 7.10229954847819e-08
653 7.10959682237444e-08
654 7.14005707891374e-08
655 7.09251466446403e-08
656 7.06419100993116e-08
657 7.03898876963649e-08
658 7.07353393636367e-08
659 7.06103193692798e-08
660 7.05232920950039e-08
661 6.98944617738562e-08
662 7.04668323692204e-08
663 7.05790128563422e-08
664 7.06834484276442e-08
665 7.05050595684042e-08
666 7.05541651768726e-08
667 7.04250950889218e-08
668 7.02627289683733e-08
669 7.01283582316137e-08
670 7.01519766721503e-08
671 7.05615761376066e-08
672 7.07046510228793e-08
673 7.05230718267558e-08
674 7.05627627439753e-08
675 7.01560338711715e-08
676 6.93522821393344e-08
677 7.00401159292596e-08
678 6.97461999266125e-08
679 6.9465372121158e-08
680 6.95746464884905e-08
681 6.92048232053821e-08
682 6.92426098680698e-08
683 6.90973678274531e-08
684 6.77442812957452e-08
685 6.89146801846618e-08
686 6.88692196604279e-08
687 6.87519161601813e-08
688 6.80426452959182e-08
689 6.83605207996152e-08
690 6.73724613875493e-08
691 6.76151259426661e-08
692 6.75478446510169e-08
693 6.79449598806059e-08
694 6.79809488701721e-08
695 6.79992240293359e-08
696 6.76119000786457e-08
697 6.76197657867306e-08
698 6.72100171072998e-08
699 6.75242901593265e-08
700 6.71783411121396e-08
701 6.68667610170814e-08
702 6.68732482722589e-08
703 6.6634335382787e-08
704 6.73412543505947e-08
705 6.67744686211336e-08
706 6.69261979169278e-08
707 6.66587851583245e-08
708 6.64841692810114e-08
709 6.67055743974743e-08
710 6.66235209223487e-08
711 6.6017541655583e-08
712 6.65357759999097e-08
713 6.6384295394073e-08
714 6.61761845321962e-08
715 6.59351684362264e-08
716 6.54304201930245e-08
717 6.55618137557212e-08
718 6.52752802920986e-08
719 6.58806627029662e-08
720 6.59247874068569e-08
721 6.58668355413283e-08
722 6.55527330195582e-08
723 6.58132393027699e-08
724 6.56063150472619e-08
725 6.58412204757042e-08
726 6.5727761011658e-08
727 6.5294997853016e-08
728 6.55633130008937e-08
729 6.54662670740436e-08
730 6.58570868949937e-08
731 6.54659402243851e-08
732 6.51337828116993e-08
733 6.53316476473265e-08
734 6.50514735411889e-08
735 6.49165770028048e-08
736 6.49355271775676e-08
737 6.48185149998426e-08
738 6.489074877436e-08
739 6.49063807145467e-08
740 6.47615578941441e-08
741 6.50024603032762e-08
742 6.51157776587752e-08
743 6.49462563728775e-08
744 6.51906333359875e-08
745 6.48567350935991e-08
746 6.4962300427851e-08
747 6.51312745958421e-08
748 6.49429523491563e-08
749 6.50068940899473e-08
750 6.48487343823945e-08
751 6.48873594855104e-08
752 6.47778080065109e-08
753 6.48152749249675e-08
754 6.46486455480044e-08
755 6.44023359086532e-08
756 6.45145661337665e-08
757 6.43885940121436e-08
758 6.45287911993364e-08
759 6.44643876057671e-08
760 6.42051034560609e-08
761 6.40454658196177e-08
762 6.39795274537391e-08
763 6.41688089331183e-08
764 6.37989430174457e-08
765 6.37417940652085e-08
766 6.36800834286078e-08
767 6.38875192748856e-08
768 6.37947721315868e-08
769 6.39605914898311e-08
770 6.34628491980038e-08
771 6.37096277955607e-08
772 6.34608028349248e-08
773 6.34789216746867e-08
774 6.35109742574969e-08
775 6.36256203279117e-08
776 6.39157491377773e-08
777 6.37263681824152e-08
778 6.37933155189785e-08
779 6.34032701896103e-08
780 6.38183337287046e-08
781 6.33494892099407e-08
782 6.34290913126279e-08
783 6.36563726175154e-08
784 6.31244745363801e-08
785 6.34060981496987e-08
786 6.330985513614e-08
787 6.35672847693058e-08
788 6.30494909614754e-08
789 6.27267482400384e-08
790 6.29013996444883e-08
791 6.29417229447427e-08
792 6.26432523631593e-08
793 6.27155785082323e-08
794 6.28520240297803e-08
795 6.28954808234994e-08
796 6.29754239866998e-08
797 6.29151770681347e-08
798 6.30115266631037e-08
799 6.28347578413013e-08
800 6.24609199917359e-08
801 6.30277341429064e-08
802 6.30873131512999e-08
803 6.2941531098204e-08
804 6.29821883535442e-08
805 6.31097734071773e-08
806 6.22237124048297e-08
807 6.09556067843187e-08
808 6.25309368729177e-08
809 6.19277926716677e-08
810 6.22195841515349e-08
811 6.19806641566356e-08
812 6.19251068201265e-08
813 6.21844407078243e-08
814 6.2079166696094e-08
815 6.20860873823403e-08
816 6.20130293782495e-08
817 6.22409146444625e-08
818 6.2059157812655e-08
819 6.23436591240534e-08
820 6.20045099708477e-08
821 6.20254283489885e-08
822 6.19300593029948e-08
823 6.20703772824527e-08
824 6.17072544173425e-08
825 6.18381150729874e-08
826 6.18941484731295e-08
827 6.18202449231831e-08
828 6.18000726149148e-08
829 6.15933473113728e-08
830 6.1619559232895e-08
831 6.16309137058124e-08
832 6.15554611727021e-08
833 6.1844048104831e-08
834 6.15085795629966e-08
835 6.14752053706979e-08
836 6.14745800930905e-08
837 6.13117521197637e-08
838 6.13909492130915e-08
839 6.11444761489111e-08
840 6.11138730732819e-08
841 6.14495903050738e-08
842 6.14626785022665e-08
843 6.11874071410057e-08
844 6.11116561799463e-08
845 6.08862649187358e-08
846 6.13708550645242e-08
847 6.08247745503832e-08
848 6.10631332165212e-08
849 6.08224794973466e-08
850 6.10275847634512e-08
851 6.08650694289281e-08
852 6.13691781836678e-08
853 6.09854282629385e-08
854 6.11721233667595e-08
855 6.08894055176279e-08
856 6.09218844260795e-08
857 6.09158519182529e-08
858 6.10265757927664e-08
859 6.06710628403562e-08
860 6.11091621749438e-08
861 6.06682704074046e-08
862 6.06680075065924e-08
863 6.08496861786989e-08
864 6.07812467023905e-08
865 6.05264034447828e-08
866 6.07337469205049e-08
867 6.05064727210447e-08
868 6.05991772317793e-08
869 6.08094197218634e-08
870 6.01855774107207e-08
871 6.02850676045819e-08
872 6.02982765940396e-08
873 6.00952176910141e-08
874 6.02134520022446e-08
875 5.98983405097897e-08
876 5.98406941776375e-08
877 5.98010245767e-08
878 5.95034777006731e-08
879 5.91101887437162e-08
880 5.92548268230075e-08
881 5.91134856620101e-08
882 5.92683235822733e-08
883 5.93401878745681e-08
884 5.90452806648045e-08
885 5.91059361454427e-08
886 5.91294870844195e-08
887 5.90851882975585e-08
888 5.91557451912195e-08
889 5.90886415352543e-08
890 5.89134998563168e-08
891 5.89662363381649e-08
892 5.87985233835298e-08
893 5.89049165000688e-08
894 5.86348889441979e-08
895 5.87657567052702e-08
896 5.87748161251511e-08
897 5.84789887625448e-08
898 5.85243959960735e-08
899 5.82978785246269e-08
900 5.8343601949673e-08
901 5.82662522674582e-08
902 5.83525867625667e-08
903 5.84398627268001e-08
904 5.81841668179095e-08
905 5.81659591603056e-08
906 5.80761430057919e-08
907 5.80786370107944e-08
908 5.75842662442483e-08
909 5.7528058761136e-08
910 5.73972585016236e-08
911 5.73828486949424e-08
912 5.71303289120806e-08
913 5.72532528053671e-08
914 5.70601308425012e-08
915 5.71191129949966e-08
916 5.7108987761012e-08
917 5.69870373112735e-08
918 5.70551215162141e-08
919 5.69466216404635e-08
920 5.69641329661863e-08
921 5.69079112722193e-08
922 5.67837581399999e-08
923 5.678650083496e-08
924 5.66967912618566e-08
925 5.65424436160811e-08
926 5.67454065958373e-08
927 5.63693021149447e-08
928 5.63078472737288e-08
929 5.6271588277923e-08
930 5.61950201927175e-08
931 5.62325581654477e-08
932 5.6348294918962e-08
933 5.61508279872669e-08
934 5.59964234980725e-08
935 5.58999566635521e-08
936 5.59573507530331e-08
937 5.59046817727449e-08
938 5.60009887351498e-08
939 5.60081687694947e-08
940 5.58381785253914e-08
941 5.56867973955377e-08
942 5.56464421208602e-08
943 5.54065948676907e-08
944 5.54206955882819e-08
945 5.53334658093263e-08
946 5.5447848978929e-08
947 5.50879697414075e-08
948 5.45779563765336e-08
949 5.50567662571666e-08
950 5.50407044386247e-08
951 5.53670851388688e-08
952 5.47605907286197e-08
953 5.49000205296579e-08
954 5.49658736304082e-08
955 5.47715224286094e-08
956 5.47373524284467e-08
957 5.45250564698563e-08
958 5.46986349547751e-08
959 5.44448255368479e-08
960 5.44043281536233e-08
961 5.41712523727256e-08
962 5.44440084127018e-08
963 5.39916769071169e-08
964 5.40723767983309e-08
965 5.37717887993949e-08
966 5.39072289029718e-08
967 5.40463425124926e-08
968 5.37043050030661e-08
969 5.36778337334454e-08
970 5.35614361751868e-08
971 5.33372173094904e-08
972 5.34621911185695e-08
973 5.31617487808944e-08
974 5.34201944901724e-08
975 5.38267208582965e-08
976 5.36612283497107e-08
977 5.33368158528447e-08
978 5.41805071918589e-08
979 5.4068127752771e-08
980 5.35162278936241e-08
981 5.5044981905894e-08
982 5.46738547768655e-08
983 5.39281934663904e-08
984 5.33337392027988e-08
985 5.35102486765027e-08
986 5.2986241172448e-08
987 5.42522684554569e-08
988 5.32367145922308e-08
989 5.28020756007663e-08
990 5.24483247943408e-08
991 5.28344017425297e-08
992 5.37873638961628e-08
993 5.26645678178284e-08
994 5.23986187772607e-08
995 5.22394003610316e-08
996 5.22211571762909e-08
997 5.217404819291e-08
998 5.33851078898806e-08
999 5.23216243664137e-08
1000 5.18363876267358e-08
1001 5.19911047547339e-08
1002 5.13039459804077e-08
1003 5.13991409434311e-08
1004 5.11438145167631e-08
1005 5.11511402123688e-08
1006 5.19914777896702e-08
1007 5.15636671138964e-08
1008 5.09770039514024e-08
1009 5.1509793763671e-08
1010 5.12252249507128e-08
1011 5.08427717704762e-08
1012 5.06435426927965e-08
1013 5.05292945263136e-08
1014 5.06675590372652e-08
1015 5.04658039801598e-08
1016 5.01824857224165e-08
1017 5.10123427943654e-08
1018 5.06079267381665e-08
1019 5.11299589334158e-08
1020 5.00008248138784e-08
1021 4.95664984612176e-08
1022 4.93094702846975e-08
1023 4.95516871978907e-08
1024 4.95037788539321e-08
1025 4.93005671842184e-08
1026 4.94162186726044e-08
1027 4.90187339607928e-08
1028 4.93410681201567e-08
1029 4.89876548215307e-08
1030 4.86437130575723e-08
1031 4.91014944259405e-08
1032 4.93515273092271e-08
1033 4.88652673880097e-08
1034 4.97311631875164e-08
1035 4.89030291817016e-08
1036 4.8588965739782e-08
1037 4.85690918594628e-08
1038 4.83279372076595e-08
1039 4.81028052945476e-08
1040 4.81087418791049e-08
1041 4.83390500960468e-08
1042 4.78880295418094e-08
1043 4.7867725783135e-08
1044 4.80396629143343e-08
1045 4.75405741440227e-08
1046 4.77755648375933e-08
1047 4.79346802251257e-08
1048 4.7569660210911e-08
1049 4.72118308891822e-08
1050 4.71920102995682e-08
1051 4.70640451055715e-08
1052 4.70857202117259e-08
1053 4.81053916701057e-08
1054 4.72714347665715e-08
1055 4.69471110875475e-08
1056 4.66328451409481e-08
1057 4.66688305778007e-08
1058 4.67580640872711e-08
1059 4.69491112653486e-08
1060 4.70188332712951e-08
1061 4.67787657498775e-08
1062 4.64097418273468e-08
1063 4.61082656499912e-08
1064 4.62073472817792e-08
1065 4.60139837343831e-08
1066 4.67401335413342e-08
1067 4.6427548028305e-08
1068 4.60419862235995e-08
1069 4.60152200787434e-08
1070 4.5810843118943e-08
1071 4.565069389173e-08
1072 4.66679921373725e-08
1073 4.56663116210621e-08
1074 4.53839845704351e-08
1075 4.42569856318187e-08
1076 4.49592363338525e-08
1077 4.46868710923809e-08
1078 4.46782415508551e-08
1079 4.46825652034022e-08
1080 4.45299228601925e-08
1081 4.50826966869045e-08
1082 4.47639401102151e-08
1083 4.37085070359444e-08
1084 4.46256009922763e-08
1085 4.38140261849185e-08
1086 4.32104876324502e-08
1087 4.38330438612411e-08
1088 4.36373319701033e-08
1089 4.33466240679081e-08
1090 4.38319887052785e-08
1091 4.32192379662411e-08
1092 4.25785451341198e-08
1093 4.2535152289247e-08
1094 4.27743280795312e-08
1095 4.19634282877723e-08
1096 4.20127221900657e-08
1097 4.1746563539391e-08
1098 4.19298977760718e-08
1099 4.15440908341225e-08
1100 4.12482492606614e-08
1101 4.12138021488317e-08
1102 4.09838918358219e-08
1103 4.11797635990752e-08
1104 4.08260873996369e-08
1105 4.04628153205522e-08
1106 4.07295566162702e-08
1107 4.08402911489247e-08
1108 4.05751023890843e-08
1109 4.03213356037213e-08
1110 4.03954807381979e-08
1111 4.01010176176442e-08
1112 4.03537399051856e-08
1113 4.01690662954479e-08
1114 4.00114466003743e-08
1115 4.0326714412231e-08
1116 4.04352498151184e-08
1117 4.0083410368652e-08
1118 4.01510433789554e-08
1119 3.93020798128418e-08
1120 3.91458812032397e-08
1121 3.97365695903318e-08
1122 3.90955854356889e-08
1123 3.94907821998913e-08
1124 3.93192394199104e-08
1125 3.95392163454744e-08
1126 3.90773848835124e-08
1127 3.92044654518031e-08
1128 3.90704855135482e-08
1129 3.9338722501725e-08
1130 3.88163350351078e-08
1131 3.88256644612284e-08
1132 3.89981700266162e-08
1133 3.83832130523842e-08
1134 3.85010530123964e-08
1135 3.86369620741789e-08
1136 3.88502066073215e-08
1137 3.88346492741221e-08
1138 3.82146225774704e-08
1139 3.8295304705116e-08
1140 3.86786354056312e-08
1141 3.82233409368382e-08
1142 3.90628294155704e-08
1143 3.84632947714181e-08
1144 3.86317751122078e-08
1145 3.88306666820881e-08
1146 3.86831047194391e-08
1147 3.86461884716027e-08
1148 3.84165126376956e-08
1149 3.79532423266937e-08
1150 3.78046678406463e-08
1151 3.76981930116926e-08
1152 3.74290998195193e-08
1153 3.75906985539132e-08
1154 3.74964415073009e-08
1155 3.73628346039823e-08
1156 3.73820085997068e-08
1157 3.77303059906353e-08
1158 3.74031579042367e-08
1159 3.75770987659507e-08
1160 3.76600262086413e-08
1161 3.76212021535594e-08
1162 3.80562994450884e-08
1163 3.77246642813134e-08
1164 3.72634652023862e-08
1165 3.75629696236501e-08
1166 3.74292881133442e-08
1167 3.69734785010678e-08
1168 3.73452415658448e-08
1169 3.63103929146291e-08
1170 3.67433585779509e-08
1171 3.68291708241486e-08
1172 3.68818753315736e-08
1173 3.65444421390748e-08
1174 3.68130130823374e-08
1175 3.65101975319249e-08
1176 3.66712598065533e-08
1177 3.67779939836055e-08
1178 3.66174184307511e-08
1179 3.64742618330638e-08
1180 3.62899257311255e-08
1181 3.6493560173767e-08
1182 3.63223655597267e-08
1183 3.59546739048255e-08
1184 3.62275365262121e-08
1185 3.61630014822367e-08
1186 3.6120436419651e-08
1187 3.59875187427861e-08
1188 3.6180640705652e-08
1189 3.58481990758719e-08
1190 3.6112517420861e-08
1191 3.58723610816014e-08
1192 3.63272221193256e-08
1193 3.61760434941516e-08
1194 3.54852573991593e-08
1195 3.5869412329248e-08
1196 3.55233495952234e-08
1197 3.56695863956702e-08
1198 3.56885507812876e-08
1199 3.56177238813871e-08
1200 3.58859502114228e-08
1201 3.56042164639803e-08
1202 3.5362010208928e-08
1203 3.5551327215444e-08
1204 3.55587346234643e-08
1205 3.5401232167942e-08
1206 3.55442466570821e-08
1207 3.55880054314639e-08
1208 3.53164857358479e-08
1209 3.528120018359e-08
1210 3.54563418625276e-08
1211 3.51970150802572e-08
1212 3.51653177688149e-08
1213 3.50615216859751e-08
1214 3.52058293628943e-08
1215 3.54116487244482e-08
1216 3.5153725264081e-08
1217 3.53217508575199e-08
1218 3.54338141050903e-08
1219 3.50761801826138e-08
1220 3.51901761064255e-08
1221 3.50295650264343e-08
1222 3.51787541319482e-08
1223 3.50207649546519e-08
1224 3.52917943757802e-08
1225 3.52742866027711e-08
1226 3.53263587271613e-08
1227 3.49399833510233e-08
1228 3.49836390967084e-08
1229 3.47092736774357e-08
1230 3.48305100317248e-08
1231 3.46589068556113e-08
1232 3.47250299626012e-08
1233 3.46051685085058e-08
1234 3.47531212696595e-08
1235 3.45469572948787e-08
1236 3.45955015745858e-08
1237 3.45476216523366e-08
1238 3.44746133862373e-08
1239 3.44760415771361e-08
1240 3.43169972438773e-08
1241 3.45315527283674e-08
1242 3.43490178522643e-08
1243 3.43222126275577e-08
1244 3.43496431298718e-08
1245 3.42955814858215e-08
1246 3.42500889871644e-08
1247 3.45132349366395e-08
1248 3.42354447013804e-08
1249 3.42308545953074e-08
1250 3.4251232960969e-08
1251 3.42260797481231e-08
1252 3.44292843124094e-08
1253 3.41410242299389e-08
1254 3.41298971306969e-08
1255 3.43415287318294e-08
1256 3.40861454617425e-08
1257 3.43450068385209e-08
1258 3.4766056700164e-08
1259 3.402421100418e-08
1260 3.40002266341344e-08
1261 3.39330199494725e-08
1262 3.39683303707261e-08
1263 3.40177344071435e-08
1264 3.39433867679872e-08
1265 3.39688135397864e-08
1266 3.39072556698738e-08
1267 3.39168444440929e-08
1268 3.38827561563448e-08
1269 3.3881228489463e-08
1270 3.39257155701489e-08
1271 3.39340147093026e-08
1272 3.39141763561202e-08
1273 3.38723609161207e-08
1274 3.41156365379902e-08
1275 3.3910705354856e-08
1276 3.38107852826397e-08
1277 3.37353789348072e-08
1278 3.38510304231932e-08
1279 3.39174697217004e-08
1280 3.37101973002518e-08
1281 3.36362191433182e-08
1282 3.37993064647435e-08
1283 3.37826122631668e-08
1284 3.3907507912545e-08
1285 3.3833224222235e-08
1286 3.38680621325693e-08
1287 3.36769439002182e-08
1288 3.37924603854844e-08
1289 3.36230989717023e-08
1290 3.41016850313736e-08
1291 3.37673426997753e-08
1292 3.37719185949936e-08
1293 3.39981340857776e-08
1294 3.37242127557147e-08
1295 3.37207772815873e-08
1296 3.37617578338723e-08
1297 3.3977578084432e-08
1298 3.39234205171124e-08
1299 3.35590577549283e-08
1300 3.35321566069524e-08
1301 3.36843086756744e-08
1302 3.38262324817151e-08
1303 3.38090551110781e-08
1304 3.34666161450059e-08
1305 3.35575727206106e-08
1306 3.35027046105552e-08
1307 3.3638222873833e-08
1308 3.3849914871098e-08
1309 3.3976107260969e-08
1310 3.39217400835423e-08
1311 3.38151977530288e-08
1312 3.38507746278083e-08
1313 3.37054757437727e-08
1314 3.3905241281218e-08
1315 3.37733645494609e-08
1316 3.37946808315337e-08
1317 3.35671117568381e-08
1318 3.36358461083819e-08
1319 3.35615411017898e-08
1320 3.36008660895004e-08
1321 3.36360983510531e-08
1322 3.36912755471985e-08
1323 3.33426157794747e-08
1324 3.35990648636653e-08
1325 3.36381091869953e-08
1326 3.34312346694787e-08
1327 3.36888028584781e-08
1328 3.37417951357111e-08
1329 3.34899894482987e-08
1330 3.3500814566878e-08
1331 3.36566543523986e-08
1332 3.36798571254349e-08
1333 3.36907000075826e-08
1334 3.36247936161271e-08
1335 3.37254739690707e-08
1336 3.36526646549373e-08
1337 3.3692785450512e-08
1338 3.34364678167276e-08
1339 3.36328866978874e-08
1340 3.40319381564314e-08
1341 3.34950840397141e-08
1342 3.34339098628789e-08
1343 3.35002745543989e-08
1344 3.33441327882156e-08
1345 3.3104551988572e-08
1346 3.30662537351145e-08
1347 3.31242659967756e-08
1348 3.31304264022947e-08
1349 3.31176721601878e-08
1350 3.30311884511048e-08
1351 3.30780984825196e-08
1352 3.32954179782519e-08
1353 3.33058522983265e-08
1354 3.3173549240928e-08
1355 3.32712062345308e-08
1356 3.32123626378689e-08
1357 3.28838396512765e-08
1358 3.31865663838471e-08
1359 3.34707443983007e-08
1360 3.28140430383428e-08
1361 3.29218643457807e-08
1362 3.28714229169691e-08
1363 3.28914957492543e-08
1364 3.29348743832725e-08
1365 3.27506590735993e-08
1366 3.28661506898698e-08
1367 3.27858167281647e-08
1368 3.3066879012722e-08
1369 3.31057030678039e-08
1370 3.27721849657792e-08
1371 3.27728422178097e-08
1372 3.27430136337625e-08
1373 3.27886624518214e-08
1374 3.27218430129506e-08
1375 3.27568301372594e-08
1376 3.29375495766726e-08
1377 3.29270442023244e-08
1378 3.34253513756266e-08
1379 3.33685470366163e-08
1380 3.32960290450046e-08
1381 3.31097034234062e-08
1382 3.25794893285547e-08
1383 3.25357909503055e-08
1384 3.25143680868223e-08
1385 3.25510569609833e-08
1386 3.24664704010047e-08
1387 3.24090336789595e-08
1388 3.25286535485247e-08
1389 3.2412280859262e-08
1390 3.23502256094343e-08
1391 3.23304156779614e-08
1392 3.22816369191514e-08
1393 3.22778213046604e-08
1394 3.22363469251741e-08
1395 3.18635535734302e-08
1396 3.16635677677368e-08
1397 3.15772652470514e-08
1398 3.21226956145892e-08
1399 3.19565955919643e-08
1400 3.08370466939323e-08
1401 3.06547001116542e-08
1402 3.07130072485506e-08
1403 3.05424627811135e-08
1404 3.04559932828852e-08
1405 3.0140807183443e-08
1406 2.99422495686485e-08
1407 2.97705842200457e-08
1408 2.96809492539296e-08
1409 2.97171087737524e-08
1410 2.97359754597437e-08
1411 2.94656210542144e-08
1412 2.93694153441493e-08
1413 2.93682873575563e-08
1414 2.93257134131863e-08
1415 2.91836261823164e-08
1416 2.94766362429755e-08
1417 2.9273532931029e-08
1418 2.92173094607051e-08
1419 2.9165457604563e-08
1420 2.89672197339996e-08
1421 2.9140625912305e-08
1422 2.88393113834218e-08
1423 2.87430008683032e-08
1424 2.87162542633723e-08
1425 2.88905486200974e-08
1426 2.86125114712377e-08
1427 2.85488237494746e-08
1428 2.84466263877903e-08
1429 2.84608230316508e-08
1430 2.85410290956634e-08
1431 2.84073262690754e-08
1432 2.82797749662222e-08
1433 2.83557692881686e-08
1434 2.80849228317948e-08
1435 2.81335346130618e-08
1436 2.80310086253621e-08
1437 2.82716072774747e-08
1438 2.7820744818996e-08
1439 2.79602225816689e-08
1440 2.78354868044062e-08
1441 2.77875216170287e-08
1442 2.77120211222837e-08
1443 2.76598122184168e-08
1444 2.77865925824017e-08
1445 2.76249654262983e-08
1446 2.77910263690728e-08
1447 2.75467346710911e-08
1448 2.77907759027585e-08
1449 2.74981850623135e-08
1450 2.78960179400656e-08
1451 2.75689835405046e-08
1452 2.73127884753421e-08
1453 2.73681344253873e-08
1454 2.72660614086817e-08
1455 2.76329945592124e-08
1456 2.7390049339715e-08
1457 2.75883920153319e-08
1458 2.72765809938846e-08
1459 2.73360676317225e-08
1460 2.76030167611907e-08
1461 2.73719624743762e-08
1462 2.75650329228938e-08
1463 2.73792615246293e-08
1464 2.73202935829886e-08
1465 2.73258571326096e-08
1466 2.7312502481891e-08
1467 2.72636899723011e-08
1468 2.73532183570069e-08
1469 2.72219402575047e-08
1470 2.74996558857765e-08
1471 2.75057630005904e-08
1472 2.74324598592557e-08
1473 2.70904170207587e-08
1474 2.74207607731114e-08
1475 2.71971654086656e-08
1476 2.70489302067745e-08
1477 2.70370854593693e-08
1478 2.72259725875301e-08
1479 2.6904205085998e-08
1480 2.73345595047658e-08
1481 2.69381086326348e-08
1482 2.70285145376192e-08
1483 2.68235460509914e-08
1484 2.68273687709097e-08
1485 2.67586184321544e-08
1486 2.70098166055277e-08
1487 2.69288324972194e-08
1488 2.67007287391152e-08
1489 2.68151421067842e-08
1490 2.66444892815798e-08
1491 2.68662141422737e-08
1492 2.65882160732644e-08
1493 2.6701206579105e-08
1494 2.6617742676649e-08
1495 2.65526978182606e-08
1496 2.65207269478651e-08
1497 2.64204018662895e-08
1498 2.66125859127442e-08
1499 2.66312980556904e-08
1500 2.64313992914822e-08
1501 2.64776680580781e-08
1502 2.66155435468818e-08
1503 2.62699089148555e-08
1504 2.64873403210686e-08
1505 2.61865462647393e-08
1506 2.63703743286214e-08
1507 2.61989878680424e-08
1508 2.66057611497672e-08
1509 2.63247841303382e-08
1510 2.65205724048201e-08
1511 2.61806167856093e-08
1512 2.62626969060875e-08
1513 2.5993317720463e-08
1514 2.61847485916178e-08
1515 2.62617501078921e-08
1516 2.62322679134286e-08
1517 2.61944901325251e-08
1518 2.61166963611004e-08
1519 2.6042947354199e-08
1520 2.61638248844065e-08
1521 2.61095962628133e-08
1522 2.61528185774296e-08
1523 2.60363393067564e-08
1524 2.58585988177629e-08
1525 2.58763641625137e-08
1526 2.61659778288958e-08
1527 2.59195509499932e-08
1528 2.58715289191969e-08
1529 2.58914134576571e-08
1530 2.63006096901108e-08
1531 2.5822561866562e-08
1532 2.65123194509442e-08
1533 2.59597712215509e-08
1534 2.56883776472705e-08
1535 2.62815795792903e-08
1536 2.57637360334684e-08
1537 2.6085221094263e-08
1538 2.56079513150098e-08
1539 2.61211159369168e-08
1540 2.57135983616763e-08
1541 2.59338204244841e-08
1542 2.55538274984701e-08
1543 2.58102783590175e-08
1544 2.56892551675492e-08
1545 2.57071075537851e-08
1546 2.56960728250988e-08
1547 2.56369823148361e-08
1548 2.59188990270331e-08
1549 2.56464467440765e-08
1550 2.55264129833677e-08
1551 2.58355985494063e-08
1552 2.56554866240322e-08
1553 2.55558632034081e-08
1554 2.53884362422241e-08
1555 2.57069494580264e-08
1556 2.55146055394562e-08
1557 2.55283438832521e-08
1558 2.5632692413069e-08
1559 2.56192667080768e-08
1560 2.54874965577301e-08
1561 2.55923993108809e-08
1562 2.55490206768627e-08
1563 2.58196806157684e-08
1564 2.53628780200188e-08
1565 2.5529827141213e-08
1566 2.57074681542235e-08
1567 2.5734451014614e-08
1568 2.5313246609926e-08
1569 2.54317829018191e-08
1570 2.53901557556446e-08
1571 2.53821692552947e-08
1572 2.52848568749187e-08
1573 2.54848746550351e-08
1574 2.52314045212643e-08
1575 2.54039207447931e-08
1576 2.53332448352239e-08
1577 2.53304683894839e-08
1578 2.53432048680224e-08
1579 2.54797249965577e-08
1580 2.55402579085739e-08
1581 2.54831515889009e-08
1582 2.5354601973504e-08
1583 2.56498928763449e-08
1584 2.52862530913944e-08
1585 2.54771084229333e-08
1586 2.52099034980802e-08
1587 2.51113494442734e-08
1588 2.52420750967985e-08
1589 2.52751615192892e-08
1590 2.51395100292484e-08
1591 2.50743710239476e-08
1592 2.53069778466397e-08
1593 2.54211212080691e-08
1594 2.50831959647257e-08
1595 2.5569224959554e-08
1596 2.54202383587199e-08
1597 2.5096067446384e-08
1598 2.52358240970807e-08
1599 2.51135503503974e-08
1600 2.48746658826349e-08
1601 2.50846845517572e-08
1602 2.50575826754584e-08
1603 2.50950442648445e-08
1604 2.53295731056369e-08
1605 2.51865621692104e-08
1606 2.53426133411949e-08
1607 2.49559111153985e-08
1608 2.51896334901858e-08
1609 2.48797782376187e-08
1610 2.51354421720862e-08
1611 2.50074361218822e-08
1612 2.51086369473796e-08
1613 2.49791920481357e-08
1614 2.47860576507719e-08
1615 2.51274929752299e-08
1616 2.48538096769835e-08
1617 2.50689033975959e-08
1618 2.48076155173749e-08
1619 2.49193750079257e-08
1620 2.47409559506195e-08
1621 2.48304239391928e-08
1622 2.50713885208143e-08
1623 2.51349536739554e-08
1624 2.50693812375857e-08
1625 2.48439757655206e-08
1626 2.52361118668887e-08
1627 2.51095322312267e-08
1628 2.49061162804765e-08
1629 2.52623379992656e-08
1630 2.49944864805229e-08
1631 2.4986311686348e-08
1632 2.49285179165781e-08
1633 2.48092941745881e-08
1634 2.47664324604102e-08
1635 2.48405367386795e-08
1636 2.47322482493928e-08
1637 2.46954545701783e-08
1638 2.4707027534987e-08
1639 2.46650149193783e-08
1640 2.47583642476457e-08
1641 2.48254679036108e-08
1642 2.47049154467049e-08
1643 2.47484557291955e-08
1644 2.48918485823424e-08
1645 2.47640610240296e-08
1646 2.46152485061657e-08
1647 2.47690952193125e-08
1648 2.48548861492282e-08
1649 2.4845817847563e-08
1650 2.46281324223219e-08
1651 2.4658923791776e-08
1652 2.45668054787984e-08
1653 2.48775755551378e-08
1654 2.46313440754875e-08
1655 2.46120404057137e-08
1656 2.45499585105335e-08
1657 2.4793610720053e-08
1658 2.45487630223806e-08
1659 2.44234250601494e-08
1660 2.44149909178759e-08
1661 2.45040094881688e-08
1662 2.4456744185386e-08
1663 2.44743318944529e-08
1664 2.45182487645934e-08
1665 2.44247182479285e-08
1666 2.46487914523641e-08
1667 2.45596414316651e-08
1668 2.43627837903659e-08
1669 2.43241746744616e-08
1670 2.46179165941385e-08
1671 2.42922570947712e-08
1672 2.43420288370544e-08
1673 2.43256028653605e-08
1674 2.43161917268253e-08
1675 2.45174245350199e-08
1676 2.42960602747644e-08
1677 2.44619826617054e-08
1678 2.44325040199556e-08
1679 2.45977069823766e-08
1680 2.4370116591399e-08
1681 2.44176163732845e-08
1682 2.44856526165904e-08
1683 2.4321614944256e-08
1684 2.45340192606136e-08
1685 2.44959945661094e-08
1686 2.43040556568985e-08
1687 2.43576199210338e-08
1688 2.4392127428996e-08
1689 2.41237199105626e-08
1690 2.42880915379828e-08
1691 2.44081714839695e-08
1692 2.42584459186901e-08
1693 2.45438460666492e-08
1694 2.42812383532964e-08
1695 2.426163270286e-08
1696 2.43602293892309e-08
1697 2.42850290987917e-08
1698 2.42851534437705e-08
1699 2.43289584034301e-08
1700 2.43382096698497e-08
1701 2.41825190983036e-08
1702 2.42869830913151e-08
1703 2.42530688865372e-08
1704 2.42938682504246e-08
1705 2.44119515713237e-08
1706 2.43450948289592e-08
1707 2.43831657087412e-08
1708 2.42425386431933e-08
1709 2.41606716855358e-08
1710 2.42259812210932e-08
1711 2.41444091386711e-08
1712 2.42339233125222e-08
1713 2.40583659660842e-08
1714 2.41016238078373e-08
1715 2.41532731592997e-08
1716 2.41600570660694e-08
1717 2.39875177499016e-08
1718 2.42279476481144e-08
1719 2.41229649589059e-08
1720 2.43857503079425e-08
1721 2.42529729632679e-08
1722 2.41608777429292e-08
1723 2.39987105743467e-08
1724 2.4330065073741e-08
1725 2.40680275709337e-08
1726 2.41960762537019e-08
1727 2.40225048742104e-08
1728 2.42428352947854e-08
1729 2.42448709997234e-08
1730 2.41214177520988e-08
1731 2.38957724718603e-08
1732 2.42220146162708e-08
1733 2.40111983629276e-08
1734 2.40948789809181e-08
1735 2.42998972055375e-08
1736 2.40879263202487e-08
1737 2.41470257122955e-08
1738 2.41008937251763e-08
1739 2.40614301816322e-08
1740 2.40215172198077e-08
1741 2.42948789974662e-08
1742 2.39569892812597e-08
1743 2.41986501947622e-08
1744 2.39522250922164e-08
1745 2.41188313765406e-08
1746 2.40889210800788e-08
1747 2.42762894231419e-08
1748 2.41131505873682e-08
1749 2.39325750328589e-08
1750 2.40387567629341e-08
1751 2.39821513758898e-08
1752 2.40891537828247e-08
1753 2.41385365029601e-08
1754 2.38450272860291e-08
1755 2.41214532792355e-08
1756 2.38364332716401e-08
1757 2.43570728031273e-08
1758 2.40284290242698e-08
1759 2.4105332840918e-08
1760 2.40995827738288e-08
1761 2.39378028510373e-08
1762 2.39749251562671e-08
1763 2.3950946115292e-08
1764 2.41006183898662e-08
1765 2.39539872382011e-08
1766 2.38425723608771e-08
1767 2.38994850576546e-08
1768 2.39860096229449e-08
1769 2.39140138802441e-08
1770 2.39321327200059e-08
1771 2.40184654387576e-08
1772 2.38456578927071e-08
1773 2.38980941702494e-08
1774 2.39109496646961e-08
1775 2.39652742095586e-08
1776 2.39399753354519e-08
1777 2.38846933342529e-08
1778 2.41146729251795e-08
1779 2.3911757907058e-08
1780 2.3989880304498e-08
1781 2.38814461539505e-08
1782 2.38582256173459e-08
1783 2.40359447900573e-08
1784 2.39271393809304e-08
1785 2.38007196173839e-08
1786 2.40299780074338e-08
1787 2.39221016329338e-08
1788 2.39941879698335e-08
1789 2.38088588844221e-08
1790 2.38712249966966e-08
1791 2.39389805756218e-08
1792 2.38853701262087e-08
1793 2.37537687297618e-08
1794 2.37912338718616e-08
1795 2.38552662068514e-08
1796 2.38086848014518e-08
1797 2.37872885833212e-08
1798 2.37855957152533e-08
1799 2.37548452020064e-08
1800 2.37847981310324e-08
1801 2.41841533465958e-08
1802 2.37619914855713e-08
1803 2.35948913740458e-08
1804 2.36940387310369e-08
1805 2.38785116124518e-08
1806 2.36433699285499e-08
1807 2.35411423687992e-08
1808 2.37773551958753e-08
1809 2.34321433367768e-08
1810 2.36186465940591e-08
1811 2.34825972000863e-08
1812 2.35540653648059e-08
1813 2.36505641737494e-08
1814 2.35954331628818e-08
1815 2.35081856203578e-08
1816 2.352164507613e-08
1817 2.33108252700731e-08
1818 2.36945840725866e-08
1819 2.36624391192208e-08
1820 2.35056596409322e-08
1821 2.34617267835802e-08
1822 2.3692853901025e-08
1823 2.36023485200576e-08
1824 2.36679742471324e-08
1825 2.36024408906133e-08
1826 2.34322836689671e-08
1827 2.3730704512559e-08
1828 2.34320367553664e-08
1829 2.35313653007552e-08
1830 2.36260664365773e-08
1831 2.37779218537071e-08
1832 2.36941648523725e-08
1833 2.34272370391864e-08
1834 2.33823715944936e-08
1835 2.36151471710855e-08
1836 2.33792381010289e-08
1837 2.33879813293925e-08
1838 2.36590551594418e-08
1839 2.34801706966437e-08
1840 2.34346817507003e-08
1841 2.3598516918355e-08
1842 2.32651196085953e-08
1843 2.33543957506299e-08
1844 2.3541554483586e-08
1845 2.3369780777216e-08
1846 2.37251605028632e-08
1847 2.32751311557422e-08
1848 2.34535928456125e-08
1849 2.35440751339411e-08
1850 2.33822703421538e-08
1851 2.3315571695548e-08
1852 2.35505268619818e-08
1853 2.34377868224556e-08
1854 2.35711627993851e-08
1855 2.35999415565402e-08
1856 2.34588650727119e-08
1857 2.42730866517604e-08
1858 2.53036631647774e-08
1859 2.57896086708342e-08
1860 2.50572185223064e-08
1861 2.520623354485e-08
1862 2.42532092187275e-08
1863 2.41681856749665e-08
1864 2.58043932888086e-08
1865 2.612629046439e-08
1866 2.41966162661811e-08
1867 2.46578029106104e-08
1868 2.41003625944813e-08
1869 2.54829490842212e-08
1870 2.5090537647543e-08
1871 2.43672531041739e-08
1872 2.45489317762804e-08
1873 2.45061766435128e-08
1874 2.40285356056802e-08
1875 2.44914879488078e-08
1876 2.45233717777182e-08
1877 2.5529772074151e-08
1878 2.40382700411601e-08
1879 2.36499957395608e-08
1880 2.34891679440352e-08
1881 2.3534386883739e-08
1882 2.34979058433282e-08
1883 2.37262085533985e-08
1884 2.35262209713483e-08
1885 2.37260984192744e-08
1886 2.34477379734699e-08
1887 2.35325146036303e-08
1888 2.36966393174498e-08
1889 2.40141115881443e-08
1890 2.42548097162398e-08
1891 2.57152539262506e-08
1892 2.5236108314175e-08
1893 2.42466828836996e-08
1894 2.54969823032525e-08
1895 2.58198280533861e-08
1896 2.55265071302802e-08
1897 2.55213734590143e-08
1898 2.45173872315263e-08
1899 2.63173909331726e-08
1900 2.60840202770396e-08
1901 2.53784548931435e-08
1902 2.58333407998634e-08
1903 2.43717188652681e-08
1904 2.49185596601365e-08
1905 2.5233664047164e-08
1906 2.60196220125408e-08
1907 2.43010784828357e-08
1908 2.54050167569631e-08
1909 2.51029650399914e-08
1910 2.51661607109099e-08
1911 2.53932697091841e-08
1912 2.40022153263908e-08
1913 2.60253028017132e-08
1914 2.43251818687895e-08
1915 2.55048000497027e-08
1916 2.54771048702196e-08
1917 2.56436933909754e-08
1918 2.47547227161249e-08
1919 2.60634553939099e-08
1920 2.55225440781714e-08
1921 2.53737617583738e-08
1922 2.53923140292045e-08
1923 2.43397000332379e-08
1924 2.50959644176874e-08
1925 2.52234393371964e-08
1926 2.51652050309303e-08
1927 2.39436754867484e-08
1928 2.49167246835214e-08
1929 2.50156126924139e-08
1930 2.50536373869181e-08
1931 2.50888412267614e-08
1932 2.42773019465403e-08
1933 2.43706494984508e-08
1934 2.39818849223639e-08
1935 2.44867219834077e-08
1936 2.48278801961987e-08
1937 2.47535627551088e-08
1938 2.41745858886588e-08
1939 2.48248017697961e-08
1940 2.4680957722012e-08
1941 2.47142128984024e-08
1942 2.4599273729109e-08
1943 2.4122510211555e-08
1944 2.44112552394427e-08
1945 2.42905553449191e-08
1946 2.40385737981796e-08
1947 2.42093776137153e-08
1948 2.42945894513014e-08
1949 2.53888767787203e-08
1950 2.52168934622432e-08
1951 2.59697099380674e-08
1952 2.53553533724471e-08
1953 2.44501485724413e-08
1954 2.5549603321906e-08
1955 2.44420661488221e-08
1956 2.59109214084674e-08
1957 2.53673011485489e-08
1958 2.5145586945996e-08
1959 2.46907596590518e-08
1960 2.5448883889112e-08
1961 2.48739162600486e-08
1962 2.54878642635958e-08
1963 2.5171248196898e-08
1964 2.55383643121831e-08
1965 2.58042316403362e-08
1966 2.55235690360678e-08
1967 2.55399221771313e-08
1968 2.62990713650879e-08
1969 2.60012864572445e-08
1970 2.55665586479381e-08
1971 2.52414373846932e-08
1972 2.48760230192602e-08
1973 2.57099426193008e-08
1974 2.53477985268091e-08
1975 2.51917224858289e-08
1976 2.55138310478742e-08
1977 2.54393413001708e-08
1978 2.47501166228403e-08
1979 2.55252068370737e-08
1980 2.56597143533099e-08
1981 2.47368987515983e-08
1982 2.47721700930015e-08
1983 2.54014711487116e-08
1984 2.54703795832256e-08
1985 2.52431355818317e-08
1986 2.46914773072149e-08
1987 2.53729766086508e-08
1988 2.5807750603235e-08
1989 2.57580836660054e-08
1990 2.42811921680186e-08
1991 2.53261998039989e-08
1992 2.52512197818078e-08
1993 2.6098881278358e-08
1994 2.44597906373656e-08
1995 2.45215296956758e-08
1996 2.46040841034301e-08
1997 2.60784940309122e-08
1998 2.56496157646779e-08
1999 2.57032617412278e-08
2000 2.50943710256024e-08
2001 2.5486402321917e-08
2002 2.5251283730654e-08
2003 2.60424393161429e-08
2004 2.49951117581304e-08
2005 2.50306513294163e-08
2006 2.53205652001043e-08
2007 2.534204845972e-08
2008 2.60715644628817e-08
2009 2.4905988382784e-08
2010 2.54345682293433e-08
2011 2.49759075643396e-08
2012 2.58722536727873e-08
2013 2.48693723392535e-08
2014 2.44774582824903e-08
2015 2.5023974004057e-08
2016 2.55750389754894e-08
2017 2.50112197619501e-08
2018 2.55194976261919e-08
2019 2.54812686506511e-08
2020 2.44319515729785e-08
2021 2.54889425121974e-08
2022 2.54700527335672e-08
2023 2.55195544696107e-08
2024 2.44460469644991e-08
2025 2.54500971408333e-08
2026 2.53473970701634e-08
2027 2.54111949260505e-08
2028 2.54578775837899e-08
2029 2.47681661846855e-08
2030 2.60722838874017e-08
2031 2.53316301268569e-08
2032 2.53451659659731e-08
2033 2.51251410787745e-08
2034 2.57633487876774e-08
2035 2.5083723542707e-08
2036 2.53056988697153e-08
2037 2.53647041148497e-08
2038 2.58883989801006e-08
2039 2.46083224908489e-08
2040 2.46563054417948e-08
2041 2.49307401389842e-08
2042 2.46292053418529e-08
2043 2.45969111745126e-08
2044 2.46384406210609e-08
2045 2.46784921387189e-08
2046 2.50300633553024e-08
2047 2.49879654745655e-08
2048 2.49486955539169e-08
2049 2.47142128984024e-08
2050 2.46815829996194e-08
2051 2.5059113895054e-08
2052 2.45520705988156e-08
2053 2.45393767528412e-08
2054 2.44551774386537e-08
2055 2.52332483796636e-08
2056 2.44700402163289e-08
2057 2.47813201070812e-08
2058 2.50483012109726e-08
2059 2.43607676253532e-08
2060 2.51384264515764e-08
2061 2.44026647777673e-08
2062 2.48024853988227e-08
2063 2.48396698765418e-08
2064 2.49908502780727e-08
2065 2.47900135974533e-08
2066 2.4485148131248e-08
2067 2.45969147272262e-08
2068 2.51103386972318e-08
2069 2.46178029073008e-08
2070 2.44605367072381e-08
2071 2.50287648384528e-08
2072 2.42597604227512e-08
2073 2.43185329651396e-08
2074 2.45457982828157e-08
2075 2.50134455370699e-08
2076 2.44942199856268e-08
2077 2.41719817495323e-08
2078 2.44482567524074e-08
2079 2.51026452957603e-08
2080 2.42215971724136e-08
2081 2.46732714259679e-08
2082 2.44985738362402e-08
2083 2.39966784221224e-08
2084 2.48849527650918e-08
2085 2.46183766705599e-08
2086 2.42036612974061e-08
2087 2.46893065991571e-08
2088 2.50089264852704e-08
2089 2.40797852768537e-08
2090 2.46051712338158e-08
2091 2.42065318900586e-08
2092 2.4366709538981e-08
2093 2.44518876257871e-08
2094 2.48202471908598e-08
2095 2.42659066174156e-08
2096 2.38974759980692e-08
2097 2.37771917710461e-08
2098 2.44753142197851e-08
2099 2.43909976660461e-08
2100 2.44314968256276e-08
2101 2.44207960520271e-08
2102 2.47388740604038e-08
2103 2.43176856429272e-08
2104 2.38398367713444e-08
2105 2.43105269248645e-08
2106 2.42533726435568e-08
2107 2.46519462621109e-08
2108 2.41725199856546e-08
2109 2.36479618109797e-08
2110 2.44318449915681e-08
2111 2.45722677760796e-08
2112 2.39231479071123e-08
2113 2.40799185036167e-08
2114 2.36952288901193e-08
2115 2.37578596795629e-08
2116 2.38206805391883e-08
2117 2.42804425454324e-08
2118 2.32659100873889e-08
2119 2.44416007433301e-08
2120 2.41509550136243e-08
2121 2.35915589286151e-08
2122 2.43399025379176e-08
2123 2.34238939356146e-08
2124 2.3020547246233e-08
2125 2.38944206643055e-08
2126 2.42379218917677e-08
2127 2.34593873216227e-08
2128 2.3928283354735e-08
2129 2.3588640374328e-08
2130 2.33834427376678e-08
2131 2.37102817379764e-08
2132 2.37599842023428e-08
2133 2.51713494492378e-08
2134 2.32817392031848e-08
2135 2.2940367827573e-08
2136 2.35802399828344e-08
2137 2.43809630262604e-08
2138 2.39419133407637e-08
2139 2.43551223633176e-08
2140 2.50755256558932e-08
2141 2.41260629252338e-08
2142 2.50644838217795e-08
2143 2.37822863624615e-08
2144 2.24109957258634e-08
2145 2.51418406094217e-08
2146 2.45343052540647e-08
2147 2.41219275665117e-08
2148 2.42936781802428e-08
2149 2.33482602141066e-08
2150 2.45765363615646e-08
2151 2.37180906026424e-08
2152 2.35766197675957e-08
2153 2.49827696308103e-08
2154 2.24965823747425e-08
2155 2.24586536035076e-08
2156 2.40278872354338e-08
2157 2.46525058145153e-08
2158 2.38108057715181e-08
2159 2.22843254960026e-08
2160 2.45098821238798e-08
2161 2.42487629975585e-08
2162 2.41537847500695e-08
2163 2.4425165889852e-08
2164 2.34957120426316e-08
2165 2.41896405128728e-08
2166 2.23394440723723e-08
2167 2.45470737070264e-08
2168 2.32732375593514e-08
2169 2.47346321202713e-08
2170 2.46487772415094e-08
2171 2.34666810428052e-08
2172 2.50407374835504e-08
2173 2.44600890653146e-08
2174 2.34498092055446e-08
2175 2.38638158123194e-08
2176 2.36643309392548e-08
2177 2.37170691974598e-08
2178 2.34896280204566e-08
2179 2.22754259482372e-08
2180 2.36743424864017e-08
2181 2.29560370712534e-08
2182 2.44694806639245e-08
2183 2.30466650208427e-08
2184 2.36749340132292e-08
2185 2.39295037118836e-08
2186 2.21768008401568e-08
2187 2.39681732239205e-08
2188 2.30474466178521e-08
2189 2.30795773603631e-08
2190 2.23054446024662e-08
2191 2.40632154202558e-08
2192 2.43328912574725e-08
2193 2.48496458965519e-08
2194 2.41574458215155e-08
2195 2.39692177217421e-08
2196 2.34414567756858e-08
2197 2.29365735293641e-08
2198 2.44538220783852e-08
2199 2.44326496812164e-08
2200 2.3874040522287e-08
2201 2.47337350600674e-08
2202 2.41685498281186e-08
2203 2.37461410534934e-08
2204 2.29802541440449e-08
2205 2.50744562890759e-08
2206 2.43594886484289e-08
2207 2.48026452709382e-08
2208 2.34575612267918e-08
2209 2.40923778704882e-08
2210 2.37423591897823e-08
2211 2.30840093706774e-08
2212 2.46237377155012e-08
2213 2.39819399894259e-08
2214 2.26765983768473e-08
2215 2.39188828743409e-08
2216 2.42035707032073e-08
2217 2.42353976886989e-08
2218 2.41873756579025e-08
2219 2.43559092893975e-08
2220 2.44638869162372e-08
2221 2.4417257549203e-08
2222 2.42286954943438e-08
2223 2.4248038243968e-08
2224 2.44412916572401e-08
2225 2.43469280292175e-08
2226 2.376941665716e-08
2227 2.44163782525675e-08
2228 2.42257485183472e-08
2229 2.3441973695526e-08
2230 2.47244642537225e-08
2231 2.40459190337106e-08
2232 2.40689779218428e-08
2233 2.51394194350496e-08
2234 2.33521948445059e-08
2235 2.53158276564136e-08
2236 2.33968826535147e-08
2237 2.52877061512891e-08
2238 2.44516158431907e-08
2239 2.54147707323682e-08
2240 2.39176181082712e-08
2241 2.39374315924579e-08
2242 2.50896352582686e-08
2243 2.44523512549222e-08
2244 2.42708075859355e-08
2245 2.38941559871364e-08
2246 2.40126389883244e-08
2247 2.35322694663864e-08
2248 2.41310171844589e-08
2249 2.37172059769364e-08
2250 2.393323761396e-08
2251 2.40269599771636e-08
2252 2.39268924673297e-08
2253 2.25360352601456e-08
2254 2.38669564112115e-08
2255 2.43489299833755e-08
2256 2.34895090045484e-08
2257 2.37621247123343e-08
2258 2.40125253014867e-08
2259 2.34760619832741e-08
2260 2.38276136599325e-08
2261 2.31318590948604e-08
2262 2.35091821565447e-08
2263 2.31127454952684e-08
2264 2.22502283264703e-08
2265 2.3721147712763e-08
2266 2.28449952288656e-08
2267 2.21791118804049e-08
2268 2.35750370336518e-08
2269 2.2724174542077e-08
2270 2.37497488342342e-08
2271 2.35516850466411e-08
2272 2.32817338741143e-08
2273 2.34861960990429e-08
2274 2.35472779053225e-08
2275 2.25054321845164e-08
2276 2.36312445167641e-08
2277 2.35502710665969e-08
2278 2.31911059245249e-08
2279 2.30750565322069e-08
2280 2.26376535295003e-08
2281 2.32254162568779e-08
2282 2.34736727833251e-08
2283 2.28314878114588e-08
2284 2.32530315003032e-08
2285 2.25703917777764e-08
2286 2.33655743642203e-08
2287 2.26951382131801e-08
2288 2.36396715536102e-08
2289 2.33202577248903e-08
2290 2.27005418906856e-08
2291 2.25918039831186e-08
2292 2.35445902774245e-08
2293 2.30320775784776e-08
2294 2.28556569226157e-08
2295 2.2402007360256e-08
2296 2.3380863467537e-08
2297 2.26825846993961e-08
2298 2.29504468762798e-08
2299 2.14940154563692e-08
2300 2.26808687386892e-08
2301 2.23752358863294e-08
2302 2.29983516675247e-08
2303 2.26552234749988e-08
2304 2.28674021940378e-08
2305 2.2141382061136e-08
2306 2.29245884497686e-08
2307 2.25704486211953e-08
2308 2.1478244960349e-08
2309 2.26365539646167e-08
2310 2.25258744990242e-08
2311 2.1569308117364e-08
2312 2.32691501622639e-08
2313 2.17242739353196e-08
2314 2.27770513561154e-08
2315 2.19703597537091e-08
2316 2.25252314578483e-08
2317 2.29590000344615e-08
2318 2.26719159002187e-08
2319 2.32193482219145e-08
2320 2.25397762676494e-08
2321 2.2511752462151e-08
2322 2.13586304198543e-08
2323 2.28251195721896e-08
2324 2.22551008732808e-08
2325 2.26581278184312e-08
2326 2.23916956088033e-08
2327 2.14354489713742e-08
2328 2.12870681082222e-08
2329 2.27943974806522e-08
2330 2.17261906243493e-08
2331 2.27996057589053e-08
2332 2.27635226224265e-08
2333 2.10703543501722e-08
2334 2.26953176252209e-08
2335 2.2404949007182e-08
2336 2.11450590370532e-08
2337 2.29666241580162e-08
2338 2.28994334605659e-08
2339 2.17904929655788e-08
2340 2.27688641274426e-08
2341 2.18833910992089e-08
2342 2.28152536863035e-08
2343 2.19512710231129e-08
2344 2.14940865106428e-08
2345 2.12476880534496e-08
2346 2.23315073100139e-08
2347 2.14532871467554e-08
2348 2.21764864249963e-08
2349 2.24598952769384e-08
2350 2.32503598596168e-08
2351 2.2767034479898e-08
2352 2.11389785675919e-08
2353 2.24692584538388e-08
2354 2.15716617901762e-08
2355 2.20560423258576e-08
2356 2.24080274335847e-08
2357 2.24431850881501e-08
2358 2.18085176584282e-08
2359 2.20671303452491e-08
2360 2.17544631198052e-08
2361 2.19477893637077e-08
2362 2.19769535902969e-08
2363 2.16504378869331e-08
2364 2.11801332028472e-08
2365 2.15455564500644e-08
2366 2.09007584572873e-08
2367 2.16184119494756e-08
2368 2.10838901892885e-08
2369 2.13559374628858e-08
2370 2.18119993178334e-08
2371 2.13637125767718e-08
2372 2.24817160443536e-08
2373 2.18748184011019e-08
2374 2.18197513390805e-08
2375 2.14989537283827e-08
2376 2.19538929258078e-08
2377 2.15475868259318e-08
2378 2.13836717222193e-08
2379 2.17700470983573e-08
2380 2.13055706410614e-08
2381 2.19165361414753e-08
2382 2.14177280355443e-08
2383 2.10824548929622e-08
2384 2.18336317914236e-08
2385 2.11184456588853e-08
2386 2.17505480293312e-08
2387 2.14535553766382e-08
2388 2.11443289543922e-08
2389 2.17668585378306e-08
2390 2.14431974399076e-08
2391 2.17753992615144e-08
2392 2.14273505605433e-08
2393 2.14537099196832e-08
2394 2.09685797614156e-08
2395 2.26327934171877e-08
2396 2.13226236667197e-08
2397 2.07720045608539e-08
2398 2.25334186865211e-08
2399 2.1819923645694e-08
2400 2.23106280117236e-08
2401 2.26649596868356e-08
2402 2.14352660066197e-08
2403 2.28958079162567e-08
2404 2.23045546476897e-08
2405 2.24985949870415e-08
2406 2.21849116854855e-08
2407 2.27069563152327e-08
2408 2.18665636708693e-08
2409 2.14868443038085e-08
2410 2.16668833985523e-08
2411 2.21034888170379e-08
2412 2.10399822009322e-08
2413 2.20130615957714e-08
2414 2.20737188527664e-08
2415 2.27665672980493e-08
2416 2.23606662075326e-08
2417 2.18093649806406e-08
2418 2.24515694924321e-08
2419 2.19577120930126e-08
2420 2.20105569326279e-08
2421 2.25093721439862e-08
2422 2.18121840589447e-08
2423 2.12914201824788e-08
2424 2.22421778772741e-08
2425 2.21449276693875e-08
2426 2.10026858127321e-08
2427 2.06117309886622e-08
2428 2.05996713020795e-08
2429 2.11825987861403e-08
2430 2.10283666035593e-08
2431 2.03247108032656e-08
2432 2.08995025730019e-08
2433 2.10038866299556e-08
2434 2.08630499543005e-08
2435 2.10271036138465e-08
2436 2.08756905095697e-08
2437 2.06238137678838e-08
2438 2.05271533104678e-08
2439 2.07206696245521e-08
2440 2.07075991909278e-08
2441 2.09699617670367e-08
2442 2.10559516489184e-08
2443 2.08219965713852e-08
2444 2.07670431962015e-08
2445 2.07820409769965e-08
2446 2.07091961357264e-08
2447 2.07103010296805e-08
2448 2.07257979667475e-08
2449 1.82328587783331e-08
2450 1.95356300025651e-08
2451 1.96689082798684e-08
2452 1.96954790254722e-08
2453 1.95605451835945e-08
2454 1.95697076321721e-08
2455 1.79524448640223e-08
2456 1.91351876566159e-08
2457 1.79739370054222e-08
2458 1.90908604480455e-08
2459 1.91588060971526e-08
2460 1.94817566523398e-08
2461 1.78848598153536e-08
2462 1.89017832497029e-08
2463 1.91755180622977e-08
2464 1.92473361693146e-08
2465 1.92628313300247e-08
2466 1.84037887152044e-08
2467 1.96405078867201e-08
2468 1.92849665126005e-08
2469 1.90850517611807e-08
2470 1.87722815070401e-08
2471 1.89857622956424e-08
2472 1.7675839458775e-08
2473 1.89213018586543e-08
2474 1.89847142451072e-08
2475 1.76946741703432e-08
2476 1.87720470279373e-08
2477 1.88696187564119e-08
2478 1.87697004605525e-08
2479 1.87066540036085e-08
2480 1.92523579300996e-08
2481 1.89282420848258e-08
2482 1.88597724104511e-08
2483 1.87025204212432e-08
2484 1.87718693922534e-08
2485 1.87154913788845e-08
2486 1.86875546148713e-08
2487 1.73591683250152e-08
2488 1.87670377016502e-08
2489 1.87382784844203e-08
2490 1.79592074545099e-08
2491 1.88473059381522e-08
2492 1.84870003749893e-08
2493 1.82776300761134e-08
2494 1.87587581024218e-08
2495 1.87042523691616e-08
2496 1.84714465945035e-08
2497 1.85714057465702e-08
2498 1.85872295332956e-08
2499 1.85917059525309e-08
2500 1.86313968697505e-08
2501 1.77487429198209e-08
2502 1.75446306371896e-08
2503 1.85884605485853e-08
2504 1.86219040188007e-08
2505 1.84956512327972e-08
2506 1.83646342577504e-08
2507 1.83709172318913e-08
2508 1.86110007405205e-08
2509 1.84701125505171e-08
2510 1.82545516480559e-08
2511 1.86067783403132e-08
2512 1.89019413454616e-08
2513 1.80791346338083e-08
2514 1.83163226807892e-08
2515 1.84716153484032e-08
2516 1.83164043932038e-08
2517 1.82898940437326e-08
2518 1.82677588611568e-08
2519 1.86864337337056e-08
2520 1.84214332676902e-08
2521 1.90291089552375e-08
2522 1.7127280926843e-08
2523 1.8180420724434e-08
2524 1.81998931481075e-08
2525 1.87085227310035e-08
2526 1.84510504652735e-08
2527 1.87067819013009e-08
2528 1.85360118365452e-08
2529 1.82760810929494e-08
2530 1.85899455829031e-08
2531 1.87026873987861e-08
2532 1.88536475320689e-08
2533 1.72231366946107e-08
2534 1.80212644806943e-08
2535 1.81189641068613e-08
2536 1.83915016549463e-08
2537 1.85199233726507e-08
2538 1.84179054230071e-08
2539 1.85162853938436e-08
2540 1.82736279441542e-08
2541 1.84195236840878e-08
2542 1.88308852955288e-08
2543 1.84655295498715e-08
2544 1.8265279067009e-08
2545 1.85698034727011e-08
2546 1.88642612641843e-08
2547 1.8819026337269e-08
2548 1.84628241584051e-08
2549 1.88821775992665e-08
2550 1.84562569671698e-08
2551 1.83697892452983e-08
2552 1.88385325117224e-08
2553 1.83253270336081e-08
2554 1.84315354090359e-08
2555 1.81514536734539e-08
2556 1.81800707821367e-08
2557 1.86233961585458e-08
2558 1.83841155632081e-08
2559 1.83919723895087e-08
2560 1.84490058785514e-08
2561 1.8337152241088e-08
2562 1.83790920260662e-08
2563 1.84597439556455e-08
2564 1.82722139641101e-08
2565 1.83617281379611e-08
2566 1.81322636905179e-08
2567 1.8515795119356e-08
2568 1.82635488954475e-08
2569 1.80022698970106e-08
2570 1.81294108614338e-08
2571 1.77956849256589e-08
2572 1.82596444631145e-08
2573 1.83098833872464e-08
2574 1.82632842182784e-08
2575 1.78847994192211e-08
2576 1.79807244649055e-08
2577 1.85552764264685e-08
2578 1.78687997731686e-08
2579 1.81708177393602e-08
2580 1.81795343223712e-08
2581 1.80054406939689e-08
2582 1.85635418148422e-08
2583 1.81869825865988e-08
2584 1.81662347387146e-08
2585 1.81133668064604e-08
2586 1.80822077311404e-08
2587 1.75831473825383e-08
2588 1.84683841553124e-08
2589 1.83073414206092e-08
2590 1.83229982297917e-08
2591 1.80696950735637e-08
2592 1.82544663829276e-08
2593 1.82567632123209e-08
2594 1.80479489131358e-08
2595 1.86145410197014e-08
2596 1.78603922762477e-08
2597 1.80690662432426e-08
2598 1.77592589523101e-08
2599 1.78122405714021e-08
2600 1.78031474007412e-08
2601 1.77780847820941e-08
2602 1.75732637330839e-08
2603 1.75890111364652e-08
2604 1.69675988814788e-08
2605 1.76092243009407e-08
2606 1.74889063231376e-08
2607 1.77908692222672e-08
2608 1.71912226676341e-08
2609 1.7899477455785e-08
2610 1.72847229862327e-08
2611 1.73116276869223e-08
2612 1.7415590747305e-08
2613 1.72624936567445e-08
2614 1.82274728643961e-08
2615 1.73101462053182e-08
2616 1.70318088521526e-08
2617 1.72178236113041e-08
2618 1.6915588929578e-08
2619 1.73350258592109e-08
2620 1.73535976699668e-08
2621 1.70837619606345e-08
2622 1.72906240436532e-08
2623 1.71759104716784e-08
2624 1.72152923028079e-08
2625 1.67600369138654e-08
2626 1.7104097693732e-08
2627 1.67343667811792e-08
2628 1.66503326681777e-08
2629 1.6691652504619e-08
2630 1.71837548634812e-08
2631 1.71724483521984e-08
2632 1.71745018207048e-08
2633 1.70737219917783e-08
2634 1.64728959362037e-08
2635 1.65750133618303e-08
2636 1.70221081674526e-08
2637 1.74539334096835e-08
2638 1.65219393721827e-08
2639 1.68897518193489e-08
2640 1.63550026854864e-08
2641 1.65683768926783e-08
2642 1.67532974160167e-08
2643 1.69207581279807e-08
2644 1.6794075463622e-08
2645 1.69730416388347e-08
2646 1.67733826827998e-08
2647 1.70983476266429e-08
2648 1.66439821924769e-08
2649 1.68230407382453e-08
2650 1.71830123463224e-08
2651 1.67016231955586e-08
2652 1.69875544742126e-08
2653 1.6445804718046e-08
2654 1.65902438453713e-08
2655 1.65870996937656e-08
2656 1.66998308515076e-08
2657 1.6768403554579e-08
2658 1.62132671732707e-08
2659 1.62211684084923e-08
2660 1.6396993984813e-08
2661 1.65133275942253e-08
2662 1.6380070633204e-08
2663 1.63286593135581e-08
2664 1.66046554284094e-08
2665 1.65174309785243e-08
2666 1.66831313208604e-08
2667 1.6191075147276e-08
2668 1.63633480099179e-08
2669 1.66286646674507e-08
2670 1.63877178493976e-08
2671 1.62231881262187e-08
2672 1.66059326289769e-08
2673 1.65157825193774e-08
2674 1.70288689815834e-08
2675 1.65145532804445e-08
2676 1.60520805536635e-08
2677 1.69031686425569e-08
2678 1.66050551086983e-08
2679 1.65108389182933e-08
2680 1.67150808749739e-08
2681 1.64824847104228e-08
2682 1.68490927876519e-08
2683 1.67111942062093e-08
2684 1.67058296085543e-08
2685 1.64613016551129e-08
2686 1.69389888782234e-08
2687 1.65411542241145e-08
2688 1.6742001562875e-08
2689 1.67838614117954e-08
2690 1.65935016838148e-08
2691 1.65741962376842e-08
2692 1.62499969036389e-08
2693 1.61085740302269e-08
2694 1.62069042630719e-08
2695 1.69710610009588e-08
2696 1.70459824033742e-08
2697 1.60855950781524e-08
2698 1.64806053248867e-08
2699 1.70901959251069e-08
2700 1.59201469784875e-08
2701 1.65667692897387e-08
2702 1.62697748606888e-08
2703 1.64887605791364e-08
2704 1.63664619634574e-08
2705 1.62747824106191e-08
2706 1.64733862106914e-08
2707 1.59265756138893e-08
2708 1.61713789026408e-08
2709 1.59078048511674e-08
2710 1.62527697966652e-08
2711 1.59111426256686e-08
2712 1.53115635725953e-08
2713 1.56875614720775e-08
2714 1.5861564506281e-08
2715 1.58235398117768e-08
2716 1.56253250338523e-08
2717 1.61888387140152e-08
2718 1.57851580695478e-08
2719 1.55893467024271e-08
2720 1.55662309708759e-08
2721 1.56661208450259e-08
2722 1.55277426472367e-08
2723 1.54027244292365e-08
2724 1.61999000880542e-08
2725 1.56845203491685e-08
2726 1.55603228080281e-08
2727 1.56927555394759e-08
2728 1.58564716912224e-08
2729 1.64965108240267e-08
2730 1.61082365224274e-08
2731 1.57610298145983e-08
2732 1.59669557575626e-08
2733 1.54571981880736e-08
2734 1.56054600353173e-08
2735 1.63943472131223e-08
2736 1.54768731164268e-08
2737 1.59301691837754e-08
2738 1.57830370994816e-08
2739 1.57570383407801e-08
2740 1.53955230786096e-08
2741 1.5362369154559e-08
2742 1.57406763179324e-08
2743 1.55555142100638e-08
2744 1.59825752632514e-08
2745 1.62460160879618e-08
2746 1.62814774995468e-08
2747 1.66396567635729e-08
2748 1.69838152430657e-08
2749 1.61247477592497e-08
2750 1.60844670915594e-08
2751 1.57500892328244e-08
2752 1.57190154226328e-08
2753 1.63560383015238e-08
2754 1.57290642732733e-08
2755 1.64012732284391e-08
2756 1.64232041299783e-08
2757 1.60021809136879e-08
2758 1.57531747646544e-08
2759 1.6023305349222e-08
2760 1.63708762102033e-08
2761 1.58855559817539e-08
2762 1.60301318885558e-08
2763 1.61947664167883e-08
2764 1.57814756818198e-08
2765 1.63338089720355e-08
2766 1.62767381794993e-08
2767 1.60194293385985e-08
2768 1.54893591286509e-08
2769 1.58070889710871e-08
2770 1.66866822581824e-08
2771 1.57290180879954e-08
2772 1.60964592765822e-08
2773 1.60962141393384e-08
2774 1.58797721638848e-08
2775 1.58375677017375e-08
2776 1.58503823399769e-08
2777 1.57346278228943e-08
2778 1.57195056971204e-08
2779 1.57921942189887e-08
2780 1.61394648756641e-08
2781 1.60778803603989e-08
2782 1.602774091225e-08
2783 1.62007971482581e-08
2784 1.55388022449188e-08
2785 1.61884585736516e-08
2786 1.59103503705182e-08
2787 1.6345616415947e-08
2788 1.63950542031444e-08
2789 1.60808859561712e-08
2790 1.57324588911933e-08
2791 1.59661581733417e-08
2792 1.59725992432413e-08
2793 1.56903023906807e-08
2794 1.56603050527337e-08
2795 1.61179496416253e-08
2796 1.58777542225153e-08
2797 1.62368039013927e-08
2798 1.55683643754401e-08
2799 1.58845558928533e-08
2800 1.61688014088668e-08
2801 1.57365978026291e-08
2802 1.62451012641895e-08
2803 1.60601221210754e-08
2804 1.60765072365621e-08
2805 1.5865921909608e-08
2806 1.62408433368455e-08
2807 1.63234741279439e-08
2808 1.62357576272143e-08
2809 1.6114412915158e-08
2810 1.60990065722899e-08
2811 1.6375569344973e-08
2812 1.60447672925557e-08
2813 1.60856892250649e-08
2814 1.61360382833209e-08
2815 1.61640549833919e-08
2816 1.59539741417802e-08
2817 1.60323789799577e-08
2818 1.58280322182236e-08
2819 1.60721551623055e-08
2820 1.59463056093045e-08
2821 1.57907020792436e-08
2822 1.58676591865969e-08
2823 1.58992659038404e-08
2824 1.58478172807008e-08
2825 1.58226196589339e-08
2826 1.61724305058897e-08
2827 1.59500039842442e-08
2828 1.60349404865201e-08
2829 1.57844493031689e-08
2830 1.59168038749158e-08
2831 1.61616586780156e-08
2832 1.59733701821096e-08
2833 1.61026179057444e-08
2834 1.60781947755595e-08
2835 1.61349298366531e-08
2836 1.56736064127472e-08
2837 1.56345798529856e-08
2838 1.59402144817022e-08
2839 1.58358641755285e-08
2840 1.58717146092613e-08
2841 1.58777080372374e-08
2842 1.59908015717747e-08
2843 1.58233781633044e-08
2844 1.58686006557218e-08
2845 1.57819641799506e-08
2846 1.58024135998858e-08
2847 1.60709472396547e-08
2848 1.59666786458956e-08
2849 1.57763704322633e-08
2850 1.59994346660142e-08
2851 1.58655666382401e-08
2852 1.60531392623398e-08
2853 1.59924855580584e-08
2854 1.58811879202858e-08
2855 1.53379868805814e-08
2856 1.62365711986467e-08
2857 1.59810529254401e-08
2858 1.59412163469597e-08
2859 1.58487196699753e-08
2860 1.58503006275623e-08
2861 1.61281228372445e-08
2862 1.55751660457781e-08
2863 1.5942395847901e-08
2864 1.58728141741449e-08
2865 1.57972248615579e-08
2866 1.6072437603043e-08
2867 1.57988910842732e-08
2868 1.57207011852734e-08
2869 1.59007313982329e-08
2870 1.59871333949013e-08
2871 1.59360524776275e-08
2872 1.59035824509601e-08
2873 1.5386946827789e-08
2874 1.55471084894998e-08
2875 1.6184770856853e-08
2876 1.60918371960861e-08
2877 1.58815325335127e-08
2878 1.62333044784191e-08
2879 1.58797224258933e-08
2880 1.54528816409538e-08
2881 1.57070321193942e-08
2882 1.55215058583735e-08
2883 1.54118477979637e-08
2884 1.57774611153627e-08
2885 1.59180100212097e-08
2886 1.56636286163803e-08
2887 1.5748655712855e-08
2888 1.56059698497302e-08
2889 1.55057691131333e-08
2890 1.59900679364e-08
2891 1.55935904189164e-08
2892 1.56755248781337e-08
2893 1.59550612721659e-08
2894 1.58938355809823e-08
2895 1.57782782395088e-08
2896 1.57053072769031e-08
2897 1.58588449039598e-08
2898 1.58627670998612e-08
2899 1.61771307460867e-08
2900 1.57961181912469e-08
2901 1.5660623020608e-08
2902 1.58220796464548e-08
2903 1.55319828110123e-08
2904 1.55790012001944e-08
2905 1.5826014276854e-08
2906 1.51034225126523e-08
2907 1.5662857677512e-08
2908 1.57089381502828e-08
2909 1.50718477698319e-08
2910 1.54905475113765e-08
2911 1.59963295942589e-08
2912 1.54555745979224e-08
2913 1.52517038998212e-08
2914 1.56800954442815e-08
2915 1.5027350031005e-08
2916 1.56624881952894e-08
2917 1.54294905740926e-08
2918 1.52412997778129e-08
2919 1.50361199047211e-08
2920 1.53735584262904e-08
2921 1.50733345805065e-08
2922 1.51578483098547e-08
2923 1.53085775167483e-08
2924 1.51311159157785e-08
2925 1.48092365037655e-08
2926 1.4893592137355e-08
2927 1.52644279438618e-08
2928 1.49481707012455e-08
2929 1.50216372674095e-08
2930 1.50370329521365e-08
2931 1.5376892648078e-08
2932 1.52537253939045e-08
2933 1.51221879463037e-08
2934 1.53193813190455e-08
2935 1.51501833300927e-08
2936 1.51139118997889e-08
2937 1.51754431243489e-08
2938 1.48009551281802e-08
2939 1.52266661501699e-08
2940 1.54496131443693e-08
2941 1.53561572346916e-08
2942 1.5145898757396e-08
2943 1.54695420917506e-08
2944 1.53990828977157e-08
2945 1.52169938871793e-08
2946 1.57966209002325e-08
2947 1.51381396307215e-08
2948 1.50716328306544e-08
2949 1.47447494214248e-08
2950 1.53758517029701e-08
2951 1.52763579563953e-08
2952 1.51548018578751e-08
2953 1.52719632495746e-08
2954 1.45965310949236e-08
2955 1.5116654594749e-08
2956 1.47846188625067e-08
2957 1.49161962781363e-08
2958 1.48052130555243e-08
2959 1.48398680011042e-08
2960 1.48063126204079e-08
2961 1.51053569652504e-08
2962 1.49504764124231e-08
2963 1.52101709005592e-08
2964 1.47553569362913e-08
2965 1.49085987999342e-08
2966 1.51313770402339e-08
2967 1.49671652849293e-08
2968 1.48791805543169e-08
2969 1.53943737757345e-08
2970 1.50506807017337e-08
2971 1.59614277350784e-08
2972 1.53313965967072e-08
2973 1.52657371188525e-08
2974 1.53120058854483e-08
2975 1.46888448071536e-08
2976 1.50387826636234e-08
2977 1.46993830441033e-08
2978 1.56427795161562e-08
2979 1.60034865359648e-08
2980 1.53814738723668e-08
2981 1.53784576184535e-08
2982 1.47395429195285e-08
2983 1.51185766128492e-08
2984 1.53629144961087e-08
2985 1.48361829488408e-08
2986 1.44564555881743e-08
2987 1.4582120400064e-08
2988 1.49133043692018e-08
2989 1.48165035795955e-08
2990 1.47621266322062e-08
2991 1.46273082535231e-08
2992 1.48366652297227e-08
2993 1.52349084459047e-08
2994 1.51426160499568e-08
2995 1.52148924570383e-08
2996 1.49845398311754e-08
2997 1.53131765046055e-08
2998 1.48125707255531e-08
2999 1.51578678497799e-08
3000 1.06411794931205e-08
3001 1.068157295947e-08
3002 1.068470911747e-08
3003 1.06847961589551e-08
3004 1.06903783603229e-08
3005 1.06893649487461e-08
3006 1.06886348660851e-08
3007 1.0687673857035e-08
3008 1.06862305671029e-08
3009 1.06859898707512e-08
3010 1.06839186386765e-08
3011 1.06825286394496e-08
3012 1.06817674705439e-08
3013 1.06810631450571e-08
3014 1.06805417843248e-08
3015 1.06794555421175e-08
3016 1.06791526732763e-08
3017 1.06783728526239e-08
3018 1.06766835372696e-08
3019 1.06775601693698e-08
3020 1.06770023933223e-08
3021 1.06756461448754e-08
3022 1.06735607019459e-08
3023 1.06749400430317e-08
3024 1.06739497240937e-08
3025 1.06711910419222e-08
3026 1.06724122872492e-08
3027 1.06699360458151e-08
3028 1.06676933953054e-08
3029 1.06693249790624e-08
3030 1.06674509225968e-08
3031 1.06675352995467e-08
3032 1.06657367382468e-08
3033 1.06663406995722e-08
3034 1.06651185660667e-08
3035 1.06658886167565e-08
3036 1.06644417741109e-08
3037 1.06654916010029e-08
3038 1.06638742281007e-08
3039 1.06646975694957e-08
3040 1.06637703112256e-08
3041 1.06626618645578e-08
3042 1.06636788288483e-08
3043 1.06630713148093e-08
3044 1.06624815643386e-08
3045 1.06617061845782e-08
3046 1.06625934748195e-08
3047 1.066328447763e-08
3048 1.06610551497965e-08
3049 1.0662349225754e-08
3050 1.06628332829928e-08
3051 1.06616830919393e-08
3052 1.06624504780939e-08
3053 1.06627418006155e-08
3054 1.06629807206104e-08
3055 1.0665352156991e-08
3056 1.06659570064949e-08
3057 1.06657029874668e-08
3058 1.06688320400394e-08
3059 1.06672848332323e-08
3060 1.0667785765861e-08
3061 1.0667017491528e-08
3062 1.06670601240921e-08
3063 1.06682653822077e-08
3064 1.06681001810216e-08
3065 1.06631770080412e-08
3066 1.06608721850421e-08
3067 1.06606057315162e-08
3068 1.06576711900175e-08
3069 1.06636903751678e-08
3070 1.06619832962451e-08
3071 1.06648787578933e-08
3072 1.06660662524405e-08
3073 1.06654090004099e-08
3074 1.06647544129146e-08
3075 1.06610427152987e-08
3076 1.06678541555993e-08
3077 1.06659072685034e-08
3078 1.06628306184575e-08
3079 1.06610436034771e-08
3080 1.0667918992624e-08
3081 1.06656452558695e-08
3082 1.06632063179291e-08
3083 1.06633057939121e-08
3084 1.06661213195025e-08
3085 1.06631734553275e-08
3086 1.06607433991712e-08
3087 1.06672706223776e-08
3088 1.06647251030267e-08
3089 1.06617381590013e-08
3090 1.06617763506733e-08
3091 1.06643440744847e-08
3092 1.06641806496555e-08
3093 1.06638955443827e-08
3094 1.06606172778356e-08
3095 1.06602655591814e-08
3096 1.06627187079766e-08
3097 1.06593365245544e-08
3098 1.06589679305102e-08
3099 1.06613837758118e-08
3100 1.06613029515756e-08
3101 1.0661238114551e-08
3102 1.06599085114567e-08
3103 1.06565787305613e-08
3104 1.06539754796131e-08
3105 1.06579056691203e-08
3106 1.06567696889215e-08
3107 1.06582875858408e-08
3108 1.06591047099869e-08
3109 1.06550972489572e-08
3110 1.06550022138663e-08
3111 1.06553059708858e-08
3112 1.06517896725222e-08
3113 1.06584661097031e-08
3114 1.06554018941551e-08
3115 1.06507993535843e-08
3116 1.06574402636284e-08
3117 1.06545599010133e-08
3118 1.06496766960618e-08
3119 1.06565005708603e-08
3120 1.06529505217168e-08
3121 1.06484110418137e-08
3122 1.06551203415961e-08
3123 1.06520818832223e-08
3124 1.06472333172292e-08
3125 1.06538458055638e-08
3126 1.06507611619122e-08
3127 1.06455333437339e-08
3128 1.06525455123574e-08
3129 1.06495123830541e-08
3130 1.06441913061417e-08
3131 1.06525970267057e-08
3132 1.06506412578256e-08
3133 1.06448494463507e-08
3134 1.065084109797e-08
3135 1.06490860574127e-08
3136 1.06480770867279e-08
3137 1.06470521288315e-08
3138 1.06498934115962e-08
3139 1.06477724415299e-08
3140 1.06461586213413e-08
3141 1.06450110948231e-08
3142 1.06407096467365e-08
3143 1.06469251193175e-08
3144 1.0646172832196e-08
3145 1.06424362655844e-08
3146 1.06401554234026e-08
3147 1.06437187952224e-08
3148 1.06423483359208e-08
3149 1.06396900179107e-08
3150 1.06385442677492e-08
3151 1.06427320289981e-08
3152 1.0639398695389e-08
3153 1.06369739683032e-08
3154 1.06409236977356e-08
3155 1.0637291047999e-08
3156 1.06316528913908e-08
3157 1.06398765353788e-08
3158 1.06390674048384e-08
3159 1.06330624305429e-08
3160 1.06373345687416e-08
3161 1.06339150818258e-08
3162 1.0631915792203e-08
3163 1.06349693496099e-08
3164 1.06327808779838e-08
3165 1.06273754241215e-08
3166 1.06314868020263e-08
3167 1.0632021485435e-08
3168 1.06280486633636e-08
3169 1.06307682656848e-08
3170 1.06262749710595e-08
3171 1.06317328274486e-08
3172 1.06278994493891e-08
3173 1.06342996630815e-08
3174 1.06323598814129e-08
3175 1.06266719868131e-08
3176 1.06272022293297e-08
3177 1.06223181361997e-08
3178 1.06269144595217e-08
3179 1.0629272573226e-08
3180 1.06233351004903e-08
3181 1.06254347542745e-08
3182 1.06267510346925e-08
3183 1.06271400568403e-08
3184 1.06198445593009e-08
3185 1.0627911883887e-08
3186 1.06236042185515e-08
3187 1.06218900342014e-08
3188 1.06224842255642e-08
3189 1.06187547643799e-08
3190 1.06231308194538e-08
3191 1.0619550572244e-08
3192 1.06246078601657e-08
3193 1.06157322932177e-08
3194 1.06202540095524e-08
3195 1.06207727057495e-08
3196 1.06226334395387e-08
3197 1.06178719150307e-08
3198 1.06200941374368e-08
3199 1.06194413262983e-08
3200 1.06193027704649e-08
3201 1.06214512740621e-08
3202 1.06156194945584e-08
3203 1.06168425162423e-08
3204 1.06193498439211e-08
3205 1.0619942258927e-08
3206 1.06150670475813e-08
3207 1.06192930005022e-08
3208 1.06169109059806e-08
3209 1.06173203562321e-08
3210 1.06181534675898e-08
3211 1.06144408817954e-08
3212 1.06170610081335e-08
3213 1.06127266974454e-08
3214 1.06144177891565e-08
3215 1.06151638590291e-08
3216 1.06145066069985e-08
3217 1.06147508560639e-08
3218 1.0614429335476e-08
3219 1.06119335541166e-08
3220 1.06135411570563e-08
3221 1.06133093424887e-08
3222 1.06109547814981e-08
3223 1.06127435728354e-08
3224 1.06124264931395e-08
3225 1.06100186414437e-08
3226 1.06101136765346e-08
3227 1.06089563800538e-08
3228 1.06092885587827e-08
3229 1.06080531026009e-08
3230 1.06084288020725e-08
3231 1.06073052563715e-08
3232 1.06076081252127e-08
3233 1.06064828031549e-08
3234 1.06070716654472e-08
3235 1.06095336960266e-08
3236 1.0603412370358e-08
3237 1.06049871106961e-08
3238 1.06028119617463e-08
3239 1.06040110026129e-08
3240 1.06011679434914e-08
3241 1.06014690359757e-08
3242 1.06014912404362e-08
3243 1.06019673040691e-08
3244 1.06003623656648e-08
3245 1.06047952641575e-08
3246 1.06004343081167e-08
3247 1.05989315102306e-08
3248 1.06011075473589e-08
3249 1.05992219445739e-08
3250 1.06032738145245e-08
3251 1.0600553324025e-08
3252 1.05980300091346e-08
3253 1.06018847034761e-08
3254 1.05972102204532e-08
3255 1.05987290055509e-08
3256 1.05975628272859e-08
3257 1.05959880869477e-08
3258 1.05973922970293e-08
3259 1.05963939844855e-08
3260 1.05971054153997e-08
3261 1.05967892238823e-08
3262 1.05921644788509e-08
3263 1.05963104957141e-08
3264 1.05955253459911e-08
3265 1.05941850847557e-08
3266 1.05953041895646e-08
3267 1.05951709628016e-08
3268 1.05932178584567e-08
3269 1.05926307725213e-08
3270 1.05932524974151e-08
3271 1.05919006898603e-08
3272 1.05930251237396e-08
3273 1.05912594250412e-08
3274 1.0591151067274e-08
3275 1.05909103709223e-08
3276 1.05927480120727e-08
3277 1.05931894367473e-08
3278 1.05893001034474e-08
3279 1.05911794889835e-08
3280 1.05908037895119e-08
3281 1.05930570981627e-08
3282 1.05900728186725e-08
3283 1.05901625246929e-08
3284 1.05908517511466e-08
3285 1.05891100332656e-08
3286 1.05913207093522e-08
3287 1.05878301681628e-08
3288 1.0588109944365e-08
3289 1.05889812473947e-08
3290 1.05886677204126e-08
3291 1.05894359947456e-08
3292 1.05850714859912e-08
3293 1.05846611475613e-08
3294 1.05850483933523e-08
3295 1.05835846753166e-08
3296 1.05853121823429e-08
3297 1.05839212949377e-08
3298 1.05811865935834e-08
3299 1.05824993212877e-08
3300 1.05858326548969e-08
3301 1.05804209837856e-08
3302 1.05833963814916e-08
3303 1.05814335071841e-08
3304 1.0585217147252e-08
3305 1.05784909720796e-08
3306 1.05806945427389e-08
3307 1.05823501073132e-08
3308 1.05817461459878e-08
3309 1.05784385695529e-08
3310 1.05784678794407e-08
3311 1.05811110984178e-08
3312 1.05803783512215e-08
3313 1.0575361919507e-08
3314 1.05781436943175e-08
3315 1.05761790436532e-08
3316 1.05792690163753e-08
3317 1.05766169156141e-08
3318 1.05760360469276e-08
3319 1.05765272095937e-08
3320 1.05742916645113e-08
3321 1.05747766099284e-08
3322 1.0576828302078e-08
3323 1.0574991549106e-08
3324 1.05779740522394e-08
3325 1.05721476018061e-08
3326 1.05729398569565e-08
3327 1.05787112403277e-08
3328 1.05710196152131e-08
3329 1.05728972243924e-08
3330 1.05720987519931e-08
3331 1.05725224130993e-08
3332 1.05685122875343e-08
3333 1.05757163026965e-08
3334 1.05731619015614e-08
3335 1.05692894436515e-08
3336 1.05691650986728e-08
3337 1.0572365205519e-08
3338 1.05681481343822e-08
3339 1.05684865303601e-08
3340 1.05713180431621e-08
3341 1.05673754191571e-08
3342 1.05682129714069e-08
3343 1.05682298467968e-08
3344 1.0569969788321e-08
3345 1.0567531738559e-08
3346 1.05651647430705e-08
3347 1.05684359041902e-08
3348 1.05655626470025e-08
3349 1.05663682248291e-08
3350 1.05674162753644e-08
3351 1.05635180602803e-08
3352 1.05647455228564e-08
3353 1.05633715108411e-08
3354 1.05637418812421e-08
3355 1.05620481249957e-08
3356 1.05649515802497e-08
3357 1.05613029433016e-08
3358 1.05623003676669e-08
3359 1.05645421299982e-08
3360 1.05602246947001e-08
3361 1.05591420052065e-08
3362 1.05622230961444e-08
3363 1.05592290466916e-08
3364 1.05634345715089e-08
3365 1.05608286560255e-08
3366 1.05573798592218e-08
3367 1.05626769553169e-08
3368 1.05600994615429e-08
3369 1.05562421026661e-08
3370 1.05559525565013e-08
3371 1.05572599551351e-08
3372 1.05575095332711e-08
3373 1.05558832785846e-08
3374 1.05506421377299e-08
3375 1.05593072063925e-08
3376 1.05558584095888e-08
3377 1.05548920714682e-08
3378 1.05555759688514e-08
3379 1.05527062643773e-08
3380 1.0556298946085e-08
3381 1.05547997009126e-08
3382 1.05570956421275e-08
3383 1.05541557715583e-08
3384 1.0553909746136e-08
3385 1.05554445184453e-08
3386 1.05523483284742e-08
3387 1.05515356452202e-08
3388 1.05519823989653e-08
3389 1.05549293749618e-08
3390 1.05507549363892e-08
3391 1.05535029604198e-08
3392 1.05532675931386e-08
3393 1.05486943624555e-08
3394 1.05500239655498e-08
3395 1.05527462324062e-08
3396 1.05512745207648e-08
3397 1.0549302764673e-08
3398 1.05496287261531e-08
3399 1.05509077030774e-08
3400 1.05474056155686e-08
3401 1.05490682855702e-08
3402 1.05480060241803e-08
3403 1.05443236364522e-08
3404 1.05479429635125e-08
3405 1.05455200127835e-08
3406 1.05460280508396e-08
3407 1.05437889530435e-08
3408 1.05514326165235e-08
3409 1.05426032348532e-08
3410 1.05448521026119e-08
3411 1.05451301024573e-08
3412 1.05428625829518e-08
3413 1.05434976305219e-08
3414 1.05434230235346e-08
3415 1.05411972484148e-08
3416 1.05428563657028e-08
3417 1.05434159181073e-08
3418 1.05405719708074e-08
3419 1.05414068585219e-08
3420 1.05414876827581e-08
3421 1.05402504502194e-08
3422 1.05426529728447e-08
3423 1.05379287518303e-08
3424 1.05399466931999e-08
3425 1.0541754136284e-08
3426 1.05379811543571e-08
3427 1.05399671213036e-08
3428 1.05403605843435e-08
3429 1.05354587276452e-08
3430 1.05360697943979e-08
3431 1.05364144076248e-08
3432 1.05411688267054e-08
3433 1.05367341518559e-08
3434 1.05386090964998e-08
3435 1.05341371181567e-08
3436 1.05352704338202e-08
3437 1.05344879486324e-08
3438 1.05325828059222e-08
3439 1.05337250033699e-08
3440 1.05333102240479e-08
3441 1.05313873177693e-08
3442 1.05316146914447e-08
3443 1.05320676624387e-08
3444 1.052914999633e-08
3445 1.05312309983674e-08
3446 1.05297557340123e-08
3447 1.05313304743504e-08
3448 1.05296766861329e-08
3449 1.0529985772223e-08
3450 1.05287503160412e-08
3451 1.05310959952476e-08
3452 1.05301616315501e-08
3453 1.05273629813496e-08
3454 1.05292778940225e-08
3455 1.05282698115161e-08
3456 1.05278834539035e-08
3457 1.05276649620123e-08
3458 1.05262536465034e-08
3459 1.0527154259421e-08
3460 1.05259179150607e-08
3461 1.05248325610319e-08
3462 1.05292867758067e-08
3463 1.05256816596011e-08
3464 1.05246531489911e-08
3465 1.05278745721193e-08
3466 1.05228492586207e-08
3467 1.05269677419528e-08
3468 1.05262518701466e-08
3469 1.05235740122112e-08
3470 1.05218154189402e-08
3471 1.0522646753941e-08
3472 1.05239106318322e-08
3473 1.05228190605544e-08
3474 1.05255137938798e-08
3475 1.05203081801619e-08
3476 1.05230562041925e-08
3477 1.05208384226785e-08
3478 1.05232826896895e-08
3479 1.05208615153174e-08
3480 1.05197948130353e-08
3481 1.0518841797591e-08
3482 1.05198063593548e-08
3483 1.05210924417065e-08
3484 1.05174589037915e-08
3485 1.05166728658901e-08
3486 1.05164028596505e-08
3487 1.05151478635435e-08
3488 1.05170059327975e-08
3489 1.05159410068723e-08
3490 1.05158441954245e-08
3491 1.05157598184746e-08
3492 1.05189679189266e-08
3493 1.05151691798255e-08
3494 1.05154951413056e-08
3495 1.0514694004371e-08
3496 1.0515794457433e-08
3497 1.05136273020889e-08
3498 1.05135526951017e-08
3499 1.05122985871731e-08
3500 1.05131734429165e-08
3501 1.05113162618409e-08
3502 1.05101811698205e-08
3503 1.05143218576131e-08
3504 1.05102948566582e-08
3505 1.05098401093073e-08
3506 1.05120019355809e-08
3507 1.05109512205104e-08
3508 1.05090007807007e-08
3509 1.05086446211544e-08
3510 1.05110542492071e-08
3511 1.05081152668163e-08
3512 1.05110462556013e-08
3513 1.05102238023846e-08
3514 1.05104422942759e-08
3515 1.05085566914909e-08
3516 1.05067510247636e-08
3517 1.05074766665325e-08
3518 1.05066737532411e-08
3519 1.05040545150814e-08
3520 1.05083062251765e-08
3521 1.05044950515776e-08
3522 1.0502637870502e-08
3523 1.05049897669574e-08
3524 1.05036015440874e-08
3525 1.0502079206276e-08
3526 1.05033031161383e-08
3527 1.05032640362879e-08
3528 1.05016502160993e-08
3529 1.05043111986447e-08
3530 1.05022843754909e-08
3531 1.05040571796167e-08
3532 1.04989723581639e-08
3533 1.05042508025122e-08
3534 1.04992041727314e-08
3535 1.05005906192446e-08
3536 1.04984847482115e-08
3537 1.05025206309506e-08
3538 1.05005248940415e-08
3539 1.04984740900704e-08
3540 1.05006616735182e-08
3541 1.05007025297255e-08
3542 1.04926751731682e-08
3543 1.04993596039549e-08
3544 1.04926600741351e-08
3545 1.04967581293636e-08
3546 1.04966710878784e-08
3547 1.04905932829524e-08
3548 1.04944435364018e-08
3549 1.04944977152854e-08
3550 1.04919122279057e-08
3551 1.0495017299661e-08
3552 1.04964552605225e-08
3553 1.0488230728356e-08
3554 1.04906296982676e-08
3555 1.04916733079108e-08
3556 1.04931032751665e-08
3557 1.04934150257918e-08
3558 1.04883435270153e-08
3559 1.04883284279822e-08
3560 1.0490304624966e-08
3561 1.04892583507876e-08
3562 1.04858504101912e-08
3563 1.04885709006908e-08
3564 1.04876747286653e-08
3565 1.04899990915897e-08
3566 1.04868220773824e-08
3567 1.04873114636916e-08
3568 1.04834780856322e-08
3569 1.04860200522694e-08
3570 1.04856123783748e-08
3571 1.04841726411564e-08
3572 1.04838919767758e-08
3573 1.04867110550799e-08
3574 1.04826503033451e-08
3575 1.04855635285617e-08
3576 1.04831334724054e-08
3577 1.04828616898089e-08
3578 1.04839470438378e-08
3579 1.04841184622728e-08
3580 1.04870014894232e-08
3581 1.04850128579415e-08
3582 1.04817621249254e-08
3583 1.04797992506178e-08
3584 1.04792237110019e-08
3585 1.04825472746484e-08
3586 1.04785060628387e-08
3587 1.04826138880298e-08
3588 1.0479211276504e-08
3589 1.04808606238294e-08
3590 1.04765529584938e-08
3591 1.04765405239959e-08
3592 1.04795363498056e-08
3593 1.04766435526926e-08
3594 1.04781916476782e-08
3595 1.04801127776e-08
3596 1.04743982376476e-08
3597 1.04778541398787e-08
3598 1.04759569907742e-08
3599 1.04746220586094e-08
3600 1.04768425046586e-08
3601 1.04769775077784e-08
3602 1.04780957244088e-08
3603 1.04737445383307e-08
3604 1.04718855808983e-08
3605 1.04725090821489e-08
3606 1.04754480645397e-08
3607 1.04711270765279e-08
3608 1.04771009645788e-08
3609 1.04716368909408e-08
3610 1.04771507025703e-08
3611 1.04680406565194e-08
3612 1.04711892490172e-08
3613 1.04755857321948e-08
3614 1.04685602408949e-08
3615 1.04749489082678e-08
3616 1.047281195099e-08
3617 1.04686082025296e-08
3618 1.04703987702237e-08
3619 1.04700239589306e-08
3620 1.04706128212229e-08
3621 1.04623394392434e-08
3622 1.04728501426621e-08
3623 1.04725010885431e-08
3624 1.04702113645772e-08
3625 1.04657580379808e-08
3626 1.04658468558227e-08
3627 1.04723270055729e-08
3628 1.04695665470445e-08
3629 1.04666462164005e-08
3630 1.04687263302594e-08
3631 1.04662820632484e-08
3632 1.04690167646027e-08
3633 1.04675104140028e-08
3634 1.04695150326961e-08
3635 1.04622763785756e-08
3636 1.04608952611329e-08
3637 1.04652775334557e-08
3638 1.04665485167743e-08
3639 1.04607398299095e-08
3640 1.04623056884634e-08
3641 1.0461149280161e-08
3642 1.04622133179078e-08
3643 1.04622932539655e-08
3644 1.04589927829579e-08
3645 1.04629549468882e-08
3646 1.04633297581813e-08
3647 1.04645314635832e-08
3648 1.0463045541087e-08
3649 1.04612141171856e-08
3650 1.04625010877157e-08
3651 1.04618402829715e-08
3652 1.04575583748101e-08
3653 1.04591588723224e-08
3654 1.04604378492468e-08
3655 1.04634079178823e-08
3656 1.0460015076319e-08
3657 1.04559427782647e-08
3658 1.04607291717684e-08
3659 1.04623811836291e-08
3660 1.04571338255255e-08
3661 1.04551522994711e-08
3662 1.0456635557432e-08
3663 1.0457382515483e-08
3664 1.0456093768596e-08
3665 1.04584856330803e-08
3666 1.04555200053369e-08
3667 1.04519513044465e-08
3668 1.0454955123862e-08
3669 1.04530286648696e-08
3670 1.04530721856122e-08
3671 1.0457570809308e-08
3672 1.04533279809971e-08
3673 1.04531006073216e-08
3674 1.04567279279877e-08
3675 1.04540678336207e-08
3676 1.04565778258348e-08
3677 1.04497388520031e-08
3678 1.04532249523004e-08
3679 1.04509156884092e-08
3680 1.0445724285546e-08
3681 1.04511519438688e-08
3682 1.04486490570821e-08
3683 1.04463877548255e-08
3684 1.0449905829546e-08
3685 1.0450077247981e-08
3686 1.04461657102206e-08
3687 1.04479980223005e-08
3688 1.04393418354221e-08
3689 1.04411714829666e-08
3690 1.04487476448867e-08
3691 1.04487032359657e-08
3692 1.04490620600473e-08
3693 1.04521697963378e-08
3694 1.04467350325876e-08
3695 1.04427568814458e-08
3696 1.04435775583056e-08
3697 1.04474144890787e-08
3698 1.04414592527746e-08
3699 1.04447375193217e-08
3700 1.045039255132e-08
3701 1.04444772830448e-08
3702 1.04441237880337e-08
3703 1.04438724335409e-08
3704 1.04464135119997e-08
3705 1.04450759152996e-08
3706 1.04431299163821e-08
3707 1.04419761726149e-08
3708 1.04480060159062e-08
3709 1.04431592262699e-08
3710 1.04390069921578e-08
3711 1.0438336417451e-08
3712 1.04394271005503e-08
3713 1.04356399077687e-08
3714 1.04437889447695e-08
3715 1.0439900499648e-08
3716 1.04382635868205e-08
3717 1.04403570233558e-08
3718 1.04362500863431e-08
3719 1.04390354138673e-08
3720 1.04402415601612e-08
3721 1.04352899654714e-08
3722 1.04389030752827e-08
3723 1.04396535860474e-08
3724 1.04393995670193e-08
3725 1.04406057133133e-08
3726 1.04352873009361e-08
3727 1.04362527508783e-08
3728 1.04358086616685e-08
3729 1.04392814392895e-08
3730 1.04394386468698e-08
3731 1.04358237607016e-08
3732 1.0437462449886e-08
3733 1.04375050824501e-08
3734 1.04399253686438e-08
3735 1.04374668907781e-08
3736 1.04374278109276e-08
3737 1.04331405736957e-08
3738 1.04290736047119e-08
3739 1.04320960758741e-08
3740 1.04342943174629e-08
3741 1.04359534347509e-08
3742 1.04301038916788e-08
3743 1.04344080043006e-08
3744 1.04345954099472e-08
3745 1.04320942995173e-08
3746 1.04360786679081e-08
3747 1.042823161157e-08
3748 1.04303428116737e-08
3749 1.0428015784214e-08
3750 1.04282245061427e-08
3751 1.04290869273882e-08
3752 1.04329336281239e-08
3753 1.04279207491231e-08
3754 1.04255457600289e-08
3755 1.04312567472675e-08
3756 1.04262518618725e-08
3757 1.04282351642837e-08
3758 1.04313855331384e-08
3759 1.04261657085658e-08
3760 1.04241673071215e-08
3761 1.04294288760798e-08
3762 1.04238839782056e-08
3763 1.04266675293729e-08
3764 1.04252517729719e-08
3765 1.04252091404078e-08
3766 1.0427816832248e-08
3767 1.04279100909821e-08
3768 1.0426290941723e-08
3769 1.04244071152948e-08
3770 1.04228101704962e-08
3771 1.04242641185692e-08
3772 1.04232444897434e-08
3773 1.04253476962413e-08
3774 1.04216981711147e-08
3775 1.04239950005081e-08
3776 1.04212460882991e-08
3777 1.04254773702905e-08
3778 1.0422314566938e-08
3779 1.04221866692455e-08
3780 1.04189048499848e-08
3781 1.04177786397486e-08
3782 1.04172226400578e-08
3783 1.04181063775854e-08
3784 1.0417104512328e-08
3785 1.04174331383433e-08
3786 1.04186881344503e-08
3787 1.04235953202192e-08
3788 1.04224460173441e-08
3789 1.04179838089635e-08
3790 1.0414715312379e-08
3791 1.041683361791e-08
3792 1.04151496316263e-08
3793 1.04144239898574e-08
3794 1.04185504667953e-08
3795 1.04143440537996e-08
3796 1.04191579808344e-08
3797 1.04180406523824e-08
3798 1.04151878232983e-08
3799 1.04143422774428e-08
3800 1.04126902655821e-08
3801 1.04200230666152e-08
3802 1.04117630073119e-08
3803 1.0416679074865e-08
3804 1.04125623678897e-08
3805 1.04123145661106e-08
3806 1.04119406429959e-08
3807 1.04110196019747e-08
3808 1.04126440803043e-08
3809 1.04142641177418e-08
3810 1.0412641415769e-08
3811 1.04121502531029e-08
3812 1.04106350207189e-08
3813 1.0409837436498e-08
3814 1.04091393282602e-08
3815 1.04099395770163e-08
3816 1.04091819608243e-08
3817 1.04084838525864e-08
3818 1.04111856913391e-08
3819 1.04098516473528e-08
3820 1.04118162980171e-08
3821 1.04087920504981e-08
3822 1.04060484673596e-08
3823 1.04090549513103e-08
3824 1.04064357131506e-08
3825 1.04067456874191e-08
3826 1.04046105064981e-08
3827 1.04038875292645e-08
3828 1.04088959673732e-08
3829 1.04074624474038e-08
3830 1.04050421612101e-08
3831 1.04072359619067e-08
3832 1.0402017913691e-08
3833 1.040127095564e-08
3834 1.04030561942636e-08
3835 1.04011892432254e-08
3836 1.04062563011098e-08
3837 1.04067696682364e-08
3838 1.0405545758374e-08
3839 1.04010933199561e-08
3840 1.04055146721294e-08
3841 1.04031698811013e-08
3842 1.04003685663656e-08
3843 1.04040349668821e-08
3844 1.03995692057879e-08
3845 1.0398356842245e-08
3846 1.04018793578575e-08
3847 1.03987183308618e-08
3848 1.03989732380683e-08
3849 1.03993427202909e-08
3850 1.039959052207e-08
3851 1.04013730961583e-08
3852 1.03989883371014e-08
3853 1.03960760000632e-08
3854 1.03992388034158e-08
3855 1.03999644451847e-08
3856 1.03961248498763e-08
3857 1.04006181445015e-08
3858 1.03988790911558e-08
3859 1.03994173272781e-08
3860 1.04026094405185e-08
3861 1.03937356499273e-08
3862 1.03954160834974e-08
3863 1.03944932661193e-08
3864 1.03992530142705e-08
3865 1.03988790911558e-08
3866 1.03935011708245e-08
3867 1.03919139959885e-08
3868 1.0392848359686e-08
3869 1.03999147071931e-08
3870 1.03929505002043e-08
3871 1.03905959392137e-08
3872 1.03930606343283e-08
3873 1.03919548521958e-08
3874 1.03907282777982e-08
3875 1.03902824122315e-08
3876 1.03926094396911e-08
3877 1.03963255781991e-08
3878 1.03908455173496e-08
3879 1.03899493453241e-08
3880 1.03892503489078e-08
3881 1.03882360491525e-08
3882 1.03890505087634e-08
3883 1.03907478177234e-08
3884 1.03862962674839e-08
3885 1.03891331093564e-08
3886 1.03937782824914e-08
3887 1.03923145644558e-08
3888 1.03894501890522e-08
3889 1.03895443359647e-08
3890 1.03854889133004e-08
3891 1.03849009391865e-08
3892 1.0385940996116e-08
3893 1.03834167930472e-08
3894 1.03856159228144e-08
3895 1.0384857418444e-08
3896 1.0383524262636e-08
3897 1.03868877943114e-08
3898 1.03856452327022e-08
3899 1.03847401788926e-08
3900 1.03819415286921e-08
3901 1.03862785039155e-08
3902 1.03820232411067e-08
3903 1.03828865505307e-08
3904 1.03825072983454e-08
3905 1.03831672149113e-08
3906 1.03823065700226e-08
3907 1.03840047671611e-08
3908 1.03865573919393e-08
3909 1.03809725260362e-08
3910 1.03808677209827e-08
3911 1.03819060015553e-08
3912 1.03847357380005e-08
3913 1.038541608267e-08
3914 1.03797033190745e-08
3915 1.03797734851696e-08
3916 1.037950347893e-08
3917 1.03839754572732e-08
3918 1.03789048466751e-08
3919 1.03780752880311e-08
3920 1.03779012050609e-08
3921 1.03775406046225e-08
3922 1.03748805102555e-08
3923 1.03782955562792e-08
3924 1.03767705539326e-08
3925 1.0377402048789e-08
3926 1.03805231077558e-08
3927 1.03801376383217e-08
3928 1.03778150517542e-08
3929 1.03733750478341e-08
3930 1.03772777038103e-08
3931 1.0376346004648e-08
3932 1.03708499565869e-08
3933 1.03723474254025e-08
3934 1.03771844450762e-08
3935 1.03749284718901e-08
3936 1.03704476117628e-08
3937 1.03734292267177e-08
3938 1.0372424696925e-08
3939 1.03741255585987e-08
3940 1.03719237642963e-08
3941 1.03701705000958e-08
3942 1.03727257894093e-08
3943 1.03707566978528e-08
3944 1.0371810965637e-08
3945 1.03735899870117e-08
3946 1.03738546641807e-08
3947 1.03723074573736e-08
3948 1.03706403464798e-08
3949 1.03677324503337e-08
3950 1.03720507738103e-08
3951 1.0369576308733e-08
3952 1.03707664678154e-08
3953 1.03688000407942e-08
3954 1.03720676492003e-08
3955 1.03713784227466e-08
3956 1.03706696563677e-08
3957 1.03659143491086e-08
3958 1.03693000852445e-08
3959 1.03694164366175e-08
3960 1.03673878371069e-08
3961 1.03677164631222e-08
3962 1.03659267836065e-08
3963 1.03664667960857e-08
3964 1.03628137182454e-08
3965 1.03683976959701e-08
3966 1.03625641401095e-08
3967 1.03631139225513e-08
3968 1.03631103698376e-08
3969 1.03637152193414e-08
3970 1.0364057168033e-08
3971 1.03645296789523e-08
3972 1.03595318989846e-08
3973 1.03628838843406e-08
3974 1.03661168537883e-08
3975 1.03658628347603e-08
3976 1.03603712275913e-08
3977 1.03630215519956e-08
3978 1.03609192336762e-08
3979 1.03617505686771e-08
3980 1.03605026779974e-08
3981 1.03610577895097e-08
3982 1.03566195619464e-08
3983 1.03578088328504e-08
3984 1.03592592282098e-08
3985 1.03611546009574e-08
3986 1.03598383205394e-08
3987 1.03590762634553e-08
3988 1.03580362065259e-08
3989 1.03619681723899e-08
3990 1.03554942398887e-08
3991 1.03581898613925e-08
3992 1.03567749931699e-08
3993 1.03581800914299e-08
3994 1.03542117102506e-08
3995 1.03595798606193e-08
3996 1.03546966556678e-08
3997 1.03549986363305e-08
3998 1.03557855624103e-08
3999 1.03558468467213e-08
4000 1.03556851982489e-08
4001 1.03583550625785e-08
4002 1.03545652052617e-08
4003 1.03553929875488e-08
4004 1.03537525220077e-08
4005 1.03563388975658e-08
4006 1.03489776748233e-08
4007 1.03544648411003e-08
4008 1.03523483119261e-08
4009 1.03538813078785e-08
4010 1.03521884398106e-08
4011 1.03531512252175e-08
4012 1.03505364279499e-08
4013 1.03486872404801e-08
4014 1.03557376007757e-08
4015 1.03487076685838e-08
4016 1.03527533212855e-08
4017 1.03518695837579e-08
4018 1.03501216486279e-08
4019 1.03499209203051e-08
4020 1.03517674432396e-08
4021 1.03521795580264e-08
4022 1.03527302286466e-08
4023 1.03531387907196e-08
4024 1.0350496459921e-08
4025 1.03500861214911e-08
4026 1.03506545556797e-08
4027 1.03479980140264e-08
4028 1.03478123847367e-08
4029 1.0348024659379e-08
4030 1.0345398315792e-08
4031 1.03464934397834e-08
4032 1.0348566448215e-08
4033 1.03519264271767e-08
4034 1.03499520065498e-08
4035 1.03473043466806e-08
4036 1.0341929979063e-08
4037 1.03436645915167e-08
4038 1.03458397404665e-08
4039 1.0348130352611e-08
4040 1.03485344737919e-08
4041 1.03447428401182e-08
4042 1.03431281317512e-08
4043 1.03450714661335e-08
4044 1.034183050308e-08
4045 1.03462705070001e-08
4046 1.03412132190783e-08
4047 1.03422639341488e-08
4048 1.0339303635476e-08
4049 1.03407282736612e-08
4050 1.03398862805193e-08
4051 1.03432098441658e-08
4052 1.03389243832908e-08
4053 1.03421324837427e-08
4054 1.03393276162933e-08
4055 1.03412221008625e-08
4056 1.03395727535371e-08
4057 1.03395931816408e-08
4058 1.03433368536798e-08
4059 1.03400745743443e-08
4060 1.03434940612601e-08
4061 1.03376986970716e-08
4062 1.03362864933843e-08
4063 1.03396118333876e-08
4064 1.03367394643783e-08
4065 1.03376365245822e-08
4066 1.03430339848387e-08
4067 1.03386943450801e-08
4068 1.03360378034267e-08
4069 1.03380397575847e-08
4070 1.03389146133281e-08
4071 1.03410977558838e-08
4072 1.03393631434301e-08
4073 1.03392725492313e-08
4074 1.03350590308082e-08
4075 1.03344941493333e-08
4076 1.03359658609747e-08
4077 1.03346451396646e-08
4078 1.03325863420878e-08
4079 1.03331254663885e-08
4080 1.03339425905347e-08
4081 1.03316377675355e-08
4082 1.03311030841269e-08
4083 1.03319468536256e-08
4084 1.03302992826571e-08
4085 1.03338182455559e-08
4086 1.03300061837786e-08
4087 1.03303685605738e-08
4088 1.03287316477463e-08
4089 1.03316928345976e-08
4090 1.03327453260249e-08
4091 1.03262181028185e-08
4092 1.03301571741099e-08
4093 1.03300843434795e-08
4094 1.03280752838941e-08
4095 1.03276196483648e-08
4096 1.03304014231753e-08
4097 1.03265298534438e-08
4098 1.03295105802204e-08
4099 1.03309565346876e-08
4100 1.03223722902612e-08
4101 1.03316004640419e-08
4102 1.03274304663614e-08
4103 1.03287991493062e-08
4104 1.03264348183529e-08
4105 1.03294013342747e-08
4106 1.03208668278398e-08
4107 1.03232906667472e-08
4108 1.03261799111465e-08
4109 1.03242978610751e-08
4110 1.03274970797429e-08
4111 1.03290975772552e-08
4112 1.03275308305228e-08
4113 1.03257908889987e-08
4114 1.03234141235475e-08
4115 1.03217860925042e-08
4116 1.0321954846404e-08
4117 1.03229895742629e-08
4118 1.03205302082188e-08
4119 1.03240340720845e-08
4120 1.03212789426266e-08
4121 1.03218802394167e-08
4122 1.03250377136987e-08
4123 1.03204955692604e-08
4124 1.03253761096767e-08
4125 1.0322048105138e-08
4126 1.03197104195374e-08
4127 1.0318602861048e-08
4128 1.03209609747523e-08
4129 1.03211057478347e-08
4130 1.03202379975187e-08
4131 1.03245776372773e-08
4132 1.03194350842273e-08
4133 1.0320168719602e-08
4134 1.03165911369274e-08
4135 1.03212007829256e-08
4136 1.0318752963201e-08
4137 1.03174464527456e-08
4138 1.03190220812621e-08
4139 1.03130055606471e-08
4140 1.03155812780642e-08
4141 1.03148574126521e-08
4142 1.03164961018365e-08
4143 1.03155475272843e-08
4144 1.03119575101118e-08
4145 1.03126396311382e-08
4146 1.03123207750855e-08
4147 1.03112061111688e-08
4148 1.03150119556972e-08
4149 1.03131405637669e-08
4150 1.0309737952241e-08
4151 1.03128865447388e-08
4152 1.03140900264975e-08
4153 1.03101216453183e-08
4154 1.03110142646301e-08
4155 1.03097068659963e-08
4156 1.03090913583515e-08
4157 1.03091641889819e-08
4158 1.03080379787457e-08
4159 1.03080992630566e-08
4160 1.03091357672724e-08
4161 1.03078106050702e-08
4162 1.03082928859521e-08
4163 1.03071506885044e-08
4164 1.03074748736276e-08
4165 1.03070760815172e-08
4166 1.03086783553863e-08
4167 1.03115720406777e-08
4168 1.0306176356778e-08
4169 1.03061328360354e-08
4170 1.03074278001714e-08
4171 1.03064730083702e-08
4172 1.03090957992436e-08
4173 1.03073460877567e-08
4174 1.03047179678128e-08
4175 1.0310008846659e-08
4176 1.03070325607746e-08
4177 1.03102326676208e-08
4178 1.03039665688698e-08
4179 1.0305014619405e-08
4180 1.03068522605554e-08
4181 1.03008597207577e-08
4182 1.0306889564049e-08
4183 1.0303615738394e-08
4184 1.03044017762954e-08
4185 1.03095718628765e-08
4186 1.03021662312131e-08
4187 1.03055821654152e-08
4188 1.03027328890448e-08
4189 1.03039718979403e-08
4190 1.03045412203073e-08
4191 1.03045376675936e-08
4192 1.03042721022462e-08
4193 1.03043689136939e-08
4194 1.03000727946778e-08
4195 1.03008845897534e-08
4196 1.03046406962903e-08
4197 1.03041735144416e-08
4198 1.03022159692046e-08
4199 1.03013677588137e-08
4200 1.0298748520654e-08
4201 1.02978132687781e-08
4202 1.02966115633762e-08
4203 1.03012514074408e-08
4204 1.03018313879488e-08
4205 1.0295038599395e-08
4206 1.02985007188749e-08
4207 1.02980646232709e-08
4208 1.0294778363118e-08
4209 1.03017727681731e-08
4210 1.02960839853949e-08
4211 1.0296735908355e-08
4212 1.02973718441035e-08
4213 1.02935633350398e-08
4214 1.02938759738436e-08
4215 1.02952748548546e-08
4216 1.02935207024757e-08
4217 1.02952117941868e-08
4218 1.0297404706705e-08
4219 1.02951274172369e-08
4220 1.02954897940322e-08
4221 1.02977564253592e-08
4222 1.02930286516312e-08
4223 1.02898276566066e-08
4224 1.02975086235801e-08
4225 1.02953165992403e-08
4226 1.02893435993678e-08
4227 1.02937249835122e-08
4228 1.02903543464095e-08
4229 1.02898471965318e-08
4230 1.02946771107781e-08
4231 1.02893755737909e-08
4232 1.02913801924842e-08
4233 1.02966755122225e-08
4234 1.02925739042803e-08
4235 1.02946318136787e-08
4236 1.02945341140526e-08
4237 1.02884971653339e-08
4238 1.02876640539762e-08
4239 1.02886055231011e-08
4240 1.02885264752217e-08
4241 1.02900230558589e-08
4242 1.02947819158317e-08
4243 1.02881534402854e-08
4244 1.02908161991877e-08
4245 1.0286294482853e-08
4246 1.02853698891181e-08
4247 1.02856034800425e-08
4248 1.02854826877774e-08
4249 1.02829114112524e-08
4250 1.02878283669838e-08
4251 1.02853139338777e-08
4252 1.02854409433917e-08
4253 1.02909787358385e-08
4254 1.02817310221326e-08
4255 1.02827888426305e-08
4256 1.02817798719457e-08
4257 1.02816617442159e-08
4258 1.02856665407103e-08
4259 1.02819273095633e-08
4260 1.02834230020221e-08
4261 1.02793284995073e-08
4262 1.02813917379763e-08
4263 1.0287826590627e-08
4264 1.02823420888853e-08
4265 1.027753970817e-08
4266 1.02819042169244e-08
4267 1.0282002804729e-08
4268 1.0279105566724e-08
4269 1.02759152298404e-08
4270 1.02773318744198e-08
4271 1.02780512989398e-08
4272 1.02794777134818e-08
4273 1.02820836289652e-08
4274 1.02795505441122e-08
4275 1.02788240141649e-08
4276 1.02785433497843e-08
4277 1.02784252220545e-08
4278 1.02810391311436e-08
4279 1.02759507569772e-08
4280 1.02796873235889e-08
4281 1.02834292192711e-08
4282 1.02776889221445e-08
4283 1.02735127072151e-08
4284 1.02764587950332e-08
4285 1.02806181345727e-08
4286 1.0278231599159e-08
4287 1.02726707140732e-08
4288 1.02708899163417e-08
4289 1.02690442815856e-08
4290 1.02720232320053e-08
4291 1.02735473461735e-08
4292 1.02694297510197e-08
4293 1.02690709269382e-08
4294 1.02713109129127e-08
4295 1.02722061967597e-08
4296 1.02740402851964e-08
4297 1.02698738402296e-08
4298 1.02728190398693e-08
4299 1.02684021285881e-08
4300 1.02681383395975e-08
4301 1.02719956984743e-08
4302 1.02674979629569e-08
4303 1.02658930245525e-08
4304 1.02668513690674e-08
4305 1.02668558099595e-08
4306 1.0265384098318e-08
4307 1.0265566174894e-08
4308 1.02671773305474e-08
4309 1.02640607124727e-08
4310 1.02630330900411e-08
4311 1.02644479582636e-08
4312 1.02646016131303e-08
4313 1.02635162591014e-08
4314 1.02661159573358e-08
4315 1.02606882990131e-08
4316 1.02651975808499e-08
4317 1.02621537934056e-08
4318 1.02612141006375e-08
4319 1.02614396979561e-08
4320 1.02616084518559e-08
4321 1.0259809890556e-08
4322 1.02612425223469e-08
4323 1.02618953334854e-08
4324 1.02644319710521e-08
4325 1.02587343064897e-08
4326 1.02617319086562e-08
4327 1.02665804746493e-08
4328 1.02606314555942e-08
4329 1.02605586249638e-08
4330 1.02624690967446e-08
4331 1.02608339602739e-08
4332 1.02657073952628e-08
4333 1.02583763705866e-08
4334 1.02555066661125e-08
4335 1.02536059642944e-08
4336 1.02556825254396e-08
4337 1.02529487122638e-08
4338 1.02560608894464e-08
4339 1.02584145622586e-08
4340 1.02541042323878e-08
4341 1.02548742830777e-08
4342 1.02556647618712e-08
4343 1.02591393158491e-08
4344 1.02565902437846e-08
4345 1.02556825254396e-08
4346 1.02505843813105e-08
4347 1.02504182919461e-08
4348 1.02538653123929e-08
4349 1.02519877032137e-08
4350 1.02504973398254e-08
4351 1.02484287722859e-08
4352 1.02507105026461e-08
4353 1.02536921176011e-08
4354 1.0256978377754e-08
4355 1.0251524962257e-08
4356 1.02514396971287e-08
4357 1.02527515366546e-08
4358 1.02525063994108e-08
4359 1.02512736077642e-08
4360 1.02474269070285e-08
4361 1.02448236560804e-08
4362 1.02459853934533e-08
4363 1.02448645122877e-08
4364 1.02427666348603e-08
4365 1.02472048624236e-08
4366 1.02453547867754e-08
4367 1.02461736872783e-08
4368 1.02481561015111e-08
4369 1.02438200144661e-08
4370 1.02467421214669e-08
4371 1.0246221648913e-08
4372 1.02424246861688e-08
4373 1.02443271643438e-08
4374 1.02390460554602e-08
4375 1.02421502390371e-08
4376 1.02401305213107e-08
4377 1.02440607108178e-08
4378 1.02419264180753e-08
4379 1.0247195092461e-08
4380 1.02420658620872e-08
4381 1.02438733051713e-08
4382 1.02417549996403e-08
4383 1.0241443249015e-08
4384 1.02350430353226e-08
4385 1.02373727273175e-08
4386 1.02410915303608e-08
4387 1.02406216839768e-08
4388 1.02376338517729e-08
4389 1.02409041247142e-08
4390 1.0236908210004e-08
4391 1.02407486934908e-08
4392 1.02387840428264e-08
4393 1.02371222610032e-08
4394 1.02398072243659e-08
4395 1.02418038494534e-08
4396 1.02394066558986e-08
4397 1.0238783154648e-08
4398 1.0235780223411e-08
4399 1.02359143383524e-08
4400 1.02398809431747e-08
4401 1.02362314180482e-08
4402 1.02390274037134e-08
4403 1.02357002873532e-08
4404 1.0235382319479e-08
4405 1.0235337910558e-08
4406 1.02362642806497e-08
4407 1.02350217190406e-08
4408 1.02387520684033e-08
4409 1.02364001719479e-08
4410 1.02375725674619e-08
4411 1.02352579745002e-08
4412 1.0233778269253e-08
4413 1.02304245075402e-08
4414 1.02349488884101e-08
4415 1.02316848327177e-08
4416 1.02296082715725e-08
4417 1.02291455306158e-08
4418 1.02273194357849e-08
4419 1.0226987257056e-08
4420 1.02269099855334e-08
4421 1.02248005617867e-08
4422 1.02299990700772e-08
4423 1.02296437987093e-08
4424 1.02290753645207e-08
4425 1.0227683588937e-08
4426 1.02299502202641e-08
4427 1.02252766254196e-08
4428 1.02238768562302e-08
4429 1.02249266831222e-08
4430 1.02230401921588e-08
4431 1.0222219515299e-08
4432 1.02229815723831e-08
4433 1.02219237518852e-08
4434 1.02271027202505e-08
4435 1.02253672196184e-08
4436 1.02250421463168e-08
4437 1.02210018226856e-08
4438 1.02251922484697e-08
4439 1.02206829666329e-08
4440 1.02206678675998e-08
4441 1.02182626804392e-08
4442 1.02215249597748e-08
4443 1.02177208916032e-08
4444 1.02221013875692e-08
4445 1.02177413197069e-08
4446 1.02169286364528e-08
4447 1.02184118944137e-08
4448 1.02170565341453e-08
4449 1.02163371096253e-08
4450 1.02169668281249e-08
4451 1.02193151718666e-08
4452 1.02148929315149e-08
4453 1.0220676749384e-08
4454 1.02168531412872e-08
4455 1.02189448014656e-08
4456 1.02188764117273e-08
4457 1.02168851157103e-08
4458 1.02206572094588e-08
4459 1.02171338056678e-08
4460 1.02208632668521e-08
4461 1.02166266557902e-08
4462 1.02126600509678e-08
4463 1.0213022427763e-08
4464 1.02137791557766e-08
4465 1.02122692524631e-08
4466 1.02171204829915e-08
4467 1.0214825429955e-08
4468 1.02093196119313e-08
4469 1.02115560451921e-08
4470 1.02115711442252e-08
4471 1.02145181202218e-08
4472 1.0211634204893e-08
4473 1.02090389475507e-08
4474 1.02138200119839e-08
4475 1.02113979494334e-08
4476 1.02097716947469e-08
4477 1.02105648380757e-08
4478 1.02120463196798e-08
4479 1.02124309009355e-08
4480 1.02099777521403e-08
4481 1.02106554322745e-08
4482 1.02097565957138e-08
4483 1.02065005336271e-08
4484 1.02093196119313e-08
4485 1.02071355811972e-08
4486 1.02091473053179e-08
4487 1.02103143717613e-08
4488 1.0205717160261e-08
4489 1.02116208822167e-08
4490 1.02067554408336e-08
4491 1.02015667025057e-08
4492 1.02021164849475e-08
4493 1.02026831427793e-08
4494 1.02034460880418e-08
4495 1.02068531404598e-08
4496 1.02028101522933e-08
4497 1.02037418514556e-08
4498 1.02015658143273e-08
4499 1.02059676265753e-08
4500 1.02052286621301e-08
4501 1.02056496587011e-08
4502 1.02032053916901e-08
4503 1.02035491167385e-08
4504 1.02008250735253e-08
4505 1.01986969980317e-08
4506 1.01979651390138e-08
4507 1.01959294340759e-08
4508 1.02013277825108e-08
4509 1.02003738788881e-08
4510 1.02005195401489e-08
4511 1.01990496048643e-08
4512 1.01980273115032e-08
4513 1.01956523224089e-08
4514 1.0195512878397e-08
4515 1.02006811886213e-08
4516 1.0197658717459e-08
4517 1.01930721640997e-08
4518 1.0191022248307e-08
4519 1.01940846874982e-08
4520 1.01971595611872e-08
4521 1.01988471001846e-08
4522 1.01958521625534e-08
4523 1.01915302863631e-08
4524 1.01924193529612e-08
4525 1.01907584593164e-08
4526 1.01962891463359e-08
4527 1.01955066611481e-08
4528 1.01915382799689e-08
4529 1.0197514832555e-08
4530 1.01951806996681e-08
4531 1.01880290870326e-08
4532 1.01880841540947e-08
4533 1.01860333501236e-08
4534 1.01881481029409e-08
4535 1.01866044488474e-08
4536 1.01874748636988e-08
4537 1.01919903627845e-08
4538 1.01879953362527e-08
4539 1.01924904072348e-08
4540 1.0190994714776e-08
4541 1.01842809741015e-08
4542 1.01875041735866e-08
4543 1.01878558922408e-08
4544 1.01875361480097e-08
4545 1.01883053105212e-08
4546 1.01877439817599e-08
4547 1.01840234023598e-08
4548 1.01831458820811e-08
4549 1.01839923161151e-08
4550 1.01840900157413e-08
4551 1.0182605869602e-08
4552 1.01840402777498e-08
4553 1.01806048036224e-08
4554 1.01817176911823e-08
4555 1.01826280740624e-08
4556 1.01837738242239e-08
4557 1.01872297264549e-08
4558 1.01805675001287e-08
4559 1.0181581799884e-08
4560 1.01799928486912e-08
4561 1.01773505178926e-08
4562 1.01783186323701e-08
4563 1.01768691251891e-08
4564 1.01745429859079e-08
4565 1.01746930880608e-08
4566 1.01765254001407e-08
4567 1.01784891626266e-08
4568 1.01737738233965e-08
4569 1.01771107097193e-08
4570 1.01745145641985e-08
4571 1.01724308976259e-08
4572 1.01723172107882e-08
4573 1.01722053003073e-08
4574 1.01719663803124e-08
4575 1.01714148215137e-08
4576 1.01726582713013e-08
4577 1.01741983726811e-08
4578 1.01745554204058e-08
4579 1.01717168021764e-08
4580 1.01701029819878e-08
4581 1.01694386245299e-08
4582 1.01679225039675e-08
4583 1.01693755638621e-08
4584 1.01717363421017e-08
4585 1.01759169979232e-08
4586 1.01734727309122e-08
4587 1.01691508547219e-08
4588 1.01695309950856e-08
4589 1.01666550733626e-08
4590 1.01650661221697e-08
4591 1.01691242093693e-08
4592 1.01683674813557e-08
4593 1.0169459940812e-08
4594 1.01719246359266e-08
4595 1.01699315635528e-08
4596 1.01650989847712e-08
4597 1.01633732541018e-08
4598 1.01629629156719e-08
4599 1.01621120407458e-08
4600 1.01658468310006e-08
4601 1.01664454632555e-08
4602 1.01701660426556e-08
4603 1.01644541672385e-08
4604 1.01688133469224e-08
4605 1.01671115970703e-08
4606 1.01649542116888e-08
4607 1.01661896678706e-08
4608 1.01642783079114e-08
4609 1.01650083905724e-08
4610 1.01600612367747e-08
4611 1.01634309856991e-08
4612 1.01604760160967e-08
4613 1.01591783874255e-08
4614 1.01629487048172e-08
4615 1.01602815050228e-08
4616 1.01593959911384e-08
4617 1.01574828548223e-08
4618 1.01581658640271e-08
4619 1.01588755185844e-08
4620 1.01579509248495e-08
4621 1.01597965596056e-08
4622 1.01588328860203e-08
4623 1.01560191367867e-08
4624 1.01540784669396e-08
4625 1.0157132912525e-08
4626 1.01556336673525e-08
4627 1.01531680840594e-08
4628 1.01539114893967e-08
4629 1.01535420071741e-08
4630 1.01549515463262e-08
4631 1.0156684382423e-08
4632 1.01550847730891e-08
4633 1.01520187811843e-08
4634 1.01521120399184e-08
4635 1.01519175288445e-08
4636 1.01524255669005e-08
4637 1.01533359497807e-08
4638 1.01515569284061e-08
4639 1.01524184614732e-08
4640 1.01548467412726e-08
4641 1.01496535620527e-08
4642 1.015552264505e-08
4643 1.0152862550683e-08
4644 1.01487573900272e-08
4645 1.01477048985998e-08
4646 1.01486516967952e-08
4647 1.01499608717859e-08
4648 1.01495682969244e-08
4649 1.01507788841104e-08
4650 1.01503960792115e-08
4651 1.01453867529244e-08
4652 1.01445216671436e-08
4653 1.01484340930824e-08
4654 1.01481498759881e-08
4655 1.01488319970144e-08
4656 1.0147974016661e-08
4657 1.01443164979287e-08
4658 1.01436219424045e-08
4659 1.01476986813509e-08
4660 1.01446948619355e-08
4661 1.01419228570876e-08
4662 1.01416564035617e-08
4663 1.01443218269992e-08
4664 1.0142852779893e-08
4665 1.01467376723008e-08
4666 1.01421813170077e-08
4667 1.01413473174716e-08
4668 1.01424113552184e-08
4669 1.01437835908769e-08
4670 1.01421342435515e-08
4671 1.01439052713204e-08
4672 1.01446744338318e-08
4673 1.01435571053798e-08
4674 1.01454382672728e-08
4675 1.01438795141462e-08
4676 1.01456754109108e-08
4677 1.01416128828191e-08
4678 1.01418109466067e-08
4679 1.01416262054954e-08
4680 1.01428065946152e-08
4681 1.01395913887359e-08
4682 1.01416262054954e-08
4683 1.01423403009449e-08
4684 1.01393551332762e-08
4685 1.01406705255158e-08
4686 1.01413331066169e-08
4687 1.01452242162736e-08
4688 1.01433270671691e-08
4689 1.01416386399933e-08
4690 1.01427914955821e-08
4691 1.01410151387427e-08
4692 1.01354453718727e-08
4693 1.01393808904504e-08
4694 1.01388453188633e-08
4695 1.01433474952728e-08
4696 1.01356336656977e-08
4697 1.01362038762431e-08
4698 1.01339336922024e-08
4699 1.01350465797623e-08
4700 1.013276040851e-08
4701 1.01400914331862e-08
4702 1.0137045869385e-08
4703 1.01404067365252e-08
4704 1.01327319868005e-08
4705 1.01342445546493e-08
4706 1.0135763339747e-08
4707 1.0137604533611e-08
4708 1.01354782344742e-08
4709 1.01346593339713e-08
4710 1.01343955449806e-08
4711 1.01349213466051e-08
4712 1.01365982274615e-08
4713 1.01362314097742e-08
4714 1.01355119852542e-08
4715 1.01356869564029e-08
4716 1.01340829061769e-08
4717 1.01353343495703e-08
4718 1.01333377244828e-08
4719 1.01339514557708e-08
4720 1.01329789004012e-08
4721 1.01322710222007e-08
4722 1.01336397051455e-08
4723 1.01305568378507e-08
4724 1.01306714128668e-08
4725 1.0129237004719e-08
4726 1.01334816093868e-08
4727 1.01307611188872e-08
4728 1.01238750715993e-08
4729 1.01246948602807e-08
4730 1.01278354591727e-08
4731 1.01283870179714e-08
4732 1.01287769282976e-08
4733 1.01275521302568e-08
4734 1.01254471474022e-08
4735 1.01274730823775e-08
4736 1.01262251916978e-08
4737 1.0129452832075e-08
4738 1.01241681704778e-08
4739 1.01264383545185e-08
4740 1.01258068596621e-08
4741 1.01264800989043e-08
4742 1.01221449000377e-08
4743 1.01255297479952e-08
4744 1.01232915383775e-08
4745 1.01244426176095e-08
4746 1.01237249694464e-08
4747 1.01236974359153e-08
4748 1.01238244454294e-08
4749 1.01247872308363e-08
4750 1.0119967974731e-08
4751 1.01228199156367e-08
4752 1.01201793611949e-08
4753 1.01194386203929e-08
4754 1.01171933053479e-08
4755 1.01196748758525e-08
4756 1.01169357336062e-08
4757 1.01129122853649e-08
4758 1.01157437981669e-08
4759 1.01172847877251e-08
4760 1.01160679832901e-08
4761 1.01158166287973e-08
4762 1.01178425637727e-08
4763 1.0123391902539e-08
4764 1.01182457967752e-08
4765 1.01153680986954e-08
4766 1.01134673968772e-08
4767 1.01151353959494e-08
4768 1.0116622206624e-08
4769 1.0116406379268e-08
4770 1.01171000466138e-08
4771 1.01137294095111e-08
4772 1.01132462404507e-08
4773 1.01130028795637e-08
4774 1.01146149233955e-08
4775 1.0111915749178e-08
4776 1.01137773711457e-08
4777 1.01137835883947e-08
4778 1.01130650520531e-08
4779 1.01124975060429e-08
4780 1.01074570935111e-08
4781 1.01067767488416e-08
4782 1.01093871052171e-08
4783 1.01082537895536e-08
4784 1.01092565429894e-08
4785 1.01120471995841e-08
4786 1.01101420568739e-08
4787 1.01101216287702e-08
4788 1.01107842098713e-08
4789 1.01073451830302e-08
4790 1.01081258918612e-08
4791 1.01091233162265e-08
4792 1.01039061561892e-08
4793 1.01070423141891e-08
4794 1.01040233957406e-08
4795 1.01039701050354e-08
4796 1.01007513464424e-08
4797 1.01056762957796e-08
4798 1.010612571406e-08
4799 1.01042108013871e-08
4800 1.01068486912936e-08
4801 1.01054382639632e-08
4802 1.0104449721382e-08
4803 1.01047170630864e-08
4804 1.01045749545392e-08
4805 1.01053565515485e-08
4806 1.010196104545e-08
4807 1.01047596956505e-08
4808 1.01019450582385e-08
4809 1.01025738885596e-08
4810 1.01029646870643e-08
4811 1.01023109877474e-08
4812 1.01026405019411e-08
4813 1.01014228093277e-08
4814 1.00947268322216e-08
4815 1.00956984994127e-08
4816 1.00969224092751e-08
4817 1.00999733021467e-08
4818 1.00970627414654e-08
4819 1.00957251447653e-08
4820 1.00959907101128e-08
4821 1.0098964331462e-08
4822 1.00980717121502e-08
4823 1.00986516926582e-08
4824 1.00977723960227e-08
4825 1.0095961400225e-08
4826 1.00911696776507e-08
4827 1.00924362200772e-08
4828 1.00900088284561e-08
4829 1.00924522072887e-08
4830 1.00925729995538e-08
4831 1.00957553428316e-08
4832 1.00942694203354e-08
4833 1.00963140070576e-08
4834 1.00957251447653e-08
4835 1.00962385118919e-08
4836 1.00936246028027e-08
4837 1.0094429292451e-08
4838 1.00917123546651e-08
4839 1.00939248071086e-08
4840 1.00894563814791e-08
4841 1.00936059510559e-08
4842 1.0088965218813e-08
4843 1.00931503155266e-08
4844 1.00927453061672e-08
4845 1.00941059955062e-08
4846 1.0094179714315e-08
4847 1.00900390265224e-08
4848 1.00921999646175e-08
4849 1.00901900168537e-08
4850 1.00909032241248e-08
4851 1.00921884182981e-08
4852 1.00899919530661e-08
4853 1.00888328802284e-08
4854 1.0090977831112e-08
4855 1.00892059151647e-08
4856 1.00892441068368e-08
4857 1.00881534237374e-08
4858 1.00883141840313e-08
4859 1.00883488229897e-08
4860 1.00876746955691e-08
4861 1.00880725995012e-08
4862 1.00862935781265e-08
4863 1.00879393727382e-08
4864 1.00841610617408e-08
4865 1.00872181718614e-08
4866 1.00869446129082e-08
4867 1.00839105954265e-08
4868 1.00851966777782e-08
4869 1.00863104535165e-08
4870 1.00839576688827e-08
4871 1.00832737714995e-08
4872 1.00833963401215e-08
4873 1.00826813564936e-08
4874 1.0085198454135e-08
4875 1.00843626782421e-08
4876 1.00833341676321e-08
4877 1.00806332170578e-08
4878 1.00822612481011e-08
4879 1.00808765779448e-08
4880 1.00819983472888e-08
4881 1.00791632817732e-08
4882 1.0076134593362e-08
4883 1.00800638946907e-08
4884 1.0080293044723e-08
4885 1.00794164126228e-08
4886 1.00797006297171e-08
4887 1.00769659283628e-08
4888 1.00793773327723e-08
4889 1.00768859923051e-08
4890 1.00779233846993e-08
4891 1.00773034361623e-08
4892 1.00809023351189e-08
4893 1.00758885679397e-08
4894 1.00776977873807e-08
4895 1.00783479339839e-08
4896 1.00760413346279e-08
4897 1.00747543640978e-08
4898 1.00750714437936e-08
4899 1.00743049458174e-08
4900 1.00755102039329e-08
4901 1.00739860897647e-08
4902 1.00735491059822e-08
4903 1.0073368805763e-08
4904 1.00723314133688e-08
4905 1.00728572149933e-08
4906 1.00719939055693e-08
4907 1.00708792416526e-08
4908 1.00711998740621e-08
4909 1.00689030446688e-08
4910 1.00678736458804e-08
4911 1.00709831585277e-08
4912 1.00701775807011e-08
4913 1.00699963923034e-08
4914 1.00680557224564e-08
4915 1.00658974488965e-08
4916 1.00655714874165e-08
4917 1.00637889133282e-08
4918 1.00669694802491e-08
4919 1.00671515568251e-08
4920 1.0067540578973e-08
4921 1.00677590708642e-08
4922 1.00631822874675e-08
4923 1.00615551446026e-08
4924 1.00640624722814e-08
4925 1.0064050925962e-08
4926 1.00641823763681e-08
4927 1.0062032096414e-08
4928 1.00607921993401e-08
4929 1.00610124675882e-08
4930 1.00620365373061e-08
4931 1.00648982481744e-08
4932 1.0064205469007e-08
4933 1.00611297071396e-08
4934 1.00635597632959e-08
4935 1.0063420319284e-08
4936 1.00617647547097e-08
4937 1.00571480032841e-08
4938 1.00601145192059e-08
4939 1.00590478169238e-08
4940 1.00573380734659e-08
4941 1.00545785031159e-08
4942 1.00567296712484e-08
4943 1.00544719217055e-08
4944 1.00568655625466e-08
4945 1.00567216776426e-08
4946 1.00552508541796e-08
4947 1.0057143562392e-08
4948 1.00579491402186e-08
4949 1.00578523287709e-08
4950 1.00571355687862e-08
4951 1.0055710930601e-08
4952 1.00547401515882e-08
4953 1.00514183642986e-08
4954 1.00522621337973e-08
4955 1.00560715310394e-08
4956 1.0052972676533e-08
4957 1.00505959110819e-08
4958 1.0053979870861e-08
4959 1.00534460756307e-08
4960 1.00554391480046e-08
4961 1.0053359922324e-08
4962 1.00516315271193e-08
4963 1.00498116495373e-08
4964 1.00525872070989e-08
4965 1.00508801281762e-08
4966 1.00493497967591e-08
4967 1.0053122778686e-08
4968 1.0055577703838e-08
4969 1.00483461551448e-08
4970 1.00455048723802e-08
4971 1.00457135943088e-08
4972 1.00461559071618e-08
4973 1.00470973762867e-08
4974 1.00469987884821e-08
4975 1.00457429041967e-08
4976 1.00473220854269e-08
4977 1.00453441120862e-08
4978 1.0043881282229e-08
4979 1.00457029361678e-08
4980 1.0044378662144e-08
4981 1.00452952622732e-08
4982 1.00447037354456e-08
4983 1.00428509952621e-08
4984 1.00462225205433e-08
4985 1.0046513843065e-08
4986 1.00472394848339e-08
4987 1.00460972873861e-08
4988 1.00481063469715e-08
4989 1.00445767259316e-08
4990 1.00433297234304e-08
4991 1.00444603745586e-08
4992 1.00433128480404e-08
4993 1.00428403371211e-08
4994 1.00439345729342e-08
4995 1.00408730219215e-08
4996 1.00436441385909e-08
4997 1.00378771961118e-08
4998 1.00402868241645e-08
4999 1.00402077762851e-08
};
\addlegendentry{Test}

\nextgroupplot[
title={ELU/SiLU $\hy$},
ymin=4.01499357844638e-09, ymax=1e-05,
]
\addplot [semithick, black, dashed]
table {%
0 0.0103848394091474
1 0.0013935467019619
2 0.000334121725649311
3 0.000210507038100332
4 0.000199637074339989
5 0.000172362014400278
6 9.93725476328109e-05
7 3.05206616478983e-05
8 1.91680535790226e-05
9 1.82584303446731e-05
10 1.77017475732555e-05
11 1.72037136537e-05
12 1.67205192112618e-05
13 1.61988198145338e-05
14 1.55714710948729e-05
15 1.47443235583893e-05
16 1.35813958041666e-05
17 1.1917030008064e-05
18 9.66347014538371e-06
19 7.08385268238487e-06
20 4.85965359585805e-06
21 3.48275638227591e-06
22 2.86209319112629e-06
23 2.63184313181597e-06
24 2.53954837754833e-06
25 2.48687120570068e-06
26 2.44822919520615e-06
27 2.41736737261355e-06
28 2.39131499222722e-06
29 2.36846097007515e-06
30 2.34785372637347e-06
31 2.3288620126305e-06
32 2.31106426980254e-06
33 2.29417942801646e-06
34 2.27801888446244e-06
35 2.26244773118367e-06
36 2.2473676438981e-06
37 2.23270538040232e-06
38 2.21839890085818e-06
39 2.20439938228978e-06
40 2.19066289248637e-06
41 2.17715097487314e-06
42 2.16382891659528e-06
43 2.15066482664383e-06
44 2.13762863672606e-06
45 2.12469405044224e-06
46 2.1118326229228e-06
47 2.09902070292145e-06
48 2.08623257439022e-06
49 2.07344565182055e-06
50 2.0606396622469e-06
51 2.04779345506445e-06
52 2.03488441809263e-06
53 2.02188436691308e-06
54 2.00876303674846e-06
55 1.99549503048679e-06
56 1.98205900517401e-06
57 1.96842942798625e-06
58 1.9545771091547e-06
59 1.94047070547576e-06
60 1.92609116831832e-06
61 1.91142879397788e-06
62 1.89643821856933e-06
63 1.8810759647323e-06
64 1.86532254996763e-06
65 1.84915221674053e-06
66 1.83252354769969e-06
67 1.81541156592324e-06
68 1.79776542092824e-06
69 1.77954607748454e-06
70 1.76070773203207e-06
71 1.74119934080785e-06
72 1.72100265462127e-06
73 1.70007301630903e-06
74 1.67837629627243e-06
75 1.6558483728133e-06
76 1.63244410642704e-06
77 1.60814037408841e-06
78 1.58288762654202e-06
79 1.55663737642087e-06
80 1.5293678034336e-06
81 1.50108155435547e-06
82 1.47183240023807e-06
83 1.44171957774297e-06
84 1.41078346300638e-06
85 1.37903190133137e-06
86 1.34648295189876e-06
87 1.31329687771853e-06
88 1.27954291542309e-06
89 1.24516933426833e-06
90 1.21038715401944e-06
91 1.17541561546908e-06
92 1.14036083968472e-06
93 1.10541825192634e-06
94 1.07073777889433e-06
95 1.03646953603764e-06
96 1.00281618214026e-06
97 9.6994068272771e-07
98 9.38002058282095e-07
99 9.07150921198863e-07
100 8.77472990696404e-07
101 8.49076515287095e-07
102 8.22130109666475e-07
103 7.96622642614153e-07
104 7.72583256937409e-07
105 7.49905001981688e-07
106 7.28725713512901e-07
107 7.08963121759609e-07
108 6.90502378713376e-07
109 6.73416592537279e-07
110 6.57548313641954e-07
111 6.42795073133584e-07
112 6.29062958850568e-07
113 6.16304088749686e-07
114 6.04481634372433e-07
115 5.93502030660886e-07
116 5.83293435797216e-07
117 5.73793632117514e-07
118 5.64945426562602e-07
119 5.56698347342177e-07
120 5.49003670816006e-07
121 5.41816664865635e-07
122 5.35096161515369e-07
123 5.28806072965793e-07
124 5.22910080460903e-07
125 5.17377891103621e-07
126 5.12180044777821e-07
127 5.07297591893163e-07
128 5.02672801301784e-07
129 4.98310948836789e-07
130 4.94194084089017e-07
131 4.90299348207657e-07
132 4.86580496277256e-07
133 4.83070874471636e-07
134 4.79668139460543e-07
135 4.76507048091435e-07
136 4.73369438104498e-07
137 4.70481273957546e-07
138 4.67591096617426e-07
139 4.6492139294152e-07
140 4.62241134846053e-07
141 4.59754091750142e-07
142 4.57256747981916e-07
143 4.54911128116464e-07
144 4.52596516160142e-07
145 4.5037566767725e-07
146 4.48163261960843e-07
147 4.46046623158836e-07
148 4.43988316346733e-07
149 4.4197519155631e-07
150 4.40011970367138e-07
151 4.38091374896032e-07
152 4.3620898902752e-07
153 4.34362231528596e-07
154 4.32552662019248e-07
155 4.30777616298172e-07
156 4.2903503365288e-07
157 4.27324992442024e-07
158 4.25636253075012e-07
159 4.23988099381489e-07
160 4.223343463714e-07
161 4.20778495309193e-07
162 4.19119510510768e-07
163 4.17648352845745e-07
164 4.1600816141063e-07
165 4.14550030678029e-07
166 4.12989832481614e-07
167 4.11578742518515e-07
168 4.10032232686319e-07
169 4.08606176190318e-07
170 4.07184486554968e-07
171 4.05736385516775e-07
172 4.04370170702606e-07
173 4.02930748268915e-07
174 4.01572919743742e-07
175 4.00195661176994e-07
176 3.98839951449403e-07
177 3.97494893366712e-07
178 3.96163256509041e-07
179 3.94850371254307e-07
180 3.93580979668684e-07
181 3.92303898436808e-07
182 3.90940368861337e-07
183 3.8973289844435e-07
184 3.88443404922612e-07
185 3.87191771336504e-07
186 3.85949695200338e-07
187 3.84716453048384e-07
188 3.83485863693522e-07
189 3.82240913932463e-07
190 3.80983355602993e-07
191 3.79801040255146e-07
192 3.78631231711779e-07
193 3.7746009137507e-07
194 3.76298758915539e-07
195 3.75147074930027e-07
196 3.74004909241776e-07
197 3.72873344799629e-07
198 3.71750326280562e-07
199 3.70636548996828e-07
200 3.69530617451375e-07
201 3.6843458502922e-07
202 3.67346453199247e-07
203 3.66267505901519e-07
204 3.65197798704031e-07
205 3.64137636214679e-07
206 3.6308755565706e-07
207 3.62047063148907e-07
208 3.61016425847183e-07
209 3.59995324666684e-07
210 3.58983797482892e-07
211 3.57980138778302e-07
212 3.56985304906132e-07
213 3.55997113857143e-07
214 3.55016729047364e-07
215 3.5403990905003e-07
216 3.53066809987368e-07
217 3.52100647436515e-07
218 3.51153491275014e-07
219 3.50213426001389e-07
220 3.49277583321594e-07
221 3.48341186661827e-07
222 3.47400420718458e-07
223 3.46473825013405e-07
224 3.45565377616097e-07
225 3.44668462091136e-07
226 3.43779908865471e-07
227 3.42903995132815e-07
228 3.42037461985356e-07
229 3.41176102825358e-07
230 3.40320570646924e-07
231 3.39468786153319e-07
232 3.38621524596405e-07
233 3.37778873664263e-07
234 3.36940522581486e-07
235 3.361066513623e-07
236 3.35275295388016e-07
237 3.34443688263519e-07
238 3.33620740949669e-07
239 3.32807486882913e-07
240 3.31998661536659e-07
241 3.31195995956435e-07
242 3.3038985346856e-07
243 3.29604717643761e-07
244 3.28837057404385e-07
245 3.2806667594798e-07
246 3.27300603240488e-07
247 3.26536828755053e-07
248 3.25780557435706e-07
249 3.25031661356334e-07
250 3.2428950485297e-07
251 3.23550899448577e-07
252 3.22817921555085e-07
253 3.22090491872373e-07
254 3.213654839751e-07
255 3.20645192820379e-07
256 3.19927770670247e-07
257 3.19209769589435e-07
258 3.18498998344907e-07
259 3.17808854660484e-07
260 3.17118566671937e-07
261 3.16413793475689e-07
262 3.15745145336166e-07
263 3.15100691466341e-07
264 3.14426126758249e-07
265 3.13777341391486e-07
266 3.13123837589302e-07
267 3.12476425792241e-07
268 3.11838458074476e-07
269 3.11210179024357e-07
270 3.10583805774201e-07
271 3.09961421212535e-07
272 3.09342478333363e-07
273 3.08731358030556e-07
274 3.08127676656333e-07
275 3.07551774569426e-07
276 3.06965979514295e-07
277 3.06337286134273e-07
278 3.05753124636432e-07
279 3.05155502968724e-07
280 3.04579996149634e-07
281 3.0399918187296e-07
282 3.03410309198782e-07
283 3.02848758265029e-07
284 3.02301750246592e-07
285 3.01729723291899e-07
286 3.01174921482072e-07
287 3.00632724798966e-07
288 3.00080931985747e-07
289 2.99550544758276e-07
290 2.99023289908717e-07
291 2.98487630819189e-07
292 2.97982151388432e-07
293 2.97434315130296e-07
294 2.96968781745122e-07
295 2.96491135209465e-07
296 2.9586731158382e-07
297 2.95459539159992e-07
298 2.94856935692422e-07
299 2.94484102951742e-07
300 2.93877832600486e-07
301 2.93499175844758e-07
302 2.92889707194988e-07
303 2.92521673236124e-07
304 2.91922747660323e-07
305 2.91550653358286e-07
306 2.90937308019323e-07
307 2.90588034772554e-07
308 2.89985974939455e-07
309 2.89658276205529e-07
310 2.89088240895907e-07
311 2.88667368760542e-07
312 2.88200686881623e-07
313 2.8779869232487e-07
314 2.87272323951981e-07
315 2.86887091741406e-07
316 2.86366995599252e-07
317 2.86025630057196e-07
318 2.85532999255267e-07
319 2.8515034688148e-07
320 2.84702430006334e-07
321 2.84298877113898e-07
322 2.83910843709378e-07
323 2.83493465662588e-07
324 2.8306049995841e-07
325 2.82850188476402e-07
326 2.82206928973849e-07
327 2.81858722290806e-07
328 2.81486200078618e-07
329 2.81064841672674e-07
330 2.80722495766383e-07
331 2.80443757480597e-07
332 2.79905444408257e-07
333 2.79647700665464e-07
334 2.79193326484517e-07
335 2.78779102336202e-07
336 2.78548345901797e-07
337 2.78090921081997e-07
338 2.77814582531732e-07
339 2.77379596692384e-07
340 2.77006395203117e-07
341 2.76818780797328e-07
342 2.76267798369645e-07
343 2.7616099357175e-07
344 2.75558222119443e-07
345 2.75467707455235e-07
346 2.74877555472841e-07
347 2.74799174594165e-07
348 2.7420436046377e-07
349 2.74125994907948e-07
350 2.73550464861216e-07
351 2.73456550049289e-07
352 2.72920484379213e-07
353 2.72750133040844e-07
354 2.72353298550243e-07
355 2.72187221431963e-07
356 2.71773746709059e-07
357 2.7148284546108e-07
358 2.71189111950676e-07
359 2.70891143141938e-07
360 2.70593067773106e-07
361 2.70251049714787e-07
362 2.69995412764956e-07
363 2.69754756295981e-07
364 2.69363466754768e-07
365 2.69089900164943e-07
366 2.68783651137028e-07
367 2.68448824523482e-07
368 2.6826656759571e-07
369 2.6787394455674e-07
370 2.67591859530292e-07
371 2.67295272128543e-07
372 2.67109781630026e-07
373 2.66726684719387e-07
374 2.66442249272636e-07
375 2.66167721663102e-07
376 2.65887135984855e-07
377 2.65610200570698e-07
378 2.65332180779509e-07
379 2.65059422680736e-07
380 2.6477588672158e-07
381 2.64506091263605e-07
382 2.64227543230078e-07
383 2.639586648856e-07
384 2.63685500957678e-07
385 2.63418976477858e-07
386 2.63151382632998e-07
387 2.6288777463801e-07
388 2.6262489277773e-07
389 2.62364491658396e-07
390 2.62106105922477e-07
391 2.61849027954675e-07
392 2.6159415893634e-07
393 2.61340646415142e-07
394 2.61089122311375e-07
395 2.6083924483089e-07
396 2.60590002857164e-07
397 2.60343360069193e-07
398 2.60096916179009e-07
399 2.59852713891284e-07
400 2.59610228890317e-07
401 2.59368541811966e-07
402 2.59129170230743e-07
403 2.58890394658984e-07
404 2.58653378009122e-07
405 2.58418322077958e-07
406 2.58183724669792e-07
407 2.57950965901976e-07
408 2.57718542161278e-07
409 2.57488921741711e-07
410 2.57259913349017e-07
411 2.57031203394043e-07
412 2.56804444356717e-07
413 2.56577862598029e-07
414 2.56353457060321e-07
415 2.56129087689239e-07
416 2.55906254425398e-07
417 2.5568523570918e-07
418 2.55464889865387e-07
419 2.55245033336848e-07
420 2.55026454825114e-07
421 2.54809984205018e-07
422 2.54593418238969e-07
423 2.54377565634734e-07
424 2.5416366798936e-07
425 2.53950647185697e-07
426 2.53737834321477e-07
427 2.5352594504735e-07
428 2.53316995890707e-07
429 2.53106340237785e-07
430 2.52898103261678e-07
431 2.52688945690416e-07
432 2.52483396549152e-07
433 2.52276064827583e-07
434 2.5207368795499e-07
435 2.51866325887029e-07
436 2.51661838675865e-07
437 2.51466260820621e-07
438 2.51274792378631e-07
439 2.51078424507334e-07
440 2.50890117188263e-07
441 2.50696780698334e-07
442 2.50506818409768e-07
443 2.50319581595271e-07
444 2.50128302187846e-07
445 2.4994508378029e-07
446 2.49755319154055e-07
447 2.49576672247009e-07
448 2.49383481248699e-07
449 2.49212280355238e-07
450 2.49013384808094e-07
451 2.48849844246557e-07
452 2.48649259029676e-07
453 2.48490180707606e-07
454 2.48287054998464e-07
455 2.48135529633764e-07
456 2.47926302613166e-07
457 2.47789106409435e-07
458 2.47560041994888e-07
459 2.4752318103971e-07
460 2.47179671489128e-07
461 2.4707506160393e-07
462 2.46948494128318e-07
463 2.46717215378389e-07
464 2.46524068243659e-07
465 2.46415958550017e-07
466 2.46199061083807e-07
467 2.46117769019172e-07
468 2.45790863059092e-07
469 2.45705122417839e-07
470 2.45606371992224e-07
471 2.45328490087182e-07
472 2.45163146969318e-07
473 2.45111524973751e-07
474 2.4480142115646e-07
475 2.44661860492101e-07
476 2.44634977653924e-07
477 2.44292662282319e-07
478 2.44277000149395e-07
479 2.44331955490118e-07
480 2.43905354519569e-07
481 2.43804293308614e-07
482 2.43725766194025e-07
483 2.43426032145777e-07
484 2.43284116325349e-07
485 2.43237301904564e-07
486 2.42913790175159e-07
487 2.42787410816625e-07
488 2.42709848920164e-07
489 2.42428896577884e-07
490 2.42280002344053e-07
491 2.42131923717359e-07
492 2.4194986977033e-07
493 2.41780858904583e-07
494 2.41623936968693e-07
495 2.41455747834962e-07
496 2.41294208999321e-07
497 2.41124838143314e-07
498 2.40963316697318e-07
499 2.40795853970965e-07
500 2.40632616640646e-07
501 2.40466957024843e-07
502 2.40302732151143e-07
503 2.40140335186823e-07
504 2.39976581046797e-07
505 2.39813538648548e-07
506 2.39651042563516e-07
507 2.39488452080217e-07
508 2.39326242510884e-07
509 2.39163933095199e-07
510 2.39002423397139e-07
511 2.38840353935288e-07
512 2.38678658355695e-07
513 2.38517964840312e-07
514 2.38356372936899e-07
515 2.38195729515667e-07
516 2.38035482978738e-07
517 2.3787665729369e-07
518 2.37718060034808e-07
519 2.37563765969284e-07
520 2.37410106913138e-07
521 2.37257084259923e-07
522 2.37098785404122e-07
523 2.36936647932673e-07
524 2.36774352424973e-07
525 2.36612481790743e-07
526 2.36451250814618e-07
527 2.36289473717122e-07
528 2.36127880125281e-07
529 2.3596590166175e-07
530 2.35804210688251e-07
531 2.35642398620506e-07
532 2.35480759035234e-07
533 2.353185738313e-07
534 2.35155749380489e-07
535 2.34994067844774e-07
536 2.34831613771469e-07
537 2.34670155565553e-07
538 2.34506834651427e-07
539 2.34344352652904e-07
540 2.34181730242256e-07
541 2.34018620870025e-07
542 2.33855940703798e-07
543 2.33692670364327e-07
544 2.33530159438722e-07
545 2.33366589154294e-07
546 2.3320283347239e-07
547 2.33039588955819e-07
548 2.32875444204694e-07
549 2.32711398973962e-07
550 2.32547936605876e-07
551 2.32383647730039e-07
552 2.3221904118742e-07
553 2.32054192594866e-07
554 2.31890145740543e-07
555 2.31725541432581e-07
556 2.31560963487532e-07
557 2.31395577401372e-07
558 2.31230967645324e-07
559 2.31067083714009e-07
560 2.30901557426222e-07
561 2.30740099093296e-07
562 2.30585442522191e-07
563 2.30430264791082e-07
564 2.30217606035588e-07
565 2.30096327895168e-07
566 2.29883949984355e-07
567 2.29762511609444e-07
568 2.29546060473318e-07
569 2.29428019782318e-07
570 2.29204832128183e-07
571 2.29093036543837e-07
572 2.28862227265481e-07
573 2.28755661619395e-07
574 2.28520930344978e-07
575 2.28416980996293e-07
576 2.28177961909104e-07
577 2.280772398775e-07
578 2.27834950270456e-07
579 2.27734626372822e-07
580 2.27491542125868e-07
581 2.27390874410816e-07
582 2.27147334245004e-07
583 2.27043939628579e-07
584 2.26804521937574e-07
585 2.26693997716687e-07
586 2.26461581323889e-07
587 2.26342848572259e-07
588 2.26117452337249e-07
589 2.25988703657976e-07
590 2.257743992784e-07
591 2.25635558987136e-07
592 2.25429940354793e-07
593 2.25280046503684e-07
594 2.25085950901871e-07
595 2.24924753505995e-07
596 2.24739392816531e-07
597 2.24569365349758e-07
598 2.24390794183016e-07
599 2.24214765265351e-07
600 2.24038238094693e-07
601 2.23862224758342e-07
602 2.2368461181177e-07
603 2.23508305976061e-07
604 2.23330165722402e-07
605 2.23152618541e-07
606 2.22974729913616e-07
607 2.22796181359008e-07
608 2.22619270370927e-07
609 2.22439653475348e-07
610 2.22261751760655e-07
611 2.22082436449611e-07
612 2.2190244207998e-07
613 2.21723178627897e-07
614 2.21541225593924e-07
615 2.2136084701696e-07
616 2.21178488551743e-07
617 2.20996180122057e-07
618 2.20813316547108e-07
619 2.20630612483674e-07
620 2.20446837767341e-07
621 2.20262806831961e-07
622 2.20078580455585e-07
623 2.19893656264958e-07
624 2.19708973249411e-07
625 2.19523429993984e-07
626 2.19337772751516e-07
627 2.19151501481996e-07
628 2.18965809322569e-07
629 2.18778535044706e-07
630 2.1859244045519e-07
631 2.18404610571987e-07
632 2.18217246144903e-07
633 2.18029670382869e-07
634 2.17841810727926e-07
635 2.17654423535052e-07
636 2.17466245001852e-07
637 2.17278099083451e-07
638 2.17089402077875e-07
639 2.16900926325536e-07
640 2.16711626944921e-07
641 2.16523292709603e-07
642 2.16334587575417e-07
643 2.16145229860132e-07
644 2.15955193818118e-07
645 2.15766053694111e-07
646 2.15575916954869e-07
647 2.15385374573884e-07
648 2.15195948063851e-07
649 2.15005359338605e-07
650 2.14813768063316e-07
651 2.14623361345367e-07
652 2.14431841980556e-07
653 2.14241288128747e-07
654 2.14049042557285e-07
655 2.13856008762114e-07
656 2.136634266936e-07
657 2.13469495235152e-07
658 2.13274597819613e-07
659 2.13078883855466e-07
660 2.12883174072864e-07
661 2.12685639233712e-07
662 2.12488698732471e-07
663 2.12290616338429e-07
664 2.12091603235542e-07
665 2.11892993212892e-07
666 2.11692897598148e-07
667 2.11492727480334e-07
668 2.11290270831554e-07
669 2.11089950584054e-07
670 2.10888637432127e-07
671 2.10686762770251e-07
672 2.10484524829191e-07
673 2.10282866464517e-07
674 2.10082038531567e-07
675 2.09880370618976e-07
676 2.09678765692445e-07
677 2.09476222191363e-07
678 2.09274330985387e-07
679 2.09073749533495e-07
680 2.0887067669495e-07
681 2.08668611280771e-07
682 2.08466075006797e-07
683 2.08263616514159e-07
684 2.08062035212286e-07
685 2.07857602063655e-07
686 2.07657637240288e-07
687 2.07448935908516e-07
688 2.07259024572082e-07
689 2.07031693349791e-07
690 2.06862661181084e-07
691 2.06613985041848e-07
692 2.06454585128135e-07
693 2.06207517988055e-07
694 2.06053430577668e-07
695 2.05793024144363e-07
696 2.05648780564616e-07
697 2.05382929626019e-07
698 2.05245939584486e-07
699 2.0496995200503e-07
700 2.04843181720982e-07
701 2.04558006906552e-07
702 2.04439098697939e-07
703 2.04145331342609e-07
704 2.04036773707728e-07
705 2.0373256553885e-07
706 2.0363309601823e-07
707 2.03319745127217e-07
708 2.03228507295705e-07
709 2.02906579684914e-07
710 2.02822971766103e-07
711 2.02492900545082e-07
712 2.02417297761492e-07
713 2.02079349644002e-07
714 2.02009710641882e-07
715 2.01665174916599e-07
716 2.01602084247909e-07
717 2.01251549013293e-07
718 2.01192323393329e-07
719 2.00837414551458e-07
720 2.00782370873398e-07
721 2.00422997683525e-07
722 2.00370771265668e-07
723 2.00008556342723e-07
724 1.99958739065309e-07
725 1.99594542754866e-07
726 1.99546022969344e-07
727 1.99179766545043e-07
728 1.99132573465377e-07
729 1.98764813639229e-07
730 1.98717610444099e-07
731 1.98349996449743e-07
732 1.98302823926966e-07
733 1.97934214257067e-07
734 1.97886671052139e-07
735 1.97519135853419e-07
736 1.97469456010957e-07
737 1.97103790802622e-07
738 1.97050906235674e-07
739 1.96688534389367e-07
740 1.96630416852805e-07
741 1.96275278683267e-07
742 1.96208674529075e-07
743 1.95862479694497e-07
744 1.95785279930583e-07
745 1.95451799731572e-07
746 1.95392731970223e-07
747 1.95020968766002e-07
748 1.94933440117673e-07
749 1.94615899712147e-07
750 1.94517558970375e-07
751 1.94198087743125e-07
752 1.94083493484598e-07
753 1.93790405854877e-07
754 1.93689086893656e-07
755 1.93397035711307e-07
756 1.93178273440786e-07
757 1.93040917523035e-07
758 1.92740025044991e-07
759 1.92613272624875e-07
760 1.92323419176077e-07
761 1.92183931284617e-07
762 1.91901318238763e-07
763 1.91818515706821e-07
764 1.91471253939923e-07
765 1.91282452425057e-07
766 1.91114835209838e-07
767 1.90891441829599e-07
768 1.90656724251781e-07
769 1.90499370010144e-07
770 1.90227179021285e-07
771 1.90049241011181e-07
772 1.89864645825821e-07
773 1.89624681389056e-07
774 1.89399142613844e-07
775 1.89234311341657e-07
776 1.88970614221518e-07
777 1.88783493495492e-07
778 1.88627165766064e-07
779 1.88388603160838e-07
780 1.88166568069192e-07
781 1.87955445603905e-07
782 1.87792623720817e-07
783 1.87564618117086e-07
784 1.87361882171722e-07
785 1.87155793916816e-07
786 1.86953020568481e-07
787 1.86749057500712e-07
788 1.86546120041875e-07
789 1.86339572682215e-07
790 1.86133274762135e-07
791 1.85927802461627e-07
792 1.85724407805843e-07
793 1.85520254790461e-07
794 1.8531654213394e-07
795 1.85114322180269e-07
796 1.8491147868005e-07
797 1.84710308760216e-07
798 1.84507456900462e-07
799 1.84306662466049e-07
800 1.841055719769e-07
801 1.83905414130159e-07
802 1.83705687260449e-07
803 1.83506264419542e-07
804 1.83305716958238e-07
805 1.83107747457001e-07
806 1.82909498517958e-07
807 1.82710647206541e-07
808 1.82513127239048e-07
809 1.82315325423232e-07
810 1.8211913164734e-07
811 1.81923576717757e-07
812 1.81726046003305e-07
813 1.81530662421103e-07
814 1.81335820206741e-07
815 1.81141798762496e-07
816 1.80947372696139e-07
817 1.80753502638353e-07
818 1.80561610865659e-07
819 1.80368010288312e-07
820 1.80177046185115e-07
821 1.79986016000555e-07
822 1.79794901105979e-07
823 1.79605739157651e-07
824 1.79415841561337e-07
825 1.79228774944029e-07
826 1.79040784227702e-07
827 1.78853585198535e-07
828 1.78667085190209e-07
829 1.78481531305508e-07
830 1.78294306298898e-07
831 1.78107918149539e-07
832 1.77922145411458e-07
833 1.77736336344658e-07
834 1.77551393575115e-07
835 1.77365299701027e-07
836 1.77181815034899e-07
837 1.76995688313752e-07
838 1.76811858258574e-07
839 1.76628147975588e-07
840 1.76443719491548e-07
841 1.76260395853056e-07
842 1.76076696575045e-07
843 1.75893451684672e-07
844 1.75709531799573e-07
845 1.75523082169615e-07
846 1.7532974741874e-07
847 1.75190412563531e-07
848 1.74957462855119e-07
849 1.74792557630532e-07
850 1.74594825709917e-07
851 1.74453638372096e-07
852 1.7421312817012e-07
853 1.74046751017176e-07
854 1.73913984717444e-07
855 1.73669464533344e-07
856 1.735107693146e-07
857 1.73362037807046e-07
858 1.73133466584652e-07
859 1.73023832831909e-07
860 1.7276779112807e-07
861 1.72679239357265e-07
862 1.72416984641988e-07
863 1.72321968413325e-07
864 1.72064241900749e-07
865 1.71946022352376e-07
866 1.71740323691161e-07
867 1.71590473782235e-07
868 1.71376809027013e-07
869 1.71264555617334e-07
870 1.71018987169536e-07
871 1.70905442928859e-07
872 1.70711378414623e-07
873 1.70542594972733e-07
874 1.70355600895622e-07
875 1.70213298668287e-07
876 1.69993087754339e-07
877 1.69870280783968e-07
878 1.6967239809329e-07
879 1.69515401081899e-07
880 1.69310159266356e-07
881 1.69174845361386e-07
882 1.68974206765427e-07
883 1.68829948067106e-07
884 1.68617025081641e-07
885 1.6848456202645e-07
886 1.68291801198261e-07
887 1.68134270341014e-07
888 1.67940125413768e-07
889 1.67791264195305e-07
890 1.67597371555583e-07
891 1.67445317768244e-07
892 1.67257831264145e-07
893 1.67098965008439e-07
894 1.66917877996831e-07
895 1.6675232931096e-07
896 1.66577020257019e-07
897 1.66407751901243e-07
898 1.66234161947276e-07
899 1.66061279939456e-07
900 1.65889407706565e-07
901 1.65715472779215e-07
902 1.65540738412684e-07
903 1.65369455824571e-07
904 1.65197220749924e-07
905 1.65027845501875e-07
906 1.64855977438982e-07
907 1.64683987818659e-07
908 1.64513297521207e-07
909 1.64341530168421e-07
910 1.64173353325303e-07
911 1.64002457345713e-07
912 1.63830570222512e-07
913 1.6365910318239e-07
914 1.63487814126562e-07
915 1.63315572903056e-07
916 1.63144972019413e-07
917 1.62972852220111e-07
918 1.628009925283e-07
919 1.62628224623873e-07
920 1.62456037527292e-07
921 1.62283891998793e-07
922 1.62111316918345e-07
923 1.61937652698896e-07
924 1.61765568053251e-07
925 1.61592155315304e-07
926 1.61417133500041e-07
927 1.61243244129139e-07
928 1.61068105912321e-07
929 1.60891019167941e-07
930 1.60711059306706e-07
931 1.60527605703287e-07
932 1.60390733393356e-07
933 1.60181007457627e-07
934 1.59995852742956e-07
935 1.59848953280584e-07
936 1.59676237715178e-07
937 1.59462167494517e-07
938 1.59325502042229e-07
939 1.59138281154991e-07
940 1.58937925431424e-07
941 1.5879625336046e-07
942 1.58601230349831e-07
943 1.58437435318515e-07
944 1.58216657062393e-07
945 1.58094740580594e-07
946 1.57862916864371e-07
947 1.57737017771442e-07
948 1.57506213815939e-07
949 1.57377759927968e-07
950 1.57148199696699e-07
951 1.57016063160764e-07
952 1.56789804850455e-07
953 1.5665531045217e-07
954 1.56431474564123e-07
955 1.56281456461738e-07
956 1.56071689527248e-07
957 1.55899814510807e-07
958 1.55744409611636e-07
959 1.55525805312529e-07
960 1.55358597646593e-07
961 1.55169116882625e-07
962 1.54985352627701e-07
963 1.54800631046559e-07
964 1.54614611254988e-07
965 1.5442960172507e-07
966 1.54243462079595e-07
967 1.54057013436848e-07
968 1.5386920107785e-07
969 1.53681919163429e-07
970 1.53494442758362e-07
971 1.5330569851324e-07
972 1.53115718232844e-07
973 1.52926560519795e-07
974 1.52735432169315e-07
975 1.52545299962892e-07
976 1.52353962036056e-07
977 1.52161254110883e-07
978 1.51969279139408e-07
979 1.51776086950939e-07
980 1.51582037832387e-07
981 1.51387941207393e-07
982 1.51194220317841e-07
983 1.50997860956625e-07
984 1.50802003708872e-07
985 1.50604742298999e-07
986 1.50407492852889e-07
987 1.50209963901027e-07
988 1.50010605054707e-07
989 1.49810609610945e-07
990 1.49608103093612e-07
991 1.49413805623055e-07
992 1.49203967354961e-07
993 1.490404446991e-07
994 1.48790255932507e-07
995 1.48626017105968e-07
996 1.48383946478869e-07
997 1.48220408104471e-07
998 1.47977986383996e-07
999 1.47812280531845e-07
1000 1.47563924252303e-07
1001 1.47402461832868e-07
1002 1.4714665867599e-07
1003 1.46986643021041e-07
1004 1.46731151880797e-07
1005 1.46560082104674e-07
1006 1.46328837860032e-07
1007 1.46121646945829e-07
1008 1.45913658613495e-07
1009 1.45697529897681e-07
1010 1.45489557138667e-07
1011 1.45272274190145e-07
1012 1.45059753199561e-07
1013 1.44842347458596e-07
1014 1.44627237164574e-07
1015 1.44407345292752e-07
1016 1.44192245711494e-07
1017 1.43969266912958e-07
1018 1.43750592793257e-07
1019 1.43529483437987e-07
1020 1.43307704419282e-07
1021 1.43085296706147e-07
1022 1.42862514862507e-07
1023 1.42637581957761e-07
1024 1.42412477620368e-07
1025 1.421867499527e-07
1026 1.41958837352352e-07
1027 1.41730974681842e-07
1028 1.4150177252148e-07
1029 1.41268312906284e-07
1030 1.41048702681612e-07
1031 1.40797126912506e-07
1032 1.40579688084674e-07
1033 1.40328257705047e-07
1034 1.40121672203009e-07
1035 1.39854214591839e-07
1036 1.39627471514192e-07
1037 1.39391215175255e-07
1038 1.39152524595954e-07
1039 1.38915447783994e-07
1040 1.38679627188054e-07
1041 1.38446568437534e-07
1042 1.38207823502157e-07
1043 1.37968890803641e-07
1044 1.37725739053352e-07
1045 1.37479401943086e-07
1046 1.37233426221517e-07
1047 1.36984640301652e-07
1048 1.36735180197967e-07
1049 1.36484572781814e-07
1050 1.36231713737001e-07
1051 1.3597932617726e-07
1052 1.35724352340549e-07
1053 1.35468392686633e-07
1054 1.35211143080571e-07
1055 1.34952735453275e-07
1056 1.34692921062829e-07
1057 1.34431297741155e-07
1058 1.34168259835921e-07
1059 1.33905307759363e-07
1060 1.33638556744753e-07
1061 1.33372844866475e-07
1062 1.33104275767515e-07
1063 1.32836696818295e-07
1064 1.32565278917252e-07
1065 1.32295151181072e-07
1066 1.32023103755596e-07
1067 1.31747216805866e-07
1068 1.31471056670662e-07
1069 1.3119191426858e-07
1070 1.30913442188074e-07
1071 1.30634042355027e-07
1072 1.30352073363316e-07
1073 1.30069147515233e-07
1074 1.29785626064027e-07
1075 1.2949884153457e-07
1076 1.29213076376367e-07
1077 1.28924087493321e-07
1078 1.28635391106524e-07
1079 1.28344048583529e-07
1080 1.28051562557019e-07
1081 1.27758089558672e-07
1082 1.27463344862022e-07
1083 1.27166677754253e-07
1084 1.2686916969562e-07
1085 1.26570600539555e-07
1086 1.26270875934198e-07
1087 1.25969943025162e-07
1088 1.25666096152832e-07
1089 1.25363675621992e-07
1090 1.25058642885145e-07
1091 1.24751893557207e-07
1092 1.24445145800678e-07
1093 1.24136556794552e-07
1094 1.23827149656197e-07
1095 1.23514966827276e-07
1096 1.23205121202341e-07
1097 1.22891053570884e-07
1098 1.22577557150949e-07
1099 1.22263030477132e-07
1100 1.21945695396475e-07
1101 1.21630255934813e-07
1102 1.21310836932453e-07
1103 1.2099277558697e-07
1104 1.20672051142856e-07
1105 1.20352949342095e-07
1106 1.20030374817226e-07
1107 1.19708828223875e-07
1108 1.19382927063816e-07
1109 1.1906104581727e-07
1110 1.18734379418006e-07
1111 1.18411865425383e-07
1112 1.18081873299758e-07
1113 1.17758102474275e-07
1114 1.17427661557201e-07
1115 1.1710247454455e-07
1116 1.16769633865088e-07
1117 1.1644420077217e-07
1118 1.16109452070967e-07
1119 1.15783419958104e-07
1120 1.15448055205469e-07
1121 1.15119910093275e-07
1122 1.14783875324154e-07
1123 1.14454582124557e-07
1124 1.14118233784266e-07
1125 1.13787875339222e-07
1126 1.13451013772359e-07
1127 1.13119021660557e-07
1128 1.12782842600856e-07
1129 1.12449491360911e-07
1130 1.12113739310082e-07
1131 1.11779976318838e-07
1132 1.11446154828387e-07
1133 1.11114425001224e-07
1134 1.10785210209352e-07
1135 1.1045372448848e-07
1136 1.10117368935292e-07
1137 1.0978086141078e-07
1138 1.09443312660273e-07
1139 1.09107292599608e-07
1140 1.08769396267139e-07
1141 1.08433728874591e-07
1142 1.08096742599706e-07
1143 1.07760716637539e-07
1144 1.07424518613897e-07
1145 1.07089963577778e-07
1146 1.06757270493585e-07
1147 1.06422567397013e-07
1148 1.06084554311492e-07
1149 1.05745572151372e-07
1150 1.05405851368534e-07
1151 1.05065995755327e-07
1152 1.04726737242045e-07
1153 1.04387926158855e-07
1154 1.04046878511177e-07
1155 1.03708024473681e-07
1156 1.03369328465996e-07
1157 1.03027639956288e-07
1158 1.02687634390142e-07
1159 1.02345840053308e-07
1160 1.02006356335282e-07
1161 1.01665405049856e-07
1162 1.01325304604716e-07
1163 1.00984921427738e-07
1164 1.00644890561385e-07
1165 1.0030500892233e-07
1166 9.99648228610095e-08
1167 9.96247428191044e-08
1168 9.92858524280216e-08
1169 9.89456106019748e-08
1170 9.86065710726347e-08
1171 9.82675203926586e-08
1172 9.79291697085216e-08
1173 9.75897173269846e-08
1174 9.72508709662367e-08
1175 9.69138792070723e-08
1176 9.65749209935751e-08
1177 9.62375062600884e-08
1178 9.58989835113666e-08
1179 9.55619776712435e-08
1180 9.52249926509108e-08
1181 9.48875933708404e-08
1182 9.45508207506052e-08
1183 9.42146682352263e-08
1184 9.38774641514595e-08
1185 9.35423184269535e-08
1186 9.32059098055493e-08
1187 9.28707633882642e-08
1188 9.25354161602776e-08
1189 9.22005482126842e-08
1190 9.1865358386789e-08
1191 9.15312461438056e-08
1192 9.11972701778119e-08
1193 9.08619715689341e-08
1194 9.05303129079549e-08
1195 9.01968944835296e-08
1196 8.98653097727298e-08
1197 8.95316938513968e-08
1198 8.92005165573639e-08
1199 8.88691986600243e-08
1200 8.8537478051931e-08
1201 8.82073445491738e-08
1202 8.78769972718629e-08
1203 8.75470022558833e-08
1204 8.72170671106787e-08
1205 8.68889285434715e-08
1206 8.65596035213301e-08
1207 8.62318846408527e-08
1208 8.5903474449589e-08
1209 8.55749752892709e-08
1210 8.5245171159487e-08
1211 8.49171306467333e-08
1212 8.4591928134925e-08
1213 8.42779655911841e-08
1214 8.39612227738762e-08
1215 8.36463131905774e-08
1216 8.33317920723165e-08
1217 8.30179408239928e-08
1218 8.27058290400196e-08
1219 8.2394517507689e-08
1220 8.2084739917887e-08
1221 8.17770250274297e-08
1222 8.14690272958885e-08
1223 8.11642913511879e-08
1224 8.08585185245736e-08
1225 8.05580171430176e-08
1226 8.02537616821919e-08
1227 7.99561612445654e-08
1228 7.96548122319329e-08
1229 7.93611519298576e-08
1230 7.90640276566723e-08
1231 7.87735264955458e-08
1232 7.84799934565594e-08
1233 7.81931968285576e-08
1234 7.79034863289674e-08
1235 7.76232889954098e-08
1236 7.73353005565358e-08
1237 7.70618362888609e-08
1238 7.67787823143795e-08
1239 7.65112314049521e-08
1240 7.62304239143585e-08
1241 7.59688508782386e-08
1242 7.56906983119254e-08
1243 7.54372747067134e-08
1244 7.51600314798928e-08
1245 7.49141323683755e-08
1246 7.46361948273933e-08
1247 7.43987862050766e-08
1248 7.41224269746876e-08
1249 7.38932943422643e-08
1250 7.36166062509369e-08
1251 7.33959313914401e-08
1252 7.31201377486279e-08
1253 7.29064032629623e-08
1254 7.26321491490989e-08
1255 7.24238460132476e-08
1256 7.21538128509724e-08
1257 7.19493594820708e-08
1258 7.16855483742229e-08
1259 7.14817223466113e-08
1260 7.12248173213581e-08
1261 7.10233274965333e-08
1262 7.07729396003387e-08
1263 7.05722076994597e-08
1264 7.0331461164308e-08
1265 7.01302453012254e-08
1266 6.98954590796497e-08
1267 6.96965795210502e-08
1268 6.94702396222269e-08
1269 6.92708762506378e-08
1270 6.90519237127596e-08
1271 6.88551534206994e-08
1272 6.86420001168031e-08
1273 6.84465119196176e-08
1274 6.8238688883504e-08
1275 6.80475200036668e-08
1276 6.78445150423812e-08
1277 6.76547760676272e-08
1278 6.74561714744115e-08
1279 6.7270285322163e-08
1280 6.70767357391355e-08
1281 6.68930444196469e-08
1282 6.67019500335542e-08
1283 6.65217843462962e-08
1284 6.63379071799142e-08
1285 6.61584917822644e-08
1286 6.59784397023522e-08
1287 6.58038677250339e-08
1288 6.56262338711855e-08
1289 6.54565196049894e-08
1290 6.52820178412838e-08
1291 6.51164268776761e-08
1292 6.4946010589928e-08
1293 6.47840496421637e-08
1294 6.46163073052541e-08
1295 6.44599301957705e-08
1296 6.42928729068792e-08
1297 6.41427194634225e-08
1298 6.39779144018249e-08
1299 6.38329345639121e-08
1300 6.36686660175911e-08
1301 6.35299138176393e-08
1302 6.33662452673711e-08
1303 6.32362339940684e-08
1304 6.30688776306343e-08
1305 6.29496334068058e-08
1306 6.27758480118423e-08
1307 6.26710422886312e-08
1308 6.24879665687139e-08
1309 6.24017402852672e-08
1310 6.2201637466508e-08
1311 6.21426172138229e-08
1312 6.19159036938655e-08
1313 6.18918269466029e-08
1314 6.16340565335882e-08
1315 6.16477282551209e-08
1316 6.13583841824195e-08
1317 6.14037745467755e-08
1318 6.10901462287927e-08
1319 6.11627612125965e-08
1320 6.08304886720035e-08
1321 6.09202215149374e-08
1322 6.0581722249875e-08
1323 6.06787294437083e-08
1324 6.03381947819059e-08
1325 6.04408602398188e-08
1326 6.00994693669144e-08
1327 6.02055179630412e-08
1328 5.98663963735646e-08
1329 5.99738300217645e-08
1330 5.96365776495134e-08
1331 5.9745788537402e-08
1332 5.94092052890183e-08
1333 5.95195218600075e-08
1334 5.91881399674143e-08
1335 5.92974039070881e-08
1336 5.89687828798091e-08
1337 5.90790998824531e-08
1338 5.87508709637419e-08
1339 5.88614610315119e-08
1340 5.85412034208588e-08
1341 5.8646831848197e-08
1342 5.83305862300598e-08
1343 5.84365033167167e-08
1344 5.81221398059384e-08
1345 5.82271211566621e-08
1346 5.7917916945982e-08
1347 5.80191603525293e-08
1348 5.77181279353134e-08
1349 5.78127386914318e-08
1350 5.75206182031884e-08
1351 5.76109066989794e-08
1352 5.73247408417465e-08
1353 5.74125582608609e-08
1354 5.71307531753718e-08
1355 5.72168949579677e-08
1356 5.69424259855289e-08
1357 5.70232436949425e-08
1358 5.67572761336343e-08
1359 5.68324673841936e-08
1360 5.65736299975583e-08
1361 5.66463219029423e-08
1362 5.63929077497249e-08
1363 5.64618448257725e-08
1364 5.6216263943476e-08
1365 5.62802612198432e-08
1366 5.60429465412682e-08
1367 5.61004685430966e-08
1368 5.58753711419335e-08
1369 5.59208527897148e-08
1370 5.57082373329187e-08
1371 5.57487533789924e-08
1372 5.55368986483451e-08
1373 5.55912279542881e-08
1374 5.53576134882938e-08
1375 5.54461473072365e-08
1376 5.51709513318954e-08
1377 5.53090045349869e-08
1378 5.49867612620325e-08
1379 5.51720589569982e-08
1380 5.48073897754797e-08
1381 5.50371414789197e-08
1382 5.46299103749703e-08
1383 5.49007583461325e-08
1384 5.44593910647428e-08
1385 5.47589357986489e-08
1386 5.42896751376531e-08
1387 5.46226232498448e-08
1388 5.41206718667198e-08
1389 5.44792302670238e-08
1390 5.39586391727909e-08
1391 5.43364681528935e-08
1392 5.37959447828307e-08
1393 5.41945669425559e-08
1394 5.36318438957295e-08
1395 5.40508202844237e-08
1396 5.34766664390673e-08
1397 5.39053170354187e-08
1398 5.33162584730462e-08
1399 5.37633485908628e-08
1400 5.31623290673799e-08
1401 5.36132403445855e-08
1402 5.30094417015015e-08
1403 5.34749952545432e-08
1404 5.28574996212505e-08
1405 5.3323489820567e-08
1406 5.27144830329007e-08
1407 5.31672408379258e-08
1408 5.25729176010792e-08
1409 5.30103786804403e-08
1410 5.24264477901326e-08
1411 5.28467401976762e-08
1412 5.23024798075866e-08
1413 5.264362278945e-08
1414 5.22048284330445e-08
1415 5.24267276582613e-08
1416 5.21451054589228e-08
1417 5.22179208757301e-08
1418 5.20917798016995e-08
1419 5.20372136398883e-08
1420 5.19721455809474e-08
1421 5.19114081836136e-08
1422 5.18169518177736e-08
1423 5.17993795381422e-08
1424 5.16776461609858e-08
1425 5.16802826813212e-08
1426 5.15418478861385e-08
1427 5.15653573676111e-08
1428 5.13969313060691e-08
1429 5.14510369669097e-08
1430 5.12586174288288e-08
1431 5.13292693731238e-08
1432 5.113263649692e-08
1433 5.12047307492658e-08
1434 5.10041225747493e-08
1435 5.1081055625346e-08
1436 5.08787357400564e-08
1437 5.09631738094818e-08
1438 5.07557597533292e-08
1439 5.08431385526364e-08
1440 5.06328958977953e-08
1441 5.07287631790199e-08
1442 5.05082615243069e-08
1443 5.06115496521353e-08
1444 5.03903657911309e-08
1445 5.04922095498905e-08
1446 5.02696426121219e-08
1447 5.03771710715917e-08
1448 5.01574052362841e-08
1449 5.02528318424567e-08
1450 5.00436560377793e-08
1451 5.01292291583688e-08
1452 4.99403216638328e-08
1453 4.99982663204612e-08
1454 4.98356438045633e-08
1455 4.98660265462192e-08
1456 4.97327826121108e-08
1457 4.97395347833773e-08
1458 4.96293612335386e-08
1459 4.96191536139001e-08
1460 4.95266504656477e-08
1461 4.94981438099451e-08
1462 4.94288292030021e-08
1463 4.93900888436372e-08
1464 4.93248272670499e-08
1465 4.92874629434148e-08
1466 4.92348840928081e-08
1467 4.9193408913073e-08
1468 4.9138755875866e-08
1469 4.90903954890598e-08
1470 4.90365501684131e-08
1471 4.89792968974001e-08
1472 4.89275463815275e-08
1473 4.88755588703427e-08
1474 4.88198660231198e-08
1475 4.87670553486197e-08
1476 4.87216015701009e-08
1477 4.86621221464922e-08
1478 4.86145736773302e-08
1479 4.8558230409057e-08
1480 4.85145494875105e-08
1481 4.84538037901494e-08
1482 4.84127405893098e-08
1483 4.83522303329487e-08
1484 4.83070102874628e-08
1485 4.82544501720916e-08
1486 4.82089668727426e-08
1487 4.815196913599e-08
1488 4.81057244563488e-08
1489 4.80542531051675e-08
1490 4.80045462021295e-08
1491 4.79481663697712e-08
1492 4.7902290985391e-08
1493 4.78561439534175e-08
1494 4.78067764726475e-08
1495 4.77581044475706e-08
1496 4.7712312831294e-08
1497 4.76610603241667e-08
1498 4.76151639570155e-08
1499 4.75695405781629e-08
1500 4.7519716035449e-08
1501 4.74767552516298e-08
1502 4.74308765494591e-08
1503 4.73831915117628e-08
1504 4.73342460449722e-08
1505 4.72925221757237e-08
1506 4.72433158780206e-08
1507 4.71978935525641e-08
1508 4.71551610479182e-08
1509 4.71057238096151e-08
1510 4.70636810590008e-08
1511 4.70193076305137e-08
1512 4.69697995408858e-08
1513 4.69314612350225e-08
1514 4.68771796584555e-08
1515 4.68428599411652e-08
1516 4.67925190692498e-08
1517 4.67523078353516e-08
1518 4.67048146122195e-08
1519 4.66668481817756e-08
1520 4.66145859259193e-08
1521 4.6576042515678e-08
1522 4.6529912819171e-08
1523 4.64865640603662e-08
1524 4.64455284228116e-08
1525 4.64009388285724e-08
1526 4.63581539240643e-08
1527 4.63157069692244e-08
1528 4.62704171835249e-08
1529 4.62327374490812e-08
1530 4.61867245888925e-08
1531 4.61459973375433e-08
1532 4.61025909790358e-08
1533 4.60612644652691e-08
1534 4.60204583041524e-08
1535 4.59780094419493e-08
1536 4.59378818287171e-08
1537 4.58951273532016e-08
1538 4.5856326666982e-08
1539 4.58160776957151e-08
1540 4.57744114505942e-08
1541 4.57333234580304e-08
1542 4.56952032665825e-08
1543 4.56529720676002e-08
1544 4.56149203360567e-08
1545 4.55730965875922e-08
1546 4.55358775837134e-08
1547 4.54924720894034e-08
1548 4.54549360200396e-08
1549 4.54157542919553e-08
1550 4.53757441243319e-08
1551 4.53371721236273e-08
1552 4.52967708239882e-08
1553 4.52579101062334e-08
1554 4.52221637576855e-08
1555 4.5178951895597e-08
1556 4.51429243901735e-08
1557 4.51029024719496e-08
1558 4.50686207740247e-08
1559 4.50274430279229e-08
1560 4.4993600301968e-08
1561 4.49583014412447e-08
1562 4.49235103676671e-08
1563 4.48906551460126e-08
1564 4.48667528698543e-08
1565 4.48268849249089e-08
1566 4.47757712307428e-08
1567 4.47438133988065e-08
1568 4.47017347684486e-08
1569 4.46658541655154e-08
1570 4.46265731701701e-08
1571 4.45908784452165e-08
1572 4.45536781144007e-08
1573 4.45148175707288e-08
1574 4.44829874739483e-08
1575 4.44397713752487e-08
1576 4.44077955021882e-08
1577 4.4369836612379e-08
1578 4.43319928860042e-08
1579 4.42959801776333e-08
1580 4.42586810929235e-08
1581 4.42231477104649e-08
1582 4.41867669194185e-08
1583 4.41515373681423e-08
1584 4.41133587967268e-08
1585 4.40775617331912e-08
1586 4.40396854817227e-08
1587 4.400550228989e-08
1588 4.39694801983581e-08
1589 4.39322216652105e-08
1590 4.38995454241642e-08
1591 4.38624468364068e-08
1592 4.38274227057001e-08
1593 4.37889420072946e-08
1594 4.37567988467169e-08
1595 4.37193096869404e-08
1596 4.36880398668738e-08
1597 4.36470710718506e-08
1598 4.36178419662703e-08
1599 4.35790327879548e-08
1600 4.35470165383034e-08
1601 4.35076038109994e-08
1602 4.34774778970137e-08
1603 4.34386368757256e-08
1604 4.34113201330799e-08
1605 4.33675199376093e-08
1606 4.3342480844144e-08
1607 4.32984890459398e-08
1608 4.32795788218865e-08
1609 4.32244220622824e-08
1610 4.32130419769639e-08
1611 4.31576379036347e-08
1612 4.31474872568671e-08
1613 4.30867084444397e-08
1614 4.30837254032923e-08
1615 4.30174756869128e-08
1616 4.30238354496293e-08
1617 4.29476525098949e-08
1618 4.29576794791497e-08
1619 4.28814980086845e-08
1620 4.28924379289874e-08
1621 4.28149116868948e-08
1622 4.28277774797703e-08
1623 4.27502704238858e-08
1624 4.27601816399559e-08
1625 4.26836792202234e-08
1626 4.269846977345e-08
1627 4.26193040334866e-08
1628 4.26287888477539e-08
1629 4.25547567888263e-08
1630 4.25658527272166e-08
1631 4.24897880966224e-08
1632 4.25022206596015e-08
1633 4.242412147204e-08
1634 4.24369188756213e-08
1635 4.23611743503027e-08
1636 4.23727202099045e-08
1637 4.22978695164833e-08
1638 4.23096997426153e-08
1639 4.2235383295175e-08
1640 4.22472580710043e-08
1641 4.21708712510238e-08
1642 4.2184939639478e-08
1643 4.21080454082734e-08
1644 4.21224662117137e-08
1645 4.20446952464903e-08
1646 4.20598660937355e-08
1647 4.198424109636e-08
1648 4.20027342158402e-08
1649 4.19214805271739e-08
1650 4.19366661845721e-08
1651 4.18617832551593e-08
1652 4.18767045811119e-08
1653 4.18002502262649e-08
1654 4.18133820037081e-08
1655 4.17435087776141e-08
1656 4.17532521119757e-08
1657 4.16836570305001e-08
1658 4.16898727346737e-08
1659 4.16242119904986e-08
1660 4.16305802881656e-08
1661 4.15673882414591e-08
1662 4.15673520965942e-08
1663 4.15110016676046e-08
1664 4.15071096082631e-08
1665 4.14556798804533e-08
1666 4.14455431676508e-08
1667 4.13998768709067e-08
1668 4.13830384016212e-08
1669 4.13434609143337e-08
1670 4.13240604772991e-08
1671 4.12895847325956e-08
1672 4.12658240922514e-08
1673 4.12341755287038e-08
1674 4.12082811915848e-08
1675 4.11763684360356e-08
1676 4.11519725742604e-08
1677 4.11201780781134e-08
1678 4.10947911777537e-08
1679 4.1065703088039e-08
1680 4.10384622242521e-08
1681 4.10096341780797e-08
1682 4.09836628789684e-08
1683 4.09569382682218e-08
1684 4.09257566891252e-08
1685 4.09009808663985e-08
1686 4.08720091693571e-08
1687 4.08467715149463e-08
1688 4.08213944773639e-08
1689 4.07916226312643e-08
1690 4.07651861389358e-08
1691 4.07401587125822e-08
1692 4.07119177672222e-08
1693 4.06857561305873e-08
1694 4.06609175662709e-08
1695 4.06326548909597e-08
1696 4.06063294475789e-08
1697 4.0582717188542e-08
1698 4.05543546224685e-08
1699 4.05290886213994e-08
1700 4.05038151389814e-08
1701 4.04777659173128e-08
1702 4.04532296678362e-08
1703 4.04265200975029e-08
1704 4.0402276947793e-08
1705 4.03760524398944e-08
1706 4.03533527044697e-08
1707 4.03276315148471e-08
1708 4.03041899728063e-08
1709 4.02791773510547e-08
1710 4.02588327481634e-08
1711 4.02384625703966e-08
1712 4.02233588217005e-08
1713 4.0211285785352e-08
1714 4.01954043007802e-08
1715 4.01674906456151e-08
1716 4.01367011855225e-08
1717 4.01119561317387e-08
1718 4.0085933746381e-08
1719 4.00616809514975e-08
1720 4.003655311835e-08
1721 4.00116022778452e-08
1722 3.99858858859403e-08
1723 3.99605236844902e-08
1724 3.99360872858079e-08
1725 3.99097200169951e-08
1726 3.98860154831038e-08
1727 3.98620727759535e-08
1728 3.98345485641549e-08
1729 3.98113557726898e-08
1730 3.97854316858126e-08
1731 3.97622571444156e-08
1732 3.97358248871438e-08
1733 3.97121494783992e-08
1734 3.96863808322934e-08
1735 3.96625478074064e-08
1736 3.96385531531429e-08
1737 3.96126026547261e-08
1738 3.95890165016954e-08
1739 3.95658154408451e-08
1740 3.95388404135222e-08
1741 3.95158285684261e-08
1742 3.94914987349448e-08
1743 3.94658287026228e-08
1744 3.94442400877271e-08
1745 3.94175655749063e-08
1746 3.93949615997702e-08
1747 3.93705142642808e-08
1748 3.93447322637464e-08
1749 3.93225242569795e-08
1750 3.92984255705997e-08
1751 3.92745652639803e-08
1752 3.92492883960482e-08
1753 3.92269956350777e-08
1754 3.9203098414875e-08
1755 3.9178333069545e-08
1756 3.91556279804028e-08
1757 3.9131091112754e-08
1758 3.91085707529548e-08
1759 3.90844484992048e-08
1760 3.90612877319008e-08
1761 3.90371952792012e-08
1762 3.90134650083418e-08
1763 3.89892565473282e-08
1764 3.89674952190333e-08
1765 3.89442030384046e-08
1766 3.89201292887442e-08
1767 3.88984457118102e-08
1768 3.88749181785553e-08
1769 3.88512450835155e-08
1770 3.88289582453627e-08
1771 3.88054419397932e-08
1772 3.87822694623008e-08
1773 3.87588593413479e-08
1774 3.87362087181931e-08
1775 3.87137290185358e-08
1776 3.86909936096469e-08
1777 3.86689320621691e-08
1778 3.86459090220281e-08
1779 3.86228170001779e-08
1780 3.86015323123878e-08
1781 3.8576623322184e-08
1782 3.85551681625351e-08
1783 3.85327347638409e-08
1784 3.85110304748082e-08
1785 3.84880933601561e-08
1786 3.84638222801215e-08
1787 3.8444444513841e-08
1788 3.84214981550279e-08
1789 3.8399518136778e-08
1790 3.83764502815964e-08
1791 3.83559889702845e-08
1792 3.83335016795883e-08
1793 3.83120096214551e-08
1794 3.82883707878978e-08
1795 3.8267552711968e-08
1796 3.82462423398344e-08
1797 3.82220530390498e-08
1798 3.82007091195291e-08
1799 3.81806440519217e-08
1800 3.81570438328094e-08
1801 3.81373738662649e-08
1802 3.81135268345822e-08
1803 3.8091416992625e-08
1804 3.80709796075074e-08
1805 3.80488116304978e-08
1806 3.80280034275593e-08
1807 3.80039449532354e-08
1808 3.79846108851112e-08
1809 3.79633181775141e-08
1810 3.79407371169371e-08
1811 3.79190207484559e-08
1812 3.78975130748138e-08
1813 3.78760852886018e-08
1814 3.78547862176504e-08
1815 3.78326880234869e-08
1816 3.7811640594354e-08
1817 3.77903284958236e-08
1818 3.7769321932446e-08
1819 3.77477107196267e-08
1820 3.77252972574293e-08
1821 3.77055883018507e-08
1822 3.76845400602566e-08
1823 3.76619721040861e-08
1824 3.76423314978336e-08
1825 3.76214352919568e-08
1826 3.75989094594242e-08
1827 3.75774293459585e-08
1828 3.75583502003618e-08
1829 3.75359415081267e-08
1830 3.75140670392238e-08
1831 3.74939373979366e-08
1832 3.74738109440997e-08
1833 3.74516674361836e-08
1834 3.7430630781321e-08
1835 3.74113793635278e-08
1836 3.73899367520636e-08
1837 3.73691681709865e-08
1838 3.73492600189707e-08
1839 3.73267123465748e-08
1840 3.7307011489629e-08
1841 3.72872246956479e-08
1842 3.72652947646657e-08
1843 3.72452302950244e-08
1844 3.72245632782597e-08
1845 3.72057488975397e-08
1846 3.71850634313109e-08
1847 3.71626939208447e-08
1848 3.71424350351646e-08
1849 3.71226465369912e-08
1850 3.71038670350377e-08
1851 3.7083268645155e-08
1852 3.70615799636376e-08
1853 3.70405963359133e-08
1854 3.70215261285001e-08
1855 3.70014555128861e-08
1856 3.69805152109492e-08
1857 3.69610014565591e-08
1858 3.69400704707257e-08
1859 3.69205499970437e-08
1860 3.69026419244634e-08
1861 3.68787903772816e-08
1862 3.68599670550473e-08
1863 3.68417175045099e-08
1864 3.68209813348397e-08
1865 3.6800910939494e-08
1866 3.67806852645813e-08
1867 3.67614439571451e-08
1868 3.67410524957368e-08
1869 3.67225429513951e-08
1870 3.67015569533447e-08
1871 3.66831765483688e-08
1872 3.66614251348096e-08
1873 3.66436824306149e-08
1874 3.66221198844219e-08
1875 3.66032816012662e-08
1876 3.65826310946282e-08
1877 3.65640362416375e-08
1878 3.65442098591018e-08
1879 3.65253359309037e-08
1880 3.65057856479556e-08
1881 3.64857052808087e-08
1882 3.64646878208141e-08
1883 3.644835753569e-08
1884 3.64272061370574e-08
1885 3.64082023165402e-08
1886 3.63885215075577e-08
1887 3.6368752551974e-08
1888 3.63501966686997e-08
1889 3.63303083523725e-08
1890 3.63127822362586e-08
1891 3.62922025147761e-08
1892 3.62725615179471e-08
1893 3.6252930023517e-08
1894 3.62381091754127e-08
1895 3.62141347229894e-08
1896 3.61979276601865e-08
1897 3.61778462529827e-08
1898 3.61597604012776e-08
1899 3.61382228157847e-08
1900 3.61239343749098e-08
1901 3.60995335266789e-08
1902 3.60836708102052e-08
1903 3.60628823996567e-08
1904 3.60487670050258e-08
1905 3.60222219766282e-08
1906 3.60114653288335e-08
1907 3.59842637052132e-08
1908 3.59763824955284e-08
1909 3.59428113920757e-08
1910 3.59437859740463e-08
1911 3.58994966676018e-08
1912 3.5917553203868e-08
1913 3.58467881125257e-08
1914 3.58978166710155e-08
1915 3.5790131832858e-08
1916 3.58865911531669e-08
1917 3.57333774210211e-08
1918 3.58620815659183e-08
1919 3.56883021239707e-08
1920 3.58271985005931e-08
1921 3.56507011662899e-08
1922 3.57908281680697e-08
1923 3.56147336142421e-08
1924 3.57557924548502e-08
1925 3.55781866518967e-08
1926 3.57175754168715e-08
1927 3.55434486327688e-08
1928 3.56811706443416e-08
1929 3.55080162048882e-08
1930 3.56434791231042e-08
1931 3.54737393633009e-08
1932 3.56105300691745e-08
1933 3.5436399343336e-08
1934 3.55721876361681e-08
1935 3.54009467322669e-08
1936 3.55358372909897e-08
1937 3.53660357590435e-08
1938 3.5501334678445e-08
1939 3.53313695966584e-08
1940 3.54642647009129e-08
1941 3.52967307419849e-08
1942 3.54317351907163e-08
1943 3.52592853218869e-08
1944 3.53951339910896e-08
1945 3.52240560155259e-08
1946 3.53601161637851e-08
1947 3.51904038731199e-08
1948 3.53232693284244e-08
1949 3.51561975597825e-08
1950 3.52877756362169e-08
1951 3.51215701099861e-08
1952 3.52546209436699e-08
1953 3.50855504074321e-08
1954 3.52204345905882e-08
1955 3.50513547067166e-08
1956 3.51854258311413e-08
1957 3.50171412761835e-08
1958 3.51496887920355e-08
1959 3.49830046575406e-08
1960 3.51172191093152e-08
1961 3.49460348829655e-08
1962 3.50811778360161e-08
1963 3.49136003161554e-08
1964 3.50462842435562e-08
1965 3.48798979163956e-08
1966 3.50141376022073e-08
1967 3.48435174359896e-08
1968 3.49805300186023e-08
1969 3.48107663572517e-08
1970 3.49471792475686e-08
1971 3.47756990914938e-08
1972 3.49120130820335e-08
1973 3.47447853463745e-08
1974 3.487760154397e-08
1975 3.47071412298838e-08
1976 3.48471439177711e-08
1977 3.46748766140159e-08
1978 3.48119792912271e-08
1979 3.46387351914945e-08
1980 3.47807931313504e-08
1981 3.46044714969462e-08
1982 3.4746882119796e-08
1983 3.45730546325029e-08
1984 3.47136930196612e-08
1985 3.45384617139732e-08
1986 3.46816260217153e-08
1987 3.45056425066215e-08
1988 3.4648825459227e-08
1989 3.44716625346742e-08
1990 3.46187666051545e-08
1991 3.44359134256678e-08
1992 3.45871344475945e-08
1993 3.44046301883338e-08
1994 3.45528101930448e-08
1995 3.43724848408389e-08
1996 3.45191242767662e-08
1997 3.43391161012896e-08
1998 3.44879705718171e-08
1999 3.43069201138579e-08
2000 3.44586834157834e-08
2001 3.42743286791958e-08
2002 3.44241819369984e-08
2003 3.42432753335231e-08
2004 3.4393652973197e-08
2005 3.42128871035907e-08
2006 3.43620321370253e-08
2007 3.41807970403174e-08
2008 3.4334485580878e-08
2009 3.41465594226875e-08
2010 3.43059484164687e-08
2011 3.41147980396084e-08
2012 3.42737862006892e-08
2013 3.40827184202031e-08
2014 3.42467022744763e-08
2015 3.40506455211997e-08
2016 3.42174528926709e-08
2017 3.4018832453242e-08
2018 3.41860072980982e-08
2019 3.39887537417205e-08
2020 3.41591949852837e-08
2021 3.39587227757843e-08
2022 3.41302309783131e-08
2023 3.39260200517e-08
2024 3.41015753710927e-08
2025 3.38994188984199e-08
2026 3.40722103406854e-08
2027 3.38630337750612e-08
2028 3.40401586302885e-08
2029 3.38389995735433e-08
2030 3.40060119079499e-08
2031 3.38075385368786e-08
2032 3.39818326804409e-08
2033 3.3778532724682e-08
2034 3.39496945518381e-08
2035 3.37466841466849e-08
2036 3.39231183181798e-08
2037 3.37170518125429e-08
2038 3.38885463608829e-08
2039 3.36900602879719e-08
2040 3.38631763678876e-08
2041 3.36599391694836e-08
2042 3.38316514150172e-08
2043 3.36288634654736e-08
2044 3.38006545095304e-08
2045 3.36007324817089e-08
2046 3.37741104514677e-08
2047 3.35729020093378e-08
2048 3.37447157683357e-08
2049 3.35409426823752e-08
2050 3.37104001517652e-08
2051 3.35185897166745e-08
2052 3.36808731407068e-08
2053 3.34860764392708e-08
2054 3.36555008790818e-08
2055 3.34524768534905e-08
2056 3.36280341020068e-08
2057 3.34251789495532e-08
2058 3.35967896136768e-08
2059 3.34000228667808e-08
2060 3.35649370846181e-08
2061 3.33701651773488e-08
2062 3.35361676002677e-08
2063 3.33442030793218e-08
2064 3.35034018941194e-08
2065 3.33173063706838e-08
2066 3.3475728686172e-08
2067 3.32886140821387e-08
2068 3.34445716143605e-08
2069 3.32612334110749e-08
2070 3.34175519838276e-08
2071 3.32330669881031e-08
2072 3.33858153505062e-08
2073 3.32053681213473e-08
2074 3.33551568969526e-08
2075 3.31844103218515e-08
2076 3.33231572398063e-08
2077 3.31528626755784e-08
2078 3.32980820392237e-08
2079 3.31214612647734e-08
2080 3.32701358358722e-08
2081 3.30973231195131e-08
2082 3.32393311761603e-08
2083 3.30682983922248e-08
2084 3.32107309033436e-08
2085 3.30443311173934e-08
2086 3.31783985725309e-08
2087 3.30178540111081e-08
2088 3.31486069262699e-08
2089 3.29947199610992e-08
2090 3.31150846746286e-08
2091 3.29706428039422e-08
2092 3.30849867229421e-08
2093 3.29431326482332e-08
2094 3.30577846958668e-08
2095 3.29164257426573e-08
2096 3.30283499556661e-08
2097 3.28906320059552e-08
2098 3.29962981262533e-08
2099 3.28671141816006e-08
2100 3.29685052027351e-08
2101 3.28388157784243e-08
2102 3.29423157541253e-08
2103 3.28115128136464e-08
2104 3.29099810310041e-08
2105 3.27899915713026e-08
2106 3.28767806971886e-08
2107 3.27688623600331e-08
2108 3.28449301698619e-08
2109 3.27409436193804e-08
2110 3.28218662624113e-08
2111 3.27155775450283e-08
2112 3.27845536689519e-08
2113 3.2696405178001e-08
2114 3.27560836475005e-08
2115 3.26693344210316e-08
2116 3.27288166304118e-08
2117 3.26420004168604e-08
2118 3.26997975894638e-08
2119 3.2619332935413e-08
2120 3.26690958820652e-08
2121 3.25962574792005e-08
2122 3.26389414022632e-08
2123 3.25719672285008e-08
2124 3.26067325830959e-08
2125 3.25511558640912e-08
2126 3.25774582821126e-08
2127 3.25272533312493e-08
2128 3.25489469104045e-08
2129 3.24990357396526e-08
2130 3.25223502168903e-08
2131 3.24769604667097e-08
2132 3.24918972698374e-08
2133 3.24526004500392e-08
2134 3.24620045966118e-08
2135 3.24313290769318e-08
2136 3.24297166007792e-08
2137 3.24064973056259e-08
2138 3.24056929941197e-08
2139 3.23823804821988e-08
2140 3.23747104788996e-08
2141 3.23584256560761e-08
2142 3.2347102263186e-08
2143 3.23339302632508e-08
2144 3.23186316777413e-08
2145 3.23083984961059e-08
2146 3.22946270552293e-08
2147 3.22804370740393e-08
2148 3.22698172567915e-08
2149 3.22521717517343e-08
2150 3.22429048247308e-08
2151 3.22295467418687e-08
2152 3.22141316508873e-08
2153 3.22067225129174e-08
2154 3.21874950890422e-08
2155 3.21760809911531e-08
2156 3.21681936759077e-08
2157 3.21495688977436e-08
2158 3.21411751240674e-08
2159 3.21266733673831e-08
2160 3.21195493389048e-08
2161 3.2109048636153e-08
2162 3.21122739745938e-08
2163 3.21210464091504e-08
2164 3.20955575721094e-08
2165 3.20798107769082e-08
2166 3.20626095522414e-08
2167 3.20530276540776e-08
2168 3.20367941970545e-08
2169 3.20293605624578e-08
2170 3.20109487983444e-08
2171 3.20027484721486e-08
2172 3.19858944060059e-08
2173 3.19782463202856e-08
2174 3.19619695647866e-08
2175 3.19483423290023e-08
2176 3.19381439097821e-08
2177 3.1925095086871e-08
2178 3.19129848558308e-08
2179 3.18956145537452e-08
2180 3.18914765531364e-08
2181 3.18693545979798e-08
2182 3.18653845041705e-08
2183 3.18444458411449e-08
2184 3.18392313389815e-08
2185 3.1821556894629e-08
2186 3.181208510461e-08
2187 3.17988662121849e-08
2188 3.17849128423919e-08
2189 3.17743707538565e-08
2190 3.17595474652599e-08
2191 3.1746173976499e-08
2192 3.17385506061196e-08
2193 3.17204536912641e-08
2194 3.17156795033302e-08
2195 3.16925765784237e-08
2196 3.16881361344112e-08
2197 3.1673489388595e-08
2198 3.16610544073281e-08
2199 3.16470999526253e-08
2200 3.16385133312114e-08
2201 3.16239412490038e-08
2202 3.16102350299552e-08
2203 3.16021463355742e-08
2204 3.15827746100172e-08
2205 3.15809845410353e-08
2206 3.15557946815037e-08
2207 3.15569884203892e-08
2208 3.15301124906853e-08
2209 3.15309651501838e-08
2210 3.15109322106544e-08
2211 3.15018743193285e-08
2212 3.14887893100302e-08
2213 3.14775199372619e-08
2214 3.14664530454145e-08
2215 3.14510458596651e-08
2216 3.14414196540547e-08
2217 3.14287722840145e-08
2218 3.14154978326986e-08
2219 3.1406357103414e-08
2220 3.13906297089161e-08
2221 3.13802956719567e-08
2222 3.1367483362077e-08
2223 3.1356068333821e-08
2224 3.1345127851079e-08
2225 3.13320572558062e-08
2226 3.13179233386585e-08
2227 3.13115760475213e-08
2228 3.12919820379953e-08
2229 3.12888775513276e-08
2230 3.1269972567971e-08
2231 3.12625539278244e-08
2232 3.12463085103687e-08
2233 3.12393607768691e-08
2234 3.12213068283107e-08
2235 3.12170154639357e-08
2236 3.1199909235724e-08
2237 3.11899131013593e-08
2238 3.1176904270902e-08
2239 3.11662458072348e-08
2240 3.11545067070984e-08
2241 3.11428218917609e-08
2242 3.11291787256085e-08
2243 3.11210789150085e-08
2244 3.11028335528984e-08
2245 3.10979339102069e-08
2246 3.10814969413098e-08
2247 3.10740712532454e-08
2248 3.10592138954568e-08
2249 3.10486047669922e-08
2250 3.10361221227407e-08
2251 3.10253077022704e-08
2252 3.1014636747706e-08
2253 3.09988917424064e-08
2254 3.09925678232403e-08
2255 3.09756300040931e-08
2256 3.09699169831479e-08
2257 3.09505519904274e-08
2258 3.09482033558073e-08
2259 3.09271079548257e-08
2260 3.09243554765803e-08
2261 3.09044322539531e-08
2262 3.09009181902464e-08
2263 3.08812293505767e-08
2264 3.08772216413011e-08
2265 3.08617185045268e-08
2266 3.08509711250959e-08
2267 3.08392484704711e-08
2268 3.08285256382934e-08
2269 3.08152993946376e-08
2270 3.08058716316228e-08
2271 3.07936422547783e-08
2272 3.07796645091507e-08
2273 3.07738125990831e-08
2274 3.07561852848082e-08
2275 3.07501698719737e-08
2276 3.07318604715334e-08
2277 3.07302436022283e-08
2278 3.07071357026345e-08
2279 3.07076501648851e-08
2280 3.06848810947002e-08
2281 3.06846102235969e-08
2282 3.06605703463969e-08
2283 3.06654034165632e-08
2284 3.06336053388367e-08
2285 3.06449024771727e-08
2286 3.06124762790017e-08
2287 3.0618297284235e-08
2288 3.05924258348078e-08
2289 3.0595148444279e-08
2290 3.05695803275441e-08
2291 3.05724436735488e-08
2292 3.05473401012168e-08
2293 3.05484075013851e-08
2294 3.05281151808945e-08
2295 3.05227565606803e-08
2296 3.05055886526961e-08
2297 3.05016411341619e-08
2298 3.04833350275313e-08
2299 3.04789479028678e-08
2300 3.04616099062738e-08
2301 3.04555025567588e-08
2302 3.04382856390895e-08
2303 3.04352867375179e-08
2304 3.0414697017811e-08
2305 3.04141404210423e-08
2306 3.0389670236719e-08
2307 3.03940688146653e-08
2308 3.03654316577973e-08
2309 3.0374986111692e-08
2310 3.03410926631997e-08
2311 3.03530701810661e-08
2312 3.03194893622738e-08
2313 3.03306105995382e-08
2314 3.02967514642649e-08
2315 3.03091215518858e-08
2316 3.02750066065727e-08
2317 3.02872066442195e-08
2318 3.02516682126885e-08
2319 3.02672646330659e-08
2320 3.0227440591668e-08
2321 3.02485999459368e-08
2322 3.02008272980991e-08
2323 3.0230852537505e-08
2324 3.01776476689941e-08
2325 3.02090793427023e-08
2326 3.01552088377566e-08
2327 3.01878856936444e-08
2328 3.01317876876261e-08
2329 3.01703132303821e-08
2330 3.0104250276608e-08
2331 3.01538588156625e-08
2332 3.00789645675259e-08
2333 3.01348930469292e-08
2334 3.00540108646441e-08
2335 3.01186567395195e-08
2336 3.00263044858973e-08
2337 3.01002667457695e-08
2338 3.00049718062745e-08
2339 3.00811161177972e-08
2340 2.99762347926169e-08
2341 3.0067470593087e-08
2342 2.99490972894834e-08
2343 3.00525701313337e-08
2344 2.99192279713534e-08
2345 3.00407292814775e-08
2346 2.98879868683155e-08
2347 3.00289007095778e-08
2348 2.98578768815894e-08
2349 3.00194053473035e-08
2350 2.98222732231235e-08
2351 3.00117266414102e-08
2352 2.97893855336628e-08
2353 3.00048213675019e-08
2354 2.97524890302547e-08
2355 3.00010595133404e-08
2356 2.97156190827152e-08
2357 2.99941667505355e-08
2358 2.96812836340088e-08
2359 2.99875109286418e-08
2360 2.96455954598152e-08
2361 2.99843674140821e-08
2362 2.96065380294142e-08
2363 2.99813920567882e-08
2364 2.95701018231398e-08
2365 2.99771139002925e-08
2366 2.95325523576029e-08
2367 2.99738895124246e-08
2368 2.94956725581663e-08
2369 2.99711983149376e-08
2370 2.94557686206787e-08
2371 2.99720352623378e-08
2372 2.94157367424219e-08
2373 2.9967977766221e-08
2374 2.93841688854268e-08
2375 2.99600770028441e-08
2376 2.93486244107299e-08
2377 2.99558940872746e-08
2378 2.93156638688785e-08
2379 2.99462255894944e-08
2380 2.92865583804147e-08
2381 2.99314739593548e-08
2382 2.92610804826232e-08
2383 2.99164572270527e-08
2384 2.92367966583384e-08
2385 2.99017871832064e-08
2386 2.92110004738166e-08
2387 2.98864187233683e-08
2388 2.9185476658089e-08
2389 2.98687195769975e-08
2390 2.91647375800785e-08
2391 2.9850032028822e-08
2392 2.91431740262471e-08
2393 2.98318842828493e-08
2394 2.91199977193735e-08
2395 2.98130416593878e-08
2396 2.90994852032433e-08
2397 2.97932315562743e-08
2398 2.90796816848626e-08
2399 2.97732045981824e-08
2400 2.9061656876106e-08
2401 2.97503698873935e-08
2402 2.9045490901991e-08
2403 2.97301090215196e-08
2404 2.90251594585733e-08
2405 2.97082908302659e-08
2406 2.90104247571144e-08
2407 2.96820384064755e-08
2408 2.89969943962909e-08
2409 2.96521493619029e-08
2410 2.89852958827996e-08
2411 2.96245236566239e-08
2412 2.89711913703528e-08
2413 2.95977544979564e-08
2414 2.89568069578028e-08
2415 2.95715647049555e-08
2416 2.89425812689537e-08
2417 2.95456778658387e-08
2418 2.89295397537526e-08
2419 2.95136486756498e-08
2420 2.89202228931362e-08
2421 2.94876052340243e-08
2422 2.89013252217085e-08
2423 2.94666905493735e-08
2424 2.88829614720765e-08
2425 2.94454511851061e-08
2426 2.88632897822438e-08
2427 2.94251870951445e-08
2428 2.88432694133256e-08
2429 2.94043131512378e-08
2430 2.88273894180069e-08
2431 2.93776404181045e-08
2432 2.88140909607826e-08
2433 2.93524332879436e-08
2434 2.87958647886555e-08
2435 2.93330418497106e-08
2436 2.87743315041666e-08
2437 2.93138970057782e-08
2438 2.87562829508481e-08
2439 2.92911345436409e-08
2440 2.87400453977682e-08
2441 2.92690111667326e-08
2442 2.87183246518996e-08
2443 2.92489758288994e-08
2444 2.87017403799439e-08
2445 2.92299350446168e-08
2446 2.86783028469184e-08
2447 2.92146544469407e-08
2448 2.86544860705895e-08
2449 2.91987211147315e-08
2450 2.86316487787541e-08
2451 2.91815248483207e-08
2452 2.86126625388405e-08
2453 2.91626172042037e-08
2454 2.85871218363987e-08
2455 2.91544031607582e-08
2456 2.8555242879813e-08
2457 2.91474250639912e-08
2458 2.85244007034446e-08
2459 2.91435624750758e-08
2460 2.84899347957612e-08
2461 2.91384505644032e-08
2462 2.84578797873358e-08
2463 2.91346710210583e-08
2464 2.84262550569458e-08
2465 2.91275270667413e-08
2466 2.83954392721331e-08
2467 2.91190697492283e-08
2468 2.83696017082669e-08
2469 2.91037472095468e-08
2470 2.83460202001873e-08
2471 2.90938721176914e-08
2472 2.83161256704467e-08
2473 2.90821969608501e-08
2474 2.82920778627016e-08
2475 2.90696041391758e-08
2476 2.82674202700672e-08
2477 2.90531787645598e-08
2478 2.82482472508949e-08
2479 2.90280612755822e-08
2480 2.82378015761253e-08
2481 2.89989506123689e-08
2482 2.82254585410868e-08
2483 2.89744957491234e-08
2484 2.82088515053403e-08
2485 2.8952606124788e-08
2486 2.8191853317594e-08
2487 2.89284699321879e-08
2488 2.81799405787986e-08
2489 2.88999538888834e-08
2490 2.81704331366317e-08
2491 2.88672531707501e-08
2492 2.81640950157147e-08
2493 2.88348793646698e-08
2494 2.81552145302566e-08
2495 2.88062063493744e-08
2496 2.81435660520746e-08
2497 2.87805279892694e-08
2498 2.81293173667319e-08
2499 2.87510696246329e-08
2500 2.81239762991436e-08
2501 2.87214784457923e-08
2502 2.81083422255524e-08
2503 2.86985384083671e-08
2504 2.80949279634068e-08
2505 2.86722979618848e-08
2506 2.8084749811752e-08
2507 2.86435225254067e-08
2508 2.80718146690173e-08
2509 2.86166470992999e-08
2510 2.80607990936765e-08
2511 2.8588659317208e-08
2512 2.80531268668227e-08
2513 2.85596280333866e-08
2514 2.80389136375625e-08
2515 2.85376035931151e-08
2516 2.80203516678235e-08
2517 2.851632158829e-08
2518 2.80059629893525e-08
2519 2.84940268007183e-08
2520 2.79893984187485e-08
2521 2.84712006934917e-08
2522 2.79745291695832e-08
2523 2.84501584502106e-08
2524 2.79553257169773e-08
2525 2.84272839727873e-08
2526 2.79432252698886e-08
2527 2.84059474797704e-08
2528 2.79216032996743e-08
2529 2.83885537357698e-08
2530 2.79027847971403e-08
2531 2.83673828187503e-08
2532 2.78904146167047e-08
2533 2.83448904621064e-08
2534 2.7869692502458e-08
2535 2.83286510505132e-08
2536 2.7849139022873e-08
2537 2.83131663898484e-08
2538 2.78273080758229e-08
2539 2.82964423989895e-08
2540 2.78050392343854e-08
2541 2.82814077556814e-08
2542 2.77860042201539e-08
2543 2.82629646068955e-08
2544 2.77674525877014e-08
2545 2.82433401639359e-08
2546 2.77506709736031e-08
2547 2.82264686226608e-08
2548 2.77279722051826e-08
2549 2.82101197697227e-08
2550 2.77107554429445e-08
2551 2.81891253319788e-08
2552 2.76923990614186e-08
2553 2.81724248232251e-08
2554 2.76727826715728e-08
2555 2.81537586155345e-08
2556 2.7654039248759e-08
2557 2.81337599816744e-08
2558 2.76406653842987e-08
2559 2.81105862602882e-08
2560 2.76245541821485e-08
2561 2.8090234529321e-08
2562 2.76065298314698e-08
2563 2.80705401300985e-08
2564 2.75932567941339e-08
2565 2.80452810788034e-08
2566 2.75789629864409e-08
2567 2.80253494882032e-08
2568 2.75625975643123e-08
2569 2.80018394533954e-08
2570 2.75510902092435e-08
2571 2.79789245622997e-08
2572 2.75341863771317e-08
2573 2.79599292503097e-08
2574 2.75169418435528e-08
2575 2.79398269775744e-08
2576 2.75030596490478e-08
2577 2.79179363371629e-08
2578 2.74880318160697e-08
2579 2.78952023904377e-08
2580 2.74751107449678e-08
2581 2.78725735847196e-08
2582 2.74613633561849e-08
2583 2.78521252410346e-08
2584 2.74440318634994e-08
2585 2.78329323055715e-08
2586 2.74275568676874e-08
2587 2.78091181702411e-08
2588 2.74179472621316e-08
2589 2.77855078021361e-08
2590 2.74035481917512e-08
2591 2.7757975404219e-08
2592 2.73950526863409e-08
2593 2.77392746295124e-08
2594 2.73759021478526e-08
2595 2.7711363103089e-08
2596 2.73714482197374e-08
2597 2.76924642470533e-08
2598 2.73497501526165e-08
2599 2.76607992251154e-08
2600 2.73508723522831e-08
2601 2.76423956735439e-08
2602 2.73266643202597e-08
2603 2.76072952944961e-08
2604 2.73291986432334e-08
2605 2.75931375759431e-08
2606 2.73036227853574e-08
2607 2.75614265070612e-08
2608 2.73040272473768e-08
2609 2.7543413090747e-08
2610 2.72824872795674e-08
2611 2.75080934837302e-08
2612 2.72825418621281e-08
2613 2.74938699811322e-08
2614 2.72596320842755e-08
2615 2.74623208316171e-08
2616 2.72557884315461e-08
2617 2.74495939271979e-08
2618 2.72283564208831e-08
2619 2.74217208255934e-08
2620 2.7227556740117e-08
2621 2.7406787454165e-08
2622 2.72018275306873e-08
2623 2.73770533834039e-08
2624 2.71979835404501e-08
2625 2.73659380136682e-08
2626 2.71722036084832e-08
2627 2.73370355861058e-08
2628 2.71680803234364e-08
2629 2.73210822676617e-08
2630 2.71485764551382e-08
2631 2.72916561137038e-08
2632 2.71426729152591e-08
2633 2.72736352886493e-08
2634 2.71229813844087e-08
2635 2.72447585463187e-08
2636 2.71152121555929e-08
2637 2.72337533915312e-08
2638 2.70887682702892e-08
2639 2.72178472215945e-08
2640 2.70689800445645e-08
2641 2.72059523231949e-08
2642 2.70525041943248e-08
2643 2.71816887147391e-08
2644 2.70395765460396e-08
2645 2.71635359609945e-08
2646 2.70223092759814e-08
2647 2.71425022007055e-08
2648 2.70066971497229e-08
2649 2.71270877423291e-08
2650 2.69891271347245e-08
2651 2.71014994634999e-08
2652 2.69806203720968e-08
2653 2.70820210699352e-08
2654 2.69645562107623e-08
2655 2.70613337141068e-08
2656 2.69476181893324e-08
2657 2.7048452831302e-08
2658 2.69290717618276e-08
2659 2.70255883803028e-08
2660 2.69175737939165e-08
2661 2.70052172997026e-08
2662 2.69025988328675e-08
2663 2.69848526344152e-08
2664 2.68887124788009e-08
2665 2.69669075985135e-08
2666 2.68723682776972e-08
2667 2.69446928575778e-08
2668 2.68602313782118e-08
2669 2.69273561588346e-08
2670 2.68456999079625e-08
2671 2.69032158013438e-08
2672 2.68336519191337e-08
2673 2.68811552199821e-08
2674 2.68201619231245e-08
2675 2.68697322267641e-08
2676 2.68001449834632e-08
2677 2.68439084500915e-08
2678 2.67905562023607e-08
2679 2.68260529712183e-08
2680 2.67756504164218e-08
2681 2.68072868923053e-08
2682 2.67631568744431e-08
2683 2.67825912259045e-08
2684 2.67516073368945e-08
2685 2.67670711693313e-08
2686 2.67371732445287e-08
2687 2.6740618773724e-08
2688 2.67244115195187e-08
2689 2.67242311045024e-08
2690 2.67117990184751e-08
2691 2.67025192166592e-08
2692 2.67008108092437e-08
2693 2.66802356942986e-08
2694 2.66869761593735e-08
2695 2.66634211540939e-08
2696 2.6672884551493e-08
2697 2.66398070984941e-08
2698 2.66597079487951e-08
2699 2.66245982167845e-08
2700 2.6645015096971e-08
2701 2.66055100150986e-08
2702 2.66284859062882e-08
2703 2.65887371353468e-08
2704 2.66125459686961e-08
2705 2.65705589832521e-08
2706 2.65992203978271e-08
2707 2.65532887548936e-08
2708 2.65819803955125e-08
2709 2.6539516024382e-08
2710 2.6561286796678e-08
2711 2.65285832083961e-08
2712 2.6537039495178e-08
2713 2.65243001373872e-08
2714 2.65068958031911e-08
2715 2.6525904177177e-08
2716 2.64672025358514e-08
2717 2.65185403975909e-08
2718 2.65122863853939e-08
2719 2.64339809925573e-08
2720 2.65063530922038e-08
2721 2.64210888401006e-08
2722 2.64819311954767e-08
2723 2.64149064626018e-08
2724 2.64557836129331e-08
2725 2.64098512479105e-08
2726 2.642824672483e-08
2727 2.640796172515e-08
2728 2.63820826391647e-08
2729 2.63989879809579e-08
2730 2.64839820884877e-08
2731 2.62741869183625e-08
2732 2.64693203548827e-08
2733 2.62638835544227e-08
2734 2.64542258745504e-08
2735 2.62230172671796e-08
2736 2.64300591876765e-08
2737 2.63458501643665e-08
2738 2.62825722048277e-08
2739 2.63435834202408e-08
2740 2.62799176655726e-08
2741 2.62778798252317e-08
2742 2.64538771177492e-08
2743 2.61225319015956e-08
2744 2.64270630543351e-08
2745 2.61044057283666e-08
2746 2.64009343371452e-08
2747 2.61217829848892e-08
2748 2.62952116842907e-08
2749 2.63335103156681e-08
2750 2.61075348260142e-08
2751 2.63151477226664e-08
2752 2.61361539695137e-08
2753 2.62679230011997e-08
2754 2.61333539697262e-08
2755 2.61821243767724e-08
2756 2.63161749023411e-08
2757 2.60173423756171e-08
2758 2.62979565455179e-08
2759 2.60401480354222e-08
2760 2.62432180164751e-08
2761 2.60826444640205e-08
2762 2.61568436827897e-08
2763 2.61919949222644e-08
2764 2.60585755229581e-08
2765 2.61273631980696e-08
2766 2.60018974940213e-08
2767 2.62957530814401e-08
2768 2.58846066387708e-08
2769 2.62521409841732e-08
2770 2.59047276864877e-08
2771 2.61232814535717e-08
2772 2.61212843477576e-08
2773 2.59566145310863e-08
2774 2.60909840739121e-08
2775 2.60804236080769e-08
2776 2.59674439566648e-08
2777 2.60513501353365e-08
2778 2.59111705342985e-08
2779 2.61975909059409e-08
2780 2.58014712097854e-08
2781 2.61469474236442e-08
2782 2.5829864993554e-08
2783 2.60320951042115e-08
2784 2.60182610292148e-08
2785 2.58856175647804e-08
2786 2.59820384693743e-08
2787 2.59093274053157e-08
2788 2.58959898962186e-08
2789 2.60444201618526e-08
2790 2.57917263219998e-08
2791 2.59885306495011e-08
2792 2.58425287276154e-08
2793 2.59366133734673e-08
2794 2.59602848426965e-08
2795 2.57919466866152e-08
2796 2.59173249594902e-08
2797 2.58113842526164e-08
2798 2.58934710033465e-08
2799 2.58504786496694e-08
2800 2.59369370751994e-08
2801 2.57412377093758e-08
2802 2.58879745036467e-08
2803 2.57404055976629e-08
2804 2.58717547108045e-08
2805 2.5819463307597e-08
2806 2.57719267489076e-08
2807 2.57724212300303e-08
2808 2.57224063027195e-08
2809 2.59161390432361e-08
2810 2.55956805852442e-08
2811 2.5798543061617e-08
2812 2.58238359918117e-08
2813 2.56212471914097e-08
2814 2.5724795609694e-08
2815 2.58582595120682e-08
2816 2.55567739344542e-08
2817 2.57378468324632e-08
2818 2.57949231194932e-08
2819 2.55770418182699e-08
2820 2.56978699413324e-08
2821 2.57794065319228e-08
2822 2.55360571626007e-08
2823 2.56784218868322e-08
2824 2.57594250987481e-08
2825 2.55257574304224e-08
2826 2.56540148260154e-08
2827 2.57285630376192e-08
2828 2.54966298425252e-08
2829 2.56228434658556e-08
2830 2.57254335047641e-08
2831 2.54652107665176e-08
2832 2.56099883844119e-08
2833 2.56875563369174e-08
2834 2.54528720555758e-08
2835 2.5581573587008e-08
2836 2.56732519725578e-08
2837 2.54241635786467e-08
2838 2.55683946375207e-08
2839 2.56333874575354e-08
2840 2.54208184182758e-08
2841 2.55307717753617e-08
2842 2.56289987392577e-08
2843 2.54023912977797e-08
2844 2.55659446419809e-08
2845 2.54562796997426e-08
2846 2.55041469960915e-08
2847 2.54628960487668e-08
2848 2.54721418599946e-08
2849 2.54740645337925e-08
2850 2.54378481052697e-08
2851 2.54451345396145e-08
2852 2.54209926884297e-08
2853 2.5534893465462e-08
2854 2.53542796773143e-08
2855 2.54176227629799e-08
2856 2.54115483220296e-08
2857 2.54423997505526e-08
2858 2.53709750657194e-08
2859 2.54109675983383e-08
2860 2.53625006447766e-08
2861 2.54205317355982e-08
2862 2.5339516106504e-08
2863 2.54077955907928e-08
2864 2.53216730323746e-08
2865 2.53942515200389e-08
2866 2.53061391806142e-08
2867 2.53840992976428e-08
2868 2.52888010150443e-08
2869 2.53624345116776e-08
2870 2.52796249298193e-08
2871 2.5353084251778e-08
2872 2.52580783453027e-08
2873 2.53171562658583e-08
2874 2.5255843355998e-08
2875 2.53236959515402e-08
2876 2.52322747897882e-08
2877 2.52766231536583e-08
2878 2.52271282623617e-08
2879 2.52390895638666e-08
2880 2.5208060073556e-08
2881 2.53307663975466e-08
2882 2.51706559455389e-08
2883 2.52027723295356e-08
2884 2.52401260409929e-08
2885 2.52103345661503e-08
2886 2.51499155092816e-08
2887 2.51974158984591e-08
2888 2.51799885109261e-08
2889 2.51590718074457e-08
2890 2.51626103373592e-08
2891 2.51650246421153e-08
2892 2.51816733809473e-08
2893 2.51146508072342e-08
2894 2.51486578750626e-08
2895 2.51219379592271e-08
2896 2.51137060196527e-08
2897 2.51189622253456e-08
2898 2.50909387806697e-08
2899 2.51038480700849e-08
2900 2.508561961867e-08
2901 2.51109905728963e-08
2902 2.50489062838533e-08
2903 2.508047055505e-08
2904 2.50581963621155e-08
2905 2.50561210153322e-08
2906 2.50515209461177e-08
2907 2.50266688479606e-08
2908 2.5041226697553e-08
2909 2.50221824866781e-08
2910 2.50205076897103e-08
2911 2.5010108272383e-08
2912 2.50067496501138e-08
2913 2.4992190902795e-08
2914 2.49955247353384e-08
2915 2.49779484231549e-08
2916 2.49786699446641e-08
2917 2.49648279986303e-08
2918 2.49630503190712e-08
2919 2.49518670993876e-08
2920 2.49501876896652e-08
2921 2.49362157194177e-08
2922 2.49345069369689e-08
2923 2.49213410390414e-08
2924 2.49238098239957e-08
2925 2.4908025516357e-08
2926 2.49060027419645e-08
2927 2.48953326185131e-08
2928 2.48912390188316e-08
2929 2.48825342579195e-08
2930 2.48757842187253e-08
2931 2.48689798370005e-08
2932 2.48596716454497e-08
2933 2.4853132368774e-08
2934 2.48482682958517e-08
2935 2.4840678271465e-08
2936 2.48339942197084e-08
2937 2.48261987558784e-08
2938 2.48172656653889e-08
2939 2.48112835730119e-08
2940 2.48023216056215e-08
2941 2.47991454480623e-08
2942 2.47880846810933e-08
2943 2.47830255963866e-08
2944 2.47755431195262e-08
2945 2.47668502579845e-08
2946 2.47607732153376e-08
2947 2.47531107795407e-08
2948 2.47453925821262e-08
2949 2.47396890367124e-08
2950 2.4732151368001e-08
2951 2.4724123484976e-08
2952 2.47155637658203e-08
2953 2.47105380464241e-08
2954 2.47002972095789e-08
2955 2.46983017755387e-08
2956 2.46904575413875e-08
2957 2.4687928720013e-08
2958 2.4677415825014e-08
2959 2.46716168008643e-08
2960 2.46638786154385e-08
2961 2.46601205788899e-08
2962 2.46487105708404e-08
2963 2.46460380430857e-08
2964 2.46374516834624e-08
2965 2.4631490322502e-08
2966 2.46239910173252e-08
2967 2.46200920470674e-08
2968 2.46118382083438e-08
2969 2.46039460169989e-08
2970 2.45958007198954e-08
2971 2.45915572845146e-08
2972 2.45836292498414e-08
2973 2.45774703997448e-08
2974 2.45680359931555e-08
2975 2.45655936517153e-08
2976 2.45538055407835e-08
2977 2.45486773104986e-08
2978 2.45413564923247e-08
2979 2.45360027788522e-08
2980 2.45277041177605e-08
2981 2.45222093746555e-08
2982 2.4516441349931e-08
2983 2.45089808168508e-08
2984 2.45006092938205e-08
2985 2.44940533644655e-08
2986 2.44891103897693e-08
2987 2.44809465643758e-08
2988 2.44744235422534e-08
2989 2.44656873527038e-08
2990 2.44614526274667e-08
2991 2.44535193190121e-08
2992 2.44485333409372e-08
2993 2.44387437409177e-08
2994 2.44352687406302e-08
2995 2.44239695135207e-08
2996 2.44204461692377e-08
2997 2.44132798759011e-08
2998 2.44058503253708e-08
2999 2.43958349159801e-08
3000 1.02381488764069e-08
3001 1.03281380785386e-08
3002 1.05042228980418e-08
3003 1.06131276146337e-08
3004 1.06563617779665e-08
3005 1.06688545891132e-08
3006 1.06704480416497e-08
3007 1.06690575367024e-08
3008 1.06668036801011e-08
3009 1.06643891404218e-08
3010 1.06620916493078e-08
3011 1.06597644212725e-08
3012 1.0657478600351e-08
3013 1.06552703470764e-08
3014 1.06531258466797e-08
3015 1.06510005122157e-08
3016 1.06489389742753e-08
3017 1.06469213314586e-08
3018 1.06449676700038e-08
3019 1.06429916054973e-08
3020 1.06410451488625e-08
3021 1.06392065476191e-08
3022 1.06373639087443e-08
3023 1.06355581858397e-08
3024 1.06337716073435e-08
3025 1.06320196106707e-08
3026 1.06302701445293e-08
3027 1.06285885411728e-08
3028 1.06268619751854e-08
3029 1.06251552069309e-08
3030 1.06235924175468e-08
3031 1.06219491559645e-08
3032 1.06203811536948e-08
3033 1.06188081810121e-08
3034 1.06172257618442e-08
3035 1.06157470051299e-08
3036 1.06141814856381e-08
3037 1.06126667444029e-08
3038 1.06111207348464e-08
3039 1.06096904115838e-08
3040 1.06081599604846e-08
3041 1.0606756869147e-08
3042 1.06054026676461e-08
3043 1.06039205176045e-08
3044 1.06025208780058e-08
3045 1.06010390216738e-08
3046 1.05997228844468e-08
3047 1.05983555212519e-08
3048 1.05969627899605e-08
3049 1.05955841329108e-08
3050 1.0594279556686e-08
3051 1.05929027826301e-08
3052 1.05916633676845e-08
3053 1.05903006844682e-08
3054 1.05889966575262e-08
3055 1.05876971342178e-08
3056 1.05864071102274e-08
3057 1.05851586941763e-08
3058 1.05838634961303e-08
3059 1.05825241940721e-08
3060 1.05812418785545e-08
3061 1.05800832135849e-08
3062 1.05788176298777e-08
3063 1.05775367980343e-08
3064 1.05763509718332e-08
3065 1.05750973562746e-08
3066 1.05739067801092e-08
3067 1.05726875228723e-08
3068 1.0571478995372e-08
3069 1.05703035006843e-08
3070 1.05690622487359e-08
3071 1.05678990365288e-08
3072 1.05666724406345e-08
3073 1.05655138493144e-08
3074 1.05642532014782e-08
3075 1.0563098815336e-08
3076 1.05619150504721e-08
3077 1.05607936806001e-08
3078 1.05596509924061e-08
3079 1.05584986044294e-08
3080 1.05572846933882e-08
3081 1.05562493526584e-08
3082 1.05550201059817e-08
3083 1.05538989689929e-08
3084 1.05527562265922e-08
3085 1.05516505466557e-08
3086 1.05504940921541e-08
3087 1.05494321654964e-08
3088 1.0548299375357e-08
3089 1.05471796939122e-08
3090 1.05460232875249e-08
3091 1.05450017461317e-08
3092 1.05438256195051e-08
3093 1.05427148953396e-08
3094 1.05415620984539e-08
3095 1.05406054392576e-08
3096 1.05394376699736e-08
3097 1.05383221839644e-08
3098 1.05372559828371e-08
3099 1.05361112209407e-08
3100 1.05350176191343e-08
3101 1.05340124700593e-08
3102 1.05329196302545e-08
3103 1.05318072436411e-08
3104 1.05307000689836e-08
3105 1.05296964837687e-08
3106 1.05286063524362e-08
3107 1.05275483289619e-08
3108 1.05264462500199e-08
3109 1.05253948072814e-08
3110 1.0524345364249e-08
3111 1.05231998530908e-08
3112 1.05222600378513e-08
3113 1.05211037396691e-08
3114 1.05201038485964e-08
3115 1.0519072321441e-08
3116 1.05179579287656e-08
3117 1.0516954219747e-08
3118 1.05158375674486e-08
3119 1.05148711281727e-08
3120 1.05138007864408e-08
3121 1.05127938110489e-08
3122 1.0511782207373e-08
3123 1.0510716443396e-08
3124 1.05097017492201e-08
3125 1.05086981424807e-08
3126 1.05076256667891e-08
3127 1.05066652416747e-08
3128 1.05055729155701e-08
3129 1.05045796032904e-08
3130 1.05035523311342e-08
3131 1.0502522176617e-08
3132 1.05015150475979e-08
3133 1.05005376409145e-08
3134 1.04994967507488e-08
3135 1.04984580039796e-08
3136 1.0497527442449e-08
3137 1.04964186101592e-08
3138 1.04954366947213e-08
3139 1.04943988958051e-08
3140 1.04934632506326e-08
3141 1.04924191091371e-08
3142 1.04914239198867e-08
3143 1.04904016119817e-08
3144 1.04893644168325e-08
3145 1.04884412638617e-08
3146 1.04874189488374e-08
3147 1.04863908770708e-08
3148 1.04854530528775e-08
3149 1.04844207740873e-08
3150 1.04834559543354e-08
3151 1.04825643788137e-08
3152 1.04815429505245e-08
3153 1.04804743039089e-08
3154 1.04795320340217e-08
3155 1.04786177620303e-08
3156 1.04775777809568e-08
3157 1.04765509131577e-08
3158 1.04756315822685e-08
3159 1.04747047837556e-08
3160 1.04735879432882e-08
3161 1.04726254879783e-08
3162 1.04717838177865e-08
3163 1.047074403486e-08
3164 1.04697979289164e-08
3165 1.04687793142794e-08
3166 1.04678491483073e-08
3167 1.04668822751425e-08
3168 1.0465884461533e-08
3169 1.04649222410352e-08
3170 1.04639611325924e-08
3171 1.04629998350508e-08
3172 1.04620191591492e-08
3173 1.04610632534752e-08
3174 1.04601435551854e-08
3175 1.04591583493796e-08
3176 1.04582175980139e-08
3177 1.04573042785244e-08
3178 1.04562655000723e-08
3179 1.04553198564533e-08
3180 1.04543970766147e-08
3181 1.04534396437178e-08
3182 1.04524634730319e-08
3183 1.04516059019205e-08
3184 1.04505913568753e-08
3185 1.04497167464634e-08
3186 1.04487144898247e-08
3187 1.04478509042327e-08
3188 1.04468219244702e-08
3189 1.04459258107731e-08
3190 1.04449778905863e-08
3191 1.04440451337146e-08
3192 1.04431177889858e-08
3193 1.04421197145987e-08
3194 1.04412124224823e-08
3195 1.04402675317888e-08
3196 1.04394032676147e-08
3197 1.04384804061608e-08
3198 1.04375274036089e-08
3199 1.04365798846012e-08
3200 1.04356233113778e-08
3201 1.04346798816268e-08
3202 1.04337961507328e-08
3203 1.04328880862897e-08
3204 1.04319288076332e-08
3205 1.04309902735217e-08
3206 1.04300384794198e-08
3207 1.04290954842934e-08
3208 1.04281588895472e-08
3209 1.04272673636802e-08
3210 1.04263385518749e-08
3211 1.04254192423298e-08
3212 1.04244798281861e-08
3213 1.04235588840595e-08
3214 1.0422591970663e-08
3215 1.04217018400965e-08
3216 1.04207808001161e-08
3217 1.04198125046029e-08
3218 1.04188902832897e-08
3219 1.04179822655037e-08
3220 1.04171161271482e-08
3221 1.04161650637674e-08
3222 1.04152322966539e-08
3223 1.04143853432576e-08
3224 1.04134751252177e-08
3225 1.04124622853502e-08
3226 1.04115725491349e-08
3227 1.04106711548146e-08
3228 1.04098276295928e-08
3229 1.04089072842789e-08
3230 1.04079306831734e-08
3231 1.04070465247463e-08
3232 1.04061709661901e-08
3233 1.04052772190028e-08
3234 1.04043399348636e-08
3235 1.04034635030059e-08
3236 1.04025777723088e-08
3237 1.04016815526131e-08
3238 1.040074800715e-08
3239 1.03998786488285e-08
3240 1.03989163107027e-08
3241 1.03980559946898e-08
3242 1.03971634762001e-08
3243 1.03962451786926e-08
3244 1.03953686880764e-08
3245 1.03944496044478e-08
3246 1.03935194589594e-08
3247 1.03926769939311e-08
3248 1.03917952045951e-08
3249 1.0390876865482e-08
3250 1.03900066920903e-08
3251 1.03890978543769e-08
3252 1.03882046201126e-08
3253 1.03873245875519e-08
3254 1.03864483950167e-08
3255 1.03855061245328e-08
3256 1.03846847500783e-08
3257 1.03837762590042e-08
3258 1.03828890920532e-08
3259 1.03819971604691e-08
3260 1.03811327387959e-08
3261 1.03801990749275e-08
3262 1.03792790324964e-08
3263 1.03784110593447e-08
3264 1.03775968672698e-08
3265 1.0376729226047e-08
3266 1.0375822782141e-08
3267 1.03748975526646e-08
3268 1.03739869402458e-08
3269 1.03731222708126e-08
3270 1.03722699297121e-08
3271 1.03713431888752e-08
3272 1.03704474264665e-08
3273 1.03695551903621e-08
3274 1.03687055001966e-08
3275 1.03678752999964e-08
3276 1.03669789747046e-08
3277 1.03660501693387e-08
3278 1.03651507860902e-08
3279 1.03643017064781e-08
3280 1.03634376120709e-08
3281 1.03625222526021e-08
3282 1.03617013439278e-08
3283 1.03608436502478e-08
3284 1.0359876472979e-08
3285 1.0359110946323e-08
3286 1.03581940749525e-08
3287 1.03573900725601e-08
3288 1.03564205508888e-08
3289 1.03555443273784e-08
3290 1.03547321371328e-08
3291 1.03538140837495e-08
3292 1.03529714127193e-08
3293 1.03520765605825e-08
3294 1.03512346803919e-08
3295 1.03503732992727e-08
3296 1.03495057507674e-08
3297 1.0348645427441e-08
3298 1.03477979498667e-08
3299 1.03469110867144e-08
3300 1.03460401244954e-08
3301 1.03451724294407e-08
3302 1.03443557413485e-08
3303 1.03435160234699e-08
3304 1.03425924843359e-08
3305 1.03416927021427e-08
3306 1.03409024764245e-08
3307 1.03400216851929e-08
3308 1.03391337702152e-08
3309 1.03382291479631e-08
3310 1.03374746053264e-08
3311 1.03365562745261e-08
3312 1.03357109274699e-08
3313 1.03348396725544e-08
3314 1.03340512420252e-08
3315 1.03331695853609e-08
3316 1.03322331457684e-08
3317 1.03313867499816e-08
3318 1.03305910300888e-08
3319 1.03296553883858e-08
3320 1.03288395884721e-08
3321 1.03280122447924e-08
3322 1.03271621485768e-08
3323 1.03262605773702e-08
3324 1.03254735662611e-08
3325 1.03246798983242e-08
3326 1.03237111920979e-08
3327 1.03229041806957e-08
3328 1.03219903193202e-08
3329 1.03211825549787e-08
3330 1.03203408831354e-08
3331 1.03194777271026e-08
3332 1.03186641162156e-08
3333 1.0317747836136e-08
3334 1.03169071813541e-08
3335 1.0316046491668e-08
3336 1.03152223832614e-08
3337 1.03143606502709e-08
3338 1.03135475961053e-08
3339 1.03127087525273e-08
3340 1.03118622285375e-08
3341 1.03109820120822e-08
3342 1.0310187368287e-08
3343 1.03093394163978e-08
3344 1.03085019367288e-08
3345 1.03076544650665e-08
3346 1.03067999275841e-08
3347 1.03059374129133e-08
3348 1.03051160735834e-08
3349 1.03042933047442e-08
3350 1.03033960341253e-08
3351 1.03026313097165e-08
3352 1.03017784828241e-08
3353 1.03009385043068e-08
3354 1.03000852501728e-08
3355 1.02993040840688e-08
3356 1.02984281350471e-08
3357 1.02976499208596e-08
3358 1.02967903330142e-08
3359 1.02959792054186e-08
3360 1.02951146712105e-08
3361 1.02943238818737e-08
3362 1.02934725434434e-08
3363 1.02926674995785e-08
3364 1.02918466039631e-08
3365 1.02910556569191e-08
3366 1.02901285412155e-08
3367 1.02893678504828e-08
3368 1.02884850228108e-08
3369 1.02876364150711e-08
3370 1.02868676085666e-08
3371 1.02859863353039e-08
3372 1.02852054252034e-08
3373 1.02843541798792e-08
3374 1.02835639479715e-08
3375 1.02827076544909e-08
3376 1.02819284314576e-08
3377 1.02810764273376e-08
3378 1.02802574870192e-08
3379 1.02794810586809e-08
3380 1.02786137601008e-08
3381 1.02777726718462e-08
3382 1.0276985714569e-08
3383 1.02761105390814e-08
3384 1.02753430086672e-08
3385 1.02745200036974e-08
3386 1.02736570634226e-08
3387 1.02728655078099e-08
3388 1.02720918244148e-08
3389 1.02712235648256e-08
3390 1.02703636609969e-08
3391 1.02695768515598e-08
3392 1.02687472173235e-08
3393 1.0267936792735e-08
3394 1.02671742922278e-08
3395 1.02663177177498e-08
3396 1.02655255979078e-08
3397 1.02646810117846e-08
3398 1.02638796277837e-08
3399 1.02630272156567e-08
3400 1.02621930977292e-08
3401 1.02614121075817e-08
3402 1.02605692500896e-08
3403 1.02597317941516e-08
3404 1.0258974184163e-08
3405 1.02581165444676e-08
3406 1.02572864238981e-08
3407 1.02564829817381e-08
3408 1.02557098450307e-08
3409 1.0254868476986e-08
3410 1.02540836868087e-08
3411 1.02532527080229e-08
3412 1.02524461911557e-08
3413 1.0251647519402e-08
3414 1.02508309648003e-08
3415 1.02499640793341e-08
3416 1.02492328037801e-08
3417 1.02483821728949e-08
3418 1.02476497820941e-08
3419 1.0246719942833e-08
3420 1.02458863628085e-08
3421 1.02450942719295e-08
3422 1.02443428370291e-08
3423 1.02435426068831e-08
3424 1.02426948082601e-08
3425 1.02418960420264e-08
3426 1.02410747558346e-08
3427 1.02403298502668e-08
3428 1.02394798669192e-08
3429 1.02386654064063e-08
3430 1.0237882752645e-08
3431 1.02370138292812e-08
3432 1.02362309504778e-08
3433 1.02354504590979e-08
3434 1.02346237453171e-08
3435 1.02338125602258e-08
3436 1.02330202472189e-08
3437 1.02322221950946e-08
3438 1.02313567380136e-08
3439 1.02305595128666e-08
3440 1.02297445500193e-08
3441 1.02290189141901e-08
3442 1.02282322240327e-08
3443 1.02273504854478e-08
3444 1.02265839145715e-08
3445 1.02257823090118e-08
3446 1.02249800028181e-08
3447 1.02241921439428e-08
3448 1.02233917242678e-08
3449 1.02225216927904e-08
3450 1.02218148081562e-08
3451 1.0220931347879e-08
3452 1.02201805413787e-08
3453 1.02193588969457e-08
3454 1.02185812720645e-08
3455 1.02177427446365e-08
3456 1.02169608761499e-08
3457 1.02161631207326e-08
3458 1.02153622539292e-08
3459 1.02146038690326e-08
3460 1.02137972337046e-08
3461 1.02130192943528e-08
3462 1.02122020335382e-08
3463 1.0211350112907e-08
3464 1.02105761358717e-08
3465 1.02098199681322e-08
3466 1.02090246837938e-08
3467 1.0208230875372e-08
3468 1.02074146844378e-08
3469 1.0206606004981e-08
3470 1.0205887478007e-08
3471 1.02050357058125e-08
3472 1.02042666966784e-08
3473 1.02034885666041e-08
3474 1.02027039215746e-08
3475 1.02019007740956e-08
3476 1.02011710417099e-08
3477 1.02003123084587e-08
3478 1.01994917006271e-08
3479 1.019873630978e-08
3480 1.01979647758599e-08
3481 1.01971689355357e-08
3482 1.01963808138289e-08
3483 1.01955995422814e-08
3484 1.01947583446976e-08
3485 1.0194017748183e-08
3486 1.01932636183272e-08
3487 1.01924498645128e-08
3488 1.01916743256575e-08
3489 1.01908351022167e-08
3490 1.01900591649717e-08
3491 1.0189307008579e-08
3492 1.01884144428771e-08
3493 1.01877085737917e-08
3494 1.01868737313049e-08
3495 1.01861510946599e-08
3496 1.01853583493045e-08
3497 1.01845924959515e-08
3498 1.01838043567448e-08
3499 1.01830454438678e-08
3500 1.01822148446812e-08
3501 1.0181435365797e-08
3502 1.01806588046482e-08
3503 1.01798405269665e-08
3504 1.01790961243159e-08
3505 1.01783185309651e-08
3506 1.0177453255239e-08
3507 1.0176713068355e-08
3508 1.01759258798739e-08
3509 1.01751423140783e-08
3510 1.01743579117852e-08
3511 1.01736601137298e-08
3512 1.0172785328541e-08
3513 1.01720883523559e-08
3514 1.01712943637589e-08
3515 1.01705416871573e-08
3516 1.0169671633406e-08
3517 1.01689432341623e-08
3518 1.01681339835236e-08
3519 1.01673907599714e-08
3520 1.01666269075179e-08
3521 1.0165818192312e-08
3522 1.01650237561979e-08
3523 1.01642970913862e-08
3524 1.0163506129271e-08
3525 1.01627251824082e-08
3526 1.01619157468896e-08
3527 1.01610792056228e-08
3528 1.01603659098254e-08
3529 1.01595362996115e-08
3530 1.01588134330394e-08
3531 1.01579826478487e-08
3532 1.01572662521976e-08
3533 1.01564957726979e-08
3534 1.01557372302885e-08
3535 1.01549437225984e-08
3536 1.01541843544606e-08
3537 1.01533396593412e-08
3538 1.01526599138452e-08
3539 1.01518422845337e-08
3540 1.01510215389372e-08
3541 1.01502871586445e-08
3542 1.01494993214324e-08
3543 1.01487364163738e-08
3544 1.01479527945259e-08
3545 1.01471416376203e-08
3546 1.01464334493345e-08
3547 1.01456676835782e-08
3548 1.01448876919791e-08
3549 1.01441009940639e-08
3550 1.01433540662665e-08
3551 1.01426356054757e-08
3552 1.01417671392462e-08
3553 1.01410886372e-08
3554 1.01402721275484e-08
3555 1.01394891123263e-08
3556 1.01386653099944e-08
3557 1.01379972639176e-08
3558 1.01371602187345e-08
3559 1.01363603001309e-08
3560 1.01356498953958e-08
3561 1.01349374341808e-08
3562 1.01341286306428e-08
3563 1.01333614223043e-08
3564 1.01326438025212e-08
3565 1.01317908742932e-08
3566 1.01310580527259e-08
3567 1.01302828868777e-08
3568 1.01295718654276e-08
3569 1.01287508109255e-08
3570 1.01280038391632e-08
3571 1.01272604209751e-08
3572 1.01264631971604e-08
3573 1.01257653229436e-08
3574 1.01249916299867e-08
3575 1.01241163634747e-08
3576 1.01233687750529e-08
3577 1.01227092096368e-08
3578 1.01218870920822e-08
3579 1.01211545164154e-08
3580 1.01204005208688e-08
3581 1.01196645996421e-08
3582 1.01188667403773e-08
3583 1.01180581831145e-08
3584 1.01173040702451e-08
3585 1.01165756222488e-08
3586 1.01157556503251e-08
3587 1.01150639908038e-08
3588 1.01142580398728e-08
3589 1.01135413671377e-08
3590 1.01127288096997e-08
3591 1.01119984352305e-08
3592 1.01112348365462e-08
3593 1.01103974408517e-08
3594 1.01097173819398e-08
3595 1.01089558671719e-08
3596 1.01081560670568e-08
3597 1.01074521439704e-08
3598 1.01066557462032e-08
3599 1.0105925442469e-08
3600 1.01051721195927e-08
3601 1.01043856253896e-08
3602 1.01037133809906e-08
3603 1.01028915733131e-08
3604 1.01021862323331e-08
3605 1.01014121333814e-08
3606 1.0100597020446e-08
3607 1.00999205292357e-08
3608 1.0099143771522e-08
3609 1.00984146721578e-08
3610 1.00976911822648e-08
3611 1.00969031092552e-08
3612 1.0096146979402e-08
3613 1.00953454007791e-08
3614 1.00946752215059e-08
3615 1.00939122982119e-08
3616 1.00930960800355e-08
3617 1.00923717846341e-08
3618 1.0091619405253e-08
3619 1.00908546235984e-08
3620 1.00900935383202e-08
3621 1.00893757663811e-08
3622 1.00886794186378e-08
3623 1.00878389576314e-08
3624 1.00871352465004e-08
3625 1.00863572030374e-08
3626 1.00855788892629e-08
3627 1.00849347833692e-08
3628 1.00841263523527e-08
3629 1.00833505097819e-08
3630 1.00826305172996e-08
3631 1.00818730513069e-08
3632 1.00811187680599e-08
3633 1.00804546074168e-08
3634 1.00796795486496e-08
3635 1.00789577290039e-08
3636 1.00781493082153e-08
3637 1.00774059510617e-08
3638 1.00766481860165e-08
3639 1.00759360171648e-08
3640 1.00751978456409e-08
3641 1.00744813518322e-08
3642 1.00736915297356e-08
3643 1.00729186390119e-08
3644 1.00721964165912e-08
3645 1.00714728414053e-08
3646 1.00707013445528e-08
3647 1.0069943881294e-08
3648 1.00692608870495e-08
3649 1.00684709472276e-08
3650 1.0067676497999e-08
3651 1.00669703732154e-08
3652 1.00662108357685e-08
3653 1.00655320804249e-08
3654 1.00647508409074e-08
3655 1.00640326538698e-08
3656 1.00632602654943e-08
3657 1.00625558096951e-08
3658 1.00617555661986e-08
3659 1.00609847261263e-08
3660 1.0060268974349e-08
3661 1.00595033599993e-08
3662 1.00588144116981e-08
3663 1.00580065820893e-08
3664 1.00572868484416e-08
3665 1.0056514617135e-08
3666 1.0055845291887e-08
3667 1.00550505703623e-08
3668 1.00543346818888e-08
3669 1.00535138767982e-08
3670 1.00528218493628e-08
3671 1.00521114810848e-08
3672 1.00513677146197e-08
3673 1.00506175028936e-08
3674 1.00498720418535e-08
3675 1.00491214368864e-08
3676 1.00484062755257e-08
3677 1.00476991784504e-08
3678 1.00468634576245e-08
3679 1.0046188193058e-08
3680 1.00454366574465e-08
3681 1.00446633071322e-08
3682 1.004389148547e-08
3683 1.00432285567165e-08
3684 1.00424696518886e-08
3685 1.00417551670839e-08
3686 1.00410364315406e-08
3687 1.00402285132806e-08
3688 1.00395453646318e-08
3689 1.00387814428171e-08
3690 1.00379952137908e-08
3691 1.00373011263089e-08
3692 1.00365625352039e-08
3693 1.00358253049826e-08
3694 1.00350478254851e-08
3695 1.0034308716933e-08
3696 1.00336389831784e-08
3697 1.00328601862626e-08
3698 1.00321391022506e-08
3699 1.00314028282644e-08
3700 1.00306936003935e-08
3701 1.00299273902565e-08
3702 1.00291602941754e-08
3703 1.0028432192638e-08
3704 1.00277223845369e-08
3705 1.00270736822922e-08
3706 1.00262638184773e-08
3707 1.00255844697472e-08
3708 1.00247354878902e-08
3709 1.00240315641792e-08
3710 1.00233562594365e-08
3711 1.0022571637458e-08
3712 1.00218780111488e-08
3713 1.00211237502174e-08
3714 1.00203609568056e-08
3715 1.00196792965912e-08
3716 1.00189869374628e-08
3717 1.00182187723757e-08
3718 1.00174970926181e-08
3719 1.00166985291666e-08
3720 1.00159770486802e-08
3721 1.00152848045015e-08
3722 1.00146056627587e-08
3723 1.00138085503132e-08
3724 1.00131012376048e-08
3725 1.00123593116674e-08
3726 1.00116700601921e-08
3727 1.00108508596375e-08
3728 1.00100962312499e-08
3729 1.00094573857873e-08
3730 1.00086782747755e-08
3731 1.00080197007607e-08
3732 1.00072318195837e-08
3733 1.00065123190968e-08
3734 1.00057774237994e-08
3735 1.00050541251284e-08
3736 1.00043217508422e-08
3737 1.00035563089934e-08
3738 1.00028874985836e-08
3739 1.00022363216934e-08
3740 1.00013876928318e-08
3741 1.00006831574712e-08
3742 9.99996095875366e-09
3743 9.99923781157264e-09
3744 9.9985071452674e-09
3745 9.99777551113912e-09
3746 9.99702015946902e-09
3747 9.99636890254563e-09
3748 9.995620811179e-09
3749 9.99489697796341e-09
3750 9.9941114040103e-09
3751 9.99344280797065e-09
3752 9.99271389338918e-09
3753 9.9920287904981e-09
3754 9.99128643246194e-09
3755 9.99056876094251e-09
3756 9.98981832731061e-09
3757 9.98910533588082e-09
3758 9.98840868021711e-09
3759 9.9876871895721e-09
3760 9.98692564103954e-09
3761 9.98620997055832e-09
3762 9.98550114456043e-09
3763 9.98475808255961e-09
3764 9.98405776758693e-09
3765 9.98331638979444e-09
3766 9.98263790361054e-09
3767 9.98189191153687e-09
3768 9.98116085861034e-09
3769 9.98038476757168e-09
3770 9.97974508619437e-09
3771 9.97902617845548e-09
3772 9.97832677289423e-09
3773 9.97754513663851e-09
3774 9.97687784123902e-09
3775 9.97613990318114e-09
3776 9.97541261787971e-09
3777 9.97471799164817e-09
3778 9.97399852643854e-09
3779 9.97325935558901e-09
3780 9.97254393128588e-09
3781 9.97178520507902e-09
3782 9.97110527396761e-09
3783 9.97032487309868e-09
3784 9.9696368877078e-09
3785 9.96888490040204e-09
3786 9.96823115168022e-09
3787 9.9674950098938e-09
3788 9.96675692736815e-09
3789 9.96604165050263e-09
3790 9.96533427162494e-09
3791 9.9646169866574e-09
3792 9.96389040518186e-09
3793 9.96316802495678e-09
3794 9.96249250334591e-09
3795 9.96179358114802e-09
3796 9.96106613250503e-09
3797 9.96027335424532e-09
3798 9.95959964766868e-09
3799 9.95882884025079e-09
3800 9.95815887400164e-09
3801 9.95746342158377e-09
3802 9.95672481647614e-09
3803 9.95607152920464e-09
3804 9.95528940572754e-09
3805 9.95455703325931e-09
3806 9.95390153289699e-09
3807 9.95317157667652e-09
3808 9.95245416046675e-09
3809 9.95174327148019e-09
3810 9.95102868839304e-09
3811 9.95032047909628e-09
3812 9.94958749976627e-09
3813 9.94896178668769e-09
3814 9.94817278708593e-09
3815 9.94750442835646e-09
3816 9.94672608171932e-09
3817 9.94602837793568e-09
3818 9.94533623618898e-09
3819 9.94465390824484e-09
3820 9.94389392923456e-09
3821 9.94319199006111e-09
3822 9.94248156539756e-09
3823 9.94179754094937e-09
3824 9.94109688197409e-09
3825 9.94038372512107e-09
3826 9.93965303540401e-09
3827 9.93897582972358e-09
3828 9.93823334360933e-09
3829 9.93755603449775e-09
3830 9.9368451330073e-09
3831 9.93616079952853e-09
3832 9.93541036876933e-09
3833 9.93469008503389e-09
3834 9.93400987427118e-09
3835 9.93328768265911e-09
3836 9.93261991132088e-09
3837 9.93184832796812e-09
3838 9.9311599655455e-09
3839 9.93049684855363e-09
3840 9.92973012885201e-09
3841 9.92908613479948e-09
3842 9.92832441988611e-09
3843 9.92768782835662e-09
3844 9.9269140707986e-09
3845 9.92625193993457e-09
3846 9.92553202920632e-09
3847 9.92483857464865e-09
3848 9.92408599345684e-09
3849 9.92345244223147e-09
3850 9.92272484466594e-09
3851 9.92199209516598e-09
3852 9.92126274101945e-09
3853 9.92059015010593e-09
3854 9.9199085001056e-09
3855 9.91921146800689e-09
3856 9.91851683644629e-09
3857 9.9177976356779e-09
3858 9.91711987959054e-09
3859 9.91636987861255e-09
3860 9.91570167166445e-09
3861 9.91499319699662e-09
3862 9.91428869882849e-09
3863 9.91356330955207e-09
3864 9.91288239403365e-09
3865 9.91223137740416e-09
3866 9.91152582914545e-09
3867 9.9107716023672e-09
3868 9.91013225684623e-09
3869 9.90938914137429e-09
3870 9.90867944450358e-09
3871 9.90800112744827e-09
3872 9.90726516489349e-09
3873 9.9066457574376e-09
3874 9.90591599302204e-09
3875 9.90516883518644e-09
3876 9.90447614830636e-09
3877 9.90382082917407e-09
3878 9.90317154885012e-09
3879 9.90242933285312e-09
3880 9.90173839381103e-09
3881 9.90102335370058e-09
3882 9.90029701283812e-09
3883 9.89966234121731e-09
3884 9.89893233685479e-09
3885 9.89821862862339e-09
3886 9.89756835664823e-09
3887 9.89682596783115e-09
3888 9.89619403975184e-09
3889 9.89542264960563e-09
3890 9.89473470876245e-09
3891 9.89402594919753e-09
3892 9.89333271224357e-09
3893 9.89264726364292e-09
3894 9.89188329489354e-09
3895 9.89123737638042e-09
3896 9.89055444217124e-09
3897 9.88986300519412e-09
3898 9.88917126297506e-09
3899 9.88841451216027e-09
3900 9.88766736197133e-09
3901 9.88704018736741e-09
3902 9.88634136590838e-09
3903 9.88561655534958e-09
3904 9.88497537378341e-09
3905 9.88420339029239e-09
3906 9.88358738271411e-09
3907 9.88281630592835e-09
3908 9.88216619854376e-09
3909 9.88150101381702e-09
3910 9.88072624005798e-09
3911 9.88004010413213e-09
3912 9.87937121513238e-09
3913 9.87868801159697e-09
3914 9.87801958925172e-09
3915 9.87729781207203e-09
3916 9.87659092616111e-09
3917 9.87592183013253e-09
3918 9.87522002940389e-09
3919 9.87455835459172e-09
3920 9.87378407847628e-09
3921 9.87312263944773e-09
3922 9.87237192440205e-09
3923 9.87171095184758e-09
3924 9.87103749482132e-09
3925 9.87032569234325e-09
3926 9.86963838534027e-09
3927 9.8689391373058e-09
3928 9.86826349295977e-09
3929 9.86752905957067e-09
3930 9.86684088569168e-09
3931 9.86620058755772e-09
3932 9.86545188290389e-09
3933 9.86476114879509e-09
3934 9.86412028189387e-09
3935 9.86341093504872e-09
3936 9.86270935357525e-09
3937 9.8620239152164e-09
3938 9.86134775819714e-09
3939 9.86064367330952e-09
3940 9.85994671236223e-09
3941 9.85926691796091e-09
3942 9.85856426603815e-09
3943 9.85785846364939e-09
3944 9.85716855592123e-09
3945 9.85647460202688e-09
3946 9.85577121390135e-09
3947 9.85512126697175e-09
3948 9.85439032409607e-09
3949 9.85371883480135e-09
3950 9.85300850406268e-09
3951 9.85227618340023e-09
3952 9.85163505817788e-09
3953 9.85094816954857e-09
3954 9.85021393183627e-09
3955 9.8495649275554e-09
3956 9.84890981399478e-09
3957 9.84818622314088e-09
3958 9.84748611822239e-09
3959 9.84679283688727e-09
3960 9.8461489525109e-09
3961 9.84549295825199e-09
3962 9.84474062852569e-09
3963 9.84409597307168e-09
3964 9.84338799767115e-09
3965 9.84267599886401e-09
3966 9.8419626870655e-09
3967 9.84136116422008e-09
3968 9.84061709850825e-09
3969 9.83997054136709e-09
3970 9.83924063366337e-09
3971 9.83856670064287e-09
3972 9.83792394385063e-09
3973 9.83721412768646e-09
3974 9.83649594650526e-09
3975 9.83587827355081e-09
3976 9.83509766804003e-09
3977 9.83446520803899e-09
3978 9.83379890802383e-09
3979 9.83306654467331e-09
3980 9.83247601174519e-09
3981 9.83172393057008e-09
3982 9.83100560740524e-09
3983 9.83034042961739e-09
3984 9.82970927443372e-09
3985 9.82893369445847e-09
3986 9.8283142605099e-09
3987 9.82762440188134e-09
3988 9.82692182785461e-09
3989 9.82630501261683e-09
3990 9.82554625766907e-09
3991 9.82489551021315e-09
3992 9.82420822644159e-09
3993 9.82353010081649e-09
3994 9.82284852492354e-09
3995 9.822165013329e-09
3996 9.82143920412459e-09
3997 9.82079958355975e-09
3998 9.82010471778372e-09
3999 9.81944620201403e-09
4000 9.81869742586183e-09
4001 9.81807418416747e-09
4002 9.81733024817133e-09
4003 9.81673998289023e-09
4004 9.81598090321612e-09
4005 9.81535547103785e-09
4006 9.8146434104468e-09
4007 9.81396619170738e-09
4008 9.8132948996299e-09
4009 9.81259811991264e-09
4010 9.81188744443584e-09
4011 9.81126747429117e-09
4012 9.81054152135452e-09
4013 9.80989332066695e-09
4014 9.80921525205181e-09
4015 9.80850407295009e-09
4016 9.80782837248229e-09
4017 9.80719447310485e-09
4018 9.80649664704403e-09
4019 9.80584514083394e-09
4020 9.80517825822924e-09
4021 9.80449703599784e-09
4022 9.80384767168552e-09
4023 9.80311439192916e-09
4024 9.80242130173281e-09
4025 9.80178718384961e-09
4026 9.8010827985634e-09
4027 9.80037839173886e-09
4028 9.79971418303693e-09
4029 9.79906842868417e-09
4030 9.79843089414512e-09
4031 9.79767275890997e-09
4032 9.79703053136105e-09
4033 9.79635713226068e-09
4034 9.79567974122852e-09
4035 9.7950252830048e-09
4036 9.79435079841162e-09
4037 9.79367574539813e-09
4038 9.79300029056718e-09
4039 9.79237239215336e-09
4040 9.79165520292868e-09
4041 9.7910144771507e-09
4042 9.79029157976985e-09
4043 9.78964583861486e-09
4044 9.78894708207612e-09
4045 9.78825924885518e-09
4046 9.78765537015813e-09
4047 9.78692849923357e-09
4048 9.78628478015553e-09
4049 9.78553201885779e-09
4050 9.78491302093543e-09
4051 9.78423982615773e-09
4052 9.78351554632717e-09
4053 9.78300905714324e-09
4054 9.78217827059258e-09
4055 9.78154458160241e-09
4056 9.78089743915167e-09
4057 9.7802120021806e-09
4058 9.77953385668251e-09
4059 9.77887075497008e-09
4060 9.77824522198356e-09
4061 9.77748917807553e-09
4062 9.77687845825359e-09
4063 9.77616196974618e-09
4064 9.77550076571021e-09
4065 9.77480060256053e-09
4066 9.77412799674227e-09
4067 9.77350873682115e-09
4068 9.77279683250787e-09
4069 9.77216284538118e-09
4070 9.77147742052542e-09
4071 9.77080217873239e-09
4072 9.77013174886798e-09
4073 9.7694253101549e-09
4074 9.76878027664219e-09
4075 9.76815045133916e-09
4076 9.76742620799331e-09
4077 9.76678081084093e-09
4078 9.76610045100301e-09
4079 9.76545358612191e-09
4080 9.76480354036857e-09
4081 9.76403227580247e-09
4082 9.76343547698377e-09
4083 9.76276569525369e-09
4084 9.76208357218733e-09
4085 9.76142870338925e-09
4086 9.76069509860122e-09
4087 9.76006661095041e-09
4088 9.75943518372047e-09
4089 9.75875597408749e-09
4090 9.7580590508739e-09
4091 9.75735434567693e-09
4092 9.75677181554169e-09
4093 9.75610037941277e-09
4094 9.75539041500606e-09
4095 9.75468279515446e-09
4096 9.75403975599087e-09
4097 9.75340895190135e-09
4098 9.75274667043557e-09
4099 9.7520480772384e-09
4100 9.75136624653539e-09
4101 9.75072764747287e-09
4102 9.75011514883772e-09
4103 9.74942233233911e-09
4104 9.74868605457813e-09
4105 9.7480141244416e-09
4106 9.74745037651825e-09
4107 9.74667578509947e-09
4108 9.7460677577349e-09
4109 9.7453573614098e-09
4110 9.74474609445608e-09
4111 9.74408238459124e-09
4112 9.74336600823022e-09
4113 9.74268599288064e-09
4114 9.74203484356562e-09
4115 9.74138658019208e-09
4116 9.74076028302517e-09
4117 9.74006671979055e-09
4118 9.73943659933474e-09
4119 9.73875770653165e-09
4120 9.73810087957772e-09
4121 9.73741971341258e-09
4122 9.73680844462699e-09
4123 9.73601113732348e-09
4124 9.73542959536999e-09
4125 9.7347817100274e-09
4126 9.7340990564132e-09
4127 9.73339475217327e-09
4128 9.73280072048832e-09
4129 9.73213136545858e-09
4130 9.73144986587959e-09
4131 9.73081333731363e-09
4132 9.73011163674098e-09
4133 9.72944492523553e-09
4134 9.72882957639421e-09
4135 9.72816767060014e-09
4136 9.72744845600948e-09
4137 9.72679510777286e-09
4138 9.72618633983402e-09
4139 9.72551748866512e-09
4140 9.72486501017722e-09
4141 9.72420551838271e-09
4142 9.72348023181579e-09
4143 9.7228532824345e-09
4144 9.72222090865615e-09
4145 9.72154625573929e-09
4146 9.72087186824899e-09
4147 9.72023146550427e-09
4148 9.71957989341632e-09
4149 9.71886804038147e-09
4150 9.71827867159158e-09
4151 9.7176099218832e-09
4152 9.71691994729185e-09
4153 9.7162544656082e-09
4154 9.71560761317547e-09
4155 9.71491634901844e-09
4156 9.71429493125398e-09
4157 9.71360592180726e-09
4158 9.71293534662854e-09
4159 9.71228082775888e-09
4160 9.71159893038698e-09
4161 9.71097541087318e-09
4162 9.71033508370689e-09
4163 9.70968926077009e-09
4164 9.70901635780064e-09
4165 9.70839186451022e-09
4166 9.70767037179743e-09
4167 9.70704295799596e-09
4168 9.70644985703872e-09
4169 9.7057204504758e-09
4170 9.70510965708771e-09
4171 9.7043730333396e-09
4172 9.70368751748718e-09
4173 9.70310272621622e-09
4174 9.70240968448111e-09
4175 9.70173397382701e-09
4176 9.70111236504867e-09
4177 9.70041948536449e-09
4178 9.69977478214312e-09
4179 9.69911733442186e-09
4180 9.69850549305262e-09
4181 9.69790882891786e-09
4182 9.69716455988256e-09
4183 9.69652385553188e-09
4184 9.6958230620392e-09
4185 9.69525185640085e-09
4186 9.69452950962124e-09
4187 9.69384737252443e-09
4188 9.69325586067105e-09
4189 9.69261083565154e-09
4190 9.6919035762616e-09
4191 9.69122905299102e-09
4192 9.69064280142984e-09
4193 9.68997905403945e-09
4194 9.68932117260957e-09
4195 9.68864004760595e-09
4196 9.6879992596971e-09
4197 9.68732729279831e-09
4198 9.68667488189523e-09
4199 9.68607362340779e-09
4200 9.68541233424547e-09
4201 9.68473576710982e-09
4202 9.68403150643227e-09
4203 9.68341650046944e-09
4204 9.68275148366393e-09
4205 9.68204123878713e-09
4206 9.68148647233213e-09
4207 9.68083933598762e-09
4208 9.68017149492739e-09
4209 9.67950421013053e-09
4210 9.67883477688558e-09
4211 9.67823405581547e-09
4212 9.67754399412712e-09
4213 9.67686620287345e-09
4214 9.67629502968137e-09
4215 9.67563709743102e-09
4216 9.67497164157394e-09
4217 9.67432323621675e-09
4218 9.6736391187735e-09
4219 9.67302623013089e-09
4220 9.67237400346932e-09
4221 9.67173361932083e-09
4222 9.67105846989835e-09
4223 9.67043995464545e-09
4224 9.6697681211122e-09
4225 9.66916204757362e-09
4226 9.66851874008301e-09
4227 9.66784150505107e-09
4228 9.66720457956649e-09
4229 9.66648927279434e-09
4230 9.66591995138488e-09
4231 9.66520714769381e-09
4232 9.66459935994313e-09
4233 9.66394709436824e-09
4234 9.66333673138586e-09
4235 9.66265753937767e-09
4236 9.66198899966797e-09
4237 9.66138486764578e-09
4238 9.66066750598959e-09
4239 9.66003538006854e-09
4240 9.65944981669298e-09
4241 9.65878191734604e-09
4242 9.65812202219363e-09
4243 9.65751617534871e-09
4244 9.65686191174708e-09
4245 9.65618692587433e-09
4246 9.65548525037885e-09
4247 9.65491996952894e-09
4248 9.65429222574143e-09
4249 9.65361541933885e-09
4250 9.65299778465933e-09
4251 9.65236850801465e-09
4252 9.65167789195032e-09
4253 9.65105231852392e-09
4254 9.6504165456035e-09
4255 9.64976428077802e-09
4256 9.64910257901519e-09
4257 9.64848508126781e-09
4258 9.64783452150897e-09
4259 9.64715129972427e-09
4260 9.6464763905818e-09
4261 9.64592372447448e-09
4262 9.64522964748415e-09
4263 9.6445987435162e-09
4264 9.64392037253181e-09
4265 9.6433225388326e-09
4266 9.64270655313959e-09
4267 9.64202330885899e-09
4268 9.64142502014875e-09
4269 9.64072378305258e-09
4270 9.64008936046867e-09
4271 9.63952308179195e-09
4272 9.63880385788929e-09
4273 9.63819939067689e-09
4274 9.63751353419417e-09
4275 9.63688509046656e-09
4276 9.63632670203385e-09
4277 9.63562339353907e-09
4278 9.63501500796105e-09
4279 9.63433196762842e-09
4280 9.63364083771123e-09
4281 9.63306122100938e-09
4282 9.6323516411978e-09
4283 9.63170027920568e-09
4284 9.63108445714567e-09
4285 9.63048005660216e-09
4286 9.62975954472978e-09
4287 9.62924575759194e-09
4288 9.62855033906362e-09
4289 9.62787049006708e-09
4290 9.62726057517072e-09
4291 9.62660039598162e-09
4292 9.62598332358844e-09
4293 9.62534404966298e-09
4294 9.62467813660217e-09
4295 9.62402825274711e-09
4296 9.62338532609075e-09
4297 9.6227979107244e-09
4298 9.62211063820773e-09
4299 9.62149264156775e-09
4300 9.62081120410574e-09
4301 9.62019067904385e-09
4302 9.61954497136208e-09
4303 9.61897416307256e-09
4304 9.61828272552645e-09
4305 9.6176526345193e-09
4306 9.6170120036515e-09
4307 9.61637582189145e-09
4308 9.61574254022823e-09
4309 9.61505614067992e-09
4310 9.61448633041151e-09
4311 9.61378501779242e-09
4312 9.6132105423391e-09
4313 9.61256650418296e-09
4314 9.61190012491175e-09
4315 9.61129951777828e-09
4316 9.61061793113005e-09
4317 9.60998750208775e-09
4318 9.60936448232697e-09
4319 9.60870371509437e-09
4320 9.60807485586579e-09
4321 9.6074115151501e-09
4322 9.60675185647519e-09
4323 9.60616481246457e-09
4324 9.60551438775048e-09
4325 9.60487434099877e-09
4326 9.60425835079548e-09
4327 9.60360078615385e-09
4328 9.60297332491811e-09
4329 9.60234899755053e-09
4330 9.60166029106979e-09
4331 9.60107112994713e-09
4332 9.60037239278178e-09
4333 9.59979901279939e-09
4334 9.59909438630335e-09
4335 9.59847233858446e-09
4336 9.59780658402187e-09
4337 9.59720375263984e-09
4338 9.59661603901207e-09
4339 9.59592035279511e-09
4340 9.59536759055635e-09
4341 9.59468972615285e-09
4342 9.59398347390172e-09
4343 9.59341982822992e-09
4344 9.59271492614877e-09
4345 9.59213999436603e-09
4346 9.59151711389661e-09
4347 9.5908615848489e-09
4348 9.5902488597005e-09
4349 9.58957666864768e-09
4350 9.58889905569582e-09
4351 9.58829013006368e-09
4352 9.58765313255339e-09
4353 9.58698041821082e-09
4354 9.58635893573423e-09
4355 9.58579771943857e-09
4356 9.58510835089022e-09
4357 9.5845024934843e-09
4358 9.58384514634925e-09
4359 9.58316780426405e-09
4360 9.58256767881471e-09
4361 9.58190106012391e-09
4362 9.58129820366471e-09
4363 9.58065320560975e-09
4364 9.58001188025581e-09
4365 9.57936101086965e-09
4366 9.5787425949817e-09
4367 9.5781268993067e-09
4368 9.57743007660994e-09
4369 9.57684797552039e-09
4370 9.57623150231457e-09
4371 9.57554488324741e-09
4372 9.57498714118654e-09
4373 9.57433429993326e-09
4374 9.57367483782334e-09
4375 9.57307502463811e-09
4376 9.5724172624323e-09
4377 9.57179894228721e-09
4378 9.57119577151e-09
4379 9.57058925683818e-09
4380 9.56989524662777e-09
4381 9.5692368073802e-09
4382 9.56865328752277e-09
4383 9.56803532826955e-09
4384 9.5673298200899e-09
4385 9.56679413049999e-09
4386 9.5661073990505e-09
4387 9.56553690556472e-09
4388 9.56484454595063e-09
4389 9.56421681932995e-09
4390 9.5635906926933e-09
4391 9.56296095867837e-09
4392 9.56232567868309e-09
4393 9.56169363496018e-09
4394 9.56101241172957e-09
4395 9.56039420223209e-09
4396 9.55981984659959e-09
4397 9.55911062264225e-09
4398 9.55854924630795e-09
4399 9.55786757708688e-09
4400 9.55726327815654e-09
4401 9.55667230884139e-09
4402 9.55601745689094e-09
4403 9.55541815138294e-09
4404 9.55470256432112e-09
4405 9.55409524866502e-09
4406 9.55350747774975e-09
4407 9.55287070439348e-09
4408 9.55215996606418e-09
4409 9.55154256580826e-09
4410 9.55092036467042e-09
4411 9.55027291012212e-09
4412 9.54966482734354e-09
4413 9.54903959268783e-09
4414 9.5483757721615e-09
4415 9.54779535122796e-09
4416 9.54713070693003e-09
4417 9.54655001798865e-09
4418 9.54591561551366e-09
4419 9.54527952033712e-09
4420 9.54466147673472e-09
4421 9.54403863386022e-09
4422 9.54342861071711e-09
4423 9.54272977136961e-09
4424 9.54212183121306e-09
4425 9.5415196614268e-09
4426 9.54087729661268e-09
4427 9.54020256083155e-09
4428 9.53967256446531e-09
4429 9.53896066642479e-09
4430 9.53834268495324e-09
4431 9.53771561787442e-09
4432 9.53713094811737e-09
4433 9.53652292089158e-09
4434 9.53588821762941e-09
4435 9.53524632579805e-09
4436 9.53462911638947e-09
4437 9.53399532206689e-09
4438 9.53332137712537e-09
4439 9.53271664118349e-09
4440 9.53217173708942e-09
4441 9.53144885980362e-09
4442 9.53083749320738e-09
4443 9.53024783398315e-09
4444 9.52956885985623e-09
4445 9.52891479823192e-09
4446 9.5283694529491e-09
4447 9.52768311988927e-09
4448 9.5270722969415e-09
4449 9.52646012504499e-09
4450 9.52582817019543e-09
4451 9.5252311667765e-09
4452 9.52457489369896e-09
4453 9.52396021300372e-09
4454 9.52337437105932e-09
4455 9.52274703699962e-09
4456 9.52200844099582e-09
4457 9.52147016827776e-09
4458 9.52086890068648e-09
4459 9.52021179941032e-09
4460 9.51959622283449e-09
4461 9.5189667991824e-09
4462 9.51833375294198e-09
4463 9.51774421982521e-09
4464 9.51709055974081e-09
4465 9.5165025311289e-09
4466 9.51584223012059e-09
4467 9.51527313809708e-09
4468 9.51461090124839e-09
4469 9.51401676346775e-09
4470 9.51337582756617e-09
4471 9.51270993983233e-09
4472 9.51215443255327e-09
4473 9.51146900786404e-09
4474 9.51088261155753e-09
4475 9.51027375739599e-09
4476 9.50964621862505e-09
4477 9.50906273564089e-09
4478 9.50835483859436e-09
4479 9.50776862053415e-09
4480 9.50721315726155e-09
4481 9.50652609521541e-09
4482 9.50587708842265e-09
4483 9.5052958192926e-09
4484 9.50469338546755e-09
4485 9.5040748219477e-09
4486 9.50347003779439e-09
4487 9.50281281414389e-09
4488 9.50220525715306e-09
4489 9.50154214243715e-09
4490 9.50094741426766e-09
4491 9.50031469039436e-09
4492 9.49973408806426e-09
4493 9.49911899630895e-09
4494 9.4984321439423e-09
4495 9.49788229766269e-09
4496 9.49726620363967e-09
4497 9.49660599461333e-09
4498 9.49603290265055e-09
4499 9.49537897598773e-09
4500 9.49479400146058e-09
4501 9.49413536892318e-09
4502 9.49353575749323e-09
4503 9.49291483250125e-09
4504 9.49225561727718e-09
4505 9.49159429701474e-09
4506 9.49102067097918e-09
4507 9.49038664757595e-09
4508 9.48979814677231e-09
4509 9.48916334511662e-09
4510 9.48853921015069e-09
4511 9.48793773530854e-09
4512 9.48728133472188e-09
4513 9.48666317288072e-09
4514 9.48606822927245e-09
4515 9.48544430212639e-09
4516 9.48486706635332e-09
4517 9.48423777961949e-09
4518 9.48357472148331e-09
4519 9.4830199744711e-09
4520 9.4824069523658e-09
4521 9.48173309352207e-09
4522 9.4811881472534e-09
4523 9.48052016015721e-09
4524 9.47999709707825e-09
4525 9.47927831926321e-09
4526 9.47871656073462e-09
4527 9.47805035225735e-09
4528 9.47746548858597e-09
4529 9.4768007565943e-09
4530 9.47626121705625e-09
4531 9.47556816606465e-09
4532 9.47499248989125e-09
4533 9.47433630854588e-09
4534 9.4737273819423e-09
4535 9.47316580834912e-09
4536 9.4725321934247e-09
4537 9.47190968007827e-09
4538 9.47134715090225e-09
4539 9.47069273955769e-09
4540 9.47009212176608e-09
4541 9.46953049604793e-09
4542 9.46888489454512e-09
4543 9.46825312975186e-09
4544 9.46765624784685e-09
4545 9.46703592161202e-09
4546 9.46639490911894e-09
4547 9.46573051320565e-09
4548 9.46515092703493e-09
4549 9.46460338395461e-09
4550 9.46397128995247e-09
4551 9.46334058631038e-09
4552 9.46270564165796e-09
4553 9.46211865641977e-09
4554 9.46155029331319e-09
4555 9.46093084104593e-09
4556 9.46029476532606e-09
4557 9.45967371533685e-09
4558 9.45904634795658e-09
4559 9.45837484760126e-09
4560 9.45777553054694e-09
4561 9.45723690969069e-09
4562 9.45658136788929e-09
4563 9.45599128700236e-09
4564 9.45534245314073e-09
4565 9.4547278271101e-09
4566 9.45411570299481e-09
4567 9.45353385545244e-09
4568 9.45290646052477e-09
4569 9.45231040927474e-09
4570 9.45168049046319e-09
4571 9.45108752645196e-09
4572 9.45044921790705e-09
4573 9.44988133512459e-09
4574 9.44923644513596e-09
4575 9.44862917082179e-09
4576 9.44794620244549e-09
4577 9.44745550601756e-09
4578 9.446784391548e-09
4579 9.44610876771335e-09
4580 9.4455552629158e-09
4581 9.44491506274514e-09
4582 9.44433517592602e-09
4583 9.44367872736385e-09
4584 9.44309543174371e-09
4585 9.44249263203079e-09
4586 9.44186389940926e-09
4587 9.4412455745041e-09
4588 9.44064321142801e-09
4589 9.44009609626928e-09
4590 9.4394654541613e-09
4591 9.43885663834409e-09
4592 9.43829428899645e-09
4593 9.43762575247864e-09
4594 9.43702934132207e-09
4595 9.43642467617079e-09
4596 9.4358285119972e-09
4597 9.43520051154001e-09
4598 9.43458584011514e-09
4599 9.43399608967221e-09
4600 9.43337370902519e-09
4601 9.43273465539574e-09
4602 9.43220943552814e-09
4603 9.43152975639572e-09
4604 9.43089725088941e-09
4605 9.43038009541819e-09
4606 9.42971150924565e-09
4607 9.4291334722385e-09
4608 9.42857104306583e-09
4609 9.42788546003104e-09
4610 9.42728177556751e-09
4611 9.42675317978925e-09
4612 9.42607946015361e-09
4613 9.42554222313485e-09
4614 9.42487318901508e-09
4615 9.42427938016577e-09
4616 9.42368570555629e-09
4617 9.42309789307705e-09
4618 9.42247208235436e-09
4619 9.42181190469182e-09
4620 9.42130448403078e-09
4621 9.42059897292291e-09
4622 9.42001624029687e-09
4623 9.41938290702216e-09
4624 9.41882401388205e-09
4625 9.41817335418926e-09
4626 9.41757702600798e-09
4627 9.41695721726199e-09
4628 9.41635909376681e-09
4629 9.41576412329115e-09
4630 9.41515048724639e-09
4631 9.41452105186757e-09
4632 9.41393674713797e-09
4633 9.41334779658298e-09
4634 9.41270262289073e-09
4635 9.41207714014181e-09
4636 9.41147434788126e-09
4637 9.41091593184562e-09
4638 9.41031973808459e-09
4639 9.40964550893986e-09
4640 9.40909175187188e-09
4641 9.40847803523881e-09
4642 9.40787932807163e-09
4643 9.40726225130695e-09
4644 9.40660825839157e-09
4645 9.40602883542363e-09
4646 9.40543520941417e-09
4647 9.40480568592528e-09
4648 9.40427612303174e-09
4649 9.40360502550697e-09
4650 9.40298810113427e-09
4651 9.40241327429536e-09
4652 9.40175131576571e-09
4653 9.40113208811044e-09
4654 9.4006113706413e-09
4655 9.39995961801721e-09
4656 9.39938909270965e-09
4657 9.39879987475745e-09
4658 9.39816904829693e-09
4659 9.39751085214657e-09
4660 9.39693132512298e-09
4661 9.39634649470278e-09
4662 9.3957686434637e-09
4663 9.39513609810039e-09
4664 9.39455912726817e-09
4665 9.39392956951501e-09
4666 9.39332027867501e-09
4667 9.39276235656372e-09
4668 9.39219544780767e-09
4669 9.39151248238734e-09
4670 9.39097242171061e-09
4671 9.39036207665833e-09
4672 9.38975828260191e-09
4673 9.38914727165174e-09
4674 9.38860434075706e-09
4675 9.38795280840121e-09
4676 9.38733235164579e-09
4677 9.38668475303217e-09
4678 9.38613249229503e-09
4679 9.3854747630906e-09
4680 9.38497599584087e-09
4681 9.38426361110634e-09
4682 9.38373187479857e-09
4683 9.38311783817147e-09
4684 9.3825072074033e-09
4685 9.38192752480971e-09
4686 9.38129850941438e-09
4687 9.38072010188418e-09
4688 9.3801192272841e-09
4689 9.37945672435658e-09
4690 9.37889759231036e-09
4691 9.37825241245638e-09
4692 9.37772307191276e-09
4693 9.37700842711109e-09
4694 9.37645044432611e-09
4695 9.37589752487977e-09
4696 9.37527932463877e-09
4697 9.37467307862705e-09
4698 9.37404963287369e-09
4699 9.37347286611434e-09
4700 9.37284504565206e-09
4701 9.37218435806408e-09
4702 9.37170764196693e-09
4703 9.37107702446427e-09
4704 9.37046793217378e-09
4705 9.3698549411686e-09
4706 9.36923172335791e-09
4707 9.3686884987676e-09
4708 9.36808370856357e-09
4709 9.36746157401136e-09
4710 9.36684134002219e-09
4711 9.36627980745175e-09
4712 9.3656558421279e-09
4713 9.36504013644701e-09
4714 9.36448157654035e-09
4715 9.36385144373331e-09
4716 9.3632608294536e-09
4717 9.36263641766744e-09
4718 9.36205664377188e-09
4719 9.36142992653821e-09
4720 9.36086196021146e-09
4721 9.36025531442231e-09
4722 9.35966184165138e-09
4723 9.3590432040519e-09
4724 9.35847917216126e-09
4725 9.35785458548721e-09
4726 9.35728512425904e-09
4727 9.3566763606362e-09
4728 9.35603497566329e-09
4729 9.35544859773096e-09
4730 9.35489382346277e-09
4731 9.35425129572343e-09
4732 9.35372400115431e-09
4733 9.3530627719024e-09
4734 9.35246540606505e-09
4735 9.35187135987781e-09
4736 9.35122558581858e-09
4737 9.35066893793801e-09
4738 9.35007766665608e-09
4739 9.34940939593953e-09
4740 9.34883260569896e-09
4741 9.34824079035224e-09
4742 9.34766140500698e-09
4743 9.34708110147953e-09
4744 9.3464968990431e-09
4745 9.34588168916006e-09
4746 9.34530854894422e-09
4747 9.34465226737347e-09
4748 9.34413537444223e-09
4749 9.34349151554548e-09
4750 9.34298972353476e-09
4751 9.34232039853655e-09
4752 9.3416932467616e-09
4753 9.34111926979841e-09
4754 9.34051110179635e-09
4755 9.33994878035693e-09
4756 9.33935739176606e-09
4757 9.33874868142004e-09
4758 9.33813226776381e-09
4759 9.33758335595891e-09
4760 9.33698445902687e-09
4761 9.33638226223443e-09
4762 9.33573853624525e-09
4763 9.33521362937728e-09
4764 9.33455748179657e-09
4765 9.33402864689015e-09
4766 9.33337546725477e-09
4767 9.33283458111334e-09
4768 9.33218646290146e-09
4769 9.33166371781813e-09
4770 9.33101181149754e-09
4771 9.33039543146719e-09
4772 9.329815747805e-09
4773 9.32924746613328e-09
4774 9.32860643436395e-09
4775 9.32806386517992e-09
4776 9.32739024288309e-09
4777 9.32691338613456e-09
4778 9.32627696388633e-09
4779 9.32566210679053e-09
4780 9.32510290731214e-09
4781 9.32449015847436e-09
4782 9.32391704175362e-09
4783 9.3232911058394e-09
4784 9.32274067010075e-09
4785 9.32215290917748e-09
4786 9.32151429476613e-09
4787 9.32097147940403e-09
4788 9.32036906832467e-09
4789 9.319768955518e-09
4790 9.31916500157559e-09
4791 9.3186259918776e-09
4792 9.31797027932391e-09
4793 9.31741440161893e-09
4794 9.31685303642848e-09
4795 9.31619171670728e-09
4796 9.31563972628169e-09
4797 9.31505290495405e-09
4798 9.31442888003897e-09
4799 9.31387519835514e-09
4800 9.31323663731776e-09
4801 9.31266100524797e-09
4802 9.31206974777443e-09
4803 9.31158306709257e-09
4804 9.31086375983992e-09
4805 9.31031605044819e-09
4806 9.30971362737842e-09
4807 9.30916620407729e-09
4808 9.30861146920814e-09
4809 9.30797383072446e-09
4810 9.30731454006073e-09
4811 9.30673130192239e-09
4812 9.30614720331957e-09
4813 9.30561337167402e-09
4814 9.30497675294406e-09
4815 9.30443540378412e-09
4816 9.3037576251731e-09
4817 9.30320054660927e-09
4818 9.30267162199683e-09
4819 9.30210353992933e-09
4820 9.30146551351985e-09
4821 9.30088075776214e-09
4822 9.30030086737643e-09
4823 9.29971476874847e-09
4824 9.29911982168463e-09
4825 9.29854557839282e-09
4826 9.29794773919801e-09
4827 9.29736277115179e-09
4828 9.2967870518601e-09
4829 9.29615956975216e-09
4830 9.29566169266538e-09
4831 9.29503071860072e-09
4832 9.29442917615986e-09
4833 9.29387109935287e-09
4834 9.29322282453016e-09
4835 9.29274131326874e-09
4836 9.29206129039739e-09
4837 9.29149354540748e-09
4838 9.29094629542604e-09
4839 9.29034548363683e-09
4840 9.28976224613687e-09
4841 9.28915415071563e-09
4842 9.28850463231823e-09
4843 9.28796965118939e-09
4844 9.28745665737529e-09
4845 9.2867826869264e-09
4846 9.28623762923297e-09
4847 9.28557977242228e-09
4848 9.28506799179052e-09
4849 9.28447463290072e-09
4850 9.28385040957491e-09
4851 9.28327441444832e-09
4852 9.28272660767615e-09
4853 9.28212977380216e-09
4854 9.28152410789584e-09
4855 9.28092237473255e-09
4856 9.28037513449331e-09
4857 9.27977770760757e-09
4858 9.27916480278346e-09
4859 9.27863324666489e-09
4860 9.27804662279041e-09
4861 9.27739413746076e-09
4862 9.27684681722996e-09
4863 9.27623973698277e-09
4864 9.27565242399286e-09
4865 9.2749821183763e-09
4866 9.27443064492606e-09
4867 9.27383377326285e-09
4868 9.27327814700951e-09
4869 9.27271817044439e-09
4870 9.27217462556862e-09
4871 9.27153174623552e-09
4872 9.27091854471818e-09
4873 9.27033171360669e-09
4874 9.26979152579555e-09
4875 9.26916833257629e-09
4876 9.2686416129778e-09
4877 9.26804947939952e-09
4878 9.26745984500266e-09
4879 9.2668538640428e-09
4880 9.26633310276148e-09
4881 9.26567811747325e-09
4882 9.26508441388696e-09
4883 9.26451167787556e-09
4884 9.26394647526863e-09
4885 9.26335280515556e-09
4886 9.26271234155673e-09
4887 9.262194122589e-09
4888 9.26155892558289e-09
4889 9.26100861581292e-09
4890 9.26041951662682e-09
4891 9.25991365213763e-09
4892 9.25930940763597e-09
4893 9.25867321378837e-09
4894 9.25807620592856e-09
4895 9.25755286107499e-09
4896 9.25694210426875e-09
4897 9.25634154518018e-09
4898 9.25577695104485e-09
4899 9.25522030038872e-09
4900 9.25455704962885e-09
4901 9.25401947643456e-09
4902 9.25343272686896e-09
4903 9.25281437050285e-09
4904 9.25227553123797e-09
4905 9.25166318976878e-09
4906 9.25109295017712e-09
4907 9.25056696378379e-09
4908 9.24992359589705e-09
4909 9.24937539045767e-09
4910 9.24875402238956e-09
4911 9.24817131327249e-09
4912 9.24761025702936e-09
4913 9.24701223306568e-09
4914 9.24640657876119e-09
4915 9.24588381488734e-09
4916 9.24529031537391e-09
4917 9.24468455174021e-09
4918 9.24413463958273e-09
4919 9.24352630020386e-09
4920 9.24293999370052e-09
4921 9.24239185845499e-09
4922 9.24173857090593e-09
4923 9.24118955063225e-09
4924 9.24055031301107e-09
4925 9.24003323253542e-09
4926 9.23944044248226e-09
4927 9.23886914655503e-09
4928 9.2382863708107e-09
4929 9.2376904816116e-09
4930 9.23708757333275e-09
4931 9.23653754789089e-09
4932 9.23591646198596e-09
4933 9.2353527347544e-09
4934 9.23475036111732e-09
4935 9.23417830976658e-09
4936 9.23361883364837e-09
4937 9.23297882815532e-09
4938 9.23241339098602e-09
4939 9.23187148123283e-09
4940 9.23125661081436e-09
4941 9.23072979601425e-09
4942 9.23011255284101e-09
4943 9.22952378326625e-09
4944 9.22895686805702e-09
4945 9.22841358012849e-09
4946 9.22782674669942e-09
4947 9.22718856112559e-09
4948 9.22664453373301e-09
4949 9.22608011053039e-09
4950 9.22548078584329e-09
4951 9.22491802489433e-09
4952 9.22430248381789e-09
4953 9.22376143877579e-09
4954 9.22309467245308e-09
4955 9.2225802691831e-09
4956 9.221988101632e-09
4957 9.22141645483265e-09
4958 9.22082803651858e-09
4959 9.22026180900926e-09
4960 9.21966002166707e-09
4961 9.21907834371127e-09
4962 9.21855155547324e-09
4963 9.21795899422317e-09
4964 9.21732477912607e-09
4965 9.21681716758993e-09
4966 9.21620182782468e-09
4967 9.21559336664046e-09
4968 9.21507574076774e-09
4969 9.21444476099931e-09
4970 9.21384397171987e-09
4971 9.21328618397332e-09
4972 9.21275996010329e-09
4973 9.21217250951512e-09
4974 9.21155657326867e-09
4975 9.21101238648969e-09
4976 9.21044559168416e-09
4977 9.20981076846239e-09
4978 9.20926777137065e-09
4979 9.20866983665503e-09
4980 9.2080924838922e-09
4981 9.20747690137669e-09
4982 9.20693888681323e-09
4983 9.2063614508392e-09
4984 9.20578066017364e-09
4985 9.2052167030704e-09
4986 9.20464001873123e-09
4987 9.20407837519399e-09
4988 9.20350408144255e-09
4989 9.20291481845692e-09
4990 9.20234284709776e-09
4991 9.20176137989004e-09
4992 9.20117546798771e-09
4993 9.20064800913334e-09
4994 9.20005533550094e-09
4995 9.19945105835873e-09
4996 9.19886987719998e-09
4997 9.19834677155784e-09
4998 9.19771837665229e-09
4999 9.19717649146279e-09
};
\addlegendentry{Train}
\addplot [semithick, black]
table {%
0 0.00202328362502158
1 0.000648917688522488
2 0.000244898081291467
3 0.000229991186643019
4 0.000213549894397147
5 0.000161681178724393
6 5.68717550777365e-05
7 2.0527917513391e-05
8 1.87026907951804e-05
9 1.81155810423661e-05
10 1.76042085513473e-05
11 1.71373085322557e-05
12 1.66656900546513e-05
13 1.61264215421397e-05
14 1.54404242493911e-05
15 1.4498127711704e-05
16 1.31280812638579e-05
17 1.1157483641e-05
18 8.60615182318725e-06
19 6.01030751568032e-06
20 4.14225814893143e-06
21 3.17889271173044e-06
22 2.79184541795985e-06
23 2.63862079918908e-06
24 2.56200223702763e-06
25 2.50798552769993e-06
26 2.46633612732694e-06
27 2.4333546662092e-06
28 2.40411327467882e-06
29 2.37874451158859e-06
30 2.35634615819436e-06
31 2.33638729696395e-06
32 2.31858030019794e-06
33 2.30254977395816e-06
34 2.28791304834886e-06
35 2.27437453759194e-06
36 2.26155111704429e-06
37 2.2490623905469e-06
38 2.23663960241538e-06
39 2.2241322312766e-06
40 2.21146797230176e-06
41 2.19857884076191e-06
42 2.18550280806085e-06
43 2.17222100218351e-06
44 2.15877707887557e-06
45 2.14523788599763e-06
46 2.13164685192169e-06
47 2.11798555938003e-06
48 2.10428811442398e-06
49 2.09053450817009e-06
50 2.07666380447336e-06
51 2.06266145141853e-06
52 2.04851062335365e-06
53 2.03432955458993e-06
54 2.02025557882735e-06
55 2.00624299395713e-06
56 1.99220198737748e-06
57 1.97805866264389e-06
58 1.9637902823888e-06
59 1.94943277165294e-06
60 1.93506502910168e-06
61 1.9205428998248e-06
62 1.90574940006627e-06
63 1.89065042377479e-06
64 1.87498744708137e-06
65 1.85868452717841e-06
66 1.84185694251937e-06
67 1.82449480234936e-06
68 1.80659310444753e-06
69 1.78816287643713e-06
70 1.76920468675235e-06
71 1.74978640643531e-06
72 1.72988893609727e-06
73 1.70920736763946e-06
74 1.68786380072561e-06
75 1.66583686223021e-06
76 1.64308528383117e-06
77 1.61970183398807e-06
78 1.59555270329292e-06
79 1.57065767325548e-06
80 1.54497070070647e-06
81 1.51843300955079e-06
82 1.49087827594485e-06
83 1.4621236914536e-06
84 1.43284432851942e-06
85 1.40286635996745e-06
86 1.37221388740727e-06
87 1.34051379063749e-06
88 1.30834223455167e-06
89 1.27608961975056e-06
90 1.24408211377158e-06
91 1.21236985251016e-06
92 1.18079867661436e-06
93 1.14950626084465e-06
94 1.11873077912605e-06
95 1.08818073840666e-06
96 1.05787182747008e-06
97 1.02832314041734e-06
98 9.99787289401866e-07
99 9.72622274275636e-07
100 9.47082355651219e-07
101 9.2327047696017e-07
102 9.01386727036879e-07
103 8.80842264905368e-07
104 8.60657848988922e-07
105 8.41086205127795e-07
106 8.22523418264609e-07
107 8.04213073024584e-07
108 7.87080352893099e-07
109 7.70886003920168e-07
110 7.5514310537983e-07
111 7.40086591122235e-07
112 7.25661607248185e-07
113 7.11934490027488e-07
114 6.98868348081305e-07
115 6.86378598402371e-07
116 6.74421414714743e-07
117 6.6294404632572e-07
118 6.51920004202111e-07
119 6.41342523977073e-07
120 6.31196826361702e-07
121 6.21489959939936e-07
122 6.122216973381e-07
123 6.03427565692982e-07
124 5.95081473875325e-07
125 5.87176259614353e-07
126 5.79710729198268e-07
127 5.72683688915276e-07
128 5.66003507174173e-07
129 5.59664727006748e-07
130 5.53670645331295e-07
131 5.47980789633584e-07
132 5.42476698228711e-07
133 5.37166727099248e-07
134 5.32079297954624e-07
135 5.27200825217733e-07
136 5.22552682014066e-07
137 5.18015440320596e-07
138 5.13709892402403e-07
139 5.09430947204237e-07
140 5.05488003454957e-07
141 5.01398801588948e-07
142 4.97764574447501e-07
143 4.94070150125481e-07
144 4.90461900426453e-07
145 4.86907538288506e-07
146 4.83693440855859e-07
147 4.8044500999822e-07
148 4.77301057344448e-07
149 4.74228443181346e-07
150 4.71252633360564e-07
151 4.68349441007376e-07
152 4.65485356926365e-07
153 4.62713671822712e-07
154 4.60045612271642e-07
155 4.57474214954345e-07
156 4.54987116427219e-07
157 4.52569622666488e-07
158 4.50238246685331e-07
159 4.47958797167303e-07
160 4.45738720600275e-07
161 4.43551897433281e-07
162 4.41426863062588e-07
163 4.39356540482549e-07
164 4.37300883504577e-07
165 4.35290189670923e-07
166 4.33336566629805e-07
167 4.31374957088337e-07
168 4.29477040597703e-07
169 4.27618942921981e-07
170 4.25771673917552e-07
171 4.24009641619705e-07
172 4.22233910057912e-07
173 4.20551856450402e-07
174 4.18866335394341e-07
175 4.17222906889947e-07
176 4.15604660020108e-07
177 4.14005143056784e-07
178 4.12422650697408e-07
179 4.10848912224537e-07
180 4.09261815548234e-07
181 4.07760325060735e-07
182 4.06274835995646e-07
183 4.04777580342852e-07
184 4.03317216068899e-07
185 4.01874160615989e-07
186 4.00452620397118e-07
187 3.99050179566984e-07
188 3.97630174120422e-07
189 3.95974552702683e-07
190 3.94730875541427e-07
191 3.93687145106014e-07
192 3.92371703128447e-07
193 3.91033921687267e-07
194 3.89695571811899e-07
195 3.88347615398743e-07
196 3.87000170576357e-07
197 3.85653152079612e-07
198 3.84298488143031e-07
199 3.82940157805933e-07
200 3.81566678697709e-07
201 3.80182200387935e-07
202 3.78791355615249e-07
203 3.77383457816904e-07
204 3.7597456525873e-07
205 3.74557032500888e-07
206 3.73158684396913e-07
207 3.71775684016029e-07
208 3.70428153928515e-07
209 3.69110438214193e-07
210 3.67821115787592e-07
211 3.6656129509538e-07
212 3.6533529623739e-07
213 3.64131892638397e-07
214 3.62955091759432e-07
215 3.61818308647344e-07
216 3.60710799895969e-07
217 3.59624692691796e-07
218 3.58553052137722e-07
219 3.57494940317338e-07
220 3.56449163518846e-07
221 3.55447241418005e-07
222 3.54455437445722e-07
223 3.53413554421422e-07
224 3.52386507529445e-07
225 3.51390838204679e-07
226 3.50426518025415e-07
227 3.49496929175075e-07
228 3.48596842059123e-07
229 3.47711022641306e-07
230 3.46847116361459e-07
231 3.46004526363686e-07
232 3.45171770277375e-07
233 3.44353225045779e-07
234 3.43538658853504e-07
235 3.42718919910112e-07
236 3.41909640155791e-07
237 3.41093624456335e-07
238 3.40251119723689e-07
239 3.39371098334595e-07
240 3.38472915473176e-07
241 3.37593121457758e-07
242 3.36824143687409e-07
243 3.36262871769577e-07
244 3.35537947648845e-07
245 3.34775052124314e-07
246 3.3401880727979e-07
247 3.33273305841431e-07
248 3.3254531217608e-07
249 3.31848099222043e-07
250 3.31155547428352e-07
251 3.30441736196008e-07
252 3.29708228719028e-07
253 3.28965313656226e-07
254 3.28209637245891e-07
255 3.27453449244786e-07
256 3.2669476013325e-07
257 3.25934990996757e-07
258 3.25193525441136e-07
259 3.24420483366339e-07
260 3.23694877124581e-07
261 3.23046265293669e-07
262 3.22363405302895e-07
263 3.21643199185928e-07
264 3.20893803973377e-07
265 3.20153958455194e-07
266 3.19419541483512e-07
267 3.18712721991687e-07
268 3.18018265943465e-07
269 3.17273276095875e-07
270 3.1651708809477e-07
271 3.1578258585796e-07
272 3.15053512167651e-07
273 3.14335409257183e-07
274 3.13629726633735e-07
275 3.12954909986729e-07
276 3.12367376409384e-07
277 3.11640889094633e-07
278 3.10955044824368e-07
279 3.10264255176662e-07
280 3.09547942833888e-07
281 3.08896233036648e-07
282 3.08467321019634e-07
283 3.07728470261281e-07
284 3.07021622347747e-07
285 3.06651088521903e-07
286 3.05888136153953e-07
287 3.05134648215244e-07
288 3.04728416722355e-07
289 3.03946222857121e-07
290 3.03396120671096e-07
291 3.02805631235969e-07
292 3.02308649224869e-07
293 3.01588784168416e-07
294 3.01107917266563e-07
295 3.00628272498216e-07
296 3.00113669027269e-07
297 2.99665259717585e-07
298 2.98990585179126e-07
299 2.98452590641318e-07
300 2.97898282042297e-07
301 2.97440323038245e-07
302 2.96886810247088e-07
303 2.96444937930573e-07
304 2.95941504191433e-07
305 2.95448586484781e-07
306 2.94854118010335e-07
307 2.9477700991265e-07
308 2.93954229846349e-07
309 2.93692295372239e-07
310 2.9375959798017e-07
311 2.9251563660182e-07
312 2.92062708240337e-07
313 2.91592584744649e-07
314 2.9124572620276e-07
315 2.90636648969667e-07
316 2.90192133434175e-07
317 2.89730962776957e-07
318 2.89319075363892e-07
319 2.88896643496628e-07
320 2.88472648435345e-07
321 2.88072840248788e-07
322 2.87819347022378e-07
323 2.87247672758895e-07
324 2.86914200842148e-07
325 2.86570752905391e-07
326 2.86081672129512e-07
327 2.85669926824994e-07
328 2.85282197864944e-07
329 2.84869116740083e-07
330 2.8455588108045e-07
331 2.84479767742596e-07
332 2.83803984757469e-07
333 2.84169260567069e-07
334 2.83040918702682e-07
335 2.82648954907927e-07
336 2.82910093574174e-07
337 2.81960154779881e-07
338 2.82350185898395e-07
339 2.81285309711166e-07
340 2.8101462135055e-07
341 2.80981282685389e-07
342 2.8033255716764e-07
343 2.80120246998194e-07
344 2.7969332450084e-07
345 2.7951520564784e-07
346 2.79090102139889e-07
347 2.78884726867545e-07
348 2.7849478101416e-07
349 2.78322772828687e-07
350 2.77936152315306e-07
351 2.77813853699627e-07
352 2.77439909268651e-07
353 2.7784861345026e-07
354 2.76925959497021e-07
355 2.76683721267545e-07
356 2.7597695861914e-07
357 2.7567156735131e-07
358 2.75278097205955e-07
359 2.75688819328934e-07
360 2.74770144415015e-07
361 2.74789243803752e-07
362 2.74128609589752e-07
363 2.74029133606746e-07
364 2.73774901415891e-07
365 2.7421751269685e-07
366 2.73180717158539e-07
367 2.72973693427048e-07
368 2.72978468274232e-07
369 2.72472192364148e-07
370 2.72057320671593e-07
371 2.71847284238902e-07
372 2.71573725285634e-07
373 2.71529700057727e-07
374 2.71123923312189e-07
375 2.7093773269371e-07
376 2.7066531060882e-07
377 2.70473009322814e-07
378 2.70327177531726e-07
379 2.70065612539838e-07
380 2.69870582769727e-07
381 2.69645283879072e-07
382 2.69441443379037e-07
383 2.69240359784817e-07
384 2.69049166945479e-07
385 2.68868831199143e-07
386 2.68695515615036e-07
387 2.68522569513152e-07
388 2.68353460342041e-07
389 2.68185260665632e-07
390 2.68011262960499e-07
391 2.67826663957749e-07
392 2.67645646090386e-07
393 2.67454254299082e-07
394 2.67264937292566e-07
395 2.67069992787583e-07
396 2.66866521769771e-07
397 2.66663107595377e-07
398 2.6645241746337e-07
399 2.66245876900939e-07
400 2.66035669937992e-07
401 2.65818073330593e-07
402 2.65605706317729e-07
403 2.6539808573034e-07
404 2.65179409097982e-07
405 2.64960704043915e-07
406 2.64745608546946e-07
407 2.64530882532199e-07
408 2.6431428068463e-07
409 2.64106063241343e-07
410 2.63884857076846e-07
411 2.63681897649803e-07
412 2.63459497773511e-07
413 2.6323249358029e-07
414 2.63025469848799e-07
415 2.62803183659344e-07
416 2.62583512267156e-07
417 2.62363357705908e-07
418 2.62135415596276e-07
419 2.61917449506655e-07
420 2.61696669667799e-07
421 2.61475463503302e-07
422 2.61252324662564e-07
423 2.61033591186788e-07
424 2.60818040942468e-07
425 2.60588336686851e-07
426 2.60373212768172e-07
427 2.60151438169487e-07
428 2.59933756296959e-07
429 2.59710986938444e-07
430 2.59492026088992e-07
431 2.59272695757318e-07
432 2.59061238239156e-07
433 2.58894630178474e-07
434 2.58824854881823e-07
435 2.58781000184172e-07
436 2.58582446122091e-07
437 2.58241101391832e-07
438 2.58008242326468e-07
439 2.57736360254057e-07
440 2.57552699167718e-07
441 2.57281527638042e-07
442 2.57120348123863e-07
443 2.56834738365797e-07
444 2.56714173474393e-07
445 2.56397896691851e-07
446 2.56329172998448e-07
447 2.559917788858e-07
448 2.5593936925361e-07
449 2.55602003562672e-07
450 2.55543994853724e-07
451 2.55217543099207e-07
452 2.5514452772768e-07
453 2.54837345892156e-07
454 2.54748755423861e-07
455 2.54462037219128e-07
456 2.54341898653365e-07
457 2.54090252838068e-07
458 2.53937372463042e-07
459 2.53606771138948e-07
460 2.53572437713956e-07
461 2.53309195841211e-07
462 2.53073068279264e-07
463 2.52979873494041e-07
464 2.52747241802354e-07
465 2.52633441277794e-07
466 2.52354453778025e-07
467 2.52074585205264e-07
468 2.5201839548572e-07
469 2.51762259040333e-07
470 2.51540200224554e-07
471 2.51417674235199e-07
472 2.51242198601176e-07
473 2.50984641070318e-07
474 2.50838041893076e-07
475 2.5071207687688e-07
476 2.5047205554074e-07
477 2.50463273232526e-07
478 2.50482287356135e-07
479 2.50113828315079e-07
480 2.49816082487087e-07
481 2.49656125106412e-07
482 2.49427102971822e-07
483 2.49236364879835e-07
484 2.49080954972669e-07
485 2.48884390430248e-07
486 2.48692458626465e-07
487 2.4855052060957e-07
488 2.48339119934826e-07
489 2.48142896452919e-07
490 2.48027760108016e-07
491 2.47822441679091e-07
492 2.47639405870359e-07
493 2.47492437210894e-07
494 2.47291694677187e-07
495 2.47128582486766e-07
496 2.46916954438348e-07
497 2.46756400201775e-07
498 2.46549916482763e-07
499 2.46380579937977e-07
500 2.46177876306319e-07
501 2.46002230142039e-07
502 2.45801430764914e-07
503 2.45619162342336e-07
504 2.45425951561629e-07
505 2.45237117724173e-07
506 2.45048255465008e-07
507 2.44857972120371e-07
508 2.44667035076418e-07
509 2.44464445131598e-07
510 2.44274019678414e-07
511 2.44075920363684e-07
512 2.43876797867415e-07
513 2.43674634248237e-07
514 2.43478211814363e-07
515 2.43271330191419e-07
516 2.430642211948e-07
517 2.4285216682074e-07
518 2.42626356339315e-07
519 2.42376870573935e-07
520 2.42104590597592e-07
521 2.41793259192491e-07
522 2.41518733901103e-07
523 2.41281497892487e-07
524 2.41068477180306e-07
525 2.40860146050181e-07
526 2.40657925587584e-07
527 2.40456898836783e-07
528 2.40255729977434e-07
529 2.40059677025783e-07
530 2.3985575126062e-07
531 2.39659584622132e-07
532 2.39462679019198e-07
533 2.39258326928393e-07
534 2.39062416085289e-07
535 2.38861645129873e-07
536 2.38664000562494e-07
537 2.38466640212209e-07
538 2.38268114571838e-07
539 2.38068679436765e-07
540 2.37870921182548e-07
541 2.3767249501816e-07
542 2.37471923014709e-07
543 2.37275102676904e-07
544 2.37075155951061e-07
545 2.36876346093595e-07
546 2.36678317833139e-07
547 2.36478626902681e-07
548 2.3627482903521e-07
549 2.36074114923213e-07
550 2.35879170418229e-07
551 2.35681895333073e-07
552 2.35478580634663e-07
553 2.35278022842067e-07
554 2.35078402965883e-07
555 2.34877617799611e-07
556 2.34683170674543e-07
557 2.34483209737846e-07
558 2.3428491147115e-07
559 2.34089938544457e-07
560 2.33897750945289e-07
561 2.33708220775952e-07
562 2.3350423816737e-07
563 2.33310103681106e-07
564 2.33112956493642e-07
565 2.329077375407e-07
566 2.32703342817331e-07
567 2.32499928642937e-07
568 2.3229146961512e-07
569 2.32093924523724e-07
570 2.31879724310602e-07
571 2.31681596574163e-07
572 2.31467055300527e-07
573 2.31270377071269e-07
574 2.31048574050874e-07
575 2.30856613825381e-07
576 2.3063159915182e-07
577 2.30443887971887e-07
578 2.30215391638922e-07
579 2.30027310976766e-07
580 2.29795872996874e-07
581 2.29608744461984e-07
582 2.29373767979268e-07
583 2.2918284514617e-07
584 2.28952742986621e-07
585 2.28756732667534e-07
586 2.28526488399439e-07
587 2.28331984430952e-07
588 2.28101825427984e-07
589 2.27909382033431e-07
590 2.27673368158321e-07
591 2.27474004077521e-07
592 2.27239908667798e-07
593 2.27031506483399e-07
594 2.26809817149842e-07
595 2.26592689500649e-07
596 2.26366708488968e-07
597 2.26145587589599e-07
598 2.25926314101343e-07
599 2.256996936012e-07
600 2.25479325877131e-07
601 2.25251454821773e-07
602 2.2502706542582e-07
603 2.24796195880117e-07
604 2.24567642703732e-07
605 2.24340595877948e-07
606 2.24106045720873e-07
607 2.23877080429702e-07
608 2.23638636498436e-07
609 2.23401542598367e-07
610 2.23164775547957e-07
611 2.22925010007202e-07
612 2.22685031303627e-07
613 2.22442253061672e-07
614 2.22198323740486e-07
615 2.2195361282229e-07
616 2.21703942315798e-07
617 2.21460211946578e-07
618 2.2121187726043e-07
619 2.20965318931121e-07
620 2.20718177956769e-07
621 2.20472330170196e-07
622 2.20220726987463e-07
623 2.19969308545842e-07
624 2.19719083816017e-07
625 2.19470877027561e-07
626 2.19218691199785e-07
627 2.18960465758755e-07
628 2.18703817722599e-07
629 2.18447368638408e-07
630 2.18186087863614e-07
631 2.17919392753174e-07
632 2.17651574985211e-07
633 2.17380062395023e-07
634 2.17105650790472e-07
635 2.16826364862754e-07
636 2.16540811948107e-07
637 2.1625673696235e-07
638 2.1597135457796e-07
639 2.15680444171085e-07
640 2.1538851058267e-07
641 2.15093265865107e-07
642 2.14789835695228e-07
643 2.14487968719368e-07
644 2.14184794344874e-07
645 2.1387562298969e-07
646 2.13564916862197e-07
647 2.13249236935553e-07
648 2.12933386478653e-07
649 2.12616555472778e-07
650 2.12296185964078e-07
651 2.11977763342475e-07
652 2.11654096915481e-07
653 2.11331865784814e-07
654 2.11008114092692e-07
655 2.10682969736808e-07
656 2.10354400564938e-07
657 2.10034912129231e-07
658 2.0970715297608e-07
659 2.09380559113015e-07
660 2.09056381095252e-07
661 2.08730227768683e-07
662 2.08395505296721e-07
663 2.08064321327583e-07
664 2.07731645218701e-07
665 2.07392787388017e-07
666 2.07046497280317e-07
667 2.06704996230656e-07
668 2.06358194532186e-07
669 2.06014817649702e-07
670 2.05670374953115e-07
671 2.05331360803029e-07
672 2.04998130470813e-07
673 2.04667273351333e-07
674 2.04342825327331e-07
675 2.04013716142981e-07
676 2.03692508193853e-07
677 2.03369054929681e-07
678 2.0304976544594e-07
679 2.02725942699544e-07
680 2.02406781113496e-07
681 2.02088742184969e-07
682 2.01767079488491e-07
683 2.01457353909973e-07
684 2.01128159460495e-07
685 2.0083803065063e-07
686 2.00486894641472e-07
687 2.00256820903633e-07
688 1.99816000190367e-07
689 1.99841380776888e-07
690 1.99165157255266e-07
691 1.99332276906716e-07
692 1.98558737452004e-07
693 1.98716918475839e-07
694 1.97938391011121e-07
695 1.9812952700704e-07
696 1.97328375861616e-07
697 1.97526020428995e-07
698 1.96719071254847e-07
699 1.96931424056856e-07
700 1.9611094614902e-07
701 1.96334639213092e-07
702 1.95508690126189e-07
703 1.9573096210479e-07
704 1.94910214190713e-07
705 1.95130496649654e-07
706 1.94310317169766e-07
707 1.94526734276224e-07
708 1.9371331916318e-07
709 1.93917500723728e-07
710 1.93114345847789e-07
711 1.93309702467559e-07
712 1.92518641028983e-07
713 1.92703538459682e-07
714 1.91925877857102e-07
715 1.9209139168197e-07
716 1.91328084042652e-07
717 1.91487814049651e-07
718 1.90737623029236e-07
719 1.90879504202712e-07
720 1.90137100730681e-07
721 1.90272146483039e-07
722 1.89552181950603e-07
723 1.89662813454561e-07
724 1.88962360425649e-07
725 1.89052684618218e-07
726 1.88364651876327e-07
727 1.88442285775636e-07
728 1.87772627668892e-07
729 1.8782708366416e-07
730 1.87176638632991e-07
731 1.87211853130975e-07
732 1.86575377369991e-07
733 1.86591663009494e-07
734 1.85969341259806e-07
735 1.85967124366471e-07
736 1.85364427807144e-07
737 1.85336574531902e-07
738 1.8475387264516e-07
739 1.84704589401008e-07
740 1.84146117021555e-07
741 1.84073570608234e-07
742 1.83546120524625e-07
743 1.8345080832205e-07
744 1.831117515394e-07
745 1.82920302904677e-07
746 1.82222962052947e-07
747 1.82122093406178e-07
748 1.81642349161848e-07
749 1.81520036335314e-07
750 1.81216122996375e-07
751 1.8101002297044e-07
752 1.80600324029001e-07
753 1.80416449779841e-07
754 1.79817547518724e-07
755 1.79458012894429e-07
756 1.79413163436948e-07
757 1.78763727376463e-07
758 1.78633612790691e-07
759 1.78438256170921e-07
760 1.78087105950908e-07
761 1.77523105548971e-07
762 1.7761698245522e-07
763 1.76875090573958e-07
764 1.7671681007414e-07
765 1.76695365894375e-07
766 1.76135102947228e-07
767 1.75723954498608e-07
768 1.75550667336211e-07
769 1.75089709841814e-07
770 1.74926611862247e-07
771 1.74570075728298e-07
772 1.74191626456377e-07
773 1.73944442849461e-07
774 1.73778119005874e-07
775 1.73323584817808e-07
776 1.73396415448224e-07
777 1.73029164329819e-07
778 1.72428471501007e-07
779 1.72161946920824e-07
780 1.71924028791182e-07
781 1.71670023974002e-07
782 1.7129042362285e-07
783 1.71010668736926e-07
784 1.70733144955193e-07
785 1.70453546388671e-07
786 1.70174999425399e-07
787 1.69895216117766e-07
788 1.69617308642955e-07
789 1.6934390600909e-07
790 1.69078120393351e-07
791 1.68813599543682e-07
792 1.68548936585466e-07
793 1.68288494251101e-07
794 1.68030709346567e-07
795 1.67778736681612e-07
796 1.67530231465207e-07
797 1.67285591601285e-07
798 1.67040212772918e-07
799 1.66798230338827e-07
800 1.66559701142432e-07
801 1.66322706718347e-07
802 1.66086024933065e-07
803 1.65851545830265e-07
804 1.65620079428663e-07
805 1.65390048323388e-07
806 1.65158340337257e-07
807 1.64928167123435e-07
808 1.64695023840977e-07
809 1.64465674856729e-07
810 1.64238926458893e-07
811 1.64010074854559e-07
812 1.63780683237746e-07
813 1.63551845844268e-07
814 1.63325296398398e-07
815 1.63099841188341e-07
816 1.62870307462981e-07
817 1.62644070655915e-07
818 1.62416114335429e-07
819 1.62189920160927e-07
820 1.6196199226215e-07
821 1.6173441963474e-07
822 1.61505525397843e-07
823 1.61279942290093e-07
824 1.61052255975846e-07
825 1.60822466455102e-07
826 1.605962722806e-07
827 1.60365360102332e-07
828 1.60136607973982e-07
829 1.59910484853754e-07
830 1.59683025913182e-07
831 1.59451403192179e-07
832 1.59224441631522e-07
833 1.58996613208728e-07
834 1.58763327817724e-07
835 1.58528337124153e-07
836 1.58295804908448e-07
837 1.58065148525566e-07
838 1.57830228886269e-07
839 1.57593149197055e-07
840 1.57358698515964e-07
841 1.57119842469911e-07
842 1.56884183866168e-07
843 1.56644958337893e-07
844 1.56408773932526e-07
845 1.56176426457932e-07
846 1.5598241986936e-07
847 1.55686763037011e-07
848 1.55413843572205e-07
849 1.55180444494363e-07
850 1.54972696009281e-07
851 1.54684272501981e-07
852 1.54407885588626e-07
853 1.54261641682751e-07
854 1.53925668655575e-07
855 1.53649722278715e-07
856 1.53570709926498e-07
857 1.53164279481643e-07
858 1.52904561900868e-07
859 1.52639174189062e-07
860 1.52432889422016e-07
861 1.52123689645123e-07
862 1.51891853761299e-07
863 1.51614557353241e-07
864 1.51474878862246e-07
865 1.51111819945982e-07
866 1.50885909988574e-07
867 1.50590125258532e-07
868 1.50449352531723e-07
869 1.50088709460761e-07
870 1.49990484032969e-07
871 1.49602911392321e-07
872 1.49352089806598e-07
873 1.4908452783402e-07
874 1.48854937265241e-07
875 1.48563486845887e-07
876 1.48394576626742e-07
877 1.48066689575899e-07
878 1.47814176898464e-07
879 1.47532077221513e-07
880 1.47320264431983e-07
881 1.47019918017577e-07
882 1.46791137467517e-07
883 1.46500397590899e-07
884 1.46298006598045e-07
885 1.46003003465012e-07
886 1.45760651548699e-07
887 1.45485756775088e-07
888 1.45260955264348e-07
889 1.44980305094577e-07
890 1.44757436260079e-07
891 1.44479869845782e-07
892 1.44252055633842e-07
893 1.43989396406141e-07
894 1.43752615144876e-07
895 1.43500102467442e-07
896 1.43256926321556e-07
897 1.4301143380635e-07
898 1.42773089351067e-07
899 1.42529245295009e-07
900 1.42292776672548e-07
901 1.42054403795555e-07
902 1.41822013688397e-07
903 1.41588515134572e-07
904 1.41353027061086e-07
905 1.41121148544698e-07
906 1.40884381494288e-07
907 1.40651820856874e-07
908 1.40419544436554e-07
909 1.40185250074865e-07
910 1.39952064159843e-07
911 1.39714074975927e-07
912 1.39479396921161e-07
913 1.39251042696742e-07
914 1.39020613687535e-07
915 1.38796053761325e-07
916 1.38570044327935e-07
917 1.38345299660614e-07
918 1.38120981318934e-07
919 1.37900457275464e-07
920 1.37681197998063e-07
921 1.37459196025702e-07
922 1.37240803610439e-07
923 1.37025949698e-07
924 1.36811522111202e-07
925 1.36596128186284e-07
926 1.3638424434248e-07
927 1.36174676867995e-07
928 1.35966558900691e-07
929 1.35766242692625e-07
930 1.3558265266056e-07
931 1.35426915903736e-07
932 1.35121894118129e-07
933 1.34959734054974e-07
934 1.34816190211495e-07
935 1.34540044882669e-07
936 1.34335110146822e-07
937 1.34189932055051e-07
938 1.33931095547268e-07
939 1.33765325927016e-07
940 1.3356974193357e-07
941 1.33321165662892e-07
942 1.33179668182493e-07
943 1.3293306722062e-07
944 1.32699909727307e-07
945 1.32514017536778e-07
946 1.32317978795982e-07
947 1.32117008888599e-07
948 1.31909047240697e-07
949 1.31724718244186e-07
950 1.31510191181405e-07
951 1.31333862896099e-07
952 1.3111427676904e-07
953 1.30940122744505e-07
954 1.30731493186431e-07
955 1.30592425762188e-07
956 1.30337639347999e-07
957 1.30160913158761e-07
958 1.29968796613866e-07
959 1.29772502077685e-07
960 1.29580854490996e-07
961 1.2938852478328e-07
962 1.29193864495392e-07
963 1.28999602111435e-07
964 1.28807101873463e-07
965 1.2861436005096e-07
966 1.28422485090596e-07
967 1.28227739537579e-07
968 1.28034628232854e-07
969 1.27838617913767e-07
970 1.27647680869813e-07
971 1.27454711673636e-07
972 1.27261230886688e-07
973 1.2706780694316e-07
974 1.26873899830571e-07
975 1.2668010640482e-07
976 1.26487520901719e-07
977 1.26295759628192e-07
978 1.26102108310988e-07
979 1.25909707548999e-07
980 1.25716852039659e-07
981 1.25523030192198e-07
982 1.25332334732775e-07
983 1.25139337114888e-07
984 1.24946751611787e-07
985 1.2475327082484e-07
986 1.24563598546956e-07
987 1.24369449849837e-07
988 1.2417596906289e-07
989 1.23982346167395e-07
990 1.23781390470867e-07
991 1.23591661349565e-07
992 1.23384609196364e-07
993 1.23271490792831e-07
994 1.23024236131641e-07
995 1.22881885999959e-07
996 1.2264509052784e-07
997 1.22469003827064e-07
998 1.2228080947807e-07
999 1.22093027243864e-07
1000 1.21904363936665e-07
1001 1.21709248901425e-07
1002 1.21533219044068e-07
1003 1.2131327764564e-07
1004 1.21170941724813e-07
1005 1.20883612453326e-07
1006 1.20667323244561e-07
1007 1.20424772376282e-07
1008 1.20249424639951e-07
1009 1.20029596928362e-07
1010 1.19855130265023e-07
1011 1.19657030950293e-07
1012 1.19479196314387e-07
1013 1.19258089625873e-07
1014 1.19077348870178e-07
1015 1.18880109312158e-07
1016 1.1868203131371e-07
1017 1.18482034849876e-07
1018 1.18286926920064e-07
1019 1.18079562128059e-07
1020 1.17878123262471e-07
1021 1.17662843024391e-07
1022 1.1744509009759e-07
1023 1.17224097095914e-07
1024 1.16999004262652e-07
1025 1.16780441317132e-07
1026 1.16567655084054e-07
1027 1.16353255918966e-07
1028 1.16138487271655e-07
1029 1.15944111200861e-07
1030 1.15720382609652e-07
1031 1.15518304255602e-07
1032 1.15317078552835e-07
1033 1.15111269849422e-07
1034 1.14952172225458e-07
1035 1.14707852105767e-07
1036 1.14505176895818e-07
1037 1.14306139664677e-07
1038 1.14120702221499e-07
1039 1.13932586032206e-07
1040 1.13725000971954e-07
1041 1.13548033198185e-07
1042 1.13340639984472e-07
1043 1.13128983514343e-07
1044 1.12928212558927e-07
1045 1.12719327205468e-07
1046 1.12514442207612e-07
1047 1.12311838051937e-07
1048 1.1210738648515e-07
1049 1.11902821231524e-07
1050 1.11699719695935e-07
1051 1.11493541510299e-07
1052 1.11289821802529e-07
1053 1.11085483922579e-07
1054 1.10882830028913e-07
1055 1.10677085274347e-07
1056 1.10474481118672e-07
1057 1.10271201947398e-07
1058 1.10070764947068e-07
1059 1.09869027653531e-07
1060 1.09668846448585e-07
1061 1.09467819697784e-07
1062 1.092697843319e-07
1063 1.09076694343457e-07
1064 1.08884336214032e-07
1065 1.08696951883758e-07
1066 1.08510242569082e-07
1067 1.08319028413462e-07
1068 1.08132795162419e-07
1069 1.07940167026754e-07
1070 1.07749187350237e-07
1071 1.07561220374919e-07
1072 1.0737368683067e-07
1073 1.07181968189707e-07
1074 1.06991848269899e-07
1075 1.06803810240308e-07
1076 1.06617029871359e-07
1077 1.06426490731337e-07
1078 1.06239575359268e-07
1079 1.06049789394547e-07
1080 1.05860983978801e-07
1081 1.05672718575534e-07
1082 1.05479287526578e-07
1083 1.05292158991688e-07
1084 1.05097839764312e-07
1085 1.04904259501382e-07
1086 1.04710608184178e-07
1087 1.04516672649879e-07
1088 1.04319852312074e-07
1089 1.04120921662343e-07
1090 1.03922197070005e-07
1091 1.03722811672924e-07
1092 1.03517251659468e-07
1093 1.03314121702169e-07
1094 1.0311163833876e-07
1095 1.02901090315299e-07
1096 1.02694279746629e-07
1097 1.02480171904062e-07
1098 1.02266469070855e-07
1099 1.0205093303739e-07
1100 1.01833109056315e-07
1101 1.01611696834425e-07
1102 1.01388494044841e-07
1103 1.01164459920255e-07
1104 1.00935615421349e-07
1105 1.00706060379707e-07
1106 1.00472831832121e-07
1107 1.00240200140433e-07
1108 9.99985942939929e-08
1109 9.97599158836238e-08
1110 9.95197879660736e-08
1111 9.92788216080953e-08
1112 9.90301245451519e-08
1113 9.87844046562714e-08
1114 9.85313519663578e-08
1115 9.8285390492947e-08
1116 9.80243441972561e-08
1117 9.77758674025608e-08
1118 9.75143592540917e-08
1119 9.72554587974628e-08
1120 9.69943698692077e-08
1121 9.67344320201846e-08
1122 9.64635589184581e-08
1123 9.62053903208471e-08
1124 9.59333021910425e-08
1125 9.56696482035113e-08
1126 9.54007290943082e-08
1127 9.51285556993753e-08
1128 9.48588763094449e-08
1129 9.45864329082724e-08
1130 9.43169666811627e-08
1131 9.40456601483675e-08
1132 9.37741475581788e-08
1133 9.35125967771455e-08
1134 9.32461858837996e-08
1135 9.29836474483636e-08
1136 9.27091576841121e-08
1137 9.24373537714018e-08
1138 9.21589986546678e-08
1139 9.18869247357179e-08
1140 9.16180127319421e-08
1141 9.13432458560237e-08
1142 9.10697792733117e-08
1143 9.08045905134713e-08
1144 9.05374406556803e-08
1145 9.02691681403667e-08
1146 9.00103884760028e-08
1147 8.97510830100146e-08
1148 8.94787959282439e-08
1149 8.92187017598189e-08
1150 8.89595170860957e-08
1151 8.87086031298168e-08
1152 8.84586768279405e-08
1153 8.82082531461492e-08
1154 8.79600818848303e-08
1155 8.77189947345869e-08
1156 8.74764793934446e-08
1157 8.72400889306846e-08
1158 8.70002239139467e-08
1159 8.67684164518323e-08
1160 8.65331273303127e-08
1161 8.6305803392861e-08
1162 8.60792042089997e-08
1163 8.58574153994596e-08
1164 8.56331965337631e-08
1165 8.54182999887598e-08
1166 8.52026502684566e-08
1167 8.49873416086666e-08
1168 8.47731200792623e-08
1169 8.45649097414025e-08
1170 8.43562233399098e-08
1171 8.41502867388044e-08
1172 8.39465315038979e-08
1173 8.37406659570661e-08
1174 8.35369817764331e-08
1175 8.33349815820839e-08
1176 8.31302386927746e-08
1177 8.29290982551356e-08
1178 8.27298620720285e-08
1179 8.25241315283165e-08
1180 8.23238650582425e-08
1181 8.21274923623605e-08
1182 8.19222591985636e-08
1183 8.17215521919934e-08
1184 8.1522429695724e-08
1185 8.13210050409907e-08
1186 8.1114585270825e-08
1187 8.09147451263925e-08
1188 8.07107980449473e-08
1189 8.05061830533305e-08
1190 8.02951447553824e-08
1191 8.00899684350043e-08
1192 7.98847139549252e-08
1193 7.96783368173237e-08
1194 7.94617918131735e-08
1195 7.92513361602687e-08
1196 7.9041498679544e-08
1197 7.88344394209162e-08
1198 7.86153506737719e-08
1199 7.83991325192801e-08
1200 7.81857423248766e-08
1201 7.79699362851716e-08
1202 7.77573703203416e-08
1203 7.75426514110222e-08
1204 7.73274848597794e-08
1205 7.71128583210157e-08
1206 7.68982104659699e-08
1207 7.66767911386523e-08
1208 7.64509806572278e-08
1209 7.62052678737746e-08
1210 7.59171214781418e-08
1211 7.55838058807967e-08
1212 7.52358886302318e-08
1213 7.49161017665756e-08
1214 7.46227186709802e-08
1215 7.43341317388513e-08
1216 7.40436618684726e-08
1217 7.37612921852815e-08
1218 7.34721368189639e-08
1219 7.31792084707195e-08
1220 7.28920497294894e-08
1221 7.26070581436034e-08
1222 7.23119981671516e-08
1223 7.20275394883174e-08
1224 7.17396915206336e-08
1225 7.14567036652625e-08
1226 7.11752647930552e-08
1227 7.08955383288412e-08
1228 7.06245373294223e-08
1229 7.03530886880799e-08
1230 7.00922342389276e-08
1231 6.98358064710192e-08
1232 6.95859796451259e-08
1233 6.93418726882555e-08
1234 6.90983412710011e-08
1235 6.88672514570499e-08
1236 6.86313086362134e-08
1237 6.8406038167268e-08
1238 6.81638141486474e-08
1239 6.79476528375744e-08
1240 6.77075462363064e-08
1241 6.74883011697602e-08
1242 6.72475053420385e-08
1243 6.70343212050284e-08
1244 6.67866046910603e-08
1245 6.65707062807996e-08
1246 6.63224355434977e-08
1247 6.61083490172132e-08
1248 6.58508909623379e-08
1249 6.56367618034892e-08
1250 6.53717862064696e-08
1251 6.5154857509242e-08
1252 6.48947349191076e-08
1253 6.46677449367417e-08
1254 6.4401319832541e-08
1255 6.41658886024743e-08
1256 6.39075494746066e-08
1257 6.36640677953437e-08
1258 6.33978132213997e-08
1259 6.31516670068777e-08
1260 6.28902228072548e-08
1261 6.26286720262215e-08
1262 6.23677749445051e-08
1263 6.21007103518423e-08
1264 6.18392164142278e-08
1265 6.15646555957028e-08
1266 6.1298266018639e-08
1267 6.102764160687e-08
1268 6.07582606448887e-08
1269 6.04898460210279e-08
1270 6.02174168307101e-08
1271 5.99358642716652e-08
1272 5.96705476141324e-08
1273 5.93922990788087e-08
1274 5.91275224337551e-08
1275 5.88501372078554e-08
1276 5.85831543276072e-08
1277 5.83105297380371e-08
1278 5.80401398053709e-08
1279 5.77712953031551e-08
1280 5.75017153892077e-08
1281 5.72317944147471e-08
1282 5.69655114190937e-08
1283 5.66935618451225e-08
1284 5.64279645232091e-08
1285 5.61676962718138e-08
1286 5.58959385443814e-08
1287 5.56339152524288e-08
1288 5.53706804851117e-08
1289 5.51018182193275e-08
1290 5.48416103640648e-08
1291 5.45912008931282e-08
1292 5.43425890953131e-08
1293 5.40950821914521e-08
1294 5.38525348758867e-08
1295 5.36221200775344e-08
1296 5.33980752948082e-08
1297 5.31805532943963e-08
1298 5.29721120301474e-08
1299 5.27638768232919e-08
1300 5.25672056994608e-08
1301 5.23676959574004e-08
1302 5.21752561155608e-08
1303 5.1987143478982e-08
1304 5.18046441300157e-08
1305 5.16220133306433e-08
1306 5.14476816704246e-08
1307 5.12674134256486e-08
1308 5.11023010574263e-08
1309 5.09219511002357e-08
1310 5.07648003633676e-08
1311 5.05891328828056e-08
1312 5.04394392919494e-08
1313 5.02650827627349e-08
1314 5.01248713646874e-08
1315 4.99488486127575e-08
1316 4.98227628042969e-08
1317 4.96424306106746e-08
1318 4.95213292595054e-08
1319 4.93486531638609e-08
1320 4.92325682444061e-08
1321 4.90647913409248e-08
1322 4.89532538949788e-08
1323 4.87886708810947e-08
1324 4.86825939560731e-08
1325 4.85196380850539e-08
1326 4.84134190514851e-08
1327 4.82556181680138e-08
1328 4.81554316422717e-08
1329 4.80051873807952e-08
1330 4.79040771494965e-08
1331 4.77564903178518e-08
1332 4.76567478813195e-08
1333 4.75139501077138e-08
1334 4.7411159442845e-08
1335 4.72741064072579e-08
1336 4.71667469525983e-08
1337 4.70388705764435e-08
1338 4.6931941000139e-08
1339 4.68080649795866e-08
1340 4.66874396920502e-08
1341 4.65725911169557e-08
1342 4.64490099716386e-08
1343 4.63362894720376e-08
1344 4.62045868232508e-08
1345 4.6099330575089e-08
1346 4.59610305370006e-08
1347 4.5866578091136e-08
1348 4.57219471172721e-08
1349 4.56237110313396e-08
1350 4.54786359682657e-08
1351 4.53913919784554e-08
1352 4.52544099971419e-08
1353 4.51598864970038e-08
1354 4.50308732524718e-08
1355 4.49390178403064e-08
1356 4.48174617417862e-08
1357 4.47242740619913e-08
1358 4.46119123864719e-08
1359 4.45127099624187e-08
1360 4.44119976350521e-08
1361 4.43083116863363e-08
1362 4.42205738693247e-08
1363 4.41090080016693e-08
1364 4.40323937311859e-08
1365 4.39074874236667e-08
1366 4.38509033529044e-08
1367 4.37148734988568e-08
1368 4.36827640726278e-08
1369 4.35166356282934e-08
1370 4.35224549732993e-08
1371 4.33268851907087e-08
1372 4.33721183412672e-08
1373 4.31360511754519e-08
1374 4.32222186930176e-08
1375 4.29569553261899e-08
1376 4.306610179583e-08
1377 4.27817461456925e-08
1378 4.29144115798863e-08
1379 4.26115356333412e-08
1380 4.2757186236031e-08
1381 4.24436628065905e-08
1382 4.26130810637915e-08
1383 4.22820214396324e-08
1384 4.24699173606768e-08
1385 4.21255101912266e-08
1386 4.23279082895078e-08
1387 4.1974161035796e-08
1388 4.21908232794976e-08
1389 4.18334558105471e-08
1390 4.20509138621128e-08
1391 4.16858227936245e-08
1392 4.19103969306889e-08
1393 4.15591117075564e-08
1394 4.1764636193875e-08
1395 4.14273486626371e-08
1396 4.162105682326e-08
1397 4.13034086932385e-08
1398 4.14817975524784e-08
1399 4.11836786895492e-08
1400 4.13382004182949e-08
1401 4.1067277578577e-08
1402 4.11984864001624e-08
1403 4.09469436135623e-08
1404 4.10570741848915e-08
1405 4.08334734913751e-08
1406 4.09100557874353e-08
1407 4.07259399537452e-08
1408 4.076493098637e-08
1409 4.06162179444891e-08
1410 4.06089561977296e-08
1411 4.05156974636611e-08
1412 4.04104838480634e-08
1413 4.03677802296443e-08
1414 4.02414173095167e-08
1415 4.02161859369699e-08
1416 4.01246573744629e-08
1417 4.00864053062833e-08
1418 4.00250605991914e-08
1419 3.99249486804365e-08
1420 3.99215167590228e-08
1421 3.97749602143449e-08
1422 3.97839414745249e-08
1423 3.96393815549345e-08
1424 3.96480537290245e-08
1425 3.9528448070314e-08
1426 3.952621341341e-08
1427 3.94216108645651e-08
1428 3.93858314851059e-08
1429 3.93017138833329e-08
1430 3.92499757140286e-08
1431 3.91873520300123e-08
1432 3.91251404607829e-08
1433 3.90654086857012e-08
1434 3.89967809155678e-08
1435 3.89514767107357e-08
1436 3.88744538781793e-08
1437 3.88299064013609e-08
1438 3.87492136155743e-08
1439 3.87143579416716e-08
1440 3.86354130910149e-08
1441 3.8596787987899e-08
1442 3.85080021203521e-08
1443 3.84822556043218e-08
1444 3.83962728278675e-08
1445 3.83667924097608e-08
1446 3.82778040375342e-08
1447 3.82504623530622e-08
1448 3.81634954749188e-08
1449 3.81420548478673e-08
1450 3.80552549472668e-08
1451 3.80284781442697e-08
1452 3.79483005019665e-08
1453 3.79094799995983e-08
1454 3.78361377784131e-08
1455 3.78004436640822e-08
1456 3.77308637666829e-08
1457 3.76831827963997e-08
1458 3.76166795490462e-08
1459 3.75597366542024e-08
1460 3.7492270621442e-08
1461 3.74353454901666e-08
1462 3.73628346039823e-08
1463 3.73019801713781e-08
1464 3.72306026008573e-08
1465 3.71684052424825e-08
1466 3.70887178746671e-08
1467 3.70090447177063e-08
1468 3.69148374090855e-08
1469 3.68519330606887e-08
1470 3.67705013104569e-08
1471 3.67402002154904e-08
1472 3.66689327790937e-08
1473 3.66388839267984e-08
1474 3.65710697280974e-08
1475 3.65419694503544e-08
1476 3.64705243782737e-08
1477 3.64436516520072e-08
1478 3.63722811869138e-08
1479 3.63484531362701e-08
1480 3.62818113330832e-08
1481 3.62554821720096e-08
1482 3.61840477580699e-08
1483 3.61682133132035e-08
1484 3.61023531070259e-08
1485 3.60708014568445e-08
1486 3.60137413224493e-08
1487 3.59804666061336e-08
1488 3.59296770113815e-08
1489 3.58953933243811e-08
1490 3.58506007103188e-08
1491 3.58221008411874e-08
1492 3.57759759594956e-08
1493 3.57399372319378e-08
1494 3.56957343683462e-08
1495 3.56581999483296e-08
1496 3.56180791527549e-08
1497 3.55759617320928e-08
1498 3.55373686034e-08
1499 3.54999798446443e-08
1500 3.54613085562505e-08
1501 3.54205518249273e-08
1502 3.5375869344989e-08
1503 3.53371198968944e-08
1504 3.52999123265363e-08
1505 3.52597702146795e-08
1506 3.52275399961854e-08
1507 3.51924640540346e-08
1508 3.51522082553402e-08
1509 3.51172921853049e-08
1510 3.50786280023385e-08
1511 3.50418112304851e-08
1512 3.50097018042561e-08
1513 3.49671935850893e-08
1514 3.49403954658101e-08
1515 3.4903287371435e-08
1516 3.48705455621712e-08
1517 3.48331603561292e-08
1518 3.48024826735127e-08
1519 3.47619923957154e-08
1520 3.4735904819172e-08
1521 3.47040298720458e-08
1522 3.46716930721414e-08
1523 3.46415447438631e-08
1524 3.46088420144497e-08
1525 3.45783739419403e-08
1526 3.45530501988378e-08
1527 3.45212889385493e-08
1528 3.44942883145904e-08
1529 3.44615500580403e-08
1530 3.44384076811366e-08
1531 3.44051649392441e-08
1532 3.43804273938986e-08
1533 3.43526416202167e-08
1534 3.43213706344159e-08
1535 3.42867174651929e-08
1536 3.42580293022365e-08
1537 3.42311601286838e-08
1538 3.41973311890342e-08
1539 3.41701102968273e-08
1540 3.41353789679033e-08
1541 3.41068329134941e-08
1542 3.40721015845702e-08
1543 3.40444046287303e-08
1544 3.40097230377978e-08
1545 3.39827970208262e-08
1546 3.39484138578428e-08
1547 3.39186776443512e-08
1548 3.38872823135716e-08
1549 3.38546009004403e-08
1550 3.38221752826939e-08
1551 3.37923395932194e-08
1552 3.37582690690397e-08
1553 3.37273426964657e-08
1554 3.36947501011764e-08
1555 3.36635110897987e-08
1556 3.36321548388696e-08
1557 3.36021628299932e-08
1558 3.35666641149146e-08
1559 3.35357874803321e-08
1560 3.35012977359384e-08
1561 3.34681118374647e-08
1562 3.34388658984608e-08
1563 3.33959420117935e-08
1564 3.335645004654e-08
1565 3.33317018430535e-08
1566 3.33082930126238e-08
1567 3.32689609194858e-08
1568 3.32428697902287e-08
1569 3.32102203515205e-08
1570 3.31829852484589e-08
1571 3.31544072196266e-08
1572 3.31214735638241e-08
1573 3.30943308313181e-08
1574 3.30621645616702e-08
1575 3.30356613176264e-08
1576 3.30002869475265e-08
1577 3.29720819536305e-08
1578 3.29410383415052e-08
1579 3.29109788310689e-08
1580 3.28838432039902e-08
1581 3.28530163073992e-08
1582 3.28267262261761e-08
1583 3.27959668311451e-08
1584 3.27675877542788e-08
1585 3.27386224796555e-08
1586 3.27099130004171e-08
1587 3.26782050308339e-08
1588 3.26520357418758e-08
1589 3.26246727411217e-08
1590 3.25954658819683e-08
1591 3.25650013621726e-08
1592 3.25406475099044e-08
1593 3.25131850331672e-08
1594 3.24880318203213e-08
1595 3.24583098176845e-08
1596 3.24269109341913e-08
1597 3.2402059702008e-08
1598 3.23724584916363e-08
1599 3.23445306094072e-08
1600 3.23209619068621e-08
1601 3.22897513171938e-08
1602 3.22650883788356e-08
1603 3.22364464011571e-08
1604 3.2210841993674e-08
1605 3.21864490615553e-08
1606 3.21506590239551e-08
1607 3.2136391325821e-08
1608 3.20910444884248e-08
1609 3.20854418589533e-08
1610 3.20416262411527e-08
1611 3.20273230158818e-08
1612 3.1988143689432e-08
1613 3.1972426484117e-08
1614 3.19312825070028e-08
1615 3.19210329280395e-08
1616 3.18734194593162e-08
1617 3.18647863650767e-08
1618 3.18224415707391e-08
1619 3.18066000204453e-08
1620 3.17714139441705e-08
1621 3.17514512460093e-08
1622 3.17194235321949e-08
1623 3.16940429456736e-08
1624 3.16692450041955e-08
1625 3.16403401257048e-08
1626 3.16140997824732e-08
1627 3.15856283350513e-08
1628 3.15636370373795e-08
1629 3.1532948696622e-08
1630 3.15140695761329e-08
1631 3.14860670869166e-08
1632 3.14612300655881e-08
1633 3.14284775981832e-08
1634 3.14099501963483e-08
1635 3.13766683746053e-08
1636 3.1360290364546e-08
1637 3.13257721984428e-08
1638 3.13109254079791e-08
1639 3.12786134770704e-08
1640 3.12660901613526e-08
1641 3.12323997775366e-08
1642 3.1217496143654e-08
1643 3.11806722663732e-08
1644 3.11751833237395e-08
1645 3.11294172661292e-08
1646 3.11259249485829e-08
1647 3.1087289187326e-08
1648 3.10792067637067e-08
1649 3.10374907996902e-08
1650 3.10394554503546e-08
1651 3.09910888063314e-08
1652 3.09926697639185e-08
1653 3.09514440743897e-08
1654 3.09511065665902e-08
1655 3.09062357928269e-08
1656 3.09070813386825e-08
1657 3.08607219778878e-08
1658 3.08614431787646e-08
1659 3.08197449783165e-08
1660 3.08188496944695e-08
1661 3.07794181253485e-08
1662 3.0777222548295e-08
1663 3.07373930752419e-08
1664 3.07361389673133e-08
1665 3.06944833994294e-08
1666 3.06909520020326e-08
1667 3.06539362782132e-08
1668 3.06493355139992e-08
1669 3.06179757103564e-08
1670 3.06046494813472e-08
1671 3.05741849615515e-08
1672 3.05622691598728e-08
1673 3.05323375471289e-08
1674 3.05158351920909e-08
1675 3.04966469855117e-08
1676 3.04776861526079e-08
1677 3.04522913552319e-08
1678 3.04377429927172e-08
1679 3.04118756844218e-08
1680 3.03943572532717e-08
1681 3.03702947235251e-08
1682 3.03540161894489e-08
1683 3.03264506840151e-08
1684 3.03103320220544e-08
1685 3.02857898759612e-08
1686 3.02658627049368e-08
1687 3.02491578452191e-08
1688 3.02205300783953e-08
1689 3.01997893359385e-08
1690 3.01781426514935e-08
1691 3.01633207300256e-08
1692 3.01338332064915e-08
1693 3.01174196692955e-08
1694 3.0090447467046e-08
1695 3.00704812161712e-08
1696 3.00443190326405e-08
1697 3.00210167836212e-08
1698 2.99978353268671e-08
1699 2.99816953486243e-08
1700 2.99545952486824e-08
1701 2.99324405261814e-08
1702 2.99062499209413e-08
1703 2.98866495995753e-08
1704 2.98589490910217e-08
1705 2.98367623940976e-08
1706 2.98127460496289e-08
1707 2.97913107516479e-08
1708 2.976983992653e-08
1709 2.97444415764403e-08
1710 2.97185742681449e-08
1711 2.9700931492016e-08
1712 2.96873636784767e-08
1713 2.96629973917106e-08
1714 2.96418747325333e-08
1715 2.96242852471096e-08
1716 2.96112503406221e-08
1717 2.95888398227362e-08
1718 2.9565459414016e-08
1719 2.95444149145396e-08
1720 2.95237505554269e-08
1721 2.95017805740372e-08
1722 2.94772650732966e-08
1723 2.94599189487599e-08
1724 2.94359345787143e-08
1725 2.94128845723662e-08
1726 2.93925843664056e-08
1727 2.93690387564993e-08
1728 2.93479480717451e-08
1729 2.93294650788312e-08
1730 2.93058324274398e-08
1731 2.92852710970237e-08
1732 2.92635231602389e-08
1733 2.92413560032401e-08
1734 2.92213471198011e-08
1735 2.91969861621055e-08
1736 2.91765935855892e-08
1737 2.91567676669047e-08
1738 2.91346822223204e-08
1739 2.91104846894541e-08
1740 2.90891470910992e-08
1741 2.9068255358311e-08
1742 2.90472375041873e-08
1743 2.90247861300941e-08
1744 2.90060775398615e-08
1745 2.89832584599026e-08
1746 2.89601960190566e-08
1747 2.89436989930891e-08
1748 2.89200947634072e-08
1749 2.89016384158458e-08
1750 2.88756378807875e-08
1751 2.8857714440278e-08
1752 2.88357107081083e-08
1753 2.88138615189837e-08
1754 2.87930834730332e-08
1755 2.8772314308867e-08
1756 2.87501276119428e-08
1757 2.87329360304511e-08
1758 2.87131349807623e-08
1759 2.86901951085383e-08
1760 2.86700210239133e-08
1761 2.86491577128345e-08
1762 2.86298771356996e-08
1763 2.86071450972258e-08
1764 2.85851093906331e-08
1765 2.85660703980284e-08
1766 2.85453101156463e-08
1767 2.85260135512999e-08
1768 2.85046173331693e-08
1769 2.84837877728705e-08
1770 2.84620753632225e-08
1771 2.84423435914505e-08
1772 2.84240773140709e-08
1773 2.84005672313015e-08
1774 2.8382421746187e-08
1775 2.8358906334347e-08
1776 2.83418977176098e-08
1777 2.8318645206582e-08
1778 2.83012937529747e-08
1779 2.82771992488051e-08
1780 2.8257302275847e-08
1781 2.82398158191199e-08
1782 2.82175296462128e-08
1783 2.81974372740024e-08
1784 2.81749024111377e-08
1785 2.81548704350598e-08
1786 2.81336038909785e-08
1787 2.81140444258199e-08
1788 2.80918470707547e-08
1789 2.80719412160124e-08
1790 2.8049237599248e-08
1791 2.80295591181812e-08
1792 2.80090013404788e-08
1793 2.79882339526694e-08
1794 2.79689462701072e-08
1795 2.79498877375772e-08
1796 2.79276779480142e-08
1797 2.79095537791818e-08
1798 2.78892766658601e-08
1799 2.78702714240353e-08
1800 2.78540674969463e-08
1801 2.78307759060681e-08
1802 2.78113354568177e-08
1803 2.77918452695758e-08
1804 2.77724510056032e-08
1805 2.77507936630172e-08
1806 2.77316090091517e-08
1807 2.77142842008971e-08
1808 2.76921863218149e-08
1809 2.76733000958984e-08
1810 2.7653804579586e-08
1811 2.763495743352e-08
1812 2.76163074630631e-08
1813 2.7596863461099e-08
1814 2.75744191924332e-08
1815 2.75544032035668e-08
1816 2.75386042858372e-08
1817 2.75199241173141e-08
1818 2.7498289867367e-08
1819 2.74818585666026e-08
1820 2.74644573750038e-08
1821 2.74422955470754e-08
1822 2.74267186739507e-08
1823 2.7405180347273e-08
1824 2.73862532651492e-08
1825 2.73686708851528e-08
1826 2.73489888513723e-08
1827 2.73317883880964e-08
1828 2.73100120296021e-08
1829 2.7291509496763e-08
1830 2.72725202421498e-08
1831 2.72571334392069e-08
1832 2.72356324160228e-08
1833 2.72209437213178e-08
1834 2.72006435153571e-08
1835 2.71854094791024e-08
1836 2.71638320725742e-08
1837 2.71448534761021e-08
1838 2.71285536257437e-08
1839 2.71100493165477e-08
1840 2.70898681264953e-08
1841 2.70755560194402e-08
1842 2.70584958883546e-08
1843 2.70412279235188e-08
1844 2.70191051754409e-08
1845 2.70049671513561e-08
1846 2.69847628686648e-08
1847 2.69701008193124e-08
1848 2.6952772458344e-08
1849 2.69339608394148e-08
1850 2.69118949347558e-08
1851 2.68946109827084e-08
1852 2.6878975489808e-08
1853 2.68622759591608e-08
1854 2.68475108811117e-08
1855 2.68264770397764e-08
1856 2.68116249202421e-08
1857 2.67926285602016e-08
1858 2.67803859088644e-08
1859 2.67582596080729e-08
1860 2.6742142722469e-08
1861 2.67253810193324e-08
1862 2.67073350102009e-08
1863 2.66906887702589e-08
1864 2.6674848996322e-08
1865 2.66587800723528e-08
1866 2.66387161218518e-08
1867 2.66234376766761e-08
1868 2.66060027342974e-08
1869 2.65892801110112e-08
1870 2.65733124393819e-08
1871 2.65570534452308e-08
1872 2.65392259279906e-08
1873 2.6522110729843e-08
1874 2.6504933359206e-08
1875 2.64899799873319e-08
1876 2.64731188082123e-08
1877 2.64601442978574e-08
1878 2.64408939187888e-08
1879 2.64245070269453e-08
1880 2.64049067055794e-08
1881 2.63912127707044e-08
1882 2.63747121920233e-08
1883 2.63616701801084e-08
1884 2.63414339229939e-08
1885 2.63255479637792e-08
1886 2.63108006492985e-08
1887 2.62930406336181e-08
1888 2.62787782645546e-08
1889 2.62611727919193e-08
1890 2.62444004306417e-08
1891 2.62273296414151e-08
1892 2.62095234404569e-08
1893 2.61946180302175e-08
1894 2.6179536760651e-08
1895 2.61585366700956e-08
1896 2.61472248297423e-08
1897 2.61267558698819e-08
1898 2.61112518273876e-08
1899 2.60935149043462e-08
1900 2.60792667461374e-08
1901 2.6057049851147e-08
1902 2.60458161704946e-08
1903 2.60252956962859e-08
1904 2.60093262482997e-08
1905 2.5988340368599e-08
1906 2.59779859845821e-08
1907 2.59532608737345e-08
1908 2.59458055040795e-08
1909 2.59161367921479e-08
1910 2.59133106084164e-08
1911 2.58749075499054e-08
1912 2.58806931441313e-08
1913 2.58192809354796e-08
1914 2.58523069618377e-08
1915 2.57629313438201e-08
1916 2.58225565374914e-08
1917 2.57138754733433e-08
1918 2.58018513221714e-08
1919 2.56864201020335e-08
1920 2.57738363984572e-08
1921 2.5655046087536e-08
1922 2.57423913296861e-08
1923 2.56225138883792e-08
1924 2.57058871966365e-08
1925 2.5588780871999e-08
1926 2.56764245420982e-08
1927 2.55566963147658e-08
1928 2.56449954605387e-08
1929 2.5522648883225e-08
1930 2.56147387744932e-08
1931 2.54934047205779e-08
1932 2.5576913031955e-08
1933 2.54586272063761e-08
1934 2.55479157829086e-08
1935 2.542812360673e-08
1936 2.55208085775394e-08
1937 2.53970338093268e-08
1938 2.54883563144404e-08
1939 2.53659582227783e-08
1940 2.54617322781314e-08
1941 2.53356162716045e-08
1942 2.54226364404531e-08
1943 2.53052032661572e-08
1944 2.53921044190974e-08
1945 2.52736729322578e-08
1946 2.53632652658098e-08
1947 2.52441036963091e-08
1948 2.53344420997337e-08
1949 2.52154475077759e-08
1950 2.53049563525565e-08
1951 2.51861553834942e-08
1952 2.52723992844039e-08
1953 2.51593874622813e-08
1954 2.52454732674323e-08
1955 2.51299230313862e-08
1956 2.52183358639968e-08
1957 2.51003235973712e-08
1958 2.51905767356675e-08
1959 2.50767904219629e-08
1960 2.51541170115388e-08
1961 2.50443790150712e-08
1962 2.51307259446776e-08
1963 2.50188119110817e-08
1964 2.51069867118758e-08
1965 2.49938594265586e-08
1966 2.50743070751014e-08
1967 2.49691751719183e-08
1968 2.50490934661229e-08
1969 2.49387017703384e-08
1970 2.50213663122167e-08
1971 2.49123957019037e-08
1972 2.49976537247676e-08
1973 2.48824925108693e-08
1974 2.49651872508139e-08
1975 2.48614213660403e-08
1976 2.49398794949229e-08
1977 2.48310527695139e-08
1978 2.49103671023931e-08
1979 2.48084521814462e-08
1980 2.4886620764164e-08
1981 2.47855336255043e-08
1982 2.48598635010921e-08
1983 2.47580160817051e-08
1984 2.4835996370598e-08
1985 2.47385845142389e-08
1986 2.48109017775278e-08
1987 2.471058913045e-08
1988 2.47852014467753e-08
1989 2.46886315835582e-08
1990 2.47532252473093e-08
1991 2.46616238541719e-08
1992 2.47247964324515e-08
1993 2.46331470776795e-08
1994 2.47014515508681e-08
1995 2.46068374565311e-08
1996 2.46730564867903e-08
1997 2.45825226841134e-08
1998 2.4649301266777e-08
1999 2.45588491765147e-08
2000 2.46172575657511e-08
2001 2.45318787506221e-08
2002 2.45930742437395e-08
2003 2.45054483372087e-08
2004 2.45643416718622e-08
2005 2.44746019006925e-08
2006 2.45356766015448e-08
2007 2.44472140309426e-08
2008 2.45050699732019e-08
2009 2.44171758367884e-08
2010 2.44664519755133e-08
2011 2.43845192926528e-08
2012 2.44373072888493e-08
2013 2.43547457756677e-08
2014 2.44031586049687e-08
2015 2.432381585038e-08
2016 2.43694557866547e-08
2017 2.42929623084365e-08
2018 2.43394211452141e-08
2019 2.4263744791142e-08
2020 2.43074804728849e-08
2021 2.42299940111934e-08
2022 2.42750068935038e-08
2023 2.42005935291445e-08
2024 2.42440005848721e-08
2025 2.41645636833709e-08
2026 2.4210807580971e-08
2027 2.41389876975973e-08
2028 2.41878268525397e-08
2029 2.410881627668e-08
2030 2.41608564266471e-08
2031 2.40881519175673e-08
2032 2.41316922000578e-08
2033 2.4059330527848e-08
2034 2.41035387205102e-08
2035 2.40325395139962e-08
2036 2.40769910675454e-08
2037 2.4009048971152e-08
2038 2.40543478469135e-08
2039 2.39842155025372e-08
2040 2.40275923601985e-08
2041 2.39576340987924e-08
2042 2.39998669826491e-08
2043 2.39344242203288e-08
2044 2.39777477872849e-08
2045 2.39127171397513e-08
2046 2.39540778323999e-08
2047 2.38863364643294e-08
2048 2.39243078681284e-08
2049 2.38647732686559e-08
2050 2.39089796849612e-08
2051 2.3842650520578e-08
2052 2.38813910868885e-08
2053 2.3820296846111e-08
2054 2.38590782686288e-08
2055 2.3799911375022e-08
2056 2.38370638783181e-08
2057 2.3777690927318e-08
2058 2.38164332699853e-08
2059 2.37557369331398e-08
2060 2.37932997748658e-08
2061 2.37374901956855e-08
2062 2.37724684382101e-08
2063 2.37132304903298e-08
2064 2.37537243208408e-08
2065 2.36929071917302e-08
2066 2.37275781245216e-08
2067 2.36733086467211e-08
2068 2.37095303390333e-08
2069 2.36550228294163e-08
2070 2.36875408177184e-08
2071 2.36352555305075e-08
2072 2.36678001641621e-08
2073 2.36159234390243e-08
2074 2.36492780913977e-08
2075 2.35967938522208e-08
2076 2.36311379353538e-08
2077 2.35784121116467e-08
2078 2.36057910996124e-08
2079 2.35601742559766e-08
2080 2.35874662024571e-08
2081 2.35407888737882e-08
2082 2.35694876948855e-08
2083 2.35252475278003e-08
2084 2.35490240640956e-08
2085 2.35051995645108e-08
2086 2.35329782327653e-08
2087 2.34878392291193e-08
2088 2.35134187676067e-08
2089 2.34710597624144e-08
2090 2.34964652179315e-08
2091 2.34522445907714e-08
2092 2.34766357465332e-08
2093 2.34374137875193e-08
2094 2.34581527536193e-08
2095 2.34209398541907e-08
2096 2.34403820797979e-08
2097 2.34056152237372e-08
2098 2.34260699727429e-08
2099 2.33904451363287e-08
2100 2.34063382009708e-08
2101 2.33743175925838e-08
2102 2.33883419298309e-08
2103 2.33578667518941e-08
2104 2.33714718689271e-08
2105 2.33406627359045e-08
2106 2.3357403122759e-08
2107 2.33261356896719e-08
2108 2.33383214975902e-08
2109 2.33132393390179e-08
2110 2.33200267985012e-08
2111 2.32943921929518e-08
2112 2.3304599139351e-08
2113 2.32801973254482e-08
2114 2.32868160310318e-08
2115 2.32674750577644e-08
2116 2.32724595150557e-08
2117 2.32511183639872e-08
2118 2.3255566361513e-08
2119 2.3239385527063e-08
2120 2.32392718402252e-08
2121 2.32252652665466e-08
2122 2.32235279895576e-08
2123 2.32093455565519e-08
2124 2.32111734277396e-08
2125 2.31956676088885e-08
2126 2.31970904707168e-08
2127 2.3182488817497e-08
2128 2.31814887285964e-08
2129 2.31674377459967e-08
2130 2.31632348857147e-08
2131 2.31553016760699e-08
2132 2.31491572577625e-08
2133 2.31422401242298e-08
2134 2.31371366510302e-08
2135 2.31257679672581e-08
2136 2.3123224224264e-08
2137 2.31138805872888e-08
2138 2.31095302893891e-08
2139 2.30979164683731e-08
2140 2.30964882774742e-08
2141 2.30862227112993e-08
2142 2.30826415759111e-08
2143 2.30742749351975e-08
2144 2.30701306946912e-08
2145 2.30619718877279e-08
2146 2.305709223549e-08
2147 2.30513137466914e-08
2148 2.30431993486491e-08
2149 2.30367103171147e-08
2150 2.30326051564589e-08
2151 2.30251391286629e-08
2152 2.30186518734854e-08
2153 2.30137739976044e-08
2154 2.30088819108687e-08
2155 2.30054126859613e-08
2156 2.29987371369589e-08
2157 2.29969465692648e-08
2158 2.29899921322385e-08
2159 2.29904504323031e-08
2160 2.29914398630626e-08
2161 2.3003414284517e-08
2162 2.30461498773593e-08
2163 2.30690009317414e-08
2164 2.30559766833949e-08
2165 2.3046009545169e-08
2166 2.3038831287181e-08
2167 2.30314736171522e-08
2168 2.30276118173833e-08
2169 2.30227605868549e-08
2170 2.30206431695024e-08
2171 2.30147723101481e-08
2172 2.30120171806902e-08
2173 2.30055121619444e-08
2174 2.29993570854958e-08
2175 2.2998536408636e-08
2176 2.29941292673175e-08
2177 2.29893473147058e-08
2178 2.29839329790593e-08
2179 2.29816059515997e-08
2180 2.29781367266924e-08
2181 2.29714132160552e-08
2182 2.29672973972583e-08
2183 2.29618102309814e-08
2184 2.29588099642797e-08
2185 2.29548895447351e-08
2186 2.29485106473248e-08
2187 2.29475549673452e-08
2188 2.29435830334523e-08
2189 2.2941186728076e-08
2190 2.29365131332315e-08
2191 2.29321450717634e-08
2192 2.29260859185842e-08
2193 2.29242917981765e-08
2194 2.29184280442496e-08
2195 2.2916189834632e-08
2196 2.29118075623091e-08
2197 2.29056666967153e-08
2198 2.29025705067443e-08
2199 2.28975149951793e-08
2200 2.2895283890989e-08
2201 2.28901981813578e-08
2202 2.28867538254462e-08
2203 2.28841621208176e-08
2204 2.28816610103877e-08
2205 2.2874523608607e-08
2206 2.2872830740539e-08
2207 2.28658745271559e-08
2208 2.28637873078696e-08
2209 2.28583001415927e-08
2210 2.2855155989987e-08
2211 2.2850233705185e-08
2212 2.28465673046685e-08
2213 2.28436594085224e-08
2214 2.28392913470543e-08
2215 2.28355734321894e-08
2216 2.28305072624835e-08
2217 2.28254091183544e-08
2218 2.28218830500282e-08
2219 2.28178507200028e-08
2220 2.28138361535457e-08
2221 2.28088374853996e-08
2222 2.28070042851414e-08
2223 2.28012613234796e-08
2224 2.27978649292027e-08
2225 2.27929604079691e-08
2226 2.27894147997176e-08
2227 2.27876153502393e-08
2228 2.27810836861408e-08
2229 2.27738947700118e-08
2230 2.27737917413151e-08
2231 2.27683667475276e-08
2232 2.27649472606117e-08
2233 2.27608172309601e-08
2234 2.27555627674292e-08
2235 2.2749492956109e-08
2236 2.27480310144301e-08
2237 2.27443646139136e-08
2238 2.27391439011626e-08
2239 2.27350991366393e-08
2240 2.27321681478543e-08
2241 2.27252616724627e-08
2242 2.27245440242996e-08
2243 2.27191936374993e-08
2244 2.2712903557931e-08
2245 2.27113918782607e-08
2246 2.27065868330101e-08
2247 2.27049810064273e-08
2248 2.26989573803849e-08
2249 2.26949889992056e-08
2250 2.26918430712431e-08
2251 2.26882104215065e-08
2252 2.26822862714471e-08
2253 2.26813057224717e-08
2254 2.26753940069102e-08
2255 2.2672001165347e-08
2256 2.26675371806095e-08
2257 2.26650964663122e-08
2258 2.26599272679096e-08
2259 2.26571632566674e-08
2260 2.26502852029853e-08
2261 2.26449650142513e-08
2262 2.26434000438758e-08
2263 2.26381509094153e-08
2264 2.2634692342649e-08
2265 2.26332499408954e-08
2266 2.26297984795565e-08
2267 2.2622437256814e-08
2268 2.26219327714716e-08
2269 2.26179217577283e-08
2270 2.26103757938745e-08
2271 2.26080398846307e-08
2272 2.26066489972254e-08
2273 2.26016236837268e-08
2274 2.25961329647362e-08
2275 2.25942198284201e-08
2276 2.25932446085153e-08
2277 2.25850858015519e-08
2278 2.25830252276182e-08
2279 2.25791207952852e-08
2280 2.25747154303235e-08
2281 2.25712390999888e-08
2282 2.25682956767059e-08
2283 2.25630305550339e-08
2284 2.25606200388029e-08
2285 2.25563923095251e-08
2286 2.25528591357715e-08
2287 2.25503384854164e-08
2288 2.25471517012465e-08
2289 2.25415046628541e-08
2290 2.25404921394556e-08
2291 2.25381366902866e-08
2292 2.25342127180284e-08
2293 2.25293579347863e-08
2294 2.25245955220998e-08
2295 2.25205809556428e-08
2296 2.25172041012911e-08
2297 2.25143885757007e-08
2298 2.25110987628341e-08
2299 2.25073808479692e-08
2300 2.25059171299336e-08
2301 2.25006289156227e-08
2302 2.24973835116771e-08
2303 2.24942713344944e-08
2304 2.24924310288088e-08
2305 2.24872049869873e-08
2306 2.24866827380765e-08
2307 2.24806981918846e-08
2308 2.24804477255702e-08
2309 2.2473537697465e-08
2310 2.24749072685881e-08
2311 2.24651124369757e-08
2312 2.24670699822127e-08
2313 2.24615241961601e-08
2314 2.24596501396945e-08
2315 2.24529959069741e-08
2316 2.24537508586309e-08
2317 2.24488427846836e-08
2318 2.24489706823761e-08
2319 2.24430074524662e-08
2320 2.24407603610643e-08
2321 2.24359784084527e-08
2322 2.24382485924934e-08
2323 2.24308376317595e-08
2324 2.24292673323134e-08
2325 2.24254623759634e-08
2326 2.24258940306754e-08
2327 2.24168736906449e-08
2328 2.24188116959567e-08
2329 2.24092993050817e-08
2330 2.24144560689865e-08
2331 2.24046647900877e-08
2332 2.2409196276385e-08
2333 2.23982112856902e-08
2334 2.24038423368711e-08
2335 2.23920384456733e-08
2336 2.23986305059043e-08
2337 2.2385615139342e-08
2338 2.23963567691499e-08
2339 2.23792167020065e-08
2340 2.23905995966334e-08
2341 2.23757066208918e-08
2342 2.23848513059011e-08
2343 2.2367132146428e-08
2344 2.23824585532384e-08
2345 2.23600142845726e-08
2346 2.2377850683597e-08
2347 2.23524097009431e-08
2348 2.23742215865741e-08
2349 2.23470184579355e-08
2350 2.23687735001477e-08
2351 2.23408047617113e-08
2352 2.23678107147407e-08
2353 2.23326743764574e-08
2354 2.23663949583397e-08
2355 2.23250768982552e-08
2356 2.23639258223329e-08
2357 2.23213429961788e-08
2358 2.23616822836448e-08
2359 2.23138609811713e-08
2360 2.23578329183738e-08
2361 2.23055547365902e-08
2362 2.23565042034579e-08
2363 2.22974243513363e-08
2364 2.23566871682124e-08
2365 2.22923084436388e-08
2366 2.23510667751725e-08
2367 2.22851053166551e-08
2368 2.23517453434852e-08
2369 2.22764846569135e-08
2370 2.23515250752371e-08
2371 2.22694271911905e-08
2372 2.23504219576398e-08
2373 2.22649028103206e-08
2374 2.23489617923178e-08
2375 2.22598792731787e-08
2376 2.23504592611334e-08
2377 2.22512781533624e-08
2378 2.23468976656704e-08
2379 2.22462244181543e-08
2380 2.23420641987104e-08
2381 2.22405933669734e-08
2382 2.23412470745643e-08
2383 2.22358540469259e-08
2384 2.23397709220308e-08
2385 2.22283844664162e-08
2386 2.23363212370487e-08
2387 2.22205507327544e-08
2388 2.23367404572627e-08
2389 2.22180194242583e-08
2390 2.233129592355e-08
2391 2.22109264313985e-08
2392 2.23274732036316e-08
2393 2.22015295037181e-08
2394 2.23298020074481e-08
2395 2.21953975199085e-08
2396 2.23279137401278e-08
2397 2.21863327709571e-08
2398 2.23310561153767e-08
2399 2.21766729424644e-08
2400 2.23316849456978e-08
2401 2.21660361177101e-08
2402 2.23307274893614e-08
2403 2.21527542976219e-08
2404 2.2325533421963e-08
2405 2.21458744675829e-08
2406 2.23115161901433e-08
2407 2.21375380249356e-08
2408 2.23018528089369e-08
2409 2.21323919191718e-08
2410 2.22974811947552e-08
2411 2.21273399603206e-08
2412 2.22829878993025e-08
2413 2.21192113514235e-08
2414 2.22701910246315e-08
2415 2.21126210675493e-08
2416 2.22604938926452e-08
2417 2.21106173370345e-08
2418 2.22497682500489e-08
2419 2.21058744642733e-08
2420 2.22402913863107e-08
2421 2.20953992879913e-08
2422 2.22302851682343e-08
2423 2.20881410939455e-08
2424 2.22204921129787e-08
2425 2.20824052377111e-08
2426 2.22131717464435e-08
2427 2.2073834315961e-08
2428 2.22029008511981e-08
2429 2.20693188168752e-08
2430 2.21913598608126e-08
2431 2.20629150504692e-08
2432 2.2182030434692e-08
2433 2.20548237450657e-08
2434 2.2172805813625e-08
2435 2.20508411530318e-08
2436 2.21645048981145e-08
2437 2.20403926221024e-08
2438 2.21585398918478e-08
2439 2.20349392066055e-08
2440 2.21457430171768e-08
2441 2.20261089367568e-08
2442 2.21386144971802e-08
2443 2.20170495168759e-08
2444 2.21304894409968e-08
2445 2.2010976152842e-08
2446 2.21203961814354e-08
2447 2.20004210405023e-08
2448 2.21126921218229e-08
2449 2.19905622600436e-08
2450 2.21010143519607e-08
2451 2.19797247069664e-08
2452 2.20876366086031e-08
2453 2.19699032300014e-08
2454 2.20803588746321e-08
2455 2.19564970649344e-08
2456 2.20683507023978e-08
2457 2.19441282922617e-08
2458 2.205822724477e-08
2459 2.1933207250413e-08
2460 2.20515232740581e-08
2461 2.19273417201293e-08
2462 2.20438245435162e-08
2463 2.19167635151507e-08
2464 2.2033638913399e-08
2465 2.19036841997422e-08
2466 2.20291767050185e-08
2467 2.18957474373838e-08
2468 2.20192415412157e-08
2469 2.18861426759531e-08
2470 2.20132196915301e-08
2471 2.18773639204528e-08
2472 2.20067253309253e-08
2473 2.18689617526024e-08
2474 2.19997922101811e-08
2475 2.18577351773774e-08
2476 2.19948752544497e-08
2477 2.18490665560012e-08
2478 2.19884590535457e-08
2479 2.18438938048848e-08
2480 2.19818048208253e-08
2481 2.18363105375374e-08
2482 2.19741931317685e-08
2483 2.18303544130549e-08
2484 2.19663967016004e-08
2485 2.18179589950296e-08
2486 2.19590763350652e-08
2487 2.1814381412355e-08
2488 2.19548468294306e-08
2489 2.18064037937893e-08
2490 2.19494307174273e-08
2491 2.17991207307477e-08
2492 2.19398668122039e-08
2493 2.1794008375764e-08
2494 2.19317968230825e-08
2495 2.17863416196451e-08
2496 2.19233324827428e-08
2497 2.17753175490998e-08
2498 2.19162394898831e-08
2499 2.17718163497693e-08
2500 2.19040199311848e-08
2501 2.17609699149079e-08
2502 2.19000497736488e-08
2503 2.17532853952207e-08
2504 2.18931663908961e-08
2505 2.1742332378949e-08
2506 2.18820304098699e-08
2507 2.17328572915676e-08
2508 2.18774491855811e-08
2509 2.17239009003833e-08
2510 2.18652758121607e-08
2511 2.17177493766485e-08
2512 2.1852033071923e-08
2513 2.17100240007539e-08
2514 2.18457145706452e-08
2515 2.16998543578484e-08
2516 2.18357136816394e-08
2517 2.16931859142733e-08
2518 2.18274269769836e-08
2519 2.1685600870569e-08
2520 2.18144311503465e-08
2521 2.16788951235003e-08
2522 2.18065103751997e-08
2523 2.1672361683045e-08
2524 2.17952766945473e-08
2525 2.16629949534308e-08
2526 2.17839435379119e-08
2527 2.16559126187121e-08
2528 2.1771466407472e-08
2529 2.16511928385898e-08
2530 2.17590603313056e-08
2531 2.16453432955177e-08
2532 2.17440963012905e-08
2533 2.16368771788211e-08
2534 2.17369180433025e-08
2535 2.16349569370777e-08
2536 2.1723407073182e-08
2537 2.16284128384814e-08
2538 2.17154756398941e-08
2539 2.16202931113685e-08
2540 2.1704725128302e-08
2541 2.16148183795895e-08
2542 2.16948521369886e-08
2543 2.16076543324561e-08
2544 2.16822559906404e-08
2545 2.16039133249524e-08
2546 2.16689493015565e-08
2547 2.15951150295268e-08
2548 2.16569766564589e-08
2549 2.15861017949237e-08
2550 2.16410089848296e-08
2551 2.15741220443988e-08
2552 2.16221938131866e-08
2553 2.15616005050379e-08
2554 2.16061977198478e-08
2555 2.15492779176429e-08
2556 2.1593653087848e-08
2557 2.15412541137994e-08
2558 2.15781721379926e-08
2559 2.15259117197775e-08
2560 2.15596376307303e-08
2561 2.15096775946222e-08
2562 2.15385540514035e-08
2563 2.1497458035924e-08
2564 2.15223359134598e-08
2565 2.1481795897671e-08
2566 2.15109494661192e-08
2567 2.14715853985581e-08
2568 2.14904467554788e-08
2569 2.14539230825039e-08
2570 2.14749480420551e-08
2571 2.14361079997616e-08
2572 2.14597335457256e-08
2573 2.14143476284789e-08
2574 2.14364828110547e-08
2575 2.13953601502226e-08
2576 2.14175095436531e-08
2577 2.13720490194191e-08
2578 2.13972768392523e-08
2579 2.13561701656317e-08
2580 2.13797619608158e-08
2581 2.13394244497067e-08
2582 2.13637569856928e-08
2583 2.13190887166093e-08
2584 2.13496011980396e-08
2585 2.13080042499314e-08
2586 2.13357118639124e-08
2587 2.12867927729121e-08
2588 2.13159339068625e-08
2589 2.12807957922223e-08
2590 2.13058317655168e-08
2591 2.12480806283111e-08
2592 2.12836646085179e-08
2593 2.12614530425981e-08
2594 2.12675033139931e-08
2595 2.12197566185068e-08
2596 2.12572359714613e-08
2597 2.12361452867071e-08
2598 2.12304342994685e-08
2599 2.11853929954486e-08
2600 2.12377049280121e-08
2601 2.12072919225648e-08
2602 2.11908428582319e-08
2603 2.11606039357548e-08
2604 2.12187085679716e-08
2605 2.11671977723427e-08
2606 2.11735500244004e-08
2607 2.11277928485742e-08
2608 2.11973976149693e-08
2609 2.11377439995886e-08
2610 2.11463859756122e-08
2611 2.1106481895572e-08
2612 2.11778043990307e-08
2613 2.1106989933628e-08
2614 2.11304644892607e-08
2615 2.10786463838986e-08
2616 2.11564472607506e-08
2617 2.10794652844015e-08
2618 2.11097503921565e-08
2619 2.10639985454009e-08
2620 2.11398596405843e-08
2621 2.1051498322322e-08
2622 2.10935020561465e-08
2623 2.10367847586213e-08
2624 2.11164703500799e-08
2625 2.10229860186928e-08
2626 2.10793462684933e-08
2627 2.10124273536394e-08
2628 2.10975681369518e-08
2629 2.09990105304314e-08
2630 2.10569091052548e-08
2631 2.09968629150126e-08
2632 2.10790336296895e-08
2633 2.09819237539932e-08
2634 2.10299937464242e-08
2635 2.09790922411912e-08
2636 2.10555963775505e-08
2637 2.09594226419085e-08
2638 2.10367119279908e-08
2639 2.09527595274039e-08
2640 2.10317132598448e-08
2641 2.09397779116216e-08
2642 2.10108286324839e-08
2643 2.09313046894977e-08
2644 2.1011281603478e-08
2645 2.09243502524714e-08
2646 2.09952784047118e-08
2647 2.09169019882438e-08
2648 2.09908073145471e-08
2649 2.09135322393195e-08
2650 2.09738413303739e-08
2651 2.09019876962202e-08
2652 2.09678159279747e-08
2653 2.08917114719043e-08
2654 2.09419983576709e-08
2655 2.08876791418788e-08
2656 2.09445172316691e-08
2657 2.08700559056751e-08
2658 2.09225987646278e-08
2659 2.08702779502801e-08
2660 2.09120578631428e-08
2661 2.08605950291485e-08
2662 2.09032471332193e-08
2663 2.08489296937842e-08
2664 2.08890345021473e-08
2665 2.08385699806968e-08
2666 2.08670680734713e-08
2667 2.08295993786578e-08
2668 2.08645083432657e-08
2669 2.08203267959561e-08
2670 2.08434354220799e-08
2671 2.08132568957353e-08
2672 2.08315373839696e-08
2673 2.07950741071272e-08
2674 2.08230801490572e-08
2675 2.07800017193449e-08
2676 2.07977421951e-08
2677 2.07812700381282e-08
2678 2.07884554015436e-08
2679 2.07668122698124e-08
2680 2.07786925443543e-08
2681 2.07494927906282e-08
2682 2.07558557008269e-08
2683 2.07352179870668e-08
2684 2.07483186187574e-08
2685 2.07289083675732e-08
2686 2.07269064134152e-08
2687 2.07182164757569e-08
2688 2.07150527842259e-08
2689 2.0704677083927e-08
2690 2.06971471072848e-08
2691 2.0691658164651e-08
2692 2.06801473723317e-08
2693 2.06789945167429e-08
2694 2.066295579084e-08
2695 2.06681818326615e-08
2696 2.06462011931308e-08
2697 2.06556070025954e-08
2698 2.06326529195167e-08
2699 2.06452277495828e-08
2700 2.06144559200538e-08
2701 2.06287129600469e-08
2702 2.05983106127405e-08
2703 2.06147365844345e-08
2704 2.05799270958096e-08
2705 2.06008099468136e-08
2706 2.05652757045982e-08
2707 2.05903578631705e-08
2708 2.0551706114702e-08
2709 2.05693613253288e-08
2710 2.05262491448366e-08
2711 2.05501073935466e-08
2712 2.050571623613e-08
2713 2.05097379080144e-08
2714 2.04714361018432e-08
2715 2.03868921744288e-08
2716 2.03898835593463e-08
2717 2.04185379715227e-08
2718 2.04923473745566e-08
2719 2.04811545501116e-08
2720 2.0487332719199e-08
2721 2.04714627471958e-08
2722 2.04597991881883e-08
2723 2.04641334988764e-08
2724 2.04341379372863e-08
2725 2.044967040149e-08
2726 2.03792236419531e-08
2727 2.02743226651592e-08
2728 2.0245721543688e-08
2729 2.0391988542201e-08
2730 2.03493755179807e-08
2731 2.03714325408555e-08
2732 2.03895016426259e-08
2733 2.03374472818041e-08
2734 2.02463503740091e-08
2735 2.01985095316104e-08
2736 2.03355767780522e-08
2737 2.02940579896449e-08
2738 2.02401544413533e-08
2739 2.01756122919505e-08
2740 2.0156074143074e-08
2741 2.02190957310222e-08
2742 2.02881516031539e-08
2743 2.02747703070827e-08
2744 2.02478993571731e-08
2745 2.0138575251849e-08
2746 2.01283647527362e-08
2747 2.01042702485665e-08
2748 2.01882333072945e-08
2749 2.01569640978505e-08
2750 2.01140277766854e-08
2751 2.01382341913359e-08
2752 2.00776248959755e-08
2753 2.00802841021641e-08
2754 2.00283363227527e-08
2755 2.01297147839341e-08
2756 2.00678957895661e-08
2757 2.00341698786133e-08
2758 2.01866843241305e-08
2759 2.00204866018794e-08
2760 2.00767171776306e-08
2761 1.99827212554737e-08
2762 2.01512264652592e-08
2763 2.01085619266905e-08
2764 2.00571914632519e-08
2765 1.99196374950361e-08
2766 2.00743919265278e-08
2767 2.01262615462383e-08
2768 1.9966636344293e-08
2769 1.99422149904649e-08
2770 1.98975467213813e-08
2771 2.00434175923192e-08
2772 2.00905319047706e-08
2773 1.98986302990534e-08
2774 2.00162162400375e-08
2775 1.99626342123338e-08
2776 1.98609289014939e-08
2777 1.98505301085561e-08
2778 1.99926688537744e-08
2779 2.0031396985587e-08
2780 1.98867677880799e-08
2781 1.98552640995331e-08
2782 1.98003231588473e-08
2783 1.9890018521096e-08
2784 1.97980014604582e-08
2785 1.98848741916891e-08
2786 1.9824748065389e-08
2787 1.97629610454442e-08
2788 1.98544736207396e-08
2789 1.97612255448121e-08
2790 1.97955962732976e-08
2791 1.98055900568761e-08
2792 1.973525520782e-08
2793 1.9659555761109e-08
2794 1.96937310903422e-08
2795 1.97289526937539e-08
2796 1.97684855152147e-08
2797 1.9734692102702e-08
2798 1.96815932440586e-08
2799 1.95751788112375e-08
2800 1.97453591255226e-08
2801 1.98352214653141e-08
2802 1.98459808586904e-08
2803 1.96846627886771e-08
2804 1.98018010877377e-08
2805 1.97983069938346e-08
2806 1.97989695749357e-08
2807 1.96128198126644e-08
2808 1.97361433862397e-08
2809 1.97190388462332e-08
2810 1.95715177397915e-08
2811 1.97040712635044e-08
2812 1.9668625839131e-08
2813 1.95376728129304e-08
2814 1.96871372537544e-08
2815 1.96046237022074e-08
2816 1.95194242991192e-08
2817 1.96670715268965e-08
2818 1.96940934671375e-08
2819 1.95068032837753e-08
2820 1.963334206323e-08
2821 1.9548915375367e-08
2822 1.94804350428512e-08
2823 1.96221989767764e-08
2824 1.96223552961783e-08
2825 1.94596374569755e-08
2826 1.95836769023572e-08
2827 1.9516772198358e-08
2828 1.94272438136522e-08
2829 1.95725480267583e-08
2830 1.9504318160557e-08
2831 1.94065528091869e-08
2832 1.95438190075947e-08
2833 1.94802662889515e-08
2834 1.93823819216732e-08
2835 1.95259755031429e-08
2836 1.94477802750725e-08
2837 1.93595077746522e-08
2838 1.95018969861849e-08
2839 1.94591844859815e-08
2840 1.93428508765692e-08
2841 1.94749052440102e-08
2842 1.93682563320863e-08
2843 1.94633287264878e-08
2844 1.93545197646472e-08
2845 1.94645473072796e-08
2846 1.93319547037163e-08
2847 1.94336458037014e-08
2848 1.93232274625643e-08
2849 1.94506935002892e-08
2850 1.94560314525916e-08
2851 1.92931395304186e-08
2852 1.93839397866213e-08
2853 1.92744042948334e-08
2854 1.9264520645379e-08
2855 1.92971114643115e-08
2856 1.93248936852797e-08
2857 1.92993816483522e-08
2858 1.92521554254199e-08
2859 1.92908515828094e-08
2860 1.93199642950503e-08
2861 1.92940703414024e-08
2862 1.92928979458884e-08
2863 1.92745552851648e-08
2864 1.92817779520738e-08
2865 1.92628579753773e-08
2866 1.92657747533076e-08
2867 1.926624548787e-08
2868 1.92305353863276e-08
2869 1.9233329595636e-08
2870 1.92390121611652e-08
2871 1.92649753927299e-08
2872 1.91554594408672e-08
2873 1.91968627660799e-08
2874 1.92104536722582e-08
2875 1.92720897018717e-08
2876 1.91138553873316e-08
2877 1.92520008823749e-08
2878 1.9186607858046e-08
2879 1.90987261561304e-08
2880 1.91469364807517e-08
2881 1.90752942330619e-08
2882 1.91601294829979e-08
2883 1.90634903418641e-08
2884 1.90809217315291e-08
2885 1.90516011855379e-08
2886 1.91588007680821e-08
2887 1.90411775236043e-08
2888 1.917605629842e-08
2889 1.91041156227811e-08
2890 1.9076100699067e-08
2891 1.90929032584108e-08
2892 1.91727025367072e-08
2893 1.91365980839464e-08
2894 1.90076701045427e-08
2895 1.91241458225022e-08
2896 1.89782518589254e-08
2897 1.91147613293197e-08
2898 1.90542834843654e-08
2899 1.90639735109244e-08
2900 1.89509989922954e-08
2901 1.90827460500032e-08
2902 1.9080834690044e-08
2903 1.90141129507992e-08
2904 1.90663040910977e-08
2905 1.89197546518471e-08
2906 1.90482776218914e-08
2907 1.90145890144322e-08
2908 1.90366691299459e-08
2909 1.89913134107655e-08
2910 1.90214137774092e-08
2911 1.89558999608153e-08
2912 1.9005952367479e-08
2913 1.89264959260527e-08
2914 1.89953830442846e-08
2915 1.89202395972643e-08
2916 1.89797564331684e-08
2917 1.89172997266951e-08
2918 1.89556388363599e-08
2919 1.89144948592457e-08
2920 1.89310735976278e-08
2921 1.89058955157861e-08
2922 1.89126705407716e-08
2923 1.88905548981211e-08
2924 1.8893166142675e-08
2925 1.88738216166939e-08
2926 1.88758626507024e-08
2927 1.88541608991954e-08
2928 1.88542585988216e-08
2929 1.88401099165958e-08
2930 1.88300379733164e-08
2931 1.8818981928348e-08
2932 1.88087483365962e-08
2933 1.88052933225435e-08
2934 1.8785964783774e-08
2935 1.879000244287e-08
2936 1.87727433598184e-08
2937 1.87587492206376e-08
2938 1.87468121026768e-08
2939 1.87385840177967e-08
2940 1.87284570074553e-08
2941 1.87166815379669e-08
2942 1.87030106957309e-08
2943 1.86915016797684e-08
2944 1.86794597567541e-08
2945 1.86690822800983e-08
2946 1.86567579163466e-08
2947 1.86404669477724e-08
2948 1.86228579224235e-08
2949 1.86086364095672e-08
2950 1.85951805065088e-08
2951 1.8579195071311e-08
2952 1.85692439202967e-08
2953 1.85636999106009e-08
2954 1.85573778566095e-08
2955 1.85573778566095e-08
2956 1.85507893490922e-08
2957 1.85535640184753e-08
2958 1.85464674729019e-08
2959 1.85420354625876e-08
2960 1.85352799775274e-08
2961 1.852971998062e-08
2962 1.85278938857891e-08
2963 1.85223836268733e-08
2964 1.85169763966542e-08
2965 1.85137665198454e-08
2966 1.85084623183229e-08
2967 1.85088406823297e-08
2968 1.85024493504216e-08
2969 1.84957560378507e-08
2970 1.84908675038287e-08
2971 1.84835506900072e-08
2972 1.84811312919919e-08
2973 1.84783068846173e-08
2974 1.84693540461467e-08
2975 1.84689792348536e-08
2976 1.8458234052332e-08
2977 1.84561592675436e-08
2978 1.84470501096712e-08
2979 1.84442239259397e-08
2980 1.84359869592754e-08
2981 1.8429576087442e-08
2982 1.84291177873774e-08
2983 1.84229680399994e-08
2984 1.84138624348407e-08
2985 1.84072526110413e-08
2986 1.84029236294236e-08
2987 1.8399578749495e-08
2988 1.83934840691791e-08
2989 1.83834103495428e-08
2990 1.8380100996751e-08
2991 1.8375887478328e-08
2992 1.83700166189738e-08
2993 1.83651263085949e-08
2994 1.83579089707564e-08
2995 1.83501516204387e-08
2996 1.8344087138189e-08
2997 1.83423694011253e-08
2998 1.83336954506785e-08
2999 1.8324742612208e-08
3000 9.09508823809801e-09
3001 9.21601106540493e-09
3002 9.37755739727208e-09
3003 9.43102040906751e-09
3004 9.46234823828718e-09
3005 9.47659906103127e-09
3006 9.4785255200236e-09
3007 9.47789224881035e-09
3008 9.47605283130315e-09
3009 9.47413791863028e-09
3010 9.47195744060991e-09
3011 9.47049727528793e-09
3012 9.46850420291412e-09
3013 9.4665333350008e-09
3014 9.46485734232283e-09
3015 9.46334832718776e-09
3016 9.46160838566357e-09
3017 9.45967926213598e-09
3018 9.45818001696352e-09
3019 9.45666123186584e-09
3020 9.45498168647418e-09
3021 9.45277278674439e-09
3022 9.45131883867134e-09
3023 9.44947231573678e-09
3024 9.44804323665949e-09
3025 9.44671718627887e-09
3026 9.44487510423642e-09
3027 9.44346734144119e-09
3028 9.4421377383469e-09
3029 9.44074329822797e-09
3030 9.43908418093997e-09
3031 9.43787803464602e-09
3032 9.43641165207509e-09
3033 9.43470457315243e-09
3034 9.4333145739256e-09
3035 9.43199918168602e-09
3036 9.43045552759258e-09
3037 9.42894562427909e-09
3038 9.42785227664444e-09
3039 9.42668432202254e-09
3040 9.42504740919503e-09
3041 9.42379774215851e-09
3042 9.42251876523414e-09
3043 9.42140410131742e-09
3044 9.41965261347377e-09
3045 9.41850331059868e-09
3046 9.41732025694364e-09
3047 9.41598354842199e-09
3048 9.41474276316967e-09
3049 9.41347622074318e-09
3050 9.41233047058176e-09
3051 9.41141298227421e-09
3052 9.40989597353337e-09
3053 9.40862943110687e-09
3054 9.40753608347222e-09
3055 9.406299739112e-09
3056 9.40555988648839e-09
3057 9.40414857097949e-09
3058 9.40311828401263e-09
3059 9.40173539021316e-09
3060 9.40080635558616e-09
3061 9.39927335963375e-09
3062 9.39821909184957e-09
3063 9.39732025528883e-09
3064 9.39563538082666e-09
3065 9.39461131110875e-09
3066 9.39341848749109e-09
3067 9.39231092900172e-09
3068 9.39142807965254e-09
3069 9.39016686629657e-09
3070 9.38934618943676e-09
3071 9.38819244566957e-09
3072 9.386805110978e-09
3073 9.38607858103069e-09
3074 9.3851051374827e-09
3075 9.38388211579877e-09
3076 9.38289979046658e-09
3077 9.3820213820095e-09
3078 9.38068467348785e-09
3079 9.37959843128056e-09
3080 9.37842870030181e-09
3081 9.37747746121431e-09
3082 9.37673139134176e-09
3083 9.37528810140975e-09
3084 9.37439192938427e-09
3085 9.37374800002999e-09
3086 9.37242550236306e-09
3087 9.37141386714302e-09
3088 9.37033206582782e-09
3089 9.36944033469445e-09
3090 9.36839583687288e-09
3091 9.36774835480492e-09
3092 9.36647293059423e-09
3093 9.36587163380409e-09
3094 9.36466104661804e-09
3095 9.36369559667583e-09
3096 9.36264754614058e-09
3097 9.36155064579225e-09
3098 9.36050081890016e-09
3099 9.3595646788458e-09
3100 9.35878397001488e-09
3101 9.35768262877446e-09
3102 9.35661059742188e-09
3103 9.35562916026811e-09
3104 9.35493815745758e-09
3105 9.35373378752047e-09
3106 9.35304278470994e-09
3107 9.3521119737261e-09
3108 9.35079746966494e-09
3109 9.35027522075416e-09
3110 9.34892074866411e-09
3111 9.34816490882895e-09
3112 9.34692412357663e-09
3113 9.34608301861317e-09
3114 9.3452952043549e-09
3115 9.34432531352059e-09
3116 9.343397167072e-09
3117 9.34273991504142e-09
3118 9.34162613930312e-09
3119 9.34056743062683e-09
3120 9.33964816596244e-09
3121 9.33908683720119e-09
3122 9.3378922372267e-09
3123 9.33697652527599e-09
3124 9.33601462804745e-09
3125 9.33552968263029e-09
3126 9.33448696116557e-09
3127 9.33316623985547e-09
3128 9.33273192060824e-09
3129 9.33154620241794e-09
3130 9.33044574935593e-09
3131 9.32967392230921e-09
3132 9.32891719429563e-09
3133 9.3278744728309e-09
3134 9.32667809649956e-09
3135 9.32622246097026e-09
3136 9.32516108775872e-09
3137 9.32448251944606e-09
3138 9.32352239857437e-09
3139 9.32297439248941e-09
3140 9.32189081481738e-09
3141 9.32090937766361e-09
3142 9.31994570407824e-09
3143 9.31934796000178e-09
3144 9.31809207571632e-09
3145 9.31716037655406e-09
3146 9.31618426847081e-09
3147 9.31527566194745e-09
3148 9.31438215445723e-09
3149 9.31366006540202e-09
3150 9.31270616177926e-09
3151 9.31200183629244e-09
3152 9.31079124910639e-09
3153 9.30978494295687e-09
3154 9.3090877228974e-09
3155 9.30815957644882e-09
3156 9.30753341066293e-09
3157 9.30658661246753e-09
3158 9.30587873426703e-09
3159 9.30500387852362e-09
3160 9.30404286947351e-09
3161 9.30312538116596e-09
3162 9.30254451247947e-09
3163 9.30156041079044e-09
3164 9.30069887772333e-09
3165 9.29963661633337e-09
3166 9.29862586929175e-09
3167 9.29804677696211e-09
3168 9.29686372330707e-09
3169 9.29590715514905e-09
3170 9.29521082326801e-09
3171 9.29463706000888e-09
3172 9.2936369711083e-09
3173 9.29249122094689e-09
3174 9.29203114452548e-09
3175 9.29094223778293e-09
3176 9.29018284523409e-09
3177 9.28927601506757e-09
3178 9.28874488437259e-09
3179 9.2880512170268e-09
3180 9.28694365853744e-09
3181 9.28613097528341e-09
3182 9.28512999820441e-09
3183 9.28424270796313e-09
3184 9.28366805652558e-09
3185 9.28261467691982e-09
3186 9.28150711843045e-09
3187 9.28075394313055e-09
3188 9.27987109378137e-09
3189 9.2790157779632e-09
3190 9.27820220653075e-09
3191 9.27748811108131e-09
3192 9.27680954276866e-09
3193 9.27583965193435e-09
3194 9.27497012526146e-09
3195 9.27415655382902e-09
3196 9.27335186418077e-09
3197 9.27248944293524e-09
3198 9.27145382689787e-09
3199 9.27084897739405e-09
3200 9.27006826856314e-09
3201 9.26919074828447e-09
3202 9.26856191796332e-09
3203 9.26759113895059e-09
3204 9.26669674328195e-09
3205 9.26580057125648e-09
3206 9.26483867402794e-09
3207 9.26428711522931e-09
3208 9.26327370365243e-09
3209 9.26254095645618e-09
3210 9.26169185788694e-09
3211 9.26077081686572e-09
3212 9.26016774371874e-09
3213 9.25916410210448e-09
3214 9.25797838391418e-09
3215 9.25749876756754e-09
3216 9.25662302364572e-09
3217 9.25604659585133e-09
3218 9.25487242398049e-09
3219 9.25415388763895e-09
3220 9.2532408402235e-09
3221 9.25257204187346e-09
3222 9.25143961438835e-09
3223 9.25083742941979e-09
3224 9.25030008147587e-09
3225 9.24923782008591e-09
3226 9.24877419095083e-09
3227 9.24773324584294e-09
3228 9.24686549552689e-09
3229 9.2457570488591e-09
3230 9.24515486389055e-09
3231 9.2444629729016e-09
3232 9.2435703535898e-09
3233 9.24261200907495e-09
3234 9.24207110841735e-09
3235 9.24122289802654e-09
3236 9.24022103276911e-09
3237 9.23935150609623e-09
3238 9.23884169168332e-09
3239 9.23804943653295e-09
3240 9.23699339239192e-09
3241 9.23643383998751e-09
3242 9.23562559762559e-09
3243 9.23460152790767e-09
3244 9.23401000108015e-09
3245 9.23308718370208e-09
3246 9.23245924155935e-09
3247 9.23139165109887e-09
3248 9.23058074420169e-09
3249 9.22976006734189e-09
3250 9.22886567167325e-09
3251 9.22790288626629e-09
3252 9.22703158323657e-09
3253 9.22651111068262e-09
3254 9.22551457449572e-09
3255 9.22505805078799e-09
3256 9.22367693334536e-09
3257 9.2233136683717e-09
3258 9.22219189902762e-09
3259 9.22200271702422e-09
3260 9.2209644364516e-09
3261 9.22010290338449e-09
3262 9.21933906994354e-09
3263 9.2188097156054e-09
3264 9.21754672589259e-09
3265 9.21676956977535e-09
3266 9.21614162763262e-09
3267 9.21532361530808e-09
3268 9.21478537918574e-09
3269 9.21398513042959e-09
3270 9.21310228108041e-09
3271 9.21217147009656e-09
3272 9.21142540022402e-09
3273 9.21058873615266e-09
3274 9.20966503059617e-09
3275 9.20910636637018e-09
3276 9.20802190051973e-09
3277 9.20764531286977e-09
3278 9.20673315363274e-09
3279 9.20582277075255e-09
3280 9.20482978727932e-09
3281 9.20394693793014e-09
3282 9.20338472099047e-09
3283 9.20239529023092e-09
3284 9.20187392949856e-09
3285 9.20124865189109e-09
3286 9.20059051168209e-09
3287 9.1991569917127e-09
3288 9.19845621893955e-09
3289 9.1980298932981e-09
3290 9.19712039859633e-09
3291 9.19658038611715e-09
3292 9.19563269974333e-09
3293 9.194463856943e-09
3294 9.1940446367289e-09
3295 9.19324705250801e-09
3296 9.19236775587251e-09
3297 9.19165721313675e-09
3298 9.19090847872894e-09
3299 9.18999898402717e-09
3300 9.18944831340696e-09
3301 9.1885352659915e-09
3302 9.18753784162618e-09
3303 9.18679976535941e-09
3304 9.18617182321668e-09
3305 9.1858343154172e-09
3306 9.18459974741381e-09
3307 9.18364939650473e-09
3308 9.18318665554807e-09
3309 9.18230114166363e-09
3310 9.18159592799839e-09
3311 9.1806811042261e-09
3312 9.17998299598821e-09
3313 9.17911346931533e-09
3314 9.17870135452858e-09
3315 9.17734777061696e-09
3316 9.1765786081055e-09
3317 9.17636810982003e-09
3318 9.17543818701461e-09
3319 9.17453135684809e-09
3320 9.17389897381327e-09
3321 9.17272746647768e-09
3322 9.17186682158899e-09
3323 9.17161191438254e-09
3324 9.17047238147006e-09
3325 9.16987374921518e-09
3326 9.16919873361621e-09
3327 9.16795883654231e-09
3328 9.16751918822456e-09
3329 9.16663545069696e-09
3330 9.16589026900283e-09
3331 9.16532361117106e-09
3332 9.16443099185926e-09
3333 9.16361297953472e-09
3334 9.16293263486523e-09
3335 9.162183012279e-09
3336 9.16136322359762e-09
3337 9.1607761376622e-09
3338 9.15992703909296e-09
3339 9.15900599807173e-09
3340 9.15874487361634e-09
3341 9.15761777520174e-09
3342 9.1568672644371e-09
3343 9.15622067054755e-09
3344 9.15498432618733e-09
3345 9.15469833273619e-09
3346 9.15354103625532e-09
3347 9.15246989308116e-09
3348 9.15204090290445e-09
3349 9.15132059020607e-09
3350 9.15042619453743e-09
3351 9.14976361343633e-09
3352 9.14892872572182e-09
3353 9.14853703903873e-09
3354 9.14739572976941e-09
3355 9.14661768547376e-09
3356 9.14588405009908e-09
3357 9.14529785234208e-09
3358 9.14420628106427e-09
3359 9.14390163586631e-09
3360 9.14324971290625e-09
3361 9.14208708735487e-09
3362 9.14172026966753e-09
3363 9.1409839697576e-09
3364 9.14022990627927e-09
3365 9.13910636057835e-09
3366 9.13874131924786e-09
3367 9.13798281487743e-09
3368 9.13715059169817e-09
3369 9.13699249593947e-09
3370 9.13566644555885e-09
3371 9.13476405628444e-09
3372 9.13403042090977e-09
3373 9.13309339267698e-09
3374 9.13284470271947e-09
3375 9.13213415998371e-09
3376 9.13123088253087e-09
3377 9.13036224403641e-09
3378 9.12969699840005e-09
3379 9.12908326711204e-09
3380 9.1278371527892e-09
3381 9.1272243096796e-09
3382 9.12620468085379e-09
3383 9.12575615075184e-09
3384 9.12501452177139e-09
3385 9.12440967226757e-09
3386 9.12323194768305e-09
3387 9.12304543021492e-09
3388 9.1222691622761e-09
3389 9.1215657249677e-09
3390 9.12062159130755e-09
3391 9.12005226894053e-09
3392 9.11893227595328e-09
3393 9.11825370764063e-09
3394 9.117873567277e-09
3395 9.11685393845119e-09
3396 9.11625086530421e-09
3397 9.11549324911221e-09
3398 9.11461572883354e-09
3399 9.11384123725156e-09
3400 9.11292463712243e-09
3401 9.11229847133654e-09
3402 9.1117104972227e-09
3403 9.11102659983953e-09
3404 9.11032493888797e-09
3405 9.10944386589563e-09
3406 9.1085903264343e-09
3407 9.10784958563227e-09
3408 9.10716213553542e-09
3409 9.106503107148e-09
3410 9.10530495445983e-09
3411 9.10462638614717e-09
3412 9.10381192653631e-09
3413 9.10263775466547e-09
3414 9.10277631049894e-09
3415 9.1017193781795e-09
3416 9.10106390250576e-09
3417 9.10035691248368e-09
3418 9.09977160290509e-09
3419 9.09851038954912e-09
3420 9.0976799427267e-09
3421 9.0971390420691e-09
3422 9.09634767509715e-09
3423 9.09592046127727e-09
3424 9.09477471111586e-09
3425 9.09412989358316e-09
3426 9.09364317180916e-09
3427 9.09283226491198e-09
3428 9.09208974775311e-09
3429 9.09125308368175e-09
3430 9.09065978049739e-09
3431 9.08983022185339e-09
3432 9.08930797294261e-09
3433 9.08814623556964e-09
3434 9.08740194205393e-09
3435 9.0867287028118e-09
3436 9.08587338699363e-09
3437 9.08519215414572e-09
3438 9.08442387981268e-09
3439 9.08372044250427e-09
3440 9.08292641099706e-09
3441 9.08227093532332e-09
3442 9.08160924240065e-09
3443 9.08081432271501e-09
3444 9.08018105150177e-09
3445 9.0791862916717e-09
3446 9.07853170417638e-09
3447 9.07781672054853e-09
3448 9.07688502138626e-09
3449 9.0764586957448e-09
3450 9.07565400609656e-09
3451 9.07505715019852e-09
3452 9.07387143200822e-09
3453 9.07362984747806e-09
3454 9.07280917061826e-09
3455 9.07207464706516e-09
3456 9.07110120351717e-09
3457 9.07052299936595e-09
3458 9.06960462287998e-09
3459 9.06863384386725e-09
3460 9.06806629785706e-09
3461 9.06746144835324e-09
3462 9.06646402398792e-09
3463 9.06572061865063e-09
3464 9.06503316855378e-09
3465 9.06434927117061e-09
3466 9.06368757824794e-09
3467 9.06310759773987e-09
3468 9.06195829486478e-09
3469 9.06168740044677e-09
3470 9.06047059601178e-09
3471 9.05994124167364e-09
3472 9.05970232167874e-09
3473 9.05849528720637e-09
3474 9.05784958149525e-09
3475 9.0574010513933e-09
3476 9.05648889215627e-09
3477 9.05566466258279e-09
3478 9.05486174929138e-09
3479 9.05410146856411e-09
3480 9.05340602486149e-09
3481 9.05284469610024e-09
3482 9.05198493938997e-09
3483 9.05187569344434e-09
3484 9.05067842893459e-09
3485 9.0501064420323e-09
3486 9.04905839149706e-09
3487 9.04877772711643e-09
3488 9.04799613010709e-09
3489 9.04737174067805e-09
3490 9.04648178590151e-09
3491 9.04587338368401e-09
3492 9.04542574176048e-09
3493 9.04439900750731e-09
3494 9.04396024736798e-09
3495 9.04267771772993e-09
3496 9.04204089380301e-09
3497 9.04134989099248e-09
3498 9.04080188490752e-09
3499 9.03990482470363e-09
3500 9.03923513817517e-09
3501 9.03855834621936e-09
3502 9.03781849359575e-09
3503 9.03721453227035e-09
3504 9.03644625793731e-09
3505 9.0356566673222e-09
3506 9.03488039938338e-09
3507 9.03419472564337e-09
3508 9.03372576743777e-09
3509 9.03270702679038e-09
3510 9.03238994709454e-09
3511 9.03114738548538e-09
3512 9.03040842104019e-09
3513 9.02963570581505e-09
3514 9.02888697140725e-09
3515 9.02818353409884e-09
3516 9.02765329158228e-09
3517 9.02697117055595e-09
3518 9.02632901755851e-09
3519 9.02552432791026e-09
3520 9.02442920391877e-09
3521 9.02402863545149e-09
3522 9.02341401598505e-09
3523 9.02248054046595e-09
3524 9.02176644501651e-09
3525 9.02134900115925e-09
3526 9.02070151909129e-09
3527 9.01998653546343e-09
3528 9.01936569874806e-09
3529 9.01858676627398e-09
3530 9.01765684346856e-09
3531 9.01697116972855e-09
3532 9.01613184112193e-09
3533 9.01550301080078e-09
3534 9.01470187386622e-09
3535 9.0136955677167e-09
3536 9.01318841783905e-09
3537 9.0126182072936e-09
3538 9.0116669682061e-09
3539 9.01051322443891e-09
3540 9.01027430444401e-09
3541 9.00949448379151e-09
3542 9.00861962804811e-09
3543 9.00852281660036e-09
3544 9.0073388747669e-09
3545 9.00690455551967e-09
3546 9.00616825560974e-09
3547 9.00573216000566e-09
3548 9.00486707422488e-09
3549 9.0037186595282e-09
3550 9.00309071738548e-09
3551 9.00260843650358e-09
3552 9.00166341466502e-09
3553 9.00084451416205e-09
3554 9.00041730034218e-09
3555 8.99974406110005e-09
3556 8.99896779316123e-09
3557 8.99811603005674e-09
3558 8.99693386458011e-09
3559 8.99669760912047e-09
3560 8.99606078519355e-09
3561 8.9950757953261e-09
3562 8.99467611503724e-09
3563 8.99384211550114e-09
3564 8.99358987282994e-09
3565 8.99262175835247e-09
3566 8.99188457026412e-09
3567 8.99120689012989e-09
3568 8.99043683944001e-09
3569 8.98988439246295e-09
3570 8.98872798416051e-09
3571 8.98840823992941e-09
3572 8.98760621481642e-09
3573 8.98670649007727e-09
3574 8.98647911640182e-09
3575 8.9855447527043e-09
3576 8.98465568610618e-09
3577 8.98415830619115e-09
3578 8.98348506694902e-09
3579 8.98270080540442e-09
3580 8.98170160468226e-09
3581 8.98188456943672e-09
3582 8.9806890812838e-09
3583 8.97977869840361e-09
3584 8.97916407893717e-09
3585 8.97856633486072e-09
3586 8.97757868045801e-09
3587 8.97723229087433e-09
3588 8.97627749907315e-09
3589 8.9752250076458e-09
3590 8.97480134653961e-09
3591 8.9740526121318e-09
3592 8.97336605021337e-09
3593 8.97270258093386e-09
3594 8.97230112428815e-09
3595 8.9713383388812e-09
3596 8.97074858841052e-09
3597 8.97010288269939e-09
3598 8.96930352212166e-09
3599 8.96781227055499e-09
3600 8.96737883948617e-09
3601 8.96696317198575e-09
3602 8.96616825230012e-09
3603 8.96562024621517e-09
3604 8.96468499433922e-09
3605 8.96356056045988e-09
3606 8.96328611332819e-09
3607 8.96252139170883e-09
3608 8.96159946250918e-09
3609 8.96134721983799e-09
3610 8.96031426833588e-09
3611 8.95973162329256e-09
3612 8.95887986018806e-09
3613 8.95858942584482e-09
3614 8.95755736252113e-09
3615 8.95674734380236e-09
3616 8.95663543332148e-09
3617 8.95598084582616e-09
3618 8.95519214338947e-09
3619 8.95400820155601e-09
3620 8.953565888703e-09
3621 8.95277629808788e-09
3622 8.95238994047531e-09
3623 8.95160567893072e-09
3624 8.95094931507856e-09
3625 8.95021745606073e-09
3626 8.94933105399787e-09
3627 8.94855833877273e-09
3628 8.94784157878803e-09
3629 8.94720830757478e-09
3630 8.94663099160198e-09
3631 8.94649421212534e-09
3632 8.94483509483734e-09
3633 8.94421603447881e-09
3634 8.94406237961221e-09
3635 8.94291396491553e-09
3636 8.94217855318402e-09
3637 8.94157903275072e-09
3638 8.94121487959865e-09
3639 8.94037910370571e-09
3640 8.93937990298355e-09
3641 8.9386968937788e-09
3642 8.93825102821211e-09
3643 8.9370963962665e-09
3644 8.93680596192326e-09
3645 8.93621709963099e-09
3646 8.93538754098699e-09
3647 8.93481466590629e-09
3648 8.93404639157325e-09
3649 8.93328344631072e-09
3650 8.93270701851634e-09
3651 8.93201157481371e-09
3652 8.9313569873184e-09
3653 8.93033647031416e-09
3654 8.92990836831586e-09
3655 8.92909568506184e-09
3656 8.92840912314341e-09
3657 8.92784246531164e-09
3658 8.92695339871352e-09
3659 8.92625262594038e-09
3660 8.92557228127089e-09
3661 8.92456775147821e-09
3662 8.92417872933038e-09
3663 8.92326390555809e-09
3664 8.92273810393363e-09
3665 8.92183482648079e-09
3666 8.92116425177392e-09
3667 8.92048390710443e-09
3668 8.91962237403732e-09
3669 8.91921292378584e-09
3670 8.91865159502458e-09
3671 8.91750850939843e-09
3672 8.91704843297703e-09
3673 8.916163807271e-09
3674 8.91602081054543e-09
3675 8.91483242781987e-09
3676 8.9142879744486e-09
3677 8.91374618561258e-09
3678 8.91264839708583e-09
3679 8.91219276155653e-09
3680 8.91146623160921e-09
3681 8.91085960574856e-09
3682 8.91020057736114e-09
3683 8.90933815611561e-09
3684 8.90892781768571e-09
3685 8.90806362008334e-09
3686 8.90709195289219e-09
3687 8.90671447706382e-09
3688 8.90602080971803e-09
3689 8.90515039486672e-09
3690 8.90470008840794e-09
3691 8.90369200590158e-09
3692 8.90282514376395e-09
3693 8.90250095864076e-09
3694 8.90191476088376e-09
3695 8.90126550245895e-09
3696 8.90064733027884e-09
3697 8.89984175245218e-09
3698 8.89911522250486e-09
3699 8.89819506966205e-09
3700 8.89756446298406e-09
3701 8.89682105764678e-09
3702 8.89647999713361e-09
3703 8.89514328861196e-09
3704 8.89511131418885e-09
3705 8.89415208149558e-09
3706 8.89367335332736e-09
3707 8.8924174690419e-09
3708 8.89209417209713e-09
3709 8.89127704795101e-09
3710 8.89073792365025e-09
3711 8.8899918537777e-09
3712 8.88932660814135e-09
3713 8.88805651300117e-09
3714 8.88825901768087e-09
3715 8.88771101159591e-09
3716 8.88718787450671e-09
3717 8.88599149817537e-09
3718 8.8852578628007e-09
3719 8.88491324957386e-09
3720 8.88400286669366e-09
3721 8.88360673911848e-09
3722 8.88282958300124e-09
3723 8.88202222881773e-09
3724 8.88136408860873e-09
3725 8.88050699643372e-09
3726 8.87973339303016e-09
3727 8.87953888195625e-09
3728 8.87874218591378e-09
3729 8.87774298519162e-09
3730 8.87716211650513e-09
3731 8.87670026372689e-09
3732 8.87579254538196e-09
3733 8.87544615579827e-09
3734 8.87467521692997e-09
3735 8.87374262958929e-09
3736 8.87293527540578e-09
3737 8.87264928195464e-09
3738 8.87198225996144e-09
3739 8.87102213908975e-09
3740 8.87069262489604e-09
3741 8.86971474045595e-09
3742 8.86934703459019e-09
3743 8.86822348888927e-09
3744 8.86789219833872e-09
3745 8.86729800697594e-09
3746 8.86640716402098e-09
3747 8.86585116433025e-09
3748 8.8654203977967e-09
3749 8.86485018725125e-09
3750 8.86367423902357e-09
3751 8.86386786191906e-09
3752 8.86256934506946e-09
3753 8.8624538818749e-09
3754 8.86154971624364e-09
3755 8.86043505232692e-09
3756 8.85979289932948e-09
3757 8.85952822216041e-09
3758 8.85850059972881e-09
3759 8.85755380153341e-09
3760 8.85725004451388e-09
3761 8.85654483084863e-09
3762 8.85548878670761e-09
3763 8.85496476143999e-09
3764 8.85430484487415e-09
3765 8.85358897306787e-09
3766 8.85312001486227e-09
3767 8.85222029012311e-09
3768 8.85148310203476e-09
3769 8.85096529401608e-09
3770 8.85043061060742e-09
3771 8.84971118608746e-09
3772 8.84886919294559e-09
3773 8.84848816440353e-09
3774 8.8474676473993e-09
3775 8.84679085544349e-09
3776 8.84628548192268e-09
3777 8.84547457502549e-09
3778 8.84495854336365e-09
3779 8.84399398159985e-09
3780 8.84300455084031e-09
3781 8.84249828914108e-09
3782 8.84190232142146e-09
3783 8.84107276277746e-09
3784 8.8409981557902e-09
3785 8.83996520428809e-09
3786 8.83964013098648e-09
3787 8.83861517309015e-09
3788 8.83810180596356e-09
3789 8.83787887318022e-09
3790 8.83683082264497e-09
3791 8.83599859946571e-09
3792 8.83537687457192e-09
3793 8.83463524559147e-09
3794 8.83385009586846e-09
3795 8.83348416635954e-09
3796 8.83248230110212e-09
3797 8.83181616728734e-09
3798 8.83137563079117e-09
3799 8.83067752255329e-09
3800 8.8302858358702e-09
3801 8.82950779157454e-09
3802 8.82848549821347e-09
3803 8.82771455934517e-09
3804 8.82720829764594e-09
3805 8.82676243207925e-09
3806 8.82594441975471e-09
3807 8.82505180044291e-09
3808 8.8244114238023e-09
3809 8.82419382008948e-09
3810 8.82314754591107e-09
3811 8.82224782117191e-09
3812 8.82164208348968e-09
3813 8.82114292721781e-09
3814 8.82026984783124e-09
3815 8.82005490865367e-09
3816 8.81902106897314e-09
3817 8.8186853375305e-09
3818 8.81798456475735e-09
3819 8.81728556834105e-09
3820 8.81632100657725e-09
3821 8.81571082800292e-09
3822 8.81518413820004e-09
3823 8.81466100111084e-09
3824 8.81376127637168e-09
3825 8.81276918107687e-09
3826 8.81243789052633e-09
3827 8.81148665143883e-09
3828 8.81111628103781e-09
3829 8.81091111182286e-09
3830 8.80974670991463e-09
3831 8.80884698517548e-09
3832 8.80840378414405e-09
3833 8.80803519009987e-09
3834 8.80698980409989e-09
3835 8.80680950388069e-09
3836 8.80588935103788e-09
3837 8.80526762614409e-09
3838 8.80445938378216e-09
3839 8.80366357591811e-09
3840 8.80302408745592e-09
3841 8.80209771736418e-09
3842 8.80184902740666e-09
3843 8.80094042088331e-09
3844 8.80011263859615e-09
3845 8.79978490075928e-09
3846 8.79883543802862e-09
3847 8.79838069067773e-09
3848 8.79778205842285e-09
3849 8.79701467226823e-09
3850 8.7960527750397e-09
3851 8.79552253252314e-09
3852 8.79481731885789e-09
3853 8.79431993894286e-09
3854 8.79373462936428e-09
3855 8.79298855949173e-09
3856 8.79231265571434e-09
3857 8.79173711609837e-09
3858 8.79141026643993e-09
3859 8.79069439463365e-09
3860 8.78947581384182e-09
3861 8.78915074054021e-09
3862 8.78865069608992e-09
3863 8.7878655463669e-09
3864 8.78732020481721e-09
3865 8.78628458877984e-09
3866 8.78562378403558e-09
3867 8.78525163727772e-09
3868 8.78470807208487e-09
3869 8.78383143998462e-09
3870 8.78305339568897e-09
3871 8.78273365145787e-09
3872 8.78173977980623e-09
3873 8.78137118576205e-09
3874 8.78066863663207e-09
3875 8.7801783621444e-09
3876 8.77929107190312e-09
3877 8.77876438210023e-09
3878 8.77866046522513e-09
3879 8.77739037008496e-09
3880 8.7771399037706e-09
3881 8.77604566795753e-09
3882 8.7755394062583e-09
3883 8.77461392434498e-09
3884 8.77396733045543e-09
3885 8.7736635734359e-09
3886 8.77266881360583e-09
3887 8.7725409159134e-09
3888 8.7712335172796e-09
3889 8.77066952398309e-09
3890 8.77003270005616e-09
3891 8.76926087300944e-09
3892 8.76840999808337e-09
3893 8.76812045191855e-09
3894 8.76758399215305e-09
3895 8.76722250353623e-09
3896 8.76618688749886e-09
3897 8.765228542984e-09
3898 8.76457306731027e-09
3899 8.76396377691435e-09
3900 8.76343264621937e-09
3901 8.76283845485659e-09
3902 8.76182504327971e-09
3903 8.76130190619051e-09
3904 8.76088090961957e-09
3905 8.76028316554311e-09
3906 8.75962946622622e-09
3907 8.75900418861875e-09
3908 8.75832917301977e-09
3909 8.75787442566889e-09
3910 8.75670913558224e-09
3911 8.75606964712006e-09
3912 8.75566641411751e-09
3913 8.75527739196968e-09
3914 8.75426309221439e-09
3915 8.75371330977259e-09
3916 8.75284111856445e-09
3917 8.75217587292809e-09
3918 8.75200178995783e-09
3919 8.75098571384569e-09
3920 8.75030004010569e-09
3921 8.74985861543109e-09
3922 8.74904948489075e-09
3923 8.74846506349058e-09
3924 8.74840111464437e-09
3925 8.74707062337166e-09
3926 8.74646577386784e-09
3927 8.74578631737677e-09
3928 8.74530492467329e-09
3929 8.74459971100805e-09
3930 8.74368843994944e-09
3931 8.74323458077697e-09
3932 8.74258088146007e-09
3933 8.74180372534283e-09
3934 8.74107897175236e-09
3935 8.74046968135644e-09
3936 8.73990479988151e-09
3937 8.73910543930378e-09
3938 8.7382447944151e-09
3939 8.73764705033864e-09
3940 8.73708572157739e-09
3941 8.7367393319937e-09
3942 8.73560868086543e-09
3943 8.7353324573769e-09
3944 8.73448868787818e-09
3945 8.73390515465644e-09
3946 8.73327987704897e-09
3947 8.73284999869384e-09
3948 8.73234817788671e-09
3949 8.73116245969641e-09
3950 8.73086491992581e-09
3951 8.73008421109489e-09
3952 8.72966499088079e-09
3953 8.72914007743475e-09
3954 8.72786287686722e-09
3955 8.72742855761999e-09
3956 8.72722960565397e-09
3957 8.72597549772536e-09
3958 8.72528627127167e-09
3959 8.72490613090804e-09
3960 8.72465300005842e-09
3961 8.72403838059199e-09
3962 8.72307648336346e-09
3963 8.72223449022158e-09
3964 8.72172467580867e-09
3965 8.72114025440851e-09
3966 8.72037286825389e-09
3967 8.72004068952492e-09
3968 8.71891447928874e-09
3969 8.71864980211967e-09
3970 8.71816308034568e-09
3971 8.71707950267364e-09
3972 8.71640182253941e-09
3973 8.71586269823865e-09
3974 8.71535732471784e-09
3975 8.71464056473314e-09
3976 8.7138989357527e-09
3977 8.71377103806026e-09
3978 8.71272654023869e-09
3979 8.71210215080964e-09
3980 8.71152394665842e-09
3981 8.71064109730924e-09
3982 8.71011529568477e-09
3983 8.70920757733984e-09
3984 8.70907257422004e-09
3985 8.70838778865846e-09
3986 8.7075511245871e-09
3987 8.70684058185134e-09
3988 8.70648353412662e-09
3989 8.70573391154039e-09
3990 8.70461747126683e-09
3991 8.70443006562027e-09
3992 8.70388916496267e-09
3993 8.7031146733807e-09
3994 8.70227090388198e-09
3995 8.70162963906296e-09
3996 8.70127170315982e-09
3997 8.70036753752856e-09
3998 8.69962413219127e-09
3999 8.69927774260759e-09
4000 8.69875727005365e-09
4001 8.69821725757447e-09
4002 8.69700400585316e-09
4003 8.69674643411145e-09
4004 8.69589467100695e-09
4005 8.69513527845811e-09
4006 8.6947062882814e-09
4007 8.69410410331284e-09
4008 8.6930471709934e-09
4009 8.6931057907691e-09
4010 8.69192362529247e-09
4011 8.69151417504099e-09
4012 8.69053451424406e-09
4013 8.69017391380567e-09
4014 8.68938521136897e-09
4015 8.68893668126702e-09
4016 8.68830074551852e-09
4017 8.68763372352532e-09
4018 8.68693117439534e-09
4019 8.68621974348116e-09
4020 8.68607941129085e-09
4021 8.68524274721949e-09
4022 8.6841929203274e-09
4023 8.6839087032331e-09
4024 8.68326299752198e-09
4025 8.68254623753728e-09
4026 8.68209237836481e-09
4027 8.68107896678794e-09
4028 8.68075211712949e-09
4029 8.68020944011505e-09
4030 8.67911520430198e-09
4031 8.67829985651269e-09
4032 8.67812932625611e-09
4033 8.67726157594007e-09
4034 8.67665317372257e-09
4035 8.67600213894093e-09
4036 8.67496741108198e-09
4037 8.67459082343203e-09
4038 8.67417071503951e-09
4039 8.67341931609644e-09
4040 8.67291927164615e-09
4041 8.67197424980759e-09
4042 8.67162075479655e-09
4043 8.67098126633437e-09
4044 8.67008420613047e-09
4045 8.66996430204381e-09
4046 8.66931060272691e-09
4047 8.66836913360203e-09
4048 8.66742944083398e-09
4049 8.66701199697673e-09
4050 8.66641869379237e-09
4051 8.66603588889348e-09
4052 8.66541771671336e-09
4053 8.6646032571025e-09
4054 8.66387850351202e-09
4055 8.66347438233106e-09
4056 8.66215366102097e-09
4057 8.66199467708384e-09
4058 8.66164029389438e-09
4059 8.66077964900569e-09
4060 8.66018545764291e-09
4061 8.65947313855031e-09
4062 8.65860450005584e-09
4063 8.6580707048256e-09
4064 8.65738414290718e-09
4065 8.65677307615442e-09
4066 8.65608118516548e-09
4067 8.65573124286811e-09
4068 8.65487947976362e-09
4069 8.65440430430908e-09
4070 8.65352323131674e-09
4071 8.65346638789788e-09
4072 8.6523312958775e-09
4073 8.65192451016128e-09
4074 8.65127791627174e-09
4075 8.65063753963113e-09
4076 8.64999361027685e-09
4077 8.64927418575689e-09
4078 8.64862315097525e-09
4079 8.64816218637543e-09
4080 8.64749427620382e-09
4081 8.64657145882575e-09
4082 8.64628635355302e-09
4083 8.64591687133043e-09
4084 8.64504645647912e-09
4085 8.64425686586401e-09
4086 8.64391491717242e-09
4087 8.64317240001355e-09
4088 8.64241123110787e-09
4089 8.64203286710108e-09
4090 8.64139781953099e-09
4091 8.64051674653865e-09
4092 8.64011884260663e-09
4093 8.63939941808667e-09
4094 8.63866400635516e-09
4095 8.63786375759901e-09
4096 8.6373450614019e-09
4097 8.63703597531185e-09
4098 8.63622062752256e-09
4099 8.63549498575367e-09
4100 8.6353839634512e-09
4101 8.6344504879321e-09
4102 8.63360050118445e-09
4103 8.63293614372651e-09
4104 8.63231441883272e-09
4105 8.6318490133408e-09
4106 8.63110383164667e-09
4107 8.63053450927964e-09
4108 8.63007976192876e-09
4109 8.62910543020234e-09
4110 8.62894378172996e-09
4111 8.62827143066625e-09
4112 8.62702620452183e-09
4113 8.62676685642327e-09
4114 8.62613624974529e-09
4115 8.6253386655244e-09
4116 8.62480931118625e-09
4117 8.62460236561446e-09
4118 8.62372129262212e-09
4119 8.6230889095873e-09
4120 8.62246540833667e-09
4121 8.62180016270031e-09
4122 8.62122817579802e-09
4123 8.62072102592037e-09
4124 8.61997939693993e-09
4125 8.61942162089235e-09
4126 8.61891091830103e-09
4127 8.61804316798498e-09
4128 8.6177465163928e-09
4129 8.61699067655763e-09
4130 8.61646309857633e-09
4131 8.61557314379979e-09
4132 8.61512461369784e-09
4133 8.61423732345656e-09
4134 8.61329141343958e-09
4135 8.61270432750416e-09
4136 8.61256932438437e-09
4137 8.61197957391369e-09
4138 8.61102122939883e-09
4139 8.61041016264608e-09
4140 8.61018367714905e-09
4141 8.60959215032153e-09
4142 8.60901749888399e-09
4143 8.6080262917676e-09
4144 8.60739390873277e-09
4145 8.60687610071409e-09
4146 8.60610960273789e-09
4147 8.60610693820263e-09
4148 8.60515481093671e-09
4149 8.60432791682797e-09
4150 8.60395665824853e-09
4151 8.60320881201915e-09
4152 8.60281712533606e-09
4153 8.60203641650514e-09
4154 8.60139515168612e-09
4155 8.60093329890788e-09
4156 8.60000959335139e-09
4157 8.59918980467e-09
4158 8.59857252066831e-09
4159 8.59849613732422e-09
4160 8.59763105154343e-09
4161 8.59736015712542e-09
4162 8.59625171045764e-09
4163 8.59565574273802e-09
4164 8.59537507835739e-09
4165 8.59421511734126e-09
4166 8.59372573103201e-09
4167 8.59279847276184e-09
4168 8.5923774761909e-09
4169 8.59209148273976e-09
4170 8.59123261420791e-09
4171 8.59042348366756e-09
4172 8.59002202702186e-09
4173 8.58886117782731e-09
4174 8.58877768905586e-09
4175 8.5880582645359e-09
4176 8.58711324269734e-09
4177 8.58648796508987e-09
4178 8.58633519840168e-09
4179 8.58534487946372e-09
4180 8.58470716735837e-09
4181 8.58407833703723e-09
4182 8.58378257362347e-09
4183 8.58289261884693e-09
4184 8.58240323253767e-09
4185 8.5818996353737e-09
4186 8.58097592981721e-09
4187 8.58033111228451e-09
4188 8.57993587288775e-09
4189 8.57924220554196e-09
4190 8.57869775217068e-09
4191 8.57830073641708e-09
4192 8.57727311398548e-09
4193 8.57703597034742e-09
4194 8.57615756189034e-09
4195 8.57566639922425e-09
4196 8.57501447626419e-09
4197 8.57441939672299e-09
4198 8.57406590171195e-09
4199 8.57340864968137e-09
4200 8.57278426025232e-09
4201 8.57193072079099e-09
4202 8.5711331365701e-09
4203 8.57073345628123e-09
4204 8.5702813734656e-09
4205 8.56976889451744e-09
4206 8.56912141244948e-09
4207 8.5689375595166e-09
4208 8.56793747061602e-09
4209 8.56701998230847e-09
4210 8.5663360849253e-09
4211 8.56582538233397e-09
4212 8.56474979826771e-09
4213 8.56468940213517e-09
4214 8.56420534489644e-09
4215 8.56358184364581e-09
4216 8.56261817006043e-09
4217 8.5621705281369e-09
4218 8.56130721871295e-09
4219 8.560930631063e-09
4220 8.5601641330868e-09
4221 8.55959569889819e-09
4222 8.55902015928223e-09
4223 8.55837711810636e-09
4224 8.55761417284384e-09
4225 8.5573779173842e-09
4226 8.55603410343519e-09
4227 8.55630322149636e-09
4228 8.55541593125508e-09
4229 8.554537522798e-09
4230 8.55416182332647e-09
4231 8.55344417516335e-09
4232 8.55296988788723e-09
4233 8.55241300001808e-09
4234 8.55151061074366e-09
4235 8.55130544152871e-09
4236 8.5502769309187e-09
4237 8.55009840705634e-09
4238 8.54957082907504e-09
4239 8.54900061852959e-09
4240 8.54854498300028e-09
4241 8.54747561618296e-09
4242 8.54674730987881e-09
4243 8.54634230051943e-09
4244 8.54561665875053e-09
4245 8.54502779645827e-09
4246 8.54425419305471e-09
4247 8.54338555456025e-09
4248 8.54344950340646e-09
4249 8.54249382342687e-09
4250 8.54176995801481e-09
4251 8.54162607311082e-09
4252 8.54070858480327e-09
4253 8.54004866823743e-09
4254 8.53895354424594e-09
4255 8.53849702053822e-09
4256 8.5381222092451e-09
4257 8.53762305297323e-09
4258 8.53650572452125e-09
4259 8.53606696438192e-09
4260 8.53535020439722e-09
4261 8.53458459459944e-09
4262 8.53457304827998e-09
4263 8.53360759833777e-09
4264 8.53296633351874e-09
4265 8.53217940743889e-09
4266 8.53211989948477e-09
4267 8.5313773823259e-09
4268 8.53049630933356e-09
4269 8.53005577283739e-09
4270 8.52957526831233e-09
4271 8.52908588200307e-09
4272 8.52830783770742e-09
4273 8.52791082195381e-09
4274 8.52702441989095e-09
4275 8.52671178108722e-09
4276 8.52610515522656e-09
4277 8.52557668906684e-09
4278 8.52512194171595e-09
4279 8.52464232536931e-09
4280 8.52395576345089e-09
4281 8.52286330399465e-09
4282 8.52240233939483e-09
4283 8.52180193078311e-09
4284 8.52076276203206e-09
4285 8.52036485810004e-09
4286 8.52011172725042e-09
4287 8.51906367671518e-09
4288 8.5187714660151e-09
4289 8.51794101919268e-09
4290 8.51729176076788e-09
4291 8.51692583125896e-09
4292 8.51604919915872e-09
4293 8.51546744229381e-09
4294 8.51496473330826e-09
4295 8.5142586314646e-09
4296 8.51346282360055e-09
4297 8.51280734792681e-09
4298 8.51209147612053e-09
4299 8.51179571270677e-09
4300 8.51132408996591e-09
4301 8.51058334916388e-09
4302 8.50998294055216e-09
4303 8.50914361194555e-09
4304 8.50905657046042e-09
4305 8.50804848795406e-09
4306 8.50741788127607e-09
4307 8.50683257169749e-09
4308 8.50608650182494e-09
4309 8.50598969037719e-09
4310 8.50502601679182e-09
4311 8.50449666245368e-09
4312 8.50396197904502e-09
4313 8.5031945928904e-09
4314 8.5022797691181e-09
4315 8.50210035707732e-09
4316 8.50150705389296e-09
4317 8.50102477301107e-09
4318 8.50048742506715e-09
4319 8.49942516367719e-09
4320 8.49895975818526e-09
4321 8.49856185425324e-09
4322 8.49794457025155e-09
4323 8.49698267302301e-09
4324 8.49668957414451e-09
4325 8.49596926144613e-09
4326 8.49565751082082e-09
4327 8.49449843798311e-09
4328 8.49404280245381e-09
4329 8.49318215756512e-09
4330 8.4927398447121e-09
4331 8.49191117424652e-09
4332 8.49199643937482e-09
4333 8.49148129589139e-09
4334 8.49117398615817e-09
4335 8.49020675985912e-09
4336 8.48959302857111e-09
4337 8.48887626858641e-09
4338 8.48819148302482e-09
4339 8.48762393701463e-09
4340 8.48663894714718e-09
4341 8.48632009109451e-09
4342 8.48567083266971e-09
4343 8.48542303089062e-09
4344 8.4847506798269e-09
4345 8.48382608609199e-09
4346 8.48384829055249e-09
4347 8.48277181830781e-09
4348 8.48219805504868e-09
4349 8.48122727603595e-09
4350 8.48120418339704e-09
4351 8.48051495694335e-09
4352 8.48020853538856e-09
4353 8.47900860634354e-09
4354 8.47898107281253e-09
4355 8.47828829364516e-09
4356 8.47771097767236e-09
4357 8.47686631999522e-09
4358 8.4760536367412e-09
4359 8.47535552850331e-09
4360 8.47480841059678e-09
4361 8.47420444927138e-09
4362 8.47357739530707e-09
4363 8.47338199605474e-09
4364 8.47261194536486e-09
4365 8.47157988204117e-09
4366 8.47125303238272e-09
4367 8.47078496235554e-09
4368 8.47019343552802e-09
4369 8.46958769784578e-09
4370 8.46901571094349e-09
4371 8.46863823511512e-09
4372 8.46725711767249e-09
4373 8.4671043509843e-09
4374 8.46660430653401e-09
4375 8.46613712468525e-09
4376 8.46567171919332e-09
4377 8.46486258865298e-09
4378 8.46416625677193e-09
4379 8.46384562436242e-09
4380 8.46313863434034e-09
4381 8.46271852594782e-09
4382 8.46162695467001e-09
4383 8.46176817503874e-09
4384 8.46056469328005e-09
4385 8.46007086607869e-09
4386 8.45944025940071e-09
4387 8.45897307755195e-09
4388 8.45819503325629e-09
4389 8.45763015178136e-09
4390 8.45724912323931e-09
4391 8.45637160296064e-09
4392 8.45590797382556e-09
4393 8.45506509250527e-09
4394 8.45420711215183e-09
4395 8.4537141731289e-09
4396 8.45331626919688e-09
4397 8.45271408422832e-09
4398 8.45246184155712e-09
4399 8.45143244276869e-09
4400 8.45088443668374e-09
4401 8.45054159981373e-09
4402 8.45015613037958e-09
4403 8.44901215657501e-09
4404 8.44844905145692e-09
4405 8.44787884091147e-09
4406 8.44705727587325e-09
4407 8.44660252852236e-09
4408 8.44588221582399e-09
4409 8.44525782639494e-09
4410 8.4449407466991e-09
4411 8.44423020396334e-09
4412 8.44385539267023e-09
4413 8.4432993929795e-09
4414 8.44239078645614e-09
4415 8.44192182825054e-09
4416 8.441446652796e-09
4417 8.44055225712737e-09
4418 8.44032221891666e-09
4419 8.43941627692857e-09
4420 8.43915692883002e-09
4421 8.43864977895237e-09
4422 8.43776160053267e-09
4423 8.43677483430838e-09
4424 8.4367623998105e-09
4425 8.43620995283345e-09
4426 8.43577119269412e-09
4427 8.43532621530585e-09
4428 8.43431191555055e-09
4429 8.43339265088616e-09
4430 8.43299829966782e-09
4431 8.43259151395159e-09
4432 8.4321678528454e-09
4433 8.43129654981567e-09
4434 8.43065262046139e-09
4435 8.42963032710031e-09
4436 8.42948466583948e-09
4437 8.42893221886243e-09
4438 8.42805558676218e-09
4439 8.42790814914451e-09
4440 8.42740455198054e-09
4441 8.42685476953875e-09
4442 8.42594971572908e-09
4443 8.42535197165262e-09
4444 8.42483593999077e-09
4445 8.42375413867558e-09
4446 8.42322567251585e-09
4447 8.42280112323124e-09
4448 8.42212610763227e-09
4449 8.42168024206558e-09
4450 8.42063396788717e-09
4451 8.42061975703245e-09
4452 8.41975289489483e-09
4453 8.41880698487785e-09
4454 8.418462371651e-09
4455 8.41806180318372e-09
4456 8.41708569510047e-09
4457 8.41694003383964e-09
4458 8.41623659653123e-09
4459 8.41565839238001e-09
4460 8.41535552353889e-09
4461 8.41440783716507e-09
4462 8.41430747300365e-09
4463 8.41370262349983e-09
4464 8.41270075824241e-09
4465 8.41237302040554e-09
4466 8.41168557030869e-09
4467 8.41126368555933e-09
4468 8.41075209478959e-09
4469 8.40987102179724e-09
4470 8.40926350775817e-09
4471 8.40868619178536e-09
4472 8.40820568726031e-09
4473 8.40784775135717e-09
4474 8.40725977724333e-09
4475 8.40638758603518e-09
4476 8.40562286441582e-09
4477 8.40547098590605e-09
4478 8.40468494800461e-09
4479 8.40380209865543e-09
4480 8.40355607323318e-09
4481 8.40274605451441e-09
4482 8.40218650211e-09
4483 8.40164471327398e-09
4484 8.40121838763253e-09
4485 8.40059399820348e-09
4486 8.40013125724681e-09
4487 8.39876079794522e-09
4488 8.39898195437172e-09
4489 8.39778024896987e-09
4490 8.39724911827489e-09
4491 8.39652525286283e-09
4492 8.39629166193845e-09
4493 8.39571789867932e-09
4494 8.39525338136582e-09
4495 8.39453928591638e-09
4496 8.39373104355445e-09
4497 8.39339620029023e-09
4498 8.39266078855871e-09
4499 8.39218383674734e-09
4500 8.39146885311948e-09
4501 8.39131786278813e-09
4502 8.39041103262161e-09
4503 8.38987990192663e-09
4504 8.38919334000821e-09
4505 8.38842062478307e-09
4506 8.3882101264976e-09
4507 8.38771185840415e-09
4508 8.38673841485615e-09
4509 8.38612557174656e-09
4510 8.38576141859448e-09
4511 8.38482261400486e-09
4512 8.38441049921812e-09
4513 8.38398506175508e-09
4514 8.38319369478313e-09
4515 8.38261104973981e-09
4516 8.38202662833964e-09
4517 8.38157721005928e-09
4518 8.3810371975801e-09
4519 8.38023961335921e-09
4520 8.37970226541529e-09
4521 8.379202220965e-09
4522 8.37882296877979e-09
4523 8.37798985742211e-09
4524 8.37791169772117e-09
4525 8.37674818399137e-09
4526 8.37675262488347e-09
4527 8.37568059353089e-09
4528 8.37534575026666e-09
4529 8.37487235116896e-09
4530 8.3744460255275e-09
4531 8.37338909320806e-09
4532 8.37306757262013e-09
4533 8.37254976460144e-09
4534 8.37158609101607e-09
4535 8.37137470455218e-09
4536 8.37063840464225e-09
4537 8.37034797029901e-09
4538 8.36942692927778e-09
4539 8.36882652066606e-09
4540 8.36830338357686e-09
4541 8.36758307087848e-09
4542 8.36709102003397e-09
4543 8.36647551238912e-09
4544 8.36676150584026e-09
4545 8.36540525739338e-09
4546 8.36468760923026e-09
4547 8.36459523867461e-09
4548 8.36409697058116e-09
4549 8.36307201268482e-09
4550 8.3625568692014e-09
4551 8.36180902297201e-09
4552 8.36156388572817e-09
4553 8.36109315116573e-09
4554 8.36008151594569e-09
4555 8.35919511388283e-09
4556 8.35918800845548e-09
4557 8.35842417501453e-09
4558 8.35820390676645e-09
4559 8.35739655258294e-09
4560 8.35663893639094e-09
4561 8.35641422725075e-09
4562 8.35548341626691e-09
4563 8.35500646445553e-09
4564 8.35468583204602e-09
4565 8.35402591548018e-09
4566 8.35358449080559e-09
4567 8.35281888100781e-09
4568 8.35265456800016e-09
4569 8.35157276668497e-09
4570 8.35125035791862e-09
4571 8.35079916328141e-09
4572 8.3496853875431e-09
4573 8.34974933638932e-09
4574 8.34870039767566e-09
4575 8.34798630222622e-09
4576 8.34763103085834e-09
4577 8.34713986819224e-09
4578 8.34669933169607e-09
4579 8.34607227773176e-09
4580 8.34565572205292e-09
4581 8.34519742198836e-09
4582 8.34420976758565e-09
4583 8.3437612374837e-09
4584 8.34291924434183e-09
4585 8.34265545535118e-09
4586 8.34208968569783e-09
4587 8.34118640824499e-09
4588 8.34092350743276e-09
4589 8.34011437689242e-09
4590 8.33941893318979e-09
4591 8.33913205156023e-09
4592 8.33893309959421e-09
4593 8.33775537500969e-09
4594 8.33745339434699e-09
4595 8.33703150959764e-09
4596 8.33624280716094e-09
4597 8.33531732524762e-09
4598 8.33517788123572e-09
4599 8.33446112125102e-09
4600 8.33389091070558e-09
4601 8.33346636142096e-09
4602 8.33274427236574e-09
4603 8.33167312919159e-09
4604 8.33196356353483e-09
4605 8.3309208420701e-09
4606 8.33054070170647e-09
4607 8.32987900878379e-09
4608 8.32942426143291e-09
4609 8.3287750030081e-09
4610 8.32838864539553e-09
4611 8.32788327187473e-09
4612 8.3271380901806e-09
4613 8.32668423100813e-09
4614 8.3257045702112e-09
4615 8.32506330539218e-09
4616 8.32468494138539e-09
4617 8.32363156177962e-09
4618 8.323350897399e-09
4619 8.32313329368617e-09
4620 8.32245561355194e-09
4621 8.32173441267514e-09
4622 8.32089686042536e-09
4623 8.32032931441518e-09
4624 8.31975643933447e-09
4625 8.31931767919514e-09
4626 8.3187750021807e-09
4627 8.31832824843559e-09
4628 8.31789215283152e-09
4629 8.31676327806008e-09
4630 8.3161690866973e-09
4631 8.31578361726315e-09
4632 8.31554558544667e-09
4633 8.31479951557412e-09
4634 8.3142213114229e-09
4635 8.31372126697261e-09
4636 8.31305690951467e-09
4637 8.31253466060389e-09
4638 8.31168733839149e-09
4639 8.31165802850364e-09
4640 8.31064017603467e-09
4641 8.31004420831505e-09
4642 8.30964363984776e-09
4643 8.30936031093188e-09
4644 8.30843394084013e-09
4645 8.3078779411494e-09
4646 8.30768698278916e-09
4647 8.30678992258527e-09
4648 8.30618684943829e-09
4649 8.30594970580023e-09
4650 8.30537238982743e-09
4651 8.30456503564392e-09
4652 8.3040365694842e-09
4653 8.30351520875183e-09
4654 8.30300539433892e-09
4655 8.30237656401778e-09
4656 8.30204349711039e-09
4657 8.30090662873317e-09
4658 8.30068902502035e-09
4659 8.30019875053267e-09
4660 8.29936208646131e-09
4661 8.29851476424892e-09
4662 8.29818436187679e-09
4663 8.29730240070603e-09
4664 8.29740898211639e-09
4665 8.29651813916144e-09
4666 8.29605806274003e-09
4667 8.29564150706119e-09
4668 8.29462543094905e-09
4669 8.29443091987514e-09
4670 8.29379764866189e-09
4671 8.29342194919036e-09
4672 8.29269186652937e-09
4673 8.29197510654467e-09
4674 8.29163582238834e-09
4675 8.29149460201961e-09
4676 8.29013391268063e-09
4677 8.29008328651071e-09
4678 8.28922708251412e-09
4679 8.28903612415388e-09
4680 8.28841084654641e-09
4681 8.28732193980386e-09
4682 8.28699242561015e-09
4683 8.28655455364924e-09
4684 8.28568591515477e-09
4685 8.28488566639862e-09
4686 8.28481727666031e-09
4687 8.28478352588036e-09
4688 8.28360136040374e-09
4689 8.28315549483705e-09
4690 8.28252311180222e-09
4691 8.28172552758133e-09
4692 8.28081070380904e-09
4693 8.28077428849383e-09
4694 8.28030088939613e-09
4695 8.27961965654822e-09
4696 8.27889401477933e-09
4697 8.27825807903082e-09
4698 8.27769230937747e-09
4699 8.27745427756099e-09
4700 8.27660961988386e-09
4701 8.27599944130952e-09
4702 8.27518942259076e-09
4703 8.27468227271311e-09
4704 8.27441670736562e-09
4705 8.2738749185296e-09
4706 8.27325052910055e-09
4707 8.27255153268425e-09
4708 8.27222290666896e-09
4709 8.27149015947271e-09
4710 8.27089330357467e-09
4711 8.2705611248457e-09
4712 8.26979640322634e-09
4713 8.2694384673232e-09
4714 8.26832291522805e-09
4715 8.26822965649399e-09
4716 8.26747115212356e-09
4717 8.26668422604371e-09
4718 8.2660660538636e-09
4719 8.26569568346258e-09
4720 8.2653581756631e-09
4721 8.26450463620176e-09
4722 8.26390422759005e-09
4723 8.26301338463509e-09
4724 8.26284374255692e-09
4725 8.26228063743883e-09
4726 8.26199908487979e-09
4727 8.26094126438193e-09
4728 8.26051405056205e-09
4729 8.26002199971754e-09
4730 8.25958146322137e-09
4731 8.25922619185349e-09
4732 8.25826873551705e-09
4733 8.25776513835308e-09
4734 8.25702173301579e-09
4735 8.25664869807952e-09
4736 8.256057171252e-09
4737 8.25572676887987e-09
4738 8.25493184919424e-09
4739 8.25416002214752e-09
4740 8.25375590096655e-09
4741 8.25312529428857e-09
4742 8.25291568418152e-09
4743 8.25207990828858e-09
4744 8.25172996599122e-09
4745 8.25097412615605e-09
4746 8.25069346177543e-09
4747 8.25009216498529e-09
4748 8.24957613332344e-09
4749 8.24892243400654e-09
4750 8.24846857483408e-09
4751 8.24766033247215e-09
4752 8.24753332295813e-09
4753 8.24643908714506e-09
4754 8.24622947703801e-09
4755 8.24602341964464e-09
4756 8.24491586115528e-09
4757 8.2448625704501e-09
4758 8.24407653254866e-09
4759 8.24352763828529e-09
4760 8.24299473123347e-09
4761 8.24196000337452e-09
4762 8.24157631029721e-09
4763 8.24115709008311e-09
4764 8.24042878377895e-09
4765 8.23997936549858e-09
4766 8.2395246181477e-09
4767 8.23865420329639e-09
4768 8.23808754546462e-09
4769 8.23760437640431e-09
4770 8.23699153329471e-09
4771 8.23612111844341e-09
4772 8.2362063835717e-09
4773 8.23514412218174e-09
4774 8.23477819267282e-09
4775 8.2343083462888e-09
4776 8.23358448087674e-09
4777 8.2328384110042e-09
4778 8.2325994910093e-09
4779 8.23221224521831e-09
4780 8.23132406679861e-09
4781 8.2308186932778e-09
4782 8.23042967112997e-09
4783 8.22979817627356e-09
4784 8.22953882817501e-09
4785 8.22859202997961e-09
4786 8.22794454791165e-09
4787 8.22765500174683e-09
4788 8.22715140458286e-09
4789 8.22624102170266e-09
4790 8.22556245339001e-09
4791 8.22544166112493e-09
4792 8.22466539318611e-09
4793 8.22431989178085e-09
4794 8.22334467187602e-09
4795 8.22319812243677e-09
4796 8.22239876185904e-09
4797 8.22181167592362e-09
4798 8.22173973347162e-09
4799 8.2206206286628e-09
4800 8.22010459700095e-09
4801 8.21935053352263e-09
4802 8.21952905738499e-09
4803 8.21823942231958e-09
4804 8.21828383124057e-09
4805 8.21768963987779e-09
4806 8.21665224748358e-09
4807 8.21655277150057e-09
4808 8.21615486756855e-09
4809 8.21545942386592e-09
4810 8.21470003131708e-09
4811 8.21446022314376e-09
4812 8.21339707357538e-09
4813 8.21333667744284e-09
4814 8.21254886318457e-09
4815 8.21225931701974e-09
4816 8.21141821205629e-09
4817 8.21126722172494e-09
4818 8.21039858323047e-09
4819 8.21031598263744e-09
4820 8.20929368927636e-09
4821 8.2085920283248e-09
4822 8.20831758119311e-09
4823 8.20768963905039e-09
4824 8.20713896843017e-09
4825 8.20665668754827e-09
4826 8.20583956340215e-09
4827 8.20558732073096e-09
4828 8.20470003048968e-09
4829 8.20402767942596e-09
4830 8.20406143020591e-09
4831 8.20316969907253e-09
4832 8.20271495172165e-09
4833 8.20198664541749e-09
4834 8.20175749538521e-09
4835 8.20113665866984e-09
4836 8.20029910642006e-09
4837 8.20014367519661e-09
4838 8.19932921558575e-09
4839 8.19865331180836e-09
4840 8.19848633426545e-09
4841 8.19762480119834e-09
4842 8.19697731913038e-09
4843 8.19670198382028e-09
4844 8.19574985655436e-09
4845 8.19560774800721e-09
4846 8.19483503278207e-09
4847 8.19404633034537e-09
4848 8.19400458595965e-09
4849 8.19282330866145e-09
4850 8.19240764116103e-09
4851 8.1921767147719e-09
4852 8.19163403775747e-09
4853 8.19086842795969e-09
4854 8.1904003579325e-09
4855 8.19032841548051e-09
4856 8.18919243528171e-09
4857 8.18877055053235e-09
4858 8.1883984037745e-09
4859 8.18788237211265e-09
4860 8.18707146521547e-09
4861 8.18675971459015e-09
4862 8.1862117085052e-09
4863 8.18540701885695e-09
4864 8.18484746645254e-09
4865 8.18458634199715e-09
4866 8.18402945412799e-09
4867 8.18356404863607e-09
4868 8.18360934573548e-09
4869 8.18232059884849e-09
4870 8.18126455470747e-09
4871 8.18054957107961e-09
4872 8.18066858698785e-09
4873 8.1796294182368e-09
4874 8.17925371876527e-09
4875 8.17876522063443e-09
4876 8.1785520578137e-09
4877 8.17747114467693e-09
4878 8.17741341307965e-09
4879 8.17682810350107e-09
4880 8.17639378425383e-09
4881 8.17576140121901e-09
4882 8.17492917803975e-09
4883 8.17461298652233e-09
4884 8.17409784303891e-09
4885 8.17342815651045e-09
4886 8.17302492350791e-09
4887 8.17229217631166e-09
4888 8.17181344814344e-09
4889 8.17149903298287e-09
4890 8.17060907820633e-09
4891 8.17047762780021e-09
4892 8.16960277205681e-09
4893 8.16890732835418e-09
4894 8.16885492582742e-09
4895 8.16797118829982e-09
4896 8.16745426845955e-09
4897 8.16698797478921e-09
4898 8.16651990476203e-09
4899 8.16569301065329e-09
4900 8.16560596916815e-09
4901 8.16448864071617e-09
4902 8.16420353544345e-09
4903 8.16332423880795e-09
4904 8.1628170889303e-09
4905 8.1623845460399e-09
4906 8.16204792641884e-09
4907 8.16112333268393e-09
4908 8.16070855336193e-09
4909 8.1607387514282e-09
4910 8.15958589583943e-09
4911 8.15929723785302e-09
4912 8.15856715519203e-09
4913 8.15780243357267e-09
4914 8.1570554755217e-09
4915 8.15683964816571e-09
4916 8.15633693918016e-09
4917 8.15589995539767e-09
4918 8.15495404538069e-09
4919 8.15427458888962e-09
4920 8.15396550279957e-09
4921 8.15358980332803e-09
4922 8.1526820849831e-09
4923 8.15218559324649e-09
4924 8.15213407889814e-09
4925 8.15149014954386e-09
4926 8.15057976666367e-09
4927 8.15058065484209e-09
4928 8.14972178631024e-09
4929 8.14912226587694e-09
4930 8.1481283942253e-09
4931 8.14756528910721e-09
4932 8.14739919974272e-09
4933 8.1467401713553e-09
4934 8.14612999278097e-09
4935 8.14515122016246e-09
4936 8.14513612112933e-09
4937 8.14448686270453e-09
4938 8.14446821095771e-09
4939 8.14356226896962e-09
4940 8.14293166229163e-09
4941 8.14247069769181e-09
4942 8.14198486409623e-09
4943 8.14093237266889e-09
4944 8.14095635348622e-09
4945 8.14031508866719e-09
4946 8.13987099945734e-09
4947 8.13886735784308e-09
4948 8.13863731963238e-09
4949 8.13780953734522e-09
4950 8.13753686657037e-09
4951 8.13695155699179e-09
4952 8.1362712123223e-09
4953 8.13589284831551e-09
4954 8.13528977516853e-09
4955 8.13508371777516e-09
4956 8.13431988433422e-09
4957 8.13386957787543e-09
4958 8.13326206383636e-09
4959 8.1323578982051e-09
4960 8.13234013463671e-09
4961 8.13151412870639e-09
4962 8.13113398834275e-09
4963 8.13024136903095e-09
4964 8.12971201469281e-09
4965 8.12932299254499e-09
4966 8.12865863508705e-09
4967 8.12835487806751e-09
4968 8.12738232269794e-09
4969 8.12688760731817e-09
4970 8.12615041922982e-09
4971 8.12593370369541e-09
4972 8.12549050266398e-09
4973 8.12503309077783e-09
4974 8.12457212617801e-09
4975 8.12374256753401e-09
4976 8.12338996070139e-09
4977 8.12264566718568e-09
4978 8.12225930957311e-09
4979 8.12166955910243e-09
4980 8.12106115688493e-09
4981 8.12075029443804e-09
4982 8.1204420965264e-09
4983 8.11955924717722e-09
4984 8.11896949670654e-09
4985 8.11851030846356e-09
4986 8.11770561881531e-09
4987 8.11729172767173e-09
4988 8.11680855861141e-09
4989 8.11648437348822e-09
4990 8.11510236786717e-09
4991 8.11555178614753e-09
4992 8.11451972282384e-09
4993 8.11385980625801e-09
4994 8.11350453489013e-09
4995 8.11271760881027e-09
4996 8.11242806264545e-09
4997 8.11176281700909e-09
4998 8.11116951382473e-09
4999 8.11058598060299e-09
};
\addlegendentry{Test}
\end{groupplot}

\end{tikzpicture}

	% This file was created by tikzplotlib v0.9.6.
\begin{tikzpicture}

\begin{groupplot}[group style={group size=1 by 2},
legend cell align={left},
legend style={fill opacity=1, draw opacity=1, text opacity=1, draw=white},
log basis y={10},
tick align=outside,
tick pos=left,
title style={at={(0.3,0.85)},anchor=north},
x grid style={white!69.0196078431373!black},
xlabel={Epoch},
x label style={yshift=13pt},
xmin=-99.95, xmax=5098.95,
xtick style={color=black},
xtick = {0,1000,4000,5000},
y grid style={white!69.0196078431373!black},
ylabel={MSE Loss},
ymode=log,
ytick style={color=black},
width=.45\textwidth,
height=.25\textwidth
]

\nextgroupplot[
title={Leaky/Tanh $\rare$},
ymin=5.06930642957777e-09, ymax=1e-05,
]
\addplot [semithick, black, dashed]
table {%
0 0.00537926897467696
1 0.000324325849460365
2 0.000191448766511712
3 0.000180987851649661
4 0.000161569847787177
5 0.000111777376539521
6 5.36363975949143e-05
7 3.04853169982948e-05
8 2.61630102617687e-05
9 2.3062706877397e-05
10 1.90714949017092e-05
11 1.44392981503927e-05
12 1.005958182364e-05
13 6.78378771342381e-06
14 4.96552498248803e-06
15 4.01256759806756e-06
16 3.42866652513152e-06
17 2.99875318925302e-06
18 2.66226517730672e-06
19 2.39634537678768e-06
20 2.18927207963659e-06
21 2.02825173611743e-06
22 1.90509641760883e-06
23 1.80711361914021e-06
24 1.72771627400081e-06
25 1.65752676944209e-06
26 1.59690033109783e-06
27 1.54408441812492e-06
28 1.49150736101689e-06
29 1.44153419581272e-06
30 1.39383434423479e-06
31 1.34720315087122e-06
32 1.30109482185148e-06
33 1.25379442448548e-06
34 1.20788310636399e-06
35 1.16203255339542e-06
36 1.11641993507305e-06
37 1.07344115927788e-06
38 1.03146026298973e-06
39 9.92806262459567e-07
40 9.56833535930457e-07
41 9.23543588534415e-07
42 8.93950566780433e-07
43 8.66367665938839e-07
44 8.41672275925021e-07
45 8.19183669047874e-07
46 7.98604961794069e-07
47 7.80519454195883e-07
48 7.63187743931582e-07
49 7.47564150056945e-07
50 7.32583129618547e-07
51 7.18796317880077e-07
52 7.05872307435129e-07
53 6.93905756676827e-07
54 6.81588794385846e-07
55 6.69819190441956e-07
56 6.58149053432311e-07
57 6.47357014059935e-07
58 6.35614143803309e-07
59 6.2502412724541e-07
60 6.1467551897465e-07
61 6.05110842084144e-07
62 5.95938256637396e-07
63 5.87134924948884e-07
64 5.78806650441521e-07
65 5.7111052216996e-07
66 5.63843811594467e-07
67 5.56850080752014e-07
68 5.50290696182287e-07
69 5.44114506315196e-07
70 5.38255490575068e-07
71 5.32821911264136e-07
72 5.27468139459586e-07
73 5.22255367513935e-07
74 5.17322025970657e-07
75 5.12748185725442e-07
76 5.08049609196704e-07
77 5.03903404998951e-07
78 4.99898509897179e-07
79 4.96254582083466e-07
80 4.92545565037972e-07
81 4.8879899746801e-07
82 4.85454735301261e-07
83 4.82024266908354e-07
84 4.78799978099076e-07
85 4.75748878411508e-07
86 4.72855693619323e-07
87 4.70037327289674e-07
88 4.66922109069401e-07
89 4.63972293232473e-07
90 4.61216015427368e-07
91 4.5861256580082e-07
92 4.5615915830588e-07
93 4.5366836721783e-07
94 4.51300973614011e-07
95 4.48962911878681e-07
96 4.46682791620745e-07
97 4.44417258396612e-07
98 4.42211226255651e-07
99 4.39953997698694e-07
100 4.37889469592889e-07
101 4.35940979087945e-07
102 4.33882396890795e-07
103 4.31655883238236e-07
104 4.29410587546641e-07
105 4.2729061793878e-07
106 4.25008231923485e-07
107 4.22903520114204e-07
108 4.21015691198789e-07
109 4.19006507200947e-07
110 4.16926578351351e-07
111 4.14874409790045e-07
112 4.1287306186355e-07
113 4.10836467100495e-07
114 4.09193907504601e-07
115 4.0705106852279e-07
116 4.05251153102526e-07
117 4.03300212616031e-07
118 4.01468570451868e-07
119 3.99724681169999e-07
120 3.97820055910714e-07
121 3.96209956400995e-07
122 3.94328917439069e-07
123 3.92485409813403e-07
124 3.90679926232096e-07
125 3.88894603554846e-07
126 3.87438976842347e-07
127 3.85627665668409e-07
128 3.83959368249975e-07
129 3.82367194930566e-07
130 3.80764984010185e-07
131 3.79130087370783e-07
132 3.77436962251565e-07
133 3.75722851503113e-07
134 3.74428929839965e-07
135 3.72895064774781e-07
136 3.713125786593e-07
137 3.69711134077022e-07
138 3.68195370530344e-07
139 3.66658514819207e-07
140 3.65324342231688e-07
141 3.63842661505132e-07
142 3.62394095015262e-07
143 3.60946005082852e-07
144 3.59498689908477e-07
145 3.58238799352151e-07
146 3.56732601048293e-07
147 3.55371864356968e-07
148 3.54011777233154e-07
149 3.5275652089517e-07
150 3.51446611023931e-07
151 3.5015020199225e-07
152 3.48811512116853e-07
153 3.4756472395614e-07
154 3.46247541019196e-07
155 3.44960876741851e-07
156 3.43705323484045e-07
157 3.42532890341474e-07
158 3.41293314154001e-07
159 3.40117012381569e-07
160 3.3890576290041e-07
161 3.37826488152615e-07
162 3.36655637404704e-07
163 3.3549127570609e-07
164 3.34314691070148e-07
165 3.33243560609375e-07
166 3.32218753024804e-07
167 3.31112521431365e-07
168 3.30052226340527e-07
169 3.29028703177414e-07
170 3.2790485362888e-07
171 3.26740986237084e-07
172 3.25842840128487e-07
173 3.2482061808814e-07
174 3.23954994453146e-07
175 3.22873094193099e-07
176 3.21928446648911e-07
177 3.20977210414242e-07
178 3.19969357258643e-07
179 3.18987769951207e-07
180 3.17876480773194e-07
181 3.17063132918349e-07
182 3.16040863511446e-07
183 3.15127243474933e-07
184 3.14169429769251e-07
185 3.13193994207595e-07
186 3.12247779437058e-07
187 3.11346675704982e-07
188 3.10469445455652e-07
189 3.09641504051328e-07
190 3.08756569928903e-07
191 3.07876854254374e-07
192 3.07015841773506e-07
193 3.06199104425531e-07
194 3.05353673512698e-07
195 3.04491107300464e-07
196 3.03661975697977e-07
197 3.02830573680524e-07
198 3.02035184652993e-07
199 3.0121672716632e-07
200 3.00380253987598e-07
201 2.99616179935569e-07
202 2.98815651586182e-07
203 2.98010291318818e-07
204 2.9723919914737e-07
205 2.96436995791183e-07
206 2.95694983979189e-07
207 2.95015608780425e-07
208 2.94198568477455e-07
209 2.93471664627454e-07
210 2.92710463754453e-07
211 2.92014826126952e-07
212 2.91273496442201e-07
213 2.90539209304441e-07
214 2.89891880011162e-07
215 2.89167747662411e-07
216 2.88402231796603e-07
217 2.87725845061537e-07
218 2.87033993460639e-07
219 2.86387271843935e-07
220 2.85706630892335e-07
221 2.85072257773855e-07
222 2.84431937942031e-07
223 2.83748340626389e-07
224 2.83091480649311e-07
225 2.82458360890026e-07
226 2.81813902758721e-07
227 2.8118964448165e-07
228 2.80575783736303e-07
229 2.79946692189448e-07
230 2.79276069519163e-07
231 2.787020724897e-07
232 2.78050531785645e-07
233 2.77453712159037e-07
234 2.76878374163125e-07
235 2.76256404389663e-07
236 2.75675015381616e-07
237 2.75072065869963e-07
238 2.74503869326814e-07
239 2.73951722277843e-07
240 2.73374394374137e-07
241 2.72794813700905e-07
242 2.72269830241534e-07
243 2.71672219271935e-07
244 2.71127169395591e-07
245 2.7054031877416e-07
246 2.70034094158156e-07
247 2.69396030485147e-07
248 2.68851049145091e-07
249 2.6818264521733e-07
250 2.6761745773296e-07
251 2.67030464711659e-07
252 2.6650430652797e-07
253 2.65959659833115e-07
254 2.65394447864331e-07
255 2.64852272238159e-07
256 2.64300880481905e-07
257 2.63756149526451e-07
258 2.63028001118215e-07
259 2.62460299039091e-07
260 2.61855232053243e-07
261 2.61331802009934e-07
262 2.60815858872832e-07
263 2.60261341647805e-07
264 2.59711425446341e-07
265 2.59245437996825e-07
266 2.58891884099555e-07
267 2.58245335019502e-07
268 2.57696187510703e-07
269 2.5720047655664e-07
270 2.56661104803157e-07
271 2.56120709238417e-07
272 2.55632109426251e-07
273 2.55149979577318e-07
274 2.54602728589681e-07
275 2.54125361564661e-07
276 2.53612250389779e-07
277 2.53115071235044e-07
278 2.52673663840852e-07
279 2.52181400984597e-07
280 2.51739133730666e-07
281 2.51280379476526e-07
282 2.5073291734401e-07
283 2.50260147627301e-07
284 2.49711764233496e-07
285 2.49278976388823e-07
286 2.48819121663857e-07
287 2.48423981211943e-07
288 2.47936454529807e-07
289 2.47451482405125e-07
290 2.46968502182199e-07
291 2.46501543998079e-07
292 2.46016386167724e-07
293 2.45580273110946e-07
294 2.45129641246677e-07
295 2.44709981313207e-07
296 2.44276135166643e-07
297 2.43637187226753e-07
298 2.43210320457266e-07
299 2.42601521474484e-07
300 2.42212586684332e-07
301 2.4164615468969e-07
302 2.41198870943649e-07
303 2.407180980466e-07
304 2.40255705200809e-07
305 2.39770088287017e-07
306 2.39291980857814e-07
307 2.38868118271895e-07
308 2.38418197893608e-07
309 2.37980143015903e-07
310 2.37512703721521e-07
311 2.37094398716664e-07
312 2.36620000382715e-07
313 2.36198308314783e-07
314 2.35787259308395e-07
315 2.35382170929554e-07
316 2.34935281659077e-07
317 2.34529268281491e-07
318 2.34083780735972e-07
319 2.33648303730405e-07
320 2.33260743527808e-07
321 2.32825503397649e-07
322 2.32393887515059e-07
323 2.31952593379603e-07
324 2.31528094644773e-07
325 2.31091467286504e-07
326 2.30653432605976e-07
327 2.30234133519858e-07
328 2.29851819817384e-07
329 2.29377622202875e-07
330 2.28941529931248e-07
331 2.2853986822291e-07
332 2.28112414939474e-07
333 2.27670627010568e-07
334 2.27240460651146e-07
335 2.2683481809338e-07
336 2.26399548362011e-07
337 2.25983697175636e-07
338 2.25576361884805e-07
339 2.25153648255372e-07
340 2.24780862881246e-07
341 2.24329024330672e-07
342 2.23981187339461e-07
343 2.23528726817079e-07
344 2.23138665113254e-07
345 2.22666132327021e-07
346 2.22294292905545e-07
347 2.21909176712387e-07
348 2.21531205626846e-07
349 2.21108317006724e-07
350 2.20705658486864e-07
351 2.20260400004335e-07
352 2.19871969619589e-07
353 2.19488365178933e-07
354 2.19081386692466e-07
355 2.18646649910781e-07
356 2.18223053263245e-07
357 2.17840906014288e-07
358 2.17447773666279e-07
359 2.17026864985925e-07
360 2.16694807967599e-07
361 2.16259201424762e-07
362 2.1583503078304e-07
363 2.15521394649087e-07
364 2.15093814734502e-07
365 2.14645060871987e-07
366 2.14278768149612e-07
367 2.13864801242636e-07
368 2.13514095570133e-07
369 2.13104509818152e-07
370 2.1270352831948e-07
371 2.12376758771171e-07
372 2.11898171878389e-07
373 2.1157079094003e-07
374 2.11146420905806e-07
375 2.10764177641209e-07
376 2.10367095585084e-07
377 2.10023715245811e-07
378 2.0962159432969e-07
379 2.09282447219294e-07
380 2.09006052751093e-07
381 2.08600960020178e-07
382 2.0816858242334e-07
383 2.0778526173082e-07
384 2.07461081103588e-07
385 2.07103965056632e-07
386 2.06752541984656e-07
387 2.06254396735162e-07
388 2.05940889507872e-07
389 2.05581420139822e-07
390 2.05206684734449e-07
391 2.04875735660082e-07
392 2.04461397396827e-07
393 2.04140243377893e-07
394 2.03670522154198e-07
395 2.03450584541187e-07
396 2.02966163723062e-07
397 2.0277340211905e-07
398 2.02377072323401e-07
399 2.02045504728687e-07
400 2.01493711454503e-07
401 2.01408080127585e-07
402 2.00910208832283e-07
403 2.00661772904454e-07
404 2.00227273856157e-07
405 2.0004771265647e-07
406 1.99559731472121e-07
407 1.99244385106212e-07
408 1.9884856933583e-07
409 1.98642940351768e-07
410 1.98209742523936e-07
411 1.9787925690995e-07
412 1.97563098382147e-07
413 1.97212684676096e-07
414 1.96908413001573e-07
415 1.96624026167491e-07
416 1.96237357271478e-07
417 1.95925344421255e-07
418 1.95556203929748e-07
419 1.95201773653508e-07
420 1.94899599655685e-07
421 1.94440916891025e-07
422 1.94267081650601e-07
423 1.93773665593255e-07
424 1.935315427124e-07
425 1.93209232795866e-07
426 1.92874546958599e-07
427 1.92506085657129e-07
428 1.92232111174917e-07
429 1.91765056094972e-07
430 1.91541054973321e-07
431 1.91146748324655e-07
432 1.90879218046724e-07
433 1.90631337413372e-07
434 1.9033152294945e-07
435 1.89945725709073e-07
436 1.89646187759074e-07
437 1.89322766004807e-07
438 1.88994187980462e-07
439 1.88678464771996e-07
440 1.88415765326688e-07
441 1.88036978630457e-07
442 1.87795632159649e-07
443 1.87423036815737e-07
444 1.87154296578562e-07
445 1.86883733770848e-07
446 1.8653575716332e-07
447 1.86317136076752e-07
448 1.86038097139019e-07
449 1.85826476996276e-07
450 1.85510107636233e-07
451 1.85214987603821e-07
452 1.84718681928153e-07
453 1.84597460289204e-07
454 1.84299972340263e-07
455 1.83956828541554e-07
456 1.83656115649633e-07
457 1.83341635929146e-07
458 1.83029795502776e-07
459 1.82906214055478e-07
460 1.82593222594285e-07
461 1.82243653442882e-07
462 1.81925875300415e-07
463 1.81367697899226e-07
464 1.81096243882628e-07
465 1.80794670625595e-07
466 1.80501837594882e-07
467 1.80158165886901e-07
468 1.79891409128885e-07
469 1.79574427755291e-07
470 1.79317500276888e-07
471 1.79010063373397e-07
472 1.78725223807241e-07
473 1.78432354378977e-07
474 1.78154695200661e-07
475 1.77869362674166e-07
476 1.77587940179968e-07
477 1.77435676273596e-07
478 1.77077641264844e-07
479 1.76741828564531e-07
480 1.76460036141357e-07
481 1.76144911609555e-07
482 1.75879508230281e-07
483 1.75505196174264e-07
484 1.75233252288187e-07
485 1.74988864164227e-07
486 1.74593626141473e-07
487 1.74357453045282e-07
488 1.74129375974275e-07
489 1.73851382595736e-07
490 1.73489901840185e-07
491 1.73380046254223e-07
492 1.72983708336183e-07
493 1.72777742926122e-07
494 1.72515633343195e-07
495 1.72313285183634e-07
496 1.72005488445315e-07
497 1.71752280692772e-07
498 1.71478896795918e-07
499 1.71131160015214e-07
500 1.70869828940212e-07
501 1.70591390844521e-07
502 1.70338167675865e-07
503 1.70140944731223e-07
504 1.69943524905847e-07
505 1.69673944451887e-07
506 1.69399835116657e-07
507 1.69055191278744e-07
508 1.68898963052833e-07
509 1.685464903316e-07
510 1.68362018093227e-07
511 1.68218424094313e-07
512 1.67981423000896e-07
513 1.67665027016817e-07
514 1.67391526479932e-07
515 1.67152652140956e-07
516 1.66942770381873e-07
517 1.66555362702603e-07
518 1.66402951630396e-07
519 1.66126789231669e-07
520 1.65971751092897e-07
521 1.65681123545092e-07
522 1.65376101433523e-07
523 1.65169757790906e-07
524 1.6483006690482e-07
525 1.64661625245088e-07
526 1.64434999072149e-07
527 1.64227067323885e-07
528 1.63905039555878e-07
529 1.63764953875045e-07
530 1.63408635656559e-07
531 1.63292031461815e-07
532 1.62969131412893e-07
533 1.62833645410032e-07
534 1.62547514436007e-07
535 1.62344810105353e-07
536 1.6223000879112e-07
537 1.62046782103076e-07
538 1.6194940303027e-07
539 1.61526954020275e-07
540 1.61308555172823e-07
541 1.60983122309943e-07
542 1.60816409048081e-07
543 1.60548277823302e-07
544 1.60293181439286e-07
545 1.60029280742435e-07
546 1.59843005403282e-07
547 1.59588148241241e-07
548 1.59389587854442e-07
549 1.59107243858259e-07
550 1.58923910622555e-07
551 1.58713202318417e-07
552 1.58522403746808e-07
553 1.58252669333336e-07
554 1.58034186673817e-07
555 1.57883096739475e-07
556 1.57738759416048e-07
557 1.57432374004785e-07
558 1.57201307017907e-07
559 1.57003296326508e-07
560 1.56820888093989e-07
561 1.56633221576996e-07
562 1.56443865297007e-07
563 1.56224910698555e-07
564 1.56055643494746e-07
565 1.55838856005452e-07
566 1.55682252435341e-07
567 1.55536764016695e-07
568 1.55362656394153e-07
569 1.55100982393463e-07
570 1.54921962827181e-07
571 1.54784955974208e-07
572 1.54597766010234e-07
573 1.54354613827401e-07
574 1.54135425088242e-07
575 1.53951064313596e-07
576 1.53778016324324e-07
577 1.53594668168999e-07
578 1.53529006569286e-07
579 1.53297743344716e-07
580 1.53099291051895e-07
581 1.52881978307207e-07
582 1.52704949650584e-07
583 1.52518762591036e-07
584 1.5236250360573e-07
585 1.52199381625806e-07
586 1.52223800215623e-07
587 1.51833477772811e-07
588 1.51697847911691e-07
589 1.51531595562915e-07
590 1.51428843730983e-07
591 1.51188097086141e-07
592 1.51015534391341e-07
593 1.50969652777277e-07
594 1.50809665878882e-07
595 1.50519695155893e-07
596 1.50493258278317e-07
597 1.50223761692381e-07
598 1.50159953516749e-07
599 1.49994593320102e-07
600 1.49933122561841e-07
601 1.49577338354412e-07
602 1.49454696793505e-07
603 1.49385890949816e-07
604 1.49139802108067e-07
605 1.49014007436499e-07
606 1.48880155549591e-07
607 1.48732757462078e-07
608 1.48514718827819e-07
609 1.48610412999517e-07
610 1.4822434298889e-07
611 1.48349138306969e-07
612 1.47939630958049e-07
613 1.48082978993713e-07
614 1.47688663300727e-07
615 1.47789761424377e-07
616 1.47435760404768e-07
617 1.47525687615335e-07
618 1.47202618155262e-07
619 1.4714802206317e-07
620 1.46932239698749e-07
621 1.46766087670169e-07
622 1.46670737136478e-07
623 1.46656839370429e-07
624 1.46451818016757e-07
625 1.46419701272826e-07
626 1.46326028629851e-07
627 1.46080902262646e-07
628 1.45928070391044e-07
629 1.45951529398225e-07
630 1.45679930827924e-07
631 1.4558240016882e-07
632 1.4554031317271e-07
633 1.45450159350524e-07
634 1.45218444095541e-07
635 1.45190369846837e-07
636 1.45074038928605e-07
637 1.44877360655826e-07
638 1.44790599472699e-07
639 1.44614654155717e-07
640 1.44586633797683e-07
641 1.44414264980242e-07
642 1.4427893903779e-07
643 1.44226019784366e-07
644 1.44287894355166e-07
645 1.44161567603884e-07
646 1.43885013656142e-07
647 1.43870274001312e-07
648 1.43801189965043e-07
649 1.435163541883e-07
650 1.43490153050596e-07
651 1.43279821318565e-07
652 1.4337752615079e-07
653 1.43281555532226e-07
654 1.42976653925597e-07
655 1.42882232308406e-07
656 1.43044091293731e-07
657 1.42810023791462e-07
658 1.42531364010434e-07
659 1.42646210095076e-07
660 1.42485420980609e-07
661 1.42276854622647e-07
662 1.42313023345775e-07
663 1.42100916975973e-07
664 1.41966335663213e-07
665 1.41953429793062e-07
666 1.41738237530387e-07
667 1.4187404160948e-07
668 1.41688332257139e-07
669 1.41727910859313e-07
670 1.41423522413309e-07
671 1.41419274794163e-07
672 1.41409380586044e-07
673 1.41058935513527e-07
674 1.41154362800489e-07
675 1.41162569775144e-07
676 1.40795779042246e-07
677 1.40888800276073e-07
678 1.40679525608256e-07
679 1.40544483146599e-07
680 1.40532143841909e-07
681 1.4050670554866e-07
682 1.40432941738489e-07
683 1.40278016644846e-07
684 1.40266155701063e-07
685 1.40132945072757e-07
686 1.39867811995931e-07
687 1.40041230119792e-07
688 1.39835203310845e-07
689 1.39649071881909e-07
690 1.39666877260858e-07
691 1.39604243356128e-07
692 1.39480758503119e-07
693 1.39429618501552e-07
694 1.39291471941139e-07
695 1.39224886883227e-07
696 1.39373204308235e-07
697 1.38990202562805e-07
698 1.38970073305877e-07
699 1.38958005189593e-07
700 1.38833719070863e-07
701 1.38865341729577e-07
702 1.38649325242657e-07
703 1.3877714671251e-07
704 1.38295324139559e-07
705 1.38328095320439e-07
706 1.38271988034955e-07
707 1.38280872987284e-07
708 1.38160360576123e-07
709 1.38112425955406e-07
710 1.38092942810975e-07
711 1.37915692180357e-07
712 1.37960050672792e-07
713 1.37660058341371e-07
714 1.37648414654468e-07
715 1.37548896535833e-07
716 1.37704554852736e-07
717 1.37246946102731e-07
718 1.37406698758724e-07
719 1.38282616322094e-07
720 1.36924560089291e-07
721 1.38047666258689e-07
722 1.36723256965787e-07
723 1.37835451157198e-07
724 1.36591847809342e-07
725 1.37268774035082e-07
726 1.36604063801116e-07
727 1.36760222798404e-07
728 1.37522121165468e-07
729 1.36262575662904e-07
730 1.36532620438778e-07
731 1.36587666908383e-07
732 1.36285042655171e-07
733 1.36367610657828e-07
734 1.3631952306703e-07
735 1.3617504265051e-07
736 1.3617151680112e-07
737 1.35928872507574e-07
738 1.36883148336153e-07
739 1.35644793554768e-07
740 1.35848459143162e-07
741 1.35717295369986e-07
742 1.35925469255227e-07
743 1.35459736120946e-07
744 1.35413008054641e-07
745 1.36477638922106e-07
746 1.35117394934881e-07
747 1.35272397977282e-07
748 1.35096170772897e-07
749 1.351372806222e-07
750 1.35049970611867e-07
751 1.35009364020444e-07
752 1.35073848297473e-07
753 1.34849044270524e-07
754 1.34702889456939e-07
755 1.34788891336779e-07
756 1.34729296408231e-07
757 1.34620654068662e-07
758 1.34481607308334e-07
759 1.34542223226397e-07
760 1.34423401215944e-07
761 1.34317759789671e-07
762 1.34254335149908e-07
763 1.34130800827581e-07
764 1.34061302912158e-07
765 1.33974288616834e-07
766 1.339397514597e-07
767 1.33814305285362e-07
768 1.33761088715101e-07
769 1.33688917313801e-07
770 1.33516693345825e-07
771 1.33551146626054e-07
772 1.33321331038161e-07
773 1.33257479574045e-07
774 1.3314653566221e-07
775 1.3307513515004e-07
776 1.32957369075815e-07
777 1.32948707491476e-07
778 1.328441522368e-07
779 1.32765396021028e-07
780 1.32660078499214e-07
781 1.32495479777006e-07
782 1.32527600574139e-07
783 1.32394918797019e-07
784 1.32384303477284e-07
785 1.32434042231289e-07
786 1.32226808759572e-07
787 1.32126254242415e-07
788 1.32083504748692e-07
789 1.32007645284649e-07
790 1.31958393757348e-07
791 1.31861627703511e-07
792 1.31940223282889e-07
793 1.31662543532141e-07
794 1.31615914091832e-07
795 1.3166125657138e-07
796 1.31522912526272e-07
797 1.31570957959237e-07
798 1.31365379189674e-07
799 1.31462527114046e-07
800 1.31439980081804e-07
801 1.31190741833542e-07
802 1.31231580866586e-07
803 1.31040102225644e-07
804 1.31053217613086e-07
805 1.30897390277163e-07
806 1.30889801103429e-07
807 1.30715777041601e-07
808 1.30756223709838e-07
809 1.30628854211334e-07
810 1.30462685251853e-07
811 1.30414655506783e-07
812 1.30407762948259e-07
813 1.30288454306626e-07
814 1.30336483670668e-07
815 1.3018150747568e-07
816 1.30121635276037e-07
817 1.30094425502936e-07
818 1.30011646203965e-07
819 1.2992151734359e-07
820 1.29839809415166e-07
821 1.2967381421225e-07
822 1.29597560563077e-07
823 1.295705061648e-07
824 1.29548155280546e-07
825 1.29344793947439e-07
826 1.29407171371554e-07
827 1.29320858660797e-07
828 1.29278127805055e-07
829 1.29198980938217e-07
830 1.29082386072898e-07
831 1.29047126630422e-07
832 1.28963352360678e-07
833 1.28903002251324e-07
834 1.28818705709577e-07
835 1.28830668928881e-07
836 1.28753102705659e-07
837 1.28663153962538e-07
838 1.28578692733861e-07
839 1.28525893883236e-07
840 1.28461451492257e-07
841 1.28395582914109e-07
842 1.28391512587811e-07
843 1.28273653727362e-07
844 1.28243403675121e-07
845 1.28135905602456e-07
846 1.2813735687045e-07
847 1.27990388346433e-07
848 1.2802488484942e-07
849 1.27859900838967e-07
850 1.2781179943433e-07
851 1.27719577855512e-07
852 1.2773484385642e-07
853 1.27607142919484e-07
854 1.2756764747035e-07
855 1.27454755442624e-07
856 1.27419393574524e-07
857 1.27326358492663e-07
858 1.27338625000473e-07
859 1.27176948094831e-07
860 1.27156185556476e-07
861 1.27055903587081e-07
862 1.27059177770672e-07
863 1.26945453957639e-07
864 1.26905267250788e-07
865 1.26817184926242e-07
866 1.26720529665114e-07
867 1.26684015450351e-07
868 1.26627241658017e-07
869 1.26523526057465e-07
870 1.26492231408815e-07
871 1.26422224465461e-07
872 1.26319076333647e-07
873 1.26284832416612e-07
874 1.26258520464884e-07
875 1.26143520967847e-07
876 1.26123683993562e-07
877 1.25995991834049e-07
878 1.25994270230745e-07
879 1.25849246395582e-07
880 1.25837716322241e-07
881 1.25760414595e-07
882 1.2569778150473e-07
883 1.25657724532768e-07
884 1.25561306333566e-07
885 1.25529986473971e-07
886 1.25437570972498e-07
887 1.25381883347764e-07
888 1.25315336428233e-07
889 1.25206956495205e-07
890 1.25182671113055e-07
891 1.25128715355416e-07
892 1.25029302290081e-07
893 1.24979792123892e-07
894 1.2492176188017e-07
895 1.24871249306491e-07
896 1.24817886916428e-07
897 1.24740170328597e-07
898 1.24701418766193e-07
899 1.24578112033369e-07
900 1.24608640034385e-07
901 1.24490363198504e-07
902 1.24423935903728e-07
903 1.24365138101545e-07
904 1.24298681522195e-07
905 1.24246052295884e-07
906 1.2418155196503e-07
907 1.24136679071629e-07
908 1.24037483121509e-07
909 1.23981106330273e-07
910 1.23975029754586e-07
911 1.23864943757379e-07
912 1.23801921202649e-07
913 1.23800400970264e-07
914 1.2364803504239e-07
915 1.2364109883789e-07
916 1.23496889059105e-07
917 1.23507249131016e-07
918 1.2344376674811e-07
919 1.23325841301813e-07
920 1.23257048115555e-07
921 1.2331594781001e-07
922 1.23116275299928e-07
923 1.2316893857145e-07
924 1.23023121719523e-07
925 1.22992693774471e-07
926 1.22878937629967e-07
927 1.22832627062586e-07
928 1.22770651644455e-07
929 1.22694384541155e-07
930 1.22650666436463e-07
931 1.22639889427845e-07
932 1.22546863666972e-07
933 1.22411796711575e-07
934 1.22418346440778e-07
935 1.2237241630686e-07
936 1.2222990939792e-07
937 1.22208444120808e-07
938 1.22142977380335e-07
939 1.22169625452972e-07
940 1.22059910295569e-07
941 1.21913441174293e-07
942 1.22010176250242e-07
943 1.21844763701251e-07
944 1.21871871846757e-07
945 1.21791860278719e-07
946 1.21614932587999e-07
947 1.21701534268048e-07
948 1.21513173215515e-07
949 1.21542157580645e-07
950 1.21465948248467e-07
951 1.21334595144962e-07
952 1.21449691094178e-07
953 1.21205511804057e-07
954 1.21259639921156e-07
955 1.21126643341629e-07
956 1.21099203512998e-07
957 1.20970506221951e-07
958 1.20978571027219e-07
959 1.20930028132626e-07
960 1.20928774110851e-07
961 1.20799354638823e-07
962 1.20673130783011e-07
963 1.20652438952096e-07
964 1.20591752028165e-07
965 1.20505554817019e-07
966 1.20495121546771e-07
967 1.20429295705105e-07
968 1.2033666901301e-07
969 1.20286372991441e-07
970 1.20216925046268e-07
971 1.20172720799783e-07
972 1.20100580447424e-07
973 1.20059384895566e-07
974 1.19870551508416e-07
975 1.19952925505817e-07
976 1.1978020986847e-07
977 1.19858777888116e-07
978 1.19622341205439e-07
979 1.19754206679978e-07
980 1.19467366006543e-07
981 1.19601308657291e-07
982 1.19401819220766e-07
983 1.19452868478476e-07
984 1.19246568589748e-07
985 1.19385652628257e-07
986 1.19014045945054e-07
987 1.19234693313874e-07
988 1.18889818970125e-07
989 1.19092462639525e-07
990 1.1875247753812e-07
991 1.19014326518396e-07
992 1.18782699111808e-07
993 1.18774333926819e-07
994 1.18601437525623e-07
995 1.18745587867952e-07
996 1.18406136232263e-07
997 1.18588774848938e-07
998 1.18331510828629e-07
999 1.18457206943035e-07
1000 1.18327288391118e-07
1001 1.1828860679941e-07
1002 1.18171067546857e-07
1003 1.18167666811608e-07
1004 1.18077708304298e-07
1005 1.18041528669011e-07
1006 1.17952378146935e-07
1007 1.17851308653716e-07
1008 1.17845310751097e-07
1009 1.17802575702264e-07
1010 1.17484612918606e-07
1011 1.17733143551568e-07
1012 1.1759902975772e-07
1013 1.17647475684546e-07
1014 1.17592685084933e-07
1015 1.17565659107033e-07
1016 1.17436752677058e-07
1017 1.17310259921233e-07
1018 1.17280786779972e-07
1019 1.17089752921906e-07
1020 1.17165992880697e-07
1021 1.1699477363436e-07
1022 1.16979458565591e-07
1023 1.16875027080798e-07
1024 1.1691639203848e-07
1025 1.1657687619504e-07
1026 1.16723137425279e-07
1027 1.16515327564404e-07
1028 1.16625204286791e-07
1029 1.16446067424558e-07
1030 1.16427153628251e-07
1031 1.16354283923314e-07
1032 1.16266580164837e-07
1033 1.16256563229999e-07
1034 1.16158198426586e-07
1035 1.1614902036472e-07
1036 1.16039866387307e-07
1037 1.15832262816529e-07
1038 1.15926651721665e-07
1039 1.1565406336933e-07
1040 1.15737860395537e-07
1041 1.15458466622087e-07
1042 1.155424351027e-07
1043 1.15267651889628e-07
1044 1.15596642147331e-07
1045 1.15268487014042e-07
1046 1.15271330384914e-07
1047 1.14962137548247e-07
1048 1.15291136848494e-07
1049 1.14847946213192e-07
1050 1.15190834858669e-07
1051 1.14710825439346e-07
1052 1.15016264221168e-07
1053 1.14565868249983e-07
1054 1.1481157616311e-07
1055 1.14480261936389e-07
1056 1.14558771930007e-07
1057 1.14633196979685e-07
1058 1.14455439633065e-07
1059 1.14238816733891e-07
1060 1.14399665126719e-07
1061 1.14151833112697e-07
1062 1.14281876839151e-07
1063 1.14121717697202e-07
1064 1.14132611764184e-07
1065 1.13912798159088e-07
1066 1.13884180562795e-07
1067 1.13976890491418e-07
1068 1.1367977784138e-07
1069 1.13688182774574e-07
1070 1.13567343898779e-07
1071 1.13623888069458e-07
1072 1.13519177852517e-07
1073 1.13451523641839e-07
1074 1.13300481866307e-07
1075 1.13332723437498e-07
1076 1.13240924983327e-07
1077 1.13148985659173e-07
1078 1.13076634114861e-07
1079 1.13031726201473e-07
1080 1.12945998923752e-07
1081 1.12876671465312e-07
1082 1.12785192566633e-07
1083 1.12776206107945e-07
1084 1.12798344784171e-07
1085 1.12494973792732e-07
1086 1.12565207766924e-07
1087 1.12357306901512e-07
1088 1.123637443321e-07
1089 1.12253228988557e-07
1090 1.12211716334354e-07
1091 1.12104510163302e-07
1092 1.1206960710286e-07
1093 1.11977439253508e-07
1094 1.11932589105912e-07
1095 1.11773434874785e-07
1096 1.11810565743831e-07
1097 1.11612962768071e-07
1098 1.1159645167913e-07
1099 1.11540086437767e-07
1100 1.11509210810823e-07
1101 1.11322369245492e-07
1102 1.11359068073913e-07
1103 1.11182769749973e-07
1104 1.11217455317281e-07
1105 1.11028761691667e-07
1106 1.1108429648532e-07
1107 1.1089796557906e-07
1108 1.10938999352772e-07
1109 1.1075545810435e-07
1110 1.10786726139178e-07
1111 1.1061356102271e-07
1112 1.10638028753218e-07
1113 1.10470988285538e-07
1114 1.1048431808014e-07
1115 1.10293081588164e-07
1116 1.10416813640768e-07
1117 1.10204453521057e-07
1118 1.10186400110734e-07
1119 1.10032594076159e-07
1120 1.10100790504664e-07
1121 1.09884401851446e-07
1122 1.09974860545314e-07
1123 1.09816414045838e-07
1124 1.09749837351458e-07
1125 1.09681992598976e-07
1126 1.09613169206657e-07
1127 1.0941496306982e-07
1128 1.09455786256429e-07
1129 1.09231630736506e-07
1130 1.09286840674905e-07
1131 1.09068101704413e-07
1132 1.09094173311064e-07
1133 1.08985955852958e-07
1134 1.08983172137922e-07
1135 1.08818126724142e-07
1136 1.08804445953847e-07
1137 1.08577826680722e-07
1138 1.08759500267563e-07
1139 1.08543335674227e-07
1140 1.08473860844782e-07
1141 1.08454586862372e-07
1142 1.08413596267898e-07
1143 1.08226259982303e-07
1144 1.08098892585673e-07
1145 1.08130716260924e-07
1146 1.0795761652016e-07
1147 1.07979801807989e-07
1148 1.07798190470465e-07
1149 1.07867866026456e-07
1150 1.07748803218843e-07
1151 1.07622438790589e-07
1152 1.07506979851735e-07
1153 1.07467208288359e-07
1154 1.07338677874935e-07
1155 1.07318710209192e-07
1156 1.07119180471571e-07
1157 1.07276342638851e-07
1158 1.07118442437493e-07
1159 1.069655763275e-07
1160 1.06974888594902e-07
1161 1.06782112546888e-07
1162 1.06718019643903e-07
1163 1.067115504787e-07
1164 1.06580819832836e-07
1165 1.0657124075264e-07
1166 1.06413620518886e-07
1167 1.06483058527118e-07
1168 1.06351474448818e-07
1169 1.06275020718805e-07
1170 1.06100385223407e-07
1171 1.06150675371008e-07
1172 1.059862570707e-07
1173 1.05890279418652e-07
1174 1.05865305336295e-07
1175 1.05762305017798e-07
1176 1.05660392594853e-07
1177 1.05526975269665e-07
1178 1.05471855732286e-07
1179 1.05415960856892e-07
1180 1.05346588691546e-07
1181 1.05343038220074e-07
1182 1.05270419716863e-07
1183 1.05181423618372e-07
1184 1.05123611743085e-07
1185 1.04972935992986e-07
1186 1.04976401444912e-07
1187 1.04836514002837e-07
1188 1.04791366572154e-07
1189 1.04640886251506e-07
1190 1.04624333612247e-07
1191 1.04557093461022e-07
1192 1.04508137620307e-07
1193 1.04472909169484e-07
1194 1.04310074201752e-07
1195 1.04239490467339e-07
1196 1.04146149061179e-07
1197 1.0403710382656e-07
1198 1.04058997495216e-07
1199 1.03709298677312e-07
1200 1.03904661825638e-07
1201 1.03811463439296e-07
1202 1.03665727048785e-07
1203 1.03703265286637e-07
1204 1.03406819302876e-07
1205 1.0335403518047e-07
1206 1.03353759138614e-07
1207 1.03263867196723e-07
1208 1.0325985418369e-07
1209 1.03108851925438e-07
1210 1.03035067132495e-07
1211 1.0294552942236e-07
1212 1.0288876663278e-07
1213 1.02876423676346e-07
1214 1.02662573236323e-07
1215 1.02678567911241e-07
1216 1.02390057315382e-07
1217 1.02407491808343e-07
1218 1.02241911179579e-07
1219 1.02207915172681e-07
1220 1.02160058817802e-07
1221 1.01964553252287e-07
1222 1.01970431314768e-07
1223 1.01941091940727e-07
1224 1.01796613222405e-07
1225 1.01697705973347e-07
1226 1.01618171594264e-07
1227 1.01715931592583e-07
1228 1.01561118837079e-07
1229 1.01375430762829e-07
1230 1.01236755834844e-07
1231 1.01312689741118e-07
1232 1.01238300313167e-07
1233 1.0102416126534e-07
1234 1.00975738874354e-07
1235 1.00945628822124e-07
1236 1.00841874891344e-07
1237 1.00698861831283e-07
1238 1.00694977623661e-07
1239 1.00593986147146e-07
1240 1.00357615229196e-07
1241 1.00404349749894e-07
1242 1.00319807234239e-07
1243 1.00157669515433e-07
1244 9.99840208835145e-08
1245 1.00118871265487e-07
1246 9.99391362292989e-08
1247 9.99027398269625e-08
1248 9.96582623469244e-08
1249 9.96339519696576e-08
1250 9.9405252533824e-08
1251 9.95907192100631e-08
1252 9.95708399300632e-08
1253 9.91101137848283e-08
1254 9.93096884380762e-08
1255 9.92001463058223e-08
1256 9.90096312967204e-08
1257 9.90505198710245e-08
1258 9.88288024137418e-08
1259 9.88051310182314e-08
1260 9.86768020045936e-08
1261 9.86374161748849e-08
1262 9.84476449188421e-08
1263 9.84186590771152e-08
1264 9.82929982056291e-08
1265 9.82753718035134e-08
1266 9.80300925088606e-08
1267 9.80353046071691e-08
1268 9.79526751945237e-08
1269 9.78678528791299e-08
1270 9.77433838103003e-08
1271 9.77294080404256e-08
1272 9.7612446272155e-08
1273 9.75639521887395e-08
1274 9.72775427428019e-08
1275 9.73426179897707e-08
1276 9.71786970112021e-08
1277 9.70942184843615e-08
1278 9.70437658636136e-08
1279 9.690014050312e-08
1280 9.68366146247845e-08
1281 9.67566769056738e-08
1282 9.66696707558334e-08
1283 9.65424662782155e-08
1284 9.64583639295746e-08
1285 9.63547184009705e-08
1286 9.62566287201483e-08
1287 9.6256001731021e-08
1288 9.61462494091236e-08
1289 9.60559344647116e-08
1290 9.59052122726511e-08
1291 9.59106940570109e-08
1292 9.57665970346966e-08
1293 9.57501474849742e-08
1294 9.56313208333448e-08
1295 9.54847253038515e-08
1296 9.54578427272423e-08
1297 9.53432140140542e-08
1298 9.53525316855774e-08
1299 9.51477408741575e-08
1300 9.51729034963655e-08
1301 9.49707996320548e-08
1302 9.49455826750878e-08
1303 9.483956462919e-08
1304 9.47691796393535e-08
1305 9.47605251435668e-08
1306 9.45748634686439e-08
1307 9.45444732765033e-08
1308 9.43062419707097e-08
1309 9.43507327733073e-08
1310 9.43299268643116e-08
1311 9.40654619157044e-08
1312 9.40579268080555e-08
1313 9.40094021886218e-08
1314 9.39390929586281e-08
1315 9.37696564848878e-08
1316 9.37139897771999e-08
1317 9.36550998340557e-08
1318 9.36536310605085e-08
1319 9.33969684999703e-08
1320 9.34000937378165e-08
1321 9.33514792529344e-08
1322 9.32543611171788e-08
1323 9.30747456351355e-08
1324 9.29778140501547e-08
1325 9.29543634371477e-08
1326 9.27843707589915e-08
1327 9.27184962034033e-08
1328 9.27009205229368e-08
1329 9.25843963313078e-08
1330 9.26100956095688e-08
1331 9.24299653677352e-08
1332 9.22939312770765e-08
1333 9.21989396900358e-08
1334 9.22646188032949e-08
1335 9.2141868859752e-08
1336 9.19170490614185e-08
1337 9.19022333305541e-08
1338 9.17333230758821e-08
1339 9.17761311423071e-08
1340 9.15972648574126e-08
1341 9.16407167470545e-08
1342 9.13953976935389e-08
1343 9.16109370994178e-08
1344 9.1441875040843e-08
1345 9.14199859329834e-08
1346 9.12718640284638e-08
1347 9.12183666228472e-08
1348 9.10676917325048e-08
1349 9.10336624397878e-08
1350 9.09409087630486e-08
1351 9.08364189822208e-08
1352 9.0792613887114e-08
1353 9.06479953615147e-08
1354 9.05673963638165e-08
1355 9.0363293946627e-08
1356 9.05175335499386e-08
1357 9.03084473629079e-08
1358 9.00589917263161e-08
1359 9.01229194583131e-08
1360 8.98698408966503e-08
1361 8.99031044507836e-08
1362 8.97484976847274e-08
1363 8.95870471100402e-08
1364 8.96799569765072e-08
1365 8.94222050686899e-08
1366 8.93247502666838e-08
1367 8.94127660706623e-08
1368 8.9011499865066e-08
1369 8.90570080729347e-08
1370 8.90579438626027e-08
1371 8.89372521282184e-08
1372 8.8675192975618e-08
1373 8.85419005451915e-08
1374 8.87074875906002e-08
1375 8.85344046643866e-08
1376 8.85895058377884e-08
1377 8.82548427569851e-08
1378 8.82028420670089e-08
1379 8.82088051752383e-08
1380 8.81064864683978e-08
1381 8.79697588653805e-08
1382 8.7908574203599e-08
1383 8.79394719586202e-08
1384 8.78683801621705e-08
1385 8.77054827026491e-08
1386 8.75559654684999e-08
1387 8.75825401345232e-08
1388 8.75099768240517e-08
1389 8.72481159563954e-08
1390 8.73480325069842e-08
1391 8.72215291272127e-08
1392 8.71810897722369e-08
1393 8.71089804559411e-08
1394 8.69473788727149e-08
1395 8.68325984533236e-08
1396 8.68650491780976e-08
1397 8.67739178778138e-08
1398 8.65550197315379e-08
1399 8.67175996797087e-08
1400 8.65171493469319e-08
1401 8.63157404404902e-08
1402 8.65059098775767e-08
1403 8.64519121299701e-08
1404 8.60518349590222e-08
1405 8.62974652466875e-08
1406 8.59781126409942e-08
1407 8.59902614709362e-08
1408 8.58873472551203e-08
1409 8.58285306923889e-08
1410 8.58368795038089e-08
1411 8.57239557237044e-08
1412 8.57412793076051e-08
1413 8.54940525392145e-08
1414 8.53279082235758e-08
1415 8.53191663221509e-08
1416 8.5151014784568e-08
1417 8.51210750103526e-08
1418 8.49792922230641e-08
1419 8.49312394022306e-08
1420 8.50152270293059e-08
1421 8.49077184210678e-08
1422 8.48281312468835e-08
1423 8.47313824792195e-08
1424 8.46283104363188e-08
1425 8.46030881707094e-08
1426 8.44901244909657e-08
1427 8.42869707229354e-08
1428 8.43629282951675e-08
1429 8.42870857775679e-08
1430 8.40115714018097e-08
1431 8.39810136921138e-08
1432 8.40693248376212e-08
1433 8.37995000524394e-08
1434 8.38681609383052e-08
1435 8.36686474992021e-08
1436 8.37365843739946e-08
1437 8.36763194396184e-08
1438 8.34690236648505e-08
1439 8.34219236489631e-08
1440 8.32799191639033e-08
1441 8.32294754058438e-08
1442 8.32576775526839e-08
1443 8.30319631353049e-08
1444 8.30114231176182e-08
1445 8.2924158902209e-08
1446 8.29269360536067e-08
1447 8.28784578108532e-08
1448 8.2734574162302e-08
1449 8.27496635080749e-08
1450 8.28148783331706e-08
1451 8.27225329582681e-08
1452 8.23384803174676e-08
1453 8.26050416051771e-08
1454 8.22565781528617e-08
1455 8.25349582389556e-08
1456 8.23445866058314e-08
1457 8.22293847733135e-08
1458 8.21004104487599e-08
1459 8.205965535657e-08
1460 8.19868429484671e-08
1461 8.19049191997934e-08
1462 8.18411173120204e-08
1463 8.17490460240933e-08
1464 8.16282133802026e-08
1465 8.15965408400565e-08
1466 8.15774665956503e-08
1467 8.14763389316298e-08
1468 8.12889197683297e-08
1469 8.13696932784325e-08
1470 8.12156045344459e-08
1471 8.09276227125011e-08
1472 8.11245255682991e-08
1473 8.10654040899195e-08
1474 8.09957131484573e-08
1475 8.08070246254289e-08
1476 8.07455075917396e-08
1477 8.05836456994058e-08
1478 8.06776746458127e-08
1479 8.06172849774356e-08
1480 8.05014292577511e-08
1481 8.02404750830377e-08
1482 8.02827702344189e-08
1483 8.03121896297832e-08
1484 8.03048164894626e-08
1485 8.01343437846214e-08
1486 8.01186196888182e-08
1487 8.00203151598566e-08
1488 7.98849858489881e-08
1489 7.98094531391236e-08
1490 7.97026963970104e-08
1491 7.97733784692056e-08
1492 7.95930978036985e-08
1493 7.94942347925875e-08
1494 7.94226729983194e-08
1495 7.92351076022158e-08
1496 7.9229406964032e-08
1497 7.92586612421431e-08
1498 7.9210077198244e-08
1499 7.91360323209034e-08
1500 7.90693211776272e-08
1501 7.89855961880193e-08
1502 7.88667724180847e-08
1503 7.88834286606921e-08
1504 7.88913125333579e-08
1505 7.86499798346441e-08
1506 7.87640513304488e-08
1507 7.84540581113546e-08
1508 7.849647276581e-08
1509 7.84900931130039e-08
1510 7.84062571845823e-08
1511 7.82908279441941e-08
1512 7.81754167045179e-08
1513 7.82052711443271e-08
1514 7.80346610427252e-08
1515 7.80318879782804e-08
1516 7.79190073836311e-08
1517 7.79103565404782e-08
1518 7.78963818688361e-08
1519 7.76692963944114e-08
1520 7.77950692119056e-08
1521 7.76068312502431e-08
1522 7.75366254184284e-08
1523 7.74748522651691e-08
1524 7.72438775142881e-08
1525 7.74285518043705e-08
1526 7.71561369994345e-08
1527 7.71889377415569e-08
1528 7.71643687724755e-08
1529 7.7031393930227e-08
1530 7.69545411003492e-08
1531 7.6964169996252e-08
1532 7.68070115690911e-08
1533 7.67987106282675e-08
1534 7.63483713144275e-08
1535 7.6716347395589e-08
1536 7.63614509877719e-08
1537 7.63141561699321e-08
1538 7.63324750630012e-08
1539 7.6253872971499e-08
1540 7.62543266117355e-08
1541 7.61177799422086e-08
1542 7.60193927069608e-08
1543 7.58623342651532e-08
1544 7.59800005312172e-08
1545 7.56018743683917e-08
1546 7.58593455874035e-08
1547 7.55035765127587e-08
1548 7.56314431984428e-08
1549 7.54209146789364e-08
1550 7.56638076131466e-08
1551 7.50676477148105e-08
1552 7.54215840181871e-08
1553 7.50596419960559e-08
1554 7.52263162189237e-08
1555 7.49706081761836e-08
1556 7.5075985059847e-08
1557 7.49663117991517e-08
1558 7.48347249812653e-08
1559 7.46992861864193e-08
1560 7.48045248539775e-08
1561 7.45544024125522e-08
1562 7.46716515354073e-08
1563 7.4502422884315e-08
1564 7.44221502593234e-08
1565 7.43695751599382e-08
1566 7.42724116995497e-08
1567 7.43567460492756e-08
1568 7.41733312423776e-08
1569 7.41038047888942e-08
1570 7.39234341242678e-08
1571 7.38665651645043e-08
1572 7.37513549178281e-08
1573 7.39271440828304e-08
1574 7.3810096528959e-08
1575 7.36539344270426e-08
1576 7.36133516987536e-08
1577 7.34937636264199e-08
1578 7.34113732980823e-08
1579 7.33009710756072e-08
1580 7.31785859215961e-08
1581 7.33839690130189e-08
1582 7.3108419474277e-08
1583 7.30625835534404e-08
1584 7.29917514910028e-08
1585 7.27672298586057e-08
1586 7.29165158612766e-08
1587 7.27098058699305e-08
1588 7.26508050110652e-08
1589 7.28849243718521e-08
1590 7.27739119170678e-08
1591 7.22443264296047e-08
1592 7.24413576702609e-08
1593 7.23247248388681e-08
1594 7.23552434620522e-08
1595 7.23208266215636e-08
1596 7.2107271912536e-08
1597 7.21756779102911e-08
1598 7.20150070820935e-08
1599 7.1941940920528e-08
1600 7.18742262253436e-08
1601 7.17497117834753e-08
1602 7.17740960269886e-08
1603 7.17993817493934e-08
1604 7.17446937024135e-08
1605 7.13116433037619e-08
1606 7.14468337239538e-08
1607 7.14358653293168e-08
1608 7.13035933870287e-08
1609 7.12688324568944e-08
1610 7.11834225888275e-08
1611 7.12231156159149e-08
1612 7.11139325235699e-08
1613 7.10655459017495e-08
1614 7.10141297783906e-08
1615 7.06676156925212e-08
1616 7.0788645651465e-08
1617 7.07839394853416e-08
1618 7.0659487949154e-08
1619 7.08130393216599e-08
1620 7.05536206795365e-08
1621 7.04987623549869e-08
1622 7.03157733972226e-08
1623 7.00814454921428e-08
1624 7.0190926804159e-08
1625 6.99542574409406e-08
1626 6.98486750634331e-08
1627 6.97887999825397e-08
1628 6.96791323964163e-08
1629 6.96486206610558e-08
1630 6.94769858609057e-08
1631 6.9402146689157e-08
1632 6.92423790851215e-08
1633 6.92870549676705e-08
1634 6.88390020875396e-08
1635 6.91961680221453e-08
1636 6.89195094634165e-08
1637 6.89679160230661e-08
1638 6.86872140520656e-08
1639 6.86087756767151e-08
1640 6.84817498139978e-08
1641 6.84727563800003e-08
1642 6.83670065195763e-08
1643 6.82449461795187e-08
1644 6.8137350059061e-08
1645 6.80775166697067e-08
1646 6.80135457762354e-08
1647 6.78287432860358e-08
1648 6.76945192283895e-08
1649 6.7621702067644e-08
1650 6.77869924570196e-08
1651 6.7624443304215e-08
1652 6.75555862117427e-08
1653 6.72132862971075e-08
1654 6.76199004483458e-08
1655 6.73826758177398e-08
1656 6.71095949704714e-08
1657 6.69977399665456e-08
1658 6.71197756205721e-08
1659 6.66854296946795e-08
1660 6.6897803709054e-08
1661 6.68473735707664e-08
1662 6.66792734960175e-08
1663 6.67418373581086e-08
1664 6.63925474921889e-08
1665 6.63574085404406e-08
1666 6.64602506379453e-08
1667 6.63019080073823e-08
1668 6.63178950364163e-08
1669 6.63321584881693e-08
1670 6.58078560431541e-08
1671 6.60409839343679e-08
1672 6.60810220658448e-08
1673 6.59381617484911e-08
1674 6.57938789596635e-08
1675 6.57937769186212e-08
1676 6.56647871606353e-08
1677 6.55878607465077e-08
1678 6.54706400791127e-08
1679 6.53903164500313e-08
1680 6.53102556422702e-08
1681 6.5342711229821e-08
1682 6.5332326573575e-08
1683 6.51763642496306e-08
1684 6.51445535835293e-08
1685 6.50518794493848e-08
1686 6.48445554012333e-08
1687 6.48893073948997e-08
1688 6.47322058835442e-08
1689 6.45631096469756e-08
1690 6.4584817149882e-08
1691 6.45634087939051e-08
1692 6.43035986844787e-08
1693 6.45609194798169e-08
1694 6.42299187170181e-08
1695 6.42096425056415e-08
1696 6.4219270334398e-08
1697 6.41014088014202e-08
1698 6.39808856903734e-08
1699 6.39845643592984e-08
1700 6.38907501646635e-08
1701 6.39727766813536e-08
1702 6.36744039721471e-08
1703 6.38004899333744e-08
1704 6.34139433746128e-08
1705 6.35313919681302e-08
1706 6.34585525056686e-08
1707 6.34174762770279e-08
1708 6.31960865633552e-08
1709 6.31240488040419e-08
1710 6.30532932928141e-08
1711 6.30201824836085e-08
1712 6.28275236591591e-08
1713 6.27910704089629e-08
1714 6.27785099847067e-08
1715 6.25812979646589e-08
1716 6.27042888332596e-08
1717 6.24951942427288e-08
1718 6.25071216546758e-08
1719 6.24593639777515e-08
1720 6.22081376757677e-08
1721 6.22073345675211e-08
1722 6.19898118934259e-08
1723 6.20416052163897e-08
1724 6.19966120911641e-08
1725 6.20146342669159e-08
1726 6.18762743278545e-08
1727 6.18465993822426e-08
1728 6.17383420986606e-08
1729 6.18484778125428e-08
1730 6.1744395777108e-08
1731 6.17160139904094e-08
1732 6.15721448204987e-08
1733 6.16102070041258e-08
1734 6.11907435348336e-08
1735 6.12615916422143e-08
1736 6.12607626080397e-08
1737 6.11861938901725e-08
1738 6.10190649612541e-08
1739 6.10093247184196e-08
1740 6.09282871155692e-08
1741 6.10181717068947e-08
1742 6.08868857021605e-08
1743 6.08084162281308e-08
1744 6.07690323914944e-08
1745 6.06269619365385e-08
1746 6.07267153585056e-08
1747 6.07257672355921e-08
1748 6.05484702833614e-08
1749 6.05543811493803e-08
1750 6.05042622821017e-08
1751 6.04116672557886e-08
1752 6.02551451023281e-08
1753 6.01835935269968e-08
1754 6.00572637470265e-08
1755 6.01147908647626e-08
1756 6.00367345424679e-08
1757 5.99943383994272e-08
1758 6.00599670597113e-08
1759 5.99035569806006e-08
1760 5.98550872425996e-08
1761 5.96361179447946e-08
1762 5.96839904352109e-08
1763 5.96168745716774e-08
1764 5.97269884319296e-08
1765 5.95091290089833e-08
1766 5.94108165432594e-08
1767 5.93198714362586e-08
1768 5.93648309377137e-08
1769 5.94101704960437e-08
1770 5.92672342465406e-08
1771 5.93726116027149e-08
1772 5.90869362846469e-08
1773 5.91855508962347e-08
1774 5.90109762912405e-08
1775 5.90131852971076e-08
1776 5.8979214995869e-08
1777 5.88853792669042e-08
1778 5.87305313914932e-08
1779 5.91384652222082e-08
1780 5.88001862897869e-08
1781 5.87910398128777e-08
1782 5.85255963834186e-08
1783 5.88223720234815e-08
1784 5.86011516561236e-08
1785 5.86183486963598e-08
1786 5.86115127227949e-08
1787 5.83808195067093e-08
1788 5.83827012241578e-08
1789 5.83591649259141e-08
1790 5.82646731328573e-08
1791 5.80800546625682e-08
1792 5.82352624816096e-08
1793 5.82184114552575e-08
1794 5.80345607654742e-08
1795 5.79979204302639e-08
1796 5.82486718934128e-08
1797 5.80414299724552e-08
1798 5.79347548588238e-08
1799 5.78054179354304e-08
1800 5.78341418790984e-08
1801 5.77776993471169e-08
1802 5.75926364492219e-08
1803 5.79574915273717e-08
1804 5.74876996628504e-08
1805 5.754194882579e-08
1806 5.7463711460759e-08
1807 5.74627288347784e-08
1808 5.74842390022035e-08
1809 5.7369950411168e-08
1810 5.73636265182031e-08
1811 5.72805136838106e-08
1812 5.7493645533313e-08
1813 5.71470002013186e-08
1814 5.69810089934286e-08
1815 5.71317879090039e-08
1816 5.69651783877134e-08
1817 5.70107216133486e-08
1818 5.68774652243142e-08
1819 5.68442275219461e-08
1820 5.68583112356968e-08
1821 5.6987781897444e-08
1822 5.6928989982552e-08
1823 5.68634933140189e-08
1824 5.65086088917255e-08
1825 5.63396369290814e-08
1826 5.66006557591869e-08
1827 5.6425867493104e-08
1828 5.64573802219748e-08
1829 5.65001917185448e-08
1830 5.63804173698834e-08
1831 5.62585484780165e-08
1832 5.63155592603337e-08
1833 5.62714200174064e-08
1834 5.63579780030743e-08
1835 5.60266123161846e-08
1836 5.59301020768288e-08
1837 5.62080115420471e-08
1838 5.59665152932709e-08
1839 5.59704066329658e-08
1840 5.59349190005776e-08
1841 5.60538430107727e-08
1842 5.58622253938879e-08
1843 5.56356004977054e-08
1844 5.58007151258977e-08
1845 5.57276024757414e-08
1846 5.57422171865163e-08
1847 5.55300961906369e-08
1848 5.57237652203391e-08
1849 5.58213007599839e-08
1850 5.54134891559421e-08
1851 5.54469368867494e-08
1852 5.54113580353288e-08
1853 5.56070035382383e-08
1854 5.55309946177474e-08
1855 5.53596809353962e-08
1856 5.50674695696252e-08
1857 5.51739624272685e-08
1858 5.519938536791e-08
1859 5.48892343896767e-08
1860 5.49918112171532e-08
1861 5.5022853382436e-08
1862 5.50559683420992e-08
1863 5.46609367240158e-08
1864 5.47077487480507e-08
1865 5.47862745658989e-08
1866 5.46905776541706e-08
1867 5.46768221569849e-08
1868 5.46509897763059e-08
1869 5.47002337754066e-08
1870 5.46165884975913e-08
1871 5.44025544346738e-08
1872 5.45138736001149e-08
1873 5.43528694354478e-08
1874 5.44364552563792e-08
1875 5.42655500139766e-08
1876 5.42620439194508e-08
1877 5.43123208824881e-08
1878 5.41044557071935e-08
1879 5.42021766278822e-08
1880 5.41184167870412e-08
1881 5.42049763261332e-08
1882 5.40472347867471e-08
1883 5.40217305085022e-08
1884 5.39589523640416e-08
1885 5.38109817376231e-08
1886 5.3884475956778e-08
1887 5.38079534369018e-08
1888 5.37056908735956e-08
1889 5.37139374854334e-08
1890 5.36947004827759e-08
1891 5.37844509782559e-08
1892 5.34891506429602e-08
1893 5.37364799972906e-08
1894 5.3495371644896e-08
1895 5.35420579415025e-08
1896 5.35452245125079e-08
1897 5.33901079382559e-08
1898 5.34230664945845e-08
1899 5.32973682370041e-08
1900 5.3458199567924e-08
1901 5.30870207460232e-08
1902 5.30364283806151e-08
1903 5.3250177284081e-08
1904 5.31221108541757e-08
1905 5.31099038889593e-08
1906 5.29953729686561e-08
1907 5.30443617821064e-08
1908 5.29177297186045e-08
1909 5.27688171994001e-08
1910 5.29282245560481e-08
1911 5.28104229791104e-08
1912 5.27554574683009e-08
1913 5.24457357951036e-08
1914 5.26512776746202e-08
1915 5.28651287998727e-08
1916 5.24775452297455e-08
1917 5.25889230003074e-08
1918 5.25325934996879e-08
1919 5.2395517416759e-08
1920 5.25371009520992e-08
1921 5.23942957664225e-08
1922 5.24876280341147e-08
1923 5.22042925212851e-08
1924 5.23762517423698e-08
1925 5.2295042753947e-08
1926 5.18997411957933e-08
1927 5.20074057330788e-08
1928 5.20566693809066e-08
1929 5.24507730617074e-08
1930 5.21053428661489e-08
1931 5.21024999453612e-08
1932 5.18922494592289e-08
1933 5.17510498037055e-08
1934 5.19441161186407e-08
1935 5.20984284260173e-08
1936 5.17096520220584e-08
1937 5.18373590514543e-08
1938 5.1797419662325e-08
1939 5.15640684182195e-08
1940 5.15856499005984e-08
1941 5.16164946393616e-08
1942 5.1608361254285e-08
1943 5.15090401540519e-08
1944 5.15716358271412e-08
1945 5.12620152739451e-08
1946 5.15418221538333e-08
1947 5.10885401694416e-08
1948 5.12017384015628e-08
1949 5.12897318274419e-08
1950 5.11519555659312e-08
1951 5.13599067422099e-08
1952 5.1249172187795e-08
1953 5.11166824741505e-08
1954 5.10709825296551e-08
1955 5.11198221715414e-08
1956 5.07369301008254e-08
1957 5.12830467367387e-08
1958 5.0679001100562e-08
1959 5.09075746921717e-08
1960 5.08067419318081e-08
1961 5.07021088775872e-08
1962 5.06760892733382e-08
1963 5.07722286218204e-08
1964 5.09148114002933e-08
1965 5.07339252195926e-08
1966 5.05940258692661e-08
1967 5.04113741359724e-08
1968 5.04967740631024e-08
1969 5.05724183557099e-08
1970 5.04573526987073e-08
1971 5.0150446344599e-08
1972 5.01246961208679e-08
1973 5.01887607087248e-08
1974 5.04022047227437e-08
1975 5.0098131119114e-08
1976 5.01196775570811e-08
1977 5.02016788965776e-08
1978 5.00683527566714e-08
1979 5.01512013957317e-08
1980 4.99581726036702e-08
1981 5.00192781531794e-08
1982 5.00527683646723e-08
1983 4.97754310431198e-08
1984 5.00179453704064e-08
1985 4.96731300696496e-08
1986 4.98523761511827e-08
1987 4.96848401434491e-08
1988 4.98968238993314e-08
1989 4.94577185232714e-08
1990 4.96236991520682e-08
1991 4.92991040954571e-08
1992 4.96093736519487e-08
1993 4.93972731714365e-08
1994 4.94233284484835e-08
1995 4.96267843805853e-08
1996 4.91556403239279e-08
1997 4.96059877548305e-08
1998 4.94041991796479e-08
1999 4.93606685143533e-08
2000 4.94508462052234e-08
2001 4.93974699380395e-08
2002 4.92071244320158e-08
2003 4.91224981566951e-08
2004 4.92879094453613e-08
2005 4.90735965597011e-08
2006 4.89144676758535e-08
2007 4.87595996983003e-08
2008 4.90756247546464e-08
2009 4.88900888009347e-08
2010 4.87745150865582e-08
2011 4.86352602231399e-08
2012 4.87579240204816e-08
2013 4.86301132527345e-08
2014 4.83020110229049e-08
2015 4.85021510052697e-08
2016 4.870522800271e-08
2017 4.83136117437333e-08
2018 4.84454972311532e-08
2019 4.84840039836065e-08
2020 4.85742232081421e-08
2021 4.81871583752902e-08
2022 4.82778206101209e-08
2023 4.82940009929322e-08
2024 4.83797451638068e-08
2025 4.84657957149359e-08
2026 4.81945328272282e-08
2027 4.79930085939273e-08
2028 4.79543918854297e-08
2029 4.8154850075921e-08
2030 4.79363767527108e-08
2031 4.75947366518348e-08
2032 4.8016013537211e-08
2033 4.80525516453056e-08
2034 4.77861640271193e-08
2035 4.77253857882332e-08
2036 4.77969944217005e-08
2037 4.75412257712193e-08
2038 4.77307008526218e-08
2039 4.7447952019164e-08
2040 4.74516378776713e-08
2041 4.76098862218777e-08
2042 4.73866667283218e-08
2043 4.72744445585427e-08
2044 4.77448702658201e-08
2045 4.76439726568856e-08
2046 4.73618248979513e-08
2047 4.73664710192168e-08
2048 4.70036071265056e-08
2049 4.70433706927231e-08
2050 4.74535952896815e-08
2051 4.70521390538714e-08
2052 4.73438699897244e-08
2053 4.70392186797675e-08
2054 4.70946917068282e-08
2055 4.68886401119306e-08
2056 4.69430438401197e-08
2057 4.68936069202286e-08
2058 4.69032205381303e-08
2059 4.68007221867683e-08
2060 4.65382708143558e-08
2061 4.69431803113984e-08
2062 4.6422078010977e-08
2063 4.63668593244648e-08
2064 4.64600919629632e-08
2065 4.62390046378491e-08
2066 4.64637353028152e-08
2067 4.64654284604293e-08
2068 4.61841626555604e-08
2069 4.62867908408438e-08
2070 4.60661608570057e-08
2071 4.64429573034941e-08
2072 4.59947905540048e-08
2073 4.61020276278923e-08
2074 4.61636989226299e-08
2075 4.61267917148955e-08
2076 4.57112341283317e-08
2077 4.58543699626279e-08
2078 4.57684975407879e-08
2079 4.60454935029553e-08
2080 4.59814484798926e-08
2081 4.57277251693089e-08
2082 4.56201977876436e-08
2083 4.57043619199737e-08
2084 4.60201521927939e-08
2085 4.55735976636618e-08
2086 4.56495329230755e-08
2087 4.56923545757437e-08
2088 4.55031476420675e-08
2089 4.57038918506569e-08
2090 4.56270427826588e-08
2091 4.54246509682399e-08
2092 4.56047028545292e-08
2093 4.53623473939402e-08
2094 4.5335112004663e-08
2095 4.52264046049144e-08
2096 4.50054434937286e-08
2097 4.54385562971282e-08
2098 4.52071506109597e-08
2099 4.50135421923292e-08
2100 4.5253903776743e-08
2101 4.51444855316119e-08
2102 4.51576029700806e-08
2103 4.46957013160443e-08
2104 4.49526635590836e-08
2105 4.50092949479952e-08
2106 4.50080782063988e-08
2107 4.45663990313872e-08
2108 4.47444851885415e-08
2109 4.45226810050769e-08
2110 4.46454950817632e-08
2111 4.48981661702597e-08
2112 4.44439802023133e-08
2113 4.45812668772305e-08
2114 4.44334123168577e-08
2115 4.45847653520737e-08
2116 4.43360117650116e-08
2117 4.40715759282284e-08
2118 4.43603498121448e-08
2119 4.39368314246735e-08
2120 4.40828665744242e-08
2121 4.41467110656735e-08
2122 4.38016868118396e-08
2123 4.40169419615444e-08
2124 4.36974139650381e-08
2125 4.39759238575199e-08
2126 4.36264297363209e-08
2127 4.36960129235331e-08
2128 4.37696554813627e-08
2129 4.36563132595325e-08
2130 4.34787538492465e-08
2131 4.35604141511003e-08
2132 4.34942383928938e-08
2133 4.34705109302325e-08
2134 4.35716780593509e-08
2135 4.33004029973816e-08
2136 4.32078359082766e-08
2137 4.35509563105896e-08
2138 4.31137216092559e-08
2139 4.33766317196049e-08
2140 4.30665818802467e-08
2141 4.33133584500922e-08
2142 4.29109249040494e-08
2143 4.30897280396803e-08
2144 4.30482639139917e-08
2145 4.30929253036894e-08
2146 4.28394240736818e-08
2147 4.30623493827831e-08
2148 4.27911144367243e-08
2149 4.29850119201269e-08
2150 4.26773064210462e-08
2151 4.28175751756577e-08
2152 4.26351009550352e-08
2153 4.28395373031076e-08
2154 4.26025690249965e-08
2155 4.27210672473422e-08
2156 4.26878403461117e-08
2157 4.25220619781896e-08
2158 4.25031196775727e-08
2159 4.24025308407572e-08
2160 4.23822862831713e-08
2161 4.25299227639897e-08
2162 4.23327446852362e-08
2163 4.22982478176515e-08
2164 4.21332814313224e-08
2165 4.21994336607945e-08
2166 4.22140030189588e-08
2167 4.22307225882435e-08
2168 4.22667500656892e-08
2169 4.19017442734315e-08
2170 4.20190090122397e-08
2171 4.19755534797339e-08
2172 4.2082648070707e-08
2173 4.19090592642668e-08
2174 4.18482998749692e-08
2175 4.18702315876374e-08
2176 4.18495606340219e-08
2177 4.15707093370799e-08
2178 4.18682227738554e-08
2179 4.166742951206e-08
2180 4.20760969044309e-08
2181 4.15040083738827e-08
2182 4.16300549248572e-08
2183 4.18916932454216e-08
2184 4.16555506492777e-08
2185 4.16016953812548e-08
2186 4.15083785063608e-08
2187 4.13752713015469e-08
2188 4.13989242797896e-08
2189 4.16442455843935e-08
2190 4.13545527737647e-08
2191 4.12905589941559e-08
2192 4.12305982937511e-08
2193 4.13851901657214e-08
2194 4.10134606338985e-08
2195 4.11826293713613e-08
2196 4.13281844078295e-08
2197 4.12726705514288e-08
2198 4.10315638375014e-08
2199 4.11469971757228e-08
2200 4.10711463501023e-08
2201 4.10852968264486e-08
2202 4.07978753966809e-08
2203 4.07849891739254e-08
2204 4.07378252198498e-08
2205 4.05655948434447e-08
2206 4.05701076182652e-08
2207 4.05951090898249e-08
2208 4.05312947320269e-08
2209 4.04975673293784e-08
2210 4.04697561193768e-08
2211 4.0429017694521e-08
2212 4.03639427999369e-08
2213 4.0298513151793e-08
2214 4.02158021071131e-08
2215 4.04039626655806e-08
2216 4.01067151143408e-08
2217 4.01984234392749e-08
2218 4.00507474738188e-08
2219 4.01034694714753e-08
2220 3.99967038522409e-08
2221 4.01047620606221e-08
2222 3.99602197624915e-08
2223 3.99375241504352e-08
2224 3.99425873882642e-08
2225 3.9881469667824e-08
2226 3.98491721860239e-08
2227 3.99310294043609e-08
2228 3.9659302673245e-08
2229 3.98352253072609e-08
2230 3.96855659552386e-08
2231 3.95437303624124e-08
2232 3.96369614024117e-08
2233 3.94381031643132e-08
2234 3.93893469314843e-08
2235 3.94060079811354e-08
2236 3.94267514944868e-08
2237 3.96296954510156e-08
2238 3.93112575360899e-08
2239 3.92945243583576e-08
2240 3.932517397498e-08
2241 3.91676114874784e-08
2242 3.91881611698786e-08
2243 3.91312532896926e-08
2244 3.90221368165022e-08
2245 3.90390239048966e-08
2246 3.90703525052771e-08
2247 3.89791707198217e-08
2248 3.9251231428894e-08
2249 3.89662446318439e-08
2250 3.88506510056175e-08
2251 3.89083121157263e-08
2252 3.88832883384538e-08
2253 3.89401462050909e-08
2254 3.86062877106852e-08
2255 3.85921635013275e-08
2256 3.87648108781669e-08
2257 3.85473116335522e-08
2258 3.85563809524037e-08
2259 3.84706245089017e-08
2260 3.85889485516877e-08
2261 3.860666741895e-08
2262 3.8551013912258e-08
2263 3.83550958842349e-08
2264 3.86174301016951e-08
2265 3.83413458289183e-08
2266 3.8447865077007e-08
2267 3.80641031867413e-08
2268 3.8412357528772e-08
2269 3.83958799867745e-08
2270 3.82190268275462e-08
2271 3.83232664913358e-08
2272 3.8337691570911e-08
2273 3.81325006229005e-08
2274 3.83506987560178e-08
2275 3.78795641684526e-08
2276 3.79339787008881e-08
2277 3.77651813407986e-08
2278 3.78493216199871e-08
2279 3.78531861211506e-08
2280 3.77625881589072e-08
2281 3.76189226387602e-08
2282 3.76873794536436e-08
2283 3.7590454087244e-08
2284 3.75692965977059e-08
2285 3.77406961751259e-08
2286 3.74833455496493e-08
2287 3.74221285146525e-08
2288 3.76004011286568e-08
2289 3.74443456312434e-08
2290 3.7422124663733e-08
2291 3.72840553795051e-08
2292 3.73524320895768e-08
2293 3.71539107721919e-08
2294 3.73811331104701e-08
2295 3.73646894806789e-08
2296 3.72398018306797e-08
2297 3.72634489922419e-08
2298 3.70246180605438e-08
2299 3.71737376578807e-08
2300 3.70975825876307e-08
2301 3.71352402885883e-08
2302 3.68811630240273e-08
2303 3.70137614225108e-08
2304 3.70192037402184e-08
2305 3.69396545094602e-08
2306 3.66985607826997e-08
2307 3.68698148416335e-08
2308 3.67036278507982e-08
2309 3.67004009067529e-08
2310 3.68016420844164e-08
2311 3.65906156800655e-08
2312 3.66867473018395e-08
2313 3.66838870813346e-08
2314 3.66165256653339e-08
2315 3.65963076491838e-08
2316 3.65916295237412e-08
2317 3.65407610396495e-08
2318 3.63484543361992e-08
2319 3.65808584170679e-08
2320 3.62858827567258e-08
2321 3.65091907688075e-08
2322 3.63320229190833e-08
2323 3.64357718516395e-08
2324 3.6380397062219e-08
2325 3.60679236979244e-08
2326 3.62363802275389e-08
2327 3.62045190880345e-08
2328 3.61748804200701e-08
2329 3.61081611384151e-08
2330 3.61909358712609e-08
2331 3.61537618180918e-08
2332 3.60060013591124e-08
2333 3.59183136415808e-08
2334 3.58976887095963e-08
2335 3.59503711457876e-08
2336 3.56913697330263e-08
2337 3.57384723397303e-08
2338 3.56857289198764e-08
2339 3.56694606877816e-08
2340 3.57764516376591e-08
2341 3.58383671841267e-08
2342 3.57145821823224e-08
2343 3.55062520918104e-08
2344 3.57128159502729e-08
2345 3.55889222567463e-08
2346 3.55715563604697e-08
2347 3.55603973023477e-08
2348 3.54496680048388e-08
2349 3.54440871661588e-08
2350 3.55987295264093e-08
2351 3.54475603336901e-08
2352 3.5363442079106e-08
2353 3.51956118258556e-08
2354 3.53108831121496e-08
2355 3.50980712782167e-08
2356 3.52909799876677e-08
2357 3.52577114899333e-08
2358 3.52030695203176e-08
2359 3.51381403520534e-08
2360 3.4913146322868e-08
2361 3.49528885030814e-08
2362 3.51526312436512e-08
2363 3.47962502469379e-08
2364 3.48923256008327e-08
2365 3.49714195673201e-08
2366 3.49391881480088e-08
2367 3.48685064239795e-08
2368 3.48834807741838e-08
2369 3.45755443476037e-08
2370 3.46819432510692e-08
2371 3.47412781676049e-08
2372 3.45843301654547e-08
2373 3.46679150013962e-08
2374 3.45216286499195e-08
2375 3.47285834125799e-08
2376 3.46104507631395e-08
2377 3.45536309107608e-08
2378 3.45006494346656e-08
2379 3.43701391138929e-08
2380 3.44203731637016e-08
2381 3.42268139679192e-08
2382 3.44101768461336e-08
2383 3.43383471888004e-08
2384 3.4386508491302e-08
2385 3.4211732640177e-08
2386 3.42511685658131e-08
2387 3.41761459643486e-08
2388 3.431537463916e-08
2389 3.41590441634843e-08
2390 3.40455063208012e-08
2391 3.40202623463348e-08
2392 3.40223799861761e-08
2393 3.4004418043132e-08
2394 3.39442764865083e-08
2395 3.39565947005127e-08
2396 3.38759212872297e-08
2397 3.38069526879536e-08
2398 3.3913764094784e-08
2399 3.38089339391168e-08
2400 3.39422509068044e-08
2401 3.36906414961646e-08
2402 3.3955243644801e-08
2403 3.36252649484337e-08
2404 3.38483330608597e-08
2405 3.36415065969042e-08
2406 3.36957129523086e-08
2407 3.35306614154263e-08
2408 3.37863758215207e-08
2409 3.35697086999431e-08
2410 3.35067842507364e-08
2411 3.33711738980114e-08
2412 3.34192201374428e-08
2413 3.33409441353183e-08
2414 3.34326243249805e-08
2415 3.32919871883774e-08
2416 3.34089956059991e-08
2417 3.32977015014002e-08
2418 3.34221069460128e-08
2419 3.31901211096941e-08
2420 3.33382303061569e-08
2421 3.31264712318813e-08
2422 3.34315236463212e-08
2423 3.33254056967824e-08
2424 3.31849593049327e-08
2425 3.31173018595088e-08
2426 3.30422557830445e-08
2427 3.31819253558407e-08
2428 3.30689351648594e-08
2429 3.28515195384682e-08
2430 3.30028069290833e-08
2431 3.28570824716934e-08
2432 3.28559528863792e-08
2433 3.28448033126705e-08
2434 3.27557961576908e-08
2435 3.27872675236485e-08
2436 3.26160164525469e-08
2437 3.27652833056469e-08
2438 3.2645938976561e-08
2439 3.28018277104825e-08
2440 3.27655873868515e-08
2441 3.28657621513351e-08
2442 3.28831704541344e-08
2443 3.25147632933565e-08
2444 3.243593565605e-08
2445 3.27105986759335e-08
2446 3.25247189412181e-08
2447 3.24606943311601e-08
2448 3.24096786292749e-08
2449 3.24640406419441e-08
2450 3.23963059223331e-08
2451 3.24952529595635e-08
2452 3.23569765536824e-08
2453 3.23272562243737e-08
2454 3.22842033271264e-08
2455 3.22668336223408e-08
2456 3.23674906934635e-08
2457 3.21988006586338e-08
2458 3.22599177460248e-08
2459 3.23331201699251e-08
2460 3.22136478536628e-08
2461 3.20699672524682e-08
2462 3.20648431211268e-08
2463 3.21377264320688e-08
2464 3.2127670112736e-08
2465 3.20360413446075e-08
2466 3.23352085693784e-08
2467 3.20554347741364e-08
2468 3.20766236296954e-08
2469 3.18504489238425e-08
2470 3.19293555017808e-08
2471 3.18462047710355e-08
2472 3.18732739934546e-08
2473 3.17777950127329e-08
2474 3.1786181219573e-08
2475 3.17295141991636e-08
2476 3.16333716856576e-08
2477 3.20693745856637e-08
2478 3.17184058356013e-08
2479 3.17004250018904e-08
2480 3.17124942572633e-08
2481 3.17981888695584e-08
2482 3.15307306606449e-08
2483 3.15691017278574e-08
2484 3.16627113949863e-08
2485 3.1922607142354e-08
2486 3.16693578179539e-08
2487 3.15486187700831e-08
2488 3.14762355135834e-08
2489 3.16393784172142e-08
2490 3.16039597887396e-08
2491 3.13986805302591e-08
2492 3.13312276913891e-08
2493 3.13335628399081e-08
2494 3.13234325246547e-08
2495 3.13311351232137e-08
2496 3.13111841783176e-08
2497 3.12885116593442e-08
2498 3.12515051414941e-08
2499 3.12672144548998e-08
2500 3.13641311140422e-08
2501 3.12836326883392e-08
2502 3.11025710011403e-08
2503 3.1471125679694e-08
2504 3.10939471930283e-08
2505 3.10194573767131e-08
2506 3.12997231035617e-08
2507 3.09931192796764e-08
2508 3.08283688814637e-08
2509 3.090146372009e-08
2510 3.09760257486413e-08
2511 3.0619991023384e-08
2512 3.08363536338785e-08
2513 3.05962913229596e-08
2514 3.07165750192961e-08
2515 3.04847531835684e-08
2516 3.0626893155361e-08
2517 3.04946102986925e-08
2518 3.05278925993857e-08
2519 3.04329012219551e-08
2520 3.04691876806817e-08
2521 3.07183757164431e-08
2522 3.05768679309537e-08
2523 3.04180235195517e-08
2524 3.02288699187692e-08
2525 3.01587709070628e-08
2526 3.02607163487068e-08
2527 3.02412624058057e-08
2528 3.01124919750428e-08
2529 2.99829449574851e-08
2530 3.02043239299721e-08
2531 3.00602640481706e-08
2532 3.00012078497947e-08
2533 2.98392811326487e-08
2534 2.99374430268173e-08
2535 2.97165742799699e-08
2536 2.99489092174809e-08
2537 2.96320191957022e-08
2538 2.96964129125055e-08
2539 2.96776630186457e-08
2540 2.9592740819373e-08
2541 2.95294483494413e-08
2542 2.94873796617967e-08
2543 2.94233509401387e-08
2544 2.94878663369413e-08
2545 2.93053503128693e-08
2546 2.93315346455714e-08
2547 2.9176413314902e-08
2548 2.92204978233901e-08
2549 2.9194896099094e-08
2550 2.91047914076081e-08
2551 2.90860093377088e-08
2552 2.90589749778469e-08
2553 2.93098290651272e-08
2554 2.92445642655625e-08
2555 2.89512363738087e-08
2556 2.87472110984677e-08
2557 2.88117497058149e-08
2558 2.86727101961315e-08
2559 2.88363496647737e-08
2560 2.86638276847384e-08
2561 2.8779031457038e-08
2562 2.88623375888086e-08
2563 2.84702722945784e-08
2564 2.84764627338507e-08
2565 2.85676800579893e-08
2566 2.8472400618984e-08
2567 2.84146559066389e-08
2568 2.84207513547852e-08
2569 2.83806162257871e-08
2570 2.82440255283678e-08
2571 2.82559548769878e-08
2572 2.82062941139305e-08
2573 2.8344118357726e-08
2574 2.81440269973565e-08
2575 2.84000229904269e-08
2576 2.83177025441894e-08
2577 2.81640791504056e-08
2578 2.81472896999091e-08
2579 2.8071855382894e-08
2580 2.81017790271232e-08
2581 2.80602533333507e-08
2582 2.82072861359506e-08
2583 2.7795820625176e-08
2584 2.82524711003873e-08
2585 2.76422521541253e-08
2586 2.80187100825113e-08
2587 2.77397364802923e-08
2588 2.80525967948364e-08
2589 2.76610551350753e-08
2590 2.77049899699566e-08
2591 2.76344335135015e-08
2592 2.80354109856162e-08
2593 2.74212940911678e-08
2594 2.7677420005201e-08
2595 2.77511582718049e-08
2596 2.75907358138205e-08
2597 2.75597819434648e-08
2598 2.76982306426365e-08
2599 2.74986764394747e-08
2600 2.76603674906895e-08
2601 2.75463719057178e-08
2602 2.76729682986421e-08
2603 2.71826362576721e-08
2604 2.75786595451688e-08
2605 2.7270840741167e-08
2606 2.71090406966934e-08
2607 2.75028502658703e-08
2608 2.70512432052339e-08
2609 2.73580675334006e-08
2610 2.71036941510427e-08
2611 2.71953976771488e-08
2612 2.7182391282965e-08
2613 2.71113177923166e-08
2614 2.69053933439345e-08
2615 2.72097166291729e-08
2616 2.68070596831649e-08
2617 2.70745572941156e-08
2618 2.69150003884278e-08
2619 2.69083960391381e-08
2620 2.69823768039856e-08
2621 2.69067121749789e-08
2622 2.69112586219755e-08
2623 2.71278215464577e-08
2624 2.68382715673532e-08
2625 2.68556654279273e-08
2626 2.67193566458879e-08
2627 2.68754729979559e-08
2628 2.67442949417607e-08
2629 2.67592372924508e-08
2630 2.68245971086767e-08
2631 2.6764109768207e-08
2632 2.67239933691155e-08
2633 2.65035055675433e-08
2634 2.66866930236365e-08
2635 2.65753609047792e-08
2636 2.66332809266689e-08
2637 2.66181373427266e-08
2638 2.64840217154561e-08
2639 2.6476515001761e-08
2640 2.65569382420505e-08
2641 2.64022498626115e-08
2642 2.65072175593684e-08
2643 2.65126467249299e-08
2644 2.63538944782482e-08
2645 2.64300677104146e-08
2646 2.63766447325953e-08
2647 2.65561127650304e-08
2648 2.6421471258864e-08
2649 2.62951695708669e-08
2650 2.64091059187788e-08
2651 2.61508262420218e-08
2652 2.64880427440772e-08
2653 2.62899565575125e-08
2654 2.63013667278766e-08
2655 2.6294397748261e-08
2656 2.62063476237362e-08
2657 2.62821462833074e-08
2658 2.63132659801268e-08
2659 2.62116843900984e-08
2660 2.61540441044428e-08
2661 2.6243553823635e-08
2662 2.61234943532696e-08
2663 2.62575376466234e-08
2664 2.6000726411235e-08
2665 2.61977301001526e-08
2666 2.60834053238401e-08
2667 2.62255113749887e-08
2668 2.60898328150461e-08
2669 2.61995046955121e-08
2670 2.59382403375863e-08
2671 2.60444401354309e-08
2672 2.60961859848141e-08
2673 2.59712677252999e-08
2674 2.6089069603108e-08
2675 2.59269275955099e-08
2676 2.606228339741e-08
2677 2.60122212498715e-08
2678 2.60316380757963e-08
2679 2.58325513944335e-08
2680 2.59541000682262e-08
2681 2.58519061753226e-08
2682 2.59506962045908e-08
2683 2.57401073657793e-08
2684 2.59182139290548e-08
2685 2.58818589706777e-08
2686 2.5615690074865e-08
2687 2.58607478953454e-08
2688 2.58395574530557e-08
2689 2.57925141651238e-08
2690 2.55385115959239e-08
2691 2.57936080738652e-08
2692 2.56948465260365e-08
2693 2.56863257195317e-08
2694 2.5648588389382e-08
2695 2.57164871331206e-08
2696 2.55983826487061e-08
2697 2.56523031816336e-08
2698 2.56296299567804e-08
2699 2.56545156096522e-08
2700 2.54044795595654e-08
2701 2.56086864551808e-08
2702 2.55218742879482e-08
2703 2.56716243354216e-08
2704 2.53370629612615e-08
2705 2.56551488819756e-08
2706 2.55466321166242e-08
2707 2.54116542850458e-08
2708 2.54119590539226e-08
2709 2.55525530052925e-08
2710 2.52765125507981e-08
2711 2.55042040364639e-08
2712 2.51586095669776e-08
2713 2.5477543834862e-08
2714 2.52169711192352e-08
2715 2.53236895835229e-08
2716 2.52962586873018e-08
2717 2.52999938887832e-08
2718 2.49370583709219e-08
2719 2.5249428912355e-08
2720 2.49931929041658e-08
2721 2.53672169221453e-08
2722 2.50168582018873e-08
2723 2.5256868053658e-08
2724 2.49065494009049e-08
2725 2.52115850836176e-08
2726 2.48568922511794e-08
2727 2.51127648625005e-08
2728 2.48516331406989e-08
2729 2.51252239200639e-08
2730 2.47425717991856e-08
2731 2.50557648180383e-08
2732 2.47105112869406e-08
2733 2.49741632669664e-08
2734 2.4724572558199e-08
2735 2.48260320721005e-08
2736 2.47123229912827e-08
2737 2.48859465528106e-08
2738 2.45979342827773e-08
2739 2.48509276892239e-08
2740 2.45459097085732e-08
2741 2.47572094409509e-08
2742 2.44751788771591e-08
2743 2.46981156348802e-08
2744 2.44703801746127e-08
2745 2.42781615722443e-08
2746 2.4666370484816e-08
2747 2.44689943715848e-08
2748 2.45353748655752e-08
2749 2.45268448753055e-08
2750 2.42744352958457e-08
2751 2.44215200986275e-08
2752 2.44233249819814e-08
2753 2.44100403707925e-08
2754 2.41830943927823e-08
2755 2.42137156580835e-08
2756 2.41953720907517e-08
2757 2.40797873436449e-08
2758 2.42666173042583e-08
2759 2.41012077852876e-08
2760 2.42643985723934e-08
2761 2.40579645127692e-08
2762 2.40377426790062e-08
2763 2.41829452400921e-08
2764 2.4064678289637e-08
2765 2.39457407362753e-08
2766 2.40772301312031e-08
2767 2.40722380719927e-08
2768 2.38380120169968e-08
2769 2.40050929443125e-08
2770 2.38689329044028e-08
2771 2.36619880531475e-08
2772 2.37946007037859e-08
2773 2.36468878760387e-08
2774 2.371229203324e-08
2775 2.37357439114572e-08
2776 2.36200998868785e-08
2777 2.37536494596124e-08
2778 2.35635367449305e-08
2779 2.35260003482729e-08
2780 2.35890035673592e-08
2781 2.36096065018288e-08
2782 2.35385435856106e-08
2783 2.36076872768276e-08
2784 2.35301393218812e-08
2785 2.33732073715576e-08
2786 2.35387138802778e-08
2787 2.33765555899268e-08
2788 2.34368392761919e-08
2789 2.35266229231534e-08
2790 2.33990208122758e-08
2791 2.32574275946806e-08
2792 2.33687779787051e-08
2793 2.3259893945804e-08
2794 2.35324522888103e-08
2795 2.32553107053501e-08
2796 2.32881672173058e-08
2797 2.31005821598185e-08
2798 2.33121999706487e-08
2799 2.31560460244307e-08
2800 2.32634538195331e-08
2801 2.32214208235604e-08
2802 2.31365842486841e-08
2803 2.31544856750254e-08
2804 2.31272857349474e-08
2805 2.31259822802699e-08
2806 2.3110235511048e-08
2807 2.31610095224877e-08
2808 2.29825287920704e-08
2809 2.30671384242598e-08
2810 2.31090420914626e-08
2811 2.29215790723547e-08
2812 2.30481557801365e-08
2813 2.31056636512683e-08
2814 2.30891954524637e-08
2815 2.30004938601702e-08
2816 2.2959012830448e-08
2817 2.30315348150878e-08
2818 2.28307143488315e-08
2819 2.30214573109233e-08
2820 2.28413158125385e-08
2821 2.29441863563995e-08
2822 2.28260129071511e-08
2823 2.29881133235743e-08
2824 2.26926152324669e-08
2825 2.2996530536501e-08
2826 2.27514305670695e-08
2827 2.2809166569715e-08
2828 2.28540182978243e-08
2829 2.30035514809934e-08
2830 2.27659600744445e-08
2831 2.2824505310659e-08
2832 2.27776229384524e-08
2833 2.2836921604874e-08
2834 2.27497578653146e-08
2835 2.2844336413419e-08
2836 2.2807121002888e-08
2837 2.26491733494871e-08
2838 2.26339909608075e-08
2839 2.26925224093844e-08
2840 2.27730745778754e-08
2841 2.26647368135602e-08
2842 2.26296446002028e-08
2843 2.25846783896699e-08
2844 2.25907932722702e-08
2845 2.25999726874981e-08
2846 2.25479837701048e-08
2847 2.26138114090002e-08
2848 2.2542416741933e-08
2849 2.27213360832756e-08
2850 2.26349979957075e-08
2851 2.25056632008336e-08
2852 2.23353564603457e-08
2853 2.25866420939802e-08
2854 2.2619719272754e-08
2855 2.25496473351683e-08
2856 2.25141176628529e-08
2857 2.24106398571955e-08
2858 2.24663132095682e-08
2859 2.23983937210903e-08
2860 2.24419492356187e-08
2861 2.24502563401785e-08
2862 2.23228852667123e-08
2863 2.23988827130484e-08
2864 2.22112555032794e-08
2865 2.24189183337664e-08
2866 2.21959817261475e-08
2867 2.24399909147799e-08
2868 2.23030748647268e-08
2869 2.22233629483259e-08
2870 2.21474948371991e-08
2871 2.22789512358545e-08
2872 2.22825205913324e-08
2873 2.22652233992093e-08
2874 2.21329003269322e-08
2875 2.24184631583135e-08
2876 2.22073571813919e-08
2877 2.22265759173279e-08
2878 2.22689593540881e-08
2879 2.22954429409672e-08
2880 2.22585519200624e-08
2881 2.2030894331726e-08
2882 2.22095676867262e-08
2883 2.21657440422351e-08
2884 2.21533533777407e-08
2885 2.21690786048612e-08
2886 2.21281864605238e-08
2887 2.20969698141449e-08
2888 2.21142558745058e-08
2889 2.20774300441207e-08
2890 2.20787158822233e-08
2891 2.19983267628621e-08
2892 2.20491144746759e-08
2893 2.20596773652382e-08
2894 2.1999141855078e-08
2895 2.19380550916526e-08
2896 2.2116774892611e-08
2897 2.20552810816788e-08
2898 2.21679196559244e-08
2899 2.22407448984274e-08
2900 2.19489200330436e-08
2901 2.19345440071184e-08
2902 2.20148278740062e-08
2903 2.19314847995644e-08
2904 2.18588494051986e-08
2905 2.1869695427279e-08
2906 2.18987359650313e-08
2907 2.18307962356334e-08
2908 2.18240683609316e-08
2909 2.17613967385955e-08
2910 2.20614823500664e-08
2911 2.17051162580972e-08
2912 2.19506263958724e-08
2913 2.17002145128653e-08
2914 2.18731398504701e-08
2915 2.1901101517896e-08
2916 2.17982806114403e-08
2917 2.18403533753797e-08
2918 2.18221577761302e-08
2919 2.17854797774919e-08
2920 2.17531063253507e-08
2921 2.18360508656978e-08
2922 2.16024710080198e-08
2923 2.1706468518623e-08
2924 2.17469932040082e-08
2925 2.16550211582511e-08
2926 2.17962689927909e-08
2927 2.16771305519003e-08
2928 2.16418025051901e-08
2929 2.17398852031536e-08
2930 2.16607447180994e-08
2931 2.16200789875387e-08
2932 2.16329168754026e-08
2933 2.16425635692907e-08
2934 2.16080323247603e-08
2935 2.16450838128601e-08
2936 2.16118870166593e-08
2937 2.16020283518858e-08
2938 2.14665680744464e-08
2939 2.16953843681367e-08
2940 2.14903230120189e-08
2941 2.16445101919316e-08
2942 2.16327984310372e-08
2943 2.14280401455991e-08
2944 2.14507077687109e-08
2945 2.14868425676418e-08
2946 2.14663371322921e-08
2947 2.14242062122061e-08
2948 2.14893828687224e-08
2949 2.14463352219418e-08
2950 2.13874170733597e-08
2951 2.14627311794224e-08
2952 2.14060599246224e-08
2953 2.13812019596027e-08
2954 2.14039126822385e-08
2955 2.1556306277315e-08
2956 2.13756746476612e-08
2957 2.14056011011987e-08
2958 2.14817671158052e-08
2959 2.14045413680086e-08
2960 2.13416232979213e-08
2961 2.12694152834647e-08
2962 2.146655144597e-08
2963 2.13090282947803e-08
2964 2.1398338920342e-08
2965 2.1288219082205e-08
2966 2.14126483724097e-08
2967 2.12223572790826e-08
2968 2.13785692144697e-08
2969 2.11916069221552e-08
2970 2.13317983113193e-08
2971 2.12602763665082e-08
2972 2.11924569817334e-08
2973 2.11878199096738e-08
2974 2.12433819932745e-08
2975 2.12031906876575e-08
2976 2.13411803968722e-08
2977 2.10974342329529e-08
2978 2.1231543078537e-08
2979 2.11775311984574e-08
2980 2.13472901602341e-08
2981 2.11202689310852e-08
2982 2.10952233845596e-08
2983 2.12360646811849e-08
2984 2.11304783492849e-08
2985 2.11342223823685e-08
2986 2.10964194837793e-08
2987 2.12179809224722e-08
2988 2.11364420807936e-08
2989 2.10793014490118e-08
2990 2.10677252623359e-08
2991 2.12305774598409e-08
2992 2.11743278768495e-08
2993 2.10248292520365e-08
2994 2.10853925901588e-08
2995 2.11108140821814e-08
2996 2.12441444413791e-08
2997 2.09955917718219e-08
2998 2.10524168804405e-08
2999 2.10043334756271e-08
3000 1.20255561248117e-08
3001 1.19033150908221e-08
3002 1.19678265467771e-08
3003 1.20036410695412e-08
3004 1.20155308742931e-08
3005 1.20149670541081e-08
3006 1.20103331195109e-08
3007 1.2005061366227e-08
3008 1.20000244762108e-08
3009 1.19944571954078e-08
3010 1.19899405973678e-08
3011 1.19852341347593e-08
3012 1.1981894113805e-08
3013 1.19765020989981e-08
3014 1.19745199394505e-08
3015 1.19693864946979e-08
3016 1.19645307725125e-08
3017 1.1962038243718e-08
3018 1.19583773186827e-08
3019 1.19558659741059e-08
3020 1.19521959999203e-08
3021 1.19485370762284e-08
3022 1.19469532485206e-08
3023 1.19429648702263e-08
3024 1.19405784523297e-08
3025 1.19381496273552e-08
3026 1.19358269749237e-08
3027 1.1933916950313e-08
3028 1.19304921227803e-08
3029 1.19272640855816e-08
3030 1.1923216852755e-08
3031 1.19212629846321e-08
3032 1.19197716918762e-08
3033 1.19170332359142e-08
3034 1.19160628968551e-08
3035 1.19134635195028e-08
3036 1.19112242913944e-08
3037 1.19100818074536e-08
3038 1.19084515731449e-08
3039 1.19071250392344e-08
3040 1.19048833751023e-08
3041 1.19020565116368e-08
3042 1.19007740767563e-08
3043 1.18987598395071e-08
3044 1.18964865678806e-08
3045 1.18942129968824e-08
3046 1.18928784097749e-08
3047 1.18921639602476e-08
3048 1.18883313172946e-08
3049 1.18877311958143e-08
3050 1.18854425695814e-08
3051 1.18834389277456e-08
3052 1.18820750746595e-08
3053 1.18793369970893e-08
3054 1.18764245108371e-08
3055 1.18763136000566e-08
3056 1.18755854856545e-08
3057 1.18719900637776e-08
3058 1.1870315620055e-08
3059 1.18684824323978e-08
3060 1.18698765109448e-08
3061 1.18677919302146e-08
3062 1.1866478549033e-08
3063 1.1863440436688e-08
3064 1.186259705932e-08
3065 1.1860377195555e-08
3066 1.18577154017419e-08
3067 1.18577120701013e-08
3068 1.1853695597891e-08
3069 1.18539194500222e-08
3070 1.18507714874094e-08
3071 1.18495580467004e-08
3072 1.18486911562521e-08
3073 1.18453490558779e-08
3074 1.18450459178077e-08
3075 1.18433139723872e-08
3076 1.18397981337393e-08
3077 1.18406298209861e-08
3078 1.18378904984395e-08
3079 1.18371697116482e-08
3080 1.18351907497927e-08
3081 1.18329283260721e-08
3082 1.18320176905606e-08
3083 1.18303210266957e-08
3084 1.18256719415233e-08
3085 1.18234285358398e-08
3086 1.18212256116357e-08
3087 1.18185173598462e-08
3088 1.18165579568175e-08
3089 1.18159312106902e-08
3090 1.18130280438999e-08
3091 1.18106315117883e-08
3092 1.18100547660538e-08
3093 1.18069208329963e-08
3094 1.18074047323891e-08
3095 1.18039966610084e-08
3096 1.18017526358205e-08
3097 1.18012338491957e-08
3098 1.1798957609388e-08
3099 1.17976753686855e-08
3100 1.17953759325717e-08
3101 1.17943955858513e-08
3102 1.17941194378302e-08
3103 1.17927902255621e-08
3104 1.1789293814457e-08
3105 1.17868813313826e-08
3106 1.1786804529873e-08
3107 1.17860301295791e-08
3108 1.17833753396357e-08
3109 1.17830238004213e-08
3110 1.1781381410142e-08
3111 1.1778629292164e-08
3112 1.17760899190633e-08
3113 1.17743590542507e-08
3114 1.17746653169326e-08
3115 1.17714337999619e-08
3116 1.17695285319352e-08
3117 1.17696138156875e-08
3118 1.17685421867908e-08
3119 1.1763833687034e-08
3120 1.17652581879957e-08
3121 1.17619856623841e-08
3122 1.17604356719703e-08
3123 1.1762118530545e-08
3124 1.17599182747063e-08
3125 1.17558597038658e-08
3126 1.17542485825739e-08
3127 1.17552109511476e-08
3128 1.17526954435399e-08
3129 1.17496671721284e-08
3130 1.17487405316141e-08
3131 1.17495985497151e-08
3132 1.17441266779261e-08
3133 1.17435200114036e-08
3134 1.17421446377275e-08
3135 1.1743336767317e-08
3136 1.17393455864867e-08
3137 1.17401308135923e-08
3138 1.17372179695985e-08
3139 1.17342881661708e-08
3140 1.1736027603293e-08
3141 1.17316074954299e-08
3142 1.17301564322436e-08
3143 1.17291948000808e-08
3144 1.17270991493723e-08
3145 1.17285290532343e-08
3146 1.17244553413387e-08
3147 1.17226593258635e-08
3148 1.17215833851936e-08
3149 1.17222777230863e-08
3150 1.17211757877655e-08
3151 1.17170698427926e-08
3152 1.17156900221738e-08
3153 1.17143866166791e-08
3154 1.17136790621641e-08
3155 1.17140280715411e-08
3156 1.17094024958964e-08
3157 1.17085968555086e-08
3158 1.17077124728859e-08
3159 1.17053225830999e-08
3160 1.17053109618681e-08
3161 1.17024061893067e-08
3162 1.17038276310211e-08
3163 1.16982055838044e-08
3164 1.1697741966521e-08
3165 1.16970173367481e-08
3166 1.16921503035272e-08
3167 1.16943141096337e-08
3168 1.16945401631985e-08
3169 1.16879086581589e-08
3170 1.16901788377866e-08
3171 1.16857017373462e-08
3172 1.16878219524608e-08
3173 1.16832044111392e-08
3174 1.16847181722279e-08
3175 1.16833824025941e-08
3176 1.16790819409407e-08
3177 1.16806509656564e-08
3178 1.16761875841587e-08
3179 1.16777890937503e-08
3180 1.16733612306186e-08
3181 1.16750883632222e-08
3182 1.16731297657158e-08
3183 1.16699067018722e-08
3184 1.16709965612416e-08
3185 1.16664541898737e-08
3186 1.16676992395204e-08
3187 1.16640338283236e-08
3188 1.16657400328346e-08
3189 1.16612615511102e-08
3190 1.1662944835511e-08
3191 1.16608331796886e-08
3192 1.16570293687279e-08
3193 1.16587145544689e-08
3194 1.16541281096338e-08
3195 1.16550483648947e-08
3196 1.16498992086544e-08
3197 1.16530477642374e-08
3198 1.16493983825239e-08
3199 1.16503191240624e-08
3200 1.1648285022231e-08
3201 1.16438895166604e-08
3202 1.16458166763422e-08
3203 1.16409889785729e-08
3204 1.16439079033692e-08
3205 1.16412993841075e-08
3206 1.1639388786705e-08
3207 1.16416938768515e-08
3208 1.16375333057139e-08
3209 1.16384943490699e-08
3210 1.1636956195632e-08
3211 1.16362657738567e-08
3212 1.16343546199549e-08
3213 1.1633171324188e-08
3214 1.16312624100501e-08
3215 1.16311748279696e-08
3216 1.16284537020339e-08
3217 1.16287370521828e-08
3218 1.16281454665967e-08
3219 1.16243749503364e-08
3220 1.16246750992699e-08
3221 1.16242233247077e-08
3222 1.16206120986118e-08
3223 1.16200296204072e-08
3224 1.16209689547397e-08
3225 1.16157618233192e-08
3226 1.16126129652561e-08
3227 1.16145796664191e-08
3228 1.16142522714746e-08
3229 1.1610101785936e-08
3230 1.16093043250609e-08
3231 1.16062170282816e-08
3232 1.16046013149962e-08
3233 1.16034410564125e-08
3234 1.16031775696801e-08
3235 1.16017379995503e-08
3236 1.16020561412855e-08
3237 1.16029818161556e-08
3238 1.1599650272337e-08
3239 1.15986570856652e-08
3240 1.15984105885369e-08
3241 1.15970005473698e-08
3242 1.15933290223136e-08
3243 1.15947987703036e-08
3244 1.15899382278351e-08
3245 1.15923493238246e-08
3246 1.15905790921966e-08
3247 1.15873332925676e-08
3248 1.1588203574886e-08
3249 1.1586969230315e-08
3250 1.15829324279781e-08
3251 1.15843438670826e-08
3252 1.158104858906e-08
3253 1.15811711585978e-08
3254 1.15804598211366e-08
3255 1.15764987933253e-08
3256 1.15744309916177e-08
3257 1.15752507109934e-08
3258 1.15722339629198e-08
3259 1.15723772268206e-08
3260 1.1571180462272e-08
3261 1.15698474231207e-08
3262 1.15684618701151e-08
3263 1.15677643162393e-08
3264 1.15663245502662e-08
3265 1.15650568554004e-08
3266 1.15640901273972e-08
3267 1.15595777339939e-08
3268 1.15611395568593e-08
3269 1.15596855326805e-08
3270 1.15589532687088e-08
3271 1.15556298122821e-08
3272 1.15536271286798e-08
3273 1.15515520138776e-08
3274 1.15509305096129e-08
3275 1.15513443280368e-08
3276 1.15493461299854e-08
3277 1.15475362435225e-08
3278 1.15466164823663e-08
3279 1.15471166251824e-08
3280 1.15463813301075e-08
3281 1.1544525019197e-08
3282 1.15403008847936e-08
3283 1.15403985951057e-08
3284 1.1541058597242e-08
3285 1.15381913698143e-08
3286 1.15360212943894e-08
3287 1.15353806482144e-08
3288 1.15354763466358e-08
3289 1.15318617247562e-08
3290 1.15322140527008e-08
3291 1.15295693967898e-08
3292 1.15312842099147e-08
3293 1.1528757786039e-08
3294 1.15288922647172e-08
3295 1.15271974143738e-08
3296 1.15238416667773e-08
3297 1.15246155803495e-08
3298 1.1523116859119e-08
3299 1.15226498693666e-08
3300 1.15190814530819e-08
3301 1.15194946392061e-08
3302 1.15182672998093e-08
3303 1.15166877884265e-08
3304 1.15144111278997e-08
3305 1.15145377916903e-08
3306 1.15142155670545e-08
3307 1.15128537709897e-08
3308 1.15095351383154e-08
3309 1.15065519214197e-08
3310 1.1507836350344e-08
3311 1.1505282383556e-08
3312 1.15054935012349e-08
3313 1.15028000785555e-08
3314 1.15021092711165e-08
3315 1.15001409570215e-08
3316 1.15005915089228e-08
3317 1.14981937280323e-08
3318 1.14979729801679e-08
3319 1.14972599696284e-08
3320 1.14955615568568e-08
3321 1.1492501691357e-08
3322 1.14928333745656e-08
3323 1.14925821455281e-08
3324 1.14903345449502e-08
3325 1.14877611814834e-08
3326 1.14877590580986e-08
3327 1.14874604302262e-08
3328 1.14860046274423e-08
3329 1.14819404979372e-08
3330 1.14834419985832e-08
3331 1.14821497932127e-08
3332 1.14814120548423e-08
3333 1.14782847251416e-08
3334 1.14781220137972e-08
3335 1.14775203997886e-08
3336 1.14768559247025e-08
3337 1.14735152535184e-08
3338 1.14742141596458e-08
3339 1.14721610897373e-08
3340 1.14701343015e-08
3341 1.14699237220572e-08
3342 1.14690531163864e-08
3343 1.14676193808116e-08
3344 1.14641251861558e-08
3345 1.14656152669923e-08
3346 1.1464133628375e-08
3347 1.14640144330547e-08
3348 1.14603328207341e-08
3349 1.14613640772865e-08
3350 1.14585851352655e-08
3351 1.14589684601929e-08
3352 1.14537547710458e-08
3353 1.14551977614119e-08
3354 1.14538382575691e-08
3355 1.1452844253329e-08
3356 1.14508703359939e-08
3357 1.1451004370111e-08
3358 1.14504248813208e-08
3359 1.14460604536959e-08
3360 1.14456295582932e-08
3361 1.14443412365661e-08
3362 1.14453147768701e-08
3363 1.14416039176668e-08
3364 1.1441623084002e-08
3365 1.14395915169763e-08
3366 1.1437233413375e-08
3367 1.14345179823838e-08
3368 1.14354412664996e-08
3369 1.1433435866709e-08
3370 1.14323348569534e-08
3371 1.14315752729466e-08
3372 1.14278537710899e-08
3373 1.14255986067846e-08
3374 1.14235587702627e-08
3375 1.14226003101459e-08
3376 1.14223260685997e-08
3377 1.1421716628629e-08
3378 1.14208406917782e-08
3379 1.14189253266062e-08
3380 1.1417505769884e-08
3381 1.14174312364212e-08
3382 1.14130718589511e-08
3383 1.14126695811567e-08
3384 1.14105040624479e-08
3385 1.14090254006716e-08
3386 1.14093273783922e-08
3387 1.14083528743036e-08
3388 1.14069829808827e-08
3389 1.14054207654424e-08
3390 1.1404770936474e-08
3391 1.14021102674555e-08
3392 1.14009337889853e-08
3393 1.1398032429305e-08
3394 1.13986517051023e-08
3395 1.13963052724986e-08
3396 1.13959641591666e-08
3397 1.13959369220651e-08
3398 1.13918863232088e-08
3399 1.13917397966123e-08
3400 1.13908510343819e-08
3401 1.13895749134785e-08
3402 1.13897123360518e-08
3403 1.13881618256928e-08
3404 1.13877864589551e-08
3405 1.13826271496698e-08
3406 1.13831594076297e-08
3407 1.13807112908093e-08
3408 1.13805524438737e-08
3409 1.13791287252585e-08
3410 1.13785787488052e-08
3411 1.13778570786371e-08
3412 1.13754275923306e-08
3413 1.13738584676115e-08
3414 1.13720006801321e-08
3415 1.13730660800393e-08
3416 1.13698076290303e-08
3417 1.13693861284181e-08
3418 1.13678469569611e-08
3419 1.13690444843717e-08
3420 1.13651011570681e-08
3421 1.13640254273129e-08
3422 1.13635473641127e-08
3423 1.13616122010829e-08
3424 1.13605055228339e-08
3425 1.13597382617459e-08
3426 1.13590459143997e-08
3427 1.13557110558715e-08
3428 1.13557396018704e-08
3429 1.13549928274781e-08
3430 1.13520127837663e-08
3431 1.1350751953243e-08
3432 1.1350496962409e-08
3433 1.13489736527084e-08
3434 1.13493690175881e-08
3435 1.13469093891705e-08
3436 1.13462425316013e-08
3437 1.13441738701092e-08
3438 1.13422830589072e-08
3439 1.13417679006023e-08
3440 1.13421343573838e-08
3441 1.13409105474138e-08
3442 1.13405177909476e-08
3443 1.13385267813959e-08
3444 1.13383158109603e-08
3445 1.133614213078e-08
3446 1.13353673548422e-08
3447 1.13341392971866e-08
3448 1.13320224937319e-08
3449 1.13318038300614e-08
3450 1.13302138601001e-08
3451 1.13284204806829e-08
3452 1.13277167812231e-08
3453 1.13262312149143e-08
3454 1.13250865829417e-08
3455 1.13242726491325e-08
3456 1.13233341456354e-08
3457 1.13218537526882e-08
3458 1.13204379770804e-08
3459 1.13198657760716e-08
3460 1.13182924013355e-08
3461 1.13166681298404e-08
3462 1.13154722187758e-08
3463 1.13144650523422e-08
3464 1.13129658822209e-08
3465 1.13118933085798e-08
3466 1.13102791394759e-08
3467 1.13101683890671e-08
3468 1.13078470241057e-08
3469 1.13077107181947e-08
3470 1.1305683388585e-08
3471 1.13052044311279e-08
3472 1.13032395273605e-08
3473 1.13031297503674e-08
3474 1.13010875952835e-08
3475 1.13000046132461e-08
3476 1.12991529888362e-08
3477 1.12980023935505e-08
3478 1.12966222419186e-08
3479 1.12958989912926e-08
3480 1.12939755917418e-08
3481 1.12939137653645e-08
3482 1.12917047313532e-08
3483 1.12908610480633e-08
3484 1.12900342686961e-08
3485 1.1288074882182e-08
3486 1.12880013859451e-08
3487 1.12852765375981e-08
3488 1.12853224801257e-08
3489 1.12831216371734e-08
3490 1.12825337741929e-08
3491 1.12815929738941e-08
3492 1.12808031685618e-08
3493 1.12790449877664e-08
3494 1.12784153094292e-08
3495 1.12769624567022e-08
3496 1.12756354909982e-08
3497 1.12742712040093e-08
3498 1.12746480159798e-08
3499 1.12723648097446e-08
3500 1.12710734855026e-08
3501 1.12696733798046e-08
3502 1.1269755019111e-08
3503 1.12686481795743e-08
3504 1.12663117832201e-08
3505 1.12661272166914e-08
3506 1.12634804286249e-08
3507 1.12638669659548e-08
3508 1.12612257275879e-08
3509 1.12612656923139e-08
3510 1.1260475317687e-08
3511 1.12575411583271e-08
3512 1.12568340590868e-08
3513 1.12560698349307e-08
3514 1.12568226810494e-08
3515 1.12549301393428e-08
3516 1.12528410841362e-08
3517 1.12516190108047e-08
3518 1.12485384365091e-08
3519 1.12458299394158e-08
3520 1.12458693161954e-08
3521 1.12464815152091e-08
3522 1.12429668025216e-08
3523 1.12441667608765e-08
3524 1.12408678364095e-08
3525 1.12401954365515e-08
3526 1.12392227401281e-08
3527 1.12373333222282e-08
3528 1.12364220836714e-08
3529 1.12361959875296e-08
3530 1.12339062477984e-08
3531 1.12327742281537e-08
3532 1.12315438736688e-08
3533 1.12297546503159e-08
3534 1.12268079242306e-08
3535 1.12256564362145e-08
3536 1.12259644365065e-08
3537 1.12257971718044e-08
3538 1.12237942340165e-08
3539 1.12220915113037e-08
3540 1.12214771742503e-08
3541 1.12199194619855e-08
3542 1.12188935110225e-08
3543 1.12179690145708e-08
3544 1.12159722680805e-08
3545 1.12142282937266e-08
3546 1.12140818882556e-08
3547 1.12122696204475e-08
3548 1.12102592066954e-08
3549 1.12090646711138e-08
3550 1.12091401292258e-08
3551 1.12066514811326e-08
3552 1.12065646510617e-08
3553 1.12039883298776e-08
3554 1.1204712601659e-08
3555 1.1202628516449e-08
3556 1.12007789540847e-08
3557 1.12010583468714e-08
3558 1.11989491469178e-08
3559 1.11987126952262e-08
3560 1.11978104838373e-08
3561 1.11954898404654e-08
3562 1.11953038104684e-08
3563 1.11938378403453e-08
3564 1.11922817537247e-08
3565 1.11909228990015e-08
3566 1.11908373221503e-08
3567 1.11893093105997e-08
3568 1.11875915368154e-08
3569 1.11868469946652e-08
3570 1.11851018665121e-08
3571 1.11849820639831e-08
3572 1.11831836482612e-08
3573 1.11818892870597e-08
3574 1.11804650719805e-08
3575 1.11796346431714e-08
3576 1.11796353212956e-08
3577 1.1177426852943e-08
3578 1.11762646347047e-08
3579 1.11747694813147e-08
3580 1.11745099128646e-08
3581 1.11725897858905e-08
3582 1.11712598455938e-08
3583 1.11702193664742e-08
3584 1.11692081705417e-08
3585 1.11683783462768e-08
3586 1.1166445792965e-08
3587 1.11659550386667e-08
3588 1.11648108272744e-08
3589 1.11641405134699e-08
3590 1.11623207721956e-08
3591 1.11611607723217e-08
3592 1.11597833773736e-08
3593 1.11631139173562e-08
3594 1.11566308375888e-08
3595 1.11565030020688e-08
3596 1.11552899219602e-08
3597 1.11576325357854e-08
3598 1.1152440149681e-08
3599 1.11512090380794e-08
3600 1.11552805044379e-08
3601 1.11478845640223e-08
3602 1.11515985663502e-08
3603 1.11457639806467e-08
3604 1.11454272074818e-08
3605 1.11482660065942e-08
3606 1.11431583967869e-08
3607 1.11427655519747e-08
3608 1.11426273598503e-08
3609 1.11437188314301e-08
3610 1.11385761470129e-08
3611 1.11380866838762e-08
3612 1.11391715808817e-08
3613 1.11355017409498e-08
3614 1.11380874280032e-08
3615 1.11322491482546e-08
3616 1.11355819187031e-08
3617 1.11329994844367e-08
3618 1.11325834070419e-08
3619 1.11320614591404e-08
3620 1.1126760543545e-08
3621 1.11290016804877e-08
3622 1.11256313617136e-08
3623 1.11244064869842e-08
3624 1.11258746655352e-08
3625 1.1123928947282e-08
3626 1.11195573554435e-08
3627 1.11197699496079e-08
3628 1.11202039746461e-08
3629 1.11171307571045e-08
3630 1.11163929376601e-08
3631 1.11140688399403e-08
3632 1.11154480904874e-08
3633 1.11106308833819e-08
3634 1.11122976456379e-08
3635 1.11099704912243e-08
3636 1.11072610857466e-08
3637 1.11077841930818e-08
3638 1.11073207539059e-08
3639 1.11051265788698e-08
3640 1.11042858708665e-08
3641 1.1103730989781e-08
3642 1.11000359715174e-08
3643 1.11002665506843e-08
3644 1.11004728040875e-08
3645 1.10987854302913e-08
3646 1.10987182294908e-08
3647 1.10974641557293e-08
3648 1.1095449371834e-08
3649 1.10947886022006e-08
3650 1.10940901472401e-08
3651 1.10925191112998e-08
3652 1.10914854287991e-08
3653 1.1091070326652e-08
3654 1.1089161918193e-08
3655 1.10851564158188e-08
3656 1.10867316848173e-08
3657 1.10858538703296e-08
3658 1.10848585445911e-08
3659 1.10841765557645e-08
3660 1.10790784660642e-08
3661 1.10807457135709e-08
3662 1.10809784209243e-08
3663 1.10793290097611e-08
3664 1.10781003506977e-08
3665 1.10779703178521e-08
3666 1.1075473351746e-08
3667 1.10750857184039e-08
3668 1.10734509837229e-08
3669 1.10698880358417e-08
3670 1.10708695825468e-08
3671 1.10698691680733e-08
3672 1.10696094463569e-08
3673 1.10681038637539e-08
3674 1.10668279501014e-08
3675 1.10657823425564e-08
3676 1.10654159732326e-08
3677 1.10638176325506e-08
3678 1.10624671294524e-08
3679 1.10615516296375e-08
3680 1.10604202225861e-08
3681 1.10593411599413e-08
3682 1.10570027187507e-08
3683 1.10558496879964e-08
3684 1.10552448611967e-08
3685 1.10549354651046e-08
3686 1.10526051523563e-08
3687 1.10525844479181e-08
3688 1.10516901162816e-08
3689 1.10506209183292e-08
3690 1.10488843599044e-08
3691 1.10484677353917e-08
3692 1.1046160849848e-08
3693 1.10463921360326e-08
3694 1.10434762385647e-08
3695 1.10429603040751e-08
3696 1.10413626183414e-08
3697 1.10419529832928e-08
3698 1.10405895736299e-08
3699 1.10402860680203e-08
3700 1.10390917107961e-08
3701 1.10372997091102e-08
3702 1.10350127995873e-08
3703 1.10343319747463e-08
3704 1.10338032425494e-08
3705 1.10330431828953e-08
3706 1.10302775975513e-08
3707 1.1030003831819e-08
3708 1.10299264047542e-08
3709 1.10274492997053e-08
3710 1.10261715888516e-08
3711 1.10269616914738e-08
3712 1.10264985932196e-08
3713 1.10244972844065e-08
3714 1.10217978401272e-08
3715 1.1022337454647e-08
3716 1.10215533882796e-08
3717 1.10189747674072e-08
3718 1.1018005428215e-08
3719 1.10120316494655e-08
3720 1.10117287442368e-08
3721 1.10151172520168e-08
3722 1.10136765716362e-08
3723 1.1012668488658e-08
3724 1.10124385709343e-08
3725 1.10105601386912e-08
3726 1.10103254968852e-08
3727 1.10092684239282e-08
3728 1.10079757965342e-08
3729 1.10077235218886e-08
3730 1.10057840554123e-08
3731 1.10046055915858e-08
3732 1.10033265574572e-08
3733 1.10016559186632e-08
3734 1.10018689788438e-08
3735 1.09994658989421e-08
3736 1.09991500069628e-08
3737 1.09983597637309e-08
3738 1.09971453224611e-08
3739 1.09952111734535e-08
3740 1.09949458915737e-08
3741 1.099315662767e-08
3742 1.0991834334062e-08
3743 1.09902180342458e-08
3744 1.09901728969353e-08
3745 1.09888834188554e-08
3746 1.09884368511837e-08
3747 1.09854409811871e-08
3748 1.09848033206239e-08
3749 1.0985363580629e-08
3750 1.09826542931679e-08
3751 1.09822648533264e-08
3752 1.0981188735798e-08
3753 1.09795724590189e-08
3754 1.09790855292446e-08
3755 1.09775968714643e-08
3756 1.09714154110929e-08
3757 1.09732226332704e-08
3758 1.09748547836575e-08
3759 1.09739409545062e-08
3760 1.09723044845189e-08
3761 1.09709570593197e-08
3762 1.09706204320936e-08
3763 1.09687763824862e-08
3764 1.09678444520367e-08
3765 1.09671160347935e-08
3766 1.09658988337036e-08
3767 1.09650826616292e-08
3768 1.09634279922832e-08
3769 1.09625095181809e-08
3770 1.0961303571172e-08
3771 1.09609027072466e-08
3772 1.09571970444144e-08
3773 1.09579159241202e-08
3774 1.09572428921012e-08
3775 1.0956066204576e-08
3776 1.09550673501324e-08
3777 1.09541009664926e-08
3778 1.09527184563862e-08
3779 1.0951839799489e-08
3780 1.09504927259529e-08
3781 1.09490181012806e-08
3782 1.09483721565107e-08
3783 1.09476834628808e-08
3784 1.09483707473879e-08
3785 1.09451827862428e-08
3786 1.09443552278876e-08
3787 1.09430923574405e-08
3788 1.09434324567226e-08
3789 1.09411305591889e-08
3790 1.09407127636685e-08
3791 1.09401472291049e-08
3792 1.09396173611498e-08
3793 1.09365155721297e-08
3794 1.09362057008067e-08
3795 1.09363540451712e-08
3796 1.09347507599722e-08
3797 1.09328197774317e-08
3798 1.09325419895001e-08
3799 1.09314339397926e-08
3800 1.09308830258115e-08
3801 1.09285095653278e-08
3802 1.09287628461685e-08
3803 1.09263088766132e-08
3804 1.09271721304705e-08
3805 1.09236062446061e-08
3806 1.0923625337389e-08
3807 1.09229837990388e-08
3808 1.09220252728082e-08
3809 1.09224663640828e-08
3810 1.09200509699203e-08
3811 1.09180060861302e-08
3812 1.0919597521003e-08
3813 1.09151858071477e-08
3814 1.0916487458712e-08
3815 1.09137703497852e-08
3816 1.09150440670835e-08
3817 1.09118625470084e-08
3818 1.09107014885645e-08
3819 1.09091357518021e-08
3820 1.09088050889716e-08
3821 1.09090082597019e-08
3822 1.09063749345328e-08
3823 1.09060092767788e-08
3824 1.09044693266391e-08
3825 1.0905035157438e-08
3826 1.09025376934802e-08
3827 1.0903266777107e-08
3828 1.09000242920765e-08
3829 1.09016026156039e-08
3830 1.08987004212857e-08
3831 1.08974235328574e-08
3832 1.08959794530017e-08
3833 1.08954668919525e-08
3834 1.0894721484328e-08
3835 1.08926291568223e-08
3836 1.08930508933014e-08
3837 1.08904474839799e-08
3838 1.08890947262408e-08
3839 1.08871432314972e-08
3840 1.08890269996942e-08
3841 1.08876814769732e-08
3842 1.08851883234007e-08
3843 1.08854836402827e-08
3844 1.08839608107536e-08
3845 1.08840211084649e-08
3846 1.08812374728651e-08
3847 1.08812263531766e-08
3848 1.08795941037021e-08
3849 1.0879401996039e-08
3850 1.08784558794095e-08
3851 1.08746757964406e-08
3852 1.08744856969767e-08
3853 1.08748260834179e-08
3854 1.08738530421593e-08
3855 1.08710393573741e-08
3856 1.08703091925288e-08
3857 1.08706657898361e-08
3858 1.08699252024946e-08
3859 1.0869025354493e-08
3860 1.08672393565323e-08
3861 1.08665151193621e-08
3862 1.0864274992417e-08
3863 1.08630388405695e-08
3864 1.0863805657596e-08
3865 1.08634560507803e-08
3866 1.08639684298362e-08
3867 1.08602162548765e-08
3868 1.08600421724891e-08
3869 1.08575656220522e-08
3870 1.08569194598451e-08
3871 1.08565385998072e-08
3872 1.0855221279249e-08
3873 1.08549563085925e-08
3874 1.0853599856947e-08
3875 1.08506172790124e-08
3876 1.08501653721393e-08
3877 1.08502828640444e-08
3878 1.08478977175197e-08
3879 1.08484220515959e-08
3880 1.08473816887722e-08
3881 1.08458300848713e-08
3882 1.08448716492626e-08
3883 1.08437005566231e-08
3884 1.08414838368431e-08
3885 1.08429571242796e-08
3886 1.0841382587029e-08
3887 1.08391120465234e-08
3888 1.08374695921565e-08
3889 1.08379179790119e-08
3890 1.08350441703764e-08
3891 1.08348140419323e-08
3892 1.0835522508662e-08
3893 1.08324157087059e-08
3894 1.08322327351806e-08
3895 1.08299769396303e-08
3896 1.08300166552222e-08
3897 1.08285704176481e-08
3898 1.08296961122689e-08
3899 1.08263658762375e-08
3900 1.08261784834418e-08
3901 1.08241465452408e-08
3902 1.08243411851838e-08
3903 1.08247368961201e-08
3904 1.08219058900239e-08
3905 1.08212500977145e-08
3906 1.08210252710017e-08
3907 1.0818803914292e-08
3908 1.08195056016114e-08
3909 1.08168409341802e-08
3910 1.08158108601819e-08
3911 1.08144398397458e-08
3912 1.08139805822249e-08
3913 1.08149922780354e-08
3914 1.08129566302462e-08
3915 1.08111525119192e-08
3916 1.08111082430529e-08
3917 1.08126958378862e-08
3918 1.08082498826156e-08
3919 1.08076065891693e-08
3920 1.0807792715728e-08
3921 1.08104935268305e-08
3922 1.08074743569164e-08
3923 1.08044388694717e-08
3924 1.08030365894851e-08
3925 1.08033637935823e-08
3926 1.08038060381843e-08
3927 1.08041274136661e-08
3928 1.08040467332593e-08
3929 1.07992292112946e-08
3930 1.08002630901938e-08
3931 1.07957090355881e-08
3932 1.07962282659413e-08
3933 1.07949210207126e-08
3934 1.07967770062445e-08
3935 1.0793272293641e-08
3936 1.07918979686816e-08
3937 1.07915251734425e-08
3938 1.07900722400855e-08
3939 1.07903938662279e-08
3940 1.07904281839155e-08
3941 1.07861284144861e-08
3942 1.07859215228989e-08
3943 1.07870496623974e-08
3944 1.0783636666456e-08
3945 1.07835695067615e-08
3946 1.07784084830143e-08
3947 1.07758701429483e-08
3948 1.07754595735921e-08
3949 1.07738373660016e-08
3950 1.07752510933035e-08
3951 1.07721128355109e-08
3952 1.07710582109566e-08
3953 1.07705987292539e-08
3954 1.0771409251098e-08
3955 1.07681148922567e-08
3956 1.07666514279348e-08
3957 1.07684301824951e-08
3958 1.07657234104663e-08
3959 1.07642070393543e-08
3960 1.07650674853188e-08
3961 1.0762511625878e-08
3962 1.07629657666031e-08
3963 1.07601211354058e-08
3964 1.07592281094471e-08
3965 1.07582495708636e-08
3966 1.07589534105168e-08
3967 1.07555540015958e-08
3968 1.07556049818436e-08
3969 1.07542268069361e-08
3970 1.07549988080657e-08
3971 1.07518293896436e-08
3972 1.07515392525348e-08
3973 1.07518284053199e-08
3974 1.07448536945665e-08
3975 1.07441448091994e-08
3976 1.07424499171949e-08
3977 1.07427140298433e-08
3978 1.07412453759781e-08
3979 1.07443799600704e-08
3980 1.0743397751839e-08
3981 1.07420780092737e-08
3982 1.07431981943296e-08
3983 1.07398842284778e-08
3984 1.07408498423722e-08
3985 1.07380638925958e-08
3986 1.07421969112453e-08
3987 1.07343302996887e-08
3988 1.07326848245992e-08
3989 1.07349032656412e-08
3990 1.07334118858438e-08
3991 1.07322675685917e-08
3992 1.07331273365985e-08
3993 1.07297401642503e-08
3994 1.07296935329682e-08
3995 1.07283922977097e-08
3996 1.07290954971251e-08
3997 1.07261562914962e-08
3998 1.07263407899127e-08
3999 1.07245826967972e-08
4000 1.07230289022875e-08
4001 1.07248408484639e-08
4002 1.072154207718e-08
4003 1.07203069630302e-08
4004 1.07217512107238e-08
4005 1.0717557938883e-08
4006 1.07197736481379e-08
4007 1.07156012073839e-08
4008 1.07159823275471e-08
4009 1.07164537286597e-08
4010 1.07139619650864e-08
4011 1.07123961542444e-08
4012 1.0713283672692e-08
4013 1.07104367157163e-08
4014 1.07108113624466e-08
4015 1.07089736004762e-08
4016 1.07073231870869e-08
4017 1.07098878231182e-08
4018 1.07060681185656e-08
4019 1.0706608529587e-08
4020 1.07044407736401e-08
4021 1.07040588870622e-08
4022 1.07023338126733e-08
4023 1.07022687298175e-08
4024 1.069839004611e-08
4025 1.0699182294821e-08
4026 1.07001936357487e-08
4027 1.06987068219933e-08
4028 1.06965548739013e-08
4029 1.06953999020054e-08
4030 1.06946007448205e-08
4031 1.06934308165552e-08
4032 1.06934082262644e-08
4033 1.06907393727096e-08
4034 1.0691238403554e-08
4035 1.06892573067396e-08
4036 1.06885689052649e-08
4037 1.06884836247323e-08
4038 1.06863743536689e-08
4039 1.06861442940309e-08
4040 1.06840398036989e-08
4041 1.06838654758135e-08
4042 1.06822335308177e-08
4043 1.06824486988677e-08
4044 1.06793614979839e-08
4045 1.06804114238246e-08
4046 1.067893980361e-08
4047 1.06787452975876e-08
4048 1.06765105940543e-08
4049 1.06755320053997e-08
4050 1.0675478453237e-08
4051 1.06749191957911e-08
4052 1.06724767623134e-08
4053 1.06722309729801e-08
4054 1.06716859148148e-08
4055 1.06691110543233e-08
4056 1.06698090988622e-08
4057 1.06681830113198e-08
4058 1.06661125231777e-08
4059 1.06671658490765e-08
4060 1.06659241038976e-08
4061 1.06645856171872e-08
4062 1.06624110348674e-08
4063 1.06625355592027e-08
4064 1.06622266734524e-08
4065 1.06600302485216e-08
4066 1.06601250712268e-08
4067 1.06589903430543e-08
4068 1.06570441638087e-08
4069 1.06577469861369e-08
4070 1.06551349068895e-08
4071 1.06551022388823e-08
4072 1.06532432644357e-08
4073 1.06528609496725e-08
4074 1.06523574176232e-08
4075 1.06512323174435e-08
4076 1.06487700916036e-08
4077 1.0648455375295e-08
4078 1.06480062646852e-08
4079 1.06468891344358e-08
4080 1.06452439001536e-08
4081 1.06452634403009e-08
4082 1.06440034537969e-08
4083 1.06432653885213e-08
4084 1.06418366724037e-08
4085 1.06408124291579e-08
4086 1.06403526679566e-08
4087 1.06389626879089e-08
4088 1.0638068643154e-08
4089 1.06378990117062e-08
4090 1.06361846831937e-08
4091 1.06360124765281e-08
4092 1.06334782684625e-08
4093 1.06331254055847e-08
4094 1.06329945207262e-08
4095 1.06311863161956e-08
4096 1.06307765904978e-08
4097 1.06300643779589e-08
4098 1.06272176198241e-08
4099 1.06280738064735e-08
4100 1.06255381224496e-08
4101 1.06258131813142e-08
4102 1.06258571386797e-08
4103 1.06231048471461e-08
4104 1.0623064497034e-08
4105 1.06221588721256e-08
4106 1.06228849635209e-08
4107 1.0621354840229e-08
4108 1.06186584230838e-08
4109 1.0618406869084e-08
4110 1.0618488049674e-08
4111 1.06169052928318e-08
4112 1.06159448534371e-08
4113 1.06136990965788e-08
4114 1.06117492596897e-08
4115 1.06083698137444e-08
4116 1.06070693723648e-08
4117 1.06070372211664e-08
4118 1.06063309552595e-08
4119 1.06045518117048e-08
4120 1.06051992176948e-08
4121 1.06031429609049e-08
4122 1.06019036477945e-08
4123 1.06015851623342e-08
4124 1.05994610043736e-08
4125 1.05997183736245e-08
4126 1.05974664627584e-08
4127 1.0597290401787e-08
4128 1.05949821797291e-08
4129 1.05947126013833e-08
4130 1.05965784200079e-08
4131 1.05931595312941e-08
4132 1.05919218882644e-08
4133 1.05914943878405e-08
4134 1.05893656227873e-08
4135 1.05893709130417e-08
4136 1.05870585333234e-08
4137 1.05887156943429e-08
4138 1.05867799534143e-08
4139 1.0584604442837e-08
4140 1.05844167571645e-08
4141 1.05832859626925e-08
4142 1.05829909766569e-08
4143 1.05813645364522e-08
4144 1.05795380985069e-08
4145 1.05799131830675e-08
4146 1.0577278424434e-08
4147 1.05790293912361e-08
4148 1.05764306308903e-08
4149 1.05758033472625e-08
4150 1.05739581807152e-08
4151 1.05728816378464e-08
4152 1.05730407735788e-08
4153 1.05711950474235e-08
4154 1.057107411917e-08
4155 1.05704864387102e-08
4156 1.05628214466802e-08
4157 1.05674827148433e-08
4158 1.05672620838299e-08
4159 1.05648325325891e-08
4160 1.05667014991123e-08
4161 1.05639332447921e-08
4162 1.05624454968117e-08
4163 1.05617954766213e-08
4164 1.0562737295064e-08
4165 1.0563090709792e-08
4166 1.05630762905623e-08
4167 1.05636073018389e-08
4168 1.05615430881778e-08
4169 1.05626686845578e-08
4170 1.05599224231867e-08
4171 1.05596608996755e-08
4172 1.05584367697809e-08
4173 1.0556990031857e-08
4174 1.05554806677371e-08
4175 1.05552697687167e-08
4176 1.05547908882814e-08
4177 1.05520788660912e-08
4178 1.05517615799633e-08
4179 1.05515742686718e-08
4180 1.05494169515691e-08
4181 1.05517147090256e-08
4182 1.05463629413854e-08
4183 1.05470327116525e-08
4184 1.05468146022886e-08
4185 1.05454185400783e-08
4186 1.05459165269245e-08
4187 1.05425047671331e-08
4188 1.05439205392577e-08
4189 1.05413659642756e-08
4190 1.05401179039538e-08
4191 1.05391520739129e-08
4192 1.05399113877191e-08
4193 1.05373271578335e-08
4194 1.05353874660513e-08
4195 1.0534325613154e-08
4196 1.05366967362297e-08
4197 1.0534604129156e-08
4198 1.05340895550504e-08
4199 1.05332088130849e-08
4200 1.05324234943582e-08
4201 1.05293825752029e-08
4202 1.05275857531784e-08
4203 1.05300866798663e-08
4204 1.05273232845898e-08
4205 1.05265278202843e-08
4206 1.05258747174625e-08
4207 1.05242449095072e-08
4208 1.05240514887417e-08
4209 1.05223572394869e-08
4210 1.05220082592533e-08
4211 1.05224697887624e-08
4212 1.05197193016188e-08
4213 1.05165801376622e-08
4214 1.05174724392976e-08
4215 1.05184265597469e-08
4216 1.05163692045718e-08
4217 1.05142056937846e-08
4218 1.05148742852584e-08
4219 1.05128690109091e-08
4220 1.05102974790189e-08
4221 1.05093441390286e-08
4222 1.05076293366174e-08
4223 1.05138945952071e-08
4224 1.0507277761182e-08
4225 1.05038617499303e-08
4226 1.05072247229138e-08
4227 1.05030242518872e-08
4228 1.05044104650176e-08
4229 1.05027714131789e-08
4230 1.05025405599812e-08
4231 1.04995903243005e-08
4232 1.04990018477275e-08
4233 1.04970838623042e-08
4234 1.0495379099168e-08
4235 1.04982980317497e-08
4236 1.04945724448691e-08
4237 1.04927502231344e-08
4238 1.04933355487696e-08
4239 1.04901716557748e-08
4240 1.04919565484252e-08
4241 1.0487614176713e-08
4242 1.04876821651334e-08
4243 1.04884907602409e-08
4244 1.04845968800504e-08
4245 1.0481561855763e-08
4246 1.04838761220694e-08
4247 1.04839706019377e-08
4248 1.04803629718264e-08
4249 1.04825712380907e-08
4250 1.04780039467678e-08
4251 1.04814073903781e-08
4252 1.04763462908664e-08
4253 1.04758546595751e-08
4254 1.04747559842855e-08
4255 1.04725048981624e-08
4256 1.04757628678898e-08
4257 1.04717095652795e-08
4258 1.04722391859602e-08
4259 1.0469139174879e-08
4260 1.04707231065315e-08
4261 1.04693089748586e-08
4262 1.04664026857049e-08
4263 1.04674627701973e-08
4264 1.04641797675775e-08
4265 1.04604752240722e-08
4266 1.04623087202882e-08
4267 1.04643915183333e-08
4268 1.04589307245162e-08
4269 1.04602344188631e-08
4270 1.0458148112652e-08
4271 1.04567908162378e-08
4272 1.04580248339037e-08
4273 1.04555492935754e-08
4274 1.04532386306366e-08
4275 1.04535427094266e-08
4276 1.04507032211154e-08
4277 1.04510125275292e-08
4278 1.04514769913022e-08
4279 1.04489992555634e-08
4280 1.0448262859869e-08
4281 1.04467340646952e-08
4282 1.04459747247154e-08
4283 1.04452635708852e-08
4284 1.04445436754641e-08
4285 1.04409778157455e-08
4286 1.04395712920702e-08
4287 1.04396353888259e-08
4288 1.04438666423401e-08
4289 1.04382912347323e-08
4290 1.04361224578242e-08
4291 1.043842489723e-08
4292 1.04368688538248e-08
4293 1.04356578959353e-08
4294 1.04317877234195e-08
4295 1.04318257573288e-08
4296 1.04343950991792e-08
4297 1.04294385840364e-08
4298 1.04286134393894e-08
4299 1.04275672453413e-08
4300 1.0427371261057e-08
4301 1.0426141623221e-08
4302 1.04249570999915e-08
4303 1.04282300609493e-08
4304 1.042265577339e-08
4305 1.04251954999579e-08
4306 1.0422156041856e-08
4307 1.04198336567662e-08
4308 1.04198179268489e-08
4309 1.0419846853349e-08
4310 1.04176264311417e-08
4311 1.04201026296102e-08
4312 1.04148188834197e-08
4313 1.04188034406039e-08
4314 1.04142265928975e-08
4315 1.04122780719418e-08
4316 1.04156736664418e-08
4317 1.04120197366719e-08
4318 1.04088442163808e-08
4319 1.0409657480337e-08
4320 1.04154032427584e-08
4321 1.04087612825821e-08
4322 1.04044366033007e-08
4323 1.04048068255425e-08
4324 1.0403720979435e-08
4325 1.04030198228855e-08
4326 1.04025449143286e-08
4327 1.04024028402538e-08
4328 1.04007933317274e-08
4329 1.039996423291e-08
4330 1.03992046436019e-08
4331 1.03970747411253e-08
4332 1.03985599297363e-08
4333 1.03958666351212e-08
4334 1.04026647519961e-08
4335 1.0395286691528e-08
4336 1.03951960381266e-08
4337 1.0395405169894e-08
4338 1.03906292485689e-08
4339 1.03924080912809e-08
4340 1.0388931676597e-08
4341 1.03912729686184e-08
4342 1.03923425940622e-08
4343 1.03880470863349e-08
4344 1.0387433645398e-08
4345 1.03880171198056e-08
4346 1.03856417868198e-08
4347 1.03846214204728e-08
4348 1.03839162001462e-08
4349 1.03836677415314e-08
4350 1.03825652449374e-08
4351 1.03816089828124e-08
4352 1.03794615584352e-08
4353 1.03764115396282e-08
4354 1.03762322862244e-08
4355 1.03752656697154e-08
4356 1.03736840266211e-08
4357 1.03738826963684e-08
4358 1.03728207289377e-08
4359 1.03715403994009e-08
4360 1.03708398133395e-08
4361 1.03730927632217e-08
4362 1.0368766501262e-08
4363 1.03680571547915e-08
4364 1.03658587669864e-08
4365 1.03660510352432e-08
4366 1.03654344626725e-08
4367 1.03642008874999e-08
4368 1.03675329790609e-08
4369 1.03613103565603e-08
4370 1.0360402898979e-08
4371 1.03615123813183e-08
4372 1.03593330801344e-08
4373 1.0361630183725e-08
4374 1.03593722617923e-08
4375 1.03594967839904e-08
4376 1.03541216883163e-08
4377 1.03539973852762e-08
4378 1.03518664984759e-08
4379 1.03565238467607e-08
4380 1.03517938650904e-08
4381 1.03509029242554e-08
4382 1.03484765454354e-08
4383 1.03517180441159e-08
4384 1.03474261180375e-08
4385 1.03462384954539e-08
4386 1.03493481560524e-08
4387 1.03438914473564e-08
4388 1.03479301666598e-08
4389 1.03424724303136e-08
4390 1.03408605096333e-08
4391 1.03420417906785e-08
4392 1.03399673820892e-08
4393 1.03380675669645e-08
4394 1.03387692045459e-08
4395 1.03380659374069e-08
4396 1.03360573024402e-08
4397 1.03362040687754e-08
4398 1.0334883029775e-08
4399 1.03346330810605e-08
4400 1.03340934655e-08
4401 1.03321477524371e-08
4402 1.03318104088396e-08
4403 1.03306571955369e-08
4404 1.0328686334049e-08
4405 1.03320024973652e-08
4406 1.03269452197824e-08
4407 1.03263546046839e-08
4408 1.03268100965781e-08
4409 1.03242331258163e-08
4410 1.03239496492408e-08
4411 1.03237053857425e-08
4412 1.03225996030826e-08
4413 1.03213521515516e-08
4414 1.03203351273368e-08
4415 1.03194549334329e-08
4416 1.03185380622012e-08
4417 1.03174760691382e-08
4418 1.03168516021462e-08
4419 1.03162692414172e-08
4420 1.03147377499557e-08
4421 1.03138561926841e-08
4422 1.03137853438989e-08
4423 1.03117486103116e-08
4424 1.03103414241384e-08
4425 1.03115222524347e-08
4426 1.03093095509127e-08
4427 1.03096918285944e-08
4428 1.03123283052287e-08
4429 1.03089288386315e-08
4430 1.03046000739376e-08
4431 1.03054139913988e-08
4432 1.03039773170083e-08
4433 1.03035168475402e-08
4434 1.03008529578064e-08
4435 1.03014285245673e-08
4436 1.02986923562892e-08
4437 1.02991940242186e-08
4438 1.02974804051803e-08
4439 1.02965072287103e-08
4440 1.02951600098183e-08
4441 1.02950258540624e-08
4442 1.02930525588546e-08
4443 1.02939184286471e-08
4444 1.02905498020056e-08
4445 1.02918284930337e-08
4446 1.02903768174345e-08
4447 1.02893316670377e-08
4448 1.02881332055688e-08
4449 1.02867710149585e-08
4450 1.0287358650593e-08
4451 1.02852715018614e-08
4452 1.02835625504505e-08
4453 1.02843273435266e-08
4454 1.02827563405322e-08
4455 1.02818221169171e-08
4456 1.02813643858279e-08
4457 1.02794241569476e-08
4458 1.0278851308318e-08
4459 1.02776352927519e-08
4460 1.02770339169833e-08
4461 1.02771621789954e-08
4462 1.02747596302799e-08
4463 1.02753681549411e-08
4464 1.02742567254371e-08
4465 1.02718865930551e-08
4466 1.02724060863923e-08
4467 1.02707254586165e-08
4468 1.02700074880863e-08
4469 1.02693915874363e-08
4470 1.02680903788094e-08
4471 1.02675238592975e-08
4472 1.02667878511126e-08
4473 1.02677572621779e-08
4474 1.02634445832972e-08
4475 1.02639106049629e-08
4476 1.02625059835715e-08
4477 1.02599530511643e-08
4478 1.02642200450476e-08
4479 1.02575800522281e-08
4480 1.02582630815556e-08
4481 1.0260487494626e-08
4482 1.02569015745513e-08
4483 1.02606692485158e-08
4484 1.02559541002578e-08
4485 1.02531297111186e-08
4486 1.0255031531875e-08
4487 1.02533335584187e-08
4488 1.02520956728191e-08
4489 1.02520075128953e-08
4490 1.02508020561609e-08
4491 1.02508266081602e-08
4492 1.02495923892665e-08
4493 1.02488082492497e-08
4494 1.02443783489281e-08
4495 1.02478997051209e-08
4496 1.02422875658675e-08
4497 1.0244317131286e-08
4498 1.02448812556721e-08
4499 1.02397850059027e-08
4500 1.02422001810876e-08
4501 1.0241131377986e-08
4502 1.02385591218696e-08
4503 1.02399393712982e-08
4504 1.02378484581889e-08
4505 1.02380257069046e-08
4506 1.02315026705713e-08
4507 1.02334725432668e-08
4508 1.02345162964895e-08
4509 1.02337809591707e-08
4510 1.02297055205175e-08
4511 1.02325672499126e-08
4512 1.02292663619052e-08
4513 1.02307364271137e-08
4514 1.02279127155297e-08
4515 1.02292997101044e-08
4516 1.02272193004543e-08
4517 1.02252144083548e-08
4518 1.02260946180655e-08
4519 1.02239529309323e-08
4520 1.02233723492107e-08
4521 1.02236387821142e-08
4522 1.02220020074539e-08
4523 1.0220153330423e-08
4524 1.02233231525362e-08
4525 1.021518128419e-08
4526 1.02182605704049e-08
4527 1.02146380091261e-08
4528 1.02174550959955e-08
4529 1.02156023486649e-08
4530 1.02180465454543e-08
4531 1.02140938926659e-08
4532 1.02096305314714e-08
4533 1.02111724056692e-08
4534 1.02111426242141e-08
4535 1.02139919597177e-08
4536 1.0209807523337e-08
4537 1.02076452111188e-08
4538 1.02080514506747e-08
4539 1.02108190787598e-08
4540 1.02043259564283e-08
4541 1.02059744940819e-08
4542 1.02045301940273e-08
4543 1.02024853246563e-08
4544 1.02068796651034e-08
4545 1.02016837864594e-08
4546 1.02031768430544e-08
4547 1.01987933721071e-08
4548 1.0198806446246e-08
4549 1.02014709336407e-08
4550 1.0196987802899e-08
4551 1.01992484869279e-08
4552 1.01942933913024e-08
4553 1.01984667000565e-08
4554 1.01910227009866e-08
4555 1.01971004179829e-08
4556 1.01915441030748e-08
4557 1.01933132546916e-08
4558 1.01903530565217e-08
4559 1.01861972482964e-08
4560 1.01882381868951e-08
4561 1.01906297681337e-08
4562 1.018600157289e-08
4563 1.01882797049618e-08
4564 1.01847811351391e-08
4565 1.01872236188544e-08
4566 1.01820057069879e-08
4567 1.01858339880689e-08
4568 1.01813519250427e-08
4569 1.01833385616479e-08
4570 1.01788630486044e-08
4571 1.01818149002425e-08
4572 1.01781421893432e-08
4573 1.01800098222388e-08
4574 1.01747646924083e-08
4575 1.0179740167926e-08
4576 1.01746498584049e-08
4577 1.01766318298807e-08
4578 1.01724215332277e-08
4579 1.01724423628297e-08
4580 1.01740964697644e-08
4581 1.01682816893572e-08
4582 1.01719496214125e-08
4583 1.01695936060714e-08
4584 1.01660812318188e-08
4585 1.01691022278694e-08
4586 1.01650711120116e-08
4587 1.01703480019366e-08
4588 1.01659618224476e-08
4589 1.01684575502387e-08
4590 1.01665374746401e-08
4591 1.01630039563705e-08
4592 1.01659355646599e-08
4593 1.01615253096515e-08
4594 1.01636387524473e-08
4595 1.015899718812e-08
4596 1.01617863012132e-08
4597 1.01581087641328e-08
4598 1.01608044395102e-08
4599 1.01547619923287e-08
4600 1.01592091642511e-08
4601 1.01543556855765e-08
4602 1.01566999750197e-08
4603 1.01557513077222e-08
4604 1.01520578439096e-08
4605 1.01546078125209e-08
4606 1.01490791605263e-08
4607 1.01527959556758e-08
4608 1.01519751488782e-08
4609 1.0147587584608e-08
4610 1.01498247277038e-08
4611 1.01450326303043e-08
4612 1.01480504552109e-08
4613 1.01472140230791e-08
4614 1.01434077719037e-08
4615 1.01461591351104e-08
4616 1.01406353102823e-08
4617 1.01444557708691e-08
4618 1.01430599478147e-08
4619 1.01383202882471e-08
4620 1.01421470611734e-08
4621 1.01374925248893e-08
4622 1.01396061677889e-08
4623 1.0135818841156e-08
4624 1.01384162430745e-08
4625 1.01339581103643e-08
4626 1.01368398994356e-08
4627 1.01354041810026e-08
4628 1.01311231132106e-08
4629 1.01333563305717e-08
4630 1.01301802667569e-08
4631 1.0131651794712e-08
4632 1.01319768101155e-08
4633 1.01264058008782e-08
4634 1.01299646651093e-08
4635 1.01254176092941e-08
4636 1.01273975759292e-08
4637 1.01278056733323e-08
4638 1.01227558429451e-08
4639 1.01251841224809e-08
4640 1.01218332480973e-08
4641 1.01231371405774e-08
4642 1.01199991522222e-08
4643 1.01214236671587e-08
4644 1.01185851399954e-08
4645 1.01169291853026e-08
4646 1.01155798322844e-08
4647 1.01191976571574e-08
4648 1.01149374790166e-08
4649 1.01165701828221e-08
4650 1.0116176156108e-08
4651 1.01121938153176e-08
4652 1.0111780680222e-08
4653 1.01101536991027e-08
4654 1.011014027201e-08
4655 1.01109876454453e-08
4656 1.01078265025006e-08
4657 1.01077732487381e-08
4658 1.01063992195144e-08
4659 1.01049317628998e-08
4660 1.01047639979313e-08
4661 1.01036755254108e-08
4662 1.01063018259351e-08
4663 1.01014178320591e-08
4664 1.01043775150716e-08
4665 1.01033831996639e-08
4666 1.00990770495746e-08
4667 1.00980821726576e-08
4668 1.00980809013135e-08
4669 1.00974521050623e-08
4670 1.00968205090785e-08
4671 1.0095121225781e-08
4672 1.00943617625526e-08
4673 1.00947079138824e-08
4674 1.00926178378674e-08
4675 1.00921609387378e-08
4676 1.00908872460448e-08
4677 1.00911325211367e-08
4678 1.0090248600339e-08
4679 1.00892682536463e-08
4680 1.00870528360747e-08
4681 1.00877685484252e-08
4682 1.00865875515693e-08
4683 1.008558873139e-08
4684 1.00845281971601e-08
4685 1.00839452007451e-08
4686 1.00827255743313e-08
4687 1.00824160939039e-08
4688 1.00814427308332e-08
4689 1.0080749137123e-08
4690 1.00791082078139e-08
4691 1.00788722653816e-08
4692 1.00782434136887e-08
4693 1.00767365653204e-08
4694 1.00766955061782e-08
4695 1.00748490508978e-08
4696 1.00746587666095e-08
4697 1.00741439556379e-08
4698 1.00731461516596e-08
4699 1.00706230698744e-08
4700 1.00700942359672e-08
4701 1.00707645401404e-08
4702 1.00686835154962e-08
4703 1.00677177690967e-08
4704 1.00683012700525e-08
4705 1.00665972841835e-08
4706 1.0064812122082e-08
4707 1.00624785672465e-08
4708 1.00634970502528e-08
4709 1.0060953663224e-08
4710 1.00613599638699e-08
4711 1.00605280340255e-08
4712 1.00588830854453e-08
4713 1.00590703518422e-08
4714 1.00586334928626e-08
4715 1.00569654042787e-08
4716 1.00548120836735e-08
4717 1.00559570859332e-08
4718 1.00550042478192e-08
4719 1.00523682907139e-08
4720 1.00532037217577e-08
4721 1.00509643068264e-08
4722 1.00511212468429e-08
4723 1.00499558391004e-08
4724 1.00497302560204e-08
4725 1.00460658822854e-08
4726 1.00477670876425e-08
4727 1.00467127648329e-08
4728 1.00451138941166e-08
4729 1.0045767084077e-08
4730 1.00447254243324e-08
4731 1.00419622040271e-08
4732 1.00433086868412e-08
4733 1.00415000790915e-08
4734 1.00393842486224e-08
4735 1.00411078361728e-08
4736 1.00400413563656e-08
4737 1.00370126141086e-08
4738 1.00380975038838e-08
4739 1.00363663013137e-08
4740 1.00344490932686e-08
4741 1.00358062894018e-08
4742 1.00348682374046e-08
4743 1.00331923782726e-08
4744 1.00325697345166e-08
4745 1.00324692541565e-08
4746 1.00294997266159e-08
4747 1.00295199359224e-08
4748 1.00287151679479e-08
4749 1.00271382679901e-08
4750 1.00265178257092e-08
4751 1.0024953460086e-08
4752 1.00247007680382e-08
4753 1.00237977148504e-08
4754 1.00244759578677e-08
4755 1.00217926837093e-08
4756 1.00227725946672e-08
4757 1.00217763286808e-08
4758 1.00187363816645e-08
4759 1.00198766735254e-08
4760 1.00188390018685e-08
4761 1.00170527028709e-08
4762 1.00179590289962e-08
4763 1.00164541366327e-08
4764 1.00140325559744e-08
4765 1.00145049905104e-08
4766 1.00137473713485e-08
4767 1.00121719464469e-08
4768 1.00114560365322e-08
4769 1.00113062757728e-08
4770 1.00099132986503e-08
4771 1.00073507090653e-08
4772 1.00083789389277e-08
4773 1.0008253539362e-08
4774 1.00062815782537e-08
4775 1.00067080155281e-08
4776 1.00047796914721e-08
4777 1.00037996365321e-08
4778 1.00030466178985e-08
4779 1.00023423364598e-08
4780 1.00023607645661e-08
4781 1.00016439172623e-08
4782 9.9996757487314e-09
4783 9.99837273907289e-09
4784 9.99960846483072e-09
4785 9.99845943398803e-09
4786 9.99639688167853e-09
4787 9.99623995599774e-09
4788 9.99498921645803e-09
4789 9.99446406481364e-09
4790 9.99136321071237e-09
4791 9.99320087506639e-09
4792 9.99273980428761e-09
4793 9.98935947477131e-09
4794 9.99064524692672e-09
4795 9.98910553828836e-09
4796 9.98686954264971e-09
4797 9.98789807087064e-09
4798 9.98742873510594e-09
4799 9.98399322514609e-09
4800 9.98482725539374e-09
4801 9.98325429021896e-09
4802 9.98335189912602e-09
4803 9.98175501278065e-09
4804 9.98121307897726e-09
4805 9.98093771981123e-09
4806 9.97721545066554e-09
4807 9.97902438461262e-09
4808 9.9789798946176e-09
4809 9.9775810205871e-09
4810 9.97576655792365e-09
4811 9.97506026124972e-09
4812 9.97484958574491e-09
4813 9.97454746104465e-09
4814 9.97118087212034e-09
4815 9.97188373738633e-09
4816 9.97249425387092e-09
4817 9.97100676793094e-09
4818 9.96952588232236e-09
4819 9.96886149999543e-09
4820 9.96689236930359e-09
4821 9.96789927229513e-09
4822 9.96529965531512e-09
4823 9.96509942986779e-09
4824 9.9647585793533e-09
4825 9.9647125103991e-09
4826 9.96407347034212e-09
4827 9.96227445609899e-09
4828 9.9611502144964e-09
4829 9.95984099771974e-09
4830 9.96047153117852e-09
4831 9.96028740860022e-09
4832 9.95682825913335e-09
4833 9.95664048526745e-09
4834 9.95614439931736e-09
4835 9.95598373462747e-09
4836 9.95440135595493e-09
4837 9.95431971041738e-09
4838 9.9533971218424e-09
4839 9.95132050768399e-09
4840 9.95023670739836e-09
4841 9.9501287309095e-09
4842 9.95113103664791e-09
4843 9.94890058535203e-09
4844 9.94684372131183e-09
4845 9.94731921948044e-09
4846 9.94755251490509e-09
4847 9.94699584030978e-09
4848 9.94534645158696e-09
4849 9.94512837010353e-09
4850 9.94250953878506e-09
4851 9.9422960281037e-09
4852 9.94256444536434e-09
4853 9.94212676605488e-09
4854 9.93852620194413e-09
4855 9.93881560067134e-09
4856 9.93777582798672e-09
4857 9.93760813977618e-09
4858 9.9359243621866e-09
4859 9.93689695479028e-09
4860 9.93440840578452e-09
4861 9.93384630823546e-09
4862 9.93270304153193e-09
4863 9.93293684579954e-09
4864 9.93167430733688e-09
4865 9.93104305667403e-09
4866 9.92833185073372e-09
4867 9.92986392511774e-09
4868 9.92865074943283e-09
4869 9.92611948590072e-09
4870 9.9250599940176e-09
4871 9.92488063496222e-09
4872 9.92470759537956e-09
4873 9.92477455512286e-09
4874 9.92339855769575e-09
4875 9.92081048598048e-09
4876 9.92006356001496e-09
4877 9.92052123002796e-09
4878 9.91888497212101e-09
4879 9.91843581123197e-09
4880 9.9181022383682e-09
4881 9.91721626995845e-09
4882 9.91605927543049e-09
4883 9.91530765082071e-09
4884 9.91439241423975e-09
4885 9.91441383117741e-09
4886 9.91332470998141e-09
4887 9.91138907914524e-09
4888 9.91157627470984e-09
4889 9.91017160109631e-09
4890 9.90949195464608e-09
4891 9.90775314603509e-09
4892 9.90710734020961e-09
4893 9.90714371550128e-09
4894 9.90592685554126e-09
4895 9.90547953769183e-09
4896 9.90488866746408e-09
4897 9.90335339284831e-09
4898 9.90323561365913e-09
4899 9.90082088449251e-09
4900 9.90128264118095e-09
4901 9.90036140798012e-09
4902 9.89851716122481e-09
4903 9.89859751780731e-09
4904 9.89775857060393e-09
4905 9.89633056752703e-09
4906 9.89682507680778e-09
4907 9.89486532861073e-09
4908 9.89377505891675e-09
4909 9.89413233405678e-09
4910 9.89308873111661e-09
4911 9.89400213424752e-09
4912 9.89189716361916e-09
4913 9.88938372052661e-09
4914 9.89000255324779e-09
4915 9.89002703498387e-09
4916 9.88817994637681e-09
4917 9.88774721109864e-09
4918 9.88728512506254e-09
4919 9.88490093718764e-09
4920 9.88449034650118e-09
4921 9.88346126197526e-09
4922 9.8824968583211e-09
4923 9.88329869130999e-09
4924 9.88127879889e-09
4925 9.88012397146187e-09
4926 9.8802398704606e-09
4927 9.87865480078787e-09
4928 9.87841008982804e-09
4929 9.87742148479687e-09
4930 9.87585710977845e-09
4931 9.8756259227506e-09
4932 9.87449085644576e-09
4933 9.87244834373124e-09
4934 9.87514479200424e-09
4935 9.8732388730538e-09
4936 9.87086656065172e-09
4937 9.87041163111324e-09
4938 9.87031456947674e-09
4939 9.87157284688955e-09
4940 9.86809108671771e-09
4941 9.86750288052563e-09
4942 9.86737464671317e-09
4943 9.86515983247838e-09
4944 9.86403121221169e-09
4945 9.8641868209598e-09
4946 9.86250120273147e-09
4947 9.86180401250925e-09
4948 9.86090350554858e-09
4949 9.86071188407989e-09
4950 9.85987571522351e-09
4951 9.85809950759808e-09
4952 9.8580690077732e-09
4953 9.85739888478832e-09
4954 9.8549047950619e-09
4955 9.85570092078281e-09
4956 9.85426372555886e-09
4957 9.85277936192685e-09
4958 9.85160664572493e-09
4959 9.85183114171617e-09
4960 9.85045560757125e-09
4961 9.84949830463988e-09
4962 9.84842949293407e-09
4963 9.84759374568489e-09
4964 9.84772560022607e-09
4965 9.84571925234656e-09
4966 9.84397650682922e-09
4967 9.84464414363617e-09
4968 9.84395285248685e-09
4969 9.84172280649176e-09
4970 9.84112696486572e-09
4971 9.84061839857941e-09
4972 9.84019943538195e-09
4973 9.83810048155176e-09
4974 9.83814790923854e-09
4975 9.83772996762666e-09
4976 9.8365597002853e-09
4977 9.83539409778189e-09
4978 9.83446162236556e-09
4979 9.8346490930018e-09
4980 9.832294568482e-09
4981 9.83123627479321e-09
4982 9.83143902318073e-09
4983 9.82955407968777e-09
4984 9.83009306713967e-09
4985 9.82674792383842e-09
4986 9.82770297762447e-09
4987 9.8265632005462e-09
4988 9.82579553691909e-09
4989 9.82469176612211e-09
4990 9.82763376448925e-09
4991 9.82218606254692e-09
4992 9.82318465207743e-09
4993 9.8211013791899e-09
4994 9.81983608705494e-09
4995 9.81979781318476e-09
4996 9.8177997702037e-09
4997 9.81720973546041e-09
4998 9.81675244242364e-09
4999 9.8149975926165e-09
};
\addlegendentry{Train}
\addplot [semithick, black]
table {%
0 0.000844808062538505
1 0.000208711557206698
2 0.000196937893633731
3 0.000183461743290536
4 0.000149782586959191
5 8.43533780425787e-05
6 3.90812929254025e-05
7 3.09006754832808e-05
8 2.76371138170362e-05
9 2.37405001826119e-05
10 1.86681008926826e-05
11 1.33545354401576e-05
12 8.87744226929499e-06
13 6.00852945353836e-06
14 4.56939051218797e-06
15 3.75559784515644e-06
16 3.18063644044742e-06
17 2.739854153333e-06
18 2.39844712268678e-06
19 2.14078932003758e-06
20 1.94881522475043e-06
21 1.8103229422195e-06
22 1.7085088757085e-06
23 1.63263359809207e-06
24 1.57770784880995e-06
25 1.52424615862401e-06
26 1.4683309927932e-06
27 1.41961857025308e-06
28 1.36831772579171e-06
29 1.32251977902342e-06
30 1.27716361930652e-06
31 1.23100915061514e-06
32 1.18576554086758e-06
33 1.14247063720541e-06
34 1.09843369955342e-06
35 1.06554364265321e-06
36 1.01784200978727e-06
37 9.78036382548453e-07
38 9.41370217333315e-07
39 9.06908098841086e-07
40 8.7309450691464e-07
41 8.43667010030913e-07
42 8.18539774627425e-07
43 7.95028995526081e-07
44 7.75821092702245e-07
45 7.55929704610026e-07
46 7.38149935841648e-07
47 7.23268897218077e-07
48 7.09294056377985e-07
49 6.9623297349608e-07
50 6.83657901845436e-07
51 6.71110001349007e-07
52 6.61495278109214e-07
53 6.50248750844185e-07
54 6.39307927485788e-07
55 6.29124087936361e-07
56 6.19705929238989e-07
57 6.03134822085849e-07
58 5.93593881603738e-07
59 5.8516548051557e-07
60 5.76683532926836e-07
61 5.6939973092085e-07
62 5.61493038730987e-07
63 5.54286941678583e-07
64 5.48190428162343e-07
65 5.42428949756868e-07
66 5.38981339559541e-07
67 5.33953425474465e-07
68 5.29370936419582e-07
69 5.25470170487097e-07
70 5.23020389664453e-07
71 5.19717559654964e-07
72 5.17404771471774e-07
73 5.15599879236106e-07
74 5.13117299760779e-07
75 5.1091973318762e-07
76 5.08224673012592e-07
77 5.05819798490847e-07
78 5.03742853652511e-07
79 5.01774991334969e-07
80 4.98819815675233e-07
81 4.96751511036564e-07
82 4.9480121333545e-07
83 4.92912988647731e-07
84 4.90857075874374e-07
85 4.89075830500951e-07
86 4.87242971303203e-07
87 4.85643624870136e-07
88 4.8382412387582e-07
89 4.81792937989667e-07
90 4.79978666589886e-07
91 4.78777678836195e-07
92 4.7686586412965e-07
93 4.75015212941798e-07
94 4.73103000331321e-07
95 4.711125995982e-07
96 4.69197317443104e-07
97 4.67405953941125e-07
98 4.65519434555972e-07
99 4.63757260149578e-07
100 4.60991543604905e-07
101 4.60338213770228e-07
102 4.58234580946737e-07
103 4.56393081549322e-07
104 4.55003856814074e-07
105 4.53174891390518e-07
106 4.52220632496392e-07
107 4.50541421059825e-07
108 4.49058632057131e-07
109 4.47349293608568e-07
110 4.45466184828547e-07
111 4.43503978431181e-07
112 4.41549445895362e-07
113 4.39636750115824e-07
114 4.37234518813057e-07
115 4.35717538493918e-07
116 4.33864698834441e-07
117 4.32095532687526e-07
118 4.30377923521519e-07
119 4.2867333149843e-07
120 4.26919569918027e-07
121 4.28820357001314e-07
122 4.26964419375508e-07
123 4.2517999077063e-07
124 4.23418498485262e-07
125 4.21756652713157e-07
126 4.20540857248852e-07
127 4.18880034658287e-07
128 4.17273213315639e-07
129 4.15573481404863e-07
130 4.13992381709249e-07
131 4.12385247727798e-07
132 4.10624352298328e-07
133 4.08998744205746e-07
134 4.07772915878013e-07
135 4.07750462727563e-07
136 4.06637809646782e-07
137 4.05437248218732e-07
138 4.04136045517589e-07
139 4.02751055617045e-07
140 4.01625499080183e-07
141 4.00324012161946e-07
142 3.98826529135476e-07
143 3.97591406908759e-07
144 3.97958785924857e-07
145 3.97476952684883e-07
146 3.95962274524209e-07
147 3.94428440131378e-07
148 3.93035548995613e-07
149 3.91381831832405e-07
150 3.89989736504504e-07
151 3.88788663485684e-07
152 3.87485727060266e-07
153 3.86974363664194e-07
154 3.83994546382382e-07
155 3.82769201223709e-07
156 3.81406238147974e-07
157 3.80140647848748e-07
158 3.78728230998604e-07
159 3.78222608787837e-07
160 3.76713188643407e-07
161 3.75466044033601e-07
162 3.73678972209746e-07
163 3.72167306750271e-07
164 3.7044185319246e-07
165 3.69100007446832e-07
166 3.67674175549837e-07
167 3.66364872661507e-07
168 3.65005604407997e-07
169 3.63648183565601e-07
170 3.61263090553621e-07
171 3.59584191755857e-07
172 3.58225861418759e-07
173 3.56780077481744e-07
174 3.54073875996619e-07
175 3.52527223412835e-07
176 3.51068990767089e-07
177 3.4970972251358e-07
178 3.48558245377717e-07
179 3.47183032545217e-07
180 3.46265693451642e-07
181 3.44637356874955e-07
182 3.4332529708081e-07
183 3.41835601602725e-07
184 3.4089325140485e-07
185 3.39665660931132e-07
186 3.38495539153882e-07
187 3.37413212037063e-07
188 3.36293936697984e-07
189 3.35519871441647e-07
190 3.34262239221061e-07
191 3.33132277319237e-07
192 3.32187738649736e-07
193 3.31371552420023e-07
194 3.30608429521817e-07
195 3.29762400497202e-07
196 3.28767441715172e-07
197 3.28054881038042e-07
198 3.27246254983038e-07
199 3.26171146980414e-07
200 3.2554314088884e-07
201 3.24539797702528e-07
202 3.23681661029696e-07
203 3.22949773590153e-07
204 3.22280669706743e-07
205 3.21527664937094e-07
206 3.20541630571824e-07
207 3.1974150260794e-07
208 3.1902146702123e-07
209 3.18272952881671e-07
210 3.17570084007457e-07
211 3.16858063342806e-07
212 3.16331636440736e-07
213 3.15530286343346e-07
214 3.14729334149888e-07
215 3.14031979087304e-07
216 3.13191122813805e-07
217 3.12157283133274e-07
218 3.11281155518373e-07
219 3.10678672121867e-07
220 3.09965031419779e-07
221 3.0897348324288e-07
222 3.08490314182563e-07
223 3.07777099806117e-07
224 3.07262098431238e-07
225 3.06590322907141e-07
226 3.05925340171598e-07
227 3.05384389776009e-07
228 3.04676689211192e-07
229 3.04088842995043e-07
230 3.03542208257568e-07
231 3.02866851598083e-07
232 3.02400849250262e-07
233 3.01491155596523e-07
234 3.00935482755449e-07
235 3.00260182939383e-07
236 2.99667107128698e-07
237 2.99313995810735e-07
238 2.98687922395402e-07
239 2.97841353358308e-07
240 2.97213262001605e-07
241 2.96567122859415e-07
242 2.96191274173907e-07
243 2.95359910751358e-07
244 2.94889474616866e-07
245 2.94281761625825e-07
246 2.93639942583468e-07
247 2.93305163268087e-07
248 2.92546218361167e-07
249 2.92216810748869e-07
250 2.91736483859495e-07
251 2.91013776632099e-07
252 2.90642901745741e-07
253 2.89909223738505e-07
254 2.89543464759845e-07
255 2.88894682398677e-07
256 2.88196616793357e-07
257 2.87474904325791e-07
258 2.86898568901961e-07
259 2.85780629383225e-07
260 2.85205743466577e-07
261 2.84559746432933e-07
262 2.84279138895727e-07
263 2.83695214875479e-07
264 2.83075308971092e-07
265 2.82265915529933e-07
266 2.81335900353952e-07
267 2.8064141588402e-07
268 2.79939627034764e-07
269 2.79169540817747e-07
270 2.78937449138539e-07
271 2.78443195611544e-07
272 2.77810300985948e-07
273 2.77105982604553e-07
274 2.76410958122142e-07
275 2.76044971769807e-07
276 2.75542959116137e-07
277 2.7479086384119e-07
278 2.74389265086938e-07
279 2.73477354539864e-07
280 2.73083259116902e-07
281 2.72413473112465e-07
282 2.72159923042636e-07
283 2.71589556177787e-07
284 2.71133131946044e-07
285 2.70454052042624e-07
286 2.70592522610968e-07
287 2.69597791202614e-07
288 2.69148642928485e-07
289 2.68527458047174e-07
290 2.67838629497419e-07
291 2.67260958253246e-07
292 2.67261299313759e-07
293 2.66473335841511e-07
294 2.66147111460668e-07
295 2.65917208253086e-07
296 2.65325354575907e-07
297 2.6490707227822e-07
298 2.63787200083243e-07
299 2.63508241005184e-07
300 2.62809095374905e-07
301 2.62265501760339e-07
302 2.61756639474697e-07
303 2.61032539583539e-07
304 2.60815482988619e-07
305 2.60269246155076e-07
306 2.5974711093113e-07
307 2.59418357018149e-07
308 2.59001836866446e-07
309 2.58572583788919e-07
310 2.58113118434267e-07
311 2.57709018569585e-07
312 2.57229856970298e-07
313 2.56893770256283e-07
314 2.56493592587503e-07
315 2.55844184948728e-07
316 2.557383140811e-07
317 2.54857468462433e-07
318 2.54541220101601e-07
319 2.54465078342037e-07
320 2.53791426985117e-07
321 2.53287424811788e-07
322 2.53177233844326e-07
323 2.52503554065697e-07
324 2.52100306852299e-07
325 2.51865060363343e-07
326 2.51368788894979e-07
327 2.50829600645375e-07
328 2.50396595902203e-07
329 2.50103795451651e-07
330 2.49880599767494e-07
331 2.49433242061059e-07
332 2.48692060722533e-07
333 2.48833345040111e-07
334 2.47914584861064e-07
335 2.47401459319008e-07
336 2.47551696475057e-07
337 2.46573677031847e-07
338 2.46664143332964e-07
339 2.45871149218146e-07
340 2.45256842390518e-07
341 2.45081366756494e-07
342 2.44597771370536e-07
343 2.44053069309302e-07
344 2.43727527049487e-07
345 2.43335250615928e-07
346 2.43120922505113e-07
347 2.42708068753927e-07
348 2.4222293859566e-07
349 2.41812642798322e-07
350 2.41191372651883e-07
351 2.41109120224792e-07
352 2.40273806184632e-07
353 2.39844382576848e-07
354 2.39412315750087e-07
355 2.3883936250968e-07
356 2.3825953121559e-07
357 2.38166592225753e-07
358 2.37299133232227e-07
359 2.36881447790438e-07
360 2.36273137943499e-07
361 2.35701634210272e-07
362 2.33968464158352e-07
363 2.32997848570449e-07
364 2.32391684562572e-07
365 2.3208987443013e-07
366 2.31504330372445e-07
367 2.30924598554338e-07
368 2.31044950282921e-07
369 2.29927735517776e-07
370 2.29836956577856e-07
371 2.2916385944427e-07
372 2.2865665982863e-07
373 2.28500923071806e-07
374 2.27651739237444e-07
375 2.27512785500039e-07
376 2.27166140120971e-07
377 2.26694595539811e-07
378 2.26519560442284e-07
379 2.25970040901302e-07
380 2.25285177180012e-07
381 2.24853252461799e-07
382 2.24575714469211e-07
383 2.24358188916085e-07
384 2.23655035824777e-07
385 2.23413792355132e-07
386 2.23031108248506e-07
387 2.22910330194281e-07
388 2.22281897777066e-07
389 2.21784617338017e-07
390 2.20917272031329e-07
391 2.21140723510871e-07
392 2.20410512952185e-07
393 2.2001032107255e-07
394 2.19877264839852e-07
395 2.19082124885972e-07
396 2.18699526044475e-07
397 2.18523794615066e-07
398 2.17944844393969e-07
399 2.17034383354076e-07
400 2.17610505615085e-07
401 2.16876401282207e-07
402 2.16612207282196e-07
403 2.15967503436332e-07
404 2.15685574289637e-07
405 2.15279271742475e-07
406 2.14905298889789e-07
407 2.14226957950814e-07
408 2.13599463450009e-07
409 2.12981731806394e-07
410 2.12571038105125e-07
411 2.12624627238256e-07
412 2.12070219163252e-07
413 2.11784808357152e-07
414 2.11318706533348e-07
415 2.11045104947516e-07
416 2.10112460763412e-07
417 2.10088003882447e-07
418 2.09786023219749e-07
419 2.09440742082734e-07
420 2.08965985848408e-07
421 2.08745419172374e-07
422 2.07855677558655e-07
423 2.07679462960186e-07
424 2.07805385343818e-07
425 2.07102601734732e-07
426 2.06890589993236e-07
427 2.06494206622665e-07
428 2.06372362754337e-07
429 2.05817769938221e-07
430 2.0557484958772e-07
431 2.05229099492499e-07
432 2.04676638304591e-07
433 2.04507443868351e-07
434 2.04162660111251e-07
435 2.03693204525734e-07
436 2.03952879473945e-07
437 2.03080247729304e-07
438 2.02900650947413e-07
439 2.02672666205217e-07
440 2.0208744899719e-07
441 2.01934327037634e-07
442 2.01445914171927e-07
443 2.00983393483511e-07
444 2.00684112883209e-07
445 2.00545883899395e-07
446 2.00423372120895e-07
447 2.00078716261487e-07
448 1.99741421624822e-07
449 1.9933409589612e-07
450 1.98867056155905e-07
451 1.98816394458845e-07
452 1.98642553073114e-07
453 1.99311656956525e-07
454 1.98615353497189e-07
455 1.98735008893891e-07
456 1.98479057189616e-07
457 1.98057165334831e-07
458 1.9815948348878e-07
459 1.98087178659989e-07
460 1.97764506992826e-07
461 1.97527882050963e-07
462 1.97275383584383e-07
463 1.97648844846299e-07
464 1.98105297499751e-07
465 1.97678858171457e-07
466 1.97499161913584e-07
467 1.97017001823951e-07
468 1.96776383631914e-07
469 1.9652185301311e-07
470 1.96349716929944e-07
471 1.96090198301135e-07
472 1.95881355580241e-07
473 1.95614560993818e-07
474 1.95283149651004e-07
475 1.95382853007686e-07
476 1.95083401877127e-07
477 2.03233227580313e-07
478 2.04084031452112e-07
479 2.01131697963319e-07
480 2.03101265583427e-07
481 2.03073540205878e-07
482 2.00585390075503e-07
483 2.01559231527426e-07
484 2.02195096221658e-07
485 1.99894529373523e-07
486 2.01185258674741e-07
487 2.01320290216245e-07
488 2.01506466623869e-07
489 1.98801657802505e-07
490 2.01094152885162e-07
491 1.98450393895655e-07
492 1.99435731929043e-07
493 2.00144199880015e-07
494 1.99753543483894e-07
495 1.99202929707099e-07
496 1.99304352577201e-07
497 1.99807985268308e-07
498 1.96566162458112e-07
499 1.9682721585923e-07
500 1.96316392475637e-07
501 1.96144384290164e-07
502 1.97901925957922e-07
503 1.97607661789334e-07
504 1.97397355350404e-07
505 1.97079273789313e-07
506 1.97546128788417e-07
507 1.95139179481885e-07
508 1.9611444201928e-07
509 1.9749707291794e-07
510 1.94813594589505e-07
511 1.9645298721116e-07
512 1.94474978343351e-07
513 1.95969889205116e-07
514 1.94117959040341e-07
515 1.95436498984236e-07
516 1.93606339848884e-07
517 1.946815757492e-07
518 1.94778976947418e-07
519 1.92802843912432e-07
520 1.95104462363815e-07
521 1.92555631883806e-07
522 1.93635599998743e-07
523 1.92106028862327e-07
524 1.93856124042213e-07
525 1.91709204955259e-07
526 1.95507226408154e-07
527 1.91437450780541e-07
528 1.93280058624623e-07
529 1.90958843404587e-07
530 1.92891363326453e-07
531 1.90727106996746e-07
532 1.92547886967986e-07
533 1.90484698237015e-07
534 1.92346050198466e-07
535 1.88896109420966e-07
536 1.91733676047079e-07
537 1.88953933388802e-07
538 1.92076143434861e-07
539 1.87840512921866e-07
540 1.91194217791235e-07
541 1.92035159329862e-07
542 1.90670135680193e-07
543 1.91556452477926e-07
544 1.90446201031591e-07
545 1.90558196777602e-07
546 1.91110089531321e-07
547 1.90009203038244e-07
548 1.90040935876823e-07
549 1.90433226521236e-07
550 1.90699410040907e-07
551 1.90076079320534e-07
552 1.87324687317414e-07
553 1.90985659287435e-07
554 1.89805206218807e-07
555 1.90129242128023e-07
556 1.89801170336068e-07
557 1.89033002584438e-07
558 1.89222674862322e-07
559 1.89191879940154e-07
560 1.89123340987862e-07
561 1.89169156783464e-07
562 1.88550288271472e-07
563 1.89358061675193e-07
564 1.89083209534147e-07
565 1.88436388270929e-07
566 1.89312316933865e-07
567 1.88273105550252e-07
568 1.85061026058975e-07
569 1.8908653487415e-07
570 1.83717716595311e-07
571 1.87931547657172e-07
572 1.83886498916763e-07
573 1.88113858712313e-07
574 1.89957432894516e-07
575 1.89689728813391e-07
576 1.89752469736959e-07
577 1.88004435131006e-07
578 1.81890655426287e-07
579 1.87482342539624e-07
580 1.89461388799828e-07
581 1.89417406204484e-07
582 1.88973018566685e-07
583 1.89324509847211e-07
584 1.89227634450617e-07
585 1.82162565920407e-07
586 1.88507428333651e-07
587 1.83216570803779e-07
588 1.87280093655318e-07
589 1.82890303790373e-07
590 1.83293494160353e-07
591 1.86790771294909e-07
592 1.82645919721836e-07
593 1.84778897960314e-07
594 1.87950050190011e-07
595 1.80814978989474e-07
596 1.8650108302154e-07
597 1.82085216238193e-07
598 1.83542255172142e-07
599 1.79599808802777e-07
600 1.87450922339849e-07
601 1.87688883102055e-07
602 1.85536791263985e-07
603 1.87528286232919e-07
604 1.85114956252619e-07
605 1.86274192515157e-07
606 1.82860532049745e-07
607 1.79265185806798e-07
608 1.78709143483502e-07
609 1.79033364133829e-07
610 1.7869054147468e-07
611 1.7911118277425e-07
612 1.78115342919227e-07
613 1.78825331431653e-07
614 1.78147473661738e-07
615 1.77092871922468e-07
616 1.77744794882528e-07
617 1.77845976168101e-07
618 1.77040291760022e-07
619 1.78858584831687e-07
620 1.76878572233363e-07
621 1.83280292276322e-07
622 1.76903370174841e-07
623 1.77915822519026e-07
624 1.76647375838002e-07
625 1.77569958736967e-07
626 1.76356678593947e-07
627 1.76436500964883e-07
628 1.76347910496588e-07
629 1.7717988498589e-07
630 1.75975614524759e-07
631 1.75900609633572e-07
632 1.77217955865672e-07
633 1.75587629769325e-07
634 1.76081059066746e-07
635 1.76721101752264e-07
636 1.75233651589224e-07
637 1.75341327235401e-07
638 1.75611447161828e-07
639 1.75419799575138e-07
640 1.74884590364854e-07
641 1.76134079765689e-07
642 1.74699579247317e-07
643 1.7396195062247e-07
644 1.75766956544976e-07
645 1.7419971243271e-07
646 1.74277744235951e-07
647 1.7546115316236e-07
648 1.74106958183984e-07
649 1.74162394728228e-07
650 1.75317595108027e-07
651 1.74385760942641e-07
652 1.73437300077239e-07
653 1.73692640714762e-07
654 1.73574562722933e-07
655 1.73387903146249e-07
656 1.74210612158276e-07
657 1.73098513300829e-07
658 1.73205521036834e-07
659 1.74268393493549e-07
660 1.7308649091774e-07
661 1.73228428934635e-07
662 1.73997861452335e-07
663 1.72488768157564e-07
664 1.72476319448833e-07
665 1.72514518226308e-07
666 1.73013972926128e-07
667 1.71854750874445e-07
668 1.72713242818645e-07
669 1.71931048953411e-07
670 1.72133724163359e-07
671 1.72064858361409e-07
672 1.72063849390724e-07
673 1.71869402265656e-07
674 1.71346016486496e-07
675 1.71444384022834e-07
676 1.71429732631623e-07
677 1.71212235500207e-07
678 1.71808167692689e-07
679 1.71624378708657e-07
680 1.70713477132267e-07
681 1.70713562397395e-07
682 1.71625941902676e-07
683 1.70651688335965e-07
684 1.69649595704868e-07
685 1.70435612290021e-07
686 1.70420904055391e-07
687 1.70043605862702e-07
688 1.70738516658275e-07
689 1.70552070244412e-07
690 1.69706325436891e-07
691 1.69789316828428e-07
692 1.70340697991378e-07
693 1.69609975841922e-07
694 1.70193189319434e-07
695 1.69863739074572e-07
696 1.68982097648041e-07
697 1.69162277074975e-07
698 1.69382403214513e-07
699 1.69544762229634e-07
700 1.68485570384291e-07
701 1.70164781820858e-07
702 1.69187075016453e-07
703 1.6856371587437e-07
704 1.68610625905785e-07
705 1.68519932230993e-07
706 1.69139511285721e-07
707 1.6737638475206e-07
708 1.69543284300744e-07
709 1.74123059082376e-07
710 1.69667487170955e-07
711 1.68747888551479e-07
712 1.67866303968367e-07
713 1.67392968819513e-07
714 1.68199321137763e-07
715 1.73418300164485e-07
716 1.68168241998501e-07
717 1.67042216503432e-07
718 1.67641886150705e-07
719 1.6620556664293e-07
720 1.67069359235938e-07
721 1.66195462725227e-07
722 1.67115587146327e-07
723 1.65926238082648e-07
724 1.6704684924207e-07
725 1.65676937058379e-07
726 1.66700402814968e-07
727 1.6647425127303e-07
728 1.6561901361456e-07
729 1.66252831945712e-07
730 1.67014434282464e-07
731 1.66326771022796e-07
732 1.66372856824637e-07
733 1.65163854148886e-07
734 1.6655260992593e-07
735 1.64965868520994e-07
736 1.6635448218949e-07
737 1.6564321470014e-07
738 1.64334565511126e-07
739 1.65173673849495e-07
740 1.65506847338293e-07
741 1.65594457257612e-07
742 1.66086110198194e-07
743 1.65011172725826e-07
744 1.65103415383783e-07
745 1.63400969199756e-07
746 1.64151074955043e-07
747 1.64772330890628e-07
748 1.65025213050285e-07
749 1.64716425388178e-07
750 1.63514613404914e-07
751 1.64683768844043e-07
752 1.63086497195764e-07
753 1.64650359124607e-07
754 1.62961569571962e-07
755 1.64566586136061e-07
756 1.64611265063286e-07
757 1.64635551413994e-07
758 1.64834347060605e-07
759 1.65162816756492e-07
760 1.65388954087575e-07
761 1.65940420515653e-07
762 1.65819386666044e-07
763 1.66130917023111e-07
764 1.65988367939462e-07
765 1.66071671969803e-07
766 1.65892302561588e-07
767 1.66044785032682e-07
768 1.65938430995993e-07
769 1.64848046324551e-07
770 1.70334416793594e-07
771 1.62698569283748e-07
772 1.65680461350348e-07
773 1.63039644007767e-07
774 1.65216377467914e-07
775 1.6238661260104e-07
776 1.65064065527076e-07
777 1.62526859526224e-07
778 1.62293588346074e-07
779 1.63644472195301e-07
780 1.61936213771696e-07
781 1.70261699850016e-07
782 1.61748630489456e-07
783 1.62629760325217e-07
784 1.64105799171921e-07
785 1.62374945489319e-07
786 1.62671852876883e-07
787 1.61723932023961e-07
788 1.64453879847315e-07
789 1.61606635629141e-07
790 1.61718347158057e-07
791 1.63683992582264e-07
792 1.61347372795717e-07
793 1.63181653078937e-07
794 1.68513977882867e-07
795 1.60995355713567e-07
796 1.63453108825706e-07
797 1.60452941599942e-07
798 1.63499606742334e-07
799 1.60606901999927e-07
800 1.76145036334674e-07
801 1.74083723436524e-07
802 1.75793118728507e-07
803 1.74355221815858e-07
804 1.74106588701761e-07
805 1.74142982700687e-07
806 1.74893401094778e-07
807 1.7584135036941e-07
808 1.7271027275001e-07
809 1.75294317728003e-07
810 1.71258932368801e-07
811 1.73022726812633e-07
812 1.72836053025094e-07
813 1.73718632368036e-07
814 1.74233775851462e-07
815 1.73047027374196e-07
816 1.7410140173979e-07
817 1.74126824958876e-07
818 1.7411261410416e-07
819 1.74085073467722e-07
820 1.72635211015404e-07
821 1.71961872297288e-07
822 1.71431182138804e-07
823 1.74538200781171e-07
824 1.68448593740322e-07
825 1.72075104387659e-07
826 1.70767279428219e-07
827 1.69122159832114e-07
828 1.75998778217945e-07
829 1.70020712175756e-07
830 1.7342003388876e-07
831 1.69492395230009e-07
832 1.74284153331428e-07
833 1.69398049365554e-07
834 1.73859760366213e-07
835 1.72317612623374e-07
836 1.71964302353445e-07
837 1.71816239458167e-07
838 1.71936505921622e-07
839 1.71946581417615e-07
840 1.71657760006383e-07
841 1.71266151482996e-07
842 1.71361421053007e-07
843 1.70980797520315e-07
844 1.7117761785812e-07
845 1.71059653553129e-07
846 1.71105739354971e-07
847 1.71110357882753e-07
848 1.70678276845138e-07
849 1.70585920500343e-07
850 1.71061259379712e-07
851 1.7077870495541e-07
852 1.70548048572527e-07
853 1.7041743660684e-07
854 1.70491048834265e-07
855 1.70712738167822e-07
856 1.70092732787452e-07
857 1.69924973647539e-07
858 1.69703099572871e-07
859 1.69678330053102e-07
860 1.68998482763527e-07
861 1.69182186482431e-07
862 1.6864107976744e-07
863 1.68696587365957e-07
864 1.68210561923843e-07
865 1.69011073580805e-07
866 1.69067320143768e-07
867 1.68944623624157e-07
868 1.68204124406657e-07
869 1.68030013014686e-07
870 1.68195583682973e-07
871 1.6795995350094e-07
872 1.67899329994725e-07
873 1.6810892589092e-07
874 1.67820189744816e-07
875 1.68129759003932e-07
876 1.67634013337192e-07
877 1.67442578913324e-07
878 1.66963019410105e-07
879 1.66852643701532e-07
880 1.67096303016478e-07
881 1.66962422554207e-07
882 1.66008348401192e-07
883 1.66715864224898e-07
884 1.66402713830394e-07
885 1.66352606356668e-07
886 1.66231373555092e-07
887 1.65873629498492e-07
888 1.66039811233532e-07
889 1.6563598137509e-07
890 1.66007424695636e-07
891 1.65646241612194e-07
892 1.64933609880791e-07
893 1.6552769466216e-07
894 1.64951046599526e-07
895 1.65064577117846e-07
896 1.65032716381575e-07
897 1.64908328770252e-07
898 1.64794244028599e-07
899 1.64534668556371e-07
900 1.63988616463939e-07
901 1.64784566436538e-07
902 1.63601569624916e-07
903 1.64374981181936e-07
904 1.63475959880088e-07
905 1.63609314540736e-07
906 1.64158279858384e-07
907 1.63348261139618e-07
908 1.63545820441868e-07
909 1.63170810196789e-07
910 1.62907682010882e-07
911 1.62706299988713e-07
912 1.6329676100213e-07
913 1.62652781909856e-07
914 1.62734963282674e-07
915 1.62704296258198e-07
916 1.6318148254868e-07
917 1.62746474074993e-07
918 1.6324209184404e-07
919 1.62740562359431e-07
920 1.61813190402427e-07
921 1.61946630328202e-07
922 1.62296785788385e-07
923 1.61297748491052e-07
924 1.62152630878154e-07
925 1.61258057573832e-07
926 1.61066481041416e-07
927 1.61620945959839e-07
928 1.61473508342169e-07
929 1.61268715714868e-07
930 1.5902121219824e-07
931 1.66026126180441e-07
932 1.58973335828705e-07
933 1.61382772034813e-07
934 1.6247341250164e-07
935 1.56424803776645e-07
936 1.67031586784105e-07
937 1.59779588670972e-07
938 1.59468527272111e-07
939 1.62728667874035e-07
940 1.56236836801327e-07
941 1.67008067819552e-07
942 1.59145869815802e-07
943 1.59577055569571e-07
944 1.62210923804196e-07
945 1.55875270024808e-07
946 1.66671910051264e-07
947 1.58740419919923e-07
948 1.60180462671633e-07
949 1.60677927851793e-07
950 1.55798986156697e-07
951 1.64541972935695e-07
952 1.56074634105607e-07
953 1.6355170373572e-07
954 1.56567352860293e-07
955 1.59553053435957e-07
956 1.58334046318487e-07
957 1.58131143734863e-07
958 1.58782185621931e-07
959 1.5865448688146e-07
960 1.58809442041274e-07
961 1.5800458186277e-07
962 1.57620178242723e-07
963 1.58069809685912e-07
964 1.57919345156188e-07
965 1.57585759552603e-07
966 1.58340924372169e-07
967 1.57769207476122e-07
968 1.57656728561051e-07
969 1.57503009745597e-07
970 1.57967008362903e-07
971 1.5752704030092e-07
972 1.60049253850048e-07
973 1.52259516994491e-07
974 1.58971388941609e-07
975 1.53289548165958e-07
976 1.62819347337972e-07
977 1.53554950088619e-07
978 1.62918695423286e-07
979 1.53076740616598e-07
980 1.61120780717283e-07
981 1.51790658264872e-07
982 1.62816661486431e-07
983 1.51896600186774e-07
984 1.62672691317312e-07
985 1.51950402482726e-07
986 1.58121295612546e-07
987 1.52618014226391e-07
988 1.58764962066016e-07
989 1.5253672813742e-07
990 1.58623862489549e-07
991 1.52478648374199e-07
992 1.60393966552874e-07
993 1.5185837298759e-07
994 1.61323484348941e-07
995 1.50541396237713e-07
996 1.56063720169186e-07
997 1.50942582877178e-07
998 1.56501457126978e-07
999 1.52878143921953e-07
1000 1.53568009864102e-07
1001 1.53721856577249e-07
1002 1.53343492570457e-07
1003 1.54076161607009e-07
1004 1.53783147993636e-07
1005 1.5386628149372e-07
1006 1.53414006831554e-07
1007 1.5315872303745e-07
1008 1.56148146857049e-07
1009 1.48807487221347e-07
1010 1.55327143147588e-07
1011 1.49870302834643e-07
1012 1.5650685725177e-07
1013 1.55884336550116e-07
1014 1.4926020242001e-07
1015 1.57059773187029e-07
1016 1.48662223864449e-07
1017 1.56900654246783e-07
1018 1.48159926993685e-07
1019 1.57824700863785e-07
1020 1.48533572996712e-07
1021 1.56538845885734e-07
1022 1.48990451975806e-07
1023 1.54673330143851e-07
1024 1.47944618333895e-07
1025 1.52551066889828e-07
1026 1.48236594554874e-07
1027 1.56323423539106e-07
1028 1.48179765346867e-07
1029 1.52951230347753e-07
1030 1.50046133740034e-07
1031 1.53419989601389e-07
1032 1.48316260606407e-07
1033 1.55260678980085e-07
1034 1.47296162822386e-07
1035 1.55554587877305e-07
1036 1.4687903160393e-07
1037 1.55438314664025e-07
1038 1.46798868172482e-07
1039 1.55342064545039e-07
1040 1.45591684486135e-07
1041 1.51433539485879e-07
1042 1.45293284958825e-07
1043 1.50506693330499e-07
1044 1.45889742952932e-07
1045 1.54552978415268e-07
1046 1.45091902936656e-07
1047 1.50960147493606e-07
1048 1.45246005445188e-07
1049 1.54588263967526e-07
1050 1.44786383771134e-07
1051 1.50440769175475e-07
1052 1.45161678233308e-07
1053 1.50380003560713e-07
1054 1.47714004583577e-07
1055 1.47452595911091e-07
1056 1.51921597080218e-07
1057 1.48840101132919e-07
1058 1.49281575545501e-07
1059 1.48354885709523e-07
1060 1.51291231986761e-07
1061 1.48326307680691e-07
1062 1.51599067521602e-07
1063 1.47180273302183e-07
1064 1.49185751752157e-07
1065 1.42416055837202e-07
1066 1.49695978279851e-07
1067 1.43887959325184e-07
1068 1.4923995195204e-07
1069 1.44249597155977e-07
1070 1.52514715523466e-07
1071 1.45308860055593e-07
1072 1.52155692489941e-07
1073 1.45736620993375e-07
1074 1.44945957458731e-07
1075 1.521400463389e-07
1076 1.45981317700716e-07
1077 1.50514310348626e-07
1078 1.46270465961607e-07
1079 1.51853598140406e-07
1080 1.45675414842117e-07
1081 1.50083778294174e-07
1082 1.45751158697749e-07
1083 1.48036718883304e-07
1084 1.41868838454684e-07
1085 1.47713421938533e-07
1086 1.4286447935774e-07
1087 1.4690073157908e-07
1088 1.41652549245919e-07
1089 1.47264643146627e-07
1090 1.41973501399661e-07
1091 1.46506778264666e-07
1092 1.41787040774943e-07
1093 1.46310298987373e-07
1094 1.41767856121078e-07
1095 1.45939580420418e-07
1096 1.42298873129221e-07
1097 1.45137462936873e-07
1098 1.41406829357038e-07
1099 1.45934777151524e-07
1100 1.41798096819912e-07
1101 1.46221879049335e-07
1102 1.41456169444609e-07
1103 1.45694272646324e-07
1104 1.41433915246125e-07
1105 1.460552567778e-07
1106 1.41156931476871e-07
1107 1.45345453006485e-07
1108 1.41197503467083e-07
1109 1.45193666867272e-07
1110 1.40859484076827e-07
1111 1.44977988725259e-07
1112 1.403815588219e-07
1113 1.44710739391485e-07
1114 1.405467173754e-07
1115 1.4490809974177e-07
1116 1.38975238428429e-07
1117 1.45001280316137e-07
1118 1.40000011583652e-07
1119 1.43551147857579e-07
1120 1.39104756158304e-07
1121 1.47275926565271e-07
1122 1.4396051994936e-07
1123 1.44062681783907e-07
1124 1.4051157393169e-07
1125 1.43315233458452e-07
1126 1.39462130732682e-07
1127 1.44134958190989e-07
1128 1.39281311817285e-07
1129 1.4272704618179e-07
1130 1.39084818329138e-07
1131 1.43001429364631e-07
1132 1.3800969611566e-07
1133 1.43518676054555e-07
1134 1.37379586817588e-07
1135 1.42638185707256e-07
1136 1.38402015181782e-07
1137 1.45556001029945e-07
1138 1.40777771662215e-07
1139 1.37025622848341e-07
1140 1.46208861906416e-07
1141 1.41470181347358e-07
1142 1.41519450380656e-07
1143 1.37343690198577e-07
1144 1.41507229045601e-07
1145 1.37159815949417e-07
1146 1.42641582101533e-07
1147 1.37142919243161e-07
1148 1.42989080131883e-07
1149 1.4053553343274e-07
1150 1.41470479775307e-07
1151 1.36667097194731e-07
1152 1.42049429996405e-07
1153 1.36426422159275e-07
1154 1.41963155897429e-07
1155 1.36521293825353e-07
1156 1.4420301397422e-07
1157 1.39405756272026e-07
1158 1.37508351372162e-07
1159 1.39112742658654e-07
1160 1.40392728553707e-07
1161 1.35367997700087e-07
1162 1.41943999665273e-07
1163 1.35559275804553e-07
1164 1.41331412351065e-07
1165 1.38102535629514e-07
1166 1.34379007477037e-07
1167 1.43181154044214e-07
1168 1.40140031135161e-07
1169 1.35035307380349e-07
1170 1.40897384426353e-07
1171 1.37545598022371e-07
1172 1.33655348122375e-07
1173 1.42371305855704e-07
1174 1.40302546469684e-07
1175 1.38832163543157e-07
1176 1.34305707888416e-07
1177 1.40880786148045e-07
1178 1.34119090944296e-07
1179 1.39905537821505e-07
1180 1.3859398961813e-07
1181 1.3872596582587e-07
1182 1.417263320036e-07
1183 1.38972509944324e-07
1184 1.3905302864714e-07
1185 1.34115552441472e-07
1186 1.39161684842293e-07
1187 1.34450075961468e-07
1188 1.41034277589824e-07
1189 1.33865555085322e-07
1190 1.40792664637956e-07
1191 1.37237975650351e-07
1192 1.41007845400054e-07
1193 1.36427971142439e-07
1194 1.40067371035002e-07
1195 1.37396071409057e-07
1196 1.40401894554998e-07
1197 1.37400718358549e-07
1198 1.38998302645632e-07
1199 1.35180982852035e-07
1200 1.43275997288583e-07
1201 1.33674475932821e-07
1202 1.39490779815787e-07
1203 1.41691714361514e-07
1204 1.34497454951088e-07
1205 1.37095113927899e-07
1206 1.36679190632094e-07
1207 1.39848367552986e-07
1208 1.39526164844028e-07
1209 1.39266262522142e-07
1210 1.36835339503705e-07
1211 1.39784205543947e-07
1212 1.36764526814659e-07
1213 1.39145740263302e-07
1214 1.36204832301701e-07
1215 1.35667519884919e-07
1216 1.31447166040743e-07
1217 1.37788319420906e-07
1218 1.34591800815542e-07
1219 1.37827427693082e-07
1220 1.34472031732003e-07
1221 1.37011554102173e-07
1222 1.37356821028334e-07
1223 1.36144279849759e-07
1224 1.32987281631358e-07
1225 1.36889681812136e-07
1226 1.37111115350308e-07
1227 1.36084310042861e-07
1228 1.32284000642358e-07
1229 1.34046587163539e-07
1230 1.30820325239256e-07
1231 1.37909921704704e-07
1232 1.35340329165956e-07
1233 1.3211523253176e-07
1234 1.33401172774938e-07
1235 1.30615291027425e-07
1236 1.35165279857574e-07
1237 1.30165986433894e-07
1238 1.36675808448672e-07
1239 1.3239491636341e-07
1240 1.33841538740853e-07
1241 1.32763702254124e-07
1242 1.30884401983167e-07
1243 1.33149939074428e-07
1244 1.28497816831441e-07
1245 1.36612669621172e-07
1246 1.30515914520402e-07
1247 1.34327905243481e-07
1248 1.29573052731757e-07
1249 1.32955264575685e-07
1250 1.29341984234088e-07
1251 1.30198429815209e-07
1252 1.35129539557965e-07
1253 1.27104087255248e-07
1254 1.35696893721615e-07
1255 1.30266258224765e-07
1256 1.27535528804401e-07
1257 1.34747523361511e-07
1258 1.27150258322217e-07
1259 1.34540030671815e-07
1260 1.26901881003505e-07
1261 1.34600512069483e-07
1262 1.29084511968358e-07
1263 1.32000053554293e-07
1264 1.28032894508578e-07
1265 1.32171905420364e-07
1266 1.30954248334092e-07
1267 1.29182510022474e-07
1268 1.30771553585873e-07
1269 1.31254068946873e-07
1270 1.27517068904126e-07
1271 1.30567187284214e-07
1272 1.27467529864589e-07
1273 1.33308944327837e-07
1274 1.28208583305423e-07
1275 1.29454051034372e-07
1276 1.28477864791421e-07
1277 1.30393345898483e-07
1278 1.27226755353149e-07
1279 1.30317786783962e-07
1280 1.2798118120827e-07
1281 1.29905700418931e-07
1282 1.27917545000855e-07
1283 1.30301671674715e-07
1284 1.27548631212449e-07
1285 1.30505981132956e-07
1286 1.28317751091345e-07
1287 1.29827583350561e-07
1288 1.2661296011629e-07
1289 1.2942848570674e-07
1290 1.26480827589148e-07
1291 1.29595647990755e-07
1292 1.28550624367563e-07
1293 1.29420115513312e-07
1294 1.26479577033933e-07
1295 1.28748709471438e-07
1296 1.28227213735954e-07
1297 1.26819116985644e-07
1298 1.28391747011847e-07
1299 1.26462111893488e-07
1300 1.28338115246152e-07
1301 1.26259095623027e-07
1302 1.28391747011847e-07
1303 1.27554315554335e-07
1304 1.2525498505056e-07
1305 1.28240216668019e-07
1306 1.2591537768003e-07
1307 1.30639591588988e-07
1308 1.2604313326392e-07
1309 1.27428634755233e-07
1310 1.27326913457182e-07
1311 1.28532903431733e-07
1312 1.25453894384009e-07
1313 1.27619088630126e-07
1314 1.27117715464919e-07
1315 1.2708697738617e-07
1316 1.2709924135379e-07
1317 1.26504161812591e-07
1318 1.2470812293941e-07
1319 1.2741259070026e-07
1320 1.27593921206426e-07
1321 1.25798280237177e-07
1322 1.26511892517556e-07
1323 1.25549789231627e-07
1324 1.26165829783531e-07
1325 1.25663817129862e-07
1326 1.23473341773206e-07
1327 1.25021969665795e-07
1328 1.2281398653613e-07
1329 1.27467430388606e-07
1330 1.22508424738044e-07
1331 1.2462551524095e-07
1332 1.24411414503811e-07
1333 1.22340921393516e-07
1334 1.27306435615537e-07
1335 1.2150914585618e-07
1336 1.26949842638169e-07
1337 1.24019379654783e-07
1338 1.24014476909906e-07
1339 1.24418292557493e-07
1340 1.23386868722264e-07
1341 1.24654405908586e-07
1342 1.23040919675077e-07
1343 1.23547025054904e-07
1344 1.22607190178314e-07
1345 1.17490138507037e-07
1346 1.2436454710496e-07
1347 1.18928824122122e-07
1348 1.17182850090103e-07
1349 1.20657361435406e-07
1350 1.18292149409172e-07
1351 1.16481558620762e-07
1352 1.19932479947238e-07
1353 1.17882272832048e-07
1354 1.1595723492519e-07
1355 1.16629813362579e-07
1356 1.19638912110531e-07
1357 1.17210660732781e-07
1358 1.21150662835134e-07
1359 1.15678794543328e-07
1360 1.19128600317708e-07
1361 1.20420153848499e-07
1362 1.17896163942532e-07
1363 1.17345862804541e-07
1364 1.18402844861976e-07
1365 1.14708804233032e-07
1366 1.18394687831369e-07
1367 1.13529338818807e-07
1368 1.16973133401643e-07
1369 1.15529807942494e-07
1370 1.1256005194582e-07
1371 1.17759597628719e-07
1372 1.16927033388947e-07
1373 1.16999459010003e-07
1374 1.15331808103747e-07
1375 1.13293751269339e-07
1376 1.17043839509279e-07
1377 1.1337164806946e-07
1378 1.14760638325606e-07
1379 1.16327733223898e-07
1380 1.17244162822772e-07
1381 1.13300892223833e-07
1382 1.14544683071927e-07
1383 1.1437448677043e-07
1384 1.13387990552383e-07
1385 1.14017900898489e-07
1386 1.14186136102035e-07
1387 1.12147084507797e-07
1388 1.16499748514798e-07
1389 1.15969307046271e-07
1390 1.13391216416403e-07
1391 1.13374774457498e-07
1392 1.11586402340436e-07
1393 1.16104764913416e-07
1394 1.15201295614042e-07
1395 1.14539453477391e-07
1396 1.16851126108486e-07
1397 1.12307354527275e-07
1398 1.16730880961313e-07
1399 1.12785834005535e-07
1400 1.11588931872575e-07
1401 1.15669621436609e-07
1402 1.1604844729618e-07
1403 1.12549656705596e-07
1404 1.11512129308267e-07
1405 1.14045072052704e-07
1406 1.1121979781592e-07
1407 1.1497303376018e-07
1408 1.15076311146822e-07
1409 1.15906615860695e-07
1410 1.15660448329891e-07
1411 1.16281775319749e-07
1412 1.12198037527378e-07
1413 1.10600694824825e-07
1414 1.14347962210104e-07
1415 1.13353856079357e-07
1416 1.12966773713197e-07
1417 1.11109230260809e-07
1418 1.12092351400861e-07
1419 1.1193034055168e-07
1420 1.13900192388883e-07
1421 1.15195142313951e-07
1422 1.13410990820739e-07
1423 1.12255506223846e-07
1424 1.127607873741e-07
1425 1.1219383821981e-07
1426 1.12176849142998e-07
1427 1.13646230204267e-07
1428 1.10364382521766e-07
1429 1.14673810003296e-07
1430 1.12400400098522e-07
1431 1.10788135998519e-07
1432 1.09118410307474e-07
1433 1.12455360579133e-07
1434 1.10410105946812e-07
1435 1.12183336398175e-07
1436 1.09592775743295e-07
1437 1.08367167683809e-07
1438 1.11232253630078e-07
1439 1.10157301946856e-07
1440 1.12127587215127e-07
1441 1.10957387278177e-07
1442 1.09687078975185e-07
1443 1.11939847613485e-07
1444 1.0955334062146e-07
1445 1.08648244179221e-07
1446 1.10539403408438e-07
1447 1.09009306470398e-07
1448 1.10510583795076e-07
1449 1.0963067609282e-07
1450 1.10892948157471e-07
1451 1.07533054460873e-07
1452 1.10410006470829e-07
1453 1.10295125921311e-07
1454 1.07317177366895e-07
1455 1.11972511263048e-07
1456 1.11399955926572e-07
1457 1.08636960760577e-07
1458 1.0888592782976e-07
1459 1.09461119279786e-07
1460 1.08429347278616e-07
1461 1.09370041911916e-07
1462 1.10302785572003e-07
1463 1.09627059430295e-07
1464 1.08742959525898e-07
1465 1.06548171174836e-07
1466 1.11125238788645e-07
1467 1.08400705300937e-07
1468 1.07664234860749e-07
1469 1.11181769568702e-07
1470 1.08122087283391e-07
1471 1.0759595880927e-07
1472 1.08878019489111e-07
1473 1.10970653111053e-07
1474 1.07564950724282e-07
1475 1.10130969233069e-07
1476 1.06628021967481e-07
1477 1.07245199387762e-07
1478 1.07551592520849e-07
1479 1.09016362159764e-07
1480 1.06066870841914e-07
1481 1.07006584926239e-07
1482 1.06925291731841e-07
1483 1.09864544128868e-07
1484 1.0631617897161e-07
1485 1.09463535125087e-07
1486 1.07294120255119e-07
1487 1.09335140052735e-07
1488 1.07362254198051e-07
1489 1.06855502224334e-07
1490 1.0835240260576e-07
1491 1.08608325888326e-07
1492 1.07858745934664e-07
1493 1.06323000181874e-07
1494 1.06821360645881e-07
1495 1.05504255998312e-07
1496 1.05670537209335e-07
1497 1.03144664365118e-07
1498 1.08547745014675e-07
1499 1.08202435455951e-07
1500 1.07119376480114e-07
1501 1.08002872423185e-07
1502 1.05203049827196e-07
1503 1.07196228782414e-07
1504 1.06861634208144e-07
1505 1.05762282487376e-07
1506 1.06448283077043e-07
1507 1.04458592886658e-07
1508 1.05174997599988e-07
1509 1.05933395389002e-07
1510 1.04000740464016e-07
1511 1.05201394262622e-07
1512 1.06011590617072e-07
1513 1.06499044250086e-07
1514 1.04108558218741e-07
1515 1.05388799909178e-07
1516 1.05061531030515e-07
1517 1.04986014548558e-07
1518 1.05628593871643e-07
1519 1.03690297237335e-07
1520 1.05346217083024e-07
1521 1.04792526656183e-07
1522 1.0379201142996e-07
1523 1.03872267231964e-07
1524 1.02031982862627e-07
1525 1.03617097124697e-07
1526 1.02682697900036e-07
1527 1.01185982259722e-07
1528 1.02762015785629e-07
1529 1.03484786961872e-07
1530 1.03544927299026e-07
1531 1.03929046701978e-07
1532 1.03286694752569e-07
1533 9.95057831687518e-08
1534 1.02326261242069e-07
1535 1.0233884495392e-07
1536 1.02597532247728e-07
1537 1.02027158277451e-07
1538 1.02155468084675e-07
1539 1.01467513502485e-07
1540 1.01217601411463e-07
1541 1.01594217483125e-07
1542 1.00739640629399e-07
1543 9.86310197959028e-08
1544 9.92858986137435e-08
1545 9.89867032785696e-08
1546 9.94127731246408e-08
1547 9.86732899832532e-08
1548 9.94997009229337e-08
1549 9.81705454705661e-08
1550 9.76206564473614e-08
1551 9.71032108054715e-08
1552 9.79709398052364e-08
1553 9.7722541170242e-08
1554 9.79417364987967e-08
1555 9.7202729421042e-08
1556 9.72312861335922e-08
1557 9.7935981102637e-08
1558 9.72040936630947e-08
1559 9.6965166562768e-08
1560 9.75499929722901e-08
1561 9.59890513740902e-08
1562 9.76307958922007e-08
1563 9.65172546329995e-08
1564 9.79469803041866e-08
1565 9.84401467007956e-08
1566 9.840200476674e-08
1567 9.71255573745111e-08
1568 9.69934887962154e-08
1569 9.67283213526571e-08
1570 9.59213934947911e-08
1571 9.51059888620875e-08
1572 9.5405177091834e-08
1573 9.5839467917358e-08
1574 9.63839141832068e-08
1575 9.60094581614612e-08
1576 9.52264187503715e-08
1577 9.569503589546e-08
1578 9.55861949591963e-08
1579 9.50341743077843e-08
1580 9.45839673249793e-08
1581 9.54300176658762e-08
1582 9.39085609275025e-08
1583 9.46907547927367e-08
1584 9.50660705711925e-08
1585 9.39365136787274e-08
1586 9.39084188189554e-08
1587 9.39109412456673e-08
1588 9.36650437211028e-08
1589 9.30653669684034e-08
1590 9.38239210768188e-08
1591 9.32704793399353e-08
1592 9.33042016981744e-08
1593 9.32971886413725e-08
1594 9.26332432982235e-08
1595 9.30154016032247e-08
1596 9.27631162994658e-08
1597 9.31550800942205e-08
1598 9.40300637353175e-08
1599 9.30075003680031e-08
1600 9.20134866078115e-08
1601 9.20425549111314e-08
1602 9.28169043845628e-08
1603 9.24748064790037e-08
1604 9.21413061405474e-08
1605 9.19106284413829e-08
1606 9.21873422043973e-08
1607 9.18365046231884e-08
1608 9.15099676035425e-08
1609 9.15925184585831e-08
1610 9.21064184922216e-08
1611 9.15600466555588e-08
1612 9.16235123327169e-08
1613 9.06040398263031e-08
1614 9.03759769244061e-08
1615 9.07161705754334e-08
1616 9.1428056236964e-08
1617 9.09329926912505e-08
1618 9.08261696963564e-08
1619 8.96745504519458e-08
1620 9.01576910905533e-08
1621 8.99637129236908e-08
1622 8.97508556363391e-08
1623 8.91621212417704e-08
1624 8.93057006123854e-08
1625 8.93913281174719e-08
1626 8.91706548600268e-08
1627 8.8620446092591e-08
1628 8.90119693508495e-08
1629 8.89556943661773e-08
1630 8.89039100115951e-08
1631 8.78394743608624e-08
1632 8.794088302011e-08
1633 8.74793073535329e-08
1634 8.82681234770644e-08
1635 8.80919373003053e-08
1636 8.78946764260036e-08
1637 8.79773978113008e-08
1638 8.95662921607254e-08
1639 8.78923103186935e-08
1640 8.769688264465e-08
1641 8.79073311921275e-08
1642 8.73859136163446e-08
1643 8.72436842769275e-08
1644 8.71770922117321e-08
1645 8.70467573577116e-08
1646 8.69903828970564e-08
1647 8.65492921775513e-08
1648 8.57668212006502e-08
1649 8.60009876646473e-08
1650 8.58992450503138e-08
1651 8.64739533312786e-08
1652 8.56726387610252e-08
1653 8.54067323530217e-08
1654 8.61024531673138e-08
1655 8.59922693052795e-08
1656 8.55517114928261e-08
1657 8.67437037754826e-08
1658 8.47497503286831e-08
1659 8.56214725786231e-08
1660 8.41187883793282e-08
1661 8.5089439494368e-08
1662 8.49081160936294e-08
1663 8.45420800033025e-08
1664 8.46301020374085e-08
1665 8.41366727399873e-08
1666 8.37823819210826e-08
1667 8.35420692624211e-08
1668 8.351480573765e-08
1669 8.32876239087454e-08
1670 8.33005202593995e-08
1671 8.31066699902294e-08
1672 8.3058040445394e-08
1673 8.28909634265074e-08
1674 8.28167969757487e-08
1675 8.42108676124553e-08
1676 8.2674475265776e-08
1677 8.24586834369256e-08
1678 8.23003318828341e-08
1679 8.25370065626885e-08
1680 8.24344823513457e-08
1681 8.21896719571669e-08
1682 8.19175696165075e-08
1683 8.22193442218122e-08
1684 8.22480430429096e-08
1685 8.21919527993487e-08
1686 8.21907306658431e-08
1687 8.18235648125665e-08
1688 8.11288671798138e-08
1689 8.16559690974827e-08
1690 8.17525389606999e-08
1691 8.19428365161912e-08
1692 8.13468687965724e-08
1693 8.11335283401604e-08
1694 8.08756723813531e-08
1695 8.08130238283411e-08
1696 8.01154982355001e-08
1697 8.01074193645945e-08
1698 8.00388093580295e-08
1699 7.961789805222e-08
1700 7.9377812767234e-08
1701 7.90283820606419e-08
1702 7.84514000429226e-08
1703 7.90716256915402e-08
1704 7.89891743124826e-08
1705 7.8750545640105e-08
1706 7.88638274684672e-08
1707 7.89200171880111e-08
1708 7.85599993946562e-08
1709 7.80685027734762e-08
1710 7.82495988005394e-08
1711 7.80291813384792e-08
1712 7.76094708498931e-08
1713 7.75736097580193e-08
1714 7.67621486374992e-08
1715 7.7405644560713e-08
1716 7.73414825516738e-08
1717 7.58526894628631e-08
1718 7.69701031799741e-08
1719 7.70419035234227e-08
1720 7.56996385575803e-08
1721 7.64824719112767e-08
1722 7.70254402482351e-08
1723 7.7524248354166e-08
1724 7.55950964048679e-08
1725 7.63507514989215e-08
1726 7.56195888129696e-08
1727 7.60884972805798e-08
1728 7.55171427613277e-08
1729 7.5511863428801e-08
1730 7.57034896992081e-08
1731 7.54243458800374e-08
1732 7.56362936726873e-08
1733 7.57175229182394e-08
1734 7.53405942077734e-08
1735 7.49269659650054e-08
1736 7.4855527998352e-08
1737 7.49690798329539e-08
1738 7.63298473316354e-08
1739 7.50060280552134e-08
1740 7.40477617000579e-08
1741 7.47711439430532e-08
1742 7.44288826126649e-08
1743 7.38085148555001e-08
1744 7.38704244440669e-08
1745 7.22411641618237e-08
1746 7.34644700628451e-08
1747 7.28217699474953e-08
1748 7.33099980720908e-08
1749 7.31741494064408e-08
1750 7.29277545019613e-08
1751 7.31485627625261e-08
1752 7.2436748155269e-08
1753 7.32290814653425e-08
1754 7.26675608575533e-08
1755 7.29144673528026e-08
1756 7.2522325922364e-08
1757 7.14593113571027e-08
1758 7.22224555715911e-08
1759 7.3266228639568e-08
1760 7.21363520028717e-08
1761 7.17003345584999e-08
1762 7.18919821451891e-08
1763 7.178116590012e-08
1764 7.15522716632222e-08
1765 7.09484808680827e-08
1766 7.12885821485543e-08
1767 7.07181158077219e-08
1768 7.1241743171413e-08
1769 7.12196523977582e-08
1770 7.12903940325305e-08
1771 7.13771868277036e-08
1772 7.04720406474735e-08
1773 7.07406186961634e-08
1774 7.10513248236566e-08
1775 7.00816187304554e-08
1776 6.98885784800041e-08
1777 7.01907367783861e-08
1778 6.92464041662788e-08
1779 6.99218745126018e-08
1780 6.97193627274828e-08
1781 6.94875907925052e-08
1782 6.99089994782298e-08
1783 7.09163217038622e-08
1784 6.9994442242205e-08
1785 6.97607447364135e-08
1786 6.96508308806187e-08
1787 6.92008015334977e-08
1788 6.96501913921566e-08
1789 6.8829898225431e-08
1790 6.94664166189796e-08
1791 7.0080716341181e-08
1792 6.96511719411319e-08
1793 6.87567549562118e-08
1794 6.82951011299338e-08
1795 6.7766670497349e-08
1796 6.75785329917744e-08
1797 6.79976821515993e-08
1798 6.79783553891866e-08
1799 6.89217429794553e-08
1800 6.81135006175282e-08
1801 6.80497649341305e-08
1802 6.73308520049432e-08
1803 6.752922843134e-08
1804 6.73699744879741e-08
1805 6.76338203220439e-08
1806 6.77142821814414e-08
1807 6.69672104436358e-08
1808 6.76460985005178e-08
1809 6.71754989411966e-08
1810 6.74642421927274e-08
1811 6.63384724930438e-08
1812 6.64164474528661e-08
1813 6.64404353756254e-08
1814 6.75106122116631e-08
1815 6.63539267975466e-08
1816 6.67559163503029e-08
1817 6.59478729403418e-08
1818 6.63728840777367e-08
1819 6.72032669513101e-08
1820 6.52037002168981e-08
1821 6.5833845042107e-08
1822 6.52634284392661e-08
1823 6.55032224017305e-08
1824 6.42823749785748e-08
1825 6.66173889385391e-08
1826 6.54956764378767e-08
1827 6.60192682744309e-08
1828 6.624834725244e-08
1829 6.50400693302799e-08
1830 6.44531468196874e-08
1831 6.55007923455742e-08
1832 6.50219220688086e-08
1833 6.38214459058872e-08
1834 6.47231317429942e-08
1835 6.46217301891738e-08
1836 6.52863292316397e-08
1837 6.43326458771298e-08
1838 6.42921165194821e-08
1839 6.3854841414468e-08
1840 6.45989572944927e-08
1841 6.41692778913239e-08
1842 6.39906332366991e-08
1843 6.51413358809805e-08
1844 6.39511057443087e-08
1845 6.36396393360883e-08
1846 6.4517358566718e-08
1847 6.32463965644092e-08
1848 6.29824512543564e-08
1849 6.3609391531827e-08
1850 6.31537915296576e-08
1851 6.38951718201497e-08
1852 6.31358503255797e-08
1853 6.30736352036365e-08
1854 6.33776551239862e-08
1855 6.37156674088146e-08
1856 6.40811563812349e-08
1857 6.34717238767735e-08
1858 6.25943386012295e-08
1859 6.40047019828671e-08
1860 6.31067038625588e-08
1861 6.20735391976268e-08
1862 6.27427070298836e-08
1863 6.27952374543383e-08
1864 6.3491683022221e-08
1865 6.29580654276651e-08
1866 6.2809675682729e-08
1867 6.21553155610854e-08
1868 6.27152445531465e-08
1869 6.26738980713526e-08
1870 6.30694287906408e-08
1871 6.24963050199767e-08
1872 6.31034637876837e-08
1873 6.13204562682768e-08
1874 6.27653164997355e-08
1875 6.21887963347945e-08
1876 6.28354612786097e-08
1877 6.1728187006338e-08
1878 6.24909972657406e-08
1879 6.23187617065923e-08
1880 6.05924057595075e-08
1881 6.31833287911832e-08
1882 6.17449629203293e-08
1883 6.19440996274534e-08
1884 6.1513375726463e-08
1885 6.23207441208251e-08
1886 6.07598451551894e-08
1887 6.19914644062192e-08
1888 6.03816943112179e-08
1889 6.16785769125272e-08
1890 5.96768572336259e-08
1891 6.21802556111106e-08
1892 6.03542815724722e-08
1893 6.14148945032866e-08
1894 6.05334804504309e-08
1895 6.09917734095689e-08
1896 6.12989481396653e-08
1897 5.9657466522367e-08
1898 5.97598202034533e-08
1899 5.98981202415416e-08
1900 6.12676203104456e-08
1901 6.14682207356054e-08
1902 6.13940045468553e-08
1903 5.9869535107282e-08
1904 6.05257142183291e-08
1905 6.15742337117808e-08
1906 6.03713914415493e-08
1907 6.0960303471802e-08
1908 5.91877302724697e-08
1909 5.9940489904875e-08
1910 6.04115086844104e-08
1911 6.02620673362253e-08
1912 6.10199535344691e-08
1913 6.09790546945987e-08
1914 6.08353261100092e-08
1915 6.0470455309769e-08
1916 5.96956795106962e-08
1917 5.93721161123995e-08
1918 6.07571664090756e-08
1919 5.91324642584823e-08
1920 6.00484710844285e-08
1921 6.00455436483571e-08
1922 6.05274976805958e-08
1923 5.92902651419536e-08
1924 5.88979247595489e-08
1925 5.84228949662702e-08
1926 5.91221791523822e-08
1927 5.8998864460591e-08
1928 5.93714055696637e-08
1929 6.02668919214011e-08
1930 5.96534803776194e-08
1931 5.84062540553987e-08
1932 6.00050000798547e-08
1933 5.99714766735815e-08
1934 5.96731979385368e-08
1935 5.96440870026527e-08
1936 5.94613283055878e-08
1937 5.81012038480822e-08
1938 5.99666307721236e-08
1939 5.87427528841999e-08
1940 5.8601550279036e-08
1941 5.88280428814869e-08
1942 5.81253907228074e-08
1943 5.93669717829926e-08
1944 5.89399249406597e-08
1945 5.93635363088652e-08
1946 5.77917873556544e-08
1947 5.90326649785311e-08
1948 5.92694391343684e-08
1949 5.76404772800743e-08
1950 5.85202215575009e-08
1951 5.87450479372364e-08
1952 5.71609426458508e-08
1953 5.87180828404144e-08
1954 5.72780614049861e-08
1955 5.80270231864688e-08
1956 5.88968411818769e-08
1957 5.76844101374263e-08
1958 5.76415430941779e-08
1959 5.84531569813862e-08
1960 5.82645007796145e-08
1961 5.69690428164904e-08
1962 5.83720520808129e-08
1963 5.78523788874463e-08
1964 5.80600492128269e-08
1965 5.61731781090202e-08
1966 5.82662664783129e-08
1967 5.65226088156123e-08
1968 5.72253675557022e-08
1969 5.7633556593828e-08
1970 5.63932118780031e-08
1971 5.78113130700331e-08
1972 5.78307499665698e-08
1973 5.77971057680315e-08
1974 5.66369209309414e-08
1975 5.72942155940837e-08
1976 5.67512792315483e-08
1977 5.74548764120664e-08
1978 5.62438096096685e-08
1979 5.67168001452956e-08
1980 5.71714835473358e-08
1981 5.63105118089879e-08
1982 5.70442075797928e-08
1983 5.77824188496834e-08
1984 5.63959119403989e-08
1985 5.7586149182498e-08
1986 5.60454935794041e-08
1987 5.68373970111224e-08
1988 5.59487602913578e-08
1989 5.5689028499728e-08
1990 5.51248682256755e-08
1991 5.54363595028917e-08
1992 5.66110713862145e-08
1993 5.52882468696225e-08
1994 5.70613032380152e-08
1995 5.53685381987634e-08
1996 5.64882327580563e-08
1997 5.51160574957521e-08
1998 5.51442376206523e-08
1999 5.67559119701855e-08
2000 5.53710179929112e-08
2001 5.49465184462861e-08
2002 5.51553434036123e-08
2003 5.66434650295378e-08
2004 5.49113998715711e-08
2005 5.42813118897811e-08
2006 5.52124994612768e-08
2007 5.63192514846378e-08
2008 5.47078933266221e-08
2009 5.41790399211095e-08
2010 5.48525811439049e-08
2011 5.43263567465146e-08
2012 5.53273942216492e-08
2013 5.44236371524676e-08
2014 5.47707976750189e-08
2015 5.49924585868666e-08
2016 5.44850209394099e-08
2017 5.45162102127961e-08
2018 5.47237384296295e-08
2019 5.49451293352377e-08
2020 5.36477458012996e-08
2021 5.48345227002756e-08
2022 5.41313589508263e-08
2023 5.44325295948056e-08
2024 5.37678559453525e-08
2025 5.31523340896456e-08
2026 5.46358656094981e-08
2027 5.38580380293752e-08
2028 5.40746825095084e-08
2029 5.38351407897153e-08
2030 5.29839212504157e-08
2031 5.43285771925639e-08
2032 5.39065112548087e-08
2033 5.25985335286805e-08
2034 5.3030749569416e-08
2035 5.31675361514772e-08
2036 5.21137906162039e-08
2037 5.36983399967994e-08
2038 5.31949098103723e-08
2039 5.34397663898289e-08
2040 5.25222283442872e-08
2041 5.21799563557579e-08
2042 5.22357872512202e-08
2043 5.24689198755368e-08
2044 5.2275940021218e-08
2045 5.23795407048055e-08
2046 5.18889251566179e-08
2047 5.17550766687691e-08
2048 5.25641858928338e-08
2049 5.27194075061743e-08
2050 5.1480604668086e-08
2051 5.30326680348026e-08
2052 5.16968121644368e-08
2053 5.16733820177251e-08
2054 5.27371781799957e-08
2055 5.16619671486751e-08
2056 5.20840011120072e-08
2057 5.12721172185593e-08
2058 5.21374055040269e-08
2059 5.19944620691604e-08
2060 5.14712574783971e-08
2061 5.1521407584687e-08
2062 5.15264453326836e-08
2063 5.20977856410809e-08
2064 5.19643847951556e-08
2065 5.08608302141056e-08
2066 5.17756042484052e-08
2067 5.10969151434892e-08
2068 5.1472586193313e-08
2069 5.11197271180208e-08
2070 5.19182137281859e-08
2071 5.09223809785908e-08
2072 5.17336395944312e-08
2073 5.08211250860313e-08
2074 5.04581407767546e-08
2075 5.02974160099257e-08
2076 5.17135738675734e-08
2077 5.08447648428501e-08
2078 5.1105136122942e-08
2079 5.04303088177949e-08
2080 5.05215993484853e-08
2081 4.94804446304897e-08
2082 5.08637292284675e-08
2083 4.98489889366738e-08
2084 5.07588389098146e-08
2085 4.96421144191572e-08
2086 4.95688574631004e-08
2087 4.91739378105649e-08
2088 4.99028729450401e-08
2089 4.91566503058039e-08
2090 4.92696159426487e-08
2091 4.88186806535396e-08
2092 4.96057097620906e-08
2093 4.99320869096209e-08
2094 4.95831393720891e-08
2095 4.93338170315383e-08
2096 4.93964549264092e-08
2097 4.9335589835664e-08
2098 5.00456813767869e-08
2099 4.92791478734489e-08
2100 4.95384711030056e-08
2101 4.90058056357157e-08
2102 4.8537522445713e-08
2103 4.95273297929089e-08
2104 4.86237041741333e-08
2105 4.83394799744019e-08
2106 4.84705395820129e-08
2107 4.98139023363819e-08
2108 4.83962772079849e-08
2109 4.90540621456148e-08
2110 4.80462887253452e-08
2111 4.79237662887044e-08
2112 4.92232175020035e-08
2113 4.8324292123425e-08
2114 4.84082711693645e-08
2115 4.78097668121791e-08
2116 4.8224535476038e-08
2117 4.90809313191676e-08
2118 4.82425228653938e-08
2119 4.84276618806234e-08
2120 4.75843968672507e-08
2121 4.83390572014741e-08
2122 4.86766396079474e-08
2123 4.79165684907912e-08
2124 4.83510937954179e-08
2125 4.81648392280931e-08
2126 4.7855802876029e-08
2127 4.83931401618065e-08
2128 4.80406754377327e-08
2129 4.79041837309069e-08
2130 4.81292623533136e-08
2131 4.81946962338498e-08
2132 4.79334580916202e-08
2133 4.79151829324564e-08
2134 4.7794824098446e-08
2135 4.76108965585809e-08
2136 4.77363819584298e-08
2137 4.69736960440059e-08
2138 4.79126143204667e-08
2139 4.72884309488109e-08
2140 4.75882551143059e-08
2141 4.69402010594422e-08
2142 4.73378385379419e-08
2143 4.71780410293832e-08
2144 4.72150993857667e-08
2145 4.71140850777374e-08
2146 4.70211780623231e-08
2147 4.7110859213717e-08
2148 4.70559484710975e-08
2149 4.7132594716004e-08
2150 4.73275534318418e-08
2151 4.6879723214488e-08
2152 4.690706489896e-08
2153 4.64550815593157e-08
2154 4.63109053328026e-08
2155 4.61048799138553e-08
2156 4.67060985442913e-08
2157 4.62728486638753e-08
2158 4.61899922754583e-08
2159 4.60441746952256e-08
2160 4.67929801573064e-08
2161 4.63244020920683e-08
2162 4.54005117944689e-08
2163 4.60899514109769e-08
2164 4.63599505451384e-08
2165 4.60482247888194e-08
2166 4.57691804456317e-08
2167 4.55917543717987e-08
2168 4.57157725008983e-08
2169 4.5756955557863e-08
2170 4.60495037657438e-08
2171 4.59301965349823e-08
2172 4.5717257535216e-08
2173 4.58226203647882e-08
2174 4.56508608692729e-08
2175 4.5717037266968e-08
2176 4.56267663651033e-08
2177 4.56928113123922e-08
2178 4.54786786008299e-08
2179 4.55694681988916e-08
2180 4.53784210208141e-08
2181 4.54429276430801e-08
2182 4.48470665048717e-08
2183 4.53777353470741e-08
2184 4.47993571128791e-08
2185 4.52324897537437e-08
2186 4.49353407816488e-08
2187 4.54603217292515e-08
2188 4.49371988509029e-08
2189 4.46888570593273e-08
2190 4.40885798980162e-08
2191 4.4968277990165e-08
2192 4.50087576098213e-08
2193 4.3704076801987e-08
2194 4.51819808233722e-08
2195 4.48539552166949e-08
2196 4.38092655485889e-08
2197 4.4251031283693e-08
2198 4.39035616750516e-08
2199 4.36624567612398e-08
2200 4.45909265067712e-08
2201 4.3455031573103e-08
2202 4.4353409833775e-08
2203 4.450528479083e-08
2204 4.41230056935638e-08
2205 4.42548291346156e-08
2206 4.40930136846873e-08
2207 4.40413252533745e-08
2208 4.4100328722152e-08
2209 4.39889049630438e-08
2210 4.38766889487852e-08
2211 4.3871946076024e-08
2212 4.41387406624472e-08
2213 4.39613074831868e-08
2214 4.3759733614479e-08
2215 4.36409948179062e-08
2216 4.36574119078159e-08
2217 4.37880309789307e-08
2218 4.35188027836375e-08
2219 4.36162821415564e-08
2220 4.37318803392373e-08
2221 4.33793339027488e-08
2222 4.34177565011851e-08
2223 4.27662456559119e-08
2224 4.34833005158453e-08
2225 4.3249876569007e-08
2226 4.29321040940067e-08
2227 4.27425135285375e-08
2228 4.32975753028586e-08
2229 4.26898232319672e-08
2230 4.32226201496633e-08
2231 4.31598330408178e-08
2232 4.31222524355235e-08
2233 4.32718287868283e-08
2234 4.29748254759943e-08
2235 4.29745981023189e-08
2236 4.22835917390785e-08
2237 4.23491677281618e-08
2238 4.24188506542578e-08
2239 4.24738288984372e-08
2240 4.26457411606407e-08
2241 4.23750776690213e-08
2242 4.24330437454046e-08
2243 4.23378452296674e-08
2244 4.21228492086811e-08
2245 4.26487467564129e-08
2246 4.254354735167e-08
2247 4.21672510242388e-08
2248 4.21643022718854e-08
2249 4.1511732717936e-08
2250 4.21233501413099e-08
2251 4.16188825624886e-08
2252 4.1623103186339e-08
2253 4.16789873725065e-08
2254 4.18475671892793e-08
2255 4.20204635531718e-08
2256 4.19610515223212e-08
2257 4.20386854216304e-08
2258 4.18107575228532e-08
2259 4.14005079107937e-08
2260 4.11564506919149e-08
2261 4.12895921897416e-08
2262 4.13836254153921e-08
2263 4.08191311862538e-08
2264 4.14905798606924e-08
2265 4.09968450298948e-08
2266 4.14348590993541e-08
2267 4.18710364158414e-08
2268 4.1216797086463e-08
2269 4.12174792074893e-08
2270 4.06157383281425e-08
2271 4.06458973145618e-08
2272 4.03441653418213e-08
2273 4.01915976055989e-08
2274 4.0159619629776e-08
2275 4.06331857050191e-08
2276 4.10739993128573e-08
2277 4.09328038131207e-08
2278 4.07904252597291e-08
2279 4.05304483308555e-08
2280 4.07069293828499e-08
2281 4.09000868728526e-08
2282 4.05419768867432e-08
2283 4.03996764930525e-08
2284 4.05198186115285e-08
2285 4.02963067358542e-08
2286 4.06084481596736e-08
2287 4.10554292784582e-08
2288 3.99557755770275e-08
2289 4.01297022278868e-08
2290 4.03174560403841e-08
2291 4.00827744329035e-08
2292 4.02362800855371e-08
2293 4.07644549227371e-08
2294 4.01759407964164e-08
2295 4.0483456587026e-08
2296 4.00080466533836e-08
2297 3.96740311714439e-08
2298 3.99224013847288e-08
2299 4.0105444298888e-08
2300 4.02487536632634e-08
2301 3.99673965034708e-08
2302 4.03032629492373e-08
2303 4.00952089307793e-08
2304 3.99262312100745e-08
2305 3.98843376103741e-08
2306 4.00773849662528e-08
2307 3.97874408974985e-08
2308 3.98255757261268e-08
2309 3.98952906266459e-08
2310 3.97452879496996e-08
2311 4.01589872467412e-08
2312 3.97421509035212e-08
2313 3.92659949000063e-08
2314 3.99212325419285e-08
2315 3.96537558344789e-08
2316 3.96505619448817e-08
2317 3.95432202537904e-08
2318 3.98335728846178e-08
2319 3.93281318622485e-08
2320 3.95635844085973e-08
2321 3.93936261389172e-08
2322 3.95199997171858e-08
2323 3.92669612381269e-08
2324 3.9323015954551e-08
2325 3.93817920496531e-08
2326 3.93113062102657e-08
2327 3.93869790116241e-08
2328 3.90906151892523e-08
2329 3.92018684181039e-08
2330 3.68153969532159e-08
2331 3.89231082920105e-08
2332 3.88178094112845e-08
2333 3.90643783987343e-08
2334 3.88751111302099e-08
2335 3.90357293156285e-08
2336 3.88393033290413e-08
2337 3.87460730166822e-08
2338 3.88498015979621e-08
2339 3.86872827107254e-08
2340 3.87448295668946e-08
2341 3.81194880105795e-08
2342 3.87578360516727e-08
2343 3.87653287248213e-08
2344 3.86703433719049e-08
2345 3.85912315437054e-08
2346 3.84925691321314e-08
2347 3.84284355448017e-08
2348 3.81665401505416e-08
2349 3.85068155139834e-08
2350 3.58636036423832e-08
2351 3.84321516833097e-08
2352 3.83607137166564e-08
2353 3.82517661989823e-08
2354 3.81875295829559e-08
2355 3.82710645396855e-08
2356 3.79334643696438e-08
2357 3.8527780077402e-08
2358 3.82526685882567e-08
2359 3.80800706523132e-08
2360 3.8141482860965e-08
2361 3.81984825992276e-08
2362 3.81464069221238e-08
2363 3.79469931033327e-08
2364 3.78307909443265e-08
2365 3.76117448297464e-08
2366 3.79737450373341e-08
2367 3.77642628279773e-08
2368 3.77376565552368e-08
2369 3.79049325260894e-08
2370 3.77990794220295e-08
2371 3.75671120877996e-08
2372 3.76975215488073e-08
2373 3.75311657307975e-08
2374 3.7730508495315e-08
2375 3.71603867677095e-08
2376 3.76222040188168e-08
2377 3.74097766098203e-08
2378 3.74272630665473e-08
2379 3.76315725247878e-08
2380 3.73388502339367e-08
2381 3.72696682404694e-08
2382 3.72012323168747e-08
2383 3.76373492372295e-08
2384 3.69405164235559e-08
2385 3.72477018117934e-08
2386 3.7232265270859e-08
2387 3.72560080563744e-08
2388 3.69422927803953e-08
2389 3.70772319513435e-08
2390 3.71201025473056e-08
2391 3.72165729345397e-08
2392 3.70841384267351e-08
2393 3.68642858461499e-08
2394 3.71668988918827e-08
2395 3.70100892155278e-08
2396 3.68627937064048e-08
2397 3.68521604343641e-08
2398 3.66901424797561e-08
2399 3.69923185417065e-08
2400 3.66620760416936e-08
2401 3.67402215317725e-08
2402 3.65521906076083e-08
2403 3.6649328905014e-08
2404 3.65905776789077e-08
2405 3.69259645083275e-08
2406 3.67301460357794e-08
2407 3.66646872862475e-08
2408 3.64055878776526e-08
2409 3.67153418778798e-08
2410 3.66455736866556e-08
2411 3.65647991884543e-08
2412 3.64536170138763e-08
2413 3.64712384737231e-08
2414 3.63787293622408e-08
2415 3.65215733211244e-08
2416 3.62784753349388e-08
2417 3.65248062905721e-08
2418 3.62760346206414e-08
2419 3.64954679810126e-08
2420 3.62495100603155e-08
2421 3.63274530457147e-08
2422 3.68108921122712e-08
2423 3.66407739704755e-08
2424 3.63630334732079e-08
2425 3.61635947854211e-08
2426 3.63392693714104e-08
2427 3.5892551153438e-08
2428 3.61064032006198e-08
2429 3.62606158432754e-08
2430 3.60802587806575e-08
2431 3.6138199988045e-08
2432 3.60560505896501e-08
2433 3.60104941421469e-08
2434 3.61392942238581e-08
2435 3.59221310475277e-08
2436 3.604127485346e-08
2437 3.58458578375576e-08
2438 3.59249057169109e-08
2439 3.52294371452899e-08
2440 3.59949829942252e-08
2441 3.58188962934491e-08
2442 3.57287781582727e-08
2443 3.59294247687103e-08
2444 3.58903129438204e-08
2445 3.57452165644645e-08
2446 3.577550700129e-08
2447 3.55211717817383e-08
2448 3.56520857280884e-08
2449 3.53452414003641e-08
2450 3.56191094397218e-08
2451 3.51263622633269e-08
2452 3.5506353412984e-08
2453 3.53937679165028e-08
2454 3.55145317598726e-08
2455 3.53221984994434e-08
2456 3.48855984100283e-08
2457 3.55044527111659e-08
2458 3.48838149477615e-08
2459 3.51195375003499e-08
2460 3.53707072520137e-08
2461 3.53499345351338e-08
2462 3.51672504450562e-08
2463 3.53063214220128e-08
2464 3.47917215037796e-08
2465 3.51752689198293e-08
2466 3.52141995563215e-08
2467 3.49685009837231e-08
2468 3.51146027810501e-08
2469 3.51591040725907e-08
2470 3.49475364203045e-08
2471 3.52191626973308e-08
2472 3.4685058381001e-08
2473 3.4915000668434e-08
2474 3.46837296660851e-08
2475 3.46809443385609e-08
2476 3.49106734631732e-08
2477 3.43981803041515e-08
2478 3.47364483843648e-08
2479 3.45927269052027e-08
2480 3.45073587482148e-08
2481 3.43723520757067e-08
2482 3.45900872389393e-08
2483 3.44050192779832e-08
2484 3.4515053926043e-08
2485 3.43603261399039e-08
2486 3.43140378333828e-08
2487 3.43520660806007e-08
2488 3.43077246611756e-08
2489 3.40817827293449e-08
2490 3.44323680678826e-08
2491 3.41438983753051e-08
2492 3.43711086259191e-08
2493 3.41655130853269e-08
2494 3.41990968877326e-08
2495 3.41605215226082e-08
2496 3.42086785565243e-08
2497 3.4092920486728e-08
2498 3.4055357645002e-08
2499 3.42058150692992e-08
2500 3.4090110290208e-08
2501 3.40123271769244e-08
2502 3.40883623550781e-08
2503 3.41288384220206e-08
2504 3.39619283806769e-08
2505 3.4238944124354e-08
2506 3.39713785990625e-08
2507 3.370174539441e-08
2508 3.38179226844204e-08
2509 3.39023351614287e-08
2510 3.35118386374234e-08
2511 3.36586865046229e-08
2512 3.3681569533428e-08
2513 3.36721193150424e-08
2514 3.37302452635413e-08
2515 3.36183916260779e-08
2516 3.37473089473406e-08
2517 3.3591568637803e-08
2518 3.36372814047081e-08
2519 3.36173862081068e-08
2520 3.32369296529578e-08
2521 3.32613900866363e-08
2522 3.33198286739389e-08
2523 3.34137482127517e-08
2524 3.34936949286657e-08
2525 3.33420580034272e-08
2526 3.34133645196744e-08
2527 3.31664296027157e-08
2528 3.31904885797485e-08
2529 3.31772938011454e-08
2530 3.30188001385068e-08
2531 3.30089662270439e-08
2532 3.28449232256389e-08
2533 3.29202727300526e-08
2534 3.29096714324351e-08
2535 3.27732188054597e-08
2536 3.27674278821632e-08
2537 3.26582814125231e-08
2538 3.2756922507815e-08
2539 3.25492308661524e-08
2540 3.27434364066903e-08
2541 3.29449640901203e-08
2542 3.24878790536332e-08
2543 3.27112097409099e-08
2544 3.23864952633812e-08
2545 3.25045697024962e-08
2546 3.22918900508284e-08
2547 3.25514442067742e-08
2548 3.22171480604538e-08
2549 3.24969668952235e-08
2550 3.20893072114359e-08
2551 3.20971551559524e-08
2552 3.23081970066141e-08
2553 3.20099609041335e-08
2554 3.20597273173462e-08
2555 3.17942543404115e-08
2556 3.20765387584743e-08
2557 3.18516022446147e-08
2558 3.22147535314343e-08
2559 3.17822532736045e-08
2560 3.17789421444559e-08
2561 3.18175352731487e-08
2562 3.22041096012526e-08
2563 3.16271240308197e-08
2564 3.17206598765551e-08
2565 3.17430952634368e-08
2566 3.15644221871025e-08
2567 3.15121688743147e-08
2568 3.16011323775456e-08
2569 3.15047259391577e-08
2570 3.17715063147261e-08
2571 3.14489625452552e-08
2572 3.13835997189926e-08
2573 3.13844559229892e-08
2574 3.14468380224753e-08
2575 3.12689820702872e-08
2576 3.15693355901203e-08
2577 3.13653600869657e-08
2578 3.14902237619208e-08
2579 3.13274810537223e-08
2580 3.11064205504863e-08
2581 3.12310319827702e-08
2582 3.12299306415298e-08
2583 3.13979278132592e-08
2584 3.14591162009492e-08
2585 3.12045784767179e-08
2586 3.14446531035628e-08
2587 3.12915133804381e-08
2588 3.13754888736639e-08
2589 3.12704493410365e-08
2590 3.08287013695008e-08
2591 3.12705594751606e-08
2592 3.08944372306996e-08
2593 3.10438181827521e-08
2594 3.09829353284385e-08
2595 3.11296020072405e-08
2596 3.08584162667103e-08
2597 3.10219796517686e-08
2598 3.06765279844967e-08
2599 3.09036458645551e-08
2600 3.06786276382809e-08
2601 3.07023668710826e-08
2602 3.081570199015e-08
2603 3.08716003871723e-08
2604 3.07084064843366e-08
2605 3.07608551963767e-08
2606 3.07680352307216e-08
2607 3.06984766496043e-08
2608 3.10301260242341e-08
2609 3.09446370749811e-08
2610 3.08833847384449e-08
2611 3.0672232753659e-08
2612 3.07927656706397e-08
2613 3.06134495531296e-08
2614 3.07147942635311e-08
2615 3.05900123009906e-08
2616 3.05955971668936e-08
2617 3.06902236957285e-08
2618 3.07414467215494e-08
2619 3.08361585155126e-08
2620 3.05548049084337e-08
2621 3.05601126626698e-08
2622 3.06821519302503e-08
2623 3.05365652764067e-08
2624 3.0542601336947e-08
2625 3.04793275063275e-08
2626 3.0469809786382e-08
2627 3.05915328624451e-08
2628 3.03447649230293e-08
2629 3.03455571781797e-08
2630 3.05495078123386e-08
2631 3.04728011712996e-08
2632 3.02992049228124e-08
2633 3.02614076019836e-08
2634 3.04669214301612e-08
2635 3.0235465686701e-08
2636 3.05124778776644e-08
2637 3.01504954336451e-08
2638 3.01596507767954e-08
2639 3.03972633730609e-08
2640 3.01054221552022e-08
2641 3.03006117974292e-08
2642 3.02944940244743e-08
2643 3.02023224207915e-08
2644 2.99962934491305e-08
2645 3.02114990802238e-08
2646 2.99183007257398e-08
2647 3.0248738625005e-08
2648 3.00552898124806e-08
2649 3.02209386404684e-08
2650 2.99460651831396e-08
2651 2.98984836888394e-08
2652 3.00689109167251e-08
2653 2.97906517232605e-08
2654 2.98488664896013e-08
2655 2.9975769422208e-08
2656 3.00284170862142e-08
2657 2.98307867296899e-08
2658 3.00268823139049e-08
2659 2.98323072911444e-08
2660 2.98889304417571e-08
2661 2.96856974557613e-08
2662 2.98327833547773e-08
2663 2.94936093325759e-08
2664 2.9906164655813e-08
2665 2.97529414439168e-08
2666 2.97789526371162e-08
2667 2.96517690401288e-08
2668 2.97901827650549e-08
2669 2.96009528000241e-08
2670 2.98219937633348e-08
2671 2.96727211690495e-08
2672 2.97782438707372e-08
2673 2.983689029179e-08
2674 2.9515785371359e-08
2675 2.95876620981517e-08
2676 2.93957551633639e-08
2677 2.94589561633529e-08
2678 2.93473814139134e-08
2679 2.96271700506168e-08
2680 2.95269924066588e-08
2681 2.95923836546308e-08
2682 2.92871735751987e-08
2683 2.93909216964039e-08
2684 2.95361104463154e-08
2685 2.92840152127383e-08
2686 2.95032140940066e-08
2687 2.93084347902095e-08
2688 2.93347728330673e-08
2689 2.91973467625439e-08
2690 2.93625550540355e-08
2691 2.91760215986869e-08
2692 2.93223809677556e-08
2693 2.90174924089115e-08
2694 2.93358066727478e-08
2695 2.90479533759935e-08
2696 2.92883477470696e-08
2697 2.89122166208244e-08
2698 2.90683175308004e-08
2699 2.88461432518261e-08
2700 2.90747088627086e-08
2701 2.8940824847723e-08
2702 2.9105619248071e-08
2703 2.88887740396149e-08
2704 2.91337265423408e-08
2705 2.87830062006833e-08
2706 2.89670634145978e-08
2707 2.88744033127841e-08
2708 2.89387429575072e-08
2709 2.87284258604359e-08
2710 2.89256441021735e-08
2711 2.86359966850114e-08
2712 2.89111792284302e-08
2713 2.86523391679339e-08
2714 2.89133001984965e-08
2715 2.85685057832552e-08
2716 2.8731566459328e-08
2717 2.85140853151233e-08
2718 2.89726820312808e-08
2719 2.89849744206094e-08
2720 2.88099784029328e-08
2721 2.84025940544552e-08
2722 2.86173911234755e-08
2723 2.82178334032324e-08
2724 2.86063741583575e-08
2725 2.83437255887975e-08
2726 2.85396417609718e-08
2727 2.83448873261705e-08
2728 2.85019865486902e-08
2729 2.81236278709684e-08
2730 2.84228978131296e-08
2731 2.83746643958693e-08
2732 2.84231322922324e-08
2733 2.79570002703622e-08
2734 2.83648056154107e-08
2735 2.80379026662558e-08
2736 2.82021304087721e-08
2737 2.79910441491893e-08
2738 2.83764247654972e-08
2739 2.79214109610848e-08
2740 2.82087544434262e-08
2741 2.8029074172764e-08
2742 2.83154175662048e-08
2743 2.7892971488086e-08
2744 2.79927316881867e-08
2745 2.81736962648438e-08
2746 2.8002826724105e-08
2747 2.77752150168453e-08
2748 2.77710210383475e-08
2749 2.72799400846679e-08
2750 2.79137317704681e-08
2751 2.73175864151654e-08
2752 2.74519980081323e-08
2753 2.73141633755358e-08
2754 2.72542433066292e-08
2755 2.68929216673541e-08
2756 2.70524846968101e-08
2757 2.75427520790572e-08
2758 2.71938844775832e-08
2759 2.70794675572006e-08
2760 2.730759085523e-08
2761 2.71450293354292e-08
2762 2.75030060947756e-08
2763 2.69720317191968e-08
2764 2.70666902224548e-08
2765 2.71410200980426e-08
2766 2.6944256603656e-08
2767 2.67894559868864e-08
2768 2.68728719277078e-08
2769 2.67465303238623e-08
2770 2.68507722722688e-08
2771 2.70395581480898e-08
2772 2.69929412155534e-08
2773 2.70923390388589e-08
2774 2.67327564529296e-08
2775 2.67407749277027e-08
2776 2.70116924383501e-08
2777 2.66365809409308e-08
2778 2.68452744478509e-08
2779 2.69718221090898e-08
2780 2.65095430052043e-08
2781 2.64589949949823e-08
2782 2.68307971396098e-08
2783 2.6707102307455e-08
2784 2.6715055057025e-08
2785 2.68777160528089e-08
2786 2.67176130108737e-08
2787 2.66842210550067e-08
2788 2.6562073429659e-08
2789 2.60942965013555e-08
2790 2.62003716500203e-08
2791 2.67031818879104e-08
2792 2.66770872059396e-08
2793 2.68313318230184e-08
2794 2.63454609239488e-08
2795 2.65360338147502e-08
2796 2.65871857862976e-08
2797 2.67662532138502e-08
2798 2.6635074590331e-08
2799 2.64407340466732e-08
2800 2.64297614904763e-08
2801 2.61562043135655e-08
2802 2.632881290765e-08
2803 2.61725059402806e-08
2804 2.6315042589431e-08
2805 2.62159005615104e-08
2806 2.6132848773841e-08
2807 2.62789985328027e-08
2808 2.64524739890248e-08
2809 2.64609703037877e-08
2810 2.63745327799825e-08
2811 2.64741615296771e-08
2812 2.63436561454e-08
2813 2.59313086559132e-08
2814 2.62042973986354e-08
2815 2.61726409434004e-08
2816 2.60767762938485e-08
2817 2.61166945847435e-08
2818 2.63272958989091e-08
2819 2.61451216232444e-08
2820 2.59287578074918e-08
2821 2.59546073522188e-08
2822 2.61853418948021e-08
2823 2.601707649319e-08
2824 2.60897099479962e-08
2825 2.58217944804073e-08
2826 2.5676802906105e-08
2827 2.58790091578476e-08
2828 2.56637697759743e-08
2829 2.56479975035973e-08
2830 2.5917628931893e-08
2831 2.58464503133382e-08
2832 2.57961634275716e-08
2833 2.57587835506001e-08
2834 2.55680205896169e-08
2835 2.57394248137643e-08
2836 2.5578716034147e-08
2837 2.54447272141078e-08
2838 2.55961403183846e-08
2839 2.55351437772333e-08
2840 2.53808991601545e-08
2841 2.56149359501023e-08
2842 2.53157583784969e-08
2843 2.55350425248935e-08
2844 2.54726568726937e-08
2845 2.54327048310188e-08
2846 2.55025884854376e-08
2847 2.53813485784349e-08
2848 2.5551484483799e-08
2849 2.53554635065711e-08
2850 2.53278606976437e-08
2851 2.54200624993928e-08
2852 2.53549714557266e-08
2853 2.52261855848701e-08
2854 2.52447787119081e-08
2855 2.52616558782393e-08
2856 2.48662832547097e-08
2857 2.50604603735383e-08
2858 2.52863756600163e-08
2859 2.49064964208401e-08
2860 2.50989486971775e-08
2861 2.50473330964951e-08
2862 2.49696885390449e-08
2863 2.5018962901413e-08
2864 2.49036897770338e-08
2865 2.51610021706483e-08
2866 2.53765275459727e-08
2867 2.50197764728455e-08
2868 2.52778455944735e-08
2869 2.5065904907251e-08
2870 2.55147831751401e-08
2871 2.50251801503509e-08
2872 2.48586502493708e-08
2873 2.50539589075061e-08
2874 2.53465870514447e-08
2875 2.50530227674517e-08
2876 2.5085183708029e-08
2877 2.50027536452535e-08
2878 2.49942821994864e-08
2879 2.51309586474235e-08
2880 2.47864964109112e-08
2881 2.53482941303673e-08
2882 2.50329090789592e-08
2883 2.49217126935264e-08
2884 2.48765434918141e-08
2885 2.47948666043385e-08
2886 2.4916031904354e-08
2887 2.45314577540512e-08
2888 2.48620093401541e-08
2889 2.4875582482764e-08
2890 2.48342661990364e-08
2891 2.49499212401361e-08
2892 2.48405829239573e-08
2893 2.48709195460606e-08
2894 2.4991388514195e-08
2895 2.49688003606252e-08
2896 2.48107117073459e-08
2897 2.46732660968974e-08
2898 2.47227376348746e-08
2899 2.46527207536928e-08
2900 2.50275977720094e-08
2901 2.48525378054865e-08
2902 2.50070080198839e-08
2903 2.47686564591731e-08
2904 2.50540104218544e-08
2905 2.47407729858651e-08
2906 2.50413023650253e-08
2907 2.4715323121427e-08
2908 2.50436240634144e-08
2909 2.47681750664697e-08
2910 2.49564084953136e-08
2911 2.47672016229217e-08
2912 2.5025777006249e-08
2913 2.54234517882423e-08
2914 2.50008955759995e-08
2915 2.55612047084242e-08
2916 2.50095482101642e-08
2917 2.54896832529994e-08
2918 2.4903588524694e-08
2919 2.45591351699659e-08
2920 2.33257271275988e-08
2921 2.44173286034766e-08
2922 2.46413165427839e-08
2923 2.44822100370357e-08
2924 2.49165257315553e-08
2925 2.44797817572362e-08
2926 2.45801512477328e-08
2927 2.49258071960412e-08
2928 2.4378953966675e-08
2929 2.440437185669e-08
2930 2.48490774623633e-08
2931 2.44963942463983e-08
2932 2.45625777495206e-08
2933 2.45036329005188e-08
2934 2.46371349987839e-08
2935 2.45024356360091e-08
2936 2.45997338055304e-08
2937 2.47299851707794e-08
2938 2.44412223793233e-08
2939 2.4722501379415e-08
2940 2.54026097934457e-08
2941 2.42594033750265e-08
2942 2.43230182661591e-08
2943 2.43939428656859e-08
2944 2.42536817296468e-08
2945 2.43917046560682e-08
2946 2.47241320749936e-08
2947 2.44190481168971e-08
2948 2.42357707236351e-08
2949 2.47667042430066e-08
2950 2.44474076538381e-08
2951 2.47661411378886e-08
2952 2.44191618037348e-08
2953 2.47621247950747e-08
2954 2.53370657787855e-08
2955 2.46990961016991e-08
2956 2.44221318723703e-08
2957 2.54650043274296e-08
2958 2.4341822779661e-08
2959 2.45831177636546e-08
2960 2.44068356636262e-08
2961 2.52449190440984e-08
2962 2.46607765319595e-08
2963 2.43697275692512e-08
2964 2.45387994368684e-08
2965 2.42314399656607e-08
2966 2.46197959796746e-08
2967 2.43457112247825e-08
2968 2.44503901569715e-08
2969 2.41353834695701e-08
2970 2.40409328000624e-08
2971 2.38802524421544e-08
2972 2.43424036483475e-08
2973 2.40018511732387e-08
2974 2.42939464101255e-08
2975 2.50835459070231e-08
2976 2.42975044528748e-08
2977 2.38393234042178e-08
2978 2.42140583139872e-08
2979 2.49100242655231e-08
2980 2.41602542416786e-08
2981 2.3892766876088e-08
2982 2.39356570119753e-08
2983 2.38474537894717e-08
2984 2.38573711897061e-08
2985 2.41930084854403e-08
2986 2.49841267674356e-08
2987 2.41182593896383e-08
2988 2.3926247649797e-08
2989 2.41143371937369e-08
2990 2.48639100419723e-08
2991 2.38222863657711e-08
2992 2.37044943673936e-08
2993 2.41944793089033e-08
2994 2.50081466646179e-08
2995 2.50156126924139e-08
2996 2.41073312423623e-08
2997 2.37097363964267e-08
2998 2.34838015700234e-08
2999 2.3978454777307e-08
3000 1.71995129250035e-08
3001 1.71553686811876e-08
3002 1.71758642864006e-08
3003 1.7185051603974e-08
3004 1.7181500666652e-08
3005 1.71726810549444e-08
3006 1.71645417879063e-08
3007 1.71568252937959e-08
3008 1.71505121215887e-08
3009 1.71490235345573e-08
3010 1.71427512185574e-08
3011 1.71375944546526e-08
3012 1.71310272634173e-08
3013 1.7126351892216e-08
3014 1.71131802062519e-08
3015 1.71103682333751e-08
3016 1.71053180508807e-08
3017 1.70975980040566e-08
3018 1.70914464803218e-08
3019 1.70880802841111e-08
3020 1.70901959251069e-08
3021 1.70877534344527e-08
3022 1.70778182706499e-08
3023 1.708018437796e-08
3024 1.70760383610968e-08
3025 1.70680820588132e-08
3026 1.70691727419126e-08
3027 1.70608291938379e-08
3028 1.70622538320231e-08
3029 1.71475420529532e-08
3030 1.71454175301733e-08
3031 1.71436536078318e-08
3032 1.71398184534155e-08
3033 1.71395431181054e-08
3034 1.71360579059865e-08
3035 1.71344645139015e-08
3036 1.71326934861327e-08
3037 1.71309917362805e-08
3038 1.71302296791964e-08
3039 1.71268403903468e-08
3040 1.71241243407394e-08
3041 1.71242646729297e-08
3042 1.71225007505882e-08
3043 1.71199445730963e-08
3044 1.71173990537454e-08
3045 1.71153047290318e-08
3046 1.71137433113699e-08
3047 1.71061849130183e-08
3048 1.71086256273156e-08
3049 1.71058207598662e-08
3050 1.71033445184321e-08
3051 1.71010103855451e-08
3052 1.70982410452325e-08
3053 1.70958678324951e-08
3054 1.70905405383337e-08
3055 1.70912208830032e-08
3056 1.7086865256033e-08
3057 1.70866556459259e-08
3058 1.70808753807705e-08
3059 1.70817902045428e-08
3060 1.70750418249099e-08
3061 1.70757790129983e-08
3062 1.70718603698106e-08
3063 1.70722689318836e-08
3064 1.70670126919958e-08
3065 1.7064710533532e-08
3066 1.70653464692805e-08
3067 1.70601683890936e-08
3068 1.70613851935286e-08
3069 1.70561893497734e-08
3070 1.70556262446553e-08
3071 1.7052112610827e-08
3072 1.70493450468712e-08
3073 1.7049158529403e-08
3074 1.70470180194116e-08
3075 1.70430372037345e-08
3076 1.70439289348678e-08
3077 1.70477676419978e-08
3078 1.70467604476698e-08
3079 1.70411436073437e-08
3080 1.70425149548237e-08
3081 1.70413727573759e-08
3082 1.70380580755136e-08
3083 1.70297482782189e-08
3084 1.70228044993337e-08
3085 1.70162817170194e-08
3086 1.70120753040237e-08
3087 1.70178697800338e-08
3088 1.70139280442072e-08
3089 1.70626730522372e-08
3090 1.70146350342293e-08
3091 1.70115121989056e-08
3092 1.70078831018827e-08
3093 1.70077374406219e-08
3094 1.7059290868815e-08
3095 1.70039502478403e-08
3096 1.69996692278573e-08
3097 1.70020459933085e-08
3098 1.69968359386985e-08
3099 1.69962230955889e-08
3100 1.69954894602142e-08
3101 1.69910521208294e-08
3102 1.69924039283842e-08
3103 1.70388325670956e-08
3104 1.69883396239356e-08
3105 1.69850267184302e-08
3106 1.69827227836095e-08
3107 1.69813763051252e-08
3108 1.69812661710012e-08
3109 1.70276379662937e-08
3110 1.69696150464915e-08
3111 1.69730398624779e-08
3112 1.6970394867144e-08
3113 1.69702421004558e-08
3114 1.70161378321154e-08
3115 1.69661937832188e-08
3116 1.6966678728636e-08
3117 1.69613318945494e-08
3118 1.69586673592903e-08
3119 1.69580935960312e-08
3120 1.69550968820431e-08
3121 1.69567453411901e-08
3122 1.69541873873413e-08
3123 1.69409748451699e-08
3124 1.69390528270696e-08
3125 1.699548057843e-08
3126 1.69454388299073e-08
3127 1.69344556155693e-08
3128 1.69307057262813e-08
3129 1.69851404052679e-08
3130 1.69280340855948e-08
3131 1.69248384196408e-08
3132 1.69315370612821e-08
3133 1.69202447608541e-08
3134 1.69210156997224e-08
3135 1.69175038422509e-08
3136 1.6915787881544e-08
3137 1.69143739014999e-08
3138 1.69110503378533e-08
3139 1.69116454173945e-08
3140 1.69086504797633e-08
3141 1.690426287837e-08
3142 1.69057763343972e-08
3143 1.69045470954643e-08
3144 1.69019553908356e-08
3145 1.68998059990599e-08
3146 1.68984275461526e-08
3147 1.68953206980405e-08
3148 1.68952674073353e-08
3149 1.68929119581662e-08
3150 1.68918035114984e-08
3151 1.68897269503532e-08
3152 1.68861689076039e-08
3153 1.68868208305639e-08
3154 1.68827476443312e-08
3155 1.6881045894479e-08
3156 1.68814722201205e-08
3157 1.68767186892183e-08
3158 1.68765570407459e-08
3159 1.68733187422276e-08
3160 1.68743277129124e-08
3161 1.68724643145879e-08
3162 1.68692508850654e-08
3163 1.68731837391078e-08
3164 1.68125016131171e-08
3165 1.69335176991581e-08
3166 1.68631864028157e-08
3167 1.6806751546028e-08
3168 1.69116241011125e-08
3169 1.68575837733442e-08
3170 1.6800711932774e-08
3171 1.69124518833996e-08
3172 1.69085794254897e-08
3173 1.68969549463327e-08
3174 1.67932263650528e-08
3175 1.69028488983258e-08
3176 1.68911462594679e-08
3177 1.68993992133437e-08
3178 1.68871459038655e-08
3179 1.68955427426454e-08
3180 1.68836837843855e-08
3181 1.68918372622784e-08
3182 1.68907501318927e-08
3183 1.68775571296464e-08
3184 1.68859806137789e-08
3185 1.68742406714273e-08
3186 1.68831668645453e-08
3187 1.68805733835597e-08
3188 1.68789195953423e-08
3189 1.68766209895921e-08
3190 1.68745604156584e-08
3191 1.68637086517265e-08
3192 1.68715725834545e-08
3193 1.68688512047765e-08
3194 1.68583778048514e-08
3195 1.68234031150405e-08
3196 1.6867563346068e-08
3197 1.68618949913935e-08
3198 1.68586495874479e-08
3199 1.6858040297052e-08
3200 1.68155356305988e-08
3201 1.68128835298376e-08
3202 1.68543721201786e-08
3203 1.68521481214157e-08
3204 1.6847156558697e-08
3205 1.67891123226127e-08
3206 1.67893912106365e-08
3207 1.68368767816673e-08
3208 1.67845168874692e-08
3209 1.68355409613241e-08
3210 1.67775802140113e-08
3211 1.6830963289749e-08
3212 1.6773912037138e-08
3213 1.68279807866156e-08
3214 1.67699010233946e-08
3215 1.68228186936403e-08
3216 1.6767152999364e-08
3217 1.68195786187653e-08
3218 1.68194027594382e-08
3219 1.67623213087609e-08
3220 1.6815389969338e-08
3221 1.68153206914212e-08
3222 1.67577596243973e-08
3223 1.68111373710644e-08
3224 1.6760893117862e-08
3225 1.67610387791228e-08
3226 1.67574558673778e-08
3227 1.68014384627213e-08
3228 1.67535834094679e-08
3229 1.68031064617935e-08
3230 1.68000102718224e-08
3231 1.68004774536712e-08
3232 1.67465135092471e-08
3233 1.67964664399278e-08
3234 1.67920592986093e-08
3235 1.67433977793507e-08
3236 1.67266165362889e-08
3237 1.67515885607372e-08
3238 1.67833551500962e-08
3239 1.67281264396024e-08
3240 1.67293325858964e-08
3241 1.67334412992659e-08
3242 1.67241722692779e-08
3243 1.6729767793322e-08
3244 1.67237796944164e-08
3245 1.67222715674598e-08
3246 1.67262541594937e-08
3247 1.67150435714802e-08
3248 1.67239431192456e-08
3249 1.67124678540631e-08
3250 1.67113309856859e-08
3251 1.67175251419849e-08
3252 1.67064282408091e-08
3253 1.67062399469842e-08
3254 1.6704969851844e-08
3255 1.670105476137e-08
3256 1.67023692654311e-08
3257 1.66985447691559e-08
3258 1.66989124750216e-08
3259 1.6695988591664e-08
3260 1.66898033171492e-08
3261 1.66920308686258e-08
3262 1.6688533222009e-08
3263 1.66872524687278e-08
3264 1.66861138239938e-08
3265 1.6687151216388e-08
3266 1.668199622884e-08
3267 1.66851013005953e-08
3268 1.66821880753787e-08
3269 1.6673402214451e-08
3270 1.66052256389548e-08
3271 1.66068065965419e-08
3272 1.6674807312711e-08
3273 1.66806994883473e-08
3274 1.66701532577918e-08
3275 1.66757665454043e-08
3276 1.66725531158818e-08
3277 1.66576512583561e-08
3278 1.66693752134961e-08
3279 1.65850924105371e-08
3280 1.66576654692108e-08
3281 1.65826641307376e-08
3282 1.66522031719296e-08
3283 1.66598610462643e-08
3284 1.65748055280801e-08
3285 1.66420068836715e-08
3286 1.66456484151922e-08
3287 1.66430602632772e-08
3288 1.65731801615721e-08
3289 1.66476290530682e-08
3290 1.66313860461287e-08
3291 1.66402482904005e-08
3292 1.66337414952977e-08
3293 1.66329403583632e-08
3294 1.6628993293466e-08
3295 1.6630608001833e-08
3296 1.66313576244193e-08
3297 1.66255791356207e-08
3298 1.6626682253218e-08
3299 1.66237867915697e-08
3300 1.66240585741662e-08
3301 1.6618695752868e-08
3302 1.66170988080694e-08
3303 1.66137503754271e-08
3304 1.66166351789343e-08
3305 1.66114926258842e-08
3306 1.65290447995403e-08
3307 1.65316826894468e-08
3308 1.64241225064643e-08
3309 1.64243285638577e-08
3310 1.65248792427519e-08
3311 1.64199018826139e-08
3312 1.64171485295128e-08
3313 1.64220530507464e-08
3314 1.65291726972328e-08
3315 1.64145923520209e-08
3316 1.64119011714092e-08
3317 1.64092064380839e-08
3318 1.64064815066922e-08
3319 1.64032591953855e-08
3320 1.64052416096183e-08
3321 1.64069859920346e-08
3322 1.6398454150135e-08
3323 1.63964344324086e-08
3324 1.63983475687246e-08
3325 1.64005093949982e-08
3326 1.63939777308997e-08
3327 1.63902491578938e-08
3328 1.65037281618652e-08
3329 1.63880429226992e-08
3330 1.63856661572481e-08
3331 1.63879416703594e-08
3332 1.63824864785056e-08
3333 1.63872986291835e-08
3334 1.63833675514979e-08
3335 1.63804187991445e-08
3336 1.63757718496527e-08
3337 1.63760649485312e-08
3338 1.63733950842015e-08
3339 1.63734377167657e-08
3340 1.63701692201812e-08
3341 1.63680002884803e-08
3342 1.63683999687692e-08
3343 1.63649040985092e-08
3344 1.63682560838652e-08
3345 1.63595945679162e-08
3346 1.63592268620505e-08
3347 1.63582729584277e-08
3348 1.63613282921915e-08
3349 1.64680695746711e-08
3350 1.63540487818636e-08
3351 1.64477302888599e-08
3352 1.63566298283513e-08
3353 1.64621205556159e-08
3354 1.63468030223157e-08
3355 1.63440958544925e-08
3356 1.6343216557857e-08
3357 1.63414366483039e-08
3358 1.64380917766493e-08
3359 1.63446536305401e-08
3360 1.64523434875719e-08
3361 1.633776314236e-08
3362 1.64500324473238e-08
3363 1.64467177654615e-08
3364 1.63479736414729e-08
3365 1.6448337802899e-08
3366 1.64473377139984e-08
3367 1.64440123739951e-08
3368 1.6441369155018e-08
3369 1.64392499613086e-08
3370 1.64218203480004e-08
3371 1.64399747148991e-08
3372 1.64372835342874e-08
3373 1.64282507597591e-08
3374 1.64306523942059e-08
3375 1.6411195957744e-08
3376 1.63171272049567e-08
3377 1.6318494999723e-08
3378 1.63143010212252e-08
3379 1.64082525344611e-08
3380 1.63093236693612e-08
3381 1.63080908777147e-08
3382 1.63961448862437e-08
3383 1.6307634354007e-08
3384 1.63908957517833e-08
3385 1.62949262971779e-08
3386 1.62994844288278e-08
3387 1.62927751290454e-08
3388 1.62948303739086e-08
3389 1.62897517697047e-08
3390 1.62984896689977e-08
3391 1.62836748529571e-08
3392 1.62956226290589e-08
3393 1.62888991184218e-08
3394 1.63706523892415e-08
3395 1.63673359310224e-08
3396 1.62794204783268e-08
3397 1.62750257715061e-08
3398 1.62746349730014e-08
3399 1.62721871532767e-08
3400 1.62703823747279e-08
3401 1.62721072172189e-08
3402 1.62631188516116e-08
3403 1.62661883962301e-08
3404 1.63683147036409e-08
3405 1.63449769274848e-08
3406 1.63624989113487e-08
3407 1.62560418459634e-08
3408 1.63440549982852e-08
3409 1.62544946391563e-08
3410 1.6252297285746e-08
3411 1.62501141431903e-08
3412 1.62450213281318e-08
3413 1.63341233871961e-08
3414 1.62471103237749e-08
3415 1.62469664388709e-08
3416 1.6238757893916e-08
3417 1.62370508149934e-08
3418 1.62426676553196e-08
3419 1.62321676100419e-08
3420 1.62332991493486e-08
3421 1.62313380513979e-08
3422 1.62353259725023e-08
3423 1.62285243021643e-08
3424 1.62240993972773e-08
3425 1.62245843426945e-08
3426 1.62226569955237e-08
3427 1.62244990775662e-08
3428 1.622097300924e-08
3429 1.62191611252638e-08
3430 1.62312296936307e-08
3431 1.62187294705518e-08
3432 1.62306310613758e-08
3433 1.62156936767133e-08
3434 1.62256998947896e-08
3435 1.62102793410668e-08
3436 1.62057673946947e-08
3437 1.62102598011415e-08
3438 1.62088404920269e-08
3439 1.62900786193632e-08
3440 1.61920699071061e-08
3441 1.61918922714221e-08
3442 1.61871618331588e-08
3443 1.61863731307221e-08
3444 1.61848099367035e-08
3445 1.61844049273441e-08
3446 1.61830211453662e-08
3447 1.61797171216449e-08
3448 1.61786495311844e-08
3449 1.61780864260663e-08
3450 1.61765445483297e-08
3451 1.61744857507529e-08
3452 1.61698352485473e-08
3453 1.61713504809313e-08
3454 1.61692774724997e-08
3455 1.61679967192185e-08
3456 1.61662061515244e-08
3457 1.61629820638609e-08
3458 1.61636730666714e-08
3459 1.61630868689144e-08
3460 1.61631312778354e-08
3461 1.61634421402823e-08
3462 1.61601327874905e-08
3463 1.61605484549909e-08
3464 1.61584381430657e-08
3465 1.61568483036945e-08
3466 1.61541269250165e-08
3467 1.61540505416724e-08
3468 1.61511408691695e-08
3469 1.61497037964864e-08
3470 1.6147694736901e-08
3471 1.61476432225527e-08
3472 1.61447815116844e-08
3473 1.61429625222809e-08
3474 1.6141500580602e-08
3475 1.61402997633786e-08
3476 1.61383368890711e-08
3477 1.61384932084729e-08
3478 1.61340274473787e-08
3479 1.61345568017168e-08
3480 1.61323523428791e-08
3481 1.61324305025801e-08
3482 1.6130760727151e-08
3483 1.61283253419242e-08
3484 1.6128614888089e-08
3485 1.61240407692276e-08
3486 1.61238506990458e-08
3487 1.6122752910519e-08
3488 1.61190900627162e-08
3489 1.61198201453772e-08
3490 1.61185376157391e-08
3491 1.61159690037493e-08
3492 1.6116294077051e-08
3493 1.611446798222e-08
3494 1.61120006225701e-08
3495 1.61107038820774e-08
3496 1.61091975314775e-08
3497 1.61055275782473e-08
3498 1.61031223910868e-08
3499 1.61019571010002e-08
3500 1.61017563726773e-08
3501 1.60994968467776e-08
3502 1.60983653074709e-08
3503 1.60955142547436e-08
3504 1.60953756989102e-08
3505 1.60924074066315e-08
3506 1.60924447101252e-08
3507 1.60885154087964e-08
3508 1.60878492749816e-08
3509 1.60867301701728e-08
3510 1.60850959218806e-08
3511 1.60862878573198e-08
3512 1.60834847662272e-08
3513 1.6084880982703e-08
3514 1.60803761417583e-08
3515 1.60795536885416e-08
3516 1.6076729281167e-08
3517 1.60792588133063e-08
3518 1.61079967142541e-08
3519 1.6090488941245e-08
3520 1.60897162260198e-08
3521 1.60903770307641e-08
3522 1.60796851389478e-08
3523 1.60773740986997e-08
3524 1.60773776514134e-08
3525 1.6073421704732e-08
3526 1.60714002106488e-08
3527 1.60724731301798e-08
3528 1.60693005568646e-08
3529 1.60660071912844e-08
3530 1.60670428073217e-08
3531 1.60656838943396e-08
3532 1.60612980693031e-08
3533 1.6071320274591e-08
3534 1.60677711136259e-08
3535 1.60634190393694e-08
3536 1.60584594510738e-08
3537 1.60541233640288e-08
3538 1.60512687585879e-08
3539 1.60495350343126e-08
3540 1.60440727370315e-08
3541 1.60446536057179e-08
3542 1.60426605333441e-08
3543 1.60375979163518e-08
3544 1.60381894431794e-08
3545 1.60371449453578e-08
3546 1.60342459309959e-08
3547 1.60332103149585e-08
3548 1.60320450248719e-08
3549 1.60301691920495e-08
3550 1.60267710214157e-08
3551 1.60284940875499e-08
3552 1.60246873548431e-08
3553 1.60248188052492e-08
3554 1.60177933139494e-08
3555 1.60182498376571e-08
3556 1.6018992354816e-08
3557 1.60136206517336e-08
3558 1.60144306704524e-08
3559 1.60098032608857e-08
3560 1.60077302524542e-08
3561 1.60076911726037e-08
3562 1.60048170272376e-08
3563 1.60027724405154e-08
3564 1.60032911367125e-08
3565 1.60013637895418e-08
3566 1.59982125325087e-08
3567 1.6000310409936e-08
3568 1.59955941825274e-08
3569 1.59932067589352e-08
3570 1.59955053646854e-08
3571 1.59920325870644e-08
3572 1.59912563191256e-08
3573 1.59890127804374e-08
3574 1.59897108886753e-08
3575 1.59879522954043e-08
3576 1.5982635659384e-08
3577 1.59827280299396e-08
3578 1.59817936662421e-08
3579 1.59804187660484e-08
3580 1.59779460773279e-08
3581 1.5978255163418e-08
3582 1.59782089781402e-08
3583 1.59767044038972e-08
3584 1.59746154082541e-08
3585 1.59732920224087e-08
3586 1.59680944022966e-08
3587 1.59687907341777e-08
3588 1.59666111443357e-08
3589 1.59671511568149e-08
3590 1.59643303021539e-08
3591 1.59635380470036e-08
3592 1.59621489359552e-08
3593 1.59571467150954e-08
3594 1.59575339608864e-08
3595 1.59579673919552e-08
3596 1.59551021283733e-08
3597 1.59532405064056e-08
3598 1.59513966480063e-08
3599 1.59499968788168e-08
3600 1.59458508619537e-08
3601 1.59456163828509e-08
3602 1.59463962035034e-08
3603 1.59440158853386e-08
3604 1.59418060974303e-08
3605 1.59384985209954e-08
3606 1.59409889732842e-08
3607 1.59374060615391e-08
3608 1.59337201210974e-08
3609 1.59365125540489e-08
3610 1.59331499105519e-08
3611 1.59310307168425e-08
3612 1.59282951273099e-08
3613 1.59266626553745e-08
3614 1.59294639701102e-08
3615 1.59252113718367e-08
3616 1.59263553456412e-08
3617 1.59268953581204e-08
3618 1.59218611628376e-08
3619 1.59223301210432e-08
3620 1.59283644052266e-08
3621 1.59188573434221e-08
3622 1.59256288156939e-08
3623 1.59202446781137e-08
3624 1.59217741213524e-08
3625 1.59169459834629e-08
3626 1.59149351475207e-08
3627 1.59211435146744e-08
3628 1.59154289747221e-08
3629 1.59183919379302e-08
3630 1.59135371546881e-08
3631 1.59072257588377e-08
3632 1.59031756652439e-08
3633 1.59048845205234e-08
3634 1.59151305467731e-08
3635 1.59053161752354e-08
3636 1.59073678673849e-08
3637 1.58975588249177e-08
3638 1.58988449072694e-08
3639 1.58938124883434e-08
3640 1.58963366914122e-08
3641 1.59018647138964e-08
3642 1.58907447200818e-08
3643 1.58829482899137e-08
3644 1.58813477924014e-08
3645 1.58761626067871e-08
3646 1.58761253032935e-08
3647 1.587358688937e-08
3648 1.58760489199494e-08
3649 1.58702722075077e-08
3650 1.58670143690642e-08
3651 1.58670037109232e-08
3652 1.5865955660388e-08
3653 1.58640443004288e-08
3654 1.586285769406e-08
3655 1.58663944205273e-08
3656 1.58579087639055e-08
3657 1.58579016584781e-08
3658 1.58567381447483e-08
3659 1.58548889572785e-08
3660 1.58539918970746e-08
3661 1.58524375848401e-08
3662 1.5850160295372e-08
3663 1.58477195810747e-08
3664 1.58493715929353e-08
3665 1.58451172183049e-08
3666 1.58481512357866e-08
3667 1.58433106633993e-08
3668 1.58461102017782e-08
3669 1.58459716459447e-08
3670 1.58405644157256e-08
3671 1.58372515102201e-08
3672 1.58348232304206e-08
3673 1.58364628077834e-08
3674 1.58356066037868e-08
3675 1.58318425036441e-08
3676 1.58316293408234e-08
3677 1.58328745669678e-08
3678 1.58317057241675e-08
3679 1.58269841676884e-08
3680 1.58285633489186e-08
3681 1.58264743532754e-08
3682 1.58215094359093e-08
3683 1.58201594047114e-08
3684 1.58184434440045e-08
3685 1.58216089118923e-08
3686 1.5815839304878e-08
3687 1.58167203778703e-08
3688 1.58117927639978e-08
3689 1.5811846054703e-08
3690 1.58125690319366e-08
3691 1.58098654168271e-08
3692 1.58098316660471e-08
3693 1.58103485858874e-08
3694 1.58069859423904e-08
3695 1.58030442065638e-08
3696 1.58041313369495e-08
3697 1.58026232099928e-08
3698 1.57948853996004e-08
3699 1.57955799551246e-08
3700 1.5794624275145e-08
3701 1.57955657442699e-08
3702 1.57925157395766e-08
3703 1.57898405461765e-08
3704 1.57910093889768e-08
3705 1.57909045839233e-08
3706 1.57852877435971e-08
3707 1.57828985436481e-08
3708 1.57860924332454e-08
3709 1.57795554400764e-08
3710 1.57797881428223e-08
3711 1.57779584952777e-08
3712 1.57761697039405e-08
3713 1.57794559640934e-08
3714 1.57760791097417e-08
3715 1.57767345854154e-08
3716 1.57679078682804e-08
3717 1.57746953277638e-08
3718 1.5779548334649e-08
3719 1.57749493467918e-08
3720 1.57782746867952e-08
3721 1.57691104618607e-08
3722 1.57622839225269e-08
3723 1.57602855210826e-08
3724 1.57624064911488e-08
3725 1.57579389536977e-08
3726 1.57592641158999e-08
3727 1.57582942250656e-08
3728 1.57539812306595e-08
3729 1.57564254976705e-08
3730 1.57549688850622e-08
3731 1.5748700121776e-08
3732 1.57480677387412e-08
3733 1.5746996595567e-08
3734 1.57514445930929e-08
3735 1.574503905033e-08
3736 1.57454191906936e-08
3737 1.57446500281821e-08
3738 1.57420814161924e-08
3739 1.57384310028874e-08
3740 1.57396424782519e-08
3741 1.57341286666224e-08
3742 1.57341695228297e-08
3743 1.57316435434041e-08
3744 1.5735736269562e-08
3745 1.57301549563726e-08
3746 1.57348765128518e-08
3747 1.57396051747583e-08
3748 1.57266324407601e-08
3749 1.57289399282945e-08
3750 1.57205644057967e-08
3751 1.57210031659361e-08
3752 1.57233550623914e-08
3753 1.57224011587687e-08
3754 1.57197614925053e-08
3755 1.57157220570525e-08
3756 1.57240549469861e-08
3757 1.57229056441111e-08
3758 1.57149031565496e-08
3759 1.57082435947586e-08
3760 1.57062363115301e-08
3761 1.5708096157141e-08
3762 1.57105795040025e-08
3763 1.57045807469558e-08
3764 1.57038613224358e-08
3765 1.57038524406516e-08
3766 1.56987400856679e-08
3767 1.56992978617154e-08
3768 1.56996406985854e-08
3769 1.56938977369236e-08
3770 1.56945940688047e-08
3771 1.56951305285702e-08
3772 1.56932955519551e-08
3773 1.56903041670375e-08
3774 1.56888404490019e-08
3775 1.56893324998464e-08
3776 1.56855328725669e-08
3777 1.56801576167709e-08
3778 1.56788484417802e-08
3779 1.56826303054913e-08
3780 1.56798272143988e-08
3781 1.56773936055288e-08
3782 1.56729829114965e-08
3783 1.56752939517446e-08
3784 1.56760826541813e-08
3785 1.56738728662731e-08
3786 1.5672590336635e-08
3787 1.56717199217837e-08
3788 1.56700785680641e-08
3789 1.56687498531483e-08
3790 1.56675241669291e-08
3791 1.56637494086453e-08
3792 1.56646589033471e-08
3793 1.56568411568969e-08
3794 1.56611186241662e-08
3795 1.56606070333964e-08
3796 1.56576440701883e-08
3797 1.56567754316939e-08
3798 1.56549386787219e-08
3799 1.5655357898936e-08
3800 1.56521071659199e-08
3801 1.56511070770193e-08
3802 1.56505670645402e-08
3803 1.56478936474969e-08
3804 1.56473092260967e-08
3805 1.56388804128937e-08
3806 1.56436961162854e-08
3807 1.56426338548954e-08
3808 1.56399853068478e-08
3809 1.56406088080985e-08
3810 1.56380046689719e-08
3811 1.56361288361495e-08
3812 1.56355888236703e-08
3813 1.56325494771181e-08
3814 1.56328354705693e-08
3815 1.56277213392286e-08
3816 1.56296504627562e-08
3817 1.5628911498311e-08
3818 1.56235611115108e-08
3819 1.56208557200443e-08
3820 1.56231756420766e-08
3821 1.56207278223519e-08
3822 1.56195341105558e-08
3823 1.56172461629467e-08
3824 1.56189425837283e-08
3825 1.56165498310656e-08
3826 1.56134536410946e-08
3827 1.56143507012985e-08
3828 1.56114730032186e-08
3829 1.56101780390827e-08
3830 1.5610009285183e-08
3831 1.56080943725101e-08
3832 1.56028168163402e-08
3833 1.56035486753581e-08
3834 1.56010813157081e-08
3835 1.56026889186478e-08
3836 1.55993777894992e-08
3837 1.56011594754091e-08
3838 1.56005306450879e-08
3839 1.56003867601839e-08
3840 1.55968837844966e-08
3841 1.5595132296653e-08
3842 1.55961128456283e-08
3843 1.55952122327108e-08
3844 1.55944963609045e-08
3845 1.55890642616896e-08
3846 1.55907713406123e-08
3847 1.55858668193787e-08
3848 1.55884212205137e-08
3849 1.55837227566735e-08
3850 1.55874211316132e-08
3851 1.55851420657882e-08
3852 1.55843444815673e-08
3853 1.55778057120415e-08
3854 1.55806088031341e-08
3855 1.55793742351307e-08
3856 1.55784700694994e-08
3857 1.55756687547637e-08
3858 1.55758694830865e-08
3859 1.55749013686091e-08
3860 1.55724961814485e-08
3861 1.55721835426448e-08
3862 1.556980322448e-08
3863 1.55692667647145e-08
3864 1.55679877877901e-08
3865 1.55680623947774e-08
3866 1.55738906215674e-08
3867 1.55654049649456e-08
3868 1.55585677674708e-08
3869 1.55615893504546e-08
3870 1.55600901052821e-08
3871 1.55578305793824e-08
3872 1.55563473214215e-08
3873 1.55552886127452e-08
3874 1.55548498526059e-08
3875 1.55533950163544e-08
3876 1.55524588763001e-08
3877 1.55485953001744e-08
3878 1.55443071747641e-08
3879 1.55496344689254e-08
3880 1.55465791351617e-08
3881 1.55408699242798e-08
3882 1.55423904857344e-08
3883 1.55385553313181e-08
3884 1.55363757414761e-08
3885 1.55400847745568e-08
3886 1.55387045452926e-08
3887 1.55312847027744e-08
3888 1.55353720998619e-08
3889 1.55317767536189e-08
3890 1.55338177876274e-08
3891 1.55247352751076e-08
3892 1.55291459691398e-08
3893 1.55260018175341e-08
3894 1.55286628000795e-08
3895 1.55215236219419e-08
3896 1.55253179201509e-08
3897 1.55170383209224e-08
3898 1.55227919407253e-08
3899 1.55154786796174e-08
3900 1.55191326456361e-08
3901 1.55135673196583e-08
3902 1.5517485962846e-08
3903 1.55182728889258e-08
3904 1.55238026877669e-08
3905 1.5516711471264e-08
3906 1.55119153077976e-08
3907 1.55065666973542e-08
3908 1.55111941069208e-08
3909 1.55026143033865e-08
3910 1.55060035922361e-08
3911 1.54997543688751e-08
3912 1.54994523882124e-08
3913 1.54999071355633e-08
3914 1.55005910329464e-08
3915 1.54984221012455e-08
3916 1.55114445732352e-08
3917 1.55052841677161e-08
3918 1.54936898866254e-08
3919 1.54907855431929e-08
3920 1.54902899396347e-08
3921 1.54914783223603e-08
3922 1.54915760219865e-08
3923 1.54873696089908e-08
3924 1.54903805338336e-08
3925 1.54851935718625e-08
3926 1.54873749380613e-08
3927 1.54868100565864e-08
3928 1.54802091145712e-08
3929 1.54907446869856e-08
3930 1.5483738735611e-08
3931 1.54759245418745e-08
3932 1.54761270465542e-08
3933 1.54803885266119e-08
3934 1.54764006055075e-08
3935 1.54736969903979e-08
3936 1.5472854997256e-08
3937 1.54784274286612e-08
3938 1.54734678403656e-08
3939 1.54720964928856e-08
3940 1.54727679557709e-08
3941 1.54662753715229e-08
3942 1.54640495964031e-08
3943 1.54641650595977e-08
3944 1.54661545792578e-08
3945 1.54700749988024e-08
3946 1.54675117158831e-08
3947 1.54673216457013e-08
3948 1.54650035000259e-08
3949 1.54640478200463e-08
3950 1.54630104276521e-08
3951 1.54610155789214e-08
3952 1.54595429791016e-08
3953 1.54573562838323e-08
3954 1.54558676968009e-08
3955 1.54554768982962e-08
3956 1.54543080554959e-08
3957 1.54517767469997e-08
3958 1.54525618967227e-08
3959 1.54504604665817e-08
3960 1.54469272928281e-08
3961 1.54484016690049e-08
3962 1.54447423739157e-08
3963 1.54438506427823e-08
3964 1.54421897491375e-08
3965 1.54414632191902e-08
3966 1.54394452778206e-08
3967 1.54395358720194e-08
3968 1.54367043592174e-08
3969 1.54372425953397e-08
3970 1.54330184187756e-08
3971 1.54320716205802e-08
3972 1.54311852185174e-08
3973 1.54289256926177e-08
3974 1.54340558111699e-08
3975 1.54385251249778e-08
3976 1.54466786028706e-08
3977 1.54427546306124e-08
3978 1.54331516455386e-08
3979 1.54268082752651e-08
3980 1.54213424252703e-08
3981 1.54211097225243e-08
3982 1.54173314115269e-08
3983 1.54148658282338e-08
3984 1.54188111167741e-08
3985 1.5413469611758e-08
3986 1.54173811495184e-08
3987 1.54197046242643e-08
3988 1.54188839474045e-08
3989 1.54119081940962e-08
3990 1.54081103431736e-08
3991 1.54072488101065e-08
3992 1.54040069588746e-08
3993 1.54024082377191e-08
3994 1.54007082642238e-08
3995 1.54018398035305e-08
3996 1.53970312055662e-08
3997 1.53970685090599e-08
3998 1.53985730833028e-08
3999 1.53961625670718e-08
4000 1.53948160885875e-08
4001 1.53915511447167e-08
4002 1.5392387808788e-08
4003 1.53903965127711e-08
4004 1.53877639519351e-08
4005 1.53879931019674e-08
4006 1.53853125794967e-08
4007 1.53855541640269e-08
4008 1.53843195960235e-08
4009 1.53815431502835e-08
4010 1.53800989721731e-08
4011 1.53810333358706e-08
4012 1.53762815813252e-08
4013 1.53757522269871e-08
4014 1.53751802400848e-08
4015 1.53750292497534e-08
4016 1.53730930207985e-08
4017 1.53675614456006e-08
4018 1.53739314612267e-08
4019 1.53678492154086e-08
4020 1.53674513114765e-08
4021 1.5366262928751e-08
4022 1.53650159262497e-08
4023 1.53648596068479e-08
4024 1.53738550778826e-08
4025 1.53690677962004e-08
4026 1.53643000544434e-08
4027 1.5362280336717e-08
4028 1.5361960592486e-08
4029 1.53602410790654e-08
4030 1.53569672534104e-08
4031 1.53558623594563e-08
4032 1.53545105519015e-08
4033 1.53553383341887e-08
4034 1.5353244009475e-08
4035 1.53532102586951e-08
4036 1.53500163690978e-08
4037 1.53480694820018e-08
4038 1.53501407140766e-08
4039 1.53459573937198e-08
4040 1.53462416108141e-08
4041 1.53433070693154e-08
4042 1.53445132156094e-08
4043 1.53392942792152e-08
4044 1.53394257296213e-08
4045 1.53359351884319e-08
4046 1.53370098843197e-08
4047 1.53328940655229e-08
4048 1.53363046706545e-08
4049 1.53346206843707e-08
4050 1.53299257732442e-08
4051 1.53293449045577e-08
4052 1.53301797922722e-08
4053 1.53269628100361e-08
4054 1.53255435009214e-08
4055 1.53269184011151e-08
4056 1.53222821097643e-08
4057 1.5324109980952e-08
4058 1.53226924481942e-08
4059 1.53191805907227e-08
4060 1.53175587769283e-08
4061 1.53187933449317e-08
4062 1.53171235695027e-08
4063 1.53147450276947e-08
4064 1.531092586049e-08
4065 1.5313503354264e-08
4066 1.53081405329658e-08
4067 1.53082684306582e-08
4068 1.53077230891085e-08
4069 1.53036499028758e-08
4070 1.53046997297679e-08
4071 1.52993067104035e-08
4072 1.53003316682998e-08
4073 1.52994097391002e-08
4074 1.52980152989812e-08
4075 1.52935513142438e-08
4076 1.52971129097068e-08
4077 1.52925974106211e-08
4078 1.52915635709405e-08
4079 1.52897623451054e-08
4080 1.52877834835863e-08
4081 1.52878367742915e-08
4082 1.52890322624444e-08
4083 1.52846926226857e-08
4084 1.5284390642023e-08
4085 1.52815804455031e-08
4086 1.52818291354606e-08
4087 1.52780987860979e-08
4088 1.52805288422542e-08
4089 1.52751695736697e-08
4090 1.52759760396748e-08
4091 1.52721000290512e-08
4092 1.5276537368436e-08
4093 1.52714996204395e-08
4094 1.52720502910597e-08
4095 1.52704728861863e-08
4096 1.52712811285483e-08
4097 1.52683536924769e-08
4098 1.52703645284191e-08
4099 1.52663304220368e-08
4100 1.52705350586757e-08
4101 1.52638062189681e-08
4102 1.5262269670302e-08
4103 1.52636356887115e-08
4104 1.52600421188254e-08
4105 1.52582533274881e-08
4106 1.52587045221253e-08
4107 1.5255654517432e-08
4108 1.52573260692179e-08
4109 1.52502970252044e-08
4110 1.52518051521611e-08
4111 1.52504959771704e-08
4112 1.52485455373608e-08
4113 1.52660692975815e-08
4114 1.52649093365653e-08
4115 1.52619907822782e-08
4116 1.52584576085246e-08
4117 1.52576067335985e-08
4118 1.52590562407795e-08
4119 1.5257104024613e-08
4120 1.52615271531431e-08
4121 1.52562709132553e-08
4122 1.5250622098506e-08
4123 1.52506896000659e-08
4124 1.52512207307609e-08
4125 1.52484940230124e-08
4126 1.52500714278858e-08
4127 1.52492045657482e-08
4128 1.52601877800862e-08
4129 1.52436783196208e-08
4130 1.52481085535783e-08
4131 1.52420316368307e-08
4132 1.52352885862683e-08
4133 1.52386050444875e-08
4134 1.52388768270839e-08
4135 1.5238400763451e-08
4136 1.52488848215171e-08
4137 1.52379211471043e-08
4138 1.52302437328444e-08
4139 1.5232847871971e-08
4140 1.52310626333474e-08
4141 1.52315777768308e-08
4142 1.52287622512404e-08
4143 1.52281902643381e-08
4144 1.5225735339186e-08
4145 1.52245362983194e-08
4146 1.5237228367937e-08
4147 1.52233710082328e-08
4148 1.52254884255854e-08
4149 1.52220689386695e-08
4150 1.52190722246814e-08
4151 1.52146775178608e-08
4152 1.52114907336909e-08
4153 1.52150576582244e-08
4154 1.5217816340396e-08
4155 1.52187737967324e-08
4156 1.52196832914342e-08
4157 1.52130130715022e-08
4158 1.52083821092219e-08
4159 1.5205660730544e-08
4160 1.52075863013579e-08
4161 1.5204200565222e-08
4162 1.52022394672713e-08
4163 1.51966510486545e-08
4164 1.52052415103299e-08
4165 1.51997383568414e-08
4166 1.51899630651542e-08
4167 1.51973384987514e-08
4168 1.51933114977965e-08
4169 1.51920485080836e-08
4170 1.51958765570726e-08
4171 1.51850443330659e-08
4172 1.51920804825068e-08
4173 1.51855825691882e-08
4174 1.5188438950986e-08
4175 1.51830459316216e-08
4176 1.51871457632069e-08
4177 1.51818539961823e-08
4178 1.51751393673294e-08
4179 1.51813210891305e-08
4180 1.51784238511254e-08
4181 1.51756367472444e-08
4182 1.51759689259734e-08
4183 1.51739012466123e-08
4184 1.51743790866021e-08
4185 1.51692276517679e-08
4186 1.51708388074212e-08
4187 1.51718015928282e-08
4188 1.51665648928656e-08
4189 1.51724961483524e-08
4190 1.51598626985106e-08
4191 1.51676147197577e-08
4192 1.51647991941672e-08
4193 1.51618610999549e-08
4194 1.51603636311393e-08
4195 1.51633781086957e-08
4196 1.51635770606617e-08
4197 1.51616514898478e-08
4198 1.51625858535454e-08
4199 1.51607970622081e-08
4200 1.51572745465955e-08
4201 1.51557930649915e-08
4202 1.5150975585243e-08
4203 1.51552459470849e-08
4204 1.51560684003016e-08
4205 1.51542263182591e-08
4206 1.51493431133076e-08
4207 1.51547716598088e-08
4208 1.51462344888387e-08
4209 1.5145301901498e-08
4210 1.51438079853961e-08
4211 1.51463535047469e-08
4212 1.51410155524445e-08
4213 1.51386423397071e-08
4214 1.51418877436527e-08
4215 1.51453320995643e-08
4216 1.51365000533588e-08
4217 1.51369548007096e-08
4218 1.51338994669459e-08
4219 1.51379069279756e-08
4220 1.51309933471566e-08
4221 1.51317092189629e-08
4222 1.51321071228949e-08
4223 1.51306878137802e-08
4224 1.51290393546333e-08
4225 1.51318335639417e-08
4226 1.51297800954353e-08
4227 1.51281032145789e-08
4228 1.51278172211278e-08
4229 1.51234207379503e-08
4230 1.51256482894269e-08
4231 1.51239358814337e-08
4232 1.51236054790616e-08
4233 1.51197365738653e-08
4234 1.51197809827863e-08
4235 1.51154520011687e-08
4236 1.51176955398569e-08
4237 1.51163934702936e-08
4238 1.5112407325546e-08
4239 1.51101691159283e-08
4240 1.51095047584704e-08
4241 1.51139403214984e-08
4242 1.51068864084891e-08
4243 1.51051189334339e-08
4244 1.51106309687066e-08
4245 1.51059982300694e-08
4246 1.51021577465826e-08
4247 1.51004080350958e-08
4248 1.51045949081663e-08
4249 1.50995624892403e-08
4250 1.51014294402785e-08
4251 1.51007721882479e-08
4252 1.50982479851791e-08
4253 1.51039518669904e-08
4254 1.51081991361934e-08
4255 1.50985552949123e-08
4256 1.50974326373898e-08
4257 1.50961341205402e-08
4258 1.50951553479217e-08
4259 1.50946455335088e-08
4260 1.50944234889039e-08
4261 1.50931978026847e-08
4262 1.50960630662667e-08
4263 1.50931320774816e-08
4264 1.5092220806423e-08
4265 1.50932866205267e-08
4266 1.50938817000679e-08
4267 1.50895509420934e-08
4268 1.50876804383415e-08
4269 1.50892027761529e-08
4270 1.5087088911514e-08
4271 1.50842041080068e-08
4272 1.50860675063313e-08
4273 1.50828096678879e-08
4274 1.50849910340867e-08
4275 1.50807082377469e-08
4276 1.50819037258998e-08
4277 1.50791148456619e-08
4278 1.50781733765371e-08
4279 1.507655866817e-08
4280 1.50770844697945e-08
4281 1.50734482673442e-08
4282 1.50741268356569e-08
4283 1.5072311398967e-08
4284 1.50725423253562e-08
4285 1.50691850109297e-08
4286 1.50695651512933e-08
4287 1.50685171007581e-08
4288 1.50660195430419e-08
4289 1.50691032985151e-08
4290 1.50663179709909e-08
4291 1.50647370134038e-08
4292 1.50616727978559e-08
4293 1.50624170913716e-08
4294 1.50587720071371e-08
4295 1.5057235458471e-08
4296 1.50603174375874e-08
4297 1.50532155629435e-08
4298 1.50544785526563e-08
4299 1.5053265300935e-08
4300 1.50496859419036e-08
4301 1.5047712409455e-08
4302 1.5046197177071e-08
4303 1.5050256152449e-08
4304 1.50466430426377e-08
4305 1.50466341608535e-08
4306 1.50427492684457e-08
4307 1.50424188660736e-08
4308 1.50426870959564e-08
4309 1.50405430332512e-08
4310 1.50400936149708e-08
4311 1.50418362210303e-08
4312 1.50370986773396e-08
4313 1.50366545881297e-08
4314 1.50369441342946e-08
4315 1.50342813753923e-08
4316 1.50352743588655e-08
4317 1.50317873703898e-08
4318 1.50344234839395e-08
4319 1.50286592059956e-08
4320 1.50319454661485e-08
4321 1.50296308731868e-08
4322 1.50296148859752e-08
4323 1.50226888706584e-08
4324 1.50250514252548e-08
4325 1.50226728834468e-08
4326 1.5021786481384e-08
4327 1.50208041560518e-08
4328 1.50201291404528e-08
4329 1.50197791981554e-08
4330 1.5020415133904e-08
4331 1.50194665593517e-08
4332 1.50159298328845e-08
4333 1.50172887458666e-08
4334 1.50124339626245e-08
4335 1.5016023979797e-08
4336 1.50134749077324e-08
4337 1.50133860898904e-08
4338 1.5006909492854e-08
4339 1.50081014282932e-08
4340 1.5010526155379e-08
4341 1.50100980533807e-08
4342 1.5008181364351e-08
4343 1.50094052742133e-08
4344 1.50035592838549e-08
4345 1.50062149373298e-08
4346 1.5002141751097e-08
4347 1.5000471975668e-08
4348 1.50004648702406e-08
4349 1.49992249731667e-08
4350 1.49967647189442e-08
4351 1.49960683870631e-08
4352 1.49984451525143e-08
4353 1.49952672501286e-08
4354 1.49900625245891e-08
4355 1.49917873670802e-08
4356 1.49885774902714e-08
4357 1.49834242790803e-08
4358 1.49872736443513e-08
4359 1.49854653130888e-08
4360 1.4983404739155e-08
4361 1.49870071908254e-08
4362 1.49810706062681e-08
4363 1.49779122438076e-08
4364 1.49799834758824e-08
4365 1.49784415981458e-08
4366 1.49787364733811e-08
4367 1.49776244739996e-08
4368 1.49777363844805e-08
4369 1.49747361177788e-08
4370 1.49740539967524e-08
4371 1.49711851804568e-08
4372 1.4971094586258e-08
4373 1.49714889374764e-08
4374 1.49692453987882e-08
4375 1.49689238782003e-08
4376 1.49630103862819e-08
4377 1.49618486489089e-08
4378 1.49682204408919e-08
4379 1.4963186245609e-08
4380 1.4962788341677e-08
4381 1.49592622733508e-08
4382 1.49611771860236e-08
4383 1.49578713859455e-08
4384 1.49594168163958e-08
4385 1.49528425197332e-08
4386 1.49570222873763e-08
4387 1.4955322313881e-08
4388 1.49517216385675e-08
4389 1.49511087954579e-08
4390 1.49491317102957e-08
4391 1.49469290278148e-08
4392 1.49464671750366e-08
4393 1.49473304844605e-08
4394 1.49439198793289e-08
4395 1.49463765808378e-08
4396 1.49419676631624e-08
4397 1.49363632573341e-08
4398 1.49445984476415e-08
4399 1.49391414794309e-08
4400 1.49390420034479e-08
4401 1.49385073200392e-08
4402 1.49341872202058e-08
4403 1.49322918474581e-08
4404 1.49354519862754e-08
4405 1.49319312470197e-08
4406 1.49302401553086e-08
4407 1.49327181730996e-08
4408 1.49275987126885e-08
4409 1.49278776007122e-08
4410 1.49241703439884e-08
4411 1.49238612578984e-08
4412 1.49239145486035e-08
4413 1.49235912516588e-08
4414 1.49205039434719e-08
4415 1.49202659116554e-08
4416 1.49165710894295e-08
4417 1.49210599431626e-08
4418 1.49151109241075e-08
4419 1.49175907182553e-08
4420 1.49134358196079e-08
4421 1.49117624914652e-08
4422 1.49108849711865e-08
4423 1.49103431823505e-08
4424 1.49098173807261e-08
4425 1.49054386611169e-08
4426 1.49111727409945e-08
4427 1.49067904686717e-08
4428 1.4904641076896e-08
4429 1.49026710971611e-08
4430 1.49029375506871e-08
4431 1.48982257641705e-08
4432 1.48988812398443e-08
4433 1.48971448510338e-08
4434 1.49007464145257e-08
4435 1.4897589828422e-08
4436 1.48961696311289e-08
4437 1.48951135869879e-08
4438 1.48934029553516e-08
4439 1.48935956900687e-08
4440 1.489095335927e-08
4441 1.48897845164697e-08
4442 1.48890357820619e-08
4443 1.48879584216388e-08
4444 1.4888371424604e-08
4445 1.48870871186091e-08
4446 1.48852112857867e-08
4447 1.48825076706771e-08
4448 1.4881358367802e-08
4449 1.48784291553739e-08
4450 1.48756695850238e-08
4451 1.48770649133212e-08
4452 1.48791263754333e-08
4453 1.48715431080859e-08
4454 1.48788190657001e-08
4455 1.4872383324871e-08
4456 1.48660648235932e-08
4457 1.48710874725566e-08
4458 1.48675391997699e-08
4459 1.48669210275898e-08
4460 1.48654821785499e-08
4461 1.48623637841183e-08
4462 1.48665515453672e-08
4463 1.48606718042288e-08
4464 1.48620662443477e-08
4465 1.48584202719348e-08
4466 1.48565106883325e-08
4467 1.48562042667777e-08
4468 1.48508458863716e-08
4469 1.48575347580504e-08
4470 1.48508583208695e-08
4471 1.48504453179044e-08
4472 1.48489052165246e-08
4473 1.48482612871703e-08
4474 1.48462779847591e-08
4475 1.48499248453504e-08
4476 1.48412127032316e-08
4477 1.48412402367626e-08
4478 1.48470746808016e-08
4479 1.48451677617345e-08
4480 1.48358010321203e-08
4481 1.48414471823344e-08
4482 1.48365213448187e-08
4483 1.48366741115069e-08
4484 1.48354688533914e-08
4485 1.48344208028561e-08
4486 1.4831242012292e-08
4487 1.48297827351485e-08
4488 1.48278873624008e-08
4489 1.48275596245639e-08
4490 1.48263969990126e-08
4491 1.48257939258656e-08
4492 1.48234402530534e-08
4493 1.48202010663567e-08
4494 1.48223389118129e-08
4495 1.48218948226031e-08
4496 1.48209089445572e-08
4497 1.48212366823941e-08
4498 1.48182284220866e-08
4499 1.48168535218929e-08
4500 1.48103156405455e-08
4501 1.48129144506015e-08
4502 1.48113121767324e-08
4503 1.48101237940068e-08
4504 1.48114169817859e-08
4505 1.48067007543773e-08
4506 1.48104719599473e-08
4507 1.48051393367155e-08
4508 1.48049572601394e-08
4509 1.48028407309653e-08
4510 1.48001015887189e-08
4511 1.4801566194933e-08
4512 1.47951393358881e-08
4513 1.47981671361208e-08
4514 1.47967993413545e-08
4515 1.47946233042262e-08
4516 1.47946028761226e-08
4517 1.47904817282551e-08
4518 1.47913414849654e-08
4519 1.47900509617216e-08
4520 1.47897578628431e-08
4521 1.47882195378202e-08
4522 1.47874903433376e-08
4523 1.47827448060411e-08
4524 1.4783387847217e-08
4525 1.47906815683996e-08
4526 1.47838221664642e-08
4527 1.47800962579936e-08
4528 1.47799124050607e-08
4529 1.47776155756674e-08
4530 1.47763206115314e-08
4531 1.47741978651084e-08
4532 1.47753578261245e-08
4533 1.47743159928382e-08
4534 1.47729561916776e-08
4535 1.47690348839546e-08
4536 1.47658418825358e-08
4537 1.47683989482061e-08
4538 1.47646925796607e-08
4539 1.47648755444152e-08
4540 1.47648506754194e-08
4541 1.47634278135911e-08
4542 1.47604461986361e-08
4543 1.47589744869947e-08
4544 1.47582639442589e-08
4545 1.47558401053516e-08
4546 1.47543017803287e-08
4547 1.47524392701825e-08
4548 1.47535281769251e-08
4549 1.47509195969064e-08
4550 1.47479903844783e-08
4551 1.47470506917102e-08
4552 1.47461243216185e-08
4553 1.47466927558071e-08
4554 1.47455532228946e-08
4555 1.47447218878938e-08
4556 1.47446339582302e-08
4557 1.4741285525588e-08
4558 1.47384140447571e-08
4559 1.4746897925022e-08
4560 1.47373606651513e-08
4561 1.47375924797188e-08
4562 1.47355008195404e-08
4563 1.4734625963797e-08
4564 1.4730998643131e-08
4565 1.47320733390188e-08
4566 1.47286272067504e-08
4567 1.4730241026939e-08
4568 1.4726374786278e-08
4569 1.47266092653808e-08
4570 1.47242937842407e-08
4571 1.47237662062594e-08
4572 1.47204781697496e-08
4573 1.4724207630934e-08
4574 1.4721486252256e-08
4575 1.47198928601711e-08
4576 1.47167931174863e-08
4577 1.47172736220114e-08
4578 1.47170657882612e-08
4579 1.47137457773283e-08
4580 1.47141481221524e-08
4581 1.47140077899621e-08
4582 1.47115084558891e-08
4583 1.47105385650548e-08
4584 1.47102880987404e-08
4585 1.47120999827166e-08
4586 1.47066288036513e-08
4587 1.47073917489138e-08
4588 1.47058454302851e-08
4589 1.47029552977074e-08
4590 1.47035219555391e-08
4591 1.47013254903072e-08
4592 1.47004799444517e-08
4593 1.4700370698506e-08
4594 1.46975596138077e-08
4595 1.46970906556021e-08
4596 1.46952752189122e-08
4597 1.46954794999488e-08
4598 1.46944048040609e-08
4599 1.46920884347423e-08
4600 1.46897960462411e-08
4601 1.46916310228562e-08
4602 1.46875160922377e-08
4603 1.4686435179101e-08
4604 1.46882079832267e-08
4605 1.46848577742276e-08
4606 1.46855088090092e-08
4607 1.46822882740594e-08
4608 1.46818628365963e-08
4609 1.46822607405284e-08
4610 1.46792933364281e-08
4611 1.46799941092013e-08
4612 1.46762895170127e-08
4613 1.46782035415072e-08
4614 1.46758232233424e-08
4615 1.4672857595599e-08
4616 1.46741587769839e-08
4617 1.46732537231742e-08
4618 1.46711300885727e-08
4619 1.46700056546933e-08
4620 1.46673588830026e-08
4621 1.46679477452949e-08
4622 1.46648782006764e-08
4623 1.46655123600681e-08
4624 1.46624525854122e-08
4625 1.46641561116212e-08
4626 1.46602081585456e-08
4627 1.46619809626714e-08
4628 1.46595278138761e-08
4629 1.46565239944607e-08
4630 1.46559999691931e-08
4631 1.46563863268057e-08
4632 1.46541356826901e-08
4633 1.46527066036128e-08
4634 1.46513317034191e-08
4635 1.46506931031354e-08
4636 1.46478589257981e-08
4637 1.46481031748635e-08
4638 1.4648546375895e-08
4639 1.46476635265458e-08
4640 1.4645200607788e-08
4641 1.4642218992833e-08
4642 1.46427696634532e-08
4643 1.46413858814753e-08
4644 1.46403573708653e-08
4645 1.46399212752613e-08
4646 1.46375329634907e-08
4647 1.46375036536028e-08
4648 1.46357947983233e-08
4649 1.46354333097065e-08
4650 1.46329455219529e-08
4651 1.46338834383641e-08
4652 1.46321763594415e-08
4653 1.46305856318918e-08
4654 1.46290188851594e-08
4655 1.4628160904806e-08
4656 1.46263632316845e-08
4657 1.46257423949692e-08
4658 1.46250931365444e-08
4659 1.46235645814841e-08
4660 1.46225014319157e-08
4661 1.46218965824119e-08
4662 1.461898069266e-08
4663 1.46181982074722e-08
4664 1.46166572179141e-08
4665 1.46175382909064e-08
4666 1.46156784452955e-08
4667 1.4614390586587e-08
4668 1.46124756739141e-08
4669 1.46123424471511e-08
4670 1.46109098153602e-08
4671 1.46092604680348e-08
4672 1.46098493303271e-08
4673 1.46070613382676e-08
4674 1.46056011729456e-08
4675 1.46054004446228e-08
4676 1.46028877878734e-08
4677 1.4601388542701e-08
4678 1.46013681145973e-08
4679 1.46009515589185e-08
4680 1.45995509015506e-08
4681 1.45998475531428e-08
4682 1.45973455545345e-08
4683 1.45943879203969e-08
4684 1.45944696328115e-08
4685 1.45929357486807e-08
4686 1.45919551997054e-08
4687 1.45912188997954e-08
4688 1.45904639481387e-08
4689 1.45884273550223e-08
4690 1.45860852285296e-08
4691 1.45871723589153e-08
4692 1.45846899002322e-08
4693 1.45837528719994e-08
4694 1.45816105856511e-08
4695 1.45819543106995e-08
4696 1.45804461837429e-08
4697 1.45802561135611e-08
4698 1.45783740634897e-08
4699 1.45803609186146e-08
4700 1.45744731838704e-08
4701 1.45789389449646e-08
4702 1.45752707680913e-08
4703 1.45753933367132e-08
4704 1.45767478088032e-08
4705 1.45773793036597e-08
4706 1.45770471249307e-08
4707 1.45741552159961e-08
4708 1.45743141999333e-08
4709 1.45720706612451e-08
4710 1.4571463147206e-08
4711 1.45654004413132e-08
4712 1.45657024219759e-08
4713 1.45627510050872e-08
4714 1.45626932734899e-08
4715 1.4568244388613e-08
4716 1.45657912398178e-08
4717 1.45643523907779e-08
4718 1.45577896404347e-08
4719 1.45630920656004e-08
4720 1.45622136571433e-08
4721 1.45586787070329e-08
4722 1.45777443449902e-08
4723 1.45575258514441e-08
4724 1.45576555254934e-08
4725 1.4578072082827e-08
4726 1.4554878191575e-08
4727 1.45689273978178e-08
4728 1.45450425037552e-08
4729 1.45457681455241e-08
4730 1.45512393245895e-08
4731 1.45487932812216e-08
4732 1.45488154856821e-08
4733 1.45711718246844e-08
4734 1.45402179185794e-08
4735 1.45456082734086e-08
4736 1.45382488270229e-08
4737 1.45409293494936e-08
4738 1.45409613239167e-08
4739 1.45585579147678e-08
4740 1.4534160541757e-08
4741 1.45387817340747e-08
4742 1.45369920545591e-08
4743 1.45286351838081e-08
4744 1.45290925956942e-08
4745 1.45276066731981e-08
4746 1.45568224141357e-08
4747 1.45244429816671e-08
4748 1.45252441186017e-08
4749 1.45523255667968e-08
4750 1.45276208840528e-08
4751 1.45500100856566e-08
4752 1.45406469087561e-08
4753 1.45241338955771e-08
4754 1.45179930299832e-08
4755 1.45215306446289e-08
4756 1.45209178015193e-08
4757 1.45202418977419e-08
4758 1.45115315319799e-08
4759 1.45344802859881e-08
4760 1.45171918930487e-08
4761 1.45149918751031e-08
4762 1.45136489493325e-08
4763 1.45136738183282e-08
4764 1.45126248796146e-08
4765 1.4528271030656e-08
4766 1.45098093540241e-08
4767 1.45091814118814e-08
4768 1.45070462309604e-08
4769 1.45065364165475e-08
4770 1.45279797081344e-08
4771 1.45270213636195e-08
4772 1.45025316200531e-08
4773 1.45028362652511e-08
4774 1.44945788704831e-08
4775 1.45000296214448e-08
4776 1.44984477756793e-08
4777 1.45210021784692e-08
4778 1.44964982240481e-08
4779 1.44890170972189e-08
4780 1.44937777335485e-08
4781 1.44930583090286e-08
4782 1.44920191402775e-08
4783 1.45088723257913e-08
4784 1.44840468507823e-08
4785 1.44899479082028e-08
4786 1.44869680696047e-08
4787 1.44864911177933e-08
4788 1.45093199677149e-08
4789 1.44839740201519e-08
4790 1.45007925667073e-08
4791 1.4476142062847e-08
4792 1.44812899449676e-08
4793 1.44794496392819e-08
4794 1.44953187231067e-08
4795 1.44779086497238e-08
4796 1.44761225229217e-08
4797 1.4475350695875e-08
4798 1.44744447538869e-08
4799 1.4472508524932e-08
4800 1.44948755220753e-08
4801 1.44694194403883e-08
4802 1.4464970554684e-08
4803 1.44689709102863e-08
4804 1.44674467961181e-08
4805 1.44831062698358e-08
4806 1.44901042276047e-08
4807 1.44638372390204e-08
4808 1.44634864085447e-08
4809 1.44646170596729e-08
4810 1.44624685560757e-08
4811 1.44805252233482e-08
4812 1.44548675251599e-08
4813 1.44591645323544e-08
4814 1.44807943414094e-08
4815 1.4458450436905e-08
4816 1.44503893295678e-08
4817 1.44520777567436e-08
4818 1.44532563695066e-08
4819 1.44522456224649e-08
4820 1.44740752716643e-08
4821 1.44449359140708e-08
4822 1.44620733166789e-08
4823 1.44499763266026e-08
4824 1.44473428775882e-08
4825 1.44452112493809e-08
4826 1.44389904477293e-08
4827 1.44588438999449e-08
4828 1.44475640340147e-08
4829 1.44674165980518e-08
4830 1.44451046679706e-08
4831 1.44437573013079e-08
4832 1.44621958853008e-08
4833 1.44636427279465e-08
4834 1.44415803760012e-08
4835 1.44417660052909e-08
4836 1.44580720728982e-08
4837 1.44393776935203e-08
4838 1.44606007168591e-08
4839 1.44609622054759e-08
4840 1.44583403027809e-08
4841 1.44371892218942e-08
4842 1.44336445018212e-08
4843 1.44521417055898e-08
4844 1.44498537579807e-08
4845 1.44310892125077e-08
4846 1.44303431426351e-08
4847 1.44305198901407e-08
4848 1.44501832721744e-08
4849 1.44515901467912e-08
4850 1.44492799947216e-08
4851 1.44460825524106e-08
4852 1.44279983516071e-08
4853 1.44251579570209e-08
4854 1.44450176264854e-08
4855 1.44469405327641e-08
4856 1.44446552496902e-08
4857 1.44239225008391e-08
4858 1.44410527980199e-08
4859 1.44226026677075e-08
4860 1.44421017367335e-08
4861 1.44199470142325e-08
4862 1.44394300960471e-08
4863 1.44190837048086e-08
4864 1.4436710493726e-08
4865 1.44148009084688e-08
4866 1.44323006878722e-08
4867 1.44137564106472e-08
4868 1.4412266047259e-08
4869 1.44317215955425e-08
4870 1.44308884841848e-08
4871 1.44301930404822e-08
4872 1.44268099688816e-08
4873 1.44079006503262e-08
4874 1.44281964153947e-08
4875 1.44243506028374e-08
4876 1.44246001809734e-08
4877 1.44042253680254e-08
4878 1.44225120735086e-08
4879 1.44219587383532e-08
4880 1.44015581682311e-08
4881 1.43996325974172e-08
4882 1.4419090810236e-08
4883 1.44162841664297e-08
4884 1.43979281830298e-08
4885 1.43934117957656e-08
4886 1.43930760643229e-08
4887 1.4412691484722e-08
4888 1.43916185635362e-08
4889 1.44104044252913e-08
4890 1.43883553960222e-08
4891 1.44099718824009e-08
4892 1.44094549625606e-08
4893 1.43860656720562e-08
4894 1.44048080130688e-08
4895 1.43840157562636e-08
4896 1.43822269649263e-08
4897 1.44020964043534e-08
4898 1.43786529349654e-08
4899 1.43999763224656e-08
4900 1.43791920592662e-08
4901 1.43968037491504e-08
4902 1.43978544642209e-08
4903 1.4398660930226e-08
4904 1.43764360416299e-08
4905 1.43925360518438e-08
4906 1.4370339584957e-08
4907 1.43866207835686e-08
4908 1.43906131455651e-08
4909 1.4388380265018e-08
4910 1.43878429170741e-08
4911 1.43658587248296e-08
4912 1.43843639222041e-08
4913 1.43852734169059e-08
4914 1.43820644282755e-08
4915 1.43806220265219e-08
4916 1.43605332070251e-08
4917 1.43775720218287e-08
4918 1.43764751214803e-08
4919 1.43744136593682e-08
4920 1.43714613543011e-08
4921 1.43776768268822e-08
4922 1.43742360236843e-08
4923 1.43729863566477e-08
4924 1.43707579169927e-08
4925 1.43697471699511e-08
4926 1.43708884792204e-08
4927 1.43671670116419e-08
4928 1.43687213238763e-08
4929 1.43678366981703e-08
4930 1.43661171847498e-08
4931 1.4364240463749e-08
4932 1.43674832031593e-08
4933 1.43604532709674e-08
4934 1.43645353389843e-08
4935 1.4362376177246e-08
4936 1.43648666295348e-08
4937 1.43594078849674e-08
4938 1.43556739828909e-08
4939 1.43556668774636e-08
4940 1.43530547447313e-08
4941 1.43523779527754e-08
4942 1.43549909736862e-08
4943 1.43501273086599e-08
4944 1.43527429941059e-08
4945 1.4348832344524e-08
4946 1.43508920302793e-08
4947 1.43502818517049e-08
4948 1.4345630461321e-08
4949 1.43445149092258e-08
4950 1.43436000854535e-08
4951 1.43477336678188e-08
4952 1.43448612988095e-08
4953 1.43409355501944e-08
4954 1.43426870380381e-08
4955 1.43391902795997e-08
4956 1.43375311623117e-08
4957 1.43368961147416e-08
4958 1.43367504534808e-08
4959 1.433969032405e-08
4960 1.433278296048e-08
4961 1.43333771518428e-08
4962 1.43314178302489e-08
4963 1.43332403723662e-08
4964 1.43289984322337e-08
4965 1.43283154230289e-08
4966 1.43272229635727e-08
4967 1.43314951017715e-08
4968 1.43287479659193e-08
4969 1.43257681273212e-08
4970 1.43233602756254e-08
4971 1.43213370051853e-08
4972 1.43223184423391e-08
4973 1.43213263470443e-08
4974 1.43198883861828e-08
4975 1.43215945769271e-08
4976 1.4317662611063e-08
4977 1.43155594045652e-08
4978 1.43185197032381e-08
4979 1.4313233265284e-08
4980 1.43120315598821e-08
4981 1.43144465170053e-08
4982 1.43117357964684e-08
4983 1.43100216121184e-08
4984 1.43080605141677e-08
4985 1.43121745566077e-08
4986 1.43117784290325e-08
4987 1.43103253691379e-08
4988 1.43074592173775e-08
4989 1.43075897796052e-08
4990 1.43031355648304e-08
4991 1.42791973800627e-08
4992 1.43066474223019e-08
4993 1.43033931365721e-08
4994 1.43074574410207e-08
4995 1.43025156162935e-08
4996 1.4302106166042e-08
4997 1.43074840863733e-08
4998 1.43014151632315e-08
4999 1.4300590933658e-08
};
\addlegendentry{Test}

\nextgroupplot[
title={ELU/SiLU $\rare$},
ymin=9.70583971209021e-09, ymax=1e-05,
]
\addplot [semithick, black, dashed]
table {%
0 0.00372863043990219
1 0.000408147071344501
2 0.000196447443246143
3 0.000188726544101883
4 0.000174258241869211
5 0.000110782021297382
6 3.45153858122558e-05
7 2.64351610895517e-05
8 2.33235597222574e-05
9 1.87566699617037e-05
10 1.24356932048215e-05
11 7.73688917409743e-06
12 6.38196917361711e-06
13 6.08756599018534e-06
14 5.9410869947456e-06
15 5.82102502512072e-06
16 5.71018875355378e-06
17 5.60303749225e-06
18 5.49489559031002e-06
19 5.38224300532164e-06
20 5.26112303563764e-06
21 5.12632605330765e-06
22 4.97061478365879e-06
23 4.7878761730189e-06
24 4.57214423662578e-06
25 4.31665689676919e-06
26 4.01731862540089e-06
27 3.67530398131066e-06
28 3.30092752500377e-06
29 2.91523232732516e-06
30 2.54891511710653e-06
31 2.23435264405225e-06
32 1.99089278152442e-06
33 1.81922558538972e-06
34 1.70580063245396e-06
35 1.63424627169206e-06
36 1.58981749192577e-06
37 1.56236189154058e-06
38 1.54405299067406e-06
39 1.53065294329124e-06
40 1.51963060792326e-06
41 1.50997599271818e-06
42 1.50122665431596e-06
43 1.49313196189738e-06
44 1.48554999753614e-06
45 1.47839432919028e-06
46 1.47160381994382e-06
47 1.46514096409334e-06
48 1.45901726944331e-06
49 1.45317260187738e-06
50 1.44758224642771e-06
51 1.44222585699083e-06
52 1.43708261748898e-06
53 1.43213182968083e-06
54 1.42734478422213e-06
55 1.42269521190741e-06
56 1.41815204843354e-06
57 1.41370246774031e-06
58 1.40933612786753e-06
59 1.40506093914539e-06
60 1.40085502964382e-06
61 1.3965753715155e-06
62 1.39221399083311e-06
63 1.38780749290035e-06
64 1.38333144930058e-06
65 1.37871523563859e-06
66 1.37386239985915e-06
67 1.36873761046274e-06
68 1.36333756404738e-06
69 1.35759506523669e-06
70 1.35141783488635e-06
71 1.34473790478751e-06
72 1.33751763855372e-06
73 1.32962543058213e-06
74 1.32093904915109e-06
75 1.31136337291338e-06
76 1.30074115678269e-06
77 1.28887688582324e-06
78 1.27559497279783e-06
79 1.26071177810516e-06
80 1.24402304397186e-06
81 1.2253173726986e-06
82 1.20437622419445e-06
83 1.18100701180879e-06
84 1.15503014684037e-06
85 1.12631966068122e-06
86 1.09498568204458e-06
87 1.06111967172495e-06
88 1.02498616030289e-06
89 9.87124024920405e-07
90 9.48386010300339e-07
91 9.09669705137617e-07
92 8.72124964487853e-07
93 8.36701352472602e-07
94 8.04088519050339e-07
95 7.7491722006684e-07
96 7.49372779965896e-07
97 7.27044804161636e-07
98 7.0799477238559e-07
99 6.91812034482098e-07
100 6.77968547863728e-07
101 6.6598512997551e-07
102 6.55530762406542e-07
103 6.46288243578042e-07
104 6.38106228267432e-07
105 6.30828311678044e-07
106 6.24326076877679e-07
107 6.18493007698007e-07
108 6.13236507700066e-07
109 6.0847379724116e-07
110 6.0413567966755e-07
111 6.00160854235199e-07
112 5.96497921906192e-07
113 5.93101392460937e-07
114 5.8993454718248e-07
115 5.86965969155528e-07
116 5.84167221875376e-07
117 5.81519191459989e-07
118 5.79001233553811e-07
119 5.7660111811586e-07
120 5.74303022734313e-07
121 5.7209925702395e-07
122 5.69978981831198e-07
123 5.67935502985506e-07
124 5.65961023809791e-07
125 5.64051106145769e-07
126 5.62199152245313e-07
127 5.60401928739296e-07
128 5.58656335860164e-07
129 5.56960190401412e-07
130 5.55310402749143e-07
131 5.53702421338897e-07
132 5.52134726252262e-07
133 5.50606313009183e-07
134 5.4911738109098e-07
135 5.47667423669296e-07
136 5.46256030700576e-07
137 5.44879484504435e-07
138 5.43535743194212e-07
139 5.42224188487594e-07
140 5.40942904800445e-07
141 5.39690835283935e-07
142 5.38467167405798e-07
143 5.37270523723876e-07
144 5.36100497349423e-07
145 5.3495645237156e-07
146 5.33837760350764e-07
147 5.32741713000107e-07
148 5.31670085619496e-07
149 5.30621131728637e-07
150 5.2959464395208e-07
151 5.28588952811049e-07
152 5.27604220385669e-07
153 5.26639167688359e-07
154 5.25695343856825e-07
155 5.24768793379593e-07
156 5.23861145858007e-07
157 5.22972100103658e-07
158 5.22100397793324e-07
159 5.21247696042337e-07
160 5.20411653905484e-07
161 5.19592305565197e-07
162 5.18788989642971e-07
163 5.18001103724686e-07
164 5.17228420383375e-07
165 5.16470411991321e-07
166 5.15726908281167e-07
167 5.14997244314586e-07
168 5.14280624576813e-07
169 5.13578183857888e-07
170 5.12887813114205e-07
171 5.12209524261564e-07
172 5.11543729029285e-07
173 5.10889387653535e-07
174 5.10246352916965e-07
175 5.09614061911634e-07
176 5.08992157007171e-07
177 5.08381853729034e-07
178 5.07780784980483e-07
179 5.07191079117675e-07
180 5.06611367566023e-07
181 5.06041979109639e-07
182 5.0548288607466e-07
183 5.04934876262908e-07
184 5.04397125517642e-07
185 5.03869310492178e-07
186 5.03352014812108e-07
187 5.02844049030671e-07
188 5.02344910890073e-07
189 5.01853563115162e-07
190 5.01370685309865e-07
191 5.00895266149826e-07
192 5.00426701247392e-07
193 4.99964965882782e-07
194 4.9950978666935e-07
195 4.99061770305786e-07
196 4.98618465481826e-07
197 4.98181225479399e-07
198 4.9774932963409e-07
199 4.97321681615759e-07
200 4.96899252649285e-07
201 4.96480924626397e-07
202 4.96067356207774e-07
203 4.95657481046408e-07
204 4.95251422496068e-07
205 4.94849243494855e-07
206 4.94451002124308e-07
207 4.94056187893221e-07
208 4.93665609385019e-07
209 4.9327701405133e-07
210 4.9289283129994e-07
211 4.92511656577932e-07
212 4.92133973788e-07
213 4.91759665150582e-07
214 4.91387891834449e-07
215 4.91019329398768e-07
216 4.90653773242755e-07
217 4.90290882298794e-07
218 4.89930456437193e-07
219 4.8957260233351e-07
220 4.89217508683026e-07
221 4.88864639208941e-07
222 4.88513271530877e-07
223 4.8816446328992e-07
224 4.87817855457706e-07
225 4.87472793972543e-07
226 4.87129437420464e-07
227 4.86787707229652e-07
228 4.86447034674598e-07
229 4.861084715575e-07
230 4.85769806896741e-07
231 4.8543388423461e-07
232 4.85098922421656e-07
233 4.84764451890385e-07
234 4.84431103824789e-07
235 4.84098741793559e-07
236 4.8376694843455e-07
237 4.83436668778481e-07
238 4.83106431111935e-07
239 4.82776878946822e-07
240 4.82448378132716e-07
241 4.82119949804627e-07
242 4.81792314273477e-07
243 4.81464538864884e-07
244 4.8113690976237e-07
245 4.80809850520458e-07
246 4.80482893083334e-07
247 4.80155803385784e-07
248 4.79829221880834e-07
249 4.79501328733534e-07
250 4.79173856977155e-07
251 4.78846092663687e-07
252 4.78517885962759e-07
253 4.78189278286578e-07
254 4.77859489466326e-07
255 4.77529891302453e-07
256 4.77198835271864e-07
257 4.76867327730446e-07
258 4.76535316909832e-07
259 4.76201553437861e-07
260 4.75867475563163e-07
261 4.75531986403155e-07
262 4.75195353830671e-07
263 4.74857501693293e-07
264 4.74518380118028e-07
265 4.74177605322979e-07
266 4.73835947998324e-07
267 4.7349216548831e-07
268 4.7314768544382e-07
269 4.72800715582977e-07
270 4.72452184286709e-07
271 4.72101605719644e-07
272 4.71749693330636e-07
273 4.71395479777925e-07
274 4.71039769442072e-07
275 4.70681513521143e-07
276 4.70321050206834e-07
277 4.69959026295541e-07
278 4.69594578209609e-07
279 4.69227827030849e-07
280 4.68858881140122e-07
281 4.68487183674426e-07
282 4.68113199328712e-07
283 4.67736385536099e-07
284 4.67357924398115e-07
285 4.66975642368084e-07
286 4.6659138822136e-07
287 4.66203931134501e-07
288 4.65813562957607e-07
289 4.65420540239236e-07
290 4.65024744858766e-07
291 4.64625989295087e-07
292 4.64223289609933e-07
293 4.63818313551911e-07
294 4.63409983336049e-07
295 4.62998122866765e-07
296 4.62583368960878e-07
297 4.62165617337362e-07
298 4.61744393257746e-07
299 4.61319245083303e-07
300 4.60891294755683e-07
301 4.60458839958378e-07
302 4.60021947992217e-07
303 4.59579846012303e-07
304 4.59133508119436e-07
305 4.58682104415686e-07
306 4.58225444896598e-07
307 4.57764044623943e-07
308 4.57299190312455e-07
309 4.5682909506084e-07
310 4.56354793156422e-07
311 4.55875298568387e-07
312 4.55388694271974e-07
313 4.54898114236713e-07
314 4.54401091554146e-07
315 4.53899251358081e-07
316 4.53391259309655e-07
317 4.52876303462801e-07
318 4.52353114562598e-07
319 4.51822839229621e-07
320 4.51284717911449e-07
321 4.5074022962055e-07
322 4.50187806752211e-07
323 4.49628460202689e-07
324 4.49062323831129e-07
325 4.48487937639186e-07
326 4.47906978068957e-07
327 4.47317338830189e-07
328 4.46720416356072e-07
329 4.46115051536466e-07
330 4.45502102884632e-07
331 4.44880177226992e-07
332 4.44250092447973e-07
333 4.4361100821888e-07
334 4.42963379368777e-07
335 4.42306873845055e-07
336 4.41640536028132e-07
337 4.40965134529847e-07
338 4.40280325657128e-07
339 4.39585711633939e-07
340 4.38880671705277e-07
341 4.38165526242429e-07
342 4.37439292424457e-07
343 4.36701997236e-07
344 4.35953074408246e-07
345 4.35193751918916e-07
346 4.34422233855258e-07
347 4.33640231058519e-07
348 4.3284546453215e-07
349 4.32038629925913e-07
350 4.31218911202791e-07
351 4.30385996873284e-07
352 4.29540072477863e-07
353 4.28680980157026e-07
354 4.27806931908137e-07
355 4.26912040868643e-07
356 4.25999318819592e-07
357 4.2507115866286e-07
358 4.24126351669329e-07
359 4.23163430429696e-07
360 4.22182999113119e-07
361 4.21183770626499e-07
362 4.20166176052561e-07
363 4.19129310650135e-07
364 4.18073201945646e-07
365 4.16997151166498e-07
366 4.15901903785709e-07
367 4.14786610436479e-07
368 4.13651236186752e-07
369 4.12494647779482e-07
370 4.1131612212908e-07
371 4.10115897505392e-07
372 4.0889271227762e-07
373 4.07646260898176e-07
374 4.06376447466883e-07
375 4.05082469169926e-07
376 4.03764083158009e-07
377 4.02420682002358e-07
378 4.01051618771575e-07
379 3.99656620741595e-07
380 3.98236072470581e-07
381 3.96789064646796e-07
382 3.95315705556953e-07
383 3.93816488985621e-07
384 3.92289311596628e-07
385 3.90734641543489e-07
386 3.89153819957855e-07
387 3.8754487388637e-07
388 3.85908550613934e-07
389 3.84243580461074e-07
390 3.82550015968874e-07
391 3.80828786830989e-07
392 3.79078039864922e-07
393 3.77297813823674e-07
394 3.75488474679386e-07
395 3.7364986675037e-07
396 3.71779560696339e-07
397 3.69879096044912e-07
398 3.67946753403459e-07
399 3.6598282945377e-07
400 3.63988785748504e-07
401 3.61964164380169e-07
402 3.59909024018279e-07
403 3.57825778694476e-07
404 3.55713920370704e-07
405 3.53574337018436e-07
406 3.51409069438979e-07
407 3.49218794934991e-07
408 3.47004987878208e-07
409 3.44768162483966e-07
410 3.42510278939834e-07
411 3.4023413138673e-07
412 3.37939882586014e-07
413 3.35629292831818e-07
414 3.33305534557127e-07
415 3.30969987961005e-07
416 3.28624690322243e-07
417 3.26272957495988e-07
418 3.23917328013401e-07
419 3.21558977607772e-07
420 3.19199202964526e-07
421 3.1683526008397e-07
422 3.14472679682432e-07
423 3.12115680817016e-07
424 3.09766026709291e-07
425 3.07427301994956e-07
426 3.05100357535615e-07
427 3.02787983525121e-07
428 3.00491786622281e-07
429 2.98213325747199e-07
430 2.9595519708181e-07
431 2.93721145090409e-07
432 2.91513914910446e-07
433 2.89335127959589e-07
434 2.8718476086631e-07
435 2.85064079091235e-07
436 2.8297332318683e-07
437 2.80914251139741e-07
438 2.78886818166235e-07
439 2.76892461841705e-07
440 2.74932256152383e-07
441 2.73005340985399e-07
442 2.71114165422581e-07
443 2.69256476775226e-07
444 2.67434799255284e-07
445 2.65646997037905e-07
446 2.63894172281454e-07
447 2.62174696119111e-07
448 2.60490009284631e-07
449 2.58840273095196e-07
450 2.57226603846306e-07
451 2.55650307608768e-07
452 2.54106650923624e-07
453 2.52597794672127e-07
454 2.51125497851668e-07
455 2.4969196913549e-07
456 2.48292265499828e-07
457 2.46923749596561e-07
458 2.45586162464839e-07
459 2.4427558194251e-07
460 2.42991922911884e-07
461 2.41731884592866e-07
462 2.40495752907854e-07
463 2.3928421400754e-07
464 2.38097916065882e-07
465 2.36940507962302e-07
466 2.35812380806166e-07
467 2.34714176996498e-07
468 2.33640298235471e-07
469 2.32590709940084e-07
470 2.3156198360752e-07
471 2.30552812658047e-07
472 2.29562360636315e-07
473 2.28589996784656e-07
474 2.27633940697913e-07
475 2.26692834407238e-07
476 2.25767104784502e-07
477 2.24855028227999e-07
478 2.23956328623487e-07
479 2.23071633382865e-07
480 2.2220174287213e-07
481 2.21343477023517e-07
482 2.2049790103118e-07
483 2.19663503443712e-07
484 2.18837621938306e-07
485 2.18020517809947e-07
486 2.17211375638904e-07
487 2.16409654909988e-07
488 2.15614259082564e-07
489 2.14824529527924e-07
490 2.14040452539699e-07
491 2.13260913816349e-07
492 2.12486086164532e-07
493 2.1171542474363e-07
494 2.10948150742851e-07
495 2.10182879386345e-07
496 2.09420725046705e-07
497 2.08661175674862e-07
498 2.07902939044935e-07
499 2.07146422790849e-07
500 2.06391954266749e-07
501 2.05637261204927e-07
502 2.0488458894885e-07
503 2.04131842492572e-07
504 2.03379557053296e-07
505 2.0262851229802e-07
506 2.01876270642387e-07
507 2.01124522051011e-07
508 2.00373244633845e-07
509 1.99620836440317e-07
510 1.98868397302654e-07
511 1.98116394676262e-07
512 1.97363473612988e-07
513 1.96610078574189e-07
514 1.95856981311415e-07
515 1.95102658726576e-07
516 1.94348581763037e-07
517 1.93595191788631e-07
518 1.92840523295601e-07
519 1.92086295577987e-07
520 1.91332275158551e-07
521 1.90577408003989e-07
522 1.89821947016e-07
523 1.89067706583224e-07
524 1.88312743440555e-07
525 1.87558361249174e-07
526 1.8680285475714e-07
527 1.8604850870485e-07
528 1.85293671856712e-07
529 1.84540336598538e-07
530 1.83786476921632e-07
531 1.83034009101135e-07
532 1.82281599756529e-07
533 1.81530664850271e-07
534 1.80780502847533e-07
535 1.8003129300137e-07
536 1.7928323286398e-07
537 1.78536797237783e-07
538 1.77791464396115e-07
539 1.77048414679248e-07
540 1.76307210876026e-07
541 1.75569254708741e-07
542 1.74833430917509e-07
543 1.74101205553612e-07
544 1.73370533071981e-07
545 1.72643471802125e-07
546 1.7191923671156e-07
547 1.71199484007012e-07
548 1.70483000956345e-07
549 1.69770761683452e-07
550 1.69064337590896e-07
551 1.68361243744641e-07
552 1.67663409039953e-07
553 1.66971040106212e-07
554 1.66281606115426e-07
555 1.65597481563751e-07
556 1.64918271168268e-07
557 1.64242208018806e-07
558 1.63570035114802e-07
559 1.62903460860342e-07
560 1.62240473990494e-07
561 1.61584498681755e-07
562 1.60934195720941e-07
563 1.6029080146307e-07
564 1.59655332807596e-07
565 1.59027187381255e-07
566 1.58406169683545e-07
567 1.57793173200638e-07
568 1.57186480822524e-07
569 1.56587651422235e-07
570 1.55995659437913e-07
571 1.55411371854264e-07
572 1.54835403577636e-07
573 1.54265864671821e-07
574 1.53706784807639e-07
575 1.53153216292168e-07
576 1.52607427956752e-07
577 1.52068064158684e-07
578 1.51536297245514e-07
579 1.51011521509758e-07
580 1.50493309179822e-07
581 1.49981604166349e-07
582 1.4947710782387e-07
583 1.48979266535676e-07
584 1.48487649183515e-07
585 1.48002414926296e-07
586 1.47524634242835e-07
587 1.4705232156409e-07
588 1.46586653343661e-07
589 1.46127322129708e-07
590 1.45673687676151e-07
591 1.45227367292833e-07
592 1.44786791667073e-07
593 1.44351789424846e-07
594 1.43925521104649e-07
595 1.43503590582128e-07
596 1.4309012667546e-07
597 1.42681885617613e-07
598 1.42281900709307e-07
599 1.41888060587014e-07
600 1.41502074426647e-07
601 1.4112394866661e-07
602 1.4076679982189e-07
603 1.4044379805167e-07
604 1.40057358432699e-07
605 1.39682924537254e-07
606 1.39324049436951e-07
607 1.38968807523732e-07
608 1.38617833747467e-07
609 1.38272092926606e-07
610 1.37930422480004e-07
611 1.37592430990985e-07
612 1.3726038601547e-07
613 1.36931854131106e-07
614 1.36607505031527e-07
615 1.36287842872562e-07
616 1.35971968933113e-07
617 1.35661062833137e-07
618 1.3535409797516e-07
619 1.35052542789005e-07
620 1.34754007381055e-07
621 1.34462073034136e-07
622 1.34175297930028e-07
623 1.33894848075577e-07
624 1.336161039589e-07
625 1.33338256134152e-07
626 1.33057876078091e-07
627 1.32779886497403e-07
628 1.32505826554663e-07
629 1.32234813478682e-07
630 1.31967503042674e-07
631 1.31703863083921e-07
632 1.31442718599217e-07
633 1.31185044203619e-07
634 1.3092963578476e-07
635 1.30677072229801e-07
636 1.30427179441561e-07
637 1.30180809400748e-07
638 1.29936798365371e-07
639 1.29694702677874e-07
640 1.29456367453429e-07
641 1.2921993900239e-07
642 1.28985613722321e-07
643 1.28755346744391e-07
644 1.28525506681854e-07
645 1.28299524551956e-07
646 1.28073863309375e-07
647 1.27852706982345e-07
648 1.27631845199261e-07
649 1.27415247706431e-07
650 1.27198914902316e-07
651 1.2698522854393e-07
652 1.26773434401706e-07
653 1.26564439925758e-07
654 1.26356387427862e-07
655 1.26150461321473e-07
656 1.25946634747187e-07
657 1.25744118695259e-07
658 1.25543533306072e-07
659 1.25345706083024e-07
660 1.25149239569744e-07
661 1.24953934794725e-07
662 1.24760150330694e-07
663 1.24568040510198e-07
664 1.24378543249648e-07
665 1.24189568873057e-07
666 1.24003258553884e-07
667 1.2381765579228e-07
668 1.23632453549405e-07
669 1.23450765727284e-07
670 1.23269354425304e-07
671 1.23089941609145e-07
672 1.22910342528648e-07
673 1.22733811772413e-07
674 1.22558044377996e-07
675 1.22383103368762e-07
676 1.22209880956348e-07
677 1.2203793848542e-07
678 1.21866845670837e-07
679 1.21697539775312e-07
680 1.21528744259081e-07
681 1.21361572604783e-07
682 1.21195417700726e-07
683 1.2103058784696e-07
684 1.20866227339533e-07
685 1.20703809670886e-07
686 1.20541875671343e-07
687 1.20380765880057e-07
688 1.20220632739532e-07
689 1.2006198598602e-07
690 1.19904353131695e-07
691 1.1974716833052e-07
692 1.19591444108558e-07
693 1.19436018377606e-07
694 1.19282029586998e-07
695 1.19129180860433e-07
696 1.18976792411907e-07
697 1.1882506441907e-07
698 1.18674348054348e-07
699 1.18524874785031e-07
700 1.18376379715457e-07
701 1.1822792154792e-07
702 1.18080373377794e-07
703 1.17934063162295e-07
704 1.17788347779424e-07
705 1.17643357246955e-07
706 1.17499305417823e-07
707 1.173554876015e-07
708 1.17212348007101e-07
709 1.17070766659566e-07
710 1.1692914187833e-07
711 1.16788458027983e-07
712 1.16648767719774e-07
713 1.16508806314641e-07
714 1.16371519003344e-07
715 1.16233159999091e-07
716 1.16095862712662e-07
717 1.15959106578245e-07
718 1.15822969175561e-07
719 1.15687920446028e-07
720 1.15553413675862e-07
721 1.15419141629047e-07
722 1.15286001141968e-07
723 1.15152935511453e-07
724 1.15021342243615e-07
725 1.14888953016923e-07
726 1.14757742866622e-07
727 1.14627121838851e-07
728 1.1449774532224e-07
729 1.14368557701638e-07
730 1.14239562624796e-07
731 1.1411168682951e-07
732 1.13983184983901e-07
733 1.13856436687598e-07
734 1.13729060205792e-07
735 1.13603906172877e-07
736 1.13478049471816e-07
737 1.1335355964448e-07
738 1.13228808291232e-07
739 1.1310464036729e-07
740 1.12980925309536e-07
741 1.1285901374869e-07
742 1.12736019158177e-07
743 1.12612979557447e-07
744 1.12492231022721e-07
745 1.12371788842935e-07
746 1.12250489089583e-07
747 1.1213093527207e-07
748 1.12010966712539e-07
749 1.1189149313573e-07
750 1.11773492173839e-07
751 1.11655507066821e-07
752 1.11537206681334e-07
753 1.11419941912771e-07
754 1.11303521078909e-07
755 1.11186201563918e-07
756 1.1107064740834e-07
757 1.10955272610624e-07
758 1.10839430989884e-07
759 1.10725663525102e-07
760 1.10611395911064e-07
761 1.10498082840138e-07
762 1.10385266285817e-07
763 1.10272021106361e-07
764 1.10159000885623e-07
765 1.10047804649405e-07
766 1.09936218342099e-07
767 1.098251588596e-07
768 1.09714542053219e-07
769 1.09604423178311e-07
770 1.09494104558294e-07
771 1.09384491588038e-07
772 1.09275923759533e-07
773 1.09165804849098e-07
774 1.09058371438486e-07
775 1.08949745491138e-07
776 1.08842444357826e-07
777 1.08734781170128e-07
778 1.08627692129382e-07
779 1.08521529667449e-07
780 1.08415654531235e-07
781 1.08309868553036e-07
782 1.08204369053233e-07
783 1.08099382219784e-07
784 1.07994806988643e-07
785 1.07890639399422e-07
786 1.07786188236147e-07
787 1.07682291907274e-07
788 1.07580379563821e-07
789 1.07476687300156e-07
790 1.0737360758295e-07
791 1.07271623931204e-07
792 1.07169460913603e-07
793 1.0706793543136e-07
794 1.06966185795088e-07
795 1.06865274080548e-07
796 1.06764554595351e-07
797 1.06664038335147e-07
798 1.06564123339759e-07
799 1.06464349325819e-07
800 1.06365098394789e-07
801 1.0626545061676e-07
802 1.06166281669928e-07
803 1.06068719824215e-07
804 1.05970090050533e-07
805 1.05871850978545e-07
806 1.0577374633236e-07
807 1.0567669667072e-07
808 1.05578770395631e-07
809 1.05481860884105e-07
810 1.05385178051698e-07
811 1.052873337688e-07
812 1.05191094444734e-07
813 1.05094254446492e-07
814 1.04998168061421e-07
815 1.04901266116286e-07
816 1.04804803130243e-07
817 1.04710489993387e-07
818 1.04615900424321e-07
819 1.04521459809881e-07
820 1.04428397812661e-07
821 1.04334857257804e-07
822 1.04242746597372e-07
823 1.04149216329397e-07
824 1.04057178050176e-07
825 1.03965425625852e-07
826 1.03874129885995e-07
827 1.03782101080974e-07
828 1.03691924961069e-07
829 1.03600579121732e-07
830 1.0351067146086e-07
831 1.03420728982506e-07
832 1.0333036133936e-07
833 1.03241278985422e-07
834 1.03152590446243e-07
835 1.03063033981776e-07
836 1.0297513878843e-07
837 1.02887132955587e-07
838 1.02798523812453e-07
839 1.02712042264308e-07
840 1.02624302769527e-07
841 1.02537733839725e-07
842 1.02451398892534e-07
843 1.02364051266512e-07
844 1.02278665043087e-07
845 1.02192160946757e-07
846 1.02107566782195e-07
847 1.02022163450677e-07
848 1.01936578046136e-07
849 1.01852086273446e-07
850 1.01768156070214e-07
851 1.01683125325103e-07
852 1.01599414790599e-07
853 1.01516458839157e-07
854 1.01432540231983e-07
855 1.01349948372409e-07
856 1.01267084948731e-07
857 1.01184374096341e-07
858 1.01101930705561e-07
859 1.01019599126673e-07
860 1.00937868184658e-07
861 1.00856728624699e-07
862 1.00774979018503e-07
863 1.00694579749039e-07
864 1.00614032057855e-07
865 1.00532648981577e-07
866 1.00452752713309e-07
867 1.00372748915944e-07
868 1.00293050685707e-07
869 1.00213816318195e-07
870 1.00134035158739e-07
871 1.00055982054315e-07
872 9.99766341349684e-08
873 9.98989246765447e-08
874 9.98200620951906e-08
875 9.97422426864958e-08
876 9.96656695626541e-08
877 9.95875502090016e-08
878 9.95116613236036e-08
879 9.94347382823335e-08
880 9.9357052012472e-08
881 9.92819395086997e-08
882 9.92067116190043e-08
883 9.91310633287057e-08
884 9.90554480608807e-08
885 9.89809642195638e-08
886 9.89063181355831e-08
887 9.88318642614061e-08
888 9.87581276428173e-08
889 9.86838774670318e-08
890 9.86099341684366e-08
891 9.85371098511934e-08
892 9.84647759780799e-08
893 9.83914109946582e-08
894 9.83185971401568e-08
895 9.82461011265201e-08
896 9.81744863919332e-08
897 9.81032099298673e-08
898 9.80305909692447e-08
899 9.7960472541736e-08
900 9.7889869421941e-08
901 9.78183804551946e-08
902 9.77487473194927e-08
903 9.76790646944892e-08
904 9.76083303525499e-08
905 9.75392070206027e-08
906 9.7469951932716e-08
907 9.74008674754501e-08
908 9.7332167749542e-08
909 9.7262588022673e-08
910 9.71945030681098e-08
911 9.71265472919924e-08
912 9.70592057125685e-08
913 9.69910914383476e-08
914 9.69237570691561e-08
915 9.68562117993343e-08
916 9.67903531270409e-08
917 9.67237199622772e-08
918 9.66569018272523e-08
919 9.65902600329471e-08
920 9.65248649977823e-08
921 9.64592179837354e-08
922 9.63934796693877e-08
923 9.63280732637628e-08
924 9.62641046173829e-08
925 9.61987970695688e-08
926 9.61353359354433e-08
927 9.60700777410395e-08
928 9.60075813756234e-08
929 9.59444322408132e-08
930 9.58813187006413e-08
931 9.58189899815665e-08
932 9.57566628603246e-08
933 9.56957888709198e-08
934 9.5634532608635e-08
935 9.55733184326846e-08
936 9.55124679675734e-08
937 9.54512587423295e-08
938 9.53904744713796e-08
939 9.53287891523047e-08
940 9.52676536369879e-08
941 9.52062585977131e-08
942 9.51461792908503e-08
943 9.50845020106783e-08
944 9.50232606768964e-08
945 9.49624499186896e-08
946 9.49032121857485e-08
947 9.48426708742289e-08
948 9.478213794889e-08
949 9.472285239287e-08
950 9.46621660107283e-08
951 9.46029192538944e-08
952 9.45440140887044e-08
953 9.44842329611006e-08
954 9.44241351339414e-08
955 9.43666868344906e-08
956 9.43073171102426e-08
957 9.42484763513107e-08
958 9.41909389107209e-08
959 9.41326146239874e-08
960 9.40740558519337e-08
961 9.40168648799755e-08
962 9.39592730366812e-08
963 9.39016181291663e-08
964 9.38441444722216e-08
965 9.37879033218181e-08
966 9.37297625567268e-08
967 9.36738852219676e-08
968 9.36166905916025e-08
969 9.35602593479246e-08
970 9.35049112218422e-08
971 9.34477159422187e-08
972 9.33928641417836e-08
973 9.33366612150266e-08
974 9.3282054903554e-08
975 9.32264996604815e-08
976 9.31709776876843e-08
977 9.31160994994684e-08
978 9.30611239651213e-08
979 9.30066059616763e-08
980 9.29520029497866e-08
981 9.28969003570757e-08
982 9.28430532471936e-08
983 9.27886198445194e-08
984 9.27354522675117e-08
985 9.26816749178272e-08
986 9.26274455022025e-08
987 9.25738611270432e-08
988 9.25211871365406e-08
989 9.24672447153441e-08
990 9.24141927187705e-08
991 9.2361543644337e-08
992 9.23087412010304e-08
993 9.22567495127424e-08
994 9.22034969779162e-08
995 9.21506427364882e-08
996 9.20987951631425e-08
997 9.20470391041484e-08
998 9.19947635420471e-08
999 9.19429998007537e-08
1000 9.18914799941106e-08
1001 9.18404569505427e-08
1002 9.17894005412201e-08
1003 9.1737167482453e-08
1004 9.16867615257111e-08
1005 9.16358475988588e-08
1006 9.15851797640421e-08
1007 9.15345765770681e-08
1008 9.14838958236963e-08
1009 9.14335053989745e-08
1010 9.13832967941452e-08
1011 9.13334147205624e-08
1012 9.12841226283767e-08
1013 9.123387925003e-08
1014 9.11847187436976e-08
1015 9.1135013221777e-08
1016 9.10860397378421e-08
1017 9.10365201471919e-08
1018 9.09882848292298e-08
1019 9.09388333876215e-08
1020 9.08902777618792e-08
1021 9.08417003326889e-08
1022 9.07928683679948e-08
1023 9.07455694671988e-08
1024 9.06960643471955e-08
1025 9.06481753135679e-08
1026 9.06003936997912e-08
1027 9.05528672907607e-08
1028 9.05042029124559e-08
1029 9.04573887607896e-08
1030 9.04091605700152e-08
1031 9.03626517878564e-08
1032 9.0315453202372e-08
1033 9.02673321219183e-08
1034 9.02204775412585e-08
1035 9.01741673580148e-08
1036 9.01272176956347e-08
1037 9.00803713395071e-08
1038 9.00332719417918e-08
1039 8.99864664156702e-08
1040 8.99408807777391e-08
1041 8.98946636707088e-08
1042 8.98485885669764e-08
1043 8.98017804518148e-08
1044 8.97567148721024e-08
1045 8.97099244894228e-08
1046 8.96646553401048e-08
1047 8.96185504228875e-08
1048 8.95729888052976e-08
1049 8.95285867437146e-08
1050 8.9482757309689e-08
1051 8.94376005238229e-08
1052 8.93923830904697e-08
1053 8.9346351074937e-08
1054 8.93026444201972e-08
1055 8.92573668727081e-08
1056 8.92138810448806e-08
1057 8.91676794361196e-08
1058 8.91232493245298e-08
1059 8.90787027727491e-08
1060 8.90344709678637e-08
1061 8.89912134622861e-08
1062 8.89462300310839e-08
1063 8.89029015609388e-08
1064 8.88588601872797e-08
1065 8.88143503723349e-08
1066 8.87721652498286e-08
1067 8.87288980844225e-08
1068 8.86862023032897e-08
1069 8.864279176235e-08
1070 8.86010947085403e-08
1071 8.85579700811157e-08
1072 8.85148414933035e-08
1073 8.84722708622832e-08
1074 8.84297221199759e-08
1075 8.83872897725624e-08
1076 8.83456322924303e-08
1077 8.83032643979043e-08
1078 8.82595783266993e-08
1079 8.8219266944467e-08
1080 8.817642199066e-08
1081 8.81352094785903e-08
1082 8.80918597356839e-08
1083 8.80533137399553e-08
1084 8.80079192029371e-08
1085 8.79724620079791e-08
1086 8.7921471380259e-08
1087 8.78960680354979e-08
1088 8.78312586247709e-08
1089 8.78240154968957e-08
1090 8.77426289962457e-08
1091 8.77462830572995e-08
1092 8.76592526877218e-08
1093 8.76662806623685e-08
1094 8.75767898671498e-08
1095 8.75872193377702e-08
1096 8.74952507476934e-08
1097 8.75082064064614e-08
1098 8.74147617451193e-08
1099 8.74288802519274e-08
1100 8.73345445273621e-08
1101 8.73511452752496e-08
1102 8.72546325645374e-08
1103 8.72725221192638e-08
1104 8.71757469003143e-08
1105 8.71958241437376e-08
1106 8.70961566707962e-08
1107 8.7117558360994e-08
1108 8.70182770182915e-08
1109 8.70406783932687e-08
1110 8.694062198078e-08
1111 8.69637693514669e-08
1112 8.68637468016686e-08
1113 8.68864337792985e-08
1114 8.67873502343741e-08
1115 8.68107485598735e-08
1116 8.67112360660727e-08
1117 8.67347486752657e-08
1118 8.66353276718357e-08
1119 8.66588314032413e-08
1120 8.65596516268496e-08
1121 8.65840134687268e-08
1122 8.64849380413091e-08
1123 8.65090472284002e-08
1124 8.64103304638597e-08
1125 8.64340824313636e-08
1126 8.63364721346294e-08
1127 8.63598102736063e-08
1128 8.62631220814869e-08
1129 8.62853951315579e-08
1130 8.61891990786567e-08
1131 8.62124033695899e-08
1132 8.6116572704853e-08
1133 8.61398361671206e-08
1134 8.60441184600269e-08
1135 8.60664031439029e-08
1136 8.59716593857307e-08
1137 8.59941059201397e-08
1138 8.59000855184533e-08
1139 8.59219365714203e-08
1140 8.58290302967468e-08
1141 8.5849491385126e-08
1142 8.57576409871719e-08
1143 8.57774520075516e-08
1144 8.56875948787028e-08
1145 8.57061396857262e-08
1146 8.56168941987256e-08
1147 8.56353976046442e-08
1148 8.55462333699819e-08
1149 8.55649901225775e-08
1150 8.54764039579514e-08
1151 8.54943547388132e-08
1152 8.54066035578249e-08
1153 8.542451854332e-08
1154 8.5338350565678e-08
1155 8.53543031462678e-08
1156 8.52687050501011e-08
1157 8.52839499834168e-08
1158 8.52004735980572e-08
1159 8.52152409289708e-08
1160 8.51314517538171e-08
1161 8.51466068159112e-08
1162 8.50643536458584e-08
1163 8.50772254508136e-08
1164 8.49960110440584e-08
1165 8.50084927006556e-08
1166 8.49284661970096e-08
1167 8.49400766944619e-08
1168 8.48611730344118e-08
1169 8.48727081264045e-08
1170 8.47934970242825e-08
1171 8.48045137695763e-08
1172 8.47271792490112e-08
1173 8.47366477101374e-08
1174 8.4660774513079e-08
1175 8.46697410370467e-08
1176 8.4594639383706e-08
1177 8.46024692977565e-08
1178 8.45283746837566e-08
1179 8.45350552967083e-08
1180 8.44623309030901e-08
1181 8.44687112659948e-08
1182 8.43972671313864e-08
1183 8.44027526438751e-08
1184 8.43310951861298e-08
1185 8.43359814335365e-08
1186 8.42665759011041e-08
1187 8.42698686467358e-08
1188 8.42011252157882e-08
1189 8.42039774240355e-08
1190 8.41366869352989e-08
1191 8.4137812871532e-08
1192 8.40712528362708e-08
1193 8.40737473715158e-08
1194 8.40070786711777e-08
1195 8.40071622403293e-08
1196 8.39430820986919e-08
1197 8.39426023202527e-08
1198 8.38787964227983e-08
1199 8.387783601016e-08
1200 8.38148301545161e-08
1201 8.38128256530624e-08
1202 8.37512253739625e-08
1203 8.37478700241867e-08
1204 8.36875812200155e-08
1205 8.36833237789492e-08
1206 8.36245365505306e-08
1207 8.36189088690986e-08
1208 8.35615930507849e-08
1209 8.35544970763102e-08
1210 8.34993397162975e-08
1211 8.34901943398414e-08
1212 8.3435835417589e-08
1213 8.34270988545605e-08
1214 8.33738550825025e-08
1215 8.33626558902623e-08
1216 8.33116351968144e-08
1217 8.33000525672922e-08
1218 8.32479299259781e-08
1219 8.32367513323717e-08
1220 8.31867665662678e-08
1221 8.31725161849484e-08
1222 8.31251798407706e-08
1223 8.31089185711065e-08
1224 8.30644063158914e-08
1225 8.30460797920374e-08
1226 8.30028129112925e-08
1227 8.29833722488793e-08
1228 8.29409944316239e-08
1229 8.29204949601703e-08
1230 8.28802622243607e-08
1231 8.28575117104435e-08
1232 8.28198565301364e-08
1233 8.27949343227274e-08
1234 8.27601576820491e-08
1235 8.27312651296985e-08
1236 8.27002418772871e-08
1237 8.2668509257644e-08
1238 8.26406430958215e-08
1239 8.26060267438145e-08
1240 8.25812059188635e-08
1241 8.25439024869645e-08
1242 8.25226102505461e-08
1243 8.24804552768121e-08
1244 8.24639944760897e-08
1245 8.24190747552578e-08
1246 8.24042353091414e-08
1247 8.23567446457396e-08
1248 8.23449259574893e-08
1249 8.22955023784822e-08
1250 8.22846286019363e-08
1251 8.22354937355385e-08
1252 8.22245969227531e-08
1253 8.21754996001012e-08
1254 8.21630933245388e-08
1255 8.21175030032428e-08
1256 8.21014573988421e-08
1257 8.20584940397495e-08
1258 8.20405686252634e-08
1259 8.19998827594759e-08
1260 8.19792240052131e-08
1261 8.19417065982542e-08
1262 8.19191965049093e-08
1263 8.18827151576329e-08
1264 8.18590197502189e-08
1265 8.18248426863732e-08
1266 8.17988242052259e-08
1267 8.17672002959568e-08
1268 8.1739034409889e-08
1269 8.17090094722417e-08
1270 8.16804539174321e-08
1271 8.16505861749306e-08
1272 8.16225137367788e-08
1273 8.15925359778369e-08
1274 8.15638973330479e-08
1275 8.15358847030545e-08
1276 8.15053808054245e-08
1277 8.1476992219276e-08
1278 8.14478518806894e-08
1279 8.1419286017681e-08
1280 8.1390514633739e-08
1281 8.13611375396483e-08
1282 8.13316255889873e-08
1283 8.13039963949436e-08
1284 8.12738803452895e-08
1285 8.12460943215854e-08
1286 8.12161784060628e-08
1287 8.11877706068387e-08
1288 8.11584114241981e-08
1289 8.11302389291946e-08
1290 8.11009234675808e-08
1291 8.10721862247732e-08
1292 8.10432848590281e-08
1293 8.10144885918795e-08
1294 8.0985415526591e-08
1295 8.09567232269437e-08
1296 8.09273573549873e-08
1297 8.08991597307518e-08
1298 8.08700364967052e-08
1299 8.08409121821896e-08
1300 8.08125868703158e-08
1301 8.07832061435754e-08
1302 8.0754618521528e-08
1303 8.07255188401967e-08
1304 8.06975238467622e-08
1305 8.06684763321464e-08
1306 8.06394303229929e-08
1307 8.06114288587345e-08
1308 8.05817716189594e-08
1309 8.05533837744399e-08
1310 8.05248986761597e-08
1311 8.04956761313313e-08
1312 8.04671938889889e-08
1313 8.04382691730332e-08
1314 8.04101232363408e-08
1315 8.03806049094469e-08
1316 8.03537706488555e-08
1317 8.0322387340459e-08
1318 8.02977602356591e-08
1319 8.02624958025078e-08
1320 8.02430809025978e-08
1321 8.02053518427215e-08
1322 8.01816001834155e-08
1323 8.01514784232182e-08
1324 8.01266004799572e-08
1325 8.00944326044828e-08
1326 8.00678066363858e-08
1327 8.00383505716873e-08
1328 8.00122158008953e-08
1329 7.99816432621725e-08
1330 7.99546455434452e-08
1331 7.99242442517389e-08
1332 7.98987869390366e-08
1333 7.98674317743497e-08
1334 7.9841036703332e-08
1335 7.98107138200166e-08
1336 7.97842641686586e-08
1337 7.97540083437021e-08
1338 7.97270363257496e-08
1339 7.96970454435275e-08
1340 7.96705586649793e-08
1341 7.96413617414338e-08
1342 7.96124550461741e-08
1343 7.95849561905371e-08
1344 7.95549121983541e-08
1345 7.95295265052509e-08
1346 7.94995203459159e-08
1347 7.94729358282176e-08
1348 7.94448911385359e-08
1349 7.94178129135403e-08
1350 7.93900669373571e-08
1351 7.93618911472116e-08
1352 7.93351382757379e-08
1353 7.93075559637479e-08
1354 7.92796861284195e-08
1355 7.92527311852531e-08
1356 7.92253148538258e-08
1357 7.91979045162705e-08
1358 7.91702418450058e-08
1359 7.91432440157003e-08
1360 7.91165960341011e-08
1361 7.90880308487729e-08
1362 7.90611475554037e-08
1363 7.9033928356953e-08
1364 7.90072395226993e-08
1365 7.8979606293661e-08
1366 7.89518214201479e-08
1367 7.89250576689327e-08
1368 7.88978489358883e-08
1369 7.88713220605253e-08
1370 7.88425342044263e-08
1371 7.88164783638301e-08
1372 7.87883669759459e-08
1373 7.87619562823139e-08
1374 7.87332735296964e-08
1375 7.87077023254312e-08
1376 7.86795233964632e-08
1377 7.86521028395271e-08
1378 7.86253623750177e-08
1379 7.85971941379415e-08
1380 7.85703751868816e-08
1381 7.85438928483373e-08
1382 7.85157824281235e-08
1383 7.84891557414902e-08
1384 7.84621017690945e-08
1385 7.84342480013578e-08
1386 7.84070640831303e-08
1387 7.83801769528303e-08
1388 7.83525572258803e-08
1389 7.83259233037015e-08
1390 7.82977864393075e-08
1391 7.82705351767277e-08
1392 7.82440626361236e-08
1393 7.82153473184444e-08
1394 7.81890101801963e-08
1395 7.81608479014651e-08
1396 7.81335201431865e-08
1397 7.81063339805321e-08
1398 7.80783582325917e-08
1399 7.80507776725337e-08
1400 7.80233677080133e-08
1401 7.79953518668108e-08
1402 7.79682084686151e-08
1403 7.79400334378622e-08
1404 7.79124773715445e-08
1405 7.78848064806326e-08
1406 7.78571340736001e-08
1407 7.78293051721946e-08
1408 7.78019424689802e-08
1409 7.77739820123635e-08
1410 7.77456021805456e-08
1411 7.77180827116197e-08
1412 7.76902832728687e-08
1413 7.7662133795009e-08
1414 7.76333130394491e-08
1415 7.76069283041636e-08
1416 7.75775152712654e-08
1417 7.75498251157636e-08
1418 7.75213585844092e-08
1419 7.74935438658808e-08
1420 7.74641294625233e-08
1421 7.74373860683575e-08
1422 7.74077920344673e-08
1423 7.73799799316244e-08
1424 7.73519934766931e-08
1425 7.73225261245791e-08
1426 7.72947047682493e-08
1427 7.72657563312684e-08
1428 7.72374232322903e-08
1429 7.72087308438252e-08
1430 7.71798375489574e-08
1431 7.7151733173686e-08
1432 7.71221533351074e-08
1433 7.70930564359951e-08
1434 7.70651059576188e-08
1435 7.70356959853835e-08
1436 7.70065558786115e-08
1437 7.69771774309369e-08
1438 7.69490062606515e-08
1439 7.69188740594728e-08
1440 7.68905014303378e-08
1441 7.68605836203307e-08
1442 7.68315937582642e-08
1443 7.68023806894114e-08
1444 7.67728998112283e-08
1445 7.6744520221439e-08
1446 7.6713959997754e-08
1447 7.66841954793485e-08
1448 7.66557426308268e-08
1449 7.66256177140434e-08
1450 7.65964450080148e-08
1451 7.65666649358288e-08
1452 7.65362178194451e-08
1453 7.65076851956259e-08
1454 7.64774769694832e-08
1455 7.64477978840716e-08
1456 7.64177950727252e-08
1457 7.63879823946212e-08
1458 7.63582361846815e-08
1459 7.6327985766067e-08
1460 7.62986109039687e-08
1461 7.62678498809066e-08
1462 7.62379654406509e-08
1463 7.62084912571126e-08
1464 7.61781193983069e-08
1465 7.61482615416753e-08
1466 7.6117548361232e-08
1467 7.60874380096865e-08
1468 7.60569823476942e-08
1469 7.60277748024229e-08
1470 7.59960386402803e-08
1471 7.59672394505806e-08
1472 7.59357019761708e-08
1473 7.59059634187764e-08
1474 7.5874759558392e-08
1475 7.58448026747516e-08
1476 7.5814421672149e-08
1477 7.57833141675768e-08
1478 7.57527310293327e-08
1479 7.57228906520524e-08
1480 7.56924820541904e-08
1481 7.56616655519515e-08
1482 7.56310478857713e-08
1483 7.56002553292667e-08
1484 7.55692892107085e-08
1485 7.55381188093374e-08
1486 7.55075267253602e-08
1487 7.54772643114521e-08
1488 7.544640040269e-08
1489 7.54154204085644e-08
1490 7.5383721468647e-08
1491 7.53539053284236e-08
1492 7.53219157827445e-08
1493 7.52913625960971e-08
1494 7.52602644777944e-08
1495 7.5229288178047e-08
1496 7.51974629471519e-08
1497 7.51672556664751e-08
1498 7.51361838835507e-08
1499 7.51048071934157e-08
1500 7.50736240897609e-08
1501 7.50428002493919e-08
1502 7.50101534010561e-08
1503 7.49800447175097e-08
1504 7.49483108144489e-08
1505 7.49179307257819e-08
1506 7.48853218635048e-08
1507 7.48546509230508e-08
1508 7.48221657653758e-08
1509 7.47917290100375e-08
1510 7.47601504360596e-08
1511 7.47283294049161e-08
1512 7.46968700977568e-08
1513 7.46653213048454e-08
1514 7.46341382953375e-08
1515 7.46021175488387e-08
1516 7.45700578970165e-08
1517 7.45387498075623e-08
1518 7.45068969889573e-08
1519 7.44757247463923e-08
1520 7.44437996269554e-08
1521 7.44117512705422e-08
1522 7.43796879798531e-08
1523 7.43481200911056e-08
1524 7.4316272247632e-08
1525 7.42835894791405e-08
1526 7.42519715757339e-08
1527 7.42198316241449e-08
1528 7.41879873142892e-08
1529 7.41564252786375e-08
1530 7.41232399978919e-08
1531 7.40914172392415e-08
1532 7.40592432668663e-08
1533 7.40273972792416e-08
1534 7.39939382907728e-08
1535 7.39630001240776e-08
1536 7.39307824493274e-08
1537 7.3897542789858e-08
1538 7.38655841052704e-08
1539 7.38326230900199e-08
1540 7.38006322373863e-08
1541 7.37680777551653e-08
1542 7.37357506421787e-08
1543 7.37033419104804e-08
1544 7.36711613065211e-08
1545 7.36382499599841e-08
1546 7.36053521066538e-08
1547 7.35732345202855e-08
1548 7.35400781710638e-08
1549 7.35068957942175e-08
1550 7.3474740637014e-08
1551 7.34426049828763e-08
1552 7.34094401710905e-08
1553 7.33764778853008e-08
1554 7.33438139284281e-08
1555 7.33111539044096e-08
1556 7.3278109701036e-08
1557 7.32451502098996e-08
1558 7.32117411899047e-08
1559 7.31798056805566e-08
1560 7.31458883431735e-08
1561 7.31134089142493e-08
1562 7.30798157064783e-08
1563 7.30476788963763e-08
1564 7.30139127882445e-08
1565 7.29807708426122e-08
1566 7.29481298553658e-08
1567 7.29152318466042e-08
1568 7.28814525801091e-08
1569 7.28479588496533e-08
1570 7.28155233420402e-08
1571 7.2782194362464e-08
1572 7.27492611578562e-08
1573 7.27160737916677e-08
1574 7.26816272762498e-08
1575 7.26493737279732e-08
1576 7.26161301125572e-08
1577 7.25826641869709e-08
1578 7.25496634297329e-08
1579 7.25154059439959e-08
1580 7.24819965176593e-08
1581 7.24494839836609e-08
1582 7.24154122933029e-08
1583 7.2381901731422e-08
1584 7.23485457583273e-08
1585 7.23157948168129e-08
1586 7.22812370850257e-08
1587 7.22484918980193e-08
1588 7.22145510820837e-08
1589 7.21810488499841e-08
1590 7.21471514273375e-08
1591 7.21133317922451e-08
1592 7.20805017424553e-08
1593 7.20467990116269e-08
1594 7.20126981432756e-08
1595 7.19794009000907e-08
1596 7.19454492141836e-08
1597 7.1910994361879e-08
1598 7.18776798867005e-08
1599 7.18440256384234e-08
1600 7.18095098548588e-08
1601 7.17764275219856e-08
1602 7.17424127398836e-08
1603 7.17078513456926e-08
1604 7.16742620237021e-08
1605 7.16399431173187e-08
1606 7.16063437224967e-08
1607 7.15723102655552e-08
1608 7.15380985871761e-08
1609 7.1503890667568e-08
1610 7.14697781467599e-08
1611 7.14352654660466e-08
1612 7.14015175664606e-08
1613 7.13666757525821e-08
1614 7.13327843548939e-08
1615 7.12985247734998e-08
1616 7.12636601121197e-08
1617 7.12299452607645e-08
1618 7.11953142191746e-08
1619 7.11603044845077e-08
1620 7.11260082835175e-08
1621 7.10923414355413e-08
1622 7.10563799648512e-08
1623 7.10225139730625e-08
1624 7.09878652846996e-08
1625 7.09532046516692e-08
1626 7.09182937925768e-08
1627 7.08836601317486e-08
1628 7.08488178866595e-08
1629 7.08144393235521e-08
1630 7.0778421382478e-08
1631 7.07443800722629e-08
1632 7.07090882863248e-08
1633 7.06741407410227e-08
1634 7.06380178918664e-08
1635 7.06037719826469e-08
1636 7.05680663877217e-08
1637 7.05337480262358e-08
1638 7.04977536170048e-08
1639 7.04624376184348e-08
1640 7.04278908449929e-08
1641 7.03925430785013e-08
1642 7.03571483295917e-08
1643 7.03219557633972e-08
1644 7.02858101488957e-08
1645 7.02505978793511e-08
1646 7.02154745124695e-08
1647 7.01797041444685e-08
1648 7.01443333577245e-08
1649 7.01088816472684e-08
1650 7.00731194296367e-08
1651 7.00369879962359e-08
1652 7.0001921501639e-08
1653 6.99666858641201e-08
1654 6.9930246841654e-08
1655 6.98948670883048e-08
1656 6.98592375418805e-08
1657 6.98234866645109e-08
1658 6.97870415216073e-08
1659 6.97515061585108e-08
1660 6.97148882244925e-08
1661 6.96797244188474e-08
1662 6.96436365212882e-08
1663 6.96072762722544e-08
1664 6.9571051661832e-08
1665 6.95348556289943e-08
1666 6.94992612046974e-08
1667 6.94619351362391e-08
1668 6.94263770664172e-08
1669 6.93899441803758e-08
1670 6.93527304003183e-08
1671 6.93171582399898e-08
1672 6.92803364188421e-08
1673 6.92447973928978e-08
1674 6.92073932029302e-08
1675 6.9170331104651e-08
1676 6.91339881973008e-08
1677 6.90978342783311e-08
1678 6.9060660055964e-08
1679 6.90239082512534e-08
1680 6.89873621277925e-08
1681 6.89500617516714e-08
1682 6.89134307680916e-08
1683 6.88768749621538e-08
1684 6.88392989225406e-08
1685 6.88022471462269e-08
1686 6.87650417900976e-08
1687 6.87281943712037e-08
1688 6.86904063686988e-08
1689 6.86543358559e-08
1690 6.86165854841825e-08
1691 6.85792881256475e-08
1692 6.85417748615969e-08
1693 6.85040458092701e-08
1694 6.84668774093566e-08
1695 6.84292362769234e-08
1696 6.83916631700754e-08
1697 6.83537331775952e-08
1698 6.83161081798112e-08
1699 6.82791363004753e-08
1700 6.82404322942531e-08
1701 6.82038421158104e-08
1702 6.81650738147965e-08
1703 6.81268970788018e-08
1704 6.80897754072873e-08
1705 6.8051495587973e-08
1706 6.80129906478477e-08
1707 6.79756046535473e-08
1708 6.79372729925909e-08
1709 6.78979244366218e-08
1710 6.78612548092872e-08
1711 6.7822136845308e-08
1712 6.77846358478895e-08
1713 6.77458776152662e-08
1714 6.77073675534601e-08
1715 6.76684467433297e-08
1716 6.76311658804885e-08
1717 6.75918248678187e-08
1718 6.75535262790738e-08
1719 6.75143692361324e-08
1720 6.74759211931608e-08
1721 6.74369114799234e-08
1722 6.73983093153474e-08
1723 6.73597483027422e-08
1724 6.73210525810042e-08
1725 6.72818294300548e-08
1726 6.72428358914345e-08
1727 6.72037588285157e-08
1728 6.71645287413369e-08
1729 6.71258455375856e-08
1730 6.70866422649574e-08
1731 6.70473237329716e-08
1732 6.700881688948e-08
1733 6.6968815858548e-08
1734 6.69298483169101e-08
1735 6.68897866402674e-08
1736 6.6850787754813e-08
1737 6.68116039830302e-08
1738 6.677188186055e-08
1739 6.6732521825319e-08
1740 6.66921801952824e-08
1741 6.66533462476693e-08
1742 6.66134111817485e-08
1743 6.65741427683386e-08
1744 6.65340432011163e-08
1745 6.64947066248978e-08
1746 6.64541262009877e-08
1747 6.64146508042585e-08
1748 6.63744130378063e-08
1749 6.63347701461703e-08
1750 6.62947038390094e-08
1751 6.62536484221654e-08
1752 6.62145804577463e-08
1753 6.61744165357803e-08
1754 6.61331242572949e-08
1755 6.60942692647382e-08
1756 6.60529941849397e-08
1757 6.60133659082796e-08
1758 6.59727948746358e-08
1759 6.59323981127002e-08
1760 6.58926800083393e-08
1761 6.58519952465575e-08
1762 6.58111083069457e-08
1763 6.57715050502716e-08
1764 6.57307252671657e-08
1765 6.56910361724883e-08
1766 6.56500643088087e-08
1767 6.56102243263845e-08
1768 6.55710529025022e-08
1769 6.55290545306109e-08
1770 6.54884544060153e-08
1771 6.54478083039756e-08
1772 6.5406975725768e-08
1773 6.53654011477656e-08
1774 6.53244481929605e-08
1775 6.52827659628485e-08
1776 6.52423349776221e-08
1777 6.52004487147195e-08
1778 6.51588381299639e-08
1779 6.5117320587671e-08
1780 6.50756197337898e-08
1781 6.50335524272627e-08
1782 6.49923343858383e-08
1783 6.4951127944024e-08
1784 6.49079997541158e-08
1785 6.48668900464777e-08
1786 6.48242012188938e-08
1787 6.47834072471376e-08
1788 6.47410601866127e-08
1789 6.46977610267996e-08
1790 6.46565563751089e-08
1791 6.46143364662066e-08
1792 6.45714698777056e-08
1793 6.45296199577317e-08
1794 6.44865840127018e-08
1795 6.44444101323138e-08
1796 6.44020215334606e-08
1797 6.43590847739794e-08
1798 6.43161156026473e-08
1799 6.42743806786861e-08
1800 6.42311230816262e-08
1801 6.41878045763988e-08
1802 6.41452694347677e-08
1803 6.41023480731917e-08
1804 6.40591909011334e-08
1805 6.40165288805505e-08
1806 6.39726341815639e-08
1807 6.39299502749324e-08
1808 6.38867462430248e-08
1809 6.38430538830725e-08
1810 6.37992819392252e-08
1811 6.37559800029663e-08
1812 6.37132659324635e-08
1813 6.36690688571306e-08
1814 6.36248876220158e-08
1815 6.35804625970238e-08
1816 6.35383909930454e-08
1817 6.34934688696909e-08
1818 6.34494402107499e-08
1819 6.34067986764464e-08
1820 6.33620315597661e-08
1821 6.33177902842341e-08
1822 6.32742041939416e-08
1823 6.32300032958888e-08
1824 6.31855213293342e-08
1825 6.31400309560881e-08
1826 6.30969767883016e-08
1827 6.3052974604183e-08
1828 6.3007719532937e-08
1829 6.29634446229765e-08
1830 6.29191344780899e-08
1831 6.28749879880708e-08
1832 6.2829098985695e-08
1833 6.27849729775143e-08
1834 6.27403344113375e-08
1835 6.26959070264554e-08
1836 6.26504930916205e-08
1837 6.26046370086897e-08
1838 6.25611179403407e-08
1839 6.25150779702821e-08
1840 6.24700937308376e-08
1841 6.24251917060725e-08
1842 6.23795242757907e-08
1843 6.2333725296515e-08
1844 6.22898775191061e-08
1845 6.2242720342276e-08
1846 6.21981210420231e-08
1847 6.21533781872685e-08
1848 6.21061856556082e-08
1849 6.20624227711808e-08
1850 6.20146665335497e-08
1851 6.19700589008509e-08
1852 6.19235669856444e-08
1853 6.18777134757664e-08
1854 6.18325506942519e-08
1855 6.17855304705373e-08
1856 6.17399468483271e-08
1857 6.16940486506401e-08
1858 6.16470723326934e-08
1859 6.1601297762337e-08
1860 6.15556288448005e-08
1861 6.15089735584995e-08
1862 6.14621134902293e-08
1863 6.1416335046971e-08
1864 6.13700586411525e-08
1865 6.13234706654531e-08
1866 6.12768565777522e-08
1867 6.12293353832172e-08
1868 6.11836057049508e-08
1869 6.11363493843164e-08
1870 6.10908154432543e-08
1871 6.10430286696051e-08
1872 6.09967538491851e-08
1873 6.09491384402361e-08
1874 6.09023131801223e-08
1875 6.08549836966787e-08
1876 6.08091297813473e-08
1877 6.07609636595896e-08
1878 6.07145391846586e-08
1879 6.06675413785673e-08
1880 6.06199230404059e-08
1881 6.05738498675734e-08
1882 6.05260139190378e-08
1883 6.04780508242087e-08
1884 6.0431181514442e-08
1885 6.03843030133611e-08
1886 6.03376012811019e-08
1887 6.02891754688528e-08
1888 6.02415411488089e-08
1889 6.01945420029004e-08
1890 6.01471130021736e-08
1891 6.00992131545297e-08
1892 6.00528446605964e-08
1893 6.00043250518212e-08
1894 5.99566107775118e-08
1895 5.99091665520746e-08
1896 5.9861494540403e-08
1897 5.98138020797556e-08
1898 5.97664913661866e-08
1899 5.97186347648382e-08
1900 5.967138203955e-08
1901 5.96229534322035e-08
1902 5.95760715431659e-08
1903 5.95281750226384e-08
1904 5.94799012487712e-08
1905 5.94315147903757e-08
1906 5.93848683494436e-08
1907 5.93364428222998e-08
1908 5.92891409172402e-08
1909 5.92406961312797e-08
1910 5.9193191808582e-08
1911 5.91448697804253e-08
1912 5.9097650680684e-08
1913 5.904894364539e-08
1914 5.90010000727226e-08
1915 5.89538589865768e-08
1916 5.89059388271807e-08
1917 5.88571268318461e-08
1918 5.88093838453929e-08
1919 5.87610243245607e-08
1920 5.87130454574591e-08
1921 5.86656144090369e-08
1922 5.86176444228315e-08
1923 5.85694885302246e-08
1924 5.85216042279235e-08
1925 5.84743013378741e-08
1926 5.84261857916601e-08
1927 5.83768689401687e-08
1928 5.83298196858983e-08
1929 5.82826058095343e-08
1930 5.82342937329727e-08
1931 5.81867239350942e-08
1932 5.81382547579778e-08
1933 5.80916336607729e-08
1934 5.80427709269138e-08
1935 5.79948235621686e-08
1936 5.79469966095481e-08
1937 5.79003529694866e-08
1938 5.78521729632797e-08
1939 5.78047035695306e-08
1940 5.77559439589592e-08
1941 5.77088784399216e-08
1942 5.76615298641947e-08
1943 5.76149919928604e-08
1944 5.75659022787889e-08
1945 5.75191440677081e-08
1946 5.74715234309409e-08
1947 5.74246460098848e-08
1948 5.73767349743015e-08
1949 5.73298869182004e-08
1950 5.72820779827232e-08
1951 5.72359872372807e-08
1952 5.71876066297605e-08
1953 5.71406906004768e-08
1954 5.70949392297848e-08
1955 5.70470616825247e-08
1956 5.70005604756396e-08
1957 5.69531874661244e-08
1958 5.69063798332436e-08
1959 5.6860067086717e-08
1960 5.68137518897061e-08
1961 5.67666839228487e-08
1962 5.67197691712096e-08
1963 5.66740676362087e-08
1964 5.66276173388225e-08
1965 5.65815991424579e-08
1966 5.65353133166191e-08
1967 5.64889379730538e-08
1968 5.64429644365205e-08
1969 5.63966369053759e-08
1970 5.63504777613133e-08
1971 5.63053401121394e-08
1972 5.62592542494578e-08
1973 5.62143603510812e-08
1974 5.61686187108279e-08
1975 5.61226332105313e-08
1976 5.60778266334161e-08
1977 5.60330586747426e-08
1978 5.59880136923319e-08
1979 5.59426551891562e-08
1980 5.5897071626454e-08
1981 5.58534221299922e-08
1982 5.58072744794025e-08
1983 5.57639954470446e-08
1984 5.57188894445559e-08
1985 5.56746803783703e-08
1986 5.56304593386514e-08
1987 5.55854905033293e-08
1988 5.5542199850489e-08
1989 5.54986131642288e-08
1990 5.54546014295276e-08
1991 5.54116930779891e-08
1992 5.53670424032759e-08
1993 5.53239520191262e-08
1994 5.52806552627239e-08
1995 5.52365189188109e-08
1996 5.51946997022767e-08
1997 5.5152307700812e-08
1998 5.51086638731491e-08
1999 5.50661505878125e-08
2000 5.50243767571956e-08
2001 5.49820178967764e-08
2002 5.49390003663497e-08
2003 5.48973433915911e-08
2004 5.4854939032456e-08
2005 5.48148342809363e-08
2006 5.47726771684687e-08
2007 5.47307022991106e-08
2008 5.46894180111224e-08
2009 5.46488475827722e-08
2010 5.46080760481793e-08
2011 5.4566294374947e-08
2012 5.45266657070442e-08
2013 5.44855340596229e-08
2014 5.44449344679343e-08
2015 5.44054479942879e-08
2016 5.43646098134509e-08
2017 5.43253523126452e-08
2018 5.42847181477235e-08
2019 5.42453417007316e-08
2020 5.42057704988608e-08
2021 5.41668317293187e-08
2022 5.41286403903385e-08
2023 5.40884211588377e-08
2024 5.40494156333615e-08
2025 5.40106437822985e-08
2026 5.39718988288307e-08
2027 5.39330162379059e-08
2028 5.38956598719054e-08
2029 5.38581058053822e-08
2030 5.38190834982188e-08
2031 5.3781607055825e-08
2032 5.37435739680348e-08
2033 5.37061002097161e-08
2034 5.36693056933935e-08
2035 5.3631589939318e-08
2036 5.35940872676832e-08
2037 5.35572274258378e-08
2038 5.35206492728868e-08
2039 5.34828260723152e-08
2040 5.34476814144647e-08
2041 5.3410597629977e-08
2042 5.33748049758032e-08
2043 5.33382545229166e-08
2044 5.33026050799634e-08
2045 5.32672254296429e-08
2046 5.32308542737781e-08
2047 5.31962048464507e-08
2048 5.31608243332649e-08
2049 5.31258195146656e-08
2050 5.30898798527524e-08
2051 5.30549933421831e-08
2052 5.30203998407863e-08
2053 5.2985993217014e-08
2054 5.29515429996685e-08
2055 5.29169668364027e-08
2056 5.28831917905315e-08
2057 5.28487110003084e-08
2058 5.28151453105252e-08
2059 5.27814904014434e-08
2060 5.2748062751462e-08
2061 5.27135792625089e-08
2062 5.26804917502233e-08
2063 5.26473177622222e-08
2064 5.26149816693078e-08
2065 5.25815238621163e-08
2066 5.25491308076553e-08
2067 5.25159870723435e-08
2068 5.24837269644252e-08
2069 5.24519081168684e-08
2070 5.24198160256617e-08
2071 5.23879083127454e-08
2072 5.23551711477666e-08
2073 5.2324445924512e-08
2074 5.22926171657723e-08
2075 5.22601020072067e-08
2076 5.22293503082416e-08
2077 5.21981515255732e-08
2078 5.21675140507938e-08
2079 5.21358721226051e-08
2080 5.21058686797637e-08
2081 5.2074713980943e-08
2082 5.20440313729331e-08
2083 5.20134607140399e-08
2084 5.19838234835923e-08
2085 5.19533010843176e-08
2086 5.19230764250977e-08
2087 5.18938365248189e-08
2088 5.1863919716677e-08
2089 5.18347032811839e-08
2090 5.18044679522767e-08
2091 5.17752752648981e-08
2092 5.17464562586056e-08
2093 5.17170811984435e-08
2094 5.16883411920688e-08
2095 5.1659155378303e-08
2096 5.16304008919555e-08
2097 5.16013841873608e-08
2098 5.1573401338878e-08
2099 5.15445594619912e-08
2100 5.15168834045454e-08
2101 5.14890581455596e-08
2102 5.14603827768134e-08
2103 5.14328090450711e-08
2104 5.14051132380899e-08
2105 5.13771901413662e-08
2106 5.13496982343931e-08
2107 5.13214599395795e-08
2108 5.12946043746965e-08
2109 5.12674675654523e-08
2110 5.12399049883427e-08
2111 5.12134583683377e-08
2112 5.1187233401695e-08
2113 5.11593533540911e-08
2114 5.11331277115445e-08
2115 5.11062574428678e-08
2116 5.10798440198634e-08
2117 5.10538929732185e-08
2118 5.1027342668597e-08
2119 5.10015116637064e-08
2120 5.09756476887446e-08
2121 5.0949671295264e-08
2122 5.09235394994256e-08
2123 5.08972808428432e-08
2124 5.08727300574385e-08
2125 5.0846201367527e-08
2126 5.08221706603251e-08
2127 5.07960543938424e-08
2128 5.07705658425728e-08
2129 5.07465655230632e-08
2130 5.07205517559228e-08
2131 5.06958312698202e-08
2132 5.06708794230093e-08
2133 5.0645950886441e-08
2134 5.06209040396222e-08
2135 5.05969733803902e-08
2136 5.05718781713682e-08
2137 5.05479071817305e-08
2138 5.05239440959926e-08
2139 5.04993572105761e-08
2140 5.04754421539744e-08
2141 5.04517017048123e-08
2142 5.04261907963155e-08
2143 5.0403394871168e-08
2144 5.03792342310128e-08
2145 5.03557150155487e-08
2146 5.03323918570331e-08
2147 5.03093502710605e-08
2148 5.02858441224774e-08
2149 5.0261534892293e-08
2150 5.02391469696661e-08
2151 5.02154790598119e-08
2152 5.01915164679012e-08
2153 5.01700670736405e-08
2154 5.01461614694776e-08
2155 5.01230792133711e-08
2156 5.01005100255192e-08
2157 5.00781239658465e-08
2158 5.00555631490762e-08
2159 5.00327298302849e-08
2160 5.00094611730617e-08
2161 4.99879705282424e-08
2162 4.99659846102674e-08
2163 4.99426742162079e-08
2164 4.99209716502413e-08
2165 4.98990443551506e-08
2166 4.98762193950064e-08
2167 4.98550369654183e-08
2168 4.98322295956477e-08
2169 4.98110704869603e-08
2170 4.97881384564636e-08
2171 4.97673993642422e-08
2172 4.97456038424104e-08
2173 4.97235115259897e-08
2174 4.97025324679434e-08
2175 4.96812474972685e-08
2176 4.96594972330122e-08
2177 4.96387341133442e-08
2178 4.9616895006821e-08
2179 4.95956719985813e-08
2180 4.95745456543517e-08
2181 4.95543001433063e-08
2182 4.95325022678017e-08
2183 4.95113767118305e-08
2184 4.94908748969181e-08
2185 4.94697444124448e-08
2186 4.94493753171454e-08
2187 4.94282713257022e-08
2188 4.94075009149775e-08
2189 4.9387330512296e-08
2190 4.936668009492e-08
2191 4.93462080930485e-08
2192 4.93258377289862e-08
2193 4.93054121539771e-08
2194 4.92854673677101e-08
2195 4.92647059573414e-08
2196 4.92443801674014e-08
2197 4.92248079830837e-08
2198 4.92040594939347e-08
2199 4.9184667150648e-08
2200 4.91640319992825e-08
2201 4.91450086599343e-08
2202 4.91245017011366e-08
2203 4.91050567492657e-08
2204 4.90852608860948e-08
2205 4.90654784961464e-08
2206 4.90458403579197e-08
2207 4.90260036238865e-08
2208 4.90067078269263e-08
2209 4.89871085065374e-08
2210 4.89679544113386e-08
2211 4.89477891454371e-08
2212 4.89283282871789e-08
2213 4.89092637532273e-08
2214 4.88903640643024e-08
2215 4.88710864909869e-08
2216 4.88526623971808e-08
2217 4.88325346506002e-08
2218 4.88135547880297e-08
2219 4.87948194392018e-08
2220 4.87751577944451e-08
2221 4.87567562759139e-08
2222 4.87369680710614e-08
2223 4.87190423870132e-08
2224 4.86999721376336e-08
2225 4.86807838107062e-08
2226 4.86622551396643e-08
2227 4.86435542561559e-08
2228 4.8624971673572e-08
2229 4.86060062359073e-08
2230 4.85883357130845e-08
2231 4.85699780221616e-08
2232 4.85508984247041e-08
2233 4.85326006569586e-08
2234 4.85140733927913e-08
2235 4.84957491257987e-08
2236 4.84770581197225e-08
2237 4.84593946494805e-08
2238 4.84414511614517e-08
2239 4.84229110693235e-08
2240 4.84045503030828e-08
2241 4.83869627334599e-08
2242 4.83681935801172e-08
2243 4.83499013430588e-08
2244 4.83325525153511e-08
2245 4.83150598777904e-08
2246 4.82959084435741e-08
2247 4.82790915938835e-08
2248 4.82617857855416e-08
2249 4.82434995263681e-08
2250 4.82254252758274e-08
2251 4.82081706305593e-08
2252 4.81900232678356e-08
2253 4.81727912648999e-08
2254 4.81557240279429e-08
2255 4.81376811856116e-08
2256 4.8120424666287e-08
2257 4.8103486007367e-08
2258 4.80847349617619e-08
2259 4.80691294231228e-08
2260 4.80503512516606e-08
2261 4.80342198252437e-08
2262 4.8016512246285e-08
2263 4.79995337459016e-08
2264 4.79818139051957e-08
2265 4.79657798111432e-08
2266 4.79471647034657e-08
2267 4.79317476109742e-08
2268 4.79138862097273e-08
2269 4.78973104316438e-08
2270 4.78799487821924e-08
2271 4.78641343906183e-08
2272 4.78469152294103e-08
2273 4.78300539406007e-08
2274 4.78134478538728e-08
2275 4.77969756338403e-08
2276 4.77804530039094e-08
2277 4.77646472041293e-08
2278 4.77472398956458e-08
2279 4.77302516155298e-08
2280 4.77147566844138e-08
2281 4.76975808800795e-08
2282 4.76822036250546e-08
2283 4.76656399905906e-08
2284 4.76485329312659e-08
2285 4.76340985220425e-08
2286 4.76162956224435e-08
2287 4.76002568468026e-08
2288 4.75828028521086e-08
2289 4.75673996831461e-08
2290 4.75513020297136e-08
2291 4.75343788908233e-08
2292 4.75189304154355e-08
2293 4.7502233949448e-08
2294 4.74853123044738e-08
2295 4.74703471278204e-08
2296 4.74532933241179e-08
2297 4.74375268875171e-08
2298 4.74210712164691e-08
2299 4.74044421041597e-08
2300 4.73883007998666e-08
2301 4.73719679456863e-08
2302 4.73556373470352e-08
2303 4.73401436185128e-08
2304 4.73232930096046e-08
2305 4.73073896913867e-08
2306 4.7291431824803e-08
2307 4.72753924802838e-08
2308 4.72587195550211e-08
2309 4.72423807034161e-08
2310 4.7226488390173e-08
2311 4.72109587579261e-08
2312 4.71942320219298e-08
2313 4.717850776359e-08
2314 4.71626581424189e-08
2315 4.71466281712907e-08
2316 4.713080803187e-08
2317 4.7114946652993e-08
2318 4.70985629563714e-08
2319 4.70828119452094e-08
2320 4.70667283689608e-08
2321 4.70509330172675e-08
2322 4.70356146511364e-08
2323 4.70185496568298e-08
2324 4.70029322801047e-08
2325 4.69878968329951e-08
2326 4.69712749318063e-08
2327 4.69562030485093e-08
2328 4.69400287932409e-08
2329 4.69245262406659e-08
2330 4.69088969867748e-08
2331 4.68937072128917e-08
2332 4.68771607038398e-08
2333 4.68620786993057e-08
2334 4.68455141824364e-08
2335 4.68304510601314e-08
2336 4.6815434395775e-08
2337 4.67983559113705e-08
2338 4.67832730266515e-08
2339 4.67682306615203e-08
2340 4.67525997014384e-08
2341 4.67358846081289e-08
2342 4.67215264219334e-08
2343 4.67053417119168e-08
2344 4.66899232787199e-08
2345 4.66752108922996e-08
2346 4.66589815961704e-08
2347 4.66441000837037e-08
2348 4.66276495818541e-08
2349 4.66130591476599e-08
2350 4.65980544017697e-08
2351 4.65826715112527e-08
2352 4.65662108690701e-08
2353 4.65521356494136e-08
2354 4.65359763306417e-08
2355 4.65210302564856e-08
2356 4.65060084606783e-08
2357 4.64900089593989e-08
2358 4.64752169904159e-08
2359 4.64597926148613e-08
2360 4.6445361427061e-08
2361 4.64290879942375e-08
2362 4.64145517398151e-08
2363 4.63993512012095e-08
2364 4.63837175708193e-08
2365 4.63688428071762e-08
2366 4.63552885658913e-08
2367 4.63395006140566e-08
2368 4.63248066986388e-08
2369 4.63104028463057e-08
2370 4.62962071741124e-08
2371 4.62818876125759e-08
2372 4.62685038882249e-08
2373 4.62542108881081e-08
2374 4.62407939170184e-08
2375 4.62264777159049e-08
2376 4.62109295158442e-08
2377 4.61950126249278e-08
2378 4.61800013975555e-08
2379 4.61648165379636e-08
2380 4.61497408026368e-08
2381 4.61343526598768e-08
2382 4.6118844005516e-08
2383 4.61037329557712e-08
2384 4.60883730140083e-08
2385 4.60740491323719e-08
2386 4.60587389086164e-08
2387 4.60438868192803e-08
2388 4.60287396975012e-08
2389 4.6013298304004e-08
2390 4.59986102132604e-08
2391 4.59846198004321e-08
2392 4.59686040525931e-08
2393 4.59543176374311e-08
2394 4.59396121588185e-08
2395 4.59245537607345e-08
2396 4.59094782625513e-08
2397 4.58947724921721e-08
2398 4.58798775815339e-08
2399 4.58650773911273e-08
2400 4.585034332516e-08
2401 4.58353505101705e-08
2402 4.58205807829515e-08
2403 4.58063277184451e-08
2404 4.57905149500171e-08
2405 4.57760817518249e-08
2406 4.57609371307122e-08
2407 4.57467562160474e-08
2408 4.57315649668999e-08
2409 4.57172441752363e-08
2410 4.57022117208439e-08
2411 4.56875167285098e-08
2412 4.56731408835509e-08
2413 4.56582628243218e-08
2414 4.56436690106088e-08
2415 4.56295570541165e-08
2416 4.56145084126725e-08
2417 4.55999333586199e-08
2418 4.5585774112844e-08
2419 4.55712242519724e-08
2420 4.55564203973857e-08
2421 4.55418406026808e-08
2422 4.55270458497026e-08
2423 4.55124166807153e-08
2424 4.54980189839027e-08
2425 4.54833420229228e-08
2426 4.54689883784276e-08
2427 4.54545760337766e-08
2428 4.54407563825754e-08
2429 4.5425246913311e-08
2430 4.5411217071134e-08
2431 4.53966958930963e-08
2432 4.53829893602986e-08
2433 4.53680827079417e-08
2434 4.53541030216442e-08
2435 4.53396434871856e-08
2436 4.53251534744403e-08
2437 4.53108913909261e-08
2438 4.52957903642748e-08
2439 4.52815858196232e-08
2440 4.5268323678016e-08
2441 4.5252770110249e-08
2442 4.52390050615925e-08
2443 4.52251441589446e-08
2444 4.52104246826401e-08
2445 4.51960382856775e-08
2446 4.51822486460252e-08
2447 4.51672473764653e-08
2448 4.51536773851124e-08
2449 4.51390581064359e-08
2450 4.51247113235631e-08
2451 4.51105056171741e-08
2452 4.50965722817998e-08
2453 4.5081966736582e-08
2454 4.50679842507462e-08
2455 4.50535688720777e-08
2456 4.50403727332294e-08
2457 4.50254745745227e-08
2458 4.5011680841256e-08
2459 4.49973593874553e-08
2460 4.4983548382227e-08
2461 4.49692376340849e-08
2462 4.49557711617032e-08
2463 4.49419129284756e-08
2464 4.4927672467665e-08
2465 4.49134916764571e-08
2466 4.48999108626502e-08
2467 4.48858140482677e-08
2468 4.48722395893775e-08
2469 4.48582630196981e-08
2470 4.48437843760807e-08
2471 4.48302370523734e-08
2472 4.48155606993517e-08
2473 4.48021357621897e-08
2474 4.47872720448217e-08
2475 4.4774151171989e-08
2476 4.4759919833437e-08
2477 4.47466000530472e-08
2478 4.47326331296516e-08
2479 4.47191123946755e-08
2480 4.47046555629438e-08
2481 4.46907516122863e-08
2482 4.4677101556756e-08
2483 4.46635775490645e-08
2484 4.46491319898712e-08
2485 4.46358768331212e-08
2486 4.46215929645888e-08
2487 4.46081123799402e-08
2488 4.45941733109301e-08
2489 4.45800764503623e-08
2490 4.45668001576305e-08
2491 4.45524220626936e-08
2492 4.45390765382303e-08
2493 4.45253346512686e-08
2494 4.45115970935106e-08
2495 4.44985472840553e-08
2496 4.44844794553756e-08
2497 4.44709816769251e-08
2498 4.44561626888884e-08
2499 4.44433183595105e-08
2500 4.44296565464963e-08
2501 4.4415978977419e-08
2502 4.44020137833068e-08
2503 4.43894033854697e-08
2504 4.43752871845948e-08
2505 4.4361579476293e-08
2506 4.43479653242651e-08
2507 4.43336195392607e-08
2508 4.43206032825216e-08
2509 4.43073842648634e-08
2510 4.42932063171586e-08
2511 4.42801855424779e-08
2512 4.42664250766001e-08
2513 4.42532712914279e-08
2514 4.42402756164473e-08
2515 4.42259831010539e-08
2516 4.42120420587333e-08
2517 4.41995622717517e-08
2518 4.41846667049717e-08
2519 4.41717115389206e-08
2520 4.41587094017759e-08
2521 4.41449098844693e-08
2522 4.41320136375101e-08
2523 4.41183556210145e-08
2524 4.41040346830235e-08
2525 4.40921923736681e-08
2526 4.40780429713516e-08
2527 4.40652516859874e-08
2528 4.40520183868642e-08
2529 4.40376632184769e-08
2530 4.402582405727e-08
2531 4.40117135596108e-08
2532 4.39992699032832e-08
2533 4.39856994807197e-08
2534 4.39727886551999e-08
2535 4.39589818801434e-08
2536 4.39464769270703e-08
2537 4.39320193339476e-08
2538 4.39203038860025e-08
2539 4.39068281372634e-08
2540 4.38931436883561e-08
2541 4.38801279749601e-08
2542 4.38670063478419e-08
2543 4.38539475102751e-08
2544 4.38404702838291e-08
2545 4.38278364807143e-08
2546 4.38145978469695e-08
2547 4.38019364790154e-08
2548 4.37887960311745e-08
2549 4.37756272066991e-08
2550 4.37631053358256e-08
2551 4.37497004477372e-08
2552 4.37366115897397e-08
2553 4.37236179340328e-08
2554 4.37107980264528e-08
2555 4.36972543427228e-08
2556 4.36856114518491e-08
2557 4.36721898473547e-08
2558 4.36591870143221e-08
2559 4.36468577129645e-08
2560 4.36330429802734e-08
2561 4.36209196903459e-08
2562 4.36075851633078e-08
2563 4.35950063559787e-08
2564 4.35821045874363e-08
2565 4.35701258949539e-08
2566 4.35568664729491e-08
2567 4.35438718600079e-08
2568 4.35312869173643e-08
2569 4.35180441187288e-08
2570 4.35049477771621e-08
2571 4.34927094377091e-08
2572 4.34796514487967e-08
2573 4.34664177222377e-08
2574 4.34551649646231e-08
2575 4.34411004706092e-08
2576 4.3428695953196e-08
2577 4.34163690934408e-08
2578 4.34028454250335e-08
2579 4.3390342515659e-08
2580 4.33779955151259e-08
2581 4.33653909728271e-08
2582 4.3352432675059e-08
2583 4.3340497935862e-08
2584 4.33275187732285e-08
2585 4.33152992145303e-08
2586 4.33023810411104e-08
2587 4.32898053206454e-08
2588 4.32771287293132e-08
2589 4.32644088517176e-08
2590 4.32524169906578e-08
2591 4.32403239445378e-08
2592 4.322783129318e-08
2593 4.32149780416058e-08
2594 4.32024143031118e-08
2595 4.31895962014206e-08
2596 4.31779786829178e-08
2597 4.31647185752393e-08
2598 4.31524545263517e-08
2599 4.31402595895136e-08
2600 4.31274063392717e-08
2601 4.31151697297683e-08
2602 4.31033302239481e-08
2603 4.30901026859942e-08
2604 4.30781640292643e-08
2605 4.30657044674376e-08
2606 4.30529735504503e-08
2607 4.30403858688866e-08
2608 4.3028594976402e-08
2609 4.30162294857706e-08
2610 4.3003296812838e-08
2611 4.29908574499649e-08
2612 4.29792330995937e-08
2613 4.29669191428506e-08
2614 4.29542717201858e-08
2615 4.29419714758072e-08
2616 4.29299317810106e-08
2617 4.29173326488286e-08
2618 4.29050251877783e-08
2619 4.2892042496856e-08
2620 4.28802331222666e-08
2621 4.28676602481914e-08
2622 4.28548913931071e-08
2623 4.28430063792451e-08
2624 4.28310592328618e-08
2625 4.28187968328775e-08
2626 4.28062961554954e-08
2627 4.27941379428365e-08
2628 4.27819666906082e-08
2629 4.27690289357407e-08
2630 4.27571997885234e-08
2631 4.27444108135333e-08
2632 4.27324895881931e-08
2633 4.27194549788012e-08
2634 4.27074864501886e-08
2635 4.26952229506394e-08
2636 4.26829434074794e-08
2637 4.26705643428171e-08
2638 4.26578930297072e-08
2639 4.26460872426926e-08
2640 4.26335239660514e-08
2641 4.26210364510293e-08
2642 4.26084480464883e-08
2643 4.2596654437288e-08
2644 4.25842601015081e-08
2645 4.25717090635214e-08
2646 4.25599055169368e-08
2647 4.25469093292552e-08
2648 4.25343802634703e-08
2649 4.25227882256962e-08
2650 4.25102284766776e-08
2651 4.24981913500488e-08
2652 4.24852486478056e-08
2653 4.24735875124149e-08
2654 4.24611567679811e-08
2655 4.24487702543885e-08
2656 4.24355110599794e-08
2657 4.24239869223353e-08
2658 4.2411216146121e-08
2659 4.23984245618847e-08
2660 4.23867014791579e-08
2661 4.23738416410746e-08
2662 4.23622779632815e-08
2663 4.23498261894473e-08
2664 4.23373767464152e-08
2665 4.23245063088107e-08
2666 4.23123904815803e-08
2667 4.23002100726766e-08
2668 4.22880070403142e-08
2669 4.22751732609417e-08
2670 4.22627615588311e-08
2671 4.22501848282852e-08
2672 4.22384421332467e-08
2673 4.22257051748698e-08
2674 4.22138066491495e-08
2675 4.22005564757466e-08
2676 4.21884330921163e-08
2677 4.2175881100226e-08
2678 4.21637529468555e-08
2679 4.21512490029752e-08
2680 4.21394473579806e-08
2681 4.21271631585451e-08
2682 4.21152116603096e-08
2683 4.21026234491695e-08
2684 4.20902948237156e-08
2685 4.20778487397744e-08
2686 4.2064796266672e-08
2687 4.2051913416552e-08
2688 4.20401034324502e-08
2689 4.20270689251989e-08
2690 4.20151527138479e-08
2691 4.20032904706602e-08
2692 4.19903134505351e-08
2693 4.19777935178889e-08
2694 4.19650008192107e-08
2695 4.19529220077042e-08
2696 4.19402330924523e-08
2697 4.19282851429337e-08
2698 4.19156716475744e-08
2699 4.1902983302311e-08
2700 4.18910407378181e-08
2701 4.18782563198494e-08
2702 4.18661814891585e-08
2703 4.18536841810813e-08
2704 4.18411931280005e-08
2705 4.182909206496e-08
2706 4.18161349862167e-08
2707 4.18040890608484e-08
2708 4.17911771639634e-08
2709 4.17790415969677e-08
2710 4.17664910117299e-08
2711 4.17531385912273e-08
2712 4.17415714610847e-08
2713 4.17292824741455e-08
2714 4.17168743143126e-08
2715 4.17045368605873e-08
2716 4.16917211456536e-08
2717 4.16787570953758e-08
2718 4.16665293183627e-08
2719 4.16541465622089e-08
2720 4.16415995450059e-08
2721 4.16289121820679e-08
2722 4.16165204113472e-08
2723 4.16039079798036e-08
2724 4.15916994227494e-08
2725 4.15797475363799e-08
2726 4.1566742515764e-08
2727 4.15538528171222e-08
2728 4.15415080763371e-08
2729 4.15296063738246e-08
2730 4.15173279799674e-08
2731 4.15050022608554e-08
2732 4.14925208234163e-08
2733 4.14798965424623e-08
2734 4.14676381326196e-08
2735 4.14552597229889e-08
2736 4.14424663379709e-08
2737 4.14304340625726e-08
2738 4.14174918970112e-08
2739 4.14057068593987e-08
2740 4.13925598221621e-08
2741 4.13800789018648e-08
2742 4.13681733657523e-08
2743 4.13556539975435e-08
2744 4.13434480208696e-08
2745 4.1331550160395e-08
2746 4.1319009320917e-08
2747 4.13066625031266e-08
2748 4.12935176410389e-08
2749 4.12822448880856e-08
2750 4.12685758914844e-08
2751 4.12570306549398e-08
2752 4.12434658692007e-08
2753 4.12322424345746e-08
2754 4.12192425771618e-08
2755 4.12076804152672e-08
2756 4.11950424721308e-08
2757 4.11829853794732e-08
2758 4.11702385993973e-08
2759 4.11579741377288e-08
2760 4.11451795565565e-08
2761 4.11335991792328e-08
2762 4.1120651765203e-08
2763 4.11080460729352e-08
2764 4.10957472321005e-08
2765 4.10835684212518e-08
2766 4.10709623424044e-08
2767 4.10582871823717e-08
2768 4.10466252838138e-08
2769 4.10341751424514e-08
2770 4.10208197989537e-08
2771 4.10089727513885e-08
2772 4.09972216903753e-08
2773 4.09850751170726e-08
2774 4.09728845043311e-08
2775 4.09602753597138e-08
2776 4.09469837761023e-08
2777 4.09353528028067e-08
2778 4.09229914399845e-08
2779 4.09113712003251e-08
2780 4.08992302041167e-08
2781 4.08861555891704e-08
2782 4.08743201549822e-08
2783 4.08627189680821e-08
2784 4.0848803514848e-08
2785 4.08368410051985e-08
2786 4.08255750634634e-08
2787 4.08127570483696e-08
2788 4.08005704790604e-08
2789 4.0788595091712e-08
2790 4.07759575409283e-08
2791 4.07637242014758e-08
2792 4.07516966345334e-08
2793 4.0739254711708e-08
2794 4.07272464026942e-08
2795 4.07146543239811e-08
2796 4.07028384783459e-08
2797 4.06896192790551e-08
2798 4.06784708899544e-08
2799 4.06658406817417e-08
2800 4.06540577382319e-08
2801 4.06423291219316e-08
2802 4.06298778514724e-08
2803 4.06169699129766e-08
2804 4.06058857758129e-08
2805 4.05936013088137e-08
2806 4.05810468739887e-08
2807 4.05696220919793e-08
2808 4.05576927791085e-08
2809 4.05439970763144e-08
2810 4.05323944945302e-08
2811 4.05200678907924e-08
2812 4.05086721475545e-08
2813 4.0496384572597e-08
2814 4.04848195847407e-08
2815 4.04722865878782e-08
2816 4.0460283568855e-08
2817 4.04483141345224e-08
2818 4.04362710793027e-08
2819 4.04240102529485e-08
2820 4.0412064130857e-08
2821 4.03998855118548e-08
2822 4.03888431674826e-08
2823 4.03762604832547e-08
2824 4.03656034921873e-08
2825 4.0352583246861e-08
2826 4.0342139850269e-08
2827 4.03294028559209e-08
2828 4.03183848878275e-08
2829 4.03067139536084e-08
2830 4.02948764024469e-08
2831 4.0283811757913e-08
2832 4.02728332040692e-08
2833 4.02625084137131e-08
2834 4.02512584269932e-08
2835 4.0242941060642e-08
2836 4.02339554761433e-08
2837 4.02296912840328e-08
2838 4.02374610457557e-08
2839 4.02628245272929e-08
2840 4.02323411170524e-08
2841 4.02164601793764e-08
2842 4.02059361788165e-08
2843 4.01935290930133e-08
2844 4.01819509689982e-08
2845 4.01696976133348e-08
2846 4.01580240616539e-08
2847 4.01473056415025e-08
2848 4.01344246605539e-08
2849 4.01226880013539e-08
2850 4.01103572138517e-08
2851 4.00996980542967e-08
2852 4.00876371871028e-08
2853 4.00748352282765e-08
2854 4.00639274393821e-08
2855 4.00520880290411e-08
2856 4.00402249767229e-08
2857 4.00284003143625e-08
2858 4.00164538125747e-08
2859 4.00050062832058e-08
2860 3.99933944585928e-08
2861 3.99820312346755e-08
2862 3.99700721931406e-08
2863 3.99589848929516e-08
2864 3.99470016732018e-08
2865 3.99354187183842e-08
2866 3.99240926995947e-08
2867 3.99117472660304e-08
2868 3.99013704093232e-08
2869 3.988907677277e-08
2870 3.98773039171907e-08
2871 3.98658189477708e-08
2872 3.98550590825497e-08
2873 3.98432726880227e-08
2874 3.98315873570976e-08
2875 3.98204207909814e-08
2876 3.98090340802959e-08
2877 3.97968808614202e-08
2878 3.97857822436176e-08
2879 3.97740736852725e-08
2880 3.97624802201957e-08
2881 3.97514675634092e-08
2882 3.97403365195448e-08
2883 3.97290203959511e-08
2884 3.97179854310092e-08
2885 3.97060666017524e-08
2886 3.96949680392389e-08
2887 3.96838458316395e-08
2888 3.96721209758866e-08
2889 3.96608068182758e-08
2890 3.96499119903471e-08
2891 3.96383766958497e-08
2892 3.96273150398141e-08
2893 3.96162319440396e-08
2894 3.96040918120288e-08
2895 3.95928828178516e-08
2896 3.95817106091378e-08
2897 3.95706966969112e-08
2898 3.95603851630622e-08
2899 3.95484433237669e-08
2900 3.95362661962384e-08
2901 3.95264668764383e-08
2902 3.95145608014236e-08
2903 3.95037918621277e-08
2904 3.94925382734002e-08
2905 3.94815547635208e-08
2906 3.9470015464893e-08
2907 3.94591748225093e-08
2908 3.94480473928649e-08
2909 3.94368272964574e-08
2910 3.94261952425534e-08
2911 3.94148467548483e-08
2912 3.9403194054044e-08
2913 3.939273379161e-08
2914 3.93808431500275e-08
2915 3.93700696510457e-08
2916 3.93589245244463e-08
2917 3.93484338006012e-08
2918 3.9337247526916e-08
2919 3.93256559092503e-08
2920 3.9314933372836e-08
2921 3.93042381008346e-08
2922 3.92926876930932e-08
2923 3.92810392988441e-08
2924 3.92708789513918e-08
2925 3.92599608161071e-08
2926 3.92489154863451e-08
2927 3.92380735125819e-08
2928 3.92271702727154e-08
2929 3.92161397719803e-08
2930 3.92053606621534e-08
2931 3.91944719582149e-08
2932 3.91828311403497e-08
2933 3.91724171775465e-08
2934 3.91612014924991e-08
2935 3.91508503194693e-08
2936 3.9139966803603e-08
2937 3.91286773147037e-08
2938 3.91177710548085e-08
2939 3.91070571734708e-08
2940 3.90958550844367e-08
2941 3.90849509313007e-08
2942 3.90733053803327e-08
2943 3.90639627285694e-08
2944 3.90521529378685e-08
2945 3.90414023840879e-08
2946 3.90308778444037e-08
2947 3.90195498007895e-08
2948 3.90096699363074e-08
2949 3.89981193644751e-08
2950 3.89878317024195e-08
2951 3.89770944528856e-08
2952 3.89661113073814e-08
2953 3.89551701120983e-08
2954 3.89447845376978e-08
2955 3.89334362476124e-08
2956 3.89231709092552e-08
2957 3.89119978940755e-08
2958 3.89014873671822e-08
2959 3.88906941566347e-08
2960 3.88802378996633e-08
2961 3.88685537262568e-08
2962 3.88582126511494e-08
2963 3.88480766884136e-08
2964 3.88367776360532e-08
2965 3.88259215835518e-08
2966 3.88151415111615e-08
2967 3.88050971873444e-08
2968 3.87940379458218e-08
2969 3.87838748268088e-08
2970 3.8772952659194e-08
2971 3.87621821404949e-08
2972 3.8751419742189e-08
2973 3.8740679893845e-08
2974 3.87301060884582e-08
2975 3.87192923381186e-08
2976 3.87092219253926e-08
2977 3.86982112625667e-08
2978 3.86876196805108e-08
2979 3.86771214988535e-08
2980 3.86668252885247e-08
2981 3.86560687164472e-08
2982 3.86456861072304e-08
2983 3.86347831522471e-08
2984 3.86244011123527e-08
2985 3.8613965930967e-08
2986 3.86033736525793e-08
2987 3.85928350701281e-08
2988 3.85817577421843e-08
2989 3.85717232795901e-08
2990 3.85609983570845e-08
2991 3.85500933730576e-08
2992 3.85405116469784e-08
2993 3.8529386935604e-08
2994 3.8518541896071e-08
2995 3.8508867390652e-08
2996 3.84976171972085e-08
2997 3.84879344430544e-08
2998 3.84773499264579e-08
2999 3.84664941417423e-08
3000 2.11408893259768e-08
3001 2.11082714002553e-08
3002 2.12747563767546e-08
3003 2.13531464478001e-08
3004 2.13815352757551e-08
3005 2.13872734388665e-08
3006 2.13847154552638e-08
3007 2.1379760114737e-08
3008 2.13743665860022e-08
3009 2.13690575125436e-08
3010 2.13639002498156e-08
3011 2.13589951990056e-08
3012 2.13542938662936e-08
3013 2.13497602631341e-08
3014 2.13453458434709e-08
3015 2.13412000558133e-08
3016 2.13371066373758e-08
3017 2.13330581326221e-08
3018 2.13291657474857e-08
3019 2.13253661611734e-08
3020 2.13215869918626e-08
3021 2.131791343013e-08
3022 2.13142751177009e-08
3023 2.13106811211561e-08
3024 2.13070900816348e-08
3025 2.13035795070815e-08
3026 2.13001032008386e-08
3027 2.12967478219195e-08
3028 2.12932935252375e-08
3029 2.128996294809e-08
3030 2.12865850288679e-08
3031 2.12832530605001e-08
3032 2.12799276129161e-08
3033 2.1276671593351e-08
3034 2.12734452683105e-08
3035 2.12701900998979e-08
3036 2.12669240460261e-08
3037 2.12637708766894e-08
3038 2.12605746823247e-08
3039 2.12574248614761e-08
3040 2.12542377527325e-08
3041 2.12511340507704e-08
3042 2.12479813312405e-08
3043 2.12448368157037e-08
3044 2.12417482480243e-08
3045 2.12386590556779e-08
3046 2.1235586574464e-08
3047 2.1232563751139e-08
3048 2.12294939284097e-08
3049 2.12264330238798e-08
3050 2.12233960452113e-08
3051 2.12203428787694e-08
3052 2.12174134703313e-08
3053 2.12143244812668e-08
3054 2.12113584865947e-08
3055 2.12083232017379e-08
3056 2.12054019342323e-08
3057 2.12024224751084e-08
3058 2.11994038351593e-08
3059 2.11964992895552e-08
3060 2.11935631489468e-08
3061 2.11905668482948e-08
3062 2.11876738498984e-08
3063 2.11847441984325e-08
3064 2.11818162407784e-08
3065 2.11788909852406e-08
3066 2.11759897225106e-08
3067 2.11730572663993e-08
3068 2.11701581528945e-08
3069 2.11673172058058e-08
3070 2.11644481814544e-08
3071 2.11615782717001e-08
3072 2.11587139300473e-08
3073 2.11558807508561e-08
3074 2.11530293685036e-08
3075 2.11501585457019e-08
3076 2.11473557761299e-08
3077 2.11445337081617e-08
3078 2.11416520725649e-08
3079 2.11387975133648e-08
3080 2.11360237719616e-08
3081 2.1133213034874e-08
3082 2.11304266652945e-08
3083 2.11276426451135e-08
3084 2.11248236931527e-08
3085 2.11220866959683e-08
3086 2.11193315147185e-08
3087 2.11165261402302e-08
3088 2.11137978070597e-08
3089 2.11110474976905e-08
3090 2.11082490993997e-08
3091 2.11055172313901e-08
3092 2.11026869427755e-08
3093 2.1099993781859e-08
3094 2.10972608216675e-08
3095 2.10944718113115e-08
3096 2.10917512644682e-08
3097 2.10890547827081e-08
3098 2.10863622669422e-08
3099 2.108364468556e-08
3100 2.10809378601295e-08
3101 2.10782674371912e-08
3102 2.10755454305711e-08
3103 2.10728331727084e-08
3104 2.10701576235373e-08
3105 2.1067493206739e-08
3106 2.10648593980123e-08
3107 2.10621265614996e-08
3108 2.10594698613065e-08
3109 2.1056776884909e-08
3110 2.10541419196075e-08
3111 2.10515245030463e-08
3112 2.10488444811197e-08
3113 2.10462212445472e-08
3114 2.10435805210185e-08
3115 2.10409217222263e-08
3116 2.10382836225431e-08
3117 2.10357198630606e-08
3118 2.10330824226279e-08
3119 2.10304320970578e-08
3120 2.10278328351965e-08
3121 2.10251405022843e-08
3122 2.10225248859497e-08
3123 2.10199570154779e-08
3124 2.10173627152588e-08
3125 2.10147197529098e-08
3126 2.10121266634444e-08
3127 2.10095150480205e-08
3128 2.10069630547371e-08
3129 2.10043544046634e-08
3130 2.10017641272264e-08
3131 2.09992095356881e-08
3132 2.09965763226516e-08
3133 2.09941350508003e-08
3134 2.09914582466886e-08
3135 2.09888857559681e-08
3136 2.09863646900565e-08
3137 2.09837943352387e-08
3138 2.09812835321732e-08
3139 2.09787125765581e-08
3140 2.09761811574838e-08
3141 2.09735882230055e-08
3142 2.09710900851978e-08
3143 2.09685341254762e-08
3144 2.0966013549728e-08
3145 2.09634856381258e-08
3146 2.09609415000034e-08
3147 2.09583931422008e-08
3148 2.09558537045962e-08
3149 2.09533916614157e-08
3150 2.0950898469041e-08
3151 2.09483531466437e-08
3152 2.09458610213598e-08
3153 2.09433297643224e-08
3154 2.09408968410707e-08
3155 2.09382779305378e-08
3156 2.09358344011035e-08
3157 2.09333498568109e-08
3158 2.09308747712855e-08
3159 2.09284214750527e-08
3160 2.09258693638081e-08
3161 2.09234161616667e-08
3162 2.09209179483083e-08
3163 2.09184750699642e-08
3164 2.09159838528428e-08
3165 2.09135234692237e-08
3166 2.09110342177521e-08
3167 2.09085636620476e-08
3168 2.09061395899379e-08
3169 2.09036772642612e-08
3170 2.09011808828263e-08
3171 2.0898786262491e-08
3172 2.08962692530568e-08
3173 2.08938697515149e-08
3174 2.08914228895796e-08
3175 2.08889742012164e-08
3176 2.08864686531696e-08
3177 2.08841149578198e-08
3178 2.08816259122946e-08
3179 2.08792160969473e-08
3180 2.08767671059928e-08
3181 2.08744048360021e-08
3182 2.08719543404734e-08
3183 2.08695299555584e-08
3184 2.08671742153999e-08
3185 2.08647492468961e-08
3186 2.08623082206816e-08
3187 2.08598638568591e-08
3188 2.08574963784791e-08
3189 2.08551257597778e-08
3190 2.08526947682031e-08
3191 2.0850287403118e-08
3192 2.08479205527357e-08
3193 2.08454695793669e-08
3194 2.08430721369557e-08
3195 2.08406745205725e-08
3196 2.08383163307624e-08
3197 2.08358732870506e-08
3198 2.08335306426388e-08
3199 2.08311355462398e-08
3200 2.08287711689903e-08
3201 2.08263612097026e-08
3202 2.08240240780477e-08
3203 2.0821695643547e-08
3204 2.08193330575868e-08
3205 2.08169826286242e-08
3206 2.08145840521201e-08
3207 2.08122252197684e-08
3208 2.08099015808205e-08
3209 2.08074944232917e-08
3210 2.08052259614289e-08
3211 2.08028696675466e-08
3212 2.08004786146354e-08
3213 2.07981937669355e-08
3214 2.07958117757201e-08
3215 2.07935053604391e-08
3216 2.07911505582525e-08
3217 2.07888115725252e-08
3218 2.07864645208611e-08
3219 2.07841278209719e-08
3220 2.07818124759451e-08
3221 2.07794938296146e-08
3222 2.07771728713557e-08
3223 2.07748690653209e-08
3224 2.07725645343659e-08
3225 2.07702342066574e-08
3226 2.07679109783254e-08
3227 2.07656199727735e-08
3228 2.07633598324763e-08
3229 2.07610228935562e-08
3230 2.07586948599014e-08
3231 2.07564215966571e-08
3232 2.07540719094346e-08
3233 2.07517907808152e-08
3234 2.07494881960812e-08
3235 2.07471467149611e-08
3236 2.07449426433137e-08
3237 2.07425822061902e-08
3238 2.07403163912101e-08
3239 2.07380372735932e-08
3240 2.07357642930117e-08
3241 2.07334923947866e-08
3242 2.07311990230163e-08
3243 2.07289416462308e-08
3244 2.0726621168643e-08
3245 2.0724375544956e-08
3246 2.07221274991509e-08
3247 2.07198387523477e-08
3248 2.07175044568131e-08
3249 2.07153047950825e-08
3250 2.07130165456038e-08
3251 2.07107081791613e-08
3252 2.07085094311443e-08
3253 2.07062527318169e-08
3254 2.07039369853335e-08
3255 2.07017264740594e-08
3256 2.06994066827559e-08
3257 2.06971902619868e-08
3258 2.06949463719686e-08
3259 2.06927055339534e-08
3260 2.0690415579061e-08
3261 2.06881844384554e-08
3262 2.06859920948155e-08
3263 2.06837471412591e-08
3264 2.06814759959872e-08
3265 2.06792749283258e-08
3266 2.06770685857172e-08
3267 2.06748375904953e-08
3268 2.06725748281844e-08
3269 2.0670374945575e-08
3270 2.06680873645615e-08
3271 2.06658755863565e-08
3272 2.06636813197547e-08
3273 2.06613937147604e-08
3274 2.06592403589889e-08
3275 2.06569996513695e-08
3276 2.06547765645992e-08
3277 2.06526250549621e-08
3278 2.06503800162516e-08
3279 2.06481940678849e-08
3280 2.06459145583593e-08
3281 2.06437879683441e-08
3282 2.06415723899567e-08
3283 2.06394130801146e-08
3284 2.06371779316594e-08
3285 2.06350062737504e-08
3286 2.06327268005846e-08
3287 2.06305757456948e-08
3288 2.06283244700844e-08
3289 2.06261803425423e-08
3290 2.06239659360508e-08
3291 2.06217957138821e-08
3292 2.06196023571081e-08
3293 2.06174135569115e-08
3294 2.0615231390575e-08
3295 2.06130530029935e-08
3296 2.06108656131687e-08
3297 2.06086574416342e-08
3298 2.06064650155047e-08
3299 2.0604314613093e-08
3300 2.06020932743689e-08
3301 2.05999101550169e-08
3302 2.05977618762399e-08
3303 2.05956025405296e-08
3304 2.0593390127388e-08
3305 2.05912135622932e-08
3306 2.05890719740531e-08
3307 2.05868455298019e-08
3308 2.05846641038177e-08
3309 2.05825775144164e-08
3310 2.05803531639903e-08
3311 2.05782421162093e-08
3312 2.05761052555764e-08
3313 2.05739647566316e-08
3314 2.05717832537089e-08
3315 2.05696053583448e-08
3316 2.05674612384632e-08
3317 2.05652806003487e-08
3318 2.05631792236094e-08
3319 2.05609896322123e-08
3320 2.05589108211446e-08
3321 2.0556742921396e-08
3322 2.05545466425705e-08
3323 2.05524675443991e-08
3324 2.05503091263437e-08
3325 2.05481236876803e-08
3326 2.05460908077604e-08
3327 2.05439003993502e-08
3328 2.05418206640218e-08
3329 2.05396286345749e-08
3330 2.05375744977698e-08
3331 2.05353608081493e-08
3332 2.0533291504421e-08
3333 2.05311614252524e-08
3334 2.05290273074254e-08
3335 2.05269064443847e-08
3336 2.05248472276987e-08
3337 2.0522712063431e-08
3338 2.05205335928049e-08
3339 2.05184333605946e-08
3340 2.05163661772256e-08
3341 2.05142246231804e-08
3342 2.05121337106262e-08
3343 2.05099776372508e-08
3344 2.05078516385959e-08
3345 2.0505716801289e-08
3346 2.05036650270385e-08
3347 2.05015698134248e-08
3348 2.04994466536657e-08
3349 2.04973580672507e-08
3350 2.04952246323775e-08
3351 2.049316038788e-08
3352 2.04910830693961e-08
3353 2.0488963785148e-08
3354 2.04868859186025e-08
3355 2.04847804708974e-08
3356 2.04827018092657e-08
3357 2.04806579801597e-08
3358 2.04785412510899e-08
3359 2.0476469179298e-08
3360 2.04743459369938e-08
3361 2.04722978112692e-08
3362 2.0470235604475e-08
3363 2.04681766767245e-08
3364 2.04660284396363e-08
3365 2.04639949488161e-08
3366 2.04619098344017e-08
3367 2.04598581342585e-08
3368 2.04578028980551e-08
3369 2.04557021080687e-08
3370 2.04536254182486e-08
3371 2.04515255949889e-08
3372 2.04495448197783e-08
3373 2.04474418022405e-08
3374 2.04453606812427e-08
3375 2.04433044749819e-08
3376 2.04412041320268e-08
3377 2.04392099483752e-08
3378 2.04371524346603e-08
3379 2.04350366999617e-08
3380 2.04330209515602e-08
3381 2.04309815934889e-08
3382 2.04289098938437e-08
3383 2.04268299076049e-08
3384 2.0424840063149e-08
3385 2.04227703606841e-08
3386 2.04207376894305e-08
3387 2.04186838805298e-08
3388 2.04166322160249e-08
3389 2.04146003700556e-08
3390 2.04125465531613e-08
3391 2.04105450022918e-08
3392 2.04084944559479e-08
3393 2.04064483063093e-08
3394 2.040438363482e-08
3395 2.04023858627056e-08
3396 2.0400314227953e-08
3397 2.03982714627182e-08
3398 2.039633091383e-08
3399 2.0394205407781e-08
3400 2.03922551036406e-08
3401 2.03901929690664e-08
3402 2.0388201591115e-08
3403 2.0386167810249e-08
3404 2.03841712202446e-08
3405 2.03821842252871e-08
3406 2.03801009713844e-08
3407 2.03780900409622e-08
3408 2.03760331548564e-08
3409 2.03740245984796e-08
3410 2.03719567070659e-08
3411 2.03699879173236e-08
3412 2.03679254177636e-08
3413 2.03659245247012e-08
3414 2.03638614489909e-08
3415 2.03618601464783e-08
3416 2.03598466405608e-08
3417 2.03578191994569e-08
3418 2.03558014820193e-08
3419 2.03537912760732e-08
3420 2.03518045051587e-08
3421 2.0349807560327e-08
3422 2.03478609813179e-08
3423 2.03458025663794e-08
3424 2.03437853042998e-08
3425 2.03417678235063e-08
3426 2.0339760123389e-08
3427 2.03377479973676e-08
3428 2.03357940645743e-08
3429 2.03337350539456e-08
3430 2.0331748909197e-08
3431 2.03297528125757e-08
3432 2.0327759524652e-08
3433 2.03257969040305e-08
3434 2.03237715966642e-08
3435 2.03217748540596e-08
3436 2.03197618461881e-08
3437 2.03178243222135e-08
3438 2.03158468791154e-08
3439 2.03138550113335e-08
3440 2.03118055893681e-08
3441 2.030982322615e-08
3442 2.03078372899013e-08
3443 2.03059383462878e-08
3444 2.03038830056124e-08
3445 2.03019116922776e-08
3446 2.0299918486899e-08
3447 2.02978901407413e-08
3448 2.02958975917822e-08
3449 2.02938901610605e-08
3450 2.02919549442404e-08
3451 2.02899945446755e-08
3452 2.02880592740096e-08
3453 2.02860731766674e-08
3454 2.02840607563815e-08
3455 2.02820926930025e-08
3456 2.02801091217508e-08
3457 2.02781236053884e-08
3458 2.02761533876772e-08
3459 2.02742281606993e-08
3460 2.02722122051857e-08
3461 2.02702175730041e-08
3462 2.02682674907417e-08
3463 2.02663054233998e-08
3464 2.02643686427195e-08
3465 2.02623471755037e-08
3466 2.02603614790631e-08
3467 2.02584312981591e-08
3468 2.02564996205079e-08
3469 2.0254510525286e-08
3470 2.02525713397517e-08
3471 2.02505682149745e-08
3472 2.02486636748378e-08
3473 2.02466878284069e-08
3474 2.02447679449325e-08
3475 2.02428034463686e-08
3476 2.02408116000141e-08
3477 2.02388811448295e-08
3478 2.02369068964536e-08
3479 2.02349684690906e-08
3480 2.02330245450688e-08
3481 2.02310837363884e-08
3482 2.02291225567808e-08
3483 2.02272117825752e-08
3484 2.02251818576249e-08
3485 2.02233039389155e-08
3486 2.02213164057219e-08
3487 2.0219377862507e-08
3488 2.02174638291308e-08
3489 2.02154580996594e-08
3490 2.02135280958915e-08
3491 2.02116305712541e-08
3492 2.02096570184329e-08
3493 2.02076688706754e-08
3494 2.02058190126864e-08
3495 2.02038633260182e-08
3496 2.02019019463484e-08
3497 2.02000067840435e-08
3498 2.01980896647469e-08
3499 2.01961228046277e-08
3500 2.01941875865308e-08
3501 2.01922964743195e-08
3502 2.01903841440254e-08
3503 2.01884230610627e-08
3504 2.0186520056531e-08
3505 2.01844701148146e-08
3506 2.01826125346138e-08
3507 2.01806859262965e-08
3508 2.01787911201512e-08
3509 2.01768803371194e-08
3510 2.01749247356053e-08
3511 2.0173027900916e-08
3512 2.01710562380275e-08
3513 2.01692026353117e-08
3514 2.0167266715998e-08
3515 2.01653433561932e-08
3516 2.01634254714533e-08
3517 2.0161491318782e-08
3518 2.01595732577386e-08
3519 2.01576774696566e-08
3520 2.01557670255204e-08
3521 2.01538807949597e-08
3522 2.01519187180699e-08
3523 2.01500641132668e-08
3524 2.0148134759701e-08
3525 2.01462171318112e-08
3526 2.01443058627793e-08
3527 2.01423918957389e-08
3528 2.01405453231773e-08
3529 2.01386062432807e-08
3530 2.01367216912107e-08
3531 2.01348015586578e-08
3532 2.01328655786703e-08
3533 2.01309418903506e-08
3534 2.0129081371445e-08
3535 2.01271780796986e-08
3536 2.01252628324045e-08
3537 2.01234130200456e-08
3538 2.01214758301371e-08
3539 2.01195501485785e-08
3540 2.01176676659087e-08
3541 2.01156881620701e-08
3542 2.01138739912565e-08
3543 2.01119219651602e-08
3544 2.01100503813834e-08
3545 2.01081669231051e-08
3546 2.01062399016183e-08
3547 2.01043595968176e-08
3548 2.01024724951204e-08
3549 2.01006030050022e-08
3550 2.00986749632204e-08
3551 2.00968095451226e-08
3552 2.00949258642447e-08
3553 2.00930840382196e-08
3554 2.00911570047979e-08
3555 2.00892963730936e-08
3556 2.0087368204913e-08
3557 2.00855096763042e-08
3558 2.00836333287269e-08
3559 2.00817507970408e-08
3560 2.00798512027811e-08
3561 2.00780020387925e-08
3562 2.00760480377205e-08
3563 2.00742218517957e-08
3564 2.00723622609256e-08
3565 2.00705012088354e-08
3566 2.00685945483947e-08
3567 2.00666576078423e-08
3568 2.00648470927711e-08
3569 2.00629999033697e-08
3570 2.0061117117498e-08
3571 2.00592549097767e-08
3572 2.00574140092891e-08
3573 2.00555050069995e-08
3574 2.0053709319845e-08
3575 2.00517557997215e-08
3576 2.00499250876396e-08
3577 2.0048042384313e-08
3578 2.00461854015233e-08
3579 2.00443242375226e-08
3580 2.00424559076429e-08
3581 2.00406244456608e-08
3582 2.00387066383589e-08
3583 2.00368698690667e-08
3584 2.00349858615612e-08
3585 2.00331720197067e-08
3586 2.00313095882754e-08
3587 2.00294643914467e-08
3588 2.00276190316373e-08
3589 2.00257212292776e-08
3590 2.00238759939242e-08
3591 2.0022039005696e-08
3592 2.00201907732955e-08
3593 2.00182820223604e-08
3594 2.00164790283286e-08
3595 2.00146573942073e-08
3596 2.00127912508008e-08
3597 2.00109190416908e-08
3598 2.00090827586763e-08
3599 2.0007262528321e-08
3600 2.00054067748257e-08
3601 2.0003584150774e-08
3602 2.00016791347957e-08
3603 1.99998756562625e-08
3604 1.99980436479952e-08
3605 1.99962583296576e-08
3606 1.99944197051827e-08
3607 1.99925277492574e-08
3608 1.99906898891156e-08
3609 1.99888743083743e-08
3610 1.99870007702163e-08
3611 1.99852152488744e-08
3612 1.99833890746626e-08
3613 1.99815517181734e-08
3614 1.99797232585674e-08
3615 1.99778518033544e-08
3616 1.99759799131005e-08
3617 1.99742034333039e-08
3618 1.9972388401901e-08
3619 1.99705364690161e-08
3620 1.99686814511546e-08
3621 1.99669064090413e-08
3622 1.99650656180217e-08
3623 1.9963228719444e-08
3624 1.99614328540987e-08
3625 1.99595691350307e-08
3626 1.99577477451029e-08
3627 1.99559462055743e-08
3628 1.99540991256963e-08
3629 1.99522255717177e-08
3630 1.9950469429153e-08
3631 1.99485730057458e-08
3632 1.99467724726343e-08
3633 1.9944953394746e-08
3634 1.99430941412726e-08
3635 1.99413105988477e-08
3636 1.99394913327766e-08
3637 1.99376451793798e-08
3638 1.99358355606472e-08
3639 1.99340363055689e-08
3640 1.99322161414939e-08
3641 1.99303780810678e-08
3642 1.99285275228056e-08
3643 1.99267329132891e-08
3644 1.99249533305856e-08
3645 1.99230927979688e-08
3646 1.99212653565484e-08
3647 1.99194941654102e-08
3648 1.99176919752908e-08
3649 1.99158270554056e-08
3650 1.99140040454404e-08
3651 1.99122834560472e-08
3652 1.99104067341027e-08
3653 1.99086215941224e-08
3654 1.99068424499016e-08
3655 1.99049983974353e-08
3656 1.99032116679487e-08
3657 1.99014100747963e-08
3658 1.98995790764989e-08
3659 1.98977762398966e-08
3660 1.98959979465507e-08
3661 1.9894175708024e-08
3662 1.98924232479025e-08
3663 1.98905905735014e-08
3664 1.98888010872755e-08
3665 1.98869674155611e-08
3666 1.98851994399618e-08
3667 1.98833852290692e-08
3668 1.9881647674469e-08
3669 1.98798377598619e-08
3670 1.98780161143608e-08
3671 1.98762188400869e-08
3672 1.98744668363782e-08
3673 1.98726228862744e-08
3674 1.98708685960058e-08
3675 1.98690516685085e-08
3676 1.98672974857095e-08
3677 1.98654446713076e-08
3678 1.98636670618035e-08
3679 1.98618547119223e-08
3680 1.98600638365853e-08
3681 1.98582491391375e-08
3682 1.98565009747465e-08
3683 1.98546957698165e-08
3684 1.98529423714455e-08
3685 1.98511336547136e-08
3686 1.98493482293505e-08
3687 1.98475727052339e-08
3688 1.98457886280146e-08
3689 1.98440232684893e-08
3690 1.98421925178272e-08
3691 1.98404554745957e-08
3692 1.98386251532567e-08
3693 1.9836838219156e-08
3694 1.983510189979e-08
3695 1.9833290835547e-08
3696 1.98314903105956e-08
3697 1.98297515600077e-08
3698 1.98279369970078e-08
3699 1.98261449709247e-08
3700 1.98244023413285e-08
3701 1.98226059536788e-08
3702 1.98208468512595e-08
3703 1.98191120262536e-08
3704 1.9817308234138e-08
3705 1.98155603695627e-08
3706 1.9813770102739e-08
3707 1.98120045151184e-08
3708 1.98101507781767e-08
3709 1.98085125118763e-08
3710 1.98066703527844e-08
3711 1.98048775131854e-08
3712 1.98031436903778e-08
3713 1.98013327298852e-08
3714 1.97996175382298e-08
3715 1.97978112401742e-08
3716 1.9796074531786e-08
3717 1.97943098375619e-08
3718 1.97925543475308e-08
3719 1.97908071226105e-08
3720 1.97890317480964e-08
3721 1.97872657487497e-08
3722 1.97855352128684e-08
3723 1.97837448842053e-08
3724 1.97819655827214e-08
3725 1.97802250059276e-08
3726 1.97784557547931e-08
3727 1.97766935662314e-08
3728 1.97749260866797e-08
3729 1.97731776517274e-08
3730 1.9771423370174e-08
3731 1.97696426610383e-08
3732 1.97679535315354e-08
3733 1.97662125343556e-08
3734 1.97643944254144e-08
3735 1.97626324059397e-08
3736 1.97609274696364e-08
3737 1.97591689275467e-08
3738 1.97574118658284e-08
3739 1.9755579135361e-08
3740 1.97538832648614e-08
3741 1.97521304481918e-08
3742 1.97503888528794e-08
3743 1.97486693205851e-08
3744 1.97469324521582e-08
3745 1.97451586305686e-08
3746 1.97434004216568e-08
3747 1.97416400935513e-08
3748 1.97399124332032e-08
3749 1.9738212196807e-08
3750 1.97364348196172e-08
3751 1.97346983533686e-08
3752 1.97329725932227e-08
3753 1.97311993617166e-08
3754 1.97294516885438e-08
3755 1.97276831383486e-08
3756 1.97259797097837e-08
3757 1.9724240863328e-08
3758 1.97225000556633e-08
3759 1.9720750645158e-08
3760 1.9719040124877e-08
3761 1.97172596537176e-08
3762 1.9715538173537e-08
3763 1.9713810578692e-08
3764 1.97120685424546e-08
3765 1.97103823286193e-08
3766 1.97086193169382e-08
3767 1.97068616329399e-08
3768 1.97051649777347e-08
3769 1.97033835885319e-08
3770 1.97016858583532e-08
3771 1.96999851544422e-08
3772 1.96981715203104e-08
3773 1.96964799508703e-08
3774 1.96947682404858e-08
3775 1.96930432892484e-08
3776 1.96913046473512e-08
3777 1.96895214702453e-08
3778 1.96878112533883e-08
3779 1.96861000325566e-08
3780 1.96843733297203e-08
3781 1.96826684423779e-08
3782 1.96809055050262e-08
3783 1.96791675292629e-08
3784 1.96774762264984e-08
3785 1.96757926307911e-08
3786 1.96740154769226e-08
3787 1.96723275727173e-08
3788 1.96706657248913e-08
3789 1.96688772698406e-08
3790 1.96671815949623e-08
3791 1.96654334080892e-08
3792 1.96637262210864e-08
3793 1.96620010378123e-08
3794 1.9660284524381e-08
3795 1.96585303601782e-08
3796 1.96568774316619e-08
3797 1.96551752277285e-08
3798 1.96534273226301e-08
3799 1.96517118216111e-08
3800 1.96499669045114e-08
3801 1.96482827196642e-08
3802 1.9646566468523e-08
3803 1.96449233609841e-08
3804 1.96432232215105e-08
3805 1.96414575303616e-08
3806 1.96397735693909e-08
3807 1.96380460004697e-08
3808 1.96363256695364e-08
3809 1.96346625033761e-08
3810 1.96329359743452e-08
3811 1.96312407319543e-08
3812 1.96295243570233e-08
3813 1.96278621280577e-08
3814 1.96260679266036e-08
3815 1.96243656827022e-08
3816 1.96226297515811e-08
3817 1.96210025514287e-08
3818 1.96192990574162e-08
3819 1.96175512526264e-08
3820 1.96158546997283e-08
3821 1.96141335312428e-08
3822 1.96124409291287e-08
3823 1.96107693338776e-08
3824 1.96090465507392e-08
3825 1.96073152564646e-08
3826 1.96056104713183e-08
3827 1.96039518030044e-08
3828 1.96022466907309e-08
3829 1.9600507739137e-08
3830 1.95988530658497e-08
3831 1.959716817862e-08
3832 1.95954705954349e-08
3833 1.95937707084815e-08
3834 1.95920552928941e-08
3835 1.95903539333964e-08
3836 1.95886702900605e-08
3837 1.95869774737845e-08
3838 1.95852992297407e-08
3839 1.95836296479346e-08
3840 1.95818928277469e-08
3841 1.95801945381269e-08
3842 1.95785146140381e-08
3843 1.9576885585737e-08
3844 1.95751663170651e-08
3845 1.95735004764885e-08
3846 1.95718069393447e-08
3847 1.95701022936978e-08
3848 1.95684720519784e-08
3849 1.95666981147036e-08
3850 1.95650365791278e-08
3851 1.95633832010822e-08
3852 1.95617168653461e-08
3853 1.95599939781799e-08
3854 1.95583672341071e-08
3855 1.95566366961719e-08
3856 1.95549696446751e-08
3857 1.95533075363907e-08
3858 1.95516824766373e-08
3859 1.95499456942527e-08
3860 1.9548234261757e-08
3861 1.95466039756842e-08
3862 1.95448493007788e-08
3863 1.95432257907302e-08
3864 1.95415390151776e-08
3865 1.95398401520164e-08
3866 1.95381613252721e-08
3867 1.95364829232991e-08
3868 1.95347560590364e-08
3869 1.95330457589127e-08
3870 1.95314181248296e-08
3871 1.95297907010228e-08
3872 1.95280788609642e-08
3873 1.95264280732355e-08
3874 1.95247501383333e-08
3875 1.95230784952871e-08
3876 1.95213633190638e-08
3877 1.95197466916763e-08
3878 1.95181229898922e-08
3879 1.95163458596714e-08
3880 1.95147519627126e-08
3881 1.95130417269818e-08
3882 1.95114298839338e-08
3883 1.95097864939542e-08
3884 1.95080482170429e-08
3885 1.95063949653962e-08
3886 1.95046941287025e-08
3887 1.95030520303563e-08
3888 1.95013546682166e-08
3889 1.94997329531765e-08
3890 1.94980411556411e-08
3891 1.94964049920476e-08
3892 1.94947280831581e-08
3893 1.94930771965085e-08
3894 1.94914457267714e-08
3895 1.94897757445633e-08
3896 1.94880971795541e-08
3897 1.94865084214291e-08
3898 1.94848146702897e-08
3899 1.94831736120005e-08
3900 1.94814885739469e-08
3901 1.94798206095692e-08
3902 1.94781883641748e-08
3903 1.94765322043544e-08
3904 1.94748991453331e-08
3905 1.94732196511227e-08
3906 1.94716290999319e-08
3907 1.94699020119038e-08
3908 1.94682888425834e-08
3909 1.94665510263592e-08
3910 1.94649567336058e-08
3911 1.94632838347863e-08
3912 1.94616313028417e-08
3913 1.9460010019734e-08
3914 1.94583400370263e-08
3915 1.94566764742388e-08
3916 1.94550259619564e-08
3917 1.94533496792881e-08
3918 1.94516888790686e-08
3919 1.94500170876966e-08
3920 1.94483924466637e-08
3921 1.94467652774311e-08
3922 1.94450752696307e-08
3923 1.94434431435853e-08
3924 1.94418350197845e-08
3925 1.94401771538288e-08
3926 1.94384893673627e-08
3927 1.9436886990909e-08
3928 1.94352414378818e-08
3929 1.94335243499655e-08
3930 1.94319127320708e-08
3931 1.9430309783075e-08
3932 1.94286809483968e-08
3933 1.94270144732722e-08
3934 1.94253831289348e-08
3935 1.94237270245701e-08
3936 1.94220926735267e-08
3937 1.94204052297309e-08
3938 1.94188344117552e-08
3939 1.94171814791866e-08
3940 1.94155507500793e-08
3941 1.94138731210436e-08
3942 1.94122986868495e-08
3943 1.94106929972127e-08
3944 1.94090183851525e-08
3945 1.94073669165795e-08
3946 1.9405668845951e-08
3947 1.94040582919275e-08
3948 1.94024268402315e-08
3949 1.94007882655112e-08
3950 1.93991926563108e-08
3951 1.93975733046026e-08
3952 1.9395914833964e-08
3953 1.93943366651461e-08
3954 1.93926798056632e-08
3955 1.93910150836363e-08
3956 1.93894256720895e-08
3957 1.93878263458069e-08
3958 1.93861841535914e-08
3959 1.93845472778453e-08
3960 1.93829241897925e-08
3961 1.93812855510123e-08
3962 1.93797065114465e-08
3963 1.9377994843639e-08
3964 1.93764009304154e-08
3965 1.93747612024509e-08
3966 1.93731828305732e-08
3967 1.93715048771304e-08
3968 1.93698787026575e-08
3969 1.93683033005709e-08
3970 1.93666516002944e-08
3971 1.93651134437478e-08
3972 1.93633860774423e-08
3973 1.93618295794562e-08
3974 1.93601456972559e-08
3975 1.93585533545537e-08
3976 1.93569192989962e-08
3977 1.93553529852508e-08
3978 1.93536792448268e-08
3979 1.93520922963097e-08
3980 1.93504104438191e-08
3981 1.93487762736311e-08
3982 1.93471465192996e-08
3983 1.934560176442e-08
3984 1.93439705112319e-08
3985 1.93423277600191e-08
3986 1.93407319339922e-08
3987 1.933904614565e-08
3988 1.93374652232015e-08
3989 1.93359138760396e-08
3990 1.93342907082172e-08
3991 1.93326848979547e-08
3992 1.93310567075944e-08
3993 1.93294005345623e-08
3994 1.93277939420922e-08
3995 1.93262112813675e-08
3996 1.93245642051143e-08
3997 1.93229335890277e-08
3998 1.93213818038829e-08
3999 1.93197809869039e-08
4000 1.93181999844083e-08
4001 1.93165564554287e-08
4002 1.93149901485667e-08
4003 1.931335152483e-08
4004 1.93117266627518e-08
4005 1.93101110225746e-08
4006 1.93085265465243e-08
4007 1.93068923288187e-08
4008 1.9305244829404e-08
4009 1.93036770348987e-08
4010 1.93020826297907e-08
4011 1.93004489920101e-08
4012 1.92988584433729e-08
4013 1.92973090712978e-08
4014 1.92956224310814e-08
4015 1.92940854316648e-08
4016 1.92924590697308e-08
4017 1.92908597602681e-08
4018 1.92892318249815e-08
4019 1.92876867103897e-08
4020 1.92860276673201e-08
4021 1.92844424550809e-08
4022 1.9282865356518e-08
4023 1.92812603133641e-08
4024 1.92796633882719e-08
4025 1.92780681531057e-08
4026 1.92764668162093e-08
4027 1.92748959411682e-08
4028 1.92733013612556e-08
4029 1.92716719092934e-08
4030 1.92700946022306e-08
4031 1.92684832929224e-08
4032 1.92669183370908e-08
4033 1.92653061304449e-08
4034 1.92637303157994e-08
4035 1.9262119490604e-08
4036 1.92605629544818e-08
4037 1.92589301692414e-08
4038 1.92574254994637e-08
4039 1.92557571079277e-08
4040 1.92542014173513e-08
4041 1.92525594854831e-08
4042 1.92509996239099e-08
4043 1.92494339213423e-08
4044 1.92478435184218e-08
4045 1.92462706193886e-08
4046 1.92446796822288e-08
4047 1.92430628274676e-08
4048 1.92414891016512e-08
4049 1.92399071329286e-08
4050 1.92383679753494e-08
4051 1.92367135350424e-08
4052 1.92351985698336e-08
4053 1.92335942034161e-08
4054 1.92320343082031e-08
4055 1.92304656254083e-08
4056 1.92288353605963e-08
4057 1.92272696495355e-08
4058 1.92257071016666e-08
4059 1.92241683963923e-08
4060 1.92225695125892e-08
4061 1.92209951519473e-08
4062 1.92193958727382e-08
4063 1.92178231424034e-08
4064 1.92162487208103e-08
4065 1.92146718906439e-08
4066 1.92130583822059e-08
4067 1.92114788684083e-08
4068 1.92099384831446e-08
4069 1.920833818575e-08
4070 1.92067652878269e-08
4071 1.92052004618914e-08
4072 1.92035993514805e-08
4073 1.92020855028785e-08
4074 1.92005396473238e-08
4075 1.91989213908506e-08
4076 1.91973459551797e-08
4077 1.9195763426294e-08
4078 1.91941995543732e-08
4079 1.9192627305209e-08
4080 1.91910320022082e-08
4081 1.91894692743166e-08
4082 1.91878543459589e-08
4083 1.91863366903466e-08
4084 1.91847726666028e-08
4085 1.91831754655092e-08
4086 1.9181611375263e-08
4087 1.91800802130659e-08
4088 1.91784905299031e-08
4089 1.91769294624611e-08
4090 1.91753538381079e-08
4091 1.91737362554845e-08
4092 1.91722369111691e-08
4093 1.91706106294487e-08
4094 1.91690621611507e-08
4095 1.91675327860796e-08
4096 1.91659588730242e-08
4097 1.9164446785791e-08
4098 1.91628552995149e-08
4099 1.91612529407692e-08
4100 1.91596477474576e-08
4101 1.91581977203592e-08
4102 1.91565533690929e-08
4103 1.91550456463685e-08
4104 1.91534668658178e-08
4105 1.91519055981471e-08
4106 1.91503906584178e-08
4107 1.91487767809972e-08
4108 1.9147214380677e-08
4109 1.91456440847837e-08
4110 1.91441392610181e-08
4111 1.91425090668718e-08
4112 1.91409982212565e-08
4113 1.91394503789022e-08
4114 1.91378283532218e-08
4115 1.91362744256796e-08
4116 1.91347447751622e-08
4117 1.91331940416206e-08
4118 1.91316463299396e-08
4119 1.91300621997237e-08
4120 1.91285015461728e-08
4121 1.91269686863893e-08
4122 1.91253901706823e-08
4123 1.91238924399095e-08
4124 1.91223426043163e-08
4125 1.91207892939471e-08
4126 1.91192150509889e-08
4127 1.91176792262993e-08
4128 1.91160957310754e-08
4129 1.91145771658019e-08
4130 1.9112999892823e-08
4131 1.91115294678745e-08
4132 1.91099433385911e-08
4133 1.91083492460664e-08
4134 1.91068589555088e-08
4135 1.91052829986438e-08
4136 1.9103808626908e-08
4137 1.9102232368895e-08
4138 1.91006622178858e-08
4139 1.909915157422e-08
4140 1.90976087561801e-08
4141 1.90960397288831e-08
4142 1.909446952858e-08
4143 1.90929827771913e-08
4144 1.90914233439421e-08
4145 1.90899081313201e-08
4146 1.90883406568365e-08
4147 1.90868001289646e-08
4148 1.90852871700398e-08
4149 1.90837237083463e-08
4150 1.90821980667333e-08
4151 1.9080628666901e-08
4152 1.90791111767119e-08
4153 1.90775774774887e-08
4154 1.90760440184068e-08
4155 1.90745134360792e-08
4156 1.90729498498188e-08
4157 1.907142366947e-08
4158 1.90698924915078e-08
4159 1.90683991067031e-08
4160 1.90667813274037e-08
4161 1.90653029645826e-08
4162 1.90637661800497e-08
4163 1.9062228538369e-08
4164 1.90607153581213e-08
4165 1.90591616540114e-08
4166 1.90575574475771e-08
4167 1.90560826708874e-08
4168 1.90545600197711e-08
4169 1.90529951363261e-08
4170 1.90514147182519e-08
4171 1.90499150128365e-08
4172 1.90483995820001e-08
4173 1.90468513964759e-08
4174 1.90453217580044e-08
4175 1.9043811670838e-08
4176 1.90422760842357e-08
4177 1.9040718639507e-08
4178 1.90392458074284e-08
4179 1.90376706933337e-08
4180 1.90361255542615e-08
4181 1.90346342802128e-08
4182 1.9033057293727e-08
4183 1.90315461248702e-08
4184 1.90300723716952e-08
4185 1.90284775907212e-08
4186 1.90269811414234e-08
4187 1.90254999888895e-08
4188 1.90239239091228e-08
4189 1.90224007041162e-08
4190 1.90208342907838e-08
4191 1.90193448812437e-08
4192 1.90178356138104e-08
4193 1.90162767255697e-08
4194 1.901478895594e-08
4195 1.90132419211064e-08
4196 1.90117576417403e-08
4197 1.90102444623252e-08
4198 1.90087368294733e-08
4199 1.90071631363531e-08
4200 1.90056109926617e-08
4201 1.9004110852594e-08
4202 1.90025980630337e-08
4203 1.90010757776804e-08
4204 1.89995793050679e-08
4205 1.89980387361732e-08
4206 1.89965459516661e-08
4207 1.8994994043231e-08
4208 1.89934734831088e-08
4209 1.89920016794298e-08
4210 1.89904542758912e-08
4211 1.89888934531424e-08
4212 1.89874601561058e-08
4213 1.89858767212781e-08
4214 1.89844443146958e-08
4215 1.89829277261744e-08
4216 1.89813590197319e-08
4217 1.89798348620429e-08
4218 1.89783534600418e-08
4219 1.89768051886974e-08
4220 1.89752965862877e-08
4221 1.89738245893745e-08
4222 1.89723015716625e-08
4223 1.89707741001244e-08
4224 1.8969267726654e-08
4225 1.8967761937716e-08
4226 1.89662580391547e-08
4227 1.89647141579097e-08
4228 1.89632111817772e-08
4229 1.8961657473171e-08
4230 1.89601936440575e-08
4231 1.89586659551377e-08
4232 1.89571707700509e-08
4233 1.89556739726981e-08
4234 1.89541679943006e-08
4235 1.89526909086246e-08
4236 1.89511803931341e-08
4237 1.89496384347954e-08
4238 1.89481353899956e-08
4239 1.89465969736569e-08
4240 1.89450910351163e-08
4241 1.89436070803795e-08
4242 1.8942149725365e-08
4243 1.89405985149271e-08
4244 1.8939152543862e-08
4245 1.89376453709755e-08
4246 1.89361751744554e-08
4247 1.89345931209672e-08
4248 1.89330927671816e-08
4249 1.89315849770111e-08
4250 1.89301067176073e-08
4251 1.89285945473849e-08
4252 1.89270717080303e-08
4253 1.89255620156037e-08
4254 1.89241031468002e-08
4255 1.89225996755638e-08
4256 1.89210633785297e-08
4257 1.8919605681178e-08
4258 1.89180647682252e-08
4259 1.89166194971002e-08
4260 1.89150965216323e-08
4261 1.89136450776672e-08
4262 1.89121379477464e-08
4263 1.89106662169536e-08
4264 1.89091271415864e-08
4265 1.89076418927181e-08
4266 1.89061736496909e-08
4267 1.89046987749686e-08
4268 1.89031709416088e-08
4269 1.89017083228604e-08
4270 1.89002264741056e-08
4271 1.88987337326751e-08
4272 1.88972545738242e-08
4273 1.88957440166448e-08
4274 1.88942724771435e-08
4275 1.88928080489537e-08
4276 1.88912601729041e-08
4277 1.88898346870081e-08
4278 1.88883785475769e-08
4279 1.88868457889901e-08
4280 1.88853349740725e-08
4281 1.88838865896046e-08
4282 1.88824562508128e-08
4283 1.88809227126274e-08
4284 1.88794721046048e-08
4285 1.88779177501819e-08
4286 1.88764967886668e-08
4287 1.88749720138603e-08
4288 1.88735104251214e-08
4289 1.88720726335134e-08
4290 1.8870525387904e-08
4291 1.88690849780571e-08
4292 1.88675991413256e-08
4293 1.88661287465197e-08
4294 1.88646793680136e-08
4295 1.8863200805519e-08
4296 1.88617219178955e-08
4297 1.88602821796224e-08
4298 1.8858753439488e-08
4299 1.88572742797211e-08
4300 1.88558223038759e-08
4301 1.88544021490211e-08
4302 1.88528810935173e-08
4303 1.88513880625962e-08
4304 1.88499085362059e-08
4305 1.88484737583539e-08
4306 1.88470241520577e-08
4307 1.88455457015568e-08
4308 1.88440852371408e-08
4309 1.88426116177198e-08
4310 1.88411693119728e-08
4311 1.88396618214237e-08
4312 1.88381373686097e-08
4313 1.88367370620168e-08
4314 1.88352267357916e-08
4315 1.88337732811017e-08
4316 1.8832302483035e-08
4317 1.88308353023803e-08
4318 1.88293587356503e-08
4319 1.88279412108028e-08
4320 1.88264635830049e-08
4321 1.88250016646962e-08
4322 1.88235147991489e-08
4323 1.88220673273398e-08
4324 1.88205726326107e-08
4325 1.88191215099165e-08
4326 1.88176457054934e-08
4327 1.88161917653029e-08
4328 1.88147255009707e-08
4329 1.88133057425766e-08
4330 1.88118088287059e-08
4331 1.88103453580557e-08
4332 1.88089078071163e-08
4333 1.88074651535919e-08
4334 1.88059171899457e-08
4335 1.88045486431421e-08
4336 1.88030567155884e-08
4337 1.88015534129116e-08
4338 1.88000969898738e-08
4339 1.87986530370554e-08
4340 1.87972417495519e-08
4341 1.87957628508539e-08
4342 1.8794285032292e-08
4343 1.87928006208093e-08
4344 1.87913770937631e-08
4345 1.87899636313327e-08
4346 1.87884744656253e-08
4347 1.87869788896566e-08
4348 1.87855032297291e-08
4349 1.87841208580675e-08
4350 1.87826619666709e-08
4351 1.87811862323861e-08
4352 1.87797291255898e-08
4353 1.87782924623014e-08
4354 1.87768286549483e-08
4355 1.87753734530449e-08
4356 1.87739520604302e-08
4357 1.87724205084061e-08
4358 1.87709391669122e-08
4359 1.87695010621658e-08
4360 1.8768084820181e-08
4361 1.87666012177179e-08
4362 1.87652183160636e-08
4363 1.87637020696296e-08
4364 1.8762292877561e-08
4365 1.87608133340733e-08
4366 1.87593307889311e-08
4367 1.87578825508183e-08
4368 1.87564623518321e-08
4369 1.87550040411932e-08
4370 1.87535566756325e-08
4371 1.87520770091598e-08
4372 1.87506240581503e-08
4373 1.87491993439981e-08
4374 1.8747759565757e-08
4375 1.87462738277799e-08
4376 1.8744812602528e-08
4377 1.87433968888151e-08
4378 1.87420083645473e-08
4379 1.87405269310437e-08
4380 1.87391328040909e-08
4381 1.87376109299919e-08
4382 1.87362086561116e-08
4383 1.87347793167703e-08
4384 1.87333099269105e-08
4385 1.87318642480283e-08
4386 1.87304670887511e-08
4387 1.87289916895039e-08
4388 1.87275616378435e-08
4389 1.87261172374387e-08
4390 1.87246487831638e-08
4391 1.87232185467345e-08
4392 1.87217363138148e-08
4393 1.87203326708907e-08
4394 1.87188925002968e-08
4395 1.8717429310422e-08
4396 1.87160241350848e-08
4397 1.87145547362599e-08
4398 1.87130998086926e-08
4399 1.8711693853396e-08
4400 1.87102388123639e-08
4401 1.87088030447757e-08
4402 1.87073669794258e-08
4403 1.87059333028594e-08
4404 1.87044808920844e-08
4405 1.87030313822389e-08
4406 1.87016546178143e-08
4407 1.87001960683386e-08
4408 1.86987371470493e-08
4409 1.86973508191357e-08
4410 1.86958842741669e-08
4411 1.86944972788428e-08
4412 1.86930441293254e-08
4413 1.86915696579471e-08
4414 1.86901960851082e-08
4415 1.86887335846264e-08
4416 1.86872696759655e-08
4417 1.86858551755043e-08
4418 1.86844330683777e-08
4419 1.86829685916157e-08
4420 1.86815495665238e-08
4421 1.8680105610458e-08
4422 1.86787169329516e-08
4423 1.86772784583344e-08
4424 1.86758856987879e-08
4425 1.86744089869795e-08
4426 1.86729851519296e-08
4427 1.86715467226373e-08
4428 1.86701594088456e-08
4429 1.86687202452795e-08
4430 1.86672761513795e-08
4431 1.8665881084623e-08
4432 1.8664424431325e-08
4433 1.86630453213044e-08
4434 1.86615200911955e-08
4435 1.86601840279077e-08
4436 1.86587073227329e-08
4437 1.86572580296862e-08
4438 1.86558210627241e-08
4439 1.86544240957098e-08
4440 1.86529920520317e-08
4441 1.86516169406181e-08
4442 1.8650171858009e-08
4443 1.86487655473577e-08
4444 1.86472978685182e-08
4445 1.8645956139568e-08
4446 1.86445443202399e-08
4447 1.86430465317911e-08
4448 1.86416545766288e-08
4449 1.86402643007066e-08
4450 1.86387875808769e-08
4451 1.86373798312933e-08
4452 1.86359355231203e-08
4453 1.86345332847393e-08
4454 1.86330759247844e-08
4455 1.86317027044414e-08
4456 1.86302226672375e-08
4457 1.86288775275156e-08
4458 1.86274273745457e-08
4459 1.86260064164168e-08
4460 1.86246086786301e-08
4461 1.86231270878634e-08
4462 1.86217722722426e-08
4463 1.86203625831272e-08
4464 1.86189603725018e-08
4465 1.86175308152237e-08
4466 1.86161249485228e-08
4467 1.86147223897037e-08
4468 1.86132914574422e-08
4469 1.86118601815388e-08
4470 1.86104702428469e-08
4471 1.86090302470854e-08
4472 1.86075882179781e-08
4473 1.86061583409003e-08
4474 1.86047725311833e-08
4475 1.86033429309673e-08
4476 1.86019685631811e-08
4477 1.86005387965982e-08
4478 1.85991963086163e-08
4479 1.85977323015896e-08
4480 1.85963038753789e-08
4481 1.85949196472301e-08
4482 1.85935233155965e-08
4483 1.85921135249234e-08
4484 1.85906919791534e-08
4485 1.85892846676916e-08
4486 1.85878538781492e-08
4487 1.85864525789892e-08
4488 1.85850672941301e-08
4489 1.85836599220779e-08
4490 1.85822551072301e-08
4491 1.85808187264935e-08
4492 1.85794243971749e-08
4493 1.8578002331765e-08
4494 1.8576618525723e-08
4495 1.85751832518233e-08
4496 1.85737469401703e-08
4497 1.85723734947019e-08
4498 1.85710224020752e-08
4499 1.85695830673682e-08
4500 1.85681682954575e-08
4501 1.85667369201059e-08
4502 1.85653205017067e-08
4503 1.85640382791841e-08
4504 1.85625540368994e-08
4505 1.85611777293315e-08
4506 1.85597767270174e-08
4507 1.85583798813227e-08
4508 1.85569655228313e-08
4509 1.85555366032097e-08
4510 1.85541469044648e-08
4511 1.85527546876507e-08
4512 1.85513533255688e-08
4513 1.85500007712502e-08
4514 1.85485392215079e-08
4515 1.85471632725698e-08
4516 1.85457658291588e-08
4517 1.85443843838395e-08
4518 1.85429845266649e-08
4519 1.85415345011208e-08
4520 1.85401369649785e-08
4521 1.85387302436002e-08
4522 1.853737671767e-08
4523 1.85359533548812e-08
4524 1.85345175116036e-08
4525 1.85331308869552e-08
4526 1.85317441686039e-08
4527 1.85303700881989e-08
4528 1.85289659659638e-08
4529 1.85275477641578e-08
4530 1.85261786188329e-08
4531 1.85247483256157e-08
4532 1.8523385021868e-08
4533 1.85220007628828e-08
4534 1.85205447920112e-08
4535 1.85191691118303e-08
4536 1.85177898952282e-08
4537 1.85163705061775e-08
4538 1.85149998510603e-08
4539 1.85135908696565e-08
4540 1.85121969303592e-08
4541 1.85108665181899e-08
4542 1.85094598689206e-08
4543 1.85080265379667e-08
4544 1.85066589097338e-08
4545 1.85053271843094e-08
4546 1.85039240390428e-08
4547 1.85024595399652e-08
4548 1.85011462779106e-08
4549 1.84997249766672e-08
4550 1.84983109880743e-08
4551 1.84969170216875e-08
4552 1.8495575956512e-08
4553 1.84941535917305e-08
4554 1.84928113422522e-08
4555 1.84913436319933e-08
4556 1.84900360399587e-08
4557 1.84886461036815e-08
4558 1.84872616443288e-08
4559 1.84858553058109e-08
4560 1.84844758415181e-08
4561 1.84830697822214e-08
4562 1.84817400897541e-08
4563 1.84802921241178e-08
4564 1.84789063949475e-08
4565 1.84775541215987e-08
4566 1.8476195963818e-08
4567 1.84747202910396e-08
4568 1.84733837828022e-08
4569 1.84720310600628e-08
4570 1.84705777161731e-08
4571 1.84692627253347e-08
4572 1.84678174707797e-08
4573 1.84664650283162e-08
4574 1.8465137571172e-08
4575 1.84637083889272e-08
4576 1.8462315601403e-08
4577 1.84609736900987e-08
4578 1.84595952423816e-08
4579 1.84582258221655e-08
4580 1.84568292304343e-08
4581 1.84554770916445e-08
4582 1.84540858658433e-08
4583 1.84527183119954e-08
4584 1.84513445410373e-08
4585 1.84499460824106e-08
4586 1.84485496411424e-08
4587 1.84472601706953e-08
4588 1.84458028358314e-08
4589 1.84445203247618e-08
4590 1.84431128041618e-08
4591 1.84417031915407e-08
4592 1.84402799349725e-08
4593 1.84389923148787e-08
4594 1.84376150861865e-08
4595 1.84362806619209e-08
4596 1.84348660620393e-08
4597 1.84334848889467e-08
4598 1.8432109585742e-08
4599 1.84307546630402e-08
4600 1.84294021124132e-08
4601 1.84280012803517e-08
4602 1.84266307836911e-08
4603 1.84252459158873e-08
4604 1.84239127409003e-08
4605 1.84224479191919e-08
4606 1.8421162168103e-08
4607 1.8419809016873e-08
4608 1.84183683483163e-08
4609 1.84170392423799e-08
4610 1.84157052794398e-08
4611 1.84143379458601e-08
4612 1.84129147823009e-08
4613 1.84115588680867e-08
4614 1.84101766031453e-08
4615 1.84088259388704e-08
4616 1.84074444473659e-08
4617 1.84060504177797e-08
4618 1.84047013647715e-08
4619 1.84033132730188e-08
4620 1.84019172785876e-08
4621 1.84005515081742e-08
4622 1.83992596418936e-08
4623 1.83978255676176e-08
4624 1.83964439710582e-08
4625 1.83951109437586e-08
4626 1.83937778673871e-08
4627 1.83923719030143e-08
4628 1.83910118711494e-08
4629 1.83896525050298e-08
4630 1.83883062370993e-08
4631 1.83869294588801e-08
4632 1.83855871849492e-08
4633 1.83842017080216e-08
4634 1.83828033753775e-08
4635 1.838148993763e-08
4636 1.83800907899434e-08
4637 1.8378749570086e-08
4638 1.8377383121243e-08
4639 1.83760453086601e-08
4640 1.83747031289594e-08
4641 1.83732825341509e-08
4642 1.83719852591413e-08
4643 1.83706075150836e-08
4644 1.83692340482156e-08
4645 1.83678500891016e-08
4646 1.83664955367147e-08
4647 1.83652036592763e-08
4648 1.8363797328308e-08
4649 1.83624119720061e-08
4650 1.8361065281941e-08
4651 1.83597462835305e-08
4652 1.83583793912367e-08
4653 1.835701570016e-08
4654 1.83556560593712e-08
4655 1.83542754212118e-08
4656 1.8352962740581e-08
4657 1.83516297686537e-08
4658 1.83502319202888e-08
4659 1.83488613728355e-08
4660 1.83474989748911e-08
4661 1.83461927760475e-08
4662 1.83448446538503e-08
4663 1.83435514421737e-08
4664 1.8342100824853e-08
4665 1.83407699495264e-08
4666 1.83393768033446e-08
4667 1.83380463673888e-08
4668 1.83366553385411e-08
4669 1.8335373267786e-08
4670 1.83339929738791e-08
4671 1.83326151862173e-08
4672 1.83312006061809e-08
4673 1.83299208029453e-08
4674 1.83285495992391e-08
4675 1.83272033533188e-08
4676 1.83258705915013e-08
4677 1.83245196086201e-08
4678 1.83231907536496e-08
4679 1.83218285259024e-08
4680 1.83204758285871e-08
4681 1.83191404880267e-08
4682 1.83177824619463e-08
4683 1.8316463688467e-08
4684 1.83150909605911e-08
4685 1.83137770508324e-08
4686 1.83124518283728e-08
4687 1.83110582015755e-08
4688 1.83097085400818e-08
4689 1.83083906599435e-08
4690 1.83070903918281e-08
4691 1.83056602973686e-08
4692 1.83043568342534e-08
4693 1.83030389117322e-08
4694 1.83016426526239e-08
4695 1.83002931440635e-08
4696 1.82989749274165e-08
4697 1.82975677125996e-08
4698 1.82963034604522e-08
4699 1.82950024270878e-08
4700 1.82935990672706e-08
4701 1.82922692617826e-08
4702 1.82909611251003e-08
4703 1.82895961704788e-08
4704 1.82882799327488e-08
4705 1.82869329247437e-08
4706 1.82855502864066e-08
4707 1.82842599119049e-08
4708 1.8282920336149e-08
4709 1.8281516942581e-08
4710 1.82801900957263e-08
4711 1.82788684581214e-08
4712 1.82775191691908e-08
4713 1.82762064051822e-08
4714 1.82748323210857e-08
4715 1.82734420234587e-08
4716 1.82721822030718e-08
4717 1.82707859883724e-08
4718 1.82695141338618e-08
4719 1.82681696654086e-08
4720 1.82667794293434e-08
4721 1.82655045528612e-08
4722 1.82640872090356e-08
4723 1.82628394356232e-08
4724 1.82614252473567e-08
4725 1.8260146521426e-08
4726 1.82588154133689e-08
4727 1.82574613536712e-08
4728 1.82561621290545e-08
4729 1.82547647804288e-08
4730 1.8253407551766e-08
4731 1.8252146790021e-08
4732 1.82508066956244e-08
4733 1.82494662933075e-08
4734 1.82481212074315e-08
4735 1.82467666884623e-08
4736 1.82454526095055e-08
4737 1.82440942427253e-08
4738 1.82427515836026e-08
4739 1.82414725230229e-08
4740 1.82401405482702e-08
4741 1.82387725781441e-08
4742 1.82374495264204e-08
4743 1.82360949942395e-08
4744 1.82347934036542e-08
4745 1.82334453893429e-08
4746 1.8232143707303e-08
4747 1.82307972949325e-08
4748 1.8229526043273e-08
4749 1.82281802722228e-08
4750 1.82268686996501e-08
4751 1.82254588399489e-08
4752 1.82241955980489e-08
4753 1.82228229897996e-08
4754 1.82215395825858e-08
4755 1.82202277588528e-08
4756 1.82189220024609e-08
4757 1.82175777978522e-08
4758 1.82162489017756e-08
4759 1.82149538180965e-08
4760 1.82135178852239e-08
4761 1.82123226833419e-08
4762 1.82109279459608e-08
4763 1.82095981804131e-08
4764 1.82083445336712e-08
4765 1.82070057237194e-08
4766 1.82056561357224e-08
4767 1.82043368971985e-08
4768 1.82030021730895e-08
4769 1.82016821461684e-08
4770 1.82003573011291e-08
4771 1.8199012339154e-08
4772 1.81977268154665e-08
4773 1.81963958462983e-08
4774 1.81950934926289e-08
4775 1.81936672908334e-08
4776 1.81924037207237e-08
4777 1.81910566161569e-08
4778 1.8189753185599e-08
4779 1.81884420976386e-08
4780 1.8187099089878e-08
4781 1.81857405522623e-08
4782 1.81844502049611e-08
4783 1.8183147162093e-08
4784 1.81818314883564e-08
4785 1.81804776839545e-08
4786 1.817918105948e-08
4787 1.81778510927044e-08
4788 1.81765261313971e-08
4789 1.81752256517287e-08
4790 1.81739113530865e-08
4791 1.8172596791155e-08
4792 1.81712706925075e-08
4793 1.8169953297037e-08
4794 1.81685965277845e-08
4795 1.81672984913839e-08
4796 1.81660000730388e-08
4797 1.81646245609735e-08
4798 1.81633238240941e-08
4799 1.81620807835625e-08
4800 1.81607450709664e-08
4801 1.81593931634916e-08
4802 1.81581016326648e-08
4803 1.81567855993825e-08
4804 1.81554603078671e-08
4805 1.81541733638435e-08
4806 1.8152834572599e-08
4807 1.81515460519199e-08
4808 1.81502046535664e-08
4809 1.81489162030812e-08
4810 1.81476249109525e-08
4811 1.81463351101585e-08
4812 1.81450158263652e-08
4813 1.81436623800102e-08
4814 1.81423547360449e-08
4815 1.81410610259058e-08
4816 1.81397177246023e-08
4817 1.81384234517745e-08
4818 1.81371492410654e-08
4819 1.81357962406592e-08
4820 1.81345129446897e-08
4821 1.81331628020531e-08
4822 1.81319217374687e-08
4823 1.81305641563934e-08
4824 1.81292596829474e-08
4825 1.81279170294035e-08
4826 1.81266758050025e-08
4827 1.81252882556771e-08
4828 1.81240235920532e-08
4829 1.81227452310528e-08
4830 1.81214061414359e-08
4831 1.81201135522613e-08
4832 1.81188153649259e-08
4833 1.81175244092502e-08
4834 1.81161586230438e-08
4835 1.81149382405821e-08
4836 1.81135950066413e-08
4837 1.81123139962047e-08
4838 1.81109815239611e-08
4839 1.81096736180386e-08
4840 1.8108378822379e-08
4841 1.81070666884497e-08
4842 1.81057711124422e-08
4843 1.81044678103648e-08
4844 1.81031378386487e-08
4845 1.81018351472773e-08
4846 1.81004865338341e-08
4847 1.80992573229344e-08
4848 1.80979538564885e-08
4849 1.80966220496293e-08
4850 1.80953997009625e-08
4851 1.80940470478619e-08
4852 1.80926934109371e-08
4853 1.80914359460549e-08
4854 1.80901606497141e-08
4855 1.80888697463855e-08
4856 1.80875309755413e-08
4857 1.80862531136972e-08
4858 1.80849177570386e-08
4859 1.80836222956338e-08
4860 1.80824143914127e-08
4861 1.80810375271789e-08
4862 1.80797459185811e-08
4863 1.80784283849711e-08
4864 1.8077143285139e-08
4865 1.80758008531401e-08
4866 1.80745337516053e-08
4867 1.80732712173615e-08
4868 1.80719362773973e-08
4869 1.80706913230633e-08
4870 1.80693642741481e-08
4871 1.80680660949728e-08
4872 1.80667625324649e-08
4873 1.806551332037e-08
4874 1.80641915605018e-08
4875 1.80628997529797e-08
4876 1.80616149168811e-08
4877 1.80603259655743e-08
4878 1.80589939862974e-08
4879 1.80577556869732e-08
4880 1.80563560935876e-08
4881 1.80551094652037e-08
4882 1.80538326027546e-08
4883 1.80525720760161e-08
4884 1.80512289814916e-08
4885 1.80499507196064e-08
4886 1.80486908044342e-08
4887 1.80473495909605e-08
4888 1.80460600134602e-08
4889 1.80447789974447e-08
4890 1.80434693055342e-08
4891 1.8042197398177e-08
4892 1.80409069630294e-08
4893 1.80396019812679e-08
4894 1.80382815938851e-08
4895 1.80370702052846e-08
4896 1.80357458298419e-08
4897 1.8034432992059e-08
4898 1.80331579102688e-08
4899 1.80319051382993e-08
4900 1.80306011178799e-08
4901 1.80292387431114e-08
4902 1.80280073473205e-08
4903 1.80267253909183e-08
4904 1.8025415210704e-08
4905 1.80241556674843e-08
4906 1.80228491270806e-08
4907 1.80215797819794e-08
4908 1.80202693576548e-08
4909 1.80190164892347e-08
4910 1.80177336309706e-08
4911 1.80164410080175e-08
4912 1.80151982963339e-08
4913 1.80138643062766e-08
4914 1.80126038054063e-08
4915 1.80113373298707e-08
4916 1.80100625117585e-08
4917 1.80088060728056e-08
4918 1.80075294697324e-08
4919 1.80062442980411e-08
4920 1.80049303215302e-08
4921 1.80036448498011e-08
4922 1.80023597111389e-08
4923 1.80010750668591e-08
4924 1.79998025489903e-08
4925 1.79985762058799e-08
4926 1.79972503325521e-08
4927 1.79960030705639e-08
4928 1.79947954190574e-08
4929 1.79934656641401e-08
4930 1.79921596143717e-08
4931 1.799088134255e-08
4932 1.79896439418681e-08
4933 1.79883543396653e-08
4934 1.79870844684571e-08
4935 1.79857890380553e-08
4936 1.79845085383767e-08
4937 1.798323900562e-08
4938 1.79819361566524e-08
4939 1.79807310529412e-08
4940 1.7979398498208e-08
4941 1.79781219916964e-08
4942 1.79768538013381e-08
4943 1.79755812794169e-08
4944 1.79743090571727e-08
4945 1.79730721845683e-08
4946 1.79717878346364e-08
4947 1.79704442596074e-08
4948 1.79692063362047e-08
4949 1.7967985277062e-08
4950 1.79666596866745e-08
4951 1.79653707012839e-08
4952 1.79641995769475e-08
4953 1.7962839354041e-08
4954 1.79615551676726e-08
4955 1.79602778132004e-08
4956 1.79590294093202e-08
4957 1.79577181521062e-08
4958 1.79564555904399e-08
4959 1.7955203976211e-08
4960 1.79539713803245e-08
4961 1.79526735663293e-08
4962 1.79513747496085e-08
4963 1.79500894072482e-08
4964 1.79488753206669e-08
4965 1.79475905541238e-08
4966 1.79462750838355e-08
4967 1.79449599359283e-08
4968 1.79437061756105e-08
4969 1.79424324685262e-08
4970 1.79412043345151e-08
4971 1.79399245144318e-08
4972 1.79386475961663e-08
4973 1.79373941794625e-08
4974 1.79360563459796e-08
4975 1.79348428394899e-08
4976 1.79335595154317e-08
4977 1.79322830679285e-08
4978 1.79310176555703e-08
4979 1.79297442188808e-08
4980 1.79284966409787e-08
4981 1.79272366168937e-08
4982 1.79259833409939e-08
4983 1.79247339864297e-08
4984 1.79234737352207e-08
4985 1.79221941352115e-08
4986 1.79209034417427e-08
4987 1.79196372642187e-08
4988 1.79183842239083e-08
4989 1.79171500042929e-08
4990 1.79158474841457e-08
4991 1.79146254644935e-08
4992 1.79133666442788e-08
4993 1.79120993194781e-08
4994 1.7910840646479e-08
4995 1.79095701227372e-08
4996 1.79082920864493e-08
4997 1.79070338247878e-08
4998 1.79057445267861e-08
4999 1.79045107808196e-08
};
\addlegendentry{Train}
\addplot [semithick, black]
table {%
0 0.0012633673613891
1 0.000211366903386079
2 0.000202950366656296
3 0.000193531421246007
4 0.000165711186127737
5 5.70414777030237e-05
6 3.15094257530291e-05
7 2.82433775282698e-05
8 2.4097207642626e-05
9 1.75431196112186e-05
10 1.0115767508978e-05
11 6.97107952873921e-06
12 6.35781407254399e-06
13 6.15505132373073e-06
14 6.01893634666339e-06
15 5.90412673773244e-06
16 5.79742481932044e-06
17 5.68670247957925e-06
18 5.57149996893713e-06
19 5.44175827599247e-06
20 5.29470162291545e-06
21 5.12844962941017e-06
22 4.93838433612837e-06
23 4.71260091217118e-06
24 4.446186721907e-06
25 4.1334687921335e-06
26 3.78007416657056e-06
27 3.39393591275439e-06
28 2.98808663501404e-06
29 2.58681984632858e-06
30 2.22563517127128e-06
31 1.94206177184242e-06
32 1.73271791936713e-06
33 1.5912784192551e-06
34 1.50132609633147e-06
35 1.44717034800124e-06
36 1.41265195452434e-06
37 1.39180997393851e-06
38 1.37621373141883e-06
39 1.36402570660721e-06
40 1.35335073991882e-06
41 1.34366803195007e-06
42 1.3347230378713e-06
43 1.32636068883585e-06
44 1.3185010629968e-06
45 1.31109004541941e-06
46 1.30408898257883e-06
47 1.29750628730108e-06
48 1.29129045944865e-06
49 1.28538999888406e-06
50 1.2797910358131e-06
51 1.27447162867611e-06
52 1.26941210965015e-06
53 1.26459156035708e-06
54 1.2599858791873e-06
55 1.25557085084438e-06
56 1.2512975899881e-06
57 1.24713653804065e-06
58 1.24305404369807e-06
59 1.23901588722219e-06
60 1.2350677707218e-06
61 1.23108611660427e-06
62 1.22700600968528e-06
63 1.22281710446259e-06
64 1.21852383472287e-06
65 1.21426398891344e-06
66 1.20977756523644e-06
67 1.20481877274869e-06
68 1.19957087463263e-06
69 1.19403159715148e-06
70 1.1882306125699e-06
71 1.18223397294059e-06
72 1.17598312954215e-06
73 1.16916828574176e-06
74 1.16150340545573e-06
75 1.15272757739149e-06
76 1.14257977656962e-06
77 1.13093881282111e-06
78 1.11785800527286e-06
79 1.10340477021964e-06
80 1.08758968053735e-06
81 1.07032178675581e-06
82 1.05141350559279e-06
83 1.03051638689067e-06
84 1.00716340512008e-06
85 9.8194755082659e-07
86 9.55758878262714e-07
87 9.27361043068231e-07
88 8.98199289167678e-07
89 8.6715414227001e-07
90 8.36435901874211e-07
91 8.09304765425622e-07
92 7.8275564874275e-07
93 7.58054738980718e-07
94 7.38667722544051e-07
95 7.21402273029526e-07
96 7.04860212863423e-07
97 6.92637001975527e-07
98 6.81350968534389e-07
99 6.71526947826351e-07
100 6.63219111629587e-07
101 6.55545818517567e-07
102 6.48220009225042e-07
103 6.41388623989769e-07
104 6.35023639006249e-07
105 6.29082876457687e-07
106 6.2351114138437e-07
107 6.18279784703191e-07
108 6.1336089629549e-07
109 6.08706841376261e-07
110 6.04295109951636e-07
111 6.00104897330311e-07
112 5.9609965319396e-07
113 5.92268122545647e-07
114 5.88577734106366e-07
115 5.85025134114403e-07
116 5.81586050429905e-07
117 5.78261676764669e-07
118 5.7503768857714e-07
119 5.7189959079551e-07
120 5.68850055060466e-07
121 5.65892776194232e-07
122 5.63012974907906e-07
123 5.60223043066799e-07
124 5.57504279186105e-07
125 5.54859639123606e-07
126 5.52295887246146e-07
127 5.49796141058323e-07
128 5.47365971215186e-07
129 5.45001512364252e-07
130 5.42701229733211e-07
131 5.40455005193508e-07
132 5.38274093742075e-07
133 5.36144511897874e-07
134 5.34072796654073e-07
135 5.3205906169751e-07
136 5.30103818618954e-07
137 5.28204679994815e-07
138 5.26353915120126e-07
139 5.24551239777793e-07
140 5.22792277024564e-07
141 5.21084587035148e-07
142 5.1941486844953e-07
143 5.17788748766179e-07
144 5.16197815159103e-07
145 5.14651276262157e-07
146 5.13135660185071e-07
147 5.11656025992124e-07
148 5.1020919045186e-07
149 5.08796688336588e-07
150 5.0741391532938e-07
151 5.06067635797081e-07
152 5.04743070450786e-07
153 5.03456760725385e-07
154 5.02191312534706e-07
155 5.00958435623033e-07
156 4.99753639360279e-07
157 4.98568795137544e-07
158 4.97409644140134e-07
159 4.96278289574548e-07
160 4.9516643230163e-07
161 4.94081916713185e-07
162 4.93018035285786e-07
163 4.91978198624565e-07
164 4.90957575038919e-07
165 4.8996508894561e-07
166 4.8899005378189e-07
167 4.88038210733066e-07
168 4.87110867197771e-07
169 4.86196711335651e-07
170 4.853110908698e-07
171 4.84442182369094e-07
172 4.83592828004475e-07
173 4.82766211007402e-07
174 4.81956703879405e-07
175 4.8116953621502e-07
176 4.80393737234408e-07
177 4.79641471429204e-07
178 4.78903473322134e-07
179 4.78187018870813e-07
180 4.77490459616092e-07
181 4.76806860660872e-07
182 4.76137785199171e-07
183 4.75484370099366e-07
184 4.74846842735133e-07
185 4.74221195645441e-07
186 4.73610072049269e-07
187 4.73010800305929e-07
188 4.72423010933198e-07
189 4.71850256644757e-07
190 4.71289212100601e-07
191 4.70732572921406e-07
192 4.70185227641196e-07
193 4.6965161004664e-07
194 4.69123506263713e-07
195 4.6860375846336e-07
196 4.68097027805925e-07
197 4.67598141540293e-07
198 4.67112215574161e-07
199 4.66628023332305e-07
200 4.66153437628236e-07
201 4.65692238549309e-07
202 4.65234791136027e-07
203 4.64790105070279e-07
204 4.6434541900453e-07
205 4.63914432202728e-07
206 4.63487651813921e-07
207 4.63073661194358e-07
208 4.62655236788123e-07
209 4.62252671695751e-07
210 4.61854142486118e-07
211 4.61462747125552e-07
212 4.61075387647725e-07
213 4.60696924164949e-07
214 4.60323747120128e-07
215 4.59955060705397e-07
216 4.59596094515291e-07
217 4.59238975736298e-07
218 4.58890468735262e-07
219 4.58548498727396e-07
220 4.58205306586024e-07
221 4.57877433746035e-07
222 4.575533409934e-07
223 4.57230896699912e-07
224 4.56921213753958e-07
225 4.56611644494842e-07
226 4.56311795460351e-07
227 4.56019165540056e-07
228 4.55725825077025e-07
229 4.55441579561011e-07
230 4.55161057288933e-07
231 4.54887810974469e-07
232 4.54619197398642e-07
233 4.54355415513419e-07
234 4.54095726354353e-07
235 4.53843938430509e-07
236 4.53596783245303e-07
237 4.53348206974624e-07
238 4.53107077191817e-07
239 4.52872512823888e-07
240 4.52636442105359e-07
241 4.52407419970768e-07
242 4.5218229161037e-07
243 4.51958811709119e-07
244 4.51739168738641e-07
245 4.51517621513631e-07
246 4.51305027127091e-07
247 4.51088283170975e-07
248 4.50873102408877e-07
249 4.50662241746613e-07
250 4.50451722144862e-07
251 4.50241060434564e-07
252 4.50026163889561e-07
253 4.49811750513618e-07
254 4.49601941454603e-07
255 4.49384970124811e-07
256 4.49170215688355e-07
257 4.48951738007963e-07
258 4.48728030733037e-07
259 4.48508160388883e-07
260 4.48282804654809e-07
261 4.48056766799709e-07
262 4.47831808969568e-07
263 4.47592498176164e-07
264 4.47358473820714e-07
265 4.47118452484574e-07
266 4.46879795390487e-07
267 4.46633492856563e-07
268 4.46387986130503e-07
269 4.46134919229735e-07
270 4.45881738642129e-07
271 4.45623868472467e-07
272 4.45358637080062e-07
273 4.45094343604069e-07
274 4.44825673184823e-07
275 4.44554984824208e-07
276 4.4427395096136e-07
277 4.43989677023637e-07
278 4.43705289399077e-07
279 4.43415814288528e-07
280 4.431242928149e-07
281 4.42822397417331e-07
282 4.42519649368478e-07
283 4.42210080109362e-07
284 4.41895593894515e-07
285 4.41573916987181e-07
286 4.41249540017452e-07
287 4.40919706079512e-07
288 4.40579157157117e-07
289 4.40235169207881e-07
290 4.39888538039668e-07
291 4.39531021356743e-07
292 4.39164551835347e-07
293 4.3879015265702e-07
294 4.38403731095605e-07
295 4.38010658854182e-07
296 4.37613010717541e-07
297 4.37203169667555e-07
298 4.36781107282513e-07
299 4.36354554267382e-07
300 4.35912767215996e-07
301 4.3546447159315e-07
302 4.35006796806192e-07
303 4.34541419735979e-07
304 4.34062428666948e-07
305 4.33576246905432e-07
306 4.33089610396564e-07
307 4.32586745091612e-07
308 4.32081748158453e-07
309 4.31568537351268e-07
310 4.31058367666992e-07
311 4.30540353590914e-07
312 4.30015319352606e-07
313 4.29489290354468e-07
314 4.28956354880938e-07
315 4.28413471809108e-07
316 4.27870020303089e-07
317 4.27320173912449e-07
318 4.26787295282338e-07
319 4.26270617026603e-07
320 4.25776079282514e-07
321 4.25295951345106e-07
322 4.24820655098301e-07
323 4.24355135919541e-07
324 4.23897859036515e-07
325 4.23444504349391e-07
326 4.2299544134039e-07
327 4.22544957245918e-07
328 4.22097343744099e-07
329 4.21647484927234e-07
330 4.21198791400457e-07
331 4.20746459894872e-07
332 4.20294355762962e-07
333 4.19833611431386e-07
334 4.19370394411089e-07
335 4.18900214071982e-07
336 4.18427191561932e-07
337 4.17945955177856e-07
338 4.17451929024537e-07
339 4.16958698679082e-07
340 4.1644932480267e-07
341 4.1593239075155e-07
342 4.15413325072223e-07
343 4.14880986454591e-07
344 4.14339496046523e-07
345 4.13789564390754e-07
346 4.13228519846598e-07
347 4.12651360193195e-07
348 4.12055527476696e-07
349 4.11439913250433e-07
350 4.10815488294247e-07
351 4.10167615427781e-07
352 4.09496664133258e-07
353 4.0881843688112e-07
354 4.08114971151008e-07
355 4.07390416512499e-07
356 4.06652759465942e-07
357 4.05888926025e-07
358 4.05102611011898e-07
359 4.04297765044248e-07
360 4.03471972276748e-07
361 4.02631997076242e-07
362 4.01779033154526e-07
363 4.00907879338774e-07
364 4.00018194568474e-07
365 3.9911725480124e-07
366 3.9820710640015e-07
367 3.9727703438075e-07
368 3.96336474750569e-07
369 3.95383437989949e-07
370 3.94415963000938e-07
371 3.93430468648148e-07
372 3.92430166584745e-07
373 3.9141650631791e-07
374 3.90383036119601e-07
375 3.89340812034789e-07
376 3.8827690218568e-07
377 3.87204408980324e-07
378 3.86108183647593e-07
379 3.85002039138271e-07
380 3.83878671073035e-07
381 3.827372268006e-07
382 3.81583390662854e-07
383 3.80414633127657e-07
384 3.79231067881847e-07
385 3.78031558057046e-07
386 3.76811470914618e-07
387 3.75578594002945e-07
388 3.74321103890907e-07
389 3.73049886093213e-07
390 3.71749365513097e-07
391 3.70421332718252e-07
392 3.69069994121674e-07
393 3.67686396884892e-07
394 3.66271706297994e-07
395 3.64820806453281e-07
396 3.63335516340157e-07
397 3.61815239102725e-07
398 3.6025417671226e-07
399 3.586554555568e-07
400 3.57016233465401e-07
401 3.55337192559091e-07
402 3.53629786786769e-07
403 3.51872131432174e-07
404 3.50081279520964e-07
405 3.48260499549724e-07
406 3.46402032391779e-07
407 3.44515484584917e-07
408 3.4259727499375e-07
409 3.40653514285805e-07
410 3.38683463496636e-07
411 3.36691556412916e-07
412 3.34676172997206e-07
413 3.32642912326264e-07
414 3.30590722796842e-07
415 3.2852005915629e-07
416 3.26432456176917e-07
417 3.24336895118904e-07
418 3.22227691640364e-07
419 3.20107886864207e-07
420 3.17981630360009e-07
421 3.15837070274938e-07
422 3.13691259634652e-07
423 3.11530641283753e-07
424 3.0936470807319e-07
425 3.07190589410311e-07
426 3.05005499967592e-07
427 3.02813276675806e-07
428 3.00610423664693e-07
429 2.9839813464605e-07
430 2.96183031878172e-07
431 2.93963722697299e-07
432 2.91751774739168e-07
433 2.89537297248899e-07
434 2.8732372925333e-07
435 2.85101009467326e-07
436 2.82875561197216e-07
437 2.80637380001281e-07
438 2.78391098618158e-07
439 2.76140866617425e-07
440 2.73892368340967e-07
441 2.71656944050847e-07
442 2.69438487521256e-07
443 2.67263317255129e-07
444 2.65132001686652e-07
445 2.6304337552574e-07
446 2.6100104832949e-07
447 2.58990041857032e-07
448 2.57000010606134e-07
449 2.55019102723963e-07
450 2.53049165621633e-07
451 2.51103386972318e-07
452 2.49205271529718e-07
453 2.47375595563426e-07
454 2.4561649070165e-07
455 2.43922386289341e-07
456 2.42277479856057e-07
457 2.40681032437351e-07
458 2.39119316347569e-07
459 2.37583208217984e-07
460 2.36066682646197e-07
461 2.34570606494344e-07
462 2.33090972301397e-07
463 2.31645188364382e-07
464 2.30239905363305e-07
465 2.28904539767427e-07
466 2.27634714633496e-07
467 2.264212071168e-07
468 2.25252065888526e-07
469 2.24120796588068e-07
470 2.23035726776288e-07
471 2.21988656790018e-07
472 2.2098933527559e-07
473 2.20040107024033e-07
474 2.19135031898077e-07
475 2.1827287355336e-07
476 2.1743309730482e-07
477 2.166010659721e-07
478 2.15752109511413e-07
479 2.14879335658225e-07
480 2.13990475117498e-07
481 2.13100847190617e-07
482 2.12224364304348e-07
483 2.11360955404416e-07
484 2.1051073417766e-07
485 2.09677537554853e-07
486 2.08854103789236e-07
487 2.08045392469103e-07
488 2.07246287686758e-07
489 2.06453094619974e-07
490 2.05667063823967e-07
491 2.04892415922586e-07
492 2.04119643854028e-07
493 2.0335838257779e-07
494 2.02600418219845e-07
495 2.01844770231219e-07
496 2.01094621843367e-07
497 2.00349958845436e-07
498 1.9961200337093e-07
499 1.98872626810953e-07
500 1.98138110363288e-07
501 1.9741027301734e-07
502 1.96683046738144e-07
503 1.95956843640488e-07
504 1.95232431110526e-07
505 1.94512963958005e-07
506 1.93790469893429e-07
507 1.93070050613642e-07
508 1.92340436910854e-07
509 1.9161409170465e-07
510 1.90876804140316e-07
511 1.90133064847942e-07
512 1.89380713777609e-07
513 1.88622109931202e-07
514 1.87850687893842e-07
515 1.87070625656816e-07
516 1.86281567948754e-07
517 1.85483017389743e-07
518 1.846803030503e-07
519 1.83869317993413e-07
520 1.83058062930286e-07
521 1.8224325515348e-07
522 1.81431460077874e-07
523 1.80619394996029e-07
524 1.7980897837333e-07
525 1.7900289606132e-07
526 1.78200863842903e-07
527 1.77399002154743e-07
528 1.76608082824714e-07
529 1.75815458192119e-07
530 1.75030891114147e-07
531 1.7424962095447e-07
532 1.73478284182238e-07
533 1.72709718526676e-07
534 1.71941650251028e-07
535 1.71187693354113e-07
536 1.70432770119078e-07
537 1.6968145644114e-07
538 1.68933596000898e-07
539 1.68192386240662e-07
540 1.674497411841e-07
541 1.66708048254804e-07
542 1.65973446542012e-07
543 1.65236926363832e-07
544 1.6450702844395e-07
545 1.63777230000051e-07
546 1.63048909485042e-07
547 1.62320034746699e-07
548 1.6159644644631e-07
549 1.60872446031135e-07
550 1.60141482297149e-07
551 1.5941819242471e-07
552 1.58693111984576e-07
553 1.57964592517601e-07
554 1.57241842657641e-07
555 1.56523725536317e-07
556 1.55807839519184e-07
557 1.55089963982391e-07
558 1.54382121309027e-07
559 1.53672814917627e-07
560 1.52969505506917e-07
561 1.52265442920907e-07
562 1.51570361595077e-07
563 1.50885512084642e-07
564 1.50211562299774e-07
565 1.49546025340896e-07
566 1.48905328956062e-07
567 1.48282680356715e-07
568 1.47691508800563e-07
569 1.47135210681881e-07
570 1.46612933349388e-07
571 1.46113507071277e-07
572 1.45625051573006e-07
573 1.45154899655608e-07
574 1.44683070857354e-07
575 1.44219399089707e-07
576 1.43757958426249e-07
577 1.43298407806469e-07
578 1.42848108453109e-07
579 1.42398192792825e-07
580 1.4194986874827e-07
581 1.41504884254573e-07
582 1.41067360459601e-07
583 1.40631499334631e-07
584 1.4019994409864e-07
585 1.39772680540773e-07
586 1.39348912853166e-07
587 1.38929920012743e-07
588 1.38512604053176e-07
589 1.38101000857205e-07
590 1.37694271984401e-07
591 1.37290285806557e-07
592 1.36889639179572e-07
593 1.36499139102852e-07
594 1.36113456505882e-07
595 1.3572734758327e-07
596 1.35343128704335e-07
597 1.34974655452424e-07
598 1.34602629486835e-07
599 1.34235278892447e-07
600 1.33865825091561e-07
601 1.33497636056745e-07
602 1.33117225686874e-07
603 1.32722135504082e-07
604 1.32327173218982e-07
605 1.31980229411965e-07
606 1.3164570589197e-07
607 1.31324455310278e-07
608 1.31008562220813e-07
609 1.30708414758374e-07
610 1.30406490939095e-07
611 1.3012254385103e-07
612 1.29842916862799e-07
613 1.29573379581416e-07
614 1.29307807128498e-07
615 1.29059188225256e-07
616 1.28815358380052e-07
617 1.285879562829e-07
618 1.28379923580724e-07
619 1.28194415083271e-07
620 1.28031430790543e-07
621 1.27909515867941e-07
622 1.27828968743415e-07
623 1.2777756808191e-07
624 1.27675960470697e-07
625 1.27469846233907e-07
626 1.27195633581323e-07
627 1.26910350672915e-07
628 1.2663342374708e-07
629 1.26361541674669e-07
630 1.26091549645935e-07
631 1.25826730368317e-07
632 1.2556435535771e-07
633 1.25305561482492e-07
634 1.25050007682148e-07
635 1.24797196576765e-07
636 1.24541045920523e-07
637 1.24297343973012e-07
638 1.24050856697977e-07
639 1.23809101637562e-07
640 1.23568241860994e-07
641 1.23329968459984e-07
642 1.2309490671214e-07
643 1.22859532325492e-07
644 1.22630368082355e-07
645 1.22401800695116e-07
646 1.22172565397705e-07
647 1.21953036114064e-07
648 1.2172843355529e-07
649 1.21510396411395e-07
650 1.21295926192033e-07
651 1.21075402148563e-07
652 1.20866332053993e-07
653 1.20654959800959e-07
654 1.20443658602198e-07
655 1.20232940048481e-07
656 1.20032311201612e-07
657 1.198310997097e-07
658 1.19628225547785e-07
659 1.19431021516903e-07
660 1.19232382189693e-07
661 1.19043292556853e-07
662 1.18845477459217e-07
663 1.18655293590564e-07
664 1.18465479204133e-07
665 1.18279132266252e-07
666 1.18094078516151e-07
667 1.17907205776646e-07
668 1.17728518489457e-07
669 1.17542313660124e-07
670 1.17362588980541e-07
671 1.17188790227374e-07
672 1.17013179590231e-07
673 1.16834264929366e-07
674 1.16665027860563e-07
675 1.16492095969534e-07
676 1.16321395182695e-07
677 1.16153927365303e-07
678 1.15984761350774e-07
679 1.15819467794154e-07
680 1.15656561661126e-07
681 1.15494245278569e-07
682 1.15333357086911e-07
683 1.15178117710002e-07
684 1.15014529455948e-07
685 1.14859865618655e-07
686 1.14704739928584e-07
687 1.1454577730774e-07
688 1.14397529671351e-07
689 1.1424648960201e-07
690 1.14094135028608e-07
691 1.13944416568756e-07
692 1.13795927347837e-07
693 1.13648042088244e-07
694 1.13504064813696e-07
695 1.13359959641457e-07
696 1.1321375126272e-07
697 1.13072850638218e-07
698 1.12929271267603e-07
699 1.12790949913233e-07
700 1.12651832750998e-07
701 1.12513511396628e-07
702 1.12375779792728e-07
703 1.12240172711608e-07
704 1.12105574601173e-07
705 1.1196739535535e-07
706 1.11836321536884e-07
707 1.11703030825083e-07
708 1.11570415128881e-07
709 1.11442311379051e-07
710 1.11314918171956e-07
711 1.11184228046568e-07
712 1.11060948881914e-07
713 1.10934074371016e-07
714 1.10805451924989e-07
715 1.10683913590037e-07
716 1.10558055155252e-07
717 1.10434100974999e-07
718 1.10314857693083e-07
719 1.10190839563984e-07
720 1.10074310555319e-07
721 1.09949830573441e-07
722 1.09831390204818e-07
723 1.09713994334015e-07
724 1.09598602193728e-07
725 1.09481092636088e-07
726 1.09365437594988e-07
727 1.09250969160257e-07
728 1.09138511561468e-07
729 1.09023126526608e-07
730 1.0890878598957e-07
731 1.08800094267281e-07
732 1.08683479993488e-07
733 1.08578205981757e-07
734 1.08465911807798e-07
735 1.08356069006277e-07
736 1.08246986485483e-07
737 1.08141257726402e-07
738 1.08032857326634e-07
739 1.0792552274097e-07
740 1.07822870631935e-07
741 1.07718598485462e-07
742 1.07613367106296e-07
743 1.07508313362814e-07
744 1.07404289906299e-07
745 1.07303357310684e-07
746 1.07196996168568e-07
747 1.07098969692743e-07
748 1.06996218107724e-07
749 1.06896209217666e-07
750 1.06796761656369e-07
751 1.06692908730111e-07
752 1.06595727800141e-07
753 1.06501140351156e-07
754 1.06404236532853e-07
755 1.06304199221086e-07
756 1.06207082239962e-07
757 1.06106682551399e-07
758 1.06012933542843e-07
759 1.05916747372703e-07
760 1.05824533136456e-07
761 1.05726527976913e-07
762 1.05635201919085e-07
763 1.05543982442668e-07
764 1.05449053933171e-07
765 1.05356775748078e-07
766 1.05266380501234e-07
767 1.05173718623064e-07
768 1.05082520462929e-07
769 1.04991869420701e-07
770 1.04903513431509e-07
771 1.04811427092955e-07
772 1.04722737148677e-07
773 1.04633109287988e-07
774 1.04546586499055e-07
775 1.04459132899137e-07
776 1.04368901077123e-07
777 1.04280857726735e-07
778 1.04191947514209e-07
779 1.04104827869378e-07
780 1.04015775548305e-07
781 1.03930481998304e-07
782 1.03845799515057e-07
783 1.03757422209583e-07
784 1.03669442808041e-07
785 1.03589385958003e-07
786 1.03497242776029e-07
787 1.03415679575392e-07
788 1.0333243238847e-07
789 1.03248517291377e-07
790 1.0316603749061e-07
791 1.03077503865734e-07
792 1.02994313522231e-07
793 1.02911492660951e-07
794 1.02826462011762e-07
795 1.02745445929031e-07
796 1.02660926870612e-07
797 1.02578049165913e-07
798 1.02495739895403e-07
799 1.02412769820148e-07
800 1.02330574236476e-07
801 1.02247405209255e-07
802 1.02163731696692e-07
803 1.02084889874732e-07
804 1.02003454571786e-07
805 1.01921209250122e-07
806 1.01841976629657e-07
807 1.01758224957393e-07
808 1.01677898101116e-07
809 1.01595205137528e-07
810 1.01516540951252e-07
811 1.01428916821078e-07
812 1.01349954206853e-07
813 1.01268256003095e-07
814 1.01186493850491e-07
815 1.01103935890023e-07
816 1.01027090693151e-07
817 1.00944134828751e-07
818 1.00863339014268e-07
819 1.0078513668077e-07
820 1.0070804279394e-07
821 1.00630977328819e-07
822 1.00552597359638e-07
823 1.00476121644988e-07
824 1.00397485880421e-07
825 1.00317201656708e-07
826 1.00244548661976e-07
827 1.00163099148176e-07
828 1.00088897170281e-07
829 1.00011739334604e-07
830 9.99363294340583e-08
831 9.98598963519726e-08
832 9.97823548232191e-08
833 9.97016726955735e-08
834 9.96257156771208e-08
835 9.95541853399118e-08
836 9.94781359509034e-08
837 9.9406378240019e-08
838 9.93263000736988e-08
839 9.92491777651594e-08
840 9.91755655377347e-08
841 9.91037367725767e-08
842 9.90257618127544e-08
843 9.89527180195182e-08
844 9.88752049124741e-08
845 9.88004487112448e-08
846 9.87234756166799e-08
847 9.86516823786587e-08
848 9.85762866889672e-08
849 9.85003438813692e-08
850 9.84276269377915e-08
851 9.83478827265571e-08
852 9.82726149345581e-08
853 9.81979297876023e-08
854 9.81236638608607e-08
855 9.8044786511764e-08
856 9.79711956006213e-08
857 9.78945635665696e-08
858 9.7815252786404e-08
859 9.77415339775689e-08
860 9.76675522679216e-08
861 9.75912755052377e-08
862 9.75168035211027e-08
863 9.74372440509796e-08
864 9.73592051423111e-08
865 9.72795035636409e-08
866 9.72060050230539e-08
867 9.71293161455833e-08
868 9.70511280229402e-08
869 9.69713980225606e-08
870 9.68928262068403e-08
871 9.68204787454852e-08
872 9.67425322073723e-08
873 9.66645856692594e-08
874 9.65899289440131e-08
875 9.65140714015433e-08
876 9.64386615009971e-08
877 9.63639550377593e-08
878 9.62908899282411e-08
879 9.62177182373125e-08
880 9.61438431090755e-08
881 9.60702877250696e-08
882 9.5999737936836e-08
883 9.59308508186041e-08
884 9.58619352786627e-08
885 9.57930765821402e-08
886 9.57231662823688e-08
887 9.56564178977715e-08
888 9.55863299623161e-08
889 9.5519546050582e-08
890 9.5454055326627e-08
891 9.53891898802794e-08
892 9.53239265299999e-08
893 9.5260126897756e-08
894 9.5192213223072e-08
895 9.51291028172818e-08
896 9.50679535094423e-08
897 9.50010914380073e-08
898 9.49384642012774e-08
899 9.48760074948041e-08
900 9.48162224290172e-08
901 9.47561034081446e-08
902 9.46954159530833e-08
903 9.46363982734511e-08
904 9.45773948046735e-08
905 9.45195921531194e-08
906 9.44574694017319e-08
907 9.4402530237403e-08
908 9.43443083656348e-08
909 9.4285617535661e-08
910 9.42304225759472e-08
911 9.4174346543241e-08
912 9.41205371418619e-08
913 9.4062201583256e-08
914 9.40057205411904e-08
915 9.39523729925895e-08
916 9.39009510148026e-08
917 9.38457347388066e-08
918 9.37921100785388e-08
919 9.37363040520722e-08
920 9.36841360044127e-08
921 9.36313853117099e-08
922 9.35784711941778e-08
923 9.35256281309194e-08
924 9.34784054607007e-08
925 9.34255979245791e-08
926 9.33763644184182e-08
927 9.33205868136611e-08
928 9.327148831062e-08
929 9.32273849230114e-08
930 9.31773271872771e-08
931 9.31264665382514e-08
932 9.30808354837609e-08
933 9.30321704117887e-08
934 9.2983853505757e-08
935 9.2939842488704e-08
936 9.28941972233588e-08
937 9.28489072293814e-08
938 9.28048820014737e-08
939 9.27565153574506e-08
940 9.27114811588581e-08
941 9.26658287880855e-08
942 9.26221019881268e-08
943 9.25730745393594e-08
944 9.25305556620515e-08
945 9.24858838402542e-08
946 9.24393930290535e-08
947 9.23947354181109e-08
948 9.23530194540945e-08
949 9.23069549685351e-08
950 9.22647700463131e-08
951 9.22191603081046e-08
952 9.21774727657976e-08
953 9.21308611623317e-08
954 9.20872267329287e-08
955 9.2046917643529e-08
956 9.2005151941521e-08
957 9.1960643544553e-08
958 9.19200147109223e-08
959 9.18758615853221e-08
960 9.1836199089812e-08
961 9.17925575549816e-08
962 9.17522200438725e-08
963 9.17123159638322e-08
964 9.16697615593876e-08
965 9.16282374419097e-08
966 9.15880136176384e-08
967 9.15456794814418e-08
968 9.15083049335408e-08
969 9.14686282271759e-08
970 9.14287099362809e-08
971 9.13870863428201e-08
972 9.13482622877382e-08
973 9.13105182576146e-08
974 9.127307265544e-08
975 9.12324296109546e-08
976 9.11957300786526e-08
977 9.11556981009198e-08
978 9.11196806896442e-08
979 9.10813753307593e-08
980 9.10423594291387e-08
981 9.10096673578664e-08
982 9.09682853489358e-08
983 9.09344279875768e-08
984 9.0900336147115e-08
985 9.08641908381469e-08
986 9.08264894405875e-08
987 9.07904293967476e-08
988 9.07527493154703e-08
989 9.07210022660365e-08
990 9.06859725091635e-08
991 9.06510351228462e-08
992 9.06179806747787e-08
993 9.05829722341878e-08
994 9.05472745671432e-08
995 9.0512678241339e-08
996 9.04753534314295e-08
997 9.04436348037052e-08
998 9.04106940424754e-08
999 9.03757779724401e-08
1000 9.03403218899257e-08
1001 9.03096974980144e-08
1002 9.02747814279792e-08
1003 9.02427714777332e-08
1004 9.02071235486801e-08
1005 9.01749359627502e-08
1006 9.01426346899825e-08
1007 9.01067167546898e-08
1008 9.00775063428227e-08
1009 9.00436063488996e-08
1010 9.00084202726248e-08
1011 8.99783785257569e-08
1012 8.99450469660223e-08
1013 8.99135343956914e-08
1014 8.98808778515559e-08
1015 8.98461181009225e-08
1016 8.98157992423876e-08
1017 8.97842724612019e-08
1018 8.97523122489474e-08
1019 8.97190162163497e-08
1020 8.96877523359763e-08
1021 8.96563463470557e-08
1022 8.96237253300569e-08
1023 8.95907490416903e-08
1024 8.95604799211469e-08
1025 8.95283633894906e-08
1026 8.94952378871494e-08
1027 8.94670719731039e-08
1028 8.94345291158061e-08
1029 8.94029597020563e-08
1030 8.93720155659139e-08
1031 8.93454270567418e-08
1032 8.93148168756852e-08
1033 8.92778260208615e-08
1034 8.92530422902382e-08
1035 8.92209257585819e-08
1036 8.91884752718397e-08
1037 8.91585045792453e-08
1038 8.91318094886628e-08
1039 8.9100737454828e-08
1040 8.9073296294373e-08
1041 8.9040312900579e-08
1042 8.90120475105505e-08
1043 8.89832776351795e-08
1044 8.89539464310474e-08
1045 8.89262707914895e-08
1046 8.88920723696174e-08
1047 8.88670541598913e-08
1048 8.88356979089622e-08
1049 8.88081075345326e-08
1050 8.87788615955287e-08
1051 8.87496724999437e-08
1052 8.87178899233731e-08
1053 8.86944917510846e-08
1054 8.86589148763051e-08
1055 8.86337261363224e-08
1056 8.86042386127883e-08
1057 8.85727615695941e-08
1058 8.85493136593141e-08
1059 8.8522249086509e-08
1060 8.84890809516037e-08
1061 8.84637003650823e-08
1062 8.8434802592019e-08
1063 8.84021531533108e-08
1064 8.83774120552516e-08
1065 8.83506530158229e-08
1066 8.83207533775021e-08
1067 8.82942075008941e-08
1068 8.82652884115487e-08
1069 8.82405828406263e-08
1070 8.82129569390599e-08
1071 8.81893740256601e-08
1072 8.81608244185372e-08
1073 8.81354438320159e-08
1074 8.81087771631428e-08
1075 8.80799220226436e-08
1076 8.80558630456107e-08
1077 8.80304895645168e-08
1078 8.80059332075689e-08
1079 8.79793233821147e-08
1080 8.79518324836681e-08
1081 8.79296138123209e-08
1082 8.79012915788735e-08
1083 8.78771473367124e-08
1084 8.78500756584799e-08
1085 8.78342873988913e-08
1086 8.77930261822257e-08
1087 8.77919319464127e-08
1088 8.77311308045137e-08
1089 8.77584511727036e-08
1090 8.76787993320249e-08
1091 8.77127206422301e-08
1092 8.76267236549211e-08
1093 8.76633379220948e-08
1094 8.75804886391052e-08
1095 8.76216645906425e-08
1096 8.75329249083734e-08
1097 8.75734684768759e-08
1098 8.74833077091353e-08
1099 8.75298056257634e-08
1100 8.74368453196439e-08
1101 8.74853327559322e-08
1102 8.73911787380166e-08
1103 8.74427144026413e-08
1104 8.73475016760494e-08
1105 8.73949161928067e-08
1106 8.73006058554893e-08
1107 8.73528662737044e-08
1108 8.72533192364244e-08
1109 8.73082157681893e-08
1110 8.72088961045847e-08
1111 8.72621654934846e-08
1112 8.71678835778766e-08
1113 8.72137775331794e-08
1114 8.71191261353488e-08
1115 8.71761898224577e-08
1116 8.70742198344487e-08
1117 8.7128945835957e-08
1118 8.70313741074824e-08
1119 8.70858727353152e-08
1120 8.69875194098313e-08
1121 8.70395169272342e-08
1122 8.69454055418828e-08
1123 8.69950582682577e-08
1124 8.69023253358137e-08
1125 8.69481482368428e-08
1126 8.6858520376154e-08
1127 8.69049827656454e-08
1128 8.68133298581597e-08
1129 8.68602469950019e-08
1130 8.67709033514075e-08
1131 8.6815759914316e-08
1132 8.67279439376034e-08
1133 8.67710312490999e-08
1134 8.66855032199965e-08
1135 8.67255209868745e-08
1136 8.66407958710624e-08
1137 8.66836202817467e-08
1138 8.65982769937546e-08
1139 8.66408527144813e-08
1140 8.65506564196039e-08
1141 8.65971543362321e-08
1142 8.65106954961448e-08
1143 8.65509264258435e-08
1144 8.64667200062286e-08
1145 8.65078675360564e-08
1146 8.6421465539388e-08
1147 8.64605382844275e-08
1148 8.6381149344561e-08
1149 8.64176712411791e-08
1150 8.63363993630628e-08
1151 8.63730988953648e-08
1152 8.62917417521203e-08
1153 8.63272262563441e-08
1154 8.62495710407529e-08
1155 8.6281602307281e-08
1156 8.62071871665648e-08
1157 8.62360067799273e-08
1158 8.61626858750242e-08
1159 8.61928981521487e-08
1160 8.61155697862159e-08
1161 8.61507629679181e-08
1162 8.60707629612989e-08
1163 8.61031494991948e-08
1164 8.60302620253606e-08
1165 8.60593303286805e-08
1166 8.59840554312541e-08
1167 8.60141398106862e-08
1168 8.59371525052666e-08
1169 8.59659436969196e-08
1170 8.58951878512926e-08
1171 8.59190762980688e-08
1172 8.58513971024877e-08
1173 8.58748023802036e-08
1174 8.58067110698357e-08
1175 8.58326174579815e-08
1176 8.57600568338057e-08
1177 8.5785217152079e-08
1178 8.57161452927357e-08
1179 8.57381508012622e-08
1180 8.56726103393157e-08
1181 8.56947721672441e-08
1182 8.56277466709798e-08
1183 8.56474642318972e-08
1184 8.55854693782021e-08
1185 8.56013926409105e-08
1186 8.55382609188382e-08
1187 8.55539781241532e-08
1188 8.5488721879301e-08
1189 8.55094626217578e-08
1190 8.54484412116108e-08
1191 8.54627160151722e-08
1192 8.54038333386598e-08
1193 8.5414924910765e-08
1194 8.53573922654505e-08
1195 8.53683133072991e-08
1196 8.53099777486932e-08
1197 8.53228740993472e-08
1198 8.52672457085646e-08
1199 8.52782946481057e-08
1200 8.52206625268082e-08
1201 8.52306385468182e-08
1202 8.51768291454391e-08
1203 8.5184019837925e-08
1204 8.5131404148342e-08
1205 8.51365058451847e-08
1206 8.50815879971378e-08
1207 8.50869099622287e-08
1208 8.50398080842751e-08
1209 8.50411865371825e-08
1210 8.49935446467498e-08
1211 8.49946033554261e-08
1212 8.49457677531973e-08
1213 8.49486951892686e-08
1214 8.49028154448206e-08
1215 8.49044710093949e-08
1216 8.48530348207532e-08
1217 8.48540580022927e-08
1218 8.4811183853617e-08
1219 8.48080361492975e-08
1220 8.47619858745929e-08
1221 8.47601313580526e-08
1222 8.47174064233513e-08
1223 8.47112033852682e-08
1224 8.46737222559568e-08
1225 8.46630001660742e-08
1226 8.46250358677025e-08
1227 8.46139300847426e-08
1228 8.45810319560769e-08
1229 8.45689243078596e-08
1230 8.45331626919688e-08
1231 8.45227532408899e-08
1232 8.44889243012403e-08
1233 8.44757508389193e-08
1234 8.44399750121738e-08
1235 8.44274765654518e-08
1236 8.43946068584955e-08
1237 8.4383330545279e-08
1238 8.43472349743024e-08
1239 8.43365199898471e-08
1240 8.430281894789e-08
1241 8.42887715180041e-08
1242 8.42566123537836e-08
1243 8.42422736013759e-08
1244 8.42118978994222e-08
1245 8.41968983422703e-08
1246 8.41644691718102e-08
1247 8.41524823158579e-08
1248 8.41172820287284e-08
1249 8.4097564467811e-08
1250 8.40712033323143e-08
1251 8.40536173996043e-08
1252 8.40238101318391e-08
1253 8.40072971186601e-08
1254 8.39819804809849e-08
1255 8.39604794578008e-08
1256 8.39346085967918e-08
1257 8.39142515474123e-08
1258 8.38856522022979e-08
1259 8.38663751778768e-08
1260 8.38400353586621e-08
1261 8.38177243167593e-08
1262 8.37962375044299e-08
1263 8.37718872048754e-08
1264 8.37485671922877e-08
1265 8.37268885334197e-08
1266 8.37042577472857e-08
1267 8.36819609162376e-08
1268 8.36603817333526e-08
1269 8.36336795373427e-08
1270 8.36129885328774e-08
1271 8.35898958939651e-08
1272 8.35731910342474e-08
1273 8.35499562867881e-08
1274 8.35308142654867e-08
1275 8.35038918012287e-08
1276 8.34850837350132e-08
1277 8.34633695490083e-08
1278 8.34432896112958e-08
1279 8.34188114140488e-08
1280 8.33974809211213e-08
1281 8.33756814699882e-08
1282 8.33545925615908e-08
1283 8.33281035284017e-08
1284 8.33092954621861e-08
1285 8.32830409080998e-08
1286 8.32591169341867e-08
1287 8.3238496983995e-08
1288 8.32150561791423e-08
1289 8.31889366281757e-08
1290 8.31692972269593e-08
1291 8.31438029536002e-08
1292 8.3122955629733e-08
1293 8.30990884992389e-08
1294 8.30756263781041e-08
1295 8.30513329219684e-08
1296 8.30268973572856e-08
1297 8.30021562592265e-08
1298 8.29808755042905e-08
1299 8.29570794280698e-08
1300 8.29327362339427e-08
1301 8.29088762088759e-08
1302 8.28848527589798e-08
1303 8.28618240689138e-08
1304 8.28371184979915e-08
1305 8.28137700636944e-08
1306 8.27892918664475e-08
1307 8.27644370815506e-08
1308 8.27424884164429e-08
1309 8.27198007868901e-08
1310 8.26934964948123e-08
1311 8.26645560891848e-08
1312 8.26466930448078e-08
1313 8.26189250346943e-08
1314 8.25972676921083e-08
1315 8.2567417791779e-08
1316 8.25515158453527e-08
1317 8.25175092700192e-08
1318 8.25115122893294e-08
1319 8.24580084213267e-08
1320 8.24818329192567e-08
1321 8.24037087454599e-08
1322 8.24331394255751e-08
1323 8.23501835611751e-08
1324 8.23960064622042e-08
1325 8.22998273974918e-08
1326 8.2340804397063e-08
1327 8.22538765987701e-08
1328 8.22933117206048e-08
1329 8.22076202666722e-08
1330 8.22458545712834e-08
1331 8.21597438971367e-08
1332 8.21955126184548e-08
1333 8.2110254595591e-08
1334 8.21403318695957e-08
1335 8.20649077581947e-08
1336 8.20848384819328e-08
1337 8.20164132164791e-08
1338 8.20284640212776e-08
1339 8.19705974208773e-08
1340 8.19728072087855e-08
1341 8.1922983952154e-08
1342 8.19169443389001e-08
1343 8.18767702526202e-08
1344 8.18606693542279e-08
1345 8.1834940601766e-08
1346 8.18121463908028e-08
1347 8.17870002833843e-08
1348 8.17653429407983e-08
1349 8.17424208321427e-08
1350 8.17170473510487e-08
1351 8.16971592598748e-08
1352 8.16746350551512e-08
1353 8.16548393345329e-08
1354 8.16275118609155e-08
1355 8.160832720705e-08
1356 8.15858953728821e-08
1357 8.15640106566207e-08
1358 8.15428649048044e-08
1359 8.15179674873434e-08
1360 8.1496757786681e-08
1361 8.14697642681494e-08
1362 8.14475882293664e-08
1363 8.14262932635756e-08
1364 8.1402873775005e-08
1365 8.13778626707062e-08
1366 8.13561058521373e-08
1367 8.13324092518997e-08
1368 8.13075047290113e-08
1369 8.12845115660821e-08
1370 8.12600120525531e-08
1371 8.12357754398363e-08
1372 8.12152904927643e-08
1373 8.11894125263279e-08
1374 8.11673714906647e-08
1375 8.11441580594874e-08
1376 8.11164966307842e-08
1377 8.10961537922594e-08
1378 8.10725850897143e-08
1379 8.10462381650723e-08
1380 8.10242610782552e-08
1381 8.1003314278405e-08
1382 8.09768962994895e-08
1383 8.0951593872669e-08
1384 8.09295812587152e-08
1385 8.09063962492473e-08
1386 8.08868563240139e-08
1387 8.08614331049284e-08
1388 8.08384257311445e-08
1389 8.08170455002255e-08
1390 8.0794841039733e-08
1391 8.07749813702685e-08
1392 8.07514339840054e-08
1393 8.07289808335554e-08
1394 8.07046944828471e-08
1395 8.06811328857293e-08
1396 8.06611595294271e-08
1397 8.06380384688055e-08
1398 8.0615457420663e-08
1399 8.05920805646565e-08
1400 8.05692650374112e-08
1401 8.0545774494567e-08
1402 8.05243445256565e-08
1403 8.04961288736195e-08
1404 8.04762976258644e-08
1405 8.04540576382351e-08
1406 8.04285704703034e-08
1407 8.04056483616478e-08
1408 8.03803956728188e-08
1409 8.03579638386509e-08
1410 8.03374078373054e-08
1411 8.03092561341145e-08
1412 8.02871724658871e-08
1413 8.02620832018874e-08
1414 8.02396868948563e-08
1415 8.02136170818812e-08
1416 8.01881228085222e-08
1417 8.01635096081554e-08
1418 8.01402890715508e-08
1419 8.01164645736208e-08
1420 8.00914889964588e-08
1421 8.0066378416177e-08
1422 8.0043228933846e-08
1423 8.00189425831377e-08
1424 7.99936188400352e-08
1425 7.99665542672301e-08
1426 7.99422537056671e-08
1427 7.99166741671797e-08
1428 7.98909027821537e-08
1429 7.98658206235814e-08
1430 7.98397152834696e-08
1431 7.98141428504096e-08
1432 7.97872701241431e-08
1433 7.97611150460398e-08
1434 7.97353152393043e-08
1435 7.97079522385502e-08
1436 7.96828629745505e-08
1437 7.96543062620003e-08
1438 7.96298280647534e-08
1439 7.96002197489543e-08
1440 7.95737733483293e-08
1441 7.95452166357791e-08
1442 7.95188697111371e-08
1443 7.94909738033311e-08
1444 7.9464761881809e-08
1445 7.94375623058841e-08
1446 7.94083305777349e-08
1447 7.93803351939459e-08
1448 7.93510466223779e-08
1449 7.93228664974777e-08
1450 7.92937413507389e-08
1451 7.92639056612643e-08
1452 7.92349084122179e-08
1453 7.92050087738971e-08
1454 7.91782568398958e-08
1455 7.91431418178945e-08
1456 7.91160417179526e-08
1457 7.9084834680998e-08
1458 7.90560363839177e-08
1459 7.90248790849546e-08
1460 7.89946099644112e-08
1461 7.89634526654481e-08
1462 7.89329561712293e-08
1463 7.89025946801303e-08
1464 7.88713094834748e-08
1465 7.88403156093409e-08
1466 7.88087035630269e-08
1467 7.87738656526926e-08
1468 7.87436675864228e-08
1469 7.87126523960069e-08
1470 7.86820635312324e-08
1471 7.86498830507298e-08
1472 7.86171341360387e-08
1473 7.8583063611859e-08
1474 7.85508831313564e-08
1475 7.85183118523491e-08
1476 7.8484688970093e-08
1477 7.84535814091214e-08
1478 7.84164697620326e-08
1479 7.83877496246532e-08
1480 7.83552991379111e-08
1481 7.83228202294595e-08
1482 7.82869733484404e-08
1483 7.82545441779803e-08
1484 7.82218236849985e-08
1485 7.81872770971859e-08
1486 7.81551889872389e-08
1487 7.81211610956234e-08
1488 7.80863089744344e-08
1489 7.80496236529871e-08
1490 7.80163560420988e-08
1491 7.79814399720635e-08
1492 7.79483002588677e-08
1493 7.79138389361833e-08
1494 7.78789939204216e-08
1495 7.78424649183762e-08
1496 7.7807548848341e-08
1497 7.77722632960831e-08
1498 7.77381288230572e-08
1499 7.77045130462284e-08
1500 7.76705348926043e-08
1501 7.76312987227357e-08
1502 7.7595004199793e-08
1503 7.75630226712565e-08
1504 7.7525697861347e-08
1505 7.74872859210518e-08
1506 7.74530803937523e-08
1507 7.74162032257664e-08
1508 7.73782105056853e-08
1509 7.73400614662023e-08
1510 7.73079875671101e-08
1511 7.72682611227538e-08
1512 7.72324497688714e-08
1513 7.71941230937045e-08
1514 7.71569332869149e-08
1515 7.71232393503851e-08
1516 7.70828165741477e-08
1517 7.70478294498389e-08
1518 7.70093890878343e-08
1519 7.69712258374966e-08
1520 7.69328707406203e-08
1521 7.68964980579767e-08
1522 7.68578232168693e-08
1523 7.68187646826846e-08
1524 7.67828964853834e-08
1525 7.6744242960558e-08
1526 7.67029106896189e-08
1527 7.66662608953084e-08
1528 7.66268399843284e-08
1529 7.65909717870272e-08
1530 7.65500658417295e-08
1531 7.65119807510928e-08
1532 7.64738459224645e-08
1533 7.64335723602017e-08
1534 7.63963754479846e-08
1535 7.63579564022621e-08
1536 7.63187486541028e-08
1537 7.62789582609003e-08
1538 7.62382867947053e-08
1539 7.61956258088503e-08
1540 7.61609797450546e-08
1541 7.61199032695004e-08
1542 7.60813136935212e-08
1543 7.6042780960961e-08
1544 7.60033671554083e-08
1545 7.59655449655838e-08
1546 7.59247313908418e-08
1547 7.58832996439196e-08
1548 7.58448877036244e-08
1549 7.58030509473429e-08
1550 7.57666782646993e-08
1551 7.57235696369207e-08
1552 7.56865077278235e-08
1553 7.56446141281231e-08
1554 7.56040918759027e-08
1555 7.55653033479575e-08
1556 7.55219957682129e-08
1557 7.54833848759517e-08
1558 7.54428768345861e-08
1559 7.54024753746307e-08
1560 7.53611288928369e-08
1561 7.5319206871427e-08
1562 7.52836584183569e-08
1563 7.5240244257202e-08
1564 7.51971711565602e-08
1565 7.51574020796397e-08
1566 7.5117071673958e-08
1567 7.50786881553722e-08
1568 7.50377182612283e-08
1569 7.49982973502483e-08
1570 7.49559774249065e-08
1571 7.4914360936873e-08
1572 7.48724531263179e-08
1573 7.48327266819615e-08
1574 7.47938386780334e-08
1575 7.47541406553864e-08
1576 7.47110320276079e-08
1577 7.46710853150034e-08
1578 7.46299448906029e-08
1579 7.45883355079968e-08
1580 7.4546612438553e-08
1581 7.45066444096665e-08
1582 7.44660155760357e-08
1583 7.4423461171591e-08
1584 7.43852197615524e-08
1585 7.4343525113818e-08
1586 7.43022283700157e-08
1587 7.42610168913416e-08
1588 7.42190309210855e-08
1589 7.41788497293783e-08
1590 7.41390664416031e-08
1591 7.40988710390411e-08
1592 7.40558405709635e-08
1593 7.40146575139988e-08
1594 7.39744123734454e-08
1595 7.39366967650312e-08
1596 7.38927354859698e-08
1597 7.38513037390476e-08
1598 7.38118970389223e-08
1599 7.37704866082822e-08
1600 7.37313357035418e-08
1601 7.3689037094482e-08
1602 7.36496374997841e-08
1603 7.36069694085018e-08
1604 7.35700709242337e-08
1605 7.35275236252164e-08
1606 7.34858360829094e-08
1607 7.34478717845377e-08
1608 7.34067739927013e-08
1609 7.3366273056763e-08
1610 7.33250260509521e-08
1611 7.32858183027929e-08
1612 7.32446423512556e-08
1613 7.32042053641635e-08
1614 7.31675768861351e-08
1615 7.31249301111347e-08
1616 7.30860776343434e-08
1617 7.30449301045155e-08
1618 7.3006532375075e-08
1619 7.29646245645199e-08
1620 7.2927647920551e-08
1621 7.28875662048267e-08
1622 7.28483939838043e-08
1623 7.28087456991489e-08
1624 7.2768571612869e-08
1625 7.27282198909052e-08
1626 7.26892466218487e-08
1627 7.26496836023216e-08
1628 7.26104474324529e-08
1629 7.25732860473727e-08
1630 7.2531605610493e-08
1631 7.24947568642165e-08
1632 7.24539219731923e-08
1633 7.24151334452472e-08
1634 7.23760820164898e-08
1635 7.23368600574759e-08
1636 7.23013400261152e-08
1637 7.22603985536807e-08
1638 7.22226971561213e-08
1639 7.21796666880437e-08
1640 7.21422068750144e-08
1641 7.21042567874974e-08
1642 7.20665482845106e-08
1643 7.20240009854933e-08
1644 7.19889072797741e-08
1645 7.19494934742215e-08
1646 7.19119555014913e-08
1647 7.18731385518367e-08
1648 7.18341865990624e-08
1649 7.17981691877867e-08
1650 7.17594161869783e-08
1651 7.17215726808718e-08
1652 7.16839352321585e-08
1653 7.16454948701539e-08
1654 7.16070616135767e-08
1655 7.15693815322993e-08
1656 7.15332504341859e-08
1657 7.14947532287624e-08
1658 7.14591266159914e-08
1659 7.14216881192442e-08
1660 7.13843135713432e-08
1661 7.1347947994127e-08
1662 7.13109642447307e-08
1663 7.12762826537983e-08
1664 7.12379915057682e-08
1665 7.11993664026522e-08
1666 7.1163377413086e-08
1667 7.11261378683048e-08
1668 7.10906888912177e-08
1669 7.10549059590448e-08
1670 7.10175172002891e-08
1671 7.09795173747807e-08
1672 7.09441394519672e-08
1673 7.09086904748801e-08
1674 7.08698664197982e-08
1675 7.08343677047196e-08
1676 7.07985634562647e-08
1677 7.07631286900323e-08
1678 7.07277010292273e-08
1679 7.06937726135948e-08
1680 7.06547069739827e-08
1681 7.06188387766815e-08
1682 7.05837024383982e-08
1683 7.0544793118188e-08
1684 7.05092304542632e-08
1685 7.04738525314497e-08
1686 7.04398459561162e-08
1687 7.04048019883885e-08
1688 7.03688698422411e-08
1689 7.03342237784454e-08
1690 7.02977160926821e-08
1691 7.02618336845262e-08
1692 7.02264983942769e-08
1693 7.01931028856961e-08
1694 7.01576610140364e-08
1695 7.01207909514778e-08
1696 7.00843543199881e-08
1697 7.00509232842705e-08
1698 7.00154103583372e-08
1699 6.99807785053963e-08
1700 6.99457132213865e-08
1701 6.99105484613938e-08
1702 6.98756110750764e-08
1703 6.98405457910667e-08
1704 6.98032422974393e-08
1705 6.97700244245425e-08
1706 6.97367141810901e-08
1707 6.97026507623377e-08
1708 6.96663633448225e-08
1709 6.96309712111542e-08
1710 6.95976680731292e-08
1711 6.95643365133947e-08
1712 6.95295199193424e-08
1713 6.94941917345204e-08
1714 6.94588777605532e-08
1715 6.9424956450348e-08
1716 6.93918593697163e-08
1717 6.9357319887331e-08
1718 6.93226240855438e-08
1719 6.92868482587983e-08
1720 6.92527848400459e-08
1721 6.92180464056946e-08
1722 6.91854538104053e-08
1723 6.91504595806691e-08
1724 6.91176609279864e-08
1725 6.90842725248331e-08
1726 6.9046379280735e-08
1727 6.90115840029648e-08
1728 6.89802917008819e-08
1729 6.89456740587957e-08
1730 6.89127119812838e-08
1731 6.88813699412094e-08
1732 6.88470649379269e-08
1733 6.88123265035756e-08
1734 6.8778774675593e-08
1735 6.87441570335068e-08
1736 6.87096743945403e-08
1737 6.86764494162162e-08
1738 6.86393448745548e-08
1739 6.86083012624294e-08
1740 6.85730938698725e-08
1741 6.85407002265492e-08
1742 6.85033825220671e-08
1743 6.84702712305807e-08
1744 6.84351988411436e-08
1745 6.84014267449129e-08
1746 6.83650895894061e-08
1747 6.8330116675952e-08
1748 6.82973961829703e-08
1749 6.82607534940871e-08
1750 6.82255318906755e-08
1751 6.81898058019215e-08
1752 6.81559484405625e-08
1753 6.81195615470642e-08
1754 6.80848657452771e-08
1755 6.80487914905825e-08
1756 6.80137048902907e-08
1757 6.79806646530778e-08
1758 6.79423735050477e-08
1759 6.79110385704007e-08
1760 6.78737066550639e-08
1761 6.78410145837915e-08
1762 6.78071572224326e-08
1763 6.77741667232112e-08
1764 6.77450628927545e-08
1765 6.77141258620395e-08
1766 6.76840627988895e-08
1767 6.76567069035627e-08
1768 6.76307365665707e-08
1769 6.76020732726101e-08
1770 6.75767708457897e-08
1771 6.75482922929405e-08
1772 6.75181794917989e-08
1773 6.74889690799318e-08
1774 6.74605757922109e-08
1775 6.74296103397864e-08
1776 6.74016291668522e-08
1777 6.73714808385739e-08
1778 6.73438194098708e-08
1779 6.7312569740352e-08
1780 6.72794939760024e-08
1781 6.7251846758154e-08
1782 6.72223166020558e-08
1783 6.71907898208701e-08
1784 6.71615367764389e-08
1785 6.71308626465361e-08
1786 6.70996342932995e-08
1787 6.70718165451945e-08
1788 6.70416042680699e-08
1789 6.70082016540618e-08
1790 6.69790125584768e-08
1791 6.69511166506709e-08
1792 6.69184956336721e-08
1793 6.68875088649656e-08
1794 6.68566215722421e-08
1795 6.68277806425976e-08
1796 6.67965025513695e-08
1797 6.67623254457794e-08
1798 6.6735552195496e-08
1799 6.67057094005941e-08
1800 6.66709638608154e-08
1801 6.66415829186917e-08
1802 6.6611704596653e-08
1803 6.65793322696118e-08
1804 6.65473436356478e-08
1805 6.6515475793949e-08
1806 6.64865709154583e-08
1807 6.64531683014502e-08
1808 6.64234534042407e-08
1809 6.63908039655325e-08
1810 6.63564634351133e-08
1811 6.6326236947134e-08
1812 6.62946888496663e-08
1813 6.62634036530108e-08
1814 6.62276065099832e-08
1815 6.61949641767023e-08
1816 6.61633023923969e-08
1817 6.61267023360779e-08
1818 6.60954242448497e-08
1819 6.60637127225527e-08
1820 6.60294148246976e-08
1821 6.59944845438076e-08
1822 6.59620411624928e-08
1823 6.59284182802367e-08
1824 6.58947243437069e-08
1825 6.58570371570022e-08
1826 6.58238477058148e-08
1827 6.57897842870625e-08
1828 6.57523315794606e-08
1829 6.57207834819928e-08
1830 6.56844605373408e-08
1831 6.5652315583975e-08
1832 6.56132144172261e-08
1833 6.55778791269768e-08
1834 6.55408030070248e-08
1835 6.55072582844696e-08
1836 6.54693295132347e-08
1837 6.54329994631553e-08
1838 6.53992131560699e-08
1839 6.5358463530174e-08
1840 6.53244001114217e-08
1841 6.52861587013831e-08
1842 6.52461125127957e-08
1843 6.52081695307061e-08
1844 6.51718252697719e-08
1845 6.51297682452423e-08
1846 6.50941700541807e-08
1847 6.50579252692296e-08
1848 6.50165574711536e-08
1849 6.49822453624438e-08
1850 6.49387885687247e-08
1851 6.48975131412044e-08
1852 6.48592717311658e-08
1853 6.48182663098851e-08
1854 6.4777950115058e-08
1855 6.47365112627085e-08
1856 6.46953637328807e-08
1857 6.46557865024988e-08
1858 6.46140208004908e-08
1859 6.4571523239465e-08
1860 6.45338289473329e-08
1861 6.44918927150684e-08
1862 6.44501199076331e-08
1863 6.44049436004934e-08
1864 6.43646771436579e-08
1865 6.43224282725896e-08
1866 6.42771453840396e-08
1867 6.42338022771582e-08
1868 6.4190601278824e-08
1869 6.41473789642077e-08
1870 6.41075033058769e-08
1871 6.40622559444637e-08
1872 6.40178896560428e-08
1873 6.39730757256984e-08
1874 6.39279207348409e-08
1875 6.38840091937709e-08
1876 6.38404387132141e-08
1877 6.37924699731229e-08
1878 6.37497734601311e-08
1879 6.37047108398292e-08
1880 6.36564649880711e-08
1881 6.36143582255499e-08
1882 6.35667234405446e-08
1883 6.3519323134642e-08
1884 6.34750350059221e-08
1885 6.34295673762608e-08
1886 6.33843342257023e-08
1887 6.33352570389434e-08
1888 6.32868193406466e-08
1889 6.32428438507304e-08
1890 6.31929566452527e-08
1891 6.31469774248217e-08
1892 6.3100131342253e-08
1893 6.30507130949809e-08
1894 6.30034406867708e-08
1895 6.29545127139863e-08
1896 6.29048528821841e-08
1897 6.28604013286349e-08
1898 6.28091072485404e-08
1899 6.2762644859049e-08
1900 6.27135179342986e-08
1901 6.2664049949035e-08
1902 6.26145109094978e-08
1903 6.25662579523123e-08
1904 6.25153191435857e-08
1905 6.24662845893909e-08
1906 6.24167455498537e-08
1907 6.23655509457421e-08
1908 6.23179374770189e-08
1909 6.22668778760271e-08
1910 6.22163298658052e-08
1911 6.21665989797293e-08
1912 6.21149069957028e-08
1913 6.20630444814196e-08
1914 6.20112459159827e-08
1915 6.19596391970845e-08
1916 6.19087714426314e-08
1917 6.18561770693304e-08
1918 6.18015008058137e-08
1919 6.17483379983241e-08
1920 6.16971078670758e-08
1921 6.16422610733025e-08
1922 6.15908959389344e-08
1923 6.15356796629385e-08
1924 6.14827158074149e-08
1925 6.1428174547018e-08
1926 6.13754949085887e-08
1927 6.13182251640865e-08
1928 6.12630302043726e-08
1929 6.12090929053011e-08
1930 6.11524484384063e-08
1931 6.10993424743356e-08
1932 6.10421224678248e-08
1933 6.09896204650795e-08
1934 6.09303896226265e-08
1935 6.08763457421446e-08
1936 6.0820141811746e-08
1937 6.07642576255785e-08
1938 6.07053323165019e-08
1939 6.06495191846079e-08
1940 6.05931518293801e-08
1941 6.05363155159466e-08
1942 6.04790031388802e-08
1943 6.04223586719854e-08
1944 6.03638881102597e-08
1945 6.03059717718679e-08
1946 6.02504499624956e-08
1947 6.01939689204301e-08
1948 6.013521414161e-08
1949 6.00763527813797e-08
1950 6.00172711529012e-08
1951 5.99588645400218e-08
1952 5.99029519321448e-08
1953 5.98444316324276e-08
1954 5.97876663732677e-08
1955 5.97268083879499e-08
1956 5.9667897289728e-08
1957 5.96091283000533e-08
1958 5.95524163315986e-08
1959 5.94921232277557e-08
1960 5.94334466086366e-08
1961 5.93761591005659e-08
1962 5.93164131146295e-08
1963 5.92568802915139e-08
1964 5.92005591215639e-08
1965 5.91403654937039e-08
1966 5.90799018596044e-08
1967 5.90210653683698e-08
1968 5.89601540923468e-08
1969 5.89014206298089e-08
1970 5.88437316650925e-08
1971 5.87866146872784e-08
1972 5.87247832584126e-08
1973 5.86667958657472e-08
1974 5.8606328678934e-08
1975 5.85467816449636e-08
1976 5.84877675180451e-08
1977 5.84295740679863e-08
1978 5.83692667532887e-08
1979 5.83095278727797e-08
1980 5.82515227165459e-08
1981 5.81907997343478e-08
1982 5.81319916648226e-08
1983 5.8074128617136e-08
1984 5.8014013148977e-08
1985 5.79550949453278e-08
1986 5.78958641028748e-08
1987 5.78369387937983e-08
1988 5.7773707595743e-08
1989 5.7718093415815e-08
1990 5.76585215128489e-08
1991 5.75965550808633e-08
1992 5.7538400710655e-08
1993 5.74796175101255e-08
1994 5.7421864596563e-08
1995 5.73615821508611e-08
1996 5.73032394868278e-08
1997 5.72451313018973e-08
1998 5.71856304532048e-08
1999 5.71281155714587e-08
2000 5.70681315537058e-08
2001 5.7009717835399e-08
2002 5.69511797721134e-08
2003 5.68891032060037e-08
2004 5.68333717865244e-08
2005 5.6775494527983e-08
2006 5.67166367204663e-08
2007 5.66579636540609e-08
2008 5.66035751603522e-08
2009 5.65419071563156e-08
2010 5.64841009520478e-08
2011 5.64263089586348e-08
2012 5.63699771305437e-08
2013 5.63151019150609e-08
2014 5.62578570395544e-08
2015 5.62005517679154e-08
2016 5.61426816148014e-08
2017 5.60879804822889e-08
2018 5.60322455100959e-08
2019 5.5975878154868e-08
2020 5.592238849772e-08
2021 5.58669803751854e-08
2022 5.58110713200222e-08
2023 5.57572583659294e-08
2024 5.57039392390379e-08
2025 5.56490391545594e-08
2026 5.55965549153825e-08
2027 5.55426922232982e-08
2028 5.54905064120703e-08
2029 5.54378800643462e-08
2030 5.53870549424573e-08
2031 5.53339383202456e-08
2032 5.5279880228909e-08
2033 5.52291226085799e-08
2034 5.5179260272098e-08
2035 5.51290177952524e-08
2036 5.50762280226991e-08
2037 5.50279963817957e-08
2038 5.49778746972152e-08
2039 5.49288756701571e-08
2040 5.48803100741679e-08
2041 5.48305614245237e-08
2042 5.47797469607758e-08
2043 5.47340803791485e-08
2044 5.46852767513428e-08
2045 5.4637613544628e-08
2046 5.45920606498385e-08
2047 5.45426068754296e-08
2048 5.44937677204871e-08
2049 5.44475398100985e-08
2050 5.4401510851676e-08
2051 5.43567360011821e-08
2052 5.4312781827548e-08
2053 5.42651612533973e-08
2054 5.4219682965595e-08
2055 5.41741442816601e-08
2056 5.41285096744559e-08
2057 5.40856497366349e-08
2058 5.40407683047306e-08
2059 5.3995591997591e-08
2060 5.39515099262644e-08
2061 5.39097086971196e-08
2062 5.38653353032714e-08
2063 5.38237969749389e-08
2064 5.3780393471925e-08
2065 5.37378141984846e-08
2066 5.36938458139957e-08
2067 5.3650694553653e-08
2068 5.36114477256433e-08
2069 5.35698667647466e-08
2070 5.35271702517548e-08
2071 5.34867119483806e-08
2072 5.34442250454958e-08
2073 5.34030277776765e-08
2074 5.33605835073558e-08
2075 5.33233688315704e-08
2076 5.32803099417833e-08
2077 5.32426014387966e-08
2078 5.3201162586447e-08
2079 5.31609920528808e-08
2080 5.31234114475865e-08
2081 5.30841042234442e-08
2082 5.30433901246852e-08
2083 5.30048289704155e-08
2084 5.29648218616785e-08
2085 5.29246584335397e-08
2086 5.28874615213226e-08
2087 5.28502610563919e-08
2088 5.28114263431689e-08
2089 5.27710000142179e-08
2090 5.27359702573449e-08
2091 5.26978070070072e-08
2092 5.26629939656686e-08
2093 5.26248342680447e-08
2094 5.25888914637562e-08
2095 5.25522487748731e-08
2096 5.25127425987648e-08
2097 5.24794678824492e-08
2098 5.24414183189492e-08
2099 5.2403571260129e-08
2100 5.23700443011421e-08
2101 5.23344674263626e-08
2102 5.22977998684837e-08
2103 5.22629513000084e-08
2104 5.2225747282364e-08
2105 5.21902698835675e-08
2106 5.21565866051787e-08
2107 5.21241609874323e-08
2108 5.20893905786579e-08
2109 5.2055654009564e-08
2110 5.20210434729051e-08
2111 5.19856797609464e-08
2112 5.19517122654634e-08
2113 5.19208249727399e-08
2114 5.18864382570428e-08
2115 5.18528580073507e-08
2116 5.18178389086188e-08
2117 5.17830436308486e-08
2118 5.17538580879773e-08
2119 5.17210203554441e-08
2120 5.16871487832304e-08
2121 5.16553662066599e-08
2122 5.16223401803018e-08
2123 5.15892395469564e-08
2124 5.15575564463688e-08
2125 5.15269356071713e-08
2126 5.14938172102575e-08
2127 5.14615159374898e-08
2128 5.14313924782073e-08
2129 5.13998799078763e-08
2130 5.1369045905858e-08
2131 5.13376754440742e-08
2132 5.13078788344501e-08
2133 5.12739148916808e-08
2134 5.12474933600515e-08
2135 5.12178530698293e-08
2136 5.11865714258875e-08
2137 5.11544584469448e-08
2138 5.11247897350131e-08
2139 5.10964142108605e-08
2140 5.10645392637343e-08
2141 5.10352720084484e-08
2142 5.10067970083128e-08
2143 5.09740729626174e-08
2144 5.09477864341079e-08
2145 5.09200965836953e-08
2146 5.08904918206099e-08
2147 5.08581017300003e-08
2148 5.08323587666837e-08
2149 5.08026509749016e-08
2150 5.0777018145709e-08
2151 5.07468023158708e-08
2152 5.07187074560989e-08
2153 5.06901436381213e-08
2154 5.06605104533264e-08
2155 5.06326820470804e-08
2156 5.06053901005998e-08
2157 5.0578389476641e-08
2158 5.05525186156319e-08
2159 5.05216171120537e-08
2160 5.04954087432452e-08
2161 5.04681985091793e-08
2162 5.04409527479766e-08
2163 5.04162436243405e-08
2164 5.03892074732448e-08
2165 5.03624093539656e-08
2166 5.03334973700476e-08
2167 5.03088699588261e-08
2168 5.02839441196556e-08
2169 5.02555259629389e-08
2170 5.02315380401797e-08
2171 5.02063741691927e-08
2172 5.01790644591438e-08
2173 5.01510548645001e-08
2174 5.0129941087107e-08
2175 5.01012280551549e-08
2176 5.00771051292759e-08
2177 5.00500298983297e-08
2178 5.00265713299086e-08
2179 5.00020576055249e-08
2180 4.99755898886178e-08
2181 4.99514136720336e-08
2182 4.99245054186304e-08
2183 4.99008159238201e-08
2184 4.98771264290099e-08
2185 4.98516925517833e-08
2186 4.98254664194064e-08
2187 4.98004837368171e-08
2188 4.97779666375209e-08
2189 4.97545329380955e-08
2190 4.97309073921315e-08
2191 4.97072214500349e-08
2192 4.96835603769341e-08
2193 4.96603682620389e-08
2194 4.96363803392796e-08
2195 4.96142291694923e-08
2196 4.95923337950899e-08
2197 4.95677312528642e-08
2198 4.95441483394643e-08
2199 4.95203700268121e-08
2200 4.9499444543244e-08
2201 4.94779577309146e-08
2202 4.94527618855045e-08
2203 4.94298184605668e-08
2204 4.94076637380658e-08
2205 4.93851608496243e-08
2206 4.93637806187053e-08
2207 4.93415228675076e-08
2208 4.93223630826378e-08
2209 4.92992491274435e-08
2210 4.92777374461184e-08
2211 4.92552594266726e-08
2212 4.92354068626355e-08
2213 4.92157496978507e-08
2214 4.91946430258849e-08
2215 4.91722573769948e-08
2216 4.9150454373148e-08
2217 4.91281042513947e-08
2218 4.91045994976957e-08
2219 4.90842815281667e-08
2220 4.90635692074193e-08
2221 4.90457097157559e-08
2222 4.90274665310153e-08
2223 4.90068359226825e-08
2224 4.89871254671925e-08
2225 4.89655036517433e-08
2226 4.89448161999917e-08
2227 4.89267968362128e-08
2228 4.89075340226464e-08
2229 4.88875002702116e-08
2230 4.88686531241456e-08
2231 4.88471840753846e-08
2232 4.88322591252199e-08
2233 4.88114828556263e-08
2234 4.87911542279562e-08
2235 4.87710529739616e-08
2236 4.87539715265939e-08
2237 4.87369433699314e-08
2238 4.87181566199979e-08
2239 4.87010645144892e-08
2240 4.86791940090825e-08
2241 4.86626667850487e-08
2242 4.86440647762265e-08
2243 4.86278395328554e-08
2244 4.8608974623221e-08
2245 4.85918718595713e-08
2246 4.8574520405964e-08
2247 4.85585829324009e-08
2248 4.85403894856518e-08
2249 4.85241216097165e-08
2250 4.85085251966666e-08
2251 4.84904347786141e-08
2252 4.84745577011836e-08
2253 4.84584674609323e-08
2254 4.84402420397601e-08
2255 4.84252673516039e-08
2256 4.84085340701768e-08
2257 4.83945896689875e-08
2258 4.83823114905135e-08
2259 4.8362917226541e-08
2260 4.83503903581095e-08
2261 4.83333231215965e-08
2262 4.83212367896613e-08
2263 4.83058606448594e-08
2264 4.82906479248868e-08
2265 4.82758650832693e-08
2266 4.82649831212711e-08
2267 4.82502358067904e-08
2268 4.82411763869095e-08
2269 4.82263686762963e-08
2270 4.82154405290203e-08
2271 4.82032049831105e-08
2272 4.81918576156204e-08
2273 4.81813238195627e-08
2274 4.81668465113216e-08
2275 4.81575561650516e-08
2276 4.81433595211911e-08
2277 4.81314046396619e-08
2278 4.81233684013205e-08
2279 4.81118362927191e-08
2280 4.809775688841e-08
2281 4.80871662489335e-08
2282 4.80753286069557e-08
2283 4.80604569474963e-08
2284 4.80493120846859e-08
2285 4.80343054221066e-08
2286 4.80221302723294e-08
2287 4.80062247731894e-08
2288 4.79949555654002e-08
2289 4.79758917037998e-08
2290 4.79619650661789e-08
2291 4.79460595670389e-08
2292 4.79297597166806e-08
2293 4.79110333628796e-08
2294 4.78931490022205e-08
2295 4.78776627232946e-08
2296 4.7858456753147e-08
2297 4.78428674455245e-08
2298 4.78232635714448e-08
2299 4.78011124016575e-08
2300 4.77861554770698e-08
2301 4.77661430409171e-08
2302 4.77466812753846e-08
2303 4.77263135678641e-08
2304 4.77080668304097e-08
2305 4.76861288234431e-08
2306 4.7668997638084e-08
2307 4.76462247434029e-08
2308 4.76266279747506e-08
2309 4.76074930588766e-08
2310 4.75858321635769e-08
2311 4.75660897336638e-08
2312 4.75445496306293e-08
2313 4.75250168108232e-08
2314 4.75072567951429e-08
2315 4.74819472628951e-08
2316 4.74624322066575e-08
2317 4.74387213955652e-08
2318 4.74208476930471e-08
2319 4.73991725868927e-08
2320 4.73760728425532e-08
2321 4.73561350133878e-08
2322 4.73352770313795e-08
2323 4.73127990119337e-08
2324 4.72921968253104e-08
2325 4.72668411077848e-08
2326 4.72467220902217e-08
2327 4.72260381911838e-08
2328 4.72044696664398e-08
2329 4.71804710855395e-08
2330 4.71613326169518e-08
2331 4.71403929225289e-08
2332 4.71186716310967e-08
2333 4.70978420707979e-08
2334 4.70735344038076e-08
2335 4.70525094442564e-08
2336 4.70291787735277e-08
2337 4.70084842163487e-08
2338 4.69883794096404e-08
2339 4.69670489167129e-08
2340 4.69417820170293e-08
2341 4.69190801766217e-08
2342 4.68961651733935e-08
2343 4.68754102200819e-08
2344 4.68526621943965e-08
2345 4.68306637912974e-08
2346 4.68103884543325e-08
2347 4.67859067043719e-08
2348 4.67668783699082e-08
2349 4.67429721595636e-08
2350 4.67191618724883e-08
2351 4.66945628829762e-08
2352 4.66721026270989e-08
2353 4.66522429576344e-08
2354 4.662607722139e-08
2355 4.66030236623283e-08
2356 4.65801015536726e-08
2357 4.65579788055948e-08
2358 4.65365950219621e-08
2359 4.65133851434985e-08
2360 4.64888785245421e-08
2361 4.64631177976571e-08
2362 4.64383838050253e-08
2363 4.64163605329304e-08
2364 4.63910154735458e-08
2365 4.6362465866423e-08
2366 4.63369325132135e-08
2367 4.63065177314093e-08
2368 4.62812899115761e-08
2369 4.6251482643811e-08
2370 4.62160798520017e-08
2371 4.61839846366274e-08
2372 4.61413911523323e-08
2373 4.61012170660524e-08
2374 4.60634730359288e-08
2375 4.60231177612513e-08
2376 4.59894664572857e-08
2377 4.59626221527287e-08
2378 4.5937433412746e-08
2379 4.59160602872544e-08
2380 4.58943567593906e-08
2381 4.5873093768023e-08
2382 4.585602653151e-08
2383 4.58351614440744e-08
2384 4.58124347346711e-08
2385 4.57933424513612e-08
2386 4.57763960071134e-08
2387 4.57549376164934e-08
2388 4.5733749232113e-08
2389 4.57146747123716e-08
2390 4.56965949524601e-08
2391 4.56773818768852e-08
2392 4.5657330360882e-08
2393 4.56363515866087e-08
2394 4.561682587223e-08
2395 4.55974316082575e-08
2396 4.55771242968694e-08
2397 4.55585755787524e-08
2398 4.55386164333049e-08
2399 4.55182380676433e-08
2400 4.55010393807243e-08
2401 4.54814284012173e-08
2402 4.54640272096185e-08
2403 4.54414426087624e-08
2404 4.54226167789784e-08
2405 4.54043238562463e-08
2406 4.53855513171675e-08
2407 4.53675852440938e-08
2408 4.53474910955265e-08
2409 4.53295498914486e-08
2410 4.53096618002746e-08
2411 4.52914434845297e-08
2412 4.52691928387594e-08
2413 4.52513582160918e-08
2414 4.52350477075925e-08
2415 4.52124737648774e-08
2416 4.51946959856286e-08
2417 4.51766943854182e-08
2418 4.51596982031788e-08
2419 4.51398030065775e-08
2420 4.51221033870297e-08
2421 4.51015367275431e-08
2422 4.50852688516079e-08
2423 4.50652670735963e-08
2424 4.50479191727027e-08
2425 4.50308057509119e-08
2426 4.50120971606793e-08
2427 4.49976518268613e-08
2428 4.49767689758573e-08
2429 4.4957936040646e-08
2430 4.49392096868451e-08
2431 4.49237766986244e-08
2432 4.49034551763816e-08
2433 4.48863737290139e-08
2434 4.48667805130754e-08
2435 4.48501396022039e-08
2436 4.48328272284471e-08
2437 4.48163319788364e-08
2438 4.47997621222385e-08
2439 4.4782556329892e-08
2440 4.47617978238668e-08
2441 4.47503403222527e-08
2442 4.47308963202886e-08
2443 4.47108234880034e-08
2444 4.46967085565575e-08
2445 4.46787815633343e-08
2446 4.466221525945e-08
2447 4.46447430135777e-08
2448 4.46282300003986e-08
2449 4.46119017283308e-08
2450 4.45939143389751e-08
2451 4.4577213031971e-08
2452 4.45613324018268e-08
2453 4.45434693574498e-08
2454 4.45279297878187e-08
2455 4.45113919056439e-08
2456 4.44934364907112e-08
2457 4.44767103147115e-08
2458 4.44603713845027e-08
2459 4.44444694380763e-08
2460 4.44284964373765e-08
2461 4.44098517959901e-08
2462 4.43948628969792e-08
2463 4.43781331682658e-08
2464 4.43611369860264e-08
2465 4.4347558514346e-08
2466 4.43286047868696e-08
2467 4.43121130899726e-08
2468 4.42941114897621e-08
2469 4.42794529931234e-08
2470 4.42655192500752e-08
2471 4.42492122942895e-08
2472 4.4232105977926e-08
2473 4.42185417171004e-08
2474 4.42004122191975e-08
2475 4.41865815048459e-08
2476 4.41707008747017e-08
2477 4.41537864048769e-08
2478 4.41376464266341e-08
2479 4.41218475089045e-08
2480 4.41038920939718e-08
2481 4.4092828943576e-08
2482 4.40753460395626e-08
2483 4.40591279016189e-08
2484 4.40451266570108e-08
2485 4.40295977455207e-08
2486 4.40144596325354e-08
2487 4.39985257116859e-08
2488 4.39819380915196e-08
2489 4.39653256023576e-08
2490 4.39498855087095e-08
2491 4.39352980663443e-08
2492 4.39177085809206e-08
2493 4.39013021491519e-08
2494 4.38863985152693e-08
2495 4.38739391483978e-08
2496 4.38549108139341e-08
2497 4.38385896472937e-08
2498 4.3824417872429e-08
2499 4.38078444631174e-08
2500 4.3792905302098e-08
2501 4.37779128503735e-08
2502 4.3763538570829e-08
2503 4.37449720891436e-08
2504 4.3731517962442e-08
2505 4.37139640041551e-08
2506 4.36977920514892e-08
2507 4.36826255167944e-08
2508 4.36676508286382e-08
2509 4.36514326906945e-08
2510 4.36331077935392e-08
2511 4.36175362494851e-08
2512 4.36026432737435e-08
2513 4.3586787512595e-08
2514 4.35705658219376e-08
2515 4.35555840283541e-08
2516 4.35409717169932e-08
2517 4.35232614393044e-08
2518 4.35076685789682e-08
2519 4.34927009962394e-08
2520 4.34768416823772e-08
2521 4.34616360678319e-08
2522 4.34465476928381e-08
2523 4.34271960614296e-08
2524 4.34103455404511e-08
2525 4.33968665447537e-08
2526 4.33795470655696e-08
2527 4.33650946263242e-08
2528 4.33478142269905e-08
2529 4.33327151938556e-08
2530 4.33154880852271e-08
2531 4.33001758892715e-08
2532 4.32827960139548e-08
2533 4.32669864380841e-08
2534 4.32520259607827e-08
2535 4.32335447442256e-08
2536 4.32178950404705e-08
2537 4.32039151121444e-08
2538 4.31845563753086e-08
2539 4.31696207670029e-08
2540 4.31547881873939e-08
2541 4.31384883370356e-08
2542 4.31184012938957e-08
2543 4.31037570081116e-08
2544 4.3089329437862e-08
2545 4.30733706480169e-08
2546 4.30579305543688e-08
2547 4.3040706998454e-08
2548 4.30238067394839e-08
2549 4.30094111436574e-08
2550 4.29913527000281e-08
2551 4.29753050923409e-08
2552 4.29573709936903e-08
2553 4.29437356785911e-08
2554 4.2927489118938e-08
2555 4.29126494339016e-08
2556 4.28951878461703e-08
2557 4.28791473439105e-08
2558 4.28617461523118e-08
2559 4.28482103131955e-08
2560 4.28321023093758e-08
2561 4.28139124153404e-08
2562 4.27978328332301e-08
2563 4.27818385162482e-08
2564 4.27665014512968e-08
2565 4.27496473776046e-08
2566 4.27319974960483e-08
2567 4.27153850068862e-08
2568 4.27023536531124e-08
2569 4.26840927048033e-08
2570 4.26707735812215e-08
2571 4.2653923060243e-08
2572 4.26365076577895e-08
2573 4.26204245229655e-08
2574 4.26055706270745e-08
2575 4.25891109046006e-08
2576 4.25735144915507e-08
2577 4.25563939643325e-08
2578 4.25384918401051e-08
2579 4.25235171519489e-08
2580 4.25061372766322e-08
2581 4.24915143071303e-08
2582 4.2474848527263e-08
2583 4.24588151304306e-08
2584 4.24429771328505e-08
2585 4.24267589949068e-08
2586 4.240975570724e-08
2587 4.23923900427781e-08
2588 4.23788613090892e-08
2589 4.23602628529807e-08
2590 4.23434016738611e-08
2591 4.23276951266871e-08
2592 4.23122905601758e-08
2593 4.22975787728319e-08
2594 4.22791259779842e-08
2595 4.22636645680541e-08
2596 4.22443839909192e-08
2597 4.22315480363977e-08
2598 4.2214146844799e-08
2599 4.21964010399734e-08
2600 4.21808792339107e-08
2601 4.21639789749406e-08
2602 4.2148766254968e-08
2603 4.21294892305468e-08
2604 4.21173851350432e-08
2605 4.20990780014563e-08
2606 4.20834531666969e-08
2607 4.20651034005459e-08
2608 4.20511234722198e-08
2609 4.20334345108131e-08
2610 4.20167758363732e-08
2611 4.20009342860794e-08
2612 4.19852419497602e-08
2613 4.19685157737604e-08
2614 4.19511998472899e-08
2615 4.19357846226376e-08
2616 4.19202130785834e-08
2617 4.19028722831172e-08
2618 4.18853289829713e-08
2619 4.18700523141524e-08
2620 4.18537382529394e-08
2621 4.18346921549073e-08
2622 4.18192094286951e-08
2623 4.18040357885729e-08
2624 4.17863006418884e-08
2625 4.17723846624085e-08
2626 4.17552428189083e-08
2627 4.17388079654302e-08
2628 4.171923251306e-08
2629 4.17052774537297e-08
2630 4.168985512365e-08
2631 4.16721235296791e-08
2632 4.16584349238747e-08
2633 4.16401420011425e-08
2634 4.1625558111491e-08
2635 4.16089633858974e-08
2636 4.15910328399605e-08
2637 4.15763068417618e-08
2638 4.15613961024519e-08
2639 4.15433731859594e-08
2640 4.15272332077166e-08
2641 4.15087875182962e-08
2642 4.14933474246482e-08
2643 4.14797867165362e-08
2644 4.14625738187624e-08
2645 4.14471230669733e-08
2646 4.14294447637076e-08
2647 4.14133332071742e-08
2648 4.1396788219572e-08
2649 4.13804919219274e-08
2650 4.13633181040041e-08
2651 4.1349526469503e-08
2652 4.13320435654896e-08
2653 4.13168166346622e-08
2654 4.12998311105639e-08
2655 4.12816163475327e-08
2656 4.12666096849534e-08
2657 4.12507432656639e-08
2658 4.12338891919717e-08
2659 4.12193017496065e-08
2660 4.12020106921318e-08
2661 4.11848191106401e-08
2662 4.11704021985315e-08
2663 4.11568841229837e-08
2664 4.11388114684996e-08
2665 4.11210301365372e-08
2666 4.11048795001534e-08
2667 4.10890166335776e-08
2668 4.1072443224266e-08
2669 4.1056736677092e-08
2670 4.10409732864991e-08
2671 4.10257499083855e-08
2672 4.10085831958895e-08
2673 4.09930294154037e-08
2674 4.09784206567565e-08
2675 4.09615523722096e-08
2676 4.09449683047569e-08
2677 4.0929769795639e-08
2678 4.09143439128457e-08
2679 4.08983176214406e-08
2680 4.08818188191162e-08
2681 4.08648119787358e-08
2682 4.08470057777777e-08
2683 4.08309830390863e-08
2684 4.08141751506719e-08
2685 4.07995024431784e-08
2686 4.07812805747199e-08
2687 4.07674072278041e-08
2688 4.07526918877466e-08
2689 4.07353759612761e-08
2690 4.07214315600868e-08
2691 4.07053093454124e-08
2692 4.068646575206e-08
2693 4.06718250189897e-08
2694 4.06564346633331e-08
2695 4.06410336495355e-08
2696 4.0627124775483e-08
2697 4.06080573611689e-08
2698 4.05937576886117e-08
2699 4.05798949998371e-08
2700 4.05621243260157e-08
2701 4.05460873764696e-08
2702 4.05315354612412e-08
2703 4.05176194817614e-08
2704 4.0499759990098e-08
2705 4.04863236269648e-08
2706 4.04674267429073e-08
2707 4.0452221128362e-08
2708 4.04404332243757e-08
2709 4.04222006977761e-08
2710 4.04064302017559e-08
2711 4.03915656477238e-08
2712 4.03763422696102e-08
2713 4.0358148822861e-08
2714 4.03458635389597e-08
2715 4.03276452232149e-08
2716 4.03146245275821e-08
2717 4.02969426716027e-08
2718 4.02841777713547e-08
2719 4.02676647581757e-08
2720 4.02528996801266e-08
2721 4.0238273157911e-08
2722 4.02222823936427e-08
2723 4.02063378146522e-08
2724 4.01941875338707e-08
2725 4.01777349168242e-08
2726 4.01628135193732e-08
2727 4.01479738343369e-08
2728 4.01347080014602e-08
2729 4.01189979015726e-08
2730 4.01034334629458e-08
2731 4.00895210361796e-08
2732 4.00746742457159e-08
2733 4.0061046036044e-08
2734 4.00442310422022e-08
2735 4.00303576952865e-08
2736 4.00138944200989e-08
2737 4.0001218337693e-08
2738 3.99852417842794e-08
2739 3.99725479383051e-08
2740 3.99579498377989e-08
2741 3.99436821396648e-08
2742 3.99286186336667e-08
2743 3.99138571083313e-08
2744 3.99009110196857e-08
2745 3.98855917183027e-08
2746 3.98685529034992e-08
2747 3.98545338953227e-08
2748 3.98400601397952e-08
2749 3.98267872014912e-08
2750 3.98105903798296e-08
2751 3.97972428345383e-08
2752 3.97846839916838e-08
2753 3.97688353359626e-08
2754 3.97552426534276e-08
2755 3.97407831087548e-08
2756 3.97286044062639e-08
2757 3.9711846255841e-08
2758 3.9699248333136e-08
2759 3.96823516268796e-08
2760 3.96695227777855e-08
2761 3.96562427340541e-08
2762 3.96423303072879e-08
2763 3.96287411774665e-08
2764 3.96141253133919e-08
2765 3.96014563364133e-08
2766 3.95862613800091e-08
2767 3.95753474435878e-08
2768 3.95595805002813e-08
2769 3.95462258495627e-08
2770 3.95318302537362e-08
2771 3.95204899916735e-08
2772 3.95052488499914e-08
2773 3.9490476666515e-08
2774 3.94770189870997e-08
2775 3.94635009115518e-08
2776 3.94502563949573e-08
2777 3.9436997667508e-08
2778 3.94232664291394e-08
2779 3.94102244172245e-08
2780 3.93965464695611e-08
2781 3.93836749879029e-08
2782 3.93691159672471e-08
2783 3.9355523284712e-08
2784 3.93427619371778e-08
2785 3.93315850999443e-08
2786 3.93160952683047e-08
2787 3.93028720679922e-08
2788 3.92893220180213e-08
2789 3.92745356236901e-08
2790 3.92637353741065e-08
2791 3.9252071815099e-08
2792 3.92377721425419e-08
2793 3.92272490046253e-08
2794 3.92112617930707e-08
2795 3.91983334679935e-08
2796 3.91858208104168e-08
2797 3.91761858509199e-08
2798 3.91616588046872e-08
2799 3.91467942506551e-08
2800 3.91335603922016e-08
2801 3.9120216399624e-08
2802 3.9107717952902e-08
2803 3.90950738449192e-08
2804 3.9082145519842e-08
2805 3.90693379870299e-08
2806 3.90578982489842e-08
2807 3.90449805820481e-08
2808 3.90296115426736e-08
2809 3.90167294028743e-08
2810 3.90067782518599e-08
2811 3.8994688367211e-08
2812 3.8980751071449e-08
2813 3.89678049828035e-08
2814 3.89528977962073e-08
2815 3.89418630675209e-08
2816 3.89298655534276e-08
2817 3.89171717074532e-08
2818 3.89036145520549e-08
2819 3.8889488962468e-08
2820 3.88764469505531e-08
2821 3.88663146111412e-08
2822 3.88535141837565e-08
2823 3.88383014637839e-08
2824 3.8825568537959e-08
2825 3.88082312952065e-08
2826 3.87942407087394e-08
2827 3.87766903031661e-08
2828 3.87604757179361e-08
2829 3.87404135437919e-08
2830 3.872229470403e-08
2831 3.86960756770804e-08
2832 3.86695440113272e-08
2833 3.8638056309992e-08
2834 3.86017795506177e-08
2835 3.85606000463667e-08
2836 3.85121303736469e-08
2837 3.84583564994045e-08
2838 3.84160401267764e-08
2839 3.83984932739168e-08
2840 3.83638365519801e-08
2841 3.83501266298936e-08
2842 3.83365090783627e-08
2843 3.83225007283272e-08
2844 3.8308677119403e-08
2845 3.8294430737551e-08
2846 3.82825717792912e-08
2847 3.82711391466728e-08
2848 3.82601825776874e-08
2849 3.82451581515397e-08
2850 3.82338782856095e-08
2851 3.82219376149351e-08
2852 3.82103735319106e-08
2853 3.81971680951665e-08
2854 3.81862435006042e-08
2855 3.81710236752042e-08
2856 3.81578999508747e-08
2857 3.81453624243022e-08
2858 3.81339191335428e-08
2859 3.81203832944266e-08
2860 3.81095190959968e-08
2861 3.80964593205135e-08
2862 3.80802873678476e-08
2863 3.80704570090984e-08
2864 3.80568607738496e-08
2865 3.8045541828069e-08
2866 3.80331357519026e-08
2867 3.80167399782749e-08
2868 3.80069025140983e-08
2869 3.79944431472268e-08
2870 3.79796283311862e-08
2871 3.79722031595975e-08
2872 3.79548232842808e-08
2873 3.7941703112665e-08
2874 3.79303308761791e-08
2875 3.79183617837953e-08
2876 3.79052238486111e-08
2877 3.78902775821643e-08
2878 3.78790225852299e-08
2879 3.78669398060083e-08
2880 3.78542956980255e-08
2881 3.78428204328429e-08
2882 3.78308477877454e-08
2883 3.78166689074533e-08
2884 3.7804184671586e-08
2885 3.77915299054621e-08
2886 3.77803210938055e-08
2887 3.77685154262508e-08
2888 3.77552176189511e-08
2889 3.77412980867575e-08
2890 3.77279860686031e-08
2891 3.77168412057927e-08
2892 3.7704847244413e-08
2893 3.76913291688652e-08
2894 3.76808486635127e-08
2895 3.76642859123422e-08
2896 3.76557238723763e-08
2897 3.76436162241589e-08
2898 3.76284567948915e-08
2899 3.76193831641558e-08
2900 3.76038897798026e-08
2901 3.75922795115002e-08
2902 3.75804667385182e-08
2903 3.75677764452576e-08
2904 3.75557966947326e-08
2905 3.7540395680935e-08
2906 3.75295385879326e-08
2907 3.75156616883032e-08
2908 3.75041473432702e-08
2909 3.74939119751616e-08
2910 3.74807704872637e-08
2911 3.74687090243242e-08
2912 3.74541606618095e-08
2913 3.74426889493407e-08
2914 3.74308690709313e-08
2915 3.74186619467309e-08
2916 3.74054316409911e-08
2917 3.73922333096743e-08
2918 3.73793156427382e-08
2919 3.73692010668947e-08
2920 3.73573882939127e-08
2921 3.7344499048686e-08
2922 3.73309667622834e-08
2923 3.73173349998979e-08
2924 3.73066875170025e-08
2925 3.72949422455804e-08
2926 3.72823798500121e-08
2927 3.72694017869435e-08
2928 3.7256675966546e-08
2929 3.72438400120245e-08
2930 3.72330042353042e-08
2931 3.72195536613162e-08
2932 3.72070942944447e-08
2933 3.71936863530209e-08
2934 3.71835717771773e-08
2935 3.71710449087459e-08
2936 3.71570720858472e-08
2937 3.71455008973953e-08
2938 3.7135073682748e-08
2939 3.71237973695315e-08
2940 3.71111283925529e-08
2941 3.7097777294548e-08
2942 3.70834563057088e-08
2943 3.70725885545653e-08
2944 3.70598840504499e-08
2945 3.70481174627457e-08
2946 3.70353312462157e-08
2947 3.70246731051793e-08
2948 3.70122208437351e-08
2949 3.69979176184643e-08
2950 3.69863109028756e-08
2951 3.69732937599565e-08
2952 3.69600776650714e-08
2953 3.69496362395694e-08
2954 3.69384487441948e-08
2955 3.69238009056971e-08
2956 3.69113131171162e-08
2957 3.68995110022752e-08
2958 3.68877870471351e-08
2959 3.68752992585542e-08
2960 3.68665311611949e-08
2961 3.6851719897868e-08
2962 3.68379957649267e-08
2963 3.68262611516457e-08
2964 3.68139829731717e-08
2965 3.68023727048694e-08
2966 3.67888368657532e-08
2967 3.67758730135392e-08
2968 3.67647032817331e-08
2969 3.67538213197349e-08
2970 3.67407118062602e-08
2971 3.67295633907361e-08
2972 3.67145389645884e-08
2973 3.67032875203677e-08
2974 3.66909489457612e-08
2975 3.66781556238038e-08
2976 3.66674939300538e-08
2977 3.66533186024753e-08
2978 3.66432075793455e-08
2979 3.66297179255071e-08
2980 3.66199444101767e-08
2981 3.66056269740511e-08
2982 3.65918531031184e-08
2983 3.65797454549011e-08
2984 3.65682559788638e-08
2985 3.65542440761146e-08
2986 3.65435788296509e-08
2987 3.65327466056442e-08
2988 3.65193422169341e-08
2989 3.65085668363463e-08
2990 3.64943559816311e-08
2991 3.6482909138158e-08
2992 3.64707233302397e-08
2993 3.64570063027259e-08
2994 3.64451615553207e-08
2995 3.64321159906922e-08
2996 3.64200332114706e-08
2997 3.64070373848335e-08
2998 3.63953596149713e-08
2999 3.63845700235288e-08
3000 2.26932410640757e-08
3001 2.27927010598705e-08
3002 2.2794488074851e-08
3003 2.28077698949392e-08
3004 2.28154970471905e-08
3005 2.28155503378957e-08
3006 2.28124417134268e-08
3007 2.28084200415424e-08
3008 2.28043575134507e-08
3009 2.28006431512995e-08
3010 2.27969110255799e-08
3011 2.27930456730974e-08
3012 2.2789388154365e-08
3013 2.27861054469258e-08
3014 2.27823999665588e-08
3015 2.27787726458928e-08
3016 2.2775628494287e-08
3017 2.2772288943429e-08
3018 2.2768693597186e-08
3019 2.27657146467664e-08
3020 2.27621814730128e-08
3021 2.27587939605201e-08
3022 2.27553815790316e-08
3023 2.27523155871268e-08
3024 2.27489902471234e-08
3025 2.2745908268007e-08
3026 2.27427570109739e-08
3027 2.27396856899986e-08
3028 2.27364225224846e-08
3029 2.27333334379409e-08
3030 2.27299743471576e-08
3031 2.27269687513854e-08
3032 2.27237411110082e-08
3033 2.27209024927788e-08
3034 2.27177583411731e-08
3035 2.2714750969044e-08
3036 2.27116512263592e-08
3037 2.27085710235997e-08
3038 2.2705503255338e-08
3039 2.27022471932514e-08
3040 2.26993588370306e-08
3041 2.26963194904783e-08
3042 2.26932801439261e-08
3043 2.26901946120961e-08
3044 2.26872156616764e-08
3045 2.26844178996544e-08
3046 2.26814069748116e-08
3047 2.26784884205244e-08
3048 2.26753389398482e-08
3049 2.26723759766401e-08
3050 2.26694059080046e-08
3051 2.26666774238993e-08
3052 2.26637322242595e-08
3053 2.26605756381559e-08
3054 2.26577174800013e-08
3055 2.26548593218467e-08
3056 2.26520260326879e-08
3057 2.26491945198859e-08
3058 2.26462670838146e-08
3059 2.26430945104994e-08
3060 2.26402434577722e-08
3061 2.26374741174595e-08
3062 2.26346585918691e-08
3063 2.26319283314069e-08
3064 2.26290097771198e-08
3065 2.2626352347288e-08
3066 2.26233254352337e-08
3067 2.26205596476348e-08
3068 2.26175007611573e-08
3069 2.26147314208447e-08
3070 2.2611885697188e-08
3071 2.2609253136352e-08
3072 2.26064607034004e-08
3073 2.26036771522331e-08
3074 2.26008722847837e-08
3075 2.25982468293751e-08
3076 2.25952412336028e-08
3077 2.25926797270404e-08
3078 2.25898162398153e-08
3079 2.25869936087975e-08
3080 2.25843344026089e-08
3081 2.25816467747109e-08
3082 2.25787317731374e-08
3083 2.25760050653889e-08
3084 2.25730829583881e-08
3085 2.25705427681078e-08
3086 2.25677538878699e-08
3087 2.25650911289677e-08
3088 2.25621423766142e-08
3089 2.25594618541436e-08
3090 2.2556493561865e-08
3091 2.25540581766381e-08
3092 2.25515481844241e-08
3093 2.25486225247096e-08
3094 2.25460610181472e-08
3095 2.25433165468303e-08
3096 2.25404512832483e-08
3097 2.25379910290258e-08
3098 2.25351097782323e-08
3099 2.25327738689884e-08
3100 2.25300063050327e-08
3101 2.2527380849624e-08
3102 2.25246292728798e-08
3103 2.25218492744261e-08
3104 2.2518985787201e-08
3105 2.25165734946131e-08
3106 2.25139480392045e-08
3107 2.2511077446552e-08
3108 2.25083667260151e-08
3109 2.25059881842071e-08
3110 2.25035154954867e-08
3111 2.2500525886926e-08
3112 2.2498335638943e-08
3113 2.24955662986304e-08
3114 2.24930865044826e-08
3115 2.2490228346328e-08
3116 2.24874856513679e-08
3117 2.24848815122414e-08
3118 2.24826166572711e-08
3119 2.24798561987427e-08
3120 2.24773479828855e-08
3121 2.24749125976587e-08
3122 2.24724363562245e-08
3123 2.24695675399289e-08
3124 2.24669687298729e-08
3125 2.24647020985458e-08
3126 2.24618297295365e-08
3127 2.2459335724534e-08
3128 2.24565912532171e-08
3129 2.24539871140905e-08
3130 2.24514646873786e-08
3131 2.24488179156879e-08
3132 2.24464997700125e-08
3133 2.24437037843472e-08
3134 2.24410943161502e-08
3135 2.24385381386583e-08
3136 2.24360388045852e-08
3137 2.24334915088775e-08
3138 2.24309744112361e-08
3139 2.24285034988725e-08
3140 2.24256684333568e-08
3141 2.2423133572147e-08
3142 2.24207319377001e-08
3143 2.24183374086806e-08
3144 2.24157812311887e-08
3145 2.24130189963034e-08
3146 2.24106742052754e-08
3147 2.24079776955932e-08
3148 2.24055956010716e-08
3149 2.24031300177785e-08
3150 2.24005969329255e-08
3151 2.23981615476987e-08
3152 2.23954774725144e-08
3153 2.23928449116784e-08
3154 2.23905072260777e-08
3155 2.23877325566946e-08
3156 2.23854623726538e-08
3157 2.23829452750124e-08
3158 2.23803766630226e-08
3159 2.23781810859691e-08
3160 2.23757634643107e-08
3161 2.23732588011671e-08
3162 2.23706688728953e-08
3163 2.23679901267815e-08
3164 2.23659171183499e-08
3165 2.23634000207085e-08
3166 2.23610516769668e-08
3167 2.2358438656056e-08
3168 2.23559144529872e-08
3169 2.235343998791e-08
3170 2.23512230945744e-08
3171 2.23484963868259e-08
3172 2.23461125159474e-08
3173 2.23435439039577e-08
3174 2.23411049660172e-08
3175 2.23388099129807e-08
3176 2.23362661699866e-08
3177 2.23340119731574e-08
3178 2.23315996805695e-08
3179 2.23289706724472e-08
3180 2.23268568078083e-08
3181 2.23242988539596e-08
3182 2.23219185357948e-08
3183 2.23193321602366e-08
3184 2.23166800594754e-08
3185 2.23144471789283e-08
3186 2.23120064646309e-08
3187 2.2309915692631e-08
3188 2.23073168825749e-08
3189 2.23048655101366e-08
3190 2.23024816392581e-08
3191 2.23001634935827e-08
3192 2.22975415908877e-08
3193 2.2295248314208e-08
3194 2.22927578619192e-08
3195 2.22903171476219e-08
3196 2.22881020306431e-08
3197 2.22856773035573e-08
3198 2.22832330365463e-08
3199 2.22810978556254e-08
3200 2.22783622660927e-08
3201 2.22760974111225e-08
3202 2.22737384092397e-08
3203 2.22716440845261e-08
3204 2.22692744245023e-08
3205 2.22667519977904e-08
3206 2.22644924718907e-08
3207 2.22618776746231e-08
3208 2.2259600385155e-08
3209 2.22571649999281e-08
3210 2.22549445538789e-08
3211 2.22525802229256e-08
3212 2.22502016811177e-08
3213 2.22478178102392e-08
3214 2.22453664378008e-08
3215 2.22428582219436e-08
3216 2.22406608685333e-08
3217 2.22382272596633e-08
3218 2.2235807861648e-08
3219 2.2233662022586e-08
3220 2.22312372955003e-08
3221 2.22289529006048e-08
3222 2.2226627649502e-08
3223 2.22243148328971e-08
3224 2.22219060930229e-08
3225 2.22196909760441e-08
3226 2.22172911179541e-08
3227 2.22147935602379e-08
3228 2.22128395677146e-08
3229 2.22104041824878e-08
3230 2.22079670209041e-08
3231 2.22057359167138e-08
3232 2.22036682373528e-08
3233 2.22011280470724e-08
3234 2.2198721083555e-08
3235 2.21964011615228e-08
3236 2.21942109135398e-08
3237 2.21917790810267e-08
3238 2.21895941621142e-08
3239 2.21871179206801e-08
3240 2.21849898451865e-08
3241 2.21828813096181e-08
3242 2.21803979627566e-08
3243 2.21780940279359e-08
3244 2.21755431795145e-08
3245 2.21736531358374e-08
3246 2.21712639358884e-08
3247 2.21687894708111e-08
3248 2.21665139576999e-08
3249 2.21641336395351e-08
3250 2.21619593787636e-08
3251 2.21598064342743e-08
3252 2.21573603909064e-08
3253 2.21552571844086e-08
3254 2.21527169941282e-08
3255 2.21503579922455e-08
3256 2.21484519613568e-08
3257 2.21458122950935e-08
3258 2.21438192227197e-08
3259 2.21415579204631e-08
3260 2.21394031996169e-08
3261 2.21372591369118e-08
3262 2.21346478923579e-08
3263 2.21323652738192e-08
3264 2.2130089760708e-08
3265 2.21278018130988e-08
3266 2.2125853149646e-08
3267 2.21233147357225e-08
3268 2.21209965900471e-08
3269 2.21188010129936e-08
3270 2.21166036595832e-08
3271 2.21141718270701e-08
3272 2.21119549337345e-08
3273 2.21097362640421e-08
3274 2.21076508211127e-08
3275 2.21053255700099e-08
3276 2.21031903890889e-08
3277 2.21010871825911e-08
3278 2.20987423915631e-08
3279 2.20966054342853e-08
3280 2.20941220874238e-08
3281 2.20922036220372e-08
3282 2.20897433678147e-08
3283 2.2087593976039e-08
3284 2.20854001753423e-08
3285 2.20833076269855e-08
3286 2.20809699413849e-08
3287 2.20787921278998e-08
3288 2.20764899694359e-08
3289 2.20744365009296e-08
3290 2.20722160548803e-08
3291 2.20698890274207e-08
3292 2.20675921980273e-08
3293 2.206552807138e-08
3294 2.20633342706833e-08
3295 2.20611724444097e-08
3296 2.20592060173885e-08
3297 2.2056900306211e-08
3298 2.20545075535483e-08
3299 2.20524238869757e-08
3300 2.20501412684371e-08
3301 2.20479972057319e-08
3302 2.20458638011678e-08
3303 2.20435474318492e-08
3304 2.20413873819325e-08
3305 2.2039424507625e-08
3306 2.20369127390541e-08
3307 2.20348272961246e-08
3308 2.20325180322334e-08
3309 2.20304805509386e-08
3310 2.20281979324e-08
3311 2.20261409111799e-08
3312 2.20238920434213e-08
3313 2.20216680446583e-08
3314 2.20196341160772e-08
3315 2.20173284048997e-08
3316 2.20153175689575e-08
3317 2.20129763306431e-08
3318 2.20108287152243e-08
3319 2.20087716940043e-08
3320 2.20065619060961e-08
3321 2.20043609999721e-08
3322 2.20022169372669e-08
3323 2.20000284656408e-08
3324 2.19977653870274e-08
3325 2.19957172475915e-08
3326 2.19934328526961e-08
3327 2.19915357035916e-08
3328 2.19892708486213e-08
3329 2.19872031692603e-08
3330 2.19850591065551e-08
3331 2.19829239256342e-08
3332 2.19806111090293e-08
3333 2.19788045541236e-08
3334 2.19764135778178e-08
3335 2.19743547802409e-08
3336 2.19723030880914e-08
3337 2.19699458625655e-08
3338 2.1967904828557e-08
3339 2.19655209576786e-08
3340 2.19636149267899e-08
3341 2.19613287555376e-08
3342 2.19592912742428e-08
3343 2.19571791859607e-08
3344 2.19548930147084e-08
3345 2.19527880318537e-08
3346 2.19507665377705e-08
3347 2.19487166219778e-08
3348 2.19463167638878e-08
3349 2.19444178384265e-08
3350 2.19422808811487e-08
3351 2.19404014956126e-08
3352 2.1938074468153e-08
3353 2.19359623798709e-08
3354 2.19338502915889e-08
3355 2.19318625482856e-08
3356 2.19294982173324e-08
3357 2.19274518542534e-08
3358 2.19254854272322e-08
3359 2.19232045850504e-08
3360 2.19211226948346e-08
3361 2.19190088301957e-08
3362 2.19167173298729e-08
3363 2.19148894586851e-08
3364 2.19125695366529e-08
3365 2.19104592247277e-08
3366 2.19082014751848e-08
3367 2.19061462303216e-08
3368 2.19043112537065e-08
3369 2.19020481750931e-08
3370 2.19000497736488e-08
3371 2.18978186694585e-08
3372 2.18957403319564e-08
3373 2.18937366014416e-08
3374 2.18917008965036e-08
3375 2.18895088721638e-08
3376 2.18873736912428e-08
3377 2.18852580502471e-08
3378 2.18829772080653e-08
3379 2.18810551899651e-08
3380 2.18788347439158e-08
3381 2.18768185789031e-08
3382 2.18747135960484e-08
3383 2.187267256204e-08
3384 2.18705178411938e-08
3385 2.18683968711275e-08
3386 2.18663629425464e-08
3387 2.18644693461556e-08
3388 2.1862296861741e-08
3389 2.18602380641642e-08
3390 2.18582556499314e-08
3391 2.18560849418736e-08
3392 2.1854289045109e-08
3393 2.18517470784718e-08
3394 2.18501163828932e-08
3395 2.18478835023461e-08
3396 2.1845949049748e-08
3397 2.18440074917226e-08
3398 2.18421156716886e-08
3399 2.18396856155323e-08
3400 2.18378222172078e-08
3401 2.18356301928679e-08
3402 2.1833638896851e-08
3403 2.18316298372656e-08
3404 2.1829501761772e-08
3405 2.18273950025605e-08
3406 2.18254569972487e-08
3407 2.18234319504518e-08
3408 2.18212452551825e-08
3409 2.18191651413235e-08
3410 2.18171685162361e-08
3411 2.1815056427954e-08
3412 2.18129034834647e-08
3413 2.18107345517637e-08
3414 2.18086668724027e-08
3415 2.1806849659356e-08
3416 2.18048548106253e-08
3417 2.18026343645761e-08
3418 2.18004583274478e-08
3419 2.17986677597537e-08
3420 2.17966658055957e-08
3421 2.17945110847495e-08
3422 2.17924629453137e-08
3423 2.17905302690724e-08
3424 2.17884768005661e-08
3425 2.17862563545168e-08
3426 2.17842224259357e-08
3427 2.17823448167564e-08
3428 2.17802718083249e-08
3429 2.17781916944659e-08
3430 2.17760582899018e-08
3431 2.17740581121006e-08
3432 2.17719371420344e-08
3433 2.1769773539404e-08
3434 2.17679243519342e-08
3435 2.17659419377014e-08
3436 2.17638262967057e-08
3437 2.17617603937015e-08
3438 2.17599485097253e-08
3439 2.17577476036013e-08
3440 2.17556070936098e-08
3441 2.17534754654025e-08
3442 2.17515001565971e-08
3443 2.1749510636937e-08
3444 2.1747817768869e-08
3445 2.17455990991766e-08
3446 2.17436291194417e-08
3447 2.17416005199311e-08
3448 2.17395257351427e-08
3449 2.17375113464868e-08
3450 2.17354063636321e-08
3451 2.17334150676152e-08
3452 2.17315818673569e-08
3453 2.17292832616067e-08
3454 2.1727377230718e-08
3455 2.1725398369199e-08
3456 2.17232329902117e-08
3457 2.1721449527945e-08
3458 2.17193676377292e-08
3459 2.17172484440198e-08
3460 2.17153779402679e-08
3461 2.17131965740691e-08
3462 2.17111377764923e-08
3463 2.17090665444175e-08
3464 2.17070503794048e-08
3465 2.1705057307031e-08
3466 2.17032507521253e-08
3467 2.17010427405739e-08
3468 2.16991669077515e-08
3469 2.1696971330698e-08
3470 2.16952749099164e-08
3471 2.16931788088459e-08
3472 2.16909423755851e-08
3473 2.16889048942903e-08
3474 2.16871853808698e-08
3475 2.16851869794255e-08
3476 2.16830766675002e-08
3477 2.1681243467242e-08
3478 2.16791207208189e-08
3479 2.16771862682208e-08
3480 2.1675015560163e-08
3481 2.16730509094987e-08
3482 2.16710915879048e-08
3483 2.16690043686185e-08
3484 2.16669722163942e-08
3485 2.16648530226848e-08
3486 2.1663247196102e-08
3487 2.16611653058862e-08
3488 2.16591367063756e-08
3489 2.16570690270146e-08
3490 2.16551168108481e-08
3491 2.16532782815193e-08
3492 2.16512408002245e-08
3493 2.16490647630962e-08
3494 2.16473559078167e-08
3495 2.16452260559663e-08
3496 2.16434177247038e-08
3497 2.1641435310471e-08
3498 2.16395044105866e-08
3499 2.16374154149435e-08
3500 2.16355999782536e-08
3501 2.16335038771831e-08
3502 2.16313384981959e-08
3503 2.1629578128568e-08
3504 2.16273789988009e-08
3505 2.1625806922998e-08
3506 2.16237250327822e-08
3507 2.1621580970077e-08
3508 2.16198206004492e-08
3509 2.16175966016863e-08
3510 2.16158255739174e-08
3511 2.16138698050372e-08
3512 2.16118962725886e-08
3513 2.1609746880813e-08
3514 2.16080362491766e-08
3515 2.16059028446125e-08
3516 2.16040731970679e-08
3517 2.16020250576321e-08
3518 2.16002167263696e-08
3519 2.15982289830663e-08
3520 2.15962536742609e-08
3521 2.15942783654555e-08
3522 2.15923190438616e-08
3523 2.15904183420434e-08
3524 2.15885442855779e-08
3525 2.15863380503833e-08
3526 2.15845261664072e-08
3527 2.15827267169288e-08
3528 2.15805577852279e-08
3529 2.1578721032256e-08
3530 2.15764579536426e-08
3531 2.15746194243138e-08
3532 2.15726423391516e-08
3533 2.15706030814999e-08
3534 2.15689688332077e-08
3535 2.15668833902782e-08
3536 2.15649436086096e-08
3537 2.1562915009099e-08
3538 2.15609983200693e-08
3539 2.15592148578025e-08
3540 2.15573994211127e-08
3541 2.15553477289632e-08
3542 2.15532107716854e-08
3543 2.15513296097924e-08
3544 2.15495941091604e-08
3545 2.1547629458496e-08
3546 2.15458868524365e-08
3547 2.15437392370177e-08
3548 2.15415525417484e-08
3549 2.15398170411163e-08
3550 2.15380051571401e-08
3551 2.15358788580033e-08
3552 2.15341078302345e-08
3553 2.1532033045446e-08
3554 2.15300861583501e-08
3555 2.15282032201003e-08
3556 2.15262527802906e-08
3557 2.15242526024895e-08
3558 2.15226751976161e-08
3559 2.15205471221225e-08
3560 2.15186872765116e-08
3561 2.15166569006442e-08
3562 2.15147508697555e-08
3563 2.15130278036213e-08
3564 2.15109796641855e-08
3565 2.15091198185746e-08
3566 2.15072422093954e-08
3567 2.15050910412629e-08
3568 2.15032436301499e-08
3569 2.15012114779256e-08
3570 2.14994226865883e-08
3571 2.14974988921313e-08
3572 2.14955893085289e-08
3573 2.14935571563046e-08
3574 2.14918216556725e-08
3575 2.14900204298374e-08
3576 2.14878852489164e-08
3577 2.1486064483156e-08
3578 2.14842614809641e-08
3579 2.1482337686507e-08
3580 2.14805027098919e-08
3581 2.14784865448792e-08
3582 2.14767155171103e-08
3583 2.1474594547044e-08
3584 2.14726352254502e-08
3585 2.14708926193907e-08
3586 2.14691766586839e-08
3587 2.14670965448249e-08
3588 2.14652136065752e-08
3589 2.14631352690731e-08
3590 2.14612629889643e-08
3591 2.14593942615693e-08
3592 2.14577013935013e-08
3593 2.1455647924995e-08
3594 2.14539372933586e-08
3595 2.14518589558566e-08
3596 2.14499902284615e-08
3597 2.14482476224021e-08
3598 2.14464321857122e-08
3599 2.14444835222594e-08
3600 2.14426538747148e-08
3601 2.14406021825653e-08
3602 2.14384954233537e-08
3603 2.14368611750615e-08
3604 2.14351256744294e-08
3605 2.14332267489681e-08
3606 2.14312692037311e-08
3607 2.1429146457308e-08
3608 2.14273132570497e-08
3609 2.14254178843021e-08
3610 2.14237108053794e-08
3611 2.14216857585825e-08
3612 2.14199182835273e-08
3613 2.14179696200745e-08
3614 2.14161346434594e-08
3615 2.14141557819403e-08
3616 2.14125321917891e-08
3617 2.14105462248426e-08
3618 2.1408661510236e-08
3619 2.14069402204586e-08
3620 2.1404812144965e-08
3621 2.14030304590551e-08
3622 2.14012150223652e-08
3623 2.13994653108784e-08
3624 2.13976036889107e-08
3625 2.13957136452336e-08
3626 2.13937170201461e-08
3627 2.13918038838301e-08
3628 2.13898943002278e-08
3629 2.13879243204929e-08
3630 2.13860325004589e-08
3631 2.13842863416858e-08
3632 2.13823909689381e-08
3633 2.13803978965643e-08
3634 2.13787298974921e-08
3635 2.13768416301718e-08
3636 2.13748325705865e-08
3637 2.13729851594735e-08
3638 2.13712922914056e-08
3639 2.13693667205916e-08
3640 2.13675388494039e-08
3641 2.13655386716027e-08
3642 2.13636646151372e-08
3643 2.13618349675926e-08
3644 2.13599271603471e-08
3645 2.13581632380055e-08
3646 2.13563957629503e-08
3647 2.13545341409827e-08
3648 2.13526174519529e-08
3649 2.13508126734041e-08
3650 2.13489350642249e-08
3651 2.13469810717015e-08
3652 2.13450466191034e-08
3653 2.13431690099242e-08
3654 2.13412256755419e-08
3655 2.13394404369183e-08
3656 2.13373265722794e-08
3657 2.1335605282502e-08
3658 2.13338928745088e-08
3659 2.13319744091223e-08
3660 2.13300666018768e-08
3661 2.13282991268215e-08
3662 2.13264268467128e-08
3663 2.13245758828862e-08
3664 2.13229309764529e-08
3665 2.1321026721921e-08
3666 2.1319113585605e-08
3667 2.13174260466076e-08
3668 2.13155608719262e-08
3669 2.13134541127147e-08
3670 2.13117381520078e-08
3671 2.13100630475083e-08
3672 2.13080220134998e-08
3673 2.1306238551233e-08
3674 2.13044319963274e-08
3675 2.13026787321269e-08
3676 2.13007460558856e-08
3677 2.12987423253708e-08
3678 2.12969570867472e-08
3679 2.12952642186792e-08
3680 2.12931343668288e-08
3681 2.12912691921474e-08
3682 2.12894235573913e-08
3683 2.12873825233828e-08
3684 2.12856203773981e-08
3685 2.12839168511891e-08
3686 2.12822452994033e-08
3687 2.12805204569122e-08
3688 2.12786126496667e-08
3689 2.12768291873999e-08
3690 2.12750048689259e-08
3691 2.12730668636141e-08
3692 2.1271331362982e-08
3693 2.1269373817745e-08
3694 2.12673576527322e-08
3695 2.1265877947485e-08
3696 2.12638120444808e-08
3697 2.12621635853338e-08
3698 2.12603197269345e-08
3699 2.12584048142617e-08
3700 2.12565502977213e-08
3701 2.12549871037027e-08
3702 2.12531894305812e-08
3703 2.12513509012524e-08
3704 2.12493933560154e-08
3705 2.1247606341035e-08
3706 2.12458157733408e-08
3707 2.124392217695e-08
3708 2.12422719414462e-08
3709 2.12403445942755e-08
3710 2.12386481734939e-08
3711 2.12368238550198e-08
3712 2.12350599326783e-08
3713 2.12330260040972e-08
3714 2.12313864267344e-08
3715 2.1229665136957e-08
3716 2.12278408184829e-08
3717 2.12259472220921e-08
3718 2.12242294850284e-08
3719 2.12224637863301e-08
3720 2.12204653848858e-08
3721 2.1218825807523e-08
3722 2.12168878022112e-08
3723 2.12150137457456e-08
3724 2.12131414656369e-08
3725 2.12113455688723e-08
3726 2.12097610585715e-08
3727 2.12079847017321e-08
3728 2.12061959103949e-08
3729 2.12042330360873e-08
3730 2.12024247048248e-08
3731 2.12005382138614e-08
3732 2.11987600806651e-08
3733 2.11971382668708e-08
3734 2.11951594053517e-08
3735 2.11931219240569e-08
3736 2.11915214265446e-08
3737 2.11896882262863e-08
3738 2.11878941058785e-08
3739 2.11860218257698e-08
3740 2.11844195519006e-08
3741 2.11825774698582e-08
3742 2.11806288064054e-08
3743 2.11791313375898e-08
3744 2.11772022140622e-08
3745 2.11752659851072e-08
3746 2.11736708166654e-08
3747 2.11717861020588e-08
3748 2.11698321095355e-08
3749 2.11681605577496e-08
3750 2.11663238047777e-08
3751 2.11646025150003e-08
3752 2.11628758961524e-08
3753 2.11610160505415e-08
3754 2.11592237064906e-08
3755 2.11573709663071e-08
3756 2.11557935614337e-08
3757 2.1153832463483e-08
3758 2.11521680171245e-08
3759 2.11504609382018e-08
3760 2.11487520829223e-08
3761 2.11465902566488e-08
3762 2.11449666664976e-08
3763 2.11433999197652e-08
3764 2.11416057993574e-08
3765 2.11397672700286e-08
3766 2.11379447279114e-08
3767 2.11361310675784e-08
3768 2.11344008960168e-08
3769 2.11324007182156e-08
3770 2.11308979203295e-08
3771 2.11289954421545e-08
3772 2.1127419813638e-08
3773 2.11256470095122e-08
3774 2.11238440073203e-08
3775 2.1122021465203e-08
3776 2.11203090572099e-08
3777 2.11186446108513e-08
3778 2.11168078578794e-08
3779 2.11150510409652e-08
3780 2.11134061345319e-08
3781 2.11115889214852e-08
3782 2.1109556769261e-08
3783 2.11080024570265e-08
3784 2.11060200427937e-08
3785 2.11045030340529e-08
3786 2.11024655527581e-08
3787 2.11008135408974e-08
3788 2.1099181068962e-08
3789 2.10974597791846e-08
3790 2.10955448665118e-08
3791 2.1093860880228e-08
3792 2.10920081400445e-08
3793 2.10902815211966e-08
3794 2.10885460205645e-08
3795 2.10867874272935e-08
3796 2.10851016646529e-08
3797 2.10831494484864e-08
3798 2.10814867784848e-08
3799 2.10797388433548e-08
3800 2.10779678155859e-08
3801 2.10763761998578e-08
3802 2.10744435236165e-08
3803 2.10727542082623e-08
3804 2.10709369952156e-08
3805 2.10691855073719e-08
3806 2.10675441536523e-08
3807 2.10655439758511e-08
3808 2.10638990694179e-08
3809 2.10619752749608e-08
3810 2.10604493844357e-08
3811 2.10585273663355e-08
3812 2.10569321978937e-08
3813 2.10551114321333e-08
3814 2.10534327749201e-08
3815 2.10516102328029e-08
3816 2.10498356523203e-08
3817 2.10479900175642e-08
3818 2.10462491878616e-08
3819 2.10447552717596e-08
3820 2.10429380587129e-08
3821 2.1041211439865e-08
3822 2.10393391597563e-08
3823 2.10377475440282e-08
3824 2.10357296026586e-08
3825 2.10341024597938e-08
3826 2.10323864990869e-08
3827 2.10306669856664e-08
3828 2.10289243796069e-08
3829 2.10273238820946e-08
3830 2.10255652888236e-08
3831 2.10238439990462e-08
3832 2.10221262619825e-08
3833 2.10203534578568e-08
3834 2.10185540083785e-08
3835 2.10168042968917e-08
3836 2.10150599144754e-08
3837 2.10136015255102e-08
3838 2.10117967469614e-08
3839 2.10101855913081e-08
3840 2.10084234453234e-08
3841 2.10065760342104e-08
3842 2.10048192172962e-08
3843 2.10033164194101e-08
3844 2.10012895962564e-08
3845 2.09998507472164e-08
3846 2.09980655085928e-08
3847 2.09962323083346e-08
3848 2.0994447069711e-08
3849 2.09928217032029e-08
3850 2.09909600812352e-08
3851 2.09891783953253e-08
3852 2.09877022427918e-08
3853 2.09858512789651e-08
3854 2.09842170306729e-08
3855 2.09823678432031e-08
3856 2.09804973394512e-08
3857 2.09790016469924e-08
3858 2.09773443060612e-08
3859 2.09753832081105e-08
3860 2.09738697520834e-08
3861 2.09723776123383e-08
3862 2.09703490128277e-08
3863 2.0968686342826e-08
3864 2.09671480178031e-08
3865 2.09652775140512e-08
3866 2.09634460901498e-08
3867 2.09619237523384e-08
3868 2.09600798939391e-08
3869 2.09582644572492e-08
3870 2.09566106690318e-08
3871 2.09550403695857e-08
3872 2.09532764472442e-08
3873 2.09516617388772e-08
3874 2.09498747238968e-08
3875 2.09483843605085e-08
3876 2.09464836586903e-08
3877 2.09446344712205e-08
3878 2.09431654241143e-08
3879 2.09411759044542e-08
3880 2.09396731065681e-08
3881 2.09380779381263e-08
3882 2.09363495429216e-08
3883 2.09345181190201e-08
3884 2.09328447908774e-08
3885 2.093115725188e-08
3886 2.09295780706498e-08
3887 2.09277430940347e-08
3888 2.09260235806141e-08
3889 2.09245243354417e-08
3890 2.09226804770424e-08
3891 2.09209289891987e-08
3892 2.09192023703508e-08
3893 2.09174260135114e-08
3894 2.09155910368963e-08
3895 2.09140971207944e-08
3896 2.09124824124274e-08
3897 2.09107238191564e-08
3898 2.0908958120458e-08
3899 2.0907387821012e-08
3900 2.09055670552516e-08
3901 2.09038635290426e-08
3902 2.09023784947249e-08
3903 2.09006980611548e-08
3904 2.08990353911531e-08
3905 2.08975148296986e-08
3906 2.08955572844616e-08
3907 2.08939141543851e-08
3908 2.08919583855049e-08
3909 2.08904520349051e-08
3910 2.08886703489952e-08
3911 2.08868957685127e-08
3912 2.08853307981371e-08
3913 2.08838404347489e-08
3914 2.08820090108475e-08
3915 2.08802823919996e-08
3916 2.0878596629359e-08
3917 2.08770050136309e-08
3918 2.08752322095052e-08
3919 2.08736548046318e-08
3920 2.0871782524523e-08
3921 2.08700523529615e-08
3922 2.08685442260048e-08
3923 2.08667856327338e-08
3924 2.08651194100185e-08
3925 2.08632275899845e-08
3926 2.08616128816175e-08
3927 2.08600567930262e-08
3928 2.08583621486014e-08
3929 2.08564685522106e-08
3930 2.08548325275615e-08
3931 2.08532142664808e-08
3932 2.08515213984128e-08
3933 2.08498729392659e-08
3934 2.08482973107493e-08
3935 2.08464143724996e-08
3936 2.08449542071776e-08
3937 2.08432080484044e-08
3938 2.0841554260187e-08
3939 2.08400106060935e-08
3940 2.08380388500018e-08
3941 2.08365538156841e-08
3942 2.08346833119322e-08
3943 2.0833159197764e-08
3944 2.08316048855295e-08
3945 2.08297716852712e-08
3946 2.08283417180155e-08
3947 2.08266168755245e-08
3948 2.08247925570504e-08
3949 2.08232524556706e-08
3950 2.08216377473036e-08
3951 2.0820007051725e-08
3952 2.0818282209234e-08
3953 2.08166586190828e-08
3954 2.08149710800853e-08
3955 2.08130863654787e-08
3956 2.08115000788212e-08
3957 2.08098906995247e-08
3958 2.08082795438713e-08
3959 2.08063468676301e-08
3960 2.08047445937609e-08
3961 2.08031973869538e-08
3962 2.080151340067e-08
3963 2.07997743473243e-08
3964 2.07982786548655e-08
3965 2.07968007259751e-08
3966 2.07951469377576e-08
3967 2.0793295973931e-08
3968 2.07916155403609e-08
3969 2.07900807680517e-08
3970 2.07882653313618e-08
3971 2.07867429935504e-08
3972 2.07851975631002e-08
3973 2.07833377174893e-08
3974 2.07817016928402e-08
3975 2.07798613871546e-08
3976 2.0778363918339e-08
3977 2.07769534910085e-08
3978 2.07751469361028e-08
3979 2.07734558443917e-08
3980 2.07716954747639e-08
3981 2.0770123398961e-08
3982 2.07684234254657e-08
3983 2.07669685892142e-08
3984 2.0765135388956e-08
3985 2.07637391724802e-08
3986 2.07617674163885e-08
3987 2.0760278829357e-08
3988 2.07586516864922e-08
3989 2.07570955979008e-08
3990 2.07551060782407e-08
3991 2.07536050567114e-08
3992 2.07519867956307e-08
3993 2.07503223492722e-08
3994 2.07485619796444e-08
3995 2.07470076674099e-08
3996 2.07454586842459e-08
3997 2.07435277843615e-08
3998 2.07420871589648e-08
3999 2.07403676455442e-08
4000 2.07388719530854e-08
4001 2.07372075067269e-08
4002 2.0735614114642e-08
4003 2.07338413105163e-08
4004 2.07323385126301e-08
4005 2.07307877531093e-08
4006 2.07292547571569e-08
4007 2.0727672023213e-08
4008 2.07257873086064e-08
4009 2.07241015459658e-08
4010 2.07223216364127e-08
4011 2.07208046276719e-08
4012 2.07191828138775e-08
4013 2.07173318500509e-08
4014 2.07158858955836e-08
4015 2.07143209252081e-08
4016 2.07128696416703e-08
4017 2.07109405181427e-08
4018 2.07091375159507e-08
4019 2.07076311653509e-08
4020 2.0705849479441e-08
4021 2.07043537869822e-08
4022 2.07026058518522e-08
4023 2.07009662744895e-08
4024 2.06994723583875e-08
4025 2.06976515926272e-08
4026 2.06960955040358e-08
4027 2.06946175751455e-08
4028 2.06930153012763e-08
4029 2.06914219091914e-08
4030 2.06896917376298e-08
4031 2.06879775532798e-08
4032 2.06864978480326e-08
4033 2.06849488648686e-08
4034 2.06833643545679e-08
4035 2.0681703460923e-08
4036 2.06800070401414e-08
4037 2.06782448941567e-08
4038 2.06767563071253e-08
4039 2.06751842313224e-08
4040 2.06734735996861e-08
4041 2.06719867890115e-08
4042 2.06702956973004e-08
4043 2.06685708548093e-08
4044 2.06670236480022e-08
4045 2.06654764411951e-08
4046 2.0663874167326e-08
4047 2.06622221554653e-08
4048 2.06606856067992e-08
4049 2.06589128026735e-08
4050 2.06571701966141e-08
4051 2.06557349002878e-08
4052 2.0654118415564e-08
4053 2.0652414889355e-08
4054 2.06507824174196e-08
4055 2.06492192234009e-08
4056 2.06477004383032e-08
4057 2.06460537555131e-08
4058 2.06443111494536e-08
4059 2.06427763771444e-08
4060 2.06412895664698e-08
4061 2.06397050561691e-08
4062 2.06380619260926e-08
4063 2.06364223487299e-08
4064 2.06349533016237e-08
4065 2.06334469510239e-08
4066 2.06314858530732e-08
4067 2.06298267357852e-08
4068 2.0628556640645e-08
4069 2.06268691016476e-08
4070 2.06253112366994e-08
4071 2.0623678764764e-08
4072 2.06219841203392e-08
4073 2.06204102681795e-08
4074 2.06186676621201e-08
4075 2.06172092731549e-08
4076 2.06157082516256e-08
4077 2.06138519587284e-08
4078 2.06122461321456e-08
4079 2.06105745803598e-08
4080 2.06089652010633e-08
4081 2.06075725373012e-08
4082 2.06059951324278e-08
4083 2.06044425965501e-08
4084 2.06028136773284e-08
4085 2.06012451542392e-08
4086 2.05994403756904e-08
4087 2.05980867917788e-08
4088 2.05961896426743e-08
4089 2.05948378351195e-08
4090 2.05932977337397e-08
4091 2.05914894024772e-08
4092 2.05900168026574e-08
4093 2.05883701198672e-08
4094 2.05866399483057e-08
4095 2.05851407031332e-08
4096 2.05834993494136e-08
4097 2.05819841170296e-08
4098 2.05802486163975e-08
4099 2.05788808216312e-08
4100 2.0577278547762e-08
4101 2.05755608106983e-08
4102 2.05738412972778e-08
4103 2.0572379355599e-08
4104 2.05707557654478e-08
4105 2.05693169164078e-08
4106 2.05677519460323e-08
4107 2.0566140790379e-08
4108 2.05644745676636e-08
4109 2.05630072969143e-08
4110 2.05613162052032e-08
4111 2.05596801805541e-08
4112 2.05582111334479e-08
4113 2.05565076072389e-08
4114 2.05549266496519e-08
4115 2.05533581265627e-08
4116 2.05517274309841e-08
4117 2.05500434447003e-08
4118 2.05486347937267e-08
4119 2.05468122516095e-08
4120 2.0545346757217e-08
4121 2.05437622469162e-08
4122 2.05421848420428e-08
4123 2.05405363828959e-08
4124 2.05391224028517e-08
4125 2.05372341355314e-08
4126 2.05358485771967e-08
4127 2.05345340731355e-08
4128 2.05327346236572e-08
4129 2.05311376788586e-08
4130 2.05295762611968e-08
4131 2.05279313547635e-08
4132 2.0526476518512e-08
4133 2.05246095674738e-08
4134 2.05231849292886e-08
4135 2.05213854798103e-08
4136 2.05197991931527e-08
4137 2.05183816603949e-08
4138 2.05170103129149e-08
4139 2.05154595533941e-08
4140 2.05139283337985e-08
4141 2.05122514529421e-08
4142 2.05106722717119e-08
4143 2.05090433524902e-08
4144 2.05074233150526e-08
4145 2.050602532222e-08
4146 2.05044532464171e-08
4147 2.05028918287553e-08
4148 2.05011758680484e-08
4149 2.04997370190085e-08
4150 2.04981418505668e-08
4151 2.04964987204903e-08
4152 2.04950136861726e-08
4153 2.04934895720044e-08
4154 2.0492022301255e-08
4155 2.04903347622576e-08
4156 2.04887609100979e-08
4157 2.04871462017309e-08
4158 2.04855599150733e-08
4159 2.04840624462577e-08
4160 2.04825525429442e-08
4161 2.0480840134951e-08
4162 2.04794812219689e-08
4163 2.04778185519672e-08
4164 2.04762180544549e-08
4165 2.04746761767183e-08
4166 2.04730188357871e-08
4167 2.04713366258602e-08
4168 2.04701127159979e-08
4169 2.04684624804941e-08
4170 2.04669010628322e-08
4171 2.04651211532791e-08
4172 2.04639079015578e-08
4173 2.04619272636819e-08
4174 2.04604244657958e-08
4175 2.0459138383444e-08
4176 2.0457568083998e-08
4177 2.04558716632164e-08
4178 2.04542303094968e-08
4179 2.04529690961408e-08
4180 2.04514982726778e-08
4181 2.04497307976226e-08
4182 2.04481906962428e-08
4183 2.04467571762734e-08
4184 2.0445158455118e-08
4185 2.04435917083856e-08
4186 2.0441984105446e-08
4187 2.04403658443653e-08
4188 2.04388825864044e-08
4189 2.04370689260713e-08
4190 2.04357206712302e-08
4191 2.0434152148141e-08
4192 2.04324095420816e-08
4193 2.04308623352745e-08
4194 2.042961710913e-08
4195 2.04278673976432e-08
4196 2.04262953218404e-08
4197 2.04248138402363e-08
4198 2.04233199241344e-08
4199 2.0421808244464e-08
4200 2.04203427500715e-08
4201 2.04185575114479e-08
4202 2.0417108004267e-08
4203 2.041545954512e-08
4204 2.04139052328856e-08
4205 2.04126351377454e-08
4206 2.04110151003078e-08
4207 2.04092636124642e-08
4208 2.04077643672917e-08
4209 2.04063486108907e-08
4210 2.04048156149383e-08
4211 2.04031298522978e-08
4212 2.04016501470505e-08
4213 2.04001224801686e-08
4214 2.0398481126449e-08
4215 2.03969463541398e-08
4216 2.03953973709758e-08
4217 2.03939460874381e-08
4218 2.03924432895519e-08
4219 2.03907415396998e-08
4220 2.03893328887261e-08
4221 2.03877661419938e-08
4222 2.03861088010626e-08
4223 2.03846575175248e-08
4224 2.03831174161451e-08
4225 2.03814245480771e-08
4226 2.03800905040907e-08
4227 2.03784704666532e-08
4228 2.03770067486175e-08
4229 2.03753529604001e-08
4230 2.03740739834757e-08
4231 2.03724503933245e-08
4232 2.03709547008657e-08
4233 2.03694483502659e-08
4234 2.03678727217493e-08
4235 2.03663184095149e-08
4236 2.03647338992141e-08
4237 2.03633359063815e-08
4238 2.03618384375659e-08
4239 2.03603462978208e-08
4240 2.03586552061097e-08
4241 2.03571826062898e-08
4242 2.0355720664611e-08
4243 2.03542001031565e-08
4244 2.03525516440095e-08
4245 2.03509245011446e-08
4246 2.03496561823613e-08
4247 2.03479508797955e-08
4248 2.03465884140996e-08
4249 2.03448458080402e-08
4250 2.03433501155814e-08
4251 2.03418579758363e-08
4252 2.03403320853113e-08
4253 2.03387795494336e-08
4254 2.03370742468678e-08
4255 2.03358911932128e-08
4256 2.0334207206929e-08
4257 2.03326155912009e-08
4258 2.0331231809223e-08
4259 2.03297183531959e-08
4260 2.03281373956088e-08
4261 2.03267731535561e-08
4262 2.0325225946749e-08
4263 2.03236183438094e-08
4264 2.03220835715001e-08
4265 2.03206607096718e-08
4266 2.03191614644993e-08
4267 2.03177386026709e-08
4268 2.03163814660456e-08
4269 2.03148093902428e-08
4270 2.03132746179335e-08
4271 2.03116066188613e-08
4272 2.03100771756226e-08
4273 2.03087608952046e-08
4274 2.03071817139744e-08
4275 2.03057020087272e-08
4276 2.03041512492064e-08
4277 2.03027372691622e-08
4278 2.03010923627289e-08
4279 2.02998133858046e-08
4280 2.02981045305251e-08
4281 2.02966390361325e-08
4282 2.02951024874665e-08
4283 2.02935908077961e-08
4284 2.02920507064164e-08
4285 2.02906331736585e-08
4286 2.0289123270345e-08
4287 2.02875281019033e-08
4288 2.02859720133119e-08
4289 2.0284522506131e-08
4290 2.02830925388753e-08
4291 2.02818242200919e-08
4292 2.02799270709875e-08
4293 2.02784917746612e-08
4294 2.02770760182602e-08
4295 2.02757117762076e-08
4296 2.02740189081396e-08
4297 2.02727292730742e-08
4298 2.02712211461176e-08
4299 2.0269778744364e-08
4300 2.0267975742172e-08
4301 2.02668388737948e-08
4302 2.02651246894447e-08
4303 2.02639078850098e-08
4304 2.02623713363437e-08
4305 2.02608312349639e-08
4306 2.02594065967787e-08
4307 2.02574046426207e-08
4308 2.02560208606428e-08
4309 2.025479162171e-08
4310 2.02531520443472e-08
4311 2.02516368119632e-08
4312 2.02501269086497e-08
4313 2.02486791778256e-08
4314 2.02472048016489e-08
4315 2.02456380549165e-08
4316 2.02441778895945e-08
4317 2.02428491746787e-08
4318 2.02414067729251e-08
4319 2.02395291637458e-08
4320 2.0238255515892e-08
4321 2.02365750823219e-08
4322 2.02350545208674e-08
4323 2.02335996846159e-08
4324 2.02323544584715e-08
4325 2.02306740249014e-08
4326 2.0229274255712e-08
4327 2.0227902908232e-08
4328 2.0226398333989e-08
4329 2.02247818492651e-08
4330 2.02233660928641e-08
4331 2.0221763818995e-08
4332 2.02201544396985e-08
4333 2.02188790154878e-08
4334 2.02174863517257e-08
4335 2.02158538797903e-08
4336 2.0214486085024e-08
4337 2.02130738813366e-08
4338 2.02115053582475e-08
4339 2.02104146751481e-08
4340 2.02086241074539e-08
4341 2.02070538080079e-08
4342 2.02057091058805e-08
4343 2.02042400587743e-08
4344 2.0202728379104e-08
4345 2.020121314672e-08
4346 2.01998062721032e-08
4347 2.01982803815781e-08
4348 2.01968894941729e-08
4349 2.0195527028477e-08
4350 2.01938750166164e-08
4351 2.01922443210378e-08
4352 2.01911269925859e-08
4353 2.0189554916783e-08
4354 2.0187895799495e-08
4355 2.01865528737244e-08
4356 2.01850003378468e-08
4357 2.0183470894608e-08
4358 2.01820462564228e-08
4359 2.01805026023294e-08
4360 2.01790122389411e-08
4361 2.01775520736192e-08
4362 2.01760528284467e-08
4363 2.01747099026761e-08
4364 2.01731307214459e-08
4365 2.01716652270534e-08
4366 2.01701464419557e-08
4367 2.01688550305335e-08
4368 2.01670804500509e-08
4369 2.01658583165454e-08
4370 2.01644212438623e-08
4371 2.01628527207731e-08
4372 2.01613250538912e-08
4373 2.01598311377893e-08
4374 2.01584793302345e-08
4375 2.01567917912371e-08
4376 2.01554914980306e-08
4377 2.01539887001445e-08
4378 2.0152578272814e-08
4379 2.01512122544045e-08
4380 2.01495993223944e-08
4381 2.01480805372967e-08
4382 2.01465937266221e-08
4383 2.01452490244947e-08
4384 2.01437568847496e-08
4385 2.01423002721413e-08
4386 2.01407690525457e-08
4387 2.01391756604608e-08
4388 2.01378149711218e-08
4389 2.01363672402977e-08
4390 2.01348324679884e-08
4391 2.01334842131473e-08
4392 2.01320098369706e-08
4393 2.01305070390845e-08
4394 2.01291605606002e-08
4395 2.01276506572867e-08
4396 2.01262171373173e-08
4397 2.01247392084269e-08
4398 2.01230996310642e-08
4399 2.0121818877783e-08
4400 2.01203569361041e-08
4401 2.01189269688484e-08
4402 2.0117372656614e-08
4403 2.01160581525528e-08
4404 2.01143759426259e-08
4405 2.01129015664492e-08
4406 2.01115462061807e-08
4407 2.01099084051748e-08
4408 2.01085494921927e-08
4409 2.01070289307381e-08
4410 2.01055883053414e-08
4411 2.01041423508741e-08
4412 2.01026999491205e-08
4413 2.01011438605292e-08
4414 2.0099774289406e-08
4415 2.00981649101095e-08
4416 2.0096853958762e-08
4417 2.00955359019872e-08
4418 2.00939869188232e-08
4419 2.00922940507553e-08
4420 2.00911056680297e-08
4421 2.00894572088828e-08
4422 2.00879206602167e-08
4423 2.0086600827085e-08
4424 2.00852898757375e-08
4425 2.00835508223918e-08
4426 2.00822487528285e-08
4427 2.00808187855728e-08
4428 2.00794438853791e-08
4429 2.00778753622899e-08
4430 2.00764773694573e-08
4431 2.00750100987079e-08
4432 2.00736565147963e-08
4433 2.00722745091753e-08
4434 2.00707592767913e-08
4435 2.00693559548881e-08
4436 2.0067709272098e-08
4437 2.00664747040946e-08
4438 2.0065110462042e-08
4439 2.00636591785042e-08
4440 2.0062056904635e-08
4441 2.00607175315781e-08
4442 2.00593905930191e-08
4443 2.00579020059877e-08
4444 2.00562269014881e-08
4445 2.00550704931857e-08
4446 2.00536849348509e-08
4447 2.00521483861849e-08
4448 2.00507557224228e-08
4449 2.00490930524211e-08
4450 2.00476222289581e-08
4451 2.00462704214033e-08
4452 2.00446113041153e-08
4453 2.00433571961867e-08
4454 2.00419307816446e-08
4455 2.00405612105214e-08
4456 2.00390655180627e-08
4457 2.00376426562343e-08
4458 2.0035976433519e-08
4459 2.00347045620219e-08
4460 2.00331964350653e-08
4461 2.0031750480598e-08
4462 2.00305354525199e-08
4463 2.00290219964927e-08
4464 2.00274854478266e-08
4465 2.0026186930977e-08
4466 2.00247889381444e-08
4467 2.00231333735701e-08
4468 2.00217868950858e-08
4469 2.00203178479796e-08
4470 2.00188914334376e-08
4471 2.00174135045472e-08
4472 2.00158698504538e-08
4473 2.00146423878778e-08
4474 2.00131058392117e-08
4475 2.00116563320307e-08
4476 2.00103524861106e-08
4477 2.00088372537266e-08
4478 2.00073895229025e-08
4479 2.00058369870248e-08
4480 2.00045509046731e-08
4481 2.00034033781549e-08
4482 2.00016039286766e-08
4483 2.00005114692203e-08
4484 1.99988328120071e-08
4485 1.99975005443775e-08
4486 1.99957206348245e-08
4487 1.99942906675687e-08
4488 1.99931324829095e-08
4489 1.99919174548313e-08
4490 1.99901375452782e-08
4491 1.99886880380973e-08
4492 1.99870715533734e-08
4493 1.998583698537e-08
4494 1.99843768200481e-08
4495 1.99829450764355e-08
4496 1.99815808343828e-08
4497 1.99802538958238e-08
4498 1.99786018839632e-08
4499 1.9977287379902e-08
4500 1.99757543839496e-08
4501 1.99744807360958e-08
4502 1.99729548455707e-08
4503 1.99713667825563e-08
4504 1.99701997161128e-08
4505 1.99686649438036e-08
4506 1.99671585932037e-08
4507 1.99657748112259e-08
4508 1.99644318854553e-08
4509 1.99628846786482e-08
4510 1.99616874141384e-08
4511 1.99601313255471e-08
4512 1.99588097160586e-08
4513 1.99574348158649e-08
4514 1.99557970148589e-08
4515 1.99543652712464e-08
4516 1.99528713551445e-08
4517 1.99516829724189e-08
4518 1.99500469477698e-08
4519 1.99488265906211e-08
4520 1.9947423268718e-08
4521 1.99459293526161e-08
4522 1.99444212256594e-08
4523 1.99429024405617e-08
4524 1.99416341217784e-08
4525 1.99400247424819e-08
4526 1.9938740436487e-08
4527 1.99373886289322e-08
4528 1.99360137287385e-08
4529 1.99346459339722e-08
4530 1.9933033001962e-08
4531 1.99317007343325e-08
4532 1.99302885306452e-08
4533 1.99291605440521e-08
4534 1.99274055034948e-08
4535 1.99260945521473e-08
4536 1.9924637939539e-08
4537 1.99231333652961e-08
4538 1.99219787333504e-08
4539 1.99203711304108e-08
4540 1.99189926775034e-08
4541 1.99176160009529e-08
4542 1.9916242877116e-08
4543 1.99148324497855e-08
4544 1.99132657030532e-08
4545 1.99119210009258e-08
4546 1.99104110976123e-08
4547 1.99092351493846e-08
4548 1.99076612972249e-08
4549 1.99065581796276e-08
4550 1.99050411708868e-08
4551 1.99035437020711e-08
4552 1.99021865654458e-08
4553 1.99007761381154e-08
4554 1.98994172251332e-08
4555 1.98978682419693e-08
4556 1.98965839359744e-08
4557 1.98951095597977e-08
4558 1.9893645841762e-08
4559 1.98921874527969e-08
4560 1.98906953130518e-08
4561 1.98894696268326e-08
4562 1.98880840684978e-08
4563 1.98866061396075e-08
4564 1.98850322874478e-08
4565 1.98839096299253e-08
4566 1.98825045316653e-08
4567 1.98811527241105e-08
4568 1.98795149231046e-08
4569 1.98783034477401e-08
4570 1.98768610459865e-08
4571 1.98756353597673e-08
4572 1.98740224277572e-08
4573 1.98727487799033e-08
4574 1.98712974963655e-08
4575 1.98699687814496e-08
4576 1.9868371836651e-08
4577 1.98671052942245e-08
4578 1.98654070970861e-08
4579 1.98640730530997e-08
4580 1.9862763878109e-08
4581 1.98613321344965e-08
4582 1.98600602629995e-08
4583 1.98586178612459e-08
4584 1.98571328269281e-08
4585 1.98557952302281e-08
4586 1.98544185536775e-08
4587 1.98531910911015e-08
4588 1.9851670529647e-08
4589 1.9850409316291e-08
4590 1.98487430935756e-08
4591 1.98473966150914e-08
4592 1.98459950695451e-08
4593 1.98445668786462e-08
4594 1.98431138187516e-08
4595 1.98416749697117e-08
4596 1.98404723761314e-08
4597 1.98389606964611e-08
4598 1.98376834958935e-08
4599 1.98361771452937e-08
4600 1.98348608648757e-08
4601 1.98335143863915e-08
4602 1.98320293520737e-08
4603 1.98306366883116e-08
4604 1.9829416331163e-08
4605 1.98279597185547e-08
4606 1.98265457385105e-08
4607 1.98251104421843e-08
4608 1.98238545578988e-08
4609 1.98224316960705e-08
4610 1.98207974477782e-08
4611 1.9819571761559e-08
4612 1.98180849508844e-08
4613 1.98167935394622e-08
4614 1.98153298214265e-08
4615 1.98140241991496e-08
4616 1.98127949602167e-08
4617 1.98112299898412e-08
4618 1.98098746295727e-08
4619 1.98084322278191e-08
4620 1.98070306822729e-08
4621 1.98056380185108e-08
4622 1.98044389776442e-08
4623 1.98030924991599e-08
4624 1.98016465446926e-08
4625 1.98003764495525e-08
4626 1.97989926675746e-08
4627 1.97975040805431e-08
4628 1.97961078640674e-08
4629 1.97947596092263e-08
4630 1.9793468197804e-08
4631 1.97919458599927e-08
4632 1.97905052345959e-08
4633 1.97891196762612e-08
4634 1.97879348462493e-08
4635 1.97864800099978e-08
4636 1.97852294547829e-08
4637 1.97837621840335e-08
4638 1.97823784020557e-08
4639 1.97810159363598e-08
4640 1.97793710299266e-08
4641 1.97780565258654e-08
4642 1.97768983412061e-08
4643 1.97755465336513e-08
4644 1.97741467644619e-08
4645 1.97727132444925e-08
4646 1.97712655136684e-08
4647 1.97699225878978e-08
4648 1.97685903202682e-08
4649 1.97670591006727e-08
4650 1.97656628841969e-08
4651 1.97641512045266e-08
4652 1.9763017888863e-08
4653 1.97616785158061e-08
4654 1.97601988105589e-08
4655 1.97589358208461e-08
4656 1.97574809845946e-08
4657 1.97561149661851e-08
4658 1.97548732927544e-08
4659 1.97534859580628e-08
4660 1.97518961186915e-08
4661 1.97505229948547e-08
4662 1.9748993551616e-08
4663 1.97478673413798e-08
4664 1.97464000706304e-08
4665 1.97450589212167e-08
4666 1.97436502702431e-08
4667 1.97425453762889e-08
4668 1.97408347446526e-08
4669 1.97396072820766e-08
4670 1.97383833722142e-08
4671 1.97370262355889e-08
4672 1.97355856101922e-08
4673 1.97341183394428e-08
4674 1.97327985063112e-08
4675 1.9731594136374e-08
4676 1.97299954152186e-08
4677 1.97286773584437e-08
4678 1.97274285795856e-08
4679 1.97261513790181e-08
4680 1.9724515354369e-08
4681 1.97231422305322e-08
4682 1.97217051578491e-08
4683 1.97203799956469e-08
4684 1.97191809547803e-08
4685 1.9717768751093e-08
4686 1.97162322024269e-08
4687 1.97148946057268e-08
4688 1.97138003699138e-08
4689 1.97122869138866e-08
4690 1.97109937261075e-08
4691 1.97096703402622e-08
4692 1.97083327435621e-08
4693 1.97069240925885e-08
4694 1.97056060358136e-08
4695 1.97042382410473e-08
4696 1.97026803760991e-08
4697 1.97015133096556e-08
4698 1.97000957768978e-08
4699 1.96985041611697e-08
4700 1.96972482768842e-08
4701 1.9695956865462e-08
4702 1.96945588726294e-08
4703 1.96932035123609e-08
4704 1.96917007144748e-08
4705 1.96902441018665e-08
4706 1.96890752590662e-08
4707 1.96877678604324e-08
4708 1.96865013180059e-08
4709 1.96849949674061e-08
4710 1.96836005272871e-08
4711 1.96824228027026e-08
4712 1.9680870266825e-08
4713 1.96796907658836e-08
4714 1.96782963257647e-08
4715 1.96770422178361e-08
4716 1.96753688896933e-08
4717 1.96740082003544e-08
4718 1.96726901435795e-08
4719 1.96711926747639e-08
4720 1.96698355381386e-08
4721 1.96685938647079e-08
4722 1.96671834373774e-08
4723 1.96657357065533e-08
4724 1.96646841033044e-08
4725 1.96632932158991e-08
4726 1.96617797598719e-08
4727 1.96605256519433e-08
4728 1.96591720680317e-08
4729 1.96577030209255e-08
4730 1.96564844401337e-08
4731 1.96550491438074e-08
4732 1.96537079943937e-08
4733 1.96523082252043e-08
4734 1.96510221428525e-08
4735 1.96497271787166e-08
4736 1.9648329185884e-08
4737 1.964712303959e-08
4738 1.96455669509987e-08
4739 1.96442382360829e-08
4740 1.96431724219792e-08
4741 1.96417939690718e-08
4742 1.96402467622647e-08
4743 1.96389819961951e-08
4744 1.96374543293132e-08
4745 1.9635985282207e-08
4746 1.96348288739046e-08
4747 1.96334308810719e-08
4748 1.96321163770108e-08
4749 1.96308374000864e-08
4750 1.96293328258434e-08
4751 1.96280467434917e-08
4752 1.96265599328171e-08
4753 1.96252631923244e-08
4754 1.96240570460304e-08
4755 1.96224636539455e-08
4756 1.9621340996423e-08
4757 1.96199749780135e-08
4758 1.96186071832471e-08
4759 1.96172553756924e-08
4760 1.96160776511078e-08
4761 1.96146832109889e-08
4762 1.96130969243313e-08
4763 1.96119955830909e-08
4764 1.96106704208887e-08
4765 1.96091161086542e-08
4766 1.96078620007256e-08
4767 1.96066700652864e-08
4768 1.96052383216738e-08
4769 1.96037959199202e-08
4770 1.96024316778676e-08
4771 1.96012077680052e-08
4772 1.95999643182176e-08
4773 1.95984366513358e-08
4774 1.95972962302449e-08
4775 1.9595779221504e-08
4776 1.95944824810113e-08
4777 1.95931217916723e-08
4778 1.95917788659017e-08
4779 1.95903826494259e-08
4780 1.95891143306426e-08
4781 1.95879543696265e-08
4782 1.95866043384285e-08
4783 1.95852312145917e-08
4784 1.95838545380411e-08
4785 1.95825435866936e-08
4786 1.95812575043419e-08
4787 1.95797600355263e-08
4788 1.95783886880463e-08
4789 1.95770883948398e-08
4790 1.95757863252766e-08
4791 1.95743332653819e-08
4792 1.95732230423573e-08
4793 1.95716296502724e-08
4794 1.95704448202605e-08
4795 1.95691853832614e-08
4796 1.95678655501297e-08
4797 1.9566336106891e-08
4798 1.95649967338341e-08
4799 1.95637515076896e-08
4800 1.95626785881586e-08
4801 1.95612983588944e-08
4802 1.95597937846514e-08
4803 1.95585290185818e-08
4804 1.95572606997985e-08
4805 1.95558556015385e-08
4806 1.95545535319752e-08
4807 1.95532852131919e-08
4808 1.95518818912888e-08
4809 1.95505531763729e-08
4810 1.95492262378139e-08
4811 1.95479046283253e-08
4812 1.95465528207706e-08
4813 1.95451690387927e-08
4814 1.95439611161419e-08
4815 1.9542488516322e-08
4816 1.95414813219941e-08
4817 1.95398985880502e-08
4818 1.95385716494911e-08
4819 1.95370830624597e-08
4820 1.95357685583986e-08
4821 1.9534525108611e-08
4822 1.9533308304176e-08
4823 1.95317824136509e-08
4824 1.95304910022287e-08
4825 1.9529160510956e-08
4826 1.95279383774505e-08
4827 1.95264266977802e-08
4828 1.95253413437513e-08
4829 1.95240641431838e-08
4830 1.95226199650733e-08
4831 1.95211544706808e-08
4832 1.95198257557649e-08
4833 1.95187617180181e-08
4834 1.95172464856341e-08
4835 1.95160545501949e-08
4836 1.95146334647234e-08
4837 1.95131502067625e-08
4838 1.95118534662697e-08
4839 1.95108196265892e-08
4840 1.95091764965127e-08
4841 1.95081106824091e-08
4842 1.95067872965637e-08
4843 1.9505401738229e-08
4844 1.95040517070311e-08
4845 1.95026377269869e-08
4846 1.95011704562376e-08
4847 1.95000069425078e-08
4848 1.94986000678909e-08
4849 1.94975147138621e-08
4850 1.94961131683158e-08
4851 1.94947276099811e-08
4852 1.94932976427253e-08
4853 1.94919138607474e-08
4854 1.9490613567541e-08
4855 1.9489386104965e-08
4856 1.94879330450703e-08
4857 1.94868494673983e-08
4858 1.94854710144909e-08
4859 1.94840676925878e-08
4860 1.94828650990075e-08
4861 1.94815630294443e-08
4862 1.94802325381715e-08
4863 1.94788860596873e-08
4864 1.94776372808292e-08
4865 1.94763440930501e-08
4866 1.94749034676533e-08
4867 1.94736671232931e-08
4868 1.9472562229339e-08
4869 1.94711198275854e-08
4870 1.94701836875311e-08
4871 1.94688709598267e-08
4872 1.94671567754767e-08
4873 1.94660518815226e-08
4874 1.94646574414037e-08
4875 1.94633056338489e-08
4876 1.94621740945422e-08
4877 1.94608276160579e-08
4878 1.94593710034496e-08
4879 1.94581861734378e-08
4880 1.94570404232763e-08
4881 1.94555660470996e-08
4882 1.94544487186477e-08
4883 1.94529619079731e-08
4884 1.94517735252475e-08
4885 1.94503684269876e-08
4886 1.94492599803198e-08
4887 1.94479117254787e-08
4888 1.94466487357658e-08
4889 1.94449807366937e-08
4890 1.94439753187226e-08
4891 1.94427460797897e-08
4892 1.94412663745425e-08
4893 1.94401472697336e-08
4894 1.94386462482043e-08
4895 1.94374294437694e-08
4896 1.94361646776997e-08
4897 1.94348803717048e-08
4898 1.9433672449054e-08
4899 1.94320435298323e-08
4900 1.94310132428654e-08
4901 1.94296472244559e-08
4902 1.94283096277559e-08
4903 1.942701466362e-08
4904 1.94257392394093e-08
4905 1.94245934892479e-08
4906 1.94230835859344e-08
4907 1.9421694474886e-08
4908 1.94204385906005e-08
4909 1.9419083230332e-08
4910 1.94180707069336e-08
4911 1.94166513978189e-08
4912 1.94154772259481e-08
4913 1.94140739040449e-08
4914 1.94127274255607e-08
4915 1.94115532536898e-08
4916 1.94103257911138e-08
4917 1.9408719964531e-08
4918 1.94074782911002e-08
4919 1.94063947134282e-08
4920 1.94050890911512e-08
4921 1.94037443890238e-08
4922 1.94024867283815e-08
4923 1.94012219623119e-08
4924 1.93998879183255e-08
4925 1.93988238805787e-08
4926 1.93974845075218e-08
4927 1.93960101313451e-08
4928 1.93948341831174e-08
4929 1.93934983627742e-08
4930 1.93920808300163e-08
4931 1.93908622492245e-08
4932 1.93894731381761e-08
4933 1.93882652155253e-08
4934 1.9386973804103e-08
4935 1.93858156194437e-08
4936 1.93845153262373e-08
4937 1.93832772055202e-08
4938 1.93818276983393e-08
4939 1.93806624082526e-08
4940 1.93793781022578e-08
4941 1.93781684032501e-08
4942 1.9376919624392e-08
4943 1.93753955102238e-08
4944 1.93743012744108e-08
4945 1.93728677544414e-08
4946 1.93717912821967e-08
4947 1.93703790785094e-08
4948 1.9369199577568e-08
4949 1.93681302107507e-08
4950 1.93666469527898e-08
4951 1.93653821867201e-08
4952 1.93639095869003e-08
4953 1.9362696335179e-08
4954 1.93615097288102e-08
4955 1.93602787135205e-08
4956 1.93588807206879e-08
4957 1.93576976670329e-08
4958 1.9356269476134e-08
4959 1.93551290550431e-08
4960 1.93536386916549e-08
4961 1.93524414271451e-08
4962 1.93509830381799e-08
4963 1.93496809686167e-08
4964 1.93484357424722e-08
4965 1.934717808183e-08
4966 1.9345982593677e-08
4967 1.93446147989107e-08
4968 1.9343328716559e-08
4969 1.93418738803075e-08
4970 1.93409341875395e-08
4971 1.93396534342583e-08
4972 1.93384206426117e-08
4973 1.93368663303772e-08
4974 1.9335619327876e-08
4975 1.93341911369771e-08
4976 1.93331590736534e-08
4977 1.93320808250519e-08
4978 1.93305140783195e-08
4979 1.93292848393867e-08
4980 1.93277802651437e-08
4981 1.93266398440528e-08
4982 1.9325133493453e-08
4983 1.93240001777895e-08
4984 1.93227798206408e-08
4985 1.93215257127122e-08
4986 1.93203586462687e-08
4987 1.93191116437674e-08
4988 1.9317711874578e-08
4989 1.93166762585406e-08
4990 1.93152285277165e-08
4991 1.93138145476723e-08
4992 1.93125213598933e-08
4993 1.93114484403623e-08
4994 1.93101286072306e-08
4995 1.93087572597506e-08
4996 1.93074551901873e-08
4997 1.93063769415858e-08
4998 1.93050855301635e-08
4999 1.93036413520531e-08
};
\addlegendentry{Test}

\end{groupplot}

\end{tikzpicture}
		\caption{Eight experiments with different combinations of activation functions for \(\hy\) (left) and \(\rare\) (right). Shown are training- and validation error over 5000 epochs.}
\end{figure}\label{Fig:Activations}
%\newpage
\chapter{Hyperparameters for the Convolutional Autoencoder}
\label{Ch:ApB}

The finding of appropriate hyperparameters for the convolutional autoencoder is described here. The hyperparameters include  batch size, non-linear activation functions, number of epochs for training and learning rate and number of layers i.e. depth. Depth comprises kernel size, stride width as well as number of channels per layer.\\
This analysis follows the prior finding of hyperparameters for the fully connected autoencoder, hence utilizes insights thereof. Additionally this analysis follows a slightly different scheme than before, as there are solely 40 examples for training and validation for both rarefaction levels \(\hy\) and \(\rare\). Hence it is tried to find a combined model that performs well on both rarefaction levels. With this measure datasets for both rarefaction levels are concatenated to one dataset of 80 examples for training and validation. Furthermore this approach gives an answer to how well a model can generalize about the BGK model for different rarefaction levels in Sod's shock tube.\\
Again we start with a small model which architecture and hyperparameters are summarized in \cref{Tab:small}.
\begin{table}[htbp!]
\centering
\caption{Initial selection of hyperparameters for training and architecture. The initial model compirses 6 layers, of which 4 are either convolutional (Conv.) or transposed convolutional (Tr.Conv.). The learning rate is switched from \num{e-4} to \num{e-4} at the 500th epoch as in \cref{Ch:ApA}. Depth counts the number of convolutional/transposed concolutional layers and excludes the fully connected layers.}
\begin{tabular*}{16cm}{ @{\extracolsep{\fill}} c c c c c c c c c c @{} }
	\toprule
	\multicolumn{4}{c}{Encoder} & \multicolumn{4}{c}{Decoder}\\ [.5ex] \hline
	Layer & Type & Channels & Output size & Layer & Type & Channels & Output size \\ 
	\hline
	1 & Conv  & 8  & $5\times 40$ & 4 & Lin.     & -  & 128            \\ \hline
	2 & Conv. & 16 & $1\times 8$  & 5 & Tr.Conv. & 16 & $1\times 8$    \\ \hline
	3 & Lin.  & -  & 5		      & 6 & Tr.Conv. & 8  & $25\times 200$ \\ 		
\end{tabular*}
\begin{tabular*}{16cm}{ @{\extracolsep{\fill}} c c c c c c c @{} }
	\toprule
	Epochs & Act. hidden/code & Batch size & Depth & Kernel size & Stride width & Learning rate\\ [.5ex]
	\hline
	1000 &  ReLU/ReLU  & 2 & 4 & \(5\times 5\) & \(5\times 5\) & \num{e-4}/\num{e-5} \\ \hline
\end{tabular*}\label{Tab:small}
\end{table} 
The activations and learning rate stem from conclusions made in \cref{Ch:ApA}. The application of activation functions follows the same sheme as described in \cref{Ch:ApA} seen in \cref{Fig:ActScheme}. Kernel size and stride width are chosen as to circumvent checkerboard artefacts as described in \cref{Ch:DimRedAl}. The number of channels is inspired by the architecture chosen in \cite{Carlberg}.\\
The number of available examples for training and validation limited as previously stated . Therefor the split in training and validation sets encompasses the risk of a bias in one of the sets. For instance, if out of the 40 examples, a few of them ,maybe 5 , contain considerable variance to the other examples and all of them are found in the training set, then the validation error could not estimate the model's ability to generalize. Therefore the k-fold algorithm described in \cite{Goodfellow} is adopted. The k-fold algorithm provides from the complete shuffled dataset, k independent splits into training and validation set with which the model is trained. By that, each example gets the chance to be either in the training or validation set once. For an 80/20 split as used in this thesis and described in \cref{Ch:DimRedAl}, five independent folds can be obtained. Hence with 80 available examples the training set \(\btch_{train}\) consists of 74 examples and the validation set \(\btch_{val}\) of the remaining 16 examples. The results of said preliminary experiment are summarized in \cref{Tab:kFold} and \cref{Fig:kFold}.
\begin{table}[htbp!]
	\centering
	\caption{Training of five independent folds. Summary of minimum training- and minimum validation error for a small model with two convolutional layers in the encoder as well as the corresponding \(\L2\) and the epoch in which those values are reached. The mean and standard deviation of the validation error are \num{3.55e-5} and \num{2.13e-5} respectively with a corresponding variance of \num{4.56e-10}.}
	\begin{tabular*}{15cm}{ @{\extracolsep{\fill}} c c c c c c c c c c @{} }
		\toprule
		Fold & Minimum training error & Minimum validation error & \(\L2\) & Epoch\\ [.5ex]
		\hline
		 1   & \num{2.4e-5}           & \num{7.0e-5}             & 0.053   & 498  \\  
		\hline
		2    & \num{1.4e-5}           & \num{1.3e-5}             & 0.044   & 1000\\
		\hline
		3    & \num{1.7e-5}           & \num{5.0e-5}             & 0.057   & 1000\\
		\hline
		4    & \num{1.7e-5}           & \num{1.8e-5}             & 0.039   & 997\\
		\hline
		5    & \num{1.6e-5}           & \num{2.7e-5}             & 0.053  & 1000\\
		\hline
	\end{tabular*}\label{Tab:kFold}
\end{table}
The lowest validation error of \num{1.3e-5} is achieved with the second fold at the last epoch. The corresponding \(L2 = 0.044\). As training error is lower than validation error it is assumed, that the \say{difficult} examples are within the training set. Therefore the training set could be biased. The second lowest validation error of \num{1.8e-5} and a corresponding \(\L2=0.039\) is achieved with fold 4 at the 997th epoch. For both folds training and validation error evolve in unison and get instable even before the 500th epoch. The instability is eliminated by lowering the learning rate at said epoch, but the training does not improve thereafter as in \cref{Ch:ApA}. For the other folds a seperation of training and validation error can be observed and less instability. The mean validation error of \num{3.55e-5} gives an estimate of the models performance with a standard deviation of \num{2.13e-5} and variance of \num{4.56e-10}. For the continuation of experiments fold 4 is used, as it provides a balanced split in training- and validation split which also manifests in the lowest \(\L2\) of all folds.\\
Next the capacity of the model is increased by successively adding convolutional layers, thus obtaining two additional models. One encompasses three convolutional layers, the other four convolutional layers in encoder and decoder. An exact description of both architectures is summarized in \cref{Tab:Layer}. The maximum number of channels for this analysis is kept at 16. This measure decreases the growth rate of channels per layer, which is quadratic for the first model and now scales to linear in order to increase capacity rather smooth. Note that it is also possible to keep the growth rate constant while adding layers. Here this is done in the following steps.\\
Kernel size and stride width shrink the input as described in \cref{Ch:DimRedAl}. In the previous model shrinkage evolved with a pace of \(\frac{1}{5}\). This reduction rate is able to reduce the input size over two succesive layers to unity. Hence the same kernel size and stride width can only be adopted by using excessive zero padding between successive convolutional layers. To overcome this issue a balance between zero padding, which cannot be omitted for four convolutional layers with the given input dimensions, and kernel size and stride width needs to be found. Here both features are chosen to have sizes \(3\times 3\). Extra padding, which is needed for one to reduce the shrinkage rate per layer in the encoder and for the other to transpose the shrinkage of the encoder in the decoder, is summarized in \cref{Tab:Layer}. In \texttt{pytorch}'s implementation of transposed convolutional layers a handy parameter can be used called output padding. This parameter provides a measure to effectively increase the output size by one, left or at the bottom of the output, to resolve the ambiguity when \(\text{stride}>1\). Then the respective convolutional layer maps multiple input sizes to the same output size. Taking for example the second and sixth layer of the model with three convolutional layers in encoder and decoder. Here the convolutional layer maps a size of 23 to 8. But the respective transposed convolutional layer maps size 8 to size 22. In this case output padding resolves the issue by increasing the output size by one.         
\begin{center}
	\begin{figure}[htbp!]
		% This file was created by tikzplotlib v0.9.6.
\begin{tikzpicture}

\begin{groupplot}[group style={group size=1 by 5}]
\nextgroupplot[
legend cell align={left},
legend style={fill opacity=0.8, draw opacity=1, text opacity=1, at={(0.91,0.5)}, anchor=east, draw=white!80!black},
log basis y={10},
tick align=outside,
tick pos=left,
title={fold0 },
x grid style={white!69.0196078431373!black},
xlabel={Epoch},
xmin=-99.95, xmax=2098.95,
xtick style={color=black},
y grid style={white!69.0196078431373!black},
ylabel={MSE Loss},
ymin=5.13626222579209e-06, ymax=0.0134974806769736,
ymode=log,
ytick style={color=black}
]
\addplot [semithick, black, dashed]
table {%
0 0.00637218570663833
1 0.00636555681795414
2 0.00635694373494289
3 0.00634543747054295
4 0.00632883801847584
5 0.00630568442301183
6 0.00627175879185415
7 0.00622269169741685
8 0.00615070119465599
9 0.00605453467414918
10 0.00593544460753037
11 0.00579535059478076
12 0.0056417863106617
13 0.00547931936671375
14 0.00530924932172638
15 0.00513132990363374
16 0.00494504698690434
17 0.00475059453674476
18 0.00455046348452015
19 0.00434519867667404
20 0.00413800447495305
21 0.00393186266774137
22 0.00372581593637733
23 0.00351972797852795
24 0.00331750622262916
25 0.00312577791555668
26 0.00294681397554086
27 0.00278088177947211
28 0.00262565217644806
29 0.00248040028691321
30 0.00234671220687233
31 0.00222279048512064
32 0.00210749112306985
33 0.00200038816547021
34 0.0019056912360611
35 0.00181777500847602
36 0.00173395965271084
37 0.00165342447905914
38 0.00157644129251366
39 0.00150477146416961
40 0.00143918772494089
41 0.00137983844041401
42 0.00132534253100403
43 0.00127430982934129
44 0.001225197542567
45 0.00117868666654886
46 0.00113512418931805
47 0.00109462839839125
48 0.00105680168866229
49 0.00102053049840833
50 0.000985031090635857
51 0.000951600068788139
52 0.000919941240169919
53 0.000890776508128965
54 0.000864017231748448
55 0.000839353689968902
56 0.000816614941868465
57 0.000795383438628505
58 0.000775267798701407
59 0.000756124072552211
60 0.000736864335635801
61 0.000717434460028699
62 0.000698551311671736
63 0.000680945698576352
64 0.000664424874457836
65 0.000648640157436375
66 0.000633239104274708
67 0.000617645256426158
68 0.000602611245852813
69 0.00058811548908011
70 0.000574277749535668
71 0.000561153566422945
72 0.000548689112150669
73 0.000536999257747084
74 0.00052577305288537
75 0.000514745993314136
76 0.000503332068944928
77 0.000491914856723952
78 0.000480974828974468
79 0.000470028856938143
80 0.000459047742793928
81 0.000448308496942218
82 0.000438029408314833
83 0.00042810107036928
84 0.00041876653527595
85 0.000410031321436577
86 0.000401779582944073
87 0.000393924861313621
88 0.000386466933562701
89 0.000379333551123295
90 0.000372507112444964
91 0.000365985708143057
92 0.000359716189848314
93 0.000353684219760453
94 0.000347918117498125
95 0.000342382307977118
96 0.000337044521799612
97 0.000331776675949413
98 0.000326637601446578
99 0.000321725835263464
100 0.000317009654011713
101 0.000312411409069568
102 0.000307589518598661
103 0.000302886088888954
104 0.000298399033539454
105 0.000294108083295441
106 0.000289958998536122
107 0.000285963577837833
108 0.000282127733854054
109 0.000278365406643388
110 0.000274530126617378
111 0.000270804345845477
112 0.000267214196952636
113 0.000263763402841732
114 0.000260379589704485
115 0.000256880319525976
116 0.000253539775897593
117 0.000250370813887457
118 0.000247342199322986
119 0.000244440169780091
120 0.000241588878935772
121 0.000238475935695703
122 0.000235625395923478
123 0.000232821712550901
124 0.000229702436747914
125 0.000226885004224187
126 0.000224295040254674
127 0.000221806237043154
128 0.000219454353540982
129 0.000217245422881263
130 0.000215030225803048
131 0.0002129361065073
132 0.000210932668828434
133 0.000209051869831001
134 0.000207222109967375
135 0.000205458018758264
136 0.000203737868261555
137 0.000202066195786443
138 0.000200396214136589
139 0.000198773762665638
140 0.000197111624885338
141 0.000195527595305833
142 0.000194013528769688
143 0.000192519099449839
144 0.000191112424339224
145 0.000189722880776344
146 0.000188385912810674
147 0.000187085351917915
148 0.000185842866784469
149 0.000184613834640857
150 0.000183450211714842
151 0.000182291657903733
152 0.000181198403947036
153 0.000180116583536005
154 0.000179065453306038
155 0.000178037767895489
156 0.000177051096747149
157 0.000176084564953882
158 0.000175141216168129
159 0.000174222774365518
160 0.000173356646069323
161 0.00017248799822589
162 0.00017165093982463
163 0.000170839287223501
164 0.000170069334883394
165 0.000169297757830478
166 0.000168559741579699
167 0.000167831371737037
168 0.000167151565248602
169 0.000166453213415352
170 0.000165792048399283
171 0.000165133413567276
172 0.000164518489784271
173 0.000163882688674732
174 0.000163280443358005
175 0.00016267802589276
176 0.000162105557999226
177 0.000161519182015013
178 0.000160947311950821
179 0.000160392715780944
180 0.000159875141516608
181 0.000159348128256909
182 0.000158829932029647
183 0.000158327406065695
184 0.000157815872736222
185 0.00015735064623712
186 0.000156849977371643
187 0.000156383803132876
188 0.000155896211643025
189 0.000155441842440318
190 0.000154982218639965
191 0.000154519314967239
192 0.000154073978396063
193 0.000153639225017876
194 0.000153200687954325
195 0.000152770289640891
196 0.000152329371987392
197 0.000151875316835692
198 0.000151462430181937
199 0.000151034832319041
200 0.000150612010586748
201 0.000150176556336312
202 0.000149762873586923
203 0.000149366651633009
204 0.000148935867607669
205 0.00014853612919552
206 0.000148113827802554
207 0.000147718337895064
208 0.000147325569884948
209 0.00014691063870842
210 0.000146492236859253
211 0.000146095598722695
212 0.000145659884629445
213 0.000145249496305233
214 0.000144824043227132
215 0.000144410866877998
216 0.000143992850421881
217 0.000143567553338997
218 0.000143168622465772
219 0.000142726587842645
220 0.00014230893995304
221 0.000141857401907863
222 0.000141437835214475
223 0.000140995191705962
224 0.000140548922038519
225 0.00014010977233081
226 0.000139673218438219
227 0.000139219905824106
228 0.000138775459127771
229 0.000138331404773595
230 0.000137844704743983
231 0.000137408439599973
232 0.000136895052029296
233 0.000136449158401497
234 0.000135867391001021
235 0.000135382734120171
236 0.000134823761916536
237 0.000134400256947664
238 0.00013378066243952
239 0.000133453117618032
240 0.000132774903589983
241 0.00013270513508834
242 0.000131924415838203
243 0.000131942320885425
244 0.000130791424887367
245 0.000130510401332273
246 0.000129400398923174
247 0.000129002923728638
248 0.000128179609125922
249 0.000127700828937094
250 0.000126956654621146
251 0.000126445017651022
252 0.000125705603885251
253 0.000125139874879387
254 0.000124379928898577
255 0.000123778614700143
256 0.000122961810514077
257 0.000122293684786001
258 0.000121520512284867
259 0.000120822935620879
260 0.000120007962198798
261 0.000119307835109161
262 0.000118439594751329
263 0.000117749061823602
264 0.000116869398357977
265 0.000116159957809003
266 0.000115226498432541
267 0.000114535346843692
268 0.000113536448219875
269 0.000112928810831026
270 0.000111853478379498
271 0.0001112292405967
272 0.000110061421426444
273 0.000109459679455526
274 0.000108180200526498
275 0.000107534479778071
276 0.000106202594888583
277 0.000105466233880858
278 0.000104183131723001
279 0.000103383582861216
280 0.000102188520253677
281 0.00010131080171849
282 9.99495530100702e-05
283 9.89900014980449e-05
284 9.77681008351716e-05
285 9.6792392352274e-05
286 9.55151592401116e-05
287 9.4544353548276e-05
288 9.32841872236523e-05
289 9.22798690088378e-05
290 9.09717736110593e-05
291 8.99957702782395e-05
292 8.85580400571406e-05
293 8.7600340499705e-05
294 8.61798395654922e-05
295 8.52183919408667e-05
296 8.37991208544331e-05
297 8.28842949820441e-05
298 8.13884395358677e-05
299 8.04756913352911e-05
300 7.90232036829153e-05
301 7.80876764499006e-05
302 7.65933287008025e-05
303 7.57259956780665e-05
304 7.42189150702011e-05
305 7.33209056305029e-05
306 7.17909454559518e-05
307 7.09253492381379e-05
308 6.93804944744691e-05
309 6.85530964652514e-05
310 6.70486336242782e-05
311 6.6246945270354e-05
312 6.47700007831986e-05
313 6.39570355542851e-05
314 6.25613016085325e-05
315 6.17904746942344e-05
316 6.0405123185614e-05
317 5.96647876283107e-05
318 5.83103261160289e-05
319 5.76302033712395e-05
320 5.63080148778816e-05
321 5.56920504855007e-05
322 5.44378489708564e-05
323 5.39418907283817e-05
324 5.26253869788107e-05
325 5.22562348317379e-05
326 5.09608775369763e-05
327 5.06447972838941e-05
328 4.9364311723199e-05
329 4.90926074911968e-05
330 4.78769379159694e-05
331 4.76013150323418e-05
332 4.64512190254673e-05
333 4.61676939855771e-05
334 4.51009919628653e-05
335 4.48268646375816e-05
336 4.37907791690506e-05
337 4.35519667267847e-05
338 4.25732893143405e-05
339 4.2361820848491e-05
340 4.14321211024138e-05
341 4.12674463277085e-05
342 4.0371437481479e-05
343 4.02466379632038e-05
344 3.93746549516294e-05
345 3.93175780895039e-05
346 3.84199806711116e-05
347 3.84381770874853e-05
348 3.75136810282228e-05
349 3.75541567949611e-05
350 3.66831992835159e-05
351 3.67399351104414e-05
352 3.58984782709015e-05
353 3.59884599561155e-05
354 3.51464983872063e-05
355 3.5317492039999e-05
356 3.44497862267978e-05
357 3.4597948247006e-05
358 3.37291889205282e-05
359 3.39338705743764e-05
360 3.31101576911408e-05
361 3.32980557080642e-05
362 3.25283591586789e-05
363 3.27209607815515e-05
364 3.19848670797818e-05
365 3.21473156355001e-05
366 3.14468529283296e-05
367 3.15987881320812e-05
368 3.09235373918693e-05
369 3.10540618366772e-05
370 3.04407651656735e-05
371 3.05744940716224e-05
372 3.00037478704329e-05
373 3.01094899313181e-05
374 2.95992427965519e-05
375 2.96616339585665e-05
376 2.9177683962045e-05
377 2.92281154274932e-05
378 2.88017859599865e-05
379 2.88150661071906e-05
380 2.84156203866903e-05
381 2.83934555305176e-05
382 2.80171552105024e-05
383 2.80672862178477e-05
384 2.76364268707585e-05
385 2.77428280217595e-05
386 2.72799099736432e-05
387 2.74776516295816e-05
388 2.69256079596936e-05
389 2.72123473004093e-05
390 2.65987513976551e-05
391 2.70239885851176e-05
392 2.62919397973072e-05
393 2.68830802765407e-05
394 2.60161822474014e-05
395 2.678710443349e-05
396 2.57527727138829e-05
397 2.6600895807416e-05
398 2.54828246442074e-05
399 2.62795471419608e-05
400 2.52012938712909e-05
401 2.5830894126555e-05
402 2.49438879946773e-05
403 2.54601009117561e-05
404 2.47033304967559e-05
405 2.51013893881602e-05
406 2.449397980131e-05
407 2.48110630673981e-05
408 2.42736666871135e-05
409 2.45999897130478e-05
410 2.40504127946295e-05
411 2.44441200598633e-05
412 2.38332447597411e-05
413 2.43504384014326e-05
414 2.36421544781429e-05
415 2.43159474115906e-05
416 2.34888528023447e-05
417 2.43639522463646e-05
418 2.33329855632292e-05
419 2.4208461809172e-05
420 2.31247796944611e-05
421 2.38596413144609e-05
422 2.29040630808086e-05
423 2.34965123813158e-05
424 2.27159126562348e-05
425 2.3229181499218e-05
426 2.25242523415048e-05
427 2.29642789992823e-05
428 2.23445662821398e-05
429 2.28393923189518e-05
430 2.21946337055634e-05
431 2.27174743585579e-05
432 2.20520449749628e-05
433 2.27260378404814e-05
434 2.194723131832e-05
435 2.27086069752858e-05
436 2.18257014281686e-05
437 2.26383325079293e-05
438 2.17038834471328e-05
439 2.2458001857073e-05
440 2.15218056436317e-05
441 2.21856760875827e-05
442 2.13581470445146e-05
443 2.18994502638736e-05
444 2.12024718857506e-05
445 2.17324242075545e-05
446 2.10803946636418e-05
447 2.16170592985954e-05
448 2.09642954969169e-05
449 2.15264029128637e-05
450 2.08768222789146e-05
451 2.1578106818021e-05
452 2.08128421732567e-05
453 2.16039707372495e-05
454 2.07465640693627e-05
455 2.15267165915023e-05
456 2.06228021595312e-05
457 2.13021549504422e-05
458 2.04512051431571e-05
459 2.10489879164388e-05
460 2.03109156844405e-05
461 2.08200932527713e-05
462 2.01814370599518e-05
463 2.07019572961986e-05
464 2.0100137262391e-05
465 2.06470370585077e-05
466 2.00224993025522e-05
467 2.06455878803968e-05
468 1.99833111644843e-05
469 2.07211731373391e-05
470 1.99510157221638e-05
471 2.06901715467112e-05
472 1.98623502667417e-05
473 2.05231600065581e-05
474 1.97080349979561e-05
475 2.02876164985266e-05
476 1.95624015755236e-05
477 2.00954514624208e-05
478 1.9461769066087e-05
479 1.99450622115904e-05
480 1.9373905622011e-05
481 1.99183217446386e-05
482 1.93273431621233e-05
483 1.99760055135112e-05
484 1.93220399709038e-05
485 2.00091237896327e-05
486 1.92826953402392e-05
487 1.99489513834639e-05
488 1.91726161511419e-05
489 1.97960352459337e-05
490 1.90477375807685e-05
491 1.95894444670053e-05
492 1.89306705633552e-05
493 1.94514285536407e-05
494 1.88546174799686e-05
495 1.93606066678509e-05
496 1.87775064817153e-05
497 1.93190922865849e-05
498 1.87431286020612e-05
499 1.93683268565614e-05
500 1.87331180603323e-05
501 1.93927232776669e-05
502 1.86955822234491e-05
503 1.93077556214227e-05
504 1.85936430578959e-05
505 1.91421273014747e-05
506 1.84674902796722e-05
507 1.89726416608416e-05
508 1.83560772704094e-05
509 1.88259950184566e-05
510 1.82691623500375e-05
511 1.87741261044039e-05
512 1.82314427832342e-05
513 1.87654291676775e-05
514 1.82268307065137e-05
515 1.88197023698322e-05
516 1.82274638618196e-05
517 1.88650522894296e-05
518 1.82020537309313e-05
519 1.87698578346662e-05
520 1.80741570883214e-05
521 1.85597467472487e-05
522 1.79461442570972e-05
523 1.8396472996951e-05
524 1.7852490279191e-05
525 1.83111411606873e-05
526 1.78056031647778e-05
527 1.82641405233053e-05
528 1.77816306143885e-05
529 1.83178823187635e-05
530 1.77872661868417e-05
531 1.83748894047042e-05
532 1.78235663188531e-05
533 1.83722971025535e-05
534 1.77357438397863e-05
535 1.82141711162487e-05
536 1.75976042680048e-05
537 1.80298120033839e-05
538 1.75085019298304e-05
539 1.79273896230292e-05
540 1.74477450460842e-05
541 1.78769446579352e-05
542 1.74223552806829e-05
543 1.79171184526705e-05
544 1.74486985593525e-05
545 1.79879912816094e-05
546 1.74326992028462e-05
547 1.79338819161767e-05
548 1.73504287859227e-05
549 1.77907558753798e-05
550 1.72443172397063e-05
551 1.76505337095012e-05
552 1.71586419477343e-05
553 1.7560365038749e-05
554 1.71160565654027e-05
555 1.75587280382139e-05
556 1.71156520960558e-05
557 1.75727988060981e-05
558 1.71073224140006e-05
559 1.75697096638228e-05
560 1.70541307673577e-05
561 1.74667327896838e-05
562 1.69645387610995e-05
563 1.73588210399256e-05
564 1.69081549277195e-05
565 1.73068972184609e-05
566 1.68849587776077e-05
567 1.73036378474656e-05
568 1.68700699632041e-05
569 1.72895263230721e-05
570 1.68212906059839e-05
571 1.7226756509503e-05
572 1.67703408759934e-05
573 1.71485688664763e-05
574 1.67319404782429e-05
575 1.71110956661558e-05
576 1.66986692882176e-05
577 1.70691165031212e-05
578 1.66542484922871e-05
579 1.7049407495362e-05
580 1.66429691794701e-05
581 1.7007509357958e-05
582 1.6598792080158e-05
583 1.69539668166685e-05
584 1.65512849228389e-05
585 1.68959729536233e-05
586 1.65242557219614e-05
587 1.68636285367008e-05
588 1.64925092733803e-05
589 1.68342251496423e-05
590 1.6469726929369e-05
591 1.68126872024388e-05
592 1.6433168434693e-05
593 1.67583535972682e-05
594 1.63761668892093e-05
595 1.66783810775506e-05
596 1.63170101949994e-05
597 1.66288716567653e-05
598 1.62871908893081e-05
599 1.65898283752686e-05
600 1.62544247126473e-05
601 1.65653597663518e-05
602 1.62518858810579e-05
603 1.65644423328892e-05
604 1.62299357480133e-05
605 1.65566021981256e-05
606 1.6211013696088e-05
607 1.65163596410611e-05
608 1.6182448661084e-05
609 1.64678057130718e-05
610 1.61241133467271e-05
611 1.63702891342599e-05
612 1.60456614308746e-05
613 1.6297758875794e-05
614 1.59947629241319e-05
615 1.62465882500573e-05
616 1.59698683070975e-05
617 1.62526060671908e-05
618 1.6013573377327e-05
619 1.63373693284896e-05
620 1.60591325710779e-05
621 1.63380226205767e-05
622 1.59850823677754e-05
623 1.62061190476237e-05
624 1.59056853892281e-05
625 1.61260183983991e-05
626 1.58118148760877e-05
627 1.60077699782324e-05
628 1.57392047199778e-05
629 1.59513151347657e-05
630 1.57125813922132e-05
631 1.59545901858849e-05
632 1.57461390202229e-05
633 1.60243688778205e-05
634 1.58540900812021e-05
635 1.61971712933084e-05
636 1.59120483682607e-05
637 1.61150798430043e-05
638 1.57121307216013e-05
639 1.58446123637024e-05
640 1.55414424427569e-05
641 1.5700314214051e-05
642 1.54724015590801e-05
643 1.5640556441987e-05
644 1.54652230053287e-05
645 1.57011734145396e-05
646 1.55539100246127e-05
647 1.58619475076893e-05
648 1.56970936813217e-05
649 1.5970038692803e-05
650 1.57225961725516e-05
651 1.59057979429722e-05
652 1.54974825383736e-05
653 1.5582473000153e-05
654 1.52896604059816e-05
655 1.54077433489608e-05
656 1.52094631549193e-05
657 1.53426954012659e-05
658 1.51982489955405e-05
659 1.54118735316011e-05
660 1.53526899271395e-05
661 1.57009407975028e-05
662 1.557577751643e-05
663 1.58267942662604e-05
664 1.55738797420479e-05
665 1.56658876542259e-05
666 1.5264445810903e-05
667 1.52980944216896e-05
668 1.5039328627342e-05
669 1.51151088365609e-05
670 1.49616828069021e-05
671 1.50674442633836e-05
672 1.49846566193101e-05
673 1.51991155510345e-05
674 1.51707174151916e-05
675 1.54588414951196e-05
676 1.54832450820486e-05
677 1.5753717666378e-05
678 1.54169957737871e-05
679 1.53857655127432e-05
680 1.50245817367445e-05
681 1.50011282580298e-05
682 1.47844043327794e-05
683 1.48173111591543e-05
684 1.47055581294353e-05
685 1.48267942334801e-05
686 1.47943051866939e-05
687 1.49886874276639e-05
688 1.51068738163396e-05
689 1.54700660468521e-05
690 1.54217376376886e-05
691 1.55124300631471e-05
692 1.51792151672492e-05
693 1.50554723581386e-05
694 1.47260095597623e-05
695 1.46836877026324e-05
696 1.45246756206863e-05
697 1.4559948184667e-05
698 1.45095845254239e-05
699 1.45871372951856e-05
700 1.46300615697648e-05
701 1.48569998195125e-05
702 1.50013849105068e-05
703 1.52851256198794e-05
704 1.53671129781152e-05
705 1.5418932331368e-05
706 1.50244094003771e-05
707 1.47864910613471e-05
708 1.44642929880145e-05
709 1.43846633455791e-05
710 1.42724132632388e-05
711 1.42780138163712e-05
712 1.42571712142825e-05
713 1.4304580360669e-05
714 1.43805219332549e-05
715 1.46168314343331e-05
716 1.48632444793018e-05
717 1.52317501500576e-05
718 1.54385063854123e-05
719 1.54163847387867e-05
720 1.48755649145293e-05
721 1.45454955453594e-05
722 1.42437429444531e-05
723 1.4135270171689e-05
724 1.40499495508983e-05
725 1.4016357815283e-05
726 1.3995593675542e-05
727 1.40367684760001e-05
728 1.40811946807151e-05
729 1.42335281450023e-05
730 1.45468322350073e-05
731 1.49930068060744e-05
732 1.53588608531319e-05
733 1.54266435659078e-05
734 1.4975051009003e-05
735 1.45174736549958e-05
736 1.41385194785126e-05
737 1.39648105430545e-05
738 1.3881061078358e-05
739 1.38140790015839e-05
740 1.38016770689919e-05
741 1.38019789928112e-05
742 1.38550579831254e-05
743 1.39293510306171e-05
744 1.41685664685198e-05
745 1.45002542619732e-05
746 1.50375804816028e-05
747 1.53174575854287e-05
748 1.51955544045457e-05
749 1.46459356147322e-05
750 1.41432195825963e-05
751 1.38473652175897e-05
752 1.37055367126493e-05
753 1.36307169853467e-05
754 1.36189869763825e-05
755 1.35914498657908e-05
756 1.36317677006392e-05
757 1.36339931848806e-05
758 1.37782902978056e-05
759 1.39616264953624e-05
760 1.44406646115369e-05
761 1.48994174189454e-05
762 1.54363198630048e-05
763 1.50813082833157e-05
764 1.43624188577718e-05
765 1.38187240239507e-05
766 1.36039274261091e-05
767 1.34592275971013e-05
768 1.34394075070876e-05
769 1.33778838227183e-05
770 1.33953365282213e-05
771 1.33730815954358e-05
772 1.34309791406384e-05
773 1.34723482423382e-05
774 1.37619471334283e-05
775 1.40879436050234e-05
776 1.48840414052609e-05
777 1.52065015035507e-05
778 1.5162550144554e-05
779 1.42962147622328e-05
780 1.37482438591352e-05
781 1.33865515064002e-05
782 1.33150612136035e-05
783 1.32161393271435e-05
784 1.3215672942879e-05
785 1.31588334386645e-05
786 1.32033010662447e-05
787 1.31604630542181e-05
788 1.32684830895791e-05
789 1.33329850702246e-05
790 1.37787921933308e-05
791 1.42212071876369e-05
792 1.5446708587552e-05
793 1.53969169858215e-05
794 1.45607096754263e-05
795 1.35655591964046e-05
796 1.32807773978172e-05
797 1.30647954796892e-05
798 1.30678697445319e-05
799 1.29821537449004e-05
800 1.2989569965649e-05
801 1.29483328772384e-05
802 1.29876409573626e-05
803 1.29489855256182e-05
804 1.30646109999244e-05
805 1.31028661063404e-05
806 1.36695769659134e-05
807 1.42170046564249e-05
808 1.57499663692162e-05
809 1.5353130166873e-05
810 1.42137378883467e-05
811 1.32231758360746e-05
812 1.30127319790674e-05
813 1.28430644514843e-05
814 1.28567285397896e-05
815 1.27813650381281e-05
816 1.27865417736128e-05
817 1.27499069948467e-05
818 1.27885474845435e-05
819 1.27290771070321e-05
820 1.28356002395691e-05
821 1.28054230270624e-05
822 1.32225162330357e-05
823 1.36339017364762e-05
824 1.54408001936668e-05
825 1.55387046726352e-05
826 1.46876994555623e-05
827 1.32397790510996e-05
828 1.28921302406937e-05
829 1.26641613904521e-05
830 1.26629432704028e-05
831 1.25959951062793e-05
832 1.26100132622486e-05
833 1.25563892605651e-05
834 1.25724265647165e-05
835 1.25229621563694e-05
836 1.25590940955878e-05
837 1.2495882807162e-05
838 1.25816014637792e-05
839 1.25885093997802e-05
840 1.30952348490476e-05
841 1.37296370719131e-05
842 1.64236036562793e-05
843 1.5877410797116e-05
844 1.40044411285345e-05
845 1.27008037964504e-05
846 1.2547312944422e-05
847 1.24274131154678e-05
848 1.24362624220886e-05
849 1.23854233122778e-05
850 1.23911880971495e-05
851 1.23485894647501e-05
852 1.23267043725672e-05
853 1.23099594677711e-05
854 1.23223488530666e-05
855 1.22764557251021e-05
856 1.22930486932127e-05
857 1.22466029692792e-05
858 1.23008731418572e-05
859 1.22906653927668e-05
860 1.28517943830397e-05
861 1.37602014280702e-05
862 1.80488168745052e-05
863 1.63158255681584e-05
864 1.3223580243138e-05
865 1.23258354625122e-05
866 1.22649277320974e-05
867 1.22097577015534e-05
868 1.22175696271043e-05
869 1.2170241877163e-05
870 1.21731193802876e-05
871 1.21259614853653e-05
872 1.21386418350244e-05
873 1.20978812898631e-05
874 1.21042938145965e-05
875 1.20625105546335e-05
876 1.20654569895073e-05
877 1.2026408567456e-05
878 1.20726400589799e-05
879 1.19911179798926e-05
880 1.2156711117095e-05
881 1.21876458261116e-05
882 1.33993800237597e-05
883 1.48386424035918e-05
884 1.76804074341064e-05
885 1.42512005308904e-05
886 1.24302463959669e-05
887 1.20515956634204e-05
888 1.20256114917172e-05
889 1.19814705836774e-05
890 1.1968687247732e-05
891 1.19294157325811e-05
892 1.19318674145497e-05
893 1.18878896648855e-05
894 1.18925200558451e-05
895 1.1857671931681e-05
896 1.18565465097031e-05
897 1.18247952742845e-05
898 1.1819737933294e-05
899 1.17866572939551e-05
900 1.18212699591336e-05
901 1.17627117341979e-05
902 1.19651874386228e-05
903 1.20059421719887e-05
904 1.38882738281998e-05
905 1.56721953269434e-05
906 1.72576210708542e-05
907 1.31804320065987e-05
908 1.20228305717784e-05
909 1.18191640363641e-05
910 1.17967789412532e-05
911 1.17566460382523e-05
912 1.17378093846821e-05
913 1.17185794046293e-05
914 1.16922125239505e-05
915 1.16782151771133e-05
916 1.16546429664854e-05
917 1.16443125017796e-05
918 1.16215083711957e-05
919 1.15996401988117e-05
920 1.15880916304878e-05
921 1.1562914996599e-05
922 1.15596903849102e-05
923 1.15380114797947e-05
924 1.15322172609122e-05
925 1.14983640457922e-05
926 1.14994749351727e-05
927 1.14696033093331e-05
928 1.14854600319347e-05
929 1.14336730769882e-05
930 1.16470192459417e-05
931 1.21702189539441e-05
932 1.86656558509579e-05
933 1.92806589054406e-05
934 1.29708322432753e-05
935 1.16717602358651e-05
936 1.15601505310536e-05
937 1.15538154510375e-05
938 1.15092773109726e-05
939 1.14956101326946e-05
940 1.14792729648538e-05
941 1.14482070843192e-05
942 1.14290441786125e-05
943 1.14193989459999e-05
944 1.13810355351074e-05
945 1.13772006857804e-05
946 1.13580177076855e-05
947 1.13355510735769e-05
948 1.13272608643911e-05
949 1.13105641383893e-05
950 1.12830643299588e-05
951 1.12774773459723e-05
952 1.12515260652923e-05
953 1.12587953791099e-05
954 1.12036769870372e-05
955 1.12809689745141e-05
956 1.11878323099113e-05
957 1.18019745276587e-05
958 1.25792977929695e-05
959 1.85194038059833e-05
960 1.55465720267367e-05
961 1.21946715949672e-05
962 1.13353340827071e-05
963 1.13051944381493e-05
964 1.1266927034681e-05
965 1.12468073183436e-05
966 1.12347678351554e-05
967 1.12031445453908e-05
968 1.11918631684471e-05
969 1.11751074525213e-05
970 1.11506099840231e-05
971 1.11315152386826e-05
972 1.11207409601022e-05
973 1.10944463880269e-05
974 1.1076552788758e-05
975 1.10617379398548e-05
976 1.10551042029616e-05
977 1.10157846753278e-05
978 1.10437439391209e-05
979 1.09790003621768e-05
980 1.11012507846286e-05
981 1.09925695126467e-05
982 1.20426191193568e-05
983 1.30531462764205e-05
984 1.81775939396833e-05
985 1.38752507314877e-05
986 1.16233711532354e-05
987 1.11077802972259e-05
988 1.10873847378734e-05
989 1.1057700640249e-05
990 1.10390977692276e-05
991 1.1010718800053e-05
992 1.09855343071175e-05
993 1.09805363672777e-05
994 1.09532973833204e-05
995 1.09373990637707e-05
996 1.09166731574462e-05
997 1.08999443368951e-05
998 1.08978017487882e-05
999 1.08517732162472e-05
1000 1.08758004142562e-05
1001 1.08026502310121e-05
1002 1.08684674016146e-05
1003 1.07629192391823e-05
1004 1.09606685214647e-05
1005 1.07963132220945e-05
1006 1.19360670477464e-05
1007 1.25708821472337e-05
1008 1.70222433994449e-05
1009 1.33723096976679e-05
1010 1.16570642805236e-05
1011 1.08820499257201e-05
1012 1.09244671296915e-05
1013 1.08538478837961e-05
1014 1.08519558568165e-05
1015 1.08136349692423e-05
1016 1.08183381495319e-05
1017 1.07719717803434e-05
1018 1.07762358465546e-05
1019 1.07429524160985e-05
1020 1.07373451958548e-05
1021 1.0714921267807e-05
1022 1.0700164388755e-05
1023 1.06665668595518e-05
1024 1.06796856016445e-05
1025 1.06408619802156e-05
1026 1.06599927871542e-05
1027 1.05870155890919e-05
1028 1.07264756030734e-05
1029 1.05644607557576e-05
1030 1.15193043866979e-05
1031 1.20815338839897e-05
1032 1.74445946803936e-05
1033 1.37880494432974e-05
1034 1.16091890882686e-05
1035 1.0728938993454e-05
1036 1.07590529180301e-05
1037 1.07079331123083e-05
1038 1.06969003532686e-05
1039 1.06635175172176e-05
1040 1.06577898113525e-05
1041 1.06340340124689e-05
1042 1.06187690924742e-05
1043 1.05823455154308e-05
1044 1.05820508358168e-05
1045 1.05588769401255e-05
1046 1.05446394293995e-05
1047 1.05220957828944e-05
1048 1.05217374007927e-05
1049 1.04786312604688e-05
1050 1.04952034687411e-05
1051 1.04348121299225e-05
1052 1.05103469931755e-05
1053 1.03766653922754e-05
1054 1.07435301283099e-05
1055 1.05799704479548e-05
1056 1.33807460999114e-05
1057 1.4061907645524e-05
1058 1.55647624293653e-05
1059 1.11738285959451e-05
1060 1.08393233767545e-05
1061 1.0549494951273e-05
1062 1.05765122065948e-05
1063 1.0536018177687e-05
1064 1.05091113529188e-05
1065 1.04972687146043e-05
1066 1.04688425525135e-05
1067 1.04721362137461e-05
1068 1.04278843546801e-05
1069 1.04400389308035e-05
1070 1.03902884553975e-05
1071 1.04031465655297e-05
1072 1.03598899149349e-05
1073 1.0383400598668e-05
1074 1.03201742682746e-05
1075 1.03698893493265e-05
1076 1.02640628678596e-05
1077 1.04373124090174e-05
1078 1.02093581122453e-05
1079 1.09652146917938e-05
1080 1.0830766690173e-05
1081 1.4749542353254e-05
1082 1.32327955145728e-05
1083 1.27631638728865e-05
1084 1.03908780602069e-05
1085 1.05875504410324e-05
1086 1.03588674481569e-05
1087 1.04045757354232e-05
1088 1.03348897373401e-05
1089 1.03588285444989e-05
1090 1.02904872465448e-05
1091 1.03320134744456e-05
1092 1.02617322974563e-05
1093 1.0294889786655e-05
1094 1.02238381741726e-05
1095 1.02718748328234e-05
1096 1.01789393688323e-05
1097 1.02906394551239e-05
1098 1.01123701989891e-05
1099 1.04028947418566e-05
1100 1.0092756375979e-05
1101 1.11246063717596e-05
1102 1.08067349928742e-05
1103 1.44374786804624e-05
1104 1.22857764093265e-05
1105 1.22738770842457e-05
1106 1.0238885542746e-05
1107 1.05340890541594e-05
1108 1.02031914922529e-05
1109 1.03319745988761e-05
1110 1.01911879499816e-05
1111 1.02685462153307e-05
1112 1.01705761998971e-05
1113 1.02168202974484e-05
1114 1.01332008410893e-05
1115 1.02010406927144e-05
1116 1.00820280266589e-05
1117 1.0216075796543e-05
1118 1.00247573802115e-05
1119 1.02936963399802e-05
1120 9.96592548019315e-06
1121 1.06932123613568e-05
1122 1.01884368954863e-05
1123 1.24942853705345e-05
1124 1.15562054610052e-05
1125 1.35551683041468e-05
1126 1.06157081717351e-05
1127 1.09168077401245e-05
1128 1.00569798263672e-05
1129 1.0326505282654e-05
1130 1.00722267700748e-05
1131 1.0205669386254e-05
1132 1.00471976125505e-05
1133 1.0168009517697e-05
1134 1.00153820085103e-05
1135 1.01325040976485e-05
1136 9.97636776756128e-06
1137 1.01266660273991e-05
1138 9.92141919531342e-06
1139 1.02321597998944e-05
1140 9.86043219231014e-06
1141 1.06220300982729e-05
1142 1.00315081095204e-05
1143 1.22080597698693e-05
1144 1.11505263100664e-05
1145 1.32837633688432e-05
1146 1.04882711806642e-05
1147 1.09534799239697e-05
1148 9.93042750740969e-06
1149 1.02932228656094e-05
1150 9.95250765123412e-06
1151 1.01561358069047e-05
1152 9.94246519314679e-06
1153 1.00839326632185e-05
1154 9.91458108745569e-06
1155 1.00720785489727e-05
1156 9.86767043109626e-06
1157 1.01038832482914e-05
1158 9.79686839486238e-06
1159 1.02423600522039e-05
1160 9.75426731120166e-06
1161 1.07096507273585e-05
1162 9.98880043767691e-06
1163 1.22301182363671e-05
1164 1.08382479286906e-05
1165 1.25316343351223e-05
1166 1.00634361613894e-05
1167 1.07458961651208e-05
1168 9.80380937931624e-06
1169 1.02303319603481e-05
1170 9.81738979644042e-06
1171 1.01180519846977e-05
1172 9.78304840804078e-06
1173 1.01089071190552e-05
1174 9.74045494084308e-06
1175 1.01620938789404e-05
1176 9.69555938001854e-06
1177 1.03458177403981e-05
1178 9.6864935146268e-06
1179 1.08771989004719e-05
1180 9.91312917753007e-06
1181 1.17285425204061e-05
1182 1.01372022049961e-05
1183 1.16864077251977e-05
1184 9.86314104156616e-06
1185 1.08139983783229e-05
1186 9.67448672639648e-06
1187 1.03330883689523e-05
1188 9.66473068197971e-06
1189 1.02180414217612e-05
1190 9.64273233761137e-06
1191 1.02276762835585e-05
1192 9.61358513751343e-06
1193 1.0333332918e-05
1194 9.60003121963648e-06
1195 1.06958461308926e-05
1196 9.69860184063531e-06
1197 1.12612534776124e-05
1198 9.85795850894089e-06
1199 1.13125174583395e-05
1200 9.70907895347928e-06
1201 1.07569609558711e-05
1202 9.58432814912591e-06
1203 1.04786781256472e-05
1204 9.5669757788075e-06
1205 1.03977113538045e-05
1206 9.55352014719857e-06
1207 1.04002877240328e-05
1208 9.54224698612727e-06
1209 1.04608885178603e-05
1210 9.54100650241152e-06
1211 1.06404852164843e-05
1212 9.56603954505031e-06
1213 1.08813030792465e-05
1214 9.6283374249051e-06
1215 1.09275942462572e-05
1216 9.56262725093637e-06
1217 1.06641516438577e-05
1218 9.50507970365333e-06
1219 1.04214050615781e-05
1220 9.47713399113148e-06
1221 1.03289188183586e-05
1222 9.46051051120556e-06
1223 1.04310951447584e-05
1224 9.47505053872089e-06
1225 1.05974674432519e-05
1226 9.48918896448436e-06
1227 1.07869593124477e-05
1228 9.50900132290489e-06
1229 1.09154102280185e-05
1230 9.53297983585788e-06
1231 1.07354115698177e-05
1232 9.4487118198483e-06
1233 1.04150905786415e-05
1234 9.41430328926796e-06
1235 1.02505967922717e-05
1236 9.40560861339979e-06
1237 1.02850377492469e-05
1238 9.39078624928591e-06
1239 1.04236780322786e-05
1240 9.39361406604444e-06
1241 1.05638072848135e-05
1242 9.40417923722681e-06
1243 1.07731374797693e-05
1244 9.45596527968107e-06
1245 1.07855301897386e-05
1246 9.39683077094688e-06
1247 1.0495178319303e-05
1248 9.35593022688419e-06
1249 1.02781806128327e-05
1250 9.34387924522717e-06
1251 1.02297647494032e-05
1252 9.32719351653954e-06
1253 1.03063167948836e-05
1254 9.3264838992857e-06
1255 1.04612906710599e-05
1256 9.32520633611933e-06
1257 1.05957899269038e-05
1258 9.33149092641194e-06
1259 1.06513492045268e-05
1260 9.32833794731192e-06
1261 1.06654247555849e-05
1262 9.33124059665502e-06
1263 1.04964974290134e-05
1264 9.28658350363154e-06
1265 1.0361099392231e-05
1266 9.27543119288998e-06
1267 1.03113354802398e-05
1268 9.26584757354743e-06
1269 1.03007077898454e-05
1270 9.2597198851907e-06
1271 1.02854313612788e-05
1272 9.23910377392811e-06
1273 1.02782994564343e-05
1274 9.23413458897571e-06
1275 1.03788518756032e-05
1276 9.24147704617351e-06
1277 1.04702959347058e-05
1278 9.22237531508774e-06
1279 1.03950231428174e-05
1280 9.21774804896192e-06
1281 1.03913871714045e-05
1282 9.20601957010714e-06
1283 1.04148972968066e-05
1284 9.19835512092071e-06
1285 1.04031549074124e-05
1286 9.20389964709223e-06
1287 1.03420105430629e-05
1288 9.18049036024993e-06
1289 1.01905501252997e-05
1290 9.17213651019377e-06
1291 1.01823609215401e-05
1292 9.15856880656651e-06
1293 1.02385542397609e-05
1294 9.15521061251034e-06
1295 1.0316289859702e-05
1296 9.14041352717554e-06
1297 1.0263615066175e-05
1298 9.13443109884504e-06
1299 1.03537571265466e-05
1300 9.13112842793051e-06
1301 1.03980480313792e-05
1302 9.12114537499065e-06
1303 1.04281752485447e-05
1304 9.13193339013851e-06
1305 1.03254342019854e-05
1306 9.10747865323636e-06
1307 1.02583985243099e-05
1308 9.09329021214766e-06
1309 1.02422037482341e-05
1310 9.08628113649446e-06
1311 1.01835511053805e-05
1312 9.08799323706688e-06
1313 1.01402181684751e-05
1314 9.07596764387897e-06
1315 1.01107762379149e-05
1316 9.06392834132408e-06
1317 1.00340175575031e-05
1318 9.05779041238119e-06
1319 1.00806843001955e-05
1320 9.0375472848736e-06
1321 1.02311121225052e-05
1322 9.04471100338089e-06
1323 1.03020153201516e-05
1324 9.0234608330908e-06
1325 1.02814881451918e-05
1326 9.02473573172191e-06
1327 1.02277946356644e-05
1328 9.01362732108613e-06
1329 1.01873530512497e-05
1330 9.0079510394192e-06
1331 1.01376755754501e-05
1332 9.00233801237604e-06
1333 1.0129960979044e-05
1334 8.9978769336474e-06
1335 1.00546152476122e-05
1336 8.98639128210643e-06
1337 9.98672719643157e-06
1338 8.97983609193798e-06
1339 1.00035686103617e-05
1340 8.96423971918914e-06
1341 1.00684492371217e-05
1342 8.96019503426881e-06
1343 1.01284864543194e-05
1344 8.94071044255096e-06
1345 1.01359624827824e-05
1346 8.93564165171945e-06
1347 1.0162406646308e-05
1348 8.92751744885345e-06
1349 1.01490716891695e-05
1350 8.92719302569844e-06
1351 1.0110972900268e-05
1352 8.91911358280595e-06
1353 1.0067571925787e-05
1354 8.91536829517037e-06
1355 1.00478533148207e-05
1356 8.911101609832e-06
1357 9.97516308265389e-06
1358 8.91340907915872e-06
1359 9.95179715901351e-06
1360 8.88690785871837e-06
1361 1.002553131213e-05
1362 8.88247457520031e-06
1363 1.00079299023559e-05
1364 8.87593141740695e-06
1365 1.00376614260433e-05
1366 8.8693074647983e-06
1367 1.0004636857075e-05
1368 8.86487875495501e-06
1369 9.99591245653964e-06
1370 8.84309597903865e-06
1371 1.00271767213833e-05
1372 8.84271677470405e-06
1373 9.94335181603301e-06
1374 8.84500868469917e-06
1375 9.92382152398275e-06
1376 8.83566250609036e-06
1377 9.89207344193055e-06
1378 8.83801886908397e-06
1379 9.80770491443295e-06
1380 8.82675860014448e-06
1381 9.84869267572908e-06
1382 8.80889794518946e-06
1383 9.89297891063057e-06
1384 8.79796511177311e-06
1385 9.87115270856265e-06
1386 8.78805134607585e-06
1387 9.94667358977797e-06
1388 8.77512749719589e-06
1389 9.95772616729784e-06
1390 8.7740093199784e-06
1391 9.87891986681255e-06
1392 8.76953424577831e-06
1393 9.86929352198018e-06
1394 8.76511815206538e-06
1395 9.89067169876456e-06
1396 8.75570802516545e-06
1397 9.8560794270508e-06
1398 8.76363356316912e-06
1399 9.7449823407425e-06
1400 8.75820010937112e-06
1401 9.79696025082966e-06
1402 8.73850477023375e-06
1403 9.80135616324151e-06
1404 8.73395394274112e-06
1405 9.85594242131071e-06
1406 8.70143058362416e-06
1407 9.9507995590109e-06
1408 8.70278217912634e-06
1409 9.91952048634559e-06
1410 8.70844304279306e-06
1411 9.83027209655596e-06
1412 8.71420177450677e-06
1413 9.75058233509962e-06
1414 8.72041643529275e-06
1415 9.7390351623261e-06
1416 8.71152504466899e-06
1417 9.69086589985846e-06
1418 8.73514880328496e-06
1419 9.67484000069518e-06
1420 8.68919970342663e-06
1421 9.74035513556792e-06
1422 8.6897391513574e-06
1423 9.74752705795723e-06
1424 8.68079815319334e-06
1425 9.70000193745157e-06
1426 8.70143540976365e-06
1427 9.69222668245617e-06
1428 8.65117197867971e-06
1429 9.7759976260825e-06
1430 8.65981455300258e-06
1431 9.72035381874292e-06
1432 8.66387205999963e-06
1433 9.69888888846526e-06
1434 8.65914743086726e-06
1435 9.63254681762837e-06
1436 8.69194860886324e-06
1437 9.53230306888209e-06
1438 8.68969554201904e-06
1439 9.54138959763373e-06
1440 8.66255875564548e-06
1441 9.56242749339253e-06
1442 8.66454628523083e-06
1443 9.62070424792572e-06
1444 8.60786110579159e-06
1445 9.70758889162759e-06
1446 8.60350833520318e-06
1447 9.69520554416903e-06
1448 8.60394501733808e-06
1449 9.67346085556375e-06
1450 8.61141937524845e-06
1451 9.66339290386919e-06
1452 8.59389443197411e-06
1453 9.61509784325187e-06
1454 8.64028124880623e-06
1455 9.49774269942694e-06
1456 8.63195663980409e-06
1457 9.53224542055153e-06
1458 8.60685348436085e-06
1459 9.50706736957407e-06
1460 8.64228681363155e-06
1461 9.41478512328509e-06
1462 8.62713347749011e-06
1463 9.48664802968846e-06
1464 8.57959738964631e-06
1465 9.51487987504995e-06
1466 8.59496707161789e-06
1467 9.56530330431526e-06
1468 8.52513643911923e-06
1469 9.6494586802609e-06
1470 8.53969226566331e-06
1471 9.6093993926516e-06
1472 8.55048913039891e-06
1473 9.55914583816719e-06
1474 8.5652706065531e-06
1475 9.53482800736261e-06
1476 8.55050199521923e-06
1477 9.48457314275952e-06
1478 8.61713947497567e-06
1479 9.32672817122171e-06
1480 8.62985505378777e-06
1481 9.34521009843259e-06
1482 8.58197388797777e-06
1483 9.33851255774387e-06
1484 8.62849482208272e-06
1485 9.257216440961e-06
1486 8.61871060708896e-06
1487 9.29728100762528e-06
1488 8.55825992096548e-06
1489 9.34316261075319e-06
1490 8.58936226599827e-06
1491 9.27359258573457e-06
1492 8.5623203558427e-06
1493 9.37031225178853e-06
1494 8.49947224335068e-06
1495 9.41004726706662e-06
1496 8.51182004579787e-06
1497 9.40034264423906e-06
1498 8.50115808315177e-06
1499 9.43273336839745e-06
1500 8.4808358329358e-06
1501 9.51198521892405e-06
1502 8.44740823779944e-06
1503 9.4881303470129e-06
1504 8.49930718216374e-06
1505 9.43098873046333e-06
1506 8.48017473675799e-06
1507 9.38095460462085e-06
1508 8.55128976062769e-06
1509 9.25027600606754e-06
1510 8.56152291728307e-06
1511 9.23260437118989e-06
1512 8.55294581136867e-06
1513 9.27538746586798e-06
1514 8.50767760329774e-06
1515 9.28072855416184e-06
1516 8.56882977029461e-06
1517 9.16718565885377e-06
1518 8.57270222687223e-06
1519 9.18470259159143e-06
1520 8.53722543392443e-06
1521 9.2638284689972e-06
1522 8.48790451524195e-06
1523 9.26377867749295e-06
1524 8.55083781459154e-06
1525 9.16490846958329e-06
1526 8.54238085590264e-06
1527 9.17992602150752e-06
1528 8.52088557534714e-06
1529 9.24761809423247e-06
1530 8.47448912399962e-06
1531 9.24580395356145e-06
1532 8.54500718339146e-06
1533 9.13712739308847e-06
1534 8.53688806101705e-06
1535 9.15420275177103e-06
1536 8.51817478741967e-06
1537 9.230766732371e-06
1538 8.47223837774358e-06
1539 9.20157915651476e-06
1540 8.55770638785724e-06
1541 9.07981518538659e-06
1542 8.55581196013144e-06
1543 9.10066497122886e-06
1544 8.53871021733177e-06
1545 9.16579411647245e-06
1546 8.47993090280497e-06
1547 9.18474842503958e-06
1548 8.54978781794635e-06
1549 9.04781674826438e-06
1550 8.55376355024529e-06
1551 9.08661095633256e-06
1552 8.51866857776162e-06
1553 9.16831892028291e-06
1554 8.46582681524044e-06
1555 9.1431860615443e-06
1556 8.56108003732814e-06
1557 9.02167901112305e-06
1558 8.55081239825584e-06
1559 9.07377900349893e-06
1560 8.52964914233034e-06
1561 9.12326428215149e-06
1562 8.4820961528953e-06
1563 9.09880308347377e-06
1564 8.58220939042864e-06
1565 9.00368670636587e-06
1566 8.57673653270608e-06
1567 9.02814949266251e-06
1568 8.5590676400793e-06
1569 9.06533077105198e-06
1570 8.52322613686951e-06
1571 9.03865372570145e-06
1572 8.62891285602352e-06
1573 8.91370223365939e-06
1574 8.64745074480311e-06
1575 8.93999397472145e-06
1576 8.61748142561014e-06
1577 9.04926025596664e-06
1578 8.5245171358439e-06
1579 9.04364564480176e-06
1580 8.63656744376495e-06
1581 8.92419642001263e-06
1582 8.65561947871463e-06
1583 8.96226063806793e-06
1584 8.61839598054193e-06
1585 9.04468913393019e-06
1586 8.55670512317497e-06
1587 9.00982732326971e-06
1588 8.67829994338765e-06
1589 8.90351411791945e-06
1590 8.72508924021353e-06
1591 8.83794219835865e-06
1592 8.86169747427168e-06
1593 8.70463394536669e-06
1594 8.8789739406403e-06
1595 8.83138907137315e-06
1596 8.75335434724622e-06
1597 8.90290584393272e-06
1598 8.74489760938069e-06
1599 8.97638871466544e-06
1600 8.66911895275324e-06
1601 8.94850258675861e-06
1602 8.8177044328952e-06
1603 8.82462492046709e-06
1604 8.95806578027347e-06
1605 8.70294372906644e-06
1606 8.96735991451347e-06
1607 8.85963181285199e-06
1608 8.81404467079072e-06
1609 9.03383721040374e-06
1610 8.70451796808336e-06
1611 8.99735674858571e-06
1612 8.89394329306459e-06
1613 8.8846542469212e-06
1614 9.00690151767192e-06
1615 8.78560626121905e-06
1616 9.0016539299187e-06
1617 8.95558166752464e-06
1618 8.88039203705837e-06
1619 9.10322398506302e-06
1620 8.80900975391841e-06
1621 9.02417749837348e-06
1622 9.01542128189936e-06
1623 8.91919176443379e-06
1624 9.062641794233e-06
1625 8.94541090612488e-06
1626 9.09197470166934e-06
1627 8.95306532128215e-06
1628 9.17051124177393e-06
1629 8.90476312420674e-06
1630 9.07627393792465e-06
1631 9.10425804334647e-06
1632 9.02773988170624e-06
1633 9.15518179961383e-06
1634 9.01536713793227e-06
1635 9.18182809600543e-06
1636 9.0510749872208e-06
1637 9.19095645568468e-06
1638 9.07771687497227e-06
1639 9.20335814375006e-06
1640 9.08993192860752e-06
1641 9.23434801847201e-06
1642 9.09897108186675e-06
1643 9.25162276821379e-06
1644 9.11717344304064e-06
1645 9.26205715767736e-06
1646 9.13141966712994e-06
1647 9.257117281114e-06
1648 9.1557255371133e-06
1649 9.26894354214003e-06
1650 9.15193165207384e-06
1651 9.26872885520513e-06
1652 9.17774385900127e-06
1653 9.26251533511779e-06
1654 9.16998308397865e-06
1655 9.26334173217969e-06
1656 9.1805743917539e-06
1657 9.26624205649373e-06
1658 9.15753134000941e-06
1659 9.25542018165304e-06
1660 9.17390425742015e-06
1661 9.23985042022846e-06
1662 9.15111081717379e-06
1663 9.23630268523379e-06
1664 9.15877094997963e-06
1665 9.18784206715983e-06
1666 9.27667775341146e-06
1667 9.06107830644931e-06
1668 9.14287223424681e-06
1669 9.16385929622399e-06
1670 9.17663481536657e-06
1671 9.08904743857075e-06
1672 9.14973973092925e-06
1673 9.1941297968301e-06
1674 8.99665021053542e-06
1675 9.07535679794957e-06
1676 9.09749176281061e-06
1677 9.10287078259575e-06
1678 9.03103938793004e-06
1679 9.07218203355376e-06
1680 9.0896637406912e-06
1681 9.00464290332081e-06
1682 9.0370708835108e-06
1683 8.97806128363765e-06
1684 9.01420074489456e-06
1685 9.04082937508432e-06
1686 8.9252550006802e-06
1687 9.02822921905466e-06
1688 8.91918101236788e-06
1689 8.93855700068169e-06
1690 8.98000509103047e-06
1691 8.87328093407413e-06
1692 8.92374002187113e-06
1693 8.91137589270619e-06
1694 8.95478995438515e-06
1695 8.78455575770154e-06
1696 8.87026530554946e-06
1697 8.85100568831509e-06
1698 8.85626787894278e-06
1699 8.87166360014069e-06
1700 8.77839214402654e-06
1701 8.82563097420741e-06
1702 8.83854249023042e-06
1703 8.74888571811283e-06
1704 8.78200483056535e-06
1705 8.81613044473317e-06
1706 8.72505078247654e-06
1707 8.76374472258412e-06
1708 8.77192220982703e-06
1709 8.69636035721877e-06
1710 8.72926394140938e-06
1711 8.75894943647149e-06
1712 8.66339734434396e-06
1713 8.69706083411748e-06
1714 8.72837496795897e-06
1715 8.6446274387697e-06
1716 8.66780024572078e-06
1717 8.70509511580186e-06
1718 8.59557657539867e-06
1719 8.6489613475571e-06
1720 8.66323833892402e-06
1721 8.58554865457162e-06
1722 8.61496624654512e-06
1723 8.60151401443909e-06
1724 8.62639222865003e-06
1725 8.54502809544133e-06
1726 8.58704177325587e-06
1727 8.5996048747039e-06
1728 8.51292430831263e-06
1729 8.54911106384915e-06
1730 8.56142853539188e-06
1731 8.47255922681311e-06
1732 8.50906938110185e-06
1733 8.53526215299771e-06
1734 8.41351072128349e-06
1735 8.50327494639203e-06
1736 8.48126503200097e-06
1737 8.40851161754141e-06
1738 8.46240292923239e-06
1739 8.46324713588142e-06
1740 8.36915384250236e-06
1741 8.42294155928291e-06
1742 8.43189988780635e-06
1743 8.35048948988337e-06
1744 8.41050882688377e-06
1745 8.40188938999731e-06
1746 8.31500718700973e-06
1747 8.3870207046588e-06
1748 8.37733550251851e-06
1749 8.27440514944433e-06
1750 8.36863024766821e-06
1751 8.34960109791227e-06
1752 8.26547211818518e-06
1753 8.33008890149811e-06
1754 8.32229584218913e-06
1755 8.23061508026246e-06
1756 8.32824304530799e-06
1757 8.2948991187437e-06
1758 8.23243394904027e-06
1759 8.27890779364626e-06
1760 8.29149616721203e-06
1761 8.19444227456723e-06
1762 8.27019009269625e-06
1763 8.25390558051309e-06
1764 8.17247676321298e-06
1765 8.24820489658817e-06
1766 8.24786998881999e-06
1767 8.1335685627737e-06
1768 8.23980122677259e-06
1769 8.20231723014375e-06
1770 8.1461314599407e-06
1771 8.18646398192957e-06
1772 8.20136026896678e-06
1773 8.09640512794418e-06
1774 8.20546207364048e-06
1775 8.17512529144304e-06
1776 8.08456858525375e-06
1777 8.16594812941274e-06
1778 8.15238642321026e-06
1779 8.07682150327782e-06
1780 8.15826456096458e-06
1781 8.11918625759356e-06
1782 8.0566055969955e-06
1783 8.12523835130863e-06
1784 8.10317834615271e-06
1785 8.02744206529482e-06
1786 8.1250410913225e-06
1787 8.07159175342242e-06
1788 8.02471626121237e-06
1789 8.09389980632735e-06
1790 8.01048735410426e-06
1791 8.10706244547266e-06
1792 7.94740321036214e-06
1793 8.09635399751096e-06
1794 8.02375960895496e-06
1795 8.00597396888314e-06
1796 8.04625367534229e-06
1797 7.98714758987185e-06
1798 8.05234281309186e-06
1799 7.9217952793087e-06
1800 8.04818927857553e-06
1801 7.92722276238278e-06
1802 8.06648988771475e-06
1803 7.84685722965417e-06
1804 8.09201452706132e-06
1805 7.89323017624799e-06
1806 7.96248084111717e-06
1807 7.95515682949777e-06
1808 7.91442787040353e-06
1809 8.01704672764858e-06
1810 7.80082174234309e-06
1811 8.07458979923403e-06
1812 7.78425932501658e-06
1813 8.08559253484553e-06
1814 7.72464476073154e-06
1815 8.10100000631087e-06
1816 7.73486779565014e-06
1817 8.11452332799556e-06
1818 7.66692263842739e-06
1819 8.12785865639265e-06
1820 7.71187735393308e-06
1821 8.01039567810413e-06
1822 7.77032348692197e-06
1823 7.9229157723959e-06
1824 7.80143695416902e-06
1825 7.95779058654356e-06
1826 7.68872155176847e-06
1827 8.0314409172666e-06
1828 7.63943130005496e-06
1829 8.1220028310125e-06
1830 7.55052372330978e-06
1831 8.1796174505322e-06
1832 7.57040589005609e-06
1833 8.09716414590378e-06
1834 7.59374126052537e-06
1835 8.05453393115663e-06
1836 7.59593955901305e-06
1837 8.08436416044334e-06
1838 7.52134061970944e-06
1839 8.16370912304354e-06
1840 7.52576945017847e-06
1841 8.11900748998173e-06
1842 7.51643975205329e-06
1843 8.15852479651991e-06
1844 7.45527058515494e-06
1845 8.26361728178959e-06
1846 7.4374372111552e-06
1847 8.22782215642581e-06
1848 7.46284667307862e-06
1849 8.15517491192796e-06
1850 7.49226302776407e-06
1851 8.09825547593013e-06
1852 7.49483702805476e-06
1853 8.12558462676138e-06
1854 7.43489377719175e-06
1855 8.16930158004103e-06
1856 7.46511181570764e-06
1857 8.11199120615003e-06
1858 7.48176049419902e-06
1859 8.13479096134495e-06
1860 7.41051892716449e-06
1861 8.24085329037549e-06
1862 7.38665004812455e-06
1863 8.32065646366731e-06
1864 7.34655615980406e-06
1865 8.33906053132605e-06
1866 7.39922461606701e-06
1867 8.23969492486087e-06
1868 7.38814297290835e-06
1869 8.22844734477135e-06
1870 7.47449456989235e-06
1871 8.05187993352385e-06
1872 7.55006847014128e-06
1873 8.00887108715065e-06
1874 7.5219892442524e-06
1875 8.02737799221465e-06
1876 7.55835018673157e-06
1877 8.00323423927329e-06
1878 7.50181174918874e-06
1879 8.05634956979606e-06
1880 7.50557956602504e-06
1881 8.08056383039268e-06
1882 7.45194230977742e-06
1883 8.13474869348907e-06
1884 7.47063707007278e-06
1885 8.1615918516853e-06
1886 7.43867198582659e-06
1887 8.12198120969665e-06
1888 7.53830104688324e-06
1889 7.96338592734536e-06
1890 7.62979300589217e-06
1891 7.85575984268982e-06
1892 7.69310774051402e-06
1893 7.83024822054168e-06
1894 7.6835700556499e-06
1895 7.90183799853628e-06
1896 7.58134500383756e-06
1897 7.96338234215765e-06
1898 7.57657632444175e-06
1899 8.01729962046638e-06
1900 7.50860650788887e-06
1901 8.02895469059761e-06
1902 7.54835619842797e-06
1903 8.04141259241886e-06
1904 7.50846984037823e-06
1905 8.0764121723198e-06
1906 7.51239709007034e-06
1907 8.01378873743763e-06
1908 7.62622663280643e-06
1909 7.88241440990989e-06
1910 7.70761579083068e-06
1911 7.84594100061708e-06
1912 7.70801096161566e-06
1913 7.88299650983371e-06
1914 7.6473851395864e-06
1915 7.94875893928015e-06
1916 7.62241911472783e-06
1917 8.00742884127237e-06
1918 7.58364071107298e-06
1919 7.9873434880029e-06
1920 7.64783756884713e-06
1921 8.0043250394457e-06
1922 7.60738921451454e-06
1923 7.99131776513251e-06
1924 7.67986854310898e-06
1925 7.98962900827593e-06
1926 7.6411539212673e-06
1927 7.97980356714856e-06
1928 7.70575273828511e-06
1929 7.98836086257637e-06
1930 7.67202961676761e-06
1931 7.98006134927753e-06
1932 7.7424989848307e-06
1933 8.00840672166014e-06
1934 7.69430704711249e-06
1935 7.9880419378564e-06
1936 7.77852393274436e-06
1937 8.0183062815542e-06
1938 7.73086254052258e-06
1939 7.99865561867952e-06
1940 7.91572758895853e-06
1941 7.83378903113086e-06
1942 7.96400136665598e-06
1943 7.87843340871452e-06
1944 7.99474119544952e-06
1945 7.84524126640207e-06
1946 8.01332013522504e-06
1947 7.90942431971864e-06
1948 8.02870839167191e-06
1949 7.86704619321066e-06
1950 8.02320864806694e-06
1951 8.02370308766731e-06
1952 7.91718637449534e-06
1953 8.03167113877512e-06
1954 8.08366874111011e-06
1955 7.90878072925993e-06
1956 8.1199720107783e-06
1957 7.9716385401607e-06
1958 8.12157769558564e-06
1959 7.96851878703819e-06
1960 8.10977442650129e-06
1961 8.10442378612608e-06
1962 8.00447243332059e-06
1963 8.12411395678669e-06
1964 8.13819707073948e-06
1965 8.01613916723287e-06
1966 8.13047550163315e-06
1967 8.14787925229599e-06
1968 8.04102306073418e-06
1969 8.13178463626718e-06
1970 8.18351857384547e-06
1971 8.03500980567362e-06
1972 8.15194468339886e-06
1973 8.1819027461516e-06
1974 8.05930528230414e-06
1975 8.14520875763725e-06
1976 8.18707669758556e-06
1977 8.10874073842172e-06
1978 8.17450893780824e-06
1979 8.08066197649504e-06
1980 8.13626821299929e-06
1981 8.16478588849812e-06
1982 8.12282144402321e-06
1983 8.17089533822202e-06
1984 8.07136447567025e-06
1985 8.12097037800052e-06
1986 8.13370709507044e-06
1987 8.09171115290219e-06
1988 8.14912259589073e-06
1989 8.08386364131719e-06
1990 8.12076884371127e-06
1991 8.06942501552221e-06
1992 8.1288967463955e-06
1993 8.02886450190776e-06
1994 8.08440517358067e-06
1995 8.06617652460151e-06
1996 8.08848821970409e-06
1997 7.99913966126331e-06
1998 8.04217231537008e-06
1999 8.04749222610424e-06
};
\addlegendentry{Train}
\addplot [semithick, black]
table {%
0 0.00943661201745272
1 0.00942239630967379
2 0.00940357614308596
3 0.00937668606638908
4 0.00933848507702351
5 0.00928496569395065
6 0.00920651480555534
7 0.00909615680575371
8 0.00894130021333694
9 0.00874812714755535
10 0.00851717963814735
11 0.00825926940888166
12 0.00798802729696035
13 0.00771354977041483
14 0.00743501028046012
15 0.00715617649257183
16 0.00686991028487682
17 0.00658787367865443
18 0.00629624957218766
19 0.00602078111842275
20 0.00574027001857758
21 0.00546252215281129
22 0.00519831012934446
23 0.00493174511939287
24 0.0046843197196722
25 0.00445051258429885
26 0.00423746230080724
27 0.00403901794925332
28 0.00385068892501295
29 0.00367525359615684
30 0.00351149006746709
31 0.00335577013902366
32 0.00320859672501683
33 0.00307445111684501
34 0.00295338057912886
35 0.00283679366111755
36 0.00272457138635218
37 0.00261569209396839
38 0.00251163821667433
39 0.00241464679129422
40 0.0023265762720257
41 0.00224687601439655
42 0.00217237742617726
43 0.00210257479920983
44 0.00203435402363539
45 0.00197057751938701
46 0.00191026087850332
47 0.00185492017772049
48 0.00180252047721297
49 0.00175080099143088
50 0.00170100585091859
51 0.00165472086519003
52 0.00161117070820183
53 0.00157083873637021
54 0.00153412681538612
55 0.00150054600089788
56 0.00146922469139099
57 0.00144009699579328
58 0.00141233846079558
59 0.00138531054835767
60 0.00135718251112849
61 0.00132987077813596
62 0.00130359223112464
63 0.00127923279069364
64 0.00125620432663709
65 0.00123390217777342
66 0.00121203495655209
67 0.00119054724927992
68 0.0011698353337124
69 0.00114968919660896
70 0.00113088986836374
71 0.00111308658961207
72 0.00109619251452386
73 0.00108031369745731
74 0.00106489297468215
75 0.00104946934152395
76 0.00103375967592001
77 0.00101849192287773
78 0.00100314000155777
79 0.000987731153145432
80 0.000971979694440961
81 0.000956856529228389
82 0.000942369224503636
83 0.000928669527638704
84 0.000915935321245342
85 0.000903968990314752
86 0.000892732001375407
87 0.000882128486409783
88 0.000872031494509429
89 0.000862377171870321
90 0.000853047531563789
91 0.000843970570713282
92 0.000835289363749325
93 0.000826922012493014
94 0.000819034175947309
95 0.000811459263786674
96 0.00080408836947754
97 0.00079667882528156
98 0.000789549259934574
99 0.000782741699367762
100 0.000776126864366233
101 0.000769327976740897
102 0.000762491894420236
103 0.000756017456296831
104 0.000749795697629452
105 0.000743865442927927
106 0.000738157948944718
107 0.000732632353901863
108 0.000727251113858074
109 0.00072157388785854
110 0.000715739442966878
111 0.00071029708487913
112 0.000705097860191017
113 0.00070015829987824
114 0.000695213559083641
115 0.000690597342327237
116 0.000686250976286829
117 0.00068213272606954
118 0.00067822216078639
119 0.00067438097903505
120 0.000670555571559817
121 0.000666719512082636
122 0.000662947888486087
123 0.000658532604575157
124 0.000654341245535761
125 0.000650528643745929
126 0.000647044740617275
127 0.000643608625978231
128 0.000640327227301896
129 0.000637236284092069
130 0.000634265074040741
131 0.000631300266832113
132 0.000628431036602706
133 0.00062574673211202
134 0.000623163185082376
135 0.000620606762822717
136 0.000618219666648656
137 0.000615802768152207
138 0.000613342097494751
139 0.000610965013038367
140 0.000608459347859025
141 0.000606094719842076
142 0.000603901979047805
143 0.000601673847995698
144 0.000599614344537258
145 0.000597559032030404
146 0.000595511519350111
147 0.000593592703808099
148 0.000591759686358273
149 0.000589894305448979
150 0.00058807956520468
151 0.000586330017540604
152 0.000584654975682497
153 0.000582940934691578
154 0.000581320724450052
155 0.000579773681238294
156 0.000578237639274448
157 0.00057675241259858
158 0.000575304089579731
159 0.000573918805457652
160 0.000572576536796987
161 0.000571152195334435
162 0.000569881172850728
163 0.000568545015994459
164 0.000567348615732044
165 0.000566137197893113
166 0.000564990972634405
167 0.000563750567380339
168 0.000562674715183675
169 0.000561534892767668
170 0.000560431100893766
171 0.00055941502796486
172 0.000558358442503959
173 0.000557412160560489
174 0.000556385144591331
175 0.000555467617232352
176 0.000554491416551173
177 0.000553515448700637
178 0.000552575220353901
179 0.000551584118511528
180 0.000550821190699935
181 0.00054984400048852
182 0.00054899969836697
183 0.000548104580957443
184 0.000547166855540127
185 0.000546344148460776
186 0.000545429298654199
187 0.000544583017472178
188 0.00054377451306209
189 0.000542843015864491
190 0.000542074558325112
191 0.000541145971510559
192 0.000540308537892997
193 0.000539441709406674
194 0.000538607477210462
195 0.000537655607331544
196 0.000536857231054455
197 0.000535890576429665
198 0.000535033410415053
199 0.000534143298864365
200 0.000533169426489621
201 0.000532222620677203
202 0.000531272788066417
203 0.000530342920683324
204 0.000529353797901422
205 0.000528425211086869
206 0.000527366239111871
207 0.000526419025845826
208 0.000525480718351901
209 0.000524508533999324
210 0.000523413880728185
211 0.000522506423294544
212 0.000521359906997532
213 0.000520345929544419
214 0.000519207213073969
215 0.00051812594756484
216 0.00051702419295907
217 0.000515872030518949
218 0.000514687329996377
219 0.000513492617756128
220 0.000512312690261751
221 0.000511037360411137
222 0.000509744510054588
223 0.000508594559505582
224 0.00050725694745779
225 0.000505872070789337
226 0.00050460931379348
227 0.00050325394840911
228 0.000501893751788884
229 0.000500526395626366
230 0.000499150948598981
231 0.000497537723276764
232 0.000496287306305021
233 0.000494492123834789
234 0.000493067491333932
235 0.000491067592520267
236 0.000489939760882407
237 0.000487680052174255
238 0.000486989854834974
239 0.000484112912090495
240 0.000484476069686934
241 0.000480418762890622
242 0.000482373987324536
243 0.000476802350021899
244 0.000478223635582253
245 0.000473256135592237
246 0.00047318241558969
247 0.000469636375783011
248 0.000468626763904467
249 0.000465813209302723
250 0.000464291428215802
251 0.000461572635686025
252 0.000459938892163336
253 0.000457140587968752
254 0.000455258297733963
255 0.000452535430667922
256 0.000450294144684449
257 0.00044745349441655
258 0.000445137586211786
259 0.000442199787357822
260 0.000439805793575943
261 0.000436755828559399
262 0.000434393208706751
263 0.0004309521173127
264 0.000428727042162791
265 0.000424934754846618
266 0.000422909244662151
267 0.000418572715716437
268 0.000416990515077487
269 0.000412191962823272
270 0.000410861393902451
271 0.000405532075092196
272 0.000404393242206424
273 0.000398589356336743
274 0.00039744918467477
275 0.000391651119571179
276 0.000389771652407944
277 0.000384523300454021
278 0.000381964258849621
279 0.000377130316337571
280 0.00037410223740153
281 0.000368816108675674
282 0.00036516398540698
283 0.000360393692972139
284 0.000356696284143254
285 0.000351236667484045
286 0.000347650260664523
287 0.000342429208103567
288 0.000338859565090388
289 0.000333353120367974
290 0.000329880916979164
291 0.000324183696648106
292 0.000320734456181526
293 0.000314911652822047
294 0.000311536103254184
295 0.000305507419398054
296 0.000302298896713182
297 0.000296316284220666
298 0.000293040036922321
299 0.000286846014205366
300 0.000283816858427599
301 0.000277685321634635
302 0.000274472986347973
303 0.00026835675816983
304 0.000265241629676893
305 0.00025895080761984
306 0.000255751307122409
307 0.000249624194111675
308 0.000246375566348433
309 0.000240326116909273
310 0.00023723830236122
311 0.000231361162150279
312 0.000228199525736272
313 0.000222509494051337
314 0.000219472261960618
315 0.000214028448681347
316 0.000210884012631141
317 0.000205622432986274
318 0.000202581883058883
319 0.000197623638086952
320 0.000194727064808831
321 0.000189762067748234
322 0.000187267272849567
323 0.000182438496267423
324 0.000180083923623897
325 0.000175305307493545
326 0.000173305787029676
327 0.000168740429216996
328 0.000166802899912
329 0.000162535070558079
330 0.000160685885930434
331 0.000156583351781592
332 0.000154705558088608
333 0.000150897540152073
334 0.00014911312609911
335 0.000145577767398208
336 0.000143950455822051
337 0.00014055211795494
338 0.000138948133098893
339 0.000135818234412
340 0.000134317553602159
341 0.000131326916744001
342 0.000130093147163279
343 0.00012724322732538
344 0.000126113402075134
345 0.000123324294690974
346 0.000122380210086703
347 0.000119764161354396
348 0.00011877190991072
349 0.000116369352326728
350 0.000115524395368993
351 0.000113198038889095
352 0.000112515052023809
353 0.000110278306237888
354 0.000109760141640436
355 0.000107424086309038
356 0.00010694436787162
357 0.000104396589449607
358 0.00010402958287159
359 0.000102073288871907
360 0.000101771613117307
361 9.98340547084808e-05
362 9.94208821794018e-05
363 9.77037052507512e-05
364 9.74069116637111e-05
365 9.56197691266425e-05
366 9.52845948631875e-05
367 9.35897551244125e-05
368 9.32637194637209e-05
369 9.17201250558719e-05
370 9.14611882762983e-05
371 8.99571605259553e-05
372 8.97522331797518e-05
373 8.83741231518798e-05
374 8.81273590493947e-05
375 8.68674542289227e-05
376 8.64637768245302e-05
377 8.53774618008174e-05
378 8.49838688736781e-05
379 8.39528511278331e-05
380 8.36162871564738e-05
381 8.25572278699838e-05
382 8.22636575321667e-05
383 8.12615107861347e-05
384 8.10632554930635e-05
385 7.99473127699457e-05
386 7.99598929006606e-05
387 7.87094249972142e-05
388 7.89667901699431e-05
389 7.75744265411049e-05
390 7.80625632614829e-05
391 7.64171563787386e-05
392 7.73318388382904e-05
393 7.54453139961697e-05
394 7.67580349929631e-05
395 7.45423094485886e-05
396 7.60675684432499e-05
397 7.36586807761341e-05
398 7.52300111344084e-05
399 7.27782244211994e-05
400 7.40834875614382e-05
401 7.20079478924163e-05
402 7.2958500823006e-05
403 7.12480614311062e-05
404 7.19484087312594e-05
405 7.04040212440304e-05
406 7.09698215359822e-05
407 6.96691786288284e-05
408 7.02856777934358e-05
409 6.89227090333588e-05
410 6.95757116773166e-05
411 6.82012687320821e-05
412 6.90830638632178e-05
413 6.75063565722667e-05
414 6.87672581989318e-05
415 6.68523061904125e-05
416 6.86139173922129e-05
417 6.63176033413038e-05
418 6.82101817801595e-05
419 6.57206765026785e-05
420 6.74790571792983e-05
421 6.51389200356789e-05
422 6.65063198539428e-05
423 6.45734762656502e-05
424 6.57650016364641e-05
425 6.40488942735828e-05
426 6.49675275781192e-05
427 6.34715033811517e-05
428 6.43629828118719e-05
429 6.29484275123104e-05
430 6.39919162495062e-05
431 6.24013264314272e-05
432 6.37779739918187e-05
433 6.19302882114425e-05
434 6.35993637843058e-05
435 6.14926830166951e-05
436 6.34294483461417e-05
437 6.10743736615404e-05
438 6.30826834822074e-05
439 6.05574532528408e-05
440 6.24491658527404e-05
441 6.02011496084742e-05
442 6.16835968685336e-05
443 5.97661128267646e-05
444 6.11630603088997e-05
445 5.93524018768221e-05
446 6.07469664828386e-05
447 5.89919618505519e-05
448 6.04115812166128e-05
449 5.85512752877548e-05
450 6.02963118581101e-05
451 5.82751272304449e-05
452 6.03163243795279e-05
453 5.79424486204516e-05
454 6.01909159740899e-05
455 5.75896956434008e-05
456 5.97443431615829e-05
457 5.72519275010563e-05
458 5.91452408116311e-05
459 5.69108233321458e-05
460 5.8514138800092e-05
461 5.65399022889324e-05
462 5.80648411414586e-05
463 5.62578534299973e-05
464 5.78842991671991e-05
465 5.59712825634051e-05
466 5.77394530409947e-05
467 5.56832419533748e-05
468 5.77935970795806e-05
469 5.54505568288732e-05
470 5.78455328650307e-05
471 5.51868251932319e-05
472 5.75521335122176e-05
473 5.49257856619079e-05
474 5.69579715374857e-05
475 5.46569863217883e-05
476 5.64594483876135e-05
477 5.44364447705448e-05
478 5.60717853659298e-05
479 5.4155145335244e-05
480 5.58602623641491e-05
481 5.39150714757852e-05
482 5.58932588319294e-05
483 5.37396699655801e-05
484 5.60636035515927e-05
485 5.3547428251477e-05
486 5.60229418624658e-05
487 5.33645034010988e-05
488 5.56753038836177e-05
489 5.31299228896387e-05
490 5.51819175598212e-05
491 5.29032586200628e-05
492 5.4879354138393e-05
493 5.27116535522509e-05
494 5.45676084584557e-05
495 5.24875795235857e-05
496 5.44343165529426e-05
497 5.2309322200017e-05
498 5.44786853424739e-05
499 5.21724214195274e-05
500 5.46228984603658e-05
501 5.20203757332638e-05
502 5.45455441169906e-05
503 5.17997141287196e-05
504 5.41989065823145e-05
505 5.16565560246818e-05
506 5.3867683163844e-05
507 5.15288629685529e-05
508 5.34941427758895e-05
509 5.13499253429472e-05
510 5.32942285644822e-05
511 5.12191290908959e-05
512 5.33086422365159e-05
513 5.10202262375969e-05
514 5.33791535417549e-05
515 5.09124329255428e-05
516 5.35852886969224e-05
517 5.08283483213745e-05
518 5.34982173121534e-05
519 5.06656033394393e-05
520 5.30627366970293e-05
521 5.04520867252722e-05
522 5.25826908415183e-05
523 5.03115770698059e-05
524 5.23819071531761e-05
525 5.01704926136881e-05
526 5.22486843692604e-05
527 4.99932029924821e-05
528 5.22527807333972e-05
529 4.98965127917472e-05
530 5.24881361343432e-05
531 4.98238150612451e-05
532 5.26495205122046e-05
533 4.9690996092977e-05
534 5.23218732269015e-05
535 4.95564709126484e-05
536 5.1889888709411e-05
537 4.94406704092398e-05
538 5.16204927407671e-05
539 4.92749422846828e-05
540 5.14804814883973e-05
541 4.914055170957e-05
542 5.15572646690998e-05
543 4.90743368573021e-05
544 5.17142289027106e-05
545 4.89280464535113e-05
546 5.17505468451418e-05
547 4.88824880449101e-05
548 5.14420389663428e-05
549 4.87049401272088e-05
550 5.11060643475503e-05
551 4.85268901684321e-05
552 5.08526318299118e-05
553 4.84187366964761e-05
554 5.07957629451994e-05
555 4.83168769278564e-05
556 5.09266355948057e-05
557 4.82650430058129e-05
558 5.09596757183317e-05
559 4.81265597045422e-05
560 5.0763577746693e-05
561 4.80019989481661e-05
562 5.05340212839656e-05
563 4.79236405226402e-05
564 5.03400187881198e-05
565 4.77953763038386e-05
566 5.03762712469324e-05
567 4.7704073949717e-05
568 5.03962910443079e-05
569 4.76503955724183e-05
570 5.02135735587217e-05
571 4.74796506750863e-05
572 5.00726346217562e-05
573 4.73757718282286e-05
574 5.00281930726487e-05
575 4.73347354272846e-05
576 4.99305715493392e-05
577 4.71872335765511e-05
578 4.98998306284193e-05
579 4.71739185741171e-05
580 4.98256922583096e-05
581 4.70271588710602e-05
582 4.97142900712788e-05
583 4.69317055831198e-05
584 4.96016000397503e-05
585 4.68543512397446e-05
586 4.95473977935035e-05
587 4.67388272227254e-05
588 4.94931082357652e-05
589 4.6677279897267e-05
590 4.94233281642664e-05
591 4.65693956357427e-05
592 4.9318445235258e-05
593 4.64947916043457e-05
594 4.92122817377094e-05
595 4.64049408037681e-05
596 4.90352504129987e-05
597 4.63055403088219e-05
598 4.89539597765543e-05
599 4.62012285424862e-05
600 4.89279409521259e-05
601 4.61481795355212e-05
602 4.8926034651231e-05
603 4.60527444374748e-05
604 4.8908383178059e-05
605 4.59554612461943e-05
606 4.88717487314716e-05
607 4.59217189927585e-05
608 4.87396391690709e-05
609 4.57709211332258e-05
610 4.85559612570796e-05
611 4.56436755484901e-05
612 4.8358862841269e-05
613 4.56067245977465e-05
614 4.82262184959836e-05
615 4.55673289252445e-05
616 4.82199538964778e-05
617 4.54538785561454e-05
618 4.84355005028192e-05
619 4.54102555522695e-05
620 4.85896853206214e-05
621 4.53598258900456e-05
622 4.83150906802621e-05
623 4.52513486379758e-05
624 4.80760900245514e-05
625 4.51064588560257e-05
626 4.77961693832185e-05
627 4.50686857220717e-05
628 4.76275781693403e-05
629 4.49877225037199e-05
630 4.75705091957934e-05
631 4.49696563009638e-05
632 4.77684297948144e-05
633 4.49212529929355e-05
634 4.82311552332249e-05
635 4.48721839347854e-05
636 4.82881769130472e-05
637 4.4793192500947e-05
638 4.76211935165338e-05
639 4.46789199486375e-05
640 4.70745944767259e-05
641 4.45275218226016e-05
642 4.69151018478442e-05
643 4.44354627688881e-05
644 4.70341146865394e-05
645 4.43990466010291e-05
646 4.73515210615005e-05
647 4.43980534328148e-05
648 4.78833244415e-05
649 4.44423603767063e-05
650 4.78261281386949e-05
651 4.42174823547248e-05
652 4.70449449494481e-05
653 4.4080432417104e-05
654 4.64320983155631e-05
655 4.3986216041958e-05
656 4.62264870293438e-05
657 4.38530369137879e-05
658 4.63186115666758e-05
659 4.39249670307618e-05
660 4.70432387373876e-05
661 4.39318719145376e-05
662 4.75885681225918e-05
663 4.39328941865824e-05
664 4.75445121992379e-05
665 4.36954142060131e-05
666 4.65130906377453e-05
667 4.36777154391166e-05
668 4.5828812289983e-05
669 4.35095134889707e-05
670 4.56292982562445e-05
671 4.34522662544623e-05
672 4.59080401924439e-05
673 4.34325665992219e-05
674 4.65898556285538e-05
675 4.35997790191323e-05
676 4.76833374705166e-05
677 4.35274305345956e-05
678 4.69836413685698e-05
679 4.33230670751072e-05
680 4.58667091152165e-05
681 4.31488915637601e-05
682 4.51708328910172e-05
683 4.31243643106427e-05
684 4.5080731069902e-05
685 4.30410364060663e-05
686 4.5458204112947e-05
687 4.30607760790735e-05
688 4.66979981865734e-05
689 4.32104570791125e-05
690 4.72968567919452e-05
691 4.31491898780223e-05
692 4.63661053800024e-05
693 4.28259845648427e-05
694 4.51130035798997e-05
695 4.27802624471951e-05
696 4.45495388703421e-05
697 4.2611933167791e-05
698 4.45942823716905e-05
699 4.26499864261132e-05
700 4.51914784207474e-05
701 4.26774895458948e-05
702 4.6464101615129e-05
703 4.29133542638738e-05
704 4.73180771223269e-05
705 4.27696977567393e-05
706 4.58212452940643e-05
707 4.23671408498194e-05
708 4.44112956756726e-05
709 4.23969686380588e-05
710 4.39013856521342e-05
711 4.22548473579809e-05
712 4.39504183304962e-05
713 4.2263123759767e-05
714 4.46038939116988e-05
715 4.2263804061804e-05
716 4.64153490611352e-05
717 4.27765626227483e-05
718 4.77074354421347e-05
719 4.24876307079103e-05
720 4.5459033572115e-05
721 4.2065297748195e-05
722 4.39255745732225e-05
723 4.19516727561131e-05
724 4.33844979852438e-05
725 4.20015348936431e-05
726 4.32868655479979e-05
727 4.18375784647651e-05
728 4.38524402852636e-05
729 4.18998642999213e-05
730 4.56626839877572e-05
731 4.22630837420002e-05
732 4.77300927741453e-05
733 4.2338822822785e-05
734 4.57711394119542e-05
735 4.1767125367187e-05
736 4.36488007835578e-05
737 4.1653482185211e-05
738 4.2996496631531e-05
739 4.1632454667706e-05
740 4.28656057920307e-05
741 4.15449358115438e-05
742 4.32175693276804e-05
743 4.14932110288646e-05
744 4.45079131168313e-05
745 4.17837800341658e-05
746 4.71742387162521e-05
747 4.22039811383002e-05
748 4.64096046925988e-05
749 4.14755086239893e-05
750 4.36156296927948e-05
751 4.12684930779506e-05
752 4.26430087827612e-05
753 4.13167763326783e-05
754 4.23727724410128e-05
755 4.12263470934704e-05
756 4.25303114752751e-05
757 4.11162436648738e-05
758 4.33233617513906e-05
759 4.11438704759348e-05
760 4.57811074738856e-05
761 4.18986710428726e-05
762 4.77034191135317e-05
763 4.14000787714031e-05
764 4.40751136920881e-05
765 4.09394415328279e-05
766 4.23315032094251e-05
767 4.09759340982419e-05
768 4.19389580201823e-05
769 4.09631429647561e-05
770 4.18620693380944e-05
771 4.08127052651253e-05
772 4.22339435317554e-05
773 4.06974504585378e-05
774 4.35978508903645e-05
775 4.10381944675464e-05
776 4.71023377031088e-05
777 4.1665345634101e-05
778 4.60401570308022e-05
779 4.06581857532728e-05
780 4.26583392254543e-05
781 4.05660975957289e-05
782 4.16095717810094e-05
783 4.06129111070186e-05
784 4.12842746300157e-05
785 4.05003229388967e-05
786 4.13204252254218e-05
787 4.03408957936335e-05
788 4.18790659750812e-05
789 4.02560654038098e-05
790 4.38964452769142e-05
791 4.10628927056678e-05
792 4.85745877085719e-05
793 4.11051369155757e-05
794 4.39036775787827e-05
795 4.01597244490404e-05
796 4.14315400121268e-05
797 4.02043115173001e-05
798 4.08806481573265e-05
799 4.01797260565218e-05
800 4.07186635129619e-05
801 4.01418037654366e-05
802 4.0781385905575e-05
803 3.98784541175701e-05
804 4.13980342273135e-05
805 3.97994481318165e-05
806 4.37581329606473e-05
807 4.08390733355191e-05
808 4.90866877953522e-05
809 4.05313767259941e-05
810 4.29231768066529e-05
811 3.96846553485375e-05
812 4.06932776968461e-05
813 3.98042175220326e-05
814 4.02770747314207e-05
815 3.97470357711427e-05
816 4.0124861698132e-05
817 3.96926479879767e-05
818 4.01623001380358e-05
819 3.94468443118967e-05
820 4.05555947509129e-05
821 3.92922775063198e-05
822 4.24641184508801e-05
823 4.00845965486951e-05
824 4.89764861413278e-05
825 4.0602135413792e-05
826 4.36022310168482e-05
827 3.9198792364914e-05
828 4.0321123378817e-05
829 3.93119189538993e-05
830 3.97603289457038e-05
831 3.93707668990828e-05
832 3.95974893763196e-05
833 3.92561560147442e-05
834 3.9532154914923e-05
835 3.91706234950107e-05
836 3.95360984839499e-05
837 3.89633642043918e-05
838 3.99335440306459e-05
839 3.87406944355462e-05
840 4.23748060711659e-05
841 4.00828430429101e-05
842 5.14428429596592e-05
843 3.98462143493816e-05
844 4.1567203879822e-05
845 3.87719665013719e-05
846 3.93893424188718e-05
847 3.88596818083897e-05
848 3.90727254853118e-05
849 3.88905573345255e-05
850 3.89716478821356e-05
851 3.87924410460982e-05
852 3.88639600714669e-05
853 3.87547224818263e-05
854 3.88671978726052e-05
855 3.86104111385066e-05
856 3.88035659852903e-05
857 3.84495651815087e-05
858 3.91130270145368e-05
859 3.81115969503298e-05
860 4.1908941057045e-05
861 4.0360217099078e-05
862 5.53971876797732e-05
863 3.89251581509598e-05
864 3.97268158849329e-05
865 3.84220475098118e-05
866 3.8553451304324e-05
867 3.83768201572821e-05
868 3.84018967451993e-05
869 3.82982470910065e-05
870 3.83333863283042e-05
871 3.81822355848271e-05
872 3.82581711164676e-05
873 3.81869067496154e-05
874 3.8264668546617e-05
875 3.80947931262199e-05
876 3.8139787648106e-05
877 3.79304874513764e-05
878 3.82553625968285e-05
879 3.76973039237782e-05
880 3.89006199839059e-05
881 3.74861810996663e-05
882 4.43233402620535e-05
883 4.09600725106429e-05
884 4.99214329465758e-05
885 3.76182870240882e-05
886 3.85959283448756e-05
887 3.78568693122361e-05
888 3.79015182261355e-05
889 3.77448304789141e-05
890 3.77294381905813e-05
891 3.76708958356176e-05
892 3.77069809474051e-05
893 3.75813397113234e-05
894 3.7603280361509e-05
895 3.75184863514733e-05
896 3.75659074052237e-05
897 3.74295850633644e-05
898 3.75910458387807e-05
899 3.73211078112945e-05
900 3.76094612875022e-05
901 3.70836132788099e-05
902 3.85092607757542e-05
903 3.70205052604433e-05
904 4.64598815597128e-05
905 4.1100342059508e-05
906 4.63977994513698e-05
907 3.69034059986006e-05
908 3.75818781321868e-05
909 3.72409886040259e-05
910 3.72301074094139e-05
911 3.71050955436658e-05
912 3.709245720529e-05
913 3.7095833249623e-05
914 3.70112793461885e-05
915 3.69645458704326e-05
916 3.69932167814113e-05
917 3.69595072697848e-05
918 3.6897854442941e-05
919 3.68758446711581e-05
920 3.68293585779611e-05
921 3.67786269634962e-05
922 3.68031833204441e-05
923 3.67343236575834e-05
924 3.67947213817388e-05
925 3.66101266990881e-05
926 3.66722924809437e-05
927 3.65381129086018e-05
928 3.67172360711265e-05
929 3.62661339750048e-05
930 3.80297824449372e-05
931 3.76056959794369e-05
932 6.35039759799838e-05
933 3.82801481464412e-05
934 3.74097435269505e-05
935 3.66806343663484e-05
936 3.64787265425548e-05
937 3.64542611350771e-05
938 3.63641775038559e-05
939 3.64066218025982e-05
940 3.6360310332384e-05
941 3.62983446393628e-05
942 3.62650316674262e-05
943 3.62709943146911e-05
944 3.6204462958267e-05
945 3.62063983629923e-05
946 3.61956917913631e-05
947 3.61458805855364e-05
948 3.61122183676343e-05
949 3.6164481571177e-05
950 3.60984740837011e-05
951 3.61018064722884e-05
952 3.59644946001936e-05
953 3.60719495802186e-05
954 3.57462267857045e-05
955 3.63622966688126e-05
956 3.54468538716901e-05
957 3.90318309655413e-05
958 3.78239747078624e-05
959 5.57732819288503e-05
960 3.60261183232069e-05
961 3.65042906196322e-05
962 3.57549542968627e-05
963 3.57273638655897e-05
964 3.56565360561945e-05
965 3.56527489202563e-05
966 3.56158925569616e-05
967 3.55430274794344e-05
968 3.55875381501392e-05
969 3.55093870894052e-05
970 3.55429147020914e-05
971 3.54616167896893e-05
972 3.54739968315698e-05
973 3.54676667484455e-05
974 3.54490985046141e-05
975 3.53981886291876e-05
976 3.5449684219202e-05
977 3.53300420101732e-05
978 3.5467033740133e-05
979 3.51637427229434e-05
980 3.59238547389396e-05
981 3.47007517120801e-05
982 4.01610268454533e-05
983 3.79761877411511e-05
984 5.11218640895095e-05
985 3.50144982803613e-05
986 3.57188873749692e-05
987 3.51919043168891e-05
988 3.51496491930448e-05
989 3.51282797055319e-05
990 3.50280861312058e-05
991 3.50231894117314e-05
992 3.49983820342459e-05
993 3.49445472238585e-05
994 3.49569236277603e-05
995 3.49372494383715e-05
996 3.48856447089929e-05
997 3.48839021171443e-05
998 3.49210313288495e-05
999 3.48212961398531e-05
1000 3.49411311617587e-05
1001 3.46916604030412e-05
1002 3.50498739862815e-05
1003 3.45081898558419e-05
1004 3.55593365384266e-05
1005 3.41856830345932e-05
1006 3.95938513975125e-05
1007 3.67846550943796e-05
1008 4.86465251015034e-05
1009 3.45984153682366e-05
1010 3.5468881833367e-05
1011 3.45599291904364e-05
1012 3.46593624271918e-05
1013 3.45745174854528e-05
1014 3.45236512657721e-05
1015 3.45029111485928e-05
1016 3.45025218848605e-05
1017 3.4410953958286e-05
1018 3.44512336596381e-05
1019 3.43903957400471e-05
1020 3.4409520594636e-05
1021 3.43556457664818e-05
1022 3.43815445376094e-05
1023 3.43071733368561e-05
1024 3.44401123584248e-05
1025 3.42602797900327e-05
1026 3.44436593877617e-05
1027 3.40862789016683e-05
1028 3.48815119650681e-05
1029 3.36704943038058e-05
1030 3.82758371415548e-05
1031 3.61469974450301e-05
1032 5.06608339492232e-05
1033 3.42416351486463e-05
1034 3.4896147553809e-05
1035 3.40871265507303e-05
1036 3.40843216690701e-05
1037 3.40725309797563e-05
1038 3.39603175234515e-05
1039 3.39960970450193e-05
1040 3.39776161126792e-05
1041 3.39619800797664e-05
1042 3.39183388859965e-05
1043 3.39019534294493e-05
1044 3.38674944941886e-05
1045 3.39052676281426e-05
1046 3.38895915774629e-05
1047 3.38724639732391e-05
1048 3.38955942424946e-05
1049 3.37755773216486e-05
1050 3.38861391355749e-05
1051 3.36670236720238e-05
1052 3.40980295732152e-05
1053 3.34354444930796e-05
1054 3.52079841832165e-05
1055 3.34187716362067e-05
1056 4.43481912952848e-05
1057 3.68348701158538e-05
1058 4.03115482185967e-05
1059 3.32326271745842e-05
1060 3.38201389240567e-05
1061 3.35376607836224e-05
1062 3.3489748602733e-05
1063 3.34547694365028e-05
1064 3.3438227546867e-05
1065 3.34016585838981e-05
1066 3.34213727910537e-05
1067 3.34283140546177e-05
1068 3.33293319272343e-05
1069 3.34663200192153e-05
1070 3.33221432811115e-05
1071 3.33977768605109e-05
1072 3.33150965161622e-05
1073 3.34311116603203e-05
1074 3.32055606122594e-05
1075 3.35302938765381e-05
1076 3.31209994328674e-05
1077 3.3912652725121e-05
1078 3.28150053974241e-05
1079 3.59516852768138e-05
1080 3.36258526658639e-05
1081 4.58568756585009e-05
1082 3.42502862622496e-05
1083 3.55881347786635e-05
1084 3.28205460391473e-05
1085 3.3323340176139e-05
1086 3.30136572301853e-05
1087 3.30628754454665e-05
1088 3.30118600686546e-05
1089 3.30343573295977e-05
1090 3.29731592501048e-05
1091 3.3032465580618e-05
1092 3.2951713365037e-05
1093 3.31013870891184e-05
1094 3.29059475916438e-05
1095 3.30886432493571e-05
1096 3.28350251947995e-05
1097 3.31688970618416e-05
1098 3.26193694490939e-05
1099 3.37928468070459e-05
1100 3.23655294778291e-05
1101 3.6267279938329e-05
1102 3.32613417413086e-05
1103 4.36811715189833e-05
1104 3.33156385750044e-05
1105 3.51496200892143e-05
1106 3.24283246300183e-05
1107 3.3019434340531e-05
1108 3.25974651786964e-05
1109 3.27841880789492e-05
1110 3.26096924254671e-05
1111 3.27565976476762e-05
1112 3.26065855915658e-05
1113 3.27236157318112e-05
1114 3.25643813994247e-05
1115 3.27968991769012e-05
1116 3.2501407986274e-05
1117 3.29189606418367e-05
1118 3.23599269904662e-05
1119 3.32809868268669e-05
1120 3.21526567859109e-05
1121 3.46751985489391e-05
1122 3.21953484672122e-05
1123 3.96156065107789e-05
1124 3.33341886289418e-05
1125 3.82544785679784e-05
1126 3.20796716550831e-05
1127 3.3347761927871e-05
1128 3.21942934533581e-05
1129 3.26429653796367e-05
1130 3.22499217872974e-05
1131 3.24807369906921e-05
1132 3.22509986290243e-05
1133 3.24612738040742e-05
1134 3.22314954246394e-05
1135 3.24977154377848e-05
1136 3.21646148222499e-05
1137 3.26203371514566e-05
1138 3.20446306432132e-05
1139 3.30079237755854e-05
1140 3.18074708047789e-05
1141 3.42546554747969e-05
1142 3.1798397685634e-05
1143 3.85923303838354e-05
1144 3.27071611536667e-05
1145 3.784676300711e-05
1146 3.17506019200664e-05
1147 3.31588271365035e-05
1148 3.17935737257358e-05
1149 3.23958920489531e-05
1150 3.1893821869744e-05
1151 3.22095547744539e-05
1152 3.18693673762027e-05
1153 3.21341503877193e-05
1154 3.18450038321316e-05
1155 3.22045671055093e-05
1156 3.17452067974955e-05
1157 3.24208122037817e-05
1158 3.15991928800941e-05
1159 3.28546593664214e-05
1160 3.13776254188269e-05
1161 3.43231149599887e-05
1162 3.15036086249165e-05
1163 3.79586817871314e-05
1164 3.20787003147416e-05
1165 3.57489261659794e-05
1166 3.12861848215107e-05
1167 3.27813322655857e-05
1168 3.13950622512493e-05
1169 3.2135776564246e-05
1170 3.14245589834172e-05
1171 3.20779581670649e-05
1172 3.14371573040262e-05
1173 3.21134539262857e-05
1174 3.13526506943163e-05
1175 3.23247149935924e-05
1176 3.12388983729761e-05
1177 3.29589893226512e-05
1178 3.10834329866339e-05
1179 3.43653373420238e-05
1180 3.12046213366557e-05
1181 3.57123244612012e-05
1182 3.12428383040242e-05
1183 3.46671222359873e-05
1184 3.09794122586027e-05
1185 3.29087015416007e-05
1186 3.09546885546297e-05
1187 3.21812804031651e-05
1188 3.0985913326731e-05
1189 3.21507432090584e-05
1190 3.09968272631522e-05
1191 3.21844017889816e-05
1192 3.0904178856872e-05
1193 3.26218832924496e-05
1194 3.07773880194873e-05
1195 3.35745207848959e-05
1196 3.0798721127212e-05
1197 3.45180160365999e-05
1198 3.080998067162e-05
1199 3.39018515660428e-05
1200 3.0661525670439e-05
1201 3.27471789205447e-05
1202 3.06392212223727e-05
1203 3.24561624438502e-05
1204 3.05976973322686e-05
1205 3.2393782021245e-05
1206 3.05769863189198e-05
1207 3.23855529131833e-05
1208 3.05518842651509e-05
1209 3.26545705320314e-05
1210 3.05226712953299e-05
1211 3.2985357393045e-05
1212 3.04670647892635e-05
1213 3.34731630573515e-05
1214 3.04725399473682e-05
1215 3.31615628965665e-05
1216 3.03684882965172e-05
1217 3.25835244439077e-05
1218 3.03741035168059e-05
1219 3.21439474646468e-05
1220 3.03061588056153e-05
1221 3.21419793181121e-05
1222 3.03361830447102e-05
1223 3.24544744216837e-05
1224 3.02580701827537e-05
1225 3.27778252540156e-05
1226 3.02454373013461e-05
1227 3.30088114424143e-05
1228 3.02046773867914e-05
1229 3.32238305418286e-05
1230 3.02128719340544e-05
1231 3.24657485180069e-05
1232 3.01217005471699e-05
1233 3.19527243846096e-05
1234 3.01340223813895e-05
1235 3.17714257107582e-05
1236 3.01117834169418e-05
1237 3.20273793477099e-05
1238 3.00553165288875e-05
1239 3.21913794323336e-05
1240 3.00072751997504e-05
1241 3.24734355672263e-05
1242 2.99996427202132e-05
1243 3.29267604683992e-05
1244 2.99662970064674e-05
1245 3.25210457958747e-05
1246 2.98866943921894e-05
1247 3.1965522794053e-05
1248 2.98931645374978e-05
1249 3.1662784749642e-05
1250 2.98997747449903e-05
1251 3.16397890856024e-05
1252 2.98648465104634e-05
1253 3.18812199111562e-05
1254 2.97999849863118e-05
1255 3.21049519698136e-05
1256 2.97881833830616e-05
1257 3.22839041473344e-05
1258 2.97620972560253e-05
1259 3.22970918205101e-05
1260 2.97370661428431e-05
1261 3.23263411701191e-05
1262 2.97227525152266e-05
1263 3.1732637580717e-05
1264 2.96829093713313e-05
1265 3.17019985232037e-05
1266 2.9662105589523e-05
1267 3.15068755298853e-05
1268 2.96368234558031e-05
1269 3.16199839289766e-05
1270 2.96129746857332e-05
1271 3.14033532049507e-05
1272 2.95632416964509e-05
1273 3.15665893140249e-05
1274 2.95497229672037e-05
1275 3.18223064823542e-05
1276 2.95134232146665e-05
1277 3.17016638291534e-05
1278 2.94809415208874e-05
1279 3.16296172968578e-05
1280 2.94782457785914e-05
1281 3.16280129482038e-05
1282 2.9418348276522e-05
1283 3.1558931368636e-05
1284 2.94154942821478e-05
1285 3.16119294438977e-05
1286 2.93793218588689e-05
1287 3.13625532726292e-05
1288 2.93785687972559e-05
1289 3.11425937979948e-05
1290 2.93250068352791e-05
1291 3.11054645862896e-05
1292 2.93320463242708e-05
1293 3.13119999191258e-05
1294 2.92656332021579e-05
1295 3.12534248223528e-05
1296 2.92218137474265e-05
1297 3.12319534714334e-05
1298 2.92097793135326e-05
1299 3.13827476929873e-05
1300 2.91919459414203e-05
1301 3.13652635668404e-05
1302 2.91414216917474e-05
1303 3.14684875775129e-05
1304 2.91377928078873e-05
1305 3.09963652398437e-05
1306 2.90956304525025e-05
1307 3.11206094920635e-05
1308 2.90665084321517e-05
1309 3.0952571250964e-05
1310 2.90351435978664e-05
1311 3.08068592858035e-05
1312 2.90109928755555e-05
1313 3.07908485410735e-05
1314 2.90236712316982e-05
1315 3.06005640595686e-05
1316 2.89673316729022e-05
1317 3.05890389427077e-05
1318 2.89376821456244e-05
1319 3.0695886380272e-05
1320 2.89253130176803e-05
1321 3.10285468003713e-05
1322 2.88724422716768e-05
1323 3.09075039695017e-05
1324 2.88192441075807e-05
1325 3.08995004161261e-05
1326 2.88181472569704e-05
1327 3.08260241581593e-05
1328 2.88002811430488e-05
1329 3.06590809486806e-05
1330 2.8751781428582e-05
1331 3.05537578242365e-05
1332 2.87245093204547e-05
1333 3.05577268591151e-05
1334 2.86665344901849e-05
1335 3.03388223983347e-05
1336 2.86760805465747e-05
1337 3.02625194308348e-05
1338 2.86296890408266e-05
1339 3.02719799947226e-05
1340 2.86059876088984e-05
1341 3.04695149679901e-05
1342 2.85617279587314e-05
1343 3.04451095871627e-05
1344 2.85101141344057e-05
1345 3.04249406326562e-05
1346 2.85056703432929e-05
1347 3.04663353745127e-05
1348 2.84548605122836e-05
1349 3.03521082969382e-05
1350 2.83994722849457e-05
1351 3.0294000680442e-05
1352 2.84001162071945e-05
1353 3.01449708786095e-05
1354 2.83836325252196e-05
1355 3.01667496387381e-05
1356 2.83671543002129e-05
1357 2.98468494293047e-05
1358 2.83303379546851e-05
1359 3.00731353490846e-05
1360 2.82890250673518e-05
1361 3.00689280265942e-05
1362 2.82444434560603e-05
1363 2.99732437269995e-05
1364 2.82119708572282e-05
1365 3.00588835671078e-05
1366 2.82096370938234e-05
1367 2.98163013212616e-05
1368 2.81727971014334e-05
1369 2.99366638500942e-05
1370 2.81602442555595e-05
1371 2.98553823085967e-05
1372 2.8110825951444e-05
1373 2.96840189548675e-05
1374 2.81057000393048e-05
1375 2.97039823635714e-05
1376 2.80398689938011e-05
1377 2.95023710350506e-05
1378 2.80241893051425e-05
1379 2.94565397780389e-05
1380 2.80206459137844e-05
1381 2.95749832730507e-05
1382 2.79600790236145e-05
1383 2.95146219286835e-05
1384 2.79585001408122e-05
1385 2.95313766400795e-05
1386 2.79132655123249e-05
1387 2.96976104436908e-05
1388 2.78877905657282e-05
1389 2.9529246603488e-05
1390 2.78402822004864e-05
1391 2.93888206215343e-05
1392 2.78445586445741e-05
1393 2.9357614039327e-05
1394 2.77916460618144e-05
1395 2.9452934541041e-05
1396 2.7774303816841e-05
1397 2.91846572508803e-05
1398 2.77521394309588e-05
1399 2.91050255327718e-05
1400 2.77391900453949e-05
1401 2.92564509436488e-05
1402 2.77137933153426e-05
1403 2.91133073915262e-05
1404 2.76787504844833e-05
1405 2.93614775728201e-05
1406 2.76381397270598e-05
1407 2.93065004370874e-05
1408 2.76012360700406e-05
1409 2.91894957626937e-05
1410 2.75600359600503e-05
1411 2.90017960651312e-05
1412 2.75611455435865e-05
1413 2.88632563751889e-05
1414 2.75275142485043e-05
1415 2.8899397875648e-05
1416 2.7531772502698e-05
1417 2.86690938082756e-05
1418 2.75129932560958e-05
1419 2.8788499548682e-05
1420 2.74561225523939e-05
1421 2.8780512366211e-05
1422 2.74455578619381e-05
1423 2.88254996121395e-05
1424 2.74062094831606e-05
1425 2.85903679468902e-05
1426 2.73765981546603e-05
1427 2.88223909592489e-05
1428 2.73544865194708e-05
1429 2.87095990643138e-05
1430 2.733539804467e-05
1431 2.85880378214642e-05
1432 2.73090081464034e-05
1433 2.86502927338006e-05
1434 2.73048208327964e-05
1435 2.84289853880182e-05
1436 2.73245204880368e-05
1437 2.82705605059164e-05
1438 2.72989382210653e-05
1439 2.8410810045898e-05
1440 2.72480010607978e-05
1441 2.82743276329711e-05
1442 2.72396737273084e-05
1443 2.85142796201399e-05
1444 2.71517328656046e-05
1445 2.85329551843461e-05
1446 2.71711851382861e-05
1447 2.84595371340401e-05
1448 2.71615208475851e-05
1449 2.83608569588978e-05
1450 2.71075896307593e-05
1451 2.84270427073352e-05
1452 2.7132271497976e-05
1453 2.81461234408198e-05
1454 2.71363605861552e-05
1455 2.80728509096662e-05
1456 2.70861164608505e-05
1457 2.82187484117458e-05
1458 2.70854052359937e-05
1459 2.80685981124407e-05
1460 2.70965920208255e-05
1461 2.79192172456533e-05
1462 2.70656528300606e-05
1463 2.81296306638978e-05
1464 2.70051677944139e-05
1465 2.79748128377832e-05
1466 2.69934862444643e-05
1467 2.82229975709924e-05
1468 2.69382417172892e-05
1469 2.82150431303307e-05
1470 2.69595766440034e-05
1471 2.80749427474802e-05
1472 2.68972635240061e-05
1473 2.79780106211547e-05
1474 2.68895983026596e-05
1475 2.80160384136252e-05
1476 2.68991352641024e-05
1477 2.77860817732289e-05
1478 2.69156753347488e-05
1479 2.76169794233283e-05
1480 2.68940075329738e-05
1481 2.77349317912012e-05
1482 2.68835829047021e-05
1483 2.76233931799652e-05
1484 2.69153442786774e-05
1485 2.75439579127124e-05
1486 2.6850628273678e-05
1487 2.76780119747855e-05
1488 2.68110870820237e-05
1489 2.75911443168297e-05
1490 2.68716303253314e-05
1491 2.75591373792849e-05
1492 2.67986051767366e-05
1493 2.77353992714779e-05
1494 2.67641771642957e-05
1495 2.77058388746809e-05
1496 2.67240629909793e-05
1497 2.76656446658308e-05
1498 2.67140712821856e-05
1499 2.77136259683175e-05
1500 2.66720617219107e-05
1501 2.78379193332512e-05
1502 2.66599163296632e-05
1503 2.7615278668236e-05
1504 2.66442402789835e-05
1505 2.76382052106783e-05
1506 2.66093957179692e-05
1507 2.7441279598861e-05
1508 2.66709976131096e-05
1509 2.73153837042628e-05
1510 2.66753522737417e-05
1511 2.72695124294842e-05
1512 2.6638035706128e-05
1513 2.73575351457112e-05
1514 2.65983962890459e-05
1515 2.72380857495591e-05
1516 2.66583847405855e-05
1517 2.71799326583277e-05
1518 2.66182541963644e-05
1519 2.71768858510768e-05
1520 2.6596282623359e-05
1521 2.73018449661322e-05
1522 2.65519047388807e-05
1523 2.71539593086345e-05
1524 2.65438065980561e-05
1525 2.71096396318171e-05
1526 2.6542102204985e-05
1527 2.71408425760455e-05
1528 2.64688751485664e-05
1529 2.71873886958929e-05
1530 2.6472343961359e-05
1531 2.70942200586433e-05
1532 2.64923928625649e-05
1533 2.69988777290564e-05
1534 2.64582740783226e-05
1535 2.69860138359945e-05
1536 2.64306127064629e-05
1537 2.70953823928721e-05
1538 2.63919610006269e-05
1539 2.69465654127998e-05
1540 2.64392965618754e-05
1541 2.68110597971827e-05
1542 2.64190366578987e-05
1543 2.68608182523167e-05
1544 2.63818146777339e-05
1545 2.69171614490915e-05
1546 2.63592537521617e-05
1547 2.68275816779351e-05
1548 2.63670408457983e-05
1549 2.67496398009825e-05
1550 2.63584606727818e-05
1551 2.67830273514846e-05
1552 2.63191013800679e-05
1553 2.68716357822996e-05
1554 2.62645862676436e-05
1555 2.67149680439616e-05
1556 2.62917219515657e-05
1557 2.66766182903666e-05
1558 2.62864286924014e-05
1559 2.66467086476041e-05
1560 2.62192625086755e-05
1561 2.67285940935835e-05
1562 2.62169123743661e-05
1563 2.65692997345468e-05
1564 2.62385183305014e-05
1565 2.65447961282916e-05
1566 2.62076428043656e-05
1567 2.65005546680186e-05
1568 2.61635959759587e-05
1569 2.65327034867369e-05
1570 2.61303175648209e-05
1571 2.64255577349104e-05
1572 2.62024277617456e-05
1573 2.63431920757284e-05
1574 2.61669392784825e-05
1575 2.63328911387362e-05
1576 2.61114528257167e-05
1577 2.6438172426424e-05
1578 2.60758697550045e-05
1579 2.63198253378505e-05
1580 2.60999677266227e-05
1581 2.62308058154304e-05
1582 2.6072748369188e-05
1583 2.62858266069088e-05
1584 2.6047469873447e-05
1585 2.62991416093428e-05
1586 2.59964217548259e-05
1587 2.61796230915934e-05
1588 2.60293418250512e-05
1589 2.61040622717701e-05
1590 2.60176657320699e-05
1591 2.60248834820231e-05
1592 2.60714168689447e-05
1593 2.59788157563889e-05
1594 2.60172710113693e-05
1595 2.60083488683449e-05
1596 2.59688258665847e-05
1597 2.5963436200982e-05
1598 2.59051976172486e-05
1599 2.60020788118709e-05
1600 2.58819327427773e-05
1601 2.5904144422384e-05
1602 2.59004045801703e-05
1603 2.58595973718911e-05
1604 2.59447806456592e-05
1605 2.58137188211549e-05
1606 2.58827985817334e-05
1607 2.58562322414946e-05
1608 2.57779047387885e-05
1609 2.58587533608079e-05
1610 2.57845276792068e-05
1611 2.57676529145101e-05
1612 2.57647807302419e-05
1613 2.57361189142102e-05
1614 2.57257543125888e-05
1615 2.57207320828456e-05
1616 2.57103383773938e-05
1617 2.57326391874813e-05
1618 2.56553230428835e-05
1619 2.57117717410438e-05
1620 2.56729035754688e-05
1621 2.56334751611575e-05
1622 2.564243368397e-05
1623 2.56293496931903e-05
1624 2.56326075032121e-05
1625 2.56154053204227e-05
1626 2.55799168371595e-05
1627 2.56149724009447e-05
1628 2.55634004133753e-05
1629 2.56119092227891e-05
1630 2.55910690611927e-05
1631 2.55746290349634e-05
1632 2.55859667959157e-05
1633 2.55685681622708e-05
1634 2.55828545050463e-05
1635 2.55328595812898e-05
1636 2.56065668509109e-05
1637 2.55062004725914e-05
1638 2.56087114394177e-05
1639 2.5528055630275e-05
1640 2.56020775850629e-05
1641 2.55215345532633e-05
1642 2.56296334555373e-05
1643 2.55720915447455e-05
1644 2.56572893704288e-05
1645 2.5573193852324e-05
1646 2.56353960139677e-05
1647 2.55892337008845e-05
1648 2.57200226769783e-05
1649 2.55947106779786e-05
1650 2.57063256867696e-05
1651 2.56292059930274e-05
1652 2.57079918810632e-05
1653 2.56364037340973e-05
1654 2.5757461116882e-05
1655 2.56913062912645e-05
1656 2.57420524576446e-05
1657 2.56699859164655e-05
1658 2.57937008427689e-05
1659 2.5704592189868e-05
1660 2.57780866377288e-05
1661 2.56978291872656e-05
1662 2.58190448221285e-05
1663 2.57361825788394e-05
1664 2.57701176451519e-05
1665 2.58010059042135e-05
1666 2.56451548921177e-05
1667 2.58092095464235e-05
1668 2.58259588008514e-05
1669 2.57991305261385e-05
1670 2.578863495728e-05
1671 2.58435593423201e-05
1672 2.58078071055934e-05
1673 2.56517578236526e-05
1674 2.58099789789412e-05
1675 2.5821891540545e-05
1676 2.5803410608205e-05
1677 2.57305055129109e-05
1678 2.58316631516209e-05
1679 2.57963092735736e-05
1680 2.57023621088592e-05
1681 2.57823048741557e-05
1682 2.57015944953309e-05
1683 2.5754850867088e-05
1684 2.5762667064555e-05
1685 2.56420316873118e-05
1686 2.57767369475914e-05
1687 2.56222010648344e-05
1688 2.56965540756937e-05
1689 2.56791081483243e-05
1690 2.55856684816536e-05
1691 2.56597395491553e-05
1692 2.56143321166746e-05
1693 2.56310304393992e-05
1694 2.54735168709885e-05
1695 2.56631283264142e-05
1696 2.55839640885824e-05
1697 2.56153107329737e-05
1698 2.55663144343998e-05
1699 2.55048889812315e-05
1700 2.55500126513653e-05
1701 2.54965980275301e-05
1702 2.54224505624734e-05
1703 2.55061913776444e-05
1704 2.54643855441827e-05
1705 2.54023671004688e-05
1706 2.54748429142637e-05
1707 2.54413862421643e-05
1708 2.53577527473681e-05
1709 2.54376682278235e-05
1710 2.53809921559878e-05
1711 2.53111229540082e-05
1712 2.53852995228954e-05
1713 2.53193375101546e-05
1714 2.52729350904701e-05
1715 2.53142970905174e-05
1716 2.52674690273125e-05
1717 2.52264380833367e-05
1718 2.52913287113188e-05
1719 2.52543541137129e-05
1720 2.52041518251644e-05
1721 2.52303980232682e-05
1722 2.51914789259899e-05
1723 2.51952260441612e-05
1724 2.5102785002673e-05
1725 2.51739802479278e-05
1726 2.51386754825944e-05
1727 2.50762277573813e-05
1728 2.51240489888005e-05
1729 2.51228502747836e-05
1730 2.50286266236799e-05
1731 2.50831508310512e-05
1732 2.50764660449931e-05
1733 2.49594359047478e-05
1734 2.50825996772619e-05
1735 2.49830500251846e-05
1736 2.49506829277379e-05
1737 2.5001685571624e-05
1738 2.49217973760096e-05
1739 2.48886153713102e-05
1740 2.49316035478842e-05
1741 2.48818923864746e-05
1742 2.48417309194338e-05
1743 2.48820415436057e-05
1744 2.4857048629201e-05
1745 2.4807055524434e-05
1746 2.48718515649671e-05
1747 2.48080523306271e-05
1748 2.47700008912943e-05
1749 2.48373380600242e-05
1750 2.4728273274377e-05
1751 2.47458719968563e-05
1752 2.47561165451771e-05
1753 2.47293974098284e-05
1754 2.47029292950174e-05
1755 2.47459483944112e-05
1756 2.46813960984582e-05
1757 2.46802919718903e-05
1758 2.46701529249549e-05
1759 2.46394029090879e-05
1760 2.462228803779e-05
1761 2.46506278926972e-05
1762 2.46198324020952e-05
1763 2.4583505364717e-05
1764 2.46034996962408e-05
1765 2.45904284383869e-05
1766 2.4530596419936e-05
1767 2.45829814957688e-05
1768 2.4529776055715e-05
1769 2.45094925048761e-05
1770 2.45190949499374e-05
1771 2.44817074417369e-05
1772 2.44860966631677e-05
1773 2.45124629145721e-05
1774 2.44541224674322e-05
1775 2.44620223384118e-05
1776 2.4474747988279e-05
1777 2.4432296413579e-05
1778 2.44061047851574e-05
1779 2.44607963395538e-05
1780 2.43792910623597e-05
1781 2.4391949409619e-05
1782 2.43901886278763e-05
1783 2.43878712353762e-05
1784 2.43724898609798e-05
1785 2.43862123170402e-05
1786 2.43406593654072e-05
1787 2.43334561673691e-05
1788 2.4362208932871e-05
1789 2.43104877881706e-05
1790 2.43225640588207e-05
1791 2.42898822762072e-05
1792 2.43373797275126e-05
1793 2.43100257648621e-05
1794 2.42897785938112e-05
1795 2.428002335364e-05
1796 2.42917776631657e-05
1797 2.42619680648204e-05
1798 2.42362038989086e-05
1799 2.42623282247223e-05
1800 2.42535115830833e-05
1801 2.4245879103546e-05
1802 2.42279365920695e-05
1803 2.42284149862826e-05
1804 2.42339010583237e-05
1805 2.42148216784699e-05
1806 2.42161331698298e-05
1807 2.41841462411685e-05
1808 2.41888465097873e-05
1809 2.41814523178618e-05
1810 2.42138503381284e-05
1811 2.42021542362636e-05
1812 2.41658326558536e-05
1813 2.4188633688027e-05
1814 2.41634115809575e-05
1815 2.42451151279965e-05
1816 2.41517209360609e-05
1817 2.42349269683473e-05
1818 2.416245843051e-05
1819 2.41883681155741e-05
1820 2.41190900851507e-05
1821 2.41442994592944e-05
1822 2.4119772206177e-05
1823 2.41217130678706e-05
1824 2.40762728935806e-05
1825 2.41740963247139e-05
1826 2.40777553699445e-05
1827 2.41780999203911e-05
1828 2.40716926782625e-05
1829 2.42121823248453e-05
1830 2.40589833992999e-05
1831 2.42330861510709e-05
1832 2.40375975408824e-05
1833 2.41852321778424e-05
1834 2.40270310314372e-05
1835 2.41864527197322e-05
1836 2.40131994360127e-05
1837 2.42247460846556e-05
1838 2.40213903452968e-05
1839 2.42222213273635e-05
1840 2.39709133893484e-05
1841 2.41994621319463e-05
1842 2.39783767028712e-05
1843 2.42541573243216e-05
1844 2.39780620177044e-05
1845 2.42931746470276e-05
1846 2.39290075114695e-05
1847 2.4243521693279e-05
1848 2.39067903748946e-05
1849 2.41625584749272e-05
1850 2.38884658756433e-05
1851 2.41444813582348e-05
1852 2.38690972764743e-05
1853 2.41512716456782e-05
1854 2.38753309531603e-05
1855 2.41721645579673e-05
1856 2.38444627029821e-05
1857 2.4129945813911e-05
1858 2.38586235354887e-05
1859 2.41419165831758e-05
1860 2.38658903981559e-05
1861 2.41890720644733e-05
1862 2.3857024643803e-05
1863 2.42362038989086e-05
1864 2.38437642110512e-05
1865 2.42028199863853e-05
1866 2.38312659348594e-05
1867 2.41558791458374e-05
1868 2.38355205510743e-05
1869 2.40682438743534e-05
1870 2.37926724366844e-05
1871 2.39637156482786e-05
1872 2.37783133343328e-05
1873 2.39184009842575e-05
1874 2.37753847613931e-05
1875 2.38925240410026e-05
1876 2.37576714425813e-05
1877 2.39171113207703e-05
1878 2.37522926909151e-05
1879 2.38968004850904e-05
1880 2.37286112678703e-05
1881 2.39304008573527e-05
1882 2.37068707065191e-05
1883 2.3936428988236e-05
1884 2.37323620240204e-05
1885 2.39126584347105e-05
1886 2.37136355281109e-05
1887 2.38691354752518e-05
1888 2.36574796872446e-05
1889 2.37574386119377e-05
1890 2.36676605709363e-05
1891 2.36848227359587e-05
1892 2.37143103731796e-05
1893 2.3683196559432e-05
1894 2.36415980907623e-05
1895 2.36966570810182e-05
1896 2.36461746681016e-05
1897 2.37136118812487e-05
1898 2.36432333622361e-05
1899 2.37306194321718e-05
1900 2.36179166677175e-05
1901 2.37072599702515e-05
1902 2.36080813920125e-05
1903 2.36881169257686e-05
1904 2.36129544646246e-05
1905 2.36773521464784e-05
1906 2.35904535657028e-05
1907 2.36351843341254e-05
1908 2.35820371017326e-05
1909 2.35819225054001e-05
1910 2.35848037846154e-05
1911 2.35594870900968e-05
1912 2.35408115258906e-05
1913 2.35435381910065e-05
1914 2.35361094382824e-05
1915 2.3550563128083e-05
1916 2.35386869462673e-05
1917 2.35481220443035e-05
1918 2.35244751820574e-05
1919 2.35126972256694e-05
1920 2.35215175052872e-05
1921 2.35264269576874e-05
1922 2.35133338719606e-05
1923 2.34903709497303e-05
1924 2.34835915762233e-05
1925 2.3464586774935e-05
1926 2.34957624343224e-05
1927 2.34771669056499e-05
1928 2.34907220146852e-05
1929 2.34317594731692e-05
1930 2.3491131287301e-05
1931 2.34159488172736e-05
1932 2.34771468967665e-05
1933 2.34244107559789e-05
1934 2.34629951592069e-05
1935 2.34021008509444e-05
1936 2.34507042478072e-05
1937 2.33725459111156e-05
1938 2.34702965826727e-05
1939 2.34144554269733e-05
1940 2.3403965315083e-05
1941 2.34351809922373e-05
1942 2.3399412384606e-05
1943 2.34182098211022e-05
1944 2.3368549591396e-05
1945 2.34384624491213e-05
1946 2.34011204156559e-05
1947 2.33982264035149e-05
1948 2.33825030591106e-05
1949 2.34308245126158e-05
1950 2.34071285376558e-05
1951 2.33558585023275e-05
1952 2.34709405049216e-05
1953 2.34477629419416e-05
1954 2.339564889553e-05
1955 2.35041916312184e-05
1956 2.33657829085132e-05
1957 2.34945837291889e-05
1958 2.34119579545222e-05
1959 2.35103980230633e-05
1960 2.35043080465402e-05
1961 2.34448070841609e-05
1962 2.35594816331286e-05
1963 2.35096322285244e-05
1964 2.34610415645875e-05
1965 2.35654133575736e-05
1966 2.35805891861673e-05
1967 2.35388397413772e-05
1968 2.36102314374875e-05
1969 2.36666219279869e-05
1970 2.35384050029097e-05
1971 2.37017193285283e-05
1972 2.37147232837742e-05
1973 2.36187061091186e-05
1974 2.37654567172285e-05
1975 2.37555013882229e-05
1976 2.37187650782289e-05
1977 2.38076590903802e-05
1978 2.37293115787907e-05
1979 2.38131669902941e-05
1980 2.38674037973396e-05
1981 2.38089123740792e-05
1982 2.38995107793016e-05
1983 2.37893746088957e-05
1984 2.38979682762874e-05
1985 2.39549663092475e-05
1986 2.387578933849e-05
1987 2.40012795984512e-05
1988 2.39162745856447e-05
1989 2.40104691329179e-05
1990 2.39301607507514e-05
1991 2.40251829382032e-05
1992 2.38935990637401e-05
1993 2.40146728174295e-05
1994 2.39925902860705e-05
1995 2.40260178543394e-05
1996 2.39146920648636e-05
1997 2.40241224673809e-05
1998 2.40402405324858e-05
1999 2.40174340433441e-05
};
\addlegendentry{Test}

\nextgroupplot[
legend cell align={left},
legend style={fill opacity=0.8, draw opacity=1, text opacity=1, at={(0.5,0.5)}, anchor=center, draw=white!80!black},
log basis y={10},
tick align=outside,
tick pos=left,
title={fold1 },
x grid style={white!69.0196078431373!black},
xlabel={Epoch},
xmin=-99.95, xmax=2098.95,
xtick style={color=black},
y grid style={white!69.0196078431373!black},
ylabel={MSE Loss},
ymin=2.96787158491106e-06, ymax=0.0269806203211474,
ymode=log,
ytick style={color=black}
]
\addplot [semithick, black, dashed]
table {%
0 0.0174682996585034
1 0.0166917029418983
2 0.0159662418009248
3 0.015269086783519
4 0.0145829427638091
5 0.0138803776353598
6 0.0131050192867406
7 0.0121929560409626
8 0.0111785219341982
9 0.0101515492569888
10 0.0092019377334509
11 0.0083847024361603
12 0.00769865809706971
13 0.00709244787140051
14 0.00654596769163618
15 0.00607052344275871
16 0.00565290632948745
17 0.00528715418477077
18 0.00497537186311092
19 0.00469161985893152
20 0.00443533330326318
21 0.00421532278232917
22 0.00403653572175244
23 0.00388705663317523
24 0.00376473488540796
25 0.0036559987356668
26 0.00355060209949443
27 0.0034493805319471
28 0.00335303273959653
29 0.00326418016629759
30 0.0031843719743847
31 0.00310496880183564
32 0.00302391471927876
33 0.0029389070994057
34 0.00285341816561413
35 0.00276622700084772
36 0.00267770032746739
37 0.00259237988132099
38 0.00251051991114082
39 0.0024311699687587
40 0.00235230666638131
41 0.00227785907316047
42 0.00220777405127137
43 0.00214167342033988
44 0.00207899286988322
45 0.00201959077210745
46 0.00196228567172341
47 0.00190719077953361
48 0.0018542043882519
49 0.00180374014919948
50 0.00175558977651491
51 0.00170939986765717
52 0.00166521620830906
53 0.00162264965188896
54 0.00158173247843507
55 0.00154130932999053
56 0.00150249139430869
57 0.00146471605055964
58 0.00142738854464852
59 0.00139127860314936
60 0.00135686240431454
61 0.00132411740389671
62 0.00129298410482903
63 0.00126319340204439
64 0.00123467428450574
65 0.00120726153430351
66 0.00118080481782812
67 0.00115491587705119
68 0.0011297294611321
69 0.00110483057676802
70 0.00108007081770189
71 0.00105640060548851
72 0.0010339326986899
73 0.00101184637537122
74 0.000990354307691454
75 0.000968920083824099
76 0.000948263006286698
77 0.00092830919576814
78 0.000908720974823307
79 0.000889957446322587
80 0.000871706413184903
81 0.00085426045911019
82 0.000837517650863706
83 0.000821401296570379
84 0.000805460751081899
85 0.000790129822127561
86 0.000775095619475419
87 0.000760374569386268
88 0.000745904757012195
89 0.000731748912579544
90 0.000718039264119952
91 0.000704550627585565
92 0.00069151462261452
93 0.000678572218232532
94 0.000665779623622242
95 0.000653354554577046
96 0.00064122502250008
97 0.00062949235063825
98 0.00061792596546173
99 0.000606536239965649
100 0.000595513989600338
101 0.000584491074027582
102 0.000573651043799472
103 0.000563128073792107
104 0.00055265905274049
105 0.000542404742526204
106 0.000532363632055555
107 0.000522494475617474
108 0.000512776798458292
109 0.000503351102354666
110 0.000494043796967603
111 0.000484803110167631
112 0.000475422515055834
113 0.000466334715526262
114 0.000457265090020531
115 0.000448008099969921
116 0.000438834298106539
117 0.000430119603395451
118 0.000421608658855321
119 0.000413195711594261
120 0.000404984400156572
121 0.000397024544484759
122 0.000389084460689304
123 0.00038136677888212
124 0.000373749110917809
125 0.000366234047126435
126 0.000358496150038579
127 0.000349330948864024
128 0.000339808409531295
129 0.000330557143630061
130 0.000321477241212675
131 0.000312768037860622
132 0.000304809891357061
133 0.000297265912308053
134 0.000290144550676996
135 0.000283338498602848
136 0.000276698675163556
137 0.00027024635475309
138 0.00026407260027983
139 0.000258060525707093
140 0.000252346562270744
141 0.00024690285964668
142 0.000241561004087343
143 0.00023633362481118
144 0.000231185348997021
145 0.000226149670925324
146 0.000221099409332959
147 0.000216109778609419
148 0.000211357913615706
149 0.000206800363221404
150 0.000202340621134001
151 0.000198044062411529
152 0.000193836054052765
153 0.000189576544947334
154 0.000185178293157229
155 0.000180727094575728
156 0.000176427932905288
157 0.000172021006624334
158 0.000167242143206514
159 0.000162744807112258
160 0.000158442845625495
161 0.00015431233660218
162 0.000150108494785084
163 0.000145980180493388
164 0.000142042550932198
165 0.000138299380406604
166 0.00013479883155193
167 0.000131572074063835
168 0.000128510769880519
169 0.00012545213591203
170 0.000122408456975587
171 0.000119176852890479
172 0.000116084528875016
173 0.000113206674605404
174 0.000110319908445344
175 0.000107602590187883
176 0.000105052226098579
177 0.000102679231610392
178 0.000100535376700961
179 9.8438238828491e-05
180 9.62415686899809e-05
181 9.42283020788914e-05
182 9.23251173201023e-05
183 9.0319730073285e-05
184 8.85249014681011e-05
185 8.68398654555458e-05
186 8.50834848842852e-05
187 8.34929914375615e-05
188 8.19505714377278e-05
189 8.03661589091065e-05
190 7.87843622294115e-05
191 7.74203452783562e-05
192 7.59379923340742e-05
193 7.46970380489653e-05
194 7.35127068490771e-05
195 7.21867014266686e-05
196 7.09647773132205e-05
197 6.98013087792326e-05
198 6.85880552955542e-05
199 6.73235168161812e-05
200 6.61948033538473e-05
201 6.50232153624197e-05
202 6.38876980176561e-05
203 6.28231256936118e-05
204 6.16934155130977e-05
205 6.05204282186633e-05
206 5.95149107371074e-05
207 5.85233073256142e-05
208 5.73044229713882e-05
209 5.63053127046742e-05
210 5.53478971504762e-05
211 5.43062585594556e-05
212 5.31842229843371e-05
213 5.19891541291007e-05
214 5.07627428376622e-05
215 4.98181649672691e-05
216 4.89416017803723e-05
217 4.79887713594196e-05
218 4.72503029884663e-05
219 4.64007628609708e-05
220 4.56374231827261e-05
221 4.49228722190775e-05
222 4.41760624762255e-05
223 4.35529262265144e-05
224 4.28879089504064e-05
225 4.23084209311975e-05
226 4.16892264068491e-05
227 4.11446753391664e-05
228 4.05972453189074e-05
229 4.00949285873642e-05
230 3.95563977694802e-05
231 3.90015974259805e-05
232 3.85476099323823e-05
233 3.80304431993883e-05
234 3.75338362895161e-05
235 3.70284267318666e-05
236 3.66128080947492e-05
237 3.61793137046429e-05
238 3.57426504439262e-05
239 3.52486918810957e-05
240 3.4847919103298e-05
241 3.43257115138773e-05
242 3.38390899390917e-05
243 3.34035383384723e-05
244 3.29785371772362e-05
245 3.25529564844107e-05
246 3.21451683458207e-05
247 3.17339489961099e-05
248 3.1337891093941e-05
249 3.10101189092415e-05
250 3.06478880326144e-05
251 3.02943460555316e-05
252 2.99426421357107e-05
253 2.96156968386807e-05
254 2.92976800747857e-05
255 2.89853169274379e-05
256 2.86779608934251e-05
257 2.84170743585221e-05
258 2.81212273315901e-05
259 2.78528437674397e-05
260 2.75913234705771e-05
261 2.73255979458542e-05
262 2.70706783233876e-05
263 2.6811602696325e-05
264 2.65774721018575e-05
265 2.63308433033904e-05
266 2.60903299525717e-05
267 2.58570473056219e-05
268 2.56265585567295e-05
269 2.54095956039868e-05
270 2.51749049041683e-05
271 2.49783292626815e-05
272 2.47474838085682e-05
273 2.45332332120818e-05
274 2.43165919928123e-05
275 2.41076610343616e-05
276 2.3897588036198e-05
277 2.3689678754435e-05
278 2.34964429440265e-05
279 2.32898462307618e-05
280 2.30995259277722e-05
281 2.28949860498207e-05
282 2.27142465145935e-05
283 2.25344446490583e-05
284 2.23461946198822e-05
285 2.21807195917578e-05
286 2.20288519544032e-05
287 2.18561929711258e-05
288 2.16723017042453e-05
289 2.15064642814788e-05
290 2.13319503318266e-05
291 2.11788976899607e-05
292 2.10218696376496e-05
293 2.08856447982686e-05
294 2.07519894819264e-05
295 2.06048780848345e-05
296 2.04659173792809e-05
297 2.03175485387619e-05
298 2.01850235272971e-05
299 2.00549300730657e-05
300 1.99116721929238e-05
301 1.97868557431369e-05
302 1.96788613069654e-05
303 1.95387639236322e-05
304 1.94239361903381e-05
305 1.92984490745007e-05
306 1.91925303076879e-05
307 1.90731551204804e-05
308 1.89677610019368e-05
309 1.88626117618163e-05
310 1.87518606694359e-05
311 1.8672930052227e-05
312 1.85490694901425e-05
313 1.84567095228294e-05
314 1.83549541648287e-05
315 1.82683744112122e-05
316 1.81609060137533e-05
317 1.80699237594395e-05
318 1.79899847511378e-05
319 1.78851278566372e-05
320 1.77929941385102e-05
321 1.77099889427657e-05
322 1.76122569652826e-05
323 1.75315572430446e-05
324 1.74405403198519e-05
325 1.73753643049612e-05
326 1.7283484660624e-05
327 1.72006392117063e-05
328 1.7116215201618e-05
329 1.7047074340748e-05
330 1.69564789680166e-05
331 1.68866016968805e-05
332 1.68050701954969e-05
333 1.67492335026442e-05
334 1.66693465886436e-05
335 1.66024063762427e-05
336 1.65252939654803e-05
337 1.64556390478687e-05
338 1.63836931232619e-05
339 1.63233824186648e-05
340 1.62429579993462e-05
341 1.61820948249847e-05
342 1.61422442996606e-05
343 1.60688580487511e-05
344 1.60058628753679e-05
345 1.59424624685822e-05
346 1.58816082282132e-05
347 1.58259322826093e-05
348 1.57660298485407e-05
349 1.57043620885033e-05
350 1.56500406306159e-05
351 1.5609296985436e-05
352 1.55429892935988e-05
353 1.54965373183602e-05
354 1.54357561293617e-05
355 1.53896277896859e-05
356 1.53330173427169e-05
357 1.52699466506867e-05
358 1.52240988975649e-05
359 1.51705292977811e-05
360 1.51281421500171e-05
361 1.50701658838126e-05
362 1.50332893456284e-05
363 1.49765245238964e-05
364 1.49472422074637e-05
365 1.49047273984848e-05
366 1.4850535318689e-05
367 1.48061914460873e-05
368 1.47645724850043e-05
369 1.47117304858668e-05
370 1.46677043416088e-05
371 1.46273536589803e-05
372 1.45858501145735e-05
373 1.45361062943161e-05
374 1.44970193244331e-05
375 1.44546114448502e-05
376 1.44188719992389e-05
377 1.43781483388505e-05
378 1.43460905747128e-05
379 1.42925792510629e-05
380 1.42489455697836e-05
381 1.4214007583141e-05
382 1.41664196565694e-05
383 1.41215013607643e-05
384 1.4083577174584e-05
385 1.40470031246981e-05
386 1.40088558921031e-05
387 1.39677960904472e-05
388 1.39364654354185e-05
389 1.38963280529869e-05
390 1.38524261441147e-05
391 1.38013707772222e-05
392 1.37828387058309e-05
393 1.37397383405402e-05
394 1.37073727933257e-05
395 1.36809754726491e-05
396 1.36272725269476e-05
397 1.36166064675508e-05
398 1.35805921434168e-05
399 1.35645500033837e-05
400 1.353829616485e-05
401 1.35090340787625e-05
402 1.34810661062135e-05
403 1.3439763163467e-05
404 1.3410306584638e-05
405 1.33862497318782e-05
406 1.33631424715092e-05
407 1.33306051560422e-05
408 1.34061977934419e-05
409 1.33633703978467e-05
410 1.33124240262114e-05
411 1.32614882183343e-05
412 1.3205464005149e-05
413 1.31836754120804e-05
414 1.31511393370379e-05
415 1.31012971975736e-05
416 1.31016725845659e-05
417 1.30547282210722e-05
418 1.30279358319801e-05
419 1.30208214542821e-05
420 1.29776664635275e-05
421 1.29691733115334e-05
422 1.29370692544539e-05
423 1.29031180342784e-05
424 1.28817071146925e-05
425 1.28676252457438e-05
426 1.28288021958925e-05
427 1.28035283569333e-05
428 1.277538702011e-05
429 1.27521351755769e-05
430 1.27159617941824e-05
431 1.26901618766675e-05
432 1.26548069027538e-05
433 1.26355308248879e-05
434 1.25971456286025e-05
435 1.25727406496551e-05
436 1.25298254361958e-05
437 1.25018809353714e-05
438 1.2471386927615e-05
439 1.24417817293984e-05
440 1.2400355723953e-05
441 1.23696565377737e-05
442 1.23478470896027e-05
443 1.23114803439323e-05
444 1.22782743122232e-05
445 1.22390279885115e-05
446 1.22306208548903e-05
447 1.21667453243449e-05
448 1.21371574513321e-05
449 1.21019913747883e-05
450 1.20604241624772e-05
451 1.20194190896053e-05
452 1.19860425297075e-05
453 1.1943136385896e-05
454 1.1904814628691e-05
455 1.18491835870843e-05
456 1.18251859803561e-05
457 1.17905506113875e-05
458 1.17272120052359e-05
459 1.17023469257196e-05
460 1.16545696897535e-05
461 1.16181367827636e-05
462 1.15717176878624e-05
463 1.15342119084566e-05
464 1.15034249746279e-05
465 1.14448367618328e-05
466 1.14141364974951e-05
467 1.13586340826921e-05
468 1.13366334676263e-05
469 1.12836705799396e-05
470 1.12496743702659e-05
471 1.12109186993087e-05
472 1.11529097575544e-05
473 1.1109914388123e-05
474 1.10870681100522e-05
475 1.10123635521236e-05
476 1.10015433452751e-05
477 1.09271099891992e-05
478 1.09063721471653e-05
479 1.08499039663326e-05
480 1.08157261337616e-05
481 1.07707420286193e-05
482 1.07264896054907e-05
483 1.06875869411993e-05
484 1.06424912520373e-05
485 1.0611284014167e-05
486 1.05552436267947e-05
487 1.05287150735656e-05
488 1.04786229250253e-05
489 1.04462122189758e-05
490 1.04087539373882e-05
491 1.03600795602621e-05
492 1.03333270849715e-05
493 1.02797517723269e-05
494 1.02574151355506e-05
495 1.02016557216311e-05
496 1.01738468436391e-05
497 1.01327743343843e-05
498 1.0101706100718e-05
499 1.00693046683753e-05
500 1.00297283991413e-05
501 9.99706231240294e-06
502 9.97253902923489e-06
503 9.92637687827203e-06
504 9.88921064243309e-06
505 9.87099236410272e-06
506 9.8191090519284e-06
507 9.79992867734814e-06
508 9.76238071732916e-06
509 9.73088573306968e-06
510 9.69097130176788e-06
511 9.66708010201323e-06
512 9.62372045239246e-06
513 9.60628083668258e-06
514 9.56566906634704e-06
515 9.53178442410219e-06
516 9.51553167720387e-06
517 9.46314684846516e-06
518 9.45223006010298e-06
519 9.41317858621504e-06
520 9.38171416176026e-06
521 9.36664515230334e-06
522 9.31851988481003e-06
523 9.31130817205528e-06
524 9.27638030842615e-06
525 9.249390970828e-06
526 9.24779305186418e-06
527 9.18996122628579e-06
528 9.2022846976958e-06
529 9.14835113038914e-06
530 9.12683033804651e-06
531 9.13586595210902e-06
532 9.0757104845296e-06
533 9.0647053024584e-06
534 9.02939415658999e-06
535 8.99488190894693e-06
536 8.99454786768006e-06
537 8.93013511699126e-06
538 8.92280913197818e-06
539 8.91612476250359e-06
540 8.83572204887773e-06
541 8.83895840567139e-06
542 8.78751151400792e-06
543 8.74091370547947e-06
544 8.73913552887351e-06
545 8.70557778920311e-06
546 8.64911280662461e-06
547 8.66389203729723e-06
548 8.60710929301289e-06
549 8.55012984246617e-06
550 8.56352553207929e-06
551 8.49312490189269e-06
552 8.47598246417336e-06
553 8.46309253826893e-06
554 8.3954508654227e-06
555 8.40104493321503e-06
556 8.36894761632734e-06
557 8.31726128225352e-06
558 8.31962495598249e-06
559 8.29419086334493e-06
560 8.23612470154944e-06
561 8.25279187552375e-06
562 8.23244142340862e-06
563 8.18047730918869e-06
564 8.20022733849513e-06
565 8.16601545453044e-06
566 8.13845985850326e-06
567 8.14656289988003e-06
568 8.11413435802821e-06
569 8.0906996672292e-06
570 8.09359121148323e-06
571 8.05597215122744e-06
572 8.05961267917599e-06
573 8.02234182963046e-06
574 8.02338165499517e-06
575 7.99005803374386e-06
576 7.96406373911041e-06
577 7.94209780660693e-06
578 7.90618443494151e-06
579 7.87543142342917e-06
580 7.8242883537194e-06
581 7.7825253261693e-06
582 7.780224879822e-06
583 7.70803274378895e-06
584 7.66240252884909e-06
585 7.64839899046466e-06
586 7.62384172189878e-06
587 7.57237978325775e-06
588 7.58821030408946e-06
589 7.61417994329244e-06
590 7.561612085559e-06
591 7.55858188015296e-06
592 7.59864130115556e-06
593 7.57798716963243e-06
594 7.49583339550164e-06
595 7.47320912782012e-06
596 7.45211388322731e-06
597 7.39198419454773e-06
598 7.32162024652538e-06
599 7.29731075854029e-06
600 7.31449981997145e-06
601 7.28332348272354e-06
602 7.3107444435172e-06
603 7.42702982370336e-06
604 7.44957609719044e-06
605 7.41906148271632e-06
606 7.49096760714607e-06
607 7.37071190162752e-06
608 7.33706385905308e-06
609 7.33238231286071e-06
610 7.16828334744457e-06
611 7.25694042516545e-06
612 7.30805888542185e-06
613 7.1213321153063e-06
614 7.26226141850236e-06
615 7.4593771267617e-06
616 7.22716362780429e-06
617 7.31170178241286e-06
618 7.61475337134401e-06
619 7.23761993612348e-06
620 7.24317649834078e-06
621 7.51010507259176e-06
622 7.14052023682854e-06
623 7.08014413801361e-06
624 7.3285618120561e-06
625 7.06242201659046e-06
626 6.99952755409128e-06
627 7.20220186849196e-06
628 7.04649070366881e-06
629 7.06842127298421e-06
630 7.0672261134469e-06
631 6.93481905326976e-06
632 7.33416162431236e-06
633 7.09378262785804e-06
634 6.82584067960429e-06
635 7.18769216744714e-06
636 7.00348377141724e-06
637 6.81966335278061e-06
638 6.94428357053689e-06
639 6.83307741783379e-06
640 6.99732747044823e-06
641 6.91152764600855e-06
642 6.81954732113799e-06
643 6.94443212577445e-06
644 6.88467122779424e-06
645 6.9937054905933e-06
646 6.93245251852381e-06
647 6.90070195628889e-06
648 6.96660699218976e-06
649 6.88663655896982e-06
650 6.99540891668793e-06
651 6.88684612540347e-06
652 6.86117191131541e-06
653 6.83141380231067e-06
654 6.75151140086538e-06
655 6.80596299079683e-06
656 6.72253186617677e-06
657 6.69971640695349e-06
658 6.68326541174158e-06
659 6.61127226277147e-06
660 6.64539962658833e-06
661 6.58889337242646e-06
662 6.55527808090389e-06
663 6.56631004283814e-06
664 6.51112824577582e-06
665 6.51122629002909e-06
666 6.51427226790874e-06
667 6.4498908937255e-06
668 6.47410817927047e-06
669 6.45349121593947e-06
670 6.40889695546765e-06
671 6.42476013861665e-06
672 6.41751349460956e-06
673 6.37447480344305e-06
674 6.38062669795769e-06
675 6.39400090504672e-06
676 6.34658093903728e-06
677 6.34314612953796e-06
678 6.37922044957606e-06
679 6.3281582430158e-06
680 6.30926371722462e-06
681 6.36234827730564e-06
682 6.32384241078254e-06
683 6.27836630691381e-06
684 6.34210967750612e-06
685 6.33819742860975e-06
686 6.27426244845242e-06
687 6.3054576880453e-06
688 6.42620467891999e-06
689 6.33345920079964e-06
690 6.29943341791048e-06
691 6.61937870513087e-06
692 6.53013174637188e-06
693 6.46835790631822e-06
694 7.10089595289635e-06
695 7.41259085331944e-06
696 7.84399453201187e-06
697 8.98206298284034e-06
698 6.7807635946715e-06
699 6.68910852892799e-06
700 6.87671761860506e-06
701 6.33200236752063e-06
702 6.56169372916654e-06
703 6.2631075415584e-06
704 6.21104907432696e-06
705 6.36484293487644e-06
706 6.20363223349102e-06
707 6.4332312546371e-06
708 6.21838079821468e-06
709 6.42961281393899e-06
710 6.21605212149795e-06
711 6.33729374154013e-06
712 6.20488844271794e-06
713 6.25174011755977e-06
714 6.24080466590099e-06
715 6.23959842761812e-06
716 6.31357005212441e-06
717 6.29053482373992e-06
718 6.38754372533912e-06
719 6.34738498093285e-06
720 6.39920640439079e-06
721 6.38354395828683e-06
722 6.34050669565817e-06
723 6.42174386167316e-06
724 6.31762583247486e-06
725 6.34444064012696e-06
726 6.27624173017705e-06
727 6.20988560925961e-06
728 6.24041613299375e-06
729 6.14530176275274e-06
730 6.16886983995785e-06
731 6.13265190066437e-06
732 6.06078395477827e-06
733 6.13379266878322e-06
734 6.04566154082525e-06
735 6.05511030641812e-06
736 6.08155123453524e-06
737 5.98657393125618e-06
738 6.07091887227379e-06
739 6.04502865093198e-06
740 5.95137277559915e-06
741 6.10349760385431e-06
742 6.02671505767899e-06
743 5.93847427787092e-06
744 6.13979039625241e-06
745 6.06042157367398e-06
746 5.95170033314274e-06
747 6.24968277406065e-06
748 6.29019665051439e-06
749 6.21884743357004e-06
750 6.68491676747513e-06
751 6.98775103914707e-06
752 6.814062273066e-06
753 7.32456973680178e-06
754 6.61047029641357e-06
755 6.80090360621888e-06
756 8.33170075990525e-06
757 6.61385682244192e-06
758 6.91690460679034e-06
759 6.12927214223985e-06
760 6.3716169667705e-06
761 5.99218477560781e-06
762 6.24050359747082e-06
763 5.96324532099735e-06
764 6.25882435971274e-06
765 6.00005341102466e-06
766 6.4191603027855e-06
767 6.10310656292551e-06
768 6.62930503768555e-06
769 6.17456920001119e-06
770 6.71116857045529e-06
771 6.19275405388392e-06
772 6.74952495767836e-06
773 6.1918419335602e-06
774 6.81566180046585e-06
775 6.25825391468299e-06
776 6.97459580595349e-06
777 6.4065522351886e-06
778 7.25569686395734e-06
779 6.55881212648438e-06
780 7.54749353024219e-06
781 6.71671683255548e-06
782 7.70799635134735e-06
783 6.70971440981313e-06
784 7.48983277098958e-06
785 6.42507694519762e-06
786 6.98707269456121e-06
787 6.10323765457693e-06
788 6.53592321526474e-06
789 5.95359492340519e-06
790 6.28496085593286e-06
791 5.94880074750864e-06
792 6.21632418610968e-06
793 5.9922452898438e-06
794 6.25987880993129e-06
795 6.0704586056573e-06
796 6.35589066319596e-06
797 6.1683621098152e-06
798 6.48543033282762e-06
799 6.29338932517723e-06
800 6.64303399712681e-06
801 6.46650360752038e-06
802 6.80169987394419e-06
803 6.6588483620296e-06
804 6.89474454412257e-06
805 6.77312086563506e-06
806 6.86094038801255e-06
807 6.81607360827896e-06
808 6.83108142026458e-06
809 6.87718605836898e-06
810 6.84299815225981e-06
811 6.93896129827509e-06
812 6.92904741425408e-06
813 7.06566952940121e-06
814 7.096961578636e-06
815 7.22544514909686e-06
816 7.26876655502036e-06
817 7.38726805820603e-06
818 7.44006512633753e-06
819 7.54643502513634e-06
820 7.54423937878411e-06
821 7.63695625608744e-06
822 7.60939208578181e-06
823 7.53861184737536e-06
824 7.42471006152847e-06
825 7.35926895872614e-06
826 7.20781291933641e-06
827 7.04138096644952e-06
828 6.91243522059608e-06
829 6.769255152193e-06
830 6.64609533500803e-06
831 6.52674890318228e-06
832 6.43867369043601e-06
833 6.36472349473116e-06
834 6.3074578531188e-06
835 6.25725640928873e-06
836 6.21833398560361e-06
837 6.1790347738247e-06
838 6.14947140415746e-06
839 6.1219853076562e-06
840 6.10978450240988e-06
841 6.09103142007981e-06
842 6.07758723463088e-06
843 6.06834389657238e-06
844 6.04777644432758e-06
845 6.04542538672204e-06
846 6.04536569434252e-06
847 6.03193037386658e-06
848 6.03633474983122e-06
849 6.03418801081723e-06
850 6.0287489647929e-06
851 6.0270410809863e-06
852 6.03065569259979e-06
853 6.02480646799165e-06
854 6.02651437441905e-06
855 6.03321914967141e-06
856 6.02593279645965e-06
857 6.03276115445674e-06
858 6.02832319716168e-06
859 6.03073672285175e-06
860 6.0346813186337e-06
861 6.02995061621189e-06
862 6.04402397485874e-06
863 6.03182585023031e-06
864 6.03659910308404e-06
865 6.03759858714727e-06
866 6.03830816359174e-06
867 6.04250858240833e-06
868 6.03186716435777e-06
869 6.0420746516316e-06
870 6.03576048367922e-06
871 6.02750531549651e-06
872 6.04140035553695e-06
873 6.02749730662516e-06
874 6.03144696894979e-06
875 6.02586256770281e-06
876 6.03649099598438e-06
877 6.02489919659249e-06
878 6.02616633792186e-06
879 6.02888583115663e-06
880 6.02011454589428e-06
881 6.02781520964244e-06
882 6.02285271116172e-06
883 6.02199691472627e-06
884 6.02414899999928e-06
885 6.02336332275813e-06
886 6.01899199224702e-06
887 6.01541448651093e-06
888 6.02638861221458e-06
889 6.01223502748587e-06
890 6.03079573173806e-06
891 6.02099073702175e-06
892 6.01885593548308e-06
893 6.02790017136499e-06
894 6.02111277594672e-06
895 6.00841570675033e-06
896 6.02430084451194e-06
897 6.0302800951352e-06
898 6.01614162730729e-06
899 6.03223160532389e-06
900 6.02154658509591e-06
901 6.0293088974088e-06
902 6.03065245707076e-06
903 6.0297832634496e-06
904 6.04283480977463e-06
905 6.01433932439447e-06
906 6.04797680765117e-06
907 6.01932415899035e-06
908 6.04500108831573e-06
909 6.02259675050237e-06
910 6.04224205695862e-06
911 6.03676277853007e-06
912 6.0478036192726e-06
913 6.03888319789751e-06
914 6.05489365614444e-06
915 6.04514133256207e-06
916 6.05736719669499e-06
917 6.05220869234822e-06
918 6.06829554532511e-06
919 6.04828189800237e-06
920 6.07853028084752e-06
921 6.05820227254136e-06
922 6.08498505107397e-06
923 6.06005964004118e-06
924 6.09680200970347e-06
925 6.05904252220949e-06
926 6.10151853208403e-06
927 6.06163899878208e-06
928 6.12284144874305e-06
929 6.06266862422276e-06
930 6.12366944068002e-06
931 6.08872344026121e-06
932 6.10502665252606e-06
933 6.09634598489246e-06
934 6.12066434124708e-06
935 6.09811626858052e-06
936 6.12153893754657e-06
937 6.11994950001186e-06
938 6.11984241482611e-06
939 6.12289289207389e-06
940 6.14170148947148e-06
941 6.15159713254109e-06
942 6.1348648149432e-06
943 6.13868387842375e-06
944 6.13829724403579e-06
945 6.15020507676184e-06
946 6.1450936510331e-06
947 6.14166379925095e-06
948 6.16286512623805e-06
949 6.14179615017425e-06
950 6.17419220955056e-06
951 6.16888925012872e-06
952 6.19330876392124e-06
953 6.17673845561045e-06
954 6.18661770385816e-06
955 6.1827354935762e-06
956 6.2016940885564e-06
957 6.20868540711743e-06
958 6.21007401637946e-06
959 6.20224767805527e-06
960 6.22590904525134e-06
961 6.21870050143176e-06
962 6.22395804442799e-06
963 6.2181386640446e-06
964 6.2241453866472e-06
965 6.22608396731516e-06
966 6.21094918379041e-06
967 6.22804872621045e-06
968 6.19945649468967e-06
969 6.19512612455255e-06
970 6.20557022909209e-06
971 6.1607266640949e-06
972 6.17023805126712e-06
973 6.15741660899682e-06
974 6.13781837621279e-06
975 6.13430612801999e-06
976 6.10556142336007e-06
977 6.0918246578212e-06
978 6.08124759167162e-06
979 6.04546544429198e-06
980 6.03525997150525e-06
981 6.01306516218844e-06
982 5.99518373007348e-06
983 5.98950084384765e-06
984 5.94880398620945e-06
985 5.94952004847492e-06
986 5.91335289793962e-06
987 5.9155779719156e-06
988 5.88297916094836e-06
989 5.87039647085776e-06
990 5.84479665793072e-06
991 5.83955758581383e-06
992 5.81276512686275e-06
993 5.8126976857636e-06
994 5.79328034802482e-06
995 5.77014204120155e-06
996 5.77619750790557e-06
997 5.7423816046735e-06
998 5.74836157867555e-06
999 5.7258258898692e-06
1000 5.7160579927897e-06
1001 5.70750483390961e-06
1002 5.70352973301778e-06
1003 5.68501298334139e-06
1004 5.68362873453574e-06
1005 5.67205873401088e-06
1006 5.6608658965214e-06
1007 5.66049776884609e-06
1008 5.64527397618447e-06
1009 5.64044493119802e-06
1010 5.64200315656857e-06
1011 5.62161822779153e-06
1012 5.63142885083871e-06
1013 5.61256967940694e-06
1014 5.61448859297049e-06
1015 5.60211254881969e-06
1016 5.60167477713105e-06
1017 5.59378798731702e-06
1018 5.59003459491317e-06
1019 5.58956416904738e-06
1020 5.57662323270711e-06
1021 5.5772017965108e-06
1022 5.56385411101687e-06
1023 5.57417595600486e-06
1024 5.55659345158176e-06
1025 5.56749110408515e-06
1026 5.55150862447436e-06
1027 5.55102020001691e-06
1028 5.54866379598284e-06
1029 5.54704152788725e-06
1030 5.55162682822077e-06
1031 5.53255247478677e-06
1032 5.5554839234727e-06
1033 5.53122461917552e-06
1034 5.54039169615272e-06
1035 5.53463700198432e-06
1036 5.53124689313718e-06
1037 5.53641412182057e-06
1038 5.53479263334485e-06
1039 5.52344522203902e-06
1040 5.54509857454702e-06
1041 5.50694517214408e-06
1042 5.53906428926978e-06
1043 5.51902316536909e-06
1044 5.51623026900413e-06
1045 5.5519699153121e-06
1046 5.49109702279664e-06
1047 5.528425219117e-06
1048 5.54185791640029e-06
1049 5.47450585440361e-06
1050 5.54871839899259e-06
1051 5.48998246801779e-06
1052 5.49026670540236e-06
1053 5.59001693469574e-06
1054 5.45271183462884e-06
1055 5.50996641250325e-06
1056 5.66163335606428e-06
1057 5.43296712708034e-06
1058 5.57839403941471e-06
1059 5.84816260404912e-06
1060 5.5611153960129e-06
1061 5.62028728046514e-06
1062 5.93404852284275e-06
1063 5.69826771664692e-06
1064 5.7933435778711e-06
1065 6.13510338932156e-06
1066 6.03995458797551e-06
1067 5.90898109149019e-06
1068 5.8275736195487e-06
1069 5.88185019366255e-06
1070 6.06436759740834e-06
1071 5.7238343568074e-06
1072 5.58681593197972e-06
1073 5.62199026703299e-06
1074 5.55899710516893e-06
1075 5.74796987441285e-06
1076 5.5525952543849e-06
1077 5.57477179616894e-06
1078 5.51401052323741e-06
1079 5.48935733450335e-06
1080 5.55719458129146e-06
1081 5.4922986230322e-06
1082 5.54278663400024e-06
1083 5.50097546207619e-06
1084 5.49573517048055e-06
1085 5.53040194795762e-06
1086 5.48915693884659e-06
1087 5.55045053956705e-06
1088 5.50078026872268e-06
1089 5.51917280292894e-06
1090 5.53263619230312e-06
1091 5.50321291421674e-06
1092 5.56212038411869e-06
1093 5.51712571155123e-06
1094 5.55551441122959e-06
1095 5.5468564708211e-06
1096 5.53129312377347e-06
1097 5.58167972206067e-06
1098 5.54440076225104e-06
1099 5.57534407099805e-06
1100 5.58341534635591e-06
1101 5.56242874012687e-06
1102 5.61787980289362e-06
1103 5.57676494166282e-06
1104 5.61093719028116e-06
1105 5.6032353815391e-06
1106 5.59493596319071e-06
1107 5.63119467726986e-06
1108 5.58793370798354e-06
1109 5.64965350444615e-06
1110 5.58649619915792e-06
1111 5.62953417790782e-06
1112 5.59107293974287e-06
1113 5.61390900340894e-06
1114 5.58584959280844e-06
1115 5.60416029244336e-06
1116 5.59207645326226e-06
1117 5.58673242524434e-06
1118 5.59299787492913e-06
1119 5.58393367335175e-06
1120 5.58897144821283e-06
1121 5.58351492880635e-06
1122 5.59240818311083e-06
1123 5.57910978866153e-06
1124 5.59727900663627e-06
1125 5.57885440601873e-06
1126 5.5887059321866e-06
1127 5.60957102486951e-06
1128 5.5846393493867e-06
1129 5.62484064397685e-06
1130 5.58634268198079e-06
1131 5.63664475902663e-06
1132 5.60599616516015e-06
1133 5.65303137245522e-06
1134 5.62489821197105e-06
1135 5.67061937532183e-06
1136 5.64752619961354e-06
1137 5.69594602982417e-06
1138 5.65516582288591e-06
1139 5.72811649835763e-06
1140 5.67943820790312e-06
1141 5.74128366059685e-06
1142 5.70088298112886e-06
1143 5.78552778846629e-06
1144 5.72871246968894e-06
1145 5.80528961403894e-06
1146 5.74801883463697e-06
1147 5.87765119771063e-06
1148 5.76464422752282e-06
1149 5.92182692691422e-06
1150 5.8024723193196e-06
1151 5.93741537571109e-06
1152 5.84998829886252e-06
1153 5.97217728905855e-06
1154 5.90697053735382e-06
1155 6.00471725548864e-06
1156 5.98937332162139e-06
1157 6.08513690202127e-06
1158 6.04964560293965e-06
1159 6.1355843870046e-06
1160 6.12240029804104e-06
1161 6.21877169797898e-06
1162 6.20070830277718e-06
1163 6.28267994104935e-06
1164 6.24308895303329e-06
1165 6.34467800481944e-06
1166 6.3097774172984e-06
1167 6.34386949450042e-06
1168 6.33292353372937e-06
1169 6.31100641640609e-06
1170 6.28789963809584e-06
1171 6.21346389369123e-06
1172 6.1623028865864e-06
1173 6.06524417477223e-06
1174 6.00287745330035e-06
1175 5.90957740284685e-06
1176 5.85103310181569e-06
1177 5.763405635762e-06
1178 5.73069484267695e-06
1179 5.68259368380542e-06
1180 5.63385728224471e-06
1181 5.59835482454187e-06
1182 5.56873800112409e-06
1183 5.54035244256743e-06
1184 5.52824028153743e-06
1185 5.49442077912837e-06
1186 5.49302131355294e-06
1187 5.47036730239784e-06
1188 5.46562078332602e-06
1189 5.45374792642534e-06
1190 5.4337716955378e-06
1191 5.44455011799203e-06
1192 5.41217273365429e-06
1193 5.42849392952834e-06
1194 5.40475560092871e-06
1195 5.4170855621378e-06
1196 5.3984551545857e-06
1197 5.39546627807561e-06
1198 5.38755942486376e-06
1199 5.37722185252905e-06
1200 5.37753184800487e-06
1201 5.36604008382493e-06
1202 5.36674562310738e-06
1203 5.35343769951281e-06
1204 5.35734951055061e-06
1205 5.34623834463317e-06
1206 5.33786506566258e-06
1207 5.33364820080938e-06
1208 5.32191250186332e-06
1209 5.32105559661322e-06
1210 5.3121168108494e-06
1211 5.30379920950577e-06
1212 5.31081998872657e-06
1213 5.29891811174358e-06
1214 5.2935412619932e-06
1215 5.28929775793575e-06
1216 5.28471555719015e-06
1217 5.27174879999621e-06
1218 5.28197088156228e-06
1219 5.25339567101695e-06
1220 5.27255403213249e-06
1221 5.2565023507677e-06
1222 5.25414000730842e-06
1223 5.24813824922993e-06
1224 5.2434717204054e-06
1225 5.24034643646807e-06
1226 5.23953969454872e-06
1227 5.23788438484596e-06
1228 5.22437397601586e-06
1229 5.23285042679233e-06
1230 5.22943385391602e-06
1231 5.216179116968e-06
1232 5.22082416496594e-06
1233 5.21679902735285e-06
1234 5.2164686325922e-06
1235 5.21065688618809e-06
1236 5.2080498411698e-06
1237 5.22404395608816e-06
1238 5.19286246235154e-06
1239 5.21419638334131e-06
1240 5.24421952216057e-06
1241 5.18299973583519e-06
1242 5.2296445538738e-06
1243 5.29751986508072e-06
1244 5.19077279208919e-06
1245 5.34313371151641e-06
1246 5.59440937867819e-06
1247 5.39105210474282e-06
1248 5.84285700642047e-06
1249 6.99047910019827e-06
1250 6.43361171420525e-06
1251 5.86328689637666e-06
1252 5.79480452766391e-06
1253 5.86160500344493e-06
1254 5.96431570065665e-06
1255 5.34134023599949e-06
1256 5.27868827904785e-06
1257 5.11339200151029e-06
1258 5.17137639676779e-06
1259 5.08563863645407e-06
1260 5.12387996791129e-06
1261 5.07833163412776e-06
1262 5.09922082036964e-06
1263 5.08304971047675e-06
1264 5.08197474379446e-06
1265 5.09440426238458e-06
1266 5.07722640211085e-06
1267 5.09125312066032e-06
1268 5.10201163458377e-06
1269 5.07622887023886e-06
1270 5.10403470482452e-06
1271 5.09397982793386e-06
1272 5.07783581146543e-06
1273 5.11888563883339e-06
1274 5.11320646767339e-06
1275 5.08046002454382e-06
1276 5.22385010242341e-06
1277 5.30698333525545e-06
1278 5.2248434850886e-06
1279 5.7766969880675e-06
1280 6.99420605268622e-06
1281 6.65476387566913e-06
1282 5.91303384702676e-06
1283 5.43199254368547e-06
1284 5.69088389751218e-06
1285 5.99941747595384e-06
1286 5.53436598680579e-06
1287 5.51160161898878e-06
1288 5.24386637346135e-06
1289 5.36188780034074e-06
1290 5.19854108060387e-06
1291 5.32434762884308e-06
1292 5.16682281078904e-06
1293 5.27114944318439e-06
1294 5.12076816419575e-06
1295 5.21192991668287e-06
1296 5.07200660025599e-06
1297 5.13314680049935e-06
1298 5.04207979132433e-06
1299 5.0790015744719e-06
1300 5.05871112058482e-06
1301 5.05637810059417e-06
1302 5.0887284943344e-06
1303 5.04399012701422e-06
1304 5.07609099889243e-06
1305 5.09398382467035e-06
1306 5.04649866268515e-06
1307 5.12303207912202e-06
1308 5.08369623602746e-06
1309 5.09153240349551e-06
1310 5.27210776264271e-06
1311 5.10153840592881e-06
1312 5.33224652159906e-06
1313 5.79271510089816e-06
1314 5.17275052071839e-06
1315 5.61353746889204e-06
1316 6.42735131303171e-06
1317 5.89006298383813e-06
1318 5.70937564536622e-06
1319 5.19299928954849e-06
1320 5.18206937009876e-06
1321 5.11626803463764e-06
1322 5.14604459895533e-06
1323 5.18569716778558e-06
1324 5.15189366338463e-06
1325 5.19706664479431e-06
1326 5.16536299088129e-06
1327 5.19956926025428e-06
1328 5.19147915225985e-06
1329 5.21379369539793e-06
1330 5.2259196370013e-06
1331 5.24334419082292e-06
1332 5.25969661267106e-06
1333 5.26546095901039e-06
1334 5.31323267498869e-06
1335 5.32524830983313e-06
1336 5.378888457778e-06
1337 5.3952694645972e-06
1338 5.46180206801954e-06
1339 5.52421680565288e-06
1340 5.55624527604111e-06
1341 5.61522877728948e-06
1342 5.64543845803041e-06
1343 5.6822541339408e-06
1344 5.72812778053738e-06
1345 5.72164524541166e-06
1346 5.75103280239862e-06
1347 5.73290118184127e-06
1348 5.75487248064048e-06
1349 5.73957992814418e-06
1350 5.76374932421004e-06
1351 5.74616030485058e-06
1352 5.77449067863501e-06
1353 5.75744070153236e-06
1354 5.78798345680316e-06
1355 5.77348033913101e-06
1356 5.78621825657583e-06
1357 5.78695136909237e-06
1358 5.795452580717e-06
1359 5.79121731716378e-06
1360 5.79866968841818e-06
1361 5.79893881326363e-06
1362 5.79896369766403e-06
1363 5.79064060944344e-06
1364 5.77718015805519e-06
1365 5.75829666317255e-06
1366 5.75022773752279e-06
1367 5.72413088098808e-06
1368 5.69722352197984e-06
1369 5.68291949711661e-06
1370 5.65353544479136e-06
1371 5.63308341544789e-06
1372 5.60023673712808e-06
1373 5.56802643536983e-06
1374 5.54403394475787e-06
1375 5.50860679340331e-06
1376 5.48168159755082e-06
1377 5.44959969235299e-06
1378 5.41813302251497e-06
1379 5.38150100764449e-06
1380 5.35952180056302e-06
1381 5.31996764959765e-06
1382 5.3033682493271e-06
1383 5.25920894668252e-06
1384 5.24849613694057e-06
1385 5.20789392944795e-06
1386 5.19511461490388e-06
1387 5.16572749450204e-06
1388 5.1531070786974e-06
1389 5.12720901679796e-06
1390 5.11877197673766e-06
1391 5.0962637372328e-06
1392 5.08312157067705e-06
1393 5.07318330236826e-06
1394 5.05660180267174e-06
1395 5.05369113082388e-06
1396 5.03172813960283e-06
1397 5.03159692356479e-06
1398 5.01515424269992e-06
1399 5.0128656183283e-06
1400 5.00528978210802e-06
1401 4.99565300707684e-06
1402 4.99043458164115e-06
1403 4.9835353524018e-06
1404 4.98282216743282e-06
1405 4.96732111846736e-06
1406 4.97442855927101e-06
1407 4.95849073724316e-06
1408 4.96805537127673e-06
1409 4.95226576811445e-06
1410 4.96017121146821e-06
1411 4.94709172785138e-06
1412 4.95149250537611e-06
1413 4.94585877006039e-06
1414 4.95403942877182e-06
1415 4.94026146147415e-06
1416 4.95164006192664e-06
1417 4.93516322800009e-06
1418 4.95415493411929e-06
1419 4.93411028305965e-06
1420 4.95256482954806e-06
1421 4.93248110242659e-06
1422 4.95668373980433e-06
1423 4.9388351674555e-06
1424 4.95632019076688e-06
1425 4.94029632657411e-06
1426 4.9596610410456e-06
1427 4.96016295923251e-06
1428 4.96462193722122e-06
1429 4.97407290086203e-06
1430 4.97495341686763e-06
1431 4.98618109454237e-06
1432 4.99121910833887e-06
1433 5.01001837481443e-06
1434 5.02063397866829e-06
1435 5.03609889112352e-06
1436 5.04854880667904e-06
1437 5.06727248217429e-06
1438 5.08425689164905e-06
1439 5.09576315715693e-06
1440 5.111112585087e-06
1441 5.12245464541505e-06
1442 5.13256503480077e-06
1443 5.14216591338988e-06
1444 5.14137195664459e-06
1445 5.14347764435006e-06
1446 5.14383064134443e-06
1447 5.13900895546509e-06
1448 5.14213369023789e-06
1449 5.13809314526501e-06
1450 5.13609813870516e-06
1451 5.1174535456384e-06
1452 5.10098817387613e-06
1453 5.091960134757e-06
1454 5.05813163041366e-06
1455 5.04528032118739e-06
1456 5.02321331407354e-06
1457 5.00768857276794e-06
1458 4.98363049117258e-06
1459 4.96762690084936e-06
1460 4.95521465966803e-06
1461 4.93291950824643e-06
1462 4.9261187767448e-06
1463 4.91231542131432e-06
1464 4.89813576179377e-06
1465 4.90143904903747e-06
1466 4.88382370782234e-06
1467 4.88213445374146e-06
1468 4.87829984783782e-06
1469 4.87202967435949e-06
1470 4.8746851939288e-06
1471 4.85988172505103e-06
1472 4.86683837061329e-06
1473 4.8570526528896e-06
1474 4.84995518196769e-06
1475 4.86648196711513e-06
1476 4.844347493671e-06
1477 4.86392744237442e-06
1478 4.85910305013645e-06
1479 4.84662797613434e-06
1480 4.86634694921747e-06
1481 4.85522961446157e-06
1482 4.84407091324988e-06
1483 4.87251826009169e-06
1484 4.85474336596318e-06
1485 4.84118156737587e-06
1486 4.89186772504991e-06
1487 4.84427752801288e-06
1488 4.85625777738811e-06
1489 4.93105471566673e-06
1490 4.87134549875438e-06
1491 4.89197970730056e-06
1492 5.07106818305989e-06
1493 5.05054201523144e-06
1494 5.14201366000642e-06
1495 5.58462692249527e-06
1496 6.01969004918506e-06
1497 6.43001178736945e-06
1498 5.58107000778012e-06
1499 5.35203284050768e-06
1500 6.09437547103241e-06
1501 5.42522873082956e-06
1502 5.7217921365249e-06
1503 5.08192859785853e-06
1504 5.51248011565692e-06
1505 5.13058704567802e-06
1506 5.53304347651411e-06
1507 5.1915849066228e-06
1508 5.54612451946022e-06
1509 5.21232675842276e-06
1510 5.50577476525191e-06
1511 5.21583766366345e-06
1512 5.42643416482269e-06
1513 5.20530725958121e-06
1514 5.3589176253718e-06
1515 5.19932915673484e-06
1516 5.29946353773083e-06
1517 5.19426440398507e-06
1518 5.25119457298729e-06
1519 5.17780075940603e-06
1520 5.2198202309495e-06
1521 5.17315366920469e-06
1522 5.19774156308822e-06
1523 5.16179577159562e-06
1524 5.17969016774262e-06
1525 5.15000250092647e-06
1526 5.16925690256063e-06
1527 5.14036454183657e-06
1528 5.14928168801913e-06
1529 5.13140334776205e-06
1530 5.13953256678373e-06
1531 5.11463142621061e-06
1532 5.13102433058161e-06
1533 5.10055192067259e-06
1534 5.1113312864733e-06
1535 5.08613500693646e-06
1536 5.08936134735086e-06
1537 5.07457887368771e-06
1538 5.0751157951977e-06
1539 5.06335694384624e-06
1540 5.05615369267187e-06
1541 5.03864558922195e-06
1542 5.04084312986632e-06
1543 5.0227081865541e-06
1544 5.02032649248465e-06
1545 4.99888243057439e-06
1546 5.00624448568495e-06
1547 4.97573205973878e-06
1548 4.99163282174873e-06
1549 4.95546421763087e-06
1550 4.97416296253486e-06
1551 4.93710772487958e-06
1552 4.94385025317365e-06
1553 4.92854527095357e-06
1554 4.92787727156911e-06
1555 4.91114128383632e-06
1556 4.91510558930035e-06
1557 4.8900353240788e-06
1558 4.90327903292784e-06
1559 4.87229552352737e-06
1560 4.88734426408357e-06
1561 4.86230329990863e-06
1562 4.87408356890856e-06
1563 4.84627436593641e-06
1564 4.85823835368949e-06
1565 4.83318070486555e-06
1566 4.851390500542e-06
1567 4.8256940446312e-06
1568 4.84415161018572e-06
1569 4.81375762746803e-06
1570 4.84563332958188e-06
1571 4.804484629823e-06
1572 4.83623319146915e-06
1573 4.79863388634291e-06
1574 4.82608449139254e-06
1575 4.7989385633196e-06
1576 4.82042916142352e-06
1577 4.7962986725093e-06
1578 4.81533156021475e-06
1579 4.79261434146224e-06
1580 4.80826968271954e-06
1581 4.79910245187347e-06
1582 4.80510178400254e-06
1583 4.80423545846746e-06
1584 4.78906608913378e-06
1585 4.81631901537904e-06
1586 4.78514480624446e-06
1587 4.79530794461884e-06
1588 4.78860257459909e-06
1589 4.77958250206774e-06
1590 4.8022350674162e-06
1591 4.76510490293215e-06
1592 4.78290335052041e-06
1593 4.78032629762728e-06
1594 4.76041830554962e-06
1595 4.7827750295754e-06
1596 4.75952386090844e-06
1597 4.7583252239605e-06
1598 4.77512029180087e-06
1599 4.75343834910693e-06
1600 4.75070640769122e-06
1601 4.76970780426611e-06
1602 4.75286255294173e-06
1603 4.76101143861605e-06
1604 4.7740945061499e-06
1605 4.77075397089689e-06
1606 4.7658142104599e-06
1607 4.7937692805143e-06
1608 4.79323973216689e-06
1609 4.77912031571605e-06
1610 4.82291590087769e-06
1611 4.85065503921019e-06
1612 4.80898567433395e-06
1613 4.8565746485466e-06
1614 4.89996494336419e-06
1615 4.86602009545985e-06
1616 4.90934225567992e-06
1617 4.94529745684379e-06
1618 4.89691454904697e-06
1619 4.92374397215075e-06
1620 4.94731687469331e-06
1621 4.89146180521927e-06
1622 4.884107550838e-06
1623 4.86931797143731e-06
1624 4.82271648029391e-06
1625 4.80611033868881e-06
1626 4.77878807092692e-06
1627 4.74112105405114e-06
1628 4.72319616277679e-06
1629 4.70265133382646e-06
1630 4.69938043906303e-06
1631 4.6865825780827e-06
1632 4.69982272574291e-06
1633 4.73297185971523e-06
1634 4.76965372365799e-06
1635 4.83428874389273e-06
1636 4.88533419492375e-06
1637 4.9304290581609e-06
1638 4.92675313022297e-06
1639 4.92034083201954e-06
1640 4.91140328405754e-06
1641 4.83324515539196e-06
1642 4.778695147683e-06
1643 4.74081753121036e-06
1644 4.70114059094293e-06
1645 4.6981353645043e-06
1646 4.7794097635059e-06
1647 4.87367739412719e-06
1648 4.84772546395399e-06
1649 4.92108041304961e-06
1650 4.94154129440292e-06
1651 4.82149169283918e-06
1652 5.17032685906659e-06
1653 4.96680009098649e-06
1654 4.972054338137e-06
1655 5.84102765397473e-06
1656 5.24651282729781e-06
1657 5.17651327235669e-06
1658 5.48148453176115e-06
1659 5.30120210718721e-06
1660 6.02613526696244e-06
1661 4.9810863470353e-06
1662 5.02169552492563e-06
1663 4.65200145632763e-06
1664 4.6466548184568e-06
1665 4.64156762562725e-06
1666 4.61193444955621e-06
1667 4.64558602608263e-06
1668 4.61369466096622e-06
1669 4.63920687644334e-06
1670 4.63302401612723e-06
1671 4.63032956924271e-06
1672 4.66535321472962e-06
1673 4.6582598788121e-06
1674 4.67692397544217e-06
1675 4.71784723518213e-06
1676 4.71340944992117e-06
1677 4.76803528558071e-06
1678 4.78690019846656e-06
1679 4.80258687679088e-06
1680 4.87133258239943e-06
1681 4.89441452306768e-06
1682 4.91034186325006e-06
1683 4.97759905522245e-06
1684 4.99980918434459e-06
1685 4.97581854763663e-06
1686 5.01129049768038e-06
1687 4.98750572462259e-06
1688 4.94477305818349e-06
1689 4.92944914053302e-06
1690 4.83789277814206e-06
1691 4.83184909805963e-06
1692 4.79185226374049e-06
1693 4.74123226288726e-06
1694 4.740565905129e-06
1695 4.67566620268415e-06
1696 4.6798412563425e-06
1697 4.65256515712505e-06
1698 4.61139237462559e-06
1699 4.6406283857682e-06
1700 4.6022298671666e-06
1701 4.58476384672069e-06
1702 4.62767838384776e-06
1703 4.57455116329707e-06
1704 4.58328589627771e-06
1705 4.64102261145483e-06
1706 4.57316867610722e-06
1707 4.58879605364523e-06
1708 4.7217380046999e-06
1709 4.60223754677371e-06
1710 4.63176085701775e-06
1711 4.93219922774175e-06
1712 4.7227298018409e-06
1713 4.73903983088503e-06
1714 5.33718728333815e-06
1715 5.08835311885175e-06
1716 5.22421890614841e-06
1717 6.1751407440852e-06
1718 5.32842911984261e-06
1719 5.33647795900125e-06
1720 6.58490137993878e-06
1721 5.16976086886585e-06
1722 5.27081016841902e-06
1723 4.62010747820699e-06
1724 4.7492666232787e-06
1725 4.56049638609913e-06
1726 4.6483871408728e-06
1727 4.54666204051434e-06
1728 4.60965109046251e-06
1729 4.54806051664128e-06
1730 4.56289978937074e-06
1731 4.56357889222452e-06
1732 4.54712632712225e-06
1733 4.56856233708373e-06
1734 4.54887945484109e-06
1735 4.56168192321309e-06
1736 4.56695396569272e-06
1737 4.55881247551188e-06
1738 4.57360125029438e-06
1739 4.5852368779073e-06
1740 4.5823948288577e-06
1741 4.60896047389331e-06
1742 4.61429924120534e-06
1743 4.61829807798039e-06
1744 4.66048650484151e-06
1745 4.69028401647458e-06
1746 4.67795837525285e-06
1747 4.74101785953008e-06
1748 4.83995819547543e-06
1749 4.85183686857523e-06
1750 4.88082829678853e-06
1751 5.29009609153859e-06
1752 5.57314929412771e-06
1753 5.3208400490731e-06
1754 5.47247450847521e-06
1755 5.95766321904593e-06
1756 6.53110509943669e-06
1757 5.7573345326305e-06
1758 5.99155262979445e-06
1759 5.11064569877922e-06
1760 5.6142070992413e-06
1761 4.93010136763729e-06
1762 5.43978637266411e-06
1763 4.76908726504576e-06
1764 5.23933835555577e-06
1765 4.64717682052595e-06
1766 5.06902452232698e-06
1767 4.57389555193267e-06
1768 4.94470743913719e-06
1769 4.53462993746064e-06
1770 4.85823039457248e-06
1771 4.51286545593752e-06
1772 4.81721717442648e-06
1773 4.50523172462584e-06
1774 4.79946793654322e-06
1775 4.50517009022969e-06
1776 4.77063366622588e-06
1777 4.49220358609545e-06
1778 4.72069476308162e-06
1779 4.49140046511286e-06
1780 4.68450542496252e-06
1781 4.49152693761254e-06
1782 4.63303861350103e-06
1783 4.4932333906087e-06
1784 4.5878762881927e-06
1785 4.51045442323559e-06
1786 4.53275276917466e-06
1787 4.55606159782319e-06
1788 4.5105397055243e-06
1789 4.56734368100095e-06
1790 4.53194877288184e-06
1791 4.52972246167088e-06
1792 4.57398805536682e-06
1793 4.52823889594161e-06
1794 4.54233225387828e-06
1795 4.57131470500341e-06
1796 4.53441296794935e-06
1797 4.52991745602315e-06
1798 4.60608011995241e-06
1799 4.57006676514021e-06
1800 4.5184061055503e-06
1801 4.64600950031076e-06
1802 4.7393100984904e-06
1803 4.64104885824281e-06
1804 4.80963707505971e-06
1805 5.60326871524152e-06
1806 5.24857988343535e-06
1807 5.85244314781037e-06
1808 7.61297947513937e-06
1809 5.67533114498509e-06
1810 5.31725442892465e-06
1811 4.69811153338519e-06
1812 4.74589508384481e-06
1813 4.60165159562482e-06
1814 4.6936815345768e-06
1815 4.6067372729993e-06
1816 4.68043795089268e-06
1817 4.64904502703032e-06
1818 4.68462401578681e-06
1819 4.68277512666815e-06
1820 4.68324109418939e-06
1821 4.72855205768939e-06
1822 4.7098987401747e-06
1823 4.74896361890955e-06
1824 4.74065395870585e-06
1825 4.76769003781417e-06
1826 4.79841698379673e-06
1827 4.78311215920604e-06
1828 4.86656496564579e-06
1829 4.80897617599946e-06
1830 4.86224992307513e-06
1831 4.90903744207268e-06
1832 4.8575229605936e-06
1833 4.95458600452282e-06
1834 4.88053721221308e-06
1835 4.93645584647036e-06
1836 4.96689254189755e-06
1837 4.93338575053442e-06
1838 4.99866159358052e-06
1839 4.96122329056866e-06
1840 5.00982607843788e-06
1841 5.01332064152816e-06
1842 5.0630383025868e-06
1843 5.09611141295227e-06
1844 5.18048478798639e-06
1845 5.26045017588233e-06
1846 5.37662981909002e-06
1847 5.51066625491692e-06
1848 5.66940189130082e-06
1849 5.88563267428036e-06
1850 6.05496685628031e-06
1851 6.33422605798083e-06
1852 6.40702110405115e-06
1853 6.67087912939096e-06
1854 6.67295669427352e-06
1855 6.87647092734739e-06
1856 6.96401012467091e-06
1857 7.01873177540742e-06
1858 7.05540429373828e-06
1859 6.88687761837007e-06
1860 6.66196373150441e-06
1861 6.38285191128254e-06
1862 6.05772355841751e-06
1863 5.74627087692386e-06
1864 5.46849244251998e-06
1865 5.27286175023807e-06
1866 5.12661660719658e-06
1867 5.00162044276131e-06
1868 4.91962953286201e-06
1869 4.86685377824341e-06
1870 4.80522615384858e-06
1871 4.77468634626255e-06
1872 4.74793632776279e-06
1873 4.71750983626102e-06
1874 4.70317784267247e-06
1875 4.6941959579378e-06
1876 4.68098596854226e-06
1877 4.67567866266526e-06
1878 4.66187109963873e-06
1879 4.66606994041716e-06
1880 4.65530777582731e-06
1881 4.6533120644485e-06
1882 4.65880220712893e-06
1883 4.65007034663016e-06
1884 4.65782035332038e-06
1885 4.64933291566025e-06
1886 4.65408919857096e-06
1887 4.65970816236987e-06
1888 4.64980109890004e-06
1889 4.66299128586636e-06
1890 4.66129143270849e-06
1891 4.66362668289155e-06
1892 4.66814721370359e-06
1893 4.6645875162798e-06
1894 4.68288733454039e-06
1895 4.67225765680289e-06
1896 4.68785555291382e-06
1897 4.68981847955329e-06
1898 4.68817591674152e-06
1899 4.69463995855847e-06
1900 4.69819586677126e-06
1901 4.70196634985924e-06
1902 4.71121254437926e-06
1903 4.70913834178263e-06
1904 4.71320155929116e-06
1905 4.73235268041586e-06
1906 4.71937684609181e-06
1907 4.74013562652174e-06
1908 4.72754572993199e-06
1909 4.75176924367722e-06
1910 4.75409742908575e-06
1911 4.74535413901603e-06
1912 4.77786422291836e-06
1913 4.75465268065571e-06
1914 4.78537900117656e-06
1915 4.75899034058031e-06
1916 4.79011352172104e-06
1917 4.77449491155054e-06
1918 4.79138247777622e-06
1919 4.79541096367234e-06
1920 4.78314568982998e-06
1921 4.8040368966235e-06
1922 4.78885888416434e-06
1923 4.79794679242136e-06
1924 4.80117310255962e-06
1925 4.79046775881311e-06
1926 4.80942696147541e-06
1927 4.78801276889896e-06
1928 4.81025371209292e-06
1929 4.79446324806799e-06
1930 4.79610586729613e-06
1931 4.80020659623999e-06
1932 4.79343711354862e-06
1933 4.80041620730127e-06
1934 4.79223065568972e-06
1935 4.79743092345977e-06
1936 4.80659728653633e-06
1937 4.79260745074098e-06
1938 4.78751806339679e-06
1939 4.80526383357406e-06
1940 4.7864428409406e-06
1941 4.80863096965192e-06
1942 4.77458368787118e-06
1943 4.8052290611539e-06
1944 4.78105460019927e-06
1945 4.80264569715213e-06
1946 4.7828372000624e-06
1947 4.80128370835031e-06
1948 4.77769358189284e-06
1949 4.80074300802114e-06
1950 4.78023438083677e-06
1951 4.80987713166677e-06
1952 4.7964284378022e-06
1953 4.78278054802697e-06
1954 4.81018402955774e-06
1955 4.79531568640352e-06
1956 4.80727955170745e-06
1957 4.80407806477036e-06
1958 4.79204737586486e-06
1959 4.8255856351712e-06
1960 4.80762294449773e-06
1961 4.81160850116861e-06
1962 4.82289124041169e-06
1963 4.82124285585954e-06
1964 4.80287520001436e-06
1965 4.82978276768695e-06
1966 4.83136662469911e-06
1967 4.83239682289177e-06
1968 4.83271410264504e-06
1969 4.83089127316434e-06
1970 4.83226299948355e-06
1971 4.84789446702953e-06
1972 4.83708334127798e-06
1973 4.85551550679138e-06
1974 4.86122124158338e-06
1975 4.83361470472163e-06
1976 4.90155177921112e-06
1977 4.83955802501876e-06
1978 4.89258171283775e-06
1979 4.86753679349393e-06
1980 4.89017452858541e-06
1981 4.88353801918011e-06
1982 4.88650143234476e-06
1983 4.91431393366567e-06
1984 4.87757590173903e-06
1985 4.90455837494409e-06
1986 4.91893954087104e-06
1987 4.90369463260198e-06
1988 4.92547232362829e-06
1989 4.92573482559918e-06
1990 4.93961781080821e-06
1991 4.93706807872714e-06
1992 4.92179845134301e-06
1993 4.94810838729077e-06
1994 4.93608943187543e-06
1995 4.97076745312406e-06
1996 4.95043284608223e-06
1997 4.95897797227481e-06
1998 4.95285656560657e-06
1999 4.97280127926823e-06
};
\addlegendentry{Train}
\addplot [semithick, black]
table {%
0 0.017828518524766
1 0.0170209277421236
2 0.0162515044212341
3 0.0154994819313288
4 0.0147450100630522
5 0.0139427240937948
6 0.0130075141787529
7 0.0119107682257891
8 0.0107442671433091
9 0.00962498784065247
10 0.00864505395293236
11 0.00783624965697527
12 0.00714802835136652
13 0.00652266899123788
14 0.00597374187782407
15 0.00549546489492059
16 0.00507495319470763
17 0.0047151381149888
18 0.00440172106027603
19 0.00410806247964501
20 0.00385456532239914
21 0.00364522775635123
22 0.00347544252872467
23 0.0033394331112504
24 0.00322615657933056
25 0.00312279770150781
26 0.00301905465312302
27 0.00292831915430725
28 0.00283771171234548
29 0.00276289135217667
30 0.00269222492352128
31 0.00262181623838842
32 0.00254625733941793
33 0.00247013242915273
34 0.00239320914261043
35 0.00231355056166649
36 0.00223502679727972
37 0.00215968117117882
38 0.00208797724917531
39 0.00201596855185926
40 0.0019465743098408
41 0.00188178650569171
42 0.00182107428554446
43 0.00176389038097113
44 0.00170971942134202
45 0.00165732682216913
46 0.00160703237634152
47 0.0015590024413541
48 0.00151314935646951
49 0.00146991782821715
50 0.00142850703559816
51 0.00138830696232617
52 0.00134953064844012
53 0.00131222908385098
54 0.00127566605806351
55 0.00124056974891573
56 0.00120660872198641
57 0.0011731389677152
58 0.00114109949208796
59 0.0011106088059023
60 0.00108164374250919
61 0.00105390360113233
62 0.00102754856925458
63 0.00100223091430962
64 0.0009781364351511
65 0.000954668968915939
66 0.000931537593714893
67 0.000909569906070828
68 0.000887808040715754
69 0.000866200716700405
70 0.000845424248836935
71 0.000825749360956252
72 0.000806348689366132
73 0.000787859200499952
74 0.000768729718402028
75 0.000750535458792001
76 0.000732753775082529
77 0.000715011381544173
78 0.000698274525348097
79 0.000682174286339432
80 0.000666834239382297
81 0.000652225280646235
82 0.000638197117950767
83 0.000625121407210827
84 0.000611735915299505
85 0.00059926719404757
86 0.000586948648560792
87 0.000575144018512219
88 0.000563131470698863
89 0.000551985925994813
90 0.000540476234164089
91 0.000529488665051758
92 0.00051869940944016
93 0.000507928198203444
94 0.000497457280289382
95 0.000487470417283475
96 0.000477757974294946
97 0.000468407990410924
98 0.000459243368823081
99 0.000450371648184955
100 0.000441495300037786
101 0.000432846427429467
102 0.00042450221371837
103 0.00041613433859311
104 0.000408191990572959
105 0.000400306540541351
106 0.000392735149944201
107 0.000385292776627466
108 0.000377958174794912
109 0.000370810914319009
110 0.000363672006642446
111 0.000356566946720704
112 0.000349687732523307
113 0.000342827319400385
114 0.000336388475261629
115 0.000328956870362163
116 0.000321707047987729
117 0.000315101613523439
118 0.000308587390463799
119 0.000302337197354063
120 0.000296400481602177
121 0.000290610769297928
122 0.000284840993117541
123 0.000279583648080006
124 0.000274059042567387
125 0.000268926291028038
126 0.000262297107838094
127 0.000255111051956192
128 0.000248270545853302
129 0.00024201582709793
130 0.000236022679018788
131 0.000230605932301842
132 0.000225468436838128
133 0.000220299189095385
134 0.0002156031987397
135 0.000210663507459685
136 0.000206380500458181
137 0.000201986025786027
138 0.000197642744751647
139 0.000193447995116003
140 0.00018954710685648
141 0.000185541954124346
142 0.000182048490387388
143 0.000178231537574902
144 0.00017488092998974
145 0.000171343010151759
146 0.000167864563991316
147 0.000164211916853674
148 0.000161232615937479
149 0.000157888571266085
150 0.00015474962128792
151 0.000151816188008524
152 0.000148933089803904
153 0.0001458923361497
154 0.000142757126013748
155 0.000139535884954967
156 0.000136314061819576
157 0.000132679881062359
158 0.000129215506603941
159 0.000126280923723243
160 0.000123045538202859
161 0.00011998260015389
162 0.000116745584819
163 0.000113578913442325
164 0.000110360095277429
165 0.000107670843135566
166 0.000105215047369711
167 0.000102708960184827
168 0.000100429009762593
169 9.82288474915549e-05
170 9.57258525886573e-05
171 9.33433257159777e-05
172 9.10816233954392e-05
173 8.86552661540918e-05
174 8.64464891492389e-05
175 8.42256849864498e-05
176 8.2425816799514e-05
177 8.05857489467598e-05
178 7.86945893196389e-05
179 7.70528786233626e-05
180 7.54668362787925e-05
181 7.38311355235055e-05
182 7.22671757102944e-05
183 7.08916631992906e-05
184 6.94254413247108e-05
185 6.79527365718968e-05
186 6.66059713694267e-05
187 6.5465850639157e-05
188 6.40087673673406e-05
189 6.28885609330609e-05
190 6.18238336755894e-05
191 6.04893793934025e-05
192 5.9502744989004e-05
193 5.83481705689337e-05
194 5.73073957639281e-05
195 5.60804764973e-05
196 5.52408455405384e-05
197 5.42049383511767e-05
198 5.32358499185648e-05
199 5.22250884387176e-05
200 5.13800005137455e-05
201 5.04717027070001e-05
202 4.94846754008904e-05
203 4.86643220938276e-05
204 4.78400324936956e-05
205 4.70629966002889e-05
206 4.63053584098816e-05
207 4.53866960015148e-05
208 4.4705746404361e-05
209 4.38049573858734e-05
210 4.29718129453249e-05
211 4.19376010540873e-05
212 4.10399916290771e-05
213 4.01052966481075e-05
214 3.92150541301817e-05
215 3.84969898732379e-05
216 3.76788921130355e-05
217 3.71166534023359e-05
218 3.64117877325043e-05
219 3.59316509275232e-05
220 3.52940551238135e-05
221 3.48457251675427e-05
222 3.43085375789087e-05
223 3.39022699336056e-05
224 3.33771276928019e-05
225 3.29807444359176e-05
226 3.24956126860343e-05
227 3.21389306918718e-05
228 3.16899277095217e-05
229 3.13324635499157e-05
230 3.10444156639278e-05
231 3.06607944366988e-05
232 3.02715707221068e-05
233 2.99734165309928e-05
234 2.95722711598501e-05
235 2.9264849217725e-05
236 2.90136740659364e-05
237 2.86225676973118e-05
238 2.82693436020054e-05
239 2.79557425528765e-05
240 2.7538722861209e-05
241 2.72327233687975e-05
242 2.68883577518864e-05
243 2.65507605945459e-05
244 2.6155519663007e-05
245 2.5932866265066e-05
246 2.56374823948136e-05
247 2.53918824455468e-05
248 2.50683478952851e-05
249 2.48906435444951e-05
250 2.46628860622877e-05
251 2.43888134718873e-05
252 2.42187070398359e-05
253 2.39890396187548e-05
254 2.3750701075187e-05
255 2.35937623074278e-05
256 2.34370654652594e-05
257 2.31836347666103e-05
258 2.30547848332208e-05
259 2.28314547712216e-05
260 2.27121308853384e-05
261 2.25914645852754e-05
262 2.23179686145158e-05
263 2.22775342990644e-05
264 2.20534057007171e-05
265 2.19606445170939e-05
266 2.17749529838329e-05
267 2.16859370993916e-05
268 2.14854026125977e-05
269 2.14243282243842e-05
270 2.12544473470189e-05
271 2.11388833122328e-05
272 2.09543868550099e-05
273 2.09018380701309e-05
274 2.06940858333837e-05
275 2.0645717086154e-05
276 2.04783991648583e-05
277 2.03194977075327e-05
278 2.02817682293244e-05
279 2.01196926354896e-05
280 1.99944042833522e-05
281 1.99234455067199e-05
282 1.97481931536458e-05
283 1.96450291696237e-05
284 1.95851862372365e-05
285 1.94411004486028e-05
286 1.93365776794963e-05
287 1.92494153452571e-05
288 1.91976960195461e-05
289 1.90419668797404e-05
290 1.89334386959672e-05
291 1.88616140803788e-05
292 1.87776167877018e-05
293 1.86816287168767e-05
294 1.86361976375338e-05
295 1.85230474016862e-05
296 1.84248438017676e-05
297 1.84006930794567e-05
298 1.82767562364461e-05
299 1.82336243597092e-05
300 1.81502146006096e-05
301 1.80185725184856e-05
302 1.79993439815007e-05
303 1.78844547917834e-05
304 1.78640639205696e-05
305 1.77549336513039e-05
306 1.77223828359274e-05
307 1.76377143361606e-05
308 1.76272151293233e-05
309 1.74928754859138e-05
310 1.74779106600909e-05
311 1.73964581335895e-05
312 1.7373242371832e-05
313 1.73014141182648e-05
314 1.7251764802495e-05
315 1.72184336406644e-05
316 1.71251376741566e-05
317 1.70865259860875e-05
318 1.70197290572105e-05
319 1.70126404555049e-05
320 1.69621744134929e-05
321 1.69172235473525e-05
322 1.6854715795489e-05
323 1.68310853041476e-05
324 1.678966327745e-05
325 1.67668185895309e-05
326 1.67105663422262e-05
327 1.67009111464722e-05
328 1.66561021615053e-05
329 1.6626716387691e-05
330 1.65793171618134e-05
331 1.6522777514183e-05
332 1.6553271052544e-05
333 1.65501733135898e-05
334 1.64904522534925e-05
335 1.65061264851829e-05
336 1.6435056750197e-05
337 1.64171869982965e-05
338 1.63410804816522e-05
339 1.63486420206027e-05
340 1.62990054377588e-05
341 1.62651522259694e-05
342 1.63068834808655e-05
343 1.62534597620834e-05
344 1.62194301083218e-05
345 1.61964890139643e-05
346 1.6175288692466e-05
347 1.61262069013901e-05
348 1.61576626851456e-05
349 1.61050011229236e-05
350 1.6127170965774e-05
351 1.61018415383296e-05
352 1.60480103659211e-05
353 1.61106363520958e-05
354 1.60244399012299e-05
355 1.60098570631817e-05
356 1.60258514370071e-05
357 1.59852479555411e-05
358 1.5953482943587e-05
359 1.60044164658757e-05
360 1.59958690346684e-05
361 1.5931418602122e-05
362 1.59720057126833e-05
363 1.59709852596279e-05
364 1.59966166393133e-05
365 1.5941997844493e-05
366 1.59766132128425e-05
367 1.59025112225208e-05
368 1.59132323460653e-05
369 1.59065384650603e-05
370 1.58893199113663e-05
371 1.58893744810484e-05
372 1.58953007485252e-05
373 1.58850016305223e-05
374 1.58262864715653e-05
375 1.58784750965424e-05
376 1.58329221449094e-05
377 1.58468101290055e-05
378 1.58317598106805e-05
379 1.57637623487972e-05
380 1.58019302034518e-05
381 1.57798531290609e-05
382 1.5743144103908e-05
383 1.57065678649815e-05
384 1.56689511641162e-05
385 1.57102767843753e-05
386 1.56841379066464e-05
387 1.56587793753715e-05
388 1.56132155098021e-05
389 1.56312162289396e-05
390 1.56316018546931e-05
391 1.5598403479089e-05
392 1.56162859639153e-05
393 1.56010155478725e-05
394 1.56496334966505e-05
395 1.56521582539426e-05
396 1.55757952597924e-05
397 1.56398891704157e-05
398 1.56123642227612e-05
399 1.5610708942404e-05
400 1.56350870383903e-05
401 1.56228143168846e-05
402 1.55381712829694e-05
403 1.56130372488406e-05
404 1.55380639625946e-05
405 1.55689212988364e-05
406 1.55212401296012e-05
407 1.5445904864464e-05
408 1.54589943122119e-05
409 1.54478584590834e-05
410 1.53298387886025e-05
411 1.5343832274084e-05
412 1.52539878399693e-05
413 1.53221280925209e-05
414 1.52445281855762e-05
415 1.52932352648349e-05
416 1.52357606566511e-05
417 1.52761731442297e-05
418 1.51843632920645e-05
419 1.52893280755961e-05
420 1.52031043398893e-05
421 1.52279371832265e-05
422 1.52151624206454e-05
423 1.51939657371258e-05
424 1.5175523913058e-05
425 1.52348629853805e-05
426 1.51290951180272e-05
427 1.51750436998555e-05
428 1.5108797015273e-05
429 1.50968517118599e-05
430 1.50653713717475e-05
431 1.50631976794102e-05
432 1.49979559864732e-05
433 1.49978159242892e-05
434 1.49478546518367e-05
435 1.49220941239037e-05
436 1.49222787513281e-05
437 1.4841764823359e-05
438 1.48559447552543e-05
439 1.48222889038152e-05
440 1.47857272168039e-05
441 1.47798500620411e-05
442 1.475132376072e-05
443 1.47077480505686e-05
444 1.46982101796311e-05
445 1.46510301419767e-05
446 1.46156544360565e-05
447 1.46070951814181e-05
448 1.44837640618789e-05
449 1.45409640026628e-05
450 1.44943714985857e-05
451 1.44513860504958e-05
452 1.44375408126507e-05
453 1.43832712637959e-05
454 1.44108144013444e-05
455 1.43239467433887e-05
456 1.42958251672098e-05
457 1.4354863196786e-05
458 1.42268181662075e-05
459 1.42313010655926e-05
460 1.42052022056305e-05
461 1.41718528539059e-05
462 1.41706668728148e-05
463 1.41556047310587e-05
464 1.41211712616496e-05
465 1.40736938192276e-05
466 1.41169293783605e-05
467 1.40683168865507e-05
468 1.40291103889467e-05
469 1.39949761432945e-05
470 1.39565736390068e-05
471 1.39799067255808e-05
472 1.39593194035115e-05
473 1.39060348374187e-05
474 1.39121084430371e-05
475 1.38274726850796e-05
476 1.38813938974636e-05
477 1.38118221002514e-05
478 1.38285795401316e-05
479 1.37703073050943e-05
480 1.37737688419293e-05
481 1.37285496748518e-05
482 1.37534079840407e-05
483 1.37013166749966e-05
484 1.36548251248314e-05
485 1.36851576826302e-05
486 1.36005082822521e-05
487 1.36846865643747e-05
488 1.35884292831179e-05
489 1.36228591145482e-05
490 1.35875470732572e-05
491 1.35359478008468e-05
492 1.35899417728069e-05
493 1.35104073706316e-05
494 1.34968931888579e-05
495 1.35340005726903e-05
496 1.34843940031715e-05
497 1.34926140162861e-05
498 1.34618876472814e-05
499 1.34304546008934e-05
500 1.34874426294118e-05
501 1.34512629301753e-05
502 1.34612173496862e-05
503 1.34459287437494e-05
504 1.34422625706065e-05
505 1.34426200020243e-05
506 1.33468984131468e-05
507 1.34244482978829e-05
508 1.34210522446665e-05
509 1.33963130792836e-05
510 1.34021074700286e-05
511 1.33982266561361e-05
512 1.3402862350631e-05
513 1.34251085910364e-05
514 1.33858111439622e-05
515 1.3411065992841e-05
516 1.3377290997596e-05
517 1.34235824589268e-05
518 1.33559133246308e-05
519 1.33833527797833e-05
520 1.33959229060565e-05
521 1.33901830849936e-05
522 1.33674548123963e-05
523 1.33831008497509e-05
524 1.33769790409133e-05
525 1.33832572828396e-05
526 1.3437439520203e-05
527 1.33904040922062e-05
528 1.33927769638831e-05
529 1.33709218061995e-05
530 1.33955545607023e-05
531 1.34490992422798e-05
532 1.33890089273336e-05
533 1.34110641738516e-05
534 1.3331280570128e-05
535 1.33930825541029e-05
536 1.34167403302854e-05
537 1.32872119138483e-05
538 1.33002695292816e-05
539 1.34284609885071e-05
540 1.32110635604477e-05
541 1.33949779410614e-05
542 1.33272878883872e-05
543 1.32773902805638e-05
544 1.33251451188698e-05
545 1.33807743623038e-05
546 1.32202321765362e-05
547 1.33842431750963e-05
548 1.32314034999581e-05
549 1.32485311041819e-05
550 1.33054518300924e-05
551 1.31764882098651e-05
552 1.31835431602667e-05
553 1.32719496832578e-05
554 1.31113674797234e-05
555 1.32430941448547e-05
556 1.32078048409312e-05
557 1.31588258227566e-05
558 1.31865290313726e-05
559 1.32182503875811e-05
560 1.30972202896373e-05
561 1.32085824589012e-05
562 1.32749974000035e-05
563 1.31127826534794e-05
564 1.32426775962813e-05
565 1.32393233798211e-05
566 1.3206228686613e-05
567 1.32681352624786e-05
568 1.32204377223388e-05
569 1.32071118059685e-05
570 1.32654813569388e-05
571 1.32074083012412e-05
572 1.33028415802983e-05
573 1.32049135572743e-05
574 1.32035784190521e-05
575 1.32441491587088e-05
576 1.31144834085717e-05
577 1.31305660033831e-05
578 1.30947928482783e-05
579 1.29521840790403e-05
580 1.29696600197349e-05
581 1.2844784578192e-05
582 1.28389938254259e-05
583 1.27586945382063e-05
584 1.26486684166593e-05
585 1.25862416098244e-05
586 1.25997821669444e-05
587 1.2476515621529e-05
588 1.25200986076379e-05
589 1.24557109302259e-05
590 1.25318510981742e-05
591 1.24612724903272e-05
592 1.24409616546473e-05
593 1.24104326459928e-05
594 1.23877935038763e-05
595 1.23309218906797e-05
596 1.23747331599589e-05
597 1.24195239550318e-05
598 1.23940617413609e-05
599 1.24260232041706e-05
600 1.27243156384793e-05
601 1.26135764730861e-05
602 1.27171051644837e-05
603 1.32090399347362e-05
604 1.26088552860892e-05
605 1.27834791783243e-05
606 1.30835724121425e-05
607 1.23765403259313e-05
608 1.25430433399742e-05
609 1.28693845908856e-05
610 1.21284738270333e-05
611 1.23876534416922e-05
612 1.30458147395984e-05
613 1.20547592814546e-05
614 1.21159382615588e-05
615 1.34413821797352e-05
616 1.21481398309697e-05
617 1.21315042633796e-05
618 1.3402481272351e-05
619 1.22347519209143e-05
620 1.21689417937887e-05
621 1.32563463921542e-05
622 1.22128330986015e-05
623 1.22141673273291e-05
624 1.33073026518105e-05
625 1.21387820399832e-05
626 1.227906159329e-05
627 1.36544076667633e-05
628 1.20555860121385e-05
629 1.26262739286176e-05
630 1.35473992486368e-05
631 1.20088498078985e-05
632 1.34180099848891e-05
633 1.2856675311923e-05
634 1.21406046673656e-05
635 1.36200123961316e-05
636 1.25332899187924e-05
637 1.26076620290405e-05
638 1.34945921672625e-05
639 1.2371109733067e-05
640 1.35685140776332e-05
641 1.28506480905344e-05
642 1.28723286252352e-05
643 1.36619219119893e-05
644 1.2759369383275e-05
645 1.36263288368355e-05
646 1.3043376384303e-05
647 1.30720272863982e-05
648 1.33572384584113e-05
649 1.28003403006005e-05
650 1.33299781737151e-05
651 1.27264111142722e-05
652 1.28357432913617e-05
653 1.28280707940576e-05
654 1.25059423226048e-05
655 1.27923840409494e-05
656 1.24884436445427e-05
657 1.25490932987304e-05
658 1.25580154417548e-05
659 1.23138870549155e-05
660 1.24829375636182e-05
661 1.23206973512424e-05
662 1.23037352750544e-05
663 1.23466670629568e-05
664 1.22136707432219e-05
665 1.22312185339979e-05
666 1.22916735563194e-05
667 1.21325256259297e-05
668 1.21560169645818e-05
669 1.22514957183739e-05
670 1.21247185234097e-05
671 1.21057555588777e-05
672 1.21802859212039e-05
673 1.2041055924783e-05
674 1.21119783216272e-05
675 1.21397652037558e-05
676 1.20475642688689e-05
677 1.20333452287014e-05
678 1.21236998893437e-05
679 1.20108616101788e-05
680 1.19984724733513e-05
681 1.20534059533384e-05
682 1.21082866826328e-05
683 1.19955584523268e-05
684 1.20218992378796e-05
685 1.21871662486228e-05
686 1.19854948934517e-05
687 1.19221758723143e-05
688 1.23691779663204e-05
689 1.20226341095986e-05
690 1.19385240395786e-05
691 1.22603596537374e-05
692 1.26103304864955e-05
693 1.22616911539808e-05
694 1.18797488539713e-05
695 1.4860331248201e-05
696 1.39332632898004e-05
697 1.29823292809306e-05
698 1.21145512821386e-05
699 1.20577560664969e-05
700 1.25894957818673e-05
701 1.18949428724591e-05
702 1.23835143313045e-05
703 1.21006078188657e-05
704 1.20618979053688e-05
705 1.27276061903103e-05
706 1.19484520837432e-05
707 1.29196196212433e-05
708 1.19834212455316e-05
709 1.2950717973581e-05
710 1.20679624160402e-05
711 1.28002220662893e-05
712 1.23028648886248e-05
713 1.26339373309747e-05
714 1.26422401081072e-05
715 1.25700562421116e-05
716 1.28776564451982e-05
717 1.26278237075894e-05
718 1.29646359710023e-05
719 1.25933447634452e-05
720 1.26947352327988e-05
721 1.25761134768254e-05
722 1.2355644685158e-05
723 1.26209106383612e-05
724 1.21179000416305e-05
725 1.23642703329097e-05
726 1.21639895951375e-05
727 1.20082268040278e-05
728 1.22126648420817e-05
729 1.18759462566231e-05
730 1.20359645734425e-05
731 1.19415544759249e-05
732 1.18290372483898e-05
733 1.20253280329052e-05
734 1.17773215606576e-05
735 1.17931422209949e-05
736 1.19404694487457e-05
737 1.16420114864013e-05
738 1.18172447400866e-05
739 1.1838454156532e-05
740 1.16545561468229e-05
741 1.18061261673574e-05
742 1.18024763651192e-05
743 1.16365745270741e-05
744 1.17370755106094e-05
745 1.18757889140397e-05
746 1.1767055184464e-05
747 1.166038964584e-05
748 1.23024765343871e-05
749 1.22201863632654e-05
750 1.19840997285792e-05
751 1.28028268591152e-05
752 1.32801023937645e-05
753 1.19184069262701e-05
754 1.23889667520416e-05
755 1.38901468744734e-05
756 1.26092772916309e-05
757 1.2547917322081e-05
758 1.20036111184163e-05
759 1.18514817586401e-05
760 1.18755206131027e-05
761 1.16976052595419e-05
762 1.19520063890377e-05
763 1.16887413241784e-05
764 1.20914437502506e-05
765 1.18184416351141e-05
766 1.22396631923039e-05
767 1.21238954307046e-05
768 1.23315603559604e-05
769 1.23039526442881e-05
770 1.22868923426722e-05
771 1.24479756777873e-05
772 1.22337778520887e-05
773 1.25124715850689e-05
774 1.22943774840678e-05
775 1.27458433780703e-05
776 1.23997906484874e-05
777 1.31300776047283e-05
778 1.25600518003921e-05
779 1.35322989081033e-05
780 1.25993237816147e-05
781 1.37015740619972e-05
782 1.24397010949906e-05
783 1.35945529109449e-05
784 1.20935283121071e-05
785 1.30616017486318e-05
786 1.18380958156195e-05
787 1.27432995213894e-05
788 1.18465941341128e-05
789 1.2672398952418e-05
790 1.19631304187351e-05
791 1.28720612337929e-05
792 1.21174152809544e-05
793 1.32889508677181e-05
794 1.22675410239026e-05
795 1.37992501549888e-05
796 1.22771261885646e-05
797 1.43673996717553e-05
798 1.22979427032988e-05
799 1.51086906043929e-05
800 1.22915371321142e-05
801 1.58546172315255e-05
802 1.23367162814247e-05
803 1.6454159776913e-05
804 1.26111744975788e-05
805 1.66071149578784e-05
806 1.31493043227238e-05
807 1.65030469361227e-05
808 1.39215007948224e-05
809 1.6255004084087e-05
810 1.4687204384245e-05
811 1.5978630472091e-05
812 1.53394994413247e-05
813 1.59680184879107e-05
814 1.56324131239671e-05
815 1.59016817633528e-05
816 1.55681718752021e-05
817 1.56491605594056e-05
818 1.53534565470181e-05
819 1.52185857587028e-05
820 1.46720285556512e-05
821 1.42702574521536e-05
822 1.39505082188407e-05
823 1.35860764203244e-05
824 1.32636987473234e-05
825 1.29839927467401e-05
826 1.27543844428146e-05
827 1.25379674500437e-05
828 1.24380612760433e-05
829 1.2300055459491e-05
830 1.22315095723025e-05
831 1.21559587569209e-05
832 1.21147559184465e-05
833 1.20817394417827e-05
834 1.20377417260897e-05
835 1.20341101137456e-05
836 1.20306458484265e-05
837 1.19838905447978e-05
838 1.20018703455571e-05
839 1.19836922749528e-05
840 1.19934647955233e-05
841 1.20110089483205e-05
842 1.19989545055432e-05
843 1.20004533528117e-05
844 1.20327231343254e-05
845 1.19977185022435e-05
846 1.20384338515578e-05
847 1.20781924124458e-05
848 1.20384229376214e-05
849 1.20608792713028e-05
850 1.21093053166987e-05
851 1.20685126603348e-05
852 1.21227431009174e-05
853 1.21505045171943e-05
854 1.21604070955073e-05
855 1.21427219710313e-05
856 1.22051787911914e-05
857 1.2185857485747e-05
858 1.22205701700295e-05
859 1.22410383482929e-05
860 1.22250139611424e-05
861 1.22703904708032e-05
862 1.22998171718791e-05
863 1.22956580526079e-05
864 1.22954897960881e-05
865 1.22960655062343e-05
866 1.23134177556494e-05
867 1.22884639495169e-05
868 1.23460158647504e-05
869 1.23500540212262e-05
870 1.23772615552298e-05
871 1.23426161735551e-05
872 1.2381905435177e-05
873 1.23951922432752e-05
874 1.23563550005201e-05
875 1.24097332445672e-05
876 1.24258249343256e-05
877 1.23749223348568e-05
878 1.24439729916048e-05
879 1.2450721442292e-05
880 1.24058724395582e-05
881 1.24929938465357e-05
882 1.24488851724891e-05
883 1.24566613521893e-05
884 1.24523066915572e-05
885 1.25097776617622e-05
886 1.24708794828621e-05
887 1.24765992950415e-05
888 1.25001915876055e-05
889 1.24932630569674e-05
890 1.25455871966551e-05
891 1.24509060697164e-05
892 1.24947446238366e-05
893 1.25247088362812e-05
894 1.24981561384629e-05
895 1.25340784506989e-05
896 1.25150108942762e-05
897 1.26068853205652e-05
898 1.25723390738131e-05
899 1.25428432511399e-05
900 1.26001505122986e-05
901 1.25954220493441e-05
902 1.25561800814467e-05
903 1.26202339743031e-05
904 1.26539580378449e-05
905 1.25624146676273e-05
906 1.26998111227294e-05
907 1.26072154671419e-05
908 1.26406521303579e-05
909 1.26541890494991e-05
910 1.269201766263e-05
911 1.26633012769162e-05
912 1.2710654118564e-05
913 1.26305803860305e-05
914 1.26918521345942e-05
915 1.26638115034439e-05
916 1.26885661302367e-05
917 1.26877621369204e-05
918 1.27494477055734e-05
919 1.26738004837534e-05
920 1.27574412545073e-05
921 1.27101920952555e-05
922 1.27845996757969e-05
923 1.2687967682723e-05
924 1.28412657431909e-05
925 1.26531303976662e-05
926 1.27948806039058e-05
927 1.27201556097134e-05
928 1.27883668028517e-05
929 1.2685274668911e-05
930 1.28304372992716e-05
931 1.27163675642805e-05
932 1.27343992062379e-05
933 1.26609884318896e-05
934 1.277897899854e-05
935 1.27456960399286e-05
936 1.26981922221603e-05
937 1.27144094221876e-05
938 1.27435714603052e-05
939 1.2774084098055e-05
940 1.26506402011728e-05
941 1.27403973237961e-05
942 1.26623117466806e-05
943 1.25939295685384e-05
944 1.26883551274659e-05
945 1.26285249280045e-05
946 1.266557683266e-05
947 1.26093127619242e-05
948 1.2613683793461e-05
949 1.25736796690035e-05
950 1.2580100701598e-05
951 1.25910446513444e-05
952 1.25913556985324e-05
953 1.25169099192135e-05
954 1.25502347145812e-05
955 1.25644519357593e-05
956 1.25541746456292e-05
957 1.24919661175227e-05
958 1.24923471958027e-05
959 1.24643102026312e-05
960 1.24355838124757e-05
961 1.23761192298844e-05
962 1.24159851111472e-05
963 1.23560857900884e-05
964 1.23316503959359e-05
965 1.22793489936157e-05
966 1.22519313663361e-05
967 1.2234252608323e-05
968 1.21816137834685e-05
969 1.21634657261893e-05
970 1.21606171887834e-05
971 1.20806216727942e-05
972 1.20750228234101e-05
973 1.2043068636558e-05
974 1.20412005344406e-05
975 1.19638025353197e-05
976 1.19537462524022e-05
977 1.19449005069328e-05
978 1.18967755042831e-05
979 1.19106925922097e-05
980 1.18149782792898e-05
981 1.18061634566402e-05
982 1.1843031643366e-05
983 1.17719591798959e-05
984 1.17962626973167e-05
985 1.17302388389362e-05
986 1.17400159069803e-05
987 1.17658682938782e-05
988 1.16924684334663e-05
989 1.17066829261603e-05
990 1.17214322017389e-05
991 1.16918708954472e-05
992 1.16839828478987e-05
993 1.16921646622359e-05
994 1.16671617433894e-05
995 1.16314877232071e-05
996 1.16480114229489e-05
997 1.16469445856637e-05
998 1.16235296445666e-05
999 1.16140381578589e-05
1000 1.1642210665741e-05
1001 1.16123319457984e-05
1002 1.16583269118564e-05
1003 1.16134151539882e-05
1004 1.16161791083869e-05
1005 1.16354076453717e-05
1006 1.16323453767109e-05
1007 1.16294586405274e-05
1008 1.15678649308393e-05
1009 1.16171913759899e-05
1010 1.15961011033505e-05
1011 1.16185174192651e-05
1012 1.16195624286775e-05
1013 1.16294513645698e-05
1014 1.15665279736277e-05
1015 1.16312512545846e-05
1016 1.16429746412905e-05
1017 1.16127830551704e-05
1018 1.1646044185909e-05
1019 1.16469218482962e-05
1020 1.1628755601123e-05
1021 1.15987804747419e-05
1022 1.16196260933066e-05
1023 1.16090186565998e-05
1024 1.16440633064485e-05
1025 1.16398268801277e-05
1026 1.16310111479834e-05
1027 1.16241544674267e-05
1028 1.16312585305423e-05
1029 1.16262363007991e-05
1030 1.1663308214338e-05
1031 1.16455894385581e-05
1032 1.16589681056212e-05
1033 1.1652969988063e-05
1034 1.16729552246397e-05
1035 1.16479905045708e-05
1036 1.16120272650733e-05
1037 1.16153178169043e-05
1038 1.16371911644819e-05
1039 1.16533037726185e-05
1040 1.16307437565411e-05
1041 1.16655637611984e-05
1042 1.16362762128119e-05
1043 1.16572537081083e-05
1044 1.16482315206667e-05
1045 1.16530382001656e-05
1046 1.16354076453717e-05
1047 1.16249793791212e-05
1048 1.16299534056452e-05
1049 1.16219225674286e-05
1050 1.15311277113506e-05
1051 1.16270030048327e-05
1052 1.16053024612484e-05
1053 1.162563876278e-05
1054 1.15706661745207e-05
1055 1.16018945846008e-05
1056 1.16949349830975e-05
1057 1.15885213745059e-05
1058 1.15949615064892e-05
1059 1.20092163342633e-05
1060 1.15777475002687e-05
1061 1.15965985969524e-05
1062 1.24642238006345e-05
1063 1.15560833364725e-05
1064 1.15618977361009e-05
1065 1.3348422726267e-05
1066 1.14566310003283e-05
1067 1.16151313704904e-05
1068 1.31598590087378e-05
1069 1.13183559733443e-05
1070 1.26094946608646e-05
1071 1.18817324619158e-05
1072 1.15936400106875e-05
1073 1.25692722576787e-05
1074 1.15035099952365e-05
1075 1.25967071653577e-05
1076 1.18444377221749e-05
1077 1.21402536024107e-05
1078 1.22654591905302e-05
1079 1.18060388558661e-05
1080 1.24794305520481e-05
1081 1.1926055776712e-05
1082 1.23109475680394e-05
1083 1.217933731823e-05
1084 1.20823651741375e-05
1085 1.2416162462614e-05
1086 1.2016505934298e-05
1087 1.24100952234585e-05
1088 1.20839777082438e-05
1089 1.22276514957775e-05
1090 1.24357457025326e-05
1091 1.20150698421639e-05
1092 1.24874140965403e-05
1093 1.21906459753518e-05
1094 1.23446916404646e-05
1095 1.24172775031184e-05
1096 1.22075598483207e-05
1097 1.24740317914984e-05
1098 1.23422742035473e-05
1099 1.23364552564453e-05
1100 1.25450460473076e-05
1101 1.22327410281287e-05
1102 1.26688146337983e-05
1103 1.22321816888871e-05
1104 1.25123215184431e-05
1105 1.25088927234174e-05
1106 1.23550453281496e-05
1107 1.26373615785269e-05
1108 1.23120144053246e-05
1109 1.27053135656752e-05
1110 1.23831605378655e-05
1111 1.25304068205878e-05
1112 1.24665439216187e-05
1113 1.25393999041989e-05
1114 1.2474764844228e-05
1115 1.24798743854626e-05
1116 1.25362093967851e-05
1117 1.243396945938e-05
1118 1.25717833725503e-05
1119 1.23851414173259e-05
1120 1.24517991935136e-05
1121 1.24584876175504e-05
1122 1.24714924822911e-05
1123 1.24285188576323e-05
1124 1.24678135762224e-05
1125 1.24883736134507e-05
1126 1.24244934340823e-05
1127 1.2583062016347e-05
1128 1.24087418953422e-05
1129 1.25892756841495e-05
1130 1.24271282402333e-05
1131 1.26055056171026e-05
1132 1.2518579751486e-05
1133 1.26256682051462e-05
1134 1.25353126350092e-05
1135 1.26266122606467e-05
1136 1.25912301882636e-05
1137 1.27169605548261e-05
1138 1.258604152099e-05
1139 1.27059929582174e-05
1140 1.26267505038413e-05
1141 1.28000565382536e-05
1142 1.2675793186645e-05
1143 1.27821249407134e-05
1144 1.27003249872359e-05
1145 1.29184873003396e-05
1146 1.26317718240898e-05
1147 1.28256551761297e-05
1148 1.26126478789956e-05
1149 1.288585463044e-05
1150 1.26214981719386e-05
1151 1.28305418911623e-05
1152 1.26076129163266e-05
1153 1.28133178805001e-05
1154 1.26628719954169e-05
1155 1.27733110275585e-05
1156 1.26443446788471e-05
1157 1.2710007467831e-05
1158 1.26739614643157e-05
1159 1.26419017760782e-05
1160 1.26062705021468e-05
1161 1.25852866403875e-05
1162 1.23979416457587e-05
1163 1.24317375593819e-05
1164 1.22453475341899e-05
1165 1.22510218716343e-05
1166 1.2050933946739e-05
1167 1.19790875032777e-05
1168 1.18840753202676e-05
1169 1.17620447781519e-05
1170 1.16410910777631e-05
1171 1.15516204459709e-05
1172 1.14793638203992e-05
1173 1.14676649900503e-05
1174 1.1429990081524e-05
1175 1.14327649498591e-05
1176 1.13682153823902e-05
1177 1.14304648377583e-05
1178 1.14817539724754e-05
1179 1.14988006316707e-05
1180 1.1490818906168e-05
1181 1.15525499495561e-05
1182 1.15910052045365e-05
1183 1.1611079571594e-05
1184 1.163983506558e-05
1185 1.16795608846587e-05
1186 1.17144600153551e-05
1187 1.17125227916404e-05
1188 1.17526114991051e-05
1189 1.16992068797117e-05
1190 1.17932513603591e-05
1191 1.17279714686447e-05
1192 1.18108318929444e-05
1193 1.17876852527843e-05
1194 1.18296002256102e-05
1195 1.1786563845817e-05
1196 1.18322177513619e-05
1197 1.18212292363751e-05
1198 1.1809162060672e-05
1199 1.18037514766911e-05
1200 1.17594672701671e-05
1201 1.17728286568308e-05
1202 1.17777071864111e-05
1203 1.1788100891863e-05
1204 1.17534364107996e-05
1205 1.17539693746949e-05
1206 1.17600129669881e-05
1207 1.16805858851876e-05
1208 1.17141144073685e-05
1209 1.16635319500347e-05
1210 1.16481451186701e-05
1211 1.16495248221327e-05
1212 1.16313140097191e-05
1213 1.16376586447586e-05
1214 1.16104283733875e-05
1215 1.16184992293711e-05
1216 1.1556563549675e-05
1217 1.15793891382054e-05
1218 1.15220500447322e-05
1219 1.15586381070898e-05
1220 1.15240827653906e-05
1221 1.14558097266126e-05
1222 1.14941067295149e-05
1223 1.14697440949385e-05
1224 1.14590893645072e-05
1225 1.14283384391456e-05
1226 1.14279146146146e-05
1227 1.14072172436863e-05
1228 1.13923733806587e-05
1229 1.14362728709239e-05
1230 1.13717187559814e-05
1231 1.13528012661845e-05
1232 1.1337054274918e-05
1233 1.13397754830658e-05
1234 1.13244559543091e-05
1235 1.13185178634012e-05
1236 1.1369731510058e-05
1237 1.1303401151963e-05
1238 1.1267000445514e-05
1239 1.13699998109951e-05
1240 1.12257212094846e-05
1241 1.11672379716765e-05
1242 1.14858248707606e-05
1243 1.113651705964e-05
1244 1.12769221232156e-05
1245 1.18922780529829e-05
1246 1.12640827865107e-05
1247 1.12384059320902e-05
1248 1.3752415725321e-05
1249 1.24253720059642e-05
1250 1.16684022941627e-05
1251 1.17370200314326e-05
1252 1.15867751446785e-05
1253 1.23572226584656e-05
1254 1.11476219899487e-05
1255 1.16101919047651e-05
1256 1.09996090031927e-05
1257 1.13811238406925e-05
1258 1.10910714283818e-05
1259 1.12780480776564e-05
1260 1.1124046977784e-05
1261 1.1211084711249e-05
1262 1.12111565613304e-05
1263 1.11771360025159e-05
1264 1.12377383629791e-05
1265 1.11084427771857e-05
1266 1.11803292384138e-05
1267 1.12683646875666e-05
1268 1.11511417344445e-05
1269 1.12641837404226e-05
1270 1.12023635665537e-05
1271 1.1115462257294e-05
1272 1.12515162982163e-05
1273 1.13259411591571e-05
1274 1.10565697468701e-05
1275 1.14097638288513e-05
1276 1.16824166980223e-05
1277 1.09479879029095e-05
1278 1.13618416435202e-05
1279 1.42497165143141e-05
1280 1.27917301142588e-05
1281 1.23286836242187e-05
1282 1.14312088044244e-05
1283 1.11299605123349e-05
1284 1.30684711621143e-05
1285 1.1006384738721e-05
1286 1.2486542800616e-05
1287 1.08708127299906e-05
1288 1.21775847219396e-05
1289 1.09010679807398e-05
1290 1.21426937766955e-05
1291 1.08991971501382e-05
1292 1.20779686767492e-05
1293 1.09796101241955e-05
1294 1.19524056572118e-05
1295 1.10410064735333e-05
1296 1.17115732791717e-05
1297 1.12264351628255e-05
1298 1.14511840365594e-05
1299 1.14329950520187e-05
1300 1.12867728603305e-05
1301 1.15585589810507e-05
1302 1.13696132757468e-05
1303 1.14256181404926e-05
1304 1.1665034435282e-05
1305 1.12591515062377e-05
1306 1.16375440484262e-05
1307 1.16988321678946e-05
1308 1.111399524234e-05
1309 1.21783314170898e-05
1310 1.16016644824413e-05
1311 1.08843441921636e-05
1312 1.3576367564383e-05
1313 1.13360210889368e-05
1314 1.10147766463342e-05
1315 1.52071261254605e-05
1316 1.1089677173004e-05
1317 1.38353079819353e-05
1318 1.12845273179119e-05
1319 1.22876235764124e-05
1320 1.19594033094472e-05
1321 1.18576290333294e-05
1322 1.24078533190186e-05
1323 1.18871103040874e-05
1324 1.24163570944802e-05
1325 1.21004904940492e-05
1326 1.24214784591459e-05
1327 1.23410072774277e-05
1328 1.24650041470886e-05
1329 1.25373835544451e-05
1330 1.2508329746197e-05
1331 1.28034171211766e-05
1332 1.26163595268736e-05
1333 1.30171056298423e-05
1334 1.29605587062542e-05
1335 1.30680764414137e-05
1336 1.33942157845013e-05
1337 1.30498528960743e-05
1338 1.38911900648964e-05
1339 1.31433098431444e-05
1340 1.40633401315426e-05
1341 1.32982577270013e-05
1342 1.39550729727489e-05
1343 1.34384536067955e-05
1344 1.38423829412204e-05
1345 1.33126941364026e-05
1346 1.37222350531374e-05
1347 1.33119101519696e-05
1348 1.35198615680565e-05
1349 1.32798368213116e-05
1350 1.33725434352527e-05
1351 1.31941287691006e-05
1352 1.32604354803334e-05
1353 1.30791931951535e-05
1354 1.30566158986767e-05
1355 1.29708942040452e-05
1356 1.28580895761843e-05
1357 1.28324372781208e-05
1358 1.2745231288136e-05
1359 1.26333097796305e-05
1360 1.25752567328163e-05
1361 1.24640755529981e-05
1362 1.23873896882287e-05
1363 1.22704086606973e-05
1364 1.21360308185103e-05
1365 1.20390241136192e-05
1366 1.19682126751286e-05
1367 1.18604311865056e-05
1368 1.17627096187789e-05
1369 1.17144891191856e-05
1370 1.16068613351672e-05
1371 1.15443290269468e-05
1372 1.14889417091035e-05
1373 1.14274571387796e-05
1374 1.13625537778717e-05
1375 1.12837124106591e-05
1376 1.12443831312703e-05
1377 1.12158968477161e-05
1378 1.11545414256398e-05
1379 1.11105382529786e-05
1380 1.10975015559234e-05
1381 1.10326536741923e-05
1382 1.10526698335889e-05
1383 1.09852398963994e-05
1384 1.09954262370593e-05
1385 1.0942051630991e-05
1386 1.09382162918337e-05
1387 1.09062366391299e-05
1388 1.09252468973864e-05
1389 1.08616040961351e-05
1390 1.09117863758001e-05
1391 1.08695776361856e-05
1392 1.08887234091526e-05
1393 1.08767280835309e-05
1394 1.08620806713589e-05
1395 1.08620661194436e-05
1396 1.08922495201114e-05
1397 1.08589201772702e-05
1398 1.08926797111053e-05
1399 1.08634540083585e-05
1400 1.09036245703464e-05
1401 1.08752474261564e-05
1402 1.08825697680004e-05
1403 1.08643534986186e-05
1404 1.09165812318679e-05
1405 1.0841055882338e-05
1406 1.09214106487343e-05
1407 1.08849317257409e-05
1408 1.09123056972749e-05
1409 1.08851090772077e-05
1410 1.09432921817643e-05
1411 1.09469974631793e-05
1412 1.0959477549477e-05
1413 1.09100092231529e-05
1414 1.10019427665975e-05
1415 1.0945348549285e-05
1416 1.10270229924936e-05
1417 1.09613938548137e-05
1418 1.10385217340081e-05
1419 1.09897391666891e-05
1420 1.11088202174869e-05
1421 1.10179107650765e-05
1422 1.10746914288029e-05
1423 1.10656264951103e-05
1424 1.11495537566952e-05
1425 1.10798364403308e-05
1426 1.11840781755745e-05
1427 1.11468989416608e-05
1428 1.11947738332674e-05
1429 1.12485795398243e-05
1430 1.1263969099673e-05
1431 1.1290568181721e-05
1432 1.13355072244303e-05
1433 1.13604246507748e-05
1434 1.14260383270448e-05
1435 1.14347340058885e-05
1436 1.14768827188527e-05
1437 1.14524300443009e-05
1438 1.15111433842685e-05
1439 1.14637959995889e-05
1440 1.15219663712196e-05
1441 1.1446180906205e-05
1442 1.14245622171438e-05
1443 1.14363519969629e-05
1444 1.13784544737427e-05
1445 1.1396577974665e-05
1446 1.13632195279934e-05
1447 1.12618909042794e-05
1448 1.1295765034447e-05
1449 1.12423231257708e-05
1450 1.12870020529954e-05
1451 1.11730250864639e-05
1452 1.11975859908853e-05
1453 1.11796498458716e-05
1454 1.10706951090833e-05
1455 1.10914761535241e-05
1456 1.10475230030715e-05
1457 1.10368146124529e-05
1458 1.09474431155832e-05
1459 1.09621296360274e-05
1460 1.09555348899448e-05
1461 1.09009051811881e-05
1462 1.09229713416426e-05
1463 1.09028815131751e-05
1464 1.0855666914722e-05
1465 1.08881240521441e-05
1466 1.08196181827225e-05
1467 1.08495250970009e-05
1468 1.0851747902052e-05
1469 1.08278909465298e-05
1470 1.08683361759176e-05
1471 1.08008607639931e-05
1472 1.0867179298657e-05
1473 1.08253088910715e-05
1474 1.08512485894607e-05
1475 1.08618824015139e-05
1476 1.08348258436308e-05
1477 1.07793930510525e-05
1478 1.09355642052833e-05
1479 1.07671057776315e-05
1480 1.09149623312987e-05
1481 1.09308602986857e-05
1482 1.07678306449088e-05
1483 1.09990314740571e-05
1484 1.08801887108712e-05
1485 1.08543581518461e-05
1486 1.10563287307741e-05
1487 1.08709518826799e-05
1488 1.07022906377097e-05
1489 1.1403357348172e-05
1490 1.0737946467998e-05
1491 1.07190217022435e-05
1492 1.2138819329266e-05
1493 1.05428871393087e-05
1494 1.06671486719279e-05
1495 1.4192321941664e-05
1496 1.06835486803902e-05
1497 1.13990836325684e-05
1498 1.30533662741072e-05
1499 1.05885073935497e-05
1500 1.35631926241331e-05
1501 1.09915072243894e-05
1502 1.32410614241962e-05
1503 1.11248227767646e-05
1504 1.33383773572859e-05
1505 1.13614933070494e-05
1506 1.33164849103196e-05
1507 1.15273714982322e-05
1508 1.30405942400103e-05
1509 1.15887760330224e-05
1510 1.25879432744114e-05
1511 1.15888187792734e-05
1512 1.21960938486154e-05
1513 1.15218099381309e-05
1514 1.19024662126321e-05
1515 1.1478324267955e-05
1516 1.16524115583161e-05
1517 1.13762644105009e-05
1518 1.14914009827771e-05
1519 1.13034056994366e-05
1520 1.1366422768333e-05
1521 1.12179004645441e-05
1522 1.12725074359332e-05
1523 1.11731851575314e-05
1524 1.12048874143511e-05
1525 1.11121162262862e-05
1526 1.11593362817075e-05
1527 1.10998535092222e-05
1528 1.11096169348457e-05
1529 1.10232622319018e-05
1530 1.10683777165832e-05
1531 1.09857355710119e-05
1532 1.10137643787311e-05
1533 1.09619286376983e-05
1534 1.09479697130155e-05
1535 1.08917502075201e-05
1536 1.09144712041598e-05
1537 1.08668027678505e-05
1538 1.08704771264456e-05
1539 1.0832370207936e-05
1540 1.08354070107453e-05
1541 1.08046542663942e-05
1542 1.08015374280512e-05
1543 1.07731129901367e-05
1544 1.08174608612899e-05
1545 1.0731063412095e-05
1546 1.07621499410016e-05
1547 1.0704433407227e-05
1548 1.07449277493288e-05
1549 1.06584038803703e-05
1550 1.07341247712611e-05
1551 1.06276684164186e-05
1552 1.06816178231384e-05
1553 1.0648499483068e-05
1554 1.06755487649934e-05
1555 1.06080115074292e-05
1556 1.06553616205929e-05
1557 1.06073348433711e-05
1558 1.06760071503231e-05
1559 1.06029929156648e-05
1560 1.0656123777153e-05
1561 1.05877370515373e-05
1562 1.06472934930935e-05
1563 1.06214565676055e-05
1564 1.06335028249305e-05
1565 1.05717372207437e-05
1566 1.0626901712385e-05
1567 1.05843737401301e-05
1568 1.06731959022e-05
1569 1.05992285170942e-05
1570 1.06511006379151e-05
1571 1.05665167211555e-05
1572 1.06541765489965e-05
1573 1.06348379631527e-05
1574 1.06820352812065e-05
1575 1.06065281215706e-05
1576 1.0678447324608e-05
1577 1.06423703982728e-05
1578 1.07165333247394e-05
1579 1.06562865767046e-05
1580 1.06722200143849e-05
1581 1.06761790448218e-05
1582 1.06765955933952e-05
1583 1.06760498965741e-05
1584 1.05739854916465e-05
1585 1.0684018889151e-05
1586 1.06208608485758e-05
1587 1.06489032987156e-05
1588 1.06286352092866e-05
1589 1.06301167761558e-05
1590 1.06660836536321e-05
1591 1.06277529994259e-05
1592 1.05648050521268e-05
1593 1.0695848686737e-05
1594 1.06238439911976e-05
1595 1.06575034806156e-05
1596 1.06533252619556e-05
1597 1.05816770883393e-05
1598 1.07025944089401e-05
1599 1.0645132533682e-05
1600 1.06155303001287e-05
1601 1.06915094875148e-05
1602 1.06282395790913e-05
1603 1.06803590824711e-05
1604 1.07950045276084e-05
1605 1.07152091004536e-05
1606 1.07051910163136e-05
1607 1.08842041299795e-05
1608 1.07758723970619e-05
1609 1.07321475297795e-05
1610 1.0992051102221e-05
1611 1.09288321255008e-05
1612 1.07546775325318e-05
1613 1.10485952973249e-05
1614 1.10219098132802e-05
1615 1.08575877675321e-05
1616 1.11424496935797e-05
1617 1.10627252070117e-05
1618 1.08610147435684e-05
1619 1.1073751920776e-05
1620 1.09574511952815e-05
1621 1.07574660432874e-05
1622 1.09001257442287e-05
1623 1.07693940663012e-05
1624 1.06777724795393e-05
1625 1.06589359347709e-05
1626 1.06269517345936e-05
1627 1.0538507012825e-05
1628 1.04918526631081e-05
1629 1.04105765785789e-05
1630 1.04261398519157e-05
1631 1.03413740362157e-05
1632 1.03320726339007e-05
1633 1.03207648862735e-05
1634 1.03395968835684e-05
1635 1.03200709418161e-05
1636 1.03087795650936e-05
1637 1.02667327155359e-05
1638 1.02538788269158e-05
1639 1.02079802672961e-05
1640 1.0254549124511e-05
1641 1.02833746495889e-05
1642 1.02550529845757e-05
1643 1.05291692307219e-05
1644 1.071003589459e-05
1645 1.07228052002029e-05
1646 1.10235132524394e-05
1647 1.11151011878974e-05
1648 1.06898705780623e-05
1649 1.104630791815e-05
1650 1.09760821942473e-05
1651 1.03795200629975e-05
1652 1.12310499389423e-05
1653 1.07363257484394e-05
1654 1.06614961623563e-05
1655 1.16583814815385e-05
1656 1.04640275822021e-05
1657 1.03845195553731e-05
1658 1.20473214337835e-05
1659 1.08066242319182e-05
1660 1.14636968646664e-05
1661 1.04188175100717e-05
1662 1.05839644675143e-05
1663 1.04188466139021e-05
1664 1.04702039607218e-05
1665 1.05404733403702e-05
1666 1.04724313132465e-05
1667 1.05583721961011e-05
1668 1.05277713373653e-05
1669 1.0579915397102e-05
1670 1.06479847090668e-05
1671 1.05926001197076e-05
1672 1.0746503903647e-05
1673 1.07808427856071e-05
1674 1.07519554148894e-05
1675 1.09413631435018e-05
1676 1.0881125490414e-05
1677 1.09991260615061e-05
1678 1.10890759970061e-05
1679 1.09614138636971e-05
1680 1.11772351374384e-05
1681 1.12245843411074e-05
1682 1.10042919914122e-05
1683 1.12977704702644e-05
1684 1.10589680843987e-05
1685 1.09567536128452e-05
1686 1.12245279524359e-05
1687 1.08149924926693e-05
1688 1.08931444628979e-05
1689 1.09005350168445e-05
1690 1.05731442090473e-05
1691 1.08730419015046e-05
1692 1.05487015389372e-05
1693 1.05701219581533e-05
1694 1.06824809336103e-05
1695 1.04043556348188e-05
1696 1.05545805126894e-05
1697 1.04710752566461e-05
1698 1.03267639133264e-05
1699 1.04768605524441e-05
1700 1.03980764833977e-05
1701 1.03160055004992e-05
1702 1.04885721157189e-05
1703 1.03143793239724e-05
1704 1.01789664768148e-05
1705 1.04914197436301e-05
1706 1.02841140687815e-05
1707 1.02093699752004e-05
1708 1.05389626696706e-05
1709 1.0352339813835e-05
1710 1.02250942290993e-05
1711 1.06176548797521e-05
1712 1.06351326394361e-05
1713 1.06005445559276e-05
1714 1.05380113382125e-05
1715 1.14452914203866e-05
1716 1.12834513856797e-05
1717 1.05638682725839e-05
1718 1.130914370151e-05
1719 1.13042688099085e-05
1720 1.1353745321685e-05
1721 1.04937262221938e-05
1722 1.05028984762612e-05
1723 1.01671857919428e-05
1724 1.03754937299527e-05
1725 1.02267031252268e-05
1726 1.03654738268233e-05
1727 1.01938749139663e-05
1728 1.03310721897287e-05
1729 1.02656231319997e-05
1730 1.02618487289874e-05
1731 1.02820395113667e-05
1732 1.02850772236707e-05
1733 1.03547708931728e-05
1734 1.03171969385585e-05
1735 1.03332295111613e-05
1736 1.03694492281647e-05
1737 1.03735819720896e-05
1738 1.03781039797468e-05
1739 1.04632345028222e-05
1740 1.04422670119675e-05
1741 1.0500785720069e-05
1742 1.05399967651465e-05
1743 1.05087710835505e-05
1744 1.05423632703605e-05
1745 1.07555770227918e-05
1746 1.06084189610556e-05
1747 1.04376358649461e-05
1748 1.1164729585289e-05
1749 1.07460700746742e-05
1750 1.03581587609369e-05
1751 1.23531499411911e-05
1752 1.10978971861186e-05
1753 1.02522144516115e-05
1754 1.28759993458516e-05
1755 1.13444875751156e-05
1756 1.26289123727474e-05
1757 1.09291331682471e-05
1758 1.19055912364274e-05
1759 1.05273493318236e-05
1760 1.1590326721489e-05
1761 1.04249684227398e-05
1762 1.11920217022998e-05
1763 1.03537513496121e-05
1764 1.08754511529696e-05
1765 1.02750109363114e-05
1766 1.06287761809654e-05
1767 1.02330313893617e-05
1768 1.04715463749017e-05
1769 1.02514759419137e-05
1770 1.03757565739215e-05
1771 1.02848925962462e-05
1772 1.02851772680879e-05
1773 1.03151587609318e-05
1774 1.02610301837558e-05
1775 1.03610509540886e-05
1776 1.02246121969074e-05
1777 1.03327474789694e-05
1778 1.01802424978814e-05
1779 1.03627289718133e-05
1780 1.02023095678305e-05
1781 1.03177080745809e-05
1782 1.01898549473844e-05
1783 1.03283746284433e-05
1784 1.02484646049561e-05
1785 1.025006895361e-05
1786 1.03074626167654e-05
1787 1.02520652944804e-05
1788 1.03865768323885e-05
1789 1.03311185739585e-05
1790 1.0366096830694e-05
1791 1.04380796983605e-05
1792 1.03922957350733e-05
1793 1.04300061138929e-05
1794 1.04172686405946e-05
1795 1.04017499324982e-05
1796 1.04649325294304e-05
1797 1.06153602246195e-05
1798 1.05062135844491e-05
1799 1.04277969512623e-05
1800 1.0794139598147e-05
1801 1.08241529233055e-05
1802 1.05317176348763e-05
1803 1.06536117527867e-05
1804 1.25917877085158e-05
1805 1.06809047792922e-05
1806 1.08237982203718e-05
1807 1.54880035552196e-05
1808 1.09029278974049e-05
1809 1.19041842481238e-05
1810 1.04192704384332e-05
1811 1.09289767351584e-05
1812 1.04752025436028e-05
1813 1.10222727016662e-05
1814 1.07240966826794e-05
1815 1.11078516056295e-05
1816 1.09721013359376e-05
1817 1.12020416054293e-05
1818 1.12713969429024e-05
1819 1.12362977233715e-05
1820 1.1489108146634e-05
1821 1.14396134449635e-05
1822 1.17123581730993e-05
1823 1.16708097266383e-05
1824 1.17464214781648e-05
1825 1.20337635962642e-05
1826 1.18232965178322e-05
1827 1.23375111797941e-05
1828 1.20088197945734e-05
1829 1.2192170288472e-05
1830 1.2640575732803e-05
1831 1.20787872219807e-05
1832 1.29188147184323e-05
1833 1.24057469292893e-05
1834 1.2705869266938e-05
1835 1.32674240376218e-05
1836 1.26135091704782e-05
1837 1.36678809212754e-05
1838 1.30262906168355e-05
1839 1.37942070068675e-05
1840 1.35801865326357e-05
1841 1.42570015668753e-05
1842 1.4123194887361e-05
1843 1.48282879308681e-05
1844 1.47911578096682e-05
1845 1.55504785652738e-05
1846 1.54027529788436e-05
1847 1.61936332006007e-05
1848 1.57714002853027e-05
1849 1.64230132213561e-05
1850 1.56685146066593e-05
1851 1.61416210175958e-05
1852 1.54502358782338e-05
1853 1.54642075358424e-05
1854 1.49203269756981e-05
1855 1.45415478982613e-05
1856 1.38797995532514e-05
1857 1.31835104184574e-05
1858 1.24989492178429e-05
1859 1.18603302325937e-05
1860 1.14779495561379e-05
1861 1.11742347144173e-05
1862 1.09787742985645e-05
1863 1.08019858089392e-05
1864 1.07343712443253e-05
1865 1.06850202428177e-05
1866 1.0633507372404e-05
1867 1.05726667243289e-05
1868 1.05564076875453e-05
1869 1.05651452031452e-05
1870 1.05373592305114e-05
1871 1.0575856322248e-05
1872 1.05429853647365e-05
1873 1.05603412521305e-05
1874 1.05463795989635e-05
1875 1.05404824353172e-05
1876 1.05743238236755e-05
1877 1.05696008176892e-05
1878 1.05337085187784e-05
1879 1.05613016785355e-05
1880 1.05749923022813e-05
1881 1.05689414340304e-05
1882 1.05648905446287e-05
1883 1.06225752460887e-05
1884 1.06184779724572e-05
1885 1.06301022242405e-05
1886 1.06481002148939e-05
1887 1.06695779322763e-05
1888 1.06370443972992e-05
1889 1.06621182567324e-05
1890 1.07207897599437e-05
1891 1.06850084193866e-05
1892 1.07244131868356e-05
1893 1.07829673652304e-05
1894 1.07572786873789e-05
1895 1.07846854007221e-05
1896 1.07933292383677e-05
1897 1.08525464384002e-05
1898 1.08617559817503e-05
1899 1.08441972770379e-05
1900 1.08214217107161e-05
1901 1.09220418380573e-05
1902 1.08841222754563e-05
1903 1.09377378976205e-05
1904 1.1001480743289e-05
1905 1.09845095721539e-05
1906 1.10041819425533e-05
1907 1.10340215542237e-05
1908 1.10290120574064e-05
1909 1.10900100480649e-05
1910 1.10407227111864e-05
1911 1.10788505480741e-05
1912 1.11719500637264e-05
1913 1.11077170004137e-05
1914 1.12051284304471e-05
1915 1.114029873861e-05
1916 1.12234056359739e-05
1917 1.11798644866212e-05
1918 1.12453617475694e-05
1919 1.11871258923202e-05
1920 1.12368506961502e-05
1921 1.12295310827903e-05
1922 1.12264760900871e-05
1923 1.11760236904956e-05
1924 1.12727830128279e-05
1925 1.12576626634109e-05
1926 1.12524357973598e-05
1927 1.12275147330365e-05
1928 1.12936686491594e-05
1929 1.12859042928903e-05
1930 1.12331754280603e-05
1931 1.12569705379428e-05
1932 1.1282882951491e-05
1933 1.1267131412751e-05
1934 1.13348241939093e-05
1935 1.12868156065815e-05
1936 1.12537773020449e-05
1937 1.12602629087633e-05
1938 1.1262709449511e-05
1939 1.12744864964043e-05
1940 1.12955385702662e-05
1941 1.13162859634031e-05
1942 1.12165143946186e-05
1943 1.13137930384255e-05
1944 1.12563993752701e-05
1945 1.13551141112112e-05
1946 1.12575144157745e-05
1947 1.13158575913985e-05
1948 1.1205817827431e-05
1949 1.13480236905161e-05
1950 1.12337784230476e-05
1951 1.13280684672645e-05
1952 1.12777270260267e-05
1953 1.12875486593111e-05
1954 1.13572477857815e-05
1955 1.12934858407243e-05
1956 1.13258074634359e-05
1957 1.13292426249245e-05
1958 1.12847610580502e-05
1959 1.1418184840295e-05
1960 1.12705120045575e-05
1961 1.13450696517248e-05
1962 1.13673077066778e-05
1963 1.13693431558204e-05
1964 1.13136547952308e-05
1965 1.13974392661476e-05
1966 1.13529995360295e-05
1967 1.13829964902834e-05
1968 1.13871592475334e-05
1969 1.13663718366297e-05
1970 1.13745390990516e-05
1971 1.14068607217632e-05
1972 1.14025397124351e-05
1973 1.13987725853804e-05
1974 1.14858248707606e-05
1975 1.13397181848995e-05
1976 1.14289523480693e-05
1977 1.13839396362891e-05
1978 1.14502981887199e-05
1979 1.13592068373691e-05
1980 1.14583544927882e-05
1981 1.13285632323823e-05
1982 1.13906335172942e-05
1983 1.14120230136905e-05
1984 1.13055111796712e-05
1985 1.13188170871581e-05
1986 1.13480164145585e-05
1987 1.13772284748848e-05
1988 1.13778778541018e-05
1989 1.13296036943211e-05
1990 1.13595478978823e-05
1991 1.13262531158398e-05
1992 1.13184014480794e-05
1993 1.13305641207262e-05
1994 1.12668403744465e-05
1995 1.12661500679678e-05
1996 1.12353855001857e-05
1997 1.12594516394893e-05
1998 1.1213890502404e-05
1999 1.12215684566763e-05
};
\addlegendentry{Test}

\nextgroupplot[
legend cell align={left},
legend style={fill opacity=0.8, draw opacity=1, text opacity=1, at={(0.91,0.5)}, anchor=east, draw=white!80!black},
log basis y={10},
tick align=outside,
tick pos=left,
title={fold2 },
x grid style={white!69.0196078431373!black},
xlabel={Epoch},
xmin=-99.95, xmax=2098.95,
xtick style={color=black},
y grid style={white!69.0196078431373!black},
ylabel={MSE Loss},
ymin=0.00601018615211926, ymax=0.0105993874902354,
ymode=log,
ytick style={color=black}
]
\addplot [semithick, black, dashed]
table {%
0 0.00616719383215198
1 0.00616719383215198
2 0.00616719383215198
3 0.00616719383215198
4 0.00616719383215198
5 0.00616719383215198
6 0.00616719383215198
7 0.00616719383215198
8 0.00616719383215198
9 0.00616719383215198
10 0.00616719383215198
11 0.00616719383215198
12 0.00616719383215198
13 0.00616719383215198
14 0.00616719383215198
15 0.00616719383215198
16 0.00616719383215198
17 0.00616719383215198
18 0.00616719383215198
19 0.00616719383215198
20 0.00616719383215198
21 0.00616719383215198
22 0.00616719383215198
23 0.00616719383215198
24 0.00616719383215198
25 0.00616719383215198
26 0.00616719383215198
27 0.00616719383215198
28 0.00616719383215198
29 0.00616719383215198
30 0.00616719383215198
31 0.00616719383215198
32 0.00616719383215198
33 0.00616719383215198
34 0.00616719383215198
35 0.00616719383215198
36 0.00616719383215198
37 0.00616719383215198
38 0.00616719383215198
39 0.00616719383215198
40 0.00616719383215198
41 0.00616719383215198
42 0.00616719383215198
43 0.00616719383215198
44 0.00616719383215198
45 0.00616719383215198
46 0.00616719383215198
47 0.00616719383215198
48 0.00616719383215198
49 0.00616719383215198
50 0.00616719383215198
51 0.00616719383215198
52 0.00616719383215198
53 0.00616719383215198
54 0.00616719383215198
55 0.00616719383215198
56 0.00616719383215198
57 0.00616719383215198
58 0.00616719383215198
59 0.00616719383215198
60 0.00616719383215198
61 0.00616719383215198
62 0.00616719383215198
63 0.00616719383215198
64 0.00616719383215198
65 0.00616719383215198
66 0.00616719383215198
67 0.00616719383215198
68 0.00616719383215198
69 0.00616719383215198
70 0.00616719383215198
71 0.00616719383215198
72 0.00616719383215198
73 0.00616719383215198
74 0.00616719383215198
75 0.00616719383215198
76 0.00616719383215198
77 0.00616719383215198
78 0.00616719383215198
79 0.00616719383215198
80 0.00616719383215198
81 0.00616719383215198
82 0.00616719383215198
83 0.00616719383215198
84 0.00616719383215198
85 0.00616719383215198
86 0.00616719383215198
87 0.00616719383215198
88 0.00616719383215198
89 0.00616719383215198
90 0.00616719383215198
91 0.00616719383215198
92 0.00616719383215198
93 0.00616719383215198
94 0.00616719383215198
95 0.00616719383215198
96 0.00616719383215198
97 0.00616719383215198
98 0.00616719383215198
99 0.00616719383215198
100 0.00616719383215198
101 0.00616719383215198
102 0.00616719383215198
103 0.00616719383215198
104 0.00616719383215198
105 0.00616719383215198
106 0.00616719383215198
107 0.00616719383215198
108 0.00616719383215198
109 0.00616719383215198
110 0.00616719383215198
111 0.00616719383215198
112 0.00616719383215198
113 0.00616719383215198
114 0.00616719383215198
115 0.00616719383215198
116 0.00616719383215198
117 0.00616719383215198
118 0.00616719383215198
119 0.00616719383215198
120 0.00616719383215198
121 0.00616719383215198
122 0.00616719383215198
123 0.00616719383215198
124 0.00616719383215198
125 0.00616719383215198
126 0.00616719383215198
127 0.00616719383215198
128 0.00616719383215198
129 0.00616719383215198
130 0.00616719383215198
131 0.00616719383215198
132 0.00616719383215198
133 0.00616719383215198
134 0.00616719383215198
135 0.00616719383215198
136 0.00616719383215198
137 0.00616719383215198
138 0.00616719383215198
139 0.00616719383215198
140 0.00616719383215198
141 0.00616719383215198
142 0.00616719383215198
143 0.00616719383215198
144 0.00616719383215198
145 0.00616719383215198
146 0.00616719383215198
147 0.00616719383215198
148 0.00616719383215198
149 0.00616719383215198
150 0.00616719383215198
151 0.00616719383215198
152 0.00616719383215198
153 0.00616719383215198
154 0.00616719383215198
155 0.00616719383215198
156 0.00616719383215198
157 0.00616719383215198
158 0.00616719383215198
159 0.00616719383215198
160 0.00616719383215198
161 0.00616719383215198
162 0.00616719383215198
163 0.00616719383215198
164 0.00616719383215198
165 0.00616719383215198
166 0.00616719383215198
167 0.00616719383215198
168 0.00616719383215198
169 0.00616719383215198
170 0.00616719383215198
171 0.00616719383215198
172 0.00616719383215198
173 0.00616719383215198
174 0.00616719383215198
175 0.00616719383215198
176 0.00616719383215198
177 0.00616719383215198
178 0.00616719383215198
179 0.00616719383215198
180 0.00616719383215198
181 0.00616719383215198
182 0.00616719383215198
183 0.00616719383215198
184 0.00616719383215198
185 0.00616719383215198
186 0.00616719383215198
187 0.00616719383215198
188 0.00616719383215198
189 0.00616719383215198
190 0.00616719383215198
191 0.00616719383215198
192 0.00616719383215198
193 0.00616719383215198
194 0.00616719383215198
195 0.00616719383215198
196 0.00616719383215198
197 0.00616719383215198
198 0.00616719383215198
199 0.00616719383215198
200 0.00616719383215198
201 0.00616719383215198
202 0.00616719383215198
203 0.00616719383215198
204 0.00616719383215198
205 0.00616719383215198
206 0.00616719383215198
207 0.00616719383215198
208 0.00616719383215198
209 0.00616719383215198
210 0.00616719383215198
211 0.00616719383215198
212 0.00616719383215198
213 0.00616719383215198
214 0.00616719383215198
215 0.00616719383215198
216 0.00616719383215198
217 0.00616719383215198
218 0.00616719383215198
219 0.00616719383215198
220 0.00616719383215198
221 0.00616719383215198
222 0.00616719383215198
223 0.00616719383215198
224 0.00616719383215198
225 0.00616719383215198
226 0.00616719383215198
227 0.00616719383215198
228 0.00616719383215198
229 0.00616719383215198
230 0.00616719383215198
231 0.00616719383215198
232 0.00616719383215198
233 0.00616719383215198
234 0.00616719383215198
235 0.00616719383215198
236 0.00616719383215198
237 0.00616719383215198
238 0.00616719383215198
239 0.00616719383215198
240 0.00616719383215198
241 0.00616719383215198
242 0.00616719383215198
243 0.00616719383215198
244 0.00616719383215198
245 0.00616719383215198
246 0.00616719383215198
247 0.00616719383215198
248 0.00616719383215198
249 0.00616719383215198
250 0.00616719383215198
251 0.00616719383215198
252 0.00616719383215198
253 0.00616719383215198
254 0.00616719383215198
255 0.00616719383215198
256 0.00616719383215198
257 0.00616719383215198
258 0.00616719383215198
259 0.00616719383215198
260 0.00616719383215198
261 0.00616719383215198
262 0.00616719383215198
263 0.00616719383215198
264 0.00616719383215198
265 0.00616719383215198
266 0.00616719383215198
267 0.00616719383215198
268 0.00616719383215198
269 0.00616719383215198
270 0.00616719383215198
271 0.00616719383215198
272 0.00616719383215198
273 0.00616719383215198
274 0.00616719383215198
275 0.00616719383215198
276 0.00616719383215198
277 0.00616719383215198
278 0.00616719383215198
279 0.00616719383215198
280 0.00616719383215198
281 0.00616719383215198
282 0.00616719383215198
283 0.00616719383215198
284 0.00616719383215198
285 0.00616719383215198
286 0.00616719383215198
287 0.00616719383215198
288 0.00616719383215198
289 0.00616719383215198
290 0.00616719383215198
291 0.00616719383215198
292 0.00616719383215198
293 0.00616719383215198
294 0.00616719383215198
295 0.00616719383215198
296 0.00616719383215198
297 0.00616719383215198
298 0.00616719383215198
299 0.00616719383215198
300 0.00616719383215198
301 0.00616719383215198
302 0.00616719383215198
303 0.00616719383215198
304 0.00616719383215198
305 0.00616719383215198
306 0.00616719383215198
307 0.00616719383215198
308 0.00616719383215198
309 0.00616719383215198
310 0.00616719383215198
311 0.00616719383215198
312 0.00616719383215198
313 0.00616719383215198
314 0.00616719383215198
315 0.00616719383215198
316 0.00616719383215198
317 0.00616719383215198
318 0.00616719383215198
319 0.00616719383215198
320 0.00616719383215198
321 0.00616719383215198
322 0.00616719383215198
323 0.00616719383215198
324 0.00616719383215198
325 0.00616719383215198
326 0.00616719383215198
327 0.00616719383215198
328 0.00616719383215198
329 0.00616719383215198
330 0.00616719383215198
331 0.00616719383215198
332 0.00616719383215198
333 0.00616719383215198
334 0.00616719383215198
335 0.00616719383215198
336 0.00616719383215198
337 0.00616719383215198
338 0.00616719383215198
339 0.00616719383215198
340 0.00616719383215198
341 0.00616719383215198
342 0.00616719383215198
343 0.00616719383215198
344 0.00616719383215198
345 0.00616719383215198
346 0.00616719383215198
347 0.00616719383215198
348 0.00616719383215198
349 0.00616719383215198
350 0.00616719383215198
351 0.00616719383215198
352 0.00616719383215198
353 0.00616719383215198
354 0.00616719383215198
355 0.00616719383215198
356 0.00616719383215198
357 0.00616719383215198
358 0.00616719383215198
359 0.00616719383215198
360 0.00616719383215198
361 0.00616719383215198
362 0.00616719383215198
363 0.00616719383215198
364 0.00616719383215198
365 0.00616719383215198
366 0.00616719383215198
367 0.00616719383215198
368 0.00616719383215198
369 0.00616719383215198
370 0.00616719383215198
371 0.00616719383215198
372 0.00616719383215198
373 0.00616719383215198
374 0.00616719383215198
375 0.00616719383215198
376 0.00616719383215198
377 0.00616719383215198
378 0.00616719383215198
379 0.00616719383215198
380 0.00616719383215198
381 0.00616719383215198
382 0.00616719383215198
383 0.00616719383215198
384 0.00616719383215198
385 0.00616719383215198
386 0.00616719383215198
387 0.00616719383215198
388 0.00616719383215198
389 0.00616719383215198
390 0.00616719383215198
391 0.00616719383215198
392 0.00616719383215198
393 0.00616719383215198
394 0.00616719383215198
395 0.00616719383215198
396 0.00616719383215198
397 0.00616719383215198
398 0.00616719383215198
399 0.00616719383215198
400 0.00616719383215198
401 0.00616719383215198
402 0.00616719383215198
403 0.00616719383215198
404 0.00616719383215198
405 0.00616719383215198
406 0.00616719383215198
407 0.00616719383215198
408 0.00616719383215198
409 0.00616719383215198
410 0.00616719383215198
411 0.00616719383215198
412 0.00616719383215198
413 0.00616719383215198
414 0.00616719383215198
415 0.00616719383215198
416 0.00616719383215198
417 0.00616719383215198
418 0.00616719383215198
419 0.00616719383215198
420 0.00616719383215198
421 0.00616719383215198
422 0.00616719383215198
423 0.00616719383215198
424 0.00616719383215198
425 0.00616719383215198
426 0.00616719383215198
427 0.00616719383215198
428 0.00616719383215198
429 0.00616719383215198
430 0.00616719383215198
431 0.00616719383215198
432 0.00616719383215198
433 0.00616719383215198
434 0.00616719383215198
435 0.00616719383215198
436 0.00616719383215198
437 0.00616719383215198
438 0.00616719383215198
439 0.00616719383215198
440 0.00616719383215198
441 0.00616719383215198
442 0.00616719383215198
443 0.00616719383215198
444 0.00616719383215198
445 0.00616719383215198
446 0.00616719383215198
447 0.00616719383215198
448 0.00616719383215198
449 0.00616719383215198
450 0.00616719383215198
451 0.00616719383215198
452 0.00616719383215198
453 0.00616719383215198
454 0.00616719383215198
455 0.00616719383215198
456 0.00616719383215198
457 0.00616719383215198
458 0.00616719383215198
459 0.00616719383215198
460 0.00616719383215198
461 0.00616719383215198
462 0.00616719383215198
463 0.00616719383215198
464 0.00616719383215198
465 0.00616719383215198
466 0.00616719383215198
467 0.00616719383215198
468 0.00616719383215198
469 0.00616719383215198
470 0.00616719383215198
471 0.00616719383215198
472 0.00616719383215198
473 0.00616719383215198
474 0.00616719383215198
475 0.00616719383215198
476 0.00616719383215198
477 0.00616719383215198
478 0.00616719383215198
479 0.00616719383215198
480 0.00616719383215198
481 0.00616719383215198
482 0.00616719383215198
483 0.00616719383215198
484 0.00616719383215198
485 0.00616719383215198
486 0.00616719383215198
487 0.00616719383215198
488 0.00616719383215198
489 0.00616719383215198
490 0.00616719383215198
491 0.00616719383215198
492 0.00616719383215198
493 0.00616719383215198
494 0.00616719383215198
495 0.00616719383215198
496 0.00616719383215198
497 0.00616719383215198
498 0.00616719383215198
499 0.00616719383215198
500 0.00616719383215198
501 0.00616719383215198
502 0.00616719383215198
503 0.00616719383215198
504 0.00616719383215198
505 0.00616719383215198
506 0.00616719383215198
507 0.00616719383215198
508 0.00616719383215198
509 0.00616719383215198
510 0.00616719383215198
511 0.00616719383215198
512 0.00616719383215198
513 0.00616719383215198
514 0.00616719383215198
515 0.00616719383215198
516 0.00616719383215198
517 0.00616719383215198
518 0.00616719383215198
519 0.00616719383215198
520 0.00616719383215198
521 0.00616719383215198
522 0.00616719383215198
523 0.00616719383215198
524 0.00616719383215198
525 0.00616719383215198
526 0.00616719383215198
527 0.00616719383215198
528 0.00616719383215198
529 0.00616719383215198
530 0.00616719383215198
531 0.00616719383215198
532 0.00616719383215198
533 0.00616719383215198
534 0.00616719383215198
535 0.00616719383215198
536 0.00616719383215198
537 0.00616719383215198
538 0.00616719383215198
539 0.00616719383215198
540 0.00616719383215198
541 0.00616719383215198
542 0.00616719383215198
543 0.00616719383215198
544 0.00616719383215198
545 0.00616719383215198
546 0.00616719383215198
547 0.00616719383215198
548 0.00616719383215198
549 0.00616719383215198
550 0.00616719383215198
551 0.00616719383215198
552 0.00616719383215198
553 0.00616719383215198
554 0.00616719383215198
555 0.00616719383215198
556 0.00616719383215198
557 0.00616719383215198
558 0.00616719383215198
559 0.00616719383215198
560 0.00616719383215198
561 0.00616719383215198
562 0.00616719383215198
563 0.00616719383215198
564 0.00616719383215198
565 0.00616719383215198
566 0.00616719383215198
567 0.00616719383215198
568 0.00616719383215198
569 0.00616719383215198
570 0.00616719383215198
571 0.00616719383215198
572 0.00616719383215198
573 0.00616719383215198
574 0.00616719383215198
575 0.00616719383215198
576 0.00616719383215198
577 0.00616719383215198
578 0.00616719383215198
579 0.00616719383215198
580 0.00616719383215198
581 0.00616719383215198
582 0.00616719383215198
583 0.00616719383215198
584 0.00616719383215198
585 0.00616719383215198
586 0.00616719383215198
587 0.00616719383215198
588 0.00616719383215198
589 0.00616719383215198
590 0.00616719383215198
591 0.00616719383215198
592 0.00616719383215198
593 0.00616719383215198
594 0.00616719383215198
595 0.00616719383215198
596 0.00616719383215198
597 0.00616719383215198
598 0.00616719383215198
599 0.00616719383215198
600 0.00616719383215198
601 0.00616719383215198
602 0.00616719383215198
603 0.00616719383215198
604 0.00616719383215198
605 0.00616719383215198
606 0.00616719383215198
607 0.00616719383215198
608 0.00616719383215198
609 0.00616719383215198
610 0.00616719383215198
611 0.00616719383215198
612 0.00616719383215198
613 0.00616719383215198
614 0.00616719383215198
615 0.00616719383215198
616 0.00616719383215198
617 0.00616719383215198
618 0.00616719383215198
619 0.00616719383215198
620 0.00616719383215198
621 0.00616719383215198
622 0.00616719383215198
623 0.00616719383215198
624 0.00616719383215198
625 0.00616719383215198
626 0.00616719383215198
627 0.00616719383215198
628 0.00616719383215198
629 0.00616719383215198
630 0.00616719383215198
631 0.00616719383215198
632 0.00616719383215198
633 0.00616719383215198
634 0.00616719383215198
635 0.00616719383215198
636 0.00616719383215198
637 0.00616719383215198
638 0.00616719383215198
639 0.00616719383215198
640 0.00616719383215198
641 0.00616719383215198
642 0.00616719383215198
643 0.00616719383215198
644 0.00616719383215198
645 0.00616719383215198
646 0.00616719383215198
647 0.00616719383215198
648 0.00616719383215198
649 0.00616719383215198
650 0.00616719383215198
651 0.00616719383215198
652 0.00616719383215198
653 0.00616719383215198
654 0.00616719383215198
655 0.00616719383215198
656 0.00616719383215198
657 0.00616719383215198
658 0.00616719383215198
659 0.00616719383215198
660 0.00616719383215198
661 0.00616719383215198
662 0.00616719383215198
663 0.00616719383215198
664 0.00616719383215198
665 0.00616719383215198
666 0.00616719383215198
667 0.00616719383215198
668 0.00616719383215198
669 0.00616719383215198
670 0.00616719383215198
671 0.00616719383215198
672 0.00616719383215198
673 0.00616719383215198
674 0.00616719383215198
675 0.00616719383215198
676 0.00616719383215198
677 0.00616719383215198
678 0.00616719383215198
679 0.00616719383215198
680 0.00616719383215198
681 0.00616719383215198
682 0.00616719383215198
683 0.00616719383215198
684 0.00616719383215198
685 0.00616719383215198
686 0.00616719383215198
687 0.00616719383215198
688 0.00616719383215198
689 0.00616719383215198
690 0.00616719383215198
691 0.00616719383215198
692 0.00616719383215198
693 0.00616719383215198
694 0.00616719383215198
695 0.00616719383215198
696 0.00616719383215198
697 0.00616719383215198
698 0.00616719383215198
699 0.00616719383215198
700 0.00616719383215198
701 0.00616719383215198
702 0.00616719383215198
703 0.00616719383215198
704 0.00616719383215198
705 0.00616719383215198
706 0.00616719383215198
707 0.00616719383215198
708 0.00616719383215198
709 0.00616719383215198
710 0.00616719383215198
711 0.00616719383215198
712 0.00616719383215198
713 0.00616719383215198
714 0.00616719383215198
715 0.00616719383215198
716 0.00616719383215198
717 0.00616719383215198
718 0.00616719383215198
719 0.00616719383215198
720 0.00616719383215198
721 0.00616719383215198
722 0.00616719383215198
723 0.00616719383215198
724 0.00616719383215198
725 0.00616719383215198
726 0.00616719383215198
727 0.00616719383215198
728 0.00616719383215198
729 0.00616719383215198
730 0.00616719383215198
731 0.00616719383215198
732 0.00616719383215198
733 0.00616719383215198
734 0.00616719383215198
735 0.00616719383215198
736 0.00616719383215198
737 0.00616719383215198
738 0.00616719383215198
739 0.00616719383215198
740 0.00616719383215198
741 0.00616719383215198
742 0.00616719383215198
743 0.00616719383215198
744 0.00616719383215198
745 0.00616719383215198
746 0.00616719383215198
747 0.00616719383215198
748 0.00616719383215198
749 0.00616719383215198
750 0.00616719383215198
751 0.00616719383215198
752 0.00616719383215198
753 0.00616719383215198
754 0.00616719383215198
755 0.00616719383215198
756 0.00616719383215198
757 0.00616719383215198
758 0.00616719383215198
759 0.00616719383215198
760 0.00616719383215198
761 0.00616719383215198
762 0.00616719383215198
763 0.00616719383215198
764 0.00616719383215198
765 0.00616719383215198
766 0.00616719383215198
767 0.00616719383215198
768 0.00616719383215198
769 0.00616719383215198
770 0.00616719383215198
771 0.00616719383215198
772 0.00616719383215198
773 0.00616719383215198
774 0.00616719383215198
775 0.00616719383215198
776 0.00616719383215198
777 0.00616719383215198
778 0.00616719383215198
779 0.00616719383215198
780 0.00616719383215198
781 0.00616719383215198
782 0.00616719383215198
783 0.00616719383215198
784 0.00616719383215198
785 0.00616719383215198
786 0.00616719383215198
787 0.00616719383215198
788 0.00616719383215198
789 0.00616719383215198
790 0.00616719383215198
791 0.00616719383215198
792 0.00616719383215198
793 0.00616719383215198
794 0.00616719383215198
795 0.00616719383215198
796 0.00616719383215198
797 0.00616719383215198
798 0.00616719383215198
799 0.00616719383215198
800 0.00616719383215198
801 0.00616719383215198
802 0.00616719383215198
803 0.00616719383215198
804 0.00616719383215198
805 0.00616719383215198
806 0.00616719383215198
807 0.00616719383215198
808 0.00616719383215198
809 0.00616719383215198
810 0.00616719383215198
811 0.00616719383215198
812 0.00616719383215198
813 0.00616719383215198
814 0.00616719383215198
815 0.00616719383215198
816 0.00616719383215198
817 0.00616719383215198
818 0.00616719383215198
819 0.00616719383215198
820 0.00616719383215198
821 0.00616719383215198
822 0.00616719383215198
823 0.00616719383215198
824 0.00616719383215198
825 0.00616719383215198
826 0.00616719383215198
827 0.00616719383215198
828 0.00616719383215198
829 0.00616719383215198
830 0.00616719383215198
831 0.00616719383215198
832 0.00616719383215198
833 0.00616719383215198
834 0.00616719383215198
835 0.00616719383215198
836 0.00616719383215198
837 0.00616719383215198
838 0.00616719383215198
839 0.00616719383215198
840 0.00616719383215198
841 0.00616719383215198
842 0.00616719383215198
843 0.00616719383215198
844 0.00616719383215198
845 0.00616719383215198
846 0.00616719383215198
847 0.00616719383215198
848 0.00616719383215198
849 0.00616719383215198
850 0.00616719383215198
851 0.00616719383215198
852 0.00616719383215198
853 0.00616719383215198
854 0.00616719383215198
855 0.00616719383215198
856 0.00616719383215198
857 0.00616719383215198
858 0.00616719383215198
859 0.00616719383215198
860 0.00616719383215198
861 0.00616719383215198
862 0.00616719383215198
863 0.00616719383215198
864 0.00616719383215198
865 0.00616719383215198
866 0.00616719383215198
867 0.00616719383215198
868 0.00616719383215198
869 0.00616719383215198
870 0.00616719383215198
871 0.00616719383215198
872 0.00616719383215198
873 0.00616719383215198
874 0.00616719383215198
875 0.00616719383215198
876 0.00616719383215198
877 0.00616719383215198
878 0.00616719383215198
879 0.00616719383215198
880 0.00616719383215198
881 0.00616719383215198
882 0.00616719383215198
883 0.00616719383215198
884 0.00616719383215198
885 0.00616719383215198
886 0.00616719383215198
887 0.00616719383215198
888 0.00616719383215198
889 0.00616719383215198
890 0.00616719383215198
891 0.00616719383215198
892 0.00616719383215198
893 0.00616719383215198
894 0.00616719383215198
895 0.00616719383215198
896 0.00616719383215198
897 0.00616719383215198
898 0.00616719383215198
899 0.00616719383215198
900 0.00616719383215198
901 0.00616719383215198
902 0.00616719383215198
903 0.00616719383215198
904 0.00616719383215198
905 0.00616719383215198
906 0.00616719383215198
907 0.00616719383215198
908 0.00616719383215198
909 0.00616719383215198
910 0.00616719383215198
911 0.00616719383215198
912 0.00616719383215198
913 0.00616719383215198
914 0.00616719383215198
915 0.00616719383215198
916 0.00616719383215198
917 0.00616719383215198
918 0.00616719383215198
919 0.00616719383215198
920 0.00616719383215198
921 0.00616719383215198
922 0.00616719383215198
923 0.00616719383215198
924 0.00616719383215198
925 0.00616719383215198
926 0.00616719383215198
927 0.00616719383215198
928 0.00616719383215198
929 0.00616719383215198
930 0.00616719383215198
931 0.00616719383215198
932 0.00616719383215198
933 0.00616719383215198
934 0.00616719383215198
935 0.00616719383215198
936 0.00616719383215198
937 0.00616719383215198
938 0.00616719383215198
939 0.00616719383215198
940 0.00616719383215198
941 0.00616719383215198
942 0.00616719383215198
943 0.00616719383215198
944 0.00616719383215198
945 0.00616719383215198
946 0.00616719383215198
947 0.00616719383215198
948 0.00616719383215198
949 0.00616719383215198
950 0.00616719383215198
951 0.00616719383215198
952 0.00616719383215198
953 0.00616719383215198
954 0.00616719383215198
955 0.00616719383215198
956 0.00616719383215198
957 0.00616719383215198
958 0.00616719383215198
959 0.00616719383215198
960 0.00616719383215198
961 0.00616719383215198
962 0.00616719383215198
963 0.00616719383215198
964 0.00616719383215198
965 0.00616719383215198
966 0.00616719383215198
967 0.00616719383215198
968 0.00616719383215198
969 0.00616719383215198
970 0.00616719383215198
971 0.00616719383215198
972 0.00616719383215198
973 0.00616719383215198
974 0.00616719383215198
975 0.00616719383215198
976 0.00616719383215198
977 0.00616719383215198
978 0.00616719383215198
979 0.00616719383215198
980 0.00616719383215198
981 0.00616719383215198
982 0.00616719383215198
983 0.00616719383215198
984 0.00616719383215198
985 0.00616719383215198
986 0.00616719383215198
987 0.00616719383215198
988 0.00616719383215198
989 0.00616719383215198
990 0.00616719383215198
991 0.00616719383215198
992 0.00616719383215198
993 0.00616719383215198
994 0.00616719383215198
995 0.00616719383215198
996 0.00616719383215198
997 0.00616719383215198
998 0.00616719383215198
999 0.00616719383215198
1000 0.00616719383215198
1001 0.00616719383215198
1002 0.00616719383215198
1003 0.00616719383215198
1004 0.00616719383215198
1005 0.00616719383215198
1006 0.00616719383215198
1007 0.00616719383215198
1008 0.00616719383215198
1009 0.00616719383215198
1010 0.00616719383215198
1011 0.00616719383215198
1012 0.00616719383215198
1013 0.00616719383215198
1014 0.00616719383215198
1015 0.00616719383215198
1016 0.00616719383215198
1017 0.00616719383215198
1018 0.00616719383215198
1019 0.00616719383215198
1020 0.00616719383215198
1021 0.00616719383215198
1022 0.00616719383215198
1023 0.00616719383215198
1024 0.00616719383215198
1025 0.00616719383215198
1026 0.00616719383215198
1027 0.00616719383215198
1028 0.00616719383215198
1029 0.00616719383215198
1030 0.00616719383215198
1031 0.00616719383215198
1032 0.00616719383215198
1033 0.00616719383215198
1034 0.00616719383215198
1035 0.00616719383215198
1036 0.00616719383215198
1037 0.00616719383215198
1038 0.00616719383215198
1039 0.00616719383215198
1040 0.00616719383215198
1041 0.00616719383215198
1042 0.00616719383215198
1043 0.00616719383215198
1044 0.00616719383215198
1045 0.00616719383215198
1046 0.00616719383215198
1047 0.00616719383215198
1048 0.00616719383215198
1049 0.00616719383215198
1050 0.00616719383215198
1051 0.00616719383215198
1052 0.00616719383215198
1053 0.00616719383215198
1054 0.00616719383215198
1055 0.00616719383215198
1056 0.00616719383215198
1057 0.00616719383215198
1058 0.00616719383215198
1059 0.00616719383215198
1060 0.00616719383215198
1061 0.00616719383215198
1062 0.00616719383215198
1063 0.00616719383215198
1064 0.00616719383215198
1065 0.00616719383215198
1066 0.00616719383215198
1067 0.00616719383215198
1068 0.00616719383215198
1069 0.00616719383215198
1070 0.00616719383215198
1071 0.00616719383215198
1072 0.00616719383215198
1073 0.00616719383215198
1074 0.00616719383215198
1075 0.00616719383215198
1076 0.00616719383215198
1077 0.00616719383215198
1078 0.00616719383215198
1079 0.00616719383215198
1080 0.00616719383215198
1081 0.00616719383215198
1082 0.00616719383215198
1083 0.00616719383215198
1084 0.00616719383215198
1085 0.00616719383215198
1086 0.00616719383215198
1087 0.00616719383215198
1088 0.00616719383215198
1089 0.00616719383215198
1090 0.00616719383215198
1091 0.00616719383215198
1092 0.00616719383215198
1093 0.00616719383215198
1094 0.00616719383215198
1095 0.00616719383215198
1096 0.00616719383215198
1097 0.00616719383215198
1098 0.00616719383215198
1099 0.00616719383215198
1100 0.00616719383215198
1101 0.00616719383215198
1102 0.00616719383215198
1103 0.00616719383215198
1104 0.00616719383215198
1105 0.00616719383215198
1106 0.00616719383215198
1107 0.00616719383215198
1108 0.00616719383215198
1109 0.00616719383215198
1110 0.00616719383215198
1111 0.00616719383215198
1112 0.00616719383215198
1113 0.00616719383215198
1114 0.00616719383215198
1115 0.00616719383215198
1116 0.00616719383215198
1117 0.00616719383215198
1118 0.00616719383215198
1119 0.00616719383215198
1120 0.00616719383215198
1121 0.00616719383215198
1122 0.00616719383215198
1123 0.00616719383215198
1124 0.00616719383215198
1125 0.00616719383215198
1126 0.00616719383215198
1127 0.00616719383215198
1128 0.00616719383215198
1129 0.00616719383215198
1130 0.00616719383215198
1131 0.00616719383215198
1132 0.00616719383215198
1133 0.00616719383215198
1134 0.00616719383215198
1135 0.00616719383215198
1136 0.00616719383215198
1137 0.00616719383215198
1138 0.00616719383215198
1139 0.00616719383215198
1140 0.00616719383215198
1141 0.00616719383215198
1142 0.00616719383215198
1143 0.00616719383215198
1144 0.00616719383215198
1145 0.00616719383215198
1146 0.00616719383215198
1147 0.00616719383215198
1148 0.00616719383215198
1149 0.00616719383215198
1150 0.00616719383215198
1151 0.00616719383215198
1152 0.00616719383215198
1153 0.00616719383215198
1154 0.00616719383215198
1155 0.00616719383215198
1156 0.00616719383215198
1157 0.00616719383215198
1158 0.00616719383215198
1159 0.00616719383215198
1160 0.00616719383215198
1161 0.00616719383215198
1162 0.00616719383215198
1163 0.00616719383215198
1164 0.00616719383215198
1165 0.00616719383215198
1166 0.00616719383215198
1167 0.00616719383215198
1168 0.00616719383215198
1169 0.00616719383215198
1170 0.00616719383215198
1171 0.00616719383215198
1172 0.00616719383215198
1173 0.00616719383215198
1174 0.00616719383215198
1175 0.00616719383215198
1176 0.00616719383215198
1177 0.00616719383215198
1178 0.00616719383215198
1179 0.00616719383215198
1180 0.00616719383215198
1181 0.00616719383215198
1182 0.00616719383215198
1183 0.00616719383215198
1184 0.00616719383215198
1185 0.00616719383215198
1186 0.00616719383215198
1187 0.00616719383215198
1188 0.00616719383215198
1189 0.00616719383215198
1190 0.00616719383215198
1191 0.00616719383215198
1192 0.00616719383215198
1193 0.00616719383215198
1194 0.00616719383215198
1195 0.00616719383215198
1196 0.00616719383215198
1197 0.00616719383215198
1198 0.00616719383215198
1199 0.00616719383215198
1200 0.00616719383215198
1201 0.00616719383215198
1202 0.00616719383215198
1203 0.00616719383215198
1204 0.00616719383215198
1205 0.00616719383215198
1206 0.00616719383215198
1207 0.00616719383215198
1208 0.00616719383215198
1209 0.00616719383215198
1210 0.00616719383215198
1211 0.00616719383215198
1212 0.00616719383215198
1213 0.00616719383215198
1214 0.00616719383215198
1215 0.00616719383215198
1216 0.00616719383215198
1217 0.00616719383215198
1218 0.00616719383215198
1219 0.00616719383215198
1220 0.00616719383215198
1221 0.00616719383215198
1222 0.00616719383215198
1223 0.00616719383215198
1224 0.00616719383215198
1225 0.00616719383215198
1226 0.00616719383215198
1227 0.00616719383215198
1228 0.00616719383215198
1229 0.00616719383215198
1230 0.00616719383215198
1231 0.00616719383215198
1232 0.00616719383215198
1233 0.00616719383215198
1234 0.00616719383215198
1235 0.00616719383215198
1236 0.00616719383215198
1237 0.00616719383215198
1238 0.00616719383215198
1239 0.00616719383215198
1240 0.00616719383215198
1241 0.00616719383215198
1242 0.00616719383215198
1243 0.00616719383215198
1244 0.00616719383215198
1245 0.00616719383215198
1246 0.00616719383215198
1247 0.00616719383215198
1248 0.00616719383215198
1249 0.00616719383215198
1250 0.00616719383215198
1251 0.00616719383215198
1252 0.00616719383215198
1253 0.00616719383215198
1254 0.00616719383215198
1255 0.00616719383215198
1256 0.00616719383215198
1257 0.00616719383215198
1258 0.00616719383215198
1259 0.00616719383215198
1260 0.00616719383215198
1261 0.00616719383215198
1262 0.00616719383215198
1263 0.00616719383215198
1264 0.00616719383215198
1265 0.00616719383215198
1266 0.00616719383215198
1267 0.00616719383215198
1268 0.00616719383215198
1269 0.00616719383215198
1270 0.00616719383215198
1271 0.00616719383215198
1272 0.00616719383215198
1273 0.00616719383215198
1274 0.00616719383215198
1275 0.00616719383215198
1276 0.00616719383215198
1277 0.00616719383215198
1278 0.00616719383215198
1279 0.00616719383215198
1280 0.00616719383215198
1281 0.00616719383215198
1282 0.00616719383215198
1283 0.00616719383215198
1284 0.00616719383215198
1285 0.00616719383215198
1286 0.00616719383215198
1287 0.00616719383215198
1288 0.00616719383215198
1289 0.00616719383215198
1290 0.00616719383215198
1291 0.00616719383215198
1292 0.00616719383215198
1293 0.00616719383215198
1294 0.00616719383215198
1295 0.00616719383215198
1296 0.00616719383215198
1297 0.00616719383215198
1298 0.00616719383215198
1299 0.00616719383215198
1300 0.00616719383215198
1301 0.00616719383215198
1302 0.00616719383215198
1303 0.00616719383215198
1304 0.00616719383215198
1305 0.00616719383215198
1306 0.00616719383215198
1307 0.00616719383215198
1308 0.00616719383215198
1309 0.00616719383215198
1310 0.00616719383215198
1311 0.00616719383215198
1312 0.00616719383215198
1313 0.00616719383215198
1314 0.00616719383215198
1315 0.00616719383215198
1316 0.00616719383215198
1317 0.00616719383215198
1318 0.00616719383215198
1319 0.00616719383215198
1320 0.00616719383215198
1321 0.00616719383215198
1322 0.00616719383215198
1323 0.00616719383215198
1324 0.00616719383215198
1325 0.00616719383215198
1326 0.00616719383215198
1327 0.00616719383215198
1328 0.00616719383215198
1329 0.00616719383215198
1330 0.00616719383215198
1331 0.00616719383215198
1332 0.00616719383215198
1333 0.00616719383215198
1334 0.00616719383215198
1335 0.00616719383215198
1336 0.00616719383215198
1337 0.00616719383215198
1338 0.00616719383215198
1339 0.00616719383215198
1340 0.00616719383215198
1341 0.00616719383215198
1342 0.00616719383215198
1343 0.00616719383215198
1344 0.00616719383215198
1345 0.00616719383215198
1346 0.00616719383215198
1347 0.00616719383215198
1348 0.00616719383215198
1349 0.00616719383215198
1350 0.00616719383215198
1351 0.00616719383215198
1352 0.00616719383215198
1353 0.00616719383215198
1354 0.00616719383215198
1355 0.00616719383215198
1356 0.00616719383215198
1357 0.00616719383215198
1358 0.00616719383215198
1359 0.00616719383215198
1360 0.00616719383215198
1361 0.00616719383215198
1362 0.00616719383215198
1363 0.00616719383215198
1364 0.00616719383215198
1365 0.00616719383215198
1366 0.00616719383215198
1367 0.00616719383215198
1368 0.00616719383215198
1369 0.00616719383215198
1370 0.00616719383215198
1371 0.00616719383215198
1372 0.00616719383215198
1373 0.00616719383215198
1374 0.00616719383215198
1375 0.00616719383215198
1376 0.00616719383215198
1377 0.00616719383215198
1378 0.00616719383215198
1379 0.00616719383215198
1380 0.00616719383215198
1381 0.00616719383215198
1382 0.00616719383215198
1383 0.00616719383215198
1384 0.00616719383215198
1385 0.00616719383215198
1386 0.00616719383215198
1387 0.00616719383215198
1388 0.00616719383215198
1389 0.00616719383215198
1390 0.00616719383215198
1391 0.00616719383215198
1392 0.00616719383215198
1393 0.00616719383215198
1394 0.00616719383215198
1395 0.00616719383215198
1396 0.00616719383215198
1397 0.00616719383215198
1398 0.00616719383215198
1399 0.00616719383215198
1400 0.00616719383215198
1401 0.00616719383215198
1402 0.00616719383215198
1403 0.00616719383215198
1404 0.00616719383215198
1405 0.00616719383215198
1406 0.00616719383215198
1407 0.00616719383215198
1408 0.00616719383215198
1409 0.00616719383215198
1410 0.00616719383215198
1411 0.00616719383215198
1412 0.00616719383215198
1413 0.00616719383215198
1414 0.00616719383215198
1415 0.00616719383215198
1416 0.00616719383215198
1417 0.00616719383215198
1418 0.00616719383215198
1419 0.00616719383215198
1420 0.00616719383215198
1421 0.00616719383215198
1422 0.00616719383215198
1423 0.00616719383215198
1424 0.00616719383215198
1425 0.00616719383215198
1426 0.00616719383215198
1427 0.00616719383215198
1428 0.00616719383215198
1429 0.00616719383215198
1430 0.00616719383215198
1431 0.00616719383215198
1432 0.00616719383215198
1433 0.00616719383215198
1434 0.00616719383215198
1435 0.00616719383215198
1436 0.00616719383215198
1437 0.00616719383215198
1438 0.00616719383215198
1439 0.00616719383215198
1440 0.00616719383215198
1441 0.00616719383215198
1442 0.00616719383215198
1443 0.00616719383215198
1444 0.00616719383215198
1445 0.00616719383215198
1446 0.00616719383215198
1447 0.00616719383215198
1448 0.00616719383215198
1449 0.00616719383215198
1450 0.00616719383215198
1451 0.00616719383215198
1452 0.00616719383215198
1453 0.00616719383215198
1454 0.00616719383215198
1455 0.00616719383215198
1456 0.00616719383215198
1457 0.00616719383215198
1458 0.00616719383215198
1459 0.00616719383215198
1460 0.00616719383215198
1461 0.00616719383215198
1462 0.00616719383215198
1463 0.00616719383215198
1464 0.00616719383215198
1465 0.00616719383215198
1466 0.00616719383215198
1467 0.00616719383215198
1468 0.00616719383215198
1469 0.00616719383215198
1470 0.00616719383215198
1471 0.00616719383215198
1472 0.00616719383215198
1473 0.00616719383215198
1474 0.00616719383215198
1475 0.00616719383215198
1476 0.00616719383215198
1477 0.00616719383215198
1478 0.00616719383215198
1479 0.00616719383215198
1480 0.00616719383215198
1481 0.00616719383215198
1482 0.00616719383215198
1483 0.00616719383215198
1484 0.00616719383215198
1485 0.00616719383215198
1486 0.00616719383215198
1487 0.00616719383215198
1488 0.00616719383215198
1489 0.00616719383215198
1490 0.00616719383215198
1491 0.00616719383215198
1492 0.00616719383215198
1493 0.00616719383215198
1494 0.00616719383215198
1495 0.00616719383215198
1496 0.00616719383215198
1497 0.00616719383215198
1498 0.00616719383215198
1499 0.00616719383215198
1500 0.00616719383215198
1501 0.00616719383215198
1502 0.00616719383215198
1503 0.00616719383215198
1504 0.00616719383215198
1505 0.00616719383215198
1506 0.00616719383215198
1507 0.00616719383215198
1508 0.00616719383215198
1509 0.00616719383215198
1510 0.00616719383215198
1511 0.00616719383215198
1512 0.00616719383215198
1513 0.00616719383215198
1514 0.00616719383215198
1515 0.00616719383215198
1516 0.00616719383215198
1517 0.00616719383215198
1518 0.00616719383215198
1519 0.00616719383215198
1520 0.00616719383215198
1521 0.00616719383215198
1522 0.00616719383215198
1523 0.00616719383215198
1524 0.00616719383215198
1525 0.00616719383215198
1526 0.00616719383215198
1527 0.00616719383215198
1528 0.00616719383215198
1529 0.00616719383215198
1530 0.00616719383215198
1531 0.00616719383215198
1532 0.00616719383215198
1533 0.00616719383215198
1534 0.00616719383215198
1535 0.00616719383215198
1536 0.00616719383215198
1537 0.00616719383215198
1538 0.00616719383215198
1539 0.00616719383215198
1540 0.00616719383215198
1541 0.00616719383215198
1542 0.00616719383215198
1543 0.00616719383215198
1544 0.00616719383215198
1545 0.00616719383215198
1546 0.00616719383215198
1547 0.00616719383215198
1548 0.00616719383215198
1549 0.00616719383215198
1550 0.00616719383215198
1551 0.00616719383215198
1552 0.00616719383215198
1553 0.00616719383215198
1554 0.00616719383215198
1555 0.00616719383215198
1556 0.00616719383215198
1557 0.00616719383215198
1558 0.00616719383215198
1559 0.00616719383215198
1560 0.00616719383215198
1561 0.00616719383215198
1562 0.00616719383215198
1563 0.00616719383215198
1564 0.00616719383215198
1565 0.00616719383215198
1566 0.00616719383215198
1567 0.00616719383215198
1568 0.00616719383215198
1569 0.00616719383215198
1570 0.00616719383215198
1571 0.00616719383215198
1572 0.00616719383215198
1573 0.00616719383215198
1574 0.00616719383215198
1575 0.00616719383215198
1576 0.00616719383215198
1577 0.00616719383215198
1578 0.00616719383215198
1579 0.00616719383215198
1580 0.00616719383215198
1581 0.00616719383215198
1582 0.00616719383215198
1583 0.00616719383215198
1584 0.00616719383215198
1585 0.00616719383215198
1586 0.00616719383215198
1587 0.00616719383215198
1588 0.00616719383215198
1589 0.00616719383215198
1590 0.00616719383215198
1591 0.00616719383215198
1592 0.00616719383215198
1593 0.00616719383215198
1594 0.00616719383215198
1595 0.00616719383215198
1596 0.00616719383215198
1597 0.00616719383215198
1598 0.00616719383215198
1599 0.00616719383215198
1600 0.00616719383215198
1601 0.00616719383215198
1602 0.00616719383215198
1603 0.00616719383215198
1604 0.00616719383215198
1605 0.00616719383215198
1606 0.00616719383215198
1607 0.00616719383215198
1608 0.00616719383215198
1609 0.00616719383215198
1610 0.00616719383215198
1611 0.00616719383215198
1612 0.00616719383215198
1613 0.00616719383215198
1614 0.00616719383215198
1615 0.00616719383215198
1616 0.00616719383215198
1617 0.00616719383215198
1618 0.00616719383215198
1619 0.00616719383215198
1620 0.00616719383215198
1621 0.00616719383215198
1622 0.00616719383215198
1623 0.00616719383215198
1624 0.00616719383215198
1625 0.00616719383215198
1626 0.00616719383215198
1627 0.00616719383215198
1628 0.00616719383215198
1629 0.00616719383215198
1630 0.00616719383215198
1631 0.00616719383215198
1632 0.00616719383215198
1633 0.00616719383215198
1634 0.00616719383215198
1635 0.00616719383215198
1636 0.00616719383215198
1637 0.00616719383215198
1638 0.00616719383215198
1639 0.00616719383215198
1640 0.00616719383215198
1641 0.00616719383215198
1642 0.00616719383215198
1643 0.00616719383215198
1644 0.00616719383215198
1645 0.00616719383215198
1646 0.00616719383215198
1647 0.00616719383215198
1648 0.00616719383215198
1649 0.00616719383215198
1650 0.00616719383215198
1651 0.00616719383215198
1652 0.00616719383215198
1653 0.00616719383215198
1654 0.00616719383215198
1655 0.00616719383215198
1656 0.00616719383215198
1657 0.00616719383215198
1658 0.00616719383215198
1659 0.00616719383215198
1660 0.00616719383215198
1661 0.00616719383215198
1662 0.00616719383215198
1663 0.00616719383215198
1664 0.00616719383215198
1665 0.00616719383215198
1666 0.00616719383215198
1667 0.00616719383215198
1668 0.00616719383215198
1669 0.00616719383215198
1670 0.00616719383215198
1671 0.00616719383215198
1672 0.00616719383215198
1673 0.00616719383215198
1674 0.00616719383215198
1675 0.00616719383215198
1676 0.00616719383215198
1677 0.00616719383215198
1678 0.00616719383215198
1679 0.00616719383215198
1680 0.00616719383215198
1681 0.00616719383215198
1682 0.00616719383215198
1683 0.00616719383215198
1684 0.00616719383215198
1685 0.00616719383215198
1686 0.00616719383215198
1687 0.00616719383215198
1688 0.00616719383215198
1689 0.00616719383215198
1690 0.00616719383215198
1691 0.00616719383215198
1692 0.00616719383215198
1693 0.00616719383215198
1694 0.00616719383215198
1695 0.00616719383215198
1696 0.00616719383215198
1697 0.00616719383215198
1698 0.00616719383215198
1699 0.00616719383215198
1700 0.00616719383215198
1701 0.00616719383215198
1702 0.00616719383215198
1703 0.00616719383215198
1704 0.00616719383215198
1705 0.00616719383215198
1706 0.00616719383215198
1707 0.00616719383215198
1708 0.00616719383215198
1709 0.00616719383215198
1710 0.00616719383215198
1711 0.00616719383215198
1712 0.00616719383215198
1713 0.00616719383215198
1714 0.00616719383215198
1715 0.00616719383215198
1716 0.00616719383215198
1717 0.00616719383215198
1718 0.00616719383215198
1719 0.00616719383215198
1720 0.00616719383215198
1721 0.00616719383215198
1722 0.00616719383215198
1723 0.00616719383215198
1724 0.00616719383215198
1725 0.00616719383215198
1726 0.00616719383215198
1727 0.00616719383215198
1728 0.00616719383215198
1729 0.00616719383215198
1730 0.00616719383215198
1731 0.00616719383215198
1732 0.00616719383215198
1733 0.00616719383215198
1734 0.00616719383215198
1735 0.00616719383215198
1736 0.00616719383215198
1737 0.00616719383215198
1738 0.00616719383215198
1739 0.00616719383215198
1740 0.00616719383215198
1741 0.00616719383215198
1742 0.00616719383215198
1743 0.00616719383215198
1744 0.00616719383215198
1745 0.00616719383215198
1746 0.00616719383215198
1747 0.00616719383215198
1748 0.00616719383215198
1749 0.00616719383215198
1750 0.00616719383215198
1751 0.00616719383215198
1752 0.00616719383215198
1753 0.00616719383215198
1754 0.00616719383215198
1755 0.00616719383215198
1756 0.00616719383215198
1757 0.00616719383215198
1758 0.00616719383215198
1759 0.00616719383215198
1760 0.00616719383215198
1761 0.00616719383215198
1762 0.00616719383215198
1763 0.00616719383215198
1764 0.00616719383215198
1765 0.00616719383215198
1766 0.00616719383215198
1767 0.00616719383215198
1768 0.00616719383215198
1769 0.00616719383215198
1770 0.00616719383215198
1771 0.00616719383215198
1772 0.00616719383215198
1773 0.00616719383215198
1774 0.00616719383215198
1775 0.00616719383215198
1776 0.00616719383215198
1777 0.00616719383215198
1778 0.00616719383215198
1779 0.00616719383215198
1780 0.00616719383215198
1781 0.00616719383215198
1782 0.00616719383215198
1783 0.00616719383215198
1784 0.00616719383215198
1785 0.00616719383215198
1786 0.00616719383215198
1787 0.00616719383215198
1788 0.00616719383215198
1789 0.00616719383215198
1790 0.00616719383215198
1791 0.00616719383215198
1792 0.00616719383215198
1793 0.00616719383215198
1794 0.00616719383215198
1795 0.00616719383215198
1796 0.00616719383215198
1797 0.00616719383215198
1798 0.00616719383215198
1799 0.00616719383215198
1800 0.00616719383215198
1801 0.00616719383215198
1802 0.00616719383215198
1803 0.00616719383215198
1804 0.00616719383215198
1805 0.00616719383215198
1806 0.00616719383215198
1807 0.00616719383215198
1808 0.00616719383215198
1809 0.00616719383215198
1810 0.00616719383215198
1811 0.00616719383215198
1812 0.00616719383215198
1813 0.00616719383215198
1814 0.00616719383215198
1815 0.00616719383215198
1816 0.00616719383215198
1817 0.00616719383215198
1818 0.00616719383215198
1819 0.00616719383215198
1820 0.00616719383215198
1821 0.00616719383215198
1822 0.00616719383215198
1823 0.00616719383215198
1824 0.00616719383215198
1825 0.00616719383215198
1826 0.00616719383215198
1827 0.00616719383215198
1828 0.00616719383215198
1829 0.00616719383215198
1830 0.00616719383215198
1831 0.00616719383215198
1832 0.00616719383215198
1833 0.00616719383215198
1834 0.00616719383215198
1835 0.00616719383215198
1836 0.00616719383215198
1837 0.00616719383215198
1838 0.00616719383215198
1839 0.00616719383215198
1840 0.00616719383215198
1841 0.00616719383215198
1842 0.00616719383215198
1843 0.00616719383215198
1844 0.00616719383215198
1845 0.00616719383215198
1846 0.00616719383215198
1847 0.00616719383215198
1848 0.00616719383215198
1849 0.00616719383215198
1850 0.00616719383215198
1851 0.00616719383215198
1852 0.00616719383215198
1853 0.00616719383215198
1854 0.00616719383215198
1855 0.00616719383215198
1856 0.00616719383215198
1857 0.00616719383215198
1858 0.00616719383215198
1859 0.00616719383215198
1860 0.00616719383215198
1861 0.00616719383215198
1862 0.00616719383215198
1863 0.00616719383215198
1864 0.00616719383215198
1865 0.00616719383215198
1866 0.00616719383215198
1867 0.00616719383215198
1868 0.00616719383215198
1869 0.00616719383215198
1870 0.00616719383215198
1871 0.00616719383215198
1872 0.00616719383215198
1873 0.00616719383215198
1874 0.00616719383215198
1875 0.00616719383215198
1876 0.00616719383215198
1877 0.00616719383215198
1878 0.00616719383215198
1879 0.00616719383215198
1880 0.00616719383215198
1881 0.00616719383215198
1882 0.00616719383215198
1883 0.00616719383215198
1884 0.00616719383215198
1885 0.00616719383215198
1886 0.00616719383215198
1887 0.00616719383215198
1888 0.00616719383215198
1889 0.00616719383215198
1890 0.00616719383215198
1891 0.00616719383215198
1892 0.00616719383215198
1893 0.00616719383215198
1894 0.00616719383215198
1895 0.00616719383215198
1896 0.00616719383215198
1897 0.00616719383215198
1898 0.00616719383215198
1899 0.00616719383215198
1900 0.00616719383215198
1901 0.00616719383215198
1902 0.00616719383215198
1903 0.00616719383215198
1904 0.00616719383215198
1905 0.00616719383215198
1906 0.00616719383215198
1907 0.00616719383215198
1908 0.00616719383215198
1909 0.00616719383215198
1910 0.00616719383215198
1911 0.00616719383215198
1912 0.00616719383215198
1913 0.00616719383215198
1914 0.00616719383215198
1915 0.00616719383215198
1916 0.00616719383215198
1917 0.00616719383215198
1918 0.00616719383215198
1919 0.00616719383215198
1920 0.00616719383215198
1921 0.00616719383215198
1922 0.00616719383215198
1923 0.00616719383215198
1924 0.00616719383215198
1925 0.00616719383215198
1926 0.00616719383215198
1927 0.00616719383215198
1928 0.00616719383215198
1929 0.00616719383215198
1930 0.00616719383215198
1931 0.00616719383215198
1932 0.00616719383215198
1933 0.00616719383215198
1934 0.00616719383215198
1935 0.00616719383215198
1936 0.00616719383215198
1937 0.00616719383215198
1938 0.00616719383215198
1939 0.00616719383215198
1940 0.00616719383215198
1941 0.00616719383215198
1942 0.00616719383215198
1943 0.00616719383215198
1944 0.00616719383215198
1945 0.00616719383215198
1946 0.00616719383215198
1947 0.00616719383215198
1948 0.00616719383215198
1949 0.00616719383215198
1950 0.00616719383215198
1951 0.00616719383215198
1952 0.00616719383215198
1953 0.00616719383215198
1954 0.00616719383215198
1955 0.00616719383215198
1956 0.00616719383215198
1957 0.00616719383215198
1958 0.00616719383215198
1959 0.00616719383215198
1960 0.00616719383215198
1961 0.00616719383215198
1962 0.00616719383215198
1963 0.00616719383215198
1964 0.00616719383215198
1965 0.00616719383215198
1966 0.00616719383215198
1967 0.00616719383215198
1968 0.00616719383215198
1969 0.00616719383215198
1970 0.00616719383215198
1971 0.00616719383215198
1972 0.00616719383215198
1973 0.00616719383215198
1974 0.00616719383215198
1975 0.00616719383215198
1976 0.00616719383215198
1977 0.00616719383215198
1978 0.00616719383215198
1979 0.00616719383215198
1980 0.00616719383215198
1981 0.00616719383215198
1982 0.00616719383215198
1983 0.00616719383215198
1984 0.00616719383215198
1985 0.00616719383215198
1986 0.00616719383215198
1987 0.00616719383215198
1988 0.00616719383215198
1989 0.00616719383215198
1990 0.00616719383215198
1991 0.00616719383215198
1992 0.00616719383215198
1993 0.00616719383215198
1994 0.00616719383215198
1995 0.00616719383215198
1996 0.00616719383215198
1997 0.00616719383215198
1998 0.00616719383215198
1999 0.00616719383215198
};
\addlegendentry{Train}
\addplot [semithick, black]
table {%
0 0.0103295426815748
1 0.0103295426815748
2 0.0103295426815748
3 0.0103295426815748
4 0.0103295426815748
5 0.0103295426815748
6 0.0103295426815748
7 0.0103295426815748
8 0.0103295426815748
9 0.0103295426815748
10 0.0103295426815748
11 0.0103295426815748
12 0.0103295426815748
13 0.0103295426815748
14 0.0103295426815748
15 0.0103295426815748
16 0.0103295426815748
17 0.0103295426815748
18 0.0103295426815748
19 0.0103295426815748
20 0.0103295426815748
21 0.0103295426815748
22 0.0103295426815748
23 0.0103295426815748
24 0.0103295426815748
25 0.0103295426815748
26 0.0103295426815748
27 0.0103295426815748
28 0.0103295426815748
29 0.0103295426815748
30 0.0103295426815748
31 0.0103295426815748
32 0.0103295426815748
33 0.0103295426815748
34 0.0103295426815748
35 0.0103295426815748
36 0.0103295426815748
37 0.0103295426815748
38 0.0103295426815748
39 0.0103295426815748
40 0.0103295426815748
41 0.0103295426815748
42 0.0103295426815748
43 0.0103295426815748
44 0.0103295426815748
45 0.0103295426815748
46 0.0103295426815748
47 0.0103295426815748
48 0.0103295426815748
49 0.0103295426815748
50 0.0103295426815748
51 0.0103295426815748
52 0.0103295426815748
53 0.0103295426815748
54 0.0103295426815748
55 0.0103295426815748
56 0.0103295426815748
57 0.0103295426815748
58 0.0103295426815748
59 0.0103295426815748
60 0.0103295426815748
61 0.0103295426815748
62 0.0103295426815748
63 0.0103295426815748
64 0.0103295426815748
65 0.0103295426815748
66 0.0103295426815748
67 0.0103295426815748
68 0.0103295426815748
69 0.0103295426815748
70 0.0103295426815748
71 0.0103295426815748
72 0.0103295426815748
73 0.0103295426815748
74 0.0103295426815748
75 0.0103295426815748
76 0.0103295426815748
77 0.0103295426815748
78 0.0103295426815748
79 0.0103295426815748
80 0.0103295426815748
81 0.0103295426815748
82 0.0103295426815748
83 0.0103295426815748
84 0.0103295426815748
85 0.0103295426815748
86 0.0103295426815748
87 0.0103295426815748
88 0.0103295426815748
89 0.0103295426815748
90 0.0103295426815748
91 0.0103295426815748
92 0.0103295426815748
93 0.0103295426815748
94 0.0103295426815748
95 0.0103295426815748
96 0.0103295426815748
97 0.0103295426815748
98 0.0103295426815748
99 0.0103295426815748
100 0.0103295426815748
101 0.0103295426815748
102 0.0103295426815748
103 0.0103295426815748
104 0.0103295426815748
105 0.0103295426815748
106 0.0103295426815748
107 0.0103295426815748
108 0.0103295426815748
109 0.0103295426815748
110 0.0103295426815748
111 0.0103295426815748
112 0.0103295426815748
113 0.0103295426815748
114 0.0103295426815748
115 0.0103295426815748
116 0.0103295426815748
117 0.0103295426815748
118 0.0103295426815748
119 0.0103295426815748
120 0.0103295426815748
121 0.0103295426815748
122 0.0103295426815748
123 0.0103295426815748
124 0.0103295426815748
125 0.0103295426815748
126 0.0103295426815748
127 0.0103295426815748
128 0.0103295426815748
129 0.0103295426815748
130 0.0103295426815748
131 0.0103295426815748
132 0.0103295426815748
133 0.0103295426815748
134 0.0103295426815748
135 0.0103295426815748
136 0.0103295426815748
137 0.0103295426815748
138 0.0103295426815748
139 0.0103295426815748
140 0.0103295426815748
141 0.0103295426815748
142 0.0103295426815748
143 0.0103295426815748
144 0.0103295426815748
145 0.0103295426815748
146 0.0103295426815748
147 0.0103295426815748
148 0.0103295426815748
149 0.0103295426815748
150 0.0103295426815748
151 0.0103295426815748
152 0.0103295426815748
153 0.0103295426815748
154 0.0103295426815748
155 0.0103295426815748
156 0.0103295426815748
157 0.0103295426815748
158 0.0103295426815748
159 0.0103295426815748
160 0.0103295426815748
161 0.0103295426815748
162 0.0103295426815748
163 0.0103295426815748
164 0.0103295426815748
165 0.0103295426815748
166 0.0103295426815748
167 0.0103295426815748
168 0.0103295426815748
169 0.0103295426815748
170 0.0103295426815748
171 0.0103295426815748
172 0.0103295426815748
173 0.0103295426815748
174 0.0103295426815748
175 0.0103295426815748
176 0.0103295426815748
177 0.0103295426815748
178 0.0103295426815748
179 0.0103295426815748
180 0.0103295426815748
181 0.0103295426815748
182 0.0103295426815748
183 0.0103295426815748
184 0.0103295426815748
185 0.0103295426815748
186 0.0103295426815748
187 0.0103295426815748
188 0.0103295426815748
189 0.0103295426815748
190 0.0103295426815748
191 0.0103295426815748
192 0.0103295426815748
193 0.0103295426815748
194 0.0103295426815748
195 0.0103295426815748
196 0.0103295426815748
197 0.0103295426815748
198 0.0103295426815748
199 0.0103295426815748
200 0.0103295426815748
201 0.0103295426815748
202 0.0103295426815748
203 0.0103295426815748
204 0.0103295426815748
205 0.0103295426815748
206 0.0103295426815748
207 0.0103295426815748
208 0.0103295426815748
209 0.0103295426815748
210 0.0103295426815748
211 0.0103295426815748
212 0.0103295426815748
213 0.0103295426815748
214 0.0103295426815748
215 0.0103295426815748
216 0.0103295426815748
217 0.0103295426815748
218 0.0103295426815748
219 0.0103295426815748
220 0.0103295426815748
221 0.0103295426815748
222 0.0103295426815748
223 0.0103295426815748
224 0.0103295426815748
225 0.0103295426815748
226 0.0103295426815748
227 0.0103295426815748
228 0.0103295426815748
229 0.0103295426815748
230 0.0103295426815748
231 0.0103295426815748
232 0.0103295426815748
233 0.0103295426815748
234 0.0103295426815748
235 0.0103295426815748
236 0.0103295426815748
237 0.0103295426815748
238 0.0103295426815748
239 0.0103295426815748
240 0.0103295426815748
241 0.0103295426815748
242 0.0103295426815748
243 0.0103295426815748
244 0.0103295426815748
245 0.0103295426815748
246 0.0103295426815748
247 0.0103295426815748
248 0.0103295426815748
249 0.0103295426815748
250 0.0103295426815748
251 0.0103295426815748
252 0.0103295426815748
253 0.0103295426815748
254 0.0103295426815748
255 0.0103295426815748
256 0.0103295426815748
257 0.0103295426815748
258 0.0103295426815748
259 0.0103295426815748
260 0.0103295426815748
261 0.0103295426815748
262 0.0103295426815748
263 0.0103295426815748
264 0.0103295426815748
265 0.0103295426815748
266 0.0103295426815748
267 0.0103295426815748
268 0.0103295426815748
269 0.0103295426815748
270 0.0103295426815748
271 0.0103295426815748
272 0.0103295426815748
273 0.0103295426815748
274 0.0103295426815748
275 0.0103295426815748
276 0.0103295426815748
277 0.0103295426815748
278 0.0103295426815748
279 0.0103295426815748
280 0.0103295426815748
281 0.0103295426815748
282 0.0103295426815748
283 0.0103295426815748
284 0.0103295426815748
285 0.0103295426815748
286 0.0103295426815748
287 0.0103295426815748
288 0.0103295426815748
289 0.0103295426815748
290 0.0103295426815748
291 0.0103295426815748
292 0.0103295426815748
293 0.0103295426815748
294 0.0103295426815748
295 0.0103295426815748
296 0.0103295426815748
297 0.0103295426815748
298 0.0103295426815748
299 0.0103295426815748
300 0.0103295426815748
301 0.0103295426815748
302 0.0103295426815748
303 0.0103295426815748
304 0.0103295426815748
305 0.0103295426815748
306 0.0103295426815748
307 0.0103295426815748
308 0.0103295426815748
309 0.0103295426815748
310 0.0103295426815748
311 0.0103295426815748
312 0.0103295426815748
313 0.0103295426815748
314 0.0103295426815748
315 0.0103295426815748
316 0.0103295426815748
317 0.0103295426815748
318 0.0103295426815748
319 0.0103295426815748
320 0.0103295426815748
321 0.0103295426815748
322 0.0103295426815748
323 0.0103295426815748
324 0.0103295426815748
325 0.0103295426815748
326 0.0103295426815748
327 0.0103295426815748
328 0.0103295426815748
329 0.0103295426815748
330 0.0103295426815748
331 0.0103295426815748
332 0.0103295426815748
333 0.0103295426815748
334 0.0103295426815748
335 0.0103295426815748
336 0.0103295426815748
337 0.0103295426815748
338 0.0103295426815748
339 0.0103295426815748
340 0.0103295426815748
341 0.0103295426815748
342 0.0103295426815748
343 0.0103295426815748
344 0.0103295426815748
345 0.0103295426815748
346 0.0103295426815748
347 0.0103295426815748
348 0.0103295426815748
349 0.0103295426815748
350 0.0103295426815748
351 0.0103295426815748
352 0.0103295426815748
353 0.0103295426815748
354 0.0103295426815748
355 0.0103295426815748
356 0.0103295426815748
357 0.0103295426815748
358 0.0103295426815748
359 0.0103295426815748
360 0.0103295426815748
361 0.0103295426815748
362 0.0103295426815748
363 0.0103295426815748
364 0.0103295426815748
365 0.0103295426815748
366 0.0103295426815748
367 0.0103295426815748
368 0.0103295426815748
369 0.0103295426815748
370 0.0103295426815748
371 0.0103295426815748
372 0.0103295426815748
373 0.0103295426815748
374 0.0103295426815748
375 0.0103295426815748
376 0.0103295426815748
377 0.0103295426815748
378 0.0103295426815748
379 0.0103295426815748
380 0.0103295426815748
381 0.0103295426815748
382 0.0103295426815748
383 0.0103295426815748
384 0.0103295426815748
385 0.0103295426815748
386 0.0103295426815748
387 0.0103295426815748
388 0.0103295426815748
389 0.0103295426815748
390 0.0103295426815748
391 0.0103295426815748
392 0.0103295426815748
393 0.0103295426815748
394 0.0103295426815748
395 0.0103295426815748
396 0.0103295426815748
397 0.0103295426815748
398 0.0103295426815748
399 0.0103295426815748
400 0.0103295426815748
401 0.0103295426815748
402 0.0103295426815748
403 0.0103295426815748
404 0.0103295426815748
405 0.0103295426815748
406 0.0103295426815748
407 0.0103295426815748
408 0.0103295426815748
409 0.0103295426815748
410 0.0103295426815748
411 0.0103295426815748
412 0.0103295426815748
413 0.0103295426815748
414 0.0103295426815748
415 0.0103295426815748
416 0.0103295426815748
417 0.0103295426815748
418 0.0103295426815748
419 0.0103295426815748
420 0.0103295426815748
421 0.0103295426815748
422 0.0103295426815748
423 0.0103295426815748
424 0.0103295426815748
425 0.0103295426815748
426 0.0103295426815748
427 0.0103295426815748
428 0.0103295426815748
429 0.0103295426815748
430 0.0103295426815748
431 0.0103295426815748
432 0.0103295426815748
433 0.0103295426815748
434 0.0103295426815748
435 0.0103295426815748
436 0.0103295426815748
437 0.0103295426815748
438 0.0103295426815748
439 0.0103295426815748
440 0.0103295426815748
441 0.0103295426815748
442 0.0103295426815748
443 0.0103295426815748
444 0.0103295426815748
445 0.0103295426815748
446 0.0103295426815748
447 0.0103295426815748
448 0.0103295426815748
449 0.0103295426815748
450 0.0103295426815748
451 0.0103295426815748
452 0.0103295426815748
453 0.0103295426815748
454 0.0103295426815748
455 0.0103295426815748
456 0.0103295426815748
457 0.0103295426815748
458 0.0103295426815748
459 0.0103295426815748
460 0.0103295426815748
461 0.0103295426815748
462 0.0103295426815748
463 0.0103295426815748
464 0.0103295426815748
465 0.0103295426815748
466 0.0103295426815748
467 0.0103295426815748
468 0.0103295426815748
469 0.0103295426815748
470 0.0103295426815748
471 0.0103295426815748
472 0.0103295426815748
473 0.0103295426815748
474 0.0103295426815748
475 0.0103295426815748
476 0.0103295426815748
477 0.0103295426815748
478 0.0103295426815748
479 0.0103295426815748
480 0.0103295426815748
481 0.0103295426815748
482 0.0103295426815748
483 0.0103295426815748
484 0.0103295426815748
485 0.0103295426815748
486 0.0103295426815748
487 0.0103295426815748
488 0.0103295426815748
489 0.0103295426815748
490 0.0103295426815748
491 0.0103295426815748
492 0.0103295426815748
493 0.0103295426815748
494 0.0103295426815748
495 0.0103295426815748
496 0.0103295426815748
497 0.0103295426815748
498 0.0103295426815748
499 0.0103295426815748
500 0.0103295426815748
501 0.0103295426815748
502 0.0103295426815748
503 0.0103295426815748
504 0.0103295426815748
505 0.0103295426815748
506 0.0103295426815748
507 0.0103295426815748
508 0.0103295426815748
509 0.0103295426815748
510 0.0103295426815748
511 0.0103295426815748
512 0.0103295426815748
513 0.0103295426815748
514 0.0103295426815748
515 0.0103295426815748
516 0.0103295426815748
517 0.0103295426815748
518 0.0103295426815748
519 0.0103295426815748
520 0.0103295426815748
521 0.0103295426815748
522 0.0103295426815748
523 0.0103295426815748
524 0.0103295426815748
525 0.0103295426815748
526 0.0103295426815748
527 0.0103295426815748
528 0.0103295426815748
529 0.0103295426815748
530 0.0103295426815748
531 0.0103295426815748
532 0.0103295426815748
533 0.0103295426815748
534 0.0103295426815748
535 0.0103295426815748
536 0.0103295426815748
537 0.0103295426815748
538 0.0103295426815748
539 0.0103295426815748
540 0.0103295426815748
541 0.0103295426815748
542 0.0103295426815748
543 0.0103295426815748
544 0.0103295426815748
545 0.0103295426815748
546 0.0103295426815748
547 0.0103295426815748
548 0.0103295426815748
549 0.0103295426815748
550 0.0103295426815748
551 0.0103295426815748
552 0.0103295426815748
553 0.0103295426815748
554 0.0103295426815748
555 0.0103295426815748
556 0.0103295426815748
557 0.0103295426815748
558 0.0103295426815748
559 0.0103295426815748
560 0.0103295426815748
561 0.0103295426815748
562 0.0103295426815748
563 0.0103295426815748
564 0.0103295426815748
565 0.0103295426815748
566 0.0103295426815748
567 0.0103295426815748
568 0.0103295426815748
569 0.0103295426815748
570 0.0103295426815748
571 0.0103295426815748
572 0.0103295426815748
573 0.0103295426815748
574 0.0103295426815748
575 0.0103295426815748
576 0.0103295426815748
577 0.0103295426815748
578 0.0103295426815748
579 0.0103295426815748
580 0.0103295426815748
581 0.0103295426815748
582 0.0103295426815748
583 0.0103295426815748
584 0.0103295426815748
585 0.0103295426815748
586 0.0103295426815748
587 0.0103295426815748
588 0.0103295426815748
589 0.0103295426815748
590 0.0103295426815748
591 0.0103295426815748
592 0.0103295426815748
593 0.0103295426815748
594 0.0103295426815748
595 0.0103295426815748
596 0.0103295426815748
597 0.0103295426815748
598 0.0103295426815748
599 0.0103295426815748
600 0.0103295426815748
601 0.0103295426815748
602 0.0103295426815748
603 0.0103295426815748
604 0.0103295426815748
605 0.0103295426815748
606 0.0103295426815748
607 0.0103295426815748
608 0.0103295426815748
609 0.0103295426815748
610 0.0103295426815748
611 0.0103295426815748
612 0.0103295426815748
613 0.0103295426815748
614 0.0103295426815748
615 0.0103295426815748
616 0.0103295426815748
617 0.0103295426815748
618 0.0103295426815748
619 0.0103295426815748
620 0.0103295426815748
621 0.0103295426815748
622 0.0103295426815748
623 0.0103295426815748
624 0.0103295426815748
625 0.0103295426815748
626 0.0103295426815748
627 0.0103295426815748
628 0.0103295426815748
629 0.0103295426815748
630 0.0103295426815748
631 0.0103295426815748
632 0.0103295426815748
633 0.0103295426815748
634 0.0103295426815748
635 0.0103295426815748
636 0.0103295426815748
637 0.0103295426815748
638 0.0103295426815748
639 0.0103295426815748
640 0.0103295426815748
641 0.0103295426815748
642 0.0103295426815748
643 0.0103295426815748
644 0.0103295426815748
645 0.0103295426815748
646 0.0103295426815748
647 0.0103295426815748
648 0.0103295426815748
649 0.0103295426815748
650 0.0103295426815748
651 0.0103295426815748
652 0.0103295426815748
653 0.0103295426815748
654 0.0103295426815748
655 0.0103295426815748
656 0.0103295426815748
657 0.0103295426815748
658 0.0103295426815748
659 0.0103295426815748
660 0.0103295426815748
661 0.0103295426815748
662 0.0103295426815748
663 0.0103295426815748
664 0.0103295426815748
665 0.0103295426815748
666 0.0103295426815748
667 0.0103295426815748
668 0.0103295426815748
669 0.0103295426815748
670 0.0103295426815748
671 0.0103295426815748
672 0.0103295426815748
673 0.0103295426815748
674 0.0103295426815748
675 0.0103295426815748
676 0.0103295426815748
677 0.0103295426815748
678 0.0103295426815748
679 0.0103295426815748
680 0.0103295426815748
681 0.0103295426815748
682 0.0103295426815748
683 0.0103295426815748
684 0.0103295426815748
685 0.0103295426815748
686 0.0103295426815748
687 0.0103295426815748
688 0.0103295426815748
689 0.0103295426815748
690 0.0103295426815748
691 0.0103295426815748
692 0.0103295426815748
693 0.0103295426815748
694 0.0103295426815748
695 0.0103295426815748
696 0.0103295426815748
697 0.0103295426815748
698 0.0103295426815748
699 0.0103295426815748
700 0.0103295426815748
701 0.0103295426815748
702 0.0103295426815748
703 0.0103295426815748
704 0.0103295426815748
705 0.0103295426815748
706 0.0103295426815748
707 0.0103295426815748
708 0.0103295426815748
709 0.0103295426815748
710 0.0103295426815748
711 0.0103295426815748
712 0.0103295426815748
713 0.0103295426815748
714 0.0103295426815748
715 0.0103295426815748
716 0.0103295426815748
717 0.0103295426815748
718 0.0103295426815748
719 0.0103295426815748
720 0.0103295426815748
721 0.0103295426815748
722 0.0103295426815748
723 0.0103295426815748
724 0.0103295426815748
725 0.0103295426815748
726 0.0103295426815748
727 0.0103295426815748
728 0.0103295426815748
729 0.0103295426815748
730 0.0103295426815748
731 0.0103295426815748
732 0.0103295426815748
733 0.0103295426815748
734 0.0103295426815748
735 0.0103295426815748
736 0.0103295426815748
737 0.0103295426815748
738 0.0103295426815748
739 0.0103295426815748
740 0.0103295426815748
741 0.0103295426815748
742 0.0103295426815748
743 0.0103295426815748
744 0.0103295426815748
745 0.0103295426815748
746 0.0103295426815748
747 0.0103295426815748
748 0.0103295426815748
749 0.0103295426815748
750 0.0103295426815748
751 0.0103295426815748
752 0.0103295426815748
753 0.0103295426815748
754 0.0103295426815748
755 0.0103295426815748
756 0.0103295426815748
757 0.0103295426815748
758 0.0103295426815748
759 0.0103295426815748
760 0.0103295426815748
761 0.0103295426815748
762 0.0103295426815748
763 0.0103295426815748
764 0.0103295426815748
765 0.0103295426815748
766 0.0103295426815748
767 0.0103295426815748
768 0.0103295426815748
769 0.0103295426815748
770 0.0103295426815748
771 0.0103295426815748
772 0.0103295426815748
773 0.0103295426815748
774 0.0103295426815748
775 0.0103295426815748
776 0.0103295426815748
777 0.0103295426815748
778 0.0103295426815748
779 0.0103295426815748
780 0.0103295426815748
781 0.0103295426815748
782 0.0103295426815748
783 0.0103295426815748
784 0.0103295426815748
785 0.0103295426815748
786 0.0103295426815748
787 0.0103295426815748
788 0.0103295426815748
789 0.0103295426815748
790 0.0103295426815748
791 0.0103295426815748
792 0.0103295426815748
793 0.0103295426815748
794 0.0103295426815748
795 0.0103295426815748
796 0.0103295426815748
797 0.0103295426815748
798 0.0103295426815748
799 0.0103295426815748
800 0.0103295426815748
801 0.0103295426815748
802 0.0103295426815748
803 0.0103295426815748
804 0.0103295426815748
805 0.0103295426815748
806 0.0103295426815748
807 0.0103295426815748
808 0.0103295426815748
809 0.0103295426815748
810 0.0103295426815748
811 0.0103295426815748
812 0.0103295426815748
813 0.0103295426815748
814 0.0103295426815748
815 0.0103295426815748
816 0.0103295426815748
817 0.0103295426815748
818 0.0103295426815748
819 0.0103295426815748
820 0.0103295426815748
821 0.0103295426815748
822 0.0103295426815748
823 0.0103295426815748
824 0.0103295426815748
825 0.0103295426815748
826 0.0103295426815748
827 0.0103295426815748
828 0.0103295426815748
829 0.0103295426815748
830 0.0103295426815748
831 0.0103295426815748
832 0.0103295426815748
833 0.0103295426815748
834 0.0103295426815748
835 0.0103295426815748
836 0.0103295426815748
837 0.0103295426815748
838 0.0103295426815748
839 0.0103295426815748
840 0.0103295426815748
841 0.0103295426815748
842 0.0103295426815748
843 0.0103295426815748
844 0.0103295426815748
845 0.0103295426815748
846 0.0103295426815748
847 0.0103295426815748
848 0.0103295426815748
849 0.0103295426815748
850 0.0103295426815748
851 0.0103295426815748
852 0.0103295426815748
853 0.0103295426815748
854 0.0103295426815748
855 0.0103295426815748
856 0.0103295426815748
857 0.0103295426815748
858 0.0103295426815748
859 0.0103295426815748
860 0.0103295426815748
861 0.0103295426815748
862 0.0103295426815748
863 0.0103295426815748
864 0.0103295426815748
865 0.0103295426815748
866 0.0103295426815748
867 0.0103295426815748
868 0.0103295426815748
869 0.0103295426815748
870 0.0103295426815748
871 0.0103295426815748
872 0.0103295426815748
873 0.0103295426815748
874 0.0103295426815748
875 0.0103295426815748
876 0.0103295426815748
877 0.0103295426815748
878 0.0103295426815748
879 0.0103295426815748
880 0.0103295426815748
881 0.0103295426815748
882 0.0103295426815748
883 0.0103295426815748
884 0.0103295426815748
885 0.0103295426815748
886 0.0103295426815748
887 0.0103295426815748
888 0.0103295426815748
889 0.0103295426815748
890 0.0103295426815748
891 0.0103295426815748
892 0.0103295426815748
893 0.0103295426815748
894 0.0103295426815748
895 0.0103295426815748
896 0.0103295426815748
897 0.0103295426815748
898 0.0103295426815748
899 0.0103295426815748
900 0.0103295426815748
901 0.0103295426815748
902 0.0103295426815748
903 0.0103295426815748
904 0.0103295426815748
905 0.0103295426815748
906 0.0103295426815748
907 0.0103295426815748
908 0.0103295426815748
909 0.0103295426815748
910 0.0103295426815748
911 0.0103295426815748
912 0.0103295426815748
913 0.0103295426815748
914 0.0103295426815748
915 0.0103295426815748
916 0.0103295426815748
917 0.0103295426815748
918 0.0103295426815748
919 0.0103295426815748
920 0.0103295426815748
921 0.0103295426815748
922 0.0103295426815748
923 0.0103295426815748
924 0.0103295426815748
925 0.0103295426815748
926 0.0103295426815748
927 0.0103295426815748
928 0.0103295426815748
929 0.0103295426815748
930 0.0103295426815748
931 0.0103295426815748
932 0.0103295426815748
933 0.0103295426815748
934 0.0103295426815748
935 0.0103295426815748
936 0.0103295426815748
937 0.0103295426815748
938 0.0103295426815748
939 0.0103295426815748
940 0.0103295426815748
941 0.0103295426815748
942 0.0103295426815748
943 0.0103295426815748
944 0.0103295426815748
945 0.0103295426815748
946 0.0103295426815748
947 0.0103295426815748
948 0.0103295426815748
949 0.0103295426815748
950 0.0103295426815748
951 0.0103295426815748
952 0.0103295426815748
953 0.0103295426815748
954 0.0103295426815748
955 0.0103295426815748
956 0.0103295426815748
957 0.0103295426815748
958 0.0103295426815748
959 0.0103295426815748
960 0.0103295426815748
961 0.0103295426815748
962 0.0103295426815748
963 0.0103295426815748
964 0.0103295426815748
965 0.0103295426815748
966 0.0103295426815748
967 0.0103295426815748
968 0.0103295426815748
969 0.0103295426815748
970 0.0103295426815748
971 0.0103295426815748
972 0.0103295426815748
973 0.0103295426815748
974 0.0103295426815748
975 0.0103295426815748
976 0.0103295426815748
977 0.0103295426815748
978 0.0103295426815748
979 0.0103295426815748
980 0.0103295426815748
981 0.0103295426815748
982 0.0103295426815748
983 0.0103295426815748
984 0.0103295426815748
985 0.0103295426815748
986 0.0103295426815748
987 0.0103295426815748
988 0.0103295426815748
989 0.0103295426815748
990 0.0103295426815748
991 0.0103295426815748
992 0.0103295426815748
993 0.0103295426815748
994 0.0103295426815748
995 0.0103295426815748
996 0.0103295426815748
997 0.0103295426815748
998 0.0103295426815748
999 0.0103295426815748
1000 0.0103295426815748
1001 0.0103295426815748
1002 0.0103295426815748
1003 0.0103295426815748
1004 0.0103295426815748
1005 0.0103295426815748
1006 0.0103295426815748
1007 0.0103295426815748
1008 0.0103295426815748
1009 0.0103295426815748
1010 0.0103295426815748
1011 0.0103295426815748
1012 0.0103295426815748
1013 0.0103295426815748
1014 0.0103295426815748
1015 0.0103295426815748
1016 0.0103295426815748
1017 0.0103295426815748
1018 0.0103295426815748
1019 0.0103295426815748
1020 0.0103295426815748
1021 0.0103295426815748
1022 0.0103295426815748
1023 0.0103295426815748
1024 0.0103295426815748
1025 0.0103295426815748
1026 0.0103295426815748
1027 0.0103295426815748
1028 0.0103295426815748
1029 0.0103295426815748
1030 0.0103295426815748
1031 0.0103295426815748
1032 0.0103295426815748
1033 0.0103295426815748
1034 0.0103295426815748
1035 0.0103295426815748
1036 0.0103295426815748
1037 0.0103295426815748
1038 0.0103295426815748
1039 0.0103295426815748
1040 0.0103295426815748
1041 0.0103295426815748
1042 0.0103295426815748
1043 0.0103295426815748
1044 0.0103295426815748
1045 0.0103295426815748
1046 0.0103295426815748
1047 0.0103295426815748
1048 0.0103295426815748
1049 0.0103295426815748
1050 0.0103295426815748
1051 0.0103295426815748
1052 0.0103295426815748
1053 0.0103295426815748
1054 0.0103295426815748
1055 0.0103295426815748
1056 0.0103295426815748
1057 0.0103295426815748
1058 0.0103295426815748
1059 0.0103295426815748
1060 0.0103295426815748
1061 0.0103295426815748
1062 0.0103295426815748
1063 0.0103295426815748
1064 0.0103295426815748
1065 0.0103295426815748
1066 0.0103295426815748
1067 0.0103295426815748
1068 0.0103295426815748
1069 0.0103295426815748
1070 0.0103295426815748
1071 0.0103295426815748
1072 0.0103295426815748
1073 0.0103295426815748
1074 0.0103295426815748
1075 0.0103295426815748
1076 0.0103295426815748
1077 0.0103295426815748
1078 0.0103295426815748
1079 0.0103295426815748
1080 0.0103295426815748
1081 0.0103295426815748
1082 0.0103295426815748
1083 0.0103295426815748
1084 0.0103295426815748
1085 0.0103295426815748
1086 0.0103295426815748
1087 0.0103295426815748
1088 0.0103295426815748
1089 0.0103295426815748
1090 0.0103295426815748
1091 0.0103295426815748
1092 0.0103295426815748
1093 0.0103295426815748
1094 0.0103295426815748
1095 0.0103295426815748
1096 0.0103295426815748
1097 0.0103295426815748
1098 0.0103295426815748
1099 0.0103295426815748
1100 0.0103295426815748
1101 0.0103295426815748
1102 0.0103295426815748
1103 0.0103295426815748
1104 0.0103295426815748
1105 0.0103295426815748
1106 0.0103295426815748
1107 0.0103295426815748
1108 0.0103295426815748
1109 0.0103295426815748
1110 0.0103295426815748
1111 0.0103295426815748
1112 0.0103295426815748
1113 0.0103295426815748
1114 0.0103295426815748
1115 0.0103295426815748
1116 0.0103295426815748
1117 0.0103295426815748
1118 0.0103295426815748
1119 0.0103295426815748
1120 0.0103295426815748
1121 0.0103295426815748
1122 0.0103295426815748
1123 0.0103295426815748
1124 0.0103295426815748
1125 0.0103295426815748
1126 0.0103295426815748
1127 0.0103295426815748
1128 0.0103295426815748
1129 0.0103295426815748
1130 0.0103295426815748
1131 0.0103295426815748
1132 0.0103295426815748
1133 0.0103295426815748
1134 0.0103295426815748
1135 0.0103295426815748
1136 0.0103295426815748
1137 0.0103295426815748
1138 0.0103295426815748
1139 0.0103295426815748
1140 0.0103295426815748
1141 0.0103295426815748
1142 0.0103295426815748
1143 0.0103295426815748
1144 0.0103295426815748
1145 0.0103295426815748
1146 0.0103295426815748
1147 0.0103295426815748
1148 0.0103295426815748
1149 0.0103295426815748
1150 0.0103295426815748
1151 0.0103295426815748
1152 0.0103295426815748
1153 0.0103295426815748
1154 0.0103295426815748
1155 0.0103295426815748
1156 0.0103295426815748
1157 0.0103295426815748
1158 0.0103295426815748
1159 0.0103295426815748
1160 0.0103295426815748
1161 0.0103295426815748
1162 0.0103295426815748
1163 0.0103295426815748
1164 0.0103295426815748
1165 0.0103295426815748
1166 0.0103295426815748
1167 0.0103295426815748
1168 0.0103295426815748
1169 0.0103295426815748
1170 0.0103295426815748
1171 0.0103295426815748
1172 0.0103295426815748
1173 0.0103295426815748
1174 0.0103295426815748
1175 0.0103295426815748
1176 0.0103295426815748
1177 0.0103295426815748
1178 0.0103295426815748
1179 0.0103295426815748
1180 0.0103295426815748
1181 0.0103295426815748
1182 0.0103295426815748
1183 0.0103295426815748
1184 0.0103295426815748
1185 0.0103295426815748
1186 0.0103295426815748
1187 0.0103295426815748
1188 0.0103295426815748
1189 0.0103295426815748
1190 0.0103295426815748
1191 0.0103295426815748
1192 0.0103295426815748
1193 0.0103295426815748
1194 0.0103295426815748
1195 0.0103295426815748
1196 0.0103295426815748
1197 0.0103295426815748
1198 0.0103295426815748
1199 0.0103295426815748
1200 0.0103295426815748
1201 0.0103295426815748
1202 0.0103295426815748
1203 0.0103295426815748
1204 0.0103295426815748
1205 0.0103295426815748
1206 0.0103295426815748
1207 0.0103295426815748
1208 0.0103295426815748
1209 0.0103295426815748
1210 0.0103295426815748
1211 0.0103295426815748
1212 0.0103295426815748
1213 0.0103295426815748
1214 0.0103295426815748
1215 0.0103295426815748
1216 0.0103295426815748
1217 0.0103295426815748
1218 0.0103295426815748
1219 0.0103295426815748
1220 0.0103295426815748
1221 0.0103295426815748
1222 0.0103295426815748
1223 0.0103295426815748
1224 0.0103295426815748
1225 0.0103295426815748
1226 0.0103295426815748
1227 0.0103295426815748
1228 0.0103295426815748
1229 0.0103295426815748
1230 0.0103295426815748
1231 0.0103295426815748
1232 0.0103295426815748
1233 0.0103295426815748
1234 0.0103295426815748
1235 0.0103295426815748
1236 0.0103295426815748
1237 0.0103295426815748
1238 0.0103295426815748
1239 0.0103295426815748
1240 0.0103295426815748
1241 0.0103295426815748
1242 0.0103295426815748
1243 0.0103295426815748
1244 0.0103295426815748
1245 0.0103295426815748
1246 0.0103295426815748
1247 0.0103295426815748
1248 0.0103295426815748
1249 0.0103295426815748
1250 0.0103295426815748
1251 0.0103295426815748
1252 0.0103295426815748
1253 0.0103295426815748
1254 0.0103295426815748
1255 0.0103295426815748
1256 0.0103295426815748
1257 0.0103295426815748
1258 0.0103295426815748
1259 0.0103295426815748
1260 0.0103295426815748
1261 0.0103295426815748
1262 0.0103295426815748
1263 0.0103295426815748
1264 0.0103295426815748
1265 0.0103295426815748
1266 0.0103295426815748
1267 0.0103295426815748
1268 0.0103295426815748
1269 0.0103295426815748
1270 0.0103295426815748
1271 0.0103295426815748
1272 0.0103295426815748
1273 0.0103295426815748
1274 0.0103295426815748
1275 0.0103295426815748
1276 0.0103295426815748
1277 0.0103295426815748
1278 0.0103295426815748
1279 0.0103295426815748
1280 0.0103295426815748
1281 0.0103295426815748
1282 0.0103295426815748
1283 0.0103295426815748
1284 0.0103295426815748
1285 0.0103295426815748
1286 0.0103295426815748
1287 0.0103295426815748
1288 0.0103295426815748
1289 0.0103295426815748
1290 0.0103295426815748
1291 0.0103295426815748
1292 0.0103295426815748
1293 0.0103295426815748
1294 0.0103295426815748
1295 0.0103295426815748
1296 0.0103295426815748
1297 0.0103295426815748
1298 0.0103295426815748
1299 0.0103295426815748
1300 0.0103295426815748
1301 0.0103295426815748
1302 0.0103295426815748
1303 0.0103295426815748
1304 0.0103295426815748
1305 0.0103295426815748
1306 0.0103295426815748
1307 0.0103295426815748
1308 0.0103295426815748
1309 0.0103295426815748
1310 0.0103295426815748
1311 0.0103295426815748
1312 0.0103295426815748
1313 0.0103295426815748
1314 0.0103295426815748
1315 0.0103295426815748
1316 0.0103295426815748
1317 0.0103295426815748
1318 0.0103295426815748
1319 0.0103295426815748
1320 0.0103295426815748
1321 0.0103295426815748
1322 0.0103295426815748
1323 0.0103295426815748
1324 0.0103295426815748
1325 0.0103295426815748
1326 0.0103295426815748
1327 0.0103295426815748
1328 0.0103295426815748
1329 0.0103295426815748
1330 0.0103295426815748
1331 0.0103295426815748
1332 0.0103295426815748
1333 0.0103295426815748
1334 0.0103295426815748
1335 0.0103295426815748
1336 0.0103295426815748
1337 0.0103295426815748
1338 0.0103295426815748
1339 0.0103295426815748
1340 0.0103295426815748
1341 0.0103295426815748
1342 0.0103295426815748
1343 0.0103295426815748
1344 0.0103295426815748
1345 0.0103295426815748
1346 0.0103295426815748
1347 0.0103295426815748
1348 0.0103295426815748
1349 0.0103295426815748
1350 0.0103295426815748
1351 0.0103295426815748
1352 0.0103295426815748
1353 0.0103295426815748
1354 0.0103295426815748
1355 0.0103295426815748
1356 0.0103295426815748
1357 0.0103295426815748
1358 0.0103295426815748
1359 0.0103295426815748
1360 0.0103295426815748
1361 0.0103295426815748
1362 0.0103295426815748
1363 0.0103295426815748
1364 0.0103295426815748
1365 0.0103295426815748
1366 0.0103295426815748
1367 0.0103295426815748
1368 0.0103295426815748
1369 0.0103295426815748
1370 0.0103295426815748
1371 0.0103295426815748
1372 0.0103295426815748
1373 0.0103295426815748
1374 0.0103295426815748
1375 0.0103295426815748
1376 0.0103295426815748
1377 0.0103295426815748
1378 0.0103295426815748
1379 0.0103295426815748
1380 0.0103295426815748
1381 0.0103295426815748
1382 0.0103295426815748
1383 0.0103295426815748
1384 0.0103295426815748
1385 0.0103295426815748
1386 0.0103295426815748
1387 0.0103295426815748
1388 0.0103295426815748
1389 0.0103295426815748
1390 0.0103295426815748
1391 0.0103295426815748
1392 0.0103295426815748
1393 0.0103295426815748
1394 0.0103295426815748
1395 0.0103295426815748
1396 0.0103295426815748
1397 0.0103295426815748
1398 0.0103295426815748
1399 0.0103295426815748
1400 0.0103295426815748
1401 0.0103295426815748
1402 0.0103295426815748
1403 0.0103295426815748
1404 0.0103295426815748
1405 0.0103295426815748
1406 0.0103295426815748
1407 0.0103295426815748
1408 0.0103295426815748
1409 0.0103295426815748
1410 0.0103295426815748
1411 0.0103295426815748
1412 0.0103295426815748
1413 0.0103295426815748
1414 0.0103295426815748
1415 0.0103295426815748
1416 0.0103295426815748
1417 0.0103295426815748
1418 0.0103295426815748
1419 0.0103295426815748
1420 0.0103295426815748
1421 0.0103295426815748
1422 0.0103295426815748
1423 0.0103295426815748
1424 0.0103295426815748
1425 0.0103295426815748
1426 0.0103295426815748
1427 0.0103295426815748
1428 0.0103295426815748
1429 0.0103295426815748
1430 0.0103295426815748
1431 0.0103295426815748
1432 0.0103295426815748
1433 0.0103295426815748
1434 0.0103295426815748
1435 0.0103295426815748
1436 0.0103295426815748
1437 0.0103295426815748
1438 0.0103295426815748
1439 0.0103295426815748
1440 0.0103295426815748
1441 0.0103295426815748
1442 0.0103295426815748
1443 0.0103295426815748
1444 0.0103295426815748
1445 0.0103295426815748
1446 0.0103295426815748
1447 0.0103295426815748
1448 0.0103295426815748
1449 0.0103295426815748
1450 0.0103295426815748
1451 0.0103295426815748
1452 0.0103295426815748
1453 0.0103295426815748
1454 0.0103295426815748
1455 0.0103295426815748
1456 0.0103295426815748
1457 0.0103295426815748
1458 0.0103295426815748
1459 0.0103295426815748
1460 0.0103295426815748
1461 0.0103295426815748
1462 0.0103295426815748
1463 0.0103295426815748
1464 0.0103295426815748
1465 0.0103295426815748
1466 0.0103295426815748
1467 0.0103295426815748
1468 0.0103295426815748
1469 0.0103295426815748
1470 0.0103295426815748
1471 0.0103295426815748
1472 0.0103295426815748
1473 0.0103295426815748
1474 0.0103295426815748
1475 0.0103295426815748
1476 0.0103295426815748
1477 0.0103295426815748
1478 0.0103295426815748
1479 0.0103295426815748
1480 0.0103295426815748
1481 0.0103295426815748
1482 0.0103295426815748
1483 0.0103295426815748
1484 0.0103295426815748
1485 0.0103295426815748
1486 0.0103295426815748
1487 0.0103295426815748
1488 0.0103295426815748
1489 0.0103295426815748
1490 0.0103295426815748
1491 0.0103295426815748
1492 0.0103295426815748
1493 0.0103295426815748
1494 0.0103295426815748
1495 0.0103295426815748
1496 0.0103295426815748
1497 0.0103295426815748
1498 0.0103295426815748
1499 0.0103295426815748
1500 0.0103295426815748
1501 0.0103295426815748
1502 0.0103295426815748
1503 0.0103295426815748
1504 0.0103295426815748
1505 0.0103295426815748
1506 0.0103295426815748
1507 0.0103295426815748
1508 0.0103295426815748
1509 0.0103295426815748
1510 0.0103295426815748
1511 0.0103295426815748
1512 0.0103295426815748
1513 0.0103295426815748
1514 0.0103295426815748
1515 0.0103295426815748
1516 0.0103295426815748
1517 0.0103295426815748
1518 0.0103295426815748
1519 0.0103295426815748
1520 0.0103295426815748
1521 0.0103295426815748
1522 0.0103295426815748
1523 0.0103295426815748
1524 0.0103295426815748
1525 0.0103295426815748
1526 0.0103295426815748
1527 0.0103295426815748
1528 0.0103295426815748
1529 0.0103295426815748
1530 0.0103295426815748
1531 0.0103295426815748
1532 0.0103295426815748
1533 0.0103295426815748
1534 0.0103295426815748
1535 0.0103295426815748
1536 0.0103295426815748
1537 0.0103295426815748
1538 0.0103295426815748
1539 0.0103295426815748
1540 0.0103295426815748
1541 0.0103295426815748
1542 0.0103295426815748
1543 0.0103295426815748
1544 0.0103295426815748
1545 0.0103295426815748
1546 0.0103295426815748
1547 0.0103295426815748
1548 0.0103295426815748
1549 0.0103295426815748
1550 0.0103295426815748
1551 0.0103295426815748
1552 0.0103295426815748
1553 0.0103295426815748
1554 0.0103295426815748
1555 0.0103295426815748
1556 0.0103295426815748
1557 0.0103295426815748
1558 0.0103295426815748
1559 0.0103295426815748
1560 0.0103295426815748
1561 0.0103295426815748
1562 0.0103295426815748
1563 0.0103295426815748
1564 0.0103295426815748
1565 0.0103295426815748
1566 0.0103295426815748
1567 0.0103295426815748
1568 0.0103295426815748
1569 0.0103295426815748
1570 0.0103295426815748
1571 0.0103295426815748
1572 0.0103295426815748
1573 0.0103295426815748
1574 0.0103295426815748
1575 0.0103295426815748
1576 0.0103295426815748
1577 0.0103295426815748
1578 0.0103295426815748
1579 0.0103295426815748
1580 0.0103295426815748
1581 0.0103295426815748
1582 0.0103295426815748
1583 0.0103295426815748
1584 0.0103295426815748
1585 0.0103295426815748
1586 0.0103295426815748
1587 0.0103295426815748
1588 0.0103295426815748
1589 0.0103295426815748
1590 0.0103295426815748
1591 0.0103295426815748
1592 0.0103295426815748
1593 0.0103295426815748
1594 0.0103295426815748
1595 0.0103295426815748
1596 0.0103295426815748
1597 0.0103295426815748
1598 0.0103295426815748
1599 0.0103295426815748
1600 0.0103295426815748
1601 0.0103295426815748
1602 0.0103295426815748
1603 0.0103295426815748
1604 0.0103295426815748
1605 0.0103295426815748
1606 0.0103295426815748
1607 0.0103295426815748
1608 0.0103295426815748
1609 0.0103295426815748
1610 0.0103295426815748
1611 0.0103295426815748
1612 0.0103295426815748
1613 0.0103295426815748
1614 0.0103295426815748
1615 0.0103295426815748
1616 0.0103295426815748
1617 0.0103295426815748
1618 0.0103295426815748
1619 0.0103295426815748
1620 0.0103295426815748
1621 0.0103295426815748
1622 0.0103295426815748
1623 0.0103295426815748
1624 0.0103295426815748
1625 0.0103295426815748
1626 0.0103295426815748
1627 0.0103295426815748
1628 0.0103295426815748
1629 0.0103295426815748
1630 0.0103295426815748
1631 0.0103295426815748
1632 0.0103295426815748
1633 0.0103295426815748
1634 0.0103295426815748
1635 0.0103295426815748
1636 0.0103295426815748
1637 0.0103295426815748
1638 0.0103295426815748
1639 0.0103295426815748
1640 0.0103295426815748
1641 0.0103295426815748
1642 0.0103295426815748
1643 0.0103295426815748
1644 0.0103295426815748
1645 0.0103295426815748
1646 0.0103295426815748
1647 0.0103295426815748
1648 0.0103295426815748
1649 0.0103295426815748
1650 0.0103295426815748
1651 0.0103295426815748
1652 0.0103295426815748
1653 0.0103295426815748
1654 0.0103295426815748
1655 0.0103295426815748
1656 0.0103295426815748
1657 0.0103295426815748
1658 0.0103295426815748
1659 0.0103295426815748
1660 0.0103295426815748
1661 0.0103295426815748
1662 0.0103295426815748
1663 0.0103295426815748
1664 0.0103295426815748
1665 0.0103295426815748
1666 0.0103295426815748
1667 0.0103295426815748
1668 0.0103295426815748
1669 0.0103295426815748
1670 0.0103295426815748
1671 0.0103295426815748
1672 0.0103295426815748
1673 0.0103295426815748
1674 0.0103295426815748
1675 0.0103295426815748
1676 0.0103295426815748
1677 0.0103295426815748
1678 0.0103295426815748
1679 0.0103295426815748
1680 0.0103295426815748
1681 0.0103295426815748
1682 0.0103295426815748
1683 0.0103295426815748
1684 0.0103295426815748
1685 0.0103295426815748
1686 0.0103295426815748
1687 0.0103295426815748
1688 0.0103295426815748
1689 0.0103295426815748
1690 0.0103295426815748
1691 0.0103295426815748
1692 0.0103295426815748
1693 0.0103295426815748
1694 0.0103295426815748
1695 0.0103295426815748
1696 0.0103295426815748
1697 0.0103295426815748
1698 0.0103295426815748
1699 0.0103295426815748
1700 0.0103295426815748
1701 0.0103295426815748
1702 0.0103295426815748
1703 0.0103295426815748
1704 0.0103295426815748
1705 0.0103295426815748
1706 0.0103295426815748
1707 0.0103295426815748
1708 0.0103295426815748
1709 0.0103295426815748
1710 0.0103295426815748
1711 0.0103295426815748
1712 0.0103295426815748
1713 0.0103295426815748
1714 0.0103295426815748
1715 0.0103295426815748
1716 0.0103295426815748
1717 0.0103295426815748
1718 0.0103295426815748
1719 0.0103295426815748
1720 0.0103295426815748
1721 0.0103295426815748
1722 0.0103295426815748
1723 0.0103295426815748
1724 0.0103295426815748
1725 0.0103295426815748
1726 0.0103295426815748
1727 0.0103295426815748
1728 0.0103295426815748
1729 0.0103295426815748
1730 0.0103295426815748
1731 0.0103295426815748
1732 0.0103295426815748
1733 0.0103295426815748
1734 0.0103295426815748
1735 0.0103295426815748
1736 0.0103295426815748
1737 0.0103295426815748
1738 0.0103295426815748
1739 0.0103295426815748
1740 0.0103295426815748
1741 0.0103295426815748
1742 0.0103295426815748
1743 0.0103295426815748
1744 0.0103295426815748
1745 0.0103295426815748
1746 0.0103295426815748
1747 0.0103295426815748
1748 0.0103295426815748
1749 0.0103295426815748
1750 0.0103295426815748
1751 0.0103295426815748
1752 0.0103295426815748
1753 0.0103295426815748
1754 0.0103295426815748
1755 0.0103295426815748
1756 0.0103295426815748
1757 0.0103295426815748
1758 0.0103295426815748
1759 0.0103295426815748
1760 0.0103295426815748
1761 0.0103295426815748
1762 0.0103295426815748
1763 0.0103295426815748
1764 0.0103295426815748
1765 0.0103295426815748
1766 0.0103295426815748
1767 0.0103295426815748
1768 0.0103295426815748
1769 0.0103295426815748
1770 0.0103295426815748
1771 0.0103295426815748
1772 0.0103295426815748
1773 0.0103295426815748
1774 0.0103295426815748
1775 0.0103295426815748
1776 0.0103295426815748
1777 0.0103295426815748
1778 0.0103295426815748
1779 0.0103295426815748
1780 0.0103295426815748
1781 0.0103295426815748
1782 0.0103295426815748
1783 0.0103295426815748
1784 0.0103295426815748
1785 0.0103295426815748
1786 0.0103295426815748
1787 0.0103295426815748
1788 0.0103295426815748
1789 0.0103295426815748
1790 0.0103295426815748
1791 0.0103295426815748
1792 0.0103295426815748
1793 0.0103295426815748
1794 0.0103295426815748
1795 0.0103295426815748
1796 0.0103295426815748
1797 0.0103295426815748
1798 0.0103295426815748
1799 0.0103295426815748
1800 0.0103295426815748
1801 0.0103295426815748
1802 0.0103295426815748
1803 0.0103295426815748
1804 0.0103295426815748
1805 0.0103295426815748
1806 0.0103295426815748
1807 0.0103295426815748
1808 0.0103295426815748
1809 0.0103295426815748
1810 0.0103295426815748
1811 0.0103295426815748
1812 0.0103295426815748
1813 0.0103295426815748
1814 0.0103295426815748
1815 0.0103295426815748
1816 0.0103295426815748
1817 0.0103295426815748
1818 0.0103295426815748
1819 0.0103295426815748
1820 0.0103295426815748
1821 0.0103295426815748
1822 0.0103295426815748
1823 0.0103295426815748
1824 0.0103295426815748
1825 0.0103295426815748
1826 0.0103295426815748
1827 0.0103295426815748
1828 0.0103295426815748
1829 0.0103295426815748
1830 0.0103295426815748
1831 0.0103295426815748
1832 0.0103295426815748
1833 0.0103295426815748
1834 0.0103295426815748
1835 0.0103295426815748
1836 0.0103295426815748
1837 0.0103295426815748
1838 0.0103295426815748
1839 0.0103295426815748
1840 0.0103295426815748
1841 0.0103295426815748
1842 0.0103295426815748
1843 0.0103295426815748
1844 0.0103295426815748
1845 0.0103295426815748
1846 0.0103295426815748
1847 0.0103295426815748
1848 0.0103295426815748
1849 0.0103295426815748
1850 0.0103295426815748
1851 0.0103295426815748
1852 0.0103295426815748
1853 0.0103295426815748
1854 0.0103295426815748
1855 0.0103295426815748
1856 0.0103295426815748
1857 0.0103295426815748
1858 0.0103295426815748
1859 0.0103295426815748
1860 0.0103295426815748
1861 0.0103295426815748
1862 0.0103295426815748
1863 0.0103295426815748
1864 0.0103295426815748
1865 0.0103295426815748
1866 0.0103295426815748
1867 0.0103295426815748
1868 0.0103295426815748
1869 0.0103295426815748
1870 0.0103295426815748
1871 0.0103295426815748
1872 0.0103295426815748
1873 0.0103295426815748
1874 0.0103295426815748
1875 0.0103295426815748
1876 0.0103295426815748
1877 0.0103295426815748
1878 0.0103295426815748
1879 0.0103295426815748
1880 0.0103295426815748
1881 0.0103295426815748
1882 0.0103295426815748
1883 0.0103295426815748
1884 0.0103295426815748
1885 0.0103295426815748
1886 0.0103295426815748
1887 0.0103295426815748
1888 0.0103295426815748
1889 0.0103295426815748
1890 0.0103295426815748
1891 0.0103295426815748
1892 0.0103295426815748
1893 0.0103295426815748
1894 0.0103295426815748
1895 0.0103295426815748
1896 0.0103295426815748
1897 0.0103295426815748
1898 0.0103295426815748
1899 0.0103295426815748
1900 0.0103295426815748
1901 0.0103295426815748
1902 0.0103295426815748
1903 0.0103295426815748
1904 0.0103295426815748
1905 0.0103295426815748
1906 0.0103295426815748
1907 0.0103295426815748
1908 0.0103295426815748
1909 0.0103295426815748
1910 0.0103295426815748
1911 0.0103295426815748
1912 0.0103295426815748
1913 0.0103295426815748
1914 0.0103295426815748
1915 0.0103295426815748
1916 0.0103295426815748
1917 0.0103295426815748
1918 0.0103295426815748
1919 0.0103295426815748
1920 0.0103295426815748
1921 0.0103295426815748
1922 0.0103295426815748
1923 0.0103295426815748
1924 0.0103295426815748
1925 0.0103295426815748
1926 0.0103295426815748
1927 0.0103295426815748
1928 0.0103295426815748
1929 0.0103295426815748
1930 0.0103295426815748
1931 0.0103295426815748
1932 0.0103295426815748
1933 0.0103295426815748
1934 0.0103295426815748
1935 0.0103295426815748
1936 0.0103295426815748
1937 0.0103295426815748
1938 0.0103295426815748
1939 0.0103295426815748
1940 0.0103295426815748
1941 0.0103295426815748
1942 0.0103295426815748
1943 0.0103295426815748
1944 0.0103295426815748
1945 0.0103295426815748
1946 0.0103295426815748
1947 0.0103295426815748
1948 0.0103295426815748
1949 0.0103295426815748
1950 0.0103295426815748
1951 0.0103295426815748
1952 0.0103295426815748
1953 0.0103295426815748
1954 0.0103295426815748
1955 0.0103295426815748
1956 0.0103295426815748
1957 0.0103295426815748
1958 0.0103295426815748
1959 0.0103295426815748
1960 0.0103295426815748
1961 0.0103295426815748
1962 0.0103295426815748
1963 0.0103295426815748
1964 0.0103295426815748
1965 0.0103295426815748
1966 0.0103295426815748
1967 0.0103295426815748
1968 0.0103295426815748
1969 0.0103295426815748
1970 0.0103295426815748
1971 0.0103295426815748
1972 0.0103295426815748
1973 0.0103295426815748
1974 0.0103295426815748
1975 0.0103295426815748
1976 0.0103295426815748
1977 0.0103295426815748
1978 0.0103295426815748
1979 0.0103295426815748
1980 0.0103295426815748
1981 0.0103295426815748
1982 0.0103295426815748
1983 0.0103295426815748
1984 0.0103295426815748
1985 0.0103295426815748
1986 0.0103295426815748
1987 0.0103295426815748
1988 0.0103295426815748
1989 0.0103295426815748
1990 0.0103295426815748
1991 0.0103295426815748
1992 0.0103295426815748
1993 0.0103295426815748
1994 0.0103295426815748
1995 0.0103295426815748
1996 0.0103295426815748
1997 0.0103295426815748
1998 0.0103295426815748
1999 0.0103295426815748
};
\addlegendentry{Test}

\nextgroupplot[
legend cell align={left},
legend style={fill opacity=0.8, draw opacity=1, text opacity=1, at={(0.91,0.5)}, anchor=east, draw=white!80!black},
log basis y={10},
tick align=outside,
tick pos=left,
title={fold3 },
x grid style={white!69.0196078431373!black},
xlabel={Epoch},
xmin=-99.95, xmax=2098.95,
xtick style={color=black},
y grid style={white!69.0196078431373!black},
ylabel={MSE Loss},
ymin=0.00680839679374005, ymax=0.00765943587595316,
ymode=log,
ytick style={color=black}
]
\addplot [semithick, black, dashed]
table {%
0 0.0068449447351955
1 0.0068449447351955
2 0.0068449447351955
3 0.0068449447351955
4 0.0068449447351955
5 0.0068449447351955
6 0.0068449447351955
7 0.0068449447351955
8 0.0068449447351955
9 0.0068449447351955
10 0.0068449447351955
11 0.0068449447351955
12 0.0068449447351955
13 0.0068449447351955
14 0.0068449447351955
15 0.0068449447351955
16 0.0068449447351955
17 0.0068449447351955
18 0.0068449447351955
19 0.0068449447351955
20 0.0068449447351955
21 0.0068449447351955
22 0.0068449447351955
23 0.0068449447351955
24 0.0068449447351955
25 0.0068449447351955
26 0.0068449447351955
27 0.0068449447351955
28 0.0068449447351955
29 0.0068449447351955
30 0.0068449447351955
31 0.0068449447351955
32 0.0068449447351955
33 0.0068449447351955
34 0.0068449447351955
35 0.0068449447351955
36 0.0068449447351955
37 0.0068449447351955
38 0.0068449447351955
39 0.0068449447351955
40 0.0068449447351955
41 0.0068449447351955
42 0.0068449447351955
43 0.0068449447351955
44 0.0068449447351955
45 0.0068449447351955
46 0.0068449447351955
47 0.0068449447351955
48 0.0068449447351955
49 0.0068449447351955
50 0.0068449447351955
51 0.0068449447351955
52 0.0068449447351955
53 0.0068449447351955
54 0.0068449447351955
55 0.0068449447351955
56 0.0068449447351955
57 0.0068449447351955
58 0.0068449447351955
59 0.0068449447351955
60 0.0068449447351955
61 0.0068449447351955
62 0.0068449447351955
63 0.0068449447351955
64 0.0068449447351955
65 0.0068449447351955
66 0.0068449447351955
67 0.0068449447351955
68 0.0068449447351955
69 0.0068449447351955
70 0.0068449447351955
71 0.0068449447351955
72 0.0068449447351955
73 0.0068449447351955
74 0.0068449447351955
75 0.0068449447351955
76 0.0068449447351955
77 0.0068449447351955
78 0.0068449447351955
79 0.0068449447351955
80 0.0068449447351955
81 0.0068449447351955
82 0.0068449447351955
83 0.0068449447351955
84 0.0068449447351955
85 0.0068449447351955
86 0.0068449447351955
87 0.0068449447351955
88 0.0068449447351955
89 0.0068449447351955
90 0.0068449447351955
91 0.0068449447351955
92 0.0068449447351955
93 0.0068449447351955
94 0.0068449447351955
95 0.0068449447351955
96 0.0068449447351955
97 0.0068449447351955
98 0.0068449447351955
99 0.0068449447351955
100 0.0068449447351955
101 0.0068449447351955
102 0.0068449447351955
103 0.0068449447351955
104 0.0068449447351955
105 0.0068449447351955
106 0.0068449447351955
107 0.0068449447351955
108 0.0068449447351955
109 0.0068449447351955
110 0.0068449447351955
111 0.0068449447351955
112 0.0068449447351955
113 0.0068449447351955
114 0.0068449447351955
115 0.0068449447351955
116 0.0068449447351955
117 0.0068449447351955
118 0.0068449447351955
119 0.0068449447351955
120 0.0068449447351955
121 0.0068449447351955
122 0.0068449447351955
123 0.0068449447351955
124 0.0068449447351955
125 0.0068449447351955
126 0.0068449447351955
127 0.0068449447351955
128 0.0068449447351955
129 0.0068449447351955
130 0.0068449447351955
131 0.0068449447351955
132 0.0068449447351955
133 0.0068449447351955
134 0.0068449447351955
135 0.0068449447351955
136 0.0068449447351955
137 0.0068449447351955
138 0.0068449447351955
139 0.0068449447351955
140 0.0068449447351955
141 0.0068449447351955
142 0.0068449447351955
143 0.0068449447351955
144 0.0068449447351955
145 0.0068449447351955
146 0.0068449447351955
147 0.0068449447351955
148 0.0068449447351955
149 0.0068449447351955
150 0.0068449447351955
151 0.0068449447351955
152 0.0068449447351955
153 0.0068449447351955
154 0.0068449447351955
155 0.0068449447351955
156 0.0068449447351955
157 0.0068449447351955
158 0.0068449447351955
159 0.0068449447351955
160 0.0068449447351955
161 0.0068449447351955
162 0.0068449447351955
163 0.0068449447351955
164 0.0068449447351955
165 0.0068449447351955
166 0.0068449447351955
167 0.0068449447351955
168 0.0068449447351955
169 0.0068449447351955
170 0.0068449447351955
171 0.0068449447351955
172 0.0068449447351955
173 0.0068449447351955
174 0.0068449447351955
175 0.0068449447351955
176 0.0068449447351955
177 0.0068449447351955
178 0.0068449447351955
179 0.0068449447351955
180 0.0068449447351955
181 0.0068449447351955
182 0.0068449447351955
183 0.0068449447351955
184 0.0068449447351955
185 0.0068449447351955
186 0.0068449447351955
187 0.0068449447351955
188 0.0068449447351955
189 0.0068449447351955
190 0.0068449447351955
191 0.0068449447351955
192 0.0068449447351955
193 0.0068449447351955
194 0.0068449447351955
195 0.0068449447351955
196 0.0068449447351955
197 0.0068449447351955
198 0.0068449447351955
199 0.0068449447351955
200 0.0068449447351955
201 0.0068449447351955
202 0.0068449447351955
203 0.0068449447351955
204 0.0068449447351955
205 0.0068449447351955
206 0.0068449447351955
207 0.0068449447351955
208 0.0068449447351955
209 0.0068449447351955
210 0.0068449447351955
211 0.0068449447351955
212 0.0068449447351955
213 0.0068449447351955
214 0.0068449447351955
215 0.0068449447351955
216 0.0068449447351955
217 0.0068449447351955
218 0.0068449447351955
219 0.0068449447351955
220 0.0068449447351955
221 0.0068449447351955
222 0.0068449447351955
223 0.0068449447351955
224 0.0068449447351955
225 0.0068449447351955
226 0.0068449447351955
227 0.0068449447351955
228 0.0068449447351955
229 0.0068449447351955
230 0.0068449447351955
231 0.0068449447351955
232 0.0068449447351955
233 0.0068449447351955
234 0.0068449447351955
235 0.0068449447351955
236 0.0068449447351955
237 0.0068449447351955
238 0.0068449447351955
239 0.0068449447351955
240 0.0068449447351955
241 0.0068449447351955
242 0.0068449447351955
243 0.0068449447351955
244 0.0068449447351955
245 0.0068449447351955
246 0.0068449447351955
247 0.0068449447351955
248 0.0068449447351955
249 0.0068449447351955
250 0.0068449447351955
251 0.0068449447351955
252 0.0068449447351955
253 0.0068449447351955
254 0.0068449447351955
255 0.0068449447351955
256 0.0068449447351955
257 0.0068449447351955
258 0.0068449447351955
259 0.0068449447351955
260 0.0068449447351955
261 0.0068449447351955
262 0.0068449447351955
263 0.0068449447351955
264 0.0068449447351955
265 0.0068449447351955
266 0.0068449447351955
267 0.0068449447351955
268 0.0068449447351955
269 0.0068449447351955
270 0.0068449447351955
271 0.0068449447351955
272 0.0068449447351955
273 0.0068449447351955
274 0.0068449447351955
275 0.0068449447351955
276 0.0068449447351955
277 0.0068449447351955
278 0.0068449447351955
279 0.0068449447351955
280 0.0068449447351955
281 0.0068449447351955
282 0.0068449447351955
283 0.0068449447351955
284 0.0068449447351955
285 0.0068449447351955
286 0.0068449447351955
287 0.0068449447351955
288 0.0068449447351955
289 0.0068449447351955
290 0.0068449447351955
291 0.0068449447351955
292 0.0068449447351955
293 0.0068449447351955
294 0.0068449447351955
295 0.0068449447351955
296 0.0068449447351955
297 0.0068449447351955
298 0.0068449447351955
299 0.0068449447351955
300 0.0068449447351955
301 0.0068449447351955
302 0.0068449447351955
303 0.0068449447351955
304 0.0068449447351955
305 0.0068449447351955
306 0.0068449447351955
307 0.0068449447351955
308 0.0068449447351955
309 0.0068449447351955
310 0.0068449447351955
311 0.0068449447351955
312 0.0068449447351955
313 0.0068449447351955
314 0.0068449447351955
315 0.0068449447351955
316 0.0068449447351955
317 0.0068449447351955
318 0.0068449447351955
319 0.0068449447351955
320 0.0068449447351955
321 0.0068449447351955
322 0.0068449447351955
323 0.0068449447351955
324 0.0068449447351955
325 0.0068449447351955
326 0.0068449447351955
327 0.0068449447351955
328 0.0068449447351955
329 0.0068449447351955
330 0.0068449447351955
331 0.0068449447351955
332 0.0068449447351955
333 0.0068449447351955
334 0.0068449447351955
335 0.0068449447351955
336 0.0068449447351955
337 0.0068449447351955
338 0.0068449447351955
339 0.0068449447351955
340 0.0068449447351955
341 0.0068449447351955
342 0.0068449447351955
343 0.0068449447351955
344 0.0068449447351955
345 0.0068449447351955
346 0.0068449447351955
347 0.0068449447351955
348 0.0068449447351955
349 0.0068449447351955
350 0.0068449447351955
351 0.0068449447351955
352 0.0068449447351955
353 0.0068449447351955
354 0.0068449447351955
355 0.0068449447351955
356 0.0068449447351955
357 0.0068449447351955
358 0.0068449447351955
359 0.0068449447351955
360 0.0068449447351955
361 0.0068449447351955
362 0.0068449447351955
363 0.0068449447351955
364 0.0068449447351955
365 0.0068449447351955
366 0.0068449447351955
367 0.0068449447351955
368 0.0068449447351955
369 0.0068449447351955
370 0.0068449447351955
371 0.0068449447351955
372 0.0068449447351955
373 0.0068449447351955
374 0.0068449447351955
375 0.0068449447351955
376 0.0068449447351955
377 0.0068449447351955
378 0.0068449447351955
379 0.0068449447351955
380 0.0068449447351955
381 0.0068449447351955
382 0.0068449447351955
383 0.0068449447351955
384 0.0068449447351955
385 0.0068449447351955
386 0.0068449447351955
387 0.0068449447351955
388 0.0068449447351955
389 0.0068449447351955
390 0.0068449447351955
391 0.0068449447351955
392 0.0068449447351955
393 0.0068449447351955
394 0.0068449447351955
395 0.0068449447351955
396 0.0068449447351955
397 0.0068449447351955
398 0.0068449447351955
399 0.0068449447351955
400 0.0068449447351955
401 0.0068449447351955
402 0.0068449447351955
403 0.0068449447351955
404 0.0068449447351955
405 0.0068449447351955
406 0.0068449447351955
407 0.0068449447351955
408 0.0068449447351955
409 0.0068449447351955
410 0.0068449447351955
411 0.0068449447351955
412 0.0068449447351955
413 0.0068449447351955
414 0.0068449447351955
415 0.0068449447351955
416 0.0068449447351955
417 0.0068449447351955
418 0.0068449447351955
419 0.0068449447351955
420 0.0068449447351955
421 0.0068449447351955
422 0.0068449447351955
423 0.0068449447351955
424 0.0068449447351955
425 0.0068449447351955
426 0.0068449447351955
427 0.0068449447351955
428 0.0068449447351955
429 0.0068449447351955
430 0.0068449447351955
431 0.0068449447351955
432 0.0068449447351955
433 0.0068449447351955
434 0.0068449447351955
435 0.0068449447351955
436 0.0068449447351955
437 0.0068449447351955
438 0.0068449447351955
439 0.0068449447351955
440 0.0068449447351955
441 0.0068449447351955
442 0.0068449447351955
443 0.0068449447351955
444 0.0068449447351955
445 0.0068449447351955
446 0.0068449447351955
447 0.0068449447351955
448 0.0068449447351955
449 0.0068449447351955
450 0.0068449447351955
451 0.0068449447351955
452 0.0068449447351955
453 0.0068449447351955
454 0.0068449447351955
455 0.0068449447351955
456 0.0068449447351955
457 0.0068449447351955
458 0.0068449447351955
459 0.0068449447351955
460 0.0068449447351955
461 0.0068449447351955
462 0.0068449447351955
463 0.0068449447351955
464 0.0068449447351955
465 0.0068449447351955
466 0.0068449447351955
467 0.0068449447351955
468 0.0068449447351955
469 0.0068449447351955
470 0.0068449447351955
471 0.0068449447351955
472 0.0068449447351955
473 0.0068449447351955
474 0.0068449447351955
475 0.0068449447351955
476 0.0068449447351955
477 0.0068449447351955
478 0.0068449447351955
479 0.0068449447351955
480 0.0068449447351955
481 0.0068449447351955
482 0.0068449447351955
483 0.0068449447351955
484 0.0068449447351955
485 0.0068449447351955
486 0.0068449447351955
487 0.0068449447351955
488 0.0068449447351955
489 0.0068449447351955
490 0.0068449447351955
491 0.0068449447351955
492 0.0068449447351955
493 0.0068449447351955
494 0.0068449447351955
495 0.0068449447351955
496 0.0068449447351955
497 0.0068449447351955
498 0.0068449447351955
499 0.0068449447351955
500 0.0068449447351955
501 0.0068449447351955
502 0.0068449447351955
503 0.0068449447351955
504 0.0068449447351955
505 0.0068449447351955
506 0.0068449447351955
507 0.0068449447351955
508 0.0068449447351955
509 0.0068449447351955
510 0.0068449447351955
511 0.0068449447351955
512 0.0068449447351955
513 0.0068449447351955
514 0.0068449447351955
515 0.0068449447351955
516 0.0068449447351955
517 0.0068449447351955
518 0.0068449447351955
519 0.0068449447351955
520 0.0068449447351955
521 0.0068449447351955
522 0.0068449447351955
523 0.0068449447351955
524 0.0068449447351955
525 0.0068449447351955
526 0.0068449447351955
527 0.0068449447351955
528 0.0068449447351955
529 0.0068449447351955
530 0.0068449447351955
531 0.0068449447351955
532 0.0068449447351955
533 0.0068449447351955
534 0.0068449447351955
535 0.0068449447351955
536 0.0068449447351955
537 0.0068449447351955
538 0.0068449447351955
539 0.0068449447351955
540 0.0068449447351955
541 0.0068449447351955
542 0.0068449447351955
543 0.0068449447351955
544 0.0068449447351955
545 0.0068449447351955
546 0.0068449447351955
547 0.0068449447351955
548 0.0068449447351955
549 0.0068449447351955
550 0.0068449447351955
551 0.0068449447351955
552 0.0068449447351955
553 0.0068449447351955
554 0.0068449447351955
555 0.0068449447351955
556 0.0068449447351955
557 0.0068449447351955
558 0.0068449447351955
559 0.0068449447351955
560 0.0068449447351955
561 0.0068449447351955
562 0.0068449447351955
563 0.0068449447351955
564 0.0068449447351955
565 0.0068449447351955
566 0.0068449447351955
567 0.0068449447351955
568 0.0068449447351955
569 0.0068449447351955
570 0.0068449447351955
571 0.0068449447351955
572 0.0068449447351955
573 0.0068449447351955
574 0.0068449447351955
575 0.0068449447351955
576 0.0068449447351955
577 0.0068449447351955
578 0.0068449447351955
579 0.0068449447351955
580 0.0068449447351955
581 0.0068449447351955
582 0.0068449447351955
583 0.0068449447351955
584 0.0068449447351955
585 0.0068449447351955
586 0.0068449447351955
587 0.0068449447351955
588 0.0068449447351955
589 0.0068449447351955
590 0.0068449447351955
591 0.0068449447351955
592 0.0068449447351955
593 0.0068449447351955
594 0.0068449447351955
595 0.0068449447351955
596 0.0068449447351955
597 0.0068449447351955
598 0.0068449447351955
599 0.0068449447351955
600 0.0068449447351955
601 0.0068449447351955
602 0.0068449447351955
603 0.0068449447351955
604 0.0068449447351955
605 0.0068449447351955
606 0.0068449447351955
607 0.0068449447351955
608 0.0068449447351955
609 0.0068449447351955
610 0.0068449447351955
611 0.0068449447351955
612 0.0068449447351955
613 0.0068449447351955
614 0.0068449447351955
615 0.0068449447351955
616 0.0068449447351955
617 0.0068449447351955
618 0.0068449447351955
619 0.0068449447351955
620 0.0068449447351955
621 0.0068449447351955
622 0.0068449447351955
623 0.0068449447351955
624 0.0068449447351955
625 0.0068449447351955
626 0.0068449447351955
627 0.0068449447351955
628 0.0068449447351955
629 0.0068449447351955
630 0.0068449447351955
631 0.0068449447351955
632 0.0068449447351955
633 0.0068449447351955
634 0.0068449447351955
635 0.0068449447351955
636 0.0068449447351955
637 0.0068449447351955
638 0.0068449447351955
639 0.0068449447351955
640 0.0068449447351955
641 0.0068449447351955
642 0.0068449447351955
643 0.0068449447351955
644 0.0068449447351955
645 0.0068449447351955
646 0.0068449447351955
647 0.0068449447351955
648 0.0068449447351955
649 0.0068449447351955
650 0.0068449447351955
651 0.0068449447351955
652 0.0068449447351955
653 0.0068449447351955
654 0.0068449447351955
655 0.0068449447351955
656 0.0068449447351955
657 0.0068449447351955
658 0.0068449447351955
659 0.0068449447351955
660 0.0068449447351955
661 0.0068449447351955
662 0.0068449447351955
663 0.0068449447351955
664 0.0068449447351955
665 0.0068449447351955
666 0.0068449447351955
667 0.0068449447351955
668 0.0068449447351955
669 0.0068449447351955
670 0.0068449447351955
671 0.0068449447351955
672 0.0068449447351955
673 0.0068449447351955
674 0.0068449447351955
675 0.0068449447351955
676 0.0068449447351955
677 0.0068449447351955
678 0.0068449447351955
679 0.0068449447351955
680 0.0068449447351955
681 0.0068449447351955
682 0.0068449447351955
683 0.0068449447351955
684 0.0068449447351955
685 0.0068449447351955
686 0.0068449447351955
687 0.0068449447351955
688 0.0068449447351955
689 0.0068449447351955
690 0.0068449447351955
691 0.0068449447351955
692 0.0068449447351955
693 0.0068449447351955
694 0.0068449447351955
695 0.0068449447351955
696 0.0068449447351955
697 0.0068449447351955
698 0.0068449447351955
699 0.0068449447351955
700 0.0068449447351955
701 0.0068449447351955
702 0.0068449447351955
703 0.0068449447351955
704 0.0068449447351955
705 0.0068449447351955
706 0.0068449447351955
707 0.0068449447351955
708 0.0068449447351955
709 0.0068449447351955
710 0.0068449447351955
711 0.0068449447351955
712 0.0068449447351955
713 0.0068449447351955
714 0.0068449447351955
715 0.0068449447351955
716 0.0068449447351955
717 0.0068449447351955
718 0.0068449447351955
719 0.0068449447351955
720 0.0068449447351955
721 0.0068449447351955
722 0.0068449447351955
723 0.0068449447351955
724 0.0068449447351955
725 0.0068449447351955
726 0.0068449447351955
727 0.0068449447351955
728 0.0068449447351955
729 0.0068449447351955
730 0.0068449447351955
731 0.0068449447351955
732 0.0068449447351955
733 0.0068449447351955
734 0.0068449447351955
735 0.0068449447351955
736 0.0068449447351955
737 0.0068449447351955
738 0.0068449447351955
739 0.0068449447351955
740 0.0068449447351955
741 0.0068449447351955
742 0.0068449447351955
743 0.0068449447351955
744 0.0068449447351955
745 0.0068449447351955
746 0.0068449447351955
747 0.0068449447351955
748 0.0068449447351955
749 0.0068449447351955
750 0.0068449447351955
751 0.0068449447351955
752 0.0068449447351955
753 0.0068449447351955
754 0.0068449447351955
755 0.0068449447351955
756 0.0068449447351955
757 0.0068449447351955
758 0.0068449447351955
759 0.0068449447351955
760 0.0068449447351955
761 0.0068449447351955
762 0.0068449447351955
763 0.0068449447351955
764 0.0068449447351955
765 0.0068449447351955
766 0.0068449447351955
767 0.0068449447351955
768 0.0068449447351955
769 0.0068449447351955
770 0.0068449447351955
771 0.0068449447351955
772 0.0068449447351955
773 0.0068449447351955
774 0.0068449447351955
775 0.0068449447351955
776 0.0068449447351955
777 0.0068449447351955
778 0.0068449447351955
779 0.0068449447351955
780 0.0068449447351955
781 0.0068449447351955
782 0.0068449447351955
783 0.0068449447351955
784 0.0068449447351955
785 0.0068449447351955
786 0.0068449447351955
787 0.0068449447351955
788 0.0068449447351955
789 0.0068449447351955
790 0.0068449447351955
791 0.0068449447351955
792 0.0068449447351955
793 0.0068449447351955
794 0.0068449447351955
795 0.0068449447351955
796 0.0068449447351955
797 0.0068449447351955
798 0.0068449447351955
799 0.0068449447351955
800 0.0068449447351955
801 0.0068449447351955
802 0.0068449447351955
803 0.0068449447351955
804 0.0068449447351955
805 0.0068449447351955
806 0.0068449447351955
807 0.0068449447351955
808 0.0068449447351955
809 0.0068449447351955
810 0.0068449447351955
811 0.0068449447351955
812 0.0068449447351955
813 0.0068449447351955
814 0.0068449447351955
815 0.0068449447351955
816 0.0068449447351955
817 0.0068449447351955
818 0.0068449447351955
819 0.0068449447351955
820 0.0068449447351955
821 0.0068449447351955
822 0.0068449447351955
823 0.0068449447351955
824 0.0068449447351955
825 0.0068449447351955
826 0.0068449447351955
827 0.0068449447351955
828 0.0068449447351955
829 0.0068449447351955
830 0.0068449447351955
831 0.0068449447351955
832 0.0068449447351955
833 0.0068449447351955
834 0.0068449447351955
835 0.0068449447351955
836 0.0068449447351955
837 0.0068449447351955
838 0.0068449447351955
839 0.0068449447351955
840 0.0068449447351955
841 0.0068449447351955
842 0.0068449447351955
843 0.0068449447351955
844 0.0068449447351955
845 0.0068449447351955
846 0.0068449447351955
847 0.0068449447351955
848 0.0068449447351955
849 0.0068449447351955
850 0.0068449447351955
851 0.0068449447351955
852 0.0068449447351955
853 0.0068449447351955
854 0.0068449447351955
855 0.0068449447351955
856 0.0068449447351955
857 0.0068449447351955
858 0.0068449447351955
859 0.0068449447351955
860 0.0068449447351955
861 0.0068449447351955
862 0.0068449447351955
863 0.0068449447351955
864 0.0068449447351955
865 0.0068449447351955
866 0.0068449447351955
867 0.0068449447351955
868 0.0068449447351955
869 0.0068449447351955
870 0.0068449447351955
871 0.0068449447351955
872 0.0068449447351955
873 0.0068449447351955
874 0.0068449447351955
875 0.0068449447351955
876 0.0068449447351955
877 0.0068449447351955
878 0.0068449447351955
879 0.0068449447351955
880 0.0068449447351955
881 0.0068449447351955
882 0.0068449447351955
883 0.0068449447351955
884 0.0068449447351955
885 0.0068449447351955
886 0.0068449447351955
887 0.0068449447351955
888 0.0068449447351955
889 0.0068449447351955
890 0.0068449447351955
891 0.0068449447351955
892 0.0068449447351955
893 0.0068449447351955
894 0.0068449447351955
895 0.0068449447351955
896 0.0068449447351955
897 0.0068449447351955
898 0.0068449447351955
899 0.0068449447351955
900 0.0068449447351955
901 0.0068449447351955
902 0.0068449447351955
903 0.0068449447351955
904 0.0068449447351955
905 0.0068449447351955
906 0.0068449447351955
907 0.0068449447351955
908 0.0068449447351955
909 0.0068449447351955
910 0.0068449447351955
911 0.0068449447351955
912 0.0068449447351955
913 0.0068449447351955
914 0.0068449447351955
915 0.0068449447351955
916 0.0068449447351955
917 0.0068449447351955
918 0.0068449447351955
919 0.0068449447351955
920 0.0068449447351955
921 0.0068449447351955
922 0.0068449447351955
923 0.0068449447351955
924 0.0068449447351955
925 0.0068449447351955
926 0.0068449447351955
927 0.0068449447351955
928 0.0068449447351955
929 0.0068449447351955
930 0.0068449447351955
931 0.0068449447351955
932 0.0068449447351955
933 0.0068449447351955
934 0.0068449447351955
935 0.0068449447351955
936 0.0068449447351955
937 0.0068449447351955
938 0.0068449447351955
939 0.0068449447351955
940 0.0068449447351955
941 0.0068449447351955
942 0.0068449447351955
943 0.0068449447351955
944 0.0068449447351955
945 0.0068449447351955
946 0.0068449447351955
947 0.0068449447351955
948 0.0068449447351955
949 0.0068449447351955
950 0.0068449447351955
951 0.0068449447351955
952 0.0068449447351955
953 0.0068449447351955
954 0.0068449447351955
955 0.0068449447351955
956 0.0068449447351955
957 0.0068449447351955
958 0.0068449447351955
959 0.0068449447351955
960 0.0068449447351955
961 0.0068449447351955
962 0.0068449447351955
963 0.0068449447351955
964 0.0068449447351955
965 0.0068449447351955
966 0.0068449447351955
967 0.0068449447351955
968 0.0068449447351955
969 0.0068449447351955
970 0.0068449447351955
971 0.0068449447351955
972 0.0068449447351955
973 0.0068449447351955
974 0.0068449447351955
975 0.0068449447351955
976 0.0068449447351955
977 0.0068449447351955
978 0.0068449447351955
979 0.0068449447351955
980 0.0068449447351955
981 0.0068449447351955
982 0.0068449447351955
983 0.0068449447351955
984 0.0068449447351955
985 0.0068449447351955
986 0.0068449447351955
987 0.0068449447351955
988 0.0068449447351955
989 0.0068449447351955
990 0.0068449447351955
991 0.0068449447351955
992 0.0068449447351955
993 0.0068449447351955
994 0.0068449447351955
995 0.0068449447351955
996 0.0068449447351955
997 0.0068449447351955
998 0.0068449447351955
999 0.0068449447351955
1000 0.0068449447351955
1001 0.0068449447351955
1002 0.0068449447351955
1003 0.0068449447351955
1004 0.0068449447351955
1005 0.0068449447351955
1006 0.0068449447351955
1007 0.0068449447351955
1008 0.0068449447351955
1009 0.0068449447351955
1010 0.0068449447351955
1011 0.0068449447351955
1012 0.0068449447351955
1013 0.0068449447351955
1014 0.0068449447351955
1015 0.0068449447351955
1016 0.0068449447351955
1017 0.0068449447351955
1018 0.0068449447351955
1019 0.0068449447351955
1020 0.0068449447351955
1021 0.0068449447351955
1022 0.0068449447351955
1023 0.0068449447351955
1024 0.0068449447351955
1025 0.0068449447351955
1026 0.0068449447351955
1027 0.0068449447351955
1028 0.0068449447351955
1029 0.0068449447351955
1030 0.0068449447351955
1031 0.0068449447351955
1032 0.0068449447351955
1033 0.0068449447351955
1034 0.0068449447351955
1035 0.0068449447351955
1036 0.0068449447351955
1037 0.0068449447351955
1038 0.0068449447351955
1039 0.0068449447351955
1040 0.0068449447351955
1041 0.0068449447351955
1042 0.0068449447351955
1043 0.0068449447351955
1044 0.0068449447351955
1045 0.0068449447351955
1046 0.0068449447351955
1047 0.0068449447351955
1048 0.0068449447351955
1049 0.0068449447351955
1050 0.0068449447351955
1051 0.0068449447351955
1052 0.0068449447351955
1053 0.0068449447351955
1054 0.0068449447351955
1055 0.0068449447351955
1056 0.0068449447351955
1057 0.0068449447351955
1058 0.0068449447351955
1059 0.0068449447351955
1060 0.0068449447351955
1061 0.0068449447351955
1062 0.0068449447351955
1063 0.0068449447351955
1064 0.0068449447351955
1065 0.0068449447351955
1066 0.0068449447351955
1067 0.0068449447351955
1068 0.0068449447351955
1069 0.0068449447351955
1070 0.0068449447351955
1071 0.0068449447351955
1072 0.0068449447351955
1073 0.0068449447351955
1074 0.0068449447351955
1075 0.0068449447351955
1076 0.0068449447351955
1077 0.0068449447351955
1078 0.0068449447351955
1079 0.0068449447351955
1080 0.0068449447351955
1081 0.0068449447351955
1082 0.0068449447351955
1083 0.0068449447351955
1084 0.0068449447351955
1085 0.0068449447351955
1086 0.0068449447351955
1087 0.0068449447351955
1088 0.0068449447351955
1089 0.0068449447351955
1090 0.0068449447351955
1091 0.0068449447351955
1092 0.0068449447351955
1093 0.0068449447351955
1094 0.0068449447351955
1095 0.0068449447351955
1096 0.0068449447351955
1097 0.0068449447351955
1098 0.0068449447351955
1099 0.0068449447351955
1100 0.0068449447351955
1101 0.0068449447351955
1102 0.0068449447351955
1103 0.0068449447351955
1104 0.0068449447351955
1105 0.0068449447351955
1106 0.0068449447351955
1107 0.0068449447351955
1108 0.0068449447351955
1109 0.0068449447351955
1110 0.0068449447351955
1111 0.0068449447351955
1112 0.0068449447351955
1113 0.0068449447351955
1114 0.0068449447351955
1115 0.0068449447351955
1116 0.0068449447351955
1117 0.0068449447351955
1118 0.0068449447351955
1119 0.0068449447351955
1120 0.0068449447351955
1121 0.0068449447351955
1122 0.0068449447351955
1123 0.0068449447351955
1124 0.0068449447351955
1125 0.0068449447351955
1126 0.0068449447351955
1127 0.0068449447351955
1128 0.0068449447351955
1129 0.0068449447351955
1130 0.0068449447351955
1131 0.0068449447351955
1132 0.0068449447351955
1133 0.0068449447351955
1134 0.0068449447351955
1135 0.0068449447351955
1136 0.0068449447351955
1137 0.0068449447351955
1138 0.0068449447351955
1139 0.0068449447351955
1140 0.0068449447351955
1141 0.0068449447351955
1142 0.0068449447351955
1143 0.0068449447351955
1144 0.0068449447351955
1145 0.0068449447351955
1146 0.0068449447351955
1147 0.0068449447351955
1148 0.0068449447351955
1149 0.0068449447351955
1150 0.0068449447351955
1151 0.0068449447351955
1152 0.0068449447351955
1153 0.0068449447351955
1154 0.0068449447351955
1155 0.0068449447351955
1156 0.0068449447351955
1157 0.0068449447351955
1158 0.0068449447351955
1159 0.0068449447351955
1160 0.0068449447351955
1161 0.0068449447351955
1162 0.0068449447351955
1163 0.0068449447351955
1164 0.0068449447351955
1165 0.0068449447351955
1166 0.0068449447351955
1167 0.0068449447351955
1168 0.0068449447351955
1169 0.0068449447351955
1170 0.0068449447351955
1171 0.0068449447351955
1172 0.0068449447351955
1173 0.0068449447351955
1174 0.0068449447351955
1175 0.0068449447351955
1176 0.0068449447351955
1177 0.0068449447351955
1178 0.0068449447351955
1179 0.0068449447351955
1180 0.0068449447351955
1181 0.0068449447351955
1182 0.0068449447351955
1183 0.0068449447351955
1184 0.0068449447351955
1185 0.0068449447351955
1186 0.0068449447351955
1187 0.0068449447351955
1188 0.0068449447351955
1189 0.0068449447351955
1190 0.0068449447351955
1191 0.0068449447351955
1192 0.0068449447351955
1193 0.0068449447351955
1194 0.0068449447351955
1195 0.0068449447351955
1196 0.0068449447351955
1197 0.0068449447351955
1198 0.0068449447351955
1199 0.0068449447351955
1200 0.0068449447351955
1201 0.0068449447351955
1202 0.0068449447351955
1203 0.0068449447351955
1204 0.0068449447351955
1205 0.0068449447351955
1206 0.0068449447351955
1207 0.0068449447351955
1208 0.0068449447351955
1209 0.0068449447351955
1210 0.0068449447351955
1211 0.0068449447351955
1212 0.0068449447351955
1213 0.0068449447351955
1214 0.0068449447351955
1215 0.0068449447351955
1216 0.0068449447351955
1217 0.0068449447351955
1218 0.0068449447351955
1219 0.0068449447351955
1220 0.0068449447351955
1221 0.0068449447351955
1222 0.0068449447351955
1223 0.0068449447351955
1224 0.0068449447351955
1225 0.0068449447351955
1226 0.0068449447351955
1227 0.0068449447351955
1228 0.0068449447351955
1229 0.0068449447351955
1230 0.0068449447351955
1231 0.0068449447351955
1232 0.0068449447351955
1233 0.0068449447351955
1234 0.0068449447351955
1235 0.0068449447351955
1236 0.0068449447351955
1237 0.0068449447351955
1238 0.0068449447351955
1239 0.0068449447351955
1240 0.0068449447351955
1241 0.0068449447351955
1242 0.0068449447351955
1243 0.0068449447351955
1244 0.0068449447351955
1245 0.0068449447351955
1246 0.0068449447351955
1247 0.0068449447351955
1248 0.0068449447351955
1249 0.0068449447351955
1250 0.0068449447351955
1251 0.0068449447351955
1252 0.0068449447351955
1253 0.0068449447351955
1254 0.0068449447351955
1255 0.0068449447351955
1256 0.0068449447351955
1257 0.0068449447351955
1258 0.0068449447351955
1259 0.0068449447351955
1260 0.0068449447351955
1261 0.0068449447351955
1262 0.0068449447351955
1263 0.0068449447351955
1264 0.0068449447351955
1265 0.0068449447351955
1266 0.0068449447351955
1267 0.0068449447351955
1268 0.0068449447351955
1269 0.0068449447351955
1270 0.0068449447351955
1271 0.0068449447351955
1272 0.0068449447351955
1273 0.0068449447351955
1274 0.0068449447351955
1275 0.0068449447351955
1276 0.0068449447351955
1277 0.0068449447351955
1278 0.0068449447351955
1279 0.0068449447351955
1280 0.0068449447351955
1281 0.0068449447351955
1282 0.0068449447351955
1283 0.0068449447351955
1284 0.0068449447351955
1285 0.0068449447351955
1286 0.0068449447351955
1287 0.0068449447351955
1288 0.0068449447351955
1289 0.0068449447351955
1290 0.0068449447351955
1291 0.0068449447351955
1292 0.0068449447351955
1293 0.0068449447351955
1294 0.0068449447351955
1295 0.0068449447351955
1296 0.0068449447351955
1297 0.0068449447351955
1298 0.0068449447351955
1299 0.0068449447351955
1300 0.0068449447351955
1301 0.0068449447351955
1302 0.0068449447351955
1303 0.0068449447351955
1304 0.0068449447351955
1305 0.0068449447351955
1306 0.0068449447351955
1307 0.0068449447351955
1308 0.0068449447351955
1309 0.0068449447351955
1310 0.0068449447351955
1311 0.0068449447351955
1312 0.0068449447351955
1313 0.0068449447351955
1314 0.0068449447351955
1315 0.0068449447351955
1316 0.0068449447351955
1317 0.0068449447351955
1318 0.0068449447351955
1319 0.0068449447351955
1320 0.0068449447351955
1321 0.0068449447351955
1322 0.0068449447351955
1323 0.0068449447351955
1324 0.0068449447351955
1325 0.0068449447351955
1326 0.0068449447351955
1327 0.0068449447351955
1328 0.0068449447351955
1329 0.0068449447351955
1330 0.0068449447351955
1331 0.0068449447351955
1332 0.0068449447351955
1333 0.0068449447351955
1334 0.0068449447351955
1335 0.0068449447351955
1336 0.0068449447351955
1337 0.0068449447351955
1338 0.0068449447351955
1339 0.0068449447351955
1340 0.0068449447351955
1341 0.0068449447351955
1342 0.0068449447351955
1343 0.0068449447351955
1344 0.0068449447351955
1345 0.0068449447351955
1346 0.0068449447351955
1347 0.0068449447351955
1348 0.0068449447351955
1349 0.0068449447351955
1350 0.0068449447351955
1351 0.0068449447351955
1352 0.0068449447351955
1353 0.0068449447351955
1354 0.0068449447351955
1355 0.0068449447351955
1356 0.0068449447351955
1357 0.0068449447351955
1358 0.0068449447351955
1359 0.0068449447351955
1360 0.0068449447351955
1361 0.0068449447351955
1362 0.0068449447351955
1363 0.0068449447351955
1364 0.0068449447351955
1365 0.0068449447351955
1366 0.0068449447351955
1367 0.0068449447351955
1368 0.0068449447351955
1369 0.0068449447351955
1370 0.0068449447351955
1371 0.0068449447351955
1372 0.0068449447351955
1373 0.0068449447351955
1374 0.0068449447351955
1375 0.0068449447351955
1376 0.0068449447351955
1377 0.0068449447351955
1378 0.0068449447351955
1379 0.0068449447351955
1380 0.0068449447351955
1381 0.0068449447351955
1382 0.0068449447351955
1383 0.0068449447351955
1384 0.0068449447351955
1385 0.0068449447351955
1386 0.0068449447351955
1387 0.0068449447351955
1388 0.0068449447351955
1389 0.0068449447351955
1390 0.0068449447351955
1391 0.0068449447351955
1392 0.0068449447351955
1393 0.0068449447351955
1394 0.0068449447351955
1395 0.0068449447351955
1396 0.0068449447351955
1397 0.0068449447351955
1398 0.0068449447351955
1399 0.0068449447351955
1400 0.0068449447351955
1401 0.0068449447351955
1402 0.0068449447351955
1403 0.0068449447351955
1404 0.0068449447351955
1405 0.0068449447351955
1406 0.0068449447351955
1407 0.0068449447351955
1408 0.0068449447351955
1409 0.0068449447351955
1410 0.0068449447351955
1411 0.0068449447351955
1412 0.0068449447351955
1413 0.0068449447351955
1414 0.0068449447351955
1415 0.0068449447351955
1416 0.0068449447351955
1417 0.0068449447351955
1418 0.0068449447351955
1419 0.0068449447351955
1420 0.0068449447351955
1421 0.0068449447351955
1422 0.0068449447351955
1423 0.0068449447351955
1424 0.0068449447351955
1425 0.0068449447351955
1426 0.0068449447351955
1427 0.0068449447351955
1428 0.0068449447351955
1429 0.0068449447351955
1430 0.0068449447351955
1431 0.0068449447351955
1432 0.0068449447351955
1433 0.0068449447351955
1434 0.0068449447351955
1435 0.0068449447351955
1436 0.0068449447351955
1437 0.0068449447351955
1438 0.0068449447351955
1439 0.0068449447351955
1440 0.0068449447351955
1441 0.0068449447351955
1442 0.0068449447351955
1443 0.0068449447351955
1444 0.0068449447351955
1445 0.0068449447351955
1446 0.0068449447351955
1447 0.0068449447351955
1448 0.0068449447351955
1449 0.0068449447351955
1450 0.0068449447351955
1451 0.0068449447351955
1452 0.0068449447351955
1453 0.0068449447351955
1454 0.0068449447351955
1455 0.0068449447351955
1456 0.0068449447351955
1457 0.0068449447351955
1458 0.0068449447351955
1459 0.0068449447351955
1460 0.0068449447351955
1461 0.0068449447351955
1462 0.0068449447351955
1463 0.0068449447351955
1464 0.0068449447351955
1465 0.0068449447351955
1466 0.0068449447351955
1467 0.0068449447351955
1468 0.0068449447351955
1469 0.0068449447351955
1470 0.0068449447351955
1471 0.0068449447351955
1472 0.0068449447351955
1473 0.0068449447351955
1474 0.0068449447351955
1475 0.0068449447351955
1476 0.0068449447351955
1477 0.0068449447351955
1478 0.0068449447351955
1479 0.0068449447351955
1480 0.0068449447351955
1481 0.0068449447351955
1482 0.0068449447351955
1483 0.0068449447351955
1484 0.0068449447351955
1485 0.0068449447351955
1486 0.0068449447351955
1487 0.0068449447351955
1488 0.0068449447351955
1489 0.0068449447351955
1490 0.0068449447351955
1491 0.0068449447351955
1492 0.0068449447351955
1493 0.0068449447351955
1494 0.0068449447351955
1495 0.0068449447351955
1496 0.0068449447351955
1497 0.0068449447351955
1498 0.0068449447351955
1499 0.0068449447351955
1500 0.0068449447351955
1501 0.0068449447351955
1502 0.0068449447351955
1503 0.0068449447351955
1504 0.0068449447351955
1505 0.0068449447351955
1506 0.0068449447351955
1507 0.0068449447351955
1508 0.0068449447351955
1509 0.0068449447351955
1510 0.0068449447351955
1511 0.0068449447351955
1512 0.0068449447351955
1513 0.0068449447351955
1514 0.0068449447351955
1515 0.0068449447351955
1516 0.0068449447351955
1517 0.0068449447351955
1518 0.0068449447351955
1519 0.0068449447351955
1520 0.0068449447351955
1521 0.0068449447351955
1522 0.0068449447351955
1523 0.0068449447351955
1524 0.0068449447351955
1525 0.0068449447351955
1526 0.0068449447351955
1527 0.0068449447351955
1528 0.0068449447351955
1529 0.0068449447351955
1530 0.0068449447351955
1531 0.0068449447351955
1532 0.0068449447351955
1533 0.0068449447351955
1534 0.0068449447351955
1535 0.0068449447351955
1536 0.0068449447351955
1537 0.0068449447351955
1538 0.0068449447351955
1539 0.0068449447351955
1540 0.0068449447351955
1541 0.0068449447351955
1542 0.0068449447351955
1543 0.0068449447351955
1544 0.0068449447351955
1545 0.0068449447351955
1546 0.0068449447351955
1547 0.0068449447351955
1548 0.0068449447351955
1549 0.0068449447351955
1550 0.0068449447351955
1551 0.0068449447351955
1552 0.0068449447351955
1553 0.0068449447351955
1554 0.0068449447351955
1555 0.0068449447351955
1556 0.0068449447351955
1557 0.0068449447351955
1558 0.0068449447351955
1559 0.0068449447351955
1560 0.0068449447351955
1561 0.0068449447351955
1562 0.0068449447351955
1563 0.0068449447351955
1564 0.0068449447351955
1565 0.0068449447351955
1566 0.0068449447351955
1567 0.0068449447351955
1568 0.0068449447351955
1569 0.0068449447351955
1570 0.0068449447351955
1571 0.0068449447351955
1572 0.0068449447351955
1573 0.0068449447351955
1574 0.0068449447351955
1575 0.0068449447351955
1576 0.0068449447351955
1577 0.0068449447351955
1578 0.0068449447351955
1579 0.0068449447351955
1580 0.0068449447351955
1581 0.0068449447351955
1582 0.0068449447351955
1583 0.0068449447351955
1584 0.0068449447351955
1585 0.0068449447351955
1586 0.0068449447351955
1587 0.0068449447351955
1588 0.0068449447351955
1589 0.0068449447351955
1590 0.0068449447351955
1591 0.0068449447351955
1592 0.0068449447351955
1593 0.0068449447351955
1594 0.0068449447351955
1595 0.0068449447351955
1596 0.0068449447351955
1597 0.0068449447351955
1598 0.0068449447351955
1599 0.0068449447351955
1600 0.0068449447351955
1601 0.0068449447351955
1602 0.0068449447351955
1603 0.0068449447351955
1604 0.0068449447351955
1605 0.0068449447351955
1606 0.0068449447351955
1607 0.0068449447351955
1608 0.0068449447351955
1609 0.0068449447351955
1610 0.0068449447351955
1611 0.0068449447351955
1612 0.0068449447351955
1613 0.0068449447351955
1614 0.0068449447351955
1615 0.0068449447351955
1616 0.0068449447351955
1617 0.0068449447351955
1618 0.0068449447351955
1619 0.0068449447351955
1620 0.0068449447351955
1621 0.0068449447351955
1622 0.0068449447351955
1623 0.0068449447351955
1624 0.0068449447351955
1625 0.0068449447351955
1626 0.0068449447351955
1627 0.0068449447351955
1628 0.0068449447351955
1629 0.0068449447351955
1630 0.0068449447351955
1631 0.0068449447351955
1632 0.0068449447351955
1633 0.0068449447351955
1634 0.0068449447351955
1635 0.0068449447351955
1636 0.0068449447351955
1637 0.0068449447351955
1638 0.0068449447351955
1639 0.0068449447351955
1640 0.0068449447351955
1641 0.0068449447351955
1642 0.0068449447351955
1643 0.0068449447351955
1644 0.0068449447351955
1645 0.0068449447351955
1646 0.0068449447351955
1647 0.0068449447351955
1648 0.0068449447351955
1649 0.0068449447351955
1650 0.0068449447351955
1651 0.0068449447351955
1652 0.0068449447351955
1653 0.0068449447351955
1654 0.0068449447351955
1655 0.0068449447351955
1656 0.0068449447351955
1657 0.0068449447351955
1658 0.0068449447351955
1659 0.0068449447351955
1660 0.0068449447351955
1661 0.0068449447351955
1662 0.0068449447351955
1663 0.0068449447351955
1664 0.0068449447351955
1665 0.0068449447351955
1666 0.0068449447351955
1667 0.0068449447351955
1668 0.0068449447351955
1669 0.0068449447351955
1670 0.0068449447351955
1671 0.0068449447351955
1672 0.0068449447351955
1673 0.0068449447351955
1674 0.0068449447351955
1675 0.0068449447351955
1676 0.0068449447351955
1677 0.0068449447351955
1678 0.0068449447351955
1679 0.0068449447351955
1680 0.0068449447351955
1681 0.0068449447351955
1682 0.0068449447351955
1683 0.0068449447351955
1684 0.0068449447351955
1685 0.0068449447351955
1686 0.0068449447351955
1687 0.0068449447351955
1688 0.0068449447351955
1689 0.0068449447351955
1690 0.0068449447351955
1691 0.0068449447351955
1692 0.0068449447351955
1693 0.0068449447351955
1694 0.0068449447351955
1695 0.0068449447351955
1696 0.0068449447351955
1697 0.0068449447351955
1698 0.0068449447351955
1699 0.0068449447351955
1700 0.0068449447351955
1701 0.0068449447351955
1702 0.0068449447351955
1703 0.0068449447351955
1704 0.0068449447351955
1705 0.0068449447351955
1706 0.0068449447351955
1707 0.0068449447351955
1708 0.0068449447351955
1709 0.0068449447351955
1710 0.0068449447351955
1711 0.0068449447351955
1712 0.0068449447351955
1713 0.0068449447351955
1714 0.0068449447351955
1715 0.0068449447351955
1716 0.0068449447351955
1717 0.0068449447351955
1718 0.0068449447351955
1719 0.0068449447351955
1720 0.0068449447351955
1721 0.0068449447351955
1722 0.0068449447351955
1723 0.0068449447351955
1724 0.0068449447351955
1725 0.0068449447351955
1726 0.0068449447351955
1727 0.0068449447351955
1728 0.0068449447351955
1729 0.0068449447351955
1730 0.0068449447351955
1731 0.0068449447351955
1732 0.0068449447351955
1733 0.0068449447351955
1734 0.0068449447351955
1735 0.0068449447351955
1736 0.0068449447351955
1737 0.0068449447351955
1738 0.0068449447351955
1739 0.0068449447351955
1740 0.0068449447351955
1741 0.0068449447351955
1742 0.0068449447351955
1743 0.0068449447351955
1744 0.0068449447351955
1745 0.0068449447351955
1746 0.0068449447351955
1747 0.0068449447351955
1748 0.0068449447351955
1749 0.0068449447351955
1750 0.0068449447351955
1751 0.0068449447351955
1752 0.0068449447351955
1753 0.0068449447351955
1754 0.0068449447351955
1755 0.0068449447351955
1756 0.0068449447351955
1757 0.0068449447351955
1758 0.0068449447351955
1759 0.0068449447351955
1760 0.0068449447351955
1761 0.0068449447351955
1762 0.0068449447351955
1763 0.0068449447351955
1764 0.0068449447351955
1765 0.0068449447351955
1766 0.0068449447351955
1767 0.0068449447351955
1768 0.0068449447351955
1769 0.0068449447351955
1770 0.0068449447351955
1771 0.0068449447351955
1772 0.0068449447351955
1773 0.0068449447351955
1774 0.0068449447351955
1775 0.0068449447351955
1776 0.0068449447351955
1777 0.0068449447351955
1778 0.0068449447351955
1779 0.0068449447351955
1780 0.0068449447351955
1781 0.0068449447351955
1782 0.0068449447351955
1783 0.0068449447351955
1784 0.0068449447351955
1785 0.0068449447351955
1786 0.0068449447351955
1787 0.0068449447351955
1788 0.0068449447351955
1789 0.0068449447351955
1790 0.0068449447351955
1791 0.0068449447351955
1792 0.0068449447351955
1793 0.0068449447351955
1794 0.0068449447351955
1795 0.0068449447351955
1796 0.0068449447351955
1797 0.0068449447351955
1798 0.0068449447351955
1799 0.0068449447351955
1800 0.0068449447351955
1801 0.0068449447351955
1802 0.0068449447351955
1803 0.0068449447351955
1804 0.0068449447351955
1805 0.0068449447351955
1806 0.0068449447351955
1807 0.0068449447351955
1808 0.0068449447351955
1809 0.0068449447351955
1810 0.0068449447351955
1811 0.0068449447351955
1812 0.0068449447351955
1813 0.0068449447351955
1814 0.0068449447351955
1815 0.0068449447351955
1816 0.0068449447351955
1817 0.0068449447351955
1818 0.0068449447351955
1819 0.0068449447351955
1820 0.0068449447351955
1821 0.0068449447351955
1822 0.0068449447351955
1823 0.0068449447351955
1824 0.0068449447351955
1825 0.0068449447351955
1826 0.0068449447351955
1827 0.0068449447351955
1828 0.0068449447351955
1829 0.0068449447351955
1830 0.0068449447351955
1831 0.0068449447351955
1832 0.0068449447351955
1833 0.0068449447351955
1834 0.0068449447351955
1835 0.0068449447351955
1836 0.0068449447351955
1837 0.0068449447351955
1838 0.0068449447351955
1839 0.0068449447351955
1840 0.0068449447351955
1841 0.0068449447351955
1842 0.0068449447351955
1843 0.0068449447351955
1844 0.0068449447351955
1845 0.0068449447351955
1846 0.0068449447351955
1847 0.0068449447351955
1848 0.0068449447351955
1849 0.0068449447351955
1850 0.0068449447351955
1851 0.0068449447351955
1852 0.0068449447351955
1853 0.0068449447351955
1854 0.0068449447351955
1855 0.0068449447351955
1856 0.0068449447351955
1857 0.0068449447351955
1858 0.0068449447351955
1859 0.0068449447351955
1860 0.0068449447351955
1861 0.0068449447351955
1862 0.0068449447351955
1863 0.0068449447351955
1864 0.0068449447351955
1865 0.0068449447351955
1866 0.0068449447351955
1867 0.0068449447351955
1868 0.0068449447351955
1869 0.0068449447351955
1870 0.0068449447351955
1871 0.0068449447351955
1872 0.0068449447351955
1873 0.0068449447351955
1874 0.0068449447351955
1875 0.0068449447351955
1876 0.0068449447351955
1877 0.0068449447351955
1878 0.0068449447351955
1879 0.0068449447351955
1880 0.0068449447351955
1881 0.0068449447351955
1882 0.0068449447351955
1883 0.0068449447351955
1884 0.0068449447351955
1885 0.0068449447351955
1886 0.0068449447351955
1887 0.0068449447351955
1888 0.0068449447351955
1889 0.0068449447351955
1890 0.0068449447351955
1891 0.0068449447351955
1892 0.0068449447351955
1893 0.0068449447351955
1894 0.0068449447351955
1895 0.0068449447351955
1896 0.0068449447351955
1897 0.0068449447351955
1898 0.0068449447351955
1899 0.0068449447351955
1900 0.0068449447351955
1901 0.0068449447351955
1902 0.0068449447351955
1903 0.0068449447351955
1904 0.0068449447351955
1905 0.0068449447351955
1906 0.0068449447351955
1907 0.0068449447351955
1908 0.0068449447351955
1909 0.0068449447351955
1910 0.0068449447351955
1911 0.0068449447351955
1912 0.0068449447351955
1913 0.0068449447351955
1914 0.0068449447351955
1915 0.0068449447351955
1916 0.0068449447351955
1917 0.0068449447351955
1918 0.0068449447351955
1919 0.0068449447351955
1920 0.0068449447351955
1921 0.0068449447351955
1922 0.0068449447351955
1923 0.0068449447351955
1924 0.0068449447351955
1925 0.0068449447351955
1926 0.0068449447351955
1927 0.0068449447351955
1928 0.0068449447351955
1929 0.0068449447351955
1930 0.0068449447351955
1931 0.0068449447351955
1932 0.0068449447351955
1933 0.0068449447351955
1934 0.0068449447351955
1935 0.0068449447351955
1936 0.0068449447351955
1937 0.0068449447351955
1938 0.0068449447351955
1939 0.0068449447351955
1940 0.0068449447351955
1941 0.0068449447351955
1942 0.0068449447351955
1943 0.0068449447351955
1944 0.0068449447351955
1945 0.0068449447351955
1946 0.0068449447351955
1947 0.0068449447351955
1948 0.0068449447351955
1949 0.0068449447351955
1950 0.0068449447351955
1951 0.0068449447351955
1952 0.0068449447351955
1953 0.0068449447351955
1954 0.0068449447351955
1955 0.0068449447351955
1956 0.0068449447351955
1957 0.0068449447351955
1958 0.0068449447351955
1959 0.0068449447351955
1960 0.0068449447351955
1961 0.0068449447351955
1962 0.0068449447351955
1963 0.0068449447351955
1964 0.0068449447351955
1965 0.0068449447351955
1966 0.0068449447351955
1967 0.0068449447351955
1968 0.0068449447351955
1969 0.0068449447351955
1970 0.0068449447351955
1971 0.0068449447351955
1972 0.0068449447351955
1973 0.0068449447351955
1974 0.0068449447351955
1975 0.0068449447351955
1976 0.0068449447351955
1977 0.0068449447351955
1978 0.0068449447351955
1979 0.0068449447351955
1980 0.0068449447351955
1981 0.0068449447351955
1982 0.0068449447351955
1983 0.0068449447351955
1984 0.0068449447351955
1985 0.0068449447351955
1986 0.0068449447351955
1987 0.0068449447351955
1988 0.0068449447351955
1989 0.0068449447351955
1990 0.0068449447351955
1991 0.0068449447351955
1992 0.0068449447351955
1993 0.0068449447351955
1994 0.0068449447351955
1995 0.0068449447351955
1996 0.0068449447351955
1997 0.0068449447351955
1998 0.0068449447351955
1999 0.0068449447351955
};
\addlegendentry{Train}
\addplot [semithick, black]
table {%
0 0.00761853903532028
1 0.00761853903532028
2 0.00761853903532028
3 0.00761853903532028
4 0.00761853903532028
5 0.00761853903532028
6 0.00761853903532028
7 0.00761853903532028
8 0.00761853903532028
9 0.00761853903532028
10 0.00761853903532028
11 0.00761853903532028
12 0.00761853903532028
13 0.00761853903532028
14 0.00761853903532028
15 0.00761853903532028
16 0.00761853903532028
17 0.00761853903532028
18 0.00761853903532028
19 0.00761853903532028
20 0.00761853903532028
21 0.00761853903532028
22 0.00761853903532028
23 0.00761853903532028
24 0.00761853903532028
25 0.00761853903532028
26 0.00761853903532028
27 0.00761853903532028
28 0.00761853903532028
29 0.00761853903532028
30 0.00761853903532028
31 0.00761853903532028
32 0.00761853903532028
33 0.00761853903532028
34 0.00761853903532028
35 0.00761853903532028
36 0.00761853903532028
37 0.00761853903532028
38 0.00761853903532028
39 0.00761853903532028
40 0.00761853903532028
41 0.00761853903532028
42 0.00761853903532028
43 0.00761853903532028
44 0.00761853903532028
45 0.00761853903532028
46 0.00761853903532028
47 0.00761853903532028
48 0.00761853903532028
49 0.00761853903532028
50 0.00761853903532028
51 0.00761853903532028
52 0.00761853903532028
53 0.00761853903532028
54 0.00761853903532028
55 0.00761853903532028
56 0.00761853903532028
57 0.00761853903532028
58 0.00761853903532028
59 0.00761853903532028
60 0.00761853903532028
61 0.00761853903532028
62 0.00761853903532028
63 0.00761853903532028
64 0.00761853903532028
65 0.00761853903532028
66 0.00761853903532028
67 0.00761853903532028
68 0.00761853903532028
69 0.00761853903532028
70 0.00761853903532028
71 0.00761853903532028
72 0.00761853903532028
73 0.00761853903532028
74 0.00761853903532028
75 0.00761853903532028
76 0.00761853903532028
77 0.00761853903532028
78 0.00761853903532028
79 0.00761853903532028
80 0.00761853903532028
81 0.00761853903532028
82 0.00761853903532028
83 0.00761853903532028
84 0.00761853903532028
85 0.00761853903532028
86 0.00761853903532028
87 0.00761853903532028
88 0.00761853903532028
89 0.00761853903532028
90 0.00761853903532028
91 0.00761853903532028
92 0.00761853903532028
93 0.00761853903532028
94 0.00761853903532028
95 0.00761853903532028
96 0.00761853903532028
97 0.00761853903532028
98 0.00761853903532028
99 0.00761853903532028
100 0.00761853903532028
101 0.00761853903532028
102 0.00761853903532028
103 0.00761853903532028
104 0.00761853903532028
105 0.00761853903532028
106 0.00761853903532028
107 0.00761853903532028
108 0.00761853903532028
109 0.00761853903532028
110 0.00761853903532028
111 0.00761853903532028
112 0.00761853903532028
113 0.00761853903532028
114 0.00761853903532028
115 0.00761853903532028
116 0.00761853903532028
117 0.00761853903532028
118 0.00761853903532028
119 0.00761853903532028
120 0.00761853903532028
121 0.00761853903532028
122 0.00761853903532028
123 0.00761853903532028
124 0.00761853903532028
125 0.00761853903532028
126 0.00761853903532028
127 0.00761853903532028
128 0.00761853903532028
129 0.00761853903532028
130 0.00761853903532028
131 0.00761853903532028
132 0.00761853903532028
133 0.00761853903532028
134 0.00761853903532028
135 0.00761853903532028
136 0.00761853903532028
137 0.00761853903532028
138 0.00761853903532028
139 0.00761853903532028
140 0.00761853903532028
141 0.00761853903532028
142 0.00761853903532028
143 0.00761853903532028
144 0.00761853903532028
145 0.00761853903532028
146 0.00761853903532028
147 0.00761853903532028
148 0.00761853903532028
149 0.00761853903532028
150 0.00761853903532028
151 0.00761853903532028
152 0.00761853903532028
153 0.00761853903532028
154 0.00761853903532028
155 0.00761853903532028
156 0.00761853903532028
157 0.00761853903532028
158 0.00761853903532028
159 0.00761853903532028
160 0.00761853903532028
161 0.00761853903532028
162 0.00761853903532028
163 0.00761853903532028
164 0.00761853903532028
165 0.00761853903532028
166 0.00761853903532028
167 0.00761853903532028
168 0.00761853903532028
169 0.00761853903532028
170 0.00761853903532028
171 0.00761853903532028
172 0.00761853903532028
173 0.00761853903532028
174 0.00761853903532028
175 0.00761853903532028
176 0.00761853903532028
177 0.00761853903532028
178 0.00761853903532028
179 0.00761853903532028
180 0.00761853903532028
181 0.00761853903532028
182 0.00761853903532028
183 0.00761853903532028
184 0.00761853903532028
185 0.00761853903532028
186 0.00761853903532028
187 0.00761853903532028
188 0.00761853903532028
189 0.00761853903532028
190 0.00761853903532028
191 0.00761853903532028
192 0.00761853903532028
193 0.00761853903532028
194 0.00761853903532028
195 0.00761853903532028
196 0.00761853903532028
197 0.00761853903532028
198 0.00761853903532028
199 0.00761853903532028
200 0.00761853903532028
201 0.00761853903532028
202 0.00761853903532028
203 0.00761853903532028
204 0.00761853903532028
205 0.00761853903532028
206 0.00761853903532028
207 0.00761853903532028
208 0.00761853903532028
209 0.00761853903532028
210 0.00761853903532028
211 0.00761853903532028
212 0.00761853903532028
213 0.00761853903532028
214 0.00761853903532028
215 0.00761853903532028
216 0.00761853903532028
217 0.00761853903532028
218 0.00761853903532028
219 0.00761853903532028
220 0.00761853903532028
221 0.00761853903532028
222 0.00761853903532028
223 0.00761853903532028
224 0.00761853903532028
225 0.00761853903532028
226 0.00761853903532028
227 0.00761853903532028
228 0.00761853903532028
229 0.00761853903532028
230 0.00761853903532028
231 0.00761853903532028
232 0.00761853903532028
233 0.00761853903532028
234 0.00761853903532028
235 0.00761853903532028
236 0.00761853903532028
237 0.00761853903532028
238 0.00761853903532028
239 0.00761853903532028
240 0.00761853903532028
241 0.00761853903532028
242 0.00761853903532028
243 0.00761853903532028
244 0.00761853903532028
245 0.00761853903532028
246 0.00761853903532028
247 0.00761853903532028
248 0.00761853903532028
249 0.00761853903532028
250 0.00761853903532028
251 0.00761853903532028
252 0.00761853903532028
253 0.00761853903532028
254 0.00761853903532028
255 0.00761853903532028
256 0.00761853903532028
257 0.00761853903532028
258 0.00761853903532028
259 0.00761853903532028
260 0.00761853903532028
261 0.00761853903532028
262 0.00761853903532028
263 0.00761853903532028
264 0.00761853903532028
265 0.00761853903532028
266 0.00761853903532028
267 0.00761853903532028
268 0.00761853903532028
269 0.00761853903532028
270 0.00761853903532028
271 0.00761853903532028
272 0.00761853903532028
273 0.00761853903532028
274 0.00761853903532028
275 0.00761853903532028
276 0.00761853903532028
277 0.00761853903532028
278 0.00761853903532028
279 0.00761853903532028
280 0.00761853903532028
281 0.00761853903532028
282 0.00761853903532028
283 0.00761853903532028
284 0.00761853903532028
285 0.00761853903532028
286 0.00761853903532028
287 0.00761853903532028
288 0.00761853903532028
289 0.00761853903532028
290 0.00761853903532028
291 0.00761853903532028
292 0.00761853903532028
293 0.00761853903532028
294 0.00761853903532028
295 0.00761853903532028
296 0.00761853903532028
297 0.00761853903532028
298 0.00761853903532028
299 0.00761853903532028
300 0.00761853903532028
301 0.00761853903532028
302 0.00761853903532028
303 0.00761853903532028
304 0.00761853903532028
305 0.00761853903532028
306 0.00761853903532028
307 0.00761853903532028
308 0.00761853903532028
309 0.00761853903532028
310 0.00761853903532028
311 0.00761853903532028
312 0.00761853903532028
313 0.00761853903532028
314 0.00761853903532028
315 0.00761853903532028
316 0.00761853903532028
317 0.00761853903532028
318 0.00761853903532028
319 0.00761853903532028
320 0.00761853903532028
321 0.00761853903532028
322 0.00761853903532028
323 0.00761853903532028
324 0.00761853903532028
325 0.00761853903532028
326 0.00761853903532028
327 0.00761853903532028
328 0.00761853903532028
329 0.00761853903532028
330 0.00761853903532028
331 0.00761853903532028
332 0.00761853903532028
333 0.00761853903532028
334 0.00761853903532028
335 0.00761853903532028
336 0.00761853903532028
337 0.00761853903532028
338 0.00761853903532028
339 0.00761853903532028
340 0.00761853903532028
341 0.00761853903532028
342 0.00761853903532028
343 0.00761853903532028
344 0.00761853903532028
345 0.00761853903532028
346 0.00761853903532028
347 0.00761853903532028
348 0.00761853903532028
349 0.00761853903532028
350 0.00761853903532028
351 0.00761853903532028
352 0.00761853903532028
353 0.00761853903532028
354 0.00761853903532028
355 0.00761853903532028
356 0.00761853903532028
357 0.00761853903532028
358 0.00761853903532028
359 0.00761853903532028
360 0.00761853903532028
361 0.00761853903532028
362 0.00761853903532028
363 0.00761853903532028
364 0.00761853903532028
365 0.00761853903532028
366 0.00761853903532028
367 0.00761853903532028
368 0.00761853903532028
369 0.00761853903532028
370 0.00761853903532028
371 0.00761853903532028
372 0.00761853903532028
373 0.00761853903532028
374 0.00761853903532028
375 0.00761853903532028
376 0.00761853903532028
377 0.00761853903532028
378 0.00761853903532028
379 0.00761853903532028
380 0.00761853903532028
381 0.00761853903532028
382 0.00761853903532028
383 0.00761853903532028
384 0.00761853903532028
385 0.00761853903532028
386 0.00761853903532028
387 0.00761853903532028
388 0.00761853903532028
389 0.00761853903532028
390 0.00761853903532028
391 0.00761853903532028
392 0.00761853903532028
393 0.00761853903532028
394 0.00761853903532028
395 0.00761853903532028
396 0.00761853903532028
397 0.00761853903532028
398 0.00761853903532028
399 0.00761853903532028
400 0.00761853903532028
401 0.00761853903532028
402 0.00761853903532028
403 0.00761853903532028
404 0.00761853903532028
405 0.00761853903532028
406 0.00761853903532028
407 0.00761853903532028
408 0.00761853903532028
409 0.00761853903532028
410 0.00761853903532028
411 0.00761853903532028
412 0.00761853903532028
413 0.00761853903532028
414 0.00761853903532028
415 0.00761853903532028
416 0.00761853903532028
417 0.00761853903532028
418 0.00761853903532028
419 0.00761853903532028
420 0.00761853903532028
421 0.00761853903532028
422 0.00761853903532028
423 0.00761853903532028
424 0.00761853903532028
425 0.00761853903532028
426 0.00761853903532028
427 0.00761853903532028
428 0.00761853903532028
429 0.00761853903532028
430 0.00761853903532028
431 0.00761853903532028
432 0.00761853903532028
433 0.00761853903532028
434 0.00761853903532028
435 0.00761853903532028
436 0.00761853903532028
437 0.00761853903532028
438 0.00761853903532028
439 0.00761853903532028
440 0.00761853903532028
441 0.00761853903532028
442 0.00761853903532028
443 0.00761853903532028
444 0.00761853903532028
445 0.00761853903532028
446 0.00761853903532028
447 0.00761853903532028
448 0.00761853903532028
449 0.00761853903532028
450 0.00761853903532028
451 0.00761853903532028
452 0.00761853903532028
453 0.00761853903532028
454 0.00761853903532028
455 0.00761853903532028
456 0.00761853903532028
457 0.00761853903532028
458 0.00761853903532028
459 0.00761853903532028
460 0.00761853903532028
461 0.00761853903532028
462 0.00761853903532028
463 0.00761853903532028
464 0.00761853903532028
465 0.00761853903532028
466 0.00761853903532028
467 0.00761853903532028
468 0.00761853903532028
469 0.00761853903532028
470 0.00761853903532028
471 0.00761853903532028
472 0.00761853903532028
473 0.00761853903532028
474 0.00761853903532028
475 0.00761853903532028
476 0.00761853903532028
477 0.00761853903532028
478 0.00761853903532028
479 0.00761853903532028
480 0.00761853903532028
481 0.00761853903532028
482 0.00761853903532028
483 0.00761853903532028
484 0.00761853903532028
485 0.00761853903532028
486 0.00761853903532028
487 0.00761853903532028
488 0.00761853903532028
489 0.00761853903532028
490 0.00761853903532028
491 0.00761853903532028
492 0.00761853903532028
493 0.00761853903532028
494 0.00761853903532028
495 0.00761853903532028
496 0.00761853903532028
497 0.00761853903532028
498 0.00761853903532028
499 0.00761853903532028
500 0.00761853903532028
501 0.00761853903532028
502 0.00761853903532028
503 0.00761853903532028
504 0.00761853903532028
505 0.00761853903532028
506 0.00761853903532028
507 0.00761853903532028
508 0.00761853903532028
509 0.00761853903532028
510 0.00761853903532028
511 0.00761853903532028
512 0.00761853903532028
513 0.00761853903532028
514 0.00761853903532028
515 0.00761853903532028
516 0.00761853903532028
517 0.00761853903532028
518 0.00761853903532028
519 0.00761853903532028
520 0.00761853903532028
521 0.00761853903532028
522 0.00761853903532028
523 0.00761853903532028
524 0.00761853903532028
525 0.00761853903532028
526 0.00761853903532028
527 0.00761853903532028
528 0.00761853903532028
529 0.00761853903532028
530 0.00761853903532028
531 0.00761853903532028
532 0.00761853903532028
533 0.00761853903532028
534 0.00761853903532028
535 0.00761853903532028
536 0.00761853903532028
537 0.00761853903532028
538 0.00761853903532028
539 0.00761853903532028
540 0.00761853903532028
541 0.00761853903532028
542 0.00761853903532028
543 0.00761853903532028
544 0.00761853903532028
545 0.00761853903532028
546 0.00761853903532028
547 0.00761853903532028
548 0.00761853903532028
549 0.00761853903532028
550 0.00761853903532028
551 0.00761853903532028
552 0.00761853903532028
553 0.00761853903532028
554 0.00761853903532028
555 0.00761853903532028
556 0.00761853903532028
557 0.00761853903532028
558 0.00761853903532028
559 0.00761853903532028
560 0.00761853903532028
561 0.00761853903532028
562 0.00761853903532028
563 0.00761853903532028
564 0.00761853903532028
565 0.00761853903532028
566 0.00761853903532028
567 0.00761853903532028
568 0.00761853903532028
569 0.00761853903532028
570 0.00761853903532028
571 0.00761853903532028
572 0.00761853903532028
573 0.00761853903532028
574 0.00761853903532028
575 0.00761853903532028
576 0.00761853903532028
577 0.00761853903532028
578 0.00761853903532028
579 0.00761853903532028
580 0.00761853903532028
581 0.00761853903532028
582 0.00761853903532028
583 0.00761853903532028
584 0.00761853903532028
585 0.00761853903532028
586 0.00761853903532028
587 0.00761853903532028
588 0.00761853903532028
589 0.00761853903532028
590 0.00761853903532028
591 0.00761853903532028
592 0.00761853903532028
593 0.00761853903532028
594 0.00761853903532028
595 0.00761853903532028
596 0.00761853903532028
597 0.00761853903532028
598 0.00761853903532028
599 0.00761853903532028
600 0.00761853903532028
601 0.00761853903532028
602 0.00761853903532028
603 0.00761853903532028
604 0.00761853903532028
605 0.00761853903532028
606 0.00761853903532028
607 0.00761853903532028
608 0.00761853903532028
609 0.00761853903532028
610 0.00761853903532028
611 0.00761853903532028
612 0.00761853903532028
613 0.00761853903532028
614 0.00761853903532028
615 0.00761853903532028
616 0.00761853903532028
617 0.00761853903532028
618 0.00761853903532028
619 0.00761853903532028
620 0.00761853903532028
621 0.00761853903532028
622 0.00761853903532028
623 0.00761853903532028
624 0.00761853903532028
625 0.00761853903532028
626 0.00761853903532028
627 0.00761853903532028
628 0.00761853903532028
629 0.00761853903532028
630 0.00761853903532028
631 0.00761853903532028
632 0.00761853903532028
633 0.00761853903532028
634 0.00761853903532028
635 0.00761853903532028
636 0.00761853903532028
637 0.00761853903532028
638 0.00761853903532028
639 0.00761853903532028
640 0.00761853903532028
641 0.00761853903532028
642 0.00761853903532028
643 0.00761853903532028
644 0.00761853903532028
645 0.00761853903532028
646 0.00761853903532028
647 0.00761853903532028
648 0.00761853903532028
649 0.00761853903532028
650 0.00761853903532028
651 0.00761853903532028
652 0.00761853903532028
653 0.00761853903532028
654 0.00761853903532028
655 0.00761853903532028
656 0.00761853903532028
657 0.00761853903532028
658 0.00761853903532028
659 0.00761853903532028
660 0.00761853903532028
661 0.00761853903532028
662 0.00761853903532028
663 0.00761853903532028
664 0.00761853903532028
665 0.00761853903532028
666 0.00761853903532028
667 0.00761853903532028
668 0.00761853903532028
669 0.00761853903532028
670 0.00761853903532028
671 0.00761853903532028
672 0.00761853903532028
673 0.00761853903532028
674 0.00761853903532028
675 0.00761853903532028
676 0.00761853903532028
677 0.00761853903532028
678 0.00761853903532028
679 0.00761853903532028
680 0.00761853903532028
681 0.00761853903532028
682 0.00761853903532028
683 0.00761853903532028
684 0.00761853903532028
685 0.00761853903532028
686 0.00761853903532028
687 0.00761853903532028
688 0.00761853903532028
689 0.00761853903532028
690 0.00761853903532028
691 0.00761853903532028
692 0.00761853903532028
693 0.00761853903532028
694 0.00761853903532028
695 0.00761853903532028
696 0.00761853903532028
697 0.00761853903532028
698 0.00761853903532028
699 0.00761853903532028
700 0.00761853903532028
701 0.00761853903532028
702 0.00761853903532028
703 0.00761853903532028
704 0.00761853903532028
705 0.00761853903532028
706 0.00761853903532028
707 0.00761853903532028
708 0.00761853903532028
709 0.00761853903532028
710 0.00761853903532028
711 0.00761853903532028
712 0.00761853903532028
713 0.00761853903532028
714 0.00761853903532028
715 0.00761853903532028
716 0.00761853903532028
717 0.00761853903532028
718 0.00761853903532028
719 0.00761853903532028
720 0.00761853903532028
721 0.00761853903532028
722 0.00761853903532028
723 0.00761853903532028
724 0.00761853903532028
725 0.00761853903532028
726 0.00761853903532028
727 0.00761853903532028
728 0.00761853903532028
729 0.00761853903532028
730 0.00761853903532028
731 0.00761853903532028
732 0.00761853903532028
733 0.00761853903532028
734 0.00761853903532028
735 0.00761853903532028
736 0.00761853903532028
737 0.00761853903532028
738 0.00761853903532028
739 0.00761853903532028
740 0.00761853903532028
741 0.00761853903532028
742 0.00761853903532028
743 0.00761853903532028
744 0.00761853903532028
745 0.00761853903532028
746 0.00761853903532028
747 0.00761853903532028
748 0.00761853903532028
749 0.00761853903532028
750 0.00761853903532028
751 0.00761853903532028
752 0.00761853903532028
753 0.00761853903532028
754 0.00761853903532028
755 0.00761853903532028
756 0.00761853903532028
757 0.00761853903532028
758 0.00761853903532028
759 0.00761853903532028
760 0.00761853903532028
761 0.00761853903532028
762 0.00761853903532028
763 0.00761853903532028
764 0.00761853903532028
765 0.00761853903532028
766 0.00761853903532028
767 0.00761853903532028
768 0.00761853903532028
769 0.00761853903532028
770 0.00761853903532028
771 0.00761853903532028
772 0.00761853903532028
773 0.00761853903532028
774 0.00761853903532028
775 0.00761853903532028
776 0.00761853903532028
777 0.00761853903532028
778 0.00761853903532028
779 0.00761853903532028
780 0.00761853903532028
781 0.00761853903532028
782 0.00761853903532028
783 0.00761853903532028
784 0.00761853903532028
785 0.00761853903532028
786 0.00761853903532028
787 0.00761853903532028
788 0.00761853903532028
789 0.00761853903532028
790 0.00761853903532028
791 0.00761853903532028
792 0.00761853903532028
793 0.00761853903532028
794 0.00761853903532028
795 0.00761853903532028
796 0.00761853903532028
797 0.00761853903532028
798 0.00761853903532028
799 0.00761853903532028
800 0.00761853903532028
801 0.00761853903532028
802 0.00761853903532028
803 0.00761853903532028
804 0.00761853903532028
805 0.00761853903532028
806 0.00761853903532028
807 0.00761853903532028
808 0.00761853903532028
809 0.00761853903532028
810 0.00761853903532028
811 0.00761853903532028
812 0.00761853903532028
813 0.00761853903532028
814 0.00761853903532028
815 0.00761853903532028
816 0.00761853903532028
817 0.00761853903532028
818 0.00761853903532028
819 0.00761853903532028
820 0.00761853903532028
821 0.00761853903532028
822 0.00761853903532028
823 0.00761853903532028
824 0.00761853903532028
825 0.00761853903532028
826 0.00761853903532028
827 0.00761853903532028
828 0.00761853903532028
829 0.00761853903532028
830 0.00761853903532028
831 0.00761853903532028
832 0.00761853903532028
833 0.00761853903532028
834 0.00761853903532028
835 0.00761853903532028
836 0.00761853903532028
837 0.00761853903532028
838 0.00761853903532028
839 0.00761853903532028
840 0.00761853903532028
841 0.00761853903532028
842 0.00761853903532028
843 0.00761853903532028
844 0.00761853903532028
845 0.00761853903532028
846 0.00761853903532028
847 0.00761853903532028
848 0.00761853903532028
849 0.00761853903532028
850 0.00761853903532028
851 0.00761853903532028
852 0.00761853903532028
853 0.00761853903532028
854 0.00761853903532028
855 0.00761853903532028
856 0.00761853903532028
857 0.00761853903532028
858 0.00761853903532028
859 0.00761853903532028
860 0.00761853903532028
861 0.00761853903532028
862 0.00761853903532028
863 0.00761853903532028
864 0.00761853903532028
865 0.00761853903532028
866 0.00761853903532028
867 0.00761853903532028
868 0.00761853903532028
869 0.00761853903532028
870 0.00761853903532028
871 0.00761853903532028
872 0.00761853903532028
873 0.00761853903532028
874 0.00761853903532028
875 0.00761853903532028
876 0.00761853903532028
877 0.00761853903532028
878 0.00761853903532028
879 0.00761853903532028
880 0.00761853903532028
881 0.00761853903532028
882 0.00761853903532028
883 0.00761853903532028
884 0.00761853903532028
885 0.00761853903532028
886 0.00761853903532028
887 0.00761853903532028
888 0.00761853903532028
889 0.00761853903532028
890 0.00761853903532028
891 0.00761853903532028
892 0.00761853903532028
893 0.00761853903532028
894 0.00761853903532028
895 0.00761853903532028
896 0.00761853903532028
897 0.00761853903532028
898 0.00761853903532028
899 0.00761853903532028
900 0.00761853903532028
901 0.00761853903532028
902 0.00761853903532028
903 0.00761853903532028
904 0.00761853903532028
905 0.00761853903532028
906 0.00761853903532028
907 0.00761853903532028
908 0.00761853903532028
909 0.00761853903532028
910 0.00761853903532028
911 0.00761853903532028
912 0.00761853903532028
913 0.00761853903532028
914 0.00761853903532028
915 0.00761853903532028
916 0.00761853903532028
917 0.00761853903532028
918 0.00761853903532028
919 0.00761853903532028
920 0.00761853903532028
921 0.00761853903532028
922 0.00761853903532028
923 0.00761853903532028
924 0.00761853903532028
925 0.00761853903532028
926 0.00761853903532028
927 0.00761853903532028
928 0.00761853903532028
929 0.00761853903532028
930 0.00761853903532028
931 0.00761853903532028
932 0.00761853903532028
933 0.00761853903532028
934 0.00761853903532028
935 0.00761853903532028
936 0.00761853903532028
937 0.00761853903532028
938 0.00761853903532028
939 0.00761853903532028
940 0.00761853903532028
941 0.00761853903532028
942 0.00761853903532028
943 0.00761853903532028
944 0.00761853903532028
945 0.00761853903532028
946 0.00761853903532028
947 0.00761853903532028
948 0.00761853903532028
949 0.00761853903532028
950 0.00761853903532028
951 0.00761853903532028
952 0.00761853903532028
953 0.00761853903532028
954 0.00761853903532028
955 0.00761853903532028
956 0.00761853903532028
957 0.00761853903532028
958 0.00761853903532028
959 0.00761853903532028
960 0.00761853903532028
961 0.00761853903532028
962 0.00761853903532028
963 0.00761853903532028
964 0.00761853903532028
965 0.00761853903532028
966 0.00761853903532028
967 0.00761853903532028
968 0.00761853903532028
969 0.00761853903532028
970 0.00761853903532028
971 0.00761853903532028
972 0.00761853903532028
973 0.00761853903532028
974 0.00761853903532028
975 0.00761853903532028
976 0.00761853903532028
977 0.00761853903532028
978 0.00761853903532028
979 0.00761853903532028
980 0.00761853903532028
981 0.00761853903532028
982 0.00761853903532028
983 0.00761853903532028
984 0.00761853903532028
985 0.00761853903532028
986 0.00761853903532028
987 0.00761853903532028
988 0.00761853903532028
989 0.00761853903532028
990 0.00761853903532028
991 0.00761853903532028
992 0.00761853903532028
993 0.00761853903532028
994 0.00761853903532028
995 0.00761853903532028
996 0.00761853903532028
997 0.00761853903532028
998 0.00761853903532028
999 0.00761853903532028
1000 0.00761853903532028
1001 0.00761853903532028
1002 0.00761853903532028
1003 0.00761853903532028
1004 0.00761853903532028
1005 0.00761853903532028
1006 0.00761853903532028
1007 0.00761853903532028
1008 0.00761853903532028
1009 0.00761853903532028
1010 0.00761853903532028
1011 0.00761853903532028
1012 0.00761853903532028
1013 0.00761853903532028
1014 0.00761853903532028
1015 0.00761853903532028
1016 0.00761853903532028
1017 0.00761853903532028
1018 0.00761853903532028
1019 0.00761853903532028
1020 0.00761853903532028
1021 0.00761853903532028
1022 0.00761853903532028
1023 0.00761853903532028
1024 0.00761853903532028
1025 0.00761853903532028
1026 0.00761853903532028
1027 0.00761853903532028
1028 0.00761853903532028
1029 0.00761853903532028
1030 0.00761853903532028
1031 0.00761853903532028
1032 0.00761853903532028
1033 0.00761853903532028
1034 0.00761853903532028
1035 0.00761853903532028
1036 0.00761853903532028
1037 0.00761853903532028
1038 0.00761853903532028
1039 0.00761853903532028
1040 0.00761853903532028
1041 0.00761853903532028
1042 0.00761853903532028
1043 0.00761853903532028
1044 0.00761853903532028
1045 0.00761853903532028
1046 0.00761853903532028
1047 0.00761853903532028
1048 0.00761853903532028
1049 0.00761853903532028
1050 0.00761853903532028
1051 0.00761853903532028
1052 0.00761853903532028
1053 0.00761853903532028
1054 0.00761853903532028
1055 0.00761853903532028
1056 0.00761853903532028
1057 0.00761853903532028
1058 0.00761853903532028
1059 0.00761853903532028
1060 0.00761853903532028
1061 0.00761853903532028
1062 0.00761853903532028
1063 0.00761853903532028
1064 0.00761853903532028
1065 0.00761853903532028
1066 0.00761853903532028
1067 0.00761853903532028
1068 0.00761853903532028
1069 0.00761853903532028
1070 0.00761853903532028
1071 0.00761853903532028
1072 0.00761853903532028
1073 0.00761853903532028
1074 0.00761853903532028
1075 0.00761853903532028
1076 0.00761853903532028
1077 0.00761853903532028
1078 0.00761853903532028
1079 0.00761853903532028
1080 0.00761853903532028
1081 0.00761853903532028
1082 0.00761853903532028
1083 0.00761853903532028
1084 0.00761853903532028
1085 0.00761853903532028
1086 0.00761853903532028
1087 0.00761853903532028
1088 0.00761853903532028
1089 0.00761853903532028
1090 0.00761853903532028
1091 0.00761853903532028
1092 0.00761853903532028
1093 0.00761853903532028
1094 0.00761853903532028
1095 0.00761853903532028
1096 0.00761853903532028
1097 0.00761853903532028
1098 0.00761853903532028
1099 0.00761853903532028
1100 0.00761853903532028
1101 0.00761853903532028
1102 0.00761853903532028
1103 0.00761853903532028
1104 0.00761853903532028
1105 0.00761853903532028
1106 0.00761853903532028
1107 0.00761853903532028
1108 0.00761853903532028
1109 0.00761853903532028
1110 0.00761853903532028
1111 0.00761853903532028
1112 0.00761853903532028
1113 0.00761853903532028
1114 0.00761853903532028
1115 0.00761853903532028
1116 0.00761853903532028
1117 0.00761853903532028
1118 0.00761853903532028
1119 0.00761853903532028
1120 0.00761853903532028
1121 0.00761853903532028
1122 0.00761853903532028
1123 0.00761853903532028
1124 0.00761853903532028
1125 0.00761853903532028
1126 0.00761853903532028
1127 0.00761853903532028
1128 0.00761853903532028
1129 0.00761853903532028
1130 0.00761853903532028
1131 0.00761853903532028
1132 0.00761853903532028
1133 0.00761853903532028
1134 0.00761853903532028
1135 0.00761853903532028
1136 0.00761853903532028
1137 0.00761853903532028
1138 0.00761853903532028
1139 0.00761853903532028
1140 0.00761853903532028
1141 0.00761853903532028
1142 0.00761853903532028
1143 0.00761853903532028
1144 0.00761853903532028
1145 0.00761853903532028
1146 0.00761853903532028
1147 0.00761853903532028
1148 0.00761853903532028
1149 0.00761853903532028
1150 0.00761853903532028
1151 0.00761853903532028
1152 0.00761853903532028
1153 0.00761853903532028
1154 0.00761853903532028
1155 0.00761853903532028
1156 0.00761853903532028
1157 0.00761853903532028
1158 0.00761853903532028
1159 0.00761853903532028
1160 0.00761853903532028
1161 0.00761853903532028
1162 0.00761853903532028
1163 0.00761853903532028
1164 0.00761853903532028
1165 0.00761853903532028
1166 0.00761853903532028
1167 0.00761853903532028
1168 0.00761853903532028
1169 0.00761853903532028
1170 0.00761853903532028
1171 0.00761853903532028
1172 0.00761853903532028
1173 0.00761853903532028
1174 0.00761853903532028
1175 0.00761853903532028
1176 0.00761853903532028
1177 0.00761853903532028
1178 0.00761853903532028
1179 0.00761853903532028
1180 0.00761853903532028
1181 0.00761853903532028
1182 0.00761853903532028
1183 0.00761853903532028
1184 0.00761853903532028
1185 0.00761853903532028
1186 0.00761853903532028
1187 0.00761853903532028
1188 0.00761853903532028
1189 0.00761853903532028
1190 0.00761853903532028
1191 0.00761853903532028
1192 0.00761853903532028
1193 0.00761853903532028
1194 0.00761853903532028
1195 0.00761853903532028
1196 0.00761853903532028
1197 0.00761853903532028
1198 0.00761853903532028
1199 0.00761853903532028
1200 0.00761853903532028
1201 0.00761853903532028
1202 0.00761853903532028
1203 0.00761853903532028
1204 0.00761853903532028
1205 0.00761853903532028
1206 0.00761853903532028
1207 0.00761853903532028
1208 0.00761853903532028
1209 0.00761853903532028
1210 0.00761853903532028
1211 0.00761853903532028
1212 0.00761853903532028
1213 0.00761853903532028
1214 0.00761853903532028
1215 0.00761853903532028
1216 0.00761853903532028
1217 0.00761853903532028
1218 0.00761853903532028
1219 0.00761853903532028
1220 0.00761853903532028
1221 0.00761853903532028
1222 0.00761853903532028
1223 0.00761853903532028
1224 0.00761853903532028
1225 0.00761853903532028
1226 0.00761853903532028
1227 0.00761853903532028
1228 0.00761853903532028
1229 0.00761853903532028
1230 0.00761853903532028
1231 0.00761853903532028
1232 0.00761853903532028
1233 0.00761853903532028
1234 0.00761853903532028
1235 0.00761853903532028
1236 0.00761853903532028
1237 0.00761853903532028
1238 0.00761853903532028
1239 0.00761853903532028
1240 0.00761853903532028
1241 0.00761853903532028
1242 0.00761853903532028
1243 0.00761853903532028
1244 0.00761853903532028
1245 0.00761853903532028
1246 0.00761853903532028
1247 0.00761853903532028
1248 0.00761853903532028
1249 0.00761853903532028
1250 0.00761853903532028
1251 0.00761853903532028
1252 0.00761853903532028
1253 0.00761853903532028
1254 0.00761853903532028
1255 0.00761853903532028
1256 0.00761853903532028
1257 0.00761853903532028
1258 0.00761853903532028
1259 0.00761853903532028
1260 0.00761853903532028
1261 0.00761853903532028
1262 0.00761853903532028
1263 0.00761853903532028
1264 0.00761853903532028
1265 0.00761853903532028
1266 0.00761853903532028
1267 0.00761853903532028
1268 0.00761853903532028
1269 0.00761853903532028
1270 0.00761853903532028
1271 0.00761853903532028
1272 0.00761853903532028
1273 0.00761853903532028
1274 0.00761853903532028
1275 0.00761853903532028
1276 0.00761853903532028
1277 0.00761853903532028
1278 0.00761853903532028
1279 0.00761853903532028
1280 0.00761853903532028
1281 0.00761853903532028
1282 0.00761853903532028
1283 0.00761853903532028
1284 0.00761853903532028
1285 0.00761853903532028
1286 0.00761853903532028
1287 0.00761853903532028
1288 0.00761853903532028
1289 0.00761853903532028
1290 0.00761853903532028
1291 0.00761853903532028
1292 0.00761853903532028
1293 0.00761853903532028
1294 0.00761853903532028
1295 0.00761853903532028
1296 0.00761853903532028
1297 0.00761853903532028
1298 0.00761853903532028
1299 0.00761853903532028
1300 0.00761853903532028
1301 0.00761853903532028
1302 0.00761853903532028
1303 0.00761853903532028
1304 0.00761853903532028
1305 0.00761853903532028
1306 0.00761853903532028
1307 0.00761853903532028
1308 0.00761853903532028
1309 0.00761853903532028
1310 0.00761853903532028
1311 0.00761853903532028
1312 0.00761853903532028
1313 0.00761853903532028
1314 0.00761853903532028
1315 0.00761853903532028
1316 0.00761853903532028
1317 0.00761853903532028
1318 0.00761853903532028
1319 0.00761853903532028
1320 0.00761853903532028
1321 0.00761853903532028
1322 0.00761853903532028
1323 0.00761853903532028
1324 0.00761853903532028
1325 0.00761853903532028
1326 0.00761853903532028
1327 0.00761853903532028
1328 0.00761853903532028
1329 0.00761853903532028
1330 0.00761853903532028
1331 0.00761853903532028
1332 0.00761853903532028
1333 0.00761853903532028
1334 0.00761853903532028
1335 0.00761853903532028
1336 0.00761853903532028
1337 0.00761853903532028
1338 0.00761853903532028
1339 0.00761853903532028
1340 0.00761853903532028
1341 0.00761853903532028
1342 0.00761853903532028
1343 0.00761853903532028
1344 0.00761853903532028
1345 0.00761853903532028
1346 0.00761853903532028
1347 0.00761853903532028
1348 0.00761853903532028
1349 0.00761853903532028
1350 0.00761853903532028
1351 0.00761853903532028
1352 0.00761853903532028
1353 0.00761853903532028
1354 0.00761853903532028
1355 0.00761853903532028
1356 0.00761853903532028
1357 0.00761853903532028
1358 0.00761853903532028
1359 0.00761853903532028
1360 0.00761853903532028
1361 0.00761853903532028
1362 0.00761853903532028
1363 0.00761853903532028
1364 0.00761853903532028
1365 0.00761853903532028
1366 0.00761853903532028
1367 0.00761853903532028
1368 0.00761853903532028
1369 0.00761853903532028
1370 0.00761853903532028
1371 0.00761853903532028
1372 0.00761853903532028
1373 0.00761853903532028
1374 0.00761853903532028
1375 0.00761853903532028
1376 0.00761853903532028
1377 0.00761853903532028
1378 0.00761853903532028
1379 0.00761853903532028
1380 0.00761853903532028
1381 0.00761853903532028
1382 0.00761853903532028
1383 0.00761853903532028
1384 0.00761853903532028
1385 0.00761853903532028
1386 0.00761853903532028
1387 0.00761853903532028
1388 0.00761853903532028
1389 0.00761853903532028
1390 0.00761853903532028
1391 0.00761853903532028
1392 0.00761853903532028
1393 0.00761853903532028
1394 0.00761853903532028
1395 0.00761853903532028
1396 0.00761853903532028
1397 0.00761853903532028
1398 0.00761853903532028
1399 0.00761853903532028
1400 0.00761853903532028
1401 0.00761853903532028
1402 0.00761853903532028
1403 0.00761853903532028
1404 0.00761853903532028
1405 0.00761853903532028
1406 0.00761853903532028
1407 0.00761853903532028
1408 0.00761853903532028
1409 0.00761853903532028
1410 0.00761853903532028
1411 0.00761853903532028
1412 0.00761853903532028
1413 0.00761853903532028
1414 0.00761853903532028
1415 0.00761853903532028
1416 0.00761853903532028
1417 0.00761853903532028
1418 0.00761853903532028
1419 0.00761853903532028
1420 0.00761853903532028
1421 0.00761853903532028
1422 0.00761853903532028
1423 0.00761853903532028
1424 0.00761853903532028
1425 0.00761853903532028
1426 0.00761853903532028
1427 0.00761853903532028
1428 0.00761853903532028
1429 0.00761853903532028
1430 0.00761853903532028
1431 0.00761853903532028
1432 0.00761853903532028
1433 0.00761853903532028
1434 0.00761853903532028
1435 0.00761853903532028
1436 0.00761853903532028
1437 0.00761853903532028
1438 0.00761853903532028
1439 0.00761853903532028
1440 0.00761853903532028
1441 0.00761853903532028
1442 0.00761853903532028
1443 0.00761853903532028
1444 0.00761853903532028
1445 0.00761853903532028
1446 0.00761853903532028
1447 0.00761853903532028
1448 0.00761853903532028
1449 0.00761853903532028
1450 0.00761853903532028
1451 0.00761853903532028
1452 0.00761853903532028
1453 0.00761853903532028
1454 0.00761853903532028
1455 0.00761853903532028
1456 0.00761853903532028
1457 0.00761853903532028
1458 0.00761853903532028
1459 0.00761853903532028
1460 0.00761853903532028
1461 0.00761853903532028
1462 0.00761853903532028
1463 0.00761853903532028
1464 0.00761853903532028
1465 0.00761853903532028
1466 0.00761853903532028
1467 0.00761853903532028
1468 0.00761853903532028
1469 0.00761853903532028
1470 0.00761853903532028
1471 0.00761853903532028
1472 0.00761853903532028
1473 0.00761853903532028
1474 0.00761853903532028
1475 0.00761853903532028
1476 0.00761853903532028
1477 0.00761853903532028
1478 0.00761853903532028
1479 0.00761853903532028
1480 0.00761853903532028
1481 0.00761853903532028
1482 0.00761853903532028
1483 0.00761853903532028
1484 0.00761853903532028
1485 0.00761853903532028
1486 0.00761853903532028
1487 0.00761853903532028
1488 0.00761853903532028
1489 0.00761853903532028
1490 0.00761853903532028
1491 0.00761853903532028
1492 0.00761853903532028
1493 0.00761853903532028
1494 0.00761853903532028
1495 0.00761853903532028
1496 0.00761853903532028
1497 0.00761853903532028
1498 0.00761853903532028
1499 0.00761853903532028
1500 0.00761853903532028
1501 0.00761853903532028
1502 0.00761853903532028
1503 0.00761853903532028
1504 0.00761853903532028
1505 0.00761853903532028
1506 0.00761853903532028
1507 0.00761853903532028
1508 0.00761853903532028
1509 0.00761853903532028
1510 0.00761853903532028
1511 0.00761853903532028
1512 0.00761853903532028
1513 0.00761853903532028
1514 0.00761853903532028
1515 0.00761853903532028
1516 0.00761853903532028
1517 0.00761853903532028
1518 0.00761853903532028
1519 0.00761853903532028
1520 0.00761853903532028
1521 0.00761853903532028
1522 0.00761853903532028
1523 0.00761853903532028
1524 0.00761853903532028
1525 0.00761853903532028
1526 0.00761853903532028
1527 0.00761853903532028
1528 0.00761853903532028
1529 0.00761853903532028
1530 0.00761853903532028
1531 0.00761853903532028
1532 0.00761853903532028
1533 0.00761853903532028
1534 0.00761853903532028
1535 0.00761853903532028
1536 0.00761853903532028
1537 0.00761853903532028
1538 0.00761853903532028
1539 0.00761853903532028
1540 0.00761853903532028
1541 0.00761853903532028
1542 0.00761853903532028
1543 0.00761853903532028
1544 0.00761853903532028
1545 0.00761853903532028
1546 0.00761853903532028
1547 0.00761853903532028
1548 0.00761853903532028
1549 0.00761853903532028
1550 0.00761853903532028
1551 0.00761853903532028
1552 0.00761853903532028
1553 0.00761853903532028
1554 0.00761853903532028
1555 0.00761853903532028
1556 0.00761853903532028
1557 0.00761853903532028
1558 0.00761853903532028
1559 0.00761853903532028
1560 0.00761853903532028
1561 0.00761853903532028
1562 0.00761853903532028
1563 0.00761853903532028
1564 0.00761853903532028
1565 0.00761853903532028
1566 0.00761853903532028
1567 0.00761853903532028
1568 0.00761853903532028
1569 0.00761853903532028
1570 0.00761853903532028
1571 0.00761853903532028
1572 0.00761853903532028
1573 0.00761853903532028
1574 0.00761853903532028
1575 0.00761853903532028
1576 0.00761853903532028
1577 0.00761853903532028
1578 0.00761853903532028
1579 0.00761853903532028
1580 0.00761853903532028
1581 0.00761853903532028
1582 0.00761853903532028
1583 0.00761853903532028
1584 0.00761853903532028
1585 0.00761853903532028
1586 0.00761853903532028
1587 0.00761853903532028
1588 0.00761853903532028
1589 0.00761853903532028
1590 0.00761853903532028
1591 0.00761853903532028
1592 0.00761853903532028
1593 0.00761853903532028
1594 0.00761853903532028
1595 0.00761853903532028
1596 0.00761853903532028
1597 0.00761853903532028
1598 0.00761853903532028
1599 0.00761853903532028
1600 0.00761853903532028
1601 0.00761853903532028
1602 0.00761853903532028
1603 0.00761853903532028
1604 0.00761853903532028
1605 0.00761853903532028
1606 0.00761853903532028
1607 0.00761853903532028
1608 0.00761853903532028
1609 0.00761853903532028
1610 0.00761853903532028
1611 0.00761853903532028
1612 0.00761853903532028
1613 0.00761853903532028
1614 0.00761853903532028
1615 0.00761853903532028
1616 0.00761853903532028
1617 0.00761853903532028
1618 0.00761853903532028
1619 0.00761853903532028
1620 0.00761853903532028
1621 0.00761853903532028
1622 0.00761853903532028
1623 0.00761853903532028
1624 0.00761853903532028
1625 0.00761853903532028
1626 0.00761853903532028
1627 0.00761853903532028
1628 0.00761853903532028
1629 0.00761853903532028
1630 0.00761853903532028
1631 0.00761853903532028
1632 0.00761853903532028
1633 0.00761853903532028
1634 0.00761853903532028
1635 0.00761853903532028
1636 0.00761853903532028
1637 0.00761853903532028
1638 0.00761853903532028
1639 0.00761853903532028
1640 0.00761853903532028
1641 0.00761853903532028
1642 0.00761853903532028
1643 0.00761853903532028
1644 0.00761853903532028
1645 0.00761853903532028
1646 0.00761853903532028
1647 0.00761853903532028
1648 0.00761853903532028
1649 0.00761853903532028
1650 0.00761853903532028
1651 0.00761853903532028
1652 0.00761853903532028
1653 0.00761853903532028
1654 0.00761853903532028
1655 0.00761853903532028
1656 0.00761853903532028
1657 0.00761853903532028
1658 0.00761853903532028
1659 0.00761853903532028
1660 0.00761853903532028
1661 0.00761853903532028
1662 0.00761853903532028
1663 0.00761853903532028
1664 0.00761853903532028
1665 0.00761853903532028
1666 0.00761853903532028
1667 0.00761853903532028
1668 0.00761853903532028
1669 0.00761853903532028
1670 0.00761853903532028
1671 0.00761853903532028
1672 0.00761853903532028
1673 0.00761853903532028
1674 0.00761853903532028
1675 0.00761853903532028
1676 0.00761853903532028
1677 0.00761853903532028
1678 0.00761853903532028
1679 0.00761853903532028
1680 0.00761853903532028
1681 0.00761853903532028
1682 0.00761853903532028
1683 0.00761853903532028
1684 0.00761853903532028
1685 0.00761853903532028
1686 0.00761853903532028
1687 0.00761853903532028
1688 0.00761853903532028
1689 0.00761853903532028
1690 0.00761853903532028
1691 0.00761853903532028
1692 0.00761853903532028
1693 0.00761853903532028
1694 0.00761853903532028
1695 0.00761853903532028
1696 0.00761853903532028
1697 0.00761853903532028
1698 0.00761853903532028
1699 0.00761853903532028
1700 0.00761853903532028
1701 0.00761853903532028
1702 0.00761853903532028
1703 0.00761853903532028
1704 0.00761853903532028
1705 0.00761853903532028
1706 0.00761853903532028
1707 0.00761853903532028
1708 0.00761853903532028
1709 0.00761853903532028
1710 0.00761853903532028
1711 0.00761853903532028
1712 0.00761853903532028
1713 0.00761853903532028
1714 0.00761853903532028
1715 0.00761853903532028
1716 0.00761853903532028
1717 0.00761853903532028
1718 0.00761853903532028
1719 0.00761853903532028
1720 0.00761853903532028
1721 0.00761853903532028
1722 0.00761853903532028
1723 0.00761853903532028
1724 0.00761853903532028
1725 0.00761853903532028
1726 0.00761853903532028
1727 0.00761853903532028
1728 0.00761853903532028
1729 0.00761853903532028
1730 0.00761853903532028
1731 0.00761853903532028
1732 0.00761853903532028
1733 0.00761853903532028
1734 0.00761853903532028
1735 0.00761853903532028
1736 0.00761853903532028
1737 0.00761853903532028
1738 0.00761853903532028
1739 0.00761853903532028
1740 0.00761853903532028
1741 0.00761853903532028
1742 0.00761853903532028
1743 0.00761853903532028
1744 0.00761853903532028
1745 0.00761853903532028
1746 0.00761853903532028
1747 0.00761853903532028
1748 0.00761853903532028
1749 0.00761853903532028
1750 0.00761853903532028
1751 0.00761853903532028
1752 0.00761853903532028
1753 0.00761853903532028
1754 0.00761853903532028
1755 0.00761853903532028
1756 0.00761853903532028
1757 0.00761853903532028
1758 0.00761853903532028
1759 0.00761853903532028
1760 0.00761853903532028
1761 0.00761853903532028
1762 0.00761853903532028
1763 0.00761853903532028
1764 0.00761853903532028
1765 0.00761853903532028
1766 0.00761853903532028
1767 0.00761853903532028
1768 0.00761853903532028
1769 0.00761853903532028
1770 0.00761853903532028
1771 0.00761853903532028
1772 0.00761853903532028
1773 0.00761853903532028
1774 0.00761853903532028
1775 0.00761853903532028
1776 0.00761853903532028
1777 0.00761853903532028
1778 0.00761853903532028
1779 0.00761853903532028
1780 0.00761853903532028
1781 0.00761853903532028
1782 0.00761853903532028
1783 0.00761853903532028
1784 0.00761853903532028
1785 0.00761853903532028
1786 0.00761853903532028
1787 0.00761853903532028
1788 0.00761853903532028
1789 0.00761853903532028
1790 0.00761853903532028
1791 0.00761853903532028
1792 0.00761853903532028
1793 0.00761853903532028
1794 0.00761853903532028
1795 0.00761853903532028
1796 0.00761853903532028
1797 0.00761853903532028
1798 0.00761853903532028
1799 0.00761853903532028
1800 0.00761853903532028
1801 0.00761853903532028
1802 0.00761853903532028
1803 0.00761853903532028
1804 0.00761853903532028
1805 0.00761853903532028
1806 0.00761853903532028
1807 0.00761853903532028
1808 0.00761853903532028
1809 0.00761853903532028
1810 0.00761853903532028
1811 0.00761853903532028
1812 0.00761853903532028
1813 0.00761853903532028
1814 0.00761853903532028
1815 0.00761853903532028
1816 0.00761853903532028
1817 0.00761853903532028
1818 0.00761853903532028
1819 0.00761853903532028
1820 0.00761853903532028
1821 0.00761853903532028
1822 0.00761853903532028
1823 0.00761853903532028
1824 0.00761853903532028
1825 0.00761853903532028
1826 0.00761853903532028
1827 0.00761853903532028
1828 0.00761853903532028
1829 0.00761853903532028
1830 0.00761853903532028
1831 0.00761853903532028
1832 0.00761853903532028
1833 0.00761853903532028
1834 0.00761853903532028
1835 0.00761853903532028
1836 0.00761853903532028
1837 0.00761853903532028
1838 0.00761853903532028
1839 0.00761853903532028
1840 0.00761853903532028
1841 0.00761853903532028
1842 0.00761853903532028
1843 0.00761853903532028
1844 0.00761853903532028
1845 0.00761853903532028
1846 0.00761853903532028
1847 0.00761853903532028
1848 0.00761853903532028
1849 0.00761853903532028
1850 0.00761853903532028
1851 0.00761853903532028
1852 0.00761853903532028
1853 0.00761853903532028
1854 0.00761853903532028
1855 0.00761853903532028
1856 0.00761853903532028
1857 0.00761853903532028
1858 0.00761853903532028
1859 0.00761853903532028
1860 0.00761853903532028
1861 0.00761853903532028
1862 0.00761853903532028
1863 0.00761853903532028
1864 0.00761853903532028
1865 0.00761853903532028
1866 0.00761853903532028
1867 0.00761853903532028
1868 0.00761853903532028
1869 0.00761853903532028
1870 0.00761853903532028
1871 0.00761853903532028
1872 0.00761853903532028
1873 0.00761853903532028
1874 0.00761853903532028
1875 0.00761853903532028
1876 0.00761853903532028
1877 0.00761853903532028
1878 0.00761853903532028
1879 0.00761853903532028
1880 0.00761853903532028
1881 0.00761853903532028
1882 0.00761853903532028
1883 0.00761853903532028
1884 0.00761853903532028
1885 0.00761853903532028
1886 0.00761853903532028
1887 0.00761853903532028
1888 0.00761853903532028
1889 0.00761853903532028
1890 0.00761853903532028
1891 0.00761853903532028
1892 0.00761853903532028
1893 0.00761853903532028
1894 0.00761853903532028
1895 0.00761853903532028
1896 0.00761853903532028
1897 0.00761853903532028
1898 0.00761853903532028
1899 0.00761853903532028
1900 0.00761853903532028
1901 0.00761853903532028
1902 0.00761853903532028
1903 0.00761853903532028
1904 0.00761853903532028
1905 0.00761853903532028
1906 0.00761853903532028
1907 0.00761853903532028
1908 0.00761853903532028
1909 0.00761853903532028
1910 0.00761853903532028
1911 0.00761853903532028
1912 0.00761853903532028
1913 0.00761853903532028
1914 0.00761853903532028
1915 0.00761853903532028
1916 0.00761853903532028
1917 0.00761853903532028
1918 0.00761853903532028
1919 0.00761853903532028
1920 0.00761853903532028
1921 0.00761853903532028
1922 0.00761853903532028
1923 0.00761853903532028
1924 0.00761853903532028
1925 0.00761853903532028
1926 0.00761853903532028
1927 0.00761853903532028
1928 0.00761853903532028
1929 0.00761853903532028
1930 0.00761853903532028
1931 0.00761853903532028
1932 0.00761853903532028
1933 0.00761853903532028
1934 0.00761853903532028
1935 0.00761853903532028
1936 0.00761853903532028
1937 0.00761853903532028
1938 0.00761853903532028
1939 0.00761853903532028
1940 0.00761853903532028
1941 0.00761853903532028
1942 0.00761853903532028
1943 0.00761853903532028
1944 0.00761853903532028
1945 0.00761853903532028
1946 0.00761853903532028
1947 0.00761853903532028
1948 0.00761853903532028
1949 0.00761853903532028
1950 0.00761853903532028
1951 0.00761853903532028
1952 0.00761853903532028
1953 0.00761853903532028
1954 0.00761853903532028
1955 0.00761853903532028
1956 0.00761853903532028
1957 0.00761853903532028
1958 0.00761853903532028
1959 0.00761853903532028
1960 0.00761853903532028
1961 0.00761853903532028
1962 0.00761853903532028
1963 0.00761853903532028
1964 0.00761853903532028
1965 0.00761853903532028
1966 0.00761853903532028
1967 0.00761853903532028
1968 0.00761853903532028
1969 0.00761853903532028
1970 0.00761853903532028
1971 0.00761853903532028
1972 0.00761853903532028
1973 0.00761853903532028
1974 0.00761853903532028
1975 0.00761853903532028
1976 0.00761853903532028
1977 0.00761853903532028
1978 0.00761853903532028
1979 0.00761853903532028
1980 0.00761853903532028
1981 0.00761853903532028
1982 0.00761853903532028
1983 0.00761853903532028
1984 0.00761853903532028
1985 0.00761853903532028
1986 0.00761853903532028
1987 0.00761853903532028
1988 0.00761853903532028
1989 0.00761853903532028
1990 0.00761853903532028
1991 0.00761853903532028
1992 0.00761853903532028
1993 0.00761853903532028
1994 0.00761853903532028
1995 0.00761853903532028
1996 0.00761853903532028
1997 0.00761853903532028
1998 0.00761853903532028
1999 0.00761853903532028
};
\addlegendentry{Test}

\nextgroupplot[
legend cell align={left},
legend style={fill opacity=0.8, draw opacity=1, text opacity=1, at={(0.91,0.5)}, anchor=east, draw=white!80!black},
log basis y={10},
tick align=outside,
tick pos=left,
title={fold4 },
x grid style={white!69.0196078431373!black},
xlabel={Epoch},
xmin=-99.95, xmax=2098.95,
xtick style={color=black},
y grid style={white!69.0196078431373!black},
ylabel={MSE Loss},
ymin=4.80992375923239e-06, ymax=0.0117216592885793,
ymode=log,
ytick style={color=black}
]
\addplot [semithick, black, dashed]
table {%
0 0.00822320894440054
1 0.00820460099282627
2 0.0081851299560185
3 0.00816224136588062
4 0.00812980631530991
5 0.00807914323058867
6 0.00800119539985644
7 0.00788148516994625
8 0.0077082929392418
9 0.00747939522079832
10 0.00719748211668048
11 0.00686815154131182
12 0.00650004400995385
13 0.00609808819535829
14 0.00568912964536139
15 0.00528664919966104
16 0.00491248612433992
17 0.00458832454251024
18 0.00431205154518466
19 0.0040851426952031
20 0.00389338512877657
21 0.00373028170542966
22 0.00357952245008164
23 0.00342988834768221
24 0.00329137935091239
25 0.00316776507838767
26 0.00305162361837574
27 0.00293673816906903
28 0.0028185691063527
29 0.00270418399418304
30 0.00259803874553199
31 0.00250061365363763
32 0.00241079959027957
33 0.00232105375823721
34 0.00222892384954321
35 0.00214372064510826
36 0.00206285230342473
37 0.00198393043558553
38 0.00190810793469609
39 0.00183418172161964
40 0.0017640985804519
41 0.00169993570955285
42 0.00163745250773673
43 0.00156907237806081
44 0.00150223762574342
45 0.00144104558233948
46 0.00138422189598941
47 0.00132887555548677
48 0.00127869340531106
49 0.00123133398534492
50 0.00118392199885875
51 0.00113858978039616
52 0.00109727190988451
53 0.00105549756216305
54 0.00101580311962834
55 0.000976343439361926
56 0.00094069511672501
57 0.000909459627507658
58 0.000882137162790286
59 0.000856747461540408
60 0.00083145675998253
61 0.000807717972122646
62 0.000784026319365694
63 0.000761965938465892
64 0.000740992169113497
65 0.00072082202166257
66 0.000702259736783617
67 0.000685670016849826
68 0.000670529818094678
69 0.000656195746515209
70 0.000642737706115781
71 0.00063021114027606
72 0.000618441019014426
73 0.000607216733115479
74 0.000596443918936984
75 0.000586264079871057
76 0.000576529174480811
77 0.000566479455301305
78 0.000555858840399992
79 0.000545916596038865
80 0.000536426170469895
81 0.00052691616888012
82 0.000517111859259956
83 0.000505845381809422
84 0.000495116907821114
85 0.000485520576397391
86 0.000476744507110993
87 0.000468560425801456
88 0.000460656482779598
89 0.000452600341112941
90 0.000444652419758995
91 0.000437257551396897
92 0.00043030874184069
93 0.000423325323872348
94 0.000416000240491599
95 0.00040795639232627
96 0.000400463297445697
97 0.000393197424042313
98 0.000386042293921918
99 0.000379302703297668
100 0.000372658170750384
101 0.000366318639997587
102 0.000360379586812343
103 0.00035469466941862
104 0.000349253282919904
105 0.000343171777362983
106 0.000336800223813327
107 0.00033100268012376
108 0.000325531852922722
109 0.000320381957720883
110 0.000315532723238476
111 0.00031086766922428
112 0.000306445532629596
113 0.000302129460756362
114 0.00029785864661136
115 0.00029369003341273
116 0.000289489988652747
117 0.000285338260102463
118 0.000281216389844019
119 0.000276763964061288
120 0.000270687263403957
121 0.000264962891552756
122 0.000259861645014681
123 0.000255136275908896
124 0.000250701788212737
125 0.000245425353853079
126 0.000240619483209414
127 0.000236154090970331
128 0.000232087709363782
129 0.000228333571980599
130 0.000224778869821307
131 0.000221331664375768
132 0.00021802558854489
133 0.000214713925284826
134 0.000211433275190132
135 0.000207984147195361
136 0.000204583332784836
137 0.000201288813336475
138 0.00019815915416288
139 0.000194940639509866
140 0.000191603430065967
141 0.000188509848888074
142 0.000185623487514164
143 0.000182803304598522
144 0.000180016668061
145 0.000177334750494751
146 0.000174848309150377
147 0.000172265168735919
148 0.000169645708224309
149 0.000167177718836342
150 0.000164854062642661
151 0.000162508622270963
152 0.000160137962650353
153 0.000157882273814636
154 0.000155587941550905
155 0.000153372259305229
156 0.000151200087730174
157 0.000149071804198542
158 0.000147045781405453
159 0.000144974706154599
160 0.000142958904337576
161 0.000141018454202779
162 0.000139109656720038
163 0.000137234057110192
164 0.000135409923641916
165 0.000133607078428355
166 0.000131866306197992
167 0.000130118314810801
168 0.000128417216563648
169 0.000126717487658468
170 0.000125067867802642
171 0.000123447896974938
172 0.000121853082310608
173 0.000120262330362664
174 0.00011858329110126
175 0.000116952078075272
176 0.000115355648363291
177 0.000113753231039126
178 0.000112077870184635
179 0.000110519840326617
180 0.000109001592937474
181 0.000107510675121603
182 0.000106057431231932
183 0.000104584263194596
184 0.00010307640115137
185 0.000101558128166701
186 0.000100123924126727
187 9.87349524423919e-05
188 9.73798828400874e-05
189 9.60422066498268e-05
190 9.47373294684084e-05
191 9.34473145370873e-05
192 9.21585083784038e-05
193 9.09250872025424e-05
194 8.96718563954124e-05
195 8.84635198286077e-05
196 8.72348798894862e-05
197 8.6029517792241e-05
198 8.48257320233969e-05
199 8.36382433107019e-05
200 8.24799041421187e-05
201 8.13064989806378e-05
202 8.01656089466363e-05
203 7.90232220735376e-05
204 7.78864422690262e-05
205 7.67615585672843e-05
206 7.56390005030028e-05
207 7.45443486760422e-05
208 7.34738697518722e-05
209 7.24346941192167e-05
210 7.13904212359395e-05
211 7.03797334624667e-05
212 6.93799667752515e-05
213 6.83907849303722e-05
214 6.7424165479224e-05
215 6.65188692021701e-05
216 6.55638159905436e-05
217 6.46702753144268e-05
218 6.37826871852099e-05
219 6.28964726985259e-05
220 6.20648149958214e-05
221 6.12033662914513e-05
222 6.03731727886192e-05
223 5.95839074435389e-05
224 5.87622810770405e-05
225 5.79961709648025e-05
226 5.72195395283082e-05
227 5.6472290957732e-05
228 5.57313603355425e-05
229 5.50149289675339e-05
230 5.4282820908913e-05
231 5.3565411012535e-05
232 5.2854470014907e-05
233 5.2199323218094e-05
234 5.15508818157429e-05
235 5.09250984093157e-05
236 5.03030394276571e-05
237 4.97075194405561e-05
238 4.91285692034715e-05
239 4.85566533012793e-05
240 4.80074943447484e-05
241 4.74691980336672e-05
242 4.69144007868572e-05
243 4.64342992185784e-05
244 4.58672267815707e-05
245 4.54342640079375e-05
246 4.48524367797631e-05
247 4.45325139248887e-05
248 4.387661818539e-05
249 4.38217442391142e-05
250 4.30060909368812e-05
251 4.36881897640795e-05
252 4.26840815064278e-05
253 4.48610940260163e-05
254 4.27697472049626e-05
255 4.42257813333402e-05
256 4.0940946600676e-05
257 4.1223012310132e-05
258 3.98962600547392e-05
259 3.99259282164444e-05
260 3.93002746523763e-05
261 3.91854530636948e-05
262 3.8705120334992e-05
263 3.85445995534361e-05
264 3.81315689497175e-05
265 3.79480747199068e-05
266 3.75864070139009e-05
267 3.73730849849707e-05
268 3.70588185367726e-05
269 3.68434608923884e-05
270 3.65411174343411e-05
271 3.6326950122012e-05
272 3.60332678577269e-05
273 3.58532958253388e-05
274 3.55325603889067e-05
275 3.53933702605946e-05
276 3.50390379992849e-05
277 3.50250416787379e-05
278 3.4565271051834e-05
279 3.48459776651211e-05
280 3.42379519469027e-05
281 3.56994880548189e-05
282 3.5050069446485e-05
283 3.89835200569344e-05
284 3.50882854586176e-05
285 3.52417986333453e-05
286 3.29251211246273e-05
287 3.29429536911618e-05
288 3.25530913516392e-05
289 3.25002666317076e-05
290 3.22314454026529e-05
291 3.21113369536885e-05
292 3.1875713095264e-05
293 3.17476696367613e-05
294 3.15273600735022e-05
295 3.14093068876087e-05
296 3.11943509299084e-05
297 3.10770848346298e-05
298 3.08560372663802e-05
299 3.0760658610296e-05
300 3.05329519145747e-05
301 3.0479475794376e-05
302 3.02026966174651e-05
303 3.02225136683543e-05
304 2.98726074117184e-05
305 3.01282200019237e-05
306 2.96186659028974e-05
307 3.06692119493857e-05
308 3.00410749622948e-05
309 3.26654165273554e-05
310 3.11784678500571e-05
311 3.33276974859298e-05
312 2.93594327380742e-05
313 2.96808696662598e-05
314 2.84991651873234e-05
315 2.86742202462165e-05
316 2.83192594232595e-05
317 2.835274265367e-05
318 2.81217013697876e-05
319 2.80795014730995e-05
320 2.78955737991193e-05
321 2.78533771638223e-05
322 2.76751672652065e-05
323 2.76362154878473e-05
324 2.74473521744145e-05
325 2.74254155063769e-05
326 2.72218717574102e-05
327 2.72385885586868e-05
328 2.69757293567263e-05
329 2.71132413172115e-05
330 2.67450381985501e-05
331 2.72207516462952e-05
332 2.6681128595385e-05
333 2.83510764781392e-05
334 2.77430130203871e-05
335 3.06091479311821e-05
336 2.81516344322053e-05
337 2.94315806503853e-05
338 2.62098571648384e-05
339 2.65220444493064e-05
340 2.57764387210369e-05
341 2.58801189974012e-05
342 2.56460970440298e-05
343 2.56682914010486e-05
344 2.54991165071772e-05
345 2.54683479541384e-05
346 2.53367740761234e-05
347 2.53080053842503e-05
348 2.51826769108732e-05
349 2.51313480883475e-05
350 2.50261871006519e-05
351 2.49949260988647e-05
352 2.48333241861243e-05
353 2.48819079036178e-05
354 2.46473878788844e-05
355 2.48448517541355e-05
356 2.44790048540278e-05
357 2.52613020318426e-05
358 2.4797091885187e-05
359 2.69385077215634e-05
360 2.67123325054519e-05
361 3.09809558553392e-05
362 2.55856303790769e-05
363 2.5233146342285e-05
364 2.38492800288537e-05
365 2.40860793347952e-05
366 2.37764054902279e-05
367 2.3779315049266e-05
368 2.36626117364747e-05
369 2.36367036283713e-05
370 2.35239217589722e-05
371 2.35111631918805e-05
372 2.33972325245357e-05
373 2.33716357882496e-05
374 2.32613422275207e-05
375 2.32477279638044e-05
376 2.31305598432652e-05
377 2.31388720517423e-05
378 2.29813497414355e-05
379 2.30491166857139e-05
380 2.28427812409837e-05
381 2.30460609542682e-05
382 2.27205670718655e-05
383 2.34735587821833e-05
384 2.31621637614698e-05
385 2.65604314069467e-05
386 2.6057081129599e-05
387 2.86139030124394e-05
388 2.33486119478599e-05
389 2.33074022158997e-05
390 2.22196889816395e-05
391 2.23497287441532e-05
392 2.21471753121261e-05
393 2.21865643010855e-05
394 2.20686008136628e-05
395 2.20553495375864e-05
396 2.19650511139591e-05
397 2.19538564514288e-05
398 2.18661858906177e-05
399 2.18599054360169e-05
400 2.17642697684095e-05
401 2.17608554280435e-05
402 2.16574280200721e-05
403 2.16817891360854e-05
404 2.15392235844991e-05
405 2.1624301387968e-05
406 2.14229939807797e-05
407 2.17152201393844e-05
408 2.13802062392077e-05
409 2.25038155978186e-05
410 2.23372189573079e-05
411 2.54278171610001e-05
412 2.46467358451241e-05
413 2.69743405938994e-05
414 2.18259475107674e-05
415 2.17564045075225e-05
416 2.09190310584972e-05
417 2.10834031005858e-05
418 2.08782893023063e-05
419 2.09440656568916e-05
420 2.08226986302407e-05
421 2.08399922274616e-05
422 2.07433558083725e-05
423 2.07651152090982e-05
424 2.06664631035025e-05
425 2.06893328293933e-05
426 2.05765671887903e-05
427 2.06385457499625e-05
428 2.04739622345906e-05
429 2.06347588378453e-05
430 2.0382208089309e-05
431 2.07729035828219e-05
432 2.04107977346268e-05
433 2.16809707781485e-05
434 2.1448416298836e-05
435 2.4251217322982e-05
436 2.31702421839564e-05
437 2.52991609340086e-05
438 2.08263211093906e-05
439 2.09046532334556e-05
440 1.99486964638229e-05
441 2.01784835394125e-05
442 1.99194060919172e-05
443 1.99857341551812e-05
444 1.9869843869591e-05
445 1.989681471537e-05
446 1.98054159307182e-05
447 1.98216525352235e-05
448 1.97284917186069e-05
449 1.97624577873423e-05
450 1.96384777420011e-05
451 1.97239271659821e-05
452 1.95500908122259e-05
453 1.97500079687218e-05
454 1.947845724104e-05
455 2.00360475215433e-05
456 1.97036119652938e-05
457 2.18379295806592e-05
458 2.17369709901438e-05
459 2.44299065687986e-05
460 2.19997433731078e-05
461 2.2588657748901e-05
462 1.93767901894848e-05
463 1.9639032084684e-05
464 1.90946521746449e-05
465 1.92226518480609e-05
466 1.90780188417117e-05
467 1.91191688659886e-05
468 1.90210635707677e-05
469 1.90498055522292e-05
470 1.89616082274524e-05
471 1.89742346562438e-05
472 1.8900743021133e-05
473 1.89193923958442e-05
474 1.88226155387605e-05
475 1.88786261658969e-05
476 1.87480419323593e-05
477 1.88900567755779e-05
478 1.86581397725316e-05
479 1.91061750359767e-05
480 1.88112325584244e-05
481 2.07495217439124e-05
482 2.10218645895488e-05
483 2.45522771111228e-05
484 2.18398002208853e-05
485 2.19680917024623e-05
486 1.85521657736221e-05
487 1.87701720759659e-05
488 1.83273621131574e-05
489 1.8419630976152e-05
490 1.8297453859617e-05
491 1.8325318411061e-05
492 1.8247458251297e-05
493 1.8264787386979e-05
494 1.8195215565231e-05
495 1.82035029870908e-05
496 1.81292615508699e-05
497 1.81504641226793e-05
498 1.80626501649073e-05
499 1.81117644078022e-05
500 1.79743981999492e-05
501 1.8118269063061e-05
502 1.7895521555289e-05
503 1.83437736746006e-05
504 1.80553408847262e-05
505 1.99876954190825e-05
506 2.02938788228524e-05
507 2.40064687196256e-05
508 2.0910354507464e-05
509 2.09925544012168e-05
510 1.77724108578747e-05
511 1.7997383127244e-05
512 1.75726350909455e-05
513 1.76645348137772e-05
514 1.75491723578558e-05
515 1.75796312735788e-05
516 1.75053153905758e-05
517 1.75149186224521e-05
518 1.74469692734158e-05
519 1.74595315760584e-05
520 1.73834916847859e-05
521 1.74083182896689e-05
522 1.73169679216212e-05
523 1.73772538791805e-05
524 1.72465252439302e-05
525 1.73653869217638e-05
526 1.71698790323305e-05
527 1.75239986492937e-05
528 1.72470479772313e-05
529 1.88662095541581e-05
530 1.91468523266791e-05
531 2.33180140695732e-05
532 2.08675901555155e-05
533 2.10811162144608e-05
534 1.71530937662645e-05
535 1.7330270030369e-05
536 1.68654036786231e-05
537 1.69582627910225e-05
538 1.68452132000008e-05
539 1.6871852635103e-05
540 1.67941989712705e-05
541 1.68036358748624e-05
542 1.67510115570702e-05
543 1.67452281963909e-05
544 1.66889351880251e-05
545 1.66967326888923e-05
546 1.6628282077602e-05
547 1.66475333719385e-05
548 1.65713359429054e-05
549 1.66191271079308e-05
550 1.64883257276482e-05
551 1.66712847154971e-05
552 1.64332371325182e-05
553 1.71310224097443e-05
554 1.70037619829366e-05
555 2.00365492113441e-05
556 2.09002341667597e-05
557 2.47374601884676e-05
558 1.77042392431881e-05
559 1.70284025856615e-05
560 1.61669459369795e-05
561 1.62772971722402e-05
562 1.61786285335364e-05
563 1.61903307240863e-05
564 1.61436735827136e-05
565 1.61281036663574e-05
566 1.60869561692811e-05
567 1.60756861911082e-05
568 1.60215723253865e-05
569 1.60288169792633e-05
570 1.59811299342838e-05
571 1.59661830954061e-05
572 1.59215507645749e-05
573 1.59237318630989e-05
574 1.58703852940478e-05
575 1.58576519843079e-05
576 1.58206207618861e-05
577 1.57962061742012e-05
578 1.57709288231e-05
579 1.57305541628183e-05
580 1.57308922158483e-05
581 1.56871188908836e-05
582 1.56720268144062e-05
583 1.56234334893757e-05
584 1.56666026989161e-05
585 1.55243933338545e-05
586 1.60536194840355e-05
587 1.63046439421999e-05
588 2.15988132500622e-05
589 2.58026308111725e-05
590 2.51537795350476e-05
591 1.58717971306399e-05
592 1.55095820056905e-05
593 1.53955394054384e-05
594 1.53271914199449e-05
595 1.53000963425409e-05
596 1.52689257516547e-05
597 1.52462304140455e-05
598 1.52209637052092e-05
599 1.51943607022975e-05
600 1.51793369052022e-05
601 1.51542660487003e-05
602 1.51211321047784e-05
603 1.51052170312971e-05
604 1.50905576338234e-05
605 1.50547382766764e-05
606 1.50278909583079e-05
607 1.50311587553409e-05
608 1.49731008162179e-05
609 1.49840178595984e-05
610 1.49279996590757e-05
611 1.49450889191938e-05
612 1.4877688306103e-05
613 1.49225239804984e-05
614 1.48021111330277e-05
615 1.4988423293949e-05
616 1.47704681294769e-05
617 1.56853635689957e-05
618 1.59316092391837e-05
619 2.08032628470495e-05
620 2.06401032683684e-05
621 2.12780462382212e-05
622 1.5011271776999e-05
623 1.49925192435285e-05
624 1.45795763878409e-05
625 1.46187206508941e-05
626 1.45524094943283e-05
627 1.45401660884836e-05
628 1.45105341449181e-05
629 1.44939253498011e-05
630 1.44699072019971e-05
631 1.44645556168221e-05
632 1.44297598223542e-05
633 1.44224069722476e-05
634 1.43904385179194e-05
635 1.43800614291756e-05
636 1.43429510863768e-05
637 1.43617498632453e-05
638 1.4279603589018e-05
639 1.43899913332968e-05
640 1.42061920422876e-05
641 1.45570411518303e-05
642 1.42921590969847e-05
643 1.56289882435878e-05
644 1.61800933146106e-05
645 2.25085845204642e-05
646 1.84513078766713e-05
647 1.68897005822788e-05
648 1.40559537027407e-05
649 1.41955467768407e-05
650 1.39783322046227e-05
651 1.40799061164598e-05
652 1.39817693937117e-05
653 1.40063179440153e-05
654 1.39407849759632e-05
655 1.39562670167237e-05
656 1.39164501105071e-05
657 1.39024178896774e-05
658 1.38868079925247e-05
659 1.38645883696054e-05
660 1.38574329423546e-05
661 1.38057834444538e-05
662 1.38497821779993e-05
663 1.37421737609333e-05
664 1.39142740575027e-05
665 1.36891383195614e-05
666 1.42119416866038e-05
667 1.39721305195128e-05
668 1.59564346453311e-05
669 1.67656337051736e-05
670 2.31444820924243e-05
671 1.63528966005688e-05
672 1.50519575844998e-05
673 1.3480590791759e-05
674 1.36233258394292e-05
675 1.35028081968969e-05
676 1.35314747419235e-05
677 1.34873273570646e-05
678 1.3481138006588e-05
679 1.3455314362032e-05
680 1.34490866647852e-05
681 1.34301929578529e-05
682 1.34064102705889e-05
683 1.34041179780109e-05
684 1.33585720485385e-05
685 1.33905257913014e-05
686 1.33067645348151e-05
687 1.34010884029756e-05
688 1.32414641726974e-05
689 1.34560754257995e-05
690 1.32143758969638e-05
691 1.37955813057644e-05
692 1.35749835805665e-05
693 1.5886047612601e-05
694 1.66555794415268e-05
695 2.09253151661759e-05
696 1.57583519740379e-05
697 1.52852952658433e-05
698 1.30584354640639e-05
699 1.33015272025805e-05
700 1.30449493543905e-05
701 1.30960856401519e-05
702 1.30223826840803e-05
703 1.304492697074e-05
704 1.29888345871132e-05
705 1.29933956541928e-05
706 1.29593761738489e-05
707 1.29513259861103e-05
708 1.29281718044272e-05
709 1.29074560274445e-05
710 1.29076234676173e-05
711 1.28576273515923e-05
712 1.29041479931535e-05
713 1.27959039480929e-05
714 1.29877693519909e-05
715 1.27531814638759e-05
716 1.34125382337258e-05
717 1.3287292143227e-05
718 1.61618883948722e-05
719 1.71302676257035e-05
720 2.1632718261555e-05
721 1.41643334342634e-05
722 1.35392248244992e-05
723 1.25761984393069e-05
724 1.26985788854173e-05
725 1.26014791995699e-05
726 1.26312456268929e-05
727 1.25783023628978e-05
728 1.25867737830188e-05
729 1.25576109767378e-05
730 1.25500477088369e-05
731 1.25294391004616e-05
732 1.25390163003836e-05
733 1.24986184515397e-05
734 1.24979694018368e-05
735 1.24658954347989e-05
736 1.24742676712586e-05
737 1.24396288525563e-05
738 1.24461285970767e-05
739 1.2403617317247e-05
740 1.24295954690989e-05
741 1.23635167472713e-05
742 1.24276649935418e-05
743 1.2308658992044e-05
744 1.26750065047032e-05
745 1.25551047589489e-05
746 1.49556710485887e-05
747 1.731739090155e-05
748 2.28973866429061e-05
749 1.53227571394465e-05
750 1.38249930849987e-05
751 1.21917891495738e-05
752 1.2298760773799e-05
753 1.22001554988538e-05
754 1.21825174868029e-05
755 1.21444086742528e-05
756 1.21257658876051e-05
757 1.21100714507438e-05
758 1.20958175127139e-05
759 1.20813041653012e-05
760 1.20673663102089e-05
761 1.20487603969543e-05
762 1.20479872716128e-05
763 1.20169375881218e-05
764 1.20161849450628e-05
765 1.19906352693544e-05
766 1.19789254786618e-05
767 1.19700240484022e-05
768 1.1942990344016e-05
769 1.19608395456883e-05
770 1.18911187412252e-05
771 1.19958809063903e-05
772 1.1840543181485e-05
773 1.23226771499008e-05
774 1.237762338413e-05
775 1.72444099263425e-05
776 1.97837957003344e-05
777 2.05188953549262e-05
778 1.22682578489464e-05
779 1.20206549202306e-05
780 1.17748461008205e-05
781 1.17556738314928e-05
782 1.17294485080244e-05
783 1.17105145869267e-05
784 1.16995092389605e-05
785 1.16945933887891e-05
786 1.16691543263769e-05
787 1.16670771250771e-05
788 1.16536580949678e-05
789 1.16419247908617e-05
790 1.16319528005171e-05
791 1.1634041148012e-05
792 1.16048332532426e-05
793 1.15954444703803e-05
794 1.15807654410549e-05
795 1.15678456307222e-05
796 1.15642135767313e-05
797 1.15336818868883e-05
798 1.1553723733293e-05
799 1.14981371439615e-05
800 1.15701809688673e-05
801 1.14489453166833e-05
802 1.17709277652756e-05
803 1.16913622557613e-05
804 1.47411722969948e-05
805 1.77666572436785e-05
806 2.33919261594662e-05
807 1.31082885878042e-05
808 1.18740709291743e-05
809 1.135616014869e-05
810 1.13814263245082e-05
811 1.13595129873634e-05
812 1.13471998643977e-05
813 1.1334883645242e-05
814 1.13216251955706e-05
815 1.13148181500877e-05
816 1.13054350168174e-05
817 1.12878156777185e-05
818 1.1284824785518e-05
819 1.1259721776069e-05
820 1.12646799210747e-05
821 1.12418733495545e-05
822 1.12344314389112e-05
823 1.12204415001482e-05
824 1.12102687015447e-05
825 1.12026844707458e-05
826 1.11808648137757e-05
827 1.11877269870497e-05
828 1.11483778051369e-05
829 1.11936650670774e-05
830 1.11037566286054e-05
831 1.12576227457861e-05
832 1.10875410874733e-05
833 1.18524624601601e-05
834 1.2220563920029e-05
835 1.70973452797152e-05
836 1.85468634174013e-05
837 1.90248071593735e-05
838 1.13999011520649e-05
839 1.12901973704504e-05
840 1.10157466084448e-05
841 1.10098092240829e-05
842 1.09885267403076e-05
843 1.09749651735491e-05
844 1.09654992723751e-05
845 1.09566285697582e-05
846 1.09429511025994e-05
847 1.09379446036062e-05
848 1.09213030199395e-05
849 1.09158094510065e-05
850 1.09070507248177e-05
851 1.08925314765695e-05
852 1.08885685434768e-05
853 1.08777368428292e-05
854 1.08627342645384e-05
855 1.08569334611452e-05
856 1.08453856015878e-05
857 1.0837737847158e-05
858 1.08243861354218e-05
859 1.08196706569696e-05
860 1.08242510759027e-05
861 1.07862791534075e-05
862 1.08344026508167e-05
863 1.07346267739228e-05
864 1.09730410260411e-05
865 1.08526872240056e-05
866 1.28215210148386e-05
867 1.57317402622037e-05
868 2.42049997055127e-05
869 1.44217683715198e-05
870 1.21252632082625e-05
871 1.07382120968502e-05
872 1.07617548016359e-05
873 1.07182656147176e-05
874 1.06891422930744e-05
875 1.06528576334641e-05
876 1.06393803855909e-05
877 1.06197020557364e-05
878 1.0600139475847e-05
879 1.06061032097982e-05
880 1.05860732273477e-05
881 1.05885267013273e-05
882 1.05655852306041e-05
883 1.05675629840096e-05
884 1.05496225477619e-05
885 1.05456467478504e-05
886 1.05328220281198e-05
887 1.05147452966747e-05
888 1.05184075347431e-05
889 1.04880904492344e-05
890 1.0515930661148e-05
891 1.04491075354485e-05
892 1.05535375480859e-05
893 1.04046420905668e-05
894 1.08428495505519e-05
895 1.08321230323227e-05
896 1.51579897859078e-05
897 1.77607899333188e-05
898 1.98485009293425e-05
899 1.10015288572463e-05
900 1.06936451512718e-05
901 1.03840158967028e-05
902 1.0378943335887e-05
903 1.03467305823113e-05
904 1.03375307318743e-05
905 1.03289723816236e-05
906 1.03220999152498e-05
907 1.03178175752117e-05
908 1.03068781147631e-05
909 1.02905940617681e-05
910 1.02907504182514e-05
911 1.02728808668573e-05
912 1.02779627579874e-05
913 1.02587570716128e-05
914 1.02692884913491e-05
915 1.02355155704448e-05
916 1.02739005496399e-05
917 1.02032195643531e-05
918 1.02985379164533e-05
919 1.01608436589018e-05
920 1.04632570081531e-05
921 1.0248548434455e-05
922 1.16638000807745e-05
923 1.23309369772784e-05
924 1.75823405065834e-05
925 1.46801960068665e-05
926 1.41576982707337e-05
927 1.01865264418022e-05
928 1.04541541410264e-05
929 1.0102140975965e-05
930 1.01724739127729e-05
931 1.0089354231857e-05
932 1.01068924862613e-05
933 1.00730763383794e-05
934 1.0080489502684e-05
935 1.0055134146203e-05
936 1.00605642416873e-05
937 1.00468610644278e-05
938 1.0043814939964e-05
939 1.00238563418564e-05
940 1.00362508248519e-05
941 9.99932454393004e-06
942 1.00333812511622e-05
943 9.96736092373496e-06
944 1.00904969388971e-05
945 9.93019410189255e-06
946 1.03559376818385e-05
947 1.02204039179599e-05
948 1.2588431057281e-05
949 1.40394067654759e-05
950 1.88100041847439e-05
951 1.27533548573977e-05
952 1.18035652139081e-05
953 9.86002565583899e-06
954 1.00611116844451e-05
955 9.90952963908498e-06
956 9.92226779217287e-06
957 9.88073750907148e-06
958 9.8763066049834e-06
959 9.86252837620327e-06
960 9.85982665535623e-06
961 9.8469722491501e-06
962 9.84307458962164e-06
963 9.83785013253247e-06
964 9.82933479787285e-06
965 9.81186106674414e-06
966 9.83222320782851e-06
967 9.78960967001896e-06
968 9.84777611523491e-06
969 9.75102699007913e-06
970 9.90412268420116e-06
971 9.73324180009172e-06
972 1.03309216623604e-05
973 1.03559870158199e-05
974 1.48048399544187e-05
975 1.57552489821189e-05
976 1.72850403808433e-05
977 1.02736750277055e-05
978 1.01647852293052e-05
979 9.69430696073914e-06
980 9.7495479065568e-06
981 9.69513564230695e-06
982 9.6896519907741e-06
983 9.66939003910205e-06
984 9.67368532411861e-06
985 9.65850274736724e-06
986 9.6553596760085e-06
987 9.64825021232407e-06
988 9.64005097658927e-06
989 9.63500988060417e-06
990 9.63219633987045e-06
991 9.61013443312453e-06
992 9.62979666474517e-06
993 9.58892295810365e-06
994 9.64739682585147e-06
995 9.55375408429049e-06
996 9.72004876587107e-06
997 9.5473184716921e-06
998 1.02444952200242e-05
999 1.0438367779142e-05
1000 1.50626689744415e-05
1001 1.58800367743295e-05
1002 1.64605087086822e-05
1003 1.00009353707353e-05
1004 9.89909343140916e-06
1005 9.46755798203469e-06
1006 9.53755725363603e-06
1007 9.50769566054888e-06
1008 9.50253097287135e-06
1009 9.48323423699549e-06
1010 9.48795730826291e-06
1011 9.47679983664962e-06
1012 9.46612353736054e-06
1013 9.47030042774966e-06
1014 9.45139892460922e-06
1015 9.44829392524049e-06
1016 9.44851887163534e-06
1017 9.43538167896829e-06
1018 9.4301152530063e-06
1019 9.41891287820074e-06
1020 9.4294606327594e-06
1021 9.4023097588769e-06
1022 9.43639433215182e-06
1023 9.36371205961795e-06
1024 9.51109431490682e-06
1025 9.35944074642592e-06
1026 1.00635510106306e-05
1027 1.03842960010869e-05
1028 1.50216018558469e-05
1029 1.68119532944488e-05
1030 1.69918323631446e-05
1031 9.75303569372699e-06
1032 9.56280876085458e-06
1033 9.27217349666609e-06
1034 9.36693574482561e-06
1035 9.3125555979956e-06
1036 9.32446169532586e-06
1037 9.29548149575243e-06
1038 9.30600720028263e-06
1039 9.28295934421453e-06
1040 9.28602911387522e-06
1041 9.26505643294995e-06
1042 9.28014956591561e-06
1043 9.2647475161689e-06
1044 9.27146490226072e-06
1045 9.25211635272039e-06
1046 9.25554160546849e-06
1047 9.24400645962997e-06
1048 9.24509553534847e-06
1049 9.21227802930247e-06
1050 9.27793954264011e-06
1051 9.18189614340648e-06
1052 9.39278787026598e-06
1053 9.22625528643195e-06
1054 1.03777676841998e-05
1055 1.1086031438623e-05
1056 1.65887206736715e-05
1057 1.49935478511998e-05
1058 1.39286444268971e-05
1059 9.24109964595132e-06
1060 9.43270690312303e-06
1061 9.1707925455875e-06
1062 9.20715332330957e-06
1063 9.154853347626e-06
1064 9.14732236956262e-06
1065 9.12824374388332e-06
1066 9.12136503306016e-06
1067 9.11344857923435e-06
1068 9.10732371495193e-06
1069 9.1102651509356e-06
1070 9.091911812531e-06
1071 9.08724882175971e-06
1072 9.07016206452838e-06
1073 9.08790141185278e-06
1074 9.0538263238793e-06
1075 9.10221024852387e-06
1076 9.01966640654006e-06
1077 9.1990862758351e-06
1078 9.05594913691576e-06
1079 9.97363795696149e-06
1080 1.04671697897518e-05
1081 1.52202032702997e-05
1082 1.49222048515529e-05
1083 1.45430912295574e-05
1084 9.24534871427074e-06
1085 9.33285627857661e-06
1086 8.99576751378639e-06
1087 9.0389003922775e-06
1088 8.99512391805413e-06
1089 8.99358333661393e-06
1090 8.96672473649218e-06
1091 8.97060829441187e-06
1092 8.95147447743661e-06
1093 8.9555297584365e-06
1094 8.93525290990294e-06
1095 8.94330418321765e-06
1096 8.93013680491883e-06
1097 8.93425907055878e-06
1098 8.90501348016492e-06
1099 8.93387851486072e-06
1100 8.87883979061588e-06
1101 8.99763617800975e-06
1102 8.88276838884039e-06
1103 9.38648496850991e-06
1104 9.57805679951962e-06
1105 1.39493605547658e-05
1106 1.53399043247848e-05
1107 1.50745618754944e-05
1108 9.87947650799015e-06
1109 9.58965181618865e-06
1110 8.84299703485247e-06
1111 8.93642907262571e-06
1112 8.86416835277348e-06
1113 8.87544860128653e-06
1114 8.84045696325497e-06
1115 8.84243969456744e-06
1116 8.82005227198501e-06
1117 8.82178526084854e-06
1118 8.8111953923109e-06
1119 8.81540767483369e-06
1120 8.79737067316366e-06
1121 8.80462475216781e-06
1122 8.77435154134076e-06
1123 8.79930743180601e-06
1124 8.75557525803483e-06
1125 8.84703212977733e-06
1126 8.75643164477413e-06
1127 9.20185733910284e-06
1128 9.40197475601812e-06
1129 1.32241260699106e-05
1130 1.52097238842186e-05
1131 1.62095949947649e-05
1132 9.46465002871211e-06
1133 9.12572495392894e-06
1134 8.7242107155161e-06
1135 8.76997974974536e-06
1136 8.73338339857366e-06
1137 8.72956838648964e-06
1138 8.71671123059414e-06
1139 8.70597430435716e-06
1140 8.70102143768481e-06
1141 8.69859496532843e-06
1142 8.68904090023381e-06
1143 8.69609349474221e-06
1144 8.6816255641109e-06
1145 8.68557913635826e-06
1146 8.66461894011294e-06
1147 8.68279534338434e-06
1148 8.65878185374314e-06
1149 8.70217540804852e-06
1150 8.63540800011275e-06
1151 8.83134806717312e-06
1152 8.8051004351053e-06
1153 1.01245717843979e-05
1154 1.19580126846652e-05
1155 1.76487869940067e-05
1156 1.34825381059578e-05
1157 1.10615856290797e-05
1158 8.61035878985117e-06
1159 8.75051598958621e-06
1160 8.63679358709124e-06
1161 8.63701129882877e-06
1162 8.61180793443328e-06
1163 8.6160918708611e-06
1164 8.5984237019332e-06
1165 8.59335338454503e-06
1166 8.57833176082234e-06
1167 8.57396016595846e-06
1168 8.57326939340819e-06
1169 8.56551086736435e-06
1170 8.56028616780247e-06
1171 8.55546143485597e-06
1172 8.5482885449073e-06
1173 8.55264138188705e-06
1174 8.55142751388094e-06
1175 8.54998058741341e-06
1176 8.54199000988043e-06
1177 8.5336876934905e-06
1178 8.54701689667614e-06
1179 8.52480446622561e-06
1180 8.58331496866871e-06
1181 8.5439033863377e-06
1182 9.22407369219336e-06
1183 1.09948005914684e-05
1184 1.99447747932235e-05
1185 1.75642731336523e-05
1186 1.15995790478385e-05
1187 8.45270908578222e-06
1188 8.54309329534342e-06
1189 8.52267518125505e-06
1190 8.49895989152305e-06
1191 8.49236632505246e-06
1192 8.4771628844349e-06
1193 8.47371825285492e-06
1194 8.46776354102019e-06
1195 8.4569992589989e-06
1196 8.45668300053504e-06
1197 8.4454691808844e-06
1198 8.45157990714274e-06
1199 8.44100818475635e-06
1200 8.44961757551221e-06
1201 8.42726696925755e-06
1202 8.45168921215222e-06
1203 8.42017803071027e-06
1204 8.46425427725173e-06
1205 8.4080420260646e-06
1206 8.573268202694e-06
1207 8.54479998169122e-06
1208 9.63926359021094e-06
1209 1.10682157608566e-05
1210 1.56598844162914e-05
1211 1.35751756206726e-05
1212 1.1518777920938e-05
1213 8.48008402787315e-06
1214 8.57892122951842e-06
1215 8.39075409631018e-06
1216 8.41772039139599e-06
1217 8.38596784447088e-06
1218 8.38938320091298e-06
1219 8.37264264386128e-06
1220 8.37769295491952e-06
1221 8.36301360651603e-06
1222 8.37117163710044e-06
1223 8.35402839494126e-06
1224 8.36351934696555e-06
1225 8.33851436310429e-06
1226 8.36973558571152e-06
1227 8.33393377275282e-06
1228 8.42665538047793e-06
1229 8.37890994831447e-06
1230 8.86809016820145e-06
1231 9.40093083096061e-06
1232 1.34117745717588e-05
1233 1.5209092373264e-05
1234 1.39592683841006e-05
1235 8.84440223136718e-06
1236 8.58061733832738e-06
1237 8.29816990488563e-06
1238 8.33683306999955e-06
1239 8.30721060485207e-06
1240 8.30832184817165e-06
1241 8.30031315879332e-06
1242 8.29647087219332e-06
1243 8.28840742239656e-06
1244 8.29358909804068e-06
1245 8.27786843826672e-06
1246 8.29034027860498e-06
1247 8.27585490026372e-06
1248 8.28822518750272e-06
1249 8.26289321331775e-06
1250 8.29630999632514e-06
1251 8.26005750476089e-06
1252 8.38105927924992e-06
1253 8.36793191816287e-06
1254 9.26029735359712e-06
1255 1.0864821007539e-05
1256 1.69126246244711e-05
1257 1.39092632143734e-05
1258 1.06965598731246e-05
1259 8.25176530749872e-06
1260 8.3409251488975e-06
1261 8.23900132185251e-06
1262 8.24727937975833e-06
1263 8.22631540020602e-06
1264 8.23302300467788e-06
1265 8.21639896053661e-06
1266 8.22266088118528e-06
1267 8.21262177885806e-06
1268 8.2123973630388e-06
1269 8.21131747097414e-06
1270 8.20700367598626e-06
1271 8.20390585187791e-06
1272 8.19526659812109e-06
1273 8.20314095595354e-06
1274 8.1892703797104e-06
1275 8.20250872735073e-06
1276 8.17973599220867e-06
1277 8.24887871364588e-06
1278 8.22906099295917e-06
1279 8.79267975961717e-06
1280 9.82884463407441e-06
1281 1.6342953184284e-05
1282 1.63584835569663e-05
1283 1.23398881649583e-05
1284 8.2298923149704e-06
1285 8.28297743715734e-06
1286 8.19015199249318e-06
1287 8.17470176828206e-06
1288 8.16410299009362e-06
1289 8.15457151137622e-06
1290 8.14818440120213e-06
1291 8.13941193067436e-06
1292 8.14004235849541e-06
1293 8.13280712363973e-06
1294 8.12531718541454e-06
1295 8.12405483263667e-06
1296 8.11660968536021e-06
1297 8.12016974183205e-06
1298 8.11332152061972e-06
1299 8.11441166337357e-06
1300 8.10514086424519e-06
1301 8.10950443408398e-06
1302 8.10433659859111e-06
1303 8.12704551544918e-06
1304 8.10710199905529e-06
1305 8.31260739642126e-06
1306 8.698553356834e-06
1307 1.20261846657255e-05
1308 1.73097311940706e-05
1309 1.71781833558093e-05
1310 8.77714996128098e-06
1311 8.28844316524968e-06
1312 8.11478176609981e-06
1313 8.10564197717412e-06
1314 8.09061341411343e-06
1315 8.08994363354909e-06
1316 8.07380207257236e-06
1317 8.06998335700637e-06
1318 8.06012550713664e-06
1319 8.05707114293064e-06
1320 8.05611990384314e-06
1321 8.04685725863674e-06
1322 8.04704965473579e-06
1323 8.03676149963017e-06
1324 8.04935041875954e-06
1325 8.03751113342965e-06
1326 8.04566343992619e-06
1327 8.02924125586912e-06
1328 8.05009642423915e-06
1329 8.0295114397444e-06
1330 8.1288610036534e-06
1331 8.20790677236172e-06
1332 9.26895241659675e-06
1333 1.1698815794825e-05
1334 1.77083231362452e-05
1335 1.28454512731968e-05
1336 9.81459645910121e-06
1337 8.02441055292036e-06
1338 8.11009000012408e-06
1339 8.0429336098975e-06
1340 8.032873160424e-06
1341 8.01306974951288e-06
1342 8.01353626800339e-06
1343 8.00273256806872e-06
1344 7.99700276399307e-06
1345 7.98926557132873e-06
1346 7.98257869893959e-06
1347 7.98348646868785e-06
1348 7.97915104422398e-06
1349 7.97502395677441e-06
1350 7.96563343907764e-06
1351 7.9771592598199e-06
1352 7.96484745624326e-06
1353 7.98404695612209e-06
1354 7.96681651238362e-06
1355 8.10070092538417e-06
1356 8.25149760452781e-06
1357 9.75802704594741e-06
1358 1.33753862970876e-05
1359 1.78481046161361e-05
1360 1.08485157140348e-05
1361 8.83490459530067e-06
1362 7.9451836046962e-06
1363 8.00310485815636e-06
1364 7.97215851111588e-06
1365 7.96163752458945e-06
1366 7.94999943387609e-06
1367 7.94274935611572e-06
1368 7.93256097442185e-06
1369 7.93492872019907e-06
1370 7.92295798446307e-06
1371 7.92676982785956e-06
1372 7.91858457094285e-06
1373 7.91918220399701e-06
1374 7.91429766466845e-06
1375 7.93049314340433e-06
1376 7.9095566664078e-06
1377 7.98808981938848e-06
1378 8.00373152887168e-06
1379 8.49441535888396e-06
1380 9.51405788390813e-06
1381 1.33992602581179e-05
1382 1.45458457678771e-05
1383 1.22873312081162e-05
1384 8.25711842666887e-06
1385 8.11230210429059e-06
1386 7.89802767842129e-06
1387 7.9140524970045e-06
1388 7.89319195437788e-06
1389 7.88647485971961e-06
1390 7.88079280478193e-06
1391 7.87943465829777e-06
1392 7.87377484312568e-06
1393 7.86634478411319e-06
1394 7.87152899484056e-06
1395 7.86412912667789e-06
1396 7.86545122832827e-06
1397 7.86090364490732e-06
1398 7.86831072130401e-06
1399 7.85619972931695e-06
1400 7.89340475670919e-06
1401 7.90429083796784e-06
1402 8.24702036494873e-06
1403 9.0796622240763e-06
1404 1.28628648549678e-05
1405 1.57750815651114e-05
1406 1.34850182003654e-05
1407 8.30208041280311e-06
1408 8.04186487779468e-06
1409 7.85548283854798e-06
1410 7.86183760692616e-06
1411 7.84187373570333e-06
1412 7.83973515172676e-06
1413 7.83177132457347e-06
1414 7.82949096422847e-06
1415 7.81893462376004e-06
1416 7.82336571580888e-06
1417 7.81584963838178e-06
1418 7.82008542921098e-06
1419 7.8097144150524e-06
1420 7.8230451109329e-06
1421 7.81094180590802e-06
1422 7.84028999334385e-06
1423 7.84363679751099e-06
1424 8.03612706967716e-06
1425 8.40549379610422e-06
1426 1.01660208840504e-05
1427 1.31128900449917e-05
1428 1.51604696608243e-05
1429 9.91177932363474e-06
1430 8.51814361746417e-06
1431 7.79278588869392e-06
1432 7.84319753455165e-06
1433 7.79113028503975e-06
1434 7.79488055291377e-06
1435 7.78307266180445e-06
1436 7.7854105395847e-06
1437 7.77454703648761e-06
1438 7.77720373590363e-06
1439 7.76897877663441e-06
1440 7.77648868743874e-06
1441 7.76703945926105e-06
1442 7.7887843195068e-06
1443 7.78228028242012e-06
1444 7.89301450376811e-06
1445 8.0367399498682e-06
1446 8.78041088842041e-06
1447 1.02544395193527e-05
1448 1.35243170291677e-05
1449 1.25413476656233e-05
1450 1.04263984374575e-05
1451 7.99870413092929e-06
1452 7.93563715673073e-06
1453 7.75248519335747e-06
1454 7.77379800809275e-06
1455 7.74030572092688e-06
1456 7.74601072095482e-06
1457 7.73342274251743e-06
1458 7.73781979324095e-06
1459 7.72955786310092e-06
1460 7.74156429628192e-06
1461 7.72818464389591e-06
1462 7.76001037505747e-06
1463 7.76245513423035e-06
1464 7.90038750442257e-06
1465 8.11505670111901e-06
1466 9.01254898671988e-06
1467 1.06373913467128e-05
1468 1.33788836467752e-05
1469 1.16923180392137e-05
1470 9.83120960368344e-06
1471 7.91738302874379e-06
1472 7.88649390803808e-06
1473 7.71555380385536e-06
1474 7.73476148163699e-06
1475 7.70282207118722e-06
1476 7.70976597563866e-06
1477 7.69378006892918e-06
1478 7.70219950729789e-06
1479 7.68750903290538e-06
1480 7.70491826340702e-06
1481 7.69911571696369e-06
1482 7.7423411957156e-06
1483 7.77898780701669e-06
1484 8.00494664310314e-06
1485 8.44008887157521e-06
1486 1.0031931643506e-05
1487 1.22101266146757e-05
1488 1.31022869491737e-05
1489 9.92050508086262e-06
1490 8.56566361934208e-06
1491 7.76112292200093e-06
1492 7.75486320747643e-06
1493 7.67202714480053e-06
1494 7.68323881317201e-06
1495 7.66272065366724e-06
1496 7.66960145781592e-06
1497 7.65912707145766e-06
1498 7.65437020178172e-06
1499 7.65677920455676e-06
1500 7.65917195999499e-06
1501 7.65694416227092e-06
1502 7.67370619092489e-06
1503 7.69139949508002e-06
1504 7.80233524033669e-06
1505 8.05418538352143e-06
1506 9.12669100017194e-06
1507 1.16172563203731e-05
1508 1.44420606611462e-05
1509 1.15051119914389e-05
1510 9.50072484101616e-06
1511 7.765925918779e-06
1512 7.76296505566876e-06
1513 7.64956709009557e-06
1514 7.65420213921786e-06
1515 7.63237018330809e-06
1516 7.63525090674566e-06
1517 7.63116286839338e-06
1518 7.62702517953517e-06
1519 7.62102203655246e-06
1520 7.62339165394366e-06
1521 7.61664731152578e-06
1522 7.61809516780509e-06
1523 7.61657931835913e-06
1524 7.63238262579957e-06
1525 7.66775976102085e-06
1526 7.78197489148535e-06
1527 8.13657021275782e-06
1528 9.45523903994605e-06
1529 1.25984254030209e-05
1530 1.45509977176883e-05
1531 1.05340316497449e-05
1532 8.63223999875151e-06
1533 7.63348646815043e-06
1534 7.66931998941178e-06
1535 7.61043810237894e-06
1536 7.60779376540732e-06
1537 7.59568800923205e-06
1538 7.59586575949101e-06
1539 7.59187262966421e-06
1540 7.58546391110837e-06
1541 7.58563309011251e-06
1542 7.57740128054518e-06
1543 7.58423622626569e-06
1544 7.57640526094505e-06
1545 7.575045975039e-06
1546 7.58202830064558e-06
1547 7.60320800474723e-06
1548 7.67531412348976e-06
1549 7.93359721029674e-06
1550 8.95290791430581e-06
1551 1.19554974430114e-05
1552 1.52763810299383e-05
1553 1.15110048974354e-05
1554 8.90253434565613e-06
1555 7.6068463498169e-06
1556 7.64250837459457e-06
1557 7.57457879041468e-06
1558 7.57476242940758e-06
1559 7.56432903137316e-06
1560 7.56064918905341e-06
1561 7.55512351580112e-06
1562 7.54516273171468e-06
1563 7.55286185094572e-06
1564 7.54974448580192e-06
1565 7.54158295734886e-06
1566 7.54229531330086e-06
1567 7.54294827198798e-06
1568 7.53810697384782e-06
1569 7.53908311512674e-06
1570 7.53582577694978e-06
1571 7.56218780140205e-06
1572 7.61389566916737e-06
1573 7.86968469712601e-06
1574 9.01026574845476e-06
1575 1.29970523709e-05
1576 1.67778265483531e-05
1577 1.12897232412701e-05
1578 8.40280050729003e-06
1579 7.53901812911018e-06
1580 7.56853824679382e-06
1581 7.54214530018782e-06
1582 7.53805276343389e-06
1583 7.53146145027195e-06
1584 7.52498869693863e-06
1585 7.52182992980188e-06
1586 7.51944391008674e-06
1587 7.51864572123839e-06
1588 7.51521238795405e-06
1589 7.50403708682512e-06
1590 7.51016251809844e-06
1591 7.50326595233552e-06
1592 7.50130779092473e-06
1593 7.49932499577444e-06
1594 7.5013788959355e-06
1595 7.49768707197784e-06
1596 7.49280432854604e-06
1597 7.48741201905823e-06
1598 7.49928888033047e-06
1599 7.49527779109194e-06
1600 7.5087706415955e-06
1601 7.56943871671467e-06
1602 8.07056056939537e-06
1603 1.08472540171212e-05
1604 1.97018624605583e-05
1605 1.62470624689159e-05
1606 8.19196526280486e-06
1607 7.66344883285086e-06
1608 7.54598684982177e-06
1609 7.52881669652261e-06
1610 7.52615951948865e-06
1611 7.51643460883411e-06
1612 7.50961062834232e-06
1613 7.49875116312637e-06
1614 7.49655408238681e-06
1615 7.49208831896109e-06
1616 7.48840744912993e-06
1617 7.48486596047204e-06
1618 7.47570500736483e-06
1619 7.47386563737518e-06
1620 7.4698876298962e-06
1621 7.47465600547947e-06
1622 7.46779963178845e-06
1623 7.46558358866167e-06
1624 7.46226600156774e-06
1625 7.47397772837832e-06
1626 7.47952404012153e-06
1627 7.59933828253789e-06
1628 7.87295520754405e-06
1629 9.13193062107576e-06
1630 1.23316677124174e-05
1631 1.51452551213349e-05
1632 1.15066630477223e-05
1633 8.10335336198786e-06
1634 7.61979862251572e-06
1635 7.47235179077155e-06
1636 7.48116638349394e-06
1637 7.46753227864883e-06
1638 7.46456624367564e-06
1639 7.45482947794862e-06
1640 7.45690610615135e-06
1641 7.45273004287839e-06
1642 7.44810320607581e-06
1643 7.44482055059503e-06
1644 7.43961045313046e-06
1645 7.43425958893429e-06
1646 7.43268600522651e-06
1647 7.43018000315399e-06
1648 7.43687530524451e-06
1649 7.43148895610002e-06
1650 7.42363219696784e-06
1651 7.42773974782285e-06
1652 7.41990437636009e-06
1653 7.44727309065496e-06
1654 7.47454241578627e-06
1655 7.84449773688634e-06
1656 9.4660388429979e-06
1657 1.61537802209621e-05
1658 1.76645104304485e-05
1659 9.54011346138151e-06
1660 7.78101197151937e-06
1661 7.44807408870063e-06
1662 7.4609824459726e-06
1663 7.4516252216994e-06
1664 7.44274070613171e-06
1665 7.43834725436265e-06
1666 7.43748398740474e-06
1667 7.42976228484959e-06
1668 7.42453266577847e-06
1669 7.41885445504353e-06
1670 7.41318203256736e-06
1671 7.41180195706903e-06
1672 7.40329343162927e-06
1673 7.41203401005697e-06
1674 7.40769004248243e-06
1675 7.40268712862147e-06
1676 7.39972981556036e-06
1677 7.39439481967086e-06
1678 7.41012851146472e-06
1679 7.40206007310285e-06
1680 7.50199098037418e-06
1681 7.67745065688441e-06
1682 8.90435766776054e-06
1683 1.20467240758981e-05
1684 1.60761477633331e-05
1685 1.18479717516051e-05
1686 7.94184326829983e-06
1687 7.56115100852828e-06
1688 7.41975488727231e-06
1689 7.42338303433288e-06
1690 7.41291165251567e-06
1691 7.40477065397371e-06
1692 7.40398277293552e-06
1693 7.39326997933887e-06
1694 7.39601349941665e-06
1695 7.39243754832586e-06
1696 7.38425910185025e-06
1697 7.38259663124463e-06
1698 7.37953705459704e-06
1699 7.37423082031174e-06
1700 7.3718696583791e-06
1701 7.36536183465475e-06
1702 7.3838602454579e-06
1703 7.3651156514698e-06
1704 7.43706579892089e-06
1705 7.49985777570572e-06
1706 8.14451226094359e-06
1707 9.9054302130086e-06
1708 1.47843156439731e-05
1709 1.41628984824949e-05
1710 9.25118168121353e-06
1711 7.7137415461781e-06
1712 7.39211562755804e-06
1713 7.39975240177104e-06
1714 7.38696419416707e-06
1715 7.38051698839781e-06
1716 7.38153648788265e-06
1717 7.37000315353953e-06
1718 7.37273947370998e-06
1719 7.36395646933063e-06
1720 7.35955817177114e-06
1721 7.36024029390769e-06
1722 7.35776047283698e-06
1723 7.34838026572593e-06
1724 7.35698105458482e-06
1725 7.34495430976256e-06
1726 7.36217976221587e-06
1727 7.3389799281065e-06
1728 7.39473326449858e-06
1729 7.40159334411405e-06
1730 7.78477921475895e-06
1731 8.71011705944058e-06
1732 1.27102078583685e-05
1733 1.52227084311285e-05
1734 1.11702486849663e-05
1735 8.12664215799064e-06
1736 7.3956879967918e-06
1737 7.39218592155089e-06
1738 7.37056347277498e-06
1739 7.36022790137625e-06
1740 7.35806614798884e-06
1741 7.34987108524932e-06
1742 7.3482148167936e-06
1743 7.33796492918826e-06
1744 7.34326356194304e-06
1745 7.3370585269017e-06
1746 7.33560971599712e-06
1747 7.32847507944889e-06
1748 7.33182344286876e-06
1749 7.31960238364859e-06
1750 7.33281135256725e-06
1751 7.31003688836385e-06
1752 7.35488860753586e-06
1753 7.34259896162381e-06
1754 7.61303846263939e-06
1755 8.11094181674132e-06
1756 1.10692681689339e-05
1757 1.47477925054595e-05
1758 1.3162966829211e-05
1759 8.80771077865772e-06
1760 7.44414777170688e-06
1761 7.38587123733092e-06
1762 7.34502521959612e-06
1763 7.34553498965562e-06
1764 7.33639615868587e-06
1765 7.3323098557232e-06
1766 7.32257014463e-06
1767 7.32328883296107e-06
1768 7.31955618704649e-06
1769 7.31569071771299e-06
1770 7.30559661665797e-06
1771 7.30700295004105e-06
1772 7.30809209348315e-06
1773 7.29400389132273e-06
1774 7.29713748742755e-06
1775 7.29581446479166e-06
1776 7.29763087936597e-06
1777 7.28495286028696e-06
1778 7.29601200177843e-06
1779 7.27535922695477e-06
1780 7.32256467766979e-06
1781 7.32084986709847e-06
1782 7.81989682069328e-06
1783 9.47737587803088e-06
1784 1.67005599120418e-05
1785 1.63809022670769e-05
1786 9.10550789440245e-06
1787 7.56176037208789e-06
1788 7.32878406456416e-06
1789 7.33692900561911e-06
1790 7.32632636984976e-06
1791 7.31971796735786e-06
1792 7.30992521996709e-06
1793 7.30793198494561e-06
1794 7.30409795357545e-06
1795 7.29156448681101e-06
1796 7.28925484744236e-06
1797 7.28672481753456e-06
1798 7.27760868596272e-06
1799 7.2771045728226e-06
1800 7.26349508145541e-06
1801 7.25856288269489e-06
1802 7.25499354209402e-06
1803 7.24670826435236e-06
1804 7.24610634400547e-06
1805 7.24198875756699e-06
1806 7.23031875760061e-06
1807 7.24729913581523e-06
1808 7.22446167733537e-06
1809 7.3306701871001e-06
1810 7.48322364219067e-06
1811 9.23031596228308e-06
1812 1.34387812413539e-05
1813 1.69657708726512e-05
1814 1.04793655900659e-05
1815 7.40530063869649e-06
1816 7.2770007555345e-06
1817 7.20208839716241e-06
1818 7.19317574643386e-06
1819 7.18137509808647e-06
1820 7.16981402892891e-06
1821 7.15823766617163e-06
1822 7.15038111986921e-06
1823 7.13847240102483e-06
1824 7.13471121310816e-06
1825 7.12115933598767e-06
1826 7.12015293796719e-06
1827 7.10808432363708e-06
1828 7.10893702904336e-06
1829 7.09383735114644e-06
1830 7.11308050721504e-06
1831 7.0772452596124e-06
1832 7.11578237039268e-06
1833 7.08186288389534e-06
1834 7.24504910820656e-06
1835 7.33403679908928e-06
1836 8.71148014719436e-06
1837 1.08267904858073e-05
1838 1.41753312880577e-05
1839 1.11773282600314e-05
1840 7.99993241257235e-06
1841 7.2866317036846e-06
1842 7.12571843902721e-06
1843 7.10459100100591e-06
1844 7.10690900357358e-06
1845 7.08871859650406e-06
1846 7.08622932910075e-06
1847 7.07509627995151e-06
1848 7.08275448074724e-06
1849 7.05847641130353e-06
1850 7.0718747768872e-06
1851 7.0408170368963e-06
1852 7.08577836799895e-06
1853 7.03264624424271e-06
1854 7.14987642380649e-06
1855 7.100399702753e-06
1856 7.57643654081264e-06
1857 7.89723143090892e-06
1858 1.02045067560397e-05
1859 1.12994984702852e-05
1860 1.15287234322414e-05
1861 8.58138332238134e-06
1862 7.54774291850113e-06
1863 7.14848820637659e-06
1864 7.11135415237818e-06
1865 7.05173641746271e-06
1866 7.0654713999474e-06
1867 7.03349800190711e-06
1868 7.04955271302321e-06
1869 7.02490496012897e-06
1870 7.0480896727787e-06
1871 7.00297593031474e-06
1872 7.04947741575479e-06
1873 6.99687824312623e-06
1874 7.08615603173257e-06
1875 7.00367684325354e-06
1876 7.26225773106659e-06
1877 7.30693927497583e-06
1878 8.68954178551151e-06
1879 1.00370839857256e-05
1880 1.26613214754545e-05
1881 1.05090310489886e-05
1882 8.23574570890662e-06
1883 7.29764691087542e-06
1884 7.10257629588096e-06
1885 7.03465378709689e-06
1886 7.05226748970444e-06
1887 7.00773355788442e-06
1888 7.03047334194018e-06
1889 6.99512076146602e-06
1890 7.03127907553114e-06
1891 6.98051497716179e-06
1892 7.01534781799573e-06
1893 6.97134274890709e-06
1894 7.0285792792002e-06
1895 6.96518364651233e-06
1896 7.14920735545022e-06
1897 7.09947285160162e-06
1898 7.94964676870125e-06
1899 8.62134116441649e-06
1900 1.18313515113377e-05
1901 1.12488652628806e-05
1902 9.49356630991494e-06
1903 7.66254510720099e-06
1904 7.19083048761338e-06
1905 7.02565534910216e-06
1906 7.04818257646345e-06
1907 6.9822635666017e-06
1908 7.01168055844414e-06
1909 6.97084993589447e-06
1910 6.99290718370049e-06
1911 6.96269907096703e-06
1912 6.97372433677845e-06
1913 6.95632266167756e-06
1914 6.96531947413925e-06
1915 6.95628309688168e-06
1916 6.94771020909313e-06
1917 6.94721066007986e-06
1918 6.94965002323578e-06
1919 6.926903113047e-06
1920 7.01565041022478e-06
1921 6.97095265678183e-06
1922 7.69321133753387e-06
1923 8.9392163740154e-06
1924 1.48267169389482e-05
1925 1.36872605878968e-05
1926 8.95694032593264e-06
1927 7.30021536654579e-06
1928 7.00856042873399e-06
1929 7.00575326545438e-06
1930 6.99801962428248e-06
1931 6.9903596341403e-06
1932 6.97984990327694e-06
1933 6.97854477421656e-06
1934 6.9657703040682e-06
1935 6.97151298473253e-06
1936 6.94964249914332e-06
1937 6.95965033764567e-06
1938 6.93209629237801e-06
1939 6.96789331933179e-06
1940 6.91565405619432e-06
1941 6.99351975869122e-06
1942 6.90857718704807e-06
1943 7.16155038898991e-06
1944 7.08142773780951e-06
1945 8.15556069616585e-06
1946 8.83365612142084e-06
1947 1.19835399360335e-05
1948 1.0545637486481e-05
1949 9.00373392176412e-06
1950 7.44069483971099e-06
1951 7.1900061600072e-06
1952 6.96765168167746e-06
1953 7.00771963457747e-06
1954 6.93360100989526e-06
1955 6.97828868739814e-06
1956 6.9191762793186e-06
1957 6.96296624458359e-06
1958 6.9058045296666e-06
1959 6.95367645991984e-06
1960 6.89902395961672e-06
1961 6.95302803532805e-06
1962 6.88255571690277e-06
1963 6.98504742041584e-06
1964 6.8857853406934e-06
1965 7.24865531487318e-06
1966 7.28822196116585e-06
1967 9.24158692561239e-06
1968 1.06143638275991e-05
1969 1.30062252537422e-05
1970 9.2980285508748e-06
1971 7.72989538638802e-06
1972 7.0436589219014e-06
1973 7.00931814756611e-06
1974 6.9370782180278e-06
1975 6.95976901093909e-06
1976 6.92356298831465e-06
1977 6.93857164929534e-06
1978 6.91454479817555e-06
1979 6.93168156357116e-06
1980 6.90445434664566e-06
1981 6.92341198949009e-06
1982 6.89154717348206e-06
1983 6.90464114333489e-06
1984 6.88904662776713e-06
1985 6.89329418923634e-06
1986 6.88533326842489e-06
1987 6.86674788830111e-06
1988 6.89229048339568e-06
1989 6.85623919943756e-06
1990 6.96219903767448e-06
1991 6.90995382279791e-06
1992 7.87927816731093e-06
1993 9.4807806159114e-06
1994 1.6139666729531e-05
1995 1.26778988254472e-05
1996 8.16657311425439e-06
1997 7.10638991680002e-06
1998 6.96532969335362e-06
1999 6.95732690547679e-06
};
\addlegendentry{Train}
\addplot [semithick, black]
table {%
0 0.00175222277175635
1 0.0017480734968558
2 0.00174395483918488
3 0.00173917540814728
4 0.0017323245992884
5 0.00172338495030999
6 0.00171246309764683
7 0.00169947603717446
8 0.00168274948373437
9 0.00166263955179602
10 0.00163420278113335
11 0.00159336091019213
12 0.00153792416676879
13 0.00146568357013166
14 0.00138178234919906
15 0.00128847267478704
16 0.00120096618775278
17 0.00111677125096321
18 0.00105203129351139
19 0.00100110564380884
20 0.000959507946390659
21 0.000922976294532418
22 0.000888904847670346
23 0.000855450343806297
24 0.000825307448394597
25 0.000796804262790829
26 0.000770472281146795
27 0.000743960146792233
28 0.000718007038813084
29 0.000693563604727387
30 0.000670653244014829
31 0.000649886031169444
32 0.000629784946795553
33 0.000609393697232008
34 0.000589608738664538
35 0.000571209820918739
36 0.000553340883925557
37 0.000535982311703265
38 0.000519325898494571
39 0.000503281189594418
40 0.00048807609709911
41 0.000474025669973344
42 0.000459869886981323
43 0.000444875098764896
44 0.000430673768278211
45 0.000417293544160202
46 0.000404453661758453
47 0.000392547197407112
48 0.000381453894078732
49 0.000371189729776233
50 0.000360967853339389
51 0.000351255323039368
52 0.000342053623171523
53 0.000332645315211266
54 0.000323815300362185
55 0.000315335608320311
56 0.000307914684526622
57 0.000301361986203119
58 0.000295544392429292
59 0.000289846590021625
60 0.000284048350295052
61 0.000278554420219734
62 0.000273341254796833
63 0.000268307834630832
64 0.000263542140601203
65 0.000258491869317368
66 0.000254285609116778
67 0.000250520475674421
68 0.00024723558453843
69 0.000244066643062979
70 0.000241167203057557
71 0.000238591281231493
72 0.000236141248024069
73 0.000233413506066427
74 0.000231072845053859
75 0.000228920878726058
76 0.000226558768190444
77 0.000224136965698563
78 0.000221832320676185
79 0.000219676658161916
80 0.000217490523937158
81 0.000215513704461046
82 0.000213318606256507
83 0.000210628029890358
84 0.000208226076210849
85 0.000206224052817561
86 0.000204341151402332
87 0.000202332492335699
88 0.000199971807887778
89 0.000197912493604235
90 0.000196210254216567
91 0.000194391919649206
92 0.000192696112208068
93 0.000190610720892437
94 0.00018851799541153
95 0.000186570527148433
96 0.000184904551133513
97 0.00018326255667489
98 0.000181646959390491
99 0.000180080067366362
100 0.000178518210304901
101 0.000176972534973174
102 0.000175615350599401
103 0.000174099754076451
104 0.000172813393874094
105 0.000171261301147752
106 0.000169666018337011
107 0.000168115628184751
108 0.000166553247254342
109 0.000164914366905577
110 0.000163565695402212
111 0.000162053271196783
112 0.000160680618137121
113 0.000159318413352594
114 0.000157912363647483
115 0.000156415713718161
116 0.000154939771164209
117 0.000153551140101627
118 0.00015203221119009
119 0.000150498963193968
120 0.000148548919241875
121 0.000146727601531893
122 0.000144975841976702
123 0.000143376950290985
124 0.00014176890545059
125 0.000139704861794598
126 0.000137617636937648
127 0.000135894981212914
128 0.000134284651721828
129 0.000132743210997432
130 0.000131192166008987
131 0.000129631749587134
132 0.000128102008602582
133 0.000126572354929522
134 0.000125044127344154
135 0.000123547404655255
136 0.00012204801896587
137 0.000120548233098816
138 0.000118950840260368
139 0.000117187424621079
140 0.000115466507850215
141 0.000113952228275593
142 0.000112603258457966
143 0.000111062741780188
144 0.000109854670881759
145 0.000108623811684083
146 0.000107335872598924
147 0.000106086445157416
148 0.000104803548310883
149 0.000103656733699609
150 0.00010250078048557
151 0.000101373632787727
152 0.00010026982636191
153 9.9087817943655e-05
154 9.80399854597636e-05
155 9.69914981396869e-05
156 9.5970528491307e-05
157 9.49609093368053e-05
158 9.38624143600464e-05
159 9.2922397016082e-05
160 9.19573212740943e-05
161 9.10232993192039e-05
162 9.0064451796934e-05
163 8.91786912688985e-05
164 8.82989043020643e-05
165 8.74506367836148e-05
166 8.66468180902302e-05
167 8.58673520269804e-05
168 8.51135337143205e-05
169 8.43842281028628e-05
170 8.36677645565942e-05
171 8.29897180665284e-05
172 8.23226655484177e-05
173 8.15497769508511e-05
174 8.08503100415692e-05
175 8.02141294116154e-05
176 7.95603118604049e-05
177 7.89073455962352e-05
178 7.8328033850994e-05
179 7.77920795371756e-05
180 7.72530256654136e-05
181 7.66978919273242e-05
182 7.61917908675969e-05
183 7.56868903408758e-05
184 7.51209008740261e-05
185 7.46410441934131e-05
186 7.41704934625886e-05
187 7.36731017241254e-05
188 7.32255793991499e-05
189 7.27753358660266e-05
190 7.22967015462928e-05
191 7.18744850018993e-05
192 7.14681955287233e-05
193 7.10315653122962e-05
194 7.06443461240269e-05
195 7.01895260135643e-05
196 6.98063377058133e-05
197 6.94024856784381e-05
198 6.90252491040155e-05
199 6.86347848386504e-05
200 6.82492755004205e-05
201 6.78754659020342e-05
202 6.75100236549042e-05
203 6.71231100568548e-05
204 6.67595813865773e-05
205 6.63677783450112e-05
206 6.60493024042808e-05
207 6.56751581118442e-05
208 6.54081086395308e-05
209 6.49862049613148e-05
210 6.47439737804234e-05
211 6.43610546831042e-05
212 6.40444632153958e-05
213 6.37380799162202e-05
214 6.34760508546606e-05
215 6.31413277005777e-05
216 6.28363268333487e-05
217 6.25523534836248e-05
218 6.22215666226111e-05
219 6.19471175014041e-05
220 6.15994504187256e-05
221 6.13245283602737e-05
222 6.10366478213109e-05
223 6.06851099291816e-05
224 6.04247543378733e-05
225 6.00929961365182e-05
226 5.98545484535862e-05
227 5.95130732108373e-05
228 5.92812029935885e-05
229 5.8966896176571e-05
230 5.87346548854839e-05
231 5.84250010433607e-05
232 5.81601962039713e-05
233 5.78787985432427e-05
234 5.76667407585774e-05
235 5.73966790398117e-05
236 5.71093078178819e-05
237 5.68993782508187e-05
238 5.66569688089658e-05
239 5.64274669159204e-05
240 5.61780616408214e-05
241 5.59433319722302e-05
242 5.57343773834873e-05
243 5.54847938474268e-05
244 5.52984056412242e-05
245 5.50157419638708e-05
246 5.48997923033312e-05
247 5.45358889212366e-05
248 5.45338734809775e-05
249 5.40367109351791e-05
250 5.43111673323438e-05
251 5.33305283170193e-05
252 5.42011948709842e-05
253 5.19859313499182e-05
254 5.52259189134929e-05
255 5.11651451233774e-05
256 5.40882210771088e-05
257 5.17024564032909e-05
258 5.26729090779554e-05
259 5.19507157150656e-05
260 5.21710389875807e-05
261 5.17384032718837e-05
262 5.18238048243802e-05
263 5.15215688210446e-05
264 5.15448446094524e-05
265 5.13038976350799e-05
266 5.12599690409843e-05
267 5.107434844831e-05
268 5.09758938278537e-05
269 5.08412813360337e-05
270 5.07547119923402e-05
271 5.05892749060877e-05
272 5.05357056681532e-05
273 5.03649716847576e-05
274 5.0312839448452e-05
275 5.01308386446908e-05
276 5.01142494613305e-05
277 4.98409244755749e-05
278 4.99841626151465e-05
279 4.952773815603e-05
280 4.989772423869e-05
281 4.87928664369974e-05
282 5.07953991473187e-05
283 4.69913102278952e-05
284 5.17059088451788e-05
285 4.75843880849425e-05
286 4.89115154778119e-05
287 4.82965078845154e-05
288 4.84944575873669e-05
289 4.82060604554135e-05
290 4.81996539747342e-05
291 4.80351627629716e-05
292 4.79559275845531e-05
293 4.77643989142962e-05
294 4.77741305076052e-05
295 4.75714950880501e-05
296 4.75722117698751e-05
297 4.73787695227657e-05
298 4.73566651635338e-05
299 4.71807870781049e-05
300 4.71952480438631e-05
301 4.69905135105364e-05
302 4.70342383778188e-05
303 4.67600511910859e-05
304 4.68740327050909e-05
305 4.6499480959028e-05
306 4.68637263111304e-05
307 4.59273978776764e-05
308 4.69250153400935e-05
309 4.50420266133733e-05
310 4.84381198475603e-05
311 4.43508542957716e-05
312 4.74844528071117e-05
313 4.5189015509095e-05
314 4.60281080449931e-05
315 4.55343433714006e-05
316 4.57682108390145e-05
317 4.54873443231918e-05
318 4.55688750662375e-05
319 4.54015353170689e-05
320 4.53968459623866e-05
321 4.53114771516994e-05
322 4.52645617770031e-05
323 4.51700907433406e-05
324 4.51700434496161e-05
325 4.50301231467165e-05
326 4.50872685178183e-05
327 4.48567734565586e-05
328 4.49751714768354e-05
329 4.4705633627018e-05
330 4.48771388619207e-05
331 4.44706129201222e-05
332 4.48948921984993e-05
333 4.38036440755241e-05
334 4.50841980637051e-05
335 4.29556421295274e-05
336 4.68163198092952e-05
337 4.28537459811196e-05
338 4.48568061983678e-05
339 4.35856527474243e-05
340 4.40585317846853e-05
341 4.37213420809712e-05
342 4.38634415331762e-05
343 4.36893024016172e-05
344 4.37001617683563e-05
345 4.36244663433172e-05
346 4.35718866356183e-05
347 4.34995599789545e-05
348 4.35127149103209e-05
349 4.33350032835733e-05
350 4.34000612585805e-05
351 4.32779270340689e-05
352 4.32432534580585e-05
353 4.31522166763898e-05
354 4.3152740545338e-05
355 4.29520077886991e-05
356 4.3144649680471e-05
357 4.25819780502934e-05
358 4.30874315497931e-05
359 4.20171345467679e-05
360 4.48017053713556e-05
361 4.07884217565879e-05
362 4.38386487076059e-05
363 4.17940827901475e-05
364 4.27558006776962e-05
365 4.2131454392802e-05
366 4.23081728513353e-05
367 4.21472177549731e-05
368 4.21868426201399e-05
369 4.20511460106354e-05
370 4.20678188675083e-05
371 4.19964599132072e-05
372 4.1937473724829e-05
373 4.18746094510425e-05
374 4.1861418139888e-05
375 4.17500596086029e-05
376 4.17665942222811e-05
377 4.16581751778722e-05
378 4.1624967707321e-05
379 4.15437571064103e-05
380 4.16047623730265e-05
381 4.13824600400403e-05
382 4.15813628933392e-05
383 4.11116743634921e-05
384 4.16873372159898e-05
385 4.02766381739639e-05
386 4.32318192906678e-05
387 3.95014249079395e-05
388 4.28368330176454e-05
389 4.05555183533579e-05
390 4.10724860557821e-05
391 4.07166953664273e-05
392 4.08648593293037e-05
393 4.07376464863773e-05
394 4.07059451390523e-05
395 4.06784238293767e-05
396 4.06474173360039e-05
397 4.05832179239951e-05
398 4.05824030167423e-05
399 4.04935817641672e-05
400 4.05067985411733e-05
401 4.04091770178638e-05
402 4.03815902245697e-05
403 4.03234262194019e-05
404 4.03353187721223e-05
405 4.01649522245862e-05
406 4.03091617044993e-05
407 4.00554599764291e-05
408 4.02557852794416e-05
409 3.96647847082932e-05
410 4.02648947783746e-05
411 3.89673768950161e-05
412 4.25649704993702e-05
413 3.85246567020658e-05
414 4.09274507546797e-05
415 3.92632464354392e-05
416 3.98505217162892e-05
417 3.95122187910601e-05
418 3.96859322790988e-05
419 3.95497627323493e-05
420 3.95846072933637e-05
421 3.95134011341725e-05
422 3.94826201954857e-05
423 3.94461276300717e-05
424 3.94777453038841e-05
425 3.93395348510239e-05
426 3.94067246816121e-05
427 3.92989277315792e-05
428 3.93147456634324e-05
429 3.91894645872526e-05
430 3.93468289985321e-05
431 3.90249078918714e-05
432 3.93426962546073e-05
433 3.86284737032838e-05
434 3.94170165236574e-05
435 3.81384525098838e-05
436 4.14334717788734e-05
437 3.77545802621171e-05
438 3.99599957745522e-05
439 3.83776532544289e-05
440 3.89953856938519e-05
441 3.86375832022168e-05
442 3.87943255191203e-05
443 3.86655337933917e-05
444 3.87519176001661e-05
445 3.86014507967047e-05
446 3.86547508242074e-05
447 3.85716775781475e-05
448 3.85907842428423e-05
449 3.85103558073752e-05
450 3.84926715923939e-05
451 3.84015620511491e-05
452 3.84867526008748e-05
453 3.82761572836898e-05
454 3.84407867386471e-05
455 3.81348763767164e-05
456 3.84980776289012e-05
457 3.75721392629202e-05
458 3.87184554710984e-05
459 3.70340931112878e-05
460 4.0678249206394e-05
461 3.72355643776245e-05
462 3.8687347114319e-05
463 3.76466014131438e-05
464 3.79347984562628e-05
465 3.77608303097077e-05
466 3.7867921491852e-05
467 3.77496726287063e-05
468 3.7777117540827e-05
469 3.76812640752178e-05
470 3.77057731384411e-05
471 3.75965246348642e-05
472 3.76360840164125e-05
473 3.75581184925977e-05
474 3.76159587176517e-05
475 3.75008676201105e-05
476 3.7558169424301e-05
477 3.74260962416884e-05
478 3.7494974094443e-05
479 3.72407739632763e-05
480 3.75335730495863e-05
481 3.67648099199869e-05
482 3.7715733924415e-05
483 3.61207530659158e-05
484 3.9979800931178e-05
485 3.64542029274162e-05
486 3.7817626434844e-05
487 3.687190474011e-05
488 3.70572015526704e-05
489 3.68934634025209e-05
490 3.69502995454241e-05
491 3.68090331903659e-05
492 3.68811524822377e-05
493 3.68050568795297e-05
494 3.68423825420905e-05
495 3.67588036169764e-05
496 3.67667053069454e-05
497 3.66956992365886e-05
498 3.67284701496828e-05
499 3.65775158570614e-05
500 3.66515196219552e-05
501 3.65417581633665e-05
502 3.66530985047575e-05
503 3.64207444363274e-05
504 3.66535641660448e-05
505 3.58804827556014e-05
506 3.68471955880523e-05
507 3.53364303009585e-05
508 3.90661007259041e-05
509 3.55885676981416e-05
510 3.68757282558363e-05
511 3.59905643563252e-05
512 3.61662787327077e-05
513 3.60254598490428e-05
514 3.60773956344929e-05
515 3.59729747287929e-05
516 3.59932309947908e-05
517 3.59307159669697e-05
518 3.59474688593764e-05
519 3.58635770680849e-05
520 3.58850884367712e-05
521 3.58152246917598e-05
522 3.5791439586319e-05
523 3.57356475433335e-05
524 3.57620483555365e-05
525 3.56614837073721e-05
526 3.57435528712813e-05
527 3.55564006895293e-05
528 3.57762073690537e-05
529 3.51248563674744e-05
530 3.58204997610301e-05
531 3.44926083926111e-05
532 3.83168371627107e-05
533 3.46370070474222e-05
534 3.60730700776912e-05
535 3.51152120856568e-05
536 3.52854185621254e-05
537 3.51586240867618e-05
538 3.51882372342516e-05
539 3.51024900737684e-05
540 3.50535192410462e-05
541 3.50115333276335e-05
542 3.50306727341376e-05
543 3.49607507814653e-05
544 3.49552756233606e-05
545 3.49128640664276e-05
546 3.4885153581854e-05
547 3.48212633980438e-05
548 3.48677021975163e-05
549 3.47167115251068e-05
550 3.47738605341874e-05
551 3.46583074133378e-05
552 3.47728964698035e-05
553 3.44606851285789e-05
554 3.47247878380585e-05
555 3.39021062245592e-05
556 3.64783700206317e-05
557 3.32623312715441e-05
558 3.52396018570289e-05
559 3.40106635121629e-05
560 3.43532628903631e-05
561 3.421009023441e-05
562 3.42609637300484e-05
563 3.41679733537603e-05
564 3.41200648108497e-05
565 3.40881451847963e-05
566 3.40605110977776e-05
567 3.39916841767263e-05
568 3.39013204211369e-05
569 3.39070393238217e-05
570 3.38594654749613e-05
571 3.38268910127226e-05
572 3.38288227794692e-05
573 3.37786332238466e-05
574 3.37424135068431e-05
575 3.36883385898545e-05
576 3.36955090460833e-05
577 3.36390767188277e-05
578 3.36162374878768e-05
579 3.35716758854687e-05
580 3.35281001753174e-05
581 3.35000549966935e-05
582 3.34792712237686e-05
583 3.3457916288171e-05
584 3.34003270836547e-05
585 3.33455536747351e-05
586 3.32185336446855e-05
587 3.3052758226404e-05
588 3.2334792194888e-05
589 3.5320692404639e-05
590 3.20634244417306e-05
591 3.37409728672355e-05
592 3.31906085193623e-05
593 3.29971408064011e-05
594 3.29167341988068e-05
595 3.28549431287684e-05
596 3.28394526150078e-05
597 3.2806728995638e-05
598 3.27651250700001e-05
599 3.27531997754704e-05
600 3.27384805132169e-05
601 3.26677800330799e-05
602 3.26174922520295e-05
603 3.25939618051052e-05
604 3.25633918691892e-05
605 3.25474538840353e-05
606 3.25397559208795e-05
607 3.24512875522487e-05
608 3.24711545545142e-05
609 3.24166539940052e-05
610 3.23876374750398e-05
611 3.23122221743688e-05
612 3.23379681503866e-05
613 3.22416817652993e-05
614 3.22703490382992e-05
615 3.21102779707871e-05
616 3.22257983498275e-05
617 3.19031132676173e-05
618 3.21842526318505e-05
619 3.1263378332369e-05
620 3.54712537955493e-05
621 3.11845215037465e-05
622 3.29276263073552e-05
623 3.19076498271897e-05
624 3.1990453862818e-05
625 3.17795638693497e-05
626 3.17373087455053e-05
627 3.1698597013019e-05
628 3.16796940751374e-05
629 3.16288533213083e-05
630 3.1634252081858e-05
631 3.16097248287406e-05
632 3.15651632263325e-05
633 3.15364668495022e-05
634 3.15106371999718e-05
635 3.1486797524849e-05
636 3.14625285682268e-05
637 3.13856726279482e-05
638 3.14341486955527e-05
639 3.12860611302312e-05
640 3.13488999381661e-05
641 3.11583971779328e-05
642 3.13109230773989e-05
643 3.09061870211735e-05
644 3.20898179779761e-05
645 3.01299205602845e-05
646 3.32194504153449e-05
647 3.04461027553771e-05
648 3.13516975438688e-05
649 3.10605137201492e-05
650 3.11954900098499e-05
651 3.10346877085976e-05
652 3.10345967591275e-05
653 3.09521019516978e-05
654 3.09517017740291e-05
655 3.08862436213531e-05
656 3.08506823785137e-05
657 3.08219860016834e-05
658 3.08091439364944e-05
659 3.07562659145333e-05
660 3.06990514218342e-05
661 3.06486908812076e-05
662 3.06050169456284e-05
663 3.063904659939e-05
664 3.04753775708377e-05
665 3.05865578411613e-05
666 3.03346187138231e-05
667 3.05102421407355e-05
668 3.01401250908384e-05
669 3.23113745253067e-05
670 2.92713029921288e-05
671 3.19905193464365e-05
672 2.97490951197688e-05
673 3.03302585962228e-05
674 3.01485888485331e-05
675 3.01609561574878e-05
676 3.0095719921519e-05
677 3.00932988466229e-05
678 3.00349074677797e-05
679 3.00486044579884e-05
680 3.0009283364052e-05
681 2.99587245535804e-05
682 2.994905844389e-05
683 2.98879967886023e-05
684 2.99138482660055e-05
685 2.98233626381261e-05
686 2.98360937449615e-05
687 2.96911057375837e-05
688 2.9756803996861e-05
689 2.96443213301245e-05
690 2.97188871627441e-05
691 2.95033332804451e-05
692 2.96501220873324e-05
693 2.90516636596294e-05
694 3.00500505545642e-05
695 2.8623902835534e-05
696 3.23246676998679e-05
697 2.92780514428159e-05
698 2.99709736282239e-05
699 2.93607099592919e-05
700 2.93825651169755e-05
701 2.92645745503251e-05
702 2.92562708636979e-05
703 2.91812593786744e-05
704 2.91457818093477e-05
705 2.91510896204272e-05
706 2.91252636088757e-05
707 2.90836214844603e-05
708 2.90401785605354e-05
709 2.90174630208639e-05
710 2.89961026282981e-05
711 2.89842719212174e-05
712 2.89249892375665e-05
713 2.89442286884878e-05
714 2.87565289909253e-05
715 2.88773881038651e-05
716 2.86142694676528e-05
717 2.88669689325616e-05
718 2.83008648693794e-05
719 3.11223993776366e-05
720 2.76967039098963e-05
721 3.01530653814552e-05
722 2.83834633592051e-05
723 2.8696158551611e-05
724 2.85893947875593e-05
725 2.8560905775521e-05
726 2.85125934169628e-05
727 2.85314799839398e-05
728 2.84726284007775e-05
729 2.84638990706299e-05
730 2.8420758098946e-05
731 2.83932731690584e-05
732 2.83755107375327e-05
733 2.84118159470381e-05
734 2.83255085378187e-05
735 2.83460722130258e-05
736 2.83388362731785e-05
737 2.83191784546943e-05
738 2.82786804746138e-05
739 2.82844412140548e-05
740 2.8243004635442e-05
741 2.82596229226328e-05
742 2.81932934740325e-05
743 2.8198754080222e-05
744 2.81148131762166e-05
745 2.81023130810354e-05
746 2.77232247754e-05
747 2.84492434730055e-05
748 2.73547884717118e-05
749 3.12250485876575e-05
750 2.82837918348378e-05
751 2.85963251371868e-05
752 2.82346900348784e-05
753 2.80779113381868e-05
754 2.79508694802644e-05
755 2.79179748758906e-05
756 2.78851020993898e-05
757 2.78946154139703e-05
758 2.78581555903656e-05
759 2.7880816560355e-05
760 2.78485404123785e-05
761 2.7829484679387e-05
762 2.78075858659577e-05
763 2.7788286388386e-05
764 2.77559865935473e-05
765 2.77830622508191e-05
766 2.77460003417218e-05
767 2.77213093795581e-05
768 2.77009239653125e-05
769 2.76808877970325e-05
770 2.76785613095853e-05
771 2.7609008611762e-05
772 2.76690207101637e-05
773 2.75713500741404e-05
774 2.78761435765773e-05
775 2.68776166194584e-05
776 3.10224204440601e-05
777 2.67526265815832e-05
778 2.8404152544681e-05
779 2.76070222753333e-05
780 2.75651837000623e-05
781 2.74426838586805e-05
782 2.73599707725225e-05
783 2.74140074907336e-05
784 2.74010526482016e-05
785 2.73110435955459e-05
786 2.73797213594662e-05
787 2.73758323601214e-05
788 2.7369993404136e-05
789 2.73098812613171e-05
790 2.73190162261017e-05
791 2.72856505034724e-05
792 2.72597462753765e-05
793 2.72601446340559e-05
794 2.72462366410764e-05
795 2.72179113380844e-05
796 2.72103807219537e-05
797 2.72231754934182e-05
798 2.71659719146555e-05
799 2.71526605502004e-05
800 2.71005628746934e-05
801 2.71630851784721e-05
802 2.71161607088288e-05
803 2.72144752671011e-05
804 2.67674804490525e-05
805 2.89131039608037e-05
806 2.63191723206546e-05
807 2.82345645246096e-05
808 2.69158172159223e-05
809 2.70620275841793e-05
810 2.7066700567957e-05
811 2.701682933548e-05
812 2.70220607490046e-05
813 2.69820957328193e-05
814 2.69761076197028e-05
815 2.69536540145054e-05
816 2.69151769316522e-05
817 2.69241572823375e-05
818 2.68687545030843e-05
819 2.68655712716281e-05
820 2.683912498469e-05
821 2.68070580204949e-05
822 2.67854320554761e-05
823 2.67587365669897e-05
824 2.67519262706628e-05
825 2.67166360572446e-05
826 2.67427258222597e-05
827 2.66943261522101e-05
828 2.66768747678725e-05
829 2.66220558842178e-05
830 2.66213282884564e-05
831 2.66497791017173e-05
832 2.66023689619033e-05
833 2.6521480322117e-05
834 2.64577083726181e-05
835 2.61258210230153e-05
836 3.02675962302601e-05
837 2.59149692283245e-05
838 2.750151361397e-05
839 2.66627430391964e-05
840 2.66044753516326e-05
841 2.65053095063195e-05
842 2.65375820163172e-05
843 2.65318503807066e-05
844 2.6485280613997e-05
845 2.64569607679732e-05
846 2.64487862295937e-05
847 2.64354075625306e-05
848 2.64227710431442e-05
849 2.6417135813972e-05
850 2.63530837401049e-05
851 2.63397450908087e-05
852 2.63314541371074e-05
853 2.63384008576395e-05
854 2.63191268459195e-05
855 2.62838439084589e-05
856 2.62869343714556e-05
857 2.62561370618641e-05
858 2.62149951595347e-05
859 2.62475059571443e-05
860 2.62103349086829e-05
861 2.61931236309465e-05
862 2.61682744167047e-05
863 2.61406003119191e-05
864 2.61371242231689e-05
865 2.6062047254527e-05
866 2.58969575952506e-05
867 2.61542190855835e-05
868 2.56819694186561e-05
869 2.96542748401407e-05
870 2.66617080342257e-05
871 2.67152663582237e-05
872 2.64364316535648e-05
873 2.62484445556765e-05
874 2.6115303626284e-05
875 2.60198339674389e-05
876 2.59228854702087e-05
877 2.58855434367433e-05
878 2.58806394413114e-05
879 2.5848963559838e-05
880 2.58335458056536e-05
881 2.58315048995428e-05
882 2.58191575994715e-05
883 2.57632254943019e-05
884 2.57391275226837e-05
885 2.56998064287473e-05
886 2.56948187598027e-05
887 2.56708372035064e-05
888 2.56069870374631e-05
889 2.55966951954179e-05
890 2.5552873921697e-05
891 2.55577087955317e-05
892 2.54934366239468e-05
893 2.55183040280826e-05
894 2.54335991485277e-05
895 2.56992516369792e-05
896 2.48133092100034e-05
897 2.84321176877711e-05
898 2.46333911491092e-05
899 2.63730453298194e-05
900 2.55153208854608e-05
901 2.54467922786716e-05
902 2.53701018664287e-05
903 2.53224334301194e-05
904 2.53201651503332e-05
905 2.52613772318e-05
906 2.5272664061049e-05
907 2.52066583925625e-05
908 2.52432128036162e-05
909 2.51881538133603e-05
910 2.51933324761922e-05
911 2.5164028556901e-05
912 2.51330602623057e-05
913 2.51265282713575e-05
914 2.51292312896112e-05
915 2.51037836278556e-05
916 2.50704160862369e-05
917 2.50552911893465e-05
918 2.50396551564336e-05
919 2.50387038249755e-05
920 2.49282511504134e-05
921 2.50167122430867e-05
922 2.47336411121069e-05
923 2.50944194704061e-05
924 2.43806607613806e-05
925 2.83580229734071e-05
926 2.46537383645773e-05
927 2.58109503192827e-05
928 2.50341381615726e-05
929 2.50602388405241e-05
930 2.48535052378429e-05
931 2.47906245931517e-05
932 2.47865937126335e-05
933 2.47565330937505e-05
934 2.46983508986887e-05
935 2.47236130235251e-05
936 2.47061161644524e-05
937 2.46815816353774e-05
938 2.46402832999593e-05
939 2.46698746195761e-05
940 2.46037870965665e-05
941 2.45911887759576e-05
942 2.45177961915033e-05
943 2.45820647251094e-05
944 2.45252740569413e-05
945 2.45306891883956e-05
946 2.44455623032991e-05
947 2.45245992118726e-05
948 2.41222605836811e-05
949 2.48805481533054e-05
950 2.39038472500397e-05
951 2.74326976068551e-05
952 2.45863084273878e-05
953 2.50822122325189e-05
954 2.46311810769839e-05
955 2.45504870690638e-05
956 2.44219500018517e-05
957 2.43799750023754e-05
958 2.43236027017701e-05
959 2.4318183932337e-05
960 2.42946043726988e-05
961 2.42880050791427e-05
962 2.42254609474912e-05
963 2.42569076362997e-05
964 2.4199289327953e-05
965 2.41726684180321e-05
966 2.41648358496604e-05
967 2.4182656488847e-05
968 2.41192683461122e-05
969 2.41174966504332e-05
970 2.4059971110546e-05
971 2.41400884988252e-05
972 2.39843957388075e-05
973 2.44644670601701e-05
974 2.33839418797288e-05
975 2.70134642050834e-05
976 2.33714126807172e-05
977 2.51690707955277e-05
978 2.41344332607696e-05
979 2.4189832402044e-05
980 2.40304907492828e-05
981 2.39521687035449e-05
982 2.38943412114168e-05
983 2.39433375099907e-05
984 2.38978554989444e-05
985 2.39261189562967e-05
986 2.38509674090892e-05
987 2.38969532802003e-05
988 2.38046814047266e-05
989 2.38632292166585e-05
990 2.37592212215532e-05
991 2.38185839407379e-05
992 2.37282201851485e-05
993 2.37698423006805e-05
994 2.36816940741846e-05
995 2.36686428252142e-05
996 2.35977204283699e-05
997 2.36073883570498e-05
998 2.34769631788367e-05
999 2.36101986956783e-05
1000 2.29639572353335e-05
1001 2.58863565250067e-05
1002 2.27711680054199e-05
1003 2.4390326871071e-05
1004 2.34459512284957e-05
1005 2.36923733609729e-05
1006 2.3590080672875e-05
1007 2.35250699915923e-05
1008 2.34899125644006e-05
1009 2.34972176258452e-05
1010 2.34458238992374e-05
1011 2.33758273679996e-05
1012 2.33987011597492e-05
1013 2.33460687013576e-05
1014 2.33169266721234e-05
1015 2.33072896662634e-05
1016 2.32457878155401e-05
1017 2.3257764041773e-05
1018 2.32432339544175e-05
1019 2.3209506252897e-05
1020 2.3201528165373e-05
1021 2.31778594752541e-05
1022 2.31396570598008e-05
1023 2.31104240810964e-05
1024 2.31167432502843e-05
1025 2.30583846132504e-05
1026 2.30037858273135e-05
1027 2.29843590204837e-05
1028 2.25103067350574e-05
1029 2.56569619523361e-05
1030 2.22243597818306e-05
1031 2.35510397033067e-05
1032 2.31804369832389e-05
1033 2.33518603636185e-05
1034 2.32697202591226e-05
1035 2.32406437135069e-05
1036 2.32168949878542e-05
1037 2.31870890274877e-05
1038 2.32087168114958e-05
1039 2.31681970035424e-05
1040 2.31194226216758e-05
1041 2.31336161959916e-05
1042 2.31028170674108e-05
1043 2.30977202591021e-05
1044 2.30412952078041e-05
1045 2.30559999181423e-05
1046 2.30254227062687e-05
1047 2.30024197662715e-05
1048 2.29510442295577e-05
1049 2.29488105105702e-05
1050 2.28979552048258e-05
1051 2.28856952162459e-05
1052 2.28740900638513e-05
1053 2.28152784984559e-05
1054 2.26708561967826e-05
1055 2.27343189180829e-05
1056 2.21640020754421e-05
1057 2.65100352407899e-05
1058 2.24193772737635e-05
1059 2.3651004084968e-05
1060 2.29272445722017e-05
1061 2.29084453167161e-05
1062 2.27031632675789e-05
1063 2.26090487558395e-05
1064 2.25243420572951e-05
1065 2.24994928430533e-05
1066 2.24882423935924e-05
1067 2.24674276978476e-05
1068 2.23973620450124e-05
1069 2.24074738071067e-05
1070 2.23568367800908e-05
1071 2.23639344767435e-05
1072 2.23470888158772e-05
1073 2.22811795538291e-05
1074 2.23060778807849e-05
1075 2.22759645112092e-05
1076 2.22669095819583e-05
1077 2.22081653191708e-05
1078 2.22019061766332e-05
1079 2.201591632911e-05
1080 2.21236514335033e-05
1081 2.16798453038791e-05
1082 2.54158549068961e-05
1083 2.16854350583162e-05
1084 2.3016526029096e-05
1085 2.23014594666893e-05
1086 2.23174629354617e-05
1087 2.2132115191198e-05
1088 2.20458678086288e-05
1089 2.1953212126391e-05
1090 2.19168450712459e-05
1091 2.19092562474543e-05
1092 2.18866352952318e-05
1093 2.184589720855e-05
1094 2.18046934605809e-05
1095 2.17938322748523e-05
1096 2.17990673263557e-05
1097 2.17752876778832e-05
1098 2.17442448047223e-05
1099 2.17169017560082e-05
1100 2.17029482882936e-05
1101 2.1711775843869e-05
1102 2.16994376387447e-05
1103 2.1613805074594e-05
1104 2.20501242438331e-05
1105 2.10788475669688e-05
1106 2.24198156502098e-05
1107 2.11954848055029e-05
1108 2.31828853429761e-05
1109 2.18378245335771e-05
1110 2.19742978515569e-05
1111 2.17422766581876e-05
1112 2.16794996958924e-05
1113 2.15620038943598e-05
1114 2.15265354199801e-05
1115 2.14693809539312e-05
1116 2.14701412915019e-05
1117 2.1423502403195e-05
1118 2.13947314477991e-05
1119 2.14016126847127e-05
1120 2.13576058740728e-05
1121 2.13337389141088e-05
1122 2.13057919609128e-05
1123 2.12858285522088e-05
1124 2.12849408853799e-05
1125 2.12511004065163e-05
1126 2.12983250094112e-05
1127 2.12125560210552e-05
1128 2.15768723137444e-05
1129 2.08231322176289e-05
1130 2.38307984545827e-05
1131 2.07512075576233e-05
1132 2.20977308345027e-05
1133 2.13465209526476e-05
1134 2.13845105463406e-05
1135 2.12730101338821e-05
1136 2.11844471778022e-05
1137 2.11258193303365e-05
1138 2.10651032830356e-05
1139 2.1012790966779e-05
1140 2.10159178095637e-05
1141 2.10288399102865e-05
1142 2.10017551580677e-05
1143 2.09841837204294e-05
1144 2.09390927921049e-05
1145 2.09495829039952e-05
1146 2.09092959266854e-05
1147 2.08877045224654e-05
1148 2.08795790967997e-05
1149 2.08469227800379e-05
1150 2.08216624741908e-05
1151 2.08001347345999e-05
1152 2.07988086913247e-05
1153 2.06012391572585e-05
1154 2.08493638638174e-05
1155 2.05574051506119e-05
1156 2.37244385061786e-05
1157 2.08376841328572e-05
1158 2.13544954021927e-05
1159 2.09983791137347e-05
1160 2.0956902517355e-05
1161 2.08276906050742e-05
1162 2.07824250537669e-05
1163 2.07137563847937e-05
1164 2.06474087462993e-05
1165 2.06112490559462e-05
1166 2.05507094506174e-05
1167 2.05187061510514e-05
1168 2.05016422114568e-05
1169 2.04791303985985e-05
1170 2.04546558961738e-05
1171 2.04369898710866e-05
1172 2.0399367713253e-05
1173 2.03890158445574e-05
1174 2.04051993932808e-05
1175 2.03635208890773e-05
1176 2.03540475922637e-05
1177 2.03531326405937e-05
1178 2.03413983399514e-05
1179 2.0319850591477e-05
1180 2.02707979042316e-05
1181 2.02532264665933e-05
1182 2.02522987819975e-05
1183 2.0127154130023e-05
1184 2.03340277948882e-05
1185 2.33621540246531e-05
1186 2.0160825442872e-05
1187 2.05731394089526e-05
1188 2.04566149477614e-05
1189 2.03773524845019e-05
1190 2.03123290702933e-05
1191 2.03082854568493e-05
1192 2.02417868422344e-05
1193 2.01964812731603e-05
1194 2.02002156584058e-05
1195 2.01438015210442e-05
1196 2.01538914552657e-05
1197 2.01060120161856e-05
1198 2.01397742785048e-05
1199 2.00926660909317e-05
1200 2.00823469640454e-05
1201 2.00760114239529e-05
1202 2.00633112399373e-05
1203 2.0050954844919e-05
1204 2.00494305317989e-05
1205 2.00412741833134e-05
1206 2.00153845071327e-05
1207 2.00993381440639e-05
1208 1.97706522158114e-05
1209 2.00160375243286e-05
1210 1.98276284208987e-05
1211 2.26593710976886e-05
1212 1.99265032279072e-05
1213 2.06231761694653e-05
1214 2.0224615582265e-05
1215 2.0225039406796e-05
1216 2.0115592633374e-05
1217 2.00786344066728e-05
1218 2.00128361029783e-05
1219 1.99473233806202e-05
1220 1.99172827706207e-05
1221 1.99064033949981e-05
1222 1.98932375496952e-05
1223 1.9855910068145e-05
1224 1.9785191398114e-05
1225 1.98239449673565e-05
1226 1.97615936485818e-05
1227 1.97812551050447e-05
1228 1.96623659576289e-05
1229 1.98079524125205e-05
1230 1.96853452507639e-05
1231 1.99981150217354e-05
1232 1.94547046703519e-05
1233 2.15432937693549e-05
1234 1.94260137504898e-05
1235 2.02162700588815e-05
1236 1.97583831322845e-05
1237 1.9848621377605e-05
1238 1.97367844521068e-05
1239 1.9732282453333e-05
1240 1.96933651750442e-05
1241 1.97005228983471e-05
1242 1.9621384126367e-05
1243 1.96589589904761e-05
1244 1.96653672901448e-05
1245 1.96222445083549e-05
1246 1.96245273400564e-05
1247 1.96053570107324e-05
1248 1.95884131244384e-05
1249 1.96200253412826e-05
1250 1.950069963641e-05
1251 1.95794964383822e-05
1252 1.95429056475405e-05
1253 1.96461769519374e-05
1254 1.93468677025521e-05
1255 2.00479607883608e-05
1256 1.95142365555512e-05
1257 2.15135605685646e-05
1258 1.93417072296143e-05
1259 1.98343332158402e-05
1260 1.95935153897153e-05
1261 1.95812426682096e-05
1262 1.95030170289101e-05
1263 1.94917374756187e-05
1264 1.94940967048751e-05
1265 1.94915919564664e-05
1266 1.946581141965e-05
1267 1.94104450201849e-05
1268 1.94758522411576e-05
1269 1.9391121895751e-05
1270 1.93955929717049e-05
1271 1.94083168025827e-05
1272 1.93866435438395e-05
1273 1.93775667867158e-05
1274 1.93506093637552e-05
1275 1.93145242519677e-05
1276 1.93561809282983e-05
1277 1.93020186998183e-05
1278 1.92729112313827e-05
1279 1.93448686331976e-05
1280 1.96133314602775e-05
1281 1.92038678505924e-05
1282 2.30711357289692e-05
1283 1.93534906429704e-05
1284 2.00056638277601e-05
1285 1.96386772586266e-05
1286 1.95594311662717e-05
1287 1.94559070223477e-05
1288 1.94199074030621e-05
1289 1.93641681107692e-05
1290 1.92968545889016e-05
1291 1.93002852029167e-05
1292 1.92449988389853e-05
1293 1.92257211892866e-05
1294 1.9180310118827e-05
1295 1.91826275113272e-05
1296 1.91133549378719e-05
1297 1.91228718904313e-05
1298 1.90919745364226e-05
1299 1.91067847481463e-05
1300 1.90825849131215e-05
1301 1.90339360415237e-05
1302 1.90897644642973e-05
1303 1.90210939763347e-05
1304 1.91097406059271e-05
1305 1.89729216799606e-05
1306 1.91290982911596e-05
1307 1.88677058758913e-05
1308 2.18130244320491e-05
1309 1.89616712304996e-05
1310 1.97301305888686e-05
1311 1.9360146325198e-05
1312 1.92681400221772e-05
1313 1.92215011338703e-05
1314 1.91584585991222e-05
1315 1.91183553397423e-05
1316 1.90487462532474e-05
1317 1.90334158105543e-05
1318 1.89890452020336e-05
1319 1.90097835002234e-05
1320 1.89947950275382e-05
1321 1.89312613656512e-05
1322 1.89244528883137e-05
1323 1.89672700798837e-05
1324 1.89321090147132e-05
1325 1.88979611266404e-05
1326 1.89115635294002e-05
1327 1.88876456377329e-05
1328 1.88689809874631e-05
1329 1.89252805284923e-05
1330 1.88050544238649e-05
1331 1.89300499187084e-05
1332 1.86868965101894e-05
1333 1.92693732969929e-05
1334 1.90500322787557e-05
1335 2.16992393688997e-05
1336 1.91820527106756e-05
1337 1.94424592336873e-05
1338 1.91863036889117e-05
1339 1.91355757124256e-05
1340 1.90343907888746e-05
1341 1.89745751413284e-05
1342 1.89330057764892e-05
1343 1.88580706890207e-05
1344 1.88602480193367e-05
1345 1.88189897016855e-05
1346 1.87563273357227e-05
1347 1.87731657206314e-05
1348 1.87521782208933e-05
1349 1.87464174814522e-05
1350 1.86896795639768e-05
1351 1.87186124094296e-05
1352 1.87421464943327e-05
1353 1.86413672054186e-05
1354 1.87184177775634e-05
1355 1.86226152436575e-05
1356 1.87690147868125e-05
1357 1.84538912435528e-05
1358 1.89365637197625e-05
1359 1.8962655303767e-05
1360 2.00209524336969e-05
1361 1.88041049113963e-05
1362 1.89713646250311e-05
1363 1.88398116733879e-05
1364 1.87974201253382e-05
1365 1.87313598871697e-05
1366 1.87014156836085e-05
1367 1.8691420336836e-05
1368 1.86241941264598e-05
1369 1.86250799742993e-05
1370 1.86105953616789e-05
1371 1.85909466381418e-05
1372 1.86060351552442e-05
1373 1.85422359209042e-05
1374 1.85954468179261e-05
1375 1.85343651537551e-05
1376 1.85935696208617e-05
1377 1.85062071977882e-05
1378 1.85983935807599e-05
1379 1.83994488907047e-05
1380 1.87661971722264e-05
1381 1.84241507668048e-05
1382 2.05023370654089e-05
1383 1.84020464075729e-05
1384 1.90529663086636e-05
1385 1.86287561518839e-05
1386 1.86778379429597e-05
1387 1.8548482330516e-05
1388 1.854953734437e-05
1389 1.85138633241877e-05
1390 1.84904674824793e-05
1391 1.85043772944482e-05
1392 1.843093559728e-05
1393 1.84247182914987e-05
1394 1.84309774340363e-05
1395 1.84226373676211e-05
1396 1.83862957783276e-05
1397 1.84061318577733e-05
1398 1.8392351194052e-05
1399 1.83752490556799e-05
1400 1.83521005965304e-05
1401 1.83837564691203e-05
1402 1.82796593435341e-05
1403 1.85024709935533e-05
1404 1.82241146831075e-05
1405 2.05802261916688e-05
1406 1.82199109985959e-05
1407 1.88266476470744e-05
1408 1.84516848094063e-05
1409 1.84636155609041e-05
1410 1.83873780770227e-05
1411 1.83548290806357e-05
1412 1.83440915861866e-05
1413 1.83430402103113e-05
1414 1.83202337211696e-05
1415 1.83051379281096e-05
1416 1.82708226930117e-05
1417 1.83085649041459e-05
1418 1.82362655323232e-05
1419 1.82761614269111e-05
1420 1.82423827936873e-05
1421 1.82526673597749e-05
1422 1.82187541213352e-05
1423 1.8255475879414e-05
1424 1.8108323274646e-05
1425 1.83418560482096e-05
1426 1.80241022462724e-05
1427 1.944888572325e-05
1428 1.82321637112182e-05
1429 1.91827402886702e-05
1430 1.8200682461611e-05
1431 1.83676311280578e-05
1432 1.82245130417868e-05
1433 1.82438434421783e-05
1434 1.81392988451989e-05
1435 1.82265212060884e-05
1436 1.81459436134901e-05
1437 1.81980194611242e-05
1438 1.81275008799275e-05
1439 1.81807536137057e-05
1440 1.81143077497836e-05
1441 1.81412760866806e-05
1442 1.80474999069702e-05
1443 1.81271989276865e-05
1444 1.80236766027519e-05
1445 1.81888553925091e-05
1446 1.79244143510005e-05
1447 1.85180451808264e-05
1448 1.80987353814999e-05
1449 2.0034953195136e-05
1450 1.79916023625992e-05
1451 1.85221906576771e-05
1452 1.81407613126794e-05
1453 1.82012372533791e-05
1454 1.80818515218562e-05
1455 1.81022824108368e-05
1456 1.80572751560248e-05
1457 1.81034502020339e-05
1458 1.79918351932429e-05
1459 1.80492552317446e-05
1460 1.80191509571159e-05
1461 1.80877650564071e-05
1462 1.79894796019653e-05
1463 1.80488314072136e-05
1464 1.78990358108422e-05
1465 1.81186169356806e-05
1466 1.78583013621392e-05
1467 1.85146709554829e-05
1468 1.80053884832887e-05
1469 1.9743974917219e-05
1470 1.79320049937814e-05
1471 1.83881256816676e-05
1472 1.80344050022541e-05
1473 1.81074992724461e-05
1474 1.79868002305739e-05
1475 1.80214519787114e-05
1476 1.79657745320583e-05
1477 1.79915241460549e-05
1478 1.79297894646879e-05
1479 1.79804828803753e-05
1480 1.78979535121471e-05
1481 1.79626349563478e-05
1482 1.78736027010018e-05
1483 1.79540293174796e-05
1484 1.78463360498426e-05
1485 1.81015548150754e-05
1486 1.77136953425361e-05
1487 1.86854394996772e-05
1488 1.79148828465259e-05
1489 1.87524547072826e-05
1490 1.7774600564735e-05
1491 1.80849074240541e-05
1492 1.79280777956592e-05
1493 1.79806575033581e-05
1494 1.79108192241983e-05
1495 1.79174821823835e-05
1496 1.7898521036841e-05
1497 1.79174112417968e-05
1498 1.78425161720952e-05
1499 1.78822956513613e-05
1500 1.78349491761765e-05
1501 1.78969949047314e-05
1502 1.78030313691124e-05
1503 1.78843984031118e-05
1504 1.77639903995441e-05
1505 1.79443450178951e-05
1506 1.76337380253244e-05
1507 1.83979591383832e-05
1508 1.79224989551585e-05
1509 1.94604981516022e-05
1510 1.77706187969306e-05
1511 1.8194223230239e-05
1512 1.78589434653986e-05
1513 1.79272628884064e-05
1514 1.78273785422789e-05
1515 1.78682112164097e-05
1516 1.78155260073254e-05
1517 1.78685659193434e-05
1518 1.78304471774027e-05
1519 1.78146037796978e-05
1520 1.77942074515158e-05
1521 1.77827459992841e-05
1522 1.77797610376729e-05
1523 1.77752845047507e-05
1524 1.77280071511632e-05
1525 1.78041664185002e-05
1526 1.77147012436762e-05
1527 1.7842321540229e-05
1528 1.7562922948855e-05
1529 1.84698146767914e-05
1530 1.78363552549854e-05
1531 1.92690258700168e-05
1532 1.77675665327115e-05
1533 1.79890666913707e-05
1534 1.77590518433135e-05
1535 1.77565088961273e-05
1536 1.77567289938452e-05
1537 1.77483725565253e-05
1538 1.76595876837382e-05
1539 1.7721862604958e-05
1540 1.76637186086737e-05
1541 1.77228885149816e-05
1542 1.76374651346123e-05
1543 1.76452649611747e-05
1544 1.76347039086977e-05
1545 1.76645207830006e-05
1546 1.75883069459815e-05
1547 1.76270150404889e-05
1548 1.75680306711001e-05
1549 1.76989760802826e-05
1550 1.74240940395975e-05
1551 1.81529958354076e-05
1552 1.7855258192867e-05
1553 1.93815139937215e-05
1554 1.76313569681952e-05
1555 1.78997179318685e-05
1556 1.76825851667672e-05
1557 1.76697594724828e-05
1558 1.7607795598451e-05
1559 1.76388257386861e-05
1560 1.7606749679544e-05
1561 1.75965542439371e-05
1562 1.75574641616549e-05
1563 1.75529112311779e-05
1564 1.75269851752091e-05
1565 1.75639470398892e-05
1566 1.75352397491224e-05
1567 1.75328477780567e-05
1568 1.75113636942115e-05
1569 1.75128243427025e-05
1570 1.74609813257121e-05
1571 1.75367731571896e-05
1572 1.74549295479665e-05
1573 1.75607165147085e-05
1574 1.73238440765999e-05
1575 1.80737497430528e-05
1576 1.80261376954149e-05
1577 1.86591842066264e-05
1578 1.74163833435159e-05
1579 1.76177654793719e-05
1580 1.75476852746215e-05
1581 1.75725726876408e-05
1582 1.75251225300599e-05
1583 1.75413442775607e-05
1584 1.74910765053937e-05
1585 1.74816323124105e-05
1586 1.75138775375672e-05
1587 1.74894266820047e-05
1588 1.74800279637566e-05
1589 1.74300148501061e-05
1590 1.74586657522013e-05
1591 1.74380274984287e-05
1592 1.7462369214627e-05
1593 1.74011129274732e-05
1594 1.74510751094203e-05
1595 1.73970584000926e-05
1596 1.74342058016919e-05
1597 1.7388396372553e-05
1598 1.73850494320504e-05
1599 1.73386979440693e-05
1600 1.73919070221018e-05
1601 1.73420103237731e-05
1602 1.73821190401213e-05
1603 1.72151540027699e-05
1604 1.86924698937219e-05
1605 1.75909244717332e-05
1606 1.77904421434505e-05
1607 1.75081695488188e-05
1608 1.75218137883348e-05
1609 1.7452137399232e-05
1610 1.74623164639343e-05
1611 1.74512551893713e-05
1612 1.74300021171803e-05
1613 1.74094275280368e-05
1614 1.7393111193087e-05
1615 1.74152683030115e-05
1616 1.73678927239962e-05
1617 1.73548996826867e-05
1618 1.73495682247449e-05
1619 1.73857861227589e-05
1620 1.73317257576855e-05
1621 1.73606622411171e-05
1622 1.72754407685716e-05
1623 1.73018706846051e-05
1624 1.73026655829744e-05
1625 1.7323862266494e-05
1626 1.72439467860386e-05
1627 1.73115822690306e-05
1628 1.71655792655656e-05
1629 1.7481896065874e-05
1630 1.72752279468114e-05
1631 1.87080677278573e-05
1632 1.7168533304357e-05
1633 1.76456815097481e-05
1634 1.73062435351312e-05
1635 1.73799726326251e-05
1636 1.73474036273547e-05
1637 1.73203225131147e-05
1638 1.73305634234566e-05
1639 1.72755935636815e-05
1640 1.73280986928148e-05
1641 1.72885083884466e-05
1642 1.72673844645033e-05
1643 1.72591026057489e-05
1644 1.72193504113238e-05
1645 1.72451727848966e-05
1646 1.72115796885919e-05
1647 1.72133750311332e-05
1648 1.72043583006598e-05
1649 1.71952542586951e-05
1650 1.71994925040053e-05
1651 1.7171521903947e-05
1652 1.7160054994747e-05
1653 1.71793835761491e-05
1654 1.7119631593232e-05
1655 1.72226064023562e-05
1656 1.69958475453313e-05
1657 1.80863589775981e-05
1658 1.77797828655457e-05
1659 1.79671278601745e-05
1660 1.72455602296395e-05
1661 1.72665877471445e-05
1662 1.7275866412092e-05
1663 1.7246851712116e-05
1664 1.7236177882296e-05
1665 1.72063555510249e-05
1666 1.72247382579371e-05
1667 1.71815390785923e-05
1668 1.71884639712516e-05
1669 1.71445244632196e-05
1670 1.71429092006292e-05
1671 1.71679675986525e-05
1672 1.71216834132792e-05
1673 1.71573119587265e-05
1674 1.71286483237054e-05
1675 1.7109448890551e-05
1676 1.70881121448474e-05
1677 1.70922776305815e-05
1678 1.70938637893414e-05
1679 1.70982530107722e-05
1680 1.71181018231437e-05
1681 1.69458689924795e-05
1682 1.72111867868807e-05
1683 1.70698767760769e-05
1684 1.85655371751636e-05
1685 1.69850136444438e-05
1686 1.7528882381157e-05
1687 1.71045230672462e-05
1688 1.71598221641034e-05
1689 1.7140320778708e-05
1690 1.71563569892896e-05
1691 1.71105875779176e-05
1692 1.71187330124667e-05
1693 1.70891980815213e-05
1694 1.70939056260977e-05
1695 1.7076030417229e-05
1696 1.70278181030881e-05
1697 1.70207349583507e-05
1698 1.70503335539252e-05
1699 1.700759275991e-05
1700 1.70507209986681e-05
1701 1.69602571986616e-05
1702 1.70174898812547e-05
1703 1.69661780091701e-05
1704 1.69964187080041e-05
1705 1.69690429174807e-05
1706 1.70979019458173e-05
1707 1.6851294276421e-05
1708 1.7980526536121e-05
1709 1.71714873431483e-05
1710 1.76089306478389e-05
1711 1.6993948520394e-05
1712 1.70989678736078e-05
1713 1.7042764739017e-05
1714 1.70720450114459e-05
1715 1.70247567439219e-05
1716 1.70317762240302e-05
1717 1.69664144777926e-05
1718 1.70186031027697e-05
1719 1.70102794072591e-05
1720 1.69711383932736e-05
1721 1.69591785379453e-05
1722 1.69316226674709e-05
1723 1.69241902767681e-05
1724 1.69566737895366e-05
1725 1.69166887644678e-05
1726 1.69242266565561e-05
1727 1.68679689522833e-05
1728 1.68881106219487e-05
1729 1.68898932315642e-05
1730 1.69805171026383e-05
1731 1.67366724781459e-05
1732 1.7468053556513e-05
1733 1.73378739418695e-05
1734 1.79275411937851e-05
1735 1.68433998624096e-05
1736 1.70574221556308e-05
1737 1.69576542248251e-05
1738 1.69833201653091e-05
1739 1.69069298863178e-05
1740 1.69354843819747e-05
1741 1.69108752743341e-05
1742 1.6917090761126e-05
1743 1.68551359820412e-05
1744 1.68876322277356e-05
1745 1.68754650076153e-05
1746 1.686959512881e-05
1747 1.68258520716336e-05
1748 1.68255392054562e-05
1749 1.68369733728468e-05
1750 1.68127771758009e-05
1751 1.67788912222022e-05
1752 1.68165879586013e-05
1753 1.67506132129347e-05
1754 1.68262158695143e-05
1755 1.66502359206788e-05
1756 1.71345654962352e-05
1757 1.71719166246476e-05
1758 1.8116572391591e-05
1759 1.66851477843011e-05
1760 1.69903087225975e-05
1761 1.68472688528709e-05
1762 1.68183287314605e-05
1763 1.6843316188897e-05
1764 1.68092228705063e-05
1765 1.68070528161479e-05
1766 1.67507278092671e-05
1767 1.67757971212268e-05
1768 1.67624257301213e-05
1769 1.67336729646195e-05
1770 1.67081398103619e-05
1771 1.66597510542488e-05
1772 1.67067410075106e-05
1773 1.66929803526727e-05
1774 1.66198897204595e-05
1775 1.6670515833539e-05
1776 1.66323861776618e-05
1777 1.66306526807602e-05
1778 1.66023110068636e-05
1779 1.65715118782828e-05
1780 1.66081699717324e-05
1781 1.657615575823e-05
1782 1.66451645782217e-05
1783 1.64286684594117e-05
1784 1.76291378011229e-05
1785 1.70805396919604e-05
1786 1.71755964402109e-05
1787 1.66669033205835e-05
1788 1.66718382388353e-05
1789 1.66541631188011e-05
1790 1.66288045875262e-05
1791 1.66252557391999e-05
1792 1.66060017363634e-05
1793 1.65623332577525e-05
1794 1.65686160471523e-05
1795 1.6510981367901e-05
1796 1.65108158398652e-05
1797 1.65135297720553e-05
1798 1.64938501256984e-05
1799 1.65005585586186e-05
1800 1.64508728630608e-05
1801 1.64234625117388e-05
1802 1.64291795954341e-05
1803 1.64178618433652e-05
1804 1.6399668311351e-05
1805 1.63899021572433e-05
1806 1.63978147611488e-05
1807 1.63572240126086e-05
1808 1.63527511176653e-05
1809 1.63791264640167e-05
1810 1.62432843353599e-05
1811 1.64763259817846e-05
1812 1.65528508659918e-05
1813 1.81148116098484e-05
1814 1.62389151228126e-05
1815 1.66105783137027e-05
1816 1.64007178682368e-05
1817 1.63994991453364e-05
1818 1.6384410628234e-05
1819 1.63879121828359e-05
1820 1.63606055139098e-05
1821 1.63336935656844e-05
1822 1.63527402037289e-05
1823 1.63138411153341e-05
1824 1.6326241166098e-05
1825 1.62559354066616e-05
1826 1.63070762937423e-05
1827 1.62617470778059e-05
1828 1.62772776093334e-05
1829 1.62233354785712e-05
1830 1.62442593136802e-05
1831 1.62024371093139e-05
1832 1.6245470760623e-05
1833 1.61566931637935e-05
1834 1.62316846399335e-05
1835 1.60737890837481e-05
1836 1.63405456987675e-05
1837 1.61207735800417e-05
1838 1.77321344381198e-05
1839 1.61339430633234e-05
1840 1.67301295732614e-05
1841 1.62438755069161e-05
1842 1.62848555191886e-05
1843 1.61876432684949e-05
1844 1.62242831720505e-05
1845 1.61722182383528e-05
1846 1.62065825861646e-05
1847 1.61397310876055e-05
1848 1.61760071932804e-05
1849 1.61268089868827e-05
1850 1.61458392540226e-05
1851 1.61005609697895e-05
1852 1.61396365001565e-05
1853 1.60960589710157e-05
1854 1.61234111146769e-05
1855 1.60262243298348e-05
1856 1.61894458869938e-05
1857 1.59403807629133e-05
1858 1.64736211445415e-05
1859 1.61619536811486e-05
1860 1.69912891578861e-05
1861 1.59647988766665e-05
1862 1.63760923896916e-05
1863 1.60829404194374e-05
1864 1.61704811034724e-05
1865 1.61059888341697e-05
1866 1.61247626238037e-05
1867 1.61027091962751e-05
1868 1.608561615285e-05
1869 1.60480976774124e-05
1870 1.60936178872362e-05
1871 1.60295057867188e-05
1872 1.6062014765339e-05
1873 1.59949486260302e-05
1874 1.60486888489686e-05
1875 1.59512146638008e-05
1876 1.60794352268567e-05
1877 1.58780185302021e-05
1878 1.61858097271761e-05
1879 1.58967341121752e-05
1880 1.72182408277877e-05
1881 1.59073315444402e-05
1882 1.66266672749771e-05
1883 1.60258077812614e-05
1884 1.61133830260951e-05
1885 1.59791015903465e-05
1886 1.6020288967411e-05
1887 1.59591199917486e-05
1888 1.59977989824256e-05
1889 1.59318624355365e-05
1890 1.59958872245625e-05
1891 1.59305636771023e-05
1892 1.5935209376039e-05
1893 1.59063456521835e-05
1894 1.59522023750469e-05
1895 1.58430011651944e-05
1896 1.59313076437684e-05
1897 1.57892500283197e-05
1898 1.60620038514026e-05
1899 1.57446283992613e-05
1900 1.65355922945309e-05
1901 1.59231367433676e-05
1902 1.6758574929554e-05
1903 1.58542297867825e-05
1904 1.61055613716599e-05
1905 1.5858860933804e-05
1906 1.5948164218571e-05
1907 1.58790371642681e-05
1908 1.59260489454027e-05
1909 1.58586899488e-05
1910 1.58810071297921e-05
1911 1.58588409249205e-05
1912 1.58614056999795e-05
1913 1.58378588821506e-05
1914 1.58103994181147e-05
1915 1.5810572222108e-05
1916 1.58023576659616e-05
1917 1.58116399688879e-05
1918 1.5791600162629e-05
1919 1.57489721459569e-05
1920 1.58051479957066e-05
1921 1.5717523638159e-05
1922 1.58432685566368e-05
1923 1.56527949002339e-05
1924 1.66592180903535e-05
1925 1.60105009854306e-05
1926 1.66004138009157e-05
1927 1.59094506670954e-05
1928 1.59372029884253e-05
1929 1.58546699822182e-05
1930 1.58138591359602e-05
1931 1.58392012963304e-05
1932 1.57824215420987e-05
1933 1.58383008965757e-05
1934 1.57538288476644e-05
1935 1.57972244778648e-05
1936 1.57488648255821e-05
1937 1.57711619976908e-05
1938 1.57233644131338e-05
1939 1.57343474711524e-05
1940 1.57214199134614e-05
1941 1.5736175555503e-05
1942 1.56617079483112e-05
1943 1.57775793923065e-05
1944 1.56189435074339e-05
1945 1.58779839694034e-05
1946 1.55703219206771e-05
1947 1.64019656949677e-05
1948 1.56711430463474e-05
1949 1.64545308507513e-05
1950 1.56422847794602e-05
1951 1.5903819075902e-05
1952 1.57069916895125e-05
1953 1.58071616169764e-05
1954 1.57031390699558e-05
1955 1.5767074728501e-05
1956 1.56986352521926e-05
1957 1.57316062541213e-05
1958 1.56652022269554e-05
1959 1.56853257067269e-05
1960 1.56735914060846e-05
1961 1.57159975060495e-05
1962 1.56408568727784e-05
1963 1.56963851623004e-05
1964 1.55716898007086e-05
1965 1.57034537551226e-05
1966 1.55130110215396e-05
1967 1.58879993250594e-05
1968 1.56491187226493e-05
1969 1.6654126739013e-05
1970 1.55586421897169e-05
1971 1.6023854186642e-05
1972 1.56536643771688e-05
1973 1.57702197611798e-05
1974 1.56867263285676e-05
1975 1.5719682778581e-05
1976 1.56947135110386e-05
1977 1.56846672325628e-05
1978 1.56594323925674e-05
1979 1.56403693836182e-05
1980 1.56208334374242e-05
1981 1.56321675603976e-05
1982 1.5647454347345e-05
1983 1.56032656377647e-05
1984 1.56185669766273e-05
1985 1.55845300469082e-05
1986 1.55870529852109e-05
1987 1.55514626385411e-05
1988 1.5558895029244e-05
1989 1.55454508785624e-05
1990 1.556461029395e-05
1991 1.55147372424835e-05
1992 1.56545993377222e-05
1993 1.54518056660891e-05
1994 1.66547361004632e-05
1995 1.56614933075616e-05
1996 1.61704647325678e-05
1997 1.56106125359656e-05
1998 1.56903552124277e-05
1999 1.56641981448047e-05
};
\addlegendentry{Test}
\end{groupplot}

\end{tikzpicture}

		\label{Fig:kFold}
		\caption{Training- and validation error over 1000 epochs for five independent folds.}
	\end{figure}
\end{center}
Results for both models are summarized in \cref{Tab:Layer}. Training- and validation error for the model with three convolutinal layers achieve slightly better results as the small model. On the contrary does the \(\L2\) not achieve the resuls of the small model with \(\L2=0.044\). In \cref{Fig:Layer} the evolution of training- and validation error is depicted over 2000 epochs. The model with three convolutional layers shows an unstable training after the 500th epoch, which is ignored for now. It can be observed that the model with four convolutional layers is underfitting and does not achieve comparable results to the small model. 
\begin{table}[htbp!]
	\setlength{\tabcolsep}{1pt}
	\footnotesize
	\caption{Architecture for the models with three (left) and four (right) convolutional layers in encoder and decoder. Kernel size and stride width are \(3\times 3\) for both features and models.}
	\begin{minipage}{.5\textwidth}
		\begin{tabular*}{.9\textwidth}{ @{\extracolsep{\fill}} c c c c c @{} }
			\toprule
			\multicolumn{4}{c}{Encoder} \\ [.5ex]\hline
			Layer & Type & Channels & Padding & Output  \\ 
			\hline
			1 & Conv  & 4  & in: 1/1 & $9\times 67$  \\ \hline
			2 & Conv. & 8  & in: 0/1 & $3\times 23$  \\ \hline
			3 & Conv. & 16 & in: 0/1 & $1\times 8$   \\ \hline
			4 & Lin.  & -  & -   & 5		      	 \\ 
			\\
			\toprule
			\multicolumn{4}{c}{Decoder}		\\ [.5ex]\hline
			Layer & Type & Channels & Padding & Output \\
			\hline
			5 & Lin.     & -  & - & 128       	 		        \\ \hline
			6 & Tr.Conv. & 16 & in: 0/1 out: 0/1 & $1\times 8$  \\ \hline
			7 & Tr.Conv. & 8  & in: 0/1 out: 0/1 & $3\times 23$ \\ \hline
			8 & Tr.Conv. & 1  & in: 0/1 & $25\times 200$        \\ \hline
			\\
		\end{tabular*}
	\end{minipage}%
	\begin{minipage}{.5\textwidth}
		\begin{tabular*}{.9\textwidth}{ @{\extracolsep{\fill}} c c c c c @{} }
			\toprule
			\multicolumn{4}{c}{Encoder} \\ [.5ex]\hline
			Layer & Type & Channels & Padding & Out  \\ 
			\hline
			1 & Conv  & 2  & in: 3/2 & $10\times 68$ \\ \hline
			2 & Conv. & 4  & in: 3/2 & $5\times 24$  \\ \hline
			3 & Conv. & 8  & in: 3/2 & $3\times 9$   \\ \hline
			4 & Conv. & 16 & in: 3/3 & $1\times 3$   \\ \hline
			5 & Lin.  & -  & & 5		      \\  
			\toprule
			\multicolumn{4}{c}{Decoder}		\\ [.5ex]\hline
			Layer & Type & Channels & Padding & Out \\
			\hline
			6  & Lin.     & -  & - & 128       	 \\ \hline
			7  & Tr.Conv. & 16 & - & $3\times 9$   \\ \hline
			8  & Tr.Conv. & 8  & in: 2/2 out: 0/1 & $5\times 24$   \\ \hline
			9  & Tr.Conv. & 4  & in: 3/2 & $9\times 68$   \\ \hline
			10 & Tr.Conv. & 1  & in: 1/2 & $25\times 200$ \\ \hline   
		\end{tabular*}
	\end{minipage}
\end{table} 
\begin{table}[htbp!]
	\centering
	\caption{Results for increasing number of layers. Summary of minimum training- and minimum validation error for a model with three and a model with convolutional layers in the encoder as well as the corresponding \(\L2\) and the epoch in which those values are reached.}
	\begin{tabular*}{15cm}{ @{\extracolsep{\fill}} c c c c c c c c c c @{} }
		\toprule
		Layer & Minimum training error & Minimum validation error & \(\L2\) & Epoch\\ [.5ex]
		\hline
		3   & \num{1.4e-5}           & \num{1.5e-5}             & 0.044   & 1975  \\  
		\hline
		4    & \num{7.3e-4}           & \num{8.1e-4}             & 0.327   & 1995\\
		\hline
	\end{tabular*}\label{Tab:Layer}
\end{table} 
\begin{center}
	\begin{figure}[htbp!]
		% This file was created by tikzplotlib v0.9.6.
\begin{tikzpicture}

\begin{groupplot}[group style={group size=2 by 1}]
\nextgroupplot[
legend cell align={left},
legend style={fill opacity=0.8, draw opacity=1, text opacity=1, draw=white!80!black},
log basis y={10},
tick align=outside,
tick pos=left,
title={Model3 },
x grid style={white!69.0196078431373!black},
xlabel={Epoch},
xmin=-99.95, xmax=2098.95,
xtick style={color=black},
y grid style={white!69.0196078431373!black},
ylabel={MSE Loss},
ymin=8.88664867882007e-06, ymax=0.01,
ymode=log,
ytick style={color=black}
]
\addplot [semithick, black, dashed]
table {%
0 0.0591322294203565
1 0.0571813012938946
2 0.0553483462426811
3 0.0536217110930011
4 0.0519817891763523
5 0.0504158382536843
6 0.048917654203251
7 0.0474803105462343
8 0.0460976826725528
9 0.044765998783987
10 0.0434846194111742
11 0.0422513766679913
12 0.041065052500926
13 0.0399211139301769
14 0.038816183921881
15 0.0377468024380505
16 0.0367096365080215
17 0.035702271386981
18 0.0347228948958218
19 0.0337699174997397
20 0.0328420310979709
21 0.0319380142027512
22 0.0310564514948055
23 0.0301966735278256
24 0.0293582719459664
25 0.0285401940054726
26 0.0277417832403444
27 0.0269628732930869
28 0.0262029257137328
29 0.0254612409044057
30 0.0247376960760448
31 0.0240321719611529
32 0.0233440008305479
33 0.0226735404867213
34 0.0220200676703826
35 0.0213822359510232
36 0.0207581442955416
37 0.0201472562039271
38 0.0195490065525519
39 0.0189631264220225
40 0.0183881274715532
41 0.0178217947832309
42 0.0172638810181525
43 0.0167132107744692
44 0.0161615614924813
45 0.0155940286931582
46 0.0150068740331335
47 0.0143202653125627
48 0.0134792069002287
49 0.0127046815177891
50 0.0120440874525229
51 0.0114065190427937
52 0.0107035632900079
53 0.0100266317967908
54 0.00939784950605826
55 0.00880874711583601
56 0.00824155005830107
57 0.00764251218060963
58 0.00706394788358011
59 0.00652519057621248
60 0.00603162538391189
61 0.00557825837677228
62 0.00515501527843298
63 0.00475763052963885
64 0.00436550728045404
65 0.00398499809671193
66 0.00363470627598872
67 0.00331176295912883
68 0.00301173736079363
69 0.00273530351660156
70 0.00250203917494218
71 0.00227434015505423
72 0.00205789824576641
73 0.00186697551180259
74 0.0017054268373613
75 0.00156439451893675
76 0.00144362724404345
77 0.00134151971906249
78 0.00124457403126144
79 0.00116278183122631
80 0.00109116772637208
81 0.00101833554435871
82 0.000948073110521364
83 0.000890652899670386
84 0.000841520695757936
85 0.000798174684859987
86 0.000759626192575524
87 0.000724814007753594
88 0.000693391650202102
89 0.000664901279378682
90 0.000638599811964013
91 0.000614826922173961
92 0.000593301690514636
93 0.000573696006767932
94 0.000555781338789529
95 0.000539393807002853
96 0.000524487673374097
97 0.000510905710598308
98 0.000498504422239421
99 0.000487080589891775
100 0.00047651646355007
101 0.000466500106085732
102 0.000456725053368245
103 0.000446446092382757
104 0.000435641426861366
105 0.000424610074901466
106 0.000415271642964399
107 0.000407080388185932
108 0.000399682721649697
109 0.000392894963511026
110 0.000386653453347208
111 0.000380852407261045
112 0.000375389330997677
113 0.000369745254488407
114 0.000363012680963948
115 0.00035668680800427
116 0.000350959695197162
117 0.000345885530578016
118 0.00034129893856516
119 0.000337053278713029
120 0.000333084214560131
121 0.000329395276366995
122 0.000325830776830571
123 0.000322551934402782
124 0.000319471587374665
125 0.000316538689048684
126 0.000313738333602487
127 0.000311032433387481
128 0.000308406275337347
129 0.000305784571082768
130 0.000302245202703944
131 0.000298671769769498
132 0.000295635817622042
133 0.000292999885289191
134 0.000290566063824826
135 0.000288329974807766
136 0.000286236328690848
137 0.000284281059037994
138 0.000282403661287844
139 0.00028063945296708
140 0.000278941881646233
141 0.000277333998241147
142 0.000275763771014681
143 0.00027427482379494
144 0.000272817700135874
145 0.000271429695487768
146 0.000270033047030438
147 0.000268651369168538
148 0.000267391312348764
149 0.000266051329390393
150 0.000264771938418562
151 0.000263578846983137
152 0.000262383683917733
153 0.000261206277116344
154 0.00026007616824586
155 0.000258946622210487
156 0.000257846936079886
157 0.000256761145209339
158 0.00025568897467565
159 0.000254625922792684
160 0.000253584286156183
161 0.00025254965268573
162 0.000251518517387694
163 0.000250518195144878
164 0.000249515684600965
165 0.000248522529773254
166 0.000247525827219874
167 0.000246547710560208
168 0.000245560649887011
169 0.000244602551575213
170 0.000243633006917321
171 0.00024266914135751
172 0.00024171285780028
173 0.000240762281350726
174 0.000239816454296715
175 0.000238868945984905
176 0.000237911273714531
177 0.000236960488791738
178 0.00023602959015534
179 0.000235081389973857
180 0.00023413582917442
181 0.000233190433718278
182 0.000232246387753321
183 0.000231294474843935
184 0.000230359799175517
185 0.000229409537396918
186 0.000228467877505523
187 0.000227511689431026
188 0.000226569585123571
189 0.000225606570097625
190 0.000224663920320722
191 0.000223688910153896
192 0.000222728172118991
193 0.000221767202930323
194 0.000220782658630014
195 0.000219812565561028
196 0.00021881907093757
197 0.000217838252126512
198 0.000216834403303778
199 0.000215845009762461
200 0.000214827645237392
201 0.000213814296827763
202 0.000212787503954814
203 0.000211752841437374
204 0.000210707577664948
205 0.00020965252970484
206 0.000208583252273797
207 0.000207533620837808
208 0.000206440667312791
209 0.000205364391050011
210 0.000204262021540558
211 0.000203136116184055
212 0.000202033330353402
213 0.000200879023580569
214 0.000199661415891228
215 0.000198392688702143
216 0.000197159864612217
217 0.000195961606721085
218 0.000194715714627591
219 0.000193487171088691
220 0.00019224733753731
221 0.000190942118393878
222 0.000189570968061048
223 0.000188210327408456
224 0.000186810408351334
225 0.000185437883644113
226 0.000184057433358475
227 0.000182647995984553
228 0.000181244723989948
229 0.000179779890174814
230 0.000178361432020324
231 0.000176877647220408
232 0.000175406197925554
233 0.000173908974772985
234 0.000172405944809384
235 0.000170901016083747
236 0.000169343418697565
237 0.000167822995670974
238 0.000166274857704707
239 0.000164687645465733
240 0.000163145781293395
241 0.000161557829045478
242 0.000159925248063075
243 0.000158307510702116
244 0.000156675372110726
245 0.000154973626919741
246 0.000153323306470554
247 0.000151725530166402
248 0.000150057919363178
249 0.00014844713319917
250 0.000146790761121451
251 0.000145165309781703
252 0.000143504971006791
253 0.000141883052982905
254 0.000140251272817693
255 0.000138610596216893
256 0.000136944163742214
257 0.000135350814723267
258 0.000133731449736274
259 0.000132113316432481
260 0.000130514987205288
261 0.000128917470476608
262 0.000127348933659732
263 0.000125763593935346
264 0.000124235192764388
265 0.000122652352629871
266 0.00012113660203994
267 0.000119616604358441
268 0.000118104572493394
269 0.000116616447364493
270 0.000115140124179902
271 0.000113688334899109
272 0.000112212479997709
273 0.000110782110077423
274 0.000109384814663827
275 0.000108027122955434
276 0.000106672385783213
277 0.000105365755906917
278 0.000104025929026363
279 0.000102777226771877
280 0.0001015095396113
281 0.000100261860175976
282 9.90431942753389e-05
283 9.78256870780569e-05
284 9.66420422230385e-05
285 9.54828764250237e-05
286 9.43263877388745e-05
287 9.3225096172489e-05
288 9.21268803821818e-05
289 9.10842492203301e-05
290 9.00189569179588e-05
291 8.89953074931782e-05
292 8.79970691300969e-05
293 8.70183442955863e-05
294 8.60646092348816e-05
295 8.51514878306148e-05
296 8.4246716880898e-05
297 8.3362635677986e-05
298 8.25047803729717e-05
299 8.16522233861861e-05
300 8.0813472656871e-05
301 8.00114185821599e-05
302 7.91841526819326e-05
303 7.83601494589448e-05
304 7.75920166589117e-05
305 7.68295561002219e-05
306 7.60169163669389e-05
307 7.54124407364998e-05
308 7.48146967239904e-05
309 7.41085991933232e-05
310 7.34330627309987e-05
311 7.27177625492459e-05
312 7.20375085876412e-05
313 7.13342257938621e-05
314 7.0696414688598e-05
315 7.00140044145314e-05
316 6.93311870136881e-05
317 6.87315396632471e-05
318 6.80952805147683e-05
319 6.7457195925158e-05
320 6.68435025019676e-05
321 6.62865632925502e-05
322 6.56928180156058e-05
323 6.51264658309003e-05
324 6.45667646352877e-05
325 6.40085497352061e-05
326 6.34940820880558e-05
327 6.29739243578342e-05
328 6.24471889274503e-05
329 6.19394299619103e-05
330 6.14403424101795e-05
331 6.09461097802466e-05
332 6.04558690149304e-05
333 5.99738897584245e-05
334 5.94940626328366e-05
335 5.8980935293107e-05
336 5.84641513476925e-05
337 5.80543696422353e-05
338 5.7607266981563e-05
339 5.71648853977536e-05
340 5.67384794187831e-05
341 5.63124016572658e-05
342 5.58629734683791e-05
343 5.55168075813128e-05
344 5.50938692676084e-05
345 5.46761151234421e-05
346 5.42820944886557e-05
347 5.3882221372703e-05
348 5.3469399436068e-05
349 5.31336961557827e-05
350 5.27688771327917e-05
351 5.23585200085108e-05
352 5.19793673241509e-05
353 5.15771569524759e-05
354 5.1243808471213e-05
355 5.09019464587368e-05
356 5.0528525576965e-05
357 5.01697122246014e-05
358 4.98004454598799e-05
359 4.94120626619576e-05
360 4.91256179522281e-05
361 4.87562798525687e-05
362 4.84060278154175e-05
363 4.80638761040098e-05
364 4.77426251173085e-05
365 4.74336112930018e-05
366 4.71094565313024e-05
367 4.67811775841653e-05
368 4.6499715050885e-05
369 4.61397517632633e-05
370 4.5871669275499e-05
371 4.55899701101714e-05
372 4.52808258160076e-05
373 4.49588641515675e-05
374 4.47215336194517e-05
375 4.44051255215072e-05
376 4.40830291132954e-05
377 4.38611666560007e-05
378 4.35498307744808e-05
379 4.3242941064392e-05
380 4.30340549115726e-05
381 4.2737348547206e-05
382 4.24281007376237e-05
383 4.22377508044747e-05
384 4.19360737993202e-05
385 4.16420891902192e-05
386 4.14568503224189e-05
387 4.1158268867747e-05
388 4.09061160269175e-05
389 4.06799686007986e-05
390 4.04067597230551e-05
391 4.01734586752411e-05
392 3.99323959818787e-05
393 3.96844751922032e-05
394 3.94802569729791e-05
395 3.92311376078425e-05
396 3.90117977087101e-05
397 3.87946101376713e-05
398 3.8593309570345e-05
399 3.83245916708574e-05
400 3.81402352758187e-05
401 3.78958515572236e-05
402 3.76230544087974e-05
403 3.74507398248625e-05
404 3.71718847169689e-05
405 3.70130155236126e-05
406 3.67675082308949e-05
407 3.6559763925581e-05
408 3.63634599089835e-05
409 3.61772201582511e-05
410 3.5939084298775e-05
411 3.58042711692264e-05
412 3.55672500074178e-05
413 3.54245014673182e-05
414 3.52356438320101e-05
415 3.50272276357089e-05
416 3.48721196061774e-05
417 3.4708548424689e-05
418 3.44973749228927e-05
419 3.43747065461031e-05
420 3.41770609981751e-05
421 3.40077060148047e-05
422 3.38670984607603e-05
423 3.36899384780764e-05
424 3.35377572699969e-05
425 3.33936487155739e-05
426 3.32223698027434e-05
427 3.30886594408497e-05
428 3.29519396657929e-05
429 3.27440044749494e-05
430 3.26378553907603e-05
431 3.25211597882458e-05
432 3.23103260448931e-05
433 3.22452841707843e-05
434 3.20762836540922e-05
435 3.19274675764092e-05
436 3.18089011486222e-05
437 3.16867211225258e-05
438 3.1509949904418e-05
439 3.14079593231043e-05
440 3.13072052300356e-05
441 3.11336010252461e-05
442 3.10493863668881e-05
443 3.09165031602276e-05
444 3.07780936701363e-05
445 3.06853001337259e-05
446 3.0546348682492e-05
447 3.04497980678775e-05
448 3.03408855693021e-05
449 3.0241986859636e-05
450 3.00842236793386e-05
451 3.0034819124225e-05
452 2.98750326201969e-05
453 2.98017399149586e-05
454 2.96897608489388e-05
455 2.9597975082396e-05
456 2.94419260189471e-05
457 2.93739637040602e-05
458 2.92870696068093e-05
459 2.91586942218203e-05
460 2.9087811798334e-05
461 2.89968421434139e-05
462 2.88556640537507e-05
463 2.88319242205404e-05
464 2.87065532074138e-05
465 2.86072336148635e-05
466 2.85475162513649e-05
467 2.84781234185516e-05
468 2.83432913903425e-05
469 2.83228849280448e-05
470 2.82105620534878e-05
471 2.81149608447606e-05
472 2.80785395077032e-05
473 2.7992816420408e-05
474 2.78853617032837e-05
475 2.78527656529803e-05
476 2.77764789995416e-05
477 2.76870507143201e-05
478 2.76512977794852e-05
479 2.75612269344805e-05
480 2.74675677403025e-05
481 2.7462965306313e-05
482 2.73711363174556e-05
483 2.72883886136555e-05
484 2.72526416118524e-05
485 2.71794790211999e-05
486 2.70736272156569e-05
487 2.70350926925289e-05
488 2.69519692324138e-05
489 2.68517481245567e-05
490 2.68078065754906e-05
491 2.67283712265254e-05
492 2.66358474796391e-05
493 2.65973722157753e-05
494 2.65447253209317e-05
495 2.64161727603351e-05
496 2.64121106390292e-05
497 2.63434165663767e-05
498 2.6239618906132e-05
499 2.60917558030371e-05
500 2.59466073639913e-05
501 2.58326446598289e-05
502 2.57630735873704e-05
503 2.57072335028141e-05
504 2.56017757980231e-05
505 2.55837350096044e-05
506 2.54953707266026e-05
507 2.54061987767784e-05
508 2.53956176532455e-05
509 2.53170927528856e-05
510 2.52407887817441e-05
511 2.5225848258259e-05
512 2.51556095545169e-05
513 2.50786457343111e-05
514 2.50627049798169e-05
515 2.50090819893245e-05
516 2.49190649395103e-05
517 2.491813605765e-05
518 2.48538990490488e-05
519 2.47910482205072e-05
520 2.47626712823745e-05
521 2.47269985251819e-05
522 2.46422028649107e-05
523 2.46341901792846e-05
524 2.45918870813711e-05
525 2.45211671074941e-05
526 2.45046780094071e-05
527 2.44632586152704e-05
528 2.43937803752203e-05
529 2.44023956899042e-05
530 2.43293952788548e-05
531 2.42882974976766e-05
532 2.42639500562802e-05
533 2.42539742263403e-05
534 2.41195549737228e-05
535 2.41902253321769e-05
536 2.40714911097228e-05
537 2.40835041616094e-05
538 2.39980208363022e-05
539 2.40978156185179e-05
540 2.38858887797733e-05
541 2.40800918440698e-05
542 2.38495781239578e-05
543 2.40958280368631e-05
544 2.38302195540996e-05
545 2.42526828682088e-05
546 2.3904121381868e-05
547 2.45620317578243e-05
548 2.41282296657985e-05
549 2.49960462870291e-05
550 2.43246320872359e-05
551 2.49729296513124e-05
552 2.41099272404455e-05
553 2.42267635943705e-05
554 2.36616118538535e-05
555 2.37467083650245e-05
556 2.34029810322056e-05
557 2.35469446430159e-05
558 2.3340013623141e-05
559 2.3406206722143e-05
560 2.3281268392239e-05
561 2.33721260638475e-05
562 2.3210590915923e-05
563 2.33169115428922e-05
564 2.317275075292e-05
565 2.33216867879804e-05
566 2.3102596721003e-05
567 2.33784465653031e-05
568 2.31337986571489e-05
569 2.34962958334251e-05
570 2.32268547346592e-05
571 2.38599340960377e-05
572 2.34859875183346e-05
573 2.44587932556328e-05
574 2.38483813363999e-05
575 2.43457881001774e-05
576 2.35504466843395e-05
577 2.35754190303084e-05
578 2.3018127650154e-05
579 2.30578455937547e-05
580 2.2800258715705e-05
581 2.28698095394719e-05
582 2.26758961350271e-05
583 2.28070296000027e-05
584 2.26466184471974e-05
585 2.2710819735039e-05
586 2.26013779478507e-05
587 2.2700639327411e-05
588 2.25220557794614e-05
589 2.27215830026495e-05
590 2.25205165556019e-05
591 2.2740481096406e-05
592 2.25485312679297e-05
593 2.30058567467495e-05
594 2.27441566913456e-05
595 2.36053885176446e-05
596 2.3306368285958e-05
597 2.43322302591054e-05
598 2.35475473289171e-05
599 2.36890256433142e-05
600 2.28383118301423e-05
601 2.26699257117602e-05
602 2.23260260492353e-05
603 2.23129778200715e-05
604 2.21322922486777e-05
605 2.22123788811146e-05
606 2.20987985271037e-05
607 2.21201107359903e-05
608 2.20741547858339e-05
609 2.21020486748102e-05
610 2.20124670313737e-05
611 2.20337756173805e-05
612 2.19708997075685e-05
613 2.20236037815624e-05
614 2.1903517374966e-05
615 2.20494954810135e-05
616 2.1899116095625e-05
617 2.21352522187246e-05
618 2.19609469720439e-05
619 2.2685855452842e-05
620 2.25501990556154e-05
621 2.41698366840737e-05
622 2.38558748666406e-05
623 2.4370444305255e-05
624 2.30260182521036e-05
625 2.22747292850656e-05
626 2.17865728409095e-05
627 2.17052018411934e-05
628 2.1617200921753e-05
629 2.15880337570695e-05
630 2.1580405089594e-05
631 2.15677014754334e-05
632 2.15023124283675e-05
633 2.15373632750371e-05
634 2.15032935031445e-05
635 2.14561701197269e-05
636 2.14972416490156e-05
637 2.14495744534915e-05
638 2.1423937159426e-05
639 2.14093506469837e-05
640 2.14339418729281e-05
641 2.13347611435211e-05
642 2.15023300018657e-05
643 2.13739870993379e-05
644 2.17731291494161e-05
645 2.1731871260755e-05
646 2.30428815104489e-05
647 2.31846275209335e-05
648 2.47548286118615e-05
649 2.34812243089522e-05
650 2.2227493154503e-05
651 2.14852272311106e-05
652 2.12209354177872e-05
653 2.11060711441391e-05
654 2.11139864392251e-05
655 2.10711768948713e-05
656 2.1036461076207e-05
657 2.10497215027416e-05
658 2.10321173934602e-05
659 2.09829412716189e-05
660 2.10025022502336e-05
661 2.09993640503114e-05
662 2.09363807242013e-05
663 2.09706264415743e-05
664 2.09241720874331e-05
665 2.09420501295909e-05
666 2.08943051411836e-05
667 2.10243610370142e-05
668 2.08786614548373e-05
669 2.13308284857661e-05
670 2.13083991686958e-05
671 2.26316359537293e-05
672 2.29110544465172e-05
673 2.44000504734387e-05
674 2.31292608212641e-05
675 2.1726881030304e-05
676 2.09839899660835e-05
677 2.07417497142259e-05
678 2.0689563046794e-05
679 2.06234422295282e-05
680 2.06364813424287e-05
681 2.06218003180325e-05
682 2.0575198000472e-05
683 2.06081302112082e-05
684 2.05696489912199e-05
685 2.05486177575764e-05
686 2.0559896968031e-05
687 2.05654681568745e-05
688 2.05043787371295e-05
689 2.05525088210479e-05
690 2.04901723845907e-05
691 2.05428533481822e-05
692 2.04790817992517e-05
693 2.06684890233788e-05
694 2.06093854888678e-05
695 2.13422825128262e-05
696 2.15910913698991e-05
697 2.37795555664277e-05
698 2.3582285731294e-05
699 2.29749252964595e-05
700 2.15272111141473e-05
701 2.05270447821704e-05
702 2.0374938643819e-05
703 2.02651110967622e-05
704 2.02321129085803e-05
705 2.0263017709965e-05
706 2.02416058012744e-05
707 2.02009378691415e-05
708 2.02184117261694e-05
709 2.02002224263254e-05
710 2.01731577282871e-05
711 2.01808602335873e-05
712 2.01530115200299e-05
713 2.01590556736519e-05
714 2.01885194348606e-05
715 2.01016446457203e-05
716 2.02053510696132e-05
717 2.01102555426047e-05
718 2.03049340354866e-05
719 2.02760400869195e-05
720 2.10559462354709e-05
721 2.14561457063667e-05
722 2.36597007914341e-05
723 2.33730917900132e-05
724 2.24358583924555e-05
725 2.10721223297128e-05
726 2.01127688521652e-05
727 2.00254283901913e-05
728 1.99023751559224e-05
729 1.99411677783701e-05
730 1.99241164127173e-05
731 1.98854451092245e-05
732 1.99272891672209e-05
733 1.98666415123228e-05
734 1.99077034146455e-05
735 1.98686208978671e-05
736 1.98573382181877e-05
737 1.98592625491045e-05
738 1.98318567123934e-05
739 1.98391799841602e-05
740 1.98316981263602e-05
741 1.98018908195152e-05
742 1.98372501580835e-05
743 1.976619831634e-05
744 1.99127438866853e-05
745 1.98472805781336e-05
746 2.03704502803603e-05
747 2.06928300450215e-05
748 2.29907103159732e-05
749 2.34939090928421e-05
750 2.34588750558373e-05
751 2.15130094147753e-05
752 1.98843553942396e-05
753 1.97514852384728e-05
754 1.96255064155437e-05
755 1.96137717329847e-05
756 1.96366896787481e-05
757 1.95943412943933e-05
758 1.96227908606517e-05
759 1.96158340743935e-05
760 1.95818809176629e-05
761 1.95927717463462e-05
762 1.95612153426339e-05
763 1.95742549502498e-05
764 1.95372477493905e-05
765 1.95712356605426e-05
766 1.95354634744405e-05
767 1.95353488106065e-05
768 1.95341281949979e-05
769 1.95713828441413e-05
770 1.95215836429341e-05
771 1.97264402990172e-05
772 1.97457853881033e-05
773 2.06826726811826e-05
774 2.12677314976872e-05
775 2.35296675317187e-05
776 2.28221624105629e-05
777 2.10377799643169e-05
778 2.00119339028859e-05
779 1.9411198102226e-05
780 1.94129971049506e-05
781 1.93493095679287e-05
782 1.9322506711994e-05
783 1.93481591050926e-05
784 1.93027991892691e-05
785 1.93495481441985e-05
786 1.92927408653176e-05
787 1.93255812082072e-05
788 1.93023023093986e-05
789 1.92855910867706e-05
790 1.92776754319723e-05
791 1.9273816268317e-05
792 1.92658259310363e-05
793 1.92535992971088e-05
794 1.92629093218599e-05
795 1.92469365805081e-05
796 1.92364256825783e-05
797 1.93014146576687e-05
798 1.92162626717618e-05
799 1.95846307819991e-05
800 1.97289235002174e-05
801 2.16706361708674e-05
802 2.28870103953227e-05
803 2.45781223995323e-05
804 2.22360721036097e-05
805 1.94447320644997e-05
806 1.91872410768568e-05
807 1.90216524487141e-05
808 1.90752612825307e-05
809 1.90364900101159e-05
810 1.9059312515779e-05
811 1.90343617196831e-05
812 1.9055486871089e-05
813 1.90049508690393e-05
814 1.90364555248124e-05
815 1.8993407092438e-05
816 1.90162540727457e-05
817 1.89809315704892e-05
818 1.90146897045551e-05
819 1.89891327000424e-05
820 1.89906551488761e-05
821 1.89760313116238e-05
822 1.90007166169792e-05
823 1.89782944071126e-05
824 1.91128403166374e-05
825 1.91125021791194e-05
826 1.97673612598592e-05
827 2.02968642675216e-05
828 2.25911332885076e-05
829 2.26951900241801e-05
830 2.16301323527901e-05
831 2.00996147556864e-05
832 1.8979339809988e-05
833 1.88992606071636e-05
834 1.87797248436539e-05
835 1.88258148625664e-05
836 1.87848426040915e-05
837 1.88271248307004e-05
838 1.87782426870342e-05
839 1.88050495273373e-05
840 1.87578592329984e-05
841 1.87950054240105e-05
842 1.87838458725054e-05
843 1.87435655902846e-05
844 1.87918831517919e-05
845 1.87401375191243e-05
846 1.88228692499237e-05
847 1.87442729169263e-05
848 1.89908615864098e-05
849 1.89743216658655e-05
850 1.98883078521561e-05
851 2.04960818255273e-05
852 2.26631653807985e-05
853 2.21212640374446e-05
854 2.04144680031604e-05
855 1.93369806087418e-05
856 1.86928663419295e-05
857 1.86662573531393e-05
858 1.85874594409441e-05
859 1.86101781558534e-05
860 1.85922933964022e-05
861 1.86008857490094e-05
862 1.85778677286308e-05
863 1.85922113864478e-05
864 1.85662352336635e-05
865 1.85785772757185e-05
866 1.85490352881956e-05
867 1.85678845801007e-05
868 1.85666204135515e-05
869 1.85544421551498e-05
870 1.85827673133687e-05
871 1.85649196540183e-05
872 1.87573114072137e-05
873 1.88296575736047e-05
874 1.97942390265382e-05
875 2.0554364670744e-05
876 2.28111202735803e-05
877 2.20354376283538e-05
878 1.98866356342364e-05
879 1.89611734482398e-05
880 1.84230292168408e-05
881 1.84430247776035e-05
882 1.83714486885478e-05
883 1.84193092973572e-05
884 1.83710845016449e-05
885 1.83981333421457e-05
886 1.83618884515013e-05
887 1.83886280265e-05
888 1.83521147132382e-05
889 1.83787449383743e-05
890 1.83517168492742e-05
891 1.83807436577865e-05
892 1.83271616132075e-05
893 1.83770784190074e-05
894 1.83179569983594e-05
895 1.84395693025863e-05
896 1.83683676322488e-05
897 1.87906319739639e-05
898 1.89715284193426e-05
899 2.05673224344949e-05
900 2.13467930016975e-05
901 2.25348727251884e-05
902 2.0907194300257e-05
903 1.87441740386873e-05
904 1.84143015324878e-05
905 1.81844212985283e-05
906 1.82108501931566e-05
907 1.82313697125736e-05
908 1.81893675090805e-05
909 1.82221347264644e-05
910 1.8184623439943e-05
911 1.82302462619077e-05
912 1.81783153703208e-05
913 1.82060449729349e-05
914 1.81645204389014e-05
915 1.82103347743379e-05
916 1.81502846867687e-05
917 1.8218248023949e-05
918 1.81578632929558e-05
919 1.82912171524663e-05
920 1.82050491996932e-05
921 1.85850560461276e-05
922 1.86883455794984e-05
923 2.00940780512582e-05
924 2.07428630005779e-05
925 2.2033349550199e-05
926 2.06733026271166e-05
927 1.86835877222791e-05
928 1.82740498573253e-05
929 1.80773779963062e-05
930 1.80668969091968e-05
931 1.80992976410721e-05
932 1.80632173725215e-05
933 1.80923675632272e-05
934 1.80560669571506e-05
935 1.80806984584336e-05
936 1.80599347761046e-05
937 1.80885056790814e-05
938 1.80575352715096e-05
939 1.80878129487638e-05
940 1.80477366749088e-05
941 1.80908269760138e-05
942 1.80356693677552e-05
943 1.81207535119299e-05
944 1.80458327196931e-05
945 1.82890640685329e-05
946 1.82235152328403e-05
947 1.89915821673381e-05
948 1.93257524401247e-05
949 2.1333522774114e-05
950 2.10570481877781e-05
951 1.98537467026583e-05
952 1.88134308967136e-05
953 1.81319549108494e-05
954 1.80636065296724e-05
955 1.80144395587334e-05
956 1.79900315613324e-05
957 1.80212118419121e-05
958 1.79904254098417e-05
959 1.80197837122975e-05
960 1.79851052513058e-05
961 1.80161682923341e-05
962 1.79682921199742e-05
963 1.80138513394823e-05
964 1.79672538269671e-05
965 1.80190365552946e-05
966 1.79620939730896e-05
967 1.80237283284868e-05
968 1.79444593051414e-05
969 1.81788367035196e-05
970 1.80298689844705e-05
971 1.83656171524049e-05
972 1.82791934735249e-05
973 1.91475776292194e-05
974 1.92336205637744e-05
975 2.03871305402004e-05
976 1.97999196158705e-05
977 1.90564555202499e-05
978 1.84316760991621e-05
979 1.80313604420945e-05
980 1.79235648465337e-05
981 1.78844479767903e-05
982 1.78366909793226e-05
983 1.7867651602721e-05
984 1.78311125536013e-05
985 1.78482309278749e-05
986 1.78124631480259e-05
987 1.78395574685908e-05
988 1.77940013950106e-05
989 1.78305115294819e-05
990 1.7788096070781e-05
991 1.78239189194684e-05
992 1.77719092935291e-05
993 1.77877044098551e-05
994 1.77755568255833e-05
995 1.77977028670284e-05
996 1.77664397762456e-05
997 1.78189418704733e-05
998 1.77595520334251e-05
999 1.79856899888975e-05
1000 1.79336914110095e-05
1001 1.87487216845028e-05
1002 1.89568160777753e-05
1003 2.0440364349561e-05
1004 1.99581886217715e-05
1005 1.90734160154804e-05
1006 1.83187411675334e-05
1007 1.77998970736226e-05
1008 1.77198077633278e-05
1009 1.76728333478593e-05
1010 1.76395049011902e-05
1011 1.76605288206844e-05
1012 1.76296796974285e-05
1013 1.76566531369104e-05
1014 1.76235861379936e-05
1015 1.76418728372596e-05
1016 1.7598065110036e-05
1017 1.76411046606262e-05
1018 1.75871712730924e-05
1019 1.76242026457274e-05
1020 1.75810952125488e-05
1021 1.76182759776822e-05
1022 1.7548158613323e-05
1023 1.76170946630805e-05
1024 1.75388405123655e-05
1025 1.76300900660564e-05
1026 1.75247895386477e-05
1027 1.76908910969331e-05
1028 1.75501588017823e-05
1029 1.78894201452451e-05
1030 1.78052515655835e-05
1031 1.84853344329383e-05
1032 1.84183428477702e-05
1033 1.92809879342448e-05
1034 1.8838947219102e-05
1035 1.85950510083188e-05
1036 1.80722507785447e-05
1037 1.77269467549479e-05
1038 1.75670541811002e-05
1039 1.75322717188564e-05
1040 1.74686552130154e-05
1041 1.74903098448809e-05
1042 1.74341187255678e-05
1043 1.74655750582176e-05
1044 1.74214471284628e-05
1045 1.74681728228876e-05
1046 1.73995080006151e-05
1047 1.74548669109598e-05
1048 1.7379427507791e-05
1049 1.74335399005798e-05
1050 1.73853460809781e-05
1051 1.74328570929916e-05
1052 1.73453636360676e-05
1053 1.74479118095761e-05
1054 1.73345566052419e-05
1055 1.74993282859859e-05
1056 1.73478255662829e-05
1057 1.76399065408006e-05
1058 1.7466882873407e-05
1059 1.80145514843133e-05
1060 1.78801932060324e-05
1061 1.85949533975105e-05
1062 1.8301820815747e-05
1063 1.85393520766297e-05
1064 1.80324410292698e-05
1065 1.78104851924488e-05
1066 1.75311706831494e-05
1067 1.74125461258612e-05
1068 1.73164752999e-05
1069 1.73388264848029e-05
1070 1.72481581834916e-05
1071 1.73039190727309e-05
1072 1.72341908468709e-05
1073 1.72860411842279e-05
1074 1.72096265158572e-05
1075 1.72753486040733e-05
1076 1.72091618013681e-05
1077 1.72601468513278e-05
1078 1.71830475377277e-05
1079 1.72540953871092e-05
1080 1.71734022273995e-05
1081 1.72434425680379e-05
1082 1.71815139244913e-05
1083 1.72699340819094e-05
1084 1.71746821879815e-05
1085 1.7370908955705e-05
1086 1.72412595671467e-05
1087 1.76409628860341e-05
1088 1.75069384646953e-05
1089 1.82556496604036e-05
1090 1.79996913272085e-05
1091 1.85879029857006e-05
1092 1.80265166807025e-05
1093 1.79275854819139e-05
1094 1.74686703742211e-05
1095 1.73683369437683e-05
1096 1.71641356576657e-05
1097 1.71826201125214e-05
1098 1.70956585874293e-05
1099 1.71221840412628e-05
1100 1.70441821421008e-05
1101 1.70803521593044e-05
1102 1.70646765820237e-05
1103 1.70576729221494e-05
1104 1.70430555650469e-05
1105 1.7041300631071e-05
1106 1.70334786675497e-05
1107 1.70291218140051e-05
1108 1.70126768690437e-05
1109 1.70196522484289e-05
1110 1.69792449167261e-05
1111 1.70511689976394e-05
1112 1.69519972033072e-05
1113 1.70624230015903e-05
1114 1.69483889038702e-05
1115 1.71131577308969e-05
1116 1.69610635585293e-05
1117 1.73088938852928e-05
1118 1.71955146148761e-05
1119 1.79987891577582e-05
1120 1.79245033384845e-05
1121 1.89843820193047e-05
1122 1.8394134478239e-05
1123 1.80970976844108e-05
1124 1.74446221308244e-05
1125 1.71897806739985e-05
1126 1.69794049673655e-05
1127 1.6942152428534e-05
1128 1.69163732035571e-05
1129 1.6892758165632e-05
1130 1.6897755713785e-05
1131 1.68671496716399e-05
1132 1.68783233718273e-05
1133 1.68574544154865e-05
1134 1.68462342235998e-05
1135 1.68704847567369e-05
1136 1.68348109008676e-05
1137 1.68459760168105e-05
1138 1.68684801122865e-05
1139 1.68111038227892e-05
1140 1.68387770722589e-05
1141 1.67916639499666e-05
1142 1.68415824202128e-05
1143 1.67711908423129e-05
1144 1.68486325167194e-05
1145 1.67615254582643e-05
1146 1.68873586741114e-05
1147 1.67918655251675e-05
1148 1.70551698195354e-05
1149 1.69427905576391e-05
1150 1.77296074239663e-05
1151 1.76599170829128e-05
1152 1.90408812787979e-05
1153 1.84744704494477e-05
1154 1.82709216560717e-05
1155 1.74574786764303e-05
1156 1.69933035403247e-05
1157 1.68186755491462e-05
1158 1.67253953033963e-05
1159 1.66844249189779e-05
1160 1.67251569758164e-05
1161 1.66722667085395e-05
1162 1.67023238351049e-05
1163 1.66643101406905e-05
1164 1.66826751852867e-05
1165 1.66749800438737e-05
1166 1.66697128811677e-05
1167 1.66623386803622e-05
1168 1.66566732913154e-05
1169 1.6619817302832e-05
1170 1.66732157902416e-05
1171 1.66109125938618e-05
1172 1.66808394137519e-05
1173 1.65906775277946e-05
1174 1.66749073784445e-05
1175 1.66143808764119e-05
1176 1.67480136163967e-05
1177 1.66320142893639e-05
1178 1.69726425394146e-05
1179 1.68193806078243e-05
1180 1.769516300687e-05
1181 1.7567588058931e-05
1182 1.87350027687216e-05
1183 1.81131612322361e-05
1184 1.78291273296161e-05
1185 1.71168950089395e-05
1186 1.68362611536743e-05
1187 1.66161882555471e-05
1188 1.66057466786107e-05
1189 1.65239221767699e-05
1190 1.65661622193447e-05
1191 1.6498443012658e-05
1192 1.65242902090412e-05
1193 1.65153180877731e-05
1194 1.6513592886902e-05
1195 1.649661555847e-05
1196 1.64979287560207e-05
1197 1.64681181678006e-05
1198 1.6519082487676e-05
1199 1.64488195890655e-05
1200 1.65122267388185e-05
1201 1.64411702288092e-05
1202 1.65047104139227e-05
1203 1.64454711066853e-05
1204 1.65408236134645e-05
1205 1.64315193624809e-05
1206 1.66988876530993e-05
1207 1.65375485887687e-05
1208 1.72444833936858e-05
1209 1.71658853780343e-05
1210 1.86030185829189e-05
1211 1.82577339735701e-05
1212 1.84005897319039e-05
1213 1.73794318256704e-05
1214 1.68463744274483e-05
1215 1.65176526034649e-05
1216 1.64469301187076e-05
1217 1.63984951453422e-05
1218 1.63630026639439e-05
1219 1.63730827789088e-05
1220 1.63567266366904e-05
1221 1.63306815257336e-05
1222 1.63692478425403e-05
1223 1.63277326308275e-05
1224 1.63519071811891e-05
1225 1.63124805045989e-05
1226 1.63336714993356e-05
1227 1.6329069044918e-05
1228 1.63107029393927e-05
1229 1.62986103289242e-05
1230 1.63407271620564e-05
1231 1.62844169757648e-05
1232 1.6334702286791e-05
1233 1.62675646722121e-05
1234 1.63565144437605e-05
1235 1.62869851574321e-05
1236 1.64787330594329e-05
1237 1.6375334416896e-05
1238 1.71574400438246e-05
1239 1.72615788889274e-05
1240 1.92895831419015e-05
1241 1.89155852914347e-05
1242 1.84869636368212e-05
1243 1.71816476752795e-05
1244 1.64658453738475e-05
1245 1.62666938807732e-05
1246 1.62014676945432e-05
1247 1.62194465058718e-05
1248 1.61873942721158e-05
1249 1.62017561171623e-05
1250 1.61772282347705e-05
1251 1.61716421445135e-05
1252 1.62019312655026e-05
1253 1.61512396434915e-05
1254 1.61696570319947e-05
1255 1.61813138701561e-05
1256 1.61536380409721e-05
1257 1.61474485231849e-05
1258 1.6177126685335e-05
1259 1.61388748480817e-05
1260 1.61659347091003e-05
1261 1.61218345713365e-05
1262 1.61568450751659e-05
1263 1.61346459579548e-05
1264 1.61694619920105e-05
1265 1.60979280838802e-05
1266 1.63088938269773e-05
1267 1.61879975757451e-05
1268 1.68630050456642e-05
1269 1.70774630769976e-05
1270 1.92773760105958e-05
1271 1.92000141305826e-05
1272 1.89644970856051e-05
1273 1.72399515623312e-05
1274 1.6301193228152e-05
1275 1.61216301819422e-05
1276 1.6019111658494e-05
1277 1.60185640316612e-05
1278 1.60355513174615e-05
1279 1.60078786852758e-05
1280 1.60094822612145e-05
1281 1.60242631683793e-05
1282 1.59952966480859e-05
1283 1.5998274490503e-05
1284 1.60215725149015e-05
1285 1.59842350999639e-05
1286 1.59858910060429e-05
1287 1.60134906534992e-05
1288 1.59664820893468e-05
1289 1.6005829904131e-05
1290 1.59626766245147e-05
1291 1.5991665151649e-05
1292 1.59687764598182e-05
1293 1.60121733778773e-05
1294 1.59428893282687e-05
1295 1.61336737529716e-05
1296 1.60251435876724e-05
1297 1.66849160510019e-05
1298 1.6913271447816e-05
1299 1.91488984184751e-05
1300 1.91814346388597e-05
1301 1.90098934518268e-05
1302 1.71892059146472e-05
1303 1.61713957105292e-05
1304 1.59802585075752e-05
1305 1.58640714187364e-05
1306 1.58772134226659e-05
1307 1.5895609838612e-05
1308 1.58603074336128e-05
1309 1.58617764065561e-05
1310 1.58789465358744e-05
1311 1.5860237677412e-05
1312 1.58477834153459e-05
1313 1.58697439727185e-05
1314 1.58483598680093e-05
1315 1.58485730739066e-05
1316 1.58605162390302e-05
1317 1.58405086674307e-05
1318 1.58407355508317e-05
1319 1.58587485383777e-05
1320 1.58256188655947e-05
1321 1.58427575800424e-05
1322 1.58461682575606e-05
1323 1.58401155911925e-05
1324 1.58114569939194e-05
1325 1.59021410661708e-05
1326 1.58133924803572e-05
1327 1.60796610959579e-05
1328 1.61243209340078e-05
1329 1.76003096852284e-05
1330 1.85605955138612e-05
1331 2.15938276162397e-05
1332 1.95742116444464e-05
1333 1.64303082752504e-05
1334 1.59057194322187e-05
1335 1.57256357646851e-05
1336 1.57940858342975e-05
1337 1.57371715863164e-05
1338 1.57431390386265e-05
1339 1.57696050324674e-05
1340 1.57303783234752e-05
1341 1.5731876888303e-05
1342 1.57545444361418e-05
1343 1.57282580195428e-05
1344 1.57261975672895e-05
1345 1.57409450873836e-05
1346 1.57183865772126e-05
1347 1.57241260372309e-05
1348 1.57339565305392e-05
1349 1.57046442175179e-05
1350 1.57220473839814e-05
1351 1.57185882598831e-05
1352 1.57122545059174e-05
1353 1.56916277140873e-05
1354 1.57531920077503e-05
1355 1.5676472611581e-05
1356 1.57852324793595e-05
1357 1.57274357674986e-05
1358 1.60602454370995e-05
1359 1.61004963903366e-05
1360 1.77484176124842e-05
1361 1.84455037453013e-05
1362 2.03924857804516e-05
1363 1.84942038781166e-05
1364 1.62942528567633e-05
1365 1.57838206518157e-05
1366 1.56199595529039e-05
1367 1.56401875623935e-05
1368 1.56511510205348e-05
1369 1.56201164451808e-05
1370 1.56187432200916e-05
1371 1.56373069155613e-05
1372 1.56155793198387e-05
1373 1.56114409484154e-05
1374 1.56267330817528e-05
1375 1.56064353649121e-05
1376 1.56060050597873e-05
1377 1.5614118474172e-05
1378 1.55934175212202e-05
1379 1.56048943784626e-05
1380 1.55889791506603e-05
1381 1.56181855617277e-05
1382 1.55728886528372e-05
1383 1.56120825969275e-05
1384 1.5585795963613e-05
1385 1.56391992631733e-05
1386 1.55651874913509e-05
1387 1.58614950103164e-05
1388 1.58241859646857e-05
1389 1.70666596961411e-05
1390 1.77184403633568e-05
1391 2.03648554628622e-05
1392 1.89487928912158e-05
1393 1.67392755825446e-05
1394 1.58426851704085e-05
1395 1.55189203292494e-05
1396 1.55274307895858e-05
1397 1.54999259804889e-05
1398 1.5521632596105e-05
1399 1.54865145209904e-05
1400 1.55017047256401e-05
1401 1.55071462222267e-05
1402 1.54802744529725e-05
1403 1.54816527304646e-05
1404 1.54911366023747e-05
1405 1.54795800626495e-05
1406 1.5475873230919e-05
1407 1.54892279873309e-05
1408 1.54637714673278e-05
1409 1.54614966003663e-05
1410 1.54746979266207e-05
1411 1.54823922544622e-05
1412 1.54609909874814e-05
1413 1.54547518809167e-05
1414 1.5483315408904e-05
1415 1.54411678958688e-05
1416 1.54844714490565e-05
1417 1.5459125618289e-05
1418 1.55237734360547e-05
1419 1.54594522707718e-05
1420 1.58823165632427e-05
1421 1.61480065017372e-05
1422 1.88750765151013e-05
1423 2.00880393226299e-05
1424 2.08829306540537e-05
1425 1.77169238266828e-05
1426 1.55458021797727e-05
1427 1.54439502493275e-05
1428 1.53650505834158e-05
1429 1.53657497943271e-05
1430 1.5353411495056e-05
1431 1.53587019569024e-05
1432 1.53774296105524e-05
1433 1.53453133315828e-05
1434 1.53424240849631e-05
1435 1.53583177442407e-05
1436 1.53314381550196e-05
1437 1.53308523622719e-05
1438 1.53602028314381e-05
1439 1.5327716458291e-05
1440 1.53325881555588e-05
1441 1.5325150861667e-05
1442 1.53484453613295e-05
1443 1.5315751029199e-05
1444 1.53301149063978e-05
1445 1.53381956482512e-05
1446 1.53343147415441e-05
1447 1.53066699448701e-05
1448 1.53769009640925e-05
1449 1.52973314122562e-05
1450 1.54922165283011e-05
1451 1.54651379129511e-05
1452 1.6322366954924e-05
1453 1.69638558187302e-05
1454 2.03194160577347e-05
1455 1.97205195231476e-05
1456 1.74329037161414e-05
1457 1.5800262739063e-05
1458 1.53029267866422e-05
1459 1.52765052781056e-05
1460 1.52251345686416e-05
1461 1.5231975960539e-05
1462 1.52227610197286e-05
1463 1.5241534213839e-05
1464 1.52122904855112e-05
1465 1.52104658686092e-05
1466 1.52393555072905e-05
1467 1.5206507075316e-05
1468 1.5211765437062e-05
1469 1.52051096127259e-05
1470 1.52258642138747e-05
1471 1.51941568136138e-05
1472 1.52055199258427e-05
1473 1.52117985048328e-05
1474 1.52080983610858e-05
1475 1.51761050291022e-05
1476 1.52674232678507e-05
1477 1.51871170119833e-05
1478 1.53999303020846e-05
1479 1.54152731348667e-05
1480 1.65385272268281e-05
1481 1.7452402444551e-05
1482 2.09011024443484e-05
1483 1.94982310310898e-05
1484 1.64513001701394e-05
1485 1.53751208347153e-05
1486 1.51356098871958e-05
1487 1.51456890309376e-05
1488 1.51411055901995e-05
1489 1.51221796031997e-05
1490 1.51210913044153e-05
1491 1.51349903383391e-05
1492 1.51130643191166e-05
1493 1.51161017218904e-05
1494 1.51124564915506e-05
1495 1.51220618667125e-05
1496 1.51164879618193e-05
1497 1.51083437400779e-05
1498 1.51277525963778e-05
1499 1.51015512446229e-05
1500 1.51186156225336e-05
1501 1.50993374616881e-05
1502 1.51513664494729e-05
1503 1.50964827563271e-05
1504 1.51928513849597e-05
1505 1.51626834177243e-05
1506 1.54815448709655e-05
1507 1.56072980859179e-05
1508 1.74901012504058e-05
1509 1.8597564699796e-05
1510 2.05917864839478e-05
1511 1.80786766326158e-05
1512 1.55882309940125e-05
1513 1.51239370662637e-05
1514 1.50172542028315e-05
1515 1.50344273541769e-05
1516 1.50202495241558e-05
1517 1.5038016802027e-05
1518 1.50128368203717e-05
1519 1.50161342138411e-05
1520 1.50122988307189e-05
1521 1.50307696151764e-05
1522 1.50044835725005e-05
1523 1.50106995722155e-05
1524 1.50235722169434e-05
1525 1.50050943932278e-05
1526 1.50060442956068e-05
1527 1.50064867412425e-05
1528 1.5021844012697e-05
1529 1.50012830433255e-05
1530 1.50149402009525e-05
1531 1.5019265214633e-05
1532 1.50105685707835e-05
1533 1.50056642902463e-05
1534 1.50098008151467e-05
1535 1.50214211322997e-05
1536 1.50228924775675e-05
1537 1.50023300382607e-05
1538 1.50739926314536e-05
1539 1.50264206606465e-05
1540 1.53838571930187e-05
1541 1.57933225914952e-05
1542 1.93742049301449e-05
1543 2.17005224700806e-05
1544 2.16789442775145e-05
1545 1.70488581634487e-05
1546 1.49426455435808e-05
1547 1.49836495104694e-05
1548 1.48814312370682e-05
1549 1.48936062336347e-05
1550 1.4898846342426e-05
1551 1.48730108837647e-05
1552 1.48735135456768e-05
1553 1.48753476936214e-05
1554 1.4886905624234e-05
1555 1.48611894918993e-05
1556 1.4867764665194e-05
1557 1.48594375404265e-05
1558 1.48825090584559e-05
1559 1.48539393971525e-05
1560 1.48650247417947e-05
1561 1.48532052710593e-05
1562 1.48836122937368e-05
1563 1.48505263197762e-05
1564 1.48737812040167e-05
1565 1.48503277093148e-05
1566 1.49162682832049e-05
1567 1.48509872763825e-05
1568 1.50075578493336e-05
1569 1.50032483996476e-05
1570 1.57112334067122e-05
1571 1.6364959642523e-05
1572 1.96178260245539e-05
1573 1.95240909421912e-05
1574 1.76023446685214e-05
1575 1.55658754095178e-05
1576 1.49555026882631e-05
1577 1.48331252964873e-05
1578 1.48046853536421e-05
1579 1.48195646958449e-05
1580 1.47940193007123e-05
1581 1.47946629818207e-05
1582 1.47926114903996e-05
1583 1.48074171493207e-05
1584 1.47842472255633e-05
1585 1.47896932332081e-05
1586 1.4785851244703e-05
1587 1.4807266650152e-05
1588 1.47823214673437e-05
1589 1.47895869102577e-05
1590 1.47994369443794e-05
1591 1.47882496133178e-05
1592 1.47904036933433e-05
1593 1.47939900143612e-05
1594 1.48027172204301e-05
1595 1.47969989510166e-05
1596 1.47930994511825e-05
1597 1.48073052019804e-05
1598 1.4810360246198e-05
1599 1.48323976731923e-05
1600 1.47978234212864e-05
1601 1.50305523063388e-05
1602 1.52757669722448e-05
1603 1.78038164837524e-05
1604 2.09483567985558e-05
1605 2.40018166999256e-05
1606 1.86751837172316e-05
1607 1.47696771763961e-05
1608 1.48438059395239e-05
1609 1.46932093003649e-05
1610 1.46959474252029e-05
1611 1.47072446616825e-05
1612 1.46767331545838e-05
1613 1.46816257076132e-05
1614 1.46757720731472e-05
1615 1.46744551763334e-05
1616 1.46893490877709e-05
1617 1.46648359304891e-05
1618 1.46725754763821e-05
1619 1.46672373420031e-05
1620 1.46860831256035e-05
1621 1.46605529756094e-05
1622 1.46735069468384e-05
1623 1.46626292827357e-05
1624 1.46878911340131e-05
1625 1.46573984816101e-05
1626 1.46779097192073e-05
1627 1.46613092244507e-05
1628 1.47024321628386e-05
1629 1.46546025630023e-05
1630 1.47212731125279e-05
1631 1.46744258429088e-05
1632 1.49163155760412e-05
1633 1.49939242537478e-05
1634 1.64630251617837e-05
1635 1.78968311965733e-05
1636 2.13070228820555e-05
1637 1.89357784354804e-05
1638 1.54168542443323e-05
1639 1.4720803022783e-05
1640 1.4589356154282e-05
1641 1.46211949383357e-05
1642 1.45995811142718e-05
1643 1.46023862828137e-05
1644 1.45964182394209e-05
1645 1.46105094187554e-05
1646 1.45882683293408e-05
1647 1.4595419268737e-05
1648 1.45898721926052e-05
1649 1.46064677979396e-05
1650 1.45887971640946e-05
1651 1.45929701407255e-05
1652 1.45898489112284e-05
1653 1.46063965269505e-05
1654 1.45857759026313e-05
1655 1.46006890511252e-05
1656 1.45839248624213e-05
1657 1.46302563934775e-05
1658 1.45804747293532e-05
1659 1.46923821540135e-05
1660 1.46598310171164e-05
1661 1.52462277984711e-05
1662 1.59009888558437e-05
1663 1.94593267468512e-05
1664 2.00841889288483e-05
1665 1.83872231178661e-05
1666 1.55601831908214e-05
1667 1.46414952957308e-05
1668 1.455117587712e-05
1669 1.45035035785668e-05
1670 1.45214459386089e-05
1671 1.4510114192845e-05
1672 1.45228168766387e-05
1673 1.45014550465561e-05
1674 1.45073273749574e-05
1675 1.45026024509498e-05
1676 1.45047600752513e-05
1677 1.45166547222964e-05
1678 1.44975119154012e-05
1679 1.45027133515718e-05
1680 1.44999872704332e-05
1681 1.45155868129798e-05
1682 1.449745697224e-05
1683 1.45069491335192e-05
1684 1.44941956627953e-05
1685 1.45304697518611e-05
1686 1.4494578576052e-05
1687 1.45551248360576e-05
1688 1.45148012467011e-05
1689 1.47852189478215e-05
1690 1.49591055587095e-05
1691 1.69848080995294e-05
1692 1.90399070727487e-05
1693 2.20075415757748e-05
1694 1.83286367914093e-05
1695 1.47698109977945e-05
1696 1.45220885539032e-05
1697 1.44101823154053e-05
1698 1.44400414607837e-05
1699 1.44193532483072e-05
1700 1.44184212613752e-05
1701 1.44150542298327e-05
1702 1.44114635469705e-05
1703 1.44212998747228e-05
1704 1.44038156766335e-05
1705 1.44070614722658e-05
1706 1.44033316638037e-05
1707 1.44034869631327e-05
1708 1.44146882603557e-05
1709 1.43969987078663e-05
1710 1.44048228558624e-05
1711 1.4397094478813e-05
1712 1.44183102235296e-05
1713 1.4392959454046e-05
1714 1.44106511510422e-05
1715 1.43938413623701e-05
1716 1.44415848817303e-05
1717 1.43903803984102e-05
1718 1.45259733188574e-05
1719 1.45306717720395e-05
1720 1.53609631770024e-05
1721 1.64119912171579e-05
1722 2.07442274748892e-05
1723 2.0203202561575e-05
1724 1.6625646241053e-05
1725 1.47777457817355e-05
1726 1.43812686874689e-05
1727 1.43414430464617e-05
1728 1.43104751293421e-05
1729 1.43156510867826e-05
1730 1.43066374058698e-05
1731 1.43080747716517e-05
1732 1.43028563108771e-05
1733 1.43144273634377e-05
1734 1.42968246974462e-05
1735 1.43011989219843e-05
1736 1.42962012228409e-05
1737 1.42996179919308e-05
1738 1.43051770309377e-05
1739 1.4298124444867e-05
1740 1.42922439514415e-05
1741 1.4304782568697e-05
1742 1.42913195375627e-05
1743 1.43020411567107e-05
1744 1.43036261968099e-05
1745 1.43185806713575e-05
1746 1.42937612701566e-05
1747 1.43630180611254e-05
1748 1.43219327135569e-05
1749 1.47170312518874e-05
1750 1.51540219253121e-05
1751 1.85787275026605e-05
1752 2.05422319989879e-05
1753 2.03840527350785e-05
1754 1.61207983908618e-05
1755 1.43271113066668e-05
1756 1.42714043933623e-05
1757 1.41839878691208e-05
1758 1.42044211335346e-05
1759 1.41896538750075e-05
1760 1.41866497389564e-05
1761 1.41839693141854e-05
1762 1.41932740755912e-05
1763 1.41742694532709e-05
1764 1.41785259657468e-05
1765 1.41760028968818e-05
1766 1.41756037721485e-05
1767 1.41843814254194e-05
1768 1.4171250866557e-05
1769 1.41748522701768e-05
1770 1.41719620008196e-05
1771 1.41749543982606e-05
1772 1.41685055541352e-05
1773 1.41894484682048e-05
1774 1.41694429975914e-05
1775 1.41797664774401e-05
1776 1.41711335137629e-05
1777 1.41952225511943e-05
1778 1.41828969764113e-05
1779 1.42612729621128e-05
1780 1.42459205600964e-05
1781 1.48960814381205e-05
1782 1.58978414743771e-05
1783 2.07842025723437e-05
1784 2.10922327794627e-05
1785 1.71934177322086e-05
1786 1.45958907005195e-05
1787 1.41250830623108e-05
1788 1.41284163435174e-05
1789 1.40821706020589e-05
1790 1.40943021453843e-05
1791 1.40824032697218e-05
1792 1.4084317049079e-05
1793 1.40794569380986e-05
1794 1.40775572048391e-05
1795 1.40766110998669e-05
1796 1.40884445247735e-05
1797 1.40711984983888e-05
1798 1.40801528165646e-05
1799 1.40742828653728e-05
1800 1.40796234955332e-05
1801 1.4076240535843e-05
1802 1.40902173013657e-05
1803 1.40771744741031e-05
1804 1.40850015846183e-05
1805 1.40781925308531e-05
1806 1.40891394186937e-05
1807 1.40752363013696e-05
1808 1.41086453675499e-05
1809 1.409649429851e-05
1810 1.41828753545958e-05
1811 1.41970849441009e-05
1812 1.50650159751464e-05
1813 1.66085838078089e-05
1814 2.22936527221052e-05
1815 2.09462533455884e-05
1816 1.55015404228465e-05
1817 1.41587074797833e-05
1818 1.39786475750192e-05
1819 1.40263942953744e-05
1820 1.39939121730492e-05
1821 1.39945597457114e-05
1822 1.39910048515546e-05
1823 1.39868062083437e-05
1824 1.39874624176528e-05
1825 1.39816340523069e-05
1826 1.39837721215841e-05
1827 1.39796663605196e-05
1828 1.39911859071695e-05
1829 1.39732334223375e-05
1830 1.39820867310902e-05
1831 1.39755747987635e-05
1832 1.39894836395094e-05
1833 1.39763740367727e-05
1834 1.40024786206006e-05
1835 1.39702314059242e-05
1836 1.40185937542725e-05
1837 1.39838492678734e-05
1838 1.41615196098677e-05
1839 1.42264913298717e-05
1840 1.54024310519141e-05
1841 1.68507386257843e-05
1842 2.0820189730486e-05
1843 1.8882643107343e-05
1844 1.51271314221546e-05
1845 1.40733407665294e-05
1846 1.39100049616481e-05
1847 1.39309890663242e-05
1848 1.39097352240825e-05
1849 1.39138469776867e-05
1850 1.39071305107485e-05
1851 1.39069292868221e-05
1852 1.39045409310867e-05
1853 1.39036992714558e-05
1854 1.39023557728102e-05
1855 1.39016453308827e-05
1856 1.39099143798838e-05
1857 1.38978082575569e-05
1858 1.3904782819818e-05
1859 1.38964946176934e-05
1860 1.39088770203522e-05
1861 1.38953917843132e-05
1862 1.39318436773728e-05
1863 1.38913999210288e-05
1864 1.3984827373914e-05
1865 1.39638252658614e-05
1866 1.44466693599021e-05
1867 1.50379949745982e-05
1868 1.8343731198911e-05
1869 1.95610229813958e-05
1870 1.86438641502917e-05
1871 1.52324219575561e-05
1872 1.40352775184205e-05
1873 1.38660586372907e-05
1874 1.38147481618667e-05
1875 1.38334178760857e-05
1876 1.38164431149335e-05
1877 1.38193532559505e-05
1878 1.38133190672463e-05
1879 1.3814457068051e-05
1880 1.38112350600572e-05
1881 1.38114923018406e-05
1882 1.38079813480907e-05
1883 1.38074915532194e-05
1884 1.38117089880652e-05
1885 1.38113057190914e-05
1886 1.38082561798036e-05
1887 1.38130984854712e-05
1888 1.38199636983138e-05
1889 1.38120678059295e-05
1890 1.38115806938011e-05
1891 1.38201831552109e-05
1892 1.3814559127745e-05
1893 1.38308736103987e-05
1894 1.38191653333841e-05
1895 1.38879226430078e-05
1896 1.38968848122367e-05
1897 1.4442471063969e-05
1898 1.54999241477327e-05
1899 2.08370452798867e-05
1900 2.17434994675969e-05
1901 1.7353305659995e-05
1902 1.42464751076155e-05
1903 1.37735577454556e-05
1904 1.37772264365843e-05
1905 1.37217474951434e-05
1906 1.37331219574044e-05
1907 1.37228309422532e-05
1908 1.37202296546235e-05
1909 1.37165214781909e-05
1910 1.37166773841457e-05
1911 1.37162294220339e-05
1912 1.37127131725201e-05
1913 1.37169037723872e-05
1914 1.37124084260698e-05
1915 1.37148610757087e-05
1916 1.37154271913076e-05
1917 1.37257498558085e-05
1918 1.37110626252657e-05
1919 1.37206606876106e-05
1920 1.37179832933043e-05
1921 1.37381124596381e-05
1922 1.37167703422314e-05
1923 1.37961405806131e-05
1924 1.38009713910314e-05
1925 1.43147895803608e-05
1926 1.51528644178889e-05
1927 1.90118989928934e-05
1928 2.00857862866499e-05
1929 1.79982985821781e-05
1930 1.46145830752786e-05
1931 1.38125370283682e-05
1932 1.36826973844428e-05
1933 1.36416223970315e-05
1934 1.36563700507963e-05
1935 1.36417472371697e-05
1936 1.36445534923979e-05
1937 1.36382091588949e-05
1938 1.36393818639391e-05
1939 1.36411076980814e-05
1940 1.36396004166706e-05
1941 1.36436347961677e-05
1942 1.3640166943496e-05
1943 1.36448033143388e-05
1944 1.3640557763317e-05
1945 1.3648158462054e-05
1946 1.36446529204193e-05
1947 1.36581252512258e-05
1948 1.36537933683734e-05
1949 1.37004208977665e-05
1950 1.37001930751168e-05
1951 1.40423935923728e-05
1952 1.45710113081954e-05
1953 1.72768915254196e-05
1954 1.94434473357319e-05
1955 1.98424617394721e-05
1956 1.57080429161738e-05
1957 1.39318497081042e-05
1958 1.3625813115592e-05
1959 1.35703322317049e-05
1960 1.35931405291778e-05
1961 1.35745272085153e-05
1962 1.35786367749802e-05
1963 1.35740820015329e-05
1964 1.35763132980138e-05
1965 1.35745566991474e-05
1966 1.35766273965388e-05
1967 1.35742863438537e-05
1968 1.35767771727302e-05
1969 1.35783782777565e-05
1970 1.35776130338883e-05
1971 1.35798863802883e-05
1972 1.35784282027096e-05
1973 1.35853748823855e-05
1974 1.35862666530429e-05
1975 1.35886219529979e-05
1976 1.35811430439503e-05
1977 1.36286585408207e-05
1978 1.36341313354826e-05
1979 1.39455022538648e-05
1980 1.45865641583143e-05
1981 1.77073794391802e-05
1982 2.04730777868534e-05
1983 2.02952088335095e-05
1984 1.5392569551409e-05
1985 1.37834954188065e-05
1986 1.35889431112091e-05
1987 1.35141811807848e-05
1988 1.35397998888287e-05
1989 1.35186903742479e-05
1990 1.35207422453654e-05
1991 1.35182081697494e-05
1992 1.35185070573307e-05
1993 1.35159637513205e-05
1994 1.35174011011152e-05
1995 1.35201052668954e-05
1996 1.35158331344698e-05
1997 1.35200391175871e-05
1998 1.35195998600679e-05
1999 1.35207009859251e-05
};
\addlegendentry{Train}
\addplot [semithick, black]
table {%
0 0.057849682867527
1 0.0559819377958775
2 0.0542273670434952
3 0.0525680147111416
4 0.0509869009256363
5 0.0494754239916801
6 0.048027615994215
7 0.046637337654829
8 0.0452980510890484
9 0.0440098457038403
10 0.0427696518599987
11 0.0415775701403618
12 0.0404295772314072
13 0.0393218956887722
14 0.0382511578500271
15 0.0372139774262905
16 0.0362074859440327
17 0.0352296009659767
18 0.034278754144907
19 0.0333534479141235
20 0.0324525609612465
21 0.031574685126543
22 0.0307186506688595
23 0.0298842433840036
24 0.0290706660598516
25 0.0282768737524748
26 0.0275025926530361
27 0.0267476458102465
28 0.0260110832750797
29 0.0252926498651505
30 0.0245923716574907
31 0.0239095874130726
32 0.0232443325221539
33 0.022596538066864
34 0.0219651032239199
35 0.0213483646512032
36 0.0207448396831751
37 0.0201545283198357
38 0.0195764489471912
39 0.0190106276422739
40 0.018454173579812
41 0.0179061610251665
42 0.0173669494688511
43 0.0168323945254087
44 0.0162893682718277
45 0.0157253164798021
46 0.0151341231539845
47 0.0143508175387979
48 0.013534658588469
49 0.0128450375050306
50 0.0122303599491715
51 0.011562742292881
52 0.0108609022572637
53 0.0102208098396659
54 0.00960944686084986
55 0.00904358457773924
56 0.00845036841928959
57 0.00784381106495857
58 0.0072660269215703
59 0.00673610204830766
60 0.00624109478667378
61 0.005775504745543
62 0.005336944013834
63 0.00491087231785059
64 0.00448384555056691
65 0.0040838485583663
66 0.0037132955621928
67 0.00336974067613482
68 0.00304863695055246
69 0.00277013028971851
70 0.00253134104423225
71 0.00227881851606071
72 0.00205948203802109
73 0.0018689880380407
74 0.00170827598776668
75 0.00156484043691307
76 0.00144698983058333
77 0.00133968994487077
78 0.00124193774536252
79 0.00116083747707307
80 0.00108586787246168
81 0.00100289424881339
82 0.000935226213186979
83 0.000878022750839591
84 0.000827959447633475
85 0.000783597060944885
86 0.000743714859709144
87 0.000707543629687279
88 0.000675055489409715
89 0.000645150663331151
90 0.000617741956375539
91 0.000593080359976739
92 0.00057066889712587
93 0.000550186785403639
94 0.000531486584804952
95 0.000514419924002141
96 0.000498924811836332
97 0.000484806776512414
98 0.000471864506835118
99 0.000459972477983683
100 0.000448913750005886
101 0.000438344897702336
102 0.000427711580414325
103 0.000416324066463858
104 0.000403849204303697
105 0.000392922927858308
106 0.000383673264877871
107 0.000375280535081401
108 0.000367595144780353
109 0.000360718870069832
110 0.00035424716770649
111 0.000348315748851746
112 0.000342820247169584
113 0.000335951888700947
114 0.000328995374729857
115 0.000322504551149905
116 0.000316820194711909
117 0.000311697891447693
118 0.000307081179926172
119 0.000302735803415999
120 0.0002987889747601
121 0.000294995668809861
122 0.00029151514172554
123 0.00028823665343225
124 0.000285149406408891
125 0.000282183173112571
126 0.000279377883998677
127 0.000276733335340396
128 0.00027425741427578
129 0.000271219585556537
130 0.000267194001935422
131 0.00026378475013189
132 0.000260900473222136
133 0.000258285348536447
134 0.000255897030001506
135 0.000253680511377752
136 0.000251627003308386
137 0.000249682052526623
138 0.000247853604378179
139 0.000246110954321921
140 0.000244473776547238
141 0.000242880312725902
142 0.000241386413108557
143 0.000239927947404794
144 0.000238546344917268
145 0.000237187778111547
146 0.000235799408983439
147 0.000234484861721285
148 0.000233333266805857
149 0.000232078280532733
150 0.000230935067520477
151 0.000229809869779274
152 0.000228704608161934
153 0.000227632088353857
154 0.000226582924369723
155 0.000225557610974647
156 0.000224553339648992
157 0.000223554627154954
158 0.000222577116801403
159 0.000221623558900319
160 0.000220674075535499
161 0.0002197425783379
162 0.00021884951274842
163 0.000217959444853477
164 0.000217065680772066
165 0.00021619965264108
166 0.00021531838865485
167 0.000214470710488968
168 0.000213639184948988
169 0.000212800805456936
170 0.000211960781598464
171 0.000211132050026208
172 0.00021030854259152
173 0.000209505786187947
174 0.000208705445402302
175 0.000207881748792715
176 0.00020707439398393
177 0.00020628928905353
178 0.000205516480491497
179 0.00020472145115491
180 0.000203946081455797
181 0.00020318383758422
182 0.00020237869466655
183 0.000201620525331236
184 0.000200835173018277
185 0.000200079171918333
186 0.000199270536541007
187 0.000198538371478207
188 0.000197742701857351
189 0.000196987602976151
190 0.000196205466636457
191 0.000195452506886795
192 0.000194674197700806
193 0.000193917658179998
194 0.000193138315808028
195 0.000192368810530752
196 0.000191590093891136
197 0.000190811813808978
198 0.000190047809155658
199 0.000189243364729919
200 0.000188474266906269
201 0.000187635247129947
202 0.000186823031981476
203 0.000186041463166475
204 0.000185199693078175
205 0.000184395001269877
206 0.00018357734370511
207 0.000182715884875506
208 0.000181890645762905
209 0.000181029317900538
210 0.000180177899892442
211 0.000179314709384926
212 0.000178416084963828
213 0.000177577152498998
214 0.000176555986399762
215 0.000175687280716375
216 0.000174681641510688
217 0.000173676962731406
218 0.00017271627439186
219 0.000171752704773098
220 0.000170745886862278
221 0.000169645398273133
222 0.000168601516634226
223 0.000167527789017186
224 0.000166454788995907
225 0.000165354605996981
226 0.000164237309945747
227 0.000163116579642519
228 0.000161966585437767
229 0.000160785595653579
230 0.000159601520863362
231 0.000158416471094824
232 0.000157192305778153
233 0.000155934452777728
234 0.000154690525960177
235 0.000153444067109376
236 0.000152161854202859
237 0.000150891777593642
238 0.000149592466186732
239 0.000148266481119208
240 0.000146990743814968
241 0.000145641010021791
242 0.000144263554830104
243 0.000142870907438919
244 0.000141525408253074
245 0.000140251824632287
246 0.000138916686410084
247 0.000137624170747586
248 0.000136258808197454
249 0.000134920526761562
250 0.000133535912027583
251 0.000132186498376541
252 0.00013077724725008
253 0.000129408334032632
254 0.000128024854348041
255 0.000126626720884815
256 0.000125177728477865
257 0.000123762394650839
258 0.000122370955068618
259 0.000120959084597416
260 0.000119554817501921
261 0.000118140385893639
262 0.000116743380203843
263 0.000115346025268082
264 0.000113911133666988
265 0.000112496396468487
266 0.000111100504000206
267 0.000109706961666234
268 0.000108225678559393
269 0.000106859668449033
270 0.00010554380423855
271 0.000104144011856988
272 0.000102763813629281
273 0.000101341553090606
274 9.99795156531036e-05
275 9.8583048384171e-05
276 9.72682319115847e-05
277 9.59080425673164e-05
278 9.45655629038811e-05
279 9.32389011722989e-05
280 9.19614394661039e-05
281 9.07016583369114e-05
282 8.9441571617499e-05
283 8.82135864230804e-05
284 8.69999930728227e-05
285 8.57568738865666e-05
286 8.46537805045955e-05
287 8.35459868540056e-05
288 8.24823000584729e-05
289 8.14124941825867e-05
290 8.04272131063044e-05
291 7.94056031736545e-05
292 7.84624062362127e-05
293 7.75144071667455e-05
294 7.66429875511676e-05
295 7.57529851398431e-05
296 7.49125683796592e-05
297 7.41077310522087e-05
298 7.32697080820799e-05
299 7.25203717593104e-05
300 7.17620205250569e-05
301 7.10021195118316e-05
302 7.02947727404535e-05
303 6.96193310432136e-05
304 6.88999425619841e-05
305 6.82643367326818e-05
306 6.77544012432918e-05
307 6.75382034387439e-05
308 6.7010878410656e-05
309 6.64043982396834e-05
310 6.57434720778838e-05
311 6.51287991786376e-05
312 6.44474202999845e-05
313 6.38605924905278e-05
314 6.31546499789692e-05
315 6.26322871539742e-05
316 6.20511855231598e-05
317 6.14383097854443e-05
318 6.08819427725393e-05
319 6.03354492341168e-05
320 5.981425056234e-05
321 5.92757969570812e-05
322 5.88100228924304e-05
323 5.83472610742319e-05
324 5.78850558667909e-05
325 5.74084151594434e-05
326 5.69583280594088e-05
327 5.65477312193252e-05
328 5.61210872547235e-05
329 5.56996346858796e-05
330 5.53107856831048e-05
331 5.49050928384531e-05
332 5.45172333659139e-05
333 5.41402441740502e-05
334 5.37736596015748e-05
335 5.33895290573128e-05
336 5.30528886883985e-05
337 5.26916919625364e-05
338 5.23423186677974e-05
339 5.19997120136395e-05
340 5.16611398779787e-05
341 5.13254271936603e-05
342 5.10239660798106e-05
343 5.0946280680364e-05
344 5.04532617924269e-05
345 5.00947098771576e-05
346 4.97664186696056e-05
347 4.94305822940078e-05
348 4.91230930492748e-05
349 4.87705729028676e-05
350 4.83653602714185e-05
351 4.78255897178315e-05
352 4.74670159746893e-05
353 4.71293424197938e-05
354 4.68224679934792e-05
355 4.67378849862143e-05
356 4.63011019746773e-05
357 4.59070797660388e-05
358 4.55421322840266e-05
359 4.51935375167523e-05
360 4.51009873358998e-05
361 4.46738631580956e-05
362 4.43312310380861e-05
363 4.39949908468407e-05
364 4.39186987932771e-05
365 4.35270594607573e-05
366 4.32201159128454e-05
367 4.29201609222218e-05
368 4.28480125265196e-05
369 4.24665086029563e-05
370 4.2162984755123e-05
371 4.20955584559124e-05
372 4.187046943116e-05
373 4.14888600062113e-05
374 4.13913585362025e-05
375 4.11694127251394e-05
376 4.08072992286179e-05
377 4.07247825933155e-05
378 4.04771744797472e-05
379 4.01731340389233e-05
380 4.00494136556517e-05
381 3.98312513425481e-05
382 3.94706039514858e-05
383 3.9487025787821e-05
384 3.90893255826086e-05
385 3.88970220228657e-05
386 3.87653053621762e-05
387 3.84672930522356e-05
388 3.8418329495471e-05
389 3.81128884328064e-05
390 3.79107950720936e-05
391 3.77904761990067e-05
392 3.75089257431682e-05
393 3.74855189875234e-05
394 3.71795722458046e-05
395 3.69907065760344e-05
396 3.68870896636508e-05
397 3.66175154340453e-05
398 3.66020540241152e-05
399 3.63145409210119e-05
400 3.62818063877057e-05
401 3.593782093958e-05
402 3.57137723767664e-05
403 3.57039607479237e-05
404 3.54529693140648e-05
405 3.53873583662789e-05
406 3.5203749575885e-05
407 3.51342205249239e-05
408 3.49153742718045e-05
409 3.4913573472295e-05
410 3.46644737874158e-05
411 3.46940105373506e-05
412 3.44008622050751e-05
413 3.43676074407995e-05
414 3.41917148034554e-05
415 3.41327940986957e-05
416 3.38494428433478e-05
417 3.39114594680723e-05
418 3.35977056238335e-05
419 3.36052216880489e-05
420 3.34621763613541e-05
421 3.33633324771654e-05
422 3.31969240505714e-05
423 3.30611655954272e-05
424 3.29275098920334e-05
425 3.28640126099344e-05
426 3.27534544339869e-05
427 3.26620029227342e-05
428 3.25402725138701e-05
429 3.22805499308743e-05
430 3.22224150295369e-05
431 3.20858525810763e-05
432 3.20616127282847e-05
433 3.19047321681865e-05
434 3.1789062632015e-05
435 3.16671830660198e-05
436 3.15342840622179e-05
437 3.14652243105229e-05
438 3.1307772587752e-05
439 3.11247167701367e-05
440 3.11176008835901e-05
441 3.09012393699959e-05
442 3.09567367366981e-05
443 3.07450573018286e-05
444 3.08220369333867e-05
445 3.05844296235591e-05
446 3.04581044474617e-05
447 3.04360710288165e-05
448 3.03567157970974e-05
449 3.03140550386161e-05
450 3.01689015032025e-05
451 3.01471463899361e-05
452 2.99520361295436e-05
453 2.99168223136803e-05
454 2.97359820251586e-05
455 2.96927446470363e-05
456 2.95279787678737e-05
457 2.94532528641867e-05
458 2.93562352453591e-05
459 2.92736167466501e-05
460 2.92311488010455e-05
461 2.91281157842604e-05
462 2.89859708573204e-05
463 2.89887357212137e-05
464 2.88786195596913e-05
465 2.87987786578014e-05
466 2.87986858893419e-05
467 2.86785052594496e-05
468 2.85920195892686e-05
469 2.85856531263562e-05
470 2.849692646123e-05
471 2.84020534309093e-05
472 2.84017587546259e-05
473 2.83100125670899e-05
474 2.82195433101151e-05
475 2.82239543594187e-05
476 2.80814219877357e-05
477 2.80560852843337e-05
478 2.80481835943647e-05
479 2.79804498859448e-05
480 2.78881361737149e-05
481 2.78870629699668e-05
482 2.77930066658882e-05
483 2.76976224995451e-05
484 2.77154322247952e-05
485 2.75812672043685e-05
486 2.745387791947e-05
487 2.74233989330241e-05
488 2.7284489988233e-05
489 2.71737117145676e-05
490 2.71014559984906e-05
491 2.70719374384498e-05
492 2.693884925975e-05
493 2.68556705123046e-05
494 2.68327403318835e-05
495 2.67183004325489e-05
496 2.66328152065398e-05
497 2.66427960013971e-05
498 2.64552854787325e-05
499 2.60942033492029e-05
500 2.59286225627875e-05
501 2.5909905161825e-05
502 2.57596075243782e-05
503 2.56959938269574e-05
504 2.56443472608225e-05
505 2.55667473538779e-05
506 2.54600690823281e-05
507 2.54187161772279e-05
508 2.53417401836487e-05
509 2.52503450610675e-05
510 2.52128411375452e-05
511 2.51823366852477e-05
512 2.50686643994413e-05
513 2.50597768172156e-05
514 2.49998774961568e-05
515 2.49717049882747e-05
516 2.48321211984148e-05
517 2.49259974225424e-05
518 2.47428361035418e-05
519 2.47617535933387e-05
520 2.46122053795261e-05
521 2.47289590333821e-05
522 2.46203726419481e-05
523 2.45605060626986e-05
524 2.45539649768034e-05
525 2.44925413426245e-05
526 2.441216565785e-05
527 2.43863432842772e-05
528 2.44067505263956e-05
529 2.43854210566496e-05
530 2.42520982283168e-05
531 2.43580871028826e-05
532 2.41866473515984e-05
533 2.42337900999701e-05
534 2.40669705817709e-05
535 2.42715595959453e-05
536 2.40055887843482e-05
537 2.42816968238913e-05
538 2.38540051213931e-05
539 2.4324737751158e-05
540 2.38330812862841e-05
541 2.44056664087111e-05
542 2.3892422177596e-05
543 2.47568714257795e-05
544 2.39768232859205e-05
545 2.53515954682371e-05
546 2.47160023718607e-05
547 2.6711990358308e-05
548 2.55497034231666e-05
549 2.82333658105927e-05
550 2.59829994320171e-05
551 2.78521092695883e-05
552 2.46881336352089e-05
553 2.5668608941487e-05
554 2.38162956520682e-05
555 2.44955535890767e-05
556 2.34504022955662e-05
557 2.40209392359247e-05
558 2.34483341046143e-05
559 2.38886714214459e-05
560 2.33224909607088e-05
561 2.37577551160939e-05
562 2.32596339628799e-05
563 2.3805814635125e-05
564 2.32638685702113e-05
565 2.38417887885589e-05
566 2.32606198551366e-05
567 2.42096302827122e-05
568 2.34625986195169e-05
569 2.48264041147195e-05
570 2.39558412431506e-05
571 2.63575584540376e-05
572 2.52744393947069e-05
573 2.8498454412329e-05
574 2.55840004683705e-05
575 2.77776471193647e-05
576 2.40450099227019e-05
577 2.53075777436607e-05
578 2.31574431381887e-05
579 2.39597047766438e-05
580 2.28422959480667e-05
581 2.34901181102032e-05
582 2.28910121222725e-05
583 2.33616319746943e-05
584 2.27775472012581e-05
585 2.32557231356623e-05
586 2.27419786824612e-05
587 2.32294951274525e-05
588 2.27945638471283e-05
589 2.34249018831179e-05
590 2.27358723350335e-05
591 2.37922758969944e-05
592 2.2989444914856e-05
593 2.48903379542753e-05
594 2.40110384766012e-05
595 2.76283790299203e-05
596 2.58263007708592e-05
597 2.98484028462553e-05
598 2.51747569564031e-05
599 2.69443153229076e-05
600 2.31199355766876e-05
601 2.41343786910875e-05
602 2.24447412620066e-05
603 2.31192025239579e-05
604 2.23512251977809e-05
605 2.2760143110645e-05
606 2.23426577576902e-05
607 2.27317123062676e-05
608 2.22877115447773e-05
609 2.25798012252199e-05
610 2.22343733184971e-05
611 2.24962041102117e-05
612 2.21871632675175e-05
613 2.24332961806795e-05
614 2.22274084080709e-05
615 2.26902848226018e-05
616 2.21325244638138e-05
617 2.32800921367016e-05
618 2.27397031267174e-05
619 2.55684153671609e-05
620 2.54798942478374e-05
621 3.20041035593022e-05
622 2.79394134850008e-05
623 3.01638065138832e-05
624 2.31973081099568e-05
625 2.37658277910668e-05
626 2.18875302380184e-05
627 2.24584437091835e-05
628 2.20254169107648e-05
629 2.21624522964703e-05
630 2.19911980821053e-05
631 2.20509009523084e-05
632 2.19208268390503e-05
633 2.19967787415953e-05
634 2.1919982827967e-05
635 2.20184792851796e-05
636 2.19826924876543e-05
637 2.17970282392344e-05
638 2.1963471226627e-05
639 2.17300166696077e-05
640 2.1972477043164e-05
641 2.16923417610815e-05
642 2.23795741476351e-05
643 2.1936466509942e-05
644 2.39328055613441e-05
645 2.37127023865469e-05
646 2.96492089546518e-05
647 2.87525890598772e-05
648 3.42160819855053e-05
649 2.42827081819996e-05
650 2.43623871938325e-05
651 2.15468844544375e-05
652 2.22378803300671e-05
653 2.1588166418951e-05
654 2.17849901673617e-05
655 2.1563866539509e-05
656 2.16450898733456e-05
657 2.1547035430558e-05
658 2.1586854927591e-05
659 2.15310210478492e-05
660 2.158396091545e-05
661 2.15872460103128e-05
662 2.14259343920276e-05
663 2.15398522414034e-05
664 2.1373951312853e-05
665 2.16638909478206e-05
666 2.1266438125167e-05
667 2.20244855881901e-05
668 2.15467662201263e-05
669 2.36124051298248e-05
670 2.35464376601158e-05
671 2.97872829833068e-05
672 2.89352738036541e-05
673 3.42299063049722e-05
674 2.38280408666469e-05
675 2.38001321122283e-05
676 2.11770784517284e-05
677 2.18069653783459e-05
678 2.12647501030006e-05
679 2.1410047338577e-05
680 2.12886261579115e-05
681 2.13246330531547e-05
682 2.1220772396191e-05
683 2.13083385460777e-05
684 2.11826154554728e-05
685 2.12883351196069e-05
686 2.12689428735757e-05
687 2.12539580388693e-05
688 2.11307669815142e-05
689 2.12836403079564e-05
690 2.10763409995707e-05
691 2.14533520193072e-05
692 2.105917337758e-05
693 2.21094123844523e-05
694 2.18106506508775e-05
695 2.54724072874524e-05
696 2.6405745302327e-05
697 3.57609933416825e-05
698 2.78570678347023e-05
699 2.8386019039317e-05
700 2.14299288927577e-05
701 2.19105095311534e-05
702 2.09019453905057e-05
703 2.12106024264358e-05
704 2.1078230929561e-05
705 2.10872349271085e-05
706 2.10123307624599e-05
707 2.1022635337431e-05
708 2.10245361813577e-05
709 2.09361223824089e-05
710 2.10300713661127e-05
711 2.08972433028976e-05
712 2.09835234272759e-05
713 2.0951651094947e-05
714 2.10645575862145e-05
715 2.07890661840793e-05
716 2.12290724448394e-05
717 2.07757639145711e-05
718 2.20045658352319e-05
719 2.16168955375906e-05
720 2.5866596843116e-05
721 2.6729952878668e-05
722 3.65874329872895e-05
723 2.66511306108441e-05
724 2.774405402306e-05
725 2.09851223189617e-05
726 2.1513891624636e-05
727 2.0747780581587e-05
728 2.09698191611096e-05
729 2.08072724490194e-05
730 2.0898318325635e-05
731 2.08829842449632e-05
732 2.09457502933219e-05
733 2.07403500098735e-05
734 2.08939745789394e-05
735 2.07029006560333e-05
736 2.08324017876294e-05
737 2.0682322428911e-05
738 2.07698958547553e-05
739 2.06514887395315e-05
740 2.08521632885095e-05
741 2.06163094844669e-05
742 2.07637240237091e-05
743 2.05183732759906e-05
744 2.12480936170323e-05
745 2.08951878448715e-05
746 2.38971697399393e-05
747 2.49896929744864e-05
748 3.64831466868054e-05
749 3.03888427879428e-05
750 3.1317107641371e-05
751 2.10693597182399e-05
752 2.13291095860768e-05
753 2.0549585315166e-05
754 2.06925233214861e-05
755 2.0680974557763e-05
756 2.06516415346414e-05
757 2.06720069400035e-05
758 2.07628363568801e-05
759 2.06445929507026e-05
760 2.06440890906379e-05
761 2.05758933589095e-05
762 2.0606217731256e-05
763 2.05641554202884e-05
764 2.0491595932981e-05
765 2.06151034944924e-05
766 2.03437957679853e-05
767 2.08164565265179e-05
768 2.03215622605057e-05
769 2.09445188374957e-05
770 2.03265863092383e-05
771 2.18529494304676e-05
772 2.13735002034809e-05
773 2.6911759050563e-05
774 2.7758622309193e-05
775 3.75131130567752e-05
776 2.48324868152849e-05
777 2.45984938374022e-05
778 2.02394930965966e-05
779 2.08788660529535e-05
780 2.03123581741238e-05
781 2.05254527827492e-05
782 2.03681393031729e-05
783 2.04755033337278e-05
784 2.04814659809927e-05
785 2.04107091121841e-05
786 2.03849303943571e-05
787 2.04048556042835e-05
788 2.02968149096705e-05
789 2.04033221962163e-05
790 2.02481023734435e-05
791 2.03586751013063e-05
792 2.01561033463804e-05
793 2.046082772722e-05
794 2.01711736735888e-05
795 2.03677391255042e-05
796 2.0060393580934e-05
797 2.05861178983469e-05
798 2.01636630663415e-05
799 2.21767295442987e-05
800 2.28420904022641e-05
801 3.31754235958215e-05
802 3.38314930559136e-05
803 3.71747046301607e-05
804 2.12874674616614e-05
805 2.07098473765654e-05
806 2.00982885871781e-05
807 2.01077509700553e-05
808 2.0212328308844e-05
809 2.00483464141143e-05
810 2.02900555450469e-05
811 2.00976046471624e-05
812 2.01601833396126e-05
813 2.00629710889189e-05
814 2.00855938601308e-05
815 2.00466565729585e-05
816 2.00649501493899e-05
817 1.99716942006489e-05
818 2.00492304429645e-05
819 2.00032318389276e-05
820 2.00579579541227e-05
821 1.98765428649494e-05
822 2.01683451450663e-05
823 1.98804737010505e-05
824 2.07491866603959e-05
825 2.05576980079059e-05
826 2.42739024542971e-05
827 2.60629676631652e-05
828 3.64705265383236e-05
829 2.77290509984596e-05
830 2.71415738097858e-05
831 2.01522998395376e-05
832 2.04333809961099e-05
833 1.97745757759549e-05
834 1.99516580323689e-05
835 1.98503912542947e-05
836 2.00201066036243e-05
837 1.98824691324262e-05
838 1.9826471543638e-05
839 1.98221860046033e-05
840 1.98355173779419e-05
841 1.97692679648753e-05
842 1.98137131519616e-05
843 1.9679855540744e-05
844 1.9999037249363e-05
845 1.96010441868566e-05
846 2.0132283680141e-05
847 1.96552337001776e-05
848 2.0979845430702e-05
849 2.06988152058329e-05
850 2.57444189628586e-05
851 2.71835106104845e-05
852 3.62650607712567e-05
853 2.48316500801593e-05
854 2.40557073993841e-05
855 1.96500150195789e-05
856 2.01226830540691e-05
857 1.95241991605144e-05
858 1.97227091121022e-05
859 1.97085519175744e-05
860 1.97497156477766e-05
861 1.9606599380495e-05
862 1.96919372683624e-05
863 1.95175507542444e-05
864 1.96563323697774e-05
865 1.94957610801794e-05
866 1.96614037122345e-05
867 1.95667198568117e-05
868 1.96949895325815e-05
869 1.94055392057635e-05
870 1.99039859580807e-05
871 1.94307613128331e-05
872 2.08955134439748e-05
873 2.07081993721658e-05
874 2.64819609583355e-05
875 2.79199593933299e-05
876 3.72695722035132e-05
877 2.38127013290068e-05
878 2.29135584959295e-05
879 1.93260857486166e-05
880 1.979545049835e-05
881 1.9429480744293e-05
882 1.95707998500438e-05
883 1.95306110981619e-05
884 1.94897729670629e-05
885 1.94591611943906e-05
886 1.94756303244503e-05
887 1.93862015294144e-05
888 1.94432304851944e-05
889 1.95231259567663e-05
890 1.9369368601474e-05
891 1.94591193576343e-05
892 1.92616225831443e-05
893 1.95469292521011e-05
894 1.91880244528875e-05
895 1.98414236365352e-05
896 1.95366847037803e-05
897 2.17537653952604e-05
898 2.20886049646651e-05
899 3.08003000100143e-05
900 2.95644058496691e-05
901 3.3748299756553e-05
902 2.08092278626282e-05
903 2.05871256184764e-05
904 1.91630588233238e-05
905 1.93972591659985e-05
906 1.92982752196258e-05
907 1.92771549336612e-05
908 1.92880797840189e-05
909 1.93296073121019e-05
910 1.93040450540138e-05
911 1.93698069779202e-05
912 1.91815961443353e-05
913 1.93287269212306e-05
914 1.91293365787715e-05
915 1.92749193956843e-05
916 1.90672326425556e-05
917 1.95445209101308e-05
918 1.9014090867131e-05
919 1.98157722479664e-05
920 1.91714334505377e-05
921 2.14875635720091e-05
922 2.13289968087338e-05
923 2.93142838927452e-05
924 2.79488976957509e-05
925 3.33823809341993e-05
926 2.0720681277453e-05
927 2.07283774216194e-05
928 1.89598376891809e-05
929 1.93636788026197e-05
930 1.92556672118371e-05
931 1.93431424122537e-05
932 1.91560702660354e-05
933 1.92705429071793e-05
934 1.92228199011879e-05
935 1.92828592844307e-05
936 1.92175211850554e-05
937 1.933992098202e-05
938 1.92417737707729e-05
939 1.92996631085407e-05
940 1.90905866475077e-05
941 1.9406803403399e-05
942 1.89641777978977e-05
943 1.96415785467252e-05
944 1.89812544704182e-05
945 2.07190369110322e-05
946 1.93651994777611e-05
947 2.48752312472789e-05
948 2.35280767810764e-05
949 3.52098140865564e-05
950 2.33247774303891e-05
951 2.49300228460925e-05
952 1.89286110980902e-05
953 2.01772400032496e-05
954 1.90753999049775e-05
955 1.95500815607375e-05
956 1.92538554983912e-05
957 1.95876236830372e-05
958 1.93274754565209e-05
959 1.94589574675774e-05
960 1.92930001503555e-05
961 1.94270614883862e-05
962 1.92693532881094e-05
963 1.94665699382313e-05
964 1.9261950001237e-05
965 1.96093369595474e-05
966 1.90877854038263e-05
967 1.9666116713779e-05
968 1.89408201549668e-05
969 2.04409334401134e-05
970 1.88752856047358e-05
971 2.21277259697672e-05
972 1.96307391888695e-05
973 2.65682629105868e-05
974 2.22040162043413e-05
975 3.10611612803768e-05
976 2.06588447326794e-05
977 2.40447971009417e-05
978 1.8770315364236e-05
979 2.05652031581849e-05
980 1.89319562196033e-05
981 1.97970475710463e-05
982 1.91410872503184e-05
983 1.97187255253084e-05
984 1.92219158634543e-05
985 1.95319134945748e-05
986 1.92228544619866e-05
987 1.942817834788e-05
988 1.9180291928933e-05
989 1.955980224011e-05
990 1.91460221685702e-05
991 1.9463133867248e-05
992 1.90384234883823e-05
993 1.94606163859135e-05
994 1.89329257409554e-05
995 1.96752207557438e-05
996 1.87538680620492e-05
997 1.99445421458222e-05
998 1.85127173608635e-05
999 2.11611204576911e-05
1000 1.89004767889855e-05
1001 2.58815161942039e-05
1002 2.18750537896995e-05
1003 3.28358437400311e-05
1004 2.08002747967839e-05
1005 2.45335832005367e-05
1006 1.84616892511258e-05
1007 2.03723611775786e-05
1008 1.88995927601354e-05
1009 1.95209031517152e-05
1010 1.91198814718518e-05
1011 1.94387812371133e-05
1012 1.92234856513096e-05
1013 1.9472807252896e-05
1014 1.91600774996914e-05
1015 1.93528194358805e-05
1016 1.91328435903415e-05
1017 1.93925498024328e-05
1018 1.9131155568175e-05
1019 1.94584208657034e-05
1020 1.9013157725567e-05
1021 1.93997020687675e-05
1022 1.89230340765789e-05
1023 1.95049015019322e-05
1024 1.88220219570212e-05
1025 1.98070647456916e-05
1026 1.85140725079691e-05
1027 2.0322491764091e-05
1028 1.83431093319086e-05
1029 2.1978125005262e-05
1030 1.86324596143095e-05
1031 2.52882709901314e-05
1032 1.99427886400372e-05
1033 2.83750105154468e-05
1034 1.94394579011714e-05
1035 2.43900540226605e-05
1036 1.83346237463411e-05
1037 2.09139670914738e-05
1038 1.84708642336773e-05
1039 1.99064579646802e-05
1040 1.87869154615328e-05
1041 1.9582988898037e-05
1042 1.8873406588682e-05
1043 1.94511121662799e-05
1044 1.90066512004705e-05
1045 1.94018648471683e-05
1046 1.89235088328132e-05
1047 1.9377486751182e-05
1048 1.87898876902182e-05
1049 1.95177581190364e-05
1050 1.87920450116508e-05
1051 1.94453459698707e-05
1052 1.86193246918265e-05
1053 1.96295986825135e-05
1054 1.84830587386386e-05
1055 2.01747225219151e-05
1056 1.81910509127192e-05
1057 2.11862388823647e-05
1058 1.82227631739806e-05
1059 2.3638405764359e-05
1060 1.89462261914741e-05
1061 2.62294470303459e-05
1062 1.92281695490237e-05
1063 2.51695328188362e-05
1064 1.83578122232575e-05
1065 2.19048142753309e-05
1066 1.81502673513023e-05
1067 2.02719602384605e-05
1068 1.84164273377974e-05
1069 1.96281853277469e-05
1070 1.85566386790015e-05
1071 1.94760013982886e-05
1072 1.86545839824248e-05
1073 1.93135583685944e-05
1074 1.85740591405192e-05
1075 1.94034928426845e-05
1076 1.86077886610292e-05
1077 1.92496154340915e-05
1078 1.85046574188164e-05
1079 1.93960222532041e-05
1080 1.84030595846707e-05
1081 1.94710228242911e-05
1082 1.8237522454001e-05
1083 1.97101362573449e-05
1084 1.81677314685658e-05
1085 2.04958032554714e-05
1086 1.80187435034895e-05
1087 2.21603713725926e-05
1088 1.8663396986085e-05
1089 2.52236004598672e-05
1090 1.92924890143331e-05
1091 2.59232238022378e-05
1092 1.85593689820962e-05
1093 2.27656146307709e-05
1094 1.80037732206983e-05
1095 2.04852276510792e-05
1096 1.8076856576954e-05
1097 1.97017816390144e-05
1098 1.83350202860311e-05
1099 1.92381830856903e-05
1100 1.83529045898467e-05
1101 1.91896961041493e-05
1102 1.84987566171912e-05
1103 1.89811580639798e-05
1104 1.84650107257767e-05
1105 1.89821421372471e-05
1106 1.84911095857387e-05
1107 1.88629855983891e-05
1108 1.84210839506704e-05
1109 1.89150541700656e-05
1110 1.84414748218842e-05
1111 1.88507856364595e-05
1112 1.8269232896273e-05
1113 1.91633298527449e-05
1114 1.80478928086814e-05
1115 1.95784868992632e-05
1116 1.78466525539989e-05
1117 2.11070237128297e-05
1118 1.82671064976603e-05
1119 2.48867072514258e-05
1120 2.0054425476701e-05
1121 2.84349189314526e-05
1122 1.92447023437126e-05
1123 2.34705330512952e-05
1124 1.78635145857697e-05
1125 2.01983275474049e-05
1126 1.80113365786383e-05
1127 1.91697527043289e-05
1128 1.82144194695866e-05
1129 1.89300626516342e-05
1130 1.84005202754634e-05
1131 1.8665452444111e-05
1132 1.84367763722548e-05
1133 1.86314155143918e-05
1134 1.85248882189626e-05
1135 1.85164626600454e-05
1136 1.8466540495865e-05
1137 1.85485805559438e-05
1138 1.84242071554763e-05
1139 1.8345837816014e-05
1140 1.84039090527222e-05
1141 1.83511656359769e-05
1142 1.85482713277452e-05
1143 1.81241048267111e-05
1144 1.86964698514203e-05
1145 1.79795115400339e-05
1146 1.92107800103258e-05
1147 1.77086640178459e-05
1148 2.03004001377849e-05
1149 1.81181439984357e-05
1150 2.41139496210963e-05
1151 2.01252787519479e-05
1152 2.9379876650637e-05
1153 1.99203277588822e-05
1154 2.4245577151305e-05
1155 1.77793699549511e-05
1156 1.97733243112452e-05
1157 1.78686423168983e-05
1158 1.86707184184343e-05
1159 1.81595987669425e-05
1160 1.8539458324085e-05
1161 1.81528212124249e-05
1162 1.84327855095034e-05
1163 1.82522089744452e-05
1164 1.84500458999537e-05
1165 1.81851628440199e-05
1166 1.83566571649862e-05
1167 1.82278527063318e-05
1168 1.83277625183109e-05
1169 1.81446484930348e-05
1170 1.83079901034944e-05
1171 1.81298273673747e-05
1172 1.8503438695916e-05
1173 1.78408045030665e-05
1174 1.87004770850763e-05
1175 1.77589608938433e-05
1176 1.92018869711319e-05
1177 1.75750064954627e-05
1178 2.06159893423319e-05
1179 1.81027644430287e-05
1180 2.42536953010131e-05
1181 2.01429302251199e-05
1182 2.80125696008326e-05
1183 1.9307295588078e-05
1184 2.29765973926987e-05
1185 1.76419871422695e-05
1186 1.9614613847807e-05
1187 1.76639023266034e-05
1188 1.86466004379326e-05
1189 1.78785921889357e-05
1190 1.84322452696506e-05
1191 1.78684094862547e-05
1192 1.82768217200646e-05
1193 1.79773523996118e-05
1194 1.82451294676866e-05
1195 1.79070975718787e-05
1196 1.81053983396851e-05
1197 1.79606395249721e-05
1198 1.81805371539667e-05
1199 1.77981328306487e-05
1200 1.82695421244716e-05
1201 1.77303809323348e-05
1202 1.82973726623459e-05
1203 1.75521963683423e-05
1204 1.86614197446033e-05
1205 1.74685410456732e-05
1206 1.946489646798e-05
1207 1.76667417690624e-05
1208 2.25234271056252e-05
1209 1.9815053747152e-05
1210 2.84170018858276e-05
1211 2.08183209906565e-05
1212 2.50025750574423e-05
1213 1.78831960511161e-05
1214 1.96741184481652e-05
1215 1.75082404894056e-05
1216 1.84624586836435e-05
1217 1.75962668436114e-05
1218 1.80444621946663e-05
1219 1.77858401002595e-05
1220 1.79470589500852e-05
1221 1.77346937562106e-05
1222 1.79736198333558e-05
1223 1.77301244548289e-05
1224 1.78699719981523e-05
1225 1.76680405274965e-05
1226 1.79306207428453e-05
1227 1.76693756657187e-05
1228 1.78029968083138e-05
1229 1.76209923665738e-05
1230 1.78755726665258e-05
1231 1.75558525370434e-05
1232 1.78632581082638e-05
1233 1.74873694049893e-05
1234 1.81478189915651e-05
1235 1.72506915987469e-05
1236 1.89396159839816e-05
1237 1.76608318724902e-05
1238 2.26232150453143e-05
1239 2.11182450584602e-05
1240 3.19222308462486e-05
1241 2.21444042836083e-05
1242 2.42752066696994e-05
1243 1.74820052052382e-05
1244 1.85145399882458e-05
1245 1.73485241248272e-05
1246 1.77627262019087e-05
1247 1.75499026227044e-05
1248 1.765073830029e-05
1249 1.75251934706466e-05
1250 1.75456643773941e-05
1251 1.76150169863831e-05
1252 1.75855257111834e-05
1253 1.75007426150842e-05
1254 1.76461344381096e-05
1255 1.74913129740162e-05
1256 1.75243003468495e-05
1257 1.74341548699886e-05
1258 1.75822842720663e-05
1259 1.74026590684662e-05
1260 1.75379682332277e-05
1261 1.72903510247124e-05
1262 1.76579542312538e-05
1263 1.72370437212521e-05
1264 1.77012370841112e-05
1265 1.71924875758123e-05
1266 1.84647578862496e-05
1267 1.73963780980557e-05
1268 2.17471861105878e-05
1269 2.11263195524225e-05
1270 3.26352237607352e-05
1271 2.42833884840365e-05
1272 2.51460241997847e-05
1273 1.74812521436252e-05
1274 1.80733222805429e-05
1275 1.72112268046476e-05
1276 1.74418200913351e-05
1277 1.72619093063986e-05
1278 1.73349162650993e-05
1279 1.73265525518218e-05
1280 1.73643111338606e-05
1281 1.72704603755847e-05
1282 1.73615353560308e-05
1283 1.73263761098497e-05
1284 1.72844429471297e-05
1285 1.72514774021693e-05
1286 1.73299067682819e-05
1287 1.7283362467424e-05
1288 1.71746323758271e-05
1289 1.73554926732322e-05
1290 1.71400752151385e-05
1291 1.73589251062367e-05
1292 1.70058574440191e-05
1293 1.75533677975181e-05
1294 1.69436152646085e-05
1295 1.81276354851434e-05
1296 1.72518903127639e-05
1297 2.12974573514657e-05
1298 2.10815178434132e-05
1299 3.25856381095946e-05
1300 2.48033375100931e-05
1301 2.53524631261826e-05
1302 1.74086799233919e-05
1303 1.79432190634543e-05
1304 1.69979066413362e-05
1305 1.71994797710795e-05
1306 1.71994452102808e-05
1307 1.72180552908685e-05
1308 1.71090177900624e-05
1309 1.71696192410309e-05
1310 1.71895371749997e-05
1311 1.71613173733931e-05
1312 1.71029387274757e-05
1313 1.72223917616066e-05
1314 1.71013434737688e-05
1315 1.71410847542575e-05
1316 1.70304738276172e-05
1317 1.71617411979241e-05
1318 1.70777602761518e-05
1319 1.70694711414399e-05
1320 1.69886916410178e-05
1321 1.71606297953986e-05
1322 1.69611666933633e-05
1323 1.70975290529896e-05
1324 1.69504892255645e-05
1325 1.73799762706039e-05
1326 1.68057849805336e-05
1327 1.85875196621055e-05
1328 1.83135198312812e-05
1329 2.70464479399379e-05
1330 2.84698307950748e-05
1331 3.84554223273881e-05
1332 1.87157274922356e-05
1333 1.80255592567846e-05
1334 1.69440536410548e-05
1335 1.70814637385774e-05
1336 1.71431474882411e-05
1337 1.69959093909711e-05
1338 1.71481260622386e-05
1339 1.70803014043486e-05
1340 1.70240273291711e-05
1341 1.70469138538465e-05
1342 1.70925268321298e-05
1343 1.70483144756872e-05
1344 1.7024698536261e-05
1345 1.70908988366136e-05
1346 1.70229177456349e-05
1347 1.70295734278625e-05
1348 1.69925388036063e-05
1349 1.70257644640515e-05
1350 1.70739494933514e-05
1351 1.68763599504018e-05
1352 1.70154453371651e-05
1353 1.69207633007318e-05
1354 1.7160460629384e-05
1355 1.67220205185004e-05
1356 1.75739423866617e-05
1357 1.67499838426011e-05
1358 1.91024682862917e-05
1359 1.85760036401916e-05
1360 2.81145858025411e-05
1361 2.68217190750875e-05
1362 3.32268391503021e-05
1363 1.83420615940122e-05
1364 1.8140137399314e-05
1365 1.67832695296966e-05
1366 1.69759787240764e-05
1367 1.70058901858283e-05
1368 1.69715276570059e-05
1369 1.69120448845206e-05
1370 1.69335180544294e-05
1371 1.6995751138893e-05
1372 1.69417799043003e-05
1373 1.69149279827252e-05
1374 1.69684535649139e-05
1375 1.69118466146756e-05
1376 1.69099766935688e-05
1377 1.68593960552244e-05
1378 1.69106388057116e-05
1379 1.69369850482326e-05
1380 1.68108726938954e-05
1381 1.69026243383996e-05
1382 1.68353635672247e-05
1383 1.70326238730922e-05
1384 1.66719710250618e-05
1385 1.71989468071843e-05
1386 1.67201669682981e-05
1387 1.8416772945784e-05
1388 1.78678947122535e-05
1389 2.496180786693e-05
1390 2.57172523561167e-05
1391 3.5604985896498e-05
1392 1.98545276361983e-05
1393 1.92174666153733e-05
1394 1.66249974427046e-05
1395 1.70174043887528e-05
1396 1.68523783941055e-05
1397 1.68453116202727e-05
1398 1.68313763424521e-05
1399 1.68935257534031e-05
1400 1.68946044141194e-05
1401 1.67914804478642e-05
1402 1.68107799254358e-05
1403 1.67882499226835e-05
1404 1.68878887052415e-05
1405 1.6809888620628e-05
1406 1.67940706887748e-05
1407 1.67576854437357e-05
1408 1.6733964002924e-05
1409 1.68352889886592e-05
1410 1.67905345733743e-05
1411 1.67244288604707e-05
1412 1.67282050824724e-05
1413 1.66961690410972e-05
1414 1.67241887538694e-05
1415 1.6739602870075e-05
1416 1.68568276421865e-05
1417 1.65560977620771e-05
1418 1.71154315466993e-05
1419 1.66311947396025e-05
1420 1.93146497622365e-05
1421 2.02308383450145e-05
1422 3.5090277378913e-05
1423 3.19316750392318e-05
1424 3.01879754260881e-05
1425 1.71080937434454e-05
1426 1.67897796927718e-05
1427 1.67326543305535e-05
1428 1.65995952556841e-05
1429 1.66760673891986e-05
1430 1.67289872479159e-05
1431 1.67260222951882e-05
1432 1.66687514138175e-05
1433 1.66638201335445e-05
1434 1.66505960805807e-05
1435 1.66586596606066e-05
1436 1.66035224538064e-05
1437 1.67273301485693e-05
1438 1.66592944879085e-05
1439 1.65986239153426e-05
1440 1.6626094293315e-05
1441 1.65855053637642e-05
1442 1.66181707754731e-05
1443 1.65201690833783e-05
1444 1.67343023349531e-05
1445 1.65262135851663e-05
1446 1.66413392435061e-05
1447 1.64526154549094e-05
1448 1.6784015315352e-05
1449 1.64082812261768e-05
1450 1.74787837750046e-05
1451 1.71616120496765e-05
1452 2.20416004594881e-05
1453 2.4670640414115e-05
1454 3.84359627787489e-05
1455 2.33945520449197e-05
1456 2.00581343960948e-05
1457 1.64862831297796e-05
1458 1.66161771630868e-05
1459 1.64351749845082e-05
1460 1.64518478413811e-05
1461 1.64700722962152e-05
1462 1.64442553796107e-05
1463 1.64564171427628e-05
1464 1.64208759088069e-05
1465 1.6545887774555e-05
1466 1.64599769050255e-05
1467 1.64223820320331e-05
1468 1.64480989042204e-05
1469 1.64005105034448e-05
1470 1.64171615324449e-05
1471 1.63603363034781e-05
1472 1.64319990290096e-05
1473 1.63513468578458e-05
1474 1.64429802680388e-05
1475 1.64119246619521e-05
1476 1.66351965162903e-05
1477 1.63316744874464e-05
1478 1.73990538314683e-05
1479 1.75147943082266e-05
1480 2.34815270232502e-05
1481 2.74632202490466e-05
1482 3.89857632399071e-05
1483 2.07288685487583e-05
1484 1.79287999344524e-05
1485 1.62695705512306e-05
1486 1.63690529006999e-05
1487 1.63499553309521e-05
1488 1.63225558935665e-05
1489 1.63337026606314e-05
1490 1.6330333892256e-05
1491 1.63272015925031e-05
1492 1.63100630743429e-05
1493 1.63170461746631e-05
1494 1.632580824662e-05
1495 1.63888598763151e-05
1496 1.63392051035771e-05
1497 1.6308693375322e-05
1498 1.63149561558384e-05
1499 1.62706619448727e-05
1500 1.6350108126062e-05
1501 1.62524593179114e-05
1502 1.63954064191785e-05
1503 1.62249507411616e-05
1504 1.66784611792536e-05
1505 1.64219945872901e-05
1506 1.80968345375732e-05
1507 1.90794162335806e-05
1508 2.8508704417618e-05
1509 3.02947755699279e-05
1510 3.26579101965763e-05
1511 1.80393417394953e-05
1512 1.66783720487729e-05
1513 1.61986881721532e-05
1514 1.62189116963418e-05
1515 1.62452670338098e-05
1516 1.62313317559892e-05
1517 1.62318738148315e-05
1518 1.62187625392107e-05
1519 1.62225114763714e-05
1520 1.62182386702625e-05
1521 1.62066044140374e-05
1522 1.61989955813624e-05
1523 1.61984680744354e-05
1524 1.61864536494249e-05
1525 1.61770003614947e-05
1526 1.61863790708594e-05
1527 1.61805783136515e-05
1528 1.61773496074602e-05
1529 1.6183916159207e-05
1530 1.61930802278221e-05
1531 1.61678854055936e-05
1532 1.61707448569359e-05
1533 1.61682146426756e-05
1534 1.6182681065402e-05
1535 1.61394127644598e-05
1536 1.61896277859341e-05
1537 1.61101179401157e-05
1538 1.64866105478723e-05
1539 1.62643718795152e-05
1540 1.81935993168736e-05
1541 2.10468879231485e-05
1542 3.98375232180115e-05
1543 3.97485891880933e-05
1544 2.72844481514767e-05
1545 1.64783032232663e-05
1546 1.60443214554107e-05
1547 1.61388143169461e-05
1548 1.60342369781574e-05
1549 1.60784366016742e-05
1550 1.60427098307991e-05
1551 1.60503313964e-05
1552 1.60670060722623e-05
1553 1.60649451572681e-05
1554 1.60562012752052e-05
1555 1.60442341439193e-05
1556 1.60515901370673e-05
1557 1.60397758008912e-05
1558 1.60444797074888e-05
1559 1.60130512085743e-05
1560 1.6043475625338e-05
1561 1.59972587425727e-05
1562 1.60391318786424e-05
1563 1.59742685355013e-05
1564 1.60698054969544e-05
1565 1.59670344146434e-05
1566 1.61532461788738e-05
1567 1.6010728359106e-05
1568 1.66648478625575e-05
1569 1.6789843357401e-05
1570 2.02899209398311e-05
1571 2.42435744439717e-05
1572 3.68997461919207e-05
1573 2.57703304669121e-05
1574 2.00416507141199e-05
1575 1.64290086104302e-05
1576 1.60657345986692e-05
1577 1.59217106556753e-05
1578 1.59454557433492e-05
1579 1.59269275172846e-05
1580 1.593121123733e-05
1581 1.59306437126361e-05
1582 1.5928844732116e-05
1583 1.59203354996862e-05
1584 1.59100745804608e-05
1585 1.59237133630086e-05
1586 1.58978982653935e-05
1587 1.58983439177973e-05
1588 1.58888087753439e-05
1589 1.58990405907389e-05
1590 1.58871789608384e-05
1591 1.59000010171439e-05
1592 1.58858274517115e-05
1593 1.59116407303372e-05
1594 1.58970087795751e-05
1595 1.59030951181194e-05
1596 1.58918846864253e-05
1597 1.59330011229031e-05
1598 1.58820712385932e-05
1599 1.60145573318005e-05
1600 1.59898427227745e-05
1601 1.70514213095885e-05
1602 1.9156595953973e-05
1603 3.2581763662165e-05
1604 4.45776822743937e-05
1605 3.6077562981518e-05
1606 1.64447355928132e-05
1607 1.59059381985571e-05
1608 1.59449809871148e-05
1609 1.58486218424514e-05
1610 1.58562907017767e-05
1611 1.5856296158745e-05
1612 1.58445127453888e-05
1613 1.58592574734939e-05
1614 1.58462735271314e-05
1615 1.58512611960759e-05
1616 1.58416914928239e-05
1617 1.58465700224042e-05
1618 1.58471084432676e-05
1619 1.58392249431927e-05
1620 1.58404509420507e-05
1621 1.58346265379805e-05
1622 1.58407056005672e-05
1623 1.5831892596907e-05
1624 1.58388211275451e-05
1625 1.58236016432056e-05
1626 1.58488946908619e-05
1627 1.58168277266668e-05
1628 1.58670209202683e-05
1629 1.58240527525777e-05
1630 1.60030194820138e-05
1631 1.59374558279524e-05
1632 1.68731967278291e-05
1633 1.79998842213536e-05
1634 2.49389486270957e-05
1635 3.1729960028315e-05
1636 3.76147618226241e-05
1637 1.89791189768584e-05
1638 1.62809192261193e-05
1639 1.58123402798083e-05
1640 1.58005714183673e-05
1641 1.58482853294117e-05
1642 1.58129005285446e-05
1643 1.5841773347347e-05
1644 1.58207840286195e-05
1645 1.58267885126406e-05
1646 1.58184320753207e-05
1647 1.58263701450778e-05
1648 1.58244802150875e-05
1649 1.58142556756502e-05
1650 1.58082857524278e-05
1651 1.58230359375011e-05
1652 1.58074217324611e-05
1653 1.58224793267436e-05
1654 1.58002021635184e-05
1655 1.58398324856535e-05
1656 1.57903778017499e-05
1657 1.58890543389134e-05
1658 1.57939957716735e-05
1659 1.62124724738533e-05
1660 1.63421245815698e-05
1661 1.9126639017486e-05
1662 2.35751977015752e-05
1663 3.83369551855139e-05
1664 2.99045495921746e-05
1665 2.11332517210394e-05
1666 1.62657615874195e-05
1667 1.57801750901854e-05
1668 1.57573394972133e-05
1669 1.57414815475931e-05
1670 1.57687318278477e-05
1671 1.57574831973761e-05
1672 1.57682388817193e-05
1673 1.57517733896384e-05
1674 1.57694266817998e-05
1675 1.57544509420404e-05
1676 1.57672602654202e-05
1677 1.57494159793714e-05
1678 1.57570357259829e-05
1679 1.57618778757751e-05
1680 1.57563317770837e-05
1681 1.57614394993288e-05
1682 1.5731469829916e-05
1683 1.57657123054378e-05
1684 1.57354261318687e-05
1685 1.5782341506565e-05
1686 1.57365666382248e-05
1687 1.5910287402221e-05
1688 1.59047885972541e-05
1689 1.69644154084381e-05
1690 1.8837490642909e-05
1691 2.85262813122245e-05
1692 3.67642533092294e-05
1693 3.5672197554959e-05
1694 1.72804757312406e-05
1695 1.57221038534772e-05
1696 1.57639224198647e-05
1697 1.56997011799831e-05
1698 1.57358263095375e-05
1699 1.57121339725563e-05
1700 1.57254307850963e-05
1701 1.57185877469601e-05
1702 1.57239501277218e-05
1703 1.57068898261059e-05
1704 1.57109370775288e-05
1705 1.57184149429668e-05
1706 1.57058857439552e-05
1707 1.57197955559241e-05
1708 1.57066278916318e-05
1709 1.56922778842272e-05
1710 1.57062895596027e-05
1711 1.56956248247297e-05
1712 1.56980022438802e-05
1713 1.5683352103224e-05
1714 1.5724888726254e-05
1715 1.56800597324036e-05
1716 1.57717913680244e-05
1717 1.57326048793038e-05
1718 1.62294127221685e-05
1719 1.6652562408126e-05
1720 2.06235436053248e-05
1721 2.723164834606e-05
1722 4.26923033955973e-05
1723 2.41739289776888e-05
1724 1.75881104951259e-05
1725 1.58273178385571e-05
1726 1.56769801833434e-05
1727 1.56904916366329e-05
1728 1.56884780153632e-05
1729 1.569974119775e-05
1730 1.56988480739528e-05
1731 1.56973710545572e-05
1732 1.56972619151929e-05
1733 1.56957285071258e-05
1734 1.56843470904278e-05
1735 1.57042923092376e-05
1736 1.56920395966154e-05
1737 1.57072936417535e-05
1738 1.56842816068092e-05
1739 1.57017366291257e-05
1740 1.56978730956325e-05
1741 1.57807353389217e-05
1742 1.57917911565164e-05
1743 1.58310467668343e-05
1744 1.58145339810289e-05
1745 1.58643306349404e-05
1746 1.58178117999341e-05
1747 1.60273393703392e-05
1748 1.6093630620162e-05
1749 1.78090813278686e-05
1750 2.15692598430905e-05
1751 3.77077521989122e-05
1752 3.77283904526848e-05
1753 2.54102724284166e-05
1754 1.64930133905727e-05
1755 1.57685008161934e-05
1756 1.57849717652425e-05
1757 1.57586091518169e-05
1758 1.57589511218248e-05
1759 1.57571557792835e-05
1760 1.57545837282669e-05
1761 1.57524245878449e-05
1762 1.57490303536179e-05
1763 1.57522190420423e-05
1764 1.57398462761194e-05
1765 1.57417434820672e-05
1766 1.57351441885112e-05
1767 1.57389913510997e-05
1768 1.57370559463743e-05
1769 1.5739768059575e-05
1770 1.57212853082456e-05
1771 1.57246522576315e-05
1772 1.57318991114153e-05
1773 1.57204594870564e-05
1774 1.57165086420719e-05
1775 1.57352933456423e-05
1776 1.57273971126415e-05
1777 1.57854992721695e-05
1778 1.57825943460921e-05
1779 1.60122308443533e-05
1780 1.64501780091086e-05
1781 1.93461764865788e-05
1782 2.66921269940212e-05
1783 4.59508046333212e-05
1784 2.86611721094232e-05
1785 1.79519356606761e-05
1786 1.59327100845985e-05
1787 1.57036483869888e-05
1788 1.56914193212288e-05
1789 1.56945225171512e-05
1790 1.56883743329672e-05
1791 1.56878668349236e-05
1792 1.56829173647566e-05
1793 1.56914029503241e-05
1794 1.5686417100369e-05
1795 1.56763071572641e-05
1796 1.56653059093514e-05
1797 1.56707810674561e-05
1798 1.56690693984274e-05
1799 1.5677176634199e-05
1800 1.56747191795148e-05
1801 1.56760270328959e-05
1802 1.56830956257181e-05
1803 1.5672885638196e-05
1804 1.5677383998991e-05
1805 1.56715686898679e-05
1806 1.56758069351781e-05
1807 1.56852238433203e-05
1808 1.56981750478735e-05
1809 1.57155918714125e-05
1810 1.59630180860404e-05
1811 1.67384077940369e-05
1812 2.06669938052073e-05
1813 3.12692973238882e-05
1814 4.93958868901245e-05
1815 2.22545786527917e-05
1816 1.60857798618963e-05
1817 1.56668666022597e-05
1818 1.56659443746321e-05
1819 1.56523128680419e-05
1820 1.56437108671525e-05
1821 1.56478054122999e-05
1822 1.56379464897327e-05
1823 1.56396436068462e-05
1824 1.56367077579489e-05
1825 1.56343558046501e-05
1826 1.56313290062826e-05
1827 1.56286860146793e-05
1828 1.56296791828936e-05
1829 1.56257810886018e-05
1830 1.56192454596749e-05
1831 1.56300811795518e-05
1832 1.5633660950698e-05
1833 1.56222486111801e-05
1834 1.56342975969892e-05
1835 1.56485602929024e-05
1836 1.5711253581685e-05
1837 1.579242052685e-05
1838 1.63013992278138e-05
1839 1.75590521394042e-05
1840 2.2521177015733e-05
1841 3.14648132189177e-05
1842 3.93720256397501e-05
1843 2.0626537661883e-05
1844 1.62895703397226e-05
1845 1.56771093315911e-05
1846 1.56397691171151e-05
1847 1.5636158423149e-05
1848 1.56317473738454e-05
1849 1.56366568262456e-05
1850 1.56233036250342e-05
1851 1.56205587700242e-05
1852 1.56078549480299e-05
1853 1.56140904437052e-05
1854 1.56050537043484e-05
1855 1.56043788592797e-05
1856 1.5614103176631e-05
1857 1.560977580084e-05
1858 1.56148071255302e-05
1859 1.56120077008381e-05
1860 1.56144651555223e-05
1861 1.56180449266685e-05
1862 1.56384321599035e-05
1863 1.56842288561165e-05
1864 1.58636939886492e-05
1865 1.62656160682673e-05
1866 1.80813985934947e-05
1867 2.30151363211917e-05
1868 3.60451558663044e-05
1869 3.37176425091457e-05
1870 2.21802274609217e-05
1871 1.66460540640401e-05
1872 1.55823890963802e-05
1873 1.55589568748837e-05
1874 1.55827729031444e-05
1875 1.55621510202764e-05
1876 1.5564524801448e-05
1877 1.55604811880039e-05
1878 1.55603174789576e-05
1879 1.55532598000718e-05
1880 1.55514899233822e-05
1881 1.55462221300695e-05
1882 1.55388479470275e-05
1883 1.55482430272968e-05
1884 1.55458237713901e-05
1885 1.55368034029379e-05
1886 1.55427187564783e-05
1887 1.55235957208788e-05
1888 1.55294346768642e-05
1889 1.55221987370169e-05
1890 1.5537840226898e-05
1891 1.55365560203791e-05
1892 1.55495363287628e-05
1893 1.55693742271978e-05
1894 1.5625691958121e-05
1895 1.57329704961739e-05
1896 1.62820815603482e-05
1897 1.84825003088918e-05
1898 2.68824278464308e-05
1899 4.84974880237132e-05
1900 3.1538318580715e-05
1901 1.75142522493843e-05
1902 1.5856503523537e-05
1903 1.55890429596184e-05
1904 1.55011530296179e-05
1905 1.5545552741969e-05
1906 1.55059915414313e-05
1907 1.55159596033627e-05
1908 1.55077814270044e-05
1909 1.55008729052497e-05
1910 1.55019752128283e-05
1911 1.54852095874958e-05
1912 1.54869958350901e-05
1913 1.54812059918186e-05
1914 1.54737354023382e-05
1915 1.54710578499362e-05
1916 1.5490542864427e-05
1917 1.54631470650202e-05
1918 1.54792760440614e-05
1919 1.54628960444825e-05
1920 1.55047891894355e-05
1921 1.54641584231285e-05
1922 1.55864145199303e-05
1923 1.56017322296975e-05
1924 1.64348621183308e-05
1925 1.78939553734381e-05
1926 2.52987047133502e-05
1927 3.8660284189973e-05
1928 3.39398211508524e-05
1929 1.93307005247334e-05
1930 1.65169258252718e-05
1931 1.55120815179544e-05
1932 1.54750341607723e-05
1933 1.55268189701019e-05
1934 1.54591998580145e-05
1935 1.54894169099862e-05
1936 1.54630961333169e-05
1937 1.54654844664037e-05
1938 1.54501449287636e-05
1939 1.54490498971427e-05
1940 1.54535446199588e-05
1941 1.54342942551011e-05
1942 1.54513036250137e-05
1943 1.5424593584612e-05
1944 1.5456880646525e-05
1945 1.5420397176058e-05
1946 1.54789777297992e-05
1947 1.53947075887118e-05
1948 1.55823163368041e-05
1949 1.54407025547698e-05
1950 1.62355827342253e-05
1951 1.66975078172982e-05
1952 2.21945028897608e-05
1953 3.14703102048952e-05
1954 3.98609518015292e-05
1955 2.42057176365051e-05
1956 1.75810564542189e-05
1957 1.54829958773917e-05
1958 1.54620665853145e-05
1959 1.55642173922388e-05
1960 1.54529079736676e-05
1961 1.54967783601023e-05
1962 1.54668960021809e-05
1963 1.54797162394971e-05
1964 1.54670015035663e-05
1965 1.54607823787956e-05
1966 1.54536755871959e-05
1967 1.54440895130392e-05
1968 1.54371937242104e-05
1969 1.5431891370099e-05
1970 1.54357003339101e-05
1971 1.54171102622058e-05
1972 1.54463159560692e-05
1973 1.54028239194304e-05
1974 1.54605440911837e-05
1975 1.53687706188066e-05
1976 1.55740908667212e-05
1977 1.5369620086858e-05
1978 1.6224037608481e-05
1979 1.64977373060538e-05
1980 2.2968002667767e-05
1981 3.41271588695236e-05
1982 4.41864503955003e-05
1983 2.25734402192757e-05
1984 1.72150812431937e-05
1985 1.56961887114448e-05
1986 1.54597728396766e-05
1987 1.56789756147191e-05
1988 1.55092875502305e-05
1989 1.55572688527172e-05
1990 1.55291872943053e-05
1991 1.55383113451535e-05
1992 1.55208090291126e-05
1993 1.55058078235015e-05
1994 1.5511268429691e-05
1995 1.54930148710264e-05
1996 1.54939261847176e-05
1997 1.54682402353501e-05
1998 1.54775589180645e-05
1999 1.5444065866177e-05
};
\addlegendentry{Test}

\nextgroupplot[
legend cell align={left},
legend style={fill opacity=0.8, draw opacity=1, text opacity=1, draw=white!80!black},
log basis y={10},
tick align=outside,
tick pos=left,
title={Model4 },
x grid style={white!69.0196078431373!black},
xlabel={Epoch},
xmin=-99.95, xmax=2098.95,
xtick style={color=black},
y grid style={white!69.0196078431373!black},
ylabel={MSE Loss},
ymin=0.000567045165310798, ymax=0.01,
ymode=log,
ytick style={color=black}
]
\addplot [semithick, black, dashed]
table {%
0 0.121380121447146
1 0.119226661510766
2 0.117086208425462
3 0.114960365230218
4 0.112838139291853
5 0.110684741754085
6 0.108491151360795
7 0.106234128819779
8 0.103888872079551
9 0.101391297066584
10 0.0985529855825007
11 0.095050988253206
12 0.0908367088995874
13 0.0865919981151819
14 0.0827232567826286
15 0.0791347286431119
16 0.0758035803446546
17 0.0727247466566041
18 0.0698695507599041
19 0.067200995516032
20 0.0646776631474495
21 0.0622945396462455
22 0.0600358163937926
23 0.0578907739836723
24 0.0558482189662755
25 0.0538967980537564
26 0.0520274562295526
27 0.0502331525785848
28 0.0485088402638212
29 0.0468493531225249
30 0.0452499521197751
31 0.043708793935366
32 0.0422217903542332
33 0.0407846077578142
34 0.039390996156726
35 0.0380345472949557
36 0.0367232903372496
37 0.0354611618677154
38 0.0342471333569847
39 0.0330793029861525
40 0.0319562916993164
41 0.0308769115945324
42 0.029840161849279
43 0.0288450825610198
44 0.0278907065512612
45 0.026976051798556
46 0.026100086630322
47 0.025261735136155
48 0.0244598609860986
49 0.0236932896368671
50 0.0229608064692002
51 0.022261158679612
52 0.0215930731792469
53 0.0209552620071918
54 0.0203464429068845
55 0.0197653295763303
56 0.0192106705217157
57 0.0186812241445296
58 0.0181757970131002
59 0.0176932263129856
60 0.0172323997248895
61 0.0167922313266899
62 0.0163717254472431
63 0.0159698794886936
64 0.015585776855005
65 0.0152185482729692
66 0.0148673606454395
67 0.0145314250548836
68 0.0142099969525589
69 0.013902317834436
70 0.013606863343739
71 0.0133216935500968
72 0.0130461832159199
73 0.0127833749284036
74 0.0125328581052599
75 0.0122928958735429
76 0.012062816938851
77 0.011842120802612
78 0.0116303357644938
79 0.0114270155463601
80 0.0112317370076198
81 0.0110441014840035
82 0.0108637273224303
83 0.0106902431434719
84 0.0105233025678899
85 0.0103625729389023
86 0.0102077360788826
87 0.0100584836181952
88 0.00991450894798618
89 0.00977550128300209
90 0.00964048365131021
91 0.00950637776986696
92 0.00937253837764729
93 0.00924567932088394
94 0.00912530535424594
95 0.00900889679905958
96 0.00889598652429413
97 0.00878636473498773
98 0.00868000467016827
99 0.00857665037619881
100 0.00847606158640701
101 0.00837804448383395
102 0.00828253606596263
103 0.00818926242209272
104 0.00809829179343069
105 0.00800920767505886
106 0.00792210168583551
107 0.00783681313623674
108 0.00775319887179649
109 0.00767126330174506
110 0.00759080013085622
111 0.0075118267632206
112 0.00743408446578542
113 0.00735775240900693
114 0.00728255239664577
115 0.00720847710908856
116 0.00713546350016259
117 0.00706342903868062
118 0.00699239641107852
119 0.00692219371558167
120 0.00685283520579105
121 0.0067842859934899
122 0.00671647450508317
123 0.00664935723034432
124 0.00658289388229605
125 0.00651704570191214
126 0.00645177416299703
127 0.00638704397715628
128 0.0063228235594579
129 0.0062590812885901
130 0.00619578778423602
131 0.0061329175150604
132 0.00607044119533384
133 0.00600832415511832
134 0.00594658774207346
135 0.00588515945855761
136 0.0058240305952495
137 0.00576323637505993
138 0.00570267670264002
139 0.00564238939114148
140 0.00558234407799318
141 0.00552254315698519
142 0.00546295262029162
143 0.00540356679994147
144 0.0053444455479621
145 0.00528545604174724
146 0.00522664795425953
147 0.00516811187844723
148 0.0051096456227242
149 0.00505144796625245
150 0.00499336754000979
151 0.00493544584605843
152 0.00487773298664251
153 0.00482018524780869
154 0.00476282990712207
155 0.00470562605914893
156 0.00464863261004211
157 0.00459183103521354
158 0.00453521676536184
159 0.00447880104911746
160 0.00442260018462548
161 0.00436661833373364
162 0.00431086312164553
163 0.00425534299574792
164 0.00420006895365077
165 0.00414506812012405
166 0.00409032630705042
167 0.0040359024060308
168 0.00398179583135061
169 0.00392808056858485
170 0.00387470296482206
171 0.0038216774664761
172 0.00376900618357467
173 0.00371669800733798
174 0.00366478616342647
175 0.00361324988989509
176 0.00356216251748265
177 0.00351147595938528
178 0.00346123021517997
179 0.00341143280820688
180 0.00336209457236691
181 0.00331321439807652
182 0.00326480176954647
183 0.0032168627003557
184 0.00316940086850082
185 0.0031224082231347
186 0.00307581833840231
187 0.00302968312098528
188 0.00298405678040581
189 0.00293891062028706
190 0.00289424053698895
191 0.00285005095065571
192 0.00280632899011835
193 0.00276332501744037
194 0.00272098029381596
195 0.00267924147192389
196 0.00263798742525978
197 0.00259724192073918
198 0.00255694790939742
199 0.00251714845217066
200 0.00247779812889348
201 0.00243893023070996
202 0.00240053129346052
203 0.00236262321050162
204 0.00232517888071015
205 0.00228821511336719
206 0.00225174604383938
207 0.00221576506010024
208 0.00218028460585629
209 0.00214531048186473
210 0.0021108514356456
211 0.00207691875220917
212 0.00204351118736668
213 0.00201066157387686
214 0.00197835833023419
215 0.00194662709327531
216 0.0019154716283083
217 0.00188490024629573
218 0.00185493374920043
219 0.00182556913568988
220 0.00179682078396581
221 0.00176870179529942
222 0.00174121126929094
223 0.00171436330310826
224 0.00168815672896017
225 0.00166264619929279
226 0.00163772097585024
227 0.0016134752859216
228 0.00158989454939729
229 0.00156696084377472
230 0.00154468452819856
231 0.00152306339714414
232 0.00150207892238541
233 0.00148165141490608
234 0.0014617644401369
235 0.00144251470464951
236 0.00142394488602804
237 0.00140598703228534
238 0.00138861602772522
239 0.001371697536797
240 0.00135524134520892
241 0.00133926314128985
242 0.00132376545388979
243 0.0013088319851704
244 0.00129447859580978
245 0.00128079956084548
246 0.00126769050712028
247 0.00125504952029587
248 0.0012427895617293
249 0.00123094269292778
250 0.00121946040803778
251 0.00120842656178866
252 0.00119768768854556
253 0.00118731816883155
254 0.00117729501039321
255 0.00116761968536139
256 0.00115828394461914
257 0.00114916610368709
258 0.00114030529562115
259 0.00113174619298206
260 0.00112345455306695
261 0.00111543839614114
262 0.00110771573997681
263 0.00110027435363236
264 0.00109313332615102
265 0.00108627371071179
266 0.0010796745081052
267 0.00107333494190698
268 0.00106723814917586
269 0.00106136933879952
270 0.00105569756328805
271 0.00105019292448105
272 0.00104489589301693
273 0.00103976153982899
274 0.00103484935121401
275 0.00103014514243682
276 0.0010256925905594
277 0.00102151372723824
278 0.00101766164499395
279 0.00101417482886745
280 0.00101106800019579
281 0.00100830758742632
282 0.00100586280291282
283 0.00100371936966326
284 0.00100185105526407
285 0.00100024503061036
286 0.000998814616508525
287 0.000997546412037309
288 0.00099636760745625
289 0.000995303129940339
290 0.000994289789844061
291 0.000993368436951414
292 0.000992511229043203
293 0.000991700476816959
294 0.000990932231246688
295 0.000990181581926208
296 0.000989459006490279
297 0.000988755294770272
298 0.000988063827861652
299 0.00098739924800384
300 0.000986728699217565
301 0.000986082184482484
302 0.000985440372346602
303 0.000984818494515594
304 0.000984184320742543
305 0.000983563471407933
306 0.000982940896648188
307 0.000982338732399057
308 0.000981716448364978
309 0.00098110956858477
310 0.000980497537739211
311 0.000979889538655243
312 0.000979282584097518
313 0.000978672928454216
314 0.000978082995970908
315 0.000977468971825601
316 0.000976870610571723
317 0.000976265313482827
318 0.000975663153525375
319 0.000975063787223007
320 0.000974462282584909
321 0.000973862426150163
322 0.000973262644009765
323 0.000972660092855904
324 0.000972076939433464
325 0.0009714706459647
326 0.000970878176957513
327 0.000970285200992294
328 0.000969693157429674
329 0.000969103771012669
330 0.000968510587171068
331 0.000967932911578373
332 0.000967336468704616
333 0.000966755610249947
334 0.000966167584863342
335 0.000965604237279649
336 0.000965017227855469
337 0.000964443629328571
338 0.000963872910347163
339 0.000963320334449236
340 0.000962752163957248
341 0.000962197287265099
342 0.000961637682252103
343 0.000960967562150472
344 0.000960450809117219
345 0.000959885726047105
346 0.000959374084061437
347 0.000958832715667768
348 0.000958292131969074
349 0.000957816039914405
350 0.000957276521546646
351 0.0009568061832681
352 0.000956303617641652
353 0.000955833270751327
354 0.000955347694713282
355 0.000954882798282597
356 0.000954436813628945
357 0.000953978173754422
358 0.000953517013584815
359 0.0009530787192773
360 0.000952637733917072
361 0.000952209156764638
362 0.000951770628688564
363 0.000951384026791402
364 0.000950952077488409
365 0.000950561420410168
366 0.000950156439188277
367 0.000949789187643546
368 0.000949409668805856
369 0.000949029115190569
370 0.000948703834495745
371 0.00094831645475324
372 0.000947978049453013
373 0.000947616904426241
374 0.000947289965438358
375 0.00094695289354263
376 0.000946619256666281
377 0.00094630087298242
378 0.000945972423210151
379 0.000945663374182004
380 0.00094534923701417
381 0.00094504395030981
382 0.000944746287643738
383 0.000944449461627528
384 0.000944155137659664
385 0.000943857933606296
386 0.000943577773739435
387 0.000943279882250181
388 0.000943018393542161
389 0.000942715449099296
390 0.000942443004731786
391 0.000942141464037149
392 0.000941888990439566
393 0.000941623076130327
394 0.000941351093132425
395 0.000941108906971522
396 0.000940847746903728
397 0.000940600817557424
398 0.000940340019724317
399 0.000940112166290419
400 0.000939853472544883
401 0.000939618165659795
402 0.000939377929313423
403 0.000939140122653725
404 0.000938896415618728
405 0.000938677036970148
406 0.000938432850659865
407 0.000938209522018951
408 0.000937972004720677
409 0.000937760471373394
410 0.000937531013192938
411 0.000937310194643715
412 0.000937091501128862
413 0.000936882922076165
414 0.000936647799306911
415 0.000936444203489373
416 0.000936221906613355
417 0.000936016612598678
418 0.000935801535206338
419 0.00093560276388871
420 0.000935386109631509
421 0.000935183978839405
422 0.000934966767943024
423 0.000934769933138568
424 0.00093456316898255
425 0.000934363931406779
426 0.000934156031235034
427 0.000933969901581122
428 0.000933756300042887
429 0.000933569860990247
430 0.000933371466857125
431 0.00093318013691146
432 0.000932977736880503
433 0.000932787281840319
434 0.000932600444969012
435 0.000932409245905319
436 0.000932210352175389
437 0.000932035522907881
438 0.000931833294913531
439 0.00093164654012412
440 0.00093145133513417
441 0.000931267255396051
442 0.000931081360391772
443 0.000930897196127489
444 0.000930711312548738
445 0.000930521886829183
446 0.00093034322242147
447 0.000930154085381218
448 0.000929975963884999
449 0.000929797537310151
450 0.000929608098573453
451 0.000929431300534134
452 0.00092924429779373
453 0.000929073897310673
454 0.000928897246240012
455 0.000928708206799911
456 0.000928533718195013
457 0.000928349844173226
458 0.000928182385450782
459 0.000927990127053135
460 0.000927827672910553
461 0.000927640413920017
462 0.000927464986858695
463 0.00092728691930688
464 0.000927108949611011
465 0.000926928430601492
466 0.00092675661230146
467 0.000926586791337058
468 0.0009264001567999
469 0.000926232199503829
470 0.000926046683218829
471 0.000925871792844646
472 0.00092569690158939
473 0.000925512498014314
474 0.000925348518734381
475 0.000925168534394061
476 0.000924983872749863
477 0.000924816410446283
478 0.000924633961489008
479 0.000924447601619249
480 0.000924285761669807
481 0.0009241010882306
482 0.000923923510327995
483 0.000923741569465619
484 0.000923563333032007
485 0.000923381593963768
486 0.000923208856335123
487 0.000923025023325863
488 0.000922844679536183
489 0.000922661708727901
490 0.000922484965229842
491 0.00092229557481005
492 0.000922125427393894
493 0.000921943124126301
494 0.000921752871164472
495 0.000921573282539612
496 0.000921382458869857
497 0.000921201268425875
498 0.000921014583212809
499 0.000920831663762556
500 0.000920640626787872
501 0.000920455569797696
502 0.000920268766464005
503 0.000920081708812859
504 0.000919894003942545
505 0.000919693300687641
506 0.000919519778619815
507 0.000919324046975589
508 0.000919121094113962
509 0.000918939524154894
510 0.00091874339051401
511 0.000918546829410616
512 0.000918354505699881
513 0.000918141416605067
514 0.000917979303636685
515 0.00091774667615141
516 0.000917595084814593
517 0.000917413504822662
518 0.000917135285675386
519 0.000916998812272141
520 0.000916703278647901
521 0.00091641393285613
522 0.000916093718217326
523 0.000915821940139949
524 0.000915543306007294
525 0.000915273144812545
526 0.000915037408049102
527 0.000914770852176616
528 0.000914495803300497
529 0.000914239817518592
530 0.00091397297308049
531 0.000913723355040474
532 0.000913451390374576
533 0.000913174115822812
534 0.000912922095665181
535 0.000912660608236138
536 0.000912386429092749
537 0.000912125453936596
538 0.000911854431109305
539 0.000911581754991175
540 0.000911318993701116
541 0.000911036238903762
542 0.000910770143406125
543 0.0009104936858364
544 0.000910209611674873
545 0.000909941990727248
546 0.000909653102183938
547 0.000909360733203357
548 0.000909035037494732
549 0.000908814600109054
550 0.000908377272878624
551 0.000908037842663134
552 0.000907727379512835
553 0.000907399292970013
554 0.000907095724727469
555 0.000906778463985347
556 0.000906455731808364
557 0.00090615146308437
558 0.000905826267285192
559 0.000905493922203959
560 0.000905180132974692
561 0.000904848378240786
562 0.000904513921881289
563 0.000904188856651444
564 0.000903844971247736
565 0.000903496126966274
566 0.000903158183376718
567 0.000902808644468678
568 0.000902453903961487
569 0.000902104390007707
570 0.000901740552961883
571 0.00090136991590839
572 0.000901019634454769
573 0.000900635710053166
574 0.000900273511973637
575 0.000899891518486129
576 0.000899505572846238
577 0.000899133084772075
578 0.000898740697465428
579 0.000898345656963784
580 0.000897955518013305
581 0.000897554263019629
582 0.000897151865899559
583 0.000896750288092107
584 0.000896337745871278
585 0.000895916362679827
586 0.000895504526425839
587 0.000895082476944253
588 0.000894662814005187
589 0.000894225845030405
590 0.000893800330459271
591 0.000893357544896389
592 0.000892926491786739
593 0.000892481475062823
594 0.000892036601157997
595 0.000891586360751262
596 0.000891127616910126
597 0.000890672363368594
598 0.000890206599535759
599 0.00088973930485281
600 0.000889276980728937
601 0.000888793304909541
602 0.000888321255416713
603 0.0008878274771007
604 0.000887349246681879
605 0.000886851256581167
606 0.000886345061502425
607 0.000885814630095183
608 0.000885287891833286
609 0.000884801823190173
610 0.000884298373989623
611 0.000883777488553505
612 0.000883248235027168
613 0.000882740090929701
614 0.000882222499001273
615 0.000881689385863638
616 0.000881150385140472
617 0.000880626265058027
618 0.000880089824477182
619 0.000879546832578626
620 0.000879004507794434
621 0.000878439871172532
622 0.000877897516943449
623 0.000877352852398872
624 0.000876791630730622
625 0.000876233146044569
626 0.000875674593174836
627 0.000875093797645832
628 0.000874520142900792
629 0.000873963105789244
630 0.000873399345465486
631 0.000872826256454573
632 0.000872263977299781
633 0.000871685401563127
634 0.000871111318332396
635 0.000870524062094091
636 0.000869945220387081
637 0.00086936366591317
638 0.000868774235641467
639 0.000868190519014433
640 0.000867604333080862
641 0.000867024145918549
642 0.000866433981428827
643 0.000865842839715469
644 0.000865257784823825
645 0.000864672511369236
646 0.000864087411088121
647 0.000863500703047748
648 0.000862911926475363
649 0.000862328706858762
650 0.000861749775339149
651 0.000861147994811517
652 0.000860576833929372
653 0.000859987604826529
654 0.000859419964882591
655 0.000858836465880586
656 0.000858270780497605
657 0.000857700917435977
658 0.000857122585358638
659 0.000856551203554545
660 0.000855953572170165
661 0.000855392873774008
662 0.000854823138610072
663 0.000854269838754362
664 0.000853706574332591
665 0.000853153943779716
666 0.000852583761258074
667 0.000852045148803882
668 0.00085149423165376
669 0.000850929714573567
670 0.000850392206501738
671 0.000849883680729135
672 0.000849348777563819
673 0.000848818826369779
674 0.000848294432671537
675 0.000847784073414459
676 0.000847260290186114
677 0.00084670621930627
678 0.000846128418629633
679 0.000845561437358811
680 0.000845014880894723
681 0.000844457324603809
682 0.000843912681460779
683 0.000843408766314724
684 0.000842857142032472
685 0.000842295145673688
686 0.000841751930806822
687 0.000841169698560407
688 0.000840482379402374
689 0.00083970852651305
690 0.000839088138974375
691 0.000838605750573151
692 0.000838122253185247
693 0.000837722880646652
694 0.000837233752406519
695 0.000836767470786981
696 0.000836359946191578
697 0.000835886536123098
698 0.000835526526515196
699 0.000835047393934474
700 0.000834685181274608
701 0.00083424653359998
702 0.000833854612295681
703 0.000833460301834066
704 0.000833060999866575
705 0.000832679260014402
706 0.000832301200290431
707 0.000831936840484104
708 0.000831564949919539
709 0.00083119627984729
710 0.000830832747681143
711 0.000830481098404334
712 0.000830152051634059
713 0.000829807797686044
714 0.00082949218042927
715 0.000829139284263647
716 0.000828825946399547
717 0.000828500172389113
718 0.000828170765572622
719 0.000827873027105852
720 0.000827545509650918
721 0.000827255031396135
722 0.000826913292740983
723 0.00082664753830386
724 0.000826341362255789
725 0.000826037688312908
726 0.000825751590497248
727 0.000825464121646746
728 0.000825189589420461
729 0.000824871367711921
730 0.000824576418608558
731 0.000824255845998323
732 0.00082395465750551
733 0.000823646473264716
734 0.000823377505639655
735 0.000823046902951319
736 0.000822797059981895
737 0.000822507672580741
738 0.00082224018137822
739 0.000821950675089056
740 0.000821679104120676
741 0.00082142238210281
742 0.000821171169206991
743 0.000820891228698883
744 0.00082065163826428
745 0.00082037707170457
746 0.000820130658382823
747 0.000819868629577059
748 0.000819605456456429
749 0.000819366369427144
750 0.000819143964037039
751 0.000818914259156145
752 0.000818632855583701
753 0.000818433661208928
754 0.000818145636515055
755 0.000817946416418636
756 0.000817671112599783
757 0.000817438018373196
758 0.000817207114437224
759 0.000816997507939732
760 0.000816717470883077
761 0.000816541029109885
762 0.000816345722398637
763 0.000816106358143998
764 0.000815861275896168
765 0.000815669663325025
766 0.000815417677245023
767 0.000815237978116556
768 0.000814993815481557
769 0.000814758376179725
770 0.000814592735821407
771 0.000814344965277769
772 0.000814164472103585
773 0.000813936487503497
774 0.000813726142183668
775 0.000813527469347264
776 0.000813279909635867
777 0.000813128239968819
778 0.000812916539274511
779 0.000812725704633976
780 0.000812470305362467
781 0.000812293689335775
782 0.000812061502017514
783 0.000811842610829672
784 0.000811659630585382
785 0.000811450738893882
786 0.000811250017591192
787 0.000811017100971867
788 0.00081076263282398
789 0.000810583505113982
790 0.000810344784156314
791 0.000810166298833792
792 0.000809939161371176
793 0.000809707540156523
794 0.000809517789917891
795 0.000809289404287483
796 0.000809094013618505
797 0.000808873317708958
798 0.000808644632655842
799 0.000808442043364721
800 0.000808177316088177
801 0.000807966236067159
802 0.000807789209034127
803 0.000807539575049532
804 0.000807327822741399
805 0.000807108336232432
806 0.000806875832950027
807 0.00080669020698565
808 0.000806424827430874
809 0.000806213777025278
810 0.000806004435588648
811 0.000805766887367554
812 0.000805524431712001
813 0.000805330381268732
814 0.000805099084232097
815 0.000804862627575176
816 0.000804655664552456
817 0.000804413681123606
818 0.000804160731576076
819 0.00080397999963111
820 0.000803749501642415
821 0.000803509629292876
822 0.000803295033364293
823 0.000803039909470726
824 0.000802818798348426
825 0.000802609565909052
826 0.000802377967971779
827 0.000802128490931864
828 0.000801930752231783
829 0.000801689931620331
830 0.00080145668420073
831 0.000801241663054952
832 0.000800985422245049
833 0.00080076710705157
834 0.000800537323698336
835 0.000800315709255983
836 0.000800085718282162
837 0.000799834922389664
838 0.000799647230707023
839 0.000799375511292055
840 0.000799132238000766
841 0.000798936378345161
842 0.000798662979008213
843 0.000798436239421108
844 0.000798188937864097
845 0.000797909489421045
846 0.000797629762558927
847 0.00079738014517261
848 0.000797153726296074
849 0.000796860131941912
850 0.000796626801189859
851 0.00079637241680075
852 0.00079613717272764
853 0.000795883862423352
854 0.000795642821572073
855 0.000795385596461529
856 0.000795156887363646
857 0.000794882714984624
858 0.00079467442409964
859 0.000794381898430174
860 0.000794122708214218
861 0.000793855476388217
862 0.000793650697062276
863 0.000793347028320568
864 0.000793114056477862
865 0.000792838317892119
866 0.000792607247973365
867 0.000792315078712136
868 0.000792090522452327
869 0.000791834349797682
870 0.000791590441423295
871 0.000791301934214061
872 0.000791072060223996
873 0.000790791707544258
874 0.000790561382672195
875 0.000790280612420702
876 0.000790061426329203
877 0.0007897946287585
878 0.000789573203718419
879 0.000789288849603054
880 0.000789071684494047
881 0.000788828092936456
882 0.000788585670875364
883 0.000788296514116382
884 0.000788050474000102
885 0.000787788242519127
886 0.000787491320920708
887 0.00078725727050255
888 0.000786973348937181
889 0.000786816156562509
890 0.000786510284200403
891 0.000786210136055843
892 0.000785819344912397
893 0.000785560187011924
894 0.000785265249163558
895 0.000784885083618292
896 0.000784655689500369
897 0.000784291630054668
898 0.0007840198847191
899 0.000783771052056181
900 0.000783436282489447
901 0.000783251941072649
902 0.000782914982039529
903 0.000782719371329677
904 0.000782380642846192
905 0.00078221920060173
906 0.000781861902254377
907 0.000781624383364488
908 0.000781409078939532
909 0.000781169910226254
910 0.000780854941950793
911 0.000780618639112163
912 0.000780319431726184
913 0.00078012493878532
914 0.000779819815249994
915 0.000779579836205357
916 0.000779274242205474
917 0.000779085638583865
918 0.000778833700934456
919 0.000778466064332406
920 0.000778310374073499
921 0.00077798367672699
922 0.000777820494164416
923 0.000777553347234061
924 0.000777203834985585
925 0.000777062177775178
926 0.000776771815679922
927 0.000776428116154193
928 0.00077625902423506
929 0.000775903079414775
930 0.000775738750121491
931 0.000775473286637407
932 0.000775169828159505
933 0.000774981007879205
934 0.000774767429817302
935 0.00077440971747933
936 0.000774237385058996
937 0.000773998428570621
938 0.000773676486147679
939 0.000773493833662542
940 0.000773251801575725
941 0.000772987618233856
942 0.000772718410132711
943 0.000772528415069473
944 0.000772258316715124
945 0.000771957197997608
946 0.00077176429812198
947 0.000771539639799812
948 0.000771227882808034
949 0.000771047372495559
950 0.000770851674360529
951 0.000770609490700735
952 0.000770322460311945
953 0.000770170618210386
954 0.000769936045486475
955 0.000769628264436051
956 0.000769503017323814
957 0.000769261938501131
958 0.000769027587807614
959 0.000768753283779233
960 0.000768594301462144
961 0.000768343796067938
962 0.000768091744333788
963 0.00076790680469685
964 0.000767703031101519
965 0.000767525888676346
966 0.000767230088314363
967 0.000767129223447682
968 0.000766901874158066
969 0.00076672131729083
970 0.000766434666843452
971 0.00076633436859197
972 0.000766107871271515
973 0.000765830624374075
974 0.000765719468120096
975 0.000765477688531746
976 0.000765248369106075
977 0.000765094436445679
978 0.00076491079764196
979 0.000764673261471671
980 0.000764529899583977
981 0.000764323557319813
982 0.000764134014758611
983 0.000764008645916192
984 0.00076380967584555
985 0.000763635905030924
986 0.000763436018957009
987 0.000763316582435891
988 0.000763131302079501
989 0.000762911010554035
990 0.000762796607460814
991 0.000762650067372306
992 0.000762407150403988
993 0.000762265268463125
994 0.000762140354339635
995 0.000761925723395507
996 0.000761791593646421
997 0.000761634271981393
998 0.000761455775545983
999 0.00076130598807822
1000 0.000761147524826811
1001 0.00076094556700923
1002 0.00076084663132292
1003 0.000760655752770845
1004 0.000760528862542742
1005 0.000760401083539364
1006 0.000760263332722388
1007 0.00076011698166667
1008 0.000759940295409933
1009 0.000759863002684824
1010 0.000759698064712211
1011 0.000759571549963312
1012 0.000759384307144728
1013 0.000759322610974777
1014 0.000759181138732856
1015 0.000758988650204628
1016 0.000758933358582681
1017 0.000758744707809456
1018 0.000758670805026895
1019 0.000758542548851437
1020 0.000758403632090676
1021 0.000758279262100814
1022 0.000758164870944711
1023 0.000758077838611371
1024 0.000757904517286079
1025 0.000757816913107945
1026 0.000757708419229175
1027 0.000757538549407855
1028 0.000757484029918487
1029 0.000757396933977361
1030 0.000757232473006297
1031 0.000757140193599071
1032 0.000757066355333791
1033 0.000756984999043198
1034 0.000756821547156505
1035 0.000756680595486614
1036 0.0007566458882593
1037 0.000756504787062795
1038 0.000756440796067182
1039 0.000756321877162236
1040 0.000756205027187207
1041 0.000756123593987468
1042 0.000756000319881878
1043 0.000755903639173994
1044 0.000755820470828894
1045 0.000755711658285918
1046 0.000755618158251536
1047 0.000755548648669446
1048 0.0007554382554531
1049 0.000755350824618972
1050 0.000755278417415184
1051 0.00075519005989122
1052 0.000755074723883808
1053 0.000755017086049747
1054 0.000754921662831975
1055 0.000754862092577469
1056 0.000754748456188281
1057 0.000754676682419131
1058 0.000754581747770544
1059 0.000754496665933857
1060 0.00075446030450621
1061 0.000754324939947537
1062 0.000754240237171189
1063 0.00075417622613827
1064 0.000754089049308959
1065 0.000753984640653016
1066 0.000753963085742271
1067 0.000753864297337259
1068 0.000753786303135939
1069 0.000753737741206351
1070 0.000753622824021249
1071 0.000753555975023801
1072 0.000753514939106026
1073 0.000753410952768263
1074 0.000753347077960598
1075 0.000753259947430251
1076 0.000753184171514931
1077 0.00075309592327244
1078 0.000753021605987669
1079 0.000752927267370751
1080 0.000752888415348707
1081 0.000752777761618972
1082 0.000752720900038639
1083 0.000752664157545269
1084 0.000752576651819936
1085 0.00075253089099192
1086 0.000752453186606772
1087 0.000752381468117846
1088 0.000752328370452915
1089 0.000752232273498521
1090 0.000752188614200122
1091 0.000752100225582808
1092 0.000752016888043272
1093 0.000751959797952395
1094 0.000751878717380805
1095 0.0007518091716463
1096 0.000751758550933346
1097 0.000751672292892636
1098 0.000751618148342459
1099 0.00075153240408099
1100 0.000751454273000718
1101 0.000751411747103248
1102 0.000751316943336633
1103 0.000751258896769968
1104 0.00075120595744238
1105 0.000751140601920497
1106 0.000751058445985109
1107 0.000751021241484295
1108 0.000750937250018069
1109 0.000750894822544979
1110 0.000750848232200951
1111 0.000750785473030646
1112 0.000750722893116063
1113 0.000750665565561803
1114 0.000750600192532147
1115 0.000750542104810847
1116 0.00075048334741723
1117 0.000750447888094641
1118 0.000750388816584291
1119 0.000750317929998801
1120 0.000750288060601179
1121 0.000750199050003175
1122 0.00075015988005589
1123 0.000750107247682763
1124 0.000750025684965294
1125 0.000749984306764873
1126 0.00074992450348077
1127 0.000749850753322789
1128 0.00074980783156775
1129 0.000749733096085947
1130 0.000749684307919551
1131 0.000749628678789804
1132 0.000749565266602303
1133 0.000749531338243514
1134 0.000749466735356918
1135 0.000749412056052279
1136 0.000749372430789208
1137 0.000749309925822672
1138 0.000749272776516818
1139 0.000749215490657207
1140 0.000749157378720611
1141 0.000749118300575446
1142 0.000749052914045478
1143 0.000749022290762014
1144 0.000748979225505764
1145 0.000748903482559626
1146 0.00074887090664788
1147 0.000748792591480196
1148 0.00074875732906321
1149 0.000748709547679027
1150 0.000748651710580361
1151 0.000748596159922954
1152 0.000748548487024436
1153 0.000748478074712011
1154 0.000748428891057529
1155 0.000748388092574714
1156 0.000748319512496209
1157 0.00074826702405062
1158 0.000748233851084024
1159 0.000748170724449437
1160 0.000748122606609058
1161 0.000748066448721829
1162 0.000748025496562832
1163 0.000747981099380013
1164 0.000747927279491023
1165 0.000747876149176818
1166 0.000747845787856249
1167 0.000747777558501639
1168 0.000747735029449359
1169 0.000747697178269391
1170 0.000747645270308794
1171 0.000747590664218478
1172 0.000747562530307277
1173 0.000747501913281212
1174 0.000747462400937593
1175 0.000747424447553158
1176 0.000747374049439031
1177 0.000747341886466302
1178 0.000747294075864602
1179 0.0007472441363916
1180 0.000747206058832717
1181 0.00074714777014151
1182 0.00074711468073474
1183 0.000747080025490732
1184 0.000747019232932189
1185 0.000746975902245595
1186 0.000746938371435135
1187 0.000746875118778689
1188 0.000746824695283976
1189 0.000746792278448538
1190 0.000746729529254253
1191 0.000746700857149563
1192 0.000746631802883257
1193 0.000746585525178034
1194 0.000746553979269038
1195 0.000746506635380229
1196 0.000746440528274661
1197 0.000746409178219665
1198 0.000746360430355253
1199 0.000746325529632941
1200 0.000746289091836161
1201 0.000746229926562592
1202 0.000746196154040035
1203 0.000746169018952969
1204 0.000746121368933927
1205 0.000746060498784118
1206 0.000746034819712804
1207 0.000745979866195512
1208 0.00074594488333446
1209 0.000745921002163641
1210 0.000745859158712392
1211 0.000745821884265752
1212 0.000745770120119005
1213 0.000745749078362223
1214 0.000745702900445622
1215 0.000745657930224297
1216 0.000745628778702212
1217 0.000745579520270212
1218 0.000745515019048071
1219 0.000745482152154864
1220 0.000745432199238394
1221 0.000745391331577139
1222 0.000745342582945341
1223 0.000745292444094048
1224 0.000745238393506042
1225 0.000745184514698849
1226 0.000745137244507532
1227 0.000745104374061611
1228 0.000745041404201174
1229 0.000744992040864645
1230 0.000744960016504592
1231 0.000744912120211438
1232 0.00074488036028697
1233 0.000744854311108156
1234 0.000744798136764757
1235 0.000744763633861112
1236 0.000744727748269725
1237 0.000744702011019172
1238 0.000744662778259908
1239 0.000744647502585849
1240 0.000744620650635852
1241 0.000744581953796342
1242 0.000744554997197611
1243 0.000744543797594588
1244 0.000744491788736923
1245 0.000744461178697975
1246 0.000744442907631537
1247 0.000744405830829464
1248 0.000744366233732308
1249 0.000744343581828844
1250 0.000744288451699049
1251 0.000744239447300288
1252 0.000744207318291501
1253 0.000744152033860246
1254 0.000744101839273981
1255 0.00074406529921589
1256 0.000744012391379556
1257 0.000743967608116236
1258 0.000743923553386594
1259 0.000743859749405829
1260 0.000743804720656271
1261 0.000743765226133064
1262 0.000743705053622534
1263 0.000743654487592948
1264 0.000743629233767251
1265 0.000743581793329895
1266 0.000743543021798132
1267 0.000743496268711397
1268 0.000743463839199876
1269 0.000743441241326082
1270 0.000743382569055484
1271 0.000743352647560869
1272 0.000743336685673057
1273 0.000743302372001153
1274 0.000743271797716716
1275 0.000743258780545375
1276 0.000743228919645844
1277 0.000743197991170064
1278 0.0007431827114317
1279 0.000743151580877566
1280 0.000743128957395811
1281 0.000743111585563838
1282 0.000743074749493644
1283 0.000743044151505501
1284 0.000743030841761083
1285 0.000742985174611022
1286 0.000742956400927142
1287 0.000742927955172945
1288 0.000742880853835004
1289 0.000742842195904814
1290 0.000742818924351241
1291 0.000742764526620476
1292 0.000742722485512104
1293 0.000742692650135268
1294 0.000742641134479527
1295 0.000742586320626515
1296 0.00074255055096728
1297 0.000742489817156411
1298 0.000742447623196085
1299 0.000742419467229638
1300 0.000742367489721119
1301 0.00074233049645045
1302 0.000742286579964002
1303 0.000742238108017546
1304 0.000742198985534515
1305 0.000742177169655633
1306 0.000742131470872209
1307 0.00074209408970205
1308 0.000742076121980517
1309 0.000742027681297941
1310 0.000741991866107128
1311 0.000741982799240759
1312 0.000741941320399064
1313 0.000741916044034951
1314 0.000741907941403497
1315 0.000741871305052655
1316 0.000741847973841914
1317 0.000741841811162658
1318 0.000741817702731851
1319 0.000741798038689012
1320 0.000741792740996061
1321 0.000741764023700853
1322 0.000741740528667378
1323 0.000741730805941643
1324 0.000741706416135912
1325 0.000741681784973025
1326 0.000741660367651775
1327 0.000741644318765111
1328 0.000741606916875526
1329 0.000741571961015097
1330 0.000741544191271259
1331 0.000741511082537727
1332 0.000741468970232972
1333 0.000741440020505024
1334 0.000741395224650887
1335 0.000741351873955409
1336 0.000741308483583225
1337 0.000741252297785877
1338 0.000741205129060063
1339 0.000741168829563321
1340 0.000741112892910678
1341 0.000741068117946497
1342 0.000741030180364532
1343 0.000740977972782275
1344 0.000740933411975675
1345 0.000740898430478865
1346 0.000740850030808815
1347 0.000740809626279315
1348 0.000740771730960432
1349 0.000740747111308337
1350 0.000740705523611496
1351 0.000740672098999084
1352 0.000740655364296572
1353 0.000740622760474707
1354 0.00074059827662154
1355 0.000740590108591732
1356 0.000740566413298893
1357 0.000740545775073542
1358 0.000740532768304547
1359 0.000740531849999115
1360 0.000740520226656827
1361 0.000740510864972066
1362 0.000740510818076245
1363 0.000740512667277926
1364 0.000740506986829814
1365 0.00074050010059068
1366 0.000740494178131712
1367 0.000740496428250026
1368 0.000740488000730011
1369 0.000740473586830603
1370 0.000740466893205394
1371 0.000740450425297468
1372 0.000740430282718307
1373 0.00074040262737185
1374 0.000740380130281437
1375 0.000740353050844078
1376 0.000740306966662274
1377 0.000740268697683177
1378 0.000740216787079362
1379 0.000740175222944117
1380 0.000740117529545614
1381 0.000740063033646265
1382 0.000740019049260354
1383 0.000739956374388839
1384 0.000739900533716309
1385 0.000739855107639187
1386 0.000739792415885177
1387 0.000739737614168234
1388 0.000739695926966988
1389 0.000739638063436132
1390 0.000739586747869225
1391 0.000739550818593671
1392 0.000739499263403331
1393 0.000739462331949881
1394 0.000739416759813594
1395 0.000739378943762858
1396 0.000739354401616765
1397 0.000739315887301473
1398 0.000739284739523782
1399 0.000739276122999399
1400 0.000739250082745002
1401 0.000739236317713221
1402 0.000739240465975399
1403 0.000739232716227889
1404 0.000739238099413342
1405 0.0007392614518551
1406 0.000739276905505903
1407 0.000739296833899061
1408 0.000739328697818564
1409 0.000739352895152479
1410 0.000739381369925241
1411 0.000739396978985951
1412 0.000739419851925049
1413 0.000739428943631992
1414 0.000739447955481864
1415 0.000739463039195698
1416 0.000739475390872713
1417 0.000739467038499697
1418 0.000739467429525575
1419 0.000739443368161119
1420 0.000739424389877286
1421 0.00073938502075066
1422 0.000739340431607616
1423 0.000739295009594798
1424 0.00073923058772607
1425 0.000739163749472027
1426 0.000739099424151846
1427 0.000739018010818882
1428 0.000738936187900663
1429 0.000738871119835949
1430 0.000738791674450567
1431 0.000738730742710914
1432 0.000738660594294061
1433 0.000738607775815581
1434 0.000738545036682581
1435 0.000738499973834905
1436 0.000738443295716706
1437 0.000738401965520552
1438 0.000738353456540608
1439 0.000738315096214137
1440 0.00073826396354093
1441 0.000738225073945387
1442 0.000738174230434652
1443 0.000738133429564414
1444 0.000738082999760081
1445 0.000738045760584782
1446 0.000737999583634519
1447 0.000737966777251131
1448 0.000737932091055882
1449 0.000737914661470995
1450 0.000737885283740525
1451 0.000737871280165336
1452 0.000737852587036514
1453 0.000737850341181456
1454 0.000737845273931725
1455 0.000737848919527551
1456 0.000737865061495313
1457 0.000737877871983983
1458 0.000737887547757055
1459 0.000737915293541391
1460 0.00073794471896349
1461 0.000737965530134943
1462 0.000738005433362332
1463 0.00073805393310522
1464 0.000738073466209244
1465 0.000738121880857534
1466 0.000738163053824792
1467 0.000738228162731502
1468 0.000738267262903491
1469 0.000738338204314459
1470 0.000738399419333291
1471 0.000738473586835653
1472 0.000738603337310906
1473 0.000738678453870989
1474 0.00073872430556321
1475 0.000738819173420779
1476 0.000738949317593551
1477 0.000738966759087134
1478 0.000739013385185672
1479 0.000739058607393872
1480 0.00073900130672655
1481 0.000738955619709714
1482 0.000738867942459365
1483 0.00073872339643799
1484 0.000738553203490255
1485 0.000738370325024107
1486 0.000738160382553588
1487 0.000737983786450513
1488 0.000737829151972846
1489 0.000737708846799023
1490 0.000737639489074127
1491 0.000737588936942757
1492 0.000737599440412851
1493 0.000737643347605399
1494 0.000737711103141692
1495 0.000737867194004593
1496 0.000737966490845565
1497 0.000738162593620473
1498 0.000738424748448097
1499 0.000738713035246974
1500 0.000739086436055914
1501 0.0007395496881486
1502 0.000740048765067058
1503 0.000740689289045804
1504 0.000741443816025367
1505 0.000742264953970562
1506 0.000742769354786788
1507 0.000742999760916518
1508 0.000742868292121557
1509 0.000742187630351054
1510 0.000741145932522613
1511 0.000740087618936514
1512 0.000739208324745277
1513 0.000738554731896102
1514 0.000738099196411213
1515 0.000737782678299936
1516 0.000737549465327447
1517 0.000737394012361392
1518 0.000737269832484344
1519 0.00073718802710232
1520 0.000737125660378979
1521 0.00073706488618086
1522 0.000737030558667584
1523 0.000736999278842632
1524 0.000736985971087734
1525 0.000736972634115318
1526 0.000736954746201945
1527 0.000736952993776185
1528 0.000736958715265246
1529 0.000736954428020908
1530 0.000736971258845642
1531 0.000736985393331224
1532 0.000737000501288776
1533 0.000737034972672745
1534 0.000737084039798219
1535 0.00073711200744242
1536 0.000737176564285846
1537 0.000737236567800892
1538 0.000737314531818356
1539 0.000737383889742205
1540 0.000737475494418049
1541 0.000737575389536005
1542 0.00073768343906977
1543 0.000737801637029634
1544 0.000737898933664383
1545 0.000738015844802931
1546 0.000738123812084268
1547 0.000738147315530568
1548 0.000738197313808087
1549 0.000738192885620492
1550 0.00073818234875489
1551 0.000738099453059249
1552 0.000738000125807048
1553 0.00073786155709854
1554 0.000737714722902183
1555 0.000737544407655832
1556 0.000737357232480917
1557 0.000737199868325433
1558 0.000737024879072123
1559 0.000736885268509013
1560 0.000736758379389357
1561 0.00073661954266413
1562 0.000736523041467763
1563 0.000736428760461649
1564 0.000736340355729226
1565 0.000736264631854056
1566 0.000736212340740394
1567 0.000736140605170021
1568 0.000736095945399029
1569 0.000736058592053723
1570 0.000736001111164342
1571 0.000735967526395598
1572 0.000735945963043605
1573 0.000735931278711632
1574 0.000735924524633447
1575 0.000735911918354759
1576 0.000735919988528622
1577 0.000735904021297529
1578 0.000735905483821853
1579 0.000735896184352214
1580 0.000735887647550726
1581 0.000735871379305308
1582 0.000735866257372209
1583 0.000735842258592356
1584 0.000735833623508597
1585 0.00073579761422593
1586 0.000735770133132974
1587 0.000735755666681825
1588 0.000735745546336375
1589 0.000735732177986392
1590 0.000735725033393919
1591 0.000735705468173364
1592 0.000735699085112174
1593 0.000735677400285795
1594 0.000735701669668742
1595 0.000735656000983909
1596 0.000735659175632009
1597 0.000735640918065883
1598 0.000735628052865422
1599 0.000735613585732153
1600 0.000735594261328743
1601 0.000735576707796781
1602 0.000735566033654322
1603 0.000735542034817627
1604 0.000735529603474561
1605 0.000735546243561203
1606 0.00073549490329583
1607 0.000735499469612932
1608 0.000735473350999882
1609 0.000735480116873077
1610 0.000735455515780359
1611 0.00073549804235995
1612 0.000735506974024247
1613 0.000735775590243293
1614 0.000736062261921688
1615 0.000737184043231309
1616 0.000737748242585212
1617 0.000737757710709275
1618 0.000736613131863351
1619 0.000735777972749929
1620 0.000735575675832933
1621 0.00073547891599901
1622 0.000735478547795765
1623 0.000735467548452107
1624 0.000735472614451282
1625 0.000735464837106292
1626 0.000735511304071679
1627 0.000735484511125151
1628 0.000735496829804561
1629 0.000735500136300971
1630 0.000735498898961851
1631 0.000735508797873763
1632 0.00073549943482476
1633 0.000735522554947465
1634 0.000735512193813292
1635 0.000735570394454044
1636 0.000735543064962485
1637 0.000735576150731276
1638 0.000735585371245406
1639 0.000735626369504416
1640 0.000735633224792309
1641 0.000735688943024115
1642 0.000735701783668219
1643 0.000735775766003144
1644 0.000735828529769833
1645 0.000735900282450075
1646 0.000735989525622927
1647 0.000736186387769067
1648 0.000736376443256859
1649 0.000736834736244418
1650 0.000737170926271347
1651 0.000737580031312746
1652 0.000738176779918831
1653 0.000738935519621009
1654 0.000739952089872986
1655 0.000741138816039211
1656 0.000742104105398766
1657 0.000742329679297882
1658 0.000741389209338195
1659 0.000739611712816668
1660 0.00073781808066542
1661 0.000736619503783231
1662 0.000736122904356762
1663 0.00073624027638175
1664 0.000736672814696249
1665 0.000737330088327326
1666 0.000738042508061199
1667 0.000738613219738227
1668 0.000738859669638714
1669 0.000738734510093764
1670 0.000738470504302313
1671 0.000738217795088758
1672 0.000737859523525231
1673 0.000737518251725078
1674 0.000737209932964333
1675 0.000736900071842683
1676 0.000736668960456655
1677 0.000736459731882633
1678 0.000736306946777177
1679 0.000736158344722071
1680 0.000736060846634246
1681 0.000736003570381172
1682 0.00073592567440528
1683 0.000735885594281172
1684 0.000735818666896648
1685 0.000735792964576376
1686 0.000735752354898978
1687 0.00073572808818767
1688 0.000735697979081351
1689 0.000735716032835398
1690 0.000735673361589306
1691 0.000735649723878851
1692 0.000735645778860317
1693 0.000735628363230489
1694 0.000735616878728251
1695 0.000735647618597568
1696 0.000735603733005519
1697 0.000735614862577449
1698 0.00073560707676279
1699 0.000735608581749148
1700 0.000735603510491956
1701 0.000735602410316005
1702 0.000735554815207706
1703 0.000735561565051057
1704 0.000735549809860458
1705 0.000735545409185079
1706 0.00073555050005325
1707 0.00073556370276151
1708 0.000735517656124784
1709 0.000735528054406132
1710 0.000735518510197153
1711 0.000735507852795081
1712 0.000735516652838442
1713 0.000735524307600599
1714 0.000735482349313088
1715 0.000735477579041799
1716 0.000735465671340307
1717 0.00073546273847569
1718 0.000735469675163358
1719 0.000735414658009859
1720 0.000735417914597747
1721 0.000735397960539785
1722 0.000735392698118176
1723 0.000735378547830123
1724 0.000735387680947497
1725 0.000735341043707649
1726 0.000735320572374576
1727 0.000735308967023229
1728 0.000735302206209099
1729 0.000735279439027181
1730 0.000735306268467184
1731 0.000735240176993557
1732 0.000735223792730721
1733 0.000735223445559541
1734 0.000735206284815604
1735 0.000735218035259777
1736 0.000735174373744485
1737 0.000735162412524915
1738 0.000735161254311834
1739 0.000735148107992245
1740 0.000735133893044804
1741 0.000735167030825323
1742 0.000735109102805609
1743 0.000735118220887898
1744 0.000735108797499606
1745 0.000735106883155368
1746 0.000735132717807119
1747 0.000735084928578544
1748 0.000735092172675422
1749 0.000735087837881565
1750 0.000735089549237955
1751 0.000735093945365861
1752 0.000735117132620644
1753 0.000735085593589702
1754 0.000735081714566377
1755 0.000735075556463016
1756 0.000735091517015007
1757 0.000735108550600216
1758 0.00073507512573201
1759 0.000735070735430554
1760 0.000735057859117205
1761 0.000735067107285658
1762 0.000735078728922645
1763 0.000735034539616208
1764 0.000735018601034199
1765 0.000735004206489975
1766 0.00073499490457607
1767 0.000735000413953912
1768 0.00073494497726756
1769 0.000734918047385236
1770 0.000734890014769007
1771 0.000734877753018282
1772 0.00073486126075295
1773 0.00073480071347376
1774 0.000734764921190845
1775 0.000734744020064682
1776 0.000734713559523925
1777 0.000734702113589947
1778 0.000734640276391474
1779 0.000734603802868605
1780 0.000734584711011621
1781 0.000734558369885008
1782 0.000734560926474614
1783 0.00073450025925581
1784 0.000734529168369136
1785 0.000734501184439296
1786 0.00073447377585012
1787 0.000734487324649535
1788 0.000734433822884739
1789 0.000734413036525439
1790 0.000734415366821395
1791 0.000734411498342524
1792 0.000734455139848933
1793 0.000734415870084604
1794 0.000734421726036771
1795 0.000734439958335997
1796 0.000734486621837505
1797 0.000734485004869612
1798 0.000734510657764531
1799 0.000734565967348999
1800 0.000734641017459126
1801 0.000734685727280748
1802 0.000734761277129792
1803 0.000734869671106253
1804 0.00073501657047359
1805 0.000735129079004082
1806 0.000735284426639282
1807 0.000735473591333857
1808 0.000735715656503544
1809 0.000735874426226246
1810 0.000736083259369025
1811 0.000736241833806162
1812 0.000736370906253114
1813 0.000736292160354424
1814 0.000736158972273415
1815 0.000735868854690125
1816 0.000735559144374065
1817 0.00073513784741408
1818 0.000734810967514932
1819 0.000734534628463734
1820 0.000734341146539919
1821 0.000734214917457621
1822 0.000734166548141957
1823 0.000734191010991481
1824 0.000734277423902086
1825 0.000734433140110013
1826 0.000734662648028461
1827 0.000735011579251932
1828 0.000735465911901656
1829 0.00073600718909006
1830 0.000736627280133462
1831 0.000737303189850991
1832 0.000737969425955498
1833 0.000738486164095775
1834 0.000738865044041859
1835 0.000738867813566912
1836 0.000738475997763999
1837 0.00073776255291591
1838 0.000736885365483886
1839 0.000736037533101808
1840 0.000735330889057195
1841 0.00073480780480395
1842 0.000734395113909159
1843 0.000734112072422022
1844 0.000733908590973442
1845 0.000733781872725103
1846 0.000733653747118979
1847 0.000733572435649421
1848 0.000733535534919838
1849 0.000733470370278155
1850 0.000733437236704049
1851 0.000733432938204714
1852 0.000733397155499915
1853 0.000733389104595972
1854 0.000733399895750608
1855 0.000733381737575201
1856 0.000733388520870903
1857 0.000733395328694542
1858 0.000733435149101069
1859 0.000733429421870824
1860 0.000733460598269176
1861 0.000733510156806005
1862 0.000733535058373036
1863 0.000733586089182836
1864 0.000733647855497566
1865 0.000733733065743536
1866 0.000733798262615437
1867 0.000733884762212256
1868 0.000733990148745534
1869 0.000734112858197022
1870 0.000734211572222421
1871 0.000734333623171324
1872 0.000734449989323593
1873 0.000734641568072902
1874 0.000734804810292644
1875 0.000734986207589827
1876 0.000735180851393125
1877 0.000735351843559329
1878 0.000735416634796593
1879 0.000735405390742017
1880 0.000735325448346202
1881 0.000735086400709406
1882 0.000734797466037662
1883 0.000734448741610549
1884 0.000734173730563725
1885 0.000733880197373082
1886 0.0007336683135577
1887 0.000733529656173459
1888 0.000733401161539859
1889 0.000733326348296259
1890 0.000733248351565408
1891 0.000733246290451461
1892 0.00073322499846995
1893 0.000733241791522232
1894 0.00073330318255671
1895 0.000733335149874392
1896 0.000733437747641119
1897 0.000733505698605086
1898 0.000733649744290688
1899 0.000733764564699868
1900 0.000733957243710393
1901 0.000734146896235188
1902 0.000734442939773317
1903 0.00073475168997561
1904 0.000735218794744696
1905 0.000735739078010056
1906 0.000736437829317538
1907 0.000737105571573693
1908 0.000737715563246866
1909 0.00073785109825053
1910 0.000737653821119011
1911 0.000737199694668789
1912 0.000736470066073025
1913 0.000735767602918713
1914 0.000734968404003666
1915 0.000734361222924917
1916 0.000733923565405803
1917 0.000733568032700305
1918 0.000733346872038965
1919 0.000733162266072895
1920 0.000733037585831653
1921 0.00073295613893265
1922 0.000732853109298048
1923 0.000732800922861543
1924 0.00073275452047028
1925 0.000732732654199708
1926 0.000732692796901802
1927 0.000732679425340166
1928 0.00073267375313435
1929 0.00073269347274163
1930 0.000732681829191506
1931 0.00073270231632705
1932 0.000732748348553969
1933 0.000732766574145671
1934 0.000732848673095532
1935 0.000732893317064054
1936 0.000732976306409228
1937 0.000733091293682264
1938 0.000733281330383306
1939 0.000733467846686153
1940 0.000733767385611372
1941 0.000734110155150347
1942 0.000734411223589859
1943 0.000734769426657067
1944 0.000735211404929714
1945 0.000735561297972254
1946 0.00073587389823615
1947 0.000735957276987165
1948 0.000735852031510831
1949 0.000735482559917955
1950 0.000734970642724875
1951 0.00073444292968361
1952 0.000733919049679344
1953 0.000733525661701151
1954 0.000733214832479234
1955 0.000733001490459628
1956 0.000732865780690872
1957 0.000732741949121873
1958 0.000732695204362699
1959 0.000732661552149239
1960 0.000732662375042992
1961 0.000732644091755219
1962 0.000732634040872426
1963 0.000732686541937255
1964 0.0007327300656641
1965 0.000732824174974667
1966 0.000732920389850733
1967 0.000733028611620057
1968 0.000733163341777754
1969 0.000733270654251328
1970 0.000733412953309198
1971 0.000733549192148075
1972 0.000733707088926394
1973 0.00073388300702959
1974 0.000734040625019361
1975 0.000734211290961184
1976 0.000734383263306881
1977 0.000734534935702413
1978 0.000734680035265001
1979 0.000734747274549363
1980 0.000734771904348008
1981 0.000734713936566322
1982 0.000734590377959421
1983 0.000734360244081245
1984 0.000734110286686018
1985 0.000733821345221486
1986 0.000733509735965754
1987 0.00073320996918369
1988 0.000732966970758753
1989 0.000732728030698127
1990 0.000732560217812761
1991 0.000732402528427656
1992 0.00073231151097275
1993 0.000732218547227603
1994 0.000732184445126904
1995 0.000732141908486028
1996 0.000732151084946508
1997 0.00073215969828766
1998 0.000732177834152026
1999 0.000732240088240133
};
\addlegendentry{Train}
\addplot [semithick, black]
table {%
0 0.119434930384159
1 0.117304101586342
2 0.11518806964159
3 0.11308541893959
4 0.110968038439751
5 0.108812496066093
6 0.106609411537647
7 0.104329437017441
8 0.101943418383598
9 0.0993392691016197
10 0.0962402373552322
11 0.0923535525798798
12 0.0880412459373474
13 0.0840616896748543
14 0.080422081053257
15 0.0770128443837166
16 0.0738680884242058
17 0.0709552466869354
18 0.0682505965232849
19 0.0656973943114281
20 0.0632878765463829
21 0.0610066577792168
22 0.0588423199951649
23 0.0567842461168766
24 0.0548209324479103
25 0.0529422946274281
26 0.0511406995356083
27 0.0494102723896503
28 0.0477463901042938
29 0.0461436323821545
30 0.044599000364542
31 0.0431104004383087
32 0.0416724719107151
33 0.0402819477021694
34 0.038929108530283
35 0.0376161001622677
36 0.0363509804010391
37 0.0351338870823383
38 0.0339630730450153
39 0.0328370705246925
40 0.0317545235157013
41 0.0307143852114677
42 0.0297157224267721
43 0.0287575796246529
44 0.0278389696031809
45 0.0269589107483625
46 0.0261163450777531
47 0.0253101736307144
48 0.0245392452925444
49 0.0238023772835732
50 0.0230983439832926
51 0.0224258936941624
52 0.0217837635427713
53 0.0211706757545471
54 0.020585373044014
55 0.0200265869498253
56 0.0194930955767632
57 0.0189837031066418
58 0.0184972602874041
59 0.0180326066911221
60 0.0175887010991573
61 0.0171645097434521
62 0.0167590416967869
63 0.0163713674992323
64 0.0160006005316973
65 0.0156459156423807
66 0.0153065090999007
67 0.0149816311895847
68 0.0146705750375986
69 0.0143722016364336
70 0.0140842543914914
71 0.0138051407411695
72 0.0135364588350058
73 0.0132818156853318
74 0.0130378473550081
75 0.0128038413822651
76 0.012579252012074
77 0.0123635735362768
78 0.0121563691645861
79 0.0119571946561337
80 0.0117656541988254
81 0.0115813557058573
82 0.0114039406180382
83 0.0112330596894026
84 0.011068370193243
85 0.0109095545485616
86 0.0107563128694892
87 0.0106083489954472
88 0.0104653304442763
89 0.0103267878293991
90 0.0101894997060299
91 0.0100520569831133
92 0.00991679634898901
93 0.00979145336896181
94 0.00967022217810154
95 0.00955258030444384
96 0.00943826697766781
97 0.0093270493671298
98 0.00921900942921638
99 0.00911383051425219
100 0.00901121646165848
101 0.00891094375401735
102 0.00881307199597359
103 0.00871721655130386
104 0.00862338114529848
105 0.00853146146982908
106 0.00844133459031582
107 0.00835289712995291
108 0.00826608296483755
109 0.00818066578358412
110 0.00809671264141798
111 0.00801408011466265
112 0.00793261267244816
113 0.00785237271338701
114 0.00777323730289936
115 0.00769512448459864
116 0.00761797139421105
117 0.0075416206382215
118 0.00746610993519425
119 0.00739154545590281
120 0.00731767108663917
121 0.00724448822438717
122 0.00717196194455028
123 0.00710004568099976
124 0.0070287031121552
125 0.00695789465680718
126 0.0068875877186656
127 0.0068177436478436
128 0.00674833683297038
129 0.00667933328077197
130 0.00661071063950658
131 0.00654243770986795
132 0.00647449167445302
133 0.00640696147456765
134 0.00633961381390691
135 0.00627252925187349
136 0.00620585167780519
137 0.00613920344039798
138 0.00607279874384403
139 0.00600674375891685
140 0.00594073254615068
141 0.0058750519528985
142 0.00580955715849996
143 0.00574433617293835
144 0.00567909283563495
145 0.00561385788023472
146 0.00554929627105594
147 0.00548448879271746
148 0.00542001100257039
149 0.00535544799640775
150 0.00529124168679118
151 0.00522693200036883
152 0.00516327749937773
153 0.00509925978258252
154 0.00503591122105718
155 0.00497265625745058
156 0.00490905111655593
157 0.00484619988128543
158 0.00478346878662705
159 0.00472091045230627
160 0.00465855095535517
161 0.00459640193730593
162 0.00453447736799717
163 0.00447279214859009
164 0.0044113602489233
165 0.00435057887807488
166 0.00429012719541788
167 0.00422960333526134
168 0.00416982732713223
169 0.00411049695685506
170 0.00405153818428516
171 0.00399296497926116
172 0.00393479317426682
173 0.0038772018160671
174 0.00382010382600129
175 0.00376354902982712
176 0.0037074820138514
177 0.00365196913480759
178 0.00359701178967953
179 0.0035426199901849
180 0.00348880863748491
181 0.00343557656742632
182 0.00338293402455747
183 0.00333089008927345
184 0.00327944871969521
185 0.00322849792428315
186 0.00317817553877831
187 0.00312843546271324
188 0.0030792998149991
189 0.00303076324053109
190 0.00298283924348652
191 0.00293546984903514
192 0.00288882316090167
193 0.00284282630309463
194 0.00279780244454741
195 0.00275343493558466
196 0.00270968745462596
197 0.0026664650067687
198 0.00262384722009301
199 0.00258175493218005
200 0.00254023517481983
201 0.00249928887933493
202 0.00245886971242726
203 0.00241904822178185
204 0.00237971963360906
205 0.00234100339002907
206 0.00230281660333276
207 0.00226520979776978
208 0.0022281683050096
209 0.00219169561751187
210 0.00215581012889743
211 0.00212048529647291
212 0.00208582892082632
213 0.00205171015113592
214 0.00201824307441711
215 0.00198540138080716
216 0.00195319578051567
217 0.00192164874169976
218 0.00189075083471835
219 0.00186051742639393
220 0.00183096807450056
221 0.00180210906546563
222 0.00177391909528524
223 0.00174644729122519
224 0.00171979935839772
225 0.00169360137078911
226 0.00166822446044534
227 0.00164356757886708
228 0.00161960860714316
229 0.00159634638112038
230 0.00157380802556872
231 0.00155195046681911
232 0.00153075810521841
233 0.00151017715688795
234 0.00149026047438383
235 0.00147104705683887
236 0.00145248754415661
237 0.0014345747185871
238 0.00141724711284041
239 0.00140025035943836
240 0.00138390704523772
241 0.0013680529082194
242 0.00135277595836669
243 0.00133807631209493
244 0.00132398854475468
245 0.00131052953656763
246 0.00129762478172779
247 0.00128521118313074
248 0.00127317907754332
249 0.00126157724298537
250 0.00125040032435209
251 0.00123962853103876
252 0.00122918048873544
253 0.0012191302375868
254 0.00120943353977054
255 0.00120007432997227
256 0.00119102781172842
257 0.00118219340220094
258 0.00117365247569978
259 0.00116541213355958
260 0.00115744373761117
261 0.0011497592786327
262 0.00114237959496677
263 0.00113529642112553
264 0.0011285183718428
265 0.00112200959119946
266 0.00111575156915933
267 0.0011097431415692
268 0.00110396987292916
269 0.00109841953963041
270 0.00109304825309664
271 0.00108786218333989
272 0.00108285038731992
273 0.00107805652078241
274 0.00107346766162664
275 0.00106911384500563
276 0.00106501113623381
277 0.00106123462319374
278 0.00105781748425215
279 0.00105481722857803
280 0.0010521961376071
281 0.00104990496765822
282 0.00104794010985643
283 0.00104626442771405
284 0.00104485068004578
285 0.00104363285936415
286 0.00104256893973798
287 0.00104161095805466
288 0.00104073295369744
289 0.00103992328513414
290 0.00103917100932449
291 0.00103846762795001
292 0.0010377992875874
293 0.00103716307785362
294 0.00103654561098665
295 0.00103594490792602
296 0.00103535503149033
297 0.00103477423544973
298 0.00103420671075583
299 0.0010336491977796
300 0.00103309575933963
301 0.00103254278656095
302 0.00103200075682253
303 0.00103146419860423
304 0.00103092647623271
305 0.00103038980159909
306 0.00102985720150173
307 0.00102932832669467
308 0.00102879491169006
309 0.0010282639414072
310 0.00102773681282997
311 0.00102720141876489
312 0.00102666765451431
313 0.00102613237686455
314 0.00102560175582767
315 0.00102506764233112
316 0.00102453690487891
317 0.00102400686591864
318 0.00102346856147051
319 0.0010229330509901
320 0.00102240254636854
321 0.0010218636598438
322 0.0010213244240731
323 0.0010207905434072
324 0.00102025619708002
325 0.0010197184747085
326 0.00101918319705874
327 0.00101865327451378
328 0.00101811636704952
329 0.00101757817901671
330 0.00101704499684274
331 0.0010165135608986
332 0.00101598317269236
333 0.00101545627694577
334 0.00101493380498141
335 0.00101441831793636
336 0.00101390166673809
337 0.00101338990498334
338 0.00101288489531726
339 0.0010123843094334
340 0.00101188442204148
341 0.00101139023900032
342 0.00101074995473027
343 0.00101025938056409
344 0.001009751111269
345 0.00100931315682828
346 0.00100883457344025
347 0.00100838381331414
348 0.00100795028265566
349 0.00100748986005783
350 0.00100708031095564
351 0.00100665527861565
352 0.00100624619517475
353 0.00100582197774202
354 0.00100542581640184
355 0.00100502441637218
356 0.00100464024581015
357 0.00100424396805465
358 0.00100387400016189
359 0.0010035039158538
360 0.0010031369747594
361 0.00100276863668114
362 0.0010024345247075
363 0.00100208283402026
364 0.00100174453109503
365 0.00100140832364559
366 0.00100111600477248
367 0.00100081402342767
368 0.00100051623303443
369 0.00100023462437093
370 0.000999930547550321
371 0.000999660696834326
372 0.000999372219666839
373 0.000999121461063623
374 0.000998859526589513
375 0.000998595729470253
376 0.00099835149012506
377 0.00099809467792511
378 0.00099785893689841
379 0.000997619819827378
380 0.000997384428046644
381 0.000997152179479599
382 0.000996921909973025
383 0.000996684073470533
384 0.000996461836621165
385 0.000996249495074153
386 0.00099602818954736
387 0.000995815498754382
388 0.000995585462078452
389 0.000995376380160451
390 0.000995123875327408
391 0.000994937028735876
392 0.000994742149487138
393 0.000994544243440032
394 0.000994353787973523
395 0.000994162866845727
396 0.000993977300822735
397 0.000993773341178894
398 0.000993603840470314
399 0.000993421534076333
400 0.000993243418633938
401 0.000993063091300428
402 0.000992885325103998
403 0.00099269999191165
404 0.000992542714811862
405 0.000992373214103281
406 0.000992213375866413
407 0.000992043293081224
408 0.000991887995041907
409 0.000991724198684096
410 0.000991565990261734
411 0.000991405686363578
412 0.000991250388324261
413 0.00099107448477298
414 0.000990931876003742
415 0.000990772037766874
416 0.000990618369542062
417 0.000990453525446355
418 0.000990305910818279
419 0.000990153639577329
420 0.000990005326457322
421 0.000989850959740579
422 0.00098969938699156
423 0.000989547464996576
424 0.000989401945844293
425 0.000989247462712228
426 0.000989104039035738
427 0.000988953514024615
428 0.000988811487331986
429 0.000988664454780519
430 0.000988521729595959
431 0.00098838156554848
432 0.000988238258287311
433 0.000988093786872923
434 0.000987955252639949
435 0.000987808452919126
436 0.000987669336609542
437 0.000987524981610477
438 0.000987383769825101
439 0.000987240113317966
440 0.00098709762096405
441 0.000986956176348031
442 0.000986823812127113
443 0.000986685277894139
444 0.000986540573649108
445 0.000986410886980593
446 0.000986260245554149
447 0.000986139522865415
448 0.000985998893156648
449 0.000985855353064835
450 0.000985728576779366
451 0.000985586084425449
452 0.000985462684184313
453 0.000985329737886786
454 0.000985188176855445
455 0.00098505534697324
456 0.000984926242381334
457 0.000984791549853981
458 0.000984652899205685
459 0.000984522164799273
460 0.00098439515568316
461 0.000984259066171944
462 0.000984125072136521
463 0.000983998528681695
464 0.000983860809355974
465 0.000983729958534241
466 0.000983607606031001
467 0.000983471749350429
468 0.000983333098702133
469 0.000983199337497354
470 0.000983074307441711
471 0.000982939498499036
472 0.000982811092399061
473 0.000982680823653936
474 0.000982545665465295
475 0.000982414931058884
476 0.000982281286269426
477 0.000982144847512245
478 0.000982014113105834
479 0.000981894321739674
480 0.000981756602413952
481 0.000981616205535829
482 0.000981477554887533
483 0.000981349148787558
484 0.000981216551735997
485 0.000981081160716712
486 0.000980946468189359
487 0.000980808981694281
488 0.000980673241429031
489 0.000980538316071033
490 0.000980397686362267
491 0.000980264972895384
492 0.000980130978859961
493 0.000979985808953643
494 0.000979837961494923
495 0.000979699194431305
496 0.000979571719653904
497 0.000979425385594368
498 0.000979284523054957
499 0.000979137374088168
500 0.000979001168161631
501 0.000978855532594025
502 0.000978713156655431
503 0.000978560768999159
504 0.000978413736447692
505 0.000978287542238832
506 0.000978130497969687
507 0.000977977411821485
508 0.000977847841568291
509 0.000977722578682005
510 0.000977611518464983
511 0.000977495685219765
512 0.000977367395535111
513 0.000977273099124432
514 0.000977060059085488
515 0.000976950163021684
516 0.000976962852291763
517 0.000976526644080877
518 0.000976591138169169
519 0.000976270355749875
520 0.000976006092969328
521 0.000975769769866019
522 0.000975607079453766
523 0.000975366856437176
524 0.000975156959611923
525 0.000974975759163499
526 0.000974764232523739
527 0.00097453745547682
528 0.000974341761320829
529 0.000974127149675041
530 0.000973940652329475
531 0.000973702000919729
532 0.000973480113316327
533 0.000973283720668405
534 0.000973077083472162
535 0.00097284495132044
536 0.00097264489158988
537 0.000972421024926007
538 0.000972216774243861
539 0.000972006178926677
540 0.000971770088654011
541 0.00097156543051824
542 0.000971339002717286
543 0.000971124216448516
544 0.000970907742157578
545 0.000970666471403092
546 0.000970431137830019
547 0.000970240740571171
548 0.000970071589108557
549 0.000969618791714311
550 0.000969339162111282
551 0.000969097134657204
552 0.000968832348007709
553 0.000968589971307665
554 0.000968323496636003
555 0.000968064065091312
556 0.000967843807302415
557 0.000967568543273956
558 0.000967299158219248
559 0.000967050727922469
560 0.000966776802670211
561 0.000966526393312961
562 0.000966263061854988
563 0.000965985294897109
564 0.000965703569818288
565 0.000965437502600253
566 0.000965144892688841
567 0.000964876904617995
568 0.000964599079452455
569 0.000964298960752785
570 0.000963998551014811
571 0.000963737547863275
572 0.000963424856308848
573 0.000963134167250246
574 0.00096282153390348
575 0.000962508202064782
576 0.0009622378856875
577 0.000961912621278316
578 0.000961592188104987
579 0.000961288227699697
580 0.000960958830546588
581 0.000960650795605034
582 0.000960329489316791
583 0.000959986879024655
584 0.000959649798460305
585 0.000959326222073287
586 0.000958983902819455
587 0.000958657823503017
588 0.000958306074608117
589 0.000957968761213124
590 0.000957609037868679
591 0.000957270327489823
592 0.000956906878855079
593 0.000956560310442001
594 0.000956202740781009
595 0.000955834926571697
596 0.000955475377850235
597 0.000955098133999854
598 0.000954719725996256
599 0.000954373739659786
600 0.000953979382757097
601 0.000953601440414786
602 0.000953206210397184
603 0.000952822272665799
604 0.000952450907789171
605 0.000952052650973201
606 0.000951579713728279
607 0.000951138266827911
608 0.000950786110479385
609 0.000950377841945738
610 0.000949941051658243
611 0.000949521025177091
612 0.000949136214330792
613 0.000948732427787036
614 0.000948305591009557
615 0.000947875261772424
616 0.00094749650452286
617 0.000947061518672854
618 0.000946621294133365
619 0.000946200161706656
620 0.000945757841691375
621 0.000945353298448026
622 0.000944933970458806
623 0.000944488507229835
624 0.000944044964853674
625 0.000943615334108472
626 0.000943164981435984
627 0.000942701590247452
628 0.000942311366088688
629 0.000941878301091492
630 0.000941431615501642
631 0.000941000529564917
632 0.000940553436521441
633 0.000940118567086756
634 0.000939664430916309
635 0.00093922670930624
636 0.000938776589464396
637 0.00093833653954789
638 0.000937882578000426
639 0.000937445787712932
640 0.000936996832024306
641 0.000936563592404127
642 0.000936113821808249
643 0.000935682095587254
644 0.0009352364577353
645 0.000934805662836879
646 0.000934374576900154
647 0.000933937553782016
648 0.000933513627387583
649 0.000933093542698771
650 0.000932608381845057
651 0.000932222930714488
652 0.000931761227548122
653 0.000931380607653409
654 0.000930920126847923
655 0.000930511450860649
656 0.000930112029891461
657 0.000929694331716746
658 0.000929285772144794
659 0.000928822497371584
660 0.000928455730900168
661 0.000928015972021967
662 0.000927635526750237
663 0.000927249784581363
664 0.000926861248444766
665 0.000926420732866973
666 0.000926065724343061
667 0.000925687549170107
668 0.000925263448152691
669 0.000924871303141117
670 0.000924559368286282
671 0.000924195861443877
672 0.00092379137640819
673 0.000923438521567732
674 0.000923081941436976
675 0.000922657724004239
676 0.00092219514772296
677 0.000921665458008647
678 0.000921114115044475
679 0.000920647580642253
680 0.000920129183214158
681 0.000919640704523772
682 0.000919238256756216
683 0.000918698846362531
684 0.000918168050702661
685 0.000917630910407752
686 0.000917026191018522
687 0.00091651629190892
688 0.000915941433049738
689 0.000915595679543912
690 0.000915273441933095
691 0.000915056094527245
692 0.000914753531105816
693 0.000914476637262851
694 0.000914177682716399
695 0.000913962547201663
696 0.000913662312086672
697 0.00091346149565652
698 0.000913163530640304
699 0.000912961957510561
700 0.00091267068637535
701 0.000912458577658981
702 0.000912196876015514
703 0.000911926268599927
704 0.000911750306840986
705 0.000911490817088634
706 0.000911309034563601
707 0.000911036971956491
708 0.000910852802917361
709 0.000910605769604445
710 0.000910367933101952
711 0.000910199887584895
712 0.000909966533072293
713 0.000909801106899977
714 0.000909556867554784
715 0.000909360125660896
716 0.000909153139218688
717 0.000908919901121408
718 0.000908767397049814
719 0.000908546033315361
720 0.000908369780518115
721 0.000908148766029626
722 0.000907972978893667
723 0.000907763373106718
724 0.000907534558791667
725 0.000907396839465946
726 0.000907162728253752
727 0.000907017849385738
728 0.000906751200091094
729 0.000906549743376672
730 0.000906219764146954
731 0.00090599455870688
732 0.000905678316485137
733 0.000905483204405755
734 0.00090519618242979
735 0.00090496928896755
736 0.000904719694517553
737 0.000904518296010792
738 0.000904247805010527
739 0.00090404786169529
740 0.000903765438124537
741 0.000903580861631781
742 0.000903312466107309
743 0.000903130799997598
744 0.000902872416190803
745 0.000902650994248688
746 0.000902391388081014
747 0.000902222061995417
748 0.000901949650142342
749 0.000901857565622777
750 0.000901742721907794
751 0.000901536084711552
752 0.000901412393432111
753 0.000901195511687547
754 0.000901057384908199
755 0.000900838640518486
756 0.000900648476090282
757 0.000900535727851093
758 0.000900350336451083
759 0.000900140265002847
760 0.000900045968592167
761 0.000899978971574455
762 0.000899777747690678
763 0.000899619481060654
764 0.000899437232874334
765 0.000899203470908105
766 0.000899090722668916
767 0.000898898113518953
768 0.000898662954568863
769 0.000898555910680443
770 0.000898326572496444
771 0.000898201891686767
772 0.0008980025886558
773 0.000897797581274062
774 0.000897679768968374
775 0.000897458638064563
776 0.000897301768418401
777 0.00089718634262681
778 0.000896998273674399
779 0.000896755198482424
780 0.000896644545719028
781 0.000896390236448497
782 0.000896253564860672
783 0.000896072189789265
784 0.000895911362022161
785 0.000895729928743094
786 0.000895548961125314
787 0.000895311997737736
788 0.000895108503755182
789 0.00089489488163963
790 0.000894778117071837
791 0.00089459033915773
792 0.000894350814633071
793 0.000894202967174351
794 0.00089399580610916
795 0.000893829681444913
796 0.000893650052603334
797 0.000893426302354783
798 0.000893239746801555
799 0.000892926589585841
800 0.00089277868391946
801 0.000892609008587897
802 0.000892376760020852
803 0.000892187294084579
804 0.000891988456714898
805 0.000891780713573098
806 0.000891575415153056
807 0.000891349860467017
808 0.000891122967004776
809 0.00089091929839924
810 0.000890699739102274
811 0.000890503404662013
812 0.000890287978108972
813 0.000890082155819982
814 0.000889842107426375
815 0.000889624294359237
816 0.000889422022737563
817 0.000889197806827724
818 0.000888976268470287
819 0.000888773298356682
820 0.000888543319888413
821 0.000888333132024854
822 0.000888091744855046
823 0.000887875736225396
824 0.000887607922777534
825 0.000887397560290992
826 0.000887167523615062
827 0.000886920315679163
828 0.000886690686456859
829 0.000886467343661934
830 0.000886223395355046
831 0.000885977991856635
832 0.000885759305674583
833 0.000885580899193883
834 0.000885285728145391
835 0.000885078392457217
836 0.000884808134287596
837 0.000884591834619641
838 0.000884339620824903
839 0.000884087756276131
840 0.00088386726565659
841 0.000883606146089733
842 0.00088336254702881
843 0.000883094500750303
844 0.000882801425177604
845 0.000882559863384813
846 0.000882221560459584
847 0.000881970743648708
848 0.000881678774021566
849 0.00088144902838394
850 0.000881198386196047
851 0.000880949373822659
852 0.000880679755937308
853 0.000880450708791614
854 0.000880146748386323
855 0.000879896688275039
856 0.000879616709426045
857 0.000879386672750115
858 0.000879101746249944
859 0.000878850056324154
860 0.000878541904967278
861 0.000878297607414424
862 0.000878035905770957
863 0.000877782702445984
864 0.000877544691320509
865 0.000877263024449348
866 0.000876979902386665
867 0.000876727630384266
868 0.000876458885613829
869 0.000876198755577207
870 0.000875927449669689
871 0.000875670404639095
872 0.000875384022947401
873 0.00087511376477778
874 0.000874826102517545
875 0.000874609628226608
876 0.000874339253641665
877 0.000874068064149469
878 0.000873833894729614
879 0.00087356602307409
880 0.000873356824740767
881 0.000873055134434253
882 0.000872806354891509
883 0.000872506003361195
884 0.00087220425484702
885 0.000871857453603297
886 0.000871598720550537
887 0.000871341791935265
888 0.000871402327902615
889 0.000871058844495565
890 0.000870660354848951
891 0.000870225892867893
892 0.000869767449330539
893 0.000869321811478585
894 0.000868972740136087
895 0.000868586532305926
896 0.000868269125930965
897 0.000867907714564353
898 0.000867568363901228
899 0.000867273600306362
900 0.000866966438479722
901 0.000866725109517574
902 0.000866424175910652
903 0.000866154150571674
904 0.000865867419634014
905 0.000865594483911991
906 0.000865339068695903
907 0.000865043257363141
908 0.000864767993334681
909 0.000864515779539943
910 0.000864182831719518
911 0.000863992841914296
912 0.00086364580783993
913 0.000863387249410152
914 0.000863102322909981
915 0.000862820481415838
916 0.00086251407628879
917 0.000862277986016124
918 0.000861986889503896
919 0.000861654465552419
920 0.000861460110172629
921 0.000861141656059772
922 0.000860931060742587
923 0.000860636122524738
924 0.000860340311191976
925 0.00086007866775617
926 0.000859773776028305
927 0.000859472085721791
928 0.000859199615661055
929 0.000858853338286281
930 0.000858674815390259
931 0.000858410145156085
932 0.000858105428051203
933 0.000857839011587203
934 0.000857616774737835
935 0.00085727300029248
936 0.000857031962368637
937 0.000856750295497477
938 0.000856464670505375
939 0.000856203841976821
940 0.000855958263855428
941 0.000855753023643047
942 0.000855410937219858
943 0.000855163554660976
944 0.00085491081699729
945 0.000854561105370522
946 0.000854334037285298
947 0.000854109937790781
948 0.000853806850500405
949 0.000853608013130724
950 0.000853396893944591
951 0.000853142119012773
952 0.000852821453008801
953 0.000852671510074288
954 0.000852472090627998
955 0.000852142518851906
956 0.000851913820952177
957 0.000851699034683406
958 0.000851415970828384
959 0.000851157528813928
960 0.000850917247589678
961 0.000850641401484609
962 0.000850374985020608
963 0.000850189360789955
964 0.000849883537739515
965 0.000849696516525
966 0.000849529809784144
967 0.000849365605972707
968 0.000849007279612124
969 0.000848856929223984
970 0.000848622468765825
971 0.000848358729854226
972 0.000848153489641845
973 0.000847925955895334
974 0.000847582414280623
975 0.000847435905598104
976 0.000847241259180009
977 0.000846947252284735
978 0.000846780254505575
979 0.000846553419250995
980 0.000846315349917859
981 0.000846178620122373
982 0.000846005510538816
983 0.000845752889290452
984 0.000845591421239078
985 0.000845408241730183
986 0.000845205271616578
987 0.000844986236188561
988 0.000844952301122248
989 0.000844587397295982
990 0.000844498048536479
991 0.000844277557916939
992 0.000844086811412126
993 0.000843837042339146
994 0.000843713933136314
995 0.000843460089527071
996 0.000843379879370332
997 0.000843143847305328
998 0.000842955545522273
999 0.000842755078338087
1000 0.000842641806229949
1001 0.000842377543449402
1002 0.000842286448460072
1003 0.000842053268570453
1004 0.000841862289234996
1005 0.000841810135170817
1006 0.000841559027321637
1007 0.000841351691633463
1008 0.000841341214254498
1009 0.000841177068650723
1010 0.000840908905956894
1011 0.000840864551719278
1012 0.000840679334942251
1013 0.000840487540699542
1014 0.000840428925585002
1015 0.000840182474348694
1016 0.000840026827063411
1017 0.000839969434309751
1018 0.000839889689814299
1019 0.000839679385535419
1020 0.000839400745462626
1021 0.000839382177218795
1022 0.00083926331717521
1023 0.000839065061882138
1024 0.0008388931164518
1025 0.000838762149214745
1026 0.000838707317598164
1027 0.000838551030028611
1028 0.00083845405606553
1029 0.00083830242510885
1030 0.000838094856590033
1031 0.000838005973491818
1032 0.000837936880998313
1033 0.00083778757834807
1034 0.000837773317471147
1035 0.000837618310470134
1036 0.000837464816868305
1037 0.000837305036839098
1038 0.000837249914184213
1039 0.00083705032011494
1040 0.000836910970974714
1041 0.000836825813166797
1042 0.000836702412925661
1043 0.000836588093079627
1044 0.000836502062156796
1045 0.000836346123833209
1046 0.000836264691315591
1047 0.00083622126840055
1048 0.000836056191474199
1049 0.00083599251229316
1050 0.000835930230095983
1051 0.000835812767036259
1052 0.000835702114272863
1053 0.000835665210615844
1054 0.000835551822092384
1055 0.000835531216580421
1056 0.000835407990962267
1057 0.000835307175293565
1058 0.000835216895211488
1059 0.000835136976093054
1060 0.000835107755847275
1061 0.000834958802442998
1062 0.000834893493447453
1063 0.000834797450806946
1064 0.00083474483108148
1065 0.000834652513731271
1066 0.000834622769616544
1067 0.000834579521324486
1068 0.000834511709399521
1069 0.000834428588859737
1070 0.000834333593957126
1071 0.000834271428175271
1072 0.000834205537103117
1073 0.000834151927847415
1074 0.00083407323108986
1075 0.000834028993267566
1076 0.000833923579193652
1077 0.000833816768135875
1078 0.000833755766507238
1079 0.000833685277029872
1080 0.000833605765365064
1081 0.000833526311907917
1082 0.000833449419587851
1083 0.000833418744150549
1084 0.000833360245451331
1085 0.000833336787763983
1086 0.000833274447359145
1087 0.000833203084766865
1088 0.000833108206279576
1089 0.00083304155850783
1090 0.00083297595847398
1091 0.000832900870591402
1092 0.00083280197577551
1093 0.000832733523566276
1094 0.000832665828056633
1095 0.000832606689073145
1096 0.000832534686196595
1097 0.000832491205073893
1098 0.000832392950542271
1099 0.000832346908282489
1100 0.000832264486234635
1101 0.000832224090117961
1102 0.000832157209515572
1103 0.000832094927318394
1104 0.000832039921078831
1105 0.000832014076877385
1106 0.000831944053061306
1107 0.000831879209727049
1108 0.000831859069876373
1109 0.000831837416626513
1110 0.000831784389447421
1111 0.000831740442663431
1112 0.000831690966151655
1113 0.000831667683087289
1114 0.000831615412607789
1115 0.000831559940706939
1116 0.000831528450362384
1117 0.000831510405987501
1118 0.000831435667350888
1119 0.000831418787129223
1120 0.000831360404845327
1121 0.000831307959742844
1122 0.000831264827866107
1123 0.000831184559501708
1124 0.000831133627798408
1125 0.000831058772746474
1126 0.000831012497656047
1127 0.000830946781206876
1128 0.000830894918181002
1129 0.000830842880532146
1130 0.000830803939606994
1131 0.000830750039312989
1132 0.000830722623504698
1133 0.000830667733680457
1134 0.000830665114335716
1135 0.000830613484140486
1136 0.000830565055366606
1137 0.000830547127407044
1138 0.0008305290248245
1139 0.000830489967484027
1140 0.000830474426038563
1141 0.000830426171887666
1142 0.000830397300887853
1143 0.000830373144708574
1144 0.000830320932436734
1145 0.000830282166134566
1146 0.000830205681268126
1147 0.000830158998724073
1148 0.000830123783089221
1149 0.000830062665045261
1150 0.000830029079224914
1151 0.000829975761007518
1152 0.000829895841889083
1153 0.000829837401397526
1154 0.000829791824799031
1155 0.000829729135148227
1156 0.00082969875074923
1157 0.000829645781777799
1158 0.000829607015475631
1159 0.000829568831250072
1160 0.000829508644528687
1161 0.000829475640784949
1162 0.000829454977065325
1163 0.000829397875349969
1164 0.000829376745969057
1165 0.000829336349852383
1166 0.000829300261102617
1167 0.000829257478471845
1168 0.000829231459647417
1169 0.00082917814143002
1170 0.000829166965559125
1171 0.000829125929158181
1172 0.000829092983622104
1173 0.000829062075354159
1174 0.000829046010039747
1175 0.000828997115604579
1176 0.000829000142402947
1177 0.000828957243356854
1178 0.0008289500256069
1179 0.000828918826300651
1180 0.000828870455734432
1181 0.000828841351903975
1182 0.000828806834761053
1183 0.000828770396765321
1184 0.000828742748126388
1185 0.000828714517410845
1186 0.000828647229354829
1187 0.000828606600407511
1188 0.000828545074909925
1189 0.000828512362204492
1190 0.000828470569103956
1191 0.00082844466669485
1192 0.00082837522495538
1193 0.000828332907985896
1194 0.000828309508506209
1195 0.000828295480459929
1196 0.000828217132948339
1197 0.000828158285003155
1198 0.00082816177746281
1199 0.000828160264063627
1200 0.000828125572297722
1201 0.00082809675950557
1202 0.000828085117973387
1203 0.000828059564810246
1204 0.000828066433314234
1205 0.000828008342068642
1206 0.000827981333713979
1207 0.000827936513815075
1208 0.000827928248327225
1209 0.000827903277240694
1210 0.000827877491246909
1211 0.000827859214041382
1212 0.000827809330075979
1213 0.000827760784886777
1214 0.000827774230856448
1215 0.000827753043267876
1216 0.000827722251415253
1217 0.000827705196570605
1218 0.000827601703349501
1219 0.000827546289656311
1220 0.00082749844295904
1221 0.000827466777991503
1222 0.000827411073260009
1223 0.000827305018901825
1224 0.000827274867333472
1225 0.000827227195259184
1226 0.000827144889626652
1227 0.000827117706649005
1228 0.000827092269901186
1229 0.000827028532512486
1230 0.000827006297186017
1231 0.00082699564518407
1232 0.000826995063107461
1233 0.00082699372433126
1234 0.000826992560178041
1235 0.000827002222649753
1236 0.000826982839498669
1237 0.000826956529635936
1238 0.000826988194603473
1239 0.000826978299301118
1240 0.000827019975986332
1241 0.00082701985957101
1242 0.000827048846986145
1243 0.00082705169916153
1244 0.000827044248580933
1245 0.000827020441647619
1246 0.000826988194603473
1247 0.00082700076745823
1248 0.000826982082799077
1249 0.000826922419946641
1250 0.000826882314868271
1251 0.000826819508802146
1252 0.000826743664219975
1253 0.000826682837214321
1254 0.000826597213745117
1255 0.000826508854515851
1256 0.00082648522220552
1257 0.0008264469797723
1258 0.000826320610940456
1259 0.000826266827061772
1260 0.000826195813715458
1261 0.000826118281111121
1262 0.000826073170173913
1263 0.000826033181510866
1264 0.000825995812192559
1265 0.000825994473416358
1266 0.000825957278721035
1267 0.000825953553430736
1268 0.000825969676952809
1269 0.000825956696644425
1270 0.000825932889711112
1271 0.000825955765321851
1272 0.000825957860797644
1273 0.000826009432785213
1274 0.000826041796244681
1275 0.000826047209557146
1276 0.000826100236736238
1277 0.000826118455734104
1278 0.000826120900455862
1279 0.000826159841381013
1280 0.000826166535262018
1281 0.00082616729196161
1282 0.000826179748401046
1283 0.000826167059130967
1284 0.000826138071715832
1285 0.000826121133286506
1286 0.000826108327601105
1287 0.000826032424811274
1288 0.000826003612019122
1289 0.000825955299660563
1290 0.000825888244435191
1291 0.00082583847688511
1292 0.000825772469397634
1293 0.000825687020551413
1294 0.000825637776870281
1295 0.000825552619062364
1296 0.000825452792923898
1297 0.000825379625894129
1298 0.000825314898975194
1299 0.000825262221042067
1300 0.000825228402391076
1301 0.000825195864308625
1302 0.000825132359750569
1303 0.00082510564243421
1304 0.00082506361650303
1305 0.000825046328827739
1306 0.000825048249680549
1307 0.000825021648779511
1308 0.000825016119051725
1309 0.000825024209916592
1310 0.000825004011858255
1311 0.000825028808321804
1312 0.000825055933091789
1313 0.000825068855192512
1314 0.000825107155833393
1315 0.000825145340058953
1316 0.000825156166683882
1317 0.000825206516310573
1318 0.000825255352538079
1319 0.000825305352918804
1320 0.000825340452138335
1321 0.000825364666525275
1322 0.00082537648268044
1323 0.000825373630505055
1324 0.000825406168587506
1325 0.000825407390948385
1326 0.000825386203359812
1327 0.000825363153126091
1328 0.000825344352051616
1329 0.000825280381832272
1330 0.000825194350909442
1331 0.000825177354272455
1332 0.000825107388664037
1333 0.000825010763946921
1334 0.000824954477138817
1335 0.000824861868750304
1336 0.000824741611722857
1337 0.000824657967314124
1338 0.000824551389086992
1339 0.000824465940240771
1340 0.000824371702037752
1341 0.000824302434921265
1342 0.000824206625111401
1343 0.000824138638563454
1344 0.000824088638182729
1345 0.000824019429273903
1346 0.000823975075036287
1347 0.000823950103949755
1348 0.000823913374915719
1349 0.000823888869490474
1350 0.00082389812450856
1351 0.000823891605250537
1352 0.000823897134978324
1353 0.000823941954877228
1354 0.000823972513899207
1355 0.000824027927592397
1356 0.000824087881483138
1357 0.000824150280095637
1358 0.000824212445877492
1359 0.000824290851596743
1360 0.000824416696559638
1361 0.000824500632006675
1362 0.00082461180863902
1363 0.000824674847535789
1364 0.000824796385131776
1365 0.000824868562631309
1366 0.000824915594421327
1367 0.00082494446542114
1368 0.000825024850200862
1369 0.000824999529868364
1370 0.000824972696136683
1371 0.000824895745608956
1372 0.000824857677798718
1373 0.000824790622573346
1374 0.000824687944259495
1375 0.000824583577923477
1376 0.000824467628262937
1377 0.000824329094029963
1378 0.000824198999907821
1379 0.000824018905404955
1380 0.00082388095324859
1381 0.000823735434096307
1382 0.000823580892756581
1383 0.000823454523924738
1384 0.000823330658022314
1385 0.000823213718831539
1386 0.000823124020826072
1387 0.000823022041004151
1388 0.000822936999611557
1389 0.000822881702333689
1390 0.000822804402559996
1391 0.000822733505629003
1392 0.000822686357423663
1393 0.000822642701677978
1394 0.000822620582766831
1395 0.000822620000690222
1396 0.000822612550109625
1397 0.000822626461740583
1398 0.000822646135929972
1399 0.000822705624159425
1400 0.000822800793685019
1401 0.00082289008423686
1402 0.000823004520498216
1403 0.000823158537968993
1404 0.000823320471681654
1405 0.000823502719867975
1406 0.000823747832328081
1407 0.000823944457806647
1408 0.000824136019218713
1409 0.000824391725473106
1410 0.000824633287265897
1411 0.000824720191303641
1412 0.000824796443339437
1413 0.000824899936560541
1414 0.000824924733024091
1415 0.000824963732156903
1416 0.000824922346509993
1417 0.000824889284558594
1418 0.00082481512799859
1419 0.000824658840429038
1420 0.000824488059151918
1421 0.000824319606181234
1422 0.000824109360110015
1423 0.00082385097630322
1424 0.000823607435449958
1425 0.000823359878268093
1426 0.000823083217255771
1427 0.000822848407551646
1428 0.000822602538391948
1429 0.000822385773062706
1430 0.000822199624963105
1431 0.000822033092845231
1432 0.00082189729437232
1433 0.000821780762635171
1434 0.000821691704913974
1435 0.000821615220047534
1436 0.000821537163574249
1437 0.000821453577373177
1438 0.000821382331196219
1439 0.000821317837107927
1440 0.000821258116047829
1441 0.000821205961983651
1442 0.000821165041998029
1443 0.000821127672679722
1444 0.000821102876216173
1445 0.000821091351099312
1446 0.000821073423139751
1447 0.00082107848720625
1448 0.000821098801679909
1449 0.000821141991764307
1450 0.000821215799078345
1451 0.000821312714833766
1452 0.000821444671601057
1453 0.000821591936983168
1454 0.000821768830064684
1455 0.000821971567347646
1456 0.000822205794975162
1457 0.000822505098767579
1458 0.000822693924419582
1459 0.000822965113911778
1460 0.000823344860691577
1461 0.00082356296479702
1462 0.0008239108719863
1463 0.000824338116217405
1464 0.000824522576294839
1465 0.000824927294161171
1466 0.000825084454845637
1467 0.000825493247248232
1468 0.000825712573714554
1469 0.000826091738417745
1470 0.000826323986984789
1471 0.000826684350613505
1472 0.000827201060019433
1473 0.000827963231131434
1474 0.000827820738777518
1475 0.000827861775178462
1476 0.000828688323963434
1477 0.000828069285489619
1478 0.000827809621114284
1479 0.000827947224024683
1480 0.000827253679744899
1481 0.000826436618808657
1482 0.000825707626063377
1483 0.000824847491458058
1484 0.00082401605322957
1485 0.00082325009861961
1486 0.000822546367999166
1487 0.000821958412416279
1488 0.000821546418592334
1489 0.000821239256765693
1490 0.000821095134597272
1491 0.00082105043111369
1492 0.000821117777377367
1493 0.000821294845081866
1494 0.000821562367491424
1495 0.000821998983155936
1496 0.00082252366701141
1497 0.00082327623385936
1498 0.000824207498226315
1499 0.000825287599582225
1500 0.000826588249765337
1501 0.000828084244858474
1502 0.000829754688311368
1503 0.000831537297926843
1504 0.000833637313917279
1505 0.000835035287309438
1506 0.000835092039778829
1507 0.000834169099107385
1508 0.000832080084364861
1509 0.000829305732622743
1510 0.000826662813778967
1511 0.000824593880679458
1512 0.000823171867523342
1513 0.000822242116555572
1514 0.000821646070107818
1515 0.000821271853055805
1516 0.00082103640306741
1517 0.000820894143544137
1518 0.000820801535155624
1519 0.000820752407889813
1520 0.000820736109744757
1521 0.000820708053652197
1522 0.000820715504232794
1523 0.00082073116209358
1524 0.000820764689706266
1525 0.000820811896119267
1526 0.000820843793917447
1527 0.000820898218080401
1528 0.000820974120870233
1529 0.000821038789581507
1530 0.000821133959107101
1531 0.000821239955257624
1532 0.00082137924619019
1533 0.000821517838630825
1534 0.000821706838905811
1535 0.000821882509626448
1536 0.000822109985165298
1537 0.000822356843855232
1538 0.000822620349936187
1539 0.000822906673420221
1540 0.000823249225504696
1541 0.000823573383968323
1542 0.000823924725409597
1543 0.000824253424070776
1544 0.00082454871153459
1545 0.000824772228952497
1546 0.000824979215394706
1547 0.000825046445243061
1548 0.000825034978333861
1549 0.000824898714199662
1550 0.000824652262963355
1551 0.00082435755757615
1552 0.000823976180981845
1553 0.000823543814476579
1554 0.000823069480247796
1555 0.000822608533781022
1556 0.000822193396743387
1557 0.000821787223685533
1558 0.000821445428300649
1559 0.000821118243038654
1560 0.000820852350443602
1561 0.000820606481283903
1562 0.000820414046756923
1563 0.000820242275949568
1564 0.000820087792817503
1565 0.000819955137558281
1566 0.000819859269540757
1567 0.000819757813587785
1568 0.000819683133158833
1569 0.000819640350528061
1570 0.000819583190605044
1571 0.000819563749246299
1572 0.000819586275611073
1573 0.000819604843854904
1574 0.000819653272628784
1575 0.00081970717292279
1576 0.000819763925392181
1577 0.000819821550976485
1578 0.000819883134681731
1579 0.000819919281639159
1580 0.000819959037471563
1581 0.000819983542896807
1582 0.000820004031993449
1583 0.000820011948235333
1584 0.000820005661807954
1585 0.000819926906842738
1586 0.000819932436570525
1587 0.000819989771116525
1588 0.000819981563836336
1589 0.000820034416392446
1590 0.000820045184809715
1591 0.000820052693597972
1592 0.000820056593511254
1593 0.000820079585537314
1594 0.000820110493805259
1595 0.000820233603008091
1596 0.000820169982034713
1597 0.000820252869743854
1598 0.000820178946014494
1599 0.000820279878098518
1600 0.00082004489377141
1601 0.000820303161162883
1602 0.000820023706182837
1603 0.000820237037260085
1604 0.000819986395072192
1605 0.000820271845441312
1606 0.000819941982626915
1607 0.000820401648525149
1608 0.000819809327367693
1609 0.000820532324723899
1610 0.00081957271322608
1611 0.000821205030661076
1612 0.000819062755908817
1613 0.000823750102426857
1614 0.000819588138256222
1615 0.000831260462291539
1616 0.000821976806037128
1617 0.000829516095109284
1618 0.000818958855234087
1619 0.00082300091162324
1620 0.000819961889646947
1621 0.000821867142803967
1622 0.000820773188024759
1623 0.000821666908450425
1624 0.000821204448584467
1625 0.000821696245111525
1626 0.000821608700789511
1627 0.000821984373033047
1628 0.000821850262582302
1629 0.000822202069684863
1630 0.00082201324403286
1631 0.000822416855953634
1632 0.000822166504804045
1633 0.000822756555862725
1634 0.000822335772681981
1635 0.000823126407340169
1636 0.000822752946987748
1637 0.000823499285615981
1638 0.000823115056846291
1639 0.000824064016342163
1640 0.000823463429696858
1641 0.00082471827045083
1642 0.000824021582957357
1643 0.000825434282887727
1644 0.00082480238052085
1645 0.000826629635412246
1646 0.000826547970063984
1647 0.000829302181955427
1648 0.00082987267524004
1649 0.000834218109957874
1650 0.000833994243294001
1651 0.000836974882986397
1652 0.000838961394038051
1653 0.000841516652144492
1654 0.000843742629513144
1655 0.000844411726575345
1656 0.000842154782731086
1657 0.000836990715470165
1658 0.00083052203990519
1659 0.000825094582978636
1660 0.000821669062133878
1661 0.000820282322820276
1662 0.000820352463051677
1663 0.000821385823655874
1664 0.000822865928057581
1665 0.000824434275273234
1666 0.000825851107947528
1667 0.000826704199425876
1668 0.000826853152830154
1669 0.000826387025881559
1670 0.000825773517135531
1671 0.000825044058728963
1672 0.000824250862933695
1673 0.000823517388198525
1674 0.000822891481220722
1675 0.000822440313640982
1676 0.000821968249510974
1677 0.000821636349428445
1678 0.000821391586214304
1679 0.000821215973701328
1680 0.000821131689008325
1681 0.000821010384242982
1682 0.0008209134102799
1683 0.000820857414510101
1684 0.000820865912828594
1685 0.00082080764696002
1686 0.000820762710645795
1687 0.000820758519694209
1688 0.000820744957309216
1689 0.000820821849629283
1690 0.000820792862214148
1691 0.000820791523437947
1692 0.000820790184661746
1693 0.000820826331619173
1694 0.000820847111754119
1695 0.00082089495845139
1696 0.000820910616312176
1697 0.000820947461761534
1698 0.000821019697468728
1699 0.000821005320176482
1700 0.000820974877569824
1701 0.000821003515738994
1702 0.000821008638013154
1703 0.000821027613710612
1704 0.000821049557998776
1705 0.000821063993498683
1706 0.000821071618702263
1707 0.000821117078885436
1708 0.000821123889181763
1709 0.000821141584310681
1710 0.000821162422653288
1711 0.000821172667201608
1712 0.00082116830162704
1713 0.000821202993392944
1714 0.000821204157546163
1715 0.000821207242552191
1716 0.000821200839709491
1717 0.000821178720798343
1718 0.000821203109808266
1719 0.000821185065433383
1720 0.000821176101453602
1721 0.000821173074655235
1722 0.000821171852294356
1723 0.000821168127004057
1724 0.000821135239675641
1725 0.000821137335151434
1726 0.00082112051313743
1727 0.000821100431494415
1728 0.000821106717921793
1729 0.000821085413917899
1730 0.000821062072645873
1731 0.000821054447442293
1732 0.000821044435724616
1733 0.000821035297121853
1734 0.00082104280591011
1735 0.00082104280591011
1736 0.000821030582301319
1737 0.000821024412289262
1738 0.000821006193291396
1739 0.000821024470496923
1740 0.000821047695353627
1741 0.00082105427281931
1742 0.000821065623313189
1743 0.000821072375401855
1744 0.000821083434857428
1745 0.000821111665572971
1746 0.000821104797068983
1747 0.0008211326203309
1748 0.000821152119897306
1749 0.000821171735879034
1750 0.000821202469523996
1751 0.000821236986666918
1752 0.000821252877358347
1753 0.000821276335045695
1754 0.000821285007987171
1755 0.000821297348011285
1756 0.000821326626464725
1757 0.000821339373942465
1758 0.0008213454275392
1759 0.000821328314486891
1760 0.00082132697571069
1761 0.000821300083771348
1762 0.000821305846329778
1763 0.000821280060335994
1764 0.000821237277705222
1765 0.000821201188955456
1766 0.000821173656731844
1767 0.000821120163891464
1768 0.000821062363684177
1769 0.00082098349230364
1770 0.000820914458017796
1771 0.000820830522570759
1772 0.000820772896986455
1773 0.000820690707769245
1774 0.000820599729195237
1775 0.00082052277866751
1776 0.000820466200821102
1777 0.000820358458440751
1778 0.00082030234625563
1779 0.000820218643639237
1780 0.000820172484964132
1781 0.000820124812889844
1782 0.000820051005575806
1783 0.000820009328890592
1784 0.000820200017187744
1785 0.000820118701085448
1786 0.000820076151285321
1787 0.000820024870336056
1788 0.000820028479211032
1789 0.000820019515231252
1790 0.000820045068394393
1791 0.000820127374026924
1792 0.000820193090476096
1793 0.000820274522993714
1794 0.000820371729787439
1795 0.00082049798220396
1796 0.000820665620267391
1797 0.000820852466858923
1798 0.000821050547529012
1799 0.00082131935050711
1800 0.000821622088551521
1801 0.000821951078251004
1802 0.000822366506326944
1803 0.000822821399196982
1804 0.000823333801236004
1805 0.000823889044113457
1806 0.000824441958684474
1807 0.000825068855192512
1808 0.000825642375275493
1809 0.000826091913040727
1810 0.000826358213089406
1811 0.000826417992357165
1812 0.000826063973363489
1813 0.000825366005301476
1814 0.000824273272883147
1815 0.000823093403596431
1816 0.00082185841165483
1817 0.000820778892375529
1818 0.000819916313048452
1819 0.000819329696241766
1820 0.000818993663415313
1821 0.000818902568425983
1822 0.000819026259705424
1823 0.000819384353235364
1824 0.000819957989733666
1825 0.000820759916678071
1826 0.000821858178824186
1827 0.000823291658889502
1828 0.000824945978820324
1829 0.000826701696496457
1830 0.000828551652375609
1831 0.000830153527203947
1832 0.00083125609671697
1833 0.000831771758385003
1834 0.000831070472486317
1835 0.000829580763820559
1836 0.00082751177251339
1837 0.000825439812615514
1838 0.000823459704406559
1839 0.000821906258352101
1840 0.000820792454760522
1841 0.000819992972537875
1842 0.000819449662230909
1843 0.000819058215711266
1844 0.0008188042556867
1845 0.000818613974843174
1846 0.000818482541944832
1847 0.000818374683149159
1848 0.000818342203274369
1849 0.000818294531200081
1850 0.000818264205008745
1851 0.000818280794192106
1852 0.000818270666059107
1853 0.000818302680272609
1854 0.000818347267340869
1855 0.000818374624941498
1856 0.000818444706965238
1857 0.000818510248791426
1858 0.000818622764199972
1859 0.00081872649025172
1860 0.000818874745164067
1861 0.000819035107269883
1862 0.000819243199657649
1863 0.000819485459942371
1864 0.000819743727333844
1865 0.000820028013549745
1866 0.000820429297164083
1867 0.000820755201857537
1868 0.000821202644146979
1869 0.000821558525785804
1870 0.000822042638901621
1871 0.000822489731945097
1872 0.000822908186819404
1873 0.000823583861347288
1874 0.000824155984446406
1875 0.000824601273052394
1876 0.000825030903797597
1877 0.000825158029329032
1878 0.000825042719952762
1879 0.000824458023998886
1880 0.000823762500658631
1881 0.000822693633381277
1882 0.00082167173968628
1883 0.000820620974991471
1884 0.000819860550109297
1885 0.000819170265458524
1886 0.000818685512058437
1887 0.000818334345240146
1888 0.000818117579910904
1889 0.000818009371869266
1890 0.000817961001303047
1891 0.000818019441794604
1892 0.000818115018773824
1893 0.00081827025860548
1894 0.000818484171759337
1895 0.000818755012005568
1896 0.000819100474473089
1897 0.000819480803329498
1898 0.000819973647594452
1899 0.000820526678580791
1900 0.000821219873614609
1901 0.000822033849544823
1902 0.000823033857159317
1903 0.000824263843242079
1904 0.000825764436740428
1905 0.000827477429993451
1906 0.000829314754810184
1907 0.000830617034807801
1908 0.000830889563076198
1909 0.000830583565402776
1910 0.000828670221380889
1911 0.000826487259473652
1912 0.000824507791548967
1913 0.000822441827040166
1914 0.000820868765003979
1915 0.000819830689579248
1916 0.000819154258351773
1917 0.000818671134766191
1918 0.000818365719169378
1919 0.000818162050563842
1920 0.000818031490780413
1921 0.000817938591353595
1922 0.00081786303780973
1923 0.000817820138763636
1924 0.000817808322608471
1925 0.000817812164314091
1926 0.00081782607594505
1927 0.000817845633719116
1928 0.000817880791146308
1929 0.000817934225779027
1930 0.000818011641968042
1931 0.000818115018773824
1932 0.000818266358692199
1933 0.00081844354281202
1934 0.00081867235712707
1935 0.000819011067505926
1936 0.000819418579339981
1937 0.000819923938252032
1938 0.000820594083052129
1939 0.000821414170786738
1940 0.000822492758743465
1941 0.000823452137410641
1942 0.000824300339445472
1943 0.000825354363769293
1944 0.00082623481284827
1945 0.000826757808681577
1946 0.000826768868137151
1947 0.00082611566176638
1948 0.000824912218376994
1949 0.000823371286969632
1950 0.000821823312435299
1951 0.000820508575998247
1952 0.000819486333057284
1953 0.00081877171760425
1954 0.000818330910988152
1955 0.000818067812360823
1956 0.000817959546111524
1957 0.000817930849734694
1958 0.000817992200609297
1959 0.000818094413261861
1960 0.0008182741003111
1961 0.000818346568848938
1962 0.000818525266367942
1963 0.000818785047158599
1964 0.00081911007873714
1965 0.0008194784168154
1966 0.000819903623778373
1967 0.000820312881842256
1968 0.000820795597974211
1969 0.000821295892819762
1970 0.000821847061160952
1971 0.000822351255919784
1972 0.000822968373540789
1973 0.000823492708150297
1974 0.000824050221126527
1975 0.000824533635750413
1976 0.000824943126644939
1977 0.000825230206828564
1978 0.000825353490654379
1979 0.00082529685460031
1980 0.000824997667223215
1981 0.000824485206976533
1982 0.000823799869976938
1983 0.000822890840936452
1984 0.000822113768663257
1985 0.00082120846491307
1986 0.000820400775410235
1987 0.00081965234130621
1988 0.00081903743557632
1989 0.000818515487480909
1990 0.000818113214336336
1991 0.000817780091892928
1992 0.000817564956378192
1993 0.000817398075014353
1994 0.000817306514363736
1995 0.000817261752672493
1996 0.000817279622424394
1997 0.000817362684756517
1998 0.000817481311969459
1999 0.000817675783764571
};
\addlegendentry{Test}
\end{groupplot}

\end{tikzpicture}

		\label{Fig:Layer}
		\caption{Increasing the number of layers. Shown are training- and validation error over 2000 epochs a model with two and a model with three convolutional layers in the encoder. Channels are chosen to reach a  maximum of 16 in the last layer of encoder. Kernel size and stride width are both \(3\times 3\).}
	\end{figure}
\end{center}
Next the batch size is increased from two to four. The initial batch size was chosen so small as the limited available data permits reasonable training time. The results for all three models are shown in \cref{Tab:Batch4}. The models with two and three convolutional layers benefit from increasing the batch size. The small model reaches the best training- and validation error with \num{6.0e-6} and \num{8.0e-6} respectively and \(\L2=0.03\). In \cref{Fig:Batch} it can be observed, that the increased batch size results in a stable training for both shallow models. The underfitting of the deep model with four convolutional layers in encoder and decoder is increased.
\begin{table}[htbp!]
	\centering
	\caption{Increasing the batch size to four. Summary of minimum training- and minimum validation error for the models with two, three and four convolutional layers in the encoder as well as the corresponding \(\L2\) and the epoch in which those values are reached.}
	\begin{tabular*}{15cm}{ @{\extracolsep{\fill}} c c c c c c c c c c @{} }
		\toprule
		Layer & Min. training error & Min. validation error & \(\L2\) & Epoch\\ [.5ex]
		\hline
		2   & \num{6.0e-6}             & \num{8.0e-6}             & 0.030   & 1999  \\
		\hline  
		3   & \num{1.0e-5}              & \num{1.3e-5}            & 0.038   & 1965  \\  
		\hline
		4    & \num{6.0e-3}            & \num{6.9e-3}             & 0.94   	& 109\\
		\hline
	\end{tabular*}\label{Tab:Batch4}
\end{table}
\begin{center}
	\begin{figure}[htbp!]
		% This file was created by tikzplotlib v0.9.6.
\begin{tikzpicture}

\begin{groupplot}[
group style={group size=2 by 2,horizontal sep=2.5cm},
legend cell align={left},
legend style={fill opacity=1, draw opacity=1, text opacity=1, draw=white},
log basis y={10},
tick align=outside,
tick pos=left,
title style={at={(0.43,0.85)},anchor=north},
x grid style={white!69.0196078431373!black},
xlabel={Epoch},
x label style={yshift=13pt},
xmin=-49.95, xmax=2048.95,
xtick style={color=black},
xtick = {0,500,1500,2000},
y grid style={white!69.0196078431373!black},
ylabel={MSE Loss},
ymode=log,
ytick style={color=black},
width=.45\textwidth,
height=.25\textwidth
]
\nextgroupplot[
title={2 Layer},
y grid style={white!69.0196078431373!black},
]
\addplot [semithick, black, dashed]
table {%
0 0.0340073617408052
1 0.0332358524901792
2 0.0324765683617443
3 0.0317228138446808
4 0.0309688663110137
5 0.0302053948398679
6 0.0294234940083697
7 0.0286121044773608
8 0.0277554992353544
9 0.0268289762316272
10 0.0258079520426691
11 0.0246811327524483
12 0.0234530753223225
13 0.0221448098309338
14 0.0208049673819914
15 0.0194812100962736
16 0.0182018360937946
17 0.0169796468690038
18 0.0158200662117451
19 0.014721826009918
20 0.0136790304677561
21 0.0126877853181213
22 0.0117368521168828
23 0.0108312364318408
24 0.00997602660208941
25 0.00916807624162175
26 0.00839779857778922
27 0.00766622024821118
28 0.00697363587096334
29 0.00631626788526773
30 0.00569439519313164
31 0.00511638610623777
32 0.0045952586515341
33 0.00413849292090163
34 0.00375459052156657
35 0.00343954785785172
36 0.00317993241696968
37 0.00296778731353697
38 0.00280000617567566
39 0.0026685543016356
40 0.00256502724187158
41 0.00248100823046116
42 0.00241139361241949
43 0.00235209371112433
44 0.00229988497358136
45 0.00225357278850424
46 0.00221161926583591
47 0.00217228300743955
48 0.00213454030654248
49 0.0020977227786716
50 0.00206140276713995
51 0.00202593227368197
52 0.00199109788036367
53 0.00195641284722115
54 0.00192149218560189
55 0.00188633010543526
56 0.00185133392619719
57 0.00181615229985255
58 0.00178069823823535
59 0.00174494601185415
60 0.00170903299863312
61 0.00167354724283086
62 0.00163836898127556
63 0.00160289505504352
64 0.00156724704459066
65 0.00153172237844501
66 0.00149676786651298
67 0.0014624783329964
68 0.00142909327109919
69 0.00139640839341837
70 0.00136494008199861
71 0.0013344912864568
72 0.00130456080484009
73 0.00127562903534795
74 0.00124728181251044
75 0.0012202292773793
76 0.00119456119438155
77 0.00116986350354864
78 0.00114613465348157
79 0.00112354719495045
80 0.00110195396582924
81 0.0010812631354753
82 0.00106144882533954
83 0.00104239477651902
84 0.00102402173908445
85 0.00100626690018402
86 0.000989204019425927
87 0.000972688552884904
88 0.00095671303425604
89 0.000941239748946998
90 0.000926215851379197
91 0.000911653858086225
92 0.000897522668331874
93 0.000883778403256485
94 0.000870373063587948
95 0.000857291972870655
96 0.000844522169813899
97 0.000832057680167253
98 0.00081986551758817
99 0.00080792339563196
100 0.000796236272142536
101 0.000784774311124892
102 0.000773553562634532
103 0.000762537725950807
104 0.000751759409281405
105 0.000741197182167141
106 0.000730833388107754
107 0.000720679545002767
108 0.000710705162277669
109 0.00070090873799078
110 0.000691298443745936
111 0.000681840015649193
112 0.000672513869224289
113 0.000663316122919255
114 0.000654278600009217
115 0.000645381946833723
116 0.000636625522053791
117 0.000627992463898774
118 0.000619507575787104
119 0.000611156613114616
120 0.000602933482984902
121 0.000594827782696594
122 0.000586835380949502
123 0.000578968324679296
124 0.000571214776658779
125 0.000563569923893725
126 0.000556054558615626
127 0.000548638777627675
128 0.000541324817587174
129 0.000534087312279219
130 0.000526948133197314
131 0.000519909003235952
132 0.000512971392428452
133 0.000506143657048264
134 0.000499428837092308
135 0.000492825345436376
136 0.000486302057785881
137 0.000479872444097751
138 0.00047354509927533
139 0.000467275092110087
140 0.000461102884768083
141 0.000455024050658892
142 0.000449025527997748
143 0.000443122708196908
144 0.000437311603013746
145 0.000431562961532905
146 0.000425917864408487
147 0.000420349342710846
148 0.000414868824159087
149 0.000409481243875631
150 0.000404176295865977
151 0.000398915082327256
152 0.00039374487914412
153 0.000388664743007472
154 0.000383685951886537
155 0.000378801510535709
156 0.000374000990278067
157 0.000369304985257224
158 0.000364657705334226
159 0.000360099677330528
160 0.000355647105035439
161 0.000351241946689784
162 0.000346918345577318
163 0.000342663159750289
164 0.000338508875462606
165 0.000334393984999126
166 0.00033036534003017
167 0.000326348044476399
168 0.000322381221674606
169 0.000318450419081628
170 0.000314599534645632
171 0.000310825313142082
172 0.000307089769470137
173 0.000303462692021306
174 0.000299865403008415
175 0.000296347722246626
176 0.000292912033867276
177 0.000289499524237291
178 0.000286207273668015
179 0.000282953852402068
180 0.000279770667469093
181 0.000276661454304161
182 0.000273586399583081
183 0.000270543299596193
184 0.000267564456698954
185 0.000264608156306911
186 0.000261707135113198
187 0.000258892258846233
188 0.000256152679529498
189 0.000253374227993763
190 0.000250562693101308
191 0.000247724211360323
192 0.000244972940463439
193 0.000242293927861237
194 0.000239650891244114
195 0.000237049029473191
196 0.000234456044289511
197 0.000231892660883082
198 0.000229378852570461
199 0.000226883007962142
200 0.000224389068424147
201 0.000221977092434877
202 0.000219591459617918
203 0.000217250963777227
204 0.000214982909095829
205 0.000212747640361055
206 0.000210551279132387
207 0.000208440323390624
208 0.000206296142058693
209 0.00020423593997565
210 0.000202136865026459
211 0.000200119869000703
212 0.000198120310648475
213 0.000196122109377139
214 0.000194222583145609
215 0.000192297411828357
216 0.000190407775754298
217 0.000188511489301546
218 0.000186621402083309
219 0.000184810578005956
220 0.000183018480983843
221 0.00018125981001138
222 0.000179509188167615
223 0.00017773956447209
224 0.000176045159836824
225 0.000174367292586797
226 0.000172736260377893
227 0.000171126610211625
228 0.000169518766011678
229 0.000167929558098412
230 0.000166308488602995
231 0.000164777018577922
232 0.000163250597346731
233 0.000161758304525861
234 0.000160313642361187
235 0.000158892253164483
236 0.000157524007091681
237 0.000156148119190735
238 0.000154830276528184
239 0.000153500823427355
240 0.00015222904985504
241 0.000150964973562095
242 0.000149711356031201
243 0.000148511599238094
244 0.000147299528300698
245 0.000146175667447623
246 0.00014502001717126
247 0.000143904823805524
248 0.000142788297087776
249 0.000141697480022213
250 0.000140627068064086
251 0.000139574390898067
252 0.00013855994391393
253 0.000137563415746597
254 0.000136569997487193
255 0.000135628564463766
256 0.000134679255819017
257 0.000133734998491369
258 0.000132808315846944
259 0.000131903342290032
260 0.000131025016399416
261 0.00013012968969367
262 0.000129286554926625
263 0.000128409663681925
264 0.000127630712115898
265 0.000126792382230664
266 0.000126027014774988
267 0.000125230762589773
268 0.000124497480851526
269 0.000123719915315484
270 0.000122968073605151
271 0.000122208447513261
272 0.000121409615394613
273 0.000120648023157116
274 0.000119846015425251
275 0.000119081064727578
276 0.000118307048570826
277 0.000117569364930237
278 0.00011679827410141
279 0.00011611132422687
280 0.000115357977807651
281 0.000114647410840996
282 0.000113984888336832
283 0.000113296225964538
284 0.00011264911148573
285 0.000112001244025883
286 0.000111353408623283
287 0.000110714996394701
288 0.000110062344125139
289 0.000109464271038306
290 0.000108827473287165
291 0.000108212244893935
292 0.000107622810911645
293 0.000106982962564339
294 0.000106407157803967
295 0.000105782637723817
296 0.000105156620245461
297 0.000104556455419735
298 0.000103962342678399
299 0.000103350335791008
300 0.000102778007618598
301 0.000102167822583266
302 0.000101621186539169
303 0.000101060759388649
304 0.000100503519895101
305 9.9968010743523e-05
306 9.94314717117639e-05
307 9.88936286658859e-05
308 9.8389002737953e-05
309 9.78636241333319e-05
310 9.73601812080949e-05
311 9.68561686190972e-05
312 9.6372832579128e-05
313 9.58609360759421e-05
314 9.53949725476244e-05
315 9.49246997682396e-05
316 9.44263374109688e-05
317 9.39550280782964e-05
318 9.34886821897241e-05
319 9.29948963535043e-05
320 9.25490902066173e-05
321 9.20612131571907e-05
322 9.16000408115458e-05
323 9.11497239552972e-05
324 9.0697384328875e-05
325 9.02240532028031e-05
326 8.97821176044644e-05
327 8.93158119323889e-05
328 8.88814877475497e-05
329 8.84281294963873e-05
330 8.80162444552024e-05
331 8.75355248828313e-05
332 8.71250448406524e-05
333 8.66659377325441e-05
334 8.62514117940805e-05
335 8.5783797931116e-05
336 8.53689365916122e-05
337 8.49084825902402e-05
338 8.44950610034978e-05
339 8.40252030411826e-05
340 8.35979788944563e-05
341 8.31268634207127e-05
342 8.27187796952344e-05
343 8.22394237656177e-05
344 8.18240444817597e-05
345 8.13538858475127e-05
346 8.09360282572058e-05
347 8.04595011443254e-05
348 8.00185230351858e-05
349 7.95260059178737e-05
350 7.90784112081333e-05
351 7.86357781987768e-05
352 7.81412121426683e-05
353 7.76941009816845e-05
354 7.71937706360859e-05
355 7.67402839967346e-05
356 7.62385880452854e-05
357 7.57881841995101e-05
358 7.52524326559367e-05
359 7.47592126497665e-05
360 7.42626159899729e-05
361 7.37327854389846e-05
362 7.3256596522242e-05
363 7.27058779697032e-05
364 7.21806440884265e-05
365 7.16369185740362e-05
366 7.10991273180639e-05
367 7.05380656027543e-05
368 6.99842262701367e-05
369 6.93780510898634e-05
370 6.87486722847908e-05
371 6.81285335977222e-05
372 6.74451881703675e-05
373 6.68050930698882e-05
374 6.61439545908138e-05
375 6.54963098201122e-05
376 6.48607120572464e-05
377 6.42033372759698e-05
378 6.35634993209067e-05
379 6.2922867442694e-05
380 6.23174096068269e-05
381 6.16925582601624e-05
382 6.10551705584328e-05
383 6.04255802496567e-05
384 5.97144648342152e-05
385 5.90509479732049e-05
386 5.83699250782388e-05
387 5.76985300533295e-05
388 5.7034003044798e-05
389 5.63612071209363e-05
390 5.56770584321242e-05
391 5.50070474396591e-05
392 5.43204916709783e-05
393 5.3709390576806e-05
394 5.30533445806114e-05
395 5.24668800929717e-05
396 5.1875249759803e-05
397 5.1295516223071e-05
398 5.07371545728219e-05
399 5.01598958040361e-05
400 4.9629109586391e-05
401 4.91050506603585e-05
402 4.8572253302126e-05
403 4.80918384839768e-05
404 4.75974406404234e-05
405 4.71343648289491e-05
406 4.66817826492161e-05
407 4.62554619033995e-05
408 4.58171588171297e-05
409 4.54082560343849e-05
410 4.49891360991828e-05
411 4.45661310148893e-05
412 4.41613256865203e-05
413 4.37526126995103e-05
414 4.33952486487144e-05
415 4.30218309435126e-05
416 4.26712880052804e-05
417 4.22976237977046e-05
418 4.19941591895956e-05
419 4.16243590404264e-05
420 4.13199566073885e-05
421 4.09863231523033e-05
422 4.06913160877487e-05
423 4.03699109066125e-05
424 4.01006185213326e-05
425 3.97776217733181e-05
426 3.95304396567653e-05
427 3.92153648047788e-05
428 3.89639842830469e-05
429 3.86881648988258e-05
430 3.84355335127928e-05
431 3.8151890147553e-05
432 3.79202464748651e-05
433 3.76105651054282e-05
434 3.74030190766206e-05
435 3.70969125742526e-05
436 3.6872241553354e-05
437 3.66022769799912e-05
438 3.6388509999874e-05
439 3.61172763150286e-05
440 3.59283184288017e-05
441 3.5658853410947e-05
442 3.54520585958973e-05
443 3.52066271189777e-05
444 3.50055375157865e-05
445 3.47743069033513e-05
446 3.45753321999709e-05
447 3.43433083487277e-05
448 3.41657246685223e-05
449 3.39253404795237e-05
450 3.37427920098499e-05
451 3.35297707736437e-05
452 3.33363310538459e-05
453 3.31336396008908e-05
454 3.29405808514593e-05
455 3.27491341285402e-05
456 3.25503124631865e-05
457 3.23642690780446e-05
458 3.21639544287322e-05
459 3.19807331146116e-05
460 3.17799391220674e-05
461 3.16147208541118e-05
462 3.14295549959098e-05
463 3.12377644586803e-05
464 3.10911733873986e-05
465 3.08864795044439e-05
466 3.07429778674262e-05
467 3.05510061804348e-05
468 3.03958241509861e-05
469 3.02190328540508e-05
470 3.00551266723126e-05
471 2.98941738432479e-05
472 2.9721159169771e-05
473 2.95722774941165e-05
474 2.93981298831181e-05
475 2.92472783784969e-05
476 2.90858255038717e-05
477 2.89373704411844e-05
478 2.87765809334761e-05
479 2.86435863188217e-05
480 2.84867759321461e-05
481 2.83467300599138e-05
482 2.81989439403496e-05
483 2.80645924555234e-05
484 2.79199555195397e-05
485 2.77891105564532e-05
486 2.765226123147e-05
487 2.75173053432276e-05
488 2.73882150527704e-05
489 2.72483148520131e-05
490 2.712105149838e-05
491 2.69795117531224e-05
492 2.68400612168929e-05
493 2.67115982000021e-05
494 2.65824270639925e-05
495 2.64493419130929e-05
496 2.63323033509177e-05
497 2.61976148618892e-05
498 2.60888729529674e-05
499 2.59723073507967e-05
500 2.58409504884094e-05
501 2.57458738776029e-05
502 2.56236653100039e-05
503 2.55059369110855e-05
504 2.54055144085497e-05
505 2.52905363896661e-05
506 2.51791161415271e-05
507 2.50827872068093e-05
508 2.49582784473956e-05
509 2.48586483699853e-05
510 2.47504421475586e-05
511 2.46505154459786e-05
512 2.45448396571923e-05
513 2.44529916840763e-05
514 2.43464178200092e-05
515 2.42474166793905e-05
516 2.41544383614567e-05
517 2.40629630468003e-05
518 2.39657306023711e-05
519 2.38759582259496e-05
520 2.37840743808615e-05
521 2.3687162390762e-05
522 2.36093900629619e-05
523 2.35141318125631e-05
524 2.34283597451146e-05
525 2.33400984398502e-05
526 2.32571624592115e-05
527 2.31721019408049e-05
528 2.30826271514495e-05
529 2.30036048977089e-05
530 2.29223568881665e-05
531 2.28345140230735e-05
532 2.27601942145439e-05
533 2.26674199570454e-05
534 2.25919086366266e-05
535 2.25087858664041e-05
536 2.24301157170093e-05
537 2.23557694740695e-05
538 2.22738190437388e-05
539 2.22011895690821e-05
540 2.21241969482477e-05
541 2.20448144847563e-05
542 2.19698721529715e-05
543 2.19022601953833e-05
544 2.18201657844475e-05
545 2.17545768389504e-05
546 2.16772796401976e-05
547 2.16090408058367e-05
548 2.15301279128388e-05
549 2.1468446051276e-05
550 2.13898093429643e-05
551 2.13269066552391e-05
552 2.1256511308243e-05
553 2.11830649267664e-05
554 2.11131714780155e-05
555 2.10502932677104e-05
556 2.09800761769574e-05
557 2.09141899866205e-05
558 2.08467033366855e-05
559 2.07811577424e-05
560 2.07179040805361e-05
561 2.06467003138755e-05
562 2.05894317807931e-05
563 2.05206900798505e-05
564 2.04598222690588e-05
565 2.03900504157639e-05
566 2.0335281226691e-05
567 2.02664858977575e-05
568 2.02096015941011e-05
569 2.01475855261535e-05
570 2.00812850195686e-05
571 2.0023894115595e-05
572 1.99636709098949e-05
573 1.99014708645029e-05
574 1.98448694196385e-05
575 1.97790032831491e-05
576 1.9726517693941e-05
577 1.96592237813875e-05
578 1.96112302575102e-05
579 1.95415433646984e-05
580 1.94906420304397e-05
581 1.94326745447948e-05
582 1.93659185508954e-05
583 1.93157287378676e-05
584 1.92524809370376e-05
585 1.91966319341708e-05
586 1.91469657736532e-05
587 1.90799434680589e-05
588 1.90357316914858e-05
589 1.89718133517536e-05
590 1.89200826865576e-05
591 1.88644279032202e-05
592 1.88076158721961e-05
593 1.87559241134894e-05
594 1.86984195314488e-05
595 1.86446648050009e-05
596 1.85871364084544e-05
597 1.85284097601324e-05
598 1.84710581948022e-05
599 1.84136297914361e-05
600 1.83530023641909e-05
601 1.82962858899516e-05
602 1.82412986942659e-05
603 1.81782393983698e-05
604 1.81328156393379e-05
605 1.8073854860523e-05
606 1.80201374320732e-05
607 1.79648541652755e-05
608 1.79178754847742e-05
609 1.7864745215368e-05
610 1.78140544085936e-05
611 1.7760164276015e-05
612 1.7714530844315e-05
613 1.76645978092749e-05
614 1.76139044917312e-05
615 1.75659678833817e-05
616 1.75110181128146e-05
617 1.74694441006928e-05
618 1.74183479021983e-05
619 1.73740921202148e-05
620 1.73202243377024e-05
621 1.72764168221984e-05
622 1.72247713627893e-05
623 1.71825738989329e-05
624 1.71310573759409e-05
625 1.70823050849589e-05
626 1.70439873879435e-05
627 1.69869955637392e-05
628 1.6936617920571e-05
629 1.6887751304373e-05
630 1.6836104847151e-05
631 1.67854033942261e-05
632 1.67419172465433e-05
633 1.66887468568411e-05
634 1.6642737823036e-05
635 1.65989251521559e-05
636 1.65538334060966e-05
637 1.65043203653181e-05
638 1.64595389231526e-05
639 1.64145910873104e-05
640 1.63668895883373e-05
641 1.63231954003473e-05
642 1.62773458156629e-05
643 1.62318666757089e-05
644 1.61858315448904e-05
645 1.61433964731006e-05
646 1.61005664521818e-05
647 1.60531398176178e-05
648 1.60104174006248e-05
649 1.5968523356058e-05
650 1.59177481079409e-05
651 1.58827145747553e-05
652 1.58376295504858e-05
653 1.57905630172595e-05
654 1.57534138107152e-05
655 1.57048222614864e-05
656 1.56599339757868e-05
657 1.56243397402833e-05
658 1.55747478577384e-05
659 1.55342866299812e-05
660 1.54961087479033e-05
661 1.54487495951372e-05
662 1.54082643502584e-05
663 1.53659199340628e-05
664 1.53223916839473e-05
665 1.5275712267393e-05
666 1.52333230157842e-05
667 1.51939191128103e-05
668 1.51459967016598e-05
669 1.51071316148998e-05
670 1.50660627944998e-05
671 1.5018726430005e-05
672 1.49848133538799e-05
673 1.49400154173529e-05
674 1.48957601123767e-05
675 1.48548524082082e-05
676 1.48220641191799e-05
677 1.47692852084935e-05
678 1.47318010156328e-05
679 1.46932594400351e-05
680 1.46493946529547e-05
681 1.46115305656447e-05
682 1.45713597627939e-05
683 1.45311098432854e-05
684 1.44927353269986e-05
685 1.4455267464214e-05
686 1.44093774544496e-05
687 1.43764940661872e-05
688 1.43355048545235e-05
689 1.42936867701948e-05
690 1.42542863200701e-05
691 1.42209775138014e-05
692 1.41836570447529e-05
693 1.41389826769089e-05
694 1.41032028745935e-05
695 1.40701788099995e-05
696 1.40262033836713e-05
697 1.39971350012202e-05
698 1.39563644960816e-05
699 1.39215603456866e-05
700 1.38810009693846e-05
701 1.38494590661287e-05
702 1.38160327109726e-05
703 1.37704903309666e-05
704 1.37381945006509e-05
705 1.37063078160526e-05
706 1.3665088886089e-05
707 1.36343549314022e-05
708 1.3596067723165e-05
709 1.35656529270667e-05
710 1.35256828015795e-05
711 1.34949955010177e-05
712 1.34597154423932e-05
713 1.34249667731734e-05
714 1.3386685876221e-05
715 1.33521514948665e-05
716 1.33187519671119e-05
717 1.32867691677863e-05
718 1.3239681320748e-05
719 1.32068180182521e-05
720 1.31675221794747e-05
721 1.31319952019759e-05
722 1.30876342079589e-05
723 1.30554664633797e-05
724 1.30212800815155e-05
725 1.29811795601953e-05
726 1.2947071863903e-05
727 1.29161351986318e-05
728 1.28832186932215e-05
729 1.28399333630402e-05
730 1.28079549657767e-05
731 1.27785891866766e-05
732 1.2742845655403e-05
733 1.27100778471723e-05
734 1.26775219280173e-05
735 1.26380726641617e-05
736 1.26139023781946e-05
737 1.25835185121881e-05
738 1.2551956793172e-05
739 1.25186349724804e-05
740 1.24838978508846e-05
741 1.2457920665801e-05
742 1.24224239849702e-05
743 1.23936902132193e-05
744 1.23576704091494e-05
745 1.23275589471727e-05
746 1.22987395690388e-05
747 1.22634118899928e-05
748 1.223836258675e-05
749 1.22085093745145e-05
750 1.21822085037832e-05
751 1.21511324855939e-05
752 1.21197120030447e-05
753 1.20921651483585e-05
754 1.20626572529536e-05
755 1.20383201732688e-05
756 1.2004793592521e-05
757 1.19798822310474e-05
758 1.19495198715319e-05
759 1.19233491209791e-05
760 1.1902666295513e-05
761 1.18771162949671e-05
762 1.18425682710671e-05
763 1.18142781152658e-05
764 1.1789767656617e-05
765 1.1770293362981e-05
766 1.17426784415481e-05
767 1.17114684436148e-05
768 1.16897884292477e-05
769 1.16642404539152e-05
770 1.16438182122119e-05
771 1.16109001549458e-05
772 1.16001469625954e-05
773 1.15684080011363e-05
774 1.15408193167993e-05
775 1.1511098335637e-05
776 1.14936913005736e-05
777 1.14774438642676e-05
778 1.14529903368771e-05
779 1.14176592533738e-05
780 1.13959183492413e-05
781 1.1376955747551e-05
782 1.1360244829714e-05
783 1.13318457607425e-05
784 1.13036858765403e-05
785 1.12815235645408e-05
786 1.12615698992513e-05
787 1.12392161888408e-05
788 1.12184038044832e-05
789 1.11932826849029e-05
790 1.11767929369955e-05
791 1.11432317211335e-05
792 1.11288752378813e-05
793 1.11047208404824e-05
794 1.10898418888353e-05
795 1.10629541000482e-05
796 1.10407242445848e-05
797 1.10211914296915e-05
798 1.10014550608953e-05
799 1.09724897520286e-05
800 1.09537091655676e-05
801 1.09302822660848e-05
802 1.09182758225301e-05
803 1.08913823468337e-05
804 1.08702401929589e-05
805 1.0843294661278e-05
806 1.08256189411404e-05
807 1.0807226762867e-05
808 1.07972565289924e-05
809 1.07736630548874e-05
810 1.07576784563468e-05
811 1.0723251448172e-05
812 1.07094479156467e-05
813 1.06929186231078e-05
814 1.0676650190021e-05
815 1.06559282754964e-05
816 1.06320832347906e-05
817 1.06201914994164e-05
818 1.05929476102995e-05
819 1.05733289909871e-05
820 1.05514755471686e-05
821 1.05387445737889e-05
822 1.05277026670875e-05
823 1.05083461530431e-05
824 1.0481167498258e-05
825 1.04613638575168e-05
826 1.04406200332544e-05
827 1.04365498929254e-05
828 1.04200442524946e-05
829 1.04042471239518e-05
830 1.03754231467607e-05
831 1.03554834577635e-05
832 1.03413002320585e-05
833 1.03357438425086e-05
834 1.03277523668954e-05
835 1.03014746307001e-05
836 1.02722525979065e-05
837 1.02594773088091e-05
838 1.02525683507299e-05
839 1.02480301329907e-05
840 1.02344518246728e-05
841 1.01998309953583e-05
842 1.0175785341876e-05
843 1.01605569758634e-05
844 1.01619777558432e-05
845 1.01551630926899e-05
846 1.014118734248e-05
847 1.00997427150279e-05
848 1.00654248760684e-05
849 1.00636554774824e-05
850 1.00578431896994e-05
851 1.00622469797959e-05
852 1.00272467060541e-05
853 9.98998402004792e-06
854 9.98312258470441e-06
855 9.99616438798756e-06
856 1.0017469137133e-05
857 1.00298663085058e-05
858 9.96278125530575e-06
859 9.87899613200116e-06
860 9.92139584686871e-06
861 9.95682123961818e-06
862 9.95479962652424e-06
863 9.97864287503175e-06
864 9.9206204657136e-06
865 9.79504881790105e-06
866 9.7859496314967e-06
867 9.81679810707278e-06
868 9.82206950378081e-06
869 9.8398453208734e-06
870 9.80002809725794e-06
871 9.70007703017528e-06
872 9.67810927808566e-06
873 9.69101317360027e-06
874 9.70050902068809e-06
875 9.72437376275254e-06
876 9.71436377754376e-06
877 9.6258626584092e-06
878 9.58028489422835e-06
879 9.58077116488404e-06
880 9.58765365727948e-06
881 9.60471809510755e-06
882 9.60795182800706e-06
883 9.557930789654e-06
884 9.50019840553762e-06
885 9.49124760993292e-06
886 9.49631575367094e-06
887 9.50796009875632e-06
888 9.53681940210616e-06
889 9.49651092078951e-06
890 9.43542956047261e-06
891 9.41062604950527e-06
892 9.41781777339529e-06
893 9.42849235954085e-06
894 9.46513890288936e-06
895 9.44567353900611e-06
896 9.37020134290922e-06
897 9.33728978461112e-06
898 9.33129464312371e-06
899 9.35315373083029e-06
900 9.39456921808524e-06
901 9.37310153035359e-06
902 9.30145423724826e-06
903 9.28875000937024e-06
904 9.30321432200329e-06
905 9.3070502571041e-06
906 9.35517133014707e-06
907 9.3691825414563e-06
908 9.27040117321143e-06
909 9.22750457471588e-06
910 9.28668048208806e-06
911 9.2886615751693e-06
912 9.29232628765764e-06
913 9.32385539270664e-06
914 9.23934899663831e-06
915 9.16379897161756e-06
916 9.22511427483078e-06
917 9.2823739569689e-06
918 9.25353025078846e-06
919 9.26537307389258e-06
920 9.21384159525279e-06
921 9.11893152676785e-06
922 9.10213039309604e-06
923 9.1590116784758e-06
924 9.16221578241977e-06
925 9.16310286254787e-06
926 9.1452833769351e-06
927 9.09266369993056e-06
928 9.03160848392259e-06
929 9.02059794123666e-06
930 9.03159999655206e-06
931 9.03724725437316e-06
932 9.04774906306666e-06
933 9.04579613116713e-06
934 9.01009182992588e-06
935 8.96060479370353e-06
936 8.942081118124e-06
937 8.93361623170719e-06
938 8.95780152025683e-06
939 8.97967480949224e-06
940 8.98089632718133e-06
941 8.92069530700201e-06
942 8.88397059292667e-06
943 8.87551354794569e-06
944 8.88768836213721e-06
945 8.92331566016563e-06
946 8.92410903146831e-06
947 8.87388516897236e-06
948 8.8338150599776e-06
949 8.82656880997479e-06
950 8.83262049197059e-06
951 8.88233454469922e-06
952 8.90953493131938e-06
953 8.86017029851777e-06
954 8.80467595582024e-06
955 8.83798596849547e-06
956 8.85173316698962e-06
957 8.84663780643946e-06
958 8.93575389535745e-06
959 8.91267581967087e-06
960 8.78463867338974e-06
961 8.7596713232857e-06
962 8.8057957070703e-06
963 8.82403307393871e-06
964 8.84318142947704e-06
965 8.95303200238416e-06
966 8.86736127770705e-06
967 8.71759661144611e-06
968 8.74360672031317e-06
969 8.79946330892452e-06
970 8.78759072225031e-06
971 8.78486144495483e-06
972 8.76439766815326e-06
973 8.67885430904519e-06
974 8.69674018641811e-06
975 8.79421719980535e-06
976 8.78710124152179e-06
977 8.77636587393038e-06
978 8.76009767022609e-06
979 8.67757356283819e-06
980 8.61330221998274e-06
981 8.6357488670516e-06
982 8.64863197266256e-06
983 8.65327163846949e-06
984 8.66643275226897e-06
985 8.65812062937621e-06
986 8.6065329239042e-06
987 8.56308354833628e-06
988 8.53934148538349e-06
989 8.54025093270649e-06
990 8.55761900170338e-06
991 8.56816455282843e-06
992 8.56881348123683e-06
993 8.53840390806215e-06
994 8.51865714906336e-06
995 8.50206407592043e-06
996 8.50963996837546e-06
997 8.50204154822717e-06
998 8.49760366616534e-06
999 8.48655873694412e-06
1000 8.47440283446854e-06
1001 8.4597770131957e-06
1002 8.45003341705397e-06
1003 8.44404144672434e-06
1004 8.44265386346424e-06
1005 8.44154982770906e-06
1006 8.42887587526597e-06
1007 8.42178067207711e-06
1008 8.40501861350262e-06
1009 8.40320220321655e-06
1010 8.39301340589682e-06
1011 8.40149989010663e-06
1012 8.39323094921351e-06
1013 8.39377267831953e-06
1014 8.37408072865142e-06
1015 8.35972806936813e-06
1016 8.35422872792757e-06
1017 8.35951208641439e-06
1018 8.36531796866158e-06
1019 8.36515520259012e-06
1020 8.34830151487387e-06
1021 8.32705012464297e-06
1022 8.31583469074887e-06
1023 8.31148711059881e-06
1024 8.32335219463215e-06
1025 8.34375980057062e-06
1026 8.36522657752387e-06
1027 8.32830438920323e-06
1028 8.29952869546569e-06
1029 8.28546245179584e-06
1030 8.29156929858452e-06
1031 8.30262761199185e-06
1032 8.38578599429707e-06
1033 8.45040445963431e-06
1034 8.36531944182779e-06
1035 8.30749160909972e-06
1036 8.45568291482635e-06
1037 8.73871442672643e-06
1038 8.70110908097587e-06
1039 8.61891872633302e-06
1040 8.79576817595321e-06
1041 8.46244517683012e-06
1042 8.28735341882464e-06
1043 8.34175010031024e-06
1044 8.42842612747125e-06
1045 8.49813683350496e-06
1046 8.47757131055486e-06
1047 8.44179167192038e-06
1048 8.39783322591214e-06
1049 8.37355008001181e-06
1050 8.37363872306478e-06
1051 8.36612772146285e-06
1052 8.32456090876267e-06
1053 8.31951171063449e-06
1054 8.3124801620037e-06
1055 8.35800949164356e-06
1056 8.49370066713373e-06
1057 8.78340838050828e-06
1058 9.0253628701048e-06
1059 8.93684026224105e-06
1060 8.66831874566081e-06
1061 8.5819757807664e-06
1062 8.9466373594969e-06
1063 9.41462871028014e-06
1064 9.39103696658526e-06
1065 8.99061782847952e-06
1066 8.54067250211955e-06
1067 8.49428094969358e-06
1068 8.89710026571121e-06
1069 8.94402603992053e-06
1070 8.47773824745254e-06
1071 8.38754981867729e-06
1072 8.43789095037306e-06
1073 8.31810924396045e-06
1074 8.20575536200196e-06
1075 8.1978968539129e-06
1076 8.21311549416281e-06
1077 8.11829569125067e-06
1078 8.05388466041529e-06
1079 8.06303940611292e-06
1080 8.06349962172506e-06
1081 8.05398539463487e-06
1082 8.06737567134782e-06
1083 8.07220338050821e-06
1084 8.03868569163382e-06
1085 8.01440332286185e-06
1086 7.98725311271231e-06
1087 7.98413224047112e-06
1088 7.98977453730573e-06
1089 7.98654936879153e-06
1090 7.99932933069625e-06
1091 7.98853668297801e-06
1092 7.98999097065267e-06
1093 7.96924517154629e-06
1094 7.95772388013233e-06
1095 7.9457140731154e-06
1096 7.94074606525613e-06
1097 7.93987000143395e-06
1098 7.94200591032468e-06
1099 7.93554848695782e-06
1100 7.9364454756408e-06
1101 7.93290820510703e-06
1102 7.92866290470588e-06
1103 7.92655497684758e-06
1104 7.91203633184606e-06
1105 7.91462768015214e-06
1106 7.90154029402246e-06
1107 7.89871625345856e-06
1108 7.89639075352705e-06
1109 7.8865344978335e-06
1110 7.88773366773923e-06
1111 7.88121699821925e-06
1112 7.87966929736289e-06
1113 7.87645208213256e-06
1114 7.86915039643898e-06
1115 7.86595073438631e-06
1116 7.86356581937431e-06
1117 7.85685063598926e-06
1118 7.85672677260443e-06
1119 7.84555068583659e-06
1120 7.84547389953011e-06
1121 7.83856965075325e-06
1122 7.83564028975642e-06
1123 7.82915751519619e-06
1124 7.82768186097238e-06
1125 7.82202631748746e-06
1126 7.8155881286443e-06
1127 7.81697397658071e-06
1128 7.80747985191371e-06
1129 7.80811448042318e-06
1130 7.80220938442766e-06
1131 7.79584046785992e-06
1132 7.79894119707497e-06
1133 7.78632453621309e-06
1134 7.7889727913838e-06
1135 7.7792996357115e-06
1136 7.77997038220457e-06
1137 7.77349566351282e-06
1138 7.77153571465683e-06
1139 7.76949661718263e-06
1140 7.7594550408483e-06
1141 7.76288468641584e-06
1142 7.75143427576147e-06
1143 7.75120381588806e-06
1144 7.74408649550223e-06
1145 7.74280468788385e-06
1146 7.73301008322874e-06
1147 7.73324099690581e-06
1148 7.72843137230445e-06
1149 7.72125282984953e-06
1150 7.72103089344972e-06
1151 7.7071694789578e-06
1152 7.70899439777456e-06
1153 7.70243172847074e-06
1154 7.6989906462703e-06
1155 7.69654446423559e-06
1156 7.69055711652422e-06
1157 7.68479939837821e-06
1158 7.67675483693761e-06
1159 7.67952974334643e-06
1160 7.67293808888805e-06
1161 7.67602327988309e-06
1162 7.67432941650714e-06
1163 7.66118855251297e-06
1164 7.65298205180134e-06
1165 7.6496997738762e-06
1166 7.64922676240939e-06
1167 7.65938276224912e-06
1168 7.6808345046038e-06
1169 7.67987433671416e-06
1170 7.64513329364405e-06
1171 7.64850137011108e-06
1172 7.71736270380749e-06
1173 7.75802597991783e-06
1174 7.69094590934508e-06
1175 7.75660615939464e-06
1176 7.77839202823144e-06
1177 7.67688225379853e-06
1178 7.65938754995754e-06
1179 7.78255997192713e-06
1180 7.90202142760075e-06
1181 7.80173541731341e-06
1182 7.85890424262817e-06
1183 7.98663115939013e-06
1184 7.80576506165254e-06
1185 7.62475537517682e-06
1186 7.56519068565581e-06
1187 7.57048768443638e-06
1188 7.58265215333211e-06
1189 7.61568017619216e-06
1190 7.63482294904492e-06
1191 7.62535666375824e-06
1192 7.61663064145768e-06
1193 7.62783055410689e-06
1194 7.62848603756122e-06
1195 7.58297533734006e-06
1196 7.54840031507407e-06
1197 7.51957392389115e-06
1198 7.52239319389957e-06
1199 7.52816599435598e-06
1200 7.54986278183993e-06
1201 7.55044520226659e-06
1202 7.58729414351844e-06
1203 7.66336220514585e-06
1204 7.89493147723697e-06
1205 8.30890212299984e-06
1206 8.80866049279838e-06
1207 8.77619419335481e-06
1208 8.57818806262057e-06
1209 8.7597717769337e-06
1210 9.72318800728216e-06
1211 1.02266356289292e-05
1212 9.72999147391511e-06
1213 8.56889198860584e-06
1214 7.9707258100761e-06
1215 8.29880389546268e-06
1216 8.37655554766781e-06
1217 7.77790637269613e-06
1218 7.59246934298186e-06
1219 7.63919992017572e-06
1220 7.56092744412084e-06
1221 7.48394124159418e-06
1222 7.46636762092787e-06
1223 7.45729761293668e-06
1224 7.40350533804529e-06
1225 7.38560062075316e-06
1226 7.38823581127729e-06
1227 7.38728962223698e-06
1228 7.38432238621585e-06
1229 7.3829373999322e-06
1230 7.37624246603674e-06
1231 7.37309522670995e-06
1232 7.35859199604491e-06
1233 7.34571433785858e-06
1234 7.34519550865675e-06
1235 7.34407670064524e-06
1236 7.34392235761036e-06
1237 7.34283158768112e-06
1238 7.3381460033127e-06
1239 7.3359883982875e-06
1240 7.33364325355285e-06
1241 7.3244080844026e-06
1242 7.31768532304111e-06
1243 7.31186311757337e-06
1244 7.30792069597586e-06
1245 7.30369554335919e-06
1246 7.30657915386929e-06
1247 7.29859487883303e-06
1248 7.30104488740058e-06
1249 7.30381024431626e-06
1250 7.29559647927314e-06
1251 7.29620433431027e-06
1252 7.28798036800371e-06
1253 7.28318628956046e-06
1254 7.28017898783684e-06
1255 7.2685332708558e-06
1256 7.26904140541801e-06
1257 7.26222352483846e-06
1258 7.26014673056719e-06
1259 7.25416661882111e-06
1260 7.25780794335468e-06
1261 7.2544860071494e-06
1262 7.25169614069349e-06
1263 7.25359606488021e-06
1264 7.24942600366374e-06
1265 7.25399001445977e-06
1266 7.25827757854132e-06
1267 7.24490090176985e-06
1268 7.24088835154374e-06
1269 7.23311093020831e-06
1270 7.22734608130798e-06
1271 7.21702938695501e-06
1272 7.21036939822783e-06
1273 7.20495463153725e-06
1274 7.20390440418772e-06
1275 7.20375375953819e-06
1276 7.20332970798043e-06
1277 7.19947937088267e-06
1278 7.19811232861162e-06
1279 7.19271024569867e-06
1280 7.18991232051547e-06
1281 7.19593830778367e-06
1282 7.19153077436557e-06
1283 7.19278895005247e-06
1284 7.19844137723677e-06
1285 7.18907200039642e-06
1286 7.18429796345982e-06
1287 7.18752873851702e-06
1288 7.18601883580028e-06
1289 7.18141315024165e-06
1290 7.17304637577569e-06
1291 7.16256018073319e-06
1292 7.15719990992611e-06
1293 7.15253851363978e-06
1294 7.14300991414007e-06
1295 7.13464901228034e-06
1296 7.13075968300503e-06
1297 7.12787706724088e-06
1298 7.12630867004405e-06
1299 7.12680936849203e-06
1300 7.12405863970145e-06
1301 7.12210074477887e-06
1302 7.11708118422694e-06
1303 7.11279555656863e-06
1304 7.11502920560519e-06
1305 7.11416993098499e-06
1306 7.11004751691863e-06
1307 7.11108436321495e-06
1308 7.11601816267074e-06
1309 7.11504930343658e-06
1310 7.10178017877816e-06
1311 7.0869930799411e-06
1312 7.09932996775103e-06
1313 7.12026040625147e-06
1314 7.10476096823023e-06
1315 7.10921478610724e-06
1316 7.21590027229126e-06
1317 7.27220367819769e-06
1318 7.28521954918016e-06
1319 7.23707697285476e-06
1320 7.18752733659504e-06
1321 7.25688777733751e-06
1322 7.36150180724593e-06
1323 7.36039117008903e-06
1324 7.33356745321462e-06
1325 7.36134895992432e-06
1326 7.33573204146284e-06
1327 7.21700941333156e-06
1328 7.2881820486711e-06
1329 7.63935641644436e-06
1330 8.06936042750932e-06
1331 8.05987085850807e-06
1332 7.82501252239234e-06
1333 7.78372572816238e-06
1334 7.93730067845388e-06
1335 7.93356931595426e-06
1336 7.63520183065872e-06
1337 7.47929948430315e-06
1338 7.73169667478424e-06
1339 8.3240872355586e-06
1340 8.28836947219802e-06
1341 7.63076760754329e-06
1342 7.5734336299595e-06
1343 7.8436014454529e-06
1344 7.64167125142718e-06
1345 7.31466280393112e-06
1346 7.1970626209564e-06
1347 7.28483085216981e-06
1348 7.25923591028341e-06
1349 7.05843032444003e-06
1350 7.00626729127803e-06
1351 7.03117623230415e-06
1352 7.02462212780197e-06
1353 7.00608816709986e-06
1354 7.00604988050201e-06
1355 7.01175944452584e-06
1356 6.97243850800167e-06
1357 6.93529635201889e-06
1358 6.92297881680487e-06
1359 6.92894508248082e-06
1360 6.93295954713388e-06
1361 6.93595122875854e-06
1362 6.93771841057955e-06
1363 6.93942822669116e-06
1364 6.93701705024863e-06
1365 6.91211428372612e-06
1366 6.89057745329597e-06
1367 6.88512823726617e-06
1368 6.88996519180446e-06
1369 6.89255542535988e-06
1370 6.89085168023226e-06
1371 6.89449138377256e-06
1372 6.9048953813336e-06
1373 6.9051124407353e-06
1374 6.90305744196693e-06
1375 6.89063966763381e-06
1376 6.86163977929588e-06
1377 6.84962876819351e-06
1378 6.85388860249825e-06
1379 6.86390263286013e-06
1380 6.86551166884791e-06
1381 6.860033244422e-06
1382 6.86147851804153e-06
1383 6.87883242446279e-06
1384 6.90824251699858e-06
1385 6.92812266545571e-06
1386 6.95303558224253e-06
1387 6.93236889961341e-06
1388 6.87116309014413e-06
1389 6.8529411095912e-06
1390 6.90342696626003e-06
1391 6.97270599389491e-06
1392 6.98380068285803e-06
1393 6.94222688718549e-06
1394 6.89917586308916e-06
1395 6.88122308561447e-06
1396 6.87679492826498e-06
1397 6.89213491503497e-06
1398 6.9556487037022e-06
1399 7.04648543476804e-06
1400 7.10360830883001e-06
1401 7.01549663434387e-06
1402 6.91722121027444e-06
1403 6.95787093550745e-06
1404 7.14526739658931e-06
1405 7.22741228782899e-06
1406 7.15054219377841e-06
1407 7.02863346917996e-06
1408 6.94688895690014e-06
1409 6.91488343868918e-06
1410 6.92783074874984e-06
1411 7.05522069614897e-06
1412 7.1636141174931e-06
1413 7.15396476724806e-06
1414 6.99482628240336e-06
1415 6.92370588855056e-06
1416 7.08802695820572e-06
1417 7.22231004766103e-06
1418 7.15055140745879e-06
1419 7.00600844896065e-06
1420 6.9274884557614e-06
1421 6.92239153832317e-06
1422 7.03934760864816e-06
1423 7.14163213439388e-06
1424 7.04071099907194e-06
1425 6.87689204944761e-06
1426 6.87558698005087e-06
1427 6.98368754081005e-06
1428 6.98262530681717e-06
1429 6.90023773988052e-06
1430 6.85402824183914e-06
1431 6.85574900904039e-06
1432 6.93958138134982e-06
1433 6.99199404262905e-06
1434 6.87766002698681e-06
1435 6.77426782358034e-06
1436 6.79774328216651e-06
1437 6.85381818003438e-06
1438 6.83775888372408e-06
1439 6.79797276985708e-06
1440 6.7797399579371e-06
1441 6.80691852111533e-06
1442 6.87173079376088e-06
1443 6.87145657611923e-06
1444 6.78774177305386e-06
1445 6.7179921169544e-06
1446 6.7481153946088e-06
1447 6.79263348845206e-06
1448 6.7774926789943e-06
1449 6.75132654636215e-06
1450 6.73846403088842e-06
1451 6.75655308681735e-06
1452 6.81473366425604e-06
1453 6.84511493990667e-06
1454 6.79998196546905e-06
1455 6.71411958231422e-06
1456 6.70588984439443e-06
1457 6.77089181460883e-06
1458 6.79490720603866e-06
1459 6.75689178301561e-06
1460 6.72809889780066e-06
1461 6.71865492843935e-06
1462 6.73636691046124e-06
1463 6.81390391055582e-06
1464 6.85186991408737e-06
1465 6.78820610438419e-06
1466 6.70344600987513e-06
1467 6.68581099187542e-06
1468 6.75452242647999e-06
1469 6.79584938773157e-06
1470 6.75914335224268e-06
1471 6.71900947325083e-06
1472 6.70023463952407e-06
1473 6.69898962022395e-06
1474 6.75396026514818e-06
1475 6.84296978404686e-06
1476 6.83512772005693e-06
1477 6.75471554925644e-06
1478 6.67574497946466e-06
1479 6.71518070353397e-06
1480 6.80164530868178e-06
1481 6.8157717628848e-06
1482 6.75030883400679e-06
1483 6.70379140759597e-06
1484 6.68669875197522e-06
1485 6.70422472905314e-06
1486 6.79095023100999e-06
1487 6.85020106000033e-06
1488 6.82958074073314e-06
1489 6.71306301184421e-06
1490 6.67265380146793e-06
1491 6.76002454222965e-06
1492 6.83508251799684e-06
1493 6.79903456223159e-06
1494 6.72074188113291e-06
1495 6.69051224017406e-06
1496 6.6792680924521e-06
1497 6.75530797536721e-06
1498 6.8451409509509e-06
1499 6.83665947639675e-06
1500 6.73224244661721e-06
1501 6.65076324091185e-06
1502 6.70926208461728e-06
1503 6.79888547167474e-06
1504 6.77639514826741e-06
1505 6.70721866891794e-06
1506 6.67015176161918e-06
1507 6.66146008720264e-06
1508 6.74574540205324e-06
1509 6.81132016304951e-06
1510 6.78550166600261e-06
1511 6.67768742196664e-06
1512 6.61386940449191e-06
1513 6.67869538404091e-06
1514 6.73696954674998e-06
1515 6.70236466399857e-06
1516 6.65242082864732e-06
1517 6.62973838652181e-06
1518 6.64466929435907e-06
1519 6.736980943816e-06
1520 6.76045710941778e-06
1521 6.70000704206669e-06
1522 6.59824998710742e-06
1523 6.58829453116169e-06
1524 6.66572420907389e-06
1525 6.69005852271852e-06
1526 6.64215569233691e-06
1527 6.60865430864715e-06
1528 6.59337472427618e-06
1529 6.63482411760949e-06
1530 6.71978576646955e-06
1531 6.71857025300038e-06
1532 6.63535628673412e-06
1533 6.55269868463093e-06
1534 6.58134079746758e-06
1535 6.64796229921929e-06
1536 6.64207286877726e-06
1537 6.59889842077183e-06
1538 6.57305981511484e-06
1539 6.56754581415816e-06
1540 6.62106035352193e-06
1541 6.6931354914786e-06
1542 6.67780553570071e-06
1543 6.59582177874041e-06
1544 6.52994900152137e-06
1545 6.57062914509845e-06
1546 6.64303113117682e-06
1547 6.63067157127438e-06
1548 6.57785437711139e-06
1549 6.55667407409039e-06
1550 6.54619767422265e-06
1551 6.60922553891424e-06
1552 6.68382424570679e-06
1553 6.6663227108292e-06
1554 6.58268931703204e-06
1555 6.51227969793514e-06
1556 6.55916614297574e-06
1557 6.62900377529948e-06
1558 6.61561711265976e-06
1559 6.56467780465152e-06
1560 6.53775337799501e-06
1561 6.52752200469466e-06
1562 6.57695664773856e-06
1563 6.66218506876971e-06
1564 6.65864755326068e-06
1565 6.57754084950474e-06
1566 6.50057162948636e-06
1567 6.5388883113425e-06
1568 6.62069185595744e-06
1569 6.60560319925597e-06
1570 6.55046412302747e-06
1571 6.52269245824885e-06
1572 6.51087236166879e-06
1573 6.56746263203077e-06
1574 6.64416326342544e-06
1575 6.64832841343332e-06
1576 6.5569741263434e-06
1577 6.48357016521869e-06
1578 6.51795430632032e-06
1579 6.5996907980991e-06
1580 6.59524944866471e-06
1581 6.5323568571498e-06
1582 6.5052521188479e-06
1583 6.49088029670001e-06
1584 6.5469920246114e-06
1585 6.63590149280083e-06
1586 6.62058474585475e-06
1587 6.52606048046067e-06
1588 6.46120485579826e-06
1589 6.50836404275472e-06
1590 6.58011938857212e-06
1591 6.56213895863789e-06
1592 6.50753062953618e-06
1593 6.48446858114607e-06
1594 6.47593123090839e-06
1595 6.54436410925591e-06
1596 6.61939659575084e-06
1597 6.59478567786517e-06
1598 6.49917950583737e-06
1599 6.44617297601222e-06
1600 6.51755533262723e-06
1601 6.58265052191204e-06
1602 6.54509702729174e-06
1603 6.49008398740519e-06
1604 6.46315837898772e-06
1605 6.46883883094864e-06
1606 6.5524701073694e-06
1607 6.6085473244791e-06
1608 6.54157980920639e-06
1609 6.44973761421141e-06
1610 6.4284246390114e-06
1611 6.506290374614e-06
1612 6.53418381435511e-06
1613 6.49088858842989e-06
1614 6.45265699003496e-06
1615 6.43476345250915e-06
1616 6.48184515835939e-06
1617 6.57235468303537e-06
1618 6.56936896293106e-06
1619 6.48110102033708e-06
1620 6.4060538865901e-06
1621 6.44821211468811e-06
1622 6.51926052756656e-06
1623 6.50164254122412e-06
1624 6.45202017319699e-06
1625 6.42620048857683e-06
1626 6.42772893745025e-06
1627 6.50781090282277e-06
1628 6.57526715039857e-06
1629 6.5253865912801e-06
1630 6.42118962513455e-06
1631 6.40028661863924e-06
1632 6.47456242897321e-06
1633 6.51213256868581e-06
1634 6.46982326497729e-06
1635 6.42475668875819e-06
1636 6.40352135049645e-06
1637 6.44565124646548e-06
1638 6.53165571337666e-06
1639 6.54745876171696e-06
1640 6.4471649581406e-06
1641 6.3769224952874e-06
1642 6.41077912649023e-06
1643 6.48582050312359e-06
1644 6.46466865311017e-06
1645 6.418094634697e-06
1646 6.39130511493648e-06
1647 6.39617766544537e-06
1648 6.47990003634529e-06
1649 6.52977087724216e-06
1650 6.46367025492617e-06
1651 6.36791391358735e-06
1652 6.36550066082582e-06
1653 6.43107654779734e-06
1654 6.45013730427762e-06
1655 6.40810364684226e-06
1656 6.37897290337247e-06
1657 6.36184234936599e-06
1658 6.42229219454507e-06
1659 6.48423877490736e-06
1660 6.47443441797318e-06
1661 6.37548258478754e-06
1662 6.32551205337686e-06
1663 6.38474975649089e-06
1664 6.44379478116261e-06
1665 6.41283907099315e-06
1666 6.36733928956711e-06
1667 6.33964952909941e-06
1668 6.35945638595515e-06
1669 6.44621925341435e-06
1670 6.46964877312403e-06
1671 6.39437254758637e-06
1672 6.31060663722552e-06
1673 6.31854510258796e-06
1674 6.38981428636596e-06
1675 6.39875654185555e-06
1676 6.35208224870724e-06
1677 6.32658095936055e-06
1678 6.31438187832802e-06
1679 6.37762703742384e-06
1680 6.43714617717516e-06
1681 6.4109442782477e-06
1682 6.32037280031555e-06
1683 6.27925265556453e-06
1684 6.34373838967819e-06
1685 6.39892119162733e-06
1686 6.36562598363335e-06
1687 6.32781629173081e-06
1688 6.29618958921635e-06
1689 6.32492563790166e-06
1690 6.40923531197779e-06
1691 6.42270545670072e-06
1692 6.34110828710496e-06
1693 6.26633665189516e-06
1694 6.28248598281952e-06
1695 6.35113627757628e-06
1696 6.35208621290123e-06
1697 6.31001145866283e-06
1698 6.2865535107986e-06
1699 6.28224383517235e-06
1700 6.35494505718537e-06
1701 6.41145210616935e-06
1702 6.36535662651774e-06
1703 6.27176746198846e-06
1704 6.25096408697178e-06
1705 6.32006275600588e-06
1706 6.35312457860694e-06
1707 6.31469904218643e-06
1708 6.28137792711442e-06
1709 6.25636166055945e-06
1710 6.30724678996053e-06
1711 6.38312485538324e-06
1712 6.36917060329277e-06
1713 6.28780259690279e-06
1714 6.22420718795775e-06
1715 6.2662386453952e-06
1716 6.33344083988757e-06
1717 6.31536312261987e-06
1718 6.27274765132614e-06
1719 6.24416481294956e-06
1720 6.25844707957679e-06
1721 6.341186318792e-06
1722 6.37966561122883e-06
1723 6.30862980212019e-06
1724 6.22390636144388e-06
1725 6.22283690069894e-06
1726 6.30125811357241e-06
1727 6.31786798040363e-06
1728 6.27006264878466e-06
1729 6.23948099147051e-06
1730 6.2185095725531e-06
1731 6.28366387700474e-06
1732 6.34954919227282e-06
1733 6.30757408435004e-06
1734 6.21770888446782e-06
1735 6.18471017108308e-06
1736 6.25273315290817e-06
1737 6.29082674256004e-06
1738 6.26222333057012e-06
1739 6.22898510082171e-06
1740 6.20011374221455e-06
1741 6.24921348102436e-06
1742 6.31804589352537e-06
1743 6.30568415546304e-06
1744 6.21679625448235e-06
1745 6.16322439443923e-06
1746 6.21204751263989e-06
1747 6.27587575292892e-06
1748 6.25357568528103e-06
1749 6.21633143248375e-06
1750 6.18517591286878e-06
1751 6.2163307839036e-06
1752 6.2964691675933e-06
1753 6.29274914977843e-06
1754 6.21710130200431e-06
1755 6.14941887205206e-06
1756 6.17368321984176e-06
1757 6.24782223042353e-06
1758 6.2447756196864e-06
1759 6.20614831891783e-06
1760 6.17497375178126e-06
1761 6.18334345603433e-06
1762 6.26364618733222e-06
1763 6.30345326045265e-06
1764 6.22347489639963e-06
1765 6.14921417299921e-06
1766 6.1523678009396e-06
1767 6.22870652588342e-06
1768 6.23929561872146e-06
1769 6.19621292503002e-06
1770 6.16744269879522e-06
1771 6.16079205345566e-06
1772 6.23058575910385e-06
1773 6.27927891990203e-06
1774 6.2249836402864e-06
1775 6.13669683658703e-06
1776 6.12750896571113e-06
1777 6.19233529657384e-06
1778 6.2185726933823e-06
1779 6.18353886524224e-06
1780 6.15076144262739e-06
1781 6.13502992147241e-06
1782 6.20015137209329e-06
1783 6.25182653729529e-06
1784 6.21742896293395e-06
1785 6.12728310513296e-06
1786 6.10218098150259e-06
1787 6.1650131271667e-06
1788 6.20337742452011e-06
1789 6.17361877729757e-06
1790 6.14253761931994e-06
1791 6.1177304188087e-06
1792 6.17308727851595e-06
1793 6.24540767860465e-06
1794 6.2124566128658e-06
1795 6.12699891382532e-06
1796 6.08902621867797e-06
1797 6.13626724708417e-06
1798 6.18585928957837e-06
1799 6.16611597871724e-06
1800 6.13085545575053e-06
1801 6.10222721525444e-06
1802 6.14426015971281e-06
1803 6.22151939399002e-06
1804 6.21616062413365e-06
1805 6.13269257914578e-06
1806 6.07922776366945e-06
1807 6.12043147136913e-06
1808 6.18482465677719e-06
1809 6.17209746444324e-06
1810 6.13063411989984e-06
1811 6.1004566655245e-06
1812 6.12854123505091e-06
1813 6.20589101368863e-06
1814 6.21795955079666e-06
1815 6.13425190543193e-06
1816 6.0676799728274e-06
1817 6.09777961183174e-06
1818 6.16411547503873e-06
1819 6.15762195849785e-06
1820 6.11896814310155e-06
1821 6.08875582152993e-06
1822 6.11220773438781e-06
1823 6.19363068796175e-06
1824 6.20791281438163e-06
1825 6.12415484346415e-06
1826 6.05924235104038e-06
1827 6.0870880296774e-06
1828 6.14870386454053e-06
1829 6.14401075731499e-06
1830 6.10530455887829e-06
1831 6.07417125788366e-06
1832 6.09682160701564e-06
1833 6.17688613072098e-06
1834 6.18195204330017e-06
1835 6.10022835508678e-06
1836 6.04199313880994e-06
1837 6.06867978770326e-06
1838 6.13495750238433e-06
1839 6.12452445620886e-06
1840 6.08724815627862e-06
1841 6.05647513382082e-06
1842 6.08767853155223e-06
1843 6.16805679176657e-06
1844 6.16587686082855e-06
1845 6.08466000230623e-06
1846 6.02804539698775e-06
1847 6.07333136620758e-06
1848 6.14110326399228e-06
1849 6.12455867808983e-06
1850 6.08193915973415e-06
1851 6.04637132299424e-06
1852 6.07545440844959e-06
1853 6.15377714896153e-06
1854 6.14029039761564e-06
1855 6.05805473780779e-06
1856 6.01023486023465e-06
1857 6.05616182573591e-06
1858 6.11943466286013e-06
1859 6.10099358719796e-06
1860 6.06342274730998e-06
1861 6.03271158960202e-06
1862 6.06768084598919e-06
1863 6.13050305632878e-06
1864 6.117623423103e-06
1865 6.03325770442802e-06
1866 5.99700977423393e-06
1867 6.05104300784552e-06
1868 6.10312646932995e-06
1869 6.08188634034466e-06
1870 6.04823772752368e-06
1871 6.01883921317441e-06
1872 6.07656487011665e-06
1873 6.12857391717365e-06
1874 6.09567324334759e-06
1875 6.01826350883801e-06
1876 5.987169459497e-06
1877 6.05695761327074e-06
1878 6.10848624931655e-06
1879 6.0790257915928e-06
1880 6.04110418437036e-06
1881 6.00685514040578e-06
1882 6.06421332910445e-06
1883 6.11552224527459e-06
1884 6.06155445796383e-06
1885 5.99048825476213e-06
1886 5.97479206678576e-06
1887 6.04263908353733e-06
1888 6.08154878387592e-06
1889 6.05348576374672e-06
1890 6.01846953402868e-06
1891 6.00124777744293e-06
1892 6.06348135506751e-06
1893 6.10082334829816e-06
1894 6.04371253232552e-06
1895 5.97139229766876e-06
1896 5.97225474441299e-06
1897 6.05003985315773e-06
1898 6.06385214634367e-06
1899 6.0283770927554e-06
1900 5.99310035402758e-06
1901 5.99377709337157e-06
1902 6.05877819434553e-06
1903 6.07536499528478e-06
1904 6.01143187067832e-06
1905 5.94871926183757e-06
1906 5.97372645885167e-06
1907 6.04384786732567e-06
1908 6.04403006604977e-06
1909 6.00861407707223e-06
1910 5.97225914442952e-06
1911 5.99571385656583e-06
1912 6.06010140166702e-06
1913 6.04433513097264e-06
1914 5.97477918125658e-06
1915 5.93060474397941e-06
1916 5.97558709406962e-06
1917 6.03331507974952e-06
1918 6.01755763613939e-06
1919 5.98802480570346e-06
1920 5.95297625239822e-06
1921 5.99676936502425e-06
1922 6.05471250680375e-06
1923 6.01404179756751e-06
1924 5.9449239249703e-06
1925 5.91809678696391e-06
1926 5.9790763410443e-06
1927 6.02026014797085e-06
1928 5.99835205059582e-06
1929 5.96626923488449e-06
1930 5.94171277107369e-06
1931 5.9951353686416e-06
1932 6.02700890144287e-06
1933 5.99439881932701e-06
1934 5.92234682967804e-06
1935 5.91040963233286e-06
1936 5.97951325040716e-06
1937 6.0079713048309e-06
1938 5.98176191453359e-06
1939 5.95243008973123e-06
1940 5.93755508421223e-06
1941 5.99899024419925e-06
1942 6.02462425899355e-06
1943 5.98114170813693e-06
1944 5.91029943090501e-06
1945 5.91524188669427e-06
1946 5.98821208114275e-06
1947 6.00794670041512e-06
1948 5.97233177990382e-06
1949 5.94249519384093e-06
1950 5.93119963127053e-06
1951 6.00234238440356e-06
1952 6.02863996940401e-06
1953 5.96173689225432e-06
1954 5.90205546121148e-06
1955 5.91240337838132e-06
1956 5.98696136970209e-06
1957 5.99322671726696e-06
1958 5.95727015605726e-06
1959 5.92183388790836e-06
1960 5.93160168421703e-06
1961 6.00084631285913e-06
1962 6.00424487013848e-06
1963 5.93599852346735e-06
1964 5.88428745126684e-06
1965 5.91904644004632e-06
1966 5.98361705872116e-06
1967 5.97323040700526e-06
1968 5.93894787641108e-06
1969 5.89853773447856e-06
1970 5.92808300209358e-06
1971 5.98775459234801e-06
1972 5.9638438742269e-06
1973 5.89496700656791e-06
1974 5.86294515911133e-06
1975 5.91528770163552e-06
1976 5.96322612826661e-06
1977 5.94547050826585e-06
1978 5.91845730942535e-06
1979 5.88448525275958e-06
1980 5.93550888017269e-06
1981 5.98594244220792e-06
1982 5.93855573092356e-06
1983 5.87372038576492e-06
1984 5.85609007009141e-06
1985 5.92480799490105e-06
1986 5.95682939137296e-06
1987 5.93079835435606e-06
1988 5.8988916099741e-06
1989 5.8859147921142e-06
1990 5.9484453694969e-06
1991 5.96821669262572e-06
1992 5.91445709682836e-06
1993 5.85045106536643e-06
1994 5.86087032775507e-06
1995 5.93133625049862e-06
1996 5.93994470074909e-06
1997 5.91034083809129e-06
1998 5.87487323826741e-06
1999 5.88189410760665e-06
};
\addlegendentry{Train}
\addplot [semithick, black]
table {%
0 0.035373117774725
1 0.0345976650714874
2 0.0338306464254856
3 0.0330664105713367
4 0.0322973392903805
5 0.0315140299499035
6 0.0307066030800343
7 0.0298618860542774
8 0.0289584808051586
9 0.0279700700193644
10 0.0268717035651207
11 0.0256658773869276
12 0.0243534538894892
13 0.0229737441986799
14 0.0215817540884018
15 0.0202187616378069
16 0.0189091768115759
17 0.0176591146737337
18 0.016472902148962
19 0.0153429694473743
20 0.0142644187435508
21 0.0132245877757668
22 0.0122215133160353
23 0.0112630929797888
24 0.0103557175025344
25 0.00948696676641703
26 0.00865495670586824
27 0.0078658564016223
28 0.00711783114820719
29 0.00640764739364386
30 0.00574184721335769
31 0.00513408426195383
32 0.00459685921669006
33 0.00413936702534556
34 0.00376653275452554
35 0.00346726295538247
36 0.00322780874557793
37 0.00304277031682432
38 0.00290563656017184
39 0.00280509819276631
40 0.00272898469120264
41 0.00266945525072515
42 0.00262230215594172
43 0.00258117169141769
44 0.00254418351687491
45 0.00251022609882057
46 0.00247808382846415
47 0.00244725937955081
48 0.00241680187173188
49 0.00238626729696989
50 0.00235506775788963
51 0.00232278369367123
52 0.00228931219317019
53 0.00225518015213311
54 0.00221955101005733
55 0.00218293187208474
56 0.00214514625258744
57 0.00210625538602471
58 0.0020664541516453
59 0.00202569691464305
60 0.00198437669314444
61 0.00194286264013499
62 0.00190043135080487
63 0.00185751798562706
64 0.00181415653787553
65 0.00177065713796765
66 0.0017277110600844
67 0.00168572645634413
68 0.00164459191728383
69 0.0016042476054281
70 0.00156574323773384
71 0.00152808963321149
72 0.00149100401904434
73 0.00145456870086491
74 0.00141945946961641
75 0.00138601067010313
76 0.00135404919274151
77 0.0013232595520094
78 0.0012937793508172
79 0.00126548216212541
80 0.00123839650768787
81 0.00121247861534357
82 0.00118762324564159
83 0.00116379419341683
84 0.00114083022344857
85 0.0011187830241397
86 0.00109758891630918
87 0.00107710296288133
88 0.00105727114714682
89 0.0010380910243839
90 0.00101954187266529
91 0.00100153218954802
92 0.000984090729616582
93 0.000967123953159899
94 0.00095062394393608
95 0.000934559968300164
96 0.000918890000320971
97 0.000903658859897405
98 0.000888784183189273
99 0.000874170858878642
100 0.000859887979459018
101 0.000845908536575735
102 0.000832231598906219
103 0.000818852568045259
104 0.000805690186098218
105 0.000792834849562496
106 0.000780248723458499
107 0.000767879537306726
108 0.00075578794348985
109 0.000743928307201713
110 0.000732294283807278
111 0.000720858457498252
112 0.000709579442627728
113 0.000698515679687262
114 0.000687635852955282
115 0.000676952418871224
116 0.000666483247186989
117 0.000656226126011461
118 0.000646147120278329
119 0.000636269862297922
120 0.000626540451776236
121 0.000616964185610414
122 0.00060755480080843
123 0.000598296173848212
124 0.00058919977163896
125 0.000580231251660734
126 0.000571442767977715
127 0.000562781176995486
128 0.000554231752175838
129 0.000545815855730325
130 0.00053754891268909
131 0.000529421900864691
132 0.000521435227710754
133 0.000513580278493464
134 0.000505913805682212
135 0.000498329754918814
136 0.00049089192179963
137 0.000483605195768178
138 0.000476422777865082
139 0.000469365884782746
140 0.000462439988041297
141 0.000455651752417907
142 0.000449018087238073
143 0.000442498945631087
144 0.000436112401075661
145 0.000429848645580932
146 0.000423688034061342
147 0.000417671195464209
148 0.000411716580856591
149 0.000405916594900191
150 0.000400205986807123
151 0.000394670758396387
152 0.000389276217902079
153 0.000383906852221116
154 0.000378639582777396
155 0.000373449409380555
156 0.00036840484244749
157 0.000363344413926825
158 0.000358415651135147
159 0.000353581999661401
160 0.000348835019394755
161 0.000344250496709719
162 0.000339695834554732
163 0.000335291610099375
164 0.000330895796651021
165 0.000326594163198024
166 0.000322359468555078
167 0.000318223697831854
168 0.000314207369228825
169 0.000310246017761528
170 0.00030637887539342
171 0.000302535481750965
172 0.000298824685160071
173 0.000295078236376867
174 0.000291441770968959
175 0.00028795370599255
176 0.000284492096398026
177 0.000281199172604829
178 0.000277912273304537
179 0.000274727586656809
180 0.000271680910373107
181 0.000268649571808055
182 0.000265623268205673
183 0.000262730725808069
184 0.000259774358710274
185 0.00025689959875308
186 0.000254215905442834
187 0.000251523801125586
188 0.000248852098593488
189 0.000246457057073712
190 0.000243904447415844
191 0.000241432018810883
192 0.000238942637224682
193 0.000236542546190321
194 0.000234267063206062
195 0.000231933343457058
196 0.000229683821089566
197 0.000227558412007056
198 0.000225370706175454
199 0.000223155206185766
200 0.000221093083382584
201 0.000218943285290152
202 0.000216841741348617
203 0.000214931060327217
204 0.000213015315239318
205 0.000211112143006176
206 0.000209355537663214
207 0.000207506265724078
208 0.000205805961741135
209 0.000204014984774403
210 0.000202303694095463
211 0.000200624243007042
212 0.000198905778233893
213 0.00019735045498237
214 0.000195738626644015
215 0.000194235966773704
216 0.000192700768820941
217 0.000191163242561743
218 0.000189712693099864
219 0.000188177611562423
220 0.000186627556104213
221 0.000185132754268125
222 0.000183618933078833
223 0.000182271935045719
224 0.000180913630174473
225 0.000179616166860797
226 0.000178183865500614
227 0.000176854635355994
228 0.000175475492142141
229 0.000173924272530712
230 0.000172486950759776
231 0.000171036415849812
232 0.000169697334058583
233 0.000168372906045988
234 0.000167104066349566
235 0.00016585795674473
236 0.000164640689035878
237 0.000163458753377199
238 0.000162294832989573
239 0.000161144402227364
240 0.000159993884153664
241 0.000158915499923751
242 0.000157859918545
243 0.000156751106260344
244 0.000155816305777989
245 0.000154796289280057
246 0.000153817309183069
247 0.000152750784764066
248 0.000151748274220154
249 0.000150719919474795
250 0.000149724801303819
251 0.000148748556966893
252 0.000147749480674975
253 0.000146760008647107
254 0.000145850804983638
255 0.000144930148962885
256 0.000144012083183043
257 0.00014294964785222
258 0.000141953612910584
259 0.000141010372317396
260 0.000140043892315589
261 0.000139112467877567
262 0.000138150222483091
263 0.00013727939222008
264 0.000136317612486891
265 0.000135472131660208
266 0.000134613874251954
267 0.000133826484670863
268 0.000132942077470943
269 0.00013209690223448
270 0.000131177046569064
271 0.000130196494865231
272 0.000129270163597539
273 0.000128292609588243
274 0.000127381470520049
275 0.000126464030472562
276 0.000125554433907382
277 0.000124668993521482
278 0.000123838137369603
279 0.000122888028272428
280 0.00012207013060106
281 0.000121217053674627
282 0.000120343691378366
283 0.000119528493087273
284 0.00011870710295625
285 0.000117891366244294
286 0.000117031333502382
287 0.000116151750262361
288 0.000115326380182523
289 0.000114457732706796
290 0.000113694215542637
291 0.00011288942914689
292 0.000112061738036573
293 0.000111357279820368
294 0.000110547793156002
295 0.000109766238892917
296 0.000109016858914401
297 0.000108268402982503
298 0.000107497653516475
299 0.000106688938103616
300 0.000105865467048716
301 0.000105130631709471
302 0.000104332088085357
303 0.000103639729786664
304 0.000102954567410052
305 0.000102249498013407
306 0.000101608944532927
307 0.000100985933386255
308 0.000100352284789551
309 9.97719398583286e-05
310 9.91837441688403e-05
311 9.85578953986987e-05
312 9.79360193014145e-05
313 9.73710048128851e-05
314 9.67414744081907e-05
315 9.61137920967303e-05
316 9.5471077656839e-05
317 9.47757362155244e-05
318 9.41348262131214e-05
319 9.35827847570181e-05
320 9.29140587686561e-05
321 9.23291372600943e-05
322 9.1731249995064e-05
323 9.11127644940279e-05
324 9.05459601199254e-05
325 8.99813021533191e-05
326 8.93755932338536e-05
327 8.88030772330239e-05
328 8.82257518242113e-05
329 8.76984049682505e-05
330 8.71050942805596e-05
331 8.66246118675917e-05
332 8.60641230246983e-05
333 8.55536054586992e-05
334 8.49962088977918e-05
335 8.44441965455189e-05
336 8.37929110275581e-05
337 8.3186969277449e-05
338 8.25487295514904e-05
339 8.19199776742607e-05
340 8.1254169344902e-05
341 8.0661935498938e-05
342 8.0042147601489e-05
343 7.94887746451423e-05
344 7.89038531365804e-05
345 7.84212752478197e-05
346 7.78720786911435e-05
347 7.73472493165173e-05
348 7.68017489463091e-05
349 7.62886629672721e-05
350 7.57560483179986e-05
351 7.52351043047383e-05
352 7.47558879083954e-05
353 7.42163174436428e-05
354 7.37129594199359e-05
355 7.32273183530197e-05
356 7.27759324945509e-05
357 7.2251154051628e-05
358 7.17014117981307e-05
359 7.1150112489704e-05
360 7.06372957210988e-05
361 7.0144742494449e-05
362 6.95858980179764e-05
363 6.90282831783406e-05
364 6.84915285091847e-05
365 6.79647637298331e-05
366 6.74504190101288e-05
367 6.69657820253633e-05
368 6.64638500893489e-05
369 6.59610814182088e-05
370 6.54733667033724e-05
371 6.49807552690618e-05
372 6.44719184492715e-05
373 6.39495192444883e-05
374 6.34527168585919e-05
375 6.28695852356032e-05
376 6.22480074525811e-05
377 6.16614415775985e-05
378 6.10752540524118e-05
379 6.05180139245931e-05
380 5.99883642280474e-05
381 5.94295379414689e-05
382 5.88384027651045e-05
383 5.8270034060115e-05
384 5.76803722651675e-05
385 5.71572600165382e-05
386 5.66618655284401e-05
387 5.62182722205762e-05
388 5.5770931794541e-05
389 5.52575220353901e-05
390 5.47345334780402e-05
391 5.42445231985766e-05
392 5.3744934120914e-05
393 5.32895683136303e-05
394 5.28030614077579e-05
395 5.22772897966206e-05
396 5.17484550073277e-05
397 5.12362821609713e-05
398 5.0762504542945e-05
399 5.02613293065224e-05
400 4.97900473419577e-05
401 4.92826802656054e-05
402 4.87470460939221e-05
403 4.82515279145446e-05
404 4.77680405310821e-05
405 4.73070795123931e-05
406 4.68665239168331e-05
407 4.64215263491496e-05
408 4.59830371255521e-05
409 4.55539848189801e-05
410 4.50830193585716e-05
411 4.46071207989007e-05
412 4.4177195377415e-05
413 4.37189846707042e-05
414 4.33094028267078e-05
415 4.29215215262957e-05
416 4.25429316237569e-05
417 4.21374716097489e-05
418 4.18132549384609e-05
419 4.13572124671191e-05
420 4.10008469771128e-05
421 4.06256040150765e-05
422 4.02690602641087e-05
423 3.99295859097037e-05
424 3.96186042053159e-05
425 3.93027185054962e-05
426 3.90292007068638e-05
427 3.86840438295621e-05
428 3.84060876967851e-05
429 3.809567715507e-05
430 3.78221666323952e-05
431 3.74691699107643e-05
432 3.72211798094213e-05
433 3.68498731404543e-05
434 3.66128770110663e-05
435 3.62557038897648e-05
436 3.59978148480877e-05
437 3.57064855052158e-05
438 3.5466488043312e-05
439 3.51587768818717e-05
440 3.49401343555655e-05
441 3.46406086464413e-05
442 3.44186591973994e-05
443 3.41586419381201e-05
444 3.39192592946347e-05
445 3.37043275067117e-05
446 3.3490785426693e-05
447 3.32643721776549e-05
448 3.30828115693294e-05
449 3.28262613038532e-05
450 3.26342342304997e-05
451 3.2441701478092e-05
452 3.22413216053974e-05
453 3.20411600114312e-05
454 3.18535858241376e-05
455 3.1673949706601e-05
456 3.1474122806685e-05
457 3.13291347993072e-05
458 3.11460171360523e-05
459 3.09739334625192e-05
460 3.07915433950257e-05
461 3.06241636280902e-05
462 3.04644636344165e-05
463 3.02718271996127e-05
464 3.01360541925533e-05
465 2.99617404380115e-05
466 2.98254726658342e-05
467 2.96694815915544e-05
468 2.95128102152376e-05
469 2.93678931484465e-05
470 2.92445893137483e-05
471 2.9081289540045e-05
472 2.89333747787168e-05
473 2.88130268017994e-05
474 2.86494723695796e-05
475 2.85222467937274e-05
476 2.83702484011883e-05
477 2.8259713872103e-05
478 2.80983167613158e-05
479 2.79639862128533e-05
480 2.78313054877799e-05
481 2.77223407465499e-05
482 2.75856909865979e-05
483 2.74530102615245e-05
484 2.73399218713166e-05
485 2.7218031391385e-05
486 2.71162734861718e-05
487 2.69916945399018e-05
488 2.68782969214953e-05
489 2.67539016931551e-05
490 2.66574606939685e-05
491 2.65566231973935e-05
492 2.64317441178719e-05
493 2.63130696112057e-05
494 2.62425437540514e-05
495 2.61156437773025e-05
496 2.59994176303735e-05
497 2.58936579484725e-05
498 2.58003146882402e-05
499 2.57249757851241e-05
500 2.56042349064955e-05
501 2.55150280281669e-05
502 2.54385977314087e-05
503 2.53234775300371e-05
504 2.52177414949983e-05
505 2.5145453037112e-05
506 2.50602752203122e-05
507 2.49716322286986e-05
508 2.48768101300811e-05
509 2.47991119977087e-05
510 2.47151183430105e-05
511 2.46298095589736e-05
512 2.4545146516175e-05
513 2.44898419623496e-05
514 2.44022721744841e-05
515 2.43303074967116e-05
516 2.42359401454451e-05
517 2.41644338530023e-05
518 2.40746721829055e-05
519 2.40032404690282e-05
520 2.391067755525e-05
521 2.38423908740515e-05
522 2.37792537518544e-05
523 2.37100448430283e-05
524 2.36230807786342e-05
525 2.35308853007155e-05
526 2.34726085182047e-05
527 2.34091221500421e-05
528 2.3333877834375e-05
529 2.32563379540807e-05
530 2.3183074517874e-05
531 2.31096764764516e-05
532 2.30443001782987e-05
533 2.29880461120047e-05
534 2.29192228289321e-05
535 2.2854133931105e-05
536 2.27696400543209e-05
537 2.27153432206251e-05
538 2.26365464186529e-05
539 2.25699368456844e-05
540 2.25202602450736e-05
541 2.24464074563002e-05
542 2.23713486775523e-05
543 2.23241058847634e-05
544 2.22424478124594e-05
545 2.21840236918069e-05
546 2.21218197111739e-05
547 2.20745550905121e-05
548 2.19957255467307e-05
549 2.19499233935494e-05
550 2.18713230424328e-05
551 2.18217355723027e-05
552 2.1775733330287e-05
553 2.17158794839634e-05
554 2.16461085074116e-05
555 2.15781510632951e-05
556 2.14981737372e-05
557 2.144934842363e-05
558 2.13925213756738e-05
559 2.13273669942282e-05
560 2.12799077417003e-05
561 2.12137365451781e-05
562 2.11547976505244e-05
563 2.1082860257593e-05
564 2.10449579753913e-05
565 2.0982013666071e-05
566 2.09245536098024e-05
567 2.08650162676349e-05
568 2.07939156098291e-05
569 2.07527427846799e-05
570 2.06918612093432e-05
571 2.06390286621172e-05
572 2.05743999686092e-05
573 2.05050437216414e-05
574 2.04826446861262e-05
575 2.04179214051692e-05
576 2.03806575882481e-05
577 2.03124018298695e-05
578 2.02645169338211e-05
579 2.02038245333824e-05
580 2.01602633751463e-05
581 2.01268776436336e-05
582 2.0065090211574e-05
583 2.00030481209978e-05
584 1.9965356841567e-05
585 1.98950729100034e-05
586 1.98610869119875e-05
587 1.97896897589089e-05
588 1.97601020772709e-05
589 1.96985547518125e-05
590 1.96687124116579e-05
591 1.95946959138382e-05
592 1.95614356925944e-05
593 1.95257562154438e-05
594 1.94691674550995e-05
595 1.94268195627956e-05
596 1.9378514480195e-05
597 1.93488176591927e-05
598 1.93061205209233e-05
599 1.92699080798775e-05
600 1.92326187971048e-05
601 1.91823710338213e-05
602 1.91643594007473e-05
603 1.91045437532011e-05
604 1.90696282516001e-05
605 1.90039700100897e-05
606 1.89822185348021e-05
607 1.89238544407999e-05
608 1.88914200407453e-05
609 1.88421745406231e-05
610 1.87970581464469e-05
611 1.87665100384038e-05
612 1.87298865057528e-05
613 1.86896886589238e-05
614 1.86425568244886e-05
615 1.86086945177522e-05
616 1.85511325980769e-05
617 1.85177250386914e-05
618 1.84687614819268e-05
619 1.84475065907463e-05
620 1.83989523065975e-05
621 1.83686333912192e-05
622 1.83068368642125e-05
623 1.82779331225902e-05
624 1.82299809239339e-05
625 1.81750601768726e-05
626 1.81583563971799e-05
627 1.81037776201265e-05
628 1.80664465005975e-05
629 1.80441620614147e-05
630 1.79855815076735e-05
631 1.79433809535112e-05
632 1.7934775314643e-05
633 1.78780574060511e-05
634 1.7828997442848e-05
635 1.77984384208685e-05
636 1.77612237166613e-05
637 1.7694772395771e-05
638 1.76859794009943e-05
639 1.76460016518831e-05
640 1.75679051608313e-05
641 1.75537279574201e-05
642 1.75153872987721e-05
643 1.74776469066273e-05
644 1.74381257238565e-05
645 1.73831213032827e-05
646 1.73538319359068e-05
647 1.73057069332572e-05
648 1.72857453435427e-05
649 1.72514883161057e-05
650 1.7211030353792e-05
651 1.71822157426504e-05
652 1.7149899576907e-05
653 1.71042738656979e-05
654 1.70757284649881e-05
655 1.70466719282558e-05
656 1.70018647622783e-05
657 1.69874365383293e-05
658 1.69450813700678e-05
659 1.69069699040847e-05
660 1.68807109730551e-05
661 1.68512724485481e-05
662 1.68008755281335e-05
663 1.67620037245797e-05
664 1.67504076671321e-05
665 1.67098824022105e-05
666 1.66815971169854e-05
667 1.66614208865212e-05
668 1.66210029419744e-05
669 1.65829933393979e-05
670 1.65583460329799e-05
671 1.64937955560163e-05
672 1.64745306392433e-05
673 1.64491502800956e-05
674 1.63972999871476e-05
675 1.63592922035605e-05
676 1.63614186021732e-05
677 1.63094427989563e-05
678 1.62688083946705e-05
679 1.62557935254881e-05
680 1.62009837367805e-05
681 1.61723073688336e-05
682 1.61403513629921e-05
683 1.61205080075888e-05
684 1.60826803039527e-05
685 1.60454383149045e-05
686 1.60106192197418e-05
687 1.59814644575818e-05
688 1.59504797920818e-05
689 1.59246410476044e-05
690 1.58869188453536e-05
691 1.58357470354531e-05
692 1.58227630890906e-05
693 1.58075799845392e-05
694 1.57651211338816e-05
695 1.57271679199766e-05
696 1.56991500261938e-05
697 1.56770838657394e-05
698 1.56412897922564e-05
699 1.56342703121481e-05
700 1.5574463759549e-05
701 1.55484303832054e-05
702 1.55416510096984e-05
703 1.55193283717381e-05
704 1.5469515346922e-05
705 1.54428053065203e-05
706 1.54373010445852e-05
707 1.54125573317287e-05
708 1.53597229655134e-05
709 1.53512773977127e-05
710 1.53045057231793e-05
711 1.52947977767326e-05
712 1.52410484588472e-05
713 1.5212302969303e-05
714 1.51974136315403e-05
715 1.51654712681193e-05
716 1.51363337863586e-05
717 1.51055992319016e-05
718 1.50736414070707e-05
719 1.50414653035114e-05
720 1.5012198673503e-05
721 1.50190735439537e-05
722 1.49789957504254e-05
723 1.49503002830897e-05
724 1.49269699249999e-05
725 1.49028028317844e-05
726 1.48925864777993e-05
727 1.48475119203795e-05
728 1.48255712701939e-05
729 1.47885466503794e-05
730 1.47605978781939e-05
731 1.47703758557327e-05
732 1.47112932609161e-05
733 1.46731745189754e-05
734 1.46459815368871e-05
735 1.46143465826754e-05
736 1.45939047797583e-05
737 1.45616913869162e-05
738 1.45362355397083e-05
739 1.45299545692978e-05
740 1.45098683788092e-05
741 1.44824243761832e-05
742 1.44458872455289e-05
743 1.44256291605416e-05
744 1.44238883876824e-05
745 1.43747947731754e-05
746 1.43791612572386e-05
747 1.4320847185445e-05
748 1.43146298796637e-05
749 1.42959206641535e-05
750 1.426351354894e-05
751 1.42447379403166e-05
752 1.42144826895674e-05
753 1.41905929922359e-05
754 1.41627751872875e-05
755 1.41490636451636e-05
756 1.41164655360626e-05
757 1.41039781738073e-05
758 1.40837310027564e-05
759 1.4058929082239e-05
760 1.40086376632098e-05
761 1.39954418045818e-05
762 1.39622597998823e-05
763 1.39618541652453e-05
764 1.39250496431487e-05
765 1.38922541736974e-05
766 1.38658351716003e-05
767 1.38323575811228e-05
768 1.38127979880664e-05
769 1.38162404255127e-05
770 1.38108543978888e-05
771 1.37488541440689e-05
772 1.372518545395e-05
773 1.36883809318533e-05
774 1.36978560476564e-05
775 1.36749722514651e-05
776 1.36547978399904e-05
777 1.36062753881561e-05
778 1.36050366563722e-05
779 1.35693326228647e-05
780 1.35521113406867e-05
781 1.35385744215455e-05
782 1.35247764774249e-05
783 1.34990168589866e-05
784 1.3479424524121e-05
785 1.34484780573985e-05
786 1.34443307615584e-05
787 1.3414985005511e-05
788 1.34121210066951e-05
789 1.33720141093363e-05
790 1.33611865749117e-05
791 1.33224257297115e-05
792 1.33089506562101e-05
793 1.32956829475006e-05
794 1.33050971271587e-05
795 1.32450413730112e-05
796 1.32068880702718e-05
797 1.31732758745784e-05
798 1.31624938148889e-05
799 1.31429369503167e-05
800 1.31359811348375e-05
801 1.31054948724341e-05
802 1.30796533994726e-05
803 1.30278476717649e-05
804 1.30209400595049e-05
805 1.30059715957032e-05
806 1.30171592900297e-05
807 1.29715317598311e-05
808 1.29642821775633e-05
809 1.29156969705946e-05
810 1.29153459056397e-05
811 1.29077225210494e-05
812 1.28827596199699e-05
813 1.28747396956896e-05
814 1.28389983728994e-05
815 1.27990970213432e-05
816 1.27628500194987e-05
817 1.2753721421177e-05
818 1.27507637444069e-05
819 1.27512976177968e-05
820 1.2725015039905e-05
821 1.26914628708619e-05
822 1.26547784020659e-05
823 1.26472978081438e-05
824 1.26589893625351e-05
825 1.26557824842166e-05
826 1.26301556520048e-05
827 1.26185395856737e-05
828 1.25635515360045e-05
829 1.25502538139699e-05
830 1.25602200569119e-05
831 1.25825681607239e-05
832 1.25497326735058e-05
833 1.24846601465833e-05
834 1.24586440506391e-05
835 1.24659418361261e-05
836 1.25114584079711e-05
837 1.25135356938699e-05
838 1.24769494505017e-05
839 1.23857744256384e-05
840 1.23632953545894e-05
841 1.24011257867096e-05
842 1.24675580082112e-05
843 1.24372900245362e-05
844 1.23867184811388e-05
845 1.22667715913849e-05
846 1.2264633369341e-05
847 1.23519366752589e-05
848 1.23792142403545e-05
849 1.23778027045773e-05
850 1.22566962090787e-05
851 1.22223973448854e-05
852 1.22670526252477e-05
853 1.23879808597849e-05
854 1.2410679119057e-05
855 1.22846877275151e-05
856 1.20735376185621e-05
857 1.20038448585547e-05
858 1.21930488603539e-05
859 1.24218822747935e-05
860 1.24774960568175e-05
861 1.24199614219833e-05
862 1.20944732771022e-05
863 1.18872776511125e-05
864 1.20120093924925e-05
865 1.22214632938267e-05
866 1.2305755262787e-05
867 1.22007504614885e-05
868 1.19948963401839e-05
869 1.17936242531869e-05
870 1.19044289021986e-05
871 1.20320328278467e-05
872 1.21479733934393e-05
873 1.20817794595496e-05
874 1.19166315926122e-05
875 1.17662448246847e-05
876 1.18275611384888e-05
877 1.18992347779567e-05
878 1.2003749361611e-05
879 1.20186241474585e-05
880 1.18650641525164e-05
881 1.17242179840105e-05
882 1.17117042464088e-05
883 1.18027046482894e-05
884 1.18988491522032e-05
885 1.19060441647889e-05
886 1.18025054689497e-05
887 1.16467663247022e-05
888 1.16325063572731e-05
889 1.16813571366947e-05
890 1.18105153887882e-05
891 1.18189109343803e-05
892 1.17417075671256e-05
893 1.15944303615834e-05
894 1.15375723908073e-05
895 1.1621442354226e-05
896 1.1741849448299e-05
897 1.17131048682495e-05
898 1.16089568109601e-05
899 1.15133980216342e-05
900 1.15062484837836e-05
901 1.16262845040183e-05
902 1.17666759251733e-05
903 1.18183397717075e-05
904 1.16948185677757e-05
905 1.15588009066414e-05
906 1.14377016871003e-05
907 1.15202064989717e-05
908 1.17104582386673e-05
909 1.18737170851091e-05
910 1.18225516416715e-05
911 1.16502787932404e-05
912 1.14500271592988e-05
913 1.13948908619932e-05
914 1.16053988676867e-05
915 1.18550851766486e-05
916 1.19104424811667e-05
917 1.18343323265435e-05
918 1.16391374831437e-05
919 1.13973901534337e-05
920 1.13978785520885e-05
921 1.16135806820239e-05
922 1.17923409561627e-05
923 1.17983308882685e-05
924 1.16594892460853e-05
925 1.14300037239445e-05
926 1.12820698632277e-05
927 1.13653641165001e-05
928 1.14855647552758e-05
929 1.15332986752037e-05
930 1.1499455467856e-05
931 1.13861351564992e-05
932 1.12352554424433e-05
933 1.12125799205387e-05
934 1.12857669591904e-05
935 1.13123205665033e-05
936 1.13449978016433e-05
937 1.12341613203171e-05
938 1.11485915113008e-05
939 1.11272738649859e-05
940 1.12030775198946e-05
941 1.12551224447088e-05
942 1.12269608507631e-05
943 1.11579893200542e-05
944 1.10836972453399e-05
945 1.10713817775832e-05
946 1.11643357740832e-05
947 1.12755169539014e-05
948 1.12967372842832e-05
949 1.12351026473334e-05
950 1.10882947410573e-05
951 1.10125683931983e-05
952 1.10828050310374e-05
953 1.13063451863127e-05
954 1.14387021312723e-05
955 1.14239001050009e-05
956 1.12562602225807e-05
957 1.10315495476243e-05
958 1.09670354504487e-05
959 1.11628623926663e-05
960 1.13630276246113e-05
961 1.14163840407855e-05
962 1.14957101686741e-05
963 1.12062716652872e-05
964 1.09439806692535e-05
965 1.09771654024371e-05
966 1.12494317363598e-05
967 1.13935866465908e-05
968 1.14995455078315e-05
969 1.14511649371707e-05
970 1.12385996544617e-05
971 1.10103683255147e-05
972 1.10923538159113e-05
973 1.13578553282423e-05
974 1.15519878818304e-05
975 1.15765151349478e-05
976 1.13295318442397e-05
977 1.10193514046841e-05
978 1.09137154140626e-05
979 1.10915507320897e-05
980 1.12302404886577e-05
981 1.13416499516461e-05
982 1.1246076610405e-05
983 1.1016128155461e-05
984 1.08469166661962e-05
985 1.08687918327632e-05
986 1.09248758235481e-05
987 1.09212187453522e-05
988 1.08771218947368e-05
989 1.08126714621903e-05
990 1.07943133116351e-05
991 1.07780979305971e-05
992 1.0798491530295e-05
993 1.08112872112542e-05
994 1.07818641481572e-05
995 1.07415271486389e-05
996 1.07557207229547e-05
997 1.0762273632281e-05
998 1.07476926132222e-05
999 1.07601117633749e-05
1000 1.07902169474983e-05
1001 1.07431578726391e-05
1002 1.07619944174075e-05
1003 1.07248615677236e-05
1004 1.07287132777856e-05
1005 1.0748879503808e-05
1006 1.07520445453702e-05
1007 1.07513042166829e-05
1008 1.07410160126165e-05
1009 1.07461000879994e-05
1010 1.07070845842827e-05
1011 1.07057285276824e-05
1012 1.07172754724161e-05
1013 1.07520427263808e-05
1014 1.07512096292339e-05
1015 1.07304012999521e-05
1016 1.06860934465658e-05
1017 1.07099249362363e-05
1018 1.06857141872752e-05
1019 1.07061441667611e-05
1020 1.07604382719728e-05
1021 1.07801542981178e-05
1022 1.07611595012713e-05
1023 1.07085988929612e-05
1024 1.06552251963876e-05
1025 1.06328743640915e-05
1026 1.06883089756593e-05
1027 1.08039312181063e-05
1028 1.08930316855549e-05
1029 1.08235581137706e-05
1030 1.07245205072104e-05
1031 1.0559345355432e-05
1032 1.05530525615904e-05
1033 1.06972584035248e-05
1034 1.10115197458072e-05
1035 1.1265005923633e-05
1036 1.15811253635911e-05
1037 1.1166681360919e-05
1038 1.07776277218363e-05
1039 1.05420767795295e-05
1040 1.07687874333351e-05
1041 1.092744696507e-05
1042 1.10513583422289e-05
1043 1.13064215838676e-05
1044 1.15250750241103e-05
1045 1.15345574158709e-05
1046 1.14853392005898e-05
1047 1.13451169454493e-05
1048 1.12276011350332e-05
1049 1.11759063656791e-05
1050 1.10812070488464e-05
1051 1.09090642581577e-05
1052 1.06324860098539e-05
1053 1.0517659575271e-05
1054 1.04302671388723e-05
1055 1.04067166830646e-05
1056 1.04617938632146e-05
1057 1.06224169940106e-05
1058 1.06372344816918e-05
1059 1.06114821392111e-05
1060 1.12072502815863e-05
1061 1.21361426863587e-05
1062 1.30961352624581e-05
1063 1.36133267005789e-05
1064 1.26492568597314e-05
1065 1.09413531390601e-05
1066 1.04226528492291e-05
1067 1.05244243968627e-05
1068 1.06200559457648e-05
1069 1.05561539385235e-05
1070 1.11867193481885e-05
1071 1.15012644528178e-05
1072 1.09429865915445e-05
1073 1.03916272564675e-05
1074 1.03215843409998e-05
1075 1.03305737866322e-05
1076 1.0397255209682e-05
1077 1.06033267002204e-05
1078 1.06824036265607e-05
1079 1.05876215457101e-05
1080 1.04381733763148e-05
1081 1.03222710094997e-05
1082 1.03032962215366e-05
1083 1.03489746834384e-05
1084 1.03823067547637e-05
1085 1.0479404409125e-05
1086 1.04761184047675e-05
1087 1.04169366750284e-05
1088 1.03911588666961e-05
1089 1.0321648005629e-05
1090 1.03501297417097e-05
1091 1.03031079561333e-05
1092 1.03617558124824e-05
1093 1.03647034848109e-05
1094 1.0394092896604e-05
1095 1.03926158772083e-05
1096 1.03651427707518e-05
1097 1.0355294762121e-05
1098 1.03549809864489e-05
1099 1.03123966255225e-05
1100 1.0305258911103e-05
1101 1.02929525382933e-05
1102 1.02947888080962e-05
1103 1.03312922874466e-05
1104 1.02974963738234e-05
1105 1.03500069599249e-05
1106 1.03141010185936e-05
1107 1.03125075838761e-05
1108 1.03330403362634e-05
1109 1.0287314580637e-05
1110 1.03021948234527e-05
1111 1.02952526503941e-05
1112 1.0296555046807e-05
1113 1.03083875728771e-05
1114 1.02916583273327e-05
1115 1.02809162854101e-05
1116 1.02838384918869e-05
1117 1.02769181467011e-05
1118 1.02944222817314e-05
1119 1.02824988061911e-05
1120 1.02688745755586e-05
1121 1.02691756183049e-05
1122 1.0285405551258e-05
1123 1.02600806712871e-05
1124 1.0276796274411e-05
1125 1.02955364127411e-05
1126 1.0242417374684e-05
1127 1.02777576103108e-05
1128 1.0274972737534e-05
1129 1.02596577562508e-05
1130 1.02845888250158e-05
1131 1.02513249657932e-05
1132 1.02837302620173e-05
1133 1.02646445157006e-05
1134 1.0292252227373e-05
1135 1.02595058706356e-05
1136 1.02541453088634e-05
1137 1.02613585113431e-05
1138 1.02395333669847e-05
1139 1.02844524008106e-05
1140 1.02487310869037e-05
1141 1.02709082057117e-05
1142 1.02450885606231e-05
1143 1.02498852356803e-05
1144 1.02473795777769e-05
1145 1.02558196886093e-05
1146 1.02490939752897e-05
1147 1.02348440123023e-05
1148 1.02582180261379e-05
1149 1.02185003925115e-05
1150 1.02639623946743e-05
1151 1.0239023140457e-05
1152 1.02462254290003e-05
1153 1.02199883258436e-05
1154 1.02218909887597e-05
1155 1.01966406873544e-05
1156 1.02248013718054e-05
1157 1.02505655377172e-05
1158 1.02557341961074e-05
1159 1.0237952665193e-05
1160 1.01911664387444e-05
1161 1.01503410405712e-05
1162 1.01826362879365e-05
1163 1.02250378404278e-05
1164 1.02559124570689e-05
1165 1.0252493666485e-05
1166 1.01540399555233e-05
1167 1.00566485343734e-05
1168 1.00498282336048e-05
1169 1.02149278973229e-05
1170 1.03431711977464e-05
1171 1.05593981061247e-05
1172 1.06042170955334e-05
1173 1.02546428024652e-05
1174 1.00961360658403e-05
1175 9.95009213511366e-06
1176 1.01795494629187e-05
1177 1.04765831565601e-05
1178 1.0624776223267e-05
1179 1.08761769297416e-05
1180 1.0542678865022e-05
1181 1.01875330074108e-05
1182 9.8784976216848e-06
1183 1.00969173217891e-05
1184 1.01574869404431e-05
1185 1.01656314654974e-05
1186 1.01947543953429e-05
1187 1.0305682735634e-05
1188 1.04506234492874e-05
1189 1.05238550531794e-05
1190 1.04605651358725e-05
1191 1.02350040833699e-05
1192 9.99347503238823e-06
1193 9.87306248134701e-06
1194 9.98120867734542e-06
1195 1.00097804534016e-05
1196 1.00097195172566e-05
1197 9.95274331216933e-06
1198 9.91816432360793e-06
1199 9.87168732535793e-06
1200 9.88448846328538e-06
1201 9.80486674961867e-06
1202 9.7836982604349e-06
1203 9.79271317191888e-06
1204 1.00323104561539e-05
1205 1.03678039522492e-05
1206 1.03691481854185e-05
1207 1.03611782833468e-05
1208 1.12824272946455e-05
1209 1.30509406517376e-05
1210 1.46338552440284e-05
1211 1.51330887092627e-05
1212 1.16711089503951e-05
1213 9.92667537502712e-06
1214 9.91694287222344e-06
1215 1.01175719464663e-05
1216 9.92678360489663e-06
1217 1.03497923191753e-05
1218 1.06059387690038e-05
1219 1.01803925645072e-05
1220 9.83129029918928e-06
1221 9.73866372078191e-06
1222 9.77578565652948e-06
1223 9.87421117315535e-06
1224 9.95912250800757e-06
1225 9.97506231215084e-06
1226 9.96183280221885e-06
1227 9.87344265013235e-06
1228 9.81467383098789e-06
1229 9.81181347015081e-06
1230 9.80539152806159e-06
1231 9.8523851193022e-06
1232 9.87670500762761e-06
1233 9.87815110420343e-06
1234 9.87595467449864e-06
1235 9.86024861049373e-06
1236 9.82563869911246e-06
1237 9.81987523118732e-06
1238 9.79640935838688e-06
1239 9.78403386397986e-06
1240 9.80766526481602e-06
1241 9.80912955128588e-06
1242 9.82049186859513e-06
1243 9.82971960183932e-06
1244 9.83058998826891e-06
1245 9.81158427748596e-06
1246 9.83265817922074e-06
1247 9.79614196694456e-06
1248 9.76899718807545e-06
1249 9.78618572844425e-06
1250 9.75467992248014e-06
1251 9.76866886048811e-06
1252 9.76073260972043e-06
1253 9.76520277617965e-06
1254 9.79769538389519e-06
1255 9.78816751739942e-06
1256 9.80692038865527e-06
1257 9.80425738816848e-06
1258 9.81386983767152e-06
1259 9.77049785433337e-06
1260 9.76957471721107e-06
1261 9.76056799117941e-06
1262 9.71981171460357e-06
1263 9.73279111349257e-06
1264 9.69690790952882e-06
1265 9.68690801528282e-06
1266 9.72293037193595e-06
1267 9.70170913205948e-06
1268 9.70069049799349e-06
1269 9.71652116277255e-06
1270 9.73246733337874e-06
1271 9.75113289314322e-06
1272 9.7827205536305e-06
1273 9.80062031885609e-06
1274 9.77935906121274e-06
1275 9.76932733465219e-06
1276 9.76623869064497e-06
1277 9.74100566963898e-06
1278 9.73031546891434e-06
1279 9.71790814219275e-06
1280 9.66911738942144e-06
1281 9.67828600551002e-06
1282 9.64810533332638e-06
1283 9.62020112638129e-06
1284 9.65077742876019e-06
1285 9.64938953984529e-06
1286 9.61012392508565e-06
1287 9.61872956395382e-06
1288 9.6167159426841e-06
1289 9.61487603490241e-06
1290 9.63665479503106e-06
1291 9.64087030297378e-06
1292 9.63442744250642e-06
1293 9.66655261436244e-06
1294 9.69383199844742e-06
1295 9.70384644460864e-06
1296 9.74051363300532e-06
1297 9.71984991338104e-06
1298 9.68754011410056e-06
1299 9.67299911280861e-06
1300 9.65525487117702e-06
1301 9.6669928097981e-06
1302 9.71879944700049e-06
1303 9.75056991592282e-06
1304 9.77034596871817e-06
1305 9.7097235993715e-06
1306 9.63804268394597e-06
1307 9.57477550400654e-06
1308 9.55363157117972e-06
1309 9.60766828939086e-06
1310 9.69986103882547e-06
1311 9.80352706392296e-06
1312 9.85765291261487e-06
1313 9.80952245299704e-06
1314 9.63023285294184e-06
1315 9.41384769248543e-06
1316 9.43372378969798e-06
1317 9.47788521443726e-06
1318 9.48818887991365e-06
1319 9.48640627029818e-06
1320 9.44570092542563e-06
1321 9.47590706346091e-06
1322 9.52675054577412e-06
1323 9.55191262619337e-06
1324 9.56257736106636e-06
1325 9.59008139034268e-06
1326 9.78771913651144e-06
1327 1.01247624115786e-05
1328 1.06123770819977e-05
1329 1.08788171928609e-05
1330 1.09660995804006e-05
1331 1.14785889309132e-05
1332 1.15193934107083e-05
1333 1.14731647045119e-05
1334 1.12038605948328e-05
1335 1.02702924777986e-05
1336 9.49935019889381e-06
1337 9.52084610617021e-06
1338 9.90420448943041e-06
1339 9.92920922726626e-06
1340 9.63380352914101e-06
1341 1.04670771179372e-05
1342 1.13102669274667e-05
1343 1.08339108919608e-05
1344 9.7813872343977e-06
1345 9.38937682803953e-06
1346 9.34063882596092e-06
1347 9.39457186177606e-06
1348 9.50263711274602e-06
1349 9.73318583419314e-06
1350 9.8474074547994e-06
1351 9.66146035352722e-06
1352 9.44084968068637e-06
1353 9.3482194643002e-06
1354 9.34995659918059e-06
1355 9.39170513447607e-06
1356 9.45623014558805e-06
1357 9.56557687459281e-06
1358 9.58354939939454e-06
1359 9.5175055321306e-06
1360 9.42599490372231e-06
1361 9.36944434215548e-06
1362 9.34351010073442e-06
1363 9.34736635826994e-06
1364 9.39655546972062e-06
1365 9.45128977036802e-06
1366 9.49094010138651e-06
1367 9.50037519942271e-06
1368 9.4711331257713e-06
1369 9.42620317800902e-06
1370 9.36036758503178e-06
1371 9.3075395852793e-06
1372 9.31946669879835e-06
1373 9.31191880226834e-06
1374 9.33947558223736e-06
1375 9.40512200031662e-06
1376 9.46825093706138e-06
1377 9.5328941824846e-06
1378 9.53569360717665e-06
1379 9.47982334764674e-06
1380 9.43810573517112e-06
1381 9.36154810915468e-06
1382 9.27780638448894e-06
1383 9.25762196857249e-06
1384 9.25963831832632e-06
1385 9.22911203815602e-06
1386 9.27349901758134e-06
1387 9.37101413001074e-06
1388 9.52470054471632e-06
1389 9.73367332335329e-06
1390 9.90622447716305e-06
1391 9.92772675090237e-06
1392 9.78594289335888e-06
1393 9.5898376457626e-06
1394 9.40785048442194e-06
1395 9.30319220060483e-06
1396 9.2310019681463e-06
1397 9.18261321203317e-06
1398 9.19880949368235e-06
1399 9.23261904972605e-06
1400 9.29665111470968e-06
1401 9.42780388868414e-06
1402 9.77254057943355e-06
1403 1.01773948699702e-05
1404 1.04115679278038e-05
1405 1.03727852547308e-05
1406 9.97491770249326e-06
1407 9.56676558416802e-06
1408 9.36151536734542e-06
1409 9.19171543500852e-06
1410 9.15436612558551e-06
1411 9.22626350075006e-06
1412 9.24345931707649e-06
1413 9.31422709982144e-06
1414 9.60704983299365e-06
1415 1.01243595054257e-05
1416 1.04176870081574e-05
1417 1.03316197055392e-05
1418 9.82203710009344e-06
1419 9.37691129365703e-06
1420 9.19554986467119e-06
1421 9.14291013032198e-06
1422 9.20093589229509e-06
1423 9.21749051485676e-06
1424 9.32670081965625e-06
1425 9.68040058069164e-06
1426 1.00008855952183e-05
1427 9.94161291600903e-06
1428 9.58745749812806e-06
1429 9.26494158193236e-06
1430 9.14200063562021e-06
1431 9.10584003577242e-06
1432 9.12921859708149e-06
1433 9.19718604563968e-06
1434 9.32937291509006e-06
1435 9.61364366958151e-06
1436 9.77258514467394e-06
1437 9.63844740908826e-06
1438 9.34899890125962e-06
1439 9.18281239137286e-06
1440 9.08928086573724e-06
1441 9.07762478163932e-06
1442 9.10124617803376e-06
1443 9.1478959802771e-06
1444 9.33961837290553e-06
1445 9.54827464738628e-06
1446 9.67293181020068e-06
1447 9.54378811002243e-06
1448 9.31992690311745e-06
1449 9.15706914383918e-06
1450 9.06497916730586e-06
1451 9.0381154222996e-06
1452 9.06341756490292e-06
1453 9.09640039026272e-06
1454 9.21577066037571e-06
1455 9.44013572734548e-06
1456 9.63768252404407e-06
1457 9.66838797467062e-06
1458 9.49639525060775e-06
1459 9.27520704863127e-06
1460 9.12103951122845e-06
1461 9.04435364645906e-06
1462 9.00403119885596e-06
1463 9.04577154869912e-06
1464 9.06942022993462e-06
1465 9.15972668735776e-06
1466 9.37364711717237e-06
1467 9.64182436291594e-06
1468 9.71688041317975e-06
1469 9.58312284637941e-06
1470 9.33143292058958e-06
1471 9.16177577892086e-06
1472 9.05689921637531e-06
1473 8.98094003787264e-06
1474 8.98069174581906e-06
1475 9.02338342712028e-06
1476 9.0574030764401e-06
1477 9.24932010093471e-06
1478 9.51461151998956e-06
1479 9.7217271104455e-06
1480 9.74913200479932e-06
1481 9.54336428549141e-06
1482 9.26771099329926e-06
1483 9.09557456907351e-06
1484 8.99418682820396e-06
1485 8.94955974217737e-06
1486 8.98437701835064e-06
1487 8.99659789865837e-06
1488 9.08991569303907e-06
1489 9.31192062125774e-06
1490 9.62952253757976e-06
1491 9.79123433353379e-06
1492 9.69713073573075e-06
1493 9.38685025175801e-06
1494 9.14212159841554e-06
1495 9.00304712558864e-06
1496 8.91294439497869e-06
1497 8.9393661255599e-06
1498 8.96093479241244e-06
1499 8.99581937119365e-06
1500 9.18285331863444e-06
1501 9.53272865444887e-06
1502 9.73781061475165e-06
1503 9.65763319982216e-06
1504 9.36794185690815e-06
1505 9.08406036614906e-06
1506 8.96033270691987e-06
1507 8.8715487436275e-06
1508 8.9153090812033e-06
1509 8.9185887190979e-06
1510 8.97529571375344e-06
1511 9.19807098398451e-06
1512 9.48746219364693e-06
1513 9.61494606599445e-06
1514 9.46726595429936e-06
1515 9.17766010388732e-06
1516 8.98198140930617e-06
1517 8.88120030140271e-06
1518 8.83543725649361e-06
1519 8.87148962647188e-06
1520 8.87199621502077e-06
1521 8.99212864169385e-06
1522 9.23869993130211e-06
1523 9.47782973526046e-06
1524 9.5005243565538e-06
1525 9.28857753024204e-06
1526 9.02275860426016e-06
1527 8.88815065991366e-06
1528 8.80565312399995e-06
1529 8.8115302787628e-06
1530 8.82418498804327e-06
1531 8.8431270341971e-06
1532 8.99738915904891e-06
1533 9.2699901870219e-06
1534 9.44967723626178e-06
1535 9.38763423619093e-06
1536 9.16637964110123e-06
1537 8.95233279152308e-06
1538 8.82146468939027e-06
1539 8.75873047334608e-06
1540 8.76862213772256e-06
1541 8.78574064699933e-06
1542 8.81694268173305e-06
1543 8.98714552022284e-06
1544 9.27913788473234e-06
1545 9.44385374168633e-06
1546 9.37405093281996e-06
1547 9.12939503905363e-06
1548 8.90917999640806e-06
1549 8.79563413036522e-06
1550 8.71926567924675e-06
1551 8.74013767315773e-06
1552 8.74818033480551e-06
1553 8.77973980095703e-06
1554 8.95872108230833e-06
1555 9.22729304875247e-06
1556 9.40131667448441e-06
1557 9.34370655159e-06
1558 9.11490769794909e-06
1559 8.87564874574309e-06
1560 8.76156991580501e-06
1561 8.6888212535996e-06
1562 8.69158157001948e-06
1563 8.70934400154511e-06
1564 8.75071236805525e-06
1565 8.91602030606009e-06
1566 9.19918966246769e-06
1567 9.39817891776329e-06
1568 9.35215302888537e-06
1569 9.10078870219877e-06
1570 8.86988891579676e-06
1571 8.74369652592577e-06
1572 8.65500078361947e-06
1573 8.6711224867031e-06
1574 8.68535335030174e-06
1575 8.7331445683958e-06
1576 8.88143495103577e-06
1577 9.18325713428203e-06
1578 9.37865388550563e-06
1579 9.33263982005883e-06
1580 9.06472996575758e-06
1581 8.83484335645335e-06
1582 8.71504198585171e-06
1583 8.62780598254176e-06
1584 8.64383764564991e-06
1585 8.67305334395496e-06
1586 8.69525956659345e-06
1587 8.87058376974892e-06
1588 9.17263605515473e-06
1589 9.33960836846381e-06
1590 9.25953645491973e-06
1591 8.98558300832519e-06
1592 8.77143611432984e-06
1593 8.66713344294112e-06
1594 8.6028949226602e-06
1595 8.61611897562398e-06
1596 8.64150115376106e-06
1597 8.70629082783125e-06
1598 8.89261900738347e-06
1599 9.19628837436903e-06
1600 9.34927174967015e-06
1601 9.21661376196425e-06
1602 8.92173284228193e-06
1603 8.72634063853184e-06
1604 8.61841454025125e-06
1605 8.57118993735639e-06
1606 8.60530417412519e-06
1607 8.62901833897922e-06
1608 8.70627991389483e-06
1609 8.95707398740342e-06
1610 9.20917955227196e-06
1611 9.24335563468048e-06
1612 9.05269098439021e-06
1613 8.79308663570555e-06
1614 8.64420053403592e-06
1615 8.56018505146494e-06
1616 8.57364921102999e-06
1617 8.59608371683862e-06
1618 8.62216438690666e-06
1619 8.77181264513638e-06
1620 9.04392891243333e-06
1621 9.23208608583082e-06
1622 9.15123928280082e-06
1623 8.88296563061886e-06
1624 8.67674134497065e-06
1625 8.57500162965152e-06
1626 8.53125311550684e-06
1627 8.55864436744014e-06
1628 8.59156898513902e-06
1629 8.65250513015781e-06
1630 8.88089653017232e-06
1631 9.15157943381928e-06
1632 9.22373328648973e-06
1633 9.03729960555211e-06
1634 8.76044578035362e-06
1635 8.59578722156584e-06
1636 8.52086031954968e-06
1637 8.52333869261201e-06
1638 8.54811696626712e-06
1639 8.58903240441578e-06
1640 8.72510463523213e-06
1641 9.00311715668067e-06
1642 9.19362992135575e-06
1643 9.09265781956492e-06
1644 8.82412950886646e-06
1645 8.62709293869557e-06
1646 8.54101290315157e-06
1647 8.49766001920216e-06
1648 8.53024812386138e-06
1649 8.54530935612274e-06
1650 8.63683362695156e-06
1651 8.88405611476628e-06
1652 9.12485938897589e-06
1653 9.13355506781954e-06
1654 8.8993519966607e-06
1655 8.66110895003658e-06
1656 8.54308746056631e-06
1657 8.47464525577379e-06
1658 8.48512172524352e-06
1659 8.49112802825402e-06
1660 8.56343467603438e-06
1661 8.7529451775481e-06
1662 9.01470048120245e-06
1663 9.15298460313352e-06
1664 9.01171733858064e-06
1665 8.72900363901863e-06
1666 8.55191410664702e-06
1667 8.47048704599729e-06
1668 8.45206432131818e-06
1669 8.48228319227928e-06
1670 8.49468415253796e-06
1671 8.61781700223219e-06
1672 8.8750357463141e-06
1673 9.09442405827576e-06
1674 9.06535387912299e-06
1675 8.80921197676798e-06
1676 8.58328439790057e-06
1677 8.48204126668861e-06
1678 8.42045483295806e-06
1679 8.43570069264388e-06
1680 8.45165959617589e-06
1681 8.52352968649939e-06
1682 8.71875454322435e-06
1683 8.99231599760242e-06
1684 9.10016387933865e-06
1685 8.95489029062446e-06
1686 8.67412109073484e-06
1687 8.50263404572615e-06
1688 8.41527344164206e-06
1689 8.40664506540634e-06
1690 8.44419173517963e-06
1691 8.45210524857976e-06
1692 8.57747181726154e-06
1693 8.8550814325572e-06
1694 9.06074728845851e-06
1695 8.99956285138614e-06
1696 8.74286160978954e-06
1697 8.53390793054132e-06
1698 8.43142515805084e-06
1699 8.38141568237916e-06
1700 8.40257962408941e-06
1701 8.42846020532306e-06
1702 8.50037122290814e-06
1703 8.72594955581008e-06
1704 8.98427242645994e-06
1705 9.03997442947002e-06
1706 8.83076972968411e-06
1707 8.57891973282676e-06
1708 8.44239184516482e-06
1709 8.3652257671929e-06
1710 8.37902098282939e-06
1711 8.39440053823637e-06
1712 8.41128621686948e-06
1713 8.59583815326914e-06
1714 8.87270471139345e-06
1715 9.03827549336711e-06
1716 8.91213949216763e-06
1717 8.64041976456065e-06
1718 8.45197973831091e-06
1719 8.35558603284881e-06
1720 8.3465538409655e-06
1721 8.37145307741594e-06
1722 8.3892846305389e-06
1723 8.50112246553181e-06
1724 8.77236107044155e-06
1725 9.02530337043572e-06
1726 9.02311785466736e-06
1727 8.76133890415076e-06
1728 8.51314780447865e-06
1729 8.38984124129638e-06
1730 8.32563455333002e-06
1731 8.35688388178824e-06
1732 8.3772310972563e-06
1733 8.43404905026546e-06
1734 8.6525005826843e-06
1735 8.91327272256603e-06
1736 8.98747293831548e-06
1737 8.79938124853652e-06
1738 8.53583969728788e-06
1739 8.38750202092342e-06
1740 8.31565193948336e-06
1741 8.33006106404355e-06
1742 8.33354897622485e-06
1743 8.38773394207237e-06
1744 8.56547740113456e-06
1745 8.82644053490367e-06
1746 8.97559129953152e-06
1747 8.83354550751392e-06
1748 8.55442431202391e-06
1749 8.39675976749277e-06
1750 8.30107455840334e-06
1751 8.307257303386e-06
1752 8.33651029097382e-06
1753 8.34416732686805e-06
1754 8.4939392763772e-06
1755 8.75932710187044e-06
1756 8.96546498552198e-06
1757 8.87764235812938e-06
1758 8.61752687342232e-06
1759 8.40304619487142e-06
1760 8.29728924145456e-06
1761 8.27149597171228e-06
1762 8.31776833365439e-06
1763 8.34557613416109e-06
1764 8.43701945996145e-06
1765 8.70562143973075e-06
1766 8.93860033102101e-06
1767 8.90108003659407e-06
1768 8.64228604768869e-06
1769 8.40559550852049e-06
1770 8.29910914035281e-06
1771 8.25399001769256e-06
1772 8.28099109639879e-06
1773 8.30650515126763e-06
1774 8.40361281007063e-06
1775 8.63858258526307e-06
1776 8.89217972144252e-06
1777 8.89952116267523e-06
1778 8.67213475430617e-06
1779 8.41309883981012e-06
1780 8.2947435657843e-06
1781 8.23244499770226e-06
1782 8.261911716545e-06
1783 8.27578969619935e-06
1784 8.3645072663785e-06
1785 8.57861505210167e-06
1786 8.83242046256782e-06
1787 8.88834074430633e-06
1788 8.69164978212211e-06
1789 8.42677218315657e-06
1790 8.29174950922607e-06
1791 8.21585763333132e-06
1792 8.24782091513043e-06
1793 8.27491840027506e-06
1794 8.3124396041967e-06
1795 8.50918604555773e-06
1796 8.78246282809414e-06
1797 8.89521652425174e-06
1798 8.72711461852305e-06
1799 8.45403064886341e-06
1800 8.29120835987851e-06
1801 8.20800050860271e-06
1802 8.2246669990127e-06
1803 8.25506231194595e-06
1804 8.2923306763405e-06
1805 8.45575232233386e-06
1806 8.74645684234565e-06
1807 8.90491173777264e-06
1808 8.78067658049986e-06
1809 8.4941921159043e-06
1810 8.29866530693835e-06
1811 8.20161403680686e-06
1812 8.2099468272645e-06
1813 8.23649861558806e-06
1814 8.26469204184832e-06
1815 8.41305791254854e-06
1816 8.67853668751195e-06
1817 8.86468023963971e-06
1818 8.77817547006998e-06
1819 8.49536081659608e-06
1820 8.29560849524569e-06
1821 8.19607703306247e-06
1822 8.18820717540802e-06
1823 8.22649872134207e-06
1824 8.25746610644273e-06
1825 8.39229505800176e-06
1826 8.67070684762439e-06
1827 8.86550242285011e-06
1828 8.75734531291528e-06
1829 8.48151103127748e-06
1830 8.27295389171923e-06
1831 8.17868112790165e-06
1832 8.17227646621177e-06
1833 8.20215427665971e-06
1834 8.22500715003116e-06
1835 8.37708739709342e-06
1836 8.64364210428903e-06
1837 8.8372835307382e-06
1838 8.71942484081956e-06
1839 8.43454017740441e-06
1840 8.2450314948801e-06
1841 8.15519160823897e-06
1842 8.16997908259509e-06
1843 8.19863180367975e-06
1844 8.22957554191817e-06
1845 8.39569293020759e-06
1846 8.68129882292124e-06
1847 8.87492387846578e-06
1848 8.73792032507481e-06
1849 8.43388079374563e-06
1850 8.26084124128101e-06
1851 8.15806561149657e-06
1852 8.16888314147945e-06
1853 8.20904915599385e-06
1854 8.23280242912006e-06
1855 8.40728625917109e-06
1856 8.67624021338997e-06
1857 8.83459688338917e-06
1858 8.68636470841011e-06
1859 8.39371023175772e-06
1860 8.2240521805943e-06
1861 8.14502527646255e-06
1862 8.1525395216886e-06
1863 8.1793732533697e-06
1864 8.23101527203107e-06
1865 8.41418477648403e-06
1866 8.69080668053357e-06
1867 8.80762581800809e-06
1868 8.63619334268151e-06
1869 8.34953425510321e-06
1870 8.19647630123654e-06
1871 8.11737572803395e-06
1872 8.16238753031939e-06
1873 8.17001910036197e-06
1874 8.23008213046705e-06
1875 8.46980310598155e-06
1876 8.76403464644682e-06
1877 8.83185293787392e-06
1878 8.62302294990513e-06
1879 8.33320609672228e-06
1880 8.20034256321378e-06
1881 8.11905556474812e-06
1882 8.1564421634539e-06
1883 8.17919863038696e-06
1884 8.23667505756021e-06
1885 8.48525087349117e-06
1886 8.73951648827642e-06
1887 8.77514503372367e-06
1888 8.55917642184068e-06
1889 8.28486918180715e-06
1890 8.15385465102736e-06
1891 8.10674282547552e-06
1892 8.14431678008987e-06
1893 8.16539250081405e-06
1894 8.26573523227125e-06
1895 8.52840639709029e-06
1896 8.75975456438027e-06
1897 8.71879001351772e-06
1898 8.46338298288174e-06
1899 8.23197751742555e-06
1900 8.11946483736392e-06
1901 8.09413904789835e-06
1902 8.12451980891638e-06
1903 8.16544252302265e-06
1904 8.31056695460575e-06
1905 8.572371370974e-06
1906 8.7700773292454e-06
1907 8.66687150846701e-06
1908 8.38616506371181e-06
1909 8.18145963421557e-06
1910 8.08736513135955e-06
1911 8.09282573754899e-06
1912 8.1153493738384e-06
1913 8.15622661320958e-06
1914 8.34543425298762e-06
1915 8.60951786307851e-06
1916 8.75047044246458e-06
1917 8.58583916851785e-06
1918 8.30370481708087e-06
1919 8.14260238257702e-06
1920 8.0597410487826e-06
1921 8.09630000730976e-06
1922 8.11736299510812e-06
1923 8.15893145045266e-06
1924 8.3862105384469e-06
1925 8.64632420416456e-06
1926 8.71048814587994e-06
1927 8.50177457323298e-06
1928 8.249908205471e-06
1929 8.11477548268158e-06
1930 8.04946921562077e-06
1931 8.07873311714502e-06
1932 8.09145876701223e-06
1933 8.20208606455708e-06
1934 8.42798090161523e-06
1935 8.66494883666746e-06
1936 8.68038023327244e-06
1937 8.45768590806983e-06
1938 8.21077628643252e-06
1939 8.07613378128735e-06
1940 8.03463262855075e-06
1941 8.07555534265703e-06
1942 8.08494223747402e-06
1943 8.20713376015192e-06
1944 8.44923124532215e-06
1945 8.6950294644339e-06
1946 8.68008828547318e-06
1947 8.41992550704163e-06
1948 8.18335956864757e-06
1949 8.06118987384252e-06
1950 8.02612794359447e-06
1951 8.06916887086118e-06
1952 8.10952496976824e-06
1953 8.21200319478521e-06
1954 8.47290084493579e-06
1955 8.70034637046047e-06
1956 8.63731293065939e-06
1957 8.36306116980268e-06
1958 8.14482427813346e-06
1959 8.02943395683542e-06
1960 8.03128750703763e-06
1961 8.06478237791453e-06
1962 8.09224275144516e-06
1963 8.24989729153458e-06
1964 8.53427809488494e-06
1965 8.71319480211241e-06
1966 8.56752376421355e-06
1967 8.29059626994422e-06
1968 8.10664096206892e-06
1969 8.00898487796076e-06
1970 8.02187241788488e-06
1971 8.05641593615292e-06
1972 8.0980626080418e-06
1973 8.30186309030978e-06
1974 8.56485075928504e-06
1975 8.68269762577256e-06
1976 8.48784384288592e-06
1977 8.21614230517298e-06
1978 8.06442221801262e-06
1979 7.99289500719169e-06
1980 8.03018156148028e-06
1981 8.06528714747401e-06
1982 8.13009592093294e-06
1983 8.3544009612524e-06
1984 8.60770433064317e-06
1985 8.63036075315904e-06
1986 8.39329459267901e-06
1987 8.15099247120088e-06
1988 8.0350691860076e-06
1989 7.99789722805144e-06
1990 8.03068087407155e-06
1991 8.0486606748309e-06
1992 8.16543251858093e-06
1993 8.42501776787685e-06
1994 8.65006313688355e-06
1995 8.58264502312522e-06
1996 8.31624129205011e-06
1997 8.09942594059976e-06
1998 7.99354711489286e-06
1999 7.98953715275275e-06
};
\addlegendentry{Test}

\nextgroupplot[
title={3 Layer},
ymin=7.2623442747922e-06, ymax=0.01,
]
\addplot [semithick, black, dashed]
table {%
0 0.00623320222439361
1 0.00621069033877575
2 0.00619208805801463
3 0.00617619798504165
4 0.00616259937669383
5 0.00615093815576984
6 0.00614089029113529
7 0.00613218327998766
8 0.00612459828698775
9 0.00611799261241686
10 0.00611223152009188
11 0.00610715135917417
12 0.00610265700015589
13 0.00609869432810228
14 0.00609518969213241
15 0.00609209828689927
16 0.00608933484909358
17 0.00608683585960534
18 0.00608457787529915
19 0.00608250952791423
20 0.00608056735291029
21 0.00607872281034361
22 0.00607697850500699
23 0.00607534516893793
24 0.00607374154060381
25 0.00607214796036715
26 0.00607047540142958
27 0.00606867103851982
28 0.0060667178240692
29 0.0060646334222838
30 0.00606236966814322
31 0.00605977422310389
32 0.00605654934770428
33 0.00605258451651025
34 0.00604766286232916
35 0.00604162899799121
36 0.00603436258643342
37 0.00602550243274891
38 0.00601474719223916
39 0.00600164554998628
40 0.00598561385049834
41 0.00596615823815227
42 0.00594262204322149
43 0.00591393203285406
44 0.00587970231572399
45 0.00583941501099616
46 0.00579223276508856
47 0.00573735094440053
48 0.0056738651728665
49 0.00560060172574595
50 0.00551620228361571
51 0.00542046879854752
52 0.00531311852682848
53 0.00519425337188295
54 0.00506594746184419
55 0.004934117268931
56 0.00480115347090759
57 0.00466742490971228
58 0.00453611400189402
59 0.00440927894851484
60 0.00428710225969553
61 0.00417405524785863
62 0.00406916688552883
63 0.00397038900518965
64 0.00387646980379941
65 0.00378624009681516
66 0.00369875389333174
67 0.00361267974585644
68 0.00353029230973334
69 0.0034494837946113
70 0.0033695483525662
71 0.00329205009438738
72 0.00321853038803965
73 0.00314898631040705
74 0.00308173566372716
75 0.00301629125533509
76 0.00295211954380648
77 0.00288690138040693
78 0.00282073895687063
79 0.00275556582437275
80 0.00269198958085326
81 0.00262977869533643
82 0.00256932893717021
83 0.00251000166008453
84 0.00245061744135455
85 0.0023908958892207
86 0.00233145862603124
87 0.00227039239325677
88 0.00220763551988057
89 0.00214390746259596
90 0.00208000529892161
91 0.00201726285104087
92 0.00195470412018039
93 0.00189239426708809
94 0.00182961588961916
95 0.00176732031832216
96 0.00170640254418686
97 0.00164684982837571
98 0.00158740212782504
99 0.00152879763709279
100 0.00147225844557397
101 0.00141724268132748
102 0.00136307091270282
103 0.00131013067766617
104 0.00125727458407709
105 0.00120613787521506
106 0.00115796830732506
107 0.00111297686225953
108 0.00107063164250576
109 0.00103097649525807
110 0.000993828909486183
111 0.00095846015983625
112 0.000924721362025593
113 0.000891988493094686
114 0.000860501183296947
115 0.000830620706665286
116 0.00080294452641283
117 0.000776939920342556
118 0.000752104331468217
119 0.000727392909311675
120 0.000701530368360181
121 0.00067660432091543
122 0.000653867279879705
123 0.000633155027912835
124 0.00061412202376232
125 0.000596774881614692
126 0.0005804839091752
127 0.000565262879604234
128 0.000551254168840387
129 0.000538094628154795
130 0.000524926521279667
131 0.000511751840804209
132 0.000499454315843195
133 0.00048774935100937
134 0.000476768061218991
135 0.000466110856848445
136 0.000455567129165502
137 0.000445541681528994
138 0.000436029009506456
139 0.000427185559374266
140 0.000419001478292103
141 0.000411367319657074
142 0.000404240528752098
143 0.000397553274865459
144 0.000391299774776144
145 0.000385285999641383
146 0.000379345376501306
147 0.000373788637631378
148 0.000368553314302744
149 0.000363592439299509
150 0.000358881689919599
151 0.000354397929498873
152 0.000350095637230652
153 0.000345980366162735
154 0.000342035836723653
155 0.000338234523638903
156 0.00033456437418522
157 0.000331004799903667
158 0.000327484681577062
159 0.00032390155816131
160 0.000320462891068019
161 0.000317087191064047
162 0.000313789705941758
163 0.000310638379012573
164 0.000307622489472692
165 0.000304677024054456
166 0.000301783624614416
167 0.000298934381476101
168 0.000296103563314887
169 0.000293098699003735
170 0.000290248906537727
171 0.000287492352818219
172 0.000284771617167223
173 0.000282077336684949
174 0.000279406637474722
175 0.000276773982541556
176 0.000274177179782953
177 0.000271605212347481
178 0.000269055927141437
179 0.00026646031127342
180 0.000263731022393188
181 0.000260966254565176
182 0.00025822536595399
183 0.000255524020186471
184 0.00025285536081654
185 0.000250204869310267
186 0.000247573539237322
187 0.00024495500867161
188 0.000242370594026653
189 0.000239801016341801
190 0.000237258020469255
191 0.000234715081823822
192 0.000232179514451758
193 0.000229657293605356
194 0.000227132112005535
195 0.000224623485163278
196 0.0002221370922868
197 0.000219646154071995
198 0.000217157422987668
199 0.000214684156464529
200 0.00021222774832097
201 0.000209761331454672
202 0.000207315209419079
203 0.000204873269641581
204 0.000202419795115816
205 0.000199999448845745
206 0.000197560922462259
207 0.000195118857170939
208 0.000192710029526921
209 0.000190327301247351
210 0.000187971638665019
211 0.000185631050394264
212 0.000183308955490347
213 0.000181008075772482
214 0.000178734959051852
215 0.000176493899845198
216 0.000174266388611954
217 0.000172072502987675
218 0.000169912335659106
219 0.000167769405939211
220 0.00016564114781481
221 0.000163521322649629
222 0.00016142805174546
223 0.000159348548166349
224 0.000157174542191996
225 0.000154843464514443
226 0.000152555947634525
227 0.000150351360781897
228 0.000148284435359969
229 0.000146297906130854
230 0.000144413740150284
231 0.000142585108775961
232 0.000140808056290354
233 0.000139067840843055
234 0.000137366272291217
235 0.000135717297801818
236 0.000134082517055489
237 0.000132504838216363
238 0.000130967715355368
239 0.000129467175751685
240 0.000127995224371347
241 0.00012657161001961
242 0.000125161580198707
243 0.000123801805242962
244 0.000122468310005175
245 0.000121160725901959
246 0.000119902625780099
247 0.000118670639306373
248 0.000117465551076279
249 0.000116275873056182
250 0.000115102187756122
251 0.000113964465951355
252 0.000112846411029821
253 0.00011175452540968
254 0.00011069233928751
255 0.000109659633576342
256 0.000108634165611576
257 0.000107634786131072
258 0.000106645639059266
259 0.000105683128138523
260 0.000104725121632399
261 0.000103790854861785
262 0.000102868121357957
263 0.000101965872953969
264 0.000101081843318696
265 0.000100215530702386
266 9.93652432974557e-05
267 9.85259851020714e-05
268 9.77193354927408e-05
269 9.69242856978525e-05
270 9.61454183716626e-05
271 9.53882454926003e-05
272 9.46547783957641e-05
273 9.39264573958098e-05
274 9.32192683116284e-05
275 9.25198092165402e-05
276 9.18378618752769e-05
277 9.11647302110907e-05
278 9.05157069865936e-05
279 8.98802264543974e-05
280 8.92548904438684e-05
281 8.86316022654654e-05
282 8.80301913532833e-05
283 8.74332268274713e-05
284 8.68534380051855e-05
285 8.62769341933856e-05
286 8.57188367335837e-05
287 8.51714042280349e-05
288 8.46301729078291e-05
289 8.41068267298795e-05
290 8.3582643601865e-05
291 8.30793175889966e-05
292 8.2571191640568e-05
293 8.20994729178182e-05
294 8.16268634284967e-05
295 8.11462815066832e-05
296 8.06808433324591e-05
297 8.02242799080588e-05
298 7.9775106663682e-05
299 7.93307053026382e-05
300 7.88915059457551e-05
301 7.84580103569965e-05
302 7.80129743862545e-05
303 7.75719498520289e-05
304 7.71377877661905e-05
305 7.67172533260307e-05
306 7.63083748935856e-05
307 7.59002199117731e-05
308 7.5499286996461e-05
309 7.51062076744802e-05
310 7.47228765813901e-05
311 7.43376991803046e-05
312 7.39634967814595e-05
313 7.35930349273772e-05
314 7.32273588974763e-05
315 7.28750224459063e-05
316 7.25164977026793e-05
317 7.21693501120058e-05
318 7.18184996664206e-05
319 7.1480269930646e-05
320 7.11401753097363e-05
321 7.08006531375815e-05
322 7.04709649168578e-05
323 7.01485550962389e-05
324 6.98289172795796e-05
325 6.95096856722088e-05
326 6.91987758667523e-05
327 6.88857020323397e-05
328 6.85814676133489e-05
329 6.82797818152281e-05
330 6.79749154741671e-05
331 6.76722119621331e-05
332 6.7377158355697e-05
333 6.70837944474201e-05
334 6.67802952190044e-05
335 6.64870702369313e-05
336 6.61992870032435e-05
337 6.59110838681087e-05
338 6.5627462162432e-05
339 6.53382308710704e-05
340 6.50496134619871e-05
341 6.47656427723575e-05
342 6.44788319812406e-05
343 6.4205398111028e-05
344 6.392755531337e-05
345 6.36537196530007e-05
346 6.33814961319423e-05
347 6.30952410531904e-05
348 6.28217916647245e-05
349 6.25603681605469e-05
350 6.22882823684279e-05
351 6.20182791108981e-05
352 6.17497785122545e-05
353 6.14830917347042e-05
354 6.1226855930574e-05
355 6.09615289670629e-05
356 6.06992325051792e-05
357 6.04416134537544e-05
358 6.01861721900576e-05
359 5.99301190447044e-05
360 5.96791342197811e-05
361 5.94266009166233e-05
362 5.91763810144919e-05
363 5.89368682164526e-05
364 5.87013580237716e-05
365 5.8457094466835e-05
366 5.82176820502767e-05
367 5.79780740985747e-05
368 5.7752029564373e-05
369 5.75165472218586e-05
370 5.72838490455752e-05
371 5.70429880655254e-05
372 5.68208842413043e-05
373 5.65987159575343e-05
374 5.6374262965786e-05
375 5.61431683059332e-05
376 5.59113289768653e-05
377 5.56883974534017e-05
378 5.5465479360528e-05
379 5.52482931883702e-05
380 5.50226527376019e-05
381 5.47985822105801e-05
382 5.45827253066733e-05
383 5.43664086949391e-05
384 5.41512709357761e-05
385 5.3919535522251e-05
386 5.36972196267804e-05
387 5.34781997028055e-05
388 5.32627928517826e-05
389 5.30490323176203e-05
390 5.28376261761387e-05
391 5.26179572331031e-05
392 5.24008151980837e-05
393 5.21733915648248e-05
394 5.19563664682643e-05
395 5.17461712270517e-05
396 5.15319839529127e-05
397 5.13205693519758e-05
398 5.11090822743476e-05
399 5.09072534558008e-05
400 5.07056074745549e-05
401 5.04929616940153e-05
402 5.02985893664487e-05
403 5.01017357681377e-05
404 4.9910058022018e-05
405 4.97171002034058e-05
406 4.95328590122135e-05
407 4.93464319766268e-05
408 4.91618498323909e-05
409 4.8980250154429e-05
410 4.88018623747699e-05
411 4.86245704620636e-05
412 4.8437281421787e-05
413 4.82584240173445e-05
414 4.80819451307291e-05
415 4.79032642268784e-05
416 4.77213016676359e-05
417 4.7550336040203e-05
418 4.73816494359625e-05
419 4.72103912745325e-05
420 4.70395574159177e-05
421 4.68760395078149e-05
422 4.67146787448414e-05
423 4.65562986109447e-05
424 4.63865433339983e-05
425 4.62330118691234e-05
426 4.60705651872395e-05
427 4.59155397782851e-05
428 4.57533477269401e-05
429 4.55983923286496e-05
430 4.54471076669627e-05
431 4.52867650841959e-05
432 4.51419418294741e-05
433 4.49957835626513e-05
434 4.48437225890608e-05
435 4.46995885052814e-05
436 4.45512196680653e-05
437 4.44072215666225e-05
438 4.42639001048128e-05
439 4.41132005635225e-05
440 4.39656229191954e-05
441 4.38215394331642e-05
442 4.36692397558502e-05
443 4.35251616681676e-05
444 4.33811555033969e-05
445 4.32356294175484e-05
446 4.31039411097345e-05
447 4.29655887934643e-05
448 4.28303204325076e-05
449 4.2692233831243e-05
450 4.25593103261512e-05
451 4.24302793611275e-05
452 4.23004440093244e-05
453 4.21652176800791e-05
454 4.20365279087953e-05
455 4.19021718678358e-05
456 4.17797538023734e-05
457 4.16491500474514e-05
458 4.15244222815581e-05
459 4.13983226255077e-05
460 4.12831215088616e-05
461 4.11601544647588e-05
462 4.10465709377306e-05
463 4.09282623992091e-05
464 4.08161027740661e-05
465 4.06929652378984e-05
466 4.05888465024873e-05
467 4.04669918872003e-05
468 4.03525510233038e-05
469 4.02341653868987e-05
470 4.01271727019648e-05
471 3.99985362760091e-05
472 3.9874944581797e-05
473 3.97435376786603e-05
474 3.96314437831613e-05
475 3.95062469089602e-05
476 3.94029665637952e-05
477 3.92774235384863e-05
478 3.91683581035807e-05
479 3.90441194482172e-05
480 3.89377667779911e-05
481 3.88323741771224e-05
482 3.87207065895012e-05
483 3.86200249273827e-05
484 3.85170247581001e-05
485 3.84244399804778e-05
486 3.83103531347295e-05
487 3.82255754405314e-05
488 3.81207508688419e-05
489 3.80262619201766e-05
490 3.79282668729886e-05
491 3.78339638613667e-05
492 3.77328498797347e-05
493 3.76271658195648e-05
494 3.75340240523769e-05
495 3.7419598807098e-05
496 3.7314116122289e-05
497 3.72117001745664e-05
498 3.71050383107807e-05
499 3.70030645542307e-05
500 3.68958333809566e-05
501 3.68035355577945e-05
502 3.67043846591919e-05
503 3.66063166934438e-05
504 3.65273467322069e-05
505 3.64400669292664e-05
506 3.63577049586894e-05
507 3.62946758465377e-05
508 3.62330320471393e-05
509 3.619713249714e-05
510 3.61648127262804e-05
511 3.61636251886566e-05
512 3.61766193801571e-05
513 3.62034259064892e-05
514 3.62305509669092e-05
515 3.62201035635223e-05
516 3.60832273824663e-05
517 3.58838968455011e-05
518 3.56085434631837e-05
519 3.53567278388311e-05
520 3.51929527848682e-05
521 3.51485179876931e-05
522 3.52745013065814e-05
523 3.55632716386722e-05
524 3.59320465284085e-05
525 3.61088334228654e-05
526 3.58134317934855e-05
527 3.51653219934356e-05
528 3.46242134310337e-05
529 3.4462759707754e-05
530 3.45530568748309e-05
531 3.47076550903935e-05
532 3.47771211801629e-05
533 3.46516462137458e-05
534 3.44217537384672e-05
535 3.41573087756508e-05
536 3.39440665730883e-05
537 3.37843523263714e-05
538 3.3690522109886e-05
539 3.36108014185044e-05
540 3.35371850734845e-05
541 3.34639371697421e-05
542 3.33923796338453e-05
543 3.33209833378589e-05
544 3.32662666790995e-05
545 3.32035285541821e-05
546 3.31613334756398e-05
547 3.31246656450901e-05
548 3.30913622588191e-05
549 3.30784271400653e-05
550 3.30454993120455e-05
551 3.30202414060565e-05
552 3.2966383924915e-05
553 3.29206464022569e-05
554 3.2828842091881e-05
555 3.27457218589622e-05
556 3.26359273845611e-05
557 3.25250866808346e-05
558 3.24108301237302e-05
559 3.23146344465286e-05
560 3.22251221849257e-05
561 3.21321019489673e-05
562 3.20706372800572e-05
563 3.2007660643707e-05
564 3.19522858864829e-05
565 3.1907215585214e-05
566 3.18638818415096e-05
567 3.18288116281096e-05
568 3.17894432475541e-05
569 3.17514001668684e-05
570 3.16988862083534e-05
571 3.16371032464957e-05
572 3.15562482731124e-05
573 3.14548390121239e-05
574 3.13452689537641e-05
575 3.12629144136878e-05
576 3.12432458930978e-05
577 3.13647418277441e-05
578 3.18073770269045e-05
579 3.26952873499486e-05
580 3.38602941645227e-05
581 3.425435329385e-05
582 3.30911086190433e-05
583 3.15843353035916e-05
584 3.09741936490582e-05
585 3.09598957146662e-05
586 3.10079279071118e-05
587 3.08910605797408e-05
588 3.0666344611241e-05
589 3.04602650018637e-05
590 3.03140529753776e-05
591 3.02416033899355e-05
592 3.02183076232865e-05
593 3.01839584762043e-05
594 3.01553566144364e-05
595 3.0111963020385e-05
596 3.00400351513908e-05
597 2.99751760755029e-05
598 2.98954970441301e-05
599 2.98282012756168e-05
600 2.97549314631773e-05
601 2.96869908038566e-05
602 2.96362677989492e-05
603 2.95677130242034e-05
604 2.95235546960981e-05
605 2.94638728091101e-05
606 2.94143854793205e-05
607 2.93801522399662e-05
608 2.93312011270253e-05
609 2.93079841013899e-05
610 2.92704861766246e-05
611 2.92494601481508e-05
612 2.92293082950756e-05
613 2.92041296181544e-05
614 2.91918028585414e-05
615 2.91639644585118e-05
616 2.91291529324056e-05
617 2.90798793360914e-05
618 2.90179406885471e-05
619 2.89453197934719e-05
620 2.88503557435149e-05
621 2.87617089136383e-05
622 2.86654902095229e-05
623 2.85791325538298e-05
624 2.85067385306093e-05
625 2.84379736577023e-05
626 2.84138694324199e-05
627 2.8402946368189e-05
628 2.8446527736925e-05
629 2.85493886185151e-05
630 2.87405172878508e-05
631 2.89878508192842e-05
632 2.92284214671312e-05
633 2.92636516991962e-05
634 2.89839025744421e-05
635 2.85025425332819e-05
636 2.81343210919438e-05
637 2.81471955512202e-05
638 2.85687423651204e-05
639 2.92679667570628e-05
640 2.97616197355577e-05
641 2.95752937735649e-05
642 2.87932986235973e-05
643 2.80580146925224e-05
644 2.76764054119027e-05
645 2.75584386519867e-05
646 2.75119657979772e-05
647 2.74660015620043e-05
648 2.73940203650014e-05
649 2.73249403619502e-05
650 2.72821140883117e-05
651 2.72421655052035e-05
652 2.72239622405301e-05
653 2.72332727733193e-05
654 2.72158134779765e-05
655 2.72177430717946e-05
656 2.71933598994245e-05
657 2.71535165934367e-05
658 2.71060842695903e-05
659 2.70497079339904e-05
660 2.69889414092894e-05
661 2.69247592683541e-05
662 2.68658167548175e-05
663 2.68033674410972e-05
664 2.67508961102614e-05
665 2.67055916247649e-05
666 2.66535417994263e-05
667 2.66205396943775e-05
668 2.65703310944332e-05
669 2.6526475888744e-05
670 2.6504112576653e-05
671 2.64631629636192e-05
672 2.64342438347498e-05
673 2.64156754470335e-05
674 2.641106395318e-05
675 2.64228648090281e-05
676 2.64764064374923e-05
677 2.65757751729545e-05
678 2.67272158978038e-05
679 2.69318127394946e-05
680 2.71257175290529e-05
681 2.72309936133475e-05
682 2.71111478462593e-05
683 2.68011402244639e-05
684 2.64175384305787e-05
685 2.61528406104716e-05
686 2.60988721438871e-05
687 2.62922156184509e-05
688 2.66737068228551e-05
689 2.71303662162659e-05
690 2.7338111852071e-05
691 2.70400073194565e-05
692 2.63924782872849e-05
693 2.58853103547452e-05
694 2.57437864075172e-05
695 2.5833181948709e-05
696 2.59869971999116e-05
697 2.60370320150916e-05
698 2.59604205483654e-05
699 2.57872271531845e-05
700 2.55992316553488e-05
701 2.5452979268259e-05
702 2.53641777656632e-05
703 2.53121499591558e-05
704 2.52862050222902e-05
705 2.52627192653954e-05
706 2.52260601447674e-05
707 2.51952731487037e-05
708 2.5148637063932e-05
709 2.51105575008737e-05
710 2.50624756779061e-05
711 2.50267019046291e-05
712 2.50145081053255e-05
713 2.50206598839675e-05
714 2.50349945392081e-05
715 2.50769076681756e-05
716 2.51191030016251e-05
717 2.51800361934507e-05
718 2.5208817719502e-05
719 2.52250039189938e-05
720 2.5176416227346e-05
721 2.50895727229583e-05
722 2.49786580797107e-05
723 2.48307289609784e-05
724 2.47224811618096e-05
725 2.46311426845836e-05
726 2.46113346773491e-05
727 2.46508179784222e-05
728 2.47725848319114e-05
729 2.49994282692789e-05
730 2.52924507506691e-05
731 2.55530466781728e-05
732 2.55999825786724e-05
733 2.53175816773421e-05
734 2.48414865744451e-05
735 2.44866093854768e-05
736 2.44640941449159e-05
737 2.47397155037632e-05
738 2.51641663571167e-05
739 2.54802096519846e-05
740 2.54057410513653e-05
741 2.49988356508801e-05
742 2.45068114250735e-05
743 2.41886802809788e-05
744 2.40554849777475e-05
745 2.40129226014574e-05
746 2.40096100672105e-05
747 2.39980903447012e-05
748 2.39454450432497e-05
749 2.38897221587209e-05
750 2.38302772177512e-05
751 2.37801459817e-05
752 2.37736921473441e-05
753 2.37787900445596e-05
754 2.37940989675778e-05
755 2.38216181984718e-05
756 2.38493088167147e-05
757 2.38632050142407e-05
758 2.38477662954928e-05
759 2.38209467680051e-05
760 2.37584127673607e-05
761 2.36911762971914e-05
762 2.36110380544829e-05
763 2.35391350758007e-05
764 2.34810560777454e-05
765 2.34277902997349e-05
766 2.33988261002516e-05
767 2.33849283088894e-05
768 2.33819068169439e-05
769 2.34086744352879e-05
770 2.34559035936499e-05
771 2.35140698534586e-05
772 2.35847548828438e-05
773 2.36472394901099e-05
774 2.36614116526646e-05
775 2.36197414422712e-05
776 2.34841818409492e-05
777 2.33083769844988e-05
778 2.31651715476566e-05
779 2.31465603754977e-05
780 2.3316662501438e-05
781 2.37967167728836e-05
782 2.45427602636461e-05
783 2.52767784394425e-05
784 2.53277671264929e-05
785 2.44976680849884e-05
786 2.35154641710089e-05
787 2.30285799309726e-05
788 2.29977730148434e-05
789 2.3089168891488e-05
790 2.31277698770072e-05
791 2.30318201439594e-05
792 2.2861064521873e-05
793 2.27055749650873e-05
794 2.2623507045072e-05
795 2.26007888035618e-05
796 2.26227386079803e-05
797 2.26470555599878e-05
798 2.26585369447196e-05
799 2.26405147878417e-05
800 2.25949001340098e-05
801 2.25267506479554e-05
802 2.24722523620713e-05
803 2.2415551645949e-05
804 2.23668235665997e-05
805 2.23327131263495e-05
806 2.22950427617441e-05
807 2.22644644356507e-05
808 2.2245229507778e-05
809 2.22173078348042e-05
810 2.21857343398746e-05
811 2.21728449032454e-05
812 2.21355889169494e-05
813 2.21136649445342e-05
814 2.20924077023454e-05
815 2.20822267080756e-05
816 2.20962780383971e-05
817 2.21252066554545e-05
818 2.21962355757555e-05
819 2.23284523075407e-05
820 2.25342398731954e-05
821 2.27922932694025e-05
822 2.30549707205263e-05
823 2.31623083704058e-05
824 2.30134139216354e-05
825 2.26321831986454e-05
826 2.22058034835548e-05
827 2.19553814639895e-05
828 2.19637570109121e-05
829 2.22231735933498e-05
830 2.2647361037631e-05
831 2.30270417089784e-05
832 2.30594038912457e-05
833 2.26348839689194e-05
834 2.20542020834547e-05
835 2.1692367729198e-05
836 2.16564725503687e-05
837 2.17753824083289e-05
838 2.18909151290525e-05
839 2.18901276711758e-05
840 2.17761249490245e-05
841 2.15954175710742e-05
842 2.14326210778992e-05
843 2.13150722094824e-05
844 2.12509203731948e-05
845 2.12255319285148e-05
846 2.12084259190171e-05
847 2.1184348885317e-05
848 2.1160192726466e-05
849 2.1116811698807e-05
850 2.10648329166396e-05
851 2.10156078761514e-05
852 2.09708555161114e-05
853 2.09395237111831e-05
854 2.09303810163597e-05
855 2.09407095512759e-05
856 2.0965433857878e-05
857 2.10139599121639e-05
858 2.1072246173981e-05
859 2.1127416886646e-05
860 2.11614117571202e-05
861 2.11548262551986e-05
862 2.10917055825099e-05
863 2.09725567128505e-05
864 2.0823452848262e-05
865 2.06789401873131e-05
866 2.05735662910378e-05
867 2.0526362496831e-05
868 2.05746329258183e-05
869 2.0717599195752e-05
870 2.09858470316959e-05
871 2.13361529231015e-05
872 2.16315612027174e-05
873 2.16462049826838e-05
874 2.1280608940133e-05
875 2.07463130335039e-05
876 2.03955773196984e-05
877 2.03944210820373e-05
878 2.06745585547008e-05
879 2.10354887961728e-05
880 2.12272934950741e-05
881 2.10917417293732e-05
882 2.07020196931751e-05
883 2.02963623436858e-05
884 2.00407993973251e-05
885 1.99419550011015e-05
886 1.9927525912955e-05
887 1.99393712265739e-05
888 1.99291655587075e-05
889 1.98899518033535e-05
890 1.98330084901777e-05
891 1.97667291179648e-05
892 1.97153862506738e-05
893 1.96861363304102e-05
894 1.96848102649305e-05
895 1.96991009726588e-05
896 1.97262284544752e-05
897 1.97589934001208e-05
898 1.97743539971373e-05
899 1.97791653417934e-05
900 1.9748312930723e-05
901 1.96986215024175e-05
902 1.96292119900576e-05
903 1.9558151640009e-05
904 1.94799719888117e-05
905 1.94193233746809e-05
906 1.93746747041423e-05
907 1.93579270191258e-05
908 1.93692377026267e-05
909 1.94269100561861e-05
910 1.95343603466291e-05
911 1.96875027551613e-05
912 1.98756718283732e-05
913 2.00368784679661e-05
914 2.00850775593509e-05
915 1.99498017294797e-05
916 1.96598407609105e-05
917 1.93682863303124e-05
918 1.9237417931306e-05
919 1.93771969665413e-05
920 1.98010374414803e-05
921 2.03806237080162e-05
922 2.07914562280198e-05
923 2.06644586583371e-05
924 2.00579677178681e-05
925 1.94073441601716e-05
926 1.90490482370187e-05
927 1.89639855991963e-05
928 1.89963639432023e-05
929 1.9031651929069e-05
930 1.90033941280809e-05
931 1.89235939256349e-05
932 1.88246521917179e-05
933 1.87448473996099e-05
934 1.87050758853058e-05
935 1.87050921638399e-05
936 1.87328737288883e-05
937 1.87661857451449e-05
938 1.87911244284855e-05
939 1.87898384556018e-05
940 1.87596328282691e-05
941 1.87125854613779e-05
942 1.86520658200795e-05
943 1.85878620919677e-05
944 1.85258164715485e-05
945 1.84781402028378e-05
946 1.84440792647678e-05
947 1.84200468491547e-05
948 1.84084551744235e-05
949 1.84110995480324e-05
950 1.84224431709623e-05
951 1.84377243983569e-05
952 1.84607908892964e-05
953 1.84837588785847e-05
954 1.84972584840182e-05
955 1.85072398299901e-05
956 1.84949297379333e-05
957 1.84451105931238e-05
958 1.83796622097443e-05
959 1.82893159836883e-05
960 1.82124752636792e-05
961 1.8174793933845e-05
962 1.82366856908978e-05
963 1.84844936743644e-05
964 1.90094654324291e-05
965 1.98494222362067e-05
966 2.06936968574212e-05
967 2.07676208912044e-05
968 1.97932320262773e-05
969 1.86861433180141e-05
970 1.82203400451719e-05
971 1.82998263316136e-05
972 1.85273363197958e-05
973 1.86062390183928e-05
974 1.84397509879197e-05
975 1.81485980894625e-05
976 1.79118137348944e-05
977 1.78012330460664e-05
978 1.77826005660631e-05
979 1.78023987051912e-05
980 1.78085008315509e-05
981 1.77901962949978e-05
982 1.77463719768234e-05
983 1.76880747473351e-05
984 1.76363104573696e-05
985 1.7590505460463e-05
986 1.75543988327753e-05
987 1.75246130975637e-05
988 1.75044587864548e-05
989 1.74844464155832e-05
990 1.74627991391674e-05
991 1.74382277942087e-05
992 1.74156273016202e-05
993 1.73901746776117e-05
994 1.7368098350401e-05
995 1.73453562575254e-05
996 1.73251017228093e-05
997 1.73128620160057e-05
998 1.73107713941079e-05
999 1.73198947752695e-05
1000 1.73472365336824e-05
1001 1.73986655729319e-05
1002 1.74805104802278e-05
1003 1.76017755715296e-05
1004 1.77554774423161e-05
1005 1.79031532452356e-05
1006 1.80100804303418e-05
1007 1.79960790946954e-05
1008 1.78271886972503e-05
1009 1.75584976402909e-05
1010 1.73012759785429e-05
1011 1.71686047050201e-05
1012 1.72361170973634e-05
1013 1.75731364377985e-05
1014 1.82026285306947e-05
1015 1.89468768097001e-05
1016 1.92941591208307e-05
1017 1.87667949913539e-05
1018 1.77726636652054e-05
1019 1.71662374609527e-05
1020 1.71300905715821e-05
1021 1.73172199373628e-05
1022 1.74176518418223e-05
1023 1.73150529853006e-05
1024 1.71025255333301e-05
1025 1.68963253832644e-05
1026 1.67679157567768e-05
1027 1.6712271101671e-05
1028 1.66977594369122e-05
1029 1.66922638853428e-05
1030 1.66790047084753e-05
1031 1.66557849397009e-05
1032 1.66298138237764e-05
1033 1.65998566650227e-05
1034 1.65764927821943e-05
1035 1.65589329075289e-05
1036 1.65498955528864e-05
1037 1.65422072198051e-05
1038 1.65342960753989e-05
1039 1.65322949499114e-05
1040 1.65266104061867e-05
1041 1.65166364025637e-05
1042 1.64968217615247e-05
1043 1.6480483364445e-05
1044 1.64546970080615e-05
1045 1.64297587064155e-05
1046 1.63984754300017e-05
1047 1.63660501861784e-05
1048 1.63373731878469e-05
1049 1.63159258397272e-05
1050 1.6293679565127e-05
1051 1.62842683524289e-05
1052 1.62962291558877e-05
1053 1.63388525609154e-05
1054 1.64402074733783e-05
1055 1.66369229517294e-05
1056 1.698801072747e-05
1057 1.74999256412622e-05
1058 1.80567332241921e-05
1059 1.82393279590798e-05
1060 1.77428756236164e-05
1061 1.69050559967188e-05
1062 1.64749957075605e-05
1063 1.6747234158454e-05
1064 1.74459618391132e-05
1065 1.79653125352841e-05
1066 1.77742997737873e-05
1067 1.70358809561399e-05
1068 1.6371534363735e-05
1069 1.60494955561141e-05
1070 1.59613918722101e-05
1071 1.59410419273698e-05
1072 1.59160349282317e-05
1073 1.58810302766099e-05
1074 1.58461156014589e-05
1075 1.58247684245438e-05
1076 1.58199501236744e-05
1077 1.58188390575464e-05
1078 1.58150748281738e-05
1079 1.58105938481423e-05
1080 1.57898145314306e-05
1081 1.57753779024272e-05
1082 1.57415724499899e-05
1083 1.57145854720042e-05
1084 1.56818958809701e-05
1085 1.56597631084665e-05
1086 1.56332626994882e-05
1087 1.56126321764205e-05
1088 1.55899845775131e-05
1089 1.5572807370301e-05
1090 1.55543009219272e-05
1091 1.55382404929405e-05
1092 1.5520572803851e-05
1093 1.55036387119445e-05
1094 1.54849741864638e-05
1095 1.54686096180967e-05
1096 1.54485255805525e-05
1097 1.54305960609058e-05
1098 1.54082649759246e-05
1099 1.53881783120369e-05
1100 1.53654625321131e-05
1101 1.53446667625623e-05
1102 1.53323698057761e-05
1103 1.53306019965438e-05
1104 1.53487221736803e-05
1105 1.54029988368443e-05
1106 1.55308762810336e-05
1107 1.578013366621e-05
1108 1.62131595911053e-05
1109 1.68471311634555e-05
1110 1.74422198616631e-05
1111 1.75220895998196e-05
1112 1.68248905594304e-05
1113 1.59013952218956e-05
1114 1.5501089585257e-05
1115 1.5797272576723e-05
1116 1.65364681299351e-05
1117 1.71757937779127e-05
1118 1.70976648736776e-05
1119 1.62706827031567e-05
1120 1.54656855813062e-05
1121 1.51269339303539e-05
1122 1.50743812545873e-05
1123 1.50744475764242e-05
1124 1.50381912091824e-05
1125 1.49801696700536e-05
1126 1.49317163571361e-05
1127 1.48997189302058e-05
1128 1.48882491148861e-05
1129 1.4885814878518e-05
1130 1.48813639140322e-05
1131 1.48691127455081e-05
1132 1.4858778212723e-05
1133 1.48378726754217e-05
1134 1.48217183975063e-05
1135 1.47998090693768e-05
1136 1.47851937755128e-05
1137 1.47701072021889e-05
1138 1.47548036464329e-05
1139 1.47440027560286e-05
1140 1.47311734606248e-05
1141 1.47192524360129e-05
1142 1.47090571154251e-05
1143 1.46938963920817e-05
1144 1.46825168165776e-05
1145 1.46687498165932e-05
1146 1.46558742497582e-05
1147 1.46384835848501e-05
1148 1.46285910105348e-05
1149 1.46140954830898e-05
1150 1.45991513638144e-05
1151 1.45876001389489e-05
1152 1.45787811058895e-05
1153 1.45724425628657e-05
1154 1.45744872623865e-05
1155 1.45884413234043e-05
1156 1.46358681738512e-05
1157 1.47315906082746e-05
1158 1.49226322725404e-05
1159 1.52772596697659e-05
1160 1.58642635619266e-05
1161 1.66135002861445e-05
1162 1.70632443254881e-05
1163 1.66107270185289e-05
1164 1.55364191072316e-05
1165 1.49121814918551e-05
1166 1.52164561623147e-05
1167 1.60074989788983e-05
1168 1.64741712578298e-05
1169 1.60949404104382e-05
1170 1.52580320320617e-05
1171 1.46406218255457e-05
1172 1.43937780494596e-05
1173 1.4334384268011e-05
1174 1.43098902825045e-05
1175 1.42881205151824e-05
1176 1.42627369927872e-05
1177 1.42520046133576e-05
1178 1.42507723523977e-05
1179 1.4255844621669e-05
1180 1.42564554503899e-05
1181 1.42548909480844e-05
1182 1.42438985184512e-05
1183 1.42255594868246e-05
1184 1.42074997406816e-05
1185 1.41875846328965e-05
1186 1.41737919534535e-05
1187 1.41506850566131e-05
1188 1.41414278620466e-05
1189 1.41291994948389e-05
1190 1.41139783145228e-05
1191 1.41055970579451e-05
1192 1.40898731495476e-05
1193 1.40859955735095e-05
1194 1.4070177814407e-05
1195 1.40646080462048e-05
1196 1.4052760413108e-05
1197 1.40432505872923e-05
1198 1.40367721854773e-05
1199 1.40239834118816e-05
1200 1.40194179238939e-05
1201 1.40130692556362e-05
1202 1.4006895269425e-05
1203 1.40060241853313e-05
1204 1.40074734913398e-05
1205 1.40166657645047e-05
1206 1.40359366151976e-05
1207 1.407276507015e-05
1208 1.41353892033624e-05
1209 1.42396239455334e-05
1210 1.44075697863233e-05
1211 1.46432614434033e-05
1212 1.49283179473603e-05
1213 1.51766628830874e-05
1214 1.52387547445176e-05
1215 1.49994397928843e-05
1216 1.45576933183733e-05
1217 1.41838745264522e-05
1218 1.41355461478199e-05
1219 1.45580710353421e-05
1220 1.54890769610283e-05
1221 1.66203988354319e-05
1222 1.6979856596766e-05
1223 1.59569102635704e-05
1224 1.45874210395469e-05
1225 1.40139656765736e-05
1226 1.40174165925711e-05
1227 1.40575029605738e-05
1228 1.39755524484197e-05
1229 1.38384340750974e-05
1230 1.37356771547914e-05
1231 1.36954882781204e-05
1232 1.36860239963177e-05
1233 1.36953626697078e-05
1234 1.3690797026733e-05
1235 1.36801791588148e-05
1236 1.36638481631479e-05
1237 1.36451917156499e-05
1238 1.36316192289598e-05
1239 1.36164410817763e-05
1240 1.36072074914395e-05
1241 1.35960186042894e-05
1242 1.35885287755322e-05
1243 1.35816734117e-05
1244 1.35686052571415e-05
1245 1.35622399066726e-05
1246 1.35539270580409e-05
1247 1.3540902762621e-05
1248 1.35363319211379e-05
1249 1.35267291958563e-05
1250 1.35189050163298e-05
1251 1.35176997493325e-05
1252 1.35104277183551e-05
1253 1.35050345169141e-05
1254 1.35130485512391e-05
1255 1.35140057548888e-05
1256 1.3521241166714e-05
1257 1.3537282028242e-05
1258 1.35545238588719e-05
1259 1.35913344374572e-05
1260 1.36367090139977e-05
1261 1.36990745387422e-05
1262 1.37912693012332e-05
1263 1.39055752277351e-05
1264 1.40278825422335e-05
1265 1.41409209684085e-05
1266 1.41922429812702e-05
1267 1.41370747841485e-05
1268 1.39561700778223e-05
1269 1.37216989362088e-05
1270 1.35620805785663e-05
1271 1.36190425428229e-05
1272 1.40357672078206e-05
1273 1.48603882283993e-05
1274 1.58002397274704e-05
1275 1.60774395685515e-05
1276 1.52416654253251e-05
1277 1.41068109522635e-05
1278 1.35531034253233e-05
1279 1.35191754031538e-05
1280 1.36406786417353e-05
1281 1.36709051115069e-05
1282 1.35727646086714e-05
1283 1.34105265727769e-05
1284 1.32875564351487e-05
1285 1.32386368103532e-05
1286 1.3248697714463e-05
1287 1.3276469529977e-05
1288 1.32919551538713e-05
1289 1.32880312793127e-05
1290 1.32618323767097e-05
1291 1.32274420829859e-05
1292 1.31960182891966e-05
1293 1.31704701380464e-05
1294 1.31507345426662e-05
1295 1.31440527741944e-05
1296 1.31431529304393e-05
1297 1.3143781330438e-05
1298 1.31426095926201e-05
1299 1.31379192080949e-05
1300 1.31345767764302e-05
1301 1.31263218716704e-05
1302 1.31105083154281e-05
1303 1.31028487229123e-05
1304 1.30893806420396e-05
1305 1.30737044692353e-05
1306 1.30726745961596e-05
1307 1.3073389054874e-05
1308 1.30860173848113e-05
1309 1.31280546553647e-05
1310 1.31941829906168e-05
1311 1.33075043824427e-05
1312 1.34886751390617e-05
1313 1.37353430096354e-05
1314 1.4003384191108e-05
1315 1.4192926349299e-05
1316 1.41701525282478e-05
1317 1.38891353511639e-05
1318 1.34960525750571e-05
1319 1.32164547315128e-05
1320 1.32159678249977e-05
1321 1.35795629883084e-05
1322 1.42934304925291e-05
1323 1.51012247080473e-05
1324 1.54344326364075e-05
1325 1.48167238345209e-05
1326 1.37702859854816e-05
1327 1.31748905229934e-05
1328 1.31277253734297e-05
1329 1.32579520319975e-05
1330 1.32813585782543e-05
1331 1.31660272337797e-05
1332 1.30063059349261e-05
1333 1.28951420701462e-05
1334 1.2848134074428e-05
1335 1.28510331407483e-05
1336 1.286014713342e-05
1337 1.28649023523053e-05
1338 1.28573084694494e-05
1339 1.28351883352806e-05
1340 1.28200827611558e-05
1341 1.28015828884109e-05
1342 1.27925660962092e-05
1343 1.27846269042386e-05
1344 1.27812919470394e-05
1345 1.27811882997264e-05
1346 1.27782814032251e-05
1347 1.27711590804935e-05
1348 1.2763146353123e-05
1349 1.27501972162491e-05
1350 1.27409709360649e-05
1351 1.27318425122702e-05
1352 1.27182202565024e-05
1353 1.27146353237073e-05
1354 1.27141013619436e-05
1355 1.2726914919714e-05
1356 1.27472002970919e-05
1357 1.27868753261318e-05
1358 1.28546508726401e-05
1359 1.29561879407269e-05
1360 1.31116176511625e-05
1361 1.33197338669788e-05
1362 1.35703509211993e-05
1363 1.37884491686613e-05
1364 1.38674574134967e-05
1365 1.3677702426218e-05
1366 1.32968988637572e-05
1367 1.29502579455654e-05
1368 1.28873259803708e-05
1369 1.32164948571933e-05
1370 1.38514174257143e-05
1371 1.44513731519247e-05
1372 1.45202845311942e-05
1373 1.39210877483364e-05
1374 1.31778966365736e-05
1375 1.2754768907719e-05
1376 1.26670030473619e-05
1377 1.27232906894648e-05
1378 1.27789876955831e-05
1379 1.27632978279557e-05
1380 1.26838938139251e-05
1381 1.25935720012293e-05
1382 1.25307188172386e-05
1383 1.25126725247782e-05
1384 1.25232519678775e-05
1385 1.25461966153928e-05
1386 1.25652795004427e-05
1387 1.25680547289342e-05
1388 1.25540771440669e-05
1389 1.2526118827072e-05
1390 1.24957164158168e-05
1391 1.24684939342146e-05
1392 1.24452463263758e-05
1393 1.24355415431765e-05
1394 1.2436607376376e-05
1395 1.24543934072463e-05
1396 1.24747764798094e-05
1397 1.2508751534579e-05
1398 1.25512394824057e-05
1399 1.25915700048829e-05
1400 1.26347710356356e-05
1401 1.26697585032254e-05
1402 1.26819167278747e-05
1403 1.26731608687258e-05
1404 1.26385411731711e-05
1405 1.25692600279059e-05
1406 1.24946985597951e-05
1407 1.24252172528649e-05
1408 1.24090614126438e-05
1409 1.24789581730234e-05
1410 1.27014078423748e-05
1411 1.31297176404477e-05
1412 1.37482875690598e-05
1413 1.43383843604905e-05
1414 1.44184212715892e-05
1415 1.38189412046685e-05
1416 1.2995834326901e-05
1417 1.25509411939007e-05
1418 1.25980399880987e-05
1419 1.29101798269637e-05
1420 1.32082481858475e-05
1421 1.32793311160562e-05
1422 1.30591559450721e-05
1423 1.269001538029e-05
1424 1.2396945882287e-05
1425 1.22697833513374e-05
1426 1.22648278950699e-05
1427 1.22995991322927e-05
1428 1.23166213756853e-05
1429 1.2301391856262e-05
1430 1.2257874106858e-05
1431 1.22109782796365e-05
1432 1.21714421590369e-05
1433 1.21551048950508e-05
1434 1.21524054184086e-05
1435 1.2162429279261e-05
1436 1.21766418654801e-05
1437 1.21935904098258e-05
1438 1.22006101004324e-05
1439 1.22049892077314e-05
1440 1.21981043457176e-05
1441 1.21845960179279e-05
1442 1.21617107549099e-05
1443 1.21390754430628e-05
1444 1.21136027568802e-05
1445 1.20959377230001e-05
1446 1.2088399603094e-05
1447 1.20995492629561e-05
1448 1.21320214185872e-05
1449 1.21961736478937e-05
1450 1.23016661870423e-05
1451 1.2451759625387e-05
1452 1.26427558102549e-05
1453 1.28369868059863e-05
1454 1.29700129853916e-05
1455 1.29598214582138e-05
1456 1.2773614614936e-05
1457 1.24880267233785e-05
1458 1.22394715891261e-05
1459 1.21576752505348e-05
1460 1.23251997163898e-05
1461 1.27786360213378e-05
1462 1.35035608965772e-05
1463 1.42180679740811e-05
1464 1.43405726364954e-05
1465 1.36097770966348e-05
1466 1.26558178186187e-05
1467 1.21967452386684e-05
1468 1.2225211950101e-05
1469 1.23677197161332e-05
1470 1.23871362733841e-05
1471 1.22584037032247e-05
1472 1.20834731491826e-05
1473 1.19619422775941e-05
1474 1.19074240676476e-05
1475 1.19005991465748e-05
1476 1.19189167859801e-05
1477 1.19248721075493e-05
1478 1.19194386916099e-05
1479 1.19084714187423e-05
1480 1.18878498249764e-05
1481 1.18679089382567e-05
1482 1.18580149255365e-05
1483 1.184840765589e-05
1484 1.18450117230218e-05
1485 1.18429579227808e-05
1486 1.18450302277751e-05
1487 1.18445856056582e-05
1488 1.18418780363783e-05
1489 1.18359879848207e-05
1490 1.18298091003055e-05
1491 1.18213227899844e-05
1492 1.18111820133215e-05
1493 1.18011712872956e-05
1494 1.17904004488523e-05
1495 1.17896472890955e-05
1496 1.17888603909932e-05
1497 1.1799045959382e-05
1498 1.18324221105581e-05
1499 1.18939830260345e-05
1500 1.20068032645193e-05
1501 1.22152997614755e-05
1502 1.25539752340309e-05
1503 1.3039091924405e-05
1504 1.35874502551481e-05
1505 1.38515396725936e-05
1506 1.3477716577448e-05
1507 1.26692173916254e-05
1508 1.21351183863805e-05
1509 1.22721803490489e-05
1510 1.28593062773685e-05
1511 1.33837572966478e-05
1512 1.33390042269532e-05
1513 1.27614398406362e-05
1514 1.21523141145552e-05
1515 1.18178361576682e-05
1516 1.17244164519192e-05
1517 1.17178654317485e-05
1518 1.17175948477488e-05
1519 1.17004230575368e-05
1520 1.16722587395479e-05
1521 1.1652486452185e-05
1522 1.1645089962542e-05
1523 1.16507639265073e-05
1524 1.16583394329517e-05
1525 1.16691552323189e-05
1526 1.16700936336755e-05
1527 1.16662425408975e-05
1528 1.1656133799498e-05
1529 1.16418136748564e-05
1530 1.16267104597334e-05
1531 1.16159180572062e-05
1532 1.16054067769156e-05
1533 1.15950058772185e-05
1534 1.15940361338218e-05
1535 1.15950354055983e-05
1536 1.15945471197421e-05
1537 1.15971728522624e-05
1538 1.16072382247978e-05
1539 1.16136310746029e-05
1540 1.1621777793458e-05
1541 1.16333303838978e-05
1542 1.16452326266447e-05
1543 1.16602703923441e-05
1544 1.16741399316389e-05
1545 1.16916455166205e-05
1546 1.17112639959061e-05
1547 1.1734950333242e-05
1548 1.17577365301713e-05
1549 1.17808462269942e-05
1550 1.1798604851343e-05
1551 1.18080783746422e-05
1552 1.18089742122685e-05
1553 1.17879104388763e-05
1554 1.1748621011165e-05
1555 1.16895315522836e-05
1556 1.16221995514199e-05
1557 1.15778053932925e-05
1558 1.16013674467119e-05
1559 1.18071037522594e-05
1560 1.23734576162882e-05
1561 1.35266442868875e-05
1562 1.50594100398749e-05
1563 1.561836901387e-05
1564 1.41578517880436e-05
1565 1.24024313183924e-05
1566 1.18938869522189e-05
1567 1.21419183169991e-05
1568 1.23236275486605e-05
1569 1.21641818799922e-05
1570 1.18345762785133e-05
1571 1.15681967169934e-05
1572 1.14526798269399e-05
1573 1.14322698536728e-05
1574 1.14384943779022e-05
1575 1.14375565547498e-05
1576 1.14233201087277e-05
1577 1.14040390517545e-05
1578 1.13948388285046e-05
1579 1.13857497847647e-05
1580 1.13821219960286e-05
1581 1.13801895058607e-05
1582 1.13780009995956e-05
1583 1.13733220290513e-05
1584 1.13720288283936e-05
1585 1.13643395236451e-05
1586 1.13602622913156e-05
1587 1.13564072803385e-05
1588 1.13530089156377e-05
1589 1.13483814285775e-05
1590 1.13490315996057e-05
1591 1.13472890501676e-05
1592 1.13447847178705e-05
1593 1.13449238019481e-05
1594 1.13430870618547e-05
1595 1.13390914360245e-05
1596 1.13417078051459e-05
1597 1.13363480322981e-05
1598 1.1331670272563e-05
1599 1.13316645653505e-05
1600 1.13240292598515e-05
1601 1.13218126418513e-05
1602 1.13149563110149e-05
1603 1.13100130114674e-05
1604 1.13012312255023e-05
1605 1.12973595960408e-05
1606 1.12923946609111e-05
1607 1.1288475716853e-05
1608 1.12849451823216e-05
1609 1.12952976323299e-05
1610 1.1321842677825e-05
1611 1.14034597982204e-05
1612 1.15920430396876e-05
1613 1.20189998027431e-05
1614 1.29536947366393e-05
1615 1.45915078255676e-05
1616 1.60843652299292e-05
1617 1.51345043888362e-05
1618 1.27398334268669e-05
1619 1.21178884641138e-05
1620 1.28914619529352e-05
1621 1.3149973445703e-05
1622 1.24696812375813e-05
1623 1.16974132211567e-05
1624 1.13290193561877e-05
1625 1.12243026211267e-05
1626 1.11985135573534e-05
1627 1.11912156950389e-05
1628 1.11911460141112e-05
1629 1.11966801756846e-05
1630 1.12016374682611e-05
1631 1.12027737662057e-05
1632 1.11957811970154e-05
1633 1.11902841479772e-05
1634 1.11834664420218e-05
1635 1.11740354125089e-05
1636 1.11683563299714e-05
1637 1.11614216420364e-05
1638 1.11590643481208e-05
1639 1.11505232287534e-05
1640 1.11503245903144e-05
1641 1.11479504132372e-05
1642 1.11425607969284e-05
1643 1.11406821927673e-05
1644 1.11382023624262e-05
1645 1.11334815959818e-05
1646 1.11312959452103e-05
1647 1.11260986694894e-05
1648 1.11267961624417e-05
1649 1.11218780336486e-05
1650 1.11180250439435e-05
1651 1.1115558283592e-05
1652 1.11132671047898e-05
1653 1.11086690215423e-05
1654 1.11061713097271e-05
1655 1.1101306032435e-05
1656 1.10990275277167e-05
1657 1.1097667765636e-05
1658 1.10929420187311e-05
1659 1.10917181501691e-05
1660 1.10838525251289e-05
1661 1.10840547848934e-05
1662 1.10818751815067e-05
1663 1.10773465911151e-05
1664 1.10733878073699e-05
1665 1.1072614157781e-05
1666 1.10663981272019e-05
1667 1.10634879835203e-05
1668 1.10643851287673e-05
1669 1.10598301199527e-05
1670 1.10574460543411e-05
1671 1.10541111359996e-05
1672 1.10516997768872e-05
1673 1.10516854916476e-05
1674 1.10546086466989e-05
1675 1.10651635183423e-05
1676 1.10852312633636e-05
1677 1.11484666245332e-05
1678 1.1287557534212e-05
1679 1.16146716164778e-05
1680 1.23307561603614e-05
1681 1.36431930253256e-05
1682 1.51570653426081e-05
1683 1.52112614191235e-05
1684 1.32626370645816e-05
1685 1.18355028271289e-05
1686 1.2134752324533e-05
1687 1.31013212625319e-05
1688 1.33721924093777e-05
1689 1.25588552868372e-05
1690 1.15728388083713e-05
1691 1.11127528850119e-05
1692 1.10047336545627e-05
1693 1.09800170724483e-05
1694 1.095614377844e-05
1695 1.09503758494167e-05
1696 1.09549763587236e-05
1697 1.09652415032357e-05
1698 1.09696021521888e-05
1699 1.09674717603259e-05
1700 1.09591473560489e-05
1701 1.09517639304002e-05
1702 1.09417525770983e-05
1703 1.09328285784471e-05
1704 1.09261089729085e-05
1705 1.09218692541102e-05
1706 1.09166377755265e-05
1707 1.09132778012277e-05
1708 1.09115669018056e-05
1709 1.09063406668053e-05
1710 1.09054065171588e-05
1711 1.09017271872069e-05
1712 1.0896405425509e-05
1713 1.08991381564394e-05
1714 1.08936568723461e-05
1715 1.0891522935097e-05
1716 1.08877818048025e-05
1717 1.08843843895645e-05
1718 1.08825480276131e-05
1719 1.08800932374642e-05
1720 1.08760108425976e-05
1721 1.08726534593373e-05
1722 1.08682021027207e-05
1723 1.08673735368381e-05
1724 1.08591770511257e-05
1725 1.08597194032889e-05
1726 1.08584487774621e-05
1727 1.08513520384879e-05
1728 1.08496795183655e-05
1729 1.08456887786268e-05
1730 1.08438621675422e-05
1731 1.0840593806849e-05
1732 1.08377972973805e-05
1733 1.08352083558749e-05
1734 1.0839041040267e-05
1735 1.08428889733503e-05
1736 1.08566523935316e-05
1737 1.08857879626001e-05
1738 1.09582441578659e-05
1739 1.11027719509327e-05
1740 1.14026264252942e-05
1741 1.19887996712364e-05
1742 1.29166936724268e-05
1743 1.3884069042458e-05
1744 1.39380080657059e-05
1745 1.27293842020482e-05
1746 1.15652099867969e-05
1747 1.14703369678004e-05
1748 1.2229292629673e-05
1749 1.30133761451212e-05
1750 1.29546803171454e-05
1751 1.20525406239747e-05
1752 1.1214929058756e-05
1753 1.0881173946764e-05
1754 1.08219646726138e-05
1755 1.08046817364205e-05
1756 1.07769306481043e-05
1757 1.07464818837055e-05
1758 1.07438742029675e-05
1759 1.07532497084506e-05
1760 1.07636015900248e-05
1761 1.07678027903013e-05
1762 1.07656383692145e-05
1763 1.07568609899467e-05
1764 1.07470231125451e-05
1765 1.07343504367474e-05
1766 1.07275151712827e-05
1767 1.07205119517229e-05
1768 1.07129624817848e-05
1769 1.07116067149526e-05
1770 1.07093021064486e-05
1771 1.07039644789975e-05
1772 1.07022496509401e-05
1773 1.0700285544285e-05
1774 1.06958125778434e-05
1775 1.06966532589237e-05
1776 1.06935426402721e-05
1777 1.06897968883413e-05
1778 1.06920093456697e-05
1779 1.06902954410959e-05
1780 1.06889234259278e-05
1781 1.0686860377751e-05
1782 1.06876426246849e-05
1783 1.06893354283599e-05
1784 1.06870663301173e-05
1785 1.06877225594104e-05
1786 1.06886639150705e-05
1787 1.06874704481896e-05
1788 1.06949778762022e-05
1789 1.0696410940314e-05
1790 1.07051159674576e-05
1791 1.07157458906215e-05
1792 1.07372457185395e-05
1793 1.07702506164653e-05
1794 1.08308022139791e-05
1795 1.09293332715144e-05
1796 1.11121240486955e-05
1797 1.14200118392915e-05
1798 1.19090873180738e-05
1799 1.25338657248619e-05
1800 1.29744558212685e-05
1801 1.27076460068043e-05
1802 1.1799650008526e-05
1803 1.11558063158324e-05
1804 1.14026410051871e-05
1805 1.23428646309209e-05
1806 1.30941147959973e-05
1807 1.27937744289763e-05
1808 1.17672180075523e-05
1809 1.09780640746848e-05
1810 1.06822771568371e-05
1811 1.06196757663213e-05
1812 1.06023909918207e-05
1813 1.05799197185341e-05
1814 1.05660076428205e-05
1815 1.05669837515698e-05
1816 1.0585286919218e-05
1817 1.06006310844897e-05
1818 1.0612276791333e-05
1819 1.06132449462226e-05
1820 1.0603626878547e-05
1821 1.05871641327138e-05
1822 1.05776017829129e-05
1823 1.05628728905849e-05
1824 1.05543981385736e-05
1825 1.054595677763e-05
1826 1.05386890902892e-05
1827 1.05355054749179e-05
1828 1.05318601257842e-05
1829 1.05289181711044e-05
1830 1.05278596178593e-05
1831 1.05242934078653e-05
1832 1.05217902863775e-05
1833 1.05242596983857e-05
1834 1.05246761383793e-05
1835 1.05251323425648e-05
1836 1.05266957808325e-05
1837 1.05289742760029e-05
1838 1.05345580587723e-05
1839 1.05385073834174e-05
1840 1.05487974741969e-05
1841 1.05574920781226e-05
1842 1.05699454329145e-05
1843 1.0591662967796e-05
1844 1.06190762987346e-05
1845 1.0652329518468e-05
1846 1.07040931776048e-05
1847 1.07616090367912e-05
1848 1.08432606134024e-05
1849 1.09329109907463e-05
1850 1.10243011119948e-05
1851 1.10906355448304e-05
1852 1.11052403739542e-05
1853 1.1044280635808e-05
1854 1.09183641447963e-05
1855 1.07512699860646e-05
1856 1.06230309224742e-05
1857 1.06179504046899e-05
1858 1.08709277897301e-05
1859 1.15709144778986e-05
1860 1.28576257845125e-05
1861 1.4337021657429e-05
1862 1.43554533096335e-05
1863 1.24559221963416e-05
1864 1.10166091028541e-05
1865 1.0922990051343e-05
1866 1.11286442873171e-05
1867 1.10263711671221e-05
1868 1.07291925273145e-05
1869 1.05114951884921e-05
1870 1.04199481589795e-05
1871 1.0406664983309e-05
1872 1.04065196444569e-05
1873 1.04065287303001e-05
1874 1.03991267055292e-05
1875 1.03898997865226e-05
1876 1.03886326341396e-05
1877 1.03829391544519e-05
1878 1.03842853647063e-05
1879 1.03821529773285e-05
1880 1.03804816922182e-05
1881 1.03764848962129e-05
1882 1.03717312185392e-05
1883 1.03702099782943e-05
1884 1.03681016339063e-05
1885 1.03638931192584e-05
1886 1.03620939557736e-05
1887 1.03588590336656e-05
1888 1.03578595977982e-05
1889 1.03532933584116e-05
1890 1.03543908858139e-05
1891 1.03506093775962e-05
1892 1.03486728839641e-05
1893 1.03462020408784e-05
1894 1.03433163605171e-05
1895 1.03422465040914e-05
1896 1.03375459679089e-05
1897 1.03407918514709e-05
1898 1.03371823574339e-05
1899 1.033997062172e-05
1900 1.03407983127468e-05
1901 1.03467244234601e-05
1902 1.03558018011984e-05
1903 1.03771772299588e-05
1904 1.04064779200552e-05
1905 1.04606979887123e-05
1906 1.05540997694664e-05
1907 1.07085127807327e-05
1908 1.09528608153164e-05
1909 1.12930429277958e-05
1910 1.16809439656418e-05
1911 1.19390585308077e-05
1912 1.18230955059584e-05
1913 1.13336132625541e-05
1914 1.08067173083537e-05
1915 1.06003912185848e-05
1916 1.09017293472569e-05
1917 1.16772568210521e-05
1918 1.26180577808555e-05
1919 1.29923517704356e-05
1920 1.223428940067e-05
1921 1.10875691552437e-05
1922 1.05090335529301e-05
1923 1.04555463840139e-05
1924 1.05111127826074e-05
1925 1.04767850179766e-05
1926 1.03772924089363e-05
1927 1.02933320289544e-05
1928 1.02539329382978e-05
1929 1.02550044098759e-05
1930 1.02684501641637e-05
1931 1.0276206885651e-05
1932 1.02790484381998e-05
1933 1.0267705482514e-05
1934 1.02564470840782e-05
1935 1.02417884946249e-05
1936 1.0231977686681e-05
1937 1.02295920250128e-05
1938 1.02250909761459e-05
1939 1.02248781317371e-05
1940 1.02237402594962e-05
1941 1.02213357366931e-05
1942 1.02201776637223e-05
1943 1.02151607057621e-05
1944 1.02141986171311e-05
1945 1.02113884363764e-05
1946 1.02071410086335e-05
1947 1.02043499077276e-05
1948 1.02045968890518e-05
1949 1.02042591325624e-05
1950 1.02065396869744e-05
1951 1.02116065603397e-05
1952 1.02214385953037e-05
1953 1.02336062404174e-05
1954 1.02547874121228e-05
1955 1.02941242356991e-05
1956 1.0344931083095e-05
1957 1.04276456356356e-05
1958 1.05552737768022e-05
1959 1.0746572411735e-05
1960 1.09991301959766e-05
1961 1.12939729131067e-05
1962 1.15197965635527e-05
1963 1.15101211985991e-05
1964 1.11779899184539e-05
1965 1.0709795684738e-05
1966 1.0462185611182e-05
1967 1.06841827354209e-05
1968 1.14012353159554e-05
1969 1.22824396735499e-05
1970 1.25983316304756e-05
1971 1.19608386364156e-05
1972 1.09778746781863e-05
1973 1.04023144018583e-05
1974 1.02465781324224e-05
1975 1.02566187598629e-05
1976 1.02653051310408e-05
1977 1.02293305075385e-05
1978 1.01707595190348e-05
1979 1.01313097247147e-05
1980 1.01218467740694e-05
1981 1.01385910040719e-05
1982 1.01569407671143e-05
1983 1.01785900261664e-05
1984 1.01828576453133e-05
1985 1.01774816054778e-05
1986 1.01624758077623e-05
1987 1.01436216703998e-05
1988 1.01260522615831e-05
1989 1.01125983740236e-05
1990 1.01004683379369e-05
1991 1.00961862217197e-05
1992 1.00935759308474e-05
1993 1.00915551037861e-05
1994 1.0091235628229e-05
1995 1.00910611238181e-05
1996 1.00903905286831e-05
1997 1.00902543114234e-05
1998 1.00850005146924e-05
1999 1.00841952672681e-05
};
\addlegendentry{Train}
\addplot [semithick, black]
table {%
0 0.0071602719835937
1 0.00713793467730284
2 0.0071184323169291
3 0.00710158888250589
4 0.00708708772435784
5 0.00707451906055212
6 0.00706360302865505
7 0.00705407001078129
8 0.00704572116956115
9 0.00703843450173736
10 0.00703203584998846
11 0.00702636642381549
12 0.00702136801555753
13 0.00701693957671523
14 0.00701304292306304
15 0.00700959004461765
16 0.00700649386271834
17 0.00700370222330093
18 0.0070011718198657
19 0.00699886400252581
20 0.00699670938774943
21 0.00699469121173024
22 0.00699282344430685
23 0.00699103064835072
24 0.00698930630460382
25 0.00698753260076046
26 0.00698565132915974
27 0.0069836089387536
28 0.0069814994931221
29 0.00697922520339489
30 0.00697667058557272
31 0.00697351526468992
32 0.00696962559595704
33 0.00696484139189124
34 0.00695886975154281
35 0.00695166131481528
36 0.00694270152598619
37 0.0069316765293479
38 0.00691814720630646
39 0.00690151937305927
40 0.00688100885599852
41 0.00685605825856328
42 0.00682561472058296
43 0.00678862584754825
44 0.00674484064802527
45 0.00669304700568318
46 0.00663235317915678
47 0.00656169559806585
48 0.00647994922474027
49 0.00638527097180486
50 0.00627632113173604
51 0.00615331809967756
52 0.00601628609001637
53 0.00586668541654944
54 0.00571004766970873
55 0.00555146532133222
56 0.00539328390732408
57 0.00523575348779559
58 0.00508482288569212
59 0.00493872817605734
60 0.00480173714458942
61 0.00467647658661008
62 0.00455890921875834
63 0.00444749835878611
64 0.00434025470167398
65 0.00423781434074044
66 0.00413621217012405
67 0.00403860025107861
68 0.00394434109330177
69 0.00385013571940362
70 0.00375850521959364
71 0.0036707182880491
72 0.0035882608499378
73 0.00350916618481278
74 0.00343200284987688
75 0.00335682439617813
76 0.00328130717389286
77 0.00320394616574049
78 0.00312709296122193
79 0.00305238366127014
80 0.00297940662130713
81 0.00290843588300049
82 0.00283930683508515
83 0.00277035869657993
84 0.00270117982290685
85 0.00263184704817832
86 0.0025623042602092
87 0.00248970836400986
88 0.00241640955209732
89 0.00234205555170774
90 0.00226942519657314
91 0.00219710264354944
92 0.00212532049044967
93 0.00205302028916776
94 0.00198094733059406
95 0.00191026472020894
96 0.00184140505734831
97 0.00177316810004413
98 0.00170504220295697
99 0.00163934857118875
100 0.00157563295215368
101 0.00151295564137399
102 0.00145171443000436
103 0.00139082432724535
104 0.00133089686278254
105 0.00127431994769722
106 0.00122136925347149
107 0.00117174501065165
108 0.00112504244316369
109 0.00108153268229216
110 0.00104024703614414
111 0.00100104627199471
112 0.000963149592280388
113 0.000926605949643999
114 0.000891587405931205
115 0.000859075225889683
116 0.000828771269880235
117 0.00080011214595288
118 0.000772125844378024
119 0.000742997042834759
120 0.000713907415047288
121 0.00068717182148248
122 0.000663019251078367
123 0.00064080412266776
124 0.000620657461695373
125 0.000601980136707425
126 0.000584316032472998
127 0.00056808500085026
128 0.000553102290723473
129 0.000538529304321855
130 0.000523560564033687
131 0.00050945149268955
132 0.0004961674567312
133 0.000483674695715308
134 0.000471821636892855
135 0.000459940230939537
136 0.0004485800745897
137 0.000437735783634707
138 0.0004276781110093
139 0.000418395851738751
140 0.000409760657930747
141 0.000401749362936243
142 0.00039424654096365
143 0.000387259817216545
144 0.000380715588107705
145 0.00037415893166326
146 0.00036806522984989
147 0.000362360500730574
148 0.000357024138793349
149 0.000351993745425716
150 0.000347240158589557
151 0.0003427205956541
152 0.000338383135385811
153 0.000334252545144409
154 0.000330304173985496
155 0.000326501991366968
156 0.000322867359500378
157 0.000319361832225695
158 0.000315809331368655
159 0.00031235913047567
160 0.000309112889226526
161 0.000305953959468752
162 0.000302969972835854
163 0.000300074461847544
164 0.000297292717732489
165 0.000294586352538317
166 0.000291929201921448
167 0.000289335381239653
168 0.000286565045826137
169 0.000283840228803456
170 0.000281275890301913
171 0.000278788997093216
172 0.00027633918216452
173 0.000273923389613628
174 0.000271555793005973
175 0.000269242387730628
176 0.000266961200395599
177 0.000264722679276019
178 0.000262509944150224
179 0.000260159838944674
180 0.000257710606092587
181 0.00025530721177347
182 0.000252951984293759
183 0.00025065615773201
184 0.000248392345383763
185 0.000246150593739003
186 0.00024394407228101
187 0.000241760819335468
188 0.000239595887251198
189 0.000237468790146522
190 0.000235355662880465
191 0.000233266851864755
192 0.000231179685215466
193 0.000229112425586209
194 0.000227050375542603
195 0.000225008392590098
196 0.000222977294470184
197 0.000220963062020019
198 0.00021895079407841
199 0.000216950866160914
200 0.000214933752431534
201 0.000212943021324463
202 0.000210941303521395
203 0.00020895832858514
204 0.000206992233870551
205 0.000204999101697467
206 0.00020297993614804
207 0.000200966896954924
208 0.000198979949345812
209 0.000196998298633844
210 0.000195035550859757
211 0.00019309954950586
212 0.000191188111784868
213 0.000189297730685212
214 0.000187400510185398
215 0.000185521712410264
216 0.000183662239578553
217 0.000181831477675587
218 0.00017999489500653
219 0.000178177782800049
220 0.000176376997842453
221 0.000174584885826334
222 0.000172781728906557
223 0.000170990242622793
224 0.000169058912433684
225 0.000166911573614925
226 0.000164810233400203
227 0.000162815500516444
228 0.000160913114086725
229 0.000159105635248125
230 0.00015741448441986
231 0.000155831148731522
232 0.00015427173639182
233 0.000152738095493987
234 0.000151224608998746
235 0.000149723506183363
236 0.000148274615639821
237 0.000146842954563908
238 0.000145438505569473
239 0.000144046905916184
240 0.000142694043461233
241 0.000141348864417523
242 0.000140032119816169
243 0.000138724542921409
244 0.00013742885494139
245 0.000136175251100212
246 0.000134940710267983
247 0.000133710767840967
248 0.000132485132780857
249 0.000131281034555286
250 0.000130093263578601
251 0.000128922547446564
252 0.000127760315081105
253 0.000126615108456463
254 0.000125503749586642
255 0.000124391881399788
256 0.000123319987324066
257 0.000122260782518424
258 0.000121237586427014
259 0.000120171011076309
260 0.000119109819934238
261 0.00011806029942818
262 0.000117031893751118
263 0.000115992210339755
264 0.000114974434836768
265 0.000113967806100845
266 0.000112975496449508
267 0.000112003850517794
268 0.000111077162728179
269 0.000110139728349168
270 0.000109234948467929
271 0.000108340238512028
272 0.000107464271422941
273 0.000106601008155849
274 0.000105763145256788
275 0.000104936712887138
276 0.000104126833321061
277 0.000103302394563798
278 0.000102537036582362
279 0.000101717494544573
280 0.000100954064691905
281 0.000100164339528419
282 9.94267684291117e-05
283 9.8667609563563e-05
284 9.79556571110152e-05
285 9.72225025179796e-05
286 9.65160579653457e-05
287 9.58572563831694e-05
288 9.51783513301052e-05
289 9.45379069889896e-05
290 9.38818993745372e-05
291 9.32700495468453e-05
292 9.26519496715628e-05
293 9.2069101810921e-05
294 9.14832417038269e-05
295 9.08868169062771e-05
296 9.03158215805888e-05
297 8.9759094407782e-05
298 8.92026437213644e-05
299 8.86593115865253e-05
300 8.81379310158081e-05
301 8.76181729836389e-05
302 8.70866133482195e-05
303 8.65503025124781e-05
304 8.60012660268694e-05
305 8.54892423376441e-05
306 8.49648204166442e-05
307 8.44617607071996e-05
308 8.39546410134062e-05
309 8.34628663142212e-05
310 8.29636701382697e-05
311 8.24841699795797e-05
312 8.20038403617218e-05
313 8.15344101283699e-05
314 8.10671335784718e-05
315 8.0621728557162e-05
316 8.01666392362677e-05
317 7.97262036940083e-05
318 7.92891150922514e-05
319 7.88640827522613e-05
320 7.84428702900186e-05
321 7.8014396422077e-05
322 7.76054366724566e-05
323 7.72107232478447e-05
324 7.67967139836401e-05
325 7.64047435950488e-05
326 7.60043985792436e-05
327 7.56272856961004e-05
328 7.5242314778734e-05
329 7.48611782910302e-05
330 7.44799253880046e-05
331 7.41047842893749e-05
332 7.37353402655572e-05
333 7.33540873625316e-05
334 7.29921812308021e-05
335 7.26388825569302e-05
336 7.22853510524146e-05
337 7.1933536673896e-05
338 7.15824207873084e-05
339 7.12292385287583e-05
340 7.08810039213859e-05
341 7.05284546711482e-05
342 7.01882745488547e-05
343 6.98371368343942e-05
344 6.95114504196681e-05
345 6.9174129748717e-05
346 6.88374711899087e-05
347 6.8507666583173e-05
348 6.81747624184936e-05
349 6.78518408676609e-05
350 6.75123228575103e-05
351 6.71614543534815e-05
352 6.68559659970924e-05
353 6.65374536765739e-05
354 6.62110760458745e-05
355 6.58846111036837e-05
356 6.55740732327104e-05
357 6.52454473311082e-05
358 6.49359499220736e-05
359 6.46038679406047e-05
360 6.43043895252049e-05
361 6.39925565337762e-05
362 6.36791155557148e-05
363 6.33874151390046e-05
364 6.30825234111398e-05
365 6.27882182016037e-05
366 6.24822350800969e-05
367 6.21658764430322e-05
368 6.18707636022009e-05
369 6.15669559920207e-05
370 6.12626608926803e-05
371 6.09562739555258e-05
372 6.06628927926067e-05
373 6.03625630901661e-05
374 6.00622952333651e-05
375 5.97666476096492e-05
376 5.94697885389905e-05
377 5.91858624829911e-05
378 5.88913171668537e-05
379 5.86075402679853e-05
380 5.8309411542723e-05
381 5.80178602831438e-05
382 5.77377795707434e-05
383 5.74560253880918e-05
384 5.71605960431043e-05
385 5.68724899494555e-05
386 5.66036615055054e-05
387 5.63386602152605e-05
388 5.60715307074133e-05
389 5.5810778576415e-05
390 5.55389269720763e-05
391 5.52644669369329e-05
392 5.49637661606539e-05
393 5.4675623687217e-05
394 5.44086578884162e-05
395 5.4132251534611e-05
396 5.38721433258615e-05
397 5.36024454049766e-05
398 5.33554193680175e-05
399 5.3107909479877e-05
400 5.28633354406338e-05
401 5.26229378010612e-05
402 5.23885937582236e-05
403 5.21603578818031e-05
404 5.1932565838797e-05
405 5.16919353685807e-05
406 5.1466344302753e-05
407 5.12472361151595e-05
408 5.10088248120155e-05
409 5.07958284288179e-05
410 5.05786047142465e-05
411 5.03516057506204e-05
412 5.0134098273702e-05
413 4.9918166041607e-05
414 4.969931251253e-05
415 4.94625310238916e-05
416 4.92629551445134e-05
417 4.90611964778509e-05
418 4.88464684167411e-05
419 4.86520984850358e-05
420 4.84506017528474e-05
421 4.82665382151026e-05
422 4.80878297821619e-05
423 4.78775473311543e-05
424 4.76887325930875e-05
425 4.74971930088941e-05
426 4.73215732199606e-05
427 4.71312159788795e-05
428 4.69361839350313e-05
429 4.67679492430761e-05
430 4.65805132989772e-05
431 4.63937503809575e-05
432 4.62177813460585e-05
433 4.6033805119805e-05
434 4.58540489489678e-05
435 4.56782763649244e-05
436 4.55087247246411e-05
437 4.53210923296865e-05
438 4.51487358077429e-05
439 4.49731196567882e-05
440 4.47907805209979e-05
441 4.45963014499284e-05
442 4.44024226453621e-05
443 4.42275886598509e-05
444 4.40514195361175e-05
445 4.38715323980432e-05
446 4.36980881204363e-05
447 4.35465735790785e-05
448 4.33650347986259e-05
449 4.32053857366554e-05
450 4.30297877755947e-05
451 4.28840176027734e-05
452 4.27153063355945e-05
453 4.25500729761552e-05
454 4.23846322519239e-05
455 4.22316916228738e-05
456 4.20721517002676e-05
457 4.19392054027412e-05
458 4.17753690271638e-05
459 4.16400798712857e-05
460 4.14781898143701e-05
461 4.13425623264629e-05
462 4.12142435379792e-05
463 4.10822685807943e-05
464 4.09277890867088e-05
465 4.079500286025e-05
466 4.06460530939512e-05
467 4.05278333346359e-05
468 4.03752492275089e-05
469 4.02477526222356e-05
470 4.0105966036208e-05
471 3.99826785724144e-05
472 3.98231532017235e-05
473 3.96998912037816e-05
474 3.95694878534414e-05
475 3.94369999412447e-05
476 3.93086229451001e-05
477 3.92122965422459e-05
478 3.90581881219987e-05
479 3.8913196476642e-05
480 3.87935615435708e-05
481 3.86725077987649e-05
482 3.85600033041555e-05
483 3.843260492431e-05
484 3.83297265216243e-05
485 3.81977224606089e-05
486 3.80974634026643e-05
487 3.79492266802117e-05
488 3.78366530640051e-05
489 3.77248506993055e-05
490 3.76144744222984e-05
491 3.75043236999772e-05
492 3.73885450244416e-05
493 3.72742215404287e-05
494 3.71457426808774e-05
495 3.70196285075508e-05
496 3.69131048501004e-05
497 3.68018918379676e-05
498 3.66803324141074e-05
499 3.6572808312485e-05
500 3.6444966099225e-05
501 3.63292238034774e-05
502 3.62434839189518e-05
503 3.61250567948446e-05
504 3.60230005753692e-05
505 3.5946439311374e-05
506 3.58488359779585e-05
507 3.5793062124867e-05
508 3.57375138264615e-05
509 3.56965210812632e-05
510 3.5693865356734e-05
511 3.56877026206348e-05
512 3.5700843000086e-05
513 3.57209828507621e-05
514 3.57085846189875e-05
515 3.55401134584099e-05
516 3.53166724380571e-05
517 3.50171030731872e-05
518 3.4760290873237e-05
519 3.46147644449957e-05
520 3.45753032888751e-05
521 3.47070672432892e-05
522 3.50084555975627e-05
523 3.53731811628677e-05
524 3.55249649146572e-05
525 3.50589398294687e-05
526 3.41674676747061e-05
527 3.34894175466616e-05
528 3.33000243699644e-05
529 3.34215910697822e-05
530 3.36423800035845e-05
531 3.37323253916111e-05
532 3.36345365212765e-05
533 3.34319629473612e-05
534 3.3167569199577e-05
535 3.29847498505842e-05
536 3.28137939504813e-05
537 3.27395500789862e-05
538 3.26372683048248e-05
539 3.25431356031913e-05
540 3.24592874676455e-05
541 3.23360254697036e-05
542 3.2250136428047e-05
543 3.216900586267e-05
544 3.20966719300486e-05
545 3.20458275382407e-05
546 3.19894825224765e-05
547 3.19670907629188e-05
548 3.19361679430585e-05
549 3.19099781336263e-05
550 3.18817728839349e-05
551 3.18657839670777e-05
552 3.18214479193557e-05
553 3.17272460961249e-05
554 3.16420446324628e-05
555 3.15212055284064e-05
556 3.14228273055051e-05
557 3.13056625600439e-05
558 3.12138909066562e-05
559 3.11145959130954e-05
560 3.0998227884993e-05
561 3.093925624853e-05
562 3.08711059915368e-05
563 3.07820955640636e-05
564 3.07341251755133e-05
565 3.06791189359501e-05
566 3.06175679725129e-05
567 3.05734320136253e-05
568 3.05007270071656e-05
569 3.04314835375408e-05
570 3.03521082969382e-05
571 3.02378593914909e-05
572 3.00955434795469e-05
573 2.99488528980874e-05
574 2.98520972137339e-05
575 2.98116774501977e-05
576 2.99873081530677e-05
577 3.05350367852952e-05
578 3.1625444535166e-05
579 3.2995823858073e-05
580 3.33007374138106e-05
581 3.1789702916285e-05
582 3.02548342006048e-05
583 2.98315553663997e-05
584 2.99387538689189e-05
585 2.99763487419114e-05
586 2.97944588965038e-05
587 2.94693127216306e-05
588 2.91890792141203e-05
589 2.90046300506219e-05
590 2.88981227640761e-05
591 2.88561222987482e-05
592 2.8822134481743e-05
593 2.88026913040085e-05
594 2.87616476271069e-05
595 2.87065468000947e-05
596 2.86410213448107e-05
597 2.85784390143817e-05
598 2.84973284578882e-05
599 2.84291327261599e-05
600 2.83579738606932e-05
601 2.83031567960279e-05
602 2.82368346233852e-05
603 2.81907614407828e-05
604 2.81163083855063e-05
605 2.80735548585653e-05
606 2.80241129075876e-05
607 2.79795931419358e-05
608 2.79504856735002e-05
609 2.79274463537149e-05
610 2.78996340057347e-05
611 2.79099640465574e-05
612 2.78782845271053e-05
613 2.78664738289081e-05
614 2.78769784927135e-05
615 2.78381467069266e-05
616 2.78074057860067e-05
617 2.7763544494519e-05
618 2.77001490758266e-05
619 2.7613034035312e-05
620 2.75131515081739e-05
621 2.74351641564863e-05
622 2.73460209427867e-05
623 2.72654579021037e-05
624 2.71937533398159e-05
625 2.71819371846505e-05
626 2.716406743275e-05
627 2.7198550014873e-05
628 2.7308627977618e-05
629 2.75211768894223e-05
630 2.77608978649369e-05
631 2.80173971987097e-05
632 2.80018266494153e-05
633 2.75535603577737e-05
634 2.6902753234026e-05
635 2.64370009972481e-05
636 2.64685950241983e-05
637 2.70172404270852e-05
638 2.78916504612425e-05
639 2.84642010228708e-05
640 2.82572982541751e-05
641 2.74067715508863e-05
642 2.67425566562451e-05
643 2.6453899408807e-05
644 2.63681995420484e-05
645 2.63266192632727e-05
646 2.62151643255493e-05
647 2.61131226579892e-05
648 2.60070355579956e-05
649 2.59456628555199e-05
650 2.5890725737554e-05
651 2.58599757216871e-05
652 2.58865984505974e-05
653 2.58653089986183e-05
654 2.58994932664791e-05
655 2.58798645518254e-05
656 2.5847408323898e-05
657 2.58292584476294e-05
658 2.5776707843761e-05
659 2.57447481999407e-05
660 2.56974308285862e-05
661 2.56247367360629e-05
662 2.55875220318558e-05
663 2.55036102316808e-05
664 2.54880815191427e-05
665 2.54228289122693e-05
666 2.54007572948467e-05
667 2.53568578045815e-05
668 2.52892332355259e-05
669 2.52861336775823e-05
670 2.52421559707727e-05
671 2.52167337748688e-05
672 2.5182585886796e-05
673 2.51985238719499e-05
674 2.52043537329882e-05
675 2.52838362939656e-05
676 2.54283386311727e-05
677 2.56169405474793e-05
678 2.58899326581741e-05
679 2.61141267401399e-05
680 2.6249199436279e-05
681 2.61017885350157e-05
682 2.57791652984452e-05
683 2.54169517575065e-05
684 2.51704623224214e-05
685 2.51407982432283e-05
686 2.53765811066842e-05
687 2.57847605098505e-05
688 2.62546345766168e-05
689 2.64153259195155e-05
690 2.59449589066207e-05
691 2.50816519837826e-05
692 2.44778548221802e-05
693 2.4340757590835e-05
694 2.45271221501753e-05
695 2.47651751124067e-05
696 2.48992619162891e-05
697 2.48423784796614e-05
698 2.47253792622359e-05
699 2.45579667534912e-05
700 2.44551793002756e-05
701 2.43681843130616e-05
702 2.43559643422486e-05
703 2.43175381910987e-05
704 2.42859186982969e-05
705 2.42139358306304e-05
706 2.41881607507821e-05
707 2.41221077885712e-05
708 2.40605022554519e-05
709 2.40220069827046e-05
710 2.39851578953676e-05
711 2.40113968175137e-05
712 2.40315439441474e-05
713 2.40709450736176e-05
714 2.41292527789483e-05
715 2.42226960835978e-05
716 2.42916812567273e-05
717 2.43766990024596e-05
718 2.4404263967881e-05
719 2.43710455833934e-05
720 2.42945188801968e-05
721 2.41970574279549e-05
722 2.40620756812859e-05
723 2.39502387557877e-05
724 2.38844368141145e-05
725 2.38612065004418e-05
726 2.391458110651e-05
727 2.40541212406242e-05
728 2.42637670453405e-05
729 2.45928622462088e-05
730 2.48471988015808e-05
731 2.48228989221388e-05
732 2.4398954337812e-05
733 2.37348249356728e-05
734 2.33197406487307e-05
735 2.3321770640905e-05
736 2.37278509302996e-05
737 2.42993774008937e-05
738 2.47446914727334e-05
739 2.46978197537828e-05
740 2.42779005930061e-05
741 2.38286957028322e-05
742 2.35209063248476e-05
743 2.34477665799204e-05
744 2.34028411796317e-05
745 2.33845912589459e-05
746 2.33454156841617e-05
747 2.32568854698911e-05
748 2.31628619076218e-05
749 2.30914283747552e-05
750 2.30095192819135e-05
751 2.30052264669212e-05
752 2.30132845899789e-05
753 2.30635287152836e-05
754 2.31037338380702e-05
755 2.31743670155993e-05
756 2.3207856429508e-05
757 2.3218011847348e-05
758 2.32271031563869e-05
759 2.31674002861837e-05
760 2.31063386308961e-05
761 2.30543719226262e-05
762 2.29921552090673e-05
763 2.29412016778952e-05
764 2.28952994802967e-05
765 2.28655699174851e-05
766 2.28607550525339e-05
767 2.28642493311781e-05
768 2.28904536925256e-05
769 2.29252091230592e-05
770 2.30106161325239e-05
771 2.30580226343591e-05
772 2.31262274610344e-05
773 2.31197027460439e-05
774 2.30407567869406e-05
775 2.28391872951761e-05
776 2.25904605031246e-05
777 2.24080704356311e-05
778 2.23904808080988e-05
779 2.263445094286e-05
780 2.32264756050427e-05
781 2.41664620261872e-05
782 2.50308512477204e-05
783 2.5058619939955e-05
784 2.4083252355922e-05
785 2.31584526773077e-05
786 2.2771439034841e-05
787 2.28708795475541e-05
788 2.30019813898252e-05
789 2.29836787184468e-05
790 2.28063545364421e-05
791 2.25452131417114e-05
792 2.2320615244098e-05
793 2.2220381652005e-05
794 2.21752779907547e-05
795 2.2221580366022e-05
796 2.22733833652455e-05
797 2.23213555727853e-05
798 2.23444421862951e-05
799 2.23495808313601e-05
800 2.22864855459193e-05
801 2.22600992856314e-05
802 2.22420057980344e-05
803 2.22032685996965e-05
804 2.21898390009301e-05
805 2.21459285967285e-05
806 2.21213631448336e-05
807 2.21129084820859e-05
808 2.20993715629447e-05
809 2.2047721358831e-05
810 2.2050344341551e-05
811 2.20048696064623e-05
812 2.19820158235962e-05
813 2.19554058276117e-05
814 2.19685480260523e-05
815 2.19957109948155e-05
816 2.20465080928989e-05
817 2.21606587729184e-05
818 2.23093520617113e-05
819 2.25904641411034e-05
820 2.29176548600662e-05
821 2.32095680985367e-05
822 2.33845021284651e-05
823 2.31938574870583e-05
824 2.2811655071564e-05
825 2.24471987166908e-05
826 2.22546750592301e-05
827 2.23531769734109e-05
828 2.26878764806315e-05
829 2.31693138630362e-05
830 2.35618936130777e-05
831 2.34752224059775e-05
832 2.27847376663703e-05
833 2.19678986468352e-05
834 2.15249838220188e-05
835 2.15181607927661e-05
836 2.17016186070396e-05
837 2.19071425817674e-05
838 2.19401772483252e-05
839 2.18944787775399e-05
840 2.17701581277652e-05
841 2.16424796235515e-05
842 2.15513246075716e-05
843 2.15236450458178e-05
844 2.15027503145393e-05
845 2.15166601265082e-05
846 2.14515821426176e-05
847 2.1417119569378e-05
848 2.13779949262971e-05
849 2.13199127756525e-05
850 2.1266600015224e-05
851 2.1258923879941e-05
852 2.12106442631921e-05
853 2.12277773243841e-05
854 2.12836639548186e-05
855 2.1335979909054e-05
856 2.14177271118388e-05
857 2.15299460251117e-05
858 2.16091320908163e-05
859 2.16986300074495e-05
860 2.17211418203078e-05
861 2.16977696254617e-05
862 2.16026292036986e-05
863 2.14951833186205e-05
864 2.13510902540293e-05
865 2.12748200283386e-05
866 2.12760187423555e-05
867 2.13603707379662e-05
868 2.15205036511179e-05
869 2.1821182599524e-05
870 2.22206799662672e-05
871 2.24752784561133e-05
872 2.24018240260193e-05
873 2.1807010853081e-05
874 2.10741218324983e-05
875 2.06272234208882e-05
876 2.06736876862124e-05
877 2.10784510272788e-05
878 2.15711443161126e-05
879 2.18365421460476e-05
880 2.17447759496281e-05
881 2.13874209293863e-05
882 2.10008583962917e-05
883 2.081599268422e-05
884 2.07513839995954e-05
885 2.07569974008948e-05
886 2.07604261959204e-05
887 2.07080811378546e-05
888 2.06127497222042e-05
889 2.05065043701325e-05
890 2.04214702534955e-05
891 2.03512463485822e-05
892 2.03109138965374e-05
893 2.03402250917861e-05
894 2.03574272745755e-05
895 2.04102780116955e-05
896 2.05119849852053e-05
897 2.05282503884519e-05
898 2.05755968636367e-05
899 2.05651485885028e-05
900 2.05512624233961e-05
901 2.04891730390955e-05
902 2.04457282961812e-05
903 2.03911113203503e-05
904 2.03373001568252e-05
905 2.03144045372028e-05
906 2.02745595743181e-05
907 2.03071522264509e-05
908 2.03593735932373e-05
909 2.04828229470877e-05
910 2.06591121241217e-05
911 2.08346427825745e-05
912 2.09956069738837e-05
913 2.09876689041266e-05
914 2.07495722861495e-05
915 2.03167364816181e-05
916 1.99401692952961e-05
917 1.97899789782241e-05
918 2.00036283786176e-05
919 2.05774122150615e-05
920 2.1331283278414e-05
921 2.18323220906314e-05
922 2.16506341530476e-05
923 2.10355847229948e-05
924 2.04348652914632e-05
925 2.01690545509337e-05
926 2.01570983335841e-05
927 2.01921902771574e-05
928 2.01977873075521e-05
929 2.00937502086163e-05
930 1.99426940525882e-05
931 1.97799090528861e-05
932 1.96526088984683e-05
933 1.96230103028938e-05
934 1.96388318727259e-05
935 1.9680166587932e-05
936 1.97472254512832e-05
937 1.979413900699e-05
938 1.98489742615493e-05
939 1.98045672732405e-05
940 1.97906065295683e-05
941 1.97472236322938e-05
942 1.96996625163592e-05
943 1.96657983906334e-05
944 1.96333021449391e-05
945 1.96052751562092e-05
946 1.95854208868695e-05
947 1.95502580027096e-05
948 1.95596385310637e-05
949 1.95726242964156e-05
950 1.95756038010586e-05
951 1.95892753254157e-05
952 1.95946322492091e-05
953 1.95951361092739e-05
954 1.95796546904603e-05
955 1.95401735254563e-05
956 1.94527092389762e-05
957 1.93682899407577e-05
958 1.92396091733826e-05
959 1.91497001651442e-05
960 1.91248018381884e-05
961 1.92017469089478e-05
962 1.9533135855454e-05
963 2.02165083464934e-05
964 2.12269605981419e-05
965 2.22121943806997e-05
966 2.21906648221193e-05
967 2.10362941288622e-05
968 1.99800997506827e-05
969 1.97172103071352e-05
970 1.98743764485698e-05
971 2.01203783944948e-05
972 2.01073144125985e-05
973 1.97738900169497e-05
974 1.93090218090219e-05
975 1.89895927178441e-05
976 1.88602825801354e-05
977 1.88618723768741e-05
978 1.89153070095927e-05
979 1.89743423106847e-05
980 1.89900001714705e-05
981 1.89837592188269e-05
982 1.89414458873216e-05
983 1.89111651707208e-05
984 1.88789163075853e-05
985 1.88536960195052e-05
986 1.88306603376986e-05
987 1.88075555342948e-05
988 1.87798323167954e-05
989 1.87541645573219e-05
990 1.87085333891446e-05
991 1.86793677130481e-05
992 1.86319230124354e-05
993 1.86291545105632e-05
994 1.86023444257444e-05
995 1.8574648493086e-05
996 1.85808385140263e-05
997 1.85914523171959e-05
998 1.86244997166796e-05
999 1.86721081263386e-05
1000 1.87285186257213e-05
1001 1.88425110536627e-05
1002 1.90045775525505e-05
1003 1.92190727830166e-05
1004 1.94092626770725e-05
1005 1.95393131434685e-05
1006 1.95627271750709e-05
1007 1.94153653865214e-05
1008 1.91422186617274e-05
1009 1.89126785699045e-05
1010 1.88213070941856e-05
1011 1.88906396942912e-05
1012 1.92882234841818e-05
1013 1.99782971321838e-05
1014 2.07522716664243e-05
1015 2.10257439903216e-05
1016 2.01502534764586e-05
1017 1.87786208698526e-05
1018 1.80481820279965e-05
1019 1.81413433892885e-05
1020 1.84809705388034e-05
1021 1.86807938007405e-05
1022 1.86655306606553e-05
1023 1.85119315574411e-05
1024 1.83628490049159e-05
1025 1.82709827640792e-05
1026 1.82408293767367e-05
1027 1.82343410415342e-05
1028 1.8201521015726e-05
1029 1.81559644261142e-05
1030 1.81048053491395e-05
1031 1.80720708158333e-05
1032 1.801578400773e-05
1033 1.79919024958508e-05
1034 1.79672715603374e-05
1035 1.79757553269155e-05
1036 1.79722628672607e-05
1037 1.79871040018043e-05
1038 1.7988324543694e-05
1039 1.80275765160332e-05
1040 1.80303450179053e-05
1041 1.80111874215072e-05
1042 1.8023081793217e-05
1043 1.80020288098603e-05
1044 1.7992417269852e-05
1045 1.79559392563533e-05
1046 1.79453400050988e-05
1047 1.79022736119805e-05
1048 1.79061626113253e-05
1049 1.78825976036023e-05
1050 1.78636764758267e-05
1051 1.7875239791465e-05
1052 1.79127691808389e-05
1053 1.80233728315216e-05
1054 1.82433541340288e-05
1055 1.86368324648356e-05
1056 1.92263323697262e-05
1057 1.98521884158254e-05
1058 1.99484547920292e-05
1059 1.91331910173176e-05
1060 1.79452199517982e-05
1061 1.74047800101107e-05
1062 1.7824237147579e-05
1063 1.87545356311603e-05
1064 1.94246167666279e-05
1065 1.92628267541295e-05
1066 1.85236149263801e-05
1067 1.79620001290459e-05
1068 1.7753771317075e-05
1069 1.77115816768492e-05
1070 1.76597732206574e-05
1071 1.75841687450884e-05
1072 1.74970246007433e-05
1073 1.74162687471835e-05
1074 1.73793378053233e-05
1075 1.73718817677582e-05
1076 1.73714015545556e-05
1077 1.74057258846005e-05
1078 1.74000233528204e-05
1079 1.74041124409996e-05
1080 1.74214328581002e-05
1081 1.74017714016372e-05
1082 1.73828811966814e-05
1083 1.7376221876475e-05
1084 1.73322241607821e-05
1085 1.73393655131804e-05
1086 1.7326747183688e-05
1087 1.72923228092259e-05
1088 1.72709696926177e-05
1089 1.72600211953977e-05
1090 1.72187847056193e-05
1091 1.72098934854148e-05
1092 1.71829688042635e-05
1093 1.71724004758289e-05
1094 1.7128448234871e-05
1095 1.71241063071648e-05
1096 1.70999192050658e-05
1097 1.70723578776233e-05
1098 1.70677230926231e-05
1099 1.70383154909359e-05
1100 1.70225430338178e-05
1101 1.70241564774187e-05
1102 1.69975955941482e-05
1103 1.70525745488703e-05
1104 1.71301144291647e-05
1105 1.72923191712471e-05
1106 1.76357989403186e-05
1107 1.81542218342656e-05
1108 1.89535112440353e-05
1109 1.96660748770228e-05
1110 1.97060107893776e-05
1111 1.88542799151037e-05
1112 1.79462167579914e-05
1113 1.76772919076029e-05
1114 1.80960978468647e-05
1115 1.89057936950121e-05
1116 1.9483481082716e-05
1117 1.91529052244732e-05
1118 1.79386079253163e-05
1119 1.69123904925073e-05
1120 1.65830697369529e-05
1121 1.66221034305636e-05
1122 1.67067100846907e-05
1123 1.67467951541767e-05
1124 1.6760932339821e-05
1125 1.67443704413017e-05
1126 1.67485286510782e-05
1127 1.6746129404055e-05
1128 1.67425841937074e-05
1129 1.67222169693559e-05
1130 1.67066209542099e-05
1131 1.66808149515418e-05
1132 1.6640879039187e-05
1133 1.66261397680501e-05
1134 1.65872279467294e-05
1135 1.65878464031266e-05
1136 1.65656529134139e-05
1137 1.65768287843093e-05
1138 1.65765322890365e-05
1139 1.656145650486e-05
1140 1.65657456818735e-05
1141 1.65694327733945e-05
1142 1.65616620506626e-05
1143 1.65643814398209e-05
1144 1.65665405802429e-05
1145 1.65548553923145e-05
1146 1.65435358212562e-05
1147 1.65474757523043e-05
1148 1.6539721400477e-05
1149 1.65274050232256e-05
1150 1.65134806593414e-05
1151 1.65069104696158e-05
1152 1.65142319019651e-05
1153 1.65056317200651e-05
1154 1.65206747624325e-05
1155 1.65779565577395e-05
1156 1.66862937476253e-05
1157 1.68935512192547e-05
1158 1.73202006408246e-05
1159 1.80267561518122e-05
1160 1.89269394468283e-05
1161 1.93844625755446e-05
1162 1.85752978723031e-05
1163 1.69964150700253e-05
1164 1.61831376317423e-05
1165 1.66750069183763e-05
1166 1.7740005205269e-05
1167 1.83550710062264e-05
1168 1.797918775992e-05
1169 1.71934862009948e-05
1170 1.6703073924873e-05
1171 1.655636333453e-05
1172 1.65180790645536e-05
1173 1.64455213962356e-05
1174 1.63809090736322e-05
1175 1.63012609846191e-05
1176 1.62674277817132e-05
1177 1.62575706781354e-05
1178 1.62721207743743e-05
1179 1.62845244631171e-05
1180 1.63025997608202e-05
1181 1.63160311785759e-05
1182 1.6307534679072e-05
1183 1.63007989613106e-05
1184 1.63077238539699e-05
1185 1.63027598318877e-05
1186 1.62661908689188e-05
1187 1.62711530720117e-05
1188 1.62674205057556e-05
1189 1.6240161130554e-05
1190 1.62410105986055e-05
1191 1.62130145326955e-05
1192 1.6224952560151e-05
1193 1.62026099133072e-05
1194 1.6199390302063e-05
1195 1.61842071975116e-05
1196 1.61862189997919e-05
1197 1.61857897182927e-05
1198 1.6171808965737e-05
1199 1.61754305736395e-05
1200 1.61813932209043e-05
1201 1.6170271919691e-05
1202 1.61836378538283e-05
1203 1.61944481078535e-05
1204 1.62174874276388e-05
1205 1.62548276421148e-05
1206 1.63080549100414e-05
1207 1.63874319696333e-05
1208 1.65370729519054e-05
1209 1.67582111316733e-05
1210 1.70376042660791e-05
1211 1.74012366187526e-05
1212 1.77136153070023e-05
1213 1.77902584255207e-05
1214 1.75206660060212e-05
1215 1.70728671946563e-05
1216 1.67684993357398e-05
1217 1.6798772776383e-05
1218 1.73124826687854e-05
1219 1.83627544174669e-05
1220 1.95777174667455e-05
1221 1.97640983969904e-05
1222 1.81889645318734e-05
1223 1.63664426509058e-05
1224 1.57668418978574e-05
1225 1.59304163389606e-05
1226 1.61189691425534e-05
1227 1.61402349476703e-05
1228 1.60956060426543e-05
1229 1.60603358381195e-05
1230 1.60574090841692e-05
1231 1.60656545631355e-05
1232 1.60735617100727e-05
1233 1.60326944751432e-05
1234 1.59975279530045e-05
1235 1.59635801537661e-05
1236 1.59270857693627e-05
1237 1.59197716129711e-05
1238 1.59108785737772e-05
1239 1.59022947627818e-05
1240 1.5904148312984e-05
1241 1.59014853124972e-05
1242 1.5918387362035e-05
1243 1.59050505317282e-05
1244 1.59121991600841e-05
1245 1.59152714331867e-05
1246 1.58972288772929e-05
1247 1.59131977852667e-05
1248 1.58963121066336e-05
1249 1.59042683662847e-05
1250 1.5911013178993e-05
1251 1.58841357915662e-05
1252 1.58867023856146e-05
1253 1.58885377459228e-05
1254 1.5882706065895e-05
1255 1.58983311848715e-05
1256 1.59071732923621e-05
1257 1.59110713866539e-05
1258 1.5954699847498e-05
1259 1.59999799507204e-05
1260 1.60576255439082e-05
1261 1.61726329679368e-05
1262 1.63024451467209e-05
1263 1.64239245350473e-05
1264 1.65336932695936e-05
1265 1.6546369806747e-05
1266 1.64380489877658e-05
1267 1.61552688950906e-05
1268 1.58170278155012e-05
1269 1.56084406626178e-05
1270 1.56862060975982e-05
1271 1.62306751008146e-05
1272 1.72838063008385e-05
1273 1.84153177542612e-05
1274 1.86612978723133e-05
1275 1.76776175067062e-05
1276 1.65949877555249e-05
1277 1.62608339451253e-05
1278 1.63393815455493e-05
1279 1.64452321769204e-05
1280 1.63624827109743e-05
1281 1.61216576088918e-05
1282 1.58137972903205e-05
1283 1.56176993186818e-05
1284 1.55474353960017e-05
1285 1.55661491589854e-05
1286 1.56102123582968e-05
1287 1.56626283569494e-05
1288 1.56939404405421e-05
1289 1.56981805048417e-05
1290 1.56727419380331e-05
1291 1.56749811139889e-05
1292 1.56617243192159e-05
1293 1.56389269250212e-05
1294 1.56214227899909e-05
1295 1.56309633894125e-05
1296 1.56249552674126e-05
1297 1.56135174620431e-05
1298 1.55809630086878e-05
1299 1.55672623805003e-05
1300 1.55505167640513e-05
1301 1.5515821360168e-05
1302 1.55034067574888e-05
1303 1.54828449012712e-05
1304 1.54619956447277e-05
1305 1.54564440890681e-05
1306 1.5460811482626e-05
1307 1.54771732923109e-05
1308 1.55366506078281e-05
1309 1.56236073962646e-05
1310 1.57539743668167e-05
1311 1.59750761667965e-05
1312 1.6269987099804e-05
1313 1.65926412591944e-05
1314 1.68210826814175e-05
1315 1.67933667398756e-05
1316 1.65168094099499e-05
1317 1.61386633408256e-05
1318 1.59435876412317e-05
1319 1.60313720698468e-05
1320 1.64761349878972e-05
1321 1.73095158970682e-05
1322 1.81299292307813e-05
1323 1.83593401743565e-05
1324 1.73522766999668e-05
1325 1.58804286911618e-05
1326 1.51548892972642e-05
1327 1.52074098878074e-05
1328 1.5473757230211e-05
1329 1.55875914060744e-05
1330 1.55619072756963e-05
1331 1.548089676362e-05
1332 1.54207318701083e-05
1333 1.54030967678409e-05
1334 1.54220997501398e-05
1335 1.54063673107885e-05
1336 1.53845376189565e-05
1337 1.53491309902165e-05
1338 1.52858083311003e-05
1339 1.52642041939544e-05
1340 1.52269522004644e-05
1341 1.52202064782614e-05
1342 1.52066259033745e-05
1343 1.52188595166081e-05
1344 1.52275679283775e-05
1345 1.52556285684113e-05
1346 1.52501816046424e-05
1347 1.5259938663803e-05
1348 1.52556740431464e-05
1349 1.52610409713816e-05
1350 1.52772081492003e-05
1351 1.5244135283865e-05
1352 1.52469292515889e-05
1353 1.52545062519494e-05
1354 1.52768225234468e-05
1355 1.52784341480583e-05
1356 1.53361615957692e-05
1357 1.54010558617301e-05
1358 1.55044581333641e-05
1359 1.56865899043623e-05
1360 1.59057308337651e-05
1361 1.62023206939921e-05
1362 1.63973927556071e-05
1363 1.64360953931464e-05
1364 1.60741819854593e-05
1365 1.55244233610574e-05
1366 1.50365904119099e-05
1367 1.4969762560213e-05
1368 1.54176323121646e-05
1369 1.62504275067477e-05
1370 1.6982625311357e-05
1371 1.70752464327961e-05
1372 1.64001794473734e-05
1373 1.57451795530505e-05
1374 1.54698118421948e-05
1375 1.5468420315301e-05
1376 1.55258985614637e-05
1377 1.55340876517585e-05
1378 1.5435569366673e-05
1379 1.52477914525662e-05
1380 1.50872656377032e-05
1381 1.49845100168022e-05
1382 1.49563811646658e-05
1383 1.49787747432129e-05
1384 1.50220148498192e-05
1385 1.50717132783029e-05
1386 1.51088297570823e-05
1387 1.51187077790382e-05
1388 1.51111835293705e-05
1389 1.51113354149857e-05
1390 1.51118692883756e-05
1391 1.50838059198577e-05
1392 1.50846763062873e-05
1393 1.50968335219659e-05
1394 1.51189715325017e-05
1395 1.51172471305472e-05
1396 1.51659933180781e-05
1397 1.52086458911072e-05
1398 1.52209768202738e-05
1399 1.52492639244883e-05
1400 1.52703069034033e-05
1401 1.5251320292009e-05
1402 1.52143293234985e-05
1403 1.5149857972574e-05
1404 1.50310734170489e-05
1405 1.49404995681834e-05
1406 1.4844503311906e-05
1407 1.48314738908084e-05
1408 1.49283014252433e-05
1409 1.52045631693909e-05
1410 1.57373597176047e-05
1411 1.64833636517869e-05
1412 1.71562496689148e-05
1413 1.71990395756438e-05
1414 1.65224064403446e-05
1415 1.57790709636174e-05
1416 1.54658500832738e-05
1417 1.56657679326599e-05
1418 1.60171330207959e-05
1419 1.62669111887226e-05
1420 1.62052274390589e-05
1421 1.57904123625485e-05
1422 1.52480506585562e-05
1423 1.48477174661821e-05
1424 1.47032551467419e-05
1425 1.47375521919457e-05
1426 1.48237941175466e-05
1427 1.4885365999362e-05
1428 1.49108436744427e-05
1429 1.48975041156518e-05
1430 1.48818699017284e-05
1431 1.48709114000667e-05
1432 1.4869900951453e-05
1433 1.48574717968586e-05
1434 1.4874079170113e-05
1435 1.48800963870599e-05
1436 1.48917788465042e-05
1437 1.48733406604151e-05
1438 1.48764902405674e-05
1439 1.48350518429652e-05
1440 1.48026056194794e-05
1441 1.47681266753352e-05
1442 1.47385471791495e-05
1443 1.46929232869297e-05
1444 1.4674164049211e-05
1445 1.46629181472235e-05
1446 1.46851743920706e-05
1447 1.47215187098482e-05
1448 1.4816833754594e-05
1449 1.49416118802037e-05
1450 1.51278936755261e-05
1451 1.53591718117241e-05
1452 1.55987381731393e-05
1453 1.57537633640459e-05
1454 1.57580998347839e-05
1455 1.55774323502555e-05
1456 1.53155597217847e-05
1457 1.51347903738497e-05
1458 1.50972409755923e-05
1459 1.53561686602188e-05
1460 1.58801249199314e-05
1461 1.67141697602347e-05
1462 1.74429678736487e-05
1463 1.73833759617992e-05
1464 1.62584383360809e-05
1465 1.4961266060709e-05
1466 1.44380328492844e-05
1467 1.45718986459542e-05
1468 1.48515082400991e-05
1469 1.49595880429843e-05
1470 1.49038833114901e-05
1471 1.48026911119814e-05
1472 1.47390537676984e-05
1473 1.47236405609874e-05
1474 1.47298469528323e-05
1475 1.47468426803243e-05
1476 1.47160790220369e-05
1477 1.46946467793896e-05
1478 1.46524525916902e-05
1479 1.46085676533403e-05
1480 1.45664134834078e-05
1481 1.45683134178398e-05
1482 1.45430021802895e-05
1483 1.45514768519206e-05
1484 1.45566400533426e-05
1485 1.4576711691916e-05
1486 1.4571100109606e-05
1487 1.45869671541732e-05
1488 1.45855510709225e-05
1489 1.46029778989032e-05
1490 1.45942476592609e-05
1491 1.45900339703076e-05
1492 1.458337828808e-05
1493 1.45779367812793e-05
1494 1.45901285577565e-05
1495 1.45813519338844e-05
1496 1.46045931614935e-05
1497 1.46444581332617e-05
1498 1.47228329296922e-05
1499 1.48496919791796e-05
1500 1.50985742948251e-05
1501 1.55183952301741e-05
1502 1.6073423466878e-05
1503 1.66930749401217e-05
1504 1.69133982126368e-05
1505 1.62382511916803e-05
1506 1.50415253301617e-05
1507 1.43268553074449e-05
1508 1.45507119668764e-05
1509 1.53292803588556e-05
1510 1.60214676725445e-05
1511 1.60210365720559e-05
1512 1.54189929162385e-05
1513 1.49061870615697e-05
1514 1.46860775203095e-05
1515 1.46532083817874e-05
1516 1.46334732562536e-05
1517 1.45887224789476e-05
1518 1.45157509905403e-05
1519 1.44369423651369e-05
1520 1.43955103339977e-05
1521 1.43703609865042e-05
1522 1.43779006975819e-05
1523 1.43914267027867e-05
1524 1.4417564671021e-05
1525 1.44242803798988e-05
1526 1.44422911034781e-05
1527 1.44435389302089e-05
1528 1.44445994010312e-05
1529 1.44462310345261e-05
1530 1.44402438309044e-05
1531 1.44386140163988e-05
1532 1.44336909215781e-05
1533 1.44312825796078e-05
1534 1.4435243429034e-05
1535 1.44246159834438e-05
1536 1.44362174978596e-05
1537 1.44453433676972e-05
1538 1.44390933201066e-05
1539 1.44450777952443e-05
1540 1.44580017149565e-05
1541 1.44511068356223e-05
1542 1.44735895446502e-05
1543 1.4485511201201e-05
1544 1.4489491150016e-05
1545 1.45124167829636e-05
1546 1.45400590554345e-05
1547 1.45539261211525e-05
1548 1.45679587149061e-05
1549 1.45704889291665e-05
1550 1.45765243360074e-05
1551 1.4559371265932e-05
1552 1.45134581543971e-05
1553 1.44557052408345e-05
1554 1.43608431244502e-05
1555 1.42557719300385e-05
1556 1.42089129440137e-05
1557 1.42301587402471e-05
1558 1.44782152347034e-05
1559 1.51634121721145e-05
1560 1.65681904036319e-05
1561 1.8351098333369e-05
1562 1.87573987204814e-05
1563 1.68666501849657e-05
1564 1.52287839227938e-05
1565 1.50932364704204e-05
1566 1.55404959514271e-05
1567 1.5631434507668e-05
1568 1.5254446225299e-05
1569 1.47416767504183e-05
1570 1.43328616104554e-05
1571 1.41878454087419e-05
1572 1.41813334266772e-05
1573 1.42263870657189e-05
1574 1.4259576346376e-05
1575 1.42716971822665e-05
1576 1.42941680678632e-05
1577 1.42939061333891e-05
1578 1.42816152219893e-05
1579 1.42941589729162e-05
1580 1.42878661790746e-05
1581 1.42854805744719e-05
1582 1.42725120895193e-05
1583 1.42703574965708e-05
1584 1.4245846614358e-05
1585 1.423693538527e-05
1586 1.42436738315155e-05
1587 1.4228963664209e-05
1588 1.42285525726038e-05
1589 1.42369017339661e-05
1590 1.42232502184925e-05
1591 1.4232287867344e-05
1592 1.4242382349039e-05
1593 1.42365179272019e-05
1594 1.42467315527028e-05
1595 1.4260311218095e-05
1596 1.42396747833118e-05
1597 1.42460130518884e-05
1598 1.42575172503712e-05
1599 1.42420940392185e-05
1600 1.42524868351757e-05
1601 1.42466269608121e-05
1602 1.42443641379941e-05
1603 1.42413937282981e-05
1604 1.42365843203152e-05
1605 1.42400167533197e-05
1606 1.42351473186864e-05
1607 1.42346643769997e-05
1608 1.42595017678104e-05
1609 1.42948829306988e-05
1610 1.44104951687041e-05
1611 1.46559414133662e-05
1612 1.52106249515782e-05
1613 1.63882523338543e-05
1614 1.84409291250631e-05
1615 2.02816354430979e-05
1616 1.85061235242756e-05
1617 1.48650078699575e-05
1618 1.41387308758567e-05
1619 1.52755583258113e-05
1620 1.56910264195176e-05
1621 1.50633959492552e-05
1622 1.44744153658394e-05
1623 1.42432863867725e-05
1624 1.42089647852117e-05
1625 1.4172279406921e-05
1626 1.41108612297103e-05
1627 1.40738020490971e-05
1628 1.40750553327962e-05
1629 1.40817292049178e-05
1630 1.40815509439562e-05
1631 1.40978318086127e-05
1632 1.4104360161582e-05
1633 1.41019672810216e-05
1634 1.40946294777677e-05
1635 1.4095845472184e-05
1636 1.40983120218152e-05
1637 1.40939146149321e-05
1638 1.4089816431806e-05
1639 1.40893780553597e-05
1640 1.40851370815653e-05
1641 1.40808842843398e-05
1642 1.40748888952658e-05
1643 1.40806832860108e-05
1644 1.40740867209388e-05
1645 1.40760848807986e-05
1646 1.40794873004779e-05
1647 1.40813126563444e-05
1648 1.40602060128003e-05
1649 1.40749243655591e-05
1650 1.40643414852093e-05
1651 1.4060118701309e-05
1652 1.40572101372527e-05
1653 1.40696147354902e-05
1654 1.40518486659857e-05
1655 1.40636557262042e-05
1656 1.40555275720544e-05
1657 1.40499278131756e-05
1658 1.40494257721002e-05
1659 1.40546135298791e-05
1660 1.40435258799698e-05
1661 1.40490974445129e-05
1662 1.40389674925245e-05
1663 1.40373795147752e-05
1664 1.4045775969862e-05
1665 1.40291404022719e-05
1666 1.40363108585007e-05
1667 1.40358324642875e-05
1668 1.40253659992595e-05
1669 1.40202873808448e-05
1670 1.40193551487755e-05
1671 1.40196889333311e-05
1672 1.40139918585191e-05
1673 1.40234105856507e-05
1674 1.40263300636434e-05
1675 1.40639212986571e-05
1676 1.41329555845005e-05
1677 1.4291183106252e-05
1678 1.46934717122349e-05
1679 1.55886555148754e-05
1680 1.71987267094664e-05
1681 1.89180227607721e-05
1682 1.86177421710454e-05
1683 1.62435171660036e-05
1684 1.50800242408877e-05
1685 1.58273433044087e-05
1686 1.69515969901113e-05
1687 1.70329094544286e-05
1688 1.57402337208623e-05
1689 1.44144296427839e-05
1690 1.38968616738566e-05
1691 1.38521818371373e-05
1692 1.39144576678518e-05
1693 1.39365665745572e-05
1694 1.39790427056141e-05
1695 1.40020338221802e-05
1696 1.40141755764489e-05
1697 1.40209549499559e-05
1698 1.40081510835444e-05
1699 1.3982345990371e-05
1700 1.3971879525343e-05
1701 1.39486364787444e-05
1702 1.39438861879171e-05
1703 1.39401045089471e-05
1704 1.39386856972123e-05
1705 1.39337389555294e-05
1706 1.39280346047599e-05
1707 1.39353969643707e-05
1708 1.39233216032153e-05
1709 1.39243447847548e-05
1710 1.39251314976718e-05
1711 1.39229696287657e-05
1712 1.39234907692298e-05
1713 1.3914048395236e-05
1714 1.39166340886732e-05
1715 1.39003068397869e-05
1716 1.39123767439742e-05
1717 1.38990517370985e-05
1718 1.38980294650537e-05
1719 1.3892423339712e-05
1720 1.38886834974983e-05
1721 1.3882484381611e-05
1722 1.38865871122107e-05
1723 1.38715222419705e-05
1724 1.38783343572868e-05
1725 1.38690766107175e-05
1726 1.3866266272089e-05
1727 1.38629429784487e-05
1728 1.3856455552741e-05
1729 1.38671948661795e-05
1730 1.38548266477301e-05
1731 1.38476161737344e-05
1732 1.38554050863604e-05
1733 1.38529740070226e-05
1734 1.38659233925864e-05
1735 1.38828099807142e-05
1736 1.39160747494316e-05
1737 1.40047950480948e-05
1738 1.4172694136505e-05
1739 1.45503818202997e-05
1740 1.52634675032459e-05
1741 1.63469794642879e-05
1742 1.74222768691834e-05
1743 1.7317024685326e-05
1744 1.58479560923297e-05
1745 1.48217568494147e-05
1746 1.50336763908854e-05
1747 1.60119980137097e-05
1748 1.68191745615331e-05
1749 1.64816428878112e-05
1750 1.51303229358746e-05
1751 1.4033427760296e-05
1752 1.37093838930014e-05
1753 1.37288488986087e-05
1754 1.37830420499085e-05
1755 1.38234272526461e-05
1756 1.38457708089845e-05
1757 1.38765208248515e-05
1758 1.38956302180304e-05
1759 1.38988107210025e-05
1760 1.38923696795246e-05
1761 1.38768009492196e-05
1762 1.38486275318428e-05
1763 1.38253808472655e-05
1764 1.38148398036719e-05
1765 1.38018367579207e-05
1766 1.3795306585962e-05
1767 1.38041214086115e-05
1768 1.37930328492075e-05
1769 1.38028099172516e-05
1770 1.37949273266713e-05
1771 1.37913912112708e-05
1772 1.38096138471155e-05
1773 1.37903734866995e-05
1774 1.37957185870619e-05
1775 1.38039331432083e-05
1776 1.37960996653419e-05
1777 1.37977831400349e-05
1778 1.37903207360068e-05
1779 1.37777060444932e-05
1780 1.38004734253627e-05
1781 1.37978249767912e-05
1782 1.37920196721097e-05
1783 1.37823963086703e-05
1784 1.37886909215013e-05
1785 1.37916231324198e-05
1786 1.37779707074515e-05
1787 1.37980841827812e-05
1788 1.37819170049625e-05
1789 1.3806386050419e-05
1790 1.38144578158972e-05
1791 1.38390623760642e-05
1792 1.388035252603e-05
1793 1.39627754833782e-05
1794 1.40872889460297e-05
1795 1.43145734909922e-05
1796 1.46944012158201e-05
1797 1.53020464495057e-05
1798 1.60737872647587e-05
1799 1.65648834808962e-05
1800 1.60101008077618e-05
1801 1.45941403388861e-05
1802 1.36438829940744e-05
1803 1.39523190227919e-05
1804 1.51423000716022e-05
1805 1.60591698659118e-05
1806 1.57018712343415e-05
1807 1.46836364365299e-05
1808 1.40627171276719e-05
1809 1.38919276650995e-05
1810 1.38711748149944e-05
1811 1.38158366098651e-05
1812 1.37168462970294e-05
1813 1.36596272568568e-05
1814 1.36206308525288e-05
1815 1.36426278913859e-05
1816 1.36502931127325e-05
1817 1.36832650241558e-05
1818 1.36964408739004e-05
1819 1.37017959787045e-05
1820 1.36943035613513e-05
1821 1.37045826704707e-05
1822 1.36743574330467e-05
1823 1.36803537316155e-05
1824 1.36902317535714e-05
1825 1.36602575366851e-05
1826 1.36791877594078e-05
1827 1.36599774123169e-05
1828 1.36502176246722e-05
1829 1.36465350806247e-05
1830 1.36564503918635e-05
1831 1.36356111397617e-05
1832 1.36491025841678e-05
1833 1.36412199935876e-05
1834 1.36409744300181e-05
1835 1.36448543344159e-05
1836 1.36510789161548e-05
1837 1.36614589791861e-05
1838 1.36664384626783e-05
1839 1.36811040647444e-05
1840 1.3692715583602e-05
1841 1.37108891067328e-05
1842 1.37459219331504e-05
1843 1.37679271574598e-05
1844 1.38067043735646e-05
1845 1.38677105496754e-05
1846 1.39442145155044e-05
1847 1.40479205583688e-05
1848 1.41425844049081e-05
1849 1.42505377880298e-05
1850 1.43462566484232e-05
1851 1.43648740049684e-05
1852 1.43189517984865e-05
1853 1.42046192195266e-05
1854 1.40562679007417e-05
1855 1.39692037919303e-05
1856 1.40259826366673e-05
1857 1.43724464578554e-05
1858 1.52443999468233e-05
1859 1.67611178767402e-05
1860 1.85052031156374e-05
1861 1.82490130100632e-05
1862 1.53645996761043e-05
1863 1.34767569761607e-05
1864 1.35811396830832e-05
1865 1.39825669975835e-05
1866 1.39814192152699e-05
1867 1.37817523864214e-05
1868 1.36818453029264e-05
1869 1.3621438483824e-05
1870 1.36275230033789e-05
1871 1.36024827952497e-05
1872 1.3571896488429e-05
1873 1.35378777486039e-05
1874 1.35148002300411e-05
1875 1.35070413307403e-05
1876 1.34963884192985e-05
1877 1.35099435283337e-05
1878 1.35081154439831e-05
1879 1.35104683067766e-05
1880 1.3530927390093e-05
1881 1.35249192680931e-05
1882 1.35304926516255e-05
1883 1.35169857458095e-05
1884 1.35277268782374e-05
1885 1.35269146994688e-05
1886 1.35110358314705e-05
1887 1.35178916025325e-05
1888 1.3507923540601e-05
1889 1.35185146064032e-05
1890 1.35038626467576e-05
1891 1.34897672978695e-05
1892 1.34966476252885e-05
1893 1.34912215798977e-05
1894 1.34934753077687e-05
1895 1.34788879222469e-05
1896 1.3494935046765e-05
1897 1.34806386995479e-05
1898 1.34804931803956e-05
1899 1.34872043418e-05
1900 1.3497819963959e-05
1901 1.35017717184382e-05
1902 1.35333830257878e-05
1903 1.35625123220962e-05
1904 1.362155762763e-05
1905 1.37260567498743e-05
1906 1.39126595968264e-05
1907 1.41892433020985e-05
1908 1.45778476507985e-05
1909 1.50266223499784e-05
1910 1.52826814883156e-05
1911 1.51362828546553e-05
1912 1.46253523780615e-05
1913 1.4173444469634e-05
1914 1.41135096782818e-05
1915 1.45734156831168e-05
1916 1.54780682350975e-05
1917 1.64967732416699e-05
1918 1.67771358974278e-05
1919 1.55416655616136e-05
1920 1.39611138365581e-05
1921 1.32945924633532e-05
1922 1.33372004711418e-05
1923 1.35086966110975e-05
1924 1.35474274429725e-05
1925 1.35148302433663e-05
1926 1.3499338820111e-05
1927 1.34955644170986e-05
1928 1.35061100081657e-05
1929 1.35195759867202e-05
1930 1.35040781970019e-05
1931 1.34937881739461e-05
1932 1.34446618176298e-05
1933 1.34291958602262e-05
1934 1.33947487483965e-05
1935 1.33905896291253e-05
1936 1.33834892039886e-05
1937 1.33763187477598e-05
1938 1.33931725940784e-05
1939 1.33857038235874e-05
1940 1.33955472847447e-05
1941 1.34044703372638e-05
1942 1.3401773685473e-05
1943 1.34072079163161e-05
1944 1.34075698952074e-05
1945 1.34123201860348e-05
1946 1.34062020151759e-05
1947 1.34117481138674e-05
1948 1.34184774651658e-05
1949 1.34160900415736e-05
1950 1.34129368234426e-05
1951 1.34456886371481e-05
1952 1.34452429847443e-05
1953 1.34769807118573e-05
1954 1.35201107696048e-05
1955 1.3582698557002e-05
1956 1.367831191601e-05
1957 1.38404693643679e-05
1958 1.40669935717597e-05
1959 1.43732868309598e-05
1960 1.47131440826342e-05
1961 1.49443994814646e-05
1962 1.48493136293837e-05
1963 1.43064580697683e-05
1964 1.36002809085767e-05
1965 1.32321574710659e-05
1966 1.34865313157206e-05
1967 1.43478446261724e-05
1968 1.54033150465693e-05
1969 1.57294380187523e-05
1970 1.50096320794546e-05
1971 1.40379734148155e-05
1972 1.36609669425525e-05
1973 1.36365279104211e-05
1974 1.36690850922605e-05
1975 1.36200287670363e-05
1976 1.35007321659941e-05
1977 1.33607472889707e-05
1978 1.32782843138557e-05
1979 1.32401673909044e-05
1980 1.32558698169305e-05
1981 1.32892646433902e-05
1982 1.33282519527711e-05
1983 1.33330131575349e-05
1984 1.33590028781327e-05
1985 1.33602825371781e-05
1986 1.33512548927683e-05
1987 1.33438579723588e-05
1988 1.3345208571991e-05
1989 1.33434377858066e-05
1990 1.33325484057423e-05
1991 1.33250669023255e-05
1992 1.33381245177588e-05
1993 1.33252515297499e-05
1994 1.33184876176529e-05
1995 1.33183593788999e-05
1996 1.33092999021756e-05
1997 1.32876248244429e-05
1998 1.32954537548358e-05
1999 1.32817740450264e-05
};
\addlegendentry{Test}

\nextgroupplot[
title={4 Layer},
ymin=0.00584487338717405, ymax=0.01,
]
\addplot [semithick, black, dashed]
table {%
0 0.0123159520589979
1 0.0120827574428404
2 0.0118545932491543
3 0.0116323878173716
4 0.0114164739643456
5 0.0112069199749385
6 0.0110036929763737
7 0.0108067119508632
8 0.010615877174132
9 0.010431077076646
10 0.0102521941371378
11 0.0100791102668154
12 0.00991170501947636
13 0.00974985445645871
14 0.00959343695649295
15 0.0094423294722219
16 0.00929641055336106
17 0.00915555826213676
18 0.00901965146113071
19 0.00888856968595064
20 0.00876219477140694
21 0.008640407515486
22 0.00852309044785216
23 0.00841012840464828
24 0.00830140591642703
25 0.00819681014945672
26 0.00809622807173582
27 0.00799954816102399
28 0.00790666139073437
29 0.00781745918266097
30 0.00773183389537735
31 0.00764968163093727
32 0.00757089744047335
33 0.00749537921501542
34 0.00742302571006803
35 0.0073537377725188
36 0.00728741778993935
37 0.00722397011008979
38 0.00716330035265855
39 0.00710531515767343
40 0.00704992400977744
41 0.00699703743202917
42 0.00694656821478645
43 0.00689843095983811
44 0.00685254111196605
45 0.00680881732282401
46 0.00676717812089578
47 0.00672754593023228
48 0.0066898445288075
49 0.00665399743314765
50 0.00661993324240484
51 0.00658758047563879
52 0.00655686900927321
53 0.00652773287561104
54 0.0065001053026208
55 0.0064739228732833
56 0.00644912402458431
57 0.00642564883128216
58 0.0064034378751785
59 0.00638243634875835
60 0.0063625882417
61 0.00634384179738845
62 0.00632614451114932
63 0.00630944855402049
64 0.00629370470414869
65 0.00627886813890655
66 0.0062648936154801
67 0.00625173926619027
68 0.00623936360170774
69 0.00622772635961155
70 0.00621679065989156
71 0.00620651975350484
72 0.0061968787385922
73 0.00618783375375642
74 0.00617935282480175
75 0.00617140502981783
76 0.00616396156328847
77 0.00615699385434709
78 0.00615047485825926
79 0.00614437980766525
80 0.00613868408254348
81 0.00613336378592066
82 0.00612839784298558
83 0.00612376444951224
84 0.00611944405318354
85 0.00611541725265852
86 0.00611166650014638
87 0.00610817456981749
88 0.00610492582200095
89 0.00610190432780655
90 0.0060990953916189
91 0.00609648583304079
92 0.00609406344301533
93 0.00609181489016919
94 0.00608972934423946
95 0.00608779534741188
96 0.00608600377745461
97 0.00608434402965941
98 0.00608280753658619
99 0.00608138583811524
100 0.00608007163464208
101 0.00607885617318971
102 0.00607773283809365
103 0.00607669583405368
104 0.00607573848901666
105 0.00607485549335252
106 0.00607404017682711
107 0.00607328958540165
108 0.00607259747630451
109 0.00607196008786559
110 0.0060713735238096
111 0.00607083332943148
112 0.00607033669984958
113 0.00606988005893072
114 0.00606946036350564
115 0.00606907475412299
116 0.00606872050411766
117 0.00606839594729536
118 0.00606809726741631
119 0.00606782359864155
120 0.00606757287278015
121 0.0060673427196889
122 0.00606713208071596
123 0.00606693900226674
124 0.00606676239294757
125 0.00606660065022879
126 0.00606645292464236
127 0.00606631742448371
128 0.00606619343307102
129 0.00606608042835433
130 0.00606597715704993
131 0.00606588281880249
132 0.00606579637678806
133 0.00606571755270124
134 0.0060656461155304
135 0.00606558021172532
136 0.00606552057251974
137 0.00606546596100088
138 0.00606541619526979
139 0.00606537074236257
140 0.00606532958590833
141 0.00606529218021024
142 0.00606525817238435
143 0.00606522682028299
144 0.00606519844404829
145 0.00606517293090292
146 0.00606514926585078
147 0.00606512790909619
148 0.00606510889156198
149 0.00606509125100274
150 0.00606507534939738
151 0.00606506081021507
152 0.00606504803727148
153 0.00606503603739839
154 0.0060650251853076
155 0.00606501523543557
156 0.00606500649519148
157 0.00606499855166476
158 0.00606499107198033
159 0.00606498466913763
160 0.00606497868648148
161 0.00606497370245052
162 0.00606496867658279
163 0.00606496454201988
164 0.00606496045111271
165 0.00606495720421663
166 0.00606495403917506
167 0.00606495137435559
168 0.00606494846942951
169 0.0060649464285234
170 0.00606494449493766
171 0.0060649428578472
172 0.00606494132625812
173 0.00606494031308102
174 0.00606493908708217
175 0.00606493785016937
176 0.00606493695886456
177 0.00606493634222716
178 0.00606493561645038
179 0.00606493530358421
180 0.00606493503255479
181 0.00606493444320222
182 0.00606493431951094
183 0.00606493387203955
184 0.00606493386658258
185 0.00606493376471917
186 0.00606493374107231
187 0.00606493400846375
188 0.00606493403756758
189 0.00606493387022056
190 0.00606493425584631
191 0.00606493473242153
192 0.00606493460873025
193 0.00606493499799399
194 0.00606493518353091
195 0.00606493550003506
196 0.00606493618761306
197 0.00606493634768412
198 0.00606493686791509
199 0.00606493711165967
200 0.00606493726809276
201 0.00606493772647809
202 0.00606493817031151
203 0.0060649383085547
204 0.00606493899249472
205 0.00606493947998388
206 0.00606494004387059
207 0.00606494035673677
208 0.0060649405058939
209 0.00606494106614264
210 0.00606494127350743
211 0.00606494215753628
212 0.0060649423430732
213 0.00606494292151183
214 0.00606494353269227
215 0.00606494388557621
216 0.00606494449675665
217 0.00606494503153954
218 0.00606494498788379
219 0.00606494592648232
220 0.00606494607200148
221 0.00606494670137181
222 0.00606494696694426
223 0.00606494738531183
224 0.00606494772364385
225 0.00606494788371492
226 0.00606494875319186
227 0.00606494930980261
228 0.00606494931707857
229 0.00606495014653774
230 0.00606495065221679
231 0.00606495062675094
232 0.00606495132524287
233 0.00606495152896969
234 0.00606495204920066
235 0.00606495240936056
236 0.00606495260581141
237 0.00606495319880196
238 0.00606495357715175
239 0.00606495403189911
240 0.00606495433021337
241 0.0060649547704088
242 0.00606495502506732
243 0.00606495536703733
244 0.00606495589090628
245 0.00606495594911394
246 0.00606495651300065
247 0.0060649566767097
248 0.00606495718602673
249 0.0060649572078546
250 0.00606495747342706
251 0.00606495831380016
252 0.00606495828105835
253 0.00606495890315273
254 0.00606495898682624
255 0.00606495927786455
256 0.00606495924148476
257 0.00606495971805998
258 0.00606495995816658
259 0.00606496078398777
260 0.00606496078762575
261 0.00606496114778565
262 0.00606496128239087
263 0.00606496146065183
264 0.00606496178079396
265 0.0060649624101643
266 0.00606496263935696
267 0.00606496275941026
268 0.00606496299587889
269 0.00606496311593219
270 0.0060649635779555
271 0.00606496378532029
272 0.00606496406180668
273 0.00606496399996104
274 0.00606496452382999
275 0.00606496459658956
276 0.0060649646766251
277 0.00606496522232192
278 0.0060649652259599
279 0.00606496562977554
280 0.0060649657571048
281 0.00606496589171002
282 0.0060649662555079
283 0.00606496626278386
284 0.00606496646287269
285 0.00606496684122249
286 0.00606496698674164
287 0.00606496698310366
288 0.00606496746331686
289 0.00606496752880048
290 0.00606496785258059
291 0.00606496812179103
292 0.00606496825639624
293 0.0060649685729004
294 0.0060649686238321
295 0.00606496882755891
296 0.00606496898762998
297 0.00606496922046063
298 0.00606496928594424
299 0.00606496936961776
300 0.00606496961700032
301 0.00606496971158776
302 0.00606496993714245
303 0.00606497010448948
304 0.00606497043918353
305 0.00606497048647725
306 0.00606497057015076
307 0.00606497077751555
308 0.00606497083572322
309 0.00606497099579428
310 0.00606497113767546
311 0.00606497127955663
312 0.00606497139960993
313 0.00606497158514685
314 0.00606497167609632
315 0.00606497182525345
316 0.00606497183252941
317 0.00606497205080814
318 0.00606497211629176
319 0.00606497214903357
320 0.00606497242188198
321 0.00606497265107464
322 0.00606497271655826
323 0.00606497263652273
324 0.00606497272019624
325 0.00606497244734783
326 0.00606497278204188
327 0.00606497286571539
328 0.00606497311673593
329 0.00606497323315125
330 0.0060649732586171
331 0.00606497353146551
332 0.00606497358603519
333 0.00606497358603519
334 0.0060649738770735
335 0.00606497386252158
336 0.00606497400076478
337 0.00606497407716233
338 0.00606497408080031
339 0.00606497408807627
340 0.00606497402986861
341 0.00606497417902574
342 0.00606497449189192
343 0.00606497459011734
344 0.00606497464468703
345 0.00606497475382639
346 0.00606497482294799
347 0.00606497479384416
348 0.00606497487387969
349 0.00606497523403959
350 0.00606497511398629
351 0.00606497510671034
352 0.00606497525950545
353 0.0060649751467281
354 0.00606497539774864
355 0.00606497536500683
356 0.00606497543049045
357 0.00606497554326779
358 0.00606497560147545
359 0.00606497571789077
360 0.00606497563421726
361 0.0060649756960629
362 0.00606497585249599
363 0.00606497586704791
364 0.0060649758852378
365 0.00606497601620504
366 0.00606497601256706
367 0.00606497601256706
368 0.00606497613625834
369 0.00606497611079249
370 0.00606497623812174
371 0.0060649762453977
372 0.00606497618355206
373 0.00606497629632941
374 0.00606497638727888
375 0.00606497640910675
376 0.00606497643093462
377 0.00606497650005622
378 0.00606497661647154
379 0.00606497658009175
380 0.00606497674016282
381 0.00606497675471473
382 0.00606497681656037
383 0.00606497687476804
384 0.00606497691842378
385 0.00606497688204399
386 0.00606497699118336
387 0.00606497697299346
388 0.00606497694752761
389 0.00606497691842378
390 0.00606497703120112
391 0.00606497703120112
392 0.00606497714397847
393 0.00606497715853038
394 0.0060649771548924
395 0.00606497716580634
396 0.00606497721673804
397 0.00606497738044709
398 0.00606497742774081
399 0.00606497749322443
400 0.00606497743501677
401 0.00606497752960422
402 0.00606497756598401
403 0.00606497766420944
404 0.00606497758053592
405 0.0060649775841739
406 0.00606497766420944
407 0.00606497770058922
408 0.00606497768239933
409 0.00606497778426274
410 0.0060649778934021
411 0.00606497784610838
412 0.00606497812623275
413 0.00606497811895679
414 0.00606497807893902
415 0.00606497816261253
416 0.00606497821354424
417 0.00606497818444041
418 0.00606497831540764
419 0.00606497830449371
420 0.00606497835542541
421 0.00606497825719998
422 0.00606497824628605
423 0.00606497831176966
424 0.00606497838816722
425 0.00606497836270137
426 0.00606497835906339
427 0.00606497838452924
428 0.00606497837725328
429 0.00606497847184073
430 0.00606497843546094
431 0.0060649784609268
432 0.00606497849730658
433 0.00606497848639265
434 0.00606497848275467
435 0.00606497850458254
436 0.00606497857734212
437 0.00606497861008393
438 0.00606497863554978
439 0.00606497867192957
440 0.00606497874468914
441 0.00606497877379297
442 0.00606497872649925
443 0.0060649787010334
444 0.0060649787519651
445 0.00606497883563861
446 0.00606497883563861
447 0.00606497878470691
448 0.00606497876287904
449 0.00606497875924106
450 0.0060649787519651
451 0.00606497879198287
452 0.00606497870830935
453 0.00606497870467138
454 0.00606497871558531
455 0.00606497867556754
456 0.0060649786828435
457 0.00606497877743095
458 0.00606497913395287
459 0.00606497896296787
460 0.00606497900298564
461 0.00606497903208947
462 0.00606497898479574
463 0.00606497902845149
464 0.00606497907210723
465 0.00606497890476021
466 0.00606497893386404
467 0.00606497891931213
468 0.00606497909757309
469 0.00606497904664138
470 0.00606497915214277
471 0.00606497915214277
472 0.0060649790611933
473 0.00606497895205393
474 0.0060649789920717
475 0.00606497899570968
476 0.00606497905755532
477 0.0060649790430034
478 0.0060649790430034
479 0.00606497902481351
480 0.00606497909029713
481 0.00606497909029713
482 0.0060649791303149
483 0.00606497911576298
484 0.00606497900662362
485 0.00606497900298564
486 0.00606497909757309
487 0.00606497909029713
488 0.00606497914850479
489 0.00606497914850479
490 0.00606497917760862
491 0.00606497908665915
492 0.00606497908302117
493 0.00606497908302117
494 0.00606497911576298
495 0.00606497911576298
496 0.00606497907938319
497 0.00606497915941873
498 0.00606497908665915
499 0.00606497909393511
500 0.0060649791303149
501 0.00606497931221384
502 0.0060649792503682
503 0.00606497920307447
504 0.00606497927947203
505 0.00606497921398841
506 0.00606497921398841
507 0.00606497922854032
508 0.00606497917760862
509 0.00606497935950756
510 0.00606497922854032
511 0.00606497924673022
512 0.00606497924673022
513 0.00606497926855809
514 0.00606497939588735
515 0.00606497935950756
516 0.00606497940680129
517 0.00606497950866469
518 0.00606497946137097
519 0.00606497952685459
520 0.00606497920307447
521 0.00606497916669468
522 0.00606497919943649
523 0.00606497940316331
524 0.00606497946137097
525 0.00606497945409501
526 0.00606497942499118
527 0.00606497947592288
528 0.00606497941407724
529 0.00606497939588735
530 0.00606497948683682
531 0.00606497946500895
532 0.00606497939224937
533 0.00606497942862916
534 0.00606497951230267
535 0.00606497949411278
536 0.00606497940316331
537 0.00606497948319884
538 0.00606497953049256
539 0.0060649794213532
540 0.00606497924309224
541 0.00606497959961416
542 0.00606497959597618
543 0.00606497928674798
544 0.00606497929038596
545 0.00606497929038596
546 0.00606497929038596
547 0.00606497926492011
548 0.00606497962871799
549 0.00606497946864692
550 0.00606497942499118
551 0.0060649794213532
552 0.0060649794213532
553 0.0060649794213532
554 0.0060649794213532
555 0.0060649794213532
556 0.0060649794213532
557 0.00606497953413054
558 0.00606497953413054
559 0.00606497946137097
560 0.00606497949775076
561 0.00606497964690789
562 0.00606497964690789
563 0.00606497964690789
564 0.00606497964690789
565 0.00606497964690789
566 0.00606497952321661
567 0.00606497948683682
568 0.00606497957051033
569 0.00606497965782182
570 0.00606497962508001
571 0.00606497958870023
572 0.00606497962871799
573 0.00606497945045703
574 0.00606497950138873
575 0.00606497954868246
576 0.00606497954868246
577 0.00606497950502671
578 0.00606497986154864
579 0.00606497978878906
580 0.00606497987973853
581 0.00606497999615385
582 0.00606498001070577
583 0.00606497994158417
584 0.00606497953413054
585 0.0060649795923382
586 0.00606497957051033
587 0.00606497962871799
588 0.00606497959961416
589 0.00606497967237374
590 0.0060649796614598
591 0.00606497965054587
592 0.00606497973421938
593 0.00606497975968523
594 0.00606497976696119
595 0.0060649795923382
596 0.00606497956323437
597 0.0060649797305814
598 0.00606497943226714
599 0.00606497942499118
600 0.00606497957414831
601 0.00606497964690789
602 0.00606497964690789
603 0.00606497964690789
604 0.00606497964690789
605 0.00606497964690789
606 0.00606497964690789
607 0.00606497964690789
608 0.00606497955232044
609 0.00606497939224937
610 0.00606497974513331
611 0.00606497977787512
612 0.00606497944318107
613 0.00606497944318107
614 0.00606497959597618
615 0.0060649796796497
616 0.00606497967237374
617 0.0060649795923382
618 0.00606497970875353
619 0.00606497975968523
620 0.00606497979242704
621 0.00606497983608278
622 0.00606497989429045
623 0.00606497972330544
624 0.00606497971602948
625 0.00606497981789289
626 0.0060649794722849
627 0.00606497944318107
628 0.00606497954868246
629 0.0060649794904748
630 0.0060649795414065
631 0.00606497963963193
632 0.00606497963963193
633 0.00606497963963193
634 0.00606497962144203
635 0.00606497957414831
636 0.00606497952321661
637 0.00606497946500895
638 0.00606497945773299
639 0.00606497945773299
640 0.00606497944318107
641 0.00606497947956086
642 0.00606497947956086
643 0.00606497947956086
644 0.00606497947956086
645 0.00606497947956086
646 0.00606497947956086
647 0.00606497946500895
648 0.00606497953413054
649 0.00606497953413054
650 0.00606497953413054
651 0.00606497952321661
652 0.00606497949775076
653 0.00606497949775076
654 0.00606497947956086
655 0.0060649794213532
656 0.0060649797305814
657 0.00606497977787512
658 0.00606497965054587
659 0.0060649797815131
660 0.00606497976332321
661 0.00606497978515108
662 0.00606497990884236
663 0.00606497987610055
664 0.00606497977059917
665 0.00606497971966746
666 0.00606497968692565
667 0.00606497959597618
668 0.00606497969783959
669 0.00606497971602948
670 0.00606497961416608
671 0.00606497945045703
672 0.006064979799703
673 0.00606497962871799
674 0.00606497960689012
675 0.00606497974513331
676 0.00606497968328767
677 0.00606497974149534
678 0.00606497970147757
679 0.00606497979242704
680 0.00606497967601172
681 0.00606497962508001
682 0.00606497962871799
683 0.00606497964326991
684 0.00606497961416608
685 0.0060649797305814
686 0.00606497965782182
687 0.00606497959961416
688 0.00606497992703225
689 0.00606497997796396
690 0.00606497997796396
691 0.00606497997796396
692 0.00606497997796396
693 0.00606497997796396
694 0.00606497997796396
695 0.00606497997796396
696 0.00606497997796396
697 0.00606497997796396
698 0.00606497997796396
699 0.00606497997796396
700 0.00606497997796396
701 0.00606497997796396
702 0.00606497997796396
703 0.00606497997796396
704 0.00606497997796396
705 0.00606497997796396
706 0.00606497997796396
707 0.00606497997796396
708 0.00606497997796396
709 0.00606497997796396
710 0.00606497997796396
711 0.00606497997796396
712 0.00606497997796396
713 0.00606497997796396
714 0.00606497997796396
715 0.00606497997796396
716 0.00606497997796396
717 0.00606497997796396
718 0.00606497997796396
719 0.00606497997796396
720 0.00606497997796396
721 0.00606497997796396
722 0.00606497997796396
723 0.00606497997796396
724 0.00606497997796396
725 0.00606497997796396
726 0.00606497997796396
727 0.00606497997796396
728 0.00606497997796396
729 0.00606497997796396
730 0.00606497997796396
731 0.00606497997796396
732 0.00606497997796396
733 0.00606497997796396
734 0.00606497997796396
735 0.00606497997796396
736 0.00606497997796396
737 0.00606497997796396
738 0.00606497997796396
739 0.00606497997796396
740 0.00606497997796396
741 0.00606497997796396
742 0.00606497965054587
743 0.00606497959961416
744 0.00606497965418384
745 0.00606497965418384
746 0.00606497965418384
747 0.00606497971966746
748 0.00606497972330544
749 0.00606497971239151
750 0.00606497960325214
751 0.00606497962871799
752 0.0060649796796497
753 0.00606497968328767
754 0.00606497968328767
755 0.00606497973421938
756 0.00606497971602948
757 0.0060649796614598
758 0.00606497971602948
759 0.0060649797305814
760 0.0060649797305814
761 0.0060649797305814
762 0.0060649797305814
763 0.0060649797305814
764 0.0060649797305814
765 0.00606497975240927
766 0.00606497981425491
767 0.00606497981425491
768 0.00606497981425491
769 0.00606497981425491
770 0.00606497981425491
771 0.00606497981425491
772 0.00606497981425491
773 0.00606497981425491
774 0.00606497981425491
775 0.00606497981425491
776 0.00606497981425491
777 0.00606497981425491
778 0.00606497981425491
779 0.00606497981425491
780 0.00606497981425491
781 0.00606497981425491
782 0.00606497981425491
783 0.00606497981425491
784 0.00606497981425491
785 0.0060649798506347
786 0.00606497973785736
787 0.00606497973785736
788 0.00606497973785736
789 0.00606497973785736
790 0.00606497973785736
791 0.00606497973785736
792 0.00606497973785736
793 0.00606497973785736
794 0.00606497973785736
795 0.00606497973785736
796 0.00606497973785736
797 0.00606497973785736
798 0.00606497973785736
799 0.00606497973785736
800 0.00606497973785736
801 0.00606497973785736
802 0.00606497973785736
803 0.00606497973785736
804 0.00606497973785736
805 0.00606497973785736
806 0.00606497978878906
807 0.00606497979242704
808 0.00606497979242704
809 0.00606497979242704
810 0.00606497979242704
811 0.00606497979242704
812 0.00606497979242704
813 0.00606497979242704
814 0.00606497979242704
815 0.00606497979242704
816 0.00606497979242704
817 0.00606497979242704
818 0.00606497979242704
819 0.00606497979242704
820 0.00606497979242704
821 0.00606497979242704
822 0.00606497979242704
823 0.00606497979242704
824 0.00606497979242704
825 0.00606497979242704
826 0.00606497979242704
827 0.00606497979242704
828 0.00606497979242704
829 0.00606497979242704
830 0.00606497972694342
831 0.00606497973421938
832 0.00606497973421938
833 0.00606497973421938
834 0.00606497973421938
835 0.00606497973421938
836 0.00606497973421938
837 0.00606497973421938
838 0.00606497973421938
839 0.00606497973421938
840 0.00606497973421938
841 0.00606497973421938
842 0.00606497973421938
843 0.00606497973421938
844 0.00606497973421938
845 0.00606497973421938
846 0.00606497973421938
847 0.00606497973421938
848 0.00606497973421938
849 0.00606497965054587
850 0.0060649796614598
851 0.0060649796614598
852 0.0060649796614598
853 0.0060649796614598
854 0.0060649796614598
855 0.0060649796614598
856 0.0060649796614598
857 0.0060649796614598
858 0.0060649796614598
859 0.0060649796614598
860 0.0060649796614598
861 0.0060649796614598
862 0.0060649796614598
863 0.0060649796614598
864 0.0060649796614598
865 0.0060649796614598
866 0.0060649796614598
867 0.0060649796614598
868 0.0060649796614598
869 0.0060649796614598
870 0.0060649796614598
871 0.0060649796614598
872 0.0060649796614598
873 0.0060649796614598
874 0.0060649796614598
875 0.0060649796614598
876 0.0060649796614598
877 0.0060649796614598
878 0.0060649796614598
879 0.0060649796796497
880 0.00606497962508001
881 0.00606497962508001
882 0.00606497962508001
883 0.00606497962508001
884 0.00606497962508001
885 0.00606497962508001
886 0.00606497962508001
887 0.00606497962508001
888 0.00606497962508001
889 0.00606497962508001
890 0.00606497962508001
891 0.00606497962508001
892 0.00606497962508001
893 0.00606497962508001
894 0.00606497962508001
895 0.00606497962508001
896 0.00606497979242704
897 0.00606497979606502
898 0.00606497979606502
899 0.00606497979606502
900 0.00606497979606502
901 0.00606497979606502
902 0.00606497979606502
903 0.00606497979606502
904 0.00606497979606502
905 0.00606497979606502
906 0.00606497979606502
907 0.00606497979606502
908 0.00606497979606502
909 0.00606497979606502
910 0.00606497979606502
911 0.00606497979606502
912 0.00606497979606502
913 0.00606497979606502
914 0.00606497979606502
915 0.00606497979606502
916 0.00606497979606502
917 0.00606497979606502
918 0.00606497979606502
919 0.00606497979606502
920 0.00606497979606502
921 0.00606497979606502
922 0.00606497979606502
923 0.00606497979606502
924 0.00606497979606502
925 0.00606497979606502
926 0.00606497979606502
927 0.00606497979606502
928 0.00606497979606502
929 0.00606497979606502
930 0.00606497979606502
931 0.00606497979606502
932 0.00606497979606502
933 0.00606497979606502
934 0.00606497979606502
935 0.00606497979606502
936 0.00606497979606502
937 0.00606497979606502
938 0.00606497979606502
939 0.00606497979606502
940 0.00606497979606502
941 0.00606497979606502
942 0.00606497979606502
943 0.00606497979606502
944 0.00606497979606502
945 0.00606497979606502
946 0.00606497979606502
947 0.00606497979606502
948 0.00606497979606502
949 0.00606497979606502
950 0.00606497979606502
951 0.00606497979606502
952 0.00606497979606502
953 0.00606497979606502
954 0.00606497979606502
955 0.00606497979606502
956 0.00606497979606502
957 0.00606497979606502
958 0.00606497979606502
959 0.00606497979606502
960 0.00606497979606502
961 0.00606497979606502
962 0.00606497979606502
963 0.00606497979606502
964 0.00606497979606502
965 0.00606497979606502
966 0.00606497979606502
967 0.00606497979606502
968 0.00606497979606502
969 0.00606497979606502
970 0.00606497979606502
971 0.00606497979606502
972 0.00606497979606502
973 0.00606497979606502
974 0.00606497979606502
975 0.00606497979606502
976 0.00606497979606502
977 0.00606497979606502
978 0.00606497979606502
979 0.00606497979606502
980 0.00606497979606502
981 0.00606497979606502
982 0.00606497979606502
983 0.00606497979606502
984 0.00606497979606502
985 0.00606497979606502
986 0.00606497979606502
987 0.00606497979606502
988 0.00606497979606502
989 0.00606497979606502
990 0.00606497979606502
991 0.00606497979606502
992 0.00606497979606502
993 0.00606497979606502
994 0.00606497979606502
995 0.00606497979606502
996 0.00606497979606502
997 0.00606497979606502
998 0.00606497979606502
999 0.00606497979606502
1000 0.00606497979606502
1001 0.00606497979606502
1002 0.00606497979606502
1003 0.00606497979606502
1004 0.00606497979606502
1005 0.00606497979606502
1006 0.00606497979606502
1007 0.00606497979606502
1008 0.00606497979606502
1009 0.00606497979606502
1010 0.00606497979606502
1011 0.00606497979606502
1012 0.00606497979606502
1013 0.00606497979606502
1014 0.00606497979606502
1015 0.00606497979606502
1016 0.00606497979606502
1017 0.00606497979606502
1018 0.00606497979606502
1019 0.00606497979606502
1020 0.00606497979606502
1021 0.00606497979606502
1022 0.00606497979606502
1023 0.00606497979606502
1024 0.00606497979606502
1025 0.00606497979606502
1026 0.00606497979606502
1027 0.00606497979606502
1028 0.00606497979606502
1029 0.00606497979606502
1030 0.00606497979606502
1031 0.00606497979606502
1032 0.00606497979606502
1033 0.00606497979606502
1034 0.00606497979606502
1035 0.00606497979606502
1036 0.00606497979606502
1037 0.00606497979606502
1038 0.00606497979606502
1039 0.00606497979606502
1040 0.00606497979606502
1041 0.00606497979606502
1042 0.00606497979606502
1043 0.00606497979606502
1044 0.00606497979606502
1045 0.00606497979606502
1046 0.00606497979606502
1047 0.00606497979606502
1048 0.00606497979606502
1049 0.00606497979606502
1050 0.00606497979606502
1051 0.00606497979606502
1052 0.00606497979606502
1053 0.00606497979606502
1054 0.00606497979606502
1055 0.00606497979606502
1056 0.00606497979606502
1057 0.00606497979606502
1058 0.00606497979606502
1059 0.00606497979606502
1060 0.00606497979606502
1061 0.00606497979606502
1062 0.00606497979606502
1063 0.00606497979606502
1064 0.00606497979606502
1065 0.00606497979606502
1066 0.00606497979606502
1067 0.00606497979606502
1068 0.00606497979606502
1069 0.00606497979606502
1070 0.00606497979606502
1071 0.00606497979606502
1072 0.00606497979606502
1073 0.00606497979606502
1074 0.00606497979606502
1075 0.00606497979606502
1076 0.00606497979606502
1077 0.00606497979606502
1078 0.00606497979606502
1079 0.00606497979606502
1080 0.00606497979606502
1081 0.00606497979606502
1082 0.00606497979606502
1083 0.00606497979606502
1084 0.00606497979606502
1085 0.00606497979606502
1086 0.00606497979606502
1087 0.00606497979606502
1088 0.00606497979606502
1089 0.00606497979606502
1090 0.00606497979606502
1091 0.00606497979606502
1092 0.00606497979606502
1093 0.00606497979606502
1094 0.00606497979606502
1095 0.00606497979606502
1096 0.00606497979606502
1097 0.00606497979606502
1098 0.00606497979606502
1099 0.00606497979606502
1100 0.00606497979606502
1101 0.00606497979606502
1102 0.00606497979606502
1103 0.00606497979606502
1104 0.00606497979606502
1105 0.00606497979606502
1106 0.00606497979606502
1107 0.00606497979606502
1108 0.00606497979606502
1109 0.00606497979606502
1110 0.00606497979606502
1111 0.00606497979606502
1112 0.00606497979606502
1113 0.00606497979606502
1114 0.00606497979606502
1115 0.00606497979606502
1116 0.00606497979606502
1117 0.00606497979606502
1118 0.00606497979606502
1119 0.00606497979606502
1120 0.00606497979606502
1121 0.00606497979606502
1122 0.00606497979606502
1123 0.00606497979606502
1124 0.00606497979606502
1125 0.00606497979606502
1126 0.00606497979606502
1127 0.00606497979606502
1128 0.00606497979606502
1129 0.00606497979606502
1130 0.00606497979606502
1131 0.00606497979606502
1132 0.00606497979606502
1133 0.00606497979606502
1134 0.00606497979606502
1135 0.00606497979606502
1136 0.00606497979606502
1137 0.00606497979606502
1138 0.00606497979606502
1139 0.00606497979606502
1140 0.00606497979606502
1141 0.00606497979606502
1142 0.00606497979606502
1143 0.00606497979606502
1144 0.00606497979606502
1145 0.00606497979606502
1146 0.00606497979606502
1147 0.00606497979606502
1148 0.00606497979606502
1149 0.00606497979606502
1150 0.00606497979606502
1151 0.00606497979606502
1152 0.00606497979606502
1153 0.00606497979606502
1154 0.00606497979606502
1155 0.00606497979606502
1156 0.00606497979606502
1157 0.00606497979606502
1158 0.00606497979606502
1159 0.00606497979606502
1160 0.00606497979606502
1161 0.00606497979606502
1162 0.00606497979606502
1163 0.00606497979606502
1164 0.00606497979606502
1165 0.00606497979606502
1166 0.00606497979606502
1167 0.00606497979606502
1168 0.00606497979606502
1169 0.00606497979606502
1170 0.00606497979606502
1171 0.00606497979606502
1172 0.00606497979606502
1173 0.00606497979606502
1174 0.00606497979606502
1175 0.00606497979606502
1176 0.00606497979606502
1177 0.00606497979606502
1178 0.00606497979606502
1179 0.00606497979606502
1180 0.00606497979606502
1181 0.00606497979606502
1182 0.00606497979606502
1183 0.00606497979606502
1184 0.00606497979606502
1185 0.00606497979606502
1186 0.00606497979606502
1187 0.00606497979606502
1188 0.00606497979606502
1189 0.00606497979606502
1190 0.00606497979606502
1191 0.00606497979606502
1192 0.00606497979606502
1193 0.00606497979606502
1194 0.00606497979606502
1195 0.00606497979606502
1196 0.00606497979606502
1197 0.00606497979606502
1198 0.00606497979606502
1199 0.00606497979606502
1200 0.00606497979606502
1201 0.00606497979606502
1202 0.00606497979606502
1203 0.00606497979606502
1204 0.00606497979606502
1205 0.00606497979606502
1206 0.00606497979606502
1207 0.00606497979606502
1208 0.00606497979606502
1209 0.00606497979606502
1210 0.00606497979606502
1211 0.00606497979606502
1212 0.00606497979606502
1213 0.00606497979606502
1214 0.00606497979606502
1215 0.00606497979606502
1216 0.00606497979606502
1217 0.00606497979606502
1218 0.00606497979606502
1219 0.00606497979606502
1220 0.00606497979606502
1221 0.00606497979606502
1222 0.00606497979606502
1223 0.00606497979606502
1224 0.00606497979606502
1225 0.00606497979606502
1226 0.00606497979606502
1227 0.00606497979606502
1228 0.00606497979606502
1229 0.00606497979606502
1230 0.00606497979606502
1231 0.00606497979606502
1232 0.00606497979606502
1233 0.00606497979606502
1234 0.00606497979606502
1235 0.00606497979606502
1236 0.00606497979606502
1237 0.00606497979606502
1238 0.00606497979606502
1239 0.00606497979606502
1240 0.00606497979606502
1241 0.00606497979606502
1242 0.00606497979606502
1243 0.00606497979606502
1244 0.00606497979606502
1245 0.00606497979606502
1246 0.00606497979606502
1247 0.00606497979606502
1248 0.00606497979606502
1249 0.00606497979606502
1250 0.00606497979606502
1251 0.00606497979606502
1252 0.00606497979606502
1253 0.00606497979606502
1254 0.00606497979606502
1255 0.00606497979606502
1256 0.00606497979606502
1257 0.00606497979606502
1258 0.00606497979606502
1259 0.00606497979606502
1260 0.00606497979606502
1261 0.00606497979606502
1262 0.00606497979606502
1263 0.00606497979606502
1264 0.00606497979606502
1265 0.00606497979606502
1266 0.00606497979606502
1267 0.00606497979606502
1268 0.00606497979606502
1269 0.00606497979606502
1270 0.00606497979606502
1271 0.00606497979606502
1272 0.00606497979606502
1273 0.00606497979606502
1274 0.00606497979606502
1275 0.00606497979606502
1276 0.00606497979606502
1277 0.00606497979606502
1278 0.00606497979606502
1279 0.00606497979606502
1280 0.00606497979606502
1281 0.00606497979606502
1282 0.00606497979606502
1283 0.00606497979606502
1284 0.00606497979606502
1285 0.00606497979606502
1286 0.00606497979606502
1287 0.00606497979606502
1288 0.00606497979606502
1289 0.00606497979606502
1290 0.00606497979606502
1291 0.00606497979606502
1292 0.00606497979606502
1293 0.00606497979606502
1294 0.00606497979606502
1295 0.00606497979606502
1296 0.00606497979606502
1297 0.00606497979606502
1298 0.00606497979606502
1299 0.00606497979606502
1300 0.00606497979606502
1301 0.00606497979606502
1302 0.00606497979606502
1303 0.00606497979606502
1304 0.00606497979606502
1305 0.00606497979606502
1306 0.00606497979606502
1307 0.00606497979606502
1308 0.00606497979606502
1309 0.00606497979606502
1310 0.00606497979606502
1311 0.00606497979606502
1312 0.00606497979606502
1313 0.00606497979606502
1314 0.00606497979606502
1315 0.00606497979606502
1316 0.00606497979606502
1317 0.00606497979606502
1318 0.00606497979606502
1319 0.00606497979606502
1320 0.00606497979606502
1321 0.00606497979606502
1322 0.00606497979606502
1323 0.00606497979606502
1324 0.00606497979606502
1325 0.00606497979606502
1326 0.00606497979606502
1327 0.00606497979606502
1328 0.00606497979606502
1329 0.00606497979606502
1330 0.00606497979606502
1331 0.00606497979606502
1332 0.00606497979606502
1333 0.00606497979606502
1334 0.00606497979606502
1335 0.00606497979606502
1336 0.00606497979606502
1337 0.00606497979606502
1338 0.00606497979606502
1339 0.00606497979606502
1340 0.00606497979606502
1341 0.00606497979606502
1342 0.00606497979606502
1343 0.00606497979606502
1344 0.00606497979606502
1345 0.00606497979606502
1346 0.00606497979606502
1347 0.00606497979606502
1348 0.00606497979606502
1349 0.00606497979606502
1350 0.00606497979606502
1351 0.00606497979606502
1352 0.00606497979606502
1353 0.00606497979606502
1354 0.00606497979606502
1355 0.00606497979606502
1356 0.00606497979606502
1357 0.00606497979606502
1358 0.00606497979606502
1359 0.00606497979606502
1360 0.00606497979606502
1361 0.00606497979606502
1362 0.00606497979606502
1363 0.00606497979606502
1364 0.00606497979606502
1365 0.00606497979606502
1366 0.00606497979606502
1367 0.00606497979606502
1368 0.00606497979606502
1369 0.00606497979606502
1370 0.00606497979606502
1371 0.00606497979606502
1372 0.00606497979606502
1373 0.00606497979606502
1374 0.00606497979606502
1375 0.00606497979606502
1376 0.00606497979606502
1377 0.00606497979606502
1378 0.00606497979606502
1379 0.00606497979606502
1380 0.00606497979606502
1381 0.00606497979606502
1382 0.00606497979606502
1383 0.00606497979606502
1384 0.00606497979606502
1385 0.00606497979606502
1386 0.00606497979606502
1387 0.00606497979606502
1388 0.00606497979606502
1389 0.00606497979606502
1390 0.00606497979606502
1391 0.00606497979606502
1392 0.00606497979606502
1393 0.00606497979606502
1394 0.00606497979606502
1395 0.00606497979606502
1396 0.00606497979606502
1397 0.00606497979606502
1398 0.00606497979606502
1399 0.00606497979606502
1400 0.00606497979606502
1401 0.00606497979606502
1402 0.00606497979606502
1403 0.00606497979606502
1404 0.00606497979606502
1405 0.00606497979606502
1406 0.00606497979606502
1407 0.00606497979606502
1408 0.00606497979606502
1409 0.00606497979606502
1410 0.00606497979606502
1411 0.00606497979606502
1412 0.00606497979606502
1413 0.00606497979606502
1414 0.00606497979606502
1415 0.00606497979606502
1416 0.00606497979606502
1417 0.00606497979606502
1418 0.00606497979606502
1419 0.00606497979606502
1420 0.00606497979606502
1421 0.00606497979606502
1422 0.00606497979606502
1423 0.00606497979606502
1424 0.00606497979606502
1425 0.00606497979606502
1426 0.00606497979606502
1427 0.00606497979606502
1428 0.00606497979606502
1429 0.00606497979606502
1430 0.00606497979606502
1431 0.00606497979606502
1432 0.00606497979606502
1433 0.00606497979606502
1434 0.00606497979606502
1435 0.00606497979606502
1436 0.00606497979606502
1437 0.00606497979606502
1438 0.00606497979606502
1439 0.00606497979606502
1440 0.00606497979606502
1441 0.00606497979606502
1442 0.00606497979606502
1443 0.00606497979606502
1444 0.00606497979606502
1445 0.00606497979606502
1446 0.00606497979606502
1447 0.00606497979606502
1448 0.00606497979606502
1449 0.00606497979606502
1450 0.00606497979606502
1451 0.00606497979606502
1452 0.00606497979606502
1453 0.00606497979606502
1454 0.00606497979606502
1455 0.00606497979606502
1456 0.00606497979606502
1457 0.00606497979606502
1458 0.00606497979606502
1459 0.00606497979606502
1460 0.00606497979606502
1461 0.00606497979606502
1462 0.00606497979606502
1463 0.00606497979606502
1464 0.00606497979606502
1465 0.00606497979606502
1466 0.00606497979606502
1467 0.00606497979606502
1468 0.00606497979606502
1469 0.00606497979606502
1470 0.00606497979606502
1471 0.00606497979606502
1472 0.00606497979606502
1473 0.00606497979606502
1474 0.00606497979606502
1475 0.00606497979606502
1476 0.00606497979606502
1477 0.00606497979606502
1478 0.00606497979606502
1479 0.00606497979606502
1480 0.00606497979606502
1481 0.00606497979606502
1482 0.00606497979606502
1483 0.00606497979606502
1484 0.00606497979606502
1485 0.00606497979606502
1486 0.00606497979606502
1487 0.00606497979606502
1488 0.00606497979606502
1489 0.00606497979606502
1490 0.00606497979606502
1491 0.00606497979606502
1492 0.00606497979606502
1493 0.00606497979606502
1494 0.00606497979606502
1495 0.00606497979606502
1496 0.00606497979606502
1497 0.00606497979606502
1498 0.00606497979606502
1499 0.00606497979606502
1500 0.00606497979606502
1501 0.00606497979606502
1502 0.00606497979606502
1503 0.00606497979606502
1504 0.00606497979606502
1505 0.00606497979606502
1506 0.00606497979606502
1507 0.00606497979606502
1508 0.00606497979606502
1509 0.00606497979606502
1510 0.00606497979606502
1511 0.00606497979606502
1512 0.00606497979606502
1513 0.00606497979606502
1514 0.00606497979606502
1515 0.00606497979606502
1516 0.00606497979606502
1517 0.00606497979606502
1518 0.00606497979606502
1519 0.00606497979606502
1520 0.00606497979606502
1521 0.00606497979606502
1522 0.00606497979606502
1523 0.00606497979606502
1524 0.00606497979606502
1525 0.00606497979606502
1526 0.00606497979606502
1527 0.00606497979606502
1528 0.00606497979606502
1529 0.00606497979606502
1530 0.00606497979606502
1531 0.00606497979606502
1532 0.00606497979606502
1533 0.00606497979606502
1534 0.00606497979606502
1535 0.00606497979606502
1536 0.00606497979606502
1537 0.00606497979606502
1538 0.00606497979606502
1539 0.00606497979606502
1540 0.00606497979606502
1541 0.00606497979606502
1542 0.00606497979606502
1543 0.00606497979606502
1544 0.00606497979606502
1545 0.00606497979606502
1546 0.00606497979606502
1547 0.00606497979606502
1548 0.00606497979606502
1549 0.00606497979606502
1550 0.00606497979606502
1551 0.00606497979606502
1552 0.00606497979606502
1553 0.00606497979606502
1554 0.00606497979606502
1555 0.00606497979606502
1556 0.00606497979606502
1557 0.00606497979606502
1558 0.00606497979606502
1559 0.00606497979606502
1560 0.00606497979606502
1561 0.00606497979606502
1562 0.00606497979606502
1563 0.00606497979606502
1564 0.00606497979606502
1565 0.00606497979606502
1566 0.00606497979606502
1567 0.00606497979606502
1568 0.00606497979606502
1569 0.00606497979606502
1570 0.00606497979606502
1571 0.00606497979606502
1572 0.00606497979606502
1573 0.00606497979606502
1574 0.00606497979606502
1575 0.00606497979606502
1576 0.00606497979606502
1577 0.00606497979606502
1578 0.00606497979606502
1579 0.00606497979606502
1580 0.00606497979606502
1581 0.00606497979606502
1582 0.00606497979606502
1583 0.00606497979606502
1584 0.00606497979606502
1585 0.00606497979606502
1586 0.00606497979606502
1587 0.00606497979606502
1588 0.00606497979606502
1589 0.00606497979606502
1590 0.00606497979606502
1591 0.00606497979606502
1592 0.00606497979606502
1593 0.00606497979606502
1594 0.00606497979606502
1595 0.00606497979606502
1596 0.00606497979606502
1597 0.00606497979606502
1598 0.00606497979606502
1599 0.00606497979606502
1600 0.00606497979606502
1601 0.00606497979606502
1602 0.00606497979606502
1603 0.00606497979606502
1604 0.00606497979606502
1605 0.00606497979606502
1606 0.00606497979606502
1607 0.00606497979606502
1608 0.00606497979606502
1609 0.00606497979606502
1610 0.00606497979606502
1611 0.00606497979606502
1612 0.00606497979606502
1613 0.00606497979606502
1614 0.00606497979606502
1615 0.00606497979606502
1616 0.00606497979606502
1617 0.00606497979606502
1618 0.00606497979606502
1619 0.00606497979606502
1620 0.00606497979606502
1621 0.00606497979606502
1622 0.00606497979606502
1623 0.00606497979606502
1624 0.00606497979606502
1625 0.00606497979606502
1626 0.00606497979606502
1627 0.00606497979606502
1628 0.00606497979606502
1629 0.00606497979606502
1630 0.00606497979606502
1631 0.00606497979606502
1632 0.00606497979606502
1633 0.00606497979606502
1634 0.00606497979606502
1635 0.00606497979606502
1636 0.00606497979606502
1637 0.00606497979606502
1638 0.00606497979606502
1639 0.00606497979606502
1640 0.00606497979606502
1641 0.00606497979606502
1642 0.00606497979606502
1643 0.00606497979606502
1644 0.00606497979606502
1645 0.00606497979606502
1646 0.00606497979606502
1647 0.00606497979606502
1648 0.00606497979606502
1649 0.00606497979606502
1650 0.00606497979606502
1651 0.00606497979606502
1652 0.00606497979606502
1653 0.00606497979606502
1654 0.00606497979606502
1655 0.00606497979606502
1656 0.00606497979606502
1657 0.00606497979606502
1658 0.00606497979606502
1659 0.00606497979606502
1660 0.00606497979606502
1661 0.00606497979606502
1662 0.00606497979606502
1663 0.00606497979606502
1664 0.00606497979606502
1665 0.00606497979606502
1666 0.00606497979606502
1667 0.00606497979606502
1668 0.00606497979606502
1669 0.00606497979606502
1670 0.00606497979606502
1671 0.00606497979606502
1672 0.00606497979606502
1673 0.00606497979606502
1674 0.00606497979606502
1675 0.00606497979606502
1676 0.00606497979606502
1677 0.00606497979606502
1678 0.00606497979606502
1679 0.00606497979606502
1680 0.00606497979606502
1681 0.00606497979606502
1682 0.00606497979606502
1683 0.00606497979606502
1684 0.00606497979606502
1685 0.00606497979606502
1686 0.00606497979606502
1687 0.00606497979606502
1688 0.00606497979606502
1689 0.00606497979606502
1690 0.00606497979606502
1691 0.00606497979606502
1692 0.00606497979606502
1693 0.00606497979606502
1694 0.00606497979606502
1695 0.00606497979606502
1696 0.00606497979606502
1697 0.00606497979606502
1698 0.00606497979606502
1699 0.00606497979606502
1700 0.00606497979606502
1701 0.00606497979606502
1702 0.00606497979606502
1703 0.00606497979606502
1704 0.00606497979606502
1705 0.00606497979606502
1706 0.00606497979606502
1707 0.00606497979606502
1708 0.00606497979606502
1709 0.00606497979606502
1710 0.00606497979606502
1711 0.00606497979606502
1712 0.00606497979606502
1713 0.00606497979606502
1714 0.00606497979606502
1715 0.00606497979606502
1716 0.00606497979606502
1717 0.00606497979606502
1718 0.00606497979606502
1719 0.00606497979606502
1720 0.00606497979606502
1721 0.00606497979606502
1722 0.00606497979606502
1723 0.00606497979606502
1724 0.00606497979606502
1725 0.00606497979606502
1726 0.00606497979606502
1727 0.00606497979606502
1728 0.00606497979606502
1729 0.00606497979606502
1730 0.00606497979606502
1731 0.00606497979606502
1732 0.00606497979606502
1733 0.00606497979606502
1734 0.00606497979606502
1735 0.00606497979606502
1736 0.00606497979606502
1737 0.00606497979606502
1738 0.00606497979606502
1739 0.00606497979606502
1740 0.00606497979606502
1741 0.00606497979606502
1742 0.00606497979606502
1743 0.00606497979606502
1744 0.00606497979606502
1745 0.00606497979606502
1746 0.00606497979606502
1747 0.00606497979606502
1748 0.00606497979606502
1749 0.00606497979606502
1750 0.00606497979606502
1751 0.00606497979606502
1752 0.00606497979606502
1753 0.00606497979606502
1754 0.00606497979606502
1755 0.00606497979606502
1756 0.00606497979606502
1757 0.00606497979606502
1758 0.00606497979606502
1759 0.00606497979606502
1760 0.00606497979606502
1761 0.00606497979606502
1762 0.00606497979606502
1763 0.00606497979606502
1764 0.00606497979606502
1765 0.00606497979606502
1766 0.00606497979606502
1767 0.00606497979606502
1768 0.00606497979606502
1769 0.00606497979606502
1770 0.00606497979606502
1771 0.00606497979606502
1772 0.00606497979606502
1773 0.00606497979606502
1774 0.00606497979606502
1775 0.00606497979606502
1776 0.00606497979606502
1777 0.00606497979606502
1778 0.00606497979606502
1779 0.00606497979606502
1780 0.00606497979606502
1781 0.00606497979606502
1782 0.00606497979606502
1783 0.00606497979606502
1784 0.00606497979606502
1785 0.00606497979606502
1786 0.00606497979606502
1787 0.00606497979606502
1788 0.00606497979606502
1789 0.00606497979606502
1790 0.00606497979606502
1791 0.00606497979606502
1792 0.00606497979606502
1793 0.00606497979606502
1794 0.00606497979606502
1795 0.00606497979606502
1796 0.00606497979606502
1797 0.00606497979606502
1798 0.00606497979606502
1799 0.00606497979606502
1800 0.00606497979606502
1801 0.00606497979606502
1802 0.00606497979606502
1803 0.00606497979606502
1804 0.00606497979606502
1805 0.00606497979606502
1806 0.00606497979606502
1807 0.00606497979606502
1808 0.00606497979606502
1809 0.00606497979606502
1810 0.00606497979606502
1811 0.00606497979606502
1812 0.00606497979606502
1813 0.00606497979606502
1814 0.00606497979606502
1815 0.00606497979606502
1816 0.00606497979606502
1817 0.00606497979606502
1818 0.00606497979606502
1819 0.00606497979606502
1820 0.00606497979606502
1821 0.00606497979606502
1822 0.00606497979606502
1823 0.00606497979606502
1824 0.00606497979606502
1825 0.00606497979606502
1826 0.00606497979606502
1827 0.00606497979606502
1828 0.00606497979606502
1829 0.00606497979606502
1830 0.00606497979606502
1831 0.00606497979606502
1832 0.00606497979606502
1833 0.00606497979606502
1834 0.00606497979606502
1835 0.00606497979606502
1836 0.00606497979606502
1837 0.00606497979606502
1838 0.00606497979606502
1839 0.00606497979606502
1840 0.00606497979606502
1841 0.00606497979606502
1842 0.00606497979606502
1843 0.00606497979606502
1844 0.00606497979606502
1845 0.00606497979606502
1846 0.00606497979606502
1847 0.00606497979606502
1848 0.00606497979606502
1849 0.00606497979606502
1850 0.00606497979606502
1851 0.00606497979606502
1852 0.00606497979606502
1853 0.00606497979606502
1854 0.00606497979606502
1855 0.00606497979606502
1856 0.00606497979606502
1857 0.00606497979606502
1858 0.00606497979606502
1859 0.00606497979606502
1860 0.00606497979606502
1861 0.00606497979606502
1862 0.00606497979606502
1863 0.00606497979606502
1864 0.00606497979606502
1865 0.00606497979606502
1866 0.00606497979606502
1867 0.00606497979606502
1868 0.00606497979606502
1869 0.00606497979606502
1870 0.00606497979606502
1871 0.00606497979606502
1872 0.00606497979606502
1873 0.00606497979606502
1874 0.00606497979606502
1875 0.00606497979606502
1876 0.00606497979606502
1877 0.00606497979606502
1878 0.00606497979606502
1879 0.00606497979606502
1880 0.00606497979606502
1881 0.00606497979606502
1882 0.00606497979606502
1883 0.00606497979606502
1884 0.00606497979606502
1885 0.00606497979606502
1886 0.00606497979606502
1887 0.00606497979606502
1888 0.00606497979606502
1889 0.00606497979606502
1890 0.00606497979606502
1891 0.00606497979606502
1892 0.00606497979606502
1893 0.00606497979606502
1894 0.00606497979606502
1895 0.00606497979606502
1896 0.00606497979606502
1897 0.00606497979606502
1898 0.00606497979606502
1899 0.00606497979606502
1900 0.00606497979606502
1901 0.00606497979606502
1902 0.00606497979606502
1903 0.00606497979606502
1904 0.00606497979606502
1905 0.00606497979606502
1906 0.00606497979606502
1907 0.00606497979606502
1908 0.00606497979606502
1909 0.00606497979606502
1910 0.00606497979606502
1911 0.00606497979606502
1912 0.00606497979606502
1913 0.00606497979606502
1914 0.00606497979606502
1915 0.00606497979606502
1916 0.00606497979606502
1917 0.00606497979606502
1918 0.00606497979606502
1919 0.00606497979606502
1920 0.00606497979606502
1921 0.00606497979606502
1922 0.00606497979606502
1923 0.00606497979606502
1924 0.00606497979606502
1925 0.00606497979606502
1926 0.00606497979606502
1927 0.00606497979606502
1928 0.00606497979606502
1929 0.00606497979606502
1930 0.00606497979606502
1931 0.00606497979606502
1932 0.00606497979606502
1933 0.00606497979606502
1934 0.00606497979606502
1935 0.00606497979606502
1936 0.00606497979606502
1937 0.00606497979606502
1938 0.00606497979606502
1939 0.00606497979606502
1940 0.00606497979606502
1941 0.00606497979606502
1942 0.00606497979606502
1943 0.00606497979606502
1944 0.00606497979606502
1945 0.00606497979606502
1946 0.00606497979606502
1947 0.00606497979606502
1948 0.00606497979606502
1949 0.00606497979606502
1950 0.00606497979606502
1951 0.00606497979606502
1952 0.00606497979606502
1953 0.00606497979606502
1954 0.00606497979606502
1955 0.00606497979606502
1956 0.00606497979606502
1957 0.00606497979606502
1958 0.00606497979606502
1959 0.00606497979606502
1960 0.00606497979606502
1961 0.00606497979606502
1962 0.00606497979606502
1963 0.00606497979606502
1964 0.00606497979606502
1965 0.00606497979606502
1966 0.00606497979606502
1967 0.00606497979606502
1968 0.00606497979606502
1969 0.00606497979606502
1970 0.00606497979606502
1971 0.00606497979606502
1972 0.00606497979606502
1973 0.00606497979606502
1974 0.00606497979606502
1975 0.00606497979606502
1976 0.00606497979606502
1977 0.00606497979606502
1978 0.00606497979606502
1979 0.00606497979606502
1980 0.00606497979606502
1981 0.00606497979606502
1982 0.00606497979606502
1983 0.00606497979606502
1984 0.00606497979606502
1985 0.00606497979606502
1986 0.00606497979606502
1987 0.00606497979606502
1988 0.00606497979606502
1989 0.00606497979606502
1990 0.00606497979606502
1991 0.00606497979606502
1992 0.00606497979606502
1993 0.00606497979606502
1994 0.00606497979606502
1995 0.00606497979606502
1996 0.00606497979606502
1997 0.00606497979606502
1998 0.00606497979606502
1999 0.00606497979606502
};
\addlegendentry{Train}
\addplot [semithick, black]
table {%
0 0.012701210565865
1 0.0124782240018249
2 0.0122608626261353
3 0.012049613520503
4 0.0118446452543139
5 0.0116459419950843
6 0.0114534450694919
7 0.0112670594826341
8 0.0110866772010922
9 0.0109121883288026
10 0.0107434624806046
11 0.0105803934857249
12 0.0104228500276804
13 0.0102707138285041
14 0.0101238526403904
15 0.0099821574985981
16 0.0098454998806119
17 0.00971375592052937
18 0.00958680920302868
19 0.00946453586220741
20 0.00934682227671146
21 0.00923354923725128
22 0.00912459660321474
23 0.00901985168457031
24 0.00891919527202845
25 0.00882252398878336
26 0.0087297186255455
27 0.00864067208021879
28 0.00855526980012655
29 0.00847341306507587
30 0.0083949901163578
31 0.00831989757716656
32 0.00824803113937378
33 0.00817929394543171
34 0.00811357982456684
35 0.00805079471319914
36 0.00799084454774857
37 0.00793362688273191
38 0.00787905603647232
39 0.00782702956348658
40 0.00777747109532356
41 0.00773028610274196
42 0.00768538936972618
43 0.00764269707724452
44 0.00760212913155556
45 0.00756359519436955
46 0.00752702914178371
47 0.00749234296381474
48 0.00745946960523725
49 0.00742832757532597
50 0.00739885633811355
51 0.00737097905948758
52 0.0073446286842227
53 0.00731973955407739
54 0.00729624787345529
55 0.00727409496903419
56 0.00725321844220161
57 0.00723355636000633
58 0.00721505889669061
59 0.00719766365364194
60 0.00718132685869932
61 0.00716598890721798
62 0.00715160369873047
63 0.00713812559843063
64 0.00712550571188331
65 0.00711370212957263
66 0.00710267527028918
67 0.00709237484261394
68 0.00708276964724064
69 0.00707381730899215
70 0.00706548662856221
71 0.00705773755908012
72 0.00705054076388478
73 0.00704386178404093
74 0.00703767454251647
75 0.00703194411471486
76 0.00702665001153946
77 0.00702175963670015
78 0.00701724970713258
79 0.00701309507712722
80 0.00700927805155516
81 0.00700577395036817
82 0.00700256088748574
83 0.00699961977079511
84 0.00699693569913507
85 0.00699448538944125
86 0.00699225906282663
87 0.00699023716151714
88 0.00698840478435159
89 0.00698675215244293
90 0.00698525970801711
91 0.00698392186313868
92 0.00698272325098515
93 0.00698165548965335
94 0.00698070647194982
95 0.00697986455634236
96 0.00697912601754069
97 0.0069784801453352
98 0.00697791948914528
99 0.00697743659839034
100 0.00697702635079622
101 0.0069766785018146
102 0.00697638746351004
103 0.00697615509852767
104 0.00697596790269017
105 0.00697582401335239
106 0.00697572017088532
107 0.0069756512530148
108 0.00697561632841825
109 0.00697560701519251
110 0.00697562377899885
111 0.00697566289454699
112 0.00697572296485305
113 0.00697579933330417
114 0.00697588780894876
115 0.0069759925827384
116 0.00697610666975379
117 0.00697623286396265
118 0.00697636464610696
119 0.00697650294750929
120 0.00697664776816964
121 0.00697679584845901
122 0.00697694532573223
123 0.00697709852829576
124 0.00697725266218185
125 0.00697740912437439
126 0.00697756325826049
127 0.00697771972045302
128 0.00697787245735526
129 0.00697802193462849
130 0.00697817374020815
131 0.00697832088917494
132 0.00697846338152885
133 0.00697860680520535
134 0.00697874277830124
135 0.00697887875139713
136 0.00697900960221887
137 0.00697913812473416
138 0.00697926385328174
139 0.00697938585653901
140 0.00697950413450599
141 0.00697961542755365
142 0.00697972858324647
143 0.00697983521968126
144 0.00697993440553546
145 0.00698003685101867
146 0.00698012998327613
147 0.00698022311553359
148 0.00698031159117818
149 0.00698039587587118
150 0.00698048016056418
151 0.0069805565290153
152 0.00698063336312771
153 0.00698070367798209
154 0.00698077585548162
155 0.00698084151372313
156 0.00698090577498078
157 0.00698096444830298
158 0.00698102312162519
159 0.00698107806965709
160 0.00698113394901156
161 0.00698118424043059
162 0.00698123313486576
163 0.00698127970099449
164 0.00698132207617164
165 0.00698136445134878
166 0.0069814040325582
167 0.0069814445450902
168 0.00698148040100932
169 0.00698151206597686
170 0.00698154792189598
171 0.00698157865554094
172 0.00698160845786333
173 0.00698163686320186
174 0.00698166573420167
175 0.00698168994858861
176 0.0069817160256207
177 0.00698173651471734
178 0.00698176119476557
179 0.0069817821495235
180 0.00698180077597499
181 0.00698181893676519
182 0.00698183802887797
183 0.00698185386136174
184 0.00698187062516809
185 0.00698188599199057
186 0.00698190229013562
187 0.00698191672563553
188 0.006981928832829
189 0.00698193954303861
190 0.00698195304721594
191 0.00698196375742555
192 0.00698197446763515
193 0.00698198284953833
194 0.00698199449107051
195 0.00698200147598982
196 0.00698201125487685
197 0.00698201870545745
198 0.00698202662169933
199 0.00698203407227993
200 0.00698204012587667
201 0.00698204804211855
202 0.00698205409571528
203 0.00698205875232816
204 0.00698206387460232
205 0.00698206899687648
206 0.0069820755161345
207 0.00698207831010222
208 0.0069820829667151
209 0.00698208808898926
210 0.00698209227994084
211 0.00698209507390857
212 0.00698209973052144
213 0.00698210299015045
214 0.00698210578411818
215 0.00698210811242461
216 0.00698211044073105
217 0.00698211370036006
218 0.00698211649432778
219 0.00698211789131165
220 0.00698211882263422
221 0.00698212115094066
222 0.00698212347924709
223 0.00698212720453739
224 0.00698212580755353
225 0.0069821304641664
226 0.00698212860152125
227 0.00698213325813413
228 0.00698213186115026
229 0.00698213558644056
230 0.00698213558644056
231 0.006982137914747
232 0.00698213698342443
233 0.006982137914747
234 0.00698214070871472
235 0.00698214070871472
236 0.00698214024305344
237 0.00698214303702116
238 0.00698214303702116
239 0.00698214303702116
240 0.00698214303702116
241 0.00698214629665017
242 0.00698214629665017
243 0.00698214629665017
244 0.00698214629665017
245 0.0069821453653276
246 0.00698214955627918
247 0.00698214955627918
248 0.00698214955627918
249 0.00698214769363403
250 0.00698214862495661
251 0.00698214769363403
252 0.00698214955627918
253 0.00698214862495661
254 0.00698215235024691
255 0.00698215095326304
256 0.00698215095326304
257 0.00698215141892433
258 0.00698215235024691
259 0.00698215235024691
260 0.00698215141892433
261 0.00698215095326304
262 0.00698215095326304
263 0.00698215141892433
264 0.00698215141892433
265 0.00698215095326304
266 0.00698215002194047
267 0.00698215235024691
268 0.00698215141892433
269 0.00698215235024691
270 0.00698215141892433
271 0.00698215235024691
272 0.00698215235024691
273 0.00698215141892433
274 0.00698215235024691
275 0.00698215235024691
276 0.00698215095326304
277 0.00698215141892433
278 0.00698215141892433
279 0.00698215141892433
280 0.00698215141892433
281 0.00698215374723077
282 0.00698215374723077
283 0.00698215374723077
284 0.00698215374723077
285 0.00698215467855334
286 0.00698215467855334
287 0.00698215467855334
288 0.00698215467855334
289 0.00698215467855334
290 0.00698215467855334
291 0.00698215467855334
292 0.00698215467855334
293 0.00698215467855334
294 0.00698215467855334
295 0.00698215374723077
296 0.00698215374723077
297 0.00698215374723077
298 0.00698215374723077
299 0.00698215374723077
300 0.00698215374723077
301 0.00698215141892433
302 0.00698215141892433
303 0.00698215141892433
304 0.00698215141892433
305 0.00698215141892433
306 0.00698215141892433
307 0.00698215141892433
308 0.00698215095326304
309 0.00698215095326304
310 0.00698215235024691
311 0.00698215235024691
312 0.00698215235024691
313 0.00698215235024691
314 0.00698215235024691
315 0.00698215141892433
316 0.00698215141892433
317 0.00698215141892433
318 0.00698215141892433
319 0.00698215235024691
320 0.00698215235024691
321 0.00698215235024691
322 0.00698215235024691
323 0.00698215141892433
324 0.00698215141892433
325 0.00698215235024691
326 0.00698215235024691
327 0.00698215235024691
328 0.00698215235024691
329 0.00698215235024691
330 0.00698215141892433
331 0.00698215141892433
332 0.00698215141892433
333 0.00698215141892433
334 0.00698215328156948
335 0.00698215235024691
336 0.00698215235024691
337 0.00698215095326304
338 0.00698215095326304
339 0.00698215095326304
340 0.00698215095326304
341 0.00698215095326304
342 0.00698215095326304
343 0.00698215095326304
344 0.00698215095326304
345 0.00698215095326304
346 0.00698215095326304
347 0.00698215095326304
348 0.00698215095326304
349 0.00698215095326304
350 0.00698215095326304
351 0.00698215141892433
352 0.00698215141892433
353 0.00698215141892433
354 0.00698215141892433
355 0.00698215141892433
356 0.00698215095326304
357 0.00698215095326304
358 0.00698215095326304
359 0.00698215095326304
360 0.00698215095326304
361 0.00698215095326304
362 0.00698215141892433
363 0.00698215141892433
364 0.00698215141892433
365 0.00698215141892433
366 0.00698215141892433
367 0.00698215141892433
368 0.00698215095326304
369 0.00698215095326304
370 0.00698215095326304
371 0.00698215095326304
372 0.00698215095326304
373 0.00698215095326304
374 0.00698215095326304
375 0.00698215095326304
376 0.00698215141892433
377 0.00698215235024691
378 0.00698215235024691
379 0.00698215235024691
380 0.00698215235024691
381 0.00698215235024691
382 0.00698215235024691
383 0.00698215235024691
384 0.00698215235024691
385 0.00698215235024691
386 0.00698215141892433
387 0.00698215141892433
388 0.00698215095326304
389 0.00698215095326304
390 0.00698215095326304
391 0.00698215095326304
392 0.00698215095326304
393 0.00698215095326304
394 0.00698215095326304
395 0.00698215095326304
396 0.00698215235024691
397 0.00698215235024691
398 0.00698215235024691
399 0.00698215235024691
400 0.00698215235024691
401 0.00698215235024691
402 0.00698215235024691
403 0.00698215235024691
404 0.00698215141892433
405 0.00698215141892433
406 0.00698215235024691
407 0.00698215141892433
408 0.00698215141892433
409 0.00698215141892433
410 0.00698215141892433
411 0.00698215095326304
412 0.00698215235024691
413 0.00698215235024691
414 0.00698215235024691
415 0.00698215235024691
416 0.00698215235024691
417 0.00698215235024691
418 0.00698215235024691
419 0.00698215235024691
420 0.00698215235024691
421 0.00698215235024691
422 0.00698215235024691
423 0.00698215235024691
424 0.00698215235024691
425 0.00698215235024691
426 0.00698215235024691
427 0.00698215235024691
428 0.00698215235024691
429 0.00698215235024691
430 0.00698215235024691
431 0.00698215235024691
432 0.00698215235024691
433 0.00698215235024691
434 0.00698215235024691
435 0.00698215235024691
436 0.00698215235024691
437 0.00698215235024691
438 0.00698215235024691
439 0.00698215235024691
440 0.00698215235024691
441 0.00698215235024691
442 0.00698215235024691
443 0.00698215235024691
444 0.00698215235024691
445 0.00698215235024691
446 0.00698215235024691
447 0.00698215235024691
448 0.00698215235024691
449 0.00698215235024691
450 0.00698215235024691
451 0.00698215141892433
452 0.00698215141892433
453 0.00698215141892433
454 0.00698215141892433
455 0.00698215141892433
456 0.00698215235024691
457 0.00698215235024691
458 0.00698215235024691
459 0.00698215235024691
460 0.00698215235024691
461 0.00698215235024691
462 0.00698215235024691
463 0.00698215235024691
464 0.00698215235024691
465 0.00698215235024691
466 0.00698215235024691
467 0.00698215235024691
468 0.00698215235024691
469 0.00698215235024691
470 0.00698215235024691
471 0.00698215235024691
472 0.00698215235024691
473 0.00698215235024691
474 0.00698215235024691
475 0.00698215235024691
476 0.00698215235024691
477 0.00698215235024691
478 0.00698215235024691
479 0.00698215235024691
480 0.00698215235024691
481 0.00698215235024691
482 0.00698215235024691
483 0.00698215235024691
484 0.00698215235024691
485 0.00698215235024691
486 0.00698215235024691
487 0.00698215235024691
488 0.00698215235024691
489 0.00698215235024691
490 0.00698215235024691
491 0.00698215235024691
492 0.00698215235024691
493 0.00698215235024691
494 0.00698215235024691
495 0.00698215235024691
496 0.00698215235024691
497 0.00698215235024691
498 0.00698215235024691
499 0.00698215235024691
500 0.00698215235024691
501 0.00698215235024691
502 0.00698215235024691
503 0.00698215235024691
504 0.00698215235024691
505 0.00698215235024691
506 0.00698215235024691
507 0.00698215235024691
508 0.00698215235024691
509 0.00698215235024691
510 0.00698215235024691
511 0.00698215235024691
512 0.00698215235024691
513 0.00698215235024691
514 0.00698215235024691
515 0.00698215235024691
516 0.00698215235024691
517 0.00698215235024691
518 0.00698215235024691
519 0.00698215235024691
520 0.00698215235024691
521 0.00698215235024691
522 0.00698215235024691
523 0.00698215235024691
524 0.00698215235024691
525 0.00698215235024691
526 0.00698215235024691
527 0.00698215235024691
528 0.00698215235024691
529 0.00698215235024691
530 0.00698215235024691
531 0.00698215235024691
532 0.00698215235024691
533 0.00698215235024691
534 0.00698215235024691
535 0.00698215235024691
536 0.00698215235024691
537 0.00698215235024691
538 0.00698215235024691
539 0.00698215235024691
540 0.00698215235024691
541 0.00698215235024691
542 0.00698215235024691
543 0.00698215235024691
544 0.00698215235024691
545 0.00698215235024691
546 0.00698215235024691
547 0.00698215235024691
548 0.00698215235024691
549 0.00698215235024691
550 0.00698215235024691
551 0.00698215235024691
552 0.00698215235024691
553 0.00698215235024691
554 0.00698215235024691
555 0.00698215235024691
556 0.00698215235024691
557 0.00698215235024691
558 0.00698215235024691
559 0.00698215235024691
560 0.00698215235024691
561 0.00698215235024691
562 0.00698215235024691
563 0.00698215235024691
564 0.00698215235024691
565 0.00698215235024691
566 0.00698215235024691
567 0.00698215235024691
568 0.00698215235024691
569 0.00698215235024691
570 0.00698215235024691
571 0.00698215235024691
572 0.00698215235024691
573 0.00698215235024691
574 0.00698215235024691
575 0.00698215235024691
576 0.00698215235024691
577 0.00698215235024691
578 0.00698215235024691
579 0.00698215141892433
580 0.00698215141892433
581 0.00698215141892433
582 0.00698215141892433
583 0.00698215235024691
584 0.00698215235024691
585 0.00698215235024691
586 0.00698215235024691
587 0.00698215235024691
588 0.00698215235024691
589 0.00698215235024691
590 0.00698215235024691
591 0.00698215235024691
592 0.00698215235024691
593 0.00698215235024691
594 0.00698215235024691
595 0.00698215235024691
596 0.00698215235024691
597 0.00698215235024691
598 0.00698215235024691
599 0.00698215235024691
600 0.00698215235024691
601 0.00698215235024691
602 0.00698215235024691
603 0.00698215235024691
604 0.00698215235024691
605 0.00698215235024691
606 0.00698215235024691
607 0.00698215235024691
608 0.00698215235024691
609 0.00698215235024691
610 0.00698215235024691
611 0.00698215235024691
612 0.00698215235024691
613 0.00698215235024691
614 0.00698215235024691
615 0.00698215235024691
616 0.00698215235024691
617 0.00698215235024691
618 0.00698215235024691
619 0.00698215235024691
620 0.00698215235024691
621 0.00698215235024691
622 0.00698215095326304
623 0.00698215141892433
624 0.00698215141892433
625 0.00698215141892433
626 0.00698215235024691
627 0.00698215235024691
628 0.00698215235024691
629 0.00698215235024691
630 0.00698215235024691
631 0.00698215235024691
632 0.00698215235024691
633 0.00698215235024691
634 0.00698215235024691
635 0.00698215235024691
636 0.00698215235024691
637 0.00698215235024691
638 0.00698215235024691
639 0.00698215235024691
640 0.00698215235024691
641 0.00698215235024691
642 0.00698215235024691
643 0.00698215235024691
644 0.00698215235024691
645 0.00698215235024691
646 0.00698215235024691
647 0.00698215235024691
648 0.00698215235024691
649 0.00698215235024691
650 0.00698215235024691
651 0.00698215235024691
652 0.00698215235024691
653 0.00698215235024691
654 0.00698215235024691
655 0.00698215141892433
656 0.00698215141892433
657 0.00698215141892433
658 0.00698215141892433
659 0.00698215141892433
660 0.00698215095326304
661 0.00698215235024691
662 0.00698215235024691
663 0.00698215235024691
664 0.00698215235024691
665 0.00698215235024691
666 0.00698215235024691
667 0.00698215235024691
668 0.00698215235024691
669 0.00698215235024691
670 0.00698215235024691
671 0.00698215235024691
672 0.00698215235024691
673 0.00698215235024691
674 0.00698215235024691
675 0.00698215235024691
676 0.00698215235024691
677 0.00698215235024691
678 0.00698215235024691
679 0.00698215235024691
680 0.00698215235024691
681 0.00698215235024691
682 0.00698215235024691
683 0.00698215235024691
684 0.00698215235024691
685 0.00698215235024691
686 0.00698215235024691
687 0.00698215235024691
688 0.00698215235024691
689 0.00698215235024691
690 0.00698215235024691
691 0.00698215235024691
692 0.00698215235024691
693 0.00698215235024691
694 0.00698215235024691
695 0.00698215235024691
696 0.00698215235024691
697 0.00698215235024691
698 0.00698215235024691
699 0.00698215235024691
700 0.00698215235024691
701 0.00698215235024691
702 0.00698215235024691
703 0.00698215235024691
704 0.00698215235024691
705 0.00698215235024691
706 0.00698215235024691
707 0.00698215235024691
708 0.00698215235024691
709 0.00698215235024691
710 0.00698215235024691
711 0.00698215235024691
712 0.00698215235024691
713 0.00698215235024691
714 0.00698215235024691
715 0.00698215235024691
716 0.00698215235024691
717 0.00698215235024691
718 0.00698215235024691
719 0.00698215235024691
720 0.00698215235024691
721 0.00698215235024691
722 0.00698215235024691
723 0.00698215235024691
724 0.00698215235024691
725 0.00698215235024691
726 0.00698215235024691
727 0.00698215235024691
728 0.00698215235024691
729 0.00698215235024691
730 0.00698215235024691
731 0.00698215235024691
732 0.00698215235024691
733 0.00698215235024691
734 0.00698215235024691
735 0.00698215235024691
736 0.00698215235024691
737 0.00698215235024691
738 0.00698215235024691
739 0.00698215235024691
740 0.00698215235024691
741 0.00698215235024691
742 0.00698215235024691
743 0.00698215235024691
744 0.00698215235024691
745 0.00698215235024691
746 0.00698215235024691
747 0.00698215235024691
748 0.00698215235024691
749 0.00698215235024691
750 0.00698215235024691
751 0.00698215235024691
752 0.00698215235024691
753 0.00698215235024691
754 0.00698215235024691
755 0.00698215235024691
756 0.00698215235024691
757 0.00698215235024691
758 0.00698215235024691
759 0.00698215235024691
760 0.00698215235024691
761 0.00698215235024691
762 0.00698215235024691
763 0.00698215235024691
764 0.00698215235024691
765 0.00698215235024691
766 0.00698215235024691
767 0.00698215235024691
768 0.00698215235024691
769 0.00698215235024691
770 0.00698215235024691
771 0.00698215235024691
772 0.00698215235024691
773 0.00698215235024691
774 0.00698215235024691
775 0.00698215235024691
776 0.00698215235024691
777 0.00698215235024691
778 0.00698215235024691
779 0.00698215235024691
780 0.00698215235024691
781 0.00698215235024691
782 0.00698215235024691
783 0.00698215235024691
784 0.00698215235024691
785 0.00698215235024691
786 0.00698215235024691
787 0.00698215235024691
788 0.00698215235024691
789 0.00698215235024691
790 0.00698215235024691
791 0.00698215235024691
792 0.00698215235024691
793 0.00698215235024691
794 0.00698215235024691
795 0.00698215235024691
796 0.00698215235024691
797 0.00698215235024691
798 0.00698215235024691
799 0.00698215235024691
800 0.00698215235024691
801 0.00698215235024691
802 0.00698215235024691
803 0.00698215235024691
804 0.00698215235024691
805 0.00698215235024691
806 0.00698215235024691
807 0.00698215235024691
808 0.00698215235024691
809 0.00698215235024691
810 0.00698215235024691
811 0.00698215235024691
812 0.00698215235024691
813 0.00698215235024691
814 0.00698215235024691
815 0.00698215235024691
816 0.00698215235024691
817 0.00698215235024691
818 0.00698215235024691
819 0.00698215235024691
820 0.00698215235024691
821 0.00698215235024691
822 0.00698215235024691
823 0.00698215235024691
824 0.00698215235024691
825 0.00698215235024691
826 0.00698215235024691
827 0.00698215235024691
828 0.00698215235024691
829 0.00698215235024691
830 0.00698215235024691
831 0.00698215235024691
832 0.00698215235024691
833 0.00698215235024691
834 0.00698215235024691
835 0.00698215235024691
836 0.00698215235024691
837 0.00698215235024691
838 0.00698215235024691
839 0.00698215235024691
840 0.00698215235024691
841 0.00698215235024691
842 0.00698215235024691
843 0.00698215235024691
844 0.00698215235024691
845 0.00698215235024691
846 0.00698215235024691
847 0.00698215235024691
848 0.00698215235024691
849 0.00698215235024691
850 0.00698215235024691
851 0.00698215235024691
852 0.00698215235024691
853 0.00698215235024691
854 0.00698215235024691
855 0.00698215235024691
856 0.00698215235024691
857 0.00698215235024691
858 0.00698215235024691
859 0.00698215235024691
860 0.00698215235024691
861 0.00698215235024691
862 0.00698215235024691
863 0.00698215235024691
864 0.00698215235024691
865 0.00698215235024691
866 0.00698215235024691
867 0.00698215235024691
868 0.00698215235024691
869 0.00698215235024691
870 0.00698215235024691
871 0.00698215235024691
872 0.00698215235024691
873 0.00698215235024691
874 0.00698215235024691
875 0.00698215235024691
876 0.00698215235024691
877 0.00698215235024691
878 0.00698215235024691
879 0.00698215235024691
880 0.00698215235024691
881 0.00698215235024691
882 0.00698215235024691
883 0.00698215235024691
884 0.00698215235024691
885 0.00698215235024691
886 0.00698215235024691
887 0.00698215235024691
888 0.00698215235024691
889 0.00698215235024691
890 0.00698215235024691
891 0.00698215235024691
892 0.00698215235024691
893 0.00698215235024691
894 0.00698215235024691
895 0.00698215235024691
896 0.00698215235024691
897 0.00698215235024691
898 0.00698215235024691
899 0.00698215235024691
900 0.00698215235024691
901 0.00698215235024691
902 0.00698215235024691
903 0.00698215235024691
904 0.00698215235024691
905 0.00698215235024691
906 0.00698215235024691
907 0.00698215235024691
908 0.00698215235024691
909 0.00698215235024691
910 0.00698215235024691
911 0.00698215235024691
912 0.00698215235024691
913 0.00698215235024691
914 0.00698215235024691
915 0.00698215235024691
916 0.00698215235024691
917 0.00698215235024691
918 0.00698215235024691
919 0.00698215235024691
920 0.00698215235024691
921 0.00698215235024691
922 0.00698215235024691
923 0.00698215235024691
924 0.00698215235024691
925 0.00698215235024691
926 0.00698215235024691
927 0.00698215235024691
928 0.00698215235024691
929 0.00698215235024691
930 0.00698215235024691
931 0.00698215235024691
932 0.00698215235024691
933 0.00698215235024691
934 0.00698215235024691
935 0.00698215235024691
936 0.00698215235024691
937 0.00698215235024691
938 0.00698215235024691
939 0.00698215235024691
940 0.00698215235024691
941 0.00698215235024691
942 0.00698215235024691
943 0.00698215235024691
944 0.00698215235024691
945 0.00698215235024691
946 0.00698215235024691
947 0.00698215235024691
948 0.00698215235024691
949 0.00698215235024691
950 0.00698215235024691
951 0.00698215235024691
952 0.00698215235024691
953 0.00698215235024691
954 0.00698215235024691
955 0.00698215235024691
956 0.00698215235024691
957 0.00698215235024691
958 0.00698215235024691
959 0.00698215235024691
960 0.00698215235024691
961 0.00698215235024691
962 0.00698215235024691
963 0.00698215235024691
964 0.00698215235024691
965 0.00698215235024691
966 0.00698215235024691
967 0.00698215235024691
968 0.00698215235024691
969 0.00698215235024691
970 0.00698215235024691
971 0.00698215235024691
972 0.00698215235024691
973 0.00698215235024691
974 0.00698215235024691
975 0.00698215235024691
976 0.00698215235024691
977 0.00698215235024691
978 0.00698215235024691
979 0.00698215235024691
980 0.00698215235024691
981 0.00698215235024691
982 0.00698215235024691
983 0.00698215235024691
984 0.00698215235024691
985 0.00698215235024691
986 0.00698215235024691
987 0.00698215235024691
988 0.00698215235024691
989 0.00698215235024691
990 0.00698215235024691
991 0.00698215235024691
992 0.00698215235024691
993 0.00698215235024691
994 0.00698215235024691
995 0.00698215235024691
996 0.00698215235024691
997 0.00698215235024691
998 0.00698215235024691
999 0.00698215235024691
1000 0.00698215235024691
1001 0.00698215235024691
1002 0.00698215235024691
1003 0.00698215235024691
1004 0.00698215235024691
1005 0.00698215235024691
1006 0.00698215235024691
1007 0.00698215235024691
1008 0.00698215235024691
1009 0.00698215235024691
1010 0.00698215235024691
1011 0.00698215235024691
1012 0.00698215235024691
1013 0.00698215235024691
1014 0.00698215235024691
1015 0.00698215235024691
1016 0.00698215235024691
1017 0.00698215235024691
1018 0.00698215235024691
1019 0.00698215235024691
1020 0.00698215235024691
1021 0.00698215235024691
1022 0.00698215235024691
1023 0.00698215235024691
1024 0.00698215235024691
1025 0.00698215235024691
1026 0.00698215235024691
1027 0.00698215235024691
1028 0.00698215235024691
1029 0.00698215235024691
1030 0.00698215235024691
1031 0.00698215235024691
1032 0.00698215235024691
1033 0.00698215235024691
1034 0.00698215235024691
1035 0.00698215235024691
1036 0.00698215235024691
1037 0.00698215235024691
1038 0.00698215235024691
1039 0.00698215235024691
1040 0.00698215235024691
1041 0.00698215235024691
1042 0.00698215235024691
1043 0.00698215235024691
1044 0.00698215235024691
1045 0.00698215235024691
1046 0.00698215235024691
1047 0.00698215235024691
1048 0.00698215235024691
1049 0.00698215235024691
1050 0.00698215235024691
1051 0.00698215235024691
1052 0.00698215235024691
1053 0.00698215235024691
1054 0.00698215235024691
1055 0.00698215235024691
1056 0.00698215235024691
1057 0.00698215235024691
1058 0.00698215235024691
1059 0.00698215235024691
1060 0.00698215235024691
1061 0.00698215235024691
1062 0.00698215235024691
1063 0.00698215235024691
1064 0.00698215235024691
1065 0.00698215235024691
1066 0.00698215235024691
1067 0.00698215235024691
1068 0.00698215235024691
1069 0.00698215235024691
1070 0.00698215235024691
1071 0.00698215235024691
1072 0.00698215235024691
1073 0.00698215235024691
1074 0.00698215235024691
1075 0.00698215235024691
1076 0.00698215235024691
1077 0.00698215235024691
1078 0.00698215235024691
1079 0.00698215235024691
1080 0.00698215235024691
1081 0.00698215235024691
1082 0.00698215235024691
1083 0.00698215235024691
1084 0.00698215235024691
1085 0.00698215235024691
1086 0.00698215235024691
1087 0.00698215235024691
1088 0.00698215235024691
1089 0.00698215235024691
1090 0.00698215235024691
1091 0.00698215235024691
1092 0.00698215235024691
1093 0.00698215235024691
1094 0.00698215235024691
1095 0.00698215235024691
1096 0.00698215235024691
1097 0.00698215235024691
1098 0.00698215235024691
1099 0.00698215235024691
1100 0.00698215235024691
1101 0.00698215235024691
1102 0.00698215235024691
1103 0.00698215235024691
1104 0.00698215235024691
1105 0.00698215235024691
1106 0.00698215235024691
1107 0.00698215235024691
1108 0.00698215235024691
1109 0.00698215235024691
1110 0.00698215235024691
1111 0.00698215235024691
1112 0.00698215235024691
1113 0.00698215235024691
1114 0.00698215235024691
1115 0.00698215235024691
1116 0.00698215235024691
1117 0.00698215235024691
1118 0.00698215235024691
1119 0.00698215235024691
1120 0.00698215235024691
1121 0.00698215235024691
1122 0.00698215235024691
1123 0.00698215235024691
1124 0.00698215235024691
1125 0.00698215235024691
1126 0.00698215235024691
1127 0.00698215235024691
1128 0.00698215235024691
1129 0.00698215235024691
1130 0.00698215235024691
1131 0.00698215235024691
1132 0.00698215235024691
1133 0.00698215235024691
1134 0.00698215235024691
1135 0.00698215235024691
1136 0.00698215235024691
1137 0.00698215235024691
1138 0.00698215235024691
1139 0.00698215235024691
1140 0.00698215235024691
1141 0.00698215235024691
1142 0.00698215235024691
1143 0.00698215235024691
1144 0.00698215235024691
1145 0.00698215235024691
1146 0.00698215235024691
1147 0.00698215235024691
1148 0.00698215235024691
1149 0.00698215235024691
1150 0.00698215235024691
1151 0.00698215235024691
1152 0.00698215235024691
1153 0.00698215235024691
1154 0.00698215235024691
1155 0.00698215235024691
1156 0.00698215235024691
1157 0.00698215235024691
1158 0.00698215235024691
1159 0.00698215235024691
1160 0.00698215235024691
1161 0.00698215235024691
1162 0.00698215235024691
1163 0.00698215235024691
1164 0.00698215235024691
1165 0.00698215235024691
1166 0.00698215235024691
1167 0.00698215235024691
1168 0.00698215235024691
1169 0.00698215235024691
1170 0.00698215235024691
1171 0.00698215235024691
1172 0.00698215235024691
1173 0.00698215235024691
1174 0.00698215235024691
1175 0.00698215235024691
1176 0.00698215235024691
1177 0.00698215235024691
1178 0.00698215235024691
1179 0.00698215235024691
1180 0.00698215235024691
1181 0.00698215235024691
1182 0.00698215235024691
1183 0.00698215235024691
1184 0.00698215235024691
1185 0.00698215235024691
1186 0.00698215235024691
1187 0.00698215235024691
1188 0.00698215235024691
1189 0.00698215235024691
1190 0.00698215235024691
1191 0.00698215235024691
1192 0.00698215235024691
1193 0.00698215235024691
1194 0.00698215235024691
1195 0.00698215235024691
1196 0.00698215235024691
1197 0.00698215235024691
1198 0.00698215235024691
1199 0.00698215235024691
1200 0.00698215235024691
1201 0.00698215235024691
1202 0.00698215235024691
1203 0.00698215235024691
1204 0.00698215235024691
1205 0.00698215235024691
1206 0.00698215235024691
1207 0.00698215235024691
1208 0.00698215235024691
1209 0.00698215235024691
1210 0.00698215235024691
1211 0.00698215235024691
1212 0.00698215235024691
1213 0.00698215235024691
1214 0.00698215235024691
1215 0.00698215235024691
1216 0.00698215235024691
1217 0.00698215235024691
1218 0.00698215235024691
1219 0.00698215235024691
1220 0.00698215235024691
1221 0.00698215235024691
1222 0.00698215235024691
1223 0.00698215235024691
1224 0.00698215235024691
1225 0.00698215235024691
1226 0.00698215235024691
1227 0.00698215235024691
1228 0.00698215235024691
1229 0.00698215235024691
1230 0.00698215235024691
1231 0.00698215235024691
1232 0.00698215235024691
1233 0.00698215235024691
1234 0.00698215235024691
1235 0.00698215235024691
1236 0.00698215235024691
1237 0.00698215235024691
1238 0.00698215235024691
1239 0.00698215235024691
1240 0.00698215235024691
1241 0.00698215235024691
1242 0.00698215235024691
1243 0.00698215235024691
1244 0.00698215235024691
1245 0.00698215235024691
1246 0.00698215235024691
1247 0.00698215235024691
1248 0.00698215235024691
1249 0.00698215235024691
1250 0.00698215235024691
1251 0.00698215235024691
1252 0.00698215235024691
1253 0.00698215235024691
1254 0.00698215235024691
1255 0.00698215235024691
1256 0.00698215235024691
1257 0.00698215235024691
1258 0.00698215235024691
1259 0.00698215235024691
1260 0.00698215235024691
1261 0.00698215235024691
1262 0.00698215235024691
1263 0.00698215235024691
1264 0.00698215235024691
1265 0.00698215235024691
1266 0.00698215235024691
1267 0.00698215235024691
1268 0.00698215235024691
1269 0.00698215235024691
1270 0.00698215235024691
1271 0.00698215235024691
1272 0.00698215235024691
1273 0.00698215235024691
1274 0.00698215235024691
1275 0.00698215235024691
1276 0.00698215235024691
1277 0.00698215235024691
1278 0.00698215235024691
1279 0.00698215235024691
1280 0.00698215235024691
1281 0.00698215235024691
1282 0.00698215235024691
1283 0.00698215235024691
1284 0.00698215235024691
1285 0.00698215235024691
1286 0.00698215235024691
1287 0.00698215235024691
1288 0.00698215235024691
1289 0.00698215235024691
1290 0.00698215235024691
1291 0.00698215235024691
1292 0.00698215235024691
1293 0.00698215235024691
1294 0.00698215235024691
1295 0.00698215235024691
1296 0.00698215235024691
1297 0.00698215235024691
1298 0.00698215235024691
1299 0.00698215235024691
1300 0.00698215235024691
1301 0.00698215235024691
1302 0.00698215235024691
1303 0.00698215235024691
1304 0.00698215235024691
1305 0.00698215235024691
1306 0.00698215235024691
1307 0.00698215235024691
1308 0.00698215235024691
1309 0.00698215235024691
1310 0.00698215235024691
1311 0.00698215235024691
1312 0.00698215235024691
1313 0.00698215235024691
1314 0.00698215235024691
1315 0.00698215235024691
1316 0.00698215235024691
1317 0.00698215235024691
1318 0.00698215235024691
1319 0.00698215235024691
1320 0.00698215235024691
1321 0.00698215235024691
1322 0.00698215235024691
1323 0.00698215235024691
1324 0.00698215235024691
1325 0.00698215235024691
1326 0.00698215235024691
1327 0.00698215235024691
1328 0.00698215235024691
1329 0.00698215235024691
1330 0.00698215235024691
1331 0.00698215235024691
1332 0.00698215235024691
1333 0.00698215235024691
1334 0.00698215235024691
1335 0.00698215235024691
1336 0.00698215235024691
1337 0.00698215235024691
1338 0.00698215235024691
1339 0.00698215235024691
1340 0.00698215235024691
1341 0.00698215235024691
1342 0.00698215235024691
1343 0.00698215235024691
1344 0.00698215235024691
1345 0.00698215235024691
1346 0.00698215235024691
1347 0.00698215235024691
1348 0.00698215235024691
1349 0.00698215235024691
1350 0.00698215235024691
1351 0.00698215235024691
1352 0.00698215235024691
1353 0.00698215235024691
1354 0.00698215235024691
1355 0.00698215235024691
1356 0.00698215235024691
1357 0.00698215235024691
1358 0.00698215235024691
1359 0.00698215235024691
1360 0.00698215235024691
1361 0.00698215235024691
1362 0.00698215235024691
1363 0.00698215235024691
1364 0.00698215235024691
1365 0.00698215235024691
1366 0.00698215235024691
1367 0.00698215235024691
1368 0.00698215235024691
1369 0.00698215235024691
1370 0.00698215235024691
1371 0.00698215235024691
1372 0.00698215235024691
1373 0.00698215235024691
1374 0.00698215235024691
1375 0.00698215235024691
1376 0.00698215235024691
1377 0.00698215235024691
1378 0.00698215235024691
1379 0.00698215235024691
1380 0.00698215235024691
1381 0.00698215235024691
1382 0.00698215235024691
1383 0.00698215235024691
1384 0.00698215235024691
1385 0.00698215235024691
1386 0.00698215235024691
1387 0.00698215235024691
1388 0.00698215235024691
1389 0.00698215235024691
1390 0.00698215235024691
1391 0.00698215235024691
1392 0.00698215235024691
1393 0.00698215235024691
1394 0.00698215235024691
1395 0.00698215235024691
1396 0.00698215235024691
1397 0.00698215235024691
1398 0.00698215235024691
1399 0.00698215235024691
1400 0.00698215235024691
1401 0.00698215235024691
1402 0.00698215235024691
1403 0.00698215235024691
1404 0.00698215235024691
1405 0.00698215235024691
1406 0.00698215235024691
1407 0.00698215235024691
1408 0.00698215235024691
1409 0.00698215235024691
1410 0.00698215235024691
1411 0.00698215235024691
1412 0.00698215235024691
1413 0.00698215235024691
1414 0.00698215235024691
1415 0.00698215235024691
1416 0.00698215235024691
1417 0.00698215235024691
1418 0.00698215235024691
1419 0.00698215235024691
1420 0.00698215235024691
1421 0.00698215235024691
1422 0.00698215235024691
1423 0.00698215235024691
1424 0.00698215235024691
1425 0.00698215235024691
1426 0.00698215235024691
1427 0.00698215235024691
1428 0.00698215235024691
1429 0.00698215235024691
1430 0.00698215235024691
1431 0.00698215235024691
1432 0.00698215235024691
1433 0.00698215235024691
1434 0.00698215235024691
1435 0.00698215235024691
1436 0.00698215235024691
1437 0.00698215235024691
1438 0.00698215235024691
1439 0.00698215235024691
1440 0.00698215235024691
1441 0.00698215235024691
1442 0.00698215235024691
1443 0.00698215235024691
1444 0.00698215235024691
1445 0.00698215235024691
1446 0.00698215235024691
1447 0.00698215235024691
1448 0.00698215235024691
1449 0.00698215235024691
1450 0.00698215235024691
1451 0.00698215235024691
1452 0.00698215235024691
1453 0.00698215235024691
1454 0.00698215235024691
1455 0.00698215235024691
1456 0.00698215235024691
1457 0.00698215235024691
1458 0.00698215235024691
1459 0.00698215235024691
1460 0.00698215235024691
1461 0.00698215235024691
1462 0.00698215235024691
1463 0.00698215235024691
1464 0.00698215235024691
1465 0.00698215235024691
1466 0.00698215235024691
1467 0.00698215235024691
1468 0.00698215235024691
1469 0.00698215235024691
1470 0.00698215235024691
1471 0.00698215235024691
1472 0.00698215235024691
1473 0.00698215235024691
1474 0.00698215235024691
1475 0.00698215235024691
1476 0.00698215235024691
1477 0.00698215235024691
1478 0.00698215235024691
1479 0.00698215235024691
1480 0.00698215235024691
1481 0.00698215235024691
1482 0.00698215235024691
1483 0.00698215235024691
1484 0.00698215235024691
1485 0.00698215235024691
1486 0.00698215235024691
1487 0.00698215235024691
1488 0.00698215235024691
1489 0.00698215235024691
1490 0.00698215235024691
1491 0.00698215235024691
1492 0.00698215235024691
1493 0.00698215235024691
1494 0.00698215235024691
1495 0.00698215235024691
1496 0.00698215235024691
1497 0.00698215235024691
1498 0.00698215235024691
1499 0.00698215235024691
1500 0.00698215235024691
1501 0.00698215235024691
1502 0.00698215235024691
1503 0.00698215235024691
1504 0.00698215235024691
1505 0.00698215235024691
1506 0.00698215235024691
1507 0.00698215235024691
1508 0.00698215235024691
1509 0.00698215235024691
1510 0.00698215235024691
1511 0.00698215235024691
1512 0.00698215235024691
1513 0.00698215235024691
1514 0.00698215235024691
1515 0.00698215235024691
1516 0.00698215235024691
1517 0.00698215235024691
1518 0.00698215235024691
1519 0.00698215235024691
1520 0.00698215235024691
1521 0.00698215235024691
1522 0.00698215235024691
1523 0.00698215235024691
1524 0.00698215235024691
1525 0.00698215235024691
1526 0.00698215235024691
1527 0.00698215235024691
1528 0.00698215235024691
1529 0.00698215235024691
1530 0.00698215235024691
1531 0.00698215235024691
1532 0.00698215235024691
1533 0.00698215235024691
1534 0.00698215235024691
1535 0.00698215235024691
1536 0.00698215235024691
1537 0.00698215235024691
1538 0.00698215235024691
1539 0.00698215235024691
1540 0.00698215235024691
1541 0.00698215235024691
1542 0.00698215235024691
1543 0.00698215235024691
1544 0.00698215235024691
1545 0.00698215235024691
1546 0.00698215235024691
1547 0.00698215235024691
1548 0.00698215235024691
1549 0.00698215235024691
1550 0.00698215235024691
1551 0.00698215235024691
1552 0.00698215235024691
1553 0.00698215235024691
1554 0.00698215235024691
1555 0.00698215235024691
1556 0.00698215235024691
1557 0.00698215235024691
1558 0.00698215235024691
1559 0.00698215235024691
1560 0.00698215235024691
1561 0.00698215235024691
1562 0.00698215235024691
1563 0.00698215235024691
1564 0.00698215235024691
1565 0.00698215235024691
1566 0.00698215235024691
1567 0.00698215235024691
1568 0.00698215235024691
1569 0.00698215235024691
1570 0.00698215235024691
1571 0.00698215235024691
1572 0.00698215235024691
1573 0.00698215235024691
1574 0.00698215235024691
1575 0.00698215235024691
1576 0.00698215235024691
1577 0.00698215235024691
1578 0.00698215235024691
1579 0.00698215235024691
1580 0.00698215235024691
1581 0.00698215235024691
1582 0.00698215235024691
1583 0.00698215235024691
1584 0.00698215235024691
1585 0.00698215235024691
1586 0.00698215235024691
1587 0.00698215235024691
1588 0.00698215235024691
1589 0.00698215235024691
1590 0.00698215235024691
1591 0.00698215235024691
1592 0.00698215235024691
1593 0.00698215235024691
1594 0.00698215235024691
1595 0.00698215235024691
1596 0.00698215235024691
1597 0.00698215235024691
1598 0.00698215235024691
1599 0.00698215235024691
1600 0.00698215235024691
1601 0.00698215235024691
1602 0.00698215235024691
1603 0.00698215235024691
1604 0.00698215235024691
1605 0.00698215235024691
1606 0.00698215235024691
1607 0.00698215235024691
1608 0.00698215235024691
1609 0.00698215235024691
1610 0.00698215235024691
1611 0.00698215235024691
1612 0.00698215235024691
1613 0.00698215235024691
1614 0.00698215235024691
1615 0.00698215235024691
1616 0.00698215235024691
1617 0.00698215235024691
1618 0.00698215235024691
1619 0.00698215235024691
1620 0.00698215235024691
1621 0.00698215235024691
1622 0.00698215235024691
1623 0.00698215235024691
1624 0.00698215235024691
1625 0.00698215235024691
1626 0.00698215235024691
1627 0.00698215235024691
1628 0.00698215235024691
1629 0.00698215235024691
1630 0.00698215235024691
1631 0.00698215235024691
1632 0.00698215235024691
1633 0.00698215235024691
1634 0.00698215235024691
1635 0.00698215235024691
1636 0.00698215235024691
1637 0.00698215235024691
1638 0.00698215235024691
1639 0.00698215235024691
1640 0.00698215235024691
1641 0.00698215235024691
1642 0.00698215235024691
1643 0.00698215235024691
1644 0.00698215235024691
1645 0.00698215235024691
1646 0.00698215235024691
1647 0.00698215235024691
1648 0.00698215235024691
1649 0.00698215235024691
1650 0.00698215235024691
1651 0.00698215235024691
1652 0.00698215235024691
1653 0.00698215235024691
1654 0.00698215235024691
1655 0.00698215235024691
1656 0.00698215235024691
1657 0.00698215235024691
1658 0.00698215235024691
1659 0.00698215235024691
1660 0.00698215235024691
1661 0.00698215235024691
1662 0.00698215235024691
1663 0.00698215235024691
1664 0.00698215235024691
1665 0.00698215235024691
1666 0.00698215235024691
1667 0.00698215235024691
1668 0.00698215235024691
1669 0.00698215235024691
1670 0.00698215235024691
1671 0.00698215235024691
1672 0.00698215235024691
1673 0.00698215235024691
1674 0.00698215235024691
1675 0.00698215235024691
1676 0.00698215235024691
1677 0.00698215235024691
1678 0.00698215235024691
1679 0.00698215235024691
1680 0.00698215235024691
1681 0.00698215235024691
1682 0.00698215235024691
1683 0.00698215235024691
1684 0.00698215235024691
1685 0.00698215235024691
1686 0.00698215235024691
1687 0.00698215235024691
1688 0.00698215235024691
1689 0.00698215235024691
1690 0.00698215235024691
1691 0.00698215235024691
1692 0.00698215235024691
1693 0.00698215235024691
1694 0.00698215235024691
1695 0.00698215235024691
1696 0.00698215235024691
1697 0.00698215235024691
1698 0.00698215235024691
1699 0.00698215235024691
1700 0.00698215235024691
1701 0.00698215235024691
1702 0.00698215235024691
1703 0.00698215235024691
1704 0.00698215235024691
1705 0.00698215235024691
1706 0.00698215235024691
1707 0.00698215235024691
1708 0.00698215235024691
1709 0.00698215235024691
1710 0.00698215235024691
1711 0.00698215235024691
1712 0.00698215235024691
1713 0.00698215235024691
1714 0.00698215235024691
1715 0.00698215235024691
1716 0.00698215235024691
1717 0.00698215235024691
1718 0.00698215235024691
1719 0.00698215235024691
1720 0.00698215235024691
1721 0.00698215235024691
1722 0.00698215235024691
1723 0.00698215235024691
1724 0.00698215235024691
1725 0.00698215235024691
1726 0.00698215235024691
1727 0.00698215235024691
1728 0.00698215235024691
1729 0.00698215235024691
1730 0.00698215235024691
1731 0.00698215235024691
1732 0.00698215235024691
1733 0.00698215235024691
1734 0.00698215235024691
1735 0.00698215235024691
1736 0.00698215235024691
1737 0.00698215235024691
1738 0.00698215235024691
1739 0.00698215235024691
1740 0.00698215235024691
1741 0.00698215235024691
1742 0.00698215235024691
1743 0.00698215235024691
1744 0.00698215235024691
1745 0.00698215235024691
1746 0.00698215235024691
1747 0.00698215235024691
1748 0.00698215235024691
1749 0.00698215235024691
1750 0.00698215235024691
1751 0.00698215235024691
1752 0.00698215235024691
1753 0.00698215235024691
1754 0.00698215235024691
1755 0.00698215235024691
1756 0.00698215235024691
1757 0.00698215235024691
1758 0.00698215235024691
1759 0.00698215235024691
1760 0.00698215235024691
1761 0.00698215235024691
1762 0.00698215235024691
1763 0.00698215235024691
1764 0.00698215235024691
1765 0.00698215235024691
1766 0.00698215235024691
1767 0.00698215235024691
1768 0.00698215235024691
1769 0.00698215235024691
1770 0.00698215235024691
1771 0.00698215235024691
1772 0.00698215235024691
1773 0.00698215235024691
1774 0.00698215235024691
1775 0.00698215235024691
1776 0.00698215235024691
1777 0.00698215235024691
1778 0.00698215235024691
1779 0.00698215235024691
1780 0.00698215235024691
1781 0.00698215235024691
1782 0.00698215235024691
1783 0.00698215235024691
1784 0.00698215235024691
1785 0.00698215235024691
1786 0.00698215235024691
1787 0.00698215235024691
1788 0.00698215235024691
1789 0.00698215235024691
1790 0.00698215235024691
1791 0.00698215235024691
1792 0.00698215235024691
1793 0.00698215235024691
1794 0.00698215235024691
1795 0.00698215235024691
1796 0.00698215235024691
1797 0.00698215235024691
1798 0.00698215235024691
1799 0.00698215235024691
1800 0.00698215235024691
1801 0.00698215235024691
1802 0.00698215235024691
1803 0.00698215235024691
1804 0.00698215235024691
1805 0.00698215235024691
1806 0.00698215235024691
1807 0.00698215235024691
1808 0.00698215235024691
1809 0.00698215235024691
1810 0.00698215235024691
1811 0.00698215235024691
1812 0.00698215235024691
1813 0.00698215235024691
1814 0.00698215235024691
1815 0.00698215235024691
1816 0.00698215235024691
1817 0.00698215235024691
1818 0.00698215235024691
1819 0.00698215235024691
1820 0.00698215235024691
1821 0.00698215235024691
1822 0.00698215235024691
1823 0.00698215235024691
1824 0.00698215235024691
1825 0.00698215235024691
1826 0.00698215235024691
1827 0.00698215235024691
1828 0.00698215235024691
1829 0.00698215235024691
1830 0.00698215235024691
1831 0.00698215235024691
1832 0.00698215235024691
1833 0.00698215235024691
1834 0.00698215235024691
1835 0.00698215235024691
1836 0.00698215235024691
1837 0.00698215235024691
1838 0.00698215235024691
1839 0.00698215235024691
1840 0.00698215235024691
1841 0.00698215235024691
1842 0.00698215235024691
1843 0.00698215235024691
1844 0.00698215235024691
1845 0.00698215235024691
1846 0.00698215235024691
1847 0.00698215235024691
1848 0.00698215235024691
1849 0.00698215235024691
1850 0.00698215235024691
1851 0.00698215235024691
1852 0.00698215235024691
1853 0.00698215235024691
1854 0.00698215235024691
1855 0.00698215235024691
1856 0.00698215235024691
1857 0.00698215235024691
1858 0.00698215235024691
1859 0.00698215235024691
1860 0.00698215235024691
1861 0.00698215235024691
1862 0.00698215235024691
1863 0.00698215235024691
1864 0.00698215235024691
1865 0.00698215235024691
1866 0.00698215235024691
1867 0.00698215235024691
1868 0.00698215235024691
1869 0.00698215235024691
1870 0.00698215235024691
1871 0.00698215235024691
1872 0.00698215235024691
1873 0.00698215235024691
1874 0.00698215235024691
1875 0.00698215235024691
1876 0.00698215235024691
1877 0.00698215235024691
1878 0.00698215235024691
1879 0.00698215235024691
1880 0.00698215235024691
1881 0.00698215235024691
1882 0.00698215235024691
1883 0.00698215235024691
1884 0.00698215235024691
1885 0.00698215235024691
1886 0.00698215235024691
1887 0.00698215235024691
1888 0.00698215235024691
1889 0.00698215235024691
1890 0.00698215235024691
1891 0.00698215235024691
1892 0.00698215235024691
1893 0.00698215235024691
1894 0.00698215235024691
1895 0.00698215235024691
1896 0.00698215235024691
1897 0.00698215235024691
1898 0.00698215235024691
1899 0.00698215235024691
1900 0.00698215235024691
1901 0.00698215235024691
1902 0.00698215235024691
1903 0.00698215235024691
1904 0.00698215235024691
1905 0.00698215235024691
1906 0.00698215235024691
1907 0.00698215235024691
1908 0.00698215235024691
1909 0.00698215235024691
1910 0.00698215235024691
1911 0.00698215235024691
1912 0.00698215235024691
1913 0.00698215235024691
1914 0.00698215235024691
1915 0.00698215235024691
1916 0.00698215235024691
1917 0.00698215235024691
1918 0.00698215235024691
1919 0.00698215235024691
1920 0.00698215235024691
1921 0.00698215235024691
1922 0.00698215235024691
1923 0.00698215235024691
1924 0.00698215235024691
1925 0.00698215235024691
1926 0.00698215235024691
1927 0.00698215235024691
1928 0.00698215235024691
1929 0.00698215235024691
1930 0.00698215235024691
1931 0.00698215235024691
1932 0.00698215235024691
1933 0.00698215235024691
1934 0.00698215235024691
1935 0.00698215235024691
1936 0.00698215235024691
1937 0.00698215235024691
1938 0.00698215235024691
1939 0.00698215235024691
1940 0.00698215235024691
1941 0.00698215235024691
1942 0.00698215235024691
1943 0.00698215235024691
1944 0.00698215235024691
1945 0.00698215235024691
1946 0.00698215235024691
1947 0.00698215235024691
1948 0.00698215235024691
1949 0.00698215235024691
1950 0.00698215235024691
1951 0.00698215235024691
1952 0.00698215235024691
1953 0.00698215235024691
1954 0.00698215235024691
1955 0.00698215235024691
1956 0.00698215235024691
1957 0.00698215235024691
1958 0.00698215235024691
1959 0.00698215235024691
1960 0.00698215235024691
1961 0.00698215235024691
1962 0.00698215235024691
1963 0.00698215235024691
1964 0.00698215235024691
1965 0.00698215235024691
1966 0.00698215235024691
1967 0.00698215235024691
1968 0.00698215235024691
1969 0.00698215235024691
1970 0.00698215235024691
1971 0.00698215235024691
1972 0.00698215235024691
1973 0.00698215235024691
1974 0.00698215235024691
1975 0.00698215235024691
1976 0.00698215235024691
1977 0.00698215235024691
1978 0.00698215235024691
1979 0.00698215235024691
1980 0.00698215235024691
1981 0.00698215235024691
1982 0.00698215235024691
1983 0.00698215235024691
1984 0.00698215235024691
1985 0.00698215235024691
1986 0.00698215235024691
1987 0.00698215235024691
1988 0.00698215235024691
1989 0.00698215235024691
1990 0.00698215235024691
1991 0.00698215235024691
1992 0.00698215235024691
1993 0.00698215235024691
1994 0.00698215235024691
1995 0.00698215235024691
1996 0.00698215235024691
1997 0.00698215235024691
1998 0.00698215235024691
1999 0.00698215235024691
};
\addlegendentry{Test}
\end{groupplot}

\end{tikzpicture}

		\label{Fig:Batch}
		\caption{Increasing the batch size to four. Shown are training- and validation error over 2000 epochs for the models with two, three and four convolutional layers in the encoder.}
	\end{figure}
\end{center}
The underfitting of the deepest model is now tackled with increasing te capacity of the model by raising the channel growth rate to quadratic, as it is found in \cite{Carlberg}. The same is performed for the model with three convolutional layers in encoder and decoder. Additionally the channel sizes of the small model are increased. The results and available channel sizes in order of appearance in the encoder are summarized in \cref{Tab:Channels}. Note that the decoder mirrors the channel sizes of the encoder. The small model does not learn at all as seen in \cref{Fig:Channels}. The training was repeated several times, but no improvement could be produced. The reason for this behavior is not known and can't be classified as underfitting and overfitting, as there is no change in training- and validation error observed. On the other hand does the deep network with four convolutional layers achieve comparable results to the small model in the previous experiment. The model with three convolutional layers improves slightly. In conclusion, the quadratic growth rate of the channels starting from eight channels is suitable for the following experiments, for which the three layer model is discarded as the other models reach the lowest validation error so far. 
\begin{table}[htbp!]
	\centering
	\caption{Increasing the channel growth rate to quadratic. Summary of minimum training- and minimum validation error for the models with two, three and four convolutional layers in the encoder as well as the corresponding \(\L2\) and the epoch in which those values are reached. Note that the channel sizes of the two layer model are only increased, the growth rate has already been quadratic.}
	\begin{tabular*}{15cm}{ @{\extracolsep{\fill}} c c c c c c c c c c @{} }
		\toprule
		Layer & Channels  & Minimum training error & Minimum validation error & \(\L2\) & Epoch \\ [.5ex]
		\hline
		2    & 16,32   	  & \num{6.8e-3}           & \num{7.6e-3}             & 1.0     & 0     \\
		\hline  
		3    & 8,16,32    & \num{9.0e-6}            & \num{1.1e-5}             & 0.037   & 1985  \\  
		\hline
		4    & 8,16,32,64 & \num{7.0e-6}           & \num{9.0e-6}             & 0.033   & 1953  \\
		\hline
	\end{tabular*}\label{Tab:Channels}
\end{table}   
\begin{center}
	\begin{figure}[htbp!]
		% This file was created by tikzplotlib v0.9.6.
\begin{tikzpicture}

\begin{groupplot}[
group style={group size=2 by 2,horizontal sep=2.5cm},
legend cell align={left},
legend style={fill opacity=1, draw opacity=1, text opacity=1, draw=white},
log basis y={10},
tick align=outside,
tick pos=left,
title style={at={(0.43,0.85)},anchor=north},
x grid style={white!69.0196078431373!black},
xlabel={Epoch},
x label style={yshift=13pt},
xmin=-49.95, xmax=2048.95,
xtick style={color=black},
xtick = {0,500,1500,2000},
y grid style={white!69.0196078431373!black},
ylabel={MSE Loss},
ymode=log,
ytick style={color=black},
width=.45\textwidth,
height=.25\textwidth
]
\nextgroupplot[
title={2 Layer},
ymin=0.00680839678125267, ymax=0.01,
]
\addplot [semithick, black, dashed]
table {%
0 0.00684494472323892
1 0.00684494472323892
2 0.00684494472323892
3 0.00684494472323892
4 0.00684494472323892
5 0.00684494472323892
6 0.00684494472323892
7 0.00684494472323892
8 0.00684494472323892
9 0.00684494472323892
10 0.00684494472323892
11 0.00684494472323892
12 0.00684494472323892
13 0.00684494472323892
14 0.00684494472323892
15 0.00684494472323892
16 0.00684494472323892
17 0.00684494472323892
18 0.00684494472323892
19 0.00684494472323892
20 0.00684494472323892
21 0.00684494472323892
22 0.00684494472323892
23 0.00684494472323892
24 0.00684494472323892
25 0.00684494472323892
26 0.00684494472323892
27 0.00684494472323892
28 0.00684494472323892
29 0.00684494472323892
30 0.00684494472323892
31 0.00684494472323892
32 0.00684494472323892
33 0.00684494472323892
34 0.00684494472323892
35 0.00684494472323892
36 0.00684494472323892
37 0.00684494472323892
38 0.00684494472323892
39 0.00684494472323892
40 0.00684494472323892
41 0.00684494472323892
42 0.00684494472323892
43 0.00684494472323892
44 0.00684494472323892
45 0.00684494472323892
46 0.00684494472323892
47 0.00684494472323892
48 0.00684494472323892
49 0.00684494472323892
50 0.00684494472323892
51 0.00684494472323892
52 0.00684494472323892
53 0.00684494472323892
54 0.00684494472323892
55 0.00684494472323892
56 0.00684494472323892
57 0.00684494472323892
58 0.00684494472323892
59 0.00684494472323892
60 0.00684494472323892
61 0.00684494472323892
62 0.00684494472323892
63 0.00684494472323892
64 0.00684494472323892
65 0.00684494472323892
66 0.00684494472323892
67 0.00684494472323892
68 0.00684494472323892
69 0.00684494472323892
70 0.00684494472323892
71 0.00684494472323892
72 0.00684494472323892
73 0.00684494472323892
74 0.00684494472323892
75 0.00684494472323892
76 0.00684494472323892
77 0.00684494472323892
78 0.00684494472323892
79 0.00684494472323892
80 0.00684494472323892
81 0.00684494472323892
82 0.00684494472323892
83 0.00684494472323892
84 0.00684494472323892
85 0.00684494472323892
86 0.00684494472323892
87 0.00684494472323892
88 0.00684494472323892
89 0.00684494472323892
90 0.00684494472323892
91 0.00684494472323892
92 0.00684494472323892
93 0.00684494472323892
94 0.00684494472323892
95 0.00684494472323892
96 0.00684494472323892
97 0.00684494472323892
98 0.00684494472323892
99 0.00684494472323892
100 0.00684494472323892
101 0.00684494472323892
102 0.00684494472323892
103 0.00684494472323892
104 0.00684494472323892
105 0.00684494472323892
106 0.00684494472323892
107 0.00684494472323892
108 0.00684494472323892
109 0.00684494472323892
110 0.00684494472323892
111 0.00684494472323892
112 0.00684494472323892
113 0.00684494472323892
114 0.00684494472323892
115 0.00684494472323892
116 0.00684494472323892
117 0.00684494472323892
118 0.00684494472323892
119 0.00684494472323892
120 0.00684494472323892
121 0.00684494472323892
122 0.00684494472323892
123 0.00684494472323892
124 0.00684494472323892
125 0.00684494472323892
126 0.00684494472323892
127 0.00684494472323892
128 0.00684494472323892
129 0.00684494472323892
130 0.00684494472323892
131 0.00684494472323892
132 0.00684494472323892
133 0.00684494472323892
134 0.00684494472323892
135 0.00684494472323892
136 0.00684494472323892
137 0.00684494472323892
138 0.00684494472323892
139 0.00684494472323892
140 0.00684494472323892
141 0.00684494472323892
142 0.00684494472323892
143 0.00684494472323892
144 0.00684494472323892
145 0.00684494472323892
146 0.00684494472323892
147 0.00684494472323892
148 0.00684494472323892
149 0.00684494472323892
150 0.00684494472323892
151 0.00684494472323892
152 0.00684494472323892
153 0.00684494472323892
154 0.00684494472323892
155 0.00684494472323892
156 0.00684494472323892
157 0.00684494472323892
158 0.00684494472323892
159 0.00684494472323892
160 0.00684494472323892
161 0.00684494472323892
162 0.00684494472323892
163 0.00684494472323892
164 0.00684494472323892
165 0.00684494472323892
166 0.00684494472323892
167 0.00684494472323892
168 0.00684494472323892
169 0.00684494472323892
170 0.00684494472323892
171 0.00684494472323892
172 0.00684494472323892
173 0.00684494472323892
174 0.00684494472323892
175 0.00684494472323892
176 0.00684494472323892
177 0.00684494472323892
178 0.00684494472323892
179 0.00684494472323892
180 0.00684494472323892
181 0.00684494472323892
182 0.00684494472323892
183 0.00684494472323892
184 0.00684494472323892
185 0.00684494472323892
186 0.00684494472323892
187 0.00684494472323892
188 0.00684494472323892
189 0.00684494472323892
190 0.00684494472323892
191 0.00684494472323892
192 0.00684494472323892
193 0.00684494472323892
194 0.00684494472323892
195 0.00684494472323892
196 0.00684494472323892
197 0.00684494472323892
198 0.00684494472323892
199 0.00684494472323892
200 0.00684494472323892
201 0.00684494472323892
202 0.00684494472323892
203 0.00684494472323892
204 0.00684494472323892
205 0.00684494472323892
206 0.00684494472323892
207 0.00684494472323892
208 0.00684494472323892
209 0.00684494472323892
210 0.00684494472323892
211 0.00684494472323892
212 0.00684494472323892
213 0.00684494472323892
214 0.00684494472323892
215 0.00684494472323892
216 0.00684494472323892
217 0.00684494472323892
218 0.00684494472323892
219 0.00684494472323892
220 0.00684494472323892
221 0.00684494472323892
222 0.00684494472323892
223 0.00684494472323892
224 0.00684494472323892
225 0.00684494472323892
226 0.00684494472323892
227 0.00684494472323892
228 0.00684494472323892
229 0.00684494472323892
230 0.00684494472323892
231 0.00684494472323892
232 0.00684494472323892
233 0.00684494472323892
234 0.00684494472323892
235 0.00684494472323892
236 0.00684494472323892
237 0.00684494472323892
238 0.00684494472323892
239 0.00684494472323892
240 0.00684494472323892
241 0.00684494472323892
242 0.00684494472323892
243 0.00684494472323892
244 0.00684494472323892
245 0.00684494472323892
246 0.00684494472323892
247 0.00684494472323892
248 0.00684494472323892
249 0.00684494472323892
250 0.00684494472323892
251 0.00684494472323892
252 0.00684494472323892
253 0.00684494472323892
254 0.00684494472323892
255 0.00684494472323892
256 0.00684494472323892
257 0.00684494472323892
258 0.00684494472323892
259 0.00684494472323892
260 0.00684494472323892
261 0.00684494472323892
262 0.00684494472323892
263 0.00684494472323892
264 0.00684494472323892
265 0.00684494472323892
266 0.00684494472323892
267 0.00684494472323892
268 0.00684494472323892
269 0.00684494472323892
270 0.00684494472323892
271 0.00684494472323892
272 0.00684494472323892
273 0.00684494472323892
274 0.00684494472323892
275 0.00684494472323892
276 0.00684494472323892
277 0.00684494472323892
278 0.00684494472323892
279 0.00684494472323892
280 0.00684494472323892
281 0.00684494472323892
282 0.00684494472323892
283 0.00684494472323892
284 0.00684494472323892
285 0.00684494472323892
286 0.00684494472323892
287 0.00684494472323892
288 0.00684494472323892
289 0.00684494472323892
290 0.00684494472323892
291 0.00684494472323892
292 0.00684494472323892
293 0.00684494472323892
294 0.00684494472323892
295 0.00684494472323892
296 0.00684494472323892
297 0.00684494472323892
298 0.00684494472323892
299 0.00684494472323892
300 0.00684494472323892
301 0.00684494472323892
302 0.00684494472323892
303 0.00684494472323892
304 0.00684494472323892
305 0.00684494472323892
306 0.00684494472323892
307 0.00684494472323892
308 0.00684494472323892
309 0.00684494472323892
310 0.00684494472323892
311 0.00684494472323892
312 0.00684494472323892
313 0.00684494472323892
314 0.00684494472323892
315 0.00684494472323892
316 0.00684494472323892
317 0.00684494472323892
318 0.00684494472323892
319 0.00684494472323892
320 0.00684494472323892
321 0.00684494472323892
322 0.00684494472323892
323 0.00684494472323892
324 0.00684494472323892
325 0.00684494472323892
326 0.00684494472323892
327 0.00684494472323892
328 0.00684494472323892
329 0.00684494472323892
330 0.00684494472323892
331 0.00684494472323892
332 0.00684494472323892
333 0.00684494472323892
334 0.00684494472323892
335 0.00684494472323892
336 0.00684494472323892
337 0.00684494472323892
338 0.00684494472323892
339 0.00684494472323892
340 0.00684494472323892
341 0.00684494472323892
342 0.00684494472323892
343 0.00684494472323892
344 0.00684494472323892
345 0.00684494472323892
346 0.00684494472323892
347 0.00684494472323892
348 0.00684494472323892
349 0.00684494472323892
350 0.00684494472323892
351 0.00684494472323892
352 0.00684494472323892
353 0.00684494472323892
354 0.00684494472323892
355 0.00684494472323892
356 0.00684494472323892
357 0.00684494472323892
358 0.00684494472323892
359 0.00684494472323892
360 0.00684494472323892
361 0.00684494472323892
362 0.00684494472323892
363 0.00684494472323892
364 0.00684494472323892
365 0.00684494472323892
366 0.00684494472323892
367 0.00684494472323892
368 0.00684494472323892
369 0.00684494472323892
370 0.00684494472323892
371 0.00684494472323892
372 0.00684494472323892
373 0.00684494472323892
374 0.00684494472323892
375 0.00684494472323892
376 0.00684494472323892
377 0.00684494472323892
378 0.00684494472323892
379 0.00684494472323892
380 0.00684494472323892
381 0.00684494472323892
382 0.00684494472323892
383 0.00684494472323892
384 0.00684494472323892
385 0.00684494472323892
386 0.00684494472323892
387 0.00684494472323892
388 0.00684494472323892
389 0.00684494472323892
390 0.00684494472323892
391 0.00684494472323892
392 0.00684494472323892
393 0.00684494472323892
394 0.00684494472323892
395 0.00684494472323892
396 0.00684494472323892
397 0.00684494472323892
398 0.00684494472323892
399 0.00684494472323892
400 0.00684494472323892
401 0.00684494472323892
402 0.00684494472323892
403 0.00684494472323892
404 0.00684494472323892
405 0.00684494472323892
406 0.00684494472323892
407 0.00684494472323892
408 0.00684494472323892
409 0.00684494472323892
410 0.00684494472323892
411 0.00684494472323892
412 0.00684494472323892
413 0.00684494472323892
414 0.00684494472323892
415 0.00684494472323892
416 0.00684494472323892
417 0.00684494472323892
418 0.00684494472323892
419 0.00684494472323892
420 0.00684494472323892
421 0.00684494472323892
422 0.00684494472323892
423 0.00684494472323892
424 0.00684494472323892
425 0.00684494472323892
426 0.00684494472323892
427 0.00684494472323892
428 0.00684494472323892
429 0.00684494472323892
430 0.00684494472323892
431 0.00684494472323892
432 0.00684494472323892
433 0.00684494472323892
434 0.00684494472323892
435 0.00684494472323892
436 0.00684494472323892
437 0.00684494472323892
438 0.00684494472323892
439 0.00684494472323892
440 0.00684494472323892
441 0.00684494472323892
442 0.00684494472323892
443 0.00684494472323892
444 0.00684494472323892
445 0.00684494472323892
446 0.00684494472323892
447 0.00684494472323892
448 0.00684494472323892
449 0.00684494472323892
450 0.00684494472323892
451 0.00684494472323892
452 0.00684494472323892
453 0.00684494472323892
454 0.00684494472323892
455 0.00684494472323892
456 0.00684494472323892
457 0.00684494472323892
458 0.00684494472323892
459 0.00684494472323892
460 0.00684494472323892
461 0.00684494472323892
462 0.00684494472323892
463 0.00684494472323892
464 0.00684494472323892
465 0.00684494472323892
466 0.00684494472323892
467 0.00684494472323892
468 0.00684494472323892
469 0.00684494472323892
470 0.00684494472323892
471 0.00684494472323892
472 0.00684494472323892
473 0.00684494472323892
474 0.00684494472323892
475 0.00684494472323892
476 0.00684494472323892
477 0.00684494472323892
478 0.00684494472323892
479 0.00684494472323892
480 0.00684494472323892
481 0.00684494472323892
482 0.00684494472323892
483 0.00684494472323892
484 0.00684494472323892
485 0.00684494472323892
486 0.00684494472323892
487 0.00684494472323892
488 0.00684494472323892
489 0.00684494472323892
490 0.00684494472323892
491 0.00684494472323892
492 0.00684494472323892
493 0.00684494472323892
494 0.00684494472323892
495 0.00684494472323892
496 0.00684494472323892
497 0.00684494472323892
498 0.00684494472323892
499 0.00684494472323892
500 0.00684494472323892
501 0.00684494472323892
502 0.00684494472323892
503 0.00684494472323892
504 0.00684494472323892
505 0.00684494472323892
506 0.00684494472323892
507 0.00684494472323892
508 0.00684494472323892
509 0.00684494472323892
510 0.00684494472323892
511 0.00684494472323892
512 0.00684494472323892
513 0.00684494472323892
514 0.00684494472323892
515 0.00684494472323892
516 0.00684494472323892
517 0.00684494472323892
518 0.00684494472323892
519 0.00684494472323892
520 0.00684494472323892
521 0.00684494472323892
522 0.00684494472323892
523 0.00684494472323892
524 0.00684494472323892
525 0.00684494472323892
526 0.00684494472323892
527 0.00684494472323892
528 0.00684494472323892
529 0.00684494472323892
530 0.00684494472323892
531 0.00684494472323892
532 0.00684494472323892
533 0.00684494472323892
534 0.00684494472323892
535 0.00684494472323892
536 0.00684494472323892
537 0.00684494472323892
538 0.00684494472323892
539 0.00684494472323892
540 0.00684494472323892
541 0.00684494472323892
542 0.00684494472323892
543 0.00684494472323892
544 0.00684494472323892
545 0.00684494472323892
546 0.00684494472323892
547 0.00684494472323892
548 0.00684494472323892
549 0.00684494472323892
550 0.00684494472323892
551 0.00684494472323892
552 0.00684494472323892
553 0.00684494472323892
554 0.00684494472323892
555 0.00684494472323892
556 0.00684494472323892
557 0.00684494472323892
558 0.00684494472323892
559 0.00684494472323892
560 0.00684494472323892
561 0.00684494472323892
562 0.00684494472323892
563 0.00684494472323892
564 0.00684494472323892
565 0.00684494472323892
566 0.00684494472323892
567 0.00684494472323892
568 0.00684494472323892
569 0.00684494472323892
570 0.00684494472323892
571 0.00684494472323892
572 0.00684494472323892
573 0.00684494472323892
574 0.00684494472323892
575 0.00684494472323892
576 0.00684494472323892
577 0.00684494472323892
578 0.00684494472323892
579 0.00684494472323892
580 0.00684494472323892
581 0.00684494472323892
582 0.00684494472323892
583 0.00684494472323892
584 0.00684494472323892
585 0.00684494472323892
586 0.00684494472323892
587 0.00684494472323892
588 0.00684494472323892
589 0.00684494472323892
590 0.00684494472323892
591 0.00684494472323892
592 0.00684494472323892
593 0.00684494472323892
594 0.00684494472323892
595 0.00684494472323892
596 0.00684494472323892
597 0.00684494472323892
598 0.00684494472323892
599 0.00684494472323892
600 0.00684494472323892
601 0.00684494472323892
602 0.00684494472323892
603 0.00684494472323892
604 0.00684494472323892
605 0.00684494472323892
606 0.00684494472323892
607 0.00684494472323892
608 0.00684494472323892
609 0.00684494472323892
610 0.00684494472323892
611 0.00684494472323892
612 0.00684494472323892
613 0.00684494472323892
614 0.00684494472323892
615 0.00684494472323892
616 0.00684494472323892
617 0.00684494472323892
618 0.00684494472323892
619 0.00684494472323892
620 0.00684494472323892
621 0.00684494472323892
622 0.00684494472323892
623 0.00684494472323892
624 0.00684494472323892
625 0.00684494472323892
626 0.00684494472323892
627 0.00684494472323892
628 0.00684494472323892
629 0.00684494472323892
630 0.00684494472323892
631 0.00684494472323892
632 0.00684494472323892
633 0.00684494472323892
634 0.00684494472323892
635 0.00684494472323892
636 0.00684494472323892
637 0.00684494472323892
638 0.00684494472323892
639 0.00684494472323892
640 0.00684494472323892
641 0.00684494472323892
642 0.00684494472323892
643 0.00684494472323892
644 0.00684494472323892
645 0.00684494472323892
646 0.00684494472323892
647 0.00684494472323892
648 0.00684494472323892
649 0.00684494472323892
650 0.00684494472323892
651 0.00684494472323892
652 0.00684494472323892
653 0.00684494472323892
654 0.00684494472323892
655 0.00684494472323892
656 0.00684494472323892
657 0.00684494472323892
658 0.00684494472323892
659 0.00684494472323892
660 0.00684494472323892
661 0.00684494472323892
662 0.00684494472323892
663 0.00684494472323892
664 0.00684494472323892
665 0.00684494472323892
666 0.00684494472323892
667 0.00684494472323892
668 0.00684494472323892
669 0.00684494472323892
670 0.00684494472323892
671 0.00684494472323892
672 0.00684494472323892
673 0.00684494472323892
674 0.00684494472323892
675 0.00684494472323892
676 0.00684494472323892
677 0.00684494472323892
678 0.00684494472323892
679 0.00684494472323892
680 0.00684494472323892
681 0.00684494472323892
682 0.00684494472323892
683 0.00684494472323892
684 0.00684494472323892
685 0.00684494472323892
686 0.00684494472323892
687 0.00684494472323892
688 0.00684494472323892
689 0.00684494472323892
690 0.00684494472323892
691 0.00684494472323892
692 0.00684494472323892
693 0.00684494472323892
694 0.00684494472323892
695 0.00684494472323892
696 0.00684494472323892
697 0.00684494472323892
698 0.00684494472323892
699 0.00684494472323892
700 0.00684494472323892
701 0.00684494472323892
702 0.00684494472323892
703 0.00684494472323892
704 0.00684494472323892
705 0.00684494472323892
706 0.00684494472323892
707 0.00684494472323892
708 0.00684494472323892
709 0.00684494472323892
710 0.00684494472323892
711 0.00684494472323892
712 0.00684494472323892
713 0.00684494472323892
714 0.00684494472323892
715 0.00684494472323892
716 0.00684494472323892
717 0.00684494472323892
718 0.00684494472323892
719 0.00684494472323892
720 0.00684494472323892
721 0.00684494472323892
722 0.00684494472323892
723 0.00684494472323892
724 0.00684494472323892
725 0.00684494472323892
726 0.00684494472323892
727 0.00684494472323892
728 0.00684494472323892
729 0.00684494472323892
730 0.00684494472323892
731 0.00684494472323892
732 0.00684494472323892
733 0.00684494472323892
734 0.00684494472323892
735 0.00684494472323892
736 0.00684494472323892
737 0.00684494472323892
738 0.00684494472323892
739 0.00684494472323892
740 0.00684494472323892
741 0.00684494472323892
742 0.00684494472323892
743 0.00684494472323892
744 0.00684494472323892
745 0.00684494472323892
746 0.00684494472323892
747 0.00684494472323892
748 0.00684494472323892
749 0.00684494472323892
750 0.00684494472323892
751 0.00684494472323892
752 0.00684494472323892
753 0.00684494472323892
754 0.00684494472323892
755 0.00684494472323892
756 0.00684494472323892
757 0.00684494472323892
758 0.00684494472323892
759 0.00684494472323892
760 0.00684494472323892
761 0.00684494472323892
762 0.00684494472323892
763 0.00684494472323892
764 0.00684494472323892
765 0.00684494472323892
766 0.00684494472323892
767 0.00684494472323892
768 0.00684494472323892
769 0.00684494472323892
770 0.00684494472323892
771 0.00684494472323892
772 0.00684494472323892
773 0.00684494472323892
774 0.00684494472323892
775 0.00684494472323892
776 0.00684494472323892
777 0.00684494472323892
778 0.00684494472323892
779 0.00684494472323892
780 0.00684494472323892
781 0.00684494472323892
782 0.00684494472323892
783 0.00684494472323892
784 0.00684494472323892
785 0.00684494472323892
786 0.00684494472323892
787 0.00684494472323892
788 0.00684494472323892
789 0.00684494472323892
790 0.00684494472323892
791 0.00684494472323892
792 0.00684494472323892
793 0.00684494472323892
794 0.00684494472323892
795 0.00684494472323892
796 0.00684494472323892
797 0.00684494472323892
798 0.00684494472323892
799 0.00684494472323892
800 0.00684494472323892
801 0.00684494472323892
802 0.00684494472323892
803 0.00684494472323892
804 0.00684494472323892
805 0.00684494472323892
806 0.00684494472323892
807 0.00684494472323892
808 0.00684494472323892
809 0.00684494472323892
810 0.00684494472323892
811 0.00684494472323892
812 0.00684494472323892
813 0.00684494472323892
814 0.00684494472323892
815 0.00684494472323892
816 0.00684494472323892
817 0.00684494472323892
818 0.00684494472323892
819 0.00684494472323892
820 0.00684494472323892
821 0.00684494472323892
822 0.00684494472323892
823 0.00684494472323892
824 0.00684494472323892
825 0.00684494472323892
826 0.00684494472323892
827 0.00684494472323892
828 0.00684494472323892
829 0.00684494472323892
830 0.00684494472323892
831 0.00684494472323892
832 0.00684494472323892
833 0.00684494472323892
834 0.00684494472323892
835 0.00684494472323892
836 0.00684494472323892
837 0.00684494472323892
838 0.00684494472323892
839 0.00684494472323892
840 0.00684494472323892
841 0.00684494472323892
842 0.00684494472323892
843 0.00684494472323892
844 0.00684494472323892
845 0.00684494472323892
846 0.00684494472323892
847 0.00684494472323892
848 0.00684494472323892
849 0.00684494472323892
850 0.00684494472323892
851 0.00684494472323892
852 0.00684494472323892
853 0.00684494472323892
854 0.00684494472323892
855 0.00684494472323892
856 0.00684494472323892
857 0.00684494472323892
858 0.00684494472323892
859 0.00684494472323892
860 0.00684494472323892
861 0.00684494472323892
862 0.00684494472323892
863 0.00684494472323892
864 0.00684494472323892
865 0.00684494472323892
866 0.00684494472323892
867 0.00684494472323892
868 0.00684494472323892
869 0.00684494472323892
870 0.00684494472323892
871 0.00684494472323892
872 0.00684494472323892
873 0.00684494472323892
874 0.00684494472323892
875 0.00684494472323892
876 0.00684494472323892
877 0.00684494472323892
878 0.00684494472323892
879 0.00684494472323892
880 0.00684494472323892
881 0.00684494472323892
882 0.00684494472323892
883 0.00684494472323892
884 0.00684494472323892
885 0.00684494472323892
886 0.00684494472323892
887 0.00684494472323892
888 0.00684494472323892
889 0.00684494472323892
890 0.00684494472323892
891 0.00684494472323892
892 0.00684494472323892
893 0.00684494472323892
894 0.00684494472323892
895 0.00684494472323892
896 0.00684494472323892
897 0.00684494472323892
898 0.00684494472323892
899 0.00684494472323892
900 0.00684494472323892
901 0.00684494472323892
902 0.00684494472323892
903 0.00684494472323892
904 0.00684494472323892
905 0.00684494472323892
906 0.00684494472323892
907 0.00684494472323892
908 0.00684494472323892
909 0.00684494472323892
910 0.00684494472323892
911 0.00684494472323892
912 0.00684494472323892
913 0.00684494472323892
914 0.00684494472323892
915 0.00684494472323892
916 0.00684494472323892
917 0.00684494472323892
918 0.00684494472323892
919 0.00684494472323892
920 0.00684494472323892
921 0.00684494472323892
922 0.00684494472323892
923 0.00684494472323892
924 0.00684494472323892
925 0.00684494472323892
926 0.00684494472323892
927 0.00684494472323892
928 0.00684494472323892
929 0.00684494472323892
930 0.00684494472323892
931 0.00684494472323892
932 0.00684494472323892
933 0.00684494472323892
934 0.00684494472323892
935 0.00684494472323892
936 0.00684494472323892
937 0.00684494472323892
938 0.00684494472323892
939 0.00684494472323892
940 0.00684494472323892
941 0.00684494472323892
942 0.00684494472323892
943 0.00684494472323892
944 0.00684494472323892
945 0.00684494472323892
946 0.00684494472323892
947 0.00684494472323892
948 0.00684494472323892
949 0.00684494472323892
950 0.00684494472323892
951 0.00684494472323892
952 0.00684494472323892
953 0.00684494472323892
954 0.00684494472323892
955 0.00684494472323892
956 0.00684494472323892
957 0.00684494472323892
958 0.00684494472323892
959 0.00684494472323892
960 0.00684494472323892
961 0.00684494472323892
962 0.00684494472323892
963 0.00684494472323892
964 0.00684494472323892
965 0.00684494472323892
966 0.00684494472323892
967 0.00684494472323892
968 0.00684494472323892
969 0.00684494472323892
970 0.00684494472323892
971 0.00684494472323892
972 0.00684494472323892
973 0.00684494472323892
974 0.00684494472323892
975 0.00684494472323892
976 0.00684494472323892
977 0.00684494472323892
978 0.00684494472323892
979 0.00684494472323892
980 0.00684494472323892
981 0.00684494472323892
982 0.00684494472323892
983 0.00684494472323892
984 0.00684494472323892
985 0.00684494472323892
986 0.00684494472323892
987 0.00684494472323892
988 0.00684494472323892
989 0.00684494472323892
990 0.00684494472323892
991 0.00684494472323892
992 0.00684494472323892
993 0.00684494472323892
994 0.00684494472323892
995 0.00684494472323892
996 0.00684494472323892
997 0.00684494472323892
998 0.00684494472323892
999 0.00684494472323892
1000 0.00684494472323892
1001 0.00684494472323892
1002 0.00684494472323892
1003 0.00684494472323892
1004 0.00684494472323892
1005 0.00684494472323892
1006 0.00684494472323892
1007 0.00684494472323892
1008 0.00684494472323892
1009 0.00684494472323892
1010 0.00684494472323892
1011 0.00684494472323892
1012 0.00684494472323892
1013 0.00684494472323892
1014 0.00684494472323892
1015 0.00684494472323892
1016 0.00684494472323892
1017 0.00684494472323892
1018 0.00684494472323892
1019 0.00684494472323892
1020 0.00684494472323892
1021 0.00684494472323892
1022 0.00684494472323892
1023 0.00684494472323892
1024 0.00684494472323892
1025 0.00684494472323892
1026 0.00684494472323892
1027 0.00684494472323892
1028 0.00684494472323892
1029 0.00684494472323892
1030 0.00684494472323892
1031 0.00684494472323892
1032 0.00684494472323892
1033 0.00684494472323892
1034 0.00684494472323892
1035 0.00684494472323892
1036 0.00684494472323892
1037 0.00684494472323892
1038 0.00684494472323892
1039 0.00684494472323892
1040 0.00684494472323892
1041 0.00684494472323892
1042 0.00684494472323892
1043 0.00684494472323892
1044 0.00684494472323892
1045 0.00684494472323892
1046 0.00684494472323892
1047 0.00684494472323892
1048 0.00684494472323892
1049 0.00684494472323892
1050 0.00684494472323892
1051 0.00684494472323892
1052 0.00684494472323892
1053 0.00684494472323892
1054 0.00684494472323892
1055 0.00684494472323892
1056 0.00684494472323892
1057 0.00684494472323892
1058 0.00684494472323892
1059 0.00684494472323892
1060 0.00684494472323892
1061 0.00684494472323892
1062 0.00684494472323892
1063 0.00684494472323892
1064 0.00684494472323892
1065 0.00684494472323892
1066 0.00684494472323892
1067 0.00684494472323892
1068 0.00684494472323892
1069 0.00684494472323892
1070 0.00684494472323892
1071 0.00684494472323892
1072 0.00684494472323892
1073 0.00684494472323892
1074 0.00684494472323892
1075 0.00684494472323892
1076 0.00684494472323892
1077 0.00684494472323892
1078 0.00684494472323892
1079 0.00684494472323892
1080 0.00684494472323892
1081 0.00684494472323892
1082 0.00684494472323892
1083 0.00684494472323892
1084 0.00684494472323892
1085 0.00684494472323892
1086 0.00684494472323892
1087 0.00684494472323892
1088 0.00684494472323892
1089 0.00684494472323892
1090 0.00684494472323892
1091 0.00684494472323892
1092 0.00684494472323892
1093 0.00684494472323892
1094 0.00684494472323892
1095 0.00684494472323892
1096 0.00684494472323892
1097 0.00684494472323892
1098 0.00684494472323892
1099 0.00684494472323892
1100 0.00684494472323892
1101 0.00684494472323892
1102 0.00684494472323892
1103 0.00684494472323892
1104 0.00684494472323892
1105 0.00684494472323892
1106 0.00684494472323892
1107 0.00684494472323892
1108 0.00684494472323892
1109 0.00684494472323892
1110 0.00684494472323892
1111 0.00684494472323892
1112 0.00684494472323892
1113 0.00684494472323892
1114 0.00684494472323892
1115 0.00684494472323892
1116 0.00684494472323892
1117 0.00684494472323892
1118 0.00684494472323892
1119 0.00684494472323892
1120 0.00684494472323892
1121 0.00684494472323892
1122 0.00684494472323892
1123 0.00684494472323892
1124 0.00684494472323892
1125 0.00684494472323892
1126 0.00684494472323892
1127 0.00684494472323892
1128 0.00684494472323892
1129 0.00684494472323892
1130 0.00684494472323892
1131 0.00684494472323892
1132 0.00684494472323892
1133 0.00684494472323892
1134 0.00684494472323892
1135 0.00684494472323892
1136 0.00684494472323892
1137 0.00684494472323892
1138 0.00684494472323892
1139 0.00684494472323892
1140 0.00684494472323892
1141 0.00684494472323892
1142 0.00684494472323892
1143 0.00684494472323892
1144 0.00684494472323892
1145 0.00684494472323892
1146 0.00684494472323892
1147 0.00684494472323892
1148 0.00684494472323892
1149 0.00684494472323892
1150 0.00684494472323892
1151 0.00684494472323892
1152 0.00684494472323892
1153 0.00684494472323892
1154 0.00684494472323892
1155 0.00684494472323892
1156 0.00684494472323892
1157 0.00684494472323892
1158 0.00684494472323892
1159 0.00684494472323892
1160 0.00684494472323892
1161 0.00684494472323892
1162 0.00684494472323892
1163 0.00684494472323892
1164 0.00684494472323892
1165 0.00684494472323892
1166 0.00684494472323892
1167 0.00684494472323892
1168 0.00684494472323892
1169 0.00684494472323892
1170 0.00684494472323892
1171 0.00684494472323892
1172 0.00684494472323892
1173 0.00684494472323892
1174 0.00684494472323892
1175 0.00684494472323892
1176 0.00684494472323892
1177 0.00684494472323892
1178 0.00684494472323892
1179 0.00684494472323892
1180 0.00684494472323892
1181 0.00684494472323892
1182 0.00684494472323892
1183 0.00684494472323892
1184 0.00684494472323892
1185 0.00684494472323892
1186 0.00684494472323892
1187 0.00684494472323892
1188 0.00684494472323892
1189 0.00684494472323892
1190 0.00684494472323892
1191 0.00684494472323892
1192 0.00684494472323892
1193 0.00684494472323892
1194 0.00684494472323892
1195 0.00684494472323892
1196 0.00684494472323892
1197 0.00684494472323892
1198 0.00684494472323892
1199 0.00684494472323892
1200 0.00684494472323892
1201 0.00684494472323892
1202 0.00684494472323892
1203 0.00684494472323892
1204 0.00684494472323892
1205 0.00684494472323892
1206 0.00684494472323892
1207 0.00684494472323892
1208 0.00684494472323892
1209 0.00684494472323892
1210 0.00684494472323892
1211 0.00684494472323892
1212 0.00684494472323892
1213 0.00684494472323892
1214 0.00684494472323892
1215 0.00684494472323892
1216 0.00684494472323892
1217 0.00684494472323892
1218 0.00684494472323892
1219 0.00684494472323892
1220 0.00684494472323892
1221 0.00684494472323892
1222 0.00684494472323892
1223 0.00684494472323892
1224 0.00684494472323892
1225 0.00684494472323892
1226 0.00684494472323892
1227 0.00684494472323892
1228 0.00684494472323892
1229 0.00684494472323892
1230 0.00684494472323892
1231 0.00684494472323892
1232 0.00684494472323892
1233 0.00684494472323892
1234 0.00684494472323892
1235 0.00684494472323892
1236 0.00684494472323892
1237 0.00684494472323892
1238 0.00684494472323892
1239 0.00684494472323892
1240 0.00684494472323892
1241 0.00684494472323892
1242 0.00684494472323892
1243 0.00684494472323892
1244 0.00684494472323892
1245 0.00684494472323892
1246 0.00684494472323892
1247 0.00684494472323892
1248 0.00684494472323892
1249 0.00684494472323892
1250 0.00684494472323892
1251 0.00684494472323892
1252 0.00684494472323892
1253 0.00684494472323892
1254 0.00684494472323892
1255 0.00684494472323892
1256 0.00684494472323892
1257 0.00684494472323892
1258 0.00684494472323892
1259 0.00684494472323892
1260 0.00684494472323892
1261 0.00684494472323892
1262 0.00684494472323892
1263 0.00684494472323892
1264 0.00684494472323892
1265 0.00684494472323892
1266 0.00684494472323892
1267 0.00684494472323892
1268 0.00684494472323892
1269 0.00684494472323892
1270 0.00684494472323892
1271 0.00684494472323892
1272 0.00684494472323892
1273 0.00684494472323892
1274 0.00684494472323892
1275 0.00684494472323892
1276 0.00684494472323892
1277 0.00684494472323892
1278 0.00684494472323892
1279 0.00684494472323892
1280 0.00684494472323892
1281 0.00684494472323892
1282 0.00684494472323892
1283 0.00684494472323892
1284 0.00684494472323892
1285 0.00684494472323892
1286 0.00684494472323892
1287 0.00684494472323892
1288 0.00684494472323892
1289 0.00684494472323892
1290 0.00684494472323892
1291 0.00684494472323892
1292 0.00684494472323892
1293 0.00684494472323892
1294 0.00684494472323892
1295 0.00684494472323892
1296 0.00684494472323892
1297 0.00684494472323892
1298 0.00684494472323892
1299 0.00684494472323892
1300 0.00684494472323892
1301 0.00684494472323892
1302 0.00684494472323892
1303 0.00684494472323892
1304 0.00684494472323892
1305 0.00684494472323892
1306 0.00684494472323892
1307 0.00684494472323892
1308 0.00684494472323892
1309 0.00684494472323892
1310 0.00684494472323892
1311 0.00684494472323892
1312 0.00684494472323892
1313 0.00684494472323892
1314 0.00684494472323892
1315 0.00684494472323892
1316 0.00684494472323892
1317 0.00684494472323892
1318 0.00684494472323892
1319 0.00684494472323892
1320 0.00684494472323892
1321 0.00684494472323892
1322 0.00684494472323892
1323 0.00684494472323892
1324 0.00684494472323892
1325 0.00684494472323892
1326 0.00684494472323892
1327 0.00684494472323892
1328 0.00684494472323892
1329 0.00684494472323892
1330 0.00684494472323892
1331 0.00684494472323892
1332 0.00684494472323892
1333 0.00684494472323892
1334 0.00684494472323892
1335 0.00684494472323892
1336 0.00684494472323892
1337 0.00684494472323892
1338 0.00684494472323892
1339 0.00684494472323892
1340 0.00684494472323892
1341 0.00684494472323892
1342 0.00684494472323892
1343 0.00684494472323892
1344 0.00684494472323892
1345 0.00684494472323892
1346 0.00684494472323892
1347 0.00684494472323892
1348 0.00684494472323892
1349 0.00684494472323892
1350 0.00684494472323892
1351 0.00684494472323892
1352 0.00684494472323892
1353 0.00684494472323892
1354 0.00684494472323892
1355 0.00684494472323892
1356 0.00684494472323892
1357 0.00684494472323892
1358 0.00684494472323892
1359 0.00684494472323892
1360 0.00684494472323892
1361 0.00684494472323892
1362 0.00684494472323892
1363 0.00684494472323892
1364 0.00684494472323892
1365 0.00684494472323892
1366 0.00684494472323892
1367 0.00684494472323892
1368 0.00684494472323892
1369 0.00684494472323892
1370 0.00684494472323892
1371 0.00684494472323892
1372 0.00684494472323892
1373 0.00684494472323892
1374 0.00684494472323892
1375 0.00684494472323892
1376 0.00684494472323892
1377 0.00684494472323892
1378 0.00684494472323892
1379 0.00684494472323892
1380 0.00684494472323892
1381 0.00684494472323892
1382 0.00684494472323892
1383 0.00684494472323892
1384 0.00684494472323892
1385 0.00684494472323892
1386 0.00684494472323892
1387 0.00684494472323892
1388 0.00684494472323892
1389 0.00684494472323892
1390 0.00684494472323892
1391 0.00684494472323892
1392 0.00684494472323892
1393 0.00684494472323892
1394 0.00684494472323892
1395 0.00684494472323892
1396 0.00684494472323892
1397 0.00684494472323892
1398 0.00684494472323892
1399 0.00684494472323892
1400 0.00684494472323892
1401 0.00684494472323892
1402 0.00684494472323892
1403 0.00684494472323892
1404 0.00684494472323892
1405 0.00684494472323892
1406 0.00684494472323892
1407 0.00684494472323892
1408 0.00684494472323892
1409 0.00684494472323892
1410 0.00684494472323892
1411 0.00684494472323892
1412 0.00684494472323892
1413 0.00684494472323892
1414 0.00684494472323892
1415 0.00684494472323892
1416 0.00684494472323892
1417 0.00684494472323892
1418 0.00684494472323892
1419 0.00684494472323892
1420 0.00684494472323892
1421 0.00684494472323892
1422 0.00684494472323892
1423 0.00684494472323892
1424 0.00684494472323892
1425 0.00684494472323892
1426 0.00684494472323892
1427 0.00684494472323892
1428 0.00684494472323892
1429 0.00684494472323892
1430 0.00684494472323892
1431 0.00684494472323892
1432 0.00684494472323892
1433 0.00684494472323892
1434 0.00684494472323892
1435 0.00684494472323892
1436 0.00684494472323892
1437 0.00684494472323892
1438 0.00684494472323892
1439 0.00684494472323892
1440 0.00684494472323892
1441 0.00684494472323892
1442 0.00684494472323892
1443 0.00684494472323892
1444 0.00684494472323892
1445 0.00684494472323892
1446 0.00684494472323892
1447 0.00684494472323892
1448 0.00684494472323892
1449 0.00684494472323892
1450 0.00684494472323892
1451 0.00684494472323892
1452 0.00684494472323892
1453 0.00684494472323892
1454 0.00684494472323892
1455 0.00684494472323892
1456 0.00684494472323892
1457 0.00684494472323892
1458 0.00684494472323892
1459 0.00684494472323892
1460 0.00684494472323892
1461 0.00684494472323892
1462 0.00684494472323892
1463 0.00684494472323892
1464 0.00684494472323892
1465 0.00684494472323892
1466 0.00684494472323892
1467 0.00684494472323892
1468 0.00684494472323892
1469 0.00684494472323892
1470 0.00684494472323892
1471 0.00684494472323892
1472 0.00684494472323892
1473 0.00684494472323892
1474 0.00684494472323892
1475 0.00684494472323892
1476 0.00684494472323892
1477 0.00684494472323892
1478 0.00684494472323892
1479 0.00684494472323892
1480 0.00684494472323892
1481 0.00684494472323892
1482 0.00684494472323892
1483 0.00684494472323892
1484 0.00684494472323892
1485 0.00684494472323892
1486 0.00684494472323892
1487 0.00684494472323892
1488 0.00684494472323892
1489 0.00684494472323892
1490 0.00684494472323892
1491 0.00684494472323892
1492 0.00684494472323892
1493 0.00684494472323892
1494 0.00684494472323892
1495 0.00684494472323892
1496 0.00684494472323892
1497 0.00684494472323892
1498 0.00684494472323892
1499 0.00684494472323892
1500 0.00684494472323892
1501 0.00684494472323892
1502 0.00684494472323892
1503 0.00684494472323892
1504 0.00684494472323892
1505 0.00684494472323892
1506 0.00684494472323892
1507 0.00684494472323892
1508 0.00684494472323892
1509 0.00684494472323892
1510 0.00684494472323892
1511 0.00684494472323892
1512 0.00684494472323892
1513 0.00684494472323892
1514 0.00684494472323892
1515 0.00684494472323892
1516 0.00684494472323892
1517 0.00684494472323892
1518 0.00684494472323892
1519 0.00684494472323892
1520 0.00684494472323892
1521 0.00684494472323892
1522 0.00684494472323892
1523 0.00684494472323892
1524 0.00684494472323892
1525 0.00684494472323892
1526 0.00684494472323892
1527 0.00684494472323892
1528 0.00684494472323892
1529 0.00684494472323892
1530 0.00684494472323892
1531 0.00684494472323892
1532 0.00684494472323892
1533 0.00684494472323892
1534 0.00684494472323892
1535 0.00684494472323892
1536 0.00684494472323892
1537 0.00684494472323892
1538 0.00684494472323892
1539 0.00684494472323892
1540 0.00684494472323892
1541 0.00684494472323892
1542 0.00684494472323892
1543 0.00684494472323892
1544 0.00684494472323892
1545 0.00684494472323892
1546 0.00684494472323892
1547 0.00684494472323892
1548 0.00684494472323892
1549 0.00684494472323892
1550 0.00684494472323892
1551 0.00684494472323892
1552 0.00684494472323892
1553 0.00684494472323892
1554 0.00684494472323892
1555 0.00684494472323892
1556 0.00684494472323892
1557 0.00684494472323892
1558 0.00684494472323892
1559 0.00684494472323892
1560 0.00684494472323892
1561 0.00684494472323892
1562 0.00684494472323892
1563 0.00684494472323892
1564 0.00684494472323892
1565 0.00684494472323892
1566 0.00684494472323892
1567 0.00684494472323892
1568 0.00684494472323892
1569 0.00684494472323892
1570 0.00684494472323892
1571 0.00684494472323892
1572 0.00684494472323892
1573 0.00684494472323892
1574 0.00684494472323892
1575 0.00684494472323892
1576 0.00684494472323892
1577 0.00684494472323892
1578 0.00684494472323892
1579 0.00684494472323892
1580 0.00684494472323892
1581 0.00684494472323892
1582 0.00684494472323892
1583 0.00684494472323892
1584 0.00684494472323892
1585 0.00684494472323892
1586 0.00684494472323892
1587 0.00684494472323892
1588 0.00684494472323892
1589 0.00684494472323892
1590 0.00684494472323892
1591 0.00684494472323892
1592 0.00684494472323892
1593 0.00684494472323892
1594 0.00684494472323892
1595 0.00684494472323892
1596 0.00684494472323892
1597 0.00684494472323892
1598 0.00684494472323892
1599 0.00684494472323892
1600 0.00684494472323892
1601 0.00684494472323892
1602 0.00684494472323892
1603 0.00684494472323892
1604 0.00684494472323892
1605 0.00684494472323892
1606 0.00684494472323892
1607 0.00684494472323892
1608 0.00684494472323892
1609 0.00684494472323892
1610 0.00684494472323892
1611 0.00684494472323892
1612 0.00684494472323892
1613 0.00684494472323892
1614 0.00684494472323892
1615 0.00684494472323892
1616 0.00684494472323892
1617 0.00684494472323892
1618 0.00684494472323892
1619 0.00684494472323892
1620 0.00684494472323892
1621 0.00684494472323892
1622 0.00684494472323892
1623 0.00684494472323892
1624 0.00684494472323892
1625 0.00684494472323892
1626 0.00684494472323892
1627 0.00684494472323892
1628 0.00684494472323892
1629 0.00684494472323892
1630 0.00684494472323892
1631 0.00684494472323892
1632 0.00684494472323892
1633 0.00684494472323892
1634 0.00684494472323892
1635 0.00684494472323892
1636 0.00684494472323892
1637 0.00684494472323892
1638 0.00684494472323892
1639 0.00684494472323892
1640 0.00684494472323892
1641 0.00684494472323892
1642 0.00684494472323892
1643 0.00684494472323892
1644 0.00684494472323892
1645 0.00684494472323892
1646 0.00684494472323892
1647 0.00684494472323892
1648 0.00684494472323892
1649 0.00684494472323892
1650 0.00684494472323892
1651 0.00684494472323892
1652 0.00684494472323892
1653 0.00684494472323892
1654 0.00684494472323892
1655 0.00684494472323892
1656 0.00684494472323892
1657 0.00684494472323892
1658 0.00684494472323892
1659 0.00684494472323892
1660 0.00684494472323892
1661 0.00684494472323892
1662 0.00684494472323892
1663 0.00684494472323892
1664 0.00684494472323892
1665 0.00684494472323892
1666 0.00684494472323892
1667 0.00684494472323892
1668 0.00684494472323892
1669 0.00684494472323892
1670 0.00684494472323892
1671 0.00684494472323892
1672 0.00684494472323892
1673 0.00684494472323892
1674 0.00684494472323892
1675 0.00684494472323892
1676 0.00684494472323892
1677 0.00684494472323892
1678 0.00684494472323892
1679 0.00684494472323892
1680 0.00684494472323892
1681 0.00684494472323892
1682 0.00684494472323892
1683 0.00684494472323892
1684 0.00684494472323892
1685 0.00684494472323892
1686 0.00684494472323892
1687 0.00684494472323892
1688 0.00684494472323892
1689 0.00684494472323892
1690 0.00684494472323892
1691 0.00684494472323892
1692 0.00684494472323892
1693 0.00684494472323892
1694 0.00684494472323892
1695 0.00684494472323892
1696 0.00684494472323892
1697 0.00684494472323892
1698 0.00684494472323892
1699 0.00684494472323892
1700 0.00684494472323892
1701 0.00684494472323892
1702 0.00684494472323892
1703 0.00684494472323892
1704 0.00684494472323892
1705 0.00684494472323892
1706 0.00684494472323892
1707 0.00684494472323892
1708 0.00684494472323892
1709 0.00684494472323892
1710 0.00684494472323892
1711 0.00684494472323892
1712 0.00684494472323892
1713 0.00684494472323892
1714 0.00684494472323892
1715 0.00684494472323892
1716 0.00684494472323892
1717 0.00684494472323892
1718 0.00684494472323892
1719 0.00684494472323892
1720 0.00684494472323892
1721 0.00684494472323892
1722 0.00684494472323892
1723 0.00684494472323892
1724 0.00684494472323892
1725 0.00684494472323892
1726 0.00684494472323892
1727 0.00684494472323892
1728 0.00684494472323892
1729 0.00684494472323892
1730 0.00684494472323892
1731 0.00684494472323892
1732 0.00684494472323892
1733 0.00684494472323892
1734 0.00684494472323892
1735 0.00684494472323892
1736 0.00684494472323892
1737 0.00684494472323892
1738 0.00684494472323892
1739 0.00684494472323892
1740 0.00684494472323892
1741 0.00684494472323892
1742 0.00684494472323892
1743 0.00684494472323892
1744 0.00684494472323892
1745 0.00684494472323892
1746 0.00684494472323892
1747 0.00684494472323892
1748 0.00684494472323892
1749 0.00684494472323892
1750 0.00684494472323892
1751 0.00684494472323892
1752 0.00684494472323892
1753 0.00684494472323892
1754 0.00684494472323892
1755 0.00684494472323892
1756 0.00684494472323892
1757 0.00684494472323892
1758 0.00684494472323892
1759 0.00684494472323892
1760 0.00684494472323892
1761 0.00684494472323892
1762 0.00684494472323892
1763 0.00684494472323892
1764 0.00684494472323892
1765 0.00684494472323892
1766 0.00684494472323892
1767 0.00684494472323892
1768 0.00684494472323892
1769 0.00684494472323892
1770 0.00684494472323892
1771 0.00684494472323892
1772 0.00684494472323892
1773 0.00684494472323892
1774 0.00684494472323892
1775 0.00684494472323892
1776 0.00684494472323892
1777 0.00684494472323892
1778 0.00684494472323892
1779 0.00684494472323892
1780 0.00684494472323892
1781 0.00684494472323892
1782 0.00684494472323892
1783 0.00684494472323892
1784 0.00684494472323892
1785 0.00684494472323892
1786 0.00684494472323892
1787 0.00684494472323892
1788 0.00684494472323892
1789 0.00684494472323892
1790 0.00684494472323892
1791 0.00684494472323892
1792 0.00684494472323892
1793 0.00684494472323892
1794 0.00684494472323892
1795 0.00684494472323892
1796 0.00684494472323892
1797 0.00684494472323892
1798 0.00684494472323892
1799 0.00684494472323892
1800 0.00684494472323892
1801 0.00684494472323892
1802 0.00684494472323892
1803 0.00684494472323892
1804 0.00684494472323892
1805 0.00684494472323892
1806 0.00684494472323892
1807 0.00684494472323892
1808 0.00684494472323892
1809 0.00684494472323892
1810 0.00684494472323892
1811 0.00684494472323892
1812 0.00684494472323892
1813 0.00684494472323892
1814 0.00684494472323892
1815 0.00684494472323892
1816 0.00684494472323892
1817 0.00684494472323892
1818 0.00684494472323892
1819 0.00684494472323892
1820 0.00684494472323892
1821 0.00684494472323892
1822 0.00684494472323892
1823 0.00684494472323892
1824 0.00684494472323892
1825 0.00684494472323892
1826 0.00684494472323892
1827 0.00684494472323892
1828 0.00684494472323892
1829 0.00684494472323892
1830 0.00684494472323892
1831 0.00684494472323892
1832 0.00684494472323892
1833 0.00684494472323892
1834 0.00684494472323892
1835 0.00684494472323892
1836 0.00684494472323892
1837 0.00684494472323892
1838 0.00684494472323892
1839 0.00684494472323892
1840 0.00684494472323892
1841 0.00684494472323892
1842 0.00684494472323892
1843 0.00684494472323892
1844 0.00684494472323892
1845 0.00684494472323892
1846 0.00684494472323892
1847 0.00684494472323892
1848 0.00684494472323892
1849 0.00684494472323892
1850 0.00684494472323892
1851 0.00684494472323892
1852 0.00684494472323892
1853 0.00684494472323892
1854 0.00684494472323892
1855 0.00684494472323892
1856 0.00684494472323892
1857 0.00684494472323892
1858 0.00684494472323892
1859 0.00684494472323892
1860 0.00684494472323892
1861 0.00684494472323892
1862 0.00684494472323892
1863 0.00684494472323892
1864 0.00684494472323892
1865 0.00684494472323892
1866 0.00684494472323892
1867 0.00684494472323892
1868 0.00684494472323892
1869 0.00684494472323892
1870 0.00684494472323892
1871 0.00684494472323892
1872 0.00684494472323892
1873 0.00684494472323892
1874 0.00684494472323892
1875 0.00684494472323892
1876 0.00684494472323892
1877 0.00684494472323892
1878 0.00684494472323892
1879 0.00684494472323892
1880 0.00684494472323892
1881 0.00684494472323892
1882 0.00684494472323892
1883 0.00684494472323892
1884 0.00684494472323892
1885 0.00684494472323892
1886 0.00684494472323892
1887 0.00684494472323892
1888 0.00684494472323892
1889 0.00684494472323892
1890 0.00684494472323892
1891 0.00684494472323892
1892 0.00684494472323892
1893 0.00684494472323892
1894 0.00684494472323892
1895 0.00684494472323892
1896 0.00684494472323892
1897 0.00684494472323892
1898 0.00684494472323892
1899 0.00684494472323892
1900 0.00684494472323892
1901 0.00684494472323892
1902 0.00684494472323892
1903 0.00684494472323892
1904 0.00684494472323892
1905 0.00684494472323892
1906 0.00684494472323892
1907 0.00684494472323892
1908 0.00684494472323892
1909 0.00684494472323892
1910 0.00684494472323892
1911 0.00684494472323892
1912 0.00684494472323892
1913 0.00684494472323892
1914 0.00684494472323892
1915 0.00684494472323892
1916 0.00684494472323892
1917 0.00684494472323892
1918 0.00684494472323892
1919 0.00684494472323892
1920 0.00684494472323892
1921 0.00684494472323892
1922 0.00684494472323892
1923 0.00684494472323892
1924 0.00684494472323892
1925 0.00684494472323892
1926 0.00684494472323892
1927 0.00684494472323892
1928 0.00684494472323892
1929 0.00684494472323892
1930 0.00684494472323892
1931 0.00684494472323892
1932 0.00684494472323892
1933 0.00684494472323892
1934 0.00684494472323892
1935 0.00684494472323892
1936 0.00684494472323892
1937 0.00684494472323892
1938 0.00684494472323892
1939 0.00684494472323892
1940 0.00684494472323892
1941 0.00684494472323892
1942 0.00684494472323892
1943 0.00684494472323892
1944 0.00684494472323892
1945 0.00684494472323892
1946 0.00684494472323892
1947 0.00684494472323892
1948 0.00684494472323892
1949 0.00684494472323892
1950 0.00684494472323892
1951 0.00684494472323892
1952 0.00684494472323892
1953 0.00684494472323892
1954 0.00684494472323892
1955 0.00684494472323892
1956 0.00684494472323892
1957 0.00684494472323892
1958 0.00684494472323892
1959 0.00684494472323892
1960 0.00684494472323892
1961 0.00684494472323892
1962 0.00684494472323892
1963 0.00684494472323892
1964 0.00684494472323892
1965 0.00684494472323892
1966 0.00684494472323892
1967 0.00684494472323892
1968 0.00684494472323892
1969 0.00684494472323892
1970 0.00684494472323892
1971 0.00684494472323892
1972 0.00684494472323892
1973 0.00684494472323892
1974 0.00684494472323892
1975 0.00684494472323892
1976 0.00684494472323892
1977 0.00684494472323892
1978 0.00684494472323892
1979 0.00684494472323892
1980 0.00684494472323892
1981 0.00684494472323892
1982 0.00684494472323892
1983 0.00684494472323892
1984 0.00684494472323892
1985 0.00684494472323892
1986 0.00684494472323892
1987 0.00684494472323892
1988 0.00684494472323892
1989 0.00684494472323892
1990 0.00684494472323892
1991 0.00684494472323892
1992 0.00684494472323892
1993 0.00684494472323892
1994 0.00684494472323892
1995 0.00684494472323892
1996 0.00684494472323892
1997 0.00684494472323892
1998 0.00684494472323892
1999 0.00684494472323892
};
\addlegendentry{Train}
\addplot [semithick, black]
table {%
0 0.00761853903532028
1 0.00761853903532028
2 0.00761853903532028
3 0.00761853903532028
4 0.00761853903532028
5 0.00761853903532028
6 0.00761853903532028
7 0.00761853903532028
8 0.00761853903532028
9 0.00761853903532028
10 0.00761853903532028
11 0.00761853903532028
12 0.00761853903532028
13 0.00761853903532028
14 0.00761853903532028
15 0.00761853903532028
16 0.00761853903532028
17 0.00761853903532028
18 0.00761853903532028
19 0.00761853903532028
20 0.00761853903532028
21 0.00761853903532028
22 0.00761853903532028
23 0.00761853903532028
24 0.00761853903532028
25 0.00761853903532028
26 0.00761853903532028
27 0.00761853903532028
28 0.00761853903532028
29 0.00761853903532028
30 0.00761853903532028
31 0.00761853903532028
32 0.00761853903532028
33 0.00761853903532028
34 0.00761853903532028
35 0.00761853903532028
36 0.00761853903532028
37 0.00761853903532028
38 0.00761853903532028
39 0.00761853903532028
40 0.00761853903532028
41 0.00761853903532028
42 0.00761853903532028
43 0.00761853903532028
44 0.00761853903532028
45 0.00761853903532028
46 0.00761853903532028
47 0.00761853903532028
48 0.00761853903532028
49 0.00761853903532028
50 0.00761853903532028
51 0.00761853903532028
52 0.00761853903532028
53 0.00761853903532028
54 0.00761853903532028
55 0.00761853903532028
56 0.00761853903532028
57 0.00761853903532028
58 0.00761853903532028
59 0.00761853903532028
60 0.00761853903532028
61 0.00761853903532028
62 0.00761853903532028
63 0.00761853903532028
64 0.00761853903532028
65 0.00761853903532028
66 0.00761853903532028
67 0.00761853903532028
68 0.00761853903532028
69 0.00761853903532028
70 0.00761853903532028
71 0.00761853903532028
72 0.00761853903532028
73 0.00761853903532028
74 0.00761853903532028
75 0.00761853903532028
76 0.00761853903532028
77 0.00761853903532028
78 0.00761853903532028
79 0.00761853903532028
80 0.00761853903532028
81 0.00761853903532028
82 0.00761853903532028
83 0.00761853903532028
84 0.00761853903532028
85 0.00761853903532028
86 0.00761853903532028
87 0.00761853903532028
88 0.00761853903532028
89 0.00761853903532028
90 0.00761853903532028
91 0.00761853903532028
92 0.00761853903532028
93 0.00761853903532028
94 0.00761853903532028
95 0.00761853903532028
96 0.00761853903532028
97 0.00761853903532028
98 0.00761853903532028
99 0.00761853903532028
100 0.00761853903532028
101 0.00761853903532028
102 0.00761853903532028
103 0.00761853903532028
104 0.00761853903532028
105 0.00761853903532028
106 0.00761853903532028
107 0.00761853903532028
108 0.00761853903532028
109 0.00761853903532028
110 0.00761853903532028
111 0.00761853903532028
112 0.00761853903532028
113 0.00761853903532028
114 0.00761853903532028
115 0.00761853903532028
116 0.00761853903532028
117 0.00761853903532028
118 0.00761853903532028
119 0.00761853903532028
120 0.00761853903532028
121 0.00761853903532028
122 0.00761853903532028
123 0.00761853903532028
124 0.00761853903532028
125 0.00761853903532028
126 0.00761853903532028
127 0.00761853903532028
128 0.00761853903532028
129 0.00761853903532028
130 0.00761853903532028
131 0.00761853903532028
132 0.00761853903532028
133 0.00761853903532028
134 0.00761853903532028
135 0.00761853903532028
136 0.00761853903532028
137 0.00761853903532028
138 0.00761853903532028
139 0.00761853903532028
140 0.00761853903532028
141 0.00761853903532028
142 0.00761853903532028
143 0.00761853903532028
144 0.00761853903532028
145 0.00761853903532028
146 0.00761853903532028
147 0.00761853903532028
148 0.00761853903532028
149 0.00761853903532028
150 0.00761853903532028
151 0.00761853903532028
152 0.00761853903532028
153 0.00761853903532028
154 0.00761853903532028
155 0.00761853903532028
156 0.00761853903532028
157 0.00761853903532028
158 0.00761853903532028
159 0.00761853903532028
160 0.00761853903532028
161 0.00761853903532028
162 0.00761853903532028
163 0.00761853903532028
164 0.00761853903532028
165 0.00761853903532028
166 0.00761853903532028
167 0.00761853903532028
168 0.00761853903532028
169 0.00761853903532028
170 0.00761853903532028
171 0.00761853903532028
172 0.00761853903532028
173 0.00761853903532028
174 0.00761853903532028
175 0.00761853903532028
176 0.00761853903532028
177 0.00761853903532028
178 0.00761853903532028
179 0.00761853903532028
180 0.00761853903532028
181 0.00761853903532028
182 0.00761853903532028
183 0.00761853903532028
184 0.00761853903532028
185 0.00761853903532028
186 0.00761853903532028
187 0.00761853903532028
188 0.00761853903532028
189 0.00761853903532028
190 0.00761853903532028
191 0.00761853903532028
192 0.00761853903532028
193 0.00761853903532028
194 0.00761853903532028
195 0.00761853903532028
196 0.00761853903532028
197 0.00761853903532028
198 0.00761853903532028
199 0.00761853903532028
200 0.00761853903532028
201 0.00761853903532028
202 0.00761853903532028
203 0.00761853903532028
204 0.00761853903532028
205 0.00761853903532028
206 0.00761853903532028
207 0.00761853903532028
208 0.00761853903532028
209 0.00761853903532028
210 0.00761853903532028
211 0.00761853903532028
212 0.00761853903532028
213 0.00761853903532028
214 0.00761853903532028
215 0.00761853903532028
216 0.00761853903532028
217 0.00761853903532028
218 0.00761853903532028
219 0.00761853903532028
220 0.00761853903532028
221 0.00761853903532028
222 0.00761853903532028
223 0.00761853903532028
224 0.00761853903532028
225 0.00761853903532028
226 0.00761853903532028
227 0.00761853903532028
228 0.00761853903532028
229 0.00761853903532028
230 0.00761853903532028
231 0.00761853903532028
232 0.00761853903532028
233 0.00761853903532028
234 0.00761853903532028
235 0.00761853903532028
236 0.00761853903532028
237 0.00761853903532028
238 0.00761853903532028
239 0.00761853903532028
240 0.00761853903532028
241 0.00761853903532028
242 0.00761853903532028
243 0.00761853903532028
244 0.00761853903532028
245 0.00761853903532028
246 0.00761853903532028
247 0.00761853903532028
248 0.00761853903532028
249 0.00761853903532028
250 0.00761853903532028
251 0.00761853903532028
252 0.00761853903532028
253 0.00761853903532028
254 0.00761853903532028
255 0.00761853903532028
256 0.00761853903532028
257 0.00761853903532028
258 0.00761853903532028
259 0.00761853903532028
260 0.00761853903532028
261 0.00761853903532028
262 0.00761853903532028
263 0.00761853903532028
264 0.00761853903532028
265 0.00761853903532028
266 0.00761853903532028
267 0.00761853903532028
268 0.00761853903532028
269 0.00761853903532028
270 0.00761853903532028
271 0.00761853903532028
272 0.00761853903532028
273 0.00761853903532028
274 0.00761853903532028
275 0.00761853903532028
276 0.00761853903532028
277 0.00761853903532028
278 0.00761853903532028
279 0.00761853903532028
280 0.00761853903532028
281 0.00761853903532028
282 0.00761853903532028
283 0.00761853903532028
284 0.00761853903532028
285 0.00761853903532028
286 0.00761853903532028
287 0.00761853903532028
288 0.00761853903532028
289 0.00761853903532028
290 0.00761853903532028
291 0.00761853903532028
292 0.00761853903532028
293 0.00761853903532028
294 0.00761853903532028
295 0.00761853903532028
296 0.00761853903532028
297 0.00761853903532028
298 0.00761853903532028
299 0.00761853903532028
300 0.00761853903532028
301 0.00761853903532028
302 0.00761853903532028
303 0.00761853903532028
304 0.00761853903532028
305 0.00761853903532028
306 0.00761853903532028
307 0.00761853903532028
308 0.00761853903532028
309 0.00761853903532028
310 0.00761853903532028
311 0.00761853903532028
312 0.00761853903532028
313 0.00761853903532028
314 0.00761853903532028
315 0.00761853903532028
316 0.00761853903532028
317 0.00761853903532028
318 0.00761853903532028
319 0.00761853903532028
320 0.00761853903532028
321 0.00761853903532028
322 0.00761853903532028
323 0.00761853903532028
324 0.00761853903532028
325 0.00761853903532028
326 0.00761853903532028
327 0.00761853903532028
328 0.00761853903532028
329 0.00761853903532028
330 0.00761853903532028
331 0.00761853903532028
332 0.00761853903532028
333 0.00761853903532028
334 0.00761853903532028
335 0.00761853903532028
336 0.00761853903532028
337 0.00761853903532028
338 0.00761853903532028
339 0.00761853903532028
340 0.00761853903532028
341 0.00761853903532028
342 0.00761853903532028
343 0.00761853903532028
344 0.00761853903532028
345 0.00761853903532028
346 0.00761853903532028
347 0.00761853903532028
348 0.00761853903532028
349 0.00761853903532028
350 0.00761853903532028
351 0.00761853903532028
352 0.00761853903532028
353 0.00761853903532028
354 0.00761853903532028
355 0.00761853903532028
356 0.00761853903532028
357 0.00761853903532028
358 0.00761853903532028
359 0.00761853903532028
360 0.00761853903532028
361 0.00761853903532028
362 0.00761853903532028
363 0.00761853903532028
364 0.00761853903532028
365 0.00761853903532028
366 0.00761853903532028
367 0.00761853903532028
368 0.00761853903532028
369 0.00761853903532028
370 0.00761853903532028
371 0.00761853903532028
372 0.00761853903532028
373 0.00761853903532028
374 0.00761853903532028
375 0.00761853903532028
376 0.00761853903532028
377 0.00761853903532028
378 0.00761853903532028
379 0.00761853903532028
380 0.00761853903532028
381 0.00761853903532028
382 0.00761853903532028
383 0.00761853903532028
384 0.00761853903532028
385 0.00761853903532028
386 0.00761853903532028
387 0.00761853903532028
388 0.00761853903532028
389 0.00761853903532028
390 0.00761853903532028
391 0.00761853903532028
392 0.00761853903532028
393 0.00761853903532028
394 0.00761853903532028
395 0.00761853903532028
396 0.00761853903532028
397 0.00761853903532028
398 0.00761853903532028
399 0.00761853903532028
400 0.00761853903532028
401 0.00761853903532028
402 0.00761853903532028
403 0.00761853903532028
404 0.00761853903532028
405 0.00761853903532028
406 0.00761853903532028
407 0.00761853903532028
408 0.00761853903532028
409 0.00761853903532028
410 0.00761853903532028
411 0.00761853903532028
412 0.00761853903532028
413 0.00761853903532028
414 0.00761853903532028
415 0.00761853903532028
416 0.00761853903532028
417 0.00761853903532028
418 0.00761853903532028
419 0.00761853903532028
420 0.00761853903532028
421 0.00761853903532028
422 0.00761853903532028
423 0.00761853903532028
424 0.00761853903532028
425 0.00761853903532028
426 0.00761853903532028
427 0.00761853903532028
428 0.00761853903532028
429 0.00761853903532028
430 0.00761853903532028
431 0.00761853903532028
432 0.00761853903532028
433 0.00761853903532028
434 0.00761853903532028
435 0.00761853903532028
436 0.00761853903532028
437 0.00761853903532028
438 0.00761853903532028
439 0.00761853903532028
440 0.00761853903532028
441 0.00761853903532028
442 0.00761853903532028
443 0.00761853903532028
444 0.00761853903532028
445 0.00761853903532028
446 0.00761853903532028
447 0.00761853903532028
448 0.00761853903532028
449 0.00761853903532028
450 0.00761853903532028
451 0.00761853903532028
452 0.00761853903532028
453 0.00761853903532028
454 0.00761853903532028
455 0.00761853903532028
456 0.00761853903532028
457 0.00761853903532028
458 0.00761853903532028
459 0.00761853903532028
460 0.00761853903532028
461 0.00761853903532028
462 0.00761853903532028
463 0.00761853903532028
464 0.00761853903532028
465 0.00761853903532028
466 0.00761853903532028
467 0.00761853903532028
468 0.00761853903532028
469 0.00761853903532028
470 0.00761853903532028
471 0.00761853903532028
472 0.00761853903532028
473 0.00761853903532028
474 0.00761853903532028
475 0.00761853903532028
476 0.00761853903532028
477 0.00761853903532028
478 0.00761853903532028
479 0.00761853903532028
480 0.00761853903532028
481 0.00761853903532028
482 0.00761853903532028
483 0.00761853903532028
484 0.00761853903532028
485 0.00761853903532028
486 0.00761853903532028
487 0.00761853903532028
488 0.00761853903532028
489 0.00761853903532028
490 0.00761853903532028
491 0.00761853903532028
492 0.00761853903532028
493 0.00761853903532028
494 0.00761853903532028
495 0.00761853903532028
496 0.00761853903532028
497 0.00761853903532028
498 0.00761853903532028
499 0.00761853903532028
500 0.00761853903532028
501 0.00761853903532028
502 0.00761853903532028
503 0.00761853903532028
504 0.00761853903532028
505 0.00761853903532028
506 0.00761853903532028
507 0.00761853903532028
508 0.00761853903532028
509 0.00761853903532028
510 0.00761853903532028
511 0.00761853903532028
512 0.00761853903532028
513 0.00761853903532028
514 0.00761853903532028
515 0.00761853903532028
516 0.00761853903532028
517 0.00761853903532028
518 0.00761853903532028
519 0.00761853903532028
520 0.00761853903532028
521 0.00761853903532028
522 0.00761853903532028
523 0.00761853903532028
524 0.00761853903532028
525 0.00761853903532028
526 0.00761853903532028
527 0.00761853903532028
528 0.00761853903532028
529 0.00761853903532028
530 0.00761853903532028
531 0.00761853903532028
532 0.00761853903532028
533 0.00761853903532028
534 0.00761853903532028
535 0.00761853903532028
536 0.00761853903532028
537 0.00761853903532028
538 0.00761853903532028
539 0.00761853903532028
540 0.00761853903532028
541 0.00761853903532028
542 0.00761853903532028
543 0.00761853903532028
544 0.00761853903532028
545 0.00761853903532028
546 0.00761853903532028
547 0.00761853903532028
548 0.00761853903532028
549 0.00761853903532028
550 0.00761853903532028
551 0.00761853903532028
552 0.00761853903532028
553 0.00761853903532028
554 0.00761853903532028
555 0.00761853903532028
556 0.00761853903532028
557 0.00761853903532028
558 0.00761853903532028
559 0.00761853903532028
560 0.00761853903532028
561 0.00761853903532028
562 0.00761853903532028
563 0.00761853903532028
564 0.00761853903532028
565 0.00761853903532028
566 0.00761853903532028
567 0.00761853903532028
568 0.00761853903532028
569 0.00761853903532028
570 0.00761853903532028
571 0.00761853903532028
572 0.00761853903532028
573 0.00761853903532028
574 0.00761853903532028
575 0.00761853903532028
576 0.00761853903532028
577 0.00761853903532028
578 0.00761853903532028
579 0.00761853903532028
580 0.00761853903532028
581 0.00761853903532028
582 0.00761853903532028
583 0.00761853903532028
584 0.00761853903532028
585 0.00761853903532028
586 0.00761853903532028
587 0.00761853903532028
588 0.00761853903532028
589 0.00761853903532028
590 0.00761853903532028
591 0.00761853903532028
592 0.00761853903532028
593 0.00761853903532028
594 0.00761853903532028
595 0.00761853903532028
596 0.00761853903532028
597 0.00761853903532028
598 0.00761853903532028
599 0.00761853903532028
600 0.00761853903532028
601 0.00761853903532028
602 0.00761853903532028
603 0.00761853903532028
604 0.00761853903532028
605 0.00761853903532028
606 0.00761853903532028
607 0.00761853903532028
608 0.00761853903532028
609 0.00761853903532028
610 0.00761853903532028
611 0.00761853903532028
612 0.00761853903532028
613 0.00761853903532028
614 0.00761853903532028
615 0.00761853903532028
616 0.00761853903532028
617 0.00761853903532028
618 0.00761853903532028
619 0.00761853903532028
620 0.00761853903532028
621 0.00761853903532028
622 0.00761853903532028
623 0.00761853903532028
624 0.00761853903532028
625 0.00761853903532028
626 0.00761853903532028
627 0.00761853903532028
628 0.00761853903532028
629 0.00761853903532028
630 0.00761853903532028
631 0.00761853903532028
632 0.00761853903532028
633 0.00761853903532028
634 0.00761853903532028
635 0.00761853903532028
636 0.00761853903532028
637 0.00761853903532028
638 0.00761853903532028
639 0.00761853903532028
640 0.00761853903532028
641 0.00761853903532028
642 0.00761853903532028
643 0.00761853903532028
644 0.00761853903532028
645 0.00761853903532028
646 0.00761853903532028
647 0.00761853903532028
648 0.00761853903532028
649 0.00761853903532028
650 0.00761853903532028
651 0.00761853903532028
652 0.00761853903532028
653 0.00761853903532028
654 0.00761853903532028
655 0.00761853903532028
656 0.00761853903532028
657 0.00761853903532028
658 0.00761853903532028
659 0.00761853903532028
660 0.00761853903532028
661 0.00761853903532028
662 0.00761853903532028
663 0.00761853903532028
664 0.00761853903532028
665 0.00761853903532028
666 0.00761853903532028
667 0.00761853903532028
668 0.00761853903532028
669 0.00761853903532028
670 0.00761853903532028
671 0.00761853903532028
672 0.00761853903532028
673 0.00761853903532028
674 0.00761853903532028
675 0.00761853903532028
676 0.00761853903532028
677 0.00761853903532028
678 0.00761853903532028
679 0.00761853903532028
680 0.00761853903532028
681 0.00761853903532028
682 0.00761853903532028
683 0.00761853903532028
684 0.00761853903532028
685 0.00761853903532028
686 0.00761853903532028
687 0.00761853903532028
688 0.00761853903532028
689 0.00761853903532028
690 0.00761853903532028
691 0.00761853903532028
692 0.00761853903532028
693 0.00761853903532028
694 0.00761853903532028
695 0.00761853903532028
696 0.00761853903532028
697 0.00761853903532028
698 0.00761853903532028
699 0.00761853903532028
700 0.00761853903532028
701 0.00761853903532028
702 0.00761853903532028
703 0.00761853903532028
704 0.00761853903532028
705 0.00761853903532028
706 0.00761853903532028
707 0.00761853903532028
708 0.00761853903532028
709 0.00761853903532028
710 0.00761853903532028
711 0.00761853903532028
712 0.00761853903532028
713 0.00761853903532028
714 0.00761853903532028
715 0.00761853903532028
716 0.00761853903532028
717 0.00761853903532028
718 0.00761853903532028
719 0.00761853903532028
720 0.00761853903532028
721 0.00761853903532028
722 0.00761853903532028
723 0.00761853903532028
724 0.00761853903532028
725 0.00761853903532028
726 0.00761853903532028
727 0.00761853903532028
728 0.00761853903532028
729 0.00761853903532028
730 0.00761853903532028
731 0.00761853903532028
732 0.00761853903532028
733 0.00761853903532028
734 0.00761853903532028
735 0.00761853903532028
736 0.00761853903532028
737 0.00761853903532028
738 0.00761853903532028
739 0.00761853903532028
740 0.00761853903532028
741 0.00761853903532028
742 0.00761853903532028
743 0.00761853903532028
744 0.00761853903532028
745 0.00761853903532028
746 0.00761853903532028
747 0.00761853903532028
748 0.00761853903532028
749 0.00761853903532028
750 0.00761853903532028
751 0.00761853903532028
752 0.00761853903532028
753 0.00761853903532028
754 0.00761853903532028
755 0.00761853903532028
756 0.00761853903532028
757 0.00761853903532028
758 0.00761853903532028
759 0.00761853903532028
760 0.00761853903532028
761 0.00761853903532028
762 0.00761853903532028
763 0.00761853903532028
764 0.00761853903532028
765 0.00761853903532028
766 0.00761853903532028
767 0.00761853903532028
768 0.00761853903532028
769 0.00761853903532028
770 0.00761853903532028
771 0.00761853903532028
772 0.00761853903532028
773 0.00761853903532028
774 0.00761853903532028
775 0.00761853903532028
776 0.00761853903532028
777 0.00761853903532028
778 0.00761853903532028
779 0.00761853903532028
780 0.00761853903532028
781 0.00761853903532028
782 0.00761853903532028
783 0.00761853903532028
784 0.00761853903532028
785 0.00761853903532028
786 0.00761853903532028
787 0.00761853903532028
788 0.00761853903532028
789 0.00761853903532028
790 0.00761853903532028
791 0.00761853903532028
792 0.00761853903532028
793 0.00761853903532028
794 0.00761853903532028
795 0.00761853903532028
796 0.00761853903532028
797 0.00761853903532028
798 0.00761853903532028
799 0.00761853903532028
800 0.00761853903532028
801 0.00761853903532028
802 0.00761853903532028
803 0.00761853903532028
804 0.00761853903532028
805 0.00761853903532028
806 0.00761853903532028
807 0.00761853903532028
808 0.00761853903532028
809 0.00761853903532028
810 0.00761853903532028
811 0.00761853903532028
812 0.00761853903532028
813 0.00761853903532028
814 0.00761853903532028
815 0.00761853903532028
816 0.00761853903532028
817 0.00761853903532028
818 0.00761853903532028
819 0.00761853903532028
820 0.00761853903532028
821 0.00761853903532028
822 0.00761853903532028
823 0.00761853903532028
824 0.00761853903532028
825 0.00761853903532028
826 0.00761853903532028
827 0.00761853903532028
828 0.00761853903532028
829 0.00761853903532028
830 0.00761853903532028
831 0.00761853903532028
832 0.00761853903532028
833 0.00761853903532028
834 0.00761853903532028
835 0.00761853903532028
836 0.00761853903532028
837 0.00761853903532028
838 0.00761853903532028
839 0.00761853903532028
840 0.00761853903532028
841 0.00761853903532028
842 0.00761853903532028
843 0.00761853903532028
844 0.00761853903532028
845 0.00761853903532028
846 0.00761853903532028
847 0.00761853903532028
848 0.00761853903532028
849 0.00761853903532028
850 0.00761853903532028
851 0.00761853903532028
852 0.00761853903532028
853 0.00761853903532028
854 0.00761853903532028
855 0.00761853903532028
856 0.00761853903532028
857 0.00761853903532028
858 0.00761853903532028
859 0.00761853903532028
860 0.00761853903532028
861 0.00761853903532028
862 0.00761853903532028
863 0.00761853903532028
864 0.00761853903532028
865 0.00761853903532028
866 0.00761853903532028
867 0.00761853903532028
868 0.00761853903532028
869 0.00761853903532028
870 0.00761853903532028
871 0.00761853903532028
872 0.00761853903532028
873 0.00761853903532028
874 0.00761853903532028
875 0.00761853903532028
876 0.00761853903532028
877 0.00761853903532028
878 0.00761853903532028
879 0.00761853903532028
880 0.00761853903532028
881 0.00761853903532028
882 0.00761853903532028
883 0.00761853903532028
884 0.00761853903532028
885 0.00761853903532028
886 0.00761853903532028
887 0.00761853903532028
888 0.00761853903532028
889 0.00761853903532028
890 0.00761853903532028
891 0.00761853903532028
892 0.00761853903532028
893 0.00761853903532028
894 0.00761853903532028
895 0.00761853903532028
896 0.00761853903532028
897 0.00761853903532028
898 0.00761853903532028
899 0.00761853903532028
900 0.00761853903532028
901 0.00761853903532028
902 0.00761853903532028
903 0.00761853903532028
904 0.00761853903532028
905 0.00761853903532028
906 0.00761853903532028
907 0.00761853903532028
908 0.00761853903532028
909 0.00761853903532028
910 0.00761853903532028
911 0.00761853903532028
912 0.00761853903532028
913 0.00761853903532028
914 0.00761853903532028
915 0.00761853903532028
916 0.00761853903532028
917 0.00761853903532028
918 0.00761853903532028
919 0.00761853903532028
920 0.00761853903532028
921 0.00761853903532028
922 0.00761853903532028
923 0.00761853903532028
924 0.00761853903532028
925 0.00761853903532028
926 0.00761853903532028
927 0.00761853903532028
928 0.00761853903532028
929 0.00761853903532028
930 0.00761853903532028
931 0.00761853903532028
932 0.00761853903532028
933 0.00761853903532028
934 0.00761853903532028
935 0.00761853903532028
936 0.00761853903532028
937 0.00761853903532028
938 0.00761853903532028
939 0.00761853903532028
940 0.00761853903532028
941 0.00761853903532028
942 0.00761853903532028
943 0.00761853903532028
944 0.00761853903532028
945 0.00761853903532028
946 0.00761853903532028
947 0.00761853903532028
948 0.00761853903532028
949 0.00761853903532028
950 0.00761853903532028
951 0.00761853903532028
952 0.00761853903532028
953 0.00761853903532028
954 0.00761853903532028
955 0.00761853903532028
956 0.00761853903532028
957 0.00761853903532028
958 0.00761853903532028
959 0.00761853903532028
960 0.00761853903532028
961 0.00761853903532028
962 0.00761853903532028
963 0.00761853903532028
964 0.00761853903532028
965 0.00761853903532028
966 0.00761853903532028
967 0.00761853903532028
968 0.00761853903532028
969 0.00761853903532028
970 0.00761853903532028
971 0.00761853903532028
972 0.00761853903532028
973 0.00761853903532028
974 0.00761853903532028
975 0.00761853903532028
976 0.00761853903532028
977 0.00761853903532028
978 0.00761853903532028
979 0.00761853903532028
980 0.00761853903532028
981 0.00761853903532028
982 0.00761853903532028
983 0.00761853903532028
984 0.00761853903532028
985 0.00761853903532028
986 0.00761853903532028
987 0.00761853903532028
988 0.00761853903532028
989 0.00761853903532028
990 0.00761853903532028
991 0.00761853903532028
992 0.00761853903532028
993 0.00761853903532028
994 0.00761853903532028
995 0.00761853903532028
996 0.00761853903532028
997 0.00761853903532028
998 0.00761853903532028
999 0.00761853903532028
1000 0.00761853903532028
1001 0.00761853903532028
1002 0.00761853903532028
1003 0.00761853903532028
1004 0.00761853903532028
1005 0.00761853903532028
1006 0.00761853903532028
1007 0.00761853903532028
1008 0.00761853903532028
1009 0.00761853903532028
1010 0.00761853903532028
1011 0.00761853903532028
1012 0.00761853903532028
1013 0.00761853903532028
1014 0.00761853903532028
1015 0.00761853903532028
1016 0.00761853903532028
1017 0.00761853903532028
1018 0.00761853903532028
1019 0.00761853903532028
1020 0.00761853903532028
1021 0.00761853903532028
1022 0.00761853903532028
1023 0.00761853903532028
1024 0.00761853903532028
1025 0.00761853903532028
1026 0.00761853903532028
1027 0.00761853903532028
1028 0.00761853903532028
1029 0.00761853903532028
1030 0.00761853903532028
1031 0.00761853903532028
1032 0.00761853903532028
1033 0.00761853903532028
1034 0.00761853903532028
1035 0.00761853903532028
1036 0.00761853903532028
1037 0.00761853903532028
1038 0.00761853903532028
1039 0.00761853903532028
1040 0.00761853903532028
1041 0.00761853903532028
1042 0.00761853903532028
1043 0.00761853903532028
1044 0.00761853903532028
1045 0.00761853903532028
1046 0.00761853903532028
1047 0.00761853903532028
1048 0.00761853903532028
1049 0.00761853903532028
1050 0.00761853903532028
1051 0.00761853903532028
1052 0.00761853903532028
1053 0.00761853903532028
1054 0.00761853903532028
1055 0.00761853903532028
1056 0.00761853903532028
1057 0.00761853903532028
1058 0.00761853903532028
1059 0.00761853903532028
1060 0.00761853903532028
1061 0.00761853903532028
1062 0.00761853903532028
1063 0.00761853903532028
1064 0.00761853903532028
1065 0.00761853903532028
1066 0.00761853903532028
1067 0.00761853903532028
1068 0.00761853903532028
1069 0.00761853903532028
1070 0.00761853903532028
1071 0.00761853903532028
1072 0.00761853903532028
1073 0.00761853903532028
1074 0.00761853903532028
1075 0.00761853903532028
1076 0.00761853903532028
1077 0.00761853903532028
1078 0.00761853903532028
1079 0.00761853903532028
1080 0.00761853903532028
1081 0.00761853903532028
1082 0.00761853903532028
1083 0.00761853903532028
1084 0.00761853903532028
1085 0.00761853903532028
1086 0.00761853903532028
1087 0.00761853903532028
1088 0.00761853903532028
1089 0.00761853903532028
1090 0.00761853903532028
1091 0.00761853903532028
1092 0.00761853903532028
1093 0.00761853903532028
1094 0.00761853903532028
1095 0.00761853903532028
1096 0.00761853903532028
1097 0.00761853903532028
1098 0.00761853903532028
1099 0.00761853903532028
1100 0.00761853903532028
1101 0.00761853903532028
1102 0.00761853903532028
1103 0.00761853903532028
1104 0.00761853903532028
1105 0.00761853903532028
1106 0.00761853903532028
1107 0.00761853903532028
1108 0.00761853903532028
1109 0.00761853903532028
1110 0.00761853903532028
1111 0.00761853903532028
1112 0.00761853903532028
1113 0.00761853903532028
1114 0.00761853903532028
1115 0.00761853903532028
1116 0.00761853903532028
1117 0.00761853903532028
1118 0.00761853903532028
1119 0.00761853903532028
1120 0.00761853903532028
1121 0.00761853903532028
1122 0.00761853903532028
1123 0.00761853903532028
1124 0.00761853903532028
1125 0.00761853903532028
1126 0.00761853903532028
1127 0.00761853903532028
1128 0.00761853903532028
1129 0.00761853903532028
1130 0.00761853903532028
1131 0.00761853903532028
1132 0.00761853903532028
1133 0.00761853903532028
1134 0.00761853903532028
1135 0.00761853903532028
1136 0.00761853903532028
1137 0.00761853903532028
1138 0.00761853903532028
1139 0.00761853903532028
1140 0.00761853903532028
1141 0.00761853903532028
1142 0.00761853903532028
1143 0.00761853903532028
1144 0.00761853903532028
1145 0.00761853903532028
1146 0.00761853903532028
1147 0.00761853903532028
1148 0.00761853903532028
1149 0.00761853903532028
1150 0.00761853903532028
1151 0.00761853903532028
1152 0.00761853903532028
1153 0.00761853903532028
1154 0.00761853903532028
1155 0.00761853903532028
1156 0.00761853903532028
1157 0.00761853903532028
1158 0.00761853903532028
1159 0.00761853903532028
1160 0.00761853903532028
1161 0.00761853903532028
1162 0.00761853903532028
1163 0.00761853903532028
1164 0.00761853903532028
1165 0.00761853903532028
1166 0.00761853903532028
1167 0.00761853903532028
1168 0.00761853903532028
1169 0.00761853903532028
1170 0.00761853903532028
1171 0.00761853903532028
1172 0.00761853903532028
1173 0.00761853903532028
1174 0.00761853903532028
1175 0.00761853903532028
1176 0.00761853903532028
1177 0.00761853903532028
1178 0.00761853903532028
1179 0.00761853903532028
1180 0.00761853903532028
1181 0.00761853903532028
1182 0.00761853903532028
1183 0.00761853903532028
1184 0.00761853903532028
1185 0.00761853903532028
1186 0.00761853903532028
1187 0.00761853903532028
1188 0.00761853903532028
1189 0.00761853903532028
1190 0.00761853903532028
1191 0.00761853903532028
1192 0.00761853903532028
1193 0.00761853903532028
1194 0.00761853903532028
1195 0.00761853903532028
1196 0.00761853903532028
1197 0.00761853903532028
1198 0.00761853903532028
1199 0.00761853903532028
1200 0.00761853903532028
1201 0.00761853903532028
1202 0.00761853903532028
1203 0.00761853903532028
1204 0.00761853903532028
1205 0.00761853903532028
1206 0.00761853903532028
1207 0.00761853903532028
1208 0.00761853903532028
1209 0.00761853903532028
1210 0.00761853903532028
1211 0.00761853903532028
1212 0.00761853903532028
1213 0.00761853903532028
1214 0.00761853903532028
1215 0.00761853903532028
1216 0.00761853903532028
1217 0.00761853903532028
1218 0.00761853903532028
1219 0.00761853903532028
1220 0.00761853903532028
1221 0.00761853903532028
1222 0.00761853903532028
1223 0.00761853903532028
1224 0.00761853903532028
1225 0.00761853903532028
1226 0.00761853903532028
1227 0.00761853903532028
1228 0.00761853903532028
1229 0.00761853903532028
1230 0.00761853903532028
1231 0.00761853903532028
1232 0.00761853903532028
1233 0.00761853903532028
1234 0.00761853903532028
1235 0.00761853903532028
1236 0.00761853903532028
1237 0.00761853903532028
1238 0.00761853903532028
1239 0.00761853903532028
1240 0.00761853903532028
1241 0.00761853903532028
1242 0.00761853903532028
1243 0.00761853903532028
1244 0.00761853903532028
1245 0.00761853903532028
1246 0.00761853903532028
1247 0.00761853903532028
1248 0.00761853903532028
1249 0.00761853903532028
1250 0.00761853903532028
1251 0.00761853903532028
1252 0.00761853903532028
1253 0.00761853903532028
1254 0.00761853903532028
1255 0.00761853903532028
1256 0.00761853903532028
1257 0.00761853903532028
1258 0.00761853903532028
1259 0.00761853903532028
1260 0.00761853903532028
1261 0.00761853903532028
1262 0.00761853903532028
1263 0.00761853903532028
1264 0.00761853903532028
1265 0.00761853903532028
1266 0.00761853903532028
1267 0.00761853903532028
1268 0.00761853903532028
1269 0.00761853903532028
1270 0.00761853903532028
1271 0.00761853903532028
1272 0.00761853903532028
1273 0.00761853903532028
1274 0.00761853903532028
1275 0.00761853903532028
1276 0.00761853903532028
1277 0.00761853903532028
1278 0.00761853903532028
1279 0.00761853903532028
1280 0.00761853903532028
1281 0.00761853903532028
1282 0.00761853903532028
1283 0.00761853903532028
1284 0.00761853903532028
1285 0.00761853903532028
1286 0.00761853903532028
1287 0.00761853903532028
1288 0.00761853903532028
1289 0.00761853903532028
1290 0.00761853903532028
1291 0.00761853903532028
1292 0.00761853903532028
1293 0.00761853903532028
1294 0.00761853903532028
1295 0.00761853903532028
1296 0.00761853903532028
1297 0.00761853903532028
1298 0.00761853903532028
1299 0.00761853903532028
1300 0.00761853903532028
1301 0.00761853903532028
1302 0.00761853903532028
1303 0.00761853903532028
1304 0.00761853903532028
1305 0.00761853903532028
1306 0.00761853903532028
1307 0.00761853903532028
1308 0.00761853903532028
1309 0.00761853903532028
1310 0.00761853903532028
1311 0.00761853903532028
1312 0.00761853903532028
1313 0.00761853903532028
1314 0.00761853903532028
1315 0.00761853903532028
1316 0.00761853903532028
1317 0.00761853903532028
1318 0.00761853903532028
1319 0.00761853903532028
1320 0.00761853903532028
1321 0.00761853903532028
1322 0.00761853903532028
1323 0.00761853903532028
1324 0.00761853903532028
1325 0.00761853903532028
1326 0.00761853903532028
1327 0.00761853903532028
1328 0.00761853903532028
1329 0.00761853903532028
1330 0.00761853903532028
1331 0.00761853903532028
1332 0.00761853903532028
1333 0.00761853903532028
1334 0.00761853903532028
1335 0.00761853903532028
1336 0.00761853903532028
1337 0.00761853903532028
1338 0.00761853903532028
1339 0.00761853903532028
1340 0.00761853903532028
1341 0.00761853903532028
1342 0.00761853903532028
1343 0.00761853903532028
1344 0.00761853903532028
1345 0.00761853903532028
1346 0.00761853903532028
1347 0.00761853903532028
1348 0.00761853903532028
1349 0.00761853903532028
1350 0.00761853903532028
1351 0.00761853903532028
1352 0.00761853903532028
1353 0.00761853903532028
1354 0.00761853903532028
1355 0.00761853903532028
1356 0.00761853903532028
1357 0.00761853903532028
1358 0.00761853903532028
1359 0.00761853903532028
1360 0.00761853903532028
1361 0.00761853903532028
1362 0.00761853903532028
1363 0.00761853903532028
1364 0.00761853903532028
1365 0.00761853903532028
1366 0.00761853903532028
1367 0.00761853903532028
1368 0.00761853903532028
1369 0.00761853903532028
1370 0.00761853903532028
1371 0.00761853903532028
1372 0.00761853903532028
1373 0.00761853903532028
1374 0.00761853903532028
1375 0.00761853903532028
1376 0.00761853903532028
1377 0.00761853903532028
1378 0.00761853903532028
1379 0.00761853903532028
1380 0.00761853903532028
1381 0.00761853903532028
1382 0.00761853903532028
1383 0.00761853903532028
1384 0.00761853903532028
1385 0.00761853903532028
1386 0.00761853903532028
1387 0.00761853903532028
1388 0.00761853903532028
1389 0.00761853903532028
1390 0.00761853903532028
1391 0.00761853903532028
1392 0.00761853903532028
1393 0.00761853903532028
1394 0.00761853903532028
1395 0.00761853903532028
1396 0.00761853903532028
1397 0.00761853903532028
1398 0.00761853903532028
1399 0.00761853903532028
1400 0.00761853903532028
1401 0.00761853903532028
1402 0.00761853903532028
1403 0.00761853903532028
1404 0.00761853903532028
1405 0.00761853903532028
1406 0.00761853903532028
1407 0.00761853903532028
1408 0.00761853903532028
1409 0.00761853903532028
1410 0.00761853903532028
1411 0.00761853903532028
1412 0.00761853903532028
1413 0.00761853903532028
1414 0.00761853903532028
1415 0.00761853903532028
1416 0.00761853903532028
1417 0.00761853903532028
1418 0.00761853903532028
1419 0.00761853903532028
1420 0.00761853903532028
1421 0.00761853903532028
1422 0.00761853903532028
1423 0.00761853903532028
1424 0.00761853903532028
1425 0.00761853903532028
1426 0.00761853903532028
1427 0.00761853903532028
1428 0.00761853903532028
1429 0.00761853903532028
1430 0.00761853903532028
1431 0.00761853903532028
1432 0.00761853903532028
1433 0.00761853903532028
1434 0.00761853903532028
1435 0.00761853903532028
1436 0.00761853903532028
1437 0.00761853903532028
1438 0.00761853903532028
1439 0.00761853903532028
1440 0.00761853903532028
1441 0.00761853903532028
1442 0.00761853903532028
1443 0.00761853903532028
1444 0.00761853903532028
1445 0.00761853903532028
1446 0.00761853903532028
1447 0.00761853903532028
1448 0.00761853903532028
1449 0.00761853903532028
1450 0.00761853903532028
1451 0.00761853903532028
1452 0.00761853903532028
1453 0.00761853903532028
1454 0.00761853903532028
1455 0.00761853903532028
1456 0.00761853903532028
1457 0.00761853903532028
1458 0.00761853903532028
1459 0.00761853903532028
1460 0.00761853903532028
1461 0.00761853903532028
1462 0.00761853903532028
1463 0.00761853903532028
1464 0.00761853903532028
1465 0.00761853903532028
1466 0.00761853903532028
1467 0.00761853903532028
1468 0.00761853903532028
1469 0.00761853903532028
1470 0.00761853903532028
1471 0.00761853903532028
1472 0.00761853903532028
1473 0.00761853903532028
1474 0.00761853903532028
1475 0.00761853903532028
1476 0.00761853903532028
1477 0.00761853903532028
1478 0.00761853903532028
1479 0.00761853903532028
1480 0.00761853903532028
1481 0.00761853903532028
1482 0.00761853903532028
1483 0.00761853903532028
1484 0.00761853903532028
1485 0.00761853903532028
1486 0.00761853903532028
1487 0.00761853903532028
1488 0.00761853903532028
1489 0.00761853903532028
1490 0.00761853903532028
1491 0.00761853903532028
1492 0.00761853903532028
1493 0.00761853903532028
1494 0.00761853903532028
1495 0.00761853903532028
1496 0.00761853903532028
1497 0.00761853903532028
1498 0.00761853903532028
1499 0.00761853903532028
1500 0.00761853903532028
1501 0.00761853903532028
1502 0.00761853903532028
1503 0.00761853903532028
1504 0.00761853903532028
1505 0.00761853903532028
1506 0.00761853903532028
1507 0.00761853903532028
1508 0.00761853903532028
1509 0.00761853903532028
1510 0.00761853903532028
1511 0.00761853903532028
1512 0.00761853903532028
1513 0.00761853903532028
1514 0.00761853903532028
1515 0.00761853903532028
1516 0.00761853903532028
1517 0.00761853903532028
1518 0.00761853903532028
1519 0.00761853903532028
1520 0.00761853903532028
1521 0.00761853903532028
1522 0.00761853903532028
1523 0.00761853903532028
1524 0.00761853903532028
1525 0.00761853903532028
1526 0.00761853903532028
1527 0.00761853903532028
1528 0.00761853903532028
1529 0.00761853903532028
1530 0.00761853903532028
1531 0.00761853903532028
1532 0.00761853903532028
1533 0.00761853903532028
1534 0.00761853903532028
1535 0.00761853903532028
1536 0.00761853903532028
1537 0.00761853903532028
1538 0.00761853903532028
1539 0.00761853903532028
1540 0.00761853903532028
1541 0.00761853903532028
1542 0.00761853903532028
1543 0.00761853903532028
1544 0.00761853903532028
1545 0.00761853903532028
1546 0.00761853903532028
1547 0.00761853903532028
1548 0.00761853903532028
1549 0.00761853903532028
1550 0.00761853903532028
1551 0.00761853903532028
1552 0.00761853903532028
1553 0.00761853903532028
1554 0.00761853903532028
1555 0.00761853903532028
1556 0.00761853903532028
1557 0.00761853903532028
1558 0.00761853903532028
1559 0.00761853903532028
1560 0.00761853903532028
1561 0.00761853903532028
1562 0.00761853903532028
1563 0.00761853903532028
1564 0.00761853903532028
1565 0.00761853903532028
1566 0.00761853903532028
1567 0.00761853903532028
1568 0.00761853903532028
1569 0.00761853903532028
1570 0.00761853903532028
1571 0.00761853903532028
1572 0.00761853903532028
1573 0.00761853903532028
1574 0.00761853903532028
1575 0.00761853903532028
1576 0.00761853903532028
1577 0.00761853903532028
1578 0.00761853903532028
1579 0.00761853903532028
1580 0.00761853903532028
1581 0.00761853903532028
1582 0.00761853903532028
1583 0.00761853903532028
1584 0.00761853903532028
1585 0.00761853903532028
1586 0.00761853903532028
1587 0.00761853903532028
1588 0.00761853903532028
1589 0.00761853903532028
1590 0.00761853903532028
1591 0.00761853903532028
1592 0.00761853903532028
1593 0.00761853903532028
1594 0.00761853903532028
1595 0.00761853903532028
1596 0.00761853903532028
1597 0.00761853903532028
1598 0.00761853903532028
1599 0.00761853903532028
1600 0.00761853903532028
1601 0.00761853903532028
1602 0.00761853903532028
1603 0.00761853903532028
1604 0.00761853903532028
1605 0.00761853903532028
1606 0.00761853903532028
1607 0.00761853903532028
1608 0.00761853903532028
1609 0.00761853903532028
1610 0.00761853903532028
1611 0.00761853903532028
1612 0.00761853903532028
1613 0.00761853903532028
1614 0.00761853903532028
1615 0.00761853903532028
1616 0.00761853903532028
1617 0.00761853903532028
1618 0.00761853903532028
1619 0.00761853903532028
1620 0.00761853903532028
1621 0.00761853903532028
1622 0.00761853903532028
1623 0.00761853903532028
1624 0.00761853903532028
1625 0.00761853903532028
1626 0.00761853903532028
1627 0.00761853903532028
1628 0.00761853903532028
1629 0.00761853903532028
1630 0.00761853903532028
1631 0.00761853903532028
1632 0.00761853903532028
1633 0.00761853903532028
1634 0.00761853903532028
1635 0.00761853903532028
1636 0.00761853903532028
1637 0.00761853903532028
1638 0.00761853903532028
1639 0.00761853903532028
1640 0.00761853903532028
1641 0.00761853903532028
1642 0.00761853903532028
1643 0.00761853903532028
1644 0.00761853903532028
1645 0.00761853903532028
1646 0.00761853903532028
1647 0.00761853903532028
1648 0.00761853903532028
1649 0.00761853903532028
1650 0.00761853903532028
1651 0.00761853903532028
1652 0.00761853903532028
1653 0.00761853903532028
1654 0.00761853903532028
1655 0.00761853903532028
1656 0.00761853903532028
1657 0.00761853903532028
1658 0.00761853903532028
1659 0.00761853903532028
1660 0.00761853903532028
1661 0.00761853903532028
1662 0.00761853903532028
1663 0.00761853903532028
1664 0.00761853903532028
1665 0.00761853903532028
1666 0.00761853903532028
1667 0.00761853903532028
1668 0.00761853903532028
1669 0.00761853903532028
1670 0.00761853903532028
1671 0.00761853903532028
1672 0.00761853903532028
1673 0.00761853903532028
1674 0.00761853903532028
1675 0.00761853903532028
1676 0.00761853903532028
1677 0.00761853903532028
1678 0.00761853903532028
1679 0.00761853903532028
1680 0.00761853903532028
1681 0.00761853903532028
1682 0.00761853903532028
1683 0.00761853903532028
1684 0.00761853903532028
1685 0.00761853903532028
1686 0.00761853903532028
1687 0.00761853903532028
1688 0.00761853903532028
1689 0.00761853903532028
1690 0.00761853903532028
1691 0.00761853903532028
1692 0.00761853903532028
1693 0.00761853903532028
1694 0.00761853903532028
1695 0.00761853903532028
1696 0.00761853903532028
1697 0.00761853903532028
1698 0.00761853903532028
1699 0.00761853903532028
1700 0.00761853903532028
1701 0.00761853903532028
1702 0.00761853903532028
1703 0.00761853903532028
1704 0.00761853903532028
1705 0.00761853903532028
1706 0.00761853903532028
1707 0.00761853903532028
1708 0.00761853903532028
1709 0.00761853903532028
1710 0.00761853903532028
1711 0.00761853903532028
1712 0.00761853903532028
1713 0.00761853903532028
1714 0.00761853903532028
1715 0.00761853903532028
1716 0.00761853903532028
1717 0.00761853903532028
1718 0.00761853903532028
1719 0.00761853903532028
1720 0.00761853903532028
1721 0.00761853903532028
1722 0.00761853903532028
1723 0.00761853903532028
1724 0.00761853903532028
1725 0.00761853903532028
1726 0.00761853903532028
1727 0.00761853903532028
1728 0.00761853903532028
1729 0.00761853903532028
1730 0.00761853903532028
1731 0.00761853903532028
1732 0.00761853903532028
1733 0.00761853903532028
1734 0.00761853903532028
1735 0.00761853903532028
1736 0.00761853903532028
1737 0.00761853903532028
1738 0.00761853903532028
1739 0.00761853903532028
1740 0.00761853903532028
1741 0.00761853903532028
1742 0.00761853903532028
1743 0.00761853903532028
1744 0.00761853903532028
1745 0.00761853903532028
1746 0.00761853903532028
1747 0.00761853903532028
1748 0.00761853903532028
1749 0.00761853903532028
1750 0.00761853903532028
1751 0.00761853903532028
1752 0.00761853903532028
1753 0.00761853903532028
1754 0.00761853903532028
1755 0.00761853903532028
1756 0.00761853903532028
1757 0.00761853903532028
1758 0.00761853903532028
1759 0.00761853903532028
1760 0.00761853903532028
1761 0.00761853903532028
1762 0.00761853903532028
1763 0.00761853903532028
1764 0.00761853903532028
1765 0.00761853903532028
1766 0.00761853903532028
1767 0.00761853903532028
1768 0.00761853903532028
1769 0.00761853903532028
1770 0.00761853903532028
1771 0.00761853903532028
1772 0.00761853903532028
1773 0.00761853903532028
1774 0.00761853903532028
1775 0.00761853903532028
1776 0.00761853903532028
1777 0.00761853903532028
1778 0.00761853903532028
1779 0.00761853903532028
1780 0.00761853903532028
1781 0.00761853903532028
1782 0.00761853903532028
1783 0.00761853903532028
1784 0.00761853903532028
1785 0.00761853903532028
1786 0.00761853903532028
1787 0.00761853903532028
1788 0.00761853903532028
1789 0.00761853903532028
1790 0.00761853903532028
1791 0.00761853903532028
1792 0.00761853903532028
1793 0.00761853903532028
1794 0.00761853903532028
1795 0.00761853903532028
1796 0.00761853903532028
1797 0.00761853903532028
1798 0.00761853903532028
1799 0.00761853903532028
1800 0.00761853903532028
1801 0.00761853903532028
1802 0.00761853903532028
1803 0.00761853903532028
1804 0.00761853903532028
1805 0.00761853903532028
1806 0.00761853903532028
1807 0.00761853903532028
1808 0.00761853903532028
1809 0.00761853903532028
1810 0.00761853903532028
1811 0.00761853903532028
1812 0.00761853903532028
1813 0.00761853903532028
1814 0.00761853903532028
1815 0.00761853903532028
1816 0.00761853903532028
1817 0.00761853903532028
1818 0.00761853903532028
1819 0.00761853903532028
1820 0.00761853903532028
1821 0.00761853903532028
1822 0.00761853903532028
1823 0.00761853903532028
1824 0.00761853903532028
1825 0.00761853903532028
1826 0.00761853903532028
1827 0.00761853903532028
1828 0.00761853903532028
1829 0.00761853903532028
1830 0.00761853903532028
1831 0.00761853903532028
1832 0.00761853903532028
1833 0.00761853903532028
1834 0.00761853903532028
1835 0.00761853903532028
1836 0.00761853903532028
1837 0.00761853903532028
1838 0.00761853903532028
1839 0.00761853903532028
1840 0.00761853903532028
1841 0.00761853903532028
1842 0.00761853903532028
1843 0.00761853903532028
1844 0.00761853903532028
1845 0.00761853903532028
1846 0.00761853903532028
1847 0.00761853903532028
1848 0.00761853903532028
1849 0.00761853903532028
1850 0.00761853903532028
1851 0.00761853903532028
1852 0.00761853903532028
1853 0.00761853903532028
1854 0.00761853903532028
1855 0.00761853903532028
1856 0.00761853903532028
1857 0.00761853903532028
1858 0.00761853903532028
1859 0.00761853903532028
1860 0.00761853903532028
1861 0.00761853903532028
1862 0.00761853903532028
1863 0.00761853903532028
1864 0.00761853903532028
1865 0.00761853903532028
1866 0.00761853903532028
1867 0.00761853903532028
1868 0.00761853903532028
1869 0.00761853903532028
1870 0.00761853903532028
1871 0.00761853903532028
1872 0.00761853903532028
1873 0.00761853903532028
1874 0.00761853903532028
1875 0.00761853903532028
1876 0.00761853903532028
1877 0.00761853903532028
1878 0.00761853903532028
1879 0.00761853903532028
1880 0.00761853903532028
1881 0.00761853903532028
1882 0.00761853903532028
1883 0.00761853903532028
1884 0.00761853903532028
1885 0.00761853903532028
1886 0.00761853903532028
1887 0.00761853903532028
1888 0.00761853903532028
1889 0.00761853903532028
1890 0.00761853903532028
1891 0.00761853903532028
1892 0.00761853903532028
1893 0.00761853903532028
1894 0.00761853903532028
1895 0.00761853903532028
1896 0.00761853903532028
1897 0.00761853903532028
1898 0.00761853903532028
1899 0.00761853903532028
1900 0.00761853903532028
1901 0.00761853903532028
1902 0.00761853903532028
1903 0.00761853903532028
1904 0.00761853903532028
1905 0.00761853903532028
1906 0.00761853903532028
1907 0.00761853903532028
1908 0.00761853903532028
1909 0.00761853903532028
1910 0.00761853903532028
1911 0.00761853903532028
1912 0.00761853903532028
1913 0.00761853903532028
1914 0.00761853903532028
1915 0.00761853903532028
1916 0.00761853903532028
1917 0.00761853903532028
1918 0.00761853903532028
1919 0.00761853903532028
1920 0.00761853903532028
1921 0.00761853903532028
1922 0.00761853903532028
1923 0.00761853903532028
1924 0.00761853903532028
1925 0.00761853903532028
1926 0.00761853903532028
1927 0.00761853903532028
1928 0.00761853903532028
1929 0.00761853903532028
1930 0.00761853903532028
1931 0.00761853903532028
1932 0.00761853903532028
1933 0.00761853903532028
1934 0.00761853903532028
1935 0.00761853903532028
1936 0.00761853903532028
1937 0.00761853903532028
1938 0.00761853903532028
1939 0.00761853903532028
1940 0.00761853903532028
1941 0.00761853903532028
1942 0.00761853903532028
1943 0.00761853903532028
1944 0.00761853903532028
1945 0.00761853903532028
1946 0.00761853903532028
1947 0.00761853903532028
1948 0.00761853903532028
1949 0.00761853903532028
1950 0.00761853903532028
1951 0.00761853903532028
1952 0.00761853903532028
1953 0.00761853903532028
1954 0.00761853903532028
1955 0.00761853903532028
1956 0.00761853903532028
1957 0.00761853903532028
1958 0.00761853903532028
1959 0.00761853903532028
1960 0.00761853903532028
1961 0.00761853903532028
1962 0.00761853903532028
1963 0.00761853903532028
1964 0.00761853903532028
1965 0.00761853903532028
1966 0.00761853903532028
1967 0.00761853903532028
1968 0.00761853903532028
1969 0.00761853903532028
1970 0.00761853903532028
1971 0.00761853903532028
1972 0.00761853903532028
1973 0.00761853903532028
1974 0.00761853903532028
1975 0.00761853903532028
1976 0.00761853903532028
1977 0.00761853903532028
1978 0.00761853903532028
1979 0.00761853903532028
1980 0.00761853903532028
1981 0.00761853903532028
1982 0.00761853903532028
1983 0.00761853903532028
1984 0.00761853903532028
1985 0.00761853903532028
1986 0.00761853903532028
1987 0.00761853903532028
1988 0.00761853903532028
1989 0.00761853903532028
1990 0.00761853903532028
1991 0.00761853903532028
1992 0.00761853903532028
1993 0.00761853903532028
1994 0.00761853903532028
1995 0.00761853903532028
1996 0.00761853903532028
1997 0.00761853903532028
1998 0.00761853903532028
1999 0.00761853903532028
};
\addlegendentry{Test}

\nextgroupplot[
title={3 Layer},
ymin=5.81725598087273e-06, ymax=0.01,
]
\addplot [semithick, black, dashed]
table {%
0 0.0325543856015429
1 0.0319750122725964
2 0.0313961076317355
3 0.0308154578087851
4 0.030227355659008
5 0.0296229132218286
6 0.0289840936311521
7 0.0282681233948097
8 0.0273984759696759
9 0.026286119769793
10 0.0249351374804974
11 0.0235147805069573
12 0.0221129137207754
13 0.0207472360343672
14 0.0194722522282973
15 0.0182859042542987
16 0.0171902960864827
17 0.0161999388656113
18 0.0152708681707736
19 0.0143773120071273
20 0.0135207428247668
21 0.0127233648963738
22 0.0119949935469776
23 0.0113276096235495
24 0.010716264921939
25 0.0101537634327542
26 0.00961569749051705
27 0.00910687878786121
28 0.00862702653103042
29 0.00818350190820638
30 0.00777514137735125
31 0.00739925735979341
32 0.00704647935344838
33 0.00670999896829017
34 0.00638713088119403
35 0.00607521317579085
36 0.00577379726746585
37 0.00547817409096751
38 0.00518890971579822
39 0.00490640452335356
40 0.00463116334867664
41 0.00437236374273198
42 0.00412331408733735
43 0.00388235782884294
44 0.00364749768050388
45 0.00341939082863973
46 0.00319195325573673
47 0.00297158459579805
48 0.00276008875152911
49 0.00255535844553378
50 0.00236401519578067
51 0.00218753240187652
52 0.00202879042990389
53 0.00188823928692727
54 0.0017648163557169
55 0.00165671333525097
56 0.00156342906848295
57 0.00148441053534043
58 0.00141769794936408
59 0.00136126488359878
60 0.00131339507424855
61 0.00127199381677201
62 0.00123584210814442
63 0.00120404243716621
64 0.00117543940723408
65 0.00114919409315917
66 0.00112496862493572
67 0.00110243622111739
68 0.0010812936743605
69 0.00106131083157379
70 0.00104206228752446
71 0.00102355287344835
72 0.00100578117780969
73 0.000989010770354071
74 0.000972663112406735
75 0.000957138385274448
76 0.000941974150919123
77 0.000927237946598325
78 0.000912464160137461
79 0.000898148438864155
80 0.000884348397448775
81 0.000871022557475953
82 0.000857880477269646
83 0.000845019521875656
84 0.000832253643238801
85 0.000819730292278109
86 0.000807246260592365
87 0.000795045976701658
88 0.000782812987381476
89 0.000770868697145488
90 0.000758807680540485
91 0.000747046433389187
92 0.000735142715711845
93 0.000723506167560117
94 0.000711754688381916
95 0.000700277425494278
96 0.000688667585563962
97 0.000677268551953603
98 0.000665820012727636
99 0.000654647652481799
100 0.000643335079075769
101 0.00063223204233509
102 0.000621044051513309
103 0.00060987804135948
104 0.000598442597038229
105 0.000587135745263367
106 0.000576049648771004
107 0.000565232881854172
108 0.000554631134036754
109 0.000544330117918435
110 0.0005341660034901
111 0.000524346780366614
112 0.000514772737915337
113 0.000505439134940389
114 0.000496354579809122
115 0.000487593616526283
116 0.000478975533042103
117 0.00047075990914891
118 0.000462719068309525
119 0.000454946785794164
120 0.000447406579041854
121 0.000440152440205566
122 0.000433144811722741
123 0.000426439086368191
124 0.000419886991039675
125 0.000413672076774674
126 0.000407648700729624
127 0.000401812737891305
128 0.00039625673844057
129 0.000390876897199632
130 0.000385721395105065
131 0.000380792112537165
132 0.000376047925328749
133 0.000371554128378193
134 0.000367219715371903
135 0.000363101292805368
136 0.000359118924734503
137 0.000355331031641981
138 0.000351693887751026
139 0.000348168185155373
140 0.000344798800142598
141 0.0003414986927055
142 0.000338387408646668
143 0.000335397920025571
144 0.000332541617808602
145 0.000329802564465353
146 0.000327214288972755
147 0.000324704196373204
148 0.000322335641612881
149 0.000320023339781983
150 0.000317816773531376
151 0.000315701968929716
152 0.000313656539674412
153 0.000311695413984125
154 0.000309759694346212
155 0.000307940539414631
156 0.000306130506032787
157 0.000304451374859127
158 0.000302715283623911
159 0.00030112840795482
160 0.000299511501680172
161 0.000298024676339992
162 0.000296529653496691
163 0.000295141195010729
164 0.000293727770213081
165 0.00029240558615129
166 0.000291067316538829
167 0.000289798731728297
168 0.000288429736201579
169 0.000287210275928373
170 0.00028601109443116
171 0.000284822603362045
172 0.000283610400629186
173 0.000282452843293868
174 0.000281273085192879
175 0.000280123058018944
176 0.000279007651442953
177 0.000277911419971133
178 0.000276847195209484
179 0.000275799254723097
180 0.000274766852840003
181 0.000273746711854983
182 0.000272761618703043
183 0.000271785427230498
184 0.000270829980308918
185 0.000269907834876904
186 0.000268974053028614
187 0.000268074474092828
188 0.000267178923650135
189 0.000266306771777636
190 0.000265439652707755
191 0.000264575118535504
192 0.000263729073481045
193 0.000262914799691316
194 0.000262084185692402
195 0.000261282649148598
196 0.00026046603238683
197 0.000259683425952062
198 0.000258899136383661
199 0.00025810843317231
200 0.000257345576187618
201 0.000256599538602131
202 0.000255852932355083
203 0.000255119852681673
204 0.000254388650205328
205 0.000253676011652715
206 0.000252974716886456
207 0.000252268404665301
208 0.000251572575848513
209 0.000250876647328369
210 0.000250191660938981
211 0.000249498535595194
212 0.000248830090072261
213 0.000248151638288618
214 0.000247461683784422
215 0.000246804745984264
216 0.000246131559833884
217 0.000245472481196884
218 0.000244805914803692
219 0.000244159441422198
220 0.000243513238388005
221 0.000242851843154313
222 0.000242200140746718
223 0.000241566359022727
224 0.000240917016299136
225 0.000240264410876989
226 0.000239628610870568
227 0.000238984708062162
228 0.000238349917481173
229 0.000237712683315294
230 0.000237068142382668
231 0.000236430182212644
232 0.000235795187450094
233 0.000235143566158058
234 0.000234502524733671
235 0.000233856164015833
236 0.000233198952514613
237 0.000232554599051582
238 0.000231882889579538
239 0.000231233756323945
240 0.000230572652924366
241 0.00022993095245738
242 0.000229272412411774
243 0.000228603992297849
244 0.000227944841697081
245 0.000227290652844658
246 0.000226635613842063
247 0.000225965130027816
248 0.000225312019267676
249 0.000224642273565223
250 0.000223973037520864
251 0.000223271828815541
252 0.000222569292702701
253 0.000221873327518551
254 0.000221172573105832
255 0.000220451392010546
256 0.00021972385582103
257 0.000219006342263128
258 0.000218298178538134
259 0.000217564019351357
260 0.00021681336659185
261 0.000216093900746728
262 0.000215341834120863
263 0.000214600615834115
264 0.000213860771168584
265 0.000213107175682126
266 0.000212337102084348
267 0.000211596834446937
268 0.000210831511992637
269 0.000210053293415058
270 0.000209278011709557
271 0.000208495622132432
272 0.000207709547851209
273 0.00020692171040082
274 0.000206115954654251
275 0.000205294003023937
276 0.000204491057957057
277 0.000203662920398529
278 0.000202827109831105
279 0.000201990264031338
280 0.00020114908954838
281 0.00020029029417401
282 0.00019943086516605
283 0.000198558365980261
284 0.000197695443375778
285 0.000196788284995364
286 0.000195892687543164
287 0.000194978689194159
288 0.000194061579065874
289 0.000193130325669699
290 0.000192178398776832
291 0.000191240895617284
292 0.000190261992770502
293 0.000189284849682281
294 0.00018830770261502
295 0.000187312108892002
296 0.000186260163104635
297 0.000185242746340464
298 0.00018419325715513
299 0.00018309180831011
300 0.000181998211019163
301 0.00018089648114028
302 0.00017977910800937
303 0.000178688398648319
304 0.000177575511486339
305 0.000176480264428847
306 0.000175350492156667
307 0.000174231101368605
308 0.000173086013887769
309 0.000171940303573592
310 0.000170770979252666
311 0.000169614094488679
312 0.000168421197656698
313 0.000167215547321575
314 0.000166006505764926
315 0.000164770245248747
316 0.000163533360989732
317 0.000162264511459398
318 0.000161022904819674
319 0.000159758962865908
320 0.000158453494975674
321 0.000157149742278762
322 0.000155840075024116
323 0.000154501898805393
324 0.00015316347054295
325 0.000151821847055089
326 0.000150481864295671
327 0.000149147603679012
328 0.000147816145158686
329 0.000146470152458278
330 0.000145119753170775
331 0.000143758510688485
332 0.000142412635540268
333 0.000141049845609587
334 0.000139664303674181
335 0.00013832155889304
336 0.000136965470119321
337 0.000135575909723684
338 0.000134219123225421
339 0.000132835603352532
340 0.00013145188572139
341 0.000130082629993922
342 0.000128685167425147
343 0.000127303993366468
344 0.000125908055139234
345 0.000124526810481029
346 0.000123160545172141
347 0.000121779563869495
348 0.000120406302670517
349 0.000119042009572468
350 0.000117670152377514
351 0.000116302211438324
352 0.000114937962706563
353 0.000113574123730587
354 0.00011225873680587
355 0.000110929664288051
356 0.000109611477626004
357 0.000108338161055599
358 0.000107025631194801
359 0.000105745030879234
360 0.000104468234184196
361 0.000103239449401826
362 0.000102009700782446
363 0.000100798898927223
364 9.96187914381608e-05
365 9.84380003785645e-05
366 9.72921673394467e-05
367 9.61505358247905e-05
368 9.50435981792452e-05
369 9.39337713248278e-05
370 9.28595265747845e-05
371 9.18109233793984e-05
372 9.07656875028806e-05
373 8.97527096412887e-05
374 8.87449959634523e-05
375 8.77455168790675e-05
376 8.67364587122665e-05
377 8.57865232717359e-05
378 8.48284212509043e-05
379 8.38745253304296e-05
380 8.29761713987409e-05
381 8.20719552621085e-05
382 8.11876763577857e-05
383 8.0345129589432e-05
384 7.94986073628934e-05
385 7.86865310544727e-05
386 7.7885259344157e-05
387 7.70960247962194e-05
388 7.63058844768238e-05
389 7.55337781583876e-05
390 7.47575121522459e-05
391 7.4014346154172e-05
392 7.33093116878081e-05
393 7.26041235452612e-05
394 7.19233631230054e-05
395 7.12587252849062e-05
396 7.06071174931822e-05
397 6.99695566481751e-05
398 6.93443737702637e-05
399 6.87613872685233e-05
400 6.81624566993833e-05
401 6.76066200071546e-05
402 6.70609263409006e-05
403 6.65237920713935e-05
404 6.59844539825372e-05
405 6.54943939366603e-05
406 6.49935263936641e-05
407 6.4498708610472e-05
408 6.40149565072079e-05
409 6.3550320817285e-05
410 6.31035976539351e-05
411 6.26416050693024e-05
412 6.22049784624323e-05
413 6.17637383300007e-05
414 6.13645924545381e-05
415 6.09370136430698e-05
416 6.05319815747407e-05
417 6.01182887578489e-05
418 5.97408835858459e-05
419 5.93426862565138e-05
420 5.89754737347903e-05
421 5.85921377478371e-05
422 5.82286894541539e-05
423 5.78509811077765e-05
424 5.7498178506421e-05
425 5.71578842425424e-05
426 5.67970208180668e-05
427 5.64575480836993e-05
428 5.6106215112095e-05
429 5.57688372282428e-05
430 5.54667350201044e-05
431 5.51540573354714e-05
432 5.48568750531331e-05
433 5.45568199754598e-05
434 5.42537273844346e-05
435 5.39615244434799e-05
436 5.3646506955829e-05
437 5.33611214166285e-05
438 5.30732663008848e-05
439 5.27986441483108e-05
440 5.25497172958467e-05
441 5.22999338699037e-05
442 5.20331377913408e-05
443 5.17922175902186e-05
444 5.15377980150333e-05
445 5.12876647604799e-05
446 5.10024789406316e-05
447 5.07504836519956e-05
448 5.04969675745315e-05
449 5.02810184315194e-05
450 5.00427650251822e-05
451 4.97988166046071e-05
452 4.95663533968127e-05
453 4.93496950184635e-05
454 4.91171351058028e-05
455 4.89128532024097e-05
456 4.86747536712073e-05
457 4.84676021414998e-05
458 4.82607221385933e-05
459 4.8066875706354e-05
460 4.78532716101654e-05
461 4.76488472855863e-05
462 4.74483782539892e-05
463 4.72820314456612e-05
464 4.70619548877949e-05
465 4.68852903452444e-05
466 4.67070103695733e-05
467 4.6554281226463e-05
468 4.63748530563635e-05
469 4.62049356002581e-05
470 4.60059391684808e-05
471 4.5842817371522e-05
472 4.5645920877746e-05
473 4.54798581159821e-05
474 4.53033275960024e-05
475 4.51220362833737e-05
476 4.49656372438767e-05
477 4.48194138584768e-05
478 4.46795485657958e-05
479 4.45128710850895e-05
480 4.43423514582264e-05
481 4.41750076589642e-05
482 4.39892867802882e-05
483 4.38685025159202e-05
484 4.37255445859819e-05
485 4.35766932866954e-05
486 4.3443040979696e-05
487 4.32537846819514e-05
488 4.3092020490576e-05
489 4.29309008325163e-05
490 4.27827283004945e-05
491 4.26220234714947e-05
492 4.24818363597979e-05
493 4.23289462503362e-05
494 4.2179659658359e-05
495 4.20290831613102e-05
496 4.18828702066776e-05
497 4.17203252709442e-05
498 4.16085529622023e-05
499 4.14898704690358e-05
500 4.13639224774442e-05
501 4.12266160907393e-05
502 4.10484632027419e-05
503 4.09470047912919e-05
504 4.08047575461978e-05
505 4.06763110731845e-05
506 4.05137927117494e-05
507 4.03811150277988e-05
508 4.02583072656171e-05
509 4.01554014501926e-05
510 4.00247143090837e-05
511 3.99048297765603e-05
512 3.97582258813145e-05
513 3.95831205679542e-05
514 3.94770701319658e-05
515 3.93544491288367e-05
516 3.9218565518695e-05
517 3.90781203094548e-05
518 3.89348760023722e-05
519 3.87966968418141e-05
520 3.86787812516332e-05
521 3.85617538682936e-05
522 3.8410539119127e-05
523 3.82785152055476e-05
524 3.81896712866592e-05
525 3.81169848537866e-05
526 3.79647891577406e-05
527 3.78392183222331e-05
528 3.77224317560376e-05
529 3.75850829215096e-05
530 3.7483298754637e-05
531 3.7376991215865e-05
532 3.72287421441797e-05
533 3.71367570721759e-05
534 3.70057509826438e-05
535 3.68922542577366e-05
536 3.67537317202959e-05
537 3.66116351671053e-05
538 3.65386643039756e-05
539 3.64230447189584e-05
540 3.63248719992271e-05
541 3.61699259769921e-05
542 3.60795606866304e-05
543 3.59300416761243e-05
544 3.58424720872108e-05
545 3.5740010474683e-05
546 3.55985175133355e-05
547 3.54853262791721e-05
548 3.53957780419023e-05
549 3.52912501995206e-05
550 3.51706323300505e-05
551 3.50520422429668e-05
552 3.49280033091759e-05
553 3.48007355768232e-05
554 3.47059452181497e-05
555 3.46138635833881e-05
556 3.44930241524821e-05
557 3.4374945769855e-05
558 3.42684362593104e-05
559 3.41691244756248e-05
560 3.40476382163502e-05
561 3.39779866180834e-05
562 3.38682114957578e-05
563 3.37874783511438e-05
564 3.36690892481784e-05
565 3.35717006549885e-05
566 3.35045417401147e-05
567 3.338279535825e-05
568 3.33089057775737e-05
569 3.32158142413164e-05
570 3.31037999714567e-05
571 3.30013659635142e-05
572 3.29292958056726e-05
573 3.28154018660598e-05
574 3.27323138336055e-05
575 3.26655925277919e-05
576 3.25405261705214e-05
577 3.24611250590578e-05
578 3.23791812348873e-05
579 3.22652077784369e-05
580 3.21422678268846e-05
581 3.20924455294858e-05
582 3.19898985310374e-05
583 3.19033673861213e-05
584 3.17730414423067e-05
585 3.17060722210272e-05
586 3.16166486769021e-05
587 3.15165642703619e-05
588 3.14483602181781e-05
589 3.13734398957877e-05
590 3.12639891220101e-05
591 3.11740237179947e-05
592 3.10897013235945e-05
593 3.10373253071816e-05
594 3.09236335027663e-05
595 3.08327785489837e-05
596 3.07820210423415e-05
597 3.06750062648575e-05
598 3.06008624377796e-05
599 3.05186794520296e-05
600 3.04207069063978e-05
601 3.03470507674319e-05
602 3.02458308354403e-05
603 3.01717561370651e-05
604 3.01046751758349e-05
605 3.00191006061823e-05
606 2.99325367834058e-05
607 2.98688187854168e-05
608 2.97702543505807e-05
609 2.96773123551475e-05
610 2.95954527089748e-05
611 2.95156116223438e-05
612 2.94010434132019e-05
613 2.93328042246799e-05
614 2.92847203766655e-05
615 2.91768054623276e-05
616 2.90827233158097e-05
617 2.90288215509804e-05
618 2.89424344543932e-05
619 2.8882403562136e-05
620 2.87473713243003e-05
621 2.86985531801065e-05
622 2.86399613926847e-05
623 2.85467540521722e-05
624 2.84325846848787e-05
625 2.83949904087422e-05
626 2.83245936785192e-05
627 2.82152458837004e-05
628 2.81719357388965e-05
629 2.80890176966864e-05
630 2.79993629774822e-05
631 2.79067480448703e-05
632 2.78389053178785e-05
633 2.77218611373087e-05
634 2.76985987568423e-05
635 2.76259362621545e-05
636 2.75327281684667e-05
637 2.74711059446986e-05
638 2.7374490301213e-05
639 2.72844964257501e-05
640 2.72240517276146e-05
641 2.71399572042696e-05
642 2.70317738326753e-05
643 2.69783617596886e-05
644 2.69293038996921e-05
645 2.68325926384705e-05
646 2.67788614038977e-05
647 2.67081254676782e-05
648 2.66407606801522e-05
649 2.65498801539366e-05
650 2.64925029114238e-05
651 2.64317721416774e-05
652 2.63613923081607e-05
653 2.62645514013116e-05
654 2.62097401915184e-05
655 2.61777568510979e-05
656 2.60830049256811e-05
657 2.60134706664417e-05
658 2.59654593790515e-05
659 2.58873092633394e-05
660 2.58179071224163e-05
661 2.57635793943223e-05
662 2.56628729289332e-05
663 2.56129519229376e-05
664 2.55695288160496e-05
665 2.54748088934775e-05
666 2.54124114107412e-05
667 2.5348189261365e-05
668 2.52841959067496e-05
669 2.52080856810721e-05
670 2.51656390020116e-05
671 2.50807183945767e-05
672 2.50222525863819e-05
673 2.49698765983908e-05
674 2.48904988850995e-05
675 2.48219550300632e-05
676 2.47906463073377e-05
677 2.47362822634045e-05
678 2.46737859441737e-05
679 2.45797361984046e-05
680 2.45664375739807e-05
681 2.45093125315066e-05
682 2.44501761983429e-05
683 2.43493386093974e-05
684 2.43383761997507e-05
685 2.42787759887619e-05
686 2.42133022823054e-05
687 2.41435846390914e-05
688 2.4098857913657e-05
689 2.40354220579775e-05
690 2.39835178845738e-05
691 2.39283468985718e-05
692 2.3859919934921e-05
693 2.38290110061712e-05
694 2.37747668521138e-05
695 2.36661183592446e-05
696 2.36410650842345e-05
697 2.36399101112283e-05
698 2.35918802289348e-05
699 2.35436894868712e-05
700 2.35253294462723e-05
701 2.34558819514064e-05
702 2.34135196919283e-05
703 2.33794401580667e-05
704 2.33120936456999e-05
705 2.3224337702743e-05
706 2.3166013548348e-05
707 2.30799700418061e-05
708 2.29893513008506e-05
709 2.29336895998244e-05
710 2.28646509086161e-05
711 2.27630865055062e-05
712 2.27479432837185e-05
713 2.26493055279775e-05
714 2.26145240631581e-05
715 2.25426552020735e-05
716 2.24831239137302e-05
717 2.23761273758782e-05
718 2.23642499506127e-05
719 2.23058106385565e-05
720 2.22376235115007e-05
721 2.21810196165961e-05
722 2.21365815917807e-05
723 2.20500888730157e-05
724 2.20159214023852e-05
725 2.19491395725413e-05
726 2.18927621915554e-05
727 2.18453066764823e-05
728 2.18234490283464e-05
729 2.17297292834928e-05
730 2.17459670039943e-05
731 2.16788268510015e-05
732 2.16481249566414e-05
733 2.16012727172199e-05
734 2.15837312005362e-05
735 2.15430365066993e-05
736 2.15070307589826e-05
737 2.14662543669419e-05
738 2.14652095280599e-05
739 2.13662552575045e-05
740 2.13821392804903e-05
741 2.13165587510389e-05
742 2.12912786459185e-05
743 2.12578595366608e-05
744 2.12403572739106e-05
745 2.11545379080746e-05
746 2.11667295921814e-05
747 2.11150030189344e-05
748 2.10719182369701e-05
749 2.10260284596586e-05
750 2.10123263890694e-05
751 2.09503027051028e-05
752 2.09470098830877e-05
753 2.0864470357651e-05
754 2.08226908888776e-05
755 2.08147499165534e-05
756 2.0791622450389e-05
757 2.07412637074356e-05
758 2.07257131386029e-05
759 2.06638472768361e-05
760 2.06473674424501e-05
761 2.05630911658261e-05
762 2.05279574139183e-05
763 2.05117009137723e-05
764 2.04767841829323e-05
765 2.04112533310763e-05
766 2.03956556532603e-05
767 2.03307622008708e-05
768 2.03100006608281e-05
769 2.02059517953046e-05
770 2.01841244233947e-05
771 2.01528757060032e-05
772 2.01197776199535e-05
773 2.00439536257591e-05
774 2.00457303627388e-05
775 1.99711960569005e-05
776 1.99205280253523e-05
777 1.99084231589097e-05
778 1.98499856480794e-05
779 1.97886184558627e-05
780 1.97941501163257e-05
781 1.97436789548533e-05
782 1.96963144460938e-05
783 1.96652108108708e-05
784 1.9663456942709e-05
785 1.95796245208157e-05
786 1.95772894251434e-05
787 1.95248350323141e-05
788 1.94973508698126e-05
789 1.94537172149012e-05
790 1.94353991354035e-05
791 1.93740427114619e-05
792 1.93620647905846e-05
793 1.92856128045094e-05
794 1.92683945812178e-05
795 1.92476689768739e-05
796 1.92228534103833e-05
797 1.91360224910397e-05
798 1.91401133626812e-05
799 1.91189775158307e-05
800 1.90750299573494e-05
801 1.90277885785406e-05
802 1.90580991841216e-05
803 1.9000818909376e-05
804 1.90061016667187e-05
805 1.89514761856913e-05
806 1.89268244419338e-05
807 1.89188686263719e-05
808 1.88833024310497e-05
809 1.87923430203796e-05
810 1.88431752476959e-05
811 1.87859644960042e-05
812 1.87148377897017e-05
813 1.87670403377638e-05
814 1.87269906959386e-05
815 1.87046492285958e-05
816 1.86845428302718e-05
817 1.86284468561837e-05
818 1.86337373797585e-05
819 1.85897806268542e-05
820 1.8587689709193e-05
821 1.85177556630833e-05
822 1.8534029972983e-05
823 1.84972093215663e-05
824 1.84068604554e-05
825 1.84350518814824e-05
826 1.83963500468565e-05
827 1.83384203005232e-05
828 1.83745563759885e-05
829 1.8330988211801e-05
830 1.82716602523669e-05
831 1.82893900131376e-05
832 1.82525706193815e-05
833 1.82405412694209e-05
834 1.82480485442227e-05
835 1.8179594963641e-05
836 1.81480629954933e-05
837 1.81525437170649e-05
838 1.81265700334166e-05
839 1.80882780611569e-05
840 1.81026283812002e-05
841 1.80423676781061e-05
842 1.80301288068563e-05
843 1.8045064191341e-05
844 1.79816654508613e-05
845 1.79246907343611e-05
846 1.79458194935478e-05
847 1.78833234869558e-05
848 1.78168026181424e-05
849 1.78353375730467e-05
850 1.7776137511305e-05
851 1.77720856271435e-05
852 1.77581373606017e-05
853 1.76878244424472e-05
854 1.77052459626736e-05
855 1.76504318218917e-05
856 1.76317058802056e-05
857 1.76194689203157e-05
858 1.75891930140892e-05
859 1.75050904118734e-05
860 1.75318757662524e-05
861 1.75083287956568e-05
862 1.74614688077668e-05
863 1.74144732341119e-05
864 1.74353298945107e-05
865 1.73600120980666e-05
866 1.73546296871052e-05
867 1.73136668735197e-05
868 1.72956926647316e-05
869 1.72832360121333e-05
870 1.72682371832877e-05
871 1.71900160097493e-05
872 1.72019144457636e-05
873 1.7179593349681e-05
874 1.7135854282202e-05
875 1.71013277352472e-05
876 1.70982460261371e-05
877 1.70518025903732e-05
878 1.70371590968443e-05
879 1.69858285197222e-05
880 1.69592485903536e-05
881 1.69616153655738e-05
882 1.69332589194937e-05
883 1.6858160183375e-05
884 1.68812298930732e-05
885 1.68486827263337e-05
886 1.67876691357094e-05
887 1.67627305671658e-05
888 1.67677841922398e-05
889 1.67174102294609e-05
890 1.67219806392893e-05
891 1.66606255049828e-05
892 1.66521545459375e-05
893 1.663980959421e-05
894 1.66209873242451e-05
895 1.65278110202394e-05
896 1.65539636469703e-05
897 1.65319762857052e-05
898 1.64758275182919e-05
899 1.6437693926008e-05
900 1.64490508147708e-05
901 1.63872301826018e-05
902 1.63999446307628e-05
903 1.63320763277852e-05
904 1.63537660853308e-05
905 1.62983377940407e-05
906 1.62976172788376e-05
907 1.62330398794097e-05
908 1.62214224168622e-05
909 1.62108603660727e-05
910 1.61715568047782e-05
911 1.6094322894844e-05
912 1.6083277888157e-05
913 1.60780667748384e-05
914 1.60317338497862e-05
915 1.5995008485703e-05
916 1.59740159624278e-05
917 1.5953192097129e-05
918 1.5928540261001e-05
919 1.58762833990522e-05
920 1.5864329114379e-05
921 1.58435410249069e-05
922 1.58286786238193e-05
923 1.57287297248843e-05
924 1.57361341379669e-05
925 1.56911291853135e-05
926 1.56815395264687e-05
927 1.55968384838445e-05
928 1.56107426469987e-05
929 1.55563237882461e-05
930 1.55520693709832e-05
931 1.54930479538962e-05
932 1.54924098723086e-05
933 1.543203437393e-05
934 1.54203818674148e-05
935 1.53731215171149e-05
936 1.53470335320094e-05
937 1.53118545611619e-05
938 1.52466075995505e-05
939 1.52147722545237e-05
940 1.51873506837319e-05
941 1.51505664298668e-05
942 1.51030340589386e-05
943 1.50938795577815e-05
944 1.50408497852084e-05
945 1.50018896079018e-05
946 1.49779933700245e-05
947 1.49298111260521e-05
948 1.49100890247666e-05
949 1.4859539881229e-05
950 1.48355787636945e-05
951 1.47800392120701e-05
952 1.4780255511937e-05
953 1.47193738797569e-05
954 1.46894688803911e-05
955 1.46725380751889e-05
956 1.46426646203679e-05
957 1.46180487625713e-05
958 1.45882615996129e-05
959 1.4559031829009e-05
960 1.45366459065599e-05
961 1.4506683658766e-05
962 1.44901006784437e-05
963 1.44515447040305e-05
964 1.44433175783831e-05
965 1.4417459833993e-05
966 1.43882516994154e-05
967 1.43678341970599e-05
968 1.4350144617481e-05
969 1.4331682407942e-05
970 1.43121968889659e-05
971 1.42836098859789e-05
972 1.42715542956751e-05
973 1.42519485599735e-05
974 1.42366275319716e-05
975 1.42368436755191e-05
976 1.41878036572507e-05
977 1.42016572652892e-05
978 1.41623325831119e-05
979 1.41576969951984e-05
980 1.41488800160516e-05
981 1.41181795392242e-05
982 1.41246705460674e-05
983 1.40974667282023e-05
984 1.40861921096302e-05
985 1.40636805561201e-05
986 1.40554306113927e-05
987 1.40471551652865e-05
988 1.40209019434678e-05
989 1.40210539960606e-05
990 1.39934638347938e-05
991 1.39940296399743e-05
992 1.39849829103866e-05
993 1.39528946832002e-05
994 1.3965923244541e-05
995 1.39199313160532e-05
996 1.392892432861e-05
997 1.39094303008847e-05
998 1.38994388478864e-05
999 1.38875931092741e-05
1000 1.38674188647769e-05
1001 1.38641419411556e-05
1002 1.38430925105126e-05
1003 1.38452279578871e-05
1004 1.38185845699468e-05
1005 1.38032333261151e-05
1006 1.38102615281355e-05
1007 1.37817029788323e-05
1008 1.37703540374901e-05
1009 1.37647197355761e-05
1010 1.37360616747628e-05
1011 1.37408426787999e-05
1012 1.37062998497584e-05
1013 1.37123944021766e-05
1014 1.37210499211449e-05
1015 1.36755174509062e-05
1016 1.36841219990913e-05
1017 1.36570234445799e-05
1018 1.37175554328905e-05
1019 1.36926006568672e-05
1020 1.3950600084911e-05
1021 1.40754287656364e-05
1022 1.49904420752023e-05
1023 1.55505595778038e-05
1024 1.63913940376403e-05
1025 1.5609538309036e-05
1026 1.4283840588547e-05
1027 1.38156426068292e-05
1028 1.3484064051994e-05
1029 1.36274133986092e-05
1030 1.34626247287883e-05
1031 1.35584520180032e-05
1032 1.3474013705661e-05
1033 1.34634292408009e-05
1034 1.34230953854342e-05
1035 1.33830423294512e-05
1036 1.34062201375684e-05
1037 1.33365437626765e-05
1038 1.3388676197934e-05
1039 1.33586398902708e-05
1040 1.34045410185024e-05
1041 1.33421764587638e-05
1042 1.34312028876593e-05
1043 1.33571745450922e-05
1044 1.34366705175637e-05
1045 1.3374745666539e-05
1046 1.34613321129962e-05
1047 1.34146667782886e-05
1048 1.35568693835353e-05
1049 1.35194760666479e-05
1050 1.37919426421718e-05
1051 1.37520721779083e-05
1052 1.40883602348651e-05
1053 1.39536711714072e-05
1054 1.40560608450357e-05
1055 1.38168416476958e-05
1056 1.36969487058991e-05
1057 1.34113066945218e-05
1058 1.32351839354783e-05
1059 1.3203349887192e-05
1060 1.31117146970894e-05
1061 1.3238625733436e-05
1062 1.31784456947059e-05
1063 1.34360107644227e-05
1064 1.33821509251675e-05
1065 1.36530772287813e-05
1066 1.35580949773839e-05
1067 1.36580433718336e-05
1068 1.34291251399077e-05
1069 1.33854128225863e-05
1070 1.31981301727535e-05
1071 1.3187625329536e-05
1072 1.30408069463073e-05
1073 1.29787823990313e-05
1074 1.29582482930601e-05
1075 1.2898906021519e-05
1076 1.29487141862228e-05
1077 1.29098963856933e-05
1078 1.31194291626002e-05
1079 1.3106707235977e-05
1080 1.36149354119652e-05
1081 1.37244544973214e-05
1082 1.44547026650343e-05
1083 1.41260946087129e-05
1084 1.37233747974363e-05
1085 1.32654718179026e-05
1086 1.29282386005514e-05
1087 1.28920367892249e-05
1088 1.27731225330052e-05
1089 1.29420105885458e-05
1090 1.28685785867333e-05
1091 1.30135572256052e-05
1092 1.28964552423838e-05
1093 1.29331214218098e-05
1094 1.28142672153331e-05
1095 1.28475321652388e-05
1096 1.27572813823917e-05
1097 1.27485812253525e-05
1098 1.26969600113824e-05
1099 1.26870987067207e-05
1100 1.26724213274088e-05
1101 1.26524225798619e-05
1102 1.26947076424244e-05
1103 1.26583670070346e-05
1104 1.28118276663258e-05
1105 1.28086327411125e-05
1106 1.32073520866527e-05
1107 1.33042820724683e-05
1108 1.3822308101652e-05
1109 1.3717794420387e-05
1110 1.37084808660148e-05
1111 1.316860543632e-05
1112 1.27470038702171e-05
1113 1.26231004742294e-05
1114 1.25273771587331e-05
1115 1.2563867553439e-05
1116 1.25007985207759e-05
1117 1.26256642154488e-05
1118 1.25670874560058e-05
1119 1.27630908490772e-05
1120 1.26871213605995e-05
1121 1.28802124148564e-05
1122 1.2796516037028e-05
1123 1.29468341860672e-05
1124 1.28122378679763e-05
1125 1.29314928294377e-05
1126 1.27120832846828e-05
1127 1.26299045497547e-05
1128 1.25056189403949e-05
1129 1.24295912957706e-05
1130 1.24146642992429e-05
1131 1.23858883434025e-05
1132 1.23859904341828e-05
1133 1.23638540348026e-05
1134 1.24000674741609e-05
1135 1.2367337372865e-05
1136 1.24857948122781e-05
1137 1.24504618899124e-05
1138 1.27597903976095e-05
1139 1.28849593910019e-05
1140 1.36381725290846e-05
1141 1.3663470848968e-05
1142 1.3772351859842e-05
1143 1.35367069979964e-05
1144 1.327582795696e-05
1145 1.2699358100221e-05
1146 1.22786888123727e-05
1147 1.23074038533844e-05
1148 1.223217236479e-05
1149 1.23806040122076e-05
1150 1.23046592985787e-05
1151 1.24423170220211e-05
1152 1.23259598954206e-05
1153 1.23780566934073e-05
1154 1.22633247148229e-05
1155 1.22369730330973e-05
1156 1.21950923528402e-05
1157 1.21604218694671e-05
1158 1.21377823631974e-05
1159 1.21322881287966e-05
1160 1.21728577529012e-05
1161 1.21272124946614e-05
1162 1.22319735655907e-05
1163 1.21914060429162e-05
1164 1.24327291715076e-05
1165 1.24354616257705e-05
1166 1.29066975809167e-05
1167 1.29366587806601e-05
1168 1.32105487722356e-05
1169 1.28606168985357e-05
1170 1.25413945326613e-05
1171 1.22794203036847e-05
1172 1.21325474982115e-05
1173 1.20565982513199e-05
1174 1.19594920953148e-05
1175 1.20290681184798e-05
1176 1.19674889766941e-05
1177 1.20572598589774e-05
1178 1.20257342590691e-05
1179 1.21526348433321e-05
1180 1.21045993104474e-05
1181 1.22989311002186e-05
1182 1.22301723646245e-05
1183 1.24912517556197e-05
1184 1.2454098875736e-05
1185 1.26749778441848e-05
1186 1.2522787878666e-05
1187 1.25851543231192e-05
1188 1.23422960545128e-05
1189 1.22262319841582e-05
1190 1.20822752869287e-05
1191 1.20525333144883e-05
1192 1.194216659961e-05
1193 1.19519587897798e-05
1194 1.18773938098116e-05
1195 1.1884994938427e-05
1196 1.1849186527968e-05
1197 1.1851726263501e-05
1198 1.18339386574462e-05
1199 1.18348765241194e-05
1200 1.18294311786116e-05
1201 1.18173332275262e-05
1202 1.18371348420965e-05
1203 1.1814890637396e-05
1204 1.18921919920467e-05
1205 1.1894327174744e-05
1206 1.22914106199801e-05
1207 1.26345219477741e-05
1208 1.40591531092582e-05
1209 1.42156470808175e-05
1210 1.34048927851182e-05
1211 1.23812964147874e-05
1212 1.17877493579499e-05
1213 1.1778949776442e-05
1214 1.16511243248851e-05
1215 1.17607856129354e-05
1216 1.16873712396881e-05
1217 1.17368365764037e-05
1218 1.16747084000224e-05
1219 1.17042689726787e-05
1220 1.16579150120089e-05
1221 1.1661208317193e-05
1222 1.16374596075275e-05
1223 1.16261359259795e-05
1224 1.16306483093354e-05
1225 1.16189884735718e-05
1226 1.16327209038758e-05
1227 1.16244864347692e-05
1228 1.16835540939775e-05
1229 1.16369047606923e-05
1230 1.17668366623747e-05
1231 1.17468169555934e-05
1232 1.20282155560147e-05
1233 1.20749230010375e-05
1234 1.26129790682938e-05
1235 1.2700170628932e-05
1236 1.278620186973e-05
1237 1.23272101024696e-05
1238 1.19932761872832e-05
1239 1.17275082160972e-05
1240 1.15496058938902e-05
1241 1.15335418282569e-05
1242 1.14630288123863e-05
1243 1.15385233065979e-05
1244 1.15028123452987e-05
1245 1.16434930870213e-05
1246 1.16017957907388e-05
1247 1.17814842344188e-05
1248 1.16933003937447e-05
1249 1.18884917288398e-05
1250 1.18319192150551e-05
1251 1.20064951136811e-05
1252 1.18176314938268e-05
1253 1.1796657148011e-05
1254 1.16383278356835e-05
1255 1.15647926914875e-05
1256 1.14756408748917e-05
1257 1.14403467534885e-05
1258 1.13919901121307e-05
1259 1.14356357645562e-05
1260 1.1387096995108e-05
1261 1.14807917022119e-05
1262 1.14235796715434e-05
1263 1.15913755180941e-05
1264 1.16049974430155e-05
1265 1.20072158722451e-05
1266 1.20544539541356e-05
1267 1.23738329573797e-05
1268 1.22181800197296e-05
1269 1.21687504570644e-05
1270 1.18944955449507e-05
1271 1.17016860148311e-05
1272 1.14737063885428e-05
1273 1.13172452209653e-05
1274 1.12781009384832e-05
1275 1.12072275690878e-05
1276 1.13205783520698e-05
1277 1.12835398340394e-05
1278 1.14918894773552e-05
1279 1.14527015142585e-05
1280 1.17121657368102e-05
1281 1.1616246023749e-05
1282 1.17771928636046e-05
1283 1.16777891445707e-05
1284 1.19345812468197e-05
1285 1.16845124633613e-05
1286 1.16825437661561e-05
1287 1.14294903195145e-05
1288 1.12836280550255e-05
1289 1.12077542375744e-05
1290 1.11284330479577e-05
1291 1.13021933270119e-05
1292 1.13066414826335e-05
1293 1.17059199400416e-05
1294 1.17300647790586e-05
1295 1.2044247652554e-05
1296 1.17156109702421e-05
1297 1.13793020286579e-05
1298 1.12148285733582e-05
1299 1.10923834952814e-05
1300 1.11365353649973e-05
1301 1.10729875881788e-05
1302 1.12302857591118e-05
1303 1.11914661040657e-05
1304 1.13637668448519e-05
1305 1.12939137544288e-05
1306 1.13848981246889e-05
1307 1.12628727144681e-05
1308 1.1254580064346e-05
1309 1.11565751517162e-05
1310 1.11748975086812e-05
1311 1.10900645662326e-05
1312 1.10597877327478e-05
1313 1.10091162461856e-05
1314 1.09928415348293e-05
1315 1.10074921355618e-05
1316 1.09802914192159e-05
1317 1.10427660597878e-05
1318 1.10171958649374e-05
1319 1.12220047530087e-05
1320 1.12654352939501e-05
1321 1.18365659034225e-05
1322 1.21140192579361e-05
1323 1.28907241094822e-05
1324 1.25081939721383e-05
1325 1.18342823949291e-05
1326 1.12988185492213e-05
1327 1.09413417668236e-05
1328 1.09967017269952e-05
1329 1.09574359381526e-05
1330 1.12309775772701e-05
1331 1.11610841315724e-05
1332 1.13463721866935e-05
1333 1.11532378372914e-05
1334 1.11302044096817e-05
1335 1.09673260926257e-05
1336 1.0879462042368e-05
1337 1.09174289057989e-05
1338 1.08601508852502e-05
1339 1.10446273087206e-05
1340 1.10255442731244e-05
1341 1.13266524781608e-05
1342 1.12422739970697e-05
1343 1.14362164413961e-05
1344 1.12368936164842e-05
1345 1.12229393778307e-05
1346 1.09911064889445e-05
1347 1.08713756912238e-05
1348 1.08531602815987e-05
1349 1.07972825951208e-05
1350 1.09944084574209e-05
1351 1.09879797403067e-05
1352 1.13783931077194e-05
1353 1.13800229435412e-05
1354 1.1600956018043e-05
1355 1.12816381090397e-05
1356 1.1044686484496e-05
1357 1.08788349635347e-05
1358 1.07404201390437e-05
1359 1.08264981353301e-05
1360 1.07610637805067e-05
1361 1.09618825003821e-05
1362 1.09280510471876e-05
1363 1.11411545145046e-05
1364 1.10470111867045e-05
1365 1.11286042923098e-05
1366 1.09783528401408e-05
1367 1.09532099532572e-05
1368 1.08398085991723e-05
1369 1.0848850489964e-05
1370 1.07521912227071e-05
1371 1.07336795185375e-05
1372 1.07022989066508e-05
1373 1.06785852445057e-05
1374 1.06971311630844e-05
1375 1.06850921959278e-05
1376 1.08084445482604e-05
1377 1.08310274686829e-05
1378 1.12672853518347e-05
1379 1.152452863451e-05
1380 1.23981799173833e-05
1381 1.2092268297792e-05
1382 1.14240201174454e-05
1383 1.10053059376725e-05
1384 1.07124563726302e-05
1385 1.06713157848048e-05
1386 1.05960178871101e-05
1387 1.07637729005461e-05
1388 1.07286204258372e-05
1389 1.09087064874558e-05
1390 1.07860504332535e-05
1391 1.08567311141172e-05
1392 1.07373885178674e-05
1393 1.07115631919896e-05
1394 1.06230669345564e-05
1395 1.05808139387875e-05
1396 1.0580599607124e-05
1397 1.05551910998258e-05
1398 1.06698323616428e-05
1399 1.06572321172393e-05
1400 1.09597528084038e-05
1401 1.10038155032299e-05
1402 1.14300031111014e-05
1403 1.1338423655971e-05
1404 1.13434759594355e-05
1405 1.10623585278802e-05
1406 1.08373800138395e-05
1407 1.06405395676745e-05
1408 1.05073550038526e-05
1409 1.05484162133962e-05
1410 1.05023424215034e-05
1411 1.06451282366749e-05
1412 1.06078949517752e-05
1413 1.0736837726455e-05
1414 1.06968986131051e-05
1415 1.07971093665782e-05
1416 1.0689379379869e-05
1417 1.07563324327486e-05
1418 1.0709299639089e-05
1419 1.09476776515294e-05
1420 1.08412190407137e-05
1421 1.09159507903911e-05
1422 1.07557246291634e-05
1423 1.07460236709045e-05
1424 1.06075585222243e-05
1425 1.05439961135545e-05
1426 1.04790878197036e-05
1427 1.04631327406679e-05
1428 1.0443433296814e-05
1429 1.04332782058236e-05
1430 1.04179071218624e-05
1431 1.04264383438135e-05
1432 1.04638093212372e-05
1433 1.04452442535319e-05
1434 1.06312007890352e-05
1435 1.07311893824402e-05
1436 1.14612785626633e-05
1437 1.18105211690533e-05
1438 1.22257010382043e-05
1439 1.18460510378782e-05
1440 1.11231241675114e-05
1441 1.06675707343129e-05
1442 1.03672584756964e-05
1443 1.0381226752898e-05
1444 1.03580038466333e-05
1445 1.05478686549532e-05
1446 1.05259513070166e-05
1447 1.07450906767781e-05
1448 1.06260766621347e-05
1449 1.06993795494503e-05
1450 1.05084744213002e-05
1451 1.04540240304374e-05
1452 1.03582404591407e-05
1453 1.03097048569367e-05
1454 1.04077485971743e-05
1455 1.0401103917701e-05
1456 1.06831689468123e-05
1457 1.07213950126805e-05
1458 1.11300581728813e-05
1459 1.10285412731059e-05
1460 1.11255577532887e-05
1461 1.07388742627279e-05
1462 1.04518073147375e-05
1463 1.03618716842391e-05
1464 1.03032426785887e-05
1465 1.05396604102026e-05
1466 1.05861713421973e-05
1467 1.09537863863807e-05
1468 1.08464691788157e-05
1469 1.08989028326789e-05
1470 1.05561651828623e-05
1471 1.03589073461308e-05
1472 1.03063158665151e-05
1473 1.02616811332723e-05
1474 1.04932726490148e-05
1475 1.05223466952964e-05
1476 1.08051974514467e-05
1477 1.06691236574363e-05
1478 1.0681912167243e-05
1479 1.04143852528438e-05
1480 1.02812786266782e-05
1481 1.02495154639115e-05
1482 1.02202497096471e-05
1483 1.04338059276898e-05
1484 1.04503235913711e-05
1485 1.07841298735423e-05
1486 1.06682702316618e-05
1487 1.07330254124349e-05
1488 1.04592844767382e-05
1489 1.03289774724402e-05
1490 1.02229501965923e-05
1491 1.01573431336988e-05
1492 1.0365642953758e-05
1493 1.04059158374525e-05
1494 1.07526524502077e-05
1495 1.07005128846538e-05
1496 1.08033977497257e-05
1497 1.0496132741622e-05
1498 1.02706591178503e-05
1499 1.01972091943736e-05
1500 1.01082141004127e-05
1501 1.02982351180003e-05
1502 1.03076002293534e-05
1503 1.06332435834133e-05
1504 1.05553666323033e-05
1505 1.06591341744178e-05
1506 1.04107369427453e-05
1507 1.02702524529263e-05
1508 1.0142037348615e-05
1509 1.00745982170025e-05
1510 1.02010417055709e-05
1511 1.02264526731233e-05
1512 1.05783643569168e-05
1513 1.05929602565169e-05
1514 1.08398942373356e-05
1515 1.05545325617129e-05
1516 1.0376593627015e-05
1517 1.01846262907657e-05
1518 1.00587712150713e-05
1519 1.01346028387184e-05
1520 1.01227972866269e-05
1521 1.04462185177567e-05
1522 1.04564727241296e-05
1523 1.0748007069239e-05
1524 1.05681337529973e-05
1525 1.05106662058319e-05
1526 1.02300889697204e-05
1527 1.00908491749152e-05
1528 1.00893646628464e-05
1529 1.00885037976894e-05
1530 1.03723661357691e-05
1531 1.04256442625683e-05
1532 1.06946860860546e-05
1533 1.04784329604257e-05
1534 1.04050787808063e-05
1535 1.01163926231607e-05
1536 9.98345727154515e-06
1537 1.00578206509994e-05
1538 1.00212816338541e-05
1539 1.02000516264411e-05
1540 1.01403353838236e-05
1541 1.0262479662515e-05
1542 1.01681475346282e-05
1543 1.02183333545725e-05
1544 1.01113442241285e-05
1545 1.00417474158121e-05
1546 1.00084736818218e-05
1547 9.94311576896223e-06
1548 9.95790098734517e-06
1549 9.94402533116556e-06
1550 1.00616789016073e-05
1551 1.00968142948688e-05
1552 1.03843188785646e-05
1553 1.04800567566343e-05
1554 1.09330517208406e-05
1555 1.08527331850183e-05
1556 1.07813627110431e-05
1557 1.03809185691972e-05
1558 1.01115643680316e-05
1559 9.99649022048743e-06
1560 9.84632469425151e-06
1561 9.97419392945176e-06
1562 9.97192387686141e-06
1563 1.02364182321679e-05
1564 1.0224903414624e-05
1565 1.04230705506581e-05
1566 1.02558015573351e-05
1567 1.01347500187643e-05
1568 9.98800424234503e-06
1569 9.8792938967307e-06
1570 9.88157342085572e-06
1571 9.83805956344952e-06
1572 9.95592140284884e-06
1573 9.96433398015029e-06
1574 1.01675188322048e-05
1575 1.01785217818673e-05
1576 1.042226152137e-05
1577 1.03775851592047e-05
1578 1.04635157960331e-05
1579 1.02573808629103e-05
1580 1.01572051711685e-05
1581 9.99304318405336e-06
1582 9.88483525610206e-06
1583 9.846430504723e-06
1584 9.77965296478089e-06
1585 9.8278245204142e-06
1586 9.80407573436537e-06
1587 9.94559823652708e-06
1588 9.9768569938874e-06
1589 1.03316112785024e-05
1590 1.04226046264699e-05
1591 1.07118036378751e-05
1592 1.06278771010437e-05
1593 1.05810348554058e-05
1594 1.03043860200103e-05
1595 1.00902675406189e-05
1596 9.89709658938409e-06
1597 9.76444311362457e-06
1598 9.80857203103369e-06
1599 9.81784188880397e-06
1600 1.00726187763911e-05
1601 1.00782759382412e-05
1602 1.03353167855147e-05
1603 1.02541534570122e-05
1604 1.02591598984958e-05
1605 1.00207905813221e-05
1606 9.86174639727722e-06
1607 9.789249524772e-06
1608 9.72242486874109e-06
1609 9.85282760090911e-06
1610 9.87855271716853e-06
1611 1.01538654533329e-05
1612 1.01972987724963e-05
1613 1.04211994429448e-05
1614 1.0260216207314e-05
1615 1.01794818689882e-05
1616 9.95140795723159e-06
1617 9.80054852739443e-06
1618 9.73560998041023e-06
1619 9.6937388569529e-06
1620 9.83627617578975e-06
1621 9.86149742310261e-06
1622 1.01208723286561e-05
1623 1.01344444036755e-05
1624 1.03389104975093e-05
1625 1.01878703127056e-05
1626 1.01371552680973e-05
1627 9.88182770456092e-06
1628 9.72828499889999e-06
1629 9.72120662012799e-06
1630 9.67856705536008e-06
1631 9.87470161284421e-06
1632 9.91196086275181e-06
1633 1.02487670705642e-05
1634 1.02342939030109e-05
1635 1.03632265879838e-05
1636 1.00876582553866e-05
1637 9.9162478530701e-06
1638 9.76065481061994e-06
1639 9.63107017781795e-06
1640 9.71227991009016e-06
1641 9.71921576642387e-06
1642 9.95104367618183e-06
1643 9.98188006740008e-06
1644 1.01814610360407e-05
1645 1.00809739809193e-05
1646 1.00580276747309e-05
1647 9.89943332818655e-06
1648 9.85886584992102e-06
1649 9.6850337758525e-06
1650 9.6363883148598e-06
1651 9.65914809292201e-06
1652 9.68013837798765e-06
1653 9.97025561666476e-06
1654 1.01058179851066e-05
1655 1.05360674407251e-05
1656 1.03080558346846e-05
1657 1.00926539126078e-05
1658 9.8807339110607e-06
1659 9.80618986545778e-06
1660 9.64750019250005e-06
1661 9.57001510748512e-06
1662 9.71566173824101e-06
1663 9.76910989614055e-06
1664 1.01642681471503e-05
1665 1.01708189959027e-05
1666 1.02663863881247e-05
1667 9.97338110408919e-06
1668 9.78185772915197e-06
1669 9.6475179702793e-06
1670 9.54183381907114e-06
1671 9.72729881709711e-06
1672 9.76016392684187e-06
1673 1.00475685247403e-05
1674 9.98918297412388e-06
1675 1.00374019726956e-05
1676 9.83480547844806e-06
1677 9.68626762443137e-06
1678 9.58236335435458e-06
1679 9.52750649574341e-06
1680 9.6046833704122e-06
1681 9.65048948309288e-06
1682 9.94330360448714e-06
1683 1.00139402476884e-05
1684 1.03263379607199e-05
1685 1.01539922834348e-05
1686 1.00855086575535e-05
1687 9.75463516539321e-06
1688 9.59124333022032e-06
1689 9.62183743880018e-06
1690 9.66085548625983e-06
1691 1.00433389498278e-05
1692 1.01062234492133e-05
1693 1.02644902337801e-05
1694 9.94451049862732e-06
1695 9.71057364296257e-06
1696 9.57142326996063e-06
1697 9.51365443313534e-06
1698 9.74081426008411e-06
1699 9.80161510177879e-06
1700 1.00278100561013e-05
1701 9.89786944138871e-06
1702 9.82283335382306e-06
1703 9.59643337061777e-06
1704 9.49116020976248e-06
1705 9.56513969541106e-06
1706 9.58506606707488e-06
1707 9.87981061406629e-06
1708 9.87621043613274e-06
1709 1.00010725923028e-05
1710 9.78027932241332e-06
1711 9.64564478245222e-06
1712 9.51542429916685e-06
1713 9.44944513392443e-06
1714 9.62438961948919e-06
1715 9.68926873667897e-06
1716 1.0011827177081e-05
1717 9.98087672599013e-06
1718 9.99936146151015e-06
1719 9.68524250488656e-06
1720 9.45949793162981e-06
1721 9.4604499611961e-06
1722 9.43409829368136e-06
1723 9.59400313682579e-06
1724 9.61277961408769e-06
1725 9.73128234882381e-06
1726 9.66869359153577e-06
1727 9.675940004783e-06
1728 9.53800632963464e-06
1729 9.45548768171989e-06
1730 9.41575705049047e-06
1731 9.38730114263819e-06
1732 9.42329548969667e-06
1733 9.4680719531226e-06
1734 9.69707735443137e-06
1735 9.83681814759052e-06
1736 1.0342070867253e-05
1737 1.03673044389296e-05
1738 1.0319499523348e-05
1739 9.80776392367488e-06
1740 9.46366283649525e-06
1741 9.4079414676429e-06
1742 9.33771825906149e-06
1743 9.44709846528724e-06
1744 9.44257008406169e-06
1745 9.55572185468156e-06
1746 9.51379190361479e-06
1747 9.58348498691919e-06
1748 9.5346116353312e-06
1749 9.59080875873042e-06
1750 9.48247226340015e-06
1751 9.45194587842479e-06
1752 9.39730169768893e-06
1753 9.44287307369507e-06
1754 9.38174645526146e-06
1755 9.36534963891233e-06
1756 9.37066532991082e-06
1757 9.37060450212357e-06
1758 9.35343812180633e-06
1759 9.37159184211112e-06
1760 9.34568313404327e-06
1761 9.38406464356945e-06
1762 9.35800415113874e-06
1763 9.51361231571468e-06
1764 9.61215439154728e-06
1765 1.03008284746409e-05
1766 1.0909676795734e-05
1767 1.15658478634373e-05
1768 1.10141704503519e-05
1769 1.00041754489411e-05
1770 9.48770325592818e-06
1771 9.2457984486316e-06
1772 9.36993604661041e-06
1773 9.34585882816918e-06
1774 9.45014970632485e-06
1775 9.39453432025061e-06
1776 9.34213808179152e-06
1777 9.2832726945602e-06
1778 9.27011165074987e-06
1779 9.26857236116518e-06
1780 9.23603815650154e-06
1781 9.37810197498834e-06
1782 9.36324507705422e-06
1783 9.561256947066e-06
1784 9.56277991015497e-06
1785 9.67651309302653e-06
1786 9.60353385259793e-06
1787 9.55971781557707e-06
1788 9.43333498426568e-06
1789 9.30421281175597e-06
1790 9.26368776710262e-06
1791 9.23089301707591e-06
1792 9.25094655279679e-06
1793 9.26948077761836e-06
1794 9.50006205080456e-06
1795 9.68563010772527e-06
1796 1.03284271499859e-05
1797 1.0474079685352e-05
1798 1.03454442310635e-05
1799 9.83696086365171e-06
1800 9.46682479074923e-06
1801 9.29783671566042e-06
1802 9.16098180603342e-06
1803 9.27800909167331e-06
1804 9.27603664813148e-06
1805 9.36472894430551e-06
1806 9.32859566304955e-06
1807 9.33878378006625e-06
1808 9.29493666745884e-06
1809 9.26994334093934e-06
1810 9.22636866995674e-06
1811 9.21319120017472e-06
1812 9.18993305809579e-06
1813 9.18065634536447e-06
1814 9.18903356605938e-06
1815 9.16299991615688e-06
1816 9.24872737151361e-06
1817 9.22939812220136e-06
1818 9.41584444724697e-06
1819 9.53778233814262e-06
1820 1.00982309128028e-05
1821 1.03530967123788e-05
1822 1.07677988943067e-05
1823 1.05822749816298e-05
1824 9.9346092135022e-06
1825 9.66056256324066e-06
1826 9.39209207473368e-06
1827 9.28394516819253e-06
1828 9.27514356252601e-06
1829 9.58359539993126e-06
1830 9.64701049532835e-06
1831 9.76074391800807e-06
1832 9.46976195415061e-06
1833 9.27086305146929e-06
1834 9.22346017162567e-06
1835 9.18639791525067e-06
1836 9.39396119115088e-06
1837 9.38899461644382e-06
1838 9.47275931650893e-06
1839 9.33303375560968e-06
1840 9.2705345409172e-06
1841 9.17447256298942e-06
1842 9.12055301505177e-06
1843 9.23321208645689e-06
1844 9.26156465475003e-06
1845 9.48838633973992e-06
1846 9.47618959301622e-06
1847 9.57715797333947e-06
1848 9.40416721562087e-06
1849 9.35467390661415e-06
1850 9.1928318664003e-06
1851 9.10465488601631e-06
1852 9.21001994491633e-06
1853 9.27723038213202e-06
1854 9.66308451744169e-06
1855 9.79048432014906e-06
1856 1.00840502010868e-05
1857 9.78570066223483e-06
1858 9.51169420382314e-06
1859 9.24167201965531e-06
1860 9.13996445284226e-06
1861 9.368165986956e-06
1862 9.43662190699968e-06
1863 9.6558417901349e-06
1864 9.48265713773822e-06
1865 9.36103240256614e-06
1866 9.17536207190039e-06
1867 9.07686779250128e-06
1868 9.18668061444805e-06
1869 9.24960672499253e-06
1870 9.4993737533855e-06
1871 9.46106675492331e-06
1872 9.52852745594868e-06
1873 9.29470442301294e-06
1874 9.15522953448544e-06
1875 9.10800912201637e-06
1876 9.10976500811955e-06
1877 9.34183466583249e-06
1878 9.43885657278543e-06
1879 9.67132466200837e-06
1880 9.52948176546897e-06
1881 9.44895895926834e-06
1882 9.19533778187542e-06
1883 9.06884305607036e-06
1884 9.16910886950006e-06
1885 9.2296366371869e-06
1886 9.57614021857012e-06
1887 9.59371429587463e-06
1888 9.68478909868509e-06
1889 9.37958474267475e-06
1890 9.18509949876523e-06
1891 9.1062100331385e-06
1892 9.09807884319491e-06
1893 9.38312567733135e-06
1894 9.47310969223736e-06
1895 9.64519133539454e-06
1896 9.41235415830022e-06
1897 9.24790520429042e-06
1898 9.09275317795277e-06
1899 9.03334196689087e-06
1900 9.24427961557228e-06
1901 9.31243963719908e-06
1902 9.54136548436679e-06
1903 9.37950756174644e-06
1904 9.28196380378665e-06
1905 9.08915719399772e-06
1906 8.98297265372605e-06
1907 9.12634225436193e-06
1908 9.20780046254066e-06
1909 9.47157235131613e-06
1910 9.42229635469971e-06
1911 9.43142685194687e-06
1912 9.17063193917045e-06
1913 9.00058953767768e-06
1914 9.01881548998062e-06
1915 9.02907029320943e-06
1916 9.28665624755354e-06
1917 9.35035448357269e-06
1918 9.50602375304754e-06
1919 9.30446626057346e-06
1920 9.11586828777899e-06
1921 9.00186593710828e-06
1922 8.94229994052864e-06
1923 9.06205088746503e-06
1924 9.14136129281928e-06
1925 9.46210800201186e-06
1926 9.48726459704119e-06
1927 9.58058928901551e-06
1928 9.30747766680895e-06
1929 9.10795650810314e-06
1930 9.00443722429145e-06
1931 8.98152932382601e-06
1932 9.27929349892054e-06
1933 9.40790902603794e-06
1934 9.68967716552527e-06
1935 9.47854077892885e-06
1936 9.26747610030532e-06
1937 9.03901926641026e-06
1938 8.94018997410484e-06
1939 9.17381487930413e-06
1940 9.26010278057277e-06
1941 9.5082699136384e-06
1942 9.32774693396254e-06
1943 9.17254057597461e-06
1944 8.98344639566062e-06
1945 8.90296216038422e-06
1946 9.07888414580782e-06
1947 9.14082143310679e-06
1948 9.36939631657197e-06
1949 9.24882918695857e-06
1950 9.16184878008153e-06
1951 8.97223990747875e-06
1952 8.85600724132019e-06
1953 8.98509080293763e-06
1954 9.03752668612867e-06
1955 9.29935078719524e-06
1956 9.25974211618552e-06
1957 9.2469854386934e-06
1958 9.0511159633877e-06
1959 8.92878742320136e-06
1960 8.88701305612472e-06
1961 8.89928274627749e-06
1962 9.15602618967171e-06
1963 9.24528500512167e-06
1964 9.53794148728093e-06
1965 9.39845971537068e-06
1966 9.26508881349264e-06
1967 9.00150441829339e-06
1968 8.85389574456497e-06
1969 8.98382258185393e-06
1970 9.03694898024554e-06
1971 9.28067967542745e-06
1972 9.23201188385292e-06
1973 9.19148967515326e-06
1974 9.02398184976505e-06
1975 8.88445486069145e-06
1976 8.8420500894415e-06
1977 8.81328993784791e-06
1978 8.98355164391518e-06
1979 9.04603394413783e-06
1980 9.26778463039568e-06
1981 9.25045723221274e-06
1982 9.22494544042252e-06
1983 9.01499192806909e-06
1984 8.85589470911441e-06
1985 8.82704753379926e-06
1986 8.77382949671102e-06
1987 8.96668880656648e-06
1988 9.06950991286237e-06
1989 9.43982228207574e-06
1990 9.49616746659387e-06
1991 9.53817240301191e-06
1992 9.17533424171779e-06
1993 8.85387608384747e-06
1994 8.83605240176166e-06
1995 8.77495183626564e-06
1996 8.93654905809171e-06
1997 8.98863621756618e-06
1998 9.23461551494142e-06
1999 9.20856176023221e-06
};
\addlegendentry{Train}
\addplot [semithick, black]
table {%
0 0.0323242284357548
1 0.0317548513412476
2 0.0311849638819695
3 0.0306114219129086
4 0.0300263669341803
5 0.0294189359992743
6 0.0287609919905663
7 0.0279939621686935
8 0.0270275287330151
9 0.0258064176887274
10 0.0244404375553131
11 0.0230877324938774
12 0.0217513777315617
13 0.0204767771065235
14 0.0193001329898834
15 0.0181972552090883
16 0.0172018446028233
17 0.0162852164357901
18 0.0154061429202557
19 0.0145565439015627
20 0.0137495575472713
21 0.0130098648369312
22 0.0123314736410975
23 0.0117060486227274
24 0.011133867315948
25 0.0105926804244518
26 0.0100730676203966
27 0.00957882776856422
28 0.00911447312682867
29 0.00868617277592421
30 0.00828820839524269
31 0.00791505724191666
32 0.00755761982873082
33 0.00721253734081984
34 0.00687585351988673
35 0.00654887547716498
36 0.00622765673324466
37 0.00590978283435106
38 0.0055969781242311
39 0.00528946286067367
40 0.00499550765380263
41 0.00471598096191883
42 0.00444303173571825
43 0.00417672470211983
44 0.00391782307997346
45 0.00366128981113434
46 0.00340677029453218
47 0.00316225527785718
48 0.00292076915502548
49 0.00269423634745181
50 0.00248303986154497
51 0.0022920488845557
52 0.00212210486643016
53 0.00197244971059263
54 0.00184359541162848
55 0.00173190154600888
56 0.00163703609723598
57 0.00155721628107131
58 0.00149017374496907
59 0.00143374071922153
60 0.00138567865360528
61 0.00134393118787557
62 0.00130728981457651
63 0.00127463263925165
64 0.00124484952539206
65 0.0012172389542684
66 0.00119157007429749
67 0.00116759643424302
68 0.0011449905578047
69 0.00112341437488794
70 0.00110248907003552
71 0.0010823376942426
72 0.00106341869104654
73 0.00104469642974436
74 0.00102718325797468
75 0.00100984727032483
76 0.000993196736089885
77 0.00097641657339409
78 0.000960123492404819
79 0.000944090832490474
80 0.000928883731830865
81 0.000913703697733581
82 0.000899048405699432
83 0.000884274078998715
84 0.000870000221766531
85 0.000855485210195184
86 0.00084156054072082
87 0.000827328069135547
88 0.000813607941381633
89 0.000799565692432225
90 0.000786048185545951
91 0.000772083061747253
92 0.000758759211748838
93 0.00074492726707831
94 0.00073171965777874
95 0.000718042952939868
96 0.00070497568231076
97 0.00069140741834417
98 0.000678557145874947
99 0.000665121711790562
100 0.000652362825348973
101 0.000639177160337567
102 0.000626374676357955
103 0.000612964504398406
104 0.000599805556703359
105 0.000586521869990975
106 0.000573958735913038
107 0.00056128588039428
108 0.000549193820916116
109 0.000537059851922095
110 0.000525665003806353
111 0.000514239131007344
112 0.000503319839481264
113 0.000492464168928564
114 0.000482176401419565
115 0.00047182702110149
116 0.000462180934846401
117 0.000452517269877717
118 0.000443479046225548
119 0.000434432411566377
120 0.000425856764195487
121 0.000417623319663107
122 0.000409641390433535
123 0.000401862809667364
124 0.000394625298213214
125 0.000387370615499094
126 0.00038056357880123
127 0.000374065304640681
128 0.000367726432159543
129 0.000361551909008995
130 0.000355845200829208
131 0.000350249843904749
132 0.000345010572345927
133 0.000339996564434841
134 0.000335209449985996
135 0.000330651993863285
136 0.000326237000990659
137 0.000322067353408784
138 0.000318077218253165
139 0.000314255361445248
140 0.000310543342493474
141 0.000307035952573642
142 0.00030367323779501
143 0.000300479878205806
144 0.000297446589684114
145 0.000294632511213422
146 0.000291826989268884
147 0.000289289717329666
148 0.000286787922959775
149 0.00028443755581975
150 0.000282149994745851
151 0.000280010863207281
152 0.000277935585472733
153 0.000275931728538126
154 0.000274047022685409
155 0.000272148987278342
156 0.000270436255959794
157 0.000268663075985387
158 0.000267064053332433
159 0.000265445880359039
160 0.000264007016085088
161 0.000262485118582845
162 0.000261140143265948
163 0.000259763182839379
164 0.000258473068242893
165 0.0002571971854195
166 0.000256024039117619
167 0.000254728336585686
168 0.000253539154073223
169 0.00025248623569496
170 0.000251450896030292
171 0.000250428158324212
172 0.000249375676503405
173 0.000248440162977204
174 0.000247506657615304
175 0.000246569223236293
176 0.000245651695877314
177 0.000244724738877267
178 0.000243869057158008
179 0.000243073765886948
180 0.000242239329963923
181 0.000241395377088338
182 0.000240643072174862
183 0.000239851666265167
184 0.000239110755501315
185 0.000238417953369208
186 0.000237631218624301
187 0.000236958236200735
188 0.000236237116041593
189 0.000235541723668575
190 0.000234912717132829
191 0.000234274339163676
192 0.000233604849199764
193 0.000232982361922041
194 0.000232342659728602
195 0.000231753292609937
196 0.00023111114569474
197 0.000230503748753108
198 0.000229960394790396
199 0.000229348152060993
200 0.000228783552302048
201 0.000228235570830293
202 0.000227651878958568
203 0.000227121345233172
204 0.000226548087084666
205 0.000226012314669788
206 0.000225504627451301
207 0.000224964809603989
208 0.000224462884943932
209 0.000223936032853089
210 0.000223441980779171
211 0.000222932765609585
212 0.000222417627810501
213 0.000221944705117494
214 0.000221426336793229
215 0.000220951696974225
216 0.000220486253965646
217 0.000219997586100362
218 0.000219512367038988
219 0.000218993751332164
220 0.000218546556425281
221 0.000218071261770092
222 0.000217581837205216
223 0.000217136403080076
224 0.000216651242226362
225 0.000216182554140687
226 0.000215707172174007
227 0.000215252177440561
228 0.000214794723433442
229 0.000214294734178111
230 0.00021384320280049
231 0.000213363164220937
232 0.000212926068343222
233 0.000212405371712521
234 0.000211953782127239
235 0.000211472550290637
236 0.000210979604162276
237 0.000210528654861264
238 0.00020997499814257
239 0.000209509176784195
240 0.000209017176530324
241 0.000208533514523879
242 0.000208052588277496
243 0.000207569159101695
244 0.000207042641704902
245 0.000206557786441408
246 0.00020609020430129
247 0.000205588090466335
248 0.00020508021407295
249 0.000204593321541324
250 0.000204096591915004
251 0.000203574338229373
252 0.000203020550543442
253 0.000202482508029789
254 0.000201962844585069
255 0.000201450107851997
256 0.000200894210138358
257 0.000200305265025236
258 0.000199773348867893
259 0.000199230169528164
260 0.000198654423002154
261 0.000198104535229504
262 0.000197517598280683
263 0.000196931519894861
264 0.000196357592358254
265 0.000195790562429465
266 0.000195160740986466
267 0.000194574036868289
268 0.0001939748035511
269 0.000193360057892278
270 0.000192742503713816
271 0.000192115068784915
272 0.000191466300748289
273 0.000190840641153045
274 0.000190213337191381
275 0.00018951429228764
276 0.00018885126337409
277 0.000188176680239849
278 0.000187498808372766
279 0.000186813354957849
280 0.00018612552958075
281 0.000185411437996663
282 0.000184708333108574
283 0.000183956872206181
284 0.00018322144751437
285 0.000182491596206091
286 0.000181752620846964
287 0.000180968927452341
288 0.00018016280955635
289 0.00017940960242413
290 0.00017857646162156
291 0.000177733352757059
292 0.000176878937054425
293 0.000176006913534366
294 0.000175083187059499
295 0.000174203552887775
296 0.000173228094354272
297 0.000172266489244066
298 0.000171362436958589
299 0.000170383354998194
300 0.000169343576999381
301 0.000168356898939237
302 0.000167330756084993
303 0.000166280151461251
304 0.000165301884408109
305 0.000164322278578766
306 0.000163275864906609
307 0.000162239550263621
308 0.000161175077664666
309 0.000160108698764816
310 0.000158988812472671
311 0.000157892529387027
312 0.000156791822519153
313 0.000155641027959064
314 0.000154480265337043
315 0.000153294822666794
316 0.000152135675307363
317 0.000150924504850991
318 0.000149703031638637
319 0.000148506296682172
320 0.000147255588672124
321 0.000146014790516347
322 0.000144727673614398
323 0.000143427576404065
324 0.000142172226333059
325 0.000140862015541643
326 0.000139556403155439
327 0.000138279297971167
328 0.000136948612635024
329 0.00013562674575951
330 0.000134306697873399
331 0.000132932807900943
332 0.000131574837723747
333 0.000130266285850666
334 0.000128920844872482
335 0.000127528765005991
336 0.00012618706387002
337 0.000124813333968632
338 0.000123463440104388
339 0.000122042918519583
340 0.000120664830319583
341 0.000119283075036947
342 0.000117868228699081
343 0.000116490766231436
344 0.000115072798507754
345 0.000113660913484637
346 0.000112287518277299
347 0.000110904715256765
348 0.000109504981082864
349 0.000108122920210008
350 0.000106726904050447
351 0.000105341910966672
352 0.000103939950349741
353 0.000102611775218975
354 0.000101261008239817
355 9.99398180283606e-05
356 9.86088198260404e-05
357 9.72683847066946e-05
358 9.59464450716041e-05
359 9.46154614211991e-05
360 9.33199262362905e-05
361 9.20540478546172e-05
362 9.08082438400015e-05
363 8.95518096513115e-05
364 8.83184038684703e-05
365 8.71010561240837e-05
366 8.5909639892634e-05
367 8.471046021441e-05
368 8.35383834782988e-05
369 8.24104645289481e-05
370 8.12855432741344e-05
371 8.01639980636537e-05
372 7.90875710663386e-05
373 7.8032178862486e-05
374 7.69863909226842e-05
375 7.59377362555824e-05
376 7.49346800148487e-05
377 7.39209135645069e-05
378 7.29065432096832e-05
379 7.19289819244295e-05
380 7.09471714799292e-05
381 6.99997617630288e-05
382 6.90805682097562e-05
383 6.8162327806931e-05
384 6.72692767693661e-05
385 6.64357648929581e-05
386 6.56131887808442e-05
387 6.47401757305488e-05
388 6.39647842035629e-05
389 6.31514776614495e-05
390 6.23585292487405e-05
391 6.15944300079718e-05
392 6.08604168519378e-05
393 6.01570282015018e-05
394 5.94535886193626e-05
395 5.87754402658902e-05
396 5.81371787120588e-05
397 5.74507685087156e-05
398 5.68702307646163e-05
399 5.62497370992787e-05
400 5.56839877390303e-05
401 5.51322154933587e-05
402 5.45771035831422e-05
403 5.403365867096e-05
404 5.35188082722016e-05
405 5.30192664882634e-05
406 5.25091636518482e-05
407 5.20495050295722e-05
408 5.15750289196149e-05
409 5.11243015353102e-05
410 5.06922842760105e-05
411 5.02623188367579e-05
412 4.98127519676927e-05
413 4.93913794343825e-05
414 4.8977726692101e-05
415 4.86180833831895e-05
416 4.82235191157088e-05
417 4.78343172289897e-05
418 4.74731787107885e-05
419 4.71182756882627e-05
420 4.67395402665716e-05
421 4.64484655822162e-05
422 4.61022136732936e-05
423 4.57936439488549e-05
424 4.54801338491961e-05
425 4.5142529415898e-05
426 4.48354330728762e-05
427 4.45055884483736e-05
428 4.42149794253055e-05
429 4.39140967500862e-05
430 4.36446134699509e-05
431 4.33743261964992e-05
432 4.3117230234202e-05
433 4.28691309934948e-05
434 4.26103033532854e-05
435 4.2291525460314e-05
436 4.20245414716192e-05
437 4.17718802054878e-05
438 4.15542235714383e-05
439 4.13628731621429e-05
440 4.1151786717819e-05
441 4.09320746257436e-05
442 4.07675142923836e-05
443 4.05529244744685e-05
444 4.0340000850847e-05
445 4.0096027078107e-05
446 3.98938427679241e-05
447 3.96267751057167e-05
448 3.94568669435102e-05
449 3.92966248909943e-05
450 3.91257854062133e-05
451 3.89148590329569e-05
452 3.87398358725477e-05
453 3.85505518352147e-05
454 3.83868064091075e-05
455 3.82132202503271e-05
456 3.80406891054008e-05
457 3.78641161660198e-05
458 3.77469186787494e-05
459 3.75737581634894e-05
460 3.74276751244906e-05
461 3.72437425539829e-05
462 3.71442511095665e-05
463 3.69949339074083e-05
464 3.68474684364628e-05
465 3.66898748325184e-05
466 3.66172607755288e-05
467 3.65064624929801e-05
468 3.6395216739038e-05
469 3.62493519787677e-05
470 3.60989142791368e-05
471 3.59545701940078e-05
472 3.58237048203591e-05
473 3.56966847903095e-05
474 3.5584071156336e-05
475 3.54054500348866e-05
476 3.53185096173547e-05
477 3.52289935108274e-05
478 3.51350063283462e-05
479 3.49962756445166e-05
480 3.48584217135794e-05
481 3.47324021277018e-05
482 3.45706612279173e-05
483 3.45157022820786e-05
484 3.44199652317911e-05
485 3.43395331583451e-05
486 3.41818158631213e-05
487 3.40304832207039e-05
488 3.38547615683638e-05
489 3.37726160068996e-05
490 3.36524099111557e-05
491 3.34743672283366e-05
492 3.33789648720995e-05
493 3.32661693391856e-05
494 3.31453666149173e-05
495 3.30454640788957e-05
496 3.29321519529913e-05
497 3.28089445247315e-05
498 3.27482448483352e-05
499 3.26414083247073e-05
500 3.26010340359062e-05
501 3.24738211929798e-05
502 3.23440654028673e-05
503 3.2267278584186e-05
504 3.21857623930555e-05
505 3.20376093441155e-05
506 3.19325481541455e-05
507 3.18163147312589e-05
508 3.1727602618048e-05
509 3.16737896355335e-05
510 3.15959186991677e-05
511 3.14696662826464e-05
512 3.13643176923506e-05
513 3.11883813992608e-05
514 3.1073021091288e-05
515 3.09982169710565e-05
516 3.08657690766267e-05
517 3.0749124562135e-05
518 3.06114197883289e-05
519 3.04979130305583e-05
520 3.03774704661919e-05
521 3.02444605040364e-05
522 3.01791842503008e-05
523 3.00392821372952e-05
524 2.99621715385001e-05
525 2.98776321869809e-05
526 2.97572842100635e-05
527 2.96879279630957e-05
528 2.96043381240452e-05
529 2.9534610803239e-05
530 2.95009776891675e-05
531 2.9426353648887e-05
532 2.9326836738619e-05
533 2.92694821837358e-05
534 2.92341992462752e-05
535 2.9107473892509e-05
536 2.90221505565569e-05
537 2.89485124085331e-05
538 2.8897540687467e-05
539 2.88859064312419e-05
540 2.87934290099656e-05
541 2.8703956559184e-05
542 2.86317481368314e-05
543 2.8548334739753e-05
544 2.85019359580474e-05
545 2.83818299067207e-05
546 2.83429617411457e-05
547 2.82210276054684e-05
548 2.81719894701382e-05
549 2.80931708402932e-05
550 2.80124440905638e-05
551 2.79161267826566e-05
552 2.78551178780617e-05
553 2.77569160971325e-05
554 2.7684924134519e-05
555 2.76446990028489e-05
556 2.75574439001502e-05
557 2.74544854619307e-05
558 2.73722871497739e-05
559 2.73304067377467e-05
560 2.72273082373431e-05
561 2.71947428700514e-05
562 2.71473818429513e-05
563 2.71638855338097e-05
564 2.70482560154051e-05
565 2.69615702563897e-05
566 2.69255397142842e-05
567 2.69056436081883e-05
568 2.68030998995528e-05
569 2.67817376879975e-05
570 2.66887600446353e-05
571 2.65938779193675e-05
572 2.65582020801958e-05
573 2.64962400251534e-05
574 2.64563022938091e-05
575 2.63703823293326e-05
576 2.63241290667793e-05
577 2.62383600784233e-05
578 2.61831719399197e-05
579 2.6113568310393e-05
580 2.59902462858008e-05
581 2.59787157119717e-05
582 2.59198604908306e-05
583 2.58526233665179e-05
584 2.57286064879736e-05
585 2.56906769209309e-05
586 2.56078765232814e-05
587 2.55204395216424e-05
588 2.55212889896939e-05
589 2.54636215686332e-05
590 2.54415081144543e-05
591 2.54023780144053e-05
592 2.53597518167226e-05
593 2.53030302701518e-05
594 2.53056750807445e-05
595 2.51973924605409e-05
596 2.5190356609528e-05
597 2.51602523348993e-05
598 2.50812699960079e-05
599 2.4979319277918e-05
600 2.49470940616447e-05
601 2.48914020630764e-05
602 2.48373689828441e-05
603 2.47815496550174e-05
604 2.46970375883393e-05
605 2.47198786382796e-05
606 2.4656572350068e-05
607 2.4612038032501e-05
608 2.45055653067539e-05
609 2.44725997617934e-05
610 2.44116636167746e-05
611 2.43475078605115e-05
612 2.42419737332966e-05
613 2.41925317823188e-05
614 2.41597754211398e-05
615 2.41254238062538e-05
616 2.40088265854865e-05
617 2.39787532336777e-05
618 2.39616329054115e-05
619 2.39090932154795e-05
620 2.37800031754887e-05
621 2.37610893236706e-05
622 2.37065160035854e-05
623 2.36608775594505e-05
624 2.357609810133e-05
625 2.35006409639027e-05
626 2.34545659623109e-05
627 2.34172166528879e-05
628 2.33517039305298e-05
629 2.32875372603303e-05
630 2.32480178965488e-05
631 2.31910580623662e-05
632 2.31446028919891e-05
633 2.3059126760927e-05
634 2.30398964049527e-05
635 2.30254699999932e-05
636 2.29501965804957e-05
637 2.29164797929116e-05
638 2.28194803639781e-05
639 2.27183609240456e-05
640 2.26926640607417e-05
641 2.2657317458652e-05
642 2.25331277761143e-05
643 2.2501100829686e-05
644 2.25621133722598e-05
645 2.24353698285995e-05
646 2.24282539420528e-05
647 2.23890037887031e-05
648 2.23355073103448e-05
649 2.22966973524308e-05
650 2.22352646233048e-05
651 2.22155649680644e-05
652 2.21846130443737e-05
653 2.20823440031381e-05
654 2.2071624698583e-05
655 2.2039594114176e-05
656 2.19924913835712e-05
657 2.19204321183497e-05
658 2.19434241444105e-05
659 2.18310724449111e-05
660 2.17608849197859e-05
661 2.17721844819607e-05
662 2.16716143768281e-05
663 2.16182725125691e-05
664 2.16023217944894e-05
665 2.15481195482425e-05
666 2.1481153453351e-05
667 2.14510218938813e-05
668 2.14202282222686e-05
669 2.13453949982068e-05
670 2.12899522011867e-05
671 2.13042003451847e-05
672 2.12217619264266e-05
673 2.12036866287235e-05
674 2.11946626222925e-05
675 2.10975376830902e-05
676 2.11120332096471e-05
677 2.11129263334442e-05
678 2.10546277230605e-05
679 2.09745667234529e-05
680 2.1020201529609e-05
681 2.0986624804209e-05
682 2.0914689230267e-05
683 2.08973542612512e-05
684 2.08847050089389e-05
685 2.08297460631002e-05
686 2.0807807231904e-05
687 2.07456014322815e-05
688 2.07469929591753e-05
689 2.07147859327961e-05
690 2.06384993362008e-05
691 2.06224103749264e-05
692 2.06398599402746e-05
693 2.06030781555455e-05
694 2.05357519007521e-05
695 2.04864318220643e-05
696 2.04406478587771e-05
697 2.03745366889052e-05
698 2.04021234821994e-05
699 2.03523868549382e-05
700 2.03607887669932e-05
701 2.03268737095641e-05
702 2.02493611141108e-05
703 2.02720275410684e-05
704 2.01627099158941e-05
705 2.00652466446627e-05
706 2.00357790163253e-05
707 1.99539244931657e-05
708 1.98407160496572e-05
709 1.98097641259665e-05
710 1.97500467038481e-05
711 1.96461733139586e-05
712 1.96050132217351e-05
713 1.95849006559001e-05
714 1.95050815818831e-05
715 1.94393014680827e-05
716 1.9390838133404e-05
717 1.92836523638107e-05
718 1.93174146261299e-05
719 1.92663301277207e-05
720 1.92430288734613e-05
721 1.92192928807344e-05
722 1.91789367818274e-05
723 1.90665505215293e-05
724 1.91215713130077e-05
725 1.90481950994581e-05
726 1.90145747183124e-05
727 1.89938800758682e-05
728 1.90215396287385e-05
729 1.89549919014098e-05
730 1.90127375390148e-05
731 1.89824895642232e-05
732 1.89981674338924e-05
733 1.89704878721386e-05
734 1.90423361345893e-05
735 1.90263908734778e-05
736 1.90323953574989e-05
737 1.90421797015006e-05
738 1.90337214007741e-05
739 1.89969141501933e-05
740 1.90588671102887e-05
741 1.90280916285701e-05
742 1.90840873983689e-05
743 1.90391783689847e-05
744 1.90791088243714e-05
745 1.89879501704127e-05
746 1.90211776498472e-05
747 1.89949587365845e-05
748 1.90158207260538e-05
749 1.89708371181041e-05
750 1.89736438187538e-05
751 1.89679212780902e-05
752 1.89554248208879e-05
753 1.8864675439545e-05
754 1.88443500519497e-05
755 1.89017991942819e-05
756 1.88971698662499e-05
757 1.88241720024962e-05
758 1.88630692719016e-05
759 1.881681055238e-05
760 1.88324411283247e-05
761 1.87290333997225e-05
762 1.87030527740717e-05
763 1.86861125257565e-05
764 1.87221758096712e-05
765 1.8619268303155e-05
766 1.8584105418995e-05
767 1.85758999577956e-05
768 1.85368608072167e-05
769 1.84641412488418e-05
770 1.83963620656868e-05
771 1.84143991646124e-05
772 1.84134660230484e-05
773 1.83347201527795e-05
774 1.82938147190725e-05
775 1.82886033144314e-05
776 1.82147377927322e-05
777 1.8204671505373e-05
778 1.82494350156048e-05
779 1.8115733837476e-05
780 1.81630784936715e-05
781 1.81477007572539e-05
782 1.81101004272932e-05
783 1.81187006091932e-05
784 1.81855557457311e-05
785 1.80422302946681e-05
786 1.80649640242336e-05
787 1.80756378540536e-05
788 1.80797833309043e-05
789 1.7991578715737e-05
790 1.80221468326636e-05
791 1.79187136382097e-05
792 1.80142360477475e-05
793 1.79221333382884e-05
794 1.7902018953464e-05
795 1.79010021383874e-05
796 1.79618091351585e-05
797 1.77964902832173e-05
798 1.78458794835024e-05
799 1.78997906914447e-05
800 1.78857462742599e-05
801 1.78445425262908e-05
802 1.79392773134168e-05
803 1.79138114617672e-05
804 1.79721646418329e-05
805 1.79603503056569e-05
806 1.79380713234423e-05
807 1.79320741153788e-05
808 1.7962762285606e-05
809 1.78456521098269e-05
810 1.7889073205879e-05
811 1.79865146492375e-05
812 1.78417540155351e-05
813 1.78923692146782e-05
814 1.804024213925e-05
815 1.79154230863787e-05
816 1.80172610271256e-05
817 1.79255948751234e-05
818 1.79733124241466e-05
819 1.79558355739573e-05
820 1.80135757545941e-05
821 1.78743957803817e-05
822 1.79319740709616e-05
823 1.79533089976758e-05
824 1.78211648744764e-05
825 1.7796051906771e-05
826 1.78962709469488e-05
827 1.78439131559571e-05
828 1.78474820131669e-05
829 1.79643666342599e-05
830 1.78276804945199e-05
831 1.79121543624206e-05
832 1.78393984242575e-05
833 1.79679391294485e-05
834 1.7968599422602e-05
835 1.78562986548059e-05
836 1.78321151906857e-05
837 1.78386835614219e-05
838 1.79286580532789e-05
839 1.7821077562985e-05
840 1.79184553417144e-05
841 1.78753289219458e-05
842 1.78456539288163e-05
843 1.79141898115631e-05
844 1.78598093043547e-05
845 1.77159072336508e-05
846 1.77952842932427e-05
847 1.77305810211692e-05
848 1.75986133399419e-05
849 1.76569574250607e-05
850 1.76530884345993e-05
851 1.75752811628627e-05
852 1.76179019035771e-05
853 1.75772438524291e-05
854 1.76187841134379e-05
855 1.75196619238704e-05
856 1.74894630617928e-05
857 1.75676432263572e-05
858 1.75409459188813e-05
859 1.73336447915062e-05
860 1.74402448465116e-05
861 1.74508513737237e-05
862 1.74616870936006e-05
863 1.7292490156251e-05
864 1.74000688275555e-05
865 1.72795789694646e-05
866 1.73679800354876e-05
867 1.72953368746676e-05
868 1.72628569998778e-05
869 1.7280977772316e-05
870 1.7336600649287e-05
871 1.71309220604599e-05
872 1.72167474374874e-05
873 1.72646541614085e-05
874 1.7214622857864e-05
875 1.71060491993558e-05
876 1.72071777342353e-05
877 1.70869843714172e-05
878 1.71256560861366e-05
879 1.70480107044568e-05
880 1.70062303368468e-05
881 1.70298426382942e-05
882 1.70454950421117e-05
883 1.69092036230722e-05
884 1.69820905284723e-05
885 1.69672312040348e-05
886 1.6884672731976e-05
887 1.6842495824676e-05
888 1.68918904819293e-05
889 1.6823902114993e-05
890 1.68972073879559e-05
891 1.68466449395055e-05
892 1.67654889082769e-05
893 1.68291680893162e-05
894 1.68800670508062e-05
895 1.66414556588279e-05
896 1.67398920893902e-05
897 1.66869485838106e-05
898 1.66507034009555e-05
899 1.66180670930771e-05
900 1.66223017004086e-05
901 1.65894416568335e-05
902 1.66176632774295e-05
903 1.65245128300739e-05
904 1.6582709577051e-05
905 1.6507839973201e-05
906 1.657694883761e-05
907 1.64593275258085e-05
908 1.64350112754619e-05
909 1.64391312864609e-05
910 1.64476023201132e-05
911 1.61768184625544e-05
912 1.62999931490049e-05
913 1.62882643053308e-05
914 1.62887336045969e-05
915 1.61510324687697e-05
916 1.6187941582757e-05
917 1.61719199240906e-05
918 1.61695152200991e-05
919 1.60627369041322e-05
920 1.6054149455158e-05
921 1.60816744028125e-05
922 1.60685303853825e-05
923 1.57665599544998e-05
924 1.58905677380972e-05
925 1.58119310071925e-05
926 1.58432721946156e-05
927 1.56212736328598e-05
928 1.56640180648537e-05
929 1.56259120558389e-05
930 1.55743327923119e-05
931 1.55663492478197e-05
932 1.55377765622688e-05
933 1.5457944755326e-05
934 1.53976616275031e-05
935 1.53465534822317e-05
936 1.5273875760613e-05
937 1.51673939399188e-05
938 1.50653777382104e-05
939 1.49485731526511e-05
940 1.4939546417736e-05
941 1.48496537804022e-05
942 1.47757173181162e-05
943 1.46681932164938e-05
944 1.46460897667566e-05
945 1.4517688214255e-05
946 1.44861860462697e-05
947 1.440436244593e-05
948 1.43114530146704e-05
949 1.41576974783675e-05
950 1.41655400511809e-05
951 1.40286228997866e-05
952 1.40048086905153e-05
953 1.3847105037712e-05
954 1.38204122777097e-05
955 1.37659981191973e-05
956 1.37286824610783e-05
957 1.36104099510703e-05
958 1.35915570353973e-05
959 1.3511626093532e-05
960 1.34937699840521e-05
961 1.3363274774747e-05
962 1.33729809022043e-05
963 1.33783414639765e-05
964 1.33022449517739e-05
965 1.32141258291085e-05
966 1.32327586470637e-05
967 1.31627075461438e-05
968 1.31267870528973e-05
969 1.31355864141369e-05
970 1.30960133901681e-05
971 1.30384914882598e-05
972 1.30379467009334e-05
973 1.30087328216177e-05
974 1.29605732581695e-05
975 1.29612408272806e-05
976 1.297060498473e-05
977 1.29030377138406e-05
978 1.29185973491985e-05
979 1.28886804304784e-05
980 1.28543042592355e-05
981 1.28532738017384e-05
982 1.2814672118111e-05
983 1.28239107652917e-05
984 1.27941511891549e-05
985 1.27877683553379e-05
986 1.27456469272147e-05
987 1.27320436149603e-05
988 1.2767070074915e-05
989 1.27030998555711e-05
990 1.27083712868625e-05
991 1.27069824884529e-05
992 1.27037674246822e-05
993 1.26834102047724e-05
994 1.26389741126332e-05
995 1.26922122944961e-05
996 1.26286231534323e-05
997 1.26308987091761e-05
998 1.25931201182539e-05
999 1.26239028759301e-05
1000 1.25766955534345e-05
1001 1.25581500469707e-05
1002 1.25992428365862e-05
1003 1.25504102470586e-05
1004 1.25499527712236e-05
1005 1.25642873172183e-05
1006 1.25435144582298e-05
1007 1.25167916849023e-05
1008 1.25115393529995e-05
1009 1.24908883663011e-05
1010 1.24972084449837e-05
1011 1.24777880046167e-05
1012 1.25056176329963e-05
1013 1.25040096463636e-05
1014 1.24800935736857e-05
1015 1.24915823107585e-05
1016 1.24493226394407e-05
1017 1.25628785099252e-05
1018 1.24597308968077e-05
1019 1.28436713566771e-05
1020 1.28906995087164e-05
1021 1.41032423925935e-05
1022 1.51827462104848e-05
1023 1.73347998497775e-05
1024 1.71209649124648e-05
1025 1.46150350701646e-05
1026 1.29493664644542e-05
1027 1.23581567095243e-05
1028 1.25465448945761e-05
1029 1.24168855109019e-05
1030 1.2592104212672e-05
1031 1.2513288311311e-05
1032 1.24777379824081e-05
1033 1.23814370454056e-05
1034 1.2363104360702e-05
1035 1.23223571790732e-05
1036 1.22672790894285e-05
1037 1.23495738080237e-05
1038 1.22788569569821e-05
1039 1.2387596143526e-05
1040 1.23244653877919e-05
1041 1.24915004562354e-05
1042 1.23732843348989e-05
1043 1.25422411656473e-05
1044 1.24356947708293e-05
1045 1.25846800074214e-05
1046 1.25081533042248e-05
1047 1.27769708342385e-05
1048 1.27344246720895e-05
1049 1.31751157823601e-05
1050 1.317670194112e-05
1051 1.37746465043165e-05
1052 1.37783063109964e-05
1053 1.38847999551217e-05
1054 1.36810376716312e-05
1055 1.35875343403313e-05
1056 1.27807015815051e-05
1057 1.25115720948088e-05
1058 1.22961228044005e-05
1059 1.21451457744115e-05
1060 1.22717365229619e-05
1061 1.22887377074221e-05
1062 1.25640344776912e-05
1063 1.26828626889619e-05
1064 1.31494416564237e-05
1065 1.31330307340249e-05
1066 1.3345278603083e-05
1067 1.30741591419792e-05
1068 1.2926555427839e-05
1069 1.26384429677273e-05
1070 1.25877977552591e-05
1071 1.22896344691981e-05
1072 1.21843513625208e-05
1073 1.20512149806018e-05
1074 1.19664082376403e-05
1075 1.19626938612782e-05
1076 1.19372571134591e-05
1077 1.21874936667155e-05
1078 1.2221230463183e-05
1079 1.29506597659201e-05
1080 1.32554350784631e-05
1081 1.47487189678941e-05
1082 1.48609751704498e-05
1083 1.40169540827628e-05
1084 1.31104015963501e-05
1085 1.23428399092518e-05
1086 1.20283993965131e-05
1087 1.18311718324549e-05
1088 1.20588701975066e-05
1089 1.20997310659732e-05
1090 1.23488825920504e-05
1091 1.2251272892172e-05
1092 1.22756882774411e-05
1093 1.2133137715864e-05
1094 1.22108012874378e-05
1095 1.20305521704722e-05
1096 1.20477270684205e-05
1097 1.1927522791666e-05
1098 1.19160067697521e-05
1099 1.18618272608728e-05
1100 1.18615716928616e-05
1101 1.18610687422915e-05
1102 1.18489824671997e-05
1103 1.20450058602728e-05
1104 1.20456543299952e-05
1105 1.2604799849214e-05
1106 1.29385989566799e-05
1107 1.37694278237177e-05
1108 1.39689618663397e-05
1109 1.42898616104503e-05
1110 1.33142184495227e-05
1111 1.23013542179251e-05
1112 1.20654331112746e-05
1113 1.18381585707539e-05
1114 1.1831037227239e-05
1115 1.17512909127981e-05
1116 1.18923298941809e-05
1117 1.18758243843331e-05
1118 1.21598050100147e-05
1119 1.21600714919623e-05
1120 1.24371827041614e-05
1121 1.24404341477202e-05
1122 1.26427858049283e-05
1123 1.26054192151059e-05
1124 1.27995726870722e-05
1125 1.24018833957962e-05
1126 1.22639057735796e-05
1127 1.20197919386555e-05
1128 1.1826468835352e-05
1129 1.17881827463862e-05
1130 1.173954387923e-05
1131 1.17012486953172e-05
1132 1.16999535748619e-05
1133 1.17453437269432e-05
1134 1.16972605610499e-05
1135 1.18454472612939e-05
1136 1.18140069389483e-05
1137 1.21865468827309e-05
1138 1.23973868539906e-05
1139 1.34166075440589e-05
1140 1.42754342959961e-05
1141 1.39047206175746e-05
1142 1.44404821185162e-05
1143 1.4072036719881e-05
1144 1.2750086170854e-05
1145 1.17143245006446e-05
1146 1.17025656436454e-05
1147 1.16043420348433e-05
1148 1.17623121695942e-05
1149 1.18385869427584e-05
1150 1.19906462714425e-05
1151 1.19832466225489e-05
1152 1.19564729175181e-05
1153 1.18578791443724e-05
1154 1.17811214295216e-05
1155 1.17121480798232e-05
1156 1.16202763820183e-05
1157 1.15781776912627e-05
1158 1.16087658170727e-05
1159 1.16031369543634e-05
1160 1.16029350465396e-05
1161 1.1683091543091e-05
1162 1.16925166366855e-05
1163 1.19854639706318e-05
1164 1.20911472549778e-05
1165 1.26854965856182e-05
1166 1.32244640553836e-05
1167 1.36262178784818e-05
1168 1.33076428028289e-05
1169 1.26573413581355e-05
1170 1.22721221487154e-05
1171 1.18154239316937e-05
1172 1.17003746709088e-05
1173 1.14900285552721e-05
1174 1.15860239020549e-05
1175 1.15138518594904e-05
1176 1.16272040031618e-05
1177 1.16504625111702e-05
1178 1.17227109512896e-05
1179 1.18263978947653e-05
1180 1.19782180263428e-05
1181 1.21101211334462e-05
1182 1.22489482237143e-05
1183 1.25821970868856e-05
1184 1.27104667626554e-05
1185 1.29106292661163e-05
1186 1.27196753965109e-05
1187 1.26560298667755e-05
1188 1.2170901754871e-05
1189 1.21320263133384e-05
1190 1.19012147479225e-05
1191 1.18052867037477e-05
1192 1.172588144982e-05
1193 1.1641579476418e-05
1194 1.16048067866359e-05
1195 1.15887723950436e-05
1196 1.15548764370033e-05
1197 1.15337961688056e-05
1198 1.15347274913802e-05
1199 1.15371476567816e-05
1200 1.15218781502335e-05
1201 1.15428210847313e-05
1202 1.15393704618327e-05
1203 1.16133778647054e-05
1204 1.16304436232895e-05
1205 1.19935984912445e-05
1206 1.23723821161548e-05
1207 1.43937522807391e-05
1208 1.62555807037279e-05
1209 1.56210917339195e-05
1210 1.3044870684098e-05
1211 1.1677367183438e-05
1212 1.15868788270745e-05
1213 1.14289050543448e-05
1214 1.15506254587672e-05
1215 1.15733755592373e-05
1216 1.15757211460732e-05
1217 1.15540588012664e-05
1218 1.15488228402683e-05
1219 1.15126003947807e-05
1220 1.15017410280416e-05
1221 1.14798740469269e-05
1222 1.14377216959838e-05
1223 1.14350423245924e-05
1224 1.14347112685209e-05
1225 1.14286840471323e-05
1226 1.14508775368449e-05
1227 1.15103466669098e-05
1228 1.15040502350894e-05
1229 1.16072515083943e-05
1230 1.16762394100078e-05
1231 1.19704936878406e-05
1232 1.22038318295381e-05
1233 1.27393986986135e-05
1234 1.35028922159108e-05
1235 1.3724633390666e-05
1236 1.31188808154548e-05
1237 1.23784348033951e-05
1238 1.1986954632448e-05
1239 1.1435299711593e-05
1240 1.14267495519016e-05
1241 1.12879497464746e-05
1242 1.13621081254678e-05
1243 1.13832675197045e-05
1244 1.1536225429154e-05
1245 1.16041865112493e-05
1246 1.18068446681718e-05
1247 1.18033312901389e-05
1248 1.19588148663752e-05
1249 1.21425309771439e-05
1250 1.2331558536971e-05
1251 1.22081355584669e-05
1252 1.20166596389026e-05
1253 1.19653686851962e-05
1254 1.16886558316764e-05
1255 1.1622979400272e-05
1256 1.14370041046641e-05
1257 1.14374615804991e-05
1258 1.14582308015088e-05
1259 1.144584530266e-05
1260 1.15251168608665e-05
1261 1.15069688035874e-05
1262 1.16500023068511e-05
1263 1.17871568363626e-05
1264 1.22351311802049e-05
1265 1.27078292280203e-05
1266 1.29439258671482e-05
1267 1.32787818074576e-05
1268 1.28224573927582e-05
1269 1.28739829960978e-05
1270 1.22495403047651e-05
1271 1.19949518193607e-05
1272 1.14682488856488e-05
1273 1.13940996016026e-05
1274 1.12335228550364e-05
1275 1.13344867713749e-05
1276 1.14066697278759e-05
1277 1.15988004836254e-05
1278 1.17548161142622e-05
1279 1.20073609650717e-05
1280 1.22543406178011e-05
1281 1.2199035154481e-05
1282 1.24064863484818e-05
1283 1.26955301311682e-05
1284 1.26572458611918e-05
1285 1.25012784337741e-05
1286 1.2161142876721e-05
1287 1.16547744255513e-05
1288 1.14520125862327e-05
1289 1.1262319276284e-05
1290 1.13742307803477e-05
1291 1.14859476525453e-05
1292 1.2030452126055e-05
1293 1.24202351798886e-05
1294 1.30837224787683e-05
1295 1.28679212139104e-05
1296 1.19417900350527e-05
1297 1.1706702025549e-05
1298 1.13155065264436e-05
1299 1.13384794531157e-05
1300 1.12811212602537e-05
1301 1.14474933070596e-05
1302 1.15767688839696e-05
1303 1.17576146294596e-05
1304 1.18509306048509e-05
1305 1.19146998258657e-05
1306 1.18759498946019e-05
1307 1.17225990834413e-05
1308 1.16978735604789e-05
1309 1.16629598778673e-05
1310 1.15186494440422e-05
1311 1.14053527795477e-05
1312 1.13522837636992e-05
1313 1.12830475700321e-05
1314 1.12825800897554e-05
1315 1.12434991024202e-05
1316 1.13091236926266e-05
1317 1.13114856503671e-05
1318 1.15082702905056e-05
1319 1.16824621727574e-05
1320 1.22777801152552e-05
1321 1.31000006149407e-05
1322 1.43757188197924e-05
1323 1.48862118294346e-05
1324 1.34867441374809e-05
1325 1.23084209917579e-05
1326 1.12836223706836e-05
1327 1.12938896563719e-05
1328 1.12950438051485e-05
1329 1.16099736260367e-05
1330 1.18653597382945e-05
1331 1.21099492389476e-05
1332 1.20787080959417e-05
1333 1.18684974950156e-05
1334 1.16412402348942e-05
1335 1.13344185592723e-05
1336 1.1305291991448e-05
1337 1.122427147493e-05
1338 1.1425854609115e-05
1339 1.15885868581245e-05
1340 1.19715850814828e-05
1341 1.21697376016527e-05
1342 1.24589450933854e-05
1343 1.23889913083985e-05
1344 1.22048613775405e-05
1345 1.18601465146639e-05
1346 1.14533977466635e-05
1347 1.13435517050675e-05
1348 1.1213484867767e-05
1349 1.14231124825892e-05
1350 1.1543434993655e-05
1351 1.20850008897833e-05
1352 1.24335774671636e-05
1353 1.29775462482939e-05
1354 1.2572009836731e-05
1355 1.19874948723009e-05
1356 1.16239943963592e-05
1357 1.125109338318e-05
1358 1.13164260255871e-05
1359 1.12605421236367e-05
1360 1.14914337245864e-05
1361 1.16229784907773e-05
1362 1.18974476208678e-05
1363 1.20002441690303e-05
1364 1.20064805742004e-05
1365 1.19770138553577e-05
1366 1.17446907097474e-05
1367 1.17165727715474e-05
1368 1.15807752081309e-05
1369 1.14783661047113e-05
1370 1.13652740765247e-05
1371 1.13179039544775e-05
1372 1.12271445686929e-05
1373 1.12496027213638e-05
1374 1.12522793642711e-05
1375 1.13586684165057e-05
1376 1.14275389933027e-05
1377 1.19198703032453e-05
1378 1.24863163364353e-05
1379 1.40653401103918e-05
1380 1.45779631566256e-05
1381 1.30387843455537e-05
1382 1.2352232261037e-05
1383 1.14453923742985e-05
1384 1.13369778773631e-05
1385 1.11851541078067e-05
1386 1.13500564111746e-05
1387 1.14982976811007e-05
1388 1.17133831736282e-05
1389 1.17614536065958e-05
1390 1.17185600174707e-05
1391 1.16808005259372e-05
1392 1.15073189590476e-05
1393 1.14182939796592e-05
1394 1.12467942017247e-05
1395 1.12219022412319e-05
1396 1.11997414933285e-05
1397 1.12794768938329e-05
1398 1.13277510536136e-05
1399 1.16739556688117e-05
1400 1.20231607070309e-05
1401 1.25442647913587e-05
1402 1.30114012790727e-05
1403 1.26780623759259e-05
1404 1.25877641039551e-05
1405 1.1966375495831e-05
1406 1.1570827155083e-05
1407 1.11932795334724e-05
1408 1.12369552880409e-05
1409 1.11840854515322e-05
1410 1.13404339572298e-05
1411 1.14305630631861e-05
1412 1.14904678412131e-05
1413 1.16846331366105e-05
1414 1.17230174510041e-05
1415 1.17224999485188e-05
1416 1.15571301648743e-05
1417 1.17564768515876e-05
1418 1.20144532047561e-05
1419 1.20870590762934e-05
1420 1.20292443170911e-05
1421 1.20090298878495e-05
1422 1.17997924462543e-05
1423 1.17177987704054e-05
1424 1.14078084152425e-05
1425 1.13810983748408e-05
1426 1.12718789750943e-05
1427 1.1258174708928e-05
1428 1.11881408884074e-05
1429 1.11856397779775e-05
1430 1.11730423668632e-05
1431 1.12220386654371e-05
1432 1.1197562344023e-05
1433 1.13770347525133e-05
1434 1.15600705612451e-05
1435 1.23635763884522e-05
1436 1.36003554871422e-05
1437 1.40077154355822e-05
1438 1.45090871228604e-05
1439 1.29160762298852e-05
1440 1.21065277198795e-05
1441 1.11366571218241e-05
1442 1.11545696199755e-05
1443 1.11026965896599e-05
1444 1.13129653982469e-05
1445 1.15542698040372e-05
1446 1.17834442789899e-05
1447 1.19715432447265e-05
1448 1.19016031021602e-05
1449 1.17907129606465e-05
1450 1.14718604891095e-05
1451 1.13106571006938e-05
1452 1.11106846816256e-05
1453 1.11763529275777e-05
1454 1.12280877146986e-05
1455 1.15553302748594e-05
1456 1.19236829050351e-05
1457 1.25022452266421e-05
1458 1.28757865240914e-05
1459 1.29894624478766e-05
1460 1.24865118777961e-05
1461 1.16017945401836e-05
1462 1.13644000521163e-05
1463 1.10867349576438e-05
1464 1.13116739157704e-05
1465 1.1660707059491e-05
1466 1.21967659652e-05
1467 1.25586220747209e-05
1468 1.25550313896383e-05
1469 1.21683524412219e-05
1470 1.14688882604241e-05
1471 1.126021197706e-05
1472 1.10961154859979e-05
1473 1.13431187855895e-05
1474 1.16699766294914e-05
1475 1.21123621283914e-05
1476 1.23114896268817e-05
1477 1.21998100439669e-05
1478 1.18914740596665e-05
1479 1.13610904008965e-05
1480 1.1234704288654e-05
1481 1.10872606455814e-05
1482 1.13159076136071e-05
1483 1.16174815047998e-05
1484 1.20778558994061e-05
1485 1.23227991934982e-05
1486 1.23476684166235e-05
1487 1.20982022053795e-05
1488 1.15543716674438e-05
1489 1.12953239295166e-05
1490 1.1045661267417e-05
1491 1.12685511339805e-05
1492 1.15122084025643e-05
1493 1.20225204227609e-05
1494 1.23950812849216e-05
1495 1.25536635096068e-05
1496 1.22421324704192e-05
1497 1.15131078928243e-05
1498 1.12966772576328e-05
1499 1.10328483060584e-05
1500 1.12255956992158e-05
1501 1.1460913810879e-05
1502 1.18800535346963e-05
1503 1.2187171705591e-05
1504 1.22971732707811e-05
1505 1.20929516924662e-05
1506 1.15654856926994e-05
1507 1.1340948731231e-05
1508 1.10431046778103e-05
1509 1.11610561361886e-05
1510 1.1301949598419e-05
1511 1.1737583008653e-05
1512 1.22018836918869e-05
1513 1.25982069221209e-05
1514 1.25073938761489e-05
1515 1.18604921226506e-05
1516 1.16120927486918e-05
1517 1.11362069219467e-05
1518 1.11483795990353e-05
1519 1.11600602394901e-05
1520 1.15172915684525e-05
1521 1.19699943752494e-05
1522 1.23900426842738e-05
1523 1.25414144349634e-05
1524 1.2226497347001e-05
1525 1.18651196316932e-05
1526 1.12808056655922e-05
1527 1.11581503006164e-05
1528 1.11017297967919e-05
1529 1.14415342977736e-05
1530 1.18983371066861e-05
1531 1.23323734442238e-05
1532 1.24076168503962e-05
1533 1.2154857358837e-05
1534 1.14989197754767e-05
1535 1.10790851977072e-05
1536 1.11544914034312e-05
1537 1.11580293378211e-05
1538 1.13474861791474e-05
1539 1.14946042231168e-05
1540 1.14968934212811e-05
1541 1.16375831566984e-05
1542 1.16175042421673e-05
1543 1.15302200356382e-05
1544 1.12902425826178e-05
1545 1.12552697828505e-05
1546 1.1043224731111e-05
1547 1.10883356683189e-05
1548 1.10512773972005e-05
1549 1.11267299871542e-05
1550 1.13053329187096e-05
1551 1.15860202640761e-05
1552 1.20457552839071e-05
1553 1.25733758977731e-05
1554 1.31537035485962e-05
1555 1.2808185601898e-05
1556 1.23036361401319e-05
1557 1.15842476589023e-05
1558 1.12771258500288e-05
1559 1.09343181975419e-05
1560 1.10524506453658e-05
1561 1.11711988211027e-05
1562 1.14407748696976e-05
1563 1.17955596579122e-05
1564 1.2025389878545e-05
1565 1.20160311780637e-05
1566 1.1707626072166e-05
1567 1.13737241917988e-05
1568 1.10662031147513e-05
1569 1.11114286482916e-05
1570 1.09870170490467e-05
1571 1.10744222183712e-05
1572 1.12381003418704e-05
1573 1.14229587779846e-05
1574 1.17228955787141e-05
1575 1.18126736197155e-05
1576 1.23196223285049e-05
1577 1.2193371730973e-05
1578 1.21362418212811e-05
1579 1.16970250019222e-05
1580 1.16189257823862e-05
1581 1.1193762475159e-05
1582 1.1187129530299e-05
1583 1.09871834865771e-05
1584 1.10509345176979e-05
1585 1.10239379864652e-05
1586 1.11112667582347e-05
1587 1.13548376248218e-05
1588 1.1641710443655e-05
1589 1.22368364827707e-05
1590 1.24359339679359e-05
1591 1.29207828649669e-05
1592 1.24710550153395e-05
1593 1.25635706353933e-05
1594 1.1724930118362e-05
1595 1.15090188046452e-05
1596 1.09754009827157e-05
1597 1.10370792754111e-05
1598 1.11067538455245e-05
1599 1.12932748379535e-05
1600 1.17502386274282e-05
1601 1.18666002890677e-05
1602 1.23153085951344e-05
1603 1.21104922072846e-05
1604 1.19141295726877e-05
1605 1.13051810330944e-05
1606 1.12298421299784e-05
1607 1.09863240140839e-05
1608 1.10921127998154e-05
1609 1.12595334940124e-05
1610 1.14827653305838e-05
1611 1.20487493404653e-05
1612 1.21828334158636e-05
1613 1.24558027891908e-05
1614 1.19823271234054e-05
1615 1.18584866868332e-05
1616 1.12374627860845e-05
1617 1.11659537651576e-05
1618 1.09865432023071e-05
1619 1.1098347385996e-05
1620 1.13246560431435e-05
1621 1.15452658064896e-05
1622 1.2020693247905e-05
1623 1.21093862617272e-05
1624 1.23928848552168e-05
1625 1.20718577818479e-05
1626 1.17066674647504e-05
1627 1.11975996333058e-05
1628 1.11387817014474e-05
1629 1.09924640128156e-05
1630 1.11490462586517e-05
1631 1.15049733722117e-05
1632 1.17372319436981e-05
1633 1.23633681141655e-05
1634 1.24194511954556e-05
1635 1.22879800983355e-05
1636 1.16646633614437e-05
1637 1.1461554095149e-05
1638 1.09703969428665e-05
1639 1.1066947990912e-05
1640 1.11508325062459e-05
1641 1.13375481305411e-05
1642 1.17715981104993e-05
1643 1.19130254461197e-05
1644 1.22153396659996e-05
1645 1.18431562441401e-05
1646 1.19508722491446e-05
1647 1.15556267701322e-05
1648 1.14415479401941e-05
1649 1.10678865894442e-05
1650 1.10610944830114e-05
1651 1.10733999463264e-05
1652 1.13463647721801e-05
1653 1.19341111712856e-05
1654 1.25889746414032e-05
1655 1.28228039102396e-05
1656 1.19806554721436e-05
1657 1.20483928185422e-05
1658 1.15131051643402e-05
1659 1.13404621515656e-05
1660 1.09637612695224e-05
1661 1.10694363684161e-05
1662 1.12888410512824e-05
1663 1.17364534162334e-05
1664 1.24129564937903e-05
1665 1.24811522255186e-05
1666 1.22480969366734e-05
1667 1.15843731691712e-05
1668 1.1330308552715e-05
1669 1.09379716377589e-05
1670 1.10901964944787e-05
1671 1.13795613287948e-05
1672 1.1764301234507e-05
1673 1.20574914035387e-05
1674 1.20055137813324e-05
1675 1.19716823974159e-05
1676 1.13695678010117e-05
1677 1.12946681838366e-05
1678 1.09625261757174e-05
1679 1.10317923827097e-05
1680 1.11751996882958e-05
1681 1.14623026092886e-05
1682 1.2043544302287e-05
1683 1.24581583804684e-05
1684 1.27432940644212e-05
1685 1.23751715364051e-05
1686 1.19465075840708e-05
1687 1.11626513898955e-05
1688 1.10889968709671e-05
1689 1.11545095933252e-05
1690 1.16138317025616e-05
1691 1.23192794490024e-05
1692 1.26009581435937e-05
1693 1.24261750897858e-05
1694 1.15712164188153e-05
1695 1.12889656520565e-05
1696 1.09390912257368e-05
1697 1.11775680124993e-05
1698 1.16282753879204e-05
1699 1.19468213597429e-05
1700 1.2220556527609e-05
1701 1.18500474854955e-05
1702 1.15416669359547e-05
1703 1.10367818706436e-05
1704 1.10658138510189e-05
1705 1.1155437277921e-05
1706 1.15200818981975e-05
1707 1.19953210742096e-05
1708 1.21044295156025e-05
1709 1.20827153295977e-05
1710 1.15383545562509e-05
1711 1.12876241473714e-05
1712 1.09453030745499e-05
1713 1.1106593774457e-05
1714 1.14097683763248e-05
1715 1.1838359569083e-05
1716 1.23447580335778e-05
1717 1.23312947835075e-05
1718 1.1829585673695e-05
1719 1.10901801235741e-05
1720 1.10742103061057e-05
1721 1.09719139800291e-05
1722 1.11616091089672e-05
1723 1.14543590825633e-05
1724 1.1476717190817e-05
1725 1.17458421300398e-05
1726 1.15638567876886e-05
1727 1.14818030851893e-05
1728 1.11249310066341e-05
1729 1.11417239168077e-05
1730 1.0950651812891e-05
1731 1.09930660983082e-05
1732 1.10625578599866e-05
1733 1.12674133561086e-05
1734 1.17647214210592e-05
1735 1.2425671229721e-05
1736 1.32397408378893e-05
1737 1.32139348352212e-05
1738 1.21350276458543e-05
1739 1.12010957309394e-05
1740 1.10888158815214e-05
1741 1.09078400782892e-05
1742 1.10171222331701e-05
1743 1.12035168058355e-05
1744 1.12325587906525e-05
1745 1.1411899322411e-05
1746 1.1301223821647e-05
1747 1.15293587441556e-05
1748 1.14524900709512e-05
1749 1.14094818854937e-05
1750 1.11426252260571e-05
1751 1.12488251033938e-05
1752 1.11514109448763e-05
1753 1.1158425877511e-05
1754 1.10046985355439e-05
1755 1.10811688500689e-05
1756 1.10238934212248e-05
1757 1.10367127490463e-05
1758 1.09919483293197e-05
1759 1.10145356302382e-05
1760 1.10070359369274e-05
1761 1.10437595139956e-05
1762 1.1126749086543e-05
1763 1.13574587885523e-05
1764 1.20189588415087e-05
1765 1.34775500555406e-05
1766 1.49342722579604e-05
1767 1.48137805808801e-05
1768 1.29185000332654e-05
1769 1.13844671432162e-05
1770 1.08175145214773e-05
1771 1.09751963464078e-05
1772 1.11285307866638e-05
1773 1.11744766400079e-05
1774 1.13981404865626e-05
1775 1.10557812149636e-05
1776 1.11289073174703e-05
1777 1.09356160464813e-05
1778 1.09450102172559e-05
1779 1.091078047466e-05
1780 1.10151668195613e-05
1781 1.11382778413827e-05
1782 1.13253099698341e-05
1783 1.16384608190856e-05
1784 1.16254168460728e-05
1785 1.18614443636034e-05
1786 1.16171686386224e-05
1787 1.15201410153531e-05
1788 1.10566670628032e-05
1789 1.1131247447338e-05
1790 1.08859549072804e-05
1791 1.09190641524037e-05
1792 1.09310321931844e-05
1793 1.11192302938434e-05
1794 1.15925759018864e-05
1795 1.24187708934187e-05
1796 1.36099033625214e-05
1797 1.32820960061508e-05
1798 1.25708320410922e-05
1799 1.15326083687251e-05
1800 1.10898627099232e-05
1801 1.0845477845578e-05
1802 1.09426873677876e-05
1803 1.10907594716991e-05
1804 1.11115441541187e-05
1805 1.13316209535697e-05
1806 1.11352483145311e-05
1807 1.12685756903375e-05
1808 1.10453602246707e-05
1809 1.10913015305414e-05
1810 1.09504544525407e-05
1811 1.09903894554009e-05
1812 1.0875824955292e-05
1813 1.09130678538349e-05
1814 1.08882513814024e-05
1815 1.09462425825768e-05
1816 1.09777802208555e-05
1817 1.10880009742687e-05
1818 1.14532676889212e-05
1819 1.21075518109137e-05
1820 1.30037542476202e-05
1821 1.3176494576328e-05
1822 1.44252717291238e-05
1823 1.22367291623959e-05
1824 1.25891920106369e-05
1825 1.14038120955229e-05
1826 1.11252993519884e-05
1827 1.09095990410424e-05
1828 1.12082934720092e-05
1829 1.20700151455821e-05
1830 1.20687009257381e-05
1831 1.20684935609461e-05
1832 1.11660183392814e-05
1833 1.10318514998653e-05
1834 1.09156444523251e-05
1835 1.11016315713641e-05
1836 1.15221782834851e-05
1837 1.14747153929784e-05
1838 1.16066194095765e-05
1839 1.12088500827667e-05
1840 1.11295566966874e-05
1841 1.08052499854239e-05
1842 1.09322954813251e-05
1843 1.11550034489483e-05
1844 1.13299756776541e-05
1845 1.17899289762136e-05
1846 1.17310164569062e-05
1847 1.18726684377179e-05
1848 1.14480999400257e-05
1849 1.12909820018103e-05
1850 1.08308131530066e-05
1851 1.09031052488717e-05
1852 1.10817090899218e-05
1853 1.15024949991493e-05
1854 1.23318686746643e-05
1855 1.27966550280689e-05
1856 1.29118197946809e-05
1857 1.19487949632457e-05
1858 1.13431706267875e-05
1859 1.0796841706906e-05
1860 1.10600140033057e-05
1861 1.15847014967585e-05
1862 1.18147490866249e-05
1863 1.21128559840145e-05
1864 1.15451066449168e-05
1865 1.13187770693912e-05
1866 1.08127114799572e-05
1867 1.09074007923482e-05
1868 1.11319814095623e-05
1869 1.14501754069352e-05
1870 1.19315409392584e-05
1871 1.18665175250499e-05
1872 1.1711157640093e-05
1873 1.11271365312859e-05
1874 1.10057753772708e-05
1875 1.08331942101358e-05
1876 1.10977653093869e-05
1877 1.16391211122391e-05
1878 1.19558744700043e-05
1879 1.22769615700236e-05
1880 1.18376356113004e-05
1881 1.14879885586561e-05
1882 1.08719823401771e-05
1883 1.09329839688144e-05
1884 1.10841974674258e-05
1885 1.15307047963142e-05
1886 1.22248939078418e-05
1887 1.22572810141719e-05
1888 1.20757404147298e-05
1889 1.1286664630461e-05
1890 1.10272985693882e-05
1891 1.08509448182303e-05
1892 1.11785138869891e-05
1893 1.18238103823387e-05
1894 1.20846725621959e-05
1895 1.21589891932672e-05
1896 1.14711665446521e-05
1897 1.11567987914896e-05
1898 1.07508258224698e-05
1899 1.09989459815552e-05
1900 1.14824160846183e-05
1901 1.17910385597497e-05
1902 1.20154027172248e-05
1903 1.15518150778371e-05
1904 1.12687557702884e-05
1905 1.07699970612884e-05
1906 1.08917547549936e-05
1907 1.12013040052261e-05
1908 1.1525085028552e-05
1909 1.20461190817878e-05
1910 1.18781590572326e-05
1911 1.15834172902396e-05
1912 1.08952517621219e-05
1913 1.08790427475469e-05
1914 1.08365220512496e-05
1915 1.11350127554033e-05
1916 1.17154595500324e-05
1917 1.1875147720275e-05
1918 1.19360847747885e-05
1919 1.12482421172899e-05
1920 1.11406834548688e-05
1921 1.07082141767023e-05
1922 1.08203348645475e-05
1923 1.11217914309236e-05
1924 1.15103784992243e-05
1925 1.2166507076472e-05
1926 1.21477405627957e-05
1927 1.21028588182526e-05
1928 1.12544739749865e-05
1929 1.1023319530068e-05
1930 1.07467440102482e-05
1931 1.10500686787418e-05
1932 1.17710096674273e-05
1933 1.21638513519429e-05
1934 1.24789230540046e-05
1935 1.17226627480704e-05
1936 1.12633460958023e-05
1937 1.07019841379952e-05
1938 1.09350949060172e-05
1939 1.1505542715895e-05
1940 1.18272691906895e-05
1941 1.2126126421208e-05
1942 1.14768372441176e-05
1943 1.11657427623868e-05
1944 1.06766356111621e-05
1945 1.08519916466321e-05
1946 1.13072510430356e-05
1947 1.15224247565493e-05
1948 1.19226151582552e-05
1949 1.14459016913315e-05
1950 1.11955105239758e-05
1951 1.06837824205286e-05
1952 1.07997611848987e-05
1953 1.10480441435357e-05
1954 1.13522955871304e-05
1955 1.1836917110486e-05
1956 1.15870070658275e-05
1957 1.15311495392234e-05
1958 1.09143475128803e-05
1959 1.08518079287023e-05
1960 1.07428822957445e-05
1961 1.09704305941705e-05
1962 1.16070405056234e-05
1963 1.1924448699574e-05
1964 1.23320396596682e-05
1965 1.17845611384837e-05
1966 1.13650494313333e-05
1967 1.06997140392195e-05
1968 1.08271560748108e-05
1969 1.10931114249979e-05
1970 1.13684709504014e-05
1971 1.17839636004646e-05
1972 1.15065249701729e-05
1973 1.15084285425837e-05
1974 1.08649610410794e-05
1975 1.08015337900724e-05
1976 1.06744610093301e-05
1977 1.08029307739343e-05
1978 1.12456173155806e-05
1979 1.13100013550138e-05
1980 1.19240448839264e-05
1981 1.1616468327702e-05
1982 1.1487555639178e-05
1983 1.08517870103242e-05
1984 1.08373524199123e-05
1985 1.05996532511199e-05
1986 1.07660280264099e-05
1987 1.12160214484902e-05
1988 1.15345692393021e-05
1989 1.24051803140901e-05
1990 1.2355117178231e-05
1991 1.19473979793838e-05
1992 1.0837957233889e-05
1993 1.08537496998906e-05
1994 1.06324805528857e-05
1995 1.07654295788961e-05
1996 1.11636682049721e-05
1997 1.12840525616775e-05
1998 1.1877910765179e-05
1999 1.15576794996741e-05
};
\addlegendentry{Test}

\nextgroupplot[
title={4 Layer},
ymin=5.00295143846847e-06, ymax=0.01,
]
\addplot [semithick, black, dashed]
table {%
0 0.00714642475759319
1 0.00703587201678602
2 0.00693506794686982
3 0.0068425506578933
4 0.00675689402032731
5 0.00667731306657515
6 0.00660223617887823
7 0.0065269829456156
8 0.00644495894630381
9 0.00635507761307963
10 0.00627095575418934
11 0.00619909983834077
12 0.00614142481754243
13 0.00609509495552629
14 0.00605493969305826
15 0.00601869250931486
16 0.00598348450876074
17 0.00594813530551619
18 0.00591214535779727
19 0.00587355555398972
20 0.00583031488895358
21 0.00578157148265745
22 0.00572555502003524
23 0.00565904074028367
24 0.00557894022858818
25 0.0054805276740808
26 0.0053579054911097
27 0.00520433714336832
28 0.00500646549335215
29 0.00476314216211904
30 0.00447181710842415
31 0.00412376560416305
32 0.00373922950529959
33 0.00337062959806644
34 0.0030837018348393
35 0.00288992887180939
36 0.00273695044234046
37 0.00259787650793442
38 0.00246894826159405
39 0.00234504990021378
40 0.00222912473782344
41 0.00212239193569985
42 0.00202351935877232
43 0.00193156901787006
44 0.00184670093949535
45 0.00176765841024462
46 0.00169411200658942
47 0.00162597648704832
48 0.00156278534632293
49 0.00150406298689632
50 0.00144951619677158
51 0.00139872835552524
52 0.00135121620496648
53 0.00130647308469634
54 0.00126352706229227
55 0.0012220152775626
56 0.00117932821922295
57 0.00113167672770942
58 0.00108170596467971
59 0.00102867675741436
60 0.000976538431132212
61 0.000935029142965504
62 0.000901650168088963
63 0.000872755857926677
64 0.000847418041757919
65 0.00082472144276835
66 0.000803825770162803
67 0.000784561505952297
68 0.000766715219015168
69 0.000750090356177679
70 0.000734565856873814
71 0.000719973811555974
72 0.000706205054939346
73 0.000693181123551767
74 0.000680799642850616
75 0.000669032240693923
76 0.000657815375689097
77 0.000647013675006747
78 0.000636612088783295
79 0.000626677157470112
80 0.000617147298271448
81 0.000608000673082643
82 0.000599171680278232
83 0.00059070757902191
84 0.000582559802296601
85 0.000574694463239211
86 0.000567092900610078
87 0.000559676594548364
88 0.000552478036297543
89 0.000545474908676624
90 0.000538661403425067
91 0.000531645505361666
92 0.000525155844798064
93 0.000518835505317838
94 0.000512673490447924
95 0.000506709963246976
96 0.000500793969422375
97 0.00049505997981214
98 0.000489390152324631
99 0.000483815689221956
100 0.000478376927276258
101 0.000473059356181693
102 0.0004677508363784
103 0.000462573174218051
104 0.000457422022805076
105 0.000452378135605613
106 0.000447361667170298
107 0.000442407498326247
108 0.000437527981489438
109 0.000432690930551871
110 0.000427824792268439
111 0.000423093825702381
112 0.000418361510696741
113 0.000413669812246553
114 0.000408958640491619
115 0.000404341545049647
116 0.000399773363483291
117 0.000395137808254731
118 0.000390449418432581
119 0.000385460298730322
120 0.000380881083628992
121 0.000376170868548797
122 0.000371551414673377
123 0.00036694811126381
124 0.000362410570005522
125 0.000357758016377829
126 0.000353128701704009
127 0.000348412477762849
128 0.000343696447316688
129 0.000338939690720963
130 0.000334138843243181
131 0.000329377467949143
132 0.000324647022239333
133 0.000319919656988077
134 0.000315188301442504
135 0.000310409718679239
136 0.000305689260130748
137 0.000300944259890912
138 0.000296151139764333
139 0.00029149565688158
140 0.000286741958575476
141 0.000282031482186085
142 0.000277172221160527
143 0.000272425374703289
144 0.000267585492053968
145 0.000262867521314547
146 0.000258138146477904
147 0.000253539308232575
148 0.000248909984861712
149 0.000244390835291597
150 0.000239841928362239
151 0.000235495305986433
152 0.000231137004561788
153 0.000226838563975207
154 0.000222654266423206
155 0.000218579179431799
156 0.000214496888617077
157 0.000210456992618901
158 0.000206550170304354
159 0.000202720272056922
160 0.000199077535881997
161 0.000195589903682958
162 0.000192168482328725
163 0.000189087533101429
164 0.000185865842723842
165 0.000182763267986275
166 0.000179802532898066
167 0.000176971195799069
168 0.000174226614461759
169 0.000171529381589153
170 0.000168965404014898
171 0.000166512210540759
172 0.000164092192989074
173 0.000161582813632322
174 0.000159321956743952
175 0.000157012413154689
176 0.000154749323939996
177 0.000152529736510587
178 0.000150385643621576
179 0.00014839551215573
180 0.000146426986532333
181 0.000144370767060309
182 0.000142436225743836
183 0.000140519194019362
184 0.000138561442611262
185 0.000136660178029047
186 0.000134828037815282
187 0.000133074404715217
188 0.000131338304939277
189 0.000129619811872317
190 0.000127751249053176
191 0.000125934079918011
192 0.000124211115618778
193 0.000122527443735976
194 0.000120872414186124
195 0.000119260182685821
196 0.000117718959685931
197 0.000116177627660363
198 0.000114715635405105
199 0.00011322281349635
200 0.00011182557530276
201 0.000110421469102562
202 0.000108995388600874
203 0.000107582191375855
204 0.000106202114523057
205 0.000104837688624571
206 0.000103493398157184
207 0.000102140715156906
208 0.000100878305275387
209 9.95890847406145e-05
210 9.83252448776284e-05
211 9.70613850199697e-05
212 9.58921418856562e-05
213 9.46877304670579e-05
214 9.35528242180794e-05
215 9.23684711864325e-05
216 9.12244180568678e-05
217 9.01362506056103e-05
218 8.90594545026602e-05
219 8.79865624909826e-05
220 8.69166799404297e-05
221 8.59709850118406e-05
222 8.50028989631824e-05
223 8.39887435404307e-05
224 8.29236539274802e-05
225 8.19409803654025e-05
226 8.09259392582362e-05
227 7.98685596166138e-05
228 7.89186289935628e-05
229 7.79193268769518e-05
230 7.71049539451951e-05
231 7.6193368983013e-05
232 7.53805707773836e-05
233 7.45446955434659e-05
234 7.37732177569228e-05
235 7.30341032024739e-05
236 7.23529903439157e-05
237 7.15976622842618e-05
238 7.09479668898894e-05
239 7.02006758928064e-05
240 6.96410945835169e-05
241 6.89844929837591e-05
242 6.83347071017693e-05
243 6.76483648049953e-05
244 6.69589795307957e-05
245 6.62545308500739e-05
246 6.56203170699143e-05
247 6.49663543050849e-05
248 6.43147839483049e-05
249 6.3618962190759e-05
250 6.29358177270944e-05
251 6.22962336365163e-05
252 6.17381332013167e-05
253 6.12009074387743e-05
254 6.07058188322185e-05
255 6.018019293208e-05
256 5.97720428068271e-05
257 5.93600296205921e-05
258 5.89189242319321e-05
259 5.85571120801376e-05
260 5.81856184425078e-05
261 5.77454178447567e-05
262 5.74002709612387e-05
263 5.69922170541304e-05
264 5.65211757113104e-05
265 5.61017928184526e-05
266 5.56243285032565e-05
267 5.5122421681375e-05
268 5.46728318653322e-05
269 5.4127148537475e-05
270 5.36091377725256e-05
271 5.30955293882585e-05
272 5.25639625656993e-05
273 5.20979210776318e-05
274 5.16634602618637e-05
275 5.11908935330041e-05
276 5.07826418179036e-05
277 5.04068585414075e-05
278 5.00933298290818e-05
279 4.97516883370963e-05
280 4.95061202911984e-05
281 4.92957412596695e-05
282 4.91412701641991e-05
283 4.90365910685853e-05
284 4.90167401210329e-05
285 4.90097222254349e-05
286 4.90265575674442e-05
287 4.90310163989705e-05
288 4.90765803018434e-05
289 4.90693460832858e-05
290 4.89250543456166e-05
291 4.8762846441619e-05
292 4.83565895113003e-05
293 4.7781598311758e-05
294 4.71093952292279e-05
295 4.6305344712394e-05
296 4.54757716639165e-05
297 4.4655261749682e-05
298 4.39066113173681e-05
299 4.32886301169333e-05
300 4.27914427731935e-05
301 4.24858284624463e-05
302 4.24985688454171e-05
303 4.285068693477e-05
304 4.38671651927791e-05
305 4.5929613257556e-05
306 4.91458127633848e-05
307 5.29357829393007e-05
308 5.4653810188654e-05
309 5.17907590857902e-05
310 4.58796112337012e-05
311 4.24618389089915e-05
312 4.3669076514874e-05
313 4.71604820084792e-05
314 4.96114605743614e-05
315 4.92246981087519e-05
316 4.64195188705219e-05
317 4.31526734772092e-05
318 4.06525185674411e-05
319 3.9208475621777e-05
320 3.84357651377343e-05
321 3.80792637741934e-05
322 3.78263816038782e-05
323 3.76675148174854e-05
324 3.75108607855168e-05
325 3.73187133959618e-05
326 3.71312407665414e-05
327 3.6946294073914e-05
328 3.67322933421121e-05
329 3.65677397020647e-05
330 3.6451055692055e-05
331 3.63467262900485e-05
332 3.6294307797391e-05
333 3.63222484018877e-05
334 3.64528642684547e-05
335 3.65911495272542e-05
336 3.68134288262212e-05
337 3.70876712842971e-05
338 3.74121666588678e-05
339 3.78115321773009e-05
340 3.82334047088762e-05
341 3.86656787068063e-05
342 3.90026824312883e-05
343 3.92683230074908e-05
344 3.92861070199757e-05
345 3.90693406959741e-05
346 3.84936544222114e-05
347 3.75602955280385e-05
348 3.65796118959594e-05
349 3.55312740243363e-05
350 3.46152553127865e-05
351 3.38991506296793e-05
352 3.35490128655636e-05
353 3.37486384189845e-05
354 3.46369818657877e-05
355 3.66381111369662e-05
356 3.99561251214209e-05
357 4.45148875414247e-05
358 4.81328061283648e-05
359 4.67237605406012e-05
360 4.06705041555711e-05
361 3.53351285902193e-05
362 3.46552197649785e-05
363 3.71000912178232e-05
364 3.94119344147015e-05
365 3.97123504773944e-05
366 3.78781810805862e-05
367 3.53661297083363e-05
368 3.31788688612278e-05
369 3.18453146448405e-05
370 3.1096603711589e-05
371 3.07468715270431e-05
372 3.06301592019054e-05
373 3.04129019816912e-05
374 3.0398040568258e-05
375 3.02376599279341e-05
376 3.01477875623846e-05
377 2.99951437874313e-05
378 2.98686862905129e-05
379 2.97082003104521e-05
380 2.96066857554678e-05
381 2.952256728328e-05
382 2.94717104356579e-05
383 2.94490647299028e-05
384 2.95005904380474e-05
385 2.95194392752052e-05
386 2.96248475244454e-05
387 2.96984854308846e-05
388 2.97692919719594e-05
389 2.9849377956026e-05
390 2.99075540226568e-05
391 2.99449437424215e-05
392 2.99498755484962e-05
393 2.99172623758892e-05
394 2.98666547138282e-05
395 2.98092092414493e-05
396 2.97130012913982e-05
397 2.95519480548734e-05
398 2.94345090825487e-05
399 2.92684085003714e-05
400 2.90572901668895e-05
401 2.88701705706274e-05
402 2.85977552625383e-05
403 2.83992996727989e-05
404 2.80821064357895e-05
405 2.78382686360601e-05
406 2.75663862137066e-05
407 2.72820000155605e-05
408 2.69464256383856e-05
409 2.67365804305086e-05
410 2.67420046498046e-05
411 2.72915351828829e-05
412 2.92162859558687e-05
413 3.4730637377578e-05
414 4.76743998039098e-05
415 6.87524914138038e-05
416 7.40352042054582e-05
417 4.61879557365208e-05
418 3.40578018285242e-05
419 3.87735907914077e-05
420 3.67647534327276e-05
421 3.12741995873722e-05
422 2.7956430459497e-05
423 2.65587028334124e-05
424 2.5914896093937e-05
425 2.55794838697199e-05
426 2.53115971222684e-05
427 2.50869549756061e-05
428 2.49259209930131e-05
429 2.47578937440096e-05
430 2.46880709546815e-05
431 2.45576018294003e-05
432 2.44826583077895e-05
433 2.43789100302649e-05
434 2.43379443212177e-05
435 2.42266208534403e-05
436 2.41631018642607e-05
437 2.4097247870003e-05
438 2.4031292014115e-05
439 2.39537912136711e-05
440 2.38726706136561e-05
441 2.37953583468453e-05
442 2.37482164457958e-05
443 2.36475242019196e-05
444 2.36058853495535e-05
445 2.35074934433754e-05
446 2.34610115406042e-05
447 2.33827531364028e-05
448 2.33504712117139e-05
449 2.32564621107656e-05
450 2.32306277414551e-05
451 2.31752027310961e-05
452 2.31682495428487e-05
453 2.31649102140352e-05
454 2.31519180431405e-05
455 2.31859263077894e-05
456 2.32592518329966e-05
457 2.33330772569929e-05
458 2.3468930244519e-05
459 2.36224905556526e-05
460 2.38135167407449e-05
461 2.40790672965119e-05
462 2.43664160635149e-05
463 2.46991704138111e-05
464 2.50179562275576e-05
465 2.52195331373883e-05
466 2.52377637828261e-05
467 2.49492809700769e-05
468 2.44537840554671e-05
469 2.38567217767383e-05
470 2.33909985087877e-05
471 2.33069212374204e-05
472 2.40077258704119e-05
473 2.58258297360925e-05
474 2.91890787202931e-05
475 3.42473306917945e-05
476 3.99167656293109e-05
477 4.25496962392913e-05
478 3.89817504471424e-05
479 3.19478592150091e-05
480 2.53820115592163e-05
481 2.25722457560806e-05
482 2.27681548263803e-05
483 2.49107702110507e-05
484 2.78580475203682e-05
485 3.00112287803245e-05
486 2.97873018091366e-05
487 2.6949219021688e-05
488 2.35121601992461e-05
489 2.18217753698013e-05
490 2.18421709670125e-05
491 2.25162874025742e-05
492 2.29680460606119e-05
493 2.28609193593243e-05
494 2.23181118341742e-05
495 2.15960664728954e-05
496 2.08789274900312e-05
497 2.04602427036349e-05
498 2.0277799862356e-05
499 2.04287202762998e-05
500 2.07589070519987e-05
501 2.13241121080898e-05
502 2.18917809835517e-05
503 2.23230152283094e-05
504 2.24923183651526e-05
505 2.22088159773648e-05
506 2.16199584954779e-05
507 2.09564158684117e-05
508 2.05439428446752e-05
509 2.05360872218563e-05
510 2.09953269090235e-05
511 2.17959262727163e-05
512 2.26472756175156e-05
513 2.34463787336381e-05
514 2.38201710898167e-05
515 2.37144804655642e-05
516 2.31062162630025e-05
517 2.20469862526329e-05
518 2.09864822906525e-05
519 2.00812748865076e-05
520 1.97908320025419e-05
521 2.04263342933331e-05
522 2.25416130881229e-05
523 2.69302261486359e-05
524 3.36801467373249e-05
525 3.9809987836037e-05
526 3.88244783753322e-05
527 2.94336762607372e-05
528 2.27813985906877e-05
529 2.30984437781245e-05
530 2.55873425234476e-05
531 2.624179858568e-05
532 2.46689665734934e-05
533 2.22972779573283e-05
534 2.02237454143273e-05
535 1.90449663808323e-05
536 1.85983030149828e-05
537 1.86257652110555e-05
538 1.88636972300316e-05
539 1.91043433357407e-05
540 1.92713885311946e-05
541 1.922751336636e-05
542 1.90513360251998e-05
543 1.87990537501292e-05
544 1.85677427548825e-05
545 1.84383655010834e-05
546 1.83933397774894e-05
547 1.84683181743139e-05
548 1.85234356244734e-05
549 1.85832481074399e-05
550 1.85407881252431e-05
551 1.84655385488952e-05
552 1.82805856479007e-05
553 1.80985367155273e-05
554 1.78856507684344e-05
555 1.78136660728967e-05
556 1.78742270371401e-05
557 1.82536889656859e-05
558 1.91854728104346e-05
559 2.08824410758979e-05
560 2.36686960626287e-05
561 2.74991421047588e-05
562 3.08350950586345e-05
563 3.05994471885285e-05
564 2.60145126222966e-05
565 2.16088157420913e-05
566 2.107341720059e-05
567 2.37561951808019e-05
568 2.68441144717357e-05
569 2.79606574205005e-05
570 2.67136519784827e-05
571 2.37391505955031e-05
572 2.04863170898406e-05
573 1.82348958652767e-05
574 1.74220495630095e-05
575 1.76751259051144e-05
576 1.85034610620249e-05
577 1.94623976454977e-05
578 2.01515739561842e-05
579 2.01959899346349e-05
580 1.95282849180245e-05
581 1.85406978836511e-05
582 1.78068140490595e-05
583 1.76554447639532e-05
584 1.80564896572477e-05
585 1.87101715178883e-05
586 1.92223372055267e-05
587 1.9275346608616e-05
588 1.89195943125497e-05
589 1.82295787638509e-05
590 1.74762518607707e-05
591 1.6921057767405e-05
592 1.67329025266838e-05
593 1.71168211728912e-05
594 1.82124339609402e-05
595 2.01730561464331e-05
596 2.29794712414133e-05
597 2.58010774381034e-05
598 2.69159844279443e-05
599 2.50469708040768e-05
600 2.11301417918364e-05
601 1.87369079185373e-05
602 1.92291538434475e-05
603 2.14322467275707e-05
604 2.34707902473374e-05
605 2.38639519567485e-05
606 2.25086614999981e-05
607 2.00821380866856e-05
608 1.79385325100867e-05
609 1.66279014486292e-05
610 1.63001332840196e-05
611 1.67959423809094e-05
612 1.78753769173312e-05
613 1.9176541860233e-05
614 2.02411568821503e-05
615 2.04873591318488e-05
616 1.97067041813881e-05
617 1.82421688572632e-05
618 1.70971642834417e-05
619 1.6986775120742e-05
620 1.77480488803283e-05
621 1.88114924952743e-05
622 1.95772197377764e-05
623 1.95929328450362e-05
624 1.88602516262293e-05
625 1.77527815532486e-05
626 1.66607106084626e-05
627 1.59515840252311e-05
628 1.58114796846576e-05
629 1.63944094158097e-05
630 1.78384464639691e-05
631 2.00906306684345e-05
632 2.28180410069001e-05
633 2.48319952156351e-05
634 2.45108052867593e-05
635 2.13382696259146e-05
636 1.81326481332178e-05
637 1.74501416880801e-05
638 1.89682792259838e-05
639 2.10397870716861e-05
640 2.19576082973738e-05
641 2.12543895967698e-05
642 1.95455274720757e-05
643 1.75376998408083e-05
644 1.60069778250005e-05
645 1.53879686717318e-05
646 1.56455654991561e-05
647 1.66471308826743e-05
648 1.81290296570324e-05
649 1.95654715255955e-05
650 2.03102264499933e-05
651 1.98232644477159e-05
652 1.82714470247092e-05
653 1.66768225025038e-05
654 1.61073013167368e-05
655 1.673501375965e-05
656 1.79389399423968e-05
657 1.89869934152931e-05
658 1.92401782008744e-05
659 1.86402825486098e-05
660 1.7443208418122e-05
661 1.61895748629348e-05
662 1.5265154084787e-05
663 1.49712852408612e-05
664 1.54265409086918e-05
665 1.6722839243144e-05
666 1.88345593521433e-05
667 2.14221603478393e-05
668 2.33443823454138e-05
669 2.30983242213156e-05
670 2.05392316017239e-05
671 1.74609055609842e-05
672 1.62929960745295e-05
673 1.72626153491784e-05
674 1.9152212510587e-05
675 2.05005123121715e-05
676 2.04996977757332e-05
677 1.91374643607745e-05
678 1.71758521005927e-05
679 1.55446087912736e-05
680 1.46647335448336e-05
681 1.46334594948172e-05
682 1.53828016475899e-05
683 1.67170635219094e-05
684 1.82777106143561e-05
685 1.94858866375824e-05
686 1.95971654211036e-05
687 1.83515257354827e-05
688 1.65308922905893e-05
689 1.54255195461417e-05
690 1.56050307467126e-05
691 1.6742058302377e-05
692 1.81198471174149e-05
693 1.89332540863063e-05
694 1.87156558997259e-05
695 1.76355582848231e-05
696 1.61525344681834e-05
697 1.49373255986518e-05
698 1.42773159144127e-05
699 1.44403818911343e-05
700 1.55144810971564e-05
701 1.75822405594772e-05
702 2.03099148681218e-05
703 2.26892797172695e-05
704 2.30340392688433e-05
705 2.07032960024556e-05
706 1.72010552814683e-05
707 1.55588686303609e-05
708 1.62577086020299e-05
709 1.8052833396176e-05
710 1.95133739642195e-05
711 1.96091769413265e-05
712 1.8365048886082e-05
713 1.64930274904052e-05
714 1.48906753469902e-05
715 1.39763713988827e-05
716 1.38918096368457e-05
717 1.45924020333865e-05
718 1.58195194028821e-05
719 1.72485462588412e-05
720 1.83606231054512e-05
721 1.84532645040036e-05
722 1.73199329811524e-05
723 1.56954204864057e-05
724 1.4640387232312e-05
725 1.48230711638497e-05
726 1.59138367812339e-05
727 1.72644127003352e-05
728 1.80731438699233e-05
729 1.79377190940855e-05
730 1.69324529615267e-05
731 1.56618582973778e-05
732 1.4454423321375e-05
733 1.36919684194936e-05
734 1.37121017633746e-05
735 1.46093283639459e-05
736 1.64308348260533e-05
737 1.89826260683645e-05
738 2.14605683392932e-05
739 2.23006788875324e-05
740 2.06549189840644e-05
741 1.72155255513395e-05
742 1.5024454983692e-05
743 1.5253269155302e-05
744 1.69521619719504e-05
745 1.86298177862199e-05
746 1.91032191003693e-05
747 1.81726343111421e-05
748 1.64854923232483e-05
749 1.47581292253562e-05
750 1.35756067045634e-05
751 1.32582577645479e-05
752 1.37929507282308e-05
753 1.49614610762683e-05
754 1.64407596248139e-05
755 1.77031902843794e-05
756 1.80573554207086e-05
757 1.71053868820437e-05
758 1.54364789466399e-05
759 1.41620371070239e-05
760 1.41664847626011e-05
761 1.52257466718986e-05
762 1.66020110086151e-05
763 1.75313179369141e-05
764 1.74794609022477e-05
765 1.64967216873535e-05
766 1.50853018752173e-05
767 1.38290173481437e-05
768 1.31305479147414e-05
769 1.3242371261768e-05
770 1.4266402095231e-05
771 1.61574705987988e-05
772 1.86298976370125e-05
773 2.0737145585592e-05
774 2.09444495613553e-05
775 1.88534275231689e-05
776 1.57584000421984e-05
777 1.4295130219999e-05
778 1.50711110178392e-05
779 1.69903137972049e-05
780 1.85102849279417e-05
781 1.8630683495946e-05
782 1.72591110629838e-05
783 1.53157565980777e-05
784 1.3679959255164e-05
785 1.28473268139473e-05
786 1.29080659565517e-05
787 1.37700176239264e-05
788 1.51515581912065e-05
789 1.66147246840609e-05
790 1.7511367664369e-05
791 1.71903766652548e-05
792 1.57381349867869e-05
793 1.40789612501635e-05
794 1.34903651725082e-05
795 1.41723401863025e-05
796 1.54866390789099e-05
797 1.66600182938126e-05
798 1.70168328104126e-05
799 1.63216199244864e-05
800 1.50018248463724e-05
801 1.3682690655159e-05
802 1.27921280608234e-05
803 1.26210977917651e-05
804 1.32992051096181e-05
805 1.48929327306035e-05
806 1.71586645087984e-05
807 1.95314157744875e-05
808 2.05871057596951e-05
809 1.93485152109041e-05
810 1.62477662257743e-05
811 1.40354464193138e-05
812 1.4156297455159e-05
813 1.58703633115209e-05
814 1.76237933990908e-05
815 1.82339973981982e-05
816 1.73168183792427e-05
817 1.56117042058312e-05
818 1.38636142761328e-05
819 1.26324104749997e-05
820 1.23550668611472e-05
821 1.29613272541462e-05
822 1.42393857913703e-05
823 1.5823417353289e-05
824 1.70553492830905e-05
825 1.7220925480288e-05
826 1.59972196005498e-05
827 1.42084985510493e-05
828 1.31440146020623e-05
829 1.35148169762012e-05
830 1.48303252778259e-05
831 1.61725335332541e-05
832 1.6788713389726e-05
833 1.62615486862805e-05
834 1.49360054582104e-05
835 1.34587875368908e-05
836 1.2430141892672e-05
837 1.21698672677084e-05
838 1.27807627965382e-05
839 1.42690885489571e-05
840 1.64268358995301e-05
841 1.86307680198894e-05
842 1.96105643244238e-05
843 1.84369642148852e-05
844 1.55681873030833e-05
845 1.34674177736382e-05
846 1.3600568048e-05
847 1.51584873071542e-05
848 1.67208777150929e-05
849 1.71398819239776e-05
850 1.62116247177657e-05
851 1.45993003548561e-05
852 1.29937160266991e-05
853 1.20574921274041e-05
854 1.19454729796686e-05
855 1.26313786870114e-05
856 1.3935565581491e-05
857 1.55147758515994e-05
858 1.67542413382904e-05
859 1.68709741430462e-05
860 1.55708003752864e-05
861 1.37203892489612e-05
862 1.2739823062935e-05
863 1.32389368661112e-05
864 1.46128119669076e-05
865 1.59776098158648e-05
866 1.64756474361205e-05
867 1.58231303020173e-05
868 1.44413256490949e-05
869 1.29986143484473e-05
870 1.19992110390044e-05
871 1.17467728175669e-05
872 1.23537513561089e-05
873 1.37707415133193e-05
874 1.58262467975234e-05
875 1.78623657740573e-05
876 1.87149069001968e-05
877 1.7582102069813e-05
878 1.49987916009398e-05
879 1.30446161838993e-05
880 1.30630582804159e-05
881 1.45581389667804e-05
882 1.62048561498374e-05
883 1.68040762793531e-05
884 1.60555133010831e-05
885 1.4476523957363e-05
886 1.27977357680109e-05
887 1.1754459417368e-05
888 1.16793677316096e-05
889 1.26511672124963e-05
890 1.46012643362781e-05
891 1.71619077828922e-05
892 1.92266579723821e-05
893 1.93185432211251e-05
894 1.70725014236872e-05
895 1.39753054693159e-05
896 1.2902357906297e-05
897 1.40839923798985e-05
898 1.58656231097343e-05
899 1.66701342310116e-05
900 1.58554899463148e-05
901 1.41095690984194e-05
902 1.24032240735872e-05
903 1.14348128947839e-05
904 1.13763309190773e-05
905 1.20914523185078e-05
906 1.32967287234109e-05
907 1.45975485654759e-05
908 1.54083552263984e-05
909 1.51803091545943e-05
910 1.39564413328408e-05
911 1.25607754688506e-05
912 1.19945830012735e-05
913 1.25982089043131e-05
914 1.38812579835168e-05
915 1.51643308403671e-05
916 1.57694246818441e-05
917 1.53260037194514e-05
918 1.41099988351101e-05
919 1.2698407331424e-05
920 1.1652686743524e-05
921 1.12830938707731e-05
922 1.17401689916363e-05
923 1.30424053805811e-05
924 1.50808778638378e-05
925 1.73202792579374e-05
926 1.85852061727054e-05
927 1.77904309393018e-05
928 1.4998945547795e-05
929 1.2657842066055e-05
930 1.24552157232394e-05
931 1.40747720287493e-05
932 1.61981049231485e-05
933 1.74460885686845e-05
934 1.70832939789811e-05
935 1.55537092387803e-05
936 1.3444195232637e-05
937 1.18419134622627e-05
938 1.11517076284429e-05
939 1.13837607509382e-05
940 1.23067521222708e-05
941 1.35581254419037e-05
942 1.46359377994898e-05
943 1.49052813851469e-05
944 1.40581758243385e-05
945 1.26005572624877e-05
946 1.15554832360587e-05
947 1.16893186774547e-05
948 1.28819380909029e-05
949 1.44364988881041e-05
950 1.56462840632532e-05
951 1.58442668314329e-05
952 1.4996223188124e-05
953 1.35209170943895e-05
954 1.2103986064993e-05
955 1.11910718869623e-05
956 1.1032112840148e-05
957 1.16710835040834e-05
958 1.30375110192205e-05
959 1.49051460400784e-05
960 1.65769632256563e-05
961 1.68952738004435e-05
962 1.54607122075934e-05
963 1.30505079041221e-05
964 1.17050218921833e-05
965 1.2342842474089e-05
966 1.43703120926153e-05
967 1.6571494436235e-05
968 1.7658473948412e-05
969 1.70133524997063e-05
970 1.52335387371139e-05
971 1.29839678826293e-05
972 1.14443886296556e-05
973 1.09596755324404e-05
974 1.148320965072e-05
975 1.27568275036793e-05
976 1.42849695210501e-05
977 1.5350836599648e-05
978 1.51195073705424e-05
979 1.35952637105863e-05
980 1.18885346214181e-05
981 1.12470977160495e-05
982 1.2091024184091e-05
983 1.38016530712992e-05
984 1.54231642042202e-05
985 1.60380945215532e-05
986 1.53283779611435e-05
987 1.37060003129186e-05
988 1.20465467499109e-05
989 1.09767515859005e-05
990 1.07825960027341e-05
991 1.14552790808631e-05
992 1.28006858926e-05
993 1.44163296189603e-05
994 1.56019685548792e-05
995 1.54754620265152e-05
996 1.39117361399599e-05
997 1.19577875847821e-05
998 1.12133919258905e-05
999 1.21282809275591e-05
1000 1.40252665250529e-05
1001 1.58364089175578e-05
1002 1.6507799212695e-05
1003 1.56812691103347e-05
1004 1.39533060763419e-05
1005 1.21125516088227e-05
1006 1.08857415943575e-05
1007 1.06083911166621e-05
1008 1.12003434251662e-05
1009 1.24677846158328e-05
1010 1.40022817696206e-05
1011 1.5142337948415e-05
1012 1.50782875043554e-05
1013 1.36665969796557e-05
1014 1.18199273704267e-05
1015 1.10738485470385e-05
1016 1.19967401639443e-05
1017 1.38871396703211e-05
1018 1.57493023653643e-05
1019 1.64324372651237e-05
1020 1.5674777712249e-05
1021 1.39335934332685e-05
1022 1.20891784396449e-05
1023 1.08219618351058e-05
1024 1.04756917043147e-05
1025 1.1035725047126e-05
1026 1.2299882717226e-05
1027 1.38627213099873e-05
1028 1.5055683752152e-05
1029 1.50341405458221e-05
1030 1.35780319379109e-05
1031 1.1733791333679e-05
1032 1.09263355543288e-05
1033 1.16873304230047e-05
1034 1.34077613758166e-05
1035 1.51173224232082e-05
1036 1.58393782618838e-05
1037 1.52014020433633e-05
1038 1.36293383050656e-05
1039 1.18875853640787e-05
1040 1.06869228058137e-05
1041 1.03211227258093e-05
1042 1.07902637784818e-05
1043 1.20129149157044e-05
1044 1.36918562709809e-05
1045 1.52697106168276e-05
1046 1.57631965325145e-05
1047 1.45508073214895e-05
1048 1.23465140886569e-05
1049 1.09251900655138e-05
1050 1.12570179162663e-05
1051 1.29106892873265e-05
1052 1.48735508658238e-05
1053 1.61324444498678e-05
1054 1.59037664202444e-05
1055 1.44851880499175e-05
1056 1.25283326806169e-05
1057 1.09884794947668e-05
1058 1.02571505955495e-05
1059 1.03575500149944e-05
1060 1.12086842776193e-05
1061 1.25672255406606e-05
1062 1.39842555513248e-05
1063 1.47382712674116e-05
1064 1.41581354369258e-05
1065 1.25134339130639e-05
1066 1.0953381136436e-05
1067 1.07092085054328e-05
1068 1.18828951369832e-05
1069 1.37170965964772e-05
1070 1.52941629176784e-05
1071 1.57001384932798e-05
1072 1.48323707414022e-05
1073 1.32260650635985e-05
1074 1.15753034726218e-05
1075 1.04328041823365e-05
1076 1.00950687946799e-05
1077 1.05446456548641e-05
1078 1.17110028861056e-05
1079 1.33079841844186e-05
1080 1.47612796750352e-05
1081 1.52216193043397e-05
1082 1.41061475795734e-05
1083 1.2051037111771e-05
1084 1.06859853261643e-05
1085 1.10071587894334e-05
1086 1.26880623393388e-05
1087 1.47071070459859e-05
1088 1.6072314297233e-05
1089 1.59216062545209e-05
1090 1.44901271608155e-05
1091 1.26109842506494e-05
1092 1.10166990365812e-05
1093 1.02028860284875e-05
1094 1.01636072602496e-05
1095 1.08111496128771e-05
1096 1.19482665077442e-05
1097 1.31683802413463e-05
1098 1.39125210492086e-05
1099 1.35103305256568e-05
1100 1.21680967823945e-05
1101 1.07187406735987e-05
1102 1.03612070926351e-05
1103 1.13799304579665e-05
1104 1.32226255741275e-05
1105 1.49928566753887e-05
1106 1.57512634657575e-05
1107 1.51351438384406e-05
1108 1.36301675466299e-05
1109 1.18404165045938e-05
1110 1.05444682207967e-05
1111 9.97100156929775e-06
1112 1.0165019907582e-05
1113 1.09572149717962e-05
1114 1.21376270203477e-05
1115 1.33374877768766e-05
1116 1.39415742406168e-05
1117 1.33635234944585e-05
1118 1.18873776659978e-05
1119 1.04933305512533e-05
1120 1.03597651817111e-05
1121 1.16425830356537e-05
1122 1.37122115467792e-05
1123 1.55757194417738e-05
1124 1.62615549357259e-05
1125 1.54419048417065e-05
1126 1.37297353317578e-05
1127 1.17507521526505e-05
1128 1.04069699866649e-05
1129 9.95904477196774e-06
1130 1.04126531033089e-05
1131 1.15655817820581e-05
1132 1.30491017578116e-05
1133 1.42077877098856e-05
1134 1.425137675648e-05
1135 1.30129785211874e-05
1136 1.11179722327437e-05
1137 1.01619778742901e-05
1138 1.08048080396106e-05
1139 1.24864046815443e-05
1140 1.42481742603229e-05
1141 1.51670466617126e-05
1142 1.47333897326796e-05
1143 1.32444510878571e-05
1144 1.15575903589793e-05
1145 1.0259504936716e-05
1146 9.75378224410406e-06
1147 1.00110131615594e-05
1148 1.0926918945664e-05
1149 1.2217200031639e-05
1150 1.33528406036643e-05
1151 1.36597220408152e-05
1152 1.27984745971199e-05
1153 1.12470786460817e-05
1154 1.01145914896961e-05
1155 1.03408657121662e-05
1156 1.17580106904214e-05
1157 1.36667000498747e-05
1158 1.50913290731802e-05
1159 1.53253403250009e-05
1160 1.42541548771025e-05
1161 1.25903723597887e-05
1162 1.09287188196916e-05
1163 9.91514504944391e-06
1164 9.75992357021838e-06
1165 1.04793447834872e-05
1166 1.19012785750883e-05
1167 1.36154898966456e-05
1168 1.47869417153057e-05
1169 1.45406321636621e-05
1170 1.28403479759953e-05
1171 1.07835873746076e-05
1172 1.0171339201559e-05
1173 1.12435862232863e-05
1174 1.31103574126934e-05
1175 1.46688332415268e-05
1176 1.50099924891567e-05
1177 1.40020775372118e-05
1178 1.2319866257382e-05
1179 1.06910862753828e-05
1180 9.70239463349287e-06
1181 9.56067203361499e-06
1182 1.019338176933e-05
1183 1.13563014263462e-05
1184 1.26451784361326e-05
1185 1.33603025311846e-05
1186 1.29924503047274e-05
1187 1.16825122524755e-05
1188 1.02718453969253e-05
1189 9.87203516866764e-06
1190 1.07559557197545e-05
1191 1.24162541885742e-05
1192 1.39861645657158e-05
1193 1.47091133596566e-05
1194 1.41856975333088e-05
1195 1.27458062666452e-05
1196 1.11415096371026e-05
1197 9.91642236591872e-06
1198 9.49215737122699e-06
1199 9.95292194705399e-06
1200 1.12367990949824e-05
1201 1.30626394387878e-05
1202 1.46492117902852e-05
1203 1.5007021154867e-05
1204 1.37186747523055e-05
1205 1.13896272634362e-05
1206 1.00601203371076e-05
1207 1.05379617548751e-05
1208 1.22142540655013e-05
1209 1.3942869280914e-05
1210 1.46598185906122e-05
1211 1.40709545488615e-05
1212 1.25596100264413e-05
1213 1.08664081834853e-05
1214 9.6997336611615e-06
1215 9.42206741427754e-06
1216 1.00443160013786e-05
1217 1.13602091400544e-05
1218 1.29516316187406e-05
1219 1.39741040727781e-05
1220 1.36463557554034e-05
1221 1.21027668749107e-05
1222 1.03515278661037e-05
1223 9.82045183373081e-06
1224 1.07144622655264e-05
1225 1.23448591941022e-05
1226 1.36598894751039e-05
1227 1.39858384010649e-05
1228 1.31608457283328e-05
1229 1.17210493413289e-05
1230 1.02830239847584e-05
1231 9.42172242535122e-06
1232 9.33364747091048e-06
1233 1.00714388620027e-05
1234 1.15056771616828e-05
1235 1.32150285649946e-05
1236 1.42867148598125e-05
1237 1.39431465951967e-05
1238 1.23166594164781e-05
1239 1.03742384896521e-05
1240 9.81894612195333e-06
1241 1.08350493208675e-05
1242 1.25805993002048e-05
1243 1.40535008213982e-05
1244 1.43785991910672e-05
1245 1.35053395582396e-05
1246 1.19619547975791e-05
1247 1.03500336801066e-05
1248 9.37444596393888e-06
1249 9.29370648883321e-06
1250 1.01035143691419e-05
1251 1.157245582184e-05
1252 1.32508682928467e-05
1253 1.4130571195925e-05
1254 1.35460592771697e-05
1255 1.18021282486946e-05
1256 1.00875847888759e-05
1257 9.78856001865314e-06
1258 1.09176056803384e-05
1259 1.26148599682296e-05
1260 1.38465180749003e-05
1261 1.39386611603953e-05
1262 1.28960993526217e-05
1263 1.13012609708196e-05
1264 9.90546836265871e-06
1265 9.17207077311133e-06
1266 9.27187696087017e-06
1267 1.01927332243346e-05
1268 1.17263722163585e-05
1269 1.33853318562416e-05
1270 1.40805069132055e-05
1271 1.32828613992331e-05
1272 1.14457666791634e-05
1273 9.8886081541516e-06
1274 9.90214150642377e-06
1275 1.12664746166891e-05
1276 1.29826578615932e-05
1277 1.40920650846255e-05
1278 1.39619187182749e-05
1279 1.27222245827063e-05
1280 1.10808935875895e-05
1281 9.69222594493502e-06
1282 9.07842642861034e-06
1283 9.37012530521919e-06
1284 1.04910990059714e-05
1285 1.2149592520716e-05
1286 1.3625667378836e-05
1287 1.38861509745292e-05
1288 1.26404074776154e-05
1289 1.0731955614518e-05
1290 9.67416494923601e-06
1291 1.02328090486603e-05
1292 1.18084020252485e-05
1293 1.32823183525144e-05
1294 1.373767004198e-05
1295 1.29516654758799e-05
1296 1.13714336191784e-05
1297 9.84827075689587e-06
1298 9.09429288786789e-06
1299 9.3693565452746e-06
1300 1.06397118795343e-05
1301 1.24913954722139e-05
1302 1.39405060289866e-05
1303 1.38682271095814e-05
1304 1.21790624133533e-05
1305 1.02388458318448e-05
1306 9.63828086369967e-06
1307 1.05825301852658e-05
1308 1.2102817012416e-05
1309 1.31734400068417e-05
1310 1.30544947100075e-05
1311 1.19065794437212e-05
1312 1.04031140513161e-05
1313 9.2421929966946e-06
1314 8.93186188055672e-06
1315 9.58351099522581e-06
1316 1.10270710651506e-05
1317 1.28068032847484e-05
1318 1.3946174303392e-05
1319 1.34624431445962e-05
1320 1.16703798342943e-05
1321 9.8995620652631e-06
1322 9.64218421839469e-06
1323 1.08477851398225e-05
1324 1.25182820349123e-05
1325 1.36387106153268e-05
1326 1.34847573028951e-05
1327 1.22493423666548e-05
1328 1.05867821893479e-05
1329 9.28272362332727e-06
1330 8.88898030471097e-06
1331 9.52270240284214e-06
1332 1.10090121709927e-05
1333 1.28786731456998e-05
1334 1.41119144230206e-05
1335 1.36193998668266e-05
1336 1.1744419673998e-05
1337 9.89739046203253e-06
1338 9.65842367439684e-06
1339 1.08819191558251e-05
1340 1.24764069451833e-05
1341 1.3397074062782e-05
1342 1.30851532326171e-05
1343 1.17475368330844e-05
1344 1.0175438080795e-05
1345 9.05999147837733e-06
1346 8.89466903108982e-06
1347 9.76168509891906e-06
1348 1.14461738938143e-05
1349 1.32961335579385e-05
1350 1.410188869444e-05
1351 1.30991620718168e-05
1352 1.10343139074209e-05
1353 9.54271648706539e-06
1354 9.7888989595285e-06
1355 1.12511511736191e-05
1356 1.27376634436183e-05
1357 1.33052083840512e-05
1358 1.26464556917671e-05
1359 1.1142409149123e-05
1360 9.61782319142124e-06
1361 8.78047599406351e-06
1362 8.86862998461524e-06
1363 9.86667057967239e-06
1364 1.14970905418232e-05
1365 1.30440828495226e-05
1366 1.34696872895468e-05
1367 1.22744124388952e-05
1368 1.03845355960974e-05
1369 9.42449953555968e-06
1370 1.00591537531836e-05
1371 1.16303932762851e-05
1372 1.30013579203725e-05
1373 1.33160345027195e-05
1374 1.2416721037356e-05
1375 1.07727842704186e-05
1376 9.34727037504857e-06
1377 8.70400610630817e-06
1378 9.06951020751556e-06
1379 1.03802291091526e-05
1380 1.22758845273463e-05
1381 1.38721222119331e-05
1382 1.3876783869271e-05
1383 1.21989039573167e-05
1384 1.01140538415478e-05
1385 9.3879244944528e-06
1386 1.03548420704591e-05
1387 1.20746146008988e-05
1388 1.34222184804411e-05
1389 1.35369655731576e-05
1390 1.23420627080595e-05
1391 1.04965204295038e-05
1392 9.09394743420222e-06
1393 8.65659221305215e-06
1394 9.29057825294066e-06
1395 1.07846503940578e-05
1396 1.25398645960395e-05
1397 1.35171708690596e-05
1398 1.28461411955105e-05
1399 1.09709920322887e-05
1400 9.45849364752949e-06
1401 9.43521737845066e-06
1402 1.06897450140031e-05
1403 1.22219663928891e-05
1404 1.30346894836597e-05
1405 1.26156874086236e-05
1406 1.12516808061791e-05
1407 9.67336979962319e-06
1408 8.71546458225403e-06
1409 8.76162758167709e-06
1410 9.82705503527193e-06
1411 1.164191004277e-05
1412 1.34176227825078e-05
1413 1.39831791345912e-05
1414 1.26829768714032e-05
1415 1.05056143366866e-05
1416 9.27304480446978e-06
1417 9.7723877283129e-06
1418 1.12943103354746e-05
1419 1.26390436761548e-05
1420 1.29502116457125e-05
1421 1.20607183595745e-05
1422 1.04043031150525e-05
1423 9.03462437729097e-06
1424 8.48645733930731e-06
1425 8.93885299446673e-06
1426 1.02798425336559e-05
1427 1.20332631012854e-05
1428 1.32916492480151e-05
1429 1.30061270723347e-05
1430 1.12730693258012e-05
1431 9.53229517614007e-06
1432 9.17458068899757e-06
1433 1.02545897210926e-05
1434 1.18972526402805e-05
1435 1.29923695377232e-05
1436 1.29047462195109e-05
1437 1.16375370748933e-05
1438 9.96664597630037e-06
1439 8.74835885866787e-06
1440 8.45083594525065e-06
1441 9.13521354783953e-06
1442 1.06310581831082e-05
1443 1.23878394141519e-05
1444 1.34164589882246e-05
1445 1.28396987164603e-05
1446 1.09645818195947e-05
1447 9.34393957618962e-06
1448 9.24118509409988e-06
1449 1.04418992521449e-05
1450 1.19808571713009e-05
1451 1.28854637843068e-05
1452 1.25731046508593e-05
1453 1.12126380560351e-05
1454 9.62617553779133e-06
1455 8.56862411424686e-06
1456 8.48779888529272e-06
1457 9.4143968729643e-06
1458 1.1148256077953e-05
1459 1.30256298300768e-05
1460 1.38545825372205e-05
1461 1.28208516275707e-05
1462 1.06697203594019e-05
1463 9.18959486922999e-06
1464 9.42851922913768e-06
1465 1.0864132680255e-05
1466 1.23651638754274e-05
1467 1.29645638434761e-05
1468 1.23578503009592e-05
1469 1.07222983403155e-05
1470 9.16181058019383e-06
1471 8.34436299435026e-06
1472 8.54995986543372e-06
1473 9.69119285842979e-06
1474 1.14082080323596e-05
1475 1.28888723764931e-05
1476 1.30590548040876e-05
1477 1.16701751676773e-05
1478 9.74980318607654e-06
1479 8.87865195658977e-06
1480 9.58083056717962e-06
1481 1.11198111931721e-05
1482 1.24517230279952e-05
1483 1.2777548198617e-05
1484 1.18547060738994e-05
1485 1.02149713130384e-05
1486 8.81997565405079e-06
1487 8.25854195807274e-06
1488 8.72492289460958e-06
1489 1.01246336544625e-05
1490 1.20167775725655e-05
1491 1.34192008753864e-05
1492 1.3226686899237e-05
1493 1.13685696966903e-05
1494 9.37503485920477e-06
1495 8.8649676804331e-06
1496 9.89415505431301e-06
1497 1.1493119649264e-05
1498 1.25621738121251e-05
1499 1.24186745295018e-05
1500 1.12198234314409e-05
1501 9.63670183429244e-06
1502 8.48130805564118e-06
1503 8.26727075664557e-06
1504 9.11949433612946e-06
1505 1.08287922750971e-05
1506 1.27563269681774e-05
1507 1.37269948378149e-05
1508 1.27735367931869e-05
1509 1.06086973326036e-05
1510 8.97273626709882e-06
1511 9.11691110783153e-06
1512 1.0512500811799e-05
1513 1.19629216037143e-05
1514 1.24908275154212e-05
1515 1.17970185360416e-05
1516 1.02509237123183e-05
1517 8.83415287211697e-06
1518 8.11592050786736e-06
1519 8.34138147509655e-06
1520 9.45017614872867e-06
1521 1.1093635173931e-05
1522 1.24950457289774e-05
1523 1.26781542840737e-05
1524 1.13324341793342e-05
1525 9.54083945936546e-06
1526 8.69157441202306e-06
1527 9.34791932227785e-06
1528 1.08463017860494e-05
1529 1.21662881281681e-05
1530 1.25391137461017e-05
1531 1.17137535272649e-05
1532 1.01675559138759e-05
1533 8.76379267489291e-06
1534 8.0891733702515e-06
1535 8.41383489547098e-06
1536 9.71277737749077e-06
1537 1.16473129323502e-05
1538 1.3322403848437e-05
1539 1.34614964467095e-05
1540 1.17861136225694e-05
1541 9.5790080549274e-06
1542 8.73734794581971e-06
1543 9.58744113815913e-06
1544 1.11954435886297e-05
1545 1.24337539946784e-05
1546 1.24778497118117e-05
1547 1.13206685934397e-05
1548 9.63208798165915e-06
1549 8.39085632531322e-06
1550 8.09432082937711e-06
1551 8.87900163004662e-06
1552 1.05392923270831e-05
1553 1.24294493427657e-05
1554 1.34143251833141e-05
1555 1.25138121752078e-05
1556 1.04324978267845e-05
1557 8.83438875276532e-06
1558 8.91009876308191e-06
1559 1.02139715775773e-05
1560 1.15922988189188e-05
1561 1.21804108692003e-05
1562 1.15367458030091e-05
1563 1.008728797669e-05
1564 8.67347193245216e-06
1565 7.9873482987658e-06
1566 8.30852407029958e-06
1567 9.56062652990308e-06
1568 1.13238166795648e-05
1569 1.26742871466856e-05
1570 1.25327971098321e-05
1571 1.08976955602857e-05
1572 9.13217073172934e-06
1573 8.67495349246572e-06
1574 9.65987216969277e-06
1575 1.11513708436761e-05
1576 1.22039642801441e-05
1577 1.20722838061749e-05
1578 1.09068612687002e-05
1579 9.36800738404742e-06
1580 8.23434787955435e-06
1581 8.07205570008129e-06
1582 9.06249825050054e-06
1583 1.10188012967249e-05
1584 1.32156877635792e-05
1585 1.42396738396222e-05
1586 1.30779828333982e-05
1587 1.05222225694757e-05
1588 8.83257043160501e-06
1589 9.1707063825941e-06
1590 1.06736372663396e-05
1591 1.20346706516394e-05
1592 1.23001976630732e-05
1593 1.13218020745265e-05
1594 9.65008159115044e-06
1595 8.31467235906658e-06
1596 7.89316556715214e-06
1597 8.52877552404152e-06
1598 9.98242839689301e-06
1599 1.16527911658837e-05
1600 1.2520162296803e-05
1601 1.18258153941064e-05
1602 1.00568342338292e-05
1603 8.63402967243854e-06
1604 8.60773024924377e-06
1605 9.76578799605221e-06
1606 1.11320613452315e-05
1607 1.18732659740162e-05
1608 1.14345767836532e-05
1609 1.01582327348693e-05
1610 8.73124269151759e-06
1611 7.89856527849686e-06
1612 8.02401399413455e-06
1613 9.13575981942305e-06
1614 1.09653334614546e-05
1615 1.2722528136444e-05
1616 1.32422778675312e-05
1617 1.19328956889597e-05
1618 9.77649284558679e-06
1619 8.5480187584086e-06
1620 9.10508113252662e-06
1621 1.06877580073328e-05
1622 1.21314621345237e-05
1623 1.25218373243641e-05
1624 1.16544057729984e-05
1625 9.9947546265966e-06
1626 8.49260940416485e-06
1627 7.87241387234516e-06
1628 8.43533472638658e-06
1629 1.00711521358043e-05
1630 1.22273115571447e-05
1631 1.36583274245594e-05
1632 1.30974474870094e-05
1633 1.08761425660353e-05
1634 8.84387485089455e-06
1635 8.61329076579098e-06
1636 9.78095727247053e-06
1637 1.11151675961096e-05
1638 1.1669026075245e-05
1639 1.10450887516045e-05
1640 9.65917006578998e-06
1641 8.32186130184098e-06
1642 7.75315855316983e-06
1643 8.17272121422796e-06
1644 9.50363772744822e-06
1645 1.12803676782791e-05
1646 1.25310462255057e-05
1647 1.2247918206143e-05
1648 1.05169334330135e-05
1649 8.79602713915739e-06
1650 8.39326291413034e-06
1651 9.39587649223528e-06
1652 1.08673896870581e-05
1653 1.18065631197761e-05
1654 1.16417643560851e-05
1655 1.0487613751331e-05
1656 9.00334591369401e-06
1657 7.93132945664521e-06
1658 7.74982087536902e-06
1659 8.57930869824219e-06
1660 1.02817574937353e-05
1661 1.22249582171774e-05
1662 1.32320851637946e-05
1663 1.23599603187585e-05
1664 1.02571553285413e-05
1665 8.60933672280906e-06
1666 8.69575088152663e-06
1667 1.0083368006425e-05
1668 1.15803215621768e-05
1669 1.22427667411662e-05
1670 1.16318655567493e-05
1671 1.00686258047711e-05
1672 8.53608576245612e-06
1673 7.75596313351556e-06
1674 8.07151739623535e-06
1675 9.46421121372776e-06
1676 1.14766440162395e-05
1677 1.30653704562e-05
1678 1.29454408835938e-05
1679 1.10634849892766e-05
1680 8.97489524140371e-06
1681 8.35679646898901e-06
1682 9.30032683477133e-06
1683 1.07354754380895e-05
1684 1.16319428271616e-05
1685 1.13785388334842e-05
1686 1.01487404153833e-05
1687 8.68383011720653e-06
1688 7.75235863104129e-06
1689 7.81056530207991e-06
1690 8.89715784468947e-06
1691 1.07091177500296e-05
1692 1.23631447894179e-05
1693 1.27161964838529e-05
1694 1.12965226999606e-05
1695 9.28442756836789e-06
1696 8.30033214235804e-06
1697 8.94942234258567e-06
1698 1.04471085131119e-05
1699 1.16650264633256e-05
1700 1.18160371242459e-05
1701 1.08265081373293e-05
1702 9.26019704805547e-06
1703 7.9983729153188e-06
1704 7.61407824723115e-06
1705 8.31318342109366e-06
1706 9.95457297192637e-06
1707 1.19197669373738e-05
1708 1.30557330568215e-05
1709 1.23339589735705e-05
1710 1.0253464779425e-05
1711 8.52181995980583e-06
1712 8.50189458248551e-06
1713 9.78736700549265e-06
1714 1.12231069127056e-05
1715 1.17999029977667e-05
1716 1.11317477206052e-05
1717 9.65603664826453e-06
1718 8.23385082626693e-06
1719 7.58844548787163e-06
1720 8.0068120825727e-06
1721 9.41454941527642e-06
1722 1.13373896858393e-05
1723 1.27196569553734e-05
1724 1.24278650347198e-05
1725 1.05918927800208e-05
1726 8.67611171895266e-06
1727 8.23221482745828e-06
1728 9.24633361906735e-06
1729 1.06746038970873e-05
1730 1.15509729626817e-05
1731 1.12455869138439e-05
1732 1.00350733963639e-05
1733 8.62970295278931e-06
1734 7.67760054953825e-06
1735 7.67005602553272e-06
1736 8.7275989735236e-06
1737 1.05725562018488e-05
1738 1.24187038885726e-05
1739 1.30535307696911e-05
1740 1.17501464956538e-05
1741 9.57661733375081e-06
1742 8.30264146242654e-06
1743 8.78975695978168e-06
1744 1.02928155726045e-05
1745 1.15911798489465e-05
1746 1.18215817586642e-05
1747 1.08254960635712e-05
1748 9.23355308157792e-06
1749 7.91808142330819e-06
1750 7.56414381786197e-06
1751 8.35848827118735e-06
1752 1.00701017629046e-05
1753 1.20178543497329e-05
1754 1.29456150474438e-05
1755 1.20105273340165e-05
1756 9.88752106248492e-06
1757 8.27837266142994e-06
1758 8.31788533767197e-06
1759 9.38858239890594e-06
1760 1.04209984865822e-05
1761 1.06657893463313e-05
1762 9.95098166112207e-06
1763 8.74739004497194e-06
1764 7.7486605136734e-06
1765 7.42058817726488e-06
1766 7.97734177193643e-06
1767 9.31756692079677e-06
1768 1.0996498351401e-05
1769 1.21789853149146e-05
1770 1.1925928660228e-05
1771 1.03513686723067e-05
1772 8.57327526326834e-06
1773 7.98602930279779e-06
1774 8.82110466277064e-06
1775 1.02320063873895e-05
1776 1.13663010356813e-05
1777 1.14759544262433e-05
1778 1.05671174491784e-05
1779 9.18081669254445e-06
1780 7.90912179504666e-06
1781 7.40145461808339e-06
1782 7.89817845792129e-06
1783 9.34507998984202e-06
1784 1.13592456168488e-05
1785 1.29644461256406e-05
1786 1.29783618955659e-05
1787 1.12665684747215e-05
1788 8.99180767474639e-06
1789 8.0880139405326e-06
1790 8.87843852273584e-06
1791 1.03751014128572e-05
1792 1.14980213087312e-05
1793 1.14916633078832e-05
1794 1.03656338309444e-05
1795 8.86627897855874e-06
1796 7.71279514089063e-06
1797 7.42761920802248e-06
1798 8.18290069748429e-06
1799 9.78849213340016e-06
1800 1.16357228399644e-05
1801 1.26232277330995e-05
1802 1.18229902952016e-05
1803 9.88998830539489e-06
1804 8.21171795523101e-06
1805 8.07051703854977e-06
1806 9.15033828235989e-06
1807 1.04599600676547e-05
1808 1.10995511801715e-05
1809 1.06762667348992e-05
1810 9.49502851566031e-06
1811 8.19827016873553e-06
1812 7.40968854784363e-06
1813 7.50793843384834e-06
1814 8.57162324729988e-06
1815 1.03343976572967e-05
1816 1.21174403387436e-05
1817 1.27415108767681e-05
1818 1.16104098339398e-05
1819 9.48292770477921e-06
1820 8.08834349297349e-06
1821 8.30505808951942e-06
1822 9.62985734961896e-06
1823 1.09979108003366e-05
1824 1.15044247578355e-05
1825 1.08718588968659e-05
1826 9.47339233281497e-06
1827 8.08289162979037e-06
1828 7.33779374373356e-06
1829 7.56351986208603e-06
1830 8.7282605386596e-06
1831 1.05107628525225e-05
1832 1.21295247843545e-05
1833 1.24818816811256e-05
1834 1.10581195639503e-05
1835 9.02216839818171e-06
1836 7.93646670797787e-06
1837 8.36603395831803e-06
1838 9.71715839959142e-06
1839 1.09669318605898e-05
1840 1.13171567252546e-05
1841 1.0590306999847e-05
1842 9.21023169342394e-06
1843 7.9034523530197e-06
1844 7.27608939599111e-06
1845 7.61422249617727e-06
1846 8.88112100350114e-06
1847 1.07132868945037e-05
1848 1.22671165242449e-05
1849 1.24300849984882e-05
1850 1.08698890853853e-05
1851 8.88897350773132e-06
1852 7.87596558157744e-06
1853 8.43932406646708e-06
1854 9.81851164183478e-06
1855 1.09810364938845e-05
1856 1.12156621660531e-05
1857 1.03585613254187e-05
1858 8.95832223457571e-06
1859 7.72853895381065e-06
1860 7.2635682181188e-06
1861 7.81821115025672e-06
1862 9.34379812617125e-06
1863 1.14134754841677e-05
1864 1.29020774852218e-05
1865 1.26022125722658e-05
1866 1.06158676130286e-05
1867 8.57245521992955e-06
1868 7.92225950844994e-06
1869 8.85107455866896e-06
1870 1.02620886745508e-05
1871 1.11347955371599e-05
1872 1.08832162648564e-05
1873 9.71890637901929e-06
1874 8.33141019229711e-06
1875 7.37563575670119e-06
1876 7.29555425871276e-06
1877 8.18761303544857e-06
1878 9.78460928091618e-06
1879 1.14565654790177e-05
1880 1.21552431936733e-05
1881 1.12061462283486e-05
1882 9.43625993651831e-06
1883 7.94907025888913e-06
1884 7.93451428909853e-06
1885 9.05241428433845e-06
1886 1.03947855089603e-05
1887 1.10388882550994e-05
1888 1.0607761676884e-05
1889 9.3893754434804e-06
1890 8.06646370965503e-06
1891 7.24599507095913e-06
1892 7.30496683098991e-06
1893 8.28699743421479e-06
1894 9.96390716290207e-06
1895 1.16848529386204e-05
1896 1.23474928876155e-05
1897 1.12273261319573e-05
1898 9.23112996026321e-06
1899 7.96729422536391e-06
1900 8.24944526188476e-06
1901 9.56781654321004e-06
1902 1.0832391166643e-05
1903 1.12191130617889e-05
1904 1.05115160953795e-05
1905 9.1772034164217e-06
1906 7.85192212415797e-06
1907 7.20645242557882e-06
1908 7.559427334769e-06
1909 8.9328016894541e-06
1910 1.0981117824882e-05
1911 1.27274526336429e-05
1912 1.28931001124877e-05
1913 1.10952044783119e-05
1914 8.86251122148018e-06
1915 7.78336684088832e-06
1916 8.39290380828805e-06
1917 9.74495377362161e-06
1918 1.07965147069233e-05
1919 1.08156886118871e-05
1920 9.81993894177968e-06
1921 8.44744157324495e-06
1922 7.39412522532401e-06
1923 7.14864626449874e-06
1924 7.87437492644205e-06
1925 9.38175306214317e-06
1926 1.11345999025958e-05
1927 1.2058991669478e-05
1928 1.13777126087289e-05
1929 9.50156865164864e-06
1930 8.00865266148421e-06
1931 7.76381741998677e-06
1932 8.84617259913623e-06
1933 1.02643927191171e-05
1934 1.11016130981412e-05
1935 1.08382770579452e-05
1936 9.64460599128802e-06
1937 8.2512088912523e-06
1938 7.29050910119877e-06
1939 7.1298989373636e-06
1940 7.9509249709675e-06
1941 9.69114847798602e-06
1942 1.17871363425892e-05
1943 1.30192528141337e-05
1944 1.22689521474262e-05
1945 1.00124768752097e-05
1946 8.18558067197661e-06
1947 7.8917682376084e-06
1948 9.06707138700891e-06
1949 1.04966356340996e-05
1950 1.11482525597395e-05
1951 1.06290357323457e-05
1952 9.32374313022954e-06
1953 7.92152116702027e-06
1954 7.11476996007043e-06
1955 7.19393650427946e-06
1956 8.15101059742362e-06
1957 9.66120169959561e-06
1958 1.10479818988818e-05
1959 1.13753246083481e-05
1960 1.02946025169331e-05
1961 8.5776190430209e-06
1962 7.57277963284175e-06
1963 7.80422286655869e-06
1964 9.00590258583189e-06
1965 1.02844742073671e-05
1966 1.08729355922377e-05
1967 1.04502615237978e-05
1968 9.22824849647341e-06
1969 7.91189092352873e-06
1970 7.11407519837781e-06
1971 7.19453714193019e-06
1972 8.22670830386008e-06
1973 1.00344904780947e-05
1974 1.19521651494869e-05
1975 1.27961967829326e-05
1976 1.17373608361365e-05
1977 9.57918664007407e-06
1978 7.9525813225928e-06
1979 8.07114620042515e-06
1980 9.45309579121478e-06
1981 1.09148394338843e-05
1982 1.14615068727564e-05
1983 1.07990810926939e-05
1984 9.34407586827835e-06
1985 7.86685722020675e-06
1986 7.10257655367474e-06
1987 7.39710376529423e-06
1988 8.71510216560178e-06
1989 1.06050618403075e-05
1990 1.21197555243979e-05
1991 1.20402676701548e-05
1992 1.02863321652935e-05
1993 8.29938821929332e-06
1994 7.62608678790688e-06
1995 8.43252458952726e-06
1996 9.89899547744477e-06
1997 1.09775177316029e-05
1998 1.07886073244523e-05
1999 9.57176761806622e-06
};
\addlegendentry{Train}
\addplot [semithick, black]
table {%
0 0.00785612966865301
1 0.00775936106219888
2 0.0076711499132216
3 0.00759026873856783
4 0.00751540949568152
5 0.00744633935391903
6 0.00737929670140147
7 0.00730900792405009
8 0.00722954235970974
9 0.00715052336454391
10 0.00708312075585127
11 0.00702984584495425
12 0.00698847929015756
13 0.00695188669487834
14 0.00691845687106252
15 0.00688604637980461
16 0.00685244332998991
17 0.00681773060932755
18 0.00678023137152195
19 0.00673716468736529
20 0.00668764626607299
21 0.00663061486557126
22 0.0065612499602139
23 0.0064771817997098
24 0.00637327041476965
25 0.00624271482229233
26 0.00607887236401439
27 0.00586763257160783
28 0.00560006219893694
29 0.00527941714972258
30 0.00489018065854907
31 0.00444313744083047
32 0.00398532627150416
33 0.00359737011604011
34 0.00333419186063111
35 0.00314612640067935
36 0.00297775375656784
37 0.00282594957388937
38 0.00267920363694429
39 0.00254178256727755
40 0.00241480325348675
41 0.0022977814078331
42 0.00218843598850071
43 0.00208724220283329
44 0.00199339585378766
45 0.00190558086615056
46 0.00182409991975874
47 0.00174866465386003
48 0.00167847576085478
49 0.00161338178440928
50 0.00155283254571259
51 0.00149624072946608
52 0.00144316465593874
53 0.00139270292129368
54 0.00134375935886055
55 0.00129587599076331
56 0.00124334637075663
57 0.0011849565198645
58 0.00112602941226214
59 0.00106292602140456
60 0.00101079314481467
61 0.00097010494209826
62 0.000935087155085057
63 0.000904672429896891
64 0.000877722341101617
65 0.000853046018164605
66 0.000830303179100156
67 0.000809360586572438
68 0.000789919926319271
69 0.000771802035160363
70 0.000754793814849108
71 0.000738799863029271
72 0.000723790260963142
73 0.000709564425051212
74 0.000696065719239414
75 0.000683234597090632
76 0.000670942885335535
77 0.00065913115395233
78 0.000647797482088208
79 0.000636982789728791
80 0.000626598426606506
81 0.000616595614701509
82 0.000606941408477724
83 0.000597578473389149
84 0.000588655646424741
85 0.000580067688133568
86 0.000571720069274306
87 0.000563619483727962
88 0.000555633625481278
89 0.000548232812434435
90 0.000540288689080626
91 0.000532906618900597
92 0.000525954586919397
93 0.000519121531397104
94 0.000512531667482108
95 0.000506029173266143
96 0.000499653688166291
97 0.000493426399771124
98 0.000487309211166576
99 0.000481325754662976
100 0.000475538254249841
101 0.00046976440353319
102 0.000464061158709228
103 0.000458523922134191
104 0.000453094369731843
105 0.000447656522737816
106 0.000442313466919586
107 0.000437088281614706
108 0.000431915279477835
109 0.000426732149207965
110 0.000421672913944349
111 0.000416659546317533
112 0.000411706860177219
113 0.000406765175284818
114 0.000401899655116722
115 0.000397105643060058
116 0.000392354209907353
117 0.000387492415029556
118 0.000382253725547343
119 0.000377539487089962
120 0.000372740265447646
121 0.000368112174328417
122 0.000363582163117826
123 0.000358817080268636
124 0.000354126299498603
125 0.000349412206560373
126 0.000344794156262651
127 0.000340075872372836
128 0.00033542129676789
129 0.000330620241584256
130 0.000326005014358088
131 0.000321283936500549
132 0.000316746882162988
133 0.000312012183712795
134 0.000307464390061796
135 0.000302826840197667
136 0.000298286642646417
137 0.000293708464596421
138 0.000289245159365237
139 0.000284741341602057
140 0.000280257896520197
141 0.000275657512247562
142 0.000271177559625357
143 0.000266605522483587
144 0.000262168177869171
145 0.000257675652392209
146 0.000253371283179149
147 0.000248966971412301
148 0.000244772323640063
149 0.000240430570556782
150 0.000236376828979701
151 0.000232234262512065
152 0.000228128759772517
153 0.000224083065404557
154 0.000220161571633071
155 0.000216197629924864
156 0.000212250088225119
157 0.000208415833185427
158 0.000204630181542598
159 0.000200976224732585
160 0.000197588873561472
161 0.000194148509763181
162 0.000191266561159864
163 0.000188069287105463
164 0.00018484122119844
165 0.000181882787728682
166 0.000178974631126039
167 0.000176169807673432
168 0.000173351712874137
169 0.000170743602211587
170 0.000168164187925868
171 0.000165684177773073
172 0.000162975236889906
173 0.000160504103405401
174 0.000158093331265263
175 0.000155573347001337
176 0.000153175176819786
177 0.000150986525113694
178 0.000148808103404008
179 0.000146696562296711
180 0.000144432706292719
181 0.000142436954774894
182 0.000140456046210602
183 0.000138324132421985
184 0.000136188973556273
185 0.000134258560137823
186 0.000132319881231524
187 0.000130546410218813
188 0.000128676008898765
189 0.000126696351799183
190 0.000124638332636096
191 0.000122847821330652
192 0.000121047210996039
193 0.00011921941768378
194 0.000117539901111741
195 0.000115867922431789
196 0.000114201684482396
197 0.000112652764073573
198 0.00011102052667411
199 0.000109543987491634
200 0.000108034706499893
201 0.000106448955193628
202 0.000105008541140705
203 0.000103436228528153
204 0.000102039230114315
205 0.000100563498563133
206 9.91294582490809e-05
207 9.77622985374182e-05
208 9.64580904110335e-05
209 9.50571047724225e-05
210 9.37868753680959e-05
211 9.24797132029198e-05
212 9.127426892519e-05
213 8.9984925580211e-05
214 8.88325448613614e-05
215 8.75390105647966e-05
216 8.64142057253048e-05
217 8.52418961585499e-05
218 8.41297805891372e-05
219 8.288770186482e-05
220 8.19539491203614e-05
221 8.10025521786883e-05
222 7.98932160250843e-05
223 7.86456803325564e-05
224 7.77079403633252e-05
225 7.66541197663173e-05
226 7.54902066546492e-05
227 7.45443903724663e-05
228 7.33991255401634e-05
229 7.26007201592438e-05
230 7.1702801506035e-05
231 7.08677325746976e-05
232 7.00197924743406e-05
233 6.92005924065597e-05
234 6.84714541421272e-05
235 6.78747310303152e-05
236 6.70036461087875e-05
237 6.6424130636733e-05
238 6.55224284855649e-05
239 6.51131704216823e-05
240 6.43828243482858e-05
241 6.36968979961239e-05
242 6.30064605502412e-05
243 6.23203231953084e-05
244 6.14987802691758e-05
245 6.09455928497482e-05
246 6.03303778916597e-05
247 5.96258396399207e-05
248 5.88967122894246e-05
249 5.83218097744975e-05
250 5.77013852307573e-05
251 5.70990960113704e-05
252 5.66298695048317e-05
253 5.62248751521111e-05
254 5.56449267605785e-05
255 5.5332242482109e-05
256 5.49476135347504e-05
257 5.44682479812764e-05
258 5.42177294846624e-05
259 5.39084430783987e-05
260 5.34186474396847e-05
261 5.31503537786193e-05
262 5.27263327967376e-05
263 5.21503498021048e-05
264 5.18213128088973e-05
265 5.13394807057921e-05
266 5.07317599840462e-05
267 5.03210903843865e-05
268 4.97988730785437e-05
269 4.92053441121243e-05
270 4.87154575239401e-05
271 4.80584712931886e-05
272 4.75330416520592e-05
273 4.71127423224971e-05
274 4.66652199975215e-05
275 4.63276955997571e-05
276 4.59334078186657e-05
277 4.5734501327388e-05
278 4.54494947916828e-05
279 4.52511885669082e-05
280 4.51133491878863e-05
281 4.49969884357415e-05
282 4.50218503829092e-05
283 4.51775194960646e-05
284 4.52976273663808e-05
285 4.54899673059117e-05
286 4.54518740298226e-05
287 4.56794477941003e-05
288 4.57959358755033e-05
289 4.55399786005728e-05
290 4.54643231933005e-05
291 4.50978659500834e-05
292 4.43096141680144e-05
293 4.36619156971574e-05
294 4.27290651714429e-05
295 4.17638511862606e-05
296 4.09298118029255e-05
297 4.01191682612989e-05
298 3.94330709241331e-05
299 3.89860870200209e-05
300 3.88057742384262e-05
301 3.89775559597183e-05
302 3.96374562114943e-05
303 4.10124921472743e-05
304 4.36332375102211e-05
305 4.77219582535326e-05
306 5.25150753674097e-05
307 5.51197044842411e-05
308 5.05875323142391e-05
309 4.21032891608775e-05
310 3.74612427549437e-05
311 3.94883209082764e-05
312 4.45867044618353e-05
313 4.76687091577332e-05
314 4.67639183625579e-05
315 4.3252879549982e-05
316 3.98781521653291e-05
317 3.73906368622556e-05
318 3.60505291610025e-05
319 3.54111725755502e-05
320 3.50992086168844e-05
321 3.49309739249293e-05
322 3.47826862707734e-05
323 3.46319102391135e-05
324 3.44717736879829e-05
325 3.42875464411918e-05
326 3.40913247782737e-05
327 3.38756508426741e-05
328 3.37509736709762e-05
329 3.36424564011395e-05
330 3.36011580657214e-05
331 3.36197117576376e-05
332 3.37057390424889e-05
333 3.38549689331558e-05
334 3.41506201948505e-05
335 3.44244144798722e-05
336 3.47627465089317e-05
337 3.52188326360192e-05
338 3.57791104761418e-05
339 3.63265098712873e-05
340 3.69664958270732e-05
341 3.72797912859824e-05
342 3.77486176148523e-05
343 3.75015952158719e-05
344 3.75337367586326e-05
345 3.66268250218127e-05
346 3.56262644345406e-05
347 3.44173131452408e-05
348 3.3340576919727e-05
349 3.23699714499526e-05
350 3.16963851219043e-05
351 3.14053941110615e-05
352 3.170176569256e-05
353 3.29407339449972e-05
354 3.53957220795564e-05
355 3.94297130696941e-05
356 4.49807666882407e-05
357 4.95040658279322e-05
358 4.74545304314233e-05
359 3.86864776373841e-05
360 3.13738783006556e-05
361 3.10894611175172e-05
362 3.49159054167103e-05
363 3.80568271793891e-05
364 3.82187281502411e-05
365 3.58844627044164e-05
366 3.32585550495423e-05
367 3.1091934943106e-05
368 2.97822080028709e-05
369 2.91527921945089e-05
370 2.88567953248275e-05
371 2.87306065729354e-05
372 2.84040779661154e-05
373 2.84053294308251e-05
374 2.8227692382643e-05
375 2.81391203316161e-05
376 2.79438008874422e-05
377 2.78385305136908e-05
378 2.76516911981162e-05
379 2.76026039500721e-05
380 2.75104393949732e-05
381 2.7540372684598e-05
382 2.75846032309346e-05
383 2.7725496693165e-05
384 2.7821921321447e-05
385 2.79960568150273e-05
386 2.81286138488213e-05
387 2.82435976259876e-05
388 2.83804893115303e-05
389 2.85227470158134e-05
390 2.8566288165166e-05
391 2.86208924080711e-05
392 2.85947789961938e-05
393 2.85812511719996e-05
394 2.84653306152904e-05
395 2.85013247776078e-05
396 2.83547251456184e-05
397 2.82121563941473e-05
398 2.81163866020506e-05
399 2.78990446531679e-05
400 2.76845930784475e-05
401 2.74165231530787e-05
402 2.71387834800407e-05
403 2.68448620772688e-05
404 2.65070520981681e-05
405 2.63047986663878e-05
406 2.59414009633474e-05
407 2.56099629041273e-05
408 2.54005844908534e-05
409 2.54639144259272e-05
410 2.61960831267061e-05
411 2.87279090116499e-05
412 3.56818018190097e-05
413 5.18614797329064e-05
414 7.88231700425968e-05
415 8.79842409631237e-05
416 4.33556415373459e-05
417 2.72317647613818e-05
418 3.74613809981383e-05
419 3.49158763128798e-05
420 2.89690106001217e-05
421 2.61185214185389e-05
422 2.49732347583631e-05
423 2.44220573222265e-05
424 2.41105099121341e-05
425 2.39149212575285e-05
426 2.37262102018576e-05
427 2.35849729506299e-05
428 2.34920189541299e-05
429 2.34112503676442e-05
430 2.33103291975567e-05
431 2.32343882089481e-05
432 2.31435442401562e-05
433 2.30927726079244e-05
434 2.30215719057014e-05
435 2.29626184591325e-05
436 2.28708504437236e-05
437 2.28210974455578e-05
438 2.2749158233637e-05
439 2.26840074901702e-05
440 2.2603182515013e-05
441 2.25466119445628e-05
442 2.24782997975126e-05
443 2.24075665755663e-05
444 2.23358965740772e-05
445 2.2279475160758e-05
446 2.22132894123206e-05
447 2.2168784198584e-05
448 2.20985548367025e-05
449 2.20581732719438e-05
450 2.20069705392234e-05
451 2.19836329051759e-05
452 2.19568373722723e-05
453 2.19459925574483e-05
454 2.19494668272091e-05
455 2.20176116272341e-05
456 2.20741949306102e-05
457 2.21918016904965e-05
458 2.23388688027626e-05
459 2.25340681936359e-05
460 2.27940963668516e-05
461 2.31044068641495e-05
462 2.34506424021674e-05
463 2.37452477449551e-05
464 2.39086166402558e-05
465 2.37864987866487e-05
466 2.33862992899958e-05
467 2.27633099711966e-05
468 2.20577039726777e-05
469 2.15569052670617e-05
470 2.17638516915031e-05
471 2.31383346545044e-05
472 2.61482564383186e-05
473 3.11118092213292e-05
474 3.74266419385094e-05
475 4.40878502558917e-05
476 4.5566979679279e-05
477 3.79851917386986e-05
478 3.06964793708175e-05
479 2.39450182561995e-05
480 2.15973559534177e-05
481 2.21612735913368e-05
482 2.46791951212799e-05
483 2.79897794825956e-05
484 3.0362974939635e-05
485 2.98006762022851e-05
486 2.59766056842636e-05
487 2.17597280425252e-05
488 2.00922022486338e-05
489 2.06225540750893e-05
490 2.18507666431833e-05
491 2.25933763431385e-05
492 2.2563335733139e-05
493 2.19497705984395e-05
494 2.1108215150889e-05
495 2.03096642508171e-05
496 1.98202233150369e-05
497 1.96197233890416e-05
498 1.9746850739466e-05
499 2.00827271328308e-05
500 2.06346649065381e-05
501 2.12077684409451e-05
502 2.16367643588455e-05
503 2.16921562241623e-05
504 2.1220754206297e-05
505 2.04201442102203e-05
506 1.96240725927055e-05
507 1.92956049431814e-05
508 1.95773063751403e-05
509 2.04419884539675e-05
510 2.17111246456625e-05
511 2.29835277423263e-05
512 2.3942275220179e-05
513 2.43051108554937e-05
514 2.39246110140812e-05
515 2.3171760403784e-05
516 2.18689401663141e-05
517 2.05595988518326e-05
518 1.95752618310507e-05
519 1.9297252947581e-05
520 2.00757949642139e-05
521 2.26333322643768e-05
522 2.77794115390861e-05
523 3.58894030796364e-05
524 4.34428038715851e-05
525 4.1513929318171e-05
526 2.82251949101919e-05
527 1.97647932509426e-05
528 2.15274358197348e-05
529 2.55135419138242e-05
530 2.64868886006298e-05
531 2.4505854526069e-05
532 2.19735975406365e-05
533 1.97891276911832e-05
534 1.86064917215845e-05
535 1.81440154847223e-05
536 1.81463747139787e-05
537 1.83763804670889e-05
538 1.85876015166286e-05
539 1.86927100003231e-05
540 1.85964327101829e-05
541 1.83216779987561e-05
542 1.80255192390177e-05
543 1.781606079021e-05
544 1.77388628799235e-05
545 1.78210939338896e-05
546 1.80281149368966e-05
547 1.82181920536095e-05
548 1.83706215466373e-05
549 1.83819374797167e-05
550 1.8300284864381e-05
551 1.80842744157417e-05
552 1.78554764715955e-05
553 1.7612690498936e-05
554 1.74813103512861e-05
555 1.75186287378892e-05
556 1.79491635208251e-05
557 1.89744077943033e-05
558 2.09313711820869e-05
559 2.42549740505638e-05
560 2.87898419628618e-05
561 3.22697524097748e-05
562 3.133919017273e-05
563 2.50671982939821e-05
564 1.91847739188233e-05
565 1.926102595462e-05
566 2.39532855630387e-05
567 2.83611680060858e-05
568 2.89134131890023e-05
569 2.72748766292352e-05
570 2.40092213061871e-05
571 2.03925756068202e-05
572 1.8007805920206e-05
573 1.72421659954125e-05
574 1.75407767528668e-05
575 1.84565160452621e-05
576 1.94617587112589e-05
577 2.00879458134295e-05
578 1.99827227334026e-05
579 1.90759146789787e-05
580 1.78324189619161e-05
581 1.70050770975649e-05
582 1.70809034898411e-05
583 1.78990721906302e-05
584 1.89869242603891e-05
585 1.97528388525825e-05
586 1.98345242097275e-05
587 1.93318683159305e-05
588 1.84285581781296e-05
589 1.75428122020094e-05
590 1.69119630299974e-05
591 1.67020043591037e-05
592 1.71234860317782e-05
593 1.8355209249421e-05
594 2.05880078283371e-05
595 2.38315878959838e-05
596 2.71999433607562e-05
597 2.83636745734839e-05
598 2.52562967943959e-05
599 1.99861060536932e-05
600 1.72334839589894e-05
601 1.8585029465612e-05
602 2.20030124182813e-05
603 2.47730131377466e-05
604 2.49662662099581e-05
605 2.31951835303335e-05
606 2.03456274903147e-05
607 1.80127408384578e-05
608 1.66344489116454e-05
609 1.63217609951971e-05
610 1.69157829077449e-05
611 1.81087780219968e-05
612 1.95357733900892e-05
613 2.0637598936446e-05
614 2.07575158128748e-05
615 1.96616565517616e-05
616 1.76874309545383e-05
617 1.63850454555359e-05
618 1.6600861272309e-05
619 1.796695869416e-05
620 1.96020810108166e-05
621 2.05805481527932e-05
622 2.05135911528487e-05
623 1.94503445527516e-05
624 1.81430477823596e-05
625 1.69332106452202e-05
626 1.61300631589256e-05
627 1.59967858053278e-05
628 1.6687148672645e-05
629 1.83131596713793e-05
630 2.08685105462791e-05
631 2.40116587519879e-05
632 2.62922603724292e-05
633 2.52528789133066e-05
634 2.07321572815999e-05
635 1.6812944522826e-05
636 1.65151996043278e-05
637 1.91969575098483e-05
638 2.22485505219083e-05
639 2.33552000281634e-05
640 2.1966432541376e-05
641 2.00474078155821e-05
642 1.79094513441669e-05
643 1.62287888088031e-05
644 1.55952093336964e-05
645 1.59111295943148e-05
646 1.70514067576732e-05
647 1.86480483534979e-05
648 2.01818074856419e-05
649 2.08845049201045e-05
650 2.00885260710493e-05
651 1.79524631676031e-05
652 1.60038071044255e-05
653 1.56038095155964e-05
654 1.68252609000774e-05
655 1.86978923011338e-05
656 2.00974845938617e-05
657 2.02905739570269e-05
658 1.93832092918456e-05
659 1.79936996573815e-05
660 1.65183264471125e-05
661 1.54983008542331e-05
662 1.51935009853332e-05
663 1.5721287127235e-05
664 1.71829669852741e-05
665 1.95401062228484e-05
666 2.24957602767972e-05
667 2.46825875365175e-05
668 2.40574445342645e-05
669 2.0289875465096e-05
670 1.64434841281036e-05
671 1.54550652951002e-05
672 1.72847358044237e-05
673 2.01067596208304e-05
674 2.18084023799747e-05
675 2.14723113458604e-05
676 1.98476063815178e-05
677 1.75457407749491e-05
678 1.58248076331802e-05
679 1.48589278978761e-05
680 1.48541357702925e-05
681 1.56831993081141e-05
682 1.71776955539826e-05
683 1.88383819477167e-05
684 2.01005041162716e-05
685 2.00384492927697e-05
686 1.82379135367228e-05
687 1.58837883645901e-05
688 1.47293649206404e-05
689 1.53886430780403e-05
690 1.72277195815695e-05
691 1.91605195141165e-05
692 2.01141265279148e-05
693 1.96175533346832e-05
694 1.82672138180351e-05
695 1.65167020895751e-05
696 1.52412976603955e-05
697 1.45231379065081e-05
698 1.4715340512339e-05
699 1.59681112563703e-05
700 1.82477560883854e-05
701 2.12945687962929e-05
702 2.39965611399384e-05
703 2.41981797444168e-05
704 2.06669501494616e-05
705 1.6231628251262e-05
706 1.46589181895251e-05
707 1.62172946147621e-05
708 1.88832073035883e-05
709 2.07067223527702e-05
710 2.06132845050888e-05
711 1.89097572729224e-05
712 1.68995356943924e-05
713 1.51536969497101e-05
714 1.42082244565245e-05
715 1.42020553539624e-05
716 1.49875668284949e-05
717 1.63274817168713e-05
718 1.78302088897908e-05
719 1.89763923117425e-05
720 1.88839931070106e-05
721 1.72893087437842e-05
722 1.51679068949306e-05
723 1.40766824188177e-05
724 1.46796919580083e-05
725 1.64064585987944e-05
726 1.82797066372586e-05
727 1.92015904758591e-05
728 1.88069152500248e-05
729 1.73792086570757e-05
730 1.60071849677479e-05
731 1.47320861287881e-05
732 1.39378007588675e-05
733 1.40328957058955e-05
734 1.50855221363599e-05
735 1.71336760104168e-05
736 1.99643582163844e-05
737 2.27624168473994e-05
738 2.3581047571497e-05
739 2.09080426429864e-05
740 1.64064622367732e-05
741 1.40396841743495e-05
742 1.49927445818321e-05
743 1.76264929905301e-05
744 1.98274883587146e-05
745 2.01499842660269e-05
746 1.86246088560438e-05
747 1.68372025655117e-05
748 1.49985007737996e-05
749 1.37639890454011e-05
750 1.34967231133487e-05
751 1.4121355889074e-05
752 1.54362733155722e-05
753 1.69856102729682e-05
754 1.82850853889249e-05
755 1.85113131010439e-05
756 1.71526244230336e-05
757 1.49615852933493e-05
758 1.35313048303942e-05
759 1.39894691528752e-05
760 1.57166778080864e-05
761 1.76331614056835e-05
762 1.86627985385712e-05
763 1.83855936484179e-05
764 1.69813210959546e-05
765 1.54304179886822e-05
766 1.40847623697482e-05
767 1.3364598089538e-05
768 1.35802119984874e-05
769 1.47745604408556e-05
770 1.69184149854118e-05
771 1.96405380847864e-05
772 2.20151032408467e-05
773 2.19863522943342e-05
774 1.88734320545336e-05
775 1.49164761751308e-05
776 1.35024629344116e-05
777 1.51102331074071e-05
778 1.78775371750817e-05
779 1.98380748770433e-05
780 1.96301116375253e-05
781 1.77619422174757e-05
782 1.56100413732929e-05
783 1.39015701279277e-05
784 1.30734288177337e-05
785 1.3243361536297e-05
786 1.42237540785572e-05
787 1.57716094690841e-05
788 1.72966319951229e-05
789 1.81254508788697e-05
790 1.75499371835031e-05
791 1.55523193825502e-05
792 1.35322388814529e-05
793 1.31590632008738e-05
794 1.44545074363123e-05
795 1.64331013365882e-05
796 1.78666505235014e-05
797 1.80916504177731e-05
798 1.70230687217554e-05
799 1.54168174049119e-05
800 1.4036224456504e-05
801 1.31022179630236e-05
802 1.29925874716719e-05
803 1.38290806717123e-05
804 1.56441055878531e-05
805 1.81127652467694e-05
806 2.070272785204e-05
807 2.17754732148023e-05
808 1.97820427274564e-05
809 1.5649766282877e-05
810 1.32029263113509e-05
811 1.39570202009054e-05
812 1.6667847376084e-05
813 1.9067969333264e-05
814 1.95511456695385e-05
815 1.79890030267416e-05
816 1.6054433217505e-05
817 1.42644576044404e-05
818 1.29801683215192e-05
819 1.27561270346632e-05
820 1.34833999254624e-05
821 1.49233364936663e-05
822 1.66146219271468e-05
823 1.78019763552584e-05
824 1.78201189555693e-05
825 1.60781328304438e-05
826 1.37540728246677e-05
827 1.27335470097023e-05
828 1.3677387869393e-05
829 1.57212252815953e-05
830 1.75337900145678e-05
831 1.80824281414971e-05
832 1.71931333170505e-05
833 1.55009274749318e-05
834 1.39157382363919e-05
835 1.28455158119323e-05
836 1.26311879284913e-05
837 1.33701250888407e-05
838 1.50660343933851e-05
839 1.73972384800436e-05
840 1.97978406504262e-05
841 2.07509274332551e-05
842 1.88417125173146e-05
843 1.50668556671008e-05
844 1.28063584270421e-05
845 1.36004418891389e-05
846 1.61260832101107e-05
847 1.81878367584432e-05
848 1.8478223864804e-05
849 1.69520590134198e-05
850 1.51755166371004e-05
851 1.3401368050836e-05
852 1.24946291180095e-05
853 1.24659463835997e-05
854 1.32670456878259e-05
855 1.47208838825463e-05
856 1.64415887411451e-05
857 1.76401790668024e-05
858 1.75360619323328e-05
859 1.5692245142418e-05
860 1.33264584292192e-05
861 1.24612133731716e-05
862 1.36196249513887e-05
863 1.57428967213491e-05
864 1.7467205907451e-05
865 1.78664358827518e-05
866 1.67921716638375e-05
867 1.51110943988897e-05
868 1.35775380840641e-05
869 1.25269943964668e-05
870 1.23290719784563e-05
871 1.3049317203695e-05
872 1.46880984175368e-05
873 1.69575814652489e-05
874 1.9136794435326e-05
875 1.99493806576356e-05
876 1.81472714757547e-05
877 1.47548835229827e-05
878 1.2557319678308e-05
879 1.31578772197827e-05
880 1.56279602379072e-05
881 1.78874361154158e-05
882 1.84641976375133e-05
883 1.71038318512728e-05
884 1.53081855387427e-05
885 1.35260579554597e-05
886 1.24932457765681e-05
887 1.24962762129144e-05
888 1.35817808768479e-05
889 1.57703088916605e-05
890 1.85051212611143e-05
891 2.06372187676607e-05
892 2.04531006602338e-05
893 1.70939438248752e-05
894 1.31388269437593e-05
895 1.24698663057643e-05
896 1.47843193190056e-05
897 1.73972239281284e-05
898 1.8293249013368e-05
899 1.70250023074914e-05
900 1.48917106344015e-05
901 1.30700582303689e-05
902 1.21689881780185e-05
903 1.21469256555429e-05
904 1.29378167912364e-05
905 1.42430526466342e-05
906 1.56059049913893e-05
907 1.62708529387601e-05
908 1.57968970597722e-05
909 1.41475702548632e-05
910 1.2458778655855e-05
911 1.213388168253e-05
912 1.33606281451648e-05
913 1.52522497955943e-05
914 1.674474151514e-05
915 1.71857391251251e-05
916 1.63138302013976e-05
917 1.48386952787405e-05
918 1.33347866722033e-05
919 1.22967712741229e-05
920 1.20208796943189e-05
921 1.26329377962975e-05
922 1.41376331157517e-05
923 1.63778422574978e-05
924 1.8753051335807e-05
925 2.00692938960856e-05
926 1.85665267053992e-05
927 1.48608114614035e-05
928 1.2283623618714e-05
929 1.27381963466178e-05
930 1.53032306116074e-05
931 1.79466860572575e-05
932 1.90328810276696e-05
933 1.79210055648582e-05
934 1.62036940309918e-05
935 1.38702980621019e-05
936 1.2331488505879e-05
937 1.18039115477586e-05
938 1.22101819215459e-05
939 1.33066578200669e-05
940 1.46782040246762e-05
941 1.57828271767357e-05
942 1.58325256052194e-05
943 1.45833128044615e-05
944 1.27281573440996e-05
945 1.16695582619286e-05
946 1.22830369946314e-05
947 1.40878046295256e-05
948 1.59821556735551e-05
949 1.70907060237369e-05
950 1.68286642292514e-05
951 1.55057787196711e-05
952 1.39554440465872e-05
953 1.26019713206915e-05
954 1.1823559361801e-05
955 1.18444422696484e-05
956 1.27136554510798e-05
957 1.42995386340772e-05
958 1.6387493815273e-05
959 1.81048653757898e-05
960 1.81731884367764e-05
961 1.60598101501819e-05
962 1.30835578602273e-05
963 1.18058633233886e-05
964 1.31228462123545e-05
965 1.59605206135893e-05
966 1.84396940312581e-05
967 1.9087008695351e-05
968 1.75312652572757e-05
969 1.56241076183505e-05
970 1.32689283418586e-05
971 1.19333053589799e-05
972 1.17308636617963e-05
973 1.25287206174107e-05
974 1.40244655995048e-05
975 1.56819551193621e-05
976 1.67312846315326e-05
977 1.61745356308529e-05
978 1.40684524012613e-05
979 1.20517879622639e-05
980 1.1613635251706e-05
981 1.30788112073787e-05
982 1.53355649672449e-05
983 1.70475541381165e-05
984 1.73137359524844e-05
985 1.59686114784563e-05
986 1.40630536407116e-05
987 1.24805719678989e-05
988 1.16245819299365e-05
989 1.16700830403715e-05
990 1.25799788293079e-05
991 1.41585969686275e-05
992 1.59154642460635e-05
993 1.70916009665234e-05
994 1.67187590704998e-05
995 1.45358944791951e-05
996 1.21422081065248e-05
997 1.15829598144046e-05
998 1.31690412672469e-05
999 1.56348924065242e-05
1000 1.75058994500432e-05
1001 1.77757083292818e-05
1002 1.62520718731685e-05
1003 1.44090736284852e-05
1004 1.26363347590086e-05
1005 1.15610582724912e-05
1006 1.14796321213362e-05
1007 1.22967903735116e-05
1008 1.37582328534336e-05
1009 1.5445404642378e-05
1010 1.66118534252746e-05
1011 1.62514552357607e-05
1012 1.42789031087887e-05
1013 1.20269442049903e-05
1014 1.14756667244365e-05
1015 1.30122834889335e-05
1016 1.55439047375694e-05
1017 1.74763299582992e-05
1018 1.77255933522247e-05
1019 1.61789030244108e-05
1020 1.43552178997197e-05
1021 1.26564973470522e-05
1022 1.14881531771971e-05
1023 1.13482292363187e-05
1024 1.21411167128826e-05
1025 1.36070366352214e-05
1026 1.53259752551094e-05
1027 1.65672809089301e-05
1028 1.62461674335646e-05
1029 1.42102462632465e-05
1030 1.1945296137128e-05
1031 1.13226806206512e-05
1032 1.27376169984927e-05
1033 1.5026446817501e-05
1034 1.67583784786984e-05
1035 1.70921775861643e-05
1036 1.58754828589736e-05
1037 1.40566698973998e-05
1038 1.23783256640309e-05
1039 1.13905698526651e-05
1040 1.11809040390654e-05
1041 1.18928064694046e-05
1042 1.33462481244351e-05
1043 1.52100928971777e-05
1044 1.69053018908016e-05
1045 1.72408053913387e-05
1046 1.54337340063648e-05
1047 1.26314580484177e-05
1048 1.12343668661197e-05
1049 1.20934855658561e-05
1050 1.43444076456944e-05
1051 1.65284891409101e-05
1052 1.74651231645839e-05
1053 1.66391291713808e-05
1054 1.49039042298682e-05
1055 1.29025156638818e-05
1056 1.16079954750603e-05
1057 1.1033044756914e-05
1058 1.13711221274571e-05
1059 1.24429270726978e-05
1060 1.39994335768279e-05
1061 1.55210418597562e-05
1062 1.6179908925551e-05
1063 1.52467036969028e-05
1064 1.30228399939369e-05
1065 1.12687284854474e-05
1066 1.13967798824888e-05
1067 1.31610267999349e-05
1068 1.53449927893234e-05
1069 1.67607449839124e-05
1070 1.67529760801699e-05
1071 1.52179563883692e-05
1072 1.36453727463959e-05
1073 1.21381353892502e-05
1074 1.11513072624803e-05
1075 1.09870288724778e-05
1076 1.16982164399815e-05
1077 1.30900934891542e-05
1078 1.48499630086008e-05
1079 1.63838594744448e-05
1080 1.67155558301602e-05
1081 1.50588721226086e-05
1082 1.24039897855255e-05
1083 1.1059003554692e-05
1084 1.19093328976305e-05
1085 1.41853824970894e-05
1086 1.64105022122385e-05
1087 1.74081505974755e-05
1088 1.66135851031868e-05
1089 1.48431472553057e-05
1090 1.29372619994683e-05
1091 1.15047932922607e-05
1092 1.09196789708221e-05
1093 1.11377012217417e-05
1094 1.2064398106304e-05
1095 1.34028077809489e-05
1096 1.46916581797996e-05
1097 1.54166300490033e-05
1098 1.47060718518333e-05
1099 1.2867089935753e-05
1100 1.11493864096701e-05
1101 1.1070342225139e-05
1102 1.26192089737742e-05
1103 1.48402687045746e-05
1104 1.65072378877085e-05
1105 1.68169281096198e-05
1106 1.55031975737074e-05
1107 1.3967719496577e-05
1108 1.22955216284026e-05
1109 1.11879899122869e-05
1110 1.08544545582845e-05
1111 1.12427915155422e-05
1112 1.23039553727722e-05
1113 1.36455237225164e-05
1114 1.48738081406918e-05
1115 1.54261942952871e-05
1116 1.45232461363776e-05
1117 1.25380156532628e-05
1118 1.09810589492554e-05
1119 1.11857452793629e-05
1120 1.30357648231438e-05
1121 1.54219505930087e-05
1122 1.70828352565877e-05
1123 1.72209929587552e-05
1124 1.56526366481557e-05
1125 1.41414848258137e-05
1126 1.21930297609651e-05
1127 1.105971114157e-05
1128 1.09174170574988e-05
1129 1.16255241664476e-05
1130 1.30691150843631e-05
1131 1.47342898344505e-05
1132 1.58532202476636e-05
1133 1.57184349518502e-05
1134 1.39033281811862e-05
1135 1.15755674414686e-05
1136 1.0768257197924e-05
1137 1.19422757052234e-05
1138 1.40653728522011e-05
1139 1.58265538630076e-05
1140 1.63503500516526e-05
1141 1.52752418216551e-05
1142 1.36245798785239e-05
1143 1.20856548164738e-05
1144 1.09866605271236e-05
1145 1.06952593341703e-05
1146 1.1205869668629e-05
1147 1.23442332551349e-05
1148 1.38189252538723e-05
1149 1.49974421219667e-05
1150 1.51129925143323e-05
1151 1.3828770534019e-05
1152 1.18377656690427e-05
1153 1.07161986306892e-05
1154 1.1362079931132e-05
1155 1.33016192194191e-05
1156 1.53749278979376e-05
1157 1.65110031957738e-05
1158 1.61807583936024e-05
1159 1.46254951687297e-05
1160 1.31348560898914e-05
1161 1.15761440611095e-05
1162 1.07420073618414e-05
1163 1.0876349733735e-05
1164 1.18470261440962e-05
1165 1.35617401610943e-05
1166 1.54124190885341e-05
1167 1.65157980518416e-05
1168 1.59396240633214e-05
1169 1.35707641675253e-05
1170 1.11552053567721e-05
1171 1.09038473965484e-05
1172 1.26341446957667e-05
1173 1.49024153870414e-05
1174 1.62601281772368e-05
1175 1.60775089170784e-05
1176 1.44012583405129e-05
1177 1.28328883874929e-05
1178 1.13568094093353e-05
1179 1.06133893496008e-05
1180 1.07059076981386e-05
1181 1.15460161396186e-05
1182 1.29330974232289e-05
1183 1.43169663715526e-05
1184 1.49315947055584e-05
1185 1.42333965413854e-05
1186 1.2446024811652e-05
1187 1.0846514669538e-05
1188 1.07415598904481e-05
1189 1.21384600788588e-05
1190 1.41507052831003e-05
1191 1.56109781528357e-05
1192 1.59138489834731e-05
1193 1.47444598042057e-05
1194 1.32216655401862e-05
1195 1.18444868348888e-05
1196 1.07794612631551e-05
1197 1.05998587969225e-05
1198 1.13156065708608e-05
1199 1.28470210256637e-05
1200 1.48628205351997e-05
1201 1.65493383974535e-05
1202 1.66019744938239e-05
1203 1.4652395293524e-05
1204 1.17963072625571e-05
1205 1.06438692455413e-05
1206 1.17651052278234e-05
1207 1.39888625199092e-05
1208 1.57387985382229e-05
1209 1.60287163453177e-05
1210 1.47075652421336e-05
1211 1.31411070469767e-05
1212 1.1506552255014e-05
1213 1.05853596323868e-05
1214 1.06024945125682e-05
1215 1.14623926492641e-05
1216 1.30402468130342e-05
1217 1.47763012137148e-05
1218 1.57551021402469e-05
1219 1.50273681356339e-05
1220 1.28770752780838e-05
1221 1.0867377568502e-05
1222 1.06753168438445e-05
1223 1.21807652249117e-05
1224 1.41638038257952e-05
1225 1.53648543346208e-05
1226 1.51547146742814e-05
1227 1.38173400046071e-05
1228 1.24226526168059e-05
1229 1.11168037619791e-05
1230 1.04457558336435e-05
1231 1.06011111711268e-05
1232 1.1582598745008e-05
1233 1.32237300931592e-05
1234 1.50981450133258e-05
1235 1.61040643433807e-05
1236 1.5387460734928e-05
1237 1.30901180455112e-05
1238 1.08831327452208e-05
1239 1.07257528725313e-05
1240 1.23676509247161e-05
1241 1.44796813401626e-05
1242 1.57572085299762e-05
1243 1.55748384713661e-05
1244 1.40554184326902e-05
1245 1.27012799566728e-05
1246 1.11631597974338e-05
1247 1.04365826700814e-05
1248 1.06292127384222e-05
1249 1.16395131044555e-05
1250 1.33384673972614e-05
1251 1.51392005136586e-05
1252 1.59414994413964e-05
1253 1.49275792864501e-05
1254 1.25194310385268e-05
1255 1.06451761894277e-05
1256 1.08260437627905e-05
1257 1.25628466776107e-05
1258 1.45363328556414e-05
1259 1.54822937474819e-05
1260 1.50894475154928e-05
1261 1.36179205583176e-05
1262 1.20220693133888e-05
1263 1.08715394162573e-05
1264 1.03203237813432e-05
1265 1.06610013972386e-05
1266 1.17840700113447e-05
1267 1.35382169901277e-05
1268 1.52845805132529e-05
1269 1.58771836140659e-05
1270 1.45799931488e-05
1271 1.20867725854623e-05
1272 1.04797045423766e-05
1273 1.10551136458525e-05
1274 1.30602638819255e-05
1275 1.49753686855547e-05
1276 1.56946334755048e-05
1277 1.49864963532309e-05
1278 1.34325664475909e-05
1279 1.19125943456311e-05
1280 1.07085552372155e-05
1281 1.02993290056475e-05
1282 1.08230542537058e-05
1283 1.22045184980379e-05
1284 1.4020921298652e-05
1285 1.55070083565079e-05
1286 1.55066300067119e-05
1287 1.35924065034487e-05
1288 1.1201325833099e-05
1289 1.03890924947336e-05
1290 1.16624378279084e-05
1291 1.38222303576185e-05
1292 1.52425764099462e-05
1293 1.51908152474789e-05
1294 1.38725872602663e-05
1295 1.22120545711368e-05
1296 1.07968253360013e-05
1297 1.0292581464455e-05
1298 1.08064577943878e-05
1299 1.23286390589783e-05
1300 1.4304336218629e-05
1301 1.57202630362008e-05
1302 1.5282683307305e-05
1303 1.28961200971389e-05
1304 1.07228288470651e-05
1305 1.06368415799807e-05
1306 1.23098679978284e-05
1307 1.41939044624451e-05
1308 1.49554098243243e-05
1309 1.43678680615267e-05
1310 1.27425637401757e-05
1311 1.12842735688901e-05
1312 1.03727143141441e-05
1313 1.03125394161907e-05
1314 1.11160306914826e-05
1315 1.2764868188242e-05
1316 1.4654649021395e-05
1317 1.57237518578768e-05
1318 1.48119161167415e-05
1319 1.23877935038763e-05
1320 1.04863447631942e-05
1321 1.07713603938464e-05
1322 1.27108660308295e-05
1323 1.47069476952311e-05
1324 1.54944154928671e-05
1325 1.47734472193406e-05
1326 1.29319741972722e-05
1327 1.15310167529969e-05
1328 1.04683686004137e-05
1329 1.0286320502928e-05
1330 1.10932342067827e-05
1331 1.28129286167677e-05
1332 1.48065264511388e-05
1333 1.59513529069955e-05
1334 1.50349587784149e-05
1335 1.2434814379958e-05
1336 1.04569817267475e-05
1337 1.08345820990507e-05
1338 1.28127303469228e-05
1339 1.46682450576918e-05
1340 1.52066777445725e-05
1341 1.42820381370257e-05
1342 1.26592658489244e-05
1343 1.11871895569493e-05
1344 1.02608137240168e-05
1345 1.03419133665739e-05
1346 1.14235699584242e-05
1347 1.33135836222209e-05
1348 1.52334687300026e-05
1349 1.59243336383952e-05
1350 1.43443685374223e-05
1351 1.166755646409e-05
1352 1.02536305348622e-05
1353 1.12041971078725e-05
1354 1.33251633087639e-05
1355 1.48870822158642e-05
1356 1.49937004607636e-05
1357 1.3692517313757e-05
1358 1.20411914394936e-05
1359 1.08101830846863e-05
1360 1.01394252851605e-05
1361 1.03951970231719e-05
1362 1.15727016236633e-05
1363 1.33757866933593e-05
1364 1.49273464558064e-05
1365 1.51418635141454e-05
1366 1.33718067445443e-05
1367 1.09892280306667e-05
1368 1.03473439594381e-05
1369 1.17535837489413e-05
1370 1.38546065500122e-05
1371 1.51156482388615e-05
1372 1.49426068674074e-05
1373 1.33958428705228e-05
1374 1.17405561468331e-05
1375 1.05852068372769e-05
1376 1.01481573437923e-05
1377 1.0680286322895e-05
1378 1.22332721730345e-05
1379 1.42932349262992e-05
1380 1.58427992573706e-05
1381 1.55302077473607e-05
1382 1.31546794364112e-05
1383 1.07098976513953e-05
1384 1.04173086583614e-05
1385 1.21632174341357e-05
1386 1.4369922610058e-05
1387 1.55349225678947e-05
1388 1.49801981024211e-05
1389 1.32837249111617e-05
1390 1.14909344119951e-05
1391 1.03590182334301e-05
1392 1.01532423286699e-05
1393 1.09871180029586e-05
1394 1.26647855722695e-05
1395 1.44919513331843e-05
1396 1.53577166202012e-05
1397 1.42149538078229e-05
1398 1.17720701382495e-05
1399 1.01951927717892e-05
1400 1.08026006273576e-05
1401 1.27441326185362e-05
1402 1.44826053656288e-05
1403 1.49376237459364e-05
1404 1.39561425385182e-05
1405 1.22442388601485e-05
1406 1.07883624878014e-05
1407 1.01333389466163e-05
1408 1.03886568467715e-05
1409 1.16973469630466e-05
1410 1.36860217025969e-05
1411 1.54677782120416e-05
1412 1.57941740326351e-05
1413 1.38258737933938e-05
1414 1.11430063043372e-05
1415 1.01812429420534e-05
1416 1.14734184535337e-05
1417 1.35684695123928e-05
1418 1.48851895573898e-05
1419 1.46466263686307e-05
1420 1.3149733604223e-05
1421 1.15396314868121e-05
1422 1.04073196780519e-05
1423 1.00460601970553e-05
1424 1.06767129182117e-05
1425 1.21860239232774e-05
1426 1.40127194754314e-05
1427 1.51969788930728e-05
1428 1.45664043884608e-05
1429 1.23072086353204e-05
1430 1.03238835436059e-05
1431 1.04217569969478e-05
1432 1.22437550089671e-05
1433 1.43153110911953e-05
1434 1.512676863058e-05
1435 1.44197192639695e-05
1436 1.27753537526587e-05
1437 1.12111083581112e-05
1438 1.02444910226041e-05
1439 1.01308505691122e-05
1440 1.1022136277461e-05
1441 1.26840141092543e-05
1442 1.44809509947663e-05
1443 1.53321198013145e-05
1444 1.43238448799821e-05
1445 1.18254210974555e-05
1446 1.01715058917762e-05
1447 1.06967227111454e-05
1448 1.26243139675353e-05
1449 1.44157729664585e-05
1450 1.49902980410843e-05
1451 1.4114799341769e-05
1452 1.23680547403637e-05
1453 1.08921703940723e-05
1454 1.01327041193144e-05
1455 1.0243204087601e-05
1456 1.13930345833069e-05
1457 1.32871564346715e-05
1458 1.51902104335022e-05
1459 1.58047514560167e-05
1460 1.4242556062527e-05
1461 1.14632211989374e-05
1462 1.00719371403102e-05
1463 1.10763221528032e-05
1464 1.32064324134262e-05
1465 1.48394547068165e-05
1466 1.49710995174246e-05
1467 1.36697171910782e-05
1468 1.19782425826997e-05
1469 1.06287643575342e-05
1470 1.00340785138542e-05
1471 1.04021619335981e-05
1472 1.17298968689283e-05
1473 1.35366135509685e-05
1474 1.49417064676527e-05
1475 1.47766704685637e-05
1476 1.28394976854906e-05
1477 1.05916806205641e-05
1478 1.00960342024337e-05
1479 1.1538030776137e-05
1480 1.35760183184175e-05
1481 1.48017943502055e-05
1482 1.46383963510743e-05
1483 1.32138229673728e-05
1484 1.15166094474262e-05
1485 1.03518632386113e-05
1486 1.00352735898923e-05
1487 1.06447023426881e-05
1488 1.22241990538896e-05
1489 1.42107901410782e-05
1490 1.55106627062196e-05
1491 1.49056413647486e-05
1492 1.24435764519149e-05
1493 1.02839194369153e-05
1494 1.03233733170782e-05
1495 1.20542863442097e-05
1496 1.39727771966136e-05
1497 1.47910723171663e-05
1498 1.41895943670534e-05
1499 1.24210337162367e-05
1500 1.10395194496959e-05
1501 1.01208588603185e-05
1502 1.01177902251948e-05
1503 1.11312019726029e-05
1504 1.30220541905146e-05
1505 1.50028536154423e-05
1506 1.57698978000553e-05
1507 1.43036631925497e-05
1508 1.155572499556e-05
1509 1.00059996839263e-05
1510 1.0883370123338e-05
1511 1.29032159748022e-05
1512 1.44351415656274e-05
1513 1.4608413039241e-05
1514 1.33163894133759e-05
1515 1.16226628961158e-05
1516 1.04674109024927e-05
1517 9.91268007055623e-06
1518 1.02921248981147e-05
1519 1.15100228867959e-05
1520 1.3250192751002e-05
1521 1.45983931361116e-05
1522 1.44605965033406e-05
1523 1.26189224829432e-05
1524 1.05361841633567e-05
1525 1.00260267572594e-05
1526 1.13944470285787e-05
1527 1.33592993734055e-05
1528 1.45832236739807e-05
1529 1.45256008181605e-05
1530 1.31046026581316e-05
1531 1.15858747449238e-05
1532 1.04275377452723e-05
1533 9.94120546238264e-06
1534 1.03840793599375e-05
1535 1.18749658213346e-05
1536 1.3894756193622e-05
1537 1.55140460265102e-05
1538 1.53106429934269e-05
1539 1.30405751406215e-05
1540 1.04988876046264e-05
1541 1.01248915598262e-05
1542 1.17496947495965e-05
1543 1.37695324156084e-05
1544 1.47959199239267e-05
1545 1.43767565532471e-05
1546 1.27019584397203e-05
1547 1.10845603558118e-05
1548 1.00964471130283e-05
1549 1.00279303296702e-05
1550 1.09628899735981e-05
1551 1.27753273773124e-05
1552 1.47234914038563e-05
1553 1.55084089783486e-05
1554 1.40665733852074e-05
1555 1.14682252387865e-05
1556 9.93087724054931e-06
1557 1.07273699541111e-05
1558 1.26427694340236e-05
1559 1.41113005156512e-05
1560 1.42964790939004e-05
1561 1.31946235342184e-05
1562 1.15066850412404e-05
1563 1.02952844827087e-05
1564 9.87897510640323e-06
1565 1.03472002592753e-05
1566 1.17741838039365e-05
1567 1.3638416021422e-05
1568 1.48562312460854e-05
1569 1.42804410643294e-05
1570 1.20778604468796e-05
1571 1.01771029221709e-05
1572 1.02944868558552e-05
1573 1.19783289846964e-05
1574 1.37943143272423e-05
1575 1.45782769322977e-05
1576 1.39750836751773e-05
1577 1.22841775009874e-05
1578 1.09347593024722e-05
1579 1.00211400422268e-05
1580 1.00756833489868e-05
1581 1.12180523501593e-05
1582 1.33532994368579e-05
1583 1.56112437252887e-05
1584 1.63970908033662e-05
1585 1.45901203723042e-05
1586 1.13745481939986e-05
1587 9.9422613857314e-06
1588 1.114693168347e-05
1589 1.32774084704579e-05
1590 1.46005731949117e-05
1591 1.43295765155926e-05
1592 1.28499359561829e-05
1593 1.11728350020712e-05
1594 1.00745155577897e-05
1595 9.86568102234742e-06
1596 1.06373900052859e-05
1597 1.22356323117856e-05
1598 1.39405901791179e-05
1599 1.46135816976312e-05
1600 1.34897045427351e-05
1601 1.12595344035071e-05
1602 9.87565181276295e-06
1603 1.04350610854453e-05
1604 1.21809453048627e-05
1605 1.37131701194448e-05
1606 1.41197006087168e-05
1607 1.32808545458829e-05
1608 1.16530254672398e-05
1609 1.04639884739299e-05
1610 9.85342012427282e-06
1611 1.00960796771687e-05
1612 1.13443529698998e-05
1613 1.32777940962114e-05
1614 1.50386122186319e-05
1615 1.52946704474743e-05
1616 1.33828161779093e-05
1617 1.08137983261258e-05
1618 9.90321768767899e-06
1619 1.12117022581515e-05
1620 1.33555076899938e-05
1621 1.47751898111892e-05
1622 1.4695822756039e-05
1623 1.31461065393523e-05
1624 1.15197954073665e-05
1625 1.02732838058728e-05
1626 9.87198745860951e-06
1627 1.05901508504758e-05
1628 1.23804065879085e-05
1629 1.4634838407801e-05
1630 1.59373212227365e-05
1631 1.48655362863792e-05
1632 1.19899978017202e-05
1633 9.90124590316555e-06
1634 1.04143537100754e-05
1635 1.22623741845018e-05
1636 1.37466668093111e-05
1637 1.39214571390767e-05
1638 1.28290175780421e-05
1639 1.12612424345571e-05
1640 1.00958313851152e-05
1641 9.77399577095639e-06
1642 1.03527099781786e-05
1643 1.18081743494258e-05
1644 1.36424450829509e-05
1645 1.47914706758456e-05
1646 1.41228356369538e-05
1647 1.18511707114521e-05
1648 1.00034130809945e-05
1649 1.014799636323e-05
1650 1.181805782835e-05
1651 1.35645532282069e-05
1652 1.42540957313031e-05
1653 1.3582457540906e-05
1654 1.20181821330334e-05
1655 1.07999439933337e-05
1656 9.9137932920712e-06
1657 9.88960073300404e-06
1658 1.08427830127766e-05
1659 1.2646275536099e-05
1660 1.46127158586751e-05
1661 1.54198678501416e-05
1662 1.40110396387172e-05
1663 1.13854384835577e-05
1664 9.82673100224929e-06
1665 1.06766010503634e-05
1666 1.27288785733981e-05
1667 1.43186980494647e-05
1668 1.45345384225948e-05
1669 1.33463145175483e-05
1670 1.16912378871348e-05
1671 1.03981046777335e-05
1672 9.78868320089532e-06
1673 1.02841122497921e-05
1674 1.18105381261557e-05
1675 1.3902511454944e-05
1676 1.54156114149373e-05
1677 1.48954668475199e-05
1678 1.24123635032447e-05
1679 1.00899078461225e-05
1680 1.00412353276624e-05
1681 1.17185272756615e-05
1682 1.34710453494336e-05
1683 1.41099553729873e-05
1684 1.3376400602283e-05
1685 1.17720846901648e-05
1686 1.04903201645357e-05
1687 9.78276420937618e-06
1688 1.0005308467953e-05
1689 1.12262023321819e-05
1690 1.31164961203467e-05
1691 1.47303790072328e-05
1692 1.47740393003915e-05
1693 1.27901666928665e-05
1694 1.04614564406802e-05
1695 9.80474669631803e-06
1696 1.12423867903999e-05
1697 1.3224026588432e-05
1698 1.43252491398016e-05
1699 1.39419671540963e-05
1700 1.24448715723702e-05
1701 1.10172341010184e-05
1702 9.98743144009495e-06
1703 9.76428145804675e-06
1704 1.05846420410671e-05
1705 1.23481077025644e-05
1706 1.43750157803879e-05
1707 1.53438213601476e-05
1708 1.40933598231641e-05
1709 1.14534295789781e-05
1710 9.76595219981391e-06
1711 1.04960481621674e-05
1712 1.24321450130083e-05
1713 1.39925450639566e-05
1714 1.41665414048475e-05
1715 1.29505024233367e-05
1716 1.13122741822735e-05
1717 1.01141185950837e-05
1718 9.71266035776353e-06
1719 1.02726144177723e-05
1720 1.17839026643196e-05
1721 1.37714232550934e-05
1722 1.50774458234082e-05
1723 1.43889747050707e-05
1724 1.19896949399845e-05
1725 9.92063633020734e-06
1726 1.00515435406123e-05
1727 1.17482713903883e-05
1728 1.34345054902951e-05
1729 1.40403299155878e-05
1730 1.32398217829177e-05
1731 1.16375467769103e-05
1732 1.04644368548179e-05
1733 9.75252260104753e-06
1734 9.89671207207721e-06
1735 1.10667942863074e-05
1736 1.30121425172547e-05
1737 1.48664057633141e-05
1738 1.52170769069926e-05
1739 1.33038565763854e-05
1740 1.07214182207827e-05
1741 9.72691759670852e-06
1742 1.10339806269621e-05
1743 1.30865064420504e-05
1744 1.43112984005711e-05
1745 1.40145921250223e-05
1746 1.25458891488961e-05
1747 1.09277361843851e-05
1748 9.87930252449587e-06
1749 9.76826686382992e-06
1750 1.0676028978196e-05
1751 1.25162387121236e-05
1752 1.4506245861412e-05
1753 1.51873673530645e-05
1754 1.37517999974079e-05
1755 1.10629453047295e-05
1756 9.66195329965558e-06
1757 1.04171449493151e-05
1758 1.19629457913106e-05
1759 1.29585205286276e-05
1760 1.28411747937207e-05
1761 1.18753814604133e-05
1762 1.05882509160438e-05
1763 9.74884187598946e-06
1764 9.64501123235095e-06
1765 1.03509737527929e-05
1766 1.18396201287396e-05
1767 1.35845139084267e-05
1768 1.46470210893312e-05
1769 1.39931426019757e-05
1770 1.18228290375555e-05
1771 9.94408674159786e-06
1772 9.81662378762849e-06
1773 1.12336092570331e-05
1774 1.29128911794396e-05
1775 1.384612005495e-05
1776 1.35967075038934e-05
1777 1.22813098641927e-05
1778 1.10176570160547e-05
1779 9.96218841464724e-06
1780 9.65480921877315e-06
1781 1.02937865449348e-05
1782 1.18711204777355e-05
1783 1.3999506336404e-05
1784 1.55351317516761e-05
1785 1.51420917973155e-05
1786 1.26337226902251e-05
1787 1.02098301795195e-05
1788 9.80025015451247e-06
1789 1.1305929547234e-05
1790 1.31196020447533e-05
1791 1.40378297146526e-05
1792 1.35915752252913e-05
1793 1.20925033115782e-05
1794 1.07487558125285e-05
1795 9.78375646809582e-06
1796 9.74477552517783e-06
1797 1.06768848127103e-05
1798 1.24118005260243e-05
1799 1.43324650707655e-05
1800 1.51181966430158e-05
1801 1.38144550874131e-05
1802 1.12586849354557e-05
1803 9.69501797953853e-06
1804 1.01302766779554e-05
1805 1.17356948976521e-05
1806 1.31587530631805e-05
1807 1.35264299387927e-05
1808 1.26773711599526e-05
1809 1.13126961878152e-05
1810 1.02053963928483e-05
1811 9.61812293098774e-06
1812 9.91762135527097e-06
1813 1.11250828922493e-05
1814 1.29903464767267e-05
1815 1.47922301039216e-05
1816 1.51496215039515e-05
1817 1.34447209347854e-05
1818 1.08523017843254e-05
1819 9.63781258178642e-06
1820 1.05354838524363e-05
1821 1.23486961456365e-05
1822 1.37121760417358e-05
1823 1.38699824674404e-05
1824 1.2752952898154e-05
1825 1.13477371996851e-05
1826 1.01190435088938e-05
1827 9.59622866503196e-06
1828 9.99500025500311e-06
1829 1.1276680197625e-05
1830 1.31760143631254e-05
1831 1.47618948176387e-05
1832 1.48227181853144e-05
1833 1.28346619021613e-05
1834 1.04689625004539e-05
1835 9.63325146585703e-06
1836 1.0712118637457e-05
1837 1.24872967717238e-05
1838 1.36447988552391e-05
1839 1.36168719109264e-05
1840 1.24681155284634e-05
1841 1.11080044007394e-05
1842 1.00101278803777e-05
1843 9.60208126343787e-06
1844 1.01321884358185e-05
1845 1.15179482236272e-05
1846 1.34664560391684e-05
1847 1.49541019709432e-05
1848 1.47764048961108e-05
1849 1.26394925246132e-05
1850 1.02164076452027e-05
1851 9.66480365605094e-06
1852 1.08877175080124e-05
1853 1.26301447380683e-05
1854 1.37138067657361e-05
1855 1.34407009682036e-05
1856 1.2199501725263e-05
1857 1.08730710053351e-05
1858 9.81852281256579e-06
1859 9.58649980020709e-06
1860 1.03417796708527e-05
1861 1.20112053991761e-05
1862 1.41845521284267e-05
1863 1.5583222193527e-05
1864 1.48431599882315e-05
1865 1.21866114568547e-05
1866 9.91858178167604e-06
1867 9.88305600913009e-06
1868 1.14989397843601e-05
1869 1.3121383744874e-05
1870 1.37694978548097e-05
1871 1.30054377223132e-05
1872 1.15714401545119e-05
1873 1.03707989183022e-05
1874 9.61125351750525e-06
1875 9.74615613813512e-06
1876 1.07931064121658e-05
1877 1.24947709991829e-05
1878 1.41886557685211e-05
1879 1.4649492186436e-05
1880 1.31998704091529e-05
1881 1.08360809463193e-05
1882 9.63281399890548e-06
1883 1.01891155281919e-05
1884 1.17729259727639e-05
1885 1.31890938064316e-05
1886 1.35679329105187e-05
1887 1.2652682926273e-05
1888 1.13183514258708e-05
1889 1.01519299278152e-05
1890 9.56827443587827e-06
1891 9.81210632744478e-06
1892 1.09317252281471e-05
1893 1.27087068904075e-05
1894 1.44397345138714e-05
1895 1.48156250361353e-05
1896 1.30883854581043e-05
1897 1.06749876067624e-05
1898 9.66906554822344e-06
1899 1.06686920844368e-05
1900 1.24469142974704e-05
1901 1.36520575324539e-05
1902 1.35510781547055e-05
1903 1.24445632536663e-05
1904 1.11255121737486e-05
1905 9.98113682726398e-06
1906 9.56199255597312e-06
1907 1.01185160019668e-05
1908 1.16664004963241e-05
1909 1.38243494802737e-05
1910 1.55444158735918e-05
1911 1.53467681229813e-05
1912 1.28384381241631e-05
1913 1.0122812454938e-05
1914 9.65574599831598e-06
1915 1.09336697278195e-05
1916 1.25809829114587e-05
1917 1.35079380925163e-05
1918 1.30358293972677e-05
1919 1.17024555947864e-05
1920 1.04877599369502e-05
1921 9.62007652560715e-06
1922 9.60693068918772e-06
1923 1.05179906313424e-05
1924 1.21539696920081e-05
1925 1.39656267492683e-05
1926 1.47315377034829e-05
1927 1.35692071125959e-05
1928 1.11846247818903e-05
1929 9.61538989940891e-06
1930 9.94049605651526e-06
1931 1.15647508209804e-05
1932 1.31082160805818e-05
1933 1.37113720484194e-05
1934 1.29081390696228e-05
1935 1.16038545456831e-05
1936 1.03576958281337e-05
1937 9.60359284363221e-06
1938 9.64596256380901e-06
1939 1.06306024463265e-05
1940 1.2523096302175e-05
1941 1.47562705024029e-05
1942 1.5814759535715e-05
1943 1.44967198139057e-05
1944 1.15753391582984e-05
1945 9.63175989454612e-06
1946 1.00562938314397e-05
1947 1.18569323603879e-05
1948 1.33709545480087e-05
1949 1.37288407131564e-05
1950 1.26252798509086e-05
1951 1.13164860522375e-05
1952 1.00529878181987e-05
1953 9.43883878790075e-06
1954 9.71763529378222e-06
1955 1.08151789390831e-05
1956 1.24229300126899e-05
1957 1.37916085805045e-05
1958 1.38362374855205e-05
1959 1.2323363989708e-05
1960 1.02989570223144e-05
1961 9.47592707234435e-06
1962 1.02255817182595e-05
1963 1.18259531518561e-05
1964 1.31032493300154e-05
1965 1.34000729303807e-05
1966 1.25257711260929e-05
1967 1.12256147986045e-05
1968 1.00819279396092e-05
1969 9.49610785028199e-06
1970 9.75316015683347e-06
1971 1.09435741251218e-05
1972 1.28925375975086e-05
1973 1.48360668390524e-05
1974 1.54155277414247e-05
1975 1.37649203679757e-05
1976 1.10282762761926e-05
1977 9.56902476900723e-06
1978 1.04418631963199e-05
1979 1.23674999485957e-05
1980 1.38297655212227e-05
1981 1.39657540785265e-05
1982 1.27322100524907e-05
1983 1.13281339508831e-05
1984 9.99898384179687e-06
1985 9.50850790104596e-06
1986 1.0044746886706e-05
1987 1.1523798093549e-05
1988 1.35638829306117e-05
1989 1.50218420458259e-05
1990 1.45300955409766e-05
1991 1.21485136332922e-05
1992 9.88609644991811e-06
1993 9.68047970673069e-06
1994 1.11765830297372e-05
1995 1.29320424093748e-05
1996 1.38091563712806e-05
1997 1.30626822283375e-05
1998 1.14944123197347e-05
1999 1.01801388154854e-05
};
\addlegendentry{Test}
\end{groupplot}

\end{tikzpicture}

		\label{Fig:Channels}
		\caption{Increasing the channel growth rate to quadratic. Shown are training- and validation error over 2000 epochs for the model with two, three and four convolutional layers in the encoder.}
	\end{figure}
\end{center}
The variation of activation functions follows the same scheme as already outlined in \cref{Ch:ApA}. Eight experiments are conducted using ReLU, ELU, Tanh, SiLU and LeakyReLU in different combinations for the hidden convolutional layers and the code layer. Results are summarized in \cref{Tab:Activations}. Training- and validation error over 2000 epochs are displayed in \cref{Fig:ActivationsC}. For both models the validation error reaches the region of \num{e-6} using combinations of ELU, SiLU and Tanh. With applying rectifiers, the validation error stays in the region of \num{e-5}. Especially, with utilizing SiLU/SiLU for the four layer model, the validation error reaches a minimum of \num{6.0e-6}. Utilizing the combinations ELU/SiLU and ELU/Tanh for the two layer model also yields a minimum validation error of \num{6.0e-6}. In addition, this combination produces the smallest \(\L2\) with \(\L2=0.024\) applying ELU/SiLU for the four layer model and applying ELU/Tanh for the two layer model produces \(\L2=0.023\). These values correspond to the minimum training error which is \num{4.0e-6} and \num{3.0e-6} respectively. Here the four layer model overfits after the 500th epoch which produces, compared to the other models, a significant generalization gap. The same logic applies to the two layer model with ELU/Tanh, but yields a smaller generalization gap. In general, specifically the training of the four layer model is instable for all combinations of activation functions. The greater amount of free parameters, compared to the two layer model, requires a smaller learning rate or a greater batch size to stabilize the training. Finally, the combination of SiLU/SiLU is chosen for the four layer model and ELU/SiLU for the two layer model. Both choices lead to the lowest validation error, while maintaining the generalization gap small.\\
\begin{table}[htbp!]
	\setlength{\tabcolsep}{.01pt}
	\centering
	\caption{Variation of activations for hidden-/code layers. Summary of minimum training- and minimum validation error for the models with two and four convolutional layers in the encoder as well as the corresponding \(\L2\) and the epoch in which those values are reached.}
	\begin{tabular*}{17cm}{ @{\extracolsep{\fill}} c c c c c c c c c @{} }
		\toprule
		Act. hid./code & \multicolumn{2}{c}{Min. training error}&\multicolumn{2}{c}{Min. validation error} & \multicolumn{2}{c}{$\L2$} &\multicolumn{2}{c}{Epoch}\\ [.5ex]
		& 2 Layer& 4 Layer& 2 Layer& 4 Layer& 2 Layer& 4 Layer\\
		\hline			%2			%4			%2                  %4     %2    %4
		ELU/ELU 	     &\num{5.0e-6} &\num{4.0e-6} &\num{7.0e-6} & \num{7.0e-6} & \num{0.026}  & \num{0.031}&1969  &1365\\ \hline
		ELU/SiLU         &\num{5.0e-6} &\num{4.0e-6} &\num{6.0e-6} & \num{8.0e-6} & \num{0.026}  & \num{0.024}&1991  &1808\\ \hline
		ELU/Tanh 	     &\num{3.0e-6} &\num{5.0e-6} &\num{6.0e-6} & \num{9.0e-6} & \num{0.023}  & \num{0.029}&1998  &1498\\ \hline
		Leaky/Leaky 	 &\num{6.0e-6} &\num{7.0e-6} &\num{1.1e-5} & \num{1.0e-5} & \num{0.032}  & \num{0.032}&1976  &1971\\ \hline
		Leaky/Tanh       &\num{5.0e-6} &\num{6.0e-6} &\num{7.0e-6} & \num{9.0e-6} & \num{0.030}  & \num{0.035}&1977  &1722\\ \hline
		ReLU/ReLU        &\num{8.0e-6} &\num{7.0e-6} &\num{1.3e-5} & \num{1.1e-5} & \num{0.036}  & \num{0.036}&1984  &1989\\ \hline
		SiLU/SiLU        &\num{6.0e-6} &\num{6.0e-6} &\num{8.0e-6} & \num{6.0e-6} & \num{0.030}  & \num{0.035}&1972  &1550\\ \hline
		Tanh/Tanh        &\num{8.0e-6} &\num{5.0e-6} &\num{8.0e-6} & \num{8.0e-6} & \num{0.033}  & \num{0.030}&1999  &975 \\ \hline
	\end{tabular*}\label{Tab:Activations}
\end{table}
A comparison with the validation loss of the fully connected model suggests, that the two- and three layer model gets stuck at a local minimum of  \(\min J(\frepar)\). Therefore the input data is augmentated for the following experiments. All examples in the training- and validation set are rotated around their center by \(180\degree\) and flipped about their central vertical axes. These methods add examples the input data without altering the information about the flow field present in each example. In \cref{Fig:DataAug} is an example in its original, rotated and flipped version. The flow field stays the same, just the direction in which it evlves has been altered. Using this method, available examples for training and testing triple from 80 to 240 examples.\\
\begin{center}
	\begin{figure}[htbp!]
		% This file was created by tikzplotlib v0.9.6.
\pgfplotsset{scaled y ticks=false}
\begin{tikzpicture}
\begin{groupplot}[
group style={group size=3 by 1,horizontal sep=1.5cm},
tick align=outside,
tick pos=left,
x grid style={white!69.0196078431373!black},
xtick style={color=black},
y grid style={white!69.0196078431373!black},
ytick style={color=black},
ytick={0,0.12},
%yticklabels={0,0.12},
xtick={0.2,0.8},
%xticklabels={0,0.8},
xlabel={\(x\)},
ylabel={\(t\)},
xmin=0.0025, xmax=0.9975,
ymin=0, ymax=0.12,
height=.19\textwidth,
width=.32\textwidth,
clip=false,
axis lines=box,
x label style={yshift=.5cm},
y label style={yshift=-.9cm}
]
\nextgroupplot[
]
\addplot graphics [includegraphics cmd=\pgfimage,xmin=0.0025, xmax=0.9975, ymin=0, ymax=0.12] {Figures/Parameterstudy/Convolutional/ExpDatAug-003.png};
\node [draw,fill=white] at (0.5,0.11) {\footnotesize original};
\nextgroupplot[
]
\addplot graphics [includegraphics cmd=\pgfimage,xmin=0.0025, xmax=0.9975, ymin=0, ymax=0.12] {Figures/Parameterstudy/Convolutional/ExpDatAug-004.png};
\node [draw,fill=white] at (0.5,0.11) {\footnotesize rotated};
\nextgroupplot[
colorbar,
colorbar style={
ylabel={$v$},
ytick={0.05,.25},
yticklabels={0.05,0.25},
y label style={yshift=1.6cm}
},
colormap/blackwhite,
point meta max=0.296770393848419,
point meta min=0.0158924162387848,
]
\addplot graphics [includegraphics cmd=\pgfimage,xmin=0.0025, xmax=0.9975, ymin=0, ymax=0.12] {Figures/Parameterstudy/Convolutional/ExpDatAug-005.png};
\node [draw,fill=white] at (0.5,0.11) {\footnotesize flipped};
\end{groupplot}

\end{tikzpicture}

		\caption{Rotated and flipped version of one original example from the dataset showing \(v\) over \(x\) and \(t\).}
		\label{Fig:DataAug}
	\end{figure}
\end{center}
Results for using augmented data are summarized in \cref{Tab:DataAug}. Two experiments are conducted for each model. One with learning rate adjustment, which decreases from \num{1e-4} to \num{1e-5} after the 1250th epoch, and one without. The minimum validation error ranges between \num{2.2e-5} and \num{1.4e-5}. Therefore data augmentation could not decrease the validation error further. Instead it increases. But, as seen in \cref{Fig:DatAug} and \cref{Fig:DatAugSched}, the training is stabilized compared to the previous experiments. In addition overfitting cannot be obeserved for the two layer model. The four layer model overfits after the around the 1150th epoch. Furthermore both models benefit from reducing the learning rate. A drop in training- and validation error can be observed after the 1250th epoch for both models.
\begin{table}[htbp!]
	\centering
	\caption{Data augmentation with and without learning rate adjustement after the 1250th epoch. Summary of minimum training- and minimum validation error for the models with two and four convolutional layers in the encoder as well as the corresponding \(\L2\) and the epoch in which those values are reached.}
	\begin{tabular*}{16cm}{ @{\extracolsep{\fill}} c c c c c @{} }
		\toprule
		Layer   & Min. training error & Min. validation error & \(\L2\) & Epoch \\ [.5ex]
		\hline
		2      & \num{1.5e-5}            & \num{1.4e-5}             & 0.048     & 2000  \\
		\hline  
		2 (lr adjusted)  & \num{2.2e-5}  & \num{2.2e-5}              & 0.064    & 2000  \\  
		\hline
		4       & \num{7.0e-6}           & \num{1.6e-5}              & 0.034    & 1991  \\
		\hline
		4 (lr adjusted)  & \num{1.1e-5}  & \num{1.9e-5}             & 0.045     & 2000  \\  
		\hline
	\end{tabular*}\label{Tab:DataAug}
\end{table}    
\begin{center}
	\begin{figure}[htbp!]
		% This file was created by tikzplotlib v0.9.6.
\begin{tikzpicture}

\begin{groupplot}[
group style={
group size=2 by 1, horizontal sep=2cm},
legend cell align={left},
legend style={fill opacity=1, draw opacity=1, text opacity=1, draw=white},
log basis y={10},
tick align=outside,
tick pos=left,
title style={at={(0.43,0.85)},anchor=north},
x grid style={white!69.0196078431373!black},
xlabel={Epoch},
x label style={yshift=13pt},
xmin=-49.95, xmax=2048.95,
xtick style={color=black},
xtick = {0,500,1500,2000},
y grid style={white!69.0196078431373!black},
ylabel={MSE Loss},
ymode=log,
ytick style={color=black},
width=.45\textwidth,
height=.25\textwidth
]
\nextgroupplot[
title={2 Layer},
ymin=9.48674748028202e-06, ymax=0.001,
]
\addplot [semithick, black, dashed]
table {%
0 0.0373725050594658
1 0.0348841479280964
2 0.0320690423250198
3 0.0283010646235198
4 0.0232329077941055
5 0.0183147593246152
6 0.0143406990682706
7 0.011252622755516
8 0.00887993141562523
9 0.00709671809454449
10 0.0057836054644819
11 0.00481779648301502
12 0.00409785606704342
13 0.00355175955216206
14 0.00312950955958513
15 0.00279685058073179
16 0.00252955873778168
17 0.00230976984403242
18 0.00212441015689061
19 0.00196439310457208
20 0.00182399836527717
21 0.00169914659939726
22 0.00158685561488407
23 0.00148478343589886
24 0.00139144748663966
25 0.00130608500224601
26 0.001227738389692
27 0.00115568102470813
28 0.00108935335174465
29 0.00102846320040347
30 0.00097259987205689
31 0.000921112972946503
32 0.000873481037312255
33 0.000829310352097915
34 0.000788294194990158
35 0.000750193777927658
36 0.000714810678952441
37 0.000682007465987529
38 0.000651656210417665
39 0.000623577891246896
40 0.00059759270129689
41 0.000573543825642749
42 0.000551294260276336
43 0.000530733024637205
44 0.000511749810812034
45 0.000494250833336688
46 0.000478120912172623
47 0.000463252589156582
48 0.000449542958828412
49 0.000436891964606427
50 0.000425207966069744
51 0.000414410083863004
52 0.000404423520914558
53 0.00039517910977338
54 0.000386613634759669
55 0.000378668877450157
56 0.000371291101373572
57 0.000364430142250664
58 0.000358040429584131
59 0.000352080932922642
60 0.000346516304981985
61 0.000341313735759741
62 0.000336443752170605
63 0.000331879796306112
64 0.000327598603424425
65 0.000323578309443443
66 0.000319798657225571
67 0.000316240840466738
68 0.000312887869180637
69 0.000309724651325875
70 0.000306737120051063
71 0.000303912030744868
72 0.000301236727295873
73 0.000298699890739347
74 0.000296291313948889
75 0.000294001679473392
76 0.00029182196946446
77 0.000289743971317572
78 0.000287759935209427
79 0.00028586276353811
80 0.000284045891002431
81 0.00028230295386796
82 0.000280628659121855
83 0.000279017685348511
84 0.000277465111594211
85 0.00027596642263461
86 0.000274517124106903
87 0.000273113080993426
88 0.000271750457860283
89 0.000270425710340305
90 0.000269135433171641
91 0.000267876582256577
92 0.000266646263905083
93 0.000265441861065104
94 0.000264260847510892
95 0.000263101126705578
96 0.000261960375382841
97 0.000260836478076953
98 0.000259727499980045
99 0.000258631647720146
100 0.000257546867042417
101 0.000256472016720484
102 0.000255405660643267
103 0.00025434642439374
104 0.000253292999881675
105 0.000252244092261359
106 0.00025119857654469
107 0.000250155326426693
108 0.000249113302980201
109 0.000248071705035121
110 0.000247029543203325
111 0.000245986089311145
112 0.000244940561998419
113 0.000243892326390475
114 0.000242840720621492
115 0.000241785205579011
116 0.000240725292731743
117 0.000239660450963205
118 0.000238590388647708
119 0.000237514681174389
120 0.000236433040148161
121 0.000235345274622508
122 0.0002342510287671
123 0.000233150137508649
124 0.000232042557987218
125 0.00023092815059537
126 0.000229806827642657
127 0.00022867871306668
128 0.000227543795252435
129 0.000226402135425019
130 0.00022525392950475
131 0.000224099383999032
132 0.00022293897167458
133 0.000221772888176967
134 0.000220601562498738
135 0.000219425123669718
136 0.000218243756431017
137 0.000217057833225454
138 0.000215867729555915
139 0.000214673759269128
140 0.000213476465376061
141 0.000212276859845891
142 0.000211075593549973
143 0.000209873432927073
144 0.000208671063451978
145 0.000207469150770597
146 0.000206268207885311
147 0.000205068895491915
148 0.000203872244502653
149 0.00020267904421208
150 0.000201489780446688
151 0.000200305808486216
152 0.000199127735541538
153 0.000197958307031361
154 0.000196798878424905
155 0.000195650764216756
156 0.000194514443222715
157 0.000193390463512818
158 0.000192279258323917
159 0.000191181155310953
160 0.00019009658418175
161 0.000189025894400174
162 0.000187969420447113
163 0.000186927517736043
164 0.000185900201264152
165 0.000184887681162612
166 0.000183890152740673
167 0.000182907731044679
168 0.000181940424174816
169 0.000180988266549775
170 0.000180051207811024
171 0.000179129014478955
172 0.000178222548010846
173 0.0001773314323259
174 0.00017645606722283
175 0.000175596297031196
176 0.000174751963517868
177 0.000173922740079509
178 0.000173108351428179
179 0.000172308486526163
180 0.000171522849569783
181 0.000170751145707489
182 0.000169993054276082
183 0.000169248261684629
184 0.000168516402946276
185 0.000167797187463255
186 0.000167090229221382
187 0.000166395200139391
188 0.000165711761795251
189 0.000165039594795265
190 0.000164378384752695
191 0.000163727813974409
192 0.000163087522319453
193 0.000162457201113853
194 0.000161836557197148
195 0.000161225174461303
196 0.000160622793558218
197 0.000160028997020352
198 0.000159443542505263
199 0.000158866071373609
200 0.000158296301466502
201 0.000157733874161181
202 0.000157178588648795
203 0.000156630124874842
204 0.000156088219538238
205 0.000155552575857124
206 0.000155022919531215
207 0.000154498973581478
208 0.000153980415504407
209 0.000153467077664023
210 0.00015295864675835
211 0.000152454914458625
212 0.000151955627266886
213 0.000151460542682003
214 0.000150969406182829
215 0.000150482033724586
216 0.000149998217056672
217 0.000149517777354428
218 0.000149040500173688
219 0.000148566215787109
220 0.000148094716960164
221 0.000147625817248809
222 0.000147159381614396
223 0.000146695194641685
224 0.000146233103303454
225 0.000145773015977587
226 0.000145314753436783
227 0.000144858149279041
228 0.000144403164791621
229 0.000143949653841939
230 0.000143497505104525
231 0.000143046637660404
232 0.000142596941425192
233 0.000142148344809338
234 0.000141700751865888
235 0.000141254120291497
236 0.000140808302785918
237 0.000140363327671385
238 0.000139919222912492
239 0.000139475914015937
240 0.000139033364725094
241 0.000138591550687295
242 0.00013815045584901
243 0.000137710061712445
244 0.000137270324257107
245 0.000136831249768932
246 0.000136392789530741
247 0.000135954980554705
248 0.000135517883980659
249 0.000135081469473164
250 0.000134645677401105
251 0.00013421051718628
252 0.000133776016300639
253 0.000133342152951836
254 0.000132908939373048
255 0.000132476447329092
256 0.000132044584089404
257 0.000131613409327732
258 0.00013118293455013
259 0.000130753246795715
260 0.000130324303846407
261 0.000129896139031397
262 0.000129468796342754
263 0.000129042252166774
264 0.000128616558541239
265 0.000128191682698287
266 0.000127767693676617
267 0.000127344613728061
268 0.000126922461539891
269 0.000126501268276513
270 0.000126081069727964
271 0.000125661972568025
272 0.000125243919131416
273 0.000124827033713378
274 0.000124411381176988
275 0.000123996982568523
276 0.000123583827350634
277 0.000123171982648292
278 0.000122761505918353
279 0.000122352452542884
280 0.000121944952994113
281 0.000121538963772612
282 0.000121134618166726
283 0.000120731976458899
284 0.000120331080713261
285 0.000119932033456394
286 0.000119535120840434
287 0.000119140710798623
288 0.000118748736066247
289 0.00011835935636384
290 0.000117972575165955
291 0.000117588478597478
292 0.000117207158486584
293 0.000116828627381456
294 0.000116452997922067
295 0.00011608037820802
296 0.000115710981513454
297 0.000115345230511821
298 0.000114983035681841
299 0.000114624288867068
300 0.000114269062753654
301 0.000113917517751361
302 0.000113569779088607
303 0.0001132258654953
304 0.000112885797046639
305 0.000112549671773839
306 0.000112217605398257
307 0.000111889696934024
308 0.000111565967114776
309 0.000111246420542462
310 0.000110931006692757
311 0.000110619829979915
312 0.000110312842534673
313 0.000110010058688677
314 0.000109711416544182
315 0.000109416861597822
316 0.000109126391362698
317 0.000108839936082461
318 0.000108557624938517
319 0.000108279146128136
320 0.000108004483472257
321 0.000107733509568201
322 0.000107466173519792
323 0.000107202362540211
324 0.000106942087346814
325 0.00010668521406861
326 0.000106431698005357
327 0.000106181531900044
328 0.000105934553997618
329 0.000105690665674748
330 0.000105449879422774
331 0.000105212099787385
332 0.000104977167500427
333 0.000104745062988817
334 0.000104515697273655
335 0.000104288984196899
336 0.000104064894330236
337 0.000103843349028428
338 0.000103624289927495
339 0.000103407664634384
340 0.000103193393044876
341 0.000102981419651371
342 0.000102771698626232
343 0.000102564183284433
344 0.000102358826526464
345 0.000102155586591361
346 0.000101954383950442
347 0.000101755200913563
348 0.000101557885573792
349 0.00010136252042751
350 0.000101169357535014
351 0.000100978206321637
352 0.000100788942316872
353 0.000100601491440718
354 0.000100415890893638
355 0.00010023203301183
356 0.000100049907833485
357 9.98694456780432e-05
358 9.96906725276858e-05
359 9.95135720517718e-05
360 9.93380449282692e-05
361 9.91640702198519e-05
362 9.89916170688332e-05
363 9.88206773785786e-05
364 9.86511887267246e-05
365 9.84830996131336e-05
366 9.8316429398911e-05
367 9.81511487765374e-05
368 9.79871789006381e-05
369 9.78245548542607e-05
370 9.76631992664068e-05
371 9.75031237130016e-05
372 9.73443317882072e-05
373 9.71868096873626e-05
374 9.70304728014071e-05
375 9.68753109541846e-05
376 9.67212902729386e-05
377 9.65683992445084e-05
378 9.64166915894775e-05
379 9.62661286211623e-05
380 9.61166878189109e-05
381 9.59682899429974e-05
382 9.58209771593962e-05
383 9.56747255867659e-05
384 9.55294984983368e-05
385 9.53852957993699e-05
386 9.52420650837856e-05
387 9.50998091025686e-05
388 9.49585322835844e-05
389 9.48181535740389e-05
390 9.46787207093773e-05
391 9.45401923336438e-05
392 9.44025470820028e-05
393 9.42657700010822e-05
394 9.41298620631414e-05
395 9.39947737824317e-05
396 9.38605366371803e-05
397 9.37271077686338e-05
398 9.35945123738217e-05
399 9.34627066015992e-05
400 9.33316831996933e-05
401 9.32014281846231e-05
402 9.30719418204736e-05
403 9.2943185787675e-05
404 9.28151780238788e-05
405 9.26878995753573e-05
406 9.25613419819153e-05
407 9.24355066788489e-05
408 9.2310330001529e-05
409 9.21858767325053e-05
410 9.20621065958471e-05
411 9.19390026451102e-05
412 9.18165681724759e-05
413 9.16947964597625e-05
414 9.15736642959075e-05
415 9.14531429074827e-05
416 9.13332467291639e-05
417 9.12138985214028e-05
418 9.10951650100837e-05
419 9.09770512270332e-05
420 9.08595387988006e-05
421 9.07427083814601e-05
422 9.06265241316836e-05
423 9.05109401709107e-05
424 9.03959723170071e-05
425 9.02816274965801e-05
426 9.01678778764866e-05
427 9.00547245995161e-05
428 8.9942116130004e-05
429 8.98301047627115e-05
430 8.97186521306992e-05
431 8.96077597604498e-05
432 8.94974242591218e-05
433 8.93876314715196e-05
434 8.92783719604514e-05
435 8.91696528834511e-05
436 8.90614719987563e-05
437 8.89537861503698e-05
438 8.88465994345703e-05
439 8.87399450715994e-05
440 8.86337838951571e-05
441 8.8527861077381e-05
442 8.84224696484637e-05
443 8.83175893188574e-05
444 8.82132337783522e-05
445 8.81094986506052e-05
446 8.80063610774566e-05
447 8.79038160957653e-05
448 8.78019068437652e-05
449 8.77005731894087e-05
450 8.75998668531016e-05
451 8.74997440559374e-05
452 8.74002036906537e-05
453 8.73012662824616e-05
454 8.7202901703165e-05
455 8.71050993372554e-05
456 8.7007858837751e-05
457 8.6911183949212e-05
458 8.68150255932676e-05
459 8.67193936038291e-05
460 8.66242725952778e-05
461 8.65296695664597e-05
462 8.64355456175285e-05
463 8.63418874198866e-05
464 8.62486954815722e-05
465 8.61559609030375e-05
466 8.60637057513713e-05
467 8.5971857741877e-05
468 8.58804312239651e-05
469 8.57894278472789e-05
470 8.56988244741785e-05
471 8.56085994840328e-05
472 8.55187603564881e-05
473 8.54293151123879e-05
474 8.53401976653364e-05
475 8.52514437106322e-05
476 8.51629898548367e-05
477 8.50748962809196e-05
478 8.4987099917555e-05
479 8.48996335041837e-05
480 8.4812441831635e-05
481 8.47255841615417e-05
482 8.46389808231152e-05
483 8.4552672404925e-05
484 8.44666015697252e-05
485 8.43808043124265e-05
486 8.42952582900125e-05
487 8.42098974006926e-05
488 8.41247525211012e-05
489 8.40398566239742e-05
490 8.39551558620144e-05
491 8.38706384745554e-05
492 8.37863160872606e-05
493 8.37021672583186e-05
494 8.36181985791977e-05
495 8.35343960865487e-05
496 8.34507653758957e-05
497 8.336727659497e-05
498 8.32838858855212e-05
499 8.32006536578926e-05
500 8.31175200962283e-05
501 8.30344811480662e-05
502 8.2951505689266e-05
503 8.28686113045762e-05
504 8.27857302982219e-05
505 8.27029059931779e-05
506 8.26201462954164e-05
507 8.25373728462845e-05
508 8.24546372750016e-05
509 8.2371912507521e-05
510 8.22891822049125e-05
511 8.22064190787823e-05
512 8.21236293321685e-05
513 8.20407999088483e-05
514 8.1957873343678e-05
515 8.18748924302781e-05
516 8.17917913534435e-05
517 8.17085732502202e-05
518 8.16252448826068e-05
519 8.15417957025962e-05
520 8.1458163377827e-05
521 8.13743951866286e-05
522 8.12904303278591e-05
523 8.12062797545347e-05
524 8.11219403118457e-05
525 8.10373743066843e-05
526 8.09526096669326e-05
527 8.08675706084709e-05
528 8.0782256418388e-05
529 8.06966863512078e-05
530 8.06108311834919e-05
531 8.05246706742461e-05
532 8.04381638002856e-05
533 8.03513610871202e-05
534 8.02641927795851e-05
535 8.01766557086599e-05
536 8.00886985625008e-05
537 8.00003979589311e-05
538 7.99116539352459e-05
539 7.98224736655584e-05
540 7.97328818222809e-05
541 7.96428435248705e-05
542 7.95522919159926e-05
543 7.94612660683924e-05
544 7.93697361487489e-05
545 7.92776678117946e-05
546 7.91849945092811e-05
547 7.90917956964658e-05
548 7.89979666417177e-05
549 7.89035339892052e-05
550 7.88084565061335e-05
551 7.87127165793322e-05
552 7.86163160029218e-05
553 7.85192049039078e-05
554 7.84213135638367e-05
555 7.83226757287518e-05
556 7.82232956488826e-05
557 7.81230884510838e-05
558 7.80220575921457e-05
559 7.792015406712e-05
560 7.78173922277858e-05
561 7.77136925928327e-05
562 7.76090413777316e-05
563 7.75034399858043e-05
564 7.73968514960662e-05
565 7.72891986263365e-05
566 7.71804385190459e-05
567 7.70705679210929e-05
568 7.69596274565743e-05
569 7.68475309852098e-05
570 7.67342669287056e-05
571 7.66198033825087e-05
572 7.65040643102566e-05
573 7.63869862924575e-05
574 7.62685883148132e-05
575 7.61487506153458e-05
576 7.60274334723476e-05
577 7.59046432510975e-05
578 7.57802133162973e-05
579 7.5654164769882e-05
580 7.55264092878368e-05
581 7.5396887687873e-05
582 7.52656145988813e-05
583 7.5132503802422e-05
584 7.49975392828617e-05
585 7.486067704221e-05
586 7.47218738072301e-05
587 7.45810879436704e-05
588 7.44382758159172e-05
589 7.42934421893438e-05
590 7.41465110311405e-05
591 7.39973978104066e-05
592 7.3846060801704e-05
593 7.36925022740327e-05
594 7.3536576407444e-05
595 7.33783660370098e-05
596 7.32179276852207e-05
597 7.30552455057894e-05
598 7.2890291905973e-05
599 7.27230215034069e-05
600 7.25533388153584e-05
601 7.23812814958554e-05
602 7.22067772696751e-05
603 7.2029752667883e-05
604 7.18506061640293e-05
605 7.16696366656796e-05
606 7.14870731475988e-05
607 7.13026824428198e-05
608 7.11163460541305e-05
609 7.09280093479e-05
610 7.07377174767506e-05
611 7.05455477287842e-05
612 7.03515752829749e-05
613 7.01558290181481e-05
614 6.99584339519295e-05
615 6.97593363862363e-05
616 6.95586535298294e-05
617 6.93564816411178e-05
618 6.91528646541428e-05
619 6.89478224747593e-05
620 6.87414563588599e-05
621 6.8533887887412e-05
622 6.83251317461497e-05
623 6.81152386761141e-05
624 6.79041632534923e-05
625 6.76920061882432e-05
626 6.74788341742764e-05
627 6.72646510035217e-05
628 6.70496117306148e-05
629 6.68337811831066e-05
630 6.66171534291493e-05
631 6.6399800759361e-05
632 6.61817870195591e-05
633 6.59632750957447e-05
634 6.57442946755774e-05
635 6.55249389135785e-05
636 6.53053030520804e-05
637 6.50854180020606e-05
638 6.4865381087742e-05
639 6.4645276759497e-05
640 6.44251905536445e-05
641 6.4205220972724e-05
642 6.39854560375852e-05
643 6.37660471033049e-05
644 6.3547049257077e-05
645 6.33285398829268e-05
646 6.31105751196515e-05
647 6.28933280696723e-05
648 6.26768308921536e-05
649 6.24611194742404e-05
650 6.22463100228288e-05
651 6.20324187323718e-05
652 6.18195700129813e-05
653 6.16078214115134e-05
654 6.13971797373362e-05
655 6.11877212139215e-05
656 6.09794617716375e-05
657 6.07725329828668e-05
658 6.05669678321874e-05
659 6.03628534451654e-05
660 6.01602502345126e-05
661 5.99591192553817e-05
662 5.97595393164359e-05
663 5.95615324409475e-05
664 5.93651486973101e-05
665 5.91704922735895e-05
666 5.89776428417584e-05
667 5.87865065793854e-05
668 5.85971673518297e-05
669 5.84096578973477e-05
670 5.82239927275907e-05
671 5.80402435067147e-05
672 5.78583608970007e-05
673 5.76784580156679e-05
674 5.75004688911918e-05
675 5.73244644073156e-05
676 5.71504422595126e-05
677 5.69784506592915e-05
678 5.68084838169132e-05
679 5.66405211657184e-05
680 5.64745950886921e-05
681 5.6310718332971e-05
682 5.61488150031172e-05
683 5.59889402100093e-05
684 5.58310532975762e-05
685 5.56751374851672e-05
686 5.55212564030683e-05
687 5.53693913083464e-05
688 5.52195052551478e-05
689 5.50716050433664e-05
690 5.492562956159e-05
691 5.47816339739882e-05
692 5.46395335424184e-05
693 5.44994238017201e-05
694 5.43611571449295e-05
695 5.42247388099308e-05
696 5.40901433213984e-05
697 5.39572991774169e-05
698 5.38262788083443e-05
699 5.36970033474896e-05
700 5.35693913773135e-05
701 5.34434448885198e-05
702 5.33191160663193e-05
703 5.31964149601549e-05
704 5.30753149755962e-05
705 5.2955744209271e-05
706 5.28377315838213e-05
707 5.27212689197161e-05
708 5.26063343725032e-05
709 5.24929512764061e-05
710 5.23809687346481e-05
711 5.22703940895042e-05
712 5.21612389832171e-05
713 5.20534054899713e-05
714 5.19468825892488e-05
715 5.18416560630897e-05
716 5.17375986971066e-05
717 5.16346815035244e-05
718 5.15329533025503e-05
719 5.14326165728107e-05
720 5.13334823916504e-05
721 5.12354727734513e-05
722 5.11386046942638e-05
723 5.10428018323239e-05
724 5.09481630596535e-05
725 5.08546755367452e-05
726 5.07624330351083e-05
727 5.06713708704846e-05
728 5.05814949782746e-05
729 5.04927033008566e-05
730 5.04049802749762e-05
731 5.0318365533523e-05
732 5.02328187685919e-05
733 5.01482831332112e-05
734 5.00647716184706e-05
735 4.99822163663547e-05
736 4.99005715835684e-05
737 4.9819832433684e-05
738 4.97399652523711e-05
739 4.96609205929606e-05
740 4.9582714018707e-05
741 4.95053105270908e-05
742 4.94286285028049e-05
743 4.93526996881618e-05
744 4.9277587565418e-05
745 4.92032342146823e-05
746 4.91295507648222e-05
747 4.90565061011713e-05
748 4.89840397754904e-05
749 4.89122159888685e-05
750 4.884098988119e-05
751 4.87703209183602e-05
752 4.87002734044969e-05
753 4.86307720137802e-05
754 4.85618782046041e-05
755 4.8493472495655e-05
756 4.84256017045936e-05
757 4.83582377495869e-05
758 4.82913516132536e-05
759 4.82249083984717e-05
760 4.81589613094968e-05
761 4.80934769117406e-05
762 4.80284476426599e-05
763 4.79638081483102e-05
764 4.78995582984254e-05
765 4.783570544357e-05
766 4.77722321014321e-05
767 4.77091684955392e-05
768 4.76465687674713e-05
769 4.75842840603017e-05
770 4.75223942763373e-05
771 4.7460827646025e-05
772 4.7399645689931e-05
773 4.7338744350256e-05
774 4.72781968679688e-05
775 4.72178675986849e-05
776 4.71578926285095e-05
777 4.70982187383839e-05
778 4.70388212138554e-05
779 4.69797196730089e-05
780 4.69209178556677e-05
781 4.68623777649668e-05
782 4.6804143614428e-05
783 4.67461553098995e-05
784 4.66883923569602e-05
785 4.66308647713021e-05
786 4.65736212831303e-05
787 4.65166178399793e-05
788 4.64598769157239e-05
789 4.64033022815613e-05
790 4.63470092177213e-05
791 4.62908980149261e-05
792 4.62350106094078e-05
793 4.61793157355099e-05
794 4.61238656939145e-05
795 4.60685847087916e-05
796 4.60135518203231e-05
797 4.59586764804953e-05
798 4.59040000677646e-05
799 4.58495843022414e-05
800 4.579528788895e-05
801 4.5741197756873e-05
802 4.56873165794273e-05
803 4.56335856267032e-05
804 4.55800628209602e-05
805 4.5526680972093e-05
806 4.54735294474299e-05
807 4.54204841175236e-05
808 4.53676694860405e-05
809 4.53150331806521e-05
810 4.52624938714526e-05
811 4.52101935296175e-05
812 4.51580203553921e-05
813 4.51059798892336e-05
814 4.50541522099002e-05
815 4.50024313938258e-05
816 4.49508910840279e-05
817 4.48994920129545e-05
818 4.48483096029406e-05
819 4.47972778238418e-05
820 4.47464666167482e-05
821 4.46957677997517e-05
822 4.46452027637415e-05
823 4.45948743722132e-05
824 4.45445264526019e-05
825 4.44942815841879e-05
826 4.44441933614333e-05
827 4.43943203075984e-05
828 4.4344680929953e-05
829 4.42952551574412e-05
830 4.42459781944867e-05
831 4.41969626354677e-05
832 4.41480795790502e-05
833 4.40993758203945e-05
834 4.4050811457789e-05
835 4.40023832689225e-05
836 4.39540868549434e-05
837 4.39059810630008e-05
838 4.38579713583446e-05
839 4.38100960410045e-05
840 4.37624177216378e-05
841 4.37148391008909e-05
842 4.36674131603828e-05
843 4.36201336118103e-05
844 4.35729828455559e-05
845 4.35259606709574e-05
846 4.34790650789552e-05
847 4.34323093821594e-05
848 4.33857198807175e-05
849 4.33391858886504e-05
850 4.32928452855919e-05
851 4.32466055446667e-05
852 4.32004831771119e-05
853 4.31545006710129e-05
854 4.31086287123369e-05
855 4.30628913736797e-05
856 4.30173329940923e-05
857 4.29718088502303e-05
858 4.29264092775365e-05
859 4.2881131444498e-05
860 4.28359194402835e-05
861 4.27908414467974e-05
862 4.27458451639495e-05
863 4.27009607732269e-05
864 4.26561222823807e-05
865 4.26114397260591e-05
866 4.25668338088769e-05
867 4.25223438581905e-05
868 4.24779076724964e-05
869 4.24336107016643e-05
870 4.23893509372183e-05
871 4.23452054908087e-05
872 4.23011562199112e-05
873 4.22572182105322e-05
874 4.22133748134712e-05
875 4.21696564796292e-05
876 4.2126002221939e-05
877 4.20824773712525e-05
878 4.20389942436354e-05
879 4.19956508519448e-05
880 4.19523268782029e-05
881 4.19091001078916e-05
882 4.18659986773188e-05
883 4.1822930753573e-05
884 4.17799391930392e-05
885 4.17370346301738e-05
886 4.16941914960963e-05
887 4.16514206798742e-05
888 4.16087311106613e-05
889 4.15660624977221e-05
890 4.15234333785245e-05
891 4.14809049390878e-05
892 4.14384487582945e-05
893 4.1396009700397e-05
894 4.13536554694597e-05
895 4.13113759651177e-05
896 4.1269093012275e-05
897 4.12269170008509e-05
898 4.11847443793079e-05
899 4.11426646061604e-05
900 4.11006073980739e-05
901 4.10586100644631e-05
902 4.10166574897156e-05
903 4.09747158898928e-05
904 4.09328151723306e-05
905 4.08909749924173e-05
906 4.08491841188408e-05
907 4.0807390824682e-05
908 4.07656408795513e-05
909 4.07239112446926e-05
910 4.06821872391087e-05
911 4.06405172318121e-05
912 4.05988773000132e-05
913 4.05572316649246e-05
914 4.05156201749672e-05
915 4.04740436182512e-05
916 4.0432501637729e-05
917 4.03909757107357e-05
918 4.03495300507946e-05
919 4.0308022406658e-05
920 4.02664624245356e-05
921 4.02249257576888e-05
922 4.01834754081657e-05
923 4.01420577773592e-05
924 4.01007389824552e-05
925 4.0059479032332e-05
926 4.0018280661395e-05
927 3.99771443871562e-05
928 3.99360511422013e-05
929 3.98950284369685e-05
930 3.98540414199289e-05
931 3.98131060664516e-05
932 3.97721492155038e-05
933 3.97312788950425e-05
934 3.96904576831503e-05
935 3.96497221212636e-05
936 3.96089924551054e-05
937 3.95682684845392e-05
938 3.95275858545337e-05
939 3.94867887122767e-05
940 3.94460370708079e-05
941 3.94053156108261e-05
942 3.93646402550019e-05
943 3.93240177161876e-05
944 3.92834208042814e-05
945 3.92428623611588e-05
946 3.92023392124704e-05
947 3.91618500425276e-05
948 3.91213806114621e-05
949 3.9081009916823e-05
950 3.9040631227986e-05
951 3.90002903631152e-05
952 3.89600320887714e-05
953 3.89198813888214e-05
954 3.88797669833944e-05
955 3.8839706951066e-05
956 3.8799654631679e-05
957 3.87597137431565e-05
958 3.8719815561213e-05
959 3.86799753731755e-05
960 3.86401990499034e-05
961 3.86004880012318e-05
962 3.85608281773623e-05
963 3.85212092422667e-05
964 3.84816697926264e-05
965 3.84421498083043e-05
966 3.84027291371325e-05
967 3.83634166182626e-05
968 3.83241693734012e-05
969 3.82850077228862e-05
970 3.82459187437216e-05
971 3.82069042679234e-05
972 3.81679808221236e-05
973 3.81291261764005e-05
974 3.8090359530211e-05
975 3.80516384244819e-05
976 3.8012997735611e-05
977 3.79744589427133e-05
978 3.79359145246857e-05
979 3.78974459642005e-05
980 3.78591563432451e-05
981 3.78210114485948e-05
982 3.77829856551178e-05
983 3.77450784257623e-05
984 3.77072714551711e-05
985 3.76695441814216e-05
986 3.76319209654715e-05
987 3.75943931102777e-05
988 3.75569464310388e-05
989 3.75196024761474e-05
990 3.74823080206236e-05
991 3.74450674485161e-05
992 3.74079107678208e-05
993 3.73708162803391e-05
994 3.73338472208218e-05
995 3.72969454674651e-05
996 3.7260117108436e-05
997 3.72233477611653e-05
998 3.71866289127591e-05
999 3.71500116168969e-05
1000 3.71134427853785e-05
1001 3.70769369890667e-05
1002 3.7040482302686e-05
1003 3.70040973140343e-05
1004 3.69677557845449e-05
1005 3.69314813729223e-05
1006 3.68952635406335e-05
1007 3.68591170284797e-05
1008 3.68230224854216e-05
1009 3.67869380865452e-05
1010 3.67508986241679e-05
1011 3.67149092141972e-05
1012 3.66789748505643e-05
1013 3.6643110381244e-05
1014 3.66072745083083e-05
1015 3.65714072051067e-05
1016 3.65356405899557e-05
1017 3.64999296470122e-05
1018 3.64642615006498e-05
1019 3.64286275858679e-05
1020 3.63930754113303e-05
1021 3.63575485593515e-05
1022 3.63220740850304e-05
1023 3.62866325017327e-05
1024 3.62512081218587e-05
1025 3.621585049333e-05
1026 3.61805479075864e-05
1027 3.61452888171243e-05
1028 3.61100836858697e-05
1029 3.60748733131781e-05
1030 3.60396881286344e-05
1031 3.60045532961095e-05
1032 3.59694421282105e-05
1033 3.59343526733132e-05
1034 3.58993260298016e-05
1035 3.58643158045642e-05
1036 3.58293759911514e-05
1037 3.57944373643448e-05
1038 3.57595064013339e-05
1039 3.57246164958506e-05
1040 3.56897756776201e-05
1041 3.56549627221388e-05
1042 3.5620168122937e-05
1043 3.55853852358479e-05
1044 3.55506583093283e-05
1045 3.55159531419957e-05
1046 3.54812556112213e-05
1047 3.54466139326585e-05
1048 3.54119592487952e-05
1049 3.53773152574159e-05
1050 3.53427303796394e-05
1051 3.53081152919543e-05
1052 3.52735109589256e-05
1053 3.52389450905359e-05
1054 3.52043951086974e-05
1055 3.51698542780573e-05
1056 3.51353570457273e-05
1057 3.51008524450691e-05
1058 3.50663303348616e-05
1059 3.50318326252837e-05
1060 3.49973382226902e-05
1061 3.49628668736557e-05
1062 3.492841428591e-05
1063 3.48939608188677e-05
1064 3.48594837695041e-05
1065 3.48250349097394e-05
1066 3.47905778011655e-05
1067 3.47560935806494e-05
1068 3.47216546394691e-05
1069 3.46872046339541e-05
1070 3.4652767032739e-05
1071 3.46182831657179e-05
1072 3.45838395764038e-05
1073 3.45494186711666e-05
1074 3.45149640577821e-05
1075 3.44805192392291e-05
1076 3.44460831875892e-05
1077 3.44116047147717e-05
1078 3.43771358076358e-05
1079 3.43426851833565e-05
1080 3.43081796193208e-05
1081 3.42736731523038e-05
1082 3.42391938132162e-05
1083 3.42046751959183e-05
1084 3.41701827076453e-05
1085 3.41356752316576e-05
1086 3.41011261439187e-05
1087 3.40665716015801e-05
1088 3.40320036874194e-05
1089 3.39974241914121e-05
1090 3.39628110512062e-05
1091 3.39282147976405e-05
1092 3.3893578307224e-05
1093 3.38589238252733e-05
1094 3.38242323110194e-05
1095 3.37895423350905e-05
1096 3.37548394545214e-05
1097 3.37200618751865e-05
1098 3.36853019096139e-05
1099 3.36505239329673e-05
1100 3.36157262044168e-05
1101 3.35808345569196e-05
1102 3.35460027229099e-05
1103 3.35111329832216e-05
1104 3.34761994071897e-05
1105 3.34412312312793e-05
1106 3.3406238524473e-05
1107 3.33712333748792e-05
1108 3.33361982208421e-05
1109 3.33011131553225e-05
1110 3.32659560040582e-05
1111 3.32308549377084e-05
1112 3.31956824606294e-05
1113 3.31604466836666e-05
1114 3.31252119979789e-05
1115 3.30899506935841e-05
1116 3.30545870692589e-05
1117 3.30192419220054e-05
1118 3.29838145205485e-05
1119 3.29483838470424e-05
1120 3.29129054872368e-05
1121 3.28773676496752e-05
1122 3.28418040040172e-05
1123 3.28062296037028e-05
1124 3.277054112516e-05
1125 3.27348362656465e-05
1126 3.26991151560208e-05
1127 3.26633379306903e-05
1128 3.26274869249706e-05
1129 3.25916256927646e-05
1130 3.25556878273403e-05
1131 3.25197175901811e-05
1132 3.24836746633158e-05
1133 3.2447588435384e-05
1134 3.24114768286421e-05
1135 3.23753191562304e-05
1136 3.23390676690849e-05
1137 3.23028108197102e-05
1138 3.22665052312487e-05
1139 3.22301048546099e-05
1140 3.21936648468579e-05
1141 3.21571654258908e-05
1142 3.21205730825132e-05
1143 3.20839468172333e-05
1144 3.20473040217687e-05
1145 3.20105830837747e-05
1146 3.19738328974732e-05
1147 3.19370067215866e-05
1148 3.19001252056713e-05
1149 3.18631511966316e-05
1150 3.18261509040241e-05
1151 3.17890972324856e-05
1152 3.17519863569279e-05
1153 3.17148191678977e-05
1154 3.16776552870361e-05
1155 3.16403618268642e-05
1156 3.16030014365272e-05
1157 3.15656371997856e-05
1158 3.15281920103179e-05
1159 3.14906511006778e-05
1160 3.14530711958601e-05
1161 3.14154392550373e-05
1162 3.1377754783198e-05
1163 3.13400126602896e-05
1164 3.13022277911301e-05
1165 3.1264388593281e-05
1166 3.12264956825459e-05
1167 3.11885346248412e-05
1168 3.11505120335435e-05
1169 3.11124589931211e-05
1170 3.10743329018e-05
1171 3.10361296200767e-05
1172 3.09979133176578e-05
1173 3.09596078977847e-05
1174 3.0921230039264e-05
1175 3.08828391636003e-05
1176 3.08443233674183e-05
1177 3.0805796792895e-05
1178 3.07671600670763e-05
1179 3.07285078247806e-05
1180 3.06897631900208e-05
1181 3.06509813050582e-05
1182 3.06121332792249e-05
1183 3.0573176942994e-05
1184 3.05341911672012e-05
1185 3.04951400208845e-05
1186 3.04560236449684e-05
1187 3.04168114588776e-05
1188 3.03775660637958e-05
1189 3.03382566890775e-05
1190 3.0298863045175e-05
1191 3.02594060255975e-05
1192 3.02198946903568e-05
1193 3.01802975754365e-05
1194 3.01406535344976e-05
1195 3.01009257418864e-05
1196 3.00611131140253e-05
1197 3.00212426089056e-05
1198 2.99813287381786e-05
1199 2.99413252283405e-05
1200 2.99012604276783e-05
1201 2.98611060269849e-05
1202 2.98208998833852e-05
1203 2.97806059362434e-05
1204 2.97402433749501e-05
1205 2.96998176298284e-05
1206 2.96593053127718e-05
1207 2.96187588150569e-05
1208 2.9578082102096e-05
1209 2.95373842978582e-05
1210 2.94966081704947e-05
1211 2.94557739864606e-05
1212 2.94148503859522e-05
1213 2.93738985372253e-05
1214 2.93328717315688e-05
1215 2.92917341451944e-05
1216 2.92505983825464e-05
1217 2.92093478666544e-05
1218 2.91680374804315e-05
1219 2.91266555674241e-05
1220 2.90852128118265e-05
1221 2.90437051374918e-05
1222 2.90021081325402e-05
1223 2.89604331911164e-05
1224 2.89187491097446e-05
1225 2.88769694704409e-05
1226 2.88351460990081e-05
1227 2.87931857082905e-05
1228 2.87512284617245e-05
1229 2.87091236845782e-05
1230 2.8667024866517e-05
1231 2.86248143689013e-05
1232 2.85825606466024e-05
1233 2.85402394884689e-05
1234 2.84978682720075e-05
1235 2.8455379075254e-05
1236 2.84128804792042e-05
1237 2.83703059800189e-05
1238 2.83276638075591e-05
1239 2.82849650261596e-05
1240 2.82422130262934e-05
1241 2.81993917106149e-05
1242 2.81564756813661e-05
1243 2.81135406462833e-05
1244 2.80705199144175e-05
1245 2.80274668096325e-05
1246 2.79843532974648e-05
1247 2.79412000339837e-05
1248 2.78979636541739e-05
1249 2.78546900685702e-05
1250 2.78113483259309e-05
1251 2.77679522024954e-05
1252 2.77245119454742e-05
1253 2.76810372703551e-05
1254 2.76374687970815e-05
1255 2.75938796541197e-05
1256 2.75502337660309e-05
1257 2.75065135078023e-05
1258 2.74627661038807e-05
1259 2.74190040900161e-05
1260 2.7375177523344e-05
1261 2.73312503473733e-05
1262 2.72873617455834e-05
1263 2.72434170831559e-05
1264 2.71993988913977e-05
1265 2.71553598458022e-05
1266 2.71112674254207e-05
1267 2.70671695764927e-05
1268 2.70229866039529e-05
1269 2.69787902382745e-05
1270 2.6934550428237e-05
1271 2.68902662193445e-05
1272 2.68459375707408e-05
1273 2.68015787447951e-05
1274 2.67571936755454e-05
1275 2.67127516189921e-05
1276 2.66683286488008e-05
1277 2.66237926821826e-05
1278 2.65792806573018e-05
1279 2.65347396701069e-05
1280 2.64901107011421e-05
1281 2.64455374091218e-05
1282 2.64008969826553e-05
1283 2.63562560303872e-05
1284 2.63115628532281e-05
1285 2.6266870911229e-05
1286 2.62221784736264e-05
1287 2.61774966615983e-05
1288 2.61327541591546e-05
1289 2.60879965979418e-05
1290 2.60432788383724e-05
1291 2.5998498832891e-05
1292 2.59537161847826e-05
1293 2.59089161354827e-05
1294 2.58641076875676e-05
1295 2.58193298575312e-05
1296 2.57745197522752e-05
1297 2.5729681883746e-05
1298 2.56849634296212e-05
1299 2.56401803486739e-05
1300 2.55954124064317e-05
1301 2.55506440520747e-05
1302 2.55059143305184e-05
1303 2.54611522200558e-05
1304 2.54164021763851e-05
1305 2.5371651134994e-05
1306 2.53268815348188e-05
1307 2.52821558669088e-05
1308 2.52374404243009e-05
1309 2.51927551383109e-05
1310 2.51480509507e-05
1311 2.51033995575985e-05
1312 2.50587585046702e-05
1313 2.50141434516848e-05
1314 2.49695461652285e-05
1315 2.49250167652093e-05
1316 2.48804647148641e-05
1317 2.4835983298388e-05
1318 2.47915159530265e-05
1319 2.47471115060923e-05
1320 2.4702703270781e-05
1321 2.46583531335887e-05
1322 2.46140327675448e-05
1323 2.45697805946558e-05
1324 2.45255470385786e-05
1325 2.44813433181908e-05
1326 2.44372568338018e-05
1327 2.4393160365103e-05
1328 2.43491160164903e-05
1329 2.43051306429246e-05
1330 2.42612199201631e-05
1331 2.42173611445888e-05
1332 2.41735893542483e-05
1333 2.41298375609489e-05
1334 2.40862108308685e-05
1335 2.40426003544532e-05
1336 2.39990521162525e-05
1337 2.39555974600828e-05
1338 2.39122586075761e-05
1339 2.38689248194627e-05
1340 2.38256983641586e-05
1341 2.3782573009612e-05
1342 2.37394782889311e-05
1343 2.36965034865477e-05
1344 2.36536234231958e-05
1345 2.36107894290664e-05
1346 2.35680638039544e-05
1347 2.35254187034665e-05
1348 2.34828841006636e-05
1349 2.34404531601247e-05
1350 2.33981018382678e-05
1351 2.33558953848245e-05
1352 2.33137721258695e-05
1353 2.32717890735756e-05
1354 2.32299123042642e-05
1355 2.31881613874663e-05
1356 2.31464864407134e-05
1357 2.31049463289186e-05
1358 2.30635283019846e-05
1359 2.30222189655886e-05
1360 2.29810651071934e-05
1361 2.29401070945083e-05
1362 2.28993037989511e-05
1363 2.28586475810033e-05
1364 2.28181129712605e-05
1365 2.27777150989539e-05
1366 2.27374189568271e-05
1367 2.26972297780274e-05
1368 2.26572085904782e-05
1369 2.26172833806733e-05
1370 2.25774833341556e-05
1371 2.2537824744262e-05
1372 2.24982912335747e-05
1373 2.24589188917435e-05
1374 2.24196503483635e-05
1375 2.23805114843619e-05
1376 2.23415183722153e-05
1377 2.23026645240765e-05
1378 2.22639578358515e-05
1379 2.22254196913241e-05
1380 2.21869978146808e-05
1381 2.21487284477462e-05
1382 2.21105800264354e-05
1383 2.20725993222241e-05
1384 2.20347729597374e-05
1385 2.19970849132759e-05
1386 2.19595544888781e-05
1387 2.19221787472653e-05
1388 2.18849140492712e-05
1389 2.18478374354921e-05
1390 2.18109236194882e-05
1391 2.17741385248142e-05
1392 2.17375379681547e-05
1393 2.1701062249709e-05
1394 2.16647587703752e-05
1395 2.16285907566061e-05
1396 2.15926272145358e-05
1397 2.15567871363713e-05
1398 2.15211035499152e-05
1399 2.14855887315698e-05
1400 2.14502463080635e-05
1401 2.14150728954365e-05
1402 2.13800280833472e-05
1403 2.13451800637661e-05
1404 2.13105061845908e-05
1405 2.12759696426682e-05
1406 2.12416445008283e-05
1407 2.12074851978864e-05
1408 2.11735112571887e-05
1409 2.11396945495324e-05
1410 2.110606147158e-05
1411 2.10726033215517e-05
1412 2.10392729738847e-05
1413 2.10061452759665e-05
1414 2.09731907787614e-05
1415 2.09403698381294e-05
1416 2.0907774489487e-05
1417 2.08753015993324e-05
1418 2.08430044003431e-05
1419 2.08109098842613e-05
1420 2.07789550887109e-05
1421 2.07471801084343e-05
1422 2.07155691628695e-05
1423 2.06841113493302e-05
1424 2.06528125753872e-05
1425 2.06216841061035e-05
1426 2.05907397135737e-05
1427 2.05599371791247e-05
1428 2.05293289307444e-05
1429 2.04988625889987e-05
1430 2.0468546569082e-05
1431 2.04384301743706e-05
1432 2.04084884665908e-05
1433 2.0378716280689e-05
1434 2.03491417867241e-05
1435 2.03197187686281e-05
1436 2.029049640484e-05
1437 2.02614665951728e-05
1438 2.02325889138895e-05
1439 2.02038699441687e-05
1440 2.01753747373819e-05
1441 2.01470189337509e-05
1442 2.01188766026187e-05
1443 2.00909210139599e-05
1444 2.00631040894924e-05
1445 2.00354735415686e-05
1446 2.00080127659893e-05
1447 1.99807621908595e-05
1448 1.99536546017972e-05
1449 1.99267315371306e-05
1450 1.98999454698405e-05
1451 1.98733997371174e-05
1452 1.98470077800531e-05
1453 1.98207705395248e-05
1454 1.97947194440218e-05
1455 1.97688199137976e-05
1456 1.97431545820119e-05
1457 1.97176147788506e-05
1458 1.96922531845445e-05
1459 1.96670668242878e-05
1460 1.96420590121704e-05
1461 1.96171997099981e-05
1462 1.95925314798728e-05
1463 1.95680338516506e-05
1464 1.95436951588116e-05
1465 1.95194731726171e-05
1466 1.94954654345973e-05
1467 1.94716359030641e-05
1468 1.94479561083464e-05
1469 1.94244714789941e-05
1470 1.94011485194305e-05
1471 1.93779977137136e-05
1472 1.93550274764457e-05
1473 1.93321783511872e-05
1474 1.93095332707098e-05
1475 1.92870590473386e-05
1476 1.92647388015388e-05
1477 1.92425948757347e-05
1478 1.92205984556419e-05
1479 1.91987895619879e-05
1480 1.91771165318026e-05
1481 1.91556443377057e-05
1482 1.91343021883246e-05
1483 1.91131369907301e-05
1484 1.90921131183093e-05
1485 1.9071319362259e-05
1486 1.90506272736239e-05
1487 1.90301166718143e-05
1488 1.90097775429147e-05
1489 1.89896233333542e-05
1490 1.89696226045773e-05
1491 1.89497544749173e-05
1492 1.89300823016936e-05
1493 1.8910568574763e-05
1494 1.88911891256064e-05
1495 1.88720086183499e-05
1496 1.88529451025469e-05
1497 1.88340778422003e-05
1498 1.88153609113802e-05
1499 1.87967921938537e-05
1500 1.87783648997926e-05
1501 1.87601236352937e-05
1502 1.87420262651727e-05
1503 1.87240630200591e-05
1504 1.87062873262533e-05
1505 1.86886431237099e-05
1506 1.86711431577891e-05
1507 1.8653828796289e-05
1508 1.86366349617847e-05
1509 1.86195996475883e-05
1510 1.8602741387023e-05
1511 1.85859928076108e-05
1512 1.8569418147744e-05
1513 1.85529654691171e-05
1514 1.85366730551806e-05
1515 1.85205405636898e-05
1516 1.85045116859091e-05
1517 1.84886419004234e-05
1518 1.84729627482246e-05
1519 1.84573639252505e-05
1520 1.84419692175197e-05
1521 1.84266817626622e-05
1522 1.84115480420151e-05
1523 1.83965696471944e-05
1524 1.83817303505881e-05
1525 1.83670621141848e-05
1526 1.83524655052262e-05
1527 1.83380504346076e-05
1528 1.83237629602928e-05
1529 1.83096488791297e-05
1530 1.82956358904344e-05
1531 1.82817922145991e-05
1532 1.82680775395738e-05
1533 1.82545321768165e-05
1534 1.82410927115721e-05
1535 1.82277860574231e-05
1536 1.82146408782558e-05
1537 1.82016416309475e-05
1538 1.81887545694555e-05
1539 1.81760390306115e-05
1540 1.81635033585555e-05
1541 1.81510304730163e-05
1542 1.81387675806851e-05
1543 1.81266282410775e-05
1544 1.8114624433944e-05
1545 1.81027640978234e-05
1546 1.80910391686477e-05
1547 1.80795156700479e-05
1548 1.80681018679962e-05
1549 1.80568581503356e-05
1550 1.80457375140719e-05
1551 1.8034817700278e-05
1552 1.80240251973866e-05
1553 1.80133712094725e-05
1554 1.80028377787522e-05
1555 1.79925496202789e-05
1556 1.79823758396462e-05
1557 1.79723696197935e-05
1558 1.79625108049919e-05
1559 1.79528174228949e-05
1560 1.79432919858632e-05
1561 1.79339337211815e-05
1562 1.79247841156638e-05
1563 1.79157433960834e-05
1564 1.79068965178904e-05
1565 1.78982325476085e-05
1566 1.78897431197811e-05
1567 1.78814329636623e-05
1568 1.78732895944241e-05
1569 1.78653352757389e-05
1570 1.7857566728452e-05
1571 1.78499604744549e-05
1572 1.78425728876164e-05
1573 1.78353695818127e-05
1574 1.78283558677587e-05
1575 1.78215770579489e-05
1576 1.78149470535516e-05
1577 1.78085346416168e-05
1578 1.78023291503848e-05
1579 1.77963804712059e-05
1580 1.77905656488046e-05
1581 1.77849785446919e-05
1582 1.77796433978491e-05
1583 1.77745074415962e-05
1584 1.77695917014849e-05
1585 1.77649077232426e-05
1586 1.77604145606599e-05
1587 1.77561749872268e-05
1588 1.77521478169259e-05
1589 1.77483162643692e-05
1590 1.77447638662898e-05
1591 1.77413849412981e-05
1592 1.77382427117105e-05
1593 1.77353527650581e-05
1594 1.77326567255183e-05
1595 1.77301939938701e-05
1596 1.7727937841272e-05
1597 1.7725865522659e-05
1598 1.7724059867182e-05
1599 1.77224351046353e-05
1600 1.77209934590176e-05
1601 1.77197841916647e-05
1602 1.77186954637421e-05
1603 1.77177979262903e-05
1604 1.77170465924803e-05
1605 1.7716454330241e-05
1606 1.77159484551955e-05
1607 1.77155656745735e-05
1608 1.7715167554518e-05
1609 1.77148516081379e-05
1610 1.77145040686592e-05
1611 1.77141337604307e-05
1612 1.77136434806139e-05
1613 1.7712979282436e-05
1614 1.77120912792835e-05
1615 1.77108785859303e-05
1616 1.77092154244699e-05
1617 1.77071310781448e-05
1618 1.77043932669818e-05
1619 1.77009840462243e-05
1620 1.76967123757995e-05
1621 1.76915878900748e-05
1622 1.76855079931926e-05
1623 1.7678351023657e-05
1624 1.7670105072772e-05
1625 1.76607014985564e-05
1626 1.7650109022919e-05
1627 1.76383758881589e-05
1628 1.76254603883782e-05
1629 1.76114551605859e-05
1630 1.75963972897838e-05
1631 1.75804244850752e-05
1632 1.75635505298051e-05
1633 1.75460641802516e-05
1634 1.75279002127364e-05
1635 1.75092372979672e-05
1636 1.74902283127665e-05
1637 1.74709619861583e-05
1638 1.74515163493538e-05
1639 1.74319976800064e-05
1640 1.74125619896159e-05
1641 1.73931192062327e-05
1642 1.73738504537132e-05
1643 1.7354767169427e-05
1644 1.73359204579763e-05
1645 1.73172899078367e-05
1646 1.7298948860045e-05
1647 1.7280844331206e-05
1648 1.72630366310026e-05
1649 1.72455140479144e-05
1650 1.72282794383231e-05
1651 1.72113197329793e-05
1652 1.71946226874553e-05
1653 1.71781895836887e-05
1654 1.71620326015661e-05
1655 1.71461393758676e-05
1656 1.71304673089641e-05
1657 1.71150476742099e-05
1658 1.70998729220884e-05
1659 1.70848670038026e-05
1660 1.70701180491524e-05
1661 1.70555421720546e-05
1662 1.70411496493254e-05
1663 1.70269524974268e-05
1664 1.7012960687642e-05
1665 1.69990690264873e-05
1666 1.69854209281321e-05
1667 1.69718665183893e-05
1668 1.69584549608951e-05
1669 1.69451895821737e-05
1670 1.69320705018331e-05
1671 1.69190571005136e-05
1672 1.69061669090809e-05
1673 1.68933947476783e-05
1674 1.68807461236042e-05
1675 1.68682046576644e-05
1676 1.68557361538054e-05
1677 1.68434149519688e-05
1678 1.6831183856425e-05
1679 1.68190484698035e-05
1680 1.68070060387511e-05
1681 1.67950575000001e-05
1682 1.67831748670485e-05
1683 1.67714491562757e-05
1684 1.67597473446079e-05
1685 1.67481882099215e-05
1686 1.67366803929999e-05
1687 1.67252558191204e-05
1688 1.671393021437e-05
1689 1.67026730230428e-05
1690 1.66914594942706e-05
1691 1.66803448318949e-05
1692 1.66693229853484e-05
1693 1.66583718600085e-05
1694 1.66474880461583e-05
1695 1.66366681400021e-05
1696 1.66259027848762e-05
1697 1.66151864447646e-05
1698 1.66045947607903e-05
1699 1.65939967843679e-05
1700 1.65834962470394e-05
1701 1.65730703483765e-05
1702 1.65626669206276e-05
1703 1.65523915282388e-05
1704 1.65421224200083e-05
1705 1.65319108355414e-05
1706 1.65217804557471e-05
1707 1.65117090036304e-05
1708 1.65016825863044e-05
1709 1.64917574216143e-05
1710 1.64818689999772e-05
1711 1.64720176997572e-05
1712 1.64622162482549e-05
1713 1.64524955981922e-05
1714 1.64427885784117e-05
1715 1.64331715630794e-05
1716 1.64235826911607e-05
1717 1.64140532872281e-05
1718 1.64045709807326e-05
1719 1.63951395976506e-05
1720 1.63857436428216e-05
1721 1.63763837202069e-05
1722 1.63671145815651e-05
1723 1.63578786995799e-05
1724 1.63486831666522e-05
1725 1.63395474392184e-05
1726 1.63304583458886e-05
1727 1.63214060574883e-05
1728 1.6312422695286e-05
1729 1.63034574232057e-05
1730 1.62945731772742e-05
1731 1.62857265344485e-05
1732 1.62768839876056e-05
1733 1.62681162117669e-05
1734 1.62594309148408e-05
1735 1.62507198918268e-05
1736 1.62420623549172e-05
1737 1.62334945696211e-05
1738 1.62249249280322e-05
1739 1.62164496293471e-05
1740 1.62079822881485e-05
1741 1.61995493277445e-05
1742 1.61911669458448e-05
1743 1.61828174866467e-05
1744 1.61744654241976e-05
1745 1.61661867806491e-05
1746 1.61578879508776e-05
1747 1.61495945277442e-05
1748 1.61413385066922e-05
1749 1.61330990628983e-05
1750 1.61248686675701e-05
1751 1.61166219783127e-05
1752 1.6108418770718e-05
1753 1.61001568116011e-05
1754 1.6091959445852e-05
1755 1.60837266998115e-05
1756 1.60755056202907e-05
1757 1.60672541671403e-05
1758 1.60590230870265e-05
1759 1.60507771305163e-05
1760 1.60425401955303e-05
1761 1.60342665071293e-05
1762 1.60260049882529e-05
1763 1.60176475638707e-05
1764 1.60093176152879e-05
1765 1.60009454193144e-05
1766 1.59925472849783e-05
1767 1.59841596361806e-05
1768 1.59757413467077e-05
1769 1.59672780976431e-05
1770 1.59588002318263e-05
1771 1.59502654234108e-05
1772 1.59417440877737e-05
1773 1.59331860490515e-05
1774 1.59245978847977e-05
1775 1.59159850792179e-05
1776 1.59073440946974e-05
1777 1.58986580528856e-05
1778 1.58899887153415e-05
1779 1.58812661374223e-05
1780 1.58725313964917e-05
1781 1.58637541967839e-05
1782 1.58549717763634e-05
1783 1.58461921907123e-05
1784 1.58373424164656e-05
1785 1.58285296413633e-05
1786 1.58196439352665e-05
1787 1.58108036846609e-05
1788 1.58019512230823e-05
1789 1.57930553041178e-05
1790 1.57841426332202e-05
1791 1.57752287840059e-05
1792 1.57663203790293e-05
1793 1.57574212155491e-05
1794 1.57485255263268e-05
1795 1.57396055664449e-05
1796 1.57307066279699e-05
1797 1.57218004597226e-05
1798 1.57128963736615e-05
1799 1.57039971423837e-05
1800 1.56951492492989e-05
1801 1.56862668762396e-05
1802 1.56774084727473e-05
1803 1.56685736130887e-05
1804 1.56597279007779e-05
1805 1.56509277703757e-05
1806 1.5642138549173e-05
1807 1.56333251487932e-05
1808 1.56245232571332e-05
1809 1.56157573115342e-05
1810 1.56070114577152e-05
1811 1.55982940232372e-05
1812 1.55895917712812e-05
1813 1.55809017157911e-05
1814 1.55722406542935e-05
1815 1.55636121735488e-05
1816 1.55550072224268e-05
1817 1.55464087014205e-05
1818 1.55378554923165e-05
1819 1.55293263303472e-05
1820 1.55207891353963e-05
1821 1.55123192981203e-05
1822 1.5503854034075e-05
1823 1.54954185231186e-05
1824 1.54870315144014e-05
1825 1.54786202329523e-05
1826 1.54702664814839e-05
1827 1.54619427119584e-05
1828 1.5453614173803e-05
1829 1.54453607098167e-05
1830 1.54371003565264e-05
1831 1.54288671512977e-05
1832 1.5420653987223e-05
1833 1.54124651962443e-05
1834 1.54042933884213e-05
1835 1.539616223963e-05
1836 1.53880937624843e-05
1837 1.53799862916415e-05
1838 1.53718993075221e-05
1839 1.53638581584422e-05
1840 1.53558698950581e-05
1841 1.53478810286008e-05
1842 1.53399243103536e-05
1843 1.53319879882356e-05
1844 1.53240563672459e-05
1845 1.531617779064e-05
1846 1.530828879955e-05
1847 1.53004136797345e-05
1848 1.52926034724388e-05
1849 1.52847807894929e-05
1850 1.52769495977978e-05
1851 1.52691687442067e-05
1852 1.52613931119205e-05
1853 1.52536444009144e-05
1854 1.52459384257971e-05
1855 1.52381901248333e-05
1856 1.52305387191021e-05
1857 1.52228473246178e-05
1858 1.52151971013481e-05
1859 1.52075384522258e-05
1860 1.51999120866127e-05
1861 1.51922840861592e-05
1862 1.51846968077981e-05
1863 1.51770967287135e-05
1864 1.51695414096764e-05
1865 1.51619492572882e-05
1866 1.51544051130159e-05
1867 1.51468923042704e-05
1868 1.5139334788176e-05
1869 1.51318086866562e-05
1870 1.51243177410147e-05
1871 1.5116785198425e-05
1872 1.51092685462354e-05
1873 1.51018002180573e-05
1874 1.50943252975229e-05
1875 1.50868573968547e-05
1876 1.50793967611899e-05
1877 1.50719563671113e-05
1878 1.50645061260946e-05
1879 1.50570582449679e-05
1880 1.50496213660034e-05
1881 1.50421856730532e-05
1882 1.50347759007019e-05
1883 1.5027357856745e-05
1884 1.50199352478471e-05
1885 1.50125517220564e-05
1886 1.50051596066921e-05
1887 1.49977813865831e-05
1888 1.49903596445474e-05
1889 1.49830084102594e-05
1890 1.49756276327887e-05
1891 1.49682495450184e-05
1892 1.49608760861379e-05
1893 1.49535226539044e-05
1894 1.49461829395866e-05
1895 1.49388491748799e-05
1896 1.49314749595675e-05
1897 1.49241237187662e-05
1898 1.49168012300767e-05
1899 1.49094957535567e-05
1900 1.49021543913719e-05
1901 1.48948211619408e-05
1902 1.48875497805179e-05
1903 1.48802150414795e-05
1904 1.48729103607674e-05
1905 1.48656110268893e-05
1906 1.48583119603529e-05
1907 1.4851049569368e-05
1908 1.48437579374772e-05
1909 1.48364585224788e-05
1910 1.48291902659636e-05
1911 1.48219375528669e-05
1912 1.48146666590539e-05
1913 1.48074090322581e-05
1914 1.48001505788642e-05
1915 1.47929071057836e-05
1916 1.47856667516895e-05
1917 1.47784489451889e-05
1918 1.47711996252298e-05
1919 1.47639750913697e-05
1920 1.4756761212098e-05
1921 1.47495521482336e-05
1922 1.47423439799491e-05
1923 1.47351109924069e-05
1924 1.47279354057635e-05
1925 1.47207290484748e-05
1926 1.47135188616569e-05
1927 1.47063598561203e-05
1928 1.46991817571163e-05
1929 1.46920082440734e-05
1930 1.46848421180105e-05
1931 1.46776812224303e-05
1932 1.46705136631435e-05
1933 1.46633534576779e-05
1934 1.46562218980388e-05
1935 1.46490847810673e-05
1936 1.46419638949643e-05
1937 1.46348273819541e-05
1938 1.46276922616077e-05
1939 1.46205956162622e-05
1940 1.46134732409801e-05
1941 1.46063918193088e-05
1942 1.45992762214201e-05
1943 1.45921882710454e-05
1944 1.45851138097323e-05
1945 1.45780109255256e-05
1946 1.45709510854057e-05
1947 1.45638566468875e-05
1948 1.45568067629516e-05
1949 1.45497669340836e-05
1950 1.45426882320147e-05
1951 1.45356657670348e-05
1952 1.45286461960363e-05
1953 1.45215818374567e-05
1954 1.45145707757995e-05
1955 1.45075673569176e-05
1956 1.45005785719595e-05
1957 1.44935600712159e-05
1958 1.448656270527e-05
1959 1.44795797973885e-05
1960 1.44725915367516e-05
1961 1.44656528557145e-05
1962 1.44587088917921e-05
1963 1.44517752701153e-05
1964 1.44448265200955e-05
1965 1.44378994457739e-05
1966 1.44309628383373e-05
1967 1.44240623107805e-05
1968 1.44171549676386e-05
1969 1.44102646603509e-05
1970 1.4403385054429e-05
1971 1.43965315955915e-05
1972 1.43896712865299e-05
1973 1.43828062354838e-05
1974 1.43759515536246e-05
1975 1.43691086765448e-05
1976 1.43622984225544e-05
1977 1.43554769754436e-05
1978 1.43486716777848e-05
1979 1.43418274148506e-05
1980 1.43350465542369e-05
1981 1.43282462901813e-05
1982 1.43215075463961e-05
1983 1.43147096644351e-05
1984 1.43079250678468e-05
1985 1.43011699919408e-05
1986 1.42944210708151e-05
1987 1.42876947760347e-05
1988 1.42809531308667e-05
1989 1.42742477589053e-05
1990 1.42675128144513e-05
1991 1.42608390193052e-05
1992 1.42541592180005e-05
1993 1.42474558174991e-05
1994 1.42407754643396e-05
1995 1.42341088720244e-05
1996 1.42274494428681e-05
1997 1.42207960888522e-05
1998 1.42141419120871e-05
1999 1.42075255353073e-05
};
\addlegendentry{Train}
\addplot [semithick, black]
table {%
0 0.029663871973753
1 0.0272275544703007
2 0.0243323985487223
3 0.0203068852424622
4 0.0158795863389969
5 0.012230983003974
6 0.00945578794926405
7 0.00737061630934477
8 0.00581964943557978
9 0.00468314625322819
10 0.00385244190692902
11 0.00324030825868249
12 0.00278383074328303
13 0.00243797921575606
14 0.00217094481922686
15 0.00196010526269674
16 0.00178942526690662
17 0.00164722767658532
18 0.00152513943612576
19 0.00141803489532322
20 0.00132278981618583
21 0.00123703083954751
22 0.0011587671469897
23 0.0010869795223698
24 0.00102105119731277
25 0.000960248580668122
26 0.000903958396520466
27 0.0008518235408701
28 0.000803589296992868
29 0.000759122369345278
30 0.000718160474207252
31 0.000680369965266436
32 0.000645451596938074
33 0.000613161828368902
34 0.000583304965402931
35 0.000555709411855787
36 0.000530232035089284
37 0.000506790936924517
38 0.000485262862639502
39 0.000465491932118312
40 0.000447337864898145
41 0.000430680462159216
42 0.000415398651966825
43 0.000401388038881123
44 0.000388552783988416
45 0.000376807583961636
46 0.000366063817637041
47 0.000356233125785366
48 0.000347231049090624
49 0.000338977028150111
50 0.000331403978634626
51 0.000324448716128245
52 0.000318052305374295
53 0.000312161224428564
54 0.000306727888528258
55 0.000301708147162572
56 0.000297060469165444
57 0.000292747572530061
58 0.000288736715447158
59 0.000284998677670956
60 0.000281505897874013
61 0.000278234860161319
62 0.000275162281468511
63 0.000272270262939855
64 0.00026954262284562
65 0.00026696443092078
66 0.000264521804638207
67 0.000262202433077618
68 0.00025999647914432
69 0.00025789364008233
70 0.000255884719081223
71 0.000253961858106777
72 0.00025211795582436
73 0.000250346638495103
74 0.000248642929363996
75 0.000247001473326236
76 0.000245417759288102
77 0.000243887247052044
78 0.000242406444158405
79 0.000240971974562854
80 0.000239580520428717
81 0.000238229622482322
82 0.000236916093854234
83 0.000235637489822693
84 0.000234391758567654
85 0.000233176106121391
86 0.000231988378800452
87 0.000230827223276719
88 0.000229690616833977
89 0.000228576609515585
90 0.000227483789785765
91 0.000226410702452995
92 0.000225355834118091
93 0.00022431806428358
94 0.000223297756747343
95 0.00022229271417018
96 0.000221301626879722
97 0.000220323447138071
98 0.000219357360037975
99 0.000218402390601113
100 0.000217457854887471
101 0.000216522632399574
102 0.000215595558984205
103 0.000214676460018381
104 0.000213764462387189
105 0.000212859260500409
106 0.000211960257729515
107 0.000211066784686409
108 0.000210178390261717
109 0.000209294637897983
110 0.000208415003726259
111 0.000207539211260155
112 0.000206666838494129
113 0.000205797623493709
114 0.000204930940526538
115 0.000204066789592616
116 0.000203204952413216
117 0.000202344992430881
118 0.000201486836886033
119 0.000200630238396116
120 0.000199774905922823
121 0.000198920999537222
122 0.000198068475583568
123 0.000197217130335048
124 0.000196366978343576
125 0.000195518077816814
126 0.000194670501514338
127 0.000193824322195724
128 0.000192979510757141
129 0.000192136081750505
130 0.000191294166143052
131 0.000190454215044156
132 0.000189616461284459
133 0.000188780977623537
134 0.000187947793165222
135 0.000187117359018885
136 0.000186289849807508
137 0.000185465498361737
138 0.000184644537512213
139 0.000183827301952988
140 0.000183014024514705
141 0.000182205272722058
142 0.000181401497684419
143 0.000180603019543923
144 0.000179809954715893
145 0.000179022666998208
146 0.00017824127280619
147 0.000177465815795586
148 0.000176696295966394
149 0.000175932858837768
150 0.000175174878677353
151 0.000174423475982621
152 0.000173679203726351
153 0.00017294219287578
154 0.000172212632605806
155 0.000171491148648784
156 0.000170777857420035
157 0.000170072642504238
158 0.00016937538748607
159 0.00016868591774255
160 0.000168004116858356
161 0.000167329679243267
162 0.000166662852279842
163 0.000166003606864251
164 0.000165351695613936
165 0.000164707031217404
166 0.000164069337188266
167 0.000163438700838014
168 0.000162814525538124
169 0.000162196811288595
170 0.000161585165187716
171 0.000160978539497592
172 0.000160377778229304
173 0.00015978267765604
174 0.000159193165018223
175 0.000158609342179261
176 0.000158030918100849
177 0.000157457703608088
178 0.000156889509526081
179 0.000156326088472269
180 0.000155767382238992
181 0.00015521320165135
182 0.000154663153807633
183 0.000154117398778908
184 0.000153575630974956
185 0.0001530377776362
186 0.000152503707795404
187 0.000151973377796821
188 0.000151446773088537
189 0.000150923704495654
190 0.000150404215673916
191 0.000149888248415664
192 0.000149375540786423
193 0.000148866194649599
194 0.000148360311868601
195 0.000147857746924274
196 0.000147358543472365
197 0.000146862628753297
198 0.000146370104630478
199 0.000145880796480924
200 0.000145394835271873
201 0.000144912148243748
202 0.000144432691740803
203 0.00014395649486687
204 0.000143483513966203
205 0.000143013763590716
206 0.000142547258292325
207 0.000142083867103793
208 0.000141623764648102
209 0.000141166849061847
210 0.000140713134896941
211 0.000140262505738065
212 0.000139815019792877
213 0.000139370647957548
214 0.000138929302920587
215 0.000138490970130078
216 0.000138055562274531
217 0.000137623108457774
218 0.000137193535920233
219 0.000136766786454245
220 0.000136342758196406
221 0.000135921451146714
222 0.00013550280709751
223 0.000135086753289215
224 0.000134673231514171
225 0.000134262183564715
226 0.000133853594888933
227 0.000133447392727248
228 0.000133043547975831
229 0.000132642002427019
230 0.000132242770632729
231 0.000131845765281469
232 0.000131450986373238
233 0.000131058375700377
234 0.00013066797691863
235 0.000130279848235659
236 0.000129893975099549
237 0.000129510313854553
238 0.000129128995467909
239 0.000128749990835786
240 0.000128373212646693
241 0.000127998762764037
242 0.000127626612083986
243 0.000127256746054627
244 0.000126889150124043
245 0.000126523766084574
246 0.000126160637591965
247 0.000125799953821115
248 0.000125441612908617
249 0.000125085498439148
250 0.000124731595860794
251 0.000124379686894827
252 0.000124030004371889
253 0.000123682402772829
254 0.000123336896649562
255 0.000122993500554003
256 0.000122652214486152
257 0.000122312863823026
258 0.000121975477668457
259 0.000121640077850316
260 0.000121306722576264
261 0.000120975382742472
262 0.000120645956485532
263 0.000120318531116936
264 0.000119993070256896
265 0.000119669515697751
266 0.000119347918371204
267 0.000119028154585976
268 0.000118710267997812
269 0.000118394280434586
270 0.000118080133688636
271 0.000117767747724429
272 0.000117457115266006
273 0.000117148054414429
274 0.00011684067430906
275 0.00011653482215479
276 0.000116230505227577
277 0.000115927556180395
278 0.000115625967737287
279 0.000115325703518465
280 0.000115026647108607
281 0.000114728703920264
282 0.000114431699330453
283 0.000114135662443005
284 0.000113840374979191
285 0.000113545764179435
286 0.000113251793663949
287 0.00011295825970592
288 0.000112665096821729
289 0.000112372115836479
290 0.000112079193058889
291 0.000111786321213003
292 0.000111493303847965
293 0.00011120000272058
294 0.000110906519694254
295 0.000110612745629624
296 0.000110318542283494
297 0.00011002407700289
298 0.000109729866380803
299 0.000109436026832554
300 0.000109142762084957
301 0.000108850297692697
302 0.000108558786450885
303 0.000108268519397825
304 0.000107979678432457
305 0.000107692430901807
306 0.000107407351606525
307 0.000107124200440012
308 0.000106843173853122
309 0.000106564235466067
310 0.000106287232483737
311 0.000106012310425285
312 0.000105739556602202
313 0.00010546908742981
314 0.000105200895632152
315 0.000104935010313056
316 0.000104671511508059
317 0.000104410479252692
318 0.000104151789855678
319 0.000103895443317015
320 0.000103641330497339
321 0.00010338948777644
322 0.000103139835118782
323 0.000102892416180111
324 0.000102647340099793
325 0.000102404417702928
326 0.000102163779956754
327 0.000101925259514246
328 0.00010168892913498
329 0.000101454832474701
330 0.000101223056844901
331 0.000100993333035149
332 0.000100765690149274
333 0.000100540120911319
334 0.000100316538009793
335 0.00010009492689278
336 9.98752730083652e-05
337 9.96575399767607e-05
338 9.94416768662632e-05
339 9.92277928162366e-05
340 9.90156913758256e-05
341 9.8805503512267e-05
342 9.85970400506631e-05
343 9.8390482889954e-05
344 9.81857447186485e-05
345 9.79828109848313e-05
346 9.77816744125448e-05
347 9.75822622422129e-05
348 9.73845963017084e-05
349 9.71898116404191e-05
350 9.69975153566338e-05
351 9.68070453382097e-05
352 9.66182560659945e-05
353 9.64311184361577e-05
354 9.62456688284874e-05
355 9.60618199314922e-05
356 9.5879688160494e-05
357 9.5699040684849e-05
358 9.55199138843454e-05
359 9.53424241743051e-05
360 9.51664333115332e-05
361 9.4992239610292e-05
362 9.48200904531404e-05
363 9.46499203564599e-05
364 9.44816711125895e-05
365 9.43153427215293e-05
366 9.41509133554064e-05
367 9.39883611863479e-05
368 9.38276207307354e-05
369 9.36686628847383e-05
370 9.35114876483567e-05
371 9.33560513658449e-05
372 9.3202390416991e-05
373 9.30503811105154e-05
374 9.29000234464183e-05
375 9.27512664929964e-05
376 9.26038046600297e-05
377 9.24578125705011e-05
378 9.23133411561139e-05
379 9.217035403708e-05
380 9.20288657653145e-05
381 9.18888254091144e-05
382 9.17501602089033e-05
383 9.16129429242574e-05
384 9.14770425879396e-05
385 9.13424737518653e-05
386 9.12092218641192e-05
387 9.10772650968283e-05
388 9.09466471057385e-05
389 9.08171787159517e-05
390 9.06889763427898e-05
391 9.05619890545495e-05
392 9.04361513676122e-05
393 9.03114778338932e-05
394 9.01877065189183e-05
395 9.00651939446107e-05
396 8.9943794591818e-05
397 8.98234357009642e-05
398 8.97042045835406e-05
399 8.95860866876319e-05
400 8.94689583219588e-05
401 8.93528995220549e-05
402 8.92378666321747e-05
403 8.912382327253e-05
404 8.90108203748241e-05
405 8.88988142833114e-05
406 8.87878049979918e-05
407 8.86776906554587e-05
408 8.85685585672036e-05
409 8.84604378370568e-05
410 8.8353204773739e-05
411 8.8246822997462e-05
412 8.81414453033358e-05
413 8.80368606885895e-05
414 8.79332073964179e-05
415 8.78303908393718e-05
416 8.77283455338329e-05
417 8.76272097229958e-05
418 8.75268888194114e-05
419 8.74273464432918e-05
420 8.73286699061282e-05
421 8.72308883117512e-05
422 8.71338634169661e-05
423 8.7037704361137e-05
424 8.69423020048998e-05
425 8.68476636242121e-05
426 8.67537673912011e-05
427 8.66606060299091e-05
428 8.65681940922514e-05
429 8.64764951984398e-05
430 8.63854875206016e-05
431 8.62951637827791e-05
432 8.62055239849724e-05
433 8.61166045069695e-05
434 8.60282889334485e-05
435 8.59407591633499e-05
436 8.58538114698604e-05
437 8.57675913721323e-05
438 8.56819387990981e-05
439 8.55970065458678e-05
440 8.55126709211618e-05
441 8.54290628922172e-05
442 8.53460587677546e-05
443 8.5263600340113e-05
444 8.51817749207839e-05
445 8.51005534059368e-05
446 8.50198848638684e-05
447 8.49398493301123e-05
448 8.48603740450926e-05
449 8.47815172164701e-05
450 8.47032351884991e-05
451 8.46254988573492e-05
452 8.45484028104693e-05
453 8.44718306325376e-05
454 8.4395804151427e-05
455 8.43203088152222e-05
456 8.42453810037114e-05
457 8.41709406813607e-05
458 8.40970096760429e-05
459 8.40236389194615e-05
460 8.39507265482098e-05
461 8.38782652863301e-05
462 8.38063569972292e-05
463 8.37348343338817e-05
464 8.36637336760759e-05
465 8.35931423353031e-05
466 8.35229075164534e-05
467 8.34530874271877e-05
468 8.33836456877179e-05
469 8.33146259537898e-05
470 8.32458899822086e-05
471 8.317763422383e-05
472 8.31096258480102e-05
473 8.30419594421983e-05
474 8.29746713861823e-05
475 8.29076088848524e-05
476 8.28408738016151e-05
477 8.27744006528519e-05
478 8.27082112664357e-05
479 8.26422838144936e-05
480 8.25766255729832e-05
481 8.25112292659469e-05
482 8.24460221338086e-05
483 8.23810769361444e-05
484 8.23163645691238e-05
485 8.2251790445298e-05
486 8.21875291876495e-05
487 8.21234716568142e-05
488 8.20596324047074e-05
489 8.19959968794137e-05
490 8.19324413896538e-05
491 8.18690969026648e-05
492 8.18058470031247e-05
493 8.17427644506097e-05
494 8.16797401057556e-05
495 8.16169485915452e-05
496 8.15542080090381e-05
497 8.14915329101495e-05
498 8.14290106063709e-05
499 8.13664519228041e-05
500 8.13040460343473e-05
501 8.12416692497209e-05
502 8.11792415333912e-05
503 8.11168574728072e-05
504 8.10544734122232e-05
505 8.09920893516392e-05
506 8.0929632531479e-05
507 8.08672048151493e-05
508 8.08047479949892e-05
509 8.07422329671681e-05
510 8.06796379038133e-05
511 8.06169482530095e-05
512 8.055421494646e-05
513 8.04913361207582e-05
514 8.0428428191226e-05
515 8.03653892944567e-05
516 8.03021976025775e-05
517 8.02388967713341e-05
518 8.01755886641331e-05
519 8.01120404503308e-05
520 8.00483685452491e-05
521 7.99845365690999e-05
522 7.99205154180527e-05
523 7.98563632997684e-05
524 7.97919856267981e-05
525 7.97274842625484e-05
526 7.96627355157398e-05
527 7.95977030065842e-05
528 7.95325031504035e-05
529 7.94670559116639e-05
530 7.94013540144078e-05
531 7.93353101471439e-05
532 7.92689461377449e-05
533 7.92023274698295e-05
534 7.91353450040333e-05
535 7.90680132922716e-05
536 7.9000397818163e-05
537 7.89323603385128e-05
538 7.88639517850243e-05
539 7.87951285019517e-05
540 7.8725912317168e-05
541 7.86562159191817e-05
542 7.8586112067569e-05
543 7.85155643825419e-05
544 7.84445001045242e-05
545 7.83728901296854e-05
546 7.83008435973898e-05
547 7.82282295404002e-05
548 7.81550916144624e-05
549 7.80814152676612e-05
550 7.80070913606323e-05
551 7.79322144808248e-05
552 7.78567409724928e-05
553 7.778056897223e-05
554 7.77037275838666e-05
555 7.76262022554874e-05
556 7.75479711592197e-05
557 7.74688887759112e-05
558 7.7389158832375e-05
559 7.73085848777555e-05
560 7.72272105677985e-05
561 7.71449340390973e-05
562 7.70617480156943e-05
563 7.69776670495048e-05
564 7.68927129684016e-05
565 7.68066092859954e-05
566 7.67193487263285e-05
567 7.66309749451466e-05
568 7.65414879424497e-05
569 7.64508440624923e-05
570 7.63590214774013e-05
571 7.62659037718549e-05
572 7.61713818064891e-05
573 7.6075506513007e-05
574 7.59781542001292e-05
575 7.58792302804068e-05
576 7.57787929615006e-05
577 7.5676609412767e-05
578 7.55726650822908e-05
579 7.54669308662415e-05
580 7.53591957618482e-05
581 7.5249576184433e-05
582 7.51378538552672e-05
583 7.50239050830714e-05
584 7.49077516957186e-05
585 7.47894155210815e-05
586 7.46686055208556e-05
587 7.45456054573879e-05
588 7.44201024645008e-05
589 7.42922056815587e-05
590 7.41615149308927e-05
591 7.40281247999519e-05
592 7.38920716685243e-05
593 7.37530572223477e-05
594 7.36113361199386e-05
595 7.34669229132123e-05
596 7.33198467059992e-05
597 7.31700129108503e-05
598 7.30174215277657e-05
599 7.28620361769572e-05
600 7.27037840988487e-05
601 7.25426652934402e-05
602 7.23787743481807e-05
603 7.22123149898835e-05
604 7.20430034562014e-05
605 7.18708688509651e-05
606 7.1695598307997e-05
607 7.15172427590005e-05
608 7.13361223461106e-05
609 7.11520988261327e-05
610 7.09655287209898e-05
611 7.07763538230211e-05
612 7.05847560311668e-05
613 7.0390866312664e-05
614 7.01946337358095e-05
615 6.99961965437979e-05
616 6.97955474606715e-05
617 6.95928611094132e-05
618 6.93881447659805e-05
619 6.91815657773986e-05
620 6.89732987666503e-05
621 6.87631036271341e-05
622 6.85511113260873e-05
623 6.83371690683998e-05
624 6.81214005453512e-05
625 6.7904191382695e-05
626 6.76850977470167e-05
627 6.74645852996036e-05
628 6.7243141529616e-05
629 6.70200097374618e-05
630 6.67951462673955e-05
631 6.65685802232474e-05
632 6.6340588091407e-05
633 6.61111989757046e-05
634 6.58805947750807e-05
635 6.56487754895352e-05
636 6.54159957775846e-05
637 6.51821246719919e-05
638 6.49474313831888e-05
639 6.47118285996839e-05
640 6.44756219116971e-05
641 6.42388986307196e-05
642 6.40019352431409e-05
643 6.37646371615119e-05
644 6.35271426290274e-05
645 6.32894734735601e-05
646 6.3051724282559e-05
647 6.28141569904983e-05
648 6.25768007012084e-05
649 6.23397863819264e-05
650 6.21032668277621e-05
651 6.18671983829699e-05
652 6.16318575339392e-05
653 6.13972370047122e-05
654 6.11633804510348e-05
655 6.09304806857836e-05
656 6.06987414357718e-05
657 6.04683118581306e-05
658 6.02392428845633e-05
659 6.0011643654434e-05
660 5.97855578234885e-05
661 5.95610472373664e-05
662 5.93381082580891e-05
663 5.91169045947026e-05
664 5.88974544371013e-05
665 5.86799942539074e-05
666 5.84643239562865e-05
667 5.82505999773275e-05
668 5.80389423703309e-05
669 5.78292274440173e-05
670 5.76215170440264e-05
671 5.74158875679132e-05
672 5.72123899473809e-05
673 5.7011035096366e-05
674 5.68118084629532e-05
675 5.66149246878922e-05
676 5.64202600799035e-05
677 5.6228025641758e-05
678 5.60381085961126e-05
679 5.58504907530732e-05
680 5.56653649255168e-05
681 5.548277113121e-05
682 5.53024401597213e-05
683 5.51243938389234e-05
684 5.49486903764773e-05
685 5.47752788406797e-05
686 5.46042138012126e-05
687 5.4435535275843e-05
688 5.42691559530795e-05
689 5.41051085747313e-05
690 5.39434513484593e-05
691 5.37841078767087e-05
692 5.36272309545893e-05
693 5.34725695615634e-05
694 5.332021828508e-05
695 5.31700461579021e-05
696 5.30220786458813e-05
697 5.28763084730599e-05
698 5.27327065356076e-05
699 5.25911964359693e-05
700 5.2451818191912e-05
701 5.23144481121562e-05
702 5.21790607308503e-05
703 5.20456596859731e-05
704 5.19141685799696e-05
705 5.17845910508186e-05
706 5.1656985306181e-05
707 5.15313367941417e-05
708 5.14074963575695e-05
709 5.12852930114605e-05
710 5.11648431711365e-05
711 5.10460849909578e-05
712 5.09289930050727e-05
713 5.08135235577356e-05
714 5.06996802869253e-05
715 5.05873431393411e-05
716 5.04765666846652e-05
717 5.03673145431094e-05
718 5.02596885780804e-05
719 5.0153503252659e-05
720 5.00489404657856e-05
721 4.99461748404428e-05
722 4.98451045132242e-05
723 4.97457367600873e-05
724 4.96479660796467e-05
725 4.95518725074362e-05
726 4.94572595926002e-05
727 4.93642000947148e-05
728 4.92725630465429e-05
729 4.91823484480847e-05
730 4.90936727146618e-05
731 4.90064085170161e-05
732 4.89206504425965e-05
733 4.8836249334272e-05
734 4.8753157898318e-05
735 4.86713179270737e-05
736 4.85907730762847e-05
737 4.85113523609471e-05
738 4.84332013002131e-05
739 4.83560579596087e-05
740 4.82800314784981e-05
741 4.82050381833687e-05
742 4.81310999020934e-05
743 4.8058256652439e-05
744 4.79865047964267e-05
745 4.79156406072434e-05
746 4.78456167911645e-05
747 4.77764842798933e-05
748 4.77080175187439e-05
749 4.76404275104869e-05
750 4.75735178042669e-05
751 4.75073902634904e-05
752 4.74420448881574e-05
753 4.73773689009249e-05
754 4.73133586638141e-05
755 4.72500214527827e-05
756 4.71873063361272e-05
757 4.71251878479961e-05
758 4.70635823148768e-05
759 4.700260979007e-05
760 4.69421247544233e-05
761 4.68822036054917e-05
762 4.68227262899745e-05
763 4.6763652790105e-05
764 4.67050158476923e-05
765 4.66468482045457e-05
766 4.65890370833222e-05
767 4.65321172669064e-05
768 4.64754812128376e-05
769 4.64192453364376e-05
770 4.63631768070627e-05
771 4.63075339212082e-05
772 4.62520401924849e-05
773 4.6196837502066e-05
774 4.61417439510114e-05
775 4.6086948714219e-05
776 4.60323644801974e-05
777 4.59779002994765e-05
778 4.5923690777272e-05
779 4.58697395515628e-05
780 4.58158247056417e-05
781 4.57621899840888e-05
782 4.57086935057305e-05
783 4.56552988907788e-05
784 4.56020243291277e-05
785 4.55489353043959e-05
786 4.54959699709434e-05
787 4.54431465186644e-05
788 4.53903667221311e-05
789 4.53377106168773e-05
790 4.5285138185136e-05
791 4.5232693082653e-05
792 4.51802625320852e-05
793 4.51279738626909e-05
794 4.50757397629786e-05
795 4.5023589336779e-05
796 4.49714898422826e-05
797 4.49194922111928e-05
798 4.48677383246832e-05
799 4.48158207291272e-05
800 4.47640886704903e-05
801 4.4712440285366e-05
802 4.46607627964113e-05
803 4.46092381025665e-05
804 4.45577825303189e-05
805 4.4506410631584e-05
806 4.44550314568914e-05
807 4.44037395936903e-05
808 4.4352636905387e-05
809 4.43014396296348e-05
810 4.42504315287806e-05
811 4.41995216533542e-05
812 4.41486008639913e-05
813 4.40978219558019e-05
814 4.40471194451675e-05
815 4.39965479017701e-05
816 4.39460782217793e-05
817 4.38958377344534e-05
818 4.38455972471274e-05
819 4.37955895904452e-05
820 4.37456692452542e-05
821 4.36958325735759e-05
822 4.36462250945624e-05
823 4.35963847849052e-05
824 4.35464862675872e-05
825 4.34968169429339e-05
826 4.34472640336026e-05
827 4.33981076639611e-05
828 4.3349435145501e-05
829 4.33007771789562e-05
830 4.32523665949702e-05
831 4.32040433224756e-05
832 4.3155927414773e-05
833 4.31078406109009e-05
834 4.30599102401175e-05
835 4.30120162491221e-05
836 4.29643005190883e-05
837 4.2916650272673e-05
838 4.28691164415795e-05
839 4.28217099397443e-05
840 4.27743907494005e-05
841 4.27271515945904e-05
842 4.26800579589326e-05
843 4.26330661866814e-05
844 4.25861871917732e-05
845 4.25393700425047e-05
846 4.24926220148336e-05
847 4.24460959038697e-05
848 4.23995406890754e-05
849 4.23532110289671e-05
850 4.23068595409859e-05
851 4.22605917265173e-05
852 4.22145058109891e-05
853 4.21684198954608e-05
854 4.21224613091908e-05
855 4.20766846218612e-05
856 4.20308679167647e-05
857 4.19851712649688e-05
858 4.19395364588127e-05
859 4.18939343944658e-05
860 4.18484596593771e-05
861 4.18030358559918e-05
862 4.17576738982461e-05
863 4.17123483202886e-05
864 4.1667135519674e-05
865 4.16219772887416e-05
866 4.15768663515337e-05
867 4.15318281739019e-05
868 4.14868991356343e-05
869 4.14419882872608e-05
870 4.13971429225057e-05
871 4.13524358009454e-05
872 4.13077505072579e-05
873 4.12631379731465e-05
874 4.12186564062722e-05
875 4.1174153011525e-05
876 4.11297478422057e-05
877 4.10854408983141e-05
878 4.10412067139987e-05
879 4.099692159798e-05
880 4.09527783631347e-05
881 4.09086860599928e-05
882 4.08646592404693e-05
883 4.08206142310519e-05
884 4.07766892749351e-05
885 4.07327534048818e-05
886 4.06888830184471e-05
887 4.06451217713766e-05
888 4.06013095926028e-05
889 4.05575519835111e-05
890 4.05138889618684e-05
891 4.04703096137382e-05
892 4.04267011617776e-05
893 4.03831654693931e-05
894 4.03396770707332e-05
895 4.02961268264335e-05
896 4.02527439291589e-05
897 4.02093282900751e-05
898 4.01659781346098e-05
899 4.01226752728689e-05
900 4.0079430618789e-05
901 4.0036225982476e-05
902 3.99930322600994e-05
903 3.99498057959136e-05
904 3.99067102989648e-05
905 3.98636293539312e-05
906 3.98205520468764e-05
907 3.97775111196097e-05
908 3.97344265365973e-05
909 3.96913565055002e-05
910 3.96482828364242e-05
911 3.96052528230939e-05
912 3.95622009818908e-05
913 3.9519218262285e-05
914 3.94762610085309e-05
915 3.94333692383952e-05
916 3.93905102100689e-05
917 3.93477203033399e-05
918 3.93049485865049e-05
919 3.9262100472115e-05
920 3.92192923754919e-05
921 3.9176557038445e-05
922 3.91338908229955e-05
923 3.90913373848889e-05
924 3.90488312405068e-05
925 3.90063541999552e-05
926 3.89639208151493e-05
927 3.89215128961951e-05
928 3.88791595469229e-05
929 3.88368534913752e-05
930 3.87945910915732e-05
931 3.87521940865554e-05
932 3.87099498766474e-05
933 3.86676474590786e-05
934 3.86255014745984e-05
935 3.85832790925633e-05
936 3.85410021408461e-05
937 3.84987506549805e-05
938 3.84562263207044e-05
939 3.8413709262386e-05
940 3.83711922040675e-05
941 3.83287369913887e-05
942 3.82862272090279e-05
943 3.82436519430485e-05
944 3.82010621251538e-05
945 3.81584541173652e-05
946 3.81159006792586e-05
947 3.80732781195547e-05
948 3.80307537852786e-05
949 3.79881930712145e-05
950 3.79456214432139e-05
951 3.79030825570226e-05
952 3.78606491722167e-05
953 3.78181393898558e-05
954 3.77757860405836e-05
955 3.7733403587481e-05
956 3.76911848434247e-05
957 3.76489806512836e-05
958 3.76068637706339e-05
959 3.75647759938147e-05
960 3.75227573385928e-05
961 3.74808078049682e-05
962 3.74388873751741e-05
963 3.73970142391045e-05
964 3.73552102246322e-05
965 3.73134462279268e-05
966 3.72718131984584e-05
967 3.72301910829265e-05
968 3.71886089851614e-05
969 3.71471105609089e-05
970 3.71056594303809e-05
971 3.70642665075138e-05
972 3.7022888136562e-05
973 3.69816698366776e-05
974 3.69404078810476e-05
975 3.68992332369089e-05
976 3.68581349903252e-05
977 3.68170585716143e-05
978 3.67760112567339e-05
979 3.67350330634508e-05
980 3.66942804248538e-05
981 3.66535932698753e-05
982 3.66130370821338e-05
983 3.65725354640745e-05
984 3.65321720892098e-05
985 3.64918778359424e-05
986 3.64516927220393e-05
987 3.64115549018607e-05
988 3.63715516868979e-05
989 3.63316139555536e-05
990 3.62916944141034e-05
991 3.62518258043565e-05
992 3.62120226782281e-05
993 3.61722995876335e-05
994 3.61327001883183e-05
995 3.60930898750667e-05
996 3.60536105290521e-05
997 3.60141275450587e-05
998 3.59747173206415e-05
999 3.59353798558004e-05
1000 3.58961115125567e-05
1001 3.58568831870798e-05
1002 3.5817691241391e-05
1003 3.57785611413419e-05
1004 3.57394783350173e-05
1005 3.57004792022053e-05
1006 3.56614909833297e-05
1007 3.56225900759455e-05
1008 3.55837291863281e-05
1009 3.5544864658732e-05
1010 3.55060728907119e-05
1011 3.5467299312586e-05
1012 3.5428620321909e-05
1013 3.53899886249565e-05
1014 3.53513532900251e-05
1015 3.53127470589243e-05
1016 3.52742026734632e-05
1017 3.52357383235358e-05
1018 3.51972958014812e-05
1019 3.51588969351724e-05
1020 3.51205562765244e-05
1021 3.50822301697917e-05
1022 3.5043998650508e-05
1023 3.5005752579309e-05
1024 3.49675901816227e-05
1025 3.49294750776608e-05
1026 3.48913999914657e-05
1027 3.48533685610164e-05
1028 3.4815395338228e-05
1029 3.47773966495879e-05
1030 3.47394925483968e-05
1031 3.47015775332693e-05
1032 3.46637098118663e-05
1033 3.46259112120606e-05
1034 3.45880944223609e-05
1035 3.45504049619194e-05
1036 3.45126973115839e-05
1037 3.44750078511424e-05
1038 3.44373802363407e-05
1039 3.43997526215389e-05
1040 3.43621832143981e-05
1041 3.43246829288546e-05
1042 3.42871608154383e-05
1043 3.42496969096828e-05
1044 3.42122766596731e-05
1045 3.41748745995574e-05
1046 3.41375307471026e-05
1047 3.41001941706054e-05
1048 3.40629048878327e-05
1049 3.40256483468693e-05
1050 3.39884318236727e-05
1051 3.39511789206881e-05
1052 3.39140169671737e-05
1053 3.38768804795109e-05
1054 3.38397439918481e-05
1055 3.38026729878038e-05
1056 3.37656092597172e-05
1057 3.37285273417365e-05
1058 3.36914963554591e-05
1059 3.36544944730122e-05
1060 3.36174452968407e-05
1061 3.35805380018428e-05
1062 3.35435724991839e-05
1063 3.35066506522708e-05
1064 3.34697106154636e-05
1065 3.34328324242961e-05
1066 3.33959069394041e-05
1067 3.3359039662173e-05
1068 3.33222051267512e-05
1069 3.32853524014354e-05
1070 3.324852877995e-05
1071 3.32116942445282e-05
1072 3.31748415192124e-05
1073 3.31379487761296e-05
1074 3.31010669469833e-05
1075 3.30642178596463e-05
1076 3.30273505824152e-05
1077 3.29905087710358e-05
1078 3.2953674235614e-05
1079 3.29168797179591e-05
1080 3.28800488205161e-05
1081 3.28432906826492e-05
1082 3.28065107169095e-05
1083 3.27697198372334e-05
1084 3.27330344589427e-05
1085 3.26962690451182e-05
1086 3.26595472870395e-05
1087 3.26228328049183e-05
1088 3.25861292367335e-05
1089 3.254942203057e-05
1090 3.2512722100364e-05
1091 3.24760985677131e-05
1092 3.24394095514435e-05
1093 3.24027678288985e-05
1094 3.23660860885866e-05
1095 3.23294771078508e-05
1096 3.22927699016873e-05
1097 3.22561354550999e-05
1098 3.22194791806396e-05
1099 3.21828156302217e-05
1100 3.21461375278886e-05
1101 3.21094194077887e-05
1102 3.20728431688622e-05
1103 3.20361614285503e-05
1104 3.19994906021748e-05
1105 3.19627761200536e-05
1106 3.19261162076145e-05
1107 3.18893871735781e-05
1108 3.18527236231603e-05
1109 3.18160164169967e-05
1110 3.17793055728544e-05
1111 3.17426711262669e-05
1112 3.17059493681882e-05
1113 3.16692276101094e-05
1114 3.16325676976703e-05
1115 3.15958313876763e-05
1116 3.15591423714068e-05
1117 3.15223769575823e-05
1118 3.14856370096095e-05
1119 3.14488897856791e-05
1120 3.14121425617486e-05
1121 3.13753553200513e-05
1122 3.1338586268248e-05
1123 3.13017844746355e-05
1124 3.12649899569806e-05
1125 3.12281663354952e-05
1126 3.1191320886137e-05
1127 3.11544827127364e-05
1128 3.11176008835901e-05
1129 3.10806972265709e-05
1130 3.10437753796577e-05
1131 3.10067989630625e-05
1132 3.09697898046579e-05
1133 3.09327806462534e-05
1134 3.08957423840184e-05
1135 3.0858711397741e-05
1136 3.08216149278451e-05
1137 3.07845148199704e-05
1138 3.07474256260321e-05
1139 3.07102236547507e-05
1140 3.06730507872999e-05
1141 3.06357760564424e-05
1142 3.05985049635638e-05
1143 3.0561161111109e-05
1144 3.05238136206754e-05
1145 3.04864443023689e-05
1146 3.04490422422532e-05
1147 3.04116365441587e-05
1148 3.03741653624456e-05
1149 3.03366923617432e-05
1150 3.02992193610407e-05
1151 3.02616754197516e-05
1152 3.02240896417061e-05
1153 3.01865420624381e-05
1154 3.01489453704562e-05
1155 3.0111288651824e-05
1156 3.00735791824991e-05
1157 3.00356259685941e-05
1158 2.99974835797912e-05
1159 2.99593284580624e-05
1160 2.99211187666515e-05
1161 2.98828636005055e-05
1162 2.98445720545715e-05
1163 2.98063041554997e-05
1164 2.97679453069577e-05
1165 2.97295973723521e-05
1166 2.96912476187572e-05
1167 2.96528105536709e-05
1168 2.96143280138494e-05
1169 2.95758418360492e-05
1170 2.95373047265457e-05
1171 2.94987130473601e-05
1172 2.94600959023228e-05
1173 2.94214260065928e-05
1174 2.93827488349052e-05
1175 2.93440189125249e-05
1176 2.93052526103565e-05
1177 2.92664426524425e-05
1178 2.92275872197933e-05
1179 2.91886863124091e-05
1180 2.91497472062474e-05
1181 2.9110773539287e-05
1182 2.90717453026446e-05
1183 2.90326916001504e-05
1184 2.89935578621225e-05
1185 2.8954416848137e-05
1186 2.89151848846814e-05
1187 2.88759074464906e-05
1188 2.88366518361727e-05
1189 2.87972907244693e-05
1190 2.87578968709568e-05
1191 2.87184629996773e-05
1192 2.86789618257899e-05
1193 2.8639413358178e-05
1194 2.85998394247144e-05
1195 2.85602272924734e-05
1196 2.85205260297516e-05
1197 2.84807756543159e-05
1198 2.84410125459544e-05
1199 2.84012021438684e-05
1200 2.83613171632169e-05
1201 2.83213666989468e-05
1202 2.82814271486131e-05
1203 2.82413657259895e-05
1204 2.82013061223552e-05
1205 2.81611337413779e-05
1206 2.8120999559178e-05
1207 2.80807307717623e-05
1208 2.80404747172724e-05
1209 2.80002041108673e-05
1210 2.79598280030768e-05
1211 2.7919373678742e-05
1212 2.78789084404707e-05
1213 2.78383577096974e-05
1214 2.7797792427009e-05
1215 2.77571780316066e-05
1216 2.77165290754056e-05
1217 2.7675805540639e-05
1218 2.76350547210313e-05
1219 2.75942093139747e-05
1220 2.75533202511724e-05
1221 2.75124093604973e-05
1222 2.74714311672142e-05
1223 2.7430380214355e-05
1224 2.73893256235169e-05
1225 2.73483001365094e-05
1226 2.73071873380104e-05
1227 2.72660035989247e-05
1228 2.72248344117543e-05
1229 2.7183566999156e-05
1230 2.71422904916108e-05
1231 2.71009794232668e-05
1232 2.70595282927388e-05
1233 2.70181008090731e-05
1234 2.69766023848206e-05
1235 2.6935076675727e-05
1236 2.68935600615805e-05
1237 2.6851976144826e-05
1238 2.6810308554559e-05
1239 2.6768640964292e-05
1240 2.67269533651415e-05
1241 2.66851711785421e-05
1242 2.66433798969956e-05
1243 2.660152495082e-05
1244 2.65596245299093e-05
1245 2.6517707738094e-05
1246 2.64756872638827e-05
1247 2.64336722466396e-05
1248 2.6391573555884e-05
1249 2.63495385297574e-05
1250 2.63073434325634e-05
1251 2.6265173801221e-05
1252 2.62229314103024e-05
1253 2.61806690105004e-05
1254 2.6138368411921e-05
1255 2.60960787272779e-05
1256 2.60537508438574e-05
1257 2.60112919931998e-05
1258 2.5968894988182e-05
1259 2.59264506894397e-05
1260 2.58839081652695e-05
1261 2.58413019764703e-05
1262 2.57987467193743e-05
1263 2.57560877798824e-05
1264 2.57134506682632e-05
1265 2.56707480730256e-05
1266 2.56280072790105e-05
1267 2.5585273760953e-05
1268 2.5542474759277e-05
1269 2.54997103183996e-05
1270 2.54568240052322e-05
1271 2.54139504249906e-05
1272 2.53710750257596e-05
1273 2.53281868936028e-05
1274 2.52852241828805e-05
1275 2.5242292394978e-05
1276 2.51993187703192e-05
1277 2.51562942139572e-05
1278 2.51132405537646e-05
1279 2.5070226911339e-05
1280 2.50271896220511e-05
1281 2.49841014010599e-05
1282 2.49409913521959e-05
1283 2.4897877665353e-05
1284 2.48547748924466e-05
1285 2.48116448346991e-05
1286 2.47685347858351e-05
1287 2.47254811256425e-05
1288 2.46824129135348e-05
1289 2.46393210545648e-05
1290 2.45962710323511e-05
1291 2.45532119151903e-05
1292 2.45101819018601e-05
1293 2.4467157345498e-05
1294 2.44241400650935e-05
1295 2.43811755353818e-05
1296 2.43382182816276e-05
1297 2.42954101850046e-05
1298 2.42527967202477e-05
1299 2.42101486946922e-05
1300 2.41675679717446e-05
1301 2.41249999817228e-05
1302 2.40824974753195e-05
1303 2.40399876929587e-05
1304 2.39975979638984e-05
1305 2.39552136918064e-05
1306 2.39128949033329e-05
1307 2.38706070376793e-05
1308 2.3828388293623e-05
1309 2.378622048127e-05
1310 2.37440581258852e-05
1311 2.37019867199706e-05
1312 2.36599407799076e-05
1313 2.36179403145798e-05
1314 2.35760271607433e-05
1315 2.35341831285041e-05
1316 2.34923409152543e-05
1317 2.34505951084429e-05
1318 2.34088911383878e-05
1319 2.33672544709407e-05
1320 2.33256851061014e-05
1321 2.32841903198278e-05
1322 2.32427519222256e-05
1323 2.32014026551042e-05
1324 2.31601115956437e-05
1325 2.31188878387911e-05
1326 2.30777750402922e-05
1327 2.30366986215813e-05
1328 2.29956986004254e-05
1329 2.29548186325701e-05
1330 2.29139805014711e-05
1331 2.28732333198423e-05
1332 2.28326171054505e-05
1333 2.27920281758998e-05
1334 2.27515301958192e-05
1335 2.27111231652088e-05
1336 2.26708343689097e-05
1337 2.26306456170278e-05
1338 2.2590547814616e-05
1339 2.25505336857168e-05
1340 2.25106468860758e-05
1341 2.24708455789369e-05
1342 2.24311406782363e-05
1343 2.23915631067939e-05
1344 2.23520873987582e-05
1345 2.23126462515211e-05
1346 2.22733451664681e-05
1347 2.22341150220018e-05
1348 2.2195026758709e-05
1349 2.2156027625897e-05
1350 2.21171958401101e-05
1351 2.20784695557086e-05
1352 2.20398560486501e-05
1353 2.20013334910618e-05
1354 2.19630092033185e-05
1355 2.19248213397805e-05
1356 2.18866571231047e-05
1357 2.18486748053692e-05
1358 2.18108798435424e-05
1359 2.17732067540055e-05
1360 2.17356537177693e-05
1361 2.1698244381696e-05
1362 2.16610278584994e-05
1363 2.16239932342432e-05
1364 2.15870877582347e-05
1365 2.15503296203678e-05
1366 2.15137843042612e-05
1367 2.14773335756036e-05
1368 2.14410629268968e-05
1369 2.14049214264378e-05
1370 2.13689418160357e-05
1371 2.133309499186e-05
1372 2.12973809539108e-05
1373 2.12618524528807e-05
1374 2.12264603760559e-05
1375 2.11911628866801e-05
1376 2.11560691241175e-05
1377 2.11211208807072e-05
1378 2.10863163374597e-05
1379 2.10516609513434e-05
1380 2.10171729122521e-05
1381 2.09828194783768e-05
1382 2.09486734092934e-05
1383 2.0914631022606e-05
1384 2.08807741728378e-05
1385 2.08470737561584e-05
1386 2.08135134016629e-05
1387 2.07801422220655e-05
1388 2.07469183806097e-05
1389 2.07138473342638e-05
1390 2.06809436349431e-05
1391 2.06482218345627e-05
1392 2.06156619242392e-05
1393 2.05832402571104e-05
1394 2.05509895749856e-05
1395 2.05189189728117e-05
1396 2.04870266315993e-05
1397 2.0455265257624e-05
1398 2.04236912395572e-05
1399 2.03922827495262e-05
1400 2.03610034077428e-05
1401 2.03299605345819e-05
1402 2.02990431716898e-05
1403 2.02683022507699e-05
1404 2.02377341338433e-05
1405 2.0207302441122e-05
1406 2.01770399144152e-05
1407 2.01468847080832e-05
1408 2.01169095817022e-05
1409 2.00870945263887e-05
1410 2.00574686459731e-05
1411 2.00279955606675e-05
1412 1.99987134692492e-05
1413 1.9969598724856e-05
1414 1.99406622414244e-05
1415 1.99118658201769e-05
1416 1.9883274944732e-05
1417 1.98548696062062e-05
1418 1.98265806830022e-05
1419 1.9798524590442e-05
1420 1.97706267499598e-05
1421 1.97428780666087e-05
1422 1.97153494809754e-05
1423 1.96879936993355e-05
1424 1.96607470570598e-05
1425 1.96337114175549e-05
1426 1.96068322111387e-05
1427 1.95801421796205e-05
1428 1.95536304090638e-05
1429 1.95272768905852e-05
1430 1.95010507013649e-05
1431 1.94750100490637e-05
1432 1.94491858565016e-05
1433 1.94234926311765e-05
1434 1.9398023141548e-05
1435 1.93726937141037e-05
1436 1.93475752894301e-05
1437 1.93226387636969e-05
1438 1.9297875041957e-05
1439 1.92732804862317e-05
1440 1.92488823813619e-05
1441 1.9224591596867e-05
1442 1.92005882126978e-05
1443 1.91766976058716e-05
1444 1.91529725270811e-05
1445 1.91294620890403e-05
1446 1.9106093532173e-05
1447 1.90829086932354e-05
1448 1.90598912013229e-05
1449 1.90370592463296e-05
1450 1.90143546205945e-05
1451 1.89918591786409e-05
1452 1.89695165317971e-05
1453 1.89473448699573e-05
1454 1.89254042197717e-05
1455 1.89035599760246e-05
1456 1.8881930373027e-05
1457 1.88604317372665e-05
1458 1.88390949915629e-05
1459 1.88179419637891e-05
1460 1.87969526450615e-05
1461 1.87761197594227e-05
1462 1.87554323929362e-05
1463 1.87349633051781e-05
1464 1.87145869858796e-05
1465 1.86943889275426e-05
1466 1.86743491212837e-05
1467 1.86544984899228e-05
1468 1.86347479029791e-05
1469 1.86151864909334e-05
1470 1.85957469511777e-05
1471 1.85764911293518e-05
1472 1.85573826456675e-05
1473 1.85383723874111e-05
1474 1.85195513040526e-05
1475 1.85008757398464e-05
1476 1.84823002200574e-05
1477 1.84639156941557e-05
1478 1.84456821443746e-05
1479 1.84275668289047e-05
1480 1.840958793764e-05
1481 1.8391729099676e-05
1482 1.83740266948007e-05
1483 1.83564498001942e-05
1484 1.83389965968672e-05
1485 1.83216889126925e-05
1486 1.83045012818184e-05
1487 1.82874500751495e-05
1488 1.82705243787495e-05
1489 1.82536914508091e-05
1490 1.82370258698938e-05
1491 1.82204566954169e-05
1492 1.82040221261559e-05
1493 1.81877239811001e-05
1494 1.81714767677477e-05
1495 1.81554605660494e-05
1496 1.81394352694042e-05
1497 1.8123608242604e-05
1498 1.81078466994222e-05
1499 1.80922615982126e-05
1500 1.80767092388123e-05
1501 1.80613096745219e-05
1502 1.80460174306063e-05
1503 1.8030776118394e-05
1504 1.80156748683657e-05
1505 1.80006936716381e-05
1506 1.79857725015609e-05
1507 1.79709913936676e-05
1508 1.79562339326367e-05
1509 1.79416456376202e-05
1510 1.79271319211693e-05
1511 1.79126891453052e-05
1512 1.78983591467841e-05
1513 1.78841401066165e-05
1514 1.78699337993748e-05
1515 1.78558839252219e-05
1516 1.78418376890477e-05
1517 1.78279642568668e-05
1518 1.78141417563893e-05
1519 1.78003574546892e-05
1520 1.7786716853152e-05
1521 1.77731017174665e-05
1522 1.77595502464101e-05
1523 1.77461224666331e-05
1524 1.77327019628137e-05
1525 1.77194124262314e-05
1526 1.77061283466173e-05
1527 1.76929825101979e-05
1528 1.76798366737785e-05
1529 1.76667981577339e-05
1530 1.76537978404667e-05
1531 1.76408902916592e-05
1532 1.76280136656715e-05
1533 1.76152207131963e-05
1534 1.76024641405093e-05
1535 1.75897439476103e-05
1536 1.75771201611497e-05
1537 1.75645127455937e-05
1538 1.75519871845609e-05
1539 1.75394634425174e-05
1540 1.75270452018594e-05
1541 1.75146415131167e-05
1542 1.75022960320348e-05
1543 1.74900178535609e-05
1544 1.74777323991293e-05
1545 1.74655215232633e-05
1546 1.74533324752701e-05
1547 1.74412089108955e-05
1548 1.74290853465209e-05
1549 1.74170454556588e-05
1550 1.74050092027755e-05
1551 1.73929856828181e-05
1552 1.73810276464792e-05
1553 1.73690950759919e-05
1554 1.73571625055047e-05
1555 1.73453372553922e-05
1556 1.73334901774069e-05
1557 1.73216703842627e-05
1558 1.73099106177688e-05
1559 1.72981817740947e-05
1560 1.72864820342511e-05
1561 1.72748132172273e-05
1562 1.72631716850447e-05
1563 1.72515774465865e-05
1564 1.72400123119587e-05
1565 1.72284926520661e-05
1566 1.72169820871204e-05
1567 1.72055479197297e-05
1568 1.71941210282966e-05
1569 1.71827250596834e-05
1570 1.71713709278265e-05
1571 1.71600622707047e-05
1572 1.71488245541696e-05
1573 1.71376104844967e-05
1574 1.71264418895589e-05
1575 1.71153296832927e-05
1576 1.71042647707509e-05
1577 1.70932689798065e-05
1578 1.70822931977455e-05
1579 1.70714301930275e-05
1580 1.70606017491082e-05
1581 1.70498333318392e-05
1582 1.70392086147331e-05
1583 1.70286257343832e-05
1584 1.70181374414824e-05
1585 1.70077746588504e-05
1586 1.69974591699429e-05
1587 1.69873601407744e-05
1588 1.69773556990549e-05
1589 1.69674731296254e-05
1590 1.69577924680198e-05
1591 1.69482627825346e-05
1592 1.69389404618414e-05
1593 1.69298327818979e-05
1594 1.69209433806827e-05
1595 1.69123195519205e-05
1596 1.69039576576324e-05
1597 1.68959104485111e-05
1598 1.68881797435461e-05
1599 1.68808419402922e-05
1600 1.68738697539084e-05
1601 1.68673286680132e-05
1602 1.68612568813842e-05
1603 1.68556853168411e-05
1604 1.68506358022569e-05
1605 1.6846192011144e-05
1606 1.68423575814813e-05
1607 1.68391707120463e-05
1608 1.68366786965635e-05
1609 1.68349070008844e-05
1610 1.68339356605429e-05
1611 1.68337228387827e-05
1612 1.68342903634766e-05
1613 1.68356582435081e-05
1614 1.68378155649407e-05
1615 1.68407004821347e-05
1616 1.68443002621643e-05
1617 1.68485603353474e-05
1618 1.6853367924341e-05
1619 1.68586138897808e-05
1620 1.68641890923027e-05
1621 1.68699698406272e-05
1622 1.68757724168245e-05
1623 1.68814422067953e-05
1624 1.688679549261e-05
1625 1.68916740221903e-05
1626 1.68959359143628e-05
1627 1.68994502018904e-05
1628 1.69020331668435e-05
1629 1.69036557053914e-05
1630 1.69042596098734e-05
1631 1.69037193700206e-05
1632 1.6902125935303e-05
1633 1.6899430193007e-05
1634 1.68956794368569e-05
1635 1.68908918567467e-05
1636 1.6885182049009e-05
1637 1.68785791174741e-05
1638 1.68711576407077e-05
1639 1.68630103871692e-05
1640 1.68541828315938e-05
1641 1.68447822943563e-05
1642 1.68348469742341e-05
1643 1.68244841916021e-05
1644 1.6813652109704e-05
1645 1.68025253515225e-05
1646 1.67910366144497e-05
1647 1.67793259606697e-05
1648 1.67673515534261e-05
1649 1.67551570484648e-05
1650 1.67428461281816e-05
1651 1.67303278431064e-05
1652 1.67176622198895e-05
1653 1.67049565789057e-05
1654 1.66921254276531e-05
1655 1.66792597156018e-05
1656 1.66662866831757e-05
1657 1.66533191077178e-05
1658 1.66402896866202e-05
1659 1.66272675414803e-05
1660 1.66142363013932e-05
1661 1.66012141562533e-05
1662 1.65881847351557e-05
1663 1.65751698659733e-05
1664 1.65622041095048e-05
1665 1.65492619998986e-05
1666 1.65363344422076e-05
1667 1.65234450832941e-05
1668 1.65105993801262e-05
1669 1.6497788237757e-05
1670 1.64850462169852e-05
1671 1.64723605848849e-05
1672 1.64597186085302e-05
1673 1.64471257448895e-05
1674 1.64346329256659e-05
1675 1.64221582963364e-05
1676 1.64097709784983e-05
1677 1.63974636961939e-05
1678 1.63851982506458e-05
1679 1.63730164786102e-05
1680 1.63608801813098e-05
1681 1.63488421094371e-05
1682 1.63368677021936e-05
1683 1.63249933393672e-05
1684 1.631318264117e-05
1685 1.63014410645701e-05
1686 1.62897922564298e-05
1687 1.62782016559504e-05
1688 1.62666910910048e-05
1689 1.62552714755293e-05
1690 1.62439173436724e-05
1691 1.62326632562326e-05
1692 1.62215146701783e-05
1693 1.62104151968379e-05
1694 1.61993957590312e-05
1695 1.61884781846311e-05
1696 1.61775915330509e-05
1697 1.61668540386017e-05
1698 1.61561565619195e-05
1699 1.6145539120771e-05
1700 1.61350417329231e-05
1701 1.61245880008209e-05
1702 1.61142470460618e-05
1703 1.61039770318894e-05
1704 1.60937652253779e-05
1705 1.60836516442942e-05
1706 1.60736144607654e-05
1707 1.60637064254843e-05
1708 1.60538402269594e-05
1709 1.60440577019472e-05
1710 1.60343643074157e-05
1711 1.60247618623544e-05
1712 1.60152230819222e-05
1713 1.60057752509601e-05
1714 1.59964183694683e-05
1715 1.59871287905844e-05
1716 1.59779337991495e-05
1717 1.59688006533543e-05
1718 1.59597511810716e-05
1719 1.5950787201291e-05
1720 1.59418741532136e-05
1721 1.59330575115746e-05
1722 1.59243263624376e-05
1723 1.59156988956966e-05
1724 1.59071296366164e-05
1725 1.5898613128229e-05
1726 1.58902257680893e-05
1727 1.58818893396528e-05
1728 1.58736329467501e-05
1729 1.58654729602858e-05
1730 1.58573948283447e-05
1731 1.58493494382128e-05
1732 1.58414131874451e-05
1733 1.58335842570523e-05
1734 1.58258426381508e-05
1735 1.5818162864889e-05
1736 1.58105649461504e-05
1737 1.58030743477866e-05
1738 1.57956546900095e-05
1739 1.57883059728192e-05
1740 1.57810391101521e-05
1741 1.57738577399869e-05
1742 1.57667473104084e-05
1743 1.57596969074802e-05
1744 1.57527283590753e-05
1745 1.57458453031722e-05
1746 1.57390004460467e-05
1747 1.57322247105185e-05
1748 1.57255035446724e-05
1749 1.5718853319413e-05
1750 1.57122449309099e-05
1751 1.5705700207036e-05
1752 1.56991955009289e-05
1753 1.56927362695569e-05
1754 1.56863552547293e-05
1755 1.56799906108063e-05
1756 1.56736950884806e-05
1757 1.56674159370596e-05
1758 1.5661184079363e-05
1759 1.56549540406559e-05
1760 1.56487931235461e-05
1761 1.56426376634045e-05
1762 1.56365076691145e-05
1763 1.5630406778655e-05
1764 1.56243022502167e-05
1765 1.56181813508738e-05
1766 1.56120877363719e-05
1767 1.56060286826687e-05
1768 1.55999259732198e-05
1769 1.55938396346755e-05
1770 1.5587720554322e-05
1771 1.55815905600321e-05
1772 1.55754460138269e-05
1773 1.55692869157065e-05
1774 1.55631150846602e-05
1775 1.55568850459531e-05
1776 1.55506440933095e-05
1777 1.55443776748143e-05
1778 1.5538083971478e-05
1779 1.55317411554279e-05
1780 1.55253655975685e-05
1781 1.5518953659921e-05
1782 1.55125253513688e-05
1783 1.55060588440392e-05
1784 1.54995286720805e-05
1785 1.54929457494291e-05
1786 1.5486337360926e-05
1787 1.54797107825289e-05
1788 1.54730259964708e-05
1789 1.54662957356777e-05
1790 1.54595145431813e-05
1791 1.54527133418014e-05
1792 1.54458557517501e-05
1793 1.54390017996775e-05
1794 1.54320878209546e-05
1795 1.54251356434543e-05
1796 1.54181634570705e-05
1797 1.5411138519994e-05
1798 1.54040935740341e-05
1799 1.53970158862649e-05
1800 1.5389921827591e-05
1801 1.53827859321609e-05
1802 1.53756191139109e-05
1803 1.53684231918305e-05
1804 1.53612618305488e-05
1805 1.53540313476697e-05
1806 1.53468154167058e-05
1807 1.5339532183134e-05
1808 1.53322471305728e-05
1809 1.5324945707107e-05
1810 1.53176479216199e-05
1811 1.53103192133131e-05
1812 1.5302994142985e-05
1813 1.52956690726569e-05
1814 1.52883021655725e-05
1815 1.52809716382762e-05
1816 1.52735919982661e-05
1817 1.52662396430969e-05
1818 1.52588818309596e-05
1819 1.52515176523593e-05
1820 1.52441562022432e-05
1821 1.52367929331376e-05
1822 1.52294442159473e-05
1823 1.52221182361245e-05
1824 1.52147576955031e-05
1825 1.52074262587121e-05
1826 1.52000793605112e-05
1827 1.51927652041195e-05
1828 1.5185456504696e-05
1829 1.51781614476931e-05
1830 1.51708964040154e-05
1831 1.51636104419595e-05
1832 1.51563717736281e-05
1833 1.51491067299503e-05
1834 1.51418807945447e-05
1835 1.51346894199378e-05
1836 1.51275125972461e-05
1837 1.51203030327451e-05
1838 1.51131534948945e-05
1839 1.51060094140121e-05
1840 1.50988989844336e-05
1841 1.50918094732333e-05
1842 1.50847354234429e-05
1843 1.50776504597161e-05
1844 1.50706164276926e-05
1845 1.50636005855631e-05
1846 1.50565629155608e-05
1847 1.50495889101876e-05
1848 1.50426321852137e-05
1849 1.50356727317558e-05
1850 1.50287232827395e-05
1851 1.50218011185643e-05
1852 1.50149026012514e-05
1853 1.50080577441258e-05
1854 1.50011819641804e-05
1855 1.49943498399807e-05
1856 1.49875559145585e-05
1857 1.4980765627115e-05
1858 1.49739726111875e-05
1859 1.49672450788785e-05
1860 1.49605075421277e-05
1861 1.49537863762816e-05
1862 1.49470915857819e-05
1863 1.4940418623155e-05
1864 1.49337474795175e-05
1865 1.49270890688058e-05
1866 1.49204688568716e-05
1867 1.49138313645381e-05
1868 1.4907222976035e-05
1869 1.49006227729842e-05
1870 1.48940353028593e-05
1871 1.48874560181866e-05
1872 1.4880901289871e-05
1873 1.48743629324599e-05
1874 1.48678245750489e-05
1875 1.48612989505637e-05
1876 1.48547833305201e-05
1877 1.4848284990876e-05
1878 1.48417766467901e-05
1879 1.48353092299658e-05
1880 1.48288236232474e-05
1881 1.48223616633913e-05
1882 1.48158951560617e-05
1883 1.48094441101421e-05
1884 1.4802982150286e-05
1885 1.47965474752709e-05
1886 1.47901118907612e-05
1887 1.47837026815978e-05
1888 1.47772834679927e-05
1889 1.47708478834829e-05
1890 1.47644323078566e-05
1891 1.47580067277886e-05
1892 1.47515975186252e-05
1893 1.47452210512711e-05
1894 1.47388354889699e-05
1895 1.47324435602059e-05
1896 1.47260307130637e-05
1897 1.47196342368261e-05
1898 1.47132595884614e-05
1899 1.47068585647503e-05
1900 1.47004757309332e-05
1901 1.46941220009467e-05
1902 1.46877373481402e-05
1903 1.46813463288709e-05
1904 1.46749707710114e-05
1905 1.46685806612368e-05
1906 1.46622005559038e-05
1907 1.46558468259173e-05
1908 1.46494503496797e-05
1909 1.46430593304103e-05
1910 1.46366855915403e-05
1911 1.46303100336809e-05
1912 1.46239099194645e-05
1913 1.46175225381739e-05
1914 1.46111387948622e-05
1915 1.46047486850875e-05
1916 1.45983867696486e-05
1917 1.45919866554323e-05
1918 1.45856038216152e-05
1919 1.45792264447664e-05
1920 1.45728236020659e-05
1921 1.45664553201641e-05
1922 1.4560036106559e-05
1923 1.45536569107207e-05
1924 1.45472722579143e-05
1925 1.45408503158251e-05
1926 1.45344647535239e-05
1927 1.45280828292016e-05
1928 1.45216936289216e-05
1929 1.45152989716735e-05
1930 1.4508896128973e-05
1931 1.45024860103149e-05
1932 1.44960822581197e-05
1933 1.44897003337974e-05
1934 1.44833029480651e-05
1935 1.44769210237428e-05
1936 1.44705200000317e-05
1937 1.44641162478365e-05
1938 1.44577215905883e-05
1939 1.44513123814249e-05
1940 1.44449313665973e-05
1941 1.44385476232856e-05
1942 1.44321247717016e-05
1943 1.44257264764747e-05
1944 1.44193345477106e-05
1945 1.44129317050101e-05
1946 1.44065279528149e-05
1947 1.44001223816304e-05
1948 1.43937468237709e-05
1949 1.43873267006711e-05
1950 1.43809111250448e-05
1951 1.43745401146589e-05
1952 1.43681190820644e-05
1953 1.43617171488586e-05
1954 1.43553224916104e-05
1955 1.43489414767828e-05
1956 1.43425377245876e-05
1957 1.43361257869401e-05
1958 1.4329733858176e-05
1959 1.43233282869915e-05
1960 1.43169390867115e-05
1961 1.43105680763256e-05
1962 1.43041852425085e-05
1963 1.42977924042498e-05
1964 1.42914041134645e-05
1965 1.4285007637227e-05
1966 1.42786420838092e-05
1967 1.42722747114021e-05
1968 1.42659027915215e-05
1969 1.42595290526515e-05
1970 1.42531953315483e-05
1971 1.42468525154982e-05
1972 1.42405115184374e-05
1973 1.42341486935038e-05
1974 1.42278177008848e-05
1975 1.42214976222022e-05
1976 1.42151784530142e-05
1977 1.42088438224164e-05
1978 1.42025255627232e-05
1979 1.41961936606094e-05
1980 1.41898890433367e-05
1981 1.41836017064634e-05
1982 1.41773089126218e-05
1983 1.41710270327167e-05
1984 1.41647178679705e-05
1985 1.41584259836236e-05
1986 1.41521604746231e-05
1987 1.41459249789477e-05
1988 1.41396503750002e-05
1989 1.41333939609467e-05
1990 1.41271511893137e-05
1991 1.41209011417232e-05
1992 1.41146701935213e-05
1993 1.41084392453195e-05
1994 1.41022028401494e-05
1995 1.40959664349793e-05
1996 1.40897609526291e-05
1997 1.40835445563425e-05
1998 1.40773299790453e-05
1999 1.40711154017481e-05
};
\addlegendentry{Test}

\nextgroupplot[
title={4 Layer},
ymin=4.34475888215787e-06, ymax=0.001,
]
\addplot [semithick, black, dashed]
table {%
0 0.0900730829065045
1 0.0858608457880716
2 0.0816006909590214
3 0.0768245672807097
4 0.0690722035554548
5 0.0512649899659057
6 0.0362110335069398
7 0.0279414476438736
8 0.0228160245654484
9 0.0191741410332421
10 0.0163463078788482
11 0.0140244579136682
12 0.0120526567916386
13 0.0103501700408136
14 0.0088795002666302
15 0.00762679703378429
16 0.00658223129721591
17 0.00572427062676676
18 0.00501912633505223
19 0.00443056004587561
20 0.00392979711129253
21 0.00349921592472432
22 0.00313005159872167
23 0.00281746068261176
24 0.00255511725542116
25 0.00233224533394605
26 0.00213603427823728
27 0.00195827052774196
28 0.00180017993860323
29 0.00166757989783643
30 0.0015577555782329
31 0.00146041289474397
32 0.00137021587336979
33 0.00128769337989828
34 0.00121496889884535
35 0.00115309797532367
36 0.00110132372140015
37 0.00105765175730236
38 0.00101989708722764
39 0.000986328655888731
40 0.000955760798963941
41 0.000927417664911445
42 0.000900783864002127
43 0.000875488108003234
44 0.000851253522720678
45 0.000827984700568625
46 0.000805739258623817
47 0.000784475359125736
48 0.000764079698437096
49 0.000744453976324166
50 0.000725534888222986
51 0.000707294820661749
52 0.000689726077988932
53 0.000672820938575569
54 0.000656561742346184
55 0.000640922702075386
56 0.000625873009672281
57 0.000611397141218125
58 0.000597520884525693
59 0.000584277647002788
60 0.000571649236784803
61 0.000559612986288963
62 0.000548166134649364
63 0.00053729408335812
64 0.000526961592375604
65 0.000517127685043306
66 0.000507756295595148
67 0.000498816577968834
68 0.000490281693032557
69 0.000482126180325319
70 0.000474326497823085
71 0.000466860545770942
72 0.000459709268843274
73 0.000452855298931354
74 0.000446284424166758
75 0.000439986161495654
76 0.000433951673935932
77 0.000428173394472727
78 0.000422643736300188
79 0.000417354536693892
80 0.000412296338907936
81 0.000407458669580289
82 0.000402830989495821
83 0.000398402552131453
84 0.000394162831630259
85 0.000390101464489589
86 0.000386208396169726
87 0.000382473350839518
88 0.00037888693106917
89 0.000375439717269425
90 0.000372122054869806
91 0.000368925481135799
92 0.000365841819444768
93 0.000362863002645023
94 0.000359982533855903
95 0.000357194233207755
96 0.000354492291895288
97 0.000351871985927232
98 0.000349327976474948
99 0.000346856504004715
100 0.00034445337729494
101 0.000342114089226205
102 0.000339835100831465
103 0.000337612969029048
104 0.000335444907226664
105 0.00033332737457196
106 0.000331257936778684
107 0.000329234559918253
108 0.000327255896640584
109 0.000325320666050288
110 0.000323428301622168
111 0.000321578140533537
112 0.000319768662156624
113 0.000317998191699568
114 0.000316265514787991
115 0.000314568969509802
116 0.000312906970904692
117 0.000311277781207764
118 0.00030968068988102
119 0.000308114419577047
120 0.000306578149334769
121 0.000305071410091765
122 0.000303593042678093
123 0.000302142550220689
124 0.000300719653651527
125 0.00029932371830436
126 0.000297954125424364
127 0.000296610418824154
128 0.00029529192080228
129 0.000293998028245331
130 0.000292728403546031
131 0.000291482373199869
132 0.000290259600674858
133 0.000289059655415258
134 0.000287882236132001
135 0.000286726884539235
136 0.000285593506262671
137 0.000284481502840587
138 0.000283390435114939
139 0.000282320748780762
140 0.000281271564214573
141 0.000280242973014803
142 0.000279234386511007
143 0.000278245811505447
144 0.000277277106093266
145 0.000276328120525212
146 0.000275398644641693
147 0.000274488446554718
148 0.000273597079247641
149 0.000272724781988624
150 0.000271871145253992
151 0.000271036162899918
152 0.000270219258032019
153 0.000269420403052815
154 0.000268639379361465
155 0.000267876006674328
156 0.000267129862701646
157 0.000266401186266307
158 0.000265689172349444
159 0.000264993900943959
160 0.000264314978352331
161 0.00026365188026034
162 0.0002630047878777
163 0.000262373139264582
164 0.0002617565031476
165 0.000261154727165073
166 0.000260567171454322
167 0.000259994087943009
168 0.000259434500423576
169 0.000258888502417657
170 0.000258355244525887
171 0.000257834811975499
172 0.00025732679687233
173 0.000256830593528434
174 0.000256345856612938
175 0.000255872146122253
176 0.000255409325435589
177 0.000254956546524928
178 0.000254513758761732
179 0.000254080408945849
180 0.000253656053674926
181 0.000253240097914424
182 0.000252832158587069
183 0.000252431753523297
184 0.000252038118989143
185 0.00025165067006346
186 0.00025126891844233
187 0.000250891605882468
188 0.00025051814275893
189 0.000250147330082958
190 0.000249778450635082
191 0.000249409777614081
192 0.000249040680910658
193 0.000248670197890988
194 0.00024829841248201
195 0.000247926514184371
196 0.000247556212613631
197 0.000247190082587186
198 0.000246829607135624
199 0.00024647473369536
200 0.000246124792901981
201 0.000245778449882531
202 0.000245434986891742
203 0.000245093250671384
204 0.000244752166750573
205 0.000244410781770436
206 0.000244067026014723
207 0.000243719603564803
208 0.000243365683308146
209 0.000243002079758507
210 0.000242624182045385
211 0.000242226792884992
212 0.000241802879891395
213 0.000241345664387419
214 0.000240849973150622
215 0.000240318356010979
216 0.000239764889919059
217 0.000239214891675961
218 0.00023869567222808
219 0.000238223658286074
220 0.000237798835546717
221 0.000237411870860645
222 0.00023705109378606
223 0.00023670810705075
224 0.000236376905907794
225 0.000236054834950039
226 0.000235739808455075
227 0.000235431319876985
228 0.000235129167180048
229 0.000234833029836068
230 0.000234543721814381
231 0.000234261395879306
232 0.000233986798084137
233 0.000233720907696503
234 0.000233464770621102
235 0.000233219841698921
236 0.000232988486885688
237 0.000232773215652552
238 0.000232576962673647
239 0.000232403907650299
240 0.000232259351745275
241 0.000232148762272288
242 0.000232076667415981
243 0.000232045370677743
244 0.000232047599477407
245 0.000232057130266602
246 0.000232021607558863
247 0.000231865334541226
248 0.00023152660127342
249 0.000231007016478676
250 0.000230379037731154
251 0.000229733577796765
252 0.000229126518066873
253 0.000228570513542081
254 0.000228058259918858
255 0.000227577374495525
256 0.000227117250616971
257 0.000226669843601239
258 0.000226229475041843
259 0.000225791113685148
260 0.000225350507726318
261 0.000224903106908888
262 0.000224444694758574
263 0.000223969117080003
264 0.00022347003527301
265 0.000222938565334137
266 0.000222364201019805
267 0.000221734502927025
268 0.000221036712706753
269 0.000220261408060196
270 0.000219404359360927
271 0.000218471353903738
272 0.000217474376637957
273 0.000216428836426511
274 0.000215348131727448
275 0.000214241019662609
276 0.000213113393077909
277 0.000211967762176357
278 0.000210806696332592
279 0.000209631241891846
280 0.000208441985880844
281 0.00020723974716456
282 0.000206023698145448
283 0.000204793586007668
284 0.000203547224401746
285 0.000202281533631776
286 0.00020099049928272
287 0.000199665194638025
288 0.000198290419488008
289 0.00019684060651078
290 0.000195269989146614
291 0.000193494089446726
292 0.000191355168666973
293 0.000188603672808559
294 0.000185059031930261
295 0.000180971110078569
296 0.000176858656123121
297 0.000172984283717407
298 0.000169377631773197
299 0.000166009356561858
300 0.000162849000564809
301 0.000159872360773268
302 0.00015706020221747
303 0.000154397582567375
304 0.00015187367361591
305 0.000149480481762036
306 0.00014721254292264
307 0.000145065246272225
308 0.000143034346739531
309 0.000141114253575362
310 0.000139298969671131
311 0.000137581691390665
312 0.000135955800814713
313 0.000134415047710945
314 0.000132953282450406
315 0.000131564454927494
316 0.000130242962574319
317 0.000128983238894875
318 0.000127779885365461
319 0.000126627998502234
320 0.000125523246159294
321 0.000124461569100731
322 0.000123439189276079
323 0.000122452768091345
324 0.000121498945271507
325 0.00012057421544398
326 0.000119675775929788
327 0.000118801157971878
328 0.000117949098021578
329 0.000117119180117697
330 0.00011631166294516
331 0.00011552730848526
332 0.000114766913213771
333 0.000114030895036403
334 0.000113319392634518
335 0.000112632058467454
336 0.000111968795664552
337 0.000111328936367746
338 0.000110711920856469
339 0.000110116862245491
340 0.000109542615798356
341 0.000108987753487592
342 0.000108450841582434
343 0.000107930201555935
344 0.000107424325593778
345 0.000106931986816032
346 0.000106451929302883
347 0.000105983136066357
348 0.000105524930956354
349 0.000105076558580682
350 0.000104637690263587
351 0.000104207903935579
352 0.000103787019592498
353 0.000103374798624619
354 0.000102971313637568
355 0.000102576621086333
356 0.000102190608001725
357 0.000101813249808439
358 0.000101444525438884
359 0.00010108402086928
360 0.000100731533573632
361 0.000100386432393407
362 0.000100048354191529
363 9.97167277626924e-05
364 9.93909963042938e-05
365 9.90707354399471e-05
366 9.87554679241498e-05
367 9.84447421762032e-05
368 9.81386234926163e-05
369 9.78366383748153e-05
370 9.7538848862655e-05
371 9.72451174969062e-05
372 9.69556807130327e-05
373 9.66702694735015e-05
374 9.63891771543028e-05
375 9.6112202013406e-05
376 9.58393408675799e-05
377 9.55705268168572e-05
378 9.53057195829388e-05
379 9.50446751772915e-05
380 9.47876455086316e-05
381 9.45341399850956e-05
382 9.42840675103677e-05
383 9.40372146137255e-05
384 9.37937207462862e-05
385 9.3552836775738e-05
386 9.33150170752128e-05
387 9.30793057207779e-05
388 9.28457472871476e-05
389 9.26140808251337e-05
390 9.23841793204853e-05
391 9.21557673265738e-05
392 9.19284638219627e-05
393 9.17025846950044e-05
394 9.14776609709141e-05
395 9.12541119385916e-05
396 9.10318740565685e-05
397 9.0810692320531e-05
398 9.05910540846359e-05
399 9.03729413958369e-05
400 9.01563341031419e-05
401 8.99411700648519e-05
402 8.97277937639274e-05
403 8.95158071259061e-05
404 8.93054370602423e-05
405 8.9096777666479e-05
406 8.88897203831599e-05
407 8.86842427618717e-05
408 8.84805204440416e-05
409 8.82781104891706e-05
410 8.8077461822896e-05
411 8.787829424269e-05
412 8.76808952886184e-05
413 8.74849161955164e-05
414 8.72905552379658e-05
415 8.70976817471577e-05
416 8.69064385066546e-05
417 8.67167519160716e-05
418 8.65286518939949e-05
419 8.63420805998771e-05
420 8.61571150139184e-05
421 8.59735848024457e-05
422 8.57916946609786e-05
423 8.56114898842482e-05
424 8.54325856304664e-05
425 8.52554709117233e-05
426 8.5079871543788e-05
427 8.49059586893001e-05
428 8.47335278611657e-05
429 8.45627549613444e-05
430 8.43934713567289e-05
431 8.42258225925245e-05
432 8.40596029073974e-05
433 8.38951736170657e-05
434 8.37324004857957e-05
435 8.35710032873749e-05
436 8.34114021301957e-05
437 8.32533306459974e-05
438 8.30967942514841e-05
439 8.29419317488108e-05
440 8.27885947316531e-05
441 8.26367473104976e-05
442 8.24863863092181e-05
443 8.23377971504631e-05
444 8.21907017917548e-05
445 8.20451649280092e-05
446 8.19010923862606e-05
447 8.17585217992208e-05
448 8.16175396707308e-05
449 8.14778971166182e-05
450 8.13398988815569e-05
451 8.12031857506668e-05
452 8.10680756468211e-05
453 8.09343000085505e-05
454 8.08021861378923e-05
455 8.06712261057404e-05
456 8.05417376786484e-05
457 8.0413546314162e-05
458 8.02868607839476e-05
459 8.01614983743093e-05
460 8.00375200284975e-05
461 7.99147667507327e-05
462 7.97934091565367e-05
463 7.96733857579094e-05
464 7.95545461945342e-05
465 7.94369946035545e-05
466 7.9320748308002e-05
467 7.92057341101327e-05
468 7.90919288154631e-05
469 7.89793797816666e-05
470 7.8868102778055e-05
471 7.87580749251523e-05
472 7.86492495308266e-05
473 7.85414827362274e-05
474 7.84349855275461e-05
475 7.83295993021701e-05
476 7.82256135707371e-05
477 7.81226659777682e-05
478 7.80209171651336e-05
479 7.79203969661542e-05
480 7.78210253926659e-05
481 7.7722689619956e-05
482 7.76255647186493e-05
483 7.75294297573244e-05
484 7.74345051060739e-05
485 7.73406690157685e-05
486 7.72475979999854e-05
487 7.71555319497945e-05
488 7.7064424331752e-05
489 7.69740923563707e-05
490 7.68846941241463e-05
491 7.67960352542711e-05
492 7.67080157899613e-05
493 7.66207569995458e-05
494 7.65341306797988e-05
495 7.64483539702117e-05
496 7.63631636786499e-05
497 7.6278728130319e-05
498 7.61951858478938e-05
499 7.61123760379216e-05
500 7.60302908820639e-05
501 7.59489008975341e-05
502 7.58682360147607e-05
503 7.57885435310375e-05
504 7.5709458111343e-05
505 7.56310919882708e-05
506 7.5553778541367e-05
507 7.54770483174146e-05
508 7.54011870801226e-05
509 7.53259001532077e-05
510 7.52513498483154e-05
511 7.51776931503893e-05
512 7.51045714579845e-05
513 7.50321714117301e-05
514 7.49606029278501e-05
515 7.48895893257403e-05
516 7.48192636166323e-05
517 7.47495840739039e-05
518 7.46805526290473e-05
519 7.46121214175351e-05
520 7.45442247366649e-05
521 7.44770720354874e-05
522 7.44105855628637e-05
523 7.4344523213199e-05
524 7.42790872389302e-05
525 7.42142190475192e-05
526 7.41498940731352e-05
527 7.40861461553758e-05
528 7.40230251755255e-05
529 7.39601405529792e-05
530 7.3898074751592e-05
531 7.38364075279681e-05
532 7.37752911857588e-05
533 7.37146694997174e-05
534 7.36543935021909e-05
535 7.35948229184658e-05
536 7.35356217790676e-05
537 7.3476814732724e-05
538 7.34186348706108e-05
539 7.33608288356891e-05
540 7.33034544625847e-05
541 7.32465694272359e-05
542 7.31900826428008e-05
543 7.31339104819521e-05
544 7.30783311150655e-05
545 7.3023227205482e-05
546 7.29682703261384e-05
547 7.29140567112078e-05
548 7.28599470001257e-05
549 7.28064211479307e-05
550 7.2753237740623e-05
551 7.27005395842658e-05
552 7.26480898502283e-05
553 7.25961598770179e-05
554 7.25446664932387e-05
555 7.24933925345302e-05
556 7.24427376634177e-05
557 7.23924185000631e-05
558 7.23423837219646e-05
559 7.229289446542e-05
560 7.22437234396504e-05
561 7.21949902136032e-05
562 7.21467209565674e-05
563 7.20987997115212e-05
564 7.20513665655176e-05
565 7.20043750526145e-05
566 7.19579153560801e-05
567 7.19119067026005e-05
568 7.18662674152881e-05
569 7.18214219948739e-05
570 7.17769397766688e-05
571 7.17332920648535e-05
572 7.16899641209541e-05
573 7.1647653418457e-05
574 7.16057654900718e-05
575 7.15647918596574e-05
576 7.15246725704333e-05
577 7.14856855156635e-05
578 7.14476016838717e-05
579 7.14108272129958e-05
580 7.13755692570335e-05
581 7.13418063481204e-05
582 7.13099635708166e-05
583 7.12798772894985e-05
584 7.12520914187091e-05
585 7.12264915160669e-05
586 7.12033450239365e-05
587 7.11825731751029e-05
588 7.1163653932634e-05
589 7.11457762439712e-05
590 7.11275899606771e-05
591 7.11069927703534e-05
592 7.10818550828662e-05
593 7.10489167670876e-05
594 7.10038539916506e-05
595 7.09415135501009e-05
596 7.08613302447721e-05
597 7.07695138828512e-05
598 7.06782927638964e-05
599 7.05972977795473e-05
600 7.05301213604533e-05
601 7.04753165147546e-05
602 7.04298488803564e-05
603 7.03904342339949e-05
604 7.03544247500076e-05
605 7.03200940061303e-05
606 7.02865479655183e-05
607 7.02532576219994e-05
608 7.02195403429566e-05
609 7.01858311866014e-05
610 7.01519636336911e-05
611 7.01176668336245e-05
612 7.00832631549038e-05
613 7.00488227505976e-05
614 7.00142249030478e-05
615 6.99792824659558e-05
616 6.99443626880717e-05
617 6.99092413221081e-05
618 6.98739972063341e-05
619 6.98385987855469e-05
620 6.9802989671075e-05
621 6.97673772632375e-05
622 6.97316068034581e-05
623 6.96958216899191e-05
624 6.96598286360484e-05
625 6.96238604251202e-05
626 6.95877751285915e-05
627 6.9551591954801e-05
628 6.95153302447219e-05
629 6.94791544297857e-05
630 6.94428135806409e-05
631 6.94066228893083e-05
632 6.93703209175093e-05
633 6.93340071684645e-05
634 6.92978463483485e-05
635 6.92616938048711e-05
636 6.92255683164926e-05
637 6.91892577660743e-05
638 6.91532151850064e-05
639 6.91172119976121e-05
640 6.90812661604904e-05
641 6.90453781887849e-05
642 6.90094781212736e-05
643 6.89735703206888e-05
644 6.89378161726021e-05
645 6.89020772502621e-05
646 6.88663002215151e-05
647 6.88305497741718e-05
648 6.87947795832144e-05
649 6.87592097155705e-05
650 6.87234119342387e-05
651 6.86876923883991e-05
652 6.86520132108607e-05
653 6.86163608089411e-05
654 6.85807037340188e-05
655 6.85450348818506e-05
656 6.85093546183661e-05
657 6.84736342080328e-05
658 6.84377518780366e-05
659 6.84020251450572e-05
660 6.83661792895881e-05
661 6.83302725441592e-05
662 6.82945184398894e-05
663 6.82585030844507e-05
664 6.8222453759148e-05
665 6.818637470829e-05
666 6.81502712585787e-05
667 6.81139996932719e-05
668 6.80776672510319e-05
669 6.80412697822893e-05
670 6.80047434222786e-05
671 6.79681981440676e-05
672 6.79315105666471e-05
673 6.78948128793877e-05
674 6.78579016444303e-05
675 6.78210224987671e-05
676 6.77840803362528e-05
677 6.77469274942647e-05
678 6.77096346729892e-05
679 6.76722056501452e-05
680 6.76347411439811e-05
681 6.75970921382903e-05
682 6.75594576714881e-05
683 6.75215572402503e-05
684 6.74835368078182e-05
685 6.74453092912586e-05
686 6.74069815242243e-05
687 6.73685338199448e-05
688 6.73298984483021e-05
689 6.72913017349212e-05
690 6.7252333604273e-05
691 6.72134431087083e-05
692 6.71741842784475e-05
693 6.71349329029643e-05
694 6.70954204906593e-05
695 6.70558398745413e-05
696 6.70162305394702e-05
697 6.69762821582746e-05
698 6.693616822299e-05
699 6.68959297319323e-05
700 6.68554988744556e-05
701 6.68150481170452e-05
702 6.67742423014299e-05
703 6.67333345205634e-05
704 6.66922643806818e-05
705 6.66509449587238e-05
706 6.66095580162581e-05
707 6.65680616004731e-05
708 6.65261940350585e-05
709 6.64843230708811e-05
710 6.64422967891672e-05
711 6.64000522760944e-05
712 6.63576100545053e-05
713 6.6315027685467e-05
714 6.62722438823001e-05
715 6.62292706081757e-05
716 6.61860407570695e-05
717 6.6142739574578e-05
718 6.6099140354936e-05
719 6.60553290596037e-05
720 6.60114796578644e-05
721 6.59673276928174e-05
722 6.59229648339495e-05
723 6.58784214214355e-05
724 6.58336921759428e-05
725 6.57887358741505e-05
726 6.57435749005231e-05
727 6.56982268433618e-05
728 6.56526751328101e-05
729 6.56067960089028e-05
730 6.55607463547388e-05
731 6.55146221599049e-05
732 6.54681150725385e-05
733 6.54214335185088e-05
734 6.53744111502874e-05
735 6.53271722512727e-05
736 6.52797864191257e-05
737 6.52321478350852e-05
738 6.51842646846035e-05
739 6.5136049137493e-05
740 6.50876819605628e-05
741 6.50391039253388e-05
742 6.49902329191339e-05
743 6.49409187663726e-05
744 6.48914352225916e-05
745 6.48417316876741e-05
746 6.47919349532344e-05
747 6.47415284191766e-05
748 6.46910406156091e-05
749 6.46402019105115e-05
750 6.45890416185087e-05
751 6.45377665667487e-05
752 6.44860943784427e-05
753 6.44340057102492e-05
754 6.43817467296988e-05
755 6.43291853184754e-05
756 6.42762804853684e-05
757 6.42231186418485e-05
758 6.41695717054821e-05
759 6.41156332955954e-05
760 6.40615366691577e-05
761 6.40071055547272e-05
762 6.39523351964044e-05
763 6.38972082545782e-05
764 6.38417014731848e-05
765 6.37858446026485e-05
766 6.37297520261389e-05
767 6.36730697110011e-05
768 6.36162870174909e-05
769 6.35590649930388e-05
770 6.35014781214712e-05
771 6.34433636127104e-05
772 6.33850682518755e-05
773 6.33264085744637e-05
774 6.32671299953813e-05
775 6.32076553846635e-05
776 6.31477394297993e-05
777 6.308741022328e-05
778 6.30266332990459e-05
779 6.29653555629754e-05
780 6.29038060135182e-05
781 6.28416644919601e-05
782 6.27791274183664e-05
783 6.27161546304933e-05
784 6.26526277495524e-05
785 6.25887516913319e-05
786 6.25242712075647e-05
787 6.24592260128528e-05
788 6.23937921607383e-05
789 6.23277706021478e-05
790 6.22614250384383e-05
791 6.21943782519452e-05
792 6.21267287120493e-05
793 6.20586910452895e-05
794 6.19900115002755e-05
795 6.19206887459238e-05
796 6.18510970558835e-05
797 6.17807757260872e-05
798 6.17100000918451e-05
799 6.16385501335988e-05
800 6.15667301886447e-05
801 6.14941668265108e-05
802 6.14211521442106e-05
803 6.13476196065221e-05
804 6.12734489801407e-05
805 6.11987889366598e-05
806 6.11236874495565e-05
807 6.10480402751061e-05
808 6.09718136743709e-05
809 6.08950781900338e-05
810 6.08178565641992e-05
811 6.0740177777537e-05
812 6.06619441351578e-05
813 6.05832521192914e-05
814 6.05040267700474e-05
815 6.04243772244217e-05
816 6.03442239205757e-05
817 6.02637182514817e-05
818 6.01827340697506e-05
819 6.01012344934778e-05
820 6.00195004286282e-05
821 5.99372255081221e-05
822 5.98546897544831e-05
823 5.97717292289227e-05
824 5.96885200465636e-05
825 5.96048956443459e-05
826 5.95210323647658e-05
827 5.94368212247787e-05
828 5.93523997736156e-05
829 5.9267734336288e-05
830 5.91826452639073e-05
831 5.90974543565418e-05
832 5.90120661527749e-05
833 5.89263019416345e-05
834 5.88402758386527e-05
835 5.87539643509179e-05
836 5.8667590577007e-05
837 5.85806847036944e-05
838 5.84935088336162e-05
839 5.84060672229233e-05
840 5.83183144584609e-05
841 5.82302020291081e-05
842 5.81418805722933e-05
843 5.80529572786759e-05
844 5.79640457871733e-05
845 5.78739709548396e-05
846 5.77846779042564e-05
847 5.76941833851189e-05
848 5.7603568793733e-05
849 5.75115191760271e-05
850 5.74208003432849e-05
851 5.73311189100423e-05
852 5.7226424206173e-05
853 5.71494269342073e-05
854 5.70523417451341e-05
855 5.69584562223705e-05
856 5.68634555572582e-05
857 5.67690546446897e-05
858 5.66731345633305e-05
859 5.65793945099813e-05
860 5.64812591813772e-05
861 5.63869110905557e-05
862 5.62881576252986e-05
863 5.61952936640372e-05
864 5.60939652724339e-05
865 5.60027189493439e-05
866 5.58988258276827e-05
867 5.58063614259178e-05
868 5.57031980600679e-05
869 5.5607647678831e-05
870 5.55055704201853e-05
871 5.54079018186826e-05
872 5.5306505863939e-05
873 5.52079520256636e-05
874 5.51069574138789e-05
875 5.50075949282321e-05
876 5.49065289708513e-05
877 5.48066829964e-05
878 5.47055243652987e-05
879 5.4605532637216e-05
880 5.45042772894059e-05
881 5.44040946053315e-05
882 5.43029863398677e-05
883 5.42028468710024e-05
884 5.41016951795351e-05
885 5.40015019225848e-05
886 5.3900388429895e-05
887 5.38004055788595e-05
888 5.36994133355506e-05
889 5.35997079088209e-05
890 5.34989041061825e-05
891 5.33995287597122e-05
892 5.32990876157176e-05
893 5.32000749302597e-05
894 5.30998759842305e-05
895 5.30012270161251e-05
896 5.2901406644897e-05
897 5.28031637519651e-05
898 5.27037322110383e-05
899 5.26060023560149e-05
900 5.25070670983988e-05
901 5.24097607339513e-05
902 5.23113218164895e-05
903 5.22145449342778e-05
904 5.21167101439346e-05
905 5.20204179963457e-05
906 5.19230850244412e-05
907 5.18272697978735e-05
908 5.17306390790395e-05
909 5.16354224586735e-05
910 5.15393224205241e-05
911 5.14445748625766e-05
912 5.13490759412131e-05
913 5.12549632768848e-05
914 5.116010531007e-05
915 5.10664268939346e-05
916 5.09722731318144e-05
917 5.0879160279275e-05
918 5.07855695322708e-05
919 5.06930072390332e-05
920 5.06000500687757e-05
921 5.05079306141928e-05
922 5.04155993397148e-05
923 5.03240767955049e-05
924 5.02322042450487e-05
925 5.01411777070378e-05
926 5.00498731123381e-05
927 4.99592265891617e-05
928 4.98684905115473e-05
929 4.97783457286497e-05
930 4.96881153987753e-05
931 4.95983968491487e-05
932 4.95085858839654e-05
933 4.94192504163493e-05
934 4.93299577849863e-05
935 4.92410329068586e-05
936 4.91521371751939e-05
937 4.90636387150782e-05
938 4.89751562658588e-05
939 4.88870892591346e-05
940 4.87990504443777e-05
941 4.87112934261328e-05
942 4.86236935823096e-05
943 4.85364012613824e-05
944 4.844928927857e-05
945 4.83623636190108e-05
946 4.82756466197998e-05
947 4.81891961309581e-05
948 4.81029656024627e-05
949 4.80170819135613e-05
950 4.79313428248436e-05
951 4.78457726463451e-05
952 4.77605943463762e-05
953 4.7675640902393e-05
954 4.7590989242489e-05
955 4.75066641172361e-05
956 4.74226760260876e-05
957 4.73389400781343e-05
958 4.7255624932537e-05
959 4.71727221006783e-05
960 4.70900885810958e-05
961 4.70078999252147e-05
962 4.6926088212255e-05
963 4.68446476631357e-05
964 4.67636170308576e-05
965 4.6682949474075e-05
966 4.66025074127666e-05
967 4.6522157762278e-05
968 4.64418685955555e-05
969 4.63612033290891e-05
970 4.62798606667055e-05
971 4.61977591399432e-05
972 4.61147006115918e-05
973 4.60305399035595e-05
974 4.59453325163395e-05
975 4.58592897582359e-05
976 4.57728012482524e-05
977 4.56861252070932e-05
978 4.55996055516531e-05
979 4.55133081255591e-05
980 4.54274154956617e-05
981 4.53421493110303e-05
982 4.5257455710157e-05
983 4.5173115057177e-05
984 4.50892428294954e-05
985 4.50059211634842e-05
986 4.49227532447101e-05
987 4.48399225945195e-05
988 4.47572130160741e-05
989 4.46745360775935e-05
990 4.45921287450801e-05
991 4.45098275747569e-05
992 4.44274801765232e-05
993 4.43450920878755e-05
994 4.42627520224429e-05
995 4.4180258741496e-05
996 4.40977983847309e-05
997 4.40152607801281e-05
998 4.39326890706582e-05
999 4.38499782132359e-05
1000 4.37671225519409e-05
1001 4.36843097801898e-05
1002 4.36014650577003e-05
1003 4.35184054386658e-05
1004 4.34352650842129e-05
1005 4.33519567918002e-05
1006 4.32684390171782e-05
1007 4.31849072922338e-05
1008 4.31011573726418e-05
1009 4.30173681304306e-05
1010 4.29333611495034e-05
1011 4.28491488785691e-05
1012 4.27647560753049e-05
1013 4.26801704023205e-05
1014 4.25952992794502e-05
1015 4.2510304851362e-05
1016 4.2425080414669e-05
1017 4.23394677871632e-05
1018 4.22536092893703e-05
1019 4.21674963414868e-05
1020 4.20810338196039e-05
1021 4.19941642822626e-05
1022 4.1906962583956e-05
1023 4.1819507403081e-05
1024 4.17316577738802e-05
1025 4.16433200847166e-05
1026 4.15545688561281e-05
1027 4.14653136703672e-05
1028 4.13757025311175e-05
1029 4.12855334831856e-05
1030 4.1194841485274e-05
1031 4.11036336155727e-05
1032 4.10117118008202e-05
1033 4.0919406032458e-05
1034 4.08263949402965e-05
1035 4.07326419219108e-05
1036 4.06383150810541e-05
1037 4.05432592742727e-05
1038 4.04480807875511e-05
1039 4.03529794894553e-05
1040 4.02652007416293e-05
1041 4.01482217391447e-05
1042 4.00519719481925e-05
1043 3.99540267927989e-05
1044 3.98564942756252e-05
1045 3.97578979149197e-05
1046 3.96580812325927e-05
1047 3.95571722139702e-05
1048 3.94552972219723e-05
1049 3.93526851828104e-05
1050 3.92494933691978e-05
1051 3.91460233082341e-05
1052 3.90421963795499e-05
1053 3.89384106362911e-05
1054 3.88345488578542e-05
1055 3.87304726968315e-05
1056 3.86259773617799e-05
1057 3.85206335119885e-05
1058 3.84135566425433e-05
1059 3.83031011222575e-05
1060 3.8187058159167e-05
1061 3.80633620954048e-05
1062 3.79300612228424e-05
1063 3.77869002576858e-05
1064 3.76355319022063e-05
1065 3.74793158333849e-05
1066 3.7322147958226e-05
1067 3.71667625861013e-05
1068 3.70148220388463e-05
1069 3.68664762149251e-05
1070 3.67209423709861e-05
1071 3.65774500300139e-05
1072 3.64346424177597e-05
1073 3.62917680523841e-05
1074 3.6147851131337e-05
1075 3.60025186694202e-05
1076 3.58554929249522e-05
1077 3.57064399783743e-05
1078 3.5555396948439e-05
1079 3.54025353270041e-05
1080 3.52478651185114e-05
1081 3.50913747126924e-05
1082 3.49333878209753e-05
1083 3.47739335495589e-05
1084 3.46132776556374e-05
1085 3.44511141250337e-05
1086 3.42880379970249e-05
1087 3.41238947948833e-05
1088 3.39586861836475e-05
1089 3.37924722776014e-05
1090 3.36256182791317e-05
1091 3.34580712646471e-05
1092 3.32898399406645e-05
1093 3.31211199077567e-05
1094 3.29520648397382e-05
1095 3.27825730982075e-05
1096 3.26128238038829e-05
1097 3.24431189208478e-05
1098 3.2273112097414e-05
1099 3.21034444018172e-05
1100 3.1933813322856e-05
1101 3.17646829917824e-05
1102 3.15957969405645e-05
1103 3.14273694617858e-05
1104 3.12595902182503e-05
1105 3.10926459950641e-05
1106 3.09265074018109e-05
1107 3.0761460635631e-05
1108 3.05975423700033e-05
1109 3.04348483245083e-05
1110 3.02737479979479e-05
1111 3.01140340278986e-05
1112 2.99559235135879e-05
1113 2.97993640098563e-05
1114 2.96441425599407e-05
1115 2.94898893429263e-05
1116 2.93357858917413e-05
1117 2.91807505945485e-05
1118 2.90232381487954e-05
1119 2.88613723071762e-05
1120 2.86935789048925e-05
1121 2.85189001066518e-05
1122 2.83378381359019e-05
1123 2.81521554962675e-05
1124 2.79648456166607e-05
1125 2.7778807603814e-05
1126 2.75966224307922e-05
1127 2.74196419098871e-05
1128 2.72484843767036e-05
1129 2.70829398706004e-05
1130 2.69222821197938e-05
1131 2.67660745928803e-05
1132 2.66134297982982e-05
1133 2.6463877932296e-05
1134 2.63168206326251e-05
1135 2.61716226788167e-05
1136 2.60283275537176e-05
1137 2.58864278303387e-05
1138 2.57458976496612e-05
1139 2.56065290855645e-05
1140 2.54683130691546e-05
1141 2.53310478986653e-05
1142 2.51947889206387e-05
1143 2.50594061128595e-05
1144 2.49248517481722e-05
1145 2.47913994731637e-05
1146 2.46589275908112e-05
1147 2.45272681998661e-05
1148 2.43965558344901e-05
1149 2.42669198679361e-05
1150 2.4138251824013e-05
1151 2.40106475525674e-05
1152 2.38840744775833e-05
1153 2.37585936275764e-05
1154 2.36340373310403e-05
1155 2.35106532168553e-05
1156 2.33883683335989e-05
1157 2.32670887948908e-05
1158 2.31469850694073e-05
1159 2.30279474777717e-05
1160 2.29100750056925e-05
1161 2.27931872937385e-05
1162 2.26775514633459e-05
1163 2.25629932041708e-05
1164 2.24496416763031e-05
1165 2.2337311811782e-05
1166 2.22260539904558e-05
1167 2.21159247075775e-05
1168 2.20068950819533e-05
1169 2.1898917263267e-05
1170 2.17921658508165e-05
1171 2.16864604685914e-05
1172 2.15818634075049e-05
1173 2.14783036369681e-05
1174 2.13758852254387e-05
1175 2.12745961742182e-05
1176 2.11743092487874e-05
1177 2.10750757195418e-05
1178 2.09769347065958e-05
1179 2.08799111452625e-05
1180 2.07838338160589e-05
1181 2.06889432755967e-05
1182 2.0594938144806e-05
1183 2.05021193480093e-05
1184 2.04103363898867e-05
1185 2.03194702474245e-05
1186 2.02297846954025e-05
1187 2.01411019015533e-05
1188 2.00533778193327e-05
1189 1.99666218065886e-05
1190 1.98808663990728e-05
1191 1.97962668835316e-05
1192 1.97124552840933e-05
1193 1.96296671788332e-05
1194 1.95478936646509e-05
1195 1.94670606887826e-05
1196 1.93871678237182e-05
1197 1.93083612671785e-05
1198 1.92303446920524e-05
1199 1.91534131476582e-05
1200 1.90772474783311e-05
1201 1.90020080133024e-05
1202 1.8927746836539e-05
1203 1.88542818904125e-05
1204 1.87817280696834e-05
1205 1.87101541596254e-05
1206 1.86393401418646e-05
1207 1.85694982614374e-05
1208 1.8500403974997e-05
1209 1.84320233316271e-05
1210 1.83645225237209e-05
1211 1.82978245154194e-05
1212 1.82319858777665e-05
1213 1.81669489762726e-05
1214 1.81026460441084e-05
1215 1.8039178389273e-05
1216 1.79762994392263e-05
1217 1.79143146645799e-05
1218 1.78529674824309e-05
1219 1.77923163597882e-05
1220 1.77323524906588e-05
1221 1.7673196852049e-05
1222 1.76147000570855e-05
1223 1.75568415118714e-05
1224 1.74994873442339e-05
1225 1.74430066384919e-05
1226 1.73870496501157e-05
1227 1.73315690770911e-05
1228 1.72769908436029e-05
1229 1.72228628028866e-05
1230 1.7169248861156e-05
1231 1.71162875931922e-05
1232 1.70638533028011e-05
1233 1.70120923582336e-05
1234 1.69609128543395e-05
1235 1.69100448014585e-05
1236 1.68598432992439e-05
1237 1.68101614571962e-05
1238 1.67609330148366e-05
1239 1.67121208463072e-05
1240 1.66639055313074e-05
1241 1.66161798927552e-05
1242 1.65688718804328e-05
1243 1.65220702198117e-05
1244 1.64756104273541e-05
1245 1.64296477957275e-05
1246 1.63841059167188e-05
1247 1.63391337011376e-05
1248 1.6294478031161e-05
1249 1.62502081823372e-05
1250 1.62062303807886e-05
1251 1.61627920943393e-05
1252 1.61197675154767e-05
1253 1.60769693557938e-05
1254 1.60346338636449e-05
1255 1.59926683238609e-05
1256 1.5951177912162e-05
1257 1.59099704587125e-05
1258 1.58691024492915e-05
1259 1.58285265759635e-05
1260 1.57884503728667e-05
1261 1.57485658019615e-05
1262 1.57088407964541e-05
1263 1.56694601732473e-05
1264 1.56305454801497e-05
1265 1.55920029515736e-05
1266 1.55537541471536e-05
1267 1.55160065015518e-05
1268 1.5478575469056e-05
1269 1.54415553896096e-05
1270 1.54046319356856e-05
1271 1.53679812472755e-05
1272 1.53313785844489e-05
1273 1.52948206502875e-05
1274 1.52581512840773e-05
1275 1.52216365224926e-05
1276 1.51853325801217e-05
1277 1.51499803351385e-05
1278 1.51153919745184e-05
1279 1.50820420815971e-05
1280 1.50494097385471e-05
1281 1.50178246727251e-05
1282 1.49861903556333e-05
1283 1.49535449468866e-05
1284 1.49189767834438e-05
1285 1.48815165571155e-05
1286 1.48424753826741e-05
1287 1.4802900620244e-05
1288 1.47662590492814e-05
1289 1.47331141218861e-05
1290 1.47055073658938e-05
1291 1.46840445722773e-05
1292 1.46720376127026e-05
1293 1.46714691477238e-05
1294 1.46702746886736e-05
1295 1.46448833717727e-05
1296 1.45753295678711e-05
1297 1.45094949921543e-05
1298 1.44617554695969e-05
1299 1.43857016669339e-05
1300 1.44112110724848e-05
1301 1.44303209346219e-05
1302 1.44835969043318e-05
1303 1.4426302864005e-05
1304 1.43958303742882e-05
1305 1.4344622370525e-05
1306 1.42318052347434e-05
1307 1.4156090986243e-05
1308 1.41964163979935e-05
1309 1.42118165887458e-05
1310 1.41297989960284e-05
1311 1.4169286914741e-05
1312 1.42806665029601e-05
1313 1.42603792253245e-05
1314 1.40899656102533e-05
1315 1.3952594313101e-05
1316 1.39378431830058e-05
1317 1.39641434830177e-05
1318 1.38990728875873e-05
1319 1.38680723426413e-05
1320 1.39505339890415e-05
1321 1.39799397042376e-05
1322 1.38853121948538e-05
1323 1.3767527741256e-05
1324 1.37229973222001e-05
1325 1.37115478520874e-05
1326 1.36744472847283e-05
1327 1.36559394820305e-05
1328 1.37014549833244e-05
1329 1.37324533220351e-05
1330 1.3686768258007e-05
1331 1.35931384335208e-05
1332 1.35295945954776e-05
1333 1.34990298666556e-05
1334 1.34728048782525e-05
1335 1.34529650424042e-05
1336 1.34843670759703e-05
1337 1.35236146003119e-05
1338 1.35197437221753e-05
1339 1.34346063826266e-05
1340 1.3357798325823e-05
1341 1.32990519906192e-05
1342 1.32847947090416e-05
1343 1.32453645586376e-05
1344 1.32801343752931e-05
1345 1.33034637125699e-05
1346 1.33863440131184e-05
1347 1.32960975882668e-05
1348 1.32276889613555e-05
1349 1.31119472127258e-05
1350 1.31061763291029e-05
1351 1.30572781943764e-05
1352 1.30634899839777e-05
1353 1.30910420719961e-05
1354 1.31945952380382e-05
1355 1.3192531047442e-05
1356 1.31093899694198e-05
1357 1.29698760590694e-05
1358 1.2922380643469e-05
1359 1.2891783388369e-05
1360 1.28699753692748e-05
1361 1.28765832153969e-05
1362 1.29718821521389e-05
1363 1.30269677593257e-05
1364 1.29873061176511e-05
1365 1.28406467065171e-05
1366 1.276052300625e-05
1367 1.27203207043181e-05
1368 1.27001269933184e-05
1369 1.2678562882229e-05
1370 1.27558355380586e-05
1371 1.28101889830627e-05
1372 1.28411085083731e-05
1373 1.26978917502167e-05
1374 1.26225389642798e-05
1375 1.25435858559323e-05
1376 1.25443658411939e-05
1377 1.24933835567257e-05
1378 1.25634078145254e-05
1379 1.25955411967027e-05
1380 1.26710905140233e-05
1381 1.25654668963193e-05
1382 1.24877137892554e-05
1383 1.23941807004257e-05
1384 1.23809254392408e-05
1385 1.23281971404765e-05
1386 1.23774769373123e-05
1387 1.23992963570648e-05
1388 1.24902103451063e-05
1389 1.24341686541148e-05
1390 1.23633181559057e-05
1391 1.22536003862213e-05
1392 1.22258011122274e-05
1393 1.21799180033596e-05
1394 1.21841485582289e-05
1395 1.22288131585435e-05
1396 1.23037249295521e-05
1397 1.23161395002569e-05
1398 1.22409895221646e-05
1399 1.21306244444952e-05
1400 1.20697307851951e-05
1401 1.20344183732376e-05
1402 1.20191498531597e-05
1403 1.20365613796025e-05
1404 1.2115070751643e-05
1405 1.21784809851514e-05
1406 1.21596291885832e-05
1407 1.20254590048792e-05
1408 1.19436057038532e-05
1409 1.18900951614857e-05
1410 1.18739434486296e-05
1411 1.18419528523835e-05
1412 1.19598845174096e-05
1413 1.19633482510058e-05
1414 1.20749968021509e-05
1415 1.18992812877157e-05
1416 1.18432751747832e-05
1417 1.1735375969361e-05
1418 1.17386393381954e-05
1419 1.16920035904874e-05
1420 1.17471312179873e-05
1421 1.17927130993678e-05
1422 1.18845154932027e-05
1423 1.18252309597722e-05
1424 1.17301622519032e-05
1425 1.16226955633181e-05
1426 1.15908669364728e-05
1427 1.15578795233754e-05
1428 1.15701796173558e-05
1429 1.16049062022228e-05
1430 1.17004379103965e-05
1431 1.17136618141463e-05
1432 1.16516864447173e-05
1433 1.1521697614351e-05
1434 1.14660726614109e-05
1435 1.14228305617322e-05
1436 1.14244615888974e-05
1437 1.14223891275837e-05
1438 1.15222256233949e-05
1439 1.15603424231855e-05
1440 1.15869071658873e-05
1441 1.14339813658641e-05
1442 1.13767621326607e-05
1443 1.12876584665855e-05
1444 1.13028394241136e-05
1445 1.12544618140475e-05
1446 1.13694213759175e-05
1447 1.13592027162213e-05
1448 1.148293643638e-05
1449 1.1349431466338e-05
1450 1.12995982372392e-05
1451 1.11666303143447e-05
1452 1.11732286960375e-05
1453 1.11253952681523e-05
1454 1.11752165151321e-05
1455 1.11980707444559e-05
1456 1.13082913877349e-05
1457 1.13025995460703e-05
1458 1.12311864045959e-05
1459 1.11005483300157e-05
1460 1.10445301153561e-05
1461 1.10111953108098e-05
1462 1.10146958493355e-05
1463 1.10313509450085e-05
1464 1.11143049927875e-05
1465 1.11868912675561e-05
1466 1.11844253112281e-05
1467 1.10702411385925e-05
1468 1.09560007762373e-05
1469 1.08982401224959e-05
1470 1.08822764997759e-05
1471 1.08852693093316e-05
1472 1.09212928715389e-05
1473 1.10037249463346e-05
1474 1.10729844943573e-05
1475 1.1047426590712e-05
1476 1.09138082050227e-05
1477 1.08131881807102e-05
1478 1.0759344062213e-05
1479 1.07610714579209e-05
1480 1.07577144454751e-05
1481 1.08314380563949e-05
1482 1.08726806521039e-05
1483 1.09910383348695e-05
1484 1.08501562886924e-05
1485 1.08216176547179e-05
1486 1.06327763352472e-05
1487 1.06861714013936e-05
1488 1.05927799308366e-05
1489 1.07447799114387e-05
1490 1.06488054771129e-05
1491 1.0845961147273e-05
1492 1.07425228854652e-05
1493 1.0763591520894e-05
1494 1.05829572234484e-05
1495 1.05630959339464e-05
1496 1.05069571330792e-05
1497 1.05401193207418e-05
1498 1.05429856750069e-05
1499 1.062801238542e-05
1500 1.06755836372467e-05
1501 1.07051821546615e-05
1502 1.05987097995808e-05
1503 1.0492846759765e-05
1504 1.04104938133768e-05
1505 1.04067396774118e-05
1506 1.04133128928652e-05
1507 1.0461334214303e-05
1508 1.05245419751062e-05
1509 1.06150131114665e-05
1510 1.05986967670428e-05
1511 1.04881145380394e-05
1512 1.03428675615191e-05
1513 1.02847817172365e-05
1514 1.0283976787188e-05
1515 1.03181056066622e-05
1516 1.03601218116959e-05
1517 1.04484016804444e-05
1518 1.0500153769281e-05
1519 1.04594350425202e-05
1520 1.02934239715561e-05
1521 1.01603183999543e-05
1522 1.01149316975579e-05
1523 1.01465865517033e-05
1524 1.0189362159944e-05
1525 1.02558852243343e-05
1526 1.03280322403994e-05
1527 1.03835583094754e-05
1528 1.02988652734837e-05
1529 1.01397457813827e-05
1530 1.00148587189134e-05
1531 1.00076086878194e-05
1532 1.00542717120788e-05
1533 1.01055967517046e-05
1534 1.0160510634923e-05
1535 1.02320627786886e-05
1536 1.02784197319276e-05
1537 1.01731683000613e-05
1538 9.99646257445382e-06
1539 9.86110082360862e-06
1540 9.85920702471314e-06
1541 9.91280015346282e-06
1542 9.97335172314706e-06
1543 1.00330967344557e-05
1544 1.01335873914365e-05
1545 1.02017198860456e-05
1546 1.0092902436772e-05
1547 9.89856104164725e-06
1548 9.74255067731633e-06
1549 9.75113458541917e-06
1550 9.81453603069819e-06
1551 9.88479949981524e-06
1552 9.94585073854637e-06
1553 1.01118834242205e-05
1554 1.00794489448219e-05
1555 1.01252378973034e-05
1556 9.72199865619435e-06
1557 9.71693203647798e-06
1558 9.62154877133988e-06
1559 9.75510985057326e-06
1560 9.71353716217038e-06
1561 9.8799359857793e-06
1562 9.89719038931488e-06
1563 1.00610475044505e-05
1564 9.83913167710663e-06
1565 9.71506382304218e-06
1566 9.53190467534171e-06
1567 9.56436412735684e-06
1568 9.58563060512082e-06
1569 9.66102254314668e-06
1570 9.71116265920576e-06
1571 9.86512819286152e-06
1572 9.88974000293297e-06
1573 9.81020425664762e-06
1574 9.58291114964519e-06
1575 9.47725170163949e-06
1576 9.46207945891805e-06
1577 9.52119806285623e-06
1578 9.55028414513966e-06
1579 9.67101213215452e-06
1580 9.72696674494955e-06
1581 9.85676225558526e-06
1582 9.69597010562021e-06
1583 9.54794038795133e-06
1584 9.3907083679549e-06
1585 9.43029413669194e-06
1586 9.42764649991545e-06
1587 9.58881266536575e-06
1588 9.49842403249098e-06
1589 9.8477689742064e-06
1590 9.64633825380664e-06
1591 9.77887764008756e-06
1592 9.4503434310648e-06
1593 9.37167252956783e-06
1594 9.29453434454312e-06
1595 9.34773383972261e-06
1596 9.36575015764163e-06
1597 9.42913244372789e-06
1598 9.50884421868636e-06
1599 9.58907908099604e-06
1600 9.56611028257726e-06
1601 9.42782079604854e-06
1602 9.28696681808579e-06
1603 9.22804946294301e-06
1604 9.24140071608311e-06
1605 9.27218876611846e-06
1606 9.32180338134951e-06
1607 9.39101415061335e-06
1608 9.48521949576767e-06
1609 9.49403899734117e-06
1610 9.40305500070811e-06
1611 9.24873794320528e-06
1612 9.17578468081122e-06
1613 9.17637183780092e-06
1614 9.22296179931929e-06
1615 9.24582680283947e-06
1616 9.35971817976148e-06
1617 9.3541317352693e-06
1618 9.55700087601485e-06
1619 9.33242078045756e-06
1620 9.35542227213186e-06
1621 9.11282236515376e-06
1622 9.17047079636764e-06
1623 9.14191326017999e-06
1624 9.25293021343994e-06
1625 9.19070588345505e-06
1626 9.34802572771787e-06
1627 9.31370317817463e-06
1628 9.35658445181768e-06
1629 9.1486900958652e-06
1630 9.05002940498889e-06
1631 8.95396064374646e-06
1632 8.99781081642459e-06
1633 9.00729275225368e-06
1634 9.09599193832378e-06
1635 9.11489385539236e-06
1636 9.26166050163602e-06
1637 9.20198866201834e-06
1638 9.17817889600769e-06
1639 8.94806501664883e-06
1640 8.91276849065055e-06
1641 8.91494655282562e-06
1642 9.03600756766802e-06
1643 9.09335119795666e-06
1644 9.2039959588656e-06
1645 9.2179104456136e-06
1646 9.37231926556592e-06
1647 9.1544567067577e-06
1648 9.05168580800364e-06
1649 8.81647829823843e-06
1650 8.85322269894573e-06
1651 8.91476529337372e-06
1652 9.01775216632927e-06
1653 8.99988562845048e-06
1654 9.11087957374453e-06
1655 9.08991927772718e-06
1656 9.11636582401343e-06
1657 8.89543203600832e-06
1658 8.75668564681575e-06
1659 8.66811800766717e-06
1660 8.71661668592575e-06
1661 8.75392420477776e-06
1662 8.83901822549641e-06
1663 8.87759714096603e-06
1664 9.03357368760756e-06
1665 9.0432156767406e-06
1666 9.0143902014006e-06
1667 8.72550774329757e-06
1668 8.63681840890251e-06
1669 8.65262731745512e-06
1670 8.80784908829924e-06
1671 8.89095982614663e-06
1672 9.00515675243222e-06
1673 9.02834415998181e-06
1674 9.29078336694052e-06
1675 9.00919078243116e-06
1676 8.91260212654288e-06
1677 8.59773340404028e-06
1678 8.59934361881661e-06
1679 8.6219117670557e-06
1680 8.70394964399187e-06
1681 8.71411416092371e-06
1682 8.80983583506634e-06
1683 8.88512155228227e-06
1684 8.91443482290792e-06
1685 8.76425898610478e-06
1686 8.58558011174182e-06
1687 8.48084715165953e-06
1688 8.48235530170882e-06
1689 8.51361675321035e-06
1690 8.55954188475986e-06
1691 8.62665797877563e-06
1692 8.75402996195855e-06
1693 8.85271509550497e-06
1694 8.82052194720965e-06
1695 8.6166881570667e-06
1696 8.45060383506772e-06
1697 8.42753814348877e-06
1698 8.50704474686381e-06
1699 8.58294318462072e-06
1700 8.69819424664797e-06
1701 8.72391842795158e-06
1702 9.04477656031114e-06
1703 8.86162963690632e-06
1704 8.89444025631292e-06
1705 8.56697144039723e-06
1706 8.55814653159352e-06
1707 8.51864221118603e-06
1708 8.59529079614655e-06
1709 8.55125628677437e-06
1710 8.63024167981526e-06
1711 8.63588647407691e-06
1712 8.72213117967628e-06
1713 8.62245277666318e-06
1714 8.53044024040628e-06
1715 8.37684289815381e-06
1716 8.33805257786461e-06
1717 8.31124005055509e-06
1718 8.35239441580408e-06
1719 8.36625719813355e-06
1720 8.46138899864476e-06
1721 8.49422283977219e-06
1722 8.59314715503009e-06
1723 8.4965448259311e-06
1724 8.44062604343776e-06
1725 8.27057826432072e-06
1726 8.2699329399342e-06
1727 8.25286887481032e-06
1728 8.34408917782525e-06
1729 8.37230447932787e-06
1730 8.54084387569287e-06
1731 8.57820004777447e-06
1732 8.77877929520802e-06
1733 8.65936314535058e-06
1734 8.61284482489566e-06
1735 8.45008491682601e-06
1736 8.53121250562102e-06
1737 8.52609987707827e-06
1738 8.59080966601292e-06
1739 8.52283866947801e-06
1740 8.57657207712729e-06
1741 8.53774053174069e-06
1742 8.56754138922611e-06
1743 8.42344774317401e-06
1744 8.3511977943355e-06
1745 8.20339219724057e-06
1746 8.19204164711114e-06
1747 8.14234851868415e-06
1748 8.21020228514158e-06
1749 8.19405565740302e-06
1750 8.32209811226411e-06
1751 8.26947604674425e-06
1752 8.37237371579628e-06
1753 8.24314375336144e-06
1754 8.22866169514214e-06
1755 8.11980937243106e-06
1756 8.1087820097044e-06
1757 8.09568323016189e-06
1758 8.15440447136477e-06
1759 8.19425356605545e-06
1760 8.30249022006531e-06
1761 8.35373521823612e-06
1762 8.48993798759542e-06
1763 8.46027857429969e-06
1764 8.48790814818775e-06
1765 8.4127093368617e-06
1766 8.46670430002897e-06
1767 8.47830274016322e-06
1768 8.53462398116752e-06
1769 8.48230450027169e-06
1770 8.45787749644463e-06
1771 8.36558619117985e-06
1772 8.30387870465188e-06
1773 8.19235397386819e-06
1774 8.12039491530451e-06
1775 8.05149860146533e-06
1776 8.03812660166159e-06
1777 8.03325129557682e-06
1778 8.05741806312691e-06
1779 8.06684634587403e-06
1780 8.08604360713616e-06
1781 8.073976655254e-06
1782 8.06626752710334e-06
1783 8.02347966436893e-06
1784 8.00473541436494e-06
1785 7.96521597952922e-06
1786 7.98915619644921e-06
1787 7.98527759471312e-06
1788 8.06301573893128e-06
1789 8.09299238350529e-06
1790 8.20148887825667e-06
1791 8.21366917117909e-06
1792 8.30596204200636e-06
1793 8.27906877868637e-06
1794 8.31130053787395e-06
1795 8.27133638618231e-06
1796 8.29903810621128e-06
1797 8.26278944169682e-06
1798 8.25366121404159e-06
1799 8.17440710676465e-06
1800 8.13309881960341e-06
1801 8.04687313760155e-06
1802 8.01414553090278e-06
1803 7.94206235058918e-06
1804 7.94767015873295e-06
1805 7.90348422915107e-06
1806 7.95749957897366e-06
1807 7.91972660252327e-06
1808 7.97733630975017e-06
1809 7.90107560177944e-06
1810 7.94310332826361e-06
1811 7.88122366657262e-06
1812 7.88454882538758e-06
1813 7.81755572513513e-06
1814 7.80367400802599e-06
1815 7.76684533931871e-06
1816 7.79168759971564e-06
1817 7.81009568484592e-06
1818 7.89060826159963e-06
1819 7.92541369894669e-06
1820 8.0629499568848e-06
1821 8.06357166756773e-06
1822 8.16351315672629e-06
1823 8.11383379328845e-06
1824 8.15370231303329e-06
1825 8.11510874617246e-06
1826 8.14006960562589e-06
1827 8.09388087930074e-06
1828 8.07126403756797e-06
1829 8.00729127921803e-06
1830 7.97336597931538e-06
1831 7.92115605522762e-06
1832 7.88587020454183e-06
1833 7.83875546576477e-06
1834 7.80774486131008e-06
1835 7.7790740847424e-06
1836 7.76714786828355e-06
1837 7.75553011559774e-06
1838 7.75060390859987e-06
1839 7.73750488273587e-06
1840 7.72816451899416e-06
1841 7.70733076886169e-06
1842 7.69739283157378e-06
1843 7.66970353064759e-06
1844 7.67621990386639e-06
1845 7.641248766177e-06
1846 7.69028864174478e-06
1847 7.66794777836329e-06
1848 7.76612711561597e-06
1849 7.76524551445353e-06
1850 7.90299259278981e-06
1851 7.89652612809277e-06
1852 8.02747403483295e-06
1853 8.0115954889474e-06
1854 8.08691610979887e-06
1855 8.05809245700099e-06
1856 8.07953149835801e-06
1857 8.02260864674054e-06
1858 7.98325851647519e-06
1859 7.90263564335684e-06
1860 7.84455007026471e-06
1861 7.78332236054761e-06
1862 7.74262315417218e-06
1863 7.71419369390477e-06
1864 7.68491554007748e-06
1865 7.68117705381618e-06
1866 7.64318862659744e-06
1867 7.65899869037412e-06
1868 7.60325983506505e-06
1869 7.63215011190255e-06
1870 7.55580171703703e-06
1871 7.58738210156669e-06
1872 7.52204654238161e-06
1873 7.55862880907898e-06
1874 7.53685978338581e-06
1875 7.5802837843734e-06
1876 7.58548424547219e-06
1877 7.64375587181121e-06
1878 7.66243782024389e-06
1879 7.74149575851387e-06
1880 7.76237235082059e-06
1881 7.86872896441082e-06
1882 7.89222652860152e-06
1883 8.0003443028905e-06
1884 8.01160424046543e-06
1885 8.0563047433202e-06
1886 7.99708583680096e-06
1887 7.94018481305632e-06
1888 7.82281694059842e-06
1889 7.73667617437468e-06
1890 7.64428533332288e-06
1891 7.59364056316751e-06
1892 7.54810849201704e-06
1893 7.52851546733287e-06
1894 7.50546110950268e-06
1895 7.4942633592201e-06
1896 7.47428431028633e-06
1897 7.46225546540321e-06
1898 7.44067397325428e-06
1899 7.4295667668404e-06
1900 7.41017010004915e-06
1901 7.4105475000863e-06
1902 7.40138943318887e-06
1903 7.43049754111288e-06
1904 7.43002535349054e-06
1905 7.50867182119919e-06
1906 7.50828323011395e-06
1907 7.62403046887054e-06
1908 7.60954188550045e-06
1909 7.7421428616257e-06
1910 7.7550491989579e-06
1911 7.90464770498526e-06
1912 7.95309602944864e-06
1913 8.05666352713767e-06
1914 8.04033257395531e-06
1915 8.01374402925603e-06
1916 7.89026035619145e-06
1917 7.77170956330527e-06
1918 7.62840383039531e-06
1919 7.52955064520222e-06
1920 7.44551373029386e-06
1921 7.40186383592819e-06
1922 7.36904632721765e-06
1923 7.35778252192176e-06
1924 7.34567537108433e-06
1925 7.34354000009318e-06
1926 7.33365557200235e-06
1927 7.33345095819497e-06
1928 7.32422790505893e-06
1929 7.33646284819163e-06
1930 7.33605290056024e-06
1931 7.3706084234478e-06
1932 7.37641069884868e-06
1933 7.42859298756097e-06
1934 7.42945859263007e-06
1935 7.49444024468933e-06
1936 7.49655512777281e-06
1937 7.58367565190099e-06
1938 7.60554888164696e-06
1939 7.70735979453245e-06
1940 7.72941054449916e-06
1941 7.79686671847912e-06
1942 7.76462433312493e-06
1943 7.73321772011097e-06
1944 7.6134033412482e-06
1945 7.51820967555735e-06
1946 7.39555784790014e-06
1947 7.32661418467728e-06
1948 7.27321606556567e-06
1949 7.25651213843529e-06
1950 7.25295173786833e-06
1951 7.26050765346524e-06
1952 7.23622183329553e-06
1953 7.25917648658481e-06
1954 7.24744407361773e-06
1955 7.2329047604569e-06
1956 7.202055985663e-06
1957 7.17788633567788e-06
1958 7.15542826649577e-06
1959 7.14866691590525e-06
1960 7.15193946021486e-06
1961 7.17194538424337e-06
1962 7.19688113785348e-06
1963 7.22561954151028e-06
1964 7.2419665884856e-06
1965 7.25336250617659e-06
1966 7.26789995475485e-06
1967 7.31576072491672e-06
1968 7.40354030999886e-06
1969 7.53192550699093e-06
1970 7.66084009562462e-06
1971 7.75957798045586e-06
1972 7.78337128319132e-06
1973 7.7449216047872e-06
1974 7.64330002607968e-06
1975 7.51256224518215e-06
1976 7.36660099794525e-06
1977 7.24715325025234e-06
1978 7.16053122751248e-06
1979 7.1158447377447e-06
1980 7.09522255881723e-06
1981 7.09455139554412e-06
1982 7.0966792673488e-06
1983 7.09730001524633e-06
1984 7.08644068521854e-06
1985 7.06671626045837e-06
1986 7.03897578091528e-06
1987 7.01287691384778e-06
1988 6.99059580237342e-06
1989 6.9808969662688e-06
1990 6.9771998922145e-06
1991 6.98973851243068e-06
1992 6.99768513593805e-06
1993 7.02172439422573e-06
1994 7.01809077874316e-06
1995 7.04368354457567e-06
1996 7.02138174150472e-06
1997 7.07498400724186e-06
1998 7.04834365272689e-06
1999 7.1688745742667e-06
};
\addlegendentry{Train}
\addplot [semithick, black]
table {%
0 0.090780146420002
1 0.0864395424723625
2 0.0818765088915825
3 0.0762186869978905
4 0.0636239573359489
5 0.0425069406628609
6 0.0306675806641579
7 0.0238344538956881
8 0.0192671436816454
9 0.0158791393041611
10 0.0131847290322185
11 0.0109455604106188
12 0.00903993938118219
13 0.00741039402782917
14 0.0060381144285202
15 0.00492167333140969
16 0.0040519293397665
17 0.00340097816661
18 0.00292235659435391
19 0.00256020622327924
20 0.00226605590432882
21 0.00201351591385901
22 0.00179630191996694
23 0.00161442393437028
24 0.00146435340866446
25 0.00133856385946274
26 0.00123035744763911
27 0.00113828480243683
28 0.00106478971429169
29 0.00100760534405708
30 0.000958102114964277
31 0.000911820505280048
32 0.000869181239977479
33 0.000831412384286523
34 0.000799110159277916
35 0.000771881022956222
36 0.000748652673792094
37 0.000728273531422019
38 0.000709852844011039
39 0.000692785135470331
40 0.000676674942951649
41 0.000661263940855861
42 0.000646364875137806
43 0.000631820934358984
44 0.000617548881564289
45 0.000603641790803522
46 0.000590138020925224
47 0.000576968071982265
48 0.000564101559575647
49 0.000551564095076174
50 0.000539412721991539
51 0.0005277045420371
52 0.000516459811478853
53 0.000505649310071021
54 0.000495211570523679
55 0.000485077733173966
56 0.000475199543870986
57 0.000465581833850592
58 0.000456270208815113
59 0.000447272381279618
60 0.000438579823821783
61 0.000430216285167262
62 0.000422204349888489
63 0.000414540409110487
64 0.000407204643124714
65 0.000400177057599649
66 0.000393437658203766
67 0.000386970525141805
68 0.000380760262487456
69 0.000374793162336573
70 0.000369060144294053
71 0.000363554601790383
72 0.000358271732693538
73 0.000353210576577112
74 0.000348367320839316
75 0.00034373759990558
76 0.000339311838615686
77 0.00033508287742734
78 0.000331041839672253
79 0.000327180954627693
80 0.000323494226904586
81 0.000319975573802367
82 0.000316616118652746
83 0.000313409749651328
84 0.00031034677522257
85 0.000307419861201197
86 0.000304621644318104
87 0.000301944994134828
88 0.000299384555546567
89 0.000296934158541262
90 0.000294588535325602
91 0.000292341428576037
92 0.000290187541395426
93 0.000288120470941067
94 0.000286134076304734
95 0.000284222740447149
96 0.000282381050055847
97 0.000280603795545176
98 0.000278885680017993
99 0.000277222192380577
100 0.000275609636446461
101 0.000274043588433415
102 0.000272520701400936
103 0.000271038239588961
104 0.000269592826953158
105 0.000268182309810072
106 0.000266804738203064
107 0.000265458103967831
108 0.000264142378000543
109 0.000262856570770964
110 0.000261599896475673
111 0.000260371598415077
112 0.000259170512435958
113 0.000257995969150215
114 0.000256845727562904
115 0.000255719060078263
116 0.000254614511504769
117 0.000253531499765813
118 0.000252468918915838
119 0.000251426623435691
120 0.000250404002144933
121 0.000249400502070785
122 0.00024841568665579
123 0.000247449177550152
124 0.000246500479988754
125 0.00024556863354519
126 0.000244653318077326
127 0.000243753689574078
128 0.000242869311477989
129 0.000241999587160535
130 0.000241143294260837
131 0.00024030075292103
132 0.000239471177337691
133 0.000238654014538042
134 0.000237849031691439
135 0.000237055952311493
136 0.000236274499911815
137 0.000235504339798354
138 0.000234745515626855
139 0.000233997940085828
140 0.000233261467656121
141 0.000232535778195597
142 0.000231821089982986
143 0.00023111718473956
144 0.000230424629990011
145 0.000229742872761562
146 0.000229071825742722
147 0.000228412071010098
148 0.000227762982831337
149 0.000227125070523471
150 0.000226498203119263
151 0.000225882205995731
152 0.00022527709370479
153 0.000224682633415796
154 0.000224098999751732
155 0.000223525858018547
156 0.000222963310079649
157 0.000222411385038868
158 0.000221869529923424
159 0.000221338254050352
160 0.000220816800720058
161 0.000220305257244036
162 0.000219803681829944
163 0.000219311361433938
164 0.000218829285586253
165 0.000218355911783874
166 0.000217892113141716
167 0.000217437147512101
168 0.000216991131310351
169 0.000216553831705824
170 0.000216124986764044
171 0.000215704465517774
172 0.000215292398934253
173 0.000214887812035158
174 0.000214491054066457
175 0.00021410193585325
176 0.000213719700695947
177 0.000213344537769444
178 0.000212975981412455
179 0.00021261336223688
180 0.000212256360100582
181 0.000211904523894191
182 0.000211557300644927
183 0.000211213497095741
184 0.000210872298339382
185 0.000210532423807308
186 0.00021019262203481
187 0.000209850928513333
188 0.000209505713428371
189 0.000209154750336893
190 0.000208795914659277
191 0.000208427562029101
192 0.000208049226785079
193 0.000207662087632343
194 0.000207270466489717
195 0.000206881828489713
196 0.000206505646929145
197 0.000206150085432455
198 0.000205818912945688
199 0.00020551061606966
200 0.000205220523639582
201 0.000204944211873226
202 0.000204677446163259
203 0.000204417781787924
204 0.00020416323968675
205 0.000203912393772043
206 0.000203663745196536
207 0.000203416217118502
208 0.000203167553991079
209 0.000202914801775478
210 0.000202653929591179
211 0.000202378811081871
212 0.000202081209863536
213 0.000201750634005293
214 0.000201377100893296
215 0.000200957656488754
216 0.000200502268853597
217 0.000200035196030512
218 0.000199584639631212
219 0.000199169633560814
220 0.000198795474716462
221 0.000198457200895064
222 0.000198146808543243
223 0.00019785737094935
224 0.000197583736735396
225 0.000197322326130234
226 0.000197070752619766
227 0.000196828274056315
228 0.000196593522559851
229 0.000196366381715052
230 0.000196146836970001
231 0.000195935062947683
232 0.000195731074200012
233 0.000195534870726988
234 0.000195347180124372
235 0.000195168249774724
236 0.000194997046492063
237 0.000194834297872148
238 0.000194677530089393
239 0.000194525040569715
240 0.000194372332771309
241 0.000194213920622133
242 0.000194041451322846
243 0.000193849686183967
244 0.000193639119970612
245 0.000193429514183663
246 0.000193261628737673
247 0.000193169311387464
248 0.000193131330888718
249 0.0001930699654622
250 0.000192924067960121
251 0.000192691179108806
252 0.0001924008684
253 0.000192079605767503
254 0.000191741244634613
255 0.000191392595297657
256 0.000191035680472851
257 0.000190671926247887
258 0.000190300997928716
259 0.000189923390280455
260 0.000189537575352006
261 0.000189144135219976
262 0.000188741338206455
263 0.000188328136573546
264 0.000187903860933147
265 0.000187465702765621
266 0.00018701289081946
267 0.000186540477443486
268 0.000186047021998093
269 0.000185525786946528
270 0.000184974516741931
271 0.000184388423804194
272 0.000183766052941792
273 0.000183106778422371
274 0.000182410367415287
275 0.000181676892680116
276 0.000180906587047502
277 0.000180100571014918
278 0.000179261391167529
279 0.000178389789653011
280 0.0001774880947778
281 0.000176557339727879
282 0.000175598543137312
283 0.00017461137031205
284 0.000173593580257148
285 0.000172542917425744
286 0.000171451698406599
287 0.000170307554071769
288 0.000169088234542869
289 0.00016775299445726
290 0.00016622370458208
291 0.000164344790391624
292 0.000161831412697211
293 0.000158358030603267
294 0.000154088484123349
295 0.000149775267345831
296 0.000145917729241773
297 0.000142584744025953
298 0.000139692303491756
299 0.000137147930217907
300 0.000134875066578388
301 0.000132811415824108
302 0.000130906148115173
303 0.000129117659525946
304 0.000127414197777398
305 0.000125772261526436
306 0.000124178972328082
307 0.000122628902317956
308 0.000121122066047974
309 0.000119662152428646
310 0.000118253279651981
311 0.000116899311251473
312 0.000115603441372514
313 0.000114367641799618
314 0.000113192116259597
315 0.000112076333607547
316 0.000111018496681936
317 0.000110017230326775
318 0.000109070570033509
319 0.000108178544905968
320 0.000107342231785879
321 0.000106564111774787
322 0.000105846040241886
323 0.00010518501221668
324 0.000104572864074726
325 0.000103996630059555
326 0.000103441780083813
327 0.000102897036413196
328 0.000102355945273302
329 0.000101813930086792
330 0.000101268604339566
331 0.000100721248600166
332 0.000100174685940146
333 9.96338058030233e-05
334 9.9105789558962e-05
335 9.85976512311026e-05
336 9.81164121185429e-05
337 9.76687733782455e-05
338 9.72577327047475e-05
339 9.68842359725386e-05
340 9.65461440500803e-05
341 9.6238472906407e-05
342 9.59552999120206e-05
343 9.56885269260965e-05
344 9.54317802097648e-05
345 9.51798065216281e-05
346 9.49289460550062e-05
347 9.46763539104722e-05
348 9.44210623856634e-05
349 9.41604303079657e-05
350 9.3893933808431e-05
351 9.36196956899948e-05
352 9.33365372475237e-05
353 9.30437308852561e-05
354 9.27426444832236e-05
355 9.24328196560964e-05
356 9.21175887924619e-05
357 9.17984652915038e-05
358 9.14775519049726e-05
359 9.11583920242265e-05
360 9.08417205209844e-05
361 9.05312408576719e-05
362 9.02285537449643e-05
363 8.99352162377909e-05
364 8.96529818419367e-05
365 8.93835967872292e-05
366 8.91279414645396e-05
367 8.88862487045117e-05
368 8.86581692611799e-05
369 8.84435212356038e-05
370 8.82417371030897e-05
371 8.80512088770047e-05
372 8.78708233358338e-05
373 8.76991762197576e-05
374 8.75346959219314e-05
375 8.73764001880772e-05
376 8.72232267283835e-05
377 8.70745061547495e-05
378 8.69295981829055e-05
379 8.67879207362421e-05
380 8.66492409841157e-05
381 8.65132824401371e-05
382 8.63796740304679e-05
383 8.62484375829808e-05
384 8.61193766468205e-05
385 8.59924621181563e-05
386 8.58676139614545e-05
387 8.57447448652238e-05
388 8.56238548294641e-05
389 8.55049875099212e-05
390 8.53880264912732e-05
391 8.52730954647996e-05
392 8.51598451845348e-05
393 8.50485521368682e-05
394 8.49382922751829e-05
395 8.48296913318336e-05
396 8.47219780553132e-05
397 8.4614752267953e-05
398 8.45086324261501e-05
399 8.44029927975498e-05
400 8.42980880406685e-05
401 8.41931323520839e-05
402 8.40889915707521e-05
403 8.39848507894203e-05
404 8.38810083223507e-05
405 8.37773623061366e-05
406 8.36744002299383e-05
407 8.35711398394778e-05
408 8.34684688015841e-05
409 8.33658705232665e-05
410 8.32633668323979e-05
411 8.31612051115371e-05
412 8.30594217404723e-05
413 8.29579221317545e-05
414 8.28563133836724e-05
415 8.27546609798446e-05
416 8.26536997919902e-05
417 8.25523529783823e-05
418 8.24520466267131e-05
419 8.23512382339686e-05
420 8.22508518467657e-05
421 8.21502690087073e-05
422 8.20501518319361e-05
423 8.19498091004789e-05
424 8.18498156149872e-05
425 8.17495165392756e-05
426 8.16494502942078e-05
427 8.15497987787239e-05
428 8.14498125691898e-05
429 8.13503647805192e-05
430 8.12504440546036e-05
431 8.11507852631621e-05
432 8.10507626738399e-05
433 8.0950849223882e-05
434 8.08513796073385e-05
435 8.07513133622706e-05
436 8.06518582976423e-05
437 8.05519375717267e-05
438 8.04521405370906e-05
439 8.03526563686319e-05
440 8.02527210908011e-05
441 8.0152640293818e-05
442 8.00527559476905e-05
443 7.99528352217749e-05
444 7.98527908045799e-05
445 7.97530592535622e-05
446 7.96530221123248e-05
447 7.95531450421549e-05
448 7.94534207670949e-05
449 7.93532744864933e-05
450 7.92534119682387e-05
451 7.91528946137987e-05
452 7.90536869317293e-05
453 7.89532423368655e-05
454 7.88543256931007e-05
455 7.87542448961176e-05
456 7.86543459980749e-05
457 7.85546653787605e-05
458 7.8455195762217e-05
459 7.8355755249504e-05
460 7.82563583925366e-05
461 7.81570925028063e-05
462 7.80581467552111e-05
463 7.79593174229376e-05
464 7.78605608502403e-05
465 7.77621025918052e-05
466 7.7663549745921e-05
467 7.75651787989773e-05
468 7.74673171690665e-05
469 7.73693391238339e-05
470 7.72720595705323e-05
471 7.71752747823484e-05
472 7.70783444750123e-05
473 7.6981887104921e-05
474 7.68858008086681e-05
475 7.6790020102635e-05
476 7.66952434787527e-05
477 7.66005687182769e-05
478 7.65061195124872e-05
479 7.64133365009911e-05
480 7.63205534894951e-05
481 7.62281415518373e-05
482 7.61367482482456e-05
483 7.60461844038218e-05
484 7.5956697401125e-05
485 7.58674941607751e-05
486 7.57791640353389e-05
487 7.56912922952324e-05
488 7.56045847083442e-05
489 7.55183718865737e-05
490 7.54327338654548e-05
491 7.53475615056232e-05
492 7.52624910091981e-05
493 7.51783227315173e-05
494 7.50943218008615e-05
495 7.50115214032121e-05
496 7.49282044125721e-05
497 7.48456950532272e-05
498 7.47645535739139e-05
499 7.46838413760997e-05
500 7.46036384953186e-05
501 7.45245488360524e-05
502 7.44456774555147e-05
503 7.43683995096944e-05
504 7.42917618481442e-05
505 7.42159245419316e-05
506 7.41414551157504e-05
507 7.40676259738393e-05
508 7.39949682611041e-05
509 7.39227398298681e-05
510 7.38517774152569e-05
511 7.37819063942879e-05
512 7.37126028980128e-05
513 7.36445945221931e-05
514 7.35773282940499e-05
515 7.3510644142516e-05
516 7.34454733901657e-05
517 7.33808046788909e-05
518 7.33169072191231e-05
519 7.32541157049127e-05
520 7.31916661607102e-05
521 7.31306063244119e-05
522 7.30700630811043e-05
523 7.30102765373886e-05
524 7.29512612451799e-05
525 7.28931699995883e-05
526 7.28360755601898e-05
527 7.27790393284522e-05
528 7.27234000805765e-05
529 7.26681391824968e-05
530 7.26136568118818e-05
531 7.25601348676719e-05
532 7.25072241039015e-05
533 7.24550845916383e-05
534 7.24028941476718e-05
535 7.23522898624651e-05
536 7.23016419215128e-05
537 7.22520708222874e-05
538 7.22031618352048e-05
539 7.21549149602652e-05
540 7.21068718121387e-05
541 7.20601601642556e-05
542 7.20137250027619e-05
543 7.19676827429794e-05
544 7.1922127972357e-05
545 7.18777664587833e-05
546 7.18334267730825e-05
547 7.17899456503801e-05
548 7.17466391506605e-05
549 7.17041912139393e-05
550 7.16621943865903e-05
551 7.16207796358503e-05
552 7.15798232704401e-05
553 7.15393834980205e-05
554 7.14995185262524e-05
555 7.14598936610855e-05
556 7.1420960011892e-05
557 7.13824774720706e-05
558 7.13439076207578e-05
559 7.13059125700966e-05
560 7.12685723556206e-05
561 7.12311884853989e-05
562 7.11940592736937e-05
563 7.11571919964626e-05
564 7.11206157575361e-05
565 7.10843887645751e-05
566 7.10482127033174e-05
567 7.10122112650424e-05
568 7.09758896846324e-05
569 7.0940179284662e-05
570 7.09037849446759e-05
571 7.08672741893679e-05
572 7.08306979504414e-05
573 7.07936269463971e-05
574 7.07561266608536e-05
575 7.07182844053023e-05
576 7.06798673490994e-05
577 7.0640591729898e-05
578 7.0601083280053e-05
579 7.0560461608693e-05
580 7.05197162460536e-05
581 7.04785124980845e-05
582 7.04381600371562e-05
583 7.03996192896739e-05
584 7.03652767697349e-05
585 7.0338930527214e-05
586 7.03265759511851e-05
587 7.03375990269706e-05
588 7.03855839674361e-05
589 7.04866833984852e-05
590 7.06560385879129e-05
591 7.08973020664416e-05
592 7.1186113927979e-05
593 7.14636989869177e-05
594 7.16587092028931e-05
595 7.17282964615151e-05
596 7.16882786946371e-05
597 7.15906571713276e-05
598 7.14841080480255e-05
599 7.13956469553523e-05
600 7.13324043317698e-05
601 7.12906548869796e-05
602 7.12658220436424e-05
603 7.12514665792696e-05
604 7.12434266461059e-05
605 7.12386463419534e-05
606 7.12359469616786e-05
607 7.12334804120474e-05
608 7.12314940756187e-05
609 7.12300607119687e-05
610 7.12282126187347e-05
611 7.12262626620941e-05
612 7.12246110197157e-05
613 7.12228647898883e-05
614 7.12204200681299e-05
615 7.12175678927451e-05
616 7.12147884769365e-05
617 7.12113978806883e-05
618 7.12070250301622e-05
619 7.12023975211196e-05
620 7.11971733835526e-05
621 7.11916436557658e-05
622 7.11857574060559e-05
623 7.1179420046974e-05
624 7.11725151631981e-05
625 7.11652537574992e-05
626 7.11575339664705e-05
627 7.11493994458579e-05
628 7.11414832039736e-05
629 7.11326574673876e-05
630 7.11243555997498e-05
631 7.11157917976379e-05
632 7.11070097167976e-05
633 7.10988388163969e-05
634 7.10908134351484e-05
635 7.10825916030444e-05
636 7.10740350768901e-05
637 7.10660096956417e-05
638 7.10585954948328e-05
639 7.10512395016849e-05
640 7.10441454430111e-05
641 7.1037036832422e-05
642 7.10305757820606e-05
643 7.10240565240383e-05
644 7.10179665475152e-05
645 7.10123858880252e-05
646 7.10062959115021e-05
647 7.10008680471219e-05
648 7.09959131199867e-05
649 7.09909581928514e-05
650 7.09861051291227e-05
651 7.09810774424113e-05
652 7.09766100044362e-05
653 7.09728265064768e-05
654 7.0968373620417e-05
655 7.09648302290589e-05
656 7.09608721081167e-05
657 7.09568048478104e-05
658 7.09533924236894e-05
659 7.09495361661538e-05
660 7.0945912739262e-05
661 7.09425075910985e-05
662 7.09392697899602e-05
663 7.09358573658392e-05
664 7.09323576302268e-05
665 7.09287196514197e-05
666 7.09252926753834e-05
667 7.09217492840253e-05
668 7.09180894773453e-05
669 7.09140658727847e-05
670 7.09102896507829e-05
671 7.09064916009083e-05
672 7.09023370291106e-05
673 7.08980078343302e-05
674 7.08937805029564e-05
675 7.08894804120064e-05
676 7.08853185642511e-05
677 7.08812003722414e-05
678 7.08757434040308e-05
679 7.08708757883869e-05
680 7.08654551999643e-05
681 7.08602892700583e-05
682 7.0855014200788e-05
683 7.0849055191502e-05
684 7.08428851794451e-05
685 7.08367151673883e-05
686 7.08301377017051e-05
687 7.08233346813358e-05
688 7.08164880052209e-05
689 7.08090883563273e-05
690 7.08015941199847e-05
691 7.07937579136342e-05
692 7.07854123902507e-05
693 7.07775470800698e-05
694 7.07687868271023e-05
695 7.07602521288209e-05
696 7.07510553183965e-05
697 7.07415165379643e-05
698 7.07320505171083e-05
699 7.07218496245332e-05
700 7.0711801527068e-05
701 7.07012950442731e-05
702 7.0690679422114e-05
703 7.067959086271e-05
704 7.06680148141459e-05
705 7.06562714185566e-05
706 7.06446880940348e-05
707 7.0632049755659e-05
708 7.06196806277148e-05
709 7.06070350133814e-05
710 7.05942293279804e-05
711 7.05807105987333e-05
712 7.05672719050199e-05
713 7.05536585883237e-05
714 7.05393322277814e-05
715 7.05246930010617e-05
716 7.05103157088161e-05
717 7.04954800312407e-05
718 7.04798949300312e-05
719 7.04642588971183e-05
720 7.0448309998028e-05
721 7.04322155797854e-05
722 7.04153499100357e-05
723 7.03985206200741e-05
724 7.03809928381816e-05
725 7.036337046884e-05
726 7.03452969901264e-05
727 7.03272089594975e-05
728 7.0307869464159e-05
729 7.02888355590403e-05
730 7.02694960637018e-05
731 7.02492106938735e-05
732 7.02285906299949e-05
733 7.02077231835574e-05
734 7.01860335539095e-05
735 7.01643220963888e-05
736 7.01419048709795e-05
737 7.01191602274776e-05
738 7.00956443324685e-05
739 7.00719829183072e-05
740 7.00476084602997e-05
741 7.00230593793094e-05
742 6.99975644238293e-05
743 6.99713054927997e-05
744 6.99450101819821e-05
745 6.991811824264e-05
746 6.989039684413e-05
747 6.98625008226372e-05
748 6.98333460604772e-05
749 6.98043004376814e-05
750 6.97745344950818e-05
751 6.97442083037458e-05
752 6.97127907187678e-05
753 6.96809947839938e-05
754 6.96488132234663e-05
755 6.96156203048304e-05
756 6.95824201102369e-05
757 6.95480921422131e-05
758 6.95134294801392e-05
759 6.94779882906005e-05
760 6.94420086801983e-05
761 6.94055561325513e-05
762 6.9368070398923e-05
763 6.93300971761346e-05
764 6.92914327373728e-05
765 6.92522662575357e-05
766 6.9212335802149e-05
767 6.91718742018566e-05
768 6.91306267981417e-05
769 6.90886445227079e-05
770 6.90461311023682e-05
771 6.90028246026486e-05
772 6.89588850946166e-05
773 6.89143926138058e-05
774 6.88688305672258e-05
775 6.88231302774511e-05
776 6.87763604219072e-05
777 6.87289139023051e-05
778 6.86807907186449e-05
779 6.8632209149655e-05
780 6.85827253619209e-05
781 6.85324994265102e-05
782 6.8481414928101e-05
783 6.84299666318111e-05
784 6.83776524965651e-05
785 6.83244943502359e-05
786 6.8270368501544e-05
787 6.82161335134879e-05
788 6.81606761645526e-05
789 6.81049423292279e-05
790 6.80481753079221e-05
791 6.79903241689317e-05
792 6.79319564369507e-05
793 6.78731084917672e-05
794 6.78132782923058e-05
795 6.77528369124047e-05
796 6.76915951771662e-05
797 6.76297058816999e-05
798 6.75663686706685e-05
799 6.75027840770781e-05
800 6.74384937155992e-05
801 6.73730028211139e-05
802 6.73068789183162e-05
803 6.72400710755028e-05
804 6.71720699756406e-05
805 6.71037123538554e-05
806 6.70341178192757e-05
807 6.69640212436207e-05
808 6.68927968945354e-05
809 6.68210777803324e-05
810 6.67485219310038e-05
811 6.6675194830168e-05
812 6.66009946144186e-05
813 6.65258266963065e-05
814 6.64503895677626e-05
815 6.63740138406865e-05
816 6.62972670397721e-05
817 6.62197780911811e-05
818 6.61415251670405e-05
819 6.6062857513316e-05
820 6.59838551655412e-05
821 6.59042998449877e-05
822 6.58242279314436e-05
823 6.57436467008665e-05
824 6.56629854347557e-05
825 6.55817275401205e-05
826 6.55001058476046e-05
827 6.5418251324445e-05
828 6.53361712465994e-05
829 6.52536036795937e-05
830 6.51704249321483e-05
831 6.50874499115162e-05
832 6.50035581202246e-05
833 6.49192661512643e-05
834 6.48344939691015e-05
835 6.47492197458632e-05
836 6.46632252028212e-05
837 6.45765685476363e-05
838 6.44891042611562e-05
839 6.4401094277855e-05
840 6.43118401058018e-05
841 6.42222221358679e-05
842 6.41313381493092e-05
843 6.40400103293359e-05
844 6.39469944871962e-05
845 6.385372398654e-05
846 6.37585180811584e-05
847 6.36630866210908e-05
848 6.35660180705599e-05
849 6.34685857221484e-05
850 6.33691306575201e-05
851 6.32735318504274e-05
852 6.31806833553128e-05
853 6.30615177215077e-05
854 6.296257924987e-05
855 6.2856990552973e-05
856 6.27504705335014e-05
857 6.2642342527397e-05
858 6.2533508753404e-05
859 6.24218519078568e-05
860 6.2312392401509e-05
861 6.2199447711464e-05
862 6.20859500486404e-05
863 6.1971353716217e-05
864 6.18568519712426e-05
865 6.1736827774439e-05
866 6.16228935541585e-05
867 6.14986929576844e-05
868 6.13804659224115e-05
869 6.12548101344146e-05
870 6.11325740464963e-05
871 6.10054048593156e-05
872 6.08800983172841e-05
873 6.07516849413514e-05
874 6.06236862950027e-05
875 6.04931265115738e-05
876 6.03626431256998e-05
877 6.0230046074139e-05
878 6.00974453845993e-05
879 5.99630293436348e-05
880 5.98282495047897e-05
881 5.9691839851439e-05
882 5.95551173319109e-05
883 5.94167722738348e-05
884 5.92781616433058e-05
885 5.91384268773254e-05
886 5.89981100347359e-05
887 5.88567017985042e-05
888 5.87147333135363e-05
889 5.85722664254718e-05
890 5.8428766351426e-05
891 5.82854481763206e-05
892 5.81404856347945e-05
893 5.79962652409449e-05
894 5.78502258576918e-05
895 5.77050559513737e-05
896 5.7558008847991e-05
897 5.74125879211351e-05
898 5.72647259104997e-05
899 5.71187301829923e-05
900 5.69702642678749e-05
901 5.68238974665292e-05
902 5.66750932193827e-05
903 5.65284972253721e-05
904 5.63795947527979e-05
905 5.62327295483556e-05
906 5.60837179364171e-05
907 5.59368818358053e-05
908 5.57881248823833e-05
909 5.56412742298562e-05
910 5.54927974008024e-05
911 5.53463323740289e-05
912 5.5198139307322e-05
913 5.50519071111921e-05
914 5.49043361388613e-05
915 5.47585004824214e-05
916 5.46114752069116e-05
917 5.44660651939921e-05
918 5.43197966180742e-05
919 5.41750268894248e-05
920 5.40293949597981e-05
921 5.38852473255247e-05
922 5.37404630449601e-05
923 5.35970139026176e-05
924 5.34529608557932e-05
925 5.3310242947191e-05
926 5.31672085344326e-05
927 5.30250945303123e-05
928 5.28829623362981e-05
929 5.27416887052823e-05
930 5.26004041603301e-05
931 5.24599963682704e-05
932 5.23195885762107e-05
933 5.21799047419336e-05
934 5.20405810675584e-05
935 5.1901883125538e-05
936 5.17633379786275e-05
937 5.16255713591818e-05
938 5.14880593982525e-05
939 5.13512459292542e-05
940 5.12147344124969e-05
941 5.10788922838401e-05
942 5.09435194544494e-05
943 5.08086304762401e-05
944 5.06742326251697e-05
945 5.05405041621998e-05
946 5.04072559124324e-05
947 5.02746079291683e-05
948 5.01425129186828e-05
949 5.00110327266157e-05
950 4.98801164212637e-05
951 4.97498367622029e-05
952 4.96202119393274e-05
953 4.94911000714637e-05
954 4.93627558171283e-05
955 4.9235033657169e-05
956 4.91078826598823e-05
957 4.89814847242087e-05
958 4.88559089717455e-05
959 4.87308971059974e-05
960 4.86067619931418e-05
961 4.84833799418993e-05
962 4.83608637296129e-05
963 4.82392570120282e-05
964 4.8118869017344e-05
965 4.7999645175878e-05
966 4.78818183182739e-05
967 4.77655703434721e-05
968 4.76512250315864e-05
969 4.75388442282565e-05
970 4.74285043310374e-05
971 4.73204308946151e-05
972 4.72140964120626e-05
973 4.71091589133721e-05
974 4.70053673780058e-05
975 4.69020633317996e-05
976 4.67994141217787e-05
977 4.66965793748386e-05
978 4.65939774585422e-05
979 4.64914264739491e-05
980 4.6389308408834e-05
981 4.62874231743626e-05
982 4.6186145482352e-05
983 4.60854607808869e-05
984 4.5985641918378e-05
985 4.58866452390794e-05
986 4.57881869806442e-05
987 4.56908564956393e-05
988 4.5594457333209e-05
989 4.54987384728156e-05
990 4.54038454336114e-05
991 4.53099346486852e-05
992 4.52166568720713e-05
993 4.5124143071007e-05
994 4.50325715064537e-05
995 4.49414328613784e-05
996 4.48508035333361e-05
997 4.47606871603057e-05
998 4.46713493147399e-05
999 4.45824625785463e-05
1000 4.44936704298016e-05
1001 4.44056749984156e-05
1002 4.43177377746906e-05
1003 4.42301679868251e-05
1004 4.4142798287794e-05
1005 4.40557523688767e-05
1006 4.39688446931541e-05
1007 4.38818788097706e-05
1008 4.37949274783023e-05
1009 4.37082308053505e-05
1010 4.3621301301755e-05
1011 4.35341426054947e-05
1012 4.34471658081748e-05
1013 4.3359694245737e-05
1014 4.32720007665921e-05
1015 4.31839398515876e-05
1016 4.30959371442441e-05
1017 4.30069121648557e-05
1018 4.29175015597139e-05
1019 4.28277526225429e-05
1020 4.27374652645085e-05
1021 4.26461374445353e-05
1022 4.25538710260298e-05
1023 4.24611862399615e-05
1024 4.23676247010008e-05
1025 4.22731282014865e-05
1026 4.21777258452494e-05
1027 4.20807518821675e-05
1028 4.19832031184342e-05
1029 4.18842428189237e-05
1030 4.17839946749154e-05
1031 4.16826478613075e-05
1032 4.15796421293635e-05
1033 4.14752321376e-05
1034 4.13693996961229e-05
1035 4.12615772802383e-05
1036 4.11514047300443e-05
1037 4.103818355361e-05
1038 4.0919599996414e-05
1039 4.07877196266782e-05
1040 4.07311381422915e-05
1041 4.05844693887047e-05
1042 4.04597303713672e-05
1043 4.0332834032597e-05
1044 4.02064761146903e-05
1045 4.00792669097427e-05
1046 3.9950551581569e-05
1047 3.98195443267468e-05
1048 3.96862342313398e-05
1049 3.95507486246061e-05
1050 3.9413233025698e-05
1051 3.92737638321705e-05
1052 3.91327812394593e-05
1053 3.89905508200172e-05
1054 3.88474873034284e-05
1055 3.87041764042806e-05
1056 3.85611128876917e-05
1057 3.84188169846311e-05
1058 3.82788821298163e-05
1059 3.81429308617953e-05
1060 3.80138008040376e-05
1061 3.78944787371438e-05
1062 3.77867399947718e-05
1063 3.76900607079733e-05
1064 3.75990821339656e-05
1065 3.75068048015237e-05
1066 3.74065966752823e-05
1067 3.72939830413088e-05
1068 3.71683527191635e-05
1069 3.70315356121864e-05
1070 3.68856053682975e-05
1071 3.6733163142344e-05
1072 3.65767045877874e-05
1073 3.64174375135917e-05
1074 3.62569408025593e-05
1075 3.60949452442583e-05
1076 3.5932578612119e-05
1077 3.5769884561887e-05
1078 3.56061864295043e-05
1079 3.54418552888092e-05
1080 3.52770184690598e-05
1081 3.51113412762061e-05
1082 3.49454458046239e-05
1083 3.47785316989757e-05
1084 3.4612061426742e-05
1085 3.44452782883309e-05
1086 3.42787752742879e-05
1087 3.4113043511752e-05
1088 3.39480429829564e-05
1089 3.37840137945022e-05
1090 3.36210505338386e-05
1091 3.34593532897998e-05
1092 3.32994677592069e-05
1093 3.31411938532256e-05
1094 3.29851609421894e-05
1095 3.28317400999367e-05
1096 3.26805929944385e-05
1097 3.25327673635911e-05
1098 3.23881467920728e-05
1099 3.22476225846913e-05
1100 3.21109691867605e-05
1101 3.19791251968127e-05
1102 3.18526290357113e-05
1103 3.17316698783543e-05
1104 3.16169061989058e-05
1105 3.15097022394184e-05
1106 3.14093340421095e-05
1107 3.13177515636198e-05
1108 3.12344382109586e-05
1109 3.11608746415004e-05
1110 3.10969662677962e-05
1111 3.10413815896027e-05
1112 3.09950992232189e-05
1113 3.09550923702773e-05
1114 3.09183888020925e-05
1115 3.08793714793865e-05
1116 3.08316957671195e-05
1117 3.07654627249576e-05
1118 3.06716210616287e-05
1119 3.05425382975955e-05
1120 3.03752021864057e-05
1121 3.01736326946411e-05
1122 2.99496878142236e-05
1123 2.97149908874417e-05
1124 2.94849014608189e-05
1125 2.92678159894422e-05
1126 2.90683674393222e-05
1127 2.88886130874744e-05
1128 2.87257262243656e-05
1129 2.85782880382612e-05
1130 2.84436628135154e-05
1131 2.83190929621924e-05
1132 2.82025102933403e-05
1133 2.8092656066292e-05
1134 2.79867090284824e-05
1135 2.78851457551355e-05
1136 2.77867457043612e-05
1137 2.76908322121017e-05
1138 2.75972179224482e-05
1139 2.75059919658815e-05
1140 2.74158992397133e-05
1141 2.73274326900719e-05
1142 2.72406941803638e-05
1143 2.71545723080635e-05
1144 2.70701493718661e-05
1145 2.69877800747054e-05
1146 2.69065239990596e-05
1147 2.68260118900798e-05
1148 2.67472732957685e-05
1149 2.66700444626622e-05
1150 2.65940634562867e-05
1151 2.65201233560219e-05
1152 2.64468490058789e-05
1153 2.637526995386e-05
1154 2.63051479123533e-05
1155 2.62362955254503e-05
1156 2.6168983822572e-05
1157 2.61030909314286e-05
1158 2.60384913417511e-05
1159 2.59749467659276e-05
1160 2.59132521023275e-05
1161 2.58521376963472e-05
1162 2.579281317594e-05
1163 2.57350366155151e-05
1164 2.56781204370782e-05
1165 2.56219955190318e-05
1166 2.55674531217664e-05
1167 2.55138638749486e-05
1168 2.54608839895809e-05
1169 2.54092992690857e-05
1170 2.53591024375055e-05
1171 2.53095622610999e-05
1172 2.52611916948808e-05
1173 2.52135760092642e-05
1174 2.51667479460593e-05
1175 2.51210785791045e-05
1176 2.50760203925893e-05
1177 2.50317698373692e-05
1178 2.49886998062721e-05
1179 2.49459353653947e-05
1180 2.49041131610284e-05
1181 2.48627584369387e-05
1182 2.48221640504198e-05
1183 2.47825155383907e-05
1184 2.4743454559939e-05
1185 2.47048119490501e-05
1186 2.46672516368562e-05
1187 2.46294857788598e-05
1188 2.45928367803572e-05
1189 2.45566297962796e-05
1190 2.45203227677848e-05
1191 2.44851635216037e-05
1192 2.44498769461643e-05
1193 2.44156781263882e-05
1194 2.43813447013963e-05
1195 2.43477661570068e-05
1196 2.43144404521445e-05
1197 2.42813948716503e-05
1198 2.42486075876513e-05
1199 2.4216327801696e-05
1200 2.41845791606465e-05
1201 2.41526086028898e-05
1202 2.4121214664774e-05
1203 2.40898298216052e-05
1204 2.40592526097316e-05
1205 2.40280132857151e-05
1206 2.39973560383078e-05
1207 2.39673336182022e-05
1208 2.39367163885618e-05
1209 2.39067157963291e-05
1210 2.38769698626129e-05
1211 2.38473239733139e-05
1212 2.3817628971301e-05
1213 2.37885033129714e-05
1214 2.37590866163373e-05
1215 2.37298845604528e-05
1216 2.37005897361087e-05
1217 2.36715713981539e-05
1218 2.36428622883977e-05
1219 2.36139021581039e-05
1220 2.35857132793171e-05
1221 2.35573606914841e-05
1222 2.3528895326308e-05
1223 2.35006937145954e-05
1224 2.3472253815271e-05
1225 2.34440867643571e-05
1226 2.34161871048855e-05
1227 2.33880200539716e-05
1228 2.33605296671158e-05
1229 2.33324826695025e-05
1230 2.33047539950348e-05
1231 2.32774382311618e-05
1232 2.3250258891494e-05
1233 2.32231941481587e-05
1234 2.31961766985478e-05
1235 2.31685662583914e-05
1236 2.31416452152189e-05
1237 2.31145386351272e-05
1238 2.30876612477005e-05
1239 2.30614241445437e-05
1240 2.3034912373987e-05
1241 2.30086952797137e-05
1242 2.29826055146987e-05
1243 2.29565339395776e-05
1244 2.29304605454672e-05
1245 2.290459087817e-05
1246 2.28787648666184e-05
1247 2.28523877012776e-05
1248 2.28266344493022e-05
1249 2.28011849685572e-05
1250 2.27762848226121e-05
1251 2.27518885367317e-05
1252 2.2727395844413e-05
1253 2.27027376240585e-05
1254 2.26782176468987e-05
1255 2.26533993554767e-05
1256 2.26289466809249e-05
1257 2.26041120185982e-05
1258 2.25790008698823e-05
1259 2.25545918510761e-05
1260 2.25307012442499e-05
1261 2.25073217734462e-05
1262 2.24849827645812e-05
1263 2.24634732148843e-05
1264 2.24421310122125e-05
1265 2.24197992793052e-05
1266 2.23971383093158e-05
1267 2.23728220589692e-05
1268 2.23477436520625e-05
1269 2.23222941713175e-05
1270 2.22972685151035e-05
1271 2.22739472519606e-05
1272 2.22526050492888e-05
1273 2.22353173739975e-05
1274 2.22197886614595e-05
1275 2.22064591071103e-05
1276 2.2191461539478e-05
1277 2.21725258597871e-05
1278 2.21478967432631e-05
1279 2.21158497879514e-05
1280 2.20785823330516e-05
1281 2.20401107071666e-05
1282 2.20059391722316e-05
1283 2.19831981667085e-05
1284 2.19771118281642e-05
1285 2.19898192881374e-05
1286 2.201734969276e-05
1287 2.20494621316902e-05
1288 2.20669226109749e-05
1289 2.20493275264744e-05
1290 2.19769899558742e-05
1291 2.18554941966431e-05
1292 2.17045289900852e-05
1293 2.15727286558831e-05
1294 2.15054678847082e-05
1295 2.15492727875244e-05
1296 2.1733339963248e-05
1297 2.2279664335656e-05
1298 2.252743797726e-05
1299 2.2300719138002e-05
1300 2.18084023799747e-05
1301 2.14058727578958e-05
1302 2.12389459193218e-05
1303 2.12524719245266e-05
1304 2.13526654988527e-05
1305 2.15473010030109e-05
1306 2.18705790757667e-05
1307 2.23560928134248e-05
1308 2.25595595111372e-05
1309 2.20951296796557e-05
1310 2.15105756069534e-05
1311 2.10541711567203e-05
1312 2.09198515221942e-05
1313 2.10830567084486e-05
1314 2.14752435567789e-05
1315 2.19270386878634e-05
1316 2.22137568925973e-05
1317 2.19994144572411e-05
1318 2.15320906136185e-05
1319 2.1135485440027e-05
1320 2.09363861358725e-05
1321 2.10030102607561e-05
1322 2.12751256185584e-05
1323 2.16322223423049e-05
1324 2.18633613258135e-05
1325 2.17834804061567e-05
1326 2.1471187210409e-05
1327 2.11509886867134e-05
1328 2.09619138331618e-05
1329 2.09659483516589e-05
1330 2.11395272344816e-05
1331 2.1408735847217e-05
1332 2.16392090806039e-05
1333 2.16661501326598e-05
1334 2.14608080568723e-05
1335 2.11684255191358e-05
1336 2.09458630706649e-05
1337 2.08724504773272e-05
1338 2.09695626836037e-05
1339 2.11875121749472e-05
1340 2.14611700357636e-05
1341 2.16086791624548e-05
1342 2.15226373256883e-05
1343 2.12217128137127e-05
1344 2.09481913771015e-05
1345 2.07491411856608e-05
1346 2.07721514016157e-05
1347 2.09265417652205e-05
1348 2.12196191569092e-05
1349 2.15504860534566e-05
1350 2.15855434362311e-05
1351 2.13725306821289e-05
1352 2.09767749765888e-05
1353 2.06868935492821e-05
1354 2.05825654120417e-05
1355 2.06716376851546e-05
1356 2.09399659070186e-05
1357 2.13356524909614e-05
1358 2.15765758184716e-05
1359 2.14631327253301e-05
1360 2.10630241781473e-05
1361 2.06812583201099e-05
1362 2.04853422474116e-05
1363 2.04916796064936e-05
1364 2.07027351279976e-05
1365 2.10646612686105e-05
1366 2.14047631743597e-05
1367 2.14224346564151e-05
1368 2.11043261515442e-05
1369 2.07080338441301e-05
1370 2.04605894396082e-05
1371 2.03886065719416e-05
1372 2.05384912987938e-05
1373 2.08236069738632e-05
1374 2.11789410968777e-05
1375 2.12901541090105e-05
1376 2.11081733141327e-05
1377 2.07423254323658e-05
1378 2.04666921490571e-05
1379 2.03298568521859e-05
1380 2.03980634978507e-05
1381 2.06179847737076e-05
1382 2.094148658216e-05
1383 2.11788392334711e-05
1384 2.10844955290668e-05
1385 2.07748362299753e-05
1386 2.04727530217497e-05
1387 2.02880783035653e-05
1388 2.02832652576035e-05
1389 2.0430221411516e-05
1390 2.07195571420016e-05
1391 2.10178131965222e-05
1392 2.10854213946732e-05
1393 2.08317487704335e-05
1394 2.04936113732401e-05
1395 2.02489663934102e-05
1396 2.01691473193932e-05
1397 2.02562387130456e-05
1398 2.05104151973501e-05
1399 2.08441524591763e-05
1400 2.10248908842914e-05
1401 2.08955625566887e-05
1402 2.0574309019139e-05
1403 2.02574647119036e-05
1404 2.00838894670596e-05
1405 2.00743525056168e-05
1406 2.02642477233894e-05
1407 2.05946162168402e-05
1408 2.09490935958456e-05
1409 2.09464560612105e-05
1410 2.06952354346868e-05
1411 2.02828105102526e-05
1412 2.00729045900516e-05
1413 1.99487585632596e-05
1414 2.00898593902821e-05
1415 2.03602940018754e-05
1416 2.071678682114e-05
1417 2.08921283046948e-05
1418 2.07133380172309e-05
1419 2.03900981432525e-05
1420 2.0076462533325e-05
1421 1.991704084503e-05
1422 1.99285150301876e-05
1423 2.01222792384215e-05
1424 2.04384086828213e-05
1425 2.0724801288452e-05
1426 2.0723990019178e-05
1427 2.04837542696623e-05
1428 2.01563634618651e-05
1429 1.99180321942549e-05
1430 1.98350826394744e-05
1431 1.99145870283246e-05
1432 2.01713755814126e-05
1433 2.0489946109592e-05
1434 2.06858749152161e-05
1435 2.05634732992621e-05
1436 2.02870705834357e-05
1437 1.99604510271456e-05
1438 1.97951867448865e-05
1439 1.97538611246273e-05
1440 1.99341502593597e-05
1441 2.01981983991573e-05
1442 2.05734686460346e-05
1443 2.05637461476726e-05
1444 2.04197440325515e-05
1445 2.00242211576551e-05
1446 1.98231973627117e-05
1447 1.9668532331707e-05
1448 1.97555655176984e-05
1449 1.99226651602658e-05
1450 2.02920182346134e-05
1451 2.0445006157388e-05
1452 2.04827720153844e-05
1453 2.01605562324403e-05
1454 1.98958823602879e-05
1455 1.96515902644023e-05
1456 1.96006421901984e-05
1457 1.96825130842626e-05
1458 1.99563819478499e-05
1459 2.02404335141182e-05
1460 2.04383777600015e-05
1461 2.0308167222538e-05
1462 2.00379181478638e-05
1463 1.97132158064051e-05
1464 1.95338889170671e-05
1465 1.94984077097615e-05
1466 1.96403452719096e-05
1467 1.99152236746158e-05
1468 2.02281153178774e-05
1469 2.03362724278122e-05
1470 2.01845177798532e-05
1471 1.98584257304901e-05
1472 1.95707634702558e-05
1473 1.94161893887212e-05
1474 1.9416967916186e-05
1475 1.95902575796936e-05
1476 1.98754514713073e-05
1477 2.01684106286848e-05
1478 2.02060873562004e-05
1479 2.00376071006758e-05
1480 1.9678791431943e-05
1481 1.94465046661207e-05
1482 1.92990901268786e-05
1483 1.93823816516669e-05
1484 1.95151933439774e-05
1485 1.99464848265052e-05
1486 2.00026825041277e-05
1487 2.00989215954905e-05
1488 1.96956007130211e-05
1489 1.95555548998527e-05
1490 1.92920451809186e-05
1491 1.92749339475995e-05
1492 1.93038540601265e-05
1493 1.9526572941686e-05
1494 1.97505596588599e-05
1495 1.99111964320764e-05
1496 1.98703928617761e-05
1497 1.9646589862532e-05
1498 1.93752548511839e-05
1499 1.91899507626658e-05
1500 1.9134315152769e-05
1501 1.92200332094217e-05
1502 1.94148142327322e-05
1503 1.96649834833806e-05
1504 1.9819804947474e-05
1505 1.97651424969081e-05
1506 1.9510665879352e-05
1507 1.92283823707839e-05
1508 1.90431801456725e-05
1509 1.90137561730808e-05
1510 1.91175604413729e-05
1511 1.93389187188586e-05
1512 1.95920183614362e-05
1513 1.97376921278192e-05
1514 1.96347627934301e-05
1515 1.9355404219823e-05
1516 1.90640257642372e-05
1517 1.89114616659936e-05
1518 1.89065049198689e-05
1519 1.90299069799948e-05
1520 1.9249964680057e-05
1521 1.9492288629408e-05
1522 1.95957436517347e-05
1523 1.94767762877746e-05
1524 1.91905037354445e-05
1525 1.89332968147937e-05
1526 1.88083886314416e-05
1527 1.88334815902635e-05
1528 1.89789298019605e-05
1529 1.92278766917298e-05
1530 1.94794029084733e-05
1531 1.95796928892378e-05
1532 1.94138810911682e-05
1533 1.90994578588288e-05
1534 1.88203175639501e-05
1535 1.87014429684496e-05
1536 1.87322184501681e-05
1537 1.88783815247007e-05
1538 1.91306899068877e-05
1539 1.94016301975353e-05
1540 1.95168413483771e-05
1541 1.93328178283991e-05
1542 1.90011032827897e-05
1543 1.87028435902903e-05
1544 1.85988155863015e-05
1545 1.86372471944196e-05
1546 1.87958048627479e-05
1547 1.90933988051256e-05
1548 1.94260064745322e-05
1549 1.95775919564767e-05
1550 1.92839506780729e-05
1551 1.89303755178116e-05
1552 1.8616514353198e-05
1553 1.85074241016991e-05
1554 1.852663444879e-05
1555 1.88834073924227e-05
1556 1.9087563487119e-05
1557 1.9371536836843e-05
1558 1.94473632291192e-05
1559 1.91131111932918e-05
1560 1.88081030501053e-05
1561 1.84989930858137e-05
1562 1.84282853297191e-05
1563 1.85191329364898e-05
1564 1.87139521585777e-05
1565 1.89765760296723e-05
1566 1.93034338735742e-05
1567 1.92745828826446e-05
1568 1.9072967916145e-05
1569 1.86828365258407e-05
1570 1.84415657713544e-05
1571 1.83612901309971e-05
1572 1.84498658200027e-05
1573 1.86480483534979e-05
1574 1.89616330317222e-05
1575 1.91931849258253e-05
1576 1.9231682017562e-05
1577 1.89878264791332e-05
1578 1.86400411621435e-05
1579 1.8410210032016e-05
1580 1.83156935236184e-05
1581 1.84114542207681e-05
1582 1.86170764209237e-05
1583 1.89040547411423e-05
1584 1.91247745533474e-05
1585 1.9260742192273e-05
1586 1.89433012565132e-05
1587 1.87386976904236e-05
1588 1.83665524673415e-05
1589 1.83270112756873e-05
1590 1.83024494617712e-05
1591 1.84562468348304e-05
1592 1.87516834557755e-05
1593 1.89539114217041e-05
1594 1.90764549188316e-05
1595 1.88490066648228e-05
1596 1.85758435691241e-05
1597 1.82958756340668e-05
1598 1.81904433702584e-05
1599 1.82022977242013e-05
1600 1.83430111064808e-05
1601 1.85440421773819e-05
1602 1.87682926480193e-05
1603 1.88649719348177e-05
1604 1.87762580026174e-05
1605 1.85288954526186e-05
1606 1.82813528226689e-05
1607 1.81247432919918e-05
1608 1.81086907105055e-05
1609 1.82070434675552e-05
1610 1.84002583409892e-05
1611 1.86382549145492e-05
1612 1.88042249646969e-05
1613 1.87969599210192e-05
1614 1.86130746442359e-05
1615 1.83150204975391e-05
1616 1.81603518285556e-05
1617 1.8050388462143e-05
1618 1.81996747414814e-05
1619 1.82418571057497e-05
1620 1.85006629180862e-05
1621 1.87092500709696e-05
1622 1.87778696272289e-05
1623 1.8625763914315e-05
1624 1.82884577952791e-05
1625 1.80825409188401e-05
1626 1.79275248228805e-05
1627 1.79481048689922e-05
1628 1.80224615178304e-05
1629 1.81922241608845e-05
1630 1.83380889211548e-05
1631 1.84559994522715e-05
1632 1.83538486453472e-05
1633 1.81912455445854e-05
1634 1.79331545950845e-05
1635 1.77996116690338e-05
1636 1.77698966581374e-05
1637 1.78382069861982e-05
1638 1.80090246431064e-05
1639 1.82501917151967e-05
1640 1.84613982128212e-05
1641 1.86275410669623e-05
1642 1.84290329343639e-05
1643 1.81603772944072e-05
1644 1.78877780854236e-05
1645 1.77870751940645e-05
1646 1.78458740265341e-05
1647 1.79205308086239e-05
1648 1.81178111233748e-05
1649 1.84602886292851e-05
1650 1.85541266546352e-05
1651 1.84694945346564e-05
1652 1.80682600330329e-05
1653 1.77809797605732e-05
1654 1.75852455868153e-05
1655 1.75821896846173e-05
1656 1.76471494341968e-05
1657 1.78418995346874e-05
1658 1.80376646312652e-05
1659 1.82211551873479e-05
1660 1.81658269866602e-05
1661 1.8016924514086e-05
1662 1.76982885022881e-05
1663 1.75072636920959e-05
1664 1.74597153090872e-05
1665 1.75391869561281e-05
1666 1.7750395272742e-05
1667 1.80430797627196e-05
1668 1.83570355147822e-05
1669 1.8610951883602e-05
1670 1.8401997294859e-05
1671 1.80310180439847e-05
1672 1.76973680936499e-05
1673 1.75731856870698e-05
1674 1.77109141077381e-05
1675 1.77202873601345e-05
1676 1.79156031663297e-05
1677 1.81984087248566e-05
1678 1.8330287275603e-05
1679 1.82277181011159e-05
1680 1.78644859261112e-05
1681 1.75557343027322e-05
1682 1.73901516973274e-05
1683 1.73846365214558e-05
1684 1.74757915374357e-05
1685 1.76407593244221e-05
1686 1.78481786861084e-05
1687 1.7994445443037e-05
1688 1.79683738679159e-05
1689 1.77718411578098e-05
1690 1.75027016666718e-05
1691 1.72993350133765e-05
1692 1.72520140040433e-05
1693 1.73356456798501e-05
1694 1.75179557118099e-05
1695 1.77937290573027e-05
1696 1.80867245944683e-05
1697 1.82841777132126e-05
1698 1.81752202479402e-05
1699 1.78309564944357e-05
1700 1.76387020474067e-05
1701 1.74729721038602e-05
1702 1.76860230567399e-05
1703 1.76641242433107e-05
1704 1.79583930730587e-05
1705 1.81200102815637e-05
1706 1.83169013325823e-05
1707 1.81689374585403e-05
1708 1.79056405613665e-05
1709 1.7583270164323e-05
1710 1.74153883563122e-05
1711 1.73268199432641e-05
1712 1.73761145561002e-05
1713 1.74353481270373e-05
1714 1.75643399416003e-05
1715 1.76438679773128e-05
1716 1.76818575710058e-05
1717 1.75946479430422e-05
1718 1.74482756847283e-05
1719 1.72580803337041e-05
1720 1.71513293025782e-05
1721 1.70886141859228e-05
1722 1.71815445355605e-05
1723 1.72552372532664e-05
1724 1.74933957168832e-05
1725 1.760386476235e-05
1726 1.77906913449988e-05
1727 1.77788660948863e-05
1728 1.76823978108587e-05
1729 1.74169545061886e-05
1730 1.73957669176161e-05
1731 1.73707176145399e-05
1732 1.76071771420538e-05
1733 1.77547062776284e-05
1734 1.8114611521014e-05
1735 1.82137009687722e-05
1736 1.82769763341639e-05
1737 1.79924809344811e-05
1738 1.77009915205417e-05
1739 1.74106717167888e-05
1740 1.72876516444376e-05
1741 1.72116997418925e-05
1742 1.72398395079654e-05
1743 1.72449399542529e-05
1744 1.73076605278766e-05
1745 1.73154949152377e-05
1746 1.73118478414835e-05
1747 1.72325417224783e-05
1748 1.7155813111458e-05
1749 1.7017937352648e-05
1750 1.6978825442493e-05
1751 1.69078703038394e-05
1752 1.69655450008577e-05
1753 1.70053881447529e-05
1754 1.71479914570227e-05
1755 1.72158834175207e-05
1756 1.73540884134127e-05
1757 1.73581520357402e-05
1758 1.73692988028051e-05
1759 1.72691779880552e-05
1760 1.72777599800611e-05
1761 1.7254726117244e-05
1762 1.74223750946112e-05
1763 1.751923628035e-05
1764 1.77545389306033e-05
1765 1.78640730155166e-05
1766 1.79454727913253e-05
1767 1.77967376657762e-05
1768 1.75838158611441e-05
1769 1.73329462995753e-05
1770 1.71511892403942e-05
1771 1.70233215612825e-05
1772 1.69568465935299e-05
1773 1.69318136613583e-05
1774 1.6932857761276e-05
1775 1.69447266671341e-05
1776 1.69320410350338e-05
1777 1.69006307260133e-05
1778 1.68465503520565e-05
1779 1.67942998814397e-05
1780 1.6761054212111e-05
1781 1.67498110386077e-05
1782 1.67782436619746e-05
1783 1.68238111655228e-05
1784 1.6910915292101e-05
1785 1.69503455254016e-05
1786 1.70646089827642e-05
1787 1.712659104669e-05
1788 1.71987885551061e-05
1789 1.71755509654758e-05
1790 1.72401541931322e-05
1791 1.72535783349304e-05
1792 1.73706812347518e-05
1793 1.74282504303847e-05
1794 1.75431523530278e-05
1795 1.75527056853753e-05
1796 1.75152017618529e-05
1797 1.73603020812152e-05
1798 1.71874762600055e-05
1799 1.70145867741667e-05
1800 1.68941587617155e-05
1801 1.68185706570512e-05
1802 1.67851139849517e-05
1803 1.67882571986411e-05
1804 1.68108163052239e-05
1805 1.68100923474412e-05
1806 1.6822563338792e-05
1807 1.6771407899796e-05
1808 1.66717018146301e-05
1809 1.66070985869737e-05
1810 1.6589998267591e-05
1811 1.65659148478881e-05
1812 1.66387671924895e-05
1813 1.66739828273421e-05
1814 1.67937050719047e-05
1815 1.68561091413721e-05
1816 1.69547711266205e-05
1817 1.6974647223833e-05
1818 1.70412295119604e-05
1819 1.7005839254125e-05
1820 1.71689807757502e-05
1821 1.71948759088991e-05
1822 1.7425029000151e-05
1823 1.74930828507058e-05
1824 1.76430239662295e-05
1825 1.75845671037678e-05
1826 1.7494287021691e-05
1827 1.72852141986368e-05
1828 1.71059091371717e-05
1829 1.69350823853165e-05
1830 1.68291135196341e-05
1831 1.67527978192084e-05
1832 1.67194575624308e-05
1833 1.67010439326987e-05
1834 1.66969566635089e-05
1835 1.66854624694679e-05
1836 1.66665577125968e-05
1837 1.6630137906759e-05
1838 1.65914734679973e-05
1839 1.65470501087839e-05
1840 1.65229685080703e-05
1841 1.6506983229192e-05
1842 1.65240762726171e-05
1843 1.65456003742293e-05
1844 1.66240442922572e-05
1845 1.6622605471639e-05
1846 1.67397720360896e-05
1847 1.67677662830101e-05
1848 1.68837250384968e-05
1849 1.68793922057375e-05
1850 1.70878538483521e-05
1851 1.71271476574475e-05
1852 1.73657499544788e-05
1853 1.74186898220796e-05
1854 1.7576396203367e-05
1855 1.75282038981095e-05
1856 1.74881733983057e-05
1857 1.73205698956735e-05
1858 1.71736846823478e-05
1859 1.69976592587773e-05
1860 1.68610731634544e-05
1861 1.67408452398377e-05
1862 1.665441050136e-05
1863 1.65963665494928e-05
1864 1.65447618201142e-05
1865 1.6543634046684e-05
1866 1.64902903634356e-05
1867 1.65360143000726e-05
1868 1.64672928804066e-05
1869 1.65148412634153e-05
1870 1.6441515981569e-05
1871 1.64731973200105e-05
1872 1.64345765369944e-05
1873 1.64768644026481e-05
1874 1.64668563229498e-05
1875 1.65448400366586e-05
1876 1.65809578902554e-05
1877 1.67049547599163e-05
1878 1.6782600141596e-05
1879 1.69687755260384e-05
1880 1.70564890140668e-05
1881 1.72704912984045e-05
1882 1.73205171449808e-05
1883 1.7449036022299e-05
1884 1.73917414940661e-05
1885 1.73221833392745e-05
1886 1.71245028468547e-05
1887 1.69460417964729e-05
1888 1.67454163602088e-05
1889 1.65913406817708e-05
1890 1.64588982443092e-05
1891 1.63692220667144e-05
1892 1.63045369845349e-05
1893 1.62697742780438e-05
1894 1.62495834956644e-05
1895 1.62463638844201e-05
1896 1.62429369083839e-05
1897 1.62473897944437e-05
1898 1.62406213348731e-05
1899 1.62424439622555e-05
1900 1.62280921358615e-05
1901 1.62366068252595e-05
1902 1.62264932441758e-05
1903 1.62716951308539e-05
1904 1.62604665092658e-05
1905 1.64020202646498e-05
1906 1.64395023602992e-05
1907 1.66596055350965e-05
1908 1.67182042787317e-05
1909 1.69949089467991e-05
1910 1.70634139067261e-05
1911 1.72524250956485e-05
1912 1.7242015019292e-05
1913 1.72461241163546e-05
1914 1.70922630786663e-05
1915 1.69216873473488e-05
1916 1.6695352314855e-05
1917 1.64975517691346e-05
1918 1.6326957847923e-05
1919 1.62006326718256e-05
1920 1.61182251758873e-05
1921 1.60676499945112e-05
1922 1.60480849444866e-05
1923 1.60412564582657e-05
1924 1.60439267347101e-05
1925 1.60501695063431e-05
1926 1.60506460815668e-05
1927 1.60574472829467e-05
1928 1.6054942534538e-05
1929 1.60765575856203e-05
1930 1.60812269314192e-05
1931 1.61428815772524e-05
1932 1.61617317644414e-05
1933 1.62749875016743e-05
1934 1.63098811754026e-05
1935 1.64586617756868e-05
1936 1.64959183166502e-05
1937 1.66516510944348e-05
1938 1.6671345292707e-05
1939 1.67762882483657e-05
1940 1.67538637469988e-05
1941 1.67497364600422e-05
1942 1.66360059665749e-05
1943 1.65314868354471e-05
1944 1.63969780260231e-05
1945 1.62823926075362e-05
1946 1.61810858116951e-05
1947 1.60926501848735e-05
1948 1.6077601685538e-05
1949 1.61442985699978e-05
1950 1.60537565534469e-05
1951 1.58801867655711e-05
1952 1.58878574438859e-05
1953 1.58986167662079e-05
1954 1.59297360369237e-05
1955 1.59334285854129e-05
1956 1.59470109792892e-05
1957 1.5932220776449e-05
1958 1.59138326125685e-05
1959 1.5891842849669e-05
1960 1.58759648911655e-05
1961 1.58824532263679e-05
1962 1.59115279529942e-05
1963 1.59796145453583e-05
1964 1.60781310114544e-05
1965 1.6218993550865e-05
1966 1.63828881341033e-05
1967 1.65556793945143e-05
1968 1.66954523592722e-05
1969 1.68169972312171e-05
1970 1.6918731489568e-05
1971 1.69932027347386e-05
1972 1.69751765497494e-05
1973 1.68634887813823e-05
1974 1.6658143067616e-05
1975 1.64231678354554e-05
1976 1.62059186550323e-05
1977 1.60436011356069e-05
1978 1.59405517479172e-05
1979 1.58903203555383e-05
1980 1.5877982150414e-05
1981 1.58836282935226e-05
1982 1.5891702787485e-05
1983 1.58882539835759e-05
1984 1.58695693244226e-05
1985 1.58379571075784e-05
1986 1.57966624101391e-05
1987 1.57574122567894e-05
1988 1.57174999912968e-05
1989 1.56930127559463e-05
1990 1.56712885654997e-05
1991 1.56773821800016e-05
1992 1.56878340931144e-05
1993 1.57389913510997e-05
1994 1.5778257875354e-05
1995 1.58730472321622e-05
1996 1.59087285283022e-05
1997 1.60297222464578e-05
1998 1.6004656572477e-05
1999 1.61513726197882e-05
};
\addlegendentry{Test}
\end{groupplot}

\end{tikzpicture}

		\caption{Using augumented data. Shown are training- and validation error over 2000 epochs for the model with two and four convolutional layers in the encoder.}
		\label{Fig:DatAug}
		\vspace{-1cm}
	\end{figure}
\end{center}
\begin{center}
	\begin{figure}[H]
		% This file was created by tikzplotlib v0.9.6.
\begin{tikzpicture}

\begin{groupplot}[group style={
group size=2 by 1, horizontal sep=2cm},
legend cell align={left},
legend style={fill opacity=1, draw opacity=1, text opacity=1, draw=white},
log basis y={10},
tick align=outside,
tick pos=left,
title style={at={(0.43,0.85)},anchor=north},
x grid style={white!69.0196078431373!black},
xlabel={Epoch},
x label style={yshift=13pt},
xmin=-49.95, xmax=2048.95,
xtick style={color=black},
xtick = {0,500,1500,2000},
y grid style={white!69.0196078431373!black},
ylabel={MSE Loss},
ymode=log,
ytick style={color=black},
width=.45\textwidth,
height=.25\textwidth
]
\nextgroupplot[
title={2 Layer},
ymin=1.50584176153479e-05, ymax=0.001,
]
\addplot [semithick, black, dashed]
table {%
0 0.0373725050594658
1 0.0348841479280964
2 0.0320690423250198
3 0.0283010646235198
4 0.0232329077941055
5 0.0183147593246152
6 0.0143406990682706
7 0.011252622755516
8 0.00887993141562523
9 0.00709671809454449
10 0.0057836054644819
11 0.00481779648301502
12 0.00409785606704342
13 0.00355175955216206
14 0.00312950955958513
15 0.00279685058073179
16 0.00252955873778168
17 0.00230976984403242
18 0.00212441015689061
19 0.00196439310457208
20 0.00182399836527717
21 0.00169914659939726
22 0.00158685561488407
23 0.00148478343589886
24 0.00139144748663966
25 0.00130608500224601
26 0.001227738389692
27 0.00115568102470813
28 0.00108935335174465
29 0.00102846320040347
30 0.00097259987205689
31 0.000921112972946503
32 0.000873481037312255
33 0.000829310352097915
34 0.000788294194990158
35 0.000750193777927658
36 0.000714810678952441
37 0.000682007465987529
38 0.000651656210417665
39 0.000623577891246896
40 0.00059759270129689
41 0.000573543825642749
42 0.000551294260276336
43 0.000530733024637205
44 0.000511749810812034
45 0.000494250833336688
46 0.000478120912172623
47 0.000463252589156582
48 0.000449542958828412
49 0.000436891964606427
50 0.000425207966069744
51 0.000414410083863004
52 0.000404423520914558
53 0.00039517910977338
54 0.000386613634759669
55 0.000378668877450157
56 0.000371291101373572
57 0.000364430142250664
58 0.000358040429584131
59 0.000352080932922642
60 0.000346516304981985
61 0.000341313735759741
62 0.000336443752170605
63 0.000331879796306112
64 0.000327598603424425
65 0.000323578309443443
66 0.000319798657225571
67 0.000316240840466738
68 0.000312887869180637
69 0.000309724651325875
70 0.000306737120051063
71 0.000303912030744868
72 0.000301236727295873
73 0.000298699890739347
74 0.000296291313948889
75 0.000294001679473392
76 0.00029182196946446
77 0.000289743971317572
78 0.000287759935209427
79 0.00028586276353811
80 0.000284045891002431
81 0.00028230295386796
82 0.000280628659121855
83 0.000279017685348511
84 0.000277465111594211
85 0.00027596642263461
86 0.000274517124106903
87 0.000273113080993426
88 0.000271750457860283
89 0.000270425710340305
90 0.000269135433171641
91 0.000267876582256577
92 0.000266646263905083
93 0.000265441861065104
94 0.000264260847510892
95 0.000263101126705578
96 0.000261960375382841
97 0.000260836478076953
98 0.000259727499980045
99 0.000258631647720146
100 0.000257546867042417
101 0.000256472016720484
102 0.000255405660643267
103 0.00025434642439374
104 0.000253292999881675
105 0.000252244092261359
106 0.00025119857654469
107 0.000250155326426693
108 0.000249113302980201
109 0.000248071705035121
110 0.000247029543203325
111 0.000245986089311145
112 0.000244940561998419
113 0.000243892326390475
114 0.000242840720621492
115 0.000241785205579011
116 0.000240725292731743
117 0.000239660450963205
118 0.000238590388647708
119 0.000237514681174389
120 0.000236433040148161
121 0.000235345274622508
122 0.0002342510287671
123 0.000233150137508649
124 0.000232042557987218
125 0.00023092815059537
126 0.000229806827642657
127 0.00022867871306668
128 0.000227543795252435
129 0.000226402135425019
130 0.00022525392950475
131 0.000224099383999032
132 0.00022293897167458
133 0.000221772888176967
134 0.000220601562498738
135 0.000219425123669718
136 0.000218243756431017
137 0.000217057833225454
138 0.000215867729555915
139 0.000214673759269128
140 0.000213476465376061
141 0.000212276859845891
142 0.000211075593549973
143 0.000209873432927073
144 0.000208671063451978
145 0.000207469150770597
146 0.000206268207885311
147 0.000205068895491915
148 0.000203872244502653
149 0.00020267904421208
150 0.000201489780446688
151 0.000200305808486216
152 0.000199127735541538
153 0.000197958307031361
154 0.000196798878424905
155 0.000195650764216756
156 0.000194514443222715
157 0.000193390463512818
158 0.000192279258323917
159 0.000191181155310953
160 0.00019009658418175
161 0.000189025894400174
162 0.000187969420447113
163 0.000186927517736043
164 0.000185900201264152
165 0.000184887681162612
166 0.000183890152740673
167 0.000182907731044679
168 0.000181940424174816
169 0.000180988266549775
170 0.000180051207811024
171 0.000179129014478955
172 0.000178222548010846
173 0.0001773314323259
174 0.00017645606722283
175 0.000175596297031196
176 0.000174751963517868
177 0.000173922740079509
178 0.000173108351428179
179 0.000172308486526163
180 0.000171522849569783
181 0.000170751145707489
182 0.000169993054276082
183 0.000169248261684629
184 0.000168516402946276
185 0.000167797187463255
186 0.000167090229221382
187 0.000166395200139391
188 0.000165711761795251
189 0.000165039594795265
190 0.000164378384752695
191 0.000163727813974409
192 0.000163087522319453
193 0.000162457201113853
194 0.000161836557197148
195 0.000161225174461303
196 0.000160622793558218
197 0.000160028997020352
198 0.000159443542505263
199 0.000158866071373609
200 0.000158296301466502
201 0.000157733874161181
202 0.000157178588648795
203 0.000156630124874842
204 0.000156088219538238
205 0.000155552575857124
206 0.000155022919531215
207 0.000154498973581478
208 0.000153980415504407
209 0.000153467077664023
210 0.00015295864675835
211 0.000152454914458625
212 0.000151955627266886
213 0.000151460542682003
214 0.000150969406182829
215 0.000150482033724586
216 0.000149998217056672
217 0.000149517777354428
218 0.000149040500173688
219 0.000148566215787109
220 0.000148094716960164
221 0.000147625817248809
222 0.000147159381614396
223 0.000146695194641685
224 0.000146233103303454
225 0.000145773015977587
226 0.000145314753436783
227 0.000144858149279041
228 0.000144403164791621
229 0.000143949653841939
230 0.000143497505104525
231 0.000143046637660404
232 0.000142596941425192
233 0.000142148344809338
234 0.000141700751865888
235 0.000141254120291497
236 0.000140808302785918
237 0.000140363327671385
238 0.000139919222912492
239 0.000139475914015937
240 0.000139033364725094
241 0.000138591550687295
242 0.00013815045584901
243 0.000137710061712445
244 0.000137270324257107
245 0.000136831249768932
246 0.000136392789530741
247 0.000135954980554705
248 0.000135517883980659
249 0.000135081469473164
250 0.000134645677401105
251 0.00013421051718628
252 0.000133776016300639
253 0.000133342152951836
254 0.000132908939373048
255 0.000132476447329092
256 0.000132044584089404
257 0.000131613409327732
258 0.00013118293455013
259 0.000130753246795715
260 0.000130324303846407
261 0.000129896139031397
262 0.000129468796342754
263 0.000129042252166774
264 0.000128616558541239
265 0.000128191682698287
266 0.000127767693676617
267 0.000127344613728061
268 0.000126922461539891
269 0.000126501268276513
270 0.000126081069727964
271 0.000125661972568025
272 0.000125243919131416
273 0.000124827033713378
274 0.000124411381176988
275 0.000123996982568523
276 0.000123583827350634
277 0.000123171982648292
278 0.000122761505918353
279 0.000122352452542884
280 0.000121944952994113
281 0.000121538963772612
282 0.000121134618166726
283 0.000120731976458899
284 0.000120331080713261
285 0.000119932033456394
286 0.000119535120840434
287 0.000119140710798623
288 0.000118748736066247
289 0.00011835935636384
290 0.000117972575165955
291 0.000117588478597478
292 0.000117207158486584
293 0.000116828627381456
294 0.000116452997922067
295 0.00011608037820802
296 0.000115710981513454
297 0.000115345230511821
298 0.000114983035681841
299 0.000114624288867068
300 0.000114269062753654
301 0.000113917517751361
302 0.000113569779088607
303 0.0001132258654953
304 0.000112885797046639
305 0.000112549671773839
306 0.000112217605398257
307 0.000111889696934024
308 0.000111565967114776
309 0.000111246420542462
310 0.000110931006692757
311 0.000110619829979915
312 0.000110312842534673
313 0.000110010058688677
314 0.000109711416544182
315 0.000109416861597822
316 0.000109126391362698
317 0.000108839936082461
318 0.000108557624938517
319 0.000108279146128136
320 0.000108004483472257
321 0.000107733509568201
322 0.000107466173519792
323 0.000107202362540211
324 0.000106942087346814
325 0.00010668521406861
326 0.000106431698005357
327 0.000106181531900044
328 0.000105934553997618
329 0.000105690665674748
330 0.000105449879422774
331 0.000105212099787385
332 0.000104977167500427
333 0.000104745062988817
334 0.000104515697273655
335 0.000104288984196899
336 0.000104064894330236
337 0.000103843349028428
338 0.000103624289927495
339 0.000103407664634384
340 0.000103193393044876
341 0.000102981419651371
342 0.000102771698626232
343 0.000102564183284433
344 0.000102358826526464
345 0.000102155586591361
346 0.000101954383950442
347 0.000101755200913563
348 0.000101557885573792
349 0.00010136252042751
350 0.000101169357535014
351 0.000100978206321637
352 0.000100788942316872
353 0.000100601491440718
354 0.000100415890893638
355 0.00010023203301183
356 0.000100049907833485
357 9.98694456780432e-05
358 9.96906725276858e-05
359 9.95135720517718e-05
360 9.93380449282692e-05
361 9.91640702198519e-05
362 9.89916170688332e-05
363 9.88206773785786e-05
364 9.86511887267246e-05
365 9.84830996131336e-05
366 9.8316429398911e-05
367 9.81511487765374e-05
368 9.79871789006381e-05
369 9.78245548542607e-05
370 9.76631992664068e-05
371 9.75031237130016e-05
372 9.73443317882072e-05
373 9.71868096873626e-05
374 9.70304728014071e-05
375 9.68753109541846e-05
376 9.67212902729386e-05
377 9.65683992445084e-05
378 9.64166915894775e-05
379 9.62661286211623e-05
380 9.61166878189109e-05
381 9.59682899429974e-05
382 9.58209771593962e-05
383 9.56747255867659e-05
384 9.55294984983368e-05
385 9.53852957993699e-05
386 9.52420650837856e-05
387 9.50998091025686e-05
388 9.49585322835844e-05
389 9.48181535740389e-05
390 9.46787207093773e-05
391 9.45401923336438e-05
392 9.44025470820028e-05
393 9.42657700010822e-05
394 9.41298620631414e-05
395 9.39947737824317e-05
396 9.38605366371803e-05
397 9.37271077686338e-05
398 9.35945123738217e-05
399 9.34627066015992e-05
400 9.33316831996933e-05
401 9.32014281846231e-05
402 9.30719418204736e-05
403 9.2943185787675e-05
404 9.28151780238788e-05
405 9.26878995753573e-05
406 9.25613419819153e-05
407 9.24355066788489e-05
408 9.2310330001529e-05
409 9.21858767325053e-05
410 9.20621065958471e-05
411 9.19390026451102e-05
412 9.18165681724759e-05
413 9.16947964597625e-05
414 9.15736642959075e-05
415 9.14531429074827e-05
416 9.13332467291639e-05
417 9.12138985214028e-05
418 9.10951650100837e-05
419 9.09770512270332e-05
420 9.08595387988006e-05
421 9.07427083814601e-05
422 9.06265241316836e-05
423 9.05109401709107e-05
424 9.03959723170071e-05
425 9.02816274965801e-05
426 9.01678778764866e-05
427 9.00547245995161e-05
428 8.9942116130004e-05
429 8.98301047627115e-05
430 8.97186521306992e-05
431 8.96077597604498e-05
432 8.94974242591218e-05
433 8.93876314715196e-05
434 8.92783719604514e-05
435 8.91696528834511e-05
436 8.90614719987563e-05
437 8.89537861503698e-05
438 8.88465994345703e-05
439 8.87399450715994e-05
440 8.86337838951571e-05
441 8.8527861077381e-05
442 8.84224696484637e-05
443 8.83175893188574e-05
444 8.82132337783522e-05
445 8.81094986506052e-05
446 8.80063610774566e-05
447 8.79038160957653e-05
448 8.78019068437652e-05
449 8.77005731894087e-05
450 8.75998668531016e-05
451 8.74997440559374e-05
452 8.74002036906537e-05
453 8.73012662824616e-05
454 8.7202901703165e-05
455 8.71050993372554e-05
456 8.7007858837751e-05
457 8.6911183949212e-05
458 8.68150255932676e-05
459 8.67193936038291e-05
460 8.66242725952778e-05
461 8.65296695664597e-05
462 8.64355456175285e-05
463 8.63418874198866e-05
464 8.62486954815722e-05
465 8.61559609030375e-05
466 8.60637057513713e-05
467 8.5971857741877e-05
468 8.58804312239651e-05
469 8.57894278472789e-05
470 8.56988244741785e-05
471 8.56085994840328e-05
472 8.55187603564881e-05
473 8.54293151123879e-05
474 8.53401976653364e-05
475 8.52514437106322e-05
476 8.51629898548367e-05
477 8.50748962809196e-05
478 8.4987099917555e-05
479 8.48996335041837e-05
480 8.4812441831635e-05
481 8.47255841615417e-05
482 8.46389808231152e-05
483 8.4552672404925e-05
484 8.44666015697252e-05
485 8.43808043124265e-05
486 8.42952582900125e-05
487 8.42098974006926e-05
488 8.41247525211012e-05
489 8.40398566239742e-05
490 8.39551558620144e-05
491 8.38706384745554e-05
492 8.37863160872606e-05
493 8.37021672583186e-05
494 8.36181985791977e-05
495 8.35343960865487e-05
496 8.34507653758957e-05
497 8.336727659497e-05
498 8.32838858855212e-05
499 8.32006536578926e-05
500 8.31175200962283e-05
501 8.30344811480662e-05
502 8.2951505689266e-05
503 8.28686113045762e-05
504 8.27857302982219e-05
505 8.27029059931779e-05
506 8.26201462954164e-05
507 8.25373728462845e-05
508 8.24546372750016e-05
509 8.2371912507521e-05
510 8.22891822049125e-05
511 8.22064190787823e-05
512 8.21236293321685e-05
513 8.20407999088483e-05
514 8.1957873343678e-05
515 8.18748924302781e-05
516 8.17917913534435e-05
517 8.17085732502202e-05
518 8.16252448826068e-05
519 8.15417957025962e-05
520 8.1458163377827e-05
521 8.13743951866286e-05
522 8.12904303278591e-05
523 8.12062797545347e-05
524 8.11219403118457e-05
525 8.10373743066843e-05
526 8.09526096669326e-05
527 8.08675706084709e-05
528 8.0782256418388e-05
529 8.06966863512078e-05
530 8.06108311834919e-05
531 8.05246706742461e-05
532 8.04381638002856e-05
533 8.03513610871202e-05
534 8.02641927795851e-05
535 8.01766557086599e-05
536 8.00886985625008e-05
537 8.00003979589311e-05
538 7.99116539352459e-05
539 7.98224736655584e-05
540 7.97328818222809e-05
541 7.96428435248705e-05
542 7.95522919159926e-05
543 7.94612660683924e-05
544 7.93697361487489e-05
545 7.92776678117946e-05
546 7.91849945092811e-05
547 7.90917956964658e-05
548 7.89979666417177e-05
549 7.89035339892052e-05
550 7.88084565061335e-05
551 7.87127165793322e-05
552 7.86163160029218e-05
553 7.85192049039078e-05
554 7.84213135638367e-05
555 7.83226757287518e-05
556 7.82232956488826e-05
557 7.81230884510838e-05
558 7.80220575921457e-05
559 7.792015406712e-05
560 7.78173922277858e-05
561 7.77136925928327e-05
562 7.76090413777316e-05
563 7.75034399858043e-05
564 7.73968514960662e-05
565 7.72891986263365e-05
566 7.71804385190459e-05
567 7.70705679210929e-05
568 7.69596274565743e-05
569 7.68475309852098e-05
570 7.67342669287056e-05
571 7.66198033825087e-05
572 7.65040643102566e-05
573 7.63869862924575e-05
574 7.62685883148132e-05
575 7.61487506153458e-05
576 7.60274334723476e-05
577 7.59046432510975e-05
578 7.57802133162973e-05
579 7.5654164769882e-05
580 7.55264092878368e-05
581 7.5396887687873e-05
582 7.52656145988813e-05
583 7.5132503802422e-05
584 7.49975392828617e-05
585 7.486067704221e-05
586 7.47218738072301e-05
587 7.45810879436704e-05
588 7.44382758159172e-05
589 7.42934421893438e-05
590 7.41465110311405e-05
591 7.39973978104066e-05
592 7.3846060801704e-05
593 7.36925022740327e-05
594 7.3536576407444e-05
595 7.33783660370098e-05
596 7.32179276852207e-05
597 7.30552455057894e-05
598 7.2890291905973e-05
599 7.27230215034069e-05
600 7.25533388153584e-05
601 7.23812814958554e-05
602 7.22067772696751e-05
603 7.2029752667883e-05
604 7.18506061640293e-05
605 7.16696366656796e-05
606 7.14870731475988e-05
607 7.13026824428198e-05
608 7.11163460541305e-05
609 7.09280093479e-05
610 7.07377174767506e-05
611 7.05455477287842e-05
612 7.03515752829749e-05
613 7.01558290181481e-05
614 6.99584339519295e-05
615 6.97593363862363e-05
616 6.95586535298294e-05
617 6.93564816411178e-05
618 6.91528646541428e-05
619 6.89478224747593e-05
620 6.87414563588599e-05
621 6.8533887887412e-05
622 6.83251317461497e-05
623 6.81152386761141e-05
624 6.79041632534923e-05
625 6.76920061882432e-05
626 6.74788341742764e-05
627 6.72646510035217e-05
628 6.70496117306148e-05
629 6.68337811831066e-05
630 6.66171534291493e-05
631 6.6399800759361e-05
632 6.61817870195591e-05
633 6.59632750957447e-05
634 6.57442946755774e-05
635 6.55249389135785e-05
636 6.53053030520804e-05
637 6.50854180020606e-05
638 6.4865381087742e-05
639 6.4645276759497e-05
640 6.44251905536445e-05
641 6.4205220972724e-05
642 6.39854560375852e-05
643 6.37660471033049e-05
644 6.3547049257077e-05
645 6.33285398829268e-05
646 6.31105751196515e-05
647 6.28933280696723e-05
648 6.26768308921536e-05
649 6.24611194742404e-05
650 6.22463100228288e-05
651 6.20324187323718e-05
652 6.18195700129813e-05
653 6.16078214115134e-05
654 6.13971797373362e-05
655 6.11877212139215e-05
656 6.09794617716375e-05
657 6.07725329828668e-05
658 6.05669678321874e-05
659 6.03628534451654e-05
660 6.01602502345126e-05
661 5.99591192553817e-05
662 5.97595393164359e-05
663 5.95615324409475e-05
664 5.93651486973101e-05
665 5.91704922735895e-05
666 5.89776428417584e-05
667 5.87865065793854e-05
668 5.85971673518297e-05
669 5.84096578973477e-05
670 5.82239927275907e-05
671 5.80402435067147e-05
672 5.78583608970007e-05
673 5.76784580156679e-05
674 5.75004688911918e-05
675 5.73244644073156e-05
676 5.71504422595126e-05
677 5.69784506592915e-05
678 5.68084838169132e-05
679 5.66405211657184e-05
680 5.64745950886921e-05
681 5.6310718332971e-05
682 5.61488150031172e-05
683 5.59889402100093e-05
684 5.58310532975762e-05
685 5.56751374851672e-05
686 5.55212564030683e-05
687 5.53693913083464e-05
688 5.52195052551478e-05
689 5.50716050433664e-05
690 5.492562956159e-05
691 5.47816339739882e-05
692 5.46395335424184e-05
693 5.44994238017201e-05
694 5.43611571449295e-05
695 5.42247388099308e-05
696 5.40901433213984e-05
697 5.39572991774169e-05
698 5.38262788083443e-05
699 5.36970033474896e-05
700 5.35693913773135e-05
701 5.34434448885198e-05
702 5.33191160663193e-05
703 5.31964149601549e-05
704 5.30753149755962e-05
705 5.2955744209271e-05
706 5.28377315838213e-05
707 5.27212689197161e-05
708 5.26063343725032e-05
709 5.24929512764061e-05
710 5.23809687346481e-05
711 5.22703940895042e-05
712 5.21612389832171e-05
713 5.20534054899713e-05
714 5.19468825892488e-05
715 5.18416560630897e-05
716 5.17375986971066e-05
717 5.16346815035244e-05
718 5.15329533025503e-05
719 5.14326165728107e-05
720 5.13334823916504e-05
721 5.12354727734513e-05
722 5.11386046942638e-05
723 5.10428018323239e-05
724 5.09481630596535e-05
725 5.08546755367452e-05
726 5.07624330351083e-05
727 5.06713708704846e-05
728 5.05814949782746e-05
729 5.04927033008566e-05
730 5.04049802749762e-05
731 5.0318365533523e-05
732 5.02328187685919e-05
733 5.01482831332112e-05
734 5.00647716184706e-05
735 4.99822163663547e-05
736 4.99005715835684e-05
737 4.9819832433684e-05
738 4.97399652523711e-05
739 4.96609205929606e-05
740 4.9582714018707e-05
741 4.95053105270908e-05
742 4.94286285028049e-05
743 4.93526996881618e-05
744 4.9277587565418e-05
745 4.92032342146823e-05
746 4.91295507648222e-05
747 4.90565061011713e-05
748 4.89840397754904e-05
749 4.89122159888685e-05
750 4.884098988119e-05
751 4.87703209183602e-05
752 4.87002734044969e-05
753 4.86307720137802e-05
754 4.85618782046041e-05
755 4.8493472495655e-05
756 4.84256017045936e-05
757 4.83582377495869e-05
758 4.82913516132536e-05
759 4.82249083984717e-05
760 4.81589613094968e-05
761 4.80934769117406e-05
762 4.80284476426599e-05
763 4.79638081483102e-05
764 4.78995582984254e-05
765 4.783570544357e-05
766 4.77722321014321e-05
767 4.77091684955392e-05
768 4.76465687674713e-05
769 4.75842840603017e-05
770 4.75223942763373e-05
771 4.7460827646025e-05
772 4.7399645689931e-05
773 4.7338744350256e-05
774 4.72781968679688e-05
775 4.72178675986849e-05
776 4.71578926285095e-05
777 4.70982187383839e-05
778 4.70388212138554e-05
779 4.69797196730089e-05
780 4.69209178556677e-05
781 4.68623777649668e-05
782 4.6804143614428e-05
783 4.67461553098995e-05
784 4.66883923569602e-05
785 4.66308647713021e-05
786 4.65736212831303e-05
787 4.65166178399793e-05
788 4.64598769157239e-05
789 4.64033022815613e-05
790 4.63470092177213e-05
791 4.62908980149261e-05
792 4.62350106094078e-05
793 4.61793157355099e-05
794 4.61238656939145e-05
795 4.60685847087916e-05
796 4.60135518203231e-05
797 4.59586764804953e-05
798 4.59040000677646e-05
799 4.58495843022414e-05
800 4.579528788895e-05
801 4.5741197756873e-05
802 4.56873165794273e-05
803 4.56335856267032e-05
804 4.55800628209602e-05
805 4.5526680972093e-05
806 4.54735294474299e-05
807 4.54204841175236e-05
808 4.53676694860405e-05
809 4.53150331806521e-05
810 4.52624938714526e-05
811 4.52101935296175e-05
812 4.51580203553921e-05
813 4.51059798892336e-05
814 4.50541522099002e-05
815 4.50024313938258e-05
816 4.49508910840279e-05
817 4.48994920129545e-05
818 4.48483096029406e-05
819 4.47972778238418e-05
820 4.47464666167482e-05
821 4.46957677997517e-05
822 4.46452027637415e-05
823 4.45948743722132e-05
824 4.45445264526019e-05
825 4.44942815841879e-05
826 4.44441933614333e-05
827 4.43943203075984e-05
828 4.4344680929953e-05
829 4.42952551574412e-05
830 4.42459781944867e-05
831 4.41969626354677e-05
832 4.41480795790502e-05
833 4.40993758203945e-05
834 4.4050811457789e-05
835 4.40023832689225e-05
836 4.39540868549434e-05
837 4.39059810630008e-05
838 4.38579713583446e-05
839 4.38100960410045e-05
840 4.37624177216378e-05
841 4.37148391008909e-05
842 4.36674131603828e-05
843 4.36201336118103e-05
844 4.35729828455559e-05
845 4.35259606709574e-05
846 4.34790650789552e-05
847 4.34323093821594e-05
848 4.33857198807175e-05
849 4.33391858886504e-05
850 4.32928452855919e-05
851 4.32466055446667e-05
852 4.32004831771119e-05
853 4.31545006710129e-05
854 4.31086287123369e-05
855 4.30628913736797e-05
856 4.30173329940923e-05
857 4.29718088502303e-05
858 4.29264092775365e-05
859 4.2881131444498e-05
860 4.28359194402835e-05
861 4.27908414467974e-05
862 4.27458451639495e-05
863 4.27009607732269e-05
864 4.26561222823807e-05
865 4.26114397260591e-05
866 4.25668338088769e-05
867 4.25223438581905e-05
868 4.24779076724964e-05
869 4.24336107016643e-05
870 4.23893509372183e-05
871 4.23452054908087e-05
872 4.23011562199112e-05
873 4.22572182105322e-05
874 4.22133748134712e-05
875 4.21696564796292e-05
876 4.2126002221939e-05
877 4.20824773712525e-05
878 4.20389942436354e-05
879 4.19956508519448e-05
880 4.19523268782029e-05
881 4.19091001078916e-05
882 4.18659986773188e-05
883 4.1822930753573e-05
884 4.17799391930392e-05
885 4.17370346301738e-05
886 4.16941914960963e-05
887 4.16514206798742e-05
888 4.16087311106613e-05
889 4.15660624977221e-05
890 4.15234333785245e-05
891 4.14809049390878e-05
892 4.14384487582945e-05
893 4.1396009700397e-05
894 4.13536554694597e-05
895 4.13113759651177e-05
896 4.1269093012275e-05
897 4.12269170008509e-05
898 4.11847443793079e-05
899 4.11426646061604e-05
900 4.11006073980739e-05
901 4.10586100644631e-05
902 4.10166574897156e-05
903 4.09747158898928e-05
904 4.09328151723306e-05
905 4.08909749924173e-05
906 4.08491841188408e-05
907 4.0807390824682e-05
908 4.07656408795513e-05
909 4.07239112446926e-05
910 4.06821872391087e-05
911 4.06405172318121e-05
912 4.05988773000132e-05
913 4.05572316649246e-05
914 4.05156201749672e-05
915 4.04740436182512e-05
916 4.0432501637729e-05
917 4.03909757107357e-05
918 4.03495300507946e-05
919 4.0308022406658e-05
920 4.02664624245356e-05
921 4.02249257576888e-05
922 4.01834754081657e-05
923 4.01420577773592e-05
924 4.01007389824552e-05
925 4.0059479032332e-05
926 4.0018280661395e-05
927 3.99771443871562e-05
928 3.99360511422013e-05
929 3.98950284369685e-05
930 3.98540414199289e-05
931 3.98131060664516e-05
932 3.97721492155038e-05
933 3.97312788950425e-05
934 3.96904576831503e-05
935 3.96497221212636e-05
936 3.96089924551054e-05
937 3.95682684845392e-05
938 3.95275858545337e-05
939 3.94867887122767e-05
940 3.94460370708079e-05
941 3.94053156108261e-05
942 3.93646402550019e-05
943 3.93240177161876e-05
944 3.92834208042814e-05
945 3.92428623611588e-05
946 3.92023392124704e-05
947 3.91618500425276e-05
948 3.91213806114621e-05
949 3.9081009916823e-05
950 3.9040631227986e-05
951 3.90002903631152e-05
952 3.89600320887714e-05
953 3.89198813888214e-05
954 3.88797669833944e-05
955 3.8839706951066e-05
956 3.8799654631679e-05
957 3.87597137431565e-05
958 3.8719815561213e-05
959 3.86799753731755e-05
960 3.86401990499034e-05
961 3.86004880012318e-05
962 3.85608281773623e-05
963 3.85212092422667e-05
964 3.84816697926264e-05
965 3.84421498083043e-05
966 3.84027291371325e-05
967 3.83634166182626e-05
968 3.83241693734012e-05
969 3.82850077228862e-05
970 3.82459187437216e-05
971 3.82069042679234e-05
972 3.81679808221236e-05
973 3.81291261764005e-05
974 3.8090359530211e-05
975 3.80516384244819e-05
976 3.8012997735611e-05
977 3.79744589427133e-05
978 3.79359145246857e-05
979 3.78974459642005e-05
980 3.78591563432451e-05
981 3.78210114485948e-05
982 3.77829856551178e-05
983 3.77450784257623e-05
984 3.77072714551711e-05
985 3.76695441814216e-05
986 3.76319209654715e-05
987 3.75943931102777e-05
988 3.75569464310388e-05
989 3.75196024761474e-05
990 3.74823080206236e-05
991 3.74450674485161e-05
992 3.74079107678208e-05
993 3.73708162803391e-05
994 3.73338472208218e-05
995 3.72969454674651e-05
996 3.7260117108436e-05
997 3.72233477611653e-05
998 3.71866289127591e-05
999 3.71500116168969e-05
1000 3.71134427853785e-05
1001 3.70769369890667e-05
1002 3.7040482302686e-05
1003 3.70040973140343e-05
1004 3.69677557845449e-05
1005 3.69314813729223e-05
1006 3.68952635406335e-05
1007 3.68591170284797e-05
1008 3.68230224854216e-05
1009 3.67869380865452e-05
1010 3.67508986241679e-05
1011 3.67149092141972e-05
1012 3.66789748505643e-05
1013 3.6643110381244e-05
1014 3.66072745083083e-05
1015 3.65714072051067e-05
1016 3.65356405899557e-05
1017 3.64999296470122e-05
1018 3.64642615006498e-05
1019 3.64286275858679e-05
1020 3.63930754113303e-05
1021 3.63575485593515e-05
1022 3.63220740850304e-05
1023 3.62866325017327e-05
1024 3.62512081218587e-05
1025 3.621585049333e-05
1026 3.61805479075864e-05
1027 3.61452888171243e-05
1028 3.61100836858697e-05
1029 3.60748733131781e-05
1030 3.60396881286344e-05
1031 3.60045532961095e-05
1032 3.59694421282105e-05
1033 3.59343526733132e-05
1034 3.58993260298016e-05
1035 3.58643158045642e-05
1036 3.58293759911514e-05
1037 3.57944373643448e-05
1038 3.57595064013339e-05
1039 3.57246164958506e-05
1040 3.56897756776201e-05
1041 3.56549627221388e-05
1042 3.5620168122937e-05
1043 3.55853852358479e-05
1044 3.55506583093283e-05
1045 3.55159531419957e-05
1046 3.54812556112213e-05
1047 3.54466139326585e-05
1048 3.54119592487952e-05
1049 3.53773152574159e-05
1050 3.53427303796394e-05
1051 3.53081152919543e-05
1052 3.52735109589256e-05
1053 3.52389450905359e-05
1054 3.52043951086974e-05
1055 3.51698542780573e-05
1056 3.51353570457273e-05
1057 3.51008524450691e-05
1058 3.50663303348616e-05
1059 3.50318326252837e-05
1060 3.49973382226902e-05
1061 3.49628668736557e-05
1062 3.492841428591e-05
1063 3.48939608188677e-05
1064 3.48594837695041e-05
1065 3.48250349097394e-05
1066 3.47905778011655e-05
1067 3.47560935806494e-05
1068 3.47216546394691e-05
1069 3.46872046339541e-05
1070 3.4652767032739e-05
1071 3.46182831657179e-05
1072 3.45838395764038e-05
1073 3.45494186711666e-05
1074 3.45149640577821e-05
1075 3.44805192392291e-05
1076 3.44460831875892e-05
1077 3.44116047147717e-05
1078 3.43771358076358e-05
1079 3.43426851833565e-05
1080 3.43081796193208e-05
1081 3.42736731523038e-05
1082 3.42391938132162e-05
1083 3.42046751959183e-05
1084 3.41701827076453e-05
1085 3.41356752316576e-05
1086 3.41011261439187e-05
1087 3.40665716015801e-05
1088 3.40320036874194e-05
1089 3.39974241914121e-05
1090 3.39628110512062e-05
1091 3.39282147976405e-05
1092 3.3893578307224e-05
1093 3.38589238252733e-05
1094 3.38242323110194e-05
1095 3.37895423350905e-05
1096 3.37548394545214e-05
1097 3.37200618751865e-05
1098 3.36853019096139e-05
1099 3.36505239329673e-05
1100 3.36157262044168e-05
1101 3.35808345569196e-05
1102 3.35460027229099e-05
1103 3.35111329832216e-05
1104 3.34761994071897e-05
1105 3.34412312312793e-05
1106 3.3406238524473e-05
1107 3.33712333748792e-05
1108 3.33361982208421e-05
1109 3.33011131553225e-05
1110 3.32659560040582e-05
1111 3.32308549377084e-05
1112 3.31956824606294e-05
1113 3.31604466836666e-05
1114 3.31252119979789e-05
1115 3.30899506935841e-05
1116 3.30545870692589e-05
1117 3.30192419220054e-05
1118 3.29838145205485e-05
1119 3.29483838470424e-05
1120 3.29129054872368e-05
1121 3.28773676496752e-05
1122 3.28418040040172e-05
1123 3.28062296037028e-05
1124 3.277054112516e-05
1125 3.27348362656465e-05
1126 3.26991151560208e-05
1127 3.26633379306903e-05
1128 3.26274869249706e-05
1129 3.25916256927646e-05
1130 3.25556878273403e-05
1131 3.25197175901811e-05
1132 3.24836746633158e-05
1133 3.2447588435384e-05
1134 3.24114768286421e-05
1135 3.23753191562304e-05
1136 3.23390676690849e-05
1137 3.23028108197102e-05
1138 3.22665052312487e-05
1139 3.22301048546099e-05
1140 3.21936648468579e-05
1141 3.21571654258908e-05
1142 3.21205730825132e-05
1143 3.20839468172333e-05
1144 3.20473040217687e-05
1145 3.20105830837747e-05
1146 3.19738328974732e-05
1147 3.19370067215866e-05
1148 3.19001252056713e-05
1149 3.18631511966316e-05
1150 3.18261509040241e-05
1151 3.17890972324856e-05
1152 3.17519863569279e-05
1153 3.17148191678977e-05
1154 3.16776552870361e-05
1155 3.16403618268642e-05
1156 3.16030014365272e-05
1157 3.15656371997856e-05
1158 3.15281920103179e-05
1159 3.14906511006778e-05
1160 3.14530711958601e-05
1161 3.14154392550373e-05
1162 3.1377754783198e-05
1163 3.13400126602896e-05
1164 3.13022277911301e-05
1165 3.1264388593281e-05
1166 3.12264956825459e-05
1167 3.11885346248412e-05
1168 3.11505120335435e-05
1169 3.11124589931211e-05
1170 3.10743329018e-05
1171 3.10361296200767e-05
1172 3.09979133176578e-05
1173 3.09596078977847e-05
1174 3.0921230039264e-05
1175 3.08828391636003e-05
1176 3.08443233674183e-05
1177 3.0805796792895e-05
1178 3.07671600670763e-05
1179 3.07285078247806e-05
1180 3.06897631900208e-05
1181 3.06509813050582e-05
1182 3.06121332792249e-05
1183 3.0573176942994e-05
1184 3.05341911672012e-05
1185 3.04951400208845e-05
1186 3.04560236449684e-05
1187 3.04168114588776e-05
1188 3.03775660637958e-05
1189 3.03382566890775e-05
1190 3.0298863045175e-05
1191 3.02594060255975e-05
1192 3.02198946903568e-05
1193 3.01802975754365e-05
1194 3.01406535344976e-05
1195 3.01009257418864e-05
1196 3.00611131140253e-05
1197 3.00212426089056e-05
1198 2.99813287381786e-05
1199 2.99413252283405e-05
1200 2.99012604276783e-05
1201 2.98611060269849e-05
1202 2.98208998833852e-05
1203 2.97806059362434e-05
1204 2.97402433749501e-05
1205 2.96998176298284e-05
1206 2.96593053127718e-05
1207 2.96187588150569e-05
1208 2.9578082102096e-05
1209 2.95373842978582e-05
1210 2.94966081704947e-05
1211 2.94557739864606e-05
1212 2.94148503859522e-05
1213 2.93738985372253e-05
1214 2.93328717315688e-05
1215 2.92917341451944e-05
1216 2.92505983825464e-05
1217 2.92093478666544e-05
1218 2.91680374804315e-05
1219 2.91266555674241e-05
1220 2.90852128118265e-05
1221 2.90437051374918e-05
1222 2.90021081325402e-05
1223 2.89604331911164e-05
1224 2.89187491097446e-05
1225 2.88769694704409e-05
1226 2.88351460990081e-05
1227 2.87931857082905e-05
1228 2.87512284617245e-05
1229 2.87091236845782e-05
1230 2.8667024866517e-05
1231 2.86248143689013e-05
1232 2.85825606466024e-05
1233 2.85402394884689e-05
1234 2.84978682720075e-05
1235 2.8455379075254e-05
1236 2.84128804792042e-05
1237 2.83703059800189e-05
1238 2.83276638075591e-05
1239 2.82849650261596e-05
1240 2.82422130262934e-05
1241 2.81993917106149e-05
1242 2.81564756813661e-05
1243 2.81135406462833e-05
1244 2.80705199144175e-05
1245 2.80274668096325e-05
1246 2.79843532974648e-05
1247 2.79412000339837e-05
1248 2.78979636541739e-05
1249 2.78546900685702e-05
1250 2.68890829104862e-05
1251 2.64672552123812e-05
1252 2.64075568476111e-05
1253 2.63866190230999e-05
1254 2.63743742442794e-05
1255 2.63652128295414e-05
1256 2.63575585967383e-05
1257 2.63507659064762e-05
1258 2.63444905736302e-05
1259 2.6338570909464e-05
1260 2.63329096844937e-05
1261 2.63274058690162e-05
1262 2.63220385683383e-05
1263 2.63167733690987e-05
1264 2.63115945375129e-05
1265 2.63064722355229e-05
1266 2.63013950168774e-05
1267 2.62963594236017e-05
1268 2.62913559841612e-05
1269 2.62863742085765e-05
1270 2.62814149127877e-05
1271 2.62764773542775e-05
1272 2.62715507304279e-05
1273 2.6266631273586e-05
1274 2.62617159485501e-05
1275 2.62568167794797e-05
1276 2.62519199800693e-05
1277 2.62470214196971e-05
1278 2.62421246747617e-05
1279 2.62372240650658e-05
1280 2.62323198185754e-05
1281 2.62274175308145e-05
1282 2.6222512547136e-05
1283 2.62175973464451e-05
1284 2.62126781252666e-05
1285 2.62077567973288e-05
1286 2.62028207818806e-05
1287 2.61978872829379e-05
1288 2.61929421399761e-05
1289 2.61879934762751e-05
1290 2.61830388665156e-05
1291 2.61780782283931e-05
1292 2.61731036381718e-05
1293 2.61681209015781e-05
1294 2.61631261218772e-05
1295 2.6158124023586e-05
1296 2.61531177396061e-05
1297 2.61481061241871e-05
1298 2.61430795038819e-05
1299 2.61380451522797e-05
1300 2.61330139513423e-05
1301 2.61279649027557e-05
1302 2.61229061907879e-05
1303 2.61178468973592e-05
1304 2.61127736086072e-05
1305 2.6107702574644e-05
1306 2.61026151407625e-05
1307 2.60975240706784e-05
1308 2.60924273446742e-05
1309 2.6087319781117e-05
1310 2.60822090017617e-05
1311 2.60771007797681e-05
1312 2.60719822945769e-05
1313 2.60668465887903e-05
1314 2.60617156708776e-05
1315 2.60565722266885e-05
1316 2.60514323334367e-05
1317 2.6046279392844e-05
1318 2.60411235532369e-05
1319 2.60359709098879e-05
1320 2.60308070725301e-05
1321 2.60256397475918e-05
1322 2.60204674666179e-05
1323 2.60152903089524e-05
1324 2.60101165713659e-05
1325 2.60049327221642e-05
1326 2.59997462670469e-05
1327 2.59945626019942e-05
1328 2.59893677304982e-05
1329 2.59841773413427e-05
1330 2.59789819647693e-05
1331 2.59737823527691e-05
1332 2.59685861436765e-05
1333 2.59633799336948e-05
1334 2.59581802755093e-05
1335 2.59529658676409e-05
1336 2.5947754081083e-05
1337 2.59425438109417e-05
1338 2.59373313854875e-05
1339 2.59321193455027e-05
1340 2.59268987162405e-05
1341 2.59216818706183e-05
1342 2.59164536130877e-05
1343 2.59112355968464e-05
1344 2.59060109370305e-05
1345 2.59007925009769e-05
1346 2.58955691793498e-05
1347 2.58903453940936e-05
1348 2.58851135654936e-05
1349 2.58798900727442e-05
1350 2.58746628502375e-05
1351 2.58694337897936e-05
1352 2.58642105096148e-05
1353 2.58589823474153e-05
1354 2.58537567224456e-05
1355 2.58485294502009e-05
1356 2.58432990788056e-05
1357 2.5838067471066e-05
1358 2.58328353979209e-05
1359 2.58276042390075e-05
1360 2.58223731156212e-05
1361 2.58171447840757e-05
1362 2.58119077714743e-05
1363 2.58066767102605e-05
1364 2.58014423219303e-05
1365 2.57962098271965e-05
1366 2.57909789673031e-05
1367 2.57857455376135e-05
1368 2.57805175281807e-05
1369 2.57752824056231e-05
1370 2.57700495787105e-05
1371 2.57648124991997e-05
1372 2.57595774986186e-05
1373 2.5754339574154e-05
1374 2.57491115801164e-05
1375 2.57438795987497e-05
1376 2.5738643256427e-05
1377 2.57334063172484e-05
1378 2.57281790586224e-05
1379 2.57229410062602e-05
1380 2.57177136688824e-05
1381 2.57124724356572e-05
1382 2.57072471434583e-05
1383 2.5702016259288e-05
1384 2.5696788462426e-05
1385 2.56915523758986e-05
1386 2.56863233989435e-05
1387 2.56810887536337e-05
1388 2.56758548313011e-05
1389 2.56706302336577e-05
1390 2.56653933874418e-05
1391 2.56601595272817e-05
1392 2.56549276015742e-05
1393 2.56496956361948e-05
1394 2.56444694001582e-05
1395 2.56392373394476e-05
1396 2.56340057796696e-05
1397 2.56287773995704e-05
1398 2.56235429928845e-05
1399 2.56183118777879e-05
1400 2.56130863534783e-05
1401 2.56078506074194e-05
1402 2.56026240303224e-05
1403 2.55973852650489e-05
1404 2.55921620653983e-05
1405 2.55869291292399e-05
1406 2.55817004924571e-05
1407 2.55764708117686e-05
1408 2.55712380973587e-05
1409 2.55660022195533e-05
1410 2.55607768219572e-05
1411 2.55555392275989e-05
1412 2.55503098767207e-05
1413 2.55450766003899e-05
1414 2.55398475615583e-05
1415 2.55346169586446e-05
1416 2.55293861396074e-05
1417 2.55241515780824e-05
1418 2.55189248905552e-05
1419 2.55136958505394e-05
1420 2.55084616001729e-05
1421 2.55032280541319e-05
1422 2.549799516031e-05
1423 2.54927650399731e-05
1424 2.54875377082205e-05
1425 2.54823033598583e-05
1426 2.54770782900001e-05
1427 2.54718463027122e-05
1428 2.54666140104831e-05
1429 2.5461383898436e-05
1430 2.54561536297733e-05
1431 2.5450920341008e-05
1432 2.54456910899018e-05
1433 2.54404596417383e-05
1434 2.54352286720068e-05
1435 2.54299968709404e-05
1436 2.54247689322658e-05
1437 2.54195354300416e-05
1438 2.54143030599489e-05
1439 2.54090750389698e-05
1440 2.54038424977547e-05
1441 2.53986166039629e-05
1442 2.53933822209618e-05
1443 2.53881511493859e-05
1444 2.53829121120338e-05
1445 2.53776845798489e-05
1446 2.53724535227799e-05
1447 2.53672189751697e-05
1448 2.53619952212958e-05
1449 2.53567581548116e-05
1450 2.53515237407242e-05
1451 2.5346296774309e-05
1452 2.53410622607457e-05
1453 2.53358384275278e-05
1454 2.53306073855579e-05
1455 2.53253674268687e-05
1456 2.53201361379851e-05
1457 2.53149014417531e-05
1458 2.53096718481061e-05
1459 2.53044381478181e-05
1460 2.52992052942602e-05
1461 2.52939724566896e-05
1462 2.52887410458295e-05
1463 2.52835106241041e-05
1464 2.52782790557404e-05
1465 2.52730424419312e-05
1466 2.5267814390754e-05
1467 2.52625778938883e-05
1468 2.52573441518559e-05
1469 2.5252113951583e-05
1470 2.52468800961599e-05
1471 2.52416419129394e-05
1472 2.52364109935381e-05
1473 2.52311773708177e-05
1474 2.52259477798352e-05
1475 2.52207066552638e-05
1476 2.52154709619035e-05
1477 2.52102345947118e-05
1478 2.5205003339579e-05
1479 2.51997674511154e-05
1480 2.51945334740119e-05
1481 2.51893046370929e-05
1482 2.51840685911257e-05
1483 2.51788303637923e-05
1484 2.51736030465466e-05
1485 2.51683641797248e-05
1486 2.51631279208908e-05
1487 2.51578976936765e-05
1488 2.51526646968259e-05
1489 2.51474283577598e-05
1490 2.51421986391757e-05
1491 2.51369659774644e-05
1492 2.51317261158827e-05
1493 2.5126490699634e-05
1494 2.51212526241792e-05
1495 2.5116015657467e-05
1496 2.511078209159e-05
1497 2.51055500773608e-05
1498 2.51003147564433e-05
1499 2.50950748785783e-05
1500 2.50898363403825e-05
1501 2.50845992516939e-05
1502 2.5079362769039e-05
1503 2.50741325672858e-05
1504 2.50688920425309e-05
1505 2.50636571719198e-05
1506 2.50584152243031e-05
1507 2.50531811281836e-05
1508 2.50479343888443e-05
1509 2.50427016646645e-05
1510 2.50374658872232e-05
1511 2.50322272741244e-05
1512 2.50269913436204e-05
1513 2.50217494954796e-05
1514 2.50165159494387e-05
1515 2.50112762252286e-05
1516 2.5006040251796e-05
1517 2.50008086402076e-05
1518 2.49955647613949e-05
1519 2.49903305077718e-05
1520 2.49850929818033e-05
1521 2.49798605983879e-05
1522 2.49746146003815e-05
1523 2.49693754250657e-05
1524 2.49641398471686e-05
1525 2.49588982574878e-05
1526 2.49536651969843e-05
1527 2.49484257868933e-05
1528 2.49431794664782e-05
1529 2.4937944928934e-05
1530 2.49327062681696e-05
1531 2.49274645994409e-05
1532 2.4922230909515e-05
1533 2.49169866277654e-05
1534 2.49117446271541e-05
1535 2.4906501204273e-05
1536 2.49012652320246e-05
1537 2.48960249518149e-05
1538 2.48907829805134e-05
1539 2.48855412960935e-05
1540 2.48802987190544e-05
1541 2.4875059187283e-05
1542 2.48698197111707e-05
1543 2.48645728652501e-05
1544 2.48593327398794e-05
1545 2.48540919812375e-05
1546 2.48488444863545e-05
1547 2.48436071368374e-05
1548 2.48383566425758e-05
1549 2.48331186606758e-05
1550 2.48278777471948e-05
1551 2.48226331522143e-05
1552 2.48173885892082e-05
1553 2.48121400877229e-05
1554 2.48068937326688e-05
1555 2.48016528502741e-05
1556 2.47964075998179e-05
1557 2.47911612604549e-05
1558 2.47859178168497e-05
1559 2.47806771129788e-05
1560 2.47754294120384e-05
1561 2.47701831534997e-05
1562 2.47649450179447e-05
1563 2.47596945725329e-05
1564 2.47544453986966e-05
1565 2.47492062186438e-05
1566 2.47439608180855e-05
1567 2.47387160348111e-05
1568 2.47334705164211e-05
1569 2.47282219199008e-05
1570 2.47229794781608e-05
1571 2.47177360875502e-05
1572 2.47124868482847e-05
1573 2.47072461154e-05
1574 2.47019958337091e-05
1575 2.4696756457665e-05
1576 2.46915028115543e-05
1577 2.46862589043199e-05
1578 2.46810072962826e-05
1579 2.46757636164219e-05
1580 2.46705140523792e-05
1581 2.4665269558947e-05
1582 2.46600244532639e-05
1583 2.46547788513851e-05
1584 2.46495348846428e-05
1585 2.46442816651538e-05
1586 2.46390355679675e-05
1587 2.4633788404375e-05
1588 2.46285437729791e-05
1589 2.46232942033113e-05
1590 2.4618040723882e-05
1591 2.46127983789535e-05
1592 2.46075488519182e-05
1593 2.46022977122469e-05
1594 2.45970494739585e-05
1595 2.45917988822934e-05
1596 2.45865523990455e-05
1597 2.45813009183138e-05
1598 2.45760544137497e-05
1599 2.45708039037969e-05
1600 2.45655593523371e-05
1601 2.45603078594669e-05
1602 2.45550506843273e-05
1603 2.45498104109269e-05
1604 2.4544555723575e-05
1605 2.45393018213728e-05
1606 2.4534053491306e-05
1607 2.45288064594599e-05
1608 2.45235510885067e-05
1609 2.45183020715819e-05
1610 2.45130535934853e-05
1611 2.45078085988245e-05
1612 2.45025549260684e-05
1613 2.4497307738495e-05
1614 2.44920474375595e-05
1615 2.44868018028181e-05
1616 2.44815494279867e-05
1617 2.44762963541589e-05
1618 2.44710466240268e-05
1619 2.44657962028919e-05
1620 2.44605414521833e-05
1621 2.44552894551238e-05
1622 2.44500397350578e-05
1623 2.44447895741591e-05
1624 2.4439538255964e-05
1625 2.44342809653612e-05
1626 2.4429030393236e-05
1627 2.44237804141179e-05
1628 2.44185237147458e-05
1629 2.44132713834351e-05
1630 2.44080192818667e-05
1631 2.44027681543339e-05
1632 2.4397517588722e-05
1633 2.43922655203122e-05
1634 2.43870121480564e-05
1635 2.43817550954854e-05
1636 2.4376504298355e-05
1637 2.43712507084955e-05
1638 2.43660006902976e-05
1639 2.43607494363474e-05
1640 2.43554911089442e-05
1641 2.43502432075715e-05
1642 2.43449884660407e-05
1643 2.43397342514958e-05
1644 2.43344890780151e-05
1645 2.43292319644558e-05
1646 2.43239827225257e-05
1647 2.43187346532873e-05
1648 2.43134797880037e-05
1649 2.43082236544012e-05
1650 2.43029767987106e-05
1651 2.42977245269079e-05
1652 2.42924709441539e-05
1653 2.42872209795427e-05
1654 2.42819660029407e-05
1655 2.42767129403632e-05
1656 2.42714641330484e-05
1657 2.42662094809276e-05
1658 2.42609648045307e-05
1659 2.42557020969277e-05
1660 2.42504552619612e-05
1661 2.42452055868962e-05
1662 2.42399517815054e-05
1663 2.42346971965901e-05
1664 2.42294416186913e-05
1665 2.42241997116347e-05
1666 2.42189485814374e-05
1667 2.42136923856625e-05
1668 2.42084387114261e-05
1669 2.42031905501131e-05
1670 2.41979409922877e-05
1671 2.41926862729613e-05
1672 2.41874333385776e-05
1673 2.41821889073179e-05
1674 2.41769367527548e-05
1675 2.41716850911307e-05
1676 2.41664259913084e-05
1677 2.41611773104111e-05
1678 2.41559319832755e-05
1679 2.41506753739055e-05
1680 2.41454250256012e-05
1681 2.41401777871057e-05
1682 2.41349185522803e-05
1683 2.4129664420928e-05
1684 2.41244124679814e-05
1685 2.41191676820425e-05
1686 2.41139079542781e-05
1687 2.41086587754088e-05
1688 2.41034044687893e-05
1689 2.4098153353691e-05
1690 2.40928997937336e-05
1691 2.4087650459137e-05
1692 2.40824024813809e-05
1693 2.40771463480745e-05
1694 2.40718906304357e-05
1695 2.40666421203647e-05
1696 2.40613869305998e-05
1697 2.4056133488178e-05
1698 2.40508881237389e-05
1699 2.40456346899028e-05
1700 2.40403891291763e-05
1701 2.40351338106256e-05
1702 2.40298855113489e-05
1703 2.40246363816254e-05
1704 2.40193826215318e-05
1705 2.40141310889896e-05
1706 2.40088825496088e-05
1707 2.40036368523988e-05
1708 2.39983853823252e-05
1709 2.39931377903346e-05
1710 2.3987886424474e-05
1711 2.39826469394799e-05
1712 2.39773992500858e-05
1713 2.39721473616801e-05
1714 2.39669009050777e-05
1715 2.39616548635505e-05
1716 2.39564040477684e-05
1717 2.39511521848238e-05
1718 2.39459096977868e-05
1719 2.39406639674182e-05
1720 2.39354204483178e-05
1721 2.39301665120687e-05
1722 2.39249215416848e-05
1723 2.39196783562434e-05
1724 2.39144273068703e-05
1725 2.39091823051041e-05
1726 2.39039321048296e-05
1727 2.38986917686645e-05
1728 2.38934456883536e-05
1729 2.38882009530409e-05
1730 2.38829523905674e-05
1731 2.38777092909833e-05
1732 2.38724607572275e-05
1733 2.38672214054603e-05
1734 2.38619738380426e-05
1735 2.38567256095242e-05
1736 2.38514798723462e-05
1737 2.38462343948124e-05
1738 2.38409913473348e-05
1739 2.38357456048277e-05
1740 2.38304993193476e-05
1741 2.38252608311858e-05
1742 2.38200147618291e-05
1743 2.38147734856753e-05
1744 2.3809529678213e-05
1745 2.38042866064584e-05
1746 2.37990453545815e-05
1747 2.37937998095461e-05
1748 2.37885588303399e-05
1749 2.37833181622923e-05
1750 2.37780777399739e-05
1751 2.37728325386636e-05
1752 2.37675917666991e-05
1753 2.37623544251756e-05
1754 2.37571122827518e-05
1755 2.37518737460363e-05
1756 2.37466291800696e-05
1757 2.37413996127718e-05
1758 2.37361555912585e-05
1759 2.37309174864938e-05
1760 2.37256824838402e-05
1761 2.372044652225e-05
1762 2.37152065262573e-05
1763 2.37099654378052e-05
1764 2.37047330760021e-05
1765 2.36994936673914e-05
1766 2.36942535547513e-05
1767 2.36890143957188e-05
1768 2.36837825194532e-05
1769 2.36785458739656e-05
1770 2.36733074796547e-05
1771 2.36680754526949e-05
1772 2.36628391260633e-05
1773 2.3657604378909e-05
1774 2.36523698727472e-05
1775 2.36471329732405e-05
1776 2.3641901779096e-05
1777 2.36366681877579e-05
1778 2.36314361107641e-05
1779 2.36262013565044e-05
1780 2.36209650603669e-05
1781 2.36157380001008e-05
1782 2.3610506117322e-05
1783 2.36052670580615e-05
1784 2.36000370001932e-05
1785 2.35948053394589e-05
1786 2.35895746332204e-05
1787 2.35843455621184e-05
1788 2.35791105798929e-05
1789 2.35738869811541e-05
1790 2.35686579805143e-05
1791 2.35634286408863e-05
1792 2.35581944766731e-05
1793 2.35529628597556e-05
1794 2.35477367264518e-05
1795 2.35425155151366e-05
1796 2.35372776238307e-05
1797 2.35320527247988e-05
1798 2.35268359339476e-05
1799 2.3521600129565e-05
1800 2.35163741842589e-05
1801 2.35111503705809e-05
1802 2.35059256243157e-05
1803 2.35006987407971e-05
1804 2.34954731063534e-05
1805 2.34902462509604e-05
1806 2.34850237861295e-05
1807 2.34797955795211e-05
1808 2.34745747456818e-05
1809 2.3469346621674e-05
1810 2.34641221092957e-05
1811 2.34589057308554e-05
1812 2.34536784645319e-05
1813 2.34484644664098e-05
1814 2.34432419358536e-05
1815 2.34380174667e-05
1816 2.34328001177767e-05
1817 2.34275831954752e-05
1818 2.34223626449648e-05
1819 2.34171401523042e-05
1820 2.34119228587441e-05
1821 2.3406697202392e-05
1822 2.3401480755858e-05
1823 2.33962648821991e-05
1824 2.33910485301081e-05
1825 2.33858304961032e-05
1826 2.33806110697306e-05
1827 2.33753946344469e-05
1828 2.33701862842513e-05
1829 2.33649653604099e-05
1830 2.33597508918493e-05
1831 2.33545416591004e-05
1832 2.33493258647854e-05
1833 2.33441055899064e-05
1834 2.33388915124403e-05
1835 2.33336816434597e-05
1836 2.33284718570796e-05
1837 2.33232599479531e-05
1838 2.33180455605127e-05
1839 2.33128331101895e-05
1840 2.33076229439651e-05
1841 2.33024120600926e-05
1842 2.32972049971636e-05
1843 2.32919903192889e-05
1844 2.32867842466788e-05
1845 2.32815761262254e-05
1846 2.32763630449995e-05
1847 2.32711591573083e-05
1848 2.32659506081608e-05
1849 2.32607396322138e-05
1850 2.3255539997713e-05
1851 2.32503251892761e-05
1852 2.32451221483151e-05
1853 2.32399237510468e-05
1854 2.32347155746382e-05
1855 2.32295126281201e-05
1856 2.32243109105449e-05
1857 2.3219112076589e-05
1858 2.3213912719496e-05
1859 2.32087107748432e-05
1860 2.32035124172469e-05
1861 2.31983066051692e-05
1862 2.31931117940694e-05
1863 2.31879109501657e-05
1864 2.31827085537262e-05
1865 2.31775142888561e-05
1866 2.31723095070654e-05
1867 2.31671120465293e-05
1868 2.31619142227283e-05
1869 2.31567211388395e-05
1870 2.31515254836741e-05
1871 2.31463288452953e-05
1872 2.31411318362499e-05
1873 2.31359396805075e-05
1874 2.31307413388985e-05
1875 2.31255493806278e-05
1876 2.31203560305815e-05
1877 2.31151633632483e-05
1878 2.31099691827552e-05
1879 2.31047783865179e-05
1880 2.30995871615865e-05
1881 2.30943978648905e-05
1882 2.30892042143438e-05
1883 2.30840190675134e-05
1884 2.30788277060986e-05
1885 2.3073644166279e-05
1886 2.30684583103861e-05
1887 2.30632676911924e-05
1888 2.30580801412472e-05
1889 2.3052899331392e-05
1890 2.30477097975523e-05
1891 2.30425276012506e-05
1892 2.30373392025029e-05
1893 2.30321619512826e-05
1894 2.30269793369449e-05
1895 2.30217981069814e-05
1896 2.30166184976473e-05
1897 2.30114393975356e-05
1898 2.30062595933944e-05
1899 2.30010831222908e-05
1900 2.29959057061653e-05
1901 2.29907286604103e-05
1902 2.29855487718922e-05
1903 2.29803718851211e-05
1904 2.29752028957364e-05
1905 2.29700255450401e-05
1906 2.29648555271448e-05
1907 2.29596797286883e-05
1908 2.29545109577955e-05
1909 2.29493307895012e-05
1910 2.29441672251104e-05
1911 2.29389977019305e-05
1912 2.29338274954453e-05
1913 2.29286627358623e-05
1914 2.29234895527952e-05
1915 2.29183235243013e-05
1916 2.29131574268256e-05
1917 2.29079906567028e-05
1918 2.29028275384735e-05
1919 2.28976597078656e-05
1920 2.28925011989863e-05
1921 2.28873355494486e-05
1922 2.28821784350094e-05
1923 2.2877008904724e-05
1924 2.2871847724204e-05
1925 2.28666903646276e-05
1926 2.2861526358812e-05
1927 2.28563719050593e-05
1928 2.28512121530263e-05
1929 2.2846053679378e-05
1930 2.28408999115951e-05
1931 2.28357380092821e-05
1932 2.28305840887325e-05
1933 2.28254308810942e-05
1934 2.28202788576937e-05
1935 2.28151218755931e-05
1936 2.28099702432871e-05
1937 2.28048141182787e-05
1938 2.27996655048912e-05
1939 2.27945171144365e-05
1940 2.27893658051315e-05
1941 2.27842176384977e-05
1942 2.27790734366925e-05
1943 2.27739311450629e-05
1944 2.27687839577939e-05
1945 2.27636399117159e-05
1946 2.27584913865542e-05
1947 2.27533505633796e-05
1948 2.27482124299054e-05
1949 2.27430712499791e-05
1950 2.27379296481682e-05
1951 2.27327868464281e-05
1952 2.27276517493398e-05
1953 2.27225125450185e-05
1954 2.27173777839577e-05
1955 2.27122461451401e-05
1956 2.27071092243255e-05
1957 2.27019736716016e-05
1958 2.26968378989056e-05
1959 2.26917052786509e-05
1960 2.26865742982696e-05
1961 2.26814487467308e-05
1962 2.26763137988565e-05
1963 2.26711826147863e-05
1964 2.26660537157031e-05
1965 2.26609295855458e-05
1966 2.26558008747565e-05
1967 2.26506725186463e-05
1968 2.26455469505282e-05
1969 2.26404200001085e-05
1970 2.26352986706739e-05
1971 2.26301757540644e-05
1972 2.26250470275839e-05
1973 2.26199311962697e-05
1974 2.26148108897206e-05
1975 2.26096907587348e-05
1976 2.26045724271984e-05
1977 2.25994541202349e-05
1978 2.25943334308809e-05
1979 2.25892173129812e-05
1980 2.25841118810521e-05
1981 2.25789925645259e-05
1982 2.25738775565532e-05
1983 2.2568765135252e-05
1984 2.25636533685384e-05
1985 2.25585455619163e-05
1986 2.25534331758463e-05
1987 2.25483281566241e-05
1988 2.25432265998175e-05
1989 2.25381109520602e-05
1990 2.25330136016666e-05
1991 2.25279081158547e-05
1992 2.25228038204979e-05
1993 2.25177008538561e-05
1994 2.25125984594972e-05
1995 2.25075008838023e-05
1996 2.25024036388059e-05
1997 2.24972994701626e-05
1998 2.24921982336923e-05
1999 2.24871004771051e-05
};
\addlegendentry{Train}
\addplot [semithick, black]
table {%
0 0.029663871973753
1 0.0272275544703007
2 0.0243323985487223
3 0.0203068852424622
4 0.0158795863389969
5 0.012230983003974
6 0.00945578794926405
7 0.00737061630934477
8 0.00581964943557978
9 0.00468314625322819
10 0.00385244190692902
11 0.00324030825868249
12 0.00278383074328303
13 0.00243797921575606
14 0.00217094481922686
15 0.00196010526269674
16 0.00178942526690662
17 0.00164722767658532
18 0.00152513943612576
19 0.00141803489532322
20 0.00132278981618583
21 0.00123703083954751
22 0.0011587671469897
23 0.0010869795223698
24 0.00102105119731277
25 0.000960248580668122
26 0.000903958396520466
27 0.0008518235408701
28 0.000803589296992868
29 0.000759122369345278
30 0.000718160474207252
31 0.000680369965266436
32 0.000645451596938074
33 0.000613161828368902
34 0.000583304965402931
35 0.000555709411855787
36 0.000530232035089284
37 0.000506790936924517
38 0.000485262862639502
39 0.000465491932118312
40 0.000447337864898145
41 0.000430680462159216
42 0.000415398651966825
43 0.000401388038881123
44 0.000388552783988416
45 0.000376807583961636
46 0.000366063817637041
47 0.000356233125785366
48 0.000347231049090624
49 0.000338977028150111
50 0.000331403978634626
51 0.000324448716128245
52 0.000318052305374295
53 0.000312161224428564
54 0.000306727888528258
55 0.000301708147162572
56 0.000297060469165444
57 0.000292747572530061
58 0.000288736715447158
59 0.000284998677670956
60 0.000281505897874013
61 0.000278234860161319
62 0.000275162281468511
63 0.000272270262939855
64 0.00026954262284562
65 0.00026696443092078
66 0.000264521804638207
67 0.000262202433077618
68 0.00025999647914432
69 0.00025789364008233
70 0.000255884719081223
71 0.000253961858106777
72 0.00025211795582436
73 0.000250346638495103
74 0.000248642929363996
75 0.000247001473326236
76 0.000245417759288102
77 0.000243887247052044
78 0.000242406444158405
79 0.000240971974562854
80 0.000239580520428717
81 0.000238229622482322
82 0.000236916093854234
83 0.000235637489822693
84 0.000234391758567654
85 0.000233176106121391
86 0.000231988378800452
87 0.000230827223276719
88 0.000229690616833977
89 0.000228576609515585
90 0.000227483789785765
91 0.000226410702452995
92 0.000225355834118091
93 0.00022431806428358
94 0.000223297756747343
95 0.00022229271417018
96 0.000221301626879722
97 0.000220323447138071
98 0.000219357360037975
99 0.000218402390601113
100 0.000217457854887471
101 0.000216522632399574
102 0.000215595558984205
103 0.000214676460018381
104 0.000213764462387189
105 0.000212859260500409
106 0.000211960257729515
107 0.000211066784686409
108 0.000210178390261717
109 0.000209294637897983
110 0.000208415003726259
111 0.000207539211260155
112 0.000206666838494129
113 0.000205797623493709
114 0.000204930940526538
115 0.000204066789592616
116 0.000203204952413216
117 0.000202344992430881
118 0.000201486836886033
119 0.000200630238396116
120 0.000199774905922823
121 0.000198920999537222
122 0.000198068475583568
123 0.000197217130335048
124 0.000196366978343576
125 0.000195518077816814
126 0.000194670501514338
127 0.000193824322195724
128 0.000192979510757141
129 0.000192136081750505
130 0.000191294166143052
131 0.000190454215044156
132 0.000189616461284459
133 0.000188780977623537
134 0.000187947793165222
135 0.000187117359018885
136 0.000186289849807508
137 0.000185465498361737
138 0.000184644537512213
139 0.000183827301952988
140 0.000183014024514705
141 0.000182205272722058
142 0.000181401497684419
143 0.000180603019543923
144 0.000179809954715893
145 0.000179022666998208
146 0.00017824127280619
147 0.000177465815795586
148 0.000176696295966394
149 0.000175932858837768
150 0.000175174878677353
151 0.000174423475982621
152 0.000173679203726351
153 0.00017294219287578
154 0.000172212632605806
155 0.000171491148648784
156 0.000170777857420035
157 0.000170072642504238
158 0.00016937538748607
159 0.00016868591774255
160 0.000168004116858356
161 0.000167329679243267
162 0.000166662852279842
163 0.000166003606864251
164 0.000165351695613936
165 0.000164707031217404
166 0.000164069337188266
167 0.000163438700838014
168 0.000162814525538124
169 0.000162196811288595
170 0.000161585165187716
171 0.000160978539497592
172 0.000160377778229304
173 0.00015978267765604
174 0.000159193165018223
175 0.000158609342179261
176 0.000158030918100849
177 0.000157457703608088
178 0.000156889509526081
179 0.000156326088472269
180 0.000155767382238992
181 0.00015521320165135
182 0.000154663153807633
183 0.000154117398778908
184 0.000153575630974956
185 0.0001530377776362
186 0.000152503707795404
187 0.000151973377796821
188 0.000151446773088537
189 0.000150923704495654
190 0.000150404215673916
191 0.000149888248415664
192 0.000149375540786423
193 0.000148866194649599
194 0.000148360311868601
195 0.000147857746924274
196 0.000147358543472365
197 0.000146862628753297
198 0.000146370104630478
199 0.000145880796480924
200 0.000145394835271873
201 0.000144912148243748
202 0.000144432691740803
203 0.00014395649486687
204 0.000143483513966203
205 0.000143013763590716
206 0.000142547258292325
207 0.000142083867103793
208 0.000141623764648102
209 0.000141166849061847
210 0.000140713134896941
211 0.000140262505738065
212 0.000139815019792877
213 0.000139370647957548
214 0.000138929302920587
215 0.000138490970130078
216 0.000138055562274531
217 0.000137623108457774
218 0.000137193535920233
219 0.000136766786454245
220 0.000136342758196406
221 0.000135921451146714
222 0.00013550280709751
223 0.000135086753289215
224 0.000134673231514171
225 0.000134262183564715
226 0.000133853594888933
227 0.000133447392727248
228 0.000133043547975831
229 0.000132642002427019
230 0.000132242770632729
231 0.000131845765281469
232 0.000131450986373238
233 0.000131058375700377
234 0.00013066797691863
235 0.000130279848235659
236 0.000129893975099549
237 0.000129510313854553
238 0.000129128995467909
239 0.000128749990835786
240 0.000128373212646693
241 0.000127998762764037
242 0.000127626612083986
243 0.000127256746054627
244 0.000126889150124043
245 0.000126523766084574
246 0.000126160637591965
247 0.000125799953821115
248 0.000125441612908617
249 0.000125085498439148
250 0.000124731595860794
251 0.000124379686894827
252 0.000124030004371889
253 0.000123682402772829
254 0.000123336896649562
255 0.000122993500554003
256 0.000122652214486152
257 0.000122312863823026
258 0.000121975477668457
259 0.000121640077850316
260 0.000121306722576264
261 0.000120975382742472
262 0.000120645956485532
263 0.000120318531116936
264 0.000119993070256896
265 0.000119669515697751
266 0.000119347918371204
267 0.000119028154585976
268 0.000118710267997812
269 0.000118394280434586
270 0.000118080133688636
271 0.000117767747724429
272 0.000117457115266006
273 0.000117148054414429
274 0.00011684067430906
275 0.00011653482215479
276 0.000116230505227577
277 0.000115927556180395
278 0.000115625967737287
279 0.000115325703518465
280 0.000115026647108607
281 0.000114728703920264
282 0.000114431699330453
283 0.000114135662443005
284 0.000113840374979191
285 0.000113545764179435
286 0.000113251793663949
287 0.00011295825970592
288 0.000112665096821729
289 0.000112372115836479
290 0.000112079193058889
291 0.000111786321213003
292 0.000111493303847965
293 0.00011120000272058
294 0.000110906519694254
295 0.000110612745629624
296 0.000110318542283494
297 0.00011002407700289
298 0.000109729866380803
299 0.000109436026832554
300 0.000109142762084957
301 0.000108850297692697
302 0.000108558786450885
303 0.000108268519397825
304 0.000107979678432457
305 0.000107692430901807
306 0.000107407351606525
307 0.000107124200440012
308 0.000106843173853122
309 0.000106564235466067
310 0.000106287232483737
311 0.000106012310425285
312 0.000105739556602202
313 0.00010546908742981
314 0.000105200895632152
315 0.000104935010313056
316 0.000104671511508059
317 0.000104410479252692
318 0.000104151789855678
319 0.000103895443317015
320 0.000103641330497339
321 0.00010338948777644
322 0.000103139835118782
323 0.000102892416180111
324 0.000102647340099793
325 0.000102404417702928
326 0.000102163779956754
327 0.000101925259514246
328 0.00010168892913498
329 0.000101454832474701
330 0.000101223056844901
331 0.000100993333035149
332 0.000100765690149274
333 0.000100540120911319
334 0.000100316538009793
335 0.00010009492689278
336 9.98752730083652e-05
337 9.96575399767607e-05
338 9.94416768662632e-05
339 9.92277928162366e-05
340 9.90156913758256e-05
341 9.8805503512267e-05
342 9.85970400506631e-05
343 9.8390482889954e-05
344 9.81857447186485e-05
345 9.79828109848313e-05
346 9.77816744125448e-05
347 9.75822622422129e-05
348 9.73845963017084e-05
349 9.71898116404191e-05
350 9.69975153566338e-05
351 9.68070453382097e-05
352 9.66182560659945e-05
353 9.64311184361577e-05
354 9.62456688284874e-05
355 9.60618199314922e-05
356 9.5879688160494e-05
357 9.5699040684849e-05
358 9.55199138843454e-05
359 9.53424241743051e-05
360 9.51664333115332e-05
361 9.4992239610292e-05
362 9.48200904531404e-05
363 9.46499203564599e-05
364 9.44816711125895e-05
365 9.43153427215293e-05
366 9.41509133554064e-05
367 9.39883611863479e-05
368 9.38276207307354e-05
369 9.36686628847383e-05
370 9.35114876483567e-05
371 9.33560513658449e-05
372 9.3202390416991e-05
373 9.30503811105154e-05
374 9.29000234464183e-05
375 9.27512664929964e-05
376 9.26038046600297e-05
377 9.24578125705011e-05
378 9.23133411561139e-05
379 9.217035403708e-05
380 9.20288657653145e-05
381 9.18888254091144e-05
382 9.17501602089033e-05
383 9.16129429242574e-05
384 9.14770425879396e-05
385 9.13424737518653e-05
386 9.12092218641192e-05
387 9.10772650968283e-05
388 9.09466471057385e-05
389 9.08171787159517e-05
390 9.06889763427898e-05
391 9.05619890545495e-05
392 9.04361513676122e-05
393 9.03114778338932e-05
394 9.01877065189183e-05
395 9.00651939446107e-05
396 8.9943794591818e-05
397 8.98234357009642e-05
398 8.97042045835406e-05
399 8.95860866876319e-05
400 8.94689583219588e-05
401 8.93528995220549e-05
402 8.92378666321747e-05
403 8.912382327253e-05
404 8.90108203748241e-05
405 8.88988142833114e-05
406 8.87878049979918e-05
407 8.86776906554587e-05
408 8.85685585672036e-05
409 8.84604378370568e-05
410 8.8353204773739e-05
411 8.8246822997462e-05
412 8.81414453033358e-05
413 8.80368606885895e-05
414 8.79332073964179e-05
415 8.78303908393718e-05
416 8.77283455338329e-05
417 8.76272097229958e-05
418 8.75268888194114e-05
419 8.74273464432918e-05
420 8.73286699061282e-05
421 8.72308883117512e-05
422 8.71338634169661e-05
423 8.7037704361137e-05
424 8.69423020048998e-05
425 8.68476636242121e-05
426 8.67537673912011e-05
427 8.66606060299091e-05
428 8.65681940922514e-05
429 8.64764951984398e-05
430 8.63854875206016e-05
431 8.62951637827791e-05
432 8.62055239849724e-05
433 8.61166045069695e-05
434 8.60282889334485e-05
435 8.59407591633499e-05
436 8.58538114698604e-05
437 8.57675913721323e-05
438 8.56819387990981e-05
439 8.55970065458678e-05
440 8.55126709211618e-05
441 8.54290628922172e-05
442 8.53460587677546e-05
443 8.5263600340113e-05
444 8.51817749207839e-05
445 8.51005534059368e-05
446 8.50198848638684e-05
447 8.49398493301123e-05
448 8.48603740450926e-05
449 8.47815172164701e-05
450 8.47032351884991e-05
451 8.46254988573492e-05
452 8.45484028104693e-05
453 8.44718306325376e-05
454 8.4395804151427e-05
455 8.43203088152222e-05
456 8.42453810037114e-05
457 8.41709406813607e-05
458 8.40970096760429e-05
459 8.40236389194615e-05
460 8.39507265482098e-05
461 8.38782652863301e-05
462 8.38063569972292e-05
463 8.37348343338817e-05
464 8.36637336760759e-05
465 8.35931423353031e-05
466 8.35229075164534e-05
467 8.34530874271877e-05
468 8.33836456877179e-05
469 8.33146259537898e-05
470 8.32458899822086e-05
471 8.317763422383e-05
472 8.31096258480102e-05
473 8.30419594421983e-05
474 8.29746713861823e-05
475 8.29076088848524e-05
476 8.28408738016151e-05
477 8.27744006528519e-05
478 8.27082112664357e-05
479 8.26422838144936e-05
480 8.25766255729832e-05
481 8.25112292659469e-05
482 8.24460221338086e-05
483 8.23810769361444e-05
484 8.23163645691238e-05
485 8.2251790445298e-05
486 8.21875291876495e-05
487 8.21234716568142e-05
488 8.20596324047074e-05
489 8.19959968794137e-05
490 8.19324413896538e-05
491 8.18690969026648e-05
492 8.18058470031247e-05
493 8.17427644506097e-05
494 8.16797401057556e-05
495 8.16169485915452e-05
496 8.15542080090381e-05
497 8.14915329101495e-05
498 8.14290106063709e-05
499 8.13664519228041e-05
500 8.13040460343473e-05
501 8.12416692497209e-05
502 8.11792415333912e-05
503 8.11168574728072e-05
504 8.10544734122232e-05
505 8.09920893516392e-05
506 8.0929632531479e-05
507 8.08672048151493e-05
508 8.08047479949892e-05
509 8.07422329671681e-05
510 8.06796379038133e-05
511 8.06169482530095e-05
512 8.055421494646e-05
513 8.04913361207582e-05
514 8.0428428191226e-05
515 8.03653892944567e-05
516 8.03021976025775e-05
517 8.02388967713341e-05
518 8.01755886641331e-05
519 8.01120404503308e-05
520 8.00483685452491e-05
521 7.99845365690999e-05
522 7.99205154180527e-05
523 7.98563632997684e-05
524 7.97919856267981e-05
525 7.97274842625484e-05
526 7.96627355157398e-05
527 7.95977030065842e-05
528 7.95325031504035e-05
529 7.94670559116639e-05
530 7.94013540144078e-05
531 7.93353101471439e-05
532 7.92689461377449e-05
533 7.92023274698295e-05
534 7.91353450040333e-05
535 7.90680132922716e-05
536 7.9000397818163e-05
537 7.89323603385128e-05
538 7.88639517850243e-05
539 7.87951285019517e-05
540 7.8725912317168e-05
541 7.86562159191817e-05
542 7.8586112067569e-05
543 7.85155643825419e-05
544 7.84445001045242e-05
545 7.83728901296854e-05
546 7.83008435973898e-05
547 7.82282295404002e-05
548 7.81550916144624e-05
549 7.80814152676612e-05
550 7.80070913606323e-05
551 7.79322144808248e-05
552 7.78567409724928e-05
553 7.778056897223e-05
554 7.77037275838666e-05
555 7.76262022554874e-05
556 7.75479711592197e-05
557 7.74688887759112e-05
558 7.7389158832375e-05
559 7.73085848777555e-05
560 7.72272105677985e-05
561 7.71449340390973e-05
562 7.70617480156943e-05
563 7.69776670495048e-05
564 7.68927129684016e-05
565 7.68066092859954e-05
566 7.67193487263285e-05
567 7.66309749451466e-05
568 7.65414879424497e-05
569 7.64508440624923e-05
570 7.63590214774013e-05
571 7.62659037718549e-05
572 7.61713818064891e-05
573 7.6075506513007e-05
574 7.59781542001292e-05
575 7.58792302804068e-05
576 7.57787929615006e-05
577 7.5676609412767e-05
578 7.55726650822908e-05
579 7.54669308662415e-05
580 7.53591957618482e-05
581 7.5249576184433e-05
582 7.51378538552672e-05
583 7.50239050830714e-05
584 7.49077516957186e-05
585 7.47894155210815e-05
586 7.46686055208556e-05
587 7.45456054573879e-05
588 7.44201024645008e-05
589 7.42922056815587e-05
590 7.41615149308927e-05
591 7.40281247999519e-05
592 7.38920716685243e-05
593 7.37530572223477e-05
594 7.36113361199386e-05
595 7.34669229132123e-05
596 7.33198467059992e-05
597 7.31700129108503e-05
598 7.30174215277657e-05
599 7.28620361769572e-05
600 7.27037840988487e-05
601 7.25426652934402e-05
602 7.23787743481807e-05
603 7.22123149898835e-05
604 7.20430034562014e-05
605 7.18708688509651e-05
606 7.1695598307997e-05
607 7.15172427590005e-05
608 7.13361223461106e-05
609 7.11520988261327e-05
610 7.09655287209898e-05
611 7.07763538230211e-05
612 7.05847560311668e-05
613 7.0390866312664e-05
614 7.01946337358095e-05
615 6.99961965437979e-05
616 6.97955474606715e-05
617 6.95928611094132e-05
618 6.93881447659805e-05
619 6.91815657773986e-05
620 6.89732987666503e-05
621 6.87631036271341e-05
622 6.85511113260873e-05
623 6.83371690683998e-05
624 6.81214005453512e-05
625 6.7904191382695e-05
626 6.76850977470167e-05
627 6.74645852996036e-05
628 6.7243141529616e-05
629 6.70200097374618e-05
630 6.67951462673955e-05
631 6.65685802232474e-05
632 6.6340588091407e-05
633 6.61111989757046e-05
634 6.58805947750807e-05
635 6.56487754895352e-05
636 6.54159957775846e-05
637 6.51821246719919e-05
638 6.49474313831888e-05
639 6.47118285996839e-05
640 6.44756219116971e-05
641 6.42388986307196e-05
642 6.40019352431409e-05
643 6.37646371615119e-05
644 6.35271426290274e-05
645 6.32894734735601e-05
646 6.3051724282559e-05
647 6.28141569904983e-05
648 6.25768007012084e-05
649 6.23397863819264e-05
650 6.21032668277621e-05
651 6.18671983829699e-05
652 6.16318575339392e-05
653 6.13972370047122e-05
654 6.11633804510348e-05
655 6.09304806857836e-05
656 6.06987414357718e-05
657 6.04683118581306e-05
658 6.02392428845633e-05
659 6.0011643654434e-05
660 5.97855578234885e-05
661 5.95610472373664e-05
662 5.93381082580891e-05
663 5.91169045947026e-05
664 5.88974544371013e-05
665 5.86799942539074e-05
666 5.84643239562865e-05
667 5.82505999773275e-05
668 5.80389423703309e-05
669 5.78292274440173e-05
670 5.76215170440264e-05
671 5.74158875679132e-05
672 5.72123899473809e-05
673 5.7011035096366e-05
674 5.68118084629532e-05
675 5.66149246878922e-05
676 5.64202600799035e-05
677 5.6228025641758e-05
678 5.60381085961126e-05
679 5.58504907530732e-05
680 5.56653649255168e-05
681 5.548277113121e-05
682 5.53024401597213e-05
683 5.51243938389234e-05
684 5.49486903764773e-05
685 5.47752788406797e-05
686 5.46042138012126e-05
687 5.4435535275843e-05
688 5.42691559530795e-05
689 5.41051085747313e-05
690 5.39434513484593e-05
691 5.37841078767087e-05
692 5.36272309545893e-05
693 5.34725695615634e-05
694 5.332021828508e-05
695 5.31700461579021e-05
696 5.30220786458813e-05
697 5.28763084730599e-05
698 5.27327065356076e-05
699 5.25911964359693e-05
700 5.2451818191912e-05
701 5.23144481121562e-05
702 5.21790607308503e-05
703 5.20456596859731e-05
704 5.19141685799696e-05
705 5.17845910508186e-05
706 5.1656985306181e-05
707 5.15313367941417e-05
708 5.14074963575695e-05
709 5.12852930114605e-05
710 5.11648431711365e-05
711 5.10460849909578e-05
712 5.09289930050727e-05
713 5.08135235577356e-05
714 5.06996802869253e-05
715 5.05873431393411e-05
716 5.04765666846652e-05
717 5.03673145431094e-05
718 5.02596885780804e-05
719 5.0153503252659e-05
720 5.00489404657856e-05
721 4.99461748404428e-05
722 4.98451045132242e-05
723 4.97457367600873e-05
724 4.96479660796467e-05
725 4.95518725074362e-05
726 4.94572595926002e-05
727 4.93642000947148e-05
728 4.92725630465429e-05
729 4.91823484480847e-05
730 4.90936727146618e-05
731 4.90064085170161e-05
732 4.89206504425965e-05
733 4.8836249334272e-05
734 4.8753157898318e-05
735 4.86713179270737e-05
736 4.85907730762847e-05
737 4.85113523609471e-05
738 4.84332013002131e-05
739 4.83560579596087e-05
740 4.82800314784981e-05
741 4.82050381833687e-05
742 4.81310999020934e-05
743 4.8058256652439e-05
744 4.79865047964267e-05
745 4.79156406072434e-05
746 4.78456167911645e-05
747 4.77764842798933e-05
748 4.77080175187439e-05
749 4.76404275104869e-05
750 4.75735178042669e-05
751 4.75073902634904e-05
752 4.74420448881574e-05
753 4.73773689009249e-05
754 4.73133586638141e-05
755 4.72500214527827e-05
756 4.71873063361272e-05
757 4.71251878479961e-05
758 4.70635823148768e-05
759 4.700260979007e-05
760 4.69421247544233e-05
761 4.68822036054917e-05
762 4.68227262899745e-05
763 4.6763652790105e-05
764 4.67050158476923e-05
765 4.66468482045457e-05
766 4.65890370833222e-05
767 4.65321172669064e-05
768 4.64754812128376e-05
769 4.64192453364376e-05
770 4.63631768070627e-05
771 4.63075339212082e-05
772 4.62520401924849e-05
773 4.6196837502066e-05
774 4.61417439510114e-05
775 4.6086948714219e-05
776 4.60323644801974e-05
777 4.59779002994765e-05
778 4.5923690777272e-05
779 4.58697395515628e-05
780 4.58158247056417e-05
781 4.57621899840888e-05
782 4.57086935057305e-05
783 4.56552988907788e-05
784 4.56020243291277e-05
785 4.55489353043959e-05
786 4.54959699709434e-05
787 4.54431465186644e-05
788 4.53903667221311e-05
789 4.53377106168773e-05
790 4.5285138185136e-05
791 4.5232693082653e-05
792 4.51802625320852e-05
793 4.51279738626909e-05
794 4.50757397629786e-05
795 4.5023589336779e-05
796 4.49714898422826e-05
797 4.49194922111928e-05
798 4.48677383246832e-05
799 4.48158207291272e-05
800 4.47640886704903e-05
801 4.4712440285366e-05
802 4.46607627964113e-05
803 4.46092381025665e-05
804 4.45577825303189e-05
805 4.4506410631584e-05
806 4.44550314568914e-05
807 4.44037395936903e-05
808 4.4352636905387e-05
809 4.43014396296348e-05
810 4.42504315287806e-05
811 4.41995216533542e-05
812 4.41486008639913e-05
813 4.40978219558019e-05
814 4.40471194451675e-05
815 4.39965479017701e-05
816 4.39460782217793e-05
817 4.38958377344534e-05
818 4.38455972471274e-05
819 4.37955895904452e-05
820 4.37456692452542e-05
821 4.36958325735759e-05
822 4.36462250945624e-05
823 4.35963847849052e-05
824 4.35464862675872e-05
825 4.34968169429339e-05
826 4.34472640336026e-05
827 4.33981076639611e-05
828 4.3349435145501e-05
829 4.33007771789562e-05
830 4.32523665949702e-05
831 4.32040433224756e-05
832 4.3155927414773e-05
833 4.31078406109009e-05
834 4.30599102401175e-05
835 4.30120162491221e-05
836 4.29643005190883e-05
837 4.2916650272673e-05
838 4.28691164415795e-05
839 4.28217099397443e-05
840 4.27743907494005e-05
841 4.27271515945904e-05
842 4.26800579589326e-05
843 4.26330661866814e-05
844 4.25861871917732e-05
845 4.25393700425047e-05
846 4.24926220148336e-05
847 4.24460959038697e-05
848 4.23995406890754e-05
849 4.23532110289671e-05
850 4.23068595409859e-05
851 4.22605917265173e-05
852 4.22145058109891e-05
853 4.21684198954608e-05
854 4.21224613091908e-05
855 4.20766846218612e-05
856 4.20308679167647e-05
857 4.19851712649688e-05
858 4.19395364588127e-05
859 4.18939343944658e-05
860 4.18484596593771e-05
861 4.18030358559918e-05
862 4.17576738982461e-05
863 4.17123483202886e-05
864 4.1667135519674e-05
865 4.16219772887416e-05
866 4.15768663515337e-05
867 4.15318281739019e-05
868 4.14868991356343e-05
869 4.14419882872608e-05
870 4.13971429225057e-05
871 4.13524358009454e-05
872 4.13077505072579e-05
873 4.12631379731465e-05
874 4.12186564062722e-05
875 4.1174153011525e-05
876 4.11297478422057e-05
877 4.10854408983141e-05
878 4.10412067139987e-05
879 4.099692159798e-05
880 4.09527783631347e-05
881 4.09086860599928e-05
882 4.08646592404693e-05
883 4.08206142310519e-05
884 4.07766892749351e-05
885 4.07327534048818e-05
886 4.06888830184471e-05
887 4.06451217713766e-05
888 4.06013095926028e-05
889 4.05575519835111e-05
890 4.05138889618684e-05
891 4.04703096137382e-05
892 4.04267011617776e-05
893 4.03831654693931e-05
894 4.03396770707332e-05
895 4.02961268264335e-05
896 4.02527439291589e-05
897 4.02093282900751e-05
898 4.01659781346098e-05
899 4.01226752728689e-05
900 4.0079430618789e-05
901 4.0036225982476e-05
902 3.99930322600994e-05
903 3.99498057959136e-05
904 3.99067102989648e-05
905 3.98636293539312e-05
906 3.98205520468764e-05
907 3.97775111196097e-05
908 3.97344265365973e-05
909 3.96913565055002e-05
910 3.96482828364242e-05
911 3.96052528230939e-05
912 3.95622009818908e-05
913 3.9519218262285e-05
914 3.94762610085309e-05
915 3.94333692383952e-05
916 3.93905102100689e-05
917 3.93477203033399e-05
918 3.93049485865049e-05
919 3.9262100472115e-05
920 3.92192923754919e-05
921 3.9176557038445e-05
922 3.91338908229955e-05
923 3.90913373848889e-05
924 3.90488312405068e-05
925 3.90063541999552e-05
926 3.89639208151493e-05
927 3.89215128961951e-05
928 3.88791595469229e-05
929 3.88368534913752e-05
930 3.87945910915732e-05
931 3.87521940865554e-05
932 3.87099498766474e-05
933 3.86676474590786e-05
934 3.86255014745984e-05
935 3.85832790925633e-05
936 3.85410021408461e-05
937 3.84987506549805e-05
938 3.84562263207044e-05
939 3.8413709262386e-05
940 3.83711922040675e-05
941 3.83287369913887e-05
942 3.82862272090279e-05
943 3.82436519430485e-05
944 3.82010621251538e-05
945 3.81584541173652e-05
946 3.81159006792586e-05
947 3.80732781195547e-05
948 3.80307537852786e-05
949 3.79881930712145e-05
950 3.79456214432139e-05
951 3.79030825570226e-05
952 3.78606491722167e-05
953 3.78181393898558e-05
954 3.77757860405836e-05
955 3.7733403587481e-05
956 3.76911848434247e-05
957 3.76489806512836e-05
958 3.76068637706339e-05
959 3.75647759938147e-05
960 3.75227573385928e-05
961 3.74808078049682e-05
962 3.74388873751741e-05
963 3.73970142391045e-05
964 3.73552102246322e-05
965 3.73134462279268e-05
966 3.72718131984584e-05
967 3.72301910829265e-05
968 3.71886089851614e-05
969 3.71471105609089e-05
970 3.71056594303809e-05
971 3.70642665075138e-05
972 3.7022888136562e-05
973 3.69816698366776e-05
974 3.69404078810476e-05
975 3.68992332369089e-05
976 3.68581349903252e-05
977 3.68170585716143e-05
978 3.67760112567339e-05
979 3.67350330634508e-05
980 3.66942804248538e-05
981 3.66535932698753e-05
982 3.66130370821338e-05
983 3.65725354640745e-05
984 3.65321720892098e-05
985 3.64918778359424e-05
986 3.64516927220393e-05
987 3.64115549018607e-05
988 3.63715516868979e-05
989 3.63316139555536e-05
990 3.62916944141034e-05
991 3.62518258043565e-05
992 3.62120226782281e-05
993 3.61722995876335e-05
994 3.61327001883183e-05
995 3.60930898750667e-05
996 3.60536105290521e-05
997 3.60141275450587e-05
998 3.59747173206415e-05
999 3.59353798558004e-05
1000 3.58961115125567e-05
1001 3.58568831870798e-05
1002 3.5817691241391e-05
1003 3.57785611413419e-05
1004 3.57394783350173e-05
1005 3.57004792022053e-05
1006 3.56614909833297e-05
1007 3.56225900759455e-05
1008 3.55837291863281e-05
1009 3.5544864658732e-05
1010 3.55060728907119e-05
1011 3.5467299312586e-05
1012 3.5428620321909e-05
1013 3.53899886249565e-05
1014 3.53513532900251e-05
1015 3.53127470589243e-05
1016 3.52742026734632e-05
1017 3.52357383235358e-05
1018 3.51972958014812e-05
1019 3.51588969351724e-05
1020 3.51205562765244e-05
1021 3.50822301697917e-05
1022 3.5043998650508e-05
1023 3.5005752579309e-05
1024 3.49675901816227e-05
1025 3.49294750776608e-05
1026 3.48913999914657e-05
1027 3.48533685610164e-05
1028 3.4815395338228e-05
1029 3.47773966495879e-05
1030 3.47394925483968e-05
1031 3.47015775332693e-05
1032 3.46637098118663e-05
1033 3.46259112120606e-05
1034 3.45880944223609e-05
1035 3.45504049619194e-05
1036 3.45126973115839e-05
1037 3.44750078511424e-05
1038 3.44373802363407e-05
1039 3.43997526215389e-05
1040 3.43621832143981e-05
1041 3.43246829288546e-05
1042 3.42871608154383e-05
1043 3.42496969096828e-05
1044 3.42122766596731e-05
1045 3.41748745995574e-05
1046 3.41375307471026e-05
1047 3.41001941706054e-05
1048 3.40629048878327e-05
1049 3.40256483468693e-05
1050 3.39884318236727e-05
1051 3.39511789206881e-05
1052 3.39140169671737e-05
1053 3.38768804795109e-05
1054 3.38397439918481e-05
1055 3.38026729878038e-05
1056 3.37656092597172e-05
1057 3.37285273417365e-05
1058 3.36914963554591e-05
1059 3.36544944730122e-05
1060 3.36174452968407e-05
1061 3.35805380018428e-05
1062 3.35435724991839e-05
1063 3.35066506522708e-05
1064 3.34697106154636e-05
1065 3.34328324242961e-05
1066 3.33959069394041e-05
1067 3.3359039662173e-05
1068 3.33222051267512e-05
1069 3.32853524014354e-05
1070 3.324852877995e-05
1071 3.32116942445282e-05
1072 3.31748415192124e-05
1073 3.31379487761296e-05
1074 3.31010669469833e-05
1075 3.30642178596463e-05
1076 3.30273505824152e-05
1077 3.29905087710358e-05
1078 3.2953674235614e-05
1079 3.29168797179591e-05
1080 3.28800488205161e-05
1081 3.28432906826492e-05
1082 3.28065107169095e-05
1083 3.27697198372334e-05
1084 3.27330344589427e-05
1085 3.26962690451182e-05
1086 3.26595472870395e-05
1087 3.26228328049183e-05
1088 3.25861292367335e-05
1089 3.254942203057e-05
1090 3.2512722100364e-05
1091 3.24760985677131e-05
1092 3.24394095514435e-05
1093 3.24027678288985e-05
1094 3.23660860885866e-05
1095 3.23294771078508e-05
1096 3.22927699016873e-05
1097 3.22561354550999e-05
1098 3.22194791806396e-05
1099 3.21828156302217e-05
1100 3.21461375278886e-05
1101 3.21094194077887e-05
1102 3.20728431688622e-05
1103 3.20361614285503e-05
1104 3.19994906021748e-05
1105 3.19627761200536e-05
1106 3.19261162076145e-05
1107 3.18893871735781e-05
1108 3.18527236231603e-05
1109 3.18160164169967e-05
1110 3.17793055728544e-05
1111 3.17426711262669e-05
1112 3.17059493681882e-05
1113 3.16692276101094e-05
1114 3.16325676976703e-05
1115 3.15958313876763e-05
1116 3.15591423714068e-05
1117 3.15223769575823e-05
1118 3.14856370096095e-05
1119 3.14488897856791e-05
1120 3.14121425617486e-05
1121 3.13753553200513e-05
1122 3.1338586268248e-05
1123 3.13017844746355e-05
1124 3.12649899569806e-05
1125 3.12281663354952e-05
1126 3.1191320886137e-05
1127 3.11544827127364e-05
1128 3.11176008835901e-05
1129 3.10806972265709e-05
1130 3.10437753796577e-05
1131 3.10067989630625e-05
1132 3.09697898046579e-05
1133 3.09327806462534e-05
1134 3.08957423840184e-05
1135 3.0858711397741e-05
1136 3.08216149278451e-05
1137 3.07845148199704e-05
1138 3.07474256260321e-05
1139 3.07102236547507e-05
1140 3.06730507872999e-05
1141 3.06357760564424e-05
1142 3.05985049635638e-05
1143 3.0561161111109e-05
1144 3.05238136206754e-05
1145 3.04864443023689e-05
1146 3.04490422422532e-05
1147 3.04116365441587e-05
1148 3.03741653624456e-05
1149 3.03366923617432e-05
1150 3.02992193610407e-05
1151 3.02616754197516e-05
1152 3.02240896417061e-05
1153 3.01865420624381e-05
1154 3.01489453704562e-05
1155 3.0111288651824e-05
1156 3.00735791824991e-05
1157 3.00356259685941e-05
1158 2.99974835797912e-05
1159 2.99593284580624e-05
1160 2.99211187666515e-05
1161 2.98828636005055e-05
1162 2.98445720545715e-05
1163 2.98063041554997e-05
1164 2.97679453069577e-05
1165 2.97295973723521e-05
1166 2.96912476187572e-05
1167 2.96528105536709e-05
1168 2.96143280138494e-05
1169 2.95758418360492e-05
1170 2.95373047265457e-05
1171 2.94987130473601e-05
1172 2.94600959023228e-05
1173 2.94214260065928e-05
1174 2.93827488349052e-05
1175 2.93440189125249e-05
1176 2.93052526103565e-05
1177 2.92664426524425e-05
1178 2.92275872197933e-05
1179 2.91886863124091e-05
1180 2.91497472062474e-05
1181 2.9110773539287e-05
1182 2.90717453026446e-05
1183 2.90326916001504e-05
1184 2.89935578621225e-05
1185 2.8954416848137e-05
1186 2.89151848846814e-05
1187 2.88759074464906e-05
1188 2.88366518361727e-05
1189 2.87972907244693e-05
1190 2.87578968709568e-05
1191 2.87184629996773e-05
1192 2.86789618257899e-05
1193 2.8639413358178e-05
1194 2.85998394247144e-05
1195 2.85602272924734e-05
1196 2.85205260297516e-05
1197 2.84807756543159e-05
1198 2.84410125459544e-05
1199 2.84012021438684e-05
1200 2.83613171632169e-05
1201 2.83213666989468e-05
1202 2.82814271486131e-05
1203 2.82413657259895e-05
1204 2.82013061223552e-05
1205 2.81611337413779e-05
1206 2.8120999559178e-05
1207 2.80807307717623e-05
1208 2.80404747172724e-05
1209 2.80002041108673e-05
1210 2.79598280030768e-05
1211 2.7919373678742e-05
1212 2.78789084404707e-05
1213 2.78383577096974e-05
1214 2.7797792427009e-05
1215 2.77571780316066e-05
1216 2.77165290754056e-05
1217 2.7675805540639e-05
1218 2.76350547210313e-05
1219 2.75942093139747e-05
1220 2.75533202511724e-05
1221 2.75124093604973e-05
1222 2.74714311672142e-05
1223 2.7430380214355e-05
1224 2.73893256235169e-05
1225 2.73483001365094e-05
1226 2.73071873380104e-05
1227 2.72660035989247e-05
1228 2.72248344117543e-05
1229 2.7183566999156e-05
1230 2.71422904916108e-05
1231 2.71009794232668e-05
1232 2.70595282927388e-05
1233 2.70181008090731e-05
1234 2.69766023848206e-05
1235 2.6935076675727e-05
1236 2.68935600615805e-05
1237 2.6851976144826e-05
1238 2.6810308554559e-05
1239 2.6768640964292e-05
1240 2.67269533651415e-05
1241 2.66851711785421e-05
1242 2.66433798969956e-05
1243 2.660152495082e-05
1244 2.65596245299093e-05
1245 2.6517707738094e-05
1246 2.64756872638827e-05
1247 2.64336722466396e-05
1248 2.6391573555884e-05
1249 2.63495385297574e-05
1250 2.56946386798518e-05
1251 2.57005322055193e-05
1252 2.56907151197083e-05
1253 2.56867388088722e-05
1254 2.56845869444078e-05
1255 2.56826242548414e-05
1256 2.56803887168644e-05
1257 2.56777511822293e-05
1258 2.56746934610419e-05
1259 2.56712792179314e-05
1260 2.5667572117527e-05
1261 2.56636176345637e-05
1262 2.5659435777925e-05
1263 2.56551120401127e-05
1264 2.5650619136286e-05
1265 2.5646007998148e-05
1266 2.56412968155928e-05
1267 2.56364619417582e-05
1268 2.56315743172308e-05
1269 2.56265993812121e-05
1270 2.56215553235961e-05
1271 2.56164421443827e-05
1272 2.56112889474025e-05
1273 2.56060757237719e-05
1274 2.56008224823745e-05
1275 2.55955219472526e-05
1276 2.55901923083002e-05
1277 2.55848190136021e-05
1278 2.55794202530524e-05
1279 2.55740033026086e-05
1280 2.55685536103556e-05
1281 2.55630620813463e-05
1282 2.555756145739e-05
1283 2.55520335485926e-05
1284 2.554649108788e-05
1285 2.55409195233369e-05
1286 2.55353679676773e-05
1287 2.55297745752614e-05
1288 2.55241611739621e-05
1289 2.55185332207475e-05
1290 2.55129016295541e-05
1291 2.55072591244243e-05
1292 2.55016111623263e-05
1293 2.54959468293237e-05
1294 2.54902752203634e-05
1295 2.54845726885833e-05
1296 2.54788956226548e-05
1297 2.54732076427899e-05
1298 2.54675160249462e-05
1299 2.54618189501343e-05
1300 2.54561182373436e-05
1301 2.54504175245529e-05
1302 2.54447004408576e-05
1303 2.54390015470563e-05
1304 2.54332935583079e-05
1305 2.54275764746126e-05
1306 2.54218703048537e-05
1307 2.54161750490312e-05
1308 2.54104652412934e-05
1309 2.54047590715345e-05
1310 2.53990656347014e-05
1311 2.53933758358471e-05
1312 2.53876769420458e-05
1313 2.53819689532975e-05
1314 2.53762682405068e-05
1315 2.53705711656949e-05
1316 2.53648759098724e-05
1317 2.53592061199015e-05
1318 2.5353519959026e-05
1319 2.53478337981505e-05
1320 2.53421585512115e-05
1321 2.533649058023e-05
1322 2.53308226092486e-05
1323 2.53251528192777e-05
1324 2.53194830293069e-05
1325 2.53138241532724e-05
1326 2.53081707342062e-05
1327 2.53025227721082e-05
1328 2.52968748100102e-05
1329 2.52912377618486e-05
1330 2.5285609808634e-05
1331 2.52799763984513e-05
1332 2.52743484452367e-05
1333 2.52687368629267e-05
1334 2.52631107287016e-05
1335 2.52575118793175e-05
1336 2.52519021159969e-05
1337 2.52462941716658e-05
1338 2.52406898653135e-05
1339 2.52350964728976e-05
1340 2.52295012614923e-05
1341 2.5223924239981e-05
1342 2.52183435804909e-05
1343 2.52127665589796e-05
1344 2.52071858994896e-05
1345 2.52016143349465e-05
1346 2.51960555033293e-05
1347 2.51905003096908e-05
1348 2.518495239201e-05
1349 2.51793935603928e-05
1350 2.51738510996802e-05
1351 2.51683031819994e-05
1352 2.5162757083308e-05
1353 2.51572182605742e-05
1354 2.51516903517768e-05
1355 2.51461588050006e-05
1356 2.51406290772138e-05
1357 2.51351139013423e-05
1358 2.51295768975979e-05
1359 2.51240526267793e-05
1360 2.5118570192717e-05
1361 2.51130586548243e-05
1362 2.51075689448044e-05
1363 2.51020719588269e-05
1364 2.50965676968917e-05
1365 2.50910798058612e-05
1366 2.50855991907883e-05
1367 2.50801112997578e-05
1368 2.50746270467062e-05
1369 2.50691464316333e-05
1370 2.5063673092518e-05
1371 2.50582070293603e-05
1372 2.50527391472133e-05
1373 2.50472785410238e-05
1374 2.50418106588768e-05
1375 2.50363555096556e-05
1376 2.50308949034661e-05
1377 2.50254470302025e-05
1378 2.50200027949177e-05
1379 2.50145585596329e-05
1380 2.50091197813163e-05
1381 2.50036700890632e-05
1382 2.49982276727678e-05
1383 2.49928107223241e-05
1384 2.49873683060287e-05
1385 2.49819440796273e-05
1386 2.49765234912047e-05
1387 2.49710992648033e-05
1388 2.49656804953702e-05
1389 2.49602690018946e-05
1390 2.49548556894297e-05
1391 2.49494551098906e-05
1392 2.49440381594468e-05
1393 2.49386339419289e-05
1394 2.49332315434003e-05
1395 2.49278382398188e-05
1396 2.49224449362373e-05
1397 2.49170552706346e-05
1398 2.49116710620001e-05
1399 2.49062741204398e-05
1400 2.49008862738265e-05
1401 2.48955002462026e-05
1402 2.48901087616105e-05
1403 2.48847372859018e-05
1404 2.48793548962567e-05
1405 2.4873974325601e-05
1406 2.486861012585e-05
1407 2.48632422881201e-05
1408 2.48578635364538e-05
1409 2.48525120696286e-05
1410 2.48471387749305e-05
1411 2.48417691182112e-05
1412 2.48364140134072e-05
1413 2.48310425376985e-05
1414 2.48256910708733e-05
1415 2.48203396040481e-05
1416 2.48149790422758e-05
1417 2.48096275754506e-05
1418 2.48042742896359e-05
1419 2.47989264607895e-05
1420 2.47935713559855e-05
1421 2.47882289841073e-05
1422 2.47828847932396e-05
1423 2.4777540602372e-05
1424 2.4772203687462e-05
1425 2.47668504016474e-05
1426 2.47615098487586e-05
1427 2.47561802098062e-05
1428 2.47508305619704e-05
1429 2.47455045609968e-05
1430 2.4740164008108e-05
1431 2.47348343691556e-05
1432 2.47295029112138e-05
1433 2.47241769102402e-05
1434 2.4718852728256e-05
1435 2.4713524908293e-05
1436 2.47082043642877e-05
1437 2.47028747253353e-05
1438 2.46975578193087e-05
1439 2.46922463702504e-05
1440 2.46869331022026e-05
1441 2.46816198341548e-05
1442 2.46763156610541e-05
1443 2.46710005740169e-05
1444 2.4665681849001e-05
1445 2.46603758569108e-05
1446 2.46550571318949e-05
1447 2.46497456828365e-05
1448 2.46444415097358e-05
1449 2.46391391556244e-05
1450 2.46338349825237e-05
1451 2.46285289904336e-05
1452 2.46232266363222e-05
1453 2.46179297391791e-05
1454 2.46126273850678e-05
1455 2.46073250309564e-05
1456 2.46020244958345e-05
1457 2.45967312366702e-05
1458 2.4591434339527e-05
1459 2.45861501753097e-05
1460 2.45808496401878e-05
1461 2.45755563810235e-05
1462 2.4570275854785e-05
1463 2.45649971475359e-05
1464 2.45597166212974e-05
1465 2.45544360950589e-05
1466 2.45491501118522e-05
1467 2.45438768615713e-05
1468 2.45385908783646e-05
1469 2.45333176280838e-05
1470 2.45280389208347e-05
1471 2.45227638515644e-05
1472 2.45174887822941e-05
1473 2.45122118940344e-05
1474 2.45069404627429e-05
1475 2.45016635744832e-05
1476 2.44963866862236e-05
1477 2.44911152549321e-05
1478 2.44858401856618e-05
1479 2.44805651163915e-05
1480 2.44753136939835e-05
1481 2.44700459006708e-05
1482 2.44647744693793e-05
1483 2.44595048570773e-05
1484 2.44542534346692e-05
1485 2.44490038312506e-05
1486 2.44437396759167e-05
1487 2.44384755205829e-05
1488 2.44332095462596e-05
1489 2.44279381149681e-05
1490 2.44226921495283e-05
1491 2.44174298131838e-05
1492 2.44121711148182e-05
1493 2.44069142354419e-05
1494 2.44016646320233e-05
1495 2.43964041146683e-05
1496 2.4391147235292e-05
1497 2.43859030888416e-05
1498 2.43806516664336e-05
1499 2.43753929680679e-05
1500 2.43701360886917e-05
1501 2.43648901232518e-05
1502 2.43596405198332e-05
1503 2.43543872784358e-05
1504 2.43491449509747e-05
1505 2.43438953475561e-05
1506 2.43386475631269e-05
1507 2.43334088736447e-05
1508 2.43281538132578e-05
1509 2.4322907847818e-05
1510 2.43176673393464e-05
1511 2.43124086409807e-05
1512 2.43071553995833e-05
1513 2.43019021581858e-05
1514 2.42966580117354e-05
1515 2.4291413865285e-05
1516 2.42861678998452e-05
1517 2.4280929210363e-05
1518 2.42756832449231e-05
1519 2.42704481934197e-05
1520 2.42652040469693e-05
1521 2.42599689954659e-05
1522 2.42547321249731e-05
1523 2.42495025304379e-05
1524 2.42442747548921e-05
1525 2.42390360654099e-05
1526 2.42338101088535e-05
1527 2.42285805143183e-05
1528 2.42233472818043e-05
1529 2.42181213252479e-05
1530 2.42128917307127e-05
1531 2.42076657741563e-05
1532 2.42024361796211e-05
1533 2.41972138610436e-05
1534 2.41919933614554e-05
1535 2.41867746808566e-05
1536 2.41815487243002e-05
1537 2.41763391386485e-05
1538 2.41710986301769e-05
1539 2.41658926825039e-05
1540 2.41606830968522e-05
1541 2.41554662352428e-05
1542 2.41502366407076e-05
1543 2.41450252360664e-05
1544 2.41398174694041e-05
1545 2.41345951508265e-05
1546 2.41293964791112e-05
1547 2.41241777985124e-05
1548 2.4118960936903e-05
1549 2.41137640841771e-05
1550 2.4108561774483e-05
1551 2.41033612837782e-05
1552 2.40981535171159e-05
1553 2.40929584833793e-05
1554 2.40877579926746e-05
1555 2.40825556829805e-05
1556 2.40773515542969e-05
1557 2.40721583395498e-05
1558 2.40669614868239e-05
1559 2.40617646340979e-05
1560 2.40565750573296e-05
1561 2.40513763856143e-05
1562 2.40461831708672e-05
1563 2.40409917751094e-05
1564 2.40357949223835e-05
1565 2.40306017076364e-05
1566 2.40254048549104e-05
1567 2.40202080021845e-05
1568 2.40150184254162e-05
1569 2.40098233916797e-05
1570 2.4004641090869e-05
1571 2.3999444238143e-05
1572 2.39942546613747e-05
1573 2.39890650846064e-05
1574 2.39838846027851e-05
1575 2.3978691388038e-05
1576 2.39735090872273e-05
1577 2.3968319510459e-05
1578 2.39631317526801e-05
1579 2.39579512708588e-05
1580 2.39527580561116e-05
1581 2.39475793932797e-05
1582 2.39423934544902e-05
1583 2.39372038777219e-05
1584 2.39320306718582e-05
1585 2.39268501900369e-05
1586 2.39216733461944e-05
1587 2.39164965023519e-05
1588 2.39113342104247e-05
1589 2.39061355387093e-05
1590 2.39009495999198e-05
1591 2.3895780032035e-05
1592 2.38906013692031e-05
1593 2.388542634435e-05
1594 2.38802549574757e-05
1595 2.38750799326226e-05
1596 2.38699121837271e-05
1597 2.38647353398846e-05
1598 2.38595603150316e-05
1599 2.38543780142209e-05
1600 2.38492120843148e-05
1601 2.38440425164299e-05
1602 2.3838854758651e-05
1603 2.38336888287449e-05
1604 2.38285210798495e-05
1605 2.3823353330954e-05
1606 2.38181764871115e-05
1607 2.3813008738216e-05
1608 2.3807851903257e-05
1609 2.38026805163827e-05
1610 2.37975127674872e-05
1611 2.37923450185917e-05
1612 2.37871772696963e-05
1613 2.37820077018114e-05
1614 2.37768526858417e-05
1615 2.37716831179569e-05
1616 2.37665281019872e-05
1617 2.37613512581447e-05
1618 2.3756187147228e-05
1619 2.37510412262054e-05
1620 2.3745882572257e-05
1621 2.37407293752767e-05
1622 2.37355798162753e-05
1623 2.37304375332315e-05
1624 2.37252734223148e-05
1625 2.37201129493769e-05
1626 2.37149579334073e-05
1627 2.37098101933952e-05
1628 2.37046515394468e-05
1629 2.36994837905513e-05
1630 2.36943324125605e-05
1631 2.36891864915378e-05
1632 2.36840369325364e-05
1633 2.3678887373535e-05
1634 2.36737341765547e-05
1635 2.36685900745215e-05
1636 2.36634296015836e-05
1637 2.36582873185398e-05
1638 2.36531323025702e-05
1639 2.36479772866005e-05
1640 2.36428350035567e-05
1641 2.36376872635446e-05
1642 2.36325431615114e-05
1643 2.3627402697457e-05
1644 2.36222585954238e-05
1645 2.36171144933905e-05
1646 2.36119612964103e-05
1647 2.36068135563983e-05
1648 2.36016767303227e-05
1649 2.35965308093e-05
1650 2.35913939832244e-05
1651 2.35862462432124e-05
1652 2.35811203310732e-05
1653 2.35759871429764e-05
1654 2.35708430409431e-05
1655 2.35657043958781e-05
1656 2.35605712077813e-05
1657 2.35554325627163e-05
1658 2.35502957366407e-05
1659 2.35451698245015e-05
1660 2.35400348174153e-05
1661 2.35349052672973e-05
1662 2.35297738981899e-05
1663 2.35246443480719e-05
1664 2.35195238929009e-05
1665 2.35143961617723e-05
1666 2.35092738876119e-05
1667 2.35041461564833e-05
1668 2.34990275203018e-05
1669 2.34939052461414e-05
1670 2.34887884289492e-05
1671 2.3483653421863e-05
1672 2.34785402426496e-05
1673 2.3473421606468e-05
1674 2.34682956943288e-05
1675 2.34631770581473e-05
1676 2.34580566029763e-05
1677 2.34529361478053e-05
1678 2.34478084166767e-05
1679 2.34426970564527e-05
1680 2.3437572963303e-05
1681 2.34324670600472e-05
1682 2.34273393289186e-05
1683 2.34222243307158e-05
1684 2.3417100237566e-05
1685 2.34119761444163e-05
1686 2.34068611462135e-05
1687 2.34017461480107e-05
1688 2.33966220548609e-05
1689 2.33915015996899e-05
1690 2.33863829635084e-05
1691 2.33812770602526e-05
1692 2.33761802519439e-05
1693 2.33710688917199e-05
1694 2.33659557125065e-05
1695 2.33608443522826e-05
1696 2.33557275350904e-05
1697 2.33506179938558e-05
1698 2.33455102716107e-05
1699 2.33404061873443e-05
1700 2.33352984650992e-05
1701 2.33301925618434e-05
1702 2.33250812016195e-05
1703 2.33199662034167e-05
1704 2.33148621191503e-05
1705 2.33097453019582e-05
1706 2.33046393987024e-05
1707 2.32995316764573e-05
1708 2.32944348681485e-05
1709 2.32893198699458e-05
1710 2.3284223061637e-05
1711 2.32791135204025e-05
1712 2.32740094361361e-05
1713 2.3268910808838e-05
1714 2.3263808543561e-05
1715 2.32587208301993e-05
1716 2.32536112889647e-05
1717 2.32485253945924e-05
1718 2.32434231293155e-05
1719 2.32383299589856e-05
1720 2.32332349696662e-05
1721 2.32281490752939e-05
1722 2.32230595429428e-05
1723 2.3217953639687e-05
1724 2.32128768402617e-05
1725 2.3207780031953e-05
1726 2.32026995945489e-05
1727 2.31976046052296e-05
1728 2.31925205298467e-05
1729 2.31874328164849e-05
1730 2.31823451031232e-05
1731 2.31772737606661e-05
1732 2.31721805903362e-05
1733 2.31670856010169e-05
1734 2.3162019715528e-05
1735 2.31569338211557e-05
1736 2.3151855202741e-05
1737 2.31467784033157e-05
1738 2.31416943279328e-05
1739 2.31366193474969e-05
1740 2.31315461860504e-05
1741 2.31264530157205e-05
1742 2.31213762162952e-05
1743 2.31163139687851e-05
1744 2.31112298934022e-05
1745 2.31061549129663e-05
1746 2.31010744755622e-05
1747 2.30959994951263e-05
1748 2.3090922695701e-05
1749 2.30858513532439e-05
1750 2.30807818297762e-05
1751 2.30757123063086e-05
1752 2.30706409638515e-05
1753 2.30655787163414e-05
1754 2.30605128308525e-05
1755 2.30554524023319e-05
1756 2.30503756029066e-05
1757 2.30453242693329e-05
1758 2.30402474699076e-05
1759 2.30351852223976e-05
1760 2.30301247938769e-05
1761 2.30250607273774e-05
1762 2.30200075748144e-05
1763 2.30149453273043e-05
1764 2.30098939937307e-05
1765 2.30048317462206e-05
1766 2.29997749556787e-05
1767 2.29947163461475e-05
1768 2.29896650125738e-05
1769 2.29846045840532e-05
1770 2.29795605264371e-05
1771 2.29745073738741e-05
1772 2.29694487643428e-05
1773 2.29644047067268e-05
1774 2.29593460971955e-05
1775 2.29542947636219e-05
1776 2.2949247068027e-05
1777 2.29441975534428e-05
1778 2.29391425818903e-05
1779 2.29341039812425e-05
1780 2.29290526476689e-05
1781 2.29240158660104e-05
1782 2.29189809033414e-05
1783 2.29139368457254e-05
1784 2.2908909159014e-05
1785 2.29038596444298e-05
1786 2.28988137678243e-05
1787 2.28937842621235e-05
1788 2.28887402045075e-05
1789 2.28836961468915e-05
1790 2.28786575462436e-05
1791 2.28736189455958e-05
1792 2.28685858019162e-05
1793 2.2863554477226e-05
1794 2.28585158765782e-05
1795 2.28534827328986e-05
1796 2.2848458684166e-05
1797 2.28434382734122e-05
1798 2.28383978537749e-05
1799 2.2833381081e-05
1800 2.28283479373204e-05
1801 2.28233147936407e-05
1802 2.28182907449082e-05
1803 2.28132648771862e-05
1804 2.28082481044112e-05
1805 2.28032276936574e-05
1806 2.27982127398718e-05
1807 2.27931886911392e-05
1808 2.27881791943219e-05
1809 2.27831533265999e-05
1810 2.27781310968567e-05
1811 2.27731197810499e-05
1812 2.27681048272643e-05
1813 2.27630880544893e-05
1814 2.27580767386826e-05
1815 2.2753061784897e-05
1816 2.2748054107069e-05
1817 2.27430427912623e-05
1818 2.27380314754555e-05
1819 2.27330147026805e-05
1820 2.2728017938789e-05
1821 2.27229993470246e-05
1822 2.27179807552602e-05
1823 2.27129821723793e-05
1824 2.27079726755619e-05
1825 2.27029722736916e-05
1826 2.26979609578848e-05
1827 2.26929532800568e-05
1828 2.26879601541441e-05
1829 2.26829415623797e-05
1830 2.26779447984882e-05
1831 2.2672940758639e-05
1832 2.26679403567687e-05
1833 2.26629399548983e-05
1834 2.26579468289856e-05
1835 2.26529646170093e-05
1836 2.26479605771601e-05
1837 2.26429820031626e-05
1838 2.26379997911863e-05
1839 2.26330066652736e-05
1840 2.26280135393608e-05
1841 2.26230458792998e-05
1842 2.26180491154082e-05
1843 2.26130650844425e-05
1844 2.2608086510445e-05
1845 2.26031024794793e-05
1846 2.25981311814394e-05
1847 2.25931489694631e-05
1848 2.25881576625397e-05
1849 2.25831845455104e-05
1850 2.25782077905023e-05
1851 2.25732273975154e-05
1852 2.25682397285709e-05
1853 2.25632611545734e-05
1854 2.25582862185547e-05
1855 2.25533094635466e-05
1856 2.25483345275279e-05
1857 2.25433796003927e-05
1858 2.25384101213422e-05
1859 2.25334388233023e-05
1860 2.25284638872836e-05
1861 2.25235107791377e-05
1862 2.2518544938066e-05
1863 2.25135809159838e-05
1864 2.25086223508697e-05
1865 2.25036583287874e-05
1866 2.24987034016522e-05
1867 2.24937357415911e-05
1868 2.24887899094028e-05
1869 2.24838386202464e-05
1870 2.24788818741217e-05
1871 2.24739269469865e-05
1872 2.24689829337876e-05
1873 2.24640298256418e-05
1874 2.24590694415383e-05
1875 2.24541327042971e-05
1876 2.24491759581724e-05
1877 2.24442228500266e-05
1878 2.24392806558171e-05
1879 2.24343293666607e-05
1880 2.24293798964936e-05
1881 2.24244431592524e-05
1882 2.24194900511065e-05
1883 2.24145514948759e-05
1884 2.24096038436983e-05
1885 2.24046634684782e-05
1886 2.23997212742688e-05
1887 2.23947699851124e-05
1888 2.23898332478711e-05
1889 2.23849019675981e-05
1890 2.23799634113675e-05
1891 2.23750321310945e-05
1892 2.23701008508215e-05
1893 2.23651641135802e-05
1894 2.2360218281392e-05
1895 2.23552851821296e-05
1896 2.2350346625899e-05
1897 2.23454298975412e-05
1898 2.23404986172682e-05
1899 2.23355636990163e-05
1900 2.23306396947009e-05
1901 2.23257120524067e-05
1902 2.23207989620278e-05
1903 2.23158767767018e-05
1904 2.23109473154182e-05
1905 2.23060233111028e-05
1906 2.2301106582745e-05
1907 2.22961880353978e-05
1908 2.22912785829976e-05
1909 2.2286367311608e-05
1910 2.22814596781973e-05
1911 2.22765575017547e-05
1912 2.22716407733969e-05
1913 2.22667422349332e-05
1914 2.22618364205118e-05
1915 2.22569397010375e-05
1916 2.22520393435843e-05
1917 2.22471317101736e-05
1918 2.2242229533731e-05
1919 2.22373328142567e-05
1920 2.22324288188247e-05
1921 2.22275284613715e-05
1922 2.22226317418972e-05
1923 2.22177350224229e-05
1924 2.22128328459803e-05
1925 2.22079397644848e-05
1926 2.22030448639998e-05
1927 2.21981408685679e-05
1928 2.21932568820193e-05
1929 2.21883619815344e-05
1930 2.21834707190283e-05
1931 2.21785849134903e-05
1932 2.21736991079524e-05
1933 2.21688114834251e-05
1934 2.21639238588978e-05
1935 2.21590416913386e-05
1936 2.21541577047901e-05
1937 2.21492809941992e-05
1938 2.21443879127037e-05
1939 2.21395112021128e-05
1940 2.21346326725325e-05
1941 2.2129757780931e-05
1942 2.21248865273083e-05
1943 2.21200025407597e-05
1944 2.2115131287137e-05
1945 2.21102473005885e-05
1946 2.21053760469658e-05
1947 2.21005029743537e-05
1948 2.20956371776992e-05
1949 2.20907695620554e-05
1950 2.20858946704539e-05
1951 2.20810288737994e-05
1952 2.20761612581555e-05
1953 2.20713100134162e-05
1954 2.20664460357511e-05
1955 2.20615929720225e-05
1956 2.20567435462726e-05
1957 2.2051890482544e-05
1958 2.20470483327517e-05
1959 2.20421934500337e-05
1960 2.20373476622626e-05
1961 2.20325000555022e-05
1962 2.20276524487417e-05
1963 2.20228121179389e-05
1964 2.20179699681466e-05
1965 2.20131132664392e-05
1966 2.20082729356363e-05
1967 2.20034289668547e-05
1968 2.19985995499883e-05
1969 2.19937610381749e-05
1970 2.19889243453508e-05
1971 2.19840785575798e-05
1972 2.19792345887981e-05
1973 2.19743997149635e-05
1974 2.19695484702243e-05
1975 2.19647208723472e-05
1976 2.19598914554808e-05
1977 2.19550584006356e-05
1978 2.19502217078116e-05
1979 2.19454013858922e-05
1980 2.19405792449834e-05
1981 2.19357461901382e-05
1982 2.19309367821552e-05
1983 2.1926112822257e-05
1984 2.19212906813482e-05
1985 2.19164776353864e-05
1986 2.1911657313467e-05
1987 2.19068442675052e-05
1988 2.19020257645752e-05
1989 2.18972054426558e-05
1990 2.1892392396694e-05
1991 2.18875975406263e-05
1992 2.18827753997175e-05
1993 2.18779769056709e-05
1994 2.18731747736456e-05
1995 2.18683726416202e-05
1996 2.18635668716161e-05
1997 2.18587774725165e-05
1998 2.18539753404912e-05
1999 2.18491823034128e-05
};
\addlegendentry{Test}

\nextgroupplot[
title={4 Layer},
ymin=6.75945588579267e-06, ymax=0.001,
]
\addplot [semithick, black, dashed]
table {%
0 0.0900730829065045
1 0.0858608457880716
2 0.0816006909590214
3 0.0768245668150485
4 0.0690722034778446
5 0.0512649902763466
6 0.0362110334293296
7 0.0279414476050685
8 0.0228160244102279
9 0.0191741408780217
10 0.0163463076848226
11 0.0140244581659014
12 0.012052656665522
13 0.0103501704094621
14 0.00887950045095446
15 0.00762679724721238
16 0.00658223129721591
17 0.00572427062919208
18 0.005019126432065
19 0.00443056007134146
20 0.00392979714827864
21 0.00349921603295419
22 0.00313005177334465
23 0.00281746088482275
24 0.00255511730332122
25 0.00233224539715593
26 0.00213603435031473
27 0.00195827060065312
28 0.00180018007942332
29 0.00166758006090125
30 0.00155775561358951
31 0.00146041295242109
32 0.00137021597773431
33 0.00128769347278042
34 0.00121496902931995
35 0.00115309806377203
36 0.00110132380518735
37 0.0010576517975475
38 0.00101989711993156
39 0.000986328694352778
40 0.000955760830000448
41 0.000927417660932406
42 0.000900783913455901
43 0.000875488226824928
44 0.000851253544681185
45 0.000827984721903855
46 0.00080573923872862
47 0.000784475350769753
48 0.000764079800167868
49 0.000744454035536061
50 0.000725534943759006
51 0.0007072948547678
52 0.000689726060481159
53 0.000672820955552803
54 0.000656561765424613
55 0.000640922687201358
56 0.000625873043607802
57 0.000611397171724093
58 0.00059752092903409
59 0.000584277677470861
60 0.000571649242469145
61 0.000559612982783619
62 0.000548166184804207
63 0.000537294116573624
64 0.000526961671596382
65 0.000517127785239306
66 0.000507756389159416
67 0.000498816660031783
68 0.000490281800030819
69 0.000482126287948859
70 0.000474326568943676
71 0.000466860654796619
72 0.000459709353663129
73 0.000452855386820753
74 0.000446284501265382
75 0.000439986218594868
76 0.000433951736937388
77 0.000428173477397801
78 0.0004226437832718
79 0.000417354658367231
80 0.000412296395249238
81 0.000407458672232982
82 0.000402830924116415
83 0.000398402462925181
84 0.000394162788741899
85 0.000390101474948779
86 0.00038620835955309
87 0.000382473275872523
88 0.000378886988376811
89 0.000375439695100492
90 0.000372121987462985
91 0.000368925433775757
92 0.000365841869002755
93 0.000362862906537013
94 0.000359982474293474
95 0.000357194246300689
96 0.000354492268021052
97 0.000351871967457858
98 0.000349328092170254
99 0.000346856525681005
100 0.000344453299784201
101 0.000342114136766251
102 0.000339835132540619
103 0.000337613018980202
104 0.000335444868601561
105 0.000333327119747651
106 0.000331257791998496
107 0.000329234627391391
108 0.000327255870189447
109 0.000325320642360793
110 0.000323428499517793
111 0.000321578172417958
112 0.000319768513525294
113 0.000317998075497409
114 0.00031626537688112
115 0.000314568944692913
116 0.00031290688336109
117 0.000311277722194821
118 0.000309680816310258
119 0.000308114492933479
120 0.000306578100667328
121 0.000305071296878623
122 0.000303593004905641
123 0.000302142486683958
124 0.00030071956021042
125 0.000299323751714079
126 0.000297954285372271
127 0.000296610441353096
128 0.000295291912024709
129 0.000293997866180007
130 0.000292728336546588
131 0.000291482196359993
132 0.000290259498740397
133 0.000289059650621463
134 0.000287882236695699
135 0.000286726919640046
136 0.000285593453137759
137 0.000284481240541368
138 0.000283390357798415
139 0.000282320664216703
140 0.00028127136376573
141 0.000280242863970178
142 0.00027923417929306
143 0.000278245810330683
144 0.000277277072015636
145 0.000276328086973384
146 0.000275398500254672
147 0.000274488276256572
148 0.000273596996668365
149 0.000272724564458334
150 0.000271871032666127
151 0.000271035898985398
152 0.000270219150597957
153 0.000269420316376075
154 0.000268639311954644
155 0.000267875814505677
156 0.000267129838358452
157 0.000266400933725208
158 0.00026568898038685
159 0.000264993733746148
160 0.00026431477709347
161 0.000263651696693993
162 0.000263004602961322
163 0.000262372831175621
164 0.000261756344732097
165 0.000261154498841639
166 0.000260567032479268
167 0.000259993872996726
168 0.000259434328844084
169 0.000258888315710711
170 0.000258355112767579
171 0.000257834678876634
172 0.000257326542363027
173 0.000256830428374618
174 0.000256345642019558
175 0.000255871971653221
176 0.000255409162204974
177 0.000254956438629013
178 0.000254513553751205
179 0.00025408024171251
180 0.000253655812433825
181 0.000253239991465648
182 0.000252832067114165
183 0.000252431502211437
184 0.00025203791908505
185 0.000251650541815233
186 0.000251268701968381
187 0.000250891447663548
188 0.000250518014809131
189 0.000250147215491362
190 0.000249778294353575
191 0.000249409639141144
192 0.000249040417926949
193 0.000248670000414582
194 0.000248298314827385
195 0.000247926378574922
196 0.000247556110110736
197 0.000247190076689681
198 0.000246829420883425
199 0.000246474676454037
200 0.000246124662929503
201 0.000245778320523489
202 0.000245434887832611
203 0.000245093137114812
204 0.000244751981812878
205 0.000244410503976648
206 0.000244066948023184
207 0.000243719428228909
208 0.000243365508772797
209 0.000243001847953413
210 0.000242623974450851
211 0.000242226609060481
212 0.000241802667370431
213 0.000241345308630514
214 0.000240849720507678
215 0.000240318114443028
216 0.000239764646155531
217 0.000239214544838736
218 0.00023869540520612
219 0.000238223388571157
220 0.000237798539037234
221 0.000237411635332307
222 0.000237050918556747
223 0.000236707849358216
224 0.000236376802182766
225 0.000236054541187987
226 0.000235739395627377
227 0.000235430965091155
228 0.000235128680439326
229 0.00023483269557308
230 0.000234543331861422
231 0.000234261038798422
232 0.000233986430577223
233 0.00023372031850973
234 0.000233464286390965
235 0.000233219203972605
236 0.000232987675289564
237 0.000232772432691301
238 0.000232575912747279
239 0.000232402940592162
240 0.000232258301854434
241 0.00023214753667844
242 0.000232075358508628
243 0.000232043882107291
244 0.000232045844095789
245 0.000232055745456705
246 0.000232020433694894
247 0.000231864943017968
248 0.000231526961144368
249 0.000231008092410908
250 0.000230380178978843
251 0.000229734533372531
252 0.000229127205588024
253 0.000228570887081503
254 0.000228058715106746
255 0.000227577700203578
256 0.000227117426173133
257 0.00022666996206766
258 0.000226229394537351
259 0.000225791057279897
260 0.000225350262300121
261 0.000224902881039194
262 0.000224444409797779
263 0.000223968749177554
264 0.000223469579649323
265 0.000222938112602359
266 0.000222363529464549
267 0.000221733878052059
268 0.000221035850051029
269 0.00022026029282074
270 0.000219403357296718
271 0.000218470089815052
272 0.000217473045168504
273 0.000216427459096262
274 0.000215346746150165
275 0.000214239643303434
276 0.000213111731883468
277 0.000211966218773322
278 0.000210805290959589
279 0.000209629882389114
280 0.000208440585214002
281 0.00020723819346576
282 0.000206022086186882
283 0.000204791962540677
284 0.000203545750257679
285 0.000202279955881105
286 0.000200988743943734
287 0.000199663418224342
288 0.000198288509982566
289 0.000196838658861035
290 0.00019526771329718
291 0.000193491255132017
292 0.000191351509381358
293 0.000188598941861073
294 0.000185053000938259
295 0.000180964683001624
296 0.0001768525988363
297 0.000172978817147396
298 0.000169372470404975
299 0.000166004461381893
300 0.000162844554495223
301 0.000159868138894126
302 0.000157056267857077
303 0.000154393961674752
304 0.000151870265715578
305 0.000149477229982153
306 0.000147209322307162
307 0.000145062267009166
308 0.000143031539977301
309 0.000141111607935803
310 0.000139296439589505
311 0.000137579358759391
312 0.000135953547920546
313 0.000134412971675602
314 0.000132951319462412
315 0.000131562557375749
316 0.00013024116062373
317 0.000128981473395129
318 0.000127778200469682
319 0.000126626445502135
320 0.000125521735147534
321 0.000124460088831787
322 0.000123437822466599
323 0.000122451442152283
324 0.000121497620777215
325 0.000120572933549094
326 0.000119674526028272
327 0.000118800007456343
328 0.000117947943380159
329 0.000117118052031628
330 0.0001163106160836
331 0.000115526234732026
332 0.000114765881126762
333 0.000114029939550638
334 0.000113318406519625
335 0.000112631159204568
336 0.000111967882297392
337 0.000111328047672998
338 0.000110711057537571
339 0.000110115983090964
340 0.000109541766396622
341 0.000108986988702024
342 0.000108450087514219
343 0.000107929455685015
344 0.000107423591567605
345 0.000106931364686128
346 0.000106451282510041
347 0.000105982468374085
348 0.000105524337537114
349 0.000105076008360072
350 0.000104637134699696
351 0.000104207302487206
352 0.000103786422509226
353 0.000103374240734884
354 0.000102970813375218
355 0.000102576073719736
356 0.000102190087697333
357 0.000101812843903796
358 0.000101443929798014
359 0.000101083529559768
360 0.000100731069373694
361 0.000100386015044288
362 0.000100047864317313
363 9.97162834792675e-05
364 9.93906407037078e-05
365 9.90703473557157e-05
366 9.87550505063458e-05
367 9.8444340473236e-05
368 9.81382226233525e-05
369 9.78361453173685e-05
370 9.75384765216821e-05
371 9.72447157820966e-05
372 9.69552347527269e-05
373 9.66698306423079e-05
374 9.63888504917539e-05
375 9.61117402458929e-05
376 9.58390094893957e-05
377 9.55701894322184e-05
378 9.53053558134798e-05
379 9.50443286313885e-05
380 9.47872831294679e-05
381 9.45338088816546e-05
382 9.42836916687876e-05
383 9.40368935668327e-05
384 9.37933946379606e-05
385 9.35524990429333e-05
386 9.33147106181309e-05
387 9.30788954036643e-05
388 9.28454962737154e-05
389 9.26137954877504e-05
390 9.2383878526429e-05
391 9.2155346420005e-05
392 9.19281294002872e-05
393 9.17022745786274e-05
394 9.1477377277253e-05
395 9.12537430390141e-05
396 9.10314304315799e-05
397 9.08104477209311e-05
398 9.05908221492761e-05
399 9.0372725608745e-05
400 9.01560873458607e-05
401 8.99408967072191e-05
402 8.97274125577496e-05
403 8.95153726574449e-05
404 8.93051705901371e-05
405 8.90964920614579e-05
406 8.8889500145702e-05
407 8.86839484834923e-05
408 8.84800489563039e-05
409 8.82778705909952e-05
410 8.80772714827079e-05
411 8.78780493730839e-05
412 8.76805322699662e-05
413 8.74846619420093e-05
414 8.72902350981993e-05
415 8.70973970871584e-05
416 8.69062275559903e-05
417 8.67165413680482e-05
418 8.65283708340788e-05
419 8.63419043450146e-05
420 8.61568957877997e-05
421 8.59733082426336e-05
422 8.57912787661045e-05
423 8.56111290990687e-05
424 8.54322648062104e-05
425 8.52553130078112e-05
426 8.50797206126686e-05
427 8.49056951039984e-05
428 8.47332477770616e-05
429 8.45623659936715e-05
430 8.43931717279626e-05
431 8.42255287641554e-05
432 8.40593977002868e-05
433 8.3894905721138e-05
434 8.37320893296578e-05
435 8.35708007282202e-05
436 8.34112216973419e-05
437 8.32531424658593e-05
438 8.30966425482416e-05
439 8.29417051235737e-05
440 8.278831441307e-05
441 8.26364580201281e-05
442 8.24863013354123e-05
443 8.23374850966729e-05
444 8.21904623506953e-05
445 8.20449351432255e-05
446 8.19009838129621e-05
447 8.1758339116315e-05
448 8.16172728169325e-05
449 8.1477708154883e-05
450 8.13396429345895e-05
451 8.12029946312502e-05
452 8.10679070137136e-05
453 8.0934237795797e-05
454 8.08019068945972e-05
455 8.06709480419461e-05
456 8.0541595986953e-05
457 8.04134623730117e-05
458 8.0286734368921e-05
459 8.01613711232108e-05
460 8.00374020878773e-05
461 7.9914595940996e-05
462 7.97932841886488e-05
463 7.9673185280645e-05
464 7.95544139838474e-05
465 7.94368482649096e-05
466 7.93206045746804e-05
467 7.92055549538873e-05
468 7.90917523569116e-05
469 7.89791891312082e-05
470 7.88679772535753e-05
471 7.87579604804023e-05
472 7.8648958452258e-05
473 7.8541298741186e-05
474 7.84347885532573e-05
475 7.83295048781459e-05
476 7.82254654131028e-05
477 7.81225187012069e-05
478 7.80208436050126e-05
479 7.79202512243652e-05
480 7.78208375299035e-05
481 7.77225857693982e-05
482 7.76254741978732e-05
483 7.7529353018709e-05
484 7.74344508300828e-05
485 7.73405221157949e-05
486 7.72475377530668e-05
487 7.71553593210683e-05
488 7.70643154031821e-05
489 7.69740343002923e-05
490 7.68845820644515e-05
491 7.67958088706185e-05
492 7.67078134723912e-05
493 7.66205932526039e-05
494 7.6534001451023e-05
495 7.64482621728272e-05
496 7.63630744889573e-05
497 7.62786924847584e-05
498 7.61951802961865e-05
499 7.61121855390456e-05
500 7.60300822975069e-05
501 7.59487319659987e-05
502 7.58681047633066e-05
503 7.57884313197602e-05
504 7.57094323968014e-05
505 7.56310284089068e-05
506 7.55536552550969e-05
507 7.547694180469e-05
508 7.5401015967221e-05
509 7.53258815227772e-05
510 7.52513735212309e-05
511 7.51776195571097e-05
512 7.51044228171812e-05
513 7.50320934580865e-05
514 7.49606004646353e-05
515 7.4889439889129e-05
516 7.48190974112125e-05
517 7.47493562478496e-05
518 7.46804277464245e-05
519 7.46119702697001e-05
520 7.45442558264623e-05
521 7.44770474625511e-05
522 7.44104298332123e-05
523 7.43443957773593e-05
524 7.42789620709061e-05
525 7.42142001162923e-05
526 7.41498881673408e-05
527 7.40861766672651e-05
528 7.40227809288285e-05
529 7.39602067352981e-05
530 7.38981063032422e-05
531 7.38364274788239e-05
532 7.37752289007669e-05
533 7.37145410821445e-05
534 7.36543473157288e-05
535 7.35947379493969e-05
536 7.35355294973298e-05
537 7.34767955255696e-05
538 7.34185725702237e-05
539 7.33607475199941e-05
540 7.33033263671246e-05
541 7.32465160299493e-05
542 7.31900407492011e-05
543 7.31339126976612e-05
544 7.30783633322574e-05
545 7.30231092624933e-05
546 7.29682106464698e-05
547 7.29139655947601e-05
548 7.28599528869722e-05
549 7.28062980475859e-05
550 7.27532635759568e-05
551 7.27004745577631e-05
552 7.26481490775181e-05
553 7.25963208007367e-05
554 7.2544616405897e-05
555 7.24934450685074e-05
556 7.24426490078163e-05
557 7.23924444929007e-05
558 7.23422569670618e-05
559 7.22928202587051e-05
560 7.22435730817021e-05
561 7.21948958511594e-05
562 7.21465734964492e-05
563 7.2098814640024e-05
564 7.2051317056084e-05
565 7.20044677713361e-05
566 7.19578069556803e-05
567 7.19118191447876e-05
568 7.18663434386239e-05
569 7.18213370175154e-05
570 7.17769387582242e-05
571 7.17330725876764e-05
572 7.16897490491419e-05
573 7.16474691166979e-05
574 7.16056693133756e-05
575 7.15647044404003e-05
576 7.15245259854669e-05
577 7.14855475401066e-05
578 7.14476685743648e-05
579 7.14107481165627e-05
580 7.13756102127168e-05
581 7.13418402942996e-05
582 7.13098618530713e-05
583 7.1279801680646e-05
584 7.12518402051406e-05
585 7.12265187630123e-05
586 7.120332834513e-05
587 7.11826380133118e-05
588 7.11636384143806e-05
589 7.11457111025974e-05
590 7.11274079255493e-05
591 7.11069678374088e-05
592 7.10817670513582e-05
593 7.10488349575655e-05
594 7.10037590631411e-05
595 7.0941545648869e-05
596 7.08612489610516e-05
597 7.07696931385726e-05
598 7.06784054026836e-05
599 7.05972982781115e-05
600 7.05301311031784e-05
601 7.04753786256636e-05
602 7.04298514833113e-05
603 7.03903755455334e-05
604 7.0354287408018e-05
605 7.03200034794331e-05
606 7.02865387710953e-05
607 7.0253025305315e-05
608 7.0219514284986e-05
609 7.01857365778362e-05
610 7.01517598002965e-05
611 7.0117599319038e-05
612 7.00832803630647e-05
613 7.00487366683452e-05
614 7.00140571865404e-05
615 6.99793095909248e-05
616 6.99443298148121e-05
617 6.99092937341087e-05
618 6.98739003972548e-05
619 6.98385220516684e-05
620 6.98030656091457e-05
621 6.9767471141328e-05
622 6.97316630334418e-05
623 6.96958205637088e-05
624 6.96598062551364e-05
625 6.96238865932249e-05
626 6.95877406104254e-05
627 6.95514878377897e-05
628 6.95153096377984e-05
629 6.94790421711389e-05
630 6.94427964695876e-05
631 6.9406696766805e-05
632 6.93703520913876e-05
633 6.93340678168397e-05
634 6.92979484130755e-05
635 6.92617342181734e-05
636 6.92254625237846e-05
637 6.91894293230651e-05
638 6.91534139178884e-05
639 6.91172493330801e-05
640 6.90811353862841e-05
641 6.90452379963347e-05
642 6.9009423355008e-05
643 6.8973517963163e-05
644 6.89377861509873e-05
645 6.89019464346075e-05
646 6.88662296136992e-05
647 6.88304666596196e-05
648 6.87947166717606e-05
649 6.87589317157252e-05
650 6.87232851935467e-05
651 6.86877156610421e-05
652 6.86519758256547e-05
653 6.8616290712716e-05
654 6.85805959778444e-05
655 6.85449358096927e-05
656 6.85092373693408e-05
657 6.84736148327166e-05
658 6.84377366759747e-05
659 6.84020939161201e-05
660 6.83660862534907e-05
661 6.83302813868636e-05
662 6.82944432703891e-05
663 6.8258361759869e-05
664 6.8222367096619e-05
665 6.81863990870113e-05
666 6.81501788773649e-05
667 6.81139868667913e-05
668 6.80777200443572e-05
669 6.80412462088498e-05
670 6.80048104904074e-05
671 6.79681415943397e-05
672 6.79315921902438e-05
673 6.78948455998807e-05
674 6.78578974095956e-05
675 6.78209597590277e-05
676 6.77840025898509e-05
677 6.77467336593907e-05
678 6.77094859492892e-05
679 6.76721500241229e-05
680 6.7634661534773e-05
681 6.75970960889079e-05
682 6.75593117283787e-05
683 6.75213849478477e-05
684 6.74833346678838e-05
685 6.7445206223482e-05
686 6.74070572159735e-05
687 6.73685812889365e-05
688 6.73300379882373e-05
689 6.72913297942538e-05
690 6.72523680596745e-05
691 6.72134608580658e-05
692 6.71742312678229e-05
693 6.71349447796861e-05
694 6.7095523295535e-05
695 6.7055929696617e-05
696 6.70162156618896e-05
697 6.69761801314432e-05
698 6.69361274911277e-05
699 6.68958492842838e-05
700 6.68554154093689e-05
701 6.6814833144709e-05
702 6.67741364708263e-05
703 6.67332351108977e-05
704 6.66922948984923e-05
705 6.66509495725146e-05
706 6.66095742308433e-05
707 6.65680770287243e-05
708 6.65261901933907e-05
709 6.6484220333507e-05
710 6.64423181954514e-05
711 6.64000436619479e-05
712 6.63575531471376e-05
713 6.63148285606023e-05
714 6.62721210555143e-05
715 6.62292387758612e-05
716 6.61859983542475e-05
717 6.61427544509744e-05
718 6.60990882982067e-05
719 6.60553314896598e-05
720 6.6011386238074e-05
721 6.59673235219316e-05
722 6.5923034523981e-05
723 6.58784558638104e-05
724 6.58336326798311e-05
725 6.57886301809185e-05
726 6.57434707681167e-05
727 6.56980743419429e-05
728 6.56525313544876e-05
729 6.56067432534731e-05
730 6.5560716945375e-05
731 6.55145509540489e-05
732 6.54681796892949e-05
733 6.54214573359013e-05
734 6.53743757036788e-05
735 6.53271472520108e-05
736 6.52797459418745e-05
737 6.52321162372497e-05
738 6.51842911049509e-05
739 6.51361348621056e-05
740 6.50876109607642e-05
741 6.50390961709491e-05
742 6.49900375530403e-05
743 6.49409569438338e-05
744 6.48914948631803e-05
745 6.48417081426563e-05
746 6.47917451862175e-05
747 6.47413899817669e-05
748 6.46909329541738e-05
749 6.46401262140254e-05
750 6.45891095084987e-05
751 6.45377090530512e-05
752 6.44858958314861e-05
753 6.44338533130432e-05
754 6.43816550720544e-05
755 6.43291157468677e-05
756 6.42762295560336e-05
757 6.42230474549403e-05
758 6.41695840902419e-05
759 6.41156156054497e-05
760 6.40615112056745e-05
761 6.40070183071847e-05
762 6.3952323401395e-05
763 6.38971328967841e-05
764 6.38416789655594e-05
765 6.37857228653615e-05
766 6.37295547250005e-05
767 6.36731239419911e-05
768 6.36163108040932e-05
769 6.35590206812253e-05
770 6.3501483322644e-05
771 6.34433976382335e-05
772 6.33851610866524e-05
773 6.3326426397244e-05
774 6.32671839279434e-05
775 6.32076605455723e-05
776 6.31477720129207e-05
777 6.3087445781207e-05
778 6.30266105474675e-05
779 6.29654663839574e-05
780 6.29037160398601e-05
781 6.28417503207856e-05
782 6.27791713251706e-05
783 6.27161115905513e-05
784 6.26526533622496e-05
785 6.2588656646767e-05
786 6.25242563145889e-05
787 6.24593388351968e-05
788 6.23938523460765e-05
789 6.23279425463844e-05
790 6.22613745851671e-05
791 6.21943856013255e-05
792 6.21268575950277e-05
793 6.20587618532416e-05
794 6.19901323991219e-05
795 6.19208500935997e-05
796 6.18511144262849e-05
797 6.17808675400511e-05
798 6.17101109365118e-05
799 6.16386324523432e-05
800 6.15666413589603e-05
801 6.14942390129158e-05
802 6.14212377338201e-05
803 6.13476971954204e-05
804 6.12736662143713e-05
805 6.11989152214202e-05
806 6.11237398023453e-05
807 6.10481564938444e-05
808 6.09719961772726e-05
809 6.08951936769131e-05
810 6.08180323832623e-05
811 6.07402886695733e-05
812 6.06620213048359e-05
813 6.05833202390234e-05
814 6.05041468044002e-05
815 6.04244285327127e-05
816 6.03443926342114e-05
817 6.02638897220231e-05
818 6.01828796575887e-05
819 6.01014538948637e-05
820 6.00195760422177e-05
821 5.99374436376365e-05
822 5.98547380974424e-05
823 5.97718432899796e-05
824 5.96886702647718e-05
825 5.96050909275429e-05
826 5.95210863352236e-05
827 5.9436906729123e-05
828 5.9352484586365e-05
829 5.92677789654772e-05
830 5.91828512241932e-05
831 5.90974986044065e-05
832 5.90120969453286e-05
833 5.89263064701602e-05
834 5.8840320553107e-05
835 5.87540931983691e-05
836 5.86674637285493e-05
837 5.85807265167659e-05
838 5.8493496920183e-05
839 5.84061036335015e-05
840 5.83183114007587e-05
841 5.82302384160016e-05
842 5.81418245696833e-05
843 5.80530381715979e-05
844 5.79637769249075e-05
845 5.7874406389639e-05
846 5.77841233090718e-05
847 5.76943449009567e-05
848 5.76034061694486e-05
849 5.75125495222058e-05
850 5.74196915176609e-05
851 5.73259907383772e-05
852 5.72379294467851e-05
853 5.71415817489651e-05
854 5.70502322408591e-05
855 5.69576613784761e-05
856 5.68622092345095e-05
857 5.67683893611578e-05
858 5.6678293610446e-05
859 5.65759113844896e-05
860 5.64889978941589e-05
861 5.63846931740386e-05
862 5.62950089767848e-05
863 5.61925562436727e-05
864 5.60986400870433e-05
865 5.59981604813705e-05
866 5.5901924343497e-05
867 5.58020512381555e-05
868 5.57044901204525e-05
869 5.560467590963e-05
870 5.55064282110607e-05
871 5.54062034581193e-05
872 5.53073262613187e-05
873 5.52069552028911e-05
874 5.51076170485961e-05
875 5.50068121754066e-05
876 5.49071811993448e-05
877 5.48061668131557e-05
878 5.47063075112912e-05
879 5.46050094006508e-05
880 5.45050283283407e-05
881 5.44035370708684e-05
882 5.43037647228554e-05
883 5.42019954063259e-05
884 5.41023917127366e-05
885 5.40006212214431e-05
886 5.39013073606043e-05
887 5.3799563602297e-05
888 5.37005050655163e-05
889 5.35988102920252e-05
890 5.35000760895817e-05
891 5.33987092599849e-05
892 5.33003704200989e-05
893 5.31992355234934e-05
894 5.31010130823499e-05
895 5.30003864677534e-05
896 5.29025242101966e-05
897 5.28025497956719e-05
898 5.27047804723679e-05
899 5.26053103844977e-05
900 5.25079137550695e-05
901 5.24092032136991e-05
902 5.2312136205046e-05
903 5.22140673713996e-05
904 5.21172691207994e-05
905 5.20199287710928e-05
906 5.19235912742981e-05
907 5.18269072434426e-05
908 5.17311145623959e-05
909 5.16349447039488e-05
910 5.1539705853069e-05
911 5.14441666770911e-05
912 5.13495318751704e-05
913 5.12546260627763e-05
914 5.11604382490797e-05
915 5.10661748667947e-05
916 5.09726052915956e-05
917 5.08789770539882e-05
918 5.07858984306608e-05
919 5.06928833938029e-05
920 5.06003004474564e-05
921 5.05078746139513e-05
922 5.04158400597513e-05
923 5.03239307481825e-05
924 5.02324850160107e-05
925 5.01410362616639e-05
926 5.00501148176606e-05
927 4.99592374841503e-05
928 4.98687503759735e-05
929 4.97783469839419e-05
930 4.96883355249148e-05
931 4.95983670631972e-05
932 4.95087656465406e-05
933 4.94193016180589e-05
934 4.93301744578882e-05
935 4.92411118244718e-05
936 4.91522949630507e-05
937 4.90636738064154e-05
938 4.8975339060083e-05
939 4.88871760377189e-05
940 4.87992087343514e-05
941 4.87113869809264e-05
942 4.86238346700437e-05
943 4.85364187691554e-05
944 4.84493113125003e-05
945 4.8362438375212e-05
946 4.82757773809794e-05
947 4.81892424749238e-05
948 4.81030599601695e-05
949 4.80171207684066e-05
950 4.79313249144298e-05
951 4.78458457659296e-05
952 4.77606820960356e-05
953 4.76757713083013e-05
954 4.75910993789815e-05
955 4.75067551472345e-05
956 4.74227523123242e-05
957 4.73390447766064e-05
958 4.72556740215661e-05
959 4.71727526326996e-05
960 4.70901107405552e-05
961 4.70079476144747e-05
962 4.69261582101884e-05
963 4.6844709054028e-05
964 4.67636652174974e-05
965 4.66830365960883e-05
966 4.66025565183751e-05
967 4.65221884221971e-05
968 4.64418289472709e-05
969 4.63611426084753e-05
970 4.62799912384071e-05
971 4.61977869387435e-05
972 4.61147077122822e-05
973 4.60306034568703e-05
974 4.59454323712786e-05
975 4.58595159263571e-05
976 4.57730025071138e-05
977 4.56862619392003e-05
978 4.55996981258977e-05
979 4.55133405547296e-05
980 4.54274814387645e-05
981 4.53422504875789e-05
982 4.52574785635799e-05
983 4.51732323535718e-05
984 4.50893751562376e-05
985 4.50059142143762e-05
986 4.49227479748515e-05
987 4.48398383809945e-05
988 4.47572499950866e-05
989 4.46747088579021e-05
990 4.45921764793411e-05
991 4.4509828247404e-05
992 4.4427508122169e-05
993 4.43450613332175e-05
994 4.42628371229622e-05
995 4.41803147523956e-05
996 4.4097878143153e-05
997 4.40153855455113e-05
998 4.39327411190978e-05
999 4.38500057515038e-05
1000 4.37672706645742e-05
1001 4.36843619304739e-05
1002 4.36015136055327e-05
1003 4.35184244492367e-05
1004 4.34352634890445e-05
1005 4.33520076631074e-05
1006 4.32685915813617e-05
1007 4.31849851487698e-05
1008 4.31012688354807e-05
1009 4.30174440258687e-05
1010 4.29333445381985e-05
1011 4.28491746866655e-05
1012 4.27648092523233e-05
1013 4.26802425070131e-05
1014 4.25953813317411e-05
1015 4.25103454257207e-05
1016 4.24249841775766e-05
1017 4.2339463649436e-05
1018 4.22535452102579e-05
1019 4.21676361893238e-05
1020 4.20810778827274e-05
1021 4.19942004180977e-05
1022 4.19070947567472e-05
1023 4.18196195326459e-05
1024 4.17316645749584e-05
1025 4.16433490203853e-05
1026 4.15546294097643e-05
1027 4.14654010150173e-05
1028 4.13757695909567e-05
1029 4.12854912911579e-05
1030 4.11948234659102e-05
1031 4.11035665095483e-05
1032 4.10117235306965e-05
1033 4.09194617200607e-05
1034 4.08263816626212e-05
1035 4.07327446853382e-05
1036 4.06381939086486e-05
1037 4.05433936307986e-05
1038 4.04483841075868e-05
1039 4.03552292015756e-05
1040 4.02596364258064e-05
1041 4.01492261291973e-05
1042 4.00500317283559e-05
1043 3.99534590549422e-05
1044 3.98563255264624e-05
1045 3.97579439474309e-05
1046 3.96582587699849e-05
1047 3.95574435003709e-05
1048 3.94554702036013e-05
1049 3.93528229546784e-05
1050 3.92496142573862e-05
1051 3.91461448773593e-05
1052 3.90423421473921e-05
1053 3.89385136531454e-05
1054 3.8834497742594e-05
1055 3.8730444429073e-05
1056 3.86260926722078e-05
1057 3.85207613220473e-05
1058 3.84135500937077e-05
1059 3.83029562781208e-05
1060 3.81870564657068e-05
1061 3.80633789684263e-05
1062 3.7930151890464e-05
1063 3.77869870990348e-05
1064 3.76356217408613e-05
1065 3.74794357919465e-05
1066 3.73220881707918e-05
1067 3.71667154723809e-05
1068 3.70147078617341e-05
1069 3.68662942117718e-05
1070 3.67207783350902e-05
1071 3.65772442284159e-05
1072 3.64345161708949e-05
1073 3.62916155083326e-05
1074 3.614767555149e-05
1075 3.60024000632582e-05
1076 3.58553653043714e-05
1077 3.57063491449594e-05
1078 3.55553518100275e-05
1079 3.54024736661055e-05
1080 3.52477092728047e-05
1081 3.50913527569219e-05
1082 3.49333385093094e-05
1083 3.47738361862563e-05
1084 3.46130602698243e-05
1085 3.44510777997205e-05
1086 3.42878985080119e-05
1087 3.41236628571551e-05
1088 3.39585001351376e-05
1089 3.37923514498092e-05
1090 3.36254922596405e-05
1091 3.34579158168443e-05
1092 3.32897579961392e-05
1093 3.31211027505181e-05
1094 3.29519933283488e-05
1095 3.27825204612016e-05
1096 3.26128633882187e-05
1097 3.24429392932757e-05
1098 3.22731389298762e-05
1099 3.21032919975058e-05
1100 3.19337138255567e-05
1101 3.17644103375869e-05
1102 3.15957168695036e-05
1103 3.14273616529211e-05
1104 3.12596476111556e-05
1105 3.10926001863739e-05
1106 3.09265407760032e-05
1107 3.07614117905549e-05
1108 3.05974941738896e-05
1109 3.04348318778125e-05
1110 3.0273803941346e-05
1111 3.0114114101328e-05
1112 2.99559683298867e-05
1113 2.97994851668667e-05
1114 2.96442414366993e-05
1115 2.94898656581684e-05
1116 2.93357610440618e-05
1117 2.91807570652243e-05
1118 2.90230623960497e-05
1119 2.88612185670445e-05
1120 2.86933582008449e-05
1121 2.85187871943056e-05
1122 2.83376528216195e-05
1123 2.81520244082382e-05
1124 2.79646401561232e-05
1125 2.7778657411659e-05
1126 2.75964520592235e-05
1127 2.74194992186949e-05
1128 2.72483395207246e-05
1129 2.70828145403357e-05
1130 2.69222428315175e-05
1131 2.67660228322105e-05
1132 2.66134394308892e-05
1133 2.64638486768831e-05
1134 2.6316615875525e-05
1135 2.61715417243143e-05
1136 2.60282691565787e-05
1137 2.58864132381594e-05
1138 2.57459484724147e-05
1139 2.56065733689563e-05
1140 2.54682542101629e-05
1141 2.53310427851261e-05
1142 2.51947316911583e-05
1143 2.50593656521877e-05
1144 2.49248039546994e-05
1145 2.47913568145464e-05
1146 2.46588551462423e-05
1147 2.45272597941456e-05
1148 2.43966364124049e-05
1149 2.42669157894208e-05
1150 2.41382899351568e-05
1151 2.40106857347655e-05
1152 2.38840561834763e-05
1153 2.37585369925834e-05
1154 2.36339978130218e-05
1155 2.35105905919871e-05
1156 2.3388312669681e-05
1157 2.32670503971614e-05
1158 2.31469036080512e-05
1159 2.30279080592292e-05
1160 2.29100694445113e-05
1161 2.2793295871774e-05
1162 2.26775430528884e-05
1163 2.25630459880222e-05
1164 2.24496255919841e-05
1165 2.23372093349402e-05
1166 2.22260379866649e-05
1167 2.21158225267952e-05
1168 2.20068898168317e-05
1169 2.18989365995033e-05
1170 2.17921384522886e-05
1171 2.16864564942891e-05
1172 2.15818840392975e-05
1173 2.14783280299002e-05
1174 2.13759265387618e-05
1175 2.12745840263058e-05
1176 2.11743049523723e-05
1177 2.10751637344705e-05
1178 2.09769911307944e-05
1179 2.08797879442576e-05
1180 2.07838180206939e-05
1181 2.06889205216498e-05
1182 2.05950078679962e-05
1183 2.05022343825097e-05
1184 2.04102658258876e-05
1185 2.0319497828325e-05
1186 2.02298618342904e-05
1187 2.01410783683779e-05
1188 2.00533446298815e-05
1189 1.99665811209115e-05
1190 1.98808531095551e-05
1191 1.97962167050036e-05
1192 1.97125487986227e-05
1193 1.96297456180143e-05
1194 1.95478921819851e-05
1195 1.94671216947976e-05
1196 1.93872452326123e-05
1197 1.93084021825977e-05
1198 1.92304293271661e-05
1199 1.91533980616517e-05
1200 1.90772750473892e-05
1201 1.90020114878564e-05
1202 1.89277291392879e-05
1203 1.88543936481267e-05
1204 1.8781811300291e-05
1205 1.87101598617308e-05
1206 1.86393366519155e-05
1207 1.85694069152476e-05
1208 1.85003085183174e-05
1209 1.84319589884296e-05
1210 1.83645477231191e-05
1211 1.82979138685369e-05
1212 1.82320102043813e-05
1213 1.81669326728695e-05
1214 1.81026574495036e-05
1215 1.80390470703173e-05
1216 1.79762721910966e-05
1217 1.79142430845047e-05
1218 1.78529172079796e-05
1219 1.77923088434303e-05
1220 1.77322853585811e-05
1221 1.76731723146399e-05
1222 1.76146852067423e-05
1223 1.75567516447946e-05
1224 1.74995300786425e-05
1225 1.744295068562e-05
1226 1.73870346730591e-05
1227 1.73317050003637e-05
1228 1.7277040739098e-05
1229 1.72228505898412e-05
1230 1.71693627990521e-05
1231 1.71163605221144e-05
1232 1.70639541228941e-05
1233 1.70120963337202e-05
1234 1.69608451751439e-05
1235 1.69099831299017e-05
1236 1.68597264137797e-05
1237 1.68101277691809e-05
1238 1.67608307156305e-05
1239 1.6712179978858e-05
1240 1.66640023652557e-05
1241 1.66163147300817e-05
1242 1.65689550766975e-05
1243 1.65221004451155e-05
1244 1.64756239738513e-05
1245 1.64295926114259e-05
1246 1.63840418233955e-05
1247 1.63391162407341e-05
1248 1.62943903170287e-05
1249 1.62500811609808e-05
1250 1.42418796974889e-05
1251 1.4153692343181e-05
1252 1.40815592170422e-05
1253 1.40484091755155e-05
1254 1.40231732637848e-05
1255 1.40034381062767e-05
1256 1.39869316253538e-05
1257 1.39725783287285e-05
1258 1.39597997836916e-05
1259 1.39482005359071e-05
1260 1.3937510293142e-05
1261 1.39275533748891e-05
1262 1.39181916155214e-05
1263 1.39093346310422e-05
1264 1.39009008458165e-05
1265 1.38928182623251e-05
1266 1.3885052118449e-05
1267 1.38775456471526e-05
1268 1.38702686687253e-05
1269 1.3863203938295e-05
1270 1.38563111740098e-05
1271 1.38495996016748e-05
1272 1.38429926366257e-05
1273 1.38365311753338e-05
1274 1.38301754836571e-05
1275 1.38239024340218e-05
1276 1.38177355673008e-05
1277 1.38116026751807e-05
1278 1.38055659100189e-05
1279 1.37995893296032e-05
1280 1.37936362675598e-05
1281 1.37877369480085e-05
1282 1.37818637491923e-05
1283 1.37760124487111e-05
1284 1.37701743732066e-05
1285 1.37643753402491e-05
1286 1.37585562747233e-05
1287 1.37527668210661e-05
1288 1.37469789679064e-05
1289 1.37411657075148e-05
1290 1.37353480524164e-05
1291 1.37295262441957e-05
1292 1.37237134799998e-05
1293 1.37178724421242e-05
1294 1.37120305213993e-05
1295 1.37061587922223e-05
1296 1.37002968839302e-05
1297 1.36944062626062e-05
1298 1.36884958994443e-05
1299 1.36825949936063e-05
1300 1.36766729668854e-05
1301 1.36707418837053e-05
1302 1.36647958986676e-05
1303 1.36588473633736e-05
1304 1.36528921933869e-05
1305 1.36469266950693e-05
1306 1.36409505510452e-05
1307 1.36349616954116e-05
1308 1.36289869174059e-05
1309 1.36229789777786e-05
1310 1.3616976547226e-05
1311 1.36109665452485e-05
1312 1.36049556746324e-05
1313 1.35989399948263e-05
1314 1.359291317371e-05
1315 1.35868898158975e-05
1316 1.35808651808844e-05
1317 1.35748218657028e-05
1318 1.35687780572861e-05
1319 1.35627408119158e-05
1320 1.35566898666895e-05
1321 1.35506473704083e-05
1322 1.35446069767416e-05
1323 1.35385499753227e-05
1324 1.35324984202138e-05
1325 1.35264328952426e-05
1326 1.35203852747641e-05
1327 1.35143247105655e-05
1328 1.35082703203911e-05
1329 1.35021992928586e-05
1330 1.34961480119008e-05
1331 1.34901000231243e-05
1332 1.34840419730627e-05
1333 1.3477982837055e-05
1334 1.34719253281901e-05
1335 1.34658903802413e-05
1336 1.34598318209574e-05
1337 1.34537951813248e-05
1338 1.34477512053384e-05
1339 1.34417072482999e-05
1340 1.34356775275772e-05
1341 1.34296550413637e-05
1342 1.34236273989785e-05
1343 1.34176003333171e-05
1344 1.34115876218033e-05
1345 1.34055779472675e-05
1346 1.33995854808925e-05
1347 1.3393586943522e-05
1348 1.33875972577376e-05
1349 1.3381618497732e-05
1350 1.33756422181127e-05
1351 1.33696601659257e-05
1352 1.33636946593185e-05
1353 1.33577300260868e-05
1354 1.33517971245093e-05
1355 1.33458478153159e-05
1356 1.33399189738981e-05
1357 1.33339869291168e-05
1358 1.33280773919608e-05
1359 1.33221518480534e-05
1360 1.33162547548693e-05
1361 1.33103657861492e-05
1362 1.3304481237005e-05
1363 1.32986078972645e-05
1364 1.32927376874648e-05
1365 1.32868784697611e-05
1366 1.32810317848471e-05
1367 1.32751862581178e-05
1368 1.32693618291787e-05
1369 1.32635341654937e-05
1370 1.3257713701383e-05
1371 1.32518991720805e-05
1372 1.32461022891069e-05
1373 1.32403076046709e-05
1374 1.32345338771008e-05
1375 1.32287549270416e-05
1376 1.32229914010888e-05
1377 1.32172350889211e-05
1378 1.32115033621242e-05
1379 1.32057710420241e-05
1380 1.3200046955338e-05
1381 1.31943246846807e-05
1382 1.31886066109625e-05
1383 1.31829014851093e-05
1384 1.31772251634743e-05
1385 1.31715470962727e-05
1386 1.31658659530132e-05
1387 1.31601981300615e-05
1388 1.31545249040244e-05
1389 1.3148887787177e-05
1390 1.31432415801195e-05
1391 1.31376128476764e-05
1392 1.3131996339529e-05
1393 1.31263839436476e-05
1394 1.31207840669371e-05
1395 1.31151867221272e-05
1396 1.31095931929318e-05
1397 1.31040368002526e-05
1398 1.30984537160354e-05
1399 1.30928920019831e-05
1400 1.30873372299334e-05
1401 1.30817990324772e-05
1402 1.30762587519465e-05
1403 1.30707200742819e-05
1404 1.30651969462529e-05
1405 1.30596806814746e-05
1406 1.30541782130678e-05
1407 1.30486968448196e-05
1408 1.30432021355394e-05
1409 1.30377176657722e-05
1410 1.30322490049887e-05
1411 1.30267889471014e-05
1412 1.30213367821597e-05
1413 1.30158814037884e-05
1414 1.30104461189736e-05
1415 1.30049955403185e-05
1416 1.29995678429159e-05
1417 1.29941606002622e-05
1418 1.29887514788152e-05
1419 1.2983352721226e-05
1420 1.29779604674714e-05
1421 1.29725693908493e-05
1422 1.29671897018587e-05
1423 1.29618178246934e-05
1424 1.29564553776144e-05
1425 1.29511076411385e-05
1426 1.29457444983198e-05
1427 1.29404153845769e-05
1428 1.29350782621292e-05
1429 1.29297530723586e-05
1430 1.2924440009184e-05
1431 1.29191242968692e-05
1432 1.29138193315133e-05
1433 1.29085214594464e-05
1434 1.29032245720732e-05
1435 1.28979486092954e-05
1436 1.28926664251239e-05
1437 1.28874049171538e-05
1438 1.28821291574729e-05
1439 1.28768813463983e-05
1440 1.2871630077349e-05
1441 1.28663955267783e-05
1442 1.28611604507019e-05
1443 1.28559348396469e-05
1444 1.28507220450066e-05
1445 1.28455040610357e-05
1446 1.28402893502984e-05
1447 1.28350973766326e-05
1448 1.28299126886944e-05
1449 1.28247100429727e-05
1450 1.28195496061541e-05
1451 1.28143653501667e-05
1452 1.28091890789032e-05
1453 1.28040388786369e-05
1454 1.27988801207716e-05
1455 1.27937406470361e-05
1456 1.27886021938176e-05
1457 1.27834605780919e-05
1458 1.27783360097666e-05
1459 1.27732317265483e-05
1460 1.27681093851824e-05
1461 1.27629974070823e-05
1462 1.27579008649311e-05
1463 1.2752799669317e-05
1464 1.27477098110044e-05
1465 1.27426335737961e-05
1466 1.27375486040175e-05
1467 1.27324779635174e-05
1468 1.27274212357401e-05
1469 1.27223493893898e-05
1470 1.27173079912761e-05
1471 1.27122683254062e-05
1472 1.27072230548346e-05
1473 1.27021866148288e-05
1474 1.26971685506424e-05
1475 1.26921463336297e-05
1476 1.26871393891411e-05
1477 1.26821333017446e-05
1478 1.2677132097257e-05
1479 1.26721337402695e-05
1480 1.26671447926441e-05
1481 1.26621550267103e-05
1482 1.26571743376639e-05
1483 1.26522056686322e-05
1484 1.2647254465333e-05
1485 1.26422761557166e-05
1486 1.26373208209074e-05
1487 1.26323893188858e-05
1488 1.26274447452464e-05
1489 1.26224956794966e-05
1490 1.26175595808557e-05
1491 1.26126519160626e-05
1492 1.26077257774545e-05
1493 1.26028042055637e-05
1494 1.25978958719723e-05
1495 1.25929980097084e-05
1496 1.25880922059451e-05
1497 1.25832060546098e-05
1498 1.25783182213605e-05
1499 1.25734435707514e-05
1500 1.25685623935112e-05
1501 1.25636983548579e-05
1502 1.25588278407918e-05
1503 1.25539772737326e-05
1504 1.25491290227468e-05
1505 1.25442819077411e-05
1506 1.25394513513418e-05
1507 1.25346087245977e-05
1508 1.25297807560543e-05
1509 1.25249445108722e-05
1510 1.25201296207559e-05
1511 1.25153200375057e-05
1512 1.25105078113326e-05
1513 1.25057047751416e-05
1514 1.25009126830851e-05
1515 1.24961069000544e-05
1516 1.24913250945842e-05
1517 1.24865509786668e-05
1518 1.24817684922599e-05
1519 1.24770005787885e-05
1520 1.24722333971761e-05
1521 1.24674740410076e-05
1522 1.24627190354332e-05
1523 1.24579562612581e-05
1524 1.24532240792045e-05
1525 1.24484761053386e-05
1526 1.24437354601289e-05
1527 1.24390135128512e-05
1528 1.24342844628108e-05
1529 1.24295533829866e-05
1530 1.24248413057397e-05
1531 1.24201500382171e-05
1532 1.24154311394638e-05
1533 1.24107414540831e-05
1534 1.24060430684025e-05
1535 1.24013587286716e-05
1536 1.23966755657771e-05
1537 1.23919843574664e-05
1538 1.23873254066117e-05
1539 1.23826486652471e-05
1540 1.23779746417085e-05
1541 1.23733341670373e-05
1542 1.23686742294164e-05
1543 1.23640231111111e-05
1544 1.23593739624894e-05
1545 1.23547337915753e-05
1546 1.23500946944688e-05
1547 1.23454696362065e-05
1548 1.23408494111151e-05
1549 1.23362309905062e-05
1550 1.23316111751611e-05
1551 1.23270051644771e-05
1552 1.23223948248115e-05
1553 1.23177905277198e-05
1554 1.23132040480793e-05
1555 1.23086118065293e-05
1556 1.2304011374494e-05
1557 1.22994397351306e-05
1558 1.2294857561083e-05
1559 1.2290285087128e-05
1560 1.22857249706954e-05
1561 1.2281153868976e-05
1562 1.22765940420895e-05
1563 1.22720347851176e-05
1564 1.22675044990454e-05
1565 1.22629558664637e-05
1566 1.22584269108827e-05
1567 1.22538853976432e-05
1568 1.22493583149345e-05
1569 1.22448245090112e-05
1570 1.22403034700606e-05
1571 1.22357935508077e-05
1572 1.22312773983187e-05
1573 1.22267723824028e-05
1574 1.22222686845438e-05
1575 1.22177738009673e-05
1576 1.22132803106467e-05
1577 1.22087959226747e-05
1578 1.22043068015998e-05
1579 1.21998212746869e-05
1580 1.21953427608309e-05
1581 1.21908786402021e-05
1582 1.21864039431448e-05
1583 1.21819535984595e-05
1584 1.21774912799448e-05
1585 1.2173042705695e-05
1586 1.21685872726355e-05
1587 1.21641542116061e-05
1588 1.21597024736649e-05
1589 1.21552651280628e-05
1590 1.21508583224757e-05
1591 1.21464218973225e-05
1592 1.21419812160184e-05
1593 1.21375885777653e-05
1594 1.21331686460853e-05
1595 1.2128753809589e-05
1596 1.2124349633306e-05
1597 1.21199427128479e-05
1598 1.21155570071233e-05
1599 1.21111663196061e-05
1600 1.21067755394222e-05
1601 1.2102387918785e-05
1602 1.20980203597298e-05
1603 1.20936272791639e-05
1604 1.20892621623033e-05
1605 1.20849080905335e-05
1606 1.20805414238016e-05
1607 1.20761796310968e-05
1608 1.20718345861803e-05
1609 1.2067489029377e-05
1610 1.20631296679126e-05
1611 1.20587953749881e-05
1612 1.20544611525257e-05
1613 1.20501206293255e-05
1614 1.20458069676014e-05
1615 1.20414700598796e-05
1616 1.2037148321357e-05
1617 1.2032843904386e-05
1618 1.20285263494798e-05
1619 1.20242181228155e-05
1620 1.20199071587853e-05
1621 1.20156072502079e-05
1622 1.20113142842252e-05
1623 1.2007029125923e-05
1624 1.2002734033049e-05
1625 1.19984514093119e-05
1626 1.19941828620185e-05
1627 1.19898873389701e-05
1628 1.19856201343064e-05
1629 1.19813705345232e-05
1630 1.19770956971503e-05
1631 1.19728378672098e-05
1632 1.19685844393776e-05
1633 1.19643299134609e-05
1634 1.19600933752295e-05
1635 1.19558541846014e-05
1636 1.1951606276206e-05
1637 1.19473707786237e-05
1638 1.19431454382498e-05
1639 1.1938913845988e-05
1640 1.19346910167906e-05
1641 1.1930463806209e-05
1642 1.19262428448508e-05
1643 1.19220339008426e-05
1644 1.19178363749602e-05
1645 1.19136288612154e-05
1646 1.19094507230836e-05
1647 1.19052439731722e-05
1648 1.1901048325195e-05
1649 1.18968692734237e-05
1650 1.18926740449628e-05
1651 1.1888488698375e-05
1652 1.18843132697795e-05
1653 1.18801485312995e-05
1654 1.18759801059909e-05
1655 1.1871813217823e-05
1656 1.18676603557688e-05
1657 1.18635015287083e-05
1658 1.18593442826054e-05
1659 1.18551899873272e-05
1660 1.18510439888198e-05
1661 1.18468982236071e-05
1662 1.1842764218765e-05
1663 1.18386308972281e-05
1664 1.18345052101769e-05
1665 1.18303749273944e-05
1666 1.18262606377447e-05
1667 1.18221425135327e-05
1668 1.1818036388019e-05
1669 1.18139074591165e-05
1670 1.18098003305533e-05
1671 1.18057047110061e-05
1672 1.1801589125208e-05
1673 1.17975030968035e-05
1674 1.17934176513401e-05
1675 1.17893292429135e-05
1676 1.17852302100966e-05
1677 1.17811502559443e-05
1678 1.17770671958321e-05
1679 1.17729973003004e-05
1680 1.1768930602211e-05
1681 1.17648576889451e-05
1682 1.17607844106378e-05
1683 1.17567298882904e-05
1684 1.17526746888548e-05
1685 1.17486367372521e-05
1686 1.17445731371291e-05
1687 1.17405308248664e-05
1688 1.17364883778966e-05
1689 1.1732448644608e-05
1690 1.17284150187302e-05
1691 1.17243932796403e-05
1692 1.17203443619947e-05
1693 1.171633268419e-05
1694 1.17123057152095e-05
1695 1.17082933117629e-05
1696 1.17042880285467e-05
1697 1.17002833833387e-05
1698 1.16962746294173e-05
1699 1.16922771781584e-05
1700 1.1688260802778e-05
1701 1.168427325234e-05
1702 1.16802866937012e-05
1703 1.16762798816339e-05
1704 1.16723221887938e-05
1705 1.16683305577681e-05
1706 1.16643505897092e-05
1707 1.16603826499547e-05
1708 1.16564052084674e-05
1709 1.16524254183403e-05
1710 1.16484816462166e-05
1711 1.16445149898799e-05
1712 1.16405697963747e-05
1713 1.16366019439577e-05
1714 1.1632654194571e-05
1715 1.16287112937504e-05
1716 1.16247636296289e-05
1717 1.16208228456334e-05
1718 1.16168856454379e-05
1719 1.16129411932656e-05
1720 1.16090257134734e-05
1721 1.16050967446194e-05
1722 1.16011716011499e-05
1723 1.15972574879682e-05
1724 1.1593331320429e-05
1725 1.15894316445875e-05
1726 1.15855189720312e-05
1727 1.15816154568904e-05
1728 1.15777097411396e-05
1729 1.15738041378914e-05
1730 1.15699144274117e-05
1731 1.15660136283206e-05
1732 1.1562126750538e-05
1733 1.15582425088689e-05
1734 1.15543641975672e-05
1735 1.15504923794418e-05
1736 1.15466106699695e-05
1737 1.15427483861434e-05
1738 1.15388649586379e-05
1739 1.1535000958555e-05
1740 1.1531135882592e-05
1741 1.15272700703291e-05
1742 1.15234142787912e-05
1743 1.15195731780204e-05
1744 1.15157183341689e-05
1745 1.15118777831806e-05
1746 1.15080330476876e-05
1747 1.15042016117783e-05
1748 1.1500366780363e-05
1749 1.14965399671263e-05
1750 1.14926900186182e-05
1751 1.14888737616781e-05
1752 1.14850528876905e-05
1753 1.14812307415354e-05
1754 1.14774196369183e-05
1755 1.1473600327309e-05
1756 1.14697947367996e-05
1757 1.14659765791577e-05
1758 1.1462174710708e-05
1759 1.14583712041612e-05
1760 1.14545828354314e-05
1761 1.14507887515695e-05
1762 1.14469951100205e-05
1763 1.14432055212295e-05
1764 1.14394197655206e-05
1765 1.14356366405962e-05
1766 1.14318661532664e-05
1767 1.14280864747703e-05
1768 1.14243015876999e-05
1769 1.14205384991924e-05
1770 1.14167847620195e-05
1771 1.14130229604825e-05
1772 1.14092622371942e-05
1773 1.14055151779387e-05
1774 1.14017491797587e-05
1775 1.13980042154201e-05
1776 1.13942584931692e-05
1777 1.13904999106869e-05
1778 1.13867610111266e-05
1779 1.13830272621129e-05
1780 1.13792968443605e-05
1781 1.13755688232094e-05
1782 1.13718418684646e-05
1783 1.13681071172896e-05
1784 1.13643947005452e-05
1785 1.13606774254649e-05
1786 1.13569641036667e-05
1787 1.13532481324323e-05
1788 1.13495362195811e-05
1789 1.13458311116569e-05
1790 1.13421286442872e-05
1791 1.13384270520693e-05
1792 1.1334729510241e-05
1793 1.13310306995018e-05
1794 1.13273400321745e-05
1795 1.1323658065443e-05
1796 1.13199817937115e-05
1797 1.13162861262358e-05
1798 1.13125967627544e-05
1799 1.1308923348885e-05
1800 1.13052519612467e-05
1801 1.13015819949898e-05
1802 1.12979109789061e-05
1803 1.12942593505731e-05
1804 1.12905956942318e-05
1805 1.12869371318898e-05
1806 1.1283277250899e-05
1807 1.12796234705096e-05
1808 1.12759797135098e-05
1809 1.12723393333643e-05
1810 1.12686856074499e-05
1811 1.12650450082204e-05
1812 1.12614121074254e-05
1813 1.12577811671362e-05
1814 1.12541443080261e-05
1815 1.12505221977107e-05
1816 1.1246885709267e-05
1817 1.12432689209167e-05
1818 1.12396506762498e-05
1819 1.12360219756506e-05
1820 1.12324101723497e-05
1821 1.12288047423211e-05
1822 1.12251981834177e-05
1823 1.12215937380829e-05
1824 1.12179859946456e-05
1825 1.12143812929233e-05
1826 1.12107998970027e-05
1827 1.12071995796252e-05
1828 1.12036110235062e-05
1829 1.12000188095725e-05
1830 1.11964209595546e-05
1831 1.11928444444705e-05
1832 1.11892779465587e-05
1833 1.11857073926321e-05
1834 1.11821214764755e-05
1835 1.11785566018578e-05
1836 1.11749844681578e-05
1837 1.11714172010835e-05
1838 1.11678570959839e-05
1839 1.11642959576367e-05
1840 1.11607459540863e-05
1841 1.11571875294203e-05
1842 1.11536421914712e-05
1843 1.11500965633837e-05
1844 1.11465469598097e-05
1845 1.11430063564436e-05
1846 1.1139459806427e-05
1847 1.1135928651503e-05
1848 1.11323976792477e-05
1849 1.11288580676809e-05
1850 1.11253411848959e-05
1851 1.11218119937343e-05
1852 1.11182744510311e-05
1853 1.11147470504373e-05
1854 1.11112368553397e-05
1855 1.11077221613224e-05
1856 1.11042151171858e-05
1857 1.11007083688127e-05
1858 1.10972018383393e-05
1859 1.10937004989727e-05
1860 1.10902098630442e-05
1861 1.10867010617947e-05
1862 1.10832070641109e-05
1863 1.10797118066941e-05
1864 1.10762312906706e-05
1865 1.10727317241081e-05
1866 1.10692457573336e-05
1867 1.10657657608944e-05
1868 1.10622816746897e-05
1869 1.10588227191973e-05
1870 1.10553416054297e-05
1871 1.10518723784499e-05
1872 1.10483999635017e-05
1873 1.10449271417679e-05
1874 1.10414665434414e-05
1875 1.10380199638271e-05
1876 1.10345622775417e-05
1877 1.10310965680445e-05
1878 1.10276612833952e-05
1879 1.10242061408528e-05
1880 1.10207638070274e-05
1881 1.10173181001964e-05
1882 1.10138746152918e-05
1883 1.10104267913395e-05
1884 1.10070014163928e-05
1885 1.10035780137944e-05
1886 1.10001399198367e-05
1887 1.09967131235085e-05
1888 1.09933068106471e-05
1889 1.09898749753867e-05
1890 1.09864586631166e-05
1891 1.098303716122e-05
1892 1.09796247681852e-05
1893 1.09762074063842e-05
1894 1.09728032520925e-05
1895 1.09693987034494e-05
1896 1.09659951164076e-05
1897 1.09625953591911e-05
1898 1.09591993009417e-05
1899 1.09557919782214e-05
1900 1.09523974571128e-05
1901 1.09490108662532e-05
1902 1.09456242934532e-05
1903 1.09422361675252e-05
1904 1.09388487382252e-05
1905 1.09354722187168e-05
1906 1.09320828768726e-05
1907 1.09287215990979e-05
1908 1.09253468210113e-05
1909 1.09219826288272e-05
1910 1.09186160447787e-05
1911 1.09152447936485e-05
1912 1.09118810064738e-05
1913 1.09085169561022e-05
1914 1.09051673007343e-05
1915 1.09018007572459e-05
1916 1.08984560286037e-05
1917 1.08951002237845e-05
1918 1.08917605802598e-05
1919 1.08884171415487e-05
1920 1.088508212721e-05
1921 1.08817487169214e-05
1922 1.08784053421592e-05
1923 1.08750660254842e-05
1924 1.08717425109835e-05
1925 1.08684179978743e-05
1926 1.08650816450506e-05
1927 1.08617667322619e-05
1928 1.08584399939697e-05
1929 1.08551223953294e-05
1930 1.08518044002655e-05
1931 1.08484971163373e-05
1932 1.08451720229515e-05
1933 1.08418775702409e-05
1934 1.08385694141215e-05
1935 1.08352420253865e-05
1936 1.08319568867069e-05
1937 1.08286646369748e-05
1938 1.08253701327499e-05
1939 1.08220706298567e-05
1940 1.0818787262501e-05
1941 1.08154918505576e-05
1942 1.08121967740497e-05
1943 1.08089108241671e-05
1944 1.08056439165338e-05
1945 1.0802363247168e-05
1946 1.07990883388235e-05
1947 1.07958065284445e-05
1948 1.07925417598409e-05
1949 1.07892718415788e-05
1950 1.07860044774218e-05
1951 1.07827394506543e-05
1952 1.07794766692019e-05
1953 1.07762218336897e-05
1954 1.0772963869717e-05
1955 1.07697139772137e-05
1956 1.07664546720917e-05
1957 1.07632001065857e-05
1958 1.07599603849096e-05
1959 1.07567052018235e-05
1960 1.07534686482798e-05
1961 1.07502257199836e-05
1962 1.07469940481645e-05
1963 1.07437484947089e-05
1964 1.07405157452334e-05
1965 1.07372945888547e-05
1966 1.07340681368602e-05
1967 1.0730842875913e-05
1968 1.07276265130333e-05
1969 1.07244100936062e-05
1970 1.07211759348876e-05
1971 1.07179714567233e-05
1972 1.07147485453041e-05
1973 1.07115264054158e-05
1974 1.0708326782923e-05
1975 1.07051219820538e-05
1976 1.07019170461816e-05
1977 1.06987070796668e-05
1978 1.06955064579732e-05
1979 1.06923238982759e-05
1980 1.06891315159174e-05
1981 1.06859377086247e-05
1982 1.06827463728365e-05
1983 1.0679545766538e-05
1984 1.06763674849001e-05
1985 1.06731862243118e-05
1986 1.06699912996907e-05
1987 1.06668178710597e-05
1988 1.06636444794361e-05
1989 1.06604655988699e-05
1990 1.06573018889833e-05
1991 1.06541399971979e-05
1992 1.06509702174999e-05
1993 1.06478040266348e-05
1994 1.06446359735557e-05
1995 1.06414732311914e-05
1996 1.06383204929728e-05
1997 1.06351574575096e-05
1998 1.06320100471772e-05
1999 1.06288660826811e-05
};
\addlegendentry{Train}
\addplot [semithick, black]
table {%
0 0.090780146420002
1 0.0864395424723625
2 0.0818765088915825
3 0.0762186869978905
4 0.0636239573359489
5 0.0425069406628609
6 0.0306675806641579
7 0.0238344538956881
8 0.0192671436816454
9 0.0158791393041611
10 0.0131847281008959
11 0.0109455604106188
12 0.00903993938118219
13 0.00741039495915174
14 0.0060381144285202
15 0.00492167240008712
16 0.0040519293397665
17 0.00340097793377936
18 0.00292235636152327
19 0.00256020622327924
20 0.00226605636999011
21 0.00201351591385901
22 0.00179630203638226
23 0.00161442440003157
24 0.00146435375791043
25 0.00133856351021677
26 0.00123035768046975
27 0.00113828480243683
28 0.00106478983070701
29 0.00100760557688773
30 0.000958102114964277
31 0.000911820563487709
32 0.000869181181769818
33 0.000831412442494184
34 0.000799110159277916
35 0.000771881139371544
36 0.000748652673792094
37 0.000728273531422019
38 0.000709852902218699
39 0.000692785135470331
40 0.00067667500115931
41 0.000661263940855861
42 0.000646364991553128
43 0.000631820934358984
44 0.000617548997979611
45 0.000603641790803522
46 0.000590138079132885
47 0.000576968188397586
48 0.000564101501367986
49 0.000551564095076174
50 0.000539412721991539
51 0.000527704600244761
52 0.000516459811478853
53 0.000505649368278682
54 0.000495211570523679
55 0.000485077733173966
56 0.000475199776701629
57 0.000465581833850592
58 0.000456270267022774
59 0.000447272148448974
60 0.000438579882029444
61 0.000430216285167262
62 0.000422204495407641
63 0.000414540350902826
64 0.000407204788643867
65 0.000400176708353683
66 0.000393437658203766
67 0.000386970583349466
68 0.000380760262487456
69 0.000374793307855725
70 0.000369060086086392
71 0.000363554718205705
72 0.000358271499862894
73 0.000353210314642638
74 0.000348367495462298
75 0.000343737425282598
76 0.000339311955031008
77 0.000335082790115848
78 0.000331041839672253
79 0.000327181012835354
80 0.000323494285112247
81 0.000319975457387045
82 0.000316616060445085
83 0.000313409720547497
84 0.000310346571495757
85 0.000307420006720349
86 0.000304621644318104
87 0.000301944819511846
88 0.000299384642858058
89 0.000296934071229771
90 0.000294588360702619
91 0.000292341428576037
92 0.000290187628706917
93 0.000288120267214254
94 0.000286134047200903
95 0.00028422279865481
96 0.000282381050055847
97 0.000280603620922193
98 0.000278885709121823
99 0.000277222396107391
100 0.000275609723757952
101 0.000274043442914262
102 0.000272520730504766
103 0.000271038385108113
104 0.000269592856056988
105 0.000268182368017733
106 0.000266804709099233
107 0.000265458133071661
108 0.000264142319792882
109 0.000262856541667134
110 0.000261599867371842
111 0.000260371685726568
112 0.000259170541539788
113 0.000257995940046385
114 0.000256845669355243
115 0.000255719060078263
116 0.000254614511504769
117 0.000253531470661983
118 0.000252468889812008
119 0.000251426681643352
120 0.000250404118560255
121 0.000249400560278445
122 0.00024841568665579
123 0.000247449119342491
124 0.000246500392677262
125 0.000245568720856681
126 0.000244653288973495
127 0.000243753718677908
128 0.000242869282374159
129 0.000241999441641383
130 0.000241143265157007
131 0.00024030078202486
132 0.000239471046370454
133 0.000238654014538042
134 0.000237849017139524
135 0.000237055937759578
136 0.000236274325288832
137 0.000235504339798354
138 0.000234745573834516
139 0.000233997910981998
140 0.00023326143855229
141 0.000232535749091767
142 0.000231821031775326
143 0.00023111721384339
144 0.00023042454267852
145 0.000229742741794325
146 0.000229071796638891
147 0.000228411969146691
148 0.000227763070142828
149 0.000227124910452403
150 0.000226498144911602
151 0.00022588214778807
152 0.000225277108256705
153 0.000224682575208135
154 0.000224098868784495
155 0.000223525668843649
156 0.000222963208216242
157 0.000222411123104393
158 0.000221869369852357
159 0.00022133813763503
160 0.00022081661154516
161 0.000220305038965307
162 0.000219803492655046
163 0.000219311274122447
164 0.000218829067307524
165 0.000218355853576213
166 0.000217892011278309
167 0.000217437060200609
168 0.000216991102206521
169 0.000216553758946247
170 0.000216124812141061
171 0.000215704305446707
172 0.000215292195207439
173 0.000214887753827497
174 0.00021449092309922
175 0.000214101775782183
176 0.000213719657040201
177 0.000213344392250292
178 0.000212975763133727
179 0.000212613143958151
180 0.000212256301892921
181 0.000211904436582699
182 0.000211557300644927
183 0.000211213497095741
184 0.000210872196475975
185 0.000210532409255393
186 0.000210192505619489
187 0.000209850884857588
188 0.000209505582461134
189 0.000209154633921571
190 0.000208795798243955
191 0.000208427431061864
192 0.000208049023058265
193 0.000207661942113191
194 0.000207270422833972
195 0.000206881726626307
196 0.000206505646929145
197 0.000206149998120964
198 0.000205818781978451
199 0.000205510470550507
200 0.000205220290808938
201 0.000204944110009819
202 0.000204677315196022
203 0.000204417709028348
204 0.000204163050511852
205 0.000203912248252891
206 0.00020366364333313
207 0.00020341610070318
208 0.000203167350264266
209 0.000202914539841004
210 0.000202653638552874
211 0.000202378607355058
212 0.000202080977032892
213 0.000201750226551667
214 0.000201376897166483
215 0.000200957380002365
216 0.000200501977815293
217 0.000200034904992208
218 0.000199584261281416
219 0.000199169255211018
220 0.000198795198230073
221 0.000198456924408674
222 0.000198146517504938
223 0.000197857050807215
224 0.000197583547560498
225 0.000197321860468946
226 0.000197070534341037
227 0.000196827793843113
228 0.000196592955035158
229 0.000196365974261425
230 0.000196146385860629
231 0.000195934466319159
232 0.000195730390259996
233 0.000195534361409955
234 0.000195346554392017
235 0.00019516734755598
236 0.000194996289792471
237 0.000194833366549574
238 0.000194676817045547
239 0.000194524269318208
240 0.000194371561519802
241 0.00019421313481871
242 0.000194040927453898
243 0.000193849133211188
244 0.000193638450582512
245 0.000193428903003223
246 0.00019326045003254
247 0.000193167783436365
248 0.00019312983204145
249 0.000193068844964728
250 0.000192923384020105
251 0.000192690815310925
252 0.000192400664673187
253 0.000192078994587064
254 0.0001917410409078
255 0.000191392580745742
256 0.000191035491297953
257 0.000190671824384481
258 0.000190300779649988
259 0.000189923273865134
260 0.000189537080586888
261 0.000189143655006774
262 0.00018874101806432
263 0.000188328063813969
264 0.000187903453479521
265 0.000187465368071571
266 0.000187012657988816
267 0.000186540142749436
268 0.000186046134331264
269 0.000185525772394612
270 0.000184973745490424
271 0.000184387405170128
272 0.000183765238034539
273 0.000183106312761083
274 0.000182409494300373
275 0.000181676135980524
276 0.000180905146407895
277 0.000180099334102124
278 0.000179260066943243
279 0.000178388887434267
280 0.000177486974280328
281 0.000176556321093813
282 0.000175597058841959
283 0.000174610206158832
284 0.000173592634382658
285 0.000172541462234221
286 0.000171450155903585
287 0.000170305735082366
288 0.000169086299138144
289 0.000167751277331263
290 0.000166221303516068
291 0.000164341458003037
292 0.000161827090778388
293 0.00015835180238355
294 0.000154081790242344
295 0.000149769097333774
296 0.000145912330481224
297 0.000142580101964995
298 0.000139688330818899
299 0.000137144437758252
300 0.000134871967020445
301 0.000132808549096808
302 0.000130903514218517
303 0.000129115214804187
304 0.00012741178215947
305 0.000125769976875745
306 0.000124176745885052
307 0.000122626792290248
308 0.000121120043331757
309 0.000119660151540302
310 0.000118251293315552
311 0.000116897615953349
312 0.000115601724246517
313 0.0001143660483649
314 0.000113190428237431
315 0.000112074834760278
316 0.000111017099698074
317 0.000110015746031422
318 0.000109069107566029
319 0.000108177402580623
320 0.000107341242255643
321 0.000106562933069654
322 0.000105844934296329
323 0.000105184240965173
324 0.000104572121927049
325 0.000103995706012938
326 0.000103440885141026
327 0.000102896497992333
328 0.000102355268609244
329 0.000101813078799751
330 0.000101267956779338
331 0.000100720244518016
332 0.000100173987448215
333 9.963296179194e-05
334 9.91051128949039e-05
335 9.85970764304511e-05
336 9.81158955255523e-05
337 9.76682058535516e-05
338 9.72569760051556e-05
339 9.68834938248619e-05
340 9.65456492849626e-05
341 9.62381091085263e-05
342 9.5954594144132e-05
343 9.56879375735298e-05
344 9.54314236878417e-05
345 9.5179580966942e-05
346 9.49285822571255e-05
347 9.46759464568458e-05
348 9.44210332818329e-05
349 9.41604084800929e-05
350 9.38937373575754e-05
351 9.36191863729618e-05
352 9.33361588977277e-05
353 9.3043599918019e-05
354 9.27419023355469e-05
355 9.24321793718264e-05
356 9.21169048524462e-05
357 9.1798065113835e-05
358 9.14770134841092e-05
359 9.11576062208042e-05
360 9.08412912394851e-05
361 9.05309352674522e-05
362 9.0227993496228e-05
363 8.99345395737328e-05
364 8.96524506970309e-05
365 8.93831675057299e-05
366 8.91274830792099e-05
367 8.88857248355635e-05
368 8.86575580807403e-05
369 8.84433757164516e-05
370 8.82414024090394e-05
371 8.80510851857252e-05
372 8.78705541254021e-05
373 8.7698943389114e-05
374 8.7534477643203e-05
375 8.7376196461264e-05
376 8.7223008449655e-05
377 8.70742442202754e-05
378 8.69294162839651e-05
379 8.67877170094289e-05
380 8.66490008775145e-05
381 8.65130350575782e-05
382 8.63794048200361e-05
383 8.62481101648882e-05
384 8.6119252955541e-05
385 8.59922729432583e-05
386 8.5867453890387e-05
387 8.5744439275004e-05
388 8.56236947583966e-05
389 8.55047546792775e-05
390 8.53878154885024e-05
391 8.52728626341559e-05
392 8.51598451845348e-05
393 8.50484866532497e-05
394 8.49381322041154e-05
395 8.48294512252323e-05
396 8.47215706016868e-05
397 8.46146722324193e-05
398 8.45086615299806e-05
399 8.44029564177617e-05
400 8.42979861772619e-05
401 8.41930523165502e-05
402 8.40887878439389e-05
403 8.39844578877091e-05
404 8.38808555272408e-05
405 8.37773914099671e-05
406 8.36742910905741e-05
407 8.35710670799017e-05
408 8.34680758998729e-05
409 8.33656085887924e-05
410 8.32636142149568e-05
411 8.31609722808935e-05
412 8.30592980491929e-05
413 8.29575874377042e-05
414 8.28560077934526e-05
415 8.27544354251586e-05
416 8.26537943794392e-05
417 8.25524111860432e-05
418 8.24517410364933e-05
419 8.23514492367394e-05
420 8.22505971882492e-05
421 8.21499197627418e-05
422 8.20494969957508e-05
423 8.19496926851571e-05
424 8.18492117105052e-05
425 8.17495674709789e-05
426 8.16495303297415e-05
427 8.15495368442498e-05
428 8.14496379462071e-05
429 8.13500082585961e-05
430 8.12500438769348e-05
431 8.1150523328688e-05
432 8.10504134278744e-05
433 8.09509147075005e-05
434 8.08508193586022e-05
435 8.07512624305673e-05
436 8.06516545708291e-05
437 8.05520758149214e-05
438 8.04520459496416e-05
439 8.0352314398624e-05
440 8.02523863967508e-05
441 8.01523055997677e-05
442 8.00527632236481e-05
443 7.99521803855896e-05
444 7.98528199084103e-05
445 7.975291373441e-05
446 7.96532331150956e-05
447 7.95529049355537e-05
448 7.94531661085784e-05
449 7.93531216913834e-05
450 7.92532009654678e-05
451 7.9152814578265e-05
452 7.90533886174671e-05
453 7.89534533396363e-05
454 7.8853641753085e-05
455 7.87536118878052e-05
456 7.86545497248881e-05
457 7.85543597885408e-05
458 7.84551011747681e-05
459 7.83556970418431e-05
460 7.82562638050877e-05
461 7.81569397076964e-05
462 7.80581613071263e-05
463 7.79591136961244e-05
464 7.78603862272575e-05
465 7.77617169660516e-05
466 7.76633969508111e-05
467 7.75650041759945e-05
468 7.7467026130762e-05
469 7.7368960774038e-05
470 7.72718267398886e-05
471 7.71752529544756e-05
472 7.70778642618097e-05
473 7.69816833781078e-05
474 7.68853933550417e-05
475 7.67897799960338e-05
476 7.66949815442786e-05
477 7.66003722674213e-05
478 7.65063159633428e-05
479 7.64127617003396e-05
480 7.632023334736e-05
481 7.6227959652897e-05
482 7.61368864914402e-05
483 7.60461552999914e-05
484 7.59566901251674e-05
485 7.5867457780987e-05
486 7.57791858632118e-05
487 7.56910667405464e-05
488 7.56042281864211e-05
489 7.55184009904042e-05
490 7.54326101741754e-05
491 7.53471540519968e-05
492 7.52623382140882e-05
493 7.51781481085345e-05
494 7.50942344893701e-05
495 7.50112812966108e-05
496 7.49279861338437e-05
497 7.48459351598285e-05
498 7.47646190575324e-05
499 7.4683477578219e-05
500 7.46035220799968e-05
501 7.45242214179598e-05
502 7.44456483516842e-05
503 7.43684504413977e-05
504 7.42915945011191e-05
505 7.42156844353303e-05
506 7.41412441129796e-05
507 7.40675313863903e-05
508 7.3994422564283e-05
509 7.39226670702919e-05
510 7.38520902814344e-05
511 7.3781753599178e-05
512 7.37123846192844e-05
513 7.36443980713375e-05
514 7.35770954634063e-05
515 7.35104622435756e-05
516 7.34448185539804e-05
517 7.33803390176035e-05
518 7.33165215933695e-05
519 7.32538683223538e-05
520 7.31917680241168e-05
521 7.31304462533444e-05
522 7.30698739062063e-05
523 7.30100291548297e-05
524 7.29513194528408e-05
525 7.28931627236307e-05
526 7.28361919755116e-05
527 7.27792357793078e-05
528 7.27228834875859e-05
529 7.26683138054796e-05
530 7.26137004676275e-05
531 7.25601494195871e-05
532 7.25069767213427e-05
533 7.24547062418424e-05
534 7.24028213880956e-05
535 7.23518824088387e-05
536 7.23017728887498e-05
537 7.22519907867536e-05
538 7.22032127669081e-05
539 7.215474761324e-05
540 7.21066826372407e-05
541 7.20600728527643e-05
542 7.2013535827864e-05
543 7.19676245353185e-05
544 7.19223899068311e-05
545 7.1877526352182e-05
546 7.18331357347779e-05
547 7.17900329618715e-05
548 7.17464354238473e-05
549 7.17041839379817e-05
550 7.16619761078618e-05
551 7.16208814992569e-05
552 7.15797723387368e-05
553 7.15394853614271e-05
554 7.14992711436935e-05
555 7.14597263140604e-05
556 7.14206253178418e-05
557 7.13820772944018e-05
558 7.13434492354281e-05
559 7.1305941673927e-05
560 7.12681139702909e-05
561 7.12309265509248e-05
562 7.11938628228381e-05
563 7.1157373895403e-05
564 7.11205357220024e-05
565 7.10846143192612e-05
566 7.10481399437413e-05
567 7.10119420546107e-05
568 7.09757732693106e-05
569 7.0939859142527e-05
570 7.09035812178627e-05
571 7.08670922904275e-05
572 7.08304942236282e-05
573 7.07935978425667e-05
574 7.07561048329808e-05
575 7.07183353370056e-05
576 7.06797291059047e-05
577 7.06404680386186e-05
578 7.06010032445192e-05
579 7.05605343682691e-05
580 7.05197162460536e-05
581 7.04786943970248e-05
582 7.04381163814105e-05
583 7.03996483935043e-05
584 7.03652549418621e-05
585 7.03390833223239e-05
586 7.03266196069308e-05
587 7.03375990269706e-05
588 7.0385322032962e-05
589 7.04863123246469e-05
590 7.06556893419474e-05
591 7.08969600964338e-05
592 7.11852553649805e-05
593 7.14634443284012e-05
594 7.1658068918623e-05
595 7.17280636308715e-05
596 7.1688016760163e-05
597 7.15909773134626e-05
598 7.14842171873897e-05
599 7.13958725100383e-05
600 7.13322006049566e-05
601 7.12907567503862e-05
602 7.12657783878967e-05
603 7.12515684426762e-05
604 7.12436376488768e-05
605 7.12385881342925e-05
606 7.12360124452971e-05
607 7.12336041033268e-05
608 7.12317269062623e-05
609 7.12298642611131e-05
610 7.12280234438367e-05
611 7.12261607986875e-05
612 7.12243709131144e-05
613 7.12221808498725e-05
614 7.1220172685571e-05
615 7.12171895429492e-05
616 7.12146866135299e-05
617 7.12110850145109e-05
618 7.1207010478247e-05
619 7.12026667315513e-05
620 7.11976244929247e-05
621 7.11919346940704e-05
622 7.11859465809539e-05
623 7.11796164978296e-05
624 7.11724715074524e-05
625 7.11651227902621e-05
626 7.11573520675302e-05
627 7.11490938556381e-05
628 7.11410393705592e-05
629 7.11323955329135e-05
630 7.11243919795379e-05
631 7.11159736965783e-05
632 7.11074535502121e-05
633 7.10988533683121e-05
634 7.10908861947246e-05
635 7.10826570866629e-05
636 7.10744861862622e-05
637 7.10668246028945e-05
638 7.10589665686712e-05
639 7.10513777448796e-05
640 7.10437816451304e-05
641 7.10371750756167e-05
642 7.10304593667388e-05
643 7.10240565240383e-05
644 7.1018272137735e-05
645 7.10124004399404e-05
646 7.10064268787391e-05
647 7.10007298039272e-05
648 7.0995360147208e-05
649 7.09903106326237e-05
650 7.09858868503943e-05
651 7.0981310273055e-05
652 7.0976595452521e-05
653 7.09726518834941e-05
654 7.09682935848832e-05
655 7.09645610186271e-05
656 7.09608575562015e-05
657 7.09567902958952e-05
658 7.09535233909264e-05
659 7.09495870978571e-05
660 7.09459491190501e-05
661 7.09424857632257e-05
662 7.09389569237828e-05
663 7.09354717400856e-05
664 7.09324085619301e-05
665 7.09285741322674e-05
666 7.0925205363892e-05
667 7.0921603764873e-05
668 7.09181185811758e-05
669 7.09144733264111e-05
670 7.09103333065286e-05
671 7.09063533577137e-05
672 7.09026789991185e-05
673 7.08983207005076e-05
674 7.08940642653033e-05
675 7.08898369339295e-05
676 7.08850639057346e-05
677 7.08808584022336e-05
678 7.08753505023196e-05
679 7.08705047145486e-05
680 7.08654260961339e-05
681 7.08603256498463e-05
682 7.08544030203484e-05
683 7.08484731148928e-05
684 7.08425213815644e-05
685 7.08364314050414e-05
686 7.08300358382985e-05
687 7.08232109900564e-05
688 7.08162333467044e-05
689 7.08091392880306e-05
690 7.08016377757303e-05
691 7.07937360857613e-05
692 7.07858125679195e-05
693 7.07776416675188e-05
694 7.07688741385937e-05
695 7.07604631315917e-05
696 7.07511353539303e-05
697 7.07415383658372e-05
698 7.07319559296593e-05
699 7.07218641764484e-05
700 7.07115614204668e-05
701 7.07012659404427e-05
702 7.0690359279979e-05
703 7.06792852724902e-05
704 7.0667898398824e-05
705 7.0656300522387e-05
706 7.06445280229673e-05
707 7.06322316545993e-05
708 7.06196660757996e-05
709 7.06071878084913e-05
710 7.05943384673446e-05
711 7.05806014593691e-05
712 7.0566929935012e-05
713 7.05530183040537e-05
714 7.05394777469337e-05
715 7.05250931787305e-05
716 7.05104248481803e-05
717 7.04951380612329e-05
718 7.04798876540735e-05
719 7.04641861375421e-05
720 7.04483682056889e-05
721 7.04322737874463e-05
722 7.04155027051456e-05
723 7.03984915162437e-05
724 7.03810874256305e-05
725 7.03632540535182e-05
726 7.0345020503737e-05
727 7.03268087818287e-05
728 7.03078621882014e-05
729 7.02886100043543e-05
730 7.02693505445495e-05
731 7.02491233823821e-05
732 7.02286997693591e-05
733 7.02077450114302e-05
734 7.01861208654009e-05
735 7.0164205681067e-05
736 7.01419412507676e-05
737 7.01191092957743e-05
738 7.00960808899254e-05
739 7.00719174346887e-05
740 7.00476011843421e-05
741 7.00227756169625e-05
742 6.99974552844651e-05
743 6.9971451011952e-05
744 6.99451848049648e-05
745 6.99179799994454e-05
746 6.98902222211473e-05
747 6.98622534400783e-05
748 6.98332587489858e-05
749 6.98040894349106e-05
750 6.977426528465e-05
751 6.97438081260771e-05
752 6.9712485128548e-05
753 6.96808201610111e-05
754 6.96486386004835e-05
755 6.96155548212118e-05
756 6.95823255227879e-05
757 6.9548244937323e-05
758 6.95136477588676e-05
759 6.94778500474058e-05
760 6.94417176418938e-05
761 6.94052359904163e-05
762 6.93678230163641e-05
763 6.93298716214485e-05
764 6.92914181854576e-05
765 6.92521862220019e-05
766 6.92121830070391e-05
767 6.91720269969665e-05
768 6.91305976943113e-05
769 6.90885717631318e-05
770 6.90460583427921e-05
771 6.90031301928684e-05
772 6.89592779963277e-05
773 6.8914465373382e-05
774 6.88690852257423e-05
775 6.88232976244763e-05
776 6.87768260831945e-05
777 6.87290739733726e-05
778 6.8681147240568e-05
779 6.86323619447649e-05
780 6.85830455040559e-05
781 6.85328777763061e-05
782 6.84818005538546e-05
783 6.8430024839472e-05
784 6.8377899879124e-05
785 6.83245962136425e-05
786 6.82709141983651e-05
787 6.82164391037077e-05
788 6.81614910718054e-05
789 6.81052770232782e-05
790 6.80483790347353e-05
791 6.79910444887355e-05
792 6.79327204125002e-05
793 6.78737269481644e-05
794 6.78140204399824e-05
795 6.77534771966748e-05
796 6.76919662510045e-05
797 6.76302079227753e-05
798 6.75672490615398e-05
799 6.75031478749588e-05
800 6.74388793413527e-05
801 6.7373963247519e-05
802 6.73075846862048e-05
803 6.72407622914761e-05
804 6.71727539156564e-05
805 6.71042580506764e-05
806 6.7034678068012e-05
807 6.69646760798059e-05
808 6.68936627334915e-05
809 6.68218272039667e-05
810 6.67491622152738e-05
811 6.66758351144381e-05
812 6.66014166199602e-05
813 6.65265542920679e-05
814 6.64509498164989e-05
815 6.63747996441089e-05
816 6.62979000480846e-05
817 6.62204183754511e-05
818 6.61421800032258e-05
819 6.60635269014165e-05
820 6.59845245536417e-05
821 6.5904860093724e-05
822 6.58246499369852e-05
823 6.57442724332213e-05
824 6.56633637845516e-05
825 6.5582207753323e-05
826 6.55004914733581e-05
827 6.54185278108343e-05
828 6.53363895253278e-05
829 6.52538074064068e-05
830 6.51709051453508e-05
831 6.50876827421598e-05
832 6.50035726721399e-05
833 6.4919498981908e-05
834 6.48345376248471e-05
835 6.47491688141599e-05
836 6.46632179268636e-05
837 6.45766194793396e-05
838 6.44890606054105e-05
839 6.44008032395504e-05
840 6.43118692096323e-05
841 6.42219893052243e-05
842 6.41313890810125e-05
843 6.40395519440062e-05
844 6.39472418697551e-05
845 6.38533165329136e-05
846 6.37590856058523e-05
847 6.36628246866167e-05
848 6.35662145214155e-05
849 6.34688040008768e-05
850 6.33729359833524e-05
851 6.32712617516518e-05
852 6.31668517598882e-05
853 6.30675713182427e-05
854 6.296401261352e-05
855 6.28585039521568e-05
856 6.2755927501712e-05
857 6.2643812270835e-05
858 6.25344255240634e-05
859 6.24296517344192e-05
860 6.23137530055828e-05
861 6.22077932348475e-05
862 6.2088125559967e-05
863 6.19763959548436e-05
864 6.18561898590997e-05
865 6.17400510236621e-05
866 6.16191391600296e-05
867 6.14995660725981e-05
868 6.13771553616971e-05
869 6.12549920333549e-05
870 6.1130260291975e-05
871 6.10055758443195e-05
872 6.08787377132103e-05
873 6.07517395110335e-05
874 6.06227404205129e-05
875 6.04932138230652e-05
876 6.03619591856841e-05
877 6.02303334744647e-05
878 6.00968924118206e-05
879 5.99632112425752e-05
880 5.98276492382865e-05
881 5.96917707298417e-05
882 5.95545570831746e-05
883 5.94170014664996e-05
884 5.92779833823442e-05
885 5.91383759456221e-05
886 5.89981573284604e-05
887 5.88567891099956e-05
888 5.87152317166328e-05
889 5.85722555115353e-05
890 5.84296794841066e-05
891 5.82854554522783e-05
892 5.81415952183306e-05
893 5.799618884339e-05
894 5.78512845095247e-05
895 5.77050413994584e-05
896 5.75593221583404e-05
897 5.74121841054875e-05
898 5.72657554585021e-05
899 5.71180862607434e-05
900 5.69711992284283e-05
901 5.6823108025128e-05
902 5.6675879022805e-05
903 5.65274713153485e-05
904 5.63801040698308e-05
905 5.62315981369466e-05
906 5.60841326660011e-05
907 5.59357904421631e-05
908 5.57884377485607e-05
909 5.56402301299386e-05
910 5.54930666112341e-05
911 5.53452155145351e-05
912 5.51983284822199e-05
913 5.50509066670202e-05
914 5.49044452782255e-05
915 5.47575691598468e-05
916 5.46116716577671e-05
917 5.446537034004e-05
918 5.43198802915867e-05
919 5.41743575013243e-05
920 5.40295441169292e-05
921 5.38846870767884e-05
922 5.37405867362395e-05
923 5.35964572918601e-05
924 5.34530954610091e-05
925 5.33098536834586e-05
926 5.31672631041147e-05
927 5.30248544237111e-05
928 5.28829987160861e-05
929 5.27413940289989e-05
930 5.26004550920334e-05
931 5.24597344337963e-05
932 5.23196140420623e-05
933 5.21798101544846e-05
934 5.20405774295796e-05
935 5.19017448823433e-05
936 5.17633743584156e-05
937 5.16254622198176e-05
938 5.14881176059134e-05
939 5.13511513418052e-05
940 5.12147671543062e-05
941 5.10788522660732e-05
942 5.09435049025342e-05
943 5.08085577166639e-05
944 5.06742326251697e-05
945 5.05404641444329e-05
946 5.04072049807291e-05
947 5.02744987898041e-05
948 5.01424801768735e-05
949 5.001098907087e-05
950 4.98800691275392e-05
951 4.9749796744436e-05
952 4.96202083013486e-05
953 4.94911182613578e-05
954 4.93627594551072e-05
955 4.92350045533385e-05
956 4.91079299536068e-05
957 4.89815392938908e-05
958 4.8855861678021e-05
959 4.873098077951e-05
960 4.86067074234597e-05
961 4.84833944938146e-05
962 4.83609219372738e-05
963 4.82393479614984e-05
964 4.81188617413864e-05
965 4.79997033835389e-05
966 4.78818801639136e-05
967 4.77655776194297e-05
968 4.7651268687332e-05
969 4.75388442282565e-05
970 4.74286389362533e-05
971 4.73204345325939e-05
972 4.72140163765289e-05
973 4.71092025691178e-05
974 4.70052946184296e-05
975 4.69022779725492e-05
976 4.67994141217787e-05
977 4.66967430838849e-05
978 4.65940793219488e-05
979 4.64916010969318e-05
980 4.63893193227705e-05
981 4.6287623263197e-05
982 4.61862364318222e-05
983 4.60854680568445e-05
984 4.59857947134878e-05
985 4.5886557927588e-05
986 4.5788361603627e-05
987 4.56909510830883e-05
988 4.5594439143315e-05
989 4.54987784905825e-05
990 4.54040455224458e-05
991 4.53099310107064e-05
992 4.5216795115266e-05
993 4.51241139671765e-05
994 4.50325023848563e-05
995 4.49413128080778e-05
996 4.48507926193997e-05
997 4.47608836111613e-05
998 4.46714366262313e-05
999 4.45823425252456e-05
1000 4.4494001485873e-05
1001 4.4405762309907e-05
1002 4.43178505520336e-05
1003 4.423039717949e-05
1004 4.41429547208827e-05
1005 4.40557705587707e-05
1006 4.3968884710921e-05
1007 4.38820134149864e-05
1008 4.37949565821327e-05
1009 4.37082453572657e-05
1010 4.3621301301755e-05
1011 4.35342481068801e-05
1012 4.34470821346622e-05
1013 4.33599234384019e-05
1014 4.32721681136172e-05
1015 4.31841690442525e-05
1016 4.30957115895581e-05
1017 4.30068575951736e-05
1018 4.29173123848159e-05
1019 4.28277817263734e-05
1020 4.2737228795886e-05
1021 4.2645915527828e-05
1022 4.25540310970973e-05
1023 4.2461517296033e-05
1024 4.23677847720683e-05
1025 4.22732591687236e-05
1026 4.21778604504652e-05
1027 4.20808573835529e-05
1028 4.19832795159891e-05
1029 4.18840681959409e-05
1030 4.17841365560889e-05
1031 4.16826478613075e-05
1032 4.15795984736178e-05
1033 4.14754467783496e-05
1034 4.13692541769706e-05
1035 4.12613881053403e-05
1036 4.11506589443889e-05
1037 4.10370121244341e-05
1038 4.09149579354562e-05
1039 4.08006417274009e-05
1040 4.0720868128119e-05
1041 4.05887294618879e-05
1042 4.04592865379527e-05
1043 4.0332230128115e-05
1044 4.02061850763857e-05
1045 4.00793287553824e-05
1046 3.99505079258233e-05
1047 3.98197844333481e-05
1048 3.96865834773052e-05
1049 3.95511196984444e-05
1050 3.94134767702781e-05
1051 3.92740330426022e-05
1052 3.91329995181877e-05
1053 3.89908091165125e-05
1054 3.88474836654495e-05
1055 3.87041109206621e-05
1056 3.85610110242851e-05
1057 3.84188533644192e-05
1058 3.82789839932229e-05
1059 3.81429053959437e-05
1060 3.80136734747794e-05
1061 3.7894285924267e-05
1062 3.77866781491321e-05
1063 3.76899006369058e-05
1064 3.75989584426861e-05
1065 3.75067756976932e-05
1066 3.74064802599605e-05
1067 3.72941212845035e-05
1068 3.7168381822994e-05
1069 3.70315647160169e-05
1070 3.68854598491453e-05
1071 3.67332140740473e-05
1072 3.65763589798007e-05
1073 3.64172228728421e-05
1074 3.62564096576534e-05
1075 3.60947378794663e-05
1076 3.59324367309455e-05
1077 3.57693534169812e-05
1078 3.56059026671574e-05
1079 3.54416661139112e-05
1080 3.52766837750096e-05
1081 3.51110429619439e-05
1082 3.49451875081286e-05
1083 3.47786226484459e-05
1084 3.46116212313063e-05
1085 3.44450381817296e-05
1086 3.42786697729025e-05
1087 3.41128397849388e-05
1088 3.39477410307154e-05
1089 3.37837227561977e-05
1090 3.36205994244665e-05
1091 3.34592805302236e-05
1092 3.32992822222877e-05
1093 3.31410483340733e-05
1094 3.29850518028252e-05
1095 3.28315145452507e-05
1096 3.26804438373074e-05
1097 3.25327018799726e-05
1098 3.23883468809072e-05
1099 3.22475316352211e-05
1100 3.21111765515525e-05
1101 3.19790087814908e-05
1102 3.18525235343259e-05
1103 3.17314879794139e-05
1104 3.16172772727441e-05
1105 3.15097313432489e-05
1106 3.14097633236088e-05
1107 3.13179662043694e-05
1108 3.1234845664585e-05
1109 3.11608964693733e-05
1110 3.10969371639658e-05
1111 3.1041960028233e-05
1112 3.09955648845062e-05
1113 3.09556089632679e-05
1114 3.09190727421083e-05
1115 3.0879946280038e-05
1116 3.08321978081949e-05
1117 3.0765681003686e-05
1118 3.06712499877904e-05
1119 3.05420617223717e-05
1120 3.03747001453303e-05
1121 3.01735635730438e-05
1122 2.99486564472318e-05
1123 2.97146725642961e-05
1124 2.94848305202322e-05
1125 2.92672757495893e-05
1126 2.90680927719222e-05
1127 2.88881983578904e-05
1128 2.87255898001604e-05
1129 2.85786063614069e-05
1130 2.84439374809153e-05
1131 2.83190966001712e-05
1132 2.82028358924435e-05
1133 2.80924796243198e-05
1134 2.7986612622044e-05
1135 2.78852185147116e-05
1136 2.77869621641003e-05
1137 2.76913051493466e-05
1138 2.759745621006e-05
1139 2.75059428531677e-05
1140 2.74158373940736e-05
1141 2.73275927611394e-05
1142 2.72405686700949e-05
1143 2.71549015451455e-05
1144 2.70705786533654e-05
1145 2.69877182290656e-05
1146 2.69066676992225e-05
1147 2.68267431238201e-05
1148 2.67477753368439e-05
1149 2.66703664237866e-05
1150 2.6594718292472e-05
1151 2.65200305875624e-05
1152 2.64471709670033e-05
1153 2.63753336184891e-05
1154 2.63049914792646e-05
1155 2.62364992522635e-05
1156 2.61691257037455e-05
1157 2.61031473201001e-05
1158 2.60388060269179e-05
1159 2.59753469435964e-05
1160 2.59135449596215e-05
1161 2.58526360994438e-05
1162 2.5793133318075e-05
1163 2.5735180315678e-05
1164 2.56779712799471e-05
1165 2.56221101153642e-05
1166 2.55675349762896e-05
1167 2.55139020737261e-05
1168 2.54613732977305e-05
1169 2.54098740697373e-05
1170 2.53595262620365e-05
1171 2.53100151894614e-05
1172 2.52614736382384e-05
1173 2.52136596827768e-05
1174 2.51669825956924e-05
1175 2.51215369644342e-05
1176 2.5076582460315e-05
1177 2.5032481062226e-05
1178 2.49885579250986e-05
1179 2.49459899350768e-05
1180 2.49043769144919e-05
1181 2.48633623414207e-05
1182 2.48231572186342e-05
1183 2.47831212618621e-05
1184 2.47437637881376e-05
1185 2.47051521000685e-05
1186 2.46673062065383e-05
1187 2.46296385739697e-05
1188 2.45930150413187e-05
1189 2.4556649805163e-05
1190 2.45205737883225e-05
1191 2.44857783400221e-05
1192 2.44505481532542e-05
1193 2.44160946749616e-05
1194 2.43817139562452e-05
1195 2.43481281358982e-05
1196 2.43151262111496e-05
1197 2.42819151026197e-05
1198 2.42494115809677e-05
1199 2.42169699049555e-05
1200 2.41849538724637e-05
1201 2.41529305640142e-05
1202 2.41215675487183e-05
1203 2.40904173551826e-05
1204 2.40591198235052e-05
1205 2.40282879531151e-05
1206 2.39977489400189e-05
1207 2.39673609030433e-05
1208 2.3937051082612e-05
1209 2.39068231167039e-05
1210 2.38773664023029e-05
1211 2.38477678067284e-05
1212 2.38179927691817e-05
1213 2.37884887610562e-05
1214 2.37589847529307e-05
1215 2.37297444982687e-05
1216 2.37010681303218e-05
1217 2.36720570683246e-05
1218 2.36431478697341e-05
1219 2.36144151131157e-05
1220 2.35858042287873e-05
1221 2.35575480473926e-05
1222 2.35292318393476e-05
1223 2.35009065363556e-05
1224 2.34727885981556e-05
1225 2.34445105888881e-05
1226 2.3416847398039e-05
1227 2.33889113587793e-05
1228 2.33610826398944e-05
1229 2.33330829360057e-05
1230 2.33055980061181e-05
1231 2.32775928452611e-05
1232 2.32501752179815e-05
1233 2.32231286645401e-05
1234 2.31958038057201e-05
1235 2.31688190979185e-05
1236 2.3142123609432e-05
1237 2.31154081120621e-05
1238 2.30889163503889e-05
1239 2.30621844821144e-05
1240 2.30355581152253e-05
1241 2.30086552619468e-05
1242 2.2982363589108e-05
1243 2.29559736908413e-05
1244 2.29300349019468e-05
1245 2.29047345783329e-05
1246 2.2879028620082e-05
1247 2.28534863708774e-05
1248 2.28280114242807e-05
1249 2.28029093705118e-05
1250 2.07921584660653e-05
1251 2.07110042538261e-05
1252 2.06753175007179e-05
1253 2.06444947252749e-05
1254 2.06172517209779e-05
1255 2.05929900403135e-05
1256 2.05709839065094e-05
1257 2.0550689441734e-05
1258 2.05318247026298e-05
1259 2.05138549063122e-05
1260 2.04967000172473e-05
1261 2.04802563530393e-05
1262 2.04643401957583e-05
1263 2.04488169401884e-05
1264 2.04337720788317e-05
1265 2.04190328076947e-05
1266 2.04046536964597e-05
1267 2.0390627469169e-05
1268 2.03768213395961e-05
1269 2.03632607735926e-05
1270 2.0349893020466e-05
1271 2.03368072106969e-05
1272 2.03238541871542e-05
1273 2.03110757865943e-05
1274 2.02985174837522e-05
1275 2.02860701392638e-05
1276 2.02738428924931e-05
1277 2.02617084141821e-05
1278 2.02498558792286e-05
1279 2.02380942937452e-05
1280 2.02265709958738e-05
1281 2.0215204131091e-05
1282 2.02041046577506e-05
1283 2.01932652998948e-05
1284 2.01825532712974e-05
1285 2.01721868506866e-05
1286 2.01620623556664e-05
1287 2.01521725102793e-05
1288 2.01425682462286e-05
1289 2.01333768927725e-05
1290 2.01244074560236e-05
1291 2.01158145500813e-05
1292 2.01075545191998e-05
1293 2.0099676476093e-05
1294 2.00919785129372e-05
1295 2.00846443476621e-05
1296 2.00776266865432e-05
1297 2.00709382625064e-05
1298 2.00644371943781e-05
1299 2.00581853277981e-05
1300 2.00521735678194e-05
1301 2.00463291548658e-05
1302 2.00406902877148e-05
1303 2.00352133106207e-05
1304 2.002990913752e-05
1305 2.00246868189424e-05
1306 2.00196700461674e-05
1307 2.00146751012653e-05
1308 2.0009787476738e-05
1309 2.00050217245007e-05
1310 2.00002577912528e-05
1311 1.99955557036446e-05
1312 1.99908972717822e-05
1313 1.99863425223157e-05
1314 1.9981782315881e-05
1315 1.99772421183297e-05
1316 1.997275830945e-05
1317 1.99683145183371e-05
1318 1.99638216145104e-05
1319 1.99594142031856e-05
1320 1.99549704120727e-05
1321 1.99505830096314e-05
1322 1.99461828742642e-05
1323 1.99417791009182e-05
1324 1.99374408111908e-05
1325 1.99330552277388e-05
1326 1.99287260329584e-05
1327 1.99244059331249e-05
1328 1.99201549548889e-05
1329 1.99157839233521e-05
1330 1.99114765564445e-05
1331 1.99072655959753e-05
1332 1.99029327632161e-05
1333 1.98987563635455e-05
1334 1.98944453586591e-05
1335 1.98902343981899e-05
1336 1.98860143427737e-05
1337 1.98818052012939e-05
1338 1.98776233446551e-05
1339 1.9873439669027e-05
1340 1.98692468984518e-05
1341 1.98651468963362e-05
1342 1.98610341612948e-05
1343 1.98569305212004e-05
1344 1.98527795873815e-05
1345 1.98487123270752e-05
1346 1.98446396098007e-05
1347 1.98405959963566e-05
1348 1.9836523279082e-05
1349 1.98325087694684e-05
1350 1.98284342332045e-05
1351 1.98244724742835e-05
1352 1.98205161723308e-05
1353 1.98165307665477e-05
1354 1.98125726456055e-05
1355 1.98086381715257e-05
1356 1.98046927835094e-05
1357 1.98008074221434e-05
1358 1.97969602595549e-05
1359 1.97930530703161e-05
1360 1.97891731659183e-05
1361 1.9785331460298e-05
1362 1.97814770217519e-05
1363 1.97776589629939e-05
1364 1.97738645510981e-05
1365 1.97700574062765e-05
1366 1.97663030121475e-05
1367 1.97625540749868e-05
1368 1.97587942238897e-05
1369 1.97550580196548e-05
1370 1.97513654711656e-05
1371 1.97476310859201e-05
1372 1.97439148905687e-05
1373 1.97402496269206e-05
1374 1.97365880012512e-05
1375 1.97329209186137e-05
1376 1.97292974917218e-05
1377 1.97256122191902e-05
1378 1.97220433619805e-05
1379 1.97184508579085e-05
1380 1.97148856386775e-05
1381 1.97113076865207e-05
1382 1.97077242773958e-05
1383 1.97042136278469e-05
1384 1.97006465896266e-05
1385 1.96971650439082e-05
1386 1.96936325664865e-05
1387 1.96901564777363e-05
1388 1.96866585611133e-05
1389 1.96831661014585e-05
1390 1.96796809177613e-05
1391 1.9676244846778e-05
1392 1.96728069568053e-05
1393 1.96693763427902e-05
1394 1.96659057110082e-05
1395 1.96625078388024e-05
1396 1.96590863197343e-05
1397 1.96556775335921e-05
1398 1.96523124031955e-05
1399 1.96489581867354e-05
1400 1.96455566765508e-05
1401 1.96422315639211e-05
1402 1.96388918993762e-05
1403 1.9635614080471e-05
1404 1.96323107957141e-05
1405 1.96290129679255e-05
1406 1.96256696654018e-05
1407 1.96224100363906e-05
1408 1.96191376744537e-05
1409 1.96158616745379e-05
1410 1.96125802176539e-05
1411 1.96093369595474e-05
1412 1.96060918824514e-05
1413 1.96028795471648e-05
1414 1.95996544789523e-05
1415 1.95964657905279e-05
1416 1.95932188944425e-05
1417 1.95900447579334e-05
1418 1.95868287846679e-05
1419 1.95836673810845e-05
1420 1.95804714167025e-05
1421 1.95773045561509e-05
1422 1.95741649804404e-05
1423 1.95710199477617e-05
1424 1.95678912859876e-05
1425 1.95648171938956e-05
1426 1.95616630662698e-05
1427 1.95586089830613e-05
1428 1.95555130630964e-05
1429 1.95524517039303e-05
1430 1.95493903447641e-05
1431 1.9546338080545e-05
1432 1.95432621694636e-05
1433 1.95402499230113e-05
1434 1.95371867448557e-05
1435 1.95341399376048e-05
1436 1.95310967683326e-05
1437 1.95281245396473e-05
1438 1.95251195691526e-05
1439 1.95220909517957e-05
1440 1.95191114471527e-05
1441 1.95161301235203e-05
1442 1.95131251530256e-05
1443 1.95101802091813e-05
1444 1.95071897906018e-05
1445 1.95042575796833e-05
1446 1.95013071788708e-05
1447 1.94983676919946e-05
1448 1.9495448214002e-05
1449 1.94924996321788e-05
1450 1.94896219909424e-05
1451 1.94867498066742e-05
1452 1.94837848539464e-05
1453 1.9480889022816e-05
1454 1.94779695448233e-05
1455 1.94751210074173e-05
1456 1.94722451851703e-05
1457 1.94693311641458e-05
1458 1.94664826267399e-05
1459 1.94636395463021e-05
1460 1.94608019228326e-05
1461 1.94579424714902e-05
1462 1.94551066670101e-05
1463 1.94522672245512e-05
1464 1.94495005416684e-05
1465 1.94467029359657e-05
1466 1.94438889593584e-05
1467 1.94410731637618e-05
1468 1.94382791960379e-05
1469 1.94355143321445e-05
1470 1.94327312783571e-05
1471 1.94299791473895e-05
1472 1.94272324733902e-05
1473 1.94244621525286e-05
1474 1.94217391253915e-05
1475 1.94189597095829e-05
1476 1.94162093976047e-05
1477 1.94134718185524e-05
1478 1.94107851712033e-05
1479 1.94080621440662e-05
1480 1.94053482118761e-05
1481 1.94026397366542e-05
1482 1.9399947632337e-05
1483 1.93971991393482e-05
1484 1.93945616047131e-05
1485 1.93918513105018e-05
1486 1.93891573871952e-05
1487 1.9386501662666e-05
1488 1.93838586710626e-05
1489 1.93812047655229e-05
1490 1.93785581359407e-05
1491 1.93758914974751e-05
1492 1.93732394109247e-05
1493 1.93706273421412e-05
1494 1.9367953427718e-05
1495 1.9365379557712e-05
1496 1.93627784028649e-05
1497 1.9360146325198e-05
1498 1.93575360754039e-05
1499 1.93549440155039e-05
1500 1.93523374036886e-05
1501 1.9349783542566e-05
1502 1.93471532838885e-05
1503 1.93446121556917e-05
1504 1.93420364666963e-05
1505 1.93394589587115e-05
1506 1.93369178305147e-05
1507 1.93343330465723e-05
1508 1.93317955563543e-05
1509 1.93292089534225e-05
1510 1.93266987480456e-05
1511 1.93241830857005e-05
1512 1.93216419575037e-05
1513 1.93191444850527e-05
1514 1.93166615645168e-05
1515 1.93141622730764e-05
1516 1.93116302398266e-05
1517 1.93091491382802e-05
1518 1.93066443898715e-05
1519 1.93041887541767e-05
1520 1.930164762598e-05
1521 1.92991974472534e-05
1522 1.9296761820442e-05
1523 1.92942497960757e-05
1524 1.92918159882538e-05
1525 1.9289351257612e-05
1526 1.92869047168642e-05
1527 1.92844345292542e-05
1528 1.92819952644641e-05
1529 1.92795887414832e-05
1530 1.9277167666587e-05
1531 1.92747629625956e-05
1532 1.92722654901445e-05
1533 1.92699426406762e-05
1534 1.9267512470833e-05
1535 1.92650968529051e-05
1536 1.92626721400302e-05
1537 1.92602874449221e-05
1538 1.92578991118353e-05
1539 1.92555271496531e-05
1540 1.9253160644439e-05
1541 1.92507650353946e-05
1542 1.9248409444117e-05
1543 1.92460011021467e-05
1544 1.9243618226028e-05
1545 1.92412862816127e-05
1546 1.92389434232609e-05
1547 1.92365787370363e-05
1548 1.9234264982515e-05
1549 1.92318912013434e-05
1550 1.92295719898539e-05
1551 1.92272073036293e-05
1552 1.92248990060762e-05
1553 1.92225688806502e-05
1554 1.922028241097e-05
1555 1.92179741134169e-05
1556 1.92156694538426e-05
1557 1.92133829841623e-05
1558 1.92110856005456e-05
1559 1.92088009498548e-05
1560 1.92065399460262e-05
1561 1.92042498383671e-05
1562 1.92019797395915e-05
1563 1.91996477951761e-05
1564 1.91974286281038e-05
1565 1.91951548913494e-05
1566 1.91929229913512e-05
1567 1.91906528925756e-05
1568 1.91884246305563e-05
1569 1.91861672647065e-05
1570 1.91839353647083e-05
1571 1.91816689039115e-05
1572 1.91795024875319e-05
1573 1.91772560356185e-05
1574 1.91750132216839e-05
1575 1.91728431673255e-05
1576 1.91706258192426e-05
1577 1.91684084711596e-05
1578 1.91661656572251e-05
1579 1.91639828699408e-05
1580 1.91617637028685e-05
1581 1.91596100194147e-05
1582 1.91574454220245e-05
1583 1.9155248082825e-05
1584 1.91530871234136e-05
1585 1.91509134310763e-05
1586 1.91487361007603e-05
1587 1.9146638805978e-05
1588 1.91444960364606e-05
1589 1.91423132491764e-05
1590 1.91401577467332e-05
1591 1.91380586329615e-05
1592 1.91359067684971e-05
1593 1.91337821888737e-05
1594 1.91316212294623e-05
1595 1.91295221156906e-05
1596 1.91274375538342e-05
1597 1.91252638614969e-05
1598 1.91231738426723e-05
1599 1.9121085642837e-05
1600 1.91190047189593e-05
1601 1.91168692253996e-05
1602 1.91148283192888e-05
1603 1.91127037396654e-05
1604 1.91106082638726e-05
1605 1.91086019185605e-05
1606 1.91064682439901e-05
1607 1.91043618542608e-05
1608 1.91022991202772e-05
1609 1.91002854990074e-05
1610 1.90981627383735e-05
1611 1.9096136384178e-05
1612 1.9094090021099e-05
1613 1.90919799933909e-05
1614 1.90899663721211e-05
1615 1.90879127330845e-05
1616 1.90858627320267e-05
1617 1.90838454727782e-05
1618 1.90818136616144e-05
1619 1.90797727555037e-05
1620 1.90777573152445e-05
1621 1.90757837117417e-05
1622 1.90737409866415e-05
1623 1.90717491932446e-05
1624 1.90697282960173e-05
1625 1.90677783393767e-05
1626 1.90657938219374e-05
1627 1.90637947525829e-05
1628 1.90617865882814e-05
1629 1.9059832993662e-05
1630 1.90578539331909e-05
1631 1.90558894246351e-05
1632 1.90539522009203e-05
1633 1.90519767784281e-05
1634 1.90500122698722e-05
1635 1.90480441233376e-05
1636 1.90460723388242e-05
1637 1.90441460290458e-05
1638 1.90421742445324e-05
1639 1.90401951840613e-05
1640 1.90382488653995e-05
1641 1.90363152796635e-05
1642 1.90343616850441e-05
1643 1.9032449927181e-05
1644 1.90305472642649e-05
1645 1.90285809367197e-05
1646 1.90267001016764e-05
1647 1.90247919817921e-05
1648 1.90229056897806e-05
1649 1.90209830179811e-05
1650 1.90190858120332e-05
1651 1.90171867870959e-05
1652 1.9015316865989e-05
1653 1.90133687283378e-05
1654 1.90115097211674e-05
1655 1.90096216101665e-05
1656 1.90077789739007e-05
1657 1.90058672160376e-05
1658 1.90040245797718e-05
1659 1.90021364687709e-05
1660 1.90003138413886e-05
1661 1.89984402823029e-05
1662 1.8996561266249e-05
1663 1.89947186299833e-05
1664 1.89928723557387e-05
1665 1.89910333574517e-05
1666 1.89892016351223e-05
1667 1.8987320800079e-05
1668 1.89855381904636e-05
1669 1.89836828212719e-05
1670 1.8981843822985e-05
1671 1.8980023014592e-05
1672 1.89782258530613e-05
1673 1.89763668458909e-05
1674 1.89745715033496e-05
1675 1.89727979886811e-05
1676 1.89709498954471e-05
1677 1.89691709238105e-05
1678 1.89673501154175e-05
1679 1.8965549315908e-05
1680 1.89637703442713e-05
1681 1.89619986485923e-05
1682 1.89602014870616e-05
1683 1.89584188774461e-05
1684 1.89566253538942e-05
1685 1.89548682101304e-05
1686 1.89530692296103e-05
1687 1.8951310266857e-05
1688 1.89495676750084e-05
1689 1.8947812350234e-05
1690 1.8946067939396e-05
1691 1.89443398994626e-05
1692 1.89425736607518e-05
1693 1.89408219739562e-05
1694 1.89390666491818e-05
1695 1.89373295143014e-05
1696 1.89356214832515e-05
1697 1.89339025382651e-05
1698 1.89321599464165e-05
1699 1.89304464583984e-05
1700 1.89286947716027e-05
1701 1.89269940165104e-05
1702 1.89253132703016e-05
1703 1.8923616153188e-05
1704 1.89218790183077e-05
1705 1.89201800822048e-05
1706 1.89184956980171e-05
1707 1.8916818589787e-05
1708 1.89151560334722e-05
1709 1.891342435556e-05
1710 1.89117326954147e-05
1711 1.89100701390998e-05
1712 1.89083966688486e-05
1713 1.89067122846609e-05
1714 1.8905040633399e-05
1715 1.89033617061796e-05
1716 1.89016882359283e-05
1717 1.89000493264757e-05
1718 1.8898364942288e-05
1719 1.88967205758672e-05
1720 1.8895083485404e-05
1721 1.88934372999938e-05
1722 1.88917711057002e-05
1723 1.88901121873641e-05
1724 1.88884441740811e-05
1725 1.88867961696815e-05
1726 1.88851954590064e-05
1727 1.8883580196416e-05
1728 1.88819358299952e-05
1729 1.88803114724578e-05
1730 1.88786852959311e-05
1731 1.88770809472771e-05
1732 1.88754474947928e-05
1733 1.88738413271494e-05
1734 1.88722242455697e-05
1735 1.88706108019687e-05
1736 1.88689627975691e-05
1737 1.88674130185973e-05
1738 1.88658032129752e-05
1739 1.88642297871411e-05
1740 1.88626254384872e-05
1741 1.88610629265895e-05
1742 1.88594749488402e-05
1743 1.88578414963558e-05
1744 1.88562662515324e-05
1745 1.88546891877195e-05
1746 1.88531521416735e-05
1747 1.88515623449348e-05
1748 1.88499961950583e-05
1749 1.88484445970971e-05
1750 1.88468402484432e-05
1751 1.8845306840376e-05
1752 1.88437425094889e-05
1753 1.88421727216337e-05
1754 1.88406429515453e-05
1755 1.88391095434781e-05
1756 1.8837594325305e-05
1757 1.88360772881424e-05
1758 1.8834520233213e-05
1759 1.88329649972729e-05
1760 1.88314279512269e-05
1761 1.88298909051809e-05
1762 1.88283920579124e-05
1763 1.88268058991525e-05
1764 1.88252906809794e-05
1765 1.88237936527003e-05
1766 1.88222384167602e-05
1767 1.88207468454493e-05
1768 1.88192207133397e-05
1769 1.88176836672937e-05
1770 1.88161739060888e-05
1771 1.88147041626507e-05
1772 1.88132053153822e-05
1773 1.88116882782197e-05
1774 1.88101930689299e-05
1775 1.88086905836826e-05
1776 1.88072044693399e-05
1777 1.88057092600502e-05
1778 1.88042104127817e-05
1779 1.88027188414708e-05
1780 1.88012345461175e-05
1781 1.87997575267218e-05
1782 1.87983441719553e-05
1783 1.87968507816549e-05
1784 1.87953992281109e-05
1785 1.87939076568e-05
1786 1.87924943020334e-05
1787 1.87910245585954e-05
1788 1.87895493581891e-05
1789 1.87880868907087e-05
1790 1.87866626220057e-05
1791 1.87852219823981e-05
1792 1.87837904377375e-05
1793 1.87823497981299e-05
1794 1.87808691407554e-05
1795 1.87794230441796e-05
1796 1.87780260603176e-05
1797 1.87766090675723e-05
1798 1.87751684279647e-05
1799 1.87737714441027e-05
1800 1.87723326234845e-05
1801 1.87709301826544e-05
1802 1.87695241038455e-05
1803 1.87681052921107e-05
1804 1.87667101272382e-05
1805 1.87652967724716e-05
1806 1.87638997886097e-05
1807 1.87625028047478e-05
1808 1.87611021829071e-05
1809 1.87596924661193e-05
1810 1.87583136721514e-05
1811 1.87569476111094e-05
1812 1.87555542652262e-05
1813 1.87541772902478e-05
1814 1.87527784873964e-05
1815 1.8751419702312e-05
1816 1.87500208994607e-05
1817 1.87486803042702e-05
1818 1.87473106052494e-05
1819 1.87459045264404e-05
1820 1.87445657502394e-05
1821 1.87431869562715e-05
1822 1.87418499990599e-05
1823 1.87404875759967e-05
1824 1.87391287909122e-05
1825 1.87377518159337e-05
1826 1.8736436686595e-05
1827 1.87350469786907e-05
1828 1.87337118404685e-05
1829 1.87323694262886e-05
1830 1.8730996089289e-05
1831 1.87296627700562e-05
1832 1.87283039849717e-05
1833 1.87269542948343e-05
1834 1.87256409844849e-05
1835 1.87242767424323e-05
1836 1.87229688890511e-05
1837 1.87216119229561e-05
1838 1.87202931556385e-05
1839 1.87189489224693e-05
1840 1.87176228791941e-05
1841 1.87163059308659e-05
1842 1.87149726116331e-05
1843 1.87136465683579e-05
1844 1.8712353266892e-05
1845 1.87109981197864e-05
1846 1.87096811714582e-05
1847 1.87083769560559e-05
1848 1.87070854735794e-05
1849 1.87057612492936e-05
1850 1.87044661288382e-05
1851 1.87031691893935e-05
1852 1.87018758879276e-05
1853 1.87005916814087e-05
1854 1.86992801900487e-05
1855 1.86980050784769e-05
1856 1.86967099580215e-05
1857 1.86954457603861e-05
1858 1.86941306310473e-05
1859 1.86928682524012e-05
1860 1.86915876838611e-05
1861 1.86903052963316e-05
1862 1.86890338227386e-05
1863 1.86877114174422e-05
1864 1.86864817806054e-05
1865 1.86852012120653e-05
1866 1.86839279194828e-05
1867 1.86826546269003e-05
1868 1.86814086191589e-05
1869 1.86801626114175e-05
1870 1.8678871128941e-05
1871 1.86776323971571e-05
1872 1.86763991223415e-05
1873 1.86751331057167e-05
1874 1.86738689080812e-05
1875 1.86726138053928e-05
1876 1.86713896255242e-05
1877 1.86701217899099e-05
1878 1.86689067049883e-05
1879 1.86676443263423e-05
1880 1.86664292414207e-05
1881 1.86651759577217e-05
1882 1.86639445018955e-05
1883 1.86627239600057e-05
1884 1.86614906851901e-05
1885 1.86602737812791e-05
1886 1.86590641533257e-05
1887 1.86578181455843e-05
1888 1.86565957847051e-05
1889 1.86554170795716e-05
1890 1.86541983566713e-05
1891 1.86529723578133e-05
1892 1.86517845577328e-05
1893 1.8650591300684e-05
1894 1.86493834917201e-05
1895 1.86481938726502e-05
1896 1.86469878826756e-05
1897 1.86458146345103e-05
1898 1.86445977305993e-05
1899 1.86434081115294e-05
1900 1.86422130354913e-05
1901 1.86410343303578e-05
1902 1.86398556252243e-05
1903 1.86386532732286e-05
1904 1.8637447283254e-05
1905 1.86362940439722e-05
1906 1.86351117008599e-05
1907 1.86339184438111e-05
1908 1.86327761184657e-05
1909 1.86315955943428e-05
1910 1.86304059752729e-05
1911 1.86292072612559e-05
1912 1.86280467460165e-05
1913 1.86268644029042e-05
1914 1.86257038876647e-05
1915 1.86245069926372e-05
1916 1.86234046850586e-05
1917 1.86222368938616e-05
1918 1.86210600077175e-05
1919 1.86199031304568e-05
1920 1.86187317012809e-05
1921 1.86175930139143e-05
1922 1.86164215847384e-05
1923 1.86152719834354e-05
1924 1.86140987352701e-05
1925 1.86129745998187e-05
1926 1.86118322744733e-05
1927 1.86106626642868e-05
1928 1.860955489974e-05
1929 1.86083761946065e-05
1930 1.86072466021869e-05
1931 1.86061224667355e-05
1932 1.86049619514961e-05
1933 1.86038450920023e-05
1934 1.86026936717099e-05
1935 1.86015840881737e-05
1936 1.86004490387859e-05
1937 1.85993103514193e-05
1938 1.85981680260738e-05
1939 1.85970438906224e-05
1940 1.85959543159697e-05
1941 1.85948301805183e-05
1942 1.85937224159716e-05
1943 1.85926182894036e-05
1944 1.8591497791931e-05
1945 1.85904227691935e-05
1946 1.85893222806044e-05
1947 1.85882017831318e-05
1948 1.85871012945427e-05
1949 1.85860462806886e-05
1950 1.85849039553432e-05
1951 1.8583803466754e-05
1952 1.85827011591755e-05
1953 1.85816370503744e-05
1954 1.85805001819972e-05
1955 1.85794069693657e-05
1956 1.8578362869448e-05
1957 1.85772387339966e-05
1958 1.8576185539132e-05
1959 1.85750559467124e-05
1960 1.85739445441868e-05
1961 1.85728604265023e-05
1962 1.85718417924363e-05
1963 1.85707158379955e-05
1964 1.85696244443534e-05
1965 1.85685512406053e-05
1966 1.85674944077618e-05
1967 1.85664102900773e-05
1968 1.85653516382445e-05
1969 1.85642693395494e-05
1970 1.85631761269178e-05
1971 1.85621120181167e-05
1972 1.85610297194216e-05
1973 1.85599183168961e-05
1974 1.85588833119255e-05
1975 1.85577664524317e-05
1976 1.85567296284717e-05
1977 1.85556436917977e-05
1978 1.85545759450179e-05
1979 1.8553535483079e-05
1980 1.85524622793309e-05
1981 1.85514090844663e-05
1982 1.85503522516228e-05
1983 1.85492826858535e-05
1984 1.85482422239147e-05
1985 1.85471835720818e-05
1986 1.85461212822702e-05
1987 1.85450753633631e-05
1988 1.85440549103078e-05
1989 1.85429507837398e-05
1990 1.85419357876526e-05
1991 1.85408953257138e-05
1992 1.8539854863775e-05
1993 1.85388089448679e-05
1994 1.85377648449503e-05
1995 1.85367753147148e-05
1996 1.85357093869243e-05
1997 1.85346616490278e-05
1998 1.85336593858665e-05
1999 1.85326007340336e-05
};
\addlegendentry{Test}
\end{groupplot}

\end{tikzpicture}

		\caption{Learning rate adjustment with augmented data. Shown are training- and validation error over 2000 epochs for the model with two and four convolutional layers in the encoder.}
		\label{Fig:DatAugSched}
	\end{figure}
\end{center}
\begin{center}
	\begin{figure}[H]
		% This file was created by tikzplotlib v0.9.6.
\begin{tikzpicture}

\begin{groupplot}[
group style={group size=1 by 8},
legend cell align={left},
legend style={fill opacity=1, draw opacity=1, text opacity=1, draw=white},
log basis y={10},
tick align=outside,
tick pos=left,
title style={at={(0.43,0.85)},anchor=north},
x grid style={white!69.0196078431373!black},
xlabel={Epoch},
x label style={yshift=13pt},
xmin=-49.95, xmax=2048.95,
xtick style={color=black},
xtick = {0,500,1500,2000},
y grid style={white!69.0196078431373!black},
ylabel={MSE Loss},
ymode=log,
ytick style={color=black},
width=.45\textwidth,
height=.25\textwidth
]
\nextgroupplot[
title={ELU/ELU},
ymin=3.13353850161057e-06, ymax=0.001,
]
\addplot [semithick, black, dashed]
table {%
0 0.0118166279280558
1 0.0114290448254906
2 0.0110621910716873
3 0.0107095161220059
4 0.0103645401832182
5 0.0100190959055908
6 0.00966285800677724
7 0.00928384298458695
8 0.00886941829230636
9 0.00840857235016301
10 0.00789793932926841
11 0.00734805522370152
12 0.00678386334038805
13 0.00623681776050944
14 0.0057315058511449
15 0.00527866560150869
16 0.00487722432444571
17 0.00452059344388545
18 0.00420113107247744
19 0.00391216127900407
20 0.0036485228483798
21 0.0034061839687638
22 0.00318202394191758
23 0.00297358404350234
24 0.00277895643012016
25 0.00259670491504949
26 0.00242572790011764
27 0.0022652232528344
28 0.00211461570870597
29 0.00197347656830971
30 0.00184145648381673
31 0.00171824663993903
32 0.0016035293610912
33 0.00149698481618543
34 0.00139828670580755
35 0.00130714467923099
36 0.00122330232079548
37 0.00114636933540169
38 0.00107591940286511
39 0.00101150787850202
40 0.000952673482970567
41 0.000898981074897165
42 0.000850061611345154
43 0.000805528332875838
44 0.000765008301641501
45 0.000728148535927176
46 0.00069459187807297
47 0.000664001295717753
48 0.000636071548342443
49 0.000610530787071184
50 0.000587137675211125
51 0.00056567741512481
52 0.000545956439054862
53 0.000527800186773675
54 0.000511051562170906
55 0.000495569718623301
56 0.000481247085190262
57 0.00046798299126749
58 0.000455670789506257
59 0.000444212335651173
60 0.000433523560559479
61 0.000423530164880503
62 0.000414170318435936
63 0.000405410074108659
64 0.000397220760078199
65 0.000389552103342794
66 0.000382351775442658
67 0.000375575560383368
68 0.000369185076124268
69 0.000363145767096285
70 0.000357426793243576
71 0.000352000691236753
72 0.000346842715771345
73 0.000341929639489535
74 0.000337241289571466
75 0.000332759521143089
76 0.000328467825738699
77 0.00032435109983453
78 0.000320395738071966
79 0.000316589835392733
80 0.000312922629291279
81 0.000309383872490798
82 0.000305965114193896
83 0.000302657866882328
84 0.00029945419669275
85 0.00029634714792337
86 0.000293330875024367
87 0.000290400107815003
88 0.000287549293943812
89 0.000284773023054186
90 0.000282066599311293
91 0.000279425674420963
92 0.00027684639076142
93 0.000274324931297087
94 0.000271857949996956
95 0.000269442215881099
96 0.00026707487108979
97 0.000264753174860743
98 0.000262474594592277
99 0.000260236796520985
100 0.000258037525100008
101 0.000255874731806216
102 0.000253746481689632
103 0.000251651363896599
104 0.000249588198471429
105 0.000247555030227886
106 0.000245550272211403
107 0.000243572953422699
108 0.000241622434486999
109 0.000239697347410583
110 0.000237796219948905
111 0.000235917828490528
112 0.000234061012406528
113 0.000232224698720529
114 0.00023040784958539
115 0.000228609559371762
116 0.000226828923132416
117 0.000225065052063655
118 0.000223317089421471
119 0.00022158431158914
120 0.000219865969029343
121 0.000218161410657558
122 0.000216469952647458
123 0.000214791022813188
124 0.000213124057097502
125 0.000211468564145889
126 0.000209824097282763
127 0.000208190161345101
128 0.000206566355132054
129 0.00020495240710261
130 0.000203347994954584
131 0.000201752841292091
132 0.000200166641604937
133 0.000198589195179011
134 0.000197020282087124
135 0.000195459987480717
136 0.000193908461994852
137 0.000192365998486821
138 0.000190832758562465
139 0.000189308633025576
140 0.000187793453278573
141 0.000186287162989629
142 0.000184789834975163
143 0.00018330151237933
144 0.000181822326169367
145 0.000180352416748519
146 0.000178891940436188
147 0.000177441095104314
148 0.000176000095507334
149 0.000174569155376503
150 0.000173148512089938
151 0.000171738423603074
152 0.000170339260080254
153 0.000168951293915143
154 0.00016757493148134
155 0.00016621048274601
156 0.000164858354651187
157 0.000163518879162439
158 0.000162192391144345
159 0.000160879201359876
160 0.000159579643707275
161 0.000158294143318471
162 0.000157023176200255
163 0.000155767154467412
164 0.000154526278436151
165 0.000153300813906299
166 0.000152090945960026
167 0.000150896872582962
168 0.000149718661248244
169 0.000148556411375012
170 0.000147410110002966
171 0.000146279737919031
172 0.000145165208550679
173 0.000144066464940806
174 0.000142983379248562
175 0.000141915822609917
176 0.000140863647743572
177 0.00013982669941015
178 0.000138804784455715
179 0.000137797727006728
180 0.000136805287297648
181 0.000135827216126927
182 0.00013486332153434
183 0.000133913353437265
184 0.000132977058711958
185 0.00013205413060291
186 0.000131144283727735
187 0.000130247244754855
188 0.000129362707014025
189 0.000128490312675922
190 0.00012762975586611
191 0.000126780671081406
192 0.000125942751679986
193 0.000125115644920015
194 0.000124299053567256
195 0.000123492650459411
196 0.000122696099907671
197 0.000121909123890873
198 0.000121131381803252
199 0.000120362630127602
200 0.000119602580582523
201 0.000118850961740691
202 0.000118107461361205
203 0.000117371824870816
204 0.000116643796815197
205 0.000115923084052838
206 0.000115209414445872
207 0.000114502512616355
208 0.000113802118875128
209 0.000113107958384262
210 0.000112419756675308
211 0.000111737303654991
212 0.000111060310132416
213 0.000110388491009417
214 0.000109721640427551
215 0.000109059530529976
216 0.000108402036232746
217 0.000107748901797322
218 0.000107099926765386
219 0.000106454871627193
220 0.000105813521599885
221 0.000105175692567627
222 0.000104541145191206
223 0.000103909697372728
224 0.000103281157691981
225 0.000102655302526955
226 0.000102031973455041
227 0.000101410982750849
228 0.000100792155706131
229 0.000100175310365103
230 9.95603144815504e-05
231 9.89470851493479e-05
232 9.83354742345455e-05
233 9.7725368874535e-05
234 9.71166781198463e-05
235 9.65092472711149e-05
236 9.59029181615279e-05
237 9.52975672134926e-05
238 9.46930799443635e-05
239 9.40893226584194e-05
240 9.34861744497084e-05
241 9.28837663991544e-05
242 9.22820306072936e-05
243 9.16808639033206e-05
244 9.10801466602607e-05
245 9.04798255021433e-05
246 8.98797814556929e-05
247 8.92799467209215e-05
248 8.86802780826201e-05
249 8.80806932315181e-05
250 8.74811196212022e-05
251 8.6881532638472e-05
252 8.62818701818924e-05
253 8.56820824566285e-05
254 8.50821612772279e-05
255 8.44820471996854e-05
256 8.38817638566525e-05
257 8.32813176714353e-05
258 8.26807042528799e-05
259 8.2079927040013e-05
260 8.14789760141821e-05
261 8.08778740548632e-05
262 8.02766437004721e-05
263 7.96753223539781e-05
264 7.90739325395862e-05
265 7.84725344544768e-05
266 7.78711403910393e-05
267 7.72698540600913e-05
268 7.66686937083705e-05
269 7.60677642688279e-05
270 7.54671157494613e-05
271 7.48668249599405e-05
272 7.42669592170841e-05
273 7.36676208816789e-05
274 7.3068865674486e-05
275 7.24708304176147e-05
276 7.18735889222444e-05
277 7.12772593232103e-05
278 7.06819352558341e-05
279 7.00877240547015e-05
280 6.94947633945731e-05
281 6.89031829921305e-05
282 6.83130829486345e-05
283 6.77246527516218e-05
284 6.71380023931079e-05
285 6.65533035686394e-05
286 6.59706699082108e-05
287 6.53902382765637e-05
288 6.48121903736865e-05
289 6.4236639190085e-05
290 6.36637899020798e-05
291 6.30937397261278e-05
292 6.25266804235025e-05
293 6.19627533495759e-05
294 6.14021249703001e-05
295 6.0844920383829e-05
296 6.02913248002324e-05
297 5.97414616834158e-05
298 5.91954846527187e-05
299 5.86535523581233e-05
300 5.81157881498484e-05
301 5.75823242172646e-05
302 5.70533412798113e-05
303 5.65289498268839e-05
304 5.60092676664681e-05
305 5.5494448730542e-05
306 5.4984608112818e-05
307 5.44798409549685e-05
308 5.39803171761832e-05
309 5.34861360534933e-05
310 5.2997411117417e-05
311 5.25142131948542e-05
312 5.20366345000411e-05
313 5.15647923151619e-05
314 5.10987421193931e-05
315 5.06385861029912e-05
316 5.01843796172352e-05
317 4.97361705527055e-05
318 4.92940481109372e-05
319 4.88580084549994e-05
320 4.84280905084233e-05
321 4.80043421049459e-05
322 4.75867746416725e-05
323 4.7175396062471e-05
324 4.67701935349396e-05
325 4.63711790814614e-05
326 4.59783271935521e-05
327 4.55916407702261e-05
328 4.5211078386842e-05
329 4.48366169223391e-05
330 4.44682469833424e-05
331 4.41058755029644e-05
332 4.37494870908495e-05
333 4.33990440313892e-05
334 4.30544814165046e-05
335 4.27157609550477e-05
336 4.23828173836682e-05
337 4.20555853111182e-05
338 4.17339825489194e-05
339 4.14179451979635e-05
340 4.11074005199907e-05
341 4.08022619282633e-05
342 4.05024631930928e-05
343 4.02079097341357e-05
344 3.99185734423213e-05
345 3.96343230448792e-05
346 3.93550981101498e-05
347 3.90807860455311e-05
348 3.88113084426323e-05
349 3.85465604821889e-05
350 3.82864606933708e-05
351 3.80309046832394e-05
352 3.77798048702971e-05
353 3.75330807074192e-05
354 3.72906036361087e-05
355 3.70523239325848e-05
356 3.68181067500473e-05
357 3.65878920689511e-05
358 3.63615757947855e-05
359 3.61390702678932e-05
360 3.5920271919565e-05
361 3.57051355948101e-05
362 3.54935186166472e-05
363 3.52853862750635e-05
364 3.50806329336706e-05
365 3.48791788198355e-05
366 3.46809478131149e-05
367 3.44858494827349e-05
368 3.42938002759752e-05
369 3.41047310854492e-05
370 3.39185663804642e-05
371 3.37352285200154e-05
372 3.35546708924994e-05
373 3.33767712419331e-05
374 3.32015319557399e-05
375 3.30288305576687e-05
376 3.28586295594846e-05
377 3.26908796495218e-05
378 3.25254918038809e-05
379 3.23624182030358e-05
380 3.22016042417772e-05
381 3.20429792495247e-05
382 3.18865062070017e-05
383 3.17321119212011e-05
384 3.15797560261899e-05
385 3.14293769037022e-05
386 3.1280934685185e-05
387 3.11343690384547e-05
388 3.09896418073663e-05
389 3.08467103735666e-05
390 3.07055144048718e-05
391 3.05660329544821e-05
392 3.04282222955976e-05
393 3.02920374579685e-05
394 3.01574329455434e-05
395 3.00243787378918e-05
396 2.98928109927488e-05
397 2.97627292979996e-05
398 2.96340844911924e-05
399 2.95068373930008e-05
400 2.93809616280782e-05
401 2.92564153312469e-05
402 2.91331744861623e-05
403 2.90112006382515e-05
404 2.88904695509018e-05
405 2.87709567245997e-05
406 2.86526120945041e-05
407 2.85354300402219e-05
408 2.84193751483031e-05
409 2.83044158564394e-05
410 2.81905534720295e-05
411 2.80777408860899e-05
412 2.7965950835096e-05
413 2.7855184022485e-05
414 2.77453811179385e-05
415 2.76365535114564e-05
416 2.75286643400818e-05
417 2.74216939928351e-05
418 2.73156128258734e-05
419 2.72104258058903e-05
420 2.71060965744141e-05
421 2.70026134430168e-05
422 2.68999517203383e-05
423 2.67980889816499e-05
424 2.66970148175005e-05
425 2.65967139512213e-05
426 2.64971667789382e-05
427 2.63983710340199e-05
428 2.63002864997475e-05
429 2.62029087494398e-05
430 2.61062444195659e-05
431 2.60102509912485e-05
432 2.59149233627909e-05
433 2.58202635947669e-05
434 2.57262430594096e-05
435 2.56328555394703e-05
436 2.55401052413617e-05
437 2.54479611569991e-05
438 2.53564236984971e-05
439 2.52654683663422e-05
440 2.51750970434728e-05
441 2.50853028660458e-05
442 2.49960539733252e-05
443 2.4907353811443e-05
444 2.48191920348972e-05
445 2.47315582271312e-05
446 2.46444422273839e-05
447 2.45578403124114e-05
448 2.44717290698304e-05
449 2.43861101196785e-05
450 2.43009708569275e-05
451 2.42163133492568e-05
452 2.41321191936095e-05
453 2.40483819737847e-05
454 2.39651025779608e-05
455 2.38822547160567e-05
456 2.37998692540486e-05
457 2.37178904924917e-05
458 2.36363465973e-05
459 2.35552181422349e-05
460 2.3474504217802e-05
461 2.33941933167614e-05
462 2.33142795948993e-05
463 2.32347736215388e-05
464 2.31556499059593e-05
465 2.30769126332575e-05
466 2.29985438977565e-05
467 2.29205614523664e-05
468 2.28429395043861e-05
469 2.2765698371785e-05
470 2.26888084782217e-05
471 2.26122813700158e-05
472 2.25361059342788e-05
473 2.24602797658235e-05
474 2.2384806712239e-05
475 2.23096773375175e-05
476 2.22348940184247e-05
477 2.21604488217508e-05
478 2.20863238027391e-05
479 2.20125426579898e-05
480 2.19390918800855e-05
481 2.18659602886362e-05
482 2.1793146778748e-05
483 2.17206754378196e-05
484 2.16485007946687e-05
485 2.15766507984938e-05
486 2.15051192888893e-05
487 2.143390385001e-05
488 2.13629740386523e-05
489 2.1292383820537e-05
490 2.12220752047187e-05
491 2.11520834483281e-05
492 2.10823899813306e-05
493 2.10130105422479e-05
494 2.09439260174804e-05
495 2.08751344317193e-05
496 2.08066450895217e-05
497 2.0738454722391e-05
498 2.06705619874015e-05
499 2.06029527625162e-05
500 2.05356464384465e-05
501 2.04686149061217e-05
502 2.04019062621796e-05
503 2.0335465269028e-05
504 2.02693244553132e-05
505 2.02034759198e-05
506 2.0137905845985e-05
507 2.00726206927015e-05
508 2.0007647258069e-05
509 1.99429326279699e-05
510 1.98785160243631e-05
511 1.98143804297501e-05
512 1.97505232257811e-05
513 1.96869549213829e-05
514 1.96236577316711e-05
515 1.95606168915674e-05
516 1.94978591494532e-05
517 1.94353672888781e-05
518 1.93731514350759e-05
519 1.9311213929285e-05
520 1.92495382016489e-05
521 1.91881273430283e-05
522 1.91269867997335e-05
523 1.90661151293625e-05
524 1.90055018478574e-05
525 1.89451554035713e-05
526 1.88850722189216e-05
527 1.88252634139019e-05
528 1.8765711001123e-05
529 1.8706415190195e-05
530 1.86473883552196e-05
531 1.858862967552e-05
532 1.85301234054691e-05
533 1.84718931208749e-05
534 1.84139146277573e-05
535 1.83562093738487e-05
536 1.82987435977111e-05
537 1.82415612250963e-05
538 1.8184623758799e-05
539 1.8127948287372e-05
540 1.80715334643367e-05
541 1.80153725573007e-05
542 1.7959481933616e-05
543 1.79038316474589e-05
544 1.78484348118957e-05
545 1.77932986886731e-05
546 1.77384117456825e-05
547 1.76837734997548e-05
548 1.76293931559712e-05
549 1.75752574698151e-05
550 1.75213726087975e-05
551 1.74677379476407e-05
552 1.74143393785187e-05
553 1.73612026408421e-05
554 1.7308301924146e-05
555 1.72556423194692e-05
556 1.72032266938515e-05
557 1.71510492847915e-05
558 1.70991186472236e-05
559 1.70474253202713e-05
560 1.69959615696769e-05
561 1.69447433862047e-05
562 1.68937525337753e-05
563 1.68429976170614e-05
564 1.6792479193839e-05
565 1.67421885599595e-05
566 1.66921339754822e-05
567 1.66422997622817e-05
568 1.65927021633649e-05
569 1.65433281686944e-05
570 1.64941800981921e-05
571 1.644525532285e-05
572 1.63965529651477e-05
573 1.63480757287005e-05
574 1.62998123904856e-05
575 1.62517756336911e-05
576 1.62039590634322e-05
577 1.61563588783054e-05
578 1.61089791532731e-05
579 1.60618113618227e-05
580 1.60148528536297e-05
581 1.5968106197306e-05
582 1.59215762529641e-05
583 1.5875253488673e-05
584 1.58291458589588e-05
585 1.57832422402748e-05
586 1.57375402949356e-05
587 1.56920503862068e-05
588 1.56467562391072e-05
589 1.56016753649624e-05
590 1.55567987150107e-05
591 1.55121143237125e-05
592 1.54676317620783e-05
593 1.54233544904514e-05
594 1.53792772898953e-05
595 1.53354050098642e-05
596 1.52917379629969e-05
597 1.52482384834229e-05
598 1.52049536126242e-05
599 1.51618526373909e-05
600 1.51189585189115e-05
601 1.50762698716278e-05
602 1.50337867523831e-05
603 1.49915103975218e-05
604 1.4949452431523e-05
605 1.49075919466668e-05
606 1.48659302148246e-05
607 1.4824471957553e-05
608 1.47831970274126e-05
609 1.47421185303642e-05
610 1.47012440514516e-05
611 1.46605475386252e-05
612 1.46200483719383e-05
613 1.45797398687364e-05
614 1.45396146713495e-05
615 1.44996710282896e-05
616 1.44599302416282e-05
617 1.4420349007338e-05
618 1.43809878565548e-05
619 1.43417931859346e-05
620 1.43027812953278e-05
621 1.42639570412939e-05
622 1.4225302766846e-05
623 1.41868438809922e-05
624 1.41485538769359e-05
625 1.41104361368605e-05
626 1.40725046477996e-05
627 1.4034733258228e-05
628 1.3997155068779e-05
629 1.39597516692902e-05
630 1.39225236921448e-05
631 1.38854672080413e-05
632 1.38485775380559e-05
633 1.38118715256041e-05
634 1.37753250974981e-05
635 1.37389609875527e-05
636 1.37027657345357e-05
637 1.36667339276642e-05
638 1.36308895122284e-05
639 1.35951995154926e-05
640 1.35596804504701e-05
641 1.35243231902393e-05
642 1.34891531082815e-05
643 1.34541290712775e-05
644 1.34192887095708e-05
645 1.33845970999857e-05
646 1.33500983210411e-05
647 1.33157382045113e-05
648 1.32815542635001e-05
649 1.32475339462701e-05
650 1.32136794732673e-05
651 1.31799847906677e-05
652 1.31464660242386e-05
653 1.31130876717123e-05
654 1.30798907349572e-05
655 1.30468498724667e-05
656 1.30139682781305e-05
657 1.29812426514775e-05
658 1.29486810429569e-05
659 1.29162905402325e-05
660 1.2884053166573e-05
661 1.28519811219974e-05
662 1.28200538078715e-05
663 1.27882988429917e-05
664 1.27566895216091e-05
665 1.27252485349061e-05
666 1.26939615796573e-05
667 1.26628255898709e-05
668 1.26318601658681e-05
669 1.26010448475711e-05
670 1.2570394879674e-05
671 1.25398879724514e-05
672 1.2509547158146e-05
673 1.24793518523347e-05
674 1.24493226003608e-05
675 1.24194368993358e-05
676 1.23897042811905e-05
677 1.23601414756536e-05
678 1.2330702858776e-05
679 1.2301444513696e-05
680 1.2272322781115e-05
681 1.2243356088959e-05
682 1.22145283327768e-05
683 1.21858732171631e-05
684 1.21573546287834e-05
685 1.21290031280807e-05
686 1.21007921478622e-05
687 1.207272110193e-05
688 1.20448161489151e-05
689 1.20170400847996e-05
690 1.19894296624068e-05
691 1.19619605811749e-05
692 1.19346294518152e-05
693 1.19074427331611e-05
694 1.18803983042426e-05
695 1.18535020980914e-05
696 1.18267446858056e-05
697 1.18001166455883e-05
698 1.17736391942458e-05
699 1.17472957938958e-05
700 1.17210880077323e-05
701 1.16950178465913e-05
702 1.16690979261591e-05
703 1.1643314444143e-05
704 1.16176740370122e-05
705 1.15921722603218e-05
706 1.15668240212585e-05
707 1.15416077903774e-05
708 1.15165463299149e-05
709 1.14916208353577e-05
710 1.14668540476259e-05
711 1.14422218970844e-05
712 1.14177338694788e-05
713 1.13933852965431e-05
714 1.13691549188388e-05
715 1.13450602761134e-05
716 1.13210685981358e-05
717 1.12972077133122e-05
718 1.12734262778247e-05
719 1.12497453734761e-05
720 1.12261501250543e-05
721 1.1202615112893e-05
722 1.11791376795622e-05
723 1.11557305011445e-05
724 1.11323856408774e-05
725 1.11091196473012e-05
726 1.10859339805813e-05
727 1.10628491327702e-05
728 1.10399032351438e-05
729 1.10171006912907e-05
730 1.09944948452068e-05
731 1.0972114310448e-05
732 1.09499922000822e-05
733 1.09281720188648e-05
734 1.09067256968132e-05
735 1.08857143228391e-05
736 1.08652772841111e-05
737 1.08456545611091e-05
738 1.08271545222749e-05
739 1.08102746754213e-05
740 1.07957432913253e-05
741 1.07844744228203e-05
742 1.07776242916913e-05
743 1.0776154208969e-05
744 1.07804392701638e-05
745 1.07890845324476e-05
746 1.07974559391266e-05
747 1.07967047497937e-05
748 1.07751556885916e-05
749 1.07241541194014e-05
750 1.06474851122584e-05
751 1.0568158721469e-05
752 1.05277646440527e-05
753 1.05802823195944e-05
754 1.07788559553512e-05
755 1.11236104487489e-05
756 1.14362210226204e-05
757 1.13727687391929e-05
758 1.0919734052095e-05
759 1.05795542477694e-05
760 1.06652154023834e-05
761 1.09877997047647e-05
762 1.11421287591895e-05
763 1.09132567658321e-05
764 1.05224516904201e-05
765 1.02866741684693e-05
766 1.02433678073055e-05
767 1.02626644231663e-05
768 1.02554929881649e-05
769 1.02182471248824e-05
770 1.01816938720134e-05
771 1.01623502697379e-05
772 1.01542420658518e-05
773 1.01445666231825e-05
774 1.01266675365252e-05
775 1.01021119114364e-05
776 1.00760943784195e-05
777 1.00527241553294e-05
778 1.00331576344814e-05
779 1.00164533751723e-05
780 1.00011447408122e-05
781 9.98628835624515e-06
782 9.97167444793945e-06
783 9.95744681908661e-06
784 9.94361424311307e-06
785 9.92990252868253e-06
786 9.91586686893697e-06
787 9.90101871956028e-06
788 9.88517205868789e-06
789 9.86844086092731e-06
790 9.85114921370212e-06
791 9.83377758068116e-06
792 9.81671032285192e-06
793 9.80019975571622e-06
794 9.78427275910576e-06
795 9.76891905679622e-06
796 9.75404008585201e-06
797 9.73956881722415e-06
798 9.72549094235831e-06
799 9.71185232145899e-06
800 9.69864950661758e-06
801 9.68585471028405e-06
802 9.67326644918387e-06
803 9.66059393370244e-06
804 9.6474916624345e-06
805 9.63354852778764e-06
806 9.61847385738679e-06
807 9.60219100853976e-06
808 9.58485154711752e-06
809 9.56686799113982e-06
810 9.54881402925878e-06
811 9.53130725100948e-06
812 9.51488189748773e-06
813 9.49979100006715e-06
814 9.48600032479874e-06
815 9.4731171564888e-06
816 9.46054903749882e-06
817 9.44766940413899e-06
818 9.4342075307452e-06
819 9.42056006891789e-06
820 9.408269550093e-06
821 9.4002738819654e-06
822 9.40117358005921e-06
823 9.41718881009024e-06
824 9.45537865426616e-06
825 9.52062809389531e-06
826 9.60892043444517e-06
827 9.69615238766153e-06
828 9.73292247863355e-06
829 9.66842620897523e-06
830 9.50610291639009e-06
831 9.33112712608875e-06
832 9.26061959205526e-06
833 9.37175251358724e-06
834 9.65573953237708e-06
835 9.96178454926167e-06
836 1.00114048180444e-05
837 9.7050538503396e-06
838 9.34768365112859e-06
839 9.25008062147015e-06
840 9.3882401248635e-06
841 9.54529357244382e-06
842 9.54178818268758e-06
843 9.37096460162934e-06
844 9.17080258666658e-06
845 9.05807566375927e-06
846 9.03869542057123e-06
847 9.05570406217748e-06
848 9.0600226929638e-06
849 9.03900234838773e-06
850 9.00758026922688e-06
851 8.9847288045064e-06
852 8.97725551851636e-06
853 8.97916071451732e-06
854 8.97979423619688e-06
855 8.97155868528898e-06
856 8.95317353055702e-06
857 8.92878495051264e-06
858 8.90436926681559e-06
859 8.88433302925762e-06
860 8.8699292479788e-06
861 8.85969607899995e-06
862 8.85101522030141e-06
863 8.84179369187166e-06
864 8.83136673834883e-06
865 8.8206248527456e-06
866 8.81128687879595e-06
867 8.80483194620751e-06
868 8.80176390793963e-06
869 8.80111491774471e-06
870 8.80066556163683e-06
871 8.79754448135373e-06
872 8.7891373112825e-06
873 8.77412077038286e-06
874 8.75317322091007e-06
875 8.72905011206626e-06
876 8.70577429878949e-06
877 8.68743859250287e-06
878 8.67685198002732e-06
879 8.67464270726259e-06
880 8.67875427879028e-06
881 8.68468552539525e-06
882 8.68641846274443e-06
883 8.67856360109442e-06
884 8.65939444416597e-06
885 8.63348048163459e-06
886 8.61268242502433e-06
887 8.61500368110057e-06
888 8.66210866767858e-06
889 8.77466423965245e-06
890 8.95854084248526e-06
891 9.1720709143317e-06
892 9.2944300522646e-06
893 9.18801166172045e-06
894 8.88062088577612e-06
895 8.60131024538191e-06
896 8.54989500354009e-06
897 8.73460273176363e-06
898 8.99350980887448e-06
899 9.08648672570678e-06
900 8.90512798967791e-06
901 8.62456679051604e-06
902 8.48094193628413e-06
903 8.51204827156948e-06
904 8.60393029000761e-06
905 8.64102739761563e-06
906 8.58600489195283e-06
907 8.48024046984364e-06
908 8.38863406471546e-06
909 8.34453017262149e-06
910 8.33994603066657e-06
911 8.34852158781985e-06
912 8.34898467250156e-06
913 8.33543844169782e-06
914 8.3150860836767e-06
915 8.29896708598454e-06
916 8.293473171328e-06
917 8.29755456344117e-06
918 8.30482965774593e-06
919 8.30769050796931e-06
920 8.3009872291484e-06
921 8.28409737341218e-06
922 8.26067745851589e-06
923 8.23653638182265e-06
924 8.21698088415701e-06
925 8.20459089290182e-06
926 8.19878022628728e-06
927 8.19650512795533e-06
928 8.19383066996693e-06
929 8.18770105404099e-06
930 8.17741442205033e-06
931 8.16514570001203e-06
932 8.15538586795839e-06
933 8.15352668848135e-06
934 8.16415339688348e-06
935 8.18929213242114e-06
936 8.22671027744803e-06
937 8.26824222244227e-06
938 8.299389016031e-06
939 8.30277169239224e-06
940 8.26708260692754e-06
941 8.19781179917101e-06
942 8.12001867345202e-06
943 8.06901512362401e-06
944 8.07594846019555e-06
945 8.15639016771286e-06
946 8.2988936842554e-06
947 8.44849930103919e-06
948 8.50899656690274e-06
949 8.41205539003909e-06
950 8.21442340637191e-06
951 8.07192507235044e-06
952 8.10203291035094e-06
953 8.30894767744894e-06
954 8.58499472933261e-06
955 8.73367661391455e-06
956 8.60089742094061e-06
957 8.27472136322172e-06
958 8.01031823094434e-06
959 7.94097558287632e-06
960 8.00660366184758e-06
961 8.07959790183332e-06
962 8.07640254141262e-06
963 8.0018955328498e-06
964 7.91988465742577e-06
965 7.88321293576644e-06
966 7.89674897205828e-06
967 7.93163442303069e-06
968 7.95450966517564e-06
969 7.94711224294531e-06
970 7.91205605565892e-06
971 7.86629358717761e-06
972 7.82815506816803e-06
973 7.80717985726653e-06
974 7.80197314309561e-06
975 7.80441837910928e-06
976 7.80548605039399e-06
977 7.79991677202219e-06
978 7.78816194735299e-06
979 7.77526460016276e-06
980 7.76775848176925e-06
981 7.77031660614824e-06
982 7.78367386367051e-06
983 7.80423693846899e-06
984 7.82470452964645e-06
985 7.83613351096335e-06
986 7.83106629764774e-06
987 7.80728920179996e-06
988 7.76982658390324e-06
989 7.72995936149101e-06
990 7.7010054813087e-06
991 7.69328710781281e-06
992 7.71077942118836e-06
993 7.74915482981697e-06
994 7.79490142832628e-06
995 7.82708389657216e-06
996 7.82504307395016e-06
997 7.78231432985166e-06
998 7.71735861171408e-06
999 7.66966291720905e-06
1000 7.68382337668072e-06
1001 7.79449983667746e-06
1002 8.0109236506587e-06
1003 8.28232784222394e-06
1004 8.45855245756866e-06
1005 8.36312062979516e-06
1006 8.02267971700132e-06
1007 7.70611968459178e-06
1008 7.62717143132363e-06
1009 7.76792132395343e-06
1010 7.95785028362417e-06
1011 8.01001870165408e-06
1012 7.87098121968199e-06
1013 7.67034505422259e-06
1014 7.56217873920662e-06
1015 7.57701724829474e-06
1016 7.64679089915177e-06
1017 7.69223958396736e-06
1018 7.67539803803174e-06
1019 7.60926491505387e-06
1020 7.53461643920872e-06
1021 7.48522076321478e-06
1022 7.46955363073454e-06
1023 7.47554698321551e-06
1024 7.48482320211963e-06
1025 7.48447310083833e-06
1026 7.47225272057506e-06
1027 7.45495951015585e-06
1028 7.44238312400114e-06
1029 7.44116710826859e-06
1030 7.45189689688175e-06
1031 7.46951326391354e-06
1032 7.48581451581742e-06
1033 7.49248344966702e-06
1034 7.484501767685e-06
1035 7.46239311588681e-06
1036 7.43227900290577e-06
1037 7.40338943305829e-06
1038 7.3842988381756e-06
1039 7.37984322096708e-06
1040 7.38959230339731e-06
1041 7.40802639498384e-06
1042 7.42587584490195e-06
1043 7.43302990890982e-06
1044 7.42291226263347e-06
1045 7.39692489482024e-06
1046 7.36629359554541e-06
1047 7.3495937034096e-06
1048 7.3680041676738e-06
1049 7.44028310339928e-06
1050 7.57578439003481e-06
1051 7.7585853990314e-06
1052 7.92315554853928e-06
1053 7.95877389947464e-06
1054 7.79927950045334e-06
1055 7.53226950500618e-06
1056 7.34531123036675e-06
1057 7.35950241281813e-06
1058 7.55734937030184e-06
1059 7.7999417413821e-06
1060 7.8849603442066e-06
1061 7.71735899451897e-06
1062 7.44606228764155e-06
1063 7.2926877789925e-06
1064 7.31783360841121e-06
1065 7.43107960587963e-06
1066 7.51139812038559e-06
1067 7.49125105325987e-06
1068 7.38756181828393e-06
1069 7.27109394649261e-06
1070 7.20080869154316e-06
1071 7.18799622845268e-06
1072 7.2078748205584e-06
1073 7.22766167982059e-06
1074 7.22710172063046e-06
1075 7.2059141098535e-06
1076 7.17844007702695e-06
1077 7.16115700249986e-06
1078 7.16247377408052e-06
1079 7.18003214394969e-06
1080 7.20386646957394e-06
1081 7.22156486965275e-06
1082 7.22349670478906e-06
1083 7.20671808185358e-06
1084 7.17636494940166e-06
1085 7.14303683224671e-06
1086 7.11794774854724e-06
1087 7.10820834193271e-06
1088 7.11462921998418e-06
1089 7.13191030055071e-06
1090 7.1505359588997e-06
1091 7.15997871392204e-06
1092 7.15319085653476e-06
1093 7.1307508289209e-06
1094 7.10233807232896e-06
1095 7.08416960559788e-06
1096 7.09399694986956e-06
1097 7.1463826669671e-06
1098 7.2477234080992e-06
1099 7.38702252345291e-06
1100 7.52224274069846e-06
1101 7.57959772723638e-06
1102 7.49834802071803e-06
1103 7.30518347058506e-06
1104 7.11946434428512e-06
1105 7.061079472237e-06
1106 7.17194621646655e-06
1107 7.39954694406464e-06
1108 7.60024252421942e-06
1109 7.60309042835416e-06
1110 7.38515955767127e-06
1111 7.13398152107914e-06
1112 7.03675498137102e-06
1113 7.11184812374199e-06
1114 7.2544388247664e-06
1115 7.33928776153192e-06
1116 7.29832548085341e-06
1117 7.16177831083797e-06
1118 7.02215956849273e-06
1119 6.94937316048083e-06
1120 6.94937321021882e-06
1121 6.9849876913608e-06
1122 7.01265859248679e-06
1123 7.0081304759384e-06
1124 6.97533027249619e-06
1125 6.93714339838891e-06
1126 6.91647784289984e-06
1127 6.92254538314785e-06
1128 6.94927670963352e-06
1129 6.98141344734893e-06
1130 7.00211809423479e-06
1131 6.99996062270003e-06
1132 6.97400876248366e-06
1133 6.93416524111257e-06
1134 6.89598723369045e-06
1135 6.87307790592229e-06
1136 6.87161720591689e-06
1137 6.88893615041763e-06
1138 6.91505706829076e-06
1139 6.93625394365682e-06
1140 6.94024803848947e-06
1141 6.92243788158464e-06
1142 6.88975340601417e-06
1143 6.85916161557998e-06
1144 6.8512748931937e-06
1145 6.88347104382814e-06
1146 6.96470366445823e-06
1147 7.08919676917219e-06
1148 7.22575724765306e-06
1149 7.31224015115828e-06
1150 7.2821625005659e-06
1151 7.12897726362627e-06
1152 6.93895277059653e-06
1153 6.83125561629083e-06
1154 6.87344259020506e-06
1155 7.0508763165833e-06
1156 7.26655330485215e-06
1157 7.36478816065755e-06
1158 7.24713801680821e-06
1159 7.00418229904187e-06
1160 6.83188955630243e-06
1161 6.82932022932903e-06
1162 6.95150515017673e-06
1163 7.08514344793798e-06
1164 7.12558941007302e-06
1165 7.04010000251998e-06
1166 6.88885853072918e-06
1167 6.76631366758329e-06
1168 6.72275095414676e-06
1169 6.74592622473824e-06
1170 6.79098202915895e-06
1171 6.81497105414763e-06
1172 6.79951850557359e-06
1173 6.75667927740875e-06
1174 6.71517303096891e-06
1175 6.69873606984339e-06
1176 6.71362575488388e-06
1177 6.74966359692775e-06
1178 6.78779312135447e-06
1179 6.80855323142993e-06
1180 6.800229614079e-06
1181 6.76460946724688e-06
1182 6.71636800220199e-06
1183 6.67528426134822e-06
1184 6.65659739507163e-06
1185 6.66524657688683e-06
1186 6.69514601625565e-06
1187 6.73164183950803e-06
1188 6.75630385948978e-06
1189 6.75459779930776e-06
1190 6.72452383732036e-06
1191 6.6802034428548e-06
1192 6.64662001614857e-06
1193 6.64901175184696e-06
1194 6.70432542904109e-06
1195 6.81538866764697e-06
1196 6.96248922249509e-06
1197 7.0924882473733e-06
1198 7.12752026910124e-06
1199 7.0210110818536e-06
1200 6.82307211885558e-06
1201 6.65825514634122e-06
1202 6.62758477609771e-06
1203 6.74881790718729e-06
1204 6.95449745524712e-06
1205 7.1088674618025e-06
1206 7.08132812654583e-06
1207 6.88253840053221e-06
1208 6.67534146536752e-06
1209 6.60177011990726e-06
1210 6.67183231417567e-06
1211 6.80193978297439e-06
1212 6.88801040515941e-06
1213 6.86502424640523e-06
1214 6.74759038821193e-06
1215 6.61311398575393e-06
1216 6.53313911058717e-06
1217 6.52654106314543e-06
1218 6.56610397076207e-06
1219 6.60819236841803e-06
1220 6.61926017908598e-06
1221 6.59215300125027e-06
1222 6.54609946160178e-06
1223 6.51006929519582e-06
1224 6.50349750763723e-06
1225 6.5281415002616e-06
1226 6.57099128531513e-06
1227 6.61151541869742e-06
1228 6.62970379394778e-06
1229 6.61472266383356e-06
1230 6.5710218910553e-06
1231 6.51693510889118e-06
1232 6.47524622454654e-06
1233 6.46236246026888e-06
1234 6.48236037292804e-06
1235 6.52644529264279e-06
1236 6.57541723558808e-06
1237 6.60526378304382e-06
1238 6.59790490598056e-06
1239 6.55364634649658e-06
1240 6.49508706018764e-06
1241 6.45681075717164e-06
1242 6.46958783612206e-06
1243 6.54982837833273e-06
1244 6.69289679633422e-06
1245 6.8617149935335e-06
1246 6.97814536865238e-06
1247 6.95423333807454e-06
1248 6.77814782656938e-06
1249 6.55794127979448e-06
1250 6.43625237550793e-06
1251 6.47950090826299e-06
1252 6.6524767667886e-06
1253 6.84029697417543e-06
1254 6.8985361654228e-06
1255 6.76906962482349e-06
1256 6.55752660883024e-06
1257 6.42373466597235e-06
1258 6.4301766622421e-06
1259 6.52974618819968e-06
1260 6.63145636714546e-06
1261 6.65859178061723e-06
1262 6.59070763786573e-06
1263 6.47272866061144e-06
1264 6.37402580050406e-06
1265 6.33559683027585e-06
1266 6.35376140234456e-06
1267 6.39674154978565e-06
1268 6.42815797124996e-06
1269 6.42600116496084e-06
1270 6.39260952706167e-06
1271 6.34995514214864e-06
1272 6.3233270646279e-06
1273 6.32681904555454e-06
1274 6.35888653732763e-06
1275 6.40561786990901e-06
1276 6.44658128789999e-06
1277 6.46211441601707e-06
1278 6.44211543976425e-06
1279 6.39273310154564e-06
1280 6.33466622801393e-06
1281 6.29294582221718e-06
1282 6.28539615732393e-06
1283 6.31647931914614e-06
1284 6.37597989960881e-06
1285 6.44040701569537e-06
1286 6.47875465187298e-06
1287 6.46702648587905e-06
1288 6.40684384567436e-06
1289 6.33069990296775e-06
1290 6.28524959633836e-06
1291 6.30786675870354e-06
1292 6.41375683230194e-06
1293 6.58836903832949e-06
1294 6.77350605160498e-06
1295 6.86533163785086e-06
1296 6.77684619665442e-06
1297 6.5447781141259e-06
1298 6.32640954778552e-06
1299 6.25797611464307e-06
1300 6.35960531081281e-06
1301 6.54762339902248e-06
1302 6.68124146629978e-06
1303 6.64788075699363e-06
1304 6.4720466141921e-06
1305 6.29666437035326e-06
1306 6.23291861856501e-06
1307 6.28266042301817e-06
1308 6.37784105927963e-06
1309 6.44052793674632e-06
1310 6.42527824012973e-06
1311 6.34172534308419e-06
1312 6.24149725991785e-06
1313 6.17589307694288e-06
1314 6.16421403076828e-06
1315 6.19296132153835e-06
1316 6.2322364433598e-06
1317 6.25346982197073e-06
1318 6.24327205134279e-06
1319 6.2089440904245e-06
1320 6.17183450923875e-06
1321 6.15325046116055e-06
1322 6.16369438510844e-06
1323 6.20033126352837e-06
1324 6.24961102246857e-06
1325 6.29168813226499e-06
1326 6.30690410918788e-06
1327 6.28482544584585e-06
1328 6.23174017144734e-06
1329 6.16967843392047e-06
1330 6.12601969685045e-06
1331 6.121219107591e-06
1332 6.16168648726045e-06
1333 6.23728979487481e-06
1334 6.32067683081772e-06
1335 6.37211227516588e-06
1336 6.35840380791564e-06
1337 6.280809332182e-06
1338 6.18370858340711e-06
1339 6.12862583881935e-06
1340 6.16008471254759e-06
1341 6.29010218400339e-06
1342 6.49087881177124e-06
1343 6.68054020280806e-06
1344 6.73522810323846e-06
1345 6.58506282391613e-06
1346 6.32021531110638e-06
1347 6.1248578639983e-06
1348 6.1072597463685e-06
1349 6.24183415354196e-06
1350 6.41335005902022e-06
1351 6.48436435346511e-06
1352 6.3936956604671e-06
1353 6.21855643512248e-06
1354 6.09139299534434e-06
1355 6.07409369379752e-06
1356 6.13845847663441e-06
1357 6.21717149940082e-06
1358 6.25030063705623e-06
1359 6.21462202188283e-06
1360 6.13238508728742e-06
1361 6.05048253188301e-06
1362 6.00599050937234e-06
1363 6.00753901736084e-06
1364 6.03929501830081e-06
1365 6.07469728874932e-06
1366 6.0904156682362e-06
1367 6.07730851864829e-06
1368 6.04412305182933e-06
1369 6.01077563189278e-06
1370 5.99635554543454e-06
1371 6.01025213509132e-06
1372 6.04997839115384e-06
1373 6.10288735902742e-06
1374 6.14937963305806e-06
1375 6.16856775925712e-06
1376 6.14771388907087e-06
1377 6.09155326358746e-06
1378 6.02313115294351e-06
1379 5.97336777730106e-06
1380 5.96718960332865e-06
1381 6.01469035510505e-06
1382 6.10705203296646e-06
1383 6.2131217961614e-06
1384 6.28222952681057e-06
1385 6.26813862147912e-06
1386 6.17052170159837e-06
1387 6.04914749047225e-06
1388 5.98393487205584e-06
1389 6.02547812267318e-06
1390 6.17773663869059e-06
1391 6.39417612902093e-06
1392 6.56913730878728e-06
1393 6.57244600033025e-06
1394 6.37092488053881e-06
1395 6.10186479788055e-06
1396 5.94655170749192e-06
1397 5.97124239476443e-06
1398 6.11480504408846e-06
1399 6.25232582462942e-06
1400 6.27130657093744e-06
1401 6.15732140074599e-06
1402 6.00567565012255e-06
1403 5.92040027935781e-06
1404 5.92988125003302e-06
1405 5.99616674179515e-06
1406 6.0602534124321e-06
1407 6.07661275608962e-06
1408 6.03385301234383e-06
1409 5.95708567008302e-06
1410 5.88724604755342e-06
1411 5.85354977555852e-06
1412 5.86084049158941e-06
1413 5.89385917582064e-06
1414 5.92845618641036e-06
1415 5.94337067738593e-06
1416 5.93023316097785e-06
1417 5.89737476897056e-06
1418 5.86418255021925e-06
1419 5.84992256946748e-06
1420 5.86506253874575e-06
1421 5.90862033522654e-06
1422 5.96893292481582e-06
1423 6.02556904416574e-06
1424 6.0542373510053e-06
1425 6.03721069181162e-06
1426 5.97610792585357e-06
1427 5.89603376255354e-06
1428 5.83450701796551e-06
1429 5.82375688429337e-06
1430 5.87887054548375e-06
1431 5.99162891568028e-06
1432 6.12418492496047e-06
1433 6.20997587530425e-06
1434 6.18705462773761e-06
1435 6.05938600362066e-06
1436 5.91263895000083e-06
1437 5.84731473463052e-06
1438 5.91124604643056e-06
1439 6.08772304566685e-06
1440 6.30121309219689e-06
1441 6.42563060981871e-06
1442 6.35323748854333e-06
1443 6.11540284456424e-06
1444 5.87971252619113e-06
1445 5.78706635545601e-06
1446 5.85006398479138e-06
1447 5.98442481525296e-06
1448 6.07603433788739e-06
1449 6.05171758971679e-06
1450 5.93459314446321e-06
1451 5.81661950693757e-06
1452 5.76912194194534e-06
1453 5.79793534249973e-06
1454 5.86386713052889e-06
1455 5.91733725840982e-06
1456 5.92334808846573e-06
1457 5.87714610444934e-06
1458 5.8041630093264e-06
1459 5.74123786023506e-06
1460 5.71370438029462e-06
1461 5.72527137476797e-06
1462 5.76151525510937e-06
1463 5.79900649944065e-06
1464 5.81594035864441e-06
1465 5.80259385474236e-06
1466 5.76677053532038e-06
1467 5.72930464937116e-06
1468 5.71232098955932e-06
1469 5.72933733344883e-06
1470 5.78145652596618e-06
1471 5.85717515733108e-06
1472 5.93278906002581e-06
1473 5.97636968624471e-06
1474 5.96115812401266e-06
1475 5.885793645799e-06
1476 5.78335660028628e-06
1477 5.70564894175618e-06
1478 5.69606211220375e-06
1479 5.77232993359544e-06
1480 5.9185065062195e-06
1481 6.07613421621522e-06
1482 6.15150121952013e-06
1483 6.07818968845208e-06
1484 5.89997051747559e-06
1485 5.74826841148735e-06
1486 5.72507152973856e-06
1487 5.84215039189928e-06
1488 6.03702169099662e-06
1489 6.19742860674677e-06
1490 6.20643371718188e-06
1491 6.03938119647651e-06
1492 5.80996637689069e-06
1493 5.66504759991204e-06
1494 5.66098950738336e-06
1495 5.7546407621345e-06
1496 5.85628195137744e-06
1497 5.88524259148571e-06
1498 5.8211279898579e-06
1499 5.71602127275028e-06
1500 5.64283932247278e-06
1501 5.63679515863669e-06
1502 5.68557654290203e-06
1503 5.75124126989124e-06
1504 5.79318436244591e-06
1505 5.78621002178537e-06
1506 5.73273641801819e-06
1507 5.66048134009733e-06
1508 5.60423310513869e-06
1509 5.58610432488393e-06
1510 5.60771730206966e-06
1511 5.6533007533055e-06
1512 5.69732529731581e-06
1513 5.71501362500726e-06
1514 5.69513429260482e-06
1515 5.64801311186969e-06
1516 5.60071230726322e-06
1517 5.58200312639201e-06
1518 5.60959008044648e-06
1519 5.68537623113485e-06
1520 5.79383594523364e-06
1521 5.89980157528203e-06
1522 5.95326081498371e-06
1523 5.91475301003896e-06
1524 5.7933026234025e-06
1525 5.65411990161024e-06
1526 5.57738591400181e-06
1527 5.61115516140021e-06
1528 5.75290902293801e-06
1529 5.94328035230518e-06
1530 6.06853788109873e-06
1531 6.02392609394542e-06
1532 5.83114583108113e-06
1533 5.64180631856814e-06
1534 5.5861800385415e-06
1535 5.67868887557665e-06
1536 5.84589871177599e-06
1537 5.9768406699412e-06
1538 5.97615167308163e-06
1539 5.83554102151496e-06
1540 5.65046290379456e-06
1541 5.53319268581376e-06
1542 5.5258459412677e-06
1543 5.5960406246669e-06
1544 5.67613615176654e-06
1545 5.70423173629209e-06
1546 5.66045177219365e-06
1547 5.57823365898713e-06
1548 5.51341555743434e-06
1549 5.50125443021798e-06
1550 5.54135920882715e-06
1551 5.60806225369248e-06
1552 5.66572910454255e-06
1553 5.68342855800097e-06
1554 5.6498583855813e-06
1555 5.58111752280155e-06
1556 5.51170580909144e-06
1557 5.47411232165729e-06
1558 5.48325467342181e-06
1559 5.53284372317719e-06
1560 5.59875661743092e-06
1561 5.64660977886433e-06
1562 5.64685525716868e-06
1563 5.59488430118549e-06
1564 5.51920251634641e-06
1565 5.46544668766913e-06
1566 5.47081809987304e-06
1567 5.55027482107562e-06
1568 5.69303231223728e-06
1569 5.85660952534539e-06
1570 5.96425106991205e-06
1571 5.93850319674516e-06
1572 5.77647780453105e-06
1573 5.58024374885946e-06
1574 5.47798816707967e-06
1575 5.53027261762651e-06
1576 5.70681944678242e-06
1577 5.89829246200324e-06
1578 5.95768790878637e-06
1579 5.82090438161487e-06
1580 5.59846962566724e-06
1581 5.46243575350047e-06
1582 5.47958249619285e-06
1583 5.59923576393118e-06
1584 5.72278966615158e-06
1585 5.76152916531569e-06
1586 5.68653734811519e-06
1587 5.54941370189965e-06
1588 5.43642465533622e-06
1589 5.39959676393309e-06
1590 5.43380096251767e-06
1591 5.49663567106506e-06
1592 5.53790549506772e-06
1593 5.52712190415861e-06
1594 5.47123722549614e-06
1595 5.40717540209457e-06
1596 5.37343550544023e-06
1597 5.38739188771231e-06
1598 5.44211673458861e-06
1599 5.51373528123733e-06
1600 5.57032825998505e-06
1601 5.58350226409488e-06
1602 5.54382036455081e-06
1603 5.46998830186496e-06
1604 5.40020903727623e-06
1605 5.37082368401443e-06
1606 5.39934535614961e-06
1607 5.47842031473067e-06
1608 5.57531808631495e-06
1609 5.63824339838703e-06
1610 5.62180951124702e-06
1611 5.52654696406307e-06
1612 5.41048526336851e-06
1613 5.3497517309431e-06
1614 5.39199279581126e-06
1615 5.54131929453305e-06
1616 5.75577826156248e-06
1617 5.93837823981147e-06
1618 5.96386617779388e-06
1619 5.78933768480283e-06
1620 5.53789358637147e-06
1621 5.39417874101389e-06
1622 5.43992367729729e-06
1623 5.61882359040311e-06
1624 5.78442937770873e-06
1625 5.78841377674166e-06
1626 5.61640094609572e-06
1627 5.41452103242079e-06
1628 5.32575675560309e-06
1629 5.36884598378862e-06
1630 5.47305471165771e-06
1631 5.55127285783996e-06
1632 5.54861131796258e-06
1633 5.46970003778569e-06
1634 5.36893316116505e-06
1635 5.30438030033054e-06
1636 5.29913771263324e-06
1637 5.33770859023619e-06
1638 5.38370641978148e-06
1639 5.4016989636807e-06
1640 5.37681305212701e-06
1641 5.32322615365288e-06
1642 5.27308078268618e-06
1643 5.25508293458543e-06
1644 5.28058632109918e-06
1645 5.34234255411548e-06
1646 5.41858493985359e-06
1647 5.47839149867002e-06
1648 5.49227514934358e-06
1649 5.44941619962103e-06
1650 5.36993344901759e-06
1651 5.29743093391488e-06
1652 5.27577423614645e-06
1653 5.32861590052391e-06
1654 5.44840790706047e-06
1655 5.58834684838416e-06
1656 5.66716992667082e-06
1657 5.61745336469244e-06
1658 5.46005090917845e-06
1659 5.3060162725771e-06
1660 5.26507757747652e-06
1661 5.37586447313387e-06
1662 5.60160595597381e-06
1663 5.83407313481388e-06
1664 5.91613038469774e-06
1665 5.76267651108964e-06
1666 5.48876989991953e-06
1667 5.31191206842863e-06
1668 5.33706031280445e-06
1669 5.50100264673503e-06
1670 5.64524709423253e-06
1671 5.63016804999705e-06
1672 5.46143417867739e-06
1673 5.28235019903178e-06
1674 5.21210203041633e-06
1675 5.25421829689066e-06
1676 5.34029903587196e-06
1677 5.39661502418909e-06
1678 5.38436179686741e-06
1679 5.3156298100987e-06
1680 5.23748549507985e-06
1681 5.19409397581327e-06
1682 5.19999274173344e-06
1683 5.23924062623138e-06
1684 5.27966653063316e-06
1685 5.29134003368625e-06
1686 5.2638310381603e-06
1687 5.21262262331845e-06
1688 5.16816611551008e-06
1689 5.15684826751794e-06
1690 5.18908024282894e-06
1691 5.25748900681577e-06
1692 5.33900593335801e-06
1693 5.39945336708314e-06
1694 5.40622484868081e-06
1695 5.34981595956552e-06
1696 5.25816003005275e-06
1697 5.1856317027088e-06
1698 5.18423302109383e-06
1699 5.27898768964974e-06
1700 5.45133632989803e-06
1701 5.62579707175104e-06
1702 5.68782204268814e-06
1703 5.57344653273617e-06
1704 5.3603397818236e-06
1705 5.21175830847653e-06
1706 5.23262463625329e-06
1707 5.41480880666967e-06
1708 5.65662786877752e-06
1709 5.79319782323395e-06
1710 5.69973834441129e-06
1711 5.44114381817451e-06
1712 5.23062904589366e-06
1713 5.20951311910878e-06
1714 5.34731663881516e-06
1715 5.50096263873812e-06
1716 5.52435189682399e-06
1717 5.39143359734595e-06
1718 5.21769988726106e-06
1719 5.12702600996562e-06
1720 5.1446405953115e-06
1721 5.21717318680004e-06
1722 5.27461997101852e-06
1723 5.27318839127489e-06
1724 5.21534539155155e-06
1725 5.14132994666738e-06
1726 5.09617415245245e-06
1727 5.10044623247552e-06
1728 5.14358730407949e-06
1729 5.1942758925172e-06
1730 5.21854491619678e-06
1731 5.1996712624458e-06
1732 5.14883784585862e-06
1733 5.09708390872277e-06
1734 5.07519341352847e-06
1735 5.09847005902841e-06
1736 5.16266007499766e-06
1737 5.24544157753937e-06
1738 5.31087520494467e-06
1739 5.32261767949294e-06
1740 5.26735096606501e-06
1741 5.17223360274599e-06
1742 5.0952183099362e-06
1743 5.09352550936981e-06
1744 5.19504238472024e-06
1745 5.3807085578228e-06
1746 5.57252758692783e-06
1747 5.64958795834514e-06
1748 5.53770389011277e-06
1749 5.31216293708781e-06
1750 5.14662757389317e-06
1751 5.15535405920531e-06
1752 5.32893475257978e-06
1753 5.55837784643742e-06
1754 5.67673627172027e-06
1755 5.57058045869141e-06
1756 5.31701952510844e-06
1757 5.12376390648939e-06
1758 5.11841670514102e-06
1759 5.2632472318237e-06
1760 5.41510611729024e-06
1761 5.43496135385624e-06
1762 5.30486243732753e-06
1763 5.13823340853037e-06
1764 5.05130364070183e-06
1765 5.06665203037215e-06
1766 5.1333060158143e-06
1767 5.18444744912472e-06
1768 5.17862390259083e-06
1769 5.11968816585551e-06
1770 5.04816617485915e-06
1771 5.00849497031552e-06
1772 5.02016900805202e-06
1773 5.07085923784345e-06
1774 5.12715763356653e-06
1775 5.15372880904863e-06
1776 5.13388126677228e-06
1777 5.07984468356426e-06
1778 5.0240149613856e-06
1779 4.99831064182388e-06
1780 5.0185729318919e-06
1781 5.08046720426591e-06
1782 5.16121767235589e-06
1783 5.22453724016714e-06
1784 5.23436249588727e-06
1785 5.17834076418922e-06
1786 5.0845411871947e-06
1787 5.01102988392788e-06
1788 5.01418728582337e-06
1789 5.12056686519102e-06
1790 5.31072442200298e-06
1791 5.50771083140944e-06
1792 5.59037263236917e-06
1793 5.48095843733165e-06
1794 5.25167245335467e-06
1795 5.07781813841035e-06
1796 5.07681004480176e-06
1797 5.2402704535659e-06
1798 5.45997595269299e-06
1799 5.57272550905097e-06
1800 5.46939768053534e-06
1801 5.2243474755187e-06
1802 5.03745066549044e-06
1803 5.03354923608867e-06
1804 5.17847163505891e-06
1805 5.33381027612023e-06
1806 5.3622925042518e-06
1807 5.24079501751373e-06
1808 5.07515969871974e-06
1809 4.98107214141896e-06
1810 4.98673374282887e-06
1811 5.0466073382438e-06
1812 5.0964334139536e-06
1813 5.0939767870517e-06
1814 5.03961535924802e-06
1815 4.9704940283668e-06
1816 4.93045056781938e-06
1817 4.94107997006665e-06
1818 4.99285081279055e-06
1819 5.05391421334878e-06
1820 5.08800751397587e-06
1821 5.07483960010546e-06
1822 5.02263306767503e-06
1823 4.9622375000169e-06
1824 4.92727570922114e-06
1825 4.93694723813221e-06
1826 4.99042844825937e-06
1827 5.06821771084631e-06
1828 5.13606184249227e-06
1829 5.15654096888696e-06
1830 5.11145743686825e-06
1831 5.02161726778638e-06
1832 4.94189492528108e-06
1833 4.93169046933417e-06
1834 5.02489839426801e-06
1835 5.21165738831542e-06
1836 5.42458991592554e-06
1837 5.5424985081487e-06
1838 5.46573793336336e-06
1839 5.23842048494316e-06
1840 5.03513792970978e-06
1841 4.99465763148521e-06
1842 5.12572578381665e-06
1843 5.33138682001777e-06
1844 5.45790942441116e-06
1845 5.38825454654557e-06
1846 5.16643142667306e-06
1847 4.97216409911516e-06
1848 4.94106505621872e-06
1849 5.06468413652428e-06
1850 5.22579903083908e-06
1851 5.28704662183799e-06
1852 5.19798106246583e-06
1853 5.03731551848574e-06
1854 4.92275410035603e-06
1855 4.90306931855145e-06
1856 4.94904967140997e-06
1857 5.00181086371398e-06
1858 5.01445899869779e-06
1859 4.97546426858619e-06
1860 4.91074279063142e-06
1861 4.86204260274903e-06
1862 4.85840268860116e-06
1863 4.90135605080155e-06
1864 4.96713073161459e-06
1865 5.01953285247936e-06
1866 5.02899003063106e-06
1867 4.99028232958665e-06
1868 4.92632708049001e-06
1869 4.87300166263083e-06
1870 4.85835057473238e-06
1871 4.89150426874474e-06
1872 4.96187711407359e-06
1873 5.0414297869672e-06
1874 5.09089865818879e-06
1875 5.07757535084252e-06
1876 5.00125032321108e-06
1877 4.90501713246161e-06
1878 4.85469059574584e-06
1879 4.90366040217793e-06
1880 5.06644166531345e-06
1881 5.29971492380099e-06
1882 5.49041682340601e-06
1883 5.49997809251579e-06
1884 5.30092567707641e-06
1885 5.04726803818301e-06
1886 4.9247457667434e-06
1887 4.98816569605154e-06
1888 5.16225970148554e-06
1889 5.30708442081362e-06
1890 5.29678190197558e-06
1891 5.12597407942295e-06
1892 4.9293464545741e-06
1893 4.84938564238035e-06
1894 4.91949412273485e-06
1895 5.06633823249558e-06
1896 5.17028130531116e-06
1897 5.14807722939992e-06
1898 5.02106379229872e-06
1899 4.88780874441375e-06
1900 4.82547819036228e-06
1901 4.84020754099035e-06
1902 4.89078033893975e-06
1903 4.9280974576682e-06
1904 4.92212951996862e-06
1905 4.87513383351512e-06
1906 4.81695500464951e-06
1907 4.78377237111971e-06
1908 4.79668159725222e-06
1909 4.85242943559783e-06
1910 4.92577922717174e-06
1911 4.9801863815091e-06
1912 4.98606473264829e-06
1913 4.93968740133965e-06
1914 4.8675482418048e-06
1915 4.80960664983598e-06
1916 4.7964409946033e-06
1917 4.83770716019905e-06
1918 4.92139759566967e-06
1919 5.01516313189398e-06
1920 5.07195713295161e-06
1921 5.05191644073477e-06
1922 4.95686739343526e-06
1923 4.84369253017647e-06
1924 4.79437687772588e-06
1925 4.87006601446183e-06
1926 5.07840315666641e-06
1927 5.34718613831586e-06
1928 5.51816451732634e-06
1929 5.44349516573561e-06
1930 5.16796757032978e-06
1931 4.91750626796517e-06
1932 4.85376720149588e-06
1933 4.9574087119808e-06
1934 5.10406277109965e-06
1935 5.16115933812955e-06
1936 5.07279452710563e-06
1937 4.90555961318861e-06
1938 4.78365300171646e-06
1939 4.77827109746087e-06
1940 4.87109759195192e-06
1941 4.98518687530591e-06
1942 5.0354597593838e-06
1943 4.98579084595718e-06
1944 4.87582172414847e-06
1945 4.78013890070628e-06
1946 4.74460655564002e-06
1947 4.76622317124864e-06
1948 4.81254677575649e-06
1949 4.8463324975323e-06
1950 4.84343809636911e-06
1951 4.80383355760949e-06
1952 4.75076452621792e-06
1953 4.71657628509092e-06
1954 4.7251458088482e-06
1955 4.7811104169071e-06
1956 4.86706059588116e-06
1957 4.94733216704901e-06
1958 4.98162279338743e-06
1959 4.94888648105984e-06
1960 4.86608406280453e-06
1961 4.78034516504522e-06
1962 4.73948583534778e-06
1963 4.76864056153659e-06
1964 4.86444153180443e-06
1965 4.99459927771895e-06
1966 5.09807720661115e-06
1967 5.10615719795382e-06
1968 4.99755080296538e-06
1969 4.83880574453366e-06
1970 4.74688363727438e-06
1971 4.81001692298477e-06
1972 5.03703056287108e-06
1973 5.32895397409305e-06
1974 5.48612006134874e-06
1975 5.35716153926558e-06
1976 5.04494777509024e-06
1977 4.80997117957571e-06
1978 4.77566394652129e-06
1979 4.87352574429778e-06
1980 4.97064241899992e-06
1981 4.97219172457264e-06
1982 4.87387897418756e-06
1983 4.75126506804813e-06
1984 4.6866660503575e-06
1985 4.70847396272234e-06
1986 4.7877672546548e-06
1987 4.86607540839401e-06
1988 4.89029291106391e-06
1989 4.84580465709428e-06
1990 4.76462514242826e-06
1991 4.69626326449202e-06
1992 4.67120156066869e-06
1993 4.68925999896896e-06
1994 4.72951781960163e-06
1995 4.764407675939e-06
1996 4.77181332669829e-06
1997 4.74521980775577e-06
1998 4.69805918301702e-06
1999 4.65758962864271e-06
};
\addlegendentry{Train}
\addplot [semithick, black]
table {%
0 0.0129028288647532
1 0.012514472939074
2 0.0121427467092872
3 0.0117810638621449
4 0.0114212287589908
5 0.0110523477196693
6 0.0106608429923654
7 0.010231489315629
8 0.0097490381449461
9 0.00920352153480053
10 0.00859939586371183
11 0.00796008761972189
12 0.00732345506548882
13 0.00672581233084202
14 0.00618677446618676
15 0.00570852914825082
16 0.00528427632525563
17 0.00490491418167949
18 0.00456201424822211
19 0.00424884120002389
20 0.00396039243787527
21 0.00369284208863974
22 0.00344324111938477
23 0.00320935901254416
24 0.00298960646614432
25 0.00278276856988668
26 0.00258795171976089
27 0.00240454776212573
28 0.00223214621655643
29 0.00207043043337762
30 0.00191916234325618
31 0.00177812040783465
32 0.0016470649279654
33 0.00152573373634368
34 0.00141379085835069
35 0.00131073838565499
36 0.00121612567454576
37 0.0011295914882794
38 0.00105075037572533
39 0.000979124568402767
40 0.000914174073841423
41 0.000855278398375958
42 0.000801867165137082
43 0.000753492815420032
44 0.000709722575265914
45 0.00067018554545939
46 0.000634500989690423
47 0.000602290267124772
48 0.00057319097686559
49 0.000546873547136784
50 0.000523045600857586
51 0.000501448754221201
52 0.000481855182442814
53 0.00046405938337557
54 0.000447875878307968
55 0.000433125096606091
56 0.000419658317696303
57 0.000407362938858569
58 0.000396126735722646
59 0.000385840161470696
60 0.000376402982510626
61 0.000367722008377314
62 0.000359707191819325
63 0.000352281087543815
64 0.000345398264471442
65 0.000339014135533944
66 0.000333081843564287
67 0.000327557732816786
68 0.000322403415339068
69 0.000317584461299703
70 0.000313069933326915
71 0.000308832357404754
72 0.000304847053484991
73 0.000301091087749228
74 0.00029754446586594
75 0.000294189027044922
76 0.00029100853134878
77 0.000287988252239302
78 0.000285114801954478
79 0.000282376393442973
80 0.000279762520221993
81 0.000277263665338978
82 0.000274871097644791
83 0.000272576580755413
84 0.000270372809609398
85 0.000268252973910421
86 0.000266211369307712
87 0.000264242349658161
88 0.000262340443441644
89 0.000260501081356779
90 0.000258719926932827
91 0.000256992963841185
92 0.000255316495895386
93 0.000253687205258757
94 0.000252101948717609
95 0.000250557728577405
96 0.000249051721766591
97 0.000247581483563408
98 0.000246144627453759
99 0.00024473894154653
100 0.000243362708715722
101 0.000242014022660442
102 0.000240691108047031
103 0.000239392393268645
104 0.000238115870160982
105 0.000236860199947841
106 0.000235624262131751
107 0.000234406703384593
108 0.000233205952099524
109 0.000232020756811835
110 0.000230849924264476
111 0.000229692552238703
112 0.000228547491133213
113 0.000227413838729262
114 0.000226290547288954
115 0.00022517672914546
116 0.000224071394768544
117 0.000222973816562444
118 0.00022188329603523
119 0.000220799673115835
120 0.000219722307519987
121 0.000218650646274909
122 0.000217583816265687
123 0.000216521279071458
124 0.000215462350752205
125 0.00021440653654281
126 0.000213353108847514
127 0.000212301674764603
128 0.000211251681321301
129 0.000210202575544827
130 0.000209153979085386
131 0.00020810539717786
132 0.000207056466024369
133 0.00020600673451554
134 0.000204955911613069
135 0.000203903735382482
136 0.000202850962523371
137 0.000201797432964668
138 0.000200742550077848
139 0.000199685848201625
140 0.0001986271381611
141 0.000197566478163935
142 0.000196504319319502
143 0.000195440617972054
144 0.000194375272258185
145 0.00019330826762598
146 0.000192239560419694
147 0.000191169150639325
148 0.00019009702373296
149 0.00018902329611592
150 0.000187947996892035
151 0.000186871184268966
152 0.000185793018317781
153 0.000184713615453802
154 0.000183633121196181
155 0.000182551826583222
156 0.000181469862582162
157 0.0001803874183679
158 0.000179304726771079
159 0.000178221991518512
160 0.000177139532752335
161 0.000176057685166597
162 0.000174976754351519
163 0.000173897104104981
164 0.000172819069121033
165 0.00017174294043798
166 0.000170668979990296
167 0.000169597478816286
168 0.000168528684298508
169 0.000167462771059945
170 0.000166400088346563
171 0.000165340970852412
172 0.000164285476785153
173 0.000163233926286921
174 0.000162186479428783
175 0.000161143252626061
176 0.000160104493261315
177 0.000159070419613272
178 0.000158041118993424
179 0.000157016809680499
180 0.000155997578985989
181 0.000154983688844368
182 0.000153975226567127
183 0.000152972294017673
184 0.000151975051267073
185 0.000150983571074903
186 0.00014999792620074
187 0.000149018160300329
188 0.000148044506204315
189 0.00014707694936078
190 0.000146115562529303
191 0.000145160433021374
192 0.000144211531733163
193 0.00014326891687233
194 0.000142332704854198
195 0.000141402866574936
196 0.000140479416586459
197 0.000139562413096428
198 0.000138651943416335
199 0.000137748007546179
200 0.000136850634589791
201 0.00013595973723568
202 0.000135075344587676
203 0.000134197296574712
204 0.000133325651404448
205 0.000132460278109647
206 0.000131601162138395
207 0.0001307482161792
208 0.00012990141112823
209 0.000129060601466335
210 0.000128225627122447
211 0.000127396444440819
212 0.000126572893350385
213 0.000125754813780077
214 0.000124942103866488
215 0.00012413457443472
216 0.000123332138173282
217 0.000122534751426429
218 0.000121742246847134
219 0.000120954537123907
220 0.000120171447633766
221 0.000119392774649896
222 0.000118618438136764
223 0.000117848190711811
224 0.000117081952339504
225 0.000116319482913241
226 0.000115560651465785
227 0.000114805276098195
228 0.000114053189463448
229 0.000113304180558771
230 0.000112558183900546
231 0.000111815068521537
232 0.00011107465979876
233 0.000110336826764978
234 0.00010960142390104
235 0.000108868240204174
236 0.000108137101051398
237 0.000107407904579304
238 0.000106680461612996
239 0.000105954641185235
240 0.000105230457847938
241 0.000104507897049189
242 0.00010378679871792
243 0.000103067060990725
244 0.000102348560176324
245 0.00010163113620365
246 0.000100914716313127
247 0.000100199162261561
248 9.94845177046955e-05
249 9.87709136097692e-05
250 9.80582772172056e-05
251 9.73465503193438e-05
252 9.66356019489467e-05
253 9.59253738983534e-05
254 9.52158661675639e-05
255 9.4506933237426e-05
256 9.37985751079395e-05
257 9.30907117435709e-05
258 9.23833358683623e-05
259 9.16764329303987e-05
260 9.0970061137341e-05
261 9.02641331776977e-05
262 8.95586927072145e-05
263 8.88537033461034e-05
264 8.8149230577983e-05
265 8.74452307471074e-05
266 8.67417911649682e-05
267 8.60389118315652e-05
268 8.53366364026442e-05
269 8.46350158099085e-05
270 8.39341228129342e-05
271 8.32339865155518e-05
272 8.25347087811679e-05
273 8.18363259895705e-05
274 8.11390054877847e-05
275 8.04427181719802e-05
276 7.97476532170549e-05
277 7.90538397268392e-05
278 7.83614232204854e-05
279 7.7670527389273e-05
280 7.69812104408629e-05
281 7.62936906539835e-05
282 7.56080116843805e-05
283 7.49243263271637e-05
284 7.42426127544604e-05
285 7.35631037969142e-05
286 7.28860395611264e-05
287 7.22114782547578e-05
288 7.15396017767489e-05
289 7.087059202604e-05
290 7.02045654179528e-05
291 6.95417111273855e-05
292 6.88821892254055e-05
293 6.82261234032921e-05
294 6.75737028359435e-05
295 6.69250803184696e-05
296 6.62804450257681e-05
297 6.56399133731611e-05
298 6.50037545710802e-05
299 6.43719831714407e-05
300 6.37448174529709e-05
301 6.31224393146113e-05
302 6.25049433438107e-05
303 6.18925550952554e-05
304 6.12853837083094e-05
305 6.06835455982946e-05
306 6.00872372160666e-05
307 5.94965458731167e-05
308 5.8911584346788e-05
309 5.83325745537877e-05
310 5.77596038056072e-05
311 5.71927303099073e-05
312 5.66320959478617e-05
313 5.60777771170251e-05
314 5.55298793187831e-05
315 5.4988526244415e-05
316 5.44539070688188e-05
317 5.39261200174224e-05
318 5.34051541762892e-05
319 5.28910641151015e-05
320 5.23838389199227e-05
321 5.18835367984138e-05
322 5.13901832164265e-05
323 5.09037417941727e-05
324 5.04242889292073e-05
325 4.99517300340813e-05
326 4.9486097850604e-05
327 4.9027399654733e-05
328 4.85756318084896e-05
329 4.81308125017677e-05
330 4.76928835269064e-05
331 4.72618085041177e-05
332 4.68375801574439e-05
333 4.64200820715632e-05
334 4.60094124719035e-05
335 4.56055204267614e-05
336 4.52083113486879e-05
337 4.48177015641704e-05
338 4.44336146756541e-05
339 4.40559488197323e-05
340 4.36846785305534e-05
341 4.33196873927955e-05
342 4.29609171987977e-05
343 4.26083788624965e-05
344 4.22618722950574e-05
345 4.19214120483957e-05
346 4.1586892621126e-05
347 4.1258168494096e-05
348 4.09351305279415e-05
349 4.06177168770228e-05
350 4.03057565563358e-05
351 3.99991913582198e-05
352 3.96979012293741e-05
353 3.94017806684133e-05
354 3.91108151234221e-05
355 3.88248008675873e-05
356 3.85437124350574e-05
357 3.82674261345528e-05
358 3.79958146368153e-05
359 3.77288124582265e-05
360 3.74663395632524e-05
361 3.72082431567833e-05
362 3.69545159628615e-05
363 3.67050270142499e-05
364 3.64596817234997e-05
365 3.62184218829498e-05
366 3.59811165253632e-05
367 3.57476819772273e-05
368 3.55180600308813e-05
369 3.52921415469609e-05
370 3.50698064721655e-05
371 3.48509820469189e-05
372 3.46356209774967e-05
373 3.44236250384711e-05
374 3.42149360221811e-05
375 3.40094557031989e-05
376 3.38071331498213e-05
377 3.36078810505569e-05
378 3.34116375597659e-05
379 3.32183335558511e-05
380 3.30278853652999e-05
381 3.28402384184301e-05
382 3.26553054037504e-05
383 3.24730208376423e-05
384 3.2293359254254e-05
385 3.21162333420943e-05
386 3.19415594276506e-05
387 3.17693047691137e-05
388 3.15994257107377e-05
389 3.14318276650738e-05
390 3.1266528822016e-05
391 3.11033581965603e-05
392 3.09422721329611e-05
393 3.07832488033455e-05
394 3.06262263620738e-05
395 3.04711938952096e-05
396 3.03180713672191e-05
397 3.0166875149007e-05
398 3.00174760923255e-05
399 2.98699487757403e-05
400 2.97241494990885e-05
401 2.95800746243913e-05
402 2.9437725970638e-05
403 2.92969907604856e-05
404 2.91578890028177e-05
405 2.90203497570474e-05
406 2.88843621092383e-05
407 2.87498678517295e-05
408 2.86168506136164e-05
409 2.84852594631957e-05
410 2.83550707536051e-05
411 2.82262553810142e-05
412 2.80987733276561e-05
413 2.79725936707109e-05
414 2.78476709354436e-05
415 2.772401785478e-05
416 2.7601616238826e-05
417 2.74803478532704e-05
418 2.73602563538589e-05
419 2.72413253696868e-05
420 2.71234857791569e-05
421 2.70067739620572e-05
422 2.68910880549811e-05
423 2.67764589807484e-05
424 2.66628467215924e-05
425 2.6550202164799e-05
426 2.64385671471246e-05
427 2.632787800394e-05
428 2.62181038124254e-05
429 2.6109259124496e-05
430 2.60012766375439e-05
431 2.58941854553996e-05
432 2.57879728451371e-05
433 2.56826060649473e-05
434 2.55780541920103e-05
435 2.54742972174427e-05
436 2.53713296842761e-05
437 2.52691443165531e-05
438 2.51677083724644e-05
439 2.50670309469569e-05
440 2.49670556513593e-05
441 2.48678388743429e-05
442 2.47693169512786e-05
443 2.46714589593466e-05
444 2.45742940023774e-05
445 2.44777638727101e-05
446 2.43819267780054e-05
447 2.42866881308146e-05
448 2.41920970438514e-05
449 2.40981007664232e-05
450 2.40047029365087e-05
451 2.39118962781504e-05
452 2.38196607824648e-05
453 2.37280401051976e-05
454 2.3636926925974e-05
455 2.35464012803277e-05
456 2.34563904086826e-05
457 2.33669252338586e-05
458 2.32779875659617e-05
459 2.31895701290341e-05
460 2.31016147154151e-05
461 2.30141922656912e-05
462 2.29272336582653e-05
463 2.28407770919148e-05
464 2.2754806195735e-05
465 2.26692500291392e-05
466 2.25841995415976e-05
467 2.24995819735341e-05
468 2.24154282477684e-05
469 2.23317310883431e-05
470 2.22485032281838e-05
471 2.21657028305344e-05
472 2.20833226194372e-05
473 2.20013935177121e-05
474 2.19198827835498e-05
475 2.18387904169504e-05
476 2.17581073229667e-05
477 2.1677811673726e-05
478 2.15979453059845e-05
479 2.15184809349012e-05
480 2.1439409465529e-05
481 2.13607345358469e-05
482 2.12824670597911e-05
483 2.12045561056584e-05
484 2.11270289582899e-05
485 2.10499165405054e-05
486 2.09731842915062e-05
487 2.08968103834195e-05
488 2.08208275580546e-05
489 2.07451848837081e-05
490 2.0669922378147e-05
491 2.0595016394509e-05
492 2.05204905796563e-05
493 2.04463121917797e-05
494 2.03725066967309e-05
495 2.02990649995627e-05
496 2.0225945263519e-05
497 2.015320751525e-05
498 2.00808026420418e-05
499 2.00087579287356e-05
500 1.99370424525114e-05
501 1.98657180590089e-05
502 1.97946956177475e-05
503 1.97240678971866e-05
504 1.9653765775729e-05
505 1.9583800167311e-05
506 1.95141728909221e-05
507 1.94448930415092e-05
508 1.93759296962526e-05
509 1.93073174159508e-05
510 1.92390161828371e-05
511 1.91710532817524e-05
512 1.91034396266332e-05
513 1.9036155208596e-05
514 1.89691963896621e-05
515 1.89025540748844e-05
516 1.88362519111251e-05
517 1.87702735274797e-05
518 1.87046643986832e-05
519 1.86393463081913e-05
520 1.85743319889298e-05
521 1.85096505447291e-05
522 1.84453037945786e-05
523 1.83812226168811e-05
524 1.83174852281809e-05
525 1.82540443347534e-05
526 1.81909199454822e-05
527 1.81280884135049e-05
528 1.80655715666944e-05
529 1.80033675860614e-05
530 1.79414291778812e-05
531 1.78797981789103e-05
532 1.78184891410638e-05
533 1.77574875124265e-05
534 1.76967714651255e-05
535 1.76363228092669e-05
536 1.75762052094797e-05
537 1.75163822859759e-05
538 1.74568322108826e-05
539 1.73976040969137e-05
540 1.73386524693342e-05
541 1.72800246218685e-05
542 1.72216568898875e-05
543 1.71635892911581e-05
544 1.71058254636591e-05
545 1.70483308465919e-05
546 1.69911327247974e-05
547 1.69342456501909e-05
548 1.68776277860161e-05
549 1.68213100550929e-05
550 1.67652724485379e-05
551 1.67095113283722e-05
552 1.66540503414581e-05
553 1.65988476510393e-05
554 1.65439305419568e-05
555 1.648933175602e-05
556 1.64349567057798e-05
557 1.63809090736322e-05
558 1.63271179189906e-05
559 1.62736196216429e-05
560 1.62203723448329e-05
561 1.61674197443062e-05
562 1.61147381732007e-05
563 1.60623330884846e-05
564 1.60101972142002e-05
565 1.59583214554004e-05
566 1.59067185450112e-05
567 1.5855408491916e-05
568 1.58043440023903e-05
569 1.57535505422857e-05
570 1.57030008267611e-05
571 1.56527330545941e-05
572 1.56027399498271e-05
573 1.55529851326719e-05
574 1.55035158968531e-05
575 1.54542740347097e-05
576 1.54053232108708e-05
577 1.53566088556545e-05
578 1.53081400640076e-05
579 1.52599186549196e-05
580 1.52119737322209e-05
581 1.51642570926924e-05
582 1.51168014781433e-05
583 1.50696305354359e-05
584 1.50226533150999e-05
585 1.49759680425632e-05
586 1.49295174196595e-05
587 1.48833005368942e-05
588 1.48373328556772e-05
589 1.47916134665138e-05
590 1.47461469168775e-05
591 1.47008968269802e-05
592 1.46559232234722e-05
593 1.46111542562721e-05
594 1.45665999298217e-05
595 1.45221974889864e-05
596 1.44778632602538e-05
597 1.44337072924827e-05
598 1.43897532325354e-05
599 1.43460238177795e-05
600 1.43026227306109e-05
601 1.42595208671992e-05
602 1.42167327794596e-05
603 1.41742766572861e-05
604 1.41320851980709e-05
605 1.40901438498986e-05
606 1.40484416988329e-05
607 1.4006959645485e-05
608 1.39657340696431e-05
609 1.39247249535401e-05
610 1.3883926840208e-05
611 1.38433852043818e-05
612 1.38030909511144e-05
613 1.37630486278795e-05
614 1.37232482302352e-05
615 1.36837243189802e-05
616 1.36444032250438e-05
617 1.36053740789066e-05
618 1.35665513880667e-05
619 1.35279442474712e-05
620 1.34895963128656e-05
621 1.34514148157905e-05
622 1.34135134430835e-05
623 1.33758167066844e-05
624 1.33383628053707e-05
625 1.33011353682377e-05
626 1.32641189338756e-05
627 1.32273671624716e-05
628 1.31908291223226e-05
629 1.31545157273649e-05
630 1.3118437891535e-05
631 1.30825574160554e-05
632 1.30469479699968e-05
633 1.30115386127727e-05
634 1.29763529912452e-05
635 1.29413692775415e-05
636 1.29066256704391e-05
637 1.28721148939803e-05
638 1.28377878354513e-05
639 1.28036881505977e-05
640 1.27697758216527e-05
641 1.27361417980865e-05
642 1.27026705740718e-05
643 1.26694449136266e-05
644 1.26364302559523e-05
645 1.26036466099322e-05
646 1.25710321299266e-05
647 1.25386532090488e-05
648 1.250649074791e-05
649 1.2474548384489e-05
650 1.24427660921356e-05
651 1.24112420962774e-05
652 1.23798890854232e-05
653 1.23487634482444e-05
654 1.23178242574795e-05
655 1.22871260828106e-05
656 1.22565843412303e-05
657 1.22262508739368e-05
658 1.21961593322339e-05
659 1.21662396850297e-05
660 1.21365292216069e-05
661 1.21070070235874e-05
662 1.2077706742275e-05
663 1.2048594726366e-05
664 1.20196973512066e-05
665 1.1990968232567e-05
666 1.19624610306346e-05
667 1.19341721074306e-05
668 1.19060405268101e-05
669 1.18781117635081e-05
670 1.18503594421782e-05
671 1.18228344945237e-05
672 1.17954650704633e-05
673 1.17683075586683e-05
674 1.17413155749091e-05
675 1.17145482363412e-05
676 1.16879818961024e-05
677 1.16615619845106e-05
678 1.1635370356089e-05
679 1.16093478936818e-05
680 1.15835309770773e-05
681 1.15579005068867e-05
682 1.15324637590675e-05
683 1.15071916297893e-05
684 1.14821405077237e-05
685 1.14572012535064e-05
686 1.14324566311552e-05
687 1.14078893602709e-05
688 1.13834812509594e-05
689 1.13592486741254e-05
690 1.13351852633059e-05
691 1.1311293746985e-05
692 1.12875804916257e-05
693 1.12640245788498e-05
694 1.12406778498553e-05
695 1.12174820969813e-05
696 1.11944891614257e-05
697 1.11716508399695e-05
698 1.11490162453265e-05
699 1.11265371742775e-05
700 1.1104254554084e-05
701 1.10821583803045e-05
702 1.10602450149599e-05
703 1.10384526124108e-05
704 1.10168666651589e-05
705 1.09953971332288e-05
706 1.09740421976312e-05
707 1.09528928078362e-05
708 1.09317888927762e-05
709 1.09108514152467e-05
710 1.0889960321947e-05
711 1.08691956484108e-05
712 1.08485246528289e-05
713 1.08279555206536e-05
714 1.08075018943055e-05
715 1.07871474028798e-05
716 1.07669957287726e-05
717 1.07469468275667e-05
718 1.0727178050729e-05
719 1.0707602086768e-05
720 1.06883298940375e-05
721 1.06693787529366e-05
722 1.06508305179887e-05
723 1.06326715467731e-05
724 1.0615001883707e-05
725 1.05978051578859e-05
726 1.05810877357726e-05
727 1.05648596218089e-05
728 1.05490289570298e-05
729 1.05335375337745e-05
730 1.05182525658165e-05
731 1.05029703263426e-05
732 1.04874770840979e-05
733 1.0471478162799e-05
734 1.04546616057632e-05
735 1.04366990854032e-05
736 1.04173095678561e-05
737 1.03962638604571e-05
738 1.03734919321141e-05
739 1.0349143849453e-05
740 1.0323739843443e-05
741 1.02980711744749e-05
742 1.02730809885543e-05
743 1.02495523606194e-05
744 1.02271378636942e-05
745 1.02031390269985e-05
746 1.01720597740496e-05
747 1.01275963970693e-05
748 1.00690513136215e-05
749 1.00105125966365e-05
750 9.98590712697478e-06
751 1.00463012131513e-05
752 1.02535614132648e-05
753 1.06644447441795e-05
754 1.12588932097424e-05
755 1.17735480671399e-05
756 1.17433037303272e-05
757 1.1184296454303e-05
758 1.07371188278194e-05
759 1.06786892501987e-05
760 1.0762273632281e-05
761 1.06486377262627e-05
762 1.02414396678796e-05
763 9.85444603429642e-06
764 9.7646116046235e-06
765 9.89087402558653e-06
766 1.00378147180891e-05
767 1.01068053481868e-05
768 1.01021596492501e-05
769 1.00638035291922e-05
770 1.00174129329389e-05
771 9.96930612018332e-06
772 9.92090463114437e-06
773 9.87651492323494e-06
774 9.84229973255424e-06
775 9.82215806288878e-06
776 9.81536777544534e-06
777 9.81730318017071e-06
778 9.82201345323119e-06
779 9.82454275799682e-06
780 9.82205074251397e-06
781 9.81386347120861e-06
782 9.80074128165143e-06
783 9.78399748419179e-06
784 9.76509909378365e-06
785 9.74540580500616e-06
786 9.72605630522594e-06
787 9.70805376709905e-06
788 9.69224220170872e-06
789 9.67907817539526e-06
790 9.66876723396126e-06
791 9.66098014032468e-06
792 9.65502385952277e-06
793 9.6499079518253e-06
794 9.64468563324772e-06
795 9.63837737799622e-06
796 9.63032562140143e-06
797 9.62014019023627e-06
798 9.60779743763851e-06
799 9.59346198214917e-06
800 9.57758948061382e-06
801 9.56052372202976e-06
802 9.54282495513326e-06
803 9.52498612605268e-06
804 9.50754747464089e-06
805 9.49104287428781e-06
806 9.47621447267011e-06
807 9.4637980510015e-06
808 9.45453120948514e-06
809 9.44908970268443e-06
810 9.44786006584764e-06
811 9.45078681979794e-06
812 9.45726060308516e-06
813 9.46611180552281e-06
814 9.47551689023385e-06
815 9.48321849136846e-06
816 9.48684828472324e-06
817 9.48427805269603e-06
818 9.47416265262291e-06
819 9.4564111350337e-06
820 9.43270151765319e-06
821 9.40669178817188e-06
822 9.38419179874472e-06
823 9.37267031986266e-06
824 9.3781245595892e-06
825 9.39881374506513e-06
826 9.41409325605491e-06
827 9.38171342568239e-06
828 9.26745997276157e-06
829 9.10295329958899e-06
830 8.99943370313849e-06
831 9.07816865947098e-06
832 9.39976234803908e-06
833 9.92565674096113e-06
834 1.04446762634325e-05
835 1.05894187072408e-05
836 1.02473986771656e-05
837 9.81633183982922e-06
838 9.62511421676027e-06
839 9.60711258812808e-06
840 9.57071461016312e-06
841 9.40528298087884e-06
842 9.16299632081063e-06
843 8.99795395525871e-06
844 8.98952202987857e-06
845 9.08985293790465e-06
846 9.20759703149088e-06
847 9.28048848436447e-06
848 9.29455472942209e-06
849 9.26883058127714e-06
850 9.22811796044698e-06
851 9.18625210033497e-06
852 9.14611609914573e-06
853 9.10659764485899e-06
854 9.06813602341572e-06
855 9.03441559785279e-06
856 9.01088515092852e-06
857 9.00167378858896e-06
858 9.00717623153469e-06
859 9.0235980678699e-06
860 9.04469925444573e-06
861 9.06364857655717e-06
862 9.07539651961997e-06
863 9.07731646293541e-06
864 9.06961122382199e-06
865 9.05433535081102e-06
866 9.0340627139085e-06
867 9.01118892215891e-06
868 8.98713733477052e-06
869 8.96247456694255e-06
870 8.93724973138887e-06
871 8.91184754436836e-06
872 8.88759041117737e-06
873 8.86730140337022e-06
874 8.85512145032408e-06
875 8.85532881511608e-06
876 8.87122769199777e-06
877 8.90378760232124e-06
878 8.95087578101084e-06
879 9.00683789950563e-06
880 9.06276090972824e-06
881 9.10760172700975e-06
882 9.13078201847384e-06
883 9.12597715796437e-06
884 9.09459959075321e-06
885 9.04684748093132e-06
886 9.00039776752237e-06
887 8.97804147825809e-06
888 9.00299073691713e-06
889 9.08465244719991e-06
890 9.18404020922026e-06
891 9.18498881219421e-06
892 8.97709742275765e-06
893 8.65194397192681e-06
894 8.48529271024745e-06
895 8.64699813973857e-06
896 9.07784124137834e-06
897 9.55287123360904e-06
898 9.77151739789406e-06
899 9.60876877798e-06
900 9.29166617424926e-06
901 9.08472247829195e-06
902 9.00916620594217e-06
903 8.96302117325831e-06
904 8.87355599843431e-06
905 8.74169745657127e-06
906 8.62482465890935e-06
907 8.57840768730966e-06
908 8.61048010847298e-06
909 8.68777169671375e-06
910 8.76683316164417e-06
911 8.81765299709514e-06
912 8.83145730767865e-06
913 8.8164524640888e-06
914 8.78674291016068e-06
915 8.75326168170432e-06
916 8.72054897627095e-06
917 8.68869756232016e-06
918 8.65638958202908e-06
919 8.62387241795659e-06
920 8.5942756413715e-06
921 8.57297345646657e-06
922 8.56521000969224e-06
923 8.57385657582199e-06
924 8.59772080730181e-06
925 8.63173772813752e-06
926 8.66839764057659e-06
927 8.69953601068119e-06
928 8.71849806571845e-06
929 8.72200325829908e-06
930 8.71064003149513e-06
931 8.68836832523812e-06
932 8.66112350195181e-06
933 8.6347545220633e-06
934 8.61344415170606e-06
935 8.59790361573687e-06
936 8.58373095979914e-06
937 8.56136375659844e-06
938 8.52011635288363e-06
939 8.45743852551095e-06
940 8.38867981656222e-06
941 8.34652746561915e-06
942 8.36896742839599e-06
943 8.48439776746091e-06
944 8.70168059918797e-06
945 8.99650603969349e-06
946 9.2886075435672e-06
947 9.44475141295698e-06
948 9.37160348257748e-06
949 9.13716212380677e-06
950 8.92044135980541e-06
951 8.84036307979841e-06
952 8.89823877514573e-06
953 8.99862061487511e-06
954 8.96839628694579e-06
955 8.70022449817043e-06
956 8.35169157653581e-06
957 8.19678007246694e-06
958 8.3051054389216e-06
959 8.54394693305949e-06
960 8.74755096447188e-06
961 8.8199940364575e-06
962 8.77102320373524e-06
963 8.67620656208601e-06
964 8.59629926708294e-06
965 8.54336576594505e-06
966 8.50152082421118e-06
967 8.45375598146347e-06
968 8.39584208733868e-06
969 8.33878493722295e-06
970 8.30083627079148e-06
971 8.29521923151333e-06
972 8.32236582937185e-06
973 8.37108927953523e-06
974 8.42472127260407e-06
975 8.46778584673302e-06
976 8.49074694997398e-06
977 8.49191019369755e-06
978 8.4761368270847e-06
979 8.45108752400847e-06
980 8.42375266074669e-06
981 8.39804579300107e-06
982 8.3743598224828e-06
983 8.34995353216073e-06
984 8.32079877000069e-06
985 8.28479915071512e-06
986 8.24489779915893e-06
987 8.2104052125942e-06
988 8.19482738734223e-06
989 8.21119283500593e-06
990 8.26710493129212e-06
991 8.36196704767644e-06
992 8.48514991957927e-06
993 8.61456373968394e-06
994 8.71841621119529e-06
995 8.76500416779891e-06
996 8.74015677254647e-06
997 8.66037589730695e-06
998 8.56692349771038e-06
999 8.50687320053112e-06
1000 8.51802815304836e-06
1001 8.61447188071907e-06
1002 8.74907345860265e-06
1003 8.775412425166e-06
1004 8.54526660987176e-06
1005 8.15995281300275e-06
1006 7.94276002125116e-06
1007 8.06706339062657e-06
1008 8.43413636175683e-06
1009 8.81323649082333e-06
1010 8.97792961040977e-06
1011 8.87039004737744e-06
1012 8.65136371430708e-06
1013 8.49243133416167e-06
1014 8.42134340928169e-06
1015 8.38153209770098e-06
1016 8.32149726193165e-06
1017 8.22985657578101e-06
1018 8.13469978311332e-06
1019 8.07714332040632e-06
1020 8.07855758466758e-06
1021 8.12999587651575e-06
1022 8.20390232547652e-06
1023 8.27104577183491e-06
1024 8.31208490126301e-06
1025 8.32196928968187e-06
1026 8.30761109682498e-06
1027 8.28068277769489e-06
1028 8.25092138256878e-06
1029 8.22306719783228e-06
1030 8.1969665188808e-06
1031 8.169881766662e-06
1032 8.13887072581565e-06
1033 8.10400433692848e-06
1034 8.06994376034709e-06
1035 8.04538831289392e-06
1036 8.03987677500118e-06
1037 8.06018942967057e-06
1038 8.10749679658329e-06
1039 8.17668023955775e-06
1040 8.25658389658201e-06
1041 8.33165540825576e-06
1042 8.38529012980871e-06
1043 8.40520988276694e-06
1044 8.38891082821647e-06
1045 8.34581805975176e-06
1046 8.29454165796051e-06
1047 8.25741972221294e-06
1048 8.25533061288297e-06
1049 8.30078806757228e-06
1050 8.3808199633495e-06
1051 8.43255838844925e-06
1052 8.35547689348459e-06
1053 8.1176576713915e-06
1054 7.85794054536382e-06
1055 7.78427238401491e-06
1056 7.98742712504463e-06
1057 8.40255142975366e-06
1058 8.84131804923527e-06
1059 9.04754051589407e-06
1060 8.90622050064849e-06
1061 8.60554519022116e-06
1062 8.39522726892028e-06
1063 8.32448404253228e-06
1064 8.30703265819466e-06
1065 8.25320421427023e-06
1066 8.13113547337707e-06
1067 7.98346536612371e-06
1068 7.88652323535644e-06
1069 7.88265424489509e-06
1070 7.95703817857429e-06
1071 8.06288880994543e-06
1072 8.15340899862349e-06
1073 8.20098830445204e-06
1074 8.20319110061973e-06
1075 8.17575437395135e-06
1076 8.13817223388469e-06
1077 8.10302663012408e-06
1078 8.07345077191712e-06
1079 8.04591581982095e-06
1080 8.01483929535607e-06
1081 7.9772535173106e-06
1082 7.93584240454948e-06
1083 7.89972364145797e-06
1084 7.880812518124e-06
1085 7.88880697655259e-06
1086 7.92690752859926e-06
1087 7.99059944256442e-06
1088 8.06852131063351e-06
1089 8.14467057352886e-06
1090 8.20211789687164e-06
1091 8.22819220047677e-06
1092 8.21970570541453e-06
1093 8.18462649476714e-06
1094 8.13909446151229e-06
1095 8.10207438917132e-06
1096 8.09003722679336e-06
1097 8.11230711406097e-06
1098 8.16101328382501e-06
1099 8.1974785643979e-06
1100 8.15550720290048e-06
1101 7.99549434304936e-06
1102 7.7797012636438e-06
1103 7.65344793762779e-06
1104 7.73044030211167e-06
1105 8.02454087533988e-06
1106 8.4492039604811e-06
1107 8.81575124367373e-06
1108 8.90615228854585e-06
1109 8.69395444169641e-06
1110 8.40599386719987e-06
1111 8.24240578367608e-06
1112 8.20965851744404e-06
1113 8.21288995211944e-06
1114 8.15741896076361e-06
1115 8.01011537987506e-06
1116 7.83054838393582e-06
1117 7.71900158724748e-06
1118 7.72582552599488e-06
1119 7.8264865805977e-06
1120 7.95916002971353e-06
1121 8.06497155281249e-06
1122 8.11191875982331e-06
1123 8.10230994829908e-06
1124 8.06123716756701e-06
1125 8.01518217485864e-06
1126 7.9779192674323e-06
1127 7.94954303273698e-06
1128 7.92240916780429e-06
1129 7.88773286330979e-06
1130 7.84193525760202e-06
1131 7.7907807281008e-06
1132 7.74847376305843e-06
1133 7.7313770816545e-06
1134 7.75010539655341e-06
1135 7.80537538958015e-06
1136 7.88774013926741e-06
1137 7.9798173828749e-06
1138 8.06034586275928e-06
1139 8.11011432233499e-06
1140 8.11927020549774e-06
1141 8.09203629614785e-06
1142 8.04496357886819e-06
1143 7.99963163444772e-06
1144 7.97529719420709e-06
1145 7.98364544607466e-06
1146 8.0221934695146e-06
1147 8.06309708423214e-06
1148 8.0495328802499e-06
1149 7.93012804933824e-06
1150 7.72817838878836e-06
1151 7.56003692004015e-06
1152 7.55022892917623e-06
1153 7.74633826949866e-06
1154 8.10513211035868e-06
1155 8.49459775054129e-06
1156 8.71528118295828e-06
1157 8.64322646521032e-06
1158 8.38854612084106e-06
1159 8.17250929685542e-06
1160 8.08961340226233e-06
1161 8.09258926892653e-06
1162 8.0872887338046e-06
1163 7.99840927356854e-06
1164 7.82542701927014e-06
1165 7.65326876717154e-06
1166 7.57646375859622e-06
1167 7.62162471801275e-06
1168 7.74781437939964e-06
1169 7.88847228250233e-06
1170 7.9870524132275e-06
1171 8.0184981925413e-06
1172 7.99402459961129e-06
1173 7.94482912169769e-06
1174 7.89797832112527e-06
1175 7.86436157795833e-06
1176 7.84024723543553e-06
1177 7.81448943598662e-06
1178 7.77616696723271e-06
1179 7.72202474763617e-06
1180 7.66139874031069e-06
1181 7.61374576541129e-06
1182 7.59932026994647e-06
1183 7.62951458455063e-06
1184 7.70273527450627e-06
1185 7.80518621468218e-06
1186 7.91378170106327e-06
1187 8.0014660852612e-06
1188 8.04595310910372e-06
1189 8.04012688604416e-06
1190 7.99630743131274e-06
1191 7.93983144831145e-06
1192 7.89740442996845e-06
1193 7.88832403486595e-06
1194 7.91855109127937e-06
1195 7.97153370513115e-06
1196 7.99739063950256e-06
1197 7.92726859799586e-06
1198 7.73934516473673e-06
1199 7.51988272895687e-06
1200 7.41353005651035e-06
1201 7.50945036998019e-06
1202 7.79680158302654e-06
1203 8.17637737782206e-06
1204 8.47443516249768e-06
1205 8.52737412060378e-06
1206 8.34673664940055e-06
1207 8.11911013443023e-06
1208 7.99086774350144e-06
1209 7.96618041931652e-06
1210 7.97224311099853e-06
1211 7.92866194387898e-06
1212 7.79849688115064e-06
1213 7.62395711717545e-06
1214 7.49607534089591e-06
1215 7.47643844078993e-06
1216 7.55987457523588e-06
1217 7.6959267971688e-06
1218 7.82408824306913e-06
1219 7.90002468420425e-06
1220 7.91156435298035e-06
1221 7.87766020948766e-06
1222 7.82919505581958e-06
1223 7.78849243943114e-06
1224 7.76188517193077e-06
1225 7.74270665715449e-06
1226 7.71800750953844e-06
1227 7.67598157835891e-06
1228 7.61450519348728e-06
1229 7.54618349674274e-06
1230 7.49465334592969e-06
1231 7.48318097976153e-06
1232 7.52411006033071e-06
1233 7.61470755605842e-06
1234 7.73796909925295e-06
1235 7.86541295383358e-06
1236 7.96321910456754e-06
1237 8.00462930783397e-06
1238 7.98475593910553e-06
1239 7.92489845480304e-06
1240 7.86038799560629e-06
1241 7.82352253736462e-06
1242 7.83316863817163e-06
1243 7.88795205153292e-06
1244 7.95303458289709e-06
1245 7.95358027971815e-06
1246 7.81662765803048e-06
1247 7.56705367166433e-06
1248 7.35173671273515e-06
1249 7.31891805116902e-06
1250 7.50809613236925e-06
1251 7.85000611358555e-06
1252 8.19511660665739e-06
1253 8.36454910313478e-06
1254 8.28656084195245e-06
1255 8.08197091828333e-06
1256 7.9206683949451e-06
1257 7.85975589678856e-06
1258 7.85588235885371e-06
1259 7.83714676799718e-06
1260 7.75119406171143e-06
1261 7.60297007218469e-06
1262 7.45682928027236e-06
1263 7.38437893232913e-06
1264 7.41168605600251e-06
1265 7.51456536818296e-06
1266 7.64361448091222e-06
1267 7.74960062699392e-06
1268 7.80226400820538e-06
1269 7.80018581281183e-06
1270 7.76491560827708e-06
1271 7.72297880757833e-06
1272 7.69097823649645e-06
1273 7.67162418924272e-06
1274 7.65687400416937e-06
1275 7.63306525186636e-06
1276 7.58823443902656e-06
1277 7.52091227695928e-06
1278 7.44597309676465e-06
1279 7.39007509764633e-06
1280 7.37924256100086e-06
1281 7.42759675631532e-06
1282 7.53301083022961e-06
1283 7.67709934734739e-06
1284 7.82642655394739e-06
1285 7.93892741057789e-06
1286 7.98072414909257e-06
1287 7.94796505942941e-06
1288 7.872245987528e-06
1289 7.80079517426202e-06
1290 7.77192326495424e-06
1291 7.80311256676214e-06
1292 7.88263241702225e-06
1293 7.95331561675994e-06
1294 7.91517231846228e-06
1295 7.70550832385197e-06
1296 7.4098370532738e-06
1297 7.22060485713882e-06
1298 7.26236885384424e-06
1299 7.51794323150534e-06
1300 7.86745658842847e-06
1301 8.13660790299764e-06
1302 8.18937132862629e-06
1303 8.05239324108697e-06
1304 7.88035958976252e-06
1305 7.78155572334072e-06
1306 7.75519674789393e-06
1307 7.74613636167487e-06
1308 7.69789312471403e-06
1309 7.58791702537565e-06
1310 7.44670478525222e-06
1311 7.33712158762501e-06
1312 7.30621513866936e-06
1313 7.35851881472627e-06
1314 7.46412206353853e-06
1315 7.57954785512993e-06
1316 7.6660708145937e-06
1317 7.70369024394313e-06
1318 7.69654980103951e-06
1319 7.6651003837469e-06
1320 7.6311380325933e-06
1321 7.60721832193667e-06
1322 7.59430213292944e-06
1323 7.58395708544413e-06
1324 7.56219242248335e-06
1325 7.51631932871533e-06
1326 7.44453564038849e-06
1327 7.36240053811343e-06
1328 7.29889870854095e-06
1329 7.2834864113247e-06
1330 7.33366368876887e-06
1331 7.44984708944685e-06
1332 7.61390538173146e-06
1333 7.78819139668485e-06
1334 7.92059472587425e-06
1335 7.96679432824021e-06
1336 7.92252285464201e-06
1337 7.83152972871903e-06
1338 7.75483385950793e-06
1339 7.73627562011825e-06
1340 7.7895501817693e-06
1341 7.88918077887502e-06
1342 7.95312917034607e-06
1343 7.86391319707036e-06
1344 7.58742680773139e-06
1345 7.27094447938725e-06
1346 7.12336577635142e-06
1347 7.2260540946445e-06
1348 7.51171091906144e-06
1349 7.83044015406631e-06
1350 8.01699297880987e-06
1351 7.99887311586645e-06
1352 7.86010787123814e-06
1353 7.73485317040468e-06
1354 7.67754590924596e-06
1355 7.66343782743206e-06
1356 7.64231845096219e-06
1357 7.57516090743593e-06
1358 7.45907937016455e-06
1359 7.33281785869622e-06
1360 7.24957999409526e-06
1361 7.24039045962854e-06
1362 7.30102192392224e-06
1363 7.40232280804776e-06
1364 7.50655499359709e-06
1365 7.58192163630156e-06
1366 7.61356568546034e-06
1367 7.6071542025602e-06
1368 7.58112446419545e-06
1369 7.5542179729382e-06
1370 7.53723725210875e-06
1371 7.53076665205299e-06
1372 7.52639380152686e-06
1373 7.5094826570421e-06
1374 7.46533032724983e-06
1375 7.38967946745106e-06
1376 7.2974648901436e-06
1377 7.22038112144219e-06
1378 7.19301533536054e-06
1379 7.23837683835882e-06
1380 7.36104084353428e-06
1381 7.54438451622264e-06
1382 7.74654199631186e-06
1383 7.90303784015123e-06
1384 7.95547293819254e-06
1385 7.89880232332507e-06
1386 7.79288711783011e-06
1387 7.71485156292329e-06
1388 7.71103805163875e-06
1389 7.78578214521986e-06
1390 7.89476871432271e-06
1391 7.93226990936091e-06
1392 7.78076173446607e-06
1393 7.454851129296e-06
1394 7.14927455192083e-06
1395 7.054330126266e-06
1396 7.19946365279611e-06
1397 7.48326101529528e-06
1398 7.75072112446651e-06
1399 7.86934560892405e-06
1400 7.82153801992536e-06
1401 7.70673523220466e-06
1402 7.62295258027734e-06
1403 7.59158319851849e-06
1404 7.58206442696974e-06
1405 7.55290830056765e-06
1406 7.47756712371483e-06
1407 7.3628452810226e-06
1408 7.24770416127285e-06
1409 7.17751800038968e-06
1410 7.17589819032582e-06
1411 7.23672110325424e-06
1412 7.33346178094507e-06
1413 7.43227792554535e-06
1414 7.50434674046119e-06
1415 7.53603353587096e-06
1416 7.53250924390159e-06
1417 7.51109564589569e-06
1418 7.48994534660596e-06
1419 7.47980584492325e-06
1420 7.48181037124596e-06
1421 7.4876274993585e-06
1422 7.48059937905055e-06
1423 7.44154294807231e-06
1424 7.36126276024152e-06
1425 7.25312884242157e-06
1426 7.15377018423169e-06
1427 7.10680842530564e-06
1428 7.14354109732085e-06
1429 7.27341466699727e-06
1430 7.47991043681395e-06
1431 7.71314444136806e-06
1432 7.89075511420378e-06
1433 7.93806793808471e-06
1434 7.85769771027844e-06
1435 7.7375852924888e-06
1436 7.67045639804564e-06
1437 7.69401776778977e-06
1438 7.79038600740023e-06
1439 7.88678335084114e-06
1440 7.86108921602136e-06
1441 7.63112257118337e-06
1442 7.28622262613499e-06
1443 7.03806335877744e-06
1444 7.01620274412562e-06
1445 7.19339323040913e-06
1446 7.45155739423353e-06
1447 7.65339154895628e-06
1448 7.71401209931355e-06
1449 7.65713411965407e-06
1450 7.57400948714348e-06
1451 7.52630194256199e-06
1452 7.51522520658909e-06
1453 7.51032212065184e-06
1454 7.47763033359661e-06
1455 7.3988308031403e-06
1456 7.28557233742322e-06
1457 7.17602961231023e-06
1458 7.11131224306882e-06
1459 7.11208258508123e-06
1460 7.17265811545076e-06
1461 7.26866983313812e-06
1462 7.36783204047242e-06
1463 7.44120779927471e-06
1464 7.47403919376666e-06
1465 7.47116337151965e-06
1466 7.45112720323959e-06
1467 7.43407599657075e-06
1468 7.43266991776181e-06
1469 7.44897261029109e-06
1470 7.47342073736945e-06
1471 7.48444199416554e-06
1472 7.45351962905261e-06
1473 7.36234687792603e-06
1474 7.22479808246135e-06
1475 7.09098276274744e-06
1476 7.02229408489075e-06
1477 7.06095670466311e-06
1478 7.21707192496979e-06
1479 7.46361274650553e-06
1480 7.72620842326432e-06
1481 7.89188561611809e-06
1482 7.8849652709323e-06
1483 7.75698572397232e-06
1484 7.64076867199037e-06
1485 7.61973524276982e-06
1486 7.69093003327725e-06
1487 7.79273523221491e-06
1488 7.81934704718878e-06
1489 7.67094206821639e-06
1490 7.36863876227289e-06
1491 7.07837943991763e-06
1492 6.95767312208773e-06
1493 7.03392697687377e-06
1494 7.23075709174736e-06
1495 7.43482542020502e-06
1496 7.5510106398724e-06
1497 7.55489145376487e-06
1498 7.49917671782896e-06
1499 7.4511540333333e-06
1500 7.43733926356072e-06
1501 7.44483440939803e-06
1502 7.44424505683128e-06
1503 7.40719997338601e-06
1504 7.32248190615792e-06
1505 7.20759408068261e-06
1506 7.10231552147889e-06
1507 7.04551712260582e-06
1508 7.05521460986347e-06
1509 7.12465862306999e-06
1510 7.22895038052229e-06
1511 7.33400338503998e-06
1512 7.40799578125007e-06
1513 7.43493274057982e-06
1514 7.42274505682872e-06
1515 7.39652796255541e-06
1516 7.38257949706167e-06
1517 7.39684628570103e-06
1518 7.4412282629055e-06
1519 7.50059234633227e-06
1520 7.53857830204652e-06
1521 7.5053271757497e-06
1522 7.37160871722153e-06
1523 7.17157263352419e-06
1524 6.99764996170416e-06
1525 6.9404986788868e-06
1526 7.0406608756457e-06
1527 7.2806901698641e-06
1528 7.58224268793128e-06
1529 7.80752998252865e-06
1530 7.83671293902444e-06
1531 7.70216411183355e-06
1532 7.56572853788384e-06
1533 7.5359803304309e-06
1534 7.60265857024933e-06
1535 7.69079360907199e-06
1536 7.70015412854264e-06
1537 7.55607698010863e-06
1538 7.29131988919107e-06
1539 7.04393823980354e-06
1540 6.93554284225684e-06
1541 6.98676967658685e-06
1542 7.1387685238733e-06
1543 7.30412739358144e-06
1544 7.40822224543081e-06
1545 7.42624388294644e-06
1546 7.39114102543681e-06
1547 7.35509547666879e-06
1548 7.34679542802041e-06
1549 7.36328183847945e-06
1550 7.38260587240802e-06
1551 7.37596656108508e-06
1552 7.32136777514825e-06
1553 7.21968262951123e-06
1554 7.10121503288974e-06
1555 7.01084036336397e-06
1556 6.98321309755556e-06
1557 7.02834131516283e-06
1558 7.13197869117721e-06
1559 7.2610951065144e-06
1560 7.37209802537109e-06
1561 7.42788461138844e-06
1562 7.42017618904356e-06
1563 7.37671007300378e-06
1564 7.34131936042104e-06
1565 7.34787954570493e-06
1566 7.40974837754038e-06
1567 7.51608013160876e-06
1568 7.62078752813977e-06
1569 7.63911702961195e-06
1570 7.49587525206152e-06
1571 7.22075810699607e-06
1572 6.96419465384679e-06
1573 6.87538067722926e-06
1574 7.00095870342921e-06
1575 7.28525037629879e-06
1576 7.59086287871469e-06
1577 7.74330783315236e-06
1578 7.67414439906133e-06
1579 7.51332618165179e-06
1580 7.42808333598077e-06
1581 7.45396891943528e-06
1582 7.52771029510768e-06
1583 7.56095550968894e-06
1584 7.48349020796013e-06
1585 7.29444673197577e-06
1586 7.07952585798921e-06
1587 6.94347772878245e-06
1588 6.93187166689313e-06
1589 7.02125043972046e-06
1590 7.15194937583874e-06
1591 7.26058669897611e-06
1592 7.30855981601053e-06
1593 7.29944395061466e-06
1594 7.26816051610513e-06
1595 7.24948313290952e-06
1596 7.25759628039668e-06
1597 7.28609848010819e-06
1598 7.31444060875219e-06
1599 7.3146507020283e-06
1600 7.26365442460519e-06
1601 7.16152635504841e-06
1602 7.04003105056472e-06
1603 6.94883055984974e-06
1604 6.92868661644752e-06
1605 6.99483962307568e-06
1606 7.13391727913404e-06
1607 7.30403780835331e-06
1608 7.44073213354568e-06
1609 7.48656520954682e-06
1610 7.43775854061823e-06
1611 7.35412822905346e-06
1612 7.31006548448931e-06
1613 7.34817649572506e-06
1614 7.47149397284375e-06
1615 7.63754451327259e-06
1616 7.7398399298545e-06
1617 7.64330161473481e-06
1618 7.32972739569959e-06
1619 6.98588792147348e-06
1620 6.83626603859011e-06
1621 6.94891105013085e-06
1622 7.23449102224549e-06
1623 7.51333345760941e-06
1624 7.60940201871563e-06
1625 7.50754770706408e-06
1626 7.36707715986995e-06
1627 7.31674253984238e-06
1628 7.35141020413721e-06
1629 7.40055793357897e-06
1630 7.39344432076905e-06
1631 7.29489011064288e-06
1632 7.13259032636415e-06
1633 6.98151779943146e-06
1634 6.90720207785489e-06
1635 6.92651246936293e-06
1636 7.01220187693252e-06
1637 7.11614529791404e-06
1638 7.19258923709276e-06
1639 7.21885908205877e-06
1640 7.20425987310591e-06
1641 7.17752118362114e-06
1642 7.1645736170467e-06
1643 7.17624925528071e-06
1644 7.20764955985942e-06
1645 7.24058781997883e-06
1646 7.24709161659121e-06
1647 7.20091293260339e-06
1648 7.09815367372357e-06
1649 6.97065343047143e-06
1650 6.87426017975667e-06
1651 6.86077464706614e-06
1652 6.95705921316403e-06
1653 7.1545950959262e-06
1654 7.39686993256328e-06
1655 7.57974385123816e-06
1656 7.61011096983566e-06
1657 7.50121898818179e-06
1658 7.37104392101173e-06
1659 7.32907938072458e-06
1660 7.40991890779696e-06
1661 7.57936504669487e-06
1662 7.72780731495004e-06
1663 7.68829522712622e-06
1664 7.38990956961061e-06
1665 7.00702412359533e-06
1666 6.80493621985079e-06
1667 6.88197451381711e-06
1668 7.14564885129221e-06
1669 7.40625910111703e-06
1670 7.49393893784145e-06
1671 7.40422592571122e-06
1672 7.28289160178974e-06
1673 7.23579523764784e-06
1674 7.25369591236813e-06
1675 7.27684755474911e-06
1676 7.25044446880929e-06
1677 7.15411442797631e-06
1678 7.01719227436115e-06
1679 6.89985290591721e-06
1680 6.85085706209065e-06
1681 6.88176305629895e-06
1682 6.97006407790468e-06
1683 7.07437357050367e-06
1684 7.15307351129013e-06
1685 7.18382125342032e-06
1686 7.17316197551554e-06
1687 7.14654834155226e-06
1688 7.12921519152587e-06
1689 7.13413874109392e-06
1690 7.15940404916182e-06
1691 7.18724413673044e-06
1692 7.18641558705713e-06
1693 7.1264844336838e-06
1694 7.0043697633082e-06
1695 6.86119483361836e-06
1696 6.76751278660959e-06
1697 6.78930882713757e-06
1698 6.96110873832367e-06
1699 7.26324287825264e-06
1700 7.59639033276471e-06
1701 7.79967740527354e-06
1702 7.77122932049679e-06
1703 7.5884304351348e-06
1704 7.42973361411714e-06
1705 7.40310997571214e-06
1706 7.50500748836203e-06
1707 7.64275500841904e-06
1708 7.64982996770414e-06
1709 7.40640234653256e-06
1710 7.0217593020061e-06
1711 6.76048694003839e-06
1712 6.7716237026616e-06
1713 7.00645887263818e-06
1714 7.29502562535345e-06
1715 7.45142688174383e-06
1716 7.41856092645321e-06
1717 7.30442070562276e-06
1718 7.22759068594314e-06
1719 7.20947809895733e-06
1720 7.20885691407602e-06
1721 7.17631837687804e-06
1722 7.08610241417773e-06
1723 6.95561311658821e-06
1724 6.83742018736666e-06
1725 6.78375727147795e-06
1726 6.81644678479643e-06
1727 6.92065214025206e-06
1728 7.05366574038635e-06
1729 7.16391195965116e-06
1730 7.21653987056925e-06
1731 7.21075866749743e-06
1732 7.17343982614693e-06
1733 7.13681401975919e-06
1734 7.1218400989892e-06
1735 7.13222607373609e-06
1736 7.15245641913498e-06
1737 7.149287739594e-06
1738 7.08693096385105e-06
1739 6.95786002324894e-06
1740 6.80447374179494e-06
1741 6.70516010359279e-06
1742 6.73483145874343e-06
1743 6.93036008669878e-06
1744 7.26585676602554e-06
1745 7.63331536290934e-06
1746 7.86394866736373e-06
1747 7.84580697654746e-06
1748 7.65233380661812e-06
1749 7.47009516999242e-06
1750 7.41646954338648e-06
1751 7.49050195736345e-06
1752 7.59795420890441e-06
1753 7.57387260819087e-06
1754 7.31102727513644e-06
1755 6.92991761752637e-06
1756 6.68765869704657e-06
1757 6.72359874442918e-06
1758 6.98655367159517e-06
1759 7.30036708773696e-06
1760 7.47310059523443e-06
1761 7.44823637433001e-06
1762 7.33186288925936e-06
1763 7.24070332580595e-06
1764 7.2014581746771e-06
1765 7.18006003808114e-06
1766 7.13147437636508e-06
1767 7.03140312907635e-06
1768 6.89746366333566e-06
1769 6.78237529427861e-06
1770 6.7387936724117e-06
1771 6.78817696098122e-06
1772 6.91209288561367e-06
1773 7.06185301169171e-06
1774 7.18097362550907e-06
1775 7.23399034541217e-06
1776 7.22253753338009e-06
1777 7.17633793101413e-06
1778 7.12992368789855e-06
1779 7.10570657247445e-06
1780 7.10806534698349e-06
1781 7.12145265424624e-06
1782 7.11200755176833e-06
1783 7.04401099937968e-06
1784 6.91126342644566e-06
1785 6.75815772410715e-06
1786 6.66432197249378e-06
1787 6.70378494760371e-06
1788 6.91006016495521e-06
1789 7.25431573300739e-06
1790 7.62977106205653e-06
1791 7.8699349614908e-06
1792 7.86003784014611e-06
1793 7.66769335314166e-06
1794 7.47799549571937e-06
1795 7.41043231755611e-06
1796 7.46668729334488e-06
1797 7.55615883463179e-06
1798 7.5201805884717e-06
1799 7.25646759747178e-06
1800 6.88100271872827e-06
1801 6.64280787532334e-06
1802 6.68179063723073e-06
1803 6.95216249368968e-06
1804 7.27962969904183e-06
1805 7.46976047594217e-06
1806 7.45833494875114e-06
1807 7.34318336981232e-06
1808 7.24214442016091e-06
1809 7.18933324606041e-06
1810 7.15696342012961e-06
1811 7.10295398675953e-06
1812 7.00238479112159e-06
1813 6.86928160575917e-06
1814 6.75290812068852e-06
1815 6.70591589368996e-06
1816 6.7526711973187e-06
1817 6.87812689648126e-06
1818 7.03518298905692e-06
1819 7.16631075192709e-06
1820 7.23158836990478e-06
1821 7.22747199688456e-06
1822 7.1811687121226e-06
1823 7.12818200554466e-06
1824 7.09385540176299e-06
1825 7.08634297552635e-06
1826 7.09433606971288e-06
1827 7.08736297383439e-06
1828 7.02839770383434e-06
1829 6.90362821842427e-06
1830 6.74817374601844e-06
1831 6.63837863612571e-06
1832 6.65321476844838e-06
1833 6.83654252497945e-06
1834 7.17300235919538e-06
1835 7.56929784984095e-06
1836 7.85938846092904e-06
1837 7.90054491517367e-06
1838 7.72417206462706e-06
1839 7.514074241044e-06
1840 7.41244775781524e-06
1841 7.43605414754711e-06
1842 7.50658182369079e-06
1843 7.4803701863857e-06
1844 7.24855499356636e-06
1845 6.89017770127975e-06
1846 6.63075525153545e-06
1847 6.62386855765362e-06
1848 6.85450550008682e-06
1849 7.17828697816003e-06
1850 7.40633595341933e-06
1851 7.44111912354128e-06
1852 7.34555123926839e-06
1853 7.23818948245025e-06
1854 7.17275133865769e-06
1855 7.13490589987487e-06
1856 7.08679090166697e-06
1857 6.99944712323486e-06
1858 6.87524288878194e-06
1859 6.7522314566304e-06
1860 6.68175789542147e-06
1861 6.6979332586925e-06
1862 6.80106131767388e-06
1863 6.9571224230458e-06
1864 7.11120810592547e-06
1865 7.21216565580107e-06
1866 7.23861649021273e-06
1867 7.20574098522775e-06
1868 7.14896668796428e-06
1869 7.10062658981769e-06
1870 7.07778599462472e-06
1871 7.07905110175489e-06
1872 7.08284824213479e-06
1873 7.05179991200566e-06
1874 6.95440894560306e-06
1875 6.79918730384088e-06
1876 6.64816616335884e-06
1877 6.58980570733547e-06
1878 6.69555629428942e-06
1879 6.98596977599664e-06
1880 7.40250334274606e-06
1881 7.78989942773478e-06
1882 7.95346659288043e-06
1883 7.83340692578349e-06
1884 7.59138538342086e-06
1885 7.42556858313037e-06
1886 7.39573988539632e-06
1887 7.44079716241686e-06
1888 7.4348922680656e-06
1889 7.26306279830169e-06
1890 6.94788013788639e-06
1891 6.66438018015469e-06
1892 6.57532200420974e-06
1893 6.71130101181916e-06
1894 6.98430085321888e-06
1895 7.24133087715018e-06
1896 7.3543019425415e-06
1897 7.31621685190476e-06
1898 7.2195930442831e-06
1899 7.14350790076423e-06
1900 7.1021154326445e-06
1901 7.06986247678287e-06
1902 7.01425960869528e-06
1903 6.91892500981339e-06
1904 6.79819868310005e-06
1905 6.69294377075857e-06
1906 6.64771459923941e-06
1907 6.68802385916933e-06
1908 6.80940274833119e-06
1909 6.9770007939951e-06
1910 7.13583722244948e-06
1911 7.23471111996332e-06
1912 7.25342761143111e-06
1913 7.21108881407417e-06
1914 7.1479007601738e-06
1915 7.09936011844547e-06
1916 7.08307834429434e-06
1917 7.095703494997e-06
1918 7.11042412149254e-06
1919 7.08080006006639e-06
1920 6.9677907958976e-06
1921 6.78357901051641e-06
1922 6.60730711388169e-06
1923 6.54654195386684e-06
1924 6.68337906972738e-06
1925 7.03407295077341e-06
1926 7.50625713408226e-06
1927 7.882194040576e-06
1928 7.94496008893475e-06
1929 7.7204012995935e-06
1930 7.46179057387053e-06
1931 7.34323111828417e-06
1932 7.3471233008604e-06
1933 7.36120773581206e-06
1934 7.2673497015785e-06
1935 7.03141449776012e-06
1936 6.75412684358889e-06
1937 6.58479893900221e-06
1938 6.5973072196357e-06
1939 6.76448462400003e-06
1940 6.99276233717683e-06
1941 7.16831527824979e-06
1942 7.22405457054265e-06
1943 7.18207957106642e-06
1944 7.11228994987323e-06
1945 7.06178070686292e-06
1946 7.03482101016562e-06
1947 7.01076669429312e-06
1948 6.96428242008551e-06
1949 6.88152613292914e-06
1950 6.77266871207394e-06
1951 6.67136873744312e-06
1952 6.61909643895342e-06
1953 6.64623757984373e-06
1954 6.7597998167912e-06
1955 6.93736819812329e-06
1956 7.12816381565062e-06
1957 7.26928010408301e-06
1958 7.31853242541547e-06
1959 7.28123677617987e-06
1960 7.20325988368131e-06
1961 7.13687950337771e-06
1962 7.11471102476935e-06
1963 7.14197130946559e-06
1964 7.19213085176307e-06
1965 7.20326943337568e-06
1966 7.10339509168989e-06
1967 6.88053660269361e-06
1968 6.62965521769365e-06
1969 6.50035553917405e-06
1970 6.60431214782875e-06
1971 6.96210418027476e-06
1972 7.45614534025663e-06
1973 7.81800463300897e-06
1974 7.82044662628323e-06
1975 7.56048257244402e-06
1976 7.33066781322123e-06
1977 7.25326208339538e-06
1978 7.25664040146512e-06
1979 7.22527329344302e-06
1980 7.08735251464532e-06
1981 6.86571138430736e-06
1982 6.66232517687604e-06
1983 6.57149848848348e-06
1984 6.61917738398188e-06
1985 6.7676469370781e-06
1986 6.94339905749075e-06
1987 7.07087201590184e-06
1988 7.11316306478693e-06
1989 7.08866400600527e-06
1990 7.04206377122318e-06
1991 7.00481314197532e-06
1992 6.98272560839541e-06
1993 6.96367214914062e-06
1994 6.9288303166104e-06
1995 6.86374460201478e-06
1996 6.76984791425639e-06
1997 6.66961204842664e-06
1998 6.59996703689103e-06
1999 6.59756778986775e-06
};
\addlegendentry{Test}

\nextgroupplot[
title={ELU,SiLU},
ymin=3.07543709269404e-06, ymax=0.001,
]
\addplot [semithick, black, dashed]
table {%
0 0.0134240260813385
1 0.0130271133384667
2 0.0126539912307635
3 0.0123001110623591
4 0.0119606663938612
5 0.0116302319220267
6 0.0113027505867649
7 0.0109708382515237
8 0.0106251401011832
9 0.0102545053814538
10 0.00984724701265804
11 0.00939239663421176
12 0.0088820485107135
13 0.00831598159857094
14 0.00770624031429179
15 0.00707746570697054
16 0.00645954950596206
17 0.00587484262359794
18 0.00533189879206475
19 0.00483110928325914
20 0.0043710003956221
21 0.003950059399358
22 0.00356662102421978
23 0.00321888898906764
24 0.00290492414933397
25 0.00262264765478903
26 0.00236964876239654
27 0.00214347754808841
28 0.00194175241267658
29 0.00176220018329332
30 0.00160266004968435
31 0.00146112896254635
32 0.00133572659979109
33 0.00122469516099954
34 0.00112642470412538
35 0.00103945458249655
36 0.000962465006523416
37 0.000894264892849606
38 0.000833794165373547
39 0.000780113318796793
40 0.000732394652004587
41 0.000689915616021608
42 0.000652049411655753
43 0.00061824152180634
44 0.000588006590987789
45 0.000560922805561859
46 0.000536618975729652
47 0.000514768938046473
48 0.000495088355364715
49 0.000477327641874581
50 0.000461267609352944
51 0.000446715721864166
52 0.000433502105806838
53 0.000421477273448545
54 0.000410509145922333
55 0.000400481355427473
56 0.000391290689549351
57 0.000382845885724237
58 0.000375066212200181
59 0.000367880136309395
60 0.000361224224889156
61 0.000355042557885099
62 0.000349285450511161
63 0.000343908856621056
64 0.000338873711825727
65 0.000334145629267368
66 0.000329693884395965
67 0.000325491386774956
68 0.000321513960898301
69 0.000317740195441729
70 0.000314151067300372
71 0.0003107297417273
72 0.000307461123952635
73 0.000304332056543899
74 0.000301330670595235
75 0.000298446226452143
76 0.00029566929140401
77 0.000292991475816962
78 0.000290405010332506
79 0.00028790303349524
80 0.000285479287754242
81 0.000283128114119791
82 0.000280844372468891
83 0.000278623512599552
84 0.000276461423368346
85 0.000274354229077289
86 0.00027229834620357
87 0.000270290499202019
88 0.000268327683329517
89 0.000266407136223279
90 0.000264526305727486
91 0.000262682782931734
92 0.000260874346508899
93 0.000259098908031774
94 0.000257354537438914
95 0.000255639469401103
96 0.000253952347520681
97 0.000252291536071425
98 0.000250655507670672
99 0.000249042815653411
100 0.000247452069174869
101 0.000245882038484524
102 0.000244331470184989
103 0.000242799142711192
104 0.000241283979903528
105 0.000239784888549366
106 0.00023830071120301
107 0.000236830532458043
108 0.000235373300711217
109 0.000233928095610736
110 0.000232493979297033
111 0.000231069958033459
112 0.000229655128180184
113 0.00022824855102499
114 0.000226849392561235
115 0.00022545671441776
116 0.00022406971760347
117 0.000222687933160159
118 0.000221310843585343
119 0.000219937808140003
120 0.000218568067225533
121 0.00021720100477296
122 0.000215836096458588
123 0.000214472806646882
124 0.000213110712650177
125 0.000211749274455997
126 0.000210387997071848
127 0.000209026343327423
128 0.000207663833236893
129 0.000206300001480031
130 0.000204934233863696
131 0.000203566056597992
132 0.000202195068936817
133 0.000200820879058483
134 0.000199443167275604
135 0.000198061641924596
136 0.000196675943584523
137 0.000195285881488871
138 0.000193891223432274
139 0.000192491806842554
140 0.0001910875251383
141 0.000189678234846724
142 0.000188263821314649
143 0.000186844177676448
144 0.000185419414208354
145 0.00018398959531396
146 0.00018255484957308
147 0.00018111531676368
148 0.000179671397120273
149 0.000178223478769723
150 0.000176771844394352
151 0.000175316906620537
152 0.000173859239453122
153 0.000172399576854332
154 0.000170938439282509
155 0.000169476523183221
156 0.000168014380790282
157 0.000166552741006853
158 0.000165092431643643
159 0.000163634155512682
160 0.000162178354628395
161 0.000160725825139707
162 0.000159277318715567
163 0.000157833552492548
164 0.000156395381281982
165 0.000154963747291958
166 0.00015353912723981
167 0.000152122384974973
168 0.000150714097713944
169 0.000149314982195392
170 0.000147925625327616
171 0.000146546661142111
172 0.00014517863888841
173 0.00014382220990683
174 0.000142477945445307
175 0.000141146325830732
176 0.000139827795152314
177 0.00013852278547688
178 0.000137231637864943
179 0.000135954729898913
180 0.000134692126465552
181 0.000133443902598174
182 0.000132210214502493
183 0.00013099117342108
184 0.000129786886560623
185 0.000128597422815346
186 0.000127422842467695
187 0.000126263154982098
188 0.000125118351718356
189 0.000123988354118865
190 0.000122873132823997
191 0.000121772554848576
192 0.000120686602627984
193 0.000119615407413676
194 0.000118558737398189
195 0.000117516458374212
196 0.000116488312869478
197 0.000115474409710714
198 0.000114474325471292
199 0.000113487927308142
200 0.000112514907016248
201 0.000111555232280125
202 0.000110608391423739
203 0.000109674431627127
204 0.000108752993327244
205 0.000107843835678523
206 0.000106946654824469
207 0.000106061210857433
208 0.000105187219190839
209 0.000104324438609638
210 0.000103472822758022
211 0.000102631962874966
212 0.000101802033469767
213 0.000100982688110207
214 0.000100173735830822
215 9.93748517998938e-05
216 9.85858006856688e-05
217 9.78063986565303e-05
218 9.70364023373804e-05
219 9.62755817681682e-05
220 9.55237727850999e-05
221 9.47807746172202e-05
222 9.40463746701425e-05
223 9.33203795341342e-05
224 9.26026204126629e-05
225 9.18928834892085e-05
226 9.11910356080625e-05
227 9.04969213877393e-05
228 8.98103616293611e-05
229 8.91311842394771e-05
230 8.84592414251983e-05
231 8.77943864168174e-05
232 8.71365051864359e-05
233 8.64854043811647e-05
234 8.58409837576346e-05
235 8.52031163844913e-05
236 8.45716803326013e-05
237 8.39465791671046e-05
238 8.33276813523298e-05
239 8.27149207509592e-05
240 8.21081781623434e-05
241 8.15073532578481e-05
242 8.09123554290636e-05
243 8.03231014003813e-05
244 7.97394843630173e-05
245 7.9161453385268e-05
246 7.85888807968149e-05
247 7.80217067699596e-05
248 7.74598285318007e-05
249 7.69032251497492e-05
250 7.63518128366059e-05
251 7.5805502248727e-05
252 7.52642395838166e-05
253 7.47279546970958e-05
254 7.41966468496003e-05
255 7.36702115062826e-05
256 7.3148597692807e-05
257 7.26317547474764e-05
258 7.21196488768783e-05
259 7.16121921016111e-05
260 7.11093789647066e-05
261 7.06111406714172e-05
262 7.01173952393219e-05
263 6.96281502712282e-05
264 6.9143335622357e-05
265 6.86628972630388e-05
266 6.8186802081982e-05
267 6.77149979395608e-05
268 6.72474605778461e-05
269 6.67841204631259e-05
270 6.63249733321436e-05
271 6.58699617730463e-05
272 6.5419060149452e-05
273 6.49722239671746e-05
274 6.45294124410611e-05
275 6.40906198157154e-05
276 6.36557978310748e-05
277 6.32249072651803e-05
278 6.27979428315939e-05
279 6.2374811477639e-05
280 6.19555593743826e-05
281 6.1540100702473e-05
282 6.11284564939751e-05
283 6.07205485181339e-05
284 6.03163766470516e-05
285 5.99159348837475e-05
286 5.95191848731247e-05
287 5.91260907469859e-05
288 5.87366322690741e-05
289 5.83507843003872e-05
290 5.79685059136636e-05
291 5.75898158672317e-05
292 5.72146349639979e-05
293 5.68429658187597e-05
294 5.64747925437814e-05
295 5.61100668505787e-05
296 5.57487815058266e-05
297 5.53909142553266e-05
298 5.50364197380304e-05
299 5.46852873384296e-05
300 5.43374858068546e-05
301 5.39930041014713e-05
302 5.36518160316746e-05
303 5.3313888173534e-05
304 5.29791938959079e-05
305 5.2647721972221e-05
306 5.23194487982437e-05
307 5.19943863821482e-05
308 5.16724543331293e-05
309 5.13536579092033e-05
310 5.10380001941257e-05
311 5.072543585527e-05
312 5.04159773413448e-05
313 5.01095741611834e-05
314 4.98062354665763e-05
315 4.95059192928693e-05
316 4.92086403482972e-05
317 4.89143306623419e-05
318 4.86230208593952e-05
319 4.83346592972111e-05
320 4.80492281980105e-05
321 4.77666953173639e-05
322 4.7487038486338e-05
323 4.72102484962988e-05
324 4.69363019135471e-05
325 4.66651447936783e-05
326 4.63967910775409e-05
327 4.61312019979232e-05
328 4.58683466320053e-05
329 4.56082251787393e-05
330 4.53507989277568e-05
331 4.5096044203774e-05
332 4.48439428453185e-05
333 4.45944734224213e-05
334 4.43475978642027e-05
335 4.41032991886914e-05
336 4.38615645492746e-05
337 4.36223579640682e-05
338 4.33856375678943e-05
339 4.315141595157e-05
340 4.2919645622419e-05
341 4.269028532633e-05
342 4.2463353452149e-05
343 4.22387867757834e-05
344 4.20165659704708e-05
345 4.17966961663296e-05
346 4.15791317749381e-05
347 4.13638494478619e-05
348 4.11508034403596e-05
349 4.09400046095243e-05
350 4.07313999204462e-05
351 4.05250064403617e-05
352 4.03207343353529e-05
353 4.0118619025975e-05
354 3.99186273085661e-05
355 3.97207013946854e-05
356 3.95248610232102e-05
357 3.93310253699042e-05
358 3.91392343601638e-05
359 3.8949418879497e-05
360 3.87615938279851e-05
361 3.85757026748479e-05
362 3.83917689461555e-05
363 3.82097206212961e-05
364 3.8029560961661e-05
365 3.78512613892212e-05
366 3.76748060730847e-05
367 3.75001618948545e-05
368 3.73273106077932e-05
369 3.71562518282076e-05
370 3.69869650072019e-05
371 3.68194432738278e-05
372 3.66536471432255e-05
373 3.64895590152514e-05
374 3.63271516974351e-05
375 3.61664221770752e-05
376 3.60073277292372e-05
377 3.5849854221226e-05
378 3.56939918262356e-05
379 3.55397104598865e-05
380 3.53869811462459e-05
381 3.52357973483208e-05
382 3.50861578795048e-05
383 3.49379853687992e-05
384 3.47913548850443e-05
385 3.46461420264177e-05
386 3.4502379953949e-05
387 3.43600511811815e-05
388 3.42191114341972e-05
389 3.40795993523102e-05
390 3.39414429291196e-05
391 3.38046791696911e-05
392 3.36692538738248e-05
393 3.35351578755194e-05
394 3.34023890005142e-05
395 3.32709198715975e-05
396 3.3140709582824e-05
397 3.30117931994778e-05
398 3.28840853711654e-05
399 3.27576198060342e-05
400 3.26323645722937e-05
401 3.25083120102931e-05
402 3.23854200914298e-05
403 3.22637054281927e-05
404 3.2143115930694e-05
405 3.20236630457771e-05
406 3.19053476971476e-05
407 3.17880897000578e-05
408 3.16719305075708e-05
409 3.15568446183079e-05
410 3.14428086909402e-05
411 3.13298289427166e-05
412 3.12178349872738e-05
413 3.11069068246184e-05
414 3.0996922063764e-05
415 3.08879184771627e-05
416 3.07799019836352e-05
417 3.06728532990519e-05
418 3.05667393476483e-05
419 3.04615491799609e-05
420 3.03572911661831e-05
421 3.02539170959903e-05
422 3.01514436173989e-05
423 3.00498572727292e-05
424 2.99491076987124e-05
425 2.98492219599211e-05
426 2.97501797135169e-05
427 2.96519634446213e-05
428 2.9554539480614e-05
429 2.94579458852695e-05
430 2.93620998945698e-05
431 2.92670467558764e-05
432 2.91727650605367e-05
433 2.90792216830482e-05
434 2.89864239135795e-05
435 2.88943467836589e-05
436 2.88029849073723e-05
437 2.87123232212139e-05
438 2.86223638710226e-05
439 2.85330617444401e-05
440 2.84444574489839e-05
441 2.83565057799251e-05
442 2.82691910413746e-05
443 2.8182526776277e-05
444 2.80965022625423e-05
445 2.80110926951238e-05
446 2.79262774043332e-05
447 2.78420767614307e-05
448 2.77584466061853e-05
449 2.76753994512546e-05
450 2.75929133692898e-05
451 2.75109922398542e-05
452 2.74296218449877e-05
453 2.73487754256507e-05
454 2.72684639028853e-05
455 2.71886626279638e-05
456 2.71093788057897e-05
457 2.70305830412099e-05
458 2.69522904474684e-05
459 2.68744799925003e-05
460 2.67971450114146e-05
461 2.67202876145234e-05
462 2.66438564580085e-05
463 2.65678808943903e-05
464 2.64923411776863e-05
465 2.64172332293811e-05
466 2.63425763833425e-05
467 2.62682900640243e-05
468 2.61944326567232e-05
469 2.61209665595175e-05
470 2.60478919003049e-05
471 2.59752084303955e-05
472 2.59028870885913e-05
473 2.58309458303074e-05
474 2.57593383068411e-05
475 2.56880807611992e-05
476 2.5617204236994e-05
477 2.55466296934515e-05
478 2.54763999620877e-05
479 2.5406490273383e-05
480 2.53369043718976e-05
481 2.52676152143749e-05
482 2.51986398964732e-05
483 2.51299457545429e-05
484 2.50615540835497e-05
485 2.49934552911668e-05
486 2.49256175948176e-05
487 2.48580418542588e-05
488 2.47907525334767e-05
489 2.47237252999355e-05
490 2.46569524939844e-05
491 2.45904189277724e-05
492 2.45241340053326e-05
493 2.44581052228909e-05
494 2.43923064573437e-05
495 2.43267303190464e-05
496 2.42613690843996e-05
497 2.41962548486185e-05
498 2.41313463860138e-05
499 2.40666595416883e-05
500 2.40021632755827e-05
501 2.39378731947681e-05
502 2.38737878355266e-05
503 2.38098791776054e-05
504 2.37461588099563e-05
505 2.36826461623707e-05
506 2.36193078535507e-05
507 2.35561293173703e-05
508 2.34931438711783e-05
509 2.34303250188361e-05
510 2.33676639922464e-05
511 2.33051707674292e-05
512 2.32428139028684e-05
513 2.31806364396903e-05
514 2.31186114625359e-05
515 2.30567355679057e-05
516 2.29949997816448e-05
517 2.29334034216322e-05
518 2.28719414501199e-05
519 2.2810643510951e-05
520 2.27494547822005e-05
521 2.26884145959616e-05
522 2.26275002752629e-05
523 2.25667057556223e-05
524 2.25060360925511e-05
525 2.24455034292248e-05
526 2.23850830600725e-05
527 2.23247689739026e-05
528 2.2264570031183e-05
529 2.22045075624067e-05
530 2.21445538457488e-05
531 2.2084706973402e-05
532 2.20249504252479e-05
533 2.19653121149577e-05
534 2.19057864931926e-05
535 2.18463669696689e-05
536 2.17870390954999e-05
537 2.17278129888143e-05
538 2.16686976273195e-05
539 2.16096636052043e-05
540 2.15507346617017e-05
541 2.14919020073978e-05
542 2.1433172154417e-05
543 2.13745487585015e-05
544 2.13160086062203e-05
545 2.12575590730069e-05
546 2.11991896961194e-05
547 2.11409173580535e-05
548 2.10827331770247e-05
549 2.10246477223563e-05
550 2.09666668347097e-05
551 2.0908757381477e-05
552 2.08509307526583e-05
553 2.07931821023521e-05
554 2.07355261814257e-05
555 2.0677962158544e-05
556 2.06204781321162e-05
557 2.05630744183338e-05
558 2.0505778536517e-05
559 2.04485583488179e-05
560 2.03914315335396e-05
561 2.03343813645063e-05
562 2.02774124815619e-05
563 2.02205334325356e-05
564 2.01637642831543e-05
565 2.01070506982148e-05
566 2.00504312175553e-05
567 1.99939101293012e-05
568 1.99374653462314e-05
569 1.9881113693998e-05
570 1.98248413205704e-05
571 1.97686623835125e-05
572 1.971257065847e-05
573 1.96565788890268e-05
574 1.96006621813183e-05
575 1.95448342914517e-05
576 1.94890960472094e-05
577 1.94334498004878e-05
578 1.93778931318889e-05
579 1.93224412932125e-05
580 1.92670671239625e-05
581 1.92117863946351e-05
582 1.91565924687609e-05
583 1.91014913468734e-05
584 1.90464914240351e-05
585 1.89915738530999e-05
586 1.8936757417265e-05
587 1.88820342685858e-05
588 1.88274078674056e-05
589 1.87728782385932e-05
590 1.87184361379877e-05
591 1.86640828445661e-05
592 1.8609836999417e-05
593 1.85556957568167e-05
594 1.85016496310197e-05
595 1.84476972897585e-05
596 1.83938485349699e-05
597 1.83401117119786e-05
598 1.82864648223813e-05
599 1.82329158846528e-05
600 1.81794831739523e-05
601 1.81261535061594e-05
602 1.80729204437569e-05
603 1.80197883601352e-05
604 1.79667644921722e-05
605 1.79138434468484e-05
606 1.78610420640268e-05
607 1.78083589155165e-05
608 1.77557629221781e-05
609 1.77032814541178e-05
610 1.7650913900269e-05
611 1.75986578589971e-05
612 1.75465252780782e-05
613 1.7494511212135e-05
614 1.74425891685814e-05
615 1.7390781749782e-05
616 1.73390977664667e-05
617 1.72875261661432e-05
618 1.72360866237398e-05
619 1.71847557339788e-05
620 1.71335469953959e-05
621 1.70824557006455e-05
622 1.70315019740741e-05
623 1.69806675778261e-05
624 1.69299641612497e-05
625 1.68793708006376e-05
626 1.68289023854129e-05
627 1.67785509219698e-05
628 1.67283513672345e-05
629 1.6678281529181e-05
630 1.66283291118674e-05
631 1.65784934242907e-05
632 1.6528797887716e-05
633 1.64792408519077e-05
634 1.64297888236575e-05
635 1.6380456743903e-05
636 1.63312568552954e-05
637 1.62821854292616e-05
638 1.62332573676593e-05
639 1.61844504980024e-05
640 1.61357642696203e-05
641 1.60872265535517e-05
642 1.60388118572996e-05
643 1.59905336687416e-05
644 1.59423758425703e-05
645 1.58943475749851e-05
646 1.58464774706601e-05
647 1.57987295246187e-05
648 1.57511245220121e-05
649 1.57036463530602e-05
650 1.56562969522156e-05
651 1.56090903402628e-05
652 1.55620175359417e-05
653 1.5515055176607e-05
654 1.54682300976816e-05
655 1.54215327903273e-05
656 1.53749740352538e-05
657 1.53285451460761e-05
658 1.52822555161691e-05
659 1.52361075116403e-05
660 1.51900999938448e-05
661 1.51442312859018e-05
662 1.50984922022701e-05
663 1.50529061730964e-05
664 1.50074538876055e-05
665 1.4962130306273e-05
666 1.491693611122e-05
667 1.48718591699293e-05
668 1.48269098563247e-05
669 1.47820799956122e-05
670 1.47373590699829e-05
671 1.46927434929722e-05
672 1.46482299516748e-05
673 1.46038270294468e-05
674 1.45595282372568e-05
675 1.45153523369856e-05
676 1.44712872209851e-05
677 1.44273473061673e-05
678 1.43835573869211e-05
679 1.43399201029126e-05
680 1.42964556619773e-05
681 1.42531798061896e-05
682 1.42101036910702e-05
683 1.41672656415182e-05
684 1.41246899048042e-05
685 1.4082418244854e-05
686 1.40404940616179e-05
687 1.39989870753254e-05
688 1.39579699105497e-05
689 1.39175564761729e-05
690 1.38778969329678e-05
691 1.38392324071646e-05
692 1.38018526065764e-05
693 1.37663014996292e-05
694 1.37333605305656e-05
695 1.37043249317514e-05
696 1.36813372293432e-05
697 1.36680061224581e-05
698 1.36702835895619e-05
699 1.36973781810212e-05
700 1.37606034869009e-05
701 1.38642297073943e-05
702 1.39794696991657e-05
703 1.40168744859892e-05
704 1.38732762930971e-05
705 1.35781640207e-05
706 1.33470633318922e-05
707 1.34550077781626e-05
708 1.40745831469502e-05
709 1.48550399803327e-05
710 1.47544180286729e-05
711 1.37418063399508e-05
712 1.31015872266715e-05
713 1.31515866623744e-05
714 1.33277262470699e-05
715 1.32675518003111e-05
716 1.30372678519564e-05
717 1.2839433580325e-05
718 1.27410990202748e-05
719 1.26974943892577e-05
720 1.26637188753165e-05
721 1.26279033150212e-05
722 1.25942447954941e-05
723 1.25660615690748e-05
724 1.25416643541598e-05
725 1.25171363762888e-05
726 1.2489496583612e-05
727 1.24579413949277e-05
728 1.24233577238897e-05
729 1.23873181028245e-05
730 1.23511897367479e-05
731 1.23157797560225e-05
732 1.22813994725135e-05
733 1.22480833137928e-05
734 1.22157830055869e-05
735 1.21844496430157e-05
736 1.21540805135822e-05
737 1.21246722013524e-05
738 1.2096119684557e-05
739 1.20682589965782e-05
740 1.20408129387073e-05
741 1.20134089254265e-05
742 1.19856332307933e-05
743 1.19571079757463e-05
744 1.19275013759079e-05
745 1.18966387496755e-05
746 1.18644933380097e-05
747 1.18312178027935e-05
748 1.17971438253761e-05
749 1.17627084303962e-05
750 1.17284558935893e-05
751 1.16949600084126e-05
752 1.16627987694073e-05
753 1.16325430816744e-05
754 1.16047018838827e-05
755 1.15797200521683e-05
756 1.15578904260616e-05
757 1.15392299004924e-05
758 1.15232483537397e-05
759 1.1508624140788e-05
760 1.14928148775562e-05
761 1.14719018018405e-05
762 1.14411500220513e-05
763 1.1396841665956e-05
764 1.13403463259942e-05
765 1.12853614702146e-05
766 1.12720132037225e-05
767 1.13974847906206e-05
768 1.18707695904163e-05
769 1.29717617216585e-05
770 1.43125288101231e-05
771 1.40081106874135e-05
772 1.21694525425653e-05
773 1.15186362439346e-05
774 1.19656989756578e-05
775 1.20958604643562e-05
776 1.16389607640599e-05
777 1.11758461649458e-05
778 1.09885721304437e-05
779 1.09485286845512e-05
780 1.09236765784004e-05
781 1.08873074751159e-05
782 1.08523784412284e-05
783 1.08255022244208e-05
784 1.08034930565992e-05
785 1.07820465160557e-05
786 1.07596934331156e-05
787 1.07369573623117e-05
788 1.07147003092223e-05
789 1.06932139996019e-05
790 1.06722849793073e-05
791 1.06515433024867e-05
792 1.06306899052555e-05
793 1.06096068090977e-05
794 1.05883089585035e-05
795 1.05668857042218e-05
796 1.05454434056185e-05
797 1.05240565204667e-05
798 1.05027660275425e-05
799 1.04815999364405e-05
800 1.04605660329327e-05
801 1.04396869744505e-05
802 1.04189634146934e-05
803 1.03984153305703e-05
804 1.0378051621629e-05
805 1.03578888914768e-05
806 1.03379247544666e-05
807 1.03181388890761e-05
808 1.02985159795566e-05
809 1.02790333134095e-05
810 1.02596300521896e-05
811 1.02402630375309e-05
812 1.02208714274354e-05
813 1.02013924472288e-05
814 1.01817744351251e-05
815 1.01619743837489e-05
816 1.014197876259e-05
817 1.01218374961576e-05
818 1.01017322862162e-05
819 1.0082084333618e-05
820 1.00638305546852e-05
821 1.00489253229696e-05
822 1.00414602872689e-05
823 1.00499557822431e-05
824 1.00918621575019e-05
825 1.02011239997424e-05
826 1.04344670353029e-05
827 1.08488261458817e-05
828 1.13771297742815e-05
829 1.16317838463331e-05
830 1.11752469402759e-05
831 1.03620459253051e-05
832 1.00761103638547e-05
833 1.06636317624975e-05
834 1.16564524894613e-05
835 1.18906297839061e-05
836 1.09502144498563e-05
837 1.00680575139123e-05
838 9.85681509924063e-06
839 9.88056736783705e-06
840 9.8414161815441e-06
841 9.75852661255772e-06
842 9.70326681404288e-06
843 9.68581215765596e-06
844 9.68211819341036e-06
845 9.67324151623927e-06
846 9.65543766717758e-06
847 9.6333319534736e-06
848 9.61190560833103e-06
849 9.59313717530108e-06
850 9.57661741018967e-06
851 9.56123379847185e-06
852 9.54619059001516e-06
853 9.53124178426634e-06
854 9.51651694247602e-06
855 9.50217874429171e-06
856 9.48829607061441e-06
857 9.47478212065533e-06
858 9.46143106617114e-06
859 9.44804168145907e-06
860 9.43441444789528e-06
861 9.42047021101899e-06
862 9.40619906319995e-06
863 9.39168050972938e-06
864 9.37703963899139e-06
865 9.36243801241687e-06
866 9.34803766750036e-06
867 9.33397273161773e-06
868 9.32036218515009e-06
869 9.30727158632294e-06
870 9.29475444610972e-06
871 9.28280479328691e-06
872 9.27140294848527e-06
873 9.26047354266046e-06
874 9.24991656692242e-06
875 9.23957542919851e-06
876 9.22924636892475e-06
877 9.21862222913461e-06
878 9.20730945797033e-06
879 9.19481770988995e-06
880 9.18053998777424e-06
881 9.16385755189708e-06
882 9.14438205779078e-06
883 9.12266273633833e-06
884 9.10195788339507e-06
885 9.092968633162e-06
886 9.12615743420275e-06
887 9.28312796233399e-06
888 9.76048130851837e-06
889 1.08806483947177e-05
890 1.24489751573975e-05
891 1.24780599488616e-05
892 1.04431861638332e-05
893 9.63778023166384e-06
894 1.04625349539589e-05
895 1.08780014604903e-05
896 1.01698889949375e-05
897 9.35439118165959e-06
898 9.02972853200623e-06
899 8.9616669631809e-06
900 8.93537237001141e-06
901 8.91824183302248e-06
902 8.91474783504265e-06
903 8.91545520431691e-06
904 8.91028161120744e-06
905 8.89771014822571e-06
906 8.88135561272207e-06
907 8.86483298767615e-06
908 8.84985433025065e-06
909 8.83664662953976e-06
910 8.82483184128091e-06
911 8.81395415053277e-06
912 8.80363567468123e-06
913 8.79360706385057e-06
914 8.78367046652784e-06
915 8.77369172336273e-06
916 8.76360608881299e-06
917 8.75339131667374e-06
918 8.74305711207057e-06
919 8.73263849676675e-06
920 8.72218295722149e-06
921 8.71172418914767e-06
922 8.70131204244728e-06
923 8.69096542110981e-06
924 8.68069369897029e-06
925 8.67050650121826e-06
926 8.66042097502628e-06
927 8.65043091557993e-06
928 8.64054074511955e-06
929 8.63076414070463e-06
930 8.62109818022816e-06
931 8.61157135201296e-06
932 8.60220308407378e-06
933 8.59302707745258e-06
934 8.58410731296999e-06
935 8.57553453226245e-06
936 8.56742937571653e-06
937 8.56000319249262e-06
938 8.55357421158942e-06
939 8.54868146404897e-06
940 8.54620440993159e-06
941 8.54765994340312e-06
942 8.55569222402153e-06
943 8.57497124773232e-06
944 8.61365689530658e-06
945 8.6855001750763e-06
946 8.81116813822302e-06
947 9.01298376909665e-06
948 9.28828623258937e-06
949 9.54990792045862e-06
950 9.59603535477527e-06
951 9.2779252183206e-06
952 8.78443269325402e-06
953 8.57447897573138e-06
954 9.03823285725025e-06
955 1.02373132877531e-05
956 1.12907304856691e-05
957 1.06648597935077e-05
958 9.15152395197794e-06
959 8.58686552174248e-06
960 8.68134030351797e-06
961 8.68940986187283e-06
962 8.52930010886155e-06
963 8.38527323843152e-06
964 8.33064340177714e-06
965 8.32665722949599e-06
966 8.32686654561599e-06
967 8.31632447440001e-06
968 8.30026386466898e-06
969 8.286019498982e-06
970 8.27575792072821e-06
971 8.26785786767203e-06
972 8.26016969934784e-06
973 8.25165311368892e-06
974 8.24238500385377e-06
975 8.23293277640857e-06
976 8.22381558851504e-06
977 8.21527040439918e-06
978 8.20724271299866e-06
979 8.19953279940222e-06
980 8.19190004275328e-06
981 8.18417812187278e-06
982 8.17628878468923e-06
983 8.16826342653343e-06
984 8.16019241245414e-06
985 8.15223406913645e-06
986 8.14455233388856e-06
987 8.13733446669573e-06
988 8.13074058125807e-06
989 8.12494959667731e-06
990 8.12014246243109e-06
991 8.11650915366613e-06
992 8.11428552704996e-06
993 8.11378238463334e-06
994 8.11539434764086e-06
995 8.11965382840896e-06
996 8.12724099219508e-06
997 8.13899472262847e-06
998 8.15584966851191e-06
999 8.17862696678162e-06
1000 8.20755312602728e-06
1001 8.24135024934947e-06
1002 8.27581073448158e-06
1003 8.30222944259162e-06
1004 8.30718343181047e-06
1005 8.27598037389521e-06
1006 8.2017487912367e-06
1007 8.09901631182441e-06
1008 8.02026728852212e-06
1009 8.08430483179023e-06
1010 8.54109948544846e-06
1011 9.80275635331651e-06
1012 1.1738618154844e-05
1013 1.19693321947167e-05
1014 9.60514731129791e-06
1015 8.45992963949271e-06
1016 8.88347212679719e-06
1017 8.98518457681519e-06
1018 8.4971700156089e-06
1019 8.06993075386231e-06
1020 7.92056679266295e-06
1021 7.89550697000152e-06
1022 7.88505262683259e-06
1023 7.87430742921202e-06
1024 7.86939681951537e-06
1025 7.8679269446269e-06
1026 7.86450818957007e-06
1027 7.85734148411876e-06
1028 7.84782652463178e-06
1029 7.83796296044414e-06
1030 7.82887481953765e-06
1031 7.82077999694764e-06
1032 7.81345735667571e-06
1033 7.80662693244949e-06
1034 7.80006730671801e-06
1035 7.79366736924914e-06
1036 7.7873506700854e-06
1037 7.78107228516234e-06
1038 7.77480172686751e-06
1039 7.76850255057582e-06
1040 7.76214720943358e-06
1041 7.75572792743873e-06
1042 7.74923900959124e-06
1043 7.7426953302151e-06
1044 7.73609927762209e-06
1045 7.72947917404565e-06
1046 7.72284520156319e-06
1047 7.71621838602243e-06
1048 7.7096081847472e-06
1049 7.7030356937513e-06
1050 7.69650959497881e-06
1051 7.69004903933279e-06
1052 7.6836683104986e-06
1053 7.67740042295628e-06
1054 7.67128858569777e-06
1055 7.66541386987285e-06
1056 7.65991891071849e-06
1057 7.65506416833261e-06
1058 7.65134471336637e-06
1059 7.64972631817074e-06
1060 7.65214561493366e-06
1061 7.66261225582099e-06
1062 7.68955830654505e-06
1063 7.75080581849608e-06
1064 7.88286301656171e-06
1065 8.15276549648303e-06
1066 8.64631193042698e-06
1067 9.33809121583806e-06
1068 9.79188223304561e-06
1069 9.33832977345617e-06
1070 8.29685403580527e-06
1071 7.87544342450985e-06
1072 8.53982925885788e-06
1073 9.62409788485274e-06
1074 9.76543717001732e-06
1075 8.69841524142601e-06
1076 7.82468705562422e-06
1077 7.63814786353123e-06
1078 7.64711097556159e-06
1079 7.59819703688436e-06
1080 7.53238291029135e-06
1081 7.50450359809918e-06
1082 7.50704025609394e-06
1083 7.51321616210276e-06
1084 7.50975622754169e-06
1085 7.49811577183124e-06
1086 7.48445884735105e-06
1087 7.47279191504191e-06
1088 7.46389678241144e-06
1089 7.45682536607006e-06
1090 7.45048070260879e-06
1091 7.44424480547679e-06
1092 7.43799204006024e-06
1093 7.43182070017667e-06
1094 7.42586400015455e-06
1095 7.42016617572006e-06
1096 7.41467516363059e-06
1097 7.4092753274968e-06
1098 7.40384357733603e-06
1099 7.39829381135593e-06
1100 7.39257170412344e-06
1101 7.38670388766849e-06
1102 7.38076954043265e-06
1103 7.37490444358002e-06
1104 7.36930713518547e-06
1105 7.36425212100045e-06
1106 7.36012383661944e-06
1107 7.35745432933754e-06
1108 7.35702767151736e-06
1109 7.36005689994101e-06
1110 7.36844523530067e-06
1111 7.38523076293518e-06
1112 7.41520993585709e-06
1113 7.46569302645383e-06
1114 7.54680175152345e-06
1115 7.6693114641202e-06
1116 7.83577053997675e-06
1117 8.01990452803381e-06
1118 8.14338288179783e-06
1119 8.09549942282217e-06
1120 7.83997526543345e-06
1121 7.51737239834682e-06
1122 7.39322838771983e-06
1123 7.7358688477247e-06
1124 8.71283592385197e-06
1125 9.93927747838796e-06
1126 1.00041475539214e-05
1127 8.57240074303434e-06
1128 7.56795976819546e-06
1129 7.56850966165956e-06
1130 7.72744601817976e-06
1131 7.621183192974e-06
1132 7.39242857505928e-06
1133 7.24445188637901e-06
1134 7.20530697240207e-06
1135 7.21134713543847e-06
1136 7.21418009330677e-06
1137 7.20351330674163e-06
1138 7.18788355680289e-06
1139 7.17567995245361e-06
1140 7.16891955043764e-06
1141 7.16534112665812e-06
1142 7.16202156958445e-06
1143 7.15731727485291e-06
1144 7.15104549886547e-06
1145 7.14390768585815e-06
1146 7.13678298946974e-06
1147 7.13029190979597e-06
1148 7.12466435270898e-06
1149 7.11979920797745e-06
1150 7.1153792093881e-06
1151 7.11104167283594e-06
1152 7.1064823616851e-06
1153 7.10152610139403e-06
1154 7.09615484151271e-06
1155 7.09055327607899e-06
1156 7.08511331115602e-06
1157 7.08047679243151e-06
1158 7.07762521834354e-06
1159 7.07804767330344e-06
1160 7.08405934801704e-06
1161 7.09932621312248e-06
1162 7.12967288940547e-06
1163 7.18410951527915e-06
1164 7.27536459077527e-06
1165 7.41742540721901e-06
1166 7.61409463390095e-06
1167 7.83177695407034e-06
1168 7.96992969398502e-06
1169 7.89500140641053e-06
1170 7.58606219175562e-06
1171 7.23847915473641e-06
1172 7.14898431297684e-06
1173 7.5506677781334e-06
1174 8.49316498374719e-06
1175 9.44832229166082e-06
1176 9.25141251606476e-06
1177 7.99734849543654e-06
1178 7.23229829802108e-06
1179 7.25741919893608e-06
1180 7.40762458661948e-06
1181 7.34000021029857e-06
1182 7.14472625595874e-06
1183 6.99712411833531e-06
1184 6.94740138618144e-06
1185 6.95176851817081e-06
1186 6.96095174923528e-06
1187 6.9556091455425e-06
1188 6.93964563058813e-06
1189 6.92338382624058e-06
1190 6.91304110755908e-06
1191 6.90886496679255e-06
1192 6.90783461632094e-06
1193 6.90666463354361e-06
1194 6.90337646247485e-06
1195 6.89758898264614e-06
1196 6.89002560871899e-06
1197 6.88187928910011e-06
1198 6.87428127532641e-06
1199 6.86798172200298e-06
1200 6.86327628818262e-06
1201 6.86003307137639e-06
1202 6.85781350195924e-06
1203 6.85600618766102e-06
1204 6.85393920107913e-06
1205 6.85097552466374e-06
1206 6.84663165184674e-06
1207 6.84067927458898e-06
1208 6.83330498096879e-06
1209 6.82541298058226e-06
1210 6.81910958988396e-06
1211 6.81875203012439e-06
1212 6.83311375926365e-06
1213 6.8796962655604e-06
1214 6.99251524771682e-06
1215 7.23167475502606e-06
1216 7.67455306771581e-06
1217 8.31630039321851e-06
1218 8.82199547103113e-06
1219 8.57461793390968e-06
1220 7.64758697346934e-06
1221 7.04063401357757e-06
1222 7.32140676973358e-06
1223 8.14092946566092e-06
1224 8.58955367455394e-06
1225 8.06471763947769e-06
1226 7.22391329865957e-06
1227 6.85054928251105e-06
1228 6.8373074979533e-06
1229 6.85569569380817e-06
1230 6.81026585835909e-06
1231 6.74868087990887e-06
1232 6.71553371578426e-06
1233 6.71226794946023e-06
1234 6.71971096943835e-06
1235 6.72238643995016e-06
1236 6.71605163660161e-06
1237 6.70418871617784e-06
1238 6.69208120518405e-06
1239 6.6830516369798e-06
1240 6.67770141316026e-06
1241 6.67484536309715e-06
1242 6.67279164412449e-06
1243 6.67018896782778e-06
1244 6.6663535012168e-06
1245 6.66124752868313e-06
1246 6.65526175325226e-06
1247 6.64898345714704e-06
1248 6.64301603992001e-06
1249 6.63783092891279e-06
1250 6.63373542231227e-06
1251 6.63083596119662e-06
1252 6.629081377163e-06
1253 6.62827049247028e-06
1254 6.6280842814237e-06
1255 6.62810697971139e-06
1256 6.62782679228258e-06
1257 6.62666044792815e-06
1258 6.62398017059473e-06
1259 6.61913815980242e-06
1260 6.6116739971811e-06
1261 6.60160497645279e-06
1262 6.59025925431322e-06
1263 6.58218051130532e-06
1264 6.58990157331729e-06
1265 6.64561414742693e-06
1266 6.82846292354711e-06
1267 7.31264782949381e-06
1268 8.35656786168926e-06
1269 9.80594949862024e-06
1270 1.00862603913399e-05
1271 8.3093175726745e-06
1272 7.02801599050673e-06
1273 7.45326208040353e-06
1274 8.12611840927957e-06
1275 7.83149157523866e-06
1276 7.04166119769312e-06
1277 6.62329628031699e-06
1278 6.5479430642057e-06
1279 6.53527887006788e-06
1280 6.51141225604945e-06
1281 6.49650514628064e-06
1282 6.49618545578434e-06
1283 6.49924894080556e-06
1284 6.49748394643268e-06
1285 6.49068665703467e-06
1286 6.48216676069069e-06
1287 6.47451763668982e-06
1288 6.4684930771719e-06
1289 6.46371275392355e-06
1290 6.45954511169222e-06
1291 6.4555699847979e-06
1292 6.45162734835836e-06
1293 6.44770444235832e-06
1294 6.44382264036736e-06
1295 6.43998785632149e-06
1296 6.43619161966313e-06
1297 6.43240732234318e-06
1298 6.42862065314276e-06
1299 6.42484334267834e-06
1300 6.42109539539248e-06
1301 6.41742509843723e-06
1302 6.41390149347387e-06
1303 6.41060430250917e-06
1304 6.40762752013302e-06
1305 6.40509100557907e-06
1306 6.40314750555149e-06
1307 6.4019897201284e-06
1308 6.4018982488534e-06
1309 6.40327104539651e-06
1310 6.40670055318537e-06
1311 6.41309137927593e-06
1312 6.42381404025727e-06
1313 6.44095811530576e-06
1314 6.46767817569227e-06
1315 6.50851182193435e-06
1316 6.56940418863883e-06
1317 6.65639303676357e-06
1318 6.77117026315344e-06
1319 6.9010244709844e-06
1320 7.00582304835606e-06
1321 7.01883272613557e-06
1322 6.88881910626549e-06
1323 6.65175071157265e-06
1324 6.4588839969737e-06
1325 6.54040613357054e-06
1326 7.17758561830095e-06
1327 8.53111266962969e-06
1328 9.75349873755249e-06
1329 8.99072580828886e-06
1330 7.17620883872172e-06
1331 6.63643387355251e-06
1332 6.88587379205075e-06
1333 6.89771833428665e-06
1334 6.61436658466741e-06
1335 6.37202270947768e-06
1336 6.29007795271974e-06
1337 6.28967879068298e-06
1338 6.29340477065199e-06
1339 6.28267000157834e-06
1340 6.26776976941912e-06
1341 6.25804927345541e-06
1342 6.2541924807924e-06
1343 6.25260400211047e-06
1344 6.2502742359527e-06
1345 6.24632675227943e-06
1346 6.24136751570248e-06
1347 6.23634979124432e-06
1348 6.23188292436794e-06
1349 6.22809782990785e-06
1350 6.22477855705128e-06
1351 6.22161183239101e-06
1352 6.21835452507824e-06
1353 6.21490937557923e-06
1354 6.21133601486079e-06
1355 6.20779662563464e-06
1356 6.20453240207652e-06
1357 6.20180227084433e-06
1358 6.19987673067257e-06
1359 6.19902038678788e-06
1360 6.19949698732469e-06
1361 6.20160366304745e-06
1362 6.20568181064485e-06
1363 6.21215713758616e-06
1364 6.22156344620706e-06
1365 6.23455461656874e-06
1366 6.25187716440934e-06
1367 6.27427771471289e-06
1368 6.30224234843269e-06
1369 6.33543379402468e-06
1370 6.37183052498713e-06
1371 6.40646289662428e-06
1372 6.4305780327345e-06
1373 6.43218594476025e-06
1374 6.40009637953298e-06
1375 6.33188001586404e-06
1376 6.24426867146255e-06
1377 6.18511748129436e-06
1378 6.25569093903877e-06
1379 6.6616767679939e-06
1380 7.7408918057742e-06
1381 9.46680195834659e-06
1382 1.00123697395205e-05
1383 8.0414549521457e-06
1384 6.62404903373215e-06
1385 6.92178275230049e-06
1386 7.23691681692173e-06
1387 6.86826393270223e-06
1388 6.35871158216617e-06
1389 6.12843121983175e-06
1390 6.08960946424375e-06
1391 6.08558423165206e-06
1392 6.07373115402154e-06
1393 6.06406404868309e-06
1394 6.06250971557643e-06
1395 6.06440873474057e-06
1396 6.06430815519587e-06
1397 6.06072339692787e-06
1398 6.05503674178109e-06
1399 6.04909005019749e-06
1400 6.04390055780613e-06
1401 6.03964672540869e-06
1402 6.03607066063105e-06
1403 6.03287118794427e-06
1404 6.02986965203556e-06
1405 6.02702300245284e-06
1406 6.0243508914759e-06
1407 6.02187966292078e-06
1408 6.01961268431594e-06
1409 6.01751286799512e-06
1410 6.015514034452e-06
1411 6.01355208962318e-06
1412 6.01155704993417e-06
1413 6.00947275541941e-06
1414 6.00724884192161e-06
1415 6.00486866630945e-06
1416 6.00231665215389e-06
1417 5.99959411617945e-06
1418 5.99671689993642e-06
1419 5.99370277010181e-06
1420 5.99058155348331e-06
1421 5.98737823942486e-06
1422 5.98412469976495e-06
1423 5.98085882197452e-06
1424 5.97761884435499e-06
1425 5.97446015371261e-06
1426 5.97144529779214e-06
1427 5.96868418334395e-06
1428 5.96635204752616e-06
1429 5.96477686487518e-06
1430 5.96460948321109e-06
1431 5.96722933288163e-06
1432 5.97575692440344e-06
1433 5.99750260743548e-06
1434 6.05013521459696e-06
1435 6.17620050480383e-06
1436 6.47217308147319e-06
1437 7.11272118048001e-06
1438 8.20183827210386e-06
1439 9.10364756467885e-06
1440 8.41780142213366e-06
1441 6.80478512338212e-06
1442 6.53068514600008e-06
1443 7.64937333830318e-06
1444 8.35379057040342e-06
1445 7.48634229630341e-06
1446 6.35592220987746e-06
1447 5.98403285145821e-06
1448 5.94417669264047e-06
1449 5.9088994341927e-06
1450 5.88269969181709e-06
1451 5.88795966560696e-06
1452 5.90191785132532e-06
1453 5.90447550541384e-06
1454 5.89508755899359e-06
1455 5.88169980408892e-06
1456 5.8701227740876e-06
1457 5.86185551698648e-06
1458 5.85614353898478e-06
1459 5.85193740842982e-06
1460 5.84855873819734e-06
1461 5.84562682393042e-06
1462 5.8429146090333e-06
1463 5.84022409633178e-06
1464 5.83741865334275e-06
1465 5.83441361312609e-06
1466 5.83121147945675e-06
1467 5.82785370450978e-06
1468 5.82441248875298e-06
1469 5.82096701817747e-06
1470 5.81758116435793e-06
1471 5.81429472612527e-06
1472 5.81113787756493e-06
1473 5.80811171557727e-06
1474 5.80521703552606e-06
1475 5.8024475113605e-06
1476 5.79978893489042e-06
1477 5.7972461013911e-06
1478 5.79481082185751e-06
1479 5.79250803722786e-06
1480 5.7903564059103e-06
1481 5.78841435006083e-06
1482 5.78677408746842e-06
1483 5.78558871477952e-06
1484 5.78511836657469e-06
1485 5.78581443999582e-06
1486 5.7884988424739e-06
1487 5.79466221140024e-06
1488 5.80715461673265e-06
1489 5.83145778154659e-06
1490 5.87827541709274e-06
1491 5.96805593922056e-06
1492 6.13691114637049e-06
1493 6.43533920596795e-06
1494 6.8860675690452e-06
1495 7.34950832015357e-06
1496 7.42101949358087e-06
1497 6.84644823634173e-06
1498 6.14401925336239e-06
1499 6.11968877128888e-06
1500 7.04081955538172e-06
1501 8.27927132096562e-06
1502 8.28171588107551e-06
1503 6.90120702584451e-06
1504 5.97847936401052e-06
1505 5.87804325302699e-06
1506 5.90212652351596e-06
1507 5.81687785050633e-06
1508 5.72774018081645e-06
1509 5.702595497592e-06
1510 5.71602869436916e-06
1511 5.72775377749579e-06
1512 5.72326123826628e-06
1513 5.70846719494256e-06
1514 5.6932207237459e-06
1515 5.68247533117017e-06
1516 5.67613176860604e-06
1517 5.67221131042572e-06
1518 5.6690972027873e-06
1519 5.66614280295141e-06
1520 5.66336162899361e-06
1521 5.66092428799791e-06
1522 5.65888499037825e-06
1523 5.65710520872287e-06
1524 5.65533463525725e-06
1525 5.65331512403233e-06
1526 5.65086321602237e-06
1527 5.64791429313871e-06
1528 5.64453395623943e-06
1529 5.64088616750169e-06
1530 5.63722187552074e-06
1531 5.63384239304909e-06
1532 5.63110572082692e-06
1533 5.62943125181903e-06
1534 5.62931996661575e-06
1535 5.63145098197992e-06
1536 5.6367687477632e-06
1537 5.64666252378387e-06
1538 5.66321638073575e-06
1539 5.68956317348324e-06
1540 5.73018030358696e-06
1541 5.79094415309456e-06
1542 5.87789175199305e-06
1543 5.99289519565005e-06
1544 6.12395157917334e-06
1545 6.23235291774549e-06
1546 6.25292838796199e-06
1547 6.13399498483602e-06
1548 5.90734879679644e-06
1549 5.71674782223752e-06
1550 5.78086330627059e-06
1551 6.36088869176987e-06
1552 7.61095348611462e-06
1553 8.79965120015314e-06
1554 8.20002987111934e-06
1555 6.47740348469483e-06
1556 5.87508113270019e-06
1557 6.11334522604423e-06
1558 6.17482500819833e-06
1559 5.91581821485221e-06
1560 5.65993339218807e-06
1561 5.56382498428931e-06
1562 5.56337305823718e-06
1563 5.57244365761989e-06
1564 5.56340358892626e-06
1565 5.54588129642042e-06
1566 5.5327963099927e-06
1567 5.52737397230274e-06
1568 5.52618959792639e-06
1569 5.52524118413444e-06
1570 5.52275136378455e-06
1571 5.51898972789289e-06
1572 5.51499783396636e-06
1573 5.51160265471751e-06
1574 5.50906332952295e-06
1575 5.50715885117015e-06
1576 5.50543888255817e-06
1577 5.5035087291877e-06
1578 5.50113167419042e-06
1579 5.49829273843727e-06
1580 5.49517307746328e-06
1581 5.49207499123128e-06
1582 5.48939307565277e-06
1583 5.4875358976858e-06
1584 5.48693780189069e-06
1585 5.48803778466223e-06
1586 5.49133591443152e-06
1587 5.49742292932365e-06
1588 5.50704304291827e-06
1589 5.52114609897458e-06
1590 5.54093565208191e-06
1591 5.56781291605191e-06
1592 5.60311416020198e-06
1593 5.64742471675928e-06
1594 5.69922623805397e-06
1595 5.75274603464848e-06
1596 5.79569534631119e-06
1597 5.80955022044805e-06
1598 5.77587571592986e-06
1599 5.69022626084603e-06
1600 5.57833795955531e-06
1601 5.50859501702305e-06
1602 5.61004554988642e-06
1603 6.1136044871013e-06
1604 7.31119594732377e-06
1605 8.85081528068099e-06
1606 8.75179384784985e-06
1607 6.76670103016974e-06
1608 5.84875323728085e-06
1609 6.2164361622763e-06
1610 6.39283371928201e-06
1611 6.01592699966957e-06
1612 5.60060208121271e-06
1613 5.43606097380689e-06
1614 5.4192700247313e-06
1615 5.42013459092772e-06
1616 5.40697330242423e-06
1617 5.3932687653635e-06
1618 5.38820710627164e-06
1619 5.38892795587387e-06
1620 5.38964092999805e-06
1621 5.38767969393561e-06
1622 5.38358843904518e-06
1623 5.37898234176382e-06
1624 5.37499762609528e-06
1625 5.37192099736572e-06
1626 5.36948416929306e-06
1627 5.36729913580558e-06
1628 5.36510085424879e-06
1629 5.3628248810611e-06
1630 5.36055165367699e-06
1631 5.35842890014848e-06
1632 5.35657476774531e-06
1633 5.35505808807457e-06
1634 5.35388912314971e-06
1635 5.35302215309841e-06
1636 5.3523640790587e-06
1637 5.35181820549013e-06
1638 5.3512748041662e-06
1639 5.35062416062715e-06
1640 5.34978589339374e-06
1641 5.34867836732289e-06
1642 5.34723759981759e-06
1643 5.34541204588734e-06
1644 5.34316158473658e-06
1645 5.34044710587267e-06
1646 5.33724793116974e-06
1647 5.33354977427791e-06
1648 5.32938953679007e-06
1649 5.32489347992993e-06
1650 5.32043596290777e-06
1651 5.31700063932661e-06
1652 5.31705226380907e-06
1653 5.32681688225622e-06
1654 5.36190865219055e-06
1655 5.46136611134784e-06
1656 5.71886072053829e-06
1657 6.32504821762581e-06
1658 7.46844853338757e-06
1659 8.5951409900531e-06
1660 8.01593891974761e-06
1661 6.2543127454795e-06
1662 5.95838822459172e-06
1663 7.00526469366025e-06
1664 7.49277827916117e-06
1665 6.64919040360701e-06
1666 5.6806743611304e-06
1667 5.33549283554535e-06
1668 5.28906127383877e-06
1669 5.26806298539739e-06
1670 5.25186773803199e-06
1671 5.25803399265001e-06
1672 5.27342966583078e-06
1673 5.28093731322343e-06
1674 5.27700861363556e-06
1675 5.26698483982457e-06
1676 5.25657913463817e-06
1677 5.24847274396834e-06
1678 5.24291457004722e-06
1679 5.23927075146702e-06
1680 5.23691063536447e-06
1681 5.23538988561612e-06
1682 5.23439557609251e-06
1683 5.23367497518024e-06
1684 5.23297739229989e-06
1685 5.23211065495133e-06
1686 5.23094495807896e-06
1687 5.22943902359074e-06
1688 5.22759455678923e-06
1689 5.22547986214761e-06
1690 5.22317343687462e-06
1691 5.2207602601051e-06
1692 5.2183167125186e-06
1693 5.21591444435643e-06
1694 5.21360876692967e-06
1695 5.21145187315497e-06
1696 5.2094885516496e-06
1697 5.20777515866655e-06
1698 5.20638790879602e-06
1699 5.20542708581928e-06
1700 5.20506309031887e-06
1701 5.20555614524909e-06
1702 5.20732250963718e-06
1703 5.2110391628446e-06
1704 5.21782382101321e-06
1705 5.22958132798124e-06
1706 5.24949932012575e-06
1707 5.28300025681716e-06
1708 5.33896327414851e-06
1709 5.43087735360714e-06
1710 5.57545478363508e-06
1711 5.78192060363136e-06
1712 6.02168273156067e-06
1713 6.18727784829076e-06
1714 6.11891505597484e-06
1715 5.78148962482317e-06
1716 5.4122151547098e-06
1717 5.38650686987552e-06
1718 6.00581508747311e-06
1719 7.25449180505322e-06
1720 8.1090234087533e-06
1721 7.20822126787368e-06
1722 5.76007316555049e-06
1723 5.3802713355644e-06
1724 5.53998756558371e-06
1725 5.52815090681946e-06
1726 5.33401685354207e-06
1727 5.18039629371359e-06
1728 5.13767971455081e-06
1729 5.14894606240546e-06
1730 5.15733994133072e-06
1731 5.14802877038534e-06
1732 5.13115205436065e-06
1733 5.11807798497799e-06
1734 5.11206887487958e-06
1735 5.11053587493038e-06
1736 5.10990534241529e-06
1737 5.10824174826396e-06
1738 5.10541389830976e-06
1739 5.10217295790483e-06
1740 5.09931441783351e-06
1741 5.09721388031181e-06
1742 5.09578399432442e-06
1743 5.09467613607484e-06
1744 5.09345678523232e-06
1745 5.09180148755561e-06
1746 5.0895914402993e-06
1747 5.08691809653072e-06
1748 5.0840694263421e-06
1749 5.08149914590916e-06
1750 5.07978149499877e-06
1751 5.07963649498677e-06
1752 5.08193072468188e-06
1753 5.08777563812401e-06
1754 5.09866867925979e-06
1755 5.11665884372903e-06
1756 5.14456904365801e-06
1757 5.18604822019597e-06
1758 5.24516539934794e-06
1759 5.32467978553441e-06
1760 5.42180689588889e-06
1761 5.52107782159084e-06
1762 5.5881385723211e-06
1763 5.57700602810129e-06
1764 5.46238471743621e-06
1765 5.2815860338562e-06
1766 5.14688117103645e-06
1767 5.22862858742812e-06
1768 5.74909869133222e-06
1769 6.86895200852078e-06
1770 8.02334830707352e-06
1771 7.63338952713966e-06
1772 6.01735535177284e-06
1773 5.34840102428547e-06
1774 5.62184574937064e-06
1775 5.78236830417467e-06
1776 5.53208043951869e-06
1777 5.20514013180318e-06
1778 5.05175082698983e-06
1779 5.034896506384e-06
1780 5.04658987043882e-06
1781 5.04021067637694e-06
1782 5.02167788019037e-06
1783 5.00725568786109e-06
1784 5.00249408519338e-06
1785 5.00353083721095e-06
1786 5.00483661269868e-06
1787 5.00362698829804e-06
1788 5.00004319192726e-06
1789 4.99558848066073e-06
1790 4.99164754286241e-06
1791 4.98884901034558e-06
1792 4.98708214546895e-06
1793 4.98584390529189e-06
1794 4.98457505226924e-06
1795 4.98291329309453e-06
1796 4.98078342303643e-06
1797 4.97836236190707e-06
1798 4.97602474691661e-06
1799 4.97424505940813e-06
1800 4.97355424133872e-06
1801 4.97448369118914e-06
1802 4.97757251194386e-06
1803 4.98337537147009e-06
1804 4.99251302432668e-06
1805 5.00568568817528e-06
1806 5.02369361576882e-06
1807 5.0473221926417e-06
1808 5.07713123321096e-06
1809 5.11290056781633e-06
1810 5.15272657963806e-06
1811 5.1917935586232e-06
1812 5.22130416147348e-06
1813 5.22903730404423e-06
1814 5.20328747111165e-06
1815 5.14083023883671e-06
1816 5.05698860164827e-06
1817 4.99577737422818e-06
1818 5.04630501030334e-06
1819 5.38061669264067e-06
1820 6.27561935084842e-06
1821 7.75331523028555e-06
1822 8.42758040953129e-06
1823 6.8946122575575e-06
1824 5.44649136280384e-06
1825 5.59953039624972e-06
1826 6.03127271681103e-06
1827 5.81343666317125e-06
1828 5.289638313144e-06
1829 4.99011250010284e-06
1830 4.92822674136306e-06
1831 4.92998782597454e-06
1832 4.91995593310257e-06
1833 4.90418374177182e-06
1834 4.89741886111617e-06
1835 4.89919015755902e-06
1836 4.90219822779636e-06
1837 4.90192777791165e-06
1838 4.8983689793225e-06
1839 4.89355018196136e-06
1840 4.88919469487925e-06
1841 4.88594932868125e-06
1842 4.88364073181202e-06
1843 4.8818186342281e-06
1844 4.88013649402319e-06
1845 4.87847688512844e-06
1846 4.87690281048003e-06
1847 4.87554016315173e-06
1848 4.87449023767184e-06
1849 4.87379546632027e-06
1850 4.87341838217858e-06
1851 4.87326094589235e-06
1852 4.87319535791286e-06
1853 4.87308716179413e-06
1854 4.87280185845762e-06
1855 4.87223213108123e-06
1856 4.87129749249249e-06
1857 4.8699431745014e-06
1858 4.86814066391617e-06
1859 4.865885745442e-06
1860 4.86320406478669e-06
1861 4.86014754175201e-06
1862 4.8568258588233e-06
1863 4.85340682132573e-06
1864 4.85025310359433e-06
1865 4.84806878020549e-06
1866 4.84832918234979e-06
1867 4.85422755192033e-06
1868 4.87280654692945e-06
1869 4.91985104256187e-06
1870 5.03031542464782e-06
1871 5.27663010307933e-06
1872 5.77724542738522e-06
1873 6.583819779582e-06
1874 7.2533659998264e-06
1875 6.83024357561735e-06
1876 5.62525459990582e-06
1877 5.27358523072508e-06
1878 6.15388712965625e-06
1879 7.14001366564077e-06
1880 6.85315965620603e-06
1881 5.68070410444932e-06
1882 4.98847430585592e-06
1883 4.86980669922588e-06
1884 4.86388949805239e-06
1885 4.82437988158502e-06
1886 4.7961967655219e-06
1887 4.8022739780329e-06
1888 4.82081112096466e-06
1889 4.82970842741182e-06
1890 4.82487196284964e-06
1891 4.81296579568635e-06
1892 4.80099181743299e-06
1893 4.79212627402248e-06
1894 4.7864852703583e-06
1895 4.78311219476169e-06
1896 4.78116388613614e-06
1897 4.78021717897903e-06
1898 4.78006268034292e-06
1899 4.78050146401898e-06
1900 4.78127102709891e-06
1901 4.78205146148269e-06
1902 4.78253609337287e-06
1903 4.78249602409164e-06
1904 4.78181666307975e-06
1905 4.78046841712043e-06
1906 4.7785285453017e-06
1907 4.77611058036231e-06
1908 4.7733745436318e-06
1909 4.77046871916187e-06
1910 4.76752313938178e-06
1911 4.7646465486384e-06
1912 4.76191255471825e-06
1913 4.75936295019075e-06
1914 4.75704214664141e-06
1915 4.75496627005967e-06
1916 4.75316342885002e-06
1917 4.75169263203234e-06
1918 4.75069706062925e-06
1919 4.75048354364738e-06
1920 4.75168806657322e-06
1921 4.75568811753035e-06
1922 4.76535420501634e-06
1923 4.78682969973221e-06
1924 4.83323085909149e-06
1925 4.93204514029344e-06
1926 5.13685495917571e-06
1927 5.53096717270662e-06
1928 6.15945903437165e-06
1929 6.77605685983629e-06
1930 6.70325426832541e-06
1931 5.77935836076904e-06
1932 5.07716423392424e-06
1933 5.43387785167937e-06
1934 6.50299324833625e-06
1935 6.98133532184286e-06
1936 6.10184395943847e-06
1937 5.09284034322377e-06
1938 4.81606862301476e-06
1939 4.83796692307692e-06
1940 4.80621964893402e-06
1941 4.73740008599677e-06
1942 4.70515739925403e-06
1943 4.71080246966515e-06
1944 4.72325615197278e-06
1945 4.72421506469978e-06
1946 4.71446802530195e-06
1947 4.70214104208999e-06
1948 4.6929116546135e-06
1949 4.687791321345e-06
1950 4.68524959607031e-06
1951 4.68359873129209e-06
1952 4.68196567027235e-06
1953 4.68024956856716e-06
1954 4.67867091469287e-06
1955 4.677399481956e-06
1956 4.67643073487345e-06
1957 4.67559761063718e-06
1958 4.67468266962534e-06
1959 4.67348447719118e-06
1960 4.67190953168384e-06
1961 4.66997273296244e-06
1962 4.66780643360032e-06
1963 4.66563039291934e-06
1964 4.66373563301481e-06
1965 4.66246111829882e-06
1966 4.66220099082371e-06
1967 4.66343220395515e-06
1968 4.66675848143083e-06
1969 4.67300565221151e-06
1970 4.68334047631913e-06
1971 4.69948361825345e-06
1972 4.72393469097554e-06
1973 4.76021382311842e-06
1974 4.81283834208313e-06
1975 4.88642186580179e-06
1976 4.9825650161317e-06
1977 5.09297015716115e-06
1978 5.18979345853765e-06
1979 5.22319006734939e-06
1980 5.1455993681504e-06
1981 4.96441528285985e-06
1982 4.77901458695307e-06
1983 4.76895155943069e-06
1984 5.17791711951432e-06
1985 6.24649255964727e-06
1986 7.61044052977056e-06
1987 7.56199189311246e-06
1988 5.86333968044528e-06
1989 4.96482292300371e-06
1990 5.17836418278961e-06
1991 5.33105514533005e-06
1992 5.07719240117055e-06
1993 4.7674020673405e-06
1994 4.63668199413014e-06
1995 4.6292542894566e-06
1996 4.63942064476974e-06
1997 4.63062407440873e-06
1998 4.61326855583089e-06
1999 4.60171353533489e-06
};
\addlegendentry{Train}
\addplot [semithick, black]
table {%
0 0.0145604303106666
1 0.0141657190397382
2 0.0137925045564771
3 0.0134361693635583
4 0.0130913155153394
5 0.0127518763765693
6 0.0124105541035533
7 0.0120578072965145
8 0.0116811823099852
9 0.0112661179155111
10 0.0107977800071239
11 0.010263810865581
12 0.00965899135917425
13 0.00899131316691637
14 0.00828416924923658
15 0.00757121620699763
16 0.00688369525596499
17 0.00623902538791299
18 0.00564165553078055
19 0.00509126670658588
20 0.00458679720759392
21 0.00412655202671885
22 0.00370855932123959
23 0.00333086680620909
24 0.0029913098551333
25 0.00268715736456215
26 0.00241555017419159
27 0.00217362097464502
28 0.00195859558880329
29 0.00176787003874779
30 0.00159902207087725
31 0.00144978519529104
32 0.00131803960539401
33 0.00120182882528752
34 0.00109936483204365
35 0.00100902502890676
36 0.000929351430386305
37 0.000859040359500796
38 0.000796933134552091
39 0.00074200943345204
40 0.000693375186529011
41 0.000650248257443309
42 0.000611949828453362
43 0.000577887694817036
44 0.000547545438166708
45 0.000520472996868193
46 0.000496277934871614
47 0.000474618194857612
48 0.000455195084214211
49 0.000437747657997534
50 0.000422047305619344
51 0.00040789321064949
52 0.000395108974771574
53 0.000383539474569261
54 0.000373048183973879
55 0.000363514467608184
56 0.000354831863660365
57 0.000346906308550388
58 0.000339654827257618
59 0.000333004019921646
60 0.000326889014104381
61 0.000321252329740673
62 0.000316043297061697
63 0.000311216979753226
64 0.000306733621982858
65 0.0003025580663234
66 0.000298659026157111
67 0.000295008852845058
68 0.000291583186481148
69 0.000288360170088708
70 0.000285320536931977
71 0.000282447203062475
72 0.000279724918073043
73 0.000277139944955707
74 0.000274680409347638
75 0.000272335659246892
76 0.000270095915766433
77 0.000267952535068616
78 0.000265897891949862
79 0.0002639249432832
80 0.000262027548160404
81 0.000260199914919212
82 0.000258437066804618
83 0.000256733968853951
84 0.000255086284596473
85 0.000253490259638056
86 0.000251942343311384
87 0.000250439305091277
88 0.000248978089075536
89 0.000247555930400267
90 0.000246170238824561
91 0.000244818395003676
92 0.000243498128838837
93 0.000242207330302335
94 0.00024094432592392
95 0.000239707180298865
96 0.000238494147197343
97 0.000237303727772087
98 0.000236134379520081
99 0.000234984792768955
100 0.00023385402164422
101 0.000232740901992656
102 0.00023164419690147
103 0.000230562756769359
104 0.000229495417443104
105 0.000228441131184809
106 0.000227398966671899
107 0.000226367876166478
108 0.000225346942897886
109 0.000224335279199295
110 0.000223331924644299
111 0.000222336006117985
112 0.000221346635953523
113 0.000220363013795577
114 0.0002193844266003
115 0.000218410059460439
116 0.000217439170228317
117 0.000216471002204344
118 0.000215504944208078
119 0.000214540457818657
120 0.000213576902751811
121 0.000212613740586676
122 0.000211650316487066
123 0.000210685917409137
124 0.000209720179555006
125 0.00020875247719232
126 0.000207782431971282
127 0.000206809461815283
128 0.000205833173822612
129 0.000204852985916659
130 0.000203868650714867
131 0.000202879731659777
132 0.000201885617570952
133 0.000200885813683271
134 0.000199879927095026
135 0.000198867477593012
136 0.000197848217794672
137 0.000196821711142547
138 0.000195787710254081
139 0.000194745865883306
140 0.000193695959751494
141 0.000192637671716511
142 0.000191570914466865
143 0.000190495367860422
144 0.000189410828170367
145 0.000188316989806481
146 0.000187213780009188
147 0.000186100849532522
148 0.000184977950993925
149 0.000183845200808719
150 0.000182702686288394
151 0.000181550523848273
152 0.000180388116859831
153 0.000179216207470745
154 0.000178035028511658
155 0.000176844885572791
156 0.000175646331626922
157 0.000174440123373643
158 0.000173226013430394
159 0.000172004103660583
160 0.000170774859725498
161 0.000169538761838339
162 0.00016829623200465
163 0.000167047706781887
164 0.00016579260409344
165 0.000164532058988698
166 0.000163266624440439
167 0.00016199704259634
168 0.000160723808221519
169 0.000159447445184924
170 0.000158168521011248
171 0.000156887443154119
172 0.000155604851897806
173 0.000154320994624868
174 0.000153036569827236
175 0.000151752348756418
176 0.000150468986248598
177 0.000149186991620809
178 0.000147906946949661
179 0.000146628735819831
180 0.000145352663821541
181 0.000144079487654381
182 0.000142809745739214
183 0.000141543874633498
184 0.000140282179927453
185 0.00013902508362662
186 0.000137772978632711
187 0.000136526170535944
188 0.000135285066789947
189 0.000134049929329194
190 0.00013282110739965
191 0.00013159892114345
192 0.000130383574287407
193 0.00012917528511025
194 0.000127974708448164
195 0.000126782339066267
196 0.0001255985989701
197 0.000124423342640512
198 0.000123257210361771
199 0.000122100682347082
200 0.000120954056910705
201 0.00011981761053903
202 0.000118691728857812
203 0.000117576295451727
204 0.000116471303044818
205 0.000115376780740917
206 0.000114292830403429
207 0.000113219379272778
208 0.000112156696559396
209 0.000111105022369884
210 0.000110064647742547
211 0.000109036154753994
212 0.000108019215986133
213 0.000107013613160234
214 0.000106019193481188
215 0.000105035971500911
216 0.000104063961771317
217 0.000103102989669424
218 0.000102153047919273
219 0.00010121406376129
220 0.000100285862572491
221 9.93685098364949e-05
222 9.84618163784035e-05
223 9.75657530943863e-05
224 9.66801526374184e-05
225 9.58049713517539e-05
226 9.4940114649944e-05
227 9.40854879445396e-05
228 9.32409893721342e-05
229 9.24064734135754e-05
230 9.1581852757372e-05
231 9.07669964362867e-05
232 8.9961824414786e-05
233 8.91661766218022e-05
234 8.83800094015896e-05
235 8.76031699590385e-05
236 8.68355127749965e-05
237 8.60769287100993e-05
238 8.53273377288133e-05
239 8.45866743475199e-05
240 8.38547930470668e-05
241 8.31315628602169e-05
242 8.24169110273942e-05
243 8.17107429611497e-05
244 8.10129349702038e-05
245 8.03233706392348e-05
246 7.96418316895142e-05
247 7.89683181210421e-05
248 7.83026844146661e-05
249 7.76448287069798e-05
250 7.69946345826611e-05
251 7.63520220061764e-05
252 7.57167799747549e-05
253 7.50889012124389e-05
254 7.44682765798643e-05
255 7.38548042136244e-05
256 7.32483895262703e-05
257 7.2648937930353e-05
258 7.20562675269321e-05
259 7.14704001438804e-05
260 7.08911975380033e-05
261 7.03185505699366e-05
262 6.97524519637227e-05
263 6.91927562002093e-05
264 6.86393832438625e-05
265 6.80922530591488e-05
266 6.75512419547886e-05
267 6.70163426548243e-05
268 6.64874460198916e-05
269 6.59644574625418e-05
270 6.54472896712832e-05
271 6.49358989903703e-05
272 6.44301471766084e-05
273 6.3929968746379e-05
274 6.34353054920211e-05
275 6.29460773780011e-05
276 6.24621970928274e-05
277 6.19835991528817e-05
278 6.15102180745453e-05
279 6.10419956501573e-05
280 6.05788336542901e-05
281 6.01206702413037e-05
282 5.96674435655586e-05
283 5.92190954193939e-05
284 5.87755966989789e-05
285 5.83368891966529e-05
286 5.79029292566702e-05
287 5.74736113776453e-05
288 5.70489319215994e-05
289 5.66287671972532e-05
290 5.62130917387549e-05
291 5.58018655283377e-05
292 5.53950303583406e-05
293 5.49924770893995e-05
294 5.45942166354507e-05
295 5.42001725989394e-05
296 5.38103377039079e-05
297 5.34246537426952e-05
298 5.30430770595558e-05
299 5.26655530848075e-05
300 5.2292049076641e-05
301 5.19224995514378e-05
302 5.15568608534522e-05
303 5.11950893269386e-05
304 5.08372067997698e-05
305 5.04830641148146e-05
306 5.01327667734586e-05
307 4.97861910844222e-05
308 4.94432861160021e-05
309 4.91041137138382e-05
310 4.87685429106932e-05
311 4.84366064483766e-05
312 4.81082533951849e-05
313 4.77834219054785e-05
314 4.74620683235116e-05
315 4.71442581329029e-05
316 4.68298785563093e-05
317 4.65189077658579e-05
318 4.62113384855911e-05
319 4.59071561635938e-05
320 4.56063426099718e-05
321 4.53087923233397e-05
322 4.50145817012526e-05
323 4.47235761384945e-05
324 4.44357756350655e-05
325 4.41511620010715e-05
326 4.38696915807668e-05
327 4.35913243563846e-05
328 4.33160530519672e-05
329 4.30438340117689e-05
330 4.27746563218534e-05
331 4.25084472226445e-05
332 4.22452321799938e-05
333 4.19849493482616e-05
334 4.1727540519787e-05
335 4.14730129705276e-05
336 4.12213739764411e-05
337 4.09724707424175e-05
338 4.07264051318634e-05
339 4.04830934712663e-05
340 4.02424302592408e-05
341 4.00045137212146e-05
342 3.97692419937812e-05
343 3.95365896110889e-05
344 3.93065529351588e-05
345 3.90791174140759e-05
346 3.88542284781579e-05
347 3.86318570235744e-05
348 3.84120430680923e-05
349 3.81946629204322e-05
350 3.7979734770488e-05
351 3.7767236790387e-05
352 3.75571471522562e-05
353 3.73494331142865e-05
354 3.71440291928593e-05
355 3.69409608538263e-05
356 3.6740140785696e-05
357 3.65415689884685e-05
358 3.63453036698047e-05
359 3.61511629307643e-05
360 3.59592195309233e-05
361 3.57694480044302e-05
362 3.55817719537299e-05
363 3.53962532244623e-05
364 3.52128408849239e-05
365 3.50315058312844e-05
366 3.48522080457769e-05
367 3.46749475284014e-05
368 3.44996515195817e-05
369 3.43263745889999e-05
370 3.41550003213342e-05
371 3.39856123900972e-05
372 3.38180725520942e-05
373 3.36524353770074e-05
374 3.34886099153664e-05
375 3.33266179950442e-05
376 3.3166437788168e-05
377 3.30080147250555e-05
378 3.28513451677281e-05
379 3.26964363921434e-05
380 3.25432265526615e-05
381 3.23917338391766e-05
382 3.22418309224304e-05
383 3.20936997013632e-05
384 3.19470709655434e-05
385 3.1802093872102e-05
386 3.16586956614628e-05
387 3.15167962980922e-05
388 3.13765231112484e-05
389 3.12377160298638e-05
390 3.11004259856418e-05
391 3.0964602046879e-05
392 3.08302332996391e-05
393 3.06972688122187e-05
394 3.05657449644059e-05
395 3.04355853586458e-05
396 3.03068209177582e-05
397 3.01793934340822e-05
398 3.00532810797449e-05
399 2.99284874927253e-05
400 2.98050035780761e-05
401 2.96827693091473e-05
402 2.95618174277479e-05
403 2.94420715363231e-05
404 2.93235534627456e-05
405 2.92062395601533e-05
406 2.90900661639171e-05
407 2.89750860247295e-05
408 2.88612664007815e-05
409 2.87485472654225e-05
410 2.86369904642925e-05
411 2.8526459573186e-05
412 2.84170764643932e-05
413 2.83087338175392e-05
414 2.82014279946452e-05
415 2.80951735476265e-05
416 2.79899650195148e-05
417 2.78857405646704e-05
418 2.77824929071357e-05
419 2.76802293228684e-05
420 2.75789134320803e-05
421 2.74785506917397e-05
422 2.73791247309418e-05
423 2.72805900749518e-05
424 2.71829303528648e-05
425 2.708620922931e-05
426 2.69903139269445e-05
427 2.68952935584821e-05
428 2.68011317530181e-05
429 2.67077612079447e-05
430 2.66151982941665e-05
431 2.65234739345033e-05
432 2.64325262833154e-05
433 2.63423789874651e-05
434 2.62529192696093e-05
435 2.61643017438473e-05
436 2.60763463302283e-05
437 2.59891894529574e-05
438 2.59026874118717e-05
439 2.58169638982508e-05
440 2.57318733929424e-05
441 2.56474759225966e-05
442 2.55637169175316e-05
443 2.54806818702491e-05
444 2.53982652793638e-05
445 2.5316463506897e-05
446 2.52353038376896e-05
447 2.51548099186039e-05
448 2.50748980761273e-05
449 2.49955737672281e-05
450 2.49168660957366e-05
451 2.48387386818649e-05
452 2.47611696977401e-05
453 2.46841682383092e-05
454 2.46077088377206e-05
455 2.4531795133953e-05
456 2.44564034801442e-05
457 2.43815647991141e-05
458 2.43072099692654e-05
459 2.42333608184708e-05
460 2.41600664594444e-05
461 2.40871759160655e-05
462 2.40148128796136e-05
463 2.39428918575868e-05
464 2.38714255829109e-05
465 2.38004759012256e-05
466 2.37298772844952e-05
467 2.36597479670309e-05
468 2.35900515690446e-05
469 2.35207608056953e-05
470 2.34518811339512e-05
471 2.338341982977e-05
472 2.33153550652787e-05
473 2.32476359087741e-05
474 2.31803187489277e-05
475 2.31134054047288e-05
476 2.30468158406438e-05
477 2.29805973503971e-05
478 2.29147317440948e-05
479 2.28492262976943e-05
480 2.27840027946513e-05
481 2.27191976591712e-05
482 2.26546162593877e-05
483 2.25904168473789e-05
484 2.252652302559e-05
485 2.24629238800844e-05
486 2.2399586669053e-05
487 2.23365623241989e-05
488 2.22738162847236e-05
489 2.2211357645574e-05
490 2.21491354750469e-05
491 2.20872225327184e-05
492 2.20255515159806e-05
493 2.19641497096745e-05
494 2.19029989239061e-05
495 2.18420682358555e-05
496 2.1781372197438e-05
497 2.17209599213675e-05
498 2.1660809579771e-05
499 2.16008229472209e-05
500 2.15410382224945e-05
501 2.1481549993041e-05
502 2.14222018257715e-05
503 2.13630664802622e-05
504 2.13041985261953e-05
505 2.12454906431958e-05
506 2.11869555641897e-05
507 2.11286369449226e-05
508 2.10705238714581e-05
509 2.10125672310824e-05
510 2.0954828869435e-05
511 2.08971941901837e-05
512 2.08397523238091e-05
513 2.07825451070676e-05
514 2.07254615816055e-05
515 2.06685235752957e-05
516 2.06117565539898e-05
517 2.05551059480058e-05
518 2.04986572498456e-05
519 2.04423722607316e-05
520 2.03862000489607e-05
521 2.03301515284693e-05
522 2.02742721739924e-05
523 2.02185365196783e-05
524 2.01628954528132e-05
525 2.01074290089309e-05
526 2.0052068066434e-05
527 1.99968071683543e-05
528 1.99417354451725e-05
529 1.98867601284292e-05
530 1.98318903130712e-05
531 1.97771296370775e-05
532 1.97224999283208e-05
533 1.96680175577058e-05
534 1.96135970327305e-05
535 1.95592765521724e-05
536 1.9505103409756e-05
537 1.94510303117568e-05
538 1.93970372492913e-05
539 1.93431460502325e-05
540 1.92893749044742e-05
541 1.92356783372816e-05
542 1.91821254702518e-05
543 1.91286471817875e-05
544 1.90752671187511e-05
545 1.90219234355027e-05
546 1.89687143574702e-05
547 1.89155998668866e-05
548 1.88625563168898e-05
549 1.88096673809923e-05
550 1.87568239198299e-05
551 1.87040677701589e-05
552 1.86513716471381e-05
553 1.85987846634816e-05
554 1.85462613444543e-05
555 1.84938435268123e-05
556 1.84415002877358e-05
557 1.83892989298329e-05
558 1.83371321327286e-05
559 1.82850653800415e-05
560 1.82330695679411e-05
561 1.81811356014805e-05
562 1.81293398782145e-05
563 1.80776441993657e-05
564 1.8025975805358e-05
565 1.79744129127357e-05
566 1.79229373316048e-05
567 1.7871561794891e-05
568 1.7820266293711e-05
569 1.77690435521072e-05
570 1.77178771991748e-05
571 1.76668199856067e-05
572 1.76159173861379e-05
573 1.75650293385843e-05
574 1.75142358784797e-05
575 1.74635388248134e-05
576 1.74129145307234e-05
577 1.73623866430717e-05
578 1.73119697137736e-05
579 1.72616382769775e-05
580 1.72113977896515e-05
581 1.71612173289759e-05
582 1.71111823874526e-05
583 1.70612129295478e-05
584 1.70113380590919e-05
585 1.69615577760851e-05
586 1.69119139172835e-05
587 1.68623228091747e-05
588 1.68128390214406e-05
589 1.67634243553039e-05
590 1.67141242854996e-05
591 1.66649278980913e-05
592 1.66158370120684e-05
593 1.65668334375368e-05
594 1.65179353643907e-05
595 1.64691391546512e-05
596 1.64204648172017e-05
597 1.63718868861906e-05
598 1.63233962666709e-05
599 1.62750384333776e-05
600 1.62267679115757e-05
601 1.6178597434191e-05
602 1.61305033543613e-05
603 1.60825456987368e-05
604 1.60347026394447e-05
605 1.59869723574957e-05
606 1.59393857757095e-05
607 1.58918610395631e-05
608 1.58444381668232e-05
609 1.57971608132357e-05
610 1.57499598572031e-05
611 1.57029426191002e-05
612 1.5656027244404e-05
613 1.56091846292838e-05
614 1.5562460248475e-05
615 1.55158686538925e-05
616 1.54693807417061e-05
617 1.54230401676614e-05
618 1.53768432937795e-05
619 1.53307009895798e-05
620 1.52847205754369e-05
621 1.52388956848881e-05
622 1.51931726577459e-05
623 1.51476169776288e-05
624 1.51021104102256e-05
625 1.50567739183316e-05
626 1.50115456563071e-05
627 1.49664729178767e-05
628 1.49215966303018e-05
629 1.48767585415044e-05
630 1.48320341395447e-05
631 1.47874916365254e-05
632 1.47431073855842e-05
633 1.46988004416926e-05
634 1.46545753523242e-05
635 1.46104885061504e-05
636 1.45665344462031e-05
637 1.45227577377227e-05
638 1.44790737977019e-05
639 1.44355426527909e-05
640 1.43921461130958e-05
641 1.43488850881113e-05
642 1.43057641253108e-05
643 1.42627650348004e-05
644 1.42198377943714e-05
645 1.41771270136815e-05
646 1.41345435622497e-05
647 1.40921056299703e-05
648 1.40498113978538e-05
649 1.40076190291438e-05
650 1.39655749080703e-05
651 1.39236399263609e-05
652 1.38818049890688e-05
653 1.38400937430561e-05
654 1.37985180117539e-05
655 1.37570477818372e-05
656 1.37156903292635e-05
657 1.36744793053367e-05
658 1.36334538183291e-05
659 1.35925438371487e-05
660 1.35517730086576e-05
661 1.35111085910466e-05
662 1.34706206154078e-05
663 1.34302526930696e-05
664 1.33899329739506e-05
665 1.33497387651005e-05
666 1.33096054923953e-05
667 1.32695586216869e-05
668 1.32295690491446e-05
669 1.3189595847507e-05
670 1.31496099129436e-05
671 1.3109630344843e-05
672 1.306968897552e-05
673 1.30297266878188e-05
674 1.29897798615275e-05
675 1.29499212562223e-05
676 1.29100972117158e-05
677 1.28704041344463e-05
678 1.28308829516754e-05
679 1.27915782286436e-05
680 1.2752547263517e-05
681 1.27138609968824e-05
682 1.26756549434504e-05
683 1.26379936773446e-05
684 1.26010563690215e-05
685 1.25650403788313e-05
686 1.25302412925521e-05
687 1.24968810268911e-05
688 1.24654734463547e-05
689 1.24365724332165e-05
690 1.24109637908987e-05
691 1.23898698802805e-05
692 1.23749732665601e-05
693 1.2368709576549e-05
694 1.23747704492416e-05
695 1.23987774713896e-05
696 1.24488296933123e-05
697 1.25368342196452e-05
698 1.26781642393325e-05
699 1.28869414766086e-05
700 1.31573915496119e-05
701 1.34198899104376e-05
702 1.35006157506723e-05
703 1.32008071886958e-05
704 1.25283531815512e-05
705 1.17595400297432e-05
706 1.12107518361881e-05
707 1.11096487671603e-05
708 1.13707847049227e-05
709 1.13301312012482e-05
710 1.11200415631174e-05
711 1.15703951450996e-05
712 1.23203644761816e-05
713 1.2694515135081e-05
714 1.24868029161007e-05
715 1.19538117360207e-05
716 1.1468547199911e-05
717 1.11804929474602e-05
718 1.10570290416945e-05
719 1.10324826891883e-05
720 1.10601977212355e-05
721 1.11098461275105e-05
722 1.11597155409981e-05
723 1.11946601464297e-05
724 1.12067009467864e-05
725 1.11945355456555e-05
726 1.11621857286082e-05
727 1.11164017653209e-05
728 1.10645041786483e-05
729 1.1012621143891e-05
730 1.09649590740446e-05
731 1.09239299490582e-05
732 1.08904250737396e-05
733 1.08643116618623e-05
734 1.08448439277709e-05
735 1.08308267954271e-05
736 1.08208860183368e-05
737 1.08134145193617e-05
738 1.08067961264169e-05
739 1.07994037534809e-05
740 1.07896457848256e-05
741 1.07760761238751e-05
742 1.07574087451212e-05
743 1.07326441138866e-05
744 1.07011728687212e-05
745 1.06626466731541e-05
746 1.06172019513906e-05
747 1.05653398350114e-05
748 1.05076624095091e-05
749 1.04452037703595e-05
750 1.03788543128758e-05
751 1.03096317616291e-05
752 1.02385720310849e-05
753 1.01666382761323e-05
754 1.00949355328339e-05
755 1.00246761576273e-05
756 9.95731988950865e-06
757 9.8946566140512e-06
758 9.83890276984312e-06
759 9.7928295872407e-06
760 9.75978491624119e-06
761 9.74453814706067e-06
762 9.75509465206414e-06
763 9.80669665295864e-06
764 9.93057619780302e-06
765 1.01918267318979e-05
766 1.07272089735488e-05
767 1.18015013867989e-05
768 1.37137531055487e-05
769 1.57665926963091e-05
770 1.49788284034003e-05
771 1.14375334305805e-05
772 9.60400120675331e-06
773 9.4319411800825e-06
774 9.4541519501945e-06
775 9.44780276768142e-06
776 9.64445825957227e-06
777 9.92534023680491e-06
778 1.0069644304167e-05
779 1.0043157999462e-05
780 9.92617606243584e-06
781 9.79969354375498e-06
782 9.70392011367949e-06
783 9.64634364208905e-06
784 9.61987279879395e-06
785 9.61395926424302e-06
786 9.61873047344852e-06
787 9.62641934165731e-06
788 9.63166166911833e-06
789 9.63135244091973e-06
790 9.62440935836639e-06
791 9.61116893449798e-06
792 9.59282715484733e-06
793 9.57097654463723e-06
794 9.54722599999513e-06
795 9.52298432821408e-06
796 9.49931290961104e-06
797 9.47685657592956e-06
798 9.45611827773973e-06
799 9.43733539315872e-06
800 9.42040696827462e-06
801 9.40537393034901e-06
802 9.39190340432106e-06
803 9.37977893045172e-06
804 9.36872220336227e-06
805 9.35837942961371e-06
806 9.34843865252333e-06
807 9.3385260697687e-06
808 9.32830698729958e-06
809 9.31737213250017e-06
810 9.30527767195599e-06
811 9.29159978113603e-06
812 9.27576093090465e-06
813 9.2572654466494e-06
814 9.23532115848502e-06
815 9.20906131796073e-06
816 9.17745819606353e-06
817 9.1390729721752e-06
818 9.09208756638691e-06
819 9.03419731912436e-06
820 8.96271012607031e-06
821 8.87482201505918e-06
822 8.76915237313369e-06
823 8.64949834067374e-06
824 8.53339679451892e-06
825 8.46585589897586e-06
826 8.51901495479979e-06
827 8.71096199261956e-06
828 8.84343080542749e-06
829 8.72490727488184e-06
830 8.78689297678648e-06
831 9.58842974796426e-06
832 1.1094047295046e-05
833 1.26563518278999e-05
834 1.27928342408268e-05
835 1.10032060547383e-05
836 9.27510245674057e-06
837 8.59179999679327e-06
838 8.45350496092578e-06
839 8.50841388455592e-06
840 8.65845595399151e-06
841 8.84052224137122e-06
842 8.9895384007832e-06
843 9.06779314391315e-06
844 9.07345747691579e-06
845 9.0280645963503e-06
846 8.95938956091413e-06
847 8.88963040779345e-06
848 8.83137272467138e-06
849 8.78913124324754e-06
850 8.76258036441868e-06
851 8.74898523761658e-06
852 8.74486613611225e-06
853 8.74677607498597e-06
854 8.75164823810337e-06
855 8.75680325407302e-06
856 8.76019112183712e-06
857 8.76031936059007e-06
858 8.75625755725196e-06
859 8.74749366630567e-06
860 8.73396038514329e-06
861 8.71591782924952e-06
862 8.69389259605668e-06
863 8.66852042236133e-06
864 8.64047888171626e-06
865 8.61048101796769e-06
866 8.57922441355186e-06
867 8.54724839882692e-06
868 8.51508593768813e-06
869 8.4831626736559e-06
870 8.45188515086193e-06
871 8.42157169245183e-06
872 8.39248514239443e-06
873 8.3650247688638e-06
874 8.3394734247122e-06
875 8.31625311548123e-06
876 8.29595410323236e-06
877 8.27937856229255e-06
878 8.26755785965361e-06
879 8.26220548333367e-06
880 8.2659107647487e-06
881 8.28311385703273e-06
882 8.32150362839457e-06
883 8.39554923004471e-06
884 8.53398432809627e-06
885 8.79745584825287e-06
886 9.31753675104119e-06
887 1.03650463643135e-05
888 1.23195486594341e-05
889 1.47488854054245e-05
890 1.44985824590549e-05
891 1.0559561815171e-05
892 8.27808253234252e-06
893 8.21251433080761e-06
894 8.24825383460848e-06
895 7.99271492724074e-06
896 8.09882749308599e-06
897 8.43777070258511e-06
898 8.58076782606076e-06
899 8.50497326609911e-06
900 8.36090111988597e-06
901 8.2451342677814e-06
902 8.18032003735425e-06
903 8.15716521174181e-06
904 8.16027841210598e-06
905 8.17571253719507e-06
906 8.19291744846851e-06
907 8.20554032543441e-06
908 8.21094181446824e-06
909 8.20935383671895e-06
910 8.20251807454042e-06
911 8.19255183159839e-06
912 8.18134503788315e-06
913 8.17027648736257e-06
914 8.16009833215503e-06
915 8.15126713860082e-06
916 8.14372469903901e-06
917 8.13738552096765e-06
918 8.13197493698681e-06
919 8.12720736576011e-06
920 8.12278085504659e-06
921 8.11840163805755e-06
922 8.11386507848511e-06
923 8.10900746728294e-06
924 8.10363508207956e-06
925 8.09765606391011e-06
926 8.09099037724081e-06
927 8.08350978331873e-06
928 8.07514152256772e-06
929 8.06589196145069e-06
930 8.05565105110873e-06
931 8.04426963441074e-06
932 8.03165676188655e-06
933 8.01764781499514e-06
934 8.00193265604321e-06
935 7.98422388470499e-06
936 7.96401218394749e-06
937 7.94071638665628e-06
938 7.91352340456797e-06
939 7.88136094342917e-06
940 7.84308849688387e-06
941 7.79757101554424e-06
942 7.74424461269518e-06
943 7.68462814448867e-06
944 7.62479157856433e-06
945 7.57915040594526e-06
946 7.57200086809462e-06
947 7.62421905164956e-06
948 7.71238410379738e-06
949 7.74732779973419e-06
950 7.70811948314076e-06
951 7.82301958679454e-06
952 8.42656572785927e-06
953 9.73858823272167e-06
954 1.17071003842284e-05
955 1.31321103253867e-05
956 1.18295693027903e-05
957 9.10423568711849e-06
958 7.80378559284145e-06
959 7.53777976569836e-06
960 7.51283369027078e-06
961 7.60619195716572e-06
962 7.80987102189101e-06
963 8.02780414232984e-06
964 8.166496627382e-06
965 8.19995784695493e-06
966 8.15539260656806e-06
967 8.0762592915562e-06
968 7.99666486273054e-06
969 7.93411982158432e-06
970 7.89334353612503e-06
971 7.87244152888888e-06
972 7.86703549238155e-06
973 7.87228600529488e-06
974 7.88371653470676e-06
975 7.8975353972055e-06
976 7.91071306593949e-06
977 7.92111859482247e-06
978 7.92728314991109e-06
979 7.9284654930234e-06
980 7.92446371633559e-06
981 7.91551792644896e-06
982 7.90200647315942e-06
983 7.88462602940854e-06
984 7.86397595220478e-06
985 7.84077565185726e-06
986 7.81552171247313e-06
987 7.78876255935756e-06
988 7.7607701314264e-06
989 7.73185729485704e-06
990 7.70215774537064e-06
991 7.67181336414069e-06
992 7.64084325055592e-06
993 7.60921238907031e-06
994 7.57698899178649e-06
995 7.5443431342137e-06
996 7.51156994738267e-06
997 7.4794229476538e-06
998 7.44900671634241e-06
999 7.42201700631995e-06
1000 7.40046425562468e-06
1001 7.38639664632501e-06
1002 7.38123208066099e-06
1003 7.3860837801476e-06
1004 7.40419454814401e-06
1005 7.44781254979898e-06
1006 7.54911889089271e-06
1007 7.77812238084152e-06
1008 8.28029169497313e-06
1009 9.35615753405727e-06
1010 1.14770427899202e-05
1011 1.43370398291154e-05
1012 1.43617226058268e-05
1013 1.00577381090261e-05
1014 7.63874868425773e-06
1015 7.45335637475364e-06
1016 7.36823903935147e-06
1017 7.28198665456148e-06
1018 7.48307547837612e-06
1019 7.76068554841913e-06
1020 7.88177476351848e-06
1021 7.84655458119232e-06
1022 7.7501008490799e-06
1023 7.66238736105151e-06
1024 7.60783223086037e-06
1025 7.58528858568752e-06
1026 7.58540909373551e-06
1027 7.59812292017159e-06
1028 7.61501451052027e-06
1029 7.63016123528359e-06
1030 7.64041851653019e-06
1031 7.64480137149803e-06
1032 7.64377273299033e-06
1033 7.63869229558622e-06
1034 7.63092793931719e-06
1035 7.6218530011829e-06
1036 7.61248020353378e-06
1037 7.60339798944187e-06
1038 7.59509111958323e-06
1039 7.58777696319157e-06
1040 7.58147280066623e-06
1041 7.57611678636749e-06
1042 7.57159432396293e-06
1043 7.56772806198569e-06
1044 7.56428516979213e-06
1045 7.56119789002696e-06
1046 7.55823975850944e-06
1047 7.55520659367903e-06
1048 7.55192604628974e-06
1049 7.54823804527405e-06
1050 7.54388838686282e-06
1051 7.53858512325678e-06
1052 7.53202357373084e-06
1053 7.52382766222581e-06
1054 7.51338120608125e-06
1055 7.49995706428308e-06
1056 7.48245383874746e-06
1057 7.45951001590583e-06
1058 7.42913925932953e-06
1059 7.38900007490884e-06
1060 7.33653723727912e-06
1061 7.27044925952214e-06
1062 7.19468107490684e-06
1063 7.12780274625402e-06
1064 7.1198132900463e-06
1065 7.25829659131705e-06
1066 7.57802490625181e-06
1067 7.81103699409869e-06
1068 7.58170381232048e-06
1069 7.42539441489498e-06
1070 8.21090907265898e-06
1071 9.8440768852015e-06
1072 1.14331087388564e-05
1073 1.13506612251513e-05
1074 9.40277095651254e-06
1075 7.80548998591257e-06
1076 7.26415328244912e-06
1077 7.16719478077721e-06
1078 7.22094773664139e-06
1079 7.36442507331958e-06
1080 7.53340464143548e-06
1081 7.65988534112694e-06
1082 7.71311079006409e-06
1083 7.70124188420596e-06
1084 7.65118238632567e-06
1085 7.58933174438425e-06
1086 7.53302128941868e-06
1087 7.48995807953179e-06
1088 7.46165642340202e-06
1089 7.44637236493872e-06
1090 7.44120370654855e-06
1091 7.44309681977029e-06
1092 7.44919589124038e-06
1093 7.45707711757859e-06
1094 7.46484602132114e-06
1095 7.47101830711472e-06
1096 7.47460626371321e-06
1097 7.47495414543664e-06
1098 7.47160538594471e-06
1099 7.46450723454473e-06
1100 7.45359875509166e-06
1101 7.4389708970557e-06
1102 7.42070642445469e-06
1103 7.39887036615983e-06
1104 7.37351319912705e-06
1105 7.34455124984379e-06
1106 7.31179397917003e-06
1107 7.27501173969358e-06
1108 7.23386074241716e-06
1109 7.18824412615504e-06
1110 7.13838107913034e-06
1111 7.08578500052681e-06
1112 7.0340961428883e-06
1113 6.99070506016142e-06
1114 6.9673806137871e-06
1115 6.97663426763029e-06
1116 7.0191167651501e-06
1117 7.06703031028155e-06
1118 7.07987464920734e-06
1119 7.08261995896464e-06
1120 7.21955302651622e-06
1121 7.69036796555156e-06
1122 8.69260293256957e-06
1123 1.0353989637224e-05
1124 1.21101238619303e-05
1125 1.19366586659453e-05
1126 9.37458207772579e-06
1127 7.46876730772783e-06
1128 6.9935163082846e-06
1129 6.92940648150397e-06
1130 6.91438162903069e-06
1131 7.01088583809906e-06
1132 7.21900596545311e-06
1133 7.42682777854498e-06
1134 7.54533994040685e-06
1135 7.55923610995524e-06
1136 7.50142544347909e-06
1137 7.41522080716095e-06
1138 7.33223441784503e-06
1139 7.26767393643968e-06
1140 7.22520189810893e-06
1141 7.20281195754069e-06
1142 7.19657373338123e-06
1143 7.20234629625338e-06
1144 7.2162329161074e-06
1145 7.23482935427455e-06
1146 7.25516974853235e-06
1147 7.27471842765226e-06
1148 7.29146449884865e-06
1149 7.30387046132819e-06
1150 7.31078944227193e-06
1151 7.31148247723468e-06
1152 7.30543024474173e-06
1153 7.29250314179808e-06
1154 7.27259293853422e-06
1155 7.24575056665344e-06
1156 7.21200103726005e-06
1157 7.17139209882589e-06
1158 7.12406472302973e-06
1159 7.07024264556821e-06
1160 7.01079443388153e-06
1161 6.94800610290258e-06
1162 6.88687168803881e-06
1163 6.83711459714686e-06
1164 6.81380470268778e-06
1165 6.83248526911484e-06
1166 6.89129956299439e-06
1167 6.94928985467413e-06
1168 6.95353855917347e-06
1169 6.9462698775169e-06
1170 7.10775839252165e-06
1171 7.63916068535764e-06
1172 8.68602364789695e-06
1173 1.02423700809595e-05
1174 1.15702450784738e-05
1175 1.10074370240909e-05
1176 8.76553349371534e-06
1177 7.24896244719275e-06
1178 6.84243286741548e-06
1179 6.7786272666126e-06
1180 6.76881199979107e-06
1181 6.84782526150229e-06
1182 7.0302498897945e-06
1183 7.23642779121292e-06
1184 7.38129165256396e-06
1185 7.43105238143471e-06
1186 7.39881534173037e-06
1187 7.3195210461563e-06
1188 7.22753702575574e-06
1189 7.14519592293072e-06
1190 7.08230572854518e-06
1191 7.04047806721064e-06
1192 7.01763519828091e-06
1193 7.01051931173424e-06
1194 7.01592034602072e-06
1195 7.03096293364069e-06
1196 7.05309184922953e-06
1197 7.07996014170931e-06
1198 7.10938866177457e-06
1199 7.13932513463078e-06
1200 7.16783915777341e-06
1201 7.1930971898837e-06
1202 7.21336527931271e-06
1203 7.22692948329495e-06
1204 7.23217544873478e-06
1205 7.227362402773e-06
1206 7.21085962140933e-06
1207 7.18099818186602e-06
1208 7.13608824298717e-06
1209 7.07464414517744e-06
1210 6.99584188623703e-06
1211 6.90078104526037e-06
1212 6.79607819620287e-06
1213 6.70191047902335e-06
1214 6.66533924231771e-06
1215 6.76574836688815e-06
1216 7.04689682606841e-06
1217 7.30968758944073e-06
1218 7.17666489435942e-06
1219 6.93368065185496e-06
1220 7.37408390705241e-06
1221 8.56674250826472e-06
1222 9.93228240986355e-06
1223 1.04257651400985e-05
1224 9.35806565394159e-06
1225 7.82467941462528e-06
1226 7.00102964401594e-06
1227 6.74911416354007e-06
1228 6.7061941990687e-06
1229 6.75576166031533e-06
1230 6.87210012983996e-06
1231 7.01708313499694e-06
1232 7.14319321559742e-06
1233 7.21878404874587e-06
1234 7.23658331480692e-06
1235 7.2078451012203e-06
1236 7.15192254574504e-06
1237 7.08717379893642e-06
1238 7.0264009082166e-06
1239 6.97636141921976e-06
1240 6.93939364282414e-06
1241 6.91521609041956e-06
1242 6.9024354161229e-06
1243 6.89939224685077e-06
1244 6.90437082084827e-06
1245 6.9159259510343e-06
1246 6.93269339535618e-06
1247 6.95357675795094e-06
1248 6.97743143973639e-06
1249 7.00334976500017e-06
1250 7.0303767643054e-06
1251 7.0575893005298e-06
1252 7.08406923877192e-06
1253 7.10877611709293e-06
1254 7.13053304934874e-06
1255 7.14793395673041e-06
1256 7.15921942173736e-06
1257 7.16222484697937e-06
1258 7.15416217644815e-06
1259 7.13158442522399e-06
1260 7.09025107425987e-06
1261 7.02526904206024e-06
1262 6.93172341925674e-06
1263 6.80749917592038e-06
1264 6.66259666104452e-06
1265 6.54484074402717e-06
1266 6.596693310712e-06
1267 7.08519928593887e-06
1268 7.98556811787421e-06
1269 8.04969386081211e-06
1270 7.04125204720185e-06
1271 7.47089825381408e-06
1272 9.14173688215669e-06
1273 1.0101681255037e-05
1274 9.3240569185582e-06
1275 7.83710038376739e-06
1276 6.98714075042517e-06
1277 6.72883606966934e-06
1278 6.70605595587404e-06
1279 6.78269998388714e-06
1280 6.89630951455911e-06
1281 6.99282327332185e-06
1282 7.04237345416914e-06
1283 7.04470858181594e-06
1284 7.01628323440673e-06
1285 6.97578343533678e-06
1286 6.93664242135128e-06
1287 6.90552360538277e-06
1288 6.8841977736156e-06
1289 6.87181636749301e-06
1290 6.86628118273802e-06
1291 6.86525845594588e-06
1292 6.86667772242799e-06
1293 6.86879138811491e-06
1294 6.87030251356191e-06
1295 6.87038573232712e-06
1296 6.86849170961068e-06
1297 6.86443263475667e-06
1298 6.85813574818894e-06
1299 6.84973292663926e-06
1300 6.83935422784998e-06
1301 6.8271569944045e-06
1302 6.81325627738261e-06
1303 6.79780669088359e-06
1304 6.78085370964254e-06
1305 6.76236504659755e-06
1306 6.7423056862026e-06
1307 6.72043233862496e-06
1308 6.69658857077593e-06
1309 6.67049516778206e-06
1310 6.64189428789541e-06
1311 6.61062676954316e-06
1312 6.5769186221587e-06
1313 6.54174573355704e-06
1314 6.50737638352439e-06
1315 6.47824390398455e-06
1316 6.46119087832631e-06
1317 6.46397620585049e-06
1318 6.48945115244715e-06
1319 6.52676635581884e-06
1320 6.5522967815923e-06
1321 6.56138809063123e-06
1322 6.6171346588817e-06
1323 6.86024714013911e-06
1324 7.49071978134452e-06
1325 8.78057744557736e-06
1326 1.08625636130455e-05
1327 1.2514987247414e-05
1328 1.09823076854809e-05
1329 7.80959999246988e-06
1330 6.58465887681814e-06
1331 6.48212107989821e-06
1332 6.42887471258291e-06
1333 6.42635495751165e-06
1334 6.60094019622193e-06
1335 6.83781490806723e-06
1336 6.98665826348588e-06
1337 7.00850614521187e-06
1338 6.94501659381785e-06
1339 6.85364284436218e-06
1340 6.77245907354518e-06
1341 6.71617135594715e-06
1342 6.68542952553253e-06
1343 6.67552194499876e-06
1344 6.68041548124165e-06
1345 6.69452083457145e-06
1346 6.71290717946249e-06
1347 6.73154863761738e-06
1348 6.74747798257158e-06
1349 6.7586452132673e-06
1350 6.76407489663688e-06
1351 6.7635191953741e-06
1352 6.75722321830108e-06
1353 6.74579860060476e-06
1354 6.73011209073593e-06
1355 6.7110127019987e-06
1356 6.6892621362058e-06
1357 6.66556570649846e-06
1358 6.64046820020303e-06
1359 6.61438252791413e-06
1360 6.58759927318897e-06
1361 6.56038218949107e-06
1362 6.53291044727666e-06
1363 6.50542961011524e-06
1364 6.47832439426566e-06
1365 6.45217596684233e-06
1366 6.4278442550858e-06
1367 6.4065520746226e-06
1368 6.38981009615236e-06
1369 6.3791108004807e-06
1370 6.37547736914712e-06
1371 6.37905350231449e-06
1372 6.38964456811664e-06
1373 6.40974030829966e-06
1374 6.45058298687218e-06
1375 6.5412164076406e-06
1376 6.7427677095111e-06
1377 7.17933153282502e-06
1378 8.10127494332846e-06
1379 9.90215448837262e-06
1380 1.24426287584356e-05
1381 1.2994947610423e-05
1382 9.43142276810249e-06
1383 6.76995205139974e-06
1384 6.51861546430155e-06
1385 6.52874223305844e-06
1386 6.34053776593646e-06
1387 6.37862058283645e-06
1388 6.63903574604774e-06
1389 6.83125517753069e-06
1390 6.85371742292773e-06
1391 6.77219759381842e-06
1392 6.67159974909737e-06
1393 6.59568013361422e-06
1394 6.55344729239005e-06
1395 6.53939696348971e-06
1396 6.5447275119368e-06
1397 6.5609633566055e-06
1398 6.58089857097366e-06
1399 6.59921897749882e-06
1400 6.61261947243474e-06
1401 6.61971034787712e-06
1402 6.62054708300275e-06
1403 6.61613694319385e-06
1404 6.60786145090242e-06
1405 6.59719853501883e-06
1406 6.58534736430738e-06
1407 6.57330519970856e-06
1408 6.56174461255432e-06
1409 6.55110898151179e-06
1410 6.54160794510972e-06
1411 6.53335155220702e-06
1412 6.52632388664642e-06
1413 6.52049993732362e-06
1414 6.51577101962175e-06
1415 6.51201526125078e-06
1416 6.50910578769981e-06
1417 6.50691526971059e-06
1418 6.5053327489295e-06
1419 6.50417177894269e-06
1420 6.50331503493362e-06
1421 6.50260290058213e-06
1422 6.50181300443364e-06
1423 6.50071933705476e-06
1424 6.4990467762982e-06
1425 6.49632875138195e-06
1426 6.49200683255913e-06
1427 6.48515970169683e-06
1428 6.47453043711721e-06
1429 6.45822183287237e-06
1430 6.43360453977948e-06
1431 6.39735526419827e-06
1432 6.3466122810496e-06
1433 6.28326506557642e-06
1434 6.22706602371181e-06
1435 6.24794347459101e-06
1436 6.50873744234559e-06
1437 7.16261456545908e-06
1438 7.73880674387328e-06
1439 7.21560854799463e-06
1440 6.72866372042336e-06
1441 7.94592142483452e-06
1442 9.96931703411974e-06
1443 1.0732920600276e-05
1444 9.04963508219225e-06
1445 7.11952588972053e-06
1446 6.42220356894541e-06
1447 6.31192369837663e-06
1448 6.37622315480257e-06
1449 6.52444850857137e-06
1450 6.67355925543234e-06
1451 6.75804585625883e-06
1452 6.76775198371615e-06
1453 6.72947953717085e-06
1454 6.67584390612319e-06
1455 6.62814818497282e-06
1456 6.59485522191972e-06
1457 6.57629198030918e-06
1458 6.5692474890966e-06
1459 6.56946804156178e-06
1460 6.57314103591489e-06
1461 6.57727014186094e-06
1462 6.57992586639011e-06
1463 6.58008593745762e-06
1464 6.57748478261055e-06
1465 6.57230430078926e-06
1466 6.56499469187111e-06
1467 6.556114385603e-06
1468 6.54621862850036e-06
1469 6.53576216791407e-06
1470 6.52514518151293e-06
1471 6.51459549771971e-06
1472 6.50428273729631e-06
1473 6.49430421617581e-06
1474 6.48466266284231e-06
1475 6.47528986519319e-06
1476 6.46612716082018e-06
1477 6.45698537482531e-06
1478 6.4476948864467e-06
1479 6.43803696220857e-06
1480 6.42772010905901e-06
1481 6.41628776065772e-06
1482 6.40326561551774e-06
1483 6.38794563201373e-06
1484 6.36946697341045e-06
1485 6.34670095678302e-06
1486 6.31836201137048e-06
1487 6.28321731710457e-06
1488 6.2409594647761e-06
1489 6.19429738435429e-06
1490 6.15357339484035e-06
1491 6.14492819295265e-06
1492 6.21640447207028e-06
1493 6.41273481960525e-06
1494 6.66601044940762e-06
1495 6.71431689625024e-06
1496 6.4925802689686e-06
1497 6.55852409181534e-06
1498 7.44231920180027e-06
1499 9.13926305656787e-06
1500 1.08450585685205e-05
1501 1.05348881334066e-05
1502 8.13830229162704e-06
1503 6.54474979455699e-06
1504 6.16920488027972e-06
1505 6.1257928791747e-06
1506 6.17844943917589e-06
1507 6.33098034086288e-06
1508 6.51846085020225e-06
1509 6.65372090224992e-06
1510 6.70015424475423e-06
1511 6.67258564135409e-06
1512 6.60732075630222e-06
1513 6.53682536722044e-06
1514 6.47937940811971e-06
1515 6.44104147795588e-06
1516 6.4209416450467e-06
1517 6.41549377178308e-06
1518 6.42047643850674e-06
1519 6.43183557258453e-06
1520 6.446105089708e-06
1521 6.46045782559668e-06
1522 6.47271417619777e-06
1523 6.48132981950766e-06
1524 6.48546438242192e-06
1525 6.48465220365324e-06
1526 6.47893239147379e-06
1527 6.4685223151173e-06
1528 6.45386171527207e-06
1529 6.43542989564594e-06
1530 6.41374435872422e-06
1531 6.38915889794589e-06
1532 6.36204458714928e-06
1533 6.33255012871814e-06
1534 6.30082240604679e-06
1535 6.26696646577329e-06
1536 6.23116011411184e-06
1537 6.19391630607424e-06
1538 6.15650469626416e-06
1539 6.1214354900585e-06
1540 6.0933462009416e-06
1541 6.07904348726152e-06
1542 6.08596428719466e-06
1543 6.1162531892478e-06
1544 6.15783756074961e-06
1545 6.18494414084125e-06
1546 6.1890723372926e-06
1547 6.22803963779006e-06
1548 6.43623161522555e-06
1549 7.00212694937363e-06
1550 8.17460750113241e-06
1551 1.00837260106346e-05
1552 1.16835517474101e-05
1553 1.04421578726033e-05
1554 7.47307058190927e-06
1555 6.2197514125728e-06
1556 6.11928771832027e-06
1557 6.0806528381363e-06
1558 6.04792876401916e-06
1559 6.18359990767203e-06
1560 6.41353926766897e-06
1561 6.58082626614487e-06
1562 6.62350703350967e-06
1563 6.57002328807721e-06
1564 6.47788465357735e-06
1565 6.39067866359255e-06
1566 6.32766477792757e-06
1567 6.2914891714172e-06
1568 6.27785675533232e-06
1569 6.28099405730609e-06
1570 6.29522128292592e-06
1571 6.31551711194334e-06
1572 6.33754189038882e-06
1573 6.35792594039231e-06
1574 6.37410494164214e-06
1575 6.3844718170003e-06
1576 6.38829442323186e-06
1577 6.38552910459111e-06
1578 6.37660741631407e-06
1579 6.36228287476115e-06
1580 6.34350908512715e-06
1581 6.32118280918803e-06
1582 6.29619034953066e-06
1583 6.26924247626448e-06
1584 6.24093399892445e-06
1585 6.21170238446211e-06
1586 6.18198464508168e-06
1587 6.15214821664267e-06
1588 6.12272469879827e-06
1589 6.09448670729762e-06
1590 6.06861385676893e-06
1591 6.04680008109426e-06
1592 6.03120270170621e-06
1593 6.02390446147183e-06
1594 6.02585305387038e-06
1595 6.03561420575716e-06
1596 6.04965543971048e-06
1597 6.06703497396666e-06
1598 6.09965400144574e-06
1599 6.18475269220653e-06
1600 6.39864401819068e-06
1601 6.88480122335022e-06
1602 7.9082055890467e-06
1603 9.78982825472485e-06
1604 1.19644037113176e-05
1605 1.14965596367256e-05
1606 8.02275917521911e-06
1607 6.2344956859306e-06
1608 6.17423802395933e-06
1609 6.14816826782771e-06
1610 5.99601844442077e-06
1611 6.07236461291905e-06
1612 6.31874718237668e-06
1613 6.49060575597105e-06
1614 6.50891570330714e-06
1615 6.43007433609455e-06
1616 6.33044373898883e-06
1617 6.25249731456279e-06
1618 6.20672290096991e-06
1619 6.18884905634332e-06
1620 6.19110869592987e-06
1621 6.20581113253138e-06
1622 6.22628976998385e-06
1623 6.24705444352003e-06
1624 6.26423843641533e-06
1625 6.27559984422987e-06
1626 6.28043835604331e-06
1627 6.27901545158238e-06
1628 6.27248346063425e-06
1629 6.26213068244397e-06
1630 6.24938047621981e-06
1631 6.23540063315886e-06
1632 6.22114202997182e-06
1633 6.20728587819031e-06
1634 6.19437287241453e-06
1635 6.18264084550901e-06
1636 6.17229716226575e-06
1637 6.16342686043936e-06
1638 6.15605313214473e-06
1639 6.15016642768751e-06
1640 6.14574082646868e-06
1641 6.14278042121441e-06
1642 6.14137570664752e-06
1643 6.14155533185112e-06
1644 6.1435453062586e-06
1645 6.14765758655267e-06
1646 6.15443559581763e-06
1647 6.16476290815626e-06
1648 6.18012700215331e-06
1649 6.20306627752143e-06
1650 6.23807272859267e-06
1651 6.29335636403994e-06
1652 6.38475194136845e-06
1653 6.54398627375485e-06
1654 6.83757207298186e-06
1655 7.4035760917468e-06
1656 8.49132811708841e-06
1657 1.02943213278195e-05
1658 1.19050237117335e-05
1659 1.06773568404606e-05
1660 7.43618284104741e-06
1661 6.22114521320327e-06
1662 6.59756506138365e-06
1663 6.67465928927413e-06
1664 6.11709174336283e-06
1665 5.99504892306868e-06
1666 6.31271495876717e-06
1667 6.47049228064134e-06
1668 6.38661867924384e-06
1669 6.23337200522656e-06
1670 6.11939685768448e-06
1671 6.06156208959874e-06
1672 6.04502611167845e-06
1673 6.05420018473524e-06
1674 6.07665742791141e-06
1675 6.1021451074339e-06
1676 6.12322173765278e-06
1677 6.13616475675371e-06
1678 6.14065129411756e-06
1679 6.13844304098166e-06
1680 6.13203610555502e-06
1681 6.12388703302713e-06
1682 6.11569475950091e-06
1683 6.10861980021582e-06
1684 6.10317692917306e-06
1685 6.09951257501962e-06
1686 6.09743938184693e-06
1687 6.09671906204312e-06
1688 6.09697599429637e-06
1689 6.09785229244153e-06
1690 6.09901371717569e-06
1691 6.10015058555291e-06
1692 6.10092320130207e-06
1693 6.10108554610633e-06
1694 6.10042889093165e-06
1695 6.09868038736749e-06
1696 6.09566131970496e-06
1697 6.09109019933385e-06
1698 6.08469736107509e-06
1699 6.07621359449695e-06
1700 6.06524963586708e-06
1701 6.05139121034881e-06
1702 6.03414173383499e-06
1703 6.01306373937405e-06
1704 5.98791348238592e-06
1705 5.95900928601623e-06
1706 5.92814831179567e-06
1707 5.90005811318406e-06
1708 5.88450575378374e-06
1709 5.89759838476311e-06
1710 5.95667643210618e-06
1711 6.05860577707062e-06
1712 6.14767122897319e-06
1713 6.14146210864419e-06
1714 6.07692118137493e-06
1715 6.18953299635905e-06
1716 6.761210897821e-06
1717 8.00531051936559e-06
1718 9.83746303973021e-06
1719 1.0884727089433e-05
1720 9.21811624721158e-06
1721 6.79441973261419e-06
1722 5.96210657022311e-06
1723 5.88854072702816e-06
1724 5.85873613090371e-06
1725 5.88486136621214e-06
1726 6.03850412517204e-06
1727 6.23348341832752e-06
1728 6.36153481536894e-06
1729 6.38741448710789e-06
1730 6.33704348729225e-06
1731 6.25580241830903e-06
1732 6.17844443695503e-06
1733 6.12106759945164e-06
1734 6.08650771027897e-06
1735 6.07167794441921e-06
1736 6.07189440415823e-06
1737 6.08258915235638e-06
1738 6.09969720244408e-06
1739 6.11958103036159e-06
1740 6.13919610259472e-06
1741 6.15609042142751e-06
1742 6.16843999523553e-06
1743 6.17511477685184e-06
1744 6.17551768300473e-06
1745 6.16960733168526e-06
1746 6.15771114098607e-06
1747 6.1404130065057e-06
1748 6.1183800426079e-06
1749 6.09235621595872e-06
1750 6.06309367867652e-06
1751 6.03121679887408e-06
1752 5.99740769757773e-06
1753 5.96244808548363e-06
1754 5.92744936511735e-06
1755 5.89417368246359e-06
1756 5.86544911129749e-06
1757 5.84541567150154e-06
1758 5.83916289542685e-06
1759 5.85049019719008e-06
1760 5.877301646251e-06
1761 5.90727813687408e-06
1762 5.92366086493712e-06
1763 5.92970809520921e-06
1764 5.97727603235398e-06
1765 6.17171826888807e-06
1766 6.66886717226589e-06
1767 7.69222697272198e-06
1768 9.3909011411597e-06
1769 1.0950081559713e-05
1770 1.00764618764515e-05
1771 7.32432727090782e-06
1772 6.01145347900456e-06
1773 5.9301164583303e-06
1774 5.91996467846911e-06
1775 5.82542634219863e-06
1776 5.88112743571401e-06
1777 6.08917980571277e-06
1778 6.2767662711849e-06
1779 6.34392472420586e-06
1780 6.30161730441614e-06
1781 6.20857690591947e-06
1782 6.1156865740486e-06
1783 6.04705701334751e-06
1784 6.00671000938746e-06
1785 5.99023042013869e-06
1786 5.99155055169831e-06
1787 6.0050124375266e-06
1788 6.02564068685751e-06
1789 6.04909337198478e-06
1790 6.07165247856756e-06
1791 6.0903989833605e-06
1792 6.10334427619819e-06
1793 6.10944016443682e-06
1794 6.10843972026487e-06
1795 6.1007081058051e-06
1796 6.08710251981393e-06
1797 6.06866842645104e-06
1798 6.04655588176684e-06
1799 6.02179397901637e-06
1800 5.9953404161206e-06
1801 5.96799054619623e-06
1802 5.94047196500469e-06
1803 5.91339721722761e-06
1804 5.88741022511385e-06
1805 5.86330725127482e-06
1806 5.84204644837882e-06
1807 5.82487882638816e-06
1808 5.81314770897734e-06
1809 5.80796677240869e-06
1810 5.80958476348314e-06
1811 5.81681479161489e-06
1812 5.82759821554646e-06
1813 5.84192184760468e-06
1814 5.86806572755449e-06
1815 5.93093091083574e-06
1816 6.08338496022043e-06
1817 6.42918939774972e-06
1818 7.17092416380183e-06
1819 8.62698379933136e-06
1820 1.0750986803032e-05
1821 1.15276734504732e-05
1822 8.81302366906311e-06
1823 6.29155465503572e-06
1824 5.97890721110161e-06
1825 6.08399068369181e-06
1826 5.89333603784326e-06
1827 5.79205743633793e-06
1828 5.97397274759714e-06
1829 6.18811145614018e-06
1830 6.25743859927752e-06
1831 6.19988804828608e-06
1832 6.09905055171112e-06
1833 6.01285819357145e-06
1834 5.95929850533139e-06
1835 5.93566710449522e-06
1836 5.93395225223503e-06
1837 5.94631137573742e-06
1838 5.96588142798282e-06
1839 5.98693713982357e-06
1840 6.00508155912394e-06
1841 6.01767123953323e-06
1842 6.02367299507023e-06
1843 6.02338013777626e-06
1844 6.01786450715736e-06
1845 6.00859448240953e-06
1846 5.99703980697086e-06
1847 5.98452970734797e-06
1848 5.97204643781879e-06
1849 5.96032123212353e-06
1850 5.94980474488693e-06
1851 5.94080574956024e-06
1852 5.93342520005535e-06
1853 5.92776541452622e-06
1854 5.92378955843742e-06
1855 5.92150172451511e-06
1856 5.9209046412434e-06
1857 5.92203741689445e-06
1858 5.92500100538018e-06
1859 5.92997776038828e-06
1860 5.9372941905167e-06
1861 5.94755101701594e-06
1862 5.96167501498712e-06
1863 5.98125780015835e-06
1864 6.0088837017247e-06
1865 6.04912111157319e-06
1866 6.11008908890653e-06
1867 6.20678702034638e-06
1868 6.36780760032707e-06
1869 6.64834533381509e-06
1870 7.14973930371343e-06
1871 8.01824444351951e-06
1872 9.27273595152656e-06
1873 1.02028961919132e-05
1874 9.33316005102824e-06
1875 7.14226416675956e-06
1876 6.01436386205023e-06
1877 6.25522579866811e-06
1878 6.70466397423297e-06
1879 6.34279740552302e-06
1880 5.81369704377721e-06
1881 5.98431734033511e-06
1882 6.31653347227257e-06
1883 6.35713604424382e-06
1884 6.18811009189812e-06
1885 6.00348084844882e-06
1886 5.88652392252698e-06
1887 5.8329819694336e-06
1888 5.82024085815647e-06
1889 5.83209248361527e-06
1890 5.85731459068484e-06
1891 5.88623561270651e-06
1892 5.91105026614969e-06
1893 5.92714104641345e-06
1894 5.93331287745968e-06
1895 5.93106142332545e-06
1896 5.92305104873958e-06
1897 5.91208481637295e-06
1898 5.90052513871342e-06
1899 5.88996317674173e-06
1900 5.88131388212787e-06
1901 5.87503245697008e-06
1902 5.87111162531073e-06
1903 5.86944224778563e-06
1904 5.86972601013258e-06
1905 5.87172326049767e-06
1906 5.875021997781e-06
1907 5.87931117479457e-06
1908 5.88424200032023e-06
1909 5.88948023505509e-06
1910 5.89474257139955e-06
1911 5.89966839470435e-06
1912 5.90393074162421e-06
1913 5.90717263548868e-06
1914 5.90895660934621e-06
1915 5.90881018069922e-06
1916 5.9061326282972e-06
1917 5.90016043133801e-06
1918 5.88988814342883e-06
1919 5.87415024710936e-06
1920 5.8515233831713e-06
1921 5.82079928790336e-06
1922 5.78211165702669e-06
1923 5.73990382690681e-06
1924 5.7102738537651e-06
1925 5.7346910580236e-06
1926 5.89218188906671e-06
1927 6.25091934125521e-06
1928 6.64264598526643e-06
1929 6.55985013509053e-06
1930 6.0979914451309e-06
1931 6.29103851679247e-06
1932 7.49304581404431e-06
1933 9.07145386008779e-06
1934 9.51755191636039e-06
1935 7.95779396867147e-06
1936 6.33052195553319e-06
1937 5.78970457354444e-06
1938 5.70877637073863e-06
1939 5.72216276850668e-06
1940 5.80964433538611e-06
1941 5.94684843235882e-06
1942 6.06698995397892e-06
1943 6.12570829616743e-06
1944 6.12136045674561e-06
1945 6.07803576713195e-06
1946 6.02298086960218e-06
1947 5.9742842495325e-06
1948 5.93948925597942e-06
1949 5.91938760408084e-06
1950 5.91178286413196e-06
1951 5.91350726608653e-06
1952 5.92145124755916e-06
1953 5.93274126003962e-06
1954 5.94497214478906e-06
1955 5.95618757870398e-06
1956 5.96492645854596e-06
1957 5.97022062720498e-06
1958 5.97150437897653e-06
1959 5.96865857005469e-06
1960 5.96177096667816e-06
1961 5.95115625401377e-06
1962 5.9372960095061e-06
1963 5.92061542192823e-06
1964 5.90157696933602e-06
1965 5.88056718697771e-06
1966 5.85792804486118e-06
1967 5.83390965402941e-06
1968 5.80880350753432e-06
1969 5.78300705456058e-06
1970 5.75717604078818e-06
1971 5.7325892157678e-06
1972 5.71147711525555e-06
1973 5.69755547985551e-06
1974 5.69595067645423e-06
1975 5.71185637454619e-06
1976 5.74612295167753e-06
1977 5.78827302888385e-06
1978 5.81516587772057e-06
1979 5.8111768339586e-06
1980 5.80859523324762e-06
1981 5.90955960433348e-06
1982 6.27725239610299e-06
1983 7.14741327101365e-06
1984 8.76410285854945e-06
1985 1.06048055386054e-05
1986 1.02599060483044e-05
1987 7.43194186725304e-06
1988 5.89892306379625e-06
1989 5.8178729887004e-06
1990 5.80878440814558e-06
1991 5.692421382264e-06
1992 5.73298575545778e-06
1993 5.92234573559836e-06
1994 6.08310847383109e-06
1995 6.12767007623916e-06
1996 6.07803804086871e-06
1997 5.9932126532658e-06
1998 5.91670504945796e-06
1999 5.8650443861552e-06
};
\addlegendentry{Test}

\nextgroupplot[
title={ELU,Tanh},
ymin=2.2720948389334e-06, ymax=0.001,
]
\addplot [semithick, black, dashed]
table {%
0 0.00703803914075252
1 0.0069301391049521
2 0.00683485183981247
3 0.00675056807813235
4 0.00667576843261486
5 0.00660908265854232
6 0.00654927413415862
7 0.00649519253784092
8 0.00644569935684558
9 0.00639954226789996
10 0.00635517523187445
11 0.00631053247343516
12 0.00626266298058908
13 0.00620732950483216
14 0.0061387671448756
15 0.00604988438863074
16 0.00593326630041702
17 0.0057834329236357
18 0.00559865494869882
19 0.00538052847696235
20 0.00513184818555601
21 0.00485754402325256
22 0.0045661425574508
23 0.00426798704938847
24 0.00397281799450866
25 0.00368822045857087
26 0.00341910272618406
27 0.0031681123382441
28 0.00293626657366985
29 0.00272339884395478
30 0.00252864716640033
31 0.00235079809135641
32 0.00218847173346148
33 0.00204025895072846
34 0.00190480239871249
35 0.00178084443541593
36 0.00166724677455932
37 0.00156300080016081
38 0.00146721790770243
39 0.00137911033834826
40 0.00129798242505785
41 0.00122321884373378
42 0.00115427238233679
43 0.00109065279775677
44 0.00103191870493902
45 0.000977669652456825
46 0.000927540873817634
47 0.000881199603099958
48 0.000838340691188932
49 0.0007986832088136
50 0.000761969163704634
51 0.000727960971289576
52 0.000696440383308072
53 0.000667206586513203
54 0.000640075712908583
55 0.000614879381828359
56 0.000591463303862838
57 0.000569686099424871
58 0.000549418157334003
59 0.00053054143177178
60 0.000512948953200976
61 0.000496541456413979
62 0.000481226842794058
63 0.000466920376311464
64 0.00045354459552982
65 0.000441028320437908
66 0.00042930652944051
67 0.000418319251593857
68 0.000408011717354384
69 0.000398339627736277
70 0.000389280665331171
71 0.000380798448304631
72 0.000372847496237227
73 0.000365384364840793
74 0.000358370864205426
75 0.000351771586792893
76 0.000345553831607504
77 0.000339687840778424
78 0.000334146500563293
79 0.000328905295191362
80 0.000323941139299677
81 0.00031923271467349
82 0.000314760629066768
83 0.000310506929281473
84 0.000306454674387169
85 0.000302588350677979
86 0.000298893759463681
87 0.000295357694426457
88 0.000291968054284553
89 0.000288713778900274
90 0.000285584482298873
91 0.000282570675153693
92 0.000279663453397916
93 0.000276854790286052
94 0.000274137380188222
95 0.000271504335501049
96 0.000268949327960399
97 0.000266466492803374
98 0.000264050460486942
99 0.000261696442407811
100 0.000259399769390711
101 0.000257156162149386
102 0.000254961765222106
103 0.000252813019699261
104 0.000250706562667347
105 0.000248639395977079
106 0.000246608679049132
107 0.000244611750702006
108 0.000242646263814095
109 0.000240709990976029
110 0.000238800843703757
111 0.000236917000222547
112 0.000235056690655711
113 0.000233218310995653
114 0.000231400360860334
115 0.000229601525177259
116 0.000227820840109416
117 0.00022605730015357
118 0.00022430980743593
119 0.000222577542956515
120 0.000220859986484356
121 0.00021915629332625
122 0.000217465816888307
123 0.000215788050638821
124 0.000214122378679349
125 0.000212468303516289
126 0.000210825392969127
127 0.000209193529713048
128 0.000207572425154012
129 0.000205961850383574
130 0.000204361637088368
131 0.000202771637589194
132 0.000201191770088371
133 0.000199622172829095
134 0.000198062800109255
135 0.000196513702121592
136 0.000194974721921426
137 0.000193445833872374
138 0.000191926898537531
139 0.000190417812689248
140 0.000188918546314198
141 0.00018742903716884
142 0.000185949211811476
143 0.000184479102927071
144 0.000183018621783049
145 0.000181567855122466
146 0.000180126730242591
147 0.000178695293755027
148 0.000177273553902069
149 0.000175861666320998
150 0.000174459613788258
151 0.00017306746281065
152 0.000171685230895946
153 0.000170312833944308
154 0.000168950296142611
155 0.000167597534016295
156 0.000166254409066369
157 0.000164920846629002
158 0.00016359675817057
159 0.000162282086563437
160 0.000160976840987814
161 0.000159681242820398
162 0.000158395428371705
163 0.000157119453973564
164 0.000155853359871116
165 0.000154597173235516
166 0.000153350793056006
167 0.000152114224420075
168 0.000150887406363154
169 0.000149670290198856
170 0.000148462804418159
171 0.000147264927790047
172 0.000146076600884726
173 0.000144897735140148
174 0.000143728495686446
175 0.000142569119219615
176 0.000141419517916574
177 0.000140279618250361
178 0.000139149306306763
179 0.000138028505489274
180 0.000136917181464469
181 0.000135815251951499
182 0.000134722558499334
183 0.000133638983754736
184 0.000132564333483742
185 0.000131498494937432
186 0.000130441351586796
187 0.000129392782355353
188 0.000128352603724124
189 0.000127320568964251
190 0.000126296445955631
191 0.000125280131385352
192 0.000124271400096632
193 0.000123269941411763
194 0.000122275498796398
195 0.000121287794058844
196 0.000120306568618389
197 0.000119331597403516
198 0.000118362624647261
199 0.000117399351353242
200 0.000116441556940572
201 0.000115488927121987
202 0.000114541212582253
203 0.000113598038666396
204 0.000112659302800466
205 0.000111724627771537
206 0.000110793839269263
207 0.000109866693151162
208 0.000108943038043208
209 0.000108022607179237
210 0.000107105268881469
211 0.000106190818002005
212 0.000105279000194969
213 0.000104369572369478
214 0.000103462362204709
215 0.000102557240921897
216 0.000101654042737209
217 0.00010075267309162
218 9.98530019273858e-05
219 9.89549082532903e-05
220 9.80582786951345e-05
221 9.71630305741655e-05
222 9.62690245103204e-05
223 9.53761705488887e-05
224 9.44844561985292e-05
225 9.35937684403143e-05
226 9.27041478178126e-05
227 9.18155500784223e-05
228 9.09280088023934e-05
229 9.00415396785093e-05
230 8.91561723364021e-05
231 8.8272121459454e-05
232 8.73894534691999e-05
233 8.65082803045425e-05
234 8.56287368691255e-05
235 8.47509712684769e-05
236 8.38750055720539e-05
237 8.30010473009679e-05
238 8.21291481827302e-05
239 8.12595443306918e-05
240 8.03924955903312e-05
241 7.95303573113415e-05
242 7.86738319504821e-05
243 7.78242047090316e-05
244 7.6983423468846e-05
245 7.61513573905859e-05
246 7.53279203991042e-05
247 7.45132439163854e-05
248 7.37074745842392e-05
249 7.29107772627913e-05
250 7.21233318046188e-05
251 7.1345383219068e-05
252 7.05770457614108e-05
253 6.98185975522847e-05
254 6.9070176195396e-05
255 6.83319402270399e-05
256 6.76039789055949e-05
257 6.68865662447615e-05
258 6.6179693618551e-05
259 6.54836447750995e-05
260 6.47985336996726e-05
261 6.41243460108853e-05
262 6.34612166408033e-05
263 6.28091446657209e-05
264 6.21682228967302e-05
265 6.15384417699261e-05
266 6.09198421415158e-05
267 6.03123738045497e-05
268 5.97160472182168e-05
269 5.9130860535106e-05
270 5.85567767785733e-05
271 5.79937740496916e-05
272 5.74417421717044e-05
273 5.69006105024528e-05
274 5.63703546987426e-05
275 5.58507967269861e-05
276 5.53418693840513e-05
277 5.4843379984959e-05
278 5.43552501710565e-05
279 5.38772886216066e-05
280 5.34093685189418e-05
281 5.29509607076761e-05
282 5.25018689927492e-05
283 5.20623829061151e-05
284 5.16319743724125e-05
285 5.12104757177667e-05
286 5.07977409824889e-05
287 5.03935595048688e-05
288 4.99977821135644e-05
289 4.9610210176354e-05
290 4.92306769430684e-05
291 4.88589677871687e-05
292 4.84948835222099e-05
293 4.81382758934501e-05
294 4.77889394545628e-05
295 4.74467042010929e-05
296 4.71114148865581e-05
297 4.67828194103959e-05
298 4.64607949623996e-05
299 4.61452644415772e-05
300 4.58360569339789e-05
301 4.55329539974514e-05
302 4.52358319691371e-05
303 4.49444735366455e-05
304 4.46587533460274e-05
305 4.43785146941877e-05
306 4.41035570801773e-05
307 4.383377640238e-05
308 4.35690063795846e-05
309 4.33091322307178e-05
310 4.30540028730775e-05
311 4.28034855204373e-05
312 4.25574473013057e-05
313 4.23157495319515e-05
314 4.20783282280013e-05
315 4.18450020944761e-05
316 4.16156874649687e-05
317 4.1390260967944e-05
318 4.11685921619664e-05
319 4.09505981906477e-05
320 4.07361501828518e-05
321 4.05251774964199e-05
322 4.03175422576396e-05
323 4.01131946361488e-05
324 3.9911997404829e-05
325 3.97138622645343e-05
326 3.95187370472172e-05
327 3.93265247424779e-05
328 3.91371203036783e-05
329 3.89504952167385e-05
330 3.87665487480149e-05
331 3.85852241677753e-05
332 3.84064070999557e-05
333 3.82300722989726e-05
334 3.80561120678635e-05
335 3.788451437714e-05
336 3.77151871688852e-05
337 3.75480667926809e-05
338 3.73831132165492e-05
339 3.72202504479446e-05
340 3.70594217002918e-05
341 3.69005955676016e-05
342 3.67436962775969e-05
343 3.6588684267258e-05
344 3.64355459652188e-05
345 3.62841776819778e-05
346 3.61345636505916e-05
347 3.59866472336989e-05
348 3.58404298310688e-05
349 3.56958043212785e-05
350 3.55527857109905e-05
351 3.541131373197e-05
352 3.52713539797378e-05
353 3.51328597929523e-05
354 3.49958066152567e-05
355 3.48601805697513e-05
356 3.47259252606591e-05
357 3.45930056937505e-05
358 3.44614259759624e-05
359 3.43311334134455e-05
360 3.42021027961437e-05
361 3.40742894593404e-05
362 3.39476901061175e-05
363 3.38222674329813e-05
364 3.36979781039304e-05
365 3.35748002413538e-05
366 3.3452750841434e-05
367 3.33317444898285e-05
368 3.32117806536303e-05
369 3.30928474951975e-05
370 3.29749069720719e-05
371 3.28579376898119e-05
372 3.27419191776812e-05
373 3.26268413530784e-05
374 3.25126777056539e-05
375 3.2399395749394e-05
376 3.22869950650784e-05
377 3.21754359404736e-05
378 3.20647199743007e-05
379 3.19548269160919e-05
380 3.18457679497897e-05
381 3.17375050755686e-05
382 3.16300273297543e-05
383 3.1523300933145e-05
384 3.1417365242703e-05
385 3.13121664277105e-05
386 3.1207691876034e-05
387 3.11039401594826e-05
388 3.10008872261847e-05
389 3.08985312642562e-05
390 3.07968408250758e-05
391 3.06958039857363e-05
392 3.05954168737799e-05
393 3.04956660741595e-05
394 3.03965395644923e-05
395 3.02980113175977e-05
396 3.02000897320909e-05
397 3.01027421549804e-05
398 3.00059760505178e-05
399 2.9909767086167e-05
400 2.98141007064601e-05
401 2.9718980435689e-05
402 2.96243880768543e-05
403 2.95303179633777e-05
404 2.94367599664724e-05
405 2.93436902261135e-05
406 2.92511239976534e-05
407 2.91590189540614e-05
408 2.90674073006869e-05
409 2.8976237977929e-05
410 2.88855233385732e-05
411 2.87952644519862e-05
412 2.87054311876034e-05
413 2.86160263094359e-05
414 2.85270421258588e-05
415 2.84384735529386e-05
416 2.83503177520572e-05
417 2.82625471790254e-05
418 2.81751688291365e-05
419 2.80881864078708e-05
420 2.80015769504871e-05
421 2.7915339444462e-05
422 2.78294631748111e-05
423 2.77439505182997e-05
424 2.76587939467277e-05
425 2.75739644024497e-05
426 2.748951952114e-05
427 2.74053752384873e-05
428 2.73215813137995e-05
429 2.72381116559473e-05
430 2.7154975864363e-05
431 2.70721386144146e-05
432 2.69896157156779e-05
433 2.69074071219677e-05
434 2.68255080193569e-05
435 2.67438825183319e-05
436 2.66625532745479e-05
437 2.65815158826399e-05
438 2.65007605548817e-05
439 2.64202839268535e-05
440 2.63401023090637e-05
441 2.62601705536269e-05
442 2.61805247774305e-05
443 2.61011319508953e-05
444 2.60220129533195e-05
445 2.59431594109572e-05
446 2.58645555000214e-05
447 2.57862072707837e-05
448 2.57081242551749e-05
449 2.56303061938468e-05
450 2.55527325911942e-05
451 2.54753984343381e-05
452 2.53983271001346e-05
453 2.53214880459041e-05
454 2.52449009501277e-05
455 2.51685545151759e-05
456 2.50924668669938e-05
457 2.50166329536228e-05
458 2.49410355586122e-05
459 2.48656829064942e-05
460 2.47905643639967e-05
461 2.47156780694979e-05
462 2.46410160293919e-05
463 2.45666028888536e-05
464 2.44924115477829e-05
465 2.44184587572249e-05
466 2.43447469188141e-05
467 2.42712422391378e-05
468 2.41979798509817e-05
469 2.41249376635722e-05
470 2.40521114314163e-05
471 2.39794770671153e-05
472 2.39070633831773e-05
473 2.3834865256589e-05
474 2.37628711516891e-05
475 2.36910787521083e-05
476 2.3619474987413e-05
477 2.35480708781211e-05
478 2.34768738884839e-05
479 2.3405853053049e-05
480 2.33350410567823e-05
481 2.32644006423754e-05
482 2.31939512680412e-05
483 2.3123676104575e-05
484 2.30535891532213e-05
485 2.29836773293357e-05
486 2.29139636012121e-05
487 2.28444133405503e-05
488 2.27750445809249e-05
489 2.27058250956702e-05
490 2.26367858005005e-05
491 2.25679230858589e-05
492 2.24992289332704e-05
493 2.24307129137458e-05
494 2.23623514976623e-05
495 2.22941740943838e-05
496 2.22261543321167e-05
497 2.21582966304368e-05
498 2.2090592093349e-05
499 2.20230504943686e-05
500 2.19556698795031e-05
501 2.18884540892361e-05
502 2.18213971230341e-05
503 2.17545012226594e-05
504 2.16877546250771e-05
505 2.16211704575642e-05
506 2.15547275139727e-05
507 2.14884378806346e-05
508 2.1422298019047e-05
509 2.13563208895096e-05
510 2.12904697143301e-05
511 2.12247878756955e-05
512 2.11592536594196e-05
513 2.1093871279021e-05
514 2.10286359312306e-05
515 2.09635481489556e-05
516 2.08986105825204e-05
517 2.08338330800473e-05
518 2.07691964249079e-05
519 2.07047083726764e-05
520 2.06403722344817e-05
521 2.05761885716527e-05
522 2.0512155092689e-05
523 2.04482809529338e-05
524 2.03845419299853e-05
525 2.03209579403563e-05
526 2.02575043104503e-05
527 2.01942176651926e-05
528 2.01310761873685e-05
529 2.00680995767755e-05
530 2.00052605308088e-05
531 1.99425864195746e-05
532 1.98800455812886e-05
533 1.9817673688749e-05
534 1.97554322518556e-05
535 1.96933561547041e-05
536 1.96314127620667e-05
537 1.95696278133539e-05
538 1.95079910554341e-05
539 1.9446509750054e-05
540 1.93851893790509e-05
541 1.9324002206389e-05
542 1.92629832120872e-05
543 1.92021017717536e-05
544 1.91413937749019e-05
545 1.90808278688337e-05
546 1.90204117487269e-05
547 1.89601451552335e-05
548 1.890003015248e-05
549 1.88400727090254e-05
550 1.8780268380425e-05
551 1.87206157065134e-05
552 1.86611123460523e-05
553 1.86017759631341e-05
554 1.85425964041031e-05
555 1.8483562815419e-05
556 1.84246814001199e-05
557 1.83659580663686e-05
558 1.83073841739656e-05
559 1.82489705657929e-05
560 1.81907004730419e-05
561 1.81325823689349e-05
562 1.80746293665379e-05
563 1.8016825087841e-05
564 1.7959179931637e-05
565 1.79016816268529e-05
566 1.78443359821756e-05
567 1.77871435518284e-05
568 1.77301015504838e-05
569 1.76732194177021e-05
570 1.76164881544594e-05
571 1.75599078779953e-05
572 1.75034868767909e-05
573 1.74472134624182e-05
574 1.73911080452172e-05
575 1.73351578283132e-05
576 1.72793373778291e-05
577 1.7223697241775e-05
578 1.71682122456218e-05
579 1.71128689991917e-05
580 1.70576829212621e-05
581 1.70026586090444e-05
582 1.69477749736302e-05
583 1.68930489294894e-05
584 1.68384822067935e-05
585 1.67840801097441e-05
586 1.67298313513697e-05
587 1.66757393955663e-05
588 1.66218078021529e-05
589 1.65680285668657e-05
590 1.65144130690464e-05
591 1.64609539581306e-05
592 1.64076443027739e-05
593 1.63545012838995e-05
594 1.63015019509771e-05
595 1.62486798664929e-05
596 1.61960200095734e-05
597 1.61435055652248e-05
598 1.60911424771371e-05
599 1.60389505801106e-05
600 1.59869098368404e-05
601 1.59350248942758e-05
602 1.58832844476819e-05
603 1.58317092839866e-05
604 1.57802931788353e-05
605 1.57290379796393e-05
606 1.56779412456842e-05
607 1.5627009293695e-05
608 1.55762460494202e-05
609 1.55256424143602e-05
610 1.5475193286818e-05
611 1.54249122061856e-05
612 1.5374796031864e-05
613 1.53248411542961e-05
614 1.5275044631835e-05
615 1.52254077434577e-05
616 1.5175939473977e-05
617 1.51266315846499e-05
618 1.50774911560347e-05
619 1.50285171365283e-05
620 1.4979702708473e-05
621 1.49310587573837e-05
622 1.48825720742707e-05
623 1.483425372939e-05
624 1.47861012429473e-05
625 1.47381259587576e-05
626 1.4690306183951e-05
627 1.46426659135557e-05
628 1.45951712369197e-05
629 1.45478645094954e-05
630 1.45007059586533e-05
631 1.44537270223566e-05
632 1.44069109140332e-05
633 1.43602600353177e-05
634 1.43137833106266e-05
635 1.42674624115102e-05
636 1.42213159151083e-05
637 1.41753385065613e-05
638 1.41295251054885e-05
639 1.40838842064284e-05
640 1.40384102067515e-05
641 1.39931101692525e-05
642 1.39479854688318e-05
643 1.39030257919615e-05
644 1.38582384821007e-05
645 1.38136207503692e-05
646 1.3769165796873e-05
647 1.37248928098188e-05
648 1.3680793543358e-05
649 1.36368541774345e-05
650 1.35930936764339e-05
651 1.3549498628862e-05
652 1.35060846879753e-05
653 1.34628200179066e-05
654 1.34197576002748e-05
655 1.33768430004011e-05
656 1.33341056276493e-05
657 1.32915373995957e-05
658 1.32491420288261e-05
659 1.32069133300661e-05
660 1.3164858593484e-05
661 1.31229721276327e-05
662 1.30812512182388e-05
663 1.30396970696722e-05
664 1.29983123855482e-05
665 1.29570966826975e-05
666 1.29160512898352e-05
667 1.28751719579157e-05
668 1.28344534466862e-05
669 1.27938904626035e-05
670 1.27535190479477e-05
671 1.27133044074412e-05
672 1.26732538134888e-05
673 1.26333691987668e-05
674 1.25936563257767e-05
675 1.25540984399208e-05
676 1.25147102494338e-05
677 1.24754950228123e-05
678 1.24364331242077e-05
679 1.23975394856757e-05
680 1.23588000153774e-05
681 1.2320237523511e-05
682 1.22818439791672e-05
683 1.22436073404231e-05
684 1.22055350235684e-05
685 1.21676302384799e-05
686 1.21298848991813e-05
687 1.20923141864182e-05
688 1.20549033884032e-05
689 1.20176466431587e-05
690 1.19805624816394e-05
691 1.19436306409426e-05
692 1.1906862294353e-05
693 1.18702562996731e-05
694 1.18337922270229e-05
695 1.17975057492004e-05
696 1.17613679240236e-05
697 1.17253895126623e-05
698 1.16895621307123e-05
699 1.16538874301853e-05
700 1.1618369974542e-05
701 1.15830038840414e-05
702 1.15477978397394e-05
703 1.15127374176183e-05
704 1.14778331994358e-05
705 1.14430723918701e-05
706 1.14084722611096e-05
707 1.13740091460812e-05
708 1.13397067149634e-05
709 1.13055486696823e-05
710 1.12715413997933e-05
711 1.12376764747069e-05
712 1.12039729778246e-05
713 1.11703974248201e-05
714 1.11369779443038e-05
715 1.11036883652105e-05
716 1.1070550291592e-05
717 1.10375552146991e-05
718 1.10047088899279e-05
719 1.09720062759777e-05
720 1.09394330447543e-05
721 1.09070189520111e-05
722 1.08747479146132e-05
723 1.08426126530503e-05
724 1.08106231166971e-05
725 1.07787701448814e-05
726 1.07470554127076e-05
727 1.07154923103536e-05
728 1.06840655096363e-05
729 1.06527733869655e-05
730 1.06216284052607e-05
731 1.05906184213467e-05
732 1.0559754166195e-05
733 1.05290249106105e-05
734 1.04984395399299e-05
735 1.04680027650517e-05
736 1.04376973926179e-05
737 1.04075336917475e-05
738 1.03775135809059e-05
739 1.0347629304519e-05
740 1.0317887817024e-05
741 1.02882887276223e-05
742 1.02588357435707e-05
743 1.02295174926326e-05
744 1.0200341177935e-05
745 1.0171316427332e-05
746 1.01424148422069e-05
747 1.01136547723257e-05
748 1.00850436730582e-05
749 1.00565733429647e-05
750 1.00282395436579e-05
751 1.00000513132414e-05
752 9.97199317431807e-06
753 9.94407550614085e-06
754 9.91629045543618e-06
755 9.88863782502847e-06
756 9.86112536871531e-06
757 9.83374819618632e-06
758 9.80649704906966e-06
759 9.77938311663706e-06
760 9.75241021272666e-06
761 9.7255665476581e-06
762 9.69885678081539e-06
763 9.67227491521783e-06
764 9.64583098017613e-06
765 9.61950868116901e-06
766 9.59331779526451e-06
767 9.56725608780573e-06
768 9.54132101504968e-06
769 9.51551472638812e-06
770 9.48982770765383e-06
771 9.46426392545163e-06
772 9.43882506732052e-06
773 9.4135134460771e-06
774 9.38831376018356e-06
775 9.36324231659569e-06
776 9.33828616034305e-06
777 9.31345058496902e-06
778 9.28874158390158e-06
779 9.26414655566532e-06
780 9.2396779969306e-06
781 9.21532474507103e-06
782 9.19109285035802e-06
783 9.16697470820793e-06
784 9.14298247423062e-06
785 9.11910930589954e-06
786 9.09535562421127e-06
787 9.0717313021571e-06
788 9.04824248593172e-06
789 9.02488396548051e-06
790 9.00165949246912e-06
791 8.97857586767969e-06
792 8.95565183434144e-06
793 8.93289455738966e-06
794 8.91032298877548e-06
795 8.88795416909005e-06
796 8.86582568782046e-06
797 8.84399160749894e-06
798 8.82251132949818e-06
799 8.80148311654949e-06
800 8.7810543334399e-06
801 8.76141586569901e-06
802 8.74291711827624e-06
803 8.72604434842827e-06
804 8.71158925708215e-06
805 8.70084407011973e-06
806 8.69592662944285e-06
807 8.7003265685226e-06
808 8.7197181644072e-06
809 8.76288644136025e-06
810 8.84166437575828e-06
811 8.9664617801688e-06
812 9.13047111339438e-06
813 9.28035815217498e-06
814 9.30778191055026e-06
815 9.13019163828466e-06
816 8.81358723869852e-06
817 8.53735429284797e-06
818 8.42733057204725e-06
819 8.49100128075264e-06
820 8.65979915687376e-06
821 8.82238233224086e-06
822 8.8591051046194e-06
823 8.72697061371497e-06
824 8.50950373365578e-06
825 8.32843159148666e-06
826 8.23175940567467e-06
827 8.19404500429499e-06
828 8.17552867449223e-06
829 8.1550388078e-06
830 8.12941605321527e-06
831 8.10257306227413e-06
832 8.07788665557752e-06
833 8.05622233723824e-06
834 8.03682436956166e-06
835 8.0185859143711e-06
836 8.000736554159e-06
837 7.98288050951612e-06
838 7.96493774934959e-06
839 7.94694119843342e-06
840 7.92897324597419e-06
841 7.91111567011171e-06
842 7.8934099319028e-06
843 7.87587960804359e-06
844 7.85855184393114e-06
845 7.84141518650472e-06
846 7.82449130021234e-06
847 7.80778434172902e-06
848 7.79129274253165e-06
849 7.77503862714468e-06
850 7.75904333316646e-06
851 7.74333526631921e-06
852 7.72795017756067e-06
853 7.71293869572531e-06
854 7.6983837287159e-06
855 7.68437568776648e-06
856 7.67106055299394e-06
857 7.65863151830359e-06
858 7.64735478764322e-06
859 7.6376126152411e-06
860 7.62993888780983e-06
861 7.62507943719015e-06
862 7.6241016007117e-06
863 7.62845545487778e-06
864 7.6400826305445e-06
865 7.66138190311239e-06
866 7.69480373286058e-06
867 7.74172239026427e-06
868 7.79997596556115e-06
869 7.85992132268376e-06
870 7.90114200643899e-06
871 7.89539333823086e-06
872 7.82136632260233e-06
873 7.68484798907565e-06
874 7.52426040584453e-06
875 7.39299186225395e-06
876 7.33821286402758e-06
877 7.3971466543199e-06
878 7.60102198604784e-06
879 7.95031617606412e-06
880 8.33106092734681e-06
881 8.46984509372817e-06
882 8.18207555397521e-06
883 7.69201766637195e-06
884 7.35745008473287e-06
885 7.2490394744662e-06
886 7.24119816197799e-06
887 7.23204961250445e-06
888 7.19642723279179e-06
889 7.14973994497825e-06
890 7.10923843527667e-06
891 7.08040743901961e-06
892 7.06063614419605e-06
893 7.04560601327842e-06
894 7.03233722809671e-06
895 7.01944564873713e-06
896 7.00654328689154e-06
897 6.99356120126282e-06
898 6.98055343217874e-06
899 6.96750318951445e-06
900 6.95443026721421e-06
901 6.94132519551971e-06
902 6.92819987069271e-06
903 6.91503965288831e-06
904 6.90185340879168e-06
905 6.88864553755053e-06
906 6.8754198077059e-06
907 6.86218482037759e-06
908 6.84895517810702e-06
909 6.83572382342845e-06
910 6.82252254602389e-06
911 6.8093738354591e-06
912 6.79631679290083e-06
913 6.78341099646929e-06
914 6.77078939137488e-06
915 6.75863121291087e-06
916 6.74724504001745e-06
917 6.73715049703816e-06
918 6.7292163654642e-06
919 6.72486185404608e-06
920 6.72643756072233e-06
921 6.73764277081546e-06
922 6.76409517730292e-06
923 6.81342398767271e-06
924 6.89373550244454e-06
925 7.00792334917821e-06
926 7.14144427949037e-06
927 7.24844924171464e-06
928 7.25899894327142e-06
929 7.13134839092078e-06
930 6.9130193800504e-06
931 6.72266531598353e-06
932 6.66488196632287e-06
933 6.78073215176767e-06
934 7.04923464489582e-06
935 7.37706902143032e-06
936 7.58379244203411e-06
937 7.49936754118607e-06
938 7.16155358304604e-06
939 6.7958928573475e-06
940 6.56109620589262e-06
941 6.45656020026308e-06
942 6.42034551745496e-06
943 6.40693228959321e-06
944 6.39865095841685e-06
945 6.39216938314036e-06
946 6.38724817836334e-06
947 6.38299494859496e-06
948 6.37812115034109e-06
949 6.37170749939742e-06
950 6.36348381100049e-06
951 6.35364836476526e-06
952 6.34260590448577e-06
953 6.33077224421186e-06
954 6.31847631815674e-06
955 6.30597402384581e-06
956 6.29341806934747e-06
957 6.28089275256372e-06
958 6.26845969797785e-06
959 6.25617001048795e-06
960 6.24403577553778e-06
961 6.23212531447592e-06
962 6.22047460652198e-06
963 6.20920043026274e-06
964 6.19848757033026e-06
965 6.18861661472181e-06
966 6.1800933979228e-06
967 6.17372053302034e-06
968 6.17084971477055e-06
969 6.17365272148618e-06
970 6.18565898946599e-06
971 6.2123000450498e-06
972 6.26130685787984e-06
973 6.34184269188154e-06
974 6.46028662742992e-06
975 6.60902304261413e-06
976 6.74994543103935e-06
977 6.81040097116892e-06
978 6.72507953058954e-06
979 6.51108405769207e-06
980 6.28097390364957e-06
981 6.15770562184537e-06
982 6.19923630829078e-06
983 6.39467703411611e-06
984 6.67615749705419e-06
985 6.9127893045362e-06
986 6.94986488625204e-06
987 6.74151748913943e-06
988 6.42022625818583e-06
989 6.15608950127466e-06
990 6.00983970144853e-06
991 5.95057626462392e-06
992 5.93110725599644e-06
993 5.92319028314137e-06
994 5.91703647145891e-06
995 5.91131883176388e-06
996 5.90638077113681e-06
997 5.90211441231503e-06
998 5.89796882355387e-06
999 5.89342959411709e-06
1000 5.88816012214721e-06
1001 5.88208734608742e-06
1002 5.8752322313893e-06
1003 5.86770955379734e-06
1004 5.85960031251886e-06
1005 5.85103175421153e-06
1006 5.84207184939345e-06
1007 5.83279525656621e-06
1008 5.82325672038309e-06
1009 5.81352029538351e-06
1010 5.80365528435323e-06
1011 5.79375704745644e-06
1012 5.78397621797677e-06
1013 5.77454914196807e-06
1014 5.76592808876342e-06
1015 5.75888986986683e-06
1016 5.75480573239417e-06
1017 5.75617141063134e-06
1018 5.76734847879123e-06
1019 5.79596809302529e-06
1020 5.85456021262587e-06
1021 5.96155414900323e-06
1022 6.13719359154885e-06
1023 6.38455279577954e-06
1024 6.64774606651264e-06
1025 6.78154369104789e-06
1026 6.63937553380123e-06
1027 6.27246195072928e-06
1028 5.93364791257045e-06
1029 5.82107856228475e-06
1030 5.94434077516581e-06
1031 6.18815079356239e-06
1032 6.38959906495984e-06
1033 6.41174510018772e-06
1034 6.23917019915154e-06
1035 5.98704419196849e-06
1036 5.77979238025605e-06
1037 5.65946850406007e-06
1038 5.60530811144488e-06
1039 5.58470636891073e-06
1040 5.57697095526777e-06
1041 5.57345587104408e-06
1042 5.57141438539688e-06
1043 5.56977884302867e-06
1044 5.56778955118631e-06
1045 5.56490110836805e-06
1046 5.56084866998319e-06
1047 5.55564683057952e-06
1048 5.54946143171975e-06
1049 5.5425114848795e-06
1050 5.53499557343073e-06
1051 5.52709556300712e-06
1052 5.51893875577747e-06
1053 5.51062297748928e-06
1054 5.50223741413447e-06
1055 5.49383715275908e-06
1056 5.48548615864064e-06
1057 5.47727819988353e-06
1058 5.46934697576162e-06
1059 5.46189733885427e-06
1060 5.45528543183238e-06
1061 5.45010890373732e-06
1062 5.44739122521065e-06
1063 5.44892513820727e-06
1064 5.45779612437514e-06
1065 5.47919598714586e-06
1066 5.52165599643217e-06
1067 5.59797774313253e-06
1068 5.72421206435081e-06
1069 5.91126953786159e-06
1070 6.14170996460217e-06
1071 6.33668248273267e-06
1072 6.36367494610823e-06
1073 6.1511853370888e-06
1074 5.81228521623345e-06
1075 5.56235610105205e-06
1076 5.52859304336906e-06
1077 5.69866343536063e-06
1078 5.96999586122138e-06
1079 6.18767699123168e-06
1080 6.20712921595157e-06
1081 6.00917321413874e-06
1082 5.72455236191871e-06
1083 5.49611090816526e-06
1084 5.36936705675117e-06
1085 5.31692201422018e-06
1086 5.29965714557434e-06
1087 5.29407006943927e-06
1088 5.29142396032967e-06
1089 5.28977707281442e-06
1090 5.28872763316812e-06
1091 5.28781299857783e-06
1092 5.28649376496659e-06
1093 5.28438330205105e-06
1094 5.28136303268312e-06
1095 5.2774414180945e-06
1096 5.27275173167752e-06
1097 5.26742174189465e-06
1098 5.26159361458056e-06
1099 5.25536070572485e-06
1100 5.24881275998368e-06
1101 5.24203222251884e-06
1102 5.23507282723656e-06
1103 5.22799758062575e-06
1104 5.2208925382935e-06
1105 5.21386298579785e-06
1106 5.20710691009896e-06
1107 5.20095451683744e-06
1108 5.19599077275501e-06
1109 5.19325542658322e-06
1110 5.19466371606825e-06
1111 5.20362338862768e-06
1112 5.22624172560882e-06
1113 5.27310334508257e-06
1114 5.36126738381881e-06
1115 5.51416256833193e-06
1116 5.75133281799367e-06
1117 6.05399628739178e-06
1118 6.30982702443106e-06
1119 6.32359195407162e-06
1120 6.01194319838072e-06
1121 5.57980558735238e-06
1122 5.32563766331151e-06
1123 5.35739233242793e-06
1124 5.58707256281821e-06
1125 5.83889650407343e-06
1126 5.93691886585646e-06
1127 5.81251059195154e-06
1128 5.55812400371281e-06
1129 5.3213836941346e-06
1130 5.1743913149771e-06
1131 5.10658959607113e-06
1132 5.08199184512392e-06
1133 5.07458141285611e-06
1134 5.072835627562e-06
1135 5.073040119985e-06
1136 5.07388928072672e-06
1137 5.07447522091198e-06
1138 5.07411199679453e-06
1139 5.0724713065442e-06
1140 5.06953873724569e-06
1141 5.06550805523531e-06
1142 5.06063520155919e-06
1143 5.05516854376609e-06
1144 5.04929191480485e-06
1145 5.04316583649356e-06
1146 5.03688919950562e-06
1147 5.03052965639128e-06
1148 5.02416117686977e-06
1149 5.01782420769814e-06
1150 5.01158840959803e-06
1151 5.00551769100355e-06
1152 4.99976368795174e-06
1153 4.99455683788952e-06
1154 4.99029879685864e-06
1155 4.98771902446293e-06
1156 4.98807722992822e-06
1157 4.99363566941469e-06
1158 5.00839153039223e-06
1159 5.03927358197487e-06
1160 5.09785600177892e-06
1161 5.20173192031237e-06
1162 5.3721171395793e-06
1163 5.61878497773449e-06
1164 5.90205374173536e-06
1165 6.09103009807654e-06
1166 6.01621486673309e-06
1167 5.66820994274408e-06
1168 5.28096673990675e-06
1169 5.10190340641259e-06
1170 5.19186412351047e-06
1171 5.45458883571825e-06
1172 5.71714505603893e-06
1173 5.80242673642317e-06
1174 5.64861464891209e-06
1175 5.36711521936795e-06
1176 5.11827221894023e-06
1177 4.97139333122121e-06
1178 4.90787039053586e-06
1179 4.88684662469652e-06
1180 4.88133164289195e-06
1181 4.88044958979472e-06
1182 4.88127068720701e-06
1183 4.88282423072306e-06
1184 4.88430104361726e-06
1185 4.88497021544632e-06
1186 4.88443649970804e-06
1187 4.88264515041692e-06
1188 4.87974556762083e-06
1189 4.87595960318998e-06
1190 4.87152970274352e-06
1191 4.8666174290446e-06
1192 4.86138480182419e-06
1193 4.85592580901084e-06
1194 4.8503187786153e-06
1195 4.84460969341072e-06
1196 4.83885957347496e-06
1197 4.83312331311936e-06
1198 4.82744791474587e-06
1199 4.82194865147889e-06
1200 4.81681584529881e-06
1201 4.81238640404058e-06
1202 4.80926200374654e-06
1203 4.80853981565943e-06
1204 4.8122428810693e-06
1205 4.82407086366266e-06
1206 4.85070019884404e-06
1207 4.90391732732576e-06
1208 5.00303925576873e-06
1209 5.17486692341862e-06
1210 5.44157437909476e-06
1211 5.77905049059524e-06
1212 6.05166690359482e-06
1213 6.03502197460415e-06
1214 5.65995411783149e-06
1215 5.19163848045068e-06
1216 4.95634056418837e-06
1217 5.03598819978635e-06
1218 5.30175138102607e-06
1219 5.54541454711455e-06
1220 5.58918081594584e-06
1221 5.4067196373353e-06
1222 5.13317275574465e-06
1223 4.91302566141272e-06
1224 4.7910116083294e-06
1225 4.74042546194653e-06
1226 4.7242872138753e-06
1227 4.72087070724214e-06
1228 4.72179113231164e-06
1229 4.72437280674143e-06
1230 4.72719804722743e-06
1231 4.72920538374666e-06
1232 4.72973780407671e-06
1233 4.72864663025874e-06
1234 4.72611475954565e-06
1235 4.72247925520364e-06
1236 4.71807059421181e-06
1237 4.71318232309059e-06
1238 4.70801132257392e-06
1239 4.70269198915574e-06
1240 4.69731847152488e-06
1241 4.69195124663813e-06
1242 4.68661240082824e-06
1243 4.68133923092751e-06
1244 4.67615047927694e-06
1245 4.671094922859e-06
1246 4.66623850181591e-06
1247 4.66170381674402e-06
1248 4.65774435731703e-06
1249 4.65477444056006e-06
1250 4.65360436185946e-06
1251 4.65567456187088e-06
1252 4.66363687001703e-06
1253 4.68236332817895e-06
1254 4.72054529820909e-06
1255 4.7931598343709e-06
1256 4.92346403957455e-06
1257 5.13926193512759e-06
1258 5.44945917724249e-06
1259 5.78620031621568e-06
1260 5.95651488133342e-06
1261 5.76343871827589e-06
1262 5.29271376237261e-06
1263 4.89586615159965e-06
1264 4.81100243421295e-06
1265 5.00901700295486e-06
1266 5.30677118604217e-06
1267 5.48452095294749e-06
1268 5.41207934467991e-06
1269 5.15225681230191e-06
1270 4.87826190997964e-06
1271 4.69822624715732e-06
1272 4.61342458768499e-06
1273 4.58331442088067e-06
1274 4.57560224953468e-06
1275 4.57557252975249e-06
1276 4.57841301892969e-06
1277 4.58226106525572e-06
1278 4.58572086614595e-06
1279 4.58779495593475e-06
1280 4.58804464287255e-06
1281 4.58655242940864e-06
1282 4.58367125499848e-06
1283 4.57980557033721e-06
1284 4.57532212339018e-06
1285 4.57048801916216e-06
1286 4.56546242810774e-06
1287 4.56039071483616e-06
1288 4.55531506515783e-06
1289 4.55028877643926e-06
1290 4.54532821558828e-06
1291 4.5404473225652e-06
1292 4.53566098634894e-06
1293 4.53098794839235e-06
1294 4.52648430782077e-06
1295 4.52223511127059e-06
1296 4.51843612347957e-06
1297 4.51544334056919e-06
1298 4.51392164491082e-06
1299 4.51509869936473e-06
1300 4.52129114103528e-06
1301 4.53681219880764e-06
1302 4.56964882600275e-06
1303 4.63405505701786e-06
1304 4.75364871821427e-06
1305 4.96072828148897e-06
1306 5.27800699057224e-06
1307 5.66075721675219e-06
1308 5.92030154855649e-06
1309 5.79968835978661e-06
1310 5.30999275527577e-06
1311 4.83122362560096e-06
1312 4.68234752304397e-06
1313 4.86013666289864e-06
1314 5.16192138810823e-06
1315 5.34495236514942e-06
1316 5.27150017415323e-06
1317 5.01026125254711e-06
1318 4.73896779684679e-06
1319 4.56339613652545e-06
1320 4.48174728306938e-06
1321 4.45317657327138e-06
1322 4.44649145325826e-06
1323 4.44768498297776e-06
1324 4.45189904452192e-06
1325 4.45685065875168e-06
1326 4.46087716499122e-06
1327 4.46295099720828e-06
1328 4.46280038168823e-06
1329 4.46072177950896e-06
1330 4.45724431052952e-06
1331 4.45287101680236e-06
1332 4.44802969790103e-06
1333 4.44300105684192e-06
1334 4.43796179716571e-06
1335 4.43298693797445e-06
1336 4.42814837819938e-06
1337 4.42344106810211e-06
1338 4.41886499746857e-06
1339 4.41441411469512e-06
1340 4.41008555052136e-06
1341 4.4058659285362e-06
1342 4.4017641651628e-06
1343 4.39780556504843e-06
1344 4.39407852770302e-06
1345 4.39071239455302e-06
1346 4.38802507529701e-06
1347 4.38657908397033e-06
1348 4.38747475861589e-06
1349 4.39278457298187e-06
1350 4.40647602495403e-06
1351 4.4360515740749e-06
1352 4.49539061708748e-06
1353 4.60851855033084e-06
1354 4.81118312123385e-06
1355 5.13646745714524e-06
1356 5.55708694616186e-06
1357 5.88819265345819e-06
1358 5.82463567866398e-06
1359 5.31464361852585e-06
1360 4.76667991744506e-06
1361 4.57336627324167e-06
1362 4.74699410446533e-06
1363 5.05591840749275e-06
1364 5.23092577786599e-06
1365 5.13636809529316e-06
1366 4.86253131226633e-06
1367 4.59527889384859e-06
1368 4.43038286990216e-06
1369 4.35690050570869e-06
1370 4.33255473675231e-06
1371 4.32818494555676e-06
1372 4.33129149790545e-06
1373 4.33714552228182e-06
1374 4.34310071817379e-06
1375 4.3472895359109e-06
1376 4.3488210588194e-06
1377 4.34773679991451e-06
1378 4.34463324205936e-06
1379 4.34024856632576e-06
1380 4.33521515930124e-06
1381 4.3299674659103e-06
1382 4.32476763290168e-06
1383 4.31975559100906e-06
1384 4.31497528907698e-06
1385 4.31042438941986e-06
1386 4.30609711621344e-06
1387 4.30196000777983e-06
1388 4.29799541690201e-06
1389 4.29416603697774e-06
1390 4.29044992644734e-06
1391 4.28682077080467e-06
1392 4.28327384049965e-06
1393 4.27981062189886e-06
1394 4.27645170830715e-06
1395 4.27327601171257e-06
1396 4.27044151463107e-06
1397 4.26827272281827e-06
1398 4.26744802117796e-06
1399 4.26925023866431e-06
1400 4.27626502297329e-06
1401 4.29359151432607e-06
1402 4.33121196286024e-06
1403 4.40815350133406e-06
1404 4.55791065423483e-06
1405 4.82873389895389e-06
1406 5.2546466315917e-06
1407 5.75589455031889e-06
1408 6.02061598797832e-06
1409 5.70720771619904e-06
1410 5.01088335136401e-06
1411 4.53914395981414e-06
1412 4.54931245341683e-06
1413 4.84147066259766e-06
1414 5.08097842200073e-06
1415 5.05328102207159e-06
1416 4.80226667143313e-06
1417 4.52285030005939e-06
1418 4.33925598430918e-06
1419 4.25400635384321e-06
1420 4.22499381924624e-06
1421 4.22007572575822e-06
1422 4.2244531877067e-06
1423 4.2319574871108e-06
1424 4.23901252055714e-06
1425 4.24333757909068e-06
1426 4.2440922931597e-06
1427 4.24171050328326e-06
1428 4.23720431230734e-06
1429 4.23161583240628e-06
1430 4.22572497926765e-06
1431 4.21999441391208e-06
1432 4.21464124711335e-06
1433 4.20973428338911e-06
1434 4.20523890598545e-06
1435 4.20111936350054e-06
1436 4.19730182077416e-06
1437 4.1937451791263e-06
1438 4.19040086363509e-06
1439 4.18722345374434e-06
1440 4.18417928527148e-06
1441 4.18122660494902e-06
1442 4.17835218913254e-06
1443 4.17552897280515e-06
1444 4.17272859420947e-06
1445 4.16993618967965e-06
1446 4.16712321760215e-06
1447 4.16429338434554e-06
1448 4.16142962444965e-06
1449 4.15856154400984e-06
1450 4.15579498325158e-06
1451 4.15335192816002e-06
1452 4.15177034662761e-06
1453 4.1522335312294e-06
1454 4.15735196090949e-06
1455 4.17277287168005e-06
1456 4.21071133382966e-06
1457 4.29687171932258e-06
1458 4.4815034687673e-06
1459 4.8459694030889e-06
1460 5.45873760326998e-06
1461 6.18509039895798e-06
1462 6.4575681033574e-06
1463 5.79611744555564e-06
1464 4.83254858707483e-06
1465 4.46789621433297e-06
1466 4.66946259614076e-06
1467 4.92250027406271e-06
1468 4.89077042065134e-06
1469 4.62768911635436e-06
1470 4.35959354394555e-06
1471 4.20074053941732e-06
1472 4.1355441537938e-06
1473 4.11869457517255e-06
1474 4.12179194864137e-06
1475 4.13178071445586e-06
1476 4.14166741036226e-06
1477 4.14730481690384e-06
1478 4.14736064424659e-06
1479 4.14287886618681e-06
1480 4.13588056336067e-06
1481 4.12819971362666e-06
1482 4.12098782875958e-06
1483 4.1147246436779e-06
1484 4.1094643918882e-06
1485 4.10505904779868e-06
1486 4.10131091754806e-06
1487 4.09804391443558e-06
1488 4.09510645260625e-06
1489 4.09241853827069e-06
1490 4.08990188027758e-06
1491 4.0875091236181e-06
1492 4.08521941874795e-06
1493 4.08301319554383e-06
1494 4.08087787473477e-06
1495 4.07880127584193e-06
1496 4.07678477687412e-06
1497 4.07482228581557e-06
1498 4.07291753345973e-06
1499 4.07106426214554e-06
1500 4.06926642870786e-06
1501 4.06752414772171e-06
1502 4.06584688805722e-06
1503 4.06424552501505e-06
1504 4.0627414357175e-06
1505 4.06135094088711e-06
1506 4.06011802933648e-06
1507 4.05909853107289e-06
1508 4.05838725114904e-06
1509 4.05812989345833e-06
1510 4.05856773211255e-06
1511 4.06010676723412e-06
1512 4.06346576342642e-06
1513 4.06995674318011e-06
1514 4.08208280333611e-06
1515 4.10474844203534e-06
1516 4.14776499724212e-06
1517 4.23068069199672e-06
1518 4.39058761525501e-06
1519 4.68745895609857e-06
1520 5.17747472983388e-06
1521 5.78440379683798e-06
1522 6.1074673212147e-06
1523 5.6891640847212e-06
1524 4.83688347330258e-06
1525 4.33624823825696e-06
1526 4.41633891901105e-06
1527 4.77636252238511e-06
1528 5.01423713217264e-06
1529 4.91692217075013e-06
1530 4.5929466381267e-06
1531 4.28306609157048e-06
1532 4.10055059640868e-06
1533 4.02235611884905e-06
1534 3.99642516990539e-06
1535 3.99162922537499e-06
1536 3.99511408843978e-06
1537 4.00109343212662e-06
1538 4.00597401961633e-06
1539 4.00781401754813e-06
1540 4.00638496245165e-06
1541 4.00251867826995e-06
1542 3.99734584677347e-06
1543 3.99179926446891e-06
1544 3.9864360630304e-06
1545 3.98151233893529e-06
1546 3.97709879473851e-06
1547 3.97316068490028e-06
1548 3.96962960280689e-06
1549 3.96644190736595e-06
1550 3.96353101006497e-06
1551 3.96085672971225e-06
1552 3.95836635158631e-06
1553 3.95604503999891e-06
1554 3.95386997587011e-06
1555 3.95182738621713e-06
1556 3.94992280527973e-06
1557 3.94817389359758e-06
1558 3.94660092162269e-06
1559 3.945273252981e-06
1560 3.94428907801725e-06
1561 3.94383368318429e-06
1562 3.94423117455744e-06
1563 3.94605452003205e-06
1564 3.95031772226773e-06
1565 3.95888096260677e-06
1566 3.97520532136042e-06
1567 4.0056966192914e-06
1568 4.06217895942973e-06
1569 4.16549181636583e-06
1570 4.3483110463427e-06
1571 4.64703834346381e-06
1572 5.05672677952873e-06
1573 5.43394205543279e-06
1574 5.47777938031402e-06
1575 5.04163601444674e-06
1576 4.44455682835354e-06
1577 4.1166151576455e-06
1578 4.16354276566722e-06
1579 4.43329713828433e-06
1580 4.70060646673431e-06
1581 4.77906201101774e-06
1582 4.62494673048042e-06
1583 4.35617557537071e-06
1584 4.12171109376303e-06
1585 3.98219861841298e-06
1586 3.91938348021093e-06
1587 3.8962620383387e-06
1588 3.88852023514019e-06
1589 3.88610949109136e-06
1590 3.88602747447564e-06
1591 3.88719993082987e-06
1592 3.88880107271561e-06
1593 3.89016237001272e-06
1594 3.89089714936119e-06
1595 3.89089825336697e-06
1596 3.89023130908939e-06
1597 3.88903007619845e-06
1598 3.88743346357856e-06
1599 3.88554362773519e-06
1600 3.88345392021705e-06
1601 3.8812319100856e-06
1602 3.87890794284473e-06
1603 3.87652276856443e-06
1604 3.87409593938237e-06
1605 3.87165592874439e-06
1606 3.86923425010899e-06
1607 3.86685125453035e-06
1608 3.86458137557888e-06
1609 3.86252232598139e-06
1610 3.86086253278961e-06
1611 3.85995429130759e-06
1612 3.86044956912635e-06
1613 3.86361883819575e-06
1614 3.87187429384817e-06
1615 3.88994070554816e-06
1616 3.9270626790433e-06
1617 4.00095425590052e-06
1618 4.14380367619316e-06
1619 4.40524568467993e-06
1620 4.82999503859105e-06
1621 5.36148166574435e-06
1622 5.6973612565514e-06
1623 5.44587462769641e-06
1624 4.73317429205977e-06
1625 4.16968538408113e-06
1626 4.06232067939882e-06
1627 4.26659172347854e-06
1628 4.50445607302896e-06
1629 4.57260670660986e-06
1630 4.43764130952928e-06
1631 4.21102584047262e-06
1632 4.01625099311076e-06
1633 3.89906050379807e-06
1634 3.84401681174928e-06
1635 3.82233010332023e-06
1636 3.81489370115951e-06
1637 3.81310962804804e-06
1638 3.81371217272886e-06
1639 3.81518099024092e-06
1640 3.81658785819639e-06
1641 3.81737000321181e-06
1642 3.81731400622698e-06
1643 3.81647512259775e-06
1644 3.81500951718294e-06
1645 3.81309314123612e-06
1646 3.81087264123003e-06
1647 3.808470811828e-06
1648 3.80596579074854e-06
1649 3.80340672534452e-06
1650 3.80084448092788e-06
1651 3.79829525032349e-06
1652 3.79579631948346e-06
1653 3.79335385813917e-06
1654 3.79101945502391e-06
1655 3.7888307042877e-06
1656 3.78688928481807e-06
1657 3.78534620715243e-06
1658 3.7844921385588e-06
1659 3.7848368341642e-06
1660 3.78733350769878e-06
1661 3.7937506136565e-06
1662 3.80744189176774e-06
1663 3.83476312215691e-06
1664 3.88769921810805e-06
1665 3.98797249712146e-06
1666 4.17109432682139e-06
1667 4.48005811826135e-06
1668 4.91958960147088e-06
1669 5.34652147621273e-06
1670 5.42835567429734e-06
1671 4.97859424086755e-06
1672 4.33660512122458e-06
1673 3.98014368085953e-06
1674 4.01775727532083e-06
1675 4.27252937695854e-06
1676 4.50719583255932e-06
1677 4.55313831970727e-06
1678 4.39414130237736e-06
1679 4.14908938850544e-06
1680 3.9438233896405e-06
1681 3.82262482223616e-06
1682 3.76706964355655e-06
1683 3.74586065854388e-06
1684 3.7386634110792e-06
1685 3.73672205911113e-06
1686 3.73701678579863e-06
1687 3.7383409119407e-06
1688 3.73988910395973e-06
1689 3.74109776712928e-06
1690 3.74169737593633e-06
1691 3.74162970429026e-06
1692 3.74096804467428e-06
1693 3.7398362799923e-06
1694 3.73834122746608e-06
1695 3.73659443742014e-06
1696 3.7346557468787e-06
1697 3.73258091945772e-06
1698 3.73040931433977e-06
1699 3.72816532245857e-06
1700 3.72587401753499e-06
1701 3.7235450607298e-06
1702 3.72120996505743e-06
1703 3.71889580241813e-06
1704 3.71665115550535e-06
1705 3.71458369308186e-06
1706 3.71288047529283e-06
1707 3.71189503800551e-06
1708 3.71231362361613e-06
1709 3.71550427957779e-06
1710 3.72412785032417e-06
1711 3.74349051091016e-06
1712 3.7841835176522e-06
1713 3.86687947440123e-06
1714 4.02935078280819e-06
1715 4.32904417912994e-06
1716 4.81065852220297e-06
1717 5.38111535774988e-06
1718 5.66054837669583e-06
1719 5.26558812197209e-06
1720 4.48439401146139e-06
1721 3.98525944200756e-06
1722 3.9704779943861e-06
1723 4.2097466512736e-06
1724 4.40681476732152e-06
1725 4.39774522487113e-06
1726 4.2120335637108e-06
1727 3.98635899889754e-06
1728 3.82061718307014e-06
1729 3.73115337581886e-06
1730 3.69236743691381e-06
1731 3.67820567892529e-06
1732 3.67425247627118e-06
1733 3.67444994719968e-06
1734 3.67623826980079e-06
1735 3.67821047198014e-06
1736 3.67952938318261e-06
1737 3.67984519100695e-06
1738 3.67915983034983e-06
1739 3.67767086872917e-06
1740 3.67561137792904e-06
1741 3.67320501504231e-06
1742 3.6706010593246e-06
1743 3.66790620698332e-06
1744 3.66519061167914e-06
1745 3.66249971262e-06
1746 3.65986209871494e-06
1747 3.65729486651034e-06
1748 3.65481404562829e-06
1749 3.65243885180888e-06
1750 3.65020712456143e-06
1751 3.64817443165855e-06
1752 3.64643247285912e-06
1753 3.64516051232222e-06
1754 3.64469994029726e-06
1755 3.64562076926589e-06
1756 3.6490144832424e-06
1757 3.65691647385979e-06
1758 3.67315433136284e-06
1759 3.70497724055951e-06
1760 3.76590012951539e-06
1761 3.87992682826521e-06
1762 4.08437816457052e-06
1763 4.41816363938585e-06
1764 4.86400933397135e-06
1765 5.23869403323829e-06
1766 5.21173399370412e-06
1767 4.6959582347128e-06
1768 4.09547970914304e-06
1769 3.82361934736508e-06
1770 3.92089041523214e-06
1771 4.1960550198894e-06
1772 4.41987015076251e-06
1773 4.43866603028198e-06
1774 4.25527069314313e-06
1775 4.00002091183183e-06
1776 3.79616721346565e-06
1777 3.68001570216769e-06
1778 3.62841384893287e-06
1779 3.60935171306309e-06
1780 3.60321456271961e-06
1781 3.60188570236453e-06
1782 3.60263105236847e-06
1783 3.60431770851655e-06
1784 3.60615280903254e-06
1785 3.60760739548383e-06
1786 3.60842182312027e-06
1787 3.60854867120786e-06
1788 3.60808151089742e-06
1789 3.60713955505076e-06
1790 3.6058337500311e-06
1791 3.6042616651244e-06
1792 3.60249774145061e-06
1793 3.60058826021081e-06
1794 3.59856305376383e-06
1795 3.59645311687196e-06
1796 3.59427042728733e-06
1797 3.59203548061515e-06
1798 3.58975927694516e-06
1799 3.58746877959781e-06
1800 3.58519512144717e-06
1801 3.58301777758641e-06
1802 3.58109529940975e-06
1803 3.57973789166799e-06
1804 3.57954863150489e-06
1805 3.58174620185281e-06
1806 3.58879667960998e-06
1807 3.60570518642689e-06
1808 3.64265392427399e-06
1809 3.72007814108954e-06
1810 3.87663833034679e-06
1811 4.17456411661021e-06
1812 4.67182006680389e-06
1813 5.29325200560038e-06
1814 5.64538758318633e-06
1815 5.27233826819185e-06
1816 4.4385322737206e-06
1817 3.88834013120842e-06
1818 3.85948958459004e-06
1819 4.09590535177529e-06
1820 4.27751921616704e-06
1821 4.245404144676e-06
1822 4.04850494684084e-06
1823 3.82889816341603e-06
1824 3.6759699471034e-06
1825 3.59665860427683e-06
1826 3.56343171681317e-06
1827 3.55201058899191e-06
1828 3.54963594650926e-06
1829 3.55089954506838e-06
1830 3.55328333134786e-06
1831 3.55536339169404e-06
1832 3.55641295279874e-06
1833 3.55622701353475e-06
1834 3.55496420523238e-06
1835 3.55292262121942e-06
1836 3.55040243982252e-06
1837 3.54762009191489e-06
1838 3.54474121655457e-06
1839 3.5418525949904e-06
1840 3.5390182751005e-06
1841 3.53626158733356e-06
1842 3.53361036831146e-06
1843 3.53106759209965e-06
1844 3.52864147967757e-06
1845 3.52634649014583e-06
1846 3.52421708393891e-06
1847 3.52229244726665e-06
1848 3.52067309061788e-06
1849 3.51952272237277e-06
1850 3.51915660012914e-06
1851 3.52013500681814e-06
1852 3.52349632359861e-06
1853 3.53116752838112e-06
1854 3.54679214487064e-06
1855 3.57724885224897e-06
1856 3.6353451400295e-06
1857 3.74385468138172e-06
1858 3.93853871205607e-06
1859 4.25819942329042e-06
1860 4.69252985801916e-06
1861 5.07736059951469e-06
1862 5.09353730571149e-06
1863 4.61918073302314e-06
1864 4.01544766992501e-06
1865 3.70927220538775e-06
1866 3.77776512983097e-06
1867 4.04986408186758e-06
1868 4.29811734026231e-06
1869 4.35195105641384e-06
1870 4.18822693859511e-06
1871 3.92743251476091e-06
1872 3.70527241067009e-06
1873 3.57297789999045e-06
1874 3.51235530793303e-06
1875 3.4895986977812e-06
1876 3.48228705049358e-06
1877 3.48065567190581e-06
1878 3.48139047612328e-06
1879 3.48319432186273e-06
1880 3.48525938953514e-06
1881 3.48699464014501e-06
1882 3.48810237427166e-06
1883 3.48852085374496e-06
1884 3.48831717822407e-06
1885 3.48761783630636e-06
1886 3.48653703263757e-06
1887 3.48516744086425e-06
1888 3.48358070523247e-06
1889 3.48183225296061e-06
1890 3.47995632954401e-06
1891 3.47797040034514e-06
1892 3.47590591376346e-06
1893 3.47376195342086e-06
1894 3.47155668656285e-06
1895 3.46930394523604e-06
1896 3.46703254239245e-06
1897 3.46480343371702e-06
1898 3.46273580742462e-06
1899 3.46107778037563e-06
1900 3.46033573550564e-06
1901 3.46155278796445e-06
1902 3.46682786234176e-06
1903 3.48053851961794e-06
1904 3.51167219103665e-06
1905 3.57864516287165e-06
1906 3.71737837223662e-06
1907 3.98944040735927e-06
1908 4.4650155626158e-06
1909 5.11130836144957e-06
1910 5.57967587155872e-06
1911 5.33693173387917e-06
1912 4.49479735697977e-06
1913 3.8396129911078e-06
1914 3.73473001391567e-06
1915 3.95250500773159e-06
1916 4.14810867965798e-06
1917 4.13481351824174e-06
1918 3.94663628577874e-06
1919 3.72595515862884e-06
1920 3.56867371209546e-06
1921 3.48590669752014e-06
1922 3.45095290343878e-06
1923 3.43899860144603e-06
1924 3.43671165325965e-06
1925 3.43823771942198e-06
1926 3.44085230419289e-06
1927 3.44302252863748e-06
1928 3.44401775187464e-06
1929 3.4436880618216e-06
1930 3.4422352026553e-06
1931 3.44000018870361e-06
1932 3.43730462204483e-06
1933 3.43439838434989e-06
1934 3.43141987624307e-06
1935 3.42847899714727e-06
1936 3.4256147156686e-06
1937 3.42285901799855e-06
1938 3.42022389432373e-06
1939 3.41771564182913e-06
1940 3.41533129399885e-06
1941 3.4130834947188e-06
1942 3.41099502865205e-06
1943 3.40909720208948e-06
1944 3.40748260252433e-06
1945 3.40630191719704e-06
1946 3.40582895930197e-06
1947 3.40657660458099e-06
1948 3.40946985222246e-06
1949 3.4162712128083e-06
1950 3.43028288241065e-06
1951 3.45773552634654e-06
1952 3.51035216006323e-06
1953 3.60922649234041e-06
1954 3.7884490193818e-06
1955 4.08856715772288e-06
1956 4.51289865566196e-06
1957 4.92648673144913e-06
1958 5.01973172539749e-06
1959 4.60853952599649e-06
1960 3.99025695330124e-06
1961 3.62233860418115e-06
1962 3.63974284089963e-06
1963 3.89594300154172e-06
1964 4.16405576419976e-06
1965 4.25743008847235e-06
1966 4.12438644659119e-06
1967 3.86789745698835e-06
1968 3.6312702933472e-06
1969 3.48217812007334e-06
1970 3.41070721687764e-06
1971 3.38293807322643e-06
1972 3.37382345660941e-06
1973 3.3716613159207e-06
1974 3.37227803548501e-06
1975 3.374129468936e-06
1976 3.37632601365989e-06
1977 3.37824003904252e-06
1978 3.37954055384948e-06
1979 3.38014903511485e-06
1980 3.38012241818397e-06
1981 3.37957891494156e-06
1982 3.37863907629021e-06
1983 3.37739952693816e-06
1984 3.37593624921695e-06
1985 3.37429804608824e-06
1986 3.37252490623108e-06
1987 3.37064487321292e-06
1988 3.36866793948509e-06
1989 3.36660702737746e-06
1990 3.36447485715929e-06
1991 3.362277658292e-06
1992 3.3600367921327e-06
1993 3.35780345417369e-06
1994 3.35567323639374e-06
1995 3.35385983962766e-06
1996 3.35280103880287e-06
1997 3.35341515111942e-06
1998 3.35756888691918e-06
1999 3.36917774457213e-06
};
\addlegendentry{Train}
\addplot [semithick, black]
table {%
0 0.00803589634597301
1 0.00792757514864206
2 0.0078312112018466
3 0.00774540612474084
4 0.00766869774088264
5 0.00759975239634514
6 0.0075373356230557
7 0.00748024741187692
8 0.00742720626294613
9 0.00737667083740234
10 0.0073266034014523
11 0.00727415736764669
12 0.00721516087651253
13 0.00714373821392655
14 0.00705233123153448
15 0.00693244906142354
16 0.00677657779306173
17 0.00658092182129622
18 0.0063458327203989
19 0.00607365882024169
20 0.00576683925464749
21 0.00543230632320046
22 0.00508170062676072
23 0.00472791027277708
24 0.00438216235488653
25 0.00405243597924709
26 0.00374337914399803
27 0.00345710851252079
28 0.00319395796395838
29 0.00295317312702537
30 0.00273344735614955
31 0.00253319973126054
32 0.00235074525699019
33 0.002184406388551
34 0.00203260174021125
35 0.00189387204591185
36 0.00176689983345568
37 0.00165053550153971
38 0.00154376169666648
39 0.00144567026291043
40 0.00135546107776463
41 0.00127242796588689
42 0.00119594228453934
43 0.0011254382552579
44 0.00106040702667087
45 0.00100039085373282
46 0.000944976403843611
47 0.000893785560037941
48 0.000846473092678934
49 0.000802723283413798
50 0.0007622474222444
51 0.000724781246390194
52 0.000690082844812423
53 0.000657930213492364
54 0.000628120149485767
55 0.000600467261392623
56 0.000574801175389439
57 0.000550966709852219
58 0.000528821430634707
59 0.000508234894368798
60 0.000489087484311312
61 0.00047127073048614
62 0.000454685097793117
63 0.000439238909166306
64 0.000424846162786707
65 0.000411428161896765
66 0.000398912583477795
67 0.000387232954381034
68 0.000376327749108896
69 0.000366136169759557
70 0.000356603675754741
71 0.000347693974617869
72 0.000339369784342125
73 0.000331590243149549
74 0.00032431518775411
75 0.000317506695864722
76 0.000311129842884839
77 0.000305152265354991
78 0.000299543695291504
79 0.000294275407213718
80 0.00028932248824276
81 0.000284662615740672
82 0.000280273437965661
83 0.000276134494924918
84 0.00027222788776271
85 0.000268536619842052
86 0.000265044567640871
87 0.000261736829997972
88 0.000258599378867075
89 0.000255618972005323
90 0.000252783996984363
91 0.000250083481660113
92 0.000247507472522557
93 0.000245047092903405
94 0.00024269380082842
95 0.000240439650951885
96 0.000238277338212356
97 0.000236199965002015
98 0.000234201230341569
99 0.000232274949667044
100 0.000230415724217892
101 0.000228618853725493
102 0.000226879783440381
103 0.00022519426420331
104 0.000223558425204828
105 0.000221968453843147
106 0.000220421017729677
107 0.000218913002754562
108 0.000217441644053906
109 0.000216004264075309
110 0.000214598359889351
111 0.000213221515878104
112 0.000211871738429181
113 0.000210546917514876
114 0.000209245248697698
115 0.000207964942092076
116 0.000206704280572012
117 0.000205461779842153
118 0.000204236115678214
119 0.000203025905648246
120 0.000201830145670101
121 0.00020064770069439
122 0.000199477639398538
123 0.000198319088667631
124 0.000197171291802078
125 0.000196033681277186
126 0.000194905936950818
127 0.000193787476746365
128 0.000192677791346796
129 0.000191576385986991
130 0.00019048283866141
131 0.000189396581845358
132 0.000188317324500531
133 0.000187244571861811
134 0.000186177712748758
135 0.000185116732609458
136 0.000184061791514978
137 0.000183012962224893
138 0.000181970041012391
139 0.000180932955117896
140 0.00017990154447034
141 0.000178875692654401
142 0.000177855254150927
143 0.000176840141648427
144 0.000175830209627748
145 0.000174825239810161
146 0.000173825188539922
147 0.000172829953953624
148 0.000171839332324453
149 0.000170852843439206
150 0.000169870821991935
151 0.000168893428053707
152 0.000167920414241962
153 0.000166952246217988
154 0.000165989040397108
155 0.000165030432981439
156 0.000164076365763322
157 0.000163126707775518
158 0.000162181138875894
159 0.000161238669534214
160 0.000160299154231325
161 0.000159362723934464
162 0.000158428883878514
163 0.00015749795420561
164 0.00015656984760426
165 0.000155644491314888
166 0.000154721739818342
167 0.000153801651322283
168 0.000152884545968845
169 0.000151970365550369
170 0.000151058862684295
171 0.000150149950059131
172 0.000149243511259556
173 0.000148339429870248
174 0.00014743790961802
175 0.000146538630360737
176 0.000145641315612011
177 0.000144745921716094
178 0.00014385228860192
179 0.000142960445373319
180 0.000142070261063054
181 0.000141181633807719
182 0.000140294330776669
183 0.000139408395625651
184 0.000138523668283597
185 0.00013764007599093
186 0.000136757633299567
187 0.0001358762383461
188 0.000134995760163292
189 0.000134116111439653
190 0.00013323723396752
191 0.00013235914229881
192 0.000131481647258624
193 0.000130604617879726
194 0.000129727952298708
195 0.000128851577755995
196 0.000127975275972858
197 0.000127098872326314
198 0.000126222294056788
199 0.000125345424748957
200 0.000124468046124093
201 0.00012359008542262
202 0.000122711368021555
203 0.000121831995784305
204 0.000120951604912989
205 0.00012006994074909
206 0.000119186872325372
207 0.000118302341434173
208 0.000117416042485274
209 0.000116527939098887
210 0.000115637754788622
211 0.000114745162136387
212 0.000113849782792386
213 0.000112951420305762
214 0.000112049972813111
215 0.000111145156552084
216 0.000110236680484377
217 0.000109324282675516
218 0.000108407664811239
219 0.000107486433989834
220 0.000106560393760446
221 0.000105628823803272
222 0.000104691476735752
223 0.000103748032415751
224 0.000102798163425177
225 0.000101841578725725
226 0.000100877878139727
227 9.99067415250465e-05
228 9.89278123597614e-05
229 9.79406759142876e-05
230 9.69451139098965e-05
231 9.59409226197749e-05
232 9.4927781901788e-05
233 9.39053643378429e-05
234 9.2873400717508e-05
235 9.18315636226907e-05
236 9.07876819837838e-05
237 8.97479840205051e-05
238 8.87130809132941e-05
239 8.76833728398196e-05
240 8.6661362729501e-05
241 8.56570914038457e-05
242 8.46736074890941e-05
243 8.37048719404265e-05
244 8.27472103992477e-05
245 8.18000989966094e-05
246 8.086326852208e-05
247 7.99366680439562e-05
248 7.90202102507465e-05
249 7.811383693479e-05
250 7.72177954786457e-05
251 7.63323187129572e-05
252 7.54571592551656e-05
253 7.4592127930373e-05
254 7.37372538424097e-05
255 7.28926170268096e-05
256 7.20581374480389e-05
257 7.12340697646141e-05
258 7.04208869137801e-05
259 6.96183778927661e-05
260 6.8826564529445e-05
261 6.80455050314777e-05
262 6.7275257606525e-05
263 6.6515822254587e-05
264 6.57672790111974e-05
265 6.50296860840172e-05
266 6.43030653009191e-05
267 6.35874457657337e-05
268 6.28827183390968e-05
269 6.21889921603724e-05
270 6.1506281781476e-05
271 6.08345799264498e-05
272 6.01738211116754e-05
273 5.95240890106652e-05
274 5.88853581575677e-05
275 5.82576140004676e-05
276 5.76408237975556e-05
277 5.70349147892557e-05
278 5.64398469578009e-05
279 5.58555402676575e-05
280 5.528188921744e-05
281 5.4718206229154e-05
282 5.41636800335255e-05
283 5.36184816155583e-05
284 5.30831021023914e-05
285 5.25574978382792e-05
286 5.20415414939635e-05
287 5.15351384819951e-05
288 5.10381578351371e-05
289 5.05505304317921e-05
290 5.00721034768503e-05
291 4.96027023473289e-05
292 4.91422761115246e-05
293 4.86906028527301e-05
294 4.82476170873269e-05
295 4.78131587442476e-05
296 4.73873915325385e-05
297 4.69704791612457e-05
298 4.65618868474849e-05
299 4.61612835351843e-05
300 4.57685491710436e-05
301 4.53834982181434e-05
302 4.500605064095e-05
303 4.46359845227562e-05
304 4.42731834482402e-05
305 4.39174837083556e-05
306 4.35687361459713e-05
307 4.32268352597021e-05
308 4.28916282544378e-05
309 4.25629477831535e-05
310 4.22406628786121e-05
311 4.19246243836824e-05
312 4.16148141084705e-05
313 4.1310991946375e-05
314 4.10129723604769e-05
315 4.07206498493906e-05
316 4.043386070407e-05
317 4.01524412154686e-05
318 3.98763258999679e-05
319 3.96053619624581e-05
320 3.93393747799564e-05
321 3.90782952308655e-05
322 3.88219923479483e-05
323 3.85703497158829e-05
324 3.83231854357291e-05
325 3.8080477679614e-05
326 3.78420409106184e-05
327 3.76077878172509e-05
328 3.73776601918507e-05
329 3.71515343431383e-05
330 3.69292429240886e-05
331 3.67107604688499e-05
332 3.64958686986938e-05
333 3.6284593079472e-05
334 3.60767844540533e-05
335 3.58723009412643e-05
336 3.56711570930202e-05
337 3.54732001142111e-05
338 3.52783863490913e-05
339 3.50866139342543e-05
340 3.48978137481026e-05
341 3.47119203070179e-05
342 3.45288208336569e-05
343 3.43484316545073e-05
344 3.41707273037173e-05
345 3.39956168318167e-05
346 3.38230020133778e-05
347 3.36529446940403e-05
348 3.34852556989063e-05
349 3.33199823217001e-05
350 3.31569244735874e-05
351 3.29961476381868e-05
352 3.28375026583672e-05
353 3.26810513797682e-05
354 3.25266992149409e-05
355 3.23744243360125e-05
356 3.22241940011736e-05
357 3.20758881571237e-05
358 3.19294340442866e-05
359 3.17848316626623e-05
360 3.16420264425687e-05
361 3.15009674523026e-05
362 3.13616146740969e-05
363 3.12239317281637e-05
364 3.10878749587573e-05
365 3.09534225380048e-05
366 3.08204689645208e-05
367 3.06890724459663e-05
368 3.05591383948922e-05
369 3.04306140606059e-05
370 3.03035285469377e-05
371 3.01777527056402e-05
372 3.00533320114482e-05
373 2.9930226446595e-05
374 2.98083596135257e-05
375 2.9687753340113e-05
376 2.95683385047596e-05
377 2.94501896860311e-05
378 2.93331995635526e-05
379 2.92173917841865e-05
380 2.91026881313883e-05
381 2.89890285785077e-05
382 2.88765240838984e-05
383 2.8765123715857e-05
384 2.86546946881572e-05
385 2.85453079413855e-05
386 2.84368743450614e-05
387 2.83293720713118e-05
388 2.82227683783276e-05
389 2.81170723610558e-05
390 2.8012233087793e-05
391 2.79082705674227e-05
392 2.78051338682417e-05
393 2.77028375421651e-05
394 2.76013070106274e-05
395 2.75005331786815e-05
396 2.74005651590414e-05
397 2.73012938123429e-05
398 2.72027846222045e-05
399 2.7104955734103e-05
400 2.70078107860172e-05
401 2.69113825197564e-05
402 2.68155690719141e-05
403 2.67204395640874e-05
404 2.66259303316474e-05
405 2.65320886683185e-05
406 2.64388145296834e-05
407 2.63461643044138e-05
408 2.62541107076686e-05
409 2.61625937127974e-05
410 2.60716642515035e-05
411 2.59812586591579e-05
412 2.58913623838453e-05
413 2.58019881584914e-05
414 2.5713115974213e-05
415 2.56247276411159e-05
416 2.55368286161683e-05
417 2.54493879765505e-05
418 2.5362422093167e-05
419 2.52759036811767e-05
420 2.51898200076539e-05
421 2.51041474257363e-05
422 2.50189023063285e-05
423 2.49340992013458e-05
424 2.48496726271696e-05
425 2.47656898864079e-05
426 2.468203092576e-05
427 2.45988212554948e-05
428 2.45159681071527e-05
429 2.44335024035536e-05
430 2.43513823079411e-05
431 2.42696023633471e-05
432 2.41882044065278e-05
433 2.41071575146634e-05
434 2.40264707827009e-05
435 2.39460914599476e-05
436 2.38660886680009e-05
437 2.37863987422315e-05
438 2.37070107687032e-05
439 2.36279938690132e-05
440 2.35492661886383e-05
441 2.34708950301865e-05
442 2.33928203670075e-05
443 2.33149967243662e-05
444 2.32375314226374e-05
445 2.31603571592132e-05
446 2.30834575631889e-05
447 2.30068944802042e-05
448 2.2930549675948e-05
449 2.28545704885619e-05
450 2.27788823394803e-05
451 2.27035397983855e-05
452 2.26285428652773e-05
453 2.25538406084524e-05
454 2.24794493988156e-05
455 2.24053674173774e-05
456 2.23315491894027e-05
457 2.22580456465948e-05
458 2.21848476940067e-05
459 2.21119698835537e-05
460 2.20393631025217e-05
461 2.19670801016036e-05
462 2.18951481656404e-05
463 2.18235163629288e-05
464 2.1752199245384e-05
465 2.16811768041225e-05
466 2.16104799619643e-05
467 2.15400777960895e-05
468 2.14699684875086e-05
469 2.140012657037e-05
470 2.13305793295149e-05
471 2.12613467738265e-05
472 2.11923907045275e-05
473 2.11237365874695e-05
474 2.10553425858961e-05
475 2.09872614505002e-05
476 2.09194367926102e-05
477 2.08518904400989e-05
478 2.07846060220618e-05
479 2.07176144613186e-05
480 2.06508848350495e-05
481 2.0584455342032e-05
482 2.05182732315734e-05
483 2.04523421416525e-05
484 2.03867093659937e-05
485 2.03213330678409e-05
486 2.02562387130456e-05
487 2.0191386283841e-05
488 2.01268285309197e-05
489 2.00624708668329e-05
490 1.99984096980188e-05
491 1.99345522560179e-05
492 1.9870996766258e-05
493 1.98076722881524e-05
494 1.97445988305844e-05
495 1.96817654796178e-05
496 1.96191613213159e-05
497 1.95567954506259e-05
498 1.94946915144101e-05
499 1.94328295037849e-05
500 1.93711730389623e-05
501 1.93097857845714e-05
502 1.92486459127394e-05
503 1.9187693396816e-05
504 1.91270228242502e-05
505 1.90666014532326e-05
506 1.90064565686043e-05
507 1.89464990398847e-05
508 1.88868070836179e-05
509 1.88273279491113e-05
510 1.87680998351425e-05
511 1.87090899999021e-05
512 1.86503129953053e-05
513 1.85917524504475e-05
514 1.85334065463394e-05
515 1.84753153007478e-05
516 1.84174168680329e-05
517 1.83597567229299e-05
518 1.83022802957566e-05
519 1.82450385182165e-05
520 1.81880295713199e-05
521 1.81312370841624e-05
522 1.80746283149347e-05
523 1.80182669282658e-05
524 1.79621147253783e-05
525 1.79061862581875e-05
526 1.78504797077039e-05
527 1.77950259967474e-05
528 1.77397778315935e-05
529 1.76847606780939e-05
530 1.76299654413015e-05
531 1.75753957591951e-05
532 1.75210425368277e-05
533 1.74669075931888e-05
534 1.74129818333313e-05
535 1.73593107319903e-05
536 1.73058433574624e-05
537 1.72525651578326e-05
538 1.71995015989523e-05
539 1.71466072060866e-05
540 1.70939329109387e-05
541 1.70414496096782e-05
542 1.69891864061356e-05
543 1.69371414813213e-05
544 1.68852711794898e-05
545 1.68335973285139e-05
546 1.67821635841392e-05
547 1.67309226526413e-05
548 1.66798763530096e-05
549 1.66290337801911e-05
550 1.65783931151964e-05
551 1.65279689099407e-05
552 1.64777811733074e-05
553 1.6427777154604e-05
554 1.63779877766501e-05
555 1.63284003065201e-05
556 1.62790001922986e-05
557 1.62298474606359e-05
558 1.61808529810514e-05
559 1.61320640472695e-05
560 1.60834897542372e-05
561 1.60351100930711e-05
562 1.59869268827606e-05
563 1.59389437612845e-05
564 1.58911589096533e-05
565 1.58435923367506e-05
566 1.57961967488518e-05
567 1.57490012497874e-05
568 1.57019967446104e-05
569 1.5655225070077e-05
570 1.56086498463992e-05
571 1.55622456077253e-05
572 1.55160632857587e-05
573 1.54700574057642e-05
574 1.54242516146041e-05
575 1.53786422742996e-05
576 1.53332057379885e-05
577 1.52879892993951e-05
578 1.5242942936311e-05
579 1.51980784721673e-05
580 1.51534286487731e-05
581 1.51089725477505e-05
582 1.5064668332343e-05
583 1.50205933096004e-05
584 1.49766729009571e-05
585 1.49329416672117e-05
586 1.4889415979269e-05
587 1.48460803757189e-05
588 1.48029021147522e-05
589 1.47599384945352e-05
590 1.47171549542691e-05
591 1.46745351230493e-05
592 1.46321053762222e-05
593 1.45898429764202e-05
594 1.4547773389495e-05
595 1.45058911584783e-05
596 1.44641844599391e-05
597 1.44226551128668e-05
598 1.43812922033248e-05
599 1.43401321111014e-05
600 1.42991475513554e-05
601 1.42583485285286e-05
602 1.42176804729388e-05
603 1.41772270580987e-05
604 1.41369200719055e-05
605 1.40968022606103e-05
606 1.40568909046124e-05
607 1.40171432576608e-05
608 1.39775656862184e-05
609 1.39381463668542e-05
610 1.38989144033985e-05
611 1.38598361445474e-05
612 1.3820963431499e-05
613 1.37822235046769e-05
614 1.37436836666893e-05
615 1.37052920763381e-05
616 1.36670732899802e-05
617 1.36290400405414e-05
618 1.35911250254139e-05
619 1.35534046421526e-05
620 1.35158625198528e-05
621 1.34784741021576e-05
622 1.34412475745194e-05
623 1.34041965793585e-05
624 1.3367333849601e-05
625 1.33306039060699e-05
626 1.3294044038048e-05
627 1.3257661521493e-05
628 1.32214390760055e-05
629 1.31853394123027e-05
630 1.31494462038972e-05
631 1.31136994241388e-05
632 1.30781136249425e-05
633 1.30426878968137e-05
634 1.30074367916677e-05
635 1.29723503050627e-05
636 1.29374020616524e-05
637 1.29026620925288e-05
638 1.28680558191263e-05
639 1.28336087072967e-05
640 1.27993343994603e-05
641 1.27652283481439e-05
642 1.27312623590115e-05
643 1.26974910017452e-05
644 1.26638688016101e-05
645 1.26304194054683e-05
646 1.25971464512986e-05
647 1.25639990073978e-05
648 1.25310207295115e-05
649 1.24982052511768e-05
650 1.24655471154256e-05
651 1.24330535982153e-05
652 1.24007028716733e-05
653 1.23685149446828e-05
654 1.23365171020851e-05
655 1.23046438602614e-05
656 1.22729252325371e-05
657 1.2241376680322e-05
658 1.22099636428175e-05
659 1.21787443276844e-05
660 1.21476368803997e-05
661 1.21166804092354e-05
662 1.20859158414532e-05
663 1.20552749649505e-05
664 1.20247759696213e-05
665 1.1994457054243e-05
666 1.19642609206494e-05
667 1.19342266771127e-05
668 1.19043425002019e-05
669 1.18745692816447e-05
670 1.18449706860702e-05
671 1.18155312520685e-05
672 1.17862591650919e-05
673 1.17570752991014e-05
674 1.17280551421572e-05
675 1.16992068797117e-05
676 1.16704868560191e-05
677 1.16419350888464e-05
678 1.16135061034583e-05
679 1.15852189992438e-05
680 1.15570828711498e-05
681 1.15290931717027e-05
682 1.15012653623126e-05
683 1.14735476017813e-05
684 1.14459890028229e-05
685 1.14185850179638e-05
686 1.13913320092252e-05
687 1.13642054202501e-05
688 1.13371834231657e-05
689 1.13103105832124e-05
690 1.12835759864538e-05
691 1.12569714474375e-05
692 1.12304687718279e-05
693 1.12041097963811e-05
694 1.11778881546343e-05
695 1.11517683762941e-05
696 1.11258050310425e-05
697 1.10999581011129e-05
698 1.1074193935201e-05
699 1.10485834738938e-05
700 1.10230930658872e-05
701 1.09976881503826e-05
702 1.09724242065568e-05
703 1.09472975964309e-05
704 1.09222237369977e-05
705 1.08972981252009e-05
706 1.08724607343902e-05
707 1.08477506728377e-05
708 1.0823154298123e-05
709 1.07986734292354e-05
710 1.07742544059874e-05
711 1.07499699879554e-05
712 1.07257883428247e-05
713 1.07017058326164e-05
714 1.06776806205744e-05
715 1.06537645478966e-05
716 1.06299403341836e-05
717 1.0606209798425e-05
718 1.05825693026418e-05
719 1.05590324892546e-05
720 1.05355529740336e-05
721 1.05121807791875e-05
722 1.04888868008857e-05
723 1.04656946859905e-05
724 1.04425535027985e-05
725 1.04195069070556e-05
726 1.03965276139206e-05
727 1.03736574601498e-05
728 1.03508409665665e-05
729 1.03281081464957e-05
730 1.0305457180948e-05
731 1.02828735180083e-05
732 1.02603635241394e-05
733 1.02379199233837e-05
734 1.02155518106883e-05
735 1.0193281013926e-05
736 1.01710638773511e-05
737 1.01489185908576e-05
738 1.01268842627178e-05
739 1.01048926808289e-05
740 1.00830220617354e-05
741 1.00611978268716e-05
742 1.00394972832873e-05
743 1.00178576758481e-05
744 9.99632084131008e-06
745 9.97487950371578e-06
746 9.95351638266584e-06
747 9.9322587630013e-06
748 9.91107754089171e-06
749 9.89002910500858e-06
750 9.86909435596317e-06
751 9.8482314569992e-06
752 9.82747224043123e-06
753 9.80679033091292e-06
754 9.78620300884359e-06
755 9.76571936917026e-06
756 9.74534759734524e-06
757 9.72507223195862e-06
758 9.70489691098919e-06
759 9.68486801866675e-06
760 9.66491461440455e-06
761 9.64509854384232e-06
762 9.6253870651708e-06
763 9.60580200626282e-06
764 9.5863497335813e-06
765 9.56697931542294e-06
766 9.54773804551223e-06
767 9.52860318648163e-06
768 9.50956018641591e-06
769 9.49068180489121e-06
770 9.47187709243735e-06
771 9.45321608014638e-06
772 9.43466602620902e-06
773 9.4162442110246e-06
774 9.39791516429977e-06
775 9.37970708037028e-06
776 9.36166452447651e-06
777 9.34374202188337e-06
778 9.32594321056968e-06
779 9.30826809053542e-06
780 9.29073303268524e-06
781 9.27333985600853e-06
782 9.25606309465365e-06
783 9.23895822779741e-06
784 9.22197978070471e-06
785 9.20518414204707e-06
786 9.18854493647814e-06
787 9.17208217288135e-06
788 9.15580858418252e-06
789 9.13974781724392e-06
790 9.12387986318208e-06
791 9.1082438302692e-06
792 9.09292884898605e-06
793 9.07792764337501e-06
794 9.06329751160229e-06
795 9.04915214050561e-06
796 9.03559248399688e-06
797 9.02278407011181e-06
798 9.01090788829606e-06
799 9.00027498573763e-06
800 8.9913310148404e-06
801 8.98468442755984e-06
802 8.98121197678847e-06
803 8.982357030618e-06
804 8.99005772225792e-06
805 9.00745362741873e-06
806 9.03930231288541e-06
807 9.09249047253979e-06
808 9.17669694899814e-06
809 9.30240548768779e-06
810 9.47326134337345e-06
811 9.66445804806426e-06
812 9.79401283984771e-06
813 9.73559417616343e-06
814 9.44167550187558e-06
815 9.05133765627397e-06
816 8.75976547831669e-06
817 8.63085824676091e-06
818 8.62204433360603e-06
819 8.67206745169824e-06
820 8.71404608915327e-06
821 8.68916504259687e-06
822 8.60997897689231e-06
823 8.55720736581134e-06
824 8.57005488796858e-06
825 8.60869840835221e-06
826 8.62538672663504e-06
827 8.60897580423625e-06
828 8.57225677464157e-06
829 8.53022265800973e-06
830 8.49216667120345e-06
831 8.46135208121268e-06
832 8.43760062707588e-06
833 8.4192952272133e-06
834 8.40474604046904e-06
835 8.39253334561363e-06
836 8.38172127259895e-06
837 8.37170409795363e-06
838 8.36209528642939e-06
839 8.35272385302233e-06
840 8.34352067613509e-06
841 8.33447666082066e-06
842 8.32551268103998e-06
843 8.31673150969436e-06
844 8.30810677143745e-06
845 8.29965756565798e-06
846 8.29144664749037e-06
847 8.28349493531277e-06
848 8.27585972729139e-06
849 8.26856830826728e-06
850 8.26167615741724e-06
851 8.25531697046245e-06
852 8.2495462265797e-06
853 8.24452490633121e-06
854 8.24039761937456e-06
855 8.23741265776334e-06
856 8.23583468445577e-06
857 8.23616392153781e-06
858 8.23888331069611e-06
859 8.24476865091128e-06
860 8.25479310151422e-06
861 8.27033454697812e-06
862 8.29309828986879e-06
863 8.32522891869303e-06
864 8.36901926959399e-06
865 8.42619374452624e-06
866 8.49615298648132e-06
867 8.57299346535001e-06
868 8.6412701421068e-06
869 8.67424114403548e-06
870 8.64145749801537e-06
871 8.52867196954321e-06
872 8.35653372632805e-06
873 8.17335421743337e-06
874 8.02641443442553e-06
875 7.9445253504673e-06
876 7.94698098616209e-06
877 8.05564195616171e-06
878 8.26916493679164e-06
879 8.47841511131264e-06
880 8.45960039441707e-06
881 8.18355329101905e-06
882 7.97349366621347e-06
883 7.99580539023736e-06
884 8.08442473498872e-06
885 8.10031269793399e-06
886 8.04472438176163e-06
887 7.96552740212064e-06
888 7.89764908404322e-06
889 7.85187421570299e-06
890 7.82467668614117e-06
891 7.80938626121497e-06
892 7.80055324867135e-06
893 7.79481342760846e-06
894 7.79016136220889e-06
895 7.78557750891196e-06
896 7.7805743785575e-06
897 7.7750319178449e-06
898 7.7689255704172e-06
899 7.76234355726046e-06
900 7.75538865127601e-06
901 7.74814452597639e-06
902 7.74065483710729e-06
903 7.73294050304685e-06
904 7.72501971368911e-06
905 7.71688974054996e-06
906 7.70858332543867e-06
907 7.70001224736916e-06
908 7.69117741583614e-06
909 7.68205063650385e-06
910 7.6725882536266e-06
911 7.66274388297461e-06
912 7.65243567002472e-06
913 7.64167998568155e-06
914 7.63040770834778e-06
915 7.61871524446178e-06
916 7.60675493438612e-06
917 7.59502609071205e-06
918 7.58439273340628e-06
919 7.57668385631405e-06
920 7.57497582526412e-06
921 7.58431815484073e-06
922 7.61198680265807e-06
923 7.66590528655797e-06
924 7.74857926444383e-06
925 7.84486655902583e-06
926 7.91128695709631e-06
927 7.89359910413623e-06
928 7.78607136453502e-06
929 7.66597531765001e-06
930 7.63202024245402e-06
931 7.72711427998729e-06
932 7.94196330389241e-06
933 8.23675964056747e-06
934 8.50961441756226e-06
935 8.57819759403355e-06
936 8.32733167044353e-06
937 7.91598540672567e-06
938 7.61349247113685e-06
939 7.48496495361906e-06
940 7.44851149647729e-06
941 7.4423310252314e-06
942 7.44671433494659e-06
943 7.45605939300731e-06
944 7.46664545658859e-06
945 7.47551075619413e-06
946 7.48094726077397e-06
947 7.48270895201131e-06
948 7.48125739846728e-06
949 7.47745025364566e-06
950 7.47203057471779e-06
951 7.46557088859845e-06
952 7.45847819416667e-06
953 7.45099032428698e-06
954 7.44326825952157e-06
955 7.43534246794297e-06
956 7.42724159863428e-06
957 7.4189683800796e-06
958 7.41049325370113e-06
959 7.40174164093332e-06
960 7.39277675165795e-06
961 7.3834671638906e-06
962 7.37380605642102e-06
963 7.36386618882534e-06
964 7.35371077098534e-06
965 7.34359900889103e-06
966 7.33395654606284e-06
967 7.32575199435814e-06
968 7.32070020603715e-06
969 7.32175976736471e-06
970 7.33384558770922e-06
971 7.36422225600109e-06
972 7.42169459044817e-06
973 7.51173092794488e-06
974 7.62494164519012e-06
975 7.72177554608788e-06
976 7.73802730691386e-06
977 7.64006017561769e-06
978 7.48664433558588e-06
979 7.39097868063254e-06
980 7.41964640837978e-06
981 7.56968665882596e-06
982 7.80495975050144e-06
983 8.0561512731947e-06
984 8.19606430013664e-06
985 8.0980607890524e-06
986 7.79597849032143e-06
987 7.49030414226581e-06
988 7.31572436052375e-06
989 7.25103382137604e-06
990 7.23405628377805e-06
991 7.231853032863e-06
992 7.23432412996772e-06
993 7.238778835017e-06
994 7.24383789929561e-06
995 7.24839173926739e-06
996 7.25184236216592e-06
997 7.25384052202571e-06
998 7.25449308447423e-06
999 7.25390145817073e-06
1000 7.25232348486315e-06
1001 7.24985875422135e-06
1002 7.24666642781813e-06
1003 7.24283881936572e-06
1004 7.23840594218927e-06
1005 7.23346738595865e-06
1006 7.22793492968776e-06
1007 7.2219013418362e-06
1008 7.21531796443742e-06
1009 7.20809521226329e-06
1010 7.20030948286876e-06
1011 7.1918889261724e-06
1012 7.1829140324553e-06
1013 7.17357079338399e-06
1014 7.16428212399478e-06
1015 7.15600754119805e-06
1016 7.1507238317281e-06
1017 7.15225314706913e-06
1018 7.16763179298141e-06
1019 7.20900879969122e-06
1020 7.29452358427807e-06
1021 7.44300905353157e-06
1022 7.65116328693694e-06
1023 7.84892017691163e-06
1024 7.88880424806848e-06
1025 7.68540849094279e-06
1026 7.39391953175073e-06
1027 7.26106691217865e-06
1028 7.34657760403934e-06
1029 7.55629525883705e-06
1030 7.76985598349711e-06
1031 7.86500822869129e-06
1032 7.76397610025015e-06
1033 7.52472806198057e-06
1034 7.29806561139412e-06
1035 7.16793056199094e-06
1036 7.11634902472724e-06
1037 7.10150379745755e-06
1038 7.10004405846121e-06
1039 7.10345557308756e-06
1040 7.10856465957477e-06
1041 7.11362781657954e-06
1042 7.11763777871965e-06
1043 7.12012479198165e-06
1044 7.12100154487416e-06
1045 7.12048222339945e-06
1046 7.11878965375945e-06
1047 7.1162253334478e-06
1048 7.11294615030056e-06
1049 7.10911126589053e-06
1050 7.10481845089816e-06
1051 7.10015865479363e-06
1052 7.09513278707163e-06
1053 7.08977358954144e-06
1054 7.08410743754939e-06
1055 7.07807384969783e-06
1056 7.07171693647979e-06
1057 7.06506671122042e-06
1058 7.05822822055779e-06
1059 7.05135016687564e-06
1060 7.04483727531624e-06
1061 7.03944988345029e-06
1062 7.03664272805327e-06
1063 7.03912064636825e-06
1064 7.05172942616628e-06
1065 7.08268180460436e-06
1066 7.14455745765008e-06
1067 7.25272275303723e-06
1068 7.41516987545765e-06
1069 7.60586226533633e-06
1070 7.73508236306952e-06
1071 7.68417612562189e-06
1072 7.4497706918919e-06
1073 7.20872321835486e-06
1074 7.13252711648238e-06
1075 7.23857192497235e-06
1076 7.45414763514418e-06
1077 7.68013887864072e-06
1078 7.78953381086467e-06
1079 7.68641166359885e-06
1080 7.42761903893552e-06
1081 7.18497631169157e-06
1082 7.05319780536229e-06
1083 7.00643340678653e-06
1084 6.99464362696745e-06
1085 6.99287102179369e-06
1086 6.99411020832486e-06
1087 6.99683278071461e-06
1088 7.00018699717475e-06
1089 7.00345754012233e-06
1090 7.00601185599226e-06
1091 7.00769169270643e-06
1092 7.00833925293409e-06
1093 7.00809277986991e-06
1094 7.00708005751949e-06
1095 7.00539703757386e-06
1096 7.00311875334592e-06
1097 7.00035297995782e-06
1098 6.99712336427183e-06
1099 6.99344764143461e-06
1100 6.98938356435974e-06
1101 6.98492794981576e-06
1102 6.98004487276194e-06
1103 6.9747952693433e-06
1104 6.96918141329661e-06
1105 6.96324605087284e-06
1106 6.95726248522988e-06
1107 6.95157723384909e-06
1108 6.94703430781374e-06
1109 6.94525806466118e-06
1110 6.94942764312145e-06
1111 6.96536562827532e-06
1112 7.00367763784016e-06
1113 7.08158131601522e-06
1114 7.222155090858e-06
1115 7.44091403248603e-06
1116 7.70381393522257e-06
1117 7.8742441473878e-06
1118 7.77137302065967e-06
1119 7.4208905971318e-06
1120 7.11789425622555e-06
1121 7.06942500983132e-06
1122 7.22810182196554e-06
1123 7.45508577892906e-06
1124 7.60944931244012e-06
1125 7.57780890126014e-06
1126 7.37054097044165e-06
1127 7.13448071110179e-06
1128 6.98817666489049e-06
1129 6.93056108502788e-06
1130 6.9146944952081e-06
1131 6.9117222665227e-06
1132 6.91260675012018e-06
1133 6.91530522090034e-06
1134 6.91877721692435e-06
1135 6.92206458552391e-06
1136 6.92452931616572e-06
1137 6.92586945660878e-06
1138 6.92610410624184e-06
1139 6.92535695634433e-06
1140 6.92385401634965e-06
1141 6.92170988259022e-06
1142 6.91912146066898e-06
1143 6.91614059178391e-06
1144 6.91285549692111e-06
1145 6.90927936375374e-06
1146 6.9054899540788e-06
1147 6.90147135173902e-06
1148 6.89721855451353e-06
1149 6.89279704602086e-06
1150 6.88818681737757e-06
1151 6.88349291522172e-06
1152 6.87888632455724e-06
1153 6.87467490934068e-06
1154 6.87142164679244e-06
1155 6.87028159518377e-06
1156 6.87335477778106e-06
1157 6.88452337271883e-06
1158 6.91070817993023e-06
1159 6.96365668773069e-06
1160 7.06119953974849e-06
1161 7.22342656445107e-06
1162 7.45340730645694e-06
1163 7.69296002545161e-06
1164 7.79151287133573e-06
1165 7.61267438065261e-06
1166 7.25545123714255e-06
1167 7.00156670063734e-06
1168 6.99135580362054e-06
1169 7.16379099685582e-06
1170 7.39650477044052e-06
1171 7.55068413127447e-06
1172 7.50615481592831e-06
1173 7.2807429205568e-06
1174 7.03923387845862e-06
1175 6.90119986757054e-06
1176 6.85322856952553e-06
1177 6.84181213728152e-06
1178 6.83889902575174e-06
1179 6.83800953993341e-06
1180 6.83868347550742e-06
1181 6.84064070810564e-06
1182 6.84305632603355e-06
1183 6.84517317495192e-06
1184 6.84659380567609e-06
1185 6.84718997945311e-06
1186 6.84695442032535e-06
1187 6.84602855471894e-06
1188 6.84455335431267e-06
1189 6.84257292959956e-06
1190 6.8402346187213e-06
1191 6.83756161379279e-06
1192 6.83463031236897e-06
1193 6.83139705870417e-06
1194 6.82796826367849e-06
1195 6.82428753862041e-06
1196 6.82041263644351e-06
1197 6.81637675370439e-06
1198 6.81223855281132e-06
1199 6.80813855069573e-06
1200 6.8043500505155e-06
1201 6.80141056363937e-06
1202 6.8003519118065e-06
1203 6.8031481532671e-06
1204 6.8135072979203e-06
1205 6.83825055602938e-06
1206 6.88964701112127e-06
1207 6.98783151165117e-06
1208 7.15957730790251e-06
1209 7.42145903132041e-06
1210 7.72781640989706e-06
1211 7.90684771345695e-06
1212 7.74705222283956e-06
1213 7.31191903469153e-06
1214 6.96580264047952e-06
1215 6.91906552674482e-06
1216 7.0897053774388e-06
1217 7.31343016013852e-06
1218 7.43328610042227e-06
1219 7.34962577553233e-06
1220 7.12134988134494e-06
1221 6.91473542246968e-06
1222 6.81180245010182e-06
1223 6.7803716774506e-06
1224 6.77317211739137e-06
1225 6.77079742672504e-06
1226 6.77009666105732e-06
1227 6.77108437230345e-06
1228 6.77317075314932e-06
1229 6.7753967414319e-06
1230 6.77704292684211e-06
1231 6.77783464198001e-06
1232 6.77774369250983e-06
1233 6.77690059092129e-06
1234 6.77543403071468e-06
1235 6.77351863487274e-06
1236 6.771288553864e-06
1237 6.76878244121326e-06
1238 6.766081241949e-06
1239 6.76319677950232e-06
1240 6.76019180900767e-06
1241 6.75706405672827e-06
1242 6.75379760650685e-06
1243 6.75045748721459e-06
1244 6.74702232572599e-06
1245 6.74362900099368e-06
1246 6.74035254633054e-06
1247 6.73738759360276e-06
1248 6.7351993493503e-06
1249 6.73454405841767e-06
1250 6.7369214775681e-06
1251 6.7450482674758e-06
1252 6.76402441968094e-06
1253 6.80308039591182e-06
1254 6.87805413690512e-06
1255 7.01310545991873e-06
1256 7.23445236872067e-06
1257 7.53897802496795e-06
1258 7.82547613198403e-06
1259 7.86954569775844e-06
1260 7.53868789615808e-06
1261 7.07382469045115e-06
1262 6.83779717292055e-06
1263 6.89857643010328e-06
1264 7.11433040123666e-06
1265 7.31654426999739e-06
1266 7.35307594368351e-06
1267 7.18162937118905e-06
1268 6.93940364726586e-06
1269 6.78152719046921e-06
1270 6.72410124025191e-06
1271 6.71122188578011e-06
1272 6.70727195029031e-06
1273 6.70433973937179e-06
1274 6.70307554173633e-06
1275 6.70373628963716e-06
1276 6.70547706249636e-06
1277 6.7072674028168e-06
1278 6.70845793138142e-06
1279 6.70885265208199e-06
1280 6.70843701300328e-06
1281 6.70743565933662e-06
1282 6.70589133733301e-06
1283 6.70400777380564e-06
1284 6.70182407702669e-06
1285 6.69948440190637e-06
1286 6.69698010824504e-06
1287 6.69434211886255e-06
1288 6.69159862809465e-06
1289 6.688775101793e-06
1290 6.6858751779364e-06
1291 6.68291295369272e-06
1292 6.67992298986064e-06
1293 6.67696349410107e-06
1294 6.6741540649673e-06
1295 6.67165932100033e-06
1296 6.66990263198386e-06
1297 6.66958158035413e-06
1298 6.67201493342873e-06
1299 6.67968970446964e-06
1300 6.69730434310623e-06
1301 6.73348586133216e-06
1302 6.8037379605812e-06
1303 6.93326592227095e-06
1304 7.15445321475272e-06
1305 7.4808053796005e-06
1306 7.83318682806566e-06
1307 7.97139909991529e-06
1308 7.6725873441319e-06
1309 7.12931296220631e-06
1310 6.79506138112629e-06
1311 6.81087431075866e-06
1312 7.01910676070838e-06
1313 7.22240793038509e-06
1314 7.25758945918642e-06
1315 7.08587822373374e-06
1316 6.85130771671538e-06
1317 6.70548524794867e-06
1318 6.65647212372278e-06
1319 6.64645949655096e-06
1320 6.64254548610188e-06
1321 6.63909213471925e-06
1322 6.63752825857955e-06
1323 6.63809896650491e-06
1324 6.63972787151579e-06
1325 6.64134768157965e-06
1326 6.64225262880791e-06
1327 6.6423517637304e-06
1328 6.64170056552393e-06
1329 6.64046592646628e-06
1330 6.63883065499249e-06
1331 6.63689525026712e-06
1332 6.63476430418086e-06
1333 6.63252922095126e-06
1334 6.63021501168259e-06
1335 6.62782622384839e-06
1336 6.62539423501585e-06
1337 6.6229067670065e-06
1338 6.6203965616296e-06
1339 6.61784997646464e-06
1340 6.61528383716359e-06
1341 6.61272179058869e-06
1342 6.61027843307238e-06
1343 6.60799150864477e-06
1344 6.60606883684522e-06
1345 6.60489013171173e-06
1346 6.60510340821929e-06
1347 6.60795922158286e-06
1348 6.61584954286809e-06
1349 6.63316222926369e-06
1350 6.6683337536233e-06
1351 6.73680278850952e-06
1352 6.8648350861622e-06
1353 7.08971538188052e-06
1354 7.43762848287588e-06
1355 7.84653366281418e-06
1356 8.06347816251218e-06
1357 7.78941375756403e-06
1358 7.17351440471248e-06
1359 6.75402407068759e-06
1360 6.73870499667828e-06
1361 6.94754407959408e-06
1362 7.14873340257327e-06
1363 7.16755312168971e-06
1364 6.98107396601699e-06
1365 6.75273349770578e-06
1366 6.62504407955566e-06
1367 6.58815042697825e-06
1368 6.58159524391522e-06
1369 6.57751115795691e-06
1370 6.573688096978e-06
1371 6.57217879052041e-06
1372 6.57297505313181e-06
1373 6.57464033793076e-06
1374 6.57603413856123e-06
1375 6.57665123071638e-06
1376 6.57637701806379e-06
1377 6.57542341286899e-06
1378 6.57397140457761e-06
1379 6.57224336464424e-06
1380 6.57030159345595e-06
1381 6.5682752392604e-06
1382 6.56620159134036e-06
1383 6.56407019050675e-06
1384 6.56191832604236e-06
1385 6.55977282804088e-06
1386 6.55762096357648e-06
1387 6.55542908134521e-06
1388 6.55325493426062e-06
1389 6.55103440294624e-06
1390 6.54883751849411e-06
1391 6.54664336252608e-06
1392 6.54450786896632e-06
1393 6.54248879072838e-06
1394 6.54073619443807e-06
1395 6.53944880468771e-06
1396 6.53910865366925e-06
1397 6.54051291348878e-06
1398 6.5452691160317e-06
1399 6.5564245232963e-06
1400 6.57991586194839e-06
1401 6.62734782963526e-06
1402 6.72029227644089e-06
1403 6.896099421283e-06
1404 7.20511752660968e-06
1405 7.66623816161882e-06
1406 8.13393307907972e-06
1407 8.20296281744959e-06
1408 7.62755416872096e-06
1409 6.91019931764458e-06
1410 6.64939352645888e-06
1411 6.79234972267295e-06
1412 7.0216055974015e-06
1413 7.09531286702259e-06
1414 6.94128993927734e-06
1415 6.70703093419434e-06
1416 6.56404336041305e-06
1417 6.52181279292563e-06
1418 6.51545951768639e-06
1419 6.51175969323958e-06
1420 6.50791798761929e-06
1421 6.50664333079476e-06
1422 6.50774154564715e-06
1423 6.5095632635348e-06
1424 6.51081927571795e-06
1425 6.51108621241292e-06
1426 6.51043137622764e-06
1427 6.50916535960278e-06
1428 6.50755464448594e-06
1429 6.50577840133337e-06
1430 6.50390302325832e-06
1431 6.50201855023624e-06
1432 6.50015317660291e-06
1433 6.49829280519043e-06
1434 6.49646062811371e-06
1435 6.49461935608997e-06
1436 6.49279581921292e-06
1437 6.4909663706203e-06
1438 6.48915465717437e-06
1439 6.48730838292977e-06
1440 6.48547711534775e-06
1441 6.48360310151475e-06
1442 6.48173681838671e-06
1443 6.47983961243881e-06
1444 6.47792512609158e-06
1445 6.47599972580792e-06
1446 6.47414753984776e-06
1447 6.4723176365078e-06
1448 6.47068463877076e-06
1449 6.46940907245153e-06
1450 6.4689083956182e-06
1451 6.46992884867359e-06
1452 6.47412662146962e-06
1453 6.48481363896281e-06
1454 6.50906986265909e-06
1455 6.56168595014606e-06
1456 6.67326276015956e-06
1457 6.90269826009171e-06
1458 7.34035802452127e-06
1459 8.03467719379114e-06
1460 8.7191920101759e-06
1461 8.63756486069178e-06
1462 7.58435726311291e-06
1463 6.72811484037084e-06
1464 6.68361326461309e-06
1465 6.93784204486292e-06
1466 7.01414683135226e-06
1467 6.8230237957323e-06
1468 6.57971804685076e-06
1469 6.46295438855304e-06
1470 6.44195642962586e-06
1471 6.4412079154863e-06
1472 6.4385367295472e-06
1473 6.43728344584815e-06
1474 6.43903740638052e-06
1475 6.44160172669217e-06
1476 6.44295505480841e-06
1477 6.442771791626e-06
1478 6.44158171780873e-06
1479 6.43996509097633e-06
1480 6.43832299829228e-06
1481 6.43669363853405e-06
1482 6.43514977127779e-06
1483 6.43367320662946e-06
1484 6.43224484520033e-06
1485 6.43086286800099e-06
1486 6.42951545160031e-06
1487 6.42819759377744e-06
1488 6.42688337393338e-06
1489 6.42559871266712e-06
1490 6.42430859443266e-06
1491 6.42299210085184e-06
1492 6.42167151454487e-06
1493 6.42034910924849e-06
1494 6.4190185184998e-06
1495 6.41766337139416e-06
1496 6.41629776509944e-06
1497 6.41491715214215e-06
1498 6.41354745312128e-06
1499 6.4121481955226e-06
1500 6.41078941043816e-06
1501 6.40939742879709e-06
1502 6.40801181361894e-06
1503 6.40669168205932e-06
1504 6.40537882645731e-06
1505 6.40412690700032e-06
1506 6.40299322185456e-06
1507 6.40200369161903e-06
1508 6.40128519080463e-06
1509 6.40097414361662e-06
1510 6.40144389763009e-06
1511 6.40326425127569e-06
1512 6.40759662928758e-06
1513 6.41686438029865e-06
1514 6.43592147753225e-06
1515 6.47502156425617e-06
1516 6.5552717387618e-06
1517 6.71724137646379e-06
1518 7.02616398484679e-06
1519 7.53399126551813e-06
1520 8.10902838566108e-06
1521 8.2334381659166e-06
1522 7.54970687921741e-06
1523 6.82543804941815e-06
1524 6.80789207763155e-06
1525 7.16397653377499e-06
1526 7.42022803024156e-06
1527 7.37507343728794e-06
1528 7.0530754783249e-06
1529 6.68500069878064e-06
1530 6.45969794277335e-06
1531 6.37270250081201e-06
1532 6.35181049801758e-06
1533 6.35586593489279e-06
1534 6.36951926935581e-06
1535 6.38453866486088e-06
1536 6.39536074231728e-06
1537 6.39991594653111e-06
1538 6.39904419585946e-06
1539 6.39470817986876e-06
1540 6.38886831438867e-06
1541 6.38273195363581e-06
1542 6.37703806205536e-06
1543 6.3719285208208e-06
1544 6.36747518001357e-06
1545 6.36353479421814e-06
1546 6.36001641396433e-06
1547 6.35677406535251e-06
1548 6.35379592495156e-06
1549 6.35094011158799e-06
1550 6.34820753475651e-06
1551 6.34555044598528e-06
1552 6.34297339274781e-06
1553 6.34039679425769e-06
1554 6.33783156445133e-06
1555 6.33526724413969e-06
1556 6.33269246463897e-06
1557 6.33009858574951e-06
1558 6.32748833595542e-06
1559 6.32492674412788e-06
1560 6.32246337772813e-06
1561 6.32031378700049e-06
1562 6.31886950941407e-06
1563 6.31889543001307e-06
1564 6.32197452432592e-06
1565 6.33123636362143e-06
1566 6.35277774563292e-06
1567 6.39840709482087e-06
1568 6.48952709525474e-06
1569 6.65992911308422e-06
1570 6.9455031734833e-06
1571 7.32801026970265e-06
1572 7.62324907555012e-06
1573 7.51185234548757e-06
1574 7.00039254297735e-06
1575 6.60748901282204e-06
1576 6.62133697915124e-06
1577 6.85549230183824e-06
1578 7.08755533196381e-06
1579 7.18782621333958e-06
1580 7.08331890564295e-06
1581 6.82032441545743e-06
1582 6.55132680549286e-06
1583 6.38129085928085e-06
1584 6.30655404165736e-06
1585 6.28349243925186e-06
1586 6.28164934823872e-06
1587 6.28765201327042e-06
1588 6.29601481705322e-06
1589 6.30389604339143e-06
1590 6.30994418315822e-06
1591 6.31377906756825e-06
1592 6.31561397312908e-06
1593 6.31591865385417e-06
1594 6.31519833405036e-06
1595 6.31375905868481e-06
1596 6.31184275334817e-06
1597 6.30963722869637e-06
1598 6.30718159300159e-06
1599 6.30457543593366e-06
1600 6.30180511507206e-06
1601 6.29890155323665e-06
1602 6.29589385425788e-06
1603 6.2927206272434e-06
1604 6.2894150687498e-06
1605 6.28592488283175e-06
1606 6.28226098342566e-06
1607 6.27836698186002e-06
1608 6.27426425126032e-06
1609 6.269998721109e-06
1610 6.26570272288518e-06
1611 6.2618164520245e-06
1612 6.25916618446354e-06
1613 6.25976144874585e-06
1614 6.26780229140422e-06
1615 6.29209625913063e-06
1616 6.35038395557785e-06
1617 6.47605656922678e-06
1618 6.72121859679464e-06
1619 7.12840255800984e-06
1620 7.61017645345419e-06
1621 7.79386391513981e-06
1622 7.34291052140179e-06
1623 6.70757890475215e-06
1624 6.55481926514767e-06
1625 6.7799087446474e-06
1626 7.01533372193808e-06
1627 7.08729885445791e-06
1628 6.9620473368559e-06
1629 6.71479074298986e-06
1630 6.47945307719056e-06
1631 6.33224317425629e-06
1632 6.26511837253929e-06
1633 6.24291442363756e-06
1634 6.24091444478836e-06
1635 6.24692438577767e-06
1636 6.25508482698933e-06
1637 6.26233941147802e-06
1638 6.26739256404107e-06
1639 6.27004646958085e-06
1640 6.27069766778732e-06
1641 6.26994051344809e-06
1642 6.26817927695811e-06
1643 6.26584187557455e-06
1644 6.26314431428909e-06
1645 6.26023711447488e-06
1646 6.25713164481567e-06
1647 6.25395159659092e-06
1648 6.25067332293838e-06
1649 6.24731364951003e-06
1650 6.24387621428468e-06
1651 6.2403223637375e-06
1652 6.23668211119366e-06
1653 6.23286314294091e-06
1654 6.22887682766304e-06
1655 6.22474317424349e-06
1656 6.22043626208324e-06
1657 6.21611252427101e-06
1658 6.21202480033389e-06
1659 6.20876244283863e-06
1660 6.20766877545975e-06
1661 6.21158324065618e-06
1662 6.22626021140604e-06
1663 6.26326209385297e-06
1664 6.34453499515075e-06
1665 6.50685024083941e-06
1666 6.7940650296805e-06
1667 7.20104935680865e-06
1668 7.5436478255142e-06
1669 7.46074374546879e-06
1670 6.92590401740745e-06
1671 6.5127696871059e-06
1672 6.54442374070641e-06
1673 6.79444201523438e-06
1674 7.01182671036804e-06
1675 7.07473009242676e-06
1676 6.94161326464382e-06
1677 6.68151369609404e-06
1678 6.43371049591224e-06
1679 6.27911003903137e-06
1680 6.20913169768755e-06
1681 6.18645617578295e-06
1682 6.1847381402913e-06
1683 6.19123920841957e-06
1684 6.19999082118738e-06
1685 6.20802984485636e-06
1686 6.21406070422381e-06
1687 6.21777962805936e-06
1688 6.2195672398957e-06
1689 6.21989738647244e-06
1690 6.21920526100439e-06
1691 6.21784101895173e-06
1692 6.21597973804455e-06
1693 6.21380695520202e-06
1694 6.21141407464165e-06
1695 6.20880473434227e-06
1696 6.20603077550186e-06
1697 6.20308264842606e-06
1698 6.20000491835526e-06
1699 6.19672709945007e-06
1700 6.19325419393135e-06
1701 6.18953572484315e-06
1702 6.1855316744186e-06
1703 6.18123294771067e-06
1704 6.17655359747005e-06
1705 6.17152454651659e-06
1706 6.16624720350956e-06
1707 6.16103079664754e-06
1708 6.15678436588496e-06
1709 6.15547469351441e-06
1710 6.16170927969506e-06
1711 6.18535568719381e-06
1712 6.24697122475482e-06
1713 6.38568326394306e-06
1714 6.66233745505451e-06
1715 7.11933671482257e-06
1716 7.62453055358492e-06
1717 7.72058774600737e-06
1718 7.13862164047896e-06
1719 6.54122459309292e-06
1720 6.52443304716144e-06
1721 6.80839866618044e-06
1722 7.01541875969269e-06
1723 7.01189037499717e-06
1724 6.81391202306258e-06
1725 6.54227869745228e-06
1726 6.32694127489231e-06
1727 6.20860510025523e-06
1728 6.16043735135463e-06
1729 6.14816917732242e-06
1730 6.1516229834524e-06
1731 6.16075385551085e-06
1732 6.17023670201888e-06
1733 6.17755222265259e-06
1734 6.18185413259198e-06
1735 6.18344074609922e-06
1736 6.18298054178013e-06
1737 6.18116382611333e-06
1738 6.17854038864607e-06
1739 6.17545492787031e-06
1740 6.17210207565222e-06
1741 6.16864826952224e-06
1742 6.1650921452383e-06
1743 6.16150828136597e-06
1744 6.15789258517907e-06
1745 6.15420412941603e-06
1746 6.15047883911757e-06
1747 6.14663076703437e-06
1748 6.14266173215583e-06
1749 6.13852444075746e-06
1750 6.1342057051661e-06
1751 6.12966096014134e-06
1752 6.12491430729278e-06
1753 6.12006897426909e-06
1754 6.11540053796489e-06
1755 6.11154928265023e-06
1756 6.11005089012906e-06
1757 6.11395807936788e-06
1758 6.12983467362938e-06
1759 6.17055866314331e-06
1760 6.26027804173646e-06
1761 6.4376663431176e-06
1762 6.74213879392482e-06
1763 7.14417546987534e-06
1764 7.41926260161563e-06
1765 7.23552693671081e-06
1766 6.70209374220576e-06
1767 6.39584186501452e-06
1768 6.50371021038154e-06
1769 6.77198886478436e-06
1770 6.98431585988146e-06
1771 7.03002660884522e-06
1772 6.87148985889507e-06
1773 6.59099941913155e-06
1774 6.3354809753946e-06
1775 6.18092144577531e-06
1776 6.11279756412841e-06
1777 6.09197286394192e-06
1778 6.09210201218957e-06
1779 6.10032020631479e-06
1780 6.11055884292e-06
1781 6.11969244346255e-06
1782 6.12643771091825e-06
1783 6.13071051702718e-06
1784 6.13287693340681e-06
1785 6.13349902778282e-06
1786 6.13308066022e-06
1787 6.1319260566961e-06
1788 6.13027259532828e-06
1789 6.12827443546848e-06
1790 6.12603298577596e-06
1791 6.12356370766065e-06
1792 6.1208966144477e-06
1793 6.11806808592519e-06
1794 6.1150199144322e-06
1795 6.11177756582038e-06
1796 6.10829420111259e-06
1797 6.10451024840586e-06
1798 6.10037341175484e-06
1799 6.095835033193e-06
1800 6.09080871072365e-06
1801 6.08521349931834e-06
1802 6.07909714744892e-06
1803 6.07263655183488e-06
1804 6.06647063250421e-06
1805 6.0622883211181e-06
1806 6.06413777859416e-06
1807 6.08117034062161e-06
1808 6.13325164522394e-06
1809 6.26015935267787e-06
1810 6.52816879664897e-06
1811 6.99707788953674e-06
1812 7.55769451643573e-06
1813 7.72549356042873e-06
1814 7.13752660885802e-06
1815 6.48230661681737e-06
1816 6.46426133243949e-06
1817 6.77453499520198e-06
1818 6.98022358847084e-06
1819 6.9474263000302e-06
1820 6.71896032145014e-06
1821 6.43689190837904e-06
1822 6.2273143157654e-06
1823 6.11754467172432e-06
1824 6.07521724305116e-06
1825 6.0668289734167e-06
1826 6.07311176281655e-06
1827 6.08410391578218e-06
1828 6.09445851296186e-06
1829 6.10170627624029e-06
1830 6.10537608736195e-06
1831 6.10605957263033e-06
1832 6.10469032835681e-06
1833 6.10202368989121e-06
1834 6.09865219303174e-06
1835 6.09496737524751e-06
1836 6.09113021710073e-06
1837 6.08728942097514e-06
1838 6.08342270425055e-06
1839 6.07958190812496e-06
1840 6.07573110755766e-06
1841 6.07187075729598e-06
1842 6.06794856139459e-06
1843 6.06395087743294e-06
1844 6.05984314461239e-06
1845 6.05554805588326e-06
1846 6.05101786277373e-06
1847 6.04624892730499e-06
1848 6.0412035054469e-06
1849 6.03593980486039e-06
1850 6.03070657234639e-06
1851 6.02604222876835e-06
1852 6.02321415499318e-06
1853 6.02511590841459e-06
1854 6.03771059104474e-06
1855 6.07295032750699e-06
1856 6.15347425991786e-06
1857 6.31616831014981e-06
1858 6.60119167150697e-06
1859 6.98983058100566e-06
1860 7.28250779502559e-06
1861 7.15193573341821e-06
1862 6.64988601783989e-06
1863 6.32379806120298e-06
1864 6.40979988020263e-06
1865 6.68192024022574e-06
1866 6.91823197485064e-06
1867 6.99868496667477e-06
1868 6.86654493620154e-06
1869 6.58416138321627e-06
1870 6.30362001174944e-06
1871 6.12287840340286e-06
1872 6.03888838668354e-06
1873 6.01116471443675e-06
1874 6.00931025473983e-06
1875 6.01782767262193e-06
1876 6.02920135861496e-06
1877 6.03966509515885e-06
1878 6.04767546974472e-06
1879 6.05291097599547e-06
1880 6.05586001256597e-06
1881 6.05705645284615e-06
1882 6.05706281930907e-06
1883 6.05626428296091e-06
1884 6.05491641181288e-06
1885 6.05318518864806e-06
1886 6.05112109042238e-06
1887 6.04882689003716e-06
1888 6.04634078626987e-06
1889 6.04362639933242e-06
1890 6.04067463427782e-06
1891 6.03751504968386e-06
1892 6.03406988375355e-06
1893 6.03034459345508e-06
1894 6.02617501499481e-06
1895 6.02152931605815e-06
1896 6.01632746111136e-06
1897 6.01043302594917e-06
1898 6.00381827098317e-06
1899 5.99651048105443e-06
1900 5.98891620029463e-06
1901 5.98231827098061e-06
1902 5.97994676354574e-06
1903 5.98950055064051e-06
1904 6.0283537095529e-06
1905 6.13280235484126e-06
1906 6.3682332438475e-06
1907 6.81036863170448e-06
1908 7.40571294954862e-06
1909 7.71334089222364e-06
1910 7.22556296750554e-06
1911 6.48424020255334e-06
1912 6.37506445855252e-06
1913 6.6907232394442e-06
1914 6.92770754540106e-06
1915 6.9128750510572e-06
1916 6.6858519858215e-06
1917 6.39341214991873e-06
1918 6.1705791267741e-06
1919 6.05144714427297e-06
1920 6.00465727984556e-06
1921 5.99519080424216e-06
1922 6.00206112721935e-06
1923 6.01411829848075e-06
1924 6.0253528317844e-06
1925 6.03311173108523e-06
1926 6.03690796197043e-06
1927 6.03743546889746e-06
1928 6.03576972935116e-06
1929 6.03277339905617e-06
1930 6.02908721702988e-06
1931 6.02513682679273e-06
1932 6.02109957981156e-06
1933 6.01707233727211e-06
1934 6.01305737291113e-06
1935 6.00910971115809e-06
1936 6.00516841586796e-06
1937 6.00122621108312e-06
1938 5.99724171479465e-06
1939 5.99316626903601e-06
1940 5.98895030634594e-06
1941 5.98453652855824e-06
1942 5.97990310780006e-06
1943 5.97494590692804e-06
1944 5.96961217524949e-06
1945 5.96396102992003e-06
1946 5.95813025938696e-06
1947 5.95253823121311e-06
1948 5.9482513279363e-06
1949 5.94771790929371e-06
1950 5.95609026277089e-06
1951 5.98404540141928e-06
1952 6.05219656790723e-06
1953 6.19547427049838e-06
1954 6.45656336928369e-06
1955 6.83450889482629e-06
1956 7.16546446710709e-06
1957 7.11856409907341e-06
1958 6.65152856527129e-06
1959 6.27658710072865e-06
1960 6.31720877208863e-06
1961 6.58579665469006e-06
1962 6.84346014168113e-06
1963 6.95873313816264e-06
1964 6.86065914123901e-06
1965 6.5884073592315e-06
1966 6.2901076489652e-06
1967 6.08349273534259e-06
1968 5.9811804931087e-06
1969 5.94452058066963e-06
1970 5.93946788285393e-06
1971 5.94752191318548e-06
1972 5.9596095525194e-06
1973 5.97117423239979e-06
1974 5.9801841416629e-06
1975 5.98630049353233e-06
1976 5.98988162892056e-06
1977 5.99155328018242e-06
1978 5.99192298977869e-06
1979 5.99141321799834e-06
1980 5.99026679992676e-06
1981 5.98872793489136e-06
1982 5.98684800934279e-06
1983 5.9847238844668e-06
1984 5.98232736592763e-06
1985 5.9797453104693e-06
1986 5.97691587245208e-06
1987 5.97382631895016e-06
1988 5.97045982431155e-06
1989 5.96676136410679e-06
1990 5.96261907048756e-06
1991 5.95793426327873e-06
1992 5.95263918512501e-06
1993 5.94656057728571e-06
1994 5.93960567130125e-06
1995 5.93172308072099e-06
1996 5.92316018810379e-06
1997 5.9149306252948e-06
1998 5.90963418289903e-06
1999 5.91391699344967e-06
};
\addlegendentry{Test}

\nextgroupplot[
title={Leaky/Leaky},,
ymin=4.42472763667762e-06, ymax=0.001,
]
\addplot [semithick, black, dashed]
table {%
0 0.00626404779177392
1 0.00624001588585088
2 0.00621866311848862
3 0.00619946165534202
4 0.00618207915977109
5 0.00616630910735694
6 0.00615168228614493
7 0.00613814463940798
8 0.00612532146988087
9 0.00611310430394951
10 0.00610137557305279
11 0.00609004912621458
12 0.00607857881550444
13 0.00606661276106024
14 0.00605319234455237
15 0.00603769588997238
16 0.00601970602292567
17 0.00599934580532135
18 0.00597630691481754
19 0.00595184201665688
20 0.00592705555754947
21 0.00590202422245056
22 0.0058749113813974
23 0.00584584267744503
24 0.00581433785009722
25 0.0057808848850982
26 0.00574591398799384
27 0.00570827529372764
28 0.00566909251574543
29 0.00562924993755587
30 0.00558826367523579
31 0.00554618288515485
32 0.00550232972818776
33 0.00545771850192978
34 0.00541111870370514
35 0.00536126663064351
36 0.00530873161915224
37 0.00525405809821677
38 0.0051962434536108
39 0.00513532823060814
40 0.00507167283831222
41 0.00500486023793201
42 0.00493317512155045
43 0.00485627947637113
44 0.00477223634243273
45 0.00467705556911824
46 0.00457147862471174
47 0.00445209493682341
48 0.00431857099920308
49 0.00416689873327414
50 0.0039883161225589
51 0.00376998701358389
52 0.00351483858048596
53 0.00323698241754755
54 0.00296563197662181
55 0.00270449958316021
56 0.00246372811670881
57 0.00224020178939099
58 0.0020353044874355
59 0.00185511683775985
60 0.00169882868021887
61 0.00156197618116494
62 0.00143531505091232
63 0.00132967083163749
64 0.00123614882613765
65 0.00115273590745346
66 0.00107832250159845
67 0.00101180787396515
68 0.000952594916270755
69 0.000900077400274313
70 0.000853254368394119
71 0.000811174473710707
72 0.000772924411194253
73 0.00073789505950117
74 0.000705801525327843
75 0.000677136624062769
76 0.000650907313456628
77 0.000625890679430086
78 0.000603015072101698
79 0.000582062788225812
80 0.000562555580017943
81 0.000544444087836382
82 0.000527581838468905
83 0.00051182539277761
84 0.000497182657454687
85 0.000483443631082991
86 0.000470338746481502
87 0.00045808265258529
88 0.000446890244575116
89 0.000435671615832689
90 0.000425483728804465
91 0.000416316383507365
92 0.000407202949531893
93 0.000398483112235226
94 0.000389941689036277
95 0.000382387507556814
96 0.00037548956856881
97 0.000368417612776284
98 0.000361487474606292
99 0.000355723906977801
100 0.000349736261910039
101 0.000344039479955427
102 0.000338438905941985
103 0.000333456228304385
104 0.000328919650200987
105 0.000324092078244576
106 0.000319246859248778
107 0.000315273714932118
108 0.000311105738660444
109 0.000306755678707304
110 0.000302758741554499
111 0.000298843148470951
112 0.000295461344990144
113 0.000291827560431557
114 0.000287826506962574
115 0.000284692610364345
116 0.000281684673268501
117 0.000278866490077689
118 0.000276041044116937
119 0.00027358576971892
120 0.000271073860005799
121 0.000268664829036425
122 0.000266390835122365
123 0.000264354559931235
124 0.00026229741985162
125 0.000260263326822496
126 0.000258349655325674
127 0.000256610717997319
128 0.000254869617037912
129 0.000253050578123748
130 0.000251284553030473
131 0.000249662171825094
132 0.000248179884863475
133 0.000246762879896778
134 0.000245311224944089
135 0.000243796956226561
136 0.0002425942830655
137 0.000241369191513741
138 0.000240137722187228
139 0.000238855752854761
140 0.000237658100161298
141 0.00023666269783007
142 0.000235595815411216
143 0.000234520594659671
144 0.000233378926623118
145 0.000232377111132109
146 0.000231411189190567
147 0.000230618640415514
148 0.000229719556472219
149 0.000228830705310656
150 0.000227691059166091
151 0.000226726513432141
152 0.000225938152198069
153 0.000225188628917294
154 0.000224386864402959
155 0.000223533291091371
156 0.000222859269200626
157 0.000221896186133108
158 0.00022113220163078
159 0.00022044738744853
160 0.000219942337452039
161 0.000219293334907888
162 0.000218671604073961
163 0.000218049222894479
164 0.000217219983795758
165 0.000216414559162104
166 0.00021585057112361
167 0.000215477273854958
168 0.0002148797379391
169 0.000214122179528431
170 0.000213577345732574
171 0.00021290351888581
172 0.000212301818706351
173 0.000211861305416505
174 0.000211458976593804
175 0.000210956492153969
176 0.000210426662192731
177 0.000209842625537249
178 0.000209211883614557
179 0.000208716794006136
180 0.00020848410348151
181 0.000207962785182758
182 0.000207478908691883
183 0.000206848469161969
184 0.0002063544032751
185 0.000205948355258556
186 0.000205788125242634
187 0.000205366422221687
188 0.000204903103977472
189 0.00020427973653625
190 0.000203739498587652
191 0.000203314370054386
192 0.000203022079958259
193 0.00020266049537554
194 0.000202185167225366
195 0.000201658159696194
196 0.000201064356161851
197 0.000200663331128226
198 0.000200454297640817
199 0.000200013312110059
200 0.000199517400758964
201 0.000198871185929761
202 0.000198369572217416
203 0.000197966625776758
204 0.000197646385871053
205 0.000197228038317121
206 0.000196635426348735
207 0.00019605981125892
208 0.000195665182403104
209 0.000195248282111038
210 0.000195006775484785
211 0.000194526475596035
212 0.000193938716108732
213 0.000193511982473638
214 0.000192993929445606
215 0.000192570668062331
216 0.000192272626861723
217 0.000191750171069316
218 0.000191183244837134
219 0.000190645035985426
220 0.00019018113995628
221 0.000189701481417615
222 0.00018937445926781
223 0.000188890759659444
224 0.00018824567092679
225 0.000187664334767135
226 0.000187190220827915
227 0.000186867788968925
228 0.000186263934217834
229 0.000185705244689416
230 0.000185025208203626
231 0.000184463045727057
232 0.000183933156726823
233 0.000183557989508643
234 0.000182927856549497
235 0.000182241122033133
236 0.000181622931570757
237 0.000181320163797238
238 0.000180764700154157
239 0.00018008477952236
240 0.000179380414671471
241 0.000178918479207368
242 0.000178386223254279
243 0.000177749592161547
244 0.00017705700867765
245 0.000176396324739869
246 0.000175999841182772
247 0.000175394434378973
248 0.000174677898513664
249 0.000173945663775044
250 0.000173359842555953
251 0.000172730419279787
252 0.000172207849061579
253 0.000171586556135139
254 0.000170752790964457
255 0.000170081338296768
256 0.000169581831869436
257 0.000168941767704212
258 0.000168166087867405
259 0.000167362745656874
260 0.000166735176570398
261 0.000166064881895522
262 0.00016534456661077
263 0.000164534111654291
264 0.00016373702115402
265 0.000163015935726207
266 0.000162338544498652
267 0.000161833745629281
268 0.000160722724039886
269 0.000160216768584576
270 0.000159327172198687
271 0.000158433245260881
272 0.000157587130729553
273 0.000156707578561566
274 0.000155691597228724
275 0.000155195535775476
276 0.000154200906671065
277 0.000153138043515355
278 0.000152106503200855
279 0.000151064609752893
280 0.000150070541060643
281 0.000149056348959675
282 0.000148073455008557
283 0.000147128475994407
284 0.000146153372632796
285 0.000145164700640521
286 0.000144248540479452
287 0.000143225326610263
288 0.000142248923317823
289 0.000141141819433699
290 0.000140560173178983
291 0.000139256713971747
292 0.000138197783478233
293 0.000137169781019963
294 0.000136214514007804
295 0.000135254597559253
296 0.000134284388018102
297 0.000133331922469893
298 0.000132390897086054
299 0.000131398079084022
300 0.000130446588521238
301 0.000129538143966101
302 0.00012856296320507
303 0.000127620085166313
304 0.000126680174901139
305 0.000125724793079485
306 0.000124727052558171
307 0.00012379732494594
308 0.000122808831477528
309 0.000121916798349275
310 0.000120891963433678
311 0.000119959883214449
312 0.0001189879143908
313 0.000118028145557503
314 0.000117068236484386
315 0.000116130322240338
316 0.000115169387179037
317 0.000114249172554537
318 0.000113283530666308
319 0.000112390520015992
320 0.000111415276784044
321 0.000110510290610932
322 0.000109605694802895
323 0.000108703493509665
324 0.00010776260026546
325 0.000106822570863585
326 0.0001059479446468
327 0.000105046599117031
328 0.000104141722061968
329 0.000103196956203533
330 0.000102306577360878
331 0.00010140416958393
332 0.000100414852006736
333 9.95006279538302e-05
334 9.85829572854868e-05
335 9.76420461782368e-05
336 9.67193795986532e-05
337 9.57966109709218e-05
338 9.48644991396463e-05
339 9.39626492311163e-05
340 9.30421352478561e-05
341 9.21366371997578e-05
342 9.12514636084438e-05
343 9.04057295585403e-05
344 8.95200858792577e-05
345 8.8679451963003e-05
346 8.78443248950589e-05
347 8.69784758847914e-05
348 8.61699027936424e-05
349 8.5318135930379e-05
350 8.45022657998129e-05
351 8.37423367840984e-05
352 8.28847106788544e-05
353 8.21537422410756e-05
354 8.13468792841832e-05
355 8.05815445446001e-05
356 7.97923618591767e-05
357 7.90405082113921e-05
358 7.82864458344079e-05
359 7.75308643952144e-05
360 7.68367416696947e-05
361 7.60714239191884e-05
362 7.53903403847289e-05
363 7.46925021459788e-05
364 7.39466047576798e-05
365 7.33048125169944e-05
366 7.26134616400032e-05
367 7.19413434211447e-05
368 7.1315032784014e-05
369 7.06143129676207e-05
370 7.00059190990032e-05
371 6.93762015799848e-05
372 6.8726295289423e-05
373 6.81265512199047e-05
374 6.75189567189705e-05
375 6.69578656555814e-05
376 6.62878462875938e-05
377 6.56451196334729e-05
378 6.51006014393829e-05
379 6.44763935682136e-05
380 6.39112054869884e-05
381 6.33646509271557e-05
382 6.27575159199978e-05
383 6.22331895172579e-05
384 6.16839370408684e-05
385 6.10792202166976e-05
386 6.05760812391054e-05
387 6.00489730686604e-05
388 5.95045248346082e-05
389 5.89731080822276e-05
390 5.84775698371232e-05
391 5.79576911619029e-05
392 5.74397744728117e-05
393 5.69402456420676e-05
394 5.64586706772729e-05
395 5.59912206909985e-05
396 5.54767206395468e-05
397 5.49873247592814e-05
398 5.45417770823065e-05
399 5.40656528542627e-05
400 5.35885826096205e-05
401 5.30948591119795e-05
402 5.26351921195101e-05
403 5.21819501386744e-05
404 5.17118647636039e-05
405 5.12885675547636e-05
406 5.08532073695278e-05
407 5.04572647272994e-05
408 5.00035216006722e-05
409 4.96241052942992e-05
410 4.91972063159096e-05
411 4.87853698487584e-05
412 4.83962415742667e-05
413 4.80498465620371e-05
414 4.76414155485827e-05
415 4.72800755346725e-05
416 4.69195715169235e-05
417 4.6523614308569e-05
418 4.61986532371839e-05
419 4.58230324156261e-05
420 4.54603195620962e-05
421 4.51443769051707e-05
422 4.48147164604507e-05
423 4.44834709654174e-05
424 4.41028498983087e-05
425 4.38024841642459e-05
426 4.34915447904416e-05
427 4.31473259858706e-05
428 4.28319628049678e-05
429 4.25497836360478e-05
430 4.22314224053366e-05
431 4.19132772364605e-05
432 4.16341578421964e-05
433 4.13487144044211e-05
434 4.10354745490338e-05
435 4.07667195219119e-05
436 4.04849278545782e-05
437 4.01829462930436e-05
438 3.99106353974332e-05
439 3.96506570332633e-05
440 3.93592900351791e-05
441 3.9102067347585e-05
442 3.88354229023946e-05
443 3.85568170528927e-05
444 3.83175634368627e-05
445 3.80668388686445e-05
446 3.78013029163071e-05
447 3.75616738637063e-05
448 3.73206100938717e-05
449 3.70476578446244e-05
450 3.68211770336302e-05
451 3.65918549540822e-05
452 3.63312383129255e-05
453 3.60998393844625e-05
454 3.58786531862165e-05
455 3.56379712087573e-05
456 3.54210008772782e-05
457 3.51991154623477e-05
458 3.49672647139698e-05
459 3.4736445925887e-05
460 3.4546680478087e-05
461 3.42922856759742e-05
462 3.41101459575555e-05
463 3.39007960299398e-05
464 3.36828856113414e-05
465 3.34656275704504e-05
466 3.32885465432753e-05
467 3.30460441908542e-05
468 3.28736609098712e-05
469 3.26610928418347e-05
470 3.24663439315032e-05
471 3.22656797138166e-05
472 3.20918541945048e-05
473 3.18945381359015e-05
474 3.16953803825015e-05
475 3.15397951524687e-05
476 3.13491832457657e-05
477 3.11771936942762e-05
478 3.09954635149268e-05
479 3.08177333607773e-05
480 3.06465969046599e-05
481 3.04732530942431e-05
482 3.03053812764631e-05
483 3.01357637475519e-05
484 2.99804784056334e-05
485 2.98088839549848e-05
486 2.96649328070941e-05
487 2.94887807399391e-05
488 2.93466939425002e-05
489 2.91926843942747e-05
490 2.90399707481015e-05
491 2.8881630683486e-05
492 2.8728743892259e-05
493 2.86013247823291e-05
494 2.84369494139014e-05
495 2.82976377725674e-05
496 2.81562486712517e-05
497 2.80069397149418e-05
498 2.78718486157459e-05
499 2.77213646313612e-05
500 2.75692170994546e-05
501 2.74528446020383e-05
502 2.73030046429312e-05
503 2.71737056092292e-05
504 2.7024079361837e-05
505 2.68914151142496e-05
506 2.6772177649903e-05
507 2.6642822041012e-05
508 2.65001315664648e-05
509 2.63709666317169e-05
510 2.62488719648957e-05
511 2.61372326733067e-05
512 2.60015092123922e-05
513 2.5877610866587e-05
514 2.57668652583476e-05
515 2.56538816323371e-05
516 2.55214670659143e-05
517 2.54041176219744e-05
518 2.52992449141232e-05
519 2.51764803209653e-05
520 2.50439991411611e-05
521 2.49444713773528e-05
522 2.48445164530153e-05
523 2.47450285755235e-05
524 2.46202357168102e-05
525 2.44964166302708e-05
526 2.44228953238945e-05
527 2.43695657893284e-05
528 2.4267030703129e-05
529 2.40911093065677e-05
530 2.39764620442173e-05
531 2.38907660055077e-05
532 2.3813279049989e-05
533 2.37057509338001e-05
534 2.3594638356883e-05
535 2.34997062555919e-05
536 2.34546097583888e-05
537 2.33847684896205e-05
538 2.32250474727635e-05
539 2.3114171852967e-05
540 2.30190958330212e-05
541 2.2936463091483e-05
542 2.28905567691839e-05
543 2.28336958638664e-05
544 2.26935790728078e-05
545 2.25591721854101e-05
546 2.24741016108254e-05
547 2.24253196261515e-05
548 2.23855546348517e-05
549 2.23024940364525e-05
550 2.21773865103359e-05
551 2.20746333798871e-05
552 2.20363038270222e-05
553 2.19580940257202e-05
554 2.18663032125477e-05
555 2.17692633857069e-05
556 2.16939587609488e-05
557 2.16599723312783e-05
558 2.15997200516682e-05
559 2.14716107826973e-05
560 2.13800391808405e-05
561 2.1289457670548e-05
562 2.12407729840436e-05
563 2.12113232436195e-05
564 2.11653933757106e-05
565 2.10724247633465e-05
566 2.0983920126838e-05
567 2.09085710025647e-05
568 2.08464953814769e-05
569 2.07568471175534e-05
570 2.06972449934284e-05
571 2.06656761037038e-05
572 2.05847264798109e-05
573 2.05198498655079e-05
574 2.04495788889147e-05
575 2.03785872994189e-05
576 2.03328605294928e-05
577 2.02623909100197e-05
578 2.0167962313522e-05
579 2.01005627022965e-05
580 2.00548109763332e-05
581 2.00037095883232e-05
582 1.9949410990705e-05
583 1.98750254742919e-05
584 1.97933487466884e-05
585 1.97031955888605e-05
586 1.96422929210627e-05
587 1.95806904343954e-05
588 1.95487002638828e-05
589 1.9548278675785e-05
590 1.9485836098454e-05
591 1.93899610732018e-05
592 1.93283750657969e-05
593 1.92651684258749e-05
594 1.9184324958843e-05
595 1.91250564611778e-05
596 1.90768090622129e-05
597 1.9020796232283e-05
598 1.89577144169562e-05
599 1.88923117008244e-05
600 1.88618440795096e-05
601 1.88371032177059e-05
602 1.87590186495612e-05
603 1.86908921264717e-05
604 1.86678928688622e-05
605 1.86096577063921e-05
606 1.85415448186177e-05
607 1.84835590797405e-05
608 1.84361230850527e-05
609 1.83834333977728e-05
610 1.83256193384551e-05
611 1.82580354302075e-05
612 1.81876212863585e-05
613 1.81562557557413e-05
614 1.81122278579693e-05
615 1.807168130874e-05
616 1.80125787814234e-05
617 1.7969663048234e-05
618 1.79165966294192e-05
619 1.78823987475596e-05
620 1.78022416381651e-05
621 1.77392819136202e-05
622 1.76987729609834e-05
623 1.76411896664064e-05
624 1.75538242590534e-05
625 1.75156278903188e-05
626 1.7465258816074e-05
627 1.74196051574427e-05
628 1.73869564719098e-05
629 1.73400059537698e-05
630 1.72908034237196e-05
631 1.72373014404315e-05
632 1.72360548091888e-05
633 1.71998052937994e-05
634 1.71562429862604e-05
635 1.70795411289504e-05
636 1.70155901297875e-05
637 1.69564970100566e-05
638 1.68824313213634e-05
639 1.68705203993369e-05
640 1.6811642071346e-05
641 1.67739502767716e-05
642 1.67136952260449e-05
643 1.66563865864333e-05
644 1.66474696801089e-05
645 1.66191192043641e-05
646 1.65998729269035e-05
647 1.65695219429551e-05
648 1.65404338456909e-05
649 1.65407589900468e-05
650 1.65413984714036e-05
651 1.65112790284638e-05
652 1.64537954514543e-05
653 1.64191219198528e-05
654 1.63408568685242e-05
655 1.62294237853189e-05
656 1.61527211588464e-05
657 1.60805336548719e-05
658 1.60082251028371e-05
659 1.59523030358599e-05
660 1.58837884018936e-05
661 1.5858189851059e-05
662 1.5791564818457e-05
663 1.57567626377642e-05
664 1.57515028949007e-05
665 1.57598648726776e-05
666 1.57602912693733e-05
667 1.57937676785735e-05
668 1.58745962863804e-05
669 1.59937229380347e-05
670 1.60757697571512e-05
671 1.61887854766007e-05
672 1.6183262374625e-05
673 1.60854793449516e-05
674 1.58945798105492e-05
675 1.5639126084821e-05
676 1.54812066970322e-05
677 1.54081928638305e-05
678 1.54193837591521e-05
679 1.53941353424614e-05
680 1.53350579896738e-05
681 1.52091122060938e-05
682 1.50708062651717e-05
683 1.49657096653044e-05
684 1.49185759266146e-05
685 1.49199995593818e-05
686 1.49927368351399e-05
687 1.50614866729626e-05
688 1.51512501052053e-05
689 1.5213243115042e-05
690 1.52337099326161e-05
691 1.52618336493049e-05
692 1.52204834087399e-05
693 1.51246981303643e-05
694 1.4964902192105e-05
695 1.4806584710314e-05
696 1.46858029985708e-05
697 1.46059912768948e-05
698 1.45518073306761e-05
699 1.45080443996193e-05
700 1.4459742036621e-05
701 1.44142515186019e-05
702 1.43433802204385e-05
703 1.43034813131493e-05
704 1.42528102777817e-05
705 1.42375332483624e-05
706 1.4261033669527e-05
707 1.43479385616985e-05
708 1.44380143130007e-05
709 1.45451336077684e-05
710 1.46866138326374e-05
711 1.48425214305803e-05
712 1.49490016845277e-05
713 1.49340543167398e-05
714 1.47591915684586e-05
715 1.44858992392471e-05
716 1.42306585200203e-05
717 1.41093538434944e-05
718 1.40811953794895e-05
719 1.40911395742904e-05
720 1.40857192523214e-05
721 1.40209956249748e-05
722 1.39052007739338e-05
723 1.37871973961978e-05
724 1.36823715202183e-05
725 1.36255738905078e-05
726 1.3614482933022e-05
727 1.36288489276382e-05
728 1.367722116008e-05
729 1.37303284972745e-05
730 1.37859260114226e-05
731 1.3821598752628e-05
732 1.38435061494135e-05
733 1.38418621453695e-05
734 1.37818837888659e-05
735 1.37300151905606e-05
736 1.36200835925138e-05
737 1.35171810367041e-05
738 1.3414022040692e-05
739 1.33574839651374e-05
740 1.33262383492649e-05
741 1.32918813360305e-05
742 1.3294882815984e-05
743 1.32681135589507e-05
744 1.3253381830225e-05
745 1.32238812877716e-05
746 1.31872143427358e-05
747 1.3110133235017e-05
748 1.30548386820806e-05
749 1.30018155726219e-05
750 1.2958372215266e-05
751 1.29727233684207e-05
752 1.30251171928109e-05
753 1.31652292942874e-05
754 1.33890896449884e-05
755 1.37868024125964e-05
756 1.4374710612941e-05
757 1.49691932520568e-05
758 1.51654584819028e-05
759 1.45079381539404e-05
760 1.36810521329522e-05
761 1.32701817463499e-05
762 1.32934921452943e-05
763 1.32818750451236e-05
764 1.307872938483e-05
765 1.28437425583883e-05
766 1.26595943434893e-05
767 1.25653602438547e-05
768 1.25301421984858e-05
769 1.25000099355788e-05
770 1.24992052370487e-05
771 1.2483663653029e-05
772 1.24622258024232e-05
773 1.24479966512325e-05
774 1.24232072273855e-05
775 1.24016864280208e-05
776 1.23773601341526e-05
777 1.23577218094084e-05
778 1.23414436146163e-05
779 1.2333975041301e-05
780 1.23276689674157e-05
781 1.22908221378282e-05
782 1.22861788351969e-05
783 1.22687507495556e-05
784 1.22941015394673e-05
785 1.22909838502494e-05
786 1.22949055523236e-05
787 1.2306700496012e-05
788 1.23253247377164e-05
789 1.23419468476271e-05
790 1.23518055978877e-05
791 1.23576206050302e-05
792 1.23655255652722e-05
793 1.23351825553897e-05
794 1.23032887699992e-05
795 1.22612583997039e-05
796 1.22181373978236e-05
797 1.21691748162789e-05
798 1.20994022161369e-05
799 1.2032476824686e-05
800 1.19898304902222e-05
801 1.19397333833149e-05
802 1.19231382473828e-05
803 1.19541520060551e-05
804 1.19688586650568e-05
805 1.20208227727403e-05
806 1.20817485527169e-05
807 1.21647822517446e-05
808 1.22373289546118e-05
809 1.22925110233041e-05
810 1.22610160353531e-05
811 1.21822743253119e-05
812 1.20413131590169e-05
813 1.18695204438524e-05
814 1.17196751006077e-05
815 1.16646742078785e-05
816 1.17186170207617e-05
817 1.18332114524833e-05
818 1.20871072795126e-05
819 1.24673824828392e-05
820 1.29277314506737e-05
821 1.34640365523353e-05
822 1.36411419582316e-05
823 1.31488717425299e-05
824 1.23897477823931e-05
825 1.19123099473484e-05
826 1.18020536898911e-05
827 1.18000805535701e-05
828 1.16981295796847e-05
829 1.15192231948669e-05
830 1.14052383750618e-05
831 1.13465611537578e-05
832 1.13208331802284e-05
833 1.1328672105293e-05
834 1.13356214432869e-05
835 1.13306975340066e-05
836 1.13298792205896e-05
837 1.13483739552223e-05
838 1.13289013645712e-05
839 1.13206615388606e-05
840 1.12993663092809e-05
841 1.12977102340039e-05
842 1.12914200460779e-05
843 1.12753682692812e-05
844 1.12622016015251e-05
845 1.12442115147715e-05
846 1.12506069394058e-05
847 1.12441763633342e-05
848 1.12384014325784e-05
849 1.12358405521817e-05
850 1.12316629294895e-05
851 1.12210879752439e-05
852 1.12406781243024e-05
853 1.12396675460147e-05
854 1.12455807315115e-05
855 1.12359050170596e-05
856 1.12550246953447e-05
857 1.1256324604858e-05
858 1.12466074320139e-05
859 1.1255718858294e-05
860 1.12364280422383e-05
861 1.12457146173028e-05
862 1.1214356669953e-05
863 1.1174174618489e-05
864 1.11269353730847e-05
865 1.10903332943835e-05
866 1.10424193673353e-05
867 1.10023644559476e-05
868 1.09881749406071e-05
869 1.09421280445332e-05
870 1.09416958000708e-05
871 1.09736038318431e-05
872 1.10251857492472e-05
873 1.11656053221054e-05
874 1.14134467139593e-05
875 1.17660009628651e-05
876 1.21608723446798e-05
877 1.24570780108968e-05
878 1.2311939393328e-05
879 1.1741295418588e-05
880 1.11478751936289e-05
881 1.09528378349211e-05
882 1.10541491213922e-05
883 1.11950595975685e-05
884 1.12541569894375e-05
885 1.12039522779384e-05
886 1.10707655256448e-05
887 1.08931519324784e-05
888 1.07750469720003e-05
889 1.06930556249552e-05
890 1.06576626102139e-05
891 1.06355263200797e-05
892 1.06026522423974e-05
893 1.05882877541319e-05
894 1.05850251230777e-05
895 1.05809604180607e-05
896 1.05926401401391e-05
897 1.0596560213294e-05
898 1.05966603731744e-05
899 1.06390879484053e-05
900 1.06520704710178e-05
901 1.06742286298811e-05
902 1.06946333939817e-05
903 1.07078987028331e-05
904 1.06946596574176e-05
905 1.07011389394174e-05
906 1.06642314179339e-05
907 1.06410893110365e-05
908 1.06063621903019e-05
909 1.05553589806462e-05
910 1.0536612887968e-05
911 1.04924330415201e-05
912 1.04798603786094e-05
913 1.04547176116299e-05
914 1.04228608543977e-05
915 1.04037520927136e-05
916 1.03914832569885e-05
917 1.03790178362928e-05
918 1.03576245464865e-05
919 1.03413699434896e-05
920 1.03350295139748e-05
921 1.03281816148382e-05
922 1.03122869461458e-05
923 1.03227911036896e-05
924 1.03108282702991e-05
925 1.03313233976721e-05
926 1.03869017165081e-05
927 1.04658616413289e-05
928 1.06617630155625e-05
929 1.09463298674228e-05
930 1.14865163656219e-05
931 1.22322322262391e-05
932 1.29349981676796e-05
933 1.29596976634616e-05
934 1.21588153536578e-05
935 1.12765168385209e-05
936 1.12012392126104e-05
937 1.13559675547492e-05
938 1.13453535419694e-05
939 1.08698964247367e-05
940 1.04231075184202e-05
941 1.01681630422235e-05
942 1.00975613097631e-05
943 1.00760591195126e-05
944 1.00685585753268e-05
945 1.00431557585878e-05
946 1.0037541199992e-05
947 1.00231911366322e-05
948 1.00080782559075e-05
949 1.00047826192551e-05
950 9.98045413869164e-06
951 9.97816841419308e-06
952 9.97576104211362e-06
953 9.96871433933677e-06
954 9.95115887469922e-06
955 9.94392150133194e-06
956 9.92851086412117e-06
957 9.91788659643333e-06
958 9.91341892842001e-06
959 9.89455501532177e-06
960 9.88177901373177e-06
961 9.88525164835607e-06
962 9.88352879893739e-06
963 9.87374131167229e-06
964 9.87522823692188e-06
965 9.88454367956138e-06
966 9.89958537456204e-06
967 9.93165047358247e-06
968 9.97970146787708e-06
969 1.00681439043626e-05
970 1.01866293213959e-05
971 1.03348929059166e-05
972 1.05530689591404e-05
973 1.08794448330229e-05
974 1.10671977324728e-05
975 1.10261976242754e-05
976 1.07393205475148e-05
977 1.03461562996898e-05
978 1.01428773184864e-05
979 1.01611122875767e-05
980 1.04077022768934e-05
981 1.06198471918617e-05
982 1.07359538432661e-05
983 1.05944076658204e-05
984 1.02986789016768e-05
985 9.93046480424198e-06
986 9.73177958485394e-06
987 9.68550697244552e-06
988 9.70263022193052e-06
989 9.69644763415545e-06
990 9.68841556225897e-06
991 9.6551066333106e-06
992 9.64501775868598e-06
993 9.61749853178873e-06
994 9.61262465271062e-06
995 9.5977309726436e-06
996 9.58360192626628e-06
997 9.57066301943144e-06
998 9.5622652871441e-06
999 9.55749891495827e-06
1000 9.55689357517997e-06
1001 9.54304570477404e-06
1002 9.53840141981743e-06
1003 9.52225641004389e-06
1004 9.51945942162524e-06
1005 9.52495834916078e-06
1006 9.53286051963431e-06
1007 9.53599953756878e-06
1008 9.54940843556074e-06
1009 9.57698326953249e-06
1010 9.61924963061023e-06
1011 9.66443640049164e-06
1012 9.76993096113432e-06
1013 9.8497827760724e-06
1014 9.96630602045911e-06
1015 1.00959041464677e-05
1016 1.02520801252837e-05
1017 1.03751458588874e-05
1018 1.03841088030521e-05
1019 1.02788802127662e-05
1020 1.00287914008845e-05
1021 9.81735864336031e-06
1022 9.68128644807109e-06
1023 9.81433955971056e-06
1024 1.0046066238445e-05
1025 1.03686696775185e-05
1026 1.06380032285358e-05
1027 1.05894606896584e-05
1028 1.01785684734068e-05
1029 9.71903912727612e-06
1030 9.46319471673007e-06
1031 9.38772908831709e-06
1032 9.40279679184641e-06
1033 9.40942617333462e-06
1034 9.40133582183478e-06
1035 9.38196322852747e-06
1036 9.34616529058019e-06
1037 9.31569783624298e-06
1038 9.30088781458949e-06
1039 9.27522834537342e-06
1040 9.26915457721833e-06
1041 9.26341716578349e-06
1042 9.24658841583437e-06
1043 9.23514444206575e-06
1044 9.22448293838585e-06
1045 9.22903988964663e-06
1046 9.2222011929266e-06
1047 9.21483253257804e-06
1048 9.21295762346119e-06
1049 9.20303249518639e-06
1050 9.18988513376462e-06
1051 9.18411461192647e-06
1052 9.17545089063054e-06
1053 9.17942398981353e-06
1054 9.18317802955926e-06
1055 9.18994644916182e-06
1056 9.21411736953814e-06
1057 9.24889567421872e-06
1058 9.34989201173408e-06
1059 9.48274352374767e-06
1060 9.64820743654116e-06
1061 9.94784698171713e-06
1062 1.04010929575793e-05
1063 1.10112357916137e-05
1064 1.14571621914195e-05
1065 1.12598844150824e-05
1066 1.05490260633445e-05
1067 9.95985335983107e-06
1068 9.95366268519149e-06
1069 1.02157422894322e-05
1070 1.03691619948165e-05
1071 1.00425057887321e-05
1072 9.54787307527738e-06
1073 9.18073546518627e-06
1074 9.05601271572465e-06
1075 9.0406190667025e-06
1076 9.03869796697876e-06
1077 9.04594534745229e-06
1078 9.02728549689868e-06
1079 9.00811393567125e-06
1080 8.99419610611574e-06
1081 8.99460763825743e-06
1082 8.99820879407542e-06
1083 9.00663149305814e-06
1084 8.99509587792835e-06
1085 8.98449203923235e-06
1086 8.97088929185941e-06
1087 8.94549178287463e-06
1088 8.9502194491331e-06
1089 8.95569832159993e-06
1090 8.94961059128008e-06
1091 8.94356116365458e-06
1092 8.93006960289711e-06
1093 8.92615816283637e-06
1094 8.92101457772299e-06
1095 8.92390800188991e-06
1096 8.925152887862e-06
1097 8.93423647330138e-06
1098 8.95120931510007e-06
1099 8.96199031963363e-06
1100 8.99607564797122e-06
1101 9.02257369439496e-06
1102 9.04270933332896e-06
1103 9.03144523078936e-06
1104 9.02696607241182e-06
1105 9.00580827867259e-06
1106 8.9814723711612e-06
1107 8.95824771696141e-06
1108 8.91981001416298e-06
1109 8.90167325273694e-06
1110 8.87253939740873e-06
1111 8.8536787021809e-06
1112 8.85797614280648e-06
1113 8.8940780598179e-06
1114 8.96939430283794e-06
1115 9.13725681161992e-06
1116 9.40488702205755e-06
1117 9.92767440699538e-06
1118 1.07809245299961e-05
1119 1.17975659676972e-05
1120 1.22058130322955e-05
1121 1.14300357516584e-05
1122 1.01928589604228e-05
1123 9.76233883953626e-06
1124 9.8171027751448e-06
1125 9.77241105104554e-06
1126 9.3584255544954e-06
1127 8.95253073274205e-06
1128 8.76683932826694e-06
1129 8.72519408545713e-06
1130 8.72293437126359e-06
1131 8.71838826999038e-06
1132 8.70226549753994e-06
1133 8.68302350731653e-06
1134 8.6740907185856e-06
1135 8.66354552098159e-06
1136 8.67007335969561e-06
1137 8.67124251513474e-06
1138 8.66260648102468e-06
1139 8.66266729904197e-06
1140 8.64483179174869e-06
1141 8.63125100192264e-06
1142 8.63707688925075e-06
1143 8.63788315275116e-06
1144 8.62669430912177e-06
1145 8.62265615531044e-06
1146 8.61646623295798e-06
1147 8.61116358041869e-06
1148 8.61421614395397e-06
1149 8.62360097997339e-06
1150 8.63033792875001e-06
1151 8.64951268209069e-06
1152 8.65339791289443e-06
1153 8.66666333543264e-06
1154 8.69234502687277e-06
1155 8.70449512824223e-06
1156 8.72687959763141e-06
1157 8.72026016551786e-06
1158 8.69728732943997e-06
1159 8.65258715521122e-06
1160 8.62373987153831e-06
1161 8.58861781072306e-06
1162 8.56018975259332e-06
1163 8.5700116683185e-06
1164 8.5706049368639e-06
1165 8.59581747736371e-06
1166 8.66656733311544e-06
1167 8.79399608333387e-06
1168 8.9556844775629e-06
1169 9.32069464454344e-06
1170 9.92971976998547e-06
1171 1.07136553424425e-05
1172 1.13391143479191e-05
1173 1.11152490944733e-05
1174 1.02114514550777e-05
1175 9.46683010027982e-06
1176 9.40281923966779e-06
1177 9.5457243958208e-06
1178 9.45951637021381e-06
1179 9.0598188222657e-06
1180 8.69172445128186e-06
1181 8.50965623655497e-06
1182 8.46115704256079e-06
1183 8.44778075581587e-06
1184 8.44697107016401e-06
1185 8.42345925722299e-06
1186 8.40970457183232e-06
1187 8.39698242938169e-06
1188 8.38893800647611e-06
1189 8.39039945788045e-06
1190 8.40158047665795e-06
1191 8.40788126765801e-06
1192 8.40807102431285e-06
1193 8.39874609681601e-06
1194 8.38783257606224e-06
1195 8.37729213465366e-06
1196 8.37876875348087e-06
1197 8.38635207944805e-06
1198 8.38369326405797e-06
1199 8.38440075856539e-06
1200 8.37181715240831e-06
1201 8.35012189703832e-06
1202 8.34401999050272e-06
1203 8.34112210590376e-06
1204 8.35290606016059e-06
1205 8.35613542982117e-06
1206 8.3534848123179e-06
1207 8.34505659330631e-06
1208 8.33548097567416e-06
1209 8.33031972202747e-06
1210 8.33982584058646e-06
1211 8.35749444227929e-06
1212 8.3626925757585e-06
1213 8.38906875166856e-06
1214 8.42966039016346e-06
1215 8.46114783215057e-06
1216 8.4645899232072e-06
1217 8.46832013934318e-06
1218 8.4670843900625e-06
1219 8.48829689203967e-06
1220 8.51579451932594e-06
1221 8.53872688999502e-06
1222 8.55154610235331e-06
1223 8.53574719528183e-06
1224 8.47966229322594e-06
1225 8.42522485289265e-06
1226 8.37522710117611e-06
1227 8.34311860753445e-06
1228 8.29894917497853e-06
1229 8.29534866131354e-06
1230 8.31567520531706e-06
1231 8.40886456465739e-06
1232 8.60922712142553e-06
1233 9.04584629424221e-06
1234 9.80651281423661e-06
1235 1.10296254014486e-05
1236 1.22599103633547e-05
1237 1.21001004877996e-05
1238 1.04960864764081e-05
1239 9.36853424704509e-06
1240 9.20216475996938e-06
1241 9.18829075402527e-06
1242 8.90916525975172e-06
1243 8.46502920914816e-06
1244 8.21456957034172e-06
1245 8.15105524942794e-06
1246 8.14586497011049e-06
1247 8.13798079590811e-06
1248 8.11395158528683e-06
1249 8.10090849334699e-06
1250 8.09211632724072e-06
1251 8.08354729553429e-06
1252 8.0838228733171e-06
1253 8.08774745042484e-06
1254 8.08935932905541e-06
1255 8.07144775372137e-06
1256 8.06400060149315e-06
1257 8.05849821450266e-06
1258 8.05012186244625e-06
1259 8.05563932626541e-06
1260 8.06083260229684e-06
1261 8.05811530923961e-06
1262 8.04327727266241e-06
1263 8.03469403365398e-06
1264 8.03054539044723e-06
1265 8.02120881182589e-06
1266 8.02307440306294e-06
1267 8.02512489439522e-06
1268 8.03136991578413e-06
1269 8.02958282797306e-06
1270 8.02930820498204e-06
1271 8.03065120358326e-06
1272 8.04580503910302e-06
1273 8.06068783365532e-06
1274 8.10446912247187e-06
1275 8.16169578499171e-06
1276 8.24533061027921e-06
1277 8.30928141759557e-06
1278 8.32849246457101e-06
1279 8.33527202104278e-06
1280 8.34403542526729e-06
1281 8.31856596938962e-06
1282 8.26561066169518e-06
1283 8.19288205811119e-06
1284 8.12042731457296e-06
1285 8.05285599092542e-06
1286 8.02204137695384e-06
1287 8.0334801548787e-06
1288 8.08906227511841e-06
1289 8.22346916340422e-06
1290 8.45386168268902e-06
1291 8.86167551783501e-06
1292 9.4574795355129e-06
1293 1.01052122474243e-05
1294 1.0488808513287e-05
1295 1.00775525773145e-05
1296 9.24194426588087e-06
1297 8.61219363912369e-06
1298 8.59330449820561e-06
1299 8.65329049837271e-06
1300 8.67270971838252e-06
1301 8.42647913223971e-06
1302 8.14775594282224e-06
1303 7.97021629406203e-06
1304 7.9060334510217e-06
1305 7.89106491350111e-06
1306 7.8838498156486e-06
1307 7.86007322783888e-06
1308 7.85136581171741e-06
1309 7.84330131242683e-06
1310 7.83081009281261e-06
1311 7.83031906159692e-06
1312 7.83716165742021e-06
1313 7.83622069100431e-06
1314 7.82935550613928e-06
1315 7.82958611011964e-06
1316 7.82504922458571e-06
1317 7.82485588057824e-06
1318 7.81625187595836e-06
1319 7.81441291408669e-06
1320 7.80770367736494e-06
1321 7.79737367917477e-06
1322 7.7882399267537e-06
1323 7.78475778062671e-06
1324 7.78997656247782e-06
1325 7.79602845657479e-06
1326 7.79694396868535e-06
1327 7.78241518784029e-06
1328 7.77431095411174e-06
1329 7.77157926190597e-06
1330 7.75829749066759e-06
1331 7.75530969399085e-06
1332 7.75930008867221e-06
1333 7.76315183781406e-06
1334 7.76111568789872e-06
1335 7.75718588119645e-06
1336 7.75567861577287e-06
1337 7.75656098106481e-06
1338 7.76958655546167e-06
1339 7.80513035358865e-06
1340 7.86819368947533e-06
1341 8.01388523541391e-06
1342 8.23473426336818e-06
1343 8.59127908725554e-06
1344 9.03243017802424e-06
1345 9.5711222662942e-06
1346 9.98284512121472e-06
1347 9.87242616545814e-06
1348 9.31264943737631e-06
1349 8.6394606375606e-06
1350 8.2405841901334e-06
1351 8.41391431016802e-06
1352 8.67409044857936e-06
1353 9.00195008490812e-06
1354 9.08793147047504e-06
1355 8.8222044833941e-06
1356 8.3269920612139e-06
1357 7.97121457285499e-06
1358 7.7744622402065e-06
1359 7.77183257572034e-06
1360 7.74705517514462e-06
1361 7.70180175280188e-06
1362 7.66444682653855e-06
1363 7.6458942945834e-06
1364 7.65262707158598e-06
1365 7.66306499322411e-06
1366 7.67039412430393e-06
1367 7.66914528682605e-06
1368 7.66092794801665e-06
1369 7.65481665165879e-06
1370 7.6470675915985e-06
1371 7.62926116149032e-06
1372 7.61757865230805e-06
1373 7.61701107521162e-06
1374 7.60058390802953e-06
1375 7.602693106179e-06
1376 7.59554418117148e-06
1377 7.59164363461906e-06
1378 7.58278368362397e-06
1379 7.58205829498593e-06
1380 7.57711615584356e-06
1381 7.58039912351904e-06
1382 7.57822370367478e-06
1383 7.57918407234826e-06
1384 7.57935320727654e-06
1385 7.57307536147778e-06
1386 7.5703455628684e-06
1387 7.56816545699479e-06
1388 7.58099720776784e-06
1389 7.59281815465584e-06
1390 7.61051706721361e-06
1391 7.65558871851368e-06
1392 7.70138964512057e-06
1393 7.85642028588995e-06
1394 8.07690492443669e-06
1395 8.46979294877315e-06
1396 9.08890517337113e-06
1397 9.7480285603524e-06
1398 1.01522758839678e-05
1399 9.84637544654277e-06
1400 8.96138189609985e-06
1401 8.37873389158972e-06
1402 8.28942917152631e-06
1403 8.55759593676453e-06
1404 8.6714849594216e-06
1405 8.44139088584939e-06
1406 8.06749187098887e-06
1407 7.77105399052402e-06
1408 7.61738674093237e-06
1409 7.54537677671863e-06
1410 7.50721773279395e-06
1411 7.48012556162081e-06
1412 7.46295381226503e-06
1413 7.45826466186372e-06
1414 7.46285870079078e-06
1415 7.46912007443257e-06
1416 7.47168125680986e-06
1417 7.47759496899647e-06
1418 7.48029131436567e-06
1419 7.47281073998352e-06
1420 7.47449665272626e-06
1421 7.45979374805472e-06
1422 7.4449202678295e-06
1423 7.43193759600302e-06
1424 7.43098526267971e-06
1425 7.43592742047383e-06
1426 7.43626316790369e-06
1427 7.43549092607765e-06
1428 7.43083712784198e-06
1429 7.41644627311189e-06
1430 7.41202666798557e-06
1431 7.40231144824577e-06
1432 7.41347309318741e-06
1433 7.42393122976637e-06
1434 7.4440599897585e-06
1435 7.46144085184142e-06
1436 7.47169492765209e-06
1437 7.48917399207016e-06
1438 7.50969424334613e-06
1439 7.53744112191868e-06
1440 7.5830188119852e-06
1441 7.65075165887197e-06
1442 7.73561992950533e-06
1443 7.84689912158854e-06
1444 7.9625930808902e-06
1445 8.04913522678419e-06
1446 8.07050370532636e-06
1447 8.00645908860531e-06
1448 7.87660890999575e-06
1449 7.71152188594471e-06
1450 7.58206991946508e-06
1451 7.50979289154685e-06
1452 7.5343069854128e-06
1453 7.65536054458948e-06
1454 8.0156821216093e-06
1455 8.58083711108293e-06
1456 9.28758948504083e-06
1457 9.89835313802701e-06
1458 9.83786359753935e-06
1459 8.95564871861154e-06
1460 8.1323304383929e-06
1461 7.81133544336399e-06
1462 7.80804164968174e-06
1463 7.84018597776992e-06
1464 7.71620577744159e-06
1465 7.53051899238244e-06
1466 7.39526916770927e-06
1467 7.33724237900191e-06
1468 7.31840316881005e-06
1469 7.30707783791473e-06
1470 7.29076572980603e-06
1471 7.27884922913091e-06
1472 7.26188655875148e-06
1473 7.2525236864962e-06
1474 7.24976833677005e-06
1475 7.2434488558315e-06
1476 7.2511002624509e-06
1477 7.26025335939795e-06
1478 7.26722582289341e-06
1479 7.26309526566382e-06
1480 7.24808074359373e-06
1481 7.23694257231955e-06
1482 7.23143469638643e-06
1483 7.22688650256487e-06
1484 7.22873862812889e-06
1485 7.22839773814599e-06
1486 7.23712085903827e-06
1487 7.23850652928348e-06
1488 7.22845549816498e-06
1489 7.22001118891313e-06
1490 7.21142497894789e-06
1491 7.21100473199954e-06
1492 7.20872784576443e-06
1493 7.21192393537962e-06
1494 7.22492049298751e-06
1495 7.22864638547094e-06
1496 7.24119526740452e-06
1497 7.25490414676955e-06
1498 7.25884709762425e-06
1499 7.34596435059132e-06
1500 7.4537871244118e-06
1501 7.70245398662439e-06
1502 8.22546738099561e-06
1503 8.85544985784747e-06
1504 9.59687849189095e-06
1505 1.00557482376473e-05
1506 9.64348725318587e-06
1507 8.7015560445991e-06
1508 8.11996764937817e-06
1509 8.28336994018741e-06
1510 8.5873538928638e-06
1511 8.60486056453169e-06
1512 8.22628109631296e-06
1513 7.73188979152906e-06
1514 7.38613212458716e-06
1515 7.23652938638963e-06
1516 7.17566195440611e-06
1517 7.1430223451685e-06
1518 7.13162933418943e-06
1519 7.13086209280789e-06
1520 7.13337048185991e-06
1521 7.14513358079927e-06
1522 7.15674768514418e-06
1523 7.16665841160591e-06
1524 7.16657488109007e-06
1525 7.15101410442287e-06
1526 7.13127623885867e-06
1527 7.11175196510538e-06
1528 7.10566779282829e-06
1529 7.10356835842418e-06
1530 7.10894647859561e-06
1531 7.10827759053956e-06
1532 7.11025557009748e-06
1533 7.10717283425311e-06
1534 7.10424121219688e-06
1535 7.1017987588462e-06
1536 7.09744712246163e-06
1537 7.1044858644953e-06
1538 7.11453452861832e-06
1539 7.13065312041294e-06
1540 7.14445383920292e-06
1541 7.15642140747264e-06
1542 7.15851250543409e-06
1543 7.16351267104898e-06
1544 7.16430776570576e-06
1545 7.17202306166342e-06
1546 7.19106458024754e-06
1547 7.21713191786932e-06
1548 7.25172594684409e-06
1549 7.28954098150325e-06
1550 7.32309566942035e-06
1551 7.33473937586382e-06
1552 7.32838154604565e-06
1553 7.31703839917941e-06
1554 7.29342574512515e-06
1555 7.26567504116815e-06
1556 7.24178294220224e-06
1557 7.20655778252421e-06
1558 7.17091368773737e-06
1559 7.11806234399148e-06
1560 7.0906033515783e-06
1561 7.09256944109171e-06
1562 7.14896400832998e-06
1563 7.37155641328258e-06
1564 7.89300203685173e-06
1565 8.74734037736857e-06
1566 9.90582093507442e-06
1567 1.05204294555961e-05
1568 9.85678362042108e-06
1569 8.49936485103342e-06
1570 7.91676664668728e-06
1571 7.90893810886928e-06
1572 7.88835562204326e-06
1573 7.61845764074565e-06
1574 7.28693759821653e-06
1575 7.09684870692229e-06
1576 7.03109534505586e-06
1577 7.00665419905988e-06
1578 6.98851090419339e-06
1579 6.97042197383979e-06
1580 6.95843985365485e-06
1581 6.94678381840674e-06
1582 6.94592502004099e-06
1583 6.94664141587253e-06
1584 6.9535959221767e-06
1585 6.9522843917369e-06
1586 6.95004281148215e-06
1587 6.94598333250696e-06
1588 6.94024205660781e-06
1589 6.93835721587988e-06
1590 6.93506730176097e-06
1591 6.93311278165964e-06
1592 6.92681438785314e-06
1593 6.92428761350783e-06
1594 6.9215255908972e-06
1595 6.91834217292353e-06
1596 6.91611895309308e-06
1597 6.91793838569765e-06
1598 6.92104999622245e-06
1599 6.92379009592514e-06
1600 6.91596152080365e-06
1601 6.91237428895874e-06
1602 6.90642460909885e-06
1603 6.90428625294004e-06
1604 6.90681741932764e-06
1605 6.92189467255133e-06
1606 6.92207716657123e-06
1607 6.94569889336805e-06
1608 6.98617372396626e-06
1609 7.07225547280643e-06
1610 7.20316514435382e-06
1611 7.43059305463589e-06
1612 7.7744742377206e-06
1613 8.15544288457204e-06
1614 8.55647054720521e-06
1615 8.58798979308517e-06
1616 8.22485417728558e-06
1617 7.7053756371015e-06
1618 7.43888049470343e-06
1619 7.60393270482496e-06
1620 8.05048640195594e-06
1621 8.35253820952175e-06
1622 8.13881896100099e-06
1623 7.65299538141306e-06
1624 7.20199165371582e-06
1625 7.01375819378569e-06
1626 6.91865019852855e-06
1627 6.92428461590566e-06
1628 6.91556327758747e-06
1629 6.89123452524143e-06
1630 6.87732012849551e-06
1631 6.86262240279234e-06
1632 6.8613773338555e-06
1633 6.86751410405861e-06
1634 6.8754009534544e-06
1635 6.86836333496643e-06
1636 6.85980364423244e-06
1637 6.84399262773638e-06
1638 6.83244566257457e-06
1639 6.82373090921118e-06
1640 6.8251213081183e-06
1641 6.81621995557435e-06
1642 6.8168437197258e-06
1643 6.8197986662355e-06
1644 6.81675299762929e-06
1645 6.81133501068132e-06
1646 6.80641216455058e-06
1647 6.80637563643671e-06
1648 6.81070338082179e-06
1649 6.80926822305139e-06
1650 6.81888406273146e-06
1651 6.86793658388751e-06
1652 6.90216933296739e-06
1653 7.03152379788463e-06
1654 7.3313645758688e-06
1655 7.6498992278573e-06
1656 8.08801761387201e-06
1657 8.2981866498244e-06
1658 8.07902113209025e-06
1659 7.66333900070748e-06
1660 7.24438721011467e-06
1661 7.22791127572009e-06
1662 7.48553089557902e-06
1663 7.80227041374815e-06
1664 7.85637254452354e-06
1665 7.57540473728113e-06
1666 7.21739757381101e-06
1667 6.97610009847693e-06
1668 6.86915098402352e-06
1669 6.82882922564687e-06
1670 6.82744943958369e-06
1671 6.81910835975685e-06
1672 6.79273470360897e-06
1673 6.77382717917396e-06
1674 6.76539787214381e-06
1675 6.76285561329593e-06
1676 6.76351957551447e-06
1677 6.76776817165603e-06
1678 6.76770859264764e-06
1679 6.77293247708377e-06
1680 6.77440301544152e-06
1681 6.77667443582664e-06
1682 6.777804564706e-06
1683 6.77108273183791e-06
1684 6.75618793710697e-06
1685 6.74534263822579e-06
1686 6.73773225923213e-06
1687 6.73456108568615e-06
1688 6.72729324335819e-06
1689 6.72345809871189e-06
1690 6.73704183551393e-06
1691 6.7425305179114e-06
1692 6.74803833256021e-06
1693 6.77150016503703e-06
1694 6.77849885111215e-06
1695 6.78752933769289e-06
1696 6.88909211277178e-06
1697 6.98830420198249e-06
1698 7.24883548386401e-06
1699 7.732079304823e-06
1700 8.13032911572975e-06
1701 8.59761825111605e-06
1702 8.43493654478777e-06
1703 7.89128882061618e-06
1704 7.39867810750638e-06
1705 7.38038053427204e-06
1706 7.7346486815344e-06
1707 8.12467160216102e-06
1708 8.03108842717393e-06
1709 7.5957336420629e-06
1710 7.17678634742924e-06
1711 6.9041156551819e-06
1712 6.78124724196039e-06
1713 6.75053906462608e-06
1714 6.72736921725203e-06
1715 6.70556169524161e-06
1716 6.68957738092502e-06
1717 6.68328453645728e-06
1718 6.67751445071474e-06
1719 6.67978642621136e-06
1720 6.67851194258162e-06
1721 6.68178205476266e-06
1722 6.67976694668226e-06
1723 6.68015915117337e-06
1724 6.67989374303346e-06
1725 6.67533773945905e-06
1726 6.66347316258253e-06
1727 6.65565427482306e-06
1728 6.64522157212133e-06
1729 6.6428357321513e-06
1730 6.63686467028413e-06
1731 6.64434627761779e-06
1732 6.64392415394843e-06
1733 6.64322710797194e-06
1734 6.63981856874329e-06
1735 6.64284804496873e-06
1736 6.64841975961394e-06
1737 6.66165508533823e-06
1738 6.67912023999406e-06
1739 6.70362572563477e-06
1740 6.7391095681657e-06
1741 6.79913750545325e-06
1742 6.9129600275275e-06
1743 7.09577529800498e-06
1744 7.37860675670277e-06
1745 7.62952216071255e-06
1746 8.0053793745094e-06
1747 8.24881434979829e-06
1748 8.02701508106907e-06
1749 7.64954980958521e-06
1750 7.20420373845343e-06
1751 6.98895590289794e-06
1752 7.07584061032662e-06
1753 7.34667537471978e-06
1754 7.62904972795297e-06
1755 7.64238477479751e-06
1756 7.39952975692404e-06
1757 7.04208649615623e-06
1758 6.78482960658755e-06
1759 6.72962574821412e-06
1760 6.75898469548741e-06
1761 6.78693302802458e-06
1762 6.75730721155787e-06
1763 6.6870483594883e-06
1764 6.62946139229348e-06
1765 6.6169780854608e-06
1766 6.62606590751125e-06
1767 6.63976316950254e-06
1768 6.65229875540518e-06
1769 6.63678502110798e-06
1770 6.61643777633003e-06
1771 6.59723297324888e-06
1772 6.57587091179579e-06
1773 6.57109301283043e-06
1774 6.56577043045559e-06
1775 6.56860244951929e-06
1776 6.57441033347084e-06
1777 6.58315359025607e-06
1778 6.5963667594815e-06
1779 6.61996703410495e-06
1780 6.64363880886754e-06
1781 6.66265417059719e-06
1782 6.67676358023783e-06
1783 6.66563660534081e-06
1784 6.65068738481267e-06
1785 6.62483277391601e-06
1786 6.59838571248628e-06
1787 6.58128973896055e-06
1788 6.57032053030804e-06
1789 6.58081536730748e-06
1790 6.62437637188873e-06
1791 6.7986691538735e-06
1792 7.18324796800118e-06
1793 7.84325188085688e-06
1794 8.78164940942838e-06
1795 9.20371358148486e-06
1796 8.85199381350787e-06
1797 8.04174020885284e-06
1798 7.47812645940371e-06
1799 7.53444400025671e-06
1800 7.80652258924874e-06
1801 7.73727552427772e-06
1802 7.26160542541265e-06
1803 6.84074552115277e-06
1804 6.66066535792709e-06
1805 6.61676893720653e-06
1806 6.59003617187182e-06
1807 6.55822203654566e-06
1808 6.53151731455637e-06
1809 6.53133212669132e-06
1810 6.53442114106895e-06
1811 6.52997280692347e-06
1812 6.51896458858658e-06
1813 6.50897766885095e-06
1814 6.50307681215878e-06
1815 6.49894488446989e-06
1816 6.4914294926055e-06
1817 6.48779884304673e-06
1818 6.48688188142899e-06
1819 6.48880030862387e-06
1820 6.48827548044295e-06
1821 6.48584810125641e-06
1822 6.48690311422229e-06
1823 6.4875047041113e-06
1824 6.49051206380591e-06
1825 6.48755822840741e-06
1826 6.49287422227474e-06
1827 6.49318956646994e-06
1828 6.49141085951044e-06
1829 6.49073135683409e-06
1830 6.48816837767185e-06
1831 6.48554404758528e-06
1832 6.49048358969395e-06
1833 6.4943352073854e-06
1834 6.49178534661843e-06
1835 6.49325842427828e-06
1836 6.49448520206874e-06
1837 6.50275585467597e-06
1838 6.50835565618735e-06
1839 6.51667990503313e-06
1840 6.54096865471132e-06
1841 6.57765137646038e-06
1842 6.62268523665688e-06
1843 6.69495797378516e-06
1844 6.85221177576523e-06
1845 7.22807677977499e-06
1846 7.75808731390271e-06
1847 8.05198840581056e-06
1848 8.07976370609254e-06
1849 7.85262653124619e-06
1850 7.37827141250591e-06
1851 7.01194497487734e-06
1852 7.06975860964576e-06
1853 7.38071002004403e-06
1854 7.94643477242118e-06
1855 8.22881020212662e-06
1856 8.08055575696187e-06
1857 7.61087477929578e-06
1858 7.0308870361302e-06
1859 6.75535361249757e-06
1860 6.67826744304989e-06
1861 6.73477837409564e-06
1862 6.68325832720029e-06
1863 6.55946695360399e-06
1864 6.47792451680118e-06
1865 6.46063043863165e-06
1866 6.4761656712875e-06
1867 6.51870762613527e-06
1868 6.50243036925957e-06
1869 6.4759660851621e-06
1870 6.44083921752525e-06
1871 6.42250133964239e-06
1872 6.43641264286288e-06
1873 6.45106090857439e-06
1874 6.45100794116615e-06
1875 6.44400193294814e-06
1876 6.42274024187373e-06
1877 6.40708014998381e-06
1878 6.39843979755028e-06
1879 6.40080814129362e-06
1880 6.41254341449127e-06
1881 6.42644258519454e-06
1882 6.46825193761913e-06
1883 6.49497007199074e-06
1884 6.58596181768445e-06
1885 6.65315646930509e-06
1886 6.69044671219865e-06
1887 6.78818242061396e-06
1888 6.72226508502405e-06
1889 6.65282594347616e-06
1890 6.55347165778863e-06
1891 6.4756743443084e-06
1892 6.5566385885063e-06
1893 6.81487853793072e-06
1894 7.23129771174058e-06
1895 7.57156655950553e-06
1896 7.73273840692923e-06
1897 7.41128034142946e-06
1898 6.98027187873151e-06
1899 6.70918009504362e-06
1900 6.5785822336295e-06
1901 6.58839694978042e-06
1902 6.66316228858932e-06
1903 6.73706268017327e-06
1904 6.74573148273794e-06
1905 6.6678595933567e-06
1906 6.55299902607709e-06
1907 6.44979380837185e-06
1908 6.42365570513448e-06
1909 6.48162552430165e-06
1910 6.56652855823836e-06
1911 6.61627413300891e-06
1912 6.61424241776842e-06
1913 6.55242800906564e-06
1914 6.47153721988758e-06
1915 6.40392606499063e-06
1916 6.37501047417999e-06
1917 6.37396925107225e-06
1918 6.39734932583735e-06
1919 6.42973113329504e-06
1920 6.48946584469456e-06
1921 6.52059111860837e-06
1922 6.59360989452296e-06
1923 6.61426084747063e-06
1924 6.61951807323646e-06
1925 6.5884714199882e-06
1926 6.49284684595131e-06
1927 6.43668272903852e-06
1928 6.38702482191889e-06
1929 6.44081165646071e-06
1930 6.58360074456965e-06
1931 6.82455769851487e-06
1932 7.10616459187463e-06
1933 7.31550539967429e-06
1934 7.34602018148678e-06
1935 7.06344662582126e-06
1936 6.79051022878241e-06
1937 6.61062273099589e-06
1938 6.48631549449163e-06
1939 6.47592165403665e-06
1940 6.53761886848514e-06
1941 6.663766485282e-06
1942 6.74341817497037e-06
1943 6.79947252191226e-06
1944 6.7694831153986e-06
1945 6.61645338073669e-06
1946 6.51179579325145e-06
1947 6.39174952254962e-06
1948 6.39791325074412e-06
1949 6.49655693329976e-06
1950 6.64003169248417e-06
1951 6.74251340271326e-06
1952 6.75432756480632e-06
1953 6.67285361100056e-06
1954 6.53473773759572e-06
1955 6.41113225086087e-06
1956 6.3492195465642e-06
1957 6.35004118976212e-06
1958 6.38820307052868e-06
1959 6.44224614276823e-06
1960 6.50745857466717e-06
1961 6.51608857893393e-06
1962 6.51226122716508e-06
1963 6.48899859978513e-06
1964 6.39713384487095e-06
1965 6.34531271570538e-06
1966 6.30581836347233e-06
1967 6.29392260620421e-06
1968 6.36451962421347e-06
1969 6.4909923240819e-06
1970 6.68073826215476e-06
1971 6.8922668940985e-06
1972 7.0169466521719e-06
1973 7.00947345322334e-06
1974 6.77583168240403e-06
1975 6.55366967095006e-06
1976 6.43482136553075e-06
1977 6.35283722871804e-06
1978 6.34618427319822e-06
1979 6.39436295468698e-06
1980 6.53293535979316e-06
1981 6.64277682194125e-06
1982 6.76481421546526e-06
1983 6.85892121765619e-06
1984 6.81946846903259e-06
1985 6.82629102577437e-06
1986 6.62163741615984e-06
1987 6.43328558069811e-06
1988 6.38519439988272e-06
1989 6.49557621024144e-06
1990 6.7170796622662e-06
1991 6.93632330772687e-06
1992 6.98668641518907e-06
1993 6.8413637954734e-06
1994 6.60850227252041e-06
1995 6.41610309681795e-06
1996 6.33264980898218e-06
1997 6.33507018665824e-06
1998 6.35897677714325e-06
1999 6.35715685692873e-06
};
\addlegendentry{Train}
\addplot [semithick, black]
table {%
0 0.00723799876868725
1 0.00721188122406602
2 0.00718800164759159
3 0.00716627156361938
4 0.00714654196053743
5 0.00712833134457469
6 0.0071114469319582
7 0.00709562655538321
8 0.00708062434569001
9 0.0070662833750248
10 0.00705254869535565
11 0.00703912181779742
12 0.00702528981491923
13 0.00701043661683798
14 0.00699329422786832
15 0.0069730132818222
16 0.00695057213306427
17 0.00692490348592401
18 0.00689728511497378
19 0.00686890305951238
20 0.0068412059918046
21 0.00681212684139609
22 0.0067802001722157
23 0.00674643088132143
24 0.00670914351940155
25 0.00667045265436172
26 0.0066291457042098
27 0.00658539542928338
28 0.00654094526544213
29 0.00649585109204054
30 0.00644909497350454
31 0.00640085060149431
32 0.00635037664324045
33 0.0062995171174407
34 0.00624475069344044
35 0.00618685875087976
36 0.00612686993554235
37 0.00606412347406149
38 0.00599810853600502
39 0.00592865794897079
40 0.0058565204963088
41 0.00577928824350238
42 0.00569637026637793
43 0.0056075775064528
44 0.00550700351595879
45 0.00539304967969656
46 0.00526535231620073
47 0.00511924969032407
48 0.00495518045499921
49 0.0047605000436306
50 0.00452188262715936
51 0.0042284931987524
52 0.00389305409044027
53 0.00355038442648947
54 0.00321558932773769
55 0.00290988502092659
56 0.00263334391638637
57 0.00238766614347696
58 0.0021643191576004
59 0.00196721823886037
60 0.00179869611747563
61 0.00165095319971442
62 0.00151769467629492
63 0.00140350963920355
64 0.00130102119874209
65 0.0012099773157388
66 0.00112882908433676
67 0.00105662015266716
68 0.000993267400190234
69 0.000936591881327331
70 0.000886452849954367
71 0.000840698485262692
72 0.000799228146206588
73 0.000761300849262625
74 0.000727212696801871
75 0.000696425442583859
76 0.000667107349727303
77 0.00063982600113377
78 0.000615123950410634
79 0.000592353346291929
80 0.000570900738239288
81 0.000551183649804443
82 0.000532732636202127
83 0.000515686697326601
84 0.000499715097248554
85 0.000484654505271465
86 0.000470294180558994
87 0.000457483372883871
88 0.000444703240646049
89 0.000432901666499674
90 0.000422454497311264
91 0.000412104127462953
92 0.000402272882638499
93 0.000392804387956858
94 0.000384045153623447
95 0.000376350159058347
96 0.000368536566384137
97 0.000360722187906504
98 0.000354207877535373
99 0.000347690569469705
100 0.00034128557308577
101 0.000335099932271987
102 0.000329546397551894
103 0.000324632826959714
104 0.000319563609082252
105 0.000314175966195762
106 0.000309842871502042
107 0.000305373105220497
108 0.000300837797112763
109 0.000296466547297314
110 0.000292135606287047
111 0.00028843927429989
112 0.000284986250335351
113 0.000280961947282776
114 0.00027763910475187
115 0.000274801830528304
116 0.000271843018708751
117 0.00026915327180177
118 0.000266381626715884
119 0.000264096801402047
120 0.000261577311903238
121 0.000259385589743033
122 0.000257113366387784
123 0.000255221530096605
124 0.000253282167250291
125 0.000251394172664732
126 0.000249580130912364
127 0.000247702730121091
128 0.000246012263232842
129 0.000244240422034636
130 0.00024267117260024
131 0.000241098154219799
132 0.000239623521338217
133 0.000238320775679313
134 0.000237007989198901
135 0.000235666913795285
136 0.000234539402299561
137 0.000233333717915229
138 0.000232262013014406
139 0.000231166908633895
140 0.000230140532949008
141 0.000229147000936791
142 0.000228220174903981
143 0.000227303811698221
144 0.000226351345190778
145 0.000225621362915263
146 0.000224617091589607
147 0.000223933879169635
148 0.00022322429867927
149 0.000222380869672634
150 0.000221446505747736
151 0.00022086180979386
152 0.000219932815525681
153 0.00021931721130386
154 0.000218628614675254
155 0.000217992826947011
156 0.000217431297642179
157 0.000216754866414703
158 0.000216276428545825
159 0.000215567342820577
160 0.000215091931750067
161 0.000214615574805066
162 0.000214190862607211
163 0.000213627659832127
164 0.000213166989851743
165 0.000212671176996082
166 0.000212150349398144
167 0.000211732709431089
168 0.000211317630601116
169 0.000210814469028264
170 0.00021047200425528
171 0.000210008802241646
172 0.000209591817110777
173 0.000209193938644603
174 0.000208844328881241
175 0.000208532452234067
176 0.000208089390071109
177 0.000207700897590257
178 0.000207430828595534
179 0.000206940021598712
180 0.000206696699024178
181 0.000206316995900124
182 0.000205948919756338
183 0.000205649514100514
184 0.00020534690702334
185 0.000204896903596818
186 0.000204662181204185
187 0.00020440696971491
188 0.000204043390112929
189 0.00020359322661534
190 0.000203252115170471
191 0.000202901239390485
192 0.000202623079530895
193 0.000202337396331131
194 0.000201954913791269
195 0.000201570335775614
196 0.000201264323550276
197 0.000200797236175276
198 0.000200452865101397
199 0.000200116061023436
200 0.000199678732315078
201 0.000199266622075811
202 0.000198749563423917
203 0.000198285400983877
204 0.000197837463929318
205 0.000197408284293488
206 0.000196985594811849
207 0.000196467794012278
208 0.000196122899069451
209 0.000195612956304103
210 0.000195236672880128
211 0.000194863241631538
212 0.000194409483810887
213 0.000193975211004727
214 0.00019362146849744
215 0.000193145460798405
216 0.000192656414583325
217 0.000192310486454517
218 0.000191745843039826
219 0.00019131060980726
220 0.000190796446986496
221 0.000190250590094365
222 0.000189719023182988
223 0.000189260448678397
224 0.000188671081559733
225 0.000188230580533855
226 0.000187628655112348
227 0.00018718269711826
228 0.000186675926670432
229 0.00018606151570566
230 0.000185361743206158
231 0.000184689371963032
232 0.00018402018758934
233 0.000183540512807667
234 0.000183033247594722
235 0.000182569623575546
236 0.000181928247911856
237 0.000181310912012123
238 0.000180805756826885
239 0.000180262679350562
240 0.000179647016921081
241 0.000179052542080171
242 0.000178468835656531
243 0.000177924681338482
244 0.000177449095644988
245 0.000176743167685345
246 0.000176161789568141
247 0.000175554901943542
248 0.000174995846464299
249 0.000174378015799448
250 0.000173767897649668
251 0.000173137508681975
252 0.000172415035194717
253 0.000171947947819717
254 0.000171269843121991
255 0.000170811428688467
256 0.000169959384948015
257 0.000169407576322556
258 0.000168854036019184
259 0.00016806842177175
260 0.000167350779520348
261 0.000166571204317734
262 0.000165892284712754
263 0.000165404548170045
264 0.000164376295288093
265 0.000163701610290445
266 0.000162896889378317
267 0.000162933487445116
268 0.000161583040608093
269 0.000161355681484565
270 0.000160787691129372
271 0.000159978691954166
272 0.000159320057719015
273 0.000158560695126653
274 0.000157235364895314
275 0.000156750305905007
276 0.000156273308675736
277 0.000155252331751399
278 0.000154223671415821
279 0.000153319328092039
280 0.000152445092680864
281 0.000151575281051919
282 0.000150793246575631
283 0.000150053383549675
284 0.000149257612065412
285 0.000148472943692468
286 0.000147743179695681
287 0.000146786798723042
288 0.000145822661579587
289 0.000144225923577324
290 0.00014352542348206
291 0.00014284354983829
292 0.000141890515806153
293 0.000141066237119958
294 0.000140306641696952
295 0.000139488620334305
296 0.000138678224175237
297 0.000137873546918854
298 0.000137050868943334
299 0.00013621203834191
300 0.00013534416211769
301 0.000134566449560225
302 0.000133636203827336
303 0.000132821063743904
304 0.000131932974909432
305 0.000131137159769423
306 0.000130213404190727
307 0.000129394262330607
308 0.000128440646221861
309 0.000127622653963044
310 0.000126705970615149
311 0.000125835606013425
312 0.000124995960504748
313 0.000124066034913994
314 0.00012316950596869
315 0.000122323879622854
316 0.000121407647384331
317 0.000120562217489351
318 0.000119669770356268
319 0.000118832620501053
320 0.00011793044541264
321 0.000117059171316214
322 0.000116171657282393
323 0.000115287584776524
324 0.000114389607915655
325 0.000113526424684096
326 0.000112649926450104
327 0.000111747809569351
328 0.000110896726255305
329 0.000110064444015734
330 0.000109151740616653
331 0.000108282321889419
332 0.000107394800579641
333 0.00010653781646397
334 0.00010565084812697
335 0.000104792998172343
336 0.000103975173260551
337 0.000103122278233059
338 0.000102278012491297
339 0.000101476973213721
340 0.000100651937827934
341 9.98616160359234e-05
342 9.90592961898074e-05
343 9.82931596809067e-05
344 9.75043731159531e-05
345 9.67287851381116e-05
346 9.59750104811974e-05
347 9.52129485085607e-05
348 9.4424540293403e-05
349 9.37033182708547e-05
350 9.2925714852754e-05
351 9.22010804060847e-05
352 9.1409427113831e-05
353 9.07219873624854e-05
354 9.00192389963195e-05
355 8.9242857939098e-05
356 8.85183253558353e-05
357 8.77962447702885e-05
358 8.70361545821652e-05
359 8.63080858835019e-05
360 8.57134436955675e-05
361 8.48205454531126e-05
362 8.41110304463655e-05
363 8.35114988149144e-05
364 8.27055409899913e-05
365 8.20418063085526e-05
366 8.13475344330072e-05
367 8.06097596068867e-05
368 8.00682755652815e-05
369 7.92331629781984e-05
370 7.85841912147589e-05
371 7.80739937908947e-05
372 7.72853236412629e-05
373 7.66584125813097e-05
374 7.6029245974496e-05
375 7.5500865932554e-05
376 7.48070087865926e-05
377 7.4201641837135e-05
378 7.37260634195991e-05
379 7.29729581507854e-05
380 7.23871416994371e-05
381 7.1946567913983e-05
382 7.11937536834739e-05
383 7.0595953729935e-05
384 7.01315657352097e-05
385 6.94209593348205e-05
386 6.88651634845883e-05
387 6.8451656261459e-05
388 6.77460557199083e-05
389 6.71749003231525e-05
390 6.67882777634077e-05
391 6.61186422803439e-05
392 6.56023839837871e-05
393 6.52142989565618e-05
394 6.45214458927512e-05
395 6.39644495095126e-05
396 6.36075928923674e-05
397 6.2982042436488e-05
398 6.24507083557546e-05
399 6.18343474343419e-05
400 6.14616001257673e-05
401 6.07594884058926e-05
402 6.0256825236138e-05
403 5.98510669078678e-05
404 5.91812968195882e-05
405 5.87588401685935e-05
406 5.82333850616124e-05
407 5.79044935875572e-05
408 5.72877434024122e-05
409 5.68839823245071e-05
410 5.65472691960167e-05
411 5.59644795430358e-05
412 5.55637816432863e-05
413 5.52909841644578e-05
414 5.46995106560644e-05
415 5.43261994607747e-05
416 5.40389992238488e-05
417 5.34633873030543e-05
418 5.31036675965879e-05
419 5.28296222910285e-05
420 5.22886221006047e-05
421 5.19130808243062e-05
422 5.15598985657562e-05
423 5.11957150592934e-05
424 5.07443219248671e-05
425 5.03699229739141e-05
426 5.01160429848824e-05
427 4.95866952405777e-05
428 4.92459985252935e-05
429 4.88982950628269e-05
430 4.85978816868737e-05
431 4.81430142826866e-05
432 4.78483052575029e-05
433 4.75882116006687e-05
434 4.71067287435289e-05
435 4.68020334665198e-05
436 4.65826196887065e-05
437 4.61380877823103e-05
438 4.58075883216225e-05
439 4.5579974539578e-05
440 4.51893865829334e-05
441 4.4841075578006e-05
442 4.46501871920191e-05
443 4.42215350631159e-05
444 4.38880488218274e-05
445 4.36823502241168e-05
446 4.32826091127936e-05
447 4.29803003498819e-05
448 4.27812483394518e-05
449 4.23663223045878e-05
450 4.20771502831485e-05
451 4.1893872548826e-05
452 4.14842106692959e-05
453 4.12451481679454e-05
454 4.09935892093927e-05
455 4.06783510698006e-05
456 4.03564336011186e-05
457 4.02457480959129e-05
458 3.99413329432718e-05
459 3.95919960283209e-05
460 3.93501686630771e-05
461 3.9111891965149e-05
462 3.8964983104961e-05
463 3.85712555726059e-05
464 3.83127189707011e-05
465 3.82014914066531e-05
466 3.79236698790919e-05
467 3.7591926229652e-05
468 3.74336559616495e-05
469 3.7169833376538e-05
470 3.69948575098533e-05
471 3.67096472473349e-05
472 3.64778752555139e-05
473 3.62792416126467e-05
474 3.61447528121062e-05
475 3.58052966475952e-05
476 3.56119198841043e-05
477 3.54612020601053e-05
478 3.52801143890247e-05
479 3.49526999343652e-05
480 3.48108151229098e-05
481 3.46835477103014e-05
482 3.45020198437851e-05
483 3.41785344062373e-05
484 3.40501828759443e-05
485 3.3902677387232e-05
486 3.37381461577024e-05
487 3.34678297804203e-05
488 3.33398165821563e-05
489 3.32851413986646e-05
490 3.29837530443911e-05
491 3.2837713661138e-05
492 3.27097477565985e-05
493 3.26183253491763e-05
494 3.24071115755942e-05
495 3.22056403092574e-05
496 3.2071922760224e-05
497 3.19486971420702e-05
498 3.18125566991512e-05
499 3.15736288030166e-05
500 3.14734752464574e-05
501 3.13773161906283e-05
502 3.12319134536665e-05
503 3.10183859255631e-05
504 3.08723429043312e-05
505 3.07500959024765e-05
506 3.06606598314829e-05
507 3.05230132653378e-05
508 3.02999615087174e-05
509 3.01846266665962e-05
510 3.00727933790768e-05
511 2.99518869724125e-05
512 2.9758151868009e-05
513 2.96274174615974e-05
514 2.9583530704258e-05
515 2.94380333798472e-05
516 2.92325021291617e-05
517 2.91552387352567e-05
518 2.9058932341286e-05
519 2.88473620457808e-05
520 2.87370385194663e-05
521 2.86144550045719e-05
522 2.85280493699247e-05
523 2.84260677290149e-05
524 2.81979919236619e-05
525 2.81408247246873e-05
526 2.81084230664419e-05
527 2.80688582279254e-05
528 2.77806266240077e-05
529 2.76205064437818e-05
530 2.7559250156628e-05
531 2.75245420198189e-05
532 2.73684727289947e-05
533 2.71993449132424e-05
534 2.71147073362954e-05
535 2.7031501304009e-05
536 2.70756336249178e-05
537 2.68065869022394e-05
538 2.66917104454478e-05
539 2.66110928350827e-05
540 2.64924437942682e-05
541 2.64126792899333e-05
542 2.64425125351408e-05
543 2.62667144852458e-05
544 2.60867163888179e-05
545 2.596765079943e-05
546 2.59559328696923e-05
547 2.58881555055268e-05
548 2.58680938713951e-05
549 2.56374951277394e-05
550 2.55352260865038e-05
551 2.55148224823643e-05
552 2.54967380897142e-05
553 2.53372254519491e-05
554 2.52068930421956e-05
555 2.51155779551482e-05
556 2.51177534664748e-05
557 2.51051606028341e-05
558 2.4927523554652e-05
559 2.47752013819991e-05
560 2.4701314032427e-05
561 2.46901527134469e-05
562 2.46242416324094e-05
563 2.46378294832539e-05
564 2.44267994276015e-05
565 2.4297381969518e-05
566 2.43253143707989e-05
567 2.4224220396718e-05
568 2.41307261603652e-05
569 2.40055342146661e-05
570 2.40260960708838e-05
571 2.39425316976849e-05
572 2.3840919311624e-05
573 2.38150023506023e-05
574 2.3653370590182e-05
575 2.36316736845765e-05
576 2.36377636610996e-05
577 2.34537037613336e-05
578 2.33891969401157e-05
579 2.33334649237804e-05
580 2.33230020967312e-05
581 2.31527410505805e-05
582 2.31483991228743e-05
583 2.30602236115374e-05
584 2.29769011639291e-05
585 2.29104061872931e-05
586 2.28283588512568e-05
587 2.27492055273615e-05
588 2.282139575982e-05
589 2.27609525609296e-05
590 2.2606323909713e-05
591 2.25494295591488e-05
592 2.25497660721885e-05
593 2.2386357159121e-05
594 2.23589049710426e-05
595 2.23097013076767e-05
596 2.22747021325631e-05
597 2.22073067561723e-05
598 2.21371465158882e-05
599 2.20760030060774e-05
600 2.21356767724501e-05
601 2.20200108742574e-05
602 2.18815184780397e-05
603 2.1921587176621e-05
604 2.18856566789327e-05
605 2.17973083636025e-05
606 2.16610824281815e-05
607 2.16707103390945e-05
608 2.16040152736241e-05
609 2.16075004573213e-05
610 2.15030486288015e-05
611 2.14124611375155e-05
612 2.13650582736591e-05
613 2.14015381061472e-05
614 2.13316488952842e-05
615 2.12993072636891e-05
616 2.12074464798206e-05
617 2.11673104786314e-05
618 2.11785154533572e-05
619 2.10594716918422e-05
620 2.10044017876498e-05
621 2.09947320399806e-05
622 2.0986521121813e-05
623 2.08286492124898e-05
624 2.08011151698884e-05
625 2.07881657843245e-05
626 2.07679095183266e-05
627 2.07297944143647e-05
628 2.07004486583173e-05
629 2.05605829250999e-05
630 2.05093801923795e-05
631 2.05213673325488e-05
632 2.05671822186559e-05
633 2.05527667276328e-05
634 2.04170446522767e-05
635 2.03509898710763e-05
636 2.03074159799144e-05
637 2.02289975277381e-05
638 2.02125338546466e-05
639 2.02019964490319e-05
640 2.01533057406778e-05
641 2.00842932827072e-05
642 1.99649293790571e-05
643 1.99658061319496e-05
644 1.99859059648588e-05
645 1.99669611902209e-05
646 1.99738315131981e-05
647 1.98163143068086e-05
648 1.9840485038003e-05
649 1.98772868316155e-05
650 1.98669113160577e-05
651 1.97690515051363e-05
652 1.9768604033743e-05
653 1.97057979676174e-05
654 1.95563789020525e-05
655 1.94821168406634e-05
656 1.94385338545544e-05
657 1.93926207430195e-05
658 1.93321429833304e-05
659 1.927624361997e-05
660 1.92367533600191e-05
661 1.91813560377341e-05
662 1.90888458746485e-05
663 1.91126600839198e-05
664 1.9112174413749e-05
665 1.90956507140072e-05
666 1.91370036191074e-05
667 1.92197967407992e-05
668 1.9391451132833e-05
669 1.94619024114218e-05
670 1.95270895346766e-05
671 1.95780830836156e-05
672 1.93421310541453e-05
673 1.92344159586355e-05
674 1.90277278306894e-05
675 1.88770973181818e-05
676 1.89370475709438e-05
677 1.90652899618726e-05
678 1.90965347428573e-05
679 1.90356786333723e-05
680 1.88395260920515e-05
681 1.86116358236177e-05
682 1.84077380254166e-05
683 1.83217034646077e-05
684 1.83073607331607e-05
685 1.83653046406107e-05
686 1.84824257303262e-05
687 1.85889657586813e-05
688 1.86535744433058e-05
689 1.86846045835409e-05
690 1.87430032383418e-05
691 1.86669749382418e-05
692 1.85938952199649e-05
693 1.83854044735199e-05
694 1.82711155503057e-05
695 1.81740015250398e-05
696 1.81893065018812e-05
697 1.81712512130616e-05
698 1.81364612217294e-05
699 1.80968163476791e-05
700 1.80590523086721e-05
701 1.79584985744441e-05
702 1.79156504600542e-05
703 1.7827045667218e-05
704 1.77877391251968e-05
705 1.77477450051811e-05
706 1.78763402800541e-05
707 1.80110928340582e-05
708 1.81651321327081e-05
709 1.82074199983617e-05
710 1.85027329280274e-05
711 1.85507597052492e-05
712 1.84885175258387e-05
713 1.82732237590244e-05
714 1.80178521986818e-05
715 1.77807323780144e-05
716 1.77520651050145e-05
717 1.78672198671848e-05
718 1.79578692041105e-05
719 1.79912403837079e-05
720 1.78587652044371e-05
721 1.76521261892049e-05
722 1.74677461473038e-05
723 1.72851941897534e-05
724 1.71781503013335e-05
725 1.71812389453407e-05
726 1.71913434314774e-05
727 1.72638410731452e-05
728 1.7259431842831e-05
729 1.73997705132933e-05
730 1.73718217411079e-05
731 1.74283231899608e-05
732 1.7409771317034e-05
733 1.73758471646579e-05
734 1.72681866388302e-05
735 1.72341933648568e-05
736 1.71001975104446e-05
737 1.69747527252184e-05
738 1.69830254890257e-05
739 1.69840750459116e-05
740 1.69817321875598e-05
741 1.70557250385173e-05
742 1.70228140632389e-05
743 1.70150124176871e-05
744 1.69846316566691e-05
745 1.69471222761786e-05
746 1.68137776199728e-05
747 1.67009202414192e-05
748 1.66667632583994e-05
749 1.65544497576775e-05
750 1.65600849868497e-05
751 1.66166009876179e-05
752 1.68372116604587e-05
753 1.71802494151052e-05
754 1.75885306816781e-05
755 1.84168548003072e-05
756 1.9114711903967e-05
757 1.90905484487303e-05
758 1.81153591256589e-05
759 1.70602797879837e-05
760 1.70401799550746e-05
761 1.73491680470761e-05
762 1.74265642272076e-05
763 1.70515086210798e-05
764 1.66373338288395e-05
765 1.63085751410108e-05
766 1.6178259102162e-05
767 1.61292482516728e-05
768 1.60855142894434e-05
769 1.60952786245616e-05
770 1.61183434101986e-05
771 1.606960540812e-05
772 1.60867457452696e-05
773 1.60383151524002e-05
774 1.59896771947388e-05
775 1.59799110406311e-05
776 1.59602095664013e-05
777 1.59381688717986e-05
778 1.59220289788209e-05
779 1.59766022989061e-05
780 1.5875795725151e-05
781 1.58248931256821e-05
782 1.58139137056423e-05
783 1.584554593137e-05
784 1.59165865625255e-05
785 1.58624570758548e-05
786 1.59132177941501e-05
787 1.58630537043791e-05
788 1.59470582730137e-05
789 1.59144256031141e-05
790 1.59527717187302e-05
791 1.5918594726827e-05
792 1.59475293912692e-05
793 1.58120656124083e-05
794 1.58171551447595e-05
795 1.57310787471943e-05
796 1.57297326950356e-05
797 1.56632322614314e-05
798 1.56331752805272e-05
799 1.5570380128338e-05
800 1.5505422197748e-05
801 1.55263387568993e-05
802 1.56062342284713e-05
803 1.57168215082493e-05
804 1.58187540364452e-05
805 1.5912446542643e-05
806 1.60975487233372e-05
807 1.62019405252067e-05
808 1.63195218192413e-05
809 1.61906827997882e-05
810 1.6013171261875e-05
811 1.57120703079272e-05
812 1.53819964907598e-05
813 1.5121724572964e-05
814 1.50123760249699e-05
815 1.51414660649607e-05
816 1.53497967403382e-05
817 1.56646310642827e-05
818 1.62361739057815e-05
819 1.67743073689053e-05
820 1.73091539181769e-05
821 1.74207125382964e-05
822 1.65218862093752e-05
823 1.56052246893523e-05
824 1.53335367940599e-05
825 1.55767647811444e-05
826 1.56635196617572e-05
827 1.54477256728569e-05
828 1.50707182910992e-05
829 1.48596600411111e-05
830 1.47655782711809e-05
831 1.47455120895756e-05
832 1.4761912098038e-05
833 1.47920964082005e-05
834 1.47494129123515e-05
835 1.47562705024029e-05
836 1.47891632877872e-05
837 1.47809323607362e-05
838 1.47744476635125e-05
839 1.46981938087265e-05
840 1.47441323861131e-05
841 1.47333739732858e-05
842 1.47601240314543e-05
843 1.4715468751092e-05
844 1.46748179759015e-05
845 1.46767579281004e-05
846 1.47014434332959e-05
847 1.4724346328876e-05
848 1.466356934543e-05
849 1.46971233334625e-05
850 1.46669144669431e-05
851 1.46685988511308e-05
852 1.47537393786479e-05
853 1.46874963320442e-05
854 1.46934180520475e-05
855 1.4696539437864e-05
856 1.47410273712012e-05
857 1.47368755278876e-05
858 1.4711522453581e-05
859 1.46558904816629e-05
860 1.46842085086973e-05
861 1.46810170917888e-05
862 1.46835727719008e-05
863 1.45905787576339e-05
864 1.45242702274118e-05
865 1.45105541378143e-05
866 1.45488975249464e-05
867 1.45496360346442e-05
868 1.44706991704879e-05
869 1.45214389704051e-05
870 1.46476522786543e-05
871 1.47916371133761e-05
872 1.50715104609844e-05
873 1.55119359988021e-05
874 1.62256073963363e-05
875 1.68145052157342e-05
876 1.73503867699765e-05
877 1.67086309375009e-05
878 1.56006626639282e-05
879 1.44333880598424e-05
880 1.42753269756213e-05
881 1.45827807500609e-05
882 1.48285025716177e-05
883 1.49637426147819e-05
884 1.47515947901411e-05
885 1.45929843711201e-05
886 1.43976085382747e-05
887 1.43082497743308e-05
888 1.42914623211254e-05
889 1.43188708534581e-05
890 1.42568524097442e-05
891 1.42234976010513e-05
892 1.41962082125247e-05
893 1.41508917295141e-05
894 1.41792916110717e-05
895 1.41764503496233e-05
896 1.42011695061228e-05
897 1.41457312565763e-05
898 1.42143308039522e-05
899 1.42237167892745e-05
900 1.4250570529839e-05
901 1.42936496558832e-05
902 1.42904946187628e-05
903 1.42159515235107e-05
904 1.42774206324248e-05
905 1.42365142892231e-05
906 1.4217313946574e-05
907 1.41681830427842e-05
908 1.41291375257424e-05
909 1.41129221447045e-05
910 1.40680340336985e-05
911 1.40869797178311e-05
912 1.41113605423016e-05
913 1.40529491545749e-05
914 1.40615547934431e-05
915 1.40579550134134e-05
916 1.40308602567529e-05
917 1.40289776027203e-05
918 1.39978883453296e-05
919 1.40065976665937e-05
920 1.3980128642288e-05
921 1.39550356834661e-05
922 1.39922376547474e-05
923 1.39555731948349e-05
924 1.39636840685853e-05
925 1.40489755722228e-05
926 1.41496711876243e-05
927 1.43659317473066e-05
928 1.48011413330096e-05
929 1.53858636622317e-05
930 1.64430384756997e-05
931 1.70545754372142e-05
932 1.68262722581858e-05
933 1.54936678882223e-05
934 1.48582748806803e-05
935 1.5400153642986e-05
936 1.59274932229891e-05
937 1.5890400391072e-05
938 1.47893124449183e-05
939 1.40631609610864e-05
940 1.3725118151342e-05
941 1.37502283905633e-05
942 1.37933575388161e-05
943 1.37994775286643e-05
944 1.37686356538325e-05
945 1.37890501719085e-05
946 1.37277747853659e-05
947 1.37504875965533e-05
948 1.3705929632124e-05
949 1.36674234454404e-05
950 1.36959188239416e-05
951 1.3691214917344e-05
952 1.37236902446602e-05
953 1.36291728267679e-05
954 1.36494427351863e-05
955 1.36546223075129e-05
956 1.36478674903628e-05
957 1.3617766853713e-05
958 1.36353701236658e-05
959 1.35531317937421e-05
960 1.3591749848274e-05
961 1.3630288776767e-05
962 1.35933087221929e-05
963 1.36049984575948e-05
964 1.36268254209426e-05
965 1.36123398988275e-05
966 1.36972630571108e-05
967 1.37800989250536e-05
968 1.38488767333911e-05
969 1.41085665745777e-05
970 1.43263723657583e-05
971 1.45002422868856e-05
972 1.49895668073441e-05
973 1.52190768858418e-05
974 1.47833106893813e-05
975 1.43511815622333e-05
976 1.39728808790096e-05
977 1.38543591674534e-05
978 1.41223517857725e-05
979 1.46772681546281e-05
980 1.50409305206267e-05
981 1.52266975419479e-05
982 1.47880637086928e-05
983 1.42983044497669e-05
984 1.36485332404845e-05
985 1.34181664179778e-05
986 1.33727307911613e-05
987 1.35002846946009e-05
988 1.3554245015257e-05
989 1.35095515361172e-05
990 1.34469701151829e-05
991 1.3421532457869e-05
992 1.33988542074803e-05
993 1.3470928024617e-05
994 1.34267866087612e-05
995 1.34240044644685e-05
996 1.33638068291475e-05
997 1.33574867504649e-05
998 1.33792500491836e-05
999 1.3414070053841e-05
1000 1.33900994114811e-05
1001 1.33727726279176e-05
1002 1.33172925416147e-05
1003 1.33187140818336e-05
1004 1.32996328829904e-05
1005 1.33637277031085e-05
1006 1.33967278088676e-05
1007 1.33681860461365e-05
1008 1.34092570078792e-05
1009 1.34639794850955e-05
1010 1.35261952891597e-05
1011 1.36370927066309e-05
1012 1.37823581098928e-05
1013 1.39160256367177e-05
1014 1.39715411933139e-05
1015 1.41815689858049e-05
1016 1.43109646160156e-05
1017 1.43075731102726e-05
1018 1.4072027624934e-05
1019 1.38530485855881e-05
1020 1.35675527417334e-05
1021 1.35912623591139e-05
1022 1.40566107802442e-05
1023 1.44781261042226e-05
1024 1.49527832036256e-05
1025 1.5331243048422e-05
1026 1.51981394083123e-05
1027 1.43711204145802e-05
1028 1.34642832563259e-05
1029 1.31139677250758e-05
1030 1.31651677293121e-05
1031 1.32847035274608e-05
1032 1.33882713271305e-05
1033 1.33561552502215e-05
1034 1.33054836624069e-05
1035 1.32537124954979e-05
1036 1.31623446577578e-05
1037 1.31784663608414e-05
1038 1.31662336571026e-05
1039 1.31957076519029e-05
1040 1.32141940412112e-05
1041 1.31967244669795e-05
1042 1.3128834325471e-05
1043 1.31408096422092e-05
1044 1.31728838823619e-05
1045 1.31461765704444e-05
1046 1.31843635244877e-05
1047 1.31692067952827e-05
1048 1.31329352370813e-05
1049 1.31354299810482e-05
1050 1.30786747831735e-05
1051 1.30996004372719e-05
1052 1.31375327327987e-05
1053 1.30853486552951e-05
1054 1.31404394778656e-05
1055 1.31407514345483e-05
1056 1.31882216010126e-05
1057 1.32974073494552e-05
1058 1.36236021717195e-05
1059 1.38188415803597e-05
1060 1.41643286042381e-05
1061 1.47041464515496e-05
1062 1.54570752783911e-05
1063 1.58735474542482e-05
1064 1.53862674778793e-05
1065 1.43511615533498e-05
1066 1.37951665237779e-05
1067 1.43032621053862e-05
1068 1.49833886098349e-05
1069 1.52462362166261e-05
1070 1.43281067721546e-05
1071 1.35520667754463e-05
1072 1.29960726553691e-05
1073 1.29548470795271e-05
1074 1.30600747070275e-05
1075 1.3147833669791e-05
1076 1.31889573822264e-05
1077 1.31252581923036e-05
1078 1.3025450243731e-05
1079 1.29975433083018e-05
1080 1.30285689010634e-05
1081 1.30689395518857e-05
1082 1.31025935843354e-05
1083 1.305875593971e-05
1084 1.30326725411578e-05
1085 1.29746822494781e-05
1086 1.29353638840257e-05
1087 1.29824065879802e-05
1088 1.29989321067114e-05
1089 1.30258240460535e-05
1090 1.3024163308728e-05
1091 1.29406153064338e-05
1092 1.29779928101925e-05
1093 1.29570953504299e-05
1094 1.29720419863588e-05
1095 1.29725240185508e-05
1096 1.30328053273843e-05
1097 1.30236312543275e-05
1098 1.30676389744622e-05
1099 1.31495471578091e-05
1100 1.31757924464182e-05
1101 1.32461473185685e-05
1102 1.31839242385468e-05
1103 1.31617925944738e-05
1104 1.30936887217104e-05
1105 1.3059335287835e-05
1106 1.30554089992074e-05
1107 1.29393411043566e-05
1108 1.29317695609643e-05
1109 1.28602505355957e-05
1110 1.28425208458793e-05
1111 1.28865240185405e-05
1112 1.29580794236972e-05
1113 1.30817234094138e-05
1114 1.33933244796935e-05
1115 1.38864515975001e-05
1116 1.44929508678615e-05
1117 1.57154463522602e-05
1118 1.69388877111487e-05
1119 1.70967559824931e-05
1120 1.54715653479798e-05
1121 1.40069923872943e-05
1122 1.40753591040266e-05
1123 1.4610407561122e-05
1124 1.45545018312987e-05
1125 1.35648951982148e-05
1126 1.28269821288995e-05
1127 1.27165549201891e-05
1128 1.28244437291869e-05
1129 1.29391546579427e-05
1130 1.2967460861546e-05
1131 1.29010586533695e-05
1132 1.28122746900772e-05
1133 1.27769317259663e-05
1134 1.27749708553893e-05
1135 1.28188194139511e-05
1136 1.28252295326092e-05
1137 1.28278252304881e-05
1138 1.28226993183489e-05
1139 1.27694138427614e-05
1140 1.2734359188471e-05
1141 1.27907960632001e-05
1142 1.28425299408264e-05
1143 1.2810655789508e-05
1144 1.2799283467757e-05
1145 1.27660332509549e-05
1146 1.27572902783868e-05
1147 1.27927060020738e-05
1148 1.27950161186163e-05
1149 1.28416977531742e-05
1150 1.28800138554652e-05
1151 1.28667497847346e-05
1152 1.28571091408958e-05
1153 1.29535519590718e-05
1154 1.29402114907862e-05
1155 1.30448815980344e-05
1156 1.30124599309056e-05
1157 1.29350464703748e-05
1158 1.28342162497574e-05
1159 1.28167121147271e-05
1160 1.27312268887181e-05
1161 1.26858649309725e-05
1162 1.27245584735647e-05
1163 1.28000228869496e-05
1164 1.2828911167162e-05
1165 1.29415311675984e-05
1166 1.3201684851083e-05
1167 1.348485602648e-05
1168 1.38748609970207e-05
1169 1.47893542816746e-05
1170 1.57552658492932e-05
1171 1.64143693837104e-05
1172 1.57952308654785e-05
1173 1.43840698001441e-05
1174 1.35739073812147e-05
1175 1.40128222483327e-05
1176 1.44885625559255e-05
1177 1.42176368171931e-05
1178 1.32665045384783e-05
1179 1.27174789668061e-05
1180 1.26244403872988e-05
1181 1.27164930745494e-05
1182 1.2809878171538e-05
1183 1.28533301904099e-05
1184 1.27748662634986e-05
1185 1.27260591398226e-05
1186 1.2645171409531e-05
1187 1.26086824820959e-05
1188 1.26596123664058e-05
1189 1.2704393157037e-05
1190 1.2768600754498e-05
1191 1.27455468827975e-05
1192 1.26915365399327e-05
1193 1.26378681670758e-05
1194 1.26314207591349e-05
1195 1.26516742966487e-05
1196 1.27134917420335e-05
1197 1.27603079818073e-05
1198 1.27389785120613e-05
1199 1.26867680592113e-05
1200 1.26177310448838e-05
1201 1.260374483536e-05
1202 1.25983697216725e-05
1203 1.26886452562758e-05
1204 1.27297116705449e-05
1205 1.27266848721774e-05
1206 1.26367403936456e-05
1207 1.26376653497573e-05
1208 1.25845126603963e-05
1209 1.26458980957977e-05
1210 1.27230432553915e-05
1211 1.27177791000577e-05
1212 1.27809071273077e-05
1213 1.29125501189264e-05
1214 1.29493200802244e-05
1215 1.28988385768025e-05
1216 1.2860164133599e-05
1217 1.2840466297348e-05
1218 1.29084864965989e-05
1219 1.30514617922017e-05
1220 1.30446069306345e-05
1221 1.30689531943062e-05
1222 1.29643885884434e-05
1223 1.28372175822733e-05
1224 1.26652084873058e-05
1225 1.26497388919233e-05
1226 1.25915212265681e-05
1227 1.25515916806762e-05
1228 1.25209526231629e-05
1229 1.26625382108614e-05
1230 1.28016154121724e-05
1231 1.31318984131212e-05
1232 1.38561781568569e-05
1233 1.50405285239685e-05
1234 1.67009529832285e-05
1235 1.80817914952058e-05
1236 1.74195502040675e-05
1237 1.47944356285734e-05
1238 1.36384605866624e-05
1239 1.39935009428882e-05
1240 1.41529881148017e-05
1241 1.35401287479908e-05
1242 1.26222639664775e-05
1243 1.23792387967114e-05
1244 1.24952412079438e-05
1245 1.2632207472052e-05
1246 1.26401955640176e-05
1247 1.25719816423953e-05
1248 1.24999905892764e-05
1249 1.24650796351489e-05
1250 1.24697044157074e-05
1251 1.2465567124309e-05
1252 1.24998778119334e-05
1253 1.25609749375144e-05
1254 1.24854586829315e-05
1255 1.24693706311518e-05
1256 1.24352254715632e-05
1257 1.24244570542942e-05
1258 1.24587350001093e-05
1259 1.25159158415045e-05
1260 1.25587785078096e-05
1261 1.25082187878434e-05
1262 1.24603775475407e-05
1263 1.24220341604087e-05
1264 1.24071166283102e-05
1265 1.2422112376953e-05
1266 1.24679681903217e-05
1267 1.24740454339189e-05
1268 1.25249052871368e-05
1269 1.24906846394879e-05
1270 1.243450333277e-05
1271 1.24972693811287e-05
1272 1.24831221910426e-05
1273 1.26486684166593e-05
1274 1.26927852761582e-05
1275 1.29638374346541e-05
1276 1.30050239022239e-05
1277 1.30569815155468e-05
1278 1.29017380459118e-05
1279 1.29659292724682e-05
1280 1.29078716781805e-05
1281 1.28024321384146e-05
1282 1.262925525225e-05
1283 1.24822490761289e-05
1284 1.23920481200912e-05
1285 1.23872869153274e-05
1286 1.24675416373066e-05
1287 1.25670039778925e-05
1288 1.28105575640802e-05
1289 1.31630449686782e-05
1290 1.3699375813303e-05
1291 1.45171652548015e-05
1292 1.53959590534214e-05
1293 1.55785728566116e-05
1294 1.4998365259089e-05
1295 1.36285943881376e-05
1296 1.30316511786077e-05
1297 1.33859657580615e-05
1298 1.36531198222656e-05
1299 1.36376820591977e-05
1300 1.29633635879145e-05
1301 1.24786101878271e-05
1302 1.23090294437134e-05
1303 1.24122470879229e-05
1304 1.25228252727538e-05
1305 1.25505193864228e-05
1306 1.24923080875305e-05
1307 1.24604794109473e-05
1308 1.23944491861039e-05
1309 1.23960671771783e-05
1310 1.23952195281163e-05
1311 1.2427638466761e-05
1312 1.24285252240952e-05
1313 1.23919990073773e-05
1314 1.24181860883255e-05
1315 1.23886429719278e-05
1316 1.24115913422429e-05
1317 1.23904692372889e-05
1318 1.2466939551814e-05
1319 1.24038542708149e-05
1320 1.23750505736098e-05
1321 1.23349918794702e-05
1322 1.23555382742779e-05
1323 1.23921836348018e-05
1324 1.24553316709353e-05
1325 1.24709658848587e-05
1326 1.24516309369938e-05
1327 1.23592781164916e-05
1328 1.23829795484198e-05
1329 1.23044756037416e-05
1330 1.23263698696974e-05
1331 1.23448699014261e-05
1332 1.24505959320231e-05
1333 1.24722091641161e-05
1334 1.24380794659373e-05
1335 1.23931549751433e-05
1336 1.23850659292657e-05
1337 1.24413663797895e-05
1338 1.25429087347584e-05
1339 1.27162038552342e-05
1340 1.30803682623082e-05
1341 1.3486377611116e-05
1342 1.42800536195864e-05
1343 1.52648408402456e-05
1344 1.58695565914968e-05
1345 1.69414270203561e-05
1346 1.55181878653821e-05
1347 1.41559630719712e-05
1348 1.25428850878961e-05
1349 1.23879253806081e-05
1350 1.31185688587721e-05
1351 1.38512250487111e-05
1352 1.40663396450691e-05
1353 1.43007455335464e-05
1354 1.36577546072658e-05
1355 1.30985645228066e-05
1356 1.26349859783659e-05
1357 1.24299658637028e-05
1358 1.25084616229287e-05
1359 1.23763129522558e-05
1360 1.2302232789807e-05
1361 1.22004648801521e-05
1362 1.22363344416954e-05
1363 1.22984711197205e-05
1364 1.23769505080418e-05
1365 1.24543548736256e-05
1366 1.2395220437611e-05
1367 1.23850104500889e-05
1368 1.23324743981357e-05
1369 1.23386234918144e-05
1370 1.23325489767012e-05
1371 1.22930459838244e-05
1372 1.23359741337481e-05
1373 1.22924111565226e-05
1374 1.23190411613905e-05
1375 1.23361332953209e-05
1376 1.23080590128666e-05
1377 1.22486144391587e-05
1378 1.22593583000707e-05
1379 1.22257961265859e-05
1380 1.2309780686337e-05
1381 1.22587043733802e-05
1382 1.23588215501513e-05
1383 1.23658091979451e-05
1384 1.23166128105368e-05
1385 1.2275410881557e-05
1386 1.22773390103248e-05
1387 1.22314077088959e-05
1388 1.22835008369293e-05
1389 1.23555346362991e-05
1390 1.24095422506798e-05
1391 1.25342139654094e-05
1392 1.2644833077502e-05
1393 1.31234519358259e-05
1394 1.34651772896177e-05
1395 1.44782270581345e-05
1396 1.500714552094e-05
1397 1.58698439918226e-05
1398 1.5176707165665e-05
1399 1.41823447847855e-05
1400 1.35903046611929e-05
1401 1.34222937049344e-05
1402 1.38338782562641e-05
1403 1.38675914058695e-05
1404 1.32176519400673e-05
1405 1.25216529340832e-05
1406 1.22142728287145e-05
1407 1.22731426017708e-05
1408 1.23554118545144e-05
1409 1.23852287288173e-05
1410 1.23198606161168e-05
1411 1.22782148537226e-05
1412 1.22226874736953e-05
1413 1.21753764688037e-05
1414 1.2192213034723e-05
1415 1.21942239275086e-05
1416 1.2232663721079e-05
1417 1.2228830200911e-05
1418 1.22393530546105e-05
1419 1.22143237604178e-05
1420 1.21998946269741e-05
1421 1.21100838441635e-05
1422 1.21078455777024e-05
1423 1.21270131785423e-05
1424 1.22000756164198e-05
1425 1.22143173939548e-05
1426 1.22619330795715e-05
1427 1.21659604701563e-05
1428 1.21423654491082e-05
1429 1.20536178656039e-05
1430 1.20731783681549e-05
1431 1.20986696856562e-05
1432 1.22163155538146e-05
1433 1.2251523003215e-05
1434 1.23278368846513e-05
1435 1.22753990581259e-05
1436 1.22275205285405e-05
1437 1.22233313959441e-05
1438 1.22833334899042e-05
1439 1.2336145118752e-05
1440 1.25753731481382e-05
1441 1.27294324556715e-05
1442 1.29179634313914e-05
1443 1.31655870063696e-05
1444 1.32802424559486e-05
1445 1.31171345856274e-05
1446 1.2877279004897e-05
1447 1.25142969409353e-05
1448 1.21982366181328e-05
1449 1.20198701551999e-05
1450 1.20152217277791e-05
1451 1.21348584798397e-05
1452 1.24087846415932e-05
1453 1.29308073155698e-05
1454 1.39497215059237e-05
1455 1.47006285260431e-05
1456 1.55405487021198e-05
1457 1.51898502736003e-05
1458 1.41484015330207e-05
1459 1.30590078697423e-05
1460 1.26831318993936e-05
1461 1.27303446788574e-05
1462 1.27479597722413e-05
1463 1.23617292047129e-05
1464 1.20225449791178e-05
1465 1.18989719339879e-05
1466 1.20171607704833e-05
1467 1.21404127639835e-05
1468 1.21852581287385e-05
1469 1.21558605314931e-05
1470 1.20536515169078e-05
1471 1.19779679152998e-05
1472 1.19598762466921e-05
1473 1.19404094220954e-05
1474 1.19596288641333e-05
1475 1.20090226118919e-05
1476 1.20614140541875e-05
1477 1.21143129945267e-05
1478 1.20592758321436e-05
1479 1.19435680971947e-05
1480 1.19061423902167e-05
1481 1.18659145300626e-05
1482 1.18907146315905e-05
1483 1.19300775622833e-05
1484 1.19747192002251e-05
1485 1.20360919027007e-05
1486 1.21160428534495e-05
1487 1.20319518828182e-05
1488 1.20122895168606e-05
1489 1.19172837003134e-05
1490 1.18833077067393e-05
1491 1.18615926112398e-05
1492 1.18771249617566e-05
1493 1.19055184768513e-05
1494 1.20191070891451e-05
1495 1.207595323649e-05
1496 1.21344055514783e-05
1497 1.21356379167992e-05
1498 1.21431839943398e-05
1499 1.23550698845065e-05
1500 1.2543659977382e-05
1501 1.33127896333463e-05
1502 1.43192719406215e-05
1503 1.53131422848674e-05
1504 1.57627928274451e-05
1505 1.5520707165706e-05
1506 1.42333801704808e-05
1507 1.33886424009688e-05
1508 1.35497357405256e-05
1509 1.40102829391253e-05
1510 1.38868599606212e-05
1511 1.29198515423923e-05
1512 1.22269275379949e-05
1513 1.19472751975991e-05
1514 1.20824570331024e-05
1515 1.21114617286366e-05
1516 1.20995327961282e-05
1517 1.20346339826938e-05
1518 1.19555515993852e-05
1519 1.19038586490205e-05
1520 1.19347605505027e-05
1521 1.19587120934739e-05
1522 1.20381573651684e-05
1523 1.20046934171114e-05
1524 1.19529695439269e-05
1525 1.18370562631753e-05
1526 1.1825535693788e-05
1527 1.18395446406794e-05
1528 1.18732286864542e-05
1529 1.19522437671549e-05
1530 1.19558762889937e-05
1531 1.19890137284528e-05
1532 1.19153719424503e-05
1533 1.18901461974019e-05
1534 1.18144371299422e-05
1535 1.18072684927029e-05
1536 1.18659099825891e-05
1537 1.19397573143942e-05
1538 1.20353706734022e-05
1539 1.20278791655437e-05
1540 1.2028006494802e-05
1541 1.19450551210321e-05
1542 1.19139594971784e-05
1543 1.18825910249143e-05
1544 1.19144433483598e-05
1545 1.19770438686828e-05
1546 1.21271168609383e-05
1547 1.21681505333981e-05
1548 1.23004838314955e-05
1549 1.23040945254616e-05
1550 1.22790261229966e-05
1551 1.21660659715417e-05
1552 1.20904351206264e-05
1553 1.20694958241074e-05
1554 1.20475724543212e-05
1555 1.20419335871702e-05
1556 1.20441436592955e-05
1557 1.19863270811038e-05
1558 1.18912839752738e-05
1559 1.18135312732193e-05
1560 1.18135330922087e-05
1561 1.1893836017407e-05
1562 1.21329439934925e-05
1563 1.29701611513156e-05
1564 1.43044962896965e-05
1565 1.6043146388256e-05
1566 1.65854235092411e-05
1567 1.56067508214619e-05
1568 1.40775273393956e-05
1569 1.32844324980397e-05
1570 1.31446313389461e-05
1571 1.29276568259229e-05
1572 1.23238114611013e-05
1573 1.17496547318297e-05
1574 1.1657236427709e-05
1575 1.18309708341258e-05
1576 1.19285250548273e-05
1577 1.19231735880021e-05
1578 1.18451962407562e-05
1579 1.17749150376767e-05
1580 1.17380086521734e-05
1581 1.17165564006427e-05
1582 1.17149475045153e-05
1583 1.1769827324315e-05
1584 1.17479085020022e-05
1585 1.17536228572135e-05
1586 1.17010940812179e-05
1587 1.1701313269441e-05
1588 1.17034378490644e-05
1589 1.17279732876341e-05
1590 1.17602376121795e-05
1591 1.17354584290297e-05
1592 1.17168219730956e-05
1593 1.17005001811776e-05
1594 1.17106701509329e-05
1595 1.16772680485155e-05
1596 1.16997998702573e-05
1597 1.17395038614632e-05
1598 1.18159368867055e-05
1599 1.17623794722022e-05
1600 1.17295903692138e-05
1601 1.16738019642071e-05
1602 1.16728515422437e-05
1603 1.1707080375345e-05
1604 1.17425188363995e-05
1605 1.17542867883458e-05
1606 1.17712197607034e-05
1607 1.18771149573149e-05
1608 1.20491922643851e-05
1609 1.2342604350124e-05
1610 1.2797365343431e-05
1611 1.35606551339151e-05
1612 1.42375693030772e-05
1613 1.48075441757101e-05
1614 1.4318557077786e-05
1615 1.31645720102824e-05
1616 1.19232590805041e-05
1617 1.13786718429765e-05
1618 1.19850010378286e-05
1619 1.31088791022194e-05
1620 1.37909528348246e-05
1621 1.36208709591301e-05
1622 1.28529854919179e-05
1623 1.22872779684258e-05
1624 1.19629839900881e-05
1625 1.17207737275749e-05
1626 1.17924973892514e-05
1627 1.16811224870617e-05
1628 1.17180125016603e-05
1629 1.16834644359187e-05
1630 1.17638601295766e-05
1631 1.17099671115284e-05
1632 1.17299487101263e-05
1633 1.17158697321429e-05
1634 1.17685012810398e-05
1635 1.17327535917866e-05
1636 1.17472618512693e-05
1637 1.17103763841442e-05
1638 1.17111994768493e-05
1639 1.1733792234736e-05
1640 1.16697974590352e-05
1641 1.17102772492217e-05
1642 1.16898772830609e-05
1643 1.16582714326796e-05
1644 1.16125283966539e-05
1645 1.15997863758821e-05
1646 1.16369046736509e-05
1647 1.16371802505455e-05
1648 1.16681485451409e-05
1649 1.17012068585609e-05
1650 1.1755768355215e-05
1651 1.17227637019823e-05
1652 1.18041789392009e-05
1653 1.20836102723842e-05
1654 1.27150251501007e-05
1655 1.32743025460513e-05
1656 1.36203825604753e-05
1657 1.34602196339983e-05
1658 1.29760201161844e-05
1659 1.25875349112903e-05
1660 1.26771510622348e-05
1661 1.28805704662227e-05
1662 1.32138220578781e-05
1663 1.2869420061179e-05
1664 1.25826627481729e-05
1665 1.17936506285332e-05
1666 1.15503553388407e-05
1667 1.15720749818138e-05
1668 1.17643667181255e-05
1669 1.1890216228494e-05
1670 1.19685737445252e-05
1671 1.19222950161202e-05
1672 1.1872540198965e-05
1673 1.17754616439925e-05
1674 1.17022627819097e-05
1675 1.16152268674341e-05
1676 1.15964376163902e-05
1677 1.15717884909827e-05
1678 1.16086639536661e-05
1679 1.16783039629809e-05
1680 1.16990795504535e-05
1681 1.17262461571954e-05
1682 1.16727997010457e-05
1683 1.16021974463365e-05
1684 1.15640514195547e-05
1685 1.15927159640705e-05
1686 1.15523198473966e-05
1687 1.15605353130377e-05
1688 1.15481170723797e-05
1689 1.15659040602623e-05
1690 1.1615596122283e-05
1691 1.16454730232363e-05
1692 1.17053778012632e-05
1693 1.17999370559119e-05
1694 1.18120433398872e-05
1695 1.18222415039781e-05
1696 1.19680180432624e-05
1697 1.21111288535758e-05
1698 1.27591483760625e-05
1699 1.35091568154166e-05
1700 1.3867251254851e-05
1701 1.38915538627771e-05
1702 1.35419250000268e-05
1703 1.3024404324824e-05
1704 1.30517255456652e-05
1705 1.32881668832852e-05
1706 1.35890559249674e-05
1707 1.34830861497903e-05
1708 1.23980880744057e-05
1709 1.16965366032673e-05
1710 1.14300864879624e-05
1711 1.16259798232932e-05
1712 1.1840000297525e-05
1713 1.19373898996855e-05
1714 1.1900304343726e-05
1715 1.18166681204457e-05
1716 1.17054123620619e-05
1717 1.15857956188847e-05
1718 1.15271996037336e-05
1719 1.15035563794663e-05
1720 1.15335087684798e-05
1721 1.15980355985812e-05
1722 1.15883613034384e-05
1723 1.16424325824482e-05
1724 1.1613357855822e-05
1725 1.15793191071134e-05
1726 1.15118591565988e-05
1727 1.15126158561907e-05
1728 1.15017310235999e-05
1729 1.14892582132597e-05
1730 1.15242892206879e-05
1731 1.15879829536425e-05
1732 1.16028368211119e-05
1733 1.15628145067603e-05
1734 1.1546295354492e-05
1735 1.1590820577112e-05
1736 1.15626853585127e-05
1737 1.16134087875253e-05
1738 1.16368801172939e-05
1739 1.16669843919226e-05
1740 1.17786084956606e-05
1741 1.20195454655914e-05
1742 1.24430098367156e-05
1743 1.30235102915321e-05
1744 1.36376438604202e-05
1745 1.38572213472798e-05
1746 1.4575674867956e-05
1747 1.34580814119545e-05
1748 1.24876805784879e-05
1749 1.14885606308235e-05
1750 1.13841279016924e-05
1751 1.18753396236571e-05
1752 1.24835387396161e-05
1753 1.29940972328768e-05
1754 1.31045981106581e-05
1755 1.27230614452856e-05
1756 1.22773672046605e-05
1757 1.18913330879877e-05
1758 1.18258994916687e-05
1759 1.17763520393055e-05
1760 1.17110321298242e-05
1761 1.15433704195311e-05
1762 1.14127115011797e-05
1763 1.13919995783363e-05
1764 1.14840158857987e-05
1765 1.15669399747276e-05
1766 1.16497203634935e-05
1767 1.16922165034339e-05
1768 1.17307499749586e-05
1769 1.16808996608597e-05
1770 1.15946259029442e-05
1771 1.15303009806667e-05
1772 1.14787244456238e-05
1773 1.14331060103723e-05
1774 1.14252006824245e-05
1775 1.14468703031889e-05
1776 1.14640415631584e-05
1777 1.15473822006606e-05
1778 1.16459295895766e-05
1779 1.16859209811082e-05
1780 1.16794153655064e-05
1781 1.16350602183957e-05
1782 1.15817774712923e-05
1783 1.14900485641556e-05
1784 1.14216427391511e-05
1785 1.135900583904e-05
1786 1.12896059363266e-05
1787 1.12680208985694e-05
1788 1.12959105535992e-05
1789 1.13607757157297e-05
1790 1.1512996024976e-05
1791 1.19956193884718e-05
1792 1.29722593555925e-05
1793 1.43021516123554e-05
1794 1.51329677464673e-05
1795 1.45168596645817e-05
1796 1.37972056108993e-05
1797 1.32801515064784e-05
1798 1.32646973725059e-05
1799 1.33029516291572e-05
1800 1.28673464132589e-05
1801 1.17356421469594e-05
1802 1.12110392365139e-05
1803 1.13344694909756e-05
1804 1.16337114377529e-05
1805 1.17803647299297e-05
1806 1.17645895443275e-05
1807 1.16245191748021e-05
1808 1.15088914753869e-05
1809 1.14069298433606e-05
1810 1.13104042611667e-05
1811 1.12934694698197e-05
1812 1.12907700895448e-05
1813 1.13619298645062e-05
1814 1.13971846076311e-05
1815 1.13903715828201e-05
1816 1.13989935925929e-05
1817 1.14014674181817e-05
1818 1.13912492452073e-05
1819 1.13858641270781e-05
1820 1.13349979073973e-05
1821 1.13271726149833e-05
1822 1.13725291157607e-05
1823 1.14184749691049e-05
1824 1.13727955977083e-05
1825 1.13621854325174e-05
1826 1.13405176307424e-05
1827 1.13594996946631e-05
1828 1.13557907752693e-05
1829 1.13163996502408e-05
1830 1.13187224997091e-05
1831 1.13445639726706e-05
1832 1.13664673335734e-05
1833 1.13537616925896e-05
1834 1.13280766527168e-05
1835 1.13617788883857e-05
1836 1.14312051664456e-05
1837 1.13976775537594e-05
1838 1.13855621748371e-05
1839 1.14520289571374e-05
1840 1.15406337499735e-05
1841 1.15895609269501e-05
1842 1.17429226520471e-05
1843 1.1919846656383e-05
1844 1.27442490338581e-05
1845 1.4167839253787e-05
1846 1.4386625480256e-05
1847 1.38845562105416e-05
1848 1.30440230350359e-05
1849 1.18615398605471e-05
1850 1.11467079477734e-05
1851 1.15657639980782e-05
1852 1.24070847959956e-05
1853 1.35261680043186e-05
1854 1.38284831336932e-05
1855 1.31468186737038e-05
1856 1.23817444546148e-05
1857 1.19102369353641e-05
1858 1.18147800094448e-05
1859 1.17471217890852e-05
1860 1.17948484330554e-05
1861 1.14787162601715e-05
1862 1.13330534077249e-05
1863 1.12370425995323e-05
1864 1.12899433588609e-05
1865 1.1327579159115e-05
1866 1.13540809252299e-05
1867 1.13428477561683e-05
1868 1.13619953481248e-05
1869 1.1359652489773e-05
1870 1.14070280687883e-05
1871 1.14199410745641e-05
1872 1.14211170512135e-05
1873 1.1344074664521e-05
1874 1.12355346573167e-05
1875 1.11921490315581e-05
1876 1.11609770101495e-05
1877 1.11688295874046e-05
1878 1.11698846012587e-05
1879 1.12137167889159e-05
1880 1.12853313112282e-05
1881 1.13170681288466e-05
1882 1.13125497591682e-05
1883 1.13636315290933e-05
1884 1.15763623398379e-05
1885 1.16484143291018e-05
1886 1.167902337329e-05
1887 1.17972449515946e-05
1888 1.17379804578377e-05
1889 1.16670944407815e-05
1890 1.15860384539701e-05
1891 1.1742115020752e-05
1892 1.20747308756108e-05
1893 1.27686198538868e-05
1894 1.31488941406133e-05
1895 1.29569780256134e-05
1896 1.25529631986865e-05
1897 1.16079118015477e-05
1898 1.112896916311e-05
1899 1.11321605800185e-05
1900 1.14383801701479e-05
1901 1.17374902401934e-05
1902 1.18906746138236e-05
1903 1.19693413580535e-05
1904 1.18768275569892e-05
1905 1.17514236990246e-05
1906 1.15438306238502e-05
1907 1.14797676360467e-05
1908 1.14861632027896e-05
1909 1.15698121589958e-05
1910 1.14978256533504e-05
1911 1.14073400254711e-05
1912 1.13226697067148e-05
1913 1.12047400762094e-05
1914 1.11414665298071e-05
1915 1.11450244730804e-05
1916 1.11860344986781e-05
1917 1.12632924356149e-05
1918 1.13444175440236e-05
1919 1.14722979560611e-05
1920 1.14806935016531e-05
1921 1.15332541099633e-05
1922 1.1670112144202e-05
1923 1.15985267257201e-05
1924 1.16162482299842e-05
1925 1.15192160592414e-05
1926 1.15129087134846e-05
1927 1.14544054667931e-05
1928 1.15896746137878e-05
1929 1.17778045023442e-05
1930 1.20022968985722e-05
1931 1.24052421597298e-05
1932 1.25377082440536e-05
1933 1.23509807963273e-05
1934 1.2098432307539e-05
1935 1.13596170194796e-05
1936 1.10201972347568e-05
1937 1.09938555397093e-05
1938 1.12244015326723e-05
1939 1.15193688543513e-05
1940 1.173928467324e-05
1941 1.19654723675922e-05
1942 1.20443255582359e-05
1943 1.19514488687855e-05
1944 1.17939152914914e-05
1945 1.15986777018406e-05
1946 1.14500999188749e-05
1947 1.14630975076579e-05
1948 1.15920774987899e-05
1949 1.17144682008075e-05
1950 1.17430081445491e-05
1951 1.15881684905617e-05
1952 1.14092745207017e-05
1953 1.12177967821481e-05
1954 1.10521414171671e-05
1955 1.10407827378367e-05
1956 1.11251092675957e-05
1957 1.12740308395587e-05
1958 1.14147733256686e-05
1959 1.1586657819862e-05
1960 1.16686260298593e-05
1961 1.17047220555833e-05
1962 1.16869568955735e-05
1963 1.15051398097421e-05
1964 1.14544109237613e-05
1965 1.13817732199095e-05
1966 1.13545629574219e-05
1967 1.1490446922835e-05
1968 1.16806504593114e-05
1969 1.18452453534701e-05
1970 1.20900867841556e-05
1971 1.21587236208143e-05
1972 1.19583819468971e-05
1973 1.16444452942233e-05
1974 1.11830377136357e-05
1975 1.09528809844051e-05
1976 1.09296597656794e-05
1977 1.10615192170371e-05
1978 1.12815569082159e-05
1979 1.15280017780606e-05
1980 1.18980360639398e-05
1981 1.20477589007351e-05
1982 1.21913217299152e-05
1983 1.21378134281258e-05
1984 1.19993092084769e-05
1985 1.18297039080062e-05
1986 1.16464825623552e-05
1987 1.16060409709462e-05
1988 1.17546833280358e-05
1989 1.20179920486407e-05
1990 1.21402390504954e-05
1991 1.2080573469575e-05
1992 1.15939601528225e-05
1993 1.11814551928546e-05
1994 1.09442835309892e-05
1995 1.09943648567423e-05
1996 1.11832450784277e-05
1997 1.13782834887388e-05
1998 1.1524764886417e-05
1999 1.16103992695571e-05
};
\addlegendentry{Test}

\nextgroupplot[
title={Leaky/Tanh},,
ymin=3.12673053821883e-06, ymax=0.001,
]
\addplot [semithick, black, dashed]
table {%
0 0.0100023580089328
1 0.00974365008005407
2 0.00950256517535308
3 0.00927750682240003
4 0.00906625838615582
5 0.0088671625853749
6 0.00867880615987815
7 0.0085000488179503
8 0.00832947335038625
9 0.00816573085285199
10 0.00800756694115989
11 0.0078535529828514
12 0.00770117539468629
13 0.00754592475095706
14 0.00738187564274995
15 0.00720091764196695
16 0.00699916237499565
17 0.00676997029404447
18 0.00651137204476981
19 0.00623146744146652
20 0.00593921056133695
21 0.00564139045673073
22 0.0053396769981191
23 0.00503450822361629
24 0.00472865574192838
25 0.00442358269356191
26 0.00411944717779988
27 0.00382017279116553
28 0.00353077420368209
29 0.00325670138772693
30 0.0029998474765307
31 0.00276096043671714
32 0.00253999380947789
33 0.0023376605604426
34 0.00215214174568246
35 0.00198127879775711
36 0.00182275364204543
37 0.00167965074342646
38 0.00155167420780344
39 0.00143855675651139
40 0.00133706371525477
41 0.00124585186404147
42 0.0011603507464315
43 0.00108191095023358
44 0.0010126660217793
45 0.000950760255364003
46 0.000895478508027736
47 0.000846117412038438
48 0.00080174497952612
49 0.0007615295908181
50 0.000724831401385018
51 0.000691179101522721
52 0.000660560842788982
53 0.000632549260899395
54 0.000606861083724652
55 0.000583257688049343
56 0.000561548694804515
57 0.000541527313998813
58 0.000523055860867316
59 0.000505991136151351
60 0.000489964995267655
61 0.000474804269060769
62 0.000460658628753663
63 0.000447657322638406
64 0.000435698937963025
65 0.00042469363620512
66 0.000414559968248795
67 0.000405244114745074
68 0.000396663523360985
69 0.000388754516507106
70 0.000381432070980736
71 0.000374656902067727
72 0.000368134515838392
73 0.000361716396582779
74 0.000355921456957731
75 0.000350668374494489
76 0.000345840996260449
77 0.000341409772545376
78 0.000337323111807564
79 0.000333548212665846
80 0.000330018419845146
81 0.00032668547078174
82 0.000323573645346187
83 0.000320647699481924
84 0.000317904320127127
85 0.000315290968046611
86 0.000312796639036605
87 0.000310440397242928
88 0.000308224482296282
89 0.000306140797192711
90 0.000304197506807213
91 0.000302350627634951
92 0.000300595852763763
93 0.000298926980917713
94 0.000297329712338978
95 0.000295787156915139
96 0.000294305246598014
97 0.00029287602342265
98 0.000291532517053383
99 0.000290246343183753
100 0.000289007370383842
101 0.000287812951683009
102 0.000286662483404143
103 0.000285547115822737
104 0.000284470727478947
105 0.000283429407659241
106 0.000282413803347481
107 0.000281427386653377
108 0.00028046642080426
109 0.000279523276958571
110 0.000278600242495486
111 0.000277699372986717
112 0.000276822843602531
113 0.000275966513299863
114 0.000275133222658042
115 0.00027431502780928
116 0.000273517466666817
117 0.00027273559180685
118 0.000271970015774059
119 0.000271217903787146
120 0.000270475095817346
121 0.00026974054060247
122 0.000268958116180329
123 0.000267923321075614
124 0.00026695198255311
125 0.000266080954816061
126 0.000265276306095075
127 0.000264520525206535
128 0.00026379064968296
129 0.000263086675204249
130 0.000262387432485411
131 0.000261704474837643
132 0.00026103671166311
133 0.00026038462726774
134 0.000259747253949172
135 0.000259122578199822
136 0.00025851345560568
137 0.000257913907717011
138 0.000257332269484323
139 0.00025676122720597
140 0.000256200674414231
141 0.000255648394443142
142 0.000255105587541493
143 0.000254572545088649
144 0.000254045929580116
145 0.000253529062831603
146 0.000253018161913587
147 0.000252515379088436
148 0.000252018361265982
149 0.000251528238720766
150 0.000251044477010964
151 0.000250567767750454
152 0.000250094761497621
153 0.000249627476648584
154 0.000249166418598179
155 0.000248708856361191
156 0.00024824320882999
157 0.000247719576236705
158 0.000247214768535287
159 0.000246743294269436
160 0.000246271506000539
161 0.000245802454628574
162 0.000245330645270769
163 0.000244859166969036
164 0.000244392219372003
165 0.000243911508562178
166 0.000243447363573068
167 0.000242994886548331
168 0.000242551964277027
169 0.000242113516549125
170 0.000241690597420074
171 0.000241260654263442
172 0.000240841393633673
173 0.00024041472687486
174 0.00024001093811421
175 0.000239585672147768
176 0.000239178941967566
177 0.000238767756627567
178 0.000238363847017808
179 0.000237950023631583
180 0.000237547632195856
181 0.000237147122504666
182 0.000236744626533891
183 0.000236328302207767
184 0.000235932367843361
185 0.000235524782738139
186 0.000235129416921609
187 0.000234721737797372
188 0.00023432219671804
189 0.000233924728320289
190 0.000233520565018352
191 0.000233123756061104
192 0.000232731058559921
193 0.000232328992666453
194 0.000231925231020114
195 0.000231534705363856
196 0.00023113418060916
197 0.000230736056607839
198 0.000230340516964134
199 0.000229939898787279
200 0.000229547742407021
201 0.000229145654088825
202 0.0002287456815111
203 0.000228344317832807
204 0.000227949626008694
205 0.000227544832426929
206 0.000227143137863095
207 0.000226737125643695
208 0.00022634676946609
209 0.000225931005786606
210 0.000225375104548675
211 0.000224735926735775
212 0.000224132401172028
213 0.000223554789016589
214 0.000223011960912345
215 0.000222494751142222
216 0.000221996570530791
217 0.000221512319740214
218 0.000221044283080118
219 0.000220587369256009
220 0.000220134360517932
221 0.000219678589587602
222 0.000219225103705867
223 0.00021877061871578
224 0.000218323004958165
225 0.00021788553496549
226 0.000217423559490726
227 0.000216974203340214
228 0.000216516339889949
229 0.000216062065362621
230 0.000215597937568646
231 0.000215140780738921
232 0.000214680037203152
233 0.000214204136000262
234 0.000213736290845645
235 0.000213261628204009
236 0.000212775544525812
237 0.000212291746024107
238 0.00021180962143319
239 0.000211307686498685
240 0.000210803160641149
241 0.000210312105153321
242 0.000209794619649983
243 0.000209276930689839
244 0.000208760731226221
245 0.000208220124747527
246 0.000207636746566209
247 0.000207062435521266
248 0.000206390372497367
249 0.000205671947924202
250 0.000204971636733831
251 0.000204282157255875
252 0.000203583026205933
253 0.000202892361670592
254 0.000202229452582969
255 0.000201582897389585
256 0.000200943853315039
257 0.000200308150681394
258 0.000199653830193824
259 0.000198992121895003
260 0.000198333133283768
261 0.000197651009472111
262 0.000196963388660265
263 0.000196281530591591
264 0.000195575008405058
265 0.000194877085235134
266 0.000194161065493859
267 0.00019343009573447
268 0.000192714249848791
269 0.000191953520030097
270 0.000191213899739751
271 0.000190437538563515
272 0.000189656454878673
273 0.000188877269152954
274 0.000188058712623729
275 0.000187255893024485
276 0.000186418382739362
277 0.000185579922003853
278 0.000184720329770016
279 0.00018385679675248
280 0.000182979707247455
281 0.000182087789454499
282 0.000181132231077186
283 0.000180149655975015
284 0.000179172824530838
285 0.000178145676940744
286 0.000177158900896757
287 0.000176138686214244
288 0.000175092597643811
289 0.000174060134369824
290 0.000173006345363547
291 0.000171923461579127
292 0.000170852361975449
293 0.000169716018007193
294 0.000168579347914033
295 0.000167443553365842
296 0.000166313528183082
297 0.000165158575327951
298 0.000163969415183374
299 0.000162770001196577
300 0.000161592086058704
301 0.000160360343357979
302 0.000159121243129334
303 0.000157848754753331
304 0.00015660160640607
305 0.000155294875028744
306 0.000153961701634842
307 0.000152619679226973
308 0.000151285848694727
309 0.000149932266907626
310 0.000148554684287205
311 0.000147147072731002
312 0.000145832893025499
313 0.000144407210399322
314 0.000142966050617588
315 0.000141570959904413
316 0.000140122807621879
317 0.000138724121448774
318 0.000137263781660124
319 0.000135812136051072
320 0.000134384528401199
321 0.000132904014826352
322 0.000131436785892447
323 0.000129913968990536
324 0.000128415411658978
325 0.000126929933166764
326 0.000125434410705338
327 0.000123957754837534
328 0.000122477843852664
329 0.000120997941927214
330 0.000119563407288581
331 0.00011806816198856
332 0.000116598842808457
333 0.000115160764330824
334 0.000113698831938791
335 0.000112269421848055
336 0.000110829752053121
337 0.000109398091915125
338 0.000107876969252629
339 0.000106396929659525
340 0.000104948018723405
341 0.00010354143813629
342 0.000102160667893259
343 0.000100795781584395
344 9.94517073706902e-05
345 9.8137308754076e-05
346 9.68210656537849e-05
347 9.55434736056304e-05
348 9.42661588467786e-05
349 9.30328777108969e-05
350 9.17812806626728e-05
351 9.05797361987482e-05
352 8.93738963050339e-05
353 8.82150173779905e-05
354 8.70655906588524e-05
355 8.59381116065094e-05
356 8.48297294560041e-05
357 8.37726190709986e-05
358 8.2710183548329e-05
359 8.16797176135253e-05
360 8.0653996558766e-05
361 7.96710791632904e-05
362 7.86494326021625e-05
363 7.77121644333079e-05
364 7.67593492465579e-05
365 7.58144711827669e-05
366 7.48938757961781e-05
367 7.400366169108e-05
368 7.31157386955061e-05
369 7.22634102459097e-05
370 7.13940203169727e-05
371 7.05266452794007e-05
372 6.97039768180474e-05
373 6.88740268923738e-05
374 6.80729534323632e-05
375 6.72991569832249e-05
376 6.65271581290483e-05
377 6.57737537892444e-05
378 6.50778457469414e-05
379 6.43459422562387e-05
380 6.36428962934588e-05
381 6.29485417000808e-05
382 6.22997046502149e-05
383 6.16234591266362e-05
384 6.09776774034287e-05
385 6.03280462492251e-05
386 5.97305504967949e-05
387 5.91460865742022e-05
388 5.85400153862992e-05
389 5.79480524081077e-05
390 5.73595118442682e-05
391 5.68047659710302e-05
392 5.62772537762157e-05
393 5.57180563314219e-05
394 5.52391664614404e-05
395 5.47224926723544e-05
396 5.42248244972399e-05
397 5.37267357323401e-05
398 5.32267822617172e-05
399 5.27596400043961e-05
400 5.23016281146482e-05
401 5.18676188585587e-05
402 5.14314078499112e-05
403 5.10160004472127e-05
404 5.05998063928814e-05
405 5.01865720821115e-05
406 4.98029576956327e-05
407 4.94094768006548e-05
408 4.90233888745806e-05
409 4.86579234904028e-05
410 4.8286256054908e-05
411 4.79303399405495e-05
412 4.75777692123813e-05
413 4.72259523469631e-05
414 4.68940049200839e-05
415 4.65580828423739e-05
416 4.62312921154506e-05
417 4.59110964969867e-05
418 4.55897700550878e-05
419 4.52798726406911e-05
420 4.49758081888518e-05
421 4.46692704958451e-05
422 4.43793709221652e-05
423 4.4084385927512e-05
424 4.38011809658079e-05
425 4.35217559875412e-05
426 4.32470403959329e-05
427 4.29722385399955e-05
428 4.27072332414014e-05
429 4.24465315287392e-05
430 4.2184014547253e-05
431 4.19313970985158e-05
432 4.16766224358156e-05
433 4.14399037746982e-05
434 4.11862707774269e-05
435 4.09449990166344e-05
436 4.07154130437348e-05
437 4.04684222048424e-05
438 4.0246694380297e-05
439 4.00121847148327e-05
440 3.97967536009958e-05
441 3.95646149393691e-05
442 3.93539355485473e-05
443 3.91334429004075e-05
444 3.8928296475671e-05
445 3.8714422364361e-05
446 3.85142930747762e-05
447 3.83059465072222e-05
448 3.81282347228229e-05
449 3.79075981413735e-05
450 3.7740409590814e-05
451 3.75355338499972e-05
452 3.73540730498334e-05
453 3.71660397031626e-05
454 3.69743528523259e-05
455 3.68173178491027e-05
456 3.66235544122517e-05
457 3.64631567797424e-05
458 3.62652126568719e-05
459 3.61230997651951e-05
460 3.59399025775531e-05
461 3.57749505002225e-05
462 3.56136944268393e-05
463 3.54542931084723e-05
464 3.5281338238935e-05
465 3.51338371533672e-05
466 3.49559253547405e-05
467 3.48293907523889e-05
468 3.46682591381153e-05
469 3.44905412892338e-05
470 3.43360624146527e-05
471 3.41972429205306e-05
472 3.4063536884954e-05
473 3.39387046950357e-05
474 3.37943596111501e-05
475 3.36624292791576e-05
476 3.35201640182525e-05
477 3.33729447588027e-05
478 3.32306556884987e-05
479 3.30791817706455e-05
480 3.29307306423665e-05
481 3.27899396639353e-05
482 3.2649240012006e-05
483 3.25148907056594e-05
484 3.23878586812043e-05
485 3.2240206475187e-05
486 3.21094245800957e-05
487 3.19806276527501e-05
488 3.18595898760599e-05
489 3.17396928029012e-05
490 3.16220697378355e-05
491 3.15042735778981e-05
492 3.13895083978721e-05
493 3.12743158876749e-05
494 3.11582281753431e-05
495 3.10454846101926e-05
496 3.09286474649006e-05
497 3.08209143997651e-05
498 3.07112765960404e-05
499 3.0605028666475e-05
500 3.05012219716794e-05
501 3.03937778705077e-05
502 3.02915372025758e-05
503 3.01928751405001e-05
504 3.00887004072159e-05
505 2.9998093491912e-05
506 2.98970086438644e-05
507 2.98067098940535e-05
508 2.97123934820398e-05
509 2.96230892784521e-05
510 2.95349540397893e-05
511 2.94421464333006e-05
512 2.93605080692316e-05
513 2.92669677364188e-05
514 2.9178430447363e-05
515 2.90901239612396e-05
516 2.89899094827462e-05
517 2.89002916229464e-05
518 2.87958955929035e-05
519 2.87046413456338e-05
520 2.86064734282832e-05
521 2.85110463633664e-05
522 2.84209419363357e-05
523 2.83328984700404e-05
524 2.82384386354195e-05
525 2.81580956036365e-05
526 2.80710818314489e-05
527 2.79907860067929e-05
528 2.79053564695708e-05
529 2.78276503791197e-05
530 2.7737576741238e-05
531 2.76508985801449e-05
532 2.75561785376688e-05
533 2.74663170394263e-05
534 2.73781550843566e-05
535 2.73306487552638e-05
536 2.73348713548671e-05
537 2.75023856612933e-05
538 2.78874653423244e-05
539 2.84375417418659e-05
540 2.85588967585682e-05
541 2.79453986586908e-05
542 2.71399132905081e-05
543 2.68246935677396e-05
544 2.68266595893074e-05
545 2.68629206046089e-05
546 2.67998234591005e-05
547 2.66384788467633e-05
548 2.64508816050224e-05
549 2.62840008804499e-05
550 2.61557375789145e-05
551 2.60575058370627e-05
552 2.59754230471643e-05
553 2.59153767179221e-05
554 2.58433919735346e-05
555 2.5778668679699e-05
556 2.56917150642799e-05
557 2.56169156638286e-05
558 2.55409803768814e-05
559 2.54714870551176e-05
560 2.5391449144152e-05
561 2.5321028190195e-05
562 2.52460963792167e-05
563 2.51715724619572e-05
564 2.50984309717595e-05
565 2.50271746260466e-05
566 2.49539840013746e-05
567 2.48721623909987e-05
568 2.48042768298262e-05
569 2.47308199057272e-05
570 2.46648906210289e-05
571 2.45979577346844e-05
572 2.45365888815741e-05
573 2.44714113808442e-05
574 2.44114600493894e-05
575 2.43527413998912e-05
576 2.42893134321243e-05
577 2.42388755777867e-05
578 2.41660374724617e-05
579 2.41124328894671e-05
580 2.40426159368035e-05
581 2.39790817269103e-05
582 2.39028142097197e-05
583 2.38316128640115e-05
584 2.37542726186923e-05
585 2.36783369469418e-05
586 2.35965134498528e-05
587 2.35301761819517e-05
588 2.34478591465415e-05
589 2.33903587743001e-05
590 2.33137838034914e-05
591 2.32641558746138e-05
592 2.32026670590191e-05
593 2.31583201866226e-05
594 2.31074034870993e-05
595 2.30718778602146e-05
596 2.30253954853765e-05
597 2.29836327840793e-05
598 2.29635926316529e-05
599 2.28985306161e-05
600 2.28355477793762e-05
601 2.2745661034218e-05
602 2.26448173477056e-05
603 2.25322534463945e-05
604 2.24417867538129e-05
605 2.23985552676442e-05
606 2.25296784193141e-05
607 2.29653179060207e-05
608 2.38900977942436e-05
609 2.4894952461274e-05
610 2.46651194357739e-05
611 2.31864797635151e-05
612 2.22160381850856e-05
613 2.20817749037039e-05
614 2.21416848127376e-05
615 2.20700835076304e-05
616 2.19323847716613e-05
617 2.17755107470019e-05
618 2.16657113665253e-05
619 2.15628466155859e-05
620 2.14981903869926e-05
621 2.14305591397235e-05
622 2.1378834679453e-05
623 2.13242663853475e-05
624 2.12706737618973e-05
625 2.12200123419493e-05
626 2.11666941185573e-05
627 2.11172182200681e-05
628 2.10666343487542e-05
629 2.10223365906259e-05
630 2.09663017933792e-05
631 2.09186679955842e-05
632 2.0867131920399e-05
633 2.08163851211918e-05
634 2.07740385553823e-05
635 2.07176977253809e-05
636 2.06772159971891e-05
637 2.06222202883133e-05
638 2.05847257359615e-05
639 2.05282553167319e-05
640 2.04884469461319e-05
641 2.04355142450208e-05
642 2.03922849861815e-05
643 2.03421401063775e-05
644 2.02984845345533e-05
645 2.02504943886961e-05
646 2.02071473283105e-05
647 2.01598630289457e-05
648 2.01129201318206e-05
649 2.00668773377988e-05
650 2.00226799531578e-05
651 1.997045604063e-05
652 1.99330097006012e-05
653 1.98785724219164e-05
654 1.98348269069282e-05
655 1.97860821364859e-05
656 1.97452781256491e-05
657 1.96852737772701e-05
658 1.96434200119278e-05
659 1.95859465004222e-05
660 1.95375003693155e-05
661 1.94831594886313e-05
662 1.94248923350848e-05
663 1.9389626764843e-05
664 1.93378044988091e-05
665 1.92947177566349e-05
666 1.92574073927254e-05
667 1.92467715887101e-05
668 1.92701141454865e-05
669 1.93756578400484e-05
670 1.96106836252152e-05
671 2.0005571877757e-05
672 2.05004305481316e-05
673 2.07607145118471e-05
674 2.04070591762751e-05
675 1.95846620361095e-05
676 1.89960071377282e-05
677 1.89184472025872e-05
678 1.92536772514185e-05
679 1.97754200847999e-05
680 2.00870057667224e-05
681 1.98656447347734e-05
682 1.92737964070933e-05
683 1.87529609634307e-05
684 1.84938348117836e-05
685 1.838811488164e-05
686 1.83293324729217e-05
687 1.8275051614447e-05
688 1.82364465783902e-05
689 1.81828388077232e-05
690 1.81437859865241e-05
691 1.81043523734825e-05
692 1.80607413320377e-05
693 1.80271606062821e-05
694 1.79828973716489e-05
695 1.79505448225648e-05
696 1.7904135188207e-05
697 1.78777116983664e-05
698 1.7827046080221e-05
699 1.7790628349168e-05
700 1.77524225108971e-05
701 1.76968648291087e-05
702 1.76672774638842e-05
703 1.76201273385246e-05
704 1.75825011297093e-05
705 1.75236640016507e-05
706 1.75057237217224e-05
707 1.74435141022666e-05
708 1.74072897569655e-05
709 1.73686541746765e-05
710 1.73289548421884e-05
711 1.72786970957794e-05
712 1.72529831692358e-05
713 1.7209697063425e-05
714 1.71583415533583e-05
715 1.71524926042554e-05
716 1.71063572689256e-05
717 1.70705267419358e-05
718 1.70413758562304e-05
719 1.70087875495106e-05
720 1.69663838462952e-05
721 1.69511305814218e-05
722 1.68754695932805e-05
723 1.68391467576523e-05
724 1.67821699386117e-05
725 1.67428871769015e-05
726 1.66914541828156e-05
727 1.66552091085403e-05
728 1.65785646406569e-05
729 1.65395342570651e-05
730 1.64550399306407e-05
731 1.64001117484158e-05
732 1.63481424104717e-05
733 1.63226625364832e-05
734 1.63613305290689e-05
735 1.65117913517587e-05
736 1.68433574658788e-05
737 1.73569080336167e-05
738 1.78873486703646e-05
739 1.79708007363999e-05
740 1.73590151356651e-05
741 1.64937540199084e-05
742 1.60261793973326e-05
743 1.60411849861042e-05
744 1.61911461478148e-05
745 1.6345935831108e-05
746 1.62518951891677e-05
747 1.59426875812052e-05
748 1.56066324170645e-05
749 1.53650269101302e-05
750 1.52523461276477e-05
751 1.52017326859877e-05
752 1.51842354598752e-05
753 1.51407069162168e-05
754 1.51058000745774e-05
755 1.5044462353897e-05
756 1.49794182062735e-05
757 1.49054082427824e-05
758 1.48823948771248e-05
759 1.48532990282835e-05
760 1.47964316679072e-05
761 1.47389365472428e-05
762 1.47320619621194e-05
763 1.46906544473691e-05
764 1.46925340194226e-05
765 1.46471131610149e-05
766 1.46238986977743e-05
767 1.45876747741358e-05
768 1.4546445674446e-05
769 1.44886796702437e-05
770 1.4428898936103e-05
771 1.43735292137404e-05
772 1.43100033935806e-05
773 1.42802908464912e-05
774 1.42318653748585e-05
775 1.42095834869682e-05
776 1.41990672750669e-05
777 1.41903479229377e-05
778 1.4163851086213e-05
779 1.40923017650785e-05
780 1.40652681517306e-05
781 1.40314014376308e-05
782 1.39919528923116e-05
783 1.39486494556973e-05
784 1.38864278134143e-05
785 1.38349474476662e-05
786 1.37519560254873e-05
787 1.36640468419991e-05
788 1.36006403324007e-05
789 1.36953738598677e-05
790 1.40441089677701e-05
791 1.47873385245578e-05
792 1.58539705834571e-05
793 1.6303446467747e-05
794 1.54049234248888e-05
795 1.40092870883901e-05
796 1.3511264449706e-05
797 1.35904357039252e-05
798 1.36887301025457e-05
799 1.36211770718209e-05
800 1.34541402490029e-05
801 1.32557650067699e-05
802 1.31555567755015e-05
803 1.30249973289143e-05
804 1.29428202912951e-05
805 1.28789849018762e-05
806 1.28440691002929e-05
807 1.28156102952026e-05
808 1.27594264123232e-05
809 1.27458418519577e-05
810 1.269313907315e-05
811 1.26576695096414e-05
812 1.26413873264841e-05
813 1.26000367426382e-05
814 1.25559946508869e-05
815 1.25302382576464e-05
816 1.25109311401861e-05
817 1.24684005171627e-05
818 1.24283272027181e-05
819 1.239911855766e-05
820 1.23885279537106e-05
821 1.23483782303868e-05
822 1.23110995227105e-05
823 1.2276056183369e-05
824 1.22587971547183e-05
825 1.22406756066784e-05
826 1.22014604040377e-05
827 1.21732214874992e-05
828 1.21496322320613e-05
829 1.21268924127005e-05
830 1.2098041009212e-05
831 1.20662456062348e-05
832 1.20564360361897e-05
833 1.20481736540778e-05
834 1.2008665025931e-05
835 1.19874320818614e-05
836 1.19496606523217e-05
837 1.19316190494434e-05
838 1.19169067847036e-05
839 1.18787781075991e-05
840 1.1823955139656e-05
841 1.17954793026342e-05
842 1.17556451781908e-05
843 1.17300904360906e-05
844 1.16777026069048e-05
845 1.16381836752222e-05
846 1.15941109104511e-05
847 1.15862499773245e-05
848 1.15671668410311e-05
849 1.16056115337937e-05
850 1.16947510704346e-05
851 1.1937642651616e-05
852 1.23365971897993e-05
853 1.2902901994849e-05
854 1.33307631952073e-05
855 1.31455417010651e-05
856 1.23843446317995e-05
857 1.16194597968855e-05
858 1.1326182707716e-05
859 1.15563788067874e-05
860 1.20973316726225e-05
861 1.25754791580057e-05
862 1.26076354296512e-05
863 1.21143999312068e-05
864 1.1518440674374e-05
865 1.11536378408772e-05
866 1.09752634296356e-05
867 1.09370821961274e-05
868 1.08875689643906e-05
869 1.08633234505717e-05
870 1.08319758540532e-05
871 1.08035916284166e-05
872 1.07945917597885e-05
873 1.07706561673915e-05
874 1.07604409622652e-05
875 1.07473690669213e-05
876 1.0738439193636e-05
877 1.07126174793137e-05
878 1.0697302146534e-05
879 1.06809935951357e-05
880 1.0644809907312e-05
881 1.06327052045163e-05
882 1.06018665284591e-05
883 1.05747894465491e-05
884 1.05547113473659e-05
885 1.05265527576837e-05
886 1.05064739117644e-05
887 1.04758889010181e-05
888 1.04545853087945e-05
889 1.04293884624074e-05
890 1.04121454678197e-05
891 1.03962060136187e-05
892 1.03739836700978e-05
893 1.03454991235985e-05
894 1.03196093674818e-05
895 1.03114637047819e-05
896 1.02812623234971e-05
897 1.02722024558632e-05
898 1.02514187551028e-05
899 1.02423935081086e-05
900 1.02407386268766e-05
901 1.02514648766538e-05
902 1.02560103645288e-05
903 1.02853759932353e-05
904 1.03164956927193e-05
905 1.0377977717857e-05
906 1.04492657229116e-05
907 1.05060474037133e-05
908 1.05415499235262e-05
909 1.05358930695409e-05
910 1.04244082369531e-05
911 1.02748606090053e-05
912 1.01112244036416e-05
913 1.00215303364681e-05
914 1.00984921410641e-05
915 1.03861095590574e-05
916 1.10169810660921e-05
917 1.18333868410758e-05
918 1.2270305300599e-05
919 1.17707336997341e-05
920 1.07219416388693e-05
921 1.00288032479634e-05
922 9.82655398262811e-06
923 9.88101182797241e-06
924 9.98052949363615e-06
925 1.00182301431273e-05
926 9.9849648399486e-06
927 9.88021081127499e-06
928 9.79444502524629e-06
929 9.69977198428396e-06
930 9.65269695440973e-06
931 9.60466533184157e-06
932 9.58252567340168e-06
933 9.56460014589489e-06
934 9.55320716367059e-06
935 9.52848290847186e-06
936 9.51193578901943e-06
937 9.50064544458229e-06
938 9.47623834846034e-06
939 9.47120993533002e-06
940 9.45325906298589e-06
941 9.43517790064075e-06
942 9.430259694021e-06
943 9.43247993456797e-06
944 9.41609750249039e-06
945 9.42875281007005e-06
946 9.43407691222919e-06
947 9.43073313253162e-06
948 9.44277216974321e-06
949 9.46703429416385e-06
950 9.47581985588641e-06
951 9.45989628675648e-06
952 9.45149713227345e-06
953 9.41412245603068e-06
954 9.38253240778675e-06
955 9.34026443033531e-06
956 9.2780121038194e-06
957 9.24389977452478e-06
958 9.21706020529101e-06
959 9.2286348069015e-06
960 9.32182885915456e-06
961 9.52640187878995e-06
962 9.84754162369761e-06
963 1.03404852068723e-05
964 1.080603191006e-05
965 1.08840133234978e-05
966 1.03932255893779e-05
967 9.66646568523188e-06
968 9.21567783873645e-06
969 9.19339878990844e-06
970 9.49394359461841e-06
971 9.92293296764313e-06
972 1.02015605487615e-05
973 1.01512688432948e-05
974 9.78189379852168e-06
975 9.3551083857335e-06
976 9.06972314940191e-06
977 8.96128781924244e-06
978 8.9279792727659e-06
979 8.91917375067663e-06
980 8.90779329409952e-06
981 8.89310112806019e-06
982 8.86568094948803e-06
983 8.83752558411555e-06
984 8.8180367406876e-06
985 8.8044588563152e-06
986 8.78761341005507e-06
987 8.79105507556321e-06
988 8.78435439732872e-06
989 8.78518325253541e-06
990 8.78529326098221e-06
991 8.78486861288774e-06
992 8.78534816883825e-06
993 8.76917949277356e-06
994 8.76589919573956e-06
995 8.75363179086985e-06
996 8.7230333678523e-06
997 8.69889214427655e-06
998 8.688020903036e-06
999 8.67691936745629e-06
1000 8.63510248116928e-06
1001 8.61947669184016e-06
1002 8.61866389412214e-06
1003 8.6244952953729e-06
1004 8.66478729277809e-06
1005 8.72864878798119e-06
1006 8.8622943001937e-06
1007 9.0053688920344e-06
1008 9.17720127069366e-06
1009 9.36094084125205e-06
1010 9.44513357414856e-06
1011 9.33378328249468e-06
1012 9.06167237701982e-06
1013 8.74391715788292e-06
1014 8.5489134846739e-06
1015 8.59122992569183e-06
1016 8.95580634308057e-06
1017 9.6579775302752e-06
1018 1.04879474575048e-05
1019 1.08183299694087e-05
1020 1.01842429960364e-05
1021 9.16676884787915e-06
1022 8.54015956142096e-06
1023 8.41150882568975e-06
1024 8.43055385169045e-06
1025 8.45555911666018e-06
1026 8.42546537138666e-06
1027 8.38342582221152e-06
1028 8.33633322872007e-06
1029 8.28962250309928e-06
1030 8.2577706814746e-06
1031 8.2390040169189e-06
1032 8.23198631516586e-06
1033 8.22956308044187e-06
1034 8.22343259687663e-06
1035 8.22076901718649e-06
1036 8.21447974852507e-06
1037 8.20481982644061e-06
1038 8.18590502849048e-06
1039 8.1741069503094e-06
1040 8.15593245451218e-06
1041 8.14528864123698e-06
1042 8.1250819474743e-06
1043 8.11261166933086e-06
1044 8.09956361569331e-06
1045 8.08620160253071e-06
1046 8.07336999897323e-06
1047 8.06402225228542e-06
1048 8.06318032275932e-06
1049 8.0514523529418e-06
1050 8.05805981884955e-06
1051 8.04999641301762e-06
1052 8.04097815176874e-06
1053 8.03019037132202e-06
1054 8.02367702190665e-06
1055 8.01416853335457e-06
1056 8.00901469422133e-06
1057 8.00126160560577e-06
1058 7.98206701424498e-06
1059 7.96920498058462e-06
1060 7.9536329276042e-06
1061 7.93366865969869e-06
1062 7.91994518911654e-06
1063 7.91823706980388e-06
1064 7.94698240424996e-06
1065 8.0856265873841e-06
1066 8.46293548550925e-06
1067 9.34457490331386e-06
1068 1.095182898736e-05
1069 1.24324051788971e-05
1070 1.15724135404394e-05
1071 9.14963463294338e-06
1072 8.19697210818759e-06
1073 8.48030650157927e-06
1074 8.73938296430676e-06
1075 8.66758952167146e-06
1076 8.37202453984887e-06
1077 8.08402772811689e-06
1078 7.90836817787799e-06
1079 7.81944060657125e-06
1080 7.77529133677568e-06
1081 7.74459393171334e-06
1082 7.73361002115447e-06
1083 7.71607368510452e-06
1084 7.7058059035906e-06
1085 7.69566381086939e-06
1086 7.6813239884288e-06
1087 7.67046840399743e-06
1088 7.66185403255903e-06
1089 7.6558813062455e-06
1090 7.64831157784585e-06
1091 7.63190959252036e-06
1092 7.63003085757497e-06
1093 7.61353312705992e-06
1094 7.6099554384701e-06
1095 7.60419970613491e-06
1096 7.59455800247721e-06
1097 7.58363886799174e-06
1098 7.57939122353335e-06
1099 7.5665916865475e-06
1100 7.56635242882275e-06
1101 7.55803076790684e-06
1102 7.54832171145914e-06
1103 7.5420916498814e-06
1104 7.53398903907687e-06
1105 7.52955118166199e-06
1106 7.52473275067445e-06
1107 7.51947660648788e-06
1108 7.51019820799392e-06
1109 7.5059904911523e-06
1110 7.49916280429197e-06
1111 7.49686339296396e-06
1112 7.49241969666414e-06
1113 7.49179525283417e-06
1114 7.48814958018151e-06
1115 7.49028947799601e-06
1116 7.48969416375367e-06
1117 7.49753690731936e-06
1118 7.51147349342673e-06
1119 7.53813571452433e-06
1120 7.57433123110829e-06
1121 7.63281361715329e-06
1122 7.73522949870653e-06
1123 7.87942485414028e-06
1124 8.10099690884414e-06
1125 8.3498919234426e-06
1126 8.5547563972721e-06
1127 8.56173666963223e-06
1128 8.28138639497045e-06
1129 7.84318122681871e-06
1130 7.54545669678919e-06
1131 7.6290516829447e-06
1132 8.2302480444163e-06
1133 9.39444091996311e-06
1134 1.04042392685688e-05
1135 1.01401345046215e-05
1136 8.84533744793803e-06
1137 7.80971373637307e-06
1138 7.3781638905146e-06
1139 7.28331053734266e-06
1140 7.26718288568406e-06
1141 7.26211956370015e-06
1142 7.24074184299539e-06
1143 7.2262822168323e-06
1144 7.21188232311043e-06
1145 7.20396760334197e-06
1146 7.19301300478037e-06
1147 7.19303693941242e-06
1148 7.18617250483611e-06
1149 7.18403902078002e-06
1150 7.17761146384888e-06
1151 7.17136128747242e-06
1152 7.16342750650689e-06
1153 7.15860323019513e-06
1154 7.15023903286038e-06
1155 7.1426426007104e-06
1156 7.13610463587422e-06
1157 7.12846025718861e-06
1158 7.12374459288512e-06
1159 7.11574059919506e-06
1160 7.10812777726666e-06
1161 7.10419965943565e-06
1162 7.09460915526705e-06
1163 7.09636159701432e-06
1164 7.0845681745535e-06
1165 7.08039634678137e-06
1166 7.07746078343341e-06
1167 7.07451293946448e-06
1168 7.07097882779451e-06
1169 7.06987566956307e-06
1170 7.06689912610692e-06
1171 7.07178980818846e-06
1172 7.07379923836626e-06
1173 7.0939429792638e-06
1174 7.11335993019979e-06
1175 7.15334244705268e-06
1176 7.20976119095162e-06
1177 7.30760077649606e-06
1178 7.45830426374106e-06
1179 7.66603735113947e-06
1180 7.94948682902863e-06
1181 8.19872545143596e-06
1182 8.28515515904904e-06
1183 8.04693821354618e-06
1184 7.58359653529883e-06
1185 7.1874405794814e-06
1186 7.17261370164479e-06
1187 7.69597575145387e-06
1188 8.63782611582042e-06
1189 9.56415855868364e-06
1190 9.74629371652824e-06
1191 8.72849193678249e-06
1192 7.61507793889393e-06
1193 7.06781655779309e-06
1194 6.92115682898731e-06
1195 6.90327971897275e-06
1196 6.89058604064208e-06
1197 6.88209037225107e-06
1198 6.86305451558233e-06
1199 6.85086968577053e-06
1200 6.84245162352326e-06
1201 6.84156442787298e-06
1202 6.83502655651758e-06
1203 6.83138027035568e-06
1204 6.82683339237045e-06
1205 6.82228769099957e-06
1206 6.82002483576305e-06
1207 6.81686428194439e-06
1208 6.80926241058977e-06
1209 6.80320130275192e-06
1210 6.79852451979102e-06
1211 6.79408095627387e-06
1212 6.78451134938918e-06
1213 6.77893382561301e-06
1214 6.78053347957785e-06
1215 6.76838949065228e-06
1216 6.76387126241629e-06
1217 6.75576120645616e-06
1218 6.75340236977462e-06
1219 6.75562735064084e-06
1220 6.747993793077e-06
1221 6.74319338234142e-06
1222 6.74339049422379e-06
1223 6.73967859277091e-06
1224 6.74433160319232e-06
1225 6.74219065577297e-06
1226 6.74595803107358e-06
1227 6.75750818546206e-06
1228 6.77173552077193e-06
1229 6.81846120365037e-06
1230 6.88007115856593e-06
1231 6.98253605202481e-06
1232 7.14484800634629e-06
1233 7.36944612333978e-06
1234 7.65135854652321e-06
1235 7.90283767515554e-06
1236 7.97463926405229e-06
1237 7.70492244073218e-06
1238 7.21439809869828e-06
1239 6.83442460325256e-06
1240 6.8542009621364e-06
1241 7.40749125405227e-06
1242 8.40634038035137e-06
1243 9.31660514291721e-06
1244 9.32937582165394e-06
1245 8.27302098105331e-06
1246 7.1820519473853e-06
1247 6.72388935241131e-06
1248 6.59042933937037e-06
1249 6.56508435215031e-06
1250 6.55896702350667e-06
1251 6.54955658752066e-06
1252 6.53240038328562e-06
1253 6.52304777459811e-06
1254 6.5158729694037e-06
1255 6.50991531259137e-06
1256 6.50916707045646e-06
1257 6.50747229169468e-06
1258 6.50323106277106e-06
1259 6.50266999291382e-06
1260 6.49866717772341e-06
1261 6.49383513384727e-06
1262 6.48963913207901e-06
1263 6.48463666230725e-06
1264 6.47899236883021e-06
1265 6.47319478797037e-06
1266 6.46981200946861e-06
1267 6.46497481815445e-06
1268 6.461036597738e-06
1269 6.45205093330325e-06
1270 6.44938989768917e-06
1271 6.44500656088454e-06
1272 6.43925114429678e-06
1273 6.4399238404178e-06
1274 6.43201909222846e-06
1275 6.43147480583472e-06
1276 6.42899391212204e-06
1277 6.42630653335807e-06
1278 6.42584572996263e-06
1279 6.43572589598129e-06
1280 6.44190988086102e-06
1281 6.45509285979173e-06
1282 6.4835895112747e-06
1283 6.52316448546131e-06
1284 6.59469685992953e-06
1285 6.71409705210735e-06
1286 6.90923966906176e-06
1287 7.20170861168157e-06
1288 7.57625303626774e-06
1289 7.91545976297492e-06
1290 7.94739620646467e-06
1291 7.53931571506605e-06
1292 6.90820133364412e-06
1293 6.54843359759738e-06
1294 6.74534762390433e-06
1295 7.49690744350495e-06
1296 8.49631116639493e-06
1297 8.9653547705737e-06
1298 8.24586271586103e-06
1299 7.00059384883822e-06
1300 6.52652268740006e-06
1301 6.5062971459362e-06
1302 6.61890015174471e-06
1303 6.67038281343402e-06
1304 6.63337448081336e-06
1305 6.55018017958753e-06
1306 6.4607218215329e-06
1307 6.38965703925187e-06
1308 6.33433974939202e-06
1309 6.29915188077312e-06
1310 6.28213592390736e-06
1311 6.26580816520672e-06
1312 6.25835640466477e-06
1313 6.2524720993995e-06
1314 6.24416295957886e-06
1315 6.24154466111193e-06
1316 6.23543646205427e-06
1317 6.23460175375179e-06
1318 6.22998665678054e-06
1319 6.22268187400188e-06
1320 6.2203334083577e-06
1321 6.21546858337041e-06
1322 6.21228816521935e-06
1323 6.21063835137825e-06
1324 6.20683102492414e-06
1325 6.19839506987852e-06
1326 6.2020006297292e-06
1327 6.20017198693645e-06
1328 6.19744056651328e-06
1329 6.20053866673409e-06
1330 6.19624921482931e-06
1331 6.1973557330397e-06
1332 6.20064636946971e-06
1333 6.20389664485188e-06
1334 6.20762954972598e-06
1335 6.21497866970522e-06
1336 6.23158574697413e-06
1337 6.24533298965169e-06
1338 6.26453721630504e-06
1339 6.29945667451892e-06
1340 6.35269074944489e-06
1341 6.42348922075264e-06
1342 6.519660265214e-06
1343 6.64793707549904e-06
1344 6.79637190126314e-06
1345 6.94850628923938e-06
1346 7.02431429999706e-06
1347 6.97375609992612e-06
1348 6.74948807111164e-06
1349 6.45363250839637e-06
1350 6.2560577507309e-06
1351 6.34126049536832e-06
1352 6.90459283847478e-06
1353 8.05883673482555e-06
1354 9.38675849404547e-06
1355 9.59917286702527e-06
1356 8.29481557185208e-06
1357 6.82443131294619e-06
1358 6.20846415388954e-06
1359 6.12750788109295e-06
1360 6.14573337931468e-06
1361 6.13342642408199e-06
1362 6.0992505379609e-06
1363 6.06274514103333e-06
1364 6.0415582576745e-06
1365 6.02630515711411e-06
1366 6.02028412544087e-06
1367 6.0152255139112e-06
1368 6.00965393604547e-06
1369 6.01360549778995e-06
1370 6.00780210535667e-06
1371 6.00581225573826e-06
1372 6.00255927452054e-06
1373 6.00204442080532e-06
1374 5.99810710721904e-06
1375 5.99382334742771e-06
1376 5.99023486280714e-06
1377 5.98777820282059e-06
1378 5.98395789919159e-06
1379 5.98225627990168e-06
1380 5.97412686542675e-06
1381 5.9736045698866e-06
1382 5.96989173629048e-06
1383 5.9668211049857e-06
1384 5.96305989786217e-06
1385 5.96293081800425e-06
1386 5.96189195678498e-06
1387 5.96191190571638e-06
1388 5.95953665416538e-06
1389 5.95644556855568e-06
1390 5.95692071914122e-06
1391 5.95930290203661e-06
1392 5.95915067425068e-06
1393 5.96638919581061e-06
1394 5.9741251059453e-06
1395 5.98840824594937e-06
1396 6.00961075791773e-06
1397 6.04885404320576e-06
1398 6.11813685758023e-06
1399 6.21594122529601e-06
1400 6.41335429807377e-06
1401 6.70425297855104e-06
1402 7.13353098169556e-06
1403 7.58456035199195e-06
1404 7.78427789516201e-06
1405 7.43717092532847e-06
1406 6.69733904545833e-06
1407 6.16982566203905e-06
1408 6.26037293172121e-06
1409 6.96669958566432e-06
1410 7.95652915663858e-06
1411 8.50197515944728e-06
1412 8.0306144589759e-06
1413 7.04541463081299e-06
1414 6.30044885907566e-06
1415 5.97764878595974e-06
1416 5.87472895130503e-06
1417 5.85029037813989e-06
1418 5.84123876179454e-06
1419 5.83363918948976e-06
1420 5.83022951494705e-06
1421 5.82693337025653e-06
1422 5.82809061189238e-06
1423 5.82881559241244e-06
1424 5.82560727480086e-06
1425 5.82702556783943e-06
1426 5.82602855669911e-06
1427 5.82260567805193e-06
1428 5.82220280631596e-06
1429 5.81656582943069e-06
1430 5.81330363180754e-06
1431 5.81201491423045e-06
1432 5.80989225196227e-06
1433 5.80396691507445e-06
1434 5.79924006971488e-06
1435 5.79847403647804e-06
1436 5.79230529318409e-06
1437 5.79261970523604e-06
1438 5.78564377140545e-06
1439 5.78650779625711e-06
1440 5.78041419907471e-06
1441 5.77468722195995e-06
1442 5.76697687826311e-06
1443 5.76819712594379e-06
1444 5.7633243923938e-06
1445 5.76178355093937e-06
1446 5.75995504425997e-06
1447 5.76467346347087e-06
1448 5.76296088894424e-06
1449 5.77019070346907e-06
1450 5.78270824358462e-06
1451 5.80045609210877e-06
1452 5.8428265188315e-06
1453 5.9085709569473e-06
1454 6.04146884097823e-06
1455 6.25627336581225e-06
1456 6.62865495892895e-06
1457 7.14725370820624e-06
1458 7.71127265464244e-06
1459 7.85993468399582e-06
1460 7.32457722008029e-06
1461 6.43309853831653e-06
1462 5.9647842568511e-06
1463 6.22032402086692e-06
1464 6.94548144597462e-06
1465 7.83115216607122e-06
1466 7.80678881384667e-06
1467 7.42215320714834e-06
1468 6.43873941985618e-06
1469 5.97721818285102e-06
1470 5.78687444185988e-06
1471 5.68984219251334e-06
1472 5.66151031433648e-06
1473 5.65335779767473e-06
1474 5.64854471751808e-06
1475 5.64008102221791e-06
1476 5.64979556583189e-06
1477 5.63777821627198e-06
1478 5.62858980246439e-06
1479 5.6329455384585e-06
1480 5.62575477247584e-06
1481 5.62886633681536e-06
1482 5.62605176845921e-06
1483 5.64308728989715e-06
1484 5.61593828063423e-06
1485 5.6072366063642e-06
1486 5.60816300332334e-06
1487 5.60073666999727e-06
1488 5.59501789476613e-06
1489 5.59518579179574e-06
1490 5.59088186435908e-06
1491 5.58282572438173e-06
1492 5.58625276725344e-06
1493 5.57940071499452e-06
1494 5.59911745789954e-06
1495 5.57613287677228e-06
1496 5.56767519266366e-06
1497 5.56831594566987e-06
1498 5.56336351964504e-06
1499 5.56463388723394e-06
1500 5.56361978443221e-06
1501 5.56220921343886e-06
1502 5.57019602975828e-06
1503 5.57125319833318e-06
1504 5.59739539562898e-06
1505 5.60084664069649e-06
1506 5.61630993201057e-06
1507 5.67862935851871e-06
1508 5.79123579269591e-06
1509 5.99541720047725e-06
1510 6.3684497622063e-06
1511 6.9295086710941e-06
1512 7.62732822168211e-06
1513 7.9485012438596e-06
1514 7.50985489617051e-06
1515 6.47770799577785e-06
1516 5.83192427439094e-06
1517 6.04761706735069e-06
1518 6.64098745950525e-06
1519 7.38991899340569e-06
1520 7.56755325492797e-06
1521 7.08262816373484e-06
1522 6.3133951373473e-06
1523 5.86818793801847e-06
1524 5.58349588408724e-06
1525 5.49458938126079e-06
1526 5.48288612711723e-06
1527 5.46470996098947e-06
1528 5.4680536334395e-06
1529 5.45767363324678e-06
1530 5.46011847446515e-06
1531 5.45620954173209e-06
1532 5.45743534585696e-06
1533 5.45330482504269e-06
1534 5.46121572475577e-06
1535 5.45010634867005e-06
1536 5.46316127891799e-06
1537 5.44832464277611e-06
1538 5.46238230647589e-06
1539 5.44296000759559e-06
1540 5.44965037452272e-06
1541 5.44006026048649e-06
1542 5.44606051899876e-06
1543 5.42982138806991e-06
1544 5.43480709458422e-06
1545 5.42397324854171e-06
1546 5.42707963280264e-06
1547 5.41430021527844e-06
1548 5.41503058393289e-06
1549 5.40921875713529e-06
1550 5.4111724299144e-06
1551 5.40103435731076e-06
1552 5.40709743956924e-06
1553 5.40140072535777e-06
1554 5.4080183065075e-06
1555 5.4090310381838e-06
1556 5.42075081599869e-06
1557 5.44392509471692e-06
1558 5.48719807036413e-06
1559 5.56253707473253e-06
1560 5.68659698596541e-06
1561 5.90526711419237e-06
1562 6.22133649019752e-06
1563 6.67426808109717e-06
1564 7.05968049974359e-06
1565 7.18295602553098e-06
1566 6.74671001688054e-06
1567 6.06424744065137e-06
1568 5.61324583425726e-06
1569 5.68503196674008e-06
1570 6.31298311115636e-06
1571 7.09040830204621e-06
1572 7.68187430977285e-06
1573 7.40705365220862e-06
1574 6.66269555105181e-06
1575 5.89634643399783e-06
1576 5.49466841537338e-06
1577 5.36488595570539e-06
1578 5.33642925293343e-06
1579 5.32616002590913e-06
1580 5.31971429729161e-06
1581 5.31231322664638e-06
1582 5.30726714220719e-06
1583 5.30373507223736e-06
1584 5.30475603444103e-06
1585 5.30362099149251e-06
1586 5.30522343278328e-06
1587 5.30634859630119e-06
1588 5.31142219406533e-06
1589 5.30428937262073e-06
1590 5.31172709994365e-06
1591 5.3017652554832e-06
1592 5.30350563732185e-06
1593 5.30061097880896e-06
1594 5.30228954875867e-06
1595 5.29323583342567e-06
1596 5.29430212869997e-06
1597 5.28789489373516e-06
1598 5.29163927365239e-06
1599 5.28043650760424e-06
1600 5.28459824944427e-06
1601 5.27513922565781e-06
1602 5.27377646353244e-06
1603 5.27163464680847e-06
1604 5.27394189364649e-06
1605 5.26620826701851e-06
1606 5.26776863085665e-06
1607 5.26527251376585e-06
1608 5.27467481226118e-06
1609 5.27703529562551e-06
1610 5.30171073376273e-06
1611 5.34200744084501e-06
1612 5.42010633974854e-06
1613 5.56921572969493e-06
1614 5.83866682823952e-06
1615 6.3077685374946e-06
1616 6.97264288795907e-06
1617 7.61958863138013e-06
1618 7.64368586669839e-06
1619 6.81159897264472e-06
1620 5.82906254487092e-06
1621 5.49639650415656e-06
1622 5.89230725955847e-06
1623 6.6196049703926e-06
1624 7.16311238502243e-06
1625 7.02590554713112e-06
1626 6.40668860518723e-06
1627 5.76065229118683e-06
1628 5.43871991909306e-06
1629 5.2633405653868e-06
1630 5.23281066455716e-06
1631 5.20116319524533e-06
1632 5.19636398665924e-06
1633 5.18748865752094e-06
1634 5.18819129635695e-06
1635 5.18661848891711e-06
1636 5.19213044136713e-06
1637 5.18485887801567e-06
1638 5.19483615590843e-06
1639 5.18714507480311e-06
1640 5.19082676819771e-06
1641 5.18295628637588e-06
1642 5.19385733466216e-06
1643 5.18315483155618e-06
1644 5.19317447933432e-06
1645 5.17806632127638e-06
1646 5.19044132318847e-06
1647 5.17476236616687e-06
1648 5.18045026920433e-06
1649 5.17058695437278e-06
1650 5.17754831386164e-06
1651 5.16257848026669e-06
1652 5.1670023391015e-06
1653 5.15814237567191e-06
1654 5.16294622698865e-06
1655 5.15218508567727e-06
1656 5.15302952819319e-06
1657 5.14678415686198e-06
1658 5.15263245026709e-06
1659 5.13922937650513e-06
1660 5.14721829669718e-06
1661 5.15094117292669e-06
1662 5.1606071151955e-06
1663 5.17662776799988e-06
1664 5.21931653785224e-06
1665 5.30139637744398e-06
1666 5.46317738914226e-06
1667 5.77482981056754e-06
1668 6.32273957568152e-06
1669 7.12765720156661e-06
1670 7.83532850157087e-06
1671 7.75318511836431e-06
1672 6.65374100516924e-06
1673 5.62061441600203e-06
1674 5.46638080956008e-06
1675 5.9846420770171e-06
1676 6.71818000297364e-06
1677 6.93298866227821e-06
1678 6.62199890677506e-06
1679 5.95867344554968e-06
1680 5.50150802447469e-06
1681 5.24382178590521e-06
1682 5.14655217953575e-06
1683 5.10067054482199e-06
1684 5.09436274942487e-06
1685 5.08255797759816e-06
1686 5.0802884297152e-06
1687 5.08087891826214e-06
1688 5.08178289115868e-06
1689 5.07970945684555e-06
1690 5.08713593516674e-06
1691 5.07842029118244e-06
1692 5.08229829088158e-06
1693 5.07951661687933e-06
1694 5.0834754414808e-06
1695 5.07436637686709e-06
1696 5.08189012937699e-06
1697 5.06927262389212e-06
1698 5.07426794582599e-06
1699 5.06478618444994e-06
1700 5.07137279726066e-06
1701 5.05976930131879e-06
1702 5.06551910017805e-06
1703 5.05489797020253e-06
1704 5.05678203732707e-06
1705 5.05121950000742e-06
1706 5.05381063398147e-06
1707 5.04190007744043e-06
1708 5.04742299267669e-06
1709 5.03949936780934e-06
1710 5.04499307840689e-06
1711 5.0355833225435e-06
1712 5.0377768601173e-06
1713 5.03540340046627e-06
1714 5.04074809537514e-06
1715 5.03880896696174e-06
1716 5.05268770800704e-06
1717 5.06602725391936e-06
1718 5.10723618329934e-06
1719 5.1691276099497e-06
1720 5.30980623891786e-06
1721 5.57918736077134e-06
1722 6.05465486458456e-06
1723 6.83674812873569e-06
1724 7.69212900553207e-06
1725 7.92042905706225e-06
1726 6.9719390635159e-06
1727 5.74027363220431e-06
1728 5.33326023155478e-06
1729 5.78333812595311e-06
1730 6.52502532516586e-06
1731 6.89577889723125e-06
1732 6.62312341104254e-06
1733 5.95539377745702e-06
1734 5.41978973567225e-06
1735 5.13543216662526e-06
1736 5.05098714698171e-06
1737 4.99703507550464e-06
1738 4.99673305798254e-06
1739 4.98392688164095e-06
1740 4.99194296166294e-06
1741 4.98152512906636e-06
1742 4.98784534186392e-06
1743 4.98099438250854e-06
1744 4.99078209847958e-06
1745 4.98066161358501e-06
1746 4.98704002249717e-06
1747 4.97671558075119e-06
1748 4.988757854818e-06
1749 4.97351953865355e-06
1750 4.98574358775983e-06
1751 4.96869807564693e-06
1752 4.9809557214342e-06
1753 4.96590641474448e-06
1754 4.96942129446509e-06
1755 4.96121881199407e-06
1756 4.9609844141596e-06
1757 4.95450386917007e-06
1758 4.96189302112704e-06
1759 4.94809403051022e-06
1760 4.95832835278875e-06
1761 4.94280119078816e-06
1762 4.94877938228555e-06
1763 4.94222549796142e-06
1764 4.94770434089631e-06
1765 4.93636139253795e-06
1766 4.94123150129866e-06
1767 4.93345971208647e-06
1768 4.94100032022438e-06
1769 4.9280918499317e-06
1770 4.94034031350843e-06
1771 4.9282338208112e-06
1772 4.9342208523484e-06
1773 4.93181472549509e-06
1774 4.94391425509555e-06
1775 4.9521268734587e-06
1776 4.99045850665958e-06
1777 5.06442822589825e-06
1778 5.23148662856521e-06
1779 5.59296300495227e-06
1780 6.33007479544645e-06
1781 7.59496698221618e-06
1782 8.86577700742741e-06
1783 8.61308128441429e-06
1784 6.66699356333122e-06
1785 5.39789743125851e-06
1786 5.62461611630649e-06
1787 6.34259058895914e-06
1788 6.62826392994198e-06
1789 6.23843351321707e-06
1790 5.69019284024641e-06
1791 5.23217077530447e-06
1792 5.05700463238057e-06
1793 4.95357025509513e-06
1794 4.94306411891188e-06
1795 4.9117199696358e-06
1796 4.92581512312462e-06
1797 4.90293795851748e-06
1798 4.92272744390121e-06
1799 4.89984582774383e-06
1800 4.90560626809078e-06
1801 4.89075051102184e-06
1802 4.9003278421722e-06
1803 4.88433560152934e-06
1804 4.88763012529425e-06
1805 4.87730035136913e-06
1806 4.88296226941465e-06
1807 4.87035523821788e-06
1808 4.87447697383736e-06
1809 4.86485773820711e-06
1810 4.87294328710419e-06
1811 4.85774778646864e-06
1812 4.86964661305933e-06
1813 4.85462744537202e-06
1814 4.86096661478364e-06
1815 4.85441576225654e-06
1816 4.85771461744555e-06
1817 4.84866372540971e-06
1818 4.85078860656962e-06
1819 4.84736574568601e-06
1820 4.85328877397606e-06
1821 4.84138750755925e-06
1822 4.85418260320891e-06
1823 4.83777625381876e-06
1824 4.85068200117844e-06
1825 4.84119891797086e-06
1826 4.84994304850161e-06
1827 4.84301085990069e-06
1828 4.85765077629097e-06
1829 4.84835460290434e-06
1830 4.86953088385533e-06
1831 4.87566930051919e-06
1832 4.92118712358547e-06
1833 4.95926071675257e-06
1834 5.08519132469409e-06
1835 5.26621241669911e-06
1836 5.61772084473056e-06
1837 6.19718162764293e-06
1838 6.88065895149137e-06
1839 7.29792239551053e-06
1840 6.95640178061119e-06
1841 6.01450192783126e-06
1842 5.23405128927124e-06
1843 5.13963032644504e-06
1844 5.6830507486616e-06
1845 6.50098513221842e-06
1846 7.04346915680887e-06
1847 6.82214241720658e-06
1848 6.00930140692846e-06
1849 5.2835462138745e-06
1850 4.93267527001962e-06
1851 4.81777038885411e-06
1852 4.79123366470269e-06
1853 4.7819057740206e-06
1854 4.77566986067934e-06
1855 4.77658716291351e-06
1856 4.77853145319784e-06
1857 4.77783497787598e-06
1858 4.78019961791532e-06
1859 4.78202353715318e-06
1860 4.78280826121669e-06
1861 4.78074003873452e-06
1862 4.78051248276046e-06
1863 4.77747423577313e-06
1864 4.77836508849805e-06
1865 4.77160214562566e-06
1866 4.77053453096232e-06
1867 4.76594096521232e-06
1868 4.76896337198163e-06
1869 4.76107822522387e-06
1870 4.76301036722582e-06
1871 4.75674324418307e-06
1872 4.7562566509729e-06
1873 4.75248169662379e-06
1874 4.75417064293993e-06
1875 4.7504063898085e-06
1876 4.75828222268326e-06
1877 4.75109487774183e-06
1878 4.76439663277262e-06
1879 4.76249347736157e-06
1880 4.77807533605379e-06
1881 4.79513639128903e-06
1882 4.83340626522555e-06
1883 4.87142216165637e-06
1884 4.967765681263e-06
1885 5.08128875009728e-06
1886 5.28107828945412e-06
1887 5.54484934256472e-06
1888 5.86345337993954e-06
1889 6.10688906976797e-06
1890 6.14425195000212e-06
1891 5.87308920596552e-06
1892 5.39102668839497e-06
1893 4.9912121051765e-06
1894 4.92463319279324e-06
1895 5.28117845899345e-06
1896 6.05225085137384e-06
1897 7.01382011492058e-06
1898 7.50459815623827e-06
1899 6.93465724910247e-06
1900 5.81147701206319e-06
1901 5.03621820335809e-06
1902 4.77711432789185e-06
1903 4.74201456523993e-06
1904 4.74905104153045e-06
1905 4.73167868419999e-06
1906 4.71754501218946e-06
1907 4.69925186208364e-06
1908 4.69576548089989e-06
1909 4.68762044558702e-06
1910 4.69104379718921e-06
1911 4.69072917042013e-06
1912 4.69513380219055e-06
1913 4.69210079678817e-06
1914 4.69713637230029e-06
1915 4.69284108373103e-06
1916 4.69568648364671e-06
1917 4.69008658132708e-06
1918 4.69603217578829e-06
1919 4.68711851486781e-06
1920 4.68843923284723e-06
1921 4.68295691113951e-06
1922 4.68397631214756e-06
1923 4.67779782020727e-06
1924 4.67833676287555e-06
1925 4.67310502716423e-06
1926 4.67946058679836e-06
1927 4.6706204226421e-06
1928 4.67714679985853e-06
1929 4.6740122912059e-06
1930 4.68986713642039e-06
1931 4.69498103283783e-06
1932 4.72524104977445e-06
1933 4.75388203624583e-06
1934 4.82158870362959e-06
1935 4.92362735793339e-06
1936 5.1067752251388e-06
1937 5.34746161406918e-06
1938 5.71001155458273e-06
1939 6.05694439481663e-06
1940 6.29348468184077e-06
1941 6.16998724645157e-06
1942 5.65896296822466e-06
1943 5.07375900338225e-06
1944 4.84872138595271e-06
1945 5.06160124347232e-06
1946 5.69463595034847e-06
1947 6.54492421348074e-06
1948 7.10843734097466e-06
1949 6.82937458074129e-06
1950 5.9036011079705e-06
1951 5.10003599685227e-06
1952 4.74588367493922e-06
1953 4.66700011769738e-06
1954 4.65916678948908e-06
1955 4.65436871754932e-06
1956 4.64244136111347e-06
1957 4.63375125714371e-06
1958 4.62343581086877e-06
1959 4.62061510697609e-06
1960 4.62071638862582e-06
1961 4.62540041512582e-06
1962 4.62445526228095e-06
1963 4.62849457005809e-06
1964 4.62619188446034e-06
1965 4.63139781881061e-06
1966 4.63012935969509e-06
1967 4.62974862824872e-06
1968 4.62675016055236e-06
1969 4.62856178296001e-06
1970 4.621506925373e-06
1971 4.62236011244954e-06
1972 4.61612401969624e-06
1973 4.61764227677186e-06
1974 4.60798786861005e-06
1975 4.61315093613912e-06
1976 4.60648964994981e-06
1977 4.60704621607633e-06
1978 4.60182044514923e-06
1979 4.6057116716014e-06
1980 4.60364487908649e-06
1981 4.61491433090266e-06
1982 4.61070172219635e-06
1983 4.63833364894306e-06
1984 4.65336719046583e-06
1985 4.70426089338449e-06
1986 4.78638231471429e-06
1987 4.94801901418374e-06
1988 5.18831509310758e-06
1989 5.60112247716127e-06
1990 6.10109260490432e-06
1991 6.55931187076852e-06
1992 6.61193876694277e-06
1993 6.01531220290052e-06
1994 5.20419284577578e-06
1995 4.80806857661165e-06
1996 4.99937309239584e-06
1997 5.63638414430301e-06
1998 6.42215522295153e-06
1999 6.83331610562554e-06
};
\addlegendentry{Train}
\addplot [semithick, black]
table {%
0 0.0104643907397985
1 0.0102256890386343
2 0.0100037269294262
3 0.00979630742222071
4 0.00960157252848148
5 0.00941805820912123
6 0.00924452114850283
7 0.00907988287508488
8 0.00892255548387766
9 0.00877136085182428
10 0.00862502586096525
11 0.008481465280056
12 0.00833634473383427
13 0.00818388536572456
14 0.00801533926278353
15 0.0078244348987937
16 0.00760523602366447
17 0.00734815979376435
18 0.00706017110496759
19 0.00675290683284402
20 0.00643459009006619
21 0.00610667932778597
22 0.00577205885201693
23 0.00543102342635393
24 0.00508642196655273
25 0.00474044494330883
26 0.00439738854765892
27 0.0040622390806675
28 0.00374160148203373
29 0.00344111933372915
30 0.00316106085665524
31 0.00290116271935403
32 0.00266249221749604
33 0.002443868201226
34 0.00224376190453768
35 0.00205644639208913
36 0.00188555161003023
37 0.00173135090153664
38 0.00159539119340479
39 0.00147375511005521
40 0.00136456394102424
41 0.00126504455693066
42 0.00116977118887007
43 0.0010863037314266
44 0.00101199490018189
45 0.00094574160175398
46 0.000887078291270882
47 0.000834705715533346
48 0.000787449651397765
49 0.000744619988836348
50 0.000705402460880578
51 0.000669868662953377
52 0.000637647463008761
53 0.000608187227044255
54 0.000581172411330044
55 0.000556373212020844
56 0.000533590035047382
57 0.000512674334459007
58 0.00049349875189364
59 0.000475687149446458
60 0.000458641996374354
61 0.000442608434241265
62 0.000427840219344944
63 0.000414299807744101
64 0.000401874538511038
65 0.000390466186217964
66 0.000379984470782802
67 0.000370373629266396
68 0.000361552578397095
69 0.000353460025507957
70 0.000345957494573668
71 0.000339021615218371
72 0.000331909803207964
73 0.000325467495713383
74 0.000319661834510043
75 0.000314484350383282
76 0.000309709459543228
77 0.000305355875752866
78 0.000301342777675018
79 0.00029760628240183
80 0.000294007186312228
81 0.00029065870330669
82 0.000287537870462984
83 0.000284634734271094
84 0.000281921675195917
85 0.000279411411611363
86 0.000277051934972405
87 0.000274858961347491
88 0.000272853678325191
89 0.000270990945864469
90 0.000269228796241805
91 0.000267563911620528
92 0.000265988783212379
93 0.000264492467977107
94 0.000263068854110315
95 0.000261677836533636
96 0.000260342581896111
97 0.000259070075117052
98 0.000257824372965842
99 0.000256641476880759
100 0.000255508755799383
101 0.000254419806879014
102 0.000253370817517862
103 0.00025234482018277
104 0.000251357501838356
105 0.000250413519097492
106 0.000249499542405829
107 0.000248612108407542
108 0.000247745803790167
109 0.000246897427132353
110 0.000246063573285937
111 0.000245252944296226
112 0.000244466529693455
113 0.000243702364969067
114 0.000242958485614508
115 0.000242228459683247
116 0.000241525645833462
117 0.000240829802351072
118 0.00024015101371333
119 0.000239491651882418
120 0.000238842068938538
121 0.000238214342971332
122 0.000237318658037111
123 0.000236347594182007
124 0.000235500046983361
125 0.000234736769925803
126 0.000234031453146599
127 0.000233357495744713
128 0.000232709586271085
129 0.000232081569265574
130 0.000231454221648164
131 0.000230854886467569
132 0.000230269419262186
133 0.000229691810091026
134 0.000229135272093117
135 0.000228580567636527
136 0.000228049990255386
137 0.000227515120059252
138 0.000227015305426903
139 0.000226519536226988
140 0.000226039264816791
141 0.000225560972467065
142 0.000225088631850667
143 0.000224628340220079
144 0.000224173010792583
145 0.000223722410737537
146 0.000223284761887044
147 0.000222840782953426
148 0.000222404807573184
149 0.000221964015509002
150 0.000221540991333313
151 0.00022112624719739
152 0.000220715824980289
153 0.000220298374188133
154 0.000219868277781643
155 0.000219445253605954
156 0.000218961882637814
157 0.000218422763282433
158 0.00021794275380671
159 0.000217492095544003
160 0.000217050910578109
161 0.000216616099351086
162 0.000216189568163827
163 0.000215751788346097
164 0.000215309133636765
165 0.000214877334656194
166 0.000214456129469909
167 0.000214081373997033
168 0.000213719045859762
169 0.000213335602893494
170 0.000212981307413429
171 0.000212604514672421
172 0.000212250160984695
173 0.000211885402677581
174 0.000211523176403716
175 0.000211162463529035
176 0.00021080588339828
177 0.000210449928999878
178 0.000210088444873691
179 0.000209736099350266
180 0.000209383433684707
181 0.000209036967135035
182 0.000208681012736633
183 0.000208320037927479
184 0.000207974095246755
185 0.000207621837034822
186 0.000207271325052716
187 0.000206922675715759
188 0.000206576820346527
189 0.000206229495233856
190 0.000205880642170087
191 0.000205535383429378
192 0.000205184318474494
193 0.000204848474822938
194 0.000204495183425024
195 0.000204155483515933
196 0.000203805087949149
197 0.000203465067897923
198 0.000203123752726242
199 0.000202776849619113
200 0.000202438968699425
201 0.000202098322915845
202 0.000201757255126722
203 0.000201410497538745
204 0.000201067174202763
205 0.000200719892745838
206 0.000200378286535852
207 0.000200026595848612
208 0.000199691843590699
209 0.000199339643586427
210 0.000198628185898997
211 0.0001982403191505
212 0.000197882473003119
213 0.000197515080799349
214 0.000197144385310821
215 0.000196773340576328
216 0.000196397319086827
217 0.000196029475773685
218 0.000195658329175785
219 0.000195295899175107
220 0.000194934647879563
221 0.000194568507140502
222 0.000194198117242195
223 0.000193822779692709
224 0.000193442858289927
225 0.000193073225091211
226 0.00019269397307653
227 0.000192312872968614
228 0.000191924948012456
229 0.000191540049854666
230 0.000191151062608697
231 0.000190761376870796
232 0.000190371458302252
233 0.000189969417988323
234 0.000189568279893138
235 0.000189165570191108
236 0.000188757141586393
237 0.000188341233297251
238 0.000187929123057984
239 0.000187523997738026
240 0.000187110825208947
241 0.000186701145139523
242 0.000186283054063097
243 0.000185799188329838
244 0.000185361684998497
245 0.00018491770606488
246 0.000184467091457918
247 0.000184017684659921
248 0.000183567564818077
249 0.000183113195816986
250 0.000182576157385483
251 0.000181999683263712
252 0.00018142418412026
253 0.000180868984898552
254 0.000180353759787977
255 0.00017983496945817
256 0.00017932451737579
257 0.000178799906279892
258 0.000178257250809111
259 0.000177716035977937
260 0.000177165638888255
261 0.00017661954916548
262 0.000176044268300757
263 0.000175497800228186
264 0.000174916785908863
265 0.000174353903275914
266 0.000173758860910311
267 0.000173167340108193
268 0.000172560059581883
269 0.000171932973898947
270 0.000171316001797095
271 0.00017069537716452
272 0.000170041268575005
273 0.000169402599567547
274 0.000168716360349208
275 0.000168064463650808
276 0.000167378078913316
277 0.000166706100571901
278 0.000165995137649588
279 0.000165298231877387
280 0.000164486395078711
281 0.000163761345902458
282 0.000163010045071132
283 0.000162221811478958
284 0.000161451360327192
285 0.00016064198280219
286 0.000159840958076529
287 0.000159048708155751
288 0.000158209892106242
289 0.000157374190166593
290 0.000156536072609015
291 0.000155654284753837
292 0.000154780253069475
293 0.000153863162267953
294 0.000152964901644737
295 0.000152008549775928
296 0.000151069500134327
297 0.000150147956446745
298 0.000149210085510276
299 0.000148208680911921
300 0.00014718713646289
301 0.000146141363075003
302 0.00014506935258396
303 0.000143964047310874
304 0.000142854347359389
305 0.000141744836582802
306 0.000140626521897502
307 0.00013949784624856
308 0.000138284536660649
309 0.000137124268803746
310 0.000135939219035208
311 0.000134701680508442
312 0.000133492023451254
313 0.000132294051581994
314 0.00013105473772157
315 0.000129832362290472
316 0.000128567087813281
317 0.000127323932247236
318 0.000126087514217943
319 0.000124806581879966
320 0.000123561956570484
321 0.000122257581097074
322 0.000121003577078227
323 0.000119699594506528
324 0.000118454787298106
325 0.00011717586312443
326 0.000115884613478556
327 0.000114630413008854
328 0.000113342583063059
329 0.000111973116872832
330 0.0001107087591663
331 0.000109405780676752
332 0.000108157582872082
333 0.000106810162833426
334 0.000105393002741039
335 0.000103992337244563
336 0.000102579397207592
337 0.00010100485815201
338 9.93571593426168e-05
339 9.7748008556664e-05
340 9.62924241321161e-05
341 9.4850744062569e-05
342 9.34686177060939e-05
343 9.20806778594851e-05
344 9.07474313862622e-05
345 8.93891992745921e-05
346 8.81035084603354e-05
347 8.67928611114621e-05
348 8.55526668601669e-05
349 8.42475128592923e-05
350 8.3046659710817e-05
351 8.18145927041769e-05
352 8.0661935498938e-05
353 7.94847728684545e-05
354 7.8381817729678e-05
355 7.7287812018767e-05
356 7.62619310989976e-05
357 7.51908664824441e-05
358 7.41888070479035e-05
359 7.31765976524912e-05
360 7.22149925422855e-05
361 7.12576875230297e-05
362 7.03254627296701e-05
363 6.93961846991442e-05
364 6.8443019699771e-05
365 6.75625997246243e-05
366 6.66667619952932e-05
367 6.58322824165225e-05
368 6.49755893391557e-05
369 6.41491787973791e-05
370 6.32853843853809e-05
371 6.24310341663659e-05
372 6.16663455730304e-05
373 6.08850350545254e-05
374 6.00934981775936e-05
375 5.93897093494888e-05
376 5.86688620387577e-05
377 5.79176885366905e-05
378 5.72607859794516e-05
379 5.65706613997463e-05
380 5.58982792426832e-05
381 5.52314522792585e-05
382 5.46490191482008e-05
383 5.40223336429335e-05
384 5.33881757291965e-05
385 5.28139207744971e-05
386 5.22697555425111e-05
387 5.17037624376826e-05
388 5.11368489242159e-05
389 5.06033393321559e-05
390 5.00668975291774e-05
391 4.9547878006706e-05
392 4.90641905344091e-05
393 4.85911368741654e-05
394 4.81618008052465e-05
395 4.7692097723484e-05
396 4.7214976802934e-05
397 4.67685204057489e-05
398 4.63345131720416e-05
399 4.59030670754146e-05
400 4.54914261354133e-05
401 4.50864718004595e-05
402 4.46879857918248e-05
403 4.4310338125797e-05
404 4.39483774243854e-05
405 4.35868205386214e-05
406 4.32316919614095e-05
407 4.28859784733504e-05
408 4.25615726271644e-05
409 4.22614430135582e-05
410 4.19598618464079e-05
411 4.16648254031315e-05
412 4.1380146285519e-05
413 4.10965658375062e-05
414 4.08143932872918e-05
415 4.05436221626587e-05
416 4.02711339120287e-05
417 4.00188146159053e-05
418 3.97564654122107e-05
419 3.95038878195919e-05
420 3.92506044590846e-05
421 3.90088061976712e-05
422 3.87658692488912e-05
423 3.8536458305316e-05
424 3.83050864911638e-05
425 3.80741184926592e-05
426 3.78545410057995e-05
427 3.76376738131512e-05
428 3.7422822060762e-05
429 3.71971254935488e-05
430 3.69920308003202e-05
431 3.67765132978093e-05
432 3.65746709576342e-05
433 3.63652434316464e-05
434 3.61607708327938e-05
435 3.59524419764057e-05
436 3.5740333260037e-05
437 3.55495103576686e-05
438 3.53450814145617e-05
439 3.51565904566087e-05
440 3.49687979905866e-05
441 3.47867971868254e-05
442 3.46022898156662e-05
443 3.44284599123057e-05
444 3.42562707373872e-05
445 3.40922779287212e-05
446 3.39235375577118e-05
447 3.37643323291559e-05
448 3.35887125402223e-05
449 3.34385622409172e-05
450 3.32656854880042e-05
451 3.31091396219563e-05
452 3.29473077727016e-05
453 3.27900343108922e-05
454 3.26529479934834e-05
455 3.24946449836716e-05
456 3.23483982356265e-05
457 3.21922925650142e-05
458 3.20721555908676e-05
459 3.19210812449455e-05
460 3.17808917316142e-05
461 3.16416189889424e-05
462 3.15062097797636e-05
463 3.13625969283748e-05
464 3.12328702420928e-05
465 3.10943760268856e-05
466 3.09906135953497e-05
467 3.08518210658804e-05
468 3.06340589304455e-05
469 3.04140321532032e-05
470 3.02503249258734e-05
471 3.01076343021123e-05
472 2.99378443742171e-05
473 2.98103204841027e-05
474 2.96924699796364e-05
475 2.95614645438036e-05
476 2.94345791189699e-05
477 2.93190150841838e-05
478 2.91757369268453e-05
479 2.90145690087229e-05
480 2.88641658698907e-05
481 2.87349757854827e-05
482 2.86131798930001e-05
483 2.84990219370229e-05
484 2.83894914900884e-05
485 2.82874152617296e-05
486 2.81820048257941e-05
487 2.80788717645919e-05
488 2.79654232144821e-05
489 2.78697989415377e-05
490 2.77635572274448e-05
491 2.766945908661e-05
492 2.757586480584e-05
493 2.74928206636105e-05
494 2.74074336630292e-05
495 2.73079986072844e-05
496 2.72198722086614e-05
497 2.7143418265041e-05
498 2.70565651590005e-05
499 2.69826814474072e-05
500 2.68941566901049e-05
501 2.68219282588689e-05
502 2.67528412223328e-05
503 2.66651913989335e-05
504 2.66040915448684e-05
505 2.65237213170622e-05
506 2.64636328211054e-05
507 2.6388725018478e-05
508 2.63308065768797e-05
509 2.62707944784779e-05
510 2.61959721683525e-05
511 2.61354743997799e-05
512 2.60800770774949e-05
513 2.60088072536746e-05
514 2.59498356172116e-05
515 2.5872357582557e-05
516 2.58203472185414e-05
517 2.57347273873165e-05
518 2.56719104072545e-05
519 2.56087751040468e-05
520 2.55364357144572e-05
521 2.54811693594093e-05
522 2.54096412390936e-05
523 2.5311541321571e-05
524 2.5263130737585e-05
525 2.52006102527957e-05
526 2.51415694947354e-05
527 2.50734265136998e-05
528 2.50222146860324e-05
529 2.49330169026507e-05
530 2.48685664701043e-05
531 2.47648822551128e-05
532 2.47095686063403e-05
533 2.46398030867567e-05
534 2.46010931732599e-05
535 2.46025701926555e-05
536 2.47964144364232e-05
537 2.52028512477409e-05
538 2.58153559116181e-05
539 2.5888066375046e-05
540 2.51795918302378e-05
541 2.4355127607123e-05
542 2.41737798205577e-05
543 2.42752521444345e-05
544 2.4355749701499e-05
545 2.43104368564673e-05
546 2.4131446480169e-05
547 2.39657565543894e-05
548 2.38199336308753e-05
549 2.36873911489965e-05
550 2.36176565522328e-05
551 2.35605857596966e-05
552 2.35147963394411e-05
553 2.34444851230364e-05
554 2.34154213103466e-05
555 2.33257469517412e-05
556 2.33043847401859e-05
557 2.32389393204357e-05
558 2.31930298468797e-05
559 2.31251015065936e-05
560 2.30881360039348e-05
561 2.3014979888103e-05
562 2.29749857680872e-05
563 2.29257839237107e-05
564 2.28532244364033e-05
565 2.28225217142608e-05
566 2.27133004955249e-05
567 2.26948795898352e-05
568 2.26016327360412e-05
569 2.25713793042814e-05
570 2.25012008741032e-05
571 2.2478698156192e-05
572 2.24081068154192e-05
573 2.23884271690622e-05
574 2.2307913241093e-05
575 2.2297333998722e-05
576 2.2257014279603e-05
577 2.21960981434677e-05
578 2.21551363210892e-05
579 2.20839083340252e-05
580 2.20477613765979e-05
581 2.19829198613297e-05
582 2.19339817704167e-05
583 2.1851243218407e-05
584 2.18192071770318e-05
585 2.17374508793e-05
586 2.1698037016904e-05
587 2.16306161746616e-05
588 2.15900781768141e-05
589 2.15399650187464e-05
590 2.15224627027055e-05
591 2.14664396480657e-05
592 2.14650517591508e-05
593 2.14250285353046e-05
594 2.14175124710891e-05
595 2.13878738577478e-05
596 2.1369123714976e-05
597 2.13465918932343e-05
598 2.13286421058001e-05
599 2.12569411814911e-05
600 2.11944970942568e-05
601 2.10974922083551e-05
602 2.10113448702032e-05
603 2.0915649656672e-05
604 2.08856181416195e-05
605 2.10363887163112e-05
606 2.15191976167262e-05
607 2.25586009037215e-05
608 2.37159329117276e-05
609 2.33824630413437e-05
610 2.16006810660474e-05
611 2.06505119422218e-05
612 2.06968616112135e-05
613 2.08174278668594e-05
614 2.07215271075256e-05
615 2.0593668523361e-05
616 2.04638563445769e-05
617 2.03806575882481e-05
618 2.02706542040687e-05
619 2.02351602638373e-05
620 2.01805796677945e-05
621 2.01618076971499e-05
622 2.01168932107976e-05
623 2.0087541997782e-05
624 2.00252525246469e-05
625 2.00205231521977e-05
626 1.99693840841064e-05
627 1.9937162505812e-05
628 1.99015448743012e-05
629 1.9867678929586e-05
630 1.98102952708723e-05
631 1.98036977963056e-05
632 1.97579192899866e-05
633 1.97358713194262e-05
634 1.97004301298875e-05
635 1.96747296286048e-05
636 1.96227065316634e-05
637 1.96721557585988e-05
638 1.96159544429975e-05
639 1.95578159036813e-05
640 1.95131833606865e-05
641 1.94989534065826e-05
642 1.94447620742721e-05
643 1.94418698811205e-05
644 1.93941523320973e-05
645 1.9372573660803e-05
646 1.93342129932716e-05
647 1.93135347217321e-05
648 1.92614061234053e-05
649 1.92547959159128e-05
650 1.92116694961442e-05
651 1.9178462025593e-05
652 1.91391100088367e-05
653 1.91101789823733e-05
654 1.90693990589352e-05
655 1.90725313586881e-05
656 1.90287773875752e-05
657 1.89806996786501e-05
658 1.89159000001382e-05
659 1.89134370884858e-05
660 1.88501180673484e-05
661 1.88167123269523e-05
662 1.87948971870355e-05
663 1.87385739991441e-05
664 1.86827801371692e-05
665 1.86501983989729e-05
666 1.86707675311482e-05
667 1.86940942512592e-05
668 1.88163630809868e-05
669 1.90754890354583e-05
670 1.95070351765025e-05
671 2.00350041268393e-05
672 2.02704341063509e-05
673 1.98273828573292e-05
674 1.8998061932507e-05
675 1.85141889232909e-05
676 1.85983772098552e-05
677 1.90523060155101e-05
678 1.96843648154754e-05
679 2.00219910766464e-05
680 1.97210029000416e-05
681 1.89678248716518e-05
682 1.84100827027578e-05
683 1.81407140189549e-05
684 1.805808278732e-05
685 1.80114857357694e-05
686 1.79812213900732e-05
687 1.79546914296225e-05
688 1.79083872353658e-05
689 1.79096296051284e-05
690 1.7872363969218e-05
691 1.78395348484628e-05
692 1.78310910996515e-05
693 1.77868423634209e-05
694 1.77640376932686e-05
695 1.7721777112456e-05
696 1.77515557879815e-05
697 1.77014735527337e-05
698 1.76392040884821e-05
699 1.759526458045e-05
700 1.75322966242675e-05
701 1.75015557033475e-05
702 1.74686847458361e-05
703 1.74362066900358e-05
704 1.7378448319505e-05
705 1.73715798155172e-05
706 1.73175612872001e-05
707 1.7291971744271e-05
708 1.7261627363041e-05
709 1.72321178979473e-05
710 1.71942047018092e-05
711 1.72378968272824e-05
712 1.71909105119994e-05
713 1.71391875483096e-05
714 1.71494084497681e-05
715 1.70869934663642e-05
716 1.70571001945063e-05
717 1.70472740137484e-05
718 1.70036491908832e-05
719 1.70014136529062e-05
720 1.70038365467917e-05
721 1.69454560818849e-05
722 1.69639679370448e-05
723 1.6906209566514e-05
724 1.68684073287295e-05
725 1.68816059158416e-05
726 1.68308633874403e-05
727 1.67345951922471e-05
728 1.6689627955202e-05
729 1.66029531101231e-05
730 1.65424571605399e-05
731 1.64860011864221e-05
732 1.64614102686755e-05
733 1.64915472851135e-05
734 1.66361951414729e-05
735 1.69909835676663e-05
736 1.75473469425924e-05
737 1.81096947926562e-05
738 1.81434443220496e-05
739 1.74511769728269e-05
740 1.66514728334732e-05
741 1.63745153258787e-05
742 1.65295004990185e-05
743 1.69160757650388e-05
744 1.7227050193469e-05
745 1.71357696672203e-05
746 1.67545967997285e-05
747 1.63430158863775e-05
748 1.60615036293166e-05
749 1.59206556418212e-05
750 1.58839811774669e-05
751 1.58875609486131e-05
752 1.58629645738984e-05
753 1.58544025907759e-05
754 1.58220591401914e-05
755 1.57769627548987e-05
756 1.57172471517697e-05
757 1.56983642227715e-05
758 1.56500700541073e-05
759 1.56538371811621e-05
760 1.56599762703991e-05
761 1.56872902152827e-05
762 1.56494352268055e-05
763 1.57206522999331e-05
764 1.56601454364136e-05
765 1.56463775056181e-05
766 1.55675461428473e-05
767 1.54798617586493e-05
768 1.54610515892273e-05
769 1.54417411977192e-05
770 1.53856999531854e-05
771 1.5319428712246e-05
772 1.5315034033847e-05
773 1.52162820086232e-05
774 1.52085240188171e-05
775 1.51614576680004e-05
776 1.51410067701363e-05
777 1.51668191392673e-05
778 1.51524300235906e-05
779 1.51165986608248e-05
780 1.51145022755372e-05
781 1.5097897630767e-05
782 1.50759724419913e-05
783 1.50158939504763e-05
784 1.49803190652165e-05
785 1.48849858305766e-05
786 1.48544504554593e-05
787 1.48914705278003e-05
788 1.50761388795217e-05
789 1.54930639837403e-05
790 1.64295015565585e-05
791 1.77532601810526e-05
792 1.82058138307184e-05
793 1.69631130120251e-05
794 1.52144993990078e-05
795 1.46302636494511e-05
796 1.47605887832469e-05
797 1.4910989193595e-05
798 1.49158558997442e-05
799 1.4809441381658e-05
800 1.46498796311789e-05
801 1.45403146234457e-05
802 1.44300174724776e-05
803 1.43866545840865e-05
804 1.44016976264538e-05
805 1.44207579069189e-05
806 1.43781480801408e-05
807 1.43442357511958e-05
808 1.43332272273256e-05
809 1.43059432957671e-05
810 1.42480157592217e-05
811 1.42505496114609e-05
812 1.42754161061021e-05
813 1.42349153975374e-05
814 1.41573336804868e-05
815 1.41497612275998e-05
816 1.41427926791948e-05
817 1.40931006171741e-05
818 1.40410147650982e-05
819 1.40771280712215e-05
820 1.40425290737767e-05
821 1.39880894494127e-05
822 1.39408321047085e-05
823 1.39397316161194e-05
824 1.39186204251018e-05
825 1.39075573315495e-05
826 1.38957884701085e-05
827 1.38622754093376e-05
828 1.38248751682113e-05
829 1.38108753162669e-05
830 1.37739925776259e-05
831 1.37602755785338e-05
832 1.37758488563122e-05
833 1.3772354577668e-05
834 1.37307852128288e-05
835 1.37017932502204e-05
836 1.36873222800205e-05
837 1.37018287205137e-05
838 1.36720636874088e-05
839 1.35455011331942e-05
840 1.35217160277534e-05
841 1.34389028971782e-05
842 1.34477431856794e-05
843 1.33872517835698e-05
844 1.33263838506537e-05
845 1.32905597638455e-05
846 1.32686436700169e-05
847 1.32137984110159e-05
848 1.32603399833897e-05
849 1.33982948682387e-05
850 1.36567168738111e-05
851 1.4079990251048e-05
852 1.47182790897205e-05
853 1.51507465488976e-05
854 1.48959143189131e-05
855 1.4064677998249e-05
856 1.33762032419327e-05
857 1.32142777147237e-05
858 1.3582119208877e-05
859 1.4219793229131e-05
860 1.47968157762079e-05
861 1.47799282785854e-05
862 1.41613618325209e-05
863 1.34703232106403e-05
864 1.30299786178512e-05
865 1.28400624817004e-05
866 1.28068086269195e-05
867 1.27665043692105e-05
868 1.27816465465003e-05
869 1.27520142996218e-05
870 1.27025423353189e-05
871 1.26693994388916e-05
872 1.26678378364886e-05
873 1.26721197375446e-05
874 1.26548820844619e-05
875 1.26317945614574e-05
876 1.26094500956242e-05
877 1.26041768453433e-05
878 1.25447177197202e-05
879 1.25361511891242e-05
880 1.2523501936812e-05
881 1.24777834571432e-05
882 1.24564558063867e-05
883 1.24271837194101e-05
884 1.23921990962117e-05
885 1.23544168673106e-05
886 1.23316422104836e-05
887 1.23160543807899e-05
888 1.22585670396802e-05
889 1.22875144370482e-05
890 1.22382989502512e-05
891 1.2249348401383e-05
892 1.22285746328998e-05
893 1.22391666081967e-05
894 1.22221481433371e-05
895 1.21386538012302e-05
896 1.21496359497542e-05
897 1.21230077638756e-05
898 1.21067851068801e-05
899 1.21197917906102e-05
900 1.21108796520275e-05
901 1.21330012916587e-05
902 1.21613265946507e-05
903 1.22313394967932e-05
904 1.23096506285947e-05
905 1.24232747111819e-05
906 1.25170508908923e-05
907 1.25231945276028e-05
908 1.25100250443211e-05
909 1.23464742500801e-05
910 1.21257035061717e-05
911 1.189786780742e-05
912 1.17398449219763e-05
913 1.17798990686424e-05
914 1.20976756079472e-05
915 1.27600787891424e-05
916 1.35855807457119e-05
917 1.3952907465864e-05
918 1.33246003315435e-05
919 1.2296920431254e-05
920 1.17651570690214e-05
921 1.16963410619064e-05
922 1.18344587463071e-05
923 1.19877895485843e-05
924 1.20020895337802e-05
925 1.19652613648213e-05
926 1.18080042739166e-05
927 1.16890832941863e-05
928 1.15731827463605e-05
929 1.15112525236327e-05
930 1.1452432772785e-05
931 1.14302574729663e-05
932 1.14085951281595e-05
933 1.13751393655548e-05
934 1.13546921056695e-05
935 1.13495752884774e-05
936 1.13213254735456e-05
937 1.13074138425873e-05
938 1.13373189378763e-05
939 1.12973029899877e-05
940 1.12946245280909e-05
941 1.12746593003976e-05
942 1.13126652649953e-05
943 1.12873240141198e-05
944 1.13248215711792e-05
945 1.12998877739301e-05
946 1.13322521428927e-05
947 1.13601636257954e-05
948 1.13545256681391e-05
949 1.13988307930413e-05
950 1.13994046841981e-05
951 1.13625446829246e-05
952 1.12875231934595e-05
953 1.12704110506456e-05
954 1.1193347745575e-05
955 1.11113204184221e-05
956 1.10300152300624e-05
957 1.09906168290763e-05
958 1.0997369827237e-05
959 1.10603796201758e-05
960 1.12298403109889e-05
961 1.16298242573976e-05
962 1.21793418657035e-05
963 1.26243658087333e-05
964 1.25846027003718e-05
965 1.20104186862591e-05
966 1.12882717075991e-05
967 1.09537559183082e-05
968 1.10857927211327e-05
969 1.15012253445457e-05
970 1.20045588118955e-05
971 1.23071849884582e-05
972 1.2210608474561e-05
973 1.17505178423016e-05
974 1.12413772512809e-05
975 1.09048169179005e-05
976 1.07832283902098e-05
977 1.07007990663988e-05
978 1.07466812551138e-05
979 1.07695159385912e-05
980 1.07274354377296e-05
981 1.06635689007817e-05
982 1.06557854451239e-05
983 1.06435818452155e-05
984 1.0636514161888e-05
985 1.06345187305124e-05
986 1.06370807770872e-05
987 1.06414972833591e-05
988 1.06483248600853e-05
989 1.06573479570216e-05
990 1.06377465272089e-05
991 1.0667730748537e-05
992 1.06397192212171e-05
993 1.06383249658393e-05
994 1.0605916941131e-05
995 1.05881235867855e-05
996 1.05628960227477e-05
997 1.04869286587927e-05
998 1.05122589957318e-05
999 1.05201434053015e-05
1000 1.04626697066124e-05
1001 1.04281243693549e-05
1002 1.04202972579515e-05
1003 1.04622549770284e-05
1004 1.05170274764532e-05
1005 1.06417119241087e-05
1006 1.07823980215471e-05
1007 1.1021074897144e-05
1008 1.12249863377656e-05
1009 1.127907398768e-05
1010 1.11290228232974e-05
1011 1.08485737655428e-05
1012 1.05152712421841e-05
1013 1.03760239653639e-05
1014 1.04711962194415e-05
1015 1.09367847471731e-05
1016 1.17577337732655e-05
1017 1.27062112369458e-05
1018 1.30838925542776e-05
1019 1.22638339234982e-05
1020 1.10175742520369e-05
1021 1.03635520645184e-05
1022 1.02750327641843e-05
1023 1.02667127066525e-05
1024 1.02621461337549e-05
1025 1.02672765933676e-05
1026 1.02695221357862e-05
1027 1.01821124189883e-05
1028 1.0136605851585e-05
1029 1.01000514405314e-05
1030 1.00961042335257e-05
1031 1.00862725957995e-05
1032 1.00984243545099e-05
1033 1.0101595762535e-05
1034 1.00855049822712e-05
1035 1.00939823823865e-05
1036 1.00953902801848e-05
1037 1.00841398307239e-05
1038 1.00477391242748e-05
1039 1.00206862043706e-05
1040 1.00169127108529e-05
1041 1.00013585324632e-05
1042 9.9901226349175e-06
1043 9.96619655779796e-06
1044 9.92762579699047e-06
1045 9.92565765045583e-06
1046 9.92131663224427e-06
1047 9.91667275229702e-06
1048 9.87493604043266e-06
1049 9.86281156656332e-06
1050 9.90996522887144e-06
1051 9.93421508610481e-06
1052 9.89257841865765e-06
1053 9.8572081697057e-06
1054 9.84578855423024e-06
1055 9.83657355391188e-06
1056 9.8278142104391e-06
1057 9.81488028628519e-06
1058 9.78098796622362e-06
1059 9.77437503024703e-06
1060 9.76234878180549e-06
1061 9.75297552940901e-06
1062 9.74820522969821e-06
1063 9.79503965936601e-06
1064 9.98287578113377e-06
1065 1.04579185062903e-05
1066 1.15335205919109e-05
1067 1.34811680254643e-05
1068 1.53149230754934e-05
1069 1.40266538437572e-05
1070 1.08387948785094e-05
1071 9.88077772490215e-06
1072 1.0264256161463e-05
1073 1.05094250102411e-05
1074 1.0418661076983e-05
1075 1.00647030194523e-05
1076 9.78624939307338e-06
1077 9.65510753303533e-06
1078 9.57325482886517e-06
1079 9.54948518483434e-06
1080 9.54781626205659e-06
1081 9.54287952481536e-06
1082 9.53384096646914e-06
1083 9.52002574194921e-06
1084 9.51875244936673e-06
1085 9.51023957895814e-06
1086 9.51082620304078e-06
1087 9.48881370277377e-06
1088 9.47969328990439e-06
1089 9.47575881582452e-06
1090 9.48040906223468e-06
1091 9.47230546444189e-06
1092 9.46214458963368e-06
1093 9.44935618463205e-06
1094 9.43713530432433e-06
1095 9.44262683333363e-06
1096 9.42826954997145e-06
1097 9.41810503718443e-06
1098 9.40440804697573e-06
1099 9.40564132179134e-06
1100 9.40741392696509e-06
1101 9.39735127758468e-06
1102 9.38795437832596e-06
1103 9.35908246901818e-06
1104 9.35869047680171e-06
1105 9.37246477406006e-06
1106 9.36170181375928e-06
1107 9.34911986405496e-06
1108 9.34147738007596e-06
1109 9.31301838136278e-06
1110 9.33918181544868e-06
1111 9.33793853619136e-06
1112 9.32343027670868e-06
1113 9.33023238758324e-06
1114 9.317903277406e-06
1115 9.32791681407252e-06
1116 9.35009848035406e-06
1117 9.36141987040173e-06
1118 9.40424524742411e-06
1119 9.43471604841761e-06
1120 9.52681602939265e-06
1121 9.65692288446007e-06
1122 9.84440157481004e-06
1123 1.01141595223453e-05
1124 1.04226746771019e-05
1125 1.06473808045848e-05
1126 1.06235575003666e-05
1127 1.02391259133583e-05
1128 9.69375741988188e-06
1129 9.27979817788582e-06
1130 9.32670809561387e-06
1131 1.00027045846218e-05
1132 1.12524130599923e-05
1133 1.22404353533057e-05
1134 1.17421805043705e-05
1135 1.0380604180682e-05
1136 9.46457748796092e-06
1137 9.14616157388082e-06
1138 9.10432936507277e-06
1139 9.12975065148203e-06
1140 9.14025986276101e-06
1141 9.10906055651139e-06
1142 9.08574929781025e-06
1143 9.08137826627353e-06
1144 9.07944468053756e-06
1145 9.05183424038114e-06
1146 9.04525404621381e-06
1147 9.03956242837012e-06
1148 9.0178045866196e-06
1149 9.03330328583252e-06
1150 9.03116779227275e-06
1151 9.01059866009746e-06
1152 9.01237672223942e-06
1153 8.99084079719614e-06
1154 8.985638487502e-06
1155 8.99933729670011e-06
1156 8.98267899174243e-06
1157 8.97187783266418e-06
1158 8.96438814379508e-06
1159 8.93596825335408e-06
1160 8.94863751454977e-06
1161 8.95826633495744e-06
1162 8.9785062300507e-06
1163 8.95382891030749e-06
1164 8.93588639883092e-06
1165 8.92409207153833e-06
1166 8.93398282642011e-06
1167 8.93599644768983e-06
1168 8.92832213139627e-06
1169 8.91519630386028e-06
1170 8.91105173650431e-06
1171 8.93277683644556e-06
1172 8.95120592758758e-06
1173 8.97638165042736e-06
1174 9.01993371371645e-06
1175 9.07698722585337e-06
1176 9.20974252949236e-06
1177 9.39526853471762e-06
1178 9.64038008532953e-06
1179 9.97157349047484e-06
1180 1.02694166344008e-05
1181 1.04093678601203e-05
1182 1.00764818853349e-05
1183 9.45494866755325e-06
1184 8.96933579497272e-06
1185 8.92863954504719e-06
1186 9.42413589655189e-06
1187 1.0438523531775e-05
1188 1.14469939944684e-05
1189 1.14941567517235e-05
1190 1.0228730388917e-05
1191 9.1452802735148e-06
1192 8.77444381330861e-06
1193 8.69890300236875e-06
1194 8.7282196545857e-06
1195 8.73969474923797e-06
1196 8.75168279890204e-06
1197 8.71629799803486e-06
1198 8.69936320668785e-06
1199 8.68399001774378e-06
1200 8.67608923726948e-06
1201 8.68238748807926e-06
1202 8.6805639512022e-06
1203 8.6670725067961e-06
1204 8.67301605467219e-06
1205 8.65588208398549e-06
1206 8.64806042955024e-06
1207 8.66987193148816e-06
1208 8.648451512272e-06
1209 8.63912009663181e-06
1210 8.6334184743464e-06
1211 8.61349781189347e-06
1212 8.61205990076996e-06
1213 8.60044383443892e-06
1214 8.61287117004395e-06
1215 8.62348497321364e-06
1216 8.60681575431954e-06
1217 8.60255113366293e-06
1218 8.62081378727453e-06
1219 8.59356987348292e-06
1220 8.61314947542269e-06
1221 8.6038180597825e-06
1222 8.59239571582293e-06
1223 8.60221553011797e-06
1224 8.59750525705749e-06
1225 8.62324395711767e-06
1226 8.64289631863357e-06
1227 8.66164191393182e-06
1228 8.69998893904267e-06
1229 8.79324943525717e-06
1230 8.92923253559275e-06
1231 9.11082679522224e-06
1232 9.37971162784379e-06
1233 9.69020129559794e-06
1234 1.0002908311435e-05
1235 1.01265168268583e-05
1236 9.77916806732537e-06
1237 9.13710846361937e-06
1238 8.63845707499422e-06
1239 8.66472237248672e-06
1240 9.24754567677155e-06
1241 1.030865314533e-05
1242 1.12122907012235e-05
1243 1.09886714199092e-05
1244 9.77470335783437e-06
1245 8.85351619217545e-06
1246 8.60366890265141e-06
1247 8.50289325171616e-06
1248 8.48541276354808e-06
1249 8.48744457471184e-06
1250 8.46738748805365e-06
1251 8.46696184453322e-06
1252 8.45348677103175e-06
1253 8.44128680910217e-06
1254 8.42352255858714e-06
1255 8.42230292619206e-06
1256 8.41527526063146e-06
1257 8.40473694552202e-06
1258 8.43571524455911e-06
1259 8.42610006657196e-06
1260 8.40395478007849e-06
1261 8.41217115521431e-06
1262 8.38544656289741e-06
1263 8.38716459838906e-06
1264 8.38089090393623e-06
1265 8.3691784311668e-06
1266 8.39567201182945e-06
1267 8.3772292782669e-06
1268 8.36357776279328e-06
1269 8.3587647168315e-06
1270 8.34369529911783e-06
1271 8.34406182548264e-06
1272 8.35858554637525e-06
1273 8.35883747640764e-06
1274 8.35011633171234e-06
1275 8.34903858049074e-06
1276 8.32903515401995e-06
1277 8.33955982670886e-06
1278 8.34319416753715e-06
1279 8.357171282114e-06
1280 8.40444590721745e-06
1281 8.45292743179016e-06
1282 8.47609135234961e-06
1283 8.56939095683629e-06
1284 8.67790549818892e-06
1285 8.93494325282518e-06
1286 9.26870143302949e-06
1287 9.71714598563267e-06
1288 1.01330278994283e-05
1289 1.01558052847395e-05
1290 9.58915461524157e-06
1291 8.81379855854902e-06
1292 8.39212134451373e-06
1293 8.59534247865668e-06
1294 9.38272114581196e-06
1295 1.04380205812049e-05
1296 1.08346221168176e-05
1297 9.61016030487372e-06
1298 8.26494397188071e-06
1299 8.09281482361257e-06
1300 8.18477110442473e-06
1301 8.38284177007154e-06
1302 8.44794521981385e-06
1303 8.40476786834188e-06
1304 8.31784927868284e-06
1305 8.22703259473201e-06
1306 8.17237923911307e-06
1307 8.10972323961323e-06
1308 8.06240223027999e-06
1309 8.04894534667255e-06
1310 8.04906903795199e-06
1311 8.05058243713574e-06
1312 8.04169485491002e-06
1313 8.02695467427839e-06
1314 8.04044429969508e-06
1315 8.02861541160382e-06
1316 8.03188049758319e-06
1317 8.05161835160106e-06
1318 8.0387117122882e-06
1319 8.04997580416966e-06
1320 8.03324837761465e-06
1321 8.04906358098378e-06
1322 8.04119827080285e-06
1323 8.04774117568741e-06
1324 8.05160561867524e-06
1325 8.04161663836567e-06
1326 8.0488662206335e-06
1327 8.09153152658837e-06
1328 8.09172070148634e-06
1329 8.06709249445703e-06
1330 8.06125262897694e-06
1331 8.07223477750085e-06
1332 8.05695162853226e-06
1333 8.07209653430618e-06
1334 8.07986816653283e-06
1335 8.08779441285878e-06
1336 8.11953577795066e-06
1337 8.15067596704466e-06
1338 8.19510205474216e-06
1339 8.24971812107833e-06
1340 8.32630394143052e-06
1341 8.45167869556462e-06
1342 8.68218012328725e-06
1343 8.84729342942592e-06
1344 9.01235125638777e-06
1345 9.06760669749929e-06
1346 8.99239694263088e-06
1347 8.73650151334004e-06
1348 8.32963087304961e-06
1349 8.09888661024161e-06
1350 8.14479153632419e-06
1351 8.73418775881873e-06
1352 1.00155002655811e-05
1353 1.14081349238404e-05
1354 1.14321555884089e-05
1355 9.79377182375174e-06
1356 8.33473313832656e-06
1357 7.92901209933916e-06
1358 7.97703251009807e-06
1359 8.03426519269124e-06
1360 8.02425256551942e-06
1361 7.99077861302067e-06
1362 7.94876814325107e-06
1363 7.92910577729344e-06
1364 7.92856189946178e-06
1365 7.93264644016745e-06
1366 7.91497041063849e-06
1367 7.91500769992126e-06
1368 7.9176461440511e-06
1369 7.90983085607877e-06
1370 7.9118981375359e-06
1371 7.89554724178743e-06
1372 7.90167268860387e-06
1373 7.89715522842016e-06
1374 7.89488331065513e-06
1375 7.89083696872694e-06
1376 7.87350472819526e-06
1377 7.89625664765481e-06
1378 7.89321529737208e-06
1379 7.88441411714302e-06
1380 7.88271245255601e-06
1381 7.85946031101048e-06
1382 7.85592328611529e-06
1383 7.85455449658912e-06
1384 7.8426965046674e-06
1385 7.91060665505938e-06
1386 7.89891510066809e-06
1387 7.88861234468641e-06
1388 7.86582677392289e-06
1389 7.84810526965884e-06
1390 7.85655265644891e-06
1391 7.84188341640402e-06
1392 7.86542477726471e-06
1393 7.84903750172816e-06
1394 7.87281805969542e-06
1395 7.89268779044505e-06
1396 7.95526284491643e-06
1397 8.04572755441768e-06
1398 8.16692227090243e-06
1399 8.4049315773882e-06
1400 8.79936305864248e-06
1401 9.31062550080242e-06
1402 9.8241216619499e-06
1403 1.00409715741989e-05
1404 9.54494589677779e-06
1405 8.58823023008881e-06
1406 7.95192408986622e-06
1407 8.03409511718201e-06
1408 8.81120286067016e-06
1409 9.84595317277126e-06
1410 1.03377960840589e-05
1411 9.67044343269663e-06
1412 8.62609431351302e-06
1413 7.97429220256163e-06
1414 7.72092516854173e-06
1415 7.69463986216579e-06
1416 7.71204031480011e-06
1417 7.71975737734465e-06
1418 7.71702343627112e-06
1419 7.71635586716002e-06
1420 7.71732538851211e-06
1421 7.7184004112496e-06
1422 7.74270938563859e-06
1423 7.72448493080446e-06
1424 7.73433566791937e-06
1425 7.72035036789021e-06
1426 7.72479143051896e-06
1427 7.719887435087e-06
1428 7.7181439337437e-06
1429 7.72080329625169e-06
1430 7.71470877225511e-06
1431 7.71397844800958e-06
1432 7.7114218584029e-06
1433 7.69398320699111e-06
1434 7.70338647271274e-06
1435 7.68185236665886e-06
1436 7.69249891163781e-06
1437 7.68422160035698e-06
1438 7.67456458561355e-06
1439 7.71003215049859e-06
1440 7.68709105614107e-06
1441 7.6690985224559e-06
1442 7.66768698667875e-06
1443 7.64504693506751e-06
1444 7.65898766985629e-06
1445 7.63501884648576e-06
1446 7.70200495026074e-06
1447 7.6784381235484e-06
1448 7.66909124649828e-06
1449 7.67199071560754e-06
1450 7.69284088164568e-06
1451 7.73452393332263e-06
1452 7.84112671681214e-06
1453 7.99151894170791e-06
1454 8.24356629891554e-06
1455 8.69192717800615e-06
1456 9.33249430090655e-06
1457 1.00327924883459e-05
1458 1.01896175692673e-05
1459 9.40799236559542e-06
1460 8.27495023258962e-06
1461 7.7140375651652e-06
1462 7.99198096501641e-06
1463 8.78023365658009e-06
1464 9.65290928434115e-06
1465 9.57165502768476e-06
1466 9.06604691408575e-06
1467 8.05513263912871e-06
1468 7.75368516769959e-06
1469 7.59306840336649e-06
1470 7.51304969526245e-06
1471 7.57972202336532e-06
1472 7.55786641093437e-06
1473 7.55917562855757e-06
1474 7.51094512452255e-06
1475 7.53017775423359e-06
1476 7.49443915992742e-06
1477 7.53519361751387e-06
1478 7.54228085497743e-06
1479 7.50084518585936e-06
1480 7.5201546678727e-06
1481 7.49876926420256e-06
1482 7.52120149627444e-06
1483 7.46158502806793e-06
1484 7.52250844016089e-06
1485 7.50939170757192e-06
1486 7.48475258660619e-06
1487 7.46808018448064e-06
1488 7.46879550206359e-06
1489 7.45366924093105e-06
1490 7.44481076253578e-06
1491 7.44841827327036e-06
1492 7.42488327887258e-06
1493 7.45118222766905e-06
1494 7.39017241357942e-06
1495 7.47041167414864e-06
1496 7.45760917197913e-06
1497 7.42212523618946e-06
1498 7.47289504943183e-06
1499 7.46073692425853e-06
1500 7.42925840313546e-06
1501 7.41884150556871e-06
1502 7.39481856726343e-06
1503 7.45910074329004e-06
1504 7.39725055609597e-06
1505 7.52621963329148e-06
1506 7.58808073442196e-06
1507 7.68820791563485e-06
1508 7.95734013081528e-06
1509 8.39767199067865e-06
1510 9.11708775674924e-06
1511 9.93571120488923e-06
1512 1.02588082881994e-05
1513 9.56112398853293e-06
1514 8.28891552373534e-06
1515 7.49196397009655e-06
1516 7.76280103309546e-06
1517 8.42257122712908e-06
1518 9.29628458834486e-06
1519 9.39529854804277e-06
1520 8.78320497577079e-06
1521 7.97107986727497e-06
1522 7.58118176236167e-06
1523 7.32541820980259e-06
1524 7.36613765184302e-06
1525 7.379056114587e-06
1526 7.32259150026948e-06
1527 7.33115894036018e-06
1528 7.30636111256899e-06
1529 7.32377293388708e-06
1530 7.30013516658801e-06
1531 7.31239470042055e-06
1532 7.29573503122083e-06
1533 7.30723650121945e-06
1534 7.28043914932641e-06
1535 7.30879946786445e-06
1536 7.27078440831974e-06
1537 7.31344289306435e-06
1538 7.27336691852543e-06
1539 7.30112924429704e-06
1540 7.27495671526412e-06
1541 7.29938074073289e-06
1542 7.24899246051791e-06
1543 7.2848956733651e-06
1544 7.25419067748589e-06
1545 7.26128428141237e-06
1546 7.24183018974145e-06
1547 7.25485506336554e-06
1548 7.23374751032679e-06
1549 7.24401752449921e-06
1550 7.21868764230749e-06
1551 7.29764087736839e-06
1552 7.26730377209606e-06
1553 7.2579628067615e-06
1554 7.23328048479743e-06
1555 7.26937423678464e-06
1556 7.28717668607715e-06
1557 7.3260885073978e-06
1558 7.41438088880386e-06
1559 7.5790462688019e-06
1560 7.83411633165088e-06
1561 8.21809408080298e-06
1562 8.7132893895614e-06
1563 9.2540840341826e-06
1564 9.31732665776508e-06
1565 8.75889963936061e-06
1566 7.84682197263464e-06
1567 7.34479999664472e-06
1568 7.43280543247238e-06
1569 8.13170117908157e-06
1570 8.95020639291033e-06
1571 9.49614513956476e-06
1572 9.03269028640352e-06
1573 8.24305152491434e-06
1574 7.60696002544137e-06
1575 7.27225096852635e-06
1576 7.18094543117331e-06
1577 7.17570173947024e-06
1578 7.16964677849319e-06
1579 7.16504882802838e-06
1580 7.14837733539753e-06
1581 7.14874840923585e-06
1582 7.13400822860422e-06
1583 7.14034422344412e-06
1584 7.13575082045281e-06
1585 7.13880808689282e-06
1586 7.1394620135834e-06
1587 7.14501720722183e-06
1588 7.12859628038132e-06
1589 7.14064299245365e-06
1590 7.12267865310423e-06
1591 7.13118606654461e-06
1592 7.12342080078088e-06
1593 7.11947450326988e-06
1594 7.10576478013536e-06
1595 7.11823849997018e-06
1596 7.10525182512356e-06
1597 7.10769290890312e-06
1598 7.08204197508167e-06
1599 7.09515779817593e-06
1600 7.08113157088519e-06
1601 7.08620245859493e-06
1602 7.07797607901739e-06
1603 7.13388317308272e-06
1604 7.11466191205545e-06
1605 7.0995793066686e-06
1606 7.07925983078894e-06
1607 7.09142750565661e-06
1608 7.0804685492476e-06
1609 7.11827260602149e-06
1610 7.17075454303995e-06
1611 7.2575044214318e-06
1612 7.43380041967612e-06
1613 7.76920296630124e-06
1614 8.32903697300935e-06
1615 9.17604575079167e-06
1616 9.94437868939713e-06
1617 9.87815019470872e-06
1618 8.75850128068123e-06
1619 7.54352504372946e-06
1620 7.18233422958292e-06
1621 7.66048833611421e-06
1622 8.42263489175821e-06
1623 8.94062941370066e-06
1624 8.71901647769846e-06
1625 8.04659066488966e-06
1626 7.41560870665126e-06
1627 7.15339547241456e-06
1628 7.01563249094761e-06
1629 7.03277100910782e-06
1630 6.99614201948862e-06
1631 7.0136770773388e-06
1632 6.99305383022875e-06
1633 6.99828024153248e-06
1634 6.99259089742554e-06
1635 6.99780866852961e-06
1636 6.99195925335516e-06
1637 6.99508245816105e-06
1638 6.99092333888984e-06
1639 6.9962625275366e-06
1640 6.98567237122916e-06
1641 6.9912885010126e-06
1642 6.97739869792713e-06
1643 6.99811334925471e-06
1644 6.97250061421073e-06
1645 6.98784560881904e-06
1646 6.96858296578284e-06
1647 6.98310486768605e-06
1648 6.96471624905826e-06
1649 6.9770767368027e-06
1650 6.94934988132445e-06
1651 6.96549568601768e-06
1652 6.9509633249254e-06
1653 6.95395920047304e-06
1654 6.94370100973174e-06
1655 6.94750906404806e-06
1656 6.93543779561878e-06
1657 6.94762002240168e-06
1658 6.92321646056371e-06
1659 6.97491213941248e-06
1660 6.98221765560447e-06
1661 6.98825260769809e-06
1662 7.06181344867218e-06
1663 7.10757376509719e-06
1664 7.17558395990636e-06
1665 7.34967443349888e-06
1666 7.71703071222873e-06
1667 8.35555783851305e-06
1668 9.31292015593499e-06
1669 1.01605992313125e-05
1670 9.97392999124713e-06
1671 8.55409052746836e-06
1672 7.26824464436504e-06
1673 7.18038654667907e-06
1674 7.78838239057222e-06
1675 8.53392521094065e-06
1676 8.68176175572444e-06
1677 8.28926386020612e-06
1678 7.55512110117706e-06
1679 7.19084800948622e-06
1680 7.04179092281265e-06
1681 6.96046345183277e-06
1682 6.90001024850062e-06
1683 6.91691775500658e-06
1684 6.89239277562592e-06
1685 6.9044631345605e-06
1686 6.89328135194955e-06
1687 6.90223623678321e-06
1688 6.89410990162287e-06
1689 6.90144634063472e-06
1690 6.8875087890774e-06
1691 6.89232092554448e-06
1692 6.89265698383679e-06
1693 6.88888576405589e-06
1694 6.87629562889924e-06
1695 6.89495573169552e-06
1696 6.86789007886546e-06
1697 6.88870068188407e-06
1698 6.86668454363826e-06
1699 6.87165402268874e-06
1700 6.86228122503962e-06
1701 6.8708877734025e-06
1702 6.85938266542507e-06
1703 6.86094608681742e-06
1704 6.85377835907275e-06
1705 6.8603453655669e-06
1706 6.84112910676049e-06
1707 6.84786436977447e-06
1708 6.83810640111915e-06
1709 6.84294536767993e-06
1710 6.83083135299967e-06
1711 6.83630605635699e-06
1712 6.83610142004909e-06
1713 6.83163534631603e-06
1714 6.83015878166771e-06
1715 6.85292025082163e-06
1716 6.86285193296499e-06
1717 6.9155935307208e-06
1718 6.97644782121642e-06
1719 7.16844624548685e-06
1720 7.51790094000171e-06
1721 8.12572397990152e-06
1722 9.02555348147871e-06
1723 1.00866482171114e-05
1724 1.02678095572628e-05
1725 8.97359495866112e-06
1726 7.44772523830761e-06
1727 7.08935385773657e-06
1728 7.63584921514848e-06
1729 8.39704080135562e-06
1730 8.69356153998524e-06
1731 8.31026227388065e-06
1732 7.58250052967924e-06
1733 7.10218455424183e-06
1734 6.88667705617263e-06
1735 6.82741392665775e-06
1736 6.78363812767202e-06
1737 6.79946288073552e-06
1738 6.79408003634308e-06
1739 6.80456423651776e-06
1740 6.79966251482256e-06
1741 6.8090453169134e-06
1742 6.80517905493616e-06
1743 6.80941047903616e-06
1744 6.80123093843576e-06
1745 6.81922210787889e-06
1746 6.8026852204639e-06
1747 6.81481515130145e-06
1748 6.79117965773912e-06
1749 6.81589835949126e-06
1750 6.79218283039518e-06
1751 6.80572884448338e-06
1752 6.78946662446833e-06
1753 6.80023867971613e-06
1754 6.78091009831405e-06
1755 6.79264212521957e-06
1756 6.77524485581671e-06
1757 6.78496098771575e-06
1758 6.76207673677709e-06
1759 6.78552214594674e-06
1760 6.76274112265673e-06
1761 6.77288471706561e-06
1762 6.77011212246725e-06
1763 6.77117259328952e-06
1764 6.75434739605407e-06
1765 6.77215120958863e-06
1766 6.7517962634156e-06
1767 6.76378522257437e-06
1768 6.73738850309746e-06
1769 6.75823093843064e-06
1770 6.73759950586827e-06
1771 6.7588507590699e-06
1772 6.74831471769721e-06
1773 6.7571295403468e-06
1774 6.76248373565613e-06
1775 6.82325480738655e-06
1776 6.8956401264586e-06
1777 7.10369340595207e-06
1778 7.52473169995937e-06
1779 8.43180441734148e-06
1780 9.95684058580082e-06
1781 1.15521088446258e-05
1782 1.10941173261381e-05
1783 8.44083206175128e-06
1784 6.96098322805483e-06
1785 7.44413409847766e-06
1786 8.25848292151932e-06
1787 8.4366374721867e-06
1788 7.95410596765578e-06
1789 7.37085929358727e-06
1790 6.953837328183e-06
1791 6.82308655086672e-06
1792 6.73501563142054e-06
1793 6.74467082717456e-06
1794 6.70719509798801e-06
1795 6.7403316279524e-06
1796 6.70704821459367e-06
1797 6.74152943247464e-06
1798 6.70793860990671e-06
1799 6.73891008773353e-06
1800 6.72121177558438e-06
1801 6.71976022204035e-06
1802 6.71262705509434e-06
1803 6.72357873554574e-06
1804 6.71815405439702e-06
1805 6.72130727252807e-06
1806 6.70896679366706e-06
1807 6.72670284984633e-06
1808 6.71225325277192e-06
1809 6.72291207592934e-06
1810 6.70022200210951e-06
1811 6.7164792199037e-06
1812 6.70089502818882e-06
1813 6.71261659590527e-06
1814 6.70848248773837e-06
1815 6.70666076985071e-06
1816 6.69975997880101e-06
1817 6.70400822855299e-06
1818 6.69510563966469e-06
1819 6.70973940941622e-06
1820 6.68945449433522e-06
1821 6.70542613079306e-06
1822 6.68252368996036e-06
1823 6.70457257001544e-06
1824 6.69762175675714e-06
1825 6.70346526021603e-06
1826 6.70115787215764e-06
1827 6.71045017952565e-06
1828 6.69630435368163e-06
1829 6.73355452818214e-06
1830 6.73053318678285e-06
1831 6.7812152337865e-06
1832 6.82636982674012e-06
1833 6.96584856996196e-06
1834 7.14776069798972e-06
1835 7.57179986976553e-06
1836 8.3101831478416e-06
1837 8.98036341823172e-06
1838 9.28088411455974e-06
1839 8.72588407219155e-06
1840 7.76177694206126e-06
1841 6.93142601448926e-06
1842 6.90226534061367e-06
1843 7.54662414692575e-06
1844 8.43934321892448e-06
1845 9.00912982615409e-06
1846 8.75257501320448e-06
1847 7.84319308877457e-06
1848 7.1096837928053e-06
1849 6.81672099744901e-06
1850 6.72182568450808e-06
1851 6.69197333991178e-06
1852 6.68072379994555e-06
1853 6.67134645482292e-06
1854 6.66099003865384e-06
1855 6.67043332214234e-06
1856 6.66537880533724e-06
1857 6.6688735387288e-06
1858 6.67421591060702e-06
1859 6.67145832267124e-06
1860 6.66749383526621e-06
1861 6.67182121105725e-06
1862 6.67050517222378e-06
1863 6.67401900500408e-06
1864 6.65559582557762e-06
1865 6.66858386466629e-06
1866 6.64947765471879e-06
1867 6.65240577291115e-06
1868 6.64488425172749e-06
1869 6.65215839035227e-06
1870 6.63942137180129e-06
1871 6.63939408696024e-06
1872 6.63363971398212e-06
1873 6.63368382447516e-06
1874 6.63110313325888e-06
1875 6.63461241856567e-06
1876 6.61719104755321e-06
1877 6.6397574300936e-06
1878 6.63787113808212e-06
1879 6.65206016492448e-06
1880 6.66613277644501e-06
1881 6.70968893246027e-06
1882 6.74382226861781e-06
1883 6.84824135532835e-06
1884 6.97627501722309e-06
1885 7.19997842679732e-06
1886 7.48254342397559e-06
1887 7.83197901910171e-06
1888 8.02754584583454e-06
1889 8.02072827355005e-06
1890 7.79164111008868e-06
1891 7.18553201295435e-06
1892 6.75583441989147e-06
1893 6.7354603743297e-06
1894 7.1505960477225e-06
1895 8.00241923570866e-06
1896 9.06347304407973e-06
1897 9.59278895606985e-06
1898 8.88559316081228e-06
1899 7.67798064771341e-06
1900 6.92274716129759e-06
1901 6.71046063871472e-06
1902 6.72093483444769e-06
1903 6.72125270284596e-06
1904 6.68801703795907e-06
1905 6.67112135488424e-06
1906 6.63947002976784e-06
1907 6.63152968627401e-06
1908 6.6226762100996e-06
1909 6.62888578517595e-06
1910 6.63362288833014e-06
1911 6.63598757455475e-06
1912 6.63500577502418e-06
1913 6.63559876556974e-06
1914 6.63112996335258e-06
1915 6.64328490529442e-06
1916 6.63959644953138e-06
1917 6.63993660054985e-06
1918 6.62515913063544e-06
1919 6.63614855511696e-06
1920 6.62973661746946e-06
1921 6.62595175526803e-06
1922 6.62276943330653e-06
1923 6.62150205243961e-06
1924 6.61566627968568e-06
1925 6.6216780396644e-06
1926 6.60475006952765e-06
1927 6.62547199681285e-06
1928 6.61378771837917e-06
1929 6.63184073346201e-06
1930 6.63193577565835e-06
1931 6.67021913614008e-06
1932 6.70071221975377e-06
1933 6.77526486470015e-06
1934 6.88156296746456e-06
1935 7.08385687175905e-06
1936 7.34830473447801e-06
1937 7.75331955082947e-06
1938 8.05469153419835e-06
1939 8.30285898700822e-06
1940 8.18839544081129e-06
1941 7.49514174458454e-06
1942 6.85138184053358e-06
1943 6.69105338602094e-06
1944 6.97346740707872e-06
1945 7.66034281696193e-06
1946 8.5827359725954e-06
1947 9.20255479286425e-06
1948 8.85033296071924e-06
1949 7.79905985837104e-06
1950 6.99858128427877e-06
1951 6.72379428579006e-06
1952 6.67440872348379e-06
1953 6.67141966914642e-06
1954 6.65626657792018e-06
1955 6.63415039525717e-06
1956 6.62293859932106e-06
1957 6.60780960970442e-06
1958 6.61212925479049e-06
1959 6.60940304442192e-06
1960 6.61489502817858e-06
1961 6.61894046061207e-06
1962 6.62342881696532e-06
1963 6.62288857711246e-06
1964 6.6197289925185e-06
1965 6.62776665194542e-06
1966 6.62386855765362e-06
1967 6.62216143609839e-06
1968 6.62418460706249e-06
1969 6.60915566186304e-06
1970 6.6226762100996e-06
1971 6.60598243484856e-06
1972 6.60872910884791e-06
1973 6.59795523461071e-06
1974 6.59116949464078e-06
1975 6.59445231576683e-06
1976 6.59596526020323e-06
1977 6.59076113151968e-06
1978 6.59756051391014e-06
1979 6.59773468214553e-06
1980 6.61240437693777e-06
1981 6.6024672378262e-06
1982 6.63207583784242e-06
1983 6.6500451794127e-06
1984 6.71044426781009e-06
1985 6.80276752973441e-06
1986 6.9701268330391e-06
1987 7.25548989066738e-06
1988 7.70340375311207e-06
1989 8.23440586827928e-06
1990 8.68707502377219e-06
1991 8.78383525559912e-06
1992 7.93376420915592e-06
1993 7.03974637872307e-06
1994 6.70407098368742e-06
1995 6.9762018028996e-06
1996 7.68612335377838e-06
1997 8.55434245750075e-06
1998 8.98697908269241e-06
1999 8.55370399222011e-06
};
\addlegendentry{Test}
\nextgroupplot[
title={ReLU/ReLU},
ymin=5.44710249170915e-06, ymax=0.001,
]
\addplot [semithick, black, dashed]
table {%
	0 0.00624040243565105
	1 0.00621615676936926
	2 0.00619671851018211
	3 0.00618020132242236
	4 0.00616607644769829
	5 0.00615396772627719
	6 0.00614341196705936
	7 0.00613411925951368
	8 0.00612577959327609
	9 0.00611821992424666
	10 0.00611136637962773
	11 0.00610506910379627
	12 0.00609908264959813
	13 0.00609312472806778
	14 0.00608685711267754
	15 0.00607998481427785
	16 0.00607184365435387
	17 0.00606199451067368
	18 0.00605014433313045
	19 0.0060356309695635
	20 0.00601585431286367
	21 0.00598775188700529
	22 0.00594722858659225
	23 0.00589093405869789
	24 0.00581697924644686
	25 0.00572206884680782
	26 0.00560551705711987
	27 0.00546758633572608
	28 0.00530784375951043
	29 0.00512906810035929
	30 0.00493882158116321
	31 0.00474375075646094
	32 0.00455382666405058
	33 0.00437184012844227
	34 0.00420076990303642
	35 0.00404318277469429
	36 0.0038991222572804
	37 0.00376662573762587
	38 0.00364286782314593
	39 0.00352702388227044
	40 0.00341791665232449
	41 0.00331414487664006
	42 0.00321436433659983
	43 0.00311830826467485
	44 0.00302565087440598
	45 0.00293631321073917
	46 0.00285131758755597
	47 0.00277034290138545
	48 0.00269314230354212
	49 0.00261824912377051
	50 0.00254538994522591
	51 0.00247637722168292
	52 0.00241074115820084
	53 0.0023477861086576
	54 0.00228778797372797
	55 0.00223033290785679
	56 0.00217500685175764
	57 0.00212128411203594
	58 0.00206997063560266
	59 0.00202029750016663
	60 0.00197247932919709
	61 0.00192662175959413
	62 0.00188198597425071
	63 0.00183907926657412
	64 0.00179783036128356
	65 0.00175759236390149
	66 0.00171821257208649
	67 0.00167993448167181
	68 0.00164271580388231
	69 0.00160710271984499
	70 0.00157282105647027
	71 0.00153974376144106
	72 0.00150791071200729
	73 0.00147672971752399
	74 0.00144576694310672
	75 0.0014143770604278
	76 0.001380821490784
	77 0.00134847116351011
	78 0.00131771583710361
	79 0.00128811726290223
	80 0.0012605298870767
	81 0.00123418494376892
	82 0.00120902545995705
	83 0.0011849749898829
	84 0.00116184249782236
	85 0.00113943289056806
	86 0.0011178162694705
	87 0.00109324034929159
	88 0.00106792403153122
	89 0.00103874485716915
	90 0.0010088924543652
	91 0.000982460591785639
	92 0.000957714356673023
	93 0.000934416509380753
	94 0.000911212325490851
	95 0.000889219748160031
	96 0.000868761788296979
	97 0.000849568839043968
	98 0.000831346246513931
	99 0.000813357662877934
	100 0.000795783281091644
	101 0.000779206700144641
	102 0.000763524932608561
	103 0.00074739229125953
	104 0.000731413029029682
	105 0.000716846362820434
	106 0.000702978360664019
	107 0.00068946997043895
	108 0.000676704704005715
	109 0.000664431603013327
	110 0.0006527217257144
	111 0.000641309287004788
	112 0.000630236076460733
	113 0.000619157466871911
	114 0.000608126256395281
	115 0.000597148839005968
	116 0.000586476833461802
	117 0.000576137815812672
	118 0.000566138307362962
	119 0.000556271234529504
	120 0.000546153839138697
	121 0.000536272189378906
	122 0.000526972114755608
	123 0.000518095078405167
	124 0.000509534084528696
	125 0.000500994117260234
	126 0.000491579517529317
	127 0.000482514102117193
	128 0.000473844032001125
	129 0.00046514599114289
	130 0.000456843264856843
	131 0.000449060413274083
	132 0.000441631131337772
	133 0.000434530973421943
	134 0.000427641679152657
	135 0.000421124726045718
	136 0.000414744887450524
	137 0.00040852318241491
	138 0.000402460065771493
	139 0.00039625485865713
	140 0.000390115193795282
	141 0.000383919205262373
	142 0.000377886500075419
	143 0.000371965570451493
	144 0.000366407915578293
	145 0.000361090006890663
	146 0.000355969838324199
	147 0.000351019162906141
	148 0.00034628603185638
	149 0.000341635454617517
	150 0.000337094262476967
	151 0.00033265194488763
	152 0.000328321003024712
	153 0.000324123322826608
	154 0.000319885847005708
	155 0.000315492783585114
	156 0.000311247737158737
	157 0.000307139093109754
	158 0.000302976426496571
	159 0.000298347328197224
	160 0.000294034397455789
	161 0.000289836872809701
	162 0.000285852113876217
	163 0.000282010527683951
	164 0.000278025212679722
	165 0.000272945302810967
	166 0.000266741652694691
	167 0.00026041948123634
	168 0.000254799941245665
	169 0.000249727876081351
	170 0.000244441575262044
	171 0.000239226317177099
	172 0.000234152057259962
	173 0.000229264998012013
	174 0.000224891203515654
	175 0.000220711215291658
	176 0.000216722384749346
	177 0.000212919707493597
	178 0.000209283881076772
	179 0.000205710860669228
	180 0.000202118504844861
	181 0.000198738570517776
	182 0.000195489142967631
	183 0.000192380026078354
	184 0.000189297531903776
	185 0.000186337196794284
	186 0.000183453884474716
	187 0.00018065160271874
	188 0.000177823454919235
	189 0.000175144973525221
	190 0.000172499578894758
	191 0.000169880427080216
	192 0.000167381914707221
	193 0.000164905318285946
	194 0.000162237886456751
	195 0.000159349397833353
	196 0.000156110283512589
	197 0.000153215015458841
	198 0.000150546140233132
	199 0.000148015639922505
	200 0.000145673743219277
	201 0.000143439834118908
	202 0.000141304760646221
	203 0.000139310426675365
	204 0.000137387209534268
	205 0.000135542794751586
	206 0.000133825669394128
	207 0.000132019818366302
	208 0.000130305935712727
	209 0.000128583585777164
	210 0.000126936317784043
	211 0.000125353324619937
	212 0.000123806380941005
	213 0.000122339858464215
	214 0.000120785896498887
	215 0.000119259653530435
	216 0.000117819017560805
	217 0.00011644601612204
	218 0.000115075512965745
	219 0.000113761049036043
	220 0.000112463800732598
	221 0.000111219612747959
	222 0.000109962428325616
	223 0.000108792445161043
	224 0.000107602136267815
	225 0.00010643686931644
	226 0.000105328615930489
	227 0.000104206530878059
	228 0.000103132123783212
	229 0.000102058366891811
	230 0.000101011996065381
	231 0.00010002173173973
	232 9.89784009419736e-05
	233 9.80091829489993e-05
	234 9.7004676476331e-05
	235 9.60670319187784e-05
	236 9.51138227094361e-05
	237 9.42135237096409e-05
	238 9.32734311618333e-05
	239 9.2381941854569e-05
	240 9.14978573405278e-05
	241 9.06204519566245e-05
	242 8.98010209198219e-05
	243 8.89530050329768e-05
	244 8.80847062916246e-05
	245 8.73062465558405e-05
	246 8.64985665742779e-05
	247 8.56901823453882e-05
	248 8.49122423147719e-05
	249 8.41311879895557e-05
	250 8.34118203130174e-05
	251 8.26304054442062e-05
	252 8.18686489481024e-05
	253 8.11953817816402e-05
	254 8.04388689203961e-05
	255 7.97063291742006e-05
	256 7.90310957512474e-05
	257 7.83340440619895e-05
	258 7.76550553212019e-05
	259 7.69842474994675e-05
	260 7.61983104808905e-05
	261 7.52323243489172e-05
	262 7.42984855577333e-05
	263 7.31830288742685e-05
	264 7.18946205608972e-05
	265 7.06481199728159e-05
	266 6.95850217837801e-05
	267 6.85928530046453e-05
	268 6.77024689679229e-05
	269 6.68506709686767e-05
	270 6.60872099516041e-05
	271 6.53147221640893e-05
	272 6.46065770908422e-05
	273 6.39136692370812e-05
	274 6.32383432161987e-05
	275 6.25932039355348e-05
	276 6.19790285085742e-05
	277 6.1364357023308e-05
	278 6.07991887235926e-05
	279 6.01871674774657e-05
	280 5.96621515214224e-05
	281 5.91411180792534e-05
	282 5.85776294741436e-05
	283 5.80703296719776e-05
	284 5.75827826168052e-05
	285 5.70918027520406e-05
	286 5.66158201920075e-05
	287 5.61386829076582e-05
	288 5.57114442614193e-05
	289 5.52246410521207e-05
	290 5.48320698143812e-05
	291 5.43583126386693e-05
	292 5.39727910222609e-05
	293 5.3550389949919e-05
	294 5.31511883465896e-05
	295 5.27450540843688e-05
	296 5.23692105574014e-05
	297 5.19989552429934e-05
	298 5.15969858483345e-05
	299 5.12522129056947e-05
	300 5.08977557345247e-05
	301 5.05233928222992e-05
	302 5.02172225296249e-05
	303 4.98299851869888e-05
	304 4.95112996290459e-05
	305 4.91863126867997e-05
	306 4.88323101635046e-05
	307 4.85360835682513e-05
	308 4.82263805849925e-05
	309 4.78995538273352e-05
	310 4.76315058151044e-05
	311 4.73048445002178e-05
	312 4.69967865370791e-05
	313 4.67250312183864e-05
	314 4.64693232729019e-05
	315 4.61369944204648e-05
	316 4.58845495927562e-05
	317 4.56081798319019e-05
	318 4.53373976796456e-05
	319 4.51065552695695e-05
	320 4.47956128653004e-05
	321 4.4567668517459e-05
	322 4.42909101465716e-05
	323 4.40465771802678e-05
	324 4.38191220837325e-05
	325 4.35299561019065e-05
	326 4.33081186983486e-05
	327 4.3067030098598e-05
	328 4.28022222251911e-05
	329 4.25702249415849e-05
	330 4.2327835650724e-05
	331 4.20690442766158e-05
	332 4.18731477509482e-05
	333 4.1617274362693e-05
	334 4.14266193615731e-05
	335 4.11970090894442e-05
	336 4.09946548245443e-05
	337 4.07695722302037e-05
	338 4.05829881415798e-05
	339 4.03548078864446e-05
	340 4.01929952218438e-05
	341 3.99905979904247e-05
	342 3.97769420175109e-05
	343 3.96220450369356e-05
	344 3.94164452615087e-05
	345 3.92247173408578e-05
	346 3.9080447962192e-05
	347 3.88659323249385e-05
	348 3.86890806893803e-05
	349 3.85220079195392e-05
	350 3.83813528870292e-05
	351 3.81551780250788e-05
	352 3.80364174574765e-05
	353 3.78562085074918e-05
	354 3.7665970481271e-05
	355 3.75219594630494e-05
	356 3.73599309426709e-05
	357 3.72194214364185e-05
	358 3.70311575963456e-05
	359 3.6902695441654e-05
	360 3.67485052024108e-05
	361 3.65790757328455e-05
	362 3.64302652968718e-05
	363 3.62964599744942e-05
	364 3.61792346410539e-05
	365 3.59934986988719e-05
	366 3.58757997851455e-05
	367 3.57200328551244e-05
	368 3.55777482106134e-05
	369 3.54560213189359e-05
	370 3.53084413156068e-05
	371 3.51671175700119e-05
	372 3.50152498036493e-05
	373 3.48831723293586e-05
	374 3.47504104922791e-05
	375 3.46106324577988e-05
	376 3.44669807894604e-05
	377 3.4354782094681e-05
	378 3.42136456730202e-05
	379 3.4088121861231e-05
	380 3.39716691719616e-05
	381 3.38549546299305e-05
	382 3.37221284887335e-05
	383 3.35910325404143e-05
	384 3.34396728547404e-05
	385 3.33179983265097e-05
	386 3.32063145549455e-05
	387 3.30954456018162e-05
	388 3.29574794015741e-05
	389 3.28563738172249e-05
	390 3.27294746398366e-05
	391 3.26050221097773e-05
	392 3.24852091182493e-05
	393 3.23762924878679e-05
	394 3.22977498932175e-05
	395 3.21626644002038e-05
	396 3.20603958066101e-05
	397 3.19579727268149e-05
	398 3.18508297141307e-05
	399 3.17556391564722e-05
	400 3.16428552942227e-05
	401 3.15667115913243e-05
	402 3.1453490091593e-05
	403 3.13615056022343e-05
	404 3.12494926646423e-05
	405 3.11602959470747e-05
	406 3.10587001060014e-05
	407 3.0937374528861e-05
	408 3.08483661193293e-05
	409 3.07559284102865e-05
	410 3.06817433610718e-05
	411 3.05607692183685e-05
	412 3.04534750235064e-05
	413 3.03815656508277e-05
	414 3.02911554719287e-05
	415 3.01950676622198e-05
	416 3.00896580505139e-05
	417 3.00115070857032e-05
	418 2.99218063126716e-05
	419 2.98298838323774e-05
	420 2.97394196877576e-05
	421 2.96446665792871e-05
	422 2.95736020099469e-05
	423 2.94882386278061e-05
	424 2.93953850487583e-05
	425 2.93055173763435e-05
	426 2.92203878231589e-05
	427 2.91625763750858e-05
	428 2.90616531941623e-05
	429 2.89792534573508e-05
	430 2.88972308020163e-05
	431 2.88337013181206e-05
	432 2.87342828322323e-05
	433 2.86566452025738e-05
	434 2.85881559669576e-05
	435 2.85097161807357e-05
	436 2.84374144996491e-05
	437 2.83626623627242e-05
	438 2.83040188726602e-05
	439 2.82028281830549e-05
	440 2.81437719102939e-05
	441 2.80567836128398e-05
	442 2.79800929838814e-05
	443 2.7926784191834e-05
	444 2.78171653569359e-05
	445 2.77573597848857e-05
	446 2.77071250316396e-05
	447 2.76397550820207e-05
	448 2.75517881682674e-05
	449 2.74802941504504e-05
	450 2.74028285751626e-05
	451 2.73328985649357e-05
	452 2.7264913068592e-05
	453 2.71883426563591e-05
	454 2.71274910730313e-05
	455 2.70597026847952e-05
	456 2.70005630937931e-05
	457 2.69086425035425e-05
	458 2.6880505611615e-05
	459 2.67780961173969e-05
	460 2.6729792416802e-05
	461 2.66632983390025e-05
	462 2.66015008030251e-05
	463 2.64997159575842e-05
	464 2.64671402554484e-05
	465 2.63827238349279e-05
	466 2.62864927549344e-05
	467 2.62542143367739e-05
	468 2.61633156846131e-05
	469 2.6116425644318e-05
	470 2.60596749512843e-05
	471 2.59807385276645e-05
	472 2.59145011405337e-05
	473 2.58750366306515e-05
	474 2.57877144456131e-05
	475 2.57526059286306e-05
	476 2.56470828716715e-05
	477 2.56029423582049e-05
	478 2.55039513739064e-05
	479 2.54507798480574e-05
	480 2.53959519085356e-05
	481 2.53350901004978e-05
	482 2.52553717245974e-05
	483 2.51931274846129e-05
	484 2.51513903855027e-05
	485 2.51031600413398e-05
	486 2.49978244077909e-05
	487 2.4975637632707e-05
	488 2.48815351007181e-05
	489 2.4823647528649e-05
	490 2.4802430910853e-05
	491 2.47224012568381e-05
	492 2.46508493724207e-05
	493 2.46084065516072e-05
	494 2.45415335591304e-05
	495 2.44857022480005e-05
	496 2.44366084700687e-05
	497 2.43652317841736e-05
	498 2.42931818199565e-05
	499 2.42636524525608e-05
	500 2.41849626085866e-05
	501 2.41194763184183e-05
	502 2.4084854768347e-05
	503 2.40054340778784e-05
	504 2.3954586199082e-05
	505 2.39108195856375e-05
	506 2.38416603650649e-05
	507 2.37983241380135e-05
	508 2.37649650802751e-05
	509 2.36635637129723e-05
	510 2.36385113154824e-05
	511 2.36112913345465e-05
	512 2.35158522166046e-05
	513 2.34532556451228e-05
	514 2.3410339601071e-05
	515 2.33734498280569e-05
	516 2.32814830329886e-05
	517 2.32284121164383e-05
	518 2.31811831259421e-05
	519 2.31375933132938e-05
	520 2.31020548042693e-05
	521 2.30462234966922e-05
	522 2.29924523615921e-05
	523 2.29268338785005e-05
	524 2.29170280583446e-05
	525 2.28776169048928e-05
	526 2.27920373383483e-05
	527 2.27126263130373e-05
	528 2.26277744737047e-05
	529 2.2580120168314e-05
	530 2.25131148958724e-05
	531 2.25363697534675e-05
	532 2.26263813107863e-05
	533 2.27645826278433e-05
	534 2.28945697440253e-05
	535 2.28809071725067e-05
	536 2.26647933310176e-05
	537 2.24228224734446e-05
	538 2.22183413800536e-05
	539 2.21766996855166e-05
	540 2.22086424699341e-05
	541 2.22495454273997e-05
	542 2.22316793117017e-05
	543 2.20954808476392e-05
	544 2.18593296708036e-05
	545 2.17524983572304e-05
	546 2.16162021811073e-05
	547 2.16018773926407e-05
	548 2.15991517382719e-05
	549 2.16073255447924e-05
	550 2.15484648808939e-05
	551 2.15324503010095e-05
	552 2.14352023650832e-05
	553 2.13558173172146e-05
	554 2.12784803643729e-05
	555 2.12426110710595e-05
	556 2.11984593736503e-05
	557 2.11421191753658e-05
	558 2.11136158849712e-05
	559 2.10357445897813e-05
	560 2.10185763602766e-05
	561 2.09590951509853e-05
	562 2.08887055315188e-05
	563 2.08472997265119e-05
	564 2.08485582806617e-05
	565 2.08125393434955e-05
	566 2.07443278927855e-05
	567 2.07553992872533e-05
	568 2.07187460272706e-05
	569 2.06707372196746e-05
	570 2.06221308882704e-05
	571 2.05647426589906e-05
	572 2.05293353765512e-05
	573 2.05165529774121e-05
	574 2.04713277405233e-05
	575 2.03718586906376e-05
	576 2.0359697572303e-05
	577 2.03508837905986e-05
	578 2.02918230574056e-05
	579 2.02479833912861e-05
	580 2.01742558640916e-05
	581 2.01559344468194e-05
	582 2.00988844127892e-05
	583 2.00239975729488e-05
	584 2.00279003390591e-05
	585 2.00800973519932e-05
	586 2.00973583677211e-05
	587 2.01253145579727e-05
	588 2.01736049660894e-05
	589 2.01785379001507e-05
	590 2.0097387125162e-05
	591 1.99277016967869e-05
	592 1.98232926553743e-05
	593 1.97710072100676e-05
	594 1.97933547649853e-05
	595 1.98012996417418e-05
	596 1.9848758686436e-05
	597 1.98070356685065e-05
	598 1.9730354582137e-05
	599 1.963104578806e-05
	600 1.94853170665255e-05
	601 1.93697129162729e-05
	602 1.93556756435953e-05
	603 1.93610362053676e-05
	604 1.93806203139246e-05
	605 1.94308819185807e-05
	606 1.94215095046246e-05
	607 1.9365455228737e-05
	608 1.92242324494885e-05
	609 1.90896789238337e-05
	610 1.90479100652396e-05
	611 1.9042929185531e-05
	612 1.8993937770162e-05
	613 1.89629287881132e-05
	614 1.89836907154017e-05
	615 1.89483087229547e-05
	616 1.88409351764562e-05
	617 1.87823051032154e-05
	618 1.87468856154283e-05
	619 1.8709076817558e-05
	620 1.86902527108401e-05
	621 1.87483079532313e-05
	622 1.87684038586156e-05
	623 1.87696508326951e-05
	624 1.87306303889301e-05
	625 1.87102272430906e-05
	626 1.86889588356109e-05
	627 1.85860015040618e-05
	628 1.8513928010222e-05
	629 1.84980869697426e-05
	630 1.85087936142025e-05
	631 1.85272185380114e-05
	632 1.84691942042292e-05
	633 1.84263380766936e-05
	634 1.83874575245824e-05
	635 1.83359642935699e-05
	636 1.82539045692209e-05
	637 1.82432104232078e-05
	638 1.82598533502443e-05
	639 1.8319960817692e-05
	640 1.83764171577394e-05
	641 1.84292875946568e-05
	642 1.84485243117649e-05
	643 1.84285748350277e-05
	644 1.82793361798161e-05
	645 1.81538449925966e-05
	646 1.81657687683412e-05
	647 1.81384010744523e-05
	648 1.80850842195213e-05
	649 1.81160723844442e-05
	650 1.81772606850927e-05
	651 1.81014947617797e-05
	652 1.80076943170349e-05
	653 1.79190809266316e-05
	654 1.78650470719077e-05
	655 1.78373509474028e-05
	656 1.78783558091311e-05
	657 1.79643297659737e-05
	658 1.8073084719461e-05
	659 1.80615923248695e-05
	660 1.80411834591609e-05
	661 1.80104440339335e-05
	662 1.79137669569229e-05
	663 1.77790815314438e-05
	664 1.77207284615122e-05
	665 1.77008831343528e-05
	666 1.77338023465978e-05
	667 1.7738219357355e-05
	668 1.7671277587894e-05
	669 1.76133019795799e-05
	670 1.75594818649927e-05
	671 1.75105177344648e-05
	672 1.74824473155866e-05
	673 1.74839416882122e-05
	674 1.75149603407476e-05
	675 1.7555392256341e-05
	676 1.75839814282597e-05
	677 1.7625994978232e-05
	678 1.76534099853853e-05
	679 1.75798672152894e-05
	680 1.74814029740844e-05
	681 1.74367688323684e-05
	682 1.7434927308102e-05
	683 1.741036705738e-05
	684 1.73669179925895e-05
	685 1.73356827115612e-05
	686 1.73325111774147e-05
	687 1.73298058179228e-05
	688 1.73382978392311e-05
	689 1.72422962076979e-05
	690 1.7166593242024e-05
	691 1.71333974492427e-05
	692 1.71468957450571e-05
	693 1.72155286861653e-05
	694 1.7270092829591e-05
	695 1.73707326229788e-05
	696 1.74232069962699e-05
	697 1.74134267716397e-05
	698 1.73044470983541e-05
	699 1.71708990261976e-05
	700 1.71384218141668e-05
	701 1.71444516325892e-05
	702 1.70804057315621e-05
	703 1.70615843497757e-05
	704 1.70879378256217e-05
	705 1.70978823863521e-05
	706 1.70309149236658e-05
	707 1.69237776646014e-05
	708 1.68567913068074e-05
	709 1.68472745940562e-05
	710 1.68466540237944e-05
	711 1.68779442102363e-05
	712 1.69558144875737e-05
	713 1.70610291974072e-05
	714 1.71250415039736e-05
	715 1.70690889067515e-05
	716 1.69753034455766e-05
	717 1.68706390972773e-05
	718 1.68356302854278e-05
	719 1.6819841631488e-05
	720 1.676204047385e-05
	721 1.67333461167374e-05
	722 1.67473185292977e-05
	723 1.67483460291606e-05
	724 1.67129757109308e-05
	725 1.66197531292767e-05
	726 1.65610358724422e-05
	727 1.65507316935987e-05
	728 1.65628385708771e-05
	729 1.65874954252132e-05
	730 1.66535964662984e-05
	731 1.6743529370089e-05
	732 1.67878237302688e-05
	733 1.67498409116007e-05
	734 1.67048390444791e-05
	735 1.66324109809324e-05
	736 1.65744302904614e-05
	737 1.64932866546508e-05
	738 1.64626416268021e-05
	739 1.64825760435861e-05
	740 1.65052889791184e-05
	741 1.64500635584375e-05
	742 1.6412839713098e-05
	743 1.64069724579718e-05
	744 1.63569753173221e-05
	745 1.62848150413453e-05
	746 1.6260879791119e-05
	747 1.63050659995889e-05
	748 1.6387819671948e-05
	749 1.64640370190483e-05
	750 1.65970701591078e-05
	751 1.66190160957314e-05
	752 1.65279107253724e-05
	753 1.64333936911731e-05
	754 1.63766044014579e-05
	755 1.62647324177811e-05
	756 1.62184808818466e-05
	757 1.62559358383163e-05
	758 1.62860892594097e-05
	759 1.62416393081344e-05
	760 1.62184243599484e-05
	761 1.61631758945902e-05
	762 1.60663818995488e-05
	763 1.60168641674119e-05
	764 1.60098893200455e-05
	765 1.60241871842715e-05
	766 1.61123381676731e-05
	767 1.62443733238149e-05
	768 1.62936796552771e-05
	769 1.62541637305225e-05
	770 1.62037505440082e-05
	771 1.6150688876948e-05
	772 1.60630614498558e-05
	773 1.59820280547507e-05
	774 1.59695103949531e-05
	775 1.60097359529487e-05
	776 1.59878797507673e-05
	777 1.59554621088631e-05
	778 1.59173789811007e-05
	779 1.58869211190904e-05
	780 1.58248992416787e-05
	781 1.57630753285787e-05
	782 1.57682831023465e-05
	783 1.57951096770859e-05
	784 1.58535026155704e-05
	785 1.59472314429365e-05
	786 1.60634858659137e-05
	787 1.60850987924732e-05
	788 1.60165314433414e-05
	789 1.5952188427093e-05
	790 1.58963931475853e-05
	791 1.58217116048576e-05
	792 1.57480624967121e-05
	793 1.5744665255113e-05
	794 1.5760804094711e-05
	795 1.57964674922795e-05
	796 1.57571916616916e-05
	797 1.56662935442142e-05
	798 1.55990215482404e-05
	799 1.55420943617202e-05
	800 1.55498371334772e-05
	801 1.55553536718145e-05
	802 1.56087405578376e-05
	803 1.57193914809284e-05
	804 1.5833559778855e-05
	805 1.58942040417998e-05
	806 1.58353477548445e-05
	807 1.57310512101105e-05
	808 1.56431827793568e-05
	809 1.55964180770241e-05
	810 1.55471933567242e-05
	811 1.55341509433526e-05
	812 1.55502860206269e-05
	813 1.55424853698349e-05
	814 1.55398607404322e-05
	815 1.55061678057677e-05
	816 1.5402522388186e-05
	817 1.5324891800006e-05
	818 1.52986519097453e-05
	819 1.53321649438709e-05
	820 1.54023583629481e-05
	821 1.55113945936591e-05
	822 1.56407709042128e-05
	823 1.56728052260746e-05
	824 1.55889794513087e-05
	825 1.5514839194708e-05
	826 1.54334326776251e-05
	827 1.53490521590527e-05
	828 1.52643841140332e-05
	829 1.52944118614329e-05
	830 1.53274679313142e-05
	831 1.53222539491793e-05
	832 1.52606605254846e-05
	833 1.52334864313985e-05
	834 1.5163838254395e-05
	835 1.50919446433306e-05
	836 1.50803785441411e-05
	837 1.51209614180914e-05
	838 1.51899287388346e-05
	839 1.52852487627086e-05
	840 1.53672770188962e-05
	841 1.53885133702403e-05
	842 1.53788720993475e-05
	843 1.53288155644304e-05
	844 1.5238439283749e-05
	845 1.51319490147017e-05
	846 1.50551825921497e-05
	847 1.50890168839624e-05
	848 1.51299856288034e-05
	849 1.5119163043309e-05
	850 1.50794348083849e-05
	851 1.50508883933753e-05
	852 1.50052154523905e-05
	853 1.49052982933995e-05
	854 1.48771469614672e-05
	855 1.49030067220224e-05
	856 1.49662900206238e-05
	857 1.50716534133011e-05
	858 1.5191179201679e-05
	859 1.52506197554914e-05
	860 1.52621056024316e-05
	861 1.52087347338181e-05
	862 1.50933899583094e-05
	863 1.49857608313653e-05
	864 1.48818707668141e-05
	865 1.48967558146751e-05
	866 1.49604155676997e-05
	867 1.49590713238723e-05
	868 1.4903187333104e-05
	869 1.48549558911526e-05
	870 1.478751561379e-05
	871 1.46992105261035e-05
	872 1.46815741484119e-05
	873 1.47284312888019e-05
	874 1.47888673911467e-05
	875 1.48965572090987e-05
	876 1.50064752677537e-05
	877 1.50227012909454e-05
	878 1.49480622493314e-05
	879 1.48666429424793e-05
	880 1.47811909076978e-05
	881 1.47672866539494e-05
	882 1.47325699568768e-05
	883 1.47076277539071e-05
	884 1.47076015313274e-05
	885 1.46890172398884e-05
	886 1.46784179229087e-05
	887 1.46224875692269e-05
	888 1.45308725318927e-05
	889 1.44936585293465e-05
	890 1.44897254248377e-05
	891 1.45357141398961e-05
	892 1.46073546432035e-05
	893 1.4684204629134e-05
	894 1.4762904179122e-05
	895 1.48142341949153e-05
	896 1.48163359803277e-05
	897 1.47260216216694e-05
	898 1.46282883211057e-05
	899 1.45326831120229e-05
	900 1.44648918638524e-05
	901 1.44831005712831e-05
	902 1.45424078201728e-05
	903 1.45389048533673e-05
	904 1.44837316771174e-05
	905 1.44493167475446e-05
	906 1.43880637750726e-05
	907 1.43015600766461e-05
	908 1.42793138468988e-05
	909 1.43109689076937e-05
	910 1.43877119942459e-05
	911 1.45015602939225e-05
	912 1.4641025510187e-05
	913 1.47139616029079e-05
	914 1.46905814553122e-05
	915 1.46058933534476e-05
	916 1.45076810884603e-05
	917 1.43963099965561e-05
	918 1.42927238862001e-05
	919 1.42973446273231e-05
	920 1.43705189348253e-05
	921 1.44344755259596e-05
	922 1.43787418540597e-05
	923 1.42932764539694e-05
	924 1.41978550587396e-05
	925 1.41246790779093e-05
	926 1.40789428204613e-05
	927 1.40817507947588e-05
	928 1.41664781949657e-05
	929 1.42654072536885e-05
	930 1.44210528461031e-05
	931 1.44989266814832e-05
	932 1.45254712684562e-05
	933 1.44392619141342e-05
	934 1.43051709198261e-05
	935 1.42081386194093e-05
	936 1.41505522108787e-05
	937 1.41165678133603e-05
	938 1.41218476628069e-05
	939 1.41400621345156e-05
	940 1.41470991934511e-05
	941 1.40755204771636e-05
	942 1.39872046478473e-05
	943 1.39072913789562e-05
	944 1.38757550036672e-05
	945 1.38749159415141e-05
	946 1.39512980066314e-05
	947 1.40481024004657e-05
	948 1.41598350786865e-05
	949 1.42561835332344e-05
	950 1.4271342868355e-05
	951 1.41573861291988e-05
	952 1.40085547535307e-05
	953 1.39266721745912e-05
	954 1.39071006515223e-05
	955 1.38864072667388e-05
	956 1.38717177886605e-05
	957 1.38899309831686e-05
	958 1.38948374353021e-05
	959 1.38698524452252e-05
	960 1.37714261967403e-05
	961 1.37038974834525e-05
	962 1.36623751068043e-05
	963 1.36646208694913e-05
	964 1.37117586724855e-05
	965 1.38039241299737e-05
	966 1.39266323220255e-05
	967 1.40197872635639e-05
	968 1.40441887097609e-05
	969 1.39942110770619e-05
	970 1.39095687341495e-05
	971 1.38386752226438e-05
	972 1.37530454527024e-05
	973 1.36755393178589e-05
	974 1.36621521704683e-05
	975 1.36863051398706e-05
	976 1.37459830771292e-05
	977 1.37290899679954e-05
	978 1.36427240047254e-05
	979 1.35462980530576e-05
	980 1.34844382699839e-05
	981 1.34347911320276e-05
	982 1.34586251974156e-05
	983 1.35288234694997e-05
	984 1.36133454873288e-05
	985 1.37435141116526e-05
	986 1.38688779909302e-05
	987 1.38796095594529e-05
	988 1.3810680028925e-05
	989 1.36708260782825e-05
	990 1.35569672856661e-05
	991 1.34916566878474e-05
	992 1.3432104795541e-05
	993 1.34773610334094e-05
	994 1.35156995888508e-05
	995 1.35033980477317e-05
	996 1.34563171148017e-05
	997 1.3399441892048e-05
	998 1.33250704799082e-05
	999 1.32427091390497e-05
	1000 1.32131908348043e-05
	1001 1.32625954609722e-05
	1002 1.33378879656476e-05
	1003 1.34553077550947e-05
	1004 1.35702860752929e-05
	1005 1.36954800442624e-05
	1006 1.36770723226931e-05
	1007 1.35528726090683e-05
	1008 1.34174547827826e-05
	1009 1.3325331059022e-05
	1010 1.32533315539973e-05
	1011 1.32112191071343e-05
	1012 1.32639130683287e-05
	1013 1.32824953720245e-05
	1014 1.32532785297457e-05
	1015 1.32122109821609e-05
	1016 1.3144923991959e-05
	1017 1.30555835990975e-05
	1018 1.2993731766997e-05
	1019 1.29931835779473e-05
	1020 1.30704198326725e-05
	1021 1.31298884991082e-05
	1022 1.32598552600172e-05
	1023 1.33697535922295e-05
	1024 1.34175938608649e-05
	1025 1.33826497688005e-05
	1026 1.32790452465059e-05
	1027 1.31491197077338e-05
	1028 1.30696384363915e-05
	1029 1.30477235167348e-05
	1030 1.30116981473805e-05
	1031 1.30282091692635e-05
	1032 1.30200966932392e-05
	1033 1.30112034515406e-05
	1034 1.29553982404218e-05
	1035 1.2886220366326e-05
	1036 1.28007692090648e-05
	1037 1.27605915274387e-05
	1038 1.2792173274434e-05
	1039 1.28506821486241e-05
	1040 1.29365934355974e-05
	1041 1.30600113781298e-05
	1042 1.31809726724441e-05
	1043 1.32009747542128e-05
	1044 1.31657986450762e-05
	1045 1.30499095210013e-05
	1046 1.29289134793709e-05
	1047 1.28425303156376e-05
	1048 1.28163311874374e-05
	1049 1.2813849005866e-05
	1050 1.28189939037071e-05
	1051 1.28339882543571e-05
	1052 1.28389463132095e-05
	1053 1.27723754843601e-05
	1054 1.26555774357939e-05
	1055 1.2568425496795e-05
	1056 1.25325669486642e-05
	1057 1.25695928687719e-05
	1058 1.26376415465757e-05
	1059 1.27506847658054e-05
	1060 1.28749920698112e-05
	1061 1.29975845872821e-05
	1062 1.2994361059171e-05
	1063 1.291862239583e-05
	1064 1.2751875015482e-05
	1065 1.26188878830646e-05
	1066 1.25842987213076e-05
	1067 1.25920554623349e-05
	1068 1.25928389280716e-05
	1069 1.25892822442353e-05
	1070 1.25894913534097e-05
	1071 1.25209603982768e-05
	1072 1.2444172034165e-05
	1073 1.23511359593209e-05
	1074 1.23044126354444e-05
	1075 1.23108981906483e-05
	1076 1.23482982807843e-05
	1077 1.24359190465384e-05
	1078 1.25319406194535e-05
	1079 1.26072432529156e-05
	1080 1.26777277351664e-05
	1081 1.26707882301957e-05
	1082 1.26317176505353e-05
	1083 1.25104357646677e-05
	1084 1.23716629261139e-05
	1085 1.2315888159975e-05
	1086 1.23312180893009e-05
	1087 1.23292095697281e-05
	1088 1.23835230070313e-05
	1089 1.23791248611838e-05
	1090 1.23186681957321e-05
	1091 1.22259738670749e-05
	1092 1.21516110382913e-05
	1093 1.20843467428244e-05
	1094 1.20580286502303e-05
	1095 1.2123727847424e-05
	1096 1.22051358353303e-05
	1097 1.233099760789e-05
	1098 1.24576180962066e-05
	1099 1.25103291210849e-05
	1100 1.24975388775539e-05
	1101 1.24172009048351e-05
	1102 1.22909737552135e-05
	1103 1.21536010411205e-05
	1104 1.2057850963032e-05
	1105 1.20831814385269e-05
	1106 1.21615390469287e-05
	1107 1.21606008924857e-05
	1108 1.21699965873745e-05
	1109 1.20987904175252e-05
	1110 1.19812705428046e-05
	1111 1.18853671366992e-05
	1112 1.18246430726998e-05
	1113 1.18593718880788e-05
	1114 1.19004758509789e-05
	1115 1.19591288303411e-05
	1116 1.20504314509162e-05
	1117 1.21571354725347e-05
	1118 1.22176298003041e-05
	1119 1.21806051804896e-05
	1120 1.20918766803868e-05
	1121 1.19913768159563e-05
	1122 1.1897251960491e-05
	1123 1.17980653620009e-05
	1124 1.17937749966046e-05
	1125 1.18401504707322e-05
	1126 1.19037175050352e-05
	1127 1.18718347046354e-05
	1128 1.18190761497772e-05
	1129 1.17307613756168e-05
	1130 1.16551739761661e-05
	1131 1.16010671149525e-05
	1132 1.15964274574054e-05
	1133 1.1632302163278e-05
	1134 1.17355116966422e-05
	1135 1.18866668721296e-05
	1136 1.20099864595602e-05
	1137 1.21086805187076e-05
	1138 1.21377122681565e-05
	1139 1.20403291106186e-05
	1140 1.18475792874051e-05
	1141 1.16606443434364e-05
	1142 1.16099673448389e-05
	1143 1.16710964661593e-05
	1144 1.17247805295762e-05
	1145 1.1743640328632e-05
	1146 1.17583923504583e-05
	1147 1.17099018162037e-05
	1148 1.16218059371676e-05
	1149 1.14881702657499e-05
	1150 1.14045838568444e-05
	1151 1.13809407729093e-05
	1152 1.144253785057e-05
	1153 1.1553515397722e-05
	1154 1.17000780726073e-05
	1155 1.18481534823189e-05
	1156 1.18838158442713e-05
	1157 1.1816454007274e-05
	1158 1.16557378895266e-05
	1159 1.1525285717795e-05
	1160 1.14302096427821e-05
	1161 1.14352354447789e-05
	1162 1.14279953624674e-05
	1163 1.14524524494897e-05
	1164 1.1491293591348e-05
	1165 1.15002280267618e-05
	1166 1.14189322832914e-05
	1167 1.13297114321398e-05
	1168 1.12461422716592e-05
	1169 1.12134125238583e-05
	1170 1.12202620830004e-05
	1171 1.12909174703191e-05
	1172 1.14092774303742e-05
	1173 1.15508455085234e-05
	1174 1.16182987976998e-05
	1175 1.16469765156779e-05
	1176 1.16237992919821e-05
	1177 1.15195022285519e-05
	1178 1.13683116342855e-05
	1179 1.12849990792085e-05
	1180 1.12682363404559e-05
	1181 1.12830965015576e-05
	1182 1.13023167234161e-05
	1183 1.13532769461244e-05
	1184 1.13418945062449e-05
	1185 1.12693651299622e-05
	1186 1.11643975699849e-05
	1187 1.10882870902884e-05
	1188 1.1024068056642e-05
	1189 1.10396087400488e-05
	1190 1.11169848402426e-05
	1191 1.12464321055938e-05
	1192 1.14050771884422e-05
	1193 1.1528603424793e-05
	1194 1.15443578376784e-05
	1195 1.14802607527054e-05
	1196 1.13533804864119e-05
	1197 1.12006040708934e-05
	1198 1.11214924185532e-05
	1199 1.10855135559973e-05
	1200 1.10999258460254e-05
	1201 1.11576618184728e-05
	1202 1.12118304453901e-05
	1203 1.11716417592334e-05
	1204 1.10841176823584e-05
	1205 1.09730775097461e-05
	1206 1.08946500709806e-05
	1207 1.08540408394475e-05
	1208 1.08804811382868e-05
	1209 1.09655550986076e-05
	1210 1.1103080092667e-05
	1211 1.12462652896994e-05
	1212 1.13429282055932e-05
	1213 1.13312891212303e-05
	1214 1.11989604114626e-05
	1215 1.10833564370694e-05
	1216 1.09747741454669e-05
	1217 1.09390825517863e-05
	1218 1.09311733194062e-05
	1219 1.09325207020561e-05
	1220 1.09463101392038e-05
	1221 1.09768016329781e-05
	1222 1.09546291664486e-05
	1223 1.09142917867899e-05
	1224 1.08160211294006e-05
	1225 1.07508976743276e-05
	1226 1.07039104815243e-05
	1227 1.07291584985347e-05
	1228 1.08053743428371e-05
	1229 1.09386926769872e-05
	1230 1.11036672372222e-05
	1231 1.12275972590226e-05
	1232 1.13012146290004e-05
	1233 1.12405001075899e-05
	1234 1.10771499706175e-05
	1235 1.09113983182141e-05
	1236 1.08175389481602e-05
	1237 1.08159330096669e-05
	1238 1.08899731934287e-05
	1239 1.09156807877042e-05
	1240 1.09703447019882e-05
	1241 1.09434103183759e-05
	1242 1.08619023713175e-05
	1243 1.07214461770866e-05
	1244 1.0627396145324e-05
	1245 1.05805057799557e-05
	1246 1.05980039020181e-05
	1247 1.06757232849475e-05
	1248 1.08136797383196e-05
	1249 1.09641999639365e-05
	1250 1.1050232028964e-05
	1251 1.10914854136723e-05
	1252 1.10134921289529e-05
	1253 1.08675821461901e-05
	1254 1.07358618741671e-05
	1255 1.06564442639012e-05
	1256 1.06891769213746e-05
	1257 1.07113988470076e-05
	1258 1.07327199678764e-05
	1259 1.07576741541493e-05
	1260 1.07131234141633e-05
	1261 1.06427036694612e-05
	1262 1.05488476567217e-05
	1263 1.04819931792122e-05
	1264 1.04530658742163e-05
	1265 1.0490795807172e-05
	1266 1.05686919837922e-05
	1267 1.07217829121709e-05
	1268 1.08653576571527e-05
	1269 1.0979303958436e-05
	1270 1.10132771062865e-05
	1271 1.09565128099121e-05
	1272 1.08116688810611e-05
	1273 1.06673684747705e-05
	1274 1.05661878695429e-05
	1275 1.05409334398843e-05
	1276 1.05958762883773e-05
	1277 1.06714903491678e-05
	1278 1.06905415915293e-05
	1279 1.06536046278194e-05
	1280 1.05539047119407e-05
	1281 1.04509399267982e-05
	1282 1.03662206871746e-05
	1283 1.03495996484071e-05
	1284 1.03879286950104e-05
	1285 1.05025172434381e-05
	1286 1.0652671434741e-05
	1287 1.08152571396403e-05
	1288 1.09179663176917e-05
	1289 1.09062729549692e-05
	1290 1.08041487241195e-05
	1291 1.06353433597661e-05
	1292 1.0464560016743e-05
	1293 1.04099956814707e-05
	1294 1.04401461662462e-05
	1295 1.05350397703319e-05
	1296 1.05748800880434e-05
	1297 1.05404441725909e-05
	1298 1.04728168377477e-05
	1299 1.03791940606612e-05
	1300 1.0306971855556e-05
	1301 1.02407028812479e-05
	1302 1.02548789033818e-05
	1303 1.03108423523679e-05
	1304 1.04476807329945e-05
	1305 1.06025531181331e-05
	1306 1.07337974988297e-05
	1307 1.08015487452207e-05
	1308 1.0725226076147e-05
	1309 1.05896411142226e-05
	1310 1.04279052433043e-05
	1311 1.03425899720122e-05
	1312 1.03494531984438e-05
	1313 1.03941615101633e-05
	1314 1.04238047331506e-05
	1315 1.04365899247227e-05
	1316 1.04236171267758e-05
	1317 1.03458111500032e-05
	1318 1.02731371249831e-05
	1319 1.01797798084391e-05
	1320 1.01399003362701e-05
	1321 1.01636964711105e-05
	1322 1.02571752478298e-05
	1323 1.03842718495173e-05
	1324 1.05413459312587e-05
	1325 1.06965720085839e-05
	1326 1.07528534289969e-05
	1327 1.066884329326e-05
	1328 1.05080682235581e-05
	1329 1.03459963742836e-05
	1330 1.02679539804029e-05
	1331 1.02743214185352e-05
	1332 1.03278147367547e-05
	1333 1.03736609382565e-05
	1334 1.03997031857261e-05
	1335 1.03881889437218e-05
	1336 1.02794171112919e-05
	1337 1.01879077227807e-05
	1338 1.00778144584979e-05
	1339 1.00479854800994e-05
	1340 1.00830510447736e-05
	1341 1.01867669410893e-05
	1342 1.03284803163461e-05
	1343 1.04972793035785e-05
	1344 1.06292429347121e-05
	1345 1.06239497839056e-05
	1346 1.05121130644648e-05
	1347 1.03575218677321e-05
	1348 1.0216292096743e-05
	1349 1.01585540921434e-05
	1350 1.01787445565549e-05
	1351 1.02002775843602e-05
	1352 1.02702276514322e-05
	1353 1.0258891210313e-05
	1354 1.02266594463885e-05
	1355 1.01141800623594e-05
	1356 1.00437679346754e-05
	1357 9.97256940671321e-06
	1358 9.96208056847081e-06
	1359 1.00049113012091e-05
	1360 1.0117716964686e-05
	1361 1.02682240505914e-05
	1362 1.04656744532861e-05
	1363 1.05331180844104e-05
	1364 1.05090396793628e-05
	1365 1.04055808058945e-05
	1366 1.02680150195766e-05
	1367 1.00990553910663e-05
	1368 1.00514744225677e-05
	1369 1.00762287349454e-05
	1370 1.01644505345888e-05
	1371 1.02215795685368e-05
	1372 1.01866545030305e-05
	1373 1.01590178456235e-05
	1374 1.00395933930741e-05
	1375 9.94653211527918e-06
	1376 9.87501523752599e-06
	1377 9.87700681598369e-06
	1378 9.95075270360246e-06
	1379 1.00772733979682e-05
	1380 1.02235480898827e-05
	1381 1.04092822379442e-05
	1382 1.05075627736539e-05
	1383 1.04223590522423e-05
	1384 1.02727057011975e-05
	1385 1.00924006343206e-05
	1386 9.98973203714115e-06
	1387 9.97654940704251e-06
	1388 1.00405601148879e-05
	1389 1.0079237174665e-05
	1390 1.01332358486417e-05
	1391 1.00937596663186e-05
	1392 1.00177230564213e-05
	1393 9.88769803100809e-06
	1394 9.81191090687616e-06
	1395 9.75469441755195e-06
	1396 9.79643522036611e-06
	1397 9.8927180456343e-06
	1398 1.00249358592208e-05
	1399 1.01760383302008e-05
	1400 1.02891107438552e-05
	1401 1.03122387677956e-05
	1402 1.02545769937734e-05
	1403 1.01141449508901e-05
	1404 9.93085651224135e-06
	1405 9.85022647803646e-06
	1406 9.84989269525727e-06
	1407 9.9095389849424e-06
	1408 9.96532586405863e-06
	1409 1.00016491462185e-05
	1410 9.97419908799202e-06
	1411 9.89688675545608e-06
	1412 9.78985236343988e-06
	1413 9.73617927435555e-06
	1414 9.69821185670128e-06
	1415 9.70519277210258e-06
	1416 9.78680818874977e-06
	1417 9.93821258887806e-06
	1418 1.01121738289578e-05
	1419 1.02776396389004e-05
	1420 1.03574086267599e-05
	1421 1.02354834119467e-05
	1422 1.00450708089284e-05
	1423 9.86874844244312e-06
	1424 9.77627623655053e-06
	1425 9.78022107922527e-06
	1426 9.83885979799481e-06
	1427 9.92718126546777e-06
	1428 9.9520761889238e-06
	1429 9.92620857509507e-06
	1430 9.81726643090042e-06
	1431 9.69239025749857e-06
	1432 9.6035604535416e-06
	1433 9.56887993730504e-06
	1434 9.61303374324984e-06
	1435 9.75398753055856e-06
	1436 9.88017282388398e-06
	1437 1.0018479107643e-05
	1438 1.0133179548788e-05
	1439 1.01167604871222e-05
	1440 1.00028247942419e-05
	1441 9.8414709164274e-06
	1442 9.71116764247881e-06
	1443 9.68121259781185e-06
	1444 9.65653931928045e-06
	1445 9.70900525842922e-06
	1446 9.74350342453079e-06
	1447 9.75436652161932e-06
	1448 9.73223335698492e-06
	1449 9.66275018399188e-06
	1450 9.57709555393649e-06
	1451 9.5051630282228e-06
	1452 9.4935370533733e-06
	1453 9.51717215258441e-06
	1454 9.62886064925783e-06
	1455 9.76714175315863e-06
	1456 9.95455536667578e-06
	1457 1.01502090998906e-05
	1458 1.01658181712594e-05
	1459 1.00670710967421e-05
	1460 9.88201268015132e-06
	1461 9.70246406595265e-06
	1462 9.62522934067067e-06
	1463 9.64336829056833e-06
	1464 9.67739966384329e-06
	1465 9.76599433499814e-06
	1466 9.82121955850346e-06
	1467 9.80073691003724e-06
	1468 9.69436891296027e-06
	1469 9.58797511430021e-06
	1470 9.46184431249719e-06
	1471 9.41659204301004e-06
	1472 9.43763746263926e-06
	1473 9.53141741533159e-06
	1474 9.67936165885419e-06
	1475 9.83472293825116e-06
	1476 9.98738122603982e-06
	1477 1.0060267738865e-05
	1478 9.92963809132164e-06
	1479 9.75964506721994e-06
	1480 9.6009138257358e-06
	1481 9.50635137719757e-06
	1482 9.51001281634944e-06
	1483 9.57347557140054e-06
	1484 9.62526880954329e-06
	1485 9.65467133262621e-06
	1486 9.65191495083673e-06
	1487 9.55312007100417e-06
	1488 9.4565859942719e-06
	1489 9.37264896094803e-06
	1490 9.3464664434606e-06
	1491 9.38774235237361e-06
	1492 9.46851884364719e-06
	1493 9.59216246698702e-06
	1494 9.76822022380475e-06
	1495 9.9203355325983e-06
	1496 1.00099084328065e-05
	1497 9.88797049927825e-06
	1498 9.71835784824293e-06
	1499 9.54750261605852e-06
	1500 9.45597204271564e-06
	1501 9.44485189968702e-06
	1502 9.51142065552801e-06
	1503 9.56941319607552e-06
	1504 9.6044531758821e-06
	1505 9.61050128189811e-06
	1506 9.53474908627072e-06
	1507 9.42666209446941e-06
	1508 9.32015965027233e-06
	1509 9.26731882167076e-06
	1510 9.2939851832341e-06
	1511 9.37532133704622e-06
	1512 9.5433471933859e-06
	1513 9.70769081298783e-06
	1514 9.84313633090039e-06
	1515 9.90434119962913e-06
	1516 9.83742861926373e-06
	1517 9.66220217613056e-06
	1518 9.49286741125377e-06
	1519 9.38298168584595e-06
	1520 9.35430302284601e-06
	1521 9.41787210528844e-06
	1522 9.51296523510337e-06
	1523 9.51434047280486e-06
	1524 9.50613509864695e-06
	1525 9.44706074967883e-06
	1526 9.3396736975393e-06
	1527 9.25399901596791e-06
	1528 9.21152650334989e-06
	1529 9.21097998052289e-06
	1530 9.28589448712103e-06
	1531 9.42320437857802e-06
	1532 9.58731131817103e-06
	1533 9.7880380014459e-06
	1534 9.92673744626416e-06
	1535 9.86378449852054e-06
	1536 9.68744435958513e-06
	1537 9.48890751928388e-06
	1538 9.34890671366873e-06
	1539 9.32330220315691e-06
	1540 9.36261177031383e-06
	1541 9.46633648624129e-06
	1542 9.52552611011015e-06
	1543 9.54690119137069e-06
	1544 9.48406084333442e-06
	1545 9.36735982293158e-06
	1546 9.23418874876347e-06
	1547 9.1459843982733e-06
	1548 9.13792904633226e-06
	1549 9.21268397569008e-06
	1550 9.38632653912919e-06
	1551 9.5518493914426e-06
	1552 9.68527466049807e-06
	1553 9.7609250566677e-06
	1554 9.72087397954624e-06
	1555 9.55773416322359e-06
	1556 9.38688050311498e-06
	1557 9.25605140267294e-06
	1558 9.24547663316844e-06
	1559 9.28687256873673e-06
	1560 9.31393089054211e-06
	1561 9.36755159308689e-06
	1562 9.37724722049893e-06
	1563 9.30655214226306e-06
	1564 9.21385760577209e-06
	1565 9.12416216269918e-06
	1566 9.08244413100334e-06
	1567 9.07953738682465e-06
	1568 9.14956106345954e-06
	1569 9.25465148338134e-06
	1570 9.41167300450019e-06
	1571 9.59526289001644e-06
	1572 9.75689596227625e-06
	1573 9.71088137902143e-06
	1574 9.54074591774656e-06
	1575 9.3567872667677e-06
	1576 9.2317064508407e-06
	1577 9.17185411708488e-06
	1578 9.22040157824711e-06
	1579 9.30268667787004e-06
	1580 9.37625993024938e-06
	1581 9.40843020202919e-06
	1582 9.36290828867925e-06
	1583 9.25189363787382e-06
	1584 9.12020640608802e-06
	1585 9.04146428837294e-06
	1586 9.00989683394471e-06
	1587 9.07802277438918e-06
	1588 9.20695495842949e-06
	1589 9.40390455461682e-06
	1590 9.62376406032917e-06
	1591 9.79383518107113e-06
	1592 9.72858792014364e-06
	1593 9.54081349391345e-06
	1594 9.30688559641624e-06
	1595 9.162874415658e-06
	1596 9.13689948234264e-06
	1597 9.1867284695013e-06
	1598 9.23591058032969e-06
	1599 9.30681522604004e-06
	1600 9.29948085115484e-06
	1601 9.22003872538824e-06
	1602 9.11014750659689e-06
	1603 8.99844316837317e-06
	1604 8.95191077265878e-06
	1605 8.97419678391742e-06
	1606 9.04470523366285e-06
	1607 9.17424817803436e-06
	1608 9.30500639562126e-06
	1609 9.44954716874946e-06
	1610 9.50879101413449e-06
	1611 9.4570650359671e-06
	1612 9.31185048536065e-06
	1613 9.17653321597101e-06
	1614 9.07834905383709e-06
	1615 9.06309725579035e-06
	1616 9.05563400621645e-06
	1617 9.10948646826171e-06
	1618 9.16403297779311e-06
	1619 9.18410494499255e-06
	1620 9.15205462881374e-06
	1621 9.08760321038926e-06
	1622 8.9939774285952e-06
	1623 8.91020151527755e-06
	1624 8.88732843051798e-06
	1625 8.94288867225868e-06
	1626 9.05942128248682e-06
	1627 9.2580079913418e-06
	1628 9.51864048559514e-06
	1629 9.77798902557225e-06
	1630 9.81106410247179e-06
	1631 9.6123908246426e-06
	1632 9.34020972032101e-06
	1633 9.10096397443994e-06
	1634 9.02509068367863e-06
	1635 9.09593275544296e-06
	1636 9.21595743541559e-06
	1637 9.30556967837504e-06
	1638 9.32062290637248e-06
	1639 9.21504157602726e-06
	1640 9.05170454945647e-06
	1641 8.91469987518434e-06
	1642 8.83999389333923e-06
	1643 8.85361824032316e-06
	1644 8.93799380374816e-06
	1645 9.05819821728215e-06
	1646 9.18548759276661e-06
	1647 9.29692247719771e-06
	1648 9.30835952939191e-06
	1649 9.28618166184947e-06
	1650 9.13388613188459e-06
	1651 8.99247919683432e-06
	1652 8.90974431655422e-06
	1653 8.91126230317951e-06
	1654 8.92275403607812e-06
	1655 8.97322921744603e-06
	1656 9.0047469525345e-06
	1657 8.99179092606062e-06
	1658 8.94657681094202e-06
	1659 8.88756821471048e-06
	1660 8.82314669059525e-06
	1661 8.7743889638503e-06
	1662 8.79148685939413e-06
	1663 8.84996631089763e-06
	1664 8.98314062336425e-06
	1665 9.16292549302256e-06
	1666 9.39499958185763e-06
	1667 9.63570299994387e-06
	1668 9.62438293505841e-06
	1669 9.43070719650052e-06
	1670 9.17591207372226e-06
	1671 8.97032414748367e-06
	1672 8.92536376984054e-06
	1673 9.00868429454249e-06
	1674 9.15409197332906e-06
	1675 9.26443740389971e-06
	1676 9.28558743229502e-06
	1677 9.15384983368739e-06
	1678 8.96497574753852e-06
	1679 8.8148096901719e-06
	1680 8.73452144212195e-06
	1681 8.76416619721709e-06
	1682 8.85776131998028e-06
	1683 9.01227598593124e-06
	1684 9.15900236897471e-06
	1685 9.29011108841138e-06
	1686 9.31767385115023e-06
	1687 9.16614221768697e-06
	1688 8.98161250351137e-06
	1689 8.85809315143149e-06
	1690 8.80176544981737e-06
	1691 8.81996444856981e-06
	1692 8.85732803102002e-06
	1693 8.91663114721553e-06
	1694 8.91543290038044e-06
	1695 8.88424205491845e-06
	1696 8.82074910890651e-06
	1697 8.73961773173448e-06
	1698 8.68538804077446e-06
	1699 8.67830620077115e-06
	1700 8.72952228547774e-06
	1701 8.83932816186928e-06
	1702 8.98015306027844e-06
	1703 9.18219234069539e-06
	1704 9.39304174174538e-06
	1705 9.39874967009757e-06
	1706 9.25816144992098e-06
	1707 9.05800650130573e-06
	1708 8.87095746815447e-06
	1709 8.794404247503e-06
	1710 8.83934491202609e-06
	1711 8.94588829858378e-06
	1712 9.07585973841663e-06
	1713 9.12466381475241e-06
	1714 9.05710333842791e-06
	1715 8.92213044512147e-06
	1716 8.77688242173491e-06
	1717 8.64997178862126e-06
	1718 8.63654534200009e-06
	1719 8.6945269632821e-06
	1720 8.83233042259235e-06
	1721 9.02789108803148e-06
	1722 9.25159440967604e-06
	1723 9.37757593266753e-06
	1724 9.30083269601312e-06
	1725 9.07140160144593e-06
	1726 8.85280440066083e-06
	1727 8.74136414363846e-06
	1728 8.7294197612664e-06
	1729 8.79511004647782e-06
	1730 8.88569401702455e-06
	1731 8.91318873730285e-06
	1732 8.88316574521042e-06
	1733 8.7952832190652e-06
	1734 8.68539603260388e-06
	1735 8.6076688070591e-06
	1736 8.57455706348986e-06
	1737 8.60263228652514e-06
	1738 8.71168353278051e-06
	1739 8.84550900792647e-06
	1740 8.99291095635135e-06
	1741 9.13864926310737e-06
	1742 9.18547191108843e-06
	1743 9.07926371773726e-06
	1744 8.90249494034379e-06
	1745 8.75128627608746e-06
	1746 8.66819852962664e-06
	1747 8.68604971149267e-06
	1748 8.73681707247442e-06
	1749 8.84040594861801e-06
	1750 8.88843706281506e-06
	1751 8.86384404097385e-06
	1752 8.78578319785106e-06
	1753 8.67282250904822e-06
	1754 8.5639179969732e-06
	1755 8.53099286235448e-06
	1756 8.55988153869447e-06
	1757 8.65981525866033e-06
	1758 8.84783918220222e-06
	1759 9.06281668466846e-06
	1760 9.27538387784921e-06
	1761 9.40486268152796e-06
	1762 9.2127841551104e-06
	1763 8.93780452404513e-06
	1764 8.71285139680822e-06
	1765 8.64719500714983e-06
	1766 8.71112615197234e-06
	1767 8.82005077684767e-06
	1768 8.91993771023891e-06
	1769 8.93760036113633e-06
	1770 8.84835367820358e-06
	1771 8.71725299234072e-06
	1772 8.5719232174597e-06
	1773 8.49482791842604e-06
	1774 8.49422539506151e-06
	1775 8.57027784739728e-06
	1776 8.69587759133594e-06
	1777 8.84687795288386e-06
	1778 8.92969514421083e-06
	1779 8.95690143654093e-06
	1780 8.8851366433218e-06
	1781 8.7532560355541e-06
	1782 8.61686742759105e-06
	1783 8.5553801980609e-06
	1784 8.55627643669976e-06
	1785 8.59444070222537e-06
	1786 8.62281844860036e-06
	1787 8.64385752574748e-06
	1788 8.65684668305278e-06
	1789 8.6087252473277e-06
	1790 8.55831238055771e-06
	1791 8.46990289105065e-06
	1792 8.42859457339529e-06
	1793 8.42960510372137e-06
	1794 8.50053270351481e-06
	1795 8.64006045198096e-06
	1796 8.84701750258898e-06
	1797 9.10702678247333e-06
	1798 9.31222946132237e-06
	1799 9.37825962932237e-06
	1800 9.20000032866852e-06
	1801 8.83487749536727e-06
	1802 8.62646798172051e-06
	1803 8.58358485888999e-06
	1804 8.71876609842559e-06
	1805 8.91350253784395e-06
	1806 9.02689129311796e-06
	1807 9.00565488137772e-06
	1808 8.84771961917608e-06
	1809 8.62788820921168e-06
	1810 8.45217826128675e-06
	1811 8.39436502619861e-06
	1812 8.43451697907227e-06
	1813 8.55981413039331e-06
	1814 8.69706288941785e-06
	1815 8.8022030038104e-06
	1816 8.82675711544323e-06
	1817 8.76953002393321e-06
	1818 8.63886858049057e-06
	1819 8.50790103257282e-06
	1820 8.43907483183415e-06
	1821 8.42588346294804e-06
	1822 8.43483806534095e-06
	1823 8.47224822209114e-06
	1824 8.50330140078626e-06
	1825 8.49672418556935e-06
	1826 8.45793861614652e-06
	1827 8.41636988813121e-06
	1828 8.35102350649208e-06
	1829 8.32504748693452e-06
	1830 8.31874772977415e-06
	1831 8.38796479207815e-06
	1832 8.51172482718709e-06
	1833 8.69322895002966e-06
	1834 8.90904443107132e-06
	1835 9.16422261187932e-06
	1836 9.13144531189403e-06
	1837 8.94673025442216e-06
	1838 8.69375940482087e-06
	1839 8.51119706624104e-06
	1840 8.47696784056495e-06
	1841 8.57884918481489e-06
	1842 8.77096800078192e-06
	1843 8.92081164760583e-06
	1844 8.9561737341981e-06
	1845 8.81212273728949e-06
	1846 8.59582798273806e-06
	1847 8.39162489718603e-06
	1848 8.28212464565325e-06
	1849 8.30823706188255e-06
	1850 8.4065961356572e-06
	1851 8.59711244949324e-06
	1852 8.73435305326353e-06
	1853 8.81221249304787e-06
	1854 8.7543874309759e-06
	1855 8.59796849628225e-06
	1856 8.45739052923733e-06
	1857 8.37094678285411e-06
	1858 8.28856264778466e-06
	1859 8.31768605014105e-06
	1860 8.37475874337912e-06
	1861 8.39826670517851e-06
	1862 8.39623101800413e-06
	1863 8.351635101711e-06
	1864 8.29071265329162e-06
	1865 8.22182437421048e-06
	1866 8.18083664633207e-06
	1867 8.17937513986067e-06
	1868 8.24139327892937e-06
	1869 8.33131489574868e-06
	1870 8.48115119911341e-06
	1871 8.64984888515608e-06
	1872 8.79615422189062e-06
	1873 8.81169452338071e-06
	1874 8.724736409782e-06
	1875 8.53158734681614e-06
	1876 8.3503564427545e-06
	1877 8.30490825176611e-06
	1878 8.36031345841093e-06
	1879 8.43780087222967e-06
	1880 8.57708232882004e-06
	1881 8.65352917500672e-06
	1882 8.60117189027676e-06
	1883 8.46680112687181e-06
	1884 8.28552932574667e-06
	1885 8.14016971251164e-06
	1886 8.08842807309418e-06
	1887 8.17005855324737e-06
	1888 8.36602950027299e-06
	1889 8.627394206151e-06
	1890 8.8933876360997e-06
	1891 8.98164739737695e-06
	1892 8.8184077835507e-06
	1893 8.55668107035967e-06
	1894 8.30782938443519e-06
	1895 8.14824608585951e-06
	1896 8.15792198238796e-06
	1897 8.22737769734516e-06
	1898 8.27898450594944e-06
	1899 8.27953821325167e-06
	1900 8.20460421646629e-06
	1901 8.0979241108281e-06
	1902 8.01709618158952e-06
	1903 7.97759643500484e-06
	1904 8.01298560038077e-06
	1905 8.0930563228776e-06
	1906 8.23843203345831e-06
	1907 8.36885824195832e-06
	1908 8.48423549282273e-06
	1909 8.513078243233e-06
	1910 8.4347289730502e-06
	1911 8.28061484003939e-06
	1912 8.13395291654473e-06
	1913 8.05281167703953e-06
	1914 8.03468950572039e-06
	1915 8.0825668469231e-06
	1916 8.14409393079529e-06
	1917 8.19393669271307e-06
	1918 8.18182849382509e-06
	1919 8.12473206490694e-06
	1920 8.02874197791681e-06
	1921 7.93268221066512e-06
	1922 7.89931655997833e-06
	1923 7.95179869061968e-06
	1924 8.08835672128083e-06
	1925 8.37264879649524e-06
	1926 8.71238528610263e-06
	1927 8.99291447176154e-06
	1928 9.05943506346318e-06
	1929 8.802939366781e-06
	1930 8.34631004664743e-06
	1931 8.07382426248893e-06
	1932 8.02791687970483e-06
	1933 8.12710995745647e-06
	1934 8.27270405157066e-06
	1935 8.32066730716008e-06
	1936 8.23416915718411e-06
	1937 8.08912963190522e-06
	1938 7.92200711430269e-06
	1939 7.83613895993795e-06
	1940 7.84662764719002e-06
	1941 7.94091704925393e-06
	1942 8.0681391940729e-06
	1943 8.20267429713084e-06
	1944 8.29547665226471e-06
	1945 8.29435842675252e-06
	1946 8.19462121182113e-06
	1947 8.06189295055049e-06
	1948 7.95068513337327e-06
	1949 7.871421466632e-06
	1950 7.85243839906968e-06
	1951 7.8695120864225e-06
	1952 7.90120725824295e-06
	1953 7.90775524528442e-06
	1954 7.88825395581227e-06
	1955 7.85145529480502e-06
	1956 7.8024984784264e-06
	1957 7.76904190402661e-06
	1958 7.76292625381814e-06
	1959 7.82905260532374e-06
	1960 7.95756637739231e-06
	1961 8.17193730817678e-06
	1962 8.43673384576249e-06
	1963 8.66143494437921e-06
	1964 8.71880335928665e-06
	1965 8.57346459959274e-06
	1966 8.24465181370471e-06
	1967 7.99055334965715e-06
	1968 7.89211923013511e-06
	1969 7.98381308531049e-06
	1970 8.1680735215528e-06
	1971 8.27081767429405e-06
	1972 8.31453027050344e-06
	1973 8.20227428111053e-06
	1974 8.00543598877823e-06
	1975 7.82652703978926e-06
	1976 7.73075564008963e-06
	1977 7.76265058988201e-06
	1978 7.89185739691334e-06
	1979 8.09775386478861e-06
	1980 8.31859079575281e-06
	1981 8.43747715784104e-06
	1982 8.38434150196576e-06
	1983 8.21376704962518e-06
	1984 8.00046367022134e-06
	1985 7.82932003939862e-06
	1986 7.77319392586406e-06
	1987 7.78034628012847e-06
	1988 7.83340567256374e-06
	1989 7.8616132856979e-06
	1990 7.84740370640691e-06
	1991 7.79507963333259e-06
	1992 7.74392471658558e-06
	1993 7.69294822422495e-06
	1994 7.66696639953324e-06
	1995 7.69773893516401e-06
	1996 7.77704193399842e-06
	1997 7.93356966255487e-06
	1998 8.14069489063485e-06
	1999 8.34626139756267e-06
};
\addlegendentry{Train}
\addplot [semithick, black]
table {%
	0 0.00714140757918358
	1 0.00712134689092636
	2 0.00710418215021491
	3 0.00708945654332638
	4 0.00707689020782709
	5 0.0070659420453012
	6 0.00705638714134693
	7 0.00704785063862801
	8 0.007040124386549
	9 0.00703308451920748
	10 0.00702664069831371
	11 0.00702061597257853
	12 0.0070145963691175
	13 0.00700833275914192
	14 0.00700145866721869
	15 0.00699351029470563
	16 0.00698360940441489
	17 0.00697133177891374
	18 0.00695657730102539
	19 0.00693771941587329
	20 0.00691054528579116
	21 0.00687075918540359
	22 0.00681291241198778
	23 0.00673530157655478
	24 0.00663501024246216
	25 0.00650888960808516
	26 0.00635754223912954
	27 0.00618060352280736
	28 0.00597624713554978
	29 0.00575656211003661
	30 0.00552649283781648
	31 0.00529965199530125
	32 0.00508131179958582
	33 0.0048745172098279
	34 0.00468392856419086
	35 0.0045097260735929
	36 0.00435045082122087
	37 0.00420262804254889
	38 0.00406397879123688
	39 0.00393431540578604
	40 0.00381129398010671
	41 0.00369356689043343
	42 0.00358001468703151
	43 0.00347085599787533
	44 0.00336514855735004
	45 0.00326427863910794
	46 0.00316862692125142
	47 0.0030770474113524
	48 0.00298960693180561
	49 0.0029033983591944
	50 0.00282140262424946
	51 0.00274400669150054
	52 0.00266958866268396
	53 0.00259847729466856
	54 0.00253055989742279
	55 0.00246562599204481
	56 0.00240237638354301
	57 0.00234196800738573
	58 0.00228367047384381
	59 0.00222746282815933
	60 0.00217358418740332
	61 0.00212155608460307
	62 0.002071124734357
	63 0.00202302448451519
	64 0.00197611353360116
	65 0.00193029851652682
	66 0.00188601471018046
	67 0.00184264488052577
	68 0.00180099159479141
	69 0.00176110374741256
	70 0.00172242883127183
	71 0.00168544193729758
	72 0.00164979277178645
	73 0.00161414791364223
	74 0.0015787819866091
	75 0.001540761673823
	76 0.00150278583168983
	77 0.00146745645906776
	78 0.00143337890040129
	79 0.0014012634055689
	80 0.00137124850880355
	81 0.00134229985997081
	82 0.00131467857863754
	83 0.00128814904019237
	84 0.00126249727327377
	85 0.00123777589760721
	86 0.00121262005995959
	87 0.00118347548414022
	88 0.00115342962089926
	89 0.00111787801142782
	90 0.0010866008233279
	91 0.00105809897650033
	92 0.00103116303216666
	93 0.00100483151618391
	94 0.000979414791800082
	95 0.000955695693846792
	96 0.000933589530177414
	97 0.000912638090085238
	98 0.000892223150003701
	99 0.000872278644237667
	100 0.000853098463267088
	101 0.00083532405551523
	102 0.000817646097857505
	103 0.000799435481894761
	104 0.000782708055339754
	105 0.000766993383876979
	106 0.000751742743887007
	107 0.000737319933250546
	108 0.000723465404007584
	109 0.000710147258359939
	110 0.000697301176842302
	111 0.000684833212289959
	112 0.000672512978781015
	113 0.000659968936815858
	114 0.00064758729422465
	115 0.00063553056679666
	116 0.000623887055553496
	117 0.000612650881521404
	118 0.000601868028752506
	119 0.000591121439356357
	120 0.000579759653192014
	121 0.000569266034290195
	122 0.000559358566533774
	123 0.000549703312572092
	124 0.000540440727490932
	125 0.000529967597685754
	126 0.000519770139362663
	127 0.000510016398038715
	128 0.000500593276228756
	129 0.00049098051385954
	130 0.000482380768517032
	131 0.000474054424557835
	132 0.00046624249080196
	133 0.000458509777672589
	134 0.000451051979325712
	135 0.000444006756879389
	136 0.000436992442701012
	137 0.000430330081144348
	138 0.000423562421929091
	139 0.000417002913309261
	140 0.000410294713219628
	141 0.00040349387563765
	142 0.000396898889448494
	143 0.000390650297049433
	144 0.000384817161830142
	145 0.00037916743895039
	146 0.000373630929971114
	147 0.00036846220609732
	148 0.000363238650606945
	149 0.000358159100869671
	150 0.000353346375050023
	151 0.000348446395946667
	152 0.000343833526130766
	153 0.000339183636242524
	154 0.000334414391545579
	155 0.000329547241562977
	156 0.000325102359056473
	157 0.000320584658766165
	158 0.000315502955345437
	159 0.000310562085360289
	160 0.000306093570543453
	161 0.000301750260405242
	162 0.00029769042157568
	163 0.000293237477308139
	164 0.000288660899968818
	165 0.00028161916998215
	166 0.000274920894298702
	167 0.000269311742158607
	168 0.000264262751443312
	169 0.000257559615420178
	170 0.000251845543971285
	171 0.000246141047682613
	172 0.000241326037212275
	173 0.00023640293511562
	174 0.000232333230087534
	175 0.000227991578867659
	176 0.000224149960558861
	177 0.000220197005546652
	178 0.000216830056160688
	179 0.000212958897463977
	180 0.000209500372875482
	181 0.000206222146516666
	182 0.000202600000193343
	183 0.000199861242435873
	184 0.000196325359866023
	185 0.000193552419659682
	186 0.000190665858099237
	187 0.000187662852113135
	188 0.000185156939551234
	189 0.000182125018909574
	190 0.000179826660314575
	191 0.00017684226622805
	192 0.000174738612258807
	193 0.000171668812981807
	194 0.000169134116731584
	195 0.000165220830240287
	196 0.000162016513058916
	197 0.000159061426529661
	198 0.000156266643898562
	199 0.000153816290549003
	200 0.000151511238072999
	201 0.000149176950799301
	202 0.000147046521306038
	203 0.000145109297591262
	204 0.000143140438012779
	205 0.000141571275889874
	206 0.000139614538056776
	207 0.000138022514875047
	208 0.000136220172862522
	209 0.000134633475681767
	210 0.000132814238895662
	211 0.000131099819554947
	212 0.000129618973005563
	213 0.000128006053273566
	214 0.000126429193187505
	215 0.000124868194689043
	216 0.000123503603390418
	217 0.000122045872558374
	218 0.000120711491035763
	219 0.000119469397759531
	220 0.000118137926619966
	221 0.000116892646474298
	222 0.000115583810838871
	223 0.000114407346700318
	224 0.00011321699275868
	225 0.000112140558485407
	226 0.000110956294520292
	227 0.000109870576125104
	228 0.000108768195786979
	229 0.000107736253994517
	230 0.000106780644273385
	231 0.000105698323750403
	232 0.000104747909063008
	233 0.00010376483987784
	234 0.000102756093838252
	235 0.000101897639979143
	236 0.000100944220321253
	237 0.000100061093689874
	238 9.91408596746624e-05
	239 9.82566198217683e-05
	240 9.73299465840682e-05
	241 9.65336439548992e-05
	242 9.56888325163163e-05
	243 9.48094602790661e-05
	244 9.40412282943726e-05
	245 9.32218827074394e-05
	246 9.24536798265763e-05
	247 9.17648067115806e-05
	248 9.0844310761895e-05
	249 9.00563463801518e-05
	250 8.93020915100351e-05
	251 8.86350098880939e-05
	252 8.79212966538034e-05
	253 8.71076554176398e-05
	254 8.64391913637519e-05
	255 8.57339255162515e-05
	256 8.50307333166711e-05
	257 8.42822555569001e-05
	258 8.36464096209966e-05
	259 8.29886557767168e-05
	260 8.24650123831816e-05
	261 8.14049926702864e-05
	262 8.00425405032001e-05
	263 7.85936717875302e-05
	264 7.72375424276106e-05
	265 7.59190515964292e-05
	266 7.47596495784819e-05
	267 7.37600421416573e-05
	268 7.28721352061257e-05
	269 7.19584131729789e-05
	270 7.11313623469323e-05
	271 7.04392659827136e-05
	272 6.96244969731197e-05
	273 6.8896850279998e-05
	274 6.83192483847961e-05
	275 6.74992988933809e-05
	276 6.69034488964826e-05
	277 6.63906175759621e-05
	278 6.56903284834698e-05
	279 6.51878508506343e-05
	280 6.46871194476262e-05
	281 6.40046782791615e-05
	282 6.35822871117853e-05
	283 6.30351059953682e-05
	284 6.25804386800155e-05
	285 6.21381404926069e-05
	286 6.1605685914401e-05
	287 6.11694413237274e-05
	288 6.07233450864442e-05
	289 6.02439031354152e-05
	290 5.97478065174073e-05
	291 5.93701988691464e-05
	292 5.88768125453498e-05
	293 5.84268491365947e-05
	294 5.8008710766444e-05
	295 5.76330639887601e-05
	296 5.72895442019217e-05
	297 5.68487266718876e-05
	298 5.64844704058487e-05
	299 5.62065470148809e-05
	300 5.56946579308715e-05
	301 5.55481892661192e-05
	302 5.51599623577204e-05
	303 5.47737108718138e-05
	304 5.45891743968241e-05
	305 5.41041190444957e-05
	306 5.38632666575722e-05
	307 5.35787075932603e-05
	308 5.31590085302014e-05
	309 5.29051285411697e-05
	310 5.26462426932994e-05
	311 5.23633389093447e-05
	312 5.20192625117488e-05
	313 5.18394263053779e-05
	314 5.13648956257384e-05
	315 5.12373953824863e-05
	316 5.09358615090605e-05
	317 5.06513715663459e-05
	318 5.05160860484466e-05
	319 5.00179412483703e-05
	320 4.98830486321822e-05
	321 4.96594329888467e-05
	322 4.93286788696423e-05
	323 4.91336504637729e-05
	324 4.88375117129181e-05
	325 4.86462340631988e-05
	326 4.84641095681582e-05
	327 4.81636852782685e-05
	328 4.80097041872796e-05
	329 4.77842804684769e-05
	330 4.74349290016107e-05
	331 4.73453910672106e-05
	332 4.70366794615984e-05
	333 4.69331789645366e-05
	334 4.66659512312617e-05
	335 4.64424410893116e-05
	336 4.61970303149428e-05
	337 4.60777919215616e-05
	338 4.574474223773e-05
	339 4.5637247239938e-05
	340 4.54107721452601e-05
	341 4.52103959105443e-05
	342 4.49937324447092e-05
	343 4.4804150093114e-05
	344 4.46058729721699e-05
	345 4.44503530161455e-05
	346 4.42180753452703e-05
	347 4.40594521933235e-05
	348 4.38448842032813e-05
	349 4.37430426245555e-05
	350 4.34787398262415e-05
	351 4.3225554691162e-05
	352 4.31787739216816e-05
	353 4.29653082392178e-05
	354 4.28029306931421e-05
	355 4.26517799496651e-05
	356 4.24606951128226e-05
	357 4.22681005147751e-05
	358 4.21658878622111e-05
	359 4.20474680140615e-05
	360 4.18224517488852e-05
	361 4.16939183196519e-05
	362 4.14825190091506e-05
	363 4.14501773775555e-05
	364 4.1232899093302e-05
	365 4.10607135563623e-05
	366 4.10058455599938e-05
	367 4.08093874284532e-05
	368 4.06936342187691e-05
	369 4.05790451623034e-05
	370 4.04726670240052e-05
	371 4.02906989620533e-05
	372 4.01832403440494e-05
	373 4.00776661990676e-05
	374 3.99242526327725e-05
	375 3.9776543417247e-05
	376 3.9638875023229e-05
	377 3.95296192436945e-05
	378 3.94529197365046e-05
	379 3.93177942896727e-05
	380 3.92441907024477e-05
	381 3.91098001273349e-05
	382 3.89837332477327e-05
	383 3.88305488741025e-05
	384 3.87830441468395e-05
	385 3.86657520721201e-05
	386 3.85573075618595e-05
	387 3.83999686164316e-05
	388 3.84224113076925e-05
	389 3.8168924220372e-05
	390 3.80479577870574e-05
	391 3.79217490262818e-05
	392 3.78386866941582e-05
	393 3.78097865905147e-05
	394 3.76357529603411e-05
	395 3.75385861843824e-05
	396 3.74830597138498e-05
	397 3.72832764696795e-05
	398 3.72342619812116e-05
	399 3.70752022718079e-05
	400 3.69762819900643e-05
	401 3.69059744116385e-05
	402 3.68483524653129e-05
	403 3.67012253263965e-05
	404 3.65935993613675e-05
	405 3.65636515198275e-05
	406 3.6429577448871e-05
	407 3.63157232641242e-05
	408 3.62805112672504e-05
	409 3.61398379027378e-05
	410 3.60914382326882e-05
	411 3.59986443072557e-05
	412 3.58548859367147e-05
	413 3.58359393430874e-05
	414 3.56832497345749e-05
	415 3.56734381057322e-05
	416 3.55861702701077e-05
	417 3.54896037606522e-05
	418 3.53836876456626e-05
	419 3.5259155993117e-05
	420 3.52907554770354e-05
	421 3.50940172211267e-05
	422 3.5094992199447e-05
	423 3.49663678207435e-05
	424 3.49385372828692e-05
	425 3.48491848853882e-05
	426 3.47889217664488e-05
	427 3.47696113749407e-05
	428 3.45345579262357e-05
	429 3.46293018083088e-05
	430 3.44720901921391e-05
	431 3.44489126291592e-05
	432 3.43305582646281e-05
	433 3.43302854162175e-05
	434 3.42790517606772e-05
	435 3.41260711138602e-05
	436 3.42130624630954e-05
	437 3.41345657943748e-05
	438 3.40024344041012e-05
	439 3.39074467774481e-05
	440 3.38462414219975e-05
	441 3.37190649588592e-05
	442 3.36389421136118e-05
	443 3.35869408445433e-05
	444 3.33803145622369e-05
	445 3.35884069500025e-05
	446 3.33171155944001e-05
	447 3.33103744196706e-05
	448 3.32602539856452e-05
	449 3.30978837155271e-05
	450 3.29828835674562e-05
	451 3.30081820720807e-05
	452 3.29526483255904e-05
	453 3.28717578668147e-05
	454 3.28231180901639e-05
	455 3.27523994201329e-05
	456 3.26618974213488e-05
	457 3.25872351822909e-05
	458 3.25088076351676e-05
	459 3.23987915180624e-05
	460 3.25185646943282e-05
	461 3.22583146044053e-05
	462 3.22441628668457e-05
	463 3.21413717756514e-05
	464 3.21206644002814e-05
	465 3.19376740662847e-05
	466 3.19124555971939e-05
	467 3.19307255267631e-05
	468 3.1757870601723e-05
	469 3.18000020342879e-05
	470 3.17251542583108e-05
	471 3.16093610308599e-05
	472 3.15531397063751e-05
	473 3.15697725454811e-05
	474 3.15369397867471e-05
	475 3.13974342134316e-05
	476 3.13416057906579e-05
	477 3.11925577989314e-05
	478 3.09798233502079e-05
	479 3.10672949126456e-05
	480 3.08953203784768e-05
	481 3.09513561660424e-05
	482 3.08562048303429e-05
	483 3.07889422401786e-05
	484 3.07334412354976e-05
	485 3.05755747831427e-05
	486 3.05714784190059e-05
	487 3.05682369798888e-05
	488 3.02908356388798e-05
	489 3.03200795315206e-05
	490 3.03850883938139e-05
	491 3.02750340779312e-05
	492 3.01239360851469e-05
	493 3.02440985251451e-05
	494 3.00975843856577e-05
	495 3.00547471852042e-05
	496 3.00780648103682e-05
	497 2.99885286949575e-05
	498 2.98997492791386e-05
	499 2.99201456073206e-05
	500 2.97255483019399e-05
	501 2.96558609989006e-05
	502 2.96259568131063e-05
	503 2.95189365715487e-05
	504 2.94548972306075e-05
	505 2.94441997539252e-05
	506 2.92375407298096e-05
	507 2.92443328362424e-05
	508 2.9101713153068e-05
	509 2.89948275167262e-05
	510 2.90469633910106e-05
	511 2.90202915493865e-05
	512 2.88894534605788e-05
	513 2.88745341094909e-05
	514 2.88677201751852e-05
	515 2.88029423245462e-05
	516 2.87121583824046e-05
	517 2.88378469122108e-05
	518 2.87108650809387e-05
	519 2.88677201751852e-05
	520 2.88739920506487e-05
	521 2.89377894659992e-05
	522 2.87986058538081e-05
	523 2.88622941297945e-05
	524 2.88639657810563e-05
	525 2.88392475340515e-05
	526 2.85792284557829e-05
	527 2.84824127447791e-05
	528 2.81745524262078e-05
	529 2.79479372693459e-05
	530 2.78004990832414e-05
	531 2.76391401712317e-05
	532 2.77218659903156e-05
	533 2.76920782198431e-05
	534 2.77785347861936e-05
	535 2.76183054666035e-05
	536 2.76942937489366e-05
	537 2.78737097687554e-05
	538 2.81070188066224e-05
	539 2.85297264781548e-05
	540 2.88242990791332e-05
	541 2.89022682409268e-05
	542 2.86206068267347e-05
	543 2.81086231552763e-05
	544 2.76882292382652e-05
	545 2.7369957024348e-05
	546 2.70699583779788e-05
	547 2.70896725851344e-05
	548 2.69082447630353e-05
	549 2.69030115305213e-05
	550 2.68407311523333e-05
	551 2.69196989393095e-05
	552 2.68351668637479e-05
	553 2.68981275439728e-05
	554 2.70380933216074e-05
	555 2.70242326223524e-05
	556 2.70407072093803e-05
	557 2.70869732048595e-05
	558 2.69483080046484e-05
	559 2.68381045316346e-05
	560 2.67756149696652e-05
	561 2.66408678726293e-05
	562 2.65532871708274e-05
	563 2.64434620476095e-05
	564 2.64154223259538e-05
	565 2.63484180322848e-05
	566 2.62331159319729e-05
	567 2.62182893493446e-05
	568 2.61405220953748e-05
	569 2.61964396486292e-05
	570 2.61222467088373e-05
	571 2.61826626228867e-05
	572 2.62633402599022e-05
	573 2.63192105194321e-05
	574 2.63229430856882e-05
	575 2.62778576143319e-05
	576 2.63523397734389e-05
	577 2.63178881141357e-05
	578 2.62725170614431e-05
	579 2.61752829828765e-05
	580 2.61351033259416e-05
	581 2.59191510849632e-05
	582 2.5773462766665e-05
	583 2.55320683208993e-05
	584 2.55713166552596e-05
	585 2.54887218034128e-05
	586 2.54053175012814e-05
	587 2.53398920904147e-05
	588 2.54484530159971e-05
	589 2.54613514698576e-05
	590 2.54427905019838e-05
	591 2.53519701800542e-05
	592 2.57601786870509e-05
	593 2.59701373579446e-05
	594 2.62271514657186e-05
	595 2.63711153820623e-05
	596 2.64703976426972e-05
	597 2.64085829257965e-05
	598 2.60420783888549e-05
	599 2.57748542935587e-05
	600 2.54634196608094e-05
	601 2.50744815275539e-05
	602 2.50349403358996e-05
	603 2.48762935370905e-05
	604 2.48693613684736e-05
	605 2.49719814746641e-05
	606 2.49421646003611e-05
	607 2.49702479777625e-05
	608 2.48974556598114e-05
	609 2.51538094744319e-05
	610 2.52583431574749e-05
	611 2.54266560659744e-05
	612 2.53890630119713e-05
	613 2.55420090979896e-05
	614 2.55383638432249e-05
	615 2.52970785368234e-05
	616 2.51881920121377e-05
	617 2.49851764237974e-05
	618 2.47336538450327e-05
	619 2.46166291617556e-05
	620 2.46190174948424e-05
	621 2.45880783040775e-05
	622 2.44849816226633e-05
	623 2.4443867005175e-05
	624 2.44177790591493e-05
	625 2.4531191229471e-05
	626 2.46114668698283e-05
	627 2.46951913140947e-05
	628 2.47821008088067e-05
	629 2.48480318987276e-05
	630 2.50088505708845e-05
	631 2.5073837605305e-05
	632 2.50194334512344e-05
	633 2.48798569373321e-05
	634 2.4812316041789e-05
	635 2.45492719841423e-05
	636 2.43549784499919e-05
	637 2.42155019805068e-05
	638 2.41671205003513e-05
	639 2.4109842343023e-05
	640 2.41374364122748e-05
	641 2.40834087890107e-05
	642 2.41927027673228e-05
	643 2.41260077018524e-05
	644 2.41214947891422e-05
	645 2.4290517103509e-05
	646 2.46096515184036e-05
	647 2.4611321350676e-05
	648 2.47396874328842e-05
	649 2.47946973104263e-05
	650 2.49229397013551e-05
	651 2.46420768235112e-05
	652 2.43873746512691e-05
	653 2.41663365159184e-05
	654 2.39768232859205e-05
	655 2.3768059691065e-05
	656 2.3675434931647e-05
	657 2.38596039707772e-05
	658 2.37561434914824e-05
	659 2.36304476857185e-05
	660 2.38597240240779e-05
	661 2.38951251958497e-05
	662 2.38415268540848e-05
	663 2.38919947150862e-05
	664 2.4021386707318e-05
	665 2.41062571149087e-05
	666 2.4289340217365e-05
	667 2.42478417931125e-05
	668 2.41360994550632e-05
	669 2.40518074861029e-05
	670 2.38380216615042e-05
	671 2.37108270084718e-05
	672 2.35601455642609e-05
	673 2.34738963627024e-05
	674 2.34749732044293e-05
	675 2.33490009122761e-05
	676 2.33936752920272e-05
	677 2.35677180171479e-05
	678 2.34651597565971e-05
	679 2.3349955881713e-05
	680 2.34026338148396e-05
	681 2.35755323956255e-05
	682 2.36661817325512e-05
	683 2.38116172113223e-05
	684 2.38547654589638e-05
	685 2.39720466197468e-05
	686 2.38265984080499e-05
	687 2.38870470639085e-05
	688 2.38137381529668e-05
	689 2.35021634580335e-05
	690 2.33032824326074e-05
	691 2.31676658586366e-05
	692 2.31354079005541e-05
	693 2.31480880756862e-05
	694 2.30475907301297e-05
	695 2.31396243179915e-05
	696 2.31676895054989e-05
	697 2.31013164011529e-05
	698 2.29798988584662e-05
	699 2.3066933863447e-05
	700 2.33824794122484e-05
	701 2.3573127691634e-05
	702 2.35992611123947e-05
	703 2.37318490690086e-05
	704 2.38003940467024e-05
	705 2.36585692618974e-05
	706 2.34314175031614e-05
	707 2.31592221098254e-05
	708 2.30116565944627e-05
	709 2.28993649216136e-05
	710 2.27465970965568e-05
	711 2.27399723371491e-05
	712 2.28135431825649e-05
	713 2.28771059482824e-05
	714 2.28225380851654e-05
	715 2.27859491133131e-05
	716 2.26754164032172e-05
	717 2.27287473535398e-05
	718 2.30738223763183e-05
	719 2.31260473810835e-05
	720 2.31471331062494e-05
	721 2.32770835282281e-05
	722 2.33290429605404e-05
	723 2.32735455938382e-05
	724 2.30438890866935e-05
	725 2.27761847781949e-05
	726 2.26386509893928e-05
	727 2.25944368139608e-05
	728 2.25052208406851e-05
	729 2.23790138988988e-05
	730 2.24910672841361e-05
	731 2.25437033805065e-05
	732 2.24183113459731e-05
	733 2.24280120164622e-05
	734 2.25317762669874e-05
	735 2.24916420847876e-05
	736 2.26459887926467e-05
	737 2.26560987357516e-05
	738 2.29037996177794e-05
	739 2.29021643463057e-05
	740 2.29259785555769e-05
	741 2.28235985559877e-05
	742 2.28626631724183e-05
	743 2.27205873670755e-05
	744 2.24868326768046e-05
	745 2.22729231609264e-05
	746 2.22308808588423e-05
	747 2.21612008317607e-05
	748 2.21810114453547e-05
	749 2.22660401050234e-05
	750 2.23508486669743e-05
	751 2.21851223614067e-05
	752 2.21178252104437e-05
	753 2.22877006308408e-05
	754 2.23162041947944e-05
	755 2.24486957449699e-05
	756 2.26920456043445e-05
	757 2.28432018047897e-05
	758 2.27829750656383e-05
	759 2.27918299060548e-05
	760 2.26849551836494e-05
	761 2.2397445718525e-05
	762 2.22727703658165e-05
	763 2.2082591385697e-05
	764 2.19454104808392e-05
	765 2.19106259464752e-05
	766 2.20481997530442e-05
	767 2.20981528400443e-05
	768 2.19686480704695e-05
	769 2.18803306779591e-05
	770 2.20114307012409e-05
	771 2.20981273741927e-05
	772 2.20280398934847e-05
	773 2.21811951632844e-05
	774 2.23625920625636e-05
	775 2.25472595047904e-05
	776 2.24477553274482e-05
	777 2.252210106235e-05
	778 2.23226779780816e-05
	779 2.21981808863347e-05
	780 2.18943532672711e-05
	781 2.183653487009e-05
	782 2.17607976082945e-05
	783 2.16746066143969e-05
	784 2.17066190089099e-05
	785 2.18519107875181e-05
	786 2.1856778403162e-05
	787 2.17075375985587e-05
	788 2.16772587009473e-05
	789 2.18409531953512e-05
	790 2.19907669816166e-05
	791 2.20333586185006e-05
	792 2.21750851778779e-05
	793 2.23672559513943e-05
	794 2.23752977035474e-05
	795 2.23780898522818e-05
	796 2.21870595851215e-05
	797 2.20637575694127e-05
	798 2.17391334444983e-05
	799 2.16717871808214e-05
	800 2.1583979105344e-05
	801 2.13979456020752e-05
	802 2.15154577745125e-05
	803 2.15414456761209e-05
	804 2.16473399632378e-05
	805 2.16106618609047e-05
	806 2.1487292542588e-05
	807 2.1514486434171e-05
	808 2.16236894630129e-05
	809 2.17735341720982e-05
	810 2.18934474105481e-05
	811 2.20979381992947e-05
	812 2.21980280912248e-05
	813 2.2052719941712e-05
	814 2.20942474697949e-05
	815 2.18296390812611e-05
	816 2.1600602849503e-05
	817 2.14457359106746e-05
	818 2.13104485737858e-05
	819 2.12570521398447e-05
	820 2.1294312318787e-05
	821 2.14572664845036e-05
	822 2.15365544136148e-05
	823 2.13240491575561e-05
	824 2.12788218050264e-05
	825 2.14420142583549e-05
	826 2.14703286474105e-05
	827 2.15780382859521e-05
	828 2.17249962588539e-05
	829 2.18845743802376e-05
	830 2.18798213609261e-05
	831 2.17638389585773e-05
	832 2.17184242501389e-05
	833 2.16447315324331e-05
	834 2.13020502997097e-05
	835 2.12431332329288e-05
	836 2.10746256925631e-05
	837 2.11056758416817e-05
	838 2.11364404094638e-05
	839 2.12311224458972e-05
	840 2.11860715353396e-05
	841 2.11527603823924e-05
	842 2.12622162507614e-05
	843 2.12404902413255e-05
	844 2.12625473068329e-05
	845 2.12797658605268e-05
	846 2.14749925362412e-05
	847 2.17053639062215e-05
	848 2.16576336242724e-05
	849 2.15951895370381e-05
	850 2.15799063880695e-05
	851 2.15065720112761e-05
	852 2.1207853933447e-05
	853 2.10295256692916e-05
	854 2.09660938708112e-05
	855 2.08807814487955e-05
	856 2.09407498914516e-05
	857 2.10547314054566e-05
	858 2.09961017390015e-05
	859 2.09808204090223e-05
	860 2.11074657272547e-05
	861 2.11073329410283e-05
	862 2.10608159250114e-05
	863 2.11505594052142e-05
	864 2.13373987207888e-05
	865 2.14935498661362e-05
	866 2.16273001569789e-05
	867 2.14859646803234e-05
	868 2.14427964237984e-05
	869 2.12277773243841e-05
	870 2.09908321266994e-05
	871 2.08239998755744e-05
	872 2.07776301976992e-05
	873 2.0712410332635e-05
	874 2.07393131859135e-05
	875 2.09242807613919e-05
	876 2.09686277230503e-05
	877 2.07599132409086e-05
	878 2.06713793886593e-05
	879 2.07209523068741e-05
	880 2.08927503990708e-05
	881 2.10409361898201e-05
	882 2.1117833966855e-05
	883 2.12607283174293e-05
	884 2.13551484193886e-05
	885 2.12107534025563e-05
	886 2.11510214285227e-05
	887 2.09402060136199e-05
	888 2.08169149118476e-05
	889 2.0642926756409e-05
	890 2.05345495487563e-05
	891 2.05749201995786e-05
	892 2.06657496164553e-05
	893 2.05733376787975e-05
	894 2.06501572392881e-05
	895 2.07329394470435e-05
	896 2.06324257305823e-05
	897 2.06167969736271e-05
	898 2.0677174688899e-05
	899 2.07036810024874e-05
	900 2.09265799639979e-05
	901 2.10506859730231e-05
	902 2.11706628761021e-05
	903 2.09837780857924e-05
	904 2.10198377317283e-05
	905 2.09088357223663e-05
	906 2.05530905077467e-05
	907 2.04498192033498e-05
	908 2.04690913960803e-05
	909 2.03765084734187e-05
	910 2.03640211111633e-05
	911 2.06423192139482e-05
	912 2.06482200155733e-05
	913 2.05053092940943e-05
	914 2.04664083867101e-05
	915 2.05363630811917e-05
	916 2.05949600058375e-05
	917 2.04790285351919e-05
	918 2.07126122404588e-05
	919 2.08808960451279e-05
	920 2.11317292269086e-05
	921 2.1076184566482e-05
	922 2.09194658964407e-05
	923 2.08473138627596e-05
	924 2.05732485483168e-05
	925 2.03987674467498e-05
	926 2.02195187739562e-05
	927 2.02269111468922e-05
	928 2.01933253265452e-05
	929 2.03277904802235e-05
	930 2.03689924092032e-05
	931 2.03995095944265e-05
	932 2.02859682758572e-05
	933 2.01919829123653e-05
	934 2.02576538868016e-05
	935 2.04141833819449e-05
	936 2.04809966817265e-05
	937 2.06475306185894e-05
	938 2.0818481061724e-05
	939 2.07932862394955e-05
	940 2.07108860195149e-05
	941 2.05416672542924e-05
	942 2.03881118068239e-05
	943 2.01520506379893e-05
	944 2.00037975446321e-05
	945 2.00282265723217e-05
	946 1.9967412299593e-05
	947 1.99488313228358e-05
	948 2.01755810849136e-05
	949 2.01516832021298e-05
	950 2.00558752112556e-05
	951 1.99492042156635e-05
	952 2.00223239517072e-05
	953 2.00599461095408e-05
	954 2.02148276002845e-05
	955 2.02553364943014e-05
	956 2.0436009435798e-05
	957 2.04325097001856e-05
	958 2.05291944439523e-05
	959 2.02710434678011e-05
	960 2.01209604711039e-05
	961 1.99935257114703e-05
	962 1.98249963432318e-05
	963 1.97814551938791e-05
	964 1.97402114281431e-05
	965 1.97652789211133e-05
	966 1.9939812773373e-05
	967 1.99611349671613e-05
	968 1.98598536371719e-05
	969 1.97693589143455e-05
	970 1.97740155272186e-05
	971 1.99210135178873e-05
	972 1.99088626686716e-05
	973 2.00259346456733e-05
	974 2.02564624487422e-05
	975 2.03460040211212e-05
	976 2.0433511963347e-05
	977 2.02366136363707e-05
	978 2.00765189219965e-05
	979 1.99284550035372e-05
	980 1.9744682504097e-05
	981 1.95534175873036e-05
	982 1.96129585674498e-05
	983 1.95182983588893e-05
	984 1.95620286831399e-05
	985 1.98063989955699e-05
	986 1.9768802303588e-05
	987 1.96537675947184e-05
	988 1.94994499906898e-05
	989 1.94665626622736e-05
	990 1.96186356333783e-05
	991 1.96602795767831e-05
	992 1.98055731743807e-05
	993 2.00824342755368e-05
	994 2.01114107767353e-05
	995 2.00364356714999e-05
	996 1.99385340238223e-05
	997 1.98449124582112e-05
	998 1.955291008926e-05
	999 1.94203439605189e-05
	1000 1.93435735127423e-05
	1001 1.94250333152013e-05
	1002 1.92959341802634e-05
	1003 1.93891119124601e-05
	1004 1.96366381715052e-05
	1005 1.95488137251232e-05
	1006 1.93618161574705e-05
	1007 1.92398783838144e-05
	1008 1.9242977941758e-05
	1009 1.94046351680299e-05
	1010 1.94362728507258e-05
	1011 1.95674347196473e-05
	1012 1.99027363123605e-05
	1013 1.97663011931581e-05
	1014 1.98189936782001e-05
	1015 1.96663640963379e-05
	1016 1.94611220649676e-05
	1017 1.92282277566846e-05
	1018 1.91795970749808e-05
	1019 1.91602757695364e-05
	1020 1.91885828826344e-05
	1021 1.91393010027241e-05
	1022 1.93367031897651e-05
	1023 1.92671323020477e-05
	1024 1.91928083950188e-05
	1025 1.91038143384503e-05
	1026 1.90256068890449e-05
	1027 1.90097380254883e-05
	1028 1.91880735656014e-05
	1029 1.93009454960702e-05
	1030 1.94376461877255e-05
	1031 1.95744414668297e-05
	1032 1.95887569134356e-05
	1033 1.95305692614056e-05
	1034 1.93473169929348e-05
	1035 1.90823939192342e-05
	1036 1.90253340406343e-05
	1037 1.8964616174344e-05
	1038 1.89421061804751e-05
	1039 1.88993599294918e-05
	1040 1.89945167221595e-05
	1041 1.91016642929753e-05
	1042 1.90841165021993e-05
	1043 1.90120608749567e-05
	1044 1.88732119568158e-05
	1045 1.88013127626618e-05
	1046 1.8789409296005e-05
	1047 1.9000446627615e-05
	1048 1.90765040315455e-05
	1049 1.93123105418636e-05
	1050 1.94350777746877e-05
	1051 1.94396779988892e-05
	1052 1.9430743122939e-05
	1053 1.90686714631738e-05
	1054 1.89277234312613e-05
	1055 1.88566700671799e-05
	1056 1.87313726200955e-05
	1057 1.87387577170739e-05
	1058 1.87318910320755e-05
	1059 1.87721707334276e-05
	1060 1.89909878827166e-05
	1061 1.88430803973461e-05
	1062 1.87686437129742e-05
	1063 1.86317829502514e-05
	1064 1.85338049050188e-05
	1065 1.8631619241205e-05
	1066 1.88755075214431e-05
	1067 1.89509082701989e-05
	1068 1.90560476767132e-05
	1069 1.92097832041327e-05
	1070 1.91071922017727e-05
	1071 1.89487491297768e-05
	1072 1.87644054676639e-05
	1073 1.85722437890945e-05
	1074 1.84598320629448e-05
	1075 1.8413906218484e-05
	1076 1.85264434549026e-05
	1077 1.85286517080385e-05
	1078 1.85310018423479e-05
	1079 1.86189536179882e-05
	1080 1.85919689101866e-05
	1081 1.84740983968368e-05
	1082 1.84915497811744e-05
	1083 1.83540996658849e-05
	1084 1.8334050764679e-05
	1085 1.85652224899968e-05
	1086 1.87343885045266e-05
	1087 1.88364228961291e-05
	1088 1.89746879186714e-05
	1089 1.88228696060833e-05
	1090 1.87278401426738e-05
	1091 1.85989338206127e-05
	1092 1.83836909855017e-05
	1093 1.82850708370097e-05
	1094 1.82235999091063e-05
	1095 1.82460662472295e-05
	1096 1.8410804841551e-05
	1097 1.84831515070982e-05
	1098 1.84353284566896e-05
	1099 1.84756863745861e-05
	1100 1.84297969099134e-05
	1101 1.82822022907203e-05
	1102 1.81564955710201e-05
	1103 1.81274044734892e-05
	1104 1.83103475137614e-05
	1105 1.86135839612689e-05
	1106 1.86824654520024e-05
	1107 1.87330533663044e-05
	1108 1.86266333912499e-05
	1109 1.84297023224644e-05
	1110 1.81821797013981e-05
	1111 1.80517436092487e-05
	1112 1.7983704310609e-05
	1113 1.79945309355389e-05
	1114 1.79307717189658e-05
	1115 1.79221769940341e-05
	1116 1.80311290023383e-05
	1117 1.81493651325582e-05
	1118 1.80122260644566e-05
	1119 1.78401496668812e-05
	1120 1.77500151039567e-05
	1121 1.78748923644889e-05
	1122 1.7728632883518e-05
	1123 1.78600803337758e-05
	1124 1.80678252945654e-05
	1125 1.82427120307693e-05
	1126 1.81975810846779e-05
	1127 1.8082788301399e-05
	1128 1.80853294295957e-05
	1129 1.79316994035617e-05
	1130 1.76999801624333e-05
	1131 1.76664325408638e-05
	1132 1.75608292920515e-05
	1133 1.76502489921404e-05
	1134 1.77839119714918e-05
	1135 1.77791571331909e-05
	1136 1.79019680217607e-05
	1137 1.79737853613915e-05
	1138 1.7851665688795e-05
	1139 1.76524608832551e-05
	1140 1.74233155121328e-05
	1141 1.75787990883691e-05
	1142 1.78524187504081e-05
	1143 1.79849866981385e-05
	1144 1.81002724275459e-05
	1145 1.81220057129394e-05
	1146 1.81827126652934e-05
	1147 1.80011957127135e-05
	1148 1.77437086676946e-05
	1149 1.74427423189627e-05
	1150 1.74199576576939e-05
	1151 1.73315274878405e-05
	1152 1.73709649970988e-05
	1153 1.74248143594014e-05
	1154 1.76323246705579e-05
	1155 1.764604166965e-05
	1156 1.75659897649894e-05
	1157 1.73589978658129e-05
	1158 1.72442705661524e-05
	1159 1.7280486645177e-05
	1160 1.73184653249336e-05
	1161 1.75068380485754e-05
	1162 1.75300065166084e-05
	1163 1.77187357621733e-05
	1164 1.78535410668701e-05
	1165 1.76756457221927e-05
	1166 1.7430609659641e-05
	1167 1.74287979461951e-05
	1168 1.72210038726917e-05
	1169 1.71174706338206e-05
	1170 1.71285955730127e-05
	1171 1.72903291968396e-05
	1172 1.74181332113221e-05
	1173 1.73330845427699e-05
	1174 1.73917542269919e-05
	1175 1.73679873114452e-05
	1176 1.73108219314599e-05
	1177 1.70963066921104e-05
	1178 1.69644135894487e-05
	1179 1.73029620782472e-05
	1180 1.73117459780769e-05
	1181 1.75152435986092e-05
	1182 1.75648892764002e-05
	1183 1.77066176547669e-05
	1184 1.74986689671641e-05
	1185 1.7384114471497e-05
	1186 1.71791944012512e-05
	1187 1.70969087776029e-05
	1188 1.69494869624032e-05
	1189 1.6916148524615e-05
	1190 1.69487630046206e-05
	1191 1.72198197105899e-05
	1192 1.73712287505623e-05
	1193 1.71847368619638e-05
	1194 1.71331885212567e-05
	1195 1.70671337400563e-05
	1196 1.69007125805365e-05
	1197 1.68431106430944e-05
	1198 1.70286530192243e-05
	1199 1.70727380464086e-05
	1200 1.73478983924724e-05
	1201 1.74572051037103e-05
	1202 1.75034128915286e-05
	1203 1.72737763932673e-05
	1204 1.71477076946758e-05
	1205 1.69544528034749e-05
	1206 1.68905935424846e-05
	1207 1.67182806762867e-05
	1208 1.67581129062455e-05
	1209 1.6885078366613e-05
	1210 1.70036273630103e-05
	1211 1.70636722032214e-05
	1212 1.71377359947655e-05
	1213 1.68177175510209e-05
	1214 1.66587997227907e-05
	1215 1.6692589269951e-05
	1216 1.67362759384559e-05
	1217 1.69754166563507e-05
	1218 1.69082068168791e-05
	1219 1.71028823388042e-05
	1220 1.71536357811419e-05
	1221 1.72005984495627e-05
	1222 1.71557894645957e-05
	1223 1.69248960446566e-05
	1224 1.67354446602985e-05
	1225 1.67353337019449e-05
	1226 1.6619738744339e-05
	1227 1.66299596457975e-05
	1228 1.67000634974102e-05
	1229 1.68750721059041e-05
	1230 1.69446975633036e-05
	1231 1.70551411429187e-05
	1232 1.69798495335272e-05
	1233 1.66409154189751e-05
	1234 1.65616329468321e-05
	1235 1.65559886227129e-05
	1236 1.67160796991084e-05
	1237 1.6869293176569e-05
	1238 1.70563307619886e-05
	1239 1.71468200278468e-05
	1240 1.72829422808718e-05
	1241 1.70969215105288e-05
	1242 1.68854203366209e-05
	1243 1.6665282601025e-05
	1244 1.66275131050497e-05
	1245 1.65046476467978e-05
	1246 1.64847751875641e-05
	1247 1.65806250151945e-05
	1248 1.67460239026695e-05
	1249 1.68212209246121e-05
	1250 1.68095521075884e-05
	1251 1.6655889339745e-05
	1252 1.64266857609618e-05
	1253 1.63761251315009e-05
	1254 1.64818920893595e-05
	1255 1.65098663273966e-05
	1256 1.67267862707376e-05
	1257 1.68108908837894e-05
	1258 1.68987717188429e-05
	1259 1.69887716765516e-05
	1260 1.6825475540827e-05
	1261 1.66987138072727e-05
	1262 1.64662833412876e-05
	1263 1.64249540830497e-05
	1264 1.63125914696138e-05
	1265 1.64060693350621e-05
	1266 1.64745360962115e-05
	1267 1.66992176673375e-05
	1268 1.66409354278585e-05
	1269 1.67042853718158e-05
	1270 1.6612166291452e-05
	1271 1.64298999152379e-05
	1272 1.64243610925041e-05
	1273 1.62816741067218e-05
	1274 1.63317654369166e-05
	1275 1.66252157214331e-05
	1276 1.6805372069939e-05
	1277 1.69305203598924e-05
	1278 1.68043752637459e-05
	1279 1.66961835930124e-05
	1280 1.65042129083304e-05
	1281 1.64608809427591e-05
	1282 1.63056283781771e-05
	1283 1.63194417837076e-05
	1284 1.62618289323291e-05
	1285 1.64113989740144e-05
	1286 1.65873316291254e-05
	1287 1.66430927492911e-05
	1288 1.65405799634755e-05
	1289 1.64473076438298e-05
	1290 1.64693992701359e-05
	1291 1.61301195475971e-05
	1292 1.61886928253807e-05
	1293 1.64181183208711e-05
	1294 1.65457076946041e-05
	1295 1.68046717590187e-05
	1296 1.66721165442141e-05
	1297 1.66110294230748e-05
	1298 1.65877972904127e-05
	1299 1.63809090736322e-05
	1300 1.63064269145252e-05
	1301 1.60915496962843e-05
	1302 1.61776897584787e-05
	1303 1.62584456120385e-05
	1304 1.63928070833208e-05
	1305 1.65359651873587e-05
	1306 1.66177596838679e-05
	1307 1.6419564417447e-05
	1308 1.62025298777735e-05
	1309 1.61290827236371e-05
	1310 1.60854433488566e-05
	1311 1.61318985192338e-05
	1312 1.64083576237317e-05
	1313 1.64396406034939e-05
	1314 1.65916717378423e-05
	1315 1.66269983310485e-05
	1316 1.65675937751075e-05
	1317 1.6399186279159e-05
	1318 1.62218730110908e-05
	1319 1.60683521244209e-05
	1320 1.61404313985258e-05
	1321 1.61280677275499e-05
	1322 1.61671250680229e-05
	1323 1.62968062795699e-05
	1324 1.65041528816801e-05
	1325 1.66420522873523e-05
	1326 1.63009008247172e-05
	1327 1.61123534780927e-05
	1328 1.5978914234438e-05
	1329 1.60416857397649e-05
	1330 1.61220195877831e-05
	1331 1.63208660524106e-05
	1332 1.63976583280601e-05
	1333 1.65504679898731e-05
	1334 1.66638819791842e-05
	1335 1.65202654898167e-05
	1336 1.62816759257112e-05
	1337 1.61351381393615e-05
	1338 1.60015169967664e-05
	1339 1.60552463057684e-05
	1340 1.60971540026367e-05
	1341 1.6124944522744e-05
	1342 1.62795040523633e-05
	1343 1.65528708748752e-05
	1344 1.63354325195542e-05
	1345 1.61774423759198e-05
	1346 1.60017461894313e-05
	1347 1.59213777806144e-05
	1348 1.58946859301068e-05
	1349 1.6111471268232e-05
	1350 1.62245851242915e-05
	1351 1.63242984854151e-05
	1352 1.6496818716405e-05
	1353 1.64201537700137e-05
	1354 1.6316897017532e-05
	1355 1.61202333401889e-05
	1356 1.60303679876961e-05
	1357 1.59386181621812e-05
	1358 1.59487262862967e-05
	1359 1.59794199134922e-05
	1360 1.61179268616252e-05
	1361 1.63364529726095e-05
	1362 1.63332479132805e-05
	1363 1.61992775247199e-05
	1364 1.61171119543724e-05
	1365 1.59828414325602e-05
	1366 1.5851297575864e-05
	1367 1.57921840582276e-05
	1368 1.59817445819499e-05
	1369 1.61928910529241e-05
	1370 1.63822842296213e-05
	1371 1.63401546160458e-05
	1372 1.63180848176125e-05
	1373 1.63183067343198e-05
	1374 1.60664676513989e-05
	1375 1.59764276759233e-05
	1376 1.58804723469075e-05
	1377 1.5878906197031e-05
	1378 1.59763221745379e-05
	1379 1.60975268954644e-05
	1380 1.62939486472169e-05
	1381 1.64563080033986e-05
	1382 1.62157539307373e-05
	1383 1.59654664457776e-05
	1384 1.58124166773632e-05
	1385 1.57502108777408e-05
	1386 1.58829188876553e-05
	1387 1.60188083100365e-05
	1388 1.61825064424193e-05
	1389 1.62632513820427e-05
	1390 1.64456396305468e-05
	1391 1.6238624084508e-05
	1392 1.61270145326853e-05
	1393 1.59427108883392e-05
	1394 1.58909970195964e-05
	1395 1.57954600581434e-05
	1396 1.59265327965841e-05
	1397 1.59191404236481e-05
	1398 1.61424504767638e-05
	1399 1.61436710186535e-05
	1400 1.61083480634261e-05
	1401 1.60787785716821e-05
	1402 1.59908977366285e-05
	1403 1.57286958710756e-05
	1404 1.56391524797073e-05
	1405 1.58195380208781e-05
	1406 1.59596947924001e-05
	1407 1.61205443873769e-05
	1408 1.61798179760808e-05
	1409 1.62359538080636e-05
	1410 1.61411990120541e-05
	1411 1.5975349015207e-05
	1412 1.59095507115126e-05
	1413 1.58188013301697e-05
	1414 1.57195991050685e-05
	1415 1.5719475413789e-05
	1416 1.58740367623977e-05
	1417 1.60388608492212e-05
	1418 1.63177101057954e-05
	1419 1.6364496332244e-05
	1420 1.60674408107297e-05
	1421 1.58111088239821e-05
	1422 1.56373789650388e-05
	1423 1.56855840032222e-05
	1424 1.57373906404246e-05
	1425 1.59634255396668e-05
	1426 1.60852741828421e-05
	1427 1.62329324666644e-05
	1428 1.625902041269e-05
	1429 1.61379066412337e-05
	1430 1.59641112986719e-05
	1431 1.58161728904815e-05
	1432 1.57500016939593e-05
	1433 1.57310696522472e-05
	1434 1.580739444762e-05
	1435 1.58538041432621e-05
	1436 1.59173432621174e-05
	1437 1.62065225595143e-05
	1438 1.6092102669063e-05
	1439 1.58388575073332e-05
	1440 1.56129663082538e-05
	1441 1.54856134031434e-05
	1442 1.56379410327645e-05
	1443 1.57333106471924e-05
	1444 1.57699287228752e-05
	1445 1.59226947289426e-05
	1446 1.60159052029485e-05
	1447 1.60706185852177e-05
	1448 1.593563320057e-05
	1449 1.57937574840616e-05
	1450 1.56756923388457e-05
	1451 1.56509995576926e-05
	1452 1.56380810949486e-05
	1453 1.5681080185459e-05
	1454 1.57823669724166e-05
	1455 1.60123818204738e-05
	1456 1.62628475663951e-05
	1457 1.60928157129092e-05
	1458 1.59160135808634e-05
	1459 1.57113936438691e-05
	1460 1.55613724928116e-05
	1461 1.54918434418505e-05
	1462 1.56960250023985e-05
	1463 1.57817048602737e-05
	1464 1.59044975589495e-05
	1465 1.61644693434937e-05
	1466 1.6246705854428e-05
	1467 1.60153376782546e-05
	1468 1.58262791956076e-05
	1469 1.56887490447843e-05
	1470 1.55890484165866e-05
	1471 1.56648420670535e-05
	1472 1.55769157572649e-05
	1473 1.57135891640792e-05
	1474 1.58779475896154e-05
	1475 1.60347954079043e-05
	1476 1.61550669872668e-05
	1477 1.58794318849687e-05
	1478 1.55758043547394e-05
	1479 1.54560093506007e-05
	1480 1.54287372424733e-05
	1481 1.55017969518667e-05
	1482 1.56321893882705e-05
	1483 1.5823179637664e-05
	1484 1.58441089297412e-05
	1485 1.59504143084632e-05
	1486 1.5849751434871e-05
	1487 1.56857022375334e-05
	1488 1.5638939657947e-05
	1489 1.55372890731087e-05
	1490 1.55436828208622e-05
	1491 1.55117595568299e-05
	1492 1.5649793567718e-05
	1493 1.57965787366265e-05
	1494 1.60850104293786e-05
	1495 1.61339048645459e-05
	1496 1.58722541527823e-05
	1497 1.55600955622504e-05
	1498 1.54191238834755e-05
	1499 1.53697274072329e-05
	1500 1.54868739628e-05
	1501 1.5543884728686e-05
	1502 1.57624017447233e-05
	1503 1.58511338668177e-05
	1504 1.59026822075248e-05
	1505 1.59022292791633e-05
	1506 1.57241156557575e-05
	1507 1.54930003191112e-05
	1508 1.55004854605068e-05
	1509 1.54758290591417e-05
	1510 1.55215257109376e-05
	1511 1.56160149344942e-05
	1512 1.57150407176232e-05
	1513 1.593328488525e-05
	1514 1.59484843607061e-05
	1515 1.57758768182248e-05
	1516 1.55084744619671e-05
	1517 1.54006702359766e-05
	1518 1.53075270645786e-05
	1519 1.52777429320849e-05
	1520 1.54467634274624e-05
	1521 1.56794721988263e-05
	1522 1.57419599418063e-05
	1523 1.57525937538594e-05
	1524 1.57254125952022e-05
	1525 1.56708156282548e-05
	1526 1.54667632159544e-05
	1527 1.54751069203485e-05
	1528 1.53179371409351e-05
	1529 1.54439476318657e-05
	1530 1.54932549776277e-05
	1531 1.56280038936529e-05
	1532 1.5885338143562e-05
	1533 1.61200914590154e-05
	1534 1.58247275976464e-05
	1535 1.5482884919038e-05
	1536 1.52923785208259e-05
	1537 1.5243183952407e-05
	1538 1.52214197441936e-05
	1539 1.5413324945257e-05
	1540 1.55536235979525e-05
	1541 1.57377871801145e-05
	1542 1.58576003741473e-05
	1543 1.5799871107447e-05
	1544 1.56985825014999e-05
	1545 1.54791850945912e-05
	1546 1.53473829413997e-05
	1547 1.53334640344838e-05
	1548 1.53492328536231e-05
	1549 1.55093512148596e-05
	1550 1.57283484440995e-05
	1551 1.57575159391854e-05
	1552 1.57780650624773e-05
	1553 1.5777626686031e-05
	1554 1.54852459672838e-05
	1555 1.51811373143573e-05
	1556 1.50963387568481e-05
	1557 1.51582953549223e-05
	1558 1.52691372932168e-05
	1559 1.53489636431914e-05
	1560 1.54549124999903e-05
	1561 1.57015292643337e-05
	1562 1.55883371917298e-05
	1563 1.54365061462158e-05
	1564 1.53092623804696e-05
	1565 1.52685734065017e-05
	1566 1.53005621541524e-05
	1567 1.52158418131876e-05
	1568 1.53227138071088e-05
	1569 1.53981436596951e-05
	1570 1.56933783728164e-05
	1571 1.59229457494803e-05
	1572 1.5716892448836e-05
	1573 1.53913515532622e-05
	1574 1.51581552927382e-05
	1575 1.51099784488906e-05
	1576 1.50927335198503e-05
	1577 1.51470758282812e-05
	1578 1.54286572069395e-05
	1579 1.54888275574194e-05
	1580 1.56749265443068e-05
	1581 1.56587484525517e-05
	1582 1.5529974916717e-05
	1583 1.53912533278344e-05
	1584 1.52287238961435e-05
	1585 1.52008742588805e-05
	1586 1.5169695871009e-05
	1587 1.5292451280402e-05
	1588 1.54180943354731e-05
	1589 1.57530193973798e-05
	1590 1.59658629854675e-05
	1591 1.57304275489878e-05
	1592 1.53437158587622e-05
	1593 1.51165468196268e-05
	1594 1.50160813063849e-05
	1595 1.50961959661799e-05
	1596 1.51336607814301e-05
	1597 1.52808952407213e-05
	1598 1.54705885506701e-05
	1599 1.56008336489322e-05
	1600 1.54920198838226e-05
	1601 1.536345371278e-05
	1602 1.51745734910946e-05
	1603 1.51741842273623e-05
	1604 1.51556996570434e-05
	1605 1.51277063196176e-05
	1606 1.51964868564392e-05
	1607 1.52878692460945e-05
	1608 1.5550804164377e-05
	1609 1.55391990119824e-05
	1610 1.53393320942996e-05
	1611 1.51652857312001e-05
	1612 1.49566531035816e-05
	1613 1.4887967154209e-05
	1614 1.50123059938778e-05
	1615 1.49836623677402e-05
	1616 1.51018048200058e-05
	1617 1.52840930240927e-05
	1618 1.54314075189177e-05
	1619 1.53392575157341e-05
	1620 1.5271112715709e-05
	1621 1.52096745296149e-05
	1622 1.50151499838103e-05
	1623 1.50356763697346e-05
	1624 1.5051594346005e-05
	1625 1.51306639963877e-05
	1626 1.5273177268682e-05
	1627 1.56386140588438e-05
	1628 1.59569954121253e-05
	1629 1.57944850798231e-05
	1630 1.54239405674161e-05
	1631 1.5092545254447e-05
	1632 1.48833478306187e-05
	1633 1.48906556205475e-05
	1634 1.4995825040387e-05
	1635 1.52337843246642e-05
	1636 1.54375047713984e-05
	1637 1.55492543854052e-05
	1638 1.54299941641511e-05
	1639 1.52393622556701e-05
	1640 1.50476853377768e-05
	1641 1.49865591083653e-05
	1642 1.50194864545483e-05
	1643 1.50228515849449e-05
	1644 1.51020794874057e-05
	1645 1.51421590999234e-05
	1646 1.53268538269913e-05
	1647 1.52955853991443e-05
	1648 1.53382952703396e-05
	1649 1.4994928278611e-05
	1650 1.47625651152339e-05
	1651 1.48010412885924e-05
	1652 1.48220115079312e-05
	1653 1.49307070387295e-05
	1654 1.4911257494532e-05
	1655 1.50998394019553e-05
	1656 1.51173662743531e-05
	1657 1.50624364323448e-05
	1658 1.50760452015675e-05
	1659 1.49115348904161e-05
	1660 1.48631870615645e-05
	1661 1.48098670251784e-05
	1662 1.48695080497419e-05
	1663 1.50244322867366e-05
	1664 1.51871345224208e-05
	1665 1.54821391333826e-05
	1666 1.58391194418073e-05
	1667 1.56074784172233e-05
	1668 1.52190559674636e-05
	1669 1.49211027746787e-05
	1670 1.47706014104187e-05
	1671 1.47508226291393e-05
	1672 1.48587369039888e-05
	1673 1.51538251884631e-05
	1674 1.54719255078817e-05
	1675 1.54605586430989e-05
	1676 1.53574765136e-05
	1677 1.51614367496222e-05
	1678 1.49569150380557e-05
	1679 1.49059887917247e-05
	1680 1.48488516060752e-05
	1681 1.49213501572376e-05
	1682 1.50288142322097e-05
	1683 1.51793310578796e-05
	1684 1.5325162166846e-05
	1685 1.54549215949373e-05
	1686 1.50379801198142e-05
	1687 1.47901992022526e-05
	1688 1.468557275075e-05
	1689 1.47078990266891e-05
	1690 1.4764147636015e-05
	1691 1.48407361848513e-05
	1692 1.49216175486799e-05
	1693 1.50213390952558e-05
	1694 1.50144696817733e-05
	1695 1.49584438986494e-05
	1696 1.48618728417205e-05
	1697 1.47845576066175e-05
	1698 1.47515875141835e-05
	1699 1.47635428220383e-05
	1700 1.49291299749166e-05
	1701 1.50033574755071e-05
	1702 1.52435295603937e-05
	1703 1.55313355207909e-05
	1704 1.54106346599292e-05
	1705 1.51325075421482e-05
	1706 1.48564604387502e-05
	1707 1.46683942148229e-05
	1708 1.46178126669838e-05
	1709 1.4691790966026e-05
	1710 1.49287279782584e-05
	1711 1.52299398905598e-05
	1712 1.52912634803215e-05
	1713 1.52797292685136e-05
	1714 1.51522663145442e-05
	1715 1.49957995745353e-05
	1716 1.48649987750105e-05
	1717 1.47479768202174e-05
	1718 1.47937862493563e-05
	1719 1.49967336255941e-05
	1720 1.51123531395569e-05
	1721 1.54224289872218e-05
	1722 1.56038840941619e-05
	1723 1.52752199937822e-05
	1724 1.49400093505392e-05
	1725 1.46974380186293e-05
	1726 1.46240854519419e-05
	1727 1.46510719787329e-05
	1728 1.47944792843191e-05
	1729 1.50054420373635e-05
	1730 1.51165822899202e-05
	1731 1.51062185977935e-05
	1732 1.50388150359504e-05
	1733 1.49244306157925e-05
	1734 1.48090957736713e-05
	1735 1.46822749229614e-05
	1736 1.47728533193003e-05
	1737 1.4826196093054e-05
	1738 1.49948218677309e-05
	1739 1.50672358358861e-05
	1740 1.52487127706991e-05
	1741 1.53193705045851e-05
	1742 1.50360219777212e-05
	1743 1.47591554195969e-05
	1744 1.45903568409267e-05
	1745 1.4617558917962e-05
	1746 1.46185793710174e-05
	1747 1.47659566209768e-05
	1748 1.49548732224503e-05
	1749 1.51242002175422e-05
	1750 1.51104550241143e-05
	1751 1.50020568980835e-05
	1752 1.49459092426696e-05
	1753 1.47369246406015e-05
	1754 1.47804494190495e-05
	1755 1.47266537169344e-05
	1756 1.47618229675572e-05
	1757 1.49455545397359e-05
	1758 1.5168122445175e-05
	1759 1.54071640281472e-05
	1760 1.56410915224114e-05
	1761 1.51497688420932e-05
	1762 1.47528644447448e-05
	1763 1.45086478369194e-05
	1764 1.45689200508059e-05
	1765 1.46634411066771e-05
	1766 1.48963117680978e-05
	1767 1.50720907186042e-05
	1768 1.51352978718933e-05
	1769 1.50977102748584e-05
	1770 1.49053985296632e-05
	1771 1.48342787724687e-05
	1772 1.46083548315801e-05
	1773 1.46580696309684e-05
	1774 1.47046903293813e-05
	1775 1.48363733387669e-05
	1776 1.49213201439125e-05
	1777 1.50338628372992e-05
	1778 1.50286341522587e-05
	1779 1.48914405144751e-05
	1780 1.46704196595238e-05
	1781 1.44921332321246e-05
	1782 1.44470923260087e-05
	1783 1.44894165714504e-05
	1784 1.4619379726355e-05
	1785 1.45624608194339e-05
	1786 1.47175442180014e-05
	1787 1.47877544804942e-05
	1788 1.4913700397301e-05
	1789 1.47228593050386e-05
	1790 1.46320680869394e-05
	1791 1.45380554386065e-05
	1792 1.46140391734662e-05
	1793 1.46574075188255e-05
	1794 1.47695345731336e-05
	1795 1.49259712998173e-05
	1796 1.53223645611433e-05
	1797 1.54812696564477e-05
	1798 1.54650006152224e-05
	1799 1.51473432197236e-05
	1800 1.46213169500697e-05
	1801 1.43190891321865e-05
	1802 1.44291125252494e-05
	1803 1.46770680657937e-05
	1804 1.50100495375227e-05
	1805 1.52631182572804e-05
	1806 1.52098500620923e-05
	1807 1.51529038703302e-05
	1808 1.48505860124715e-05
	1809 1.47003129313816e-05
	1810 1.45228768815286e-05
	1811 1.45859285112238e-05
	1812 1.46682605191018e-05
	1813 1.47849195855088e-05
	1814 1.4858190297673e-05
	1815 1.49173465615604e-05
	1816 1.47397558976081e-05
	1817 1.45604071803973e-05
	1818 1.43800489240675e-05
	1819 1.4312393432192e-05
	1820 1.43316892717849e-05
	1821 1.43948846016428e-05
	1822 1.44232444654335e-05
	1823 1.45538951983326e-05
	1824 1.45523872561171e-05
	1825 1.45629001053749e-05
	1826 1.45157873703283e-05
	1827 1.44873702083714e-05
	1828 1.44100749821519e-05
	1829 1.44434698086116e-05
	1830 1.44346877277712e-05
	1831 1.45547637657728e-05
	1832 1.47642904266831e-05
	1833 1.49894649439375e-05
	1834 1.53095825226046e-05
	1835 1.50944379129214e-05
	1836 1.47120263136458e-05
	1837 1.44283912959509e-05
	1838 1.42827120725997e-05
	1839 1.43007991937338e-05
	1840 1.44266932693426e-05
	1841 1.4711031326442e-05
	1842 1.50559080793755e-05
	1843 1.51201247717836e-05
	1844 1.50713558468851e-05
	1845 1.4845219993731e-05
	1846 1.45630829138099e-05
	1847 1.44141131386277e-05
	1848 1.44507348522893e-05
	1849 1.45402682392159e-05
	1850 1.46983502418152e-05
	1851 1.47960627145949e-05
	1852 1.48448743857443e-05
	1853 1.46457641676534e-05
	1854 1.44005507536349e-05
	1855 1.42964690894587e-05
	1856 1.42462040457758e-05
	1857 1.4124902918411e-05
	1858 1.41789760164102e-05
	1859 1.43374645631411e-05
	1860 1.45163548950222e-05
	1861 1.44147661558236e-05
	1862 1.44144178193528e-05
	1863 1.43238867167383e-05
	1864 1.43400829983875e-05
	1865 1.42430453706766e-05
	1866 1.42522276291857e-05
	1867 1.42141461765277e-05
	1868 1.4327642020362e-05
	1869 1.44332789204782e-05
	1870 1.46656275319401e-05
	1871 1.48157305375207e-05
	1872 1.47332348205964e-05
	1873 1.44875402838807e-05
	1874 1.42440403578803e-05
	1875 1.39922049129382e-05
	1876 1.39688218041556e-05
	1877 1.41791861096863e-05
	1878 1.42624166983296e-05
	1879 1.44731011459953e-05
	1880 1.46335432873457e-05
	1881 1.47452265082393e-05
	1882 1.45214498843416e-05
	1883 1.43726820169832e-05
	1884 1.40464690048248e-05
	1885 1.40377960633487e-05
	1886 1.40255806400091e-05
	1887 1.42523640533909e-05
	1888 1.44347304740222e-05
	1889 1.4768871551496e-05
	1890 1.47755054058507e-05
	1891 1.44770292536123e-05
	1892 1.41477703436976e-05
	1893 1.38849018185283e-05
	1894 1.36805201691459e-05
	1895 1.37665238071349e-05
	1896 1.39763651532121e-05
	1897 1.41227765197982e-05
	1898 1.41665514092892e-05
	1899 1.40097108669579e-05
	1900 1.39563662742148e-05
	1901 1.38276645884616e-05
	1902 1.37974529934581e-05
	1903 1.38210589284427e-05
	1904 1.38474115374265e-05
	1905 1.39051253427169e-05
	1906 1.404296108376e-05
	1907 1.417249586666e-05
	1908 1.41837545015733e-05
	1909 1.39752564791706e-05
	1910 1.37831711981562e-05
	1911 1.35958634928102e-05
	1912 1.35206946652033e-05
	1913 1.36328317239531e-05
	1914 1.36736780405045e-05
	1915 1.38525256261346e-05
	1916 1.38710702231037e-05
	1917 1.40044867293909e-05
	1918 1.38867162604583e-05
	1919 1.38748146127909e-05
	1920 1.35986156237777e-05
	1921 1.36340077006025e-05
	1922 1.36322169055347e-05
	1923 1.37997685669689e-05
	1924 1.40404354169732e-05
	1925 1.44674850162119e-05
	1926 1.47524606290972e-05
	1927 1.46636275530909e-05
	1928 1.43532179208705e-05
	1929 1.37029555844492e-05
	1930 1.34284327941714e-05
	1931 1.35445579871885e-05
	1932 1.37651959448704e-05
	1933 1.40477995955735e-05
	1934 1.41456675919471e-05
	1935 1.40766524054925e-05
	1936 1.38328132379684e-05
	1937 1.37880651891464e-05
	1938 1.35897316795308e-05
	1939 1.36261724037468e-05
	1940 1.36102225951618e-05
	1941 1.37283859658055e-05
	1942 1.38137502290192e-05
	1943 1.3950162610854e-05
	1944 1.38553077704273e-05
	1945 1.37083916342817e-05
	1946 1.35451637106598e-05
	1947 1.33953608383308e-05
	1948 1.33834491862217e-05
	1949 1.33743014885113e-05
	1950 1.34180290842778e-05
	1951 1.35375357785961e-05
	1952 1.362257899018e-05
	1953 1.35605478135403e-05
	1954 1.35311420308426e-05
	1955 1.34687197714811e-05
	1956 1.34946003527148e-05
	1957 1.33884341266821e-05
	1958 1.35008513098001e-05
	1959 1.35985219458234e-05
	1960 1.39325229611131e-05
	1961 1.40696165544796e-05
	1962 1.44162677315762e-05
	1963 1.43824972838047e-05
	1964 1.40184847623459e-05
	1965 1.35843556563486e-05
	1966 1.33657895275974e-05
	1967 1.33161365738488e-05
	1968 1.35385480461991e-05
	1969 1.39124240376987e-05
	1970 1.40437778100022e-05
	1971 1.40858137456235e-05
	1972 1.40794300023117e-05
	1973 1.39339190354804e-05
	1974 1.36250673676841e-05
	1975 1.34704478114145e-05
	1976 1.3383033547143e-05
	1977 1.35331147248507e-05
	1978 1.37364968395559e-05
	1979 1.40254715006449e-05
	1980 1.40196680149529e-05
	1981 1.38469576995703e-05
	1982 1.36995149659924e-05
	1983 1.34024257931742e-05
	1984 1.31854931169073e-05
	1985 1.32127088363632e-05
	1986 1.33426992761088e-05
	1987 1.34707270262879e-05
	1988 1.35155923999264e-05
	1989 1.3546593436331e-05
	1990 1.34625461214455e-05
	1991 1.35044647322502e-05
	1992 1.3447171113512e-05
	1993 1.32985323944013e-05
	1994 1.32993845909368e-05
	1995 1.33579342218582e-05
	1996 1.35235832203762e-05
	1997 1.37006991280941e-05
	1998 1.40439778988366e-05
	1999 1.41891432576813e-05
};
\addlegendentry{Test}

\end{groupplot}

\end{tikzpicture}

		% This file was created by tikzplotlib v0.9.6.
\begin{tikzpicture}

\begin{groupplot}[
group style={group size=1 by 8},
legend cell align={left},
legend style={fill opacity=1, draw opacity=1, text opacity=1, draw=white},
log basis y={10},
tick align=outside,
tick pos=left,
title style={at={(0.43,0.85)},anchor=north},
x grid style={white!69.0196078431373!black},
xlabel={Epoch},
x label style={yshift=13pt},
xmin=-49.95, xmax=2048.95,
xtick style={color=black},
xtick = {0,500,1500,2000},
y grid style={white!69.0196078431373!black},
ylabel={MSE Loss},
ymode=log,
ytick style={color=black},
width=.45\textwidth,
height=.25\textwidth
]
\nextgroupplot[
title={ELU/ELU},
ymin=2.43868075070043e-06, ymax=0.001,
]
\addplot [semithick, black, dashed]
table {%
0 0.0150232173036784
1 0.0143799920624588
2 0.0137457957171137
3 0.0130832278518938
4 0.0123157096095383
5 0.0113429611665197
6 0.0102342990794568
7 0.0092870187581866
8 0.00856164262222592
9 0.00801335956930416
10 0.00758995221258374
11 0.00725159366993466
12 0.00697620671417098
13 0.00674876009725267
14 0.00655842925334582
15 0.00639763677463634
16 0.00626091233425541
17 0.00614401548955357
18 0.00604348784690956
19 0.00595639014136395
20 0.00588014781533275
21 0.0058124925053562
22 0.00575143636160647
23 0.00569524578168057
24 0.00564241303800372
25 0.00559161465571378
26 0.00554167844529729
27 0.00549153435713379
28 0.00544017443826306
29 0.00538661952668917
30 0.00532989054408972
31 0.00526895076473011
32 0.00520263154612621
33 0.00512957073078724
34 0.00504807094694115
35 0.00495603321905946
36 0.00485189429309685
37 0.00473265390610322
38 0.0045940099080326
39 0.00443110454216367
40 0.00423713746204157
41 0.00400371644718689
42 0.00372116099606501
43 0.00338065977484803
44 0.00298132634088688
45 0.00254318471161241
46 0.00211766547181469
47 0.00176775410727714
48 0.00151517678386881
49 0.00133276108863356
50 0.00119201381676248
51 0.00108099132012285
52 0.000993279774775147
53 0.000923236953894957
54 0.000866335056343814
55 0.000819171609691693
56 0.000779372370743658
57 0.000745255722904403
58 0.000715605677214626
59 0.000689534576849837
60 0.000666360569084645
61 0.000645560791781463
62 0.000626730322437652
63 0.000609552554124093
64 0.000593778342590667
65 0.000579218230996048
66 0.000565713022297132
67 0.00055312514086836
68 0.000541348123078933
69 0.000530282262843684
70 0.000519841353707307
71 0.000509954882545571
72 0.000500563581226743
73 0.000491618122396176
74 0.000483077310036606
75 0.000474906013096188
76 0.000467075057713373
77 0.000459558686998207
78 0.000452335027148365
79 0.00044538540441863
80 0.000438693424257508
81 0.000432244686180638
82 0.000426026595960138
83 0.000420027911786747
84 0.000414238570101588
85 0.000408649273140327
86 0.000403251713123609
87 0.000398038297134917
88 0.00039300173330048
89 0.000388134918466676
90 0.000383431118734734
91 0.000378884181827743
92 0.00037448794410011
93 0.000370236143680813
94 0.00036612306803363
95 0.000362143282018224
96 0.000358291232714691
97 0.000354561578205903
98 0.000350949218500318
99 0.000347449221408169
100 0.000344056743870169
101 0.000340767063335079
102 0.000337575711000682
103 0.000334478100967317
104 0.000331469905404447
105 0.000328547057961259
106 0.000325705830846346
107 0.000322944056051711
108 0.000320257212479191
109 0.000317642598929524
110 0.000315096940312287
111 0.000312617199142551
112 0.000310200501417057
113 0.000307844022245263
114 0.000305544929233292
115 0.000303300629184378
116 0.000301108614621626
117 0.000298966240734444
118 0.000296871171713065
119 0.000294821092666098
120 0.000292813826263227
121 0.000290847061933164
122 0.000288918744672628
123 0.000287026704654636
124 0.000285168962136595
125 0.000283343385945045
126 0.000281547826034512
127 0.000279780575738187
128 0.000278039810609698
129 0.000276323602747652
130 0.000274629960131278
131 0.000272957042852795
132 0.000271302704618392
133 0.000269664878828735
134 0.000268041436697786
135 0.00026643048738606
136 0.000264829842421932
137 0.000263237253079751
138 0.000261650390825707
139 0.00026006737004991
140 0.000258485998301694
141 0.000256903915783369
142 0.000255318596032339
143 0.000253727363656253
144 0.000252127780754563
145 0.000250517046424648
146 0.000248892382160193
147 0.000247250948632427
148 0.000245589585574635
149 0.000243906488663015
150 0.000242200844127183
151 0.000240469553546063
152 0.000238709066252341
153 0.000236916039057178
154 0.000235087115925126
155 0.00023321950970967
156 0.000231310348681291
157 0.000229356427098537
158 0.000227354141884462
159 0.000225300399279149
160 0.00022319251627323
161 0.000221027011775732
162 0.000218801963114856
163 0.000216516304419656
164 0.000214166954151551
165 0.000211751055758214
166 0.000209266457318336
167 0.000206711274813642
168 0.000204083920550602
169 0.000201383273122246
170 0.00019860888983203
171 0.000195760743451956
172 0.000192839498879493
173 0.000189846359887724
174 0.000186783260346601
175 0.00018365297512446
176 0.000180461771435603
177 0.000177217571888377
178 0.000173929556694929
179 0.000170607957500124
180 0.000167263298749276
181 0.000163907067133096
182 0.00016055157863093
183 0.000157212020326369
184 0.00015390279793337
185 0.000150636629939527
186 0.000147425874956753
187 0.000144282950714114
188 0.000141218699752699
189 0.000138242858213289
190 0.000135364399511673
191 0.000132590930661536
192 0.000129929244451432
193 0.000127383381311574
194 0.000124956310571633
195 0.000122648423371174
196 0.000120457726382028
197 0.000118380690821596
198 0.000116413894772904
199 0.000114552627508147
200 0.000112790004209273
201 0.000111120176313761
202 0.000109535954635476
203 0.000108029931396914
204 0.000106595073901872
205 0.000105224862522846
206 0.000103913058921989
207 0.000102654063425689
208 0.000101442604403701
209 0.000100273981587407
210 9.91439914344028e-05
211 9.80488333368612e-05
212 9.69851106731312e-05
213 9.59498538861681e-05
214 9.49403268464266e-05
215 9.39541987747816e-05
216 9.29893871841614e-05
217 9.2044085931775e-05
218 9.11167490187381e-05
219 9.02059819907208e-05
220 8.93105918180481e-05
221 8.84296543972596e-05
222 8.75621305169716e-05
223 8.67072553774051e-05
224 8.58643182084506e-05
225 8.5032687451303e-05
226 8.42118126627156e-05
227 8.34012915618132e-05
228 8.26007174339338e-05
229 8.18097899184522e-05
230 8.10282090242254e-05
231 8.02557607357812e-05
232 7.94922129330189e-05
233 7.8737391248751e-05
234 7.7991140415179e-05
235 7.72533110193763e-05
236 7.65237962241372e-05
237 7.58025482525682e-05
238 7.50894280656667e-05
239 7.43845338604388e-05
240 7.36878569966848e-05
241 7.29993544723584e-05
242 7.23191194538231e-05
243 7.16471155612908e-05
244 7.09834814927035e-05
245 7.03282443623721e-05
246 6.9681494593965e-05
247 6.90433058991857e-05
248 6.84137337714219e-05
249 6.7792858374105e-05
250 6.71807464840413e-05
251 6.65774685728593e-05
252 6.59830768228176e-05
253 6.53976195224004e-05
254 6.48213438552148e-05
255 6.42546630871266e-05
256 6.36977853076814e-05
257 6.31507422355071e-05
258 6.26131763539206e-05
259 6.20849728960593e-05
260 6.15660607508062e-05
261 6.10561783389585e-05
262 6.05551215357991e-05
263 6.00626473499233e-05
264 5.95785512160774e-05
265 5.91026471141731e-05
266 5.86346633184576e-05
267 5.81743883572017e-05
268 5.77216337092068e-05
269 5.72761123720511e-05
270 5.68376577376739e-05
271 5.64060397039157e-05
272 5.59809693641e-05
273 5.55622856523996e-05
274 5.51498468723821e-05
275 5.47433528481633e-05
276 5.4342695520404e-05
277 5.39476547629647e-05
278 5.35579984415335e-05
279 5.31735768305452e-05
280 5.27942231514089e-05
281 5.24197613742672e-05
282 5.20500550891256e-05
283 5.16848840703688e-05
284 5.13241782300611e-05
285 5.09677392415142e-05
286 5.06154369048772e-05
287 5.02671630755458e-05
288 4.99227440684535e-05
289 4.95820744959019e-05
290 4.92450170810343e-05
291 4.89114435282545e-05
292 4.85813512511868e-05
293 4.82544966331488e-05
294 4.7930866145407e-05
295 4.76102919861887e-05
296 4.72927503381015e-05
297 4.69781114702528e-05
298 4.66663031915004e-05
299 4.63572137903157e-05
300 4.60508448298924e-05
301 4.57470338091071e-05
302 4.54458277658887e-05
303 4.51470553741729e-05
304 4.48506977761554e-05
305 4.45567073086295e-05
306 4.42649962053565e-05
307 4.39755347656501e-05
308 4.36881973442382e-05
309 4.3403011957821e-05
310 4.31199236885504e-05
311 4.28388684383663e-05
312 4.25598310940245e-05
313 4.22827650794488e-05
314 4.20076119596047e-05
315 4.17343536085468e-05
316 4.14629424625446e-05
317 4.11933602038062e-05
318 4.09255860347457e-05
319 4.0659594596093e-05
320 4.03953456213912e-05
321 4.01328026669034e-05
322 3.98719890029042e-05
323 3.96128343354007e-05
324 3.93553563924343e-05
325 3.90995557850715e-05
326 3.88454275395134e-05
327 3.85929184858469e-05
328 3.8342051865925e-05
329 3.80928367320621e-05
330 3.78452102225424e-05
331 3.75992121774971e-05
332 3.73547865848423e-05
333 3.71119473570047e-05
334 3.68707126270351e-05
335 3.66310291539662e-05
336 3.63929074111979e-05
337 3.61564032616002e-05
338 3.59214197303004e-05
339 3.56879950729194e-05
340 3.54561452482471e-05
341 3.52258051634635e-05
342 3.49970513582321e-05
343 3.47698426210741e-05
344 3.45441896811849e-05
345 3.43201094636925e-05
346 3.40975520387587e-05
347 3.38765453733458e-05
348 3.36570763792565e-05
349 3.34391810312695e-05
350 3.32228247685862e-05
351 3.30079875894285e-05
352 3.27947093694547e-05
353 3.25830467815535e-05
354 3.23729415541152e-05
355 3.21644129144261e-05
356 3.19574795213384e-05
357 3.17521404014087e-05
358 3.15483295025842e-05
359 3.13460927117148e-05
360 3.11454086912022e-05
361 3.09462199794552e-05
362 3.07485038746336e-05
363 3.05522517365375e-05
364 3.03574476774315e-05
365 3.01640748148202e-05
366 2.99720845475804e-05
367 2.97814481200476e-05
368 2.95920564781227e-05
369 2.94039994983564e-05
370 2.92172941911417e-05
371 2.90318677897972e-05
372 2.88476288829997e-05
373 2.86645988580858e-05
374 2.84826775427405e-05
375 2.83018824305259e-05
376 2.81221643874119e-05
377 2.79434909131737e-05
378 2.77658344529641e-05
379 2.75891516920979e-05
380 2.74134322495456e-05
381 2.72386214419384e-05
382 2.70647770790333e-05
383 2.68917714905115e-05
384 2.67195825287558e-05
385 2.65482818591067e-05
386 2.63779116487228e-05
387 2.62084781148531e-05
388 2.60398343598922e-05
389 2.58720175949634e-05
390 2.57050859140406e-05
391 2.55390393348875e-05
392 2.53738911766277e-05
393 2.52095688750842e-05
394 2.50460385124995e-05
395 2.48834538894016e-05
396 2.47219189333236e-05
397 2.45614169145369e-05
398 2.44019681936436e-05
399 2.42435677755282e-05
400 2.40862250890927e-05
401 2.39300217934613e-05
402 2.3774949323041e-05
403 2.36209010324728e-05
404 2.34680182202851e-05
405 2.33163274039327e-05
406 2.31657441780442e-05
407 2.30162076526597e-05
408 2.28677376270525e-05
409 2.27204222973398e-05
410 2.25742452286681e-05
411 2.24291926187448e-05
412 2.22852919300465e-05
413 2.21425756947724e-05
414 2.20010349352151e-05
415 2.18606992667958e-05
416 2.17215659930048e-05
417 2.15837091559479e-05
418 2.14470255848198e-05
419 2.13115376652695e-05
420 2.11772241094366e-05
421 2.10441272621154e-05
422 2.09121894307884e-05
423 2.07814124735251e-05
424 2.06518482102069e-05
425 2.05234806536225e-05
426 2.03963015863451e-05
427 2.02703344100996e-05
428 2.01455300192777e-05
429 2.00218696662091e-05
430 1.98994148021825e-05
431 1.97781486654947e-05
432 1.96580771039123e-05
433 1.95391509478782e-05
434 1.94214286359795e-05
435 1.93048625547476e-05
436 1.91894458936304e-05
437 1.90751350253038e-05
438 1.89619559236576e-05
439 1.88499676809784e-05
440 1.87391913151203e-05
441 1.86295973207962e-05
442 1.85211807703922e-05
443 1.84139404701966e-05
444 1.83078420832317e-05
445 1.82029250730409e-05
446 1.8099170521424e-05
447 1.79965078004329e-05
448 1.78948940678936e-05
449 1.77943929280389e-05
450 1.76949784354008e-05
451 1.75966856303944e-05
452 1.74993643504706e-05
453 1.74030858950402e-05
454 1.7307746077222e-05
455 1.7213271412686e-05
456 1.71197412157653e-05
457 1.70271538415534e-05
458 1.69355023231788e-05
459 1.68448227668705e-05
460 1.67551166896374e-05
461 1.66664161653785e-05
462 1.65786788954847e-05
463 1.64916191920383e-05
464 1.64053170372824e-05
465 1.63200000393715e-05
466 1.62356079869141e-05
467 1.6152064908681e-05
468 1.60693706767745e-05
469 1.59875823122491e-05
470 1.5906734752491e-05
471 1.58268078784829e-05
472 1.57478646656273e-05
473 1.56698263751309e-05
474 1.55926598033318e-05
475 1.55163962745064e-05
476 1.54410043826658e-05
477 1.53663957469519e-05
478 1.52925279550686e-05
479 1.52194468583389e-05
480 1.51471193987618e-05
481 1.50754739038916e-05
482 1.50045112619068e-05
483 1.49342077069292e-05
484 1.48645790698509e-05
485 1.4795592097272e-05
486 1.4727262483305e-05
487 1.46595654850756e-05
488 1.4592537059599e-05
489 1.45261698687449e-05
490 1.44604342828814e-05
491 1.43953843849687e-05
492 1.43310454028267e-05
493 1.42673612710809e-05
494 1.42043216104781e-05
495 1.41419044261681e-05
496 1.40801002181945e-05
497 1.40188288533238e-05
498 1.3958199668096e-05
499 1.38981696444773e-05
500 1.38387523609396e-05
501 1.37798623729424e-05
502 1.37214749234005e-05
503 1.3663609497172e-05
504 1.36062586868491e-05
505 1.35494280755211e-05
506 1.34930755919527e-05
507 1.34371751769891e-05
508 1.33818032317379e-05
509 1.3326930325519e-05
510 1.32725240202802e-05
511 1.32186052450578e-05
512 1.31651470915983e-05
513 1.31122086539648e-05
514 1.3059735660903e-05
515 1.30077590672073e-05
516 1.29562676498551e-05
517 1.29051970709781e-05
518 1.28545669824121e-05
519 1.28042800628947e-05
520 1.27543164243349e-05
521 1.27048006248742e-05
522 1.26556500852359e-05
523 1.26068211852015e-05
524 1.25583714289945e-05
525 1.25102791432852e-05
526 1.24625644257748e-05
527 1.24153356289014e-05
528 1.23687186537325e-05
529 1.23226577013469e-05
530 1.2277184563203e-05
531 1.22323377276246e-05
532 1.21881519401512e-05
533 1.2144645987533e-05
534 1.21017312277871e-05
535 1.20593735406516e-05
536 1.20174849840282e-05
537 1.19759750063508e-05
538 1.19347663005698e-05
539 1.18937796695917e-05
540 1.18528867627532e-05
541 1.18119839882525e-05
542 1.17710675944238e-05
543 1.1730106800556e-05
544 1.16890178052387e-05
545 1.16478895559879e-05
546 1.16068009319292e-05
547 1.15658780739381e-05
548 1.15251989516452e-05
549 1.14849167669462e-05
550 1.14451181225661e-05
551 1.14059784692699e-05
552 1.1367684589203e-05
553 1.13303260675934e-05
554 1.12939379253163e-05
555 1.12585156237799e-05
556 1.12241301799543e-05
557 1.11906113566818e-05
558 1.11577688937103e-05
559 1.11253683385826e-05
560 1.10930784700258e-05
561 1.10605007428433e-05
562 1.1027221418658e-05
563 1.09927929390352e-05
564 1.09568722699294e-05
565 1.09191754980031e-05
566 1.08795779141246e-05
567 1.08381511321198e-05
568 1.07950018666259e-05
569 1.07505336917058e-05
570 1.07052206832492e-05
571 1.06596490923039e-05
572 1.06147043528182e-05
573 1.05715258449379e-05
574 1.05318115224406e-05
575 1.04980082706874e-05
576 1.04742128232971e-05
577 1.04665517604019e-05
578 1.04844093247181e-05
579 1.05414497575396e-05
580 1.06560180572757e-05
581 1.0843618746037e-05
582 1.10873689376234e-05
583 1.12876263251849e-05
584 1.12854814879881e-05
585 1.10666378407842e-05
586 1.09084434285478e-05
587 1.11401319315974e-05
588 1.15540488785371e-05
589 1.12452902776283e-05
590 1.02672747743782e-05
591 9.69300838171705e-06
592 9.63067422432573e-06
593 9.65800510321912e-06
594 9.56151761144497e-06
595 9.39427620849642e-06
596 9.29834570939647e-06
597 9.29670090243917e-06
598 9.31404989756857e-06
599 9.31976222773301e-06
600 9.36569600984427e-06
601 9.51021170436661e-06
602 9.76871666402701e-06
603 1.01257285862744e-05
604 1.05461717900823e-05
605 1.09618226034058e-05
606 1.12647255434695e-05
607 1.1343589804369e-05
608 1.11486914171621e-05
609 1.0730958450722e-05
610 1.02197239755242e-05
611 9.75113208845357e-06
612 9.39888670714595e-06
613 9.16243776316605e-06
614 8.99996466863229e-06
615 8.87195947196773e-06
616 8.76384291714771e-06
617 8.67953353900219e-06
618 8.62437553816875e-06
619 8.59765821026315e-06
620 8.59626517879519e-06
621 8.62069338314342e-06
622 8.67833214268643e-06
623 8.78329554243606e-06
624 8.95417193014225e-06
625 9.20936633086455e-06
626 9.55734784824358e-06
627 9.97891953247176e-06
628 1.04077406071212e-05
629 1.07310060712962e-05
630 1.08276405015317e-05
631 1.06353527709047e-05
632 1.02065964977527e-05
633 9.70134251598864e-06
634 9.29321123877003e-06
635 9.05947508478278e-06
636 8.92808381180998e-06
637 8.7537833337592e-06
638 8.4954979175933e-06
639 8.26318663449399e-06
640 8.15172573087608e-06
641 8.14955020711494e-06
642 8.19305429278216e-06
643 8.2277834083655e-06
644 8.23829836971157e-06
645 8.25072247767622e-06
646 8.30803584683792e-06
647 8.43867103395013e-06
648 8.6424674829999e-06
649 8.89689145644468e-06
650 9.1734990714798e-06
651 9.4505643026821e-06
652 9.70762915208923e-06
653 9.90500756437029e-06
654 9.97272578651121e-06
655 9.84053998553946e-06
656 9.51064549958858e-06
657 9.10534458409273e-06
658 8.7989379977671e-06
659 8.6794279834379e-06
660 8.64659025623382e-06
661 8.48474560299906e-06
662 8.14991220110528e-06
663 7.83987373598194e-06
664 7.70221200596666e-06
665 7.71439436864085e-06
666 7.78769046849703e-06
667 7.84580774038091e-06
668 7.8572486241768e-06
669 7.84853344715231e-06
670 7.87738043506181e-06
671 7.9849705034718e-06
672 8.17339175895881e-06
673 8.41636374371291e-06
674 8.68496349504255e-06
675 8.96464243638206e-06
676 9.24522466050348e-06
677 9.49029335117757e-06
678 9.61947725564016e-06
679 9.5410011917707e-06
680 9.23669702501684e-06
681 8.81893582782567e-06
682 8.47258290459507e-06
683 8.31081880470208e-06
684 8.2622459025572e-06
685 8.12397017924127e-06
686 7.81689998152757e-06
687 7.49682312406463e-06
688 7.31752073335201e-06
689 7.28449803677478e-06
690 7.32866724728609e-06
691 7.38178550196977e-06
692 7.40401584753414e-06
693 7.39801179783228e-06
694 7.40167497959021e-06
695 7.4562669309941e-06
696 7.58194829675318e-06
697 7.77819788044809e-06
698 8.03833973783696e-06
699 8.35926534392684e-06
700 8.7347039912089e-06
701 9.1319948030133e-06
702 9.46595082851331e-06
703 9.60382151937722e-06
704 9.43704290445879e-06
705 8.99570595880306e-06
706 8.47793089153726e-06
707 8.11999273153674e-06
708 8.00696343894458e-06
709 7.97333309598969e-06
710 7.76921775891282e-06
711 7.40147451949724e-06
712 7.08914405667116e-06
713 6.94456227456897e-06
714 6.93151229658184e-06
715 6.97679914374305e-06
716 7.01907299127669e-06
717 7.02475864677865e-06
718 7.00193951885808e-06
719 6.9881129247662e-06
720 7.01713160822237e-06
721 7.10155442451565e-06
722 7.2412013016887e-06
723 7.44029925225931e-06
724 7.71449641145949e-06
725 8.08086531201013e-06
726 8.53159492741895e-06
727 9.00186853414198e-06
728 9.35755297604146e-06
729 9.43478224790795e-06
730 9.15274780677322e-06
731 8.61923020867295e-06
732 8.08891473891293e-06
733 7.78336231554988e-06
734 7.71147790956661e-06
735 7.64294654054254e-06
736 7.37947664841698e-06
737 7.01204175923564e-06
738 6.74113480680916e-06
739 6.62726660127788e-06
740 6.6223119961073e-06
741 6.65964073487402e-06
742 6.68731511233034e-06
743 6.68213445154109e-06
744 6.65649019415682e-06
745 6.64161494690063e-06
746 6.66118953374095e-06
747 6.72256315414899e-06
748 6.82765057558044e-06
749 6.98682699695752e-06
750 7.22374193173891e-06
751 7.5667093089038e-06
752 8.02692496471025e-06
753 8.56642790836304e-06
754 9.06980868720098e-06
755 9.35470718488318e-06
756 9.25844552668309e-06
757 8.78022492134534e-06
758 8.14290993744748e-06
759 7.65665235391566e-06
760 7.49089831408867e-06
761 7.50939003069107e-06
762 7.38859961302296e-06
763 7.02752888681601e-06
764 6.64615463819018e-06
765 6.42643180004399e-06
766 6.36184356972436e-06
767 6.37143600901169e-06
768 6.3874923719176e-06
769 6.37951241433399e-06
770 6.35463304865524e-06
771 6.33712288600208e-06
772 6.34146531997004e-06
773 6.36626572436683e-06
774 6.40816609154271e-06
775 6.47586848856463e-06
776 6.5932243020761e-06
777 6.79374454382042e-06
778 7.11142306464296e-06
779 7.56581831495851e-06
780 8.13593274529012e-06
781 8.72639394344787e-06
782 9.15790563738028e-06
783 9.23092367877132e-06
784 8.8666042348251e-06
785 8.2252681861128e-06
786 7.6310825694037e-06
787 7.34232555821279e-06
788 7.33058904067718e-06
789 7.27259417665493e-06
790 6.94094580033777e-06
791 6.51251715844126e-06
792 6.24262419934851e-06
793 6.15036103601696e-06
794 6.13420347805871e-06
795 6.12095165708482e-06
796 6.09651226257668e-06
797 6.0787018938413e-06
798 6.0808254667144e-06
799 6.09502984838528e-06
800 6.10472114992433e-06
801 6.1058359683841e-06
802 6.11350278090228e-06
803 6.15354762523168e-06
804 6.25304451418174e-06
805 6.43871114247574e-06
806 6.73844626497555e-06
807 7.17641461278617e-06
808 7.75090259530486e-06
809 8.39482997960772e-06
810 8.94816670182763e-06
811 9.19336879334764e-06
812 8.9790954280744e-06
813 8.37550276067134e-06
814 7.68766668901577e-06
815 7.24710771216053e-06
816 7.16528545652295e-06
817 7.17617484902178e-06
818 6.92113077160883e-06
819 6.45316422298592e-06
820 6.11010135731505e-06
821 5.9788658570703e-06
822 5.94449258883856e-06
823 5.90759820662257e-06
824 5.86299921945255e-06
825 5.84275000869994e-06
826 5.86321315276805e-06
827 5.90017671209253e-06
828 5.9155754428275e-06
829 5.89617879054316e-06
830 5.86357595899756e-06
831 5.85117132212432e-06
832 5.88210201613037e-06
833 5.96809880315519e-06
834 6.12063068183843e-06
835 6.35959446082524e-06
836 6.71239494920428e-06
837 7.1985474221492e-06
838 7.79699125530442e-06
839 8.4093349919101e-06
840 8.85870678857259e-06
841 8.95275875123502e-06
842 8.6052271486281e-06
843 7.94999388453732e-06
844 7.28408100680866e-06
845 6.90541292058811e-06
846 6.88585489960758e-06
847 6.95585765786433e-06
848 6.73214314783621e-06
849 6.27282250809102e-06
850 5.93185799413476e-06
851 5.81657957798853e-06
852 5.7840531719755e-06
853 5.73493817856985e-06
854 5.66808176882816e-06
855 5.6397591303714e-06
856 5.67273972640336e-06
857 5.73563639827412e-06
858 5.76762603587966e-06
859 5.73941143988677e-06
860 5.67967113296319e-06
861 5.63875318926321e-06
862 5.64485031118167e-06
863 5.70061163029933e-06
864 5.80388179916014e-06
865 5.9652782207209e-06
866 6.21310217763238e-06
867 6.58462871427901e-06
868 7.10246304080897e-06
869 7.73684049093504e-06
870 8.37392662234038e-06
871 8.8317652497949e-06
872 8.92353603987317e-06
873 8.56274080085484e-06
874 7.86545500242042e-06
875 7.12012349168845e-06
876 6.66483578370958e-06
877 6.6488784593588e-06
878 6.84507000769941e-06
879 6.73729373135146e-06
880 6.2508197502531e-06
881 5.83345701565818e-06
882 5.6953951954597e-06
883 5.65466912716417e-06
884 5.58929183203105e-06
885 5.49878416933325e-06
886 5.46783269683715e-06
887 5.51085419342456e-06
888 5.58424118013434e-06
889 5.60819203387908e-06
890 5.5618932810475e-06
891 5.49067361221844e-06
892 5.44536630897596e-06
893 5.44022816750811e-06
894 5.46425278979967e-06
895 5.50802823262586e-06
896 5.5770253304388e-06
897 5.69024363095139e-06
898 5.87249800609868e-06
899 6.14812632804274e-06
900 6.53789654059267e-06
901 7.05304217873959e-06
902 7.67618404395876e-06
903 8.32960733276877e-06
904 8.84678448187515e-06
905 8.98120050152329e-06
906 8.54802193650528e-06
907 7.67069676710364e-06
908 6.80284691867428e-06
909 6.44447763598066e-06
910 6.69561582178346e-06
911 7.0401931768771e-06
912 6.78242100793369e-06
913 6.1607099390848e-06
914 5.72751085226031e-06
915 5.58391244886991e-06
916 5.47773069436985e-06
917 5.39524426601545e-06
918 5.32723128543466e-06
919 5.37531158384397e-06
920 5.41128535447299e-06
921 5.43637998440261e-06
922 5.37712643255972e-06
923 5.31188552654882e-06
924 5.27016562301696e-06
925 5.25350988089812e-06
926 5.24994209372132e-06
927 5.25030397602322e-06
928 5.25702080578228e-06
929 5.2790290459015e-06
930 5.32372643657908e-06
931 5.39506088426123e-06
932 5.5001246273001e-06
933 5.6604319325082e-06
934 5.92008927569054e-06
935 6.34221816753211e-06
936 6.98445486335686e-06
937 7.84837084211176e-06
938 8.81431202515159e-06
939 9.57326427553085e-06
940 9.64988704765801e-06
941 8.83296457487859e-06
942 7.57747956914301e-06
943 6.73656874106143e-06
944 6.69018050913905e-06
945 6.90718672302637e-06
946 6.53089968727372e-06
947 5.85672264197257e-06
948 5.45783398386845e-06
949 5.37188192106797e-06
950 5.2573528299682e-06
951 5.24440634963597e-06
952 5.18833536000685e-06
953 5.38675558114932e-06
954 5.29223028777892e-06
955 5.41841714962743e-06
956 5.18420973971701e-06
957 5.2174551337103e-06
958 5.0961879582978e-06
959 5.11080515241957e-06
960 5.0910631124168e-06
961 5.07886823397641e-06
962 5.08240796426307e-06
963 5.0968436517973e-06
964 5.1184244473923e-06
965 5.13997472451067e-06
966 5.15429676184453e-06
967 5.16560057040039e-06
968 5.19328412473818e-06
969 5.26695088476359e-06
970 5.42212893783756e-06
971 5.70070281025181e-06
972 6.14827871814327e-06
973 6.79665652825179e-06
974 7.62822130440099e-06
975 8.52458088473895e-06
976 9.19170336644015e-06
977 9.21010607157768e-06
978 8.47495191447223e-06
979 7.39634830981828e-06
980 6.6130642650819e-06
981 6.36149037802625e-06
982 6.40250232208928e-06
983 6.11847787501674e-06
984 5.70384593023832e-06
985 5.30815845323218e-06
986 5.3963971080595e-06
987 5.46591375449879e-06
988 6.3475832385862e-06
989 7.77833829257446e-06
990 1.67838321019076e-05
991 4.57882654671948e-05
992 5.95558583995626e-05
993 1.49318543698485e-05
994 9.14412216701521e-06
995 1.49397320257094e-05
996 1.87890184406569e-05
997 2.58555926535564e-05
998 3.96139911771343e-05
999 5.99334221860204e-05
1000 7.23708488656882e-05
1001 5.27075871836757e-05
1002 2.41295365341898e-05
1003 1.16736521711402e-05
1004 8.173828442537e-06
1005 6.45302857726904e-06
1006 5.46803737400126e-06
1007 5.17435117686205e-06
1008 5.26112749099639e-06
1009 5.43353004500702e-06
1010 5.55838530225117e-06
1011 5.61371892171536e-06
1012 5.6180714071985e-06
1013 5.59446173120648e-06
1014 5.55929244194786e-06
1015 5.52200615056009e-06
1016 5.48711238401722e-06
1017 5.45621153369424e-06
1018 5.42944792769617e-06
1019 5.40631089407384e-06
1020 5.38605947753013e-06
1021 5.36796351280344e-06
1022 5.35140443558291e-06
1023 5.33589635098863e-06
1024 5.32101025951803e-06
1025 5.30642938301185e-06
1026 5.29191988540845e-06
1027 5.27724278986952e-06
1028 5.26225333619124e-06
1029 5.24677444913024e-06
1030 5.23066792812443e-06
1031 5.21378371898251e-06
1032 5.19597159609475e-06
1033 5.17713571213108e-06
1034 5.15715426629981e-06
1035 5.13592459983059e-06
1036 5.11339636855013e-06
1037 5.089528058555e-06
1038 5.06434316882576e-06
1039 5.03789956995959e-06
1040 5.01034148880919e-06
1041 4.98188888653495e-06
1042 4.95285791912714e-06
1043 4.92366321847015e-06
1044 4.89479437537632e-06
1045 4.86684293687922e-06
1046 4.84046455584419e-06
1047 4.81634295534938e-06
1048 4.79515569096201e-06
1049 4.77752767125139e-06
1050 4.76396187276684e-06
1051 4.75479737627182e-06
1052 4.75014874767865e-06
1053 4.74991094723265e-06
1054 4.75374581765386e-06
1055 4.76111246894284e-06
1056 4.77132482123999e-06
1057 4.78359757494218e-06
1058 4.79715966905658e-06
1059 4.81124769491359e-06
1060 4.82514438693293e-06
1061 4.83830398190221e-06
1062 4.8502348639623e-06
1063 4.86058521875776e-06
1064 4.86913231023678e-06
1065 4.87576047847327e-06
1066 4.88041106505577e-06
1067 4.88309530366493e-06
1068 4.88390255748428e-06
1069 4.8830134478095e-06
1070 4.88057213754445e-06
1071 4.87673018056967e-06
1072 4.87168675933347e-06
1073 4.86564215718133e-06
1074 4.85879431355585e-06
1075 4.85134009098331e-06
1076 4.8434559261068e-06
1077 4.83525629180193e-06
1078 4.8268923285022e-06
1079 4.8185225196562e-06
1080 4.81023492060473e-06
1081 4.80208015263273e-06
1082 4.79412750875596e-06
1083 4.78644088430791e-06
1084 4.77907469775829e-06
1085 4.77202976978397e-06
1086 4.76529838855377e-06
1087 4.75890050077865e-06
1088 4.75282439493796e-06
1089 4.74703501796014e-06
1090 4.74150972373799e-06
1091 4.73627932984222e-06
1092 4.73134417133814e-06
1093 4.72665363626668e-06
1094 4.7222030321592e-06
1095 4.71799110068893e-06
1096 4.71400259471899e-06
1097 4.71019081826896e-06
1098 4.70653212425454e-06
1099 4.70303086053114e-06
1100 4.69965246829851e-06
1101 4.69638537525796e-06
1102 4.69321836926717e-06
1103 4.69015062254385e-06
1104 4.68718492885323e-06
1105 4.6843179872802e-06
1106 4.6815513292664e-06
1107 4.67889500033181e-06
1108 4.67635037959546e-06
1109 4.67391584946242e-06
1110 4.67157815142727e-06
1111 4.66934733722724e-06
1112 4.66721151881622e-06
1113 4.66512409547093e-06
1114 4.66310404401149e-06
1115 4.66108085173111e-06
1116 4.65907749935823e-06
1117 4.65700599860952e-06
1118 4.65487346401616e-06
1119 4.65259232873549e-06
1120 4.65016214867831e-06
1121 4.64750987094753e-06
1122 4.64464555838262e-06
1123 4.641544118833e-06
1124 4.63833152064552e-06
1125 4.63512476267525e-06
1126 4.63261624639166e-06
1127 4.63193186162059e-06
1128 4.63703517139535e-06
1129 4.6570817138214e-06
1130 4.72085169755943e-06
1131 4.90489530968752e-06
1132 5.40000002979824e-06
1133 6.38681689690834e-06
1134 7.23898390475597e-06
1135 6.75139883687947e-06
1136 6.03001343080223e-06
1137 6.42989050625431e-06
1138 6.88851644881083e-06
1139 6.1334784191569e-06
1140 6.56532935039422e-06
1141 8.59308770273515e-06
1142 1.80661435287277e-05
1143 2.1176115021504e-05
1144 9.94817911781354e-06
1145 9.90291477620531e-06
1146 9.38849228404592e-06
1147 7.7022068101229e-06
1148 7.36165088843777e-06
1149 7.0622902796913e-06
1150 7.23608319930236e-06
1151 5.72816262200959e-06
1152 5.93275265003967e-06
1153 5.78469415168925e-06
1154 5.82121603898145e-06
1155 4.97557365353174e-06
1156 5.28340022976082e-06
1157 5.25507396709202e-06
1158 5.12790163575971e-06
1159 4.78887746346857e-06
1160 5.0065559049628e-06
1161 5.00791306734527e-06
1162 4.76533242976807e-06
1163 4.80384552670188e-06
1164 4.83438054077467e-06
1165 4.80178742545689e-06
1166 4.65093553403406e-06
1167 4.7596595578181e-06
1168 4.72716168875564e-06
1169 4.61740188040238e-06
1170 4.68759343319469e-06
1171 4.64828356294689e-06
1172 4.60225386778035e-06
1173 4.62116454236217e-06
1174 4.59842530076138e-06
1175 4.57424878796964e-06
1176 4.58125884539307e-06
1177 4.56594900621887e-06
1178 4.55279678757137e-06
1179 4.55732050697044e-06
1180 4.54542966932259e-06
1181 4.54232134061705e-06
1182 4.53832686897471e-06
1183 4.53385752452462e-06
1184 4.53039087977558e-06
1185 4.5262506607191e-06
1186 4.52332158662472e-06
1187 4.51915832644367e-06
1188 4.51694623948562e-06
1189 4.51231841269717e-06
1190 4.51171363891056e-06
1191 4.50516915306842e-06
1192 4.50935738793845e-06
1193 4.49685506098696e-06
1194 4.52155396035714e-06
1195 4.50171934440746e-06
1196 4.67698154915652e-06
1197 4.9176104039983e-06
1198 7.10356122013422e-06
1199 1.22372143067651e-05
1200 2.29910080893347e-05
1201 2.31665879653065e-05
1202 1.14670569848485e-05
1203 1.34506958104907e-05
1204 9.44233715749476e-06
1205 8.36875674536941e-06
1206 5.73042730067641e-06
1207 5.40207725707731e-06
1208 5.40897064560397e-06
1209 5.30672032006407e-06
1210 4.8252819098149e-06
1211 4.76938511329017e-06
1212 4.93520151323423e-06
1213 4.94584165444678e-06
1214 4.78017808513975e-06
1215 4.75850135539524e-06
1216 4.82952014602134e-06
1217 4.81603357060933e-06
1218 4.73286391478034e-06
1219 4.73216961549561e-06
1220 4.75879832961823e-06
1221 4.72523076533449e-06
1222 4.67913715618273e-06
1223 4.68240033679734e-06
1224 4.67564483952465e-06
1225 4.63615214951751e-06
1226 4.6173041339248e-06
1227 4.61841650700734e-06
1228 4.59863540003269e-06
1229 4.5756345321557e-06
1230 4.57568832135102e-06
1231 4.56624245459558e-06
1232 4.54630650970067e-06
1233 4.54100514990508e-06
1234 4.53274138023652e-06
1235 4.51421856850764e-06
1236 4.50544157337696e-06
1237 4.49826439186118e-06
1238 4.48429043142795e-06
1239 4.47809175452285e-06
1240 4.47317542229264e-06
1241 4.4645603158866e-06
1242 4.46185712710268e-06
1243 4.45935799464614e-06
1244 4.45855610586854e-06
1245 4.46340322057814e-06
1246 4.47084951660237e-06
1247 4.4829409109326e-06
1248 4.49497017340761e-06
1249 4.50284798825251e-06
1250 4.49814426128903e-06
1251 4.47828843586073e-06
1252 4.44483352102232e-06
1253 4.41674219797683e-06
1254 4.42024453040091e-06
1255 4.51554712577185e-06
1256 4.78305550544178e-06
1257 5.3406818225632e-06
1258 6.00143882301118e-06
1259 6.15321691221915e-06
1260 5.54233691452133e-06
1261 5.31472021547685e-06
1262 5.85556044629953e-06
1263 7.68966502118218e-06
1264 1.0475332067994e-05
1265 1.91685766299798e-05
1266 1.73695196430401e-05
1267 8.86053437554324e-06
1268 1.15384695291709e-05
1269 8.56681579364249e-06
1270 8.20517365340834e-06
1271 5.99327747874412e-06
1272 6.74712854298321e-06
1273 5.98163484788472e-06
1274 5.37753288742948e-06
1275 4.81838320265382e-06
1276 5.12260762453032e-06
1277 5.01254863749523e-06
1278 4.64530872523738e-06
1279 4.61579209476426e-06
1280 4.72918733573913e-06
1281 4.6674155402826e-06
1282 4.47002388059836e-06
1283 4.57538383735745e-06
1284 4.57835643885574e-06
1285 4.48508698625005e-06
1286 4.45491746470239e-06
1287 4.51529674250217e-06
1288 4.46756733518683e-06
1289 4.40253732003626e-06
1290 4.46483531479913e-06
1291 4.43094795699039e-06
1292 4.38170000593985e-06
1293 4.41764622305385e-06
1294 4.39846042232794e-06
1295 4.36307685180992e-06
1296 4.38195567542543e-06
1297 4.37004380837713e-06
1298 4.34446088237728e-06
1299 4.35779898988464e-06
1300 4.3443831401202e-06
1301 4.33088385309865e-06
1302 4.3381185625968e-06
1303 4.32474341027778e-06
1304 4.3228615507207e-06
1305 4.31852938431732e-06
1306 4.31400311384778e-06
1307 4.31064322237162e-06
1308 4.30637503412434e-06
1309 4.30360047021239e-06
1310 4.29914621191685e-06
1311 4.2973671736668e-06
1312 4.29197761842026e-06
1313 4.29268062118737e-06
1314 4.28388314532846e-06
1315 4.29318978878435e-06
1316 4.27420669080192e-06
1317 4.32888059709313e-06
1318 4.32240228542824e-06
1319 4.85015722473392e-06
1320 6.09737926726694e-06
1321 1.36162112518434e-05
1322 1.9450532350973e-05
1323 1.28189891199781e-05
1324 7.45835612292467e-06
1325 8.46061772108442e-06
1326 6.08572453586476e-06
1327 7.46723700828511e-06
1328 6.22475941369771e-06
1329 6.13809599769866e-06
1330 5.14568143028526e-06
1331 5.72085425165625e-06
1332 5.37522792498635e-06
1333 4.91480255337962e-06
1334 4.87794324977031e-06
1335 4.95130991651216e-06
1336 4.87562639861494e-06
1337 4.47096601341457e-06
1338 4.76499743662373e-06
1339 4.63666894745529e-06
1340 4.50883834091087e-06
1341 4.46826265632083e-06
1342 4.54991811027483e-06
1343 4.48377811324008e-06
1344 4.33218957196502e-06
1345 4.51377458210445e-06
1346 4.40365896681172e-06
1347 4.311074535579e-06
1348 4.42823096635792e-06
1349 4.33203743699373e-06
1350 4.2780981079904e-06
1351 4.33336411731489e-06
1352 4.27679458625541e-06
1353 4.24378560981431e-06
1354 4.279667748186e-06
1355 4.24400063314678e-06
1356 4.23068449051378e-06
1357 4.24770976992228e-06
1358 4.23256500825531e-06
1359 4.25328164155303e-06
1360 4.28284088282993e-06
1361 4.35226145545364e-06
1362 4.44323430093263e-06
1363 4.53616954443348e-06
1364 4.54020188400683e-06
1365 4.44761285534057e-06
1366 4.30579354326888e-06
1367 4.32923286375342e-06
1368 4.5771652441573e-06
1369 5.22573764460965e-06
1370 5.76525243567438e-06
1371 5.98131903828403e-06
1372 5.84773819367612e-06
1373 8.10464005951417e-06
1374 1.17971708997189e-05
1375 1.41694397708392e-05
1376 8.83075289337398e-06
1377 5.91994163468001e-06
1378 7.52443008744663e-06
1379 5.94735030734483e-06
1380 6.02388974790813e-06
1381 5.50262741860763e-06
1382 5.54936486607538e-06
1383 5.53252604262866e-06
1384 4.69898292987381e-06
1385 5.40978510565893e-06
1386 4.90881900994111e-06
1387 4.73721305471742e-06
1388 4.83357752534275e-06
1389 4.65648421377196e-06
1390 4.6501484929351e-06
1391 4.39162235510437e-06
1392 4.59493218940565e-06
1393 4.45045319619197e-06
1394 4.24580803315422e-06
1395 4.48901023863968e-06
1396 4.29446226579699e-06
1397 4.19383889282976e-06
1398 4.34657937953276e-06
1399 4.20355700936348e-06
1400 4.163572080218e-06
1401 4.2252112608665e-06
1402 4.15722330537882e-06
1403 4.16453162377373e-06
1404 4.15560060584852e-06
1405 4.1457672848022e-06
1406 4.14977167562114e-06
1407 4.14018376071468e-06
1408 4.14448940300183e-06
1409 4.13294293322153e-06
1410 4.14264039472201e-06
1411 4.12464760324305e-06
1412 4.1520103413184e-06
1413 4.12117602976991e-06
1414 4.23301079910399e-06
1415 4.25628666533839e-06
1416 5.07989634002115e-06
1417 6.60608479297053e-06
1418 1.31823111946261e-05
1419 1.53559469007192e-05
1420 8.09847945504316e-06
1421 7.04987174371396e-06
1422 7.26103924719723e-06
1423 5.58558309204926e-06
1424 6.84854705212956e-06
1425 5.67753172742158e-06
1426 5.77896392606192e-06
1427 5.04181109395319e-06
1428 5.46980486881843e-06
1429 5.28513021436083e-06
1430 4.66023042378083e-06
1431 5.14239467364064e-06
1432 4.77956163846827e-06
1433 4.741671863151e-06
1434 4.51014437485853e-06
1435 4.69118665380464e-06
1436 4.58385177282139e-06
1437 4.26784558849036e-06
1438 4.61530598294679e-06
1439 4.37690055932194e-06
1440 4.24151261957384e-06
1441 4.43714402376116e-06
1442 4.25317458718766e-06
1443 4.20385780897092e-06
1444 4.29155869863251e-06
1445 4.19009234819967e-06
1446 4.15289075816894e-06
1447 4.21282432472303e-06
1448 4.13521438491493e-06
1449 4.10394867089714e-06
1450 4.13458173109671e-06
1451 4.08850026989427e-06
1452 4.09816870394764e-06
1453 4.10289214336501e-06
1454 4.12335945032538e-06
1455 4.14850893637464e-06
1456 4.1786472610994e-06
1457 4.19830928621856e-06
1458 4.20076155061011e-06
1459 4.17121185258829e-06
1460 4.13186495107354e-06
1461 4.12336221500276e-06
1462 4.29896816167563e-06
1463 4.79867147018531e-06
1464 5.55810612468122e-06
1465 5.54112240980231e-06
1466 5.18833604856717e-06
1467 5.58597659261473e-06
1468 1.03448527193351e-05
1469 1.46822178219708e-05
1470 1.29374419852724e-05
1471 6.86079772727055e-06
1472 6.54302546543306e-06
1473 6.35945637172952e-06
1474 5.65146590614241e-06
1475 5.83319906866109e-06
1476 4.7879115463445e-06
1477 5.59921104481553e-06
1478 5.01055321233679e-06
1479 4.80033371097477e-06
1480 4.8732691300124e-06
1481 4.70762681614723e-06
1482 4.71516032618524e-06
1483 4.30808969564822e-06
1484 4.6980754286885e-06
1485 4.4474339944145e-06
1486 4.21778065717859e-06
1487 4.54253104109625e-06
1488 4.25352857513772e-06
1489 4.18712608363236e-06
1490 4.31437381109845e-06
1491 4.14901619905095e-06
1492 4.10926307647941e-06
1493 4.16648616630155e-06
1494 4.07827527637572e-06
1495 4.04104893680923e-06
1496 4.09386747202145e-06
1497 4.031825942441e-06
1498 4.01951342587914e-06
1499 4.04150549115911e-06
1500 4.01436785546494e-06
1501 4.02558848189294e-06
1502 4.01046920139869e-06
1503 4.02082452710673e-06
1504 4.00421375812421e-06
1505 4.02266710963772e-06
1506 3.99761430447398e-06
1507 4.04572768653022e-06
1508 4.0095810454055e-06
1509 4.21599440014742e-06
1510 4.35878401372491e-06
1511 5.97536812341204e-06
1512 8.61299055454623e-06
1513 1.47516643931311e-05
1514 1.31548039483675e-05
1515 6.8586961425865e-06
1516 9.5055821578427e-06
1517 7.26049541022178e-06
1518 6.93969037612874e-06
1519 5.82166988793986e-06
1520 6.20670376472177e-06
1521 5.86491411169732e-06
1522 4.89472188358597e-06
1523 5.08056587067429e-06
1524 4.89264362069086e-06
1525 4.79773881956191e-06
1526 4.27117903134544e-06
1527 4.63546523210567e-06
1528 4.47314385265685e-06
1529 4.26994345636977e-06
1530 4.27516839129893e-06
1531 4.33513507513972e-06
1532 4.24461490400319e-06
1533 4.08916764982337e-06
1534 4.27462650964827e-06
1535 4.16203093900336e-06
1536 4.05764743693915e-06
1537 4.18839434512819e-06
1538 4.10963742858428e-06
1539 4.05088075705606e-06
1540 4.11568416680552e-06
1541 4.08464703449685e-06
1542 4.03124194403581e-06
1543 4.06696382881933e-06
1544 4.03984013463976e-06
1545 3.99043232879315e-06
1546 4.02432139656739e-06
1547 3.99362545144832e-06
1548 3.98338425089406e-06
1549 4.02195582827058e-06
1550 4.01820222628579e-06
1551 4.05073734310868e-06
1552 4.06563908050828e-06
1553 4.07869407759165e-06
1554 4.06166442790123e-06
1555 4.03230208156913e-06
1556 3.99575785636763e-06
1557 4.05689623916849e-06
1558 4.33814354505735e-06
1559 5.01867938562661e-06
1560 5.49262058147093e-06
1561 5.12805960561913e-06
1562 4.52883534629578e-06
1563 5.47327957289845e-06
1564 6.96488095508485e-06
1565 1.20529558866878e-05
1566 1.32935914045618e-05
1567 7.27369917630227e-06
1568 6.22887007195594e-06
1569 6.32018354451702e-06
1570 4.84914138088044e-06
1571 6.01499660257687e-06
1572 5.14743480817259e-06
1573 4.75941195876572e-06
1574 5.30610893600958e-06
1575 4.5966465187508e-06
1576 4.63409394590997e-06
1577 4.51934838219259e-06
1578 4.46145256871588e-06
1579 4.39996183132507e-06
1580 4.13759024731064e-06
1581 4.40731038842124e-06
1582 4.1898427056708e-06
1583 4.01415310502529e-06
1584 4.28188877776847e-06
1585 4.04847065427205e-06
1586 3.95856803581118e-06
1587 4.11199007555041e-06
1588 3.96091399079701e-06
1589 3.94953764704109e-06
1590 3.96524223467143e-06
1591 3.91283486811389e-06
1592 3.95241029449434e-06
1593 3.90964893459689e-06
1594 3.92628280698748e-06
1595 3.90588900422983e-06
1596 3.92471271359884e-06
1597 3.89922399701703e-06
1598 3.93584002755798e-06
1599 3.89752339224891e-06
1600 4.01230339597269e-06
1601 4.0039621167498e-06
1602 4.66608663129975e-06
1603 5.63311279355716e-06
1604 1.04062768375712e-05
1605 1.23975835339962e-05
1606 8.22715054837886e-06
1607 5.17313334125191e-06
1608 5.63569099121253e-06
1609 4.67594998543319e-06
1610 5.17457616944217e-06
1611 4.8181614582532e-06
1612 4.27668240021717e-06
1613 5.02323459694587e-06
1614 4.34827957818129e-06
1615 4.19258551365154e-06
1616 4.60985492334309e-06
1617 4.10374885850828e-06
1618 4.06635915584097e-06
1619 4.26816026166676e-06
1620 3.97705659582748e-06
1621 3.93831740508332e-06
1622 4.07153940829019e-06
1623 3.90215089685064e-06
1624 3.87897989595487e-06
1625 3.94426903338463e-06
1626 3.873240687291e-06
1627 3.89465729955241e-06
1628 3.87461253259502e-06
1629 3.88304507181747e-06
1630 3.87167165782909e-06
1631 3.8851839119225e-06
1632 3.87314199301603e-06
1633 3.90364379165842e-06
1634 3.88816671326175e-06
1635 3.97780111849322e-06
1636 3.97402463869767e-06
1637 4.417317225891e-06
1638 5.04366645381893e-06
1639 8.9992497520619e-06
1640 1.23928635016313e-05
1641 1.15558610236377e-05
1642 5.83310064516951e-06
1643 5.72979746316093e-06
1644 5.89210917123495e-06
1645 5.1603995527838e-06
1646 5.17691011481247e-06
1647 4.4454871510613e-06
1648 5.21605150272109e-06
1649 4.78482896859589e-06
1650 4.64549783596446e-06
1651 5.03391113659823e-06
1652 4.32787690141101e-06
1653 4.29428834403112e-06
1654 4.42894942964678e-06
1655 4.39397646401218e-06
1656 4.22133790101142e-06
1657 4.04444202439791e-06
1658 4.1847555061203e-06
1659 4.08356418413192e-06
1660 4.10178254206528e-06
1661 4.20344438012421e-06
1662 3.95022118238053e-06
1663 3.9057039662449e-06
1664 3.96937990920776e-06
1665 3.94364848754769e-06
1666 3.98345655805432e-06
1667 3.91301441893255e-06
1668 3.86385560702251e-06
1669 3.87452308248104e-06
1670 3.93424173994461e-06
1671 4.00857016213951e-06
1672 4.01000081606817e-06
1673 3.95737927449957e-06
1674 3.87958601011373e-06
1675 3.89206345929871e-06
1676 3.91585952330686e-06
1677 4.06144845022105e-06
1678 4.07693467052717e-06
1679 4.57257573571734e-06
1680 5.42973244899159e-06
1681 1.04894939125799e-05
1682 1.37734974057224e-05
1683 9.49476514655601e-06
1684 5.57275839963012e-06
1685 5.98548867714044e-06
1686 4.84461165228822e-06
1687 5.62116753055619e-06
1688 4.93901774589744e-06
1689 4.4911431902328e-06
1690 5.24081527997211e-06
1691 4.42503082243029e-06
1692 4.46993839764431e-06
1693 4.48905282013357e-06
1694 4.37477173997891e-06
1695 4.21957482554802e-06
1696 4.037539106716e-06
1697 4.36689905392029e-06
1698 4.01224811419354e-06
1699 3.89843726100736e-06
1700 4.23729927412353e-06
1701 3.90017320994751e-06
1702 3.85832143701137e-06
1703 4.0800644172112e-06
1704 3.84287542898321e-06
1705 3.82335457316252e-06
1706 3.89554818958082e-06
1707 3.79534806493886e-06
1708 3.83690356908772e-06
1709 3.81309108032912e-06
1710 3.81136197291632e-06
1711 3.80747980477381e-06
1712 3.79902036706792e-06
1713 3.80303958813499e-06
1714 3.81911417146608e-06
1715 3.86479421288222e-06
1716 3.91200278693216e-06
1717 3.9697083871193e-06
1718 3.95674407005497e-06
1719 3.96279086412044e-06
1720 3.88574298804123e-06
1721 4.26192453439889e-06
1722 5.10330171876738e-06
1723 1.01888642616288e-05
1724 1.44221647424558e-05
1725 1.14467521825645e-05
1726 5.58622663859865e-06
1727 6.2003413496825e-06
1728 5.47085207802667e-06
1729 5.83342728077696e-06
1730 5.07677498262638e-06
1731 4.54227666590867e-06
1732 5.42528747260462e-06
1733 4.40893187003688e-06
1734 4.36863598096693e-06
1735 4.50424803610794e-06
1736 4.42710529946488e-06
1737 4.20760662311892e-06
1738 4.05603015185552e-06
1739 4.44663040499726e-06
1740 3.96725044327084e-06
1741 3.90269966343837e-06
1742 4.31913183895904e-06
1743 3.85587011564326e-06
1744 3.83438534568015e-06
1745 4.13032479396058e-06
1746 3.80890993068039e-06
1747 3.78753412966226e-06
1748 3.96689898507496e-06
1749 3.76130033485911e-06
1750 3.76377257249594e-06
1751 3.83299436879092e-06
1752 3.74961139149033e-06
1753 3.78535681044845e-06
1754 3.75720863954321e-06
1755 3.7673318808995e-06
1756 3.76198079465873e-06
1757 3.77920655991737e-06
1758 3.77199672929596e-06
1759 3.79272428285837e-06
1760 3.76734825624503e-06
1761 3.81212294864319e-06
1762 3.774760431563e-06
1763 4.03816827065917e-06
1764 4.35448180224896e-06
1765 6.93536348372881e-06
1766 1.04792262263231e-05
1767 1.39811017678682e-05
1768 8.63532421035984e-06
1769 6.24128382353817e-06
1770 7.98312125915857e-06
1771 5.73102088563537e-06
1772 5.89259448346979e-06
1773 5.1166514776213e-06
1774 5.67696502340453e-06
1775 4.67419941330149e-06
1776 4.43119822701732e-06
1777 5.07607205557647e-06
1778 4.33114756925868e-06
1779 4.08550380859296e-06
1780 4.23009279360009e-06
1781 4.54664053273213e-06
1782 3.99585445043371e-06
1783 3.97821264686016e-06
1784 4.39869097634826e-06
1785 3.86740194402435e-06
1786 3.87031699267126e-06
1787 4.23860055598357e-06
1788 3.94590380770055e-06
1789 3.79563326613486e-06
1790 3.99070743095642e-06
1791 3.9286916364567e-06
1792 3.84361225647112e-06
1793 4.01798997651248e-06
1794 3.95900137017957e-06
1795 3.80024083301844e-06
1796 3.86670001173428e-06
1797 3.82895460337984e-06
1798 3.88123874806023e-06
1799 4.03905205548227e-06
1800 4.00526977295534e-06
1801 3.94075475829148e-06
1802 3.82302116475053e-06
1803 3.79811077388936e-06
1804 3.93823051747511e-06
1805 4.14919516567025e-06
1806 4.21068919576051e-06
1807 4.0355163961614e-06
1808 3.87448721062e-06
1809 3.89223029761965e-06
1810 4.0589572462002e-06
1811 4.12918796510731e-06
1812 4.10340323331315e-06
1813 3.91714782554953e-06
1814 4.28646354455609e-06
1815 5.39417536438158e-06
1816 1.10375223441039e-05
1817 1.44565141884101e-05
1818 9.84099298406704e-06
1819 5.82847686181331e-06
1820 6.0781083206507e-06
1821 5.15798511435683e-06
1822 6.45346968930482e-06
1823 4.68273408316122e-06
1824 4.61837522447439e-06
1825 5.49927075521595e-06
1826 4.38613924069742e-06
1827 4.15555758737085e-06
1828 4.47146089177508e-06
1829 4.63482201240595e-06
1830 3.86614427361209e-06
1831 4.0607686174754e-06
1832 4.65255050841051e-06
1833 3.80393519772149e-06
1834 3.90690725926568e-06
1835 4.44842393854117e-06
1836 3.81511565916703e-06
1837 3.82340549653915e-06
1838 4.23003374905306e-06
1839 3.81235639856925e-06
1840 3.77913381111039e-06
1841 4.06497543914419e-06
1842 3.77323941513907e-06
1843 3.76042367622453e-06
1844 3.93601550408018e-06
1845 3.73624787808335e-06
1846 3.76688715797435e-06
1847 3.8081460322914e-06
1848 3.72948040072707e-06
1849 3.77520962724365e-06
1850 3.73230146122339e-06
1851 3.74860968166413e-06
1852 3.72384420677108e-06
1853 3.74442964201638e-06
1854 3.71551716238372e-06
1855 3.75969486365868e-06
1856 3.72374022872268e-06
1857 3.87986802385942e-06
1858 3.96536674340808e-06
1859 5.09191698450095e-06
1860 7.04932556816473e-06
1861 1.2544505530343e-05
1862 1.09975441322518e-05
1863 6.36945035781267e-06
1864 8.48984950607701e-06
1865 5.84279063176041e-06
1866 5.5459588779172e-06
1867 6.99239732071533e-06
1868 4.76482370537212e-06
1869 4.53235953612463e-06
1870 5.08164068913075e-06
1871 5.19363089068747e-06
1872 3.89964187275105e-06
1873 4.39809550578651e-06
1874 5.25785095817355e-06
1875 3.9661619064546e-06
1876 4.20192872740976e-06
1877 5.1079753173866e-06
1878 4.31988734206357e-06
1879 4.16495715926946e-06
1880 4.92494805826027e-06
1881 4.64867908789302e-06
1882 4.37151872589858e-06
1883 4.96966447460778e-06
1884 4.86433927093799e-06
1885 4.43529824778466e-06
1886 4.85783623460634e-06
1887 4.98956766925573e-06
1888 4.85106819070324e-06
1889 5.3888013766823e-06
1890 5.48888748141962e-06
1891 5.09900775114147e-06
1892 5.33994971974039e-06
1893 5.60597393217677e-06
1894 5.6540912254377e-06
1895 5.90547395518115e-06
1896 5.91527487614485e-06
1897 5.96160764132492e-06
1898 6.62730621314722e-06
1899 7.34030167137689e-06
1900 8.07495514720813e-06
1901 9.72112164987493e-06
1902 1.2707423771019e-05
1903 1.87234942075065e-05
1904 3.42935080226425e-05
1905 7.69878063522356e-05
1906 0.000135328904082854
1907 7.17976141260124e-05
1908 2.40100858661663e-05
1909 1.77366323335093e-05
1910 7.32115277024548e-06
1911 4.08905162441187e-06
1912 3.99698774167234e-06
1913 3.90640745662196e-06
1914 3.81715345554312e-06
1915 3.76916926159154e-06
1916 3.74647221734747e-06
1917 3.73312323942443e-06
1918 3.72268670245646e-06
1919 3.71448276803577e-06
1920 3.70847615105419e-06
1921 3.70422736173381e-06
1922 3.70121764881048e-06
1923 3.69909553432635e-06
1924 3.69764016161511e-06
1925 3.69674977584999e-06
1926 3.69636120023387e-06
1927 3.69640519792824e-06
1928 3.69686574619443e-06
1929 3.69771899144666e-06
1930 3.69895594820591e-06
1931 3.70056940224295e-06
1932 3.70257014714603e-06
1933 3.70497416346538e-06
1934 3.70777977520831e-06
1935 3.7110127206752e-06
1936 3.71468430993005e-06
1937 3.71883134209217e-06
1938 3.72347580879229e-06
1939 3.72866794418059e-06
1940 3.73444563439573e-06
1941 3.74087141741253e-06
1942 3.74800980706969e-06
1943 3.75592255086321e-06
1944 3.76468281015896e-06
1945 3.77440942389562e-06
1946 3.78520503008239e-06
1947 3.7972049127255e-06
1948 3.81055224130122e-06
1949 3.82543334676022e-06
1950 3.84205753989342e-06
1951 3.86066718105482e-06
1952 3.88152677932574e-06
1953 3.90500266456151e-06
1954 3.93146951216394e-06
1955 3.96138477620767e-06
1956 3.99528856664944e-06
1957 4.03383819702618e-06
1958 4.07778999633734e-06
1959 4.12800138205327e-06
1960 4.18547716662943e-06
1961 4.25136890847e-06
1962 4.32700453489865e-06
1963 4.41381832727572e-06
1964 4.51340398210931e-06
1965 4.62738000361895e-06
1966 4.75734048510645e-06
1967 4.90462787228196e-06
1968 5.07012659656247e-06
1969 5.25384952387853e-06
1970 5.45449810163134e-06
1971 5.66884484975994e-06
1972 5.89100013415056e-06
1973 6.11180259824096e-06
1974 6.31853606991584e-06
1975 6.495038416654e-06
1976 6.62259401251752e-06
1977 6.68198216446925e-06
1978 6.65643950092942e-06
1979 6.53546645512648e-06
1980 6.3178333233882e-06
1981 6.01356691687371e-06
1982 5.64344196218869e-06
1983 5.2369510635808e-06
1984 4.82897900033752e-06
1985 4.45719286834567e-06
1986 4.16184855644985e-06
1987 3.98861288219976e-06
1988 3.99487055124936e-06
1989 4.25978651996672e-06
1990 4.89767451128387e-06
1991 6.07193430113284e-06
1992 7.99481893132281e-06
1993 1.0858603324948e-05
1994 1.45510332307808e-05
1995 1.79968033926503e-05
1996 1.88088444854628e-05
1997 1.54260764055891e-05
1998 1.02193727506972e-05
1999 6.95496640457094e-06
};
\addlegendentry{Train}
\addplot [semithick, black]
table {%
0 0.0151413101702929
1 0.0145246451720595
2 0.0139026269316673
3 0.013222137466073
4 0.0123859150335193
5 0.0113420393317938
6 0.0103444578126073
7 0.00959832221269608
8 0.00902600679546595
9 0.00858955644071102
10 0.00824683997780085
11 0.00796984601765871
12 0.00774148060008883
13 0.00755056738853455
14 0.00738917291164398
15 0.00725156161934137
16 0.00713344104588032
17 0.00703139184042811
18 0.00694253435358405
19 0.00686434097588062
20 0.00679455837234855
21 0.006731191650033
22 0.00667248293757439
23 0.00661688391119242
24 0.00656302459537983
25 0.00650967378169298
26 0.00645569898188114
27 0.00640002405270934
28 0.00634158495813608
29 0.00627931440249085
30 0.00621210830286145
31 0.00613871542736888
32 0.00605767173692584
33 0.00596718583256006
34 0.005864926148206
35 0.00574852107092738
36 0.00561631331220269
37 0.00546315172687173
38 0.00528315780684352
39 0.0050692274235189
40 0.00481237564235926
41 0.00450162589550018
42 0.00412523373961449
43 0.00367609784007072
44 0.0031647807918489
45 0.00263561028987169
46 0.00216450751759112
47 0.0018045041942969
48 0.0015436967369169
49 0.00134636415168643
50 0.00119183713104576
51 0.00106935750227422
52 0.000971493136603385
53 0.000893083750270307
54 0.000829847122076899
55 0.000778156332671642
56 0.00073518167482689
57 0.000698858755640686
58 0.000667726097162813
59 0.000640711397863925
60 0.00061699515208602
61 0.000595944467931986
62 0.000577075697947294
63 0.000560018059331924
64 0.000544481619726866
65 0.000530232209712267
66 0.000517100736033171
67 0.000504949013702571
68 0.000493666098918766
69 0.000483145326143131
70 0.000473290332593024
71 0.00046401986037381
72 0.0004552676982712
73 0.000446977355750278
74 0.000439102062955499
75 0.000431601976742968
76 0.000424443045631051
77 0.00041759671876207
78 0.00041103849071078
79 0.000404746999265626
80 0.000398704403778538
81 0.0003928950172849
82 0.000387305073672906
83 0.000381922844098881
84 0.000376737909391522
85 0.00037174072349444
86 0.00036692270077765
87 0.000362276274245232
88 0.000357794458977878
89 0.000353470706613734
90 0.000349298847140744
91 0.000345272885169834
92 0.000341387349180877
93 0.000337637320626527
94 0.000334017415298149
95 0.0003305223362986
96 0.000327146903146058
97 0.00032388637191616
98 0.000320736260619015
99 0.00031769199995324
100 0.000314749049721286
101 0.000311903306283057
102 0.000309150607790798
103 0.000306486792396754
104 0.000303908018395305
105 0.000301410560496151
106 0.00029899034416303
107 0.000296642654575408
108 0.000294365483568981
109 0.00029215554241091
110 0.00029001064831391
111 0.000287928094621748
112 0.000285905291093513
113 0.000283939589280635
114 0.000282028369838372
115 0.000280169624602422
116 0.000278360967058688
117 0.000276600214419886
118 0.000274885358521715
119 0.00027321424568072
120 0.00027158492594026
121 0.000269995623966679
122 0.000268444651737809
123 0.000266930233919993
124 0.000265450566075742
125 0.00026400393107906
126 0.00026258869911544
127 0.000261203327681869
128 0.000259846274275333
129 0.000258515967288986
130 0.000257210893323645
131 0.000255929422564805
132 0.000254670099820942
133 0.000253431586315855
134 0.000252212019404396
135 0.00025100982747972
136 0.000249823147896677
137 0.000248650263529271
138 0.000247490184847265
139 0.000246341136517003
140 0.000245201430516317
141 0.000244069189648144
142 0.000242942522163503
143 0.000241819405346178
144 0.000240697685512714
145 0.000239575136220083
146 0.000238449472817592
147 0.000237318512517959
148 0.000236179228522815
149 0.000235029336181469
150 0.000233866740018129
151 0.000232689417316578
152 0.00023149503977038
153 0.000230280638788827
154 0.000229043289436959
155 0.000227779833949171
156 0.000226487536565401
157 0.000225163763388991
158 0.000223805414861999
159 0.000222408954869024
160 0.000220970789087005
161 0.00021948765788693
162 0.000217955297557637
163 0.000216370666748844
164 0.000214730403968133
165 0.000213031613384373
166 0.000211270875297487
167 0.00020944511925336
168 0.000207550896448083
169 0.000205584859941155
170 0.000203543677343987
171 0.000201424161787145
172 0.000199224115931429
173 0.000196941080503166
174 0.000194574284250848
175 0.00019212243205402
176 0.000189585742191412
177 0.000186969366041012
178 0.000184278571396135
179 0.00018151622498408
180 0.000178687419975176
181 0.00017579851555638
182 0.000172857427969575
183 0.000169872786500491
184 0.000166853293194436
185 0.000163807460921817
186 0.000160746887559071
187 0.000157682196004316
188 0.000154625129653141
189 0.000151589192682877
190 0.000148585095303133
191 0.00014562405704055
192 0.000142714809044264
193 0.000139866140671074
194 0.000137087190523744
195 0.000134384536067955
196 0.000131763343233615
197 0.000129227279103361
198 0.000126779588754289
199 0.00012442194565665
200 0.00012215408787597
201 0.000119976757559925
202 0.000117887851956766
203 0.000115884089609608
204 0.000113961810711771
205 0.000112117522803601
206 0.00011034728231607
207 0.000108646832813974
208 0.000107011932414025
209 0.00010543813550612
210 0.000103921338450164
211 0.000102457554021385
212 0.000101042933238205
213 9.9673903605435e-05
214 9.83471109066159e-05
215 9.70594483078457e-05
216 9.58080127020366e-05
217 9.45902793318965e-05
218 9.34038762352429e-05
219 9.22468971111812e-05
220 9.11175375222228e-05
221 9.00140585144982e-05
222 8.89350776560605e-05
223 8.78792707226239e-05
224 8.68455099407583e-05
225 8.58327257446945e-05
226 8.48402196425013e-05
227 8.38670966913924e-05
228 8.29127529868856e-05
229 8.19764682091773e-05
230 8.10577257652767e-05
231 8.01557544036768e-05
232 7.92699938756414e-05
233 7.8399789344985e-05
234 7.75444641476497e-05
235 7.67033707234077e-05
236 7.58758033043705e-05
237 7.50610779505223e-05
238 7.42584888939746e-05
239 7.34675995772704e-05
240 7.26877624401823e-05
241 7.19185773050413e-05
242 7.11595275788568e-05
243 7.04102858435363e-05
244 6.96705683367327e-05
245 6.89401713316329e-05
246 6.82188401697204e-05
247 6.75064220558852e-05
248 6.68028151267208e-05
249 6.61079975543544e-05
250 6.5421852923464e-05
251 6.4744395785965e-05
252 6.40756406937726e-05
253 6.34156094747595e-05
254 6.27642657491378e-05
255 6.21217623120174e-05
256 6.14885429968126e-05
257 6.08651025686413e-05
258 6.02517939114477e-05
259 5.96487043367233e-05
260 5.9055815654574e-05
261 5.84731096751057e-05
262 5.79004226892721e-05
263 5.73377001273911e-05
264 5.6784749176586e-05
265 5.62414752494078e-05
266 5.57076564291492e-05
267 5.51831944903824e-05
268 5.46678929822519e-05
269 5.41616427653935e-05
270 5.36642110091634e-05
271 5.31754012627061e-05
272 5.2695075282827e-05
273 5.22229784110095e-05
274 5.17589069204405e-05
275 5.13026461703703e-05
276 5.08540688315406e-05
277 5.04128693137318e-05
278 4.99788875458762e-05
279 4.95518797833938e-05
280 4.91316786792595e-05
281 4.87180950585753e-05
282 4.83109106426127e-05
283 4.79099617223255e-05
284 4.7515073674731e-05
285 4.71260609629098e-05
286 4.67427889816463e-05
287 4.63649848825298e-05
288 4.59926814073697e-05
289 4.56255475000944e-05
290 4.52634994871914e-05
291 4.49063991254661e-05
292 4.45540608779993e-05
293 4.42063537775539e-05
294 4.38631468568929e-05
295 4.35242945968639e-05
296 4.31896769441664e-05
297 4.28590610681567e-05
298 4.25323487434071e-05
299 4.22094999521505e-05
300 4.18903800891712e-05
301 4.15749527746812e-05
302 4.12630834034644e-05
303 4.09547210438177e-05
304 4.0649734728504e-05
305 4.03480044042226e-05
306 4.00494973291643e-05
307 3.97540570702404e-05
308 3.94616763514932e-05
309 3.91722023778129e-05
310 3.88855223718565e-05
311 3.86016254196875e-05
312 3.83203732781112e-05
313 3.80417513952125e-05
314 3.77656397176906e-05
315 3.74919873138424e-05
316 3.72207614418585e-05
317 3.69518238585442e-05
318 3.66851782018784e-05
319 3.64207844540942e-05
320 3.61585662176367e-05
321 3.58984834747389e-05
322 3.56404671038035e-05
323 3.53843679476995e-05
324 3.51302187482361e-05
325 3.48781031789258e-05
326 3.46279230143409e-05
327 3.4379667340545e-05
328 3.41333034157287e-05
329 3.38888021360617e-05
330 3.36460943799466e-05
331 3.34051583195105e-05
332 3.31660157826263e-05
333 3.29285503539722e-05
334 3.26927874993999e-05
335 3.24587272189092e-05
336 3.2226260373136e-05
337 3.19954378937837e-05
338 3.17662343150005e-05
339 3.15386132569984e-05
340 3.13125783577561e-05
341 3.1088125979295e-05
342 3.08652524836361e-05
343 3.06439542328008e-05
344 3.04241802950855e-05
345 3.02059524983633e-05
346 2.99892781185918e-05
347 2.97741462418344e-05
348 2.95605495921336e-05
349 2.93484790745424e-05
350 2.91378964902833e-05
351 2.89288509520702e-05
352 2.8721287890221e-05
353 2.85152545984602e-05
354 2.831077472365e-05
355 2.81077736872248e-05
356 2.79062915069517e-05
357 2.77062536042649e-05
358 2.7507679988048e-05
359 2.73105670203222e-05
360 2.71148855972569e-05
361 2.69205684162444e-05
362 2.67276645899983e-05
363 2.65360922639957e-05
364 2.63459314737702e-05
365 2.61570949078305e-05
366 2.59696353168692e-05
367 2.57834435615223e-05
368 2.55986924457829e-05
369 2.54153965215664e-05
370 2.52334448305191e-05
371 2.50527555181179e-05
372 2.48732849286171e-05
373 2.46949039137689e-05
374 2.45176852331497e-05
375 2.43416270677699e-05
376 2.41667894442799e-05
377 2.39930650423048e-05
378 2.38205793721136e-05
379 2.3649277864024e-05
380 2.34791405091528e-05
381 2.3310145479627e-05
382 2.31422636716161e-05
383 2.29754386964487e-05
384 2.28096687351353e-05
385 2.2645021090284e-05
386 2.24814775720006e-05
387 2.23189890675712e-05
388 2.21574118768331e-05
389 2.199691698479e-05
390 2.18375043914421e-05
391 2.16790140257217e-05
392 2.15215331991203e-05
393 2.13650073419558e-05
394 2.12095292226877e-05
395 2.10552425414789e-05
396 2.09020854526898e-05
397 2.07500597753096e-05
398 2.05991800612537e-05
399 2.04494008357869e-05
400 2.03008385142311e-05
401 2.01535112864804e-05
402 2.00073718588101e-05
403 1.98624256881885e-05
404 1.97187546291389e-05
405 1.95764059753856e-05
406 1.94351960089989e-05
407 1.92951483768411e-05
408 1.91563831322128e-05
409 1.90188256965484e-05
410 1.88825579243712e-05
411 1.87475052371155e-05
412 1.86137349373894e-05
413 1.8481297956896e-05
414 1.835008879425e-05
415 1.82201565621654e-05
416 1.80915558303241e-05
417 1.79642393050017e-05
418 1.78382033482194e-05
419 1.77134388650302e-05
420 1.75899858732009e-05
421 1.74678043549648e-05
422 1.73468815773958e-05
423 1.7227248463314e-05
424 1.71087995113339e-05
425 1.69916274899151e-05
426 1.68756760103861e-05
427 1.67609723575879e-05
428 1.66474837897113e-05
429 1.65352612384595e-05
430 1.6424288332928e-05
431 1.63145577971591e-05
432 1.62060787260998e-05
433 1.60988693096442e-05
434 1.59928767970996e-05
435 1.58881248353282e-05
436 1.57845588546479e-05
437 1.56821279233554e-05
438 1.55809775606031e-05
439 1.54811023094226e-05
440 1.53824657900259e-05
441 1.52850880112965e-05
442 1.51889616972767e-05
443 1.50940149978851e-05
444 1.50002833834151e-05
445 1.4907820514054e-05
446 1.48164090205682e-05
447 1.47261498568696e-05
448 1.4637058484368e-05
449 1.45490921568125e-05
450 1.44623782034614e-05
451 1.43767611007206e-05
452 1.42921835504239e-05
453 1.42086937557906e-05
454 1.41261962198769e-05
455 1.40446736622835e-05
456 1.39642770591308e-05
457 1.38848972710548e-05
458 1.38065515784547e-05
459 1.3729296370002e-05
460 1.36531234602444e-05
461 1.35779373522382e-05
462 1.35035716084531e-05
463 1.3429958926281e-05
464 1.33573084895033e-05
465 1.32858422148274e-05
466 1.32154682432883e-05
467 1.31459801195888e-05
468 1.30774033095804e-05
469 1.30097860164824e-05
470 1.2943118235853e-05
471 1.28774372569751e-05
472 1.28127367133857e-05
473 1.27489529404556e-05
474 1.26860259115347e-05
475 1.2624031114683e-05
476 1.25629276226391e-05
477 1.2502672689152e-05
478 1.2443252671801e-05
479 1.23846466522082e-05
480 1.23268046081648e-05
481 1.22697319966392e-05
482 1.22133669719915e-05
483 1.21577904792503e-05
484 1.21028824651148e-05
485 1.20487084132037e-05
486 1.19952037493931e-05
487 1.19423893920612e-05
488 1.1890277164639e-05
489 1.18389116323669e-05
490 1.17881327241776e-05
491 1.17381177915377e-05
492 1.16888459160691e-05
493 1.16402352432488e-05
494 1.15922985060024e-05
495 1.15450438897824e-05
496 1.14983777166344e-05
497 1.14522608782863e-05
498 1.14067834147136e-05
499 1.13619207695592e-05
500 1.13176329250564e-05
501 1.12738453026395e-05
502 1.12306379378424e-05
503 1.11879990072339e-05
504 1.11458584797219e-05
505 1.11042845674092e-05
506 1.10632099676877e-05
507 1.10226492324728e-05
508 1.09825414256193e-05
509 1.09428920040955e-05
510 1.09037437141524e-05
511 1.08650838228641e-05
512 1.08268832264002e-05
513 1.07892665255349e-05
514 1.07520545498119e-05
515 1.07154028228251e-05
516 1.06791885627899e-05
517 1.06433617474977e-05
518 1.06079833130934e-05
519 1.05729686765699e-05
520 1.05383378468105e-05
521 1.05041426650132e-05
522 1.04703167380649e-05
523 1.04369082691846e-05
524 1.0403937267256e-05
525 1.03713764474378e-05
526 1.03392103483202e-05
527 1.03076708910521e-05
528 1.02766671261634e-05
529 1.02462108770851e-05
530 1.02162784969551e-05
531 1.01869318314129e-05
532 1.01581808849005e-05
533 1.01299629022833e-05
534 1.01022496892256e-05
535 1.00749730336247e-05
536 1.00481029221555e-05
537 1.00214638223406e-05
538 9.99508210952627e-06
539 9.96884045889601e-06
540 9.94256515696179e-06
541 9.9162534752395e-06
542 9.8898881333298e-06
543 9.86349095910555e-06
544 9.83706104307203e-06
545 9.81068751571001e-06
546 9.78453772404464e-06
547 9.75877264863811e-06
548 9.73351234279107e-06
549 9.70898690866306e-06
550 9.68529366218718e-06
551 9.66271545621566e-06
552 9.64129958447302e-06
553 9.62107151281089e-06
554 9.60193756327499e-06
555 9.58379132498521e-06
556 9.56645544647472e-06
557 9.54945699049858e-06
558 9.53234848566353e-06
559 9.51466699916637e-06
560 9.4957558758324e-06
561 9.47500757320086e-06
562 9.45181636780035e-06
563 9.42575206863694e-06
564 9.39655092224712e-06
565 9.36415017349645e-06
566 9.32892544369679e-06
567 9.29155430640094e-06
568 9.25317635847023e-06
569 9.21533501241356e-06
570 9.17989927984308e-06
571 9.14919655770063e-06
572 9.12626546778483e-06
573 9.11471033759881e-06
574 9.11953247850761e-06
575 9.14756765268976e-06
576 9.209249583364e-06
577 9.31966587813804e-06
578 9.50062258198159e-06
579 9.77949366642861e-06
580 1.01751365946257e-05
581 1.06461093309917e-05
582 1.10023238448775e-05
583 1.09286847873591e-05
584 1.03133088487084e-05
585 9.52797381614801e-06
586 9.19589820114197e-06
587 9.51418860495323e-06
588 9.47729313338641e-06
589 8.86936868482735e-06
590 8.91297713678796e-06
591 9.29972975427518e-06
592 9.40025893214624e-06
593 9.14344127522781e-06
594 8.7503804024891e-06
595 8.46000421006465e-06
596 8.31770285003586e-06
597 8.25377537694294e-06
598 8.24769631435629e-06
599 8.35201262816554e-06
600 8.59836927702418e-06
601 8.9601444415166e-06
602 9.39133133215364e-06
603 9.83941026788671e-06
604 1.02116955531528e-05
605 1.03589281934546e-05
606 1.01633468148066e-05
607 9.69400389294606e-06
608 9.20210732147098e-06
609 8.89272359927418e-06
610 8.7677199189784e-06
611 8.70992334967013e-06
612 8.63052082422655e-06
613 8.51320328365546e-06
614 8.38229334476637e-06
615 8.26995892566629e-06
616 8.19738670543302e-06
617 8.16564443084644e-06
618 8.16129340819316e-06
619 8.16987721918849e-06
620 8.18647913547466e-06
621 8.21888897917233e-06
622 8.28649990580743e-06
623 8.4171942944522e-06
624 8.64187040860998e-06
625 8.98132111615269e-06
626 9.41985308600124e-06
627 9.86955728876637e-06
628 1.01676550912089e-05
629 1.01579453257727e-05
630 9.81982429948403e-06
631 9.30604892346309e-06
632 8.83399025042308e-06
633 8.54731206345605e-06
634 8.45870908960933e-06
635 8.45774320623605e-06
636 8.37556672195205e-06
637 8.1770122051239e-06
638 8.02231261332054e-06
639 8.0228583101416e-06
640 8.12670714367414e-06
641 8.23052050691331e-06
642 8.26689392852131e-06
643 8.22701258584857e-06
644 8.1507960203453e-06
645 8.09512312116567e-06
646 8.10425626696087e-06
647 8.19904016680084e-06
648 8.38001324154902e-06
649 8.63633431436028e-06
650 8.95346420293208e-06
651 9.30432361201383e-06
652 9.62393278314266e-06
653 9.79495871433755e-06
654 9.69429765973473e-06
655 9.30465557757998e-06
656 8.78364426171174e-06
657 8.37637617223663e-06
658 8.23461232357658e-06
659 8.2844298958662e-06
660 8.25320148578612e-06
661 8.0371182775707e-06
662 7.8678322097403e-06
663 7.89331807027338e-06
664 8.02987324277638e-06
665 8.15237854112638e-06
666 8.18807529867627e-06
667 8.12785492598778e-06
668 8.02194881543983e-06
669 7.9421633927268e-06
670 7.93800518295029e-06
671 8.02648537501227e-06
672 8.20406876300694e-06
673 8.46204329718603e-06
674 8.79493472893955e-06
675 9.18430032470496e-06
676 9.56081112235552e-06
677 9.78824937192257e-06
678 9.71847657638136e-06
679 9.32246985030361e-06
680 8.76973717822693e-06
681 8.32499335956527e-06
682 8.15806470200187e-06
683 8.20889454189455e-06
684 8.21166122477734e-06
685 8.02077465777984e-06
686 7.81958442530595e-06
687 7.778056897223e-06
688 7.85369411460124e-06
689 7.94199877418578e-06
690 7.97721986600664e-06
691 7.94106836110586e-06
692 7.85941483627539e-06
693 7.78401954448782e-06
694 7.76014348957688e-06
695 7.81307608122006e-06
696 7.95701089373324e-06
697 8.20591412775684e-06
698 8.57336999615654e-06
699 9.05297656572657e-06
700 9.58216241997434e-06
701 1.00149700301699e-05
702 1.01498917501885e-05
703 9.85829956334783e-06
704 9.23829247767571e-06
705 8.59671854414046e-06
706 8.22848051029723e-06
707 8.21604680822929e-06
708 8.34294587548357e-06
709 8.2574979387573e-06
710 7.95739106251858e-06
711 7.75054195401026e-06
712 7.73448027757695e-06
713 7.7997574408073e-06
714 7.84885560278781e-06
715 7.84070653025992e-06
716 7.77417062636232e-06
717 7.68249628890771e-06
718 7.61127148507512e-06
719 7.58979285819805e-06
720 7.62918443797389e-06
721 7.73923784436192e-06
722 7.9414458014071e-06
723 8.26834912004415e-06
724 8.74331453815103e-06
725 9.34206673264271e-06
726 9.94371839624364e-06
727 1.03222637335421e-05
728 1.02534768302576e-05
729 9.70994551607873e-06
730 8.95804350875551e-06
731 8.37253901408985e-06
732 8.16630745248403e-06
733 8.27230451250216e-06
734 8.36270100990077e-06
735 8.16918873169925e-06
736 7.85748670750763e-06
737 7.70176484365948e-06
738 7.70291808294132e-06
739 7.74488762544934e-06
740 7.75779335526749e-06
741 7.7214272096171e-06
742 7.64637752581621e-06
743 7.5646807999874e-06
744 7.50921071812627e-06
745 7.49626724427799e-06
746 7.53068388803513e-06
747 7.62217177907587e-06
748 7.79522451921366e-06
749 8.08783443062566e-06
750 8.53842448123032e-06
751 9.15335749596125e-06
752 9.85462884273147e-06
753 1.04342007034575e-05
754 1.06055158539675e-05
755 1.02097319540917e-05
756 9.41547841648571e-06
757 8.63830609887373e-06
758 8.22288802737603e-06
759 8.23798745841486e-06
760 8.44067926664138e-06
761 8.40676966618048e-06
762 8.0599966167938e-06
763 7.76796605350683e-06
764 7.69934831623686e-06
765 7.72766179579776e-06
766 7.73314150137594e-06
767 7.68302652431885e-06
768 7.59407339501195e-06
769 7.50314711694955e-06
770 7.44101225791383e-06
771 7.41559688322013e-06
772 7.4210192906321e-06
773 7.45775923860492e-06
774 7.54253323975718e-06
775 7.70583847042872e-06
776 7.98546989244642e-06
777 8.41915516502922e-06
778 9.02634110389045e-06
779 9.7619958978612e-06
780 1.04533492049086e-05
781 1.07967280200683e-05
782 1.0535865840211e-05
783 9.74295744526898e-06
784 8.8453853095416e-06
785 8.28963129606564e-06
786 8.21747471491108e-06
787 8.43519592308439e-06
788 8.50709238875424e-06
789 8.19903561932733e-06
790 7.83522409619763e-06
791 7.70655515225371e-06
792 7.71668965171557e-06
793 7.71081886341562e-06
794 7.64341530157253e-06
795 7.54335451347288e-06
796 7.45616762287682e-06
797 7.40510222385637e-06
798 7.38320204618503e-06
799 7.37479240342509e-06
800 7.38008156986325e-06
801 7.41807798476657e-06
802 7.51261177356355e-06
803 7.68468362366548e-06
804 7.95753112470265e-06
805 8.36309664009605e-06
806 8.93436026672134e-06
807 9.66634615906514e-06
808 1.04376003946527e-05
809 1.09561487988685e-05
810 1.08829999589943e-05
811 1.01564237411367e-05
812 9.16020053409738e-06
813 8.43594989419216e-06
814 8.23266600491479e-06
815 8.42419922264526e-06
816 8.611465091235e-06
817 8.39761651150184e-06
818 7.94634161138674e-06
819 7.70992573961848e-06
820 7.69637972553028e-06
821 7.71091708884342e-06
822 7.65177719586063e-06
823 7.54505163058639e-06
824 7.44972294342006e-06
825 7.39960933060502e-06
826 7.38314383852412e-06
827 7.36928859623731e-06
828 7.35074854674167e-06
829 7.3534574767109e-06
830 7.4063955253223e-06
831 7.51800962461857e-06
832 7.68476274970453e-06
833 7.91133879829431e-06
834 8.22228594188346e-06
835 8.65904166857945e-06
836 9.25533731788164e-06
837 9.97764891508268e-06
838 1.06444431366981e-05
839 1.09328348116833e-05
840 1.05988383438671e-05
841 9.7616384664434e-06
842 8.87789155967766e-06
843 8.35179253044771e-06
844 8.25260303827235e-06
845 8.44436453917297e-06
846 8.62169872561935e-06
847 8.39654967421666e-06
848 7.91095953900367e-06
849 7.64873038860969e-06
850 7.67286564951064e-06
851 7.72985913499724e-06
852 7.69319649407407e-06
853 7.57544694351964e-06
854 7.46176237953478e-06
855 7.40417362976586e-06
856 7.393391115329e-06
857 7.38147764423047e-06
858 7.34443665351137e-06
859 7.31869158698828e-06
860 7.35449975763913e-06
861 7.46200521462015e-06
862 7.61905903345905e-06
863 7.80711161496583e-06
864 8.03420061856741e-06
865 8.33909234643215e-06
866 8.78035916684894e-06
867 9.40227164392127e-06
868 1.01650739452452e-05
869 1.08615004137391e-05
870 1.11414601633442e-05
871 1.07709984149551e-05
872 9.90113312582253e-06
873 9.0065695985686e-06
874 8.45947215566412e-06
875 8.30207227409119e-06
876 8.45726390252821e-06
877 8.7185244410648e-06
878 8.57561462908052e-06
879 7.99437020759797e-06
880 7.62724994274322e-06
881 7.67451365391025e-06
882 7.76458819018444e-06
883 7.72879502619617e-06
884 7.58058104111115e-06
885 7.45301394999842e-06
886 7.39908773539355e-06
887 7.39661436455208e-06
888 7.38069547878695e-06
889 7.33272418074193e-06
890 7.30809642845998e-06
891 7.35019784769975e-06
892 7.44427143217763e-06
893 7.54915754441754e-06
894 7.63951902627014e-06
895 7.7203467299114e-06
896 7.82183724368224e-06
897 7.99004919826984e-06
898 8.27969051897526e-06
899 8.7466860350105e-06
900 9.42515453061787e-06
901 1.02701742434874e-05
902 1.10746659629513e-05
903 1.14794411274488e-05
904 1.11967729026219e-05
905 1.02945023172651e-05
906 9.24706182559021e-06
907 8.53554865898332e-06
908 8.36277104099281e-06
909 8.70717121870257e-06
910 9.0579669631552e-06
911 8.61585158418166e-06
912 7.95069354353473e-06
913 7.77601508161752e-06
914 7.87952103564749e-06
915 7.76597426010994e-06
916 7.60279817768605e-06
917 7.41513804314309e-06
918 7.3800561040116e-06
919 7.36756328478805e-06
920 7.35941375751281e-06
921 7.32970966055291e-06
922 7.32476382836467e-06
923 7.37050231691683e-06
924 7.4209137892467e-06
925 7.44830049370648e-06
926 7.44423869036837e-06
927 7.41959956940264e-06
928 7.39520555725903e-06
929 7.39029565011151e-06
930 7.42197016734281e-06
931 7.51439620216843e-06
932 7.71329268900445e-06
933 8.09676566859707e-06
934 8.76981084729778e-06
935 9.81498124019708e-06
936 1.11621811811347e-05
937 1.24127182061784e-05
938 1.28953934108722e-05
939 1.21343045975664e-05
940 1.04494974948466e-05
941 8.9657360149431e-06
942 8.44891837914474e-06
943 8.77931688592071e-06
944 9.10359267436434e-06
945 8.56906353874365e-06
946 7.91673301137052e-06
947 7.68802146922098e-06
948 7.76351043896284e-06
949 7.57272800910869e-06
950 7.52622236177558e-06
951 7.34681179892505e-06
952 7.45718034522724e-06
953 7.31839872969431e-06
954 7.43279724702006e-06
955 7.26770349501749e-06
956 7.35324010747718e-06
957 7.34371951693902e-06
958 7.37376194592798e-06
959 7.39992538001388e-06
960 7.36162292014342e-06
961 7.33389242668636e-06
962 7.30346255295444e-06
963 7.27685892343288e-06
964 7.25845666238456e-06
965 7.24731808077195e-06
966 7.25655036148964e-06
967 7.31523323338479e-06
968 7.46072828405886e-06
969 7.73837564338464e-06
970 8.21220874058781e-06
971 8.96244637260679e-06
972 1.00339257187443e-05
973 1.1309713954688e-05
974 1.2379297913867e-05
975 1.26138893392636e-05
976 1.15906414066558e-05
977 9.803856301005e-06
978 8.54445352160838e-06
979 8.30585941002937e-06
980 8.58321436680853e-06
981 8.76000740390737e-06
982 8.32174646347994e-06
983 7.90393823990598e-06
984 7.39132610760862e-06
985 7.73557530919788e-06
986 7.4748572842509e-06
987 8.15489693195559e-06
988 8.13875885796733e-06
989 1.07329369711806e-05
990 3.28250062011648e-05
991 0.000128803338157013
992 3.43366227752995e-05
993 1.08342228486435e-05
994 2.49836593866348e-05
995 4.12360132031608e-05
996 5.34832106495742e-05
997 7.73597348597832e-05
998 0.000103262551419903
999 9.83277277555317e-05
1000 5.00360547448508e-05
1001 1.67468042491237e-05
1002 1.00596398624475e-05
1003 9.60608213063097e-06
1004 8.6163863670663e-06
1005 7.71353461459512e-06
1006 7.521872248617e-06
1007 7.77315472078044e-06
1008 8.08906679594656e-06
1009 8.30191584100248e-06
1010 8.39781932882033e-06
1011 8.41490145830903e-06
1012 8.39199765323428e-06
1013 8.3546592577477e-06
1014 8.31663601275068e-06
1015 8.28394695417956e-06
1016 8.25831648398889e-06
1017 8.23936443339335e-06
1018 8.22592392069055e-06
1019 8.21660523797618e-06
1020 8.2100996223744e-06
1021 8.20543937152252e-06
1022 8.20169316284591e-06
1023 8.19824344944209e-06
1024 8.19459273770917e-06
1025 8.19026263343403e-06
1026 8.18489843368297e-06
1027 8.17818363429978e-06
1028 8.16972351458389e-06
1029 8.15923976915656e-06
1030 8.14635313872714e-06
1031 8.13066708360566e-06
1032 8.11178324511275e-06
1033 8.08944969321601e-06
1034 8.06314619694604e-06
1035 8.03262082627043e-06
1036 7.99759800429456e-06
1037 7.9578621807741e-06
1038 7.91328784544021e-06
1039 7.86397140473127e-06
1040 7.81019025453134e-06
1041 7.75238822825486e-06
1042 7.6913147495361e-06
1043 7.62804302212317e-06
1044 7.5638872658601e-06
1045 7.50045910535846e-06
1046 7.43962800697773e-06
1047 7.38329117666581e-06
1048 7.33349543224904e-06
1049 7.29206794858328e-06
1050 7.26053531252546e-06
1051 7.23995526641374e-06
1052 7.23079801900894e-06
1053 7.23286530046607e-06
1054 7.24530809748103e-06
1055 7.26677080820082e-06
1056 7.29542216504342e-06
1057 7.32926037017023e-06
1058 7.36619222152513e-06
1059 7.40424957257346e-06
1060 7.44162935006898e-06
1061 7.47683543522726e-06
1062 7.50872231947142e-06
1063 7.53635276851128e-06
1064 7.55921655581915e-06
1065 7.57704219722655e-06
1066 7.5897164606431e-06
1067 7.59725389798405e-06
1068 7.59995646149036e-06
1069 7.59820932216826e-06
1070 7.592382189614e-06
1071 7.58305304771056e-06
1072 7.57071484258631e-06
1073 7.55595692680799e-06
1074 7.53922677176888e-06
1075 7.52108962842613e-06
1076 7.50193703424884e-06
1077 7.48208594814059e-06
1078 7.46199611967313e-06
1079 7.44200860935962e-06
1080 7.42230849937187e-06
1081 7.40308087188168e-06
1082 7.38452581572346e-06
1083 7.36667925593792e-06
1084 7.34973173166509e-06
1085 7.33366141503211e-06
1086 7.31846057533403e-06
1087 7.30428973838571e-06
1088 7.29105749996961e-06
1089 7.27873430150794e-06
1090 7.26723646948813e-06
1091 7.25659947420354e-06
1092 7.24664369045058e-06
1093 7.23732546248357e-06
1094 7.22856202628464e-06
1095 7.22034792488557e-06
1096 7.21261312719434e-06
1097 7.20531761544407e-06
1098 7.19848867447581e-06
1099 7.19207582733361e-06
1100 7.18605497240787e-06
1101 7.18042383596185e-06
1102 7.17513694326044e-06
1103 7.17015018381062e-06
1104 7.16542899681372e-06
1105 7.16089971319889e-06
1106 7.15652367944131e-06
1107 7.15227497494197e-06
1108 7.14808538759826e-06
1109 7.14393627276877e-06
1110 7.13978670319193e-06
1111 7.13566805643495e-06
1112 7.13160034138127e-06
1113 7.12755445420044e-06
1114 7.12368455424439e-06
1115 7.11994880475686e-06
1116 7.11655457052984e-06
1117 7.11349821358453e-06
1118 7.11103439243743e-06
1119 7.10925951352692e-06
1120 7.10849280949333e-06
1121 7.10901076672599e-06
1122 7.11139955456019e-06
1123 7.1163531174534e-06
1124 7.12528117219335e-06
1125 7.14025463821599e-06
1126 7.16592057870002e-06
1127 7.21066271580639e-06
1128 7.29525390852359e-06
1129 7.46558725950308e-06
1130 7.84368603490293e-06
1131 8.68510869622696e-06
1132 1.02842341220821e-05
1133 1.16336432256503e-05
1134 1.07783598650713e-05
1135 9.01694602362113e-06
1136 8.96172605280299e-06
1137 9.04484477359802e-06
1138 8.08819186204346e-06
1139 9.07340745470719e-06
1140 1.00614079201478e-05
1141 1.95575685211224e-05
1142 5.31180594407488e-05
1143 1.99987644009525e-05
1144 1.37034230647259e-05
1145 1.52694283315213e-05
1146 1.22589208331192e-05
1147 1.30582184283412e-05
1148 8.54591780807823e-06
1149 1.25868091345183e-05
1150 9.13473468244774e-06
1151 9.867621884041e-06
1152 7.51140078136814e-06
1153 1.0323807146051e-05
1154 8.0038798841997e-06
1155 8.24732887849677e-06
1156 7.75042917666724e-06
1157 8.8439001046936e-06
1158 7.73361716710497e-06
1159 7.49491709939321e-06
1160 8.15421208244516e-06
1161 7.79033143771812e-06
1162 7.66809444030514e-06
1163 7.39287588658044e-06
1164 7.97664233687101e-06
1165 7.44380486139562e-06
1166 7.36338051865459e-06
1167 7.69346024753759e-06
1168 7.38647167963791e-06
1169 7.39852976039401e-06
1170 7.41313988328329e-06
1171 7.3828246058838e-06
1172 7.32405032977113e-06
1173 7.29906241758727e-06
1174 7.31636873751995e-06
1175 7.25493555364665e-06
1176 7.2565812843095e-06
1177 7.24641495253309e-06
1178 7.21988817531383e-06
1179 7.22649838280631e-06
1180 7.208381248347e-06
1181 7.2028924478218e-06
1182 7.1968925112742e-06
1183 7.18987848813413e-06
1184 7.1841686803964e-06
1185 7.17794273441541e-06
1186 7.1731669777364e-06
1187 7.16707609171863e-06
1188 7.16324939276092e-06
1189 7.15702435627463e-06
1190 7.15493706593406e-06
1191 7.14766383680399e-06
1192 7.15084661351284e-06
1193 7.14310908733751e-06
1194 7.1773752097215e-06
1195 7.22331515135011e-06
1196 7.63161642680643e-06
1197 8.87544319994049e-06
1198 1.64495231729234e-05
1199 3.23238673445303e-05
1200 6.06446774327196e-05
1201 1.35344826048822e-05
1202 2.47132302320097e-05
1203 1.08877375168959e-05
1204 1.41596610774286e-05
1205 1.06506258816808e-05
1206 8.57020222611027e-06
1207 7.59334398026112e-06
1208 9.0517623902997e-06
1209 8.70214535098057e-06
1210 7.79108358983649e-06
1211 7.72800376580562e-06
1212 8.35593709780369e-06
1213 8.23352911538677e-06
1214 7.87184217188042e-06
1215 7.94008064985974e-06
1216 8.21283629193204e-06
1217 8.05444142315537e-06
1218 7.84356234362349e-06
1219 7.88662873674184e-06
1220 7.94584138930077e-06
1221 7.78750563767971e-06
1222 7.69215512264054e-06
1223 7.74487943999702e-06
1224 7.72094881540397e-06
1225 7.61168939789059e-06
1226 7.58958731239545e-06
1227 7.60686953071854e-06
1228 7.52799769543344e-06
1229 7.46508658266976e-06
1230 7.46453770261724e-06
1231 7.42661040931125e-06
1232 7.36894162400858e-06
1233 7.35865069145802e-06
1234 7.3502510531398e-06
1235 7.31616592020146e-06
1236 7.31162572265021e-06
1237 7.31519276087056e-06
1238 7.29643898012e-06
1239 7.29124894860433e-06
1240 7.28519444237463e-06
1241 7.26267808204284e-06
1242 7.24421261111274e-06
1243 7.21985225027311e-06
1244 7.19054605724523e-06
1245 7.16367412678665e-06
1246 7.13677081876085e-06
1247 7.11290522303898e-06
1248 7.09007508703507e-06
1249 7.06667606209521e-06
1250 7.04346120983246e-06
1251 7.03105160937412e-06
1252 7.05385764376842e-06
1253 7.15585792931961e-06
1254 7.40438918001018e-06
1255 7.93228537077084e-06
1256 8.83264237927506e-06
1257 9.97404549707426e-06
1258 1.0105258297699e-05
1259 9.35707248572726e-06
1260 8.1547477748245e-06
1261 8.89211514731869e-06
1262 1.01316936707008e-05
1263 1.30724720293074e-05
1264 2.78570951195434e-05
1265 4.83258954773191e-05
1266 1.0012060556619e-05
1267 2.158149072784e-05
1268 1.02361955214292e-05
1269 1.71711944858544e-05
1270 9.46413092606235e-06
1271 1.02953699752106e-05
1272 8.86652651388431e-06
1273 1.04602204373805e-05
1274 8.17211730463896e-06
1275 7.6626347436104e-06
1276 8.43575344333658e-06
1277 8.37490279081976e-06
1278 7.71083705330966e-06
1279 7.42841803003103e-06
1280 8.18688477011165e-06
1281 7.69635062169982e-06
1282 7.48959155316697e-06
1283 7.57829548092559e-06
1284 7.78049798100255e-06
1285 7.46791511119227e-06
1286 7.37584059606888e-06
1287 7.63692423788598e-06
1288 7.42458996683126e-06
1289 7.35571802579216e-06
1290 7.4420763667149e-06
1291 7.39473307476146e-06
1292 7.32374564904603e-06
1293 7.32574881112669e-06
1294 7.34348213882186e-06
1295 7.27546785128652e-06
1296 7.26893813407514e-06
1297 7.27886526874499e-06
1298 7.23319635653752e-06
1299 7.23366656529834e-06
1300 7.22263121133437e-06
1301 7.20139269105857e-06
1302 7.20255229680333e-06
1303 7.18530827725772e-06
1304 7.17824150342494e-06
1305 7.17028569852118e-06
1306 7.16176919013378e-06
1307 7.15425449016038e-06
1308 7.14651241651154e-06
1309 7.13977942723432e-06
1310 7.13212193659274e-06
1311 7.12619112164248e-06
1312 7.11857819624129e-06
1313 7.11397069608211e-06
1314 7.10661015546066e-06
1315 7.10763970346306e-06
1316 7.1134058998723e-06
1317 7.18115234121797e-06
1318 7.46329578760196e-06
1319 8.94935601536417e-06
1320 1.47499376907945e-05
1321 4.51831547252368e-05
1322 3.39739926857874e-05
1323 8.07986180006992e-06
1324 1.81884970515966e-05
1325 8.6343443399528e-06
1326 1.32989762278157e-05
1327 8.46507009555353e-06
1328 1.23696572700283e-05
1329 8.55702182889218e-06
1330 8.79433537193108e-06
1331 9.1005458671134e-06
1332 8.96983419806929e-06
1333 8.39186304801842e-06
1334 7.36555602998124e-06
1335 9.02246756595559e-06
1336 7.63824846217176e-06
1337 7.84376061346848e-06
1338 7.70050428400282e-06
1339 8.12653615867021e-06
1340 7.64096057537245e-06
1341 7.3058536145254e-06
1342 8.06832213129383e-06
1343 7.36910851628636e-06
1344 7.4140943979728e-06
1345 7.60388184062322e-06
1346 7.34297600502032e-06
1347 7.38972994440701e-06
1348 7.31899399397662e-06
1349 7.33771457817056e-06
1350 7.2950278990902e-06
1351 7.26649159332737e-06
1352 7.30870351617341e-06
1353 7.26064854461583e-06
1354 7.28271743355435e-06
1355 7.25587005945272e-06
1356 7.21905644240906e-06
1357 7.21127071301453e-06
1358 7.15380747351446e-06
1359 7.12505197952851e-06
1360 7.1090153141995e-06
1361 7.12424798621214e-06
1362 7.14942962076748e-06
1363 7.11262600816553e-06
1364 7.00318105373299e-06
1365 6.91563309374033e-06
1366 7.08651305103558e-06
1367 7.67047640692908e-06
1368 8.58243947732262e-06
1369 9.79034120973665e-06
1370 9.38073026190978e-06
1371 1.04468535937485e-05
1372 1.04739692687872e-05
1373 2.11808237509103e-05
1374 2.80362655757926e-05
1375 2.20889269257896e-05
1376 8.2548331192811e-06
1377 1.40326619657571e-05
1378 9.88594001682941e-06
1379 1.16405763037619e-05
1380 1.03265965663013e-05
1381 7.69876078265952e-06
1382 1.12364241431351e-05
1383 7.72345538280206e-06
1384 9.15314831217984e-06
1385 8.57931627251673e-06
1386 8.52982338983566e-06
1387 8.69783161761006e-06
1388 7.31874479242833e-06
1389 8.82879885466537e-06
1390 7.64105334383203e-06
1391 7.43331156627391e-06
1392 8.29764576337766e-06
1393 7.30049350750051e-06
1394 7.54636130295694e-06
1395 7.74735235609114e-06
1396 7.2382640610158e-06
1397 7.39092229196103e-06
1398 7.44183944334509e-06
1399 7.1815011324361e-06
1400 7.21409833204234e-06
1401 7.26523376215482e-06
1402 7.15565238351701e-06
1403 7.15694932296174e-06
1404 7.15111173121841e-06
1405 7.13521149009466e-06
1406 7.13258214091184e-06
1407 7.1212243710761e-06
1408 7.11899701855145e-06
1409 7.10750873622601e-06
1410 7.11043594492367e-06
1411 7.10120502844802e-06
1412 7.12310065864585e-06
1413 7.15028045306099e-06
1414 7.33793422114104e-06
1415 7.88033685239498e-06
1416 1.04029750218615e-05
1417 1.71571791725e-05
1418 4.10992834076751e-05
1419 1.87223249668023e-05
1420 9.06959030544385e-06
1421 1.60758972924668e-05
1422 8.35369246487971e-06
1423 1.28476376630715e-05
1424 8.44485111883841e-06
1425 1.17687959573232e-05
1426 8.96950677997665e-06
1427 8.15687144495314e-06
1428 1.04371920315316e-05
1429 8.07458036433673e-06
1430 9.12941868591588e-06
1431 7.72201383369975e-06
1432 9.15166583581595e-06
1433 7.98414566816064e-06
1434 7.38528797228355e-06
1435 8.83448956301436e-06
1436 7.42484007787425e-06
1437 7.81127073423704e-06
1438 7.98739438323537e-06
1439 7.57883117330493e-06
1440 7.76793149270816e-06
1441 7.46591967981658e-06
1442 7.56610916141653e-06
1443 7.47218109609094e-06
1444 7.25094787412672e-06
1445 7.33122942619957e-06
1446 7.23781704436988e-06
1447 7.17165085006854e-06
1448 7.14695579517866e-06
1449 7.13523013473605e-06
1450 7.18705177860102e-06
1451 7.18714591130265e-06
1452 7.2584748522786e-06
1453 7.33515889805858e-06
1454 7.4280701483076e-06
1455 7.51660763853579e-06
1456 7.59461499910685e-06
1457 7.62365152695565e-06
1458 7.59289378038375e-06
1459 7.48029196984135e-06
1460 7.34818468117737e-06
1461 7.29609337213333e-06
1462 7.60015973355621e-06
1463 8.31549186841585e-06
1464 8.15422936284449e-06
1465 7.69166763348039e-06
1466 8.13333826954477e-06
1467 1.31449787659221e-05
1468 3.09558308799751e-05
1469 3.08571725327056e-05
1470 1.36040061988751e-05
1471 1.16866194730392e-05
1472 1.29095669763046e-05
1473 7.50295885154628e-06
1474 1.26209015434142e-05
1475 7.97785560280317e-06
1476 9.67412961472291e-06
1477 8.99370206752792e-06
1478 8.96571964403847e-06
1479 9.06496825336944e-06
1480 7.3069786594715e-06
1481 9.50786670728121e-06
1482 7.62635318096727e-06
1483 7.83502036938444e-06
1484 8.52424000186147e-06
1485 7.50959043216426e-06
1486 8.05034324002918e-06
1487 7.63140178605681e-06
1488 7.65985441830708e-06
1489 7.72636212786892e-06
1490 7.30716328689596e-06
1491 7.54573784433887e-06
1492 7.41543590265792e-06
1493 7.23299808669253e-06
1494 7.31359568817425e-06
1495 7.24026540410705e-06
1496 7.20852813174133e-06
1497 7.16786735210917e-06
1498 7.15362693881616e-06
1499 7.16427166480571e-06
1500 7.13604913471499e-06
1501 7.13848839950515e-06
1502 7.12394830770791e-06
1503 7.12666633262415e-06
1504 7.11200664227363e-06
1505 7.12144446879392e-06
1506 7.11415987098007e-06
1507 7.16325939720264e-06
1508 7.24840720067732e-06
1509 7.68160407460527e-06
1510 8.93966807780089e-06
1511 1.45605736179277e-05
1512 2.42247897404013e-05
1513 3.85163075407036e-05
1514 9.07399225980043e-06
1515 1.77576766873244e-05
1516 1.16449027700583e-05
1517 1.4891498722136e-05
1518 1.05036997410934e-05
1519 8.9654758994584e-06
1520 1.1470404388092e-05
1521 9.34880972636165e-06
1522 9.20859110919992e-06
1523 7.50776052882429e-06
1524 9.96116705209715e-06
1525 7.84316671342822e-06
1526 7.86935106589226e-06
1527 8.0471245382796e-06
1528 8.27963594929315e-06
1529 7.69061171013163e-06
1530 7.33080332793179e-06
1531 8.23634854896227e-06
1532 7.48351885704324e-06
1533 7.45260831536143e-06
1534 7.7694412539131e-06
1535 7.54047687223647e-06
1536 7.4862450674118e-06
1537 7.40851601221948e-06
1538 7.51936568121891e-06
1539 7.35777666704962e-06
1540 7.23918083167518e-06
1541 7.38183553039562e-06
1542 7.23597077012528e-06
1543 7.18102273822296e-06
1544 7.26138387108222e-06
1545 7.20761318007135e-06
1546 7.23530774848768e-06
1547 7.28270742911263e-06
1548 7.32238549971953e-06
1549 7.41726125852438e-06
1550 7.47104422771372e-06
1551 7.55065593693871e-06
1552 7.59072281653062e-06
1553 7.59569684305461e-06
1554 7.52428013583994e-06
1555 7.41319036023924e-06
1556 7.29669818611001e-06
1557 7.37393474992132e-06
1558 7.87545923230937e-06
1559 8.34009460959351e-06
1560 7.79642141424119e-06
1561 7.08180505171185e-06
1562 8.50690139486687e-06
1563 1.14157583084307e-05
1564 1.65990113600856e-05
1565 3.61139718734194e-05
1566 1.6821260942379e-05
1567 8.39192580315284e-06
1568 1.42905573738972e-05
1569 7.97332631918835e-06
1570 1.02349486041931e-05
1571 1.05790213638102e-05
1572 8.56281803862657e-06
1573 1.02730291473563e-05
1574 7.76469823904335e-06
1575 9.23850075196242e-06
1576 8.55160760693252e-06
1577 7.3178803177143e-06
1578 9.0300109150121e-06
1579 7.59116528570303e-06
1580 7.5623956945492e-06
1581 8.37112111184979e-06
1582 7.30842612028937e-06
1583 7.55525798012968e-06
1584 7.85989323048852e-06
1585 7.23020775694749e-06
1586 7.35247249394888e-06
1587 7.52751066102064e-06
1588 7.21358492228319e-06
1589 7.19333411325351e-06
1590 7.2482166615373e-06
1591 7.18388764653355e-06
1592 7.17823331797263e-06
1593 7.14942189006251e-06
1594 7.15018768460141e-06
1595 7.1384397415386e-06
1596 7.14406633051112e-06
1597 7.13711006028461e-06
1598 7.1587282945984e-06
1599 7.19280160410563e-06
1600 7.35384492145386e-06
1601 7.82646020525135e-06
1602 9.72176439972827e-06
1603 1.48005810842733e-05
1604 3.19431455864105e-05
1605 2.0851823137491e-05
1606 7.59825843488215e-06
1607 1.26360528156511e-05
1608 8.74085617397213e-06
1609 8.28746397019131e-06
1610 1.04495147752459e-05
1611 7.46429259379511e-06
1612 9.10718790692044e-06
1613 8.58354360389058e-06
1614 7.68892186897574e-06
1615 8.6568115875707e-06
1616 7.63640036893776e-06
1617 7.66033190302551e-06
1618 7.93022900325013e-06
1619 7.33133083485882e-06
1620 7.37017535357154e-06
1621 7.45247416489292e-06
1622 7.26372445569723e-06
1623 7.17242573955446e-06
1624 7.20623393135611e-06
1625 7.21457718100282e-06
1626 7.16291060598451e-06
1627 7.161109806475e-06
1628 7.15003761797561e-06
1629 7.14205725671491e-06
1630 7.12417067916249e-06
1631 7.10944823367754e-06
1632 7.09061214365647e-06
1633 7.07545359546202e-06
1634 7.09010646460229e-06
1635 7.12833616489661e-06
1636 7.4636732279032e-06
1637 8.41951441543642e-06
1638 1.24654052342521e-05
1639 2.71636472461978e-05
1640 2.88768314931076e-05
1641 1.20777867778088e-05
1642 1.10987275547814e-05
1643 1.2228043487994e-05
1644 8.03998318588128e-06
1645 1.16944665933261e-05
1646 8.0297304521082e-06
1647 9.1448182502063e-06
1648 9.17473698791582e-06
1649 7.9268274930655e-06
1650 9.16719454835402e-06
1651 7.3145970418409e-06
1652 8.35287664813222e-06
1653 8.37499919725815e-06
1654 7.7578461059602e-06
1655 8.61246007843874e-06
1656 7.66747871239204e-06
1657 7.5633224696503e-06
1658 7.83519135438837e-06
1659 7.29636849428061e-06
1660 7.31857699065586e-06
1661 7.27065344108269e-06
1662 7.11606799086439e-06
1663 7.28257282389677e-06
1664 7.47677722756634e-06
1665 7.48949696571799e-06
1666 7.43103100830922e-06
1667 7.32643729861593e-06
1668 7.24145729691372e-06
1669 7.19969830242917e-06
1670 7.171602192102e-06
1671 7.10236872691894e-06
1672 7.02235684002517e-06
1673 6.98811118127196e-06
1674 7.09519372321665e-06
1675 7.26299276720965e-06
1676 7.49058244764456e-06
1677 7.70027509133797e-06
1678 8.00843554316089e-06
1679 9.67692540143616e-06
1680 1.42430690175388e-05
1681 3.50061563949566e-05
1682 2.47302668867633e-05
1683 7.73692045186181e-06
1684 1.34803622131585e-05
1685 8.7444786913693e-06
1686 9.42374117585132e-06
1687 1.06631005110103e-05
1688 8.08574077382218e-06
1689 1.0372504220868e-05
1690 7.77858440415002e-06
1691 8.8822198449634e-06
1692 8.63093555381056e-06
1693 7.33248816686682e-06
1694 8.71749125508359e-06
1695 7.51881725591375e-06
1696 7.86310374678578e-06
1697 8.17241834738525e-06
1698 7.29166413293569e-06
1699 7.87486624176381e-06
1700 7.63363186706556e-06
1701 7.16622889740393e-06
1702 7.50986191633274e-06
1703 7.32696207705885e-06
1704 7.1107615440269e-06
1705 7.21589003660483e-06
1706 7.22616141501931e-06
1707 7.22961567589664e-06
1708 7.23105358702014e-06
1709 7.24543542673928e-06
1710 7.23803987057181e-06
1711 7.2096136136679e-06
1712 7.17276907380437e-06
1713 7.13227836968144e-06
1714 7.10675840309705e-06
1715 7.08606876287377e-06
1716 7.08085053702234e-06
1717 7.02717898093397e-06
1718 7.02866600477137e-06
1719 6.97619452694198e-06
1720 7.46142404750572e-06
1721 8.56125552672893e-06
1722 1.38349378175917e-05
1723 3.4341213904554e-05
1724 3.01077452604659e-05
1725 9.14437623578124e-06
1726 1.35736909214756e-05
1727 1.01538107628585e-05
1728 9.02519605006091e-06
1729 1.06705065263668e-05
1730 7.83372342993971e-06
1731 1.06556208265829e-05
1732 8.01734040578594e-06
1733 8.85689041751903e-06
1734 8.88563863554737e-06
1735 7.55147902964382e-06
1736 8.60062846186338e-06
1737 7.45049419492716e-06
1738 8.11954851087648e-06
1739 7.95966207078891e-06
1740 7.27778660802869e-06
1741 8.29636246635346e-06
1742 7.5034818109998e-06
1743 7.24323444956099e-06
1744 7.90408284956357e-06
1745 7.25232803233666e-06
1746 7.13195049684145e-06
1747 7.4974773269787e-06
1748 7.19320769348997e-06
1749 7.14624366082717e-06
1750 7.26333382772282e-06
1751 7.19503759682993e-06
1752 7.21565174899297e-06
1753 7.16364820618764e-06
1754 7.15557325747795e-06
1755 7.12275777914329e-06
1756 7.11107441020431e-06
1757 7.07721756043611e-06
1758 7.079598162818e-06
1759 7.04295689502032e-06
1760 7.10031235939823e-06
1761 7.08093102730345e-06
1762 7.41068697607261e-06
1763 7.86794316809392e-06
1764 1.06489951576805e-05
1765 2.01598468265729e-05
1766 3.09374008793384e-05
1767 2.38695920415921e-05
1768 9.65433810051763e-06
1769 1.71562733157771e-05
1770 8.34103684610454e-06
1771 1.29232075778418e-05
1772 8.97176232683705e-06
1773 9.85647966444958e-06
1774 9.37272034207126e-06
1775 8.25089955469593e-06
1776 1.02371968750958e-05
1777 7.72971816331847e-06
1778 8.34597176435636e-06
1779 7.94158950156998e-06
1780 8.34618185763247e-06
1781 7.57381894800346e-06
1782 7.10753647581441e-06
1783 8.52039011078887e-06
1784 7.28153418094735e-06
1785 7.31161753719789e-06
1786 8.42426015879028e-06
1787 7.55205201130593e-06
1788 7.38551761969575e-06
1789 7.85084284871118e-06
1790 7.41306530471775e-06
1791 7.18561113899341e-06
1792 7.47857211536029e-06
1793 7.20305433787871e-06
1794 6.994260729698e-06
1795 7.28867917132447e-06
1796 7.2288512455998e-06
1797 7.3977030297101e-06
1798 7.75298667576862e-06
1799 7.71917893871432e-06
1800 7.67870460549602e-06
1801 7.48578577258741e-06
1802 7.33577280698228e-06
1803 7.31828140487778e-06
1804 7.37560094421497e-06
1805 7.32161242922302e-06
1806 7.07161461832584e-06
1807 6.98156918588211e-06
1808 7.18071896699257e-06
1809 7.57997486289241e-06
1810 7.77741934143705e-06
1811 7.81646758696297e-06
1812 7.59996464694268e-06
1813 7.85604061093181e-06
1814 9.05941396922572e-06
1815 1.53394303197274e-05
1816 3.65235791832674e-05
1817 2.65434555331012e-05
1818 8.0836452980293e-06
1819 1.45561216413626e-05
1820 8.66670779942069e-06
1821 1.26414515762008e-05
1822 9.22852086659987e-06
1823 8.02411159384064e-06
1824 1.16074461402604e-05
1825 7.71781651565107e-06
1826 8.22222773422254e-06
1827 8.65432048158254e-06
1828 9.00811846804572e-06
1829 7.67690107750241e-06
1830 7.3588926170487e-06
1831 9.56137591856532e-06
1832 7.33951765141683e-06
1833 7.07619028617046e-06
1834 9.2221798695391e-06
1835 7.34097420718172e-06
1836 7.09177811586414e-06
1837 8.68384995555971e-06
1838 7.35921184968902e-06
1839 7.11317034074455e-06
1840 8.20041259430582e-06
1841 7.26138705431367e-06
1842 7.14557609171607e-06
1843 7.78723824623739e-06
1844 7.1674958235235e-06
1845 7.26913958715159e-06
1846 7.40074347049813e-06
1847 7.16387694410514e-06
1848 7.33928072804702e-06
1849 7.18523415343952e-06
1850 7.25563540981966e-06
1851 7.16281965651433e-06
1852 7.22936374586425e-06
1853 7.11252459950629e-06
1854 7.2422130870109e-06
1855 7.0549458541791e-06
1856 7.44184444556595e-06
1857 7.21281230653403e-06
1858 9.12179439183092e-06
1859 1.15748880489264e-05
1860 2.30739278777037e-05
1861 3.29668728227261e-05
1862 1.04705522971926e-05
1863 1.62397955136839e-05
1864 1.27405673993053e-05
1865 8.78985338204075e-06
1866 1.56238056661095e-05
1867 8.28041538625257e-06
1868 8.71882275532698e-06
1869 9.67005144048017e-06
1870 1.06429006336839e-05
1871 7.48002048567287e-06
1872 7.56698955228785e-06
1873 1.18130255941651e-05
1874 8.01950955064967e-06
1875 7.50867184251547e-06
1876 1.15566499516717e-05
1877 9.34192394197453e-06
1878 7.88052511779824e-06
1879 1.08608464870485e-05
1880 1.02244248409988e-05
1881 8.34513321024133e-06
1882 1.04219661807292e-05
1883 1.06124471130897e-05
1884 8.74557645147434e-06
1885 1.03517886600457e-05
1886 1.13017094918177e-05
1887 1.01186078609317e-05
1888 1.17116014735075e-05
1889 1.27390594570898e-05
1890 1.14158583528479e-05
1891 1.23377658383106e-05
1892 1.32611303342856e-05
1893 1.24109401440364e-05
1894 1.31671540657408e-05
1895 1.40021857077954e-05
1896 1.40994034154573e-05
1897 1.62395517691039e-05
1898 1.90642986126477e-05
1899 2.14390456676483e-05
1900 2.64620175585151e-05
1901 3.5206427128287e-05
1902 5.02493021485861e-05
1903 8.327863906743e-05
1904 0.000150657680933364
1905 0.000176734843989834
1906 3.75942945538554e-05
1907 1.81668510776944e-05
1908 2.3810745915398e-05
1909 1.05305925899302e-05
1910 7.05350748830824e-06
1911 7.08210473021609e-06
1912 7.08325023879297e-06
1913 7.08522202330641e-06
1914 7.07333219907014e-06
1915 7.06500759406481e-06
1916 7.06229729985353e-06
1917 7.06134414940607e-06
1918 7.06049468135461e-06
1919 7.05972888681572e-06
1920 7.05928050592775e-06
1921 7.05906450093607e-06
1922 7.05896172803477e-06
1923 7.05893899066723e-06
1924 7.05894672137219e-06
1925 7.05901493347483e-06
1926 7.05913453202811e-06
1927 7.05933234712575e-06
1928 7.05963748259819e-06
1929 7.06009086570702e-06
1930 7.06065929989563e-06
1931 7.06140690454049e-06
1932 7.06237688063993e-06
1933 7.06357195667806e-06
1934 7.06501896274858e-06
1935 7.06673563399818e-06
1936 7.06881564838113e-06
1937 7.07129538568552e-06
1938 7.07419576428947e-06
1939 7.0775954554847e-06
1940 7.08158904672018e-06
1941 7.08622337697307e-06
1942 7.09162031853339e-06
1943 7.09786763763987e-06
1944 7.10511130819214e-06
1945 7.1134818426799e-06
1946 7.12318114892696e-06
1947 7.13436156729585e-06
1948 7.14727320882957e-06
1949 7.16224667485221e-06
1950 7.17953480489086e-06
1951 7.19957870387589e-06
1952 7.22281447451678e-06
1953 7.24982464817003e-06
1954 7.28124041415867e-06
1955 7.3178275670216e-06
1956 7.36058109396254e-06
1957 7.41052326702629e-06
1958 7.46901923776022e-06
1959 7.53755102778086e-06
1960 7.61789215175668e-06
1961 7.71205668570474e-06
1962 7.82230927143246e-06
1963 7.95111009210814e-06
1964 8.10100755188614e-06
1965 8.27459825814003e-06
1966 8.47389674163423e-06
1967 8.70035455591278e-06
1968 8.9538843894843e-06
1969 9.2322316049831e-06
1970 9.53015387494816e-06
1971 9.8380742201698e-06
1972 1.01415307653951e-05
1973 1.04207365438924e-05
1974 1.0651418051566e-05
1975 1.08068252302473e-05
1976 1.08609210656141e-05
1977 1.07931318780174e-05
1978 1.05933413578896e-05
1979 1.02650446933694e-05
1980 9.82674464466982e-06
1981 9.30993883230258e-06
1982 8.75513705977937e-06
1983 8.20859804662177e-06
1984 7.72007933846908e-06
1985 7.34532750357175e-06
1986 7.15304577170173e-06
1987 7.23735320207197e-06
1988 7.73524880059995e-06
1989 8.84632390807383e-06
1990 1.08456060843309e-05
1991 1.4058660781302e-05
1992 1.87049063242739e-05
1993 2.44065613514977e-05
1994 2.92097283818293e-05
1995 2.92284930765163e-05
1996 2.23516963160364e-05
1997 1.35413183670607e-05
1998 9.41083817451727e-06
1999 1.00560009741457e-05
};
\addlegendentry{Test}

\nextgroupplot[
title={ELU/SiLU},
ymin=2.43456877973265e-06, ymax=0.001,
]
\addplot [semithick, black, dashed]
table {%
0 0.00788341589213815
1 0.00768992121447809
2 0.00751548440894112
3 0.00735840213019401
4 0.00721588448504917
5 0.00708472834958229
6 0.00696034026623238
7 0.00683559190656524
8 0.00670518715196522
9 0.00657447641424369
10 0.00645316220470704
11 0.00634745175921125
12 0.00625829976343084
13 0.0061837035609642
14 0.00612082514271606
15 0.00606715086178156
16 0.00602069285378093
17 0.00597982777253492
18 0.00594305474805878
19 0.00590883776749251
20 0.00587533866928425
21 0.00583996126442798
22 0.00579877732161549
23 0.00574491051520454
24 0.00566515909304144
25 0.00553316616787924
26 0.00529406724672299
27 0.0048413401855214
28 0.00403966893281904
29 0.0029387892500381
30 0.00194979852494725
31 0.00134945576064638
32 0.0010521555395826
33 0.000884628435414925
34 0.000776877057433012
35 0.000702427354553947
36 0.000647008266241755
37 0.000603560290528549
38 0.00056852583111322
39 0.000539686563570285
40 0.000515278308739653
41 0.00049414653221902
42 0.000475594546060165
43 0.000459145958302543
44 0.000444437891928828
45 0.000431167043416281
46 0.00041911384892046
47 0.00040809543065734
48 0.000397961119006141
49 0.00038859928781676
50 0.000379904073270154
51 0.000371799546655893
52 0.000364236099585469
53 0.000357151785010501
54 0.000350494556187186
55 0.000344227926916574
56 0.000338301367719396
57 0.000332680005271868
58 0.000327329782976449
59 0.000322222580052767
60 0.000317331646783714
61 0.000312631448764478
62 0.000308100664369704
63 0.000303718265740827
64 0.000299465005809907
65 0.000295322412171117
66 0.000291275661084001
67 0.000287312375007787
68 0.000283417607420233
69 0.000279577484661786
70 0.000275780407605453
71 0.000272013665608029
72 0.000268265966496983
73 0.000264526055161696
74 0.00026078290420628
75 0.000257026884469269
76 0.000253247545117574
77 0.000249438353591813
78 0.000245589958524306
79 0.00024169221808279
80 0.000237734463610195
81 0.000233712333908898
82 0.00022961461269233
83 0.000225434838966976
84 0.00022116623108559
85 0.000216806243599876
86 0.000212361286457963
87 0.000207843712985323
88 0.000203260548346407
89 0.000198619480613615
90 0.000193931646492729
91 0.000189212796101401
92 0.000184478935750576
93 0.000179749954156705
94 0.000175050206053129
95 0.000170403064316815
96 0.000165834479787463
97 0.000161370970829466
98 0.00015703440527659
99 0.000152844831063703
100 0.000148822245193969
101 0.000144982470743571
102 0.000141335234786766
103 0.000137885459224663
104 0.000134631988714773
105 0.000131569143263732
106 0.000128687768892632
107 0.000125977594819915
108 0.00012342721666414
109 0.000121023948878474
110 0.000118755169211227
111 0.000116608190666057
112 0.000114570998221097
113 0.000112632087621023
114 0.000110780939223787
115 0.000109008144221434
116 0.000107304947789544
117 0.000105663664783151
118 0.000104077372242273
119 0.000102539878525931
120 0.000101045864454363
121 9.95907804508533e-05
122 9.81705171909653e-05
123 9.6781675551938e-05
124 9.542134688445e-05
125 9.40870179420017e-05
126 9.2776667884209e-05
127 9.14889316447898e-05
128 9.02228141512751e-05
129 8.89772786081267e-05
130 8.77513820967124e-05
131 8.65445735769299e-05
132 8.5356348051846e-05
133 8.41864722929131e-05
134 8.30349361535809e-05
135 8.19018370350477e-05
136 8.07873352073329e-05
137 7.96916060892272e-05
138 7.86148378324469e-05
139 7.75573412568065e-05
140 7.65194844234429e-05
141 7.55015769016154e-05
142 7.45039112075574e-05
143 7.352686694162e-05
144 7.25707798494568e-05
145 7.16359964769708e-05
146 7.07228388137082e-05
147 6.98314324267812e-05
148 6.89620318610196e-05
149 6.81146771341901e-05
150 6.72895315290134e-05
151 6.64864416108912e-05
152 6.57054119130862e-05
153 6.49461455282108e-05
154 6.42085558979488e-05
155 6.34922074596034e-05
156 6.27967121999973e-05
157 6.21215283445054e-05
158 6.14660715996251e-05
159 6.08298125825968e-05
160 6.02121121460186e-05
161 5.96123851579478e-05
162 5.90298991482996e-05
163 5.84640527563351e-05
164 5.79141819514462e-05
165 5.73795464191562e-05
166 5.68594762029306e-05
167 5.63533284747564e-05
168 5.58603843785477e-05
169 5.53799507514441e-05
170 5.49113942156509e-05
171 5.44540642692937e-05
172 5.40074217951769e-05
173 5.35708246616196e-05
174 5.31436968458365e-05
175 5.27255242417368e-05
176 5.23158183938222e-05
177 5.19140908039617e-05
178 5.15198618700197e-05
179 5.11326869911954e-05
180 5.07521571364578e-05
181 5.03779390328418e-05
182 5.00096840880815e-05
183 4.96471108561991e-05
184 4.92899330879482e-05
185 4.89377730588103e-05
186 4.85905122786789e-05
187 4.82479749877029e-05
188 4.7909871398133e-05
189 4.75759723599367e-05
190 4.72460196050406e-05
191 4.69199556505373e-05
192 4.65974774712663e-05
193 4.62784609709388e-05
194 4.59626623694476e-05
195 4.56499906746899e-05
196 4.53402820852489e-05
197 4.50334138548669e-05
198 4.47292539291766e-05
199 4.44277853119956e-05
200 4.41288285273345e-05
201 4.38322922633461e-05
202 4.35381230445842e-05
203 4.32462221695573e-05
204 4.29564749495626e-05
205 4.26688752384052e-05
206 4.23832827181059e-05
207 4.20996636734117e-05
208 4.18179675847341e-05
209 4.15381104588164e-05
210 4.12600501107363e-05
211 4.09838058246237e-05
212 4.07092899408212e-05
213 4.04365214947688e-05
214 4.01653236110633e-05
215 3.98956892198044e-05
216 3.9627696580169e-05
217 3.93613086657751e-05
218 3.90967231780337e-05
219 3.88338052843551e-05
220 3.85725937448456e-05
221 3.83129883658739e-05
222 3.80549711422873e-05
223 3.77985130910474e-05
224 3.75437166226789e-05
225 3.72904979215605e-05
226 3.70388490509299e-05
227 3.67887232144426e-05
228 3.65400563353546e-05
229 3.62928649551009e-05
230 3.60470893880915e-05
231 3.58027630014135e-05
232 3.55598843597704e-05
233 3.53183759997933e-05
234 3.50782965483631e-05
235 3.48396360010383e-05
236 3.46022669930335e-05
237 3.43662769637376e-05
238 3.41317049716849e-05
239 3.38985309795703e-05
240 3.36667794442747e-05
241 3.34364551477506e-05
242 3.32075734235104e-05
243 3.29801244589589e-05
244 3.2754203289187e-05
245 3.25297260133084e-05
246 3.23067289045298e-05
247 3.20851492432439e-05
248 3.18649738844101e-05
249 3.16462148219898e-05
250 3.14288942568908e-05
251 3.12129616730772e-05
252 3.09985243553967e-05
253 3.07855241317156e-05
254 3.05740484272121e-05
255 3.03640658891879e-05
256 3.01556217543464e-05
257 2.99487415986732e-05
258 2.97433048750406e-05
259 2.95394461105047e-05
260 2.93371220401184e-05
261 2.91363450273252e-05
262 2.8937113498273e-05
263 2.87394474796088e-05
264 2.8543332696529e-05
265 2.83488196011206e-05
266 2.81559559027755e-05
267 2.79646803065248e-05
268 2.77750655293119e-05
269 2.75870739905315e-05
270 2.74007123763909e-05
271 2.72159628629254e-05
272 2.70329020004567e-05
273 2.6851465438682e-05
274 2.66716616579288e-05
275 2.64935395897226e-05
276 2.63170583458816e-05
277 2.61422225555918e-05
278 2.59690774093713e-05
279 2.57975664865739e-05
280 2.56277527128645e-05
281 2.54596438402643e-05
282 2.52931358382114e-05
283 2.51283294510074e-05
284 2.49652049078009e-05
285 2.48036959007436e-05
286 2.46439147240096e-05
287 2.44857879287963e-05
288 2.43293210964168e-05
289 2.41744840323577e-05
290 2.40212669098128e-05
291 2.38696849841347e-05
292 2.3719720925186e-05
293 2.35713300398288e-05
294 2.34245416024237e-05
295 2.32793409438159e-05
296 2.31357854900693e-05
297 2.29937862457064e-05
298 2.28534459161267e-05
299 2.27146868070349e-05
300 2.25775391236027e-05
301 2.24420054628638e-05
302 2.23080285657318e-05
303 2.21755600016138e-05
304 2.20446043250888e-05
305 2.19150730913498e-05
306 2.17869858865072e-05
307 2.16603559977102e-05
308 2.15351272920827e-05
309 2.14112580287917e-05
310 2.12887591253264e-05
311 2.11676173087483e-05
312 2.10478112627754e-05
313 2.09292852737519e-05
314 2.08121076425982e-05
315 2.06962331112948e-05
316 2.05816580916007e-05
317 2.04683651752191e-05
318 2.03563479068691e-05
319 2.0245599323232e-05
320 2.01360197600309e-05
321 2.00276703843372e-05
322 1.99204977349154e-05
323 1.98144800886979e-05
324 1.97096120171381e-05
325 1.96058602881521e-05
326 1.95032222602975e-05
327 1.94016848720224e-05
328 1.93012076881161e-05
329 1.92018074329781e-05
330 1.91034002465784e-05
331 1.90060324918306e-05
332 1.8909618393792e-05
333 1.88142244965661e-05
334 1.87197867891342e-05
335 1.86263394166275e-05
336 1.85338698575066e-05
337 1.84423161151415e-05
338 1.8351698837904e-05
339 1.826200962185e-05
340 1.81731836601529e-05
341 1.80852656033181e-05
342 1.79982520105426e-05
343 1.79120848127212e-05
344 1.78267501027563e-05
345 1.77422791409754e-05
346 1.76586195053119e-05
347 1.7575801791736e-05
348 1.74937944770193e-05
349 1.74125608012332e-05
350 1.7332098963152e-05
351 1.72524137145302e-05
352 1.71734664711209e-05
353 1.70952796114676e-05
354 1.70178308760427e-05
355 1.69410270860482e-05
356 1.68649759419992e-05
357 1.67896262421863e-05
358 1.67149891225904e-05
359 1.66410328308331e-05
360 1.65677674921483e-05
361 1.64951444574513e-05
362 1.642317707784e-05
363 1.63518076643498e-05
364 1.62810308204087e-05
365 1.62109351933282e-05
366 1.61413811241573e-05
367 1.60724400402046e-05
368 1.60040918437687e-05
369 1.59362658198603e-05
370 1.58690483740287e-05
371 1.58023645884242e-05
372 1.57362253947468e-05
373 1.56706651299743e-05
374 1.56057338980276e-05
375 1.55413655509307e-05
376 1.54776065297568e-05
377 1.54144649577859e-05
378 1.53519166126159e-05
379 1.5289916129646e-05
380 1.52285217325243e-05
381 1.51676724442495e-05
382 1.51073600811458e-05
383 1.50475759053137e-05
384 1.49882954261216e-05
385 1.49295037843444e-05
386 1.48711390828282e-05
387 1.48132879740359e-05
388 1.47558341208054e-05
389 1.46987925191411e-05
390 1.46422327791385e-05
391 1.45860718827606e-05
392 1.45303018737053e-05
393 1.44749547175138e-05
394 1.44200279645901e-05
395 1.4365526213922e-05
396 1.43113957982166e-05
397 1.42577197586036e-05
398 1.42044744695369e-05
399 1.4151659565087e-05
400 1.40992703041576e-05
401 1.40472888752186e-05
402 1.39957543723312e-05
403 1.39445957962891e-05
404 1.38939000660088e-05
405 1.38435810743687e-05
406 1.37936527515592e-05
407 1.37441217731293e-05
408 1.3695007503145e-05
409 1.36462652893243e-05
410 1.35979222370963e-05
411 1.3549954690717e-05
412 1.3502386966735e-05
413 1.3455154597608e-05
414 1.34083267848695e-05
415 1.33618288504778e-05
416 1.33156625512498e-05
417 1.32698847323809e-05
418 1.3224424462166e-05
419 1.31793062703167e-05
420 1.31345124749771e-05
421 1.30900440673543e-05
422 1.30459541907157e-05
423 1.3002232044812e-05
424 1.29587839019507e-05
425 1.29157010686498e-05
426 1.28729338637612e-05
427 1.28304685214076e-05
428 1.27882981786342e-05
429 1.2746413619702e-05
430 1.27048159486165e-05
431 1.26634995369912e-05
432 1.26224296508326e-05
433 1.25816196456796e-05
434 1.25410943017101e-05
435 1.25008093743162e-05
436 1.24607350375783e-05
437 1.24209424647859e-05
438 1.23814538399714e-05
439 1.23422214484137e-05
440 1.23033012631169e-05
441 1.22647456501923e-05
442 1.2226537450033e-05
443 1.21886603032806e-05
444 1.21511708233157e-05
445 1.21140095430405e-05
446 1.20771245155638e-05
447 1.20404755419301e-05
448 1.20039877273825e-05
449 1.19675810426045e-05
450 1.19312797961513e-05
451 1.18950476553081e-05
452 1.18589573183669e-05
453 1.18229947352333e-05
454 1.17872737517288e-05
455 1.17517422886237e-05
456 1.17162180011121e-05
457 1.16804146195193e-05
458 1.16439370110299e-05
459 1.16066548834226e-05
460 1.15693298008424e-05
461 1.15349055240799e-05
462 1.15101100650605e-05
463 1.15074228705936e-05
464 1.15449892934549e-05
465 1.16382625128253e-05
466 1.17723263413083e-05
467 1.18622939728752e-05
468 1.17788779663286e-05
469 1.15214124090102e-05
470 1.13727829251786e-05
471 1.18024929474103e-05
472 1.31632334774423e-05
473 1.47688221181497e-05
474 1.44311672514164e-05
475 1.2257963801332e-05
476 1.10023591552988e-05
477 1.12351567027602e-05
478 1.17204896730172e-05
479 1.14997859297361e-05
480 1.09036540294483e-05
481 1.06195905846818e-05
482 1.06181160495211e-05
483 1.06052180344207e-05
484 1.05151242681956e-05
485 1.04194964940518e-05
486 1.03578409662575e-05
487 1.0321418249859e-05
488 1.02908173342797e-05
489 1.02597102866753e-05
490 1.02324419293254e-05
491 1.02105784014483e-05
492 1.01906468543689e-05
493 1.01693020901905e-05
494 1.0145781916826e-05
495 1.01215614973782e-05
496 1.00985590325564e-05
497 1.00776533109581e-05
498 1.00585748121773e-05
499 1.00407070036113e-05
500 1.00237709634854e-05
501 1.00079356712968e-05
502 9.99330594630976e-06
503 9.97967788762821e-06
504 9.9667075890153e-06
505 9.95423155636388e-06
506 9.94235004725397e-06
507 9.93127134130134e-06
508 9.92119980480766e-06
509 9.91210591472225e-06
510 9.90410705270506e-06
511 9.8975528626255e-06
512 9.89293629594101e-06
513 9.89049834032585e-06
514 9.89023375730369e-06
515 9.89182640331876e-06
516 9.89489539371391e-06
517 9.89941077023104e-06
518 9.90582188364897e-06
519 9.91499370606164e-06
520 9.92831556523299e-06
521 9.94736382864403e-06
522 9.9734305205601e-06
523 1.00069233521793e-05
524 1.00472517132744e-05
525 1.00933519391333e-05
526 1.01453647989302e-05
527 1.02073110337386e-05
528 1.02907318968448e-05
529 1.04177413895457e-05
530 1.06204223015993e-05
531 1.09294115908654e-05
532 1.13439591871156e-05
533 1.17938429848863e-05
534 1.21694122849902e-05
535 1.25026530000127e-05
536 1.31406766961106e-05
537 1.45637913568919e-05
538 1.66650619330966e-05
539 1.77974060591168e-05
540 1.66225202820058e-05
541 1.53937646372171e-05
542 1.55504530656714e-05
543 1.50317815252876e-05
544 1.34327404843759e-05
545 1.292599186975e-05
546 1.42466920358686e-05
547 1.75847536869611e-05
548 2.34103851148859e-05
549 3.08497243572958e-05
550 3.52539740031688e-05
551 3.05702689722409e-05
552 1.98132042283916e-05
553 1.20238773932968e-05
554 9.16766077097009e-06
555 8.56699911810921e-06
556 8.46601720727591e-06
557 8.43442997666699e-06
558 8.43964440999656e-06
559 8.46959878408882e-06
560 8.49606523356528e-06
561 8.49896727839194e-06
562 8.4760772978143e-06
563 8.43659644367278e-06
564 8.39163356491923e-06
565 8.34884568590155e-06
566 8.31162292858068e-06
567 8.28055714219289e-06
568 8.25489141931257e-06
569 8.23352419576651e-06
570 8.2153567628751e-06
571 8.19949122732311e-06
572 8.18524037438095e-06
573 8.17210033510207e-06
574 8.15972091139372e-06
575 8.1478556595016e-06
576 8.13632293361621e-06
577 8.12502703340101e-06
578 8.11390292465575e-06
579 8.10291662123319e-06
580 8.09203457752261e-06
581 8.08130161278342e-06
582 8.07073532627101e-06
583 8.06038626777195e-06
584 8.05030986050781e-06
585 8.04055613734533e-06
586 8.03121942904994e-06
587 8.02236816710433e-06
588 8.01408507999923e-06
589 8.00645903353825e-06
590 7.99954881092191e-06
591 7.99342756518939e-06
592 7.98811961288948e-06
593 7.98367188981075e-06
594 7.98010103508773e-06
595 7.97736837210294e-06
596 7.97543921748911e-06
597 7.97421099729689e-06
598 7.97361610427316e-06
599 7.97352815684604e-06
600 7.9737975964278e-06
601 7.9743054932635e-06
602 7.97489266535223e-06
603 7.97543178254756e-06
604 7.97581604050634e-06
605 7.97594563461956e-06
606 7.97576553157597e-06
607 7.97518044670653e-06
608 7.9741541716416e-06
609 7.97267463514828e-06
610 7.97072761127282e-06
611 7.96833330429791e-06
612 7.96551634429932e-06
613 7.96221581644829e-06
614 7.95851833235162e-06
615 7.95440402878711e-06
616 7.94995624620043e-06
617 7.94517496416347e-06
618 7.94007083904091e-06
619 7.93471329707529e-06
620 7.92906262159221e-06
621 7.92321403952911e-06
622 7.91713862824395e-06
623 7.91083350648591e-06
624 7.90432686859077e-06
625 7.89761001307454e-06
626 7.89070689499738e-06
627 7.88361924364267e-06
628 7.87637229215932e-06
629 7.86897593041402e-06
630 7.8614655940612e-06
631 7.85380958667758e-06
632 7.84600077530229e-06
633 7.83790099845305e-06
634 7.82926009712526e-06
635 7.81969738028465e-06
636 7.80877930584012e-06
637 7.79630807823395e-06
638 7.78330205442046e-06
639 7.77419427322457e-06
640 7.78212495067265e-06
641 7.84180264989942e-06
642 8.04052667291444e-06
643 8.58210121279512e-06
644 9.81540160971406e-06
645 1.17131752244148e-05
646 1.24595427402063e-05
647 1.0380040493807e-05
648 8.48041153655998e-06
649 1.00054953335871e-05
650 1.30030605713927e-05
651 1.18919952676322e-05
652 8.88685582367543e-06
653 8.24469209348422e-06
654 8.21462094524605e-06
655 8.00740207118622e-06
656 7.83081692468102e-06
657 7.72865037346193e-06
658 7.6490108726901e-06
659 7.56561598125316e-06
660 7.49175650893719e-06
661 7.4359952111358e-06
662 7.39900007573624e-06
663 7.37659361726983e-06
664 7.36508696164151e-06
665 7.35610076452176e-06
666 7.34177799843749e-06
667 7.3227952426258e-06
668 7.3047645905433e-06
669 7.29115006015135e-06
670 7.28050255816015e-06
671 7.26922822380516e-06
672 7.25584108440813e-06
673 7.24145716457514e-06
674 7.22776359651789e-06
675 7.21539784276359e-06
676 7.2040967564746e-06
677 7.19346704780577e-06
678 7.18365106688879e-06
679 7.17522102355872e-06
680 7.16884529605721e-06
681 7.16512845944095e-06
682 7.1642802712546e-06
683 7.16648282672594e-06
684 7.17188265308977e-06
685 7.18066317872967e-06
686 7.19302916607489e-06
687 7.20909998719321e-06
688 7.2290129553565e-06
689 7.25239711130143e-06
690 7.27903814112807e-06
691 7.30791891268012e-06
692 7.33811451603117e-06
693 7.36784797439327e-06
694 7.39529743931655e-06
695 7.41804067949658e-06
696 7.43389660229354e-06
697 7.44084857906557e-06
698 7.43825909754747e-06
699 7.42749486359884e-06
700 7.41344897647878e-06
701 7.40605845628295e-06
702 7.42115973562107e-06
703 7.47785985133476e-06
704 7.5848648268817e-06
705 7.71262670440365e-06
706 7.78252105781974e-06
707 7.73797532271914e-06
708 7.64193889413178e-06
709 7.62182896396624e-06
710 7.73698454370475e-06
711 7.94464478826029e-06
712 8.11447440085544e-06
713 8.061253137015e-06
714 7.72280828620353e-06
715 7.35155769726603e-06
716 7.43503155131719e-06
717 8.45514264558034e-06
718 1.03756534297617e-05
719 1.15552184096757e-05
720 1.02158171113587e-05
721 8.34631423884957e-06
722 8.04446830215966e-06
723 8.34469425603501e-06
724 8.04336535553674e-06
725 7.32785158730209e-06
726 6.96154109025571e-06
727 7.03700950666075e-06
728 7.14072200658222e-06
729 7.02243263805968e-06
730 6.80242317407931e-06
731 6.65697431045942e-06
732 6.62328284262159e-06
733 6.62481179736218e-06
734 6.61670427559713e-06
735 6.58788114815678e-06
736 6.56619386685264e-06
737 6.55426599793785e-06
738 6.56600020221276e-06
739 6.57687068184032e-06
740 6.60693676213242e-06
741 6.6239471641083e-06
742 6.68548408144787e-06
743 6.72939608392653e-06
744 6.87492758233432e-06
745 6.9651509835289e-06
746 7.26071725454247e-06
747 7.37347956913936e-06
748 7.90017786655994e-06
749 7.91107732034391e-06
750 8.73765998399278e-06
751 8.39559735865691e-06
752 9.47825162711524e-06
753 8.64134987743626e-06
754 9.11447910389995e-06
755 8.15838463630314e-06
756 8.16166336470303e-06
757 7.60871329363866e-06
758 7.4380522336881e-06
759 7.08602962262717e-06
760 7.04668391993124e-06
761 6.9789212329141e-06
762 6.88874053622612e-06
763 6.67767665740726e-06
764 6.50708012006618e-06
765 6.47647712348487e-06
766 6.65859387938283e-06
767 6.91610858094549e-06
768 7.0469204502821e-06
769 6.86768778823676e-06
770 6.56421134292984e-06
771 6.41559607172937e-06
772 6.60702457988549e-06
773 6.95715165877075e-06
774 7.27704863523471e-06
775 7.25485110608659e-06
776 6.97695330842407e-06
777 6.519667453464e-06
778 6.39056936879712e-06
779 6.46320491481944e-06
780 7.03354021958091e-06
781 7.17118634119274e-06
782 7.38817156431537e-06
783 6.66632422730018e-06
784 7.46230645987112e-06
785 7.51966397150028e-06
786 1.02738459322183e-05
787 9.46517975464189e-06
788 9.21792838770585e-06
789 7.02811704567097e-06
790 6.73225527947352e-06
791 6.49968980326321e-06
792 6.51048655431197e-06
793 6.4877653880302e-06
794 6.43461159288705e-06
795 6.34912306729518e-06
796 6.29733953738665e-06
797 6.30956702929453e-06
798 6.34399227728011e-06
799 6.35896827905214e-06
800 6.32918761978019e-06
801 6.26956714366855e-06
802 6.20412251395308e-06
803 6.16217432058974e-06
804 6.14502688245366e-06
805 6.14392569220357e-06
806 6.13104386992802e-06
807 6.10320096505745e-06
808 6.06031583005873e-06
809 6.04056329267522e-06
810 6.0526796485938e-06
811 6.13481253797232e-06
812 6.25190824887767e-06
813 6.44365572011196e-06
814 6.61055016859535e-06
815 6.85784088894081e-06
816 6.97497563795224e-06
817 7.32160871308452e-06
818 7.45562205040073e-06
819 8.36396004100237e-06
820 8.87229364110453e-06
821 1.02952849756477e-05
822 9.9645749820354e-06
823 8.99897890960233e-06
824 7.44250462858531e-06
825 7.57978479359878e-06
826 7.92396229654102e-06
827 8.21608021617237e-06
828 7.52540624571196e-06
829 6.70134657188015e-06
830 6.13187569920015e-06
831 6.08402131918595e-06
832 6.19266332346768e-06
833 6.22387555537784e-06
834 6.10135217682384e-06
835 5.98985677946473e-06
836 5.94755152616422e-06
837 6.00001257478056e-06
838 6.03610445715219e-06
839 6.05659132313363e-06
840 5.99751554863914e-06
841 5.96222588544748e-06
842 5.89404992146214e-06
843 5.93238103197002e-06
844 5.90887718532329e-06
845 6.02273471406178e-06
846 5.92487042405665e-06
847 6.09584466859658e-06
848 5.8645548168812e-06
849 6.45879553573536e-06
850 6.35543082738721e-06
851 8.64622477880772e-06
852 8.70774517913731e-06
853 1.10905449819398e-05
854 8.34477688904656e-06
855 7.00971142908458e-06
856 6.50604745811023e-06
857 6.33820931139617e-06
858 6.13780939673347e-06
859 6.22277948547634e-06
860 6.31767544412298e-06
861 6.30210007468435e-06
862 6.2158637517129e-06
863 6.07803005725494e-06
864 5.98911891414033e-06
865 6.01194175819941e-06
866 6.15455757912997e-06
867 6.32861037264121e-06
868 6.42976804332562e-06
869 6.36784939622714e-06
870 6.15431316575155e-06
871 5.89411837426113e-06
872 5.74489185423133e-06
873 5.81698116208784e-06
874 6.13517974690225e-06
875 6.55178207553675e-06
876 6.74317442550532e-06
877 6.46130756454255e-06
878 5.98413118257923e-06
879 5.83498197803678e-06
880 6.21399061384587e-06
881 6.85081938289755e-06
882 7.30055919850869e-06
883 7.11954477683463e-06
884 6.45926387221607e-06
885 5.82866802112392e-06
886 5.84511133139287e-06
887 6.30867790629708e-06
888 6.89007036314138e-06
889 6.64866765687222e-06
890 6.30440672466648e-06
891 5.81531885623932e-06
892 6.65832451263171e-06
893 6.81977250716059e-06
894 9.04637921728124e-06
895 8.89082930299878e-06
896 1.02535889032751e-05
897 8.0603950074476e-06
898 6.72090127196157e-06
899 6.08632168841439e-06
900 5.87035591781415e-06
901 5.79884288720578e-06
902 5.84407730741532e-06
903 5.85947916276197e-06
904 5.84018068705561e-06
905 5.84670194347936e-06
906 5.85655598372981e-06
907 5.86721808470969e-06
908 5.84894628108401e-06
909 5.80174854558635e-06
910 5.73652776836298e-06
911 5.66850641714467e-06
912 5.60082496070535e-06
913 5.54489503334921e-06
914 5.50813323796362e-06
915 5.49082261125378e-06
916 5.48416300327759e-06
917 5.48264721067504e-06
918 5.48201163819151e-06
919 5.4843375882907e-06
920 5.48973660996488e-06
921 5.50142145594457e-06
922 5.51622864009005e-06
923 5.53772024902699e-06
924 5.5609320988026e-06
925 5.59631380880887e-06
926 5.63584589752253e-06
927 5.70645914610424e-06
928 5.79058027261325e-06
929 5.95747355713883e-06
930 6.16086902294199e-06
931 6.57136062720554e-06
932 7.03754006714519e-06
933 7.88759731218391e-06
934 8.52523495886714e-06
935 9.24730293672127e-06
936 8.65477643952062e-06
937 8.25279541771806e-06
938 7.66110176453338e-06
939 8.77821727129913e-06
940 8.76525239057457e-06
941 7.70693763474384e-06
942 6.31658109817579e-06
943 7.05699523795289e-06
944 7.57134073037236e-06
945 7.37049541577051e-06
946 6.50318959483087e-06
947 5.97257090229064e-06
948 5.71157369844144e-06
949 5.60148833539387e-06
950 5.50760010775697e-06
951 5.45637362714402e-06
952 5.42027457184346e-06
953 5.42877092168936e-06
954 5.45417741237841e-06
955 5.49621898393582e-06
956 5.52220836436135e-06
957 5.54212785264241e-06
958 5.53113499801583e-06
959 5.50958957701297e-06
960 5.46128280820568e-06
961 5.42462331765137e-06
962 5.38089494162008e-06
963 5.36476635426553e-06
964 5.33487444043246e-06
965 5.33649467682196e-06
966 5.30662272124616e-06
967 5.32859210355952e-06
968 5.28680374101143e-06
969 5.34939655860711e-06
970 5.26905426645463e-06
971 5.44583622552608e-06
972 5.29715634911554e-06
973 5.94011569710062e-06
974 5.93971008289174e-06
975 8.77638771434874e-06
976 8.66533094701083e-06
977 1.07698443914117e-05
978 7.45025510795472e-06
979 6.26256093916311e-06
980 5.8845454657952e-06
981 5.73269653347808e-06
982 5.40825753603258e-06
983 5.41476374449346e-06
984 5.39533711441109e-06
985 5.38468655686408e-06
986 5.43045845802226e-06
987 5.51210087618159e-06
988 5.57023005409718e-06
989 5.57689099656145e-06
990 5.53125787039477e-06
991 5.49894012635832e-06
992 5.53398692115792e-06
993 5.6543335489323e-06
994 5.84428917616009e-06
995 6.05268615361254e-06
996 6.14605711524163e-06
997 5.99507549958744e-06
998 5.63255691177211e-06
999 5.32271404418694e-06
1000 5.42766376065984e-06
1001 6.18139842667453e-06
1002 7.20227947459406e-06
1003 7.26040468679301e-06
1004 6.10068446071921e-06
1005 5.38979622444202e-06
1006 5.75082132558968e-06
1007 6.34023137413564e-06
1008 6.42111473059259e-06
1009 6.06644345602092e-06
1010 5.72351376071367e-06
1011 5.70644682884591e-06
1012 5.77877689611483e-06
1013 5.79769570840938e-06
1014 5.60131489679705e-06
1015 5.47333146050377e-06
1016 5.223729907744e-06
1017 5.19957273481531e-06
1018 5.11244431322666e-06
1019 5.24218411057653e-06
1020 5.14779863758363e-06
1021 5.50461528359136e-06
1022 5.40254322256573e-06
1023 6.91444654998818e-06
1024 7.58525882638139e-06
1025 1.10790459841859e-05
1026 1.01092521198609e-05
1027 6.89655245444953e-06
1028 5.93217657351985e-06
1029 5.74358348348625e-06
1030 5.66422696834934e-06
1031 5.46349911711985e-06
1032 5.29085981004762e-06
1033 5.21325935043393e-06
1034 5.23129851637272e-06
1035 5.27944147643566e-06
1036 5.30047553048973e-06
1037 5.30118584674e-06
1038 5.32480969628324e-06
1039 5.36653251304031e-06
1040 5.3907200887604e-06
1041 5.36876548662413e-06
1042 5.30301969958735e-06
1043 5.21545810006074e-06
1044 5.13008823777028e-06
1045 5.05944308337725e-06
1046 5.01491373006147e-06
1047 5.01065779734411e-06
1048 5.05212184798154e-06
1049 5.118301478646e-06
1050 5.16742061584807e-06
1051 5.15973870762743e-06
1052 5.09531402737018e-06
1053 5.02817745973161e-06
1054 5.04460127626061e-06
1055 5.21244045081914e-06
1056 5.55635434995594e-06
1057 6.0247104674005e-06
1058 6.45584686687073e-06
1059 6.56350778616144e-06
1060 6.18378434191769e-06
1061 5.51939041981697e-06
1062 5.10310165258154e-06
1063 5.32470526026785e-06
1064 6.24689246175869e-06
1065 6.99557695305053e-06
1066 6.55203431931817e-06
1067 5.41166327172604e-06
1068 5.2370303675886e-06
1069 5.69791403814968e-06
1070 6.40897577541821e-06
1071 6.09088668479529e-06
1072 6.98880949334679e-06
1073 7.09042988589204e-06
1074 1.04137305090291e-05
1075 9.86948767689455e-06
1076 6.8639043186991e-06
1077 5.80568613095522e-06
1078 5.64443138273418e-06
1079 5.51341256560534e-06
1080 5.31204069265101e-06
1081 5.11584068751958e-06
1082 5.01276819653285e-06
1083 5.00147597959355e-06
1084 5.04052640404318e-06
1085 5.10572116052543e-06
1086 5.20682387161742e-06
1087 5.35925640265944e-06
1088 5.52779813212112e-06
1089 5.63295338973369e-06
1090 5.58647149251357e-06
1091 5.38701191166169e-06
1092 5.15169989645514e-06
1093 5.0047941382303e-06
1094 4.95015670987087e-06
1095 4.92686243824814e-06
1096 4.90345104675605e-06
1097 4.8804316792328e-06
1098 4.86103252317349e-06
1099 4.84242840581572e-06
1100 4.82355968656201e-06
1101 4.8065710309686e-06
1102 4.79661198804493e-06
1103 4.79788320029151e-06
1104 4.81597152246493e-06
1105 4.85583788112365e-06
1106 4.92889767356175e-06
1107 5.0457730136344e-06
1108 5.22428979365941e-06
1109 5.45928996675826e-06
1110 5.72939002463357e-06
1111 5.9097687752363e-06
1112 5.88280105984218e-06
1113 5.55577196292845e-06
1114 5.26435831149641e-06
1115 5.22554732240721e-06
1116 5.47824555363619e-06
1117 5.76292763598829e-06
1118 6.013174206565e-06
1119 6.03794967179994e-06
1120 6.12206757688583e-06
1121 6.18517198169855e-06
1122 6.84154588892483e-06
1123 7.72200859167071e-06
1124 8.90255806895723e-06
1125 8.43348549617318e-06
1126 7.05176408999364e-06
1127 5.74655400509272e-06
1128 5.98381071403864e-06
1129 5.97714775940617e-06
1130 6.10840681058988e-06
1131 5.19534384757492e-06
1132 5.27201625954987e-06
1133 4.97110179553673e-06
1134 5.3612080979093e-06
1135 5.00719441642872e-06
1136 5.3855673181058e-06
1137 4.88420100652753e-06
1138 5.37331264949259e-06
1139 4.90719384882432e-06
1140 5.29926881198683e-06
1141 4.85435831842906e-06
1142 5.30349027316745e-06
1143 4.8419164682123e-06
1144 5.46822139524394e-06
1145 5.01595730284521e-06
1146 5.87640336457973e-06
1147 5.31782940660364e-06
1148 6.27300758493377e-06
1149 5.3921456553141e-06
1150 6.01936751598586e-06
1151 5.09728239173057e-06
1152 5.5419183562222e-06
1153 4.83854350363799e-06
1154 5.2485300050531e-06
1155 4.74494313618123e-06
1156 5.15341294238247e-06
1157 4.79945768150358e-06
1158 5.26653232735264e-06
1159 4.95983773829423e-06
1160 5.45818454034119e-06
1161 5.06365010632948e-06
1162 5.43109655004592e-06
1163 4.98798964265745e-06
1164 5.19746700011758e-06
1165 4.95143261503017e-06
1166 5.2194800823635e-06
1167 5.12039502265083e-06
1168 5.3668262545159e-06
1169 5.16698452690534e-06
1170 5.23744709202134e-06
1171 4.8839017969815e-06
1172 5.0025074747051e-06
1173 4.86761579088224e-06
1174 5.6916950534891e-06
1175 6.10073662077326e-06
1176 7.48069689215924e-06
1177 6.68067082720825e-06
1178 7.292063292752e-06
1179 5.66621881237239e-06
1180 6.53044750720255e-06
1181 6.07239122407677e-06
1182 6.25416924959765e-06
1183 5.38326828269931e-06
1184 5.17872203875669e-06
1185 4.94188711153143e-06
1186 5.0411735563749e-06
1187 5.03283780872721e-06
1188 5.09649416446933e-06
1189 5.10645952989108e-06
1190 5.20146892757367e-06
1191 5.26426239400024e-06
1192 5.3447126600048e-06
1193 5.38967587804251e-06
1194 5.42808245018733e-06
1195 5.43432959876355e-06
1196 5.45439219390431e-06
1197 5.5062222372726e-06
1198 5.61830243395178e-06
1199 5.78475092183339e-06
1200 6.02646913794302e-06
1201 6.38589967749681e-06
1202 6.95903616954041e-06
1203 7.91148921219786e-06
1204 9.63538459775037e-06
1205 1.32006090778347e-05
1206 2.16898768172769e-05
1207 4.32766921036887e-05
1208 8.41229828068002e-05
1209 7.4712649857922e-05
1210 2.73527945200414e-05
1211 1.71832578939046e-05
1212 9.54101352412806e-06
1213 5.55660057832696e-06
1214 5.04991826932866e-06
1215 4.85915969461459e-06
1216 4.80665164603877e-06
1217 4.71101930199502e-06
1218 4.64769796471032e-06
1219 4.61664011908169e-06
1220 4.59411430453471e-06
1221 4.57086120042582e-06
1222 4.5468138063498e-06
1223 4.52459961652885e-06
1224 4.50541331997734e-06
1225 4.48885456361126e-06
1226 4.47411857829039e-06
1227 4.46076727134148e-06
1228 4.44863620230329e-06
1229 4.43771195524079e-06
1230 4.42790716514985e-06
1231 4.41911754589519e-06
1232 4.41120341743151e-06
1233 4.4040488509367e-06
1234 4.39756076398012e-06
1235 4.39165359011895e-06
1236 4.38628339072267e-06
1237 4.38138785274766e-06
1238 4.3769054749454e-06
1239 4.3728039031965e-06
1240 4.36904008260797e-06
1241 4.3655642731899e-06
1242 4.3623555900929e-06
1243 4.35938862319851e-06
1244 4.35663603193248e-06
1245 4.35407785515096e-06
1246 4.35170561596809e-06
1247 4.34949688632447e-06
1248 4.34744760091643e-06
1249 4.34554958972377e-06
1250 4.34378479519104e-06
1251 4.34215610711775e-06
1252 4.34065252219451e-06
1253 4.33927671017464e-06
1254 4.3380167056295e-06
1255 4.33687075995781e-06
1256 4.33585139525405e-06
1257 4.334947937501e-06
1258 4.33416544320941e-06
1259 4.33350019402035e-06
1260 4.33296037249953e-06
1261 4.33253951181989e-06
1262 4.33224962570478e-06
1263 4.33208637162785e-06
1264 4.33205501670919e-06
1265 4.33215482698035e-06
1266 4.33240369301924e-06
1267 4.33278629796696e-06
1268 4.33331410143456e-06
1269 4.33399338573004e-06
1270 4.3348334851645e-06
1271 4.33583141412619e-06
1272 4.33698322677145e-06
1273 4.33830411838976e-06
1274 4.33977886571402e-06
1275 4.34141129712629e-06
1276 4.34322115583363e-06
1277 4.34520794956317e-06
1278 4.34737395593743e-06
1279 4.34971742391266e-06
1280 4.35227283013262e-06
1281 4.35502151852063e-06
1282 4.3579778268299e-06
1283 4.36117686697379e-06
1284 4.36463111275209e-06
1285 4.36836984785138e-06
1286 4.37244840911788e-06
1287 4.37691046717426e-06
1288 4.3818370503157e-06
1289 4.38728556306778e-06
1290 4.39335460189483e-06
1291 4.40018163350153e-06
1292 4.4079076157999e-06
1293 4.41673915130281e-06
1294 4.42690334567786e-06
1295 4.43871059330192e-06
1296 4.45254898551184e-06
1297 4.46894312178259e-06
1298 4.48854805723542e-06
1299 4.51222672248619e-06
1300 4.54113059245032e-06
1301 4.57686949673075e-06
1302 4.62171897375185e-06
1303 4.67888249477966e-06
1304 4.75317320436375e-06
1305 4.85199100719313e-06
1306 4.98724820197083e-06
1307 5.17912372188789e-06
1308 5.46380751842435e-06
1309 5.91029218632855e-06
1310 6.65769383800807e-06
1311 7.99519107808067e-06
1312 1.04978502744046e-05
1313 1.50636645708246e-05
1314 2.21424468791476e-05
1315 2.93570373315344e-05
1316 2.90074441915422e-05
1317 1.79626886680495e-05
1318 1.01728676860091e-05
1319 1.19911489910862e-05
1320 1.56742134187482e-05
1321 1.4177740344179e-05
1322 9.06788492316402e-06
1323 5.80608042399788e-06
1324 4.68220692328991e-06
1325 4.43672545080709e-06
1326 4.56409359461674e-06
1327 4.75478193440182e-06
1328 4.82760706321805e-06
1329 4.7627545940454e-06
1330 4.63014900708991e-06
1331 4.50221641901116e-06
1332 4.41153677233075e-06
1333 4.35778625851313e-06
1334 4.32838552355541e-06
1335 4.31138776391116e-06
1336 4.29925184697311e-06
1337 4.28829518872575e-06
1338 4.27726056195787e-06
1339 4.26607333392504e-06
1340 4.25508844892342e-06
1341 4.24467504078763e-06
1342 4.23506848279409e-06
1343 4.22636314767466e-06
1344 4.21853966203756e-06
1345 4.21154524010703e-06
1346 4.20527955524719e-06
1347 4.19964445064203e-06
1348 4.1945449398284e-06
1349 4.18989447403817e-06
1350 4.18561459269995e-06
1351 4.18163756599199e-06
1352 4.17791658458011e-06
1353 4.17441251887585e-06
1354 4.1710898610603e-06
1355 4.16791384438842e-06
1356 4.16486316190401e-06
1357 4.16191759677886e-06
1358 4.15908157513911e-06
1359 4.15633565187257e-06
1360 4.15367983686021e-06
1361 4.15110697715715e-06
1362 4.14861681996559e-06
1363 4.14620938549159e-06
1364 4.1438962886664e-06
1365 4.14166961215656e-06
1366 4.13954256250904e-06
1367 4.13750920413847e-06
1368 4.1355793728437e-06
1369 4.13374561325508e-06
1370 4.13202635607401e-06
1371 4.13039821678396e-06
1372 4.12886732170659e-06
1373 4.12742311872716e-06
1374 4.12606391619885e-06
1375 4.12478782274572e-06
1376 4.12357414569797e-06
1377 4.12242211833558e-06
1378 4.12132989524583e-06
1379 4.12028805030218e-06
1380 4.11929968546776e-06
1381 4.11838257530217e-06
1382 4.11755002904801e-06
1383 4.11686595647165e-06
1384 4.11639963071586e-06
1385 4.11631863628248e-06
1386 4.11687422940954e-06
1387 4.11853259618677e-06
1388 4.12212014189173e-06
1389 4.12912718361635e-06
1390 4.14233048129375e-06
1391 4.16704327332873e-06
1392 4.21382897863865e-06
1393 4.3041736474958e-06
1394 4.48255300611322e-06
1395 4.83820911911792e-06
1396 5.52294974243139e-06
1397 6.6591880014677e-06
1398 7.92662412951728e-06
1399 8.40974924321003e-06
1400 8.22226593921727e-06
1401 9.14648250116556e-06
1402 1.18200608092422e-05
1403 1.41941685569691e-05
1404 1.65112036514259e-05
1405 1.98705937295429e-05
1406 1.77990188841015e-05
1407 1.23290446514268e-05
1408 9.91648789216981e-06
1409 1.14726192297354e-05
1410 1.58787280954975e-05
1411 1.87686441495138e-05
1412 1.61795926105768e-05
1413 1.03377348690259e-05
1414 6.18185919964986e-06
1415 4.7070486264289e-06
1416 4.41488896685271e-06
1417 4.29704568816192e-06
1418 4.16728697283375e-06
1419 4.09083034391688e-06
1420 4.09400795442494e-06
1421 4.13860595793736e-06
1422 4.1779488128002e-06
1423 4.19119067551765e-06
1424 4.18076593278194e-06
1425 4.15773783524198e-06
1426 4.13185283110185e-06
1427 4.10857560095845e-06
1428 4.08988972810942e-06
1429 4.07583561690128e-06
1430 4.06569956645786e-06
1431 4.05864461427985e-06
1432 4.05395281721788e-06
1433 4.05108409140542e-06
1434 4.0496413226121e-06
1435 4.04930817643478e-06
1436 4.04982867130421e-06
1437 4.05096574174202e-06
1438 4.05251715041022e-06
1439 4.05428548821973e-06
1440 4.05611295117847e-06
1441 4.05786040413858e-06
1442 4.05941668057963e-06
1443 4.06071348102444e-06
1444 4.06169259137279e-06
1445 4.06232594607481e-06
1446 4.06262129770685e-06
1447 4.06258854790398e-06
1448 4.06224959637047e-06
1449 4.06163317090424e-06
1450 4.06078344761251e-06
1451 4.0597415331689e-06
1452 4.05853365992304e-06
1453 4.05720300067358e-06
1454 4.05578331896805e-06
1455 4.0542882966399e-06
1456 4.05274697135027e-06
1457 4.05118240631719e-06
1458 4.04960153799205e-06
1459 4.04801226083773e-06
1460 4.04643216467271e-06
1461 4.04485430016699e-06
1462 4.04329094394473e-06
1463 4.04173736279212e-06
1464 4.04019943966993e-06
1465 4.03867415776915e-06
1466 4.03715586805298e-06
1467 4.03565414186513e-06
1468 4.03415444683031e-06
1469 4.03266672788227e-06
1470 4.03118750902376e-06
1471 4.02972046598116e-06
1472 4.02826041767668e-06
1473 4.02681964395413e-06
1474 4.02537826471594e-06
1475 4.02395417475887e-06
1476 4.02254113462952e-06
1477 4.0211494722886e-06
1478 4.01977570629874e-06
1479 4.01843156017101e-06
1480 4.01712548381283e-06
1481 4.01587132992098e-06
1482 4.01467790833721e-06
1483 4.01356967616451e-06
1484 4.01257279358802e-06
1485 4.01172831065999e-06
1486 4.01109432801583e-06
1487 4.01073420741582e-06
1488 4.01077956069251e-06
1489 4.01139668482209e-06
1490 4.01283964834853e-06
1491 4.01551916473508e-06
1492 4.02008585798086e-06
1493 4.02758868300879e-06
1494 4.039764489705e-06
1495 4.05954895699168e-06
1496 4.09179678040328e-06
1497 4.1446638743281e-06
1498 4.23138473815143e-06
1499 4.37261766328412e-06
1500 4.59772663186087e-06
1501 4.94116153615387e-06
1502 5.42413561599453e-06
1503 6.00620835822241e-06
1504 6.51187854039392e-06
1505 6.70990054629783e-06
1506 6.89986396729125e-06
1507 8.27249853774248e-06
1508 1.13609351410915e-05
1509 1.41411522429635e-05
1510 1.78753737913873e-05
1511 3.00129257300163e-05
1512 4.63267491097596e-05
1513 4.31306996322434e-05
1514 1.9563764407593e-05
1515 1.54378802159627e-05
1516 1.73219713768269e-05
1517 1.10615663850844e-05
1518 5.8917586873708e-06
1519 4.52268206330153e-06
1520 4.46830413625143e-06
1521 4.76933291237991e-06
1522 4.83973574738616e-06
1523 4.67270956261423e-06
1524 4.47006190607002e-06
1525 4.34097746582207e-06
1526 4.28404963692675e-06
1527 4.25982341922815e-06
1528 4.23904345003834e-06
1529 4.21180180665282e-06
1530 4.18046065286948e-06
1531 4.15045927848467e-06
1532 4.12514360914606e-06
1533 4.1049617269806e-06
1534 4.08877424451504e-06
1535 4.07518234823989e-06
1536 4.06320696466622e-06
1537 4.0523610627119e-06
1538 4.04248357188663e-06
1539 4.0335036646244e-06
1540 4.02537938470893e-06
1541 4.01805731997662e-06
1542 4.01143067696452e-06
1543 4.00541172895785e-06
1544 3.99989755650587e-06
1545 3.99482163637721e-06
1546 3.99011546614592e-06
1547 3.98572431303812e-06
1548 3.98160634695888e-06
1549 3.97772578186739e-06
1550 3.97404268581703e-06
1551 3.97054756495763e-06
1552 3.96721068907091e-06
1553 3.96400684898879e-06
1554 3.96092034793405e-06
1555 3.95794200214183e-06
1556 3.95505453043565e-06
1557 3.9522378095791e-06
1558 3.94949359394658e-06
1559 3.94680151738491e-06
1560 3.94417235649591e-06
1561 3.94158166461267e-06
1562 3.93902487783038e-06
1563 3.93649327223855e-06
1564 3.93398954068314e-06
1565 3.93150385924468e-06
1566 3.92902980783649e-06
1567 3.92657610182034e-06
1568 3.92412960281696e-06
1569 3.92170209928544e-06
1570 3.91928991760881e-06
1571 3.91689674594797e-06
1572 3.91451932824083e-06
1573 3.91216382578108e-06
1574 3.90983964149161e-06
1575 3.90755515855545e-06
1576 3.90530848415338e-06
1577 3.9031046172866e-06
1578 3.90095213664843e-06
1579 3.89887452267867e-06
1580 3.89686259183453e-06
1581 3.89494717778494e-06
1582 3.89312340287606e-06
1583 3.89140574297286e-06
1584 3.88981998100668e-06
1585 3.88837966769362e-06
1586 3.88710695720107e-06
1587 3.88604168821693e-06
1588 3.88523999328427e-06
1589 3.88477287893707e-06
1590 3.88475410617595e-06
1591 3.88535640838406e-06
1592 3.88686093799606e-06
1593 3.88975295706473e-06
1594 3.89481484641685e-06
1595 3.90339993572297e-06
1596 3.91791216935999e-06
1597 3.9426808304821e-06
1598 3.98564708992133e-06
1599 4.06176053213692e-06
1600 4.19957344721666e-06
1601 4.45349301858577e-06
1602 4.92372848404088e-06
1603 5.77179631333635e-06
1604 7.15839764864334e-06
1605 8.9223724606402e-06
1606 1.02134263713083e-05
1607 1.09518189077562e-05
1608 1.45300998353193e-05
1609 2.12544556745797e-05
1610 2.19262622616068e-05
1611 1.6923547737191e-05
1612 1.17667212906625e-05
1613 1.28928382689963e-05
1614 1.78019301690924e-05
1615 1.74144092324191e-05
1616 1.16313292441994e-05
1617 6.49942557506833e-06
1618 4.52779931459091e-06
1619 4.1386095965823e-06
1620 3.99931973427137e-06
1621 3.90391196680717e-06
1622 3.91127976095973e-06
1623 3.9792213709644e-06
1624 4.02796670240235e-06
1625 4.02624083972736e-06
1626 3.98921133148988e-06
1627 3.94277407977928e-06
1628 3.90296656771483e-06
1629 3.87442719196329e-06
1630 3.85583062778583e-06
1631 3.84439248868329e-06
1632 3.83777375367877e-06
1633 3.83436278017513e-06
1634 3.83304303841481e-06
1635 3.83300784889684e-06
1636 3.83361163902585e-06
1637 3.83436670214898e-06
1638 3.8349392960102e-06
1639 3.83512436508138e-06
1640 3.83483074528623e-06
1641 3.83407211379527e-06
1642 3.83291310646072e-06
1643 3.83143691928822e-06
1644 3.82975160961863e-06
1645 3.82795734177321e-06
1646 3.82613251892483e-06
1647 3.82435179291374e-06
1648 3.82265906861967e-06
1649 3.82108771623813e-06
1650 3.81966992291094e-06
1651 3.81841252661275e-06
1652 3.81731907950211e-06
1653 3.81639032598358e-06
1654 3.8156297937908e-06
1655 3.81501593949007e-06
1656 3.81455255549223e-06
1657 3.81422331074965e-06
1658 3.81401556748262e-06
1659 3.81393155290954e-06
1660 3.81394188275763e-06
1661 3.81404615212855e-06
1662 3.81423527096025e-06
1663 3.81448251629557e-06
1664 3.81479860744882e-06
1665 3.81516118541647e-06
1666 3.81556030937258e-06
1667 3.81599044307901e-06
1668 3.8164487676795e-06
1669 3.81693161410901e-06
1670 3.81742671273777e-06
1671 3.81793755210857e-06
1672 3.81845844366069e-06
1673 3.81899953616482e-06
1674 3.81956235384617e-06
1675 3.8201505837554e-06
1676 3.82077279537096e-06
1677 3.82144540922447e-06
1678 3.82218631023168e-06
1679 3.82300875345631e-06
1680 3.82395938047786e-06
1681 3.82507733731607e-06
1682 3.82642746221684e-06
1683 3.82809336052503e-06
1684 3.83021881100731e-06
1685 3.83300605721892e-06
1686 3.83679158555417e-06
1687 3.8421672650113e-06
1688 3.85020804327674e-06
1689 3.86303745436578e-06
1690 3.88526787853394e-06
1691 3.92749649602564e-06
1692 4.01578207953612e-06
1693 4.21440949505136e-06
1694 4.66451478864371e-06
1695 5.53773370537414e-06
1696 6.50406264202097e-06
1697 6.24692331396837e-06
1698 4.81810630548196e-06
1699 4.48758922333425e-06
1700 5.78916910232152e-06
1701 6.47449531476774e-06
1702 5.50660144060799e-06
1703 5.31999637498615e-06
1704 6.10545442114585e-06
1705 6.77932009907067e-06
1706 7.53412390874786e-06
1707 8.60816710357426e-06
1708 9.29819465511628e-06
1709 1.0600510044867e-05
1710 1.55683308706855e-05
1711 2.22562280569605e-05
1712 2.41571448380551e-05
1713 1.86130236095039e-05
1714 1.29364674437138e-05
1715 1.44036380920198e-05
1716 1.81342742058277e-05
1717 1.6076285052069e-05
1718 9.79961662128659e-06
1719 5.50780588959299e-06
1720 4.15796285979653e-06
1721 3.88777780080396e-06
1722 3.81022926243535e-06
1723 3.82498730155945e-06
1724 3.89705568815124e-06
1725 3.95083984239264e-06
1726 3.94861167463123e-06
1727 3.90617860002163e-06
1728 3.85443524553164e-06
1729 3.81201354338057e-06
1730 3.78331123784825e-06
1731 3.76593890705212e-06
1732 3.7562742565278e-06
1733 3.75145904818464e-06
1734 3.74950412840303e-06
1735 3.74906107158957e-06
1736 3.74920632439935e-06
1737 3.7493511305664e-06
1738 3.74916097789502e-06
1739 3.74851037698853e-06
1740 3.74740477093738e-06
1741 3.74593721486072e-06
1742 3.74420749205129e-06
1743 3.74234531341777e-06
1744 3.74045729767403e-06
1745 3.73862826541504e-06
1746 3.73692862309927e-06
1747 3.73539440368109e-06
1748 3.73405507569924e-06
1749 3.73292836730599e-06
1750 3.73201590408989e-06
1751 3.73130709263592e-06
1752 3.73080496807177e-06
1753 3.73049336388487e-06
1754 3.73035313283587e-06
1755 3.73037916823193e-06
1756 3.73055853297721e-06
1757 3.7308681344328e-06
1758 3.7312946669088e-06
1759 3.73183138191457e-06
1760 3.73246231555324e-06
1761 3.73317862789602e-06
1762 3.7339629794797e-06
1763 3.73481285387278e-06
1764 3.7357194709875e-06
1765 3.73668981779041e-06
1766 3.73771528017208e-06
1767 3.73880829596107e-06
1768 3.73996880365102e-06
1769 3.74120814661261e-06
1770 3.74255761781406e-06
1771 3.74402240310712e-06
1772 3.74565638794344e-06
1773 3.74749807441344e-06
1774 3.74959202775749e-06
1775 3.75202330227253e-06
1776 3.75487389203677e-06
1777 3.75826717546879e-06
1778 3.76236994714674e-06
1779 3.7673982200559e-06
1780 3.77363918868667e-06
1781 3.78149622326518e-06
1782 3.79151580998105e-06
1783 3.80445992065326e-06
1784 3.82141996024821e-06
1785 3.84391543750695e-06
1786 3.87414419700249e-06
1787 3.91524192444592e-06
1788 3.97171557198206e-06
1789 4.05006491988047e-06
1790 4.15962847588602e-06
1791 4.31388683930223e-06
1792 4.53310568548915e-06
1793 4.85090448520964e-06
1794 5.3342704280368e-06
1795 6.14699955825415e-06
1796 7.73784634944263e-06
1797 1.12451398734947e-05
1798 1.83083738196643e-05
1799 2.83691298612609e-05
1800 3.61326552482666e-05
1801 2.63954499319929e-05
1802 1.23234061639721e-05
1803 1.41357176985935e-05
1804 1.70431621493439e-05
1805 1.17394458662989e-05
1806 6.14111323171684e-06
1807 4.29279804503579e-06
1808 4.01127909510635e-06
1809 4.26827589206091e-06
1810 4.46861951086674e-06
1811 4.39839556298782e-06
1812 4.19261305861784e-06
1813 4.01675031169724e-06
1814 3.92480448363486e-06
1815 3.89228766073391e-06
1816 3.88236718429358e-06
1817 3.87267671719815e-06
1818 3.85615982334553e-06
1819 3.83494302791387e-06
1820 3.81370185520424e-06
1821 3.79555659435482e-06
1822 3.78132358225258e-06
1823 3.77034026344258e-06
1824 3.76154437609078e-06
1825 3.75406800401379e-06
1826 3.7473641801844e-06
1827 3.74120609192286e-06
1828 3.73549437970855e-06
1829 3.73019785360196e-06
1830 3.72529366221563e-06
1831 3.72077513577374e-06
1832 3.71662169318387e-06
1833 3.7127851707508e-06
1834 3.70922990267086e-06
1835 3.70592038434481e-06
1836 3.70282401762623e-06
1837 3.69992178306244e-06
1838 3.69719227177878e-06
1839 3.69461145055539e-06
1840 3.69216457407617e-06
1841 3.68982540399632e-06
1842 3.68757551694188e-06
1843 3.68541348516604e-06
1844 3.68334110234514e-06
1845 3.68134443551327e-06
1846 3.67940964551838e-06
1847 3.67753829533246e-06
1848 3.67571577819525e-06
1849 3.67393815947636e-06
1850 3.67220921926315e-06
1851 3.67051802807605e-06
1852 3.66886889491269e-06
1853 3.66725944689339e-06
1854 3.66569148002593e-06
1855 3.66415711283707e-06
1856 3.66266522300318e-06
1857 3.66121109129924e-06
1858 3.6597939742089e-06
1859 3.65842183125409e-06
1860 3.65708781224772e-06
1861 3.65580260841547e-06
1862 3.65457444306827e-06
1863 3.65340036490025e-06
1864 3.65228523624417e-06
1865 3.6512400091171e-06
1866 3.65027357729364e-06
1867 3.64940252539636e-06
1868 3.64863376256519e-06
1869 3.64798877594552e-06
1870 3.64750624359811e-06
1871 3.64722674583984e-06
1872 3.64721309054072e-06
1873 3.6475878847364e-06
1874 3.64853104539709e-06
1875 3.65033962945738e-06
1876 3.65355064391171e-06
1877 3.65914170663562e-06
1878 3.66893003456159e-06
1879 3.6864403920589e-06
1880 3.71870622317694e-06
1881 3.78019551439124e-06
1882 3.90136265027152e-06
1883 4.14691047079252e-06
1884 4.65068998645179e-06
1885 5.65444077826882e-06
1886 7.41762024025761e-06
1887 9.67173599142335e-06
1888 1.12472500450878e-05
1889 1.23698300669162e-05
1890 1.60485351039519e-05
1891 1.86148442846701e-05
1892 1.49642561506624e-05
1893 1.06087386795295e-05
1894 9.64121539848151e-06
1895 1.42707366075356e-05
1896 1.77113843542465e-05
1897 1.46784105741204e-05
1898 8.58776728751565e-06
1899 4.9886996302817e-06
1900 4.13594660830796e-06
1901 4.0043812135071e-06
1902 3.81835057750024e-06
1903 3.6836017721642e-06
1904 3.6969584155111e-06
1905 3.77591981370884e-06
1906 3.81959580275648e-06
1907 3.80336318617225e-06
1908 3.75422142540049e-06
1909 3.70279021122411e-06
1910 3.66364123260965e-06
1911 3.63825712978993e-06
1912 3.62315242075084e-06
1913 3.61477400678822e-06
1914 3.61080993394047e-06
1915 3.60989079128959e-06
1916 3.61106589708005e-06
1917 3.61349167155023e-06
1918 3.61638931101194e-06
1919 3.61909697466078e-06
1920 3.62115245433614e-06
1921 3.62229644590428e-06
1922 3.62246368668107e-06
1923 3.62172759482693e-06
1924 3.62025235389751e-06
1925 3.61825182659814e-06
1926 3.61592788378218e-06
1927 3.61346721555744e-06
1928 3.61102068624497e-06
1929 3.60869357707294e-06
1930 3.60657180231705e-06
1931 3.60469179117029e-06
1932 3.60308885849481e-06
1933 3.60177317926258e-06
1934 3.60074209992955e-06
1935 3.60000399557414e-06
1936 3.59955060558192e-06
1937 3.59938614025168e-06
1938 3.5995110092002e-06
1939 3.59994461474056e-06
1940 3.60069224913939e-06
1941 3.6017790175924e-06
1942 3.60323378356675e-06
1943 3.60510868135933e-06
1944 3.60745320360234e-06
1945 3.6103530540732e-06
1946 3.61391191916471e-06
1947 3.61827468742248e-06
1948 3.62364424089101e-06
1949 3.63028497052387e-06
1950 3.63858242946602e-06
1951 3.64904902716567e-06
1952 3.66245074556737e-06
1953 3.67985889759836e-06
1954 3.70282714046155e-06
1955 3.73364771388829e-06
1956 3.77567875453622e-06
1957 3.83395546865728e-06
1958 3.91600997939889e-06
1959 4.03320927588879e-06
1960 4.2028507949432e-06
1961 4.45172325580501e-06
1962 4.82312775407223e-06
1963 5.3948656741909e-06
1964 6.33128884075518e-06
1965 8.03174855512623e-06
1966 1.13913903661e-05
1967 1.72774934776854e-05
1968 2.36651616567496e-05
1969 2.47955838794667e-05
1970 1.60399702930913e-05
1971 9.00296318917526e-06
1972 1.26165819303026e-05
1973 1.70091780242743e-05
1974 1.38153778928007e-05
1975 7.49408632105819e-06
1976 4.55148890110557e-06
1977 3.8397375194954e-06
1978 3.83570928941257e-06
1979 4.07711473804362e-06
1980 4.20604305240158e-06
1981 4.12298142715706e-06
1982 3.94258660862956e-06
1983 3.79390131044488e-06
1984 3.71666442844365e-06
1985 3.69006004596883e-06
1986 3.68555995833297e-06
1987 3.68531020655816e-06
1988 3.68140728812527e-06
1989 3.67251562183313e-06
1990 3.66058409551062e-06
1991 3.64818856690707e-06
1992 3.63704379946483e-06
1993 3.62778026463673e-06
1994 3.6203375975985e-06
1995 3.61434291218377e-06
1996 3.60940864341242e-06
1997 3.60519386122249e-06
1998 3.60146686706475e-06
1999 3.59807395344802e-06
};
\addlegendentry{Train}
\addplot [semithick, black]
table {%
0 0.00889264233410358
1 0.00869682710617781
2 0.00851981900632381
3 0.0083593251183629
4 0.00821245741099119
5 0.00807565730065107
6 0.00794346258044243
7 0.00780930323526263
8 0.00767433363944292
9 0.00754718855023384
10 0.00743523938581347
11 0.00734081864356995
12 0.00726209999993443
13 0.00719614140689373
14 0.00714016240090132
15 0.00709197297692299
16 0.00704985344782472
17 0.00701226200908422
18 0.00697763497009873
19 0.00694418186321855
20 0.00690942350775003
21 0.00686959223821759
22 0.00681864516809583
23 0.00674469489604235
24 0.00662509305402637
25 0.00641284789890051
26 0.00601165695115924
27 0.00525782629847527
28 0.00404340960085392
29 0.00266184494830668
30 0.00167168059851974
31 0.00120662979315966
32 0.000971056229900569
33 0.000816231593489647
34 0.000712474109604955
35 0.000637085409834981
36 0.000578750332351774
37 0.000533187936525792
38 0.000496921420563012
39 0.00046734869829379
40 0.000442629941971973
41 0.000421579723479226
42 0.000403394107706845
43 0.00038749712985009
44 0.000373468472389504
45 0.000360981532139704
46 0.000349801412085071
47 0.000339730380801484
48 0.000330611830577254
49 0.000322311243508011
50 0.000314717530272901
51 0.000307734764646739
52 0.000301288207992911
53 0.000295316596748307
54 0.000289767340291291
55 0.000284590350929648
56 0.000279741536360234
57 0.000275181664619595
58 0.000270876917056739
59 0.000266798102529719
60 0.000262920511886477
61 0.000259220396401361
62 0.000255676539381966
63 0.000252270081546158
64 0.000248983997153118
65 0.000245803210418671
66 0.00024271420261357
67 0.000239703222177923
68 0.000236757681705058
69 0.000233867758652195
70 0.000231022422667593
71 0.000228212302317843
72 0.000225427953409962
73 0.000222661386942491
74 0.00021990506502334
75 0.000217149470699951
76 0.000214386571315117
77 0.000211609905818477
78 0.000208812445634976
79 0.000205987264052965
80 0.000203132178285159
81 0.000200239388504997
82 0.000197302200831473
83 0.000194313906831667
84 0.000191277606063522
85 0.000188189616892487
86 0.000185050201253034
87 0.000181860668817535
88 0.000178621936356649
89 0.000175332519575022
90 0.000171995387063362
91 0.000168618280440569
92 0.000165209363331087
93 0.000161776333698072
94 0.000158328548423015
95 0.000154878609464504
96 0.000151439860928804
97 0.000148025996168144
98 0.000144652731250972
99 0.000141335724038072
100 0.000138089410029352
101 0.000134928312036209
102 0.000131863707792945
103 0.000128908242913894
104 0.000126068131066859
105 0.00012334507482592
106 0.000120740245620254
107 0.000118252894026227
108 0.000115879833174404
109 0.000113617417810019
110 0.000111460525658913
111 0.000109403677925002
112 0.000107440377178136
113 0.000105564278783277
114 0.000103769380075391
115 0.000102048608823679
116 0.000100395176559687
117 9.88034080364741e-05
118 9.72674097283743e-05
119 9.57817755988799e-05
120 9.43414852372371e-05
121 9.29418238229118e-05
122 9.15785858524032e-05
123 9.02479150681756e-05
124 8.89460643520579e-05
125 8.76699486980215e-05
126 8.64167741383426e-05
127 8.51842050906271e-05
128 8.3970204286743e-05
129 8.27730400487781e-05
130 8.15909515949897e-05
131 8.04234950919636e-05
132 7.92698483564891e-05
133 7.8129414760042e-05
134 7.70019032643177e-05
135 7.58871974539943e-05
136 7.47853700886481e-05
137 7.36965303076431e-05
138 7.26208454580046e-05
139 7.155862840591e-05
140 7.05100173945539e-05
141 6.94754126016051e-05
142 6.84556216583587e-05
143 6.74510301905684e-05
144 6.64625404169783e-05
145 6.54908071737736e-05
146 6.45365071250126e-05
147 6.36005061096512e-05
148 6.26832552370615e-05
149 6.17854311713018e-05
150 6.09073395025916e-05
151 6.00494968239218e-05
152 5.92123142268974e-05
153 5.83958790230099e-05
154 5.7600365835242e-05
155 5.68258510611486e-05
156 5.60723419766873e-05
157 5.5339729442494e-05
158 5.46278570254799e-05
159 5.39364664291497e-05
160 5.32652884430718e-05
161 5.26139556313865e-05
162 5.19822751812171e-05
163 5.13698432769161e-05
164 5.07756703882478e-05
165 5.01994836668018e-05
166 4.96406137244776e-05
167 4.90985075884964e-05
168 4.85723066958599e-05
169 4.80615126434714e-05
170 4.75654705951456e-05
171 4.70835802843794e-05
172 4.6615103201475e-05
173 4.61594099760987e-05
174 4.57159985671751e-05
175 4.52841850346886e-05
176 4.48634818894789e-05
177 4.44532124674879e-05
178 4.40528783656191e-05
179 4.36619193351362e-05
180 4.32799279224128e-05
181 4.29063038609456e-05
182 4.25406688009389e-05
183 4.21827244281303e-05
184 4.1832092392724e-05
185 4.14885907957796e-05
186 4.11519831686746e-05
187 4.08218838856556e-05
188 4.04978163714986e-05
189 4.01795659854542e-05
190 3.98667725676205e-05
191 3.95590504922438e-05
192 3.92561378248502e-05
193 3.89577617170289e-05
194 3.8663660234306e-05
195 3.8373691495508e-05
196 3.80875426344573e-05
197 3.78050790459383e-05
198 3.75261442968622e-05
199 3.72505637642462e-05
200 3.6978173739044e-05
201 3.67088323400822e-05
202 3.64423503924627e-05
203 3.61786405846942e-05
204 3.59176556230523e-05
205 3.56591990566812e-05
206 3.54032017639838e-05
207 3.51495145878289e-05
208 3.4898090234492e-05
209 3.46488486684393e-05
210 3.44016661983915e-05
211 3.41565646522213e-05
212 3.39134494424798e-05
213 3.36722296196967e-05
214 3.34328033204656e-05
215 3.3195290598087e-05
216 3.29596259689424e-05
217 3.27258385368623e-05
218 3.24939719575923e-05
219 3.2263902539853e-05
220 3.20355466101319e-05
221 3.18089551001322e-05
222 3.15840115945321e-05
223 3.13607597490773e-05
224 3.11391268041916e-05
225 3.09191163978539e-05
226 3.07005502691027e-05
227 3.04835430142703e-05
228 3.02680819004308e-05
229 3.00540286843898e-05
230 2.98414761346066e-05
231 2.96303587674629e-05
232 2.9420678401948e-05
233 2.92124805127969e-05
234 2.90056996163912e-05
235 2.8800295694964e-05
236 2.85962578345789e-05
237 2.83935805782676e-05
238 2.81923767033732e-05
239 2.79924915957963e-05
240 2.77939016086748e-05
241 2.75967267953092e-05
242 2.74009398708586e-05
243 2.72064353339374e-05
244 2.70132386503974e-05
245 2.68213734671008e-05
246 2.66307997662807e-05
247 2.64414484263398e-05
248 2.6253386749886e-05
249 2.60666365647921e-05
250 2.58811942330794e-05
251 2.56969979091082e-05
252 2.55142094829353e-05
253 2.53327834798256e-05
254 2.5152732632705e-05
255 2.49740260187536e-05
256 2.47967527684523e-05
257 2.46207982854685e-05
258 2.44462844420923e-05
259 2.42731093749171e-05
260 2.41013603954343e-05
261 2.39310047618346e-05
262 2.37620624830015e-05
263 2.35945553868078e-05
264 2.34285434999038e-05
265 2.32639777095756e-05
266 2.31009107665159e-05
267 2.29392917390214e-05
268 2.27791824727319e-05
269 2.26205520448275e-05
270 2.24634204641916e-05
271 2.23078041017288e-05
272 2.21537175093545e-05
273 2.20011370402062e-05
274 2.18501281779027e-05
275 2.17006818274967e-05
276 2.15527343243593e-05
277 2.14063493331196e-05
278 2.12616250792053e-05
279 2.11183942155913e-05
280 2.09767495107371e-05
281 2.08366909646429e-05
282 2.06982422241708e-05
283 2.05613996513421e-05
284 2.04261741600931e-05
285 2.02925384655828e-05
286 2.01605216716416e-05
287 2.0030089217471e-05
288 1.99012665689224e-05
289 1.97740373550914e-05
290 1.96484124899143e-05
291 1.95243683265289e-05
292 1.94018648471683e-05
293 1.9280931155663e-05
294 1.91616054507904e-05
295 1.90438440768048e-05
296 1.89276615856215e-05
297 1.88130998139968e-05
298 1.8700133296079e-05
299 1.85887838597409e-05
300 1.84789860213641e-05
301 1.83708016265882e-05
302 1.82640451384941e-05
303 1.81587984116049e-05
304 1.80550687218783e-05
305 1.79527742147911e-05
306 1.78519003384281e-05
307 1.77525107574183e-05
308 1.76545290742069e-05
309 1.75579207279952e-05
310 1.74627184605924e-05
311 1.73689022631152e-05
312 1.72764393937541e-05
313 1.71853753272444e-05
314 1.70956354850205e-05
315 1.70072617038386e-05
316 1.6920259440667e-05
317 1.68345068232156e-05
318 1.67501093528699e-05
319 1.66669760801597e-05
320 1.65850869962014e-05
321 1.65044293680694e-05
322 1.64249486260815e-05
323 1.63467102538561e-05
324 1.62696123879869e-05
325 1.61936859512934e-05
326 1.61188982019667e-05
327 1.60452000272926e-05
328 1.59726514539216e-05
329 1.59011542564258e-05
330 1.58307393576251e-05
331 1.57613812916679e-05
332 1.56930418597767e-05
333 1.56257301568985e-05
334 1.55593934323406e-05
335 1.54940535139758e-05
336 1.5429637642228e-05
337 1.53662240336416e-05
338 1.53037326526828e-05
339 1.5242147128447e-05
340 1.51814811033546e-05
341 1.51217236634693e-05
342 1.50628084156779e-05
343 1.50047771967365e-05
344 1.49475927173626e-05
345 1.48912549775559e-05
346 1.48357867146842e-05
347 1.47811033457401e-05
348 1.47272166941548e-05
349 1.46740958371083e-05
350 1.4621746231569e-05
351 1.45701524161268e-05
352 1.4519249816658e-05
353 1.44691521200002e-05
354 1.44196865221602e-05
355 1.4370846656675e-05
356 1.43227844091598e-05
357 1.42753688123776e-05
358 1.42286417030846e-05
359 1.41825466926093e-05
360 1.41371201607399e-05
361 1.40923166327411e-05
362 1.40480988193303e-05
363 1.40044776344439e-05
364 1.39614367071772e-05
365 1.39189987748978e-05
366 1.38770810735878e-05
367 1.38357427204028e-05
368 1.37948645715369e-05
369 1.37545548568596e-05
370 1.37148435896961e-05
371 1.36756116262404e-05
372 1.36369199026376e-05
373 1.35988439069479e-05
374 1.35613063321216e-05
375 1.35243681143038e-05
376 1.34879346660455e-05
377 1.34520705614705e-05
378 1.34166748466669e-05
379 1.33817429741612e-05
380 1.33473567984765e-05
381 1.33134499264997e-05
382 1.32799623315805e-05
383 1.32468603624147e-05
384 1.32141612994019e-05
385 1.31818578665843e-05
386 1.31498782138806e-05
387 1.31182969198562e-05
388 1.30870130305993e-05
389 1.30560320030781e-05
390 1.30254693431198e-05
391 1.2995225006307e-05
392 1.29653035401134e-05
393 1.29356958495919e-05
394 1.29064701468451e-05
395 1.28775218399824e-05
396 1.28489491544315e-05
397 1.28207148009096e-05
398 1.27928569781943e-05
399 1.27653747767908e-05
400 1.27382272694376e-05
401 1.27114353745128e-05
402 1.26849199659773e-05
403 1.26587519844179e-05
404 1.26329459817498e-05
405 1.26074191939551e-05
406 1.2582189810928e-05
407 1.25572114484385e-05
408 1.25324368127622e-05
409 1.25079595818534e-05
410 1.24838106785319e-05
411 1.24599964692607e-05
412 1.24364696603152e-05
413 1.24132138807909e-05
414 1.23902600535075e-05
415 1.23675326904049e-05
416 1.23450836326811e-05
417 1.232293107023e-05
418 1.23009695016663e-05
419 1.22792907859548e-05
420 1.22578330774559e-05
421 1.22366172945476e-05
422 1.22156689030817e-05
423 1.21949206004501e-05
424 1.21744233183563e-05
425 1.21541825137683e-05
426 1.21341608974035e-05
427 1.21143038995797e-05
428 1.20947197501664e-05
429 1.20754220915842e-05
430 1.20563427117304e-05
431 1.203755800816e-05
432 1.20190161396749e-05
433 1.20006616270985e-05
434 1.19825353976921e-05
435 1.19645574159222e-05
436 1.1946738595725e-05
437 1.19289106805809e-05
438 1.19111700769281e-05
439 1.18934312922647e-05
440 1.18757398013258e-05
441 1.18580901471432e-05
442 1.18403986562043e-05
443 1.18228044811985e-05
444 1.18054431368364e-05
445 1.17884010251146e-05
446 1.17718818728463e-05
447 1.1756234925997e-05
448 1.17415966087719e-05
449 1.17283225335996e-05
450 1.17168465294526e-05
451 1.17074396257522e-05
452 1.17002928163856e-05
453 1.16952269308968e-05
454 1.16918636194896e-05
455 1.16890032586525e-05
456 1.16846631499357e-05
457 1.16762639663648e-05
458 1.16604442155221e-05
459 1.16339833766688e-05
460 1.15951561383554e-05
461 1.15458969958127e-05
462 1.14946806206717e-05
463 1.14593376565608e-05
464 1.14629037852865e-05
465 1.15085031211493e-05
466 1.15293414637563e-05
467 1.13960250018863e-05
468 1.11193039629143e-05
469 1.1068442290707e-05
470 1.18204616228468e-05
471 1.37277529574931e-05
472 1.58577186084585e-05
473 1.54318768181838e-05
474 1.26607383208466e-05
475 1.10463897726731e-05
476 1.11576600829721e-05
477 1.1662978977256e-05
478 1.14537815534277e-05
479 1.08569856820395e-05
480 1.07000696516479e-05
481 1.08815520434291e-05
482 1.09868597064633e-05
483 1.09180728031788e-05
484 1.07713321995107e-05
485 1.06474981294014e-05
486 1.05959252323373e-05
487 1.06072830021731e-05
488 1.06419856820139e-05
489 1.06675033748616e-05
490 1.06753268482862e-05
491 1.06765710370382e-05
492 1.06826400951832e-05
493 1.06924999272451e-05
494 1.06978513940703e-05
495 1.06941588455811e-05
496 1.06850156953442e-05
497 1.06775969470618e-05
498 1.06760144262807e-05
499 1.06794650491793e-05
500 1.06851912278216e-05
501 1.06919133031624e-05
502 1.07003597804578e-05
503 1.07115993159823e-05
504 1.07255291368347e-05
505 1.07408777694218e-05
506 1.0756143637991e-05
507 1.07705609480035e-05
508 1.0783965990413e-05
509 1.07964824564988e-05
510 1.08084987004986e-05
511 1.08205913420534e-05
512 1.08338026620913e-05
513 1.08494023152161e-05
514 1.08689828266506e-05
515 1.08940948848613e-05
516 1.0926480172202e-05
517 1.09686043288093e-05
518 1.10229775600601e-05
519 1.10919681901578e-05
520 1.1177355190739e-05
521 1.1279565114819e-05
522 1.13972346298397e-05
523 1.15265484055271e-05
524 1.16623159556184e-05
525 1.17993668027339e-05
526 1.19365768114221e-05
527 1.20810545922723e-05
528 1.22528726933524e-05
529 1.24870148283662e-05
530 1.28255032905145e-05
531 1.32903442136012e-05
532 1.38428413265501e-05
533 1.44062660183408e-05
534 1.50682099047117e-05
535 1.62789765454363e-05
536 1.85502412932692e-05
537 2.14783431147225e-05
538 2.26440042752074e-05
539 2.00844406208489e-05
540 1.68523456522962e-05
541 1.54892459249822e-05
542 1.40697893584729e-05
543 1.19845071822056e-05
544 1.21181101349066e-05
545 1.53859364218079e-05
546 2.21835434786044e-05
547 3.28057467413601e-05
548 4.36952686868608e-05
549 4.58816466561984e-05
550 3.39510042977054e-05
551 1.87141486094333e-05
552 1.11173212644644e-05
553 9.44281964621041e-06
554 9.50024877965916e-06
555 9.60023862717208e-06
556 9.62463582254713e-06
557 9.71420286077773e-06
558 9.85111455520382e-06
559 9.95474692899734e-06
560 9.98745144897839e-06
561 9.96043945633573e-06
562 9.90227545116795e-06
563 9.83630889095366e-06
564 9.77526997303357e-06
565 9.72366433416028e-06
566 9.68164658843307e-06
567 9.64780883805361e-06
568 9.62046033237129e-06
569 9.59796852839645e-06
570 9.57917563937372e-06
571 9.5630584837636e-06
572 9.5488994702464e-06
573 9.53615290200105e-06
574 9.52439586399123e-06
575 9.51328820519848e-06
576 9.50257162912749e-06
577 9.49217519519152e-06
578 9.48187971516745e-06
579 9.47163971432019e-06
580 9.46149793890072e-06
581 9.45146348385606e-06
582 9.44162729865639e-06
583 9.43206396186724e-06
584 9.42299811867997e-06
585 9.41457165026804e-06
586 9.40703466767445e-06
587 9.40060454013292e-06
588 9.3956205091672e-06
589 9.39224719331833e-06
590 9.39077381190145e-06
591 9.39145502343308e-06
592 9.39448000281118e-06
593 9.39996607485227e-06
594 9.40795780479675e-06
595 9.41845883062342e-06
596 9.4313763838727e-06
597 9.44654584600357e-06
598 9.46368345466908e-06
599 9.4824154075468e-06
600 9.50235244090436e-06
601 9.52310801949352e-06
602 9.54424103838392e-06
603 9.56541407504119e-06
604 9.58623240876477e-06
605 9.60649504122557e-06
606 9.6259545898647e-06
607 9.64448554441333e-06
608 9.66198786045425e-06
609 9.67842515819939e-06
610 9.69381744653219e-06
611 9.70829296420561e-06
612 9.72186717262957e-06
613 9.73466194409411e-06
614 9.74677914200583e-06
615 9.75818693405017e-06
616 9.76907176664099e-06
617 9.77937543211738e-06
618 9.7890842880588e-06
619 9.79814558377257e-06
620 9.80647746473551e-06
621 9.81386892817682e-06
622 9.82015990302898e-06
623 9.82508936431259e-06
624 9.82831988949329e-06
625 9.82957590167644e-06
626 9.82863093668129e-06
627 9.82513574854238e-06
628 9.8189384516445e-06
629 9.80994991550688e-06
630 9.79825199465267e-06
631 9.78408388618845e-06
632 9.76809133135248e-06
633 9.75126840785379e-06
634 9.73518126556883e-06
635 9.72238467511488e-06
636 9.71707140706712e-06
637 9.72651105257683e-06
638 9.76446335698711e-06
639 9.85831138677895e-06
640 1.0064981324831e-05
641 1.05062208604068e-05
642 1.14308513730066e-05
643 1.3211953046266e-05
644 1.5710820662207e-05
645 1.66396985150641e-05
646 1.38670575324795e-05
647 1.09072298073443e-05
648 1.16421315397019e-05
649 1.47217833728064e-05
650 1.34600231831428e-05
651 1.05215713119833e-05
652 1.05190410977229e-05
653 1.04354521681671e-05
654 1.01119758255663e-05
655 9.91513479675632e-06
656 9.74423073785147e-06
657 9.54942061071051e-06
658 9.43311351875309e-06
659 9.42920360103017e-06
660 9.45161445997655e-06
661 9.44145267567364e-06
662 9.41222424444277e-06
663 9.4033130153548e-06
664 9.41637063078815e-06
665 9.42765655054245e-06
666 9.4244996944326e-06
667 9.41305188462138e-06
668 9.40514746616827e-06
669 9.40374502533814e-06
670 9.40375048230635e-06
671 9.40062636800576e-06
672 9.39444726100191e-06
673 9.38770426728297e-06
674 9.38237917580409e-06
675 9.378948561789e-06
676 9.37691038416233e-06
677 9.37596632866189e-06
678 9.37629647523863e-06
679 9.37839467951562e-06
680 9.3828466560808e-06
681 9.39009260036983e-06
682 9.40043082664488e-06
683 9.41415873967344e-06
684 9.43160921451636e-06
685 9.45320152823115e-06
686 9.47929675021442e-06
687 9.5101695478661e-06
688 9.54580536927097e-06
689 9.58579039433971e-06
690 9.62926424108446e-06
691 9.67466075962875e-06
692 9.71970530372346e-06
693 9.76137380348518e-06
694 9.79599462880287e-06
695 9.81950779532781e-06
696 9.82817709882511e-06
697 9.81915309239412e-06
698 9.79202650341904e-06
699 9.75031252892222e-06
700 9.70353539742064e-06
701 9.6699686764623e-06
702 9.67708183452487e-06
703 9.75285729509778e-06
704 9.89961972663878e-06
705 1.00722281786148e-05
706 1.02241365311784e-05
707 1.03733964351704e-05
708 1.05560775409685e-05
709 1.07932437458658e-05
710 1.10939627120388e-05
711 1.13832884380827e-05
712 1.14649938041111e-05
713 1.12186253318214e-05
714 1.08438443930936e-05
715 1.07882760858047e-05
716 1.15150405690656e-05
717 1.31031601995346e-05
718 1.39483554448816e-05
719 1.18825328172534e-05
720 9.91104298009304e-06
721 1.05537146737333e-05
722 1.13532069008215e-05
723 1.08071999420645e-05
724 9.81426273938268e-06
725 9.38507764658425e-06
726 9.43248596740887e-06
727 9.49355489865411e-06
728 9.38602079258999e-06
729 9.23168136068853e-06
730 9.15607324714074e-06
731 9.17579927772749e-06
732 9.20631828194018e-06
733 9.21462469705148e-06
734 9.19164267543238e-06
735 9.16875978873577e-06
736 9.14603333512787e-06
737 9.14036627364112e-06
738 9.1379433797556e-06
739 9.15872988116462e-06
740 9.1951405920554e-06
741 9.27331893763039e-06
742 9.39528945309576e-06
743 9.56417261477327e-06
744 9.83195877779508e-06
745 1.00790366559522e-05
746 1.05832423287211e-05
747 1.07383066278999e-05
748 1.15822995212511e-05
749 1.1124574484711e-05
750 1.23778054330614e-05
751 1.09319789771689e-05
752 1.2700636034424e-05
753 1.11408617158304e-05
754 1.14178155854461e-05
755 1.08687299871235e-05
756 1.0210353138973e-05
757 1.02384419733426e-05
758 9.64065202424536e-06
759 9.78902880888199e-06
760 9.92862896964652e-06
761 1.00296756500029e-05
762 9.87811472441535e-06
763 9.65579783951398e-06
764 9.48609158513136e-06
765 9.46873296925332e-06
766 9.52583377511473e-06
767 9.50536468735663e-06
768 9.26441862247884e-06
769 9.00852410268271e-06
770 9.01604289538227e-06
771 9.43486975302221e-06
772 9.96937251329655e-06
773 1.04572800410097e-05
774 1.048740978149e-05
775 1.02602962215315e-05
776 9.71785721048946e-06
777 9.4751821961836e-06
778 9.339397365693e-06
779 9.61401565291453e-06
780 9.65543495112797e-06
781 9.61298883339623e-06
782 9.01327712199418e-06
783 9.64187620411394e-06
784 9.91036540654022e-06
785 1.22483515951899e-05
786 1.36696990011842e-05
787 1.2421938663465e-05
788 1.08605681816698e-05
789 9.88204737950582e-06
790 9.66601328400429e-06
791 9.50085905060405e-06
792 9.29053203435615e-06
793 9.18825480766827e-06
794 9.1781112132594e-06
795 9.23886273085373e-06
796 9.32762577576796e-06
797 9.3883345471113e-06
798 9.41153302846942e-06
799 9.38314406084828e-06
800 9.32014154386707e-06
801 9.23359039006755e-06
802 9.15253895072965e-06
803 9.08019683265593e-06
804 9.02440297068097e-06
805 8.96921756066149e-06
806 8.92498337634606e-06
807 8.89498460310278e-06
808 8.91898980626138e-06
809 9.00127633940428e-06
810 9.17280704015866e-06
811 9.4026472652331e-06
812 9.70978089753771e-06
813 1.00442639450193e-05
814 1.03959791886155e-05
815 1.08064123196527e-05
816 1.10665287138545e-05
817 1.18509688036283e-05
818 1.17817207865301e-05
819 1.37746956170304e-05
820 1.21682223834796e-05
821 1.4129696864984e-05
822 1.10318360384554e-05
823 1.03764750747359e-05
824 1.09105258161435e-05
825 1.1235280908295e-05
826 1.16907804112998e-05
827 1.05646095107659e-05
828 9.69920438365079e-06
829 9.15983764571138e-06
830 9.21162973099854e-06
831 9.32021430344321e-06
832 9.29128782445332e-06
833 9.12596078705974e-06
834 9.04419721337035e-06
835 9.077003596758e-06
836 9.22419440030353e-06
837 9.30584974412341e-06
838 9.36318065214437e-06
839 9.28755707718665e-06
840 9.2542895799852e-06
841 9.13387157197576e-06
842 9.14479187485995e-06
843 9.02860574569786e-06
844 9.1176325440756e-06
845 8.90299907041481e-06
846 9.15179862204241e-06
847 8.70383792062057e-06
848 9.57725751504768e-06
849 8.80945935932687e-06
850 1.13569258246571e-05
851 1.16877527034376e-05
852 1.39034400490345e-05
853 1.47063847180107e-05
854 1.07529622255242e-05
855 1.05554117908468e-05
856 9.9883773145848e-06
857 9.56495114223799e-06
858 9.65764320426388e-06
859 9.68512904364616e-06
860 9.5776786110946e-06
861 9.35849402594613e-06
862 9.12745963432826e-06
863 9.06389141164254e-06
864 9.18183923204197e-06
865 9.43515806284267e-06
866 9.6891544671962e-06
867 9.85590759228216e-06
868 9.85261340247234e-06
869 9.70436576608336e-06
870 9.49070636124816e-06
871 9.33225419430528e-06
872 9.30350233829813e-06
873 9.43525446928106e-06
874 9.64477294473909e-06
875 9.67251071415376e-06
876 9.30457554204622e-06
877 8.87398437043885e-06
878 8.92964271770325e-06
879 9.55448285822058e-06
880 1.0350863703934e-05
881 1.0903958354902e-05
882 1.07467512862058e-05
883 1.01196555988281e-05
884 9.4606130005559e-06
885 9.34430227061966e-06
886 9.62347621680237e-06
887 9.88204919849522e-06
888 9.63720776780974e-06
889 9.20641286938917e-06
890 8.99792121344944e-06
891 9.80191089183791e-06
892 1.00713259598706e-05
893 1.12560883280821e-05
894 1.33388994072448e-05
895 1.24858152048546e-05
896 1.43892548294389e-05
897 1.03434304037364e-05
898 9.91314300335944e-06
899 9.39611800276907e-06
900 9.14254178496776e-06
901 9.29117777559441e-06
902 9.48195247474359e-06
903 9.50381399889011e-06
904 9.41509097174276e-06
905 9.28905774344457e-06
906 9.22860181162832e-06
907 9.18837577046361e-06
908 9.14554675546242e-06
909 9.07520279724849e-06
910 8.98999041964998e-06
911 8.91211038833717e-06
912 8.87976148078451e-06
913 8.89773400558624e-06
914 8.94624736247351e-06
915 8.99499627848854e-06
916 9.03043110156432e-06
917 9.05086108105024e-06
918 9.06587592908181e-06
919 9.08046968106646e-06
920 9.09891969058663e-06
921 9.11903316591633e-06
922 9.14390329853632e-06
923 9.17427041713381e-06
924 9.21856280911015e-06
925 9.28466306504561e-06
926 9.38549510465236e-06
927 9.55143059400143e-06
928 9.78830848907819e-06
929 1.02098265415407e-05
930 1.07232572190696e-05
931 1.17015943033039e-05
932 1.24998159662937e-05
933 1.40150277729845e-05
934 1.3874979231332e-05
935 1.41307682497427e-05
936 1.17621484605479e-05
937 1.20826034617494e-05
938 1.18316675070673e-05
939 1.18839870992815e-05
940 1.13408814286231e-05
941 9.25250060390681e-06
942 1.05138988146791e-05
943 1.08068134068162e-05
944 1.08031717900303e-05
945 1.0043399015558e-05
946 9.56090934778331e-06
947 9.1235824584146e-06
948 9.01178373169387e-06
949 9.01255771168508e-06
950 9.06801051314687e-06
951 9.07537742023123e-06
952 9.09345817490248e-06
953 9.11629376787459e-06
954 9.17501438379986e-06
955 9.19714057090459e-06
956 9.19915237318492e-06
957 9.1318388513173e-06
958 9.06882814888377e-06
959 8.98710550245596e-06
960 8.96351230039727e-06
961 8.93550168257207e-06
962 8.95510311238468e-06
963 8.93303240445675e-06
964 8.96071742317872e-06
965 8.92162006493891e-06
966 8.97951304068556e-06
967 8.90690262167482e-06
968 9.03746604308253e-06
969 8.85420104168588e-06
970 9.18716614251025e-06
971 8.71278916747542e-06
972 9.71703047980554e-06
973 8.70112398843048e-06
974 1.18978932732716e-05
975 1.25511132864631e-05
976 1.42947792483028e-05
977 1.40550519063254e-05
978 1.05151893876609e-05
979 1.03152551673702e-05
980 9.81455377768725e-06
981 9.13723943085643e-06
982 9.22923936741427e-06
983 9.38230459723854e-06
984 9.45664942264557e-06
985 9.34553190745646e-06
986 9.16966655495344e-06
987 9.03213822311955e-06
988 8.96686924534151e-06
989 8.93855940375943e-06
990 8.97830886970041e-06
991 9.08486890693894e-06
992 9.27769360714592e-06
993 9.57295742409769e-06
994 9.94695801637135e-06
995 1.02379653981188e-05
996 1.02603044069838e-05
997 9.98428004095331e-06
998 9.63005095400149e-06
999 9.52891605265904e-06
1000 9.94860147329746e-06
1001 1.06926763692172e-05
1002 1.05720355350059e-05
1003 9.22408980841283e-06
1004 8.69045379658928e-06
1005 9.38823541218881e-06
1006 1.01447903944063e-05
1007 1.02536396298092e-05
1008 9.95838581729913e-06
1009 9.55768791754963e-06
1010 9.47179069044068e-06
1011 9.33485171117354e-06
1012 9.30870282900287e-06
1013 8.92052958079148e-06
1014 8.81196683621965e-06
1015 8.45922477310523e-06
1016 8.62728666106705e-06
1017 8.70448002388002e-06
1018 9.0325274868519e-06
1019 9.02621013665339e-06
1020 9.31500017031794e-06
1021 9.42177757679019e-06
1022 9.81551693257643e-06
1023 1.24395401144284e-05
1024 1.21544489957159e-05
1025 2.02388000616338e-05
1026 1.10687524284003e-05
1027 1.02114081528271e-05
1028 9.72000816545915e-06
1029 9.84618782240432e-06
1030 9.78649222815875e-06
1031 9.66417519521201e-06
1032 9.28341978578828e-06
1033 9.05331216927152e-06
1034 9.0486319095362e-06
1035 9.1148112915107e-06
1036 9.14301199372858e-06
1037 9.14252996153664e-06
1038 9.12422456167405e-06
1039 9.10045400814852e-06
1040 9.07714365894208e-06
1041 9.07177218323341e-06
1042 9.08025958779035e-06
1043 9.074127774511e-06
1044 9.0316016212455e-06
1045 8.97183144843439e-06
1046 8.92919979378348e-06
1047 8.91892523213755e-06
1048 8.92244224814931e-06
1049 8.90651790541597e-06
1050 8.84515793586615e-06
1051 8.75408841238823e-06
1052 8.69856103236089e-06
1053 8.77121783560142e-06
1054 9.02943247638177e-06
1055 9.48624438024126e-06
1056 1.00611641755677e-05
1057 1.05890376289608e-05
1058 1.07519554148894e-05
1059 1.04141099654953e-05
1060 9.7244464996038e-06
1061 9.23747757042293e-06
1062 9.30336591409286e-06
1063 1.00088418548694e-05
1064 1.05998515209649e-05
1065 9.92460627458058e-06
1066 8.73772660270333e-06
1067 8.88102476892527e-06
1068 9.37911227083532e-06
1069 1.03612374005024e-05
1070 9.56164149101824e-06
1071 1.03350739664165e-05
1072 1.15285383799346e-05
1073 1.18794796435395e-05
1074 1.94462736544665e-05
1075 1.11372373794438e-05
1076 1.01475652627414e-05
1077 9.85478891379898e-06
1078 9.77552099357126e-06
1079 9.64982154982863e-06
1080 9.42170754569815e-06
1081 9.16434692044277e-06
1082 9.04874559637392e-06
1083 9.08906622498762e-06
1084 9.20031379791908e-06
1085 9.32142665988067e-06
1086 9.44862131291302e-06
1087 9.52768550632754e-06
1088 9.49066543398658e-06
1089 9.30055830394849e-06
1090 9.05192428035662e-06
1091 8.89363673195476e-06
1092 8.88686008693185e-06
1093 8.93964261194924e-06
1094 8.96772780833999e-06
1095 8.97438530955696e-06
1096 8.98198868526379e-06
1097 8.98941107152496e-06
1098 8.98247890290804e-06
1099 8.95681660040282e-06
1100 8.91918625711696e-06
1101 8.88306658453075e-06
1102 8.85895587998675e-06
1103 8.85929330252111e-06
1104 8.89319198904559e-06
1105 8.98215694178361e-06
1106 9.13253552425886e-06
1107 9.37024015001953e-06
1108 9.63241291174199e-06
1109 9.88695410342189e-06
1110 9.85395035968395e-06
1111 9.58585314947413e-06
1112 9.0746952992049e-06
1113 8.97311474545859e-06
1114 9.2474683697219e-06
1115 9.78249772742856e-06
1116 1.0265509445162e-05
1117 1.07115638456889e-05
1118 1.11666859083925e-05
1119 1.12809993879637e-05
1120 1.20612376122153e-05
1121 1.16537221401813e-05
1122 1.3669278814632e-05
1123 1.2177266398794e-05
1124 1.31063279695809e-05
1125 1.03592583400314e-05
1126 9.59776207309915e-06
1127 1.05849449028028e-05
1128 9.87365092441905e-06
1129 1.10491564555559e-05
1130 9.0116973296972e-06
1131 9.77650142885977e-06
1132 8.76237390912138e-06
1133 9.75397870206507e-06
1134 8.57141640153714e-06
1135 9.86339091468835e-06
1136 8.3100276242476e-06
1137 9.79340165940812e-06
1138 8.44788519316353e-06
1139 9.72966063272906e-06
1140 8.57233135320712e-06
1141 9.82413894234924e-06
1142 8.64532830746612e-06
1143 9.97598272078903e-06
1144 8.81933010532521e-06
1145 1.04481741800555e-05
1146 9.11565803107806e-06
1147 1.10512928586104e-05
1148 9.34074614633573e-06
1149 1.09583661469514e-05
1150 9.05403067008592e-06
1151 1.04762621049304e-05
1152 8.74256147653796e-06
1153 1.00890019893995e-05
1154 8.60650106915273e-06
1155 9.92009790934389e-06
1156 8.55081088957377e-06
1157 1.00182369351387e-05
1158 8.54975405673031e-06
1159 1.01746991276741e-05
1160 8.59094052430009e-06
1161 9.96578910417156e-06
1162 8.69158157001948e-06
1163 9.6115863925661e-06
1164 8.93549258762505e-06
1165 9.72543148236582e-06
1166 9.34689524001442e-06
1167 9.98339965008199e-06
1168 9.6205676527461e-06
1169 9.9643702924368e-06
1170 9.50917728914646e-06
1171 9.67642154137138e-06
1172 9.32762941374676e-06
1173 9.95990012597758e-06
1174 9.83660356723703e-06
1175 1.18236002890626e-05
1176 9.45362808124628e-06
1177 1.32490813484765e-05
1178 8.69967152539175e-06
1179 1.22153405754943e-05
1180 9.61218302109046e-06
1181 1.14131580630783e-05
1182 9.67728101386456e-06
1183 9.98021369014168e-06
1184 9.40676545724273e-06
1185 9.61527348408708e-06
1186 9.40669724514009e-06
1187 9.58982491283678e-06
1188 9.58065811573761e-06
1189 9.86692612059414e-06
1190 1.00189481599955e-05
1191 1.02799122032593e-05
1192 1.04450382423238e-05
1193 1.06441193565843e-05
1194 1.07590176412486e-05
1195 1.09038219306967e-05
1196 1.10760893221595e-05
1197 1.13738306026789e-05
1198 1.17999024951132e-05
1199 1.24457583297044e-05
1200 1.34147503558779e-05
1201 1.49608295032522e-05
1202 1.75058739841916e-05
1203 2.1981208192301e-05
1204 3.06037072732579e-05
1205 4.88254045194481e-05
1206 8.62518063513562e-05
1207 0.000124135782243684
1208 5.56484556000214e-05
1209 1.0026595191448e-05
1210 2.11868828046136e-05
1211 1.28683504954097e-05
1212 9.21023547562072e-06
1213 8.50356354931137e-06
1214 7.74382715462707e-06
1215 7.81872677180218e-06
1216 7.90859121480025e-06
1217 7.952885425766e-06
1218 7.95716732682195e-06
1219 7.94747120380634e-06
1220 7.95034520706395e-06
1221 7.9643259596196e-06
1222 7.97886787040625e-06
1223 7.9894116424839e-06
1224 7.99703320808476e-06
1225 8.00365614850307e-06
1226 8.01015903562075e-06
1227 8.01648639026098e-06
1228 8.02228623797419e-06
1229 8.02735485194717e-06
1230 8.03164493845543e-06
1231 8.0353256635135e-06
1232 8.03840612206841e-06
1233 8.04099454398965e-06
1234 8.04308911028784e-06
1235 8.04472801974043e-06
1236 8.04595765657723e-06
1237 8.04675619292539e-06
1238 8.04719911684515e-06
1239 8.04727278591599e-06
1240 8.04705632617697e-06
1241 8.04653154773405e-06
1242 8.04575665824814e-06
1243 8.04476894700201e-06
1244 8.04359751782613e-06
1245 8.04223236627877e-06
1246 8.04073624749435e-06
1247 8.03916373115499e-06
1248 8.03743569122162e-06
1249 8.0356803664472e-06
1250 8.03386956249597e-06
1251 8.03203693067189e-06
1252 8.03023976914119e-06
1253 8.02846352598863e-06
1254 8.02672366262414e-06
1255 8.02505928731989e-06
1256 8.0235258792527e-06
1257 8.02208433015039e-06
1258 8.0208037616103e-06
1259 8.01970054453705e-06
1260 8.0187546700472e-06
1261 8.01802707428578e-06
1262 8.01756414148258e-06
1263 8.01733585831244e-06
1264 8.01734859123826e-06
1265 8.01764963398455e-06
1266 8.01832084107446e-06
1267 8.01924852567026e-06
1268 8.02053909865208e-06
1269 8.02218164608348e-06
1270 8.02420163381612e-06
1271 8.02661725174403e-06
1272 8.02936574473279e-06
1273 8.03251805336913e-06
1274 8.03604962129612e-06
1275 8.03997681941837e-06
1276 8.04432602308225e-06
1277 8.04904266260564e-06
1278 8.05418494564947e-06
1279 8.05973650130909e-06
1280 8.06574917078251e-06
1281 8.07213200459955e-06
1282 8.07895685284166e-06
1283 8.08632648841012e-06
1284 8.09421999292681e-06
1285 8.10268375062151e-06
1286 8.11189420346636e-06
1287 8.12188682175474e-06
1288 8.13286169432104e-06
1289 8.1448979472043e-06
1290 8.15827388578327e-06
1291 8.17327418189961e-06
1292 8.19027900433866e-06
1293 8.20970490167383e-06
1294 8.23213395051425e-06
1295 8.25824554340215e-06
1296 8.28912652650615e-06
1297 8.3260374594829e-06
1298 8.37059269542806e-06
1299 8.42511963128345e-06
1300 8.49281695991522e-06
1301 8.57826489664149e-06
1302 8.68821462063352e-06
1303 8.83242319105193e-06
1304 9.02649480849504e-06
1305 9.29544421524042e-06
1306 9.68103177001467e-06
1307 1.02570647868561e-05
1308 1.11587942228653e-05
1309 1.26448539958801e-05
1310 1.5212226571748e-05
1311 1.97476911125705e-05
1312 2.73605837719515e-05
1313 3.74302253476344e-05
1314 4.299021748011e-05
1315 3.3375239581801e-05
1316 1.53879991557915e-05
1317 9.1285073722247e-06
1318 1.43990901051438e-05
1319 1.97423596546287e-05
1320 1.6731259165681e-05
1321 1.06227280411986e-05
1322 8.73335920914542e-06
1323 8.36727485875599e-06
1324 7.82541519583901e-06
1325 7.6710502980859e-06
1326 7.8450566434185e-06
1327 8.02194881543983e-06
1328 8.07299693406094e-06
1329 8.0436357166036e-06
1330 8.00095585873351e-06
1331 7.97018037701491e-06
1332 7.95275536802365e-06
1333 7.94642801338341e-06
1334 7.94892184785567e-06
1335 7.95708001533058e-06
1336 7.96760377852479e-06
1337 7.97806387708988e-06
1338 7.98727069195593e-06
1339 7.99479039415019e-06
1340 8.0007457654574e-06
1341 8.0053669080371e-06
1342 8.00889574747998e-06
1343 8.01157966634491e-06
1344 8.01355054136366e-06
1345 8.01493843027856e-06
1346 8.01584064902272e-06
1347 8.01631176727824e-06
1348 8.01641181169543e-06
1349 8.01618443801999e-06
1350 8.01568785391282e-06
1351 8.01496662461432e-06
1352 8.01402347860858e-06
1353 8.01295027486049e-06
1354 8.01172245701309e-06
1355 8.01043006504187e-06
1356 8.00906582298921e-06
1357 8.00764792074915e-06
1358 8.006236384972e-06
1359 8.00484122009948e-06
1360 8.00349607743556e-06
1361 8.00220095698023e-06
1362 8.00098951003747e-06
1363 7.99986992205959e-06
1364 7.99887129687704e-06
1365 7.99799272499513e-06
1366 7.99724421085557e-06
1367 7.99661756900605e-06
1368 7.99612371338299e-06
1369 7.9957208072301e-06
1370 7.99546069174539e-06
1371 7.99525059846928e-06
1372 7.99512054072693e-06
1373 7.99502413428854e-06
1374 7.99491954239784e-06
1375 7.99483677837998e-06
1376 7.99473309598397e-06
1377 7.99453664512839e-06
1378 7.99430472397944e-06
1379 7.99400095274905e-06
1380 7.99360623204848e-06
1381 7.9931542131817e-06
1382 7.992643986654e-06
1383 7.9921828728402e-06
1384 7.99186727817869e-06
1385 7.9920273492462e-06
1386 7.9930541687645e-06
1387 7.99582903709961e-06
1388 8.00194084149553e-06
1389 8.01446276454953e-06
1390 8.03941020421917e-06
1391 8.08859294920694e-06
1392 8.18617354525486e-06
1393 8.38159576233011e-06
1394 8.77250931807794e-06
1395 9.5237628556788e-06
1396 1.0780770026031e-05
1397 1.2267146303202e-05
1398 1.31079941638745e-05
1399 1.33763996927883e-05
1400 1.46351330840844e-05
1401 1.73824992089067e-05
1402 1.98092657228699e-05
1403 2.26080646825721e-05
1404 2.68138464889489e-05
1405 2.23983006435446e-05
1406 1.39189241963322e-05
1407 1.37441575134289e-05
1408 2.04290736292023e-05
1409 2.68765888904454e-05
1410 2.71203207375947e-05
1411 2.00618414964993e-05
1412 1.20977192636929e-05
1413 8.29456985229626e-06
1414 7.74379532231251e-06
1415 8.0630788943381e-06
1416 8.20814148028148e-06
1417 8.21945650386624e-06
1418 8.27267922431929e-06
1419 8.35950322652934e-06
1420 8.40880238683894e-06
1421 8.39592394186184e-06
1422 8.33929243526654e-06
1423 8.26663472253131e-06
1424 8.19735032564495e-06
1425 8.14047416497488e-06
1426 8.09797347756103e-06
1427 8.06842490419513e-06
1428 8.04926366981817e-06
1429 8.03786224423675e-06
1430 8.03211332822684e-06
1431 8.03042803454446e-06
1432 8.03157217887929e-06
1433 8.03472084953682e-06
1434 8.03917737357551e-06
1435 8.04444880486699e-06
1436 8.05006584414514e-06
1437 8.05571198725374e-06
1438 8.06106163508957e-06
1439 8.06589287094539e-06
1440 8.07003016234376e-06
1441 8.07336436992045e-06
1442 8.07579272077419e-06
1443 8.07735432317713e-06
1444 8.07807100500213e-06
1445 8.07794640422799e-06
1446 8.07714695838513e-06
1447 8.07572178018745e-06
1448 8.07378546596738e-06
1449 8.0714380601421e-06
1450 8.06885509518906e-06
1451 8.06605839898111e-06
1452 8.06314073997783e-06
1453 8.06014941190369e-06
1454 8.05720083008055e-06
1455 8.0543049989501e-06
1456 8.05142826720839e-06
1457 8.04868432169314e-06
1458 8.04603405413218e-06
1459 8.04347018856788e-06
1460 8.04098090156913e-06
1461 8.03859074949287e-06
1462 8.03627517598215e-06
1463 8.03401781013235e-06
1464 8.0317831816501e-06
1465 8.02960857981816e-06
1466 8.02743943495443e-06
1467 8.02531030785758e-06
1468 8.02318754722364e-06
1469 8.02115482656518e-06
1470 8.01913756731665e-06
1471 8.01718579168664e-06
1472 8.01528858573874e-06
1473 8.0134686868405e-06
1474 8.01167061581509e-06
1475 8.01001624495257e-06
1476 8.00841189629864e-06
1477 8.00697944214335e-06
1478 8.00568523118272e-06
1479 8.00458201410947e-06
1480 8.00360885477858e-06
1481 8.00289762992179e-06
1482 8.0023901318782e-06
1483 8.0022364272736e-06
1484 8.00249654275831e-06
1485 8.00324687588727e-06
1486 8.00473662820878e-06
1487 8.00702491687844e-06
1488 8.01065652922262e-06
1489 8.01600162958493e-06
1490 8.02371505415067e-06
1491 8.03502734925132e-06
1492 8.05184845376061e-06
1493 8.07687956694281e-06
1494 8.11494555819081e-06
1495 8.17389354779152e-06
1496 8.26610539661488e-06
1497 8.41208566271234e-06
1498 8.64214325702051e-06
1499 8.99832957657054e-06
1500 9.5253017207142e-06
1501 1.02441499620909e-05
1502 1.11166482383851e-05
1503 1.21138227768824e-05
1504 1.34204892674461e-05
1505 1.49540774145862e-05
1506 1.76164248841815e-05
1507 2.36064370255917e-05
1508 3.09435054077767e-05
1509 3.90453096770216e-05
1510 5.62628629268147e-05
1511 6.75157061778009e-05
1512 3.93316840927582e-05
1513 1.06306561065139e-05
1514 1.70833936863346e-05
1515 2.15302261494799e-05
1516 1.27646617329447e-05
1517 8.85785539139761e-06
1518 8.29375221655937e-06
1519 7.66261018725345e-06
1520 7.77170225774171e-06
1521 7.93884555605473e-06
1522 7.94212974142283e-06
1523 7.89283512858674e-06
1524 7.84626718086656e-06
1525 7.82017195888329e-06
1526 7.82070219429443e-06
1527 7.83717769081704e-06
1528 7.85624342825031e-06
1529 7.87156204751227e-06
1530 7.88263514550636e-06
1531 7.89121986599639e-06
1532 7.89909881859785e-06
1533 7.90722242527409e-06
1534 7.91569163993699e-06
1535 7.92419541539857e-06
1536 7.93231993156951e-06
1537 7.93979779700749e-06
1538 7.94650804891717e-06
1539 7.95249343354953e-06
1540 7.95778669271385e-06
1541 7.96249332779553e-06
1542 7.96666336100316e-06
1543 7.97037046140758e-06
1544 7.97363099991344e-06
1545 7.97649136075052e-06
1546 7.97897246229695e-06
1547 7.9811015893938e-06
1548 7.98292421677615e-06
1549 7.98444671090692e-06
1550 7.98567543824902e-06
1551 7.98666587797925e-06
1552 7.98743531049695e-06
1553 7.98796372691868e-06
1554 7.98830569692655e-06
1555 7.98847213445697e-06
1556 7.98847031546757e-06
1557 7.98830114945304e-06
1558 7.98801556811668e-06
1559 7.98753808339825e-06
1560 7.98697965365136e-06
1561 7.986279342731e-06
1562 7.98549444880337e-06
1563 7.98460860096384e-06
1564 7.98365272203228e-06
1565 7.98260771261994e-06
1566 7.98150085756788e-06
1567 7.98037035565358e-06
1568 7.97917618911015e-06
1569 7.97797929408262e-06
1570 7.97673601482529e-06
1571 7.97553457232425e-06
1572 7.97434768173844e-06
1573 7.97316079115262e-06
1574 7.9720211942913e-06
1575 7.97094799054321e-06
1576 7.96992299001431e-06
1577 7.96895164967282e-06
1578 7.96802851255052e-06
1579 7.96719541540369e-06
1580 7.96639415057143e-06
1581 7.96564836491598e-06
1582 7.96495351096382e-06
1583 7.96426320448518e-06
1584 7.96354379417608e-06
1585 7.96279709902592e-06
1586 7.96195308794267e-06
1587 7.96093354438199e-06
1588 7.95976393419551e-06
1589 7.95838423073292e-06
1590 7.95676896814257e-06
1591 7.95500091044232e-06
1592 7.95320738689043e-06
1593 7.95182040747022e-06
1594 7.95172945800005e-06
1595 7.95477535575628e-06
1596 7.96468884800561e-06
1597 7.98919063527137e-06
1598 8.04357478045858e-06
1599 8.15851399238454e-06
1600 8.39475524117006e-06
1601 8.86707766767358e-06
1602 9.76754654402612e-06
1603 1.13202631837339e-05
1604 1.3515349564841e-05
1605 1.57983358803904e-05
1606 1.77504589373711e-05
1607 2.14907722693169e-05
1608 2.83332337858155e-05
1609 2.68206404143712e-05
1610 1.78920126927551e-05
1611 1.5081758647284e-05
1612 2.30357200052822e-05
1613 2.9357113817241e-05
1614 2.37962703977246e-05
1615 1.37985580295208e-05
1616 8.6358295448008e-06
1617 7.78304274717811e-06
1618 8.02206977823516e-06
1619 8.17714590084506e-06
1620 8.31229226605501e-06
1621 8.48597392177908e-06
1622 8.57977738633053e-06
1623 8.5366682469612e-06
1624 8.41003293317044e-06
1625 8.27324038255028e-06
1626 8.16555257188156e-06
1627 8.0946756497724e-06
1628 8.05381114332704e-06
1629 8.03315651864978e-06
1630 8.02480371930869e-06
1631 8.02326303528389e-06
1632 8.02516206022119e-06
1633 8.02835529611912e-06
1634 8.03156399342697e-06
1635 8.03391958470456e-06
1636 8.03495731815929e-06
1637 8.03446346253622e-06
1638 8.03247803560225e-06
1639 8.02914291853085e-06
1640 8.02469639893388e-06
1641 8.01949227025034e-06
1642 8.01385976956226e-06
1643 8.00809721113183e-06
1644 8.00246561993845e-06
1645 7.99717599875294e-06
1646 7.99242025095737e-06
1647 7.98825203673914e-06
1648 7.98479777586181e-06
1649 7.98205201135715e-06
1650 7.97999200585764e-06
1651 7.97860775492154e-06
1652 7.97785014583496e-06
1653 7.97765460447408e-06
1654 7.97801112639718e-06
1655 7.97878783487249e-06
1656 7.97995016910136e-06
1657 7.98146902525332e-06
1658 7.98323162598535e-06
1659 7.98520977696171e-06
1660 7.98728251538705e-06
1661 7.98952623881632e-06
1662 7.991823622433e-06
1663 7.99413464847021e-06
1664 7.99646659288555e-06
1665 7.99875397206051e-06
1666 8.00098678155337e-06
1667 8.00318593974225e-06
1668 8.00539510237286e-06
1669 8.00758061814122e-06
1670 8.00967063696589e-06
1671 8.01176793174818e-06
1672 8.01388614490861e-06
1673 8.01608348410809e-06
1674 8.01836540631484e-06
1675 8.02079830464208e-06
1676 8.02340218797326e-06
1677 8.02625800133683e-06
1678 8.02946851763409e-06
1679 8.03310467745177e-06
1680 8.03734656074084e-06
1681 8.04231603979133e-06
1682 8.04825685918331e-06
1683 8.05544277682202e-06
1684 8.0643576438888e-06
1685 8.07553169579478e-06
1686 8.09007724456023e-06
1687 8.1096559370053e-06
1688 8.13727365311934e-06
1689 8.17857835500035e-06
1690 8.24594735604478e-06
1691 8.36631079437211e-06
1692 8.60262298374437e-06
1693 9.0861321950797e-06
1694 9.97841198113747e-06
1695 1.10110922832973e-05
1696 1.10406399471685e-05
1697 1.00967172329547e-05
1698 9.72029556578491e-06
1699 1.02413678177982e-05
1700 1.02719404821983e-05
1701 8.8476417658967e-06
1702 8.89854982233373e-06
1703 1.00679644674528e-05
1704 1.0849009413505e-05
1705 1.16118662845111e-05
1706 1.30542166516534e-05
1707 1.47075261338614e-05
1708 1.68312126334058e-05
1709 2.25537733058445e-05
1710 3.11491166939959e-05
1711 3.0982839234639e-05
1712 1.88462072401308e-05
1713 1.48994176925044e-05
1714 2.52362551691476e-05
1715 2.95872450806201e-05
1716 2.13614512176719e-05
1717 1.18988255053409e-05
1718 8.08781078376342e-06
1719 7.73618376115337e-06
1720 8.02607974037528e-06
1721 8.24317339720437e-06
1722 8.46932380227372e-06
1723 8.61946227814769e-06
1724 8.60083400766598e-06
1725 8.46524744702037e-06
1726 8.3091581473127e-06
1727 8.18833086668747e-06
1728 8.11348854767857e-06
1729 8.07412652648054e-06
1730 8.05648414825555e-06
1731 8.05024228611728e-06
1732 8.04898263595533e-06
1733 8.04919000074733e-06
1734 8.04900810180698e-06
1735 8.04760020400863e-06
1736 8.0447607615497e-06
1737 8.04059618531028e-06
1738 8.03535385784926e-06
1739 8.02940576249966e-06
1740 8.02309477876406e-06
1741 8.01672740635695e-06
1742 8.01058104116237e-06
1743 8.00488760432927e-06
1744 7.99982899479801e-06
1745 7.99547706265002e-06
1746 7.9918709161575e-06
1747 7.98904329712968e-06
1748 7.98698602011427e-06
1749 7.98561904957751e-06
1750 7.98494056652999e-06
1751 7.98482233221876e-06
1752 7.9852552516968e-06
1753 7.98619203123963e-06
1754 7.98750625108369e-06
1755 7.98919882072369e-06
1756 7.99116514826892e-06
1757 7.99334065959556e-06
1758 7.99568260845263e-06
1759 7.99825738795334e-06
1760 8.00090856500901e-06
1761 8.00361522124149e-06
1762 8.00633097242098e-06
1763 8.00909401732497e-06
1764 8.01192618382629e-06
1765 8.01483201939845e-06
1766 8.01775513536995e-06
1767 8.02077374828514e-06
1768 8.02385329734534e-06
1769 8.02709018898895e-06
1770 8.03058446763316e-06
1771 8.03436887508724e-06
1772 8.03858438302996e-06
1773 8.04335195425665e-06
1774 8.04877981863683e-06
1775 8.0551262726658e-06
1776 8.06263506092364e-06
1777 8.07166634331224e-06
1778 8.08265940577257e-06
1779 8.09618450148264e-06
1780 8.11297923064558e-06
1781 8.1341268014512e-06
1782 8.16103147371905e-06
1783 8.19570414023474e-06
1784 8.24089147499762e-06
1785 8.30046246846905e-06
1786 8.3800032371073e-06
1787 8.48775471240515e-06
1788 8.63579862198094e-06
1789 8.84260225575417e-06
1790 9.13721669348888e-06
1791 9.56771418714197e-06
1792 1.02217800304061e-05
1793 1.12737025119714e-05
1794 1.31008036987623e-05
1795 1.65462042787112e-05
1796 2.3389595298795e-05
1797 3.58171309926547e-05
1798 4.9235051847063e-05
1799 4.73171385237947e-05
1800 2.25831681746058e-05
1801 9.71664485405199e-06
1802 1.67649614013499e-05
1803 2.09812315006275e-05
1804 1.32501481857616e-05
1805 8.93869400897529e-06
1806 8.49137632030761e-06
1807 7.67841811466496e-06
1808 7.53191534386133e-06
1809 7.74987893237267e-06
1810 7.84940493758768e-06
1811 7.83522591518704e-06
1812 7.80137270339765e-06
1813 7.77230434323428e-06
1814 7.75725857238285e-06
1815 7.76281740400009e-06
1816 7.78287812863709e-06
1817 7.80644131737063e-06
1818 7.82681854616385e-06
1819 7.84233270678669e-06
1820 7.85400970926275e-06
1821 7.86340751801617e-06
1822 7.87177850725129e-06
1823 7.87981025496265e-06
1824 7.88764191383962e-06
1825 7.89523619459942e-06
1826 7.90232752478914e-06
1827 7.90882677392801e-06
1828 7.9146684583975e-06
1829 7.91988986748038e-06
1830 7.92451101006009e-06
1831 7.92866740084719e-06
1832 7.93233721196884e-06
1833 7.93563322076807e-06
1834 7.93858998804353e-06
1835 7.94124389358331e-06
1836 7.94361494627083e-06
1837 7.94573134044185e-06
1838 7.94762308942154e-06
1839 7.94931747805094e-06
1840 7.9508354247082e-06
1841 7.95221330918139e-06
1842 7.95343839854468e-06
1843 7.95453252067091e-06
1844 7.95548203313956e-06
1845 7.95633241068572e-06
1846 7.95708547229879e-06
1847 7.95775576989399e-06
1848 7.95834512246074e-06
1849 7.95886171545135e-06
1850 7.95931919128634e-06
1851 7.95973028289154e-06
1852 7.96008407633053e-06
1853 7.96042058937019e-06
1854 7.9607370935264e-06
1855 7.96101357991574e-06
1856 7.96128642832628e-06
1857 7.96159656601958e-06
1858 7.96189397078706e-06
1859 7.96221502241679e-06
1860 7.96254607848823e-06
1861 7.96290350990603e-06
1862 7.96326185081853e-06
1863 7.96366293798201e-06
1864 7.96406220615609e-06
1865 7.9644260040368e-06
1866 7.96476979303407e-06
1867 7.96508902567439e-06
1868 7.96534459368559e-06
1869 7.96547192294383e-06
1870 7.96546828496503e-06
1871 7.96524545876309e-06
1872 7.96475978859235e-06
1873 7.96393032942433e-06
1874 7.96276526671136e-06
1875 7.96138465375407e-06
1876 7.96010590420337e-06
1877 7.96000858827028e-06
1878 7.96370477473829e-06
1879 7.97722532297485e-06
1880 8.01435089670122e-06
1881 8.10627898317762e-06
1882 8.32216483104276e-06
1883 8.80898460309254e-06
1884 9.83986501523759e-06
1885 1.17446161311818e-05
1886 1.44075147545664e-05
1887 1.69054510479327e-05
1888 1.9114437236567e-05
1889 2.21695299842395e-05
1890 2.2991725927568e-05
1891 1.70083076227456e-05
1892 1.37686147354543e-05
1893 1.78018744918518e-05
1894 2.58841209870297e-05
1895 2.68739804596407e-05
1896 1.85543976840563e-05
1897 1.04582841231604e-05
1898 7.7638505899813e-06
1899 7.85527481639292e-06
1900 8.15342809801223e-06
1901 8.22454967419617e-06
1902 8.34740058053285e-06
1903 8.52372340887086e-06
1904 8.58497514855117e-06
1905 8.49675507197389e-06
1906 8.34025831863983e-06
1907 8.19341494207038e-06
1908 8.08921868156176e-06
1909 8.02848080638796e-06
1910 7.9999790614238e-06
1911 7.99169629317475e-06
1912 7.99471581558464e-06
1913 8.00342240836471e-06
1914 8.01441819930915e-06
1915 8.02540489530656e-06
1916 8.03498551249504e-06
1917 8.04205865279073e-06
1918 8.04618048277916e-06
1919 8.04709361545974e-06
1920 8.04505725682247e-06
1921 8.04049886937719e-06
1922 8.033939593588e-06
1923 8.02608428784879e-06
1924 8.01756505097728e-06
1925 8.00895213615149e-06
1926 8.00072939455276e-06
1927 7.99325516709359e-06
1928 7.98677046986995e-06
1929 7.981460839801e-06
1930 7.9773553807172e-06
1931 7.97451866674237e-06
1932 7.97288976173149e-06
1933 7.97243410488591e-06
1934 7.97301436250564e-06
1935 7.97459415480262e-06
1936 7.9770970842219e-06
1937 7.98047676653368e-06
1938 7.98460587247973e-06
1939 7.98950804892229e-06
1940 7.99513145466335e-06
1941 8.00153884483734e-06
1942 8.00872021500254e-06
1943 8.01678834250197e-06
1944 8.02586237114156e-06
1945 8.03618968348019e-06
1946 8.04799037723569e-06
1947 8.06165280664572e-06
1948 8.07773631095188e-06
1949 8.09687844594009e-06
1950 8.12013877293793e-06
1951 8.14876420918154e-06
1952 8.18474836705718e-06
1953 8.23078607936623e-06
1954 8.29072723718127e-06
1955 8.3704107964877e-06
1956 8.47821047500474e-06
1957 8.62703745951876e-06
1958 8.83658503880724e-06
1959 9.13839357963298e-06
1960 9.5838850029395e-06
1961 1.02629528555553e-05
1962 1.13444075395819e-05
1963 1.31696951939375e-05
1964 1.64348348334897e-05
1965 2.231249709439e-05
1966 3.11690164380707e-05
1967 3.71431233361363e-05
1968 3.01687141472939e-05
1969 1.45505337059149e-05
1970 9.23889820114709e-06
1971 1.56485821207752e-05
1972 2.08762048714561e-05
1973 1.55992402142147e-05
1974 9.60932629823219e-06
1975 9.01716703083366e-06
1976 8.29970213089837e-06
1977 7.56387998990249e-06
1978 7.57828047426301e-06
1979 7.7995518950047e-06
1980 7.89124351285864e-06
1981 7.8907651186455e-06
1982 7.8771709013381e-06
1983 7.86016425990965e-06
1984 7.84178064350272e-06
1985 7.8346947702812e-06
1986 7.84307394496864e-06
1987 7.86051441536983e-06
1988 7.87944372859783e-06
1989 7.8957882578834e-06
1990 7.90855938248569e-06
1991 7.91816364653641e-06
1992 7.92547507444397e-06
1993 7.93126218923135e-06
1994 7.9361625466845e-06
1995 7.94047809904441e-06
1996 7.94448169472162e-06
1997 7.94820789451478e-06
1998 7.95171581557952e-06
1999 7.95500545791583e-06
};
\addlegendentry{Test}

\nextgroupplot[
title={ELU/Tanh},
ymin=3.05420661743733e-06, ymax=0.001,
]
\addplot [semithick, black, dashed]
table {%
0 0.0193993477732874
1 0.0187371889478527
2 0.0181102653732523
3 0.0175048801465891
4 0.0169044834328815
5 0.0162941868766211
6 0.0156646718969569
7 0.0150067252106965
8 0.0143133896053769
9 0.0135824052267708
10 0.0128197928424925
11 0.0120393640827388
12 0.0112567066389602
13 0.0104875731049106
14 0.00975045494851656
15 0.00906576460693032
16 0.00845041536376812
17 0.0079138804867398
18 0.00746007694397122
19 0.00708857872814406
20 0.00679392702295445
21 0.0065664505091263
22 0.00639367568510352
23 0.00626269615895581
24 0.00616254379565362
25 0.00608488002035301
26 0.00602370035267086
27 0.00597476527400431
28 0.00593507764642709
29 0.00590249297238188
30 0.00587545779126231
31 0.00585282456813729
32 0.00583373088011285
33 0.00581751522622653
34 0.0058036625450768
35 0.0057917639715015
36 0.00578149123975891
37 0.00577257966142497
38 0.00576481085590785
39 0.0057580050706747
40 0.00575201311585261
41 0.00574671061622212
42 0.00574199245602358
43 0.00573776927194558
44 0.00573396166510065
45 0.00573049046579399
46 0.00572724725134321
47 0.0057239340058004
48 0.00571872007640195
49 0.00569729796188767
50 0.00560802803011029
51 0.00534423621502356
52 0.00472887834257563
53 0.00407472292863531
54 0.00372403963410761
55 0.00347613276153425
56 0.00326957598463196
57 0.00308815882817726
58 0.00292698998964624
59 0.00278101457570301
60 0.00264533135350575
61 0.0025179505782944
62 0.00239747382693167
63 0.00228316635957526
64 0.00217451284970593
65 0.00207105593426604
66 0.00197239648514369
67 0.00187821797135257
68 0.00178829515971302
69 0.00170248531503603
70 0.00162073567753396
71 0.00154303013005119
72 0.00146934491340289
73 0.00139964239065193
74 0.00133385933281716
75 0.00127189938530137
76 0.0012136284801727
77 0.00115888920231555
78 0.0011075102158884
79 0.00105931617417809
80 0.00101412656704269
81 0.000971766226030013
82 0.000932064778226049
83 0.000894855986416587
84 0.00085997842074903
85 0.000827275457822907
86 0.000796596533291449
87 0.000767801530628276
88 0.000740760806138496
89 0.000715351585427015
90 0.000691458604251238
91 0.000668971956429232
92 0.000647790210791754
93 0.000627818678140102
94 0.000608968018582345
95 0.00059115408555499
96 0.000574302344602984
97 0.000558340069460428
98 0.000543206611041569
99 0.000528840830952504
100 0.000515193043383988
101 0.000502209981505075
102 0.000489847502990415
103 0.00047805811402668
104 0.000466804492020856
105 0.000456045800660831
106 0.000445749454797806
107 0.000435888726087796
108 0.000426436969519273
109 0.000417373334585136
110 0.000408681598514704
111 0.000400345330035634
112 0.000392351215623421
113 0.000384686577831417
114 0.000377338412363315
115 0.000370293727428361
116 0.000363539599106844
117 0.000357061920567503
118 0.000350847097479345
119 0.000344881700982569
120 0.000339152116396235
121 0.00033364569293326
122 0.000328349139181228
123 0.000323250588621704
124 0.000318337750343289
125 0.000313599887022065
126 0.000309025729677614
127 0.000304605197925412
128 0.000300328387879745
129 0.000296186433160983
130 0.000292170298791916
131 0.000288272179545856
132 0.000284483979896777
133 0.000280798989194864
134 0.000277210061994992
135 0.000273711398506293
136 0.000270296814107951
137 0.000266961375245955
138 0.000263699753077162
139 0.000260507651319131
140 0.000257380617426861
141 0.000254314969197367
142 0.000251306851737354
143 0.000248352963012621
144 0.000245450274690029
145 0.000242595965005421
146 0.000239787403927494
147 0.000237022047940627
148 0.00023429759710325
149 0.000231611990898273
150 0.000228963024369477
151 0.000226348858461733
152 0.000223767788270379
153 0.000221218201545526
154 0.000218698624451008
155 0.000216207565870263
156 0.000213743827828239
157 0.000211306246882259
158 0.000208894155861117
159 0.000206506628757097
160 0.000204143029492343
161 0.000201802576981436
162 0.000199484654984872
163 0.000197188716214214
164 0.000194914367426691
165 0.000192661390371995
166 0.000190429497735067
167 0.000188218524073136
168 0.000186028626103507
169 0.000183860606000508
170 0.000181714357125884
171 0.000179590483696757
172 0.000177489336067538
173 0.000175411533376746
174 0.000173357453434164
175 0.00017132774794959
176 0.000169322876985234
177 0.000167343286108235
178 0.000165389563619556
179 0.000163462907153189
180 0.000161563741244208
181 0.000159692727976335
182 0.000157850610634114
183 0.00015603815245413
184 0.000154255651978019
185 0.000152503618096489
186 0.000150782634023017
187 0.00014909321080836
188 0.00014743583165
189 0.000145810589515349
190 0.000144217776721689
191 0.00014265758863985
192 0.000141129945092189
193 0.000139634755072393
194 0.00013817192294141
195 0.00013674085130333
196 0.000135341115139909
197 0.000133972270219829
198 0.000132633466620291
199 0.000131324003746158
200 0.000130043071635555
201 0.000128789810247554
202 0.000127563348087278
203 0.000126362748574138
204 0.000125187078936051
205 0.000124035632637742
206 0.000122907590565546
207 0.000121802143837613
208 0.00012071845296191
209 0.000119655811460007
210 0.000118613462859685
211 0.000117590758094366
212 0.000116586917698669
213 0.000115601343566141
214 0.000114633450863266
215 0.000113682606610155
216 0.000112748003061824
217 0.000111829107169115
218 0.000110925210691448
219 0.00011003580493707
220 0.000109160191982483
221 0.000108297734215057
222 0.000107447927035764
223 0.00010661023085845
224 0.00010578415526652
225 0.000104969171303537
226 0.000104164706073107
227 0.000103370357948052
228 0.000102585648392051
229 0.000101810030884053
230 0.000101043035186876
231 0.000100284259957562
232 9.95331949980027e-05
233 9.87895230082358e-05
234 9.80527728415836e-05
235 9.73224938718431e-05
236 9.65983117566793e-05
237 9.58798281374129e-05
238 9.51666939386087e-05
239 9.44585824242949e-05
240 9.37552279651754e-05
241 9.30563495273873e-05
242 9.23616443344599e-05
243 9.16709045384323e-05
244 9.09837917220102e-05
245 9.03001075016618e-05
246 8.96197570909862e-05
247 8.89422954344354e-05
248 8.82676436901875e-05
249 8.75956137065259e-05
250 8.69259390583466e-05
251 8.62584242895537e-05
252 8.55930671264105e-05
253 8.49296495744056e-05
254 8.42681249082489e-05
255 8.36082745365729e-05
256 8.29501294674628e-05
257 8.22936321185352e-05
258 8.16387583597589e-05
259 8.09854072940652e-05
260 8.03335778698511e-05
261 7.96833181908596e-05
262 7.90346583841028e-05
263 7.83877631818086e-05
264 7.7742514221768e-05
265 7.70990888554479e-05
266 7.64575322165228e-05
267 7.58180867421743e-05
268 7.51808216392647e-05
269 7.45458111879316e-05
270 7.39132849219004e-05
271 7.32834849230812e-05
272 7.2656406402416e-05
273 7.20322876759383e-05
274 7.14113900812663e-05
275 7.07939642268229e-05
276 7.01800503009054e-05
277 6.95700275343825e-05
278 6.89640254165624e-05
279 6.83622716195487e-05
280 6.77649940854508e-05
281 6.71724737344448e-05
282 6.65849455430134e-05
283 6.60025440311074e-05
284 6.54255000540616e-05
285 6.48540919172547e-05
286 6.4288490861486e-05
287 6.372889531292e-05
288 6.31754963080766e-05
289 6.26284621745299e-05
290 6.20879589092738e-05
291 6.15540793376113e-05
292 6.10270349596931e-05
293 6.05068051555691e-05
294 5.99936296197257e-05
295 5.94876822077595e-05
296 5.89889090178985e-05
297 5.84972868722389e-05
298 5.80129862157719e-05
299 5.75359876933135e-05
300 5.70664048922254e-05
301 5.66041427845221e-05
302 5.61492167463484e-05
303 5.57016270334998e-05
304 5.5261262744466e-05
305 5.48280715548799e-05
306 5.44020840749226e-05
307 5.39832067403268e-05
308 5.35714391034503e-05
309 5.31664992706737e-05
310 5.27683271016599e-05
311 5.23769757592163e-05
312 5.19921858241901e-05
313 5.1613894477498e-05
314 5.12418718301433e-05
315 5.08762397757323e-05
316 5.05166020801084e-05
317 5.01629330571518e-05
318 4.98150946626197e-05
319 4.94728558919633e-05
320 4.91361299168602e-05
321 4.88047841358252e-05
322 4.84787901200434e-05
323 4.81578059492449e-05
324 4.78418279001858e-05
325 4.75307094873756e-05
326 4.72243014186802e-05
327 4.69224075203556e-05
328 4.66249438417776e-05
329 4.63318193766327e-05
330 4.604280907472e-05
331 4.5757895549059e-05
332 4.54768378546078e-05
333 4.51996637593766e-05
334 4.49261673765022e-05
335 4.46562546088103e-05
336 4.43898494353334e-05
337 4.41268122131078e-05
338 4.38669591744656e-05
339 4.36104403860327e-05
340 4.33568941531348e-05
341 4.31063976620294e-05
342 4.28588486443005e-05
343 4.26140386906582e-05
344 4.23719514586196e-05
345 4.21325241148907e-05
346 4.18957350873939e-05
347 4.16614755991418e-05
348 4.142961281417e-05
349 4.1200087999016e-05
350 4.09729360342226e-05
351 4.07480160191653e-05
352 4.05252662645239e-05
353 4.03046082908531e-05
354 4.00859757903049e-05
355 3.98692991367966e-05
356 3.96546080310145e-05
357 3.94418316318479e-05
358 3.9230747482577e-05
359 3.90214692345126e-05
360 3.88140040286089e-05
361 3.86082071273108e-05
362 3.84040185323897e-05
363 3.82014258946128e-05
364 3.80004160831504e-05
365 3.78008868722191e-05
366 3.7602774320078e-05
367 3.74062276833342e-05
368 3.72109887578631e-05
369 3.70171282000342e-05
370 3.68246407163042e-05
371 3.66333818959674e-05
372 3.64434226156618e-05
373 3.62547136845137e-05
374 3.60671969588111e-05
375 3.5880859968529e-05
376 3.56956370808348e-05
377 3.55115577903575e-05
378 3.5328577517646e-05
379 3.51466870398554e-05
380 3.49658218468107e-05
381 3.47860093654617e-05
382 3.4607239598472e-05
383 3.44294022625036e-05
384 3.42525150713868e-05
385 3.40765686743794e-05
386 3.39016042332219e-05
387 3.37275731183695e-05
388 3.35544838421242e-05
389 3.33822705016473e-05
390 3.32109643039757e-05
391 3.3040625993408e-05
392 3.28710118751019e-05
393 3.27023006150284e-05
394 3.25343839691072e-05
395 3.23673284512438e-05
396 3.22011525355492e-05
397 3.20357260434889e-05
398 3.18710913447262e-05
399 3.1707308991713e-05
400 3.15443568297269e-05
401 3.13822074673453e-05
402 3.12208170072381e-05
403 3.10601389372778e-05
404 3.09003111453876e-05
405 3.07413166922288e-05
406 3.05830985070088e-05
407 3.04257137386799e-05
408 3.02691206286454e-05
409 3.01132837492446e-05
410 2.99582067384563e-05
411 2.98039390926874e-05
412 2.96505413430737e-05
413 2.94979238191218e-05
414 2.93459998985668e-05
415 2.91948744006731e-05
416 2.90445872295209e-05
417 2.88951299793894e-05
418 2.87464513561986e-05
419 2.85985259083077e-05
420 2.84513862851554e-05
421 2.83050910141469e-05
422 2.81595971856063e-05
423 2.80149045792655e-05
424 2.7871032543203e-05
425 2.77278867955033e-05
426 2.75855746068032e-05
427 2.74441221961297e-05
428 2.73034415911866e-05
429 2.71635916533342e-05
430 2.70244412021725e-05
431 2.6886231808021e-05
432 2.674881675091e-05
433 2.66122002301472e-05
434 2.64763790838174e-05
435 2.63413830907666e-05
436 2.62072262060542e-05
437 2.60738418305095e-05
438 2.59412580092544e-05
439 2.58095309462192e-05
440 2.56786247874174e-05
441 2.55485315392434e-05
442 2.54192397832753e-05
443 2.52908115712103e-05
444 2.51632120651379e-05
445 2.5036407841128e-05
446 2.49104344050011e-05
447 2.4785247596526e-05
448 2.46609680942811e-05
449 2.45374395575482e-05
450 2.44147327350674e-05
451 2.4292899219347e-05
452 2.41718930809043e-05
453 2.40516917813238e-05
454 2.39322321107238e-05
455 2.38136292765034e-05
456 2.36958807846577e-05
457 2.35788961404637e-05
458 2.34627656467978e-05
459 2.33473603188372e-05
460 2.32327872566884e-05
461 2.31191016055732e-05
462 2.30062356365579e-05
463 2.2894139625862e-05
464 2.27828734153945e-05
465 2.26723401723916e-05
466 2.25626002361423e-05
467 2.245367768694e-05
468 2.23456335390892e-05
469 2.22383644015167e-05
470 2.2131835784478e-05
471 2.20260540189088e-05
472 2.19210597549591e-05
473 2.18169214249997e-05
474 2.1713545777402e-05
475 2.16109536808062e-05
476 2.15090960935527e-05
477 2.14080289353547e-05
478 2.13077003863305e-05
479 2.12082120896184e-05
480 2.11094981636961e-05
481 2.10115533079147e-05
482 2.09142717366717e-05
483 2.08177803031617e-05
484 2.07220431178712e-05
485 2.06271100040567e-05
486 2.05329381870456e-05
487 2.04394474252467e-05
488 2.03467204755725e-05
489 2.0254728575253e-05
490 2.01634516727722e-05
491 2.00729102957098e-05
492 1.99831487677216e-05
493 1.98940562583516e-05
494 1.98057535669705e-05
495 1.97181536165658e-05
496 1.96312365119411e-05
497 1.95450566806699e-05
498 1.94595790503627e-05
499 1.93747595176319e-05
500 1.92906854579178e-05
501 1.92072554980882e-05
502 1.91245380207761e-05
503 1.90425482529122e-05
504 1.89612129375405e-05
505 1.88805899270506e-05
506 1.8800683903919e-05
507 1.87214525979584e-05
508 1.86428339077338e-05
509 1.8564869364468e-05
510 1.84875515145677e-05
511 1.84109109326869e-05
512 1.83349356746021e-05
513 1.82596297619853e-05
514 1.81850074412182e-05
515 1.81110101422632e-05
516 1.80376031906349e-05
517 1.7964852808916e-05
518 1.78927159950604e-05
519 1.78212676331668e-05
520 1.7750431318575e-05
521 1.7680145269594e-05
522 1.76104974229929e-05
523 1.75414864429513e-05
524 1.74730561539604e-05
525 1.74052488404186e-05
526 1.73380274830492e-05
527 1.72713715542727e-05
528 1.7205333975312e-05
529 1.71398904029729e-05
530 1.70750724635127e-05
531 1.70107220327509e-05
532 1.69469899304886e-05
533 1.68838463423526e-05
534 1.6821207289297e-05
535 1.67591629605113e-05
536 1.66976916560202e-05
537 1.66368033660547e-05
538 1.65763977619804e-05
539 1.65164723071598e-05
540 1.64571249214873e-05
541 1.63983131287182e-05
542 1.63400633752531e-05
543 1.62822771230253e-05
544 1.62249764059652e-05
545 1.61682864998625e-05
546 1.61120998782849e-05
547 1.60563480378073e-05
548 1.60011304188856e-05
549 1.59463985056618e-05
550 1.58921533923717e-05
551 1.58383819908181e-05
552 1.57851031588052e-05
553 1.57323103238127e-05
554 1.56799409012365e-05
555 1.5628094558906e-05
556 1.55766487921483e-05
557 1.55257209826232e-05
558 1.54751768803862e-05
559 1.54250764978769e-05
560 1.53754513831927e-05
561 1.53262819253541e-05
562 1.52775373294389e-05
563 1.52292734370008e-05
564 1.51814393163363e-05
565 1.51339710114939e-05
566 1.50869126827047e-05
567 1.504031573063e-05
568 1.49941431146772e-05
569 1.49483774336545e-05
570 1.49029939535694e-05
571 1.48579671304105e-05
572 1.48133512567483e-05
573 1.47691856113852e-05
574 1.47253577935658e-05
575 1.46819653181751e-05
576 1.46388768129668e-05
577 1.45962241120401e-05
578 1.45539493416891e-05
579 1.4512022062263e-05
580 1.44703990017092e-05
581 1.44292042421057e-05
582 1.43883632190978e-05
583 1.43479172862726e-05
584 1.43077103800238e-05
585 1.42678948478192e-05
586 1.42284378839008e-05
587 1.41893047072017e-05
588 1.41505434498868e-05
589 1.41120661893979e-05
590 1.40738840244126e-05
591 1.40360539404583e-05
592 1.39986189324759e-05
593 1.39614461005522e-05
594 1.39245201893345e-05
595 1.38879032860473e-05
596 1.38516051890747e-05
597 1.38156149418478e-05
598 1.37798438260006e-05
599 1.37444343053517e-05
600 1.37093278240741e-05
601 1.36744951859669e-05
602 1.36398951511296e-05
603 1.3605569328945e-05
604 1.35715730422703e-05
605 1.3537862486146e-05
606 1.35044088835912e-05
607 1.34711478807503e-05
608 1.34381287395513e-05
609 1.34054045375365e-05
610 1.33729551237138e-05
611 1.3340695197428e-05
612 1.33086501179491e-05
613 1.32769605372118e-05
614 1.32455295585032e-05
615 1.32142796331891e-05
616 1.31831710206143e-05
617 1.31523354625074e-05
618 1.31218985046644e-05
619 1.30916924874214e-05
620 1.30615367623932e-05
621 1.30315135038472e-05
622 1.3001866363993e-05
623 1.29725336535103e-05
624 1.29433952835711e-05
625 1.29142631735135e-05
626 1.2885313424249e-05
627 1.28567624386733e-05
628 1.28285370735171e-05
629 1.28003061377058e-05
630 1.27720697875588e-05
631 1.27441538069206e-05
632 1.27167342327539e-05
633 1.26894757599416e-05
634 1.26620663820631e-05
635 1.26347822302364e-05
636 1.26079581406202e-05
637 1.25815302780552e-05
638 1.25550978253841e-05
639 1.25284451399921e-05
640 1.25020694454747e-05
641 1.24762229418707e-05
642 1.24507167242882e-05
643 1.24249719775094e-05
644 1.23990788338801e-05
645 1.23736098771587e-05
646 1.23488144794237e-05
647 1.23241308145339e-05
648 1.22990050783756e-05
649 1.22737698262654e-05
650 1.22492208731728e-05
651 1.22254217487239e-05
652 1.22014522005998e-05
653 1.21768882905826e-05
654 1.21523651657185e-05
655 1.21288553103227e-05
656 1.21059880626717e-05
657 1.20825185661033e-05
658 1.20583216229875e-05
659 1.20345261933608e-05
660 1.20120474278451e-05
661 1.19899836477089e-05
662 1.19669134406308e-05
663 1.19430323692882e-05
664 1.19200684309817e-05
665 1.18986625281536e-05
666 1.187729585439e-05
667 1.18543970302198e-05
668 1.18307979661836e-05
669 1.18087315357229e-05
670 1.17885588934996e-05
671 1.17678760247486e-05
672 1.17449217427179e-05
673 1.17214535819699e-05
674 1.17004198756376e-05
675 1.1681542659403e-05
676 1.1661317415701e-05
677 1.16380675976302e-05
678 1.16147600408567e-05
679 1.15950923813557e-05
680 1.15777629545732e-05
681 1.15578031270047e-05
682 1.15337808210825e-05
683 1.15104978846148e-05
684 1.14925017129508e-05
685 1.14769087531386e-05
686 1.14568420528371e-05
687 1.14315320232095e-05
688 1.14082867241905e-05
689 1.13925809728244e-05
690 1.13791123581564e-05
691 1.13586918359943e-05
692 1.13312048313219e-05
693 1.130787144632e-05
694 1.12952023556545e-05
695 1.12846892577068e-05
696 1.12632318263195e-05
697 1.12323380037083e-05
698 1.12087025740948e-05
699 1.12003935655025e-05
700 1.11938800628764e-05
701 1.11707268857231e-05
702 1.11343543309772e-05
703 1.11100634114791e-05
704 1.11080537337216e-05
705 1.11075578459463e-05
706 1.10817111433903e-05
707 1.10368328449795e-05
708 1.1011248915338e-05
709 1.10185487969261e-05
710 1.10273588660448e-05
711 1.0997685230052e-05
712 1.09390550022681e-05
713 1.09107735113412e-05
714 1.0931718342988e-05
715 1.09556279497269e-05
716 1.09213841525957e-05
717 1.08405787599963e-05
718 1.08068488060553e-05
719 1.08479993770061e-05
720 1.08969236762846e-05
721 1.08589731340203e-05
722 1.07417649104491e-05
723 1.06973537015165e-05
724 1.0768059560462e-05
725 1.08587489151546e-05
726 1.08234854607758e-05
727 1.0646668414438e-05
728 1.05822647569198e-05
729 1.06967359130294e-05
730 1.08554471651701e-05
731 1.08436031780457e-05
732 1.05688952665162e-05
733 1.04687202480136e-05
734 1.06503524719415e-05
735 1.09121628213416e-05
736 1.09781053119207e-05
737 1.05440136231039e-05
738 1.03853394293196e-05
739 1.06813312896747e-05
740 1.10745031349779e-05
741 1.13354801811738e-05
742 1.0638779070149e-05
743 1.04082901621894e-05
744 1.0911375767364e-05
745 1.1408082876585e-05
746 1.20391577880241e-05
747 1.08674481609228e-05
748 1.06785945988008e-05
749 1.14383975287069e-05
750 1.1948315794541e-05
751 1.28946710837852e-05
752 1.08906629314731e-05
753 1.13528206142632e-05
754 1.17307479712281e-05
755 1.26766666568301e-05
756 1.25477608889923e-05
757 1.05491625674858e-05
758 1.18767307206724e-05
759 1.13640813950155e-05
760 1.27665449767278e-05
761 1.06361340606043e-05
762 1.0834097892598e-05
763 1.0921400068753e-05
764 1.15765431587533e-05
765 1.09059862491989e-05
766 1.00406520147089e-05
767 1.05808899562021e-05
768 1.06693925303603e-05
769 1.08655799699875e-05
770 9.83476819627072e-06
771 1.01774769909468e-05
772 1.02433344437713e-05
773 1.06195635041217e-05
774 9.92443428415868e-06
775 9.85706591194457e-06
776 1.00478975184615e-05
777 1.03278836576237e-05
778 1.00824213653539e-05
779 9.69432944586401e-06
780 9.90288502578096e-06
781 1.00785276231363e-05
782 1.01689345264333e-05
783 9.70490634344401e-06
784 9.74205693893282e-06
785 9.90324090111017e-06
786 1.01179588902767e-05
787 9.84671036619034e-06
788 9.60769589397614e-06
789 9.77157522896732e-06
790 9.9604423056121e-06
791 9.99697004999689e-06
792 9.60478276113008e-06
793 9.62493074041504e-06
794 9.79350127039424e-06
795 1.00048170352807e-05
796 9.77529131418464e-06
797 9.4982914049524e-06
798 9.6536812748127e-06
799 9.85242929374408e-06
800 9.96923810703265e-06
801 9.55703801608365e-06
802 9.48821388391252e-06
803 9.68429142034211e-06
804 9.94526220310377e-06
805 9.84840766449224e-06
806 9.40332094145901e-06
807 9.53106672696435e-06
808 9.74615858240213e-06
809 1.00467548946881e-05
810 9.67047213862315e-06
811 9.34290775234103e-06
812 9.60365331081903e-06
813 9.87688930820241e-06
814 1.01258671740823e-05
815 9.47485637681211e-06
816 9.39720580461767e-06
817 9.69980240483892e-06
818 1.0112798129569e-05
819 1.01337857927319e-05
820 9.31636107281975e-06
821 9.58031283460059e-06
822 9.83138237131698e-06
823 1.04624330248271e-05
824 9.99010934066291e-06
825 9.29118414561003e-06
826 9.84240764978495e-06
827 1.00581032143054e-05
828 1.07976917185226e-05
829 9.64461853492082e-06
830 9.53474679832311e-06
831 9.99245073529664e-06
832 1.05018243488075e-05
833 1.06975230131923e-05
834 9.29937684546189e-06
835 9.97500663046935e-06
836 9.9783763545247e-06
837 1.0990698974922e-05
838 9.87558229148533e-06
839 9.41759077122128e-06
840 9.99372515053665e-06
841 1.02833176001127e-05
842 1.0613197144238e-05
843 9.14464013490601e-06
844 9.7985331528605e-06
845 9.72640136609471e-06
846 1.05958824541119e-05
847 9.43670411501785e-06
848 9.25968903331409e-06
849 9.58456542576869e-06
850 1.00142290122562e-05
851 9.87072670710631e-06
852 8.91588749851735e-06
853 9.43622893956331e-06
854 9.49422405227551e-06
855 1.00057096297235e-05
856 8.97136605715332e-06
857 9.11197343711478e-06
858 9.26076394591746e-06
859 9.76638914806927e-06
860 9.28552431567198e-06
861 8.80706479833293e-06
862 9.16416644258788e-06
863 9.3875509605823e-06
864 9.5743693364625e-06
865 8.75186934479188e-06
866 8.99021353362173e-06
867 9.1254618723724e-06
868 9.5891573757001e-06
869 9.00165888140236e-06
870 8.73930736844386e-06
871 9.02792946533282e-06
872 9.33328563235136e-06
873 9.37655472554866e-06
874 8.64399034483654e-06
875 8.91312347306439e-06
876 9.06669992417619e-06
877 9.5329775540165e-06
878 8.89727286690345e-06
879 8.67924915581852e-06
880 8.97593304927113e-06
881 9.32198406644602e-06
882 9.37345170726189e-06
883 8.58038115580939e-06
884 8.88903782225725e-06
885 9.04390495293228e-06
886 9.60479750489185e-06
887 8.91643522216157e-06
888 8.63540908468963e-06
889 8.99588799541107e-06
890 9.35113876820992e-06
891 9.52907439355499e-06
892 8.55485910378206e-06
893 8.92820670372885e-06
894 9.05472626300252e-06
895 9.76720542311682e-06
896 9.0131773937685e-06
897 8.62214192665078e-06
898 9.07722276366485e-06
899 9.4044191030207e-06
900 9.72899102791303e-06
901 8.53595653538264e-06
902 9.01712490986029e-06
903 9.06900548613976e-06
904 9.92212865114084e-06
905 9.01212358428438e-06
906 8.64252103838226e-06
907 9.10883593974177e-06
908 9.46450730765491e-06
909 9.70161792679392e-06
910 8.44515017206504e-06
911 9.06806927147841e-06
912 9.02368114097385e-06
913 9.9116266980559e-06
914 8.72138998708749e-06
915 8.69077802789775e-06
916 8.94513777183192e-06
917 9.5166354050491e-06
918 9.25174422583552e-06
919 8.34323328646747e-06
920 8.94098718617897e-06
921 8.99735476522778e-06
922 9.58212364210453e-06
923 8.33080657081098e-06
924 8.71071629404696e-06
925 8.72698961273954e-06
926 9.46914955335387e-06
927 8.63556441821345e-06
928 8.35362386197858e-06
929 8.68282218036143e-06
930 9.06685364654436e-06
931 9.03573799249102e-06
932 8.13697005597191e-06
933 8.6255263163082e-06
934 8.68457938452138e-06
935 9.24854539618991e-06
936 8.2099685663195e-06
937 8.40930467660428e-06
938 8.51132887902395e-06
939 9.13012282310888e-06
940 8.54672127914569e-06
941 8.13333990379306e-06
942 8.50082468062396e-06
943 8.79267182085641e-06
944 8.95304036596656e-06
945 8.019677366633e-06
946 8.44347991701966e-06
947 8.49840012051573e-06
948 9.14118409056641e-06
949 8.21330420919253e-06
950 8.22282706991473e-06
951 8.42102787501631e-06
952 8.97099717178662e-06
953 8.66724769998939e-06
954 7.99043905530539e-06
955 8.46230110340684e-06
956 8.61933263252013e-06
957 9.10134380305294e-06
958 8.0047047266163e-06
959 8.36506482215782e-06
960 8.40901337895161e-06
961 9.17550591239547e-06
962 8.37228209071839e-06
963 8.09358978770547e-06
964 8.44826250556707e-06
965 8.8632434689373e-06
966 8.92544741581958e-06
967 7.91054941018388e-06
968 8.49725496721021e-06
969 8.49439494388093e-06
970 9.27405252681979e-06
971 8.04346636940068e-06
972 8.31116160071588e-06
973 8.37033867640002e-06
974 9.16548604124046e-06
975 8.47988475349837e-06
976 7.99690991826196e-06
977 8.44986192660713e-06
978 8.75438074743329e-06
979 8.97293813117983e-06
980 7.8238370839756e-06
981 8.45501717705588e-06
982 8.38105162515035e-06
983 9.20399361348245e-06
984 7.92131620741543e-06
985 8.24075533145674e-06
986 8.23634800184436e-06
987 9.05771686632306e-06
988 8.24382776443144e-06
989 7.92754331158108e-06
990 8.26258231967358e-06
991 8.68109960805441e-06
992 8.63121309535586e-06
993 7.70655283588439e-06
994 8.2760683142169e-06
995 8.30361897286025e-06
996 8.88756098049726e-06
997 7.68286729169176e-06
998 8.15080046479011e-06
999 8.07976187289228e-06
1000 8.89337801979195e-06
1001 7.86555115617205e-06
1002 7.90763862923427e-06
1003 8.03364585166833e-06
1004 8.66849156366811e-06
1005 8.19199172852336e-06
1006 7.66361577220209e-06
1007 8.07640381239594e-06
1008 8.33712535808218e-06
1009 8.5363186972387e-06
1010 7.53815945486735e-06
1011 8.07368098065808e-06
1012 8.05190040154002e-06
1013 8.74484574353573e-06
1014 7.60108749098265e-06
1015 7.94398234305049e-06
1016 7.91973070590757e-06
1017 8.71950160430401e-06
1018 7.85872298969537e-06
1019 7.72135477689062e-06
1020 7.94642686230418e-06
1021 8.48388913254894e-06
1022 8.24230816576232e-06
1023 7.52144527638166e-06
1024 8.0291799413601e-06
1025 8.16804704584229e-06
1026 8.60490542287096e-06
1027 7.45605860963394e-06
1028 8.02665169175043e-06
1029 7.92757535350574e-06
1030 8.78238241952545e-06
1031 7.5786090754093e-06
1032 7.87817240777144e-06
1033 7.85243626921783e-06
1034 8.70353276383184e-06
1035 7.87086345432897e-06
1036 7.6466207552528e-06
1037 7.91605560124253e-06
1038 8.43214792922709e-06
1039 8.24451816683336e-06
1040 7.44660371765349e-06
1041 8.00330218098111e-06
1042 8.10692847608152e-06
1043 8.56441657681728e-06
1044 7.35986116850995e-06
1045 7.99887191149651e-06
1046 7.85309033446424e-06
1047 8.7151595717927e-06
1048 7.41135519888303e-06
1049 7.86770366190126e-06
1050 7.7287280788596e-06
1051 8.66345123995416e-06
1052 7.58357453101155e-06
1053 7.65660168511317e-06
1054 7.72029866524804e-06
1055 8.45626155410173e-06
1056 7.83282465555146e-06
1057 7.43960212901129e-06
1058 7.77042780342185e-06
1059 8.17522927931691e-06
1060 8.09925397504685e-06
1061 7.27465082661638e-06
1062 7.81233088176236e-06
1063 7.89765930875319e-06
1064 8.31958378277875e-06
1065 7.19329194076579e-06
1066 7.79746851264917e-06
1067 7.67926509581685e-06
1068 8.4437972596163e-06
1069 7.20613148175175e-06
1070 7.70927015025791e-06
1071 7.54893830290371e-06
1072 8.45004626626178e-06
1073 7.31118548635834e-06
1074 7.56282597080826e-06
1075 7.51005525501114e-06
1076 8.34597010168636e-06
1077 7.49592666515753e-06
1078 7.39123336046532e-06
1079 7.54371119526809e-06
1080 8.1612148008503e-06
1081 7.73546151044968e-06
1082 7.23137254965422e-06
1083 7.61383202885213e-06
1084 7.93703233981091e-06
1085 7.99031954201723e-06
1086 7.11382552864848e-06
1087 7.6760277281096e-06
1088 7.71757811790508e-06
1089 8.21133150985531e-06
1090 7.05757282837283e-06
1091 7.69205517414662e-06
1092 7.54010936176996e-06
1093 8.35437161583741e-06
1094 7.06913483838889e-06
1095 7.64445672274405e-06
1096 7.42654530583309e-06
1097 8.39684905429294e-06
1098 7.14404084334319e-06
1099 7.54046412154707e-06
1100 7.37966287900349e-06
1101 8.34264448101862e-06
1102 7.27005118505986e-06
1103 7.40299639900854e-06
1104 7.38668012090216e-06
1105 8.2151474103398e-06
1106 7.43007415060504e-06
1107 7.25773373488892e-06
1108 7.42641438833402e-06
1109 8.04431030232422e-06
1110 7.60392252985298e-06
1111 7.12445246620064e-06
1112 7.47623609598236e-06
1113 7.85796176216991e-06
1114 7.77167431920134e-06
1115 7.01465364727483e-06
1116 7.51703440116103e-06
1117 7.67700477410926e-06
1118 7.91585176429521e-06
1119 6.93268878748654e-06
1120 7.53659111651928e-06
1121 7.51516113695061e-06
1122 8.02525340581894e-06
1123 6.87822999267951e-06
1124 7.53006207077078e-06
1125 7.37911260983992e-06
1126 8.09562098069705e-06
1127 6.84851021581778e-06
1128 7.49905493080405e-06
1129 7.27069355122012e-06
1130 8.12896721136269e-06
1131 6.84006237605672e-06
1132 7.44889420545292e-06
1133 7.18859400805627e-06
1134 8.13142940714329e-06
1135 6.84986250654163e-06
1136 7.38615513995455e-06
1137 7.13026153675855e-06
1138 8.11031571856802e-06
1139 6.87553113110084e-06
1140 7.31669786091516e-06
1141 7.09291084532992e-06
1142 8.07270042990638e-06
1143 6.91546456188519e-06
1144 7.24526816853199e-06
1145 7.07399104982187e-06
1146 8.02392760945736e-06
1147 6.96790716503415e-06
1148 7.17502208757992e-06
1149 7.07091948726202e-06
1150 7.96790555490645e-06
1151 7.03069596674766e-06
1152 7.10827972838501e-06
1153 7.08085899603361e-06
1154 7.90739960265796e-06
1155 7.10036409756754e-06
1156 7.04623921521375e-06
1157 7.09981518198788e-06
1158 7.84456315727766e-06
1159 7.17207120004559e-06
1160 6.98988881353557e-06
1161 7.12315627282578e-06
1162 7.78149613367418e-06
1163 7.23935506385942e-06
1164 6.93956103958726e-06
1165 7.14536191637194e-06
1166 7.72029660822682e-06
1167 7.29554508449581e-06
1168 6.89516652485622e-06
1169 7.16113616761049e-06
1170 7.66296347975981e-06
1171 7.33444290368368e-06
1172 6.85638665487431e-06
1173 7.16625780228242e-06
1174 7.61134475624203e-06
1175 7.35178371424894e-06
1176 6.82280868957719e-06
1177 7.15852979382703e-06
1178 7.56690748104916e-06
1179 7.34640922583907e-06
1180 6.79425271243872e-06
1181 7.13784854866617e-06
1182 7.53062410652205e-06
1183 7.31984084811899e-06
1184 6.77066674370508e-06
1185 7.10598330400103e-06
1186 7.50308765162799e-06
1187 7.27593742588795e-06
1188 6.75240388403608e-06
1189 7.06619872303804e-06
1190 7.4847234223796e-06
1191 7.21985326190833e-06
1192 6.73999539912984e-06
1193 7.02195943347306e-06
1194 7.47577321646986e-06
1195 7.15653115612724e-06
1196 6.73384942828648e-06
1197 6.97649332082051e-06
1198 7.47632227948714e-06
1199 7.08994483300529e-06
1200 6.73462864142493e-06
1201 6.93205478441428e-06
1202 7.48644017356526e-06
1203 7.02246917949623e-06
1204 6.74260815003436e-06
1205 6.88968425421876e-06
1206 7.50588644748973e-06
1207 6.95496779190563e-06
1208 6.75815660855505e-06
1209 6.84942951778567e-06
1210 7.53420414945083e-06
1211 6.88671899773396e-06
1212 6.78130583864345e-06
1213 6.81040269157052e-06
1214 7.57010599983232e-06
1215 6.81635855048768e-06
1216 6.81199151131295e-06
1217 6.77165563445215e-06
1218 7.61145554939446e-06
1219 6.74224569063853e-06
1220 6.84938691364323e-06
1221 6.732429781664e-06
1222 7.65473403774308e-06
1223 6.66376677660452e-06
1224 6.89204772896801e-06
1225 6.6932397331243e-06
1226 7.69463207817012e-06
1227 6.58158718991331e-06
1228 6.93720740585491e-06
1229 6.65685223211199e-06
1230 7.72432490947494e-06
1231 6.499163340834e-06
1232 6.98127422893435e-06
1233 6.62881647528479e-06
1234 7.73567535183872e-06
1235 6.42243931103792e-06
1236 7.01891108434438e-06
1237 6.61720837413782e-06
1238 7.71952967326683e-06
1239 6.35943556126506e-06
1240 7.04365669434992e-06
1241 6.63161296188264e-06
1242 7.66728707013442e-06
1243 6.31868777745126e-06
1244 7.04796707395872e-06
1245 6.68102003942295e-06
1246 7.57239844340063e-06
1247 6.30794247946653e-06
1248 7.02446705247439e-06
1249 6.77067523469077e-06
1250 7.43157616600598e-06
1251 6.33155709195421e-06
1252 6.96703021674239e-06
1253 6.89889419014378e-06
1254 7.24677811980712e-06
1255 6.38907787120502e-06
1256 6.87412977740109e-06
1257 7.05518056065557e-06
1258 7.02745147407313e-06
1259 6.47474479009702e-06
1260 6.75228644198e-06
1261 7.21936192960015e-06
1262 6.79127738489882e-06
1263 6.57732955033907e-06
1264 6.61834834403408e-06
1265 7.36539505297173e-06
1266 6.5627623087039e-06
1267 6.68244389601114e-06
1268 6.49872688285313e-06
1269 7.46773801729717e-06
1270 6.36804768028298e-06
1271 6.77503998858242e-06
1272 6.42317107235613e-06
1273 7.50673752492048e-06
1274 6.22929703553154e-06
1275 6.84127087779984e-06
1276 6.4168075653015e-06
1277 7.47166042103231e-06
1278 6.16039237044674e-06
1279 6.86915711867186e-06
1280 6.49412250197656e-06
1281 7.36002146695114e-06
1282 6.16604912195839e-06
1283 6.8478985824072e-06
1284 6.65487977613566e-06
1285 7.17476749212409e-06
1286 6.24185589614967e-06
1287 6.77024565476358e-06
1288 6.88105497381741e-06
1289 6.92773581167927e-06
1290 6.37380541235189e-06
1291 6.63973038328436e-06
1292 7.13338050317702e-06
1293 6.64605782851879e-06
1294 6.53685346563293e-06
1295 6.48309096007438e-06
1296 7.35298570297971e-06
1297 6.37516904689051e-06
1298 6.69555000598621e-06
1299 6.35479435118214e-06
1300 7.47406872925183e-06
1301 6.17054941187689e-06
1302 6.80813668552105e-06
1303 6.322042644058e-06
1304 7.44686138887118e-06
1305 6.07756433090856e-06
1306 6.83615832031137e-06
1307 6.43121872734298e-06
1308 7.2583627899192e-06
1309 6.11188777099869e-06
1310 6.75707001107639e-06
1311 6.67527732112205e-06
1312 6.94068982909357e-06
1313 6.25183422187092e-06
1314 6.58157454935804e-06
1315 6.9837892393565e-06
1316 6.56699808665451e-06
1317 6.44509303882046e-06
1318 6.37019751437151e-06
1319 7.24463852463941e-06
1320 6.23389046072731e-06
1321 6.62443618892894e-06
1322 6.22510211911731e-06
1323 7.35276268493124e-06
1324 6.02825698514664e-06
1325 6.72591501427178e-06
1326 6.24225843459669e-06
1327 7.2553354071303e-06
1328 5.99315858806904e-06
1329 6.70547602510396e-06
1330 6.45275914834542e-06
1331 6.9715096682188e-06
1332 6.11463099886578e-06
1333 6.55770155688629e-06
1334 6.79439905582058e-06
1335 6.58314839618868e-06
1336 6.33116640358367e-06
1337 6.33767333724222e-06
1338 7.12930835256032e-06
1339 6.21149720192449e-06
1340 6.55447285069499e-06
1341 6.16618010695902e-06
1342 7.30341165322557e-06
1343 5.97662410761757e-06
1344 6.69215817339364e-06
1345 6.17870758468086e-06
1346 7.22046712731128e-06
1347 5.94752719251801e-06
1348 6.67324760073029e-06
1349 6.4317277139736e-06
1350 6.89168246381655e-06
1351 6.10961690838963e-06
1352 6.48592699903361e-06
1353 6.83645015620016e-06
1354 6.43922424625742e-06
1355 6.37098825961857e-06
1356 6.22480044398799e-06
1357 7.18276379352289e-06
1358 6.04956679239876e-06
1359 6.59800927849119e-06
1360 6.08351209763214e-06
1361 7.26054894339967e-06
1362 5.88154765956972e-06
1363 6.66491447631756e-06
1364 6.22563047159019e-06
1365 6.99862029840403e-06
1366 5.97616979902682e-06
1367 6.51390717010258e-06
1368 6.62486741909163e-06
1369 6.51580766142246e-06
1370 6.24023789086436e-06
1371 6.22404785133313e-06
1372 7.04278903285172e-06
1373 6.05265770659003e-06
1374 6.5042950723182e-06
1375 6.02279770589575e-06
1376 7.1920579696183e-06
1377 5.82918594105308e-06
1378 6.60391235740576e-06
1379 6.13276276340002e-06
1380 6.95313686094323e-06
1381 5.91298430308029e-06
1382 6.46014005667439e-06
1383 6.5494204530836e-06
1384 6.45461420756277e-06
1385 6.19701469162237e-06
1386 6.1598685228148e-06
1387 6.99817558569293e-06
1388 5.97978010130618e-06
1389 6.47888298388466e-06
1390 5.96671710084706e-06
1391 7.14027756387736e-06
1392 5.7812239102617e-06
1393 6.56617671523918e-06
1394 6.13175174812852e-06
1395 6.84894909763756e-06
1396 5.9205362745729e-06
1397 6.3788557760347e-06
1398 6.61426880910199e-06
1399 6.30337242935042e-06
1400 6.24899203138796e-06
1401 6.05472370551752e-06
1402 7.05314959326131e-06
1403 5.8532233939701e-06
1404 6.51534151430155e-06
1405 5.93357196265742e-06
1406 7.07051649140311e-06
1407 5.76121128226248e-06
1408 6.50920723188619e-06
1409 6.24901738888184e-06
1410 6.62061308087658e-06
1411 6.01605988137521e-06
1412 6.2111796594877e-06
1413 6.80139627817766e-06
1414 6.02962234630411e-06
1415 6.36929800634789e-06
1416 5.90714870796205e-06
1417 7.0901207411822e-06
1418 5.71519949410515e-06
1419 6.52001663326018e-06
1420 6.00392215766021e-06
1421 6.82047382660755e-06
1422 5.83832317602173e-06
1423 6.31784901727883e-06
1424 6.52921893795622e-06
1425 6.20238913384696e-06
1426 6.20594669342722e-06
1427 5.94520336250071e-06
1428 6.99356090017034e-06
1429 5.73119116964449e-06
1430 6.46571228912762e-06
1431 5.86945780689518e-06
1432 6.90988963913952e-06
1433 5.72440404411623e-06
1434 6.3670307746122e-06
1435 6.31595507982752e-06
1436 6.3304835062894e-06
1437 6.07377092443073e-06
1438 5.9896861195341e-06
1439 6.87824727307884e-06
1440 5.76596677248631e-06
1441 6.40444954669306e-06
1442 5.80259711391307e-06
1443 6.94323747119086e-06
1444 5.65865450496972e-06
1445 6.38752085890815e-06
1446 6.17668614744105e-06
1447 6.41147662072683e-06
1448 5.98729584311286e-06
1449 6.0197131048767e-06
1450 6.79336768527605e-06
1451 5.78915902860189e-06
1452 6.36202368831107e-06
1453 5.76668691820714e-06
1454 6.94889645735941e-06
1455 5.62014460925297e-06
1456 6.38853769530101e-06
1457 6.10314032023496e-06
1458 6.43258685162351e-06
1459 5.94430444067484e-06
1460 6.01460611093785e-06
1461 6.75160918994777e-06
1462 5.7726010442849e-06
1463 6.34051095360633e-06
1464 5.73549536042606e-06
1465 6.92479485842057e-06
1466 5.59414054102092e-06
1467 6.36137045262331e-06
1468 6.08869740936768e-06
1469 6.37449335716411e-06
1470 5.94170618306578e-06
1471 5.9588788197118e-06
1472 6.7519466835364e-06
1473 5.70526539078742e-06
1474 6.3346564669331e-06
1475 5.70468382932177e-06
1476 6.86187084308898e-06
1477 5.58303723074971e-06
1478 6.29913191119158e-06
1479 6.13563049789434e-06
1480 6.23828903467682e-06
1481 5.98001086338229e-06
1482 5.85939353925369e-06
1483 6.7852317116035e-06
1484 5.60736337540391e-06
1485 6.33811419170627e-06
1486 5.69638506142667e-06
1487 6.75398939176119e-06
1488 5.60533825932907e-06
1489 6.20142311191785e-06
1490 6.24997786147219e-06
1491 6.04517540114813e-06
1492 6.05895012206759e-06
1493 5.74115111007245e-06
1494 6.82972536480975e-06
1495 5.51667444703696e-06
1496 6.33676941674821e-06
1497 5.74663297747691e-06
1498 6.58663260111325e-06
1499 5.68205599194016e-06
1500 6.06242143419422e-06
1501 6.42841189879562e-06
1502 5.81798560617131e-06
1503 6.16671887776477e-06
1504 5.63672801234816e-06
1505 6.83732668882442e-06
1506 5.47371596582025e-06
1507 6.29598393686592e-06
1508 5.89296228703517e-06
1509 6.33124188542666e-06
1510 5.82399313486803e-06
1511 5.87258748296904e-06
1512 6.63523637811636e-06
1513 5.59041459524678e-06
1514 6.26502989931055e-06
1515 5.6008132496288e-06
1516 6.72780850941024e-06
1517 5.5220423682556e-06
1518 6.16348811988132e-06
1519 6.1545779956873e-06
1520 5.98229272785744e-06
1521 6.01384142839834e-06
1522 5.66074135122108e-06
1523 6.77550320560272e-06
1524 5.43628535609741e-06
1525 6.28442122163619e-06
1526 5.71794762294076e-06
1527 6.43399153599233e-06
1528 5.68929341682178e-06
1529 5.91638213265355e-06
1530 6.47992806790398e-06
1531 5.61883979965927e-06
1532 6.18745160885226e-06
1533 5.54028296129161e-06
1534 6.71587855283207e-06
1535 5.44790577095e-06
1536 6.15243580526226e-06
1537 6.0347061836552e-06
1538 5.98759938252869e-06
1539 5.94227295280803e-06
1540 5.63080140381089e-06
1541 6.71159223042395e-06
1542 5.39403218802192e-06
1543 6.2427949103494e-06
1544 5.66315942052142e-06
1545 6.38190551782714e-06
1546 5.65632610260991e-06
1547 5.86064798113739e-06
1548 6.4519116298456e-06
1549 5.54954616038827e-06
1550 6.16750781823328e-06
1551 5.4973145982018e-06
1552 6.64719379361856e-06
1553 5.43313480250163e-06
1554 6.08140161872939e-06
1555 6.07064263569157e-06
1556 5.85588023227501e-06
1557 5.97257467660484e-06
1558 5.54345271552847e-06
1559 6.69870238567682e-06
1560 5.34339893398794e-06
1561 6.19945664581678e-06
1562 5.71964040219086e-06
1563 6.20417430763354e-06
1564 5.72753488725652e-06
1565 5.71859347786585e-06
1566 6.54246818143633e-06
1567 5.4074116988545e-06
1568 6.1893803433577e-06
1569 5.50005784738516e-06
1570 6.48566198080403e-06
1571 5.50196147308668e-06
1572 5.91933960691904e-06
1573 6.2552974942065e-06
1574 5.59714428849389e-06
1575 6.07391731222151e-06
1576 5.43751261261605e-06
1577 6.63281055679477e-06
1578 5.34663240081557e-06
1579 6.07281170061924e-06
1580 5.93563314410517e-06
1581 5.85516719464607e-06
1582 5.89588343835601e-06
1583 5.5000657397386e-06
1584 6.63149611579428e-06
1585 5.28633931651257e-06
1586 6.14860375947046e-06
1587 5.66137263291466e-06
1588 6.11994010846217e-06
1589 5.69833275143594e-06
1590 5.63300024580826e-06
1591 6.50801046653271e-06
1592 5.32102684758229e-06
1593 6.14744048199611e-06
1594 5.47414987472905e-06
1595 6.34328466109224e-06
1596 5.51534534842801e-06
1597 5.78526601557883e-06
1598 6.30739298124183e-06
1599 5.43293595534067e-06
1600 6.0847704510536e-06
1601 5.38220247214838e-06
1602 6.49638883576387e-06
1603 5.36931063876978e-06
1604 5.9211011489424e-06
1605 6.07639381122738e-06
1606 5.5942143566412e-06
1607 5.98029635945352e-06
1608 5.37147471035837e-06
1609 6.56899205164763e-06
1610 5.2711455840182e-06
1611 6.02033012242487e-06
1612 5.85264725216916e-06
1613 5.7745142685306e-06
1614 5.85386611895444e-06
1615 5.4167097154334e-06
1616 6.56673024668208e-06
1617 5.22155889015607e-06
1618 6.07670898489232e-06
1619 5.66032088933355e-06
1620 5.94755311578155e-06
1621 5.72202947246581e-06
1622 5.49117421755341e-06
1623 6.50662076484565e-06
1624 5.21359567429158e-06
1625 6.09420032660069e-06
1626 5.51014200311783e-06
1627 6.0962572083767e-06
1628 5.59740450256641e-06
1629 5.57415255997284e-06
1630 6.41042746618581e-06
1631 5.23672152041854e-06
1632 6.0827577836875e-06
1633 5.40240635693578e-06
1634 6.2137574277088e-06
1635 5.48769992292364e-06
1636 5.65263052720866e-06
1637 6.29792010542474e-06
1638 5.27940430039564e-06
1639 6.05240312978594e-06
1640 5.33068789287938e-06
1641 6.29992753253816e-06
1642 5.39603950677758e-06
1643 5.72030890566566e-06
1644 6.18407297370283e-06
1645 5.33147481185381e-06
1646 6.01222955154412e-06
1647 5.28663820631436e-06
1648 6.35858966457903e-06
1649 5.32248041196226e-06
1650 5.77452368588638e-06
1651 6.07829109977587e-06
1652 5.38436356478655e-06
1653 5.96879313796705e-06
1654 5.26147471369143e-06
1655 6.39436956140216e-06
1656 5.26511754195269e-06
1657 5.81456837167593e-06
1658 5.98615613878195e-06
1659 5.43125658314381e-06
1660 5.92685774236656e-06
1661 5.24749769859056e-06
1662 6.41208510776536e-06
1663 5.22134020641829e-06
1664 5.84071528564323e-06
1665 5.91062907950857e-06
1666 5.46675836066157e-06
1667 5.88954030256161e-06
1668 5.23834352250674e-06
1669 6.41594907957455e-06
1670 5.18842955443688e-06
1671 5.85380088935494e-06
1672 5.85279635867408e-06
1673 5.48742811901093e-06
1674 5.8592066576324e-06
1675 5.22961326510085e-06
1676 6.41002646162292e-06
1677 5.16420242568927e-06
1678 5.85539520159273e-06
1679 5.81304897284696e-06
1680 5.49188455778449e-06
1681 5.83725239344801e-06
1682 5.21854453339188e-06
1683 6.39747063324592e-06
1684 5.14717900657047e-06
1685 5.84692604022052e-06
1686 5.79129691846703e-06
1687 5.47995852162586e-06
1688 5.82474361010554e-06
1689 5.20432181261299e-06
1690 6.38035460109876e-06
1691 5.1369705733606e-06
1692 5.82957638428638e-06
1693 5.7874126992985e-06
1694 5.45253401984525e-06
1695 5.82195807297126e-06
1696 5.18732264964683e-06
1697 6.35922124914146e-06
1698 5.13398730639381e-06
1699 5.80346888412464e-06
1700 5.80115860326913e-06
1701 5.41050040059332e-06
1702 5.82812374005925e-06
1703 5.1686819588781e-06
1704 6.33222842250092e-06
1705 5.13931800227851e-06
1706 5.76759504511415e-06
1707 5.83219698047088e-06
1708 5.3551874552582e-06
1709 5.84191152697144e-06
1710 5.1508739780104e-06
1711 6.29617125502691e-06
1712 5.1547838513244e-06
1713 5.72022548794848e-06
1714 5.87961217934918e-06
1715 5.28853178405342e-06
1716 5.86061894303214e-06
1717 5.13745996144621e-06
1718 6.24617316669074e-06
1719 5.18252637959904e-06
1720 5.65932716156681e-06
1721 5.94148428945118e-06
1722 5.21394558727906e-06
1723 5.88064403039823e-06
1724 5.13357853115792e-06
1725 6.17716457584905e-06
1726 5.22461037633093e-06
1727 5.58359183955304e-06
1728 6.01425948421053e-06
1729 5.13680411096118e-06
1730 5.89753706314866e-06
1731 5.14571655862994e-06
1732 6.08456275497105e-06
1733 5.28257250387298e-06
1734 5.49365639557209e-06
1735 6.09210357360723e-06
1736 5.064439186242e-06
1737 5.90574805592325e-06
1738 5.18111334812943e-06
1739 5.96588125079123e-06
1740 5.35683078162208e-06
1741 5.39256244991293e-06
1742 6.16652200147172e-06
1743 5.00571141159867e-06
1744 5.89904104053574e-06
1745 5.24638865684324e-06
1746 5.82076995447878e-06
1747 5.44547178726873e-06
1748 5.28579035474763e-06
1749 6.22607714806378e-06
1750 4.96978890573985e-06
1751 5.87051702183317e-06
1752 5.34633931970063e-06
1753 5.65257450979573e-06
1754 5.54442532951782e-06
1755 5.18291123796644e-06
1756 6.25676701426769e-06
1757 4.96549031048232e-06
1758 5.81300232660453e-06
1759 5.48156797286481e-06
1760 5.4686729171749e-06
1761 5.64588990670245e-06
1762 5.09708806362141e-06
1763 6.24281719652231e-06
1764 4.99968380118787e-06
1765 5.72084538497108e-06
1766 5.64639986055937e-06
1767 5.28214207395905e-06
1768 5.73860651265079e-06
1769 5.04559722713793e-06
1770 6.16999714919686e-06
1771 5.07566799967663e-06
1772 5.59261384935894e-06
1773 5.82625475065157e-06
1774 5.11183663665093e-06
1775 5.8081975140567e-06
1776 5.04730831529798e-06
1777 6.03038481905571e-06
1778 5.19095326545482e-06
1779 5.43462395441452e-06
1780 5.99708791337861e-06
1781 4.98034058082908e-06
1782 5.83936799358753e-06
1783 5.11826651283798e-06
1784 5.82771383239589e-06
1785 5.33626347021254e-06
1786 5.26392727451253e-06
1787 6.12736787353185e-06
1788 4.90894978177892e-06
1789 5.81708055236163e-06
1790 5.26464955008521e-06
1791 5.57942769541242e-06
1792 5.49495061719085e-06
1793 5.10858994573482e-06
1794 6.18314525535624e-06
1795 4.91264234092625e-06
1796 5.73070260578845e-06
1797 5.47638599268652e-06
1798 5.31697868932923e-06
1799 5.64363574717675e-06
1800 5.00529914049253e-06
1801 6.13654108283512e-06
1802 4.99565105727129e-06
1803 5.57983438120857e-06
1804 5.72319780900443e-06
1805 5.08172999591494e-06
1806 5.7542446558756e-06
1807 4.99147829735591e-06
1808 5.97477534958202e-06
1809 5.147335174982e-06
1810 5.37883459372779e-06
1811 5.95357193855506e-06
1812 4.91676482639036e-06
1813 5.79663404742803e-06
1814 5.0931017376854e-06
1815 5.71216384059881e-06
1816 5.34156227693927e-06
1817 5.16510860748554e-06
1818 6.10369734355132e-06
1819 4.85716207876408e-06
1820 5.74657895846542e-06
1821 5.30900410566915e-06
1822 5.39540928912174e-06
1823 5.53915625900459e-06
1824 4.99806928999647e-06
1825 6.11587450327633e-06
1826 4.91725318241265e-06
1827 5.59704047731202e-06
1828 5.5990056671007e-06
1829 5.09547349825112e-06
1830 5.69405389327926e-06
1831 4.94395734307318e-06
1832 5.96327277113673e-06
1833 5.08287525669004e-06
1834 5.36920810745301e-06
1835 5.88385327571217e-06
1836 4.88671508769301e-06
1837 5.75960084425731e-06
1838 5.04786238408883e-06
1839 5.66791392309085e-06
1840 5.3109792315098e-06
1841 5.12084718806705e-06
1842 6.06363097199747e-06
1843 4.82270963075848e-06
1844 5.70046484860143e-06
1845 5.30499807371854e-06
1846 5.30557345079785e-06
1847 5.53834133398823e-06
1848 4.9424922181629e-06
1849 6.05407663689661e-06
1850 4.91690404480494e-06
1851 5.51187867259273e-06
1852 5.64392851210016e-06
1853 4.98520880487519e-06
1854 5.69540009465186e-06
1855 4.92688157471832e-06
1856 5.83190464098493e-06
1857 5.13446982530752e-06
1858 5.23902275340049e-06
1859 5.93767348711083e-06
1860 4.80841091032858e-06
1861 5.71981365737884e-06
1862 5.11737894903774e-06
1863 5.45939121554539e-06
1864 5.39985841285784e-06
1865 4.98445088226163e-06
1866 6.0478183629975e-06
1867 4.82750139951804e-06
1868 5.57996423111717e-06
1869 5.4645112541607e-06
1870 5.07052123133178e-06
1871 5.61842159463311e-06
1872 4.87998222808983e-06
1873 5.89817653118274e-06
1874 5.02266983382071e-06
1875 5.30588523339759e-06
1876 5.82397300430415e-06
1877 4.81382854822243e-06
1878 5.70138498368067e-06
1879 5.01598764124367e-06
1880 5.5312009119568e-06
1881 5.30855308866762e-06
1882 5.00903357192328e-06
1883 6.01053923610095e-06
1884 4.78171874007671e-06
1885 5.59382635501038e-06
1886 5.36471626411128e-06
1887 5.10192630187589e-06
1888 5.56515775329558e-06
1889 4.85560245877537e-06
1890 5.90069871231336e-06
1891 4.96774916491916e-06
1892 5.31550270466141e-06
1893 5.76258131879115e-06
1894 4.80242046485557e-06
1895 5.6764867775172e-06
1896 4.97353774342457e-06
1897 5.52435308209809e-06
1898 5.27244916881386e-06
1899 4.993616485649e-06
1900 5.97880900388148e-06
1901 4.75790662157749e-06
1902 5.57034868897688e-06
1903 5.3418948686712e-06
1904 5.06746758688337e-06
1905 5.54789574458425e-06
1906 4.83143118046314e-06
1907 5.85735597402604e-06
1908 4.96239976754964e-06
1909 5.26992561411532e-06
1910 5.76140546915127e-06
1911 4.76417168826515e-06
1912 5.65330821977739e-06
1913 4.97971102930705e-06
1914 5.44293618043312e-06
1915 5.29045693120622e-06
1916 4.93612379592889e-06
1917 5.95531380787051e-06
1918 4.75452393011722e-06
1919 5.5105130751798e-06
1920 5.39262678245223e-06
1921 4.97185836412228e-06
1922 5.56240271443187e-06
1923 4.81275942831871e-06
1924 5.76325414414214e-06
1925 5.00771156275448e-06
1926 5.16990912835524e-06
1927 5.80867006760855e-06
1928 4.71294655568499e-06
1929 5.62053966746134e-06
1930 5.04463385908593e-06
1931 5.28854561387959e-06
1932 5.35636826626273e-06
1933 4.85140265737982e-06
1934 5.9143546549123e-06
1935 4.78816389382075e-06
1936 5.40212009880747e-06
1937 5.51246941427053e-06
1938 4.83683822771752e-06
1939 5.58999753330625e-06
1940 4.83000548534562e-06
1941 5.59776856290384e-06
1942 5.10893304550564e-06
1943 5.02202991548018e-06
1944 5.87029595644495e-06
1945 4.68403953135521e-06
1946 5.55080764463867e-06
1947 5.18863894871657e-06
1948 5.07226843460273e-06
1949 5.45321589662251e-06
1950 4.7776281100198e-06
1951 5.80781403014541e-06
1952 4.88397838616095e-06
1953 5.23060948864895e-06
1954 5.67776693127797e-06
1955 4.70695933074694e-06
1956 5.59242980635588e-06
1957 4.93074085206047e-06
1958 5.34365928794855e-06
1959 5.25965615771895e-06
1960 4.85469449706954e-06
1961 5.87869648605022e-06
1962 4.72485331215466e-06
1963 5.40600350262821e-06
1964 5.41502463846655e-06
1965 4.83399228823345e-06
1966 5.5407918750916e-06
1967 4.78203630471796e-06
1968 5.58262530514497e-06
1969 5.05507988179943e-06
1970 5.00246279955263e-06
1971 5.81984076930553e-06
1972 4.65045253683627e-06
1973 5.51477195287475e-06
1974 5.15407550327751e-06
1975 5.02360834619964e-06
1976 5.42513731893735e-06
1977 4.74211485013143e-06
1978 5.74641631345685e-06
1979 4.87326369480456e-06
1980 5.16601029687536e-06
1981 5.66989070494017e-06
1982 4.65807312366451e-06
1983 5.5523877104946e-06
1984 4.94179477250611e-06
1985 5.23167843091699e-06
1986 5.27654931659072e-06
1987 4.78538336068368e-06
1988 5.82104833179997e-06
1989 4.73461525807295e-06
1990 5.30839841239583e-06
1991 5.47521541083285e-06
1992 4.73072511297801e-06
1993 5.53078150211661e-06
1994 4.79765598981885e-06
1995 5.42226322242101e-06
1996 5.11995546048638e-06
1997 4.87739229448891e-06
1998 5.81414000100366e-06
1999 4.64626131524426e-06
};
\addlegendentry{Train}
\addplot [semithick, black]
table {%
0 0.020474249497056
1 0.0198213085532188
2 0.0191966816782951
3 0.0185853652656078
4 0.0179698001593351
5 0.0173383913934231
6 0.0166820697486401
7 0.0159917417913675
8 0.0152621334418654
9 0.0144942197948694
10 0.0136984512209892
11 0.0128902522847056
12 0.0120851937681437
13 0.0113010704517365
14 0.0105591211467981
15 0.009880306199193
16 0.00927885994315147
17 0.00876185670495033
18 0.00833152607083321
19 0.00798497721552849
20 0.00771411508321762
21 0.00750731490552425
22 0.00735100684687495
23 0.00723260641098022
24 0.00714202271774411
25 0.00707175675779581
26 0.00701644038781524
27 0.00697227520868182
28 0.00693656085059047
29 0.00690735783427954
30 0.00688325334340334
31 0.00686320010572672
32 0.0068464046344161
33 0.00683225831016898
34 0.00682028336450458
35 0.00681010261178017
36 0.00680140731856227
37 0.0067939511500299
38 0.00678753200918436
39 0.00678197992965579
40 0.00677715754136443
41 0.00677294656634331
42 0.00676925014704466
43 0.00676598260179162
44 0.00676305964589119
45 0.00676038814708591
46 0.00675775948911905
47 0.00675423163920641
48 0.0067419339902699
49 0.00667765643447638
50 0.00647336849942803
51 0.00591946532949805
52 0.00492316717281938
53 0.00436262646690011
54 0.00403013965114951
55 0.0037663436960429
56 0.00353518128395081
57 0.00333744450472295
58 0.00316350325010717
59 0.00300490017980337
60 0.00285872304812074
61 0.00272224238142371
62 0.00259374966844916
63 0.0024721659719944
64 0.00235665426589549
65 0.00224656285718083
66 0.00214136531576514
67 0.00204066419973969
68 0.00194418698083609
69 0.00185176730155945
70 0.00176331482362002
71 0.00167882174719125
72 0.00159831251949072
73 0.00152179773431271
74 0.001449239323847
75 0.00138052576221526
76 0.00131549371872097
77 0.00125396507792175
78 0.00119578698650002
79 0.0011408660793677
80 0.00108916591852903
81 0.00104066357016563
82 0.00099530688021332
83 0.000953002134338021
84 0.000913613475859165
85 0.000876983976922929
86 0.000842946406919509
87 0.000811329984571785
88 0.000781970738898963
89 0.000754710228648037
90 0.000729395367670804
91 0.000705877260770649
92 0.000684010912664235
93 0.000663653365336359
94 0.000644664629362524
95 0.000626908848062158
96 0.000610255054198205
97 0.000594579323660582
98 0.000579765997827053
99 0.000565713795367628
100 0.000552329991478473
101 0.000539540080353618
102 0.000527276599314064
103 0.00051548914052546
104 0.000504129857290536
105 0.000493167841341347
106 0.000482580362586305
107 0.0004723476304207
108 0.000462464697193354
109 0.000452932436019182
110 0.000443749595433474
111 0.000434920395491645
112 0.000426443555625156
113 0.000418313924456015
114 0.000410523585742339
115 0.000403060985263437
116 0.000395909009966999
117 0.000389050459489226
118 0.000382467900635675
119 0.000376141397282481
120 0.000370054680388421
121 0.000364188192179427
122 0.000358525838237256
123 0.000353050651028752
124 0.000347748980857432
125 0.000342604558682069
126 0.00033760498627089
127 0.000332736817654222
128 0.00032799132168293
129 0.000323356420267373
130 0.000318825332215056
131 0.000314386503305286
132 0.000310035335132852
133 0.000305761554045603
134 0.000301561871310696
135 0.0002974278468173
136 0.000293357472401112
137 0.000289343879558146
138 0.000285386573523283
139 0.000281480344710872
140 0.000277624756563455
141 0.000273816724075004
142 0.000270056334557012
143 0.000266341579845175
144 0.000262672372628003
145 0.000259047112194821
146 0.00025546588585712
147 0.000251927238423377
148 0.000248430937062949
149 0.000244976370595396
150 0.000241562214796431
151 0.000238188193179667
152 0.000234853287111036
153 0.000231557554798201
154 0.000228300108574331
155 0.000225080249947496
156 0.000221897280425765
157 0.000218752378714271
158 0.000215643944102339
159 0.00021257197658997
160 0.000209535166504793
161 0.000206533921300434
162 0.000203568313736469
163 0.000200638998649083
164 0.000197746208868921
165 0.000194889609701931
166 0.000192069332115352
167 0.000189285579835996
168 0.000186543722520582
169 0.000183841388206929
170 0.000181180279469118
171 0.000178562040673569
172 0.000175988097907975
173 0.000173459498910233
174 0.000170976447407156
175 0.000168539190781303
176 0.000166148485732265
177 0.000163804696057923
178 0.000161511983606033
179 0.000159270959557034
180 0.000157081143697724
181 0.000154943190864287
182 0.000152857057400979
183 0.00015082213212736
184 0.00014883853145875
185 0.000146906822919846
186 0.000145027152029797
187 0.00014319978072308
188 0.000141422933666036
189 0.000139696698170155
190 0.000138021292514168
191 0.000136394854052924
192 0.000134816829813644
193 0.000133287278003991
194 0.000131803681142628
195 0.000130364860524423
196 0.000128970219520852
197 0.000127617327962071
198 0.000126305327285081
199 0.00012503273319453
200 0.000123797726701014
201 0.000122599085443653
202 0.000121434874017723
203 0.000120303782750852
204 0.000119204662041739
205 0.000118136369565036
206 0.000117097348265816
207 0.000116086303023621
208 0.000115102106065024
209 0.000114143680548295
210 0.000113210007839371
211 0.00011230020027142
212 0.000111413377453573
213 0.000110548950033262
214 0.000109705899376422
215 0.000108882923086639
216 0.000108079213532619
217 0.00010729376663221
218 0.000106525832961779
219 0.000105774510302581
220 0.000105038714536931
221 0.000104317565273959
222 0.000103610545920674
223 0.000102916979813017
224 0.000102236052043736
225 0.000101567311503459
226 0.000100909608590882
227 0.000100262433988973
228 9.96254821075127e-05
229 9.89974359981716e-05
230 9.83779682428576e-05
231 9.77667514234781e-05
232 9.71626723185182e-05
233 9.65657236520201e-05
234 9.59749304456636e-05
235 9.53898488660343e-05
236 9.48099695960991e-05
237 9.42345141083933e-05
238 9.36633587116376e-05
239 9.30959286051802e-05
240 9.25322237890214e-05
241 9.19715821510181e-05
242 9.14139818632975e-05
243 9.08590809558518e-05
244 9.03064501471817e-05
245 8.97560676094145e-05
246 8.9207424025517e-05
247 8.86601628735662e-05
248 8.81144296727143e-05
249 8.75695332069881e-05
250 8.70253425091505e-05
251 8.6481733887922e-05
252 8.59385618241504e-05
253 8.53956589708105e-05
254 8.48526033223607e-05
255 8.43094094307162e-05
256 8.37664701975882e-05
257 8.32228761282749e-05
258 8.26789764687419e-05
259 8.2134545664303e-05
260 8.1589532783255e-05
261 8.10440396890044e-05
262 8.04982555564493e-05
263 7.9951852967497e-05
264 7.94051811681129e-05
265 7.88582765380852e-05
266 7.83112045610324e-05
267 7.77636960265227e-05
268 7.72162093198858e-05
269 7.66684970585629e-05
270 7.61209594202228e-05
271 7.55734799895436e-05
272 7.50260878703557e-05
273 7.44791468605399e-05
274 7.3932773375418e-05
275 7.33870911062695e-05
276 7.28422892279923e-05
277 7.22984623280354e-05
278 7.17555230949074e-05
279 7.12142355041578e-05
280 7.06744394847192e-05
281 7.01363969710656e-05
282 6.96005081408657e-05
283 6.90666929585859e-05
284 6.85352279106155e-05
285 6.80062876199372e-05
286 6.74802940920927e-05
287 6.69568617013283e-05
288 6.64366962155327e-05
289 6.59197103232145e-05
290 6.54058676445857e-05
291 6.48955610813573e-05
292 6.43887469777837e-05
293 6.38857309240848e-05
294 6.3386614783667e-05
295 6.28914494882338e-05
296 6.24003369011916e-05
297 6.19132260908373e-05
298 6.14302771282382e-05
299 6.0952028434258e-05
300 6.04780107096303e-05
301 6.00086095801089e-05
302 5.95438468735665e-05
303 5.9083631640533e-05
304 5.86281457799487e-05
305 5.81772910663858e-05
306 5.77313476242125e-05
307 5.72905628359877e-05
308 5.68544492125511e-05
309 5.6423043133691e-05
310 5.5996686569415e-05
311 5.5575274018338e-05
312 5.51584453205578e-05
313 5.47467461728957e-05
314 5.43399910384323e-05
315 5.39380671398249e-05
316 5.35411454620771e-05
317 5.31489895365667e-05
318 5.27617157786153e-05
319 5.23791131854523e-05
320 5.20013309142087e-05
321 5.16284817422275e-05
322 5.1260143663967e-05
323 5.08965349581558e-05
324 5.05377647641581e-05
325 5.0183469284093e-05
326 4.98336594318971e-05
327 4.9488342483528e-05
328 4.91475511807948e-05
329 4.88109144498594e-05
330 4.84788652101997e-05
331 4.8150908696698e-05
332 4.78273395856377e-05
333 4.75078195449896e-05
334 4.71924067824148e-05
335 4.68812759208959e-05
336 4.65741577500012e-05
337 4.62707321275957e-05
338 4.59715120086912e-05
339 4.56759698863607e-05
340 4.53842912975233e-05
341 4.50963561888784e-05
342 4.48120226792526e-05
343 4.45312543888576e-05
344 4.42540695075877e-05
345 4.3980387999909e-05
346 4.37102135038003e-05
347 4.34432731708512e-05
348 4.31797489000019e-05
349 4.29194551543333e-05
350 4.2662511987146e-05
351 4.24084973928984e-05
352 4.21575714426581e-05
353 4.19095995312091e-05
354 4.1664457967272e-05
355 4.14222777180839e-05
356 4.11829532822594e-05
357 4.09461645176634e-05
358 4.07120678573847e-05
359 4.04806960432325e-05
360 4.02519108320121e-05
361 4.00256176362745e-05
362 3.98017327825073e-05
363 3.95802017010283e-05
364 3.93608934246004e-05
365 3.91439607483335e-05
366 3.89292945328634e-05
367 3.87167965527624e-05
368 3.8506248529302e-05
369 3.82978105335496e-05
370 3.80913443223108e-05
371 3.78868644475006e-05
372 3.76842144760303e-05
373 3.74834125977941e-05
374 3.72845061065163e-05
375 3.7087327655172e-05
376 3.68916844308842e-05
377 3.66978456440847e-05
378 3.65056257578544e-05
379 3.63150356861297e-05
380 3.61259408236947e-05
381 3.59383993782103e-05
382 3.57522876583971e-05
383 3.55674819729757e-05
384 3.53840696334373e-05
385 3.52021597791463e-05
386 3.50216687365901e-05
387 3.48423309333157e-05
388 3.46644592355005e-05
389 3.44878062605858e-05
390 3.43124011124019e-05
391 3.41382037731819e-05
392 3.39650869136676e-05
393 3.37930687237531e-05
394 3.36222474288661e-05
395 3.34527321683709e-05
396 3.32841445924714e-05
397 3.31165501847863e-05
398 3.29500435327645e-05
399 3.27846755681094e-05
400 3.26203444274142e-05
401 3.24570646625943e-05
402 3.22945161315147e-05
403 3.21331863233354e-05
404 3.19728678732645e-05
405 3.18133934342768e-05
406 3.16551704599988e-05
407 3.14978715323377e-05
408 3.13413911499083e-05
409 3.1185889383778e-05
410 3.10312279907521e-05
411 3.08777925965842e-05
412 3.07252703350969e-05
413 3.05735266010743e-05
414 3.04224904539296e-05
415 3.02726566587808e-05
416 3.01237105304608e-05
417 2.99756629829062e-05
418 2.98283575830283e-05
419 2.96819998766296e-05
420 2.95365207421128e-05
421 2.93920365947997e-05
422 2.92484164674534e-05
423 2.91056476271478e-05
424 2.89636718662223e-05
425 2.88224691757932e-05
426 2.86823978967732e-05
427 2.85430251096841e-05
428 2.8404652766767e-05
429 2.82668406725861e-05
430 2.81300599453971e-05
431 2.79942869383376e-05
432 2.78592069662409e-05
433 2.7724950996344e-05
434 2.75914844678482e-05
435 2.74590511253336e-05
436 2.73273344646441e-05
437 2.71963654085994e-05
438 2.70663731498644e-05
439 2.69372394541278e-05
440 2.68088879238348e-05
441 2.66813385678688e-05
442 2.65547805611277e-05
443 2.6429044737597e-05
444 2.63040928984992e-05
445 2.61799450527178e-05
446 2.60566503129667e-05
447 2.59343505604193e-05
448 2.58127620327286e-05
449 2.56920448009623e-05
450 2.55720515269786e-05
451 2.54531369137112e-05
452 2.53349007834913e-05
453 2.52175923378672e-05
454 2.51009823841741e-05
455 2.49853455898119e-05
456 2.48704345722217e-05
457 2.47563275479479e-05
458 2.46431391133228e-05
459 2.45306237047771e-05
460 2.44191160163609e-05
461 2.43084814428585e-05
462 2.41986617766088e-05
463 2.40895205934066e-05
464 2.39813016378321e-05
465 2.38737211475382e-05
466 2.37671156355646e-05
467 2.36614414461656e-05
468 2.35565821640193e-05
469 2.3452384994016e-05
470 2.33489481615834e-05
471 2.3246373530128e-05
472 2.31446683756076e-05
473 2.30437453865306e-05
474 2.29436718655052e-05
475 2.28444096137537e-05
476 2.27459004236152e-05
477 2.26480333367363e-05
478 2.25512430915842e-05
479 2.24552604777273e-05
480 2.23599436139921e-05
481 2.22654507524567e-05
482 2.21716927626403e-05
483 2.20787569560343e-05
484 2.19866651605116e-05
485 2.18953664443688e-05
486 2.18048353417544e-05
487 2.17149608943146e-05
488 2.16260377783328e-05
489 2.15377494896529e-05
490 2.14501887967344e-05
491 2.13635139516555e-05
492 2.12775885302108e-05
493 2.11923652386758e-05
494 2.11080332519487e-05
495 2.10243124456611e-05
496 2.09414247365203e-05
497 2.08591754926601e-05
498 2.07777011382859e-05
499 2.06969307328109e-05
500 2.0616986148525e-05
501 2.05376200028695e-05
502 2.04590432986151e-05
503 2.03812469408149e-05
504 2.03041272470728e-05
505 2.02277624339331e-05
506 2.0152207071078e-05
507 2.00773138203658e-05
508 2.00030299311038e-05
509 1.99293044715887e-05
510 1.98563775484217e-05
511 1.97841400222387e-05
512 1.97127392311813e-05
513 1.9641856852104e-05
514 1.95718093891628e-05
515 1.95023530977778e-05
516 1.9433307897998e-05
517 1.93652103916975e-05
518 1.92976895050379e-05
519 1.92309526028112e-05
520 1.91646977327764e-05
521 1.90991395356832e-05
522 1.90342252608389e-05
523 1.89698839676566e-05
524 1.89062520803418e-05
525 1.88432532013394e-05
526 1.87807163456455e-05
527 1.87188925337978e-05
528 1.86577053682413e-05
529 1.8597145754029e-05
530 1.85372755368007e-05
531 1.84777436516015e-05
532 1.84189666470047e-05
533 1.83607462531654e-05
534 1.83030588232214e-05
535 1.82460389623884e-05
536 1.81895447894931e-05
537 1.81335981324082e-05
538 1.80782590177841e-05
539 1.80234310391825e-05
540 1.79690396180376e-05
541 1.79152084456291e-05
542 1.78620521182893e-05
543 1.78093341673957e-05
544 1.77572001121007e-05
545 1.77056554093724e-05
546 1.76545017893659e-05
547 1.760398663464e-05
548 1.75539244082756e-05
549 1.75042587216012e-05
550 1.74551732925465e-05
551 1.74065971805248e-05
552 1.73585031006951e-05
553 1.73108328453964e-05
554 1.72638065123465e-05
555 1.72170985024422e-05
556 1.71708834386664e-05
557 1.71251194842625e-05
558 1.70797993632732e-05
559 1.70349121617619e-05
560 1.69905488291988e-05
561 1.69466311490396e-05
562 1.69032355188392e-05
563 1.68602582562016e-05
564 1.68176284205401e-05
565 1.67754369613249e-05
566 1.67337111633969e-05
567 1.66923346114345e-05
568 1.66514146258123e-05
569 1.66110003192443e-05
570 1.65708061103942e-05
571 1.65310520969797e-05
572 1.64918037626194e-05
573 1.64529283210868e-05
574 1.64144021255197e-05
575 1.63762178999605e-05
576 1.63385393534554e-05
577 1.63011136464775e-05
578 1.6264122677967e-05
579 1.62273754540365e-05
580 1.6191177564906e-05
581 1.61552761710482e-05
582 1.61197567649651e-05
583 1.60845811478794e-05
584 1.60496947501088e-05
585 1.60152430908056e-05
586 1.598107337486e-05
587 1.59472638188163e-05
588 1.59136670845328e-05
589 1.58804559760028e-05
590 1.58476177603006e-05
591 1.58151269715745e-05
592 1.57828944793437e-05
593 1.57509348355234e-05
594 1.57192935148487e-05
595 1.56877867993899e-05
596 1.56567075464409e-05
597 1.56258174683899e-05
598 1.55953275680076e-05
599 1.55651014210889e-05
600 1.55351208377397e-05
601 1.55053839989705e-05
602 1.54759218276013e-05
603 1.5446847100975e-05
604 1.5417936083395e-05
605 1.53892415255541e-05
606 1.53608434629859e-05
607 1.53326927829767e-05
608 1.53047149069607e-05
609 1.52768261614256e-05
610 1.524928939034e-05
611 1.52221573443967e-05
612 1.51952563101077e-05
613 1.51684625961934e-05
614 1.51417279994348e-05
615 1.51153872138821e-05
616 1.50893874888425e-05
617 1.5063512364577e-05
618 1.50378518810612e-05
619 1.50122696140897e-05
620 1.49870465975255e-05
621 1.49620827869512e-05
622 1.49372335727094e-05
623 1.49124398376443e-05
624 1.48879789776402e-05
625 1.48637973325094e-05
626 1.48397411976475e-05
627 1.48157678268035e-05
628 1.47918963193661e-05
629 1.47683067552862e-05
630 1.47449909491115e-05
631 1.47219561767997e-05
632 1.46988149936078e-05
633 1.46758593473351e-05
634 1.46531865539146e-05
635 1.46309084811946e-05
636 1.46085731103085e-05
637 1.45862459248747e-05
638 1.4564043340215e-05
639 1.45423409776413e-05
640 1.45206713568768e-05
641 1.44990590342786e-05
642 1.44772911880864e-05
643 1.44559471664252e-05
644 1.44351042763446e-05
645 1.4414344150282e-05
646 1.43932384162326e-05
647 1.43722172651906e-05
648 1.43516608659411e-05
649 1.43315091918339e-05
650 1.43112838486559e-05
651 1.42908156703925e-05
652 1.42705675898469e-05
653 1.42509124998469e-05
654 1.42315238917945e-05
655 1.42119288284448e-05
656 1.41918417284614e-05
657 1.41721830004826e-05
658 1.41532209454454e-05
659 1.41345526571968e-05
660 1.41152004289324e-05
661 1.40956099130563e-05
662 1.407667514286e-05
663 1.40585807457683e-05
664 1.40403817567858e-05
665 1.40213251142995e-05
666 1.40020147227915e-05
667 1.39838302857243e-05
668 1.39666672112071e-05
669 1.39488975037239e-05
670 1.39300773298601e-05
671 1.39110834425082e-05
672 1.38937411975348e-05
673 1.38772957143374e-05
674 1.38598479679786e-05
675 1.38408504426479e-05
676 1.38224268084741e-05
677 1.38060740937362e-05
678 1.37906863528769e-05
679 1.37733959491015e-05
680 1.37541783260531e-05
681 1.37360475491732e-05
682 1.37208189698867e-05
683 1.37062252179021e-05
684 1.36887611006387e-05
685 1.36690550789353e-05
686 1.36513199322508e-05
687 1.36376465889043e-05
688 1.36241906147916e-05
689 1.36061498778872e-05
690 1.35857044369914e-05
691 1.35684304041206e-05
692 1.3556669728132e-05
693 1.3544364264817e-05
694 1.35254140332108e-05
695 1.35035097628133e-05
696 1.34867932501948e-05
697 1.34778138090041e-05
698 1.34670044644736e-05
699 1.344630800304e-05
700 1.3422438314592e-05
701 1.34061692733667e-05
702 1.34009178509586e-05
703 1.33921157612349e-05
704 1.33689145513927e-05
705 1.33420253405347e-05
706 1.33261501105153e-05
707 1.33269786601886e-05
708 1.3321384358278e-05
709 1.3293913070811e-05
710 1.32626473714481e-05
711 1.32461545945262e-05
712 1.32562363432953e-05
713 1.32557743199868e-05
714 1.32213972392492e-05
715 1.31843298731837e-05
716 1.31652177515207e-05
717 1.31898395920871e-05
718 1.31989072542638e-05
719 1.31529877762659e-05
720 1.31093302115914e-05
721 1.30813950818265e-05
722 1.31293027152424e-05
723 1.31576207422768e-05
724 1.30899452415179e-05
725 1.30437074403744e-05
726 1.29945483422489e-05
727 1.30781136249425e-05
728 1.31504029923235e-05
729 1.30369508042349e-05
730 1.3005554137635e-05
731 1.29096270029549e-05
732 1.30401995193097e-05
733 1.32192635646788e-05
734 1.3001037586946e-05
735 1.30356875160942e-05
736 1.28506944747642e-05
737 1.30198059196118e-05
738 1.34576021082466e-05
739 1.29974559968105e-05
740 1.32183395180618e-05
741 1.2874430467491e-05
742 1.30326416183379e-05
743 1.40122501761653e-05
744 1.30353864733479e-05
745 1.36892376758624e-05
746 1.29679046949605e-05
747 1.32040822791168e-05
748 1.48623166751349e-05
749 1.30522339532035e-05
750 1.44730774991331e-05
751 1.26809773064451e-05
752 1.40511510835495e-05
753 1.49296365634655e-05
754 1.33166458908818e-05
755 1.44298710438306e-05
756 1.21382499855827e-05
757 1.52834527398227e-05
758 1.32969043988851e-05
759 1.40684614962083e-05
760 1.25164806377143e-05
761 1.33890807774151e-05
762 1.39056874104426e-05
763 1.30570397232077e-05
764 1.29945829030476e-05
765 1.23159279610263e-05
766 1.37106562760891e-05
767 1.27361799968639e-05
768 1.29738236864796e-05
769 1.21613720693858e-05
770 1.31398137455108e-05
771 1.28064812088269e-05
772 1.27707835417823e-05
773 1.23317640827736e-05
774 1.26345257740468e-05
775 1.29024156194646e-05
776 1.26137747429311e-05
777 1.25084825413069e-05
778 1.23463923955569e-05
779 1.28499978018226e-05
780 1.2586206139531e-05
781 1.25662008940708e-05
782 1.2289918231545e-05
783 1.26271352201002e-05
784 1.26585946418345e-05
785 1.25155474961502e-05
786 1.23797381093027e-05
787 1.23593008538592e-05
788 1.2687052731053e-05
789 1.24794169096276e-05
790 1.24530442917603e-05
791 1.22390474643908e-05
792 1.25224114526645e-05
793 1.25535216284334e-05
794 1.24129064715817e-05
795 1.23156814879621e-05
796 1.22453584481264e-05
797 1.25971400848357e-05
798 1.23961799545214e-05
799 1.23830905067734e-05
800 1.2163423889433e-05
801 1.23665422506747e-05
802 1.25451169878943e-05
803 1.23113204608671e-05
804 1.23201189126121e-05
805 1.20732202049112e-05
806 1.25304632092593e-05
807 1.24099860840943e-05
808 1.23197778520989e-05
809 1.21974180729012e-05
810 1.21045595733449e-05
811 1.26692293633823e-05
812 1.22643486974994e-05
813 1.24008611237514e-05
814 1.20214936032426e-05
815 1.23088520922465e-05
816 1.27126095321728e-05
817 1.22035098684137e-05
818 1.24879370559938e-05
819 1.18441666927538e-05
820 1.27140383483493e-05
821 1.25822289192001e-05
822 1.23506415548036e-05
823 1.24302587209968e-05
824 1.1830442417704e-05
825 1.32032409965177e-05
826 1.2285548109503e-05
827 1.2741519640258e-05
828 1.2044337381667e-05
829 1.23022045954713e-05
830 1.33003932205611e-05
831 1.21728926387732e-05
832 1.29520176415099e-05
833 1.15569037006935e-05
834 1.3256789316074e-05
835 1.25976675917627e-05
836 1.26649583762628e-05
837 1.22614846986835e-05
838 1.19381438707933e-05
839 1.34301153593697e-05
840 1.20830891319201e-05
841 1.28482861327939e-05
842 1.14254298750893e-05
843 1.30454282043502e-05
844 1.23593918033293e-05
845 1.25096885312814e-05
846 1.18689804367023e-05
847 1.19978885777527e-05
848 1.28359015434398e-05
849 1.20132290248876e-05
850 1.23065792649868e-05
851 1.14231379484409e-05
852 1.27379707919317e-05
853 1.19788564916234e-05
854 1.22763276522164e-05
855 1.15122929855715e-05
856 1.21345474326517e-05
857 1.22702431326616e-05
858 1.19730129881646e-05
859 1.18671405289206e-05
860 1.15578095574165e-05
861 1.24241942103254e-05
862 1.1811745025625e-05
863 1.20527647595736e-05
864 1.14043814392062e-05
865 1.21506691357354e-05
866 1.19828964670887e-05
867 1.19196201922023e-05
868 1.16563005576609e-05
869 1.16270421131048e-05
870 1.22436940728221e-05
871 1.17290210255305e-05
872 1.1935338989133e-05
873 1.13510886876611e-05
874 1.21333487186348e-05
875 1.18599109555362e-05
876 1.18796751849004e-05
877 1.1576234101085e-05
878 1.16036380859441e-05
879 1.22126230053254e-05
880 1.16606051960844e-05
881 1.19361639008275e-05
882 1.12723919301061e-05
883 1.21499670058256e-05
884 1.18379502964672e-05
885 1.18717443911009e-05
886 1.15892289613839e-05
887 1.15132652354077e-05
888 1.23251211334718e-05
889 1.1598981473071e-05
890 1.20454724310548e-05
891 1.11786121124169e-05
892 1.22091032608296e-05
893 1.18996658784454e-05
894 1.18923453555908e-05
895 1.16695518954657e-05
896 1.13979749585269e-05
897 1.2521214557637e-05
898 1.15505808935268e-05
899 1.21955354188685e-05
900 1.1075288966822e-05
901 1.23132240332779e-05
902 1.19286887638737e-05
903 1.19526221169508e-05
904 1.16533383334172e-05
905 1.13721935122157e-05
906 1.25975648188614e-05
907 1.15134953375673e-05
908 1.22141027532052e-05
909 1.09359880298143e-05
910 1.24585239973385e-05
911 1.17238578241086e-05
912 1.20473041533842e-05
913 1.13385312943137e-05
914 1.15816383186029e-05
915 1.23010859169881e-05
916 1.15532493509818e-05
917 1.18992902571335e-05
918 1.09221000457183e-05
919 1.24418966152007e-05
920 1.14232025225647e-05
921 1.2010296813969e-05
922 1.0932542863884e-05
923 1.19219685075223e-05
924 1.17452473205049e-05
925 1.16847768367734e-05
926 1.13545311251073e-05
927 1.12053257907974e-05
928 1.2098639672331e-05
929 1.13472469820408e-05
930 1.17188774311217e-05
931 1.08207577795838e-05
932 1.20597642307985e-05
933 1.13433479782543e-05
934 1.17313757073134e-05
935 1.09329157567117e-05
936 1.15751445264323e-05
937 1.16598630484077e-05
938 1.14496142487042e-05
939 1.13349597086199e-05
940 1.09946840893826e-05
941 1.19536534839426e-05
942 1.12137367977994e-05
943 1.16432001959765e-05
944 1.07537825897452e-05
945 1.18385178211611e-05
946 1.13316846181988e-05
947 1.15839138743468e-05
948 1.10015607788227e-05
949 1.12933294076356e-05
950 1.17418803711189e-05
951 1.12715224531712e-05
952 1.14715903691831e-05
953 1.07665564428316e-05
954 1.19846208690433e-05
955 1.11440076580038e-05
956 1.1695665307343e-05
957 1.07345049400465e-05
958 1.16612754936796e-05
959 1.14734466478694e-05
960 1.14674076030497e-05
961 1.11982135422295e-05
962 1.09809907371528e-05
963 1.1967696082138e-05
964 1.11238241515821e-05
965 1.16848495963495e-05
966 1.0578482033452e-05
967 1.20266331578023e-05
968 1.11659410322318e-05
969 1.17263489300967e-05
970 1.07817677417188e-05
971 1.14657677841024e-05
972 1.16351602628129e-05
973 1.13568048618617e-05
974 1.13294845505152e-05
975 1.07626119643101e-05
976 1.20596878332435e-05
977 1.10360369944829e-05
978 1.17219578896766e-05
979 1.04631581052672e-05
980 1.19821797852637e-05
981 1.11047738755587e-05
982 1.16808423626935e-05
983 1.06703737401403e-05
984 1.14221447802265e-05
985 1.14844078780152e-05
986 1.1330076631566e-05
987 1.11207955342252e-05
988 1.07792075141333e-05
989 1.18367506729555e-05
990 1.09986067400314e-05
991 1.14847434815601e-05
992 1.04154805740109e-05
993 1.18683219625382e-05
994 1.09255415736698e-05
995 1.15637176349992e-05
996 1.04412147265975e-05
997 1.15228558570379e-05
998 1.11343697426491e-05
999 1.13622463686625e-05
1000 1.07510104498942e-05
1001 1.09803877421655e-05
1002 1.14680178739945e-05
1003 1.10477049020119e-05
1004 1.11407862277701e-05
1005 1.05105646071024e-05
1006 1.16932433229522e-05
1007 1.0835468856385e-05
1008 1.14027679956052e-05
1009 1.03174743344425e-05
1010 1.16244773380458e-05
1011 1.08713184090448e-05
1012 1.1408297723392e-05
1013 1.04566870504641e-05
1014 1.12420857476536e-05
1015 1.11503795778845e-05
1016 1.1178901331732e-05
1017 1.0827515325218e-05
1018 1.07215801108396e-05
1019 1.15076763904653e-05
1020 1.08847507362952e-05
1021 1.12284515125793e-05
1022 1.0325707080483e-05
1023 1.16977480502101e-05
1024 1.07418563857209e-05
1025 1.14487484097481e-05
1026 1.0240560186503e-05
1027 1.15482180262916e-05
1028 1.08698686744901e-05
1029 1.13842152131838e-05
1030 1.04764567367965e-05
1031 1.10986202344066e-05
1032 1.12106008600676e-05
1033 1.11079170892481e-05
1034 1.08868716779398e-05
1035 1.05722483567661e-05
1036 1.15653738248511e-05
1037 1.08115145849297e-05
1038 1.12674379124655e-05
1039 1.02085650723893e-05
1040 1.17203153422452e-05
1041 1.06715842775884e-05
1042 1.14566155389184e-05
1043 1.01240548247006e-05
1044 1.15753118734574e-05
1045 1.07509313238552e-05
1046 1.14050917545683e-05
1047 1.02903195511317e-05
1048 1.11917224785429e-05
1049 1.09918446469237e-05
1050 1.11788276626612e-05
1051 1.05968829302583e-05
1052 1.07255318653188e-05
1053 1.12752732093213e-05
1054 1.08980102595524e-05
1055 1.09236925709411e-05
1056 1.03261181720882e-05
1057 1.14841222966788e-05
1058 1.06721690826816e-05
1059 1.11763938548393e-05
1060 1.00832940006512e-05
1061 1.15389611892169e-05
1062 1.05682674984564e-05
1063 1.12987381726271e-05
1064 1.00244578788988e-05
1065 1.14154763650731e-05
1066 1.06051384136663e-05
1067 1.12760708361748e-05
1068 1.01293835541583e-05
1069 1.11424087663181e-05
1070 1.0760443728941e-05
1071 1.11330482468475e-05
1072 1.03515567388968e-05
1073 1.07861515061813e-05
1074 1.09853071990074e-05
1075 1.09209013317013e-05
1076 1.06332026916789e-05
1077 1.04256969279959e-05
1078 1.12158350020763e-05
1079 1.07023151940666e-05
1080 1.09137818071758e-05
1081 1.01334271676023e-05
1082 1.13848664113902e-05
1083 1.05367598735029e-05
1084 1.11362141979043e-05
1085 9.96061862679198e-06
1086 1.143660483649e-05
1087 1.04670762084424e-05
1088 1.12567859105184e-05
1089 9.92742479866138e-06
1090 1.13461846922291e-05
1091 1.05074286693707e-05
1092 1.12589632408344e-05
1093 1.00210572782089e-05
1094 1.11300032585859e-05
1095 1.06405741462368e-05
1096 1.11576937342761e-05
1097 1.02045014500618e-05
1098 1.08370259113144e-05
1099 1.08278045445331e-05
1100 1.09902994154254e-05
1101 1.04326727523585e-05
1102 1.05259923657286e-05
1103 1.10248774944921e-05
1104 1.08004787762184e-05
1105 1.06653706097859e-05
1106 1.02455014712177e-05
1107 1.11936851681094e-05
1108 1.06247380244895e-05
1109 1.08714557427447e-05
1110 1.00255274446681e-05
1111 1.13079704533448e-05
1112 1.04870368886623e-05
1113 1.10316213977057e-05
1114 9.87799739959883e-06
1115 1.13552850962151e-05
1116 1.03977654362097e-05
1117 1.11359668153455e-05
1118 9.80066306510707e-06
1119 1.1336112038407e-05
1120 1.03570655483054e-05
1121 1.11854342321749e-05
1122 9.78305342869135e-06
1123 1.12604693640606e-05
1124 1.03576476249145e-05
1125 1.11869285319699e-05
1126 9.81111588771455e-06
1127 1.11434828795609e-05
1128 1.03895827123779e-05
1129 1.11512872535968e-05
1130 9.87140174402157e-06
1131 1.10010068965494e-05
1132 1.04434293461964e-05
1133 1.10900082290755e-05
1134 9.95332720776787e-06
1135 1.08461581476149e-05
1136 1.05119079307769e-05
1137 1.10126957224566e-05
1138 1.00494926300598e-05
1139 1.06882880572812e-05
1140 1.05898679976235e-05
1141 1.09271240944508e-05
1142 1.01549512692145e-05
1143 1.05338031062274e-05
1144 1.06739744296647e-05
1145 1.0838475645869e-05
1146 1.02658850664739e-05
1147 1.03867159850779e-05
1148 1.07608984762919e-05
1149 1.07508703877102e-05
1150 1.03787733678473e-05
1151 1.02503045127378e-05
1152 1.08473414002219e-05
1153 1.06677289295476e-05
1154 1.04893842944875e-05
1155 1.01271743915277e-05
1156 1.09285638245638e-05
1157 1.05916069514933e-05
1158 1.05928793345811e-05
1159 1.00197539723013e-05
1160 1.09999591586529e-05
1161 1.05249773696414e-05
1162 1.06834531834465e-05
1163 9.92965487967012e-06
1164 1.10567116280436e-05
1165 1.04694172478048e-05
1166 1.07563346318784e-05
1167 9.8575665106182e-06
1168 1.10956052594702e-05
1169 1.04252503660973e-05
1170 1.08077956610941e-05
1171 9.8027285275748e-06
1172 1.11150702650775e-05
1173 1.03920156107051e-05
1174 1.08360691228881e-05
1175 9.76362025539856e-06
1176 1.11160034066415e-05
1177 1.03688253147993e-05
1178 1.08422154880827e-05
1179 9.73858550423756e-06
1180 1.11008839667193e-05
1181 1.03544543890166e-05
1182 1.08286958493409e-05
1183 9.7254724096274e-06
1184 1.10731079985271e-05
1185 1.03479769677506e-05
1186 1.07993737401557e-05
1187 9.72309499047697e-06
1188 1.1036680007237e-05
1189 1.03490092442371e-05
1190 1.07588448372553e-05
1191 9.73010901361704e-06
1192 1.09952234197408e-05
1193 1.03572756415815e-05
1194 1.07109826785745e-05
1195 9.7455276772962e-06
1196 1.09514858195325e-05
1197 1.03727907116991e-05
1198 1.06587776826927e-05
1199 9.76896899373969e-06
1200 1.0907300747931e-05
1201 1.03961001514108e-05
1202 1.06035377029912e-05
1203 9.80033655650914e-06
1204 1.08629255919368e-05
1205 1.04276077763643e-05
1206 1.05452782008797e-05
1207 9.84035978035536e-06
1208 1.0817085239978e-05
1209 1.04682649180177e-05
1210 1.04820019259932e-05
1211 9.89043837762438e-06
1212 1.0767534149636e-05
1213 1.05188501038356e-05
1214 1.04109121821239e-05
1215 9.95289246930042e-06
1216 1.0711106369854e-05
1217 1.0580019079498e-05
1218 1.03282636700897e-05
1219 1.0030032171926e-05
1220 1.06443039840087e-05
1221 1.06519200926414e-05
1222 1.02311687442125e-05
1223 1.01243376775528e-05
1224 1.0564517651801e-05
1225 1.07331452454673e-05
1226 1.01176819953253e-05
1227 1.02365083876066e-05
1228 1.04715572888381e-05
1229 1.08203157651587e-05
1230 9.98913128569257e-06
1231 1.03656784631312e-05
1232 1.03688325907569e-05
1233 1.09075872387621e-05
1234 9.85066981229465e-06
1235 1.05069848359562e-05
1236 1.02641151897842e-05
1237 1.09858819996589e-05
1238 9.71174995356705e-06
1239 1.06512325146468e-05
1240 1.01695586636197e-05
1241 1.10431756183971e-05
1242 9.58567125053378e-06
1243 1.07844152807957e-05
1244 1.01002178780618e-05
1245 1.10657792902202e-05
1246 9.48899469221942e-06
1247 1.08888507384108e-05
1248 1.00715424196096e-05
1249 1.10393766590278e-05
1250 9.43864233704517e-06
1251 1.09447428258136e-05
1252 1.00954630397609e-05
1253 1.09518196040881e-05
1254 9.4495335360989e-06
1255 1.09351822175086e-05
1256 1.01772675407119e-05
1257 1.07973719423171e-05
1258 9.53143080550944e-06
1259 1.08522308437387e-05
1260 1.03117281469167e-05
1261 1.05806811916409e-05
1262 9.68464610195952e-06
1263 1.07023370219395e-05
1264 1.04827577160904e-05
1265 1.031934243656e-05
1266 9.89761429082137e-06
1267 1.05090894066961e-05
1268 1.0665684385458e-05
1269 1.00410543382168e-05
1270 1.01476589406957e-05
1271 1.03084512375062e-05
1272 1.08319682112779e-05
1273 9.77890431386186e-06
1274 1.04040291262208e-05
1275 1.01398045444512e-05
1276 1.09537668322446e-05
1277 9.5650075309095e-06
1278 1.06321513158036e-05
1279 1.00384904726525e-05
1280 1.10057662823237e-05
1281 9.42787482927088e-06
1282 1.07955074781785e-05
1283 1.00298639154062e-05
1284 1.09640177470283e-05
1285 9.39202163863229e-06
1286 1.08592630567728e-05
1287 1.01259465736803e-05
1288 1.08112008092576e-05
1289 9.47593707678607e-06
1290 1.07990517790313e-05
1291 1.03191614471143e-05
1292 1.05463395811967e-05
1293 9.68367839959683e-06
1294 1.06191719169146e-05
1295 1.05747567431536e-05
1296 1.01979812825448e-05
1297 9.99182429950451e-06
1298 1.03679803942214e-05
1299 1.08305839603418e-05
1300 9.8285318017588e-06
1301 1.03409256553277e-05
1302 1.01328432720038e-05
1303 1.10094797491911e-05
1304 9.52213122218382e-06
1305 1.06424758996582e-05
1306 1.00088482213323e-05
1307 1.10410628622049e-05
1308 9.35992102313321e-06
1309 1.08041331259301e-05
1310 1.00560473583755e-05
1311 1.08850072138011e-05
1312 9.39359051699284e-06
1313 1.07677133200923e-05
1314 1.02695121313445e-05
1315 1.05523640741012e-05
1316 9.62571630225284e-06
1317 1.055003485817e-05
1318 1.05764493127936e-05
1319 1.01151108538033e-05
1320 9.99624080577632e-06
1321 1.02547464848612e-05
1322 1.08619306047331e-05
1323 9.69068332778988e-06
1324 1.03910697362153e-05
1325 1.00294955700519e-05
1326 1.10100791061996e-05
1327 9.40212066780077e-06
1328 1.06746783785638e-05
1329 9.99281564872945e-06
1330 1.09428883661167e-05
1331 9.33791216084501e-06
1332 1.07414589365362e-05
1333 1.01773102869629e-05
1334 1.06469669844955e-05
1335 9.52579921431607e-06
1336 1.05717190308496e-05
1337 1.05145318229916e-05
1338 1.01899959190632e-05
1339 9.91234901448479e-06
1340 1.02678432085668e-05
1341 1.08596277641482e-05
1342 9.7116235338035e-06
1343 1.03602105809841e-05
1344 1.0019696674135e-05
1345 1.10465525722248e-05
1346 9.38080665946472e-06
1347 1.0682332685974e-05
1348 9.99969324766425e-06
1349 1.0953838682326e-05
1350 9.33032242755871e-06
1351 1.07243122329237e-05
1352 1.02537433122052e-05
1353 1.05667204479687e-05
1354 9.59586759563535e-06
1355 1.04769451354514e-05
1356 1.0662281056284e-05
1357 1.00095521702315e-05
1358 1.00722654678975e-05
1359 1.01308851299109e-05
1360 1.09957391032367e-05
1361 9.51094625634141e-06
1362 1.0526970072533e-05
1363 9.96571361611132e-06
1364 1.10372793642455e-05
1365 9.2957907327218e-06
1366 1.07087098513148e-05
1367 1.01377781902556e-05
1368 1.07067007775186e-05
1369 9.46843829296995e-06
1370 1.05247809187858e-05
1371 1.05568460639915e-05
1372 1.01219475254766e-05
1373 9.9401067927829e-06
1374 1.01593368526665e-05
1375 1.09483717096737e-05
1376 9.55633095145458e-06
1377 1.04423679658794e-05
1378 9.95581649476662e-06
1379 1.10335940917139e-05
1380 9.29035923036281e-06
1381 1.06660954770632e-05
1382 1.01267469290178e-05
1383 1.07016476249555e-05
1384 9.45839929045178e-06
1385 1.04876535260701e-05
1386 1.05779654404614e-05
1387 1.00858860605513e-05
1388 9.95922528090887e-06
1389 1.01171390269883e-05
1390 1.09854818219901e-05
1391 9.50501362240175e-06
1392 1.04752325569279e-05
1393 9.9530325314845e-06
1394 1.10276141640497e-05
1395 9.28039207792608e-06
1396 1.06521465568221e-05
1397 1.02117110145628e-05
1398 1.06016659628949e-05
1399 9.54177357925801e-06
1400 1.0394559467386e-05
1401 1.07253836176824e-05
1402 9.91454908216838e-06
1403 1.01153118521324e-05
1404 1.00247489172034e-05
1405 1.10778373709763e-05
1406 9.37333243200555e-06
1407 1.05812678157235e-05
1408 1.00025745268795e-05
1409 1.09351231003529e-05
1410 9.31820522964699e-06
1411 1.0577298780845e-05
1412 1.0435104741191e-05
1413 1.0311812729924e-05
1414 9.77474610408535e-06
1415 1.01901205198374e-05
1416 1.09575294118258e-05
1417 9.59346380113857e-06
1418 1.03791498986538e-05
1419 9.94660513242707e-06
1420 1.10731598397251e-05
1421 9.26532902667532e-06
1422 1.0616868166835e-05
1423 1.02207959571388e-05
1424 1.05945364339277e-05
1425 9.54249935603002e-06
1426 1.0327658856113e-05
1427 1.0795280104503e-05
1428 9.81764515017858e-06
1429 1.01711730167153e-05
1430 9.96801827568561e-06
1431 1.10998898890102e-05
1432 9.30002715904266e-06
1433 1.05800227174768e-05
1434 1.00945544545539e-05
1435 1.07733412733069e-05
1436 9.4107545010047e-06
1437 1.04144291981356e-05
1438 1.06654133560369e-05
1439 9.99752865027403e-06
1440 1.00156303233234e-05
1441 1.00104134617141e-05
1442 1.10957198558026e-05
1443 9.35736989049474e-06
1444 1.05319022623007e-05
1445 1.00327879408724e-05
1446 1.08854064819752e-05
1447 9.34514628170291e-06
1448 1.04629098132136e-05
1449 1.05922563307104e-05
1450 1.01151899798424e-05
1451 9.92516288533807e-06
1452 1.00424786069198e-05
1453 1.10979535747902e-05
1454 9.39647088671336e-06
1455 1.05008111859206e-05
1456 1.00133547675796e-05
1457 1.09403417809517e-05
1458 9.32115017349133e-06
1459 1.04732880572556e-05
1460 1.05814060589182e-05
1461 1.01485256891465e-05
1462 9.90293119684793e-06
1463 1.00422812465695e-05
1464 1.11146027848008e-05
1465 9.39224628382362e-06
1466 1.04942164398381e-05
1467 1.00263996500871e-05
1468 1.09291513581411e-05
1469 9.33112278289627e-06
1470 1.04399396150257e-05
1471 1.0634431419021e-05
1472 1.00848737929482e-05
1473 9.94997299130773e-06
1474 1.0007156561187e-05
1475 1.11403087430517e-05
1476 9.34566651267232e-06
1477 1.0507046681596e-05
1478 1.00787083283649e-05
1479 1.08460653791553e-05
1480 9.38251105253585e-06
1481 1.03639331427985e-05
1482 1.07478063000599e-05
1483 9.93720004771603e-06
1484 1.00599245342892e-05
1485 9.96021117316559e-06
1486 1.11595136331744e-05
1487 9.28495683183428e-06
1488 1.05219360193587e-05
1489 1.01901778180036e-05
1490 1.06888965092367e-05
1491 9.49191417021211e-06
1492 1.02515841717832e-05
1493 1.09093525679782e-05
1494 9.73287205852102e-06
1495 1.02165331554716e-05
1496 9.94043512037024e-06
1497 1.11400686364505e-05
1498 9.2516211225302e-06
1499 1.05063463706756e-05
1500 1.0382375876361e-05
1501 1.04446098703193e-05
1502 9.67798860074254e-06
1503 1.01117357189651e-05
1504 1.10859336928115e-05
1505 9.50368939811597e-06
1506 1.03846623460413e-05
1507 9.99813892121892e-06
1508 1.10220235001179e-05
1509 9.29856742004631e-06
1510 1.04146456578746e-05
1511 1.06609704744187e-05
1512 1.01001278380863e-05
1513 9.94494712358573e-06
1514 9.98053383227671e-06
1515 1.12011157398229e-05
1516 9.31023532757536e-06
1517 1.04923428807524e-05
1518 1.0192668014497e-05
1519 1.07354135252535e-05
1520 9.4836486823624e-06
1521 1.02245985544869e-05
1522 1.09770262497477e-05
1523 9.69530447036959e-06
1524 1.02438398243976e-05
1525 9.95454774965765e-06
1526 1.11392446342506e-05
1527 9.25729727896396e-06
1528 1.04485689007561e-05
1529 1.05452136267559e-05
1530 1.02695476016379e-05
1531 9.81959146884037e-06
1532 1.00112183645251e-05
1533 1.11988347271108e-05
1534 9.35880234465003e-06
1535 1.04493092294433e-05
1536 1.0150637535844e-05
1537 1.08114863905939e-05
1538 9.43997747526737e-06
1539 1.02333115137299e-05
1540 1.09649672594969e-05
1541 9.73239184531849e-06
1542 1.02105304904399e-05
1543 9.95631125988439e-06
1544 1.11548879431211e-05
1545 9.26009579416132e-06
1546 1.04242935776711e-05
1547 1.05856097434298e-05
1548 1.02416424851981e-05
1549 9.84696271189023e-06
1550 9.99036092252936e-06
1551 1.12331090349471e-05
1552 9.33172123041004e-06
1553 1.04501650639577e-05
1554 1.02265066743712e-05
1555 1.07349796962808e-05
1556 9.50667799770599e-06
1557 1.01663654277218e-05
1558 1.10772298285156e-05
1559 9.61736714089056e-06
1560 1.02907879409031e-05
1561 1.00036659205216e-05
1562 1.10929358925205e-05
1563 9.29828911466757e-06
1564 1.03513011708856e-05
1565 1.07806172309211e-05
1566 1.00265269793454e-05
1567 1.00122870207997e-05
1568 9.9513417808339e-06
1569 1.12590696517145e-05
1570 9.26661414268892e-06
1571 1.04430773717468e-05
1572 1.04543614725117e-05
1573 1.04584723885637e-05
1574 9.71000099525554e-06
1575 1.00322540674824e-05
1576 1.12403549792361e-05
1577 9.39879555517109e-06
1578 1.0406901310489e-05
1579 1.01849955171929e-05
1580 1.08328931673896e-05
1581 9.46144518820802e-06
1582 1.01746636573807e-05
1583 1.10851951831137e-05
1584 9.64981700235512e-06
1585 1.02629819593858e-05
1586 1.00163624665583e-05
1587 1.1104071745649e-05
1588 9.30916939978488e-06
1589 1.03107913673739e-05
1590 1.08553767859121e-05
1591 9.96559901977889e-06
1592 1.00566621767939e-05
1593 9.95368009171216e-06
1594 1.12581574285286e-05
1595 9.26430220715702e-06
1596 1.03970432974165e-05
1597 1.06066891021328e-05
1598 1.02972371678334e-05
1599 9.8340742624714e-06
1600 9.97612642095191e-06
1601 1.13035021058749e-05
1602 9.31612612475874e-06
1603 1.04166456367238e-05
1604 1.03806332845124e-05
1605 1.06057595985476e-05
1606 9.63078673521522e-06
1607 1.00510833362932e-05
1608 1.12614716272219e-05
1609 9.4423039627145e-06
1610 1.03732763818698e-05
1611 1.02011454146123e-05
1612 1.08655103758792e-05
1613 9.46965519688092e-06
1614 1.01456271295319e-05
1615 1.11595791167929e-05
1616 9.61557270784397e-06
1617 1.02833810160519e-05
1618 1.00763318187091e-05
1619 1.10643804873689e-05
1620 9.35937896429095e-06
1621 1.02341355159297e-05
1622 1.10257687992998e-05
1623 9.80956792773213e-06
1624 1.01677769634989e-05
1625 1.00018651210121e-05
1626 1.12030920718098e-05
1627 9.29783436731668e-06
1628 1.03027568911784e-05
1629 1.08835283754161e-05
1630 1.00043534985161e-05
1631 1.00457264124998e-05
1632 9.96752987703076e-06
1633 1.12910520329024e-05
1634 9.27685050555738e-06
1635 1.03478269011248e-05
1636 1.07486930573941e-05
1637 1.01864688986097e-05
1638 9.92986133496743e-06
1639 9.96085509541444e-06
1640 1.13402056740597e-05
1641 9.28546160139376e-06
1642 1.03718966784072e-05
1643 1.06305069493828e-05
1644 1.03486163425259e-05
1645 9.82801248028409e-06
1646 9.97113602352329e-06
1647 1.13628439066815e-05
1648 9.31317663344089e-06
1649 1.03798101918073e-05
1650 1.05328053905396e-05
1651 1.04867067420855e-05
1652 9.74339400272584e-06
1653 9.98960422293749e-06
1654 1.13687274279073e-05
1655 9.35043408389902e-06
1656 1.03766142274253e-05
1657 1.04565815490787e-05
1658 1.05989902294823e-05
1659 9.67737923929235e-06
1660 1.00096804089844e-05
1661 1.13660016722861e-05
1662 9.38907578529324e-06
1663 1.03671700344421e-05
1664 1.04013088275678e-05
1665 1.06847555798595e-05
1666 9.62920603342354e-06
1667 1.00268298410811e-05
1668 1.13607957246131e-05
1669 9.42296992434422e-06
1670 1.03556440080865e-05
1671 1.03655402199365e-05
1672 1.07445357571123e-05
1673 9.59791213972494e-06
1674 1.00385514087975e-05
1675 1.13577934826026e-05
1676 9.44778003031388e-06
1677 1.03452593975817e-05
1678 1.03478887467645e-05
1679 1.07798205135623e-05
1680 9.58186774369096e-06
1681 1.00434408523142e-05
1682 1.13600572149153e-05
1683 9.46093314269092e-06
1684 1.03381480585085e-05
1685 1.03471547845402e-05
1686 1.07921496237395e-05
1687 9.58008695306489e-06
1688 1.00419674708974e-05
1689 1.13693895400502e-05
1690 9.4619499577675e-06
1691 1.0335249498894e-05
1692 1.03631937236059e-05
1693 1.07829482658417e-05
1694 9.59191584115615e-06
1695 1.00347751867957e-05
1696 1.13859787234105e-05
1697 9.45116153161507e-06
1698 1.03360844150302e-05
1699 1.03963529909379e-05
1700 1.07520709207165e-05
1701 9.61713885772042e-06
1702 1.0022734386439e-05
1703 1.14076901809312e-05
1704 9.43021041166503e-06
1705 1.03389629657613e-05
1706 1.04476885098848e-05
1707 1.06984816738986e-05
1708 9.65643084782641e-06
1709 1.0007778655563e-05
1710 1.14312970254105e-05
1711 9.40198697207961e-06
1712 1.03410129668191e-05
1713 1.05185890788562e-05
1714 1.06206198324799e-05
1715 9.71021927398397e-06
1716 9.992536433856e-06
1717 1.14511740321177e-05
1718 9.37078311835648e-06
1719 1.03382690213039e-05
1720 1.06104716905975e-05
1721 1.05171884570154e-05
1722 9.77870604401687e-06
1723 9.98131145024672e-06
1724 1.14603262773016e-05
1725 9.3423204816645e-06
1726 1.03264637800748e-05
1727 1.07231689980836e-05
1728 1.0388695045549e-05
1729 9.86094710242469e-06
1730 9.98049927147804e-06
1731 1.14504518933245e-05
1732 9.32376406126423e-06
1733 1.03017982837628e-05
1734 1.08552085293923e-05
1735 1.02375543065136e-05
1736 9.95414120552596e-06
1737 9.99814255919773e-06
1738 1.1413088941481e-05
1739 9.32339844439412e-06
1740 1.02621688711224e-05
1741 1.10018627310637e-05
1742 1.00693550848518e-05
1743 1.00530260169762e-05
1744 1.00432143881335e-05
1745 1.13392334242235e-05
1746 9.34859053813852e-06
1747 1.02069452623255e-05
1748 1.1154979802086e-05
1749 9.89197451417567e-06
1750 1.01499153970508e-05
1751 1.01250798252295e-05
1752 1.12213310785592e-05
1753 9.40619338507531e-06
1754 1.0140101039724e-05
1755 1.13022624645964e-05
1756 9.71643567027058e-06
1757 1.02339463410317e-05
1758 1.02510994111071e-05
1759 1.10534601844847e-05
1760 9.50018420553533e-06
1761 1.00702227427973e-05
1762 1.14260938062216e-05
1763 9.55708037508884e-06
1764 1.02923804661259e-05
1765 1.04250366348424e-05
1766 1.08344047475839e-05
1767 9.62990816333331e-06
1768 1.00128600024618e-05
1769 1.15050361273461e-05
1770 9.43129725783365e-06
1771 1.031295323628e-05
1772 1.06434063127381e-05
1773 1.05704084489844e-05
1774 9.78797288553324e-06
1775 9.98929863271769e-06
1776 1.15163666123408e-05
1777 9.35743264562916e-06
1778 1.02877984318184e-05
1779 1.0891997590079e-05
1780 1.02771546153235e-05
1781 9.95887421595398e-06
1782 1.00237266451586e-05
1783 1.14412796392571e-05
1784 9.35127900447696e-06
1785 1.02192589110928e-05
1786 1.11457902676193e-05
1787 9.97903316601878e-06
1788 1.01192053989507e-05
1789 1.01367495517479e-05
1790 1.12690804598969e-05
1791 9.42159476835513e-06
1792 1.01241985248635e-05
1793 1.13690020953072e-05
1794 9.70703149505425e-06
1795 1.02405792858917e-05
1796 1.03375696198782e-05
1797 1.1001877282979e-05
1798 9.56533858698094e-06
1799 1.00350198408705e-05
1800 1.15195907710586e-05
1801 9.4959596026456e-06
1802 1.02973253888194e-05
1803 1.06178240457666e-05
1804 1.06579318526201e-05
1805 9.76413866737857e-06
1806 9.9962271633558e-06
1807 1.15548336907523e-05
1808 9.3784346972825e-06
1809 1.02748626886751e-05
1810 1.09453130789916e-05
1811 1.02720141512691e-05
1812 9.98209088720614e-06
1813 1.00538773040171e-05
1814 1.14433669295977e-05
1815 9.37803633860312e-06
1816 1.01839268609183e-05
1817 1.126384086092e-05
1818 9.8956807050854e-06
1819 1.01694122349727e-05
1820 1.02382091426989e-05
1821 1.11791696326691e-05
1822 9.49767036217963e-06
1823 1.00692186606466e-05
1824 1.1501215340104e-05
1825 9.58937926043291e-06
1826 1.02762842288939e-05
1827 1.05465323940734e-05
1828 1.07894638858852e-05
1829 9.71222834778018e-06
1830 1.00031629699515e-05
1831 1.15867915155832e-05
1832 9.40954487305135e-06
1833 1.02734275060357e-05
1834 1.0932033546851e-05
1835 1.03329448393197e-05
1836 9.9652461358346e-06
1837 1.00612351161544e-05
1838 1.14723407023121e-05
1839 9.39199708227534e-06
1840 1.01748109955224e-05
1841 1.13066507765325e-05
1842 9.89132968243212e-06
1843 1.01783334685024e-05
1844 1.02882304418017e-05
1845 1.11544140963815e-05
1846 9.53552444116212e-06
1847 1.00511897471733e-05
1848 1.15601324068848e-05
1849 9.5548020908609e-06
1850 1.02788580988999e-05
1851 1.06680872704601e-05
1852 1.06844872789225e-05
1853 9.79133073997218e-06
1854 1.00126144388923e-05
1855 1.1595648174989e-05
1856 9.39827532420168e-06
1857 1.02380063253804e-05
1858 1.11142708192347e-05
1859 1.01614596133004e-05
1860 1.00649740488734e-05
1861 1.0156562893826e-05
1862 1.13680935101002e-05
1863 9.45323517953511e-06
1864 1.01066480056033e-05
1865 1.14848271550727e-05
1866 9.70980181591585e-06
1867 1.02452095234185e-05
1868 1.05087201518472e-05
1869 1.09123093352537e-05
1870 9.68457698036218e-06
1871 1.00140223366907e-05
1872 1.16338778752834e-05
1873 9.44428666116437e-06
1874 1.02599542515236e-05
1875 1.09912170955795e-05
1876 1.03407910501119e-05
1877 9.98702216747915e-06
1878 1.01048171927687e-05
1879 1.14742424557335e-05
1880 9.42941096582217e-06
1881 1.01374052974279e-05
1882 1.14358144855942e-05
1883 9.81172706815414e-06
1884 1.02156873253989e-05
1885 1.04440014183638e-05
1886 1.10242881419254e-05
1887 9.64215632848209e-06
1888 1.00222187029431e-05
1889 1.16503679237212e-05
1890 9.47871103562647e-06
1891 1.02612548289471e-05
1892 1.09575585156563e-05
1893 1.04125947473221e-05
1894 9.96182552626124e-06
1895 1.00995212051203e-05
1896 1.15109933176427e-05
1897 9.43333907343913e-06
1898 1.01402129075723e-05
1899 1.14451404442661e-05
1900 9.83483914751559e-06
1901 1.0210292202828e-05
1902 1.04606478998903e-05
1903 1.10329810922849e-05
1904 9.65333401836688e-06
1905 1.00236356956884e-05
1906 1.16701494334848e-05
1907 9.48104388953652e-06
1908 1.02524008980254e-05
1909 1.10151095213951e-05
1910 1.03757720353315e-05
1911 9.98980158328777e-06
1912 1.01328560049296e-05
1913 1.14888161988347e-05
1914 9.45590545597952e-06
1915 1.01178438853822e-05
1916 1.15147631731816e-05
1917 9.77860054263147e-06
1918 1.02280600913218e-05
1919 1.05588314909255e-05
1920 1.09402126327041e-05
1921 9.71530062088277e-06
1922 1.0025421033788e-05
1923 1.16840328701073e-05
1924 9.4589249783894e-06
1925 1.0228885912511e-05
1926 1.11591662061983e-05
1927 1.02367048384622e-05
1928 1.00622755780932e-05
1929 1.02212261481327e-05
1930 1.13915111796814e-05
1931 9.50940557231661e-06
1932 1.00763882073807e-05
1933 1.16209985208116e-05
1934 9.66392508416902e-06
1935 1.02501217043027e-05
1936 1.07492369352258e-05
1937 1.07397681858856e-05
1938 9.8300688478048e-06
1939 1.00555726021412e-05
1940 1.16528717626352e-05
1941 9.44475777941989e-06
1942 1.01787672974751e-05
1943 1.13716550913523e-05
1944 1.00144507086952e-05
1945 1.01579735201085e-05
1946 1.03988750197459e-05
1947 1.11867548184819e-05
1948 9.61836940405192e-06
1949 1.00371416920098e-05
1950 1.17103809316177e-05
1951 9.53574271989055e-06
1952 1.02442481875187e-05
1953 1.10350811155513e-05
1954 1.0432799172122e-05
1955 9.98842551780399e-06
1956 1.01657697086921e-05
1957 1.15130469566793e-05
1958 9.48799515754217e-06
1959 1.01008672572789e-05
1960 1.15994916995987e-05
1961 9.75450075202389e-06
1962 1.02331223388319e-05
1963 1.06990755739389e-05
1964 1.08462245407281e-05
1965 9.79782817012165e-06
1966 1.00553534139181e-05
1967 1.16943847388029e-05
1968 9.46391446632333e-06
1969 1.01806172096985e-05
1970 1.13805554065038e-05
1971 1.00542774816859e-05
1972 1.01500818345812e-05
1973 1.04204891613335e-05
1974 1.12031530079548e-05
1975 9.63391903496813e-06
1976 1.00394518085523e-05
1977 1.17395475172088e-05
1978 9.54071037995163e-06
1979 1.02319982033805e-05
1980 1.11145973278326e-05
1981 1.03867623693077e-05
1982 1.0021715752373e-05
1983 1.02229496405926e-05
1984 1.14752510853577e-05
1985 9.52487425820436e-06
1986 1.00781035143882e-05
1987 1.16773899208056e-05
1988 9.69582833931781e-06
1989 1.02375224741991e-05
1990 1.08502399598365e-05
1991 1.07123760244576e-05
1992 9.87960265774745e-06
1993 1.01056048151804e-05
1994 1.16548490041168e-05
1995 9.4800079750712e-06
1996 1.0134791409655e-05
1997 1.15437069325708e-05
1998 9.90044645732269e-06
1999 1.01996447483543e-05
};
\addlegendentry{Test}

\nextgroupplot[
title={Leaky/Leaky},
ymin=4.42458717768711e-06, ymax=0.001,
]
\addplot [semithick, black, dashed]
table {%
0 0.0324028624454513
1 0.0315859417896718
2 0.0307781375013292
3 0.0299706895602867
4 0.0291532761184499
5 0.0283061427180655
6 0.0273825976764783
7 0.0263119075680152
8 0.0249915274325758
9 0.0234845610684715
10 0.022059433045797
11 0.0207231173990294
12 0.0195336704491638
13 0.0184753039502539
14 0.017529676144477
15 0.0166786473128013
16 0.0158992739743553
17 0.0151765971968416
18 0.0144963497587014
19 0.0138447370263748
20 0.0132157597690821
21 0.0125807316508144
22 0.0119582111074124
23 0.0113585430954117
24 0.010771713568829
25 0.0101943282643333
26 0.00962657039053738
27 0.00907135686429683
28 0.00853504589758813
29 0.0080270049511455
30 0.00755756255239248
31 0.00713727528636809
32 0.00676299103361089
33 0.00642959932156373
34 0.00612665983499028
35 0.00584717436868232
36 0.0055859457643237
37 0.00533995308796875
38 0.00510728689550888
39 0.00488646476878785
40 0.00467607036989648
41 0.00447531740064733
42 0.0042827079523704
43 0.00409840253269067
44 0.00392121494223829
45 0.00375089171575382
46 0.00358696701732697
47 0.00342917231319007
48 0.00327674930304056
49 0.00312968323851237
50 0.0029876003973186
51 0.0028502312998171
52 0.00271739813615568
53 0.0025888614836731
54 0.00246449370024493
55 0.00234416910097934
56 0.00222768499952508
57 0.00211493409733521
58 0.00200594685884425
59 0.00190060279055615
60 0.00179903263051528
61 0.00170072964101564
62 0.00160610006423667
63 0.00151599642413203
64 0.00143046234734356
65 0.0013493426667992
66 0.00127237894048449
67 0.00119961040036287
68 0.00113224869346595
69 0.00107040375405631
70 0.0010139606474695
71 0.000962901040111319
72 0.000917024202863104
73 0.000876071109814802
74 0.000839780033857096
75 0.000807481441370328
76 0.000778771907789633
77 0.00075321078656998
78 0.000730070209101541
79 0.000709117071892251
80 0.00068977849696239
81 0.000671438183417195
82 0.000653005045023747
83 0.000634834830634645
84 0.000617217277977034
85 0.000600143481278792
86 0.000583650916269107
87 0.000567847736419935
88 0.000552562086340913
89 0.00053750350343762
90 0.000522983435985225
91 0.000508585067109379
92 0.000494383471959736
93 0.000480548369523603
94 0.000467097713226394
95 0.000454025834187632
96 0.000441268597569433
97 0.000427769453381188
98 0.000413847032177728
99 0.000400647501464846
100 0.000388741296319495
101 0.000377979799850436
102 0.000368124256510782
103 0.000358906204382947
104 0.000351219896856492
105 0.000344464937370503
106 0.000338269953772397
107 0.000332408584881705
108 0.000327356971865811
109 0.000322681677062064
110 0.000318418717597524
111 0.000314476777248274
112 0.000310928074668482
113 0.000307673992892887
114 0.000304598992670435
115 0.000301725694043853
116 0.000299057571510275
117 0.000296578150482674
118 0.00029420413943626
119 0.000292019810331112
120 0.000289886226028102
121 0.000287907226379502
122 0.000286044978679456
123 0.000284308073219108
124 0.000282663154735019
125 0.000281091081205886
126 0.000279587227623779
127 0.000278160495554403
128 0.000276634708257006
129 0.000275342797294797
130 0.000274055485192548
131 0.000272859352662636
132 0.000271642566190167
133 0.000270521792799627
134 0.000269360816218978
135 0.000268286603727574
136 0.000267242005747903
137 0.000266184534439162
138 0.000265215700437693
139 0.000264177211533934
140 0.000263149842908206
141 0.000262210784171657
142 0.000261273741671175
143 0.000260328334888982
144 0.000259535267332467
145 0.000258590593830377
146 0.000257755847997032
147 0.000256849395896097
148 0.000256157147305203
149 0.000255232086715296
150 0.000254493342026763
151 0.000253636993420514
152 0.000252927680662651
153 0.000252100871648508
154 0.000251393133680722
155 0.000250675839993164
156 0.000249915208257789
157 0.000249226774542421
158 0.000248426593827844
159 0.000247855222880844
160 0.000247056877242358
161 0.000246431708546879
162 0.000245690344229388
163 0.000245078143393584
164 0.000244328959979612
165 0.000243811164352792
166 0.000243033893355005
167 0.000242454186832219
168 0.000241732953668361
169 0.000241117254688561
170 0.000240519230771952
171 0.000239821561365261
172 0.000239301048338803
173 0.000238563517484636
174 0.000238037138842628
175 0.000237387076538198
176 0.000236787219193957
177 0.000236176781186259
178 0.000235632808312403
179 0.000234923458265257
180 0.000234401872603485
181 0.000233705480411572
182 0.000233164226074223
183 0.000232518393204373
184 0.000231934036037273
185 0.000231317740883696
186 0.000230648476645001
187 0.000229658007867783
188 0.00022912845929568
189 0.000228447308188606
190 0.000227899883554983
191 0.000227196503999494
192 0.000226708142747611
193 0.000225984844234972
194 0.000225414605665719
195 0.000224791686775916
196 0.000224217966888318
197 0.000223521933662596
198 0.00022291008576758
199 0.000222328779159398
200 0.000221622783556086
201 0.000221030778845943
202 0.00022038040259531
203 0.000219802498349964
204 0.000219100947219886
205 0.000218578094518307
206 0.000217819479644277
207 0.000217189301395138
208 0.00021660594666173
209 0.000215922224356291
210 0.000215330771027311
211 0.000214637253350247
212 0.000214027026117947
213 0.000213368661150071
214 0.000212772969973685
215 0.0002120763305129
216 0.000211443084651819
217 0.000210816075252751
218 0.000210117028927925
219 0.000209551326747714
220 0.000208803855628048
221 0.000208188851559044
222 0.000207576278512533
223 0.000206841880412867
224 0.000206248973199763
225 0.000205586915967615
226 0.000204926343940315
227 0.000204309331536479
228 0.000203612538427933
229 0.000202943993258486
230 0.000202261405320314
231 0.000201678564110352
232 0.000200926473496565
233 0.000200286365668489
234 0.000199676181637187
235 0.00019896332571534
236 0.000198255795254454
237 0.000197563808853829
238 0.000196799745623366
239 0.000196121944611605
240 0.000195337503981818
241 0.00019457936565459
242 0.000193849002300794
243 0.000192998268516931
244 0.000192320560614689
245 0.000191454041669203
246 0.000190866211994489
247 0.000189981148167817
248 0.000189337928844679
249 0.000188475881884642
250 0.000187858213990921
251 0.000187041642249142
252 0.000186370261218372
253 0.000185551132176442
254 0.000184877563810915
255 0.000184092494379229
256 0.000183403298279927
257 0.000182621459401844
258 0.000181946607085592
259 0.000181170226994709
260 0.000180546254597402
261 0.000179744210328181
262 0.000179025884193607
263 0.000178301771484257
264 0.00017758996003181
265 0.000176914630287683
266 0.000176145991389376
267 0.000175492381231379
268 0.000174709606994838
269 0.000173989280568776
270 0.000173241531513213
271 0.000172544254979812
272 0.000171801337955912
273 0.000170989982478886
274 0.000170255540041353
275 0.000169445968936088
276 0.000168739631590142
277 0.000167944060933678
278 0.000167173361546702
279 0.000166403024337569
280 0.000165639835529419
281 0.000164889308024385
282 0.00016406626684784
283 0.000163350213881586
284 0.00016256596896369
285 0.000161748737667722
286 0.000160916823119805
287 0.000160127116544118
288 0.000159353712774646
289 0.000158642449008539
290 0.000157764515705594
291 0.000157012170035387
292 0.000156216553847344
293 0.000155430484994667
294 0.000154687048052438
295 0.000153933703785469
296 0.000153116860332148
297 0.000152295624360477
298 0.000151531456140219
299 0.000150719414961031
300 0.000149946490523689
301 0.000149119956176946
302 0.000148342257972445
303 0.000147552871737844
304 0.000146776361702905
305 0.000146010653850226
306 0.000145217314923229
307 0.000144417000541353
308 0.000143668392695417
309 0.000142921748590652
310 0.000142167660925452
311 0.000141391865241758
312 0.000140624924199528
313 0.000139913518466983
314 0.000139154377336581
315 0.000138434057618042
316 0.000137657791867696
317 0.000136910830576653
318 0.000136179074985421
319 0.000135460942743748
320 0.000134737986996925
321 0.000134047796677805
322 0.000133297511922592
323 0.000132730095799616
324 0.000131963349531361
325 0.000131249373680475
326 0.000130519469081491
327 0.000129932629107543
328 0.000129265599241535
329 0.000128572633911972
330 0.000127818560315518
331 0.000127050844412224
332 0.00012633344238111
333 0.000125527451302787
334 0.000124830774353768
335 0.000124043973272592
336 0.000123320643496072
337 0.000122616096234651
338 0.000121992404331195
339 0.000121263598828136
340 0.00012062788951539
341 0.000119945686407164
342 0.000119386688510303
343 0.000118693119929958
344 0.000117983976196001
345 0.000117333701524558
346 0.00011673808580781
347 0.000115763492345877
348 0.000114737459767866
349 0.000114230392824766
350 0.000113686678787417
351 0.000112990054134343
352 0.000112494362582538
353 0.000111745948629505
354 0.000111103995237727
355 0.000110326938795424
356 0.000109664502957685
357 0.000108978179753194
358 0.000108166692399436
359 0.000107476959243513
360 0.000106628591396429
361 0.000105976264791252
362 0.000105190844550407
363 0.000104641855756427
364 0.000103835485106174
365 0.00010323435412829
366 0.000102507745907587
367 0.000101886253560224
368 0.000101321047083047
369 0.000100814010636441
370 0.000100165033217081
371 9.96671776931635e-05
372 9.91750820560355e-05
373 9.85280107386188e-05
374 9.80683189197862e-05
375 9.74078923263733e-05
376 9.69121124967387e-05
377 9.64269169685394e-05
378 9.5795575447255e-05
379 9.52843735149145e-05
380 9.46593116566419e-05
381 9.42126366965113e-05
382 9.36674369427237e-05
383 9.31034433619971e-05
384 9.2627184272942e-05
385 9.212809300152e-05
386 9.15598900235182e-05
387 9.11312367577466e-05
388 9.05136729443257e-05
389 8.99933646678619e-05
390 8.94763527696796e-05
391 8.89099247558534e-05
392 8.84329560051356e-05
393 8.7967338913586e-05
394 8.7461517466636e-05
395 8.69502153761914e-05
396 8.65512853067685e-05
397 8.6023099271415e-05
398 8.55423784855702e-05
399 8.50371446006193e-05
400 8.46838346433287e-05
401 8.41537199960385e-05
402 8.36143740912121e-05
403 8.31075383018742e-05
404 8.27018139375468e-05
405 8.21159177633035e-05
406 8.16762077135991e-05
407 8.11779451908023e-05
408 8.06623859119782e-05
409 8.02691397723265e-05
410 7.97964186176614e-05
411 7.9253729438733e-05
412 7.88536274853868e-05
413 7.84252756176329e-05
414 7.79400090493709e-05
415 7.75461805062605e-05
416 7.70812108896735e-05
417 7.666326783351e-05
418 7.62626470418581e-05
419 7.57584875827888e-05
420 7.53318305015682e-05
421 7.48938418269063e-05
422 7.45185542001536e-05
423 7.40969321100238e-05
424 7.37403181716445e-05
425 7.33706626903086e-05
426 7.28998481491772e-05
427 7.24840395207593e-05
428 7.20045703417327e-05
429 7.15449747872299e-05
430 7.11220633888843e-05
431 7.06566564616651e-05
432 7.02941792241063e-05
433 6.98850815155083e-05
434 6.94725383709738e-05
435 6.9086846210098e-05
436 6.86725827421242e-05
437 6.82749459031129e-05
438 6.79183782352766e-05
439 6.75293631502427e-05
440 6.71462416903523e-05
441 6.68459994415116e-05
442 6.64649556085806e-05
443 6.61103799330931e-05
444 6.57868156253016e-05
445 6.54434328168918e-05
446 6.51125917130457e-05
447 6.48599599628596e-05
448 6.45224813382583e-05
449 6.41917246184676e-05
450 6.3876305759436e-05
451 6.36087201826285e-05
452 6.32894387138094e-05
453 6.30125954614869e-05
454 6.26671354240216e-05
455 6.24644535776042e-05
456 6.20942380891165e-05
457 6.17520803132265e-05
458 6.15014331799557e-05
459 6.11735599704843e-05
460 6.08672792310472e-05
461 6.0552644413292e-05
462 6.02768019888344e-05
463 5.99331786048651e-05
464 5.9752988619266e-05
465 5.94064031389507e-05
466 5.90663766928401e-05
467 5.87997845116206e-05
468 5.8511127704719e-05
469 5.82109500015804e-05
470 5.79943185385901e-05
471 5.77009592888089e-05
472 5.74729660201001e-05
473 5.71426984805612e-05
474 5.69637959699776e-05
475 5.672228677156e-05
476 5.6429988106288e-05
477 5.61465286352814e-05
478 5.5958060499961e-05
479 5.57368149145532e-05
480 5.5440070624968e-05
481 5.52609871249388e-05
482 5.50236864711451e-05
483 5.4819880034529e-05
484 5.45319687574874e-05
485 5.43590528394589e-05
486 5.41585613262896e-05
487 5.39058904962531e-05
488 5.33130396291881e-05
489 5.32445339729293e-05
490 5.244172176333e-05
491 5.23157550560427e-05
492 5.17556526347107e-05
493 5.14189844267321e-05
494 5.110319079904e-05
495 5.06868745731026e-05
496 5.04800205476386e-05
497 5.00252255477562e-05
498 4.97791082167964e-05
499 4.95908027389191e-05
500 4.92290221316694e-05
501 4.90405118895865e-05
502 4.85114789796626e-05
503 4.78057398112242e-05
504 4.70371851406526e-05
505 4.63624671453999e-05
506 4.58311935318534e-05
507 4.53150880090902e-05
508 4.48934017072133e-05
509 4.44579033569426e-05
510 4.40987365806222e-05
511 4.3740906519929e-05
512 4.34303297964789e-05
513 4.3113106599435e-05
514 4.28385519057883e-05
515 4.25143872888611e-05
516 4.22458732032283e-05
517 4.1965425381818e-05
518 4.17104840551019e-05
519 4.14461792104248e-05
520 4.12024165257208e-05
521 4.09763582638334e-05
522 4.0781916005983e-05
523 4.05363758488875e-05
524 4.02932393726019e-05
525 4.01013705371156e-05
526 3.98429454762095e-05
527 3.96614495912218e-05
528 3.94385185629176e-05
529 3.92497694363669e-05
530 3.90329303385784e-05
531 3.88744869042057e-05
532 3.8689468546238e-05
533 3.85227587003101e-05
534 3.8355608118934e-05
535 3.81944743708118e-05
536 3.80660679297762e-05
537 3.79191449297878e-05
538 3.78108713476877e-05
539 3.7683886631612e-05
540 3.75686498088612e-05
541 3.74581183351097e-05
542 3.73576021743816e-05
543 3.72632714871202e-05
544 3.71530646461338e-05
545 3.70589055478376e-05
546 3.69494374936608e-05
547 3.68755121371578e-05
548 3.67788576340899e-05
549 3.66741270667603e-05
550 3.65807176478938e-05
551 3.64680176545562e-05
552 3.63588474954213e-05
553 3.62590930507167e-05
554 3.61381294737839e-05
555 3.60321722325807e-05
556 3.58882789655013e-05
557 3.57776120267772e-05
558 3.56533459964226e-05
559 3.55135203591317e-05
560 3.5369762343862e-05
561 3.52176847258079e-05
562 3.50905810293511e-05
563 3.49488290964928e-05
564 3.48016836397846e-05
565 3.46503778914098e-05
566 3.45118072146988e-05
567 3.43559620645806e-05
568 3.42109063353746e-05
569 3.40743674200894e-05
570 3.39329345493411e-05
571 3.37874387383863e-05
572 3.36597461725319e-05
573 3.35207149575467e-05
574 3.34035598044125e-05
575 3.32768852189247e-05
576 3.31669771682641e-05
577 3.30645875123992e-05
578 3.29627997217585e-05
579 3.28601370327419e-05
580 3.27686315344522e-05
581 3.26996988775363e-05
582 3.26051092329749e-05
583 3.25364475486367e-05
584 3.24635594566303e-05
585 3.2395655154005e-05
586 3.23763180034575e-05
587 3.23188437576505e-05
588 3.22595943273996e-05
589 3.22143994537782e-05
590 3.2062796584853e-05
591 3.19690344880996e-05
592 3.1868362263765e-05
593 3.17960813447371e-05
594 3.16780895914803e-05
595 3.15741901459887e-05
596 3.14669033159021e-05
597 3.13521606969402e-05
598 3.12316137112134e-05
599 3.10980723838838e-05
600 3.09662340640671e-05
601 3.08202418750625e-05
602 3.06912434879791e-05
603 3.05337825992069e-05
604 3.04117533644899e-05
605 3.0264682614245e-05
606 3.01067820593914e-05
607 2.9980792845663e-05
608 2.97846801089463e-05
609 2.96129085057828e-05
610 2.94720756173206e-05
611 2.93471781063204e-05
612 2.92391683842652e-05
613 2.91394225655495e-05
614 2.90510866349791e-05
615 2.89657181760106e-05
616 2.88850230347748e-05
617 2.88164080330944e-05
618 2.87483956675771e-05
619 2.8702792576496e-05
620 2.8657010272326e-05
621 2.86202258230617e-05
622 2.85868289395808e-05
623 2.85606697545404e-05
624 2.85366963446165e-05
625 2.8532979616358e-05
626 2.85154649795061e-05
627 2.85160174620103e-05
628 2.84850577862983e-05
629 2.84821424152426e-05
630 2.84531350800421e-05
631 2.84089080651029e-05
632 2.83544472132746e-05
633 2.82962657429664e-05
634 2.8213441236602e-05
635 2.81104408834665e-05
636 2.80194970905256e-05
637 2.78867824512474e-05
638 2.77582791774478e-05
639 2.75843871833104e-05
640 2.74326506399802e-05
641 2.72565679217962e-05
642 2.70737687451117e-05
643 2.68657449566945e-05
644 2.66839963956045e-05
645 2.64743815669988e-05
646 2.6278502240018e-05
647 2.60747977733899e-05
648 2.58732413982443e-05
649 2.56594415404265e-05
650 2.54893436810733e-05
651 2.52869143189116e-05
652 2.51262544885833e-05
653 2.49490729444801e-05
654 2.47982249277356e-05
655 2.4659826880935e-05
656 2.45228953659193e-05
657 2.44139135077148e-05
658 2.43128577785967e-05
659 2.42140229289589e-05
660 2.41411855554929e-05
661 2.4087325563471e-05
662 2.4030651729845e-05
663 2.40056851055215e-05
664 2.39962840566932e-05
665 2.3986652227137e-05
666 2.40131865822946e-05
667 2.40512033933271e-05
668 2.4093971550343e-05
669 2.41631640669482e-05
670 2.42605226894455e-05
671 2.43522224998571e-05
672 2.44984604584886e-05
673 2.46107220434055e-05
674 2.48127919597607e-05
675 2.50605888538757e-05
676 2.52547796684155e-05
677 2.54906202989957e-05
678 2.56118249026827e-05
679 2.58084180693174e-05
680 2.58703740492194e-05
681 2.59798858444071e-05
682 2.58570064985975e-05
683 2.57968387451513e-05
684 2.55171764536044e-05
685 2.52107688147873e-05
686 2.47747010782007e-05
687 2.43467187210911e-05
688 2.3893751759374e-05
689 2.34579360665066e-05
690 2.30645502661275e-05
691 2.27180256224813e-05
692 2.24300119242571e-05
693 2.22093251451838e-05
694 2.20893286417834e-05
695 2.20425493395737e-05
696 2.21109059097557e-05
697 2.22376376690647e-05
698 2.24505797632446e-05
699 2.27530654193231e-05
700 2.30950278243824e-05
701 2.35149461325079e-05
702 2.39685360270414e-05
703 2.45251341688402e-05
704 2.51291166577516e-05
705 2.57040929518837e-05
706 2.63397407742616e-05
707 2.68654177375538e-05
708 2.74074342208053e-05
709 2.75671615632689e-05
710 2.77070525491752e-05
711 2.71935547075941e-05
712 2.63582212198799e-05
713 2.52686921875522e-05
714 2.4070318097813e-05
715 2.28066721668085e-05
716 2.18493973456191e-05
717 2.11073238318704e-05
718 2.06400909341653e-05
719 2.04586073380142e-05
720 2.0488199762525e-05
721 2.06879253106251e-05
722 2.11871002804287e-05
723 2.18692716877911e-05
724 2.27292314853855e-05
725 2.3690808767185e-05
726 2.454304398114e-05
727 2.51539861260142e-05
728 2.53173556501451e-05
729 2.48357246235287e-05
730 2.38080224441717e-05
731 2.25072969257667e-05
732 2.12679946933747e-05
733 2.03557611193617e-05
734 1.98269003455209e-05
735 1.96480647964847e-05
736 1.96921213486689e-05
737 1.98327464531189e-05
738 1.99856871319071e-05
739 2.01152022505369e-05
740 2.01236476922162e-05
741 2.00612706766634e-05
742 1.98780928286624e-05
743 1.96144688828781e-05
744 1.92894095718543e-05
745 1.89402601513677e-05
746 1.85932808758338e-05
747 1.82886837230001e-05
748 1.80545193675385e-05
749 1.79082400215691e-05
750 1.78486073423301e-05
751 1.78939946451351e-05
752 1.80313839237556e-05
753 1.82250765377034e-05
754 1.85640918246577e-05
755 1.89726452326511e-05
756 1.9498937543716e-05
757 2.01647635371671e-05
758 2.09610853367792e-05
759 2.18697459040129e-05
760 2.27818307916294e-05
761 2.36574663894373e-05
762 2.41999758010536e-05
763 2.42788807014449e-05
764 2.36504691031314e-05
765 2.24615218407109e-05
766 2.1148104170976e-05
767 2.00015263160935e-05
768 1.9257210718493e-05
769 1.90131378161595e-05
770 1.92447593718725e-05
771 1.98292356365926e-05
772 2.06291072153419e-05
773 2.14982536235198e-05
774 2.22758375727494e-05
775 2.27617057255713e-05
776 2.28520902627594e-05
777 2.25038593413274e-05
778 2.1718741017196e-05
779 2.06067446359981e-05
780 1.93979200222572e-05
781 1.83646095592849e-05
782 1.76148387289743e-05
783 1.72506011519857e-05
784 1.7203672538102e-05
785 1.74202744211982e-05
786 1.78090496305572e-05
787 1.83592029898705e-05
788 1.90229836061206e-05
789 1.9721343473833e-05
790 2.02885458691071e-05
791 2.06487361964491e-05
792 2.06065369514619e-05
793 2.01908631591863e-05
794 1.94674357558711e-05
795 1.86159946800046e-05
796 1.7827753254096e-05
797 1.72427721647495e-05
798 1.69380382040174e-05
799 1.68655600560896e-05
800 1.69944342474082e-05
801 1.72363966246536e-05
802 1.75350774256344e-05
803 1.78077290762246e-05
804 1.80135678480298e-05
805 1.81293125223192e-05
806 1.81203922515749e-05
807 1.79714009114207e-05
808 1.77007476143132e-05
809 1.73436529919968e-05
810 1.69288530162959e-05
811 1.65205613633645e-05
812 1.62067597386795e-05
813 1.59509678780978e-05
814 1.58223863007123e-05
815 1.57982105726262e-05
816 1.5884686430212e-05
817 1.61128187698978e-05
818 1.64875220249883e-05
819 1.70033670130465e-05
820 1.77869735011882e-05
821 1.88188870495765e-05
822 2.00908999534732e-05
823 2.15112705834741e-05
824 2.29350859157762e-05
825 2.39975355835753e-05
826 2.42316285952171e-05
827 2.3286557876645e-05
828 2.14464596908215e-05
829 1.95183303297597e-05
830 1.8131407385269e-05
831 1.76201599302317e-05
832 1.78869768454604e-05
833 1.86665434647182e-05
834 1.96737946254189e-05
835 2.05877513330677e-05
836 2.11131095646522e-05
837 2.10369001152344e-05
838 2.03909334448582e-05
839 1.92743781557425e-05
840 1.79883000424041e-05
841 1.68010991572487e-05
842 1.59097330261204e-05
843 1.54106844139079e-05
844 1.53059507823627e-05
845 1.55590123398497e-05
846 1.59997777959831e-05
847 1.66079584804635e-05
848 1.7250395645263e-05
849 1.78021759431601e-05
850 1.81597412094447e-05
851 1.81524543414469e-05
852 1.77983642721813e-05
853 1.7176236376315e-05
854 1.64669405471329e-05
855 1.58217379162551e-05
856 1.53608438857589e-05
857 1.50981509356996e-05
858 1.50363746715954e-05
859 1.51025674348659e-05
860 1.52728442408545e-05
861 1.54642329057708e-05
862 1.56741418049222e-05
863 1.58272897508027e-05
864 1.59104306423075e-05
865 1.59059497235603e-05
866 1.58102056193599e-05
867 1.56232390882138e-05
868 1.53716549373684e-05
869 1.50778228427129e-05
870 1.47841967095275e-05
871 1.45490076910448e-05
872 1.43648277521891e-05
873 1.42761959356363e-05
874 1.42577169661706e-05
875 1.43266160410604e-05
876 1.45071362815941e-05
877 1.47857665098172e-05
878 1.52395402182037e-05
879 1.58829384417913e-05
880 1.6805538569642e-05
881 1.80347423075489e-05
882 1.94957406858265e-05
883 2.11285434321695e-05
884 2.27101023657639e-05
885 2.37944735737017e-05
886 2.37424797706609e-05
887 2.23610181322442e-05
888 2.00561094736429e-05
889 1.79959757602433e-05
890 1.69464828054799e-05
891 1.70816700446608e-05
892 1.80717869113778e-05
893 1.94620227702558e-05
894 2.08018686080891e-05
895 2.15346148220164e-05
896 2.1470284275793e-05
897 2.05515336428874e-05
898 1.89917521051797e-05
899 1.72701909058048e-05
900 1.57390430572946e-05
901 1.4713450449122e-05
902 1.42454490337229e-05
903 1.43110264367863e-05
904 1.47138313213446e-05
905 1.53074195008429e-05
906 1.58640736405147e-05
907 1.62604829299084e-05
908 1.63569286666387e-05
909 1.61057471412818e-05
910 1.55659243947781e-05
911 1.4936456281589e-05
912 1.4358055780761e-05
913 1.39525179374544e-05
914 1.37211902728751e-05
915 1.36628864861876e-05
916 1.36987047643089e-05
917 1.3781141495528e-05
918 1.38918622134554e-05
919 1.3964416046619e-05
920 1.39812456225741e-05
921 1.39690314586716e-05
922 1.38882907450011e-05
923 1.37491552525404e-05
924 1.35901948885575e-05
925 1.33975545457687e-05
926 1.32335982785392e-05
927 1.31050230960383e-05
928 1.30284311765649e-05
929 1.30280331234189e-05
930 1.30735819112715e-05
931 1.32059882300695e-05
932 1.34194812595467e-05
933 1.37458868447737e-05
934 1.41815161178727e-05
935 1.47841442377228e-05
936 1.55925635585419e-05
937 1.66191512658287e-05
938 1.78305618447894e-05
939 1.91493542267551e-05
940 2.03312465858474e-05
941 2.10563780704831e-05
942 2.09787302782871e-05
943 1.98793618348958e-05
944 1.81173016287062e-05
945 1.65037692863024e-05
946 1.56600516705296e-05
947 1.57731206922307e-05
948 1.67299507296548e-05
949 1.82071235634851e-05
950 1.98146070982119e-05
951 2.11031924326477e-05
952 2.17053867661576e-05
953 2.12323733261854e-05
954 1.97440294336104e-05
955 1.76465490184796e-05
956 1.558740675911e-05
957 1.4155166710772e-05
958 1.34672581282302e-05
959 1.34800011899472e-05
960 1.39066656057452e-05
961 1.45818564973865e-05
962 1.52067138881051e-05
963 1.56396137036552e-05
964 1.57065627348629e-05
965 1.53231823283306e-05
966 1.46287719395843e-05
967 1.38456662650555e-05
968 1.3206472475602e-05
969 1.28121192233266e-05
970 1.26711026355864e-05
971 1.2732731910603e-05
972 1.2876798555439e-05
973 1.30303418899302e-05
974 1.31704013588418e-05
975 1.32139276578513e-05
976 1.31582452658208e-05
977 1.30102037996238e-05
978 1.27974305863532e-05
979 1.25437099001147e-05
980 1.23119487955847e-05
981 1.21114470221784e-05
982 1.19971518284245e-05
983 1.19969714083012e-05
984 1.20722187819666e-05
985 1.22672527655965e-05
986 1.25659737548744e-05
987 1.2982724996391e-05
988 1.35381857901962e-05
989 1.427127881648e-05
990 1.51018478469211e-05
991 1.60406934890389e-05
992 1.69163488861201e-05
993 1.7517675752643e-05
994 1.76359763353062e-05
995 1.71226422391868e-05
996 1.60940354909656e-05
997 1.49458241978095e-05
998 1.4015442062032e-05
999 1.35872766326983e-05
1000 1.37456755755494e-05
1001 1.4448852777349e-05
1002 1.55477534597992e-05
1003 1.68526508730338e-05
1004 1.80640095681639e-05
1005 1.88798551734592e-05
1006 1.90000181294891e-05
1007 1.82906904599989e-05
1008 1.68798537600878e-05
1009 1.52187308337659e-05
1010 1.37353217031233e-05
1011 1.28195565203271e-05
1012 1.25723296573454e-05
1013 1.28279343805104e-05
1014 1.34909396702909e-05
1015 1.44292250752187e-05
1016 1.53883286433398e-05
1017 1.61798002995539e-05
1018 1.64620989542286e-05
1019 1.60569397351651e-05
1020 1.50395919504831e-05
1021 1.38127673245947e-05
1022 1.28173867111059e-05
1023 1.23180748010654e-05
1024 1.22948833212178e-05
1025 1.25880793842015e-05
1026 1.29880967012497e-05
1027 1.33331363354472e-05
1028 1.35441482207455e-05
1029 1.35314918301432e-05
1030 1.3291670916793e-05
1031 1.28764323754638e-05
1032 1.23820174122713e-05
1033 1.1907298837599e-05
1034 1.15274493577289e-05
1035 1.13419654601543e-05
1036 1.13611169396677e-05
1037 1.15812948386917e-05
1038 1.19784499457154e-05
1039 1.25189588917607e-05
1040 1.32178127056903e-05
1041 1.40030368802613e-05
1042 1.48105985253721e-05
1043 1.54513695873604e-05
1044 1.57389096875349e-05
1045 1.55383100119977e-05
1046 1.4873987479902e-05
1047 1.39030217933822e-05
1048 1.29701521345282e-05
1049 1.23678758869872e-05
1050 1.22057251203955e-05
1051 1.2442058432427e-05
1052 1.30018107107333e-05
1053 1.37496721315244e-05
1054 1.4543049498883e-05
1055 1.5212819768351e-05
1056 1.55990319168353e-05
1057 1.56045722192744e-05
1058 1.51396928940173e-05
1059 1.4316635459366e-05
1060 1.33141272007364e-05
1061 1.24012352786806e-05
1062 1.1807910544448e-05
1063 1.16233836902069e-05
1064 1.17680837536938e-05
1065 1.22253906162939e-05
1066 1.30419213526523e-05
1067 1.40947067883701e-05
1068 1.5388320644405e-05
1069 1.65551910358275e-05
1070 1.72256227362055e-05
1071 1.70098673635266e-05
1072 1.58731872019757e-05
1073 1.42688523521173e-05
1074 1.29084507154431e-05
1075 1.22874306560661e-05
1076 1.24232854652462e-05
1077 1.30707892456172e-05
1078 1.38615022819977e-05
1079 1.45217030622646e-05
1080 1.47774494063668e-05
1081 1.4554794082855e-05
1082 1.38977728596501e-05
1083 1.30080934583887e-05
1084 1.20914511931858e-05
1085 1.13738440958855e-05
1086 1.10292845336701e-05
1087 1.10843922929149e-05
1088 1.14740875556407e-05
1089 1.21490563866899e-05
1090 1.29768319201418e-05
1091 1.38551233401785e-05
1092 1.46077520790655e-05
1093 1.49754588747442e-05
1094 1.47795861495936e-05
1095 1.40355337334697e-05
1096 1.30274776388717e-05
1097 1.20951434894323e-05
1098 1.15424002462561e-05
1099 1.1450551720138e-05
1100 1.17352086554945e-05
1101 1.22457023206124e-05
1102 1.27969755592261e-05
1103 1.32420323746274e-05
1104 1.34514678462594e-05
1105 1.33635707530999e-05
1106 1.3003826248692e-05
1107 1.24305182476192e-05
1108 1.17834663697636e-05
1109 1.12050411855691e-05
1110 1.08418761239903e-05
1111 1.07441883772097e-05
1112 1.0926823740931e-05
1113 1.13635578244953e-05
1114 1.20437126067685e-05
1115 1.2937236430588e-05
1116 1.40635417835711e-05
1117 1.52304182234886e-05
1118 1.61366806255003e-05
1119 1.63595861053523e-05
1120 1.57484803748886e-05
1121 1.44842718139415e-05
1122 1.30647470371059e-05
1123 1.21020692525065e-05
1124 1.18901121215487e-05
1125 1.23368035804816e-05
1126 1.32060836932624e-05
1127 1.41465180760747e-05
1128 1.48785037081822e-05
1129 1.513199245462e-05
1130 1.48229874055161e-05
1131 1.40522169402857e-05
1132 1.29618333097881e-05
1133 1.18929332577622e-05
1134 1.10875693417611e-05
1135 1.07640356290517e-05
1136 1.08934280795836e-05
1137 1.13889676907064e-05
1138 1.2222452603794e-05
1139 1.32157985941461e-05
1140 1.42119551638231e-05
1141 1.49412247871794e-05
1142 1.50593580592329e-05
1143 1.44556355472503e-05
1144 1.33287879773292e-05
1145 1.21571276228138e-05
1146 1.13594444597354e-05
1147 1.11289456699026e-05
1148 1.13945935407145e-05
1149 1.19308968109522e-05
1150 1.25261854293512e-05
1151 1.29655478362523e-05
1152 1.31086895143184e-05
1153 1.28957741676317e-05
1154 1.24014328815036e-05
1155 1.17383339333088e-05
1156 1.10700475177694e-05
1157 1.05511459018715e-05
1158 1.02990588182195e-05
1159 1.03763635070919e-05
1160 1.07316829218718e-05
1161 1.13638102039459e-05
1162 1.22206432937588e-05
1163 1.32271097053049e-05
1164 1.42699529268597e-05
1165 1.50467223525297e-05
1166 1.52231745200737e-05
1167 1.46511370067159e-05
1168 1.35213439236281e-05
1169 1.23014884128025e-05
1170 1.14522875502843e-05
1171 1.12187988330348e-05
1172 1.15599884722428e-05
1173 1.22494186296507e-05
1174 1.30651586225383e-05
1175 1.37345430069047e-05
1176 1.40522628884199e-05
1177 1.38857589817576e-05
1178 1.32937133097144e-05
1179 1.24198542126308e-05
1180 1.14774921442518e-05
1181 1.07134109974538e-05
1182 1.03487058682461e-05
1183 1.04205414235281e-05
1184 1.08355083678902e-05
1185 1.16070161553239e-05
1186 1.2619342767195e-05
1187 1.38343027016319e-05
1188 1.48685045866515e-05
1189 1.53379601348647e-05
1190 1.49601603993688e-05
1191 1.38343949860342e-05
1192 1.24294550314374e-05
1193 1.13691769882962e-05
1194 1.09910381453915e-05
1195 1.12602871347178e-05
1196 1.19439614323369e-05
1197 1.27037627093429e-05
1198 1.32815146525189e-05
1199 1.34544238523659e-05
1200 1.31313983811054e-05
1201 1.24490155046786e-05
1202 1.15786678840379e-05
1203 1.07486746667718e-05
1204 1.01857786170001e-05
1205 1.00155349178621e-05
1206 1.02505352348459e-05
1207 1.08174192980925e-05
1208 1.1679495161232e-05
1209 1.26920248346707e-05
1210 1.37450260409011e-05
1211 1.44235414971661e-05
1212 1.44735889975323e-05
1213 1.37635728556518e-05
1214 1.26060441001385e-05
1215 1.14541153841685e-05
1216 1.07544206251475e-05
1217 1.06747143879815e-05
1218 1.10880792547619e-05
1219 1.1760008725048e-05
1220 1.24561870924467e-05
1221 1.29184797685866e-05
1222 1.29957281496118e-05
1223 1.2653510732541e-05
1224 1.1995309930235e-05
1225 1.11886111877624e-05
1226 1.04535639486869e-05
1227 9.96699184163674e-06
1228 9.85969967093325e-06
1229 1.0147690906237e-05
1230 1.07059956260258e-05
1231 1.16088829695826e-05
1232 1.26895727223797e-05
1233 1.38435559149386e-05
1234 1.46668984655207e-05
1235 1.48090869247497e-05
1236 1.41188733469022e-05
1237 1.28650693582699e-05
1238 1.15808419476338e-05
1239 1.08089620782792e-05
1240 1.0734931639611e-05
1241 1.1241668854467e-05
1242 1.20565644081694e-05
1243 1.28608212435211e-05
1244 1.33946435108001e-05
1245 1.34155836004624e-05
1246 1.29153744481414e-05
1247 1.20711218274394e-05
1248 1.10892884332969e-05
1249 1.02801357559912e-05
1250 9.82833500096092e-06
1251 9.85600560987621e-06
1252 1.02733684812506e-05
1253 1.10478205375131e-05
1254 1.2076755590229e-05
1255 1.32188278758605e-05
1256 1.41074533681618e-05
1257 1.44130648838114e-05
1258 1.39014325943876e-05
1259 1.27652697354819e-05
1260 1.14834892581683e-05
1261 1.06190837350084e-05
1262 1.04007848102938e-05
1263 1.07831628817223e-05
1264 1.14816813976404e-05
1265 1.21968762689662e-05
1266 1.27197278025903e-05
1267 1.28064839888253e-05
1268 1.23845100015174e-05
1269 1.16477826552774e-05
1270 1.07896172156785e-05
1271 1.00291105002981e-05
1272 9.59098825870086e-06
1273 9.57221275488251e-06
1274 9.94968225676018e-06
1275 1.06452155019809e-05
1276 1.16123022539938e-05
1277 1.26699603884006e-05
1278 1.36196082340945e-05
1279 1.40671751549526e-05
1280 1.37829219646335e-05
1281 1.28260893292875e-05
1282 1.1550539094074e-05
1283 1.06111231472639e-05
1284 1.02096386740413e-05
1285 1.04727798539983e-05
1286 1.11475996824595e-05
1287 1.1930972859453e-05
1288 1.26117791952041e-05
1289 1.28767528169149e-05
1290 1.26097921704371e-05
1291 1.19133890397194e-05
1292 1.09910747703168e-05
1293 1.01306160029679e-05
1294 9.53623816135618e-06
1295 9.38331407773774e-06
1296 9.66081468334323e-06
1297 1.02635063505829e-05
1298 1.11762971979346e-05
1299 1.22333347327697e-05
1300 1.32629298619236e-05
1301 1.38720272024884e-05
1302 1.37610762269702e-05
1303 1.29203429288793e-05
1304 1.16809184014244e-05
1305 1.0605039688194e-05
1306 1.00949366088621e-05
1307 1.02516040225709e-05
1308 1.08786667141203e-05
1309 1.16936337590801e-05
1310 1.23613407252421e-05
1311 1.26792452750379e-05
1312 1.24952210804885e-05
1313 1.1829775655503e-05
1314 1.089705745283e-05
1315 1.00037076240511e-05
1316 9.39814815792417e-06
1317 9.25787917704923e-06
1318 9.5433680105117e-06
1319 1.02070407681509e-05
1320 1.11911089089034e-05
1321 1.22853243160748e-05
1322 1.33214223012601e-05
1323 1.38657459780589e-05
1324 1.36446772538434e-05
1325 1.26523321117844e-05
1326 1.13458758530527e-05
1327 1.03276521690177e-05
1328 9.95367264522429e-06
1329 1.02637270806838e-05
1330 1.09833663355374e-05
1331 1.17782483055606e-05
1332 1.23714705941858e-05
1333 1.24834361434623e-05
1334 1.21092442610937e-05
1335 1.13201923843675e-05
1336 1.0373221370763e-05
1337 9.55797023305394e-06
1338 9.13669894231361e-06
1339 9.19181954195381e-06
1340 9.61824613021633e-06
1341 1.04186794924743e-05
1342 1.14348040769485e-05
1343 1.25283081304772e-05
1344 1.33848275924997e-05
1345 1.36015736607042e-05
1346 1.29858018915741e-05
1347 1.18128757815583e-05
1348 1.06101047760987e-05
1349 9.88808764290638e-06
1350 9.87251354622742e-06
1351 1.04045402125053e-05
1352 1.11960768407471e-05
1353 1.19341760242975e-05
1354 1.23087532069022e-05
1355 1.22045348178723e-05
1356 1.1631360021358e-05
1357 1.07403186317967e-05
1358 9.83774130158821e-06
1359 9.17992965154824e-06
1360 8.96706850106455e-06
1361 9.21560093125606e-06
1362 9.82993584308645e-06
1363 1.07354865734877e-05
1364 1.1855456509835e-05
1365 1.29009539890923e-05
1366 1.34798317166229e-05
1367 1.32815200877268e-05
1368 1.23230572457445e-05
1369 1.10507056625408e-05
1370 1.00541698344614e-05
1371 9.69150387053475e-06
1372 1.00052734115508e-05
1373 1.0739813681937e-05
1374 1.15782067542369e-05
1375 1.2168865477058e-05
1376 1.22728334055466e-05
1377 1.19196081227457e-05
1378 1.11081550562275e-05
1379 1.01489865569349e-05
1380 9.32744588810763e-06
1381 8.90908604489482e-06
1382 9.0033456094929e-06
1383 9.52300998235955e-06
1384 1.0367731789529e-05
1385 1.14796375880744e-05
1386 1.25907999812824e-05
1387 1.32972716682112e-05
1388 1.32837742334857e-05
1389 1.24365374958302e-05
1390 1.11693450683958e-05
1391 1.00627629295857e-05
1392 9.553790824679e-06
1393 9.76819065989787e-06
1394 1.04569249490538e-05
1395 1.12894414616704e-05
1396 1.19024851459315e-05
1397 1.20805771999244e-05
1398 1.17532029446643e-05
1399 1.09785212605296e-05
1400 1.00329760313045e-05
1401 9.20936946213757e-06
1402 8.77199714466315e-06
1403 8.8495884051909e-06
1404 9.34512671157961e-06
1405 1.0135161994107e-05
1406 1.12469919919533e-05
1407 1.23648042968938e-05
1408 1.31378744843147e-05
1409 1.31709001860436e-05
1410 1.23586281857158e-05
1411 1.10914657547312e-05
1412 9.97415866610396e-06
1413 9.43723908086724e-06
1414 9.62084572719846e-06
1415 1.03068890027203e-05
1416 1.11316295017261e-05
1417 1.17932405068188e-05
1418 1.20123156452934e-05
1419 1.16651832740011e-05
1420 1.08849888511564e-05
1421 9.92241894426371e-06
1422 9.08688157075233e-06
1423 8.65450890819019e-06
1424 8.73574167314928e-06
1425 9.21819177612804e-06
1426 1.00260941024999e-05
1427 1.11164149165788e-05
1428 1.21859308004346e-05
1429 1.2901732745263e-05
1430 1.28949649540466e-05
1431 1.21149124643338e-05
1432 1.08964884084628e-05
1433 9.81005651823352e-06
1434 9.31230422285267e-06
1435 9.49571009378758e-06
1436 1.01861883816667e-05
1437 1.10275853888453e-05
1438 1.16920143398147e-05
1439 1.19005984666742e-05
1440 1.15752092786359e-05
1441 1.08040185913261e-05
1442 9.84107612733709e-06
1443 9.00668644465696e-06
1444 8.56757190303981e-06
1445 8.66489313544605e-06
1446 9.17345592021945e-06
1447 1.00105145746099e-05
1448 1.11427004920728e-05
1449 1.2255632525271e-05
1450 1.29514760760507e-05
1451 1.28872369677957e-05
1452 1.20047816576374e-05
1453 1.07202931776129e-05
1454 9.65311096257082e-06
1455 9.22930909563746e-06
1456 9.51772322110855e-06
1457 1.02580086194237e-05
1458 1.11260406878699e-05
1459 1.17305923654776e-05
1460 1.18068110097624e-05
1461 1.13451144030385e-05
1462 1.04799337483641e-05
1463 9.50631744611741e-06
1464 8.74972849551625e-06
1465 8.45200637389354e-06
1466 8.68710939094797e-06
1467 9.31085924449349e-06
1468 1.02907054460033e-05
1469 1.14476153312282e-05
1470 1.24610773872824e-05
1471 1.29040671525971e-05
1472 1.24863416361976e-05
1473 1.13594642234816e-05
1474 1.00922791048674e-05
1475 9.25326075718047e-06
1476 9.15913114774014e-06
1477 9.68372342491364e-06
1478 1.05147489977142e-05
1479 1.12492718518098e-05
1480 1.16368473035422e-05
1481 1.14622455660296e-05
1482 1.08031210799275e-05
1483 9.84320037389352e-06
1484 8.96667062155743e-06
1485 8.4093199368418e-06
1486 8.37332505732036e-06
1487 8.78313666241581e-06
1488 9.56339789492944e-06
1489 1.0623433947643e-05
1490 1.17346049179901e-05
1491 1.25166570161817e-05
1492 1.25854365680134e-05
1493 1.18416567040924e-05
1494 1.06227760383604e-05
1495 9.50628567020217e-06
1496 8.99413210930788e-06
1497 9.18485821088666e-06
1498 9.86966431337777e-06
1499 1.0734986905625e-05
1500 1.1379445451265e-05
1501 1.15592292466715e-05
1502 1.11920540302179e-05
1503 1.04013081601018e-05
1504 9.42239815326928e-06
1505 8.63936488437744e-06
1506 8.25195271270118e-06
1507 8.382905560822e-06
1508 8.89533125025821e-06
1509 9.77371279908823e-06
1510 1.0876757434275e-05
1511 1.19463039771617e-05
1512 1.25596468598665e-05
1513 1.23918296361936e-05
1514 1.1451503613813e-05
1515 1.02003485782376e-05
1516 9.21780454898879e-06
1517 8.92113229689073e-06
1518 9.29624461143419e-06
1519 1.00950901265051e-05
1520 1.09515154300688e-05
1521 1.14982167742639e-05
1522 1.15277853396378e-05
1523 1.10216403848007e-05
1524 1.00983182171888e-05
1525 9.12942924102822e-06
1526 8.41241248572544e-06
1527 8.20316005079036e-06
1528 8.49144322856432e-06
1529 9.17742399408894e-06
1530 1.02093491571864e-05
1531 1.13786438324937e-05
1532 1.23477649371928e-05
1533 1.26593090010729e-05
1534 1.20757288870621e-05
1535 1.09940458932201e-05
1536 9.67717874367224e-06
1537 8.91909036493388e-06
1538 8.95066857342641e-06
1539 9.55560694393398e-06
1540 1.04002782359558e-05
1541 1.11108047509134e-05
1542 1.13772989922545e-05
1543 1.10761298066642e-05
1544 1.02902162049112e-05
1545 9.32194714575729e-06
1546 8.48510396345148e-06
1547 8.07904575284013e-06
1548 8.21891834057453e-06
1549 8.76915759739916e-06
1550 9.68785514654513e-06
1551 1.08331395054506e-05
1552 1.18748297506244e-05
1553 1.23925664206048e-05
1554 1.20805424717041e-05
1555 1.10318207626747e-05
1556 9.76670350905096e-06
1557 8.89740729714816e-06
1558 8.74545390150416e-06
1559 9.22639509504108e-06
1560 1.00554185626933e-05
1561 1.08259181907933e-05
1562 1.12069502522516e-05
1563 1.10439850757871e-05
1564 1.03751015880782e-05
1565 9.44151353721168e-06
1566 8.55831987234268e-06
1567 8.0097491967912e-06
1568 7.99544887097881e-06
1569 8.44641421693382e-06
1570 9.19261492704493e-06
1571 1.02917659763335e-05
1572 1.13709990472444e-05
1573 1.21194476956354e-05
1574 1.21223454923047e-05
1575 1.13318999801493e-05
1576 1.0096410363758e-05
1577 9.04899800557857e-06
1578 8.62049012972932e-06
1579 8.89600617526298e-06
1580 9.65649748696507e-06
1581 1.05201621298789e-05
1582 1.11218547944603e-05
1583 1.12512586496472e-05
1584 1.07826977950864e-05
1585 9.89459870126552e-06
1586 8.91395010693685e-06
1587 8.17329897406793e-06
1588 7.92341253674778e-06
1589 8.21517537019645e-06
1590 8.88797259790408e-06
1591 9.94428056877439e-06
1592 1.11058306728751e-05
1593 1.20598584989473e-05
1594 1.23387479686876e-05
1595 1.1733676644532e-05
1596 1.04951690591548e-05
1597 9.25932743189861e-06
1598 8.6106433156985e-06
1599 8.73765358022638e-06
1600 9.44332990915342e-06
1601 1.03284519390456e-05
1602 1.09938614381022e-05
1603 1.11625193484066e-05
1604 1.07172846721681e-05
1605 9.85453829382976e-06
1606 8.85616830359481e-06
1607 8.08755157155616e-06
1608 7.80287450741213e-06
1609 8.07704181404745e-06
1610 8.74377381787639e-06
1611 9.74986894153318e-06
1612 1.08350606025098e-05
1613 1.1717231313213e-05
1614 1.19514539398402e-05
1615 1.13548317091094e-05
1616 1.01781968551151e-05
1617 9.03282968378605e-06
1618 8.45742067889788e-06
1619 8.61211715452725e-06
1620 9.32498178762842e-06
1621 1.02024966315284e-05
1622 1.08540455752149e-05
1623 1.10221929454646e-05
1624 1.06065173137715e-05
1625 9.74938111930612e-06
1626 8.76884301970193e-06
1627 8.00360996366933e-06
1628 7.73196314174385e-06
1629 8.03765098300602e-06
1630 8.79637026240943e-06
1631 9.93477436672663e-06
1632 1.11831481359737e-05
1633 1.21684778573439e-05
1634 1.23660930819725e-05
1635 1.15705859791149e-05
1636 1.01944749770766e-05
1637 8.96312101517083e-06
1638 8.48012615417559e-06
1639 8.86428184010413e-06
1640 9.7564790406679e-06
1641 1.06716583578859e-05
1642 1.12128799418798e-05
1643 1.10727286113388e-05
1644 1.03276306702149e-05
1645 9.263970516038e-06
1646 8.2806134402702e-06
1647 7.73871901493806e-06
1648 7.83158193051747e-06
1649 8.47364037870335e-06
1650 9.500091468162e-06
1651 1.07092425865041e-05
1652 1.16452688505575e-05
1653 1.18757498257516e-05
1654 1.11786650500711e-05
1655 9.90365561204243e-06
1656 8.73588507843692e-06
1657 8.23856770226783e-06
1658 8.50072353841824e-06
1659 9.25226202053153e-06
1660 1.0082950744561e-05
1661 1.05643900316643e-05
1662 1.05109173564877e-05
1663 9.89742596413379e-06
1664 8.98722156206233e-06
1665 8.11691396940262e-06
1666 7.60400840826847e-06
1667 7.62220135364089e-06
1668 8.14535728732579e-06
1669 9.03230916993181e-06
1670 1.01815610582534e-05
1671 1.11969463185702e-05
1672 1.16807985373057e-05
1673 1.13372884738006e-05
1674 1.02704843065737e-05
1675 9.05086997349258e-06
1676 8.27822678628465e-06
1677 8.26232430384266e-06
1678 8.87632708845842e-06
1679 9.76794031082306e-06
1680 1.05345372767296e-05
1681 1.08358985810852e-05
1682 1.04861804701706e-05
1683 9.67657303263536e-06
1684 8.68698325318107e-06
1685 7.87425516435292e-06
1686 7.54086770049511e-06
1687 7.77337279878054e-06
1688 8.45917909675364e-06
1689 9.52881946503226e-06
1690 1.07933883684019e-05
1691 1.1779114509114e-05
1692 1.20189660504089e-05
1693 1.12972335375439e-05
1694 9.95662422287324e-06
1695 8.73108449983562e-06
1696 8.21506336157185e-06
1697 8.54134262695538e-06
1698 9.43916443318926e-06
1699 1.04086991967733e-05
1700 1.10105905699598e-05
1701 1.09201787550361e-05
1702 1.02126735206376e-05
1703 9.12317279411212e-06
1704 8.11705269398999e-06
1705 7.54997514196276e-06
1706 7.58345584994657e-06
1707 8.16196943187464e-06
1708 9.10538157672391e-06
1709 1.02819844398638e-05
1710 1.1243813503814e-05
1711 1.15481450029264e-05
1712 1.09529839225075e-05
1713 9.73674557158688e-06
1714 8.56843165530918e-06
1715 7.99909599802362e-06
1716 8.1937001130683e-06
1717 8.93196584872413e-06
1718 9.78619768376987e-06
1719 1.03492361001045e-05
1720 1.03720071491509e-05
1721 9.830846202874e-06
1722 8.92637963456622e-06
1723 8.01873974598521e-06
1724 7.44545735598123e-06
1725 7.40430312973217e-06
1726 7.88533147133563e-06
1727 8.68899208938956e-06
1728 9.82698407181459e-06
1729 1.08862211076755e-05
1730 1.14701541429341e-05
1731 1.12124785669465e-05
1732 1.01898916096133e-05
1733 8.94068237222712e-06
1734 8.10875811074752e-06
1735 8.0390681311826e-06
1736 8.65683394568606e-06
1737 9.58863955347056e-06
1738 1.03982848571604e-05
1739 1.07338993382555e-05
1740 1.04410683254841e-05
1741 9.59033594005732e-06
1742 8.55786315234752e-06
1743 7.71081330563916e-06
1744 7.35613069346996e-06
1745 7.58085299423783e-06
1746 8.24271633081963e-06
1747 9.28548058887202e-06
1748 1.05003547989924e-05
1749 1.14390389702734e-05
1750 1.16176900233356e-05
1751 1.08584235953835e-05
1752 9.54714858458772e-06
1753 8.39520824724715e-06
1754 7.96666998725826e-06
1755 8.36351095401255e-06
1756 9.25667593509871e-06
1757 1.0183345236392e-05
1758 1.07337096482141e-05
1759 1.06124129244378e-05
1760 9.87384111494904e-06
1761 8.79867349290464e-06
1762 7.84337048020944e-06
1763 7.35138292373705e-06
1764 7.45573973937042e-06
1765 8.08020924925046e-06
1766 9.07385151638351e-06
1767 1.02330047204724e-05
1768 1.11219163654308e-05
1769 1.12954093158812e-05
1770 1.0579672538924e-05
1771 9.3397370957149e-06
1772 8.24161640400334e-06
1773 7.81255083337484e-06
1774 8.13529739840391e-06
1775 8.93963458636904e-06
1776 9.81575819292146e-06
1777 1.03277399294832e-05
1778 1.02704727860115e-05
1779 9.61401124754957e-06
1780 8.64978931724991e-06
1781 7.75254901475009e-06
1782 7.26183392485424e-06
1783 7.31248324647993e-06
1784 7.86449282319523e-06
1785 8.78504273682523e-06
1786 9.95930794189803e-06
1787 1.09611198619852e-05
1788 1.13647784876214e-05
1789 1.08584211675478e-05
1790 9.68363350040136e-06
1791 8.47389436042789e-06
1792 7.83773094159557e-06
1793 7.9947994562346e-06
1794 8.73884330054864e-06
1795 9.6889112919385e-06
1796 1.03700119478134e-05
1797 1.04857497813526e-05
1798 9.9529376109686e-06
1799 8.98654630709927e-06
1800 7.97787683204376e-06
1801 7.3071877033648e-06
1802 7.18257594733984e-06
1803 7.61063574827858e-06
1804 8.38123342639108e-06
1805 9.51739465016033e-06
1806 1.05958599525557e-05
1807 1.1219906141946e-05
1808 1.10006216256231e-05
1809 9.97132337232642e-06
1810 8.71075890973572e-06
1811 7.86737894697964e-06
1812 7.80474329875602e-06
1813 8.41034305665289e-06
1814 9.33511221168359e-06
1815 1.01408972588146e-05
1816 1.04430506300268e-05
1817 1.01318888874324e-05
1818 9.26146743829293e-06
1819 8.20703375081777e-06
1820 7.42073804160803e-06
1821 7.14139944513192e-06
1822 7.43822007454042e-06
1823 8.17276644138332e-06
1824 9.28634630614766e-06
1825 1.04301548895513e-05
1826 1.12102779992185e-05
1827 1.11658468870957e-05
1828 1.02293307469026e-05
1829 8.92481516601151e-06
1830 7.94145260307744e-06
1831 7.75237372963034e-06
1832 8.30425116937761e-06
1833 9.23691038590135e-06
1834 1.00871282642245e-05
1835 1.0449083379882e-05
1836 1.01402306413867e-05
1837 9.26329886175026e-06
1838 8.19633869220127e-06
1839 7.3769835247095e-06
1840 7.08648326686045e-06
1841 7.39237670988047e-06
1842 8.12089264989879e-06
1843 9.23589459311813e-06
1844 1.03806464601952e-05
1845 1.10865479299171e-05
1846 1.09599522994941e-05
1847 9.97170756722454e-06
1848 8.69544503778741e-06
1849 7.79512516513314e-06
1850 7.65687752224409e-06
1851 8.24576199320859e-06
1852 9.1518135252322e-06
1853 9.90384168542136e-06
1854 1.01819848530305e-05
1855 9.78552457020498e-06
1856 8.92951742903492e-06
1857 7.92997403253892e-06
1858 7.2093985563626e-06
1859 6.99210146848372e-06
1860 7.34246691891371e-06
1861 8.07673813074672e-06
1862 9.15992296146584e-06
1863 1.02136732547109e-05
1864 1.08669365739278e-05
1865 1.07011967340753e-05
1866 9.74843896184296e-06
1867 8.52820834928991e-06
1868 7.69142681233959e-06
1869 7.58477708728833e-06
1870 8.16139492254564e-06
1871 9.08602377691636e-06
1872 9.88810170010623e-06
1873 1.02063959332988e-05
1874 9.87613940450771e-06
1875 9.03477270286501e-06
1876 8.0063046765666e-06
1877 7.22066532388155e-06
1878 6.9556820876393e-06
1879 7.26953888996462e-06
1880 7.9968145234588e-06
1881 9.10136817466878e-06
1882 1.0223256025732e-05
1883 1.0972979905377e-05
1884 1.08971249757062e-05
1885 9.9584492208038e-06
1886 8.67146734506719e-06
1887 7.72922685388089e-06
1888 7.55701678922804e-06
1889 8.10942359308342e-06
1890 9.05344030455169e-06
1891 9.8864221249606e-06
1892 1.02144324234388e-05
1893 9.85579868295616e-06
1894 8.9663543976215e-06
1895 7.91925419463269e-06
1896 7.15597476208529e-06
1897 6.92581900585409e-06
1898 7.2802347315104e-06
1899 8.04983204538701e-06
1900 9.13397025392726e-06
1901 1.01899117308513e-05
1902 1.07935301196793e-05
1903 1.05764011841813e-05
1904 9.57607372331637e-06
1905 8.35513651864517e-06
1906 7.54587022133713e-06
1907 7.50874297938964e-06
1908 8.12992005183588e-06
1909 9.04212174202002e-06
1910 9.81008234779424e-06
1911 1.00521275290255e-05
1912 9.63432187517377e-06
1913 8.7392530683239e-06
1914 7.74693230320977e-06
1915 7.03740478713577e-06
1916 6.89496915473597e-06
1917 7.32540245707014e-06
1918 8.08871324942118e-06
1919 9.17887994766886e-06
1920 1.01603663389405e-05
1921 1.06740624188362e-05
1922 1.03679258760536e-05
1923 9.3514216077395e-06
1924 8.17652842854955e-06
1925 7.44342238623297e-06
1926 7.47744748430534e-06
1927 8.13544707778391e-06
1928 9.05630787961798e-06
1929 9.82216248957002e-06
1930 1.00404448901692e-05
1931 9.59012441503759e-06
1932 8.68346225990635e-06
1933 7.68059979794344e-06
1934 7.00819408039344e-06
1935 6.88520901270806e-06
1936 7.33177050271649e-06
1937 8.13665621191006e-06
1938 9.28270301070455e-06
1939 1.03255612984299e-05
1940 1.08506969520583e-05
1941 1.05024423717737e-05
1942 9.40165350460376e-06
1943 8.16146311333199e-06
1944 7.4476661271472e-06
1945 7.54936948510476e-06
1946 8.31433764325595e-06
1947 9.2913644884618e-06
1948 1.002032937647e-05
1949 1.01178026596926e-05
1950 9.4976404589886e-06
1951 8.47761191735685e-06
1952 7.4769514837314e-06
1953 6.88824824601753e-06
1954 6.93382411842691e-06
1955 7.53730020353061e-06
1956 8.51535254042801e-06
1957 9.70849398296281e-06
1958 1.06132551196403e-05
1959 1.07801359252591e-05
1960 1.00562800136039e-05
1961 8.78176042462542e-06
1962 7.69942639156085e-06
1963 7.34877589003702e-06
1964 7.80503486286221e-06
1965 8.72864970169474e-06
1966 9.60233394398102e-06
1967 1.00082547307423e-05
1968 9.6868371235459e-06
1969 8.80165991068438e-06
1970 7.77100209159443e-06
1971 6.99236594869923e-06
1972 6.76504573204895e-06
1973 7.1383047428597e-06
1974 7.94088613886856e-06
1975 9.04092242848265e-06
1976 1.00437835284239e-05
1977 1.05240199170797e-05
1978 1.01667085732338e-05
1979 9.08605960869835e-06
1980 7.90511470150079e-06
1981 7.25715423932627e-06
1982 7.39491732471009e-06
1983 8.11440624026716e-06
1984 9.01571984046612e-06
1985 9.67705464649526e-06
1986 9.70996674909408e-06
1987 9.11211857901328e-06
1988 8.16219099908366e-06
1989 7.26140094586825e-06
1990 6.75937584659891e-06
1991 6.84688720653526e-06
1992 7.45205789876024e-06
1993 8.40979041694112e-06
1994 9.58269215933782e-06
1995 1.04599197627842e-05
1996 1.06409406142038e-05
1997 9.93043272634608e-06
1998 8.69251531110393e-06
1999 7.62238900131607e-06
};
\addlegendentry{Train}
\addplot [semithick, black]
table {%
0 0.0320607274770737
1 0.0312626957893372
2 0.0304698720574379
3 0.02967225946486
4 0.028857646510005
5 0.0279943775385618
6 0.0270217880606651
7 0.0258523300290108
8 0.0244098268449306
9 0.0229741875082254
10 0.0216257255524397
11 0.0203962940722704
12 0.0193079821765423
13 0.0183341447263956
14 0.0174581427127123
15 0.0166599322110415
16 0.0159199945628643
17 0.015227141790092
18 0.0145625909790397
19 0.0139241255819798
20 0.0132865719497204
21 0.0126423630863428
22 0.0120190605521202
23 0.0114085953682661
24 0.0108017791062593
25 0.0102013163268566
26 0.00960825756192207
27 0.00902844965457916
28 0.00847236905246973
29 0.00795213878154755
30 0.00748155638575554
31 0.00706594763323665
32 0.00669761467725039
33 0.00636943383142352
34 0.00607005506753922
35 0.00579582946375012
36 0.00553922960534692
37 0.00529682310298085
38 0.00506703555583954
39 0.00484841503202915
40 0.00464014569297433
41 0.00444045756012201
42 0.00424951687455177
43 0.00406575249508023
44 0.00388954114168882
45 0.00371981807984412
46 0.00355659774504602
47 0.00339898397214711
48 0.00324710388667881
49 0.00310034467838705
50 0.00295835942961276
51 0.00282115652225912
52 0.00268834899179637
53 0.00255991239100695
54 0.00243555870838463
55 0.00231524114497006
56 0.00219875341281295
57 0.00208611018024385
58 0.00197708304040134
59 0.0018720286898315
60 0.00177071057260036
61 0.00167190621141344
62 0.00157826347276568
63 0.0014891701284796
64 0.0014045819407329
65 0.00132458982989192
66 0.00124806386884302
67 0.00117705075535923
68 0.00111149344593287
69 0.00105161429382861
70 0.000996851129457355
71 0.000947532418649644
72 0.000903385924175382
73 0.000864113622810692
74 0.000829097523819655
75 0.000798017834313214
76 0.000770169543102384
77 0.000745110854040831
78 0.00072241120506078
79 0.000701627694070339
80 0.000682107405737042
81 0.000662811973597854
82 0.00064306560670957
83 0.000624634849373251
84 0.00060663215117529
85 0.000588855589739978
86 0.00057190062943846
87 0.000555643637198955
88 0.00053983589168638
89 0.000524255156051368
90 0.000508939614519477
91 0.000493615807499737
92 0.000478641013614833
93 0.000464006996480748
94 0.000449649523943663
95 0.00043574208393693
96 0.000421469856519252
97 0.000405552797019482
98 0.000390349770896137
99 0.000376210955437273
100 0.000363697152351961
101 0.000352009432390332
102 0.00034152262378484
103 0.000331527000525966
104 0.000324327498674393
105 0.000317971513140947
106 0.000311384093947709
107 0.000305651046801358
108 0.000300562183838338
109 0.000295944773824885
110 0.000291611620923504
111 0.000287837436189875
112 0.000284161505987868
113 0.00028090295381844
114 0.000277881103102118
115 0.000275058555416763
116 0.000272444449365139
117 0.000270134070888162
118 0.000267829338554293
119 0.000265779817709699
120 0.000263828667812049
121 0.000262006302364171
122 0.000260289001744241
123 0.000258706713793799
124 0.000257278908975422
125 0.000255882332567126
126 0.000254547398071736
127 0.000253160367719829
128 0.000251955061685294
129 0.000250845943810418
130 0.000249752774834633
131 0.000248745200224221
132 0.00024763192050159
133 0.000246689218329266
134 0.000245761591941118
135 0.000244822935201228
136 0.000243962422246113
137 0.00024308888532687
138 0.000242261026869528
139 0.000241235771682113
140 0.000240432418650016
141 0.000239675588090904
142 0.000238978420384228
143 0.000238139604334719
144 0.000237487023696303
145 0.000236802719882689
146 0.000236158797633834
147 0.000235398096265271
148 0.000234854524023831
149 0.000234181643463671
150 0.000233589496929199
151 0.000232960548601113
152 0.000232433594646864
153 0.000231799625908025
154 0.000231229874771088
155 0.000230709454626776
156 0.000230152189033106
157 0.000229670011322014
158 0.000229061639402062
159 0.000228628792683594
160 0.000228094024350867
161 0.00022763351444155
162 0.000227124284720048
163 0.000226690593990497
164 0.000226160540478304
165 0.000225743642658927
166 0.000225262207095511
167 0.000224862422328442
168 0.000224389310460538
169 0.000223956914851442
170 0.00022357128909789
171 0.000223105191253126
172 0.000222771690459922
173 0.000222341099288315
174 0.000221958733163774
175 0.000221580499783158
176 0.000221217502257787
177 0.000220829810132273
178 0.000220503134187311
179 0.000220032117795199
180 0.000219684661715291
181 0.000219269320950843
182 0.00021894191741012
183 0.000218593399040401
184 0.000218245113501325
185 0.000217872715438716
186 0.000217412918573245
187 0.000216971413465217
188 0.000216613712836988
189 0.000216233180253766
190 0.000215904714423232
191 0.000215498701436445
192 0.000215205087442882
193 0.000214810264878906
194 0.000214476152905263
195 0.000214113606489263
196 0.000213811916182749
197 0.000213424442335963
198 0.000213077059015632
199 0.000212770552025177
200 0.000212400293094106
201 0.000212066835956648
202 0.000211711449082941
203 0.000211408652830869
204 0.000211020989809185
205 0.000210764599614777
206 0.000210391328437254
207 0.000210025857086293
208 0.000209747027838603
209 0.000209377787541598
210 0.000209069577977061
211 0.000208720754017122
212 0.000208396115340292
213 0.000208045137696899
214 0.000207725446671247
215 0.00020737353770528
216 0.00020701572066173
217 0.000206718585104682
218 0.000206305528990924
219 0.000206032622372732
220 0.000205657139304094
221 0.000205260221264325
222 0.000204989788471721
223 0.000204612733796239
224 0.000204279276658781
225 0.000203974050236866
226 0.000203588642762043
227 0.000203238523681648
228 0.000202866023755632
229 0.00020252971444279
230 0.000202130366233177
231 0.000201835908228531
232 0.000201410861336626
233 0.000200992668396793
234 0.00020063498232048
235 0.000200281283468939
236 0.000199904359760694
237 0.000199561807676218
238 0.000199092290131375
239 0.00019862964109052
240 0.000198215318960138
241 0.000197655885131098
242 0.000197234970983118
243 0.000196697539649904
244 0.000196304870769382
245 0.000195774395251647
246 0.000195407075807452
247 0.000194793785340153
248 0.00019443451310508
249 0.000193880856386386
250 0.000193424435565248
251 0.000192986510228366
252 0.000192529318155721
253 0.000192070277989842
254 0.00019160145893693
255 0.00019112104200758
256 0.000190730657777749
257 0.000190217673662119
258 0.00018979019660037
259 0.000189251819392666
260 0.000188715959666297
261 0.000188174264621921
262 0.000187730576726608
263 0.000187289930181578
264 0.000186775447218679
265 0.000186330929864198
266 0.000185714292456396
267 0.000185242242878303
268 0.000184685486601666
269 0.000183978860150091
270 0.000183118812856264
271 0.000182467585545965
272 0.000181892886757851
273 0.000181257652002387
274 0.000180666072992608
275 0.000180080387508497
276 0.000179420298081823
277 0.000178783579031006
278 0.00017814380407799
279 0.000177559602889232
280 0.000176869929418899
281 0.000176211280631833
282 0.000175566921825521
283 0.000174940985743888
284 0.00017417743219994
285 0.000173513195477426
286 0.000172842366737314
287 0.000172361222212203
288 0.000171598643646576
289 0.00017103752179537
290 0.000170400468050502
291 0.000169680497492664
292 0.000168996659340337
293 0.000168402199051343
294 0.000167704623891041
295 0.000166898578754626
296 0.000166155732586049
297 0.000165460296557285
298 0.000164745797519572
299 0.000163947290275246
300 0.000163225835422054
301 0.000162471027579159
302 0.000161755961016752
303 0.00016101443907246
304 0.000160307085025124
305 0.000159491726662964
306 0.00015877241094131
307 0.000158002672833391
308 0.000157223534188233
309 0.00015647774853278
310 0.000155749527039006
311 0.000154928609845228
312 0.000154152396135032
313 0.000153452143422328
314 0.000152635999256745
315 0.000151795233250596
316 0.000151105632539839
317 0.000150234263855964
318 0.000149539497215301
319 0.000148804378113709
320 0.000148000122862868
321 0.000147139871842228
322 0.000146604666952044
323 0.000145626021549106
324 0.000144976700539701
325 0.000144053454278037
326 0.000143598968861625
327 0.000142600125400349
328 0.000142240241984837
329 0.000141286087455228
330 0.000140400647069328
331 0.000139518044306897
332 0.000138563642394729
333 0.00013775379920844
334 0.000136751143145375
335 0.00013610596943181
336 0.000135208334540948
337 0.000134570989757776
338 0.000133556488435715
339 0.000132903864141554
340 0.0001319452712778
341 0.000131489054183476
342 0.000130472966702655
343 0.000129706910229288
344 0.000128789848531596
345 0.000128137282445095
346 0.000127159772091545
347 0.000126065628137439
348 0.000125291480799206
349 0.000124716520076618
350 0.000123848396469839
351 0.000123133868328296
352 0.000122104494948871
353 0.000121180921269115
354 0.000120098768093158
355 0.000119084892503452
356 0.000118139207188506
357 0.000117021845653653
358 0.000115946459118277
359 0.000114835500426125
360 0.000113833993964363
361 0.000112667104986031
362 0.000111942455987446
363 0.000110737797513139
364 0.000109884545963723
365 0.000108988475403748
366 0.000108143729448784
367 0.000107226223917678
368 0.000106548868643586
369 0.000105666375020519
370 0.000104859362181742
371 0.000104169863334391
372 0.000103220707387663
373 0.000102535268524662
374 0.000101600177004002
375 0.000100794168247376
376 0.000100108809419908
377 9.9147655419074e-05
378 9.84265934675932e-05
379 9.75426737568341e-05
380 9.67764062806964e-05
381 9.61400364758447e-05
382 9.52717527979985e-05
383 9.45080028031953e-05
384 9.38760858844034e-05
385 9.30145251913927e-05
386 9.23397237784229e-05
387 9.17982761166058e-05
388 9.09893133211881e-05
389 9.03634936548769e-05
390 8.9481400209479e-05
391 8.88251815922558e-05
392 8.83071261341684e-05
393 8.75238547450863e-05
394 8.68259885464795e-05
395 8.61725784488954e-05
396 8.55881517054513e-05
397 8.48566851345822e-05
398 8.4123617853038e-05
399 8.37543339002877e-05
400 8.31672223284841e-05
401 8.24422822915949e-05
402 8.17297914181836e-05
403 8.12656944617629e-05
404 8.03827060735784e-05
405 7.97558968770318e-05
406 7.92170030763373e-05
407 7.84973162808456e-05
408 7.79439651523717e-05
409 7.74081345298328e-05
410 7.6610580435954e-05
411 7.60336333769374e-05
412 7.55967921577394e-05
413 7.49311075196601e-05
414 7.42819756851532e-05
415 7.38394082873128e-05
416 7.3141920438502e-05
417 7.26803409634158e-05
418 7.19245363143273e-05
419 7.13965055183508e-05
420 7.09380838088691e-05
421 7.0282731030602e-05
422 6.99110023560934e-05
423 6.93568945280276e-05
424 6.88644649926573e-05
425 6.81695892126299e-05
426 6.78704818710685e-05
427 6.73080867272802e-05
428 6.66993510094471e-05
429 6.63698883727193e-05
430 6.56872289255261e-05
431 6.53586612315848e-05
432 6.49305438855663e-05
433 6.42881568637677e-05
434 6.40295402263291e-05
435 6.35284013696946e-05
436 6.29193091299385e-05
437 6.27000699751079e-05
438 6.22058869339526e-05
439 6.16591205471195e-05
440 6.14303426118568e-05
441 6.10460920142941e-05
442 6.04624146944843e-05
443 6.02073341724463e-05
444 5.98499282205012e-05
445 5.93504519201815e-05
446 5.91881434957031e-05
447 5.88267066632397e-05
448 5.82730681344401e-05
449 5.80937739869114e-05
450 5.77039754716679e-05
451 5.74149598833174e-05
452 5.70706069993321e-05
453 5.65608352189884e-05
454 5.63569301448297e-05
455 5.60337393835653e-05
456 5.54877988179214e-05
457 5.52391466044355e-05
458 5.49602264072746e-05
459 5.438269363367e-05
460 5.42123307241127e-05
461 5.39555949217174e-05
462 5.34726241312455e-05
463 5.32266785739921e-05
464 5.30452016391791e-05
465 5.24833958479576e-05
466 5.23087255714927e-05
467 5.20207286172081e-05
468 5.16666295879986e-05
469 5.13682243763469e-05
470 5.10805111844093e-05
471 5.07460827066097e-05
472 5.02261100336909e-05
473 4.9923804908758e-05
474 4.9630478315521e-05
475 4.91981481900439e-05
476 4.87375946249813e-05
477 4.848804383073e-05
478 4.82763862237334e-05
479 4.78616457257885e-05
480 4.73919135401957e-05
481 4.7076271584956e-05
482 4.7224519221345e-05
483 4.69175283797085e-05
484 4.64639015262946e-05
485 4.63111427961849e-05
486 4.60278934042435e-05
487 4.45093974121846e-05
488 4.30950858572032e-05
489 4.26412225351669e-05
490 4.2012474295916e-05
491 4.1769406379899e-05
492 4.11938963225111e-05
493 4.05410064558964e-05
494 4.05601713282522e-05
495 4.03103986172937e-05
496 3.98281699744985e-05
497 3.96740761061665e-05
498 3.93542286474258e-05
499 3.9267455576919e-05
500 3.90578497899696e-05
501 3.89018096029758e-05
502 3.81163226848003e-05
503 3.6893838114338e-05
504 3.58641082129907e-05
505 3.50353438989259e-05
506 3.43956671713386e-05
507 3.38664067385253e-05
508 3.32550880557392e-05
509 3.29925933328923e-05
510 3.25289547618013e-05
511 3.22492705890909e-05
512 3.19356804538984e-05
513 3.17403355438728e-05
514 3.13504096993711e-05
515 3.11839685309678e-05
516 3.10061332129408e-05
517 3.07282753055915e-05
518 3.05024677800247e-05
519 3.04185050481465e-05
520 3.03208616969641e-05
521 3.00909832731122e-05
522 2.99624971376033e-05
523 2.98053619189886e-05
524 2.9556544177467e-05
525 2.93696029984858e-05
526 2.92736021947348e-05
527 2.90348943963181e-05
528 2.89905692625325e-05
529 2.88429691863712e-05
530 2.86613012576709e-05
531 2.85706209979253e-05
532 2.84757661574986e-05
533 2.84145826299209e-05
534 2.83244426100282e-05
535 2.82129039987922e-05
536 2.81804113910766e-05
537 2.81825141428271e-05
538 2.80280110018793e-05
539 2.80373260466149e-05
540 2.8001266400679e-05
541 2.78898969554575e-05
542 2.79298783425475e-05
543 2.78354764304822e-05
544 2.77740564342821e-05
545 2.77769559033914e-05
546 2.77586423180765e-05
547 2.76049813692225e-05
548 2.75719939963892e-05
549 2.75638576567871e-05
550 2.73708392342087e-05
551 2.73501900664996e-05
552 2.72711622528732e-05
553 2.7189489628654e-05
554 2.69616630248493e-05
555 2.69146021310007e-05
556 2.68234580289572e-05
557 2.66370770987123e-05
558 2.65392245637486e-05
559 2.63723250100156e-05
560 2.62558987742523e-05
561 2.60908363998169e-05
562 2.5957995603676e-05
563 2.58573236351367e-05
564 2.56924267887371e-05
565 2.55372760875616e-05
566 2.54190945270238e-05
567 2.53162907029036e-05
568 2.51881447184132e-05
569 2.50531138590304e-05
570 2.49432705459185e-05
571 2.48482738243183e-05
572 2.47671232500579e-05
573 2.46131894527934e-05
574 2.45716091740178e-05
575 2.45427199843107e-05
576 2.4501243387931e-05
577 2.443104312988e-05
578 2.43450012931135e-05
579 2.43700342252851e-05
580 2.43758076976519e-05
581 2.42893129325239e-05
582 2.43248341575963e-05
583 2.42480309680104e-05
584 2.42713322222698e-05
585 2.42168243858032e-05
586 2.45243500103243e-05
587 2.44368202402256e-05
588 2.43586855503963e-05
589 2.4248209228972e-05
590 2.42236728809075e-05
591 2.41127163462806e-05
592 2.39998043980449e-05
593 2.3954202333698e-05
594 2.38616903516231e-05
595 2.37300773733296e-05
596 2.35942807194078e-05
597 2.34964118135395e-05
598 2.33362734434195e-05
599 2.32241891353624e-05
600 2.3047312424751e-05
601 2.28993230848573e-05
602 2.27509044634644e-05
603 2.26915308303433e-05
604 2.24655523197725e-05
605 2.23718452616595e-05
606 2.2263482605922e-05
607 2.21794452954782e-05
608 2.209600097558e-05
609 2.19569883483928e-05
610 2.1930620277999e-05
611 2.18842087633675e-05
612 2.18769746425096e-05
613 2.18512359424494e-05
614 2.18512032006402e-05
615 2.18236727960175e-05
616 2.18389777728589e-05
617 2.18957575270906e-05
618 2.19174544326961e-05
619 2.1976597054163e-05
620 2.20518104470102e-05
621 2.20436086237896e-05
622 2.21817681449465e-05
623 2.22461476369062e-05
624 2.23222450586036e-05
625 2.2408967197407e-05
626 2.24646955757635e-05
627 2.25381136260694e-05
628 2.26127140194876e-05
629 2.25557196245063e-05
630 2.25924868573202e-05
631 2.2588379579247e-05
632 2.24974955926882e-05
633 2.24004270421574e-05
634 2.22398030018667e-05
635 2.21148511627689e-05
636 2.18715467781294e-05
637 2.16679291042965e-05
638 2.14051233342616e-05
639 2.11631559068337e-05
640 2.08941273740493e-05
641 2.06361237360397e-05
642 2.03729177883361e-05
643 2.01034363271901e-05
644 1.98501620616298e-05
645 1.9621487808763e-05
646 1.93903561012121e-05
647 1.91460403584642e-05
648 1.89771853911225e-05
649 1.87923542398494e-05
650 1.86564611794893e-05
651 1.85430908459239e-05
652 1.84299733518856e-05
653 1.83390529855387e-05
654 1.82363801286556e-05
655 1.82028834387893e-05
656 1.81869563675718e-05
657 1.81640716618858e-05
658 1.81878040166339e-05
659 1.81750019692117e-05
660 1.82594758371124e-05
661 1.83350057341158e-05
662 1.84341279236833e-05
663 1.85681146831485e-05
664 1.87681598617928e-05
665 1.89116090041352e-05
666 1.91947237908607e-05
667 1.94230906345183e-05
668 1.96332330233417e-05
669 1.99964761122828e-05
670 2.03318741114344e-05
671 2.07958146347664e-05
672 2.1109874069225e-05
673 2.15388263313798e-05
674 2.19754510908388e-05
675 2.25765543291345e-05
676 2.30707901209826e-05
677 2.34204708249308e-05
678 2.38237116718665e-05
679 2.38994452956831e-05
680 2.40838871832239e-05
681 2.36842006415827e-05
682 2.32839702221099e-05
683 2.24716950469883e-05
684 2.1619349354296e-05
685 2.04562638828065e-05
686 1.93675768969115e-05
687 1.8241813450004e-05
688 1.73313874256564e-05
689 1.66490626725135e-05
690 1.62782835104736e-05
691 1.6171714378288e-05
692 1.63363747560652e-05
693 1.66542922670487e-05
694 1.72284890140872e-05
695 1.80142451426946e-05
696 1.89059974218253e-05
697 1.99481637537247e-05
698 2.10526577575365e-05
699 2.23253373405896e-05
700 2.37175245274557e-05
701 2.50403099926189e-05
702 2.66007409663871e-05
703 2.8068252504454e-05
704 2.96415328193689e-05
705 3.05619032587856e-05
706 3.1807128834771e-05
707 3.20632243528962e-05
708 3.25513065035921e-05
709 3.17448138957843e-05
710 3.03277811326552e-05
711 2.79068171948893e-05
712 2.55284776358167e-05
713 2.26148986257613e-05
714 2.01150196517119e-05
715 1.82224557647714e-05
716 1.70657367561944e-05
717 1.65799319802318e-05
718 1.65613491844852e-05
719 1.71382544067455e-05
720 1.78214686457068e-05
721 1.89637521543773e-05
722 2.03784020413877e-05
723 2.19630946958205e-05
724 2.3702696125838e-05
725 2.50654575211229e-05
726 2.57782576227328e-05
727 2.54656188189983e-05
728 2.40466888499213e-05
729 2.16212010855088e-05
730 1.88674985110993e-05
731 1.64597931870958e-05
732 1.48406115840771e-05
733 1.41944465212873e-05
734 1.43841962199076e-05
735 1.51374042616226e-05
736 1.61085754371015e-05
737 1.69916238519363e-05
738 1.76631710928632e-05
739 1.80468323378591e-05
740 1.8052038285532e-05
741 1.78018835867988e-05
742 1.73256757989293e-05
743 1.66767258633627e-05
744 1.60361287271371e-05
745 1.53813653014367e-05
746 1.48285344039323e-05
747 1.44055211421801e-05
748 1.41729487950215e-05
749 1.41339714900823e-05
750 1.4289990758698e-05
751 1.46642796607921e-05
752 1.51258136611432e-05
753 1.58191014634212e-05
754 1.66889403772075e-05
755 1.76986086444231e-05
756 1.89247621165123e-05
757 2.03283507289598e-05
758 2.194354783569e-05
759 2.34150247706566e-05
760 2.47825373662636e-05
761 2.53416856139665e-05
762 2.48176129389321e-05
763 2.31172853091266e-05
764 2.03866420633858e-05
765 1.75126242538681e-05
766 1.51794520206749e-05
767 1.38759369292529e-05
768 1.38558716571424e-05
769 1.5023711057438e-05
770 1.70232215168653e-05
771 1.93760242837016e-05
772 2.17122669710079e-05
773 2.3546330339741e-05
774 2.46033596340567e-05
775 2.44574966927757e-05
776 2.35461848205887e-05
777 2.19345402001636e-05
778 1.98120978893712e-05
779 1.76271241798531e-05
780 1.58388356794603e-05
781 1.4680141248391e-05
782 1.42111093737185e-05
783 1.43236829899251e-05
784 1.48971239468665e-05
785 1.57625727297273e-05
786 1.68423703144072e-05
787 1.79826874955324e-05
788 1.91372419067193e-05
789 1.99477362912148e-05
790 2.02594492293429e-05
791 1.97980571101652e-05
792 1.86130400834372e-05
793 1.69900322362082e-05
794 1.52366856127628e-05
795 1.38250825330033e-05
796 1.29655190903577e-05
797 1.27859484564397e-05
798 1.31802016767324e-05
799 1.39544308694894e-05
800 1.49190136653488e-05
801 1.59050869115163e-05
802 1.67949801834766e-05
803 1.74082033481682e-05
804 1.77291585714556e-05
805 1.77538331627147e-05
806 1.74474116647616e-05
807 1.68969418155029e-05
808 1.61777370522032e-05
809 1.54019999172306e-05
810 1.4634051694884e-05
811 1.40301508508855e-05
812 1.35719665195211e-05
813 1.33556659420719e-05
814 1.33567509692512e-05
815 1.35946511363727e-05
816 1.40648880915251e-05
817 1.48442977661034e-05
818 1.5858076949371e-05
819 1.72642758116126e-05
820 1.91387625818606e-05
821 2.13662551686866e-05
822 2.37325875787064e-05
823 2.59328862739494e-05
824 2.72444030997576e-05
825 2.69177544396371e-05
826 2.43892391154077e-05
827 2.02074515982531e-05
828 1.61879706865875e-05
829 1.3560271327151e-05
830 1.30461939988891e-05
831 1.44981740959338e-05
832 1.71139054145897e-05
833 1.9971919755335e-05
834 2.24117065954488e-05
835 2.36717878578929e-05
836 2.34993331105215e-05
837 2.20082183659542e-05
838 1.98292600543937e-05
839 1.74774304468883e-05
840 1.54092322190991e-05
841 1.39759777084691e-05
842 1.3258161743579e-05
843 1.32081413539709e-05
844 1.37265351440874e-05
845 1.4544147234119e-05
846 1.56001624418423e-05
847 1.66650461324025e-05
848 1.74969027284533e-05
849 1.78654299816117e-05
850 1.75917648448376e-05
851 1.66533573064953e-05
852 1.53108212543884e-05
853 1.38882533065043e-05
854 1.27523717310396e-05
855 1.20943423098652e-05
856 1.19650985652697e-05
857 1.22795227071038e-05
858 1.28491201394354e-05
859 1.35477475851076e-05
860 1.42732260428602e-05
861 1.48990720845177e-05
862 1.53729324665619e-05
863 1.56498717842624e-05
864 1.56775713548996e-05
865 1.5487685232074e-05
866 1.51451404235559e-05
867 1.46280699482304e-05
868 1.40666688821511e-05
869 1.34814999910304e-05
870 1.30162461573491e-05
871 1.26492377603427e-05
872 1.24767029774375e-05
873 1.24687676361646e-05
874 1.26055283544702e-05
875 1.30031676235376e-05
876 1.35383152155555e-05
877 1.44185123645002e-05
878 1.560547570989e-05
879 1.72755408129888e-05
880 1.9494542357279e-05
881 2.21428654185729e-05
882 2.48580472543836e-05
883 2.72231281996937e-05
884 2.83927911368664e-05
885 2.73831210506614e-05
886 2.37808199017309e-05
887 1.87661826203112e-05
888 1.44775613080128e-05
889 1.26578279378009e-05
890 1.36481012305012e-05
891 1.67950511240633e-05
892 2.06216664082604e-05
893 2.40933495661011e-05
894 2.58585514529841e-05
895 2.53763246291783e-05
896 2.34258968703216e-05
897 2.04481038963422e-05
898 1.73954576894175e-05
899 1.48760937008774e-05
900 1.33026551338844e-05
901 1.27100465761032e-05
902 1.29107320390176e-05
903 1.36761345856939e-05
904 1.46867632793146e-05
905 1.55842972162645e-05
906 1.61303360073362e-05
907 1.61013449542224e-05
908 1.54686567839235e-05
909 1.43275001391885e-05
910 1.31220258481335e-05
911 1.2120945029892e-05
912 1.15453976832214e-05
913 1.14087924885098e-05
914 1.16267383418744e-05
915 1.20562672236701e-05
916 1.2558063644974e-05
917 1.2997384146729e-05
918 1.33295234263642e-05
919 1.35230593514279e-05
920 1.35216741909971e-05
921 1.34509973577224e-05
922 1.32309969558264e-05
923 1.29074423966813e-05
924 1.25609112728853e-05
925 1.22197916425648e-05
926 1.19496226034244e-05
927 1.17843055704725e-05
928 1.17543422675226e-05
929 1.18350399134215e-05
930 1.20660588436294e-05
931 1.24627931654686e-05
932 1.30510388771654e-05
933 1.38740233524004e-05
934 1.49504430737579e-05
935 1.63722415891243e-05
936 1.81468931259587e-05
937 2.0217061319272e-05
938 2.2385640477296e-05
939 2.40671979554463e-05
940 2.47060761466855e-05
941 2.37787662626943e-05
942 2.09425124921836e-05
943 1.70321873156354e-05
944 1.37270981213078e-05
945 1.22371875477256e-05
946 1.31061733554816e-05
947 1.60413110279478e-05
948 2.01099865080323e-05
949 2.41604375332827e-05
950 2.68910171143943e-05
951 2.77038707281463e-05
952 2.62042867689161e-05
953 2.29259021580219e-05
954 1.91953586181626e-05
955 1.56783480633749e-05
956 1.34992369567044e-05
957 1.25938131532166e-05
958 1.27663461171323e-05
959 1.35613881866448e-05
960 1.46781203511637e-05
961 1.57352187670767e-05
962 1.63493987201946e-05
963 1.62400883709779e-05
964 1.53786295413738e-05
965 1.39886524266331e-05
966 1.25490987556987e-05
967 1.15083612399758e-05
968 1.10700857476331e-05
969 1.11978279164759e-05
970 1.17170957310009e-05
971 1.23957524920115e-05
972 1.30226399051026e-05
973 1.34986294142436e-05
974 1.37583865580382e-05
975 1.37360393637209e-05
976 1.35141490318347e-05
977 1.31139167933725e-05
978 1.26582472148584e-05
979 1.22049996207352e-05
980 1.17971703730291e-05
981 1.15668517537415e-05
982 1.15115071821492e-05
983 1.16227702164906e-05
984 1.19387250379077e-05
985 1.2457067896321e-05
986 1.31740443976014e-05
987 1.41128830364323e-05
988 1.53094733832404e-05
989 1.66860882018227e-05
990 1.81282848643605e-05
991 1.9369801520952e-05
992 2.00536251213634e-05
993 1.98093675862765e-05
994 1.84179771167692e-05
995 1.61803527589655e-05
996 1.37865836222772e-05
997 1.20707372843754e-05
998 1.16241417345009e-05
999 1.26460336105083e-05
1000 1.49135048559401e-05
1001 1.79509970621439e-05
1002 2.11479382414836e-05
1003 2.3827049517422e-05
1004 2.51075580308679e-05
1005 2.48181040660711e-05
1006 2.31306821660837e-05
1007 2.01156035473105e-05
1008 1.69651975738816e-05
1009 1.44442283271928e-05
1010 1.29550480778562e-05
1011 1.26020340758259e-05
1012 1.30962162074866e-05
1013 1.42097815114539e-05
1014 1.59031769726425e-05
1015 1.7492326151114e-05
1016 1.85977369255852e-05
1017 1.87770183401881e-05
1018 1.76894336618716e-05
1019 1.55718680616701e-05
1020 1.32800678329659e-05
1021 1.16230094135972e-05
1022 1.11061990537564e-05
1023 1.16581559268525e-05
1024 1.28480924104224e-05
1025 1.4173246199789e-05
1026 1.52461552715977e-05
1027 1.58559596457053e-05
1028 1.58704642672092e-05
1029 1.54114804900018e-05
1030 1.47031005326426e-05
1031 1.37010392791126e-05
1032 1.2767615771736e-05
1033 1.20169752335642e-05
1034 1.16547298603109e-05
1035 1.16039836939308e-05
1036 1.19257301776088e-05
1037 1.25507558550453e-05
1038 1.34565316329827e-05
1039 1.46127904372406e-05
1040 1.58607781486353e-05
1041 1.70168314070906e-05
1042 1.78737191163236e-05
1043 1.80345250555547e-05
1044 1.72059735632502e-05
1045 1.56643764057662e-05
1046 1.37297056426178e-05
1047 1.20886770673678e-05
1048 1.12663337858976e-05
1049 1.1487330084492e-05
1050 1.26526410895167e-05
1051 1.44424493555562e-05
1052 1.64985449373489e-05
1053 1.84448035724927e-05
1054 1.98593934328528e-05
1055 2.05422275030287e-05
1056 2.02702158276225e-05
1057 1.90440641745226e-05
1058 1.72429136000574e-05
1059 1.53901892190333e-05
1060 1.36169401230291e-05
1061 1.2530220374174e-05
1062 1.21559423860162e-05
1063 1.24345242511481e-05
1064 1.3233819117886e-05
1065 1.46538868648349e-05
1066 1.64361827046378e-05
1067 1.84719647222664e-05
1068 2.02013925445499e-05
1069 2.09338231798029e-05
1070 1.99949481611839e-05
1071 1.74788674485171e-05
1072 1.43482438943465e-05
1073 1.19822816486703e-05
1074 1.13140413304791e-05
1075 1.2311784303165e-05
1076 1.43735969686531e-05
1077 1.66245790751418e-05
1078 1.83572101377649e-05
1079 1.90114769793581e-05
1080 1.86316065082792e-05
1081 1.7211134036188e-05
1082 1.54502486111596e-05
1083 1.37518582050689e-05
1084 1.24552980196313e-05
1085 1.17865020001773e-05
1086 1.18087473310879e-05
1087 1.23890840768581e-05
1088 1.34653264467488e-05
1089 1.4826842743787e-05
1090 1.61935913638445e-05
1091 1.72621603269363e-05
1092 1.75730765477056e-05
1093 1.69084269145969e-05
1094 1.52486200022395e-05
1095 1.33018993437872e-05
1096 1.17303952720249e-05
1097 1.10623032014701e-05
1098 1.14144340841449e-05
1099 1.25755141198169e-05
1100 1.41380305649363e-05
1101 1.5692357919761e-05
1102 1.68000824487535e-05
1103 1.72739291883772e-05
1104 1.71310875884956e-05
1105 1.62297874339856e-05
1106 1.50632613440393e-05
1107 1.37845299832406e-05
1108 1.26564764286741e-05
1109 1.18730113172205e-05
1110 1.16310948214959e-05
1111 1.18663592729717e-05
1112 1.25466403915198e-05
1113 1.37452507260605e-05
1114 1.5204462215479e-05
1115 1.69680879480438e-05
1116 1.87689620361198e-05
1117 1.99733149202075e-05
1118 1.98627494683024e-05
1119 1.82372841663891e-05
1120 1.5535501006525e-05
1121 1.28948568089982e-05
1122 1.14436961666797e-05
1123 1.16825949589838e-05
1124 1.33554576677852e-05
1125 1.58022266987246e-05
1126 1.82750427484279e-05
1127 2.00573849724606e-05
1128 2.0542202037177e-05
1129 1.96560995391337e-05
1130 1.80658244062215e-05
1131 1.58221682795556e-05
1132 1.38832001539413e-05
1133 1.24624857562594e-05
1134 1.17858571684337e-05
1135 1.19364240163122e-05
1136 1.2708702342934e-05
1137 1.40439851747942e-05
1138 1.570065614942e-05
1139 1.72553827724187e-05
1140 1.81665436684852e-05
1141 1.80361148522934e-05
1142 1.66014342539711e-05
1143 1.43994211612153e-05
1144 1.22951360026491e-05
1145 1.11598055809736e-05
1146 1.13081378003699e-05
1147 1.24747493828181e-05
1148 1.41385535243899e-05
1149 1.5700985386502e-05
1150 1.6857375157997e-05
1151 1.7275433492614e-05
1152 1.67895905178739e-05
1153 1.57645063154632e-05
1154 1.44354462463525e-05
1155 1.31071919895476e-05
1156 1.20583235911909e-05
1157 1.15120583359385e-05
1158 1.15305492727202e-05
1159 1.2025389878545e-05
1160 1.30304661070113e-05
1161 1.44109826578642e-05
1162 1.60164581757272e-05
1163 1.75947297975654e-05
1164 1.86026372830383e-05
1165 1.84741165867308e-05
1166 1.70088605955243e-05
1167 1.46980282806908e-05
1168 1.24812040667166e-05
1169 1.12641882878961e-05
1170 1.14777385533671e-05
1171 1.29141089928453e-05
1172 1.49860125020496e-05
1173 1.70379062183201e-05
1174 1.86642864719033e-05
1175 1.93107571249129e-05
1176 1.88616504601669e-05
1177 1.74444103322458e-05
1178 1.56577752932208e-05
1179 1.38300001708558e-05
1180 1.24379093904281e-05
1181 1.16602022899315e-05
1182 1.16606706797029e-05
1183 1.22713545351871e-05
1184 1.35321952257073e-05
1185 1.51594513226883e-05
1186 1.70203675224911e-05
1187 1.85548906301847e-05
1188 1.90036225831136e-05
1189 1.790797796275e-05
1190 1.55579691636376e-05
1191 1.29506497614784e-05
1192 1.12985089799622e-05
1193 1.12048637674889e-05
1194 1.25101141748019e-05
1195 1.45598523886292e-05
1196 1.65528635989176e-05
1197 1.79328108060872e-05
1198 1.83675765583757e-05
1199 1.76380472112214e-05
1200 1.62433552759467e-05
1201 1.44852110679494e-05
1202 1.2902944945381e-05
1203 1.1799162166426e-05
1204 1.13489886643947e-05
1205 1.16285636977409e-05
1206 1.2465307918319e-05
1207 1.38266887006466e-05
1208 1.5437399270013e-05
1209 1.70107032317901e-05
1210 1.78862446773564e-05
1211 1.76384673977736e-05
1212 1.6038044122979e-05
1213 1.37697070385912e-05
1214 1.18195721370284e-05
1215 1.0973088137689e-05
1216 1.1459462257335e-05
1217 1.29970321722794e-05
1218 1.49124662129907e-05
1219 1.66188001458067e-05
1220 1.77143447217532e-05
1221 1.78856607817579e-05
1222 1.70672738022404e-05
1223 1.57149570441106e-05
1224 1.40671354529331e-05
1225 1.26167260532384e-05
1226 1.16289802463143e-05
1227 1.12995367089752e-05
1228 1.16512874228647e-05
1229 1.2502422578109e-05
1230 1.3937070434622e-05
1231 1.56420428538695e-05
1232 1.73599401023239e-05
1233 1.84783475560835e-05
1234 1.83473384822719e-05
1235 1.67705984495115e-05
1236 1.42572216645931e-05
1237 1.20135628094431e-05
1238 1.10258242784766e-05
1239 1.1595733667491e-05
1240 1.33822095449432e-05
1241 1.5636374882888e-05
1242 1.76167104655178e-05
1243 1.87439563887892e-05
1244 1.88236863323255e-05
1245 1.76585708686616e-05
1246 1.58941347763175e-05
1247 1.40032434501336e-05
1248 1.2402993888827e-05
1249 1.14933291115449e-05
1250 1.13888436317211e-05
1251 1.19739015644882e-05
1252 1.31575716295629e-05
1253 1.47922537507839e-05
1254 1.65643086802447e-05
1255 1.77991896634921e-05
1256 1.79882572410861e-05
1257 1.67522448464297e-05
1258 1.44989780892502e-05
1259 1.22404480862315e-05
1260 1.10011142169242e-05
1261 1.12526658995193e-05
1262 1.27230132420664e-05
1263 1.47692544487654e-05
1264 1.66162062669173e-05
1265 1.7784444935387e-05
1266 1.79010276042391e-05
1267 1.70449857250787e-05
1268 1.55243596964283e-05
1269 1.38084442369291e-05
1270 1.23454474305618e-05
1271 1.1404342330934e-05
1272 1.12002408059197e-05
1273 1.16821183837601e-05
1274 1.27319181046914e-05
1275 1.42551925819134e-05
1276 1.58995626406977e-05
1277 1.72188720171107e-05
1278 1.76711218955461e-05
1279 1.68136884894921e-05
1280 1.48467779581551e-05
1281 1.25865162772243e-05
1282 1.1148795238114e-05
1283 1.11346153062186e-05
1284 1.23219861052348e-05
1285 1.4378290870809e-05
1286 1.63872227858519e-05
1287 1.79226954060141e-05
1288 1.84569344128249e-05
1289 1.78352293005446e-05
1290 1.63949789566686e-05
1291 1.45433423313079e-05
1292 1.28138572108583e-05
1293 1.16617547973874e-05
1294 1.12104144136538e-05
1295 1.15155398816569e-05
1296 1.23700610856758e-05
1297 1.38284740387462e-05
1298 1.54648150783032e-05
1299 1.69157738127979e-05
1300 1.76338617166039e-05
1301 1.71216488524806e-05
1302 1.53423279698472e-05
1303 1.30570751935011e-05
1304 1.13484502435313e-05
1305 1.09587754195672e-05
1306 1.19880260172067e-05
1307 1.39454032250796e-05
1308 1.6063344446593e-05
1309 1.76899575308198e-05
1310 1.82263793249149e-05
1311 1.79213166120462e-05
1312 1.6496751413797e-05
1313 1.46345728353481e-05
1314 1.28816627693595e-05
1315 1.16428072942654e-05
1316 1.11824183477438e-05
1317 1.14834138003062e-05
1318 1.24091275210958e-05
1319 1.39345611387398e-05
1320 1.56171699927654e-05
1321 1.70641651493497e-05
1322 1.76681623997865e-05
1323 1.69480117619969e-05
1324 1.49788584167254e-05
1325 1.26599170471309e-05
1326 1.11125400508172e-05
1327 1.1063898455177e-05
1328 1.24118996609468e-05
1329 1.45231006172253e-05
1330 1.65547880897066e-05
1331 1.7893193216878e-05
1332 1.81213708856376e-05
1333 1.72625859704567e-05
1334 1.56804908328922e-05
1335 1.38181385409553e-05
1336 1.22501796795405e-05
1337 1.12771376734599e-05
1338 1.11640538307256e-05
1339 1.1711926163116e-05
1340 1.28767160276766e-05
1341 1.44546174851712e-05
1342 1.60803501785267e-05
1343 1.71914834936615e-05
1344 1.72608579305233e-05
1345 1.59093906404451e-05
1346 1.37089573399862e-05
1347 1.16798819362884e-05
1348 1.08425365397125e-05
1349 1.15106004159315e-05
1350 1.32997975015314e-05
1351 1.54608223965624e-05
1352 1.72138588823145e-05
1353 1.80440547410399e-05
1354 1.776476528903e-05
1355 1.65135716088116e-05
1356 1.4749899492017e-05
1357 1.29677546283347e-05
1358 1.16408773465082e-05
1359 1.1068286767113e-05
1360 1.12545421870891e-05
1361 1.21372850117041e-05
1362 1.34936090034898e-05
1363 1.51674776134314e-05
1364 1.66737190738786e-05
1365 1.73362641362473e-05
1366 1.66817590070423e-05
1367 1.47763475979446e-05
1368 1.25223396025831e-05
1369 1.10304445115617e-05
1370 1.10021537693683e-05
1371 1.23828876894549e-05
1372 1.45274279930163e-05
1373 1.66137288033497e-05
1374 1.79363596544135e-05
1375 1.81635459739482e-05
1376 1.7275442587561e-05
1377 1.5612944480381e-05
1378 1.37197248477605e-05
1379 1.21105358630302e-05
1380 1.11721337816562e-05
1381 1.11023337012739e-05
1382 1.17788076750003e-05
1383 1.30300695673213e-05
1384 1.4741195627721e-05
1385 1.6357942513423e-05
1386 1.72508807736449e-05
1387 1.68936421687249e-05
1388 1.51707763507147e-05
1389 1.28927640616894e-05
1390 1.11759118226473e-05
1391 1.08552012534346e-05
1392 1.20074309961637e-05
1393 1.40653664857382e-05
1394 1.6202369806706e-05
1395 1.76459325302858e-05
1396 1.788818553905e-05
1397 1.71486171893775e-05
1398 1.55434099724516e-05
1399 1.36526014102856e-05
1400 1.20889717436512e-05
1401 1.11459494291921e-05
1402 1.10062837848091e-05
1403 1.16546034405474e-05
1404 1.28046658574021e-05
1405 1.44321702464367e-05
1406 1.60613944899524e-05
1407 1.70780476764776e-05
1408 1.68164406204596e-05
1409 1.51793447003001e-05
1410 1.29180352814728e-05
1411 1.11579574877396e-05
1412 1.07928717625327e-05
1413 1.19031719805207e-05
1414 1.39464164021774e-05
1415 1.60886429512175e-05
1416 1.75353252416244e-05
1417 1.79242015292402e-05
1418 1.71432966453722e-05
1419 1.5519499356742e-05
1420 1.35914142447291e-05
1421 1.19687356345821e-05
1422 1.10476503323298e-05
1423 1.09344437078107e-05
1424 1.15639941213885e-05
1425 1.27088987937896e-05
1426 1.43315310197067e-05
1427 1.58787006512284e-05
1428 1.67489797604503e-05
1429 1.64404082170222e-05
1430 1.48662411447731e-05
1431 1.27073853946058e-05
1432 1.10461860458599e-05
1433 1.07474525066209e-05
1434 1.18848056445131e-05
1435 1.3913761904405e-05
1436 1.60503477673046e-05
1437 1.75147906702477e-05
1438 1.7892909454531e-05
1439 1.71064511960139e-05
1440 1.54685531015275e-05
1441 1.3514053534891e-05
1442 1.19192391139222e-05
1443 1.09917273221072e-05
1444 1.09032316686353e-05
1445 1.15561706479639e-05
1446 1.27537959997426e-05
1447 1.44185287354048e-05
1448 1.60118179337587e-05
1449 1.68408059835201e-05
1450 1.64535103976959e-05
1451 1.47304353959044e-05
1452 1.24822836369276e-05
1453 1.09100310510257e-05
1454 1.08040658233222e-05
1455 1.21496050269343e-05
1456 1.43027245940175e-05
1457 1.64065932040103e-05
1458 1.76915109477704e-05
1459 1.77419460669626e-05
1460 1.6733341908548e-05
1461 1.49244515341707e-05
1462 1.29498821479501e-05
1463 1.15105431177653e-05
1464 1.08210942926235e-05
1465 1.09836710180389e-05
1466 1.18123916763579e-05
1467 1.32355462483247e-05
1468 1.49458119267365e-05
1469 1.63341264851624e-05
1470 1.67506186699029e-05
1471 1.57484500959981e-05
1472 1.36823482534965e-05
1473 1.16243918455439e-05
1474 1.06146671896568e-05
1475 1.11717636173125e-05
1476 1.29502177514951e-05
1477 1.51558097059024e-05
1478 1.68647784448694e-05
1479 1.75998466147576e-05
1480 1.71701394720003e-05
1481 1.57105441758176e-05
1482 1.37772167363437e-05
1483 1.20860422612168e-05
1484 1.09830443761894e-05
1485 1.06945726656704e-05
1486 1.11671233753441e-05
1487 1.22470928545226e-05
1488 1.37961651489604e-05
1489 1.53775508806575e-05
1490 1.63570475706365e-05
1491 1.61802854563575e-05
1492 1.47163209476275e-05
1493 1.25863161883899e-05
1494 1.09273687485256e-05
1495 1.06210500234738e-05
1496 1.17531717478414e-05
1497 1.3787215721095e-05
1498 1.58742295752745e-05
1499 1.72449490491999e-05
1500 1.75194800249301e-05
1501 1.66682475537527e-05
1502 1.50095593198785e-05
1503 1.30873049783986e-05
1504 1.15589336928679e-05
1505 1.07345340438769e-05
1506 1.07401192508405e-05
1507 1.1407227248128e-05
1508 1.2647440598812e-05
1509 1.4266865036916e-05
1510 1.576716931595e-05
1511 1.64615921676159e-05
1512 1.5877954865573e-05
1513 1.41019681905163e-05
1514 1.19901105790632e-05
1515 1.06484540083329e-05
1516 1.07900486909784e-05
1517 1.22881083370885e-05
1518 1.44774421642069e-05
1519 1.64759239851264e-05
1520 1.75984623638215e-05
1521 1.74506621988257e-05
1522 1.63216318469495e-05
1523 1.44706882565515e-05
1524 1.25797778309789e-05
1525 1.1211984201509e-05
1526 1.06357620097697e-05
1527 1.08965277831885e-05
1528 1.18194602691801e-05
1529 1.33363510030904e-05
1530 1.50291170939454e-05
1531 1.63049044203945e-05
1532 1.65174060384743e-05
1533 1.52842440002132e-05
1534 1.31614879137487e-05
1535 1.13524520202191e-05
1536 1.05646431620698e-05
1537 1.1376017027942e-05
1538 1.32977556859259e-05
1539 1.54710760398302e-05
1540 1.6998212231556e-05
1541 1.7500435205875e-05
1542 1.67881553352345e-05
1543 1.51421290865983e-05
1544 1.31792839965783e-05
1545 1.15738694148604e-05
1546 1.07224841485731e-05
1547 1.07315236164141e-05
1548 1.14567574200919e-05
1549 1.27512721519452e-05
1550 1.44223622555728e-05
1551 1.58887924044393e-05
1552 1.63947897817707e-05
1553 1.55954676301917e-05
1554 1.36698299684213e-05
1555 1.15976299639442e-05
1556 1.0536692570895e-05
1557 1.09872808025102e-05
1558 1.26749928313075e-05
1559 1.48266144606168e-05
1560 1.65798755915603e-05
1561 1.7309623217443e-05
1562 1.68813603522722e-05
1563 1.54495392052922e-05
1564 1.35565696837148e-05
1565 1.19062478916021e-05
1566 1.08349331640056e-05
1567 1.05665658338694e-05
1568 1.11203771666624e-05
1569 1.21583361760713e-05
1570 1.37137467390858e-05
1571 1.52610900840955e-05
1572 1.61665557243396e-05
1573 1.58673155965516e-05
1574 1.4335778359964e-05
1575 1.22161136459908e-05
1576 1.07295290945331e-05
1577 1.06161051007803e-05
1578 1.19115748020704e-05
1579 1.40165011544013e-05
1580 1.60732943186304e-05
1581 1.72774562088307e-05
1582 1.73828066181159e-05
1583 1.63513177540153e-05
1584 1.45233489092789e-05
1585 1.26138002087828e-05
1586 1.12179577627103e-05
1587 1.05833541965694e-05
1588 1.08586627902696e-05
1589 1.17599674922531e-05
1590 1.33000403366168e-05
1591 1.49887409861549e-05
1592 1.62143132911297e-05
1593 1.63691911438946e-05
1594 1.50752630361239e-05
1595 1.28790879898588e-05
1596 1.10317632788792e-05
1597 1.05113876998075e-05
1598 1.15493203338701e-05
1599 1.3601688806375e-05
1600 1.57513513840968e-05
1601 1.71242681972217e-05
1602 1.73183816514211e-05
1603 1.63223394338274e-05
1604 1.45244594023097e-05
1605 1.26101249406929e-05
1606 1.11752651719144e-05
1607 1.05206299849669e-05
1608 1.07327996374806e-05
1609 1.16048649942968e-05
1610 1.30672633531503e-05
1611 1.46347128975322e-05
1612 1.57491922436748e-05
1613 1.58666898641968e-05
1614 1.46244956340524e-05
1615 1.25817359730718e-05
1616 1.08894091681577e-05
1617 1.04782156995498e-05
1618 1.15316452138359e-05
1619 1.35455875351909e-05
1620 1.56534279085463e-05
1621 1.69890663528349e-05
1622 1.71599531313404e-05
1623 1.6198682715185e-05
1624 1.44399873533985e-05
1625 1.25391752590076e-05
1626 1.11233175630332e-05
1627 1.04848986666184e-05
1628 1.07428222690942e-05
1629 1.1735094631149e-05
1630 1.33541607283405e-05
1631 1.51410304169985e-05
1632 1.63967433763901e-05
1633 1.64362390933093e-05
1634 1.48875706145191e-05
1635 1.25633696370642e-05
1636 1.07946843854734e-05
1637 1.06431152744335e-05
1638 1.21390385174891e-05
1639 1.44694504342624e-05
1640 1.65535348060075e-05
1641 1.75375171238557e-05
1642 1.70984021679033e-05
1643 1.55387460836209e-05
1644 1.35033042170107e-05
1645 1.1653445653792e-05
1646 1.05912176877609e-05
1647 1.05236540548503e-05
1648 1.13312644316466e-05
1649 1.27396579046035e-05
1650 1.44685800478328e-05
1651 1.57530648721149e-05
1652 1.57824779307703e-05
1653 1.44624673339422e-05
1654 1.23486051961663e-05
1655 1.06851266536978e-05
1656 1.04611935967114e-05
1657 1.16882947622798e-05
1658 1.37009501486318e-05
1659 1.55654906848213e-05
1660 1.6524332750123e-05
1661 1.62332307809265e-05
1662 1.49851948663127e-05
1663 1.32392015075311e-05
1664 1.15887414722238e-05
1665 1.05606986835483e-05
1666 1.03650563687552e-05
1667 1.09740503830835e-05
1668 1.21637658594409e-05
1669 1.37918823384098e-05
1670 1.52048460222431e-05
1671 1.5659448763472e-05
1672 1.48667104440392e-05
1673 1.30228127090959e-05
1674 1.1138545232825e-05
1675 1.03492648122483e-05
1676 1.10539103843621e-05
1677 1.28743731693248e-05
1678 1.50059549923753e-05
1679 1.65802957781125e-05
1680 1.70219973369967e-05
1681 1.62035321409348e-05
1682 1.45300118674641e-05
1683 1.25862234199303e-05
1684 1.11096305772662e-05
1685 1.03956635939539e-05
1686 1.05269009509357e-05
1687 1.14335543912603e-05
1688 1.28981582747656e-05
1689 1.4660873603134e-05
1690 1.59932733367896e-05
1691 1.60770177899394e-05
1692 1.46885049616685e-05
1693 1.24455418699654e-05
1694 1.0675339581212e-05
1695 1.04554601421114e-05
1696 1.1838576028822e-05
1697 1.41231921588769e-05
1698 1.62781288963743e-05
1699 1.73848111444386e-05
1700 1.70305775100132e-05
1701 1.55186917254468e-05
1702 1.34417114168173e-05
1703 1.1579119018279e-05
1704 1.0459733857715e-05
1705 1.03260754258372e-05
1706 1.10493956526625e-05
1707 1.23625022752094e-05
1708 1.40113033921807e-05
1709 1.53221208165633e-05
1710 1.55030502355658e-05
1711 1.43225961437565e-05
1712 1.23228483062121e-05
1713 1.06317665995448e-05
1714 1.02835565485293e-05
1715 1.13850283014472e-05
1716 1.33461117002298e-05
1717 1.52872416947503e-05
1718 1.63649547175737e-05
1719 1.62086453201482e-05
1720 1.50494988702121e-05
1721 1.3268887414597e-05
1722 1.15632055894821e-05
1723 1.04494756669737e-05
1724 1.01676341728307e-05
1725 1.07371315607452e-05
1726 1.183850599773e-05
1727 1.34121291921474e-05
1728 1.48841745613026e-05
1729 1.55054767674301e-05
1730 1.48541057569673e-05
1731 1.30528742374736e-05
1732 1.11254730654764e-05
1733 1.02271869764081e-05
1734 1.08497815745068e-05
1735 1.26676868603681e-05
1736 1.48828812598367e-05
1737 1.65464753081324e-05
1738 1.7029300579452e-05
1739 1.62432625074871e-05
1740 1.45175590660074e-05
1741 1.2544047422125e-05
1742 1.09945631265873e-05
1743 1.02339499790105e-05
1744 1.0376840691606e-05
1745 1.1256896868872e-05
1746 1.27051589515759e-05
1747 1.43865936479415e-05
1748 1.56177666212898e-05
1749 1.56174646690488e-05
1750 1.41933942359174e-05
1751 1.20483273349237e-05
1752 1.04682158053038e-05
1753 1.04358950920869e-05
1754 1.19175520012504e-05
1755 1.41676782732247e-05
1756 1.61850130098173e-05
1757 1.70852872543037e-05
1758 1.66236895893235e-05
1759 1.50882688103593e-05
1760 1.30323942357791e-05
1761 1.1231352800678e-05
1762 1.02349104054156e-05
1763 1.02292369774659e-05
1764 1.1019565135939e-05
1765 1.23973914014641e-05
1766 1.40436586661963e-05
1767 1.52396050907555e-05
1768 1.51896056195255e-05
1769 1.38825234898832e-05
1770 1.18665948320995e-05
1771 1.0373446457379e-05
1772 1.02957319541019e-05
1773 1.16258070192998e-05
1774 1.36859789563459e-05
1775 1.55745965457754e-05
1776 1.64753982971888e-05
1777 1.61225398187526e-05
1778 1.47639420902124e-05
1779 1.28863066493068e-05
1780 1.12149427877739e-05
1781 1.02389167295769e-05
1782 1.01139257822069e-05
1783 1.07882115116809e-05
1784 1.20435888675274e-05
1785 1.36739818117348e-05
1786 1.50457644849666e-05
1787 1.53764103743015e-05
1788 1.43495763040846e-05
1789 1.2370755939628e-05
1790 1.06077932287008e-05
1791 1.01619298220612e-05
1792 1.12174539026455e-05
1793 1.3268490874907e-05
1794 1.53939563460881e-05
1795 1.66829977388261e-05
1796 1.66698118846398e-05
1797 1.54755907715298e-05
1798 1.35620139189996e-05
1799 1.16797646114719e-05
1800 1.03837619462865e-05
1801 1.00061060948065e-05
1802 1.04982646007556e-05
1803 1.15280154204811e-05
1804 1.3080820281175e-05
1805 1.45822577906074e-05
1806 1.52797019836726e-05
1807 1.46869642776437e-05
1808 1.29120016936213e-05
1809 1.10049286377034e-05
1810 1.00909346656408e-05
1811 1.07374826257001e-05
1812 1.25661099446006e-05
1813 1.4755434676772e-05
1814 1.63608256116277e-05
1815 1.66975078172982e-05
1816 1.58185484906426e-05
1817 1.40927832035231e-05
1818 1.21347575259279e-05
1819 1.06344914456713e-05
1820 9.98512223304715e-06
1821 1.02884287116467e-05
1822 1.12490160972811e-05
1823 1.28353131003678e-05
1824 1.44558362080716e-05
1825 1.53842065628851e-05
1826 1.50558544191881e-05
1827 1.34161200548988e-05
1828 1.13935766421491e-05
1829 1.01958812592784e-05
1830 1.06067154774792e-05
1831 1.2362207598926e-05
1832 1.46151214721613e-05
1833 1.6287733160425e-05
1834 1.67783473443706e-05
1835 1.59361807163805e-05
1836 1.41898344736546e-05
1837 1.21846805996029e-05
1838 1.06408197098062e-05
1839 9.98256473394576e-06
1840 1.02853146017878e-05
1841 1.12453572000959e-05
1842 1.27837802210706e-05
1843 1.43626493809279e-05
1844 1.52353359226254e-05
1845 1.47741884575225e-05
1846 1.30756825456046e-05
1847 1.11250237750937e-05
1848 1.00988363556098e-05
1849 1.06380193756195e-05
1850 1.24284788398654e-05
1851 1.45582735058269e-05
1852 1.60851941473084e-05
1853 1.63827371579828e-05
1854 1.53809505718527e-05
1855 1.36554954224266e-05
1856 1.18059542728588e-05
1857 1.04152713902295e-05
1858 9.87884868663969e-06
1859 1.02513085948885e-05
1860 1.12169900603476e-05
1861 1.26987151816138e-05
1862 1.41743385029258e-05
1863 1.49341458381969e-05
1864 1.44370969792362e-05
1865 1.27953408082249e-05
1866 1.0965168257826e-05
1867 1.0032304089691e-05
1868 1.06075258372584e-05
1869 1.23759382404387e-05
1870 1.45061521834577e-05
1871 1.60835224960465e-05
1872 1.64895045600133e-05
1873 1.55528668983607e-05
1874 1.38696923386306e-05
1875 1.19562910185778e-05
1876 1.04176251625177e-05
1877 9.80570985120721e-06
1878 1.01251225714805e-05
1879 1.10820483314455e-05
1880 1.26294817164307e-05
1881 1.41967857416603e-05
1882 1.50856449181447e-05
1883 1.47296959767118e-05
1884 1.30821090351674e-05
1885 1.1095151421614e-05
1886 9.97874212771421e-06
1887 1.04491782622063e-05
1888 1.21910361485789e-05
1889 1.43842444231268e-05
1890 1.6093139493023e-05
1891 1.64930697792443e-05
1892 1.5491890735575e-05
1893 1.37475335577619e-05
1894 1.17871550173732e-05
1895 1.03164957181434e-05
1896 9.75755756371655e-06
1897 1.01346013252623e-05
1898 1.11573363028583e-05
1899 1.27177772810683e-05
1900 1.41602095027338e-05
1901 1.48279286804609e-05
1902 1.42517865242553e-05
1903 1.25519409266417e-05
1904 1.07272544482839e-05
1905 9.89380896498915e-06
1906 1.06011611933354e-05
1907 1.24192902148934e-05
1908 1.45160220199614e-05
1909 1.59797582455212e-05
1910 1.61789230332943e-05
1911 1.51390340761282e-05
1912 1.34017109303386e-05
1913 1.15115026346757e-05
1914 1.01617542895838e-05
1915 9.72651560005033e-06
1916 1.02214880826068e-05
1917 1.12974794319598e-05
1918 1.28212523122784e-05
1919 1.41958726089797e-05
1920 1.46992024383508e-05
1921 1.39970252348576e-05
1922 1.22771016322076e-05
1923 1.05542594610597e-05
1924 9.86176291917218e-06
1925 1.06660618257592e-05
1926 1.25449087136076e-05
1927 1.46277097883285e-05
1928 1.60006784426514e-05
1929 1.61749740072992e-05
1930 1.51461135828868e-05
1931 1.32998584376764e-05
1932 1.13809637696249e-05
1933 1.00804754765704e-05
1934 9.69809843809344e-06
1935 1.02440517366631e-05
1936 1.13627584141796e-05
1937 1.29570526041789e-05
1938 1.44065197673626e-05
1939 1.49291809066199e-05
1940 1.41627906486974e-05
1941 1.22934561659349e-05
1942 1.0487946383364e-05
1943 9.87227576842997e-06
1944 1.08681442725356e-05
1945 1.29069412651006e-05
1946 1.50511859828839e-05
1947 1.63084732776042e-05
1948 1.62292235472705e-05
1949 1.48933349919389e-05
1950 1.28824576677289e-05
1951 1.09904876808287e-05
1952 9.87026487564435e-06
1953 9.73392434389098e-06
1954 1.05094022728736e-05
1955 1.18513044071733e-05
1956 1.35321033667424e-05
1957 1.47825203384855e-05
1958 1.47434766404331e-05
1959 1.33983267005533e-05
1960 1.13739788503153e-05
1961 9.97101233224384e-06
1962 1.01255254776333e-05
1963 1.17059598778724e-05
1964 1.39026133183506e-05
1965 1.56290043378249e-05
1966 1.61980369739467e-05
1967 1.53271485032747e-05
1968 1.35721647893661e-05
1969 1.16349374366109e-05
1970 1.0173876034969e-05
1971 9.6145149655058e-06
1972 1.00160468718968e-05
1973 1.10831006168155e-05
1974 1.26348522826447e-05
1975 1.40121401273063e-05
1976 1.45023896038765e-05
1977 1.37291990540689e-05
1978 1.19602436825517e-05
1979 1.03065394796431e-05
1980 9.80672120931558e-06
1981 1.07997475424781e-05
1982 1.27202183648478e-05
1983 1.4661022760265e-05
1984 1.57629328896292e-05
1985 1.55894686031388e-05
1986 1.43078041219269e-05
1987 1.24537300507654e-05
1988 1.07335781649454e-05
1989 9.73060650721891e-06
1990 9.68034146353602e-06
1991 1.04577184174559e-05
1992 1.17930176202208e-05
1993 1.34000956677482e-05
1994 1.45996746141464e-05
1995 1.45852945934166e-05
1996 1.32615514303325e-05
1997 1.13040396172437e-05
1998 9.937441063812e-06
1999 1.00792103694403e-05
};
\addlegendentry{Test}

\nextgroupplot[
title={Leaky/Tanh},
ymin=3.73909664548189e-06, ymax=0.001,
]
\addplot [semithick, black, dashed]
table {%
0 0.100131925661117
1 0.098722318187356
2 0.0972828739322722
3 0.0957801439799368
4 0.0941811073571444
5 0.0924023268744349
6 0.0902047529816628
7 0.0869999271817505
8 0.0822098557837307
9 0.0757899505551904
10 0.0683129096869379
11 0.0611332431435585
12 0.0554903906304389
13 0.0510386067908257
14 0.0470881599467248
15 0.0435721036046743
16 0.040368928690441
17 0.0375021973159164
18 0.0349545563803986
19 0.0326681354781613
20 0.0305613996461034
21 0.0285476568387821
22 0.026679499191232
23 0.0248776295920834
24 0.0231670296634547
25 0.0215379704604857
26 0.0200078035122715
27 0.0185607442981564
28 0.0171574162668549
29 0.015782957081683
30 0.0144430151558481
31 0.0131445542210713
32 0.0119035847019404
33 0.0107426090398803
34 0.00969007893581875
35 0.0087467975099571
36 0.0079072003136389
37 0.00716381572419778
38 0.00650207597936969
39 0.00590298807946965
40 0.00535931412014179
41 0.00486469551105984
42 0.00441738261724822
43 0.00400810012069996
44 0.0036236842352082
45 0.00327165512135252
46 0.00295344915502938
47 0.00267111305583967
48 0.0024177910599974
49 0.00219146343806642
50 0.00198990265562315
51 0.00181220336526167
52 0.00165578691303381
53 0.0015186953896773
54 0.00139776520882151
55 0.00129228916375723
56 0.00119975966481434
57 0.00111943050615082
58 0.00104764380012057
59 0.000967444002526463
60 0.000886008650923031
61 0.000823661620415805
62 0.000771824787079822
63 0.000726970207324484
64 0.000687443993228953
65 0.000654785809274472
66 0.000627131958935934
67 0.000601674867539259
68 0.000580188703679596
69 0.000561320681754296
70 0.000544653567885689
71 0.000529514155914512
72 0.000516459845130157
73 0.000504723394897155
74 0.000492492168632452
75 0.000482500926636931
76 0.000473261604156505
77 0.000465465024944933
78 0.000458051821397021
79 0.000451255228881564
80 0.000445296294174113
81 0.000439233211409373
82 0.000433889605119475
83 0.000428617482612026
84 0.000424527820769072
85 0.000420315191149712
86 0.00041611460369495
87 0.000412223177363558
88 0.00040901780812419
89 0.000405759739123823
90 0.000402679193484801
91 0.000400348397761263
92 0.000397167306346091
93 0.000394441284697677
94 0.000392124166751273
95 0.000389706652413224
96 0.000386771241551287
97 0.000384376875331327
98 0.000382579665824778
99 0.000380348823568966
100 0.00037833852275071
101 0.000376519483552329
102 0.000375008049104508
103 0.000372890529547476
104 0.000371393181012536
105 0.000369539899566007
106 0.000368046027574565
107 0.000366116441227859
108 0.000365038006634677
109 0.000363366200645032
110 0.0003620991092248
111 0.000360179128961136
112 0.000358921019937952
113 0.00035766808707649
114 0.000356122790662994
115 0.000354023115733071
116 0.000352676255147344
117 0.000351559529121914
118 0.000350181391809201
119 0.000348339057268277
120 0.000347575948467238
121 0.000345231032497395
122 0.000344740307696156
123 0.000342898611279452
124 0.000341947157380673
125 0.000340769384138184
126 0.000339281041306094
127 0.000338090671903046
128 0.000336865653480345
129 0.000336064254952362
130 0.000334000634666154
131 0.000333882387963058
132 0.00033184141318543
133 0.000330804177337995
134 0.000329416477143241
135 0.000328276243180881
136 0.000326996791613965
137 0.000326168454080289
138 0.000324994563470682
139 0.000323462872302116
140 0.000322775614904458
141 0.000321416312772271
142 0.000320049831543656
143 0.000319501924650467
144 0.000317948940846691
145 0.000317058128189274
146 0.000315679326718055
147 0.00031484306089169
148 0.000313821268264292
149 0.000312684643859029
150 0.000311818100726668
151 0.000309918069660853
152 0.000308256156699827
153 0.000307715102735528
154 0.000305996996303293
155 0.000305140546288385
156 0.000303820361182261
157 0.000302687900102683
158 0.000301674744264346
159 0.000300734433494654
160 0.000299215423297028
161 0.00029847718496967
162 0.000297264970186006
163 0.000296541687362151
164 0.000295314593131479
165 0.00029423413798213
166 0.000293247155354948
167 0.000291994505460025
168 0.000291438227577601
169 0.000289974084239475
170 0.000288832462160826
171 0.000287686943238441
172 0.000286666074430286
173 0.000285762747353147
174 0.000284687372413828
175 0.000284118647869036
176 0.000282720576933571
177 0.000281887460346297
178 0.000280724501635632
179 0.00028005285355448
180 0.000278904566016536
181 0.000277946041592259
182 0.000276864178999858
183 0.00027568325788252
184 0.000274523351322387
185 0.00027384640065975
186 0.000272605153099903
187 0.000271804992735269
188 0.000270522663640804
189 0.000269662892264932
190 0.000268599904984512
191 0.000267755154368388
192 0.00026642374552921
193 0.000265442829913809
194 0.000264317496430522
195 0.000263215820837104
196 0.000262218578086504
197 0.000260965180359563
198 0.000259908296470712
199 0.000258651058175019
200 0.000257474557997739
201 0.000256418786307222
202 0.00025522723569793
203 0.000253847420140119
204 0.00025249467859112
205 0.000251244423054686
206 0.000249698021775657
207 0.000248514101997444
208 0.000247007555515211
209 0.000245705891529724
210 0.00024439833185852
211 0.000243066094014921
212 0.000241730972902587
213 0.00024035226306296
214 0.000238851388985495
215 0.000237205713801814
216 0.000235701722260728
217 0.000234191829463271
218 0.000232562977430462
219 0.000230924374704955
220 0.000229388081265824
221 0.000227720642101303
222 0.000226181172820361
223 0.000224401360185311
224 0.000222863027545372
225 0.000221178940876143
226 0.000219599405625104
227 0.000217955364405498
228 0.000216267333769338
229 0.000214536589908221
230 0.000212937690491799
231 0.000211426395139824
232 0.000209740861009777
233 0.000207976600506754
234 0.000206314709657818
235 0.000204732287443221
236 0.000203090265983974
237 0.000201478454755488
238 0.000199848399006441
239 0.000198243531258413
240 0.000196601512584493
241 0.000194987610427688
242 0.000193356626823515
243 0.000191707990325085
244 0.000190109532184124
245 0.00018848532050697
246 0.000186961854126366
247 0.000185359877150404
248 0.000183671538877661
249 0.00018208480054227
250 0.000180338712198136
251 0.000178537384897481
252 0.000176778920945253
253 0.000174962471248818
254 0.000173112576646872
255 0.000171219205697071
256 0.000169171599509355
257 0.000167140778785324
258 0.000165067165141863
259 0.000162948856939238
260 0.000160991584436943
261 0.000159017901978586
262 0.000157109318536186
263 0.000155256851030572
264 0.000153444129409763
265 0.000151700729389859
266 0.00015010924906278
267 0.000148521516280198
268 0.000146896120156725
269 0.000145390921403532
270 0.000143973788198082
271 0.000142515210797001
272 0.000141077019691238
273 0.000139790613445712
274 0.000138182171042445
275 0.000136482882993505
276 0.000135099755368628
277 0.000133857966375217
278 0.000132643627864582
279 0.000131535260891269
280 0.000130584069722772
281 0.000129628964145923
282 0.000128682348673692
283 0.000127807507084299
284 0.000127056166178363
285 0.000126204858844403
286 0.000125480242274989
287 0.000124756746430421
288 0.000123412016847624
289 0.000121619158818476
290 0.000120085098245681
291 0.000118796434747992
292 0.000117668843216734
293 0.000116657468993253
294 0.000115767378588316
295 0.000114900585771238
296 0.000114092031623159
297 0.000113303107042384
298 0.000112550950206014
299 0.000111772943839128
300 0.000111102486457071
301 0.000110294702054148
302 0.000109762986667761
303 0.000108950913997319
304 0.000108432611796161
305 0.000107677137933138
306 0.000107121198723803
307 0.000106409203894486
308 0.000105953747507215
309 0.000105220389812644
310 0.000104756924883986
311 0.000104032879264082
312 0.000103620420858874
313 0.00010290837036564
314 0.000102522870534472
315 0.000101757134586933
316 0.000101466082838897
317 0.000100620110515592
318 0.00010052193329102
319 9.95247728212689e-05
320 9.99082611201629e-05
321 9.85881642492359e-05
322 0.000100131123986102
323 9.85572302596438e-05
324 0.000102737879871029
325 0.000101644603176965
326 0.000107968310061324
327 0.000104332589941691
328 0.000106000355060587
329 9.89294336761759e-05
330 9.79723964604773e-05
331 9.43228232443971e-05
332 9.4002399805504e-05
333 9.27407716631024e-05
334 9.25208674829037e-05
335 9.1766346258737e-05
336 9.14798168309972e-05
337 9.08459411306239e-05
338 9.0503290863353e-05
339 8.99274462966559e-05
340 8.95825055096111e-05
341 8.9031523160088e-05
342 8.8506469865024e-05
343 8.78482289436988e-05
344 8.75418486003809e-05
345 8.68858971472264e-05
346 8.67965579374186e-05
347 8.5972948681956e-05
348 8.62115633566418e-05
349 8.5212494013831e-05
350 8.63813477565145e-05
351 8.54367797984423e-05
352 8.92545735808881e-05
353 8.92767251912119e-05
354 9.75586077061052e-05
355 9.6609433313688e-05
356 9.84358418349984e-05
357 8.9658680053617e-05
358 8.61972294501356e-05
359 8.17142720848096e-05
360 8.082001488674e-05
361 7.98876145040595e-05
362 7.94317118391064e-05
363 7.89811682722075e-05
364 7.84959261324047e-05
365 7.81277881856113e-05
366 7.76109111342294e-05
367 7.72538795814626e-05
368 7.67430414043702e-05
369 7.63845067126567e-05
370 7.58517469421349e-05
371 7.55168499324554e-05
372 7.4982526186318e-05
373 7.46269967635271e-05
374 7.41107333794844e-05
375 7.37605430884969e-05
376 7.31922747263525e-05
377 7.29042831864035e-05
378 7.21875506286551e-05
379 7.19345996031961e-05
380 7.11589189421602e-05
381 7.13659459705696e-05
382 7.03972774687145e-05
383 7.23200343912822e-05
384 7.22889886048961e-05
385 7.87468232203992e-05
386 8.25424880588344e-05
387 9.20748186103992e-05
388 8.68785523948645e-05
389 7.9958354746168e-05
390 6.97041601540604e-05
391 6.69517176135059e-05
392 6.56514988008894e-05
393 6.50027020014932e-05
394 6.46002798276868e-05
395 6.41484310790474e-05
396 6.38672456005906e-05
397 6.33545588328843e-05
398 6.30536627141964e-05
399 6.25175538715439e-05
400 6.22534614649339e-05
401 6.17503604303238e-05
402 6.14796147857533e-05
403 6.09761929553088e-05
404 6.0769329607524e-05
405 6.01655835339443e-05
406 6.00283475193919e-05
407 5.94167288809899e-05
408 5.94146738279733e-05
409 5.86903012802509e-05
410 5.90021861768264e-05
411 5.81003787303302e-05
412 5.91561635445714e-05
413 5.83230693891323e-05
414 6.12379520958939e-05
415 6.10924913786448e-05
416 6.72207193588292e-05
417 6.81010336052168e-05
418 7.52340704934795e-05
419 7.02707479405262e-05
420 6.86893477990225e-05
421 5.93742689645183e-05
422 5.7273038677863e-05
423 5.45046550683992e-05
424 5.41921660612843e-05
425 5.35493429651979e-05
426 5.3296342628073e-05
427 5.29840203853382e-05
428 5.26395555056069e-05
429 5.24072546923549e-05
430 5.20803729102681e-05
431 5.18105449103246e-05
432 5.15195995305362e-05
433 5.1231480462377e-05
434 5.09826471315478e-05
435 5.06547942720204e-05
436 5.0483735130058e-05
437 5.00955364088895e-05
438 5.00528886391294e-05
439 4.94944403328645e-05
440 4.965054065309e-05
441 4.8987622555785e-05
442 4.94564035804501e-05
443 4.85544474813082e-05
444 4.96877267224249e-05
445 4.85835867678475e-05
446 5.12826702419034e-05
447 5.05039524654194e-05
448 5.69793286047116e-05
449 5.76653758628254e-05
450 6.76861999124867e-05
451 6.51314386743707e-05
452 6.64078084753328e-05
453 5.4755828045927e-05
454 5.12502305980433e-05
455 4.65868254622137e-05
456 4.61522413743864e-05
457 4.55253589350946e-05
458 4.54265002076681e-05
459 4.51459003301125e-05
460 4.49434068769961e-05
461 4.47322152723473e-05
462 4.4564778846734e-05
463 4.43757007317913e-05
464 4.41592856006423e-05
465 4.39865443055965e-05
466 4.3762832362404e-05
467 4.36511930601569e-05
468 4.34143078464899e-05
469 4.32644277168492e-05
470 4.30509214623953e-05
471 4.28807711685408e-05
472 4.27503368953808e-05
473 4.25046450089894e-05
474 4.24496513886652e-05
475 4.2135381072228e-05
476 4.21801757610751e-05
477 4.17581183000948e-05
478 4.20502633033948e-05
479 4.14479423582748e-05
480 4.22156773112192e-05
481 4.1370347737768e-05
482 4.33710008636012e-05
483 4.24928415299064e-05
484 4.73354809429338e-05
485 4.8166839633268e-05
486 5.94629516399436e-05
487 6.18026531356008e-05
488 6.91989407641813e-05
489 5.77665224739121e-05
490 5.07398810611903e-05
491 4.16943360335154e-05
492 4.03212232313876e-05
493 3.96289502617719e-05
494 3.9435325930981e-05
495 3.92172785623757e-05
496 3.91005863846772e-05
497 3.90100279972216e-05
498 3.88106759316997e-05
499 3.8701539843089e-05
500 3.85259197202004e-05
501 3.84809674400799e-05
502 3.82499871278696e-05
503 3.82106930203463e-05
504 3.79534736509868e-05
505 3.79941078847423e-05
506 3.76794927987589e-05
507 3.77648568417044e-05
508 3.74073771070016e-05
509 3.7555881512219e-05
510 3.71369323204362e-05
511 3.74512228660251e-05
512 3.68971568569521e-05
513 3.74320101457215e-05
514 3.67415458164544e-05
515 3.77939162135021e-05
516 3.6913658071569e-05
517 3.8910942947723e-05
518 3.80866818261438e-05
519 4.21852543865953e-05
520 4.20185521363692e-05
521 4.91578719277186e-05
522 4.87547236787123e-05
523 5.48440617649248e-05
524 4.92246643091221e-05
525 4.72338859509591e-05
526 3.9519617953232e-05
527 3.81075970921074e-05
528 3.55349021106122e-05
529 3.55688460587089e-05
530 3.48826164611182e-05
531 3.49802069194993e-05
532 3.46618115329989e-05
533 3.46532342163641e-05
534 3.44262379741167e-05
535 3.44068067832382e-05
536 3.42166857407733e-05
537 3.41749208274678e-05
538 3.3986645494366e-05
539 3.39516627505532e-05
540 3.37653217812317e-05
541 3.37441706115271e-05
542 3.35367126567121e-05
543 3.35384761847024e-05
544 3.33107273320365e-05
545 3.33697501737618e-05
546 3.30855464341084e-05
547 3.32579076420814e-05
548 3.28694756177583e-05
549 3.32964704803373e-05
550 3.2792632367773e-05
551 3.38123850198713e-05
552 3.32486050886871e-05
553 3.57782534905482e-05
554 3.60490758311016e-05
555 4.26185013964187e-05
556 4.58994390157841e-05
557 5.67552951338257e-05
558 5.70420900487534e-05
559 5.40810541309611e-05
560 4.08696208182846e-05
561 3.55274323879939e-05
562 3.20333206111911e-05
563 3.16375880871078e-05
564 3.14285477642784e-05
565 3.14218545405254e-05
566 3.12396467876397e-05
567 3.11065687768064e-05
568 3.10463390320592e-05
569 3.09592531095859e-05
570 3.08642597275366e-05
571 3.07410142674769e-05
572 3.06792105355669e-05
573 3.05647293643574e-05
574 3.04935449264576e-05
575 3.03749832895051e-05
576 3.03155785523757e-05
577 3.01919267471362e-05
578 3.01342614363875e-05
579 3.00083782001082e-05
580 2.99620577592918e-05
581 2.98242342253729e-05
582 2.97921371554821e-05
583 2.96463355198284e-05
584 2.96016549370393e-05
585 2.94647213721078e-05
586 2.94427716269752e-05
587 2.92759508333518e-05
588 2.92909037469258e-05
589 2.90907019788733e-05
590 2.91893381074715e-05
591 2.8919128556737e-05
592 2.92596307360782e-05
593 2.89023718345049e-05
594 2.9843404739438e-05
595 2.97991710951351e-05
596 3.3086396719284e-05
597 3.50777443856032e-05
598 4.51221179389449e-05
599 5.44887029434449e-05
600 6.61662256220552e-05
601 5.83360550052703e-05
602 4.31154924811494e-05
603 3.08270856450577e-05
604 2.85447061187938e-05
605 2.82092329229044e-05
606 2.78615259539094e-05
607 2.76930410088028e-05
608 2.77121289293802e-05
609 2.76302618367197e-05
610 2.74894727780861e-05
611 2.74029964373312e-05
612 2.735417334776e-05
613 2.72779584307159e-05
614 2.71838698466809e-05
615 2.71034598604558e-05
616 2.70520703580246e-05
617 2.69484762185357e-05
618 2.69056367301346e-05
619 2.67914386604673e-05
620 2.67503391420121e-05
621 2.66422191756988e-05
622 2.66112390576723e-05
623 2.64921841655053e-05
624 2.64787410841905e-05
625 2.63332350414203e-05
626 2.63441443415502e-05
627 2.61871979212458e-05
628 2.62625666280769e-05
629 2.60473112945192e-05
630 2.62119825116258e-05
631 2.59596354439395e-05
632 2.63340947199708e-05
633 2.60537442571263e-05
634 2.69623589659318e-05
635 2.68776207335009e-05
636 2.92348177026724e-05
637 3.02088079564555e-05
638 3.58663628787781e-05
639 3.96391305059751e-05
640 4.73227813131416e-05
641 4.63532650769594e-05
642 4.19066937240586e-05
643 3.22318991372583e-05
644 2.79582578031068e-05
645 2.53923742228324e-05
646 2.50664571517234e-05
647 2.48361486896442e-05
648 2.48017533799327e-05
649 2.46890959871848e-05
650 2.45916118934986e-05
651 2.45329097303681e-05
652 2.44580147032991e-05
653 2.43879332124664e-05
654 2.42977641597975e-05
655 2.42395908180981e-05
656 2.41583021818315e-05
657 2.41003018999209e-05
658 2.40074724331407e-05
659 2.39440734013385e-05
660 2.38620486214813e-05
661 2.38024141907545e-05
662 2.37155244278142e-05
663 2.36543719864812e-05
664 2.35772342165319e-05
665 2.3516450358585e-05
666 2.34145940822827e-05
667 2.33898001162913e-05
668 2.32704777936021e-05
669 2.32898952603477e-05
670 2.3127448997684e-05
671 2.32588517121712e-05
672 2.30655075945663e-05
673 2.34682613644566e-05
674 2.3310163793866e-05
675 2.44612243598397e-05
676 2.49516756412049e-05
677 2.85802635247023e-05
678 3.19539435338356e-05
679 4.21140203457071e-05
680 5.13332466027805e-05
681 5.64945993346555e-05
682 4.52403759112485e-05
683 3.17327552679103e-05
684 2.37092471913058e-05
685 2.24897392229195e-05
686 2.23152830542972e-05
687 2.21820271875117e-05
688 2.20030804101157e-05
689 2.19731247561583e-05
690 2.19299194093026e-05
691 2.18510567329133e-05
692 2.1771586840913e-05
693 2.17116550231822e-05
694 2.16636956480443e-05
695 2.1594403172287e-05
696 2.14965298006575e-05
697 2.1393464503916e-05
698 2.12703358606348e-05
699 2.1186507105142e-05
700 2.10862008920287e-05
701 2.10235392472669e-05
702 2.09459191005124e-05
703 2.0887825357363e-05
704 2.08218844761632e-05
705 2.07495016759651e-05
706 2.06580245496468e-05
707 2.058640026803e-05
708 2.05172843301682e-05
709 2.04559324323839e-05
710 2.03774574707438e-05
711 2.03128618778692e-05
712 2.02452112283424e-05
713 2.01929963701275e-05
714 2.01202349643381e-05
715 2.00677079575939e-05
716 1.99960787625741e-05
717 1.99538734158011e-05
718 1.98713968817543e-05
719 1.98269097602122e-05
720 1.97493064231224e-05
721 1.97122540939176e-05
722 1.95618989753399e-05
723 1.95701077529975e-05
724 1.94274693061658e-05
725 1.96129651399701e-05
726 1.94610535757533e-05
727 2.0063124203773e-05
728 2.01441905325339e-05
729 2.20386144462736e-05
730 2.36731167895243e-05
731 3.04324205586681e-05
732 4.0770043028715e-05
733 6.23685233449578e-05
734 7.20840245946874e-05
735 5.09641795076732e-05
736 2.81786049072252e-05
737 2.1503380750687e-05
738 1.99198985129101e-05
739 1.91646176652682e-05
740 1.89593487434081e-05
741 1.88896109563075e-05
742 1.86098335976936e-05
743 1.84884674361285e-05
744 1.84993360292651e-05
745 1.84360181947341e-05
746 1.83366610855273e-05
747 1.82911965165289e-05
748 1.82611416263967e-05
749 1.82103135344391e-05
750 1.8047840338653e-05
751 1.78825173193786e-05
752 1.77240158834024e-05
753 1.76091518966359e-05
754 1.75178597743297e-05
755 1.74500402092548e-05
756 1.73929212241575e-05
757 1.73312674185411e-05
758 1.72680223968769e-05
759 1.72150011934491e-05
760 1.71467670639913e-05
761 1.70996454773586e-05
762 1.70370368337558e-05
763 1.69903463369536e-05
764 1.69303811947685e-05
765 1.68839128598108e-05
766 1.68237413564754e-05
767 1.67844852949628e-05
768 1.67171370026864e-05
769 1.66852731044997e-05
770 1.66168759232477e-05
771 1.65921933614754e-05
772 1.65135302356845e-05
773 1.65134506442399e-05
774 1.64171808751945e-05
775 1.64816354644159e-05
776 1.63583893950658e-05
777 1.65822530533433e-05
778 1.64905012525196e-05
779 1.72107916256437e-05
780 1.74721197190308e-05
781 1.97258902367992e-05
782 2.18913344980365e-05
783 2.91274564432342e-05
784 3.74560848968031e-05
785 4.88304596046873e-05
786 4.7779881285237e-05
787 3.45979386366935e-05
788 2.12000080921371e-05
789 1.69344879807909e-05
790 1.60795793195945e-05
791 1.58531704315124e-05
792 1.57597105747698e-05
793 1.57713168604801e-05
794 1.56644604416556e-05
795 1.55830836305881e-05
796 1.55569512969578e-05
797 1.55362196068154e-05
798 1.54824272904364e-05
799 1.54327494943374e-05
800 1.54083405821126e-05
801 1.53697679081688e-05
802 1.53287756088361e-05
803 1.52868130758321e-05
804 1.52576628273948e-05
805 1.5224406837433e-05
806 1.51804833699032e-05
807 1.51458570361029e-05
808 1.51079315120484e-05
809 1.5076465508912e-05
810 1.50426683447336e-05
811 1.49995191485175e-05
812 1.49688277204518e-05
813 1.492677055559e-05
814 1.49053218212458e-05
815 1.4860063609845e-05
816 1.48419129271105e-05
817 1.47818281419632e-05
818 1.47769810396881e-05
819 1.47144669035981e-05
820 1.47345166592316e-05
821 1.46476503477544e-05
822 1.47183029390874e-05
823 1.46109268275296e-05
824 1.48072316807912e-05
825 1.47032293185134e-05
826 1.52453803128338e-05
827 1.5388637759628e-05
828 1.69718053335544e-05
829 1.82319241943674e-05
830 2.33022451396891e-05
831 2.99903815061953e-05
832 4.19367951565164e-05
833 4.81889436230176e-05
834 4.05145687949471e-05
835 2.54067291294291e-05
836 1.7112631553573e-05
837 1.47855724073054e-05
838 1.44251000264717e-05
839 1.42933786886346e-05
840 1.42717914108914e-05
841 1.41240661406528e-05
842 1.40676131259454e-05
843 1.40815741911382e-05
844 1.40427525483489e-05
845 1.39931678528882e-05
846 1.39541804404786e-05
847 1.39406876638049e-05
848 1.39088018631384e-05
849 1.38786593417706e-05
850 1.3843730476637e-05
851 1.38230423818442e-05
852 1.37954375762916e-05
853 1.37730553877446e-05
854 1.37336207686189e-05
855 1.37131167541327e-05
856 1.36838165385456e-05
857 1.36696857495622e-05
858 1.36275019801957e-05
859 1.3610810775333e-05
860 1.3574033182806e-05
861 1.35650639432328e-05
862 1.35224128055e-05
863 1.35192752388491e-05
864 1.3464552901965e-05
865 1.34739658221861e-05
866 1.34132438418533e-05
867 1.34527530590844e-05
868 1.33688063641557e-05
869 1.34677716392417e-05
870 1.33657911458585e-05
871 1.36257214631286e-05
872 1.35496377726696e-05
873 1.42180838995642e-05
874 1.43684901701135e-05
875 1.62053323755629e-05
876 1.79667276185569e-05
877 2.40190339724222e-05
878 3.1931976678834e-05
879 4.32266775760581e-05
880 4.46785995364962e-05
881 3.32344544631269e-05
882 1.95633097774817e-05
883 1.44163645714457e-05
884 1.32929674858673e-05
885 1.31141899224474e-05
886 1.30491054530779e-05
887 1.30565540263916e-05
888 1.29651155695854e-05
889 1.28996948802751e-05
890 1.29023015311702e-05
891 1.28887124368759e-05
892 1.28485912380683e-05
893 1.28142193371872e-05
894 1.27912506204808e-05
895 1.27820375013243e-05
896 1.27467087303046e-05
897 1.27292641156629e-05
898 1.26964554745257e-05
899 1.26836370242245e-05
900 1.26468660628376e-05
901 1.26340580060003e-05
902 1.25908701686939e-05
903 1.2578733286972e-05
904 1.25414412952551e-05
905 1.25399225954226e-05
906 1.24892257904463e-05
907 1.24989398653241e-05
908 1.24359001425489e-05
909 1.24624646051785e-05
910 1.2392232923375e-05
911 1.2452749192704e-05
912 1.23570813066465e-05
913 1.24884414809401e-05
914 1.23806099896484e-05
915 1.2684556070397e-05
916 1.26189662807974e-05
917 1.33790904861542e-05
918 1.36234565211169e-05
919 1.57319342122264e-05
920 1.78409183462236e-05
921 2.37724473350909e-05
922 3.03868554567543e-05
923 3.82613196450166e-05
924 3.63145670974063e-05
925 2.66756466587026e-05
926 1.68068883503736e-05
927 1.30762038672572e-05
928 1.21797263012979e-05
929 1.20598878616818e-05
930 1.19952110626542e-05
931 1.20104206935423e-05
932 1.19608465052323e-05
933 1.19103952407329e-05
934 1.18899613816836e-05
935 1.18733123244397e-05
936 1.18518116636324e-05
937 1.18196999050468e-05
938 1.18002343150181e-05
939 1.1777437620708e-05
940 1.17588502313026e-05
941 1.1728517062437e-05
942 1.17144405269443e-05
943 1.16858255729824e-05
944 1.16766081070807e-05
945 1.16401864875826e-05
946 1.1630648003802e-05
947 1.15966305429538e-05
948 1.15963058178181e-05
949 1.15524990729199e-05
950 1.15595101348731e-05
951 1.15081908074188e-05
952 1.15390095700718e-05
953 1.14701663402883e-05
954 1.15377393399285e-05
955 1.14500895218583e-05
956 1.16150915960844e-05
957 1.15215357112675e-05
958 1.19359275405628e-05
959 1.19604284591901e-05
960 1.31247305965587e-05
961 1.38937642510939e-05
962 1.7435103714547e-05
963 2.15570224639805e-05
964 3.10044569857837e-05
965 3.8503610483076e-05
966 3.80289969257319e-05
967 2.69448362111291e-05
968 1.63253597875723e-05
969 1.19838899692581e-05
970 1.14127736896563e-05
971 1.13087587241267e-05
972 1.12658682205335e-05
973 1.11529741513294e-05
974 1.11275356040608e-05
975 1.11375401505853e-05
976 1.11141890819511e-05
977 1.10769800620858e-05
978 1.10541307289935e-05
979 1.10430364159697e-05
980 1.10293603317047e-05
981 1.10058829569937e-05
982 1.09858249537353e-05
983 1.09674462027698e-05
984 1.09574630648979e-05
985 1.0933975421068e-05
986 1.09269637196263e-05
987 1.08978602408882e-05
988 1.08943841130582e-05
989 1.08665154847642e-05
990 1.08678841463927e-05
991 1.08326929559865e-05
992 1.08354752690332e-05
993 1.07988074731225e-05
994 1.08155700164758e-05
995 1.07656907832876e-05
996 1.08030245442592e-05
997 1.0736851814741e-05
998 1.08187571097318e-05
999 1.07376147724381e-05
1000 1.09251115603115e-05
1001 1.08463819419313e-05
1002 1.1315205910023e-05
1003 1.13927056393237e-05
1004 1.26302241838516e-05
1005 1.3460556022693e-05
1006 1.70694419452389e-05
1007 2.13148431811305e-05
1008 3.03983975982192e-05
1009 3.6679141373952e-05
1010 3.50870486443e-05
1011 2.44902443853334e-05
1012 1.50286413855838e-05
1013 1.12582240596737e-05
1014 1.07215556983675e-05
1015 1.06288716992253e-05
1016 1.06039225720167e-05
1017 1.04917397347748e-05
1018 1.04724072951257e-05
1019 1.04740556388094e-05
1020 1.04649405550106e-05
1021 1.04343649720562e-05
1022 1.04132718110606e-05
1023 1.04053517162583e-05
1024 1.03933183694949e-05
1025 1.03788643617264e-05
1026 1.0354786978084e-05
1027 1.0340555181898e-05
1028 1.03377238058755e-05
1029 1.03146163450418e-05
1030 1.03069337562545e-05
1031 1.02853277450521e-05
1032 1.0287491639005e-05
1033 1.02556056500447e-05
1034 1.02704865696523e-05
1035 1.02264900654347e-05
1036 1.02663025884908e-05
1037 1.02088417675006e-05
1038 1.02991446357947e-05
1039 1.0226897045662e-05
1040 1.04594741703323e-05
1041 1.04273117003117e-05
1042 1.10668748654064e-05
1043 1.13549880786934e-05
1044 1.33363974192946e-05
1045 1.52960500443555e-05
1046 2.07903505327067e-05
1047 2.58510478374774e-05
1048 3.05925440642341e-05
1049 2.75270320884147e-05
1050 2.1119115764634e-05
1051 1.43548863125176e-05
1052 1.13268755157492e-05
1053 1.03428310715969e-05
1054 1.02104099912737e-05
1055 1.00652225469844e-05
1056 1.00834565550656e-05
1057 1.00179303643699e-05
1058 1.00194277727894e-05
1059 9.97692275994666e-06
1060 9.98501030302634e-06
1061 9.96162374633514e-06
1062 9.96110718176624e-06
1063 9.93326196585542e-06
1064 9.93229401657914e-06
1065 9.90627055319493e-06
1066 9.91526462890135e-06
1067 9.86259608737328e-06
1068 9.86206420527935e-06
1069 9.81102026376135e-06
1070 9.83532692089284e-06
1071 9.78729284106805e-06
1072 9.84206002030419e-06
1073 9.77647096611634e-06
1074 9.89288500718999e-06
1075 9.817093885367e-06
1076 1.00870119528196e-05
1077 1.00413805625976e-05
1078 1.0708714556884e-05
1079 1.10324256894501e-05
1080 1.30027776634734e-05
1081 1.49684425991836e-05
1082 2.04039776008358e-05
1083 2.54384435010024e-05
1084 3.13979698489675e-05
1085 2.97012045535894e-05
1086 2.17554000698783e-05
1087 1.39850683424925e-05
1088 1.06979382508143e-05
1089 9.80709553743964e-06
1090 9.70219596929667e-06
1091 9.61053174375337e-06
1092 9.63982352608639e-06
1093 9.59576555725761e-06
1094 9.57876928175949e-06
1095 9.55627583376639e-06
1096 9.55424859405696e-06
1097 9.54324162005804e-06
1098 9.5312583106022e-06
1099 9.51333590037962e-06
1100 9.50771561480934e-06
1101 9.49624874824195e-06
1102 9.49083857459954e-06
1103 9.47021497488265e-06
1104 9.46311908300856e-06
1105 9.44754274723181e-06
1106 9.4524017484332e-06
1107 9.42760052424774e-06
1108 9.43685267440486e-06
1109 9.4038181996936e-06
1110 9.42387694990998e-06
1111 9.38561421470752e-06
1112 9.43724795554601e-06
1113 9.37701265435464e-06
1114 9.49142203587883e-06
1115 9.42544977622362e-06
1116 9.71269049365731e-06
1117 9.72177553926201e-06
1118 1.05203231282047e-05
1119 1.09983564939853e-05
1120 1.34587235756101e-05
1121 1.5800022474366e-05
1122 2.21111289988585e-05
1123 2.8066148729522e-05
1124 3.25074491769328e-05
1125 2.7361268781334e-05
1126 1.8088870845645e-05
1127 1.16175404567542e-05
1128 9.74167827649808e-06
1129 9.37210226847895e-06
1130 9.31581259955294e-06
1131 9.24784180789118e-06
1132 9.25762652226325e-06
1133 9.24613962904886e-06
1134 9.22953003623661e-06
1135 9.21101911810496e-06
1136 9.20239345614249e-06
1137 9.19862497728019e-06
1138 9.18778698988376e-06
1139 9.17554993140612e-06
1140 9.16449181875834e-06
1141 9.154457350391e-06
1142 9.14733643853083e-06
1143 9.14227731030337e-06
1144 9.12843301037469e-06
1145 9.12188421420268e-06
1146 9.10986750835008e-06
1147 9.10682450516731e-06
1148 9.0992406827084e-06
1149 9.08916893394007e-06
1150 9.07990601639597e-06
1151 9.07090399682176e-06
1152 9.06758415286646e-06
1153 9.05680023421951e-06
1154 9.05654058946936e-06
1155 9.03673003183059e-06
1156 9.04048984207151e-06
1157 9.01883630977096e-06
1158 9.03635179838602e-06
1159 9.00119746649608e-06
1160 9.04455951467042e-06
1161 8.98959287276568e-06
1162 9.07880974665431e-06
1163 9.02392921009465e-06
1164 9.27160450370224e-06
1165 9.27276636630836e-06
1166 9.96798142693933e-06
1167 1.02553365728397e-05
1168 1.21486881337063e-05
1169 1.40130354822077e-05
1170 2.0731493656001e-05
1171 3.05642888633884e-05
1172 4.52166521256459e-05
1173 4.51120870081922e-05
1174 2.53321400158768e-05
1175 1.31926635731361e-05
1176 1.03495578542834e-05
1177 9.52142998045247e-06
1178 9.10105685125728e-06
1179 9.05568651088373e-06
1180 9.0341055170029e-06
1181 8.95421385926909e-06
1182 8.87119474679565e-06
1183 8.8657891677002e-06
1184 8.87676544802929e-06
1185 8.85761572533283e-06
1186 8.83279720831354e-06
1187 8.82450526784595e-06
1188 8.82318104267199e-06
1189 8.8134852527233e-06
1190 8.80066850417194e-06
1191 8.79059739045118e-06
1192 8.78555507455303e-06
1193 8.77587761927145e-06
1194 8.77357516859689e-06
1195 8.75982854875446e-06
1196 8.75565980962278e-06
1197 8.74574898723779e-06
1198 8.74853572963019e-06
1199 8.72977091681548e-06
1200 8.73183994265503e-06
1201 8.71300982563383e-06
1202 8.72921142747174e-06
1203 8.70035759792387e-06
1204 8.72293650111544e-06
1205 8.68301221679246e-06
1206 8.73009045232465e-06
1207 8.67852677544079e-06
1208 8.77489001993581e-06
1209 8.70411110742708e-06
1210 8.91092054722264e-06
1211 8.85480720924647e-06
1212 9.36218658864618e-06
1213 9.47978058896126e-06
1214 1.08574733239664e-05
1215 1.19455857969797e-05
1216 1.57848617945433e-05
1217 1.96011249471439e-05
1218 2.6151105132044e-05
1219 2.83058342773757e-05
1220 2.46776270387272e-05
1221 1.65690592872636e-05
1222 1.09466190991725e-05
1223 8.92288769627214e-06
1224 8.67395306336505e-06
1225 8.61384425565603e-06
1226 8.62658060984955e-06
1227 8.58780933388203e-06
1228 8.56401925375394e-06
1229 8.55603547744721e-06
1230 8.55547507239152e-06
1231 8.55034715385727e-06
1232 8.53655096033634e-06
1233 8.52651839000629e-06
1234 8.51676308410987e-06
1235 8.51997466977394e-06
1236 8.50739051383442e-06
1237 8.5083775891448e-06
1238 8.48883435544678e-06
1239 8.49424989723957e-06
1240 8.47982979923501e-06
1241 8.49314025863634e-06
1242 8.46467976955978e-06
1243 8.48328487723649e-06
1244 8.4497683534579e-06
1245 8.48798420083341e-06
1246 8.44193368809698e-06
1247 8.51329740036988e-06
1248 8.44945413902565e-06
1249 8.60018586301692e-06
1250 8.53576990511584e-06
1251 8.89147167804083e-06
1252 8.90518907148419e-06
1253 9.81350328288499e-06
1254 1.04138580745428e-05
1255 1.31494578496927e-05
1256 1.58938008212317e-05
1257 2.27121748537229e-05
1258 2.82347191102872e-05
1259 3.03854328080888e-05
1260 2.31569607365145e-05
1261 1.38725699514453e-05
1262 9.28672731248525e-06
1263 8.5337933999341e-06
1264 8.41522231986858e-06
1265 8.40553506797903e-06
1266 8.35754704642255e-06
1267 8.34414460371136e-06
1268 8.34320390108445e-06
1269 8.33793071741695e-06
1270 8.32928966332247e-06
1271 8.31696157721495e-06
1272 8.31150573699091e-06
1273 8.30650326832938e-06
1274 8.30211900293421e-06
1275 8.29380447253669e-06
1276 8.28710148415013e-06
1277 8.27875145859025e-06
1278 8.28091931204256e-06
1279 8.26963260980307e-06
1280 8.26895038397168e-06
1281 8.2573478188408e-06
1282 8.2617413106334e-06
1283 8.24600708337897e-06
1284 8.25015809713392e-06
1285 8.23371937652695e-06
1286 8.24657335929402e-06
1287 8.2217778167859e-06
1288 8.24209473471171e-06
1289 8.20967517611848e-06
1290 8.25019530115156e-06
1291 8.20150359359673e-06
1292 8.28277569286939e-06
1293 8.2167373181008e-06
1294 8.39562450494213e-06
1295 8.33160189195326e-06
1296 8.7586414663221e-06
1297 8.80556626192686e-06
1298 9.90026067526628e-06
1299 1.06729098305891e-05
1300 1.39030992301059e-05
1301 1.7998437414235e-05
1302 2.69968660404629e-05
1303 3.44376528005341e-05
1304 3.3401788883225e-05
1305 2.21370129906973e-05
1306 1.18250683911469e-05
1307 8.75228272612105e-06
1308 8.36478873367241e-06
1309 8.23079339085098e-06
1310 8.22162811164873e-06
1311 8.15857483171811e-06
1312 8.11925558341642e-06
1313 8.12386958415345e-06
1314 8.12500128688498e-06
1315 8.11118578791792e-06
1316 8.09441226223839e-06
1317 8.09063341034033e-06
1318 8.08807487384655e-06
1319 8.08353005599116e-06
1320 8.07426115567722e-06
1321 8.06467045766368e-06
1322 8.06588626645066e-06
1323 8.05690826322092e-06
1324 8.0595660207905e-06
1325 8.04391943720617e-06
1326 8.04746483495933e-06
1327 8.03440656760301e-06
1328 8.0434141649377e-06
1329 8.02337173766432e-06
1330 8.03300370844084e-06
1331 8.01195417743372e-06
1332 8.03127419235494e-06
1333 7.99980081289675e-06
1334 8.02922361575753e-06
1335 7.98970748583372e-06
1336 8.04391737219134e-06
1337 7.98640373389503e-06
1338 8.09058029549448e-06
1339 8.01527547622527e-06
1340 8.23910870462186e-06
1341 8.17729767188524e-06
1342 8.68819272614729e-06
1343 8.8214356601668e-06
1344 1.01780010179908e-05
1345 1.11224753425176e-05
1346 1.51729143738066e-05
1347 1.96571675274981e-05
1348 2.76379469710264e-05
1349 3.08976461838029e-05
1350 2.51253603380519e-05
1351 1.56637523751613e-05
1352 9.87532357044074e-06
1353 8.18169569960503e-06
1354 7.99341411550358e-06
1355 7.9277670765876e-06
1356 7.94560684980183e-06
1357 7.91056129756385e-06
1358 7.88040795463019e-06
1359 7.87182686057264e-06
1360 7.87374224131554e-06
1361 7.87276452030028e-06
1362 7.86221278215038e-06
1363 7.85219016918859e-06
1364 7.84631646499889e-06
1365 7.84503480844023e-06
1366 7.84133159648093e-06
1367 7.83276353288898e-06
1368 7.82792970976942e-06
1369 7.82794078713067e-06
1370 7.81794614201914e-06
1371 7.81818452999516e-06
1372 7.80936120392539e-06
1373 7.81514456260624e-06
1374 7.7976607801844e-06
1375 7.80785225362735e-06
1376 7.786437513424e-06
1377 7.80914001463628e-06
1378 7.77635658089082e-06
1379 7.81769730107129e-06
1380 7.769578557415e-06
1381 7.85432109129403e-06
1382 7.78527027112119e-06
1383 7.96433518157613e-06
1384 7.89375392429292e-06
1385 8.31229779585385e-06
1386 8.34373674685196e-06
1387 9.37202753625854e-06
1388 9.98307122301867e-06
1389 1.26201528054537e-05
1390 1.52203394634398e-05
1391 2.21873628447611e-05
1392 2.8645016129758e-05
1393 3.06373493543788e-05
1394 2.30351993515399e-05
1395 1.27930214794958e-05
1396 8.5062439616479e-06
1397 7.87506520971704e-06
1398 7.76847296624084e-06
1399 7.75589061330351e-06
1400 7.71223324402825e-06
1401 7.70177028464047e-06
1402 7.70181933340552e-06
1403 7.69863469063381e-06
1404 7.6903667620698e-06
1405 7.68026551689616e-06
1406 7.67551083313833e-06
1407 7.67263925638417e-06
1408 7.66954095343664e-06
1409 7.66296680243528e-06
1410 7.65648933764851e-06
1411 7.6567408191508e-06
1412 7.65031291738438e-06
1413 7.64888158055754e-06
1414 7.6407076514684e-06
1415 7.643474457808e-06
1416 7.63464002684344e-06
1417 7.63932638925269e-06
1418 7.62378919993978e-06
1419 7.63028444517033e-06
1420 7.61442125352119e-06
1421 7.62727545478015e-06
1422 7.61029064477725e-06
1423 7.63006673309974e-06
1424 7.59743728728779e-06
1425 7.62998794279213e-06
1426 7.59269472538904e-06
1427 7.66349726166737e-06
1428 7.6054200723874e-06
1429 7.75128730090557e-06
1430 7.68744536738808e-06
1431 8.02937480504085e-06
1432 8.04525738473671e-06
1433 8.90805509978776e-06
1434 9.34361210802592e-06
1435 1.16175703510635e-05
1436 1.39600887933966e-05
1437 2.09947441902614e-05
1438 2.91839033934593e-05
1439 3.44449484117604e-05
1440 2.80862043702257e-05
1441 1.44944492355137e-05
1442 8.64223458130198e-06
1443 7.81348935596782e-06
1444 7.63849024298935e-06
1445 7.60472183625183e-06
1446 7.55041650624122e-06
1447 7.54419813731033e-06
1448 7.5421807874676e-06
1449 7.53408915077358e-06
1450 7.52358142186438e-06
1451 7.51789978270523e-06
1452 7.5158290151478e-06
1453 7.50906479130009e-06
1454 7.50982781383414e-06
1455 7.50043999175887e-06
1456 7.50038393793062e-06
1457 7.49242706010733e-06
1458 7.49186998305618e-06
1459 7.48802314021191e-06
1460 7.48458801425045e-06
1461 7.47878191820206e-06
1462 7.47566529479826e-06
1463 7.47252256338982e-06
1464 7.46999296463713e-06
1465 7.46587750644068e-06
1466 7.46192137413004e-06
1467 7.45880813379785e-06
1468 7.45367211862913e-06
1469 7.45491788745056e-06
1470 7.44745850056461e-06
1471 7.44812429864794e-06
1472 7.43903126565471e-06
1473 7.4430727137198e-06
1474 7.43245925871605e-06
1475 7.44330883595268e-06
1476 7.42080906679377e-06
1477 7.44196680546594e-06
1478 7.41304105211782e-06
1479 7.45926947676168e-06
1480 7.40860252790299e-06
1481 7.49117753873207e-06
1482 7.42601931147391e-06
1483 7.5958384808672e-06
1484 7.5429488948231e-06
1485 7.93366483886615e-06
1486 7.96707650163597e-06
1487 8.92819064368666e-06
1488 9.57022236569571e-06
1489 1.22042938919265e-05
1490 1.5092298692565e-05
1491 2.35434918920419e-05
1492 3.42477328274526e-05
1493 3.74234205082757e-05
1494 2.81233122052527e-05
1495 1.32913378898536e-05
1496 8.30278347940805e-06
1497 7.79730626021546e-06
1498 7.54176579187416e-06
1499 7.47239882148421e-06
1500 7.37551241591916e-06
1501 7.36384061461592e-06
1502 7.37310169984795e-06
1503 7.36379701038459e-06
1504 7.34517519163091e-06
1505 7.33821176712723e-06
1506 7.33793178309128e-06
1507 7.33558249166322e-06
1508 7.32918929102766e-06
1509 7.32360360000683e-06
1510 7.3204967447893e-06
1511 7.31768112505193e-06
1512 7.31008246201981e-06
1513 7.30992981079481e-06
1514 7.30276482308057e-06
1515 7.30332324039296e-06
1516 7.29596000503818e-06
1517 7.29370975527388e-06
1518 7.28657494875051e-06
1519 7.28520427450974e-06
1520 7.28051297382848e-06
1521 7.27897140695433e-06
1522 7.27201289585366e-06
1523 7.27210812545565e-06
1524 7.26631011138679e-06
1525 7.26400603578981e-06
1526 7.26141119855583e-06
1527 7.25961935099662e-06
1528 7.2500720191826e-06
1529 7.25651567012875e-06
1530 7.24383883587976e-06
1531 7.25441114113323e-06
1532 7.23779757993981e-06
1533 7.24821510278417e-06
1534 7.23049634787643e-06
1535 7.24431630150235e-06
1536 7.22529063912347e-06
1537 7.24291138620714e-06
1538 7.21812317605597e-06
1539 7.24508326754858e-06
1540 7.21252810098605e-06
1541 7.2628104526018e-06
1542 7.2097586754083e-06
1543 7.29963682211121e-06
1544 7.22657084528677e-06
1545 7.38115938148098e-06
1546 7.28703572860923e-06
1547 7.56513206390252e-06
1548 7.47790348309962e-06
1549 7.98843512939129e-06
1550 7.93409367272346e-06
1551 8.82781246147601e-06
1552 9.07969182240009e-06
1553 1.10954713932188e-05
1554 1.28065883533424e-05
1555 1.90674038549332e-05
1556 3.04762042304674e-05
1557 4.64281881704665e-05
1558 4.65462685994567e-05
1559 2.13010542182701e-05
1560 1.02233773162652e-05
1561 9.00318921281951e-06
1562 7.84502534756371e-06
1563 7.46671852702718e-06
1564 7.36794394562423e-06
1565 7.3268093583323e-06
1566 7.24076362246251e-06
1567 7.1664955827444e-06
1568 7.16183887083943e-06
1569 7.16838627301541e-06
1570 7.15329775236029e-06
1571 7.13540664154522e-06
1572 7.13350830316983e-06
1573 7.13398161344969e-06
1574 7.12808267699927e-06
1575 7.12114225098759e-06
1576 7.11476614334572e-06
1577 7.11507180461979e-06
1578 7.11109301132495e-06
1579 7.11206854386859e-06
1580 7.10317823848783e-06
1581 7.09964497591642e-06
1582 7.09694423495222e-06
1583 7.09441689838997e-06
1584 7.09273740184813e-06
1585 7.08591251896706e-06
1586 7.08624160328242e-06
1587 7.07782590447792e-06
1588 7.08129061788298e-06
1589 7.07372828179231e-06
1590 7.0736539239391e-06
1591 7.06806130157389e-06
1592 7.06499918923242e-06
1593 7.06359372948384e-06
1594 7.06098379898989e-06
1595 7.05675201917444e-06
1596 7.05821288793373e-06
1597 7.04709531884617e-06
1598 7.05515473642393e-06
1599 7.04056825329147e-06
1600 7.05573868975407e-06
1601 7.0345122020754e-06
1602 7.05186855043394e-06
1603 7.02760913728895e-06
1604 7.05768083797409e-06
1605 7.0224340613656e-06
1606 7.07976198111027e-06
1607 7.02662615736926e-06
1608 7.14334973039854e-06
1609 7.07698697155479e-06
1610 7.33599380886574e-06
1611 7.31906882922573e-06
1612 7.99875481849455e-06
1613 8.32183899035499e-06
1614 1.02548636622402e-05
1615 1.21639720838118e-05
1616 1.84120553186062e-05
1617 2.67720754720813e-05
1618 3.6183320503369e-05
1619 3.25366937339311e-05
1620 1.64564791198529e-05
1621 8.75009155620887e-06
1622 7.5399713157509e-06
1623 7.15488612357262e-06
1624 7.0635717164258e-06
1625 6.99915412027963e-06
1626 7.00113027241045e-06
1627 6.99740567045026e-06
1628 6.98064715187741e-06
1629 6.96800731603275e-06
1630 6.96407585509817e-06
1631 6.96899474039725e-06
1632 6.96287051571431e-06
1633 6.95529702943176e-06
1634 6.95124015548387e-06
1635 6.95102167425077e-06
1636 6.9485087679233e-06
1637 6.94384842159224e-06
1638 6.94026147884941e-06
1639 6.93710545807136e-06
1640 6.93604684665061e-06
1641 6.93263522189014e-06
1642 6.92813335145814e-06
1643 6.9270661464671e-06
1644 6.92251213729733e-06
1645 6.92107424127286e-06
1646 6.91760877735703e-06
1647 6.91649247785975e-06
1648 6.91247671591455e-06
1649 6.90905665434727e-06
1650 6.90723652052583e-06
1651 6.90449609841437e-06
1652 6.9035135501494e-06
1653 6.89861258873492e-06
1654 6.89985747559518e-06
1655 6.89276587539922e-06
1656 6.89317186619576e-06
1657 6.88825695949191e-06
1658 6.88971975115749e-06
1659 6.88511140856107e-06
1660 6.88455981823211e-06
1661 6.88073506482922e-06
1662 6.87794798492902e-06
1663 6.876510803e-06
1664 6.87260070675322e-06
1665 6.8738823664205e-06
1666 6.86889738288343e-06
1667 6.86713034747655e-06
1668 6.8654878306873e-06
1669 6.86531188298645e-06
1670 6.86035226094006e-06
1671 6.85828910684805e-06
1672 6.85576684800182e-06
1673 6.85456594018419e-06
1674 6.85251662124742e-06
1675 6.85185881987849e-06
1676 6.8449450347785e-06
1677 6.85135051003982e-06
1678 6.84341602896765e-06
1679 6.8437766520546e-06
1680 6.8384567017965e-06
1681 6.83894286179765e-06
1682 6.83579885540908e-06
1683 6.82813015417949e-06
1684 6.83179842697612e-06
1685 6.82771085802614e-06
1686 6.82429474396784e-06
1687 6.82412043584435e-06
1688 6.8217154298722e-06
1689 6.82068577617656e-06
1690 6.81845939665138e-06
1691 6.81931517165779e-06
1692 6.82280043085015e-06
1693 6.81323278994839e-06
1694 6.81159598991954e-06
1695 6.8153894359213e-06
1696 6.80714094647428e-06
1697 6.79798265590392e-06
1698 6.79846080675617e-06
1699 6.79921806412409e-06
1700 6.79041136519842e-06
1701 6.78385451013952e-06
1702 6.78867288872453e-06
1703 6.7943543542448e-06
1704 6.78947880849989e-06
1705 6.78329641745634e-06
1706 6.79746669973724e-06
1707 6.82521855566165e-06
1708 6.84283099694483e-06
1709 6.83924616273757e-06
1710 6.84628689207045e-06
1711 6.9371433948362e-06
1712 7.04657232608596e-06
1713 7.07698207591534e-06
1714 7.03129356072196e-06
1715 7.24066294832681e-06
1716 7.61987136410625e-06
1717 7.80457140070467e-06
1718 7.47753134078266e-06
1719 7.91873484473626e-06
1720 9.05355849489808e-06
1721 9.47887083668775e-06
1722 8.28364417770899e-06
1723 9.37158926284098e-06
1724 1.02187365804696e-05
1725 9.11576229256639e-06
1726 8.83807660656544e-06
1727 9.81476583206131e-06
1728 8.50911829353151e-06
1729 8.74520920923771e-06
1730 9.03689284825759e-06
1731 8.93144866598305e-06
1732 9.79448104487801e-06
1733 8.75158063351478e-06
1734 7.79144965434853e-06
1735 7.8518518549231e-06
1736 8.22674833500514e-06
1737 9.13080150954926e-06
1738 9.36665253092883e-06
1739 9.1216227353641e-06
1740 8.47971463358022e-06
1741 8.14231487566985e-06
1742 8.47058653352661e-06
1743 8.20346615171275e-06
1744 7.71040849656401e-06
1745 7.34025762216817e-06
1746 7.43442148021245e-06
1747 7.78111703692019e-06
1748 7.84567421341364e-06
1749 7.70920047798995e-06
1750 7.32315579199394e-06
1751 7.25824964753485e-06
1752 7.2979649385907e-06
1753 7.13971067334285e-06
1754 6.94420256763806e-06
1755 6.85726168114797e-06
1756 6.97605999278039e-06
1757 7.17534917660245e-06
1758 7.24378863559139e-06
1759 7.14390644773744e-06
1760 6.96616696949093e-06
1761 6.89409560017396e-06
1762 6.95174877840543e-06
1763 6.9815560861386e-06
1764 6.9027925428955e-06
1765 6.78518587271526e-06
1766 6.73028029840594e-06
1767 6.79301201422788e-06
1768 6.96810336053844e-06
1769 7.12001560110309e-06
1770 7.17523408422238e-06
1771 7.11505830786052e-06
1772 6.95605759482021e-06
1773 6.82836014176402e-06
1774 6.87248922304207e-06
1775 7.02343034575392e-06
1776 7.13895542325815e-06
1777 7.14139907476152e-06
1778 7.03140116264933e-06
1779 6.88550773730867e-06
1780 6.7988254350837e-06
1781 6.86032305097228e-06
1782 7.12398680491333e-06
1783 7.52023973493721e-06
1784 7.90589502308592e-06
1785 8.14185614395058e-06
1786 8.16292297400878e-06
1787 7.90782183202055e-06
1788 7.42870203929158e-06
1789 7.30235591905881e-06
1790 7.86985508582916e-06
1791 8.65466807908888e-06
1792 8.98995599207808e-06
1793 8.89237027834611e-06
1794 8.52437977982845e-06
1795 8.05833526662525e-06
1796 7.8723836409722e-06
1797 8.7382581206441e-06
1798 1.0295113115788e-05
1799 1.11892400482816e-05
1800 1.11119814647509e-05
1801 1.00194991894398e-05
1802 8.97183683612468e-06
1803 1.01543360280232e-05
1804 9.41510469942841e-06
1805 7.97132662455624e-06
1806 7.78634088050012e-06
1807 8.24169668867114e-06
1808 8.52371194071111e-06
1809 7.8121311126722e-06
1810 7.1507577450447e-06
1811 7.08840154040047e-06
1812 6.89583731094956e-06
1813 6.69828112664916e-06
1814 6.75996867371964e-06
1815 6.92485138076293e-06
1816 6.92457458484341e-06
1817 6.74217243190611e-06
1818 6.58107129636321e-06
1819 6.5272306564168e-06
1820 6.50188146833486e-06
1821 6.47214441817923e-06
1822 6.48480684706243e-06
1823 6.54811500888641e-06
1824 6.61831134340929e-06
1825 6.63441424109124e-06
1826 6.59147557868778e-06
1827 6.52683772051077e-06
1828 6.50497953413876e-06
1829 6.51451364008793e-06
1830 6.50803172286274e-06
1831 6.47509764251097e-06
1832 6.4495309906043e-06
1833 6.46867080567048e-06
1834 6.54792184073827e-06
1835 6.63500656195026e-06
1836 6.66854966890895e-06
1837 6.64436013320113e-06
1838 6.57695578087925e-06
1839 6.51135182661022e-06
1840 6.5072204518124e-06
1841 6.5804940625469e-06
1842 6.66791569070568e-06
1843 6.69088773719295e-06
1844 6.63781522369788e-06
1845 6.55620619305353e-06
1846 6.50889258801968e-06
1847 6.55634317059395e-06
1848 6.76226049556305e-06
1849 7.11204593528691e-06
1850 7.4893520478625e-06
1851 7.71278316946677e-06
1852 7.69590130289544e-06
1853 7.48684120210896e-06
1854 7.13446756073211e-06
1855 6.89619117810736e-06
1856 7.16852179749594e-06
1857 7.98268672141944e-06
1858 8.88451678271451e-06
1859 9.22690077231891e-06
1860 9.28531773070063e-06
1861 8.56488101064201e-06
1862 8.13059478055322e-06
1863 8.43261365091053e-06
1864 9.83536908449878e-06
1865 1.24924795166237e-05
1866 1.37590963760204e-05
1867 1.19403046383582e-05
1868 9.79594721428612e-06
1869 1.03091072922723e-05
1870 9.14476640190287e-06
1871 7.8118496160684e-06
1872 7.80792116472639e-06
1873 8.15277873655873e-06
1874 7.58725999361332e-06
1875 6.93169190180143e-06
1876 6.803553264767e-06
1877 6.54891295237547e-06
1878 6.46944504101299e-06
1879 6.59534298330655e-06
1880 6.63432743586156e-06
1881 6.51392005046603e-06
1882 6.36630668271465e-06
1883 6.30649771382608e-06
1884 6.27488226045969e-06
1885 6.27055462842918e-06
1886 6.30377343124877e-06
1887 6.3528442231231e-06
1888 6.37252269841682e-06
1889 6.33493834545362e-06
1890 6.28407048175461e-06
1891 6.25548853205871e-06
1892 6.23877668104456e-06
1893 6.23441097591382e-06
1894 6.2411887333802e-06
1895 6.26725782915116e-06
1896 6.3025978693787e-06
1897 6.32936773792281e-06
1898 6.32573724068664e-06
1899 6.29939678642444e-06
1900 6.26985263973978e-06
1901 6.25204665638535e-06
1902 6.24411973726424e-06
1903 6.23112212494448e-06
1904 6.22625853674563e-06
1905 6.236236808288e-06
1906 6.25438636747688e-06
1907 6.30463811912563e-06
1908 6.42748994916786e-06
1909 6.62372136250866e-06
1910 6.80038116085413e-06
1911 6.83343868068675e-06
1912 6.75501023295055e-06
1913 6.65133044464028e-06
1914 6.51345729618669e-06
1915 6.39562684900952e-06
1916 6.46671720083702e-06
1917 6.7272486905523e-06
1918 7.01207635689372e-06
1919 7.1391995764003e-06
1920 7.03015895453518e-06
1921 6.80967567223689e-06
1922 6.69333870018107e-06
1923 6.74639250242137e-06
1924 7.1906299936586e-06
1925 8.25716572094848e-06
1926 9.29196209931149e-06
1927 9.85872982539604e-06
1928 1.04596279868474e-05
1929 9.79761238539822e-06
1930 8.84166935488651e-06
1931 1.09527033487211e-05
1932 1.2173818028316e-05
1933 9.72776363283856e-06
1934 8.29952926739708e-06
1935 8.34078008793426e-06
1936 9.4879048582186e-06
1937 9.24556453441028e-06
1938 7.82001674970445e-06
1939 7.70709569808403e-06
1940 7.35672395535403e-06
1941 6.77544801952479e-06
1942 6.77544419591669e-06
1943 6.91930347507252e-06
1944 6.74038010295419e-06
1945 6.47212448479095e-06
1946 6.4069110368159e-06
1947 6.30820048375114e-06
1948 6.20440169551983e-06
1949 6.21253305688185e-06
1950 6.27569557698493e-06
1951 6.29455290734882e-06
1952 6.22423214657886e-06
1953 6.16392585506276e-06
1954 6.14970248946634e-06
1955 6.13192406540009e-06
1956 6.10518869059717e-06
1957 6.101640659395e-06
1958 6.12953323164689e-06
1959 6.16761786220721e-06
1960 6.17412234227288e-06
1961 6.14017549160906e-06
1962 6.10744398965579e-06
1963 6.0977125824202e-06
1964 6.10183590188385e-06
1965 6.09624086234106e-06
1966 6.08343007790779e-06
1967 6.0759555715606e-06
1968 6.08885303243767e-06
1969 6.12410384004392e-06
1970 6.16987357426879e-06
1971 6.19256398604051e-06
1972 6.17462429630677e-06
1973 6.13362932400108e-06
1974 6.10390363409152e-06
1975 6.10039863824241e-06
1976 6.13583415454144e-06
1977 6.19427998405087e-06
1978 6.23934526045034e-06
1979 6.24368795110186e-06
1980 6.20659755989905e-06
1981 6.14677666499475e-06
1982 6.11532488781208e-06
1983 6.13699899787434e-06
1984 6.27329133529031e-06
1985 6.56289385547737e-06
1986 6.93317685751538e-06
1987 7.29318463044137e-06
1988 7.45494005904845e-06
1989 7.29298141166623e-06
1990 6.94191295025348e-06
1991 6.62393573147568e-06
1992 6.76132896693105e-06
1993 7.68359069880376e-06
1994 9.1824248151795e-06
1995 1.02512757500506e-05
1996 1.02246128292904e-05
1997 9.76601659985477e-06
1998 8.87843481045536e-06
1999 9.14468189705531e-06
};
\addlegendentry{Train}
\addplot [semithick, black]
table {%
0 0.0987212285399437
1 0.0973140746355057
2 0.0958584174513817
3 0.0943244025111198
4 0.0926668345928192
5 0.0907467380166054
6 0.0881559327244759
7 0.0841977670788765
8 0.0785378366708755
9 0.0714866295456886
10 0.0639841631054878
11 0.0576856732368469
12 0.0529791489243507
13 0.0489454828202724
14 0.0453288033604622
15 0.0420628450810909
16 0.0390974767506123
17 0.036471288651228
18 0.0341250188648701
19 0.0320044867694378
20 0.0299808215349913
21 0.0280875116586685
22 0.0263065844774246
23 0.0245735216885805
24 0.0229357685893774
25 0.0213677175343037
26 0.019874295219779
27 0.0184203386306763
28 0.0169897582381964
29 0.015588216483593
30 0.014221828430891
31 0.0129071800038218
32 0.0116553846746683
33 0.0105036841705441
34 0.00945712719112635
35 0.00851171743124723
36 0.00767253385856748
37 0.00693238060921431
38 0.00626848498359323
39 0.00567516079172492
40 0.00514534115791321
41 0.00466687511652708
42 0.00423154840245843
43 0.0038199897389859
44 0.00343502801842988
45 0.00308691780082881
46 0.00277262507006526
47 0.00249660131521523
48 0.00224956916645169
49 0.00202898355200887
50 0.00183330208528787
51 0.00166190753225237
52 0.00151364679913968
53 0.00138367596082389
54 0.00126950617413968
55 0.00117078423500061
56 0.00108429510146379
57 0.00100925331935287
58 0.000936758879106492
59 0.00085051724454388
60 0.000780958158429712
61 0.000726918340660632
62 0.000681717763654888
63 0.000641040678601712
64 0.000607305846642703
65 0.000579446961637586
66 0.000554901373106986
67 0.000533279904630035
68 0.000515688676387072
69 0.000499777961522341
70 0.000486207893118262
71 0.000474016269436106
72 0.000463525037048385
73 0.000453873712103814
74 0.00044420690392144
75 0.000437209033407271
76 0.000430227926699445
77 0.000424464582465589
78 0.000418608513427898
79 0.000414170353906229
80 0.00040983033250086
81 0.000405590719310567
82 0.000401262048399076
83 0.000398427248001099
84 0.000395410199416801
85 0.000392122310586274
86 0.000389255699701607
87 0.000387505773687735
88 0.00038505787961185
89 0.000383467879146338
90 0.000381663499865681
91 0.000379751203581691
92 0.000378053839085624
93 0.000376500363927335
94 0.000375152769265696
95 0.000372930197045207
96 0.000371050904504955
97 0.000369959831004962
98 0.000368412758689374
99 0.00036766633274965
100 0.000366581720300019
101 0.000365581421647221
102 0.000364046194590628
103 0.000363388244295493
104 0.000362558261258528
105 0.000361334852641448
106 0.000360428413841873
107 0.000359953381121159
108 0.00035889251739718
109 0.000357916433131322
110 0.000356906821252778
111 0.00035603842115961
112 0.000355767348082736
113 0.000354838499333709
114 0.000353663723217323
115 0.000352419301634654
116 0.000352295523043722
117 0.000351263559423387
118 0.000350221729604527
119 0.000349267706042156
120 0.000348098983522505
121 0.00034726012381725
122 0.000346511980751529
123 0.000345735606970266
124 0.000344968313584104
125 0.000343848689226434
126 0.000342966697644442
127 0.000342035200446844
128 0.000341405044309795
129 0.000339970079949126
130 0.000339662074111402
131 0.000338524114340544
132 0.000337183417286724
133 0.000336367229465395
134 0.000335615972289816
135 0.000334694312186912
136 0.000333835429046303
137 0.000333205796778202
138 0.000331954972352833
139 0.000331242015818134
140 0.000330407958244905
141 0.000329491798765957
142 0.000328593683661893
143 0.000328038149746135
144 0.000326761102769524
145 0.000326149543980137
146 0.000325118860928342
147 0.000324499706039205
148 0.000323652086080983
149 0.000322724517900497
150 0.000321590836392716
151 0.000319906888762489
152 0.00031916203442961
153 0.000318180565955117
154 0.000317108148010448
155 0.000316293706418946
156 0.000315157260047272
157 0.000314341596094891
158 0.000313434982672334
159 0.000312120129819959
160 0.000311057985527441
161 0.000310620263917372
162 0.000309379189275205
163 0.000308467861032113
164 0.000307585200062022
165 0.000306489644572139
166 0.000305613793898374
167 0.000304772285744548
168 0.000303852575598285
169 0.000302698608720675
170 0.000301796098938212
171 0.000300751096801832
172 0.00030009335023351
173 0.000299537554383278
174 0.0002985060273204
175 0.000297603459330276
176 0.000296691723633558
177 0.000295418984023854
178 0.000294776196824387
179 0.000293605087790638
180 0.000292657816316932
181 0.000291693111648783
182 0.000290528638288379
183 0.000289730698568746
184 0.000288903480395675
185 0.000288111710688099
186 0.000286943482933566
187 0.00028636257047765
188 0.000285096029983833
189 0.000284437235677615
190 0.000283375644357875
191 0.000282137538306415
192 0.000280933425528929
193 0.0002801803057082
194 0.000278941821306944
195 0.000278012157650664
196 0.000277083629043773
197 0.000275790865998715
198 0.000274896912742406
199 0.000273860030574724
200 0.000272686273092404
201 0.000271687895292416
202 0.000270448246737942
203 0.000268843490630388
204 0.000267431605607271
205 0.000266089016804472
206 0.000264809321379289
207 0.000263305031694472
208 0.000262267451034859
209 0.000260816625086591
210 0.000259563967119902
211 0.000258416897850111
212 0.000257090287050232
213 0.000255684746662155
214 0.000254178274190053
215 0.000252710015047342
216 0.000250707584200427
217 0.00024912302615121
218 0.000247086514718831
219 0.000245557399466634
220 0.00024349927844014
221 0.00024166883667931
222 0.000239737171796151
223 0.00023771969426889
224 0.000235898609389551
225 0.000233876562560908
226 0.000231888901907951
227 0.000229941797442734
228 0.000228197372052819
229 0.000226232572458684
230 0.000224218296352774
231 0.000222308197407983
232 0.000220674744923599
233 0.000218786852201447
234 0.000216936823562719
235 0.00021513068350032
236 0.000213390681892633
237 0.000211609120015055
238 0.00020984276488889
239 0.000208033467060886
240 0.000206358643481508
241 0.000204510986804962
242 0.000202876515686512
243 0.000201109840418212
244 0.000199482048628852
245 0.00019781528681051
246 0.00019605690613389
247 0.000194374704733491
248 0.000192604828043841
249 0.000190702121471986
250 0.000188807054655626
251 0.000186816207133234
252 0.000184656048077159
253 0.000182588468305767
254 0.000180437389644794
255 0.000178187809069641
256 0.000176026442204602
257 0.000173760112375021
258 0.000171520063304342
259 0.000169300852576271
260 0.000167126694577746
261 0.000164967976161279
262 0.000162962664035149
263 0.000160941403009929
264 0.000158891358296387
265 0.00015695889305789
266 0.00015522088506259
267 0.000153599263285287
268 0.000151937842019834
269 0.000150131701957434
270 0.000148616076330654
271 0.000147081125760451
272 0.000145444835652597
273 0.000143875178764574
274 0.000142014047014527
275 0.000140178890433162
276 0.00013859459431842
277 0.000137324590468779
278 0.000136108690639958
279 0.000134799076477066
280 0.000133613968500867
281 0.000132613495225087
282 0.000131560111185536
283 0.000130473708850332
284 0.000129602907691151
285 0.000128602478071116
286 0.000127639577840455
287 0.000126365222968161
288 0.000123673627967946
289 0.000121368488180451
290 0.000119755648483988
291 0.000118421674415004
292 0.000117168718134053
293 0.00011606645421125
294 0.000115039176307619
295 0.000114064641820733
296 0.00011311376147205
297 0.000112209949293174
298 0.000111403023765888
299 0.000110557397420052
300 0.000109884807898197
301 0.000108841151813976
302 0.000108307074697223
303 0.000107277162896935
304 0.000106739127659239
305 0.000105834289570339
306 0.00010526244295761
307 0.000104414320958313
308 0.000103983998997137
309 0.000103108832263388
310 0.000102702266303822
311 0.000101781530247536
312 0.000101377459941432
313 0.000100452358310577
314 0.000100109136838
315 9.91418201010674e-05
316 9.89309191936627e-05
317 9.77995878201909e-05
318 9.78259704425e-05
319 9.6448500698898e-05
320 9.70783221418969e-05
321 9.52184782363474e-05
322 9.7118885605596e-05
323 9.49129171203822e-05
324 9.98253526631743e-05
325 9.65899089351296e-05
326 0.000105104780232068
327 9.6468640549574e-05
328 0.000101575227745343
329 9.18620935408399e-05
330 9.43613558774814e-05
331 9.0013985754922e-05
332 9.0891720901709e-05
333 8.91415547812358e-05
334 8.9178116468247e-05
335 8.82128297234885e-05
336 8.78766732057557e-05
337 8.7111140601337e-05
338 8.6889871454332e-05
339 8.62234956002794e-05
340 8.55056423461065e-05
341 8.48370327730663e-05
342 8.4281105955597e-05
343 8.35432656458579e-05
344 8.32731020636857e-05
345 8.23847149149515e-05
346 8.21682042442262e-05
347 8.11423888080753e-05
348 8.15709863672964e-05
349 8.00456764409319e-05
350 8.14411469036713e-05
351 7.95397136243992e-05
352 8.34232996567152e-05
353 8.16324100014754e-05
354 9.0745474153664e-05
355 8.71273878146894e-05
356 9.02289102668874e-05
357 7.78374524088576e-05
358 7.95536470832303e-05
359 7.50195249565877e-05
360 7.5627576734405e-05
361 7.43324126233347e-05
362 7.40981049602851e-05
363 7.34508867026307e-05
364 7.29153252905235e-05
365 7.24624114809558e-05
366 7.17662769602612e-05
367 7.1404043410439e-05
368 7.07285071257502e-05
369 7.03452969901264e-05
370 6.96815332048573e-05
371 6.93201436661184e-05
372 6.85987251927145e-05
373 6.81651654304005e-05
374 6.76127747283317e-05
375 6.71775633236393e-05
376 6.64876351947896e-05
377 6.62011443637311e-05
378 6.56934716971591e-05
379 6.53525348752737e-05
380 6.46426124149002e-05
381 6.46854896331206e-05
382 6.38088604318909e-05
383 6.52624003123492e-05
384 6.49441135465167e-05
385 7.01177850714885e-05
386 7.23239791113883e-05
387 8.48211493575945e-05
388 7.19313393346965e-05
389 6.97247160132974e-05
390 6.10122478974517e-05
391 6.0858808865305e-05
392 5.97203252254985e-05
393 5.93955446674954e-05
394 5.88596922170836e-05
395 5.82495740673039e-05
396 5.7722078054212e-05
397 5.71880445932038e-05
398 5.6927452533273e-05
399 5.63045250601135e-05
400 5.60165390197653e-05
401 5.5442207667511e-05
402 5.50289842067286e-05
403 5.45455804967787e-05
404 5.43559071957134e-05
405 5.36851403012406e-05
406 5.35930666956119e-05
407 5.28790478711016e-05
408 5.26943076692987e-05
409 5.20309367857408e-05
410 5.22230147907976e-05
411 5.13334016432054e-05
412 5.20849280292168e-05
413 5.13996928930283e-05
414 5.31142213731073e-05
415 5.2992985729361e-05
416 5.79986444790848e-05
417 5.81208041694481e-05
418 6.59504512441345e-05
419 5.80444720981177e-05
420 5.79355000809301e-05
421 4.90015554532874e-05
422 4.95555359520949e-05
423 4.73387044621632e-05
424 4.75417291454505e-05
425 4.6924647904234e-05
426 4.67622230644338e-05
427 4.64785371150356e-05
428 4.61961899418384e-05
429 4.59657057945151e-05
430 4.56676461908501e-05
431 4.54314722446725e-05
432 4.52018066425808e-05
433 4.49295694124885e-05
434 4.46934027422685e-05
435 4.44317884102929e-05
436 4.41249540017452e-05
437 4.39042451034766e-05
438 4.38191636931151e-05
439 4.34547109762207e-05
440 4.32829256169498e-05
441 4.30252475780435e-05
442 4.29607025580481e-05
443 4.27727354690433e-05
444 4.29800893471111e-05
445 4.29948813689407e-05
446 4.3612173612928e-05
447 4.50045154138934e-05
448 4.72413230454549e-05
449 5.15552856086288e-05
450 5.73361066926736e-05
451 5.60737280466128e-05
452 5.54308280698024e-05
453 4.40414405602496e-05
454 4.31646840297617e-05
455 4.06703038606793e-05
456 4.07498882850632e-05
457 4.02120313083287e-05
458 4.01129909732845e-05
459 4.00151766370982e-05
460 3.98727133870125e-05
461 3.95902316085994e-05
462 3.95333299820777e-05
463 3.92870279029012e-05
464 3.92719848605338e-05
465 3.9090959035093e-05
466 3.88068292522803e-05
467 3.87997715733945e-05
468 3.85671701224055e-05
469 3.8569083699258e-05
470 3.82742691726889e-05
471 3.82480029657017e-05
472 3.80397505068686e-05
473 3.80291021429002e-05
474 3.77597280021291e-05
475 3.78283839381766e-05
476 3.75640847778413e-05
477 3.76582829630934e-05
478 3.72700778825674e-05
479 3.77099931938574e-05
480 3.71079513570294e-05
481 3.80339261027984e-05
482 3.72695794794708e-05
483 3.98680494981818e-05
484 3.89978740713559e-05
485 4.53896136605181e-05
486 4.77279900223948e-05
487 5.94498742430005e-05
488 5.85142224736046e-05
489 4.707525295089e-05
490 4.0557628381066e-05
491 3.64807638106868e-05
492 3.61466445610859e-05
493 3.59872174158227e-05
494 3.59669211320579e-05
495 3.5768734960584e-05
496 3.56809141521808e-05
497 3.5647397453431e-05
498 3.56443124474026e-05
499 3.54095645889174e-05
500 3.53183386323508e-05
501 3.52669812855311e-05
502 3.52967654180247e-05
503 3.50376103597227e-05
504 3.50992777384818e-05
505 3.48828725691419e-05
506 3.48768953699619e-05
507 3.4725344448816e-05
508 3.48913854395505e-05
509 3.45283879141789e-05
510 3.46980996255297e-05
511 3.4380722354399e-05
512 3.49016190739349e-05
513 3.42450985044707e-05
514 3.50260052073281e-05
515 3.4156513720518e-05
516 3.5633744118968e-05
517 3.44195650541224e-05
518 3.77820542780682e-05
519 3.6063738662051e-05
520 4.1883293306455e-05
521 4.07011066272389e-05
522 4.92625513288658e-05
523 4.59625698567834e-05
524 4.49282160843723e-05
525 3.95207498513628e-05
526 3.56165146513376e-05
527 3.43762076226994e-05
528 3.36638149747159e-05
529 3.34027899953071e-05
530 3.32489289576188e-05
531 3.3176092983922e-05
532 3.32193812937476e-05
533 3.31016635755077e-05
534 3.30309994751588e-05
535 3.2963009289233e-05
536 3.29103968397249e-05
537 3.2819571060827e-05
538 3.27735979226418e-05
539 3.26439112541266e-05
540 3.26379486068618e-05
541 3.25130640703719e-05
542 3.25139008054975e-05
543 3.23138119711075e-05
544 3.24280372296926e-05
545 3.2135249057319e-05
546 3.23653912346344e-05
547 3.19272257911507e-05
548 3.24986795021687e-05
549 3.16787045449018e-05
550 3.26102090184577e-05
551 3.14304343191907e-05
552 3.3981392334681e-05
553 3.16897021548357e-05
554 3.6934347008355e-05
555 3.48547582689207e-05
556 4.59077418781817e-05
557 4.77126959594898e-05
558 5.52548226551153e-05
559 4.44694160250947e-05
560 3.41396662406623e-05
561 3.20699073199648e-05
562 3.12773045152426e-05
563 3.0880342819728e-05
564 3.07388872897718e-05
565 3.09236093016807e-05
566 3.08570961351506e-05
567 3.06804759020451e-05
568 3.06215806631371e-05
569 3.06416186504066e-05
570 3.05811554426327e-05
571 3.0469762350549e-05
572 3.04152072203578e-05
573 3.03759334201459e-05
574 3.02969456242863e-05
575 3.023389763257e-05
576 3.01710115309106e-05
577 3.01410127576673e-05
578 3.0032704671612e-05
579 2.99975563393673e-05
580 2.98935101454845e-05
581 2.98850591207156e-05
582 2.97533679258777e-05
583 2.9777746021864e-05
584 2.96317812171765e-05
585 2.96439320663922e-05
586 2.95020272460533e-05
587 2.9565058866865e-05
588 2.9310627724044e-05
589 2.94817000394687e-05
590 2.90310945274541e-05
591 2.9487096981029e-05
592 2.87377351924079e-05
593 2.99526054732269e-05
594 2.8417485737009e-05
595 3.09900460706558e-05
596 2.8834354452556e-05
597 3.59813348040916e-05
598 3.4801440051524e-05
599 5.57793973712251e-05
600 5.86846799706109e-05
601 4.8485799197806e-05
602 3.31530463881791e-05
603 2.85997120954562e-05
604 2.80254298559157e-05
605 2.85061232716544e-05
606 2.84742909570923e-05
607 2.80402109638089e-05
608 2.80539261439117e-05
609 2.82208766293479e-05
610 2.80923086393159e-05
611 2.79466912616044e-05
612 2.79270607279614e-05
613 2.79366904578637e-05
614 2.78523730230518e-05
615 2.7775462513091e-05
616 2.77594353974564e-05
617 2.77355193247786e-05
618 2.76465470960829e-05
619 2.76143837254494e-05
620 2.74786234513158e-05
621 2.7503749151947e-05
622 2.73397745331749e-05
623 2.73930309049319e-05
624 2.72386896540411e-05
625 2.73422047030181e-05
626 2.70381915470352e-05
627 2.72002034762409e-05
628 2.69146730715875e-05
629 2.72719735221472e-05
630 2.67074883595342e-05
631 2.72820943791885e-05
632 2.65380822384031e-05
633 2.76526207017014e-05
634 2.64037244050996e-05
635 2.85741080006119e-05
636 2.68752628471702e-05
637 3.1793606467545e-05
638 3.04154655168531e-05
639 4.01452452933881e-05
640 4.06366234528832e-05
641 4.44089273514692e-05
642 3.56180717062671e-05
643 2.90889456664445e-05
644 2.68711773969699e-05
645 2.64185655396432e-05
646 2.60462165897479e-05
647 2.5903125788318e-05
648 2.59401967923623e-05
649 2.59252428804757e-05
650 2.57868068729294e-05
651 2.56947187153855e-05
652 2.56742168858182e-05
653 2.56151433859486e-05
654 2.55352915701224e-05
655 2.54632232099539e-05
656 2.5406456188648e-05
657 2.53638045251137e-05
658 2.53041616815608e-05
659 2.51852216024417e-05
660 2.52055342571111e-05
661 2.50577468250412e-05
662 2.50154553214088e-05
663 2.49096792686032e-05
664 2.49360273301136e-05
665 2.47671305260155e-05
666 2.47684583882801e-05
667 2.45660321525065e-05
668 2.47151619987562e-05
669 2.4416276573902e-05
670 2.46897179749794e-05
671 2.41418147197692e-05
672 2.47284442593809e-05
673 2.39008295466192e-05
674 2.51671099249506e-05
675 2.37157601077342e-05
676 2.64241753029637e-05
677 2.46466861426597e-05
678 3.21418156090658e-05
679 3.26691915688571e-05
680 5.19805907970294e-05
681 5.0683203880908e-05
682 3.52101196767762e-05
683 2.64191075984854e-05
684 2.41999714489793e-05
685 2.34009821724612e-05
686 2.36044488701737e-05
687 2.38624434132362e-05
688 2.36089781537885e-05
689 2.34246235777391e-05
690 2.34857197938254e-05
691 2.34998351515969e-05
692 2.3372935174848e-05
693 2.33044083870482e-05
694 2.32698293984868e-05
695 2.32009078899864e-05
696 2.3097260054783e-05
697 2.29627748922212e-05
698 2.29311826842604e-05
699 2.28619592235191e-05
700 2.2801783416071e-05
701 2.2745169189875e-05
702 2.26659376494354e-05
703 2.26104275498074e-05
704 2.26068441406824e-05
705 2.24303894356126e-05
706 2.23818351514637e-05
707 2.22562157432549e-05
708 2.22193375520874e-05
709 2.2192245523911e-05
710 2.20990477828309e-05
711 2.20123420149321e-05
712 2.19417124753818e-05
713 2.18775094253942e-05
714 2.18118457269156e-05
715 2.17143369809492e-05
716 2.17232663999312e-05
717 2.16233838727931e-05
718 2.16119588003494e-05
719 2.1441041099024e-05
720 2.14782576222206e-05
721 2.13404982787324e-05
722 2.13762032217346e-05
723 2.10354464798002e-05
724 2.13184921449283e-05
725 2.07809407584136e-05
726 2.143489291484e-05
727 2.04561711143469e-05
728 2.19204339373391e-05
729 2.03023300855421e-05
730 2.38232023548335e-05
731 2.20623551285826e-05
732 3.73472976207267e-05
733 5.19388377142604e-05
734 7.23451157682575e-05
735 3.24697321048006e-05
736 2.14729534491198e-05
737 2.14887568290578e-05
738 2.14733590837568e-05
739 2.09269837796455e-05
740 2.00447593670106e-05
741 2.0724188289023e-05
742 2.0770967239514e-05
743 2.03028175747022e-05
744 2.03088329726597e-05
745 2.04709158424521e-05
746 2.03824165510014e-05
747 2.0215808035573e-05
748 2.02334940695437e-05
749 2.0142579160165e-05
750 1.99234309548046e-05
751 1.97253884834936e-05
752 1.96019227587385e-05
753 1.95579905266641e-05
754 1.94600543181878e-05
755 1.93625783140305e-05
756 1.93201994989067e-05
757 1.92534225789132e-05
758 1.92104071174981e-05
759 1.91448234545533e-05
760 1.90945866052061e-05
761 1.90295340871671e-05
762 1.89782404049765e-05
763 1.8915714463219e-05
764 1.88981921382947e-05
765 1.88129451998975e-05
766 1.87882142199669e-05
767 1.87135683518136e-05
768 1.86852921615355e-05
769 1.85845692612929e-05
770 1.86153756658314e-05
771 1.8478611309547e-05
772 1.85285025509074e-05
773 1.8324129996472e-05
774 1.8468284906703e-05
775 1.81536670424975e-05
776 1.85277767741354e-05
777 1.79344588104868e-05
778 1.87517507583834e-05
779 1.77231886482332e-05
780 1.95961765712127e-05
781 1.81171835720306e-05
782 2.30139648920158e-05
783 2.20511319639627e-05
784 3.88024200219661e-05
785 4.35871952504385e-05
786 4.38581591879483e-05
787 2.52478148468072e-05
788 1.83900210686261e-05
789 1.77125602931483e-05
790 1.81268114829436e-05
791 1.79098897206131e-05
792 1.74519791471539e-05
793 1.76984940480907e-05
794 1.78181035153102e-05
795 1.7624806787353e-05
796 1.75239620148204e-05
797 1.75800996657927e-05
798 1.75815348484321e-05
799 1.74909182533156e-05
800 1.74588058143854e-05
801 1.7468129954068e-05
802 1.74212273122976e-05
803 1.73708103829995e-05
804 1.73407224792754e-05
805 1.73388471012004e-05
806 1.73034968611319e-05
807 1.7251633835258e-05
808 1.72197997017065e-05
809 1.71975825651316e-05
810 1.71870487974957e-05
811 1.71474148373818e-05
812 1.70872935996158e-05
813 1.70738730957964e-05
814 1.70309813256608e-05
815 1.70287457876839e-05
816 1.69706854649121e-05
817 1.69797240232583e-05
818 1.68696496984921e-05
819 1.6927548131207e-05
820 1.67859052453423e-05
821 1.69290469784755e-05
822 1.66705103765707e-05
823 1.69726190506481e-05
824 1.65029432537267e-05
825 1.71707470144611e-05
826 1.63103923114249e-05
827 1.76446337718517e-05
828 1.63406566571211e-05
829 1.99396654352313e-05
830 1.81838768185116e-05
831 2.85928927041823e-05
832 3.39503349096049e-05
833 4.88911209686194e-05
834 3.30171751556918e-05
835 1.82923013198888e-05
836 1.75411187228747e-05
837 1.70690109371208e-05
838 1.640601476538e-05
839 1.6156316632987e-05
840 1.66527715919074e-05
841 1.65892797667766e-05
842 1.63199656526558e-05
843 1.63606910064118e-05
844 1.64718439918943e-05
845 1.64065386343282e-05
846 1.63254171638982e-05
847 1.63150452863192e-05
848 1.63294080266496e-05
849 1.62922169693047e-05
850 1.62449960043887e-05
851 1.6217974916799e-05
852 1.62078995344928e-05
853 1.62000615091529e-05
854 1.61700863827718e-05
855 1.61205516633345e-05
856 1.61076459335163e-05
857 1.6091575162136e-05
858 1.60835825226968e-05
859 1.60249401233159e-05
860 1.60235194925917e-05
861 1.59677092597121e-05
862 1.59901355800685e-05
863 1.59357605298283e-05
864 1.5962430552463e-05
865 1.5839888874325e-05
866 1.59284445544472e-05
867 1.57815884449519e-05
868 1.59490973601351e-05
869 1.56667811097577e-05
870 1.60222516569775e-05
871 1.55252073454903e-05
872 1.61828156706179e-05
873 1.53981472976739e-05
874 1.7000398656819e-05
875 1.54405588546069e-05
876 1.95156026165932e-05
877 1.8345675925957e-05
878 3.04965451505268e-05
879 3.69900881196372e-05
880 4.42896343884058e-05
881 2.4997281798278e-05
882 1.6409012459917e-05
883 1.57087852130644e-05
884 1.59614064614289e-05
885 1.56285859702621e-05
886 1.52487391460454e-05
887 1.55438610818237e-05
888 1.5636875104974e-05
889 1.54363588080741e-05
890 1.53619257616811e-05
891 1.54603330884129e-05
892 1.54608533193823e-05
893 1.53775818034774e-05
894 1.53280852828175e-05
895 1.53387081809342e-05
896 1.53216187754879e-05
897 1.52960892592091e-05
898 1.52729626279324e-05
899 1.52223110490013e-05
900 1.52123911902891e-05
901 1.51982858369593e-05
902 1.5183520190476e-05
903 1.51213880599244e-05
904 1.51327667481382e-05
905 1.50931564348866e-05
906 1.51260446727974e-05
907 1.50409487105208e-05
908 1.5103365512914e-05
909 1.49630932355649e-05
910 1.5088812688191e-05
911 1.48958142744959e-05
912 1.5124091078178e-05
913 1.48046565300319e-05
914 1.52180500663235e-05
915 1.46907177622779e-05
916 1.55339748744154e-05
917 1.46085649248562e-05
918 1.65026176546235e-05
919 1.49522793435608e-05
920 1.94473086594371e-05
921 1.87876976269763e-05
922 3.11442054226063e-05
923 3.2824907975737e-05
924 3.48253634001594e-05
925 2.17684773815563e-05
926 1.57617232616758e-05
927 1.4810746506555e-05
928 1.50088480950217e-05
929 1.47676528285956e-05
930 1.44643345265649e-05
931 1.45598751259968e-05
932 1.46435022543301e-05
933 1.45454496305319e-05
934 1.44787918543443e-05
935 1.44843243106152e-05
936 1.44770801853156e-05
937 1.44600762723712e-05
938 1.44260047818534e-05
939 1.44236682899646e-05
940 1.44084688145085e-05
941 1.43859206218622e-05
942 1.43574170579086e-05
943 1.43532060974394e-05
944 1.43368852150161e-05
945 1.43311108331545e-05
946 1.42794333441998e-05
947 1.42812468766351e-05
948 1.42412482091459e-05
949 1.42745266202837e-05
950 1.41743312269682e-05
951 1.42371291076415e-05
952 1.41231621455518e-05
953 1.42589788083569e-05
954 1.40477104650927e-05
955 1.43092975122272e-05
956 1.39350677272887e-05
957 1.4477514923783e-05
958 1.38314717332833e-05
959 1.4930757060938e-05
960 1.38099103423883e-05
961 1.65098845172906e-05
962 1.48068038470228e-05
963 2.286528069817e-05
964 2.32211205002386e-05
965 4.13065499742515e-05
966 3.37632809532806e-05
967 2.02926603378728e-05
968 1.58605416800128e-05
969 1.46501124618226e-05
970 1.38928735395893e-05
971 1.39436715471675e-05
972 1.43678753374843e-05
973 1.42012386277202e-05
974 1.3972355191072e-05
975 1.40179672598606e-05
976 1.41425962283392e-05
977 1.41075888677733e-05
978 1.40224046845105e-05
979 1.400311703037e-05
980 1.40255970109138e-05
981 1.40182755785645e-05
982 1.39851508720312e-05
983 1.3951492292108e-05
984 1.39478397613857e-05
985 1.39411658892641e-05
986 1.39205731102265e-05
987 1.39185167427058e-05
988 1.38680034069694e-05
989 1.38887671710108e-05
990 1.38538271130528e-05
991 1.38771829369944e-05
992 1.38060968311038e-05
993 1.38540144689614e-05
994 1.37785464175977e-05
995 1.38553959914134e-05
996 1.37144461405114e-05
997 1.38738632813329e-05
998 1.3644478713104e-05
999 1.39541161843226e-05
1000 1.35281470647897e-05
1001 1.41692798933946e-05
1002 1.34185493152472e-05
1003 1.45822386912187e-05
1004 1.34314723254647e-05
1005 1.62499654834392e-05
1006 1.44287914736196e-05
1007 2.25458043132676e-05
1008 2.29734869208187e-05
1009 4.0058479498839e-05
1010 3.11118601530325e-05
1011 1.98241305042757e-05
1012 1.5338175217039e-05
1013 1.44227660712204e-05
1014 1.35908676384133e-05
1015 1.36052140078391e-05
1016 1.39408748509595e-05
1017 1.38017139761359e-05
1018 1.3579166989075e-05
1019 1.35876325657591e-05
1020 1.36741991809686e-05
1021 1.36521839522175e-05
1022 1.35799464260344e-05
1023 1.35554892040091e-05
1024 1.35671525640646e-05
1025 1.35951859192573e-05
1026 1.35676064019208e-05
1027 1.35263144329656e-05
1028 1.35250375024043e-05
1029 1.35394648168585e-05
1030 1.34925394377206e-05
1031 1.34992860694183e-05
1032 1.34664678625995e-05
1033 1.35032414618763e-05
1034 1.34147167045739e-05
1035 1.35037862492027e-05
1036 1.33498533614329e-05
1037 1.35544396471232e-05
1038 1.32627383209183e-05
1039 1.36898188429768e-05
1040 1.31247334138607e-05
1041 1.40097645271453e-05
1042 1.30031939988839e-05
1043 1.50952064359444e-05
1044 1.3447689525492e-05
1045 1.85057688213419e-05
1046 1.67593480000505e-05
1047 2.68794465227984e-05
1048 2.72088291239925e-05
1049 2.88576484308578e-05
1050 1.90182981896214e-05
1051 1.52659085870255e-05
1052 1.36022636070265e-05
1053 1.3781973393634e-05
1054 1.34603496917407e-05
1055 1.3324967767403e-05
1056 1.32497116283048e-05
1057 1.33567418743041e-05
1058 1.32832674353267e-05
1059 1.32544573716586e-05
1060 1.31921169668203e-05
1061 1.32268587549333e-05
1062 1.31932893054909e-05
1063 1.32018203657935e-05
1064 1.31592496472877e-05
1065 1.3163929907023e-05
1066 1.3094751011522e-05
1067 1.30459065985633e-05
1068 1.29769905470312e-05
1069 1.30587632156676e-05
1070 1.29307418319513e-05
1071 1.30467587950989e-05
1072 1.28777801364777e-05
1073 1.31185524878674e-05
1074 1.27873599922168e-05
1075 1.3229119758762e-05
1076 1.2670602700382e-05
1077 1.35895215862547e-05
1078 1.26068771351129e-05
1079 1.46384172694525e-05
1080 1.31243286887184e-05
1081 1.79476737685036e-05
1082 1.65357250807574e-05
1083 2.81066495517734e-05
1084 2.78375646303175e-05
1085 2.91455617116299e-05
1086 1.90651935554342e-05
1087 1.42179751492222e-05
1088 1.30457356135594e-05
1089 1.31797069116146e-05
1090 1.30246680782875e-05
1091 1.27993107525981e-05
1092 1.2818005416193e-05
1093 1.29259460663889e-05
1094 1.28827423395705e-05
1095 1.2818383765989e-05
1096 1.2793772839359e-05
1097 1.28114834296866e-05
1098 1.28064812088269e-05
1099 1.27772682390059e-05
1100 1.27455441543134e-05
1101 1.27365819935221e-05
1102 1.27528182929382e-05
1103 1.27529228848289e-05
1104 1.27129014799721e-05
1105 1.27100256577251e-05
1106 1.26904005810502e-05
1107 1.2716579476546e-05
1108 1.26638396977796e-05
1109 1.27065850392682e-05
1110 1.26047843878041e-05
1111 1.26986396935536e-05
1112 1.25576043501496e-05
1113 1.27572266137577e-05
1114 1.24810494526173e-05
1115 1.28861602206598e-05
1116 1.23767122204299e-05
1117 1.31948427224415e-05
1118 1.23284653454903e-05
1119 1.44014265970327e-05
1120 1.2964716006536e-05
1121 1.82019703061087e-05
1122 1.74892429640749e-05
1123 2.98422037303681e-05
1124 3.0165450880304e-05
1125 2.54314454650739e-05
1126 1.66227782756323e-05
1127 1.35197797135334e-05
1128 1.26550266941194e-05
1129 1.27777948364383e-05
1130 1.27807315948303e-05
1131 1.26252562040463e-05
1132 1.25484884847538e-05
1133 1.25886735986569e-05
1134 1.25940614452702e-05
1135 1.25538044812856e-05
1136 1.25227907119552e-05
1137 1.25170645333128e-05
1138 1.25207852761378e-05
1139 1.24999487525201e-05
1140 1.24820644487045e-05
1141 1.24577618407784e-05
1142 1.24440357467392e-05
1143 1.24491043607122e-05
1144 1.2439108104445e-05
1145 1.24153875731281e-05
1146 1.24070702440804e-05
1147 1.23883228297927e-05
1148 1.23993286251789e-05
1149 1.23787258416996e-05
1150 1.23615436677937e-05
1151 1.23488189274212e-05
1152 1.23301151688793e-05
1153 1.23334666568553e-05
1154 1.23232348414604e-05
1155 1.23192739920341e-05
1156 1.2271281775611e-05
1157 1.23001091196784e-05
1158 1.22461387945805e-05
1159 1.2297533430683e-05
1160 1.22081692097709e-05
1161 1.23767713375855e-05
1162 1.21473476610845e-05
1163 1.24486741697183e-05
1164 1.20445392894908e-05
1165 1.26940840345924e-05
1166 1.19989499580697e-05
1167 1.33523462864105e-05
1168 1.22028231999138e-05
1169 1.57872800627956e-05
1170 1.44251653182437e-05
1171 3.05244284390938e-05
1172 4.0392427763436e-05
1173 4.19358111685142e-05
1174 1.8562983314041e-05
1175 1.3849117749487e-05
1176 1.22593446576502e-05
1177 1.31556489577633e-05
1178 1.29585487229633e-05
1179 1.2235487702128e-05
1180 1.22393439596635e-05
1181 1.25272272271104e-05
1182 1.23994450405007e-05
1183 1.22070450743195e-05
1184 1.22433420983725e-05
1185 1.23067857202841e-05
1186 1.22666251627379e-05
1187 1.22158944577677e-05
1188 1.22143892440363e-05
1189 1.22285455290694e-05
1190 1.22177698358428e-05
1191 1.2200883247715e-05
1192 1.21881121231127e-05
1193 1.21758357636281e-05
1194 1.21779512483045e-05
1195 1.21693119581323e-05
1196 1.21453495012247e-05
1197 1.21445082186256e-05
1198 1.21393823064864e-05
1199 1.21556022349978e-05
1200 1.21188459161203e-05
1201 1.2124476597819e-05
1202 1.20991053336184e-05
1203 1.21390685308143e-05
1204 1.20824497571448e-05
1205 1.21404373203404e-05
1206 1.20458780656918e-05
1207 1.21660323202377e-05
1208 1.20142103696708e-05
1209 1.2249641258677e-05
1210 1.19745136544225e-05
1211 1.24804018923896e-05
1212 1.20018639790942e-05
1213 1.30707021526177e-05
1214 1.22785322673735e-05
1215 1.49934194269008e-05
1216 1.40824668051209e-05
1217 2.14045703614829e-05
1218 2.24010218516923e-05
1219 3.10243922285736e-05
1220 2.30732293857727e-05
1221 1.49259994941531e-05
1222 1.28875981317833e-05
1223 1.25020205814508e-05
1224 1.22369065138628e-05
1225 1.20340419016429e-05
1226 1.21057137221214e-05
1227 1.21661150842556e-05
1228 1.21095563372364e-05
1229 1.20496333693154e-05
1230 1.20396625788999e-05
1231 1.2053888895025e-05
1232 1.2049442375428e-05
1233 1.20251734188059e-05
1234 1.19971309686662e-05
1235 1.19947799248621e-05
1236 1.19925871331361e-05
1237 1.19906590043684e-05
1238 1.19826254376676e-05
1239 1.19537207865505e-05
1240 1.19549204100622e-05
1241 1.19448450277559e-05
1242 1.1963075849053e-05
1243 1.19288542919094e-05
1244 1.19484375318279e-05
1245 1.18879488582024e-05
1246 1.19625383376842e-05
1247 1.18626712719561e-05
1248 1.19976366477204e-05
1249 1.1804950190708e-05
1250 1.21155590022681e-05
1251 1.17807903734501e-05
1252 1.24939479064778e-05
1253 1.187677935377e-05
1254 1.37427878144081e-05
1255 1.26959839690244e-05
1256 1.81664945557714e-05
1257 1.83744086825754e-05
1258 3.09810857288539e-05
1259 2.92912573058857e-05
1260 1.98973830265459e-05
1261 1.41707923830836e-05
1262 1.24923581097391e-05
1263 1.20276063171332e-05
1264 1.1959907169512e-05
1265 1.2081497516192e-05
1266 1.20489685286884e-05
1267 1.19699052447686e-05
1268 1.19443575385958e-05
1269 1.19513706522412e-05
1270 1.19461947178934e-05
1271 1.19272799565806e-05
1272 1.19124315460795e-05
1273 1.18972566269804e-05
1274 1.18930292956065e-05
1275 1.18904099508654e-05
1276 1.18811449283385e-05
1277 1.18555890367134e-05
1278 1.18525267680525e-05
1279 1.18496473078267e-05
1280 1.18346970339189e-05
1281 1.18327498057624e-05
1282 1.18210091386572e-05
1283 1.18180778372334e-05
1284 1.17931194836274e-05
1285 1.18010248115752e-05
1286 1.17835461423965e-05
1287 1.17933141154936e-05
1288 1.1743471986847e-05
1289 1.17925101221772e-05
1290 1.17255585792009e-05
1291 1.18028074211907e-05
1292 1.16821292976965e-05
1293 1.18506595754297e-05
1294 1.16239325507195e-05
1295 1.1960434676439e-05
1296 1.15799539344152e-05
1297 1.23932195492671e-05
1298 1.16806413643644e-05
1299 1.36932812893065e-05
1300 1.25836641018395e-05
1301 1.9223496565246e-05
1302 2.0470653907978e-05
1303 3.79416524083354e-05
1304 3.20078142976854e-05
1305 1.64486791618401e-05
1306 1.31078895719838e-05
1307 1.25092710732133e-05
1308 1.19080450531328e-05
1309 1.16616993182106e-05
1310 1.19547512440477e-05
1311 1.19863152576727e-05
1312 1.18158723125816e-05
1313 1.1764075679821e-05
1314 1.18124726213864e-05
1315 1.18201969598886e-05
1316 1.17810232040938e-05
1317 1.17575600597775e-05
1318 1.175282523036e-05
1319 1.17516219688696e-05
1320 1.17451063488261e-05
1321 1.17181252790033e-05
1322 1.17062336357776e-05
1323 1.17007330118213e-05
1324 1.16992123366799e-05
1325 1.16907649498899e-05
1326 1.16685996545129e-05
1327 1.16694700409425e-05
1328 1.16675910248887e-05
1329 1.16700657599722e-05
1330 1.16330438686418e-05
1331 1.16495593829313e-05
1332 1.16222136057331e-05
1333 1.16474702736014e-05
1334 1.15846314656665e-05
1335 1.16450291898218e-05
1336 1.15645407277043e-05
1337 1.16614901344292e-05
1338 1.15198045023135e-05
1339 1.17465933726635e-05
1340 1.14706708700396e-05
1341 1.19113774417201e-05
1342 1.14622407636489e-05
1343 1.24426815091283e-05
1344 1.17183581096469e-05
1345 1.44179339258699e-05
1346 1.33987359731691e-05
1347 2.17265060200589e-05
1348 2.38428638112964e-05
1349 3.32024937961251e-05
1350 2.41347879637033e-05
1351 1.40622132676071e-05
1352 1.22584251585067e-05
1353 1.21638140626601e-05
1354 1.18507487059105e-05
1355 1.15726097646984e-05
1356 1.16661794891115e-05
1357 1.17499157568091e-05
1358 1.16891305879108e-05
1359 1.16173050628277e-05
1360 1.16012242870056e-05
1361 1.16171122499509e-05
1362 1.16188475658419e-05
1363 1.15973889478482e-05
1364 1.15716084110318e-05
1365 1.15584380182554e-05
1366 1.15585016828845e-05
1367 1.15471893877839e-05
1368 1.15270777314436e-05
1369 1.15270959213376e-05
1370 1.15190232463647e-05
1371 1.14982613013126e-05
1372 1.14994836621918e-05
1373 1.14927825052291e-05
1374 1.14913837023778e-05
1375 1.14665072032949e-05
1376 1.14815347842523e-05
1377 1.14489048428368e-05
1378 1.14810482045868e-05
1379 1.14208414743189e-05
1380 1.14944759843638e-05
1381 1.13820105980267e-05
1382 1.15452749014366e-05
1383 1.13342857730458e-05
1384 1.16792034532409e-05
1385 1.13045962280012e-05
1386 1.20304866868537e-05
1387 1.1423997420934e-05
1388 1.30510561575647e-05
1389 1.21386019600322e-05
1390 1.7787095202948e-05
1391 1.8161286789109e-05
1392 3.19857099384535e-05
1393 3.03067608911078e-05
1394 1.93402247532504e-05
1395 1.35300188048859e-05
1396 1.20982740554609e-05
1397 1.16297051135916e-05
1398 1.15440634544939e-05
1399 1.1663497389236e-05
1400 1.16489136416931e-05
1401 1.15605898827198e-05
1402 1.15356442620396e-05
1403 1.15451657620724e-05
1404 1.15468465082813e-05
1405 1.1529531548149e-05
1406 1.1511821867316e-05
1407 1.14997455966659e-05
1408 1.14972854134976e-05
1409 1.14952053991146e-05
1410 1.14708345790859e-05
1411 1.14671693154378e-05
1412 1.14560643851291e-05
1413 1.14472622954054e-05
1414 1.14458098323666e-05
1415 1.1426318451413e-05
1416 1.14316253529978e-05
1417 1.14255935841356e-05
1418 1.14246704470133e-05
1419 1.13988953671651e-05
1420 1.14104368549306e-05
1421 1.13762498585857e-05
1422 1.13950100057991e-05
1423 1.13723781396402e-05
1424 1.14092335934401e-05
1425 1.1335160706949e-05
1426 1.14028089228668e-05
1427 1.12993211587309e-05
1428 1.14417725853855e-05
1429 1.12629868453951e-05
1430 1.15395350803738e-05
1431 1.12151401481242e-05
1432 1.17912286441424e-05
1433 1.1267268746451e-05
1434 1.25202623166842e-05
1435 1.16779865493299e-05
1436 1.63600307132583e-05
1437 1.61228635988664e-05
1438 3.16067526000552e-05
1439 3.48628345818724e-05
1440 2.30315836233785e-05
1441 1.39569156090147e-05
1442 1.20487620733911e-05
1443 1.14742488221964e-05
1444 1.15026568892063e-05
1445 1.16361834443524e-05
1446 1.15520670078695e-05
1447 1.14382828542148e-05
1448 1.14388139991206e-05
1449 1.14507183752721e-05
1450 1.14342929009581e-05
1451 1.14029826363549e-05
1452 1.13899504867732e-05
1453 1.1376819202269e-05
1454 1.1371076652722e-05
1455 1.13599435280776e-05
1456 1.13476144179003e-05
1457 1.13425594463479e-05
1458 1.13299429358449e-05
1459 1.13279429569957e-05
1460 1.13274709292455e-05
1461 1.13196174424957e-05
1462 1.13083497126354e-05
1463 1.13001451609307e-05
1464 1.12989646368078e-05
1465 1.12908255687216e-05
1466 1.12882225948852e-05
1467 1.12784900920815e-05
1468 1.12709276436362e-05
1469 1.12615534817451e-05
1470 1.12670586531749e-05
1471 1.1255368917773e-05
1472 1.12545276351739e-05
1473 1.12333827928524e-05
1474 1.12441048258916e-05
1475 1.12303996502305e-05
1476 1.12472207547398e-05
1477 1.11958552224678e-05
1478 1.12373245428898e-05
1479 1.11896497401176e-05
1480 1.12824782263488e-05
1481 1.11757644845056e-05
1482 1.13349706225563e-05
1483 1.11393001134275e-05
1484 1.14452968773548e-05
1485 1.11323306555278e-05
1486 1.17122781375656e-05
1487 1.11630752144265e-05
1488 1.25164269775269e-05
1489 1.16787387014483e-05
1490 1.64084885909688e-05
1491 1.71137635334162e-05
1492 3.53023533534724e-05
1493 3.94703383790329e-05
1494 1.98507477762178e-05
1495 1.3278051483212e-05
1496 1.25056112665334e-05
1497 1.15402635856299e-05
1498 1.12911548058037e-05
1499 1.16750688903267e-05
1500 1.16349528980209e-05
1501 1.14053736979258e-05
1502 1.1390405234124e-05
1503 1.14529711936484e-05
1504 1.14482982098707e-05
1505 1.14007698357455e-05
1506 1.13827363747987e-05
1507 1.13811302071554e-05
1508 1.1381070180505e-05
1509 1.13651321953512e-05
1510 1.13519163278397e-05
1511 1.1344489394105e-05
1512 1.13246314867865e-05
1513 1.13149608296226e-05
1514 1.1311803064018e-05
1515 1.12960251499317e-05
1516 1.12948118839995e-05
1517 1.12899670057232e-05
1518 1.12864227048703e-05
1519 1.12709249151521e-05
1520 1.12696216092445e-05
1521 1.12611005533836e-05
1522 1.12624957182561e-05
1523 1.12566340249032e-05
1524 1.12503648779239e-05
1525 1.12455227281316e-05
1526 1.12402613012819e-05
1527 1.12300867840531e-05
1528 1.1232691576879e-05
1529 1.12175666799885e-05
1530 1.12164143502014e-05
1531 1.1208468094992e-05
1532 1.12053121483768e-05
1533 1.11873523565009e-05
1534 1.11934486994869e-05
1535 1.11697145257494e-05
1536 1.11776553239906e-05
1537 1.11567505882704e-05
1538 1.11693198050489e-05
1539 1.11228746391134e-05
1540 1.11636663859827e-05
1541 1.1116118002974e-05
1542 1.1157959306729e-05
1543 1.10768614831613e-05
1544 1.11695026134839e-05
1545 1.10566388684674e-05
1546 1.12006928247865e-05
1547 1.10194305307232e-05
1548 1.13117521323147e-05
1549 1.10974269773578e-05
1550 1.14402446342865e-05
1551 1.11683439172339e-05
1552 1.20904542200151e-05
1553 1.15633101813728e-05
1554 1.39687144837808e-05
1555 1.37414754135534e-05
1556 2.93125267489813e-05
1557 4.42663658759557e-05
1558 3.98213305743411e-05
1559 1.53318596858298e-05
1560 1.41255077323876e-05
1561 1.09953352875891e-05
1562 1.19468204502482e-05
1563 1.21070088425768e-05
1564 1.14401482278481e-05
1565 1.11509280031896e-05
1566 1.14734184535337e-05
1567 1.14495433081174e-05
1568 1.12443221951253e-05
1569 1.1245143468841e-05
1570 1.13150344986934e-05
1571 1.12942852865672e-05
1572 1.12491425170447e-05
1573 1.12469697342021e-05
1574 1.12630032162997e-05
1575 1.1252224794589e-05
1576 1.12272500700783e-05
1577 1.12133693619398e-05
1578 1.12164807433146e-05
1579 1.12275138235418e-05
1580 1.12096658995142e-05
1581 1.12058069134946e-05
1582 1.11906292659114e-05
1583 1.1191042176506e-05
1584 1.11860972538125e-05
1585 1.11892140921555e-05
1586 1.1172759514011e-05
1587 1.11725103124627e-05
1588 1.11691369966138e-05
1589 1.11571353045292e-05
1590 1.11652025225339e-05
1591 1.11555664261687e-05
1592 1.11518365883967e-05
1593 1.11457266029902e-05
1594 1.11355220724363e-05
1595 1.11463959910907e-05
1596 1.11215113065555e-05
1597 1.11278313852381e-05
1598 1.11197487058234e-05
1599 1.111791425501e-05
1600 1.11151548480848e-05
1601 1.1114245353383e-05
1602 1.10859509732109e-05
1603 1.11071703940979e-05
1604 1.10706396299065e-05
1605 1.11159815787687e-05
1606 1.10479468276026e-05
1607 1.1131637620565e-05
1608 1.10093324110494e-05
1609 1.12065099528991e-05
1610 1.09769662230974e-05
1611 1.13895976028289e-05
1612 1.10012042568997e-05
1613 1.19389706014772e-05
1614 1.13988608063664e-05
1615 1.42856888487586e-05
1616 1.4516529518005e-05
1617 2.8842981919297e-05
1618 3.54858275386505e-05
1619 2.94349956675433e-05
1620 1.45905341923935e-05
1621 1.20064914881368e-05
1622 1.10387945824186e-05
1623 1.1385266589059e-05
1624 1.14541817310965e-05
1625 1.12396055556019e-05
1626 1.11212366391555e-05
1627 1.12008765427163e-05
1628 1.12210855149897e-05
1629 1.11749232019065e-05
1630 1.11310000647791e-05
1631 1.11406370706391e-05
1632 1.11475192170474e-05
1633 1.11332938104169e-05
1634 1.11055260276771e-05
1635 1.10986902654986e-05
1636 1.110724042519e-05
1637 1.11090939753922e-05
1638 1.10955925265444e-05
1639 1.10894188765087e-05
1640 1.10904256871436e-05
1641 1.10897581180325e-05
1642 1.10889413917903e-05
1643 1.10830333142076e-05
1644 1.10780474642524e-05
1645 1.10723030957161e-05
1646 1.10644996311748e-05
1647 1.1068744242948e-05
1648 1.1061257282563e-05
1649 1.10578848762088e-05
1650 1.10502778625232e-05
1651 1.10546698124381e-05
1652 1.10404216684401e-05
1653 1.1041922334698e-05
1654 1.10435230453731e-05
1655 1.103732665797e-05
1656 1.10329383460339e-05
1657 1.10264491013368e-05
1658 1.10329028757405e-05
1659 1.10166074591689e-05
1660 1.10240989670274e-05
1661 1.10167766251834e-05
1662 1.10109522211133e-05
1663 1.09981629066169e-05
1664 1.10016717371764e-05
1665 1.10007222247077e-05
1666 1.09858865471324e-05
1667 1.09863103716634e-05
1668 1.09751972559025e-05
1669 1.09783377411077e-05
1670 1.09719931060681e-05
1671 1.09727570816176e-05
1672 1.09556804090971e-05
1673 1.09496586446767e-05
1674 1.09578249976039e-05
1675 1.09281445475062e-05
1676 1.09406755655073e-05
1677 1.09290331238299e-05
1678 1.09182328742463e-05
1679 1.09104303191998e-05
1680 1.09200382212293e-05
1681 1.09199190774234e-05
1682 1.089905435947e-05
1683 1.09021948446753e-05
1684 1.09035581772332e-05
1685 1.08690819615731e-05
1686 1.08797739812871e-05
1687 1.08714530142606e-05
1688 1.08440899566631e-05
1689 1.08272397483233e-05
1690 1.08232297861832e-05
1691 1.08341200757422e-05
1692 1.08051626739325e-05
1693 1.07958121589036e-05
1694 1.08360827653087e-05
1695 1.08491358332685e-05
1696 1.08087342596264e-05
1697 1.08407302832347e-05
1698 1.08570839074673e-05
1699 1.08441008705995e-05
1700 1.0794493391586e-05
1701 1.07979203676223e-05
1702 1.08080566860735e-05
1703 1.07288942672312e-05
1704 1.06282213891973e-05
1705 1.06233383121435e-05
1706 1.06320376289659e-05
1707 1.04941436802619e-05
1708 1.03540205600439e-05
1709 1.03030615719035e-05
1710 1.0320620276616e-05
1711 1.02560206869384e-05
1712 1.00702536656172e-05
1713 9.98497762338957e-06
1714 1.00497854873538e-05
1715 1.02512776720687e-05
1716 1.03334796222043e-05
1717 1.01562136478606e-05
1718 9.91964770946652e-06
1719 1.06182187664672e-05
1720 1.14754047899623e-05
1721 1.04283890323131e-05
1722 9.40118661674205e-06
1723 1.09081756818341e-05
1724 9.80423374130623e-06
1725 9.62427202466642e-06
1726 1.41310238177539e-05
1727 1.46923630381934e-05
1728 1.31993592731305e-05
1729 1.56868027261226e-05
1730 1.48501394505729e-05
1731 1.3829426279699e-05
1732 1.52661941683618e-05
1733 1.42338985824608e-05
1734 1.3399675481196e-05
1735 1.19418391477666e-05
1736 1.06242232504883e-05
1737 1.04096152426791e-05
1738 1.1691407962644e-05
1739 1.4444794942392e-05
1740 1.39265548568801e-05
1741 1.27435650938423e-05
1742 1.29148302221438e-05
1743 1.30273474496789e-05
1744 1.273965335713e-05
1745 1.15247858047951e-05
1746 1.05223634818685e-05
1747 1.04830605778261e-05
1748 1.16288492790773e-05
1749 1.25072001537774e-05
1750 1.21618340926943e-05
1751 1.1985417586402e-05
1752 1.20014246931532e-05
1753 1.18689431474195e-05
1754 1.14802314783446e-05
1755 1.09105094452389e-05
1756 1.05121825981769e-05
1757 1.05536128103267e-05
1758 1.10468308776035e-05
1759 1.15467873911257e-05
1760 1.17218905870686e-05
1761 1.16880291898269e-05
1762 1.17845829663565e-05
1763 1.1838438695122e-05
1764 1.1697522495524e-05
1765 1.13609776235535e-05
1766 1.09375241663656e-05
1767 1.05799072116497e-05
1768 1.04085356724681e-05
1769 1.04634291346883e-05
1770 1.08039948827354e-05
1771 1.12423458631383e-05
1772 1.14735848910641e-05
1773 1.16296478154254e-05
1774 1.19465694297105e-05
1775 1.22705105241039e-05
1776 1.2419114682416e-05
1777 1.23624904517783e-05
1778 1.21356806630502e-05
1779 1.17483050416922e-05
1780 1.12568804979674e-05
1781 1.07473488242249e-05
1782 1.03790607681731e-05
1783 1.0154980373045e-05
1784 1.01624727903982e-05
1785 1.07357636807137e-05
1786 1.19052856462076e-05
1787 1.27214043459389e-05
1788 1.25835085782455e-05
1789 1.24839134514332e-05
1790 1.35047330331872e-05
1791 1.45519670695649e-05
1792 1.47323589771986e-05
1793 1.48101225931896e-05
1794 1.45543135658954e-05
1795 1.39082339956076e-05
1796 1.24375592349679e-05
1797 1.10703567770543e-05
1798 1.05275594250998e-05
1799 1.26494051073678e-05
1800 1.7361220670864e-05
1801 1.66222725965781e-05
1802 1.23114841699135e-05
1803 1.37523711600807e-05
1804 1.4122944776318e-05
1805 1.31087235786254e-05
1806 1.08606336652883e-05
1807 1.07708074210677e-05
1808 1.26258291857084e-05
1809 1.2699748367595e-05
1810 1.15941866170033e-05
1811 1.1480758075777e-05
1812 1.14624126581475e-05
1813 1.09738448372809e-05
1814 1.05369026641711e-05
1815 1.07130563264946e-05
1816 1.1093071407231e-05
1817 1.12209745566361e-05
1818 1.10952005343279e-05
1819 1.09796210381319e-05
1820 1.09672228063573e-05
1821 1.08328895294107e-05
1822 1.06225252238801e-05
1823 1.05451790659572e-05
1824 1.06713014247362e-05
1825 1.09255406641751e-05
1826 1.10845767267165e-05
1827 1.11179724626709e-05
1828 1.11337230919162e-05
1829 1.11543831735617e-05
1830 1.11211666080635e-05
1831 1.09955508378334e-05
1832 1.07726673377329e-05
1833 1.06046618384426e-05
1834 1.05403050838504e-05
1835 1.05572344182292e-05
1836 1.06959323602496e-05
1837 1.09504335341626e-05
1838 1.11214621938416e-05
1839 1.11787376226857e-05
1840 1.12735124275787e-05
1841 1.14854456114699e-05
1842 1.170427458419e-05
1843 1.17587514978368e-05
1844 1.15926141006639e-05
1845 1.12562456706655e-05
1846 1.08745562101831e-05
1847 1.05174449345213e-05
1848 1.03014053820516e-05
1849 1.02943195088301e-05
1850 1.04056889540516e-05
1851 1.07139321698924e-05
1852 1.14105441753054e-05
1853 1.21565608424135e-05
1854 1.22921555885114e-05
1855 1.20161867016577e-05
1856 1.2497543139034e-05
1857 1.39422845677473e-05
1858 1.52359962157789e-05
1859 1.55510660988512e-05
1860 1.59839455591282e-05
1861 1.55243196786614e-05
1862 1.46186075653532e-05
1863 1.26299137264141e-05
1864 1.05381677713012e-05
1865 1.28772944663069e-05
1866 1.98360194190172e-05
1867 1.90420832950622e-05
1868 1.2891829101136e-05
1869 1.3777130334347e-05
1870 1.41444043038064e-05
1871 1.1810760042863e-05
1872 1.03181127997232e-05
1873 1.22831788758049e-05
1874 1.28039819173864e-05
1875 1.14903996291105e-05
1876 1.13233254523948e-05
1877 1.11111075966619e-05
1878 1.06592624433688e-05
1879 1.06127017716062e-05
1880 1.1068726053054e-05
1881 1.11394610939897e-05
1882 1.09472648546216e-05
1883 1.08849699245184e-05
1884 1.08025160443503e-05
1885 1.06746774690691e-05
1886 1.05895487649832e-05
1887 1.06454799606581e-05
1888 1.07358009699965e-05
1889 1.07646883407142e-05
1890 1.07804889921681e-05
1891 1.07784508145414e-05
1892 1.07379419205245e-05
1893 1.06668785520014e-05
1894 1.0600147106743e-05
1895 1.05451808849466e-05
1896 1.05110848380718e-05
1897 1.05269300547661e-05
1898 1.06219404187868e-05
1899 1.0711313734646e-05
1900 1.0801141797856e-05
1901 1.08712056317017e-05
1902 1.08897165773669e-05
1903 1.08309359347913e-05
1904 1.07246487459633e-05
1905 1.06053539639106e-05
1906 1.04952641777345e-05
1907 1.03527409009985e-05
1908 1.02287040135707e-05
1909 1.01454825198743e-05
1910 1.01244113466237e-05
1911 1.02196217994788e-05
1912 1.06818524727714e-05
1913 1.11537383418181e-05
1914 1.12098568934016e-05
1915 1.11298813862959e-05
1916 1.14102613224532e-05
1917 1.20406657515559e-05
1918 1.26324503071373e-05
1919 1.29068994283443e-05
1920 1.273939415114e-05
1921 1.22754936455749e-05
1922 1.17976651381468e-05
1923 1.12326943053631e-05
1924 1.07353598650661e-05
1925 1.05879662442021e-05
1926 1.06202551251044e-05
1927 1.34094916575123e-05
1928 1.84028449439211e-05
1929 1.80233055289136e-05
1930 1.27910370792961e-05
1931 1.49025345308473e-05
1932 1.73661501321476e-05
1933 1.58017355715856e-05
1934 1.30722819449147e-05
1935 1.03347520052921e-05
1936 1.41813188747619e-05
1937 1.54156950884499e-05
1938 1.25909491544007e-05
1939 1.21923249025713e-05
1940 1.24388288895716e-05
1941 1.13025998871308e-05
1942 1.06566685644793e-05
1943 1.16121764222044e-05
1944 1.20309559861198e-05
1945 1.14215999929002e-05
1946 1.1257708138146e-05
1947 1.1190877557965e-05
1948 1.08986014311085e-05
1949 1.07055248008692e-05
1950 1.09222028186196e-05
1951 1.12014840851771e-05
1952 1.11707213363843e-05
1953 1.10601840788149e-05
1954 1.10349983515334e-05
1955 1.09756501842639e-05
1956 1.08522262962651e-05
1957 1.07387349999044e-05
1958 1.07570149339153e-05
1959 1.08857966552023e-05
1960 1.09802376755397e-05
1961 1.09887596408953e-05
1962 1.09876864371472e-05
1963 1.09859747681185e-05
1964 1.09820120997028e-05
1965 1.09525681182276e-05
1966 1.08652930066455e-05
1967 1.07636415123125e-05
1968 1.06900997707271e-05
1969 1.06816451079794e-05
1970 1.07561036202242e-05
1971 1.08225940493867e-05
1972 1.08783060568385e-05
1973 1.09567763502127e-05
1974 1.10330993265961e-05
1975 1.11217777885031e-05
1976 1.12324014480691e-05
1977 1.1353504305589e-05
1978 1.14637614387902e-05
1979 1.14765898615588e-05
1980 1.1324971637805e-05
1981 1.10633136500837e-05
1982 1.07714486148325e-05
1983 1.05037461253232e-05
1984 1.03182592283702e-05
1985 1.03636275525787e-05
1986 1.06069292087341e-05
1987 1.09941647679079e-05
1988 1.14905096779694e-05
1989 1.21595821838127e-05
1990 1.25783917610534e-05
1991 1.22721776278922e-05
1992 1.21918619697681e-05
1993 1.35987620524247e-05
1994 1.61150037456537e-05
1995 1.7576670870767e-05
1996 1.72019172168802e-05
1997 1.81623508979101e-05
1998 1.61417010531295e-05
1999 1.1820494364656e-05
};
\addlegendentry{Test}
\nextgroupplot[
title={ReLU/ReLU},
ymin=4.6128576662894e-06, ymax=0.001,
]
\addplot [semithick, black, dashed]
table {%
	0 0.0362998008495197
	1 0.0355594963766634
	2 0.0348150421632454
	3 0.0340617520269006
	4 0.0332899373024702
	5 0.0324815498897806
	6 0.0316139849601313
	7 0.0306654408341274
	8 0.0296021361718886
	9 0.0283599068643525
	10 0.0268141263513826
	11 0.0248598802136257
	12 0.0229002053383738
	13 0.021146732266061
	14 0.0195342632941902
	15 0.0180464221630245
	16 0.0166378724970855
	17 0.0153247232665308
	18 0.0141268924344331
	19 0.0129740504780784
	20 0.0118674863188062
	21 0.0109783637453802
	22 0.0102700615098001
	23 0.00967441037937533
	24 0.00918815002660267
	25 0.00879243180679623
	26 0.00846077395544853
	27 0.00816804661008064
	28 0.00787878516712226
	29 0.00756195456779096
	30 0.00723609740089159
	31 0.00694794296578038
	32 0.00670422585244523
	33 0.00648830391583033
	34 0.00628919010341633
	35 0.00609659792826278
	36 0.00590885798010277
	37 0.00572073883813573
	38 0.0055355231379508
	39 0.00534877635800513
	40 0.0051558871200541
	41 0.00496201717032818
	42 0.00475934761198005
	43 0.00455329856049502
	44 0.00433609771425836
	45 0.00411347783665406
	46 0.0038809848119854
	47 0.0036519208952086
	48 0.00343102996703237
	49 0.00321463205909822
	50 0.00299504497525049
	51 0.00279128163674613
	52 0.00260257470290526
	53 0.0024274835268443
	54 0.00226881622802466
	55 0.00212393311448977
	56 0.00199623185108067
	57 0.0018842367680918
	58 0.00178679733653553
	59 0.00170173024525866
	60 0.00162636646200554
	61 0.00155750945123145
	62 0.00149311354834936
	63 0.00143329186903429
	64 0.00138040101955994
	65 0.0013342997408472
	66 0.00129121179998037
	67 0.00125062094593886
	68 0.00121256712009199
	69 0.00117636543654953
	70 0.00114230384679104
	71 0.00110909133945825
	72 0.00107835882408835
	73 0.00104809398362704
	74 0.00101967645969125
	75 0.000992023453363799
	76 0.000965758570600883
	77 0.000940634849030175
	78 0.000916307015359052
	79 0.000893245631232276
	80 0.00087093546972028
	81 0.000849598403874552
	82 0.000829321430501295
	83 0.000810073725915572
	84 0.000791448392192251
	85 0.000773899900195829
	86 0.000757105659431545
	87 0.000741029017262917
	88 0.000725671829059138
	89 0.000711303381649486
	90 0.000697245206538355
	91 0.000684220089169685
	92 0.000671796171445749
	93 0.000659900608297903
	94 0.000648632838419871
	95 0.000637852691397711
	96 0.000627529693701945
	97 0.000617678695562063
	98 0.000608326464316633
	99 0.000599169688030088
	100 0.000590391022342374
	101 0.00058205696495861
	102 0.00057405766165175
	103 0.000566194835300848
	104 0.000558734680453199
	105 0.000551393568912317
	106 0.00054444187799163
	107 0.000537627419816999
	108 0.000531096236954909
	109 0.000524623268574942
	110 0.000518640453265107
	111 0.00051247786359454
	112 0.000506731294080964
	113 0.000500939893299801
	114 0.000495518196203193
	115 0.00049004754419002
	116 0.00048478146072739
	117 0.000479741795970767
	118 0.000474677335205342
	119 0.000469830826204998
	120 0.000465059909402044
	121 0.000460479910543654
	122 0.000455816858902836
	123 0.000451257981694653
	124 0.000446949654815398
	125 0.000442612639290019
	126 0.000438520408806653
	127 0.000434532200870308
	128 0.000430649143709161
	129 0.000426893926942284
	130 0.000423132148398508
	131 0.000419567892095074
	132 0.00041602711712585
	133 0.000412570264870737
	134 0.000409389645483316
	135 0.000405879167828971
	136 0.000402928342282394
	137 0.000399589204107542
	138 0.000396775056060505
	139 0.000393571792073999
	140 0.000390801257935891
	141 0.000387915846431497
	142 0.000385091792395542
	143 0.000382422573238728
	144 0.00037961469433867
	145 0.000377035582914687
	146 0.000374491151205802
	147 0.000371927720607346
	148 0.000369481430652741
	149 0.000367157559367115
	150 0.000364649183893562
	151 0.00036239391079107
	152 0.000360167236749476
	153 0.000357817463964238
	154 0.000355698034354646
	155 0.000353552469277929
	156 0.000351475211346042
	157 0.00034941335206895
	158 0.000347385715713244
	159 0.000345519028996932
	160 0.000343489039778433
	161 0.000341699583259469
	162 0.000339793698231006
	163 0.000337992042432234
	164 0.000336202316702838
	165 0.000334468807068333
	166 0.000332727991349202
	167 0.000331075425606286
	168 0.000329393135416467
	169 0.000327754271893355
	170 0.000326302111830046
	171 0.000324605462537875
	172 0.000323317603715623
	173 0.000321734894669135
	174 0.000320388524187365
	175 0.000318922063115679
	176 0.000317533820407334
	177 0.000316341315624413
	178 0.000314871564455643
	179 0.000313694755959659
	180 0.000312337479840608
	181 0.000311133821014664
	182 0.000309854501210793
	183 0.000308630267113585
	184 0.000307517805254065
	185 0.000306392779407361
	186 0.000305147044059595
	187 0.000304192017551941
	188 0.000303035191677736
	189 0.000301970584246192
	190 0.000300985747117011
	191 0.000299919492817935
	192 0.000298964028957016
	193 0.000298024199082647
	194 0.000297031693889949
	195 0.000296080572638857
	196 0.000295288482448086
	197 0.000294258790063395
	198 0.000293453017434331
	199 0.000292600988018421
	200 0.000291692784628594
	201 0.000290975667098792
	202 0.000290043944119134
	203 0.000289341229745332
	204 0.000288478301058603
	205 0.000287788790501509
	206 0.000287019667666755
	207 0.000286292647331265
	208 0.000285432570649391
	209 0.000284869544884714
	210 0.000284085637986209
	211 0.000283459357547144
	212 0.000282705637118852
	213 0.00028212115887527
	214 0.000281352716172023
	215 0.000280808352442818
	216 0.000280162158503572
	217 0.000279552378344761
	218 0.000278808681002829
	219 0.000278362731592097
	220 0.000277731115374991
	221 0.000277124073704726
	222 0.000276593185219554
	223 0.000275931695341569
	224 0.000275379022468769
	225 0.000274699238843823
	226 0.000274302323020947
	227 0.000273665389670441
	228 0.000273113182856832
	229 0.000272661494818749
	230 0.000272024244907243
	231 0.000271617746875563
	232 0.000271146246745957
	233 0.000270581364077316
	234 0.000270037430709635
	235 0.000269749846609102
	236 0.000268993056181444
	237 0.000268663312738227
	238 0.000268329051266392
	239 0.000267660283043369
	240 0.000267319074907846
	241 0.000266803827400963
	242 0.000266395868663949
	243 0.000265925244150367
	244 0.000265503558381397
	245 0.000265128874872289
	246 0.000264652076822358
	247 0.00026421713300806
	248 0.000263839127910614
	249 0.000263427112116688
	250 0.000262972730666888
	251 0.000262582018478952
	252 0.000262048983188379
	253 0.000261733410894749
	254 0.000261196891216287
	255 0.000260780015480577
	256 0.000260575757891957
	257 0.00025996883871926
	258 0.000259693866723865
	259 0.000259226924896439
	260 0.000258881037382253
	261 0.000258507801362384
	262 0.000257923942058369
	263 0.000257804708212461
	264 0.000257287039289622
	265 0.000256848335823179
	266 0.000256581989958704
	267 0.000256056219654965
	268 0.000255743454857793
	269 0.000255367421317487
	270 0.000254951937165515
	271 0.000254659254579792
	272 0.000254168363767349
	273 0.000253862550579242
	274 0.000253386797908206
	275 0.000253003891685921
	276 0.00025279075092044
	277 0.000252234342383417
	278 0.000251865593867251
	279 0.00025162348040908
	280 0.000251106760885023
	281 0.000250545522817447
	282 0.000250320055044995
	283 0.000249351996814084
	284 0.000248456880882486
	285 0.000248111967493969
	286 0.000247669808544515
	287 0.000246708519796357
	288 0.000246035368917319
	289 0.000244874245510118
	290 0.000243925267540135
	291 0.000243110288238313
	292 0.000242082068780292
	293 0.000241211893182935
	294 0.000240157663739637
	295 0.000239517446544824
	296 0.000238453758356627
	297 0.000237539680568943
	298 0.000236898615241898
	299 0.000235877096912418
	300 0.000235077533631056
	301 0.000234270741799492
	302 0.000233462161901343
	303 0.000232473063704219
	304 0.000231918292968203
	305 0.000231051498019497
	306 0.000230200978592165
	307 0.000229375356610717
	308 0.000228703457366919
	309 0.000227860746022657
	310 0.00022715896729153
	311 0.000226306831180523
	312 0.000225679638901966
	313 0.000224794907907722
	314 0.000224021654474882
	315 0.000223256796331839
	316 0.000222458069444542
	317 0.000221728834290502
	318 0.000220937578774283
	319 0.000220108085841275
	320 0.000219041030348421
	321 0.000218206421848777
	322 0.000217474454672129
	323 0.000216668670347531
	324 0.000215820557031066
	325 0.000215188896547147
	326 0.000214317522093666
	327 0.000213622222133836
	328 0.000212794480034972
	329 0.000212112290810751
	330 0.000211339138473932
	331 0.000210584539942715
	332 0.000209832874773497
	333 0.000209171945087405
	334 0.000208264496663446
	335 0.000207670122065906
	336 0.000206856141858225
	337 0.000206123753599741
	338 0.000205338090751184
	339 0.000204642500662544
	340 0.000203838011572088
	341 0.000203111640487919
	342 0.000202391897900611
	343 0.000201517466763335
	344 0.000200857110655761
	345 0.000200102569351657
	346 0.000199371504393753
	347 0.000198519569138966
	348 0.000197825998270673
	349 0.00019708576036237
	350 0.000196330573373871
	351 0.000195539736424166
	352 0.00019483115899277
	353 0.000194081779298472
	354 0.000193377950864715
	355 0.000192569054732417
	356 0.000191948709939993
	357 0.00019127974384503
	358 0.00019060137240956
	359 0.000189873212946168
	360 0.000189235741927973
	361 0.000188500023284632
	362 0.000187873870061139
	363 0.000187185748814045
	364 0.000186515121413322
	365 0.000185801850975054
	366 0.000185129035401133
	367 0.00018439969744577
	368 0.000183703804836455
	369 0.000182849752789593
	370 0.000182069031453125
	371 0.000181193938757929
	372 0.000180391757396592
	373 0.000179678111464909
	374 0.000178917328071293
	375 0.000178140454792697
	376 0.000177418040905764
	377 0.000176636954833498
	378 0.000175900917696481
	379 0.000175126874836451
	380 0.000174376139185028
	381 0.000173636472823091
	382 0.000172890659328573
	383 0.000172163555561156
	384 0.000171411987082593
	385 0.000170691904145315
	386 0.000169964335725581
	387 0.00016927295608582
	388 0.000168544080111133
	389 0.000167778984589972
	390 0.000167071638500715
	391 0.000166395974616762
	392 0.000165682060242034
	393 0.000164988830562152
	394 0.000164251934805293
	395 0.000163526583733642
	396 0.000162850205668974
	397 0.000162146568030153
	398 0.00016141736409736
	399 0.000160711990233153
	400 0.000160026631803589
	401 0.000159296759633776
	402 0.00015859027864451
	403 0.000157882610494653
	404 0.000157167184084983
	405 0.000156489993884179
	406 0.000155798396576756
	407 0.000155132412928083
	408 0.000154426175825506
	409 0.000153730435869193
	410 0.000153043807742392
	411 0.00015233491149047
	412 0.000151615612537626
	413 0.000150865624533481
	414 0.000150147255467914
	415 0.000149439299661935
	416 0.000148734846234788
	417 0.000148074161259615
	418 0.000147404630723713
	419 0.000146713281822031
	420 0.000146011413789893
	421 0.000145260939859782
	422 0.000144555933019319
	423 0.000143872297599046
	424 0.000143225705045325
	425 0.000142528297203626
	426 0.000141613421376974
	427 0.000140787225333838
	428 0.000140025194582449
	429 0.000139246214402533
	430 0.000138437588290685
	431 0.00013765462310289
	432 0.000136879761384989
	433 0.000136073359442435
	434 0.000135337460378082
	435 0.000134631991585366
	436 0.000133854467861738
	437 0.000133195473949854
	438 0.000132495780263753
	439 0.000131806855932837
	440 0.000131065532883667
	441 0.000130385848450487
	442 0.000129707111923949
	443 0.000128951424684942
	444 0.000128295457841432
	445 0.000127678271738318
	446 0.000126974247052658
	447 0.000126321120902162
	448 0.000125616314505805
	449 0.000124988573119822
	450 0.000124281587773112
	451 0.000123711664045345
	452 0.000122971606444366
	453 0.000122342695803468
	454 0.000121675284688649
	455 0.000121032234275731
	456 0.000120434980814821
	457 0.000119799573397472
	458 0.000119229451129854
	459 0.000118569599536045
	460 0.000118032375794996
	461 0.00011735239763766
	462 0.000116760227427903
	463 0.000116207209700292
	464 0.000115595368043842
	465 0.000115007930062916
	466 0.000114440321482334
	467 0.000113812056696361
	468 0.000113202917376043
	469 0.000112549161585207
	470 0.000111949991577376
	471 0.000111396500457772
	472 0.000110733068737545
	473 0.00011008391913947
	474 0.000109707533894721
	475 0.000108986873698314
	476 0.00010864520534426
	477 0.000107968088983057
	478 0.000107246772302005
	479 0.000106840051998347
	480 0.000106045044461212
	481 0.000105637941913983
	482 0.000104860556831454
	483 0.000104524006459883
	484 0.000103754517084553
	485 0.000103394742879459
	486 0.000102656733545814
	487 0.000102334399642245
	488 0.000101503021966209
	489 0.000101277002158895
	490 0.000100511357629784
	491 0.000100204239231516
	492 9.94595601042647e-05
	493 9.91365441791459e-05
	494 9.83949840218656e-05
	495 9.80704410977751e-05
	496 9.73204759162627e-05
	497 9.70193472227265e-05
	498 9.62944587428183e-05
	499 9.59716712003456e-05
	500 9.51823382138173e-05
	501 9.48651263286138e-05
	502 9.40909594220329e-05
	503 9.37796650930522e-05
	504 9.31623393967129e-05
	505 9.25169269834214e-05
	506 9.21753031946082e-05
	507 9.14223845995821e-05
	508 9.11130605913968e-05
	509 9.05097485883744e-05
	510 8.99485888226081e-05
	511 8.96139952146768e-05
	512 8.89627689844019e-05
	513 8.82976037956951e-05
	514 8.80108732133067e-05
	515 8.73211741065916e-05
	516 8.68143871599614e-05
	517 8.61206759168454e-05
	518 8.58639974552489e-05
	519 8.52666565265281e-05
	520 8.47443297544714e-05
	521 8.42425015150639e-05
	522 8.38537278156082e-05
	523 8.34056201028943e-05
	524 8.28232641794102e-05
	525 8.22001654228188e-05
	526 8.20961445526791e-05
	527 8.14053850035634e-05
	528 8.0945044459213e-05
	529 8.03464555332312e-05
	530 8.00340439042202e-05
	531 7.95069734067511e-05
	532 7.90548817803938e-05
	533 7.83614433714774e-05
	534 7.82078392376206e-05
	535 7.75814503128913e-05
	536 7.71779183992294e-05
	537 7.66309814821398e-05
	538 7.63550919629097e-05
	539 7.57672803644027e-05
	540 7.53324324875848e-05
	541 7.48806449024642e-05
	542 7.45768543879421e-05
	543 7.39692880245002e-05
	544 7.34581251009558e-05
	545 7.32268615593057e-05
	546 7.26950607372601e-05
	547 7.2306552837631e-05
	548 7.18926240779183e-05
	549 7.12401216986791e-05
	550 7.11040506473637e-05
	551 7.05920489849632e-05
	552 7.00619634272925e-05
	553 6.98843157351803e-05
	554 6.93480299531757e-05
	555 6.89301881067195e-05
	556 6.8551420056906e-05
	557 6.82707783568048e-05
	558 6.78473588209272e-05
	559 6.74360939001417e-05
	560 6.69745112986675e-05
	561 6.68180441891764e-05
	562 6.63592037710714e-05
	563 6.60058921511109e-05
	564 6.55262176962879e-05
	565 6.53905422751677e-05
	566 6.49648082315935e-05
	567 6.45828889531685e-05
	568 6.4211758598276e-05
	569 6.40401970315452e-05
	570 6.36086536474068e-05
	571 6.32187643248017e-05
	572 6.29192845593707e-05
	573 6.27000941264555e-05
	574 6.23190243516092e-05
	575 6.19810251265562e-05
	576 6.18273665438096e-05
	577 6.1429724844686e-05
	578 6.10677941281779e-05
	579 6.07365476241739e-05
	580 6.06628759243222e-05
	581 6.02365712580877e-05
	582 5.99480626632953e-05
	583 5.96746937233661e-05
	584 5.92619547816753e-05
	585 5.91964390537214e-05
	586 5.88675883719247e-05
	587 5.85256604495044e-05
	588 5.82117236973545e-05
	589 5.80866704211758e-05
	590 5.77712051565982e-05
	591 5.7401242663957e-05
	592 5.73086526074462e-05
	593 5.69629971352015e-05
	594 5.66744127610264e-05
	595 5.64388679720196e-05
	596 5.63115186338337e-05
	597 5.59832259483528e-05
	598 5.57239085097194e-05
	599 5.55761643994401e-05
	600 5.52506580007162e-05
	601 5.4977018649538e-05
	602 5.49259705735494e-05
	603 5.46469309341546e-05
	604 5.43738208804712e-05
	605 5.42032818060534e-05
	606 5.41104777269652e-05
	607 5.37899195904856e-05
	608 5.35788938478277e-05
	609 5.33788512981914e-05
	610 5.31108474035591e-05
	611 5.28188466617507e-05
	612 5.28007250579776e-05
	613 5.25375042030873e-05
	614 5.22463739329737e-05
	615 5.21817207186359e-05
	616 5.19564236469705e-05
	617 5.17028267807973e-05
	618 5.15279064643437e-05
	619 5.13922980971415e-05
	620 5.11431310883381e-05
	621 5.10245679379295e-05
	622 5.07657468205025e-05
	623 5.05659771832256e-05
	624 5.04689343117093e-05
	625 5.03116765315781e-05
	626 5.00654054036431e-05
	627 4.99466667918114e-05
	628 4.97733557764946e-05
	629 4.95453231010856e-05
	630 4.93225792581597e-05
	631 4.92921230801358e-05
	632 4.91486851430523e-05
	633 4.88745055449158e-05
	634 4.87464270122473e-05
	635 4.85773501850417e-05
	636 4.83277820322314e-05
	637 4.83459440374645e-05
	638 4.81695036995688e-05
	639 4.79123683945204e-05
	640 4.79079372723845e-05
	641 4.77323975829336e-05
	642 4.7494558799599e-05
	643 4.7396956738055e-05
	644 4.72744508641654e-05
	645 4.71205403016484e-05
	646 4.69334318111692e-05
	647 4.67690931742482e-05
	648 4.65905854838411e-05
	649 4.64726459412645e-05
	650 4.63275180564438e-05
	651 4.61864582064209e-05
	652 4.59533570946746e-05
	653 4.58458102343684e-05
	654 4.58385133441652e-05
	655 4.55803687060552e-05
	656 4.53877548807213e-05
	657 4.53289828570291e-05
	658 4.51361612903156e-05
	659 4.5016913297502e-05
	660 4.48068293152915e-05
	661 4.47114794042136e-05
	662 4.45855084976188e-05
	663 4.434320204183e-05
	664 4.42037484447155e-05
	665 4.41920857738864e-05
	666 4.40129823644497e-05
	667 4.38157554185636e-05
	668 4.38180487094542e-05
	669 4.36554166114433e-05
	670 4.34700610725258e-05
	671 4.33566648823103e-05
	672 4.32284514104708e-05
	673 4.31947998222881e-05
	674 4.29492958318178e-05
	675 4.281442117815e-05
	676 4.27304983006138e-05
	677 4.25821641840685e-05
	678 4.24564990879617e-05
	679 4.24084854770967e-05
	680 4.21743781302553e-05
	681 4.20118611543785e-05
	682 4.20246025143456e-05
	683 4.18178223497989e-05
	684 4.1699751719193e-05
	685 4.16173691064614e-05
	686 4.14777486739126e-05
	687 4.1259769119506e-05
	688 4.12774028006879e-05
	689 4.11750146369627e-05
	690 4.09933881044822e-05
	691 4.0860978096191e-05
	692 4.07656899241715e-05
	693 4.06403074109107e-05
	694 4.05609610041324e-05
	695 4.04217291247733e-05
	696 4.0262691534565e-05
	697 4.02234163416892e-05
	698 4.01227536883653e-05
	699 3.99695151855894e-05
	700 3.98945714294996e-05
	701 3.97533340859013e-05
	702 3.97377708907243e-05
	703 3.956212041345e-05
	704 3.94104593794964e-05
	705 3.93683406869627e-05
	706 3.92140251577189e-05
	707 3.90401766168225e-05
	708 3.90396915861402e-05
	709 3.89296300014053e-05
	710 3.87322351613761e-05
	711 3.86550960556065e-05
	712 3.86227078692514e-05
	713 3.84998430646988e-05
	714 3.83517194109118e-05
	715 3.82502669253881e-05
	716 3.82305168074026e-05
	717 3.80805621702507e-05
	718 3.79494042803685e-05
	719 3.78410640280435e-05
	720 3.77380975820074e-05
	721 3.77001690452516e-05
	722 3.75314759821777e-05
	723 3.73702576013102e-05
	724 3.72733115625579e-05
	725 3.72107439545744e-05
	726 3.70920024792554e-05
	727 3.68946228661571e-05
	728 3.68152907412878e-05
	729 3.67610980518407e-05
	730 3.66475628652552e-05
	731 3.64738038882706e-05
	732 3.63450549727418e-05
	733 3.6272140988558e-05
	734 3.6160379167427e-05
	735 3.60240753778385e-05
	736 3.59532322917744e-05
	737 3.58425654667371e-05
	738 3.5704837962669e-05
	739 3.55532745501819e-05
	740 3.55047534839059e-05
	741 3.54611640887015e-05
	742 3.52895300324008e-05
	743 3.51118790788973e-05
	744 3.50708558514157e-05
	745 3.49948883311413e-05
	746 3.48368472273819e-05
	747 3.4721806429161e-05
	748 3.46566176148144e-05
	749 3.45588537342678e-05
	750 3.44931836622209e-05
	751 3.43860178162458e-05
	752 3.42577831276003e-05
	753 3.41983335729878e-05
	754 3.40633114106481e-05
	755 3.39985817916499e-05
	756 3.39334042180894e-05
	757 3.38417712484329e-05
	758 3.36835930170309e-05
	759 3.36504476052824e-05
	760 3.3532204916753e-05
	761 3.35048824808837e-05
	762 3.33641117862271e-05
	763 3.3342815051185e-05
	764 3.32056965888228e-05
	765 3.3136142654655e-05
	766 3.30305248397167e-05
	767 3.29396321774311e-05
	768 3.28765050454649e-05
	769 3.28288355007089e-05
	770 3.2678121087315e-05
	771 3.25615988003847e-05
	772 3.25555768441177e-05
	773 3.25049557829971e-05
	774 3.23509808737299e-05
	775 3.22760829263302e-05
	776 3.22041087770231e-05
	777 3.21332412553943e-05
	778 3.20790370480495e-05
	779 3.19487588384959e-05
	780 3.18611168808047e-05
	781 3.18456304952974e-05
	782 3.17438110712942e-05
	783 3.16947840772741e-05
	784 3.16074703476943e-05
	785 3.14965190284511e-05
	786 3.14388997040282e-05
	787 3.1393148287151e-05
	788 3.13718092819215e-05
	789 3.12176037446932e-05
	790 3.12156189750112e-05
	791 3.11636992087472e-05
	792 3.09791517771885e-05
	793 3.09937957361228e-05
	794 3.09617763427639e-05
	795 3.08239544537514e-05
	796 3.07583675756007e-05
	797 3.06969656058698e-05
	798 3.06174425794836e-05
	799 3.05322052973622e-05
	800 3.05434125777992e-05
	801 3.03930397045349e-05
	802 3.0416096137742e-05
	803 3.03206056280203e-05
	804 3.02272253094316e-05
	805 3.01128138566753e-05
	806 3.01147749901531e-05
	807 3.00220973805665e-05
	808 2.99459551946768e-05
	809 2.99120246118889e-05
	810 2.9845228525005e-05
	811 2.97918040637057e-05
	812 2.96315949412929e-05
	813 2.96391174003929e-05
	814 2.95647160655221e-05
	815 2.94583449438335e-05
	816 2.93857633231198e-05
	817 2.93308808210213e-05
	818 2.92594721926775e-05
	819 2.9170187637817e-05
	820 2.9143089598449e-05
	821 2.89989355337639e-05
	822 2.89384909422097e-05
	823 2.89211914363818e-05
	824 2.89144072311842e-05
	825 2.87705050610043e-05
	826 2.86370426394456e-05
	827 2.8625126798687e-05
	828 2.85427627524371e-05
	829 2.84170055984134e-05
	830 2.83650633150501e-05
	831 2.82896655718901e-05
	832 2.81911652706412e-05
	833 2.81347444399671e-05
	834 2.80629652458231e-05
	835 2.79526961293186e-05
	836 2.78052504789628e-05
	837 2.78105944282458e-05
	838 2.77452436776571e-05
	839 2.75994861418383e-05
	840 2.74679731475658e-05
	841 2.74301486662409e-05
	842 2.73470232237116e-05
	843 2.72693887630737e-05
	844 2.71535525406819e-05
	845 2.70226213636704e-05
	846 2.69077817023344e-05
	847 2.69397562604468e-05
	848 2.67731453362785e-05
	849 2.66556710428745e-05
	850 2.66548161071967e-05
	851 2.6521012770786e-05
	852 2.63985140342982e-05
	853 2.63110077689532e-05
	854 2.6234632663602e-05
	855 2.61577829512305e-05
	856 2.60004034906558e-05
	857 2.59830337228095e-05
	858 2.58757151669897e-05
	859 2.58107316319922e-05
	860 2.56689344126926e-05
	861 2.56175256581059e-05
	862 2.55613029622737e-05
	863 2.54670272070712e-05
	864 2.53016378550797e-05
	865 2.53339849045631e-05
	866 2.52546713213064e-05
	867 2.50842133375784e-05
	868 2.50096405949307e-05
	869 2.50103238883526e-05
	870 2.48616743796504e-05
	871 2.48075455857588e-05
	872 2.47583757051473e-05
	873 2.46293703014544e-05
	874 2.46083344848103e-05
	875 2.45324306114014e-05
	876 2.4402606388918e-05
	877 2.43889602309366e-05
	878 2.43431025275243e-05
	879 2.42027662444855e-05
	880 2.41701094232383e-05
	881 2.40781202194285e-05
	882 2.40366926185231e-05
	883 2.40090417413796e-05
	884 2.38752615970839e-05
	885 2.38550041409269e-05
	886 2.37604718940077e-05
	887 2.37258859563383e-05
	888 2.36086361162791e-05
	889 2.36726436995127e-05
	890 2.34688065496869e-05
	891 2.35085337365604e-05
	892 2.34483966927712e-05
	893 2.33013666850468e-05
	894 2.33673171088356e-05
	895 2.31893096653835e-05
	896 2.31672948096673e-05
	897 2.31777901689156e-05
	898 2.30367925091457e-05
	899 2.298937353018e-05
	900 2.29650578553731e-05
	901 2.290344901823e-05
	902 2.28412733420669e-05
	903 2.28031322109246e-05
	904 2.27010512325876e-05
	905 2.27541877215742e-05
	906 2.26031791186188e-05
	907 2.25320103979243e-05
	908 2.25569552902982e-05
	909 2.24851928969372e-05
	910 2.2392003483418e-05
	911 2.24315581185408e-05
	912 2.22436571739593e-05
	913 2.23037703968032e-05
	914 2.22042077062667e-05
	915 2.22149565267671e-05
	916 2.20278366853677e-05
	917 2.20980512679603e-05
	918 2.19291288097168e-05
	919 2.20344685857299e-05
	920 2.18720976512543e-05
	921 2.18260611468679e-05
	922 2.17984030292939e-05
	923 2.17695763033987e-05
	924 2.17176197772062e-05
	925 2.16637844943079e-05
	926 2.15510100893823e-05
	927 2.1637097560756e-05
	928 2.14986506108517e-05
	929 2.14356168406482e-05
	930 2.14832781537666e-05
	931 2.13703416385158e-05
	932 2.12828277845745e-05
	933 2.1428451972838e-05
	934 2.11579506661508e-05
	935 2.12965959676126e-05
	936 2.11670797440888e-05
	937 2.10916663831995e-05
	938 2.1205658139678e-05
	939 2.10527160291463e-05
	940 2.10648182914497e-05
	941 2.08910227286196e-05
	942 2.10740372565965e-05
	943 2.08938451109475e-05
	944 2.09533019877028e-05
	945 2.0814381858969e-05
	946 2.08445423943715e-05
	947 2.07439882800031e-05
	948 2.08214596923995e-05
	949 2.07156544362874e-05
	950 2.06757206768771e-05
	951 2.07442881396958e-05
	952 2.05077005368537e-05
	953 2.07169994048684e-05
	954 2.0619315467485e-05
	955 2.05246739248821e-05
	956 2.05518795475257e-05
	957 2.0485919556279e-05
	958 2.04856901611095e-05
	959 2.05044797532139e-05
	960 2.03967674821115e-05
	961 2.04296432890771e-05
	962 2.03443856960916e-05
	963 2.0483536324889e-05
	964 2.02557530144531e-05
	965 2.04224137938525e-05
	966 2.02278778118625e-05
	967 2.04258126821344e-05
	968 2.00737116529126e-05
	969 2.04881508274468e-05
	970 2.02178726311786e-05
	971 2.02622482774473e-05
	972 2.02521859478111e-05
	973 2.01418401495346e-05
	974 2.04147715194836e-05
	975 2.01209081254206e-05
	976 2.02324635409923e-05
	977 2.03404477403524e-05
	978 2.00519084465611e-05
	979 2.04033332202869e-05
	980 2.02696072584274e-05
	981 2.01541711568609e-05
	982 2.03981596058611e-05
	983 2.02719078465918e-05
	984 2.03939398275566e-05
	985 2.02475620518783e-05
	986 2.03877719329171e-05
	987 2.04846115572366e-05
	988 2.03870968853437e-05
	989 2.05470529621721e-05
	990 2.04960205003601e-05
	991 2.06022066926437e-05
	992 2.06419847224026e-05
	993 2.06494524022105e-05
	994 2.06692437707545e-05
	995 2.06574152628036e-05
	996 2.08012937257251e-05
	997 2.0861102875358e-05
	998 2.08042492424454e-05
	999 2.08189833124095e-05
	1000 2.08881541752248e-05
	1001 2.08796084777418e-05
	1002 2.09125846737379e-05
	1003 2.08592326274015e-05
	1004 2.08352362562891e-05
	1005 2.0789538751842e-05
	1006 2.07334684887428e-05
	1007 2.07413431496661e-05
	1008 2.06144292960175e-05
	1009 2.0453957922939e-05
	1010 2.04466933197978e-05
	1011 2.02330288203711e-05
	1012 2.00264984826504e-05
	1013 1.99320888754073e-05
	1014 1.978703516059e-05
	1015 1.97056486932468e-05
	1016 1.96074964016191e-05
	1017 1.9488753935093e-05
	1018 1.94503773265353e-05
	1019 1.94331949288085e-05
	1020 1.93962338492781e-05
	1021 1.94152867543096e-05
	1022 1.95558681959085e-05
	1023 1.96231510827261e-05
	1024 1.96526725133594e-05
	1025 1.99205945534686e-05
	1026 2.00540819221828e-05
	1027 2.03759649153312e-05
	1028 2.06955063717373e-05
	1029 2.11226677109266e-05
	1030 2.16598913382882e-05
	1031 2.23363956024514e-05
	1032 2.32037229679349e-05
	1033 2.43434496738359e-05
	1034 2.59179434607404e-05
	1035 2.82249234828669e-05
	1036 3.0805123653721e-05
	1037 3.35010250260837e-05
	1038 3.55309605097887e-05
	1039 3.67389307598387e-05
	1040 3.65354539049179e-05
	1041 3.55639688578435e-05
	1042 3.18967074655063e-05
	1043 2.81888095940985e-05
	1044 2.48707114245406e-05
	1045 2.27942880961507e-05
	1046 2.09793521754875e-05
	1047 1.99838882082304e-05
	1048 1.88880502953737e-05
	1049 1.80400076352782e-05
	1050 1.73227901179018e-05
	1051 1.68777720901403e-05
	1052 1.65043542175169e-05
	1053 1.6292401635809e-05
	1054 1.61443662207716e-05
	1055 1.60657370145145e-05
	1056 1.60423271751142e-05
	1057 1.60421633417229e-05
	1058 1.61016520294766e-05
	1059 1.62152940426097e-05
	1060 1.63360979676952e-05
	1061 1.64755284011164e-05
	1062 1.66042319520443e-05
	1063 1.67109275857058e-05
	1064 1.68077767845887e-05
	1065 1.68986750317401e-05
	1066 1.69912244594173e-05
	1067 1.71448731194346e-05
	1068 1.71875627898999e-05
	1069 1.73494998776391e-05
	1070 1.73327668697709e-05
	1071 1.74829060171078e-05
	1072 1.73928942288626e-05
	1073 1.75273734051018e-05
	1074 1.73969557728526e-05
	1075 1.76547612866784e-05
	1076 1.74302439681639e-05
	1077 1.74052082755338e-05
	1078 1.74567691750838e-05
	1079 1.72835022240747e-05
	1080 1.72636735484843e-05
	1081 1.72661280757325e-05
	1082 1.70850536989064e-05
	1083 1.70428695653868e-05
	1084 1.6970822464657e-05
	1085 1.68139938203637e-05
	1086 1.68228563950379e-05
	1087 1.66732868542852e-05
	1088 1.64919545255771e-05
	1089 1.65605617112874e-05
	1090 1.63597066489274e-05
	1091 1.62840696624755e-05
	1092 1.62011193456379e-05
	1093 1.61597925192325e-05
	1094 1.59976401228334e-05
	1095 1.6024680066451e-05
	1096 1.58756696961859e-05
	1097 1.58527106375317e-05
	1098 1.57772665563272e-05
	1099 1.57401946676572e-05
	1100 1.5662676183581e-05
	1101 1.5644300018991e-05
	1102 1.55528709928632e-05
	1103 1.5586712702742e-05
	1104 1.54691374376625e-05
	1105 1.55032595010596e-05
	1106 1.54091578323801e-05
	1107 1.54186222065533e-05
	1108 1.53306399752751e-05
	1109 1.54127074498689e-05
	1110 1.52940790023592e-05
	1111 1.53040648314118e-05
	1112 1.52452429826866e-05
	1113 1.52854140917924e-05
	1114 1.52548972938149e-05
	1115 1.52217243947206e-05
	1116 1.52056977178461e-05
	1117 1.52381980047522e-05
	1118 1.51388850913037e-05
	1119 1.51817648017527e-05
	1120 1.51743210743405e-05
	1121 1.51169624373892e-05
	1122 1.51426301400193e-05
	1123 1.51498927607463e-05
	1124 1.51020999297202e-05
	1125 1.50803921918907e-05
	1126 1.51166843771477e-05
	1127 1.50759406700729e-05
	1128 1.50538168242065e-05
	1129 1.5111215759589e-05
	1130 1.50089676758114e-05
	1131 1.51096179585863e-05
	1132 1.4975500530312e-05
	1133 1.50879681335425e-05
	1134 1.49800968642921e-05
	1135 1.50246464869497e-05
	1136 1.49752604539088e-05
	1137 1.49846122603492e-05
	1138 1.50689078139976e-05
	1139 1.48584976642496e-05
	1140 1.51929217437896e-05
	1141 1.48393292818128e-05
	1142 1.52574523113458e-05
	1143 1.48671214468266e-05
	1144 1.51200759432868e-05
	1145 1.48196776681431e-05
	1146 1.5086920061691e-05
	1147 1.48091786247306e-05
	1148 1.50433192906974e-05
	1149 1.48032038982393e-05
	1150 1.50720396554505e-05
	1151 1.47289965966024e-05
	1152 1.51200796523199e-05
	1153 1.48096872738535e-05
	1154 1.49251597818534e-05
	1155 1.47392886979958e-05
	1156 1.48685198428922e-05
	1157 1.47292296617252e-05
	1158 1.48465037987933e-05
	1159 1.4636882934127e-05
	1160 1.49495977765923e-05
	1161 1.46501441307123e-05
	1162 1.48457468860386e-05
	1163 1.46258210627082e-05
	1164 1.51224115914061e-05
	1165 1.45721227617912e-05
	1166 1.53457809126678e-05
	1167 1.46367154734151e-05
	1168 1.56558829580433e-05
	1169 1.47336026792289e-05
	1170 1.58919068589114e-05
	1171 1.48601410714377e-05
	1172 1.62038779834006e-05
	1173 1.50454983121051e-05
	1174 1.5924686007196e-05
	1175 1.47889204669127e-05
	1176 1.5642028255769e-05
	1177 1.4697314735912e-05
	1178 1.55068014642268e-05
	1179 1.4633114904683e-05
	1180 1.54564877234975e-05
	1181 1.46223665460354e-05
	1182 1.55926884204405e-05
	1183 1.47624267476942e-05
	1184 1.56326149145514e-05
	1185 1.47386205995303e-05
	1186 1.55747705932185e-05
	1187 1.48222844948975e-05
	1188 1.57176838051498e-05
	1189 1.48996963389436e-05
	1190 1.58864956709692e-05
	1191 1.51814032030018e-05
	1192 1.6060500717785e-05
	1193 1.537749725955e-05
	1194 1.6086209708277e-05
	1195 1.54508393599428e-05
	1196 1.63691372261354e-05
	1197 1.60511923343876e-05
	1198 1.66907065661803e-05
	1199 1.63315982995016e-05
	1200 1.72013851003072e-05
	1201 1.69768146385252e-05
	1202 1.78616368131657e-05
	1203 1.7697618289958e-05
	1204 1.91063793764101e-05
	1205 1.94013043675056e-05
	1206 2.1215213282133e-05
	1207 2.18334644301876e-05
	1208 2.41901548303503e-05
	1209 2.5693598098897e-05
	1210 2.84909558185831e-05
	1211 3.13660925748138e-05
	1212 3.38282101957077e-05
	1213 3.4986460850206e-05
	1214 3.36928567143957e-05
	1215 3.13499251731741e-05
	1216 2.67258324129216e-05
	1217 2.30521756190427e-05
	1218 2.2657139872706e-05
	1219 2.31297629937899e-05
	1220 2.55390448415937e-05
	1221 2.80530704728221e-05
	1222 2.90230204136321e-05
	1223 2.60563359493204e-05
	1224 2.07628887629596e-05
	1225 1.65586731526446e-05
	1226 1.47889112032118e-05
	1227 1.46030598209279e-05
	1228 1.47617291208491e-05
	1229 1.45649975475948e-05
	1230 1.3985502130609e-05
	1231 1.33055332103282e-05
	1232 1.27804399294718e-05
	1233 1.25597831726054e-05
	1234 1.26090975083315e-05
	1235 1.28131133294573e-05
	1236 1.30194168264097e-05
	1237 1.31471728543886e-05
	1238 1.31606327737899e-05
	1239 1.30697193139184e-05
	1240 1.29052092532334e-05
	1241 1.27127713493991e-05
	1242 1.25301920625986e-05
	1243 1.23714124029561e-05
	1244 1.22464105345443e-05
	1245 1.2149136095374e-05
	1246 1.20833891639194e-05
	1247 1.2022184067817e-05
	1248 1.19876242123951e-05
	1249 1.19418227964019e-05
	1250 1.19174373356401e-05
	1251 1.18875821417674e-05
	1252 1.18719368753517e-05
	1253 1.18529982771065e-05
	1254 1.18434577149884e-05
	1255 1.18241890962878e-05
	1256 1.18127178883043e-05
	1257 1.17899804887855e-05
	1258 1.17758191500172e-05
	1259 1.17540804787097e-05
	1260 1.17405021917705e-05
	1261 1.17232624159413e-05
	1262 1.16986283060783e-05
	1263 1.16848605227204e-05
	1264 1.16626000679076e-05
	1265 1.16405721932722e-05
	1266 1.16318542850991e-05
	1267 1.16065704656165e-05
	1268 1.1593510768293e-05
	1269 1.15745436826131e-05
	1270 1.15608740056672e-05
	1271 1.15407177929683e-05
	1272 1.15287473487768e-05
	1273 1.15088344703196e-05
	1274 1.14901496264253e-05
	1275 1.1476848705172e-05
	1276 1.14651927098919e-05
	1277 1.14397569834779e-05
	1278 1.14326380415974e-05
	1279 1.14110257705136e-05
	1280 1.14068486283259e-05
	1281 1.13762506526172e-05
	1282 1.13773735002098e-05
	1283 1.13512444084307e-05
	1284 1.13426728347576e-05
	1285 1.13235987502236e-05
	1286 1.13187425032635e-05
	1287 1.12876900164594e-05
	1288 1.12936724026014e-05
	1289 1.12593916767878e-05
	1290 1.12628097639345e-05
	1291 1.12349548757606e-05
	1292 1.12266481746559e-05
	1293 1.12160759737634e-05
	1294 1.11991591591476e-05
	1295 1.11840469365632e-05
	1296 1.11738933021854e-05
	1297 1.11658122143865e-05
	1298 1.1151054387426e-05
	1299 1.11287951121142e-05
	1300 1.11304323766603e-05
	1301 1.1107791507925e-05
	1302 1.1103128105816e-05
	1303 1.10906855894655e-05
	1304 1.10759967455465e-05
	1305 1.10662890726587e-05
	1306 1.10584523955737e-05
	1307 1.1045916620489e-05
	1308 1.10390956340467e-05
	1309 1.10290445167749e-05
	1310 1.10260212622393e-05
	1311 1.10005345117514e-05
	1312 1.10115661264842e-05
	1313 1.09902392431138e-05
	1314 1.09886737131859e-05
	1315 1.09705700630514e-05
	1316 1.09751893067056e-05
	1317 1.09640125973698e-05
	1318 1.09421436711443e-05
	1319 1.09511003412166e-05
	1320 1.09261240179848e-05
	1321 1.09377474739603e-05
	1322 1.090411330118e-05
	1323 1.09208174521314e-05
	1324 1.08903773057278e-05
	1325 1.09124203166999e-05
	1326 1.08633963442628e-05
	1327 1.09017900520314e-05
	1328 1.08353973899966e-05
	1329 1.0921966394406e-05
	1330 1.08006394707871e-05
	1331 1.10038631770948e-05
	1332 1.08156489808664e-05
	1333 1.13759569817518e-05
	1334 1.1165042543837e-05
	1335 1.28408389432622e-05
	1336 1.30665275932529e-05
	1337 1.69595956194257e-05
	1338 1.62124639473404e-05
	1339 1.65327599823684e-05
	1340 1.32490764741533e-05
	1341 1.15958030786345e-05
	1342 1.14711865943917e-05
	1343 1.10806927509088e-05
	1344 1.12597280566007e-05
	1345 1.12730705819075e-05
	1346 1.12323892267341e-05
	1347 1.13134399324366e-05
	1348 1.13392545326718e-05
	1349 1.13348613695052e-05
	1350 1.13385362734419e-05
	1351 1.1332110574358e-05
	1352 1.14385670251238e-05
	1353 1.14487236260175e-05
	1354 1.14518070262193e-05
	1355 1.14511484063939e-05
	1356 1.15987445674648e-05
	1357 1.16220723320737e-05
	1358 1.16101256111278e-05
	1359 1.16795154170291e-05
	1360 1.16866624857437e-05
	1361 1.18050711819251e-05
	1362 1.18495347400938e-05
	1363 1.19444472232999e-05
	1364 1.20217599182126e-05
	1365 1.20963663974294e-05
	1366 1.22292770008414e-05
	1367 1.23091813755138e-05
	1368 1.24458429233698e-05
	1369 1.25385809930378e-05
	1370 1.27329099441909e-05
	1371 1.28977958873122e-05
	1372 1.31946076180611e-05
	1373 1.32917298110158e-05
	1374 1.37497327621361e-05
	1375 1.40075865111555e-05
	1376 1.4630584214359e-05
	1377 1.49851880770768e-05
	1378 1.56501989181379e-05
	1379 1.65552148629899e-05
	1380 1.77956400051471e-05
	1381 1.94193464864156e-05
	1382 2.11572209920519e-05
	1383 2.36598590461767e-05
	1384 2.71700262182151e-05
	1385 3.10563957448551e-05
	1386 3.44378855388072e-05
	1387 3.59370109066504e-05
	1388 3.44411225938757e-05
	1389 3.27522159366822e-05
	1390 3.51600865613477e-05
	1391 4.14222058395808e-05
	1392 4.7155252588027e-05
	1393 4.69880403066725e-05
	1394 3.74354488457129e-05
	1395 2.47172111116356e-05
	1396 1.61805900376066e-05
	1397 1.22859224109817e-05
	1398 1.07449695576634e-05
	1399 1.03080280311474e-05
	1400 1.02089967448649e-05
	1401 1.02362542335754e-05
	1402 1.02972559368197e-05
	1403 1.03497798367158e-05
	1404 1.03767344636907e-05
	1405 1.03778572881907e-05
	1406 1.03534713034037e-05
	1407 1.03033599909708e-05
	1408 1.02435984352667e-05
	1409 1.0170822950073e-05
	1410 1.01020903127136e-05
	1411 1.00393669040244e-05
	1412 9.99200563889246e-06
	1413 9.94829311196099e-06
	1414 9.91435078212533e-06
	1415 9.88703381210598e-06
	1416 9.86602410613102e-06
	1417 9.84796511360742e-06
	1418 9.83623564820846e-06
	1419 9.82450298003812e-06
	1420 9.81209337780342e-06
	1421 9.80289250840372e-06
	1422 9.78945015184252e-06
	1423 9.78210206881158e-06
	1424 9.77002109081582e-06
	1425 9.75890230670018e-06
	1426 9.74845087498011e-06
	1427 9.73704934814634e-06
	1428 9.72333505622203e-06
	1429 9.706895021111e-06
	1430 9.69440993259241e-06
	1431 9.67490834646867e-06
	1432 9.65858139423403e-06
	1433 9.64092349597934e-06
	1434 9.62135932702779e-06
	1435 9.60744671907321e-06
	1436 9.58731375888533e-06
	1437 9.57140844981552e-06
	1438 9.56090358883444e-06
	1439 9.54919736173565e-06
	1440 9.5352158311357e-06
	1441 9.52405005705259e-06
	1442 9.51893257905567e-06
	1443 9.50959759471459e-06
	1444 9.50415462419585e-06
	1445 9.49968002927903e-06
	1446 9.49225515611829e-06
	1447 9.49080652823397e-06
	1448 9.48474520612308e-06
	1449 9.4836545923016e-06
	1450 9.4733079887277e-06
	1451 9.47676029028344e-06
	1452 9.46673259427655e-06
	1453 9.46784271604884e-06
	1454 9.46201865659191e-06
	1455 9.46344334806781e-06
	1456 9.46003829938036e-06
	1457 9.45773411409334e-06
	1458 9.45658840301178e-06
	1459 9.45555363784933e-06
	1460 9.45068939550708e-06
	1461 9.45236057603438e-06
	1462 9.45684516118206e-06
	1463 9.45244216588037e-06
	1464 9.44789876911045e-06
	1465 9.45293240128819e-06
	1466 9.44587786477769e-06
	1467 9.4527387126675e-06
	1468 9.45288495302066e-06
	1469 9.45010381236955e-06
	1470 9.44570548710999e-06
	1471 9.45967403076509e-06
	1472 9.44598103558292e-06
	1473 9.45101966287609e-06
	1474 9.4541738047127e-06
	1475 9.45604616120477e-06
	1476 9.44795134749654e-06
	1477 9.45606186597558e-06
	1478 9.45985186895371e-06
	1479 9.44248254519664e-06
	1480 9.45834849730431e-06
	1481 9.47502463866101e-06
	1482 9.4538344121986e-06
	1483 9.47503968973251e-06
	1484 9.4625275046667e-06
	1485 9.47492952008133e-06
	1486 9.48506608011712e-06
	1487 9.48330935912622e-06
	1488 9.47347911761653e-06
	1489 9.47061240630376e-06
	1490 9.48925675281487e-06
	1491 9.49590418031221e-06
	1492 9.46664168921529e-06
	1493 9.49415660578268e-06
	1494 9.51157191231289e-06
	1495 9.50451407177866e-06
	1496 9.49853147957924e-06
	1497 9.51461483111871e-06
	1498 9.52741144999436e-06
	1499 9.50282137424097e-06
	1500 9.52851559965495e-06
	1501 9.53530477687536e-06
	1502 9.52443513035917e-06
	1503 9.52691031486097e-06
	1504 9.53791614222155e-06
	1505 9.53617298904419e-06
	1506 9.53217527843719e-06
	1507 9.52856773572819e-06
	1508 9.55161498694679e-06
	1509 9.53827818861441e-06
	1510 9.55298538585225e-06
	1511 9.53649933954637e-06
	1512 9.5415618641681e-06
	1513 9.53718543428295e-06
	1514 9.55956042147932e-06
	1515 9.54097017924482e-06
	1516 9.5670006530213e-06
	1517 9.4933633754124e-06
	1518 9.63629258166065e-06
	1519 9.45104097738181e-06
	1520 9.7652208790322e-06
	1521 9.41001862209134e-06
	1522 1.0311552140152e-05
	1523 9.66673574964716e-06
	1524 1.24654672752911e-05
	1525 1.18703488993788e-05
	1526 1.78086874527139e-05
	1527 1.42714894479923e-05
	1528 1.35099441749276e-05
	1529 1.1366533758661e-05
	1530 1.066859809562e-05
	1531 1.14649800266875e-05
	1532 1.16273256338673e-05
	1533 1.19509452431998e-05
	1534 1.26115482537159e-05
	1535 1.31580993283364e-05
	1536 1.36992605348496e-05
	1537 1.46809503576151e-05
	1538 1.59178571337293e-05
	1539 1.73264414158325e-05
	1540 1.92555277429562e-05
	1541 2.17805590168041e-05
	1542 2.54590798665078e-05
	1543 2.93403278703863e-05
	1544 3.27388333261069e-05
	1545 3.43441260639565e-05
	1546 3.25983719733358e-05
	1547 3.14519934576651e-05
	1548 3.43312344668334e-05
	1549 4.23922746790595e-05
	1550 4.93209785332738e-05
	1551 4.74051332091108e-05
	1552 3.56338108318255e-05
	1553 2.27599387194743e-05
	1554 1.47459087660451e-05
	1555 1.11112159082438e-05
	1556 9.65141765796318e-06
	1557 9.16033494746671e-06
	1558 9.00194845954161e-06
	1559 8.97203157634863e-06
	1560 8.9978001565072e-06
	1561 9.04173351834459e-06
	1562 9.09150503858314e-06
	1563 9.14600237678087e-06
	1564 9.19437479929286e-06
	1565 9.21874558024172e-06
	1566 9.22071855491424e-06
	1567 9.21032846612491e-06
	1568 9.18048795561788e-06
	1569 9.12494064664315e-06
	1570 9.065588399082e-06
	1571 8.99034002443955e-06
	1572 8.93362965026512e-06
	1573 8.86732238303267e-06
	1574 8.81972447963619e-06
	1575 8.76710514052803e-06
	1576 8.73337464923907e-06
	1577 8.69377436618635e-06
	1578 8.67190171405241e-06
	1579 8.64326022842477e-06
	1580 8.62650925359532e-06
	1581 8.60977650241068e-06
	1582 8.60008743330809e-06
	1583 8.59088726556934e-06
	1584 8.58542897752557e-06
	1585 8.5764454107462e-06
	1586 8.57432833978322e-06
	1587 8.56613657163052e-06
	1588 8.55584589931624e-06
	1589 8.55213398232024e-06
	1590 8.54522673510161e-06
	1591 8.54339133393722e-06
	1592 8.53376105691694e-06
	1593 8.52958435082485e-06
	1594 8.52111876614714e-06
	1595 8.51341620666801e-06
	1596 8.50815328856669e-06
	1597 8.49848793738062e-06
	1598 8.49333570940303e-06
	1599 8.48975144318587e-06
	1600 8.47680560944752e-06
	1601 8.47528371927808e-06
	1602 8.47364731093592e-06
	1603 8.46906993423602e-06
	1604 8.46416328492694e-06
	1605 8.47854073526833e-06
	1606 8.49015969528466e-06
	1607 8.49368235300574e-06
	1608 8.49331821761723e-06
	1609 8.49605492625471e-06
	1610 8.49965954863308e-06
	1611 8.50780225292169e-06
	1612 8.51334966611716e-06
	1613 8.51860217743194e-06
	1614 8.5254614470287e-06
	1615 8.54049741594309e-06
	1616 8.54411311212999e-06
	1617 8.56190014975056e-06
	1618 8.5731073511397e-06
	1619 8.58994980657002e-06
	1620 8.60950221071022e-06
	1621 8.62461838568862e-06
	1622 8.63229924430442e-06
	1623 8.65978762742969e-06
	1624 8.68592281655367e-06
	1625 8.69808725845189e-06
	1626 8.73197301753947e-06
	1627 8.75774221409387e-06
	1628 8.79715995605324e-06
	1629 8.83008168184318e-06
	1630 8.86633645258428e-06
	1631 8.91831892779749e-06
	1632 8.9685540123341e-06
	1633 9.0100333025589e-06
	1634 9.05893884262099e-06
	1635 9.11937467940049e-06
	1636 9.18614177614074e-06
	1637 9.22956333582192e-06
	1638 9.3383300932004e-06
	1639 9.40403391602729e-06
	1640 9.52686217736698e-06
	1641 9.64186630891817e-06
	1642 9.74087504435062e-06
	1643 9.8538959276695e-06
	1644 1.00714645219213e-05
	1645 1.02142805769034e-05
	1646 1.04309875021968e-05
	1647 1.06571896942853e-05
	1648 1.0931620942678e-05
	1649 1.12772360321145e-05
	1650 1.16377905809983e-05
	1651 1.2012235307779e-05
	1652 1.25331133293827e-05
	1653 1.30310527666211e-05
	1654 1.37062888176942e-05
	1655 1.44960765551261e-05
	1656 1.56134329802882e-05
	1657 1.66750744554633e-05
	1658 1.76916793446225e-05
	1659 1.84093561017562e-05
	1660 1.87287807129621e-05
	1661 1.85153992475762e-05
	1662 1.81432838814999e-05
	1663 1.7984330078491e-05
	1664 1.82883498522912e-05
	1665 1.95362945198951e-05
	1666 2.19270890227108e-05
	1667 2.60062840737874e-05
	1668 3.19264902017835e-05
	1669 3.76858007271608e-05
	1670 3.89638729032526e-05
	1671 3.33524064330959e-05
	1672 2.29863897018845e-05
	1673 1.49903127084627e-05
	1674 1.13755623640799e-05
	1675 9.95127441250077e-06
	1676 9.31201149434457e-06
	1677 8.80467122410522e-06
	1678 8.44351638207286e-06
	1679 8.25793454950485e-06
	1680 8.20115297450741e-06
	1681 8.25608977628178e-06
	1682 8.38530529989612e-06
	1683 8.53642351295036e-06
	1684 8.66896203799428e-06
	1685 8.74604823497549e-06
	1686 8.74857111732297e-06
	1687 8.69381680956849e-06
	1688 8.59939346486982e-06
	1689 8.51861632611417e-06
	1690 8.40082872599623e-06
	1691 8.35182232528098e-06
	1692 8.24562995660472e-06
	1693 8.21057953093174e-06
	1694 8.13061445992247e-06
	1695 8.1095161927891e-06
	1696 8.06073134018703e-06
	1697 8.04245380603419e-06
	1698 8.01888257573324e-06
	1699 8.00210538631774e-06
	1700 7.98299385174062e-06
	1701 7.97582646772099e-06
	1702 7.96139900316462e-06
	1703 7.95894520955187e-06
	1704 7.95011462884077e-06
	1705 7.95013628440699e-06
	1706 7.94973816731215e-06
	1707 7.94705930928785e-06
	1708 7.94556587635498e-06
	1709 7.94609552290382e-06
	1710 7.94598432740656e-06
	1711 7.94886587662802e-06
	1712 7.94622033861714e-06
	1713 7.94450266461411e-06
	1714 7.94318010921558e-06
	1715 7.94521585945063e-06
	1716 7.93773704366174e-06
	1717 7.93731259740582e-06
	1718 7.93482164418435e-06
	1719 7.92935515381288e-06
	1720 7.92874768063001e-06
	1721 7.92478977285072e-06
	1722 7.92567041774106e-06
	1723 7.92163976903737e-06
	1724 7.921109856035e-06
	1725 7.92143907979437e-06
	1726 7.92129901494576e-06
	1727 7.91956115264725e-06
	1728 7.9217463024861e-06
	1729 7.92748162581347e-06
	1730 7.92903392987654e-06
	1731 7.92859630127651e-06
	1732 7.93478176319695e-06
	1733 7.94135614690106e-06
	1734 7.9462947519815e-06
	1735 7.94632846901067e-06
	1736 7.95910833240043e-06
	1737 7.96594480245716e-06
	1738 7.96145480919108e-06
	1739 7.98255974032713e-06
	1740 7.9870855529407e-06
	1741 7.98761263087755e-06
	1742 8.01449952891886e-06
	1743 8.00651189525325e-06
	1744 8.01592495758285e-06
	1745 8.04456080505389e-06
	1746 8.04669452314499e-06
	1747 8.05037649875828e-06
	1748 8.07393669521872e-06
	1749 8.06550042131704e-06
	1750 8.09965177950289e-06
	1751 8.11974055547893e-06
	1752 8.1064668693287e-06
	1753 8.14624405265363e-06
	1754 8.15381836982709e-06
	1755 8.16323594943213e-06
	1756 8.21066058431796e-06
	1757 8.20722496897019e-06
	1758 8.26054034863688e-06
	1759 8.32954238383365e-06
	1760 8.36978921725517e-06
	1761 8.35719813974123e-06
	1762 8.44966240531875e-06
	1763 8.52970108411455e-06
	1764 8.56233342894086e-06
	1765 8.64662958122153e-06
	1766 8.70594967672389e-06
	1767 8.76239857738881e-06
	1768 8.83780687033209e-06
	1769 8.94732461453884e-06
	1770 9.05800198047757e-06
	1771 9.15492588937639e-06
	1772 9.18890446932608e-06
	1773 9.36514724259041e-06
	1774 9.50267858357279e-06
	1775 9.68670956780215e-06
	1776 9.6967041383067e-06
	1777 1.00394501032497e-05
	1778 1.02390893612636e-05
	1779 1.0567913802717e-05
	1780 1.08898671076219e-05
	1781 1.12968772345567e-05
	1782 1.17264139571915e-05
	1783 1.22571469791666e-05
	1784 1.29052374333583e-05
	1785 1.37437938647622e-05
	1786 1.47289121184002e-05
	1787 1.58987590328508e-05
	1788 1.76077554439047e-05
	1789 1.98125731749599e-05
	1790 2.23665995138234e-05
	1791 2.50528255829607e-05
	1792 2.68549441315713e-05
	1793 2.73018197702868e-05
	1794 2.63417516777054e-05
	1795 2.60951862429692e-05
	1796 2.89608225330085e-05
	1797 3.56977748481313e-05
	1798 4.28102145164644e-05
	1799 4.49452758957136e-05
	1800 3.75545000679267e-05
	1801 2.42499707532318e-05
	1802 1.45691510837764e-05
	1803 1.0170208939897e-05
	1804 8.64044807613595e-06
	1805 8.13494643914225e-06
	1806 7.95089842231533e-06
	1807 8.00154082103433e-06
	1808 8.15182235314182e-06
	1809 8.32165543229735e-06
	1810 8.46239202800803e-06
	1811 8.5288443720799e-06
	1812 8.52187952737893e-06
	1813 8.44968974877958e-06
	1814 8.31756040753362e-06
	1815 8.16780334567113e-06
	1816 8.02179461611985e-06
	1817 7.89323695027377e-06
	1818 7.80713231662844e-06
	1819 7.7345359565939e-06
	1820 7.6901008556618e-06
	1821 7.6563947182251e-06
	1822 7.63439816964251e-06
	1823 7.62716684210574e-06
	1824 7.62501952866756e-06
	1825 7.63322068397798e-06
	1826 7.63701847361631e-06
	1827 7.64852603918342e-06
	1828 7.65657609314019e-06
	1829 7.66668131912951e-06
	1830 7.66621653092159e-06
	1831 7.66646391792136e-06
	1832 7.66254062956051e-06
	1833 7.65433970251195e-06
	1834 7.64371244610373e-06
	1835 7.62489017436252e-06
	1836 7.61337702748222e-06
	1837 7.59404987782375e-06
	1838 7.5760141324821e-06
	1839 7.55605085966948e-06
	1840 7.53969223943329e-06
	1841 7.52062511821805e-06
	1842 7.50304797758616e-06
	1843 7.4848947422268e-06
	1844 7.46963178244187e-06
	1845 7.46140704244169e-06
	1846 7.44615101133661e-06
	1847 7.44032210864276e-06
	1848 7.4373200975586e-06
	1849 7.43600910713127e-06
	1850 7.43272710845844e-06
	1851 7.43598904850984e-06
	1852 7.44284271991802e-06
	1853 7.44046221079486e-06
	1854 7.44704699862098e-06
	1855 7.45478686070555e-06
	1856 7.4595352836937e-06
	1857 7.46709440324622e-06
	1858 7.47893136576749e-06
	1859 7.48528784111357e-06
	1860 7.49539145683542e-06
	1861 7.50631686230463e-06
	1862 7.51257850328813e-06
	1863 7.53250347251821e-06
	1864 7.53909048878398e-06
	1865 7.54945068770496e-06
	1866 7.57166208309457e-06
	1867 7.58696976355111e-06
	1868 7.60125816867685e-06
	1869 7.61906247781496e-06
	1870 7.6480809321211e-06
	1871 7.65768774346753e-06
	1872 7.68128130346213e-06
	1873 7.70207604006146e-06
	1874 7.72066508147873e-06
	1875 7.74888078325375e-06
	1876 7.78160221415192e-06
	1877 7.79103576054041e-06
	1878 7.83274109128484e-06
	1879 7.8921604655946e-06
	1880 7.93930541931331e-06
	1881 8.00832451908207e-06
	1882 8.06429324828883e-06
	1883 8.16268311343293e-06
	1884 8.20941785661944e-06
	1885 8.32716621967933e-06
	1886 8.48506471395893e-06
	1887 8.59687561405309e-06
	1888 8.73794299494079e-06
	1889 8.97200908411833e-06
	1890 9.19206351390756e-06
	1891 9.35411965308219e-06
	1892 9.69329886668646e-06
	1893 1.00598753309811e-05
	1894 1.0448134935892e-05
	1895 1.07790197043656e-05
	1896 1.14001466382518e-05
	1897 1.21765298990439e-05
	1898 1.29976236120655e-05
	1899 1.40278362792401e-05
	1900 1.5304271699712e-05
	1901 1.67523970278083e-05
	1902 1.83447014530103e-05
	1903 1.96598392712133e-05
	1904 2.05822788537091e-05
	1905 2.06211024647018e-05
	1906 1.98642856759079e-05
	1907 1.93915620840812e-05
	1908 2.02073694897109e-05
	1909 2.32860677442659e-05
	1910 2.89909367836572e-05
	1911 3.60411066573363e-05
	1912 4.12526347908226e-05
	1913 3.98950879336724e-05
	1914 3.03335769764601e-05
	1915 1.91387954906475e-05
	1916 1.24961224621245e-05
	1917 1.0335237437431e-05
	1918 9.77121165135486e-06
	1919 9.29338491939546e-06
	1920 8.57551540178747e-06
	1921 7.93107823415795e-06
	1922 7.59209225087432e-06
	1923 7.59640983094556e-06
	1924 7.8342709119994e-06
	1925 8.15678509802353e-06
	1926 8.40932296153341e-06
	1927 8.49920033463292e-06
	1928 8.41789944239224e-06
	1929 8.20955045721661e-06
	1930 7.95728553981689e-06
	1931 7.73432741230096e-06
	1932 7.58632331088194e-06
	1933 7.45124590117996e-06
	1934 7.3702227307848e-06
	1935 7.30275248628232e-06
	1936 7.26572932485681e-06
	1937 7.23148368919624e-06
	1938 7.21464339381583e-06
	1939 7.2035760227962e-06
	1940 7.19730252907169e-06
	1941 7.20003952281445e-06
	1942 7.20766780837323e-06
	1943 7.21655253865805e-06
	1944 7.22609011294395e-06
	1945 7.23958380977052e-06
	1946 7.25220140207483e-06
	1947 7.26553249297268e-06
	1948 7.28146431949739e-06
	1949 7.28009638795157e-06
	1950 7.28463788579603e-06
	1951 7.28278116390868e-06
	1952 7.27699043512331e-06
	1953 7.26736783462911e-06
	1954 7.25251115962777e-06
	1955 7.23239427280475e-06
	1956 7.21751806231907e-06
	1957 7.20176745261369e-06
	1958 7.18098394791866e-06
	1959 7.16810281886637e-06
	1960 7.14682344593598e-06
	1961 7.13435006893803e-06
	1962 7.11690925658814e-06
	1963 7.10334967291004e-06
	1964 7.09416330568757e-06
	1965 7.08120876691254e-06
	1966 7.07697913338023e-06
	1967 7.07116107001582e-06
	1968 7.07240121933239e-06
	1969 7.07176230108075e-06
	1970 7.07207939143473e-06
	1971 7.07580216996462e-06
	1972 7.07801916899342e-06
	1973 7.08134614235689e-06
	1974 7.0857010534553e-06
	1975 7.09112381080956e-06
	1976 7.0957258637705e-06
	1977 7.10362083733429e-06
	1978 7.11651550933112e-06
	1979 7.11931971775215e-06
	1980 7.13378045436031e-06
	1981 7.14705454374354e-06
	1982 7.15609196610956e-06
	1983 7.17342464184867e-06
	1984 7.1903203702206e-06
	1985 7.20339670756687e-06
	1986 7.22510598194503e-06
	1987 7.24015777731069e-06
	1988 7.26824749186505e-06
	1989 7.27465122718485e-06
	1990 7.3069930355274e-06
	1991 7.34345129593805e-06
	1992 7.32327720864845e-06
	1993 7.3818679258153e-06
	1994 7.41278009019908e-06
	1995 7.41511930080208e-06
	1996 7.4837031798225e-06
	1997 7.51012351685176e-06
	1998 7.54510457046109e-06
	1999 7.62219330763259e-06
};
\addlegendentry{Train}
\addplot [semithick, black]
table {%
	0 0.0359308160841465
	1 0.0352004282176495
	2 0.0344642400741577
	3 0.0337156169116497
	4 0.0329410471022129
	5 0.032118633389473
	6 0.031227171421051
	7 0.0302433036267757
	8 0.0291199460625648
	9 0.0277717914432287
	10 0.0260425675660372
	11 0.0240272022783756
	12 0.0223290789872408
	13 0.0207076724618673
	14 0.0192192997783422
	15 0.0178448017686605
	16 0.0165313445031643
	17 0.0153278177604079
	18 0.014218064956367
	19 0.0130960447713733
	20 0.0121293822303414
	21 0.0113847032189369
	22 0.0107572954148054
	23 0.010235195979476
	24 0.00981390941888094
	25 0.00946333818137646
	26 0.00916030444204807
	27 0.00888009928166866
	28 0.00857629533857107
	29 0.0082465847954154
	30 0.00793490093201399
	31 0.00767049379646778
	32 0.00743669969961047
	33 0.00722302542999387
	34 0.00701657542958856
	35 0.00681303068995476
	36 0.00661036837846041
	37 0.0064018415287137
	38 0.00619920017197728
	39 0.00598439248278737
	40 0.00576833402737975
	41 0.00553843379020691
	42 0.0053065218962729
	43 0.00505657726898789
	44 0.00479927891865373
	45 0.00452796369791031
	46 0.00424775527790189
	47 0.00397478230297565
	48 0.00371251930482686
	49 0.00343915401026607
	50 0.00317396968603134
	51 0.00292997690849006
	52 0.00270248181186616
	53 0.00249227927997708
	54 0.00229737930931151
	55 0.00212628277949989
	56 0.00197598873637617
	57 0.0018461657455191
	58 0.0017346452223137
	59 0.00163833191618323
	60 0.00155356305185705
	61 0.00147600390482694
	62 0.00140528241172433
	63 0.00134208227973431
	64 0.00129013264086097
	65 0.00124213506933302
	66 0.00119746010750532
	67 0.00115606072358787
	68 0.00111665832810104
	69 0.00107979623135179
	70 0.00104429759085178
	71 0.00101172958966345
	72 0.000979266129434109
	73 0.000949280802160501
	74 0.000920108228456229
	75 0.000892666867002845
	76 0.00086634635226801
	77 0.000840943423099816
	78 0.000817009306047112
	79 0.000793793820776045
	80 0.000771578983403742
	81 0.000750778417568654
	82 0.00073102954775095
	83 0.000711977132596076
	84 0.000694266171194613
	85 0.00067720643710345
	86 0.000661022553686053
	87 0.000645414693281054
	88 0.000630941009148955
	89 0.000617157085798681
	90 0.000604397268034518
	91 0.000592095719184726
	92 0.000580447784159333
	93 0.00056956266053021
	94 0.000559251406230032
	95 0.000549322983715683
	96 0.000540017616003752
	97 0.000531273020897061
	98 0.000522464397363365
	99 0.000514570449013263
	100 0.000507018528878689
	101 0.000499756715726107
	102 0.00049257866339758
	103 0.000486022268887609
	104 0.000479377631563693
	105 0.000473353982670233
	106 0.000467238220153376
	107 0.000461465970147401
	108 0.000455812667496502
	109 0.000450654304586351
	110 0.000445198704255745
	111 0.000440283183706924
	112 0.000435351277701557
	113 0.000430702872108668
	114 0.000425980833824724
	115 0.000421581033151597
	116 0.000417296221712604
	117 0.000413033179938793
	118 0.000408934021834284
	119 0.000405007827794179
	120 0.000401168916141614
	121 0.0003973045386374
	122 0.000393523863749579
	123 0.000390024128137156
	124 0.000386350118787959
	125 0.00038308955845423
	126 0.000379742705263197
	127 0.000376479962142184
	128 0.000373386254068464
	129 0.000370264606317505
	130 0.00036729738349095
	131 0.000364327861461788
	132 0.000361553888069466
	133 0.000358855293598026
	134 0.000355912081431597
	135 0.000353554059984162
	136 0.000350693357177079
	137 0.00034834208781831
	138 0.000345705891959369
	139 0.00034355215029791
	140 0.000341061357175931
	141 0.000338776735588908
	142 0.000336607365170494
	143 0.00033429657923989
	144 0.000332257681293413
	145 0.00033004701253958
	146 0.000327939284034073
	147 0.000325965316733345
	148 0.000323994841892272
	149 0.000321918720146641
	150 0.000320148275932297
	151 0.000318173319101334
	152 0.000316322024445981
	153 0.000314624339807779
	154 0.000312848569592461
	155 0.000311074429191649
	156 0.000309412716887891
	157 0.000307796988636255
	158 0.000306253670714796
	159 0.000304623506963253
	160 0.000303159526083618
	161 0.000301566906273365
	162 0.000300127576338127
	163 0.000298720930004492
	164 0.000297264603432268
	165 0.000295963400276378
	166 0.000294658122584224
	167 0.000293312768917531
	168 0.000292020500637591
	169 0.000290861498797312
	170 0.000289554358460009
	171 0.000288543320493773
	172 0.000287237780867144
	173 0.000286202470306307
	174 0.000285044050542638
	175 0.000284026667941362
	176 0.000283009023405612
	177 0.00028187659336254
	178 0.000280981650575995
	179 0.000279947038507089
	180 0.000279056665021926
	181 0.000278066436294466
	182 0.000277160055702552
	183 0.000276272388873622
	184 0.000275355472695082
	185 0.000274478748906404
	186 0.000273728248430416
	187 0.000272903445875272
	188 0.000272058503469452
	189 0.000271349330432713
	190 0.000270560529315844
	191 0.00026987589080818
	192 0.000269110430963337
	193 0.000268444826360792
	194 0.000267742609139532
	195 0.000267164665274322
	196 0.000266430899500847
	197 0.000265887385467067
	198 0.000265280716121197
	199 0.000264691334450617
	200 0.000264087721006945
	201 0.000263508147327229
	202 0.000263028894551098
	203 0.000262483838014305
	204 0.000262004177784547
	205 0.000261509703705087
	206 0.000260994274867699
	207 0.00026042212266475
	208 0.00026007485575974
	209 0.000259574182564393
	210 0.000259139720583335
	211 0.000258669228060171
	212 0.000258314918028191
	213 0.000257769337622449
	214 0.000257491366937757
	215 0.000257068371865898
	216 0.000256645522313192
	217 0.000256247789366171
	218 0.000255989172728732
	219 0.000255563907558098
	220 0.000255185354035348
	221 0.000254910031799227
	222 0.000254553888225928
	223 0.000254298065556213
	224 0.000253952341154218
	225 0.000253696198342368
	226 0.000253349717240781
	227 0.000253055768553168
	228 0.000252786529017612
	229 0.000252441299380735
	230 0.00025227534933947
	231 0.000252036697929725
	232 0.000251656048931181
	233 0.000251441961154342
	234 0.000251220393693075
	235 0.000250874174525961
	236 0.000250744284130633
	237 0.000250550277996808
	238 0.000250168202910572
	239 0.00025006418582052
	240 0.000249787408392876
	241 0.000249617907684296
	242 0.000249287870246917
	243 0.000249209231697023
	244 0.000249030388658866
	245 0.000248748925514519
	246 0.000248536583967507
	247 0.000248404365265742
	248 0.000248246098635718
	249 0.000248050346272066
	250 0.000247797055635601
	251 0.000247452640905976
	252 0.000247275573201478
	253 0.000246998766670004
	254 0.000246732262894511
	255 0.000246709736529738
	256 0.000246452720602974
	257 0.00024630650295876
	258 0.000246082956437021
	259 0.000245953444391489
	260 0.000245855510002002
	261 0.000245465460466221
	262 0.000245486182393506
	263 0.000245308357989416
	264 0.000245106115471572
	265 0.000245094095589593
	266 0.000244875292992219
	267 0.000244744180236012
	268 0.000244657974690199
	269 0.000244500435655937
	270 0.000244412309257314
	271 0.000244177761487663
	272 0.000244208960793912
	273 0.000244007082073949
	274 0.000243975300691091
	275 0.000243927526753396
	276 0.000243685804889537
	277 0.000243623566348106
	278 0.000243563496042043
	279 0.000243353380938061
	280 0.000243225062149577
	281 0.000243297094129957
	282 0.000243105649133213
	283 0.000242956186411902
	284 0.000242915484705009
	285 0.000242825437453575
	286 0.000242484282352962
	287 0.000242185007664375
	288 0.000241475048824213
	289 0.000240898298216052
	290 0.000240473644225858
	291 0.000239847751799971
	292 0.000239298766246065
	293 0.000238789070863277
	294 0.000238305205130018
	295 0.000237716507399455
	296 0.00023716501891613
	297 0.00023683428298682
	298 0.000236149135162123
	299 0.000235741317737848
	300 0.000235201921896078
	301 0.000234728140640073
	302 0.000234149600146338
	303 0.000233827842748724
	304 0.000233294937061146
	305 0.000232744467211887
	306 0.000232287784456275
	307 0.000231937781791203
	308 0.000231408164836466
	309 0.000231031692237593
	310 0.0002304965746589
	311 0.000230150166316889
	312 0.00022968124540057
	313 0.000229280136409216
	314 0.000228840610361658
	315 0.000228400516789407
	316 0.000227999058552086
	317 0.000227616110350937
	318 0.000227174532483332
	319 0.000226745833060704
	320 0.000226408708840609
	321 0.000226030009798706
	322 0.000225577590754256
	323 0.000225156749365851
	324 0.000224755160161294
	325 0.000224328730837442
	326 0.000223920433199964
	327 0.000223553885007277
	328 0.000223164941417053
	329 0.000222775532165542
	330 0.000222370668780059
	331 0.000222000540816225
	332 0.000221607464482076
	333 0.000221177411731333
	334 0.000220862639253028
	335 0.000220414061914198
	336 0.000220003144931979
	337 0.00021963266772218
	338 0.000219228735659271
	339 0.000218818517168984
	340 0.000218422952457331
	341 0.000218027984374203
	342 0.000217545501072891
	343 0.000217186781810597
	344 0.000216800515772775
	345 0.00021637327154167
	346 0.000215933221625164
	347 0.000215567051782273
	348 0.000215165317058563
	349 0.000214724292163737
	350 0.000214297047932632
	351 0.000213903593248688
	352 0.000213486811844632
	353 0.000213066712603904
	354 0.000212603699765168
	355 0.000212295475648716
	356 0.000211988255614415
	357 0.000211589474929497
	358 0.000211229576962069
	359 0.000210905447602272
	360 0.000210537953535095
	361 0.000210209895158187
	362 0.000209830090170726
	363 0.000209503836231306
	364 0.000209073841688223
	365 0.000208745230338536
	366 0.000208320023375563
	367 0.000207743549253792
	368 0.000206857293960638
	369 0.000206205048016272
	370 0.000205535747227259
	371 0.00020503562700469
	372 0.000204629934160039
	373 0.000204167270567268
	374 0.000203669565962628
	375 0.000203273099032231
	376 0.000202761744731106
	377 0.00020230702648405
	378 0.000201781935174949
	379 0.000201256363652647
	380 0.000200798691366799
	381 0.000200292459339835
	382 0.000199760615942068
	383 0.000199154324945994
	384 0.000198760390048847
	385 0.000198176217963919
	386 0.000197743822354823
	387 0.000197193454368971
	388 0.000196605586097576
	389 0.000196101289475337
	390 0.000195593922398984
	391 0.000195028522284701
	392 0.000194405962247401
	393 0.000193951738765463
	394 0.000193316693184897
	395 0.0001928134879563
	396 0.000192192121176049
	397 0.000191612751223147
	398 0.000190916864085011
	399 0.000190528939128853
	400 0.000189816535566933
	401 0.000189255544682965
	402 0.000188644320587628
	403 0.000188014979357831
	404 0.000187406258191913
	405 0.000186802484677173
	406 0.000186062781722285
	407 0.000185483484528959
	408 0.000184725038707256
	409 0.000184097167220898
	410 0.000183261203346774
	411 0.000182566305738874
	412 0.000181745592271909
	413 0.000180990085937083
	414 0.000180170507519506
	415 0.000179413458681665
	416 0.000178633941686712
	417 0.000177939247805625
	418 0.000177083784365095
	419 0.00017622574523557
	420 0.000175482156919315
	421 0.000174642438651063
	422 0.000173933411133476
	423 0.000173120919498615
	424 0.000172412444953807
	425 0.000171633597346954
	426 0.000170471786987036
	427 0.000169415128766559
	428 0.000168351776665077
	429 0.000167285514180548
	430 0.000166237950907089
	431 0.000165179269970395
	432 0.000164051030878909
	433 0.000163053467986174
	434 0.000162028372869827
	435 0.000161027244757861
	436 0.000160074923769571
	437 0.000159218179760501
	438 0.000158155336976051
	439 0.000157171845785342
	440 0.000156250287545845
	441 0.000155245637870394
	442 0.000154180146637373
	443 0.000153304266859777
	444 0.000152383479871787
	445 0.00015143706696108
	446 0.000150445121107623
	447 0.000149529791087843
	448 0.000148481820360757
	449 0.000147629965795204
	450 0.00014664868649561
	451 0.000145649042678997
	452 0.000144725083373487
	453 0.000143731158459559
	454 0.000142858756589703
	455 0.000142016258905642
	456 0.000141058058943599
	457 0.000140239368192852
	458 0.000139330193633214
	459 0.000138397925184108
	460 0.000137485403683968
	461 0.000136653223307803
	462 0.000135706533910707
	463 0.000134871850605123
	464 0.000134012283524498
	465 0.000133163397549652
	466 0.000132226690766402
	467 0.000131223889184184
	468 0.000130318207084201
	469 0.000129465130157769
	470 0.000128601139294915
	471 0.000127741644973867
	472 0.000126817845739424
	473 0.0001260242133867
	474 0.000125374222989194
	475 0.00012457040429581
	476 0.000123691817861982
	477 0.000122724581160583
	478 0.000121880155347753
	479 0.000121025608677883
	480 0.000120203098049387
	481 0.00011934863141505
	482 0.000118572628707625
	483 0.000117736381071154
	484 0.000116866394819226
	485 0.000116146562504582
	486 0.000115382943477016
	487 0.000114538954221644
	488 0.000113818285171874
	489 0.000113019392301794
	490 0.000112198955321219
	491 0.000111396715510637
	492 0.000110666740511078
	493 0.00010984704567818
	494 0.00010901853966061
	495 0.000108275875390973
	496 0.000107519910670817
	497 0.000106716812297236
	498 0.000105910701677203
	499 0.000105093786260113
	500 0.000104325372376479
	501 0.000103529484476894
	502 0.000102753925602883
	503 0.000101963065390009
	504 0.000101153731520753
	505 0.000100331351859495
	506 9.96162416413426e-05
	507 9.89045001915656e-05
	508 9.81927369139157e-05
	509 9.73913483903743e-05
	510 9.66816369327717e-05
	511 9.59094177233055e-05
	512 9.51229158090428e-05
	513 9.44293351494707e-05
	514 9.35797797865234e-05
	515 9.28426743485034e-05
	516 9.20690363273025e-05
	517 9.14647462195717e-05
	518 9.07554785953835e-05
	519 9.00268205441535e-05
	520 8.93908872967586e-05
	521 8.87837741174735e-05
	522 8.8169559603557e-05
	523 8.73465542099439e-05
	524 8.67582493810914e-05
	525 8.62262386363e-05
	526 8.54070458444767e-05
	527 8.47214978421107e-05
	528 8.40118373162113e-05
	529 8.34656457300298e-05
	530 8.27618569019251e-05
	531 8.20484492578544e-05
	532 8.13418009784073e-05
	533 8.09725606814027e-05
	534 8.0204481491819e-05
	535 7.9673518484924e-05
	536 7.89790792623535e-05
	537 7.83796203904785e-05
	538 7.77858585934155e-05
	539 7.720707799308e-05
	540 7.65626755310223e-05
	541 7.59878603275865e-05
	542 7.53188505768776e-05
	543 7.4767172918655e-05
	544 7.42698393878527e-05
	545 7.37143054720946e-05
	546 7.31589607312344e-05
	547 7.24498822819442e-05
	548 7.18173832865432e-05
	549 7.14360794518143e-05
	550 7.08099250914529e-05
	551 7.0264984969981e-05
	552 6.97835785103962e-05
	553 6.92250032443553e-05
	554 6.87608480802737e-05
	555 6.82032259646803e-05
	556 6.7773784394376e-05
	557 6.72216192469932e-05
	558 6.67035201331601e-05
	559 6.62003294564784e-05
	560 6.58542485325597e-05
	561 6.53097231406718e-05
	562 6.48371642455459e-05
	563 6.43103558104485e-05
	564 6.39870995655656e-05
	565 6.34592288406566e-05
	566 6.30835274932906e-05
	567 6.25811298959889e-05
	568 6.22043080511503e-05
	569 6.17213372606784e-05
	570 6.13134034210816e-05
	571 6.08833543083165e-05
	572 6.0597692936426e-05
	573 6.01746432948858e-05
	574 5.97422949795146e-05
	575 5.93761360505596e-05
	576 5.89621304243337e-05
	577 5.85739107918926e-05
	578 5.81865169806406e-05
	579 5.79278457735199e-05
	580 5.74659934500232e-05
	581 5.71709315408953e-05
	582 5.67464448977262e-05
	583 5.6394659623038e-05
	584 5.61937486054376e-05
	585 5.57150233362336e-05
	586 5.54107755306177e-05
	587 5.50579534319695e-05
	588 5.48561547475401e-05
	589 5.44262766197789e-05
	590 5.41218250873499e-05
	591 5.38968415639829e-05
	592 5.3536998166237e-05
	593 5.32408594153821e-05
	594 5.29460194229614e-05
	595 5.2684496040456e-05
	596 5.23480703122914e-05
	597 5.21714428032283e-05
	598 5.18175038450863e-05
	599 5.15098363393918e-05
	600 5.12798251293134e-05
	601 5.11560720042326e-05
	602 5.08021948917303e-05
	603 5.06180331285577e-05
	604 5.03707997268066e-05
	605 5.01765352964867e-05
	606 4.98221270390786e-05
	607 4.96377142553683e-05
	608 4.93443731102161e-05
	609 4.90932252432685e-05
	610 4.88349251099862e-05
	611 4.87517900182866e-05
	612 4.84348893223796e-05
	613 4.82146569993347e-05
	614 4.80828821309842e-05
	615 4.7831104893703e-05
	616 4.76344030175824e-05
	617 4.7415142034879e-05
	618 4.72636311315e-05
	619 4.70287559437566e-05
	620 4.68772595922928e-05
	621 4.66087512904778e-05
	622 4.64839758933522e-05
	623 4.63699434476439e-05
	624 4.6125212975312e-05
	625 4.58675567642786e-05
	626 4.57762507721782e-05
	627 4.55071494798176e-05
	628 4.53184838988818e-05
	629 4.51783889729995e-05
	630 4.50810730399098e-05
	631 4.48253013018984e-05
	632 4.46122321591247e-05
	633 4.45084260718431e-05
	634 4.4267471821513e-05
	635 4.41132724517956e-05
	636 4.40491712652147e-05
	637 4.38417300756555e-05
	638 4.36797308793757e-05
	639 4.35834226664156e-05
	640 4.33720088039991e-05
	641 4.32139131589793e-05
	642 4.31129519711249e-05
	643 4.2960477003362e-05
	644 4.27861923526507e-05
	645 4.25936086685397e-05
	646 4.24208119511604e-05
	647 4.22611337853596e-05
	648 4.21473596361466e-05
	649 4.20164215029217e-05
	650 4.18568888562731e-05
	651 4.17087430832908e-05
	652 4.15588910982478e-05
	653 4.14749774790835e-05
	654 4.13006309827324e-05
	655 4.1151881305268e-05
	656 4.09847998525947e-05
	657 4.0893744881032e-05
	658 4.07639272452798e-05
	659 4.06068320444319e-05
	660 4.04228158004116e-05
	661 4.03034828195814e-05
	662 4.01746947318316e-05
	663 4.00938224629499e-05
	664 4.00169374188408e-05
	665 3.9852977351984e-05
	666 3.97536859964021e-05
	667 3.97023140976671e-05
	668 3.95200113416649e-05
	669 3.94144954043441e-05
	670 3.92994697904214e-05
	671 3.92115871363785e-05
	672 3.91350658901501e-05
	673 3.8901769585209e-05
	674 3.88463668059558e-05
	675 3.86531173717231e-05
	676 3.85917483072262e-05
	677 3.85112944059074e-05
	678 3.83609876735136e-05
	679 3.81810968974605e-05
	680 3.81401951017324e-05
	681 3.80877791030798e-05
	682 3.79040902771521e-05
	683 3.78882177756168e-05
	684 3.76739699277095e-05
	685 3.757387821679e-05
	686 3.75097260985058e-05
	687 3.74720148101915e-05
	688 3.72700014850125e-05
	689 3.71776259271428e-05
	690 3.70854177162983e-05
	691 3.69405242963694e-05
	692 3.68759574485011e-05
	693 3.67673383152578e-05
	694 3.6631081457017e-05
	695 3.65886517101899e-05
	696 3.65068881365005e-05
	697 3.63758263119962e-05
	698 3.62721293640789e-05
	699 3.61797428922728e-05
	700 3.61483907909133e-05
	701 3.60508092853706e-05
	702 3.58569013769738e-05
	703 3.58152756234631e-05
	704 3.56798955181148e-05
	705 3.55595984729007e-05
	706 3.55186275555752e-05
	707 3.5425338865025e-05
	708 3.53227769664954e-05
	709 3.51936105289496e-05
	710 3.50888294633478e-05
	711 3.50623704434838e-05
	712 3.49622241628822e-05
	713 3.48482717527077e-05
	714 3.47478417097591e-05
	715 3.47361929016188e-05
	716 3.46268534485716e-05
	717 3.44647414749488e-05
	718 3.43604733643588e-05
	719 3.43347237503622e-05
	720 3.42427811119705e-05
	721 3.40629267157055e-05
	722 3.39751459250692e-05
	723 3.39076614181977e-05
	724 3.37712299369741e-05
	725 3.36662596964743e-05
	726 3.345124059706e-05
	727 3.33513926307205e-05
	728 3.33588650391903e-05
	729 3.32197087118402e-05
	730 3.30314251186792e-05
	731 3.29600516124628e-05
	732 3.29167742165737e-05
	733 3.28069691022392e-05
	734 3.27246489177924e-05
	735 3.26985355059151e-05
	736 3.25968612742145e-05
	737 3.23450367432088e-05
	738 3.22733212669846e-05
	739 3.23068925354164e-05
	740 3.21691550198011e-05
	741 3.19962564390153e-05
	742 3.19748287438415e-05
	743 3.18994934787042e-05
	744 3.18806596624199e-05
	745 3.16557707265019e-05
	746 3.15565594064537e-05
	747 3.1532566936221e-05
	748 3.14978751703165e-05
	749 3.14549761242233e-05
	750 3.13806449412368e-05
	751 3.1312221835833e-05
	752 3.10964933305513e-05
	753 3.11178737320006e-05
	754 3.10946670651902e-05
	755 3.10061732307076e-05
	756 3.09445349557791e-05
	757 3.08134112856351e-05
	758 3.07388218061533e-05
	759 3.08251655951608e-05
	760 3.06719521177001e-05
	761 3.06538095173892e-05
	762 3.04167206195416e-05
	763 3.04920358757954e-05
	764 3.03652032016544e-05
	765 3.03828473988688e-05
	766 3.02512344205752e-05
	767 3.02252410619985e-05
	768 3.0065104510868e-05
	769 3.00807532767067e-05
	770 3.0017929020687e-05
	771 2.99206403724384e-05
	772 2.98577888315776e-05
	773 2.98977483907947e-05
	774 2.96568086923799e-05
	775 2.96683247142937e-05
	776 2.96545404125936e-05
	777 2.95873669529101e-05
	778 2.93870562018128e-05
	779 2.93520861305296e-05
	780 2.94010460493155e-05
	781 2.94059918815037e-05
	782 2.93301600322593e-05
	783 2.92960339720594e-05
	784 2.90960961137898e-05
	785 2.90403440885711e-05
	786 2.9079477826599e-05
	787 2.9038139473414e-05
	788 2.91397773253266e-05
	789 2.87988568743458e-05
	790 2.89164054265711e-05
	791 2.8972852305742e-05
	792 2.87240436591674e-05
	793 2.87868915620493e-05
	794 2.86650465568528e-05
	795 2.86138438241323e-05
	796 2.86034137388924e-05
	797 2.85519690805813e-05
	798 2.84716861642664e-05
	799 2.84611123788636e-05
	800 2.84964917227626e-05
	801 2.84165289485827e-05
	802 2.82645942206727e-05
	803 2.82370747299865e-05
	804 2.81958509731339e-05
	805 2.80026488326257e-05
	806 2.81097873084946e-05
	807 2.79699288512347e-05
	808 2.8022068363498e-05
	809 2.80199765256839e-05
	810 2.77428498520749e-05
	811 2.78708084806567e-05
	812 2.77339386229869e-05
	813 2.77202743745875e-05
	814 2.77106100838864e-05
	815 2.76840582955629e-05
	816 2.74489193543559e-05
	817 2.74315952992765e-05
	818 2.74548219749704e-05
	819 2.73317546088947e-05
	820 2.73013647529297e-05
	821 2.72292672889307e-05
	822 2.71949793386739e-05
	823 2.71197677648161e-05
	824 2.69968495558714e-05
	825 2.69661904894747e-05
	826 2.68072963081067e-05
	827 2.6801171770785e-05
	828 2.67544164671563e-05
	829 2.66257466137176e-05
	830 2.65429989667609e-05
	831 2.63356141658733e-05
	832 2.63168531091651e-05
	833 2.63578822341515e-05
	834 2.62069152086042e-05
	835 2.61368022620445e-05
	836 2.59811004070798e-05
	837 2.59640564763686e-05
	838 2.59398893831531e-05
	839 2.58319960266817e-05
	840 2.57049159699818e-05
	841 2.55986797128571e-05
	842 2.5555740648997e-05
	843 2.53454854828306e-05
	844 2.53261296165874e-05
	845 2.52720728894928e-05
	846 2.51425262831617e-05
	847 2.50957727985224e-05
	848 2.50484899879666e-05
	849 2.49050735874334e-05
	850 2.49104341492057e-05
	851 2.48124506470049e-05
	852 2.47532516368665e-05
	853 2.4647926693433e-05
	854 2.45815845119068e-05
	855 2.45773699134588e-05
	856 2.44823640969116e-05
	857 2.44134571403265e-05
	858 2.4231725546997e-05
	859 2.41899833781645e-05
	860 2.42062778852414e-05
	861 2.41284305957379e-05
	862 2.40840217884397e-05
	863 2.40825993387261e-05
	864 2.39369037444703e-05
	865 2.39583914662944e-05
	866 2.39266155404039e-05
	867 2.3832593797124e-05
	868 2.37643780565122e-05
	869 2.37874610320432e-05
	870 2.36131782003213e-05
	871 2.36484229390044e-05
	872 2.3634191165911e-05
	873 2.3536382286693e-05
	874 2.35221141338116e-05
	875 2.3473525288864e-05
	876 2.33850441873074e-05
	877 2.3406530090142e-05
	878 2.33417085837573e-05
	879 2.32952424994437e-05
	880 2.32950060308212e-05
	881 2.32057263929164e-05
	882 2.31546928262105e-05
	883 2.31875710596796e-05
	884 2.31011963478522e-05
	885 2.30639434448676e-05
	886 2.30505775107304e-05
	887 2.29703473451082e-05
	888 2.28969602176221e-05
	889 2.29656980081927e-05
	890 2.28604349103989e-05
	891 2.28767530643381e-05
	892 2.29036395467119e-05
	893 2.27474974963116e-05
	894 2.28466415137518e-05
	895 2.2695581719745e-05
	896 2.2718069885741e-05
	897 2.27312266360968e-05
	898 2.26044103328604e-05
	899 2.2585702026845e-05
	900 2.26046740863239e-05
	901 2.25522890104912e-05
	902 2.25411386054475e-05
	903 2.25372968998272e-05
	904 2.2468957467936e-05
	905 2.25754338316619e-05
	906 2.23856859520311e-05
	907 2.23350634769304e-05
	908 2.23630177060841e-05
	909 2.24172908929177e-05
	910 2.22828766709426e-05
	911 2.23869192268467e-05
	912 2.22014805331128e-05
	913 2.22325852519134e-05
	914 2.22057606151793e-05
	915 2.2291555069387e-05
	916 2.20633683056803e-05
	917 2.22245380427921e-05
	918 2.20168640225893e-05
	919 2.22588150791125e-05
	920 2.20780439121881e-05
	921 2.20780384552199e-05
	922 2.20740294025745e-05
	923 2.20342444663402e-05
	924 2.194446278736e-05
	925 2.19951871258672e-05
	926 2.18787845369661e-05
	927 2.20918554987293e-05
	928 2.1865274902666e-05
	929 2.17698670894606e-05
	930 2.19977391680004e-05
	931 2.18465211219154e-05
	932 2.17428660107544e-05
	933 2.2060527044232e-05
	934 2.16193875530735e-05
	935 2.19258236029418e-05
	936 2.17761589738075e-05
	937 2.17605374928098e-05
	938 2.19741650653305e-05
	939 2.17462566070026e-05
	940 2.1790716346004e-05
	941 2.15695272345329e-05
	942 2.19896883209003e-05
	943 2.17240958590992e-05
	944 2.18960685742786e-05
	945 2.17183369386476e-05
	946 2.18593613681151e-05
	947 2.16950902540702e-05
	948 2.19037883653073e-05
	949 2.17653177969623e-05
	950 2.17612978303805e-05
	951 2.18670775211649e-05
	952 2.14483043237124e-05
	953 2.19070589082548e-05
	954 2.18743225559592e-05
	955 2.17473316297401e-05
	956 2.18490040424513e-05
	957 2.17941342270933e-05
	958 2.1926311092102e-05
	959 2.19574139919132e-05
	960 2.18195364141138e-05
	961 2.18991772271693e-05
	962 2.17511878872756e-05
	963 2.21525588131044e-05
	964 2.18555051105795e-05
	965 2.22184644371737e-05
	966 2.20989277295303e-05
	967 2.24294562940486e-05
	968 2.1685136744054e-05
	969 2.26735319301952e-05
	970 2.24236737267347e-05
	971 2.24512787099229e-05
	972 2.24229388550157e-05
	973 2.22927974391496e-05
	974 2.29425841098418e-05
	975 2.23307342821499e-05
	976 2.25925978156738e-05
	977 2.27836026169825e-05
	978 2.20908968913136e-05
	979 2.2925731173018e-05
	980 2.29075394599931e-05
	981 2.25687472266145e-05
	982 2.31548820011085e-05
	983 2.28961962420726e-05
	984 2.31707272178028e-05
	985 2.28200970013859e-05
	986 2.30815621762304e-05
	987 2.3267779397429e-05
	988 2.30868190556066e-05
	989 2.34040671784896e-05
	990 2.33584014495136e-05
	991 2.33743339776993e-05
	992 2.33992504945491e-05
	993 2.34088583965786e-05
	994 2.31903723033611e-05
	995 2.31348549277755e-05
	996 2.30789210036164e-05
	997 2.31534322665539e-05
	998 2.29033612413332e-05
	999 2.26211632252671e-05
	1000 2.25382318603806e-05
	1001 2.22212092921836e-05
	1002 2.19632438529516e-05
	1003 2.1712294255849e-05
	1004 2.13509520108346e-05
	1005 2.10268190130591e-05
	1006 2.06764489121269e-05
	1007 2.04520765691996e-05
	1008 2.0163512090221e-05
	1009 1.98874349734979e-05
	1010 1.96535147551913e-05
	1011 1.95312113646651e-05
	1012 1.94159001694061e-05
	1013 1.9446028090897e-05
	1014 1.95916763914283e-05
	1015 1.98105262825266e-05
	1016 2.00050981220556e-05
	1017 2.03505624085665e-05
	1018 2.0736692022183e-05
	1019 2.11601618502755e-05
	1020 2.18267578020459e-05
	1021 2.2290054403129e-05
	1022 2.30811492656358e-05
	1023 2.33895807468798e-05
	1024 2.43039121414768e-05
	1025 2.53801154030953e-05
	1026 2.62185712927021e-05
	1027 2.78638835879974e-05
	1028 2.90947191388113e-05
	1029 3.12534757540561e-05
	1030 3.34103569912259e-05
	1031 3.60587291652337e-05
	1032 3.93143709516153e-05
	1033 4.35745423601475e-05
	1034 4.86437820654828e-05
	1035 5.4801523219794e-05
	1036 5.93613840464968e-05
	1037 6.06049761699978e-05
	1038 5.80651940254029e-05
	1039 5.41522713319864e-05
	1040 4.83636795252096e-05
	1041 3.88311491406057e-05
	1042 3.13266173179727e-05
	1043 2.59213611570885e-05
	1044 2.37425701925531e-05
	1045 2.23153219849337e-05
	1046 2.16392236325191e-05
	1047 2.14124884223565e-05
	1048 1.99140195036307e-05
	1049 1.93508367374307e-05
	1050 1.85538592631929e-05
	1051 1.80689203261863e-05
	1052 1.76767971424852e-05
	1053 1.74734268512111e-05
	1054 1.74093056557467e-05
	1055 1.74537526618224e-05
	1056 1.7616575860302e-05
	1057 1.78924638021272e-05
	1058 1.83053598448168e-05
	1059 1.87619225471281e-05
	1060 1.91830786206992e-05
	1061 1.95911125047132e-05
	1062 1.99351961782668e-05
	1063 2.01952188945143e-05
	1064 2.04463212867267e-05
	1065 2.06688637263142e-05
	1066 2.09353493119124e-05
	1067 2.11676524486393e-05
	1068 2.13526654988527e-05
	1069 2.16222815652145e-05
	1070 2.17202086787438e-05
	1071 2.18236600630917e-05
	1072 2.18042841879651e-05
	1073 2.18763980228687e-05
	1074 2.18823279283242e-05
	1075 2.19084176933393e-05
	1076 2.18668774323305e-05
	1077 2.17304714169586e-05
	1078 2.16272237594239e-05
	1079 2.15173840842908e-05
	1080 2.14874762605177e-05
	1081 2.13954390346771e-05
	1082 2.12789655051893e-05
	1083 2.11516635317821e-05
	1084 2.10123471333645e-05
	1085 2.08641977224033e-05
	1086 2.07560296985321e-05
	1087 2.06870317924768e-05
	1088 2.04656080313725e-05
	1089 2.03426097868942e-05
	1090 2.02530700335046e-05
	1091 2.02235059987288e-05
	1092 2.00802660401678e-05
	1093 2.00638205569703e-05
	1094 1.99097248696489e-05
	1095 1.99327041627839e-05
	1096 1.98115485545713e-05
	1097 1.97277095139725e-05
	1098 1.96919372683624e-05
	1099 1.96944565686863e-05
	1100 1.96825985767646e-05
	1101 1.9685088773258e-05
	1102 1.95844950212631e-05
	1103 1.97434837900801e-05
	1104 1.96720084204571e-05
	1105 1.97312383534154e-05
	1106 1.96777837118134e-05
	1107 1.98054840438999e-05
	1108 1.96459750441136e-05
	1109 1.99365877051605e-05
	1110 1.98477537196595e-05
	1111 1.98460329556838e-05
	1112 1.98304132936755e-05
	1113 2.00225877051707e-05
	1114 2.01041239051847e-05
	1115 2.01079565158579e-05
	1116 2.01530147023732e-05
	1117 2.03980598598719e-05
	1118 2.01509046746651e-05
	1119 2.03231193154352e-05
	1120 2.05956239369698e-05
	1121 2.03726322069997e-05
	1122 2.05624783120584e-05
	1123 2.07745106308721e-05
	1124 2.07535085792188e-05
	1125 2.07133107323898e-05
	1126 2.10125053854426e-05
	1127 2.0972489437554e-05
	1128 2.09850222745445e-05
	1129 2.14363462873735e-05
	1130 2.10466641874518e-05
	1131 2.16956541407853e-05
	1132 2.1126583305886e-05
	1133 2.17968390643364e-05
	1134 2.14812371268636e-05
	1135 2.18682125705527e-05
	1136 2.17450469790492e-05
	1137 2.18698623939417e-05
	1138 2.24822342715925e-05
	1139 2.12599770748056e-05
	1140 2.30510358960601e-05
	1141 2.11050646612421e-05
	1142 2.35780080402037e-05
	1143 2.17187280213693e-05
	1144 2.33156169997528e-05
	1145 2.17672732105711e-05
	1146 2.34785966313211e-05
	1147 2.21545778913423e-05
	1148 2.37357617152156e-05
	1149 2.22586986637907e-05
	1150 2.38511820498388e-05
	1151 2.21345744648715e-05
	1152 2.46213894570246e-05
	1153 2.26637039304478e-05
	1154 2.39263372350251e-05
	1155 2.29358083743136e-05
	1156 2.39998098550132e-05
	1157 2.31889680435415e-05
	1158 2.41438046941767e-05
	1159 2.28819844778627e-05
	1160 2.50143257289892e-05
	1161 2.29661945922999e-05
	1162 2.46303334279219e-05
	1163 2.33600458159344e-05
	1164 2.62671637756284e-05
	1165 2.19056164496578e-05
	1166 2.72963552561123e-05
	1167 2.15466243389528e-05
	1168 2.83140470855869e-05
	1169 2.11956466955598e-05
	1170 2.91717169602634e-05
	1171 2.16695007111412e-05
	1172 3.1163108360488e-05
	1173 2.13397506740876e-05
	1174 2.94726487481967e-05
	1175 2.24750911002047e-05
	1176 2.91885662591085e-05
	1177 2.30915484280558e-05
	1178 2.89652070932789e-05
	1179 2.26999163714936e-05
	1180 2.8877013392048e-05
	1181 2.32147558563156e-05
	1182 3.00355004583253e-05
	1183 2.45573737629456e-05
	1184 3.04802433674922e-05
	1185 2.37829990510363e-05
	1186 3.0518916901201e-05
	1187 2.55834456766024e-05
	1188 3.15447541652247e-05
	1189 2.56774474109989e-05
	1190 3.26649133057799e-05
	1191 2.83322005998343e-05
	1192 3.45102489518467e-05
	1193 2.86456579488004e-05
	1194 3.3934320526896e-05
	1195 2.99877010547789e-05
	1196 3.58080142177641e-05
	1197 3.39149992214516e-05
	1198 3.77430624212138e-05
	1199 3.4997305192519e-05
	1200 3.97058756789193e-05
	1201 3.75584168068599e-05
	1202 4.26876576966606e-05
	1203 4.05423816118855e-05
	1204 4.78767724416684e-05
	1205 4.74367370770779e-05
	1206 5.5598979088245e-05
	1207 5.57450621272437e-05
	1208 6.29611749900505e-05
	1209 6.51006703265011e-05
	1210 6.85235208948143e-05
	1211 7.16599315637723e-05
	1212 6.63126338622533e-05
	1213 5.64805450267158e-05
	1214 4.61525196442381e-05
	1215 3.38394165737554e-05
	1216 2.61440181930084e-05
	1217 2.48985634243581e-05
	1218 2.8694310458377e-05
	1219 3.09899078274611e-05
	1220 3.8169055187609e-05
	1221 3.94372946175281e-05
	1222 3.42679340974428e-05
	1223 2.44560123974225e-05
	1224 1.67910839081742e-05
	1225 1.46492893691175e-05
	1226 1.60936069732998e-05
	1227 1.79090438905405e-05
	1228 1.8446900867275e-05
	1229 1.77087622432737e-05
	1230 1.66574209288228e-05
	1231 1.57171743921936e-05
	1232 1.52263428390143e-05
	1233 1.52255706780124e-05
	1234 1.54507015395211e-05
	1235 1.57073918671813e-05
	1236 1.58262228069361e-05
	1237 1.57626200234517e-05
	1238 1.55640391312772e-05
	1239 1.52178108692169e-05
	1240 1.48621465996257e-05
	1241 1.45804970088648e-05
	1242 1.44206342156394e-05
	1243 1.43872139233281e-05
	1244 1.43950219353428e-05
	1245 1.44387631735299e-05
	1246 1.45044368764502e-05
	1247 1.46038455568487e-05
	1248 1.46292413774063e-05
	1249 1.46591601151158e-05
	1250 1.46329684866942e-05
	1251 1.46290803968441e-05
	1252 1.46346119436203e-05
	1253 1.46429501910461e-05
	1254 1.46180600495427e-05
	1255 1.46408228829387e-05
	1256 1.46060774568468e-05
	1257 1.46043494169135e-05
	1258 1.45688973134384e-05
	1259 1.45503845487838e-05
	1260 1.45405629154993e-05
	1261 1.45608064485714e-05
	1262 1.45143740155618e-05
	1263 1.45321064337622e-05
	1264 1.44964469654951e-05
	1265 1.44817322507151e-05
	1266 1.4481314792647e-05
	1267 1.44968325912487e-05
	1268 1.4451979950536e-05
	1269 1.44575951708248e-05
	1270 1.44305586218252e-05
	1271 1.44143323268509e-05
	1272 1.44217219713028e-05
	1273 1.4417954844248e-05
	1274 1.43748366099317e-05
	1275 1.43775996548356e-05
	1276 1.43582656164654e-05
	1277 1.43616780405864e-05
	1278 1.43093884616974e-05
	1279 1.43376319101662e-05
	1280 1.43078532346408e-05
	1281 1.43125635077013e-05
	1282 1.42665558087174e-05
	1283 1.4277633454185e-05
	1284 1.42446488098358e-05
	1285 1.42349599627778e-05
	1286 1.4225950508262e-05
	1287 1.42301641972153e-05
	1288 1.41849704959895e-05
	1289 1.41767468448961e-05
	1290 1.41610407808912e-05
	1291 1.41709015224478e-05
	1292 1.41198925120989e-05
	1293 1.41077171065263e-05
	1294 1.40998117785784e-05
	1295 1.41180125865503e-05
	1296 1.40664724312956e-05
	1297 1.40503607326536e-05
	1298 1.40404254125315e-05
	1299 1.40165739139775e-05
	1300 1.40116180773475e-05
	1301 1.39817057061009e-05
	1302 1.39669355121441e-05
	1303 1.3976786249259e-05
	1304 1.39556523208739e-05
	1305 1.39275489345891e-05
	1306 1.39261692311266e-05
	1307 1.39238245537854e-05
	1308 1.38934910864918e-05
	1309 1.38938694362878e-05
	1310 1.38658642754308e-05
	1311 1.38634468385135e-05
	1312 1.38594950840343e-05
	1313 1.3828901501256e-05
	1314 1.38055675051874e-05
	1315 1.38156474349671e-05
	1316 1.38048199005425e-05
	1317 1.37891393023892e-05
	1318 1.3770713849226e-05
	1319 1.37834604174714e-05
	1320 1.37328042910667e-05
	1321 1.37694232762442e-05
	1322 1.37080769491149e-05
	1323 1.37875404107035e-05
	1324 1.36672824737616e-05
	1325 1.37817733047996e-05
	1326 1.36343533085892e-05
	1327 1.3816059436067e-05
	1328 1.35331183628296e-05
	1329 1.39825315272901e-05
	1330 1.34312622321886e-05
	1331 1.41356094900402e-05
	1332 1.33347657538252e-05
	1333 1.52236607391387e-05
	1334 1.35093241624418e-05
	1335 1.87947607628303e-05
	1336 1.54979516082676e-05
	1337 2.68031526502455e-05
	1338 1.64534885698231e-05
	1339 2.29380930250045e-05
	1340 1.2909084034618e-05
	1341 1.57705071615055e-05
	1342 1.46668962770491e-05
	1343 1.4468454537564e-05
	1344 1.51426966112922e-05
	1345 1.4989280316513e-05
	1346 1.49714014696656e-05
	1347 1.54203298734501e-05
	1348 1.53172586578876e-05
	1349 1.54371446114965e-05
	1350 1.52661341417115e-05
	1351 1.55103953147773e-05
	1352 1.58610728249187e-05
	1353 1.58865805133246e-05
	1354 1.5803359929123e-05
	1355 1.60473555297358e-05
	1356 1.67581056302879e-05
	1357 1.6520825738553e-05
	1358 1.66912923305063e-05
	1359 1.68275255418848e-05
	1360 1.69817649293691e-05
	1361 1.7497257431387e-05
	1362 1.75265195139218e-05
	1363 1.80917522811797e-05
	1364 1.82450021384284e-05
	1365 1.86981396836927e-05
	1366 1.91132912732428e-05
	1367 1.94888161786366e-05
	1368 1.9820408851956e-05
	1369 2.02874834940303e-05
	1370 2.08153996936744e-05
	1371 2.17727301787818e-05
	1372 2.25715903070522e-05
	1373 2.29623256018385e-05
	1374 2.45498667936772e-05
	1375 2.55009053944377e-05
	1376 2.78931365755852e-05
	1377 2.88358532998245e-05
	1378 3.13876043946948e-05
	1379 3.41943850798998e-05
	1380 3.82478719984647e-05
	1381 4.31100197602063e-05
	1382 4.75652341265231e-05
	1383 5.32502235728316e-05
	1384 5.88971706747543e-05
	1385 6.03754197072703e-05
	1386 5.25711475347634e-05
	1387 3.71932728739921e-05
	1388 2.20793026528554e-05
	1389 2.3832197257434e-05
	1390 4.74932385259308e-05
	1391 7.41545445634983e-05
	1392 7.59418835514225e-05
	1393 5.23860508110374e-05
	1394 3.02100434055319e-05
	1395 1.98684647330083e-05
	1396 1.66321242431877e-05
	1397 1.51532167365076e-05
	1398 1.42010703712003e-05
	1399 1.36642811412457e-05
	1400 1.35021746245911e-05
	1401 1.35744567160145e-05
	1402 1.37990755320061e-05
	1403 1.40560969157377e-05
	1404 1.42756698551239e-05
	1405 1.44267032737844e-05
	1406 1.44882915265043e-05
	1407 1.44794403240667e-05
	1408 1.44294272104162e-05
	1409 1.43558554555057e-05
	1410 1.42643311846768e-05
	1411 1.4175024261931e-05
	1412 1.40872016345384e-05
	1413 1.40056663440191e-05
	1414 1.39397270686459e-05
	1415 1.38669565785676e-05
	1416 1.3828444025421e-05
	1417 1.37778515636455e-05
	1418 1.37571150844451e-05
	1419 1.37141523737228e-05
	1420 1.37039123728755e-05
	1421 1.36823646244011e-05
	1422 1.36897397169378e-05
	1423 1.36626867970335e-05
	1424 1.366803462588e-05
	1425 1.36438929985161e-05
	1426 1.36492044475744e-05
	1427 1.36265989567619e-05
	1428 1.36163289425895e-05
	1429 1.35738937387941e-05
	1430 1.35507325467188e-05
	1431 1.35160498757614e-05
	1432 1.34738920678501e-05
	1433 1.343668645859e-05
	1434 1.33894527607481e-05
	1435 1.33348012241186e-05
	1436 1.32895938804722e-05
	1437 1.32416216729325e-05
	1438 1.31860933834105e-05
	1439 1.3129499166098e-05
	1440 1.30880916913156e-05
	1441 1.30451080622151e-05
	1442 1.29991658468498e-05
	1443 1.29663940242608e-05
	1444 1.29303334688302e-05
	1445 1.28999063235824e-05
	1446 1.28675801533973e-05
	1447 1.28340152514284e-05
	1448 1.28062301882892e-05
	1449 1.27873890960473e-05
	1450 1.27485318444087e-05
	1451 1.2729574336845e-05
	1452 1.27008870549616e-05
	1453 1.26927070596139e-05
	1454 1.26514250951004e-05
	1455 1.26508875837317e-05
	1456 1.26097365864553e-05
	1457 1.26127661133069e-05
	1458 1.25662500067847e-05
	1459 1.25852047858643e-05
	1460 1.25351843962562e-05
	1461 1.25323440443026e-05
	1462 1.25243868751568e-05
	1463 1.2512129615061e-05
	1464 1.24881662486587e-05
	1465 1.25021260828362e-05
	1466 1.24629868878401e-05
	1467 1.24947328004055e-05
	1468 1.24714233606937e-05
	1469 1.24662255984731e-05
	1470 1.24578464237857e-05
	1471 1.24746293295175e-05
	1472 1.24351226986619e-05
	1473 1.24538464660873e-05
	1474 1.2448123015929e-05
	1475 1.24721946122008e-05
	1476 1.24472271636478e-05
	1477 1.24771158880321e-05
	1478 1.24625994430971e-05
	1479 1.24153993965592e-05
	1480 1.24908683574176e-05
	1481 1.25051019495004e-05
	1482 1.24585412777378e-05
	1483 1.25205942822504e-05
	1484 1.24820317068952e-05
	1485 1.25359592857421e-05
	1486 1.25726446640329e-05
	1487 1.25563628898817e-05
	1488 1.25051074064686e-05
	1489 1.25215292428038e-05
	1490 1.26437498693122e-05
	1491 1.25903306980035e-05
	1492 1.25295482575893e-05
	1493 1.26494433061453e-05
	1494 1.27092362163239e-05
	1495 1.26788372654119e-05
	1496 1.26624490803806e-05
	1497 1.27860721477191e-05
	1498 1.27610128402011e-05
	1499 1.26904187709442e-05
	1500 1.28478468468529e-05
	1501 1.28371084429091e-05
	1502 1.28106757983915e-05
	1503 1.28493784359307e-05
	1504 1.29206655401504e-05
	1505 1.28709025375429e-05
	1506 1.28816391224973e-05
	1507 1.28895862872014e-05
	1508 1.30018606796511e-05
	1509 1.29391710288473e-05
	1510 1.30115477077197e-05
	1511 1.29401860249345e-05
	1512 1.2983187843929e-05
	1513 1.30009257190977e-05
	1514 1.3131011655787e-05
	1515 1.30339931274648e-05
	1516 1.31338383653201e-05
	1517 1.29335221572546e-05
	1518 1.34753436213941e-05
	1519 1.26750201161485e-05
	1520 1.40170895974734e-05
	1521 1.23377139971126e-05
	1522 1.58214916154975e-05
	1523 1.19743244795245e-05
	1524 2.19336325244512e-05
	1525 1.3845921785105e-05
	1526 3.32261079165619e-05
	1527 1.19076394184958e-05
	1528 2.16937332879752e-05
	1529 1.52311631609336e-05
	1530 1.70599596458487e-05
	1531 1.95779120986117e-05
	1532 1.97185454453574e-05
	1533 2.12891081901034e-05
	1534 2.3694794435869e-05
	1535 2.51930596277816e-05
	1536 2.70579203061061e-05
	1537 3.03731139865704e-05
	1538 3.43650317518041e-05
	1539 3.85655293939635e-05
	1540 4.38027745985892e-05
	1541 4.90219354105648e-05
	1542 5.52687160961796e-05
	1543 5.69156400160864e-05
	1544 5.01205831824336e-05
	1545 3.575621303753e-05
	1546 2.06983295356622e-05
	1547 2.34477902267827e-05
	1548 5.0441190978745e-05
	1549 8.01353817223571e-05
	1550 7.73003048379906e-05
	1551 4.80161179439165e-05
	1552 2.8833006581408e-05
	1553 1.92204861377832e-05
	1554 1.59275623445865e-05
	1555 1.44967107189586e-05
	1556 1.36359722091584e-05
	1557 1.31280357891228e-05
	1558 1.28934962049243e-05
	1559 1.2822839380533e-05
	1560 1.2892996892333e-05
	1561 1.30874568640138e-05
	1562 1.33125331558404e-05
	1563 1.35357186081819e-05
	1564 1.37250326588401e-05
	1565 1.389441149513e-05
	1566 1.39870371640427e-05
	1567 1.40446463774424e-05
	1568 1.40393194669741e-05
	1569 1.39747526191059e-05
	1570 1.39018457048223e-05
	1571 1.37962815642823e-05
	1572 1.36804728754214e-05
	1573 1.35673999466235e-05
	1574 1.34549463837175e-05
	1575 1.33536723296857e-05
	1576 1.32495806610677e-05
	1577 1.31592387333512e-05
	1578 1.30616363094305e-05
	1579 1.2973859156773e-05
	1580 1.29082227431354e-05
	1581 1.28591209431761e-05
	1582 1.27917210193118e-05
	1583 1.27621915453346e-05
	1584 1.27123903439497e-05
	1585 1.26863478726591e-05
	1586 1.2634644008358e-05
	1587 1.25963770187809e-05
	1588 1.25614978969679e-05
	1589 1.25466713143396e-05
	1590 1.25070955618867e-05
	1591 1.24767730085296e-05
	1592 1.24483713079826e-05
	1593 1.24115613289177e-05
	1594 1.23829267977271e-05
	1595 1.23473328130785e-05
	1596 1.231315673067e-05
	1597 1.2283925570955e-05
	1598 1.22435767480056e-05
	1599 1.22060619105469e-05
	1600 1.21782059068209e-05
	1601 1.21460334412404e-05
	1602 1.2121997315262e-05
	1603 1.21087759907823e-05
	1604 1.20434515338275e-05
	1605 1.20101467473432e-05
	1606 1.19793621706776e-05
	1607 1.19509468277101e-05
	1608 1.19342294055969e-05
	1609 1.19022752187448e-05
	1610 1.18869365905994e-05
	1611 1.1874162737513e-05
	1612 1.18525531433988e-05
	1613 1.18482485049753e-05
	1614 1.18213220048347e-05
	1615 1.18258067232091e-05
	1616 1.18042753456393e-05
	1617 1.18206062325044e-05
	1618 1.18004609248601e-05
	1619 1.18249663501047e-05
	1620 1.18091056720004e-05
	1621 1.18338166430476e-05
	1622 1.18064735943335e-05
	1623 1.18440784717677e-05
	1624 1.18556317829643e-05
	1625 1.18721709441161e-05
	1626 1.1908654414583e-05
	1627 1.19601891128696e-05
	1628 1.20220820463146e-05
	1629 1.209184210893e-05
	1630 1.21487628348405e-05
	1631 1.22567398648243e-05
	1632 1.23276113299653e-05
	1633 1.24143489301787e-05
	1634 1.25040887724026e-05
	1635 1.26221284517669e-05
	1636 1.27528173834435e-05
	1637 1.28466781461611e-05
	1638 1.31384485939634e-05
	1639 1.3281875908433e-05
	1640 1.36209755510208e-05
	1641 1.39206140374881e-05
	1642 1.41194832394831e-05
	1643 1.43889483297244e-05
	1644 1.4941739209462e-05
	1645 1.52889042510651e-05
	1646 1.58543462021044e-05
	1647 1.6347854398191e-05
	1648 1.70673210959649e-05
	1649 1.79952803591732e-05
	1650 1.88351896213135e-05
	1651 1.96308865270112e-05
	1652 2.0775267330464e-05
	1653 2.18167242564959e-05
	1654 2.28011340368539e-05
	1655 2.40603803831618e-05
	1656 2.56880575761897e-05
	1657 2.64545487880241e-05
	1658 2.59357566392282e-05
	1659 2.41736979660345e-05
	1660 2.10577873076545e-05
	1661 1.73694625118515e-05
	1662 1.45477351907175e-05
	1663 1.47559821925825e-05
	1664 2.02023838937748e-05
	1665 3.10725517920218e-05
	1666 4.566063216771e-05
	1667 5.98131991864648e-05
	1668 7.15577261871658e-05
	1669 6.88167929183692e-05
	1670 5.39317370567005e-05
	1671 3.35635850206017e-05
	1672 2.14942501770565e-05
	1673 1.71987758221803e-05
	1674 1.57256035890896e-05
	1675 1.48055160025251e-05
	1676 1.38176274049329e-05
	1677 1.29546815514914e-05
	1678 1.23306208479335e-05
	1679 1.20161594168167e-05
	1680 1.206852721225e-05
	1681 1.24445023175213e-05
	1682 1.30129583340022e-05
	1683 1.35884492920013e-05
	1684 1.4035257663636e-05
	1685 1.42921744554769e-05
	1686 1.43497345561627e-05
	1687 1.42721646625432e-05
	1688 1.40802630994585e-05
	1689 1.3898928045819e-05
	1690 1.36916278279386e-05
	1691 1.34703277581139e-05
	1692 1.33123112391331e-05
	1693 1.31362348838593e-05
	1694 1.2994429198443e-05
	1695 1.28596284412197e-05
	1696 1.27516459542676e-05
	1697 1.2643582522287e-05
	1698 1.25408932944993e-05
	1699 1.24356411106419e-05
	1700 1.23432864711503e-05
	1701 1.22591472972999e-05
	1702 1.21759621833917e-05
	1703 1.2116376638005e-05
	1704 1.2056331797794e-05
	1705 1.19994419947034e-05
	1706 1.19530805022805e-05
	1707 1.18952839329722e-05
	1708 1.187260113511e-05
	1709 1.1841107152577e-05
	1710 1.18137777462834e-05
	1711 1.1779482520069e-05
	1712 1.17703402793268e-05
	1713 1.17448034870904e-05
	1714 1.17257814054028e-05
	1715 1.17044755825191e-05
	1716 1.16823921416653e-05
	1717 1.16629098556587e-05
	1718 1.16343389890972e-05
	1719 1.16191322376835e-05
	1720 1.15957618618268e-05
	1721 1.15775983431377e-05
	1722 1.15567027023644e-05
	1723 1.15329921754892e-05
	1724 1.1521431588335e-05
	1725 1.14869890239788e-05
	1726 1.14983813546132e-05
	1727 1.14593840407906e-05
	1728 1.14549275167519e-05
	1729 1.14362856038497e-05
	1730 1.14242629933869e-05
	1731 1.13998739834642e-05
	1732 1.14035483420594e-05
	1733 1.13683909148676e-05
	1734 1.13729820441222e-05
	1735 1.13397099994472e-05
	1736 1.13544001578703e-05
	1737 1.1330784218444e-05
	1738 1.13248415800626e-05
	1739 1.13374735519756e-05
	1740 1.13227824840578e-05
	1741 1.13232845251332e-05
	1742 1.1341915524099e-05
	1743 1.13162823254243e-05
	1744 1.13226988105453e-05
	1745 1.13792657430167e-05
	1746 1.13562837213976e-05
	1747 1.13761689135572e-05
	1748 1.13935020635836e-05
	1749 1.1360756616341e-05
	1750 1.14549930003705e-05
	1751 1.1489550161059e-05
	1752 1.14452514026198e-05
	1753 1.15402290248312e-05
	1754 1.15256589197088e-05
	1755 1.15714810817735e-05
	1756 1.16721139420406e-05
	1757 1.16508454084396e-05
	1758 1.17944300654926e-05
	1759 1.20103086374002e-05
	1760 1.2009491911158e-05
	1761 1.19689266284695e-05
	1762 1.22902929433621e-05
	1763 1.24836205941392e-05
	1764 1.25597698570346e-05
	1765 1.27966759464471e-05
	1766 1.29229129015584e-05
	1767 1.30651424115058e-05
	1768 1.33005232783034e-05
	1769 1.36474845930934e-05
	1770 1.39873027364956e-05
	1771 1.41416885526269e-05
	1772 1.42650387715548e-05
	1773 1.4856826965115e-05
	1774 1.53111159306718e-05
	1775 1.56768492161063e-05
	1776 1.56920850713504e-05
	1777 1.69512568390928e-05
	1778 1.75035293068504e-05
	1779 1.86155575647717e-05
	1780 1.96205837710295e-05
	1781 2.08411565836286e-05
	1782 2.21299360418925e-05
	1783 2.38382981478935e-05
	1784 2.56323874054942e-05
	1785 2.79720861726673e-05
	1786 3.07634581986349e-05
	1787 3.33090283675119e-05
	1788 3.70525158359669e-05
	1789 4.07523220928852e-05
	1790 4.31213557021692e-05
	1791 4.18566924054176e-05
	1792 3.54356707248371e-05
	1793 2.51139899773989e-05
	1794 1.7480340829934e-05
	1795 2.33191803999944e-05
	1796 4.80910166515969e-05
	1797 7.50744657125324e-05
	1798 7.58600581320934e-05
	1799 5.56496961507946e-05
	1800 3.18521451845299e-05
	1801 1.94103249668842e-05
	1802 1.53680357470876e-05
	1803 1.36915468829102e-05
	1804 1.26316235764534e-05
	1805 1.19667483886587e-05
	1806 1.17819326987956e-05
	1807 1.20183958642883e-05
	1808 1.25357682918548e-05
	1809 1.31299466374912e-05
	1810 1.3597699762613e-05
	1811 1.38700779643841e-05
	1812 1.39376306833583e-05
	1813 1.38374089146964e-05
	1814 1.36284143081866e-05
	1815 1.33670000650454e-05
	1816 1.31146334751975e-05
	1817 1.28849442262435e-05
	1818 1.26805971376598e-05
	1819 1.2513364708866e-05
	1820 1.23773570521735e-05
	1821 1.22545734484447e-05
	1822 1.21767379823723e-05
	1823 1.21381472126814e-05
	1824 1.21129132821807e-05
	1825 1.21253706311109e-05
	1826 1.21465791380615e-05
	1827 1.22019609989366e-05
	1828 1.22497303891578e-05
	1829 1.22927212942159e-05
	1830 1.23374638860696e-05
	1831 1.23824229376623e-05
	1832 1.24014013636042e-05
	1833 1.24071875688969e-05
	1834 1.23921499834978e-05
	1835 1.23748422993231e-05
	1836 1.23377540148795e-05
	1837 1.22959736472694e-05
	1838 1.22422288768576e-05
	1839 1.21816510727513e-05
	1840 1.20994263852481e-05
	1841 1.20439735837863e-05
	1842 1.19418809845229e-05
	1843 1.18563939395244e-05
	1844 1.17701820272487e-05
	1845 1.16915234684711e-05
	1846 1.16127894216334e-05
	1847 1.15490147436503e-05
	1848 1.1481569345051e-05
	1849 1.14160320663359e-05
	1850 1.13709538709372e-05
	1851 1.13158166641369e-05
	1852 1.12796096800594e-05
	1853 1.12332800199511e-05
	1854 1.11984736577142e-05
	1855 1.11654408101458e-05
	1856 1.11383869807469e-05
	1857 1.10992268673726e-05
	1858 1.10839964690967e-05
	1859 1.10490473161917e-05
	1860 1.10406381281791e-05
	1861 1.10091086753528e-05
	1862 1.09963220893405e-05
	1863 1.09864422483952e-05
	1864 1.09631755549344e-05
	1865 1.09584470919799e-05
	1866 1.09498450910905e-05
	1867 1.09551356217708e-05
	1868 1.09548855107278e-05
	1869 1.09606417026953e-05
	1870 1.09758584585506e-05
	1871 1.09729580799467e-05
	1872 1.0990410373779e-05
	1873 1.10056726043695e-05
	1874 1.10205610326375e-05
	1875 1.10512764877058e-05
	1876 1.10890159703558e-05
	1877 1.11030813059187e-05
	1878 1.11741919681663e-05
	1879 1.12601737782825e-05
	1880 1.13498608698137e-05
	1881 1.15101829578634e-05
	1882 1.16087294372846e-05
	1883 1.17855306598358e-05
	1884 1.18538309834548e-05
	1885 1.2160926416982e-05
	1886 1.25362948892871e-05
	1887 1.27932735267677e-05
	1888 1.31460219563451e-05
	1889 1.37735851239995e-05
	1890 1.42905200846144e-05
	1891 1.46637312354869e-05
	1892 1.55999605340185e-05
	1893 1.66070421983022e-05
	1894 1.75280965777347e-05
	1895 1.81687282747589e-05
	1896 1.97538756765425e-05
	1897 2.17487304325914e-05
	1898 2.36710875469726e-05
	1899 2.55814702541102e-05
	1900 2.78093830274884e-05
	1901 2.93306347884936e-05
	1902 3.00471056107199e-05
	1903 2.85191417788155e-05
	1904 2.46756717388052e-05
	1905 1.93052746908506e-05
	1906 1.5115472706384e-05
	1907 1.64409266290022e-05
	1908 2.73078785539838e-05
	1909 4.80748021800537e-05
	1910 6.79868171573617e-05
	1911 7.2724767960608e-05
	1912 6.42358500044793e-05
	1913 4.36562404502183e-05
	1914 2.62716457655188e-05
	1915 1.83753727469593e-05
	1916 1.62630039994838e-05
	1917 1.56864643940935e-05
	1918 1.50840251080808e-05
	1919 1.39192934511811e-05
	1920 1.24255002447171e-05
	1921 1.13592441266519e-05
	1922 1.11238041426986e-05
	1923 1.17311537906062e-05
	1924 1.27712428366067e-05
	1925 1.37676515805651e-05
	1926 1.43864745041355e-05
	1927 1.45411922858329e-05
	1928 1.43157385537052e-05
	1929 1.38918667289545e-05
	1930 1.34542224259349e-05
	1931 1.30729104057536e-05
	1932 1.27399534903816e-05
	1933 1.24987745948602e-05
	1934 1.22872143037966e-05
	1935 1.21234470498166e-05
	1936 1.19960186566459e-05
	1937 1.18790840133443e-05
	1938 1.17774725367781e-05
	1939 1.17049203254282e-05
	1940 1.16825867735315e-05
	1941 1.16665505629499e-05
	1942 1.17014496936463e-05
	1943 1.17363824756467e-05
	1944 1.17954396046116e-05
	1945 1.18742955237394e-05
	1946 1.19713276944822e-05
	1947 1.20672775665298e-05
	1948 1.21556568046799e-05
	1949 1.22265646496089e-05
	1950 1.22908086268581e-05
	1951 1.23335239550215e-05
	1952 1.23421732496354e-05
	1953 1.23431946121855e-05
	1954 1.23017598525621e-05
	1955 1.22625124276965e-05
	1956 1.22097280836897e-05
	1957 1.21556968224468e-05
	1958 1.20692056952976e-05
	1959 1.19957649076241e-05
	1960 1.19022261060309e-05
	1961 1.18068637675606e-05
	1962 1.17028248496354e-05
	1963 1.16031924335402e-05
	1964 1.14953181764577e-05
	1965 1.14149315777468e-05
	1966 1.13501646410441e-05
	1967 1.12899506348185e-05
	1968 1.12407360575162e-05
	1969 1.11910503619583e-05
	1970 1.11442732304567e-05
	1971 1.11018116513151e-05
	1972 1.10597457023687e-05
	1973 1.10237997432705e-05
	1974 1.09865422928124e-05
	1975 1.09600150608458e-05
	1976 1.09278907984844e-05
	1977 1.08874874058529e-05
	1978 1.08674839793821e-05
	1979 1.0837542504305e-05
	1980 1.08128124338691e-05
	1981 1.07923224277329e-05
	1982 1.07735349956783e-05
	1983 1.07581035990734e-05
	1984 1.07488540379563e-05
	1985 1.07311434476287e-05
	1986 1.07422420114744e-05
	1987 1.07387977550388e-05
	1988 1.07537307485472e-05
	1989 1.072772374755e-05
	1990 1.07833220681641e-05
	1991 1.07866408143309e-05
	1992 1.07309124359745e-05
	1993 1.08484500742634e-05
	1994 1.08724143501604e-05
	1995 1.08637777884724e-05
	1996 1.09987149699009e-05
	1997 1.10333066913881e-05
	1998 1.11115277832141e-05
	1999 1.12625093606766e-05
};
\addlegendentry{Test}
\end{groupplot}

\end{tikzpicture}

	\end{figure}
\end{center}
\begin{center}
	\begin{figure}[H]
		% This file was created by tikzplotlib v0.9.6.
\begin{tikzpicture}

\begin{groupplot}[
group style={group size=1 by 8},
legend cell align={left},
legend style={fill opacity=1, draw opacity=1, text opacity=1, draw=white},
log basis y={10},
tick align=outside,
tick pos=left,
title style={at={(0.43,0.85)},anchor=north},
x grid style={white!69.0196078431373!black},
xlabel={Epoch},
x label style={yshift=13pt},
xmin=-49.95, xmax=2048.95,
xtick style={color=black},
xtick = {0,500,1500,2000},
y grid style={white!69.0196078431373!black},
ylabel={MSE Loss},
ymode=log,
ytick style={color=black},
width=.45\textwidth,
height=.25\textwidth
]

\nextgroupplot[
title={SiLU/SiLU},
ymin=4.07329674874023e-06, ymax=0.001,
]
\addplot [semithick, black, dashed]
table {%
	0 0.0308093957137316
	1 0.0301686491584405
	2 0.0295290873618796
	3 0.0288900367449969
	4 0.028249450318981
	5 0.0276043486082926
	6 0.026950518542435
	7 0.0262816242175177
	8 0.0255869955872186
	9 0.0248478493886068
	10 0.0240312696550973
	11 0.0230814952519722
	12 0.021916494704783
	13 0.0204523791908287
	14 0.0186814042972401
	15 0.0167554478393868
	16 0.0149231665418483
	17 0.0133348110539373
	18 0.0119904467719607
	19 0.0108503636001842
	20 0.0098807002796093
	21 0.00904987947433256
	22 0.00833139417227358
	23 0.00770417728926986
	24 0.00715168115857523
	25 0.00666101573733613
	26 0.00622215391194914
	27 0.00582723008847097
	28 0.00546998218487715
	29 0.00514535387628712
	30 0.00484920166491065
	31 0.00457808115606895
	32 0.00432909580558771
	33 0.00409978243988007
	34 0.0038880297979631
	35 0.00369201690045884
	36 0.00351016487547895
	37 0.00334109805407934
	38 0.00318361332756467
	39 0.00303665242245188
	40 0.00289928069105372
	41 0.00277066983835539
	42 0.00265008191490779
	43 0.00253685761344968
	44 0.00243040564237162
	45 0.00233019556617364
	46 0.00223574812116567
	47 0.00214663098813617
	48 0.00206245279878203
	49 0.00198285859551106
	50 0.00190752557682572
	51 0.00183615968308004
	52 0.00176849323725037
	53 0.00170428115234245
	54 0.00164329988911049
	55 0.00158534385809617
	56 0.00153022429003613
	57 0.0014777679953113
	58 0.00142781492650101
	59 0.0013802172816213
	60 0.00133483805439027
	61 0.00129155028480454
	62 0.00125023600412533
	63 0.00121078469783242
	64 0.00117309368943097
	65 0.00113706663069024
	66 0.00110261347253982
	67 0.00106964921906183
	68 0.00103809425309009
	69 0.00100787391147605
	70 0.000978917381871725
	71 0.000951158293901244
	72 0.000924533799661731
	73 0.000898984952073079
	74 0.000874455835401022
	75 0.000850894050927309
	76 0.000828250250378915
	77 0.00080647800132283
	78 0.000785533873568056
	79 0.0007653770107936
	80 0.000745969347008213
	81 0.000727275230929081
	82 0.000709261758856883
	83 0.00069189817077131
	84 0.000675156250508735
	85 0.00065900951904041
	86 0.000643433771983837
	87 0.000628406727628317
	88 0.000613907865954388
	89 0.000599917997533339
	90 0.000586419459068566
	91 0.000573395963328949
	92 0.000560832248993393
	93 0.000548714013348217
	94 0.000537027872269391
	95 0.000525761021435756
	96 0.000514901222231856
	97 0.000504436601659108
	98 0.000494355865157559
	99 0.000484647702251095
	100 0.000475301139886142
	101 0.000466305289592128
	102 0.000457649258805759
	103 0.000449322274107544
	104 0.000441313479768723
	105 0.000433612142387574
	106 0.000426207419877755
	107 0.000419088651597121
	108 0.000412244997278322
	109 0.000405665818561829
	110 0.000399340543026483
	111 0.000393258679650899
	112 0.000387409885661327
	113 0.000381784004275687
	114 0.00037637108061972
	115 0.000371161378552642
	116 0.000366145509360649
	117 0.000361314236670296
	118 0.000356658722012071
	119 0.000352170343376201
	120 0.000347840895983609
	121 0.000343662434261205
	122 0.000339627379617014
	123 0.000335728527034007
	124 0.000331958953893263
	125 0.000328312160036148
	126 0.000324781830840948
	127 0.000321362090062394
	128 0.000318047224482143
	129 0.000314832054527869
	130 0.000311711468611975
	131 0.000308680734406153
	132 0.000305735406527674
	133 0.00030287119670902
	134 0.000300084147056623
	135 0.000297370499765748
	136 0.000294726647553034
	137 0.000292149259394137
	138 0.000289635152967094
	139 0.000287181371732004
	140 0.000284785010762789
	141 0.000282443450259962
	142 0.000280154168876834
	143 0.000277914745993257
	144 0.000275722929700351
	145 0.000273576592690006
	146 0.000271473718612469
	147 0.000269412340003328
	148 0.000267390651288224
	149 0.00026540691669652
	150 0.000263459553934808
	151 0.000261546949104741
	152 0.000259667629052274
	153 0.000257820204978998
	154 0.000256003399499605
	155 0.000254215898621624
	156 0.000252456546377289
	157 0.000250724185661966
	158 0.000249017781015937
	159 0.000247336319489477
	160 0.000245678890223644
	161 0.000244044494820628
	162 0.000242432400000325
	163 0.000240841702066064
	164 0.00023927171048399
	165 0.000237721752114339
	166 0.000236191114936446
	167 0.000234679148775285
	168 0.000233185379215683
	169 0.000231709169725036
	170 0.000230250091703965
	171 0.00022880757785515
	172 0.000227381267450255
	173 0.000225970696988043
	174 0.000224575548372741
	175 0.000223195456101166
	176 0.000221829988959144
	177 0.000220478957771775
	178 0.000219142022388041
	179 0.000217818909050038
	180 0.000216509384927122
	181 0.000215213221963495
	182 0.00021393017959781
	183 0.000212660020224575
	184 0.000211402644822556
	185 0.000210157747005724
	186 0.000208925219340017
	187 0.000207704917784213
	188 0.00020649662485539
	189 0.000205300152288146
	190 0.000204115439032648
	191 0.000202942264536432
	192 0.000201780536485785
	193 0.00020063007968929
	194 0.000199490805925961
	195 0.000198362515561712
	196 0.000197245083882081
	197 0.000196138423461889
	198 0.00019504237639012
	199 0.000193956793737016
	200 0.000192881580233006
	201 0.000191816598203332
	202 0.000190761691783337
	203 0.000189716784348093
	204 0.000188681712870675
	205 0.000187656346383847
	206 0.000186640571428143
	207 0.000185634267040768
	208 0.000184637247343744
	209 0.00018364944367022
	210 0.000182670747108205
	211 0.000181700935286244
	212 0.000180739978191014
	213 0.000179787717002
	214 0.000178843990397581
	215 0.000177908668320015
	216 0.000176981640720442
	217 0.000176062817217826
	218 0.000175151967027887
	219 0.000174249080487243
	220 0.000173353971604229
	221 0.000172466504864133
	222 0.000171586581132033
	223 0.000170714053069787
	224 0.000169848777147763
	225 0.00016899067122722
	226 0.000168139576317117
	227 0.000167295388791899
	228 0.000166457902707862
	229 0.000165627088051679
	230 0.000164802806239095
	231 0.000163984893447378
	232 0.000163173232238023
	233 0.000162367737686964
	234 0.000161568260466538
	235 0.000160774706898792
	236 0.000159986857681815
	237 0.000159204708552352
	238 0.000158428064423788
	239 0.000157656837529885
	240 0.000156890907987872
	241 0.000156130116806708
	242 0.000155374415328424
	243 0.000154623654339048
	244 0.000153877671664304
	245 0.000153136391020325
	246 0.000152399730666275
	247 0.000151667535192246
	248 0.000150939685852336
	249 0.000150216082886345
	250 0.000149496645292402
	251 0.000148781194297953
	252 0.000148069714839494
	253 0.000147361986307715
	254 0.000146657975733433
	255 0.000145957581537459
	256 0.000145260684348614
	257 0.000144567139500396
	258 0.000143876880940752
	259 0.000143189829771018
	260 0.000142505834617168
	261 0.000141824853983508
	262 0.000141146758096511
	263 0.000140471452027668
	264 0.000139798807992975
	265 0.000139128795922261
	266 0.000138461315145832
	267 0.000137796211333807
	268 0.000137133488635754
	269 0.000136473013242266
	270 0.000135814638213105
	271 0.000135158384523493
	272 0.000134504138259217
	273 0.000133851785165007
	274 0.000133201286587337
	275 0.000132552566299182
	276 0.000131905519026532
	277 0.000131260065643346
	278 0.000130616173748876
	279 0.000129973786556548
	280 0.000129332797939696
	281 0.000128693150088566
	282 0.000128054831691315
	283 0.000127417715646061
	284 0.000126781789504093
	285 0.0001261470158056
	286 0.000125513278078415
	287 0.00012488058331428
	288 0.000124248873362376
	289 0.000123618113775592
	290 0.000122988261750834
	291 0.000122359290742224
	292 0.00012173115845826
	293 0.000121103835965641
	294 0.000120477309394573
	295 0.000119851532360826
	296 0.00011922649122198
	297 0.000118602208374341
	298 0.000117978684045283
	299 0.000117355880661307
	300 0.000116733768948052
	301 0.000116112383352629
	302 0.000115491749454577
	303 0.000114871841276454
	304 0.000114252715661678
	305 0.000113634382898908
	306 0.000113016863110715
	307 0.000112400166074167
	308 0.000111784352839095
	309 0.00011116945898948
	310 0.000110555580249638
	311 0.000109942669666907
	312 0.000109330875716296
	313 0.000108720225114212
	314 0.00010811076168693
	315 0.000107502591447428
	316 0.000106895727185474
	317 0.000106290334713321
	318 0.000105686440122099
	319 0.000105084144934153
	320 0.000104483515769971
	321 0.000103884660632048
	322 0.000103287686897602
	323 0.000102692633959123
	324 0.000102099657397048
	325 0.000101508788986848
	326 0.000100920191584919
	327 0.000100333886791759
	328 9.97500050630151e-05
	329 9.91686567317629e-05
	330 9.85899231977783e-05
	331 9.80138966610866e-05
	332 9.74406210616507e-05
	333 9.68702481998207e-05
	334 9.6302837789608e-05
	335 9.57384176558662e-05
	336 9.51771652353273e-05
	337 9.4619090617698e-05
	338 9.40642688362914e-05
	339 9.3512785412031e-05
	340 9.29646918734761e-05
	341 9.24200757879134e-05
	342 9.18789649801965e-05
	343 9.13414384342559e-05
	344 9.08075313077461e-05
	345 9.02773267057455e-05
	346 8.97508425623528e-05
	347 8.92280991422467e-05
	348 8.87091840127141e-05
	349 8.81940886472421e-05
	350 8.76828892444337e-05
	351 8.71755719913381e-05
	352 8.66722060379743e-05
	353 8.61728003371809e-05
	354 8.56773920645537e-05
	355 8.51859396107102e-05
	356 8.46985060434235e-05
	357 8.42150991786639e-05
	358 8.37357177942977e-05
	359 8.32603915341679e-05
	360 8.27890957850741e-05
	361 8.23218383914082e-05
	362 8.18586574382607e-05
	363 8.13995357873409e-05
	364 8.09444094898026e-05
	365 8.04933380322836e-05
	366 8.00462903214338e-05
	367 7.96032903735977e-05
	368 7.91643005868536e-05
	369 7.87293040502846e-05
	370 7.82982453699788e-05
	371 7.78712011708649e-05
	372 7.74480827772095e-05
	373 7.70288985449952e-05
	374 7.66136456604727e-05
	375 7.62022798710404e-05
	376 7.57947916270041e-05
	377 7.53911187985068e-05
	378 7.49912788648999e-05
	379 7.45952061151911e-05
	380 7.42029314153569e-05
	381 7.38144147476305e-05
	382 7.34296150994851e-05
	383 7.30484806581444e-05
	384 7.2670981239753e-05
	385 7.22970966080538e-05
	386 7.19268325042322e-05
	387 7.15601575507208e-05
	388 7.11969745736951e-05
	389 7.08373063105228e-05
	390 7.04811233447344e-05
	391 7.01283753699045e-05
	392 6.97790077595073e-05
	393 6.94330082069428e-05
	394 6.9090366821456e-05
	395 6.87510019758975e-05
	396 6.8414936578165e-05
	397 6.80821055993874e-05
	398 6.77524703860399e-05
	399 6.74259903803431e-05
	400 6.71026490124405e-05
	401 6.67824136542094e-05
	402 6.6465224648482e-05
	403 6.61511038231311e-05
	404 6.58399771396034e-05
	405 6.55318072801947e-05
	406 6.52265561598142e-05
	407 6.49241974315373e-05
	408 6.46247113991194e-05
	409 6.43280876886365e-05
	410 6.40342431381669e-05
	411 6.37431451480097e-05
	412 6.34547839126753e-05
	413 6.31691469266116e-05
	414 6.28861646987389e-05
	415 6.26058100010596e-05
	416 6.23280715785768e-05
	417 6.20529282002735e-05
	418 6.17803013369667e-05
	419 6.15101909886562e-05
	420 6.12425732811062e-05
	421 6.09773999826757e-05
	422 6.07146310187545e-05
	423 6.04542746316383e-05
	424 6.01962770190312e-05
	425 5.99405955483689e-05
	426 5.96872356481981e-05
	427 5.94361639514318e-05
	428 5.91873598523307e-05
	429 5.89407986240076e-05
	430 5.86963943192131e-05
	431 5.84541590740173e-05
	432 5.8214085555619e-05
	433 5.79761343146856e-05
	434 5.77402501562574e-05
	435 5.75064715633289e-05
	436 5.7274698974652e-05
	437 5.7044935488193e-05
	438 5.68172020791735e-05
	439 5.65914184562644e-05
	440 5.63675748992409e-05
	441 5.61456532892635e-05
	442 5.59256445171741e-05
	443 5.57075244245198e-05
	444 5.54912710271083e-05
	445 5.52768403281334e-05
	446 5.50642051990735e-05
	447 5.48533642188431e-05
	448 5.46443001780972e-05
	449 5.4436972845906e-05
	450 5.42313803038041e-05
	451 5.40275084688346e-05
	452 5.38253184316773e-05
	453 5.36247722777716e-05
	454 5.34259251168123e-05
	455 5.3228675938044e-05
	456 5.30330632244613e-05
	457 5.28389892480163e-05
	458 5.26465189096825e-05
	459 5.24556153465028e-05
	460 5.22662489856884e-05
	461 5.20783750346254e-05
	462 5.18920138716794e-05
	463 5.17071286196824e-05
	464 5.15237168627891e-05
	465 5.13417395637816e-05
	466 5.11612222027225e-05
	467 5.09820857530485e-05
	468 5.08043561779914e-05
	469 5.06279980356794e-05
	470 5.04530007674475e-05
	471 5.02793498498022e-05
	472 5.01070335729992e-05
	473 4.99360086081424e-05
	474 4.97663235705659e-05
	475 4.95978876955405e-05
	476 4.94307375902281e-05
	477 4.92647857157635e-05
	478 4.91001153903881e-05
	479 4.89366472038455e-05
	480 4.87743728143641e-05
	481 4.86132972383757e-05
	482 4.84533828313261e-05
	483 4.82946127959849e-05
	484 4.81370135503312e-05
	485 4.79805533046829e-05
	486 4.78251905917659e-05
	487 4.76708937640069e-05
	488 4.75177091772139e-05
	489 4.73655849333454e-05
	490 4.72144860594881e-05
	491 4.7064458513546e-05
	492 4.69154569628927e-05
	493 4.67674665998175e-05
	494 4.66204955387184e-05
	495 4.64744879593582e-05
	496 4.63294474286613e-05
	497 4.61853587836458e-05
	498 4.60422462680299e-05
	499 4.59000746815263e-05
	500 4.57588175777346e-05
	501 4.56184581736352e-05
	502 4.54789983734827e-05
	503 4.53404144309388e-05
	504 4.52027294812751e-05
	505 4.50658898927259e-05
	506 4.4929909563507e-05
	507 4.47947585229258e-05
	508 4.46604542645446e-05
	509 4.45269385807023e-05
	510 4.43942321766144e-05
	511 4.42623427971967e-05
	512 4.41312418217876e-05
	513 4.40008717248475e-05
	514 4.38712735615354e-05
	515 4.37424409511777e-05
	516 4.36143464384031e-05
	517 4.34869978107599e-05
	518 4.33603584468756e-05
	519 4.32344519509797e-05
	520 4.31092390158483e-05
	521 4.29847037679565e-05
	522 4.28608724121204e-05
	523 4.2737723049413e-05
	524 4.26152303418803e-05
	525 4.2493396861687e-05
	526 4.23722134996751e-05
	527 4.22516691713781e-05
	528 4.21317851504455e-05
	529 4.2012462955654e-05
	530 4.18937929254071e-05
	531 4.17757211295111e-05
	532 4.16582420683653e-05
	533 4.15413472438786e-05
	534 4.14250499147784e-05
	535 4.1309332928563e-05
	536 4.11941713309716e-05
	537 4.10795731369262e-05
	538 4.09655309141499e-05
	539 4.08520244548072e-05
	540 4.07390467671576e-05
	541 4.06266214980633e-05
	542 4.05147128219596e-05
	543 4.04033147702876e-05
	544 4.02924289772955e-05
	545 4.01820536524156e-05
	546 4.00721978195406e-05
	547 3.9962786033243e-05
	548 3.9853894648445e-05
	549 3.97454754903492e-05
	550 3.96375220930167e-05
	551 3.95300432955992e-05
	552 3.94230313816024e-05
	553 3.93164610414942e-05
	554 3.92103578263914e-05
	555 3.91046873602363e-05
	556 3.89994607843391e-05
	557 3.88946650389244e-05
	558 3.87903139937862e-05
	559 3.86863588772712e-05
	560 3.85828363533847e-05
	561 3.84797134529435e-05
	562 3.83770079821488e-05
	563 3.82746908655918e-05
	564 3.8172774466716e-05
	565 3.80712479568501e-05
	566 3.79701199477722e-05
	567 3.78693497822269e-05
	568 3.77689824517802e-05
	569 3.7668977171279e-05
	570 3.75693332870242e-05
	571 3.74700741190281e-05
	572 3.73711867638349e-05
	573 3.72726314736838e-05
	574 3.71744558123055e-05
	575 3.70766358486208e-05
	576 3.69791414982501e-05
	577 3.68820449665463e-05
	578 3.67852633900156e-05
	579 3.66888746441418e-05
	580 3.65928155900974e-05
	581 3.64971253503654e-05
	582 3.64018163736546e-05
	583 3.63068737954109e-05
	584 3.62122770809492e-05
	585 3.61181123764709e-05
	586 3.60243325729925e-05
	587 3.59309763524607e-05
	588 3.5838045604919e-05
	589 3.57455953832186e-05
	590 3.56536070142965e-05
	591 3.55621263139483e-05
	592 3.54712393857426e-05
	593 3.53809168700536e-05
	594 3.52912462346922e-05
	595 3.5202297240744e-05
	596 3.51140915313408e-05
	597 3.50267458770759e-05
	598 3.49403537285298e-05
	599 3.48549846194146e-05
	600 3.47707549366305e-05
	601 3.46878222501346e-05
	602 3.46063804386176e-05
	603 3.45265540317996e-05
	604 3.44486171002245e-05
	605 3.43728011813482e-05
	606 3.42994715367695e-05
	607 3.42290155828096e-05
	608 3.41619531241122e-05
	609 3.40989461307117e-05
	610 3.40408554109217e-05
	611 3.39888058249471e-05
	612 3.39443138841489e-05
	613 3.3909408927002e-05
	614 3.38869839566769e-05
	615 3.38811002364992e-05
	616 3.38974125639879e-05
	617 3.39440578045469e-05
	618 3.40325934331531e-05
	619 3.41791866418362e-05
	620 3.44052934622141e-05
	621 3.47357003249726e-05
	622 3.5188829158983e-05
	623 3.57501394887549e-05
	624 3.63222839609989e-05
	625 3.66827360878119e-05
	626 3.65507478861105e-05
	627 3.58251008165666e-05
	628 3.47565541716222e-05
	629 3.37500481109032e-05
	630 3.30349986086276e-05
	631 3.26056359085669e-05
	632 3.23601991993883e-05
	633 3.22079292587318e-05
	634 3.20952580707967e-05
	635 3.19967807271837e-05
	636 3.19026491553132e-05
	637 3.18102491831951e-05
	638 3.17196534282971e-05
	639 3.1631548360167e-05
	640 3.15464652018704e-05
	641 3.14645173915551e-05
	642 3.13855066806923e-05
	643 3.13091103691931e-05
	644 3.12348641173799e-05
	645 3.11623626174651e-05
	646 3.10912300705013e-05
	647 3.1021176020829e-05
	648 3.09519516861201e-05
	649 3.08834019548954e-05
	650 3.08153982402359e-05
	651 3.07478635050984e-05
	652 3.06807385754837e-05
	653 3.06139946246731e-05
	654 3.05476411526229e-05
	655 3.04816636571559e-05
	656 3.04161201185593e-05
	657 3.03510243924165e-05
	658 3.02864377417222e-05
	659 3.02223998929207e-05
	660 3.01590299756072e-05
	661 3.00963789214848e-05
	662 3.00345735126939e-05
	663 2.99737432882807e-05
	664 2.99140344992566e-05
	665 2.98556100020164e-05
	666 2.97986807709094e-05
	667 2.97434673157682e-05
	668 2.96902421439427e-05
	669 2.96392748850849e-05
	670 2.95909186860399e-05
	671 2.95455616807772e-05
	672 2.95035931401344e-05
	673 2.94654418055984e-05
	674 2.94315976603343e-05
	675 2.94024722293784e-05
	676 2.93785090832444e-05
	677 2.93600525083093e-05
	678 2.93472542054474e-05
	679 2.93400182158621e-05
	680 2.93379029585594e-05
	681 2.93398087407581e-05
	682 2.93439494711834e-05
	683 2.93476422683625e-05
	684 2.93470832559706e-05
	685 2.93375057864864e-05
	686 2.93135588051996e-05
	687 2.92698030861516e-05
	688 2.92017943195333e-05
	689 2.91070357434364e-05
	690 2.89859850113316e-05
	691 2.88421217575774e-05
	692 2.86817009111928e-05
	693 2.85124742731568e-05
	694 2.83423393412363e-05
	695 2.81781995425945e-05
	696 2.80252304065698e-05
	697 2.78868388150499e-05
	698 2.77648574069644e-05
	699 2.76599818889167e-05
	700 2.75723039564468e-05
	701 2.75015471444817e-05
	702 2.7447504912459e-05
	703 2.74100387045451e-05
	704 2.73889594453181e-05
	705 2.73839906910212e-05
	706 2.7394506517453e-05
	707 2.74190094273763e-05
	708 2.74547685350512e-05
	709 2.74970892419901e-05
	710 2.75391865756092e-05
	711 2.75718943996139e-05
	712 2.75846441795125e-05
	713 2.75668676366081e-05
	714 2.75107479765779e-05
	715 2.74135737896586e-05
	716 2.72791716966481e-05
	717 2.71172400729824e-05
	718 2.69408280715311e-05
	719 2.67628750449944e-05
	720 2.65938284016443e-05
	721 2.64401764198396e-05
	722 2.63048764779228e-05
	723 2.61881668564001e-05
	724 2.60887539695887e-05
	725 2.60048309641547e-05
	726 2.59345293969204e-05
	727 2.58762922200617e-05
	728 2.58289843415582e-05
	729 2.57919039228227e-05
	730 2.57647753372225e-05
	731 2.57476054983385e-05
	732 2.57405829771074e-05
	733 2.57440352697813e-05
	734 2.57581420797237e-05
	735 2.57827687377699e-05
	736 2.58172304441473e-05
	737 2.58597844293718e-05
	738 2.59074726542963e-05
	739 2.59555111767895e-05
	740 2.59972631582173e-05
	741 2.60243509799807e-05
	742 2.60276027717055e-05
	743 2.59985989217171e-05
	744 2.59316463768755e-05
	745 2.58257477767643e-05
	746 2.56852771514104e-05
	747 2.55195005465225e-05
	748 2.53402200698361e-05
	749 2.51591061370959e-05
	750 2.49855810992017e-05
	751 2.48258051414041e-05
	752 2.46827728531684e-05
	753 2.45569012733426e-05
	754 2.44470215235992e-05
	755 2.43512157069858e-05
	756 2.42672604855443e-05
	757 2.41930177509175e-05
	758 2.41266433818055e-05
	759 2.40664875477137e-05
	760 2.40112210363463e-05
	761 2.39597756532817e-05
	762 2.39112382161011e-05
	763 2.38649252466416e-05
	764 2.38202121636277e-05
	765 2.37766164374875e-05
	766 2.37337341815191e-05
	767 2.36911842677046e-05
	768 2.36485971214506e-05
	769 2.36057993205918e-05
	770 2.35624310533922e-05
	771 2.35183153947105e-05
	772 2.34731927974963e-05
	773 2.34269178847057e-05
	774 2.33793253983094e-05
	775 2.33302822465475e-05
	776 2.32796781602929e-05
	777 2.32275005771498e-05
	778 2.31736826421525e-05
	779 2.3118283365875e-05
	780 2.30613733194218e-05
	781 2.30030663530556e-05
	782 2.29435415803891e-05
	783 2.28830332744678e-05
	784 2.28217638920114e-05
	785 2.27600872335643e-05
	786 2.2698322453607e-05
	787 2.263695937188e-05
	788 2.25764716219601e-05
	789 2.25174959993524e-05
	790 2.24606968615149e-05
	791 2.24069008183392e-05
	792 2.23571752258067e-05
	793 2.23126836971232e-05
	794 2.22749675629075e-05
	795 2.22458294629746e-05
	796 2.22275119341475e-05
	797 2.22227876633951e-05
	798 2.22351283838407e-05
	799 2.22689351048189e-05
	800 2.23297929551336e-05
	801 2.24249707443391e-05
	802 2.25636381117056e-05
	803 2.27571777671187e-05
	804 2.3018001940045e-05
	805 2.33555871247404e-05
	806 2.37671169998066e-05
	807 2.42186494432417e-05
	808 2.462222294497e-05
	809 2.48311527641931e-05
	810 2.46970925914525e-05
	811 2.41853096909495e-05
	812 2.34398645702072e-05
	813 2.26925610675721e-05
	814 2.21072184771742e-05
	815 2.17218584595003e-05
	816 2.14954247539367e-05
	817 2.13692621002792e-05
	818 2.12966040322726e-05
	819 2.1247683704928e-05
	820 2.1205727861684e-05
	821 2.11624556669676e-05
	822 2.11145347606134e-05
	823 2.10614736317893e-05
	824 2.10042061041804e-05
	825 2.09442845999774e-05
	826 2.08832406016768e-05
	827 2.08223521838136e-05
	828 2.07626113137849e-05
	829 2.07045711064779e-05
	830 2.06485682809898e-05
	831 2.05947053188993e-05
	832 2.05429283894887e-05
	833 2.04931234648598e-05
	834 2.04450949397028e-05
	835 2.03987274716155e-05
	836 2.03538369305534e-05
	837 2.03102977138769e-05
	838 2.02680397336508e-05
	839 2.02269570337421e-05
	840 2.01870208229593e-05
	841 2.01482942046027e-05
	842 2.01107685526836e-05
	843 2.00745389946633e-05
	844 2.00397138954145e-05
	845 2.00064503204089e-05
	846 1.99749671523364e-05
	847 1.99455085621025e-05
	848 1.99183595555041e-05
	849 1.98938730449072e-05
	850 1.98724842519482e-05
	851 1.98546692402601e-05
	852 1.98409386982235e-05
	853 1.98318770401329e-05
	854 1.98281045058479e-05
	855 1.98303144642864e-05
	856 1.9839039055114e-05
	857 1.98548104108909e-05
	858 1.98779251974202e-05
	859 1.99083987126869e-05
	860 1.99456464926584e-05
	861 1.99882474234414e-05
	862 2.00338214213502e-05
	863 2.00785898556433e-05
	864 2.0117325803426e-05
	865 2.01435696425278e-05
	866 2.01498991145854e-05
	867 2.01294814843322e-05
	868 2.00772773553126e-05
	869 1.9991523835472e-05
	870 1.98749228701445e-05
	871 1.97341474077461e-05
	872 1.95788273060771e-05
	873 1.94194620064536e-05
	874 1.92655706499067e-05
	875 1.91243668865582e-05
	876 1.90005455280584e-05
	877 1.88964067362463e-05
	878 1.88127276743444e-05
	879 1.87492871432937e-05
	880 1.87054752913696e-05
	881 1.86805882478325e-05
	882 1.86740159264787e-05
	883 1.86852094685719e-05
	884 1.87133536684314e-05
	885 1.87572205021524e-05
	886 1.88144681274593e-05
	887 1.88812747552447e-05
	888 1.89517706914444e-05
	889 1.90178106151961e-05
	890 1.90692234909307e-05
	891 1.90951505771864e-05
	892 1.90860804067938e-05
	893 1.90364768926088e-05
	894 1.89466424558304e-05
	895 1.88233943916316e-05
	896 1.86782200302105e-05
	897 1.85242581522971e-05
	898 1.83734051617535e-05
	899 1.82341581833612e-05
	900 1.81111678259072e-05
	901 1.80058968979324e-05
	902 1.7917699331349e-05
	903 1.78448444785317e-05
	904 1.77852518135069e-05
	905 1.77370175578062e-05
	906 1.76985234219273e-05
	907 1.76685966195578e-05
	908 1.76464960972567e-05
	909 1.76317765436806e-05
	910 1.76243874108195e-05
	911 1.76244586995722e-05
	912 1.76323898628539e-05
	913 1.76486767458073e-05
	914 1.76737778403435e-05
	915 1.77081386638633e-05
	916 1.77518631971907e-05
	917 1.78045206880029e-05
	918 1.78648670754455e-05
	919 1.79305161580601e-05
	920 1.79974747496203e-05
	921 1.80600071217896e-05
	922 1.81106376189177e-05
	923 1.81406505106452e-05
	924 1.8141544494199e-05
	925 1.81066902840143e-05
	926 1.80333921164788e-05
	927 1.79240952675741e-05
	928 1.77863709751591e-05
	929 1.76311344048941e-05
	930 1.74701615023309e-05
	931 1.73139560715185e-05
	932 1.7169989263266e-05
	933 1.70426237602328e-05
	934 1.6933342422476e-05
	935 1.68418272181725e-05
	936 1.67666163832791e-05
	937 1.67057789042246e-05
	938 1.66574071300829e-05
	939 1.66197481092922e-05
	940 1.65913345284707e-05
	941 1.65710112085549e-05
	942 1.65577449422472e-05
	943 1.65507740845783e-05
	944 1.65493955250895e-05
	945 1.65528930224923e-05
	946 1.65604752524473e-05
	947 1.65712421065223e-05
	948 1.65840039798582e-05
	949 1.65973000321173e-05
	950 1.66093594060612e-05
	951 1.66181398952858e-05
	952 1.66214289691879e-05
	953 1.66170535251808e-05
	954 1.66030035444464e-05
	955 1.65777852814131e-05
	956 1.65406891312614e-05
	957 1.64918315590512e-05
	958 1.64322871469835e-05
	959 1.63639386840941e-05
	960 1.62892361856848e-05
	961 1.62109003554178e-05
	962 1.61315800824013e-05
	963 1.60536726028226e-05
	964 1.59790975331475e-05
	965 1.59093439435765e-05
	966 1.58455046275208e-05
	967 1.57883171780782e-05
	968 1.57383632526376e-05
	969 1.56961440822556e-05
	970 1.56622279234853e-05
	971 1.5637395627266e-05
	972 1.56226293412942e-05
	973 1.56193214877476e-05
	974 1.56293264623741e-05
	975 1.56550643666264e-05
	976 1.5699541016545e-05
	977 1.57665384250549e-05
	978 1.58604473909918e-05
	979 1.5986170986082e-05
	980 1.61483510510152e-05
	981 1.63501548655631e-05
	982 1.65903132156586e-05
	983 1.6858261794539e-05
	984 1.7128179685244e-05
	985 1.73541790289278e-05
	986 1.74743545429124e-05
	987 1.74308449132354e-05
	988 1.72021550213231e-05
	989 1.68253246499717e-05
	990 1.63828612045336e-05
	991 1.5961098526418e-05
	992 1.56146479426411e-05
	993 1.53588789331138e-05
	994 1.51834974317921e-05
	995 1.5068692093223e-05
	996 1.49951897654432e-05
	997 1.49478847220053e-05
	998 1.49161520752727e-05
	999 1.48929436889489e-05
	1000 1.48735991913895e-05
	1001 1.4855356923249e-05
	1002 1.48364919851929e-05
	1003 1.48160767921013e-05
	1004 1.47937117347396e-05
	1005 1.47693018419659e-05
	1006 1.4743012663132e-05
	1007 1.47150530835916e-05
	1008 1.46857187814931e-05
	1009 1.46553385249604e-05
	1010 1.46242035867772e-05
	1011 1.45925866590346e-05
	1012 1.45607318771113e-05
	1013 1.45288122297416e-05
	1014 1.44970202526906e-05
	1015 1.44654554361523e-05
	1016 1.44342391408259e-05
	1017 1.44034712050711e-05
	1018 1.43732193080837e-05
	1019 1.43435447341744e-05
	1020 1.43145005964129e-05
	1021 1.42861946770267e-05
	1022 1.42587056402022e-05
	1023 1.42321865084227e-05
	1024 1.42067165924686e-05
	1025 1.41825255539629e-05
	1026 1.41598586012037e-05
	1027 1.41390019194887e-05
	1028 1.41203540948709e-05
	1029 1.41044262917944e-05
	1030 1.40918387643296e-05
	1031 1.40834279918067e-05
	1032 1.40801798167445e-05
	1033 1.40834184563232e-05
	1034 1.40947555777871e-05
	1035 1.41162236033665e-05
	1036 1.41504597834796e-05
	1037 1.42006554000318e-05
	1038 1.42709197987756e-05
	1039 1.4366172820246e-05
	1040 1.4492277955469e-05
	1041 1.46557198590358e-05
	1042 1.48624294098454e-05
	1043 1.51153503722412e-05
	1044 1.54094050799358e-05
	1045 1.57235786630849e-05
	1046 1.60117457497222e-05
	1047 1.62003130341759e-05
	1048 1.62068175768582e-05
	1049 1.59827864081308e-05
	1050 1.55570252573511e-05
	1051 1.5032732619602e-05
	1052 1.45311989072638e-05
	1053 1.41328887579562e-05
	1054 1.38608145832109e-05
	1055 1.36995845352317e-05
	1056 1.36202088327764e-05
	1057 1.35949229118637e-05
	1058 1.36017232321706e-05
	1059 1.3624343701224e-05
	1060 1.36509883077451e-05
	1061 1.3673143094195e-05
	1062 1.36849792156113e-05
	1063 1.36830743358018e-05
	1064 1.36662741141436e-05
	1065 1.36353301058989e-05
	1066 1.35925540654114e-05
	1067 1.35410946384695e-05
	1068 1.34843629062686e-05
	1069 1.34254661006139e-05
	1070 1.3367008421028e-05
	1071 1.33107927524634e-05
	1072 1.32579383667064e-05
	1073 1.32089809099512e-05
	1074 1.31640205331962e-05
	1075 1.31229440292202e-05
	1076 1.30854364037702e-05
	1077 1.30511342426587e-05
	1078 1.30196435890184e-05
	1079 1.29906284094261e-05
	1080 1.29638342443172e-05
	1081 1.29389566936311e-05
	1082 1.29158835626697e-05
	1083 1.28944951853782e-05
	1084 1.28747151961761e-05
	1085 1.28566003070318e-05
	1086 1.28402214052414e-05
	1087 1.28257248519503e-05
	1088 1.28133752497206e-05
	1089 1.28035176309993e-05
	1090 1.27965817959819e-05
	1091 1.27931168165674e-05
	1092 1.27938703968766e-05
	1093 1.27997526604418e-05
	1094 1.28118882045669e-05
	1095 1.28316504550696e-05
	1096 1.28607488782961e-05
	1097 1.29011840606097e-05
	1098 1.29553099981194e-05
	1099 1.30258210937484e-05
	1100 1.31155749727441e-05
	1101 1.32273500703661e-05
	1102 1.33631233545373e-05
	1103 1.35229405273662e-05
	1104 1.3703024738021e-05
	1105 1.38931193234271e-05
	1106 1.40737452767326e-05
	1107 1.42152044411148e-05
	1108 1.42812877079734e-05
	1109 1.42397861218058e-05
	1110 1.4077737354512e-05
	1111 1.38126644486647e-05
	1112 1.34889636953517e-05
	1113 1.31605060644802e-05
	1114 1.28709839692931e-05
	1115 1.26439638208353e-05
	1116 1.24846127675937e-05
	1117 1.23868404067196e-05
	1118 1.23401881175766e-05
	1119 1.23338248663174e-05
	1120 1.23580417614733e-05
	1121 1.24043387117467e-05
	1122 1.24648197434851e-05
	1123 1.25316313450696e-05
	1124 1.2596727547276e-05
	1125 1.26519310583717e-05
	1126 1.26895441070474e-05
	1127 1.27035722528035e-05
	1128 1.26907349304872e-05
	1129 1.26511542539731e-05
	1130 1.25885151902594e-05
	1131 1.2508982116799e-05
	1132 1.24199058078034e-05
	1133 1.23282472408448e-05
	1134 1.22396460255914e-05
	1135 1.2157833054971e-05
	1136 1.2084741264573e-05
	1137 1.20209679330685e-05
	1138 1.19661933268844e-05
	1139 1.19195686600904e-05
	1140 1.1880107116724e-05
	1141 1.18468104872704e-05
	1142 1.18187905400191e-05
	1143 1.17953862215359e-05
	1144 1.17760339826134e-05
	1145 1.17604385074799e-05
	1146 1.17484279407165e-05
	1147 1.17400541057577e-05
	1148 1.17355050193169e-05
	1149 1.17351634578711e-05
	1150 1.17396196515074e-05
	1151 1.17495997855599e-05
	1152 1.17661227072574e-05
	1153 1.1790430793468e-05
	1154 1.18240165143391e-05
	1155 1.18686237051691e-05
	1156 1.19262002371556e-05
	1157 1.19988456965814e-05
	1158 1.20885524061976e-05
	1159 1.21967844251003e-05
	1160 1.23239065068503e-05
	1161 1.24680209943051e-05
	1162 1.26235007336106e-05
	1163 1.27792047379671e-05
	1164 1.29175743595056e-05
	1165 1.30149215422648e-05
	1166 1.30457330946854e-05
	1167 1.29905979555645e-05
	1168 1.28452365117937e-05
	1169 1.26254057519759e-05
	1170 1.23631880342145e-05
	1171 1.20958693017315e-05
	1172 1.18543325591247e-05
	1173 1.16570056327703e-05
	1174 1.15100648443445e-05
	1175 1.14114855271907e-05
	1176 1.13553699208069e-05
	1177 1.13347946317788e-05
	1178 1.13433141351038e-05
	1179 1.13752291923674e-05
	1180 1.14253497294214e-05
	1181 1.14885575328572e-05
	1182 1.15593021767779e-05
	1183 1.16312571023514e-05
	1184 1.1697348483608e-05
	1185 1.1750075810113e-05
	1186 1.17824318870419e-05
	1187 1.17891349091792e-05
	1188 1.17677559821061e-05
	1189 1.17194364896989e-05
	1190 1.16487435626311e-05
	1191 1.15625415730847e-05
	1192 1.14685732555131e-05
	1193 1.1373955551619e-05
	1194 1.12841502932781e-05
	1195 1.12026488672257e-05
	1196 1.11311741441966e-05
	1197 1.10701225040089e-05
	1198 1.1019037653881e-05
	1199 1.09770125824582e-05
	1200 1.09430919508213e-05
	1201 1.09163748014396e-05
	1202 1.08961190292689e-05
	1203 1.08818340613936e-05
	1204 1.08733340624667e-05
	1205 1.08706600236985e-05
	1206 1.08741695505898e-05
	1207 1.08845340349717e-05
	1208 1.09026974932647e-05
	1209 1.09299805650664e-05
	1210 1.09680728037631e-05
	1211 1.10189826578733e-05
	1212 1.1085061046856e-05
	1213 1.11688499409013e-05
	1214 1.127283591984e-05
	1215 1.13989927754687e-05
	1216 1.1547761662456e-05
	1217 1.17167864708279e-05
	1218 1.18987862265385e-05
	1219 1.2079436313428e-05
	1220 1.22359077714407e-05
	1221 1.2338277556978e-05
	1222 1.23561803988537e-05
	1223 1.22697977360531e-05
	1224 1.20814525637059e-05
	1225 1.18184065982518e-05
	1226 1.15240696416663e-05
	1227 1.12416526256709e-05
	1228 1.10014485876775e-05
	1229 1.08170969426169e-05
	1230 1.06890479827371e-05
	1231 1.06106701842634e-05
	1232 1.0572958611732e-05
	1233 1.05671496548609e-05
	1234 1.05856644196933e-05
	1235 1.06220681317382e-05
	1236 1.06706700435666e-05
	1237 1.07261904389588e-05
	1238 1.07833178830674e-05
	1239 1.08367346989269e-05
	1240 1.08811283325849e-05
	1241 1.0911680845993e-05
	1242 1.09245711072958e-05
	1243 1.09177995142318e-05
	1244 1.08912402083661e-05
	1245 1.08468992330302e-05
	1246 1.0788529344552e-05
	1247 1.07208145330162e-05
	1248 1.06485694182368e-05
	1249 1.05761247333191e-05
	1250 1.05067833935379e-05
	1251 1.04427964515708e-05
	1252 1.03853982977853e-05
	1253 1.03350813347447e-05
	1254 1.02918080635561e-05
	1255 1.02552402623246e-05
	1256 1.02249471360949e-05
	1257 1.02005356623636e-05
	1258 1.01817245585778e-05
	1259 1.01683765016958e-05
	1260 1.01606175206825e-05
	1261 1.01588257486185e-05
	1262 1.01636391569571e-05
	1263 1.01760902211367e-05
	1264 1.01975985913327e-05
	1265 1.02300387538889e-05
	1266 1.02758389850521e-05
	1267 1.03379871667642e-05
	1268 1.04201408497318e-05
	1269 1.05263735648009e-05
	1270 1.06609658985235e-05
	1271 1.08277082873087e-05
	1272 1.10283306256065e-05
	1273 1.12599932506896e-05
	1274 1.15112062033518e-05
	1275 1.17572388766973e-05
	1276 1.19575376125169e-05
	1277 1.20602187294594e-05
	1278 1.20189474408505e-05
	1279 1.18177834771416e-05
	1280 1.14879385932909e-05
	1281 1.10982421297479e-05
	1282 1.07235710729014e-05
	1283 1.04156920137655e-05
	1284 1.01942410992706e-05
	1285 1.00550819439604e-05
	1286 9.98310884625653e-06
	1287 9.96115338125492e-06
	1288 9.97402731428565e-06
	1289 1.00093119534961e-05
	1290 1.00569760199676e-05
	1291 1.01087176389569e-05
	1292 1.0157486258322e-05
	1293 1.01974506918623e-05
	1294 1.02240145025689e-05
	1295 1.023429272351e-05
	1296 1.02271417645738e-05
	1297 1.02033500795073e-05
	1298 1.01652890656112e-05
	1299 1.01164483901073e-05
	1300 1.00607951729614e-05
	1301 1.00021043785148e-05
	1302 9.94358544659235e-06
	1303 9.88758918651911e-06
	1304 9.83567576184896e-06
	1305 9.78867638679048e-06
	1306 9.74686892618593e-06
	1307 9.71018684481351e-06
	1308 9.678317894668e-06
	1309 9.65093170179898e-06
	1310 9.62761621536856e-06
	1311 9.60811515682281e-06
	1312 9.59221094731788e-06
	1313 9.57982124560885e-06
	1314 9.57102465903859e-06
	1315 9.56607828328515e-06
	1316 9.5654405960488e-06
	1317 9.56979124211443e-06
	1318 9.58012665464025e-06
	1319 9.5977457235108e-06
	1320 9.62444874730295e-06
	1321 9.66243138833534e-06
	1322 9.71460190513085e-06
	1323 9.78444766985831e-06
	1324 9.87621964299024e-06
	1325 9.99484970165554e-06
	1326 1.01453689858033e-05
	1327 1.0332341005892e-05
	1328 1.0557744705153e-05
	1329 1.08175511002173e-05
	1330 1.10966193815898e-05
	1331 1.13627134155081e-05
	1332 1.15639803617285e-05
	1333 1.16377857111161e-05
	1334 1.15354379772725e-05
	1335 1.12544654129465e-05
	1336 1.08513258716414e-05
	1337 1.04173079051861e-05
	1338 1.00347631999398e-05
	1339 9.74843723966501e-06
	1340 9.56592084122576e-06
	1341 9.4729475321742e-06
	1342 9.44834046023857e-06
	1343 9.47156948427619e-06
	1344 9.52523616959411e-06
	1345 9.59475345752026e-06
	1346 9.66768291377207e-06
	1347 9.73326161357591e-06
	1348 9.78252908367949e-06
	1349 9.80915989856612e-06
	1350 9.80993806365404e-06
	1351 9.78531248918557e-06
	1352 9.73888200306305e-06
	1353 9.67650504435369e-06
	1354 9.60497777668934e-06
	1355 9.53059459973815e-06
	1356 9.45832550414138e-06
	1357 9.39149625622804e-06
	1358 9.33186120732898e-06
	1359 9.28003221289941e-06
	1360 9.235809642405e-06
	1361 9.19851598624177e-06
	1362 9.16731498179502e-06
	1363 9.14135886276313e-06
	1364 9.11987362073319e-06
	1365 9.10223476680017e-06
	1366 9.08797084164803e-06
	1367 9.07671630656637e-06
	1368 9.06832670111157e-06
	1369 9.06280750712085e-06
	1370 9.06027682745503e-06
	1371 9.06108381926174e-06
	1372 9.06568659431173e-06
	1373 9.07485699386257e-06
	1374 9.08961026979682e-06
	1375 9.11120635649354e-06
	1376 9.14133471141554e-06
	1377 9.18209472189346e-06
	1378 9.23609615099963e-06
	1379 9.30651210495625e-06
	1380 9.39701559232731e-06
	1381 9.51162616047441e-06
	1382 9.65431389232663e-06
	1383 9.82797613602315e-06
	1384 1.00327756307195e-05
	1385 1.02631616236692e-05
	1386 1.0503709027887e-05
	1387 1.07253969829912e-05
	1388 1.0885076353162e-05
	1389 1.09335474753891e-05
	1390 1.08349705101318e-05
	1391 1.05894453348299e-05
	1392 1.02414857003907e-05
	1393 9.86271838954167e-06
	1394 9.52072629090139e-06
	1395 9.25691179176624e-06
	1396 9.08374307329041e-06
	1397 8.9946417958231e-06
	1398 8.97487760909144e-06
	1399 9.0084445467653e-06
	1400 9.08034092717003e-06
	1401 9.17660192101266e-06
	1402 9.28348264039869e-06
	1403 9.3869524775414e-06
	1404 9.4731464272968e-06
	1405 9.52970190937208e-06
	1406 9.54809512165866e-06
	1407 9.5256269219135e-06
	1408 9.46617470987121e-06
	1409 9.3790073378841e-06
	1410 9.27615231915979e-06
	1411 9.16926807903451e-06
	1412 9.06744812922966e-06
	1413 8.97636254038048e-06
	1414 8.89852938001923e-06
	1415 8.83416267427606e-06
	1416 8.78214660815502e-06
	1417 8.74077548829177e-06
	1418 8.70825184406954e-06
	1419 8.68290484845602e-06
	1420 8.66340471716853e-06
	1421 8.64865213046073e-06
	1422 8.63786589277993e-06
	1423 8.63053466559904e-06
	1424 8.62637864429416e-06
	1425 8.62533016032785e-06
	1426 8.62749064722834e-06
	1427 8.63319460364664e-06
	1428 8.64296179514668e-06
	1429 8.65756262768969e-06
	1430 8.6779831249828e-06
	1431 8.70550189802088e-06
	1432 8.74173717235749e-06
	1433 8.78865033016041e-06
	1434 8.84853789884232e-06
	1435 8.9240893892395e-06
	1436 9.01811555920062e-06
	1437 9.13333458996135e-06
	1438 9.27164871811215e-06
	1439 9.43314921286742e-06
	1440 9.61432322199585e-06
	1441 9.80571721598267e-06
	1442 9.98952821262833e-06
	1443 1.01388212918607e-05
	1444 1.02201928520174e-05
	1445 1.02030151083454e-05
	1446 1.0072956470708e-05
	1447 9.84272350379456e-06
	1448 9.55099026000994e-06
	1449 9.24834768056826e-06
	1450 8.97909395192187e-06
	1451 8.77036114843577e-06
	1452 8.63170616227649e-06
	1453 8.56085417311192e-06
	1454 8.54994724264202e-06
	1455 8.58956463289928e-06
	1456 8.6701818453605e-06
	1457 8.78193194964183e-06
	1458 8.91360540578034e-06
	1459 9.05149268604077e-06
	1460 9.17918903198256e-06
	1461 9.27878586765019e-06
	1462 9.33393186741682e-06
	1463 9.33415386228376e-06
	1464 9.27852540755225e-06
	1465 9.17646211107126e-06
	1466 9.04473602503231e-06
	1467 8.90223930483103e-06
	1468 8.76483497691538e-06
	1469 8.6428399654892e-06
	1470 8.54089965329763e-06
	1471 8.45942359717355e-06
	1472 8.39641139904757e-06
	1473 8.34892555801048e-06
	1474 8.31394824984955e-06
	1475 8.28883020886906e-06
	1476 8.27146200776951e-06
	1477 8.26022746025501e-06
	1478 8.25406136684137e-06
	1479 8.2522447613087e-06
	1480 8.25441572160912e-06
	1481 8.26053305935659e-06
	1482 8.27078920373481e-06
	1483 8.28567059052432e-06
	1484 8.30581064548142e-06
	1485 8.33220818741154e-06
	1486 8.36600095688311e-06
	1487 8.40860700357382e-06
	1488 8.46171290191933e-06
	1489 8.52712371646192e-06
	1490 8.60667206659116e-06
	1491 8.70198087277174e-06
	1492 8.81396807628221e-06
	1493 8.94209928326006e-06
	1494 9.08347021422173e-06
	1495 9.23133241315099e-06
	1496 9.37391914845875e-06
	1497 9.49408131845075e-06
	1498 9.57058169781533e-06
	1499 9.58258687511204e-06
	1500 9.51664036286104e-06
	1501 9.37348106333502e-06
	1502 9.17048316928515e-06
	1503 8.93715226979452e-06
	1504 8.70586081624936e-06
	1505 8.50305316824063e-06
	1506 8.34503612701099e-06
	1507 8.23840607644399e-06
	1508 8.18324124196579e-06
	1509 8.17643438821847e-06
	1510 8.2137889894085e-06
	1511 8.29083450959445e-06
	1512 8.40254991985034e-06
	1513 8.54223216961714e-06
	1514 8.69996607200108e-06
	1515 8.86125994270515e-06
	1516 9.0068277103228e-06
	1517 9.11439555117965e-06
	1518 9.16343666190755e-06
	1519 9.14182102640382e-06
	1520 9.05126967687409e-06
	1521 8.90765987549003e-06
	1522 8.73562713543663e-06
	1523 8.56027626028322e-06
	1524 8.4003990039605e-06
	1525 8.26608854431754e-06
	1526 8.15997246839117e-06
	1527 8.08008034347552e-06
	1528 8.02234355923304e-06
	1529 7.98233627286038e-06
	1530 7.95616466220395e-06
	1531 7.94069632625849e-06
	1532 7.93366240259274e-06
	1533 7.93351481753746e-06
	1534 7.93933230447408e-06
	1535 7.95068540426769e-06
	1536 7.9675521096334e-06
	1537 7.99022128816773e-06
	1538 8.01936116090474e-06
	1539 8.05580572738052e-06
	1540 8.10063832723529e-06
	1541 8.15502262163648e-06
	1542 8.22018554735848e-06
	1543 8.29716706896022e-06
	1544 8.38655213719619e-06
	1545 8.48801083819239e-06
	1546 8.59976220546343e-06
	1547 8.71770534693894e-06
	1548 8.83480586644225e-06
	1549 8.94056940481391e-06
	1550 9.02153818138629e-06
	1551 9.06296958191888e-06
	1552 9.05238319326429e-06
	1553 8.98364820223208e-06
	1554 8.86034401936797e-06
	1555 8.69596905950232e-06
	1556 8.5108453244942e-06
	1557 8.32655707938557e-06
	1558 8.16111685608689e-06
	1559 8.02661397081295e-06
	1560 7.92915437841657e-06
	1561 7.87049215666968e-06
	1562 7.85002346859187e-06
	1563 7.86624634585564e-06
	1564 7.91756153084577e-06
	1565 8.00228468911257e-06
	1566 8.11815301204888e-06
	1567 8.26108429308192e-06
	1568 8.42357626495982e-06
	1569 8.59311179013389e-06
	1570 8.75126703192564e-06
	1571 8.87499220780796e-06
	1572 8.94086346470147e-06
	1573 8.93228371801058e-06
	1574 8.84669479006561e-06
	1575 8.6981975133682e-06
	1576 8.51297973980536e-06
	1577 8.31996195138629e-06
	1578 8.14209776933694e-06
	1579 7.99252299010789e-06
	1580 7.87537313406972e-06
	1581 7.78885455510192e-06
	1582 7.72849756991434e-06
	1583 7.68930647332411e-06
	1584 7.66680581776313e-06
	1585 7.65751615539045e-06
	1586 7.65898962384881e-06
	1587 7.66965363929728e-06
	1588 7.68872047274272e-06
	1589 7.71602111981906e-06
	1590 7.7518401564447e-06
	1591 7.79687143470653e-06
	1592 7.85204508879644e-06
	1593 7.91836539626445e-06
	1594 7.99672460871648e-06
	1595 8.08751472902713e-06
	1596 8.19030683807398e-06
	1597 8.30309317123579e-06
	1598 8.42167012393702e-06
	1599 8.538873746744e-06
	1600 8.64422260882236e-06
	1601 8.72449356670302e-06
	1602 8.76549902883994e-06
	1603 8.75549351775362e-06
	1604 8.689045525756e-06
	1605 8.5701348879752e-06
	1606 8.41208931134929e-06
	1607 8.23429355989447e-06
	1608 8.05714834761773e-06
	1609 7.89750930074717e-06
	1610 7.76643919842002e-06
	1611 7.6692777195575e-06
	1612 7.60722284631754e-06
	1613 7.57916427929217e-06
	1614 7.58307850468043e-06
	1615 7.61688428774221e-06
	1616 7.67863472717067e-06
	1617 7.76634175636559e-06
	1618 7.87719460859648e-06
	1619 8.00667910816344e-06
	1620 8.14741785859496e-06
	1621 8.28813755759938e-06
	1622 8.41374639293235e-06
	1623 8.50659179185698e-06
	1624 8.550071957103e-06
	1625 8.53358460872755e-06
	1626 8.45688711947901e-06
	1627 8.33118481580186e-06
	1628 8.17587234713812e-06
	1629 8.01235851355386e-06
	1630 7.85842753892041e-06
	1631 7.72529505432829e-06
	1632 7.61773068624905e-06
	1633 7.53580028067802e-06
	1634 7.47708789461399e-06
	1635 7.43828995375395e-06
	1636 7.41624447009315e-06
	1637 7.40843009516823e-06
	1638 7.41313625063356e-06
	1639 7.42946240706033e-06
	1640 7.45729479412205e-06
	1641 7.49717474235467e-06
	1642 7.55026176335605e-06
	1643 7.61814144745188e-06
	1644 7.70262625948703e-06
	1645 7.80548149137417e-06
	1646 7.92780753933187e-06
	1647 8.06923053175979e-06
	1648 8.22666532052096e-06
	1649 8.39275354991287e-06
	1650 8.55426780432822e-06
	1651 8.69175903694952e-06
	1652 8.78115771740795e-06
	1653 8.79913895701634e-06
	1654 8.73156552927412e-06
	1655 8.58093522282388e-06
	1656 8.36842204066102e-06
	1657 8.12766917945851e-06
	1658 7.8933455665009e-06
	1659 7.69163552050145e-06
	1660 7.53666332720115e-06
	1661 7.43192280161509e-06
	1662 7.37440073628193e-06
	1663 7.35810518115443e-06
	1664 7.37646866078023e-06
	1665 7.4232989923928e-06
	1666 7.49279577227924e-06
	1667 7.57921986505039e-06
	1668 7.67630587006352e-06
	1669 7.77665211515455e-06
	1670 7.87162047544143e-06
	1671 7.95159079913788e-06
	1672 8.00717248061744e-06
	1673 8.03078734357143e-06
	1674 8.01848275244765e-06
	1675 7.97130307716998e-06
	1676 7.89517047650179e-06
	1677 7.79948885565318e-06
	1678 7.69483304452478e-06
	1679 7.59077982870338e-06
	1680 7.49444035896829e-06
	1681 7.41011242588741e-06
	1682 7.33968342103708e-06
	1683 7.28328603116779e-06
	1684 7.24008059371783e-06
	1685 7.20884076699235e-06
	1686 7.18840238889129e-06
	1687 7.17786402937293e-06
	1688 7.17672865846453e-06
	1689 7.185024834655e-06
	1690 7.20333669868012e-06
	1691 7.23283482884085e-06
	1692 7.2753047142271e-06
	1693 7.33322584878238e-06
	1694 7.40970476975633e-06
	1695 7.50843401853274e-06
	1696 7.63330207220747e-06
	1697 7.7878362887418e-06
	1698 7.97387196804777e-06
	1699 8.189401055958e-06
	1700 8.42536769951607e-06
	1701 8.66160927337489e-06
	1702 8.86428762925107e-06
	1703 8.98857550879484e-06
	1704 8.99078190563785e-06
	1705 8.84883997009922e-06
	1706 8.57974388068783e-06
	1707 8.23786042580821e-06
	1708 7.89191976657833e-06
	1709 7.59730923860502e-06
	1710 7.38239374875604e-06
	1711 7.25117925171048e-06
	1712 7.19367979229446e-06
	1713 7.1948161206592e-06
	1714 7.23939427160758e-06
	1715 7.31356915650849e-06
	1716 7.40480123617715e-06
	1717 7.50123468939989e-06
	1718 7.59148192130965e-06
	1719 7.66487680436967e-06
	1720 7.71256024556521e-06
	1721 7.72877581844966e-06
	1722 7.71204912730639e-06
	1723 7.66546665431633e-06
	1724 7.59594814425668e-06
	1725 7.51240849972135e-06
	1726 7.42374718143424e-06
	1727 7.33729909718761e-06
	1728 7.2581447323472e-06
	1729 7.18903381446978e-06
	1730 7.13088407700013e-06
	1731 7.0834506882278e-06
	1732 7.04578626198327e-06
	1733 7.01672203717862e-06
	1734 6.99509122448205e-06
	1735 6.9799234694301e-06
	1736 6.97052136366949e-06
	1737 6.96644682562919e-06
	1738 6.96756883122873e-06
	1739 6.97410390415598e-06
	1740 6.98658897668736e-06
	1741 7.00591878022294e-06
	1742 7.03343657093569e-06
	1743 7.0708849779777e-06
	1744 7.12060280427806e-06
	1745 7.18542058386618e-06
	1746 7.26872224099395e-06
	1747 7.3743052535491e-06
	1748 7.50602738008865e-06
	1749 7.66705931454226e-06
	1750 7.85834536465302e-06
	1751 8.07617141873607e-06
	1752 8.30885335645348e-06
	1753 8.53269913569932e-06
	1754 8.71052357886981e-06
	1755 8.79676032106147e-06
	1756 8.75212053763619e-06
	1757 8.56487667277861e-06
	1758 8.26421702448954e-06
	1759 7.91204421268787e-06
	1760 7.57661438832713e-06
	1761 7.30737367327094e-06
	1762 7.1262210994405e-06
	1763 7.03318714023737e-06
	1764 7.01685702964028e-06
	1765 7.06197967303979e-06
	1766 7.15300682685438e-06
	1767 7.27445288717377e-06
	1768 7.41020457439845e-06
	1769 7.54279480119635e-06
	1770 7.65412795011855e-06
	1771 7.72744960286786e-06
	1772 7.75097713123785e-06
	1773 7.72125935633028e-06
	1774 7.64435410793851e-06
	1775 7.53416458820766e-06
	1776 7.4078070877448e-06
	1777 7.28103349523224e-06
	1778 7.16509299536483e-06
	1779 7.06598972044503e-06
	1780 6.98543901123116e-06
	1781 6.92246848466027e-06
	1782 6.87471688998187e-06
	1783 6.83947232005266e-06
	1784 6.81422810089316e-06
	1785 6.79685587989809e-06
	1786 6.78571856838062e-06
	1787 6.77964953332832e-06
	1788 6.77786511182887e-06
	1789 6.77993632614005e-06
	1790 6.78574989976255e-06
	1791 6.79544884718553e-06
	1792 6.80944390119009e-06
	1793 6.82839273657976e-06
	1794 6.8532292498702e-06
	1795 6.88517505942343e-06
	1796 6.92578443306502e-06
	1797 6.97690211381996e-06
	1798 7.04068785140066e-06
	1799 7.11953904364293e-06
	1800 7.21586162200083e-06
	1801 7.33170735678357e-06
	1802 7.46805159757002e-06
	1803 7.62358796002616e-06
	1804 7.79326267696945e-06
	1805 7.96616528830896e-06
	1806 8.12385458104359e-06
	1807 8.24086996864537e-06
	1808 8.28867802660227e-06
	1809 8.24451322678499e-06
	1810 8.10235604831178e-06
	1811 7.87959277381844e-06
	1812 7.61324128362162e-06
	1813 7.34732791585202e-06
	1814 7.11889491711304e-06
	1815 6.95046413490275e-06
	1816 6.85056147275986e-06
	1817 6.81839943617035e-06
	1818 6.84856861710159e-06
	1819 6.93382329508552e-06
	1820 7.06565942998338e-06
	1821 7.23314512995898e-06
	1822 7.42101311956844e-06
	1823 7.60801736277017e-06
	1824 7.76722149531395e-06
	1825 7.86991732226738e-06
	1826 7.89363094177986e-06
	1827 7.83104848167682e-06
	1828 7.69458824834146e-06
	1829 7.5122034086661e-06
	1830 7.31653677199517e-06
	1831 7.13417341557943e-06
	1832 6.98034214430265e-06
	1833 6.85977735415122e-06
	1834 6.77039641949762e-06
	1835 6.70708094041572e-06
	1836 6.6640710052468e-06
	1837 6.63628783748038e-06
	1838 6.61969734672141e-06
	1839 6.61132278523269e-06
	1840 6.60910506766754e-06
	1841 6.61169956295282e-06
	1842 6.61831836090698e-06
	1843 6.6285816338052e-06
	1844 6.64247756176906e-06
	1845 6.66019730566347e-06
	1846 6.68223436939286e-06
	1847 6.70921517631484e-06
	1848 6.74197067063176e-06
	1849 6.7814960349466e-06
	1850 6.82892699543913e-06
	1851 6.88540564830475e-06
	1852 6.95206507295154e-06
	1853 7.02979771638468e-06
	1854 7.11891600957415e-06
	1855 7.21877349896261e-06
	1856 7.32705239769871e-06
	1857 7.43918693935086e-06
	1858 7.5476488614612e-06
	1859 7.64176301970565e-06
	1860 7.70828848750682e-06
	1861 7.7335669708134e-06
	1862 7.70694983742715e-06
	1863 7.62475070104074e-06
	1864 7.49284221868152e-06
	1865 7.32634052447168e-06
	1866 7.14597607753831e-06
	1867 6.97293972606872e-06
	1868 6.82466429680062e-06
	1869 6.71296382392939e-06
	1870 6.64450211296241e-06
	1871 6.62236485204915e-06
	1872 6.64792184856822e-06
	1873 6.721875051241e-06
	1874 6.84436822062651e-06
	1875 7.01388771773281e-06
	1876 7.22498344174483e-06
	1877 7.46459575218239e-06
	1878 7.70852832410185e-06
	1879 7.91994632365345e-06
	1880 8.05432317996235e-06
	1881 8.073349498261e-06
	1882 7.96378474099413e-06
	1883 7.74820165005963e-06
	1884 7.47648316945515e-06
	1885 7.20298907630479e-06
	1886 6.96617670037369e-06
	1887 6.78237693918504e-06
	1888 6.65113152287233e-06
	1889 6.56378851360273e-06
	1890 6.50976421390226e-06
	1891 6.47970983891355e-06
	1892 6.46640849311098e-06
	1893 6.46478094168401e-06
	1894 6.47139140319553e-06
	1895 6.48414742521197e-06
	1896 6.50178658290201e-06
	1897 6.52370182407935e-06
	1898 6.54971220548362e-06
	1899 6.57988278973676e-06
	1900 6.6144668267043e-06
	1901 6.65376724207078e-06
	1902 6.69806804509676e-06
	1903 6.74758541485687e-06
	1904 6.80223908133826e-06
	1905 6.86159405560716e-06
	1906 6.92468901597465e-06
	1907 6.98974182533618e-06
	1908 7.05415457424863e-06
	1909 7.11426014810002e-06
	1910 7.16547809176404e-06
	1911 7.2026948778614e-06
	1912 7.22071193415275e-06
	1913 7.21522419944165e-06
	1914 7.18390273135583e-06
	1915 7.12700479255091e-06
	1916 7.04790839556324e-06
	1917 6.95264706429555e-06
	1918 6.84891463365034e-06
	1919 6.74488967611353e-06
	1920 6.64811434170076e-06
	1921 6.56481654814911e-06
	1922 6.49982023936957e-06
	1923 6.45680595923181e-06
	1924 6.43887610074501e-06
	1925 6.4490449318555e-06
	1926 6.49081868253631e-06
	1927 6.5683674268513e-06
	1928 6.68637013134799e-06
	1929 6.84916802029534e-06
	1930 7.05886892049534e-06
	1931 7.31194144876213e-06
	1932 7.59388318449794e-06
	1933 7.87358307441366e-06
	1934 8.10121214200876e-06
	1935 8.21678089835132e-06
	1936 8.17388649565487e-06
	1937 7.96874188324637e-06
	1938 7.65003127867203e-06
	1939 7.29562867540778e-06
	1940 6.97428188445315e-06
	1941 6.72277665536569e-06
	1942 6.54743041295092e-06
	1943 6.43754469287217e-06
	1944 6.37746648912696e-06
	1945 6.35291355699508e-06
	1946 6.3530067517803e-06
	1947 6.3702028363366e-06
	1948 6.39961140613821e-06
	1949 6.43817515744161e-06
	1950 6.48398542324458e-06
	1951 6.53583846421668e-06
	1952 6.59281875137196e-06
	1953 6.65392431642431e-06
	1954 6.71784285444232e-06
	1955 6.78272682641534e-06
	1956 6.84604648348852e-06
	1957 6.90448422169254e-06
	1958 6.95412970941334e-06
	1959 6.99071189025346e-06
	1960 7.01016169379898e-06
	1961 7.00922240781665e-06
	1962 6.98628541151436e-06
	1963 6.94182975280455e-06
	1964 6.87860861958711e-06
	1965 6.80123069329852e-06
	1966 6.71554739639646e-06
	1967 6.62771773995985e-06
	1968 6.54328981575247e-06
	1969 6.46675142235154e-06
	1970 6.40138510554777e-06
	1971 6.34934395193909e-06
	1972 6.31199428546836e-06
	1973 6.29022127540679e-06
	1974 6.28483876852215e-06
	1975 6.29685737152386e-06
	1976 6.32763773822376e-06
	1977 6.37911254486312e-06
	1978 6.45351684092077e-06
	1979 6.5532551243308e-06
	1980 6.68024670602563e-06
	1981 6.83482233121424e-06
	1982 7.01395458158061e-06
	1983 7.20888551875021e-06
	1984 7.40275374866428e-06
	1985 7.56958124270568e-06
	1986 7.67720369942992e-06
	1987 7.69596408289885e-06
	1988 7.6114875966482e-06
	1989 7.43404957859184e-06
	1990 7.19720958031189e-06
	1991 6.94418926094897e-06
	1992 6.71219540393508e-06
	1993 6.52337430651073e-06
	1994 6.3850060421089e-06
	1995 6.29476241442717e-06
	1996 6.24615760536074e-06
	1997 6.23229236751399e-06
	1998 6.24758389200508e-06
	1999 6.28830455084994e-06
};
\addlegendentry{Train}
\addplot [semithick, black]
table {%
	0 0.0305759701877832
	1 0.0299463551491499
	2 0.0293179992586374
	3 0.0286893118172884
	4 0.028057811781764
	5 0.0274200011044741
	6 0.0267707947641611
	7 0.0261020623147488
	8 0.0253994110971689
	9 0.0246372669935226
	10 0.0237707681953907
	11 0.0227279290556908
	12 0.0214171409606934
	13 0.0197803545743227
	14 0.0178924091160297
	15 0.0159893035888672
	16 0.0142955398187041
	17 0.0128632169216871
	18 0.0116492994129658
	19 0.0106171909719706
	20 0.00973521824926138
	21 0.00897479243576527
	22 0.00831248424947262
	23 0.00772987026721239
	24 0.00721272407099605
	25 0.00675006257370114
	26 0.00633333809673786
	27 0.00595581578090787
	28 0.00561211444437504
	29 0.00529786758124828
	30 0.00500946817919612
	31 0.00474390154704452
	32 0.00449861958622932
	33 0.00427144253626466
	34 0.00406049564480782
	35 0.00386416004039347
	36 0.00368102639913559
	37 0.00350986863486469
	38 0.00334960920736194
	39 0.00319930189289153
	40 0.00305810803547502
	41 0.00292528420686722
	42 0.00280016660690308
	43 0.00268215988762677
	44 0.00257073156535625
	45 0.00246539991348982
	46 0.00236572953872383
	47 0.00227132649160922
	48 0.00218183267861605
	49 0.00209692120552063
	50 0.0020162935834378
	51 0.00193967751692981
	52 0.00186682236380875
	53 0.00179749866947532
	54 0.0017314946744591
	55 0.0016686151502654
	56 0.00160867988597602
	57 0.00155152240768075
	58 0.00149698741734028
	59 0.00144493079278618
	60 0.00139521877281368
	61 0.00134772527962923
	62 0.00130233354866505
	63 0.00125893275253475
	64 0.00121741998009384
	65 0.00117769802454859
	66 0.0011396745685488
	67 0.00110326358117163
	68 0.00106838322244585
	69 0.00103495665825903
	70 0.00100291101261973
	71 0.000972177600488067
	72 0.00094269122928381
	73 0.000914390722755343
	74 0.000887218047864735
	75 0.000861118838656694
	76 0.000836041697766632
	77 0.00081193883670494
	78 0.000788764969911426
	79 0.000766478013247252
	80 0.000745038385502994
	81 0.000724409474059939
	82 0.000704557576682419
	83 0.000685450970195234
	84 0.000667060434352607
	85 0.000649358844384551
	86 0.000632321229204535
	87 0.000615924247540534
	88 0.00060014653718099
	89 0.000584967725444585
	90 0.000570369011256844
	91 0.000556332583073527
	92 0.000542841327842325
	93 0.000529879063833505
	94 0.000517430482432246
	95 0.000505480275023729
	96 0.000494013831485063
	97 0.000483016483485699
	98 0.000472474173875526
	99 0.000462372758192942
	100 0.000452698033768684
	101 0.00044343588524498
	102 0.000434572604717687
	103 0.000426093843998387
	104 0.000417985895182937
	105 0.000410234875744209
	106 0.000402827048674226
	107 0.000395748560549691
	108 0.000388985907193273
	109 0.000382525700842962
	110 0.00037635478656739
	111 0.000370460125850514
	112 0.000364829116733745
	113 0.000359449244569987
	114 0.000354308605892584
	115 0.000349395471857861
	116 0.00034469849197194
	117 0.000340206810506061
	118 0.000335909833665937
	119 0.00033179740421474
	120 0.000327859830576926
	121 0.000324087799526751
	122 0.000320472434395924
	123 0.000317005207762122
	124 0.000313678261591122
	125 0.000310483737848699
	126 0.00030741433147341
	127 0.000304463173961267
	128 0.000301623862469569
	129 0.000298890052363276
	130 0.000296255981083959
	131 0.00029371606069617
	132 0.000291265081614256
	133 0.000288898212602362
	134 0.000286610797047615
	135 0.000284398440271616
	136 0.000282257067738101
	137 0.000280182750429958
	138 0.000278171879472211
	139 0.000276221107924357
	140 0.000274327117949724
	141 0.000272486737230793
	142 0.000270697404630482
	143 0.000268956122454256
	144 0.000267260649707168
	145 0.000265608308836818
	146 0.000263996887952089
	147 0.000262424407992512
	148 0.000260888744378462
	149 0.000259388005360961
	150 0.000257920386502519
	151 0.00025648417067714
	152 0.000255077873589471
	153 0.000253699807217345
	154 0.000252348749199882
	155 0.000251023157034069
	156 0.000249721808359027
	157 0.000248443568125367
	158 0.000247187272179872
	159 0.000245951727265492
	160 0.000244736002059653
	161 0.000243539237999357
	162 0.000242360401898623
	163 0.000241198533331044
	164 0.000240053050220013
	165 0.000238923035794869
	166 0.000237807835219428
	167 0.000236706677242182
	168 0.00023561911075376
	169 0.000234544218983501
	170 0.000233481710893102
	171 0.000232430931646377
	172 0.000231391328270547
	173 0.000230362435104325
	174 0.000229343786486425
	175 0.000228335047722794
	176 0.000227335593081079
	177 0.000226345131522976
	178 0.000225363372010179
	179 0.00022438979067374
	180 0.000223424212890677
	181 0.000222466231207363
	182 0.000221515452722088
	183 0.000220571819227189
	184 0.000219634908717126
	185 0.000218704357394017
	186 0.000217780150705948
	187 0.000216861881199293
	188 0.000215949286939576
	189 0.000215042295167223
	190 0.000214140702155419
	191 0.000213244187762029
	192 0.000212352562812157
	193 0.000211465827305801
	194 0.000210583777516149
	195 0.000209706151508726
	196 0.000208832978387363
	197 0.000207963981665671
	198 0.000207099103135988
	199 0.000206238211831078
	200 0.000205381191335618
	201 0.000204527997993864
	202 0.000203678559046239
	203 0.000202832670765929
	204 0.000201990449568257
	205 0.000201151633518748
	206 0.000200316237169318
	207 0.00019948418776039
	208 0.000198655499843881
	209 0.000197830057004467
	210 0.000197007801034488
	211 0.000196188702830113
	212 0.000195372747839428
	213 0.000194559805095196
	214 0.000193749961908907
	215 0.000192943145520985
	216 0.000192139210412279
	217 0.00019133830210194
	218 0.000190540231415071
	219 0.000189745085663162
	220 0.000188952704775147
	221 0.000188163234270178
	222 0.000187376499525271
	223 0.000186592471436597
	224 0.000185811208211817
	225 0.000185032651643269
	226 0.000184256743523292
	227 0.000183483498403803
	228 0.000182712814421393
	229 0.00018194479343947
	230 0.000181179304490797
	231 0.000180416274815798
	232 0.000179655879037455
	233 0.000178897826117463
	234 0.000178142276126891
	235 0.000177389156306162
	236 0.000176638437551446
	237 0.000175890134414658
	238 0.000175144043168984
	239 0.000174400251125917
	240 0.000173658729181625
	241 0.000172919506439939
	242 0.000172182495589368
	243 0.00017144760931842
	244 0.000170714905834757
	245 0.000169984312378801
	246 0.000169255843502469
	247 0.000168529310030863
	248 0.000167804872035049
	249 0.000167082398547791
	250 0.00016636194777675
	251 0.000165643228683621
	252 0.000164926474099047
	253 0.000164211553055793
	254 0.000163498465553857
	255 0.000162787080626003
	256 0.000162077471031807
	257 0.000161369491252117
	258 0.000160663184942678
	259 0.000159958435688168
	260 0.000159255287144333
	261 0.000158553637447767
	262 0.000157853428390808
	263 0.000157154645421542
	264 0.000156457215780392
	265 0.000155761168571189
	266 0.000155066343722865
	267 0.000154372813994996
	268 0.000153680419316515
	269 0.000152989217895083
	270 0.000152298991451971
	271 0.000151609841850586
	272 0.000150921638123691
	273 0.000150234263855964
	274 0.000149547937326133
	275 0.000148862338392064
	276 0.000148177525261417
	277 0.000147493308759294
	278 0.000146809863508679
	279 0.000146126883919351
	280 0.000145444457302801
	281 0.0001447625545552
	282 0.000144081015605479
	283 0.00014339984045364
	284 0.000142719072755426
	285 0.00014203843602445
	286 0.000141358090331778
	287 0.000140677861054428
	288 0.000139997791848145
	289 0.000139317649882287
	290 0.000138637609779835
	291 0.000137957467813976
	292 0.000137277238536626
	293 0.0001365969364997
	294 0.000135916387080215
	295 0.000135235663037747
	296 0.00013455466250889
	297 0.000133873356389813
	298 0.000133191802888177
	299 0.000132509841932915
	300 0.000131827560835518
	301 0.000131144901388325
	302 0.00013046179083176
	303 0.000129778272821568
	304 0.000129094361909665
	305 0.000128409985336475
	306 0.000127725186757743
	307 0.000127040024381131
	308 0.000126354410895146
	309 0.000125668448163196
	310 0.000124982165289111
	311 0.000124295605928637
	312 0.000123608726426028
	313 0.00012292165774852
	314 0.0001222343853442
	315 0.000121547098387964
	316 0.000120859716844279
	317 0.000120172364404425
	318 0.000119485172035638
	319 0.000118798263429198
	320 0.000118111580377445
	321 0.000117425392090809
	322 0.000116739676741417
	323 0.000116054616228212
	324 0.000115370232379064
	325 0.000114686830784194
	326 0.000114004324132111
	327 0.000113322996185161
	328 0.000112642883323133
	329 0.000111964116513263
	330 0.000111286753963213
	331 0.000110611152194906
	332 0.000109937209344935
	333 0.000109265172795858
	334 0.000108595129859168
	335 0.000107927131466568
	336 0.000107261512312107
	337 0.000106598112324718
	338 0.000105937288026325
	339 0.00010527900303714
	340 0.000104623461083975
	341 0.000103970705822576
	342 0.000103320882772096
	343 0.000102674101071898
	344 0.000102030382549856
	345 0.000101389930932783
	346 0.000100752782600466
	347 0.000100118995760567
	348 9.94886868284084e-05
	349 9.88620085990988e-05
	350 9.82388810371049e-05
	351 9.76194860413671e-05
	352 9.70038236118853e-05
	353 9.63920538197272e-05
	354 9.5784169388935e-05
	355 9.51802139752544e-05
	356 9.45802385103889e-05
	357 9.39843521337025e-05
	358 9.33925693971105e-05
	359 9.28048903006129e-05
	360 9.2221460363362e-05
	361 9.16421777219512e-05
	362 9.10672169993632e-05
	363 9.04964981600642e-05
	364 8.99300939636305e-05
	365 8.93680262379348e-05
	366 8.88102877070196e-05
	367 8.82569220266305e-05
	368 8.77080092323013e-05
	369 8.71634256327525e-05
	370 8.66232585394755e-05
	371 8.60875443322584e-05
	372 8.55562539072707e-05
	373 8.50293145049363e-05
	374 8.45069080241956e-05
	375 8.39888816699386e-05
	376 8.34752718219534e-05
	377 8.29660930321552e-05
	378 8.24613525765017e-05
	379 8.19609995232895e-05
	380 8.14650484244339e-05
	381 8.09734847280197e-05
	382 8.04862793302163e-05
	383 8.00034758867696e-05
	384 7.95249652583152e-05
	385 7.90508493082598e-05
	386 7.8580982517451e-05
	387 7.81154158175923e-05
	388 7.76541346567683e-05
	389 7.71970808273181e-05
	390 7.67442543292418e-05
	391 7.62956624384969e-05
	392 7.58512251195498e-05
	393 7.5410986028146e-05
	394 7.49747850932181e-05
	395 7.45427532820031e-05
	396 7.41147450753488e-05
	397 7.36908332328312e-05
	398 7.3270894063171e-05
	399 7.2854949394241e-05
	400 7.24430065019988e-05
	401 7.2034919867292e-05
	402 7.16307768016122e-05
	403 7.12304608896375e-05
	404 7.08340085111558e-05
	405 7.04413541825488e-05
	406 7.00524615240283e-05
	407 6.96673087077215e-05
	408 6.92858957336284e-05
	409 6.89081498421729e-05
	410 6.85340637573972e-05
	411 6.81635501678102e-05
	412 6.77966309012845e-05
	413 6.74332841299474e-05
	414 6.70733934384771e-05
	415 6.6717024310492e-05
	416 6.63641112623736e-05
	417 6.60145815345459e-05
	418 6.56684351270087e-05
	419 6.53256502118893e-05
	420 6.4986183133442e-05
	421 6.46500193397515e-05
	422 6.43170933471993e-05
	423 6.39873396721669e-05
	424 6.3660838350188e-05
	425 6.33374584140256e-05
	426 6.3017250795383e-05
	427 6.27000772510655e-05
	428 6.23859741608612e-05
	429 6.20749124209397e-05
	430 6.17668483755551e-05
	431 6.14617601968348e-05
	432 6.11595824011602e-05
	433 6.08603222644888e-05
	434 6.05639434070326e-05
	435 6.02704276388977e-05
	436 5.99796767346561e-05
	437 5.96917780057993e-05
	438 5.94065786572173e-05
	439 5.91241478105076e-05
	440 5.88443799642846e-05
	441 5.85672896704637e-05
	442 5.82928951189388e-05
	443 5.80210544285364e-05
	444 5.77517857891507e-05
	445 5.74851146666333e-05
	446 5.72209464735352e-05
	447 5.69593175896443e-05
	448 5.67001443414483e-05
	449 5.64433976251166e-05
	450 5.61891320103314e-05
	451 5.59372165298555e-05
	452 5.5687651183689e-05
	453 5.5440490541514e-05
	454 5.51955999981146e-05
	455 5.49530268472154e-05
	456 5.47127347090282e-05
	457 5.44746508239768e-05
	458 5.42388079338707e-05
	459 5.40051551070064e-05
	460 5.37736632395536e-05
	461 5.35442923137452e-05
	462 5.33170677954331e-05
	463 5.30919423908927e-05
	464 5.2868876082357e-05
	465 5.26478288520593e-05
	466 5.24288225278724e-05
	467 5.22118316439446e-05
	468 5.19968125445303e-05
	469 5.17837688676082e-05
	470 5.15726133016869e-05
	471 5.13633458467666e-05
	472 5.11559956066776e-05
	473 5.09505007357802e-05
	474 5.07468212163076e-05
	475 5.05450007040054e-05
	476 5.03449118696153e-05
	477 5.01466456626076e-05
	478 4.99500929436181e-05
	479 4.97552318847738e-05
	480 4.95621097797994e-05
	481 4.93706793349702e-05
	482 4.91808823426254e-05
	483 4.89927406306379e-05
	484 4.88062178192195e-05
	485 4.86212775285821e-05
	486 4.84379233967047e-05
	487 4.82561081298627e-05
	488 4.80757917102892e-05
	489 4.78970287076663e-05
	490 4.7719764552312e-05
	491 4.75439301226288e-05
	492 4.73695399705321e-05
	493 4.7196608647937e-05
	494 4.70250670332462e-05
	495 4.68549405923113e-05
	496 4.66861674794927e-05
	497 4.65187295048963e-05
	498 4.63526339444797e-05
	499 4.61878662463278e-05
	500 4.60243682027794e-05
	501 4.58622016594745e-05
	502 4.57012356491759e-05
	503 4.5541539293481e-05
	504 4.53830361948349e-05
	505 4.52257190772798e-05
	506 4.50696134066675e-05
	507 4.49146900791675e-05
	508 4.47608981630765e-05
	509 4.46082704002038e-05
	510 4.44567194790579e-05
	511 4.43062890553847e-05
	512 4.41569209215231e-05
	513 4.40086514572613e-05
	514 4.38614224549383e-05
	515 4.37152630183846e-05
	516 4.35700785601512e-05
	517 4.34259163739625e-05
	518 4.32827509939671e-05
	519 4.31405642302707e-05
	520 4.29993306170218e-05
	521 4.28590574301779e-05
	522 4.27197264798451e-05
	523 4.25813159381505e-05
	524 4.24437748733908e-05
	525 4.2307197873015e-05
	526 4.21714357798919e-05
	527 4.20366050093435e-05
	528 4.19025782321114e-05
	529 4.1769424569793e-05
	530 4.1637056710897e-05
	531 4.15055728808511e-05
	532 4.13748748542275e-05
	533 4.12449517170899e-05
	534 4.11158071074169e-05
	535 4.09874592151027e-05
	536 4.08598643844016e-05
	537 4.07329789595678e-05
	538 4.06069157179445e-05
	539 4.04815182264429e-05
	540 4.03568592446391e-05
	541 4.0232902392745e-05
	542 4.01096585846972e-05
	543 3.99870550609194e-05
	544 3.98651864088606e-05
	545 3.97439653170295e-05
	546 3.96233990613837e-05
	547 3.95034840039443e-05
	548 3.93842346966267e-05
	549 3.92656293115579e-05
	550 3.91475768992677e-05
	551 3.90301865991205e-05
	552 3.89133965654764e-05
	553 3.87972031603567e-05
	554 3.8681613659719e-05
	555 3.85666207876056e-05
	556 3.84521772502922e-05
	557 3.83383085136302e-05
	558 3.82250400434714e-05
	559 3.81122772523668e-05
	560 3.80000747099984e-05
	561 3.78884578822181e-05
	562 3.77773758373223e-05
	563 3.76668031094596e-05
	564 3.75567542505451e-05
	565 3.74472256225999e-05
	566 3.73382536054123e-05
	567 3.7229765439406e-05
	568 3.71218302461784e-05
	569 3.70143498003017e-05
	570 3.69074259651825e-05
	571 3.68010078091174e-05
	572 3.66950735042337e-05
	573 3.65896594303194e-05
	574 3.64847655873746e-05
	575 3.63803665095475e-05
	576 3.62765349564143e-05
	577 3.61731836164836e-05
	578 3.60704216291197e-05
	579 3.59681944246404e-05
	580 3.58665420208126e-05
	581 3.57654607796576e-05
	582 3.56650525645819e-05
	583 3.55652300640941e-05
	584 3.54660696757492e-05
	585 3.53676441591233e-05
	586 3.52700044459198e-05
	587 3.51731578120962e-05
	588 3.50772388628684e-05
	589 3.49822585121728e-05
	590 3.48883513652254e-05
	591 3.47955319739413e-05
	592 3.47040513588581e-05
	593 3.46140113833826e-05
	594 3.45255211868789e-05
	595 3.44388572557364e-05
	596 3.43541614711285e-05
	597 3.42717830790207e-05
	598 3.41919658239931e-05
	599 3.41150225722231e-05
	600 3.40413971571252e-05
	601 3.39716134476475e-05
	602 3.39061298291199e-05
	603 3.38457393809222e-05
	604 3.37911369570065e-05
	605 3.37433448294178e-05
	606 3.37034871336073e-05
	607 3.36731209245045e-05
	608 3.36539960699156e-05
	609 3.36484918079805e-05
	610 3.3659627661109e-05
	611 3.36913726641797e-05
	612 3.37488672812469e-05
	613 3.38389654643834e-05
	614 3.39712714776397e-05
	615 3.41584091074765e-05
	616 3.44172367476858e-05
	617 3.47701898135711e-05
	618 3.52456190739758e-05
	619 3.58750367013272e-05
	620 3.6682678910438e-05
	621 3.76587413484231e-05
	622 3.86997162422631e-05
	623 3.9527214539703e-05
	624 3.96760005969554e-05
	625 3.87334039260168e-05
	626 3.68084583897144e-05
	627 3.46241031365935e-05
	628 3.2906547858147e-05
	629 3.18700076604728e-05
	630 3.13438504235819e-05
	631 3.10942268697545e-05
	632 3.09640563500579e-05
	633 3.08742201013956e-05
	634 3.07921654894017e-05
	635 3.07083428197075e-05
	636 3.06225228996482e-05
	637 3.05372414004523e-05
	638 3.04549594147829e-05
	639 3.03769811580423e-05
	640 3.03035285469377e-05
	641 3.02342050417792e-05
	642 3.01682102872292e-05
	643 3.01046056847554e-05
	644 3.00425763271051e-05
	645 2.99814582831459e-05
	646 2.99207167699933e-05
	647 2.98600789392367e-05
	648 2.97992137348047e-05
	649 2.97381357086124e-05
	650 2.967669206555e-05
	651 2.96150683425367e-05
	652 2.95532699965406e-05
	653 2.94914243568201e-05
	654 2.94297024083789e-05
	655 2.93682332994649e-05
	656 2.930722439487e-05
	657 2.9246801204863e-05
	658 2.91871965600876e-05
	659 2.91286578431027e-05
	660 2.90715033770539e-05
	661 2.90158914140193e-05
	662 2.89622603304451e-05
	663 2.89108447759645e-05
	664 2.88621067738859e-05
	665 2.88164792436874e-05
	666 2.87745406239992e-05
	667 2.87367602140876e-05
	668 2.87038074020529e-05
	669 2.8676357032964e-05
	670 2.86551821773173e-05
	671 2.86411632259842e-05
	672 2.86350150418002e-05
	673 2.86375452560605e-05
	674 2.86495851469226e-05
	675 2.86715076072142e-05
	676 2.87037983071059e-05
	677 2.87459552055225e-05
	678 2.8797298000427e-05
	679 2.88556457235245e-05
	680 2.89177296508569e-05
	681 2.89789732050849e-05
	682 2.90323132503545e-05
	683 2.90695516014239e-05
	684 2.90806910925312e-05
	685 2.90555799438152e-05
	686 2.89852705464e-05
	687 2.88637893390842e-05
	688 2.86901577055687e-05
	689 2.84691031993134e-05
	690 2.82109485851834e-05
	691 2.79297873930773e-05
	692 2.76409864454763e-05
	693 2.73587156698341e-05
	694 2.70938799076248e-05
	695 2.6853882445721e-05
	696 2.66427396127256e-05
	697 2.64619338850025e-05
	698 2.63114361587213e-05
	699 2.6190529752057e-05
	700 2.60985070781317e-05
	701 2.60348460869864e-05
	702 2.5999464924098e-05
	703 2.5992410883191e-05
	704 2.60133565461729e-05
	705 2.60613378486596e-05
	706 2.61334716924466e-05
	707 2.62241810560226e-05
	708 2.63243964582216e-05
	709 2.64208574662916e-05
	710 2.64967584371334e-05
	711 2.65340622718213e-05
	712 2.65164908341831e-05
	713 2.64347127085784e-05
	714 2.62898374785436e-05
	715 2.60943852481432e-05
	716 2.58691998169525e-05
	717 2.5637587896199e-05
	718 2.54196183959721e-05
	719 2.52283007284859e-05
	720 2.50693574344041e-05
	721 2.49427157541504e-05
	722 2.48450469371164e-05
	723 2.47719490289455e-05
	724 2.47191237576772e-05
	725 2.46832096308935e-05
	726 2.46619729296071e-05
	727 2.4654214939801e-05
	728 2.46597792283865e-05
	729 2.46789841185091e-05
	730 2.47127081820508e-05
	731 2.4762020984781e-05
	732 2.48278047365602e-05
	733 2.49105232796865e-05
	734 2.50094435614301e-05
	735 2.51224118983373e-05
	736 2.52448280662065e-05
	737 2.53694379352964e-05
	738 2.54852220678004e-05
	739 2.5578503482393e-05
	740 2.56329476542305e-05
	741 2.56326366070425e-05
	742 2.55653048952809e-05
	743 2.5425946660107e-05
	744 2.52191439358285e-05
	745 2.49592631007545e-05
	746 2.46672243520152e-05
	747 2.43659160332754e-05
	748 2.40752033278113e-05
	749 2.38089232880156e-05
	750 2.3574448277941e-05
	751 2.33735681831604e-05
	752 2.32045367738465e-05
	753 2.30634195759194e-05
	754 2.29457946261391e-05
	755 2.28474254981847e-05
	756 2.27642685786122e-05
	757 2.2693260689266e-05
	758 2.26317042688606e-05
	759 2.25775602302747e-05
	760 2.25291987590026e-05
	761 2.24853611143772e-05
	762 2.24449649977032e-05
	763 2.24072391574737e-05
	764 2.23714250751073e-05
	765 2.23369443119736e-05
	766 2.23032966459868e-05
	767 2.22699181904318e-05
	768 2.22364724322688e-05
	769 2.22024227696238e-05
	770 2.21674708882347e-05
	771 2.21311875066021e-05
	772 2.2093265215517e-05
	773 2.20533911488019e-05
	774 2.20113324758131e-05
	775 2.19669091165997e-05
	776 2.19199628190836e-05
	777 2.18704717553919e-05
	778 2.18184159166412e-05
	779 2.17638971662382e-05
	780 2.17070628423244e-05
	781 2.16481857933104e-05
	782 2.15876043512253e-05
	783 2.15257750824094e-05
	784 2.14631872950122e-05
	785 2.14004503504839e-05
	786 2.13383900700137e-05
	787 2.12779377761763e-05
	788 2.12202576221898e-05
	789 2.1166808437556e-05
	790 2.11193219001871e-05
	791 2.10800099011976e-05
	792 2.1051668227301e-05
	793 2.10376729228301e-05
	794 2.10424313991098e-05
	795 2.10711896215798e-05
	796 2.11305468837963e-05
	797 2.1228543118923e-05
	798 2.1374960851972e-05
	799 2.15816380659817e-05
	800 2.18627028516494e-05
	801 2.22338330786442e-05
	802 2.27114323934074e-05
	803 2.33074661082355e-05
	804 2.4018700059969e-05
	805 2.48051546805073e-05
	806 2.55583399848547e-05
	807 2.60746437561465e-05
	808 2.60886117757764e-05
	809 2.54200513154501e-05
	810 2.41693669522647e-05
	811 2.27247255679686e-05
	812 2.14968022191897e-05
	813 2.06720251298975e-05
	814 2.02147130039521e-05
	815 2.00028334802482e-05
	816 1.99249443539884e-05
	817 1.99070291273529e-05
	818 1.99071673705475e-05
	819 1.9904338842025e-05
	820 1.98900670511648e-05
	821 1.98625202756375e-05
	822 1.98236302821897e-05
	823 1.97766075871186e-05
	824 1.97248555195983e-05
	825 1.9671331756399e-05
	826 1.96182063518791e-05
	827 1.95667598745786e-05
	828 1.9517729015206e-05
	829 1.9471337509458e-05
	830 1.94274653040338e-05
	831 1.93859505088767e-05
	832 1.9346398403286e-05
	833 1.93085961655015e-05
	834 1.92722145584412e-05
	835 1.92371535376878e-05
	836 1.92032075574389e-05
	837 1.91703784366837e-05
	838 1.91386388905812e-05
	839 1.91080835065804e-05
	840 1.9078832337982e-05
	841 1.90510600077687e-05
	842 1.90251321328105e-05
	843 1.90012433449738e-05
	844 1.89798192877788e-05
	845 1.89613656402798e-05
	846 1.89463989954675e-05
	847 1.89355523616541e-05
	848 1.89296697499231e-05
	849 1.89294478332158e-05
	850 1.89359816431534e-05
	851 1.89502989087487e-05
	852 1.89734400919406e-05
	853 1.9006680304301e-05
	854 1.90510982065462e-05
	855 1.91076596820494e-05
	856 1.91770050150808e-05
	857 1.92592415260151e-05
	858 1.93535142898327e-05
	859 1.94578024093062e-05
	860 1.9568298739614e-05
	861 1.96789715118939e-05
	862 1.97815188585082e-05
	863 1.98650031961733e-05
	864 1.9917126337532e-05
	865 1.99256064661313e-05
	866 1.98800225916784e-05
	867 1.97752578969812e-05
	868 1.96132896235213e-05
	869 1.94037329492858e-05
	870 1.91626422747504e-05
	871 1.89082893484738e-05
	872 1.86591805686476e-05
	873 1.84295568033122e-05
	874 1.82292569661513e-05
	875 1.80630304384977e-05
	876 1.79325852514012e-05
	877 1.78371701622382e-05
	878 1.77754354808712e-05
	879 1.77457623067312e-05
	880 1.77470119524514e-05
	881 1.77782330865739e-05
	882 1.78387854248285e-05
	883 1.79274102265481e-05
	884 1.80418010131689e-05
	885 1.81774703378323e-05
	886 1.83266038220609e-05
	887 1.84777527465485e-05
	888 1.86153301910963e-05
	889 1.872114262369e-05
	890 1.87771911441814e-05
	891 1.87693403859157e-05
	892 1.86927845788887e-05
	893 1.85539083759068e-05
	894 1.83691590791568e-05
	895 1.81611503649037e-05
	896 1.79523503902601e-05
	897 1.77601887116907e-05
	898 1.75948825926753e-05
	899 1.74598844751017e-05
	900 1.73540665855398e-05
	901 1.7273734556511e-05
	902 1.72144773387117e-05
	903 1.7172360458062e-05
	904 1.71442261489574e-05
	905 1.71278443303891e-05
	906 1.71219144249335e-05
	907 1.71259198396001e-05
	908 1.71398933161981e-05
	909 1.71644696820294e-05
	910 1.72004583873786e-05
	911 1.72491418197751e-05
	912 1.73116568475962e-05
	913 1.73889548022998e-05
	914 1.74817014340078e-05
	915 1.75896020664368e-05
	916 1.77113179233856e-05
	917 1.78431710082805e-05
	918 1.79793223651359e-05
	919 1.8111226381734e-05
	920 1.82268140633823e-05
	921 1.8312288375455e-05
	922 1.83526517503196e-05
	923 1.83351276064059e-05
	924 1.82524854608346e-05
	925 1.81058658199618e-05
	926 1.79052076418884e-05
	927 1.76684025063878e-05
	928 1.74164706550073e-05
	929 1.71691026480403e-05
	930 1.69417417055229e-05
	931 1.6743384549045e-05
	932 1.65777983056614e-05
	933 1.64445791597245e-05
	934 1.63408694788814e-05
	935 1.62629294209182e-05
	936 1.62068481586175e-05
	937 1.61691659741336e-05
	938 1.61469779413892e-05
	939 1.61381813086336e-05
	940 1.61410389409866e-05
	941 1.61542720888974e-05
	942 1.61768693942577e-05
	943 1.62079650181113e-05
	944 1.62465203175088e-05
	945 1.62913893291261e-05
	946 1.63410204550019e-05
	947 1.63932782015763e-05
	948 1.64455741469283e-05
	949 1.64946413860889e-05
	950 1.65368892339757e-05
	951 1.65683795785299e-05
	952 1.65855308296159e-05
	953 1.65851379279047e-05
	954 1.65652909345226e-05
	955 1.65253968589241e-05
	956 1.6466681699967e-05
	957 1.63916247402085e-05
	958 1.63042150234105e-05
	959 1.62088726938237e-05
	960 1.61102925630985e-05
	961 1.60126310220221e-05
	962 1.59194132720586e-05
	963 1.58332204591716e-05
	964 1.57559061335633e-05
	965 1.56886671902612e-05
	966 1.56322512339102e-05
	967 1.55874804477207e-05
	968 1.55550915224012e-05
	969 1.55362613440957e-05
	970 1.55325524247019e-05
	971 1.55462912516668e-05
	972 1.55806155817118e-05
	973 1.56394708028529e-05
	974 1.57281192514347e-05
	975 1.58526472660014e-05
	976 1.60203271661885e-05
	977 1.62390224431874e-05
	978 1.65159290190786e-05
	979 1.68563365150476e-05
	980 1.72590989677701e-05
	981 1.77115816768492e-05
	982 1.81805007741787e-05
	983 1.86035649676342e-05
	984 1.88890189747326e-05
	985 1.89335969480453e-05
	986 1.8667733456823e-05
	987 1.81082505150698e-05
	988 1.7371168723912e-05
	989 1.66190438903868e-05
	990 1.5982015611371e-05
	991 1.55154957610648e-05
	992 1.5211638128676e-05
	993 1.50330033648061e-05
	994 1.4938753338356e-05
	995 1.48962153616594e-05
	996 1.48825793075957e-05
	997 1.48831868500565e-05
	998 1.48890458149253e-05
	999 1.48949320646352e-05
	1000 1.48980288940948e-05
	1001 1.48969766087248e-05
	1002 1.48913368320791e-05
	1003 1.48811277540517e-05
	1004 1.48668295878451e-05
	1005 1.48490453284467e-05
	1006 1.48283388625714e-05
	1007 1.48053195516695e-05
	1008 1.47805685628555e-05
	1009 1.47545633808477e-05
	1010 1.47277623909758e-05
	1011 1.47004402606399e-05
	1012 1.46728925756179e-05
	1013 1.46453585330164e-05
	1014 1.46179936564295e-05
	1015 1.4590968930861e-05
	1016 1.4564354387403e-05
	1017 1.453831919207e-05
	1018 1.45129197335336e-05
	1019 1.44883451866917e-05
	1020 1.44647710840218e-05
	1021 1.44422319863224e-05
	1022 1.4421157175093e-05
	1023 1.44017303682631e-05
	1024 1.43844081321731e-05
	1025 1.43697143357713e-05
	1026 1.43582592500024e-05
	1027 1.43509178087697e-05
	1028 1.43487750392524e-05
	1029 1.43532097354182e-05
	1030 1.43659226523596e-05
	1031 1.43892139021773e-05
	1032 1.44257037391071e-05
	1033 1.44791065395111e-05
	1034 1.45536268973956e-05
	1035 1.46546180985752e-05
	1036 1.47885812111781e-05
	1037 1.49631496242364e-05
	1038 1.51872318383539e-05
	1039 1.54699791892199e-05
	1040 1.58199254656211e-05
	1041 1.62412325153127e-05
	1042 1.67280468303943e-05
	1043 1.7254631529795e-05
	1044 1.77620113390731e-05
	1045 1.81493360287277e-05
	1046 1.82840867637424e-05
	1047 1.80512342922157e-05
	1048 1.74345186678693e-05
	1049 1.65606397786178e-05
	1050 1.56473361130338e-05
	1051 1.48834615174565e-05
	1052 1.43526367537561e-05
	1053 1.40441115945578e-05
	1054 1.39036710606888e-05
	1055 1.38736331791733e-05
	1056 1.3908357686887e-05
	1057 1.39754529300262e-05
	1058 1.4052668120712e-05
	1059 1.41244554470177e-05
	1060 1.41804703162052e-05
	1061 1.42142171171145e-05
	1062 1.42227791002369e-05
	1063 1.42067056003725e-05
	1064 1.41691825774615e-05
	1065 1.41152613650775e-05
	1066 1.40507609103224e-05
	1067 1.39813300847891e-05
	1068 1.39115772981313e-05
	1069 1.38449340738589e-05
	1070 1.37833221742767e-05
	1071 1.37278566398891e-05
	1072 1.36784756250563e-05
	1073 1.36349744934705e-05
	1074 1.35966165544232e-05
	1075 1.35627315103193e-05
	1076 1.35326708914363e-05
	1077 1.35058189698611e-05
	1078 1.3481653695635e-05
	1079 1.34599649754819e-05
	1080 1.34404108393937e-05
	1081 1.34229003379005e-05
	1082 1.34074298330233e-05
	1083 1.33940575324232e-05
	1084 1.33830308186589e-05
	1085 1.33746279971092e-05
	1086 1.3369276530284e-05
	1087 1.33675766846864e-05
	1088 1.33702869788976e-05
	1089 1.33783578348812e-05
	1090 1.33929797812016e-05
	1091 1.34157089632936e-05
	1092 1.34483216243098e-05
	1093 1.34931706270436e-05
	1094 1.35529890030739e-05
	1095 1.36311055030092e-05
	1096 1.37313154482399e-05
	1097 1.38579007398221e-05
	1098 1.40152424137341e-05
	1099 1.42077788041206e-05
	1100 1.44386494866922e-05
	1101 1.47082128023612e-05
	1102 1.50116629811237e-05
	1103 1.53353485075058e-05
	1104 1.56529567902908e-05
	1105 1.59220289788209e-05
	1106 1.60867366503226e-05
	1107 1.60894996952266e-05
	1108 1.58930397446966e-05
	1109 1.55045800056541e-05
	1110 1.49831921589794e-05
	1111 1.44210125654354e-05
	1112 1.39057365231565e-05
	1113 1.34931442516972e-05
	1114 1.32012837639195e-05
	1115 1.30219159473199e-05
	1116 1.29353411466582e-05
	1117 1.292036176892e-05
	1118 1.29587851915858e-05
	1119 1.30356829686207e-05
	1120 1.31386004795786e-05
	1121 1.32559580379166e-05
	1122 1.33759285745327e-05
	1123 1.34863666971796e-05
	1124 1.35747413878562e-05
	1125 1.36300886879326e-05
	1126 1.3644676982949e-05
	1127 1.36161070258822e-05
	1128 1.35477148432983e-05
	1129 1.34482679641224e-05
	1130 1.33298053697217e-05
	1131 1.32044015117572e-05
	1132 1.30826228996739e-05
	1133 1.29715172079159e-05
	1134 1.28749952637008e-05
	1135 1.27939038065961e-05
	1136 1.27274552141898e-05
	1137 1.26738368635415e-05
	1138 1.26309851111728e-05
	1139 1.25969554574112e-05
	1140 1.25701362776454e-05
	1141 1.25492915685754e-05
	1142 1.25335873235599e-05
	1143 1.25225678857532e-05
	1144 1.25161268442753e-05
	1145 1.25143669720273e-05
	1146 1.25177466543391e-05
	1147 1.25269916679827e-05
	1148 1.25431060951087e-05
	1149 1.25674541777698e-05
	1150 1.26016257127048e-05
	1151 1.264773618459e-05
	1152 1.27082257677102e-05
	1153 1.27859111671569e-05
	1154 1.28839938042802e-05
	1155 1.30059297589469e-05
	1156 1.31548940771609e-05
	1157 1.33335988721228e-05
	1158 1.35428808789584e-05
	1159 1.37805782287614e-05
	1160 1.40392712637549e-05
	1161 1.43040970215225e-05
	1162 1.45500653161434e-05
	1163 1.47422606460168e-05
	1164 1.48394819916575e-05
	1165 1.48046219692333e-05
	1166 1.46189686347498e-05
	1167 1.42954468174139e-05
	1168 1.38790719574899e-05
	1169 1.34332076413557e-05
	1170 1.3017661331105e-05
	1171 1.26724280562485e-05
	1172 1.24137759485166e-05
	1173 1.22399624160607e-05
	1174 1.21400789794279e-05
	1175 1.21007142297458e-05
	1176 1.21096327347914e-05
	1177 1.21567691167002e-05
	1178 1.22338315122761e-05
	1179 1.23334475574666e-05
	1180 1.2448174857127e-05
	1181 1.25694332382409e-05
	1182 1.26871909742476e-05
	1183 1.27901539599407e-05
	1184 1.28666770251584e-05
	1185 1.29066556837643e-05
	1186 1.29036634461954e-05
	1187 1.28567589854356e-05
	1188 1.27709381558816e-05
	1189 1.26565091704833e-05
	1190 1.2526127648016e-05
	1191 1.23924173749401e-05
	1192 1.22654937513289e-05
	1193 1.21521588880569e-05
	1194 1.20553540909896e-05
	1195 1.1975701454503e-05
	1196 1.1911954061361e-05
	1197 1.18621610454284e-05
	1198 1.18243106044247e-05
	1199 1.17965691970312e-05
	1200 1.17775716717006e-05
	1201 1.17664558274555e-05
	1202 1.17628887892351e-05
	1203 1.17669851533719e-05
	1204 1.17794361358392e-05
	1205 1.18013413157314e-05
	1206 1.18343969006673e-05
	1207 1.18807547551114e-05
	1208 1.19431533676106e-05
	1209 1.20249815154239e-05
	1210 1.21299763122806e-05
	1211 1.22622977869469e-05
	1212 1.24261732707964e-05
	1213 1.26248341985047e-05
	1214 1.2859868547821e-05
	1215 1.31289152704994e-05
	1216 1.34227857415681e-05
	1217 1.37226879814989e-05
	1218 1.39971116368542e-05
	1219 1.42016351674101e-05
	1220 1.42863600558485e-05
	1221 1.42097978823585e-05
	1222 1.39588828460546e-05
	1223 1.35626787596266e-05
	1224 1.30867238112842e-05
	1225 1.26088052638806e-05
	1226 1.21917955766548e-05
	1227 1.18685156849097e-05
	1228 1.16443679871736e-05
	1229 1.1508192983456e-05
	1230 1.14427730295574e-05
	1231 1.14313697849866e-05
	1232 1.1460059795354e-05
	1233 1.15178227133583e-05
	1234 1.15959101094631e-05
	1235 1.16866640382796e-05
	1236 1.17827012218186e-05
	1237 1.18762991405674e-05
	1238 1.19597480079392e-05
	1239 1.2025347132294e-05
	1240 1.20662580229691e-05
	1241 1.20778749987949e-05
	1242 1.20586728371563e-05
	1243 1.20104068628279e-05
	1244 1.19381647891714e-05
	1245 1.18489952001255e-05
	1246 1.17510080599459e-05
	1247 1.16515266199713e-05
	1248 1.15564089355757e-05
	1249 1.14695758384187e-05
	1250 1.13930527732009e-05
	1251 1.13275127660017e-05
	1252 1.12726547740749e-05
	1253 1.12276402433054e-05
	1254 1.11915869638324e-05
	1255 1.11635908979224e-05
	1256 1.11430663309875e-05
	1257 1.11298277261085e-05
	1258 1.1123943295388e-05
	1259 1.11260242192657e-05
	1260 1.11372128230869e-05
	1261 1.11590215965407e-05
	1262 1.11938115878729e-05
	1263 1.12445723061683e-05
	1264 1.13151099867537e-05
	1265 1.14101940198452e-05
	1266 1.15354878289509e-05
	1267 1.16975479613757e-05
	1268 1.1903171071026e-05
	1269 1.2158535355411e-05
	1270 1.2467198757804e-05
	1271 1.28269848573836e-05
	1272 1.32241902974783e-05
	1273 1.36279477374046e-05
	1274 1.3983547432872e-05
	1275 1.42143289849628e-05
	1276 1.42380013130605e-05
	1277 1.40024758366053e-05
	1278 1.35239088194794e-05
	1279 1.289601004828e-05
	1280 1.22512738016667e-05
	1281 1.17006466098246e-05
	1282 1.12983552753576e-05
	1283 1.10458859126084e-05
	1284 1.09166576294228e-05
	1285 1.08775102489744e-05
	1286 1.08993226604071e-05
	1287 1.09594957393711e-05
	1288 1.10412938738591e-05
	1289 1.11315785034094e-05
	1290 1.12196066766046e-05
	1291 1.12964953586925e-05
	1292 1.13549585876171e-05
	1293 1.13896594484686e-05
	1294 1.13979558591382e-05
	1295 1.13799014798133e-05
	1296 1.13385094664409e-05
	1297 1.12787847683649e-05
	1298 1.12068692033063e-05
	1299 1.11291046778206e-05
	1300 1.10508435682277e-05
	1301 1.09761194835301e-05
	1302 1.09075408545323e-05
	1303 1.08463655124069e-05
	1304 1.07929372461513e-05
	1305 1.07469259091886e-05
	1306 1.07077339634998e-05
	1307 1.0674638360797e-05
	1308 1.06469433376333e-05
	1309 1.06241313915234e-05
	1310 1.06058805613429e-05
	1311 1.05921517388197e-05
	1312 1.05829640233424e-05
	1313 1.05788403743645e-05
	1314 1.05804620034178e-05
	1315 1.05889503174694e-05
	1316 1.06058687379118e-05
	1317 1.0633309102559e-05
	1318 1.06739344118978e-05
	1319 1.07313553598942e-05
	1320 1.08099602584844e-05
	1321 1.09152742879814e-05
	1322 1.10539549496025e-05
	1323 1.12333209472126e-05
	1324 1.14616150312941e-05
	1325 1.17459712782875e-05
	1326 1.20903205242939e-05
	1327 1.24913021863904e-05
	1328 1.29313093566452e-05
	1329 1.336976129096e-05
	1330 1.37372999233776e-05
	1331 1.39387975650607e-05
	1332 1.3880593542126e-05
	1333 1.35197742565651e-05
	1334 1.29076452139998e-05
	1335 1.21820212370949e-05
	1336 1.15016155177727e-05
	1337 1.09733846329618e-05
	1338 1.06296301964903e-05
	1339 1.04500049928902e-05
	1340 1.03946940726019e-05
	1341 1.04246173577849e-05
	1342 1.05083017842844e-05
	1343 1.06214874904254e-05
	1344 1.07447112895898e-05
	1345 1.08616432044073e-05
	1346 1.09582651930396e-05
	1347 1.1023432307411e-05
	1348 1.10501514427597e-05
	1349 1.10368855530396e-05
	1350 1.09874017653055e-05
	1351 1.09099582914496e-05
	1352 1.08151944004931e-05
	1353 1.07137666418566e-05
	1354 1.06145480458508e-05
	1355 1.05234221337014e-05
	1356 1.04434684544685e-05
	1357 1.03755955933593e-05
	1358 1.03191450762097e-05
	1359 1.02728045021649e-05
	1360 1.02349804365076e-05
	1361 1.02041212812765e-05
	1362 1.01790019471082e-05
	1363 1.01586574601242e-05
	1364 1.01423465821426e-05
	1365 1.01296418506536e-05
	1366 1.01204468592186e-05
	1367 1.01148107205518e-05
	1368 1.01130071925581e-05
	1369 1.01156720120343e-05
	1370 1.012353823171e-05
	1371 1.01378254839801e-05
	1372 1.01601108326577e-05
	1373 1.0192390618613e-05
	1374 1.02372150649899e-05
	1375 1.02979192888597e-05
	1376 1.03784777820692e-05
	1377 1.04839291452663e-05
	1378 1.06197812783648e-05
	1379 1.07923688119627e-05
	1380 1.10080891317921e-05
	1381 1.12718444142956e-05
	1382 1.15851435111836e-05
	1383 1.19419582915725e-05
	1384 1.23237014122424e-05
	1385 1.26930244732648e-05
	1386 1.29896880025626e-05
	1387 1.31376382341841e-05
	1388 1.30660873765009e-05
	1389 1.27460161820636e-05
	1390 1.22203600767534e-05
	1391 1.15962457130081e-05
	1392 1.09990669443505e-05
	1393 1.05196641015937e-05
	1394 1.01938994703232e-05
	1395 1.00149627542123e-05
	1396 9.95633126876783e-06
	1397 9.98843734123511e-06
	1398 1.00853949334123e-05
	1399 1.02256162790582e-05
	1400 1.03894672065508e-05
	1401 1.05573981272755e-05
	1402 1.0709069101722e-05
	1403 1.08246622403385e-05
	1404 1.08877911770833e-05
	1405 1.08892800199101e-05
	1406 1.08302892840584e-05
	1407 1.07219857454766e-05
	1408 1.05827521110768e-05
	1409 1.0432445378683e-05
	1410 1.02877302197157e-05
	1411 1.01593768704333e-05
	1412 1.0052224752144e-05
	1413 9.96646213025087e-06
	1414 9.89974978438113e-06
	1415 9.84872713161167e-06
	1416 9.80999266175786e-06
	1417 9.78073967417004e-06
	1418 9.75867624219973e-06
	1419 9.74228169070557e-06
	1420 9.7305000963388e-06
	1421 9.72273574006977e-06
	1422 9.71871759247733e-06
	1423 9.71854387898929e-06
	1424 9.72250199993141e-06
	1425 9.73127680481412e-06
	1426 9.74562135525048e-06
	1427 9.7668707894627e-06
	1428 9.79654123511864e-06
	1429 9.83659720077412e-06
	1430 9.88960346148815e-06
	1431 9.95848677121103e-06
	1432 1.00469687822624e-05
	1433 1.01591504062526e-05
	1434 1.02995127235772e-05
	1435 1.04723758340697e-05
	1436 1.06813440652331e-05
	1437 1.09273487396422e-05
	1438 1.12067455120268e-05
	1439 1.1507933777466e-05
	1440 1.18078150990186e-05
	1441 1.2069122021785e-05
	1442 1.22419351100689e-05
	1443 1.22742494568229e-05
	1444 1.21291332106921e-05
	1445 1.18079942694749e-05
	1446 1.13584555947455e-05
	1447 1.08611120595015e-05
	1448 1.03979791674647e-05
	1449 1.00258530437713e-05
	1450 9.76795490714721e-06
	1451 9.62224748946028e-06
	1452 9.57451902650064e-06
	1453 9.60776833380805e-06
	1454 9.70647124631796e-06
	1455 9.85664701147471e-06
	1456 1.00440101959975e-05
	1457 1.02515596154262e-05
	1458 1.04578366517671e-05
	1459 1.06367633634363e-05
	1460 1.0761124030978e-05
	1461 1.08086460386403e-05
	1462 1.07692621895694e-05
	1463 1.0649923751771e-05
	1464 1.04732862382662e-05
	1465 1.0270233360643e-05
	1466 1.00699080576305e-05
	1467 9.89257659966825e-06
	1468 9.747836884344e-06
	1469 9.63659385888604e-06
	1470 9.55462564888876e-06
	1471 9.49604145716876e-06
	1472 9.45511419558898e-06
	1473 9.42715723795118e-06
	1474 9.40866266319063e-06
	1475 9.39735855354229e-06
	1476 9.39178426051512e-06
	1477 9.39118399401195e-06
	1478 9.39534129429376e-06
	1479 9.40442623686977e-06
	1480 9.41894631978357e-06
	1481 9.43975919653894e-06
	1482 9.46804539125878e-06
	1483 9.50537014432484e-06
	1484 9.55350697040558e-06
	1485 9.61483510764083e-06
	1486 9.6919238785631e-06
	1487 9.78760272118961e-06
	1488 9.90493390418123e-06
	1489 1.00465813375195e-05
	1490 1.02145722848945e-05
	1491 1.04088394436985e-05
	1492 1.06261277323938e-05
	1493 1.08579652078333e-05
	1494 1.1088994142483e-05
	1495 1.12949410322472e-05
	1496 1.14443782877061e-05
	1497 1.15032044050167e-05
	1498 1.1443780749687e-05
	1499 1.12571979116183e-05
	1500 1.096103915188e-05
	1501 1.05974586404045e-05
	1502 1.02201893241727e-05
	1503 9.87765633908566e-06
	1504 9.6018420663313e-06
	1505 9.40645804803353e-06
	1506 9.29200268728891e-06
	1507 9.25238691706909e-06
	1508 9.2798400146421e-06
	1509 9.36739434109768e-06
	1510 9.50868525251281e-06
	1511 9.69651591731235e-06
	1512 9.92027526081074e-06
	1513 1.01632485893788e-05
	1514 1.04005530374707e-05
	1515 1.05996386992047e-05
	1516 1.07257501440472e-05
	1517 1.07508121800493e-05
	1518 1.06651114037959e-05
	1519 1.04834361991379e-05
	1520 1.02416524896398e-05
	1521 9.98318046185886e-06
	1522 9.74451450019842e-06
	1523 9.54698134592036e-06
	1524 9.39658002607757e-06
	1525 9.28924055187963e-06
	1526 9.21653827390401e-06
	1527 9.16972749109846e-06
	1528 9.14154770725872e-06
	1529 9.12667655939003e-06
	1530 9.12164796318393e-06
	1531 9.12446921574883e-06
	1532 9.13411531655584e-06
	1533 9.15042983251624e-06
	1534 9.17363922781078e-06
	1535 9.20472575671738e-06
	1536 9.244750799553e-06
	1537 9.29534144233912e-06
	1538 9.3582211775356e-06
	1539 9.43543182074791e-06
	1540 9.52907248574775e-06
	1541 9.64095306699164e-06
	1542 9.77242507360643e-06
	1543 9.92364675767021e-06
	1544 1.00926772574894e-05
	1545 1.02744770629215e-05
	1546 1.04598575489945e-05
	1547 1.06342404251336e-05
	1548 1.07781397673534e-05
	1549 1.08686253952328e-05
	1550 1.08838612504769e-05
	1551 1.08094409370096e-05
	1552 1.06445140772848e-05
	1553 1.04050423033186e-05
	1554 1.0120680599357e-05
	1555 9.8267892099102e-06
	1556 9.55530958890449e-06
	1557 9.32835064304527e-06
	1558 9.15735927264905e-06
	1559 9.04532498680055e-06
	1560 8.99063161341473e-06
	1561 8.99000770004932e-06
	1562 9.04063381312881e-06
	1563 9.14041629584972e-06
	1564 9.28756162466016e-06
	1565 9.47879289014963e-06
	1566 9.70726796367671e-06
	1567 9.95926802715985e-06
	1568 1.02118001450435e-05
	1569 1.0432634553581e-05
	1570 1.05837370938389e-05
	1571 1.06312427305966e-05
	1572 1.05580775198177e-05
	1573 1.03739394035074e-05
	1574 1.01144560176181e-05
	1575 9.8285327112535e-06
	1576 9.56017902353778e-06
	1577 9.33690080273664e-06
	1578 9.16797034733463e-06
	1579 9.04968510440085e-06
	1580 8.97266818356002e-06
	1581 8.92672960617347e-06
	1582 8.90335741132731e-06
	1583 8.89660623215605e-06
	1584 8.9026880232268e-06
	1585 8.91951913217781e-06
	1586 8.94644017535029e-06
	1587 8.98369853530312e-06
	1588 9.03225281945197e-06
	1589 9.09353184397332e-06
	1590 9.16935368877603e-06
	1591 9.26163284020731e-06
	1592 9.37221375352237e-06
	1593 9.50225967244478e-06
	1594 9.65186791290762e-06
	1595 9.8189366326551e-06
	1596 9.99845178739633e-06
	1597 1.0181094694417e-05
	1598 1.03524307633052e-05
	1599 1.04933897091541e-05
	1600 1.05817325675162e-05
	1601 1.05968911157106e-05
	1602 1.05256021925015e-05
	1603 1.03678994491929e-05
	1604 1.01396226455108e-05
	1605 9.86928625934524e-06
	1606 9.59028147917707e-06
	1607 9.33219962462317e-06
	1608 9.11551160243107e-06
	1609 8.95029916136991e-06
	1610 8.83878237800673e-06
	1611 8.77853290148778e-06
	1612 8.76564990903717e-06
	1613 8.79640901985113e-06
	1614 8.86798625288066e-06
	1615 8.97827430890175e-06
	1616 9.12486211746e-06
	1617 9.30366059037624e-06
	1618 9.50682351685828e-06
	1619 9.72127327258931e-06
	1620 9.92691821011249e-06
	1621 1.00979204944451e-05
	1622 1.02064723250805e-05
	1623 1.02294643511414e-05
	1624 1.01577143141185e-05
	1625 1.00007455330342e-05
	1626 9.78545631369343e-06
	1627 9.5475152193103e-06
	1628 9.31955855776323e-06
	1629 9.12358882487752e-06
	1630 8.96890924195759e-06
	1631 8.85540794115514e-06
	1632 8.77773618412903e-06
	1633 8.72912460181396e-06
	1634 8.70351959747495e-06
	1635 8.69633367983624e-06
	1636 8.70481744641438e-06
	1637 8.72776672622422e-06
	1638 8.76527428772533e-06
	1639 8.81854248291347e-06
	1640 8.8897077148431e-06
	1641 8.98155394679634e-06
	1642 9.09740629140288e-06
	1643 9.24039159144741e-06
	1644 9.41306734603131e-06
	1645 9.61613477556966e-06
	1646 9.84677535598166e-06
	1647 1.00962988653919e-05
	1648 1.03482097983942e-05
	1649 1.05760054793791e-05
	1650 1.07443265733309e-05
	1651 1.08152016764507e-05
	1652 1.07586010926752e-05
	1653 1.05664530565264e-05
	1654 1.02611447800882e-05
	1655 9.89119780570036e-06
	1656 9.51490892475704e-06
	1657 9.18156820262084e-06
	1658 8.91998570295982e-06
	1659 8.73880799190374e-06
	1660 8.63334480527556e-06
	1661 8.59283318277448e-06
	1662 8.60582986206282e-06
	1663 8.66217305883765e-06
	1664 8.75347359396983e-06
	1665 8.87220448930748e-06
	1666 9.01074599823914e-06
	1667 9.16000590223121e-06
	1668 9.30868554860353e-06
	1669 9.44334715313744e-06
	1670 9.54922688833904e-06
	1671 9.61277328315191e-06
	1672 9.62470767262857e-06
	1673 9.58269993134309e-06
	1674 9.492472599959e-06
	1675 9.3672506409348e-06
	1676 9.22353956411825e-06
	1677 9.07752837520093e-06
	1678 8.94176810106728e-06
	1679 8.82388485479169e-06
	1680 8.72692726261448e-06
	1681 8.65077072376153e-06
	1682 8.59358169691404e-06
	1683 8.5529300122289e-06
	1684 8.52661105454899e-06
	1685 8.51304230309324e-06
	1686 8.5114279499976e-06
	1687 8.52185530675342e-06
	1688 8.54520021675853e-06
	1689 8.58338898979127e-06
	1690 8.6391655713669e-06
	1691 8.71640531840967e-06
	1692 8.81989853951382e-06
	1693 8.95539960765745e-06
	1694 9.12915857043117e-06
	1695 9.34725721890572e-06
	1696 9.61365003604442e-06
	1697 9.92792683973676e-06
	1698 1.02802896435605e-05
	1699 1.06467468867777e-05
	1700 1.09836273622932e-05
	1701 1.1226888091187e-05
	1702 1.13047526610899e-05
	1703 1.11631452455185e-05
	1704 1.08002068373025e-05
	1705 1.02812782643014e-05
	1706 9.71663575910497e-06
	1707 9.21263290365459e-06
	1708 8.833020729071e-06
	1709 8.59352348925313e-06
	1710 8.47852788865566e-06
	1711 8.46103375806706e-06
	1712 8.51453569339355e-06
	1713 8.61673743202118e-06
	1714 8.74912529980065e-06
	1715 8.89531474967953e-06
	1716 9.0395533334231e-06
	1717 9.16629596758867e-06
	1718 9.26120992517099e-06
	1719 9.31319664232433e-06
	1720 9.31665726966457e-06
	1721 9.27304427023046e-06
	1722 9.19065678317565e-06
	1723 9.0824878498097e-06
	1724 8.96293204277754e-06
	1725 8.84446035342989e-06
	1726 8.73596036399249e-06
	1727 8.64211324369535e-06
	1728 8.56444694363745e-06
	1729 8.50221749715274e-06
	1730 8.45364138513105e-06
	1731 8.41659584693843e-06
	1732 8.38923006085679e-06
	1733 8.36995059216861e-06
	1734 8.35778428154299e-06
	1735 8.35215541883372e-06
	1736 8.35301398183219e-06
	1737 8.36073741083965e-06
	1738 8.37627067085123e-06
	1739 8.40097254695138e-06
	1740 8.43686666485155e-06
	1741 8.48671879793983e-06
	1742 8.55390226206509e-06
	1743 8.64275170897599e-06
	1744 8.75850128068123e-06
	1745 8.9070481408271e-06
	1746 9.09451500774594e-06
	1747 9.32613238546764e-06
	1748 9.60495435720077e-06
	1749 9.92793229670497e-06
	1750 1.02816102298675e-05
	1751 1.06368170236237e-05
	1752 1.09440734377131e-05
	1753 1.11360714072362e-05
	1754 1.11433555503027e-05
	1755 1.09248530861805e-05
	1756 1.04983901110245e-05
	1757 9.94685251498595e-06
	1758 9.38563061936293e-06
	1759 8.91311719897203e-06
	1760 8.58013936522184e-06
	1761 8.39278709463542e-06
	1762 8.33187550597358e-06
	1763 8.37082188809291e-06
	1764 8.48437503009336e-06
	1765 8.65020410856232e-06
	1766 8.84692963154521e-06
	1767 9.051813322003e-06
	1768 9.23950210562907e-06
	1769 9.38377615966601e-06
	1770 9.46219734032638e-06
	1771 9.4618999355589e-06
	1772 9.38482298806775e-06
	1773 9.24758478504373e-06
	1774 9.07622870727209e-06
	1775 8.89788952918025e-06
	1776 8.7334537965944e-06
	1777 8.59457486512838e-06
	1778 8.48450054036221e-06
	1779 8.40114898892352e-06
	1780 8.34008369565709e-06
	1781 8.29630153020844e-06
	1782 8.26553605293157e-06
	1783 8.24452945380472e-06
	1784 8.23089249024633e-06
	1785 8.22318452264881e-06
	1786 8.22050060378388e-06
	1787 8.2224623838556e-06
	1788 8.22904348751763e-06
	1789 8.24074959382415e-06
	1790 8.25820370664587e-06
	1791 8.28252996143419e-06
	1792 8.3151580838603e-06
	1793 8.35795799503103e-06
	1794 8.41321889311075e-06
	1795 8.48382387630409e-06
	1796 8.57305894896854e-06
	1797 8.68467213877011e-06
	1798 8.8225060608238e-06
	1799 8.98998314369237e-06
	1800 9.18945352168521e-06
	1801 9.42023689276539e-06
	1802 9.67664982454153e-06
	1803 9.94501351669896e-06
	1804 1.02014319054433e-05
	1805 1.04099781310651e-05
	1806 1.05269728010171e-05
	1807 1.05110675576725e-05
	1808 1.03398197097704e-05
	1809 1.00252073025331e-05
	1810 9.61648038355634e-06
	1811 9.18405748961959e-06
	1812 8.79498202266404e-06
	1813 8.49365733301966e-06
	1814 8.29804048407823e-06
	1815 8.20739023765782e-06
	1816 8.21194953459781e-06
	1817 8.30004773888504e-06
	1818 8.46031889523147e-06
	1819 8.68038205226185e-06
	1820 8.94370623427676e-06
	1821 9.22552408155752e-06
	1822 9.49083550949581e-06
	1823 9.69647135207197e-06
	1824 9.80013737716945e-06
	1825 9.7750807981356e-06
	1826 9.62413287197705e-06
	1827 9.38157972996123e-06
	1828 9.10032395040616e-06
	1829 8.83038501342526e-06
	1830 8.60383624967653e-06
	1831 8.43178986542625e-06
	1832 8.31036504678195e-06
	1833 8.22914807940833e-06
	1834 8.1770622273325e-06
	1835 8.14502254797844e-06
	1836 8.12652160675498e-06
	1837 8.11732570582535e-06
	1838 8.11482095741667e-06
	1839 8.1176085586776e-06
	1840 8.12500547908712e-06
	1841 8.13690530776512e-06
	1842 8.15354542282876e-06
	1843 8.17548607301433e-06
	1844 8.20355580799514e-06
	1845 8.23883510747692e-06
	1846 8.28263000585139e-06
	1847 8.33646936371224e-06
	1848 8.40217307995772e-06
	1849 8.48155559651786e-06
	1850 8.57657687447499e-06
	1851 8.68877577886451e-06
	1852 8.81897631188622e-06
	1853 8.96702658792492e-06
	1854 9.13010990188923e-06
	1855 9.30268288357183e-06
	1856 9.47460466704797e-06
	1857 9.63106776907807e-06
	1858 9.7528127298574e-06
	1859 9.8180162240169e-06
	1860 9.80759523372399e-06
	1861 9.71047666098457e-06
	1862 9.5302802947117e-06
	1863 9.28520967136137e-06
	1864 9.00631766853621e-06
	1865 8.72789314598776e-06
	1866 8.47978299134411e-06
	1867 8.282585440611e-06
	1868 8.14754002931295e-06
	1869 8.07890864962246e-06
	1870 8.07796277513262e-06
	1871 8.14536815596512e-06
	1872 8.28237625682959e-06
	1873 8.48984836920863e-06
	1874 8.76521880854852e-06
	1875 9.09791106096236e-06
	1876 9.46270120039117e-06
	1877 9.81435732683167e-06
	1878 1.00886518339394e-05
	1879 1.0215784641332e-05
	1880 1.0148194633075e-05
	1881 9.89074305834947e-06
	1882 9.50712819758337e-06
	1883 9.0917737907148e-06
	1884 8.72502914717188e-06
	1885 8.44686746859225e-06
	1886 8.25940151116811e-06
	1887 8.1446432886878e-06
	1888 8.08055119705386e-06
	1889 8.04892897576792e-06
	1890 8.03743023425341e-06
	1891 8.03848070063395e-06
	1892 8.04790852271253e-06
	1893 8.06359366833931e-06
	1894 8.08463846624363e-06
	1895 8.11096924735466e-06
	1896 8.14283885119949e-06
	1897 8.18074204289587e-06
	1898 8.22546007839264e-06
	1899 8.27749045129167e-06
	1900 8.33764170238283e-06
	1901 8.40628581499914e-06
	1902 8.48357740323991e-06
	1903 8.56920269143302e-06
	1904 8.66201480675954e-06
	1905 8.75973546499154e-06
	1906 8.85897588887019e-06
	1907 8.95459743333049e-06
	1908 9.04002172319451e-06
	1909 9.10740891413298e-06
	1910 9.14860720513389e-06
	1911 9.15625332709169e-06
	1912 9.12516043172218e-06
	1913 9.05398792383494e-06
	1914 8.94622007763246e-06
	1915 8.80964853422483e-06
	1916 8.65553738549352e-06
	1917 8.4963457993581e-06
	1918 8.34389174997341e-06
	1919 8.20804325485369e-06
	1920 8.09621087682899e-06
	1921 8.01370424596826e-06
	1922 7.96465883468045e-06
	1923 7.9530418588547e-06
	1924 7.98359087639255e-06
	1925 8.06251773610711e-06
	1926 8.19752767711179e-06
	1927 8.39700533106225e-06
	1928 8.66793652676279e-06
	1929 9.01146722753765e-06
	1930 9.41635789786233e-06
	1931 9.8496175269247e-06
	1932 1.02485546449316e-05
	1933 1.05238968899357e-05
	1934 1.05833960333257e-05
	1935 1.03786123872851e-05
	1936 9.946931641025e-06
	1937 9.40567861107411e-06
	1938 8.89070634002564e-06
	1939 8.49012303660857e-06
	1940 8.22564925329061e-06
	1941 8.07536707725376e-06
	1942 8.00458201410947e-06
	1943 7.98360997578129e-06
	1944 7.99252120486926e-06
	1945 8.01963142293971e-06
	1946 8.05904528533574e-06
	1947 8.10792971606134e-06
	1948 8.16523970570415e-06
	1949 8.23056052468019e-06
	1950 8.30362751003122e-06
	1951 8.38382220536005e-06
	1952 8.4700332081411e-06
	1953 8.56007773109013e-06
	1954 8.65076253830921e-06
	1955 8.73756198416231e-06
	1956 8.81492996995803e-06
	1957 8.87639635038795e-06
	1958 8.91527179192053e-06
	1959 8.92606203706237e-06
	1960 8.90490809979383e-06
	1961 8.85120334714884e-06
	1962 8.76772355695721e-06
	1963 8.6604541138513e-06
	1964 8.5378487710841e-06
	1965 8.4090361269773e-06
	1966 8.28284555609571e-06
	1967 8.16645024315221e-06
	1968 8.06493153504562e-06
	1969 7.98152450443013e-06
	1970 7.91818820289336e-06
	1971 7.87592398410197e-06
	1972 7.85570591688156e-06
	1973 7.85873544373317e-06
	1974 7.88708348409273e-06
	1975 7.94362404121784e-06
	1976 8.03221246314934e-06
	1977 8.15733164927224e-06
	1978 8.32335445011267e-06
	1979 8.53305573400576e-06
	1980 8.78545506566297e-06
	1981 9.07229605218163e-06
	1982 9.37385266297497e-06
	1983 9.65649178397143e-06
	1984 9.87296334642451e-06
	1985 9.9715043688775e-06
	1986 9.91420438367641e-06
	1987 9.69781831372529e-06
	1988 9.36267497309018e-06
	1989 8.97909194463864e-06
	1990 8.61790704220766e-06
	1991 8.32538353279233e-06
	1992 8.1171447163797e-06
	1993 7.98722521722084e-06
	1994 7.92063019616762e-06
	1995 7.90261583460961e-06
	1996 7.92230912338709e-06
	1997 7.97349912318168e-06
	1998 8.05360014055623e-06
	1999 8.16281681181863e-06
};
\addlegendentry{Test}

\nextgroupplot[
title={Tanh/Tanh},
ymin=5.31543784038883e-06, ymax=0.001,
]
\addplot [semithick, black, dashed]
table {%
	0 0.0142884069064166
	1 0.0138785560266115
	2 0.0134795071935514
	3 0.0130875279864995
	4 0.0126996813633014
	5 0.0123137547780061
	6 0.0119286776462104
	7 0.0115444746770663
	8 0.0111620487296022
	9 0.0107829777989537
	10 0.010409117974632
	11 0.0100419410591712
	12 0.00968185131932842
	13 0.00932782390009379
	14 0.00897751020966098
	15 0.00862786934158066
	16 0.00827648986160057
	17 0.0079231485069613
	18 0.00757021203753538
	19 0.00722162748570554
	20 0.00688154521776596
	21 0.00655335559349624
	22 0.00623936058400432
	23 0.00594084686235874
	24 0.00565819995244965
	25 0.00539105243660742
	26 0.0051385126025707
	27 0.00489945672234171
	28 0.00467280152952299
	29 0.00445766999655461
	30 0.0042534147924016
	31 0.00405954653615481
	32 0.00387564046832267
	33 0.00370127271889942
	34 0.00353600414928223
	35 0.00337938980555919
	36 0.00323099473644106
	37 0.00309039827152446
	38 0.00295719425776042
	39 0.00283099179250712
	40 0.00271141305165656
	41 0.00259809508952458
	42 0.00249068928042107
	43 0.00238886311126407
	44 0.0022922995940462
	45 0.00220069913120824
	46 0.00211377706455096
	47 0.00203126471478754
	48 0.00195290874125931
	49 0.00187847066899849
	50 0.00180772616840841
	51 0.00174046452957555
	52 0.00167648772094253
	53 0.0016156099136424
	54 0.00155765714680456
	55 0.00150246563407563
	56 0.00144988237298094
	57 0.00139976362925154
	58 0.0013519747758437
	59 0.00130638936570904
	60 0.00126288890260184
	61 0.00122136211894031
	62 0.00118170476298474
	63 0.00114381866114854
	64 0.00110761194946463
	65 0.00107299796172811
	66 0.00103989576473396
	67 0.00100822866625094
	68 0.000977924698872812
	69 0.000948916144807299
	70 0.000921139262572979
	71 0.000894533856353519
	72 0.000869043470174802
	73 0.00084461463484331
	74 0.000821196739707375
	75 0.000798742215010861
	76 0.00077720603076159
	77 0.000756545653530338
	78 0.000736720959366721
	79 0.00071769368423702
	80 0.00069942780578458
	81 0.000681889380075518
	82 0.000665045883351922
	83 0.00064886654968177
	84 0.000633322434396177
	85 0.00061838577562412
	86 0.000604030502017849
	87 0.000590231419892007
	88 0.000576965072241364
	89 0.000564208907917418
	90 0.000551941431695013
	91 0.000540142335466953
	92 0.00052879193458466
	93 0.000517872014142995
	94 0.000507364739178229
	95 0.000497253409093901
	96 0.000487521856939566
	97 0.000478154808206455
	98 0.000469137634354411
	99 0.000460456403516218
	100 0.000452097558081732
	101 0.000444048487452164
	102 0.000436296939142267
	103 0.000428831195222301
	104 0.000421640015247249
	105 0.000414712672863971
	106 0.000408038860769011
	107 0.000401608752554239
	108 0.000395412915850102
	109 0.000389442182722632
	110 0.000383687909788932
	111 0.000378141627379591
	112 0.000372795336488707
	113 0.00036764129004041
	114 0.000362672108849438
	115 0.000357880620640572
	116 0.00035325992757862
	117 0.000348803413317
	118 0.000344504698432502
	119 0.000340357702839356
	120 0.000336356430807427
	121 0.000332495219822704
	122 0.000328768699887405
	123 0.000325171458143814
	124 0.000321698436209772
	125 0.000318344755328326
	126 0.000315105763775136
	127 0.000311976784360013
	128 0.000308953470721463
	129 0.000306031578134025
	130 0.000303207111528536
	131 0.0003004760640124
	132 0.000297834654816143
	133 0.000295279205374754
	134 0.000292806234369891
	135 0.000290412387357719
	136 0.000288094339680356
	137 0.000285848898670338
	138 0.000283673142234875
	139 0.000281564096894726
	140 0.000279518935144552
	141 0.000277534953283975
	142 0.000275609528671339
	143 0.00027374020123716
	144 0.000271924531944023
	145 0.00027016016053949
	146 0.000268444907419507
	147 0.000266776541252511
	148 0.000265153027726228
	149 0.00026357247395481
	150 0.000262032836872095
	151 0.000260532296920246
	152 0.000259069133107914
	153 0.000257641634334504
	154 0.000256248109963053
	155 0.000254887104006229
	156 0.000253557011205885
	157 0.000252256457201838
	158 0.000250983999194432
	159 0.000249738320803772
	160 0.000248518172838885
	161 0.000247322297923347
	162 0.000246149559302467
	163 0.000244998731318447
	164 0.000243868862526142
	165 0.00024275881446556
	166 0.000241667559009784
	167 0.000240594294439234
	168 0.000239537949255464
	169 0.000238497747147903
	170 0.000237472801757121
	171 0.000236462252416914
	172 0.000235465381365429
	173 0.000234481435825273
	174 0.000233509719919311
	175 0.000232549533791371
	176 0.000231600196173076
	177 0.000230661076670913
	178 0.000229731628223817
	179 0.00022881118911755
	180 0.000227899328933745
	181 0.000226995364073446
	182 0.000226098881455528
	183 0.000225209293773787
	184 0.000224326256102358
	185 0.000223449232919393
	186 0.00022257781239432
	187 0.000221711601000152
	188 0.0002208500419556
	189 0.000219992996449037
	190 0.000219139951752823
	191 0.000218290544466981
	192 0.0002174444754246
	193 0.000216601440342856
	194 0.000215761074571219
	195 0.000214923092755726
	196 0.000214087209485569
	197 0.00021325315157128
	198 0.000212420651450884
	199 0.000211589412344892
	200 0.00021075920955127
	201 0.000209929811092024
	202 0.000209101040695714
	203 0.000208272627162387
	204 0.000207444367561038
	205 0.000206616104435398
	206 0.000205787510481059
	207 0.000204958601159433
	208 0.000204129059056868
	209 0.000203298769577032
	210 0.000202467583449106
	211 0.000201635346115836
	212 0.000200801882783708
	213 0.000199967103526433
	214 0.000199130858334229
	215 0.000198293036817176
	216 0.000197453552175375
	217 0.000196612257923334
	218 0.000195769059018858
	219 0.000194923908566125
	220 0.00019407667446103
	221 0.00019322731640159
	222 0.0001923757834561
	223 0.000191522013551548
	224 0.000190665940010604
	225 0.000189807482740889
	226 0.000188946675166335
	227 0.000188083471698519
	228 0.000187217830557529
	229 0.000186349754358162
	230 0.000185479287779344
	231 0.000184606390519093
	232 0.0001837311123154
	233 0.000182853423893903
	234 0.000181973399094204
	235 0.000181091099193509
	236 0.000180206544712291
	237 0.000179319765862829
	238 0.000178430906430549
	239 0.000177540011407018
	240 0.000176647144627395
	241 0.000175752430664033
	242 0.000174855945829222
	243 0.00017395780815832
	244 0.000173058175249707
	245 0.000172157101388848
	246 0.000171254783197128
	247 0.000170351341864716
	248 0.000169446903129256
	249 0.000168541648577047
	250 0.00016763572182299
	251 0.000166729266794619
	252 0.000165822533915616
	253 0.000164915610582739
	254 0.000164008711351471
	255 0.000163102037049612
	256 0.000162195744735527
	257 0.000161290065960884
	258 0.000160385136467767
	259 0.000159481212648416
	260 0.000158578462105652
	261 0.000157677078732377
	262 0.000156777308831124
	263 0.000155879328190167
	264 0.000154983359436756
	265 0.000154089589784689
	266 0.000153198295407719
	267 0.000152309611763712
	268 0.00015142378046562
	269 0.000150541045428554
	270 0.00014966159017149
	271 0.000148785650480932
	272 0.000147913429231039
	273 0.00014704515032804
	274 0.000146180989474942
	275 0.000145321229410911
	276 0.000144465997664156
	277 0.000143615556510213
	278 0.000142770089524902
	279 0.000141929810155261
	280 0.000141094885890425
	281 0.000140265536600737
	282 0.000139441909766447
	283 0.000138624233130713
	284 0.000137812659090741
	285 0.000137007317874804
	286 0.000136208431626983
	287 0.000135416093968388
	288 0.00013463047915252
	289 0.000133851686769049
	290 0.000133079875411113
	291 0.000132315144867334
	292 0.000131557583159747
	293 0.000130807279958844
	294 0.000130064356852699
	295 0.000129328812533913
	296 0.000128600748922736
	297 0.000127880204075836
	298 0.00012716719470518
	299 0.000126461765916019
	300 0.000125763866719808
	301 0.000125073567801337
	302 0.000124390781479633
	303 0.000123715530151003
	304 0.000123047729374548
	305 0.000122387339786201
	306 0.000121734291610665
	307 0.000121088489521526
	308 0.000120449866415129
	309 0.000119818295758023
	310 0.000119193683843832
	311 0.000118575882112282
	312 0.000117964784550395
	313 0.000117360225601715
	314 0.00011676204098876
	315 0.000116170093917844
	316 0.000115584197601493
	317 0.000115004176080902
	318 0.000114429868347088
	319 0.000113861060043519
	320 0.000113297554264591
	321 0.000112739163597553
	322 0.000112185687243027
	323 0.000111636904762236
	324 0.000111092612939956
	325 0.000110552600432356
	326 0.000110016662006274
	327 0.000109484576341856
	328 0.000108956125217219
	329 0.0001084311100783
	330 0.000107909308781018
	331 0.000107390516404848
	332 0.000106874531127232
	333 0.000106361144702305
	334 0.000105850150220022
	335 0.000105341360892908
	336 0.000104834569242485
	337 0.000104329610536524
	338 0.000103826284828301
	339 0.000103324391915294
	340 0.000102823802563989
	341 0.000102324334477544
	342 0.000101825808656031
	343 0.000101328099617604
	344 0.000100831044861138
	345 0.000100334490952037
	346 9.98383264771974e-05
	347 9.93424091149109e-05
	348 9.88466175186886e-05
	349 9.83508393090915e-05
	350 9.78549681036611e-05
	351 9.73588907129397e-05
	352 9.68625519703892e-05
	353 9.63658096964082e-05
	354 9.58686227932048e-05
	355 9.53709049014151e-05
	356 9.48725746212631e-05
	357 9.4373584886398e-05
	358 9.38738931779426e-05
	359 9.33734295642807e-05
	360 9.28721533881571e-05
	361 9.23700435890851e-05
	362 9.18670611724792e-05
	363 9.136319367542e-05
	364 9.08583761969339e-05
	365 9.03526600097848e-05
	366 8.98459924485451e-05
	367 8.93383838871387e-05
	368 8.88298365850915e-05
	369 8.83203644264086e-05
	370 8.78099694716639e-05
	371 8.72986514082186e-05
	372 8.67864733606893e-05
	373 8.62734233777473e-05
	374 8.5759554437459e-05
	375 8.52448744979029e-05
	376 8.47294338086613e-05
	377 8.42132872094226e-05
	378 8.36964482573421e-05
	379 8.31789927673299e-05
	380 8.26609598334471e-05
	381 8.21423863470727e-05
	382 8.16233400939836e-05
	383 8.11039000438996e-05
	384 8.05841053477252e-05
	385 8.00640338951553e-05
	386 7.95437364757845e-05
	387 7.90232958678416e-05
	388 7.85027740732858e-05
	389 7.79822618426351e-05
	390 7.74618062280297e-05
	391 7.69415064922896e-05
	392 7.6421426498996e-05
	393 7.59016446210126e-05
	394 7.53822583021702e-05
	395 7.48633522391629e-05
	396 7.43449955109554e-05
	397 7.38272544680285e-05
	398 7.3310264895099e-05
	399 7.27940616798151e-05
	400 7.22787587648099e-05
	401 7.17644309702337e-05
	402 7.1251185033816e-05
	403 7.0739087860261e-05
	404 7.02282396787268e-05
	405 6.97187281559764e-05
	406 6.92106519579738e-05
	407 6.87040652849191e-05
	408 6.8199081923126e-05
	409 6.76958187426635e-05
	410 6.71943337806624e-05
	411 6.66946784377842e-05
	412 6.6196992477785e-05
	413 6.57013580109833e-05
	414 6.52078380340981e-05
	415 6.4716557503175e-05
	416 6.42275807507531e-05
	417 6.37409667376687e-05
	418 6.32568560234859e-05
	419 6.2775281818972e-05
	420 6.22963496255124e-05
	421 6.18201465556467e-05
	422 6.13467405941037e-05
	423 6.08762454987755e-05
	424 6.04087021116584e-05
	425 5.99441813591284e-05
	426 5.94827985480606e-05
	427 5.90246264664529e-05
	428 5.85697029151788e-05
	429 5.81181463701341e-05
	430 5.76700054040202e-05
	431 5.72253440864756e-05
	432 5.67842660643691e-05
	433 5.63468074759044e-05
	434 5.59130640880312e-05
	435 5.54830581123156e-05
	436 5.5056882516169e-05
	437 5.463461516797e-05
	438 5.42162996453044e-05
	439 5.38019743387963e-05
	440 5.33917061531497e-05
	441 5.29855692832371e-05
	442 5.25836188174367e-05
	443 5.21858897855054e-05
	444 5.17924185743368e-05
	445 5.14032791301133e-05
	446 5.10184984889861e-05
	447 5.06381361091712e-05
	448 5.02621995650543e-05
	449 4.98907614385757e-05
	450 4.95238348960925e-05
	451 4.91614545339303e-05
	452 4.8803655580798e-05
	453 4.84504567950239e-05
	454 4.81018891633767e-05
	455 4.77579819815332e-05
	456 4.74187302472728e-05
	457 4.70841646489362e-05
	458 4.67542750328676e-05
	459 4.64291143345008e-05
	460 4.61086562140167e-05
	461 4.57929183781403e-05
	462 4.54819113073768e-05
	463 4.5175570321021e-05
	464 4.48739622527228e-05
	465 4.45770456209971e-05
	466 4.42848319650579e-05
	467 4.39972679018297e-05
	468 4.37143650913185e-05
	469 4.3436105102046e-05
	470 4.31624320498258e-05
	471 4.28933508516138e-05
	472 4.26288385213525e-05
	473 4.23688119752796e-05
	474 4.21132862271634e-05
	475 4.18622214652942e-05
	476 4.16155527602768e-05
	477 4.13732657307264e-05
	478 4.1135282600635e-05
	479 4.09015867504081e-05
	480 4.06721231982488e-05
	481 4.04468281232084e-05
	482 4.02256458471584e-05
	483 4.00085496039537e-05
	484 3.97954615323215e-05
	485 3.95863305300281e-05
	486 3.93811015655388e-05
	487 3.91796934522404e-05
	488 3.8982061930426e-05
	489 3.87881415448987e-05
	490 3.85978766672679e-05
	491 3.84111688518374e-05
	492 3.82279833033294e-05
	493 3.80482721098474e-05
	494 3.78719233964375e-05
	495 3.76989113064496e-05
	496 3.75291586180992e-05
	497 3.73625785599074e-05
	498 3.71990974059599e-05
	499 3.70386881982654e-05
	500 3.68812480786573e-05
	501 3.67267327874288e-05
	502 3.65750618200877e-05
	503 3.64261651313313e-05
	504 3.62799996551644e-05
	505 3.61364969947431e-05
	506 3.59955691564551e-05
	507 3.5857156561292e-05
	508 3.57212379782368e-05
	509 3.55876921531717e-05
	510 3.54564892930398e-05
	511 3.53275884634741e-05
	512 3.52008960362582e-05
	513 3.50763678014232e-05
	514 3.49539439383761e-05
	515 3.48335490372165e-05
	516 3.47151762696285e-05
	517 3.45987378835844e-05
	518 3.44841770640869e-05
	519 3.43714520880667e-05
	520 3.42605382428474e-05
	521 3.41513510093705e-05
	522 3.40438670107801e-05
	523 3.39380009535262e-05
	524 3.38337558005719e-05
	525 3.37310556872694e-05
	526 3.36298466763196e-05
	527 3.35301186709103e-05
	528 3.34318071182338e-05
	529 3.33349087000556e-05
	530 3.32393228106298e-05
	531 3.31450685564505e-05
	532 3.30520804254775e-05
	533 3.29603385438304e-05
	534 3.28698026024199e-05
	535 3.27804194384385e-05
	536 3.26921836091287e-05
	537 3.26050504995123e-05
	538 3.25190021683852e-05
	539 3.24339789230521e-05
	540 3.23499875349853e-05
	541 3.22669956318578e-05
	542 3.21849522322282e-05
	543 3.2103842912079e-05
	544 3.20236623991832e-05
	545 3.19443534664288e-05
	546 3.18659284204159e-05
	547 3.17883265026353e-05
	548 3.17115361525566e-05
	549 3.16355474083707e-05
	550 3.15603365450556e-05
	551 3.14858749277391e-05
	552 3.14121669120482e-05
	553 3.13391613318004e-05
	554 3.12668485236145e-05
	555 3.11952461444776e-05
	556 3.11242823229918e-05
	557 3.10539830934431e-05
	558 3.09843167727308e-05
	559 3.09152471942298e-05
	560 3.08468033267673e-05
	561 3.07789368036993e-05
	562 3.07116320783507e-05
	563 3.06449001712394e-05
	564 3.05787194321283e-05
	565 3.05130628390771e-05
	566 3.04479296246996e-05
	567 3.03832971724205e-05
	568 3.03191718984408e-05
	569 3.02555475002464e-05
	570 3.01924010841503e-05
	571 3.01297031981562e-05
	572 3.00674553912472e-05
	573 3.00056724142905e-05
	574 2.99443024189827e-05
	575 2.98833706011692e-05
	576 2.98228490365204e-05
	577 2.97627481877782e-05
	578 2.97030531513087e-05
	579 2.96437433213725e-05
	580 2.95848076454774e-05
	581 2.95262548704045e-05
	582 2.94680700392291e-05
	583 2.94102493256787e-05
	584 2.9352786317105e-05
	585 2.92956460512528e-05
	586 2.9238856111391e-05
	587 2.91824284985864e-05
	588 2.91262994345232e-05
	589 2.90705159180504e-05
	590 2.90150372350695e-05
	591 2.89598504998878e-05
	592 2.89049786985629e-05
	593 2.88504178733717e-05
	594 2.87961410450066e-05
	595 2.8742159958739e-05
	596 2.86884324900427e-05
	597 2.86349858065194e-05
	598 2.85818287366624e-05
	599 2.85289391754873e-05
	600 2.84763098434837e-05
	601 2.84239408152587e-05
	602 2.8371832328844e-05
	603 2.83199753248198e-05
	604 2.82683412642371e-05
	605 2.8216969180761e-05
	606 2.81658463769929e-05
	607 2.81149407577175e-05
	608 2.80642531755859e-05
	609 2.80137987118678e-05
	610 2.79635800062294e-05
	611 2.79135753125104e-05
	612 2.7863772466219e-05
	613 2.78141888720995e-05
	614 2.77648168491851e-05
	615 2.77156585539728e-05
	616 2.76666995269181e-05
	617 2.76179515026342e-05
	618 2.75693866527149e-05
	619 2.75210094820011e-05
	620 2.7472839629894e-05
	621 2.74248538900679e-05
	622 2.73770586467492e-05
	623 2.73294545536373e-05
	624 2.72820132067864e-05
	625 2.72347521317329e-05
	626 2.71876570252516e-05
	627 2.71407491645448e-05
	628 2.70940136246622e-05
	629 2.70474349619576e-05
	630 2.7001036638552e-05
	631 2.69548025002564e-05
	632 2.69087187199091e-05
	633 2.686279435693e-05
	634 2.6817037266369e-05
	635 2.67714285193676e-05
	636 2.6725960484697e-05
	637 2.66806698121513e-05
	638 2.66355174822763e-05
	639 2.65904957359453e-05
	640 2.65456231325345e-05
	641 2.65009064506216e-05
	642 2.64563407270657e-05
	643 2.64119078678959e-05
	644 2.63676122393974e-05
	645 2.6323460904365e-05
	646 2.62794336620686e-05
	647 2.6235529549723e-05
	648 2.61917811670287e-05
	649 2.61481534806762e-05
	650 2.61046442595614e-05
	651 2.60612474534128e-05
	652 2.60179912139336e-05
	653 2.59748641582291e-05
	654 2.5931882397856e-05
	655 2.58889922548633e-05
	656 2.58462367455081e-05
	657 2.58035812308322e-05
	658 2.57610614156079e-05
	659 2.57186452330416e-05
	660 2.56763675317018e-05
	661 2.5634194095403e-05
	662 2.55921384351154e-05
	663 2.55501927632906e-05
	664 2.55083335183315e-05
	665 2.54665872425619e-05
	666 2.54249260898121e-05
	667 2.53833889765076e-05
	668 2.53419707298974e-05
	669 2.53006651824705e-05
	670 2.52594548868501e-05
	671 2.52183473996581e-05
	672 2.51773548072265e-05
	673 2.51364675314392e-05
	674 2.50956540774894e-05
	675 2.5054952974557e-05
	676 2.50143474715969e-05
	677 2.49738511293174e-05
	678 2.49334475661556e-05
	679 2.48931148831844e-05
	680 2.48528993509467e-05
	681 2.48127733151193e-05
	682 2.47727469577796e-05
	683 2.47328130988933e-05
	684 2.46929708467292e-05
	685 2.46532409740041e-05
	686 2.46135858930074e-05
	687 2.45740278330686e-05
	688 2.45345499365612e-05
	689 2.44951787031766e-05
	690 2.44558915518667e-05
	691 2.44167043348398e-05
	692 2.4377620004401e-05
	693 2.43386052822814e-05
	694 2.42996696151465e-05
	695 2.42608222826846e-05
	696 2.42220758437384e-05
	697 2.41833930303414e-05
	698 2.41448189051141e-05
	699 2.4106321806272e-05
	700 2.40679193019844e-05
	701 2.40296015334707e-05
	702 2.39913801785008e-05
	703 2.39532475880821e-05
	704 2.39152015240052e-05
	705 2.38772378260421e-05
	706 2.38393611056154e-05
	707 2.38015601929931e-05
	708 2.37638374720461e-05
	709 2.37261989255444e-05
	710 2.3688663983279e-05
	711 2.36512037297132e-05
	712 2.36138188469681e-05
	713 2.35765335538929e-05
	714 2.3539332246969e-05
	715 2.35021773669075e-05
	716 2.34651296402433e-05
	717 2.34281739714959e-05
	718 2.33913041896017e-05
	719 2.33545154593173e-05
	720 2.33178226025643e-05
	721 2.32812084064449e-05
	722 2.32446634136352e-05
	723 2.32082225544161e-05
	724 2.31718277667881e-05
	725 2.31355286821611e-05
	726 2.309929765687e-05
	727 2.30631728221908e-05
	728 2.30271225944989e-05
	729 2.29911457125809e-05
	730 2.295525985474e-05
	731 2.29194535279476e-05
	732 2.28837320932485e-05
	733 2.28480884061355e-05
	734 2.28125134533741e-05
	735 2.27770224334733e-05
	736 2.27416212403853e-05
	737 2.27063095934454e-05
	738 2.2671089141113e-05
	739 2.26359344388527e-05
	740 2.2600848147647e-05
	741 2.25658416788121e-05
	742 2.25309116679284e-05
	743 2.24960646129091e-05
	744 2.24612931418733e-05
	745 2.24265829871229e-05
	746 2.23919629966929e-05
	747 2.23574162241391e-05
	748 2.23229567914984e-05
	749 2.22886052156923e-05
	750 2.22543249179807e-05
	751 2.22201207513706e-05
	752 2.21859851556871e-05
	753 2.2151917242752e-05
	754 2.21179302464236e-05
	755 2.208402563042e-05
	756 2.20501855672239e-05
	757 2.20164284776558e-05
	758 2.19827534451156e-05
	759 2.19491492075008e-05
	760 2.19156212004634e-05
	761 2.18821895749954e-05
	762 2.18488243817205e-05
	763 2.18155363036487e-05
	764 2.1782315080543e-05
	765 2.17491758718324e-05
	766 2.17161107975983e-05
	767 2.16831203267986e-05
	768 2.16502363130644e-05
	769 2.16174131466573e-05
	770 2.15846569098233e-05
	771 2.15519411739251e-05
	772 2.15192960979493e-05
	773 2.14867515957451e-05
	774 2.14542835657028e-05
	775 2.14219114731407e-05
	776 2.13896037273287e-05
	777 2.13573811009837e-05
	778 2.13252268643771e-05
	779 2.12931354539592e-05
	780 2.12611246013239e-05
	781 2.1229199447248e-05
	782 2.11973428676515e-05
	783 2.11655421864521e-05
	784 2.11338149469498e-05
	785 2.1102177647947e-05
	786 2.10705735028682e-05
	787 2.10390531734106e-05
	788 2.10076163611461e-05
	789 2.09762771632427e-05
	790 2.09450211805517e-05
	791 2.09138348381543e-05
	792 2.08827159227098e-05
	793 2.08516697206562e-05
	794 2.08206920540022e-05
	795 2.07897846991045e-05
	796 2.07589213943038e-05
	797 2.07281481507948e-05
	798 2.06974438761165e-05
	799 2.06668255096076e-05
	800 2.0636277493935e-05
	801 2.06058070091331e-05
	802 2.05754011055603e-05
	803 2.05450682884134e-05
	804 2.05148104086561e-05
	805 2.04846099691736e-05
	806 2.04544859485623e-05
	807 2.04244326305059e-05
	808 2.03944557206626e-05
	809 2.03645586154266e-05
	810 2.03347501930295e-05
	811 2.030499065242e-05
	812 2.02753133784483e-05
	813 2.02457108215981e-05
	814 2.02161483855434e-05
	815 2.01866455817878e-05
	816 2.01572310132292e-05
	817 2.01278868274812e-05
	818 2.00986335627817e-05
	819 2.00694627103815e-05
	820 2.00403556895878e-05
	821 2.00113101023192e-05
	822 1.99823375197639e-05
	823 1.99534317282257e-05
	824 1.9924581085462e-05
	825 1.98957726453841e-05
	826 1.98670584623528e-05
	827 1.98384458549583e-05
	828 1.98098986068374e-05
	829 1.97814231377436e-05
	830 1.97530125305434e-05
	831 1.97246825344166e-05
	832 1.96963924636862e-05
	833 1.96681777637764e-05
	834 1.9640044730096e-05
	835 1.96119510889048e-05
	836 1.95839457077795e-05
	837 1.95560060376465e-05
	838 1.95281677086712e-05
	839 1.95003904082114e-05
	840 1.94726967315262e-05
	841 1.94450548427483e-05
	842 1.941748764267e-05
	843 1.93899782381379e-05
	844 1.93625163475986e-05
	845 1.9335112963148e-05
	846 1.93077846084577e-05
	847 1.92805390852868e-05
	848 1.9253378482631e-05
	849 1.92263043530261e-05
	850 1.91992895288706e-05
	851 1.91723554117118e-05
	852 1.91454644387079e-05
	853 1.91186447970892e-05
	854 1.90918718985245e-05
	855 1.90651695781696e-05
	856 1.90385496132706e-05
	857 1.90119872023331e-05
	858 1.89855182561871e-05
	859 1.89591316619442e-05
	860 1.89327948234563e-05
	861 1.89065185587367e-05
	862 1.88803031058171e-05
	863 1.88541341863413e-05
	864 1.88280735606838e-05
	865 1.8802077772051e-05
	866 1.87761288863442e-05
	867 1.87502534565454e-05
	868 1.87244623646166e-05
	869 1.86987402450711e-05
	870 1.86730599764928e-05
	871 1.86474623475874e-05
	872 1.86219596152171e-05
	873 1.85965248107323e-05
	874 1.85711475886308e-05
	875 1.85458208115108e-05
	876 1.85205512863718e-05
	877 1.84953533342025e-05
	878 1.84702174443885e-05
	879 1.84451347244874e-05
	880 1.84201212505286e-05
	881 1.83952067942528e-05
	882 1.83703846730054e-05
	883 1.83456250937297e-05
	884 1.83209023809638e-05
	885 1.82962601904535e-05
	886 1.82716734826727e-05
	887 1.82471373051385e-05
	888 1.8222686797742e-05
	889 1.81983175018274e-05
	890 1.81740006368614e-05
	891 1.81497320816959e-05
	892 1.81255191016305e-05
	893 1.81013939304364e-05
	894 1.80773418421154e-05
	895 1.8053348313174e-05
	896 1.80294256644231e-05
	897 1.80055694514181e-05
	898 1.798179528123e-05
	899 1.79580838235438e-05
	900 1.7934429450861e-05
	901 1.79108266173955e-05
	902 1.78873003235935e-05
	903 1.78638444481294e-05
	904 1.78404553956568e-05
	905 1.78171312583686e-05
	906 1.77938582481829e-05
	907 1.77706502313413e-05
	908 1.77475265061844e-05
	909 1.77244760237727e-05
	910 1.77014760076588e-05
	911 1.76785160341808e-05
	912 1.76556321491717e-05
	913 1.76328306800144e-05
	914 1.76100922786304e-05
	915 1.75874303423029e-05
	916 1.75648362237268e-05
	917 1.75422898180955e-05
	918 1.75197993961262e-05
	919 1.74973579376569e-05
	920 1.74749884997993e-05
	921 1.7452687117725e-05
	922 1.74304504199085e-05
	923 1.74082841404299e-05
	924 1.73861897287964e-05
	925 1.73641430549765e-05
	926 1.73421782001526e-05
	927 1.73202874940159e-05
	928 1.72984666200193e-05
	929 1.72767109063443e-05
	930 1.72550011967587e-05
	931 1.72333268970704e-05
	932 1.72117296344254e-05
	933 1.71902090428944e-05
	934 1.71687512704466e-05
	935 1.7147353563729e-05
	936 1.71260105545912e-05
	937 1.71047375765454e-05
	938 1.70835206922959e-05
	939 1.70623900217493e-05
	940 1.70413273004044e-05
	941 1.70203080855913e-05
	942 1.69993557861403e-05
	943 1.69784387580307e-05
	944 1.69575992181592e-05
	945 1.69368335605213e-05
	946 1.69161173211307e-05
	947 1.68954634460761e-05
	948 1.68748642543903e-05
	949 1.68543523955123e-05
	950 1.68338987514005e-05
	951 1.68134949092291e-05
	952 1.67931382186737e-05
	953 1.67728440025883e-05
	954 1.67526198708856e-05
	955 1.67324597946106e-05
	956 1.67123607930364e-05
	957 1.66923288382748e-05
	958 1.66723638450605e-05
	959 1.66524676927793e-05
	960 1.66326163721919e-05
	961 1.6612840649799e-05
	962 1.65931041102851e-05
	963 1.65734190211708e-05
	964 1.65537801564142e-05
	965 1.65342101574595e-05
	966 1.65147016559786e-05
	967 1.64952658927575e-05
	968 1.64758901775031e-05
	969 1.64565898224112e-05
	970 1.64373339188728e-05
	971 1.64181385748918e-05
	972 1.63989987562729e-05
	973 1.63799294519151e-05
	974 1.63609316032876e-05
	975 1.63419778509422e-05
	976 1.63230713923213e-05
	977 1.63041979774903e-05
	978 1.62854013439073e-05
	979 1.62666509559983e-05
	980 1.62479642327185e-05
	981 1.62293445491457e-05
	982 1.62107677965651e-05
	983 1.61922556571881e-05
	984 1.61738121704502e-05
	985 1.61554270654563e-05
	986 1.61371118068132e-05
	987 1.61188676663926e-05
	988 1.61006567616084e-05
	989 1.60824964012818e-05
	990 1.60643933355686e-05
	991 1.60463297405045e-05
	992 1.60283186936283e-05
	993 1.60103820405766e-05
	994 1.59925033251795e-05
	995 1.59746943850791e-05
	996 1.59569257895953e-05
	997 1.59392002814229e-05
	998 1.59215351551723e-05
	999 1.59039147291651e-05
	1000 1.58863605435045e-05
	1001 1.58688368472326e-05
	1002 1.58513892678513e-05
	1003 1.58340153966208e-05
	1004 1.58167035735346e-05
	1005 1.57994293310537e-05
	1006 1.5782213591109e-05
	1007 1.57650360037564e-05
	1008 1.57479146132289e-05
	1009 1.57308332831008e-05
	1010 1.57138232346199e-05
	1011 1.569686545011e-05
	1012 1.56799516517481e-05
	1013 1.56630997416585e-05
	1014 1.56462871778729e-05
	1015 1.56295362572223e-05
	1016 1.5612850450708e-05
	1017 1.55962178780555e-05
	1018 1.55796230920657e-05
	1019 1.55630367650872e-05
	1020 1.55465264555232e-05
	1021 1.55300601569763e-05
	1022 1.55136553203761e-05
	1023 1.54973120416457e-05
	1024 1.54810349535239e-05
	1025 1.54648046759576e-05
	1026 1.5448641764948e-05
	1027 1.54325329049243e-05
	1028 1.54164663257461e-05
	1029 1.5400416756961e-05
	1030 1.53844307888562e-05
	1031 1.53684715016311e-05
	1032 1.53525662689447e-05
	1033 1.53367139184013e-05
	1034 1.53209032305313e-05
	1035 1.53051445259678e-05
	1036 1.52894499905187e-05
	1037 1.52737942080705e-05
	1038 1.52582126453638e-05
	1039 1.52426872013223e-05
	1040 1.52272158153721e-05
	1041 1.52117716218925e-05
	1042 1.51963598682414e-05
	1043 1.51809765327471e-05
	1044 1.5165636014558e-05
	1045 1.51503598360136e-05
	1046 1.51351241157727e-05
	1047 1.51199218727527e-05
	1048 1.51047886518541e-05
	1049 1.50897086257373e-05
	1050 1.5074672109705e-05
	1051 1.50596783292656e-05
	1052 1.50447561004796e-05
	1053 1.50298857768405e-05
	1054 1.50150601605503e-05
	1055 1.50002688812378e-05
	1056 1.49855028972468e-05
	1057 1.49707546270861e-05
	1058 1.49560401787596e-05
	1059 1.4941375042099e-05
	1060 1.49267535434205e-05
	1061 1.49122115473688e-05
	1062 1.48977307148357e-05
	1063 1.48833006399229e-05
	1064 1.48689275896174e-05
	1065 1.48545939389066e-05
	1066 1.48402973998429e-05
	1067 1.48260133379097e-05
	1068 1.48117558680383e-05
	1069 1.47975235051945e-05
	1070 1.47833458470359e-05
	1071 1.47692242293829e-05
	1072 1.47551334102047e-05
	1073 1.47411095241523e-05
	1074 1.47271297237239e-05
	1075 1.47131739076656e-05
	1076 1.46992554022063e-05
	1077 1.46853809539493e-05
	1078 1.46715737017189e-05
	1079 1.465780288612e-05
	1080 1.46440827428762e-05
	1081 1.46303987946794e-05
	1082 1.46167409695863e-05
	1083 1.46031124543811e-05
	1084 1.45894995888796e-05
	1085 1.45759234584375e-05
	1086 1.45623901950387e-05
	1087 1.45489341782934e-05
	1088 1.45355141611958e-05
	1089 1.45221512433125e-05
	1090 1.45088212128996e-05
	1091 1.44955501824029e-05
	1092 1.44822921051002e-05
	1093 1.44690460288643e-05
	1094 1.44558635852832e-05
	1095 1.44426926489416e-05
	1096 1.44295732305011e-05
	1097 1.44164627222665e-05
	1098 1.44033889384332e-05
	1099 1.43903557088265e-05
	1100 1.43773522474078e-05
	1101 1.43644229240181e-05
	1102 1.435153357221e-05
	1103 1.43386907218712e-05
	1104 1.43258843259275e-05
	1105 1.43131093253146e-05
	1106 1.4300366469655e-05
	1107 1.42876675468528e-05
	1108 1.427501180018e-05
	1109 1.4262384631536e-05
	1110 1.42497787471996e-05
	1111 1.42371852298595e-05
	1112 1.42246231931153e-05
	1113 1.42120841353233e-05
	1114 1.41995770022163e-05
	1115 1.41871282863804e-05
	1116 1.41747081841004e-05
	1117 1.41623364591226e-05
	1118 1.41499967369896e-05
	1119 1.41377072608861e-05
	1120 1.41254718641903e-05
	1121 1.41132592474946e-05
	1122 1.41010737237934e-05
	1123 1.40889070685546e-05
	1124 1.40767568304057e-05
	1125 1.40646437181147e-05
	1126 1.40525830865101e-05
	1127 1.40405488586737e-05
	1128 1.40285322061118e-05
	1129 1.40165589961327e-05
	1130 1.40046388779069e-05
	1131 1.3992730970358e-05
	1132 1.3980834616234e-05
	1133 1.39689747022942e-05
	1134 1.39571452315579e-05
	1135 1.39453749881113e-05
	1136 1.39336137934265e-05
	1137 1.39218901082927e-05
	1138 1.39101884499837e-05
	1139 1.38985144104709e-05
	1140 1.38868839556494e-05
	1141 1.38752824128119e-05
	1142 1.38637237583339e-05
	1143 1.38522008406028e-05
	1144 1.3840689334188e-05
	1145 1.38292298643705e-05
	1146 1.38177917143878e-05
	1147 1.38063437091773e-05
	1148 1.37949117231528e-05
	1149 1.37835240039408e-05
	1150 1.37721725472773e-05
	1151 1.37608470218709e-05
	1152 1.37495681045152e-05
	1153 1.37383327860618e-05
	1154 1.37271180520315e-05
	1155 1.37159372002316e-05
	1156 1.37047885289121e-05
	1157 1.36936595644954e-05
	1158 1.36825630185911e-05
	1159 1.36714878564703e-05
	1160 1.36604334528556e-05
	1161 1.36494094746809e-05
	1162 1.36384095767994e-05
	1163 1.36274157718219e-05
	1164 1.36164366040248e-05
	1165 1.36054813246744e-05
	1166 1.35945536499094e-05
	1167 1.35836851420379e-05
	1168 1.35728567265403e-05
	1169 1.35620680978832e-05
	1170 1.35512946961569e-05
	1171 1.35405481564987e-05
	1172 1.35298053614008e-05
	1173 1.3519084291147e-05
	1174 1.3508393934103e-05
	1175 1.34977121177826e-05
	1176 1.34870436205858e-05
	1177 1.3476423131209e-05
	1178 1.34658450754443e-05
	1179 1.3455306312693e-05
	1180 1.34447828266104e-05
	1181 1.3434277281732e-05
	1182 1.34238019739996e-05
	1183 1.34133369691369e-05
	1184 1.34028929963392e-05
	1185 1.33924500005378e-05
	1186 1.33820378813709e-05
	1187 1.33716442221044e-05
	1188 1.3361292797498e-05
	1189 1.33509684268063e-05
	1190 1.33406810576275e-05
	1191 1.33304223695063e-05
	1192 1.3320174502951e-05
	1193 1.33099546850701e-05
	1194 1.32997505737364e-05
	1195 1.32895540225775e-05
	1196 1.32793863834024e-05
	1197 1.32692439045456e-05
	1198 1.32591131674076e-05
	1199 1.32490163622379e-05
	1200 1.32389370683939e-05
	1201 1.3228870553661e-05
	1202 1.32188403725308e-05
	1203 1.32088381654683e-05
	1204 1.31988470144506e-05
	1205 1.31889076335767e-05
	1206 1.3178987135376e-05
	1207 1.31690859213052e-05
	1208 1.31591999448233e-05
	1209 1.31492981019221e-05
	1210 1.3139416086716e-05
	1211 1.31295516361263e-05
	1212 1.31197283224083e-05
	1213 1.31099332172369e-05
	1214 1.31001540921716e-05
	1215 1.30903732049603e-05
	1216 1.30806157194741e-05
	1217 1.30709079861902e-05
	1218 1.30612319821921e-05
	1219 1.30515790850438e-05
	1220 1.3041927559243e-05
	1221 1.30322858460374e-05
	1222 1.30226793650934e-05
	1223 1.30130932696204e-05
	1224 1.30035128833583e-05
	1225 1.29939481858798e-05
	1226 1.2984392352422e-05
	1227 1.2974872429794e-05
	1228 1.29653806091312e-05
	1229 1.29558878967373e-05
	1230 1.29464126388257e-05
	1231 1.29369626868936e-05
	1232 1.29275223237357e-05
	1233 1.29180960968256e-05
	1234 1.29086872924233e-05
	1235 1.28993048384984e-05
	1236 1.28899666371751e-05
	1237 1.28806718677765e-05
	1238 1.28713872591391e-05
	1239 1.28621232491355e-05
	1240 1.2852859065049e-05
	1241 1.28435932005289e-05
	1242 1.28343282277399e-05
	1243 1.28250766771032e-05
	1244 1.28158390850786e-05
	1245 1.2806633741036e-05
	1246 1.27974478729698e-05
	1247 1.27882773632848e-05
	1248 1.27791280029044e-05
	1249 1.27700079879389e-05
	1250 1.27608994127115e-05
	1251 1.27518039150232e-05
	1252 1.27427293286075e-05
	1253 1.27336713902082e-05
	1254 1.27246310164253e-05
	1255 1.27156149858365e-05
	1256 1.27066016553101e-05
	1257 1.26975990113465e-05
	1258 1.2688616131129e-05
	1259 1.26796302382104e-05
	1260 1.26706750087635e-05
	1261 1.26617602660417e-05
	1262 1.26528648713986e-05
	1263 1.26439723402427e-05
	1264 1.26350866302971e-05
	1265 1.26262025794688e-05
	1266 1.26173436960642e-05
	1267 1.26085141864962e-05
	1268 1.25997017512702e-05
	1269 1.25909128456669e-05
	1270 1.25821437961804e-05
	1271 1.25733610509826e-05
	1272 1.25645788422446e-05
	1273 1.25557992554093e-05
	1274 1.25470414502615e-05
	1275 1.25382988329648e-05
	1276 1.25295346933285e-05
	1277 1.2520802197713e-05
	1278 1.25121165481801e-05
	1279 1.25034731759399e-05
	1280 1.24948497024491e-05
	1281 1.24862539898629e-05
	1282 1.24776746019961e-05
	1283 1.24690952993944e-05
	1284 1.24605226652363e-05
	1285 1.24519641353515e-05
	1286 1.24434122987793e-05
	1287 1.2434857634247e-05
	1288 1.24262875829118e-05
	1289 1.24177297564643e-05
	1290 1.2409158873794e-05
	1291 1.24006339454752e-05
	1292 1.23921313885944e-05
	1293 1.23836590226745e-05
	1294 1.23751942346928e-05
	1295 1.23667630482771e-05
	1296 1.23583599105359e-05
	1297 1.23499765329882e-05
	1298 1.23416061548198e-05
	1299 1.23332349453165e-05
	1300 1.23248579306789e-05
	1301 1.23164774734619e-05
	1302 1.23080936234032e-05
	1303 1.22997279348169e-05
	1304 1.22913724567297e-05
	1305 1.22830144739794e-05
	1306 1.22746860284906e-05
	1307 1.22663738117978e-05
	1308 1.2258078360361e-05
	1309 1.22498137109517e-05
	1310 1.2241567876714e-05
	1311 1.22333299614752e-05
	1312 1.22251097671722e-05
	1313 1.22169038299091e-05
	1314 1.22087144305283e-05
	1315 1.22005412741544e-05
	1316 1.2192369315045e-05
	1317 1.21841784732624e-05
	1318 1.21759950957312e-05
	1319 1.21678107447565e-05
	1320 1.21596218072284e-05
	1321 1.21514529887179e-05
	1322 1.21433143434047e-05
	1323 1.21351787889523e-05
	1324 1.21270605752954e-05
	1325 1.21189710426961e-05
	1326 1.21108990498442e-05
	1327 1.21028585446936e-05
	1328 1.20948379702668e-05
	1329 1.20868178079547e-05
	1330 1.20788023281193e-05
	1331 1.207079355936e-05
	1332 1.20628113329246e-05
	1333 1.20548160538192e-05
	1334 1.20468060984535e-05
	1335 1.20388011630723e-05
	1336 1.20307872535363e-05
	1337 1.20227871782674e-05
	1338 1.20148220190686e-05
	1339 1.20068587818878e-05
	1340 1.19989027780321e-05
	1341 1.19909619762382e-05
	1342 1.19830449953895e-05
	1343 1.19751492384523e-05
	1344 1.19672855944941e-05
	1345 1.19594515339827e-05
	1346 1.19516015963939e-05
	1347 1.19437449370707e-05
	1348 1.19358933794445e-05
	1349 1.19280451400527e-05
	1350 1.19201741490826e-05
	1351 1.19122965926977e-05
	1352 1.19044239248467e-05
	1353 1.18965598012721e-05
	1354 1.18887111923982e-05
	1355 1.18809206490766e-05
	1356 1.18731621832069e-05
	1357 1.18654201273216e-05
	1358 1.18576824483796e-05
	1359 1.18499792023385e-05
	1360 1.18422809727292e-05
	1361 1.18345887827331e-05
	1362 1.18268941697863e-05
	1363 1.18192132028128e-05
	1364 1.18115358098692e-05
	1365 1.18038389160802e-05
	1366 1.17961067900296e-05
	1367 1.1788386576228e-05
	1368 1.17806570294476e-05
	1369 1.17728987767407e-05
	1370 1.17651242845795e-05
	1371 1.17573793190218e-05
	1372 1.17496623737168e-05
	1373 1.17419846397127e-05
	1374 1.17343708438966e-05
	1375 1.17267906709628e-05
	1376 1.17192523916287e-05
	1377 1.1711725697694e-05
	1378 1.17042092142583e-05
	1379 1.16966949050834e-05
	1380 1.16891781232198e-05
	1381 1.16816744473169e-05
	1382 1.16741470179704e-05
	1383 1.16666048946001e-05
	1384 1.16590416396889e-05
	1385 1.16514699364245e-05
	1386 1.16438732611357e-05
	1387 1.16362629469791e-05
	1388 1.16286364679752e-05
	1389 1.1621015104879e-05
	1390 1.16134014795932e-05
	1391 1.16058609371805e-05
	1392 1.15983537476438e-05
	1393 1.15908863485004e-05
	1394 1.15834665983527e-05
	1395 1.15760497649831e-05
	1396 1.15686632007339e-05
	1397 1.15612837170431e-05
	1398 1.15538954332806e-05
	1399 1.15465162409123e-05
	1400 1.15391233919127e-05
	1401 1.15317181546004e-05
	1402 1.1524324186496e-05
	1403 1.15169097547607e-05
	1404 1.15094751365064e-05
	1405 1.15020224988882e-05
	1406 1.14945704119407e-05
	1407 1.14870938823231e-05
	1408 1.14796089398794e-05
	1409 1.14721349184776e-05
	1410 1.1464711359821e-05
	1411 1.1457322820263e-05
	1412 1.14499631465037e-05
	1413 1.14426391171207e-05
	1414 1.14353484157448e-05
	1415 1.14280835319391e-05
	1416 1.14208276542627e-05
	1417 1.14135991999831e-05
	1418 1.14063754743654e-05
	1419 1.13991569143934e-05
	1420 1.13919211379709e-05
	1421 1.13846632494585e-05
	1422 1.1377384925737e-05
	1423 1.13700913608739e-05
	1424 1.1362777268431e-05
	1425 1.13554354541634e-05
	1426 1.13480549970291e-05
	1427 1.13406535113825e-05
	1428 1.1333252160739e-05
	1429 1.13258534746308e-05
	1430 1.13184629739749e-05
	1431 1.13111223214446e-05
	1432 1.1303831168874e-05
	1433 1.12966111380786e-05
	1434 1.12894503558891e-05
	1435 1.12823485629576e-05
	1436 1.12752884540157e-05
	1437 1.12682984898527e-05
	1438 1.12613368656866e-05
	1439 1.12543657024844e-05
	1440 1.12473647142508e-05
	1441 1.12403308634157e-05
	1442 1.12332312411922e-05
	1443 1.12260933349262e-05
	1444 1.12188810241776e-05
	1445 1.12115660932943e-05
	1446 1.1204176317392e-05
	1447 1.11967502363086e-05
	1448 1.11892864964602e-05
	1449 1.11817995147589e-05
	1450 1.11743263673247e-05
	1451 1.11668846827229e-05
	1452 1.11595156440103e-05
	1453 1.11522524157692e-05
	1454 1.11451017730246e-05
	1455 1.11380719687304e-05
	1456 1.11311859321006e-05
	1457 1.1124422684361e-05
	1458 1.11177816037866e-05
	1459 1.11112340306363e-05
	1460 1.11047439723677e-05
	1461 1.10982854337749e-05
	1462 1.10917833850976e-05
	1463 1.10851782686439e-05
	1464 1.1078397143649e-05
	1465 1.10714252272714e-05
	1466 1.10642003434691e-05
	1467 1.10567223892133e-05
	1468 1.10489998483843e-05
	1469 1.10410743943135e-05
	1470 1.10330002023318e-05
	1471 1.10248268683222e-05
	1472 1.10166383677779e-05
	1473 1.10085363651535e-05
	1474 1.1000622944124e-05
	1475 1.09930012293091e-05
	1476 1.09857444101635e-05
	1477 1.09789023383655e-05
	1478 1.09725543744332e-05
	1479 1.09667070482544e-05
	1480 1.09613119114726e-05
	1481 1.09562938988006e-05
	1482 1.09515586466102e-05
	1483 1.09469521163419e-05
	1484 1.09422854954744e-05
	1485 1.0937347866502e-05
	1486 1.0931910086498e-05
	1487 1.09257625204862e-05
	1488 1.09187244774489e-05
	1489 1.09106677328441e-05
	1490 1.09015507803178e-05
	1491 1.08914326162335e-05
	1492 1.08804515654981e-05
	1493 1.08688042281813e-05
	1494 1.08568236285578e-05
	1495 1.08448991653631e-05
	1496 1.08334111601494e-05
	1497 1.08227307862307e-05
	1498 1.08132428700003e-05
	1499 1.08053109464379e-05
	1500 1.07992201705542e-05
	1501 1.0795197429303e-05
	1502 1.07934185322733e-05
	1503 1.07938956617204e-05
	1504 1.07965121820541e-05
	1505 1.08009556605282e-05
	1506 1.08066395583251e-05
	1507 1.0812710304009e-05
	1508 1.0818015578451e-05
	1509 1.08211634035627e-05
	1510 1.08205831885755e-05
	1511 1.08148954929277e-05
	1512 1.08030493741751e-05
	1513 1.0784670561037e-05
	1514 1.07601410537228e-05
	1515 1.07306032788301e-05
	1516 1.06977199312297e-05
	1517 1.0663463880789e-05
	1518 1.06297456170523e-05
	1519 1.05982273943539e-05
	1520 1.05702046511169e-05
	1521 1.05465663686743e-05
	1522 1.05279241076062e-05
	1523 1.05147452167387e-05
	1524 1.05075876817295e-05
	1525 1.0507296668294e-05
	1526 1.05153491922749e-05
	1527 1.05342325724678e-05
	1528 1.05678557176248e-05
	1529 1.06217770330375e-05
	1530 1.07027648361679e-05
	1531 1.0816366362576e-05
	1532 1.09609265770416e-05
	1533 1.11177047124045e-05
	1534 1.124307670608e-05
	1535 1.12789273085667e-05
	1536 1.11898919445963e-05
	1537 1.09981527529612e-05
	1538 1.07694144002579e-05
	1539 1.05627690025756e-05
	1540 1.04036359491744e-05
	1541 1.02949694742449e-05
	1542 1.0234655865915e-05
	1543 1.02213226256254e-05
	1544 1.02546850602181e-05
	1545 1.03380278932974e-05
	1546 1.04825632778471e-05
	1547 1.07095732992946e-05
	1548 1.10443901171209e-05
	1549 1.1491039423106e-05
	1550 1.19771129192259e-05
	1551 1.23016107060891e-05
	1552 1.22179283259527e-05
	1553 1.17151872593979e-05
	1554 1.11509126412557e-05
	1555 1.09294492602885e-05
	1556 1.11403371434449e-05
	1557 1.15298373444261e-05
	1558 1.17137618573793e-05
	1559 1.15360257240127e-05
	1560 1.11890318734709e-05
	1561 1.08791319224366e-05
	1562 1.06370876569173e-05
	1563 1.04457913163714e-05
	1564 1.03125572259444e-05
	1565 1.02405049737797e-05
	1566 1.02150940328727e-05
	1567 1.02167189730551e-05
	1568 1.02303786313485e-05
	1569 1.02467627183245e-05
	1570 1.02602798239815e-05
	1571 1.02679959699259e-05
	1572 1.02691976149316e-05
	1573 1.02648583162335e-05
	1574 1.02568486770593e-05
	1575 1.02472263421305e-05
	1576 1.02375881443351e-05
	1577 1.02289287031709e-05
	1578 1.02216753763429e-05
	1579 1.02158300148147e-05
	1580 1.02111115314329e-05
	1581 1.02071669854098e-05
	1582 1.0203674978726e-05
	1583 1.02003835902309e-05
	1584 1.01971209893748e-05
	1585 1.01938018879366e-05
	1586 1.0190342468519e-05
	1587 1.01867542010581e-05
	1588 1.01830115042389e-05
	1589 1.01791472815194e-05
	1590 1.01751451975218e-05
	1591 1.01710409996514e-05
	1592 1.01668700178692e-05
	1593 1.01626396560306e-05
	1594 1.01583587195364e-05
	1595 1.01540248031995e-05
	1596 1.01496318389849e-05
	1597 1.01451821912235e-05
	1598 1.01406519696923e-05
	1599 1.01360722926103e-05
	1600 1.01314427869426e-05
	1601 1.01267346312994e-05
	1602 1.01219622088422e-05
	1603 1.01171308060088e-05
	1604 1.01122111537677e-05
	1605 1.01072194507168e-05
	1606 1.01021683178715e-05
	1607 1.00970489071983e-05
	1608 1.00918536141137e-05
	1609 1.00866151147017e-05
	1610 1.00813288312906e-05
	1611 1.00759811978435e-05
	1612 1.00705964545256e-05
	1613 1.00651650214445e-05
	1614 1.00596905614481e-05
	1615 1.00541660899012e-05
	1616 1.00486157954549e-05
	1617 1.00430398397577e-05
	1618 1.00374246745361e-05
	1619 1.00317891789103e-05
	1620 1.00261209681207e-05
	1621 1.00204008504079e-05
	1622 1.00146605195306e-05
	1623 1.00088927705855e-05
	1624 1.00031041956328e-05
	1625 9.99729132544758e-06
	1626 9.99144518942785e-06
	1627 9.98558573961361e-06
	1628 9.97968938065696e-06
	1629 9.97379306788559e-06
	1630 9.96789908391804e-06
	1631 9.962002922137e-06
	1632 9.95609433651623e-06
	1633 9.95014639038061e-06
	1634 9.94418132371777e-06
	1635 9.9381900042772e-06
	1636 9.93217528488799e-06
	1637 9.92614110373324e-06
	1638 9.92009274725092e-06
	1639 9.91404498762449e-06
	1640 9.9079747180042e-06
	1641 9.90189175453793e-06
	1642 9.89578558652227e-06
	1643 9.88968544035629e-06
	1644 9.88358793385657e-06
	1645 9.87749101177826e-06
	1646 9.87139604902154e-06
	1647 9.86529389201962e-06
	1648 9.85917701257222e-06
	1649 9.85306733269908e-06
	1650 9.84694781713813e-06
	1651 9.84081607668941e-06
	1652 9.83468264514897e-06
	1653 9.8285577099233e-06
	1654 9.82241490632418e-06
	1655 9.8162687880432e-06
	1656 9.81012180822916e-06
	1657 9.80395669358813e-06
	1658 9.79779072451947e-06
	1659 9.79159551484088e-06
	1660 9.78537369711319e-06
	1661 9.77914896083121e-06
	1662 9.77289359305189e-06
	1663 9.76662288998398e-06
	1664 9.76033861554981e-06
	1665 9.75404473457786e-06
	1666 9.74775944229123e-06
	1667 9.74146226440098e-06
	1668 9.73516622693182e-06
	1669 9.72885859873429e-06
	1670 9.72257337394922e-06
	1671 9.71631313539945e-06
	1672 9.71008108230365e-06
	1673 9.703904584768e-06
	1674 9.69778404424915e-06
	1675 9.69169974496253e-06
	1676 9.68567051629066e-06
	1677 9.67970998466683e-06
	1678 9.67378971417077e-06
	1679 9.66786420342203e-06
	1680 9.66194334139914e-06
	1681 9.65600547786494e-06
	1682 9.65001616393124e-06
	1683 9.6439389078995e-06
	1684 9.6377420355509e-06
	1685 9.63138136533814e-06
	1686 9.62484410571562e-06
	1687 9.61810425792464e-06
	1688 9.61115459219286e-06
	1689 9.60398249283401e-06
	1690 9.5965847570767e-06
	1691 9.58901343217633e-06
	1692 9.58132013906265e-06
	1693 9.57356451181113e-06
	1694 9.56582477762424e-06
	1695 9.55821173143079e-06
	1696 9.55081769227206e-06
	1697 9.54378311845971e-06
	1698 9.53726705255065e-06
	1699 9.53143148407776e-06
	1700 9.5264120680838e-06
	1701 9.52235501472387e-06
	1702 9.51939568061277e-06
	1703 9.51764757495255e-06
	1704 9.517139853088e-06
	1705 9.51786117830977e-06
	1706 9.51969527918095e-06
	1707 9.52238319840149e-06
	1708 9.52549261512559e-06
	1709 9.52837775614057e-06
	1710 9.5301337967868e-06
	1711 9.52967104517199e-06
	1712 9.52572570867005e-06
	1713 9.51704971896561e-06
	1714 9.5025842679064e-06
	1715 9.48163619085562e-06
	1716 9.45410343788922e-06
	1717 9.42050831298502e-06
	1718 9.38189083932173e-06
	1719 9.33974447470121e-06
	1720 9.29572773067378e-06
	1721 9.25157524989118e-06
	1722 9.20908419921318e-06
	1723 9.17040120285151e-06
	1724 9.1391093341997e-06
	1725 9.12266188812794e-06
	1726 9.13844887051596e-06
	1727 9.22861489982552e-06
	1728 9.49567466967949e-06
	1729 1.01760297379627e-05
	1730 1.1705313573529e-05
	1731 1.4295992997404e-05
	1732 1.61218842436028e-05
	1733 1.42409783592257e-05
	1734 1.14781568711919e-05
	1735 1.12940226060232e-05
	1736 1.16349632577339e-05
	1737 1.09570505344436e-05
	1738 1.00735495021098e-05
	1739 9.50334230154226e-06
	1740 9.1729684434938e-06
	1741 9.0161797654531e-06
	1742 8.96183476228885e-06
	1743 8.96207325062903e-06
	1744 8.98881694944009e-06
	1745 9.01936694042149e-06
	1746 9.03998118317872e-06
	1747 9.04742772256384e-06
	1748 9.04484899244551e-06
	1749 9.0371561327629e-06
	1750 9.02842475625221e-06
	1751 9.02110448919302e-06
	1752 9.01615309878423e-06
	1753 9.01349228143999e-06
	1754 9.01249053697484e-06
	1755 9.01238050232678e-06
	1756 9.01249703488816e-06
	1757 9.0124564522398e-06
	1758 9.01208104053808e-06
	1759 9.01137740783042e-06
	1760 9.01042101020266e-06
	1761 9.00931643776914e-06
	1762 9.00815089543983e-06
	1763 9.00697029315722e-06
	1764 9.00582599250299e-06
	1765 9.00470579523471e-06
	1766 9.00359945887885e-06
	1767 9.00250048552209e-06
	1768 9.00138667248029e-06
	1769 9.00023975525244e-06
	1770 8.99903825235526e-06
	1771 8.99778326513001e-06
	1772 8.99644626883855e-06
	1773 8.99502962248278e-06
	1774 8.99353364047784e-06
	1775 8.99194211534393e-06
	1776 8.99026331779851e-06
	1777 8.98848813335462e-06
	1778 8.9866215695622e-06
	1779 8.98466761078964e-06
	1780 8.98261320259053e-06
	1781 8.98045091801691e-06
	1782 8.97815967704219e-06
	1783 8.97576751412998e-06
	1784 8.97325600313081e-06
	1785 8.97065679517084e-06
	1786 8.96795062921285e-06
	1787 8.96512942638594e-06
	1788 8.96218465662457e-06
	1789 8.95914036647127e-06
	1790 8.9559776750292e-06
	1791 8.95269742962057e-06
	1792 8.94931733341764e-06
	1793 8.94582801258537e-06
	1794 8.94225223468936e-06
	1795 8.93858401340708e-06
	1796 8.93483120023575e-06
	1797 8.93101744381397e-06
	1798 8.92711888589304e-06
	1799 8.92311977018778e-06
	1800 8.91904311650649e-06
	1801 8.91489974641502e-06
	1802 8.91069548458745e-06
	1803 8.9064326722621e-06
	1804 8.90212069926122e-06
	1805 8.89774826795531e-06
	1806 8.89332856957026e-06
	1807 8.88884780536614e-06
	1808 8.88431667256384e-06
	1809 8.87972553620386e-06
	1810 8.87507589197867e-06
	1811 8.87039391450628e-06
	1812 8.86568460423121e-06
	1813 8.86093746821359e-06
	1814 8.85615783197125e-06
	1815 8.85134013195454e-06
	1816 8.84648492061046e-06
	1817 8.8415823000787e-06
	1818 8.83662911732586e-06
	1819 8.83163792941843e-06
	1820 8.82663455747945e-06
	1821 8.82162188808877e-06
	1822 8.81658315954326e-06
	1823 8.81153431286918e-06
	1824 8.8064801904153e-06
	1825 8.80141568515569e-06
	1826 8.79633643613431e-06
	1827 8.79124423747157e-06
	1828 8.78612443777627e-06
	1829 8.78099119283604e-06
	1830 8.77585057601493e-06
	1831 8.77068531046632e-06
	1832 8.7655203167003e-06
	1833 8.76032588337239e-06
	1834 8.7551220815385e-06
	1835 8.74989016175221e-06
	1836 8.74462783428953e-06
	1837 8.7393342216302e-06
	1838 8.73403008050389e-06
	1839 8.72869618717687e-06
	1840 8.72335480472941e-06
	1841 8.71799412038854e-06
	1842 8.71258003876108e-06
	1843 8.70715045131476e-06
	1844 8.70170231159761e-06
	1845 8.69622986421348e-06
	1846 8.69074735732056e-06
	1847 8.68529210329427e-06
	1848 8.67986130970166e-06
	1849 8.67446720853593e-06
	1850 8.66910466967852e-06
	1851 8.66383179776165e-06
	1852 8.65862018706309e-06
	1853 8.65349429091111e-06
	1854 8.64850139237205e-06
	1855 8.64364405828155e-06
	1856 8.63890690183666e-06
	1857 8.63431230158085e-06
	1858 8.62988617100768e-06
	1859 8.62561404169071e-06
	1860 8.62148686664455e-06
	1861 8.61747552782788e-06
	1862 8.61353375469776e-06
	1863 8.60959177906295e-06
	1864 8.60558906445874e-06
	1865 8.6013913538352e-06
	1866 8.59689941101749e-06
	1867 8.59193028546201e-06
	1868 8.58631750233485e-06
	1869 8.57986709412728e-06
	1870 8.57235990636696e-06
	1871 8.56357732104129e-06
	1872 8.55330642401952e-06
	1873 8.54132493799398e-06
	1874 8.52747905710771e-06
	1875 8.51162858417354e-06
	1876 8.49367993893679e-06
	1877 8.47354547950374e-06
	1878 8.45117889092251e-06
	1879 8.42654362287476e-06
	1880 8.39962500620572e-06
	1881 8.37046088975058e-06
	1882 8.33929934707101e-06
	1883 8.30712865607097e-06
	1884 8.27686302429242e-06
	1885 8.25695340544996e-06
	1886 8.27114743984225e-06
	1887 8.38555534876662e-06
	1888 8.78237723611619e-06
	1889 9.9272437772413e-06
	1890 1.2659542306892e-05
	1891 1.65838499022186e-05
	1892 1.67163654651858e-05
	1893 1.2541150084644e-05
	1894 1.10573637499556e-05
	1895 1.11979355992275e-05
	1896 1.042921094907e-05
	1897 9.35593581630201e-06
	1898 8.57900044692883e-06
	1899 8.2053643968294e-06
	1900 8.08638019123009e-06
	1901 8.06923236140733e-06
	1902 8.09284467173654e-06
	1903 8.12481003897858e-06
	1904 8.14661241577141e-06
	1905 8.15317426194895e-06
	1906 8.14839015994551e-06
	1907 8.13887035100436e-06
	1908 8.12967145691346e-06
	1909 8.12317551890374e-06
	1910 8.11974081216249e-06
	1911 8.11861170113559e-06
	1912 8.11871970185507e-06
	1913 8.11915785448036e-06
	1914 8.11938482847552e-06
	1915 8.1192094061322e-06
	1916 8.11867696448587e-06
	1917 8.11794225441531e-06
	1918 8.11716565607412e-06
	1919 8.11645452181153e-06
	1920 8.11589095839338e-06
	1921 8.11545735857067e-06
	1922 8.11515976195665e-06
	1923 8.11497937114325e-06
	1924 8.1148567172562e-06
	1925 8.11476754947194e-06
	1926 8.11469491068806e-06
	1927 8.11462441330235e-06
	1928 8.11455577309772e-06
	1929 8.11450475524111e-06
	1930 8.11444244686044e-06
	1931 8.11437787540115e-06
	1932 8.11429863212254e-06
	1933 8.11422305613263e-06
	1934 8.11414802637245e-06
	1935 8.11405555722899e-06
	1936 8.11394941191423e-06
	1937 8.11383179666336e-06
	1938 8.11367761244242e-06
	1939 8.1134698364238e-06
	1940 8.11320284999084e-06
	1941 8.11287978574882e-06
	1942 8.1124987350023e-06
	1943 8.11206876960568e-06
	1944 8.11157930069584e-06
	1945 8.1110120406791e-06
	1946 8.11033906966685e-06
	1947 8.109589349381e-06
	1948 8.10873318979333e-06
	1949 8.10776921600365e-06
	1950 8.10668930117941e-06
	1951 8.10550141672195e-06
	1952 8.1041928430281e-06
	1953 8.10275398155369e-06
	1954 8.1011790706853e-06
	1955 8.099456073829e-06
	1956 8.09758287712015e-06
	1957 8.09558654779607e-06
	1958 8.09343372143445e-06
	1959 8.09111728283796e-06
	1960 8.0886378777123e-06
	1961 8.08603162383292e-06
	1962 8.08328438228756e-06
	1963 8.08043209321596e-06
	1964 8.07747267472791e-06
	1965 8.07440447214702e-06
	1966 8.07123445589752e-06
	1967 8.06801353103737e-06
	1968 8.06472587822071e-06
	1969 8.0614008082236e-06
	1970 8.05803914172287e-06
	1971 8.05469399534076e-06
	1972 8.05136517190164e-06
	1973 8.04806869503238e-06
	1974 8.04480301219712e-06
	1975 8.04158112543973e-06
	1976 8.0384198977157e-06
	1977 8.03533141002788e-06
	1978 8.03231890689204e-06
	1979 8.02937458743713e-06
	1980 8.02646886910452e-06
	1981 8.02357354512395e-06
	1982 8.020692386701e-06
	1983 8.01779061720964e-06
	1984 8.01483524526247e-06
	1985 8.01178243836631e-06
	1986 8.00858064220478e-06
	1987 8.00514727306023e-06
	1988 8.00142755075228e-06
	1989 7.9973180984183e-06
	1990 7.99275381702103e-06
	1991 7.98765061293949e-06
	1992 7.98189948270078e-06
	1993 7.97545031261393e-06
	1994 7.96823694049209e-06
	1995 7.96021071991504e-06
	1996 7.9513302253531e-06
	1997 7.9415638474245e-06
	1998 7.93091692408154e-06
	1999 7.91939244315643e-06
};
\addlegendentry{Train}
\addplot [semithick, black]
table {%
	0 0.0145310219377279
	1 0.0141394129022956
	2 0.0137565331533551
	3 0.01337921153754
	4 0.0130049102008343
	5 0.0126322489231825
	6 0.0122610358521342
	7 0.0118919247761369
	8 0.0115261059254408
	9 0.0111650126054883
	10 0.0108097679913044
	11 0.0104605024680495
	12 0.0101159233599901
	13 0.00977340247482061
	14 0.00942955445498228
	15 0.0090814670547843
	16 0.008728570304811
	17 0.00837336573749781
	18 0.00801966339349747
	19 0.0076706656254828
	20 0.00732881296426058
	21 0.00699608959257603
	22 0.00667411647737026
	23 0.00636415556073189
	24 0.00606703897938132
	25 0.00578313879668713
	26 0.00551245454698801
	27 0.00525472545996308
	28 0.00500952219590545
	29 0.00477629480883479
	30 0.00455442816019058
	31 0.00434330524876714
	32 0.00414235610514879
	33 0.00395109318196774
	34 0.00376911996863782
	35 0.0035961028188467
	36 0.00343173136934638
	37 0.00327568803913891
	38 0.0031276373192668
	39 0.00298722740262747
	40 0.00285410019569099
	41 0.00272789783775806
	42 0.00260826875455678
	43 0.00249487045221031
	44 0.00238737324252725
	45 0.00228545977734029
	46 0.0021888252813369
	47 0.00209718011319637
	48 0.00201024767011404
	49 0.00192776555195451
	50 0.00184948625974357
	51 0.00177517463453114
	52 0.00170460890512913
	53 0.00163758010603487
	54 0.00157389079686254
	55 0.00151335564441979
	56 0.00145579979289323
	57 0.00140105921309441
	58 0.00134897907264531
	59 0.00129941420163959
	60 0.00125222816132009
	61 0.00120729219634086
	62 0.00116448570042849
	63 0.00112369551789016
	64 0.00108481443021446
	65 0.00104774220380932
	66 0.00101238477509469
	67 0.000978653086349368
	68 0.000946463260333985
	69 0.000915736774913967
	70 0.000886399357113987
	71 0.000858381274156272
	72 0.000831616576761007
	73 0.000806043215561658
	74 0.000781602517236024
	75 0.000758239242713898
	76 0.000735901121515781
	77 0.00071453902637586
	78 0.000694106216542423
	79 0.000674558454193175
	80 0.000655854237265885
	81 0.000637954100966454
	82 0.000620820675976574
	83 0.00060441845562309
	84 0.000588714028708637
	85 0.000573675497435033
	86 0.000559272826649249
	87 0.000545477319974452
	88 0.000532261736225337
	89 0.0005196004640311
	90 0.000507468881551176
	91 0.000495843414682895
	92 0.000484702060930431
	93 0.000474023807328194
	94 0.000463788310298696
	95 0.000453976623248309
	96 0.000444570439867675
	97 0.000435552152339369
	98 0.000426905520725995
	99 0.000418614567024633
	100 0.000410663982620463
	101 0.000403039623051882
	102 0.000395727693103254
	103 0.000388714746804908
	104 0.000381988473236561
	105 0.000375536677893251
	106 0.000369347893865779
	107 0.000363411119906232
	108 0.000357715791324154
	109 0.000352251896401867
	110 0.000347009568940848
	111 0.000341979670338333
	112 0.000337153178406879
	113 0.000332521769450977
	114 0.000328077119775116
	115 0.000323811429552734
	116 0.000319717160891742
	117 0.000315787096042186
	118 0.000312014512019232
	119 0.000308392569422722
	120 0.000304914719890803
	121 0.000301575142657384
	122 0.000298367958748713
	123 0.000295287318294868
	124 0.000292327749775723
	125 0.000289484392851591
	126 0.000286752037936822
	127 0.000284125795587897
	128 0.000281601125607267
	129 0.000279173691524193
	130 0.000276839069556445
	131 0.000274593359790742
	132 0.000272432371275499
	133 0.000270352582447231
	134 0.000268350209807977
	135 0.000266421906417236
	136 0.000264564121607691
	137 0.000262773712165654
	138 0.000261047942331061
	139 0.000259383348748088
	140 0.000257777341175824
	141 0.000256227154750377
	142 0.000254730170127004
	143 0.000253283884376287
	144 0.000251885940087959
	145 0.000250533863436431
	146 0.00024922561715357
	147 0.000247958902036771
	148 0.000246731797233224
	149 0.000245542381890118
	150 0.000244388618739322
	151 0.00024326890707016
	152 0.000242181326029822
	153 0.000241124245803803
	154 0.00024009621120058
	155 0.000239095665165223
	156 0.000238121079746634
	157 0.000237171072512865
	158 0.000236244333791547
	159 0.000235339583014138
	160 0.000234455656027421
	161 0.000233591243159026
	162 0.000232745354878716
	163 0.000231916710617952
	164 0.000231104510021396
	165 0.00023030761803966
	166 0.00022952510335017
	167 0.000228756136493757
	168 0.000227999800699763
	169 0.000227255339268595
	170 0.000226521777221933
	171 0.000225798503379337
	172 0.000225084615522064
	173 0.000224379720748402
	174 0.000223682945943438
	175 0.000222993563511409
	176 0.000222311078687198
	177 0.000221634982153773
	178 0.000220964502659626
	179 0.000220299421926029
	180 0.000219638866838068
	181 0.000218982590013184
	182 0.000218329907511361
	183 0.00021768051374238
	184 0.000217033957596868
	185 0.000216389846173115
	186 0.000215747728361748
	187 0.000215107153053395
	188 0.000214467901969329
	189 0.00021382950944826
	190 0.00021319164079614
	191 0.000212554019526578
	192 0.000211916412808932
	193 0.00021127842774149
	194 0.000210639758734033
	195 0.00021000012930017
	196 0.000209359335713089
	197 0.00020871713059023
	198 0.000208073295652866
	199 0.000207427452551201
	200 0.000206779528525658
	201 0.000206129348953255
	202 0.000205476608243771
	203 0.000204821291845292
	204 0.000204163123271428
	205 0.000203501665964723
	206 0.000202837225515395
	207 0.000202169467229396
	208 0.000201498172827996
	209 0.000200823225895874
	210 0.000200144742848352
	211 0.000199462287127972
	212 0.000198776018805802
	213 0.000198085734155029
	214 0.000197391331312247
	215 0.000196692955796607
	216 0.000195990229258314
	217 0.000195283311768435
	218 0.000194572159671225
	219 0.000193856641999446
	220 0.000193136962479912
	221 0.00019241294648964
	222 0.00019168462313246
	223 0.000190952036064118
	224 0.000190215214388445
	225 0.000189474274520762
	226 0.000188729012734257
	227 0.000187979836482555
	228 0.000187226571142673
	229 0.000186469493201002
	230 0.000185708631761372
	231 0.000184944059583358
	232 0.000184175922186114
	233 0.000183404365088791
	234 0.000182629490154795
	235 0.000181851341039874
	236 0.000181070325197652
	237 0.000180286384420469
	238 0.000179499722435139
	239 0.000178710601176135
	240 0.000177919078851119
	241 0.000177125359186903
	242 0.000176329733221792
	243 0.00017553212819621
	244 0.000174732937011868
	245 0.000173932261532173
	246 0.000173130276380107
	247 0.000172327112522908
	248 0.000171522973687388
	249 0.000170718049048446
	250 0.000169912382261828
	251 0.000169106366229244
	252 0.000168299811775796
	253 0.000167493155458942
	254 0.000166686368174851
	255 0.000165879668202251
	256 0.000165073171956465
	257 0.000164266981300898
	258 0.000163461314514279
	259 0.000162656142492779
	260 0.000161851712618954
	261 0.00016104819951579
	262 0.000160245588631369
	263 0.000159444156452082
	264 0.000158643888426013
	265 0.000157845046487637
	266 0.000157047703396529
	267 0.000156251961016096
	268 0.000155458124936558
	269 0.00015466615150217
	270 0.000153876389958896
	271 0.00015308876754716
	272 0.000152303618961014
	273 0.000151520987856202
	274 0.000150741194374859
	275 0.000149964209413156
	276 0.000149190324009396
	277 0.000148419654578902
	278 0.000147652448504232
	279 0.000146888734889217
	280 0.000146128833875991
	281 0.000145372760016471
	282 0.000144620848004706
	283 0.000143873039633036
	284 0.000143129684147425
	285 0.000142390868859366
	286 0.000141656724736094
	287 0.000140927426400594
	288 0.000140203090268187
	289 0.00013948384730611
	290 0.000138769813929684
	291 0.000138061179313809
	292 0.000137357943458483
	293 0.000136660280986689
	294 0.000135968293761835
	295 0.000135282025439665
	296 0.000134601563331671
	297 0.00013392698019743
	298 0.000133258334244601
	299 0.000132595654577017
	300 0.000131938999402337
	301 0.000131288397824392
	302 0.000130643835291266
	303 0.000130005340906791
	304 0.000129372900119051
	305 0.000128746512928046
	306 0.0001281261065742
	307 0.000127511695609428
	308 0.000126903236377984
	309 0.000126300597912632
	310 0.000125703838421032
	311 0.000125112768728286
	312 0.000124527316074818
	313 0.000123947422252968
	314 0.000123373058158904
	315 0.000122803932754323
	316 0.000122240104246885
	317 0.000121681332529988
	318 0.000121127552120015
	319 0.000120578537462279
	320 0.00012003425945295
	321 0.000119494507089257
	322 0.00011895909847226
	323 0.000118427939014509
	324 0.000117900824989192
	325 0.00011737753811758
	326 0.000116858012916055
	327 0.000116342016553972
	328 0.000115829381684307
	329 0.000115319919132162
	330 0.000114813483378384
	331 0.000114309856144246
	332 0.000113808913738467
	333 0.000113310430606361
	334 0.000112814210297074
	335 0.000112320143671241
	336 0.000111828012450133
	337 0.000111337649286725
	338 0.000110848901385907
	339 0.000110361572296824
	340 0.000109875509224366
	341 0.000109390566649381
	342 0.00010890655539697
	343 0.000108423322672024
	344 0.000107940788439009
	345 0.000107458705315366
	346 0.000106976985989604
	347 0.00010649548494257
	348 0.00010601403482724
	349 0.000105532570159994
	350 0.000105050967249554
	351 0.000104569066024851
	352 0.000104086757346522
	353 0.000103603932075202
	354 0.000103120517451316
	355 0.000102636367955711
	356 0.000102151461760513
	357 0.000101665682450403
	358 0.000101178920886014
	359 0.000100691126135644
	360 0.000100202269095462
	361 9.97122479020618e-05
	362 9.92209679679945e-05
	363 9.87284438451752e-05
	364 9.8234559118282e-05
	365 9.77393283392303e-05
	366 9.72427369561046e-05
	367 9.67446394497529e-05
	368 9.62451304076239e-05
	369 9.57441152422689e-05
	370 9.52416012296453e-05
	371 9.47375883697532e-05
	372 9.42320111789741e-05
	373 9.37248987611383e-05
	374 9.32162729441188e-05
	375 9.2706126451958e-05
	376 9.21944665606134e-05
	377 9.16812787181698e-05
	378 9.11665920284577e-05
	379 9.06504792510532e-05
	380 9.01328967302106e-05
	381 8.96138735697605e-05
	382 8.90934461494908e-05
	383 8.85716872289777e-05
	384 8.80485968082212e-05
	385 8.75242112670094e-05
	386 8.69985524332151e-05
	387 8.64717803779058e-05
	388 8.59437859617174e-05
	389 8.5414714703802e-05
	390 8.48846029839478e-05
	391 8.43534653540701e-05
	392 8.38214255054481e-05
	393 8.32885270938277e-05
	394 8.27547773951665e-05
	395 8.22202709969133e-05
	396 8.16850952105597e-05
	397 8.11492718639784e-05
	398 8.06128955446184e-05
	399 8.00760462880135e-05
	400 7.9538811405655e-05
	401 7.90011981735006e-05
	402 7.84633302828297e-05
	403 7.79252732172608e-05
	404 7.73870851844549e-05
	405 7.68488534959033e-05
	406 7.63106800150126e-05
	407 7.57726156734861e-05
	408 7.52347477828152e-05
	409 7.46971491025761e-05
	410 7.41599287721328e-05
	411 7.36231522751041e-05
	412 7.30868923710659e-05
	413 7.2551250923425e-05
	414 7.20162934157997e-05
	415 7.1482099883724e-05
	416 7.09487721906044e-05
	417 7.04163903719746e-05
	418 6.98850635671988e-05
	419 6.9354849983938e-05
	420 6.88258660375141e-05
	421 6.82981626596302e-05
	422 6.77718271617778e-05
	423 6.72470196150243e-05
	424 6.67237254674546e-05
	425 6.62020829622634e-05
	426 6.56822303426452e-05
	427 6.51641676086001e-05
	428 6.46480184514076e-05
	429 6.41339211142622e-05
	430 6.36218974250369e-05
	431 6.31120637990534e-05
	432 6.26045148237608e-05
	433 6.20993232587352e-05
	434 6.15965982433408e-05
	435 6.10964343650267e-05
	436 6.05988716415595e-05
	437 6.0104070144007e-05
	438 5.96120335103478e-05
	439 5.91229363635648e-05
	440 5.86368369113188e-05
	441 5.81537460675463e-05
	442 5.76738348172512e-05
	443 5.7197186833946e-05
	444 5.67238566873129e-05
	445 5.62538843951188e-05
	446 5.57874300284311e-05
	447 5.5324529967038e-05
	448 5.48652642464731e-05
	449 5.44097056263126e-05
	450 5.39579632459208e-05
	451 5.35100552951917e-05
	452 5.30660981894471e-05
	453 5.26261319464538e-05
	454 5.21902184118517e-05
	455 5.17584303452168e-05
	456 5.1330844144104e-05
	457 5.09075434820261e-05
	458 5.04884810652584e-05
	459 5.00738242408261e-05
	460 4.96635620947927e-05
	461 4.92577746626921e-05
	462 4.88564764964394e-05
	463 4.8459736717632e-05
	464 4.80675698781852e-05
	465 4.76800305477809e-05
	466 4.72971587441862e-05
	467 4.69189617433585e-05
	468 4.65454759250861e-05
	469 4.61767231172416e-05
	470 4.58127178717405e-05
	471 4.54535002063494e-05
	472 4.50990628451109e-05
	473 4.47494094260037e-05
	474 4.44045472249854e-05
	475 4.40645235357806e-05
	476 4.37292728747707e-05
	477 4.33988607255742e-05
	478 4.30732070526574e-05
	479 4.27523409598507e-05
	480 4.24362224293873e-05
	481 4.21248769271187e-05
	482 4.18182862631511e-05
	483 4.1516381315887e-05
	484 4.1219169361284e-05
	485 4.09266031056177e-05
	486 4.06386789109092e-05
	487 4.03553458454553e-05
	488 4.00765711674467e-05
	489 3.98023075831588e-05
	490 3.95325587305706e-05
	491 3.92672045563813e-05
	492 3.90062632504851e-05
	493 3.87496729672421e-05
	494 3.8497364585055e-05
	495 3.82493053621147e-05
	496 3.80054516426753e-05
	497 3.77657306671608e-05
	498 3.75301096937619e-05
	499 3.7298505048966e-05
	500 3.70708548871335e-05
	501 3.6847144656349e-05
	502 3.66272761311848e-05
	503 3.6411227483768e-05
	504 3.61989004886709e-05
	505 3.59902514901478e-05
	506 3.57851968146861e-05
	507 3.55837219103705e-05
	508 3.53857103618793e-05
	509 3.5191133065382e-05
	510 3.49999281752389e-05
	511 3.48119938280433e-05
	512 3.46273263858166e-05
	513 3.44458348990884e-05
	514 3.42674502462614e-05
	515 3.40920960297808e-05
	516 3.3919743145816e-05
	517 3.37503406626638e-05
	518 3.35837721650023e-05
	519 3.34200303768739e-05
	520 3.32590389007237e-05
	521 3.31007395288907e-05
	522 3.294505586382e-05
	523 3.27919333358295e-05
	524 3.2641357393004e-05
	525 3.2493222533958e-05
	526 3.23474778269883e-05
	527 3.22040941682644e-05
	528 3.20630133501254e-05
	529 3.19241880788468e-05
	530 3.17875201290008e-05
	531 3.16530276904814e-05
	532 3.15206016239244e-05
	533 3.13902201014571e-05
	534 3.12618249154184e-05
	535 3.11354124278296e-05
	536 3.10108662233688e-05
	537 3.08881790260784e-05
	538 3.07673071802128e-05
	539 3.06481997540686e-05
	540 3.05308240058366e-05
	541 3.04151344607817e-05
	542 3.03010783682112e-05
	543 3.01886320812628e-05
	544 3.00777373922756e-05
	545 2.99683943012496e-05
	546 2.98605500574922e-05
	547 2.97541282634484e-05
	548 2.96491343760863e-05
	549 2.95455320156179e-05
	550 2.94432757073082e-05
	551 2.93423527182313e-05
	552 2.92426939267898e-05
	553 2.9144304789952e-05
	554 2.9047145289951e-05
	555 2.89511790469987e-05
	556 2.88563660433283e-05
	557 2.87627444777172e-05
	558 2.86701961158542e-05
	559 2.85787282336969e-05
	560 2.84883517451817e-05
	561 2.8399022994563e-05
	562 2.8310671041254e-05
	563 2.82233268080745e-05
	564 2.81369520962471e-05
	565 2.80515214399202e-05
	566 2.79669984593056e-05
	567 2.78833795164246e-05
	568 2.78006664302666e-05
	569 2.77187991741812e-05
	570 2.76377559202956e-05
	571 2.7557542125578e-05
	572 2.74781486950815e-05
	573 2.73995246971026e-05
	574 2.73216774075991e-05
	575 2.72445795417298e-05
	576 2.71682201855583e-05
	577 2.70925811491907e-05
	578 2.70176460617222e-05
	579 2.69433967332589e-05
	580 2.68698113359278e-05
	581 2.67969044216443e-05
	582 2.67246359726414e-05
	583 2.66530023509404e-05
	584 2.65819689957425e-05
	585 2.65115377260372e-05
	586 2.64417194557609e-05
	587 2.63724796241149e-05
	588 2.63038218690781e-05
	589 2.62357116298517e-05
	590 2.61681307165418e-05
	591 2.61010827671271e-05
	592 2.60345750575652e-05
	593 2.59685984929092e-05
	594 2.59031112364028e-05
	595 2.58381242019823e-05
	596 2.57735810009763e-05
	597 2.57095543929609e-05
	598 2.56459788943175e-05
	599 2.55828617810039e-05
	600 2.55201866821153e-05
	601 2.5457968149567e-05
	602 2.53961916314438e-05
	603 2.53348098340211e-05
	604 2.52738536801189e-05
	605 2.52133304456947e-05
	606 2.51532037509605e-05
	607 2.50934444920858e-05
	608 2.50340872298693e-05
	609 2.49751246883534e-05
	610 2.49165223067394e-05
	611 2.48583073698683e-05
	612 2.48004125751322e-05
	613 2.47429088631179e-05
	614 2.46857525780797e-05
	615 2.46289218921447e-05
	616 2.45724731939845e-05
	617 2.45163500949275e-05
	618 2.44605125772068e-05
	619 2.44050443143351e-05
	620 2.43498689087573e-05
	621 2.42950336541981e-05
	622 2.42404985328903e-05
	623 2.41862453549402e-05
	624 2.41323177760933e-05
	625 2.40786484937416e-05
	626 2.40253084484721e-05
	627 2.39722357946448e-05
	628 2.39194469031645e-05
	629 2.38669508689782e-05
	630 2.38147331401706e-05
	631 2.37627737078583e-05
	632 2.37110962189035e-05
	633 2.36596861213911e-05
	634 2.36085325013846e-05
	635 2.35576117120218e-05
	636 2.35069765039952e-05
	637 2.34565704886336e-05
	638 2.34064209507778e-05
	639 2.33565151575021e-05
	640 2.33068458328489e-05
	641 2.32574366236804e-05
	642 2.32082293223357e-05
	643 2.31592912314227e-05
	644 2.31105823331745e-05
	645 2.30620844376972e-05
	646 2.30138011829695e-05
	647 2.29657416639384e-05
	648 2.29179167945404e-05
	649 2.28702992899343e-05
	650 2.28228927880991e-05
	651 2.27757009270135e-05
	652 2.27287109737517e-05
	653 2.26819411182078e-05
	654 2.26353640755406e-05
	655 2.25889743887819e-05
	656 2.25427975237835e-05
	657 2.24967989197467e-05
	658 2.24509985855548e-05
	659 2.24054037971655e-05
	660 2.2360020011547e-05
	661 2.23147890210385e-05
	662 2.22697544813855e-05
	663 2.22248854697682e-05
	664 2.21802056330489e-05
	665 2.21357095142594e-05
	666 2.20913316297811e-05
	667 2.2047184756957e-05
	668 2.20032034121687e-05
	669 2.19594094232889e-05
	670 2.19157482206356e-05
	671 2.18722852878273e-05
	672 2.18289569602348e-05
	673 2.17857887037098e-05
	674 2.17427841562312e-05
	675 2.16999269468943e-05
	676 2.16572443605401e-05
	677 2.16146981983911e-05
	678 2.15722902794369e-05
	679 2.15300278796349e-05
	680 2.14879546547309e-05
	681 2.14459905691911e-05
	682 2.14042029256234e-05
	683 2.13625135074835e-05
	684 2.13210114452522e-05
	685 2.12796257983427e-05
	686 2.12383893085644e-05
	687 2.11972710530972e-05
	688 2.11562946788035e-05
	689 2.11154583666939e-05
	690 2.10747366509167e-05
	691 2.10341950150905e-05
	692 2.09937479667133e-05
	693 2.09534064197214e-05
	694 2.09132049349137e-05
	695 2.08731398743112e-05
	696 2.08331912290305e-05
	697 2.07933535421034e-05
	698 2.07536613743287e-05
	699 2.07140947168227e-05
	700 2.0674637198681e-05
	701 2.06352906388929e-05
	702 2.05960859602783e-05
	703 2.05569904210279e-05
	704 2.05180240300251e-05
	705 2.04791467695031e-05
	706 2.04404204851016e-05
	707 2.04017633222975e-05
	708 2.03632589546032e-05
	709 2.03248309844639e-05
	710 2.02865521714557e-05
	711 2.02483515749918e-05
	712 2.02102855837438e-05
	713 2.01723414647859e-05
	714 2.01344719243934e-05
	715 2.00967206183122e-05
	716 2.00591020984575e-05
	717 2.00215872609988e-05
	718 1.99841942958301e-05
	719 1.99468886421528e-05
	720 1.99096921278397e-05
	721 1.98726374947e-05
	722 1.98356556211365e-05
	723 1.97987737919902e-05
	724 1.97620083781658e-05
	725 1.97253575606737e-05
	726 1.96887704078108e-05
	727 1.9652332412079e-05
	728 1.96160162886372e-05
	729 1.95797474589199e-05
	730 1.95436277863337e-05
	731 1.9507600882207e-05
	732 1.94716612895718e-05
	733 1.94358490261948e-05
	734 1.94000931514893e-05
	735 1.93645009858301e-05
	736 1.93289815797471e-05
	737 1.92935640370706e-05
	738 1.9258262909716e-05
	739 1.92230600077892e-05
	740 1.91879171325127e-05
	741 1.91529434232507e-05
	742 1.91180079127662e-05
	743 1.90832051885081e-05
	744 1.90484752238262e-05
	745 1.90138543985086e-05
	746 1.89793408935657e-05
	747 1.89449510799022e-05
	748 1.89106358448043e-05
	749 1.88764388440177e-05
	750 1.88423382496694e-05
	751 1.88083195098443e-05
	752 1.87743844435317e-05
	753 1.87405403266894e-05
	754 1.87068453669781e-05
	755 1.86732177098747e-05
	756 1.86396773642628e-05
	757 1.86062716238666e-05
	758 1.85729149961844e-05
	759 1.8539665688877e-05
	760 1.85065273399232e-05
	761 1.84735017683124e-05
	762 1.84405671461718e-05
	763 1.84077198355226e-05
	764 1.83749489224283e-05
	765 1.83422998816241e-05
	766 1.8309732695343e-05
	767 1.82772619155003e-05
	768 1.82449075509794e-05
	769 1.82126241270453e-05
	770 1.81804243766237e-05
	771 1.81482791958842e-05
	772 1.81162795342971e-05
	773 1.80843799171271e-05
	774 1.80525894393213e-05
	775 1.80208644451341e-05
	776 1.79892413143534e-05
	777 1.79577073140536e-05
	778 1.79262824531179e-05
	779 1.78949267137796e-05
	780 1.78636873897631e-05
	781 1.78325353772379e-05
	782 1.78014433913631e-05
	783 1.77704623638419e-05
	784 1.7739590475685e-05
	785 1.77087695192313e-05
	786 1.76780285983114e-05
	787 1.76473822648404e-05
	788 1.76168796315324e-05
	789 1.75864697666839e-05
	790 1.75561581272632e-05
	791 1.75258992385352e-05
	792 1.74957531271502e-05
	793 1.74656797753414e-05
	794 1.74356955540134e-05
	795 1.74057950061979e-05
	796 1.73759726749267e-05
	797 1.73462631209986e-05
	798 1.73166317836149e-05
	799 1.72871132235741e-05
	800 1.72576728800777e-05
	801 1.72283162100939e-05
	802 1.7199026842718e-05
	803 1.71698484336957e-05
	804 1.71407682501012e-05
	805 1.71117553691147e-05
	806 1.70828097907361e-05
	807 1.70539824466687e-05
	808 1.70252333191456e-05
	809 1.69966060639126e-05
	810 1.69680315593723e-05
	811 1.69395570992492e-05
	812 1.69111717696069e-05
	813 1.68828610185301e-05
	814 1.68546011991566e-05
	815 1.68264614330838e-05
	816 1.679838533164e-05
	817 1.67704456544016e-05
	818 1.67425696417922e-05
	819 1.67148100445047e-05
	820 1.66871013789205e-05
	821 1.66594763868488e-05
	822 1.66319314303109e-05
	823 1.66044828802114e-05
	824 1.65770761668682e-05
	825 1.6549791325815e-05
	826 1.6522570149391e-05
	827 1.64954872161616e-05
	828 1.64684843184659e-05
	829 1.6441534171463e-05
	830 1.6414678611909e-05
	831 1.63878848979948e-05
	832 1.63611803145614e-05
	833 1.63345866894815e-05
	834 1.63080312631791e-05
	835 1.62815831572516e-05
	836 1.62552314577624e-05
	837 1.62289743457222e-05
	838 1.62027863552794e-05
	839 1.61766874953173e-05
	840 1.61506759468466e-05
	841 1.61247153300792e-05
	842 1.60988693096442e-05
	843 1.6073026927188e-05
	844 1.60473227879265e-05
	845 1.60216659423895e-05
	846 1.59961346071213e-05
	847 1.59706978593022e-05
	848 1.59453247761121e-05
	849 1.5920046280371e-05
	850 1.58948569151107e-05
	851 1.58697293954901e-05
	852 1.58446764544351e-05
	853 1.58196926349774e-05
	854 1.57947815750958e-05
	855 1.57699687406421e-05
	856 1.57452541316161e-05
	857 1.5720579540357e-05
	858 1.5696045011282e-05
	859 1.56715541379526e-05
	860 1.56471487571253e-05
	861 1.56227961269906e-05
	862 1.55985289893579e-05
	863 1.55743509822059e-05
	864 1.55502657435136e-05
	865 1.55262496264186e-05
	866 1.55023026309209e-05
	867 1.5478462955798e-05
	868 1.54546742123784e-05
	869 1.54309691424714e-05
	870 1.54073204612359e-05
	871 1.53837972902693e-05
	872 1.53603559738258e-05
	873 1.53369892359478e-05
	874 1.53136752487626e-05
	875 1.52904303831747e-05
	876 1.52672801050358e-05
	877 1.52441962200101e-05
	878 1.52211605382035e-05
	879 1.51982094394043e-05
	880 1.51753602040117e-05
	881 1.5152630112425e-05
	882 1.51299545905204e-05
	883 1.51073336382979e-05
	884 1.50848027260508e-05
	885 1.50623300214647e-05
	886 1.50399073390872e-05
	887 1.50175892486004e-05
	888 1.49953812069725e-05
	889 1.49732150021009e-05
	890 1.49511051859008e-05
	891 1.49291054185596e-05
	892 1.49071320265648e-05
	893 1.48852868733229e-05
	894 1.4863502656226e-05
	895 1.48417720993166e-05
	896 1.48201434058137e-05
	897 1.479857837694e-05
	898 1.47770997500629e-05
	899 1.47556929732673e-05
	900 1.47343516800902e-05
	901 1.47130676850793e-05
	902 1.46918991958955e-05
	903 1.46707707244786e-05
	904 1.46497322930372e-05
	905 1.46287447932991e-05
	906 1.46078155012219e-05
	907 1.45869926200248e-05
	908 1.45662779686972e-05
	909 1.45455769597902e-05
	910 1.45249541674275e-05
	911 1.45043995871674e-05
	912 1.44839495987981e-05
	913 1.44635587275843e-05
	914 1.44432387969573e-05
	915 1.44230034493376e-05
	916 1.44028381328098e-05
	917 1.43827192005119e-05
	918 1.43626657518325e-05
	919 1.4342685972224e-05
	920 1.43228007800644e-05
	921 1.43029446917353e-05
	922 1.42832250276115e-05
	923 1.42635235533817e-05
	924 1.42439112096326e-05
	925 1.42243798109121e-05
	926 1.42048947964213e-05
	927 1.41855252877576e-05
	928 1.41662121677655e-05
	929 1.41469708978548e-05
	930 1.41277250804706e-05
	931 1.41086038638605e-05
	932 1.40895544973318e-05
	933 1.40705787998741e-05
	934 1.40516694955295e-05
	935 1.40327774715843e-05
	936 1.40140309667913e-05
	937 1.39953117468394e-05
	938 1.39766798383789e-05
	939 1.39581361509045e-05
	940 1.39396242957446e-05
	941 1.39211651912774e-05
	942 1.39027843033546e-05
	943 1.3884469808545e-05
	944 1.38662308017956e-05
	945 1.38480781970429e-05
	946 1.38299637910677e-05
	947 1.38119221446686e-05
	948 1.37939596243086e-05
	949 1.37760607685777e-05
	950 1.37582210300025e-05
	951 1.37404322231305e-05
	952 1.37227207233082e-05
	953 1.37050892590196e-05
	954 1.36875160023919e-05
	955 1.36700018629199e-05
	956 1.36525595735293e-05
	957 1.36351927721989e-05
	958 1.36178741740878e-05
	959 1.36006292450475e-05
	960 1.35834643515409e-05
	961 1.35663449327694e-05
	962 1.35492809931748e-05
	963 1.35322534333682e-05
	964 1.35153231894947e-05
	965 1.34984538817662e-05
	966 1.34816555146244e-05
	967 1.346492626908e-05
	968 1.34482497742283e-05
	969 1.34316414914792e-05
	970 1.3415088687907e-05
	971 1.33986095534055e-05
	972 1.33821904455544e-05
	973 1.33658404593007e-05
	974 1.33495486807078e-05
	975 1.33333051053341e-05
	976 1.33171079141903e-05
	977 1.33010053104954e-05
	978 1.32849127112422e-05
	979 1.32689247038797e-05
	980 1.32529867187259e-05
	981 1.32371051222435e-05
	982 1.32212817334221e-05
	983 1.32055156427668e-05
	984 1.3189836863603e-05
	985 1.31742090161424e-05
	986 1.31586521092686e-05
	987 1.31431561385398e-05
	988 1.3127687452652e-05
	989 1.31122969833086e-05
	990 1.3096971088089e-05
	991 1.30816752061946e-05
	992 1.30664493553923e-05
	993 1.30512935356819e-05
	994 1.30362213894841e-05
	995 1.30211592477281e-05
	996 1.3006167137064e-05
	997 1.29912295960821e-05
	998 1.2976365724171e-05
	999 1.29615391415427e-05
	1000 1.29467807710171e-05
	1001 1.29320769701735e-05
	1002 1.29174350149697e-05
	1003 1.29028676383314e-05
	1004 1.28883311845129e-05
	1005 1.28738729472389e-05
	1006 1.28594483612687e-05
	1007 1.2845077435486e-05
	1008 1.28307610793854e-05
	1009 1.28165074784192e-05
	1010 1.28023339129868e-05
	1011 1.27881639855332e-05
	1012 1.27740622701822e-05
	1013 1.27600424093544e-05
	1014 1.27460525618517e-05
	1015 1.2732132745441e-05
	1016 1.27182793221436e-05
	1017 1.27044731925707e-05
	1018 1.26906779769342e-05
	1019 1.26769482449163e-05
	1020 1.26632630781387e-05
	1021 1.26496670418419e-05
	1022 1.26361028378597e-05
	1023 1.26226168504218e-05
	1024 1.26091836136766e-05
	1025 1.25957876662142e-05
	1026 1.25824626593385e-05
	1027 1.2569194950629e-05
	1028 1.25559390653507e-05
	1029 1.2542751392175e-05
	1030 1.25296082842397e-05
	1031 1.25165015560924e-05
	1032 1.25034630400478e-05
	1033 1.24904581753071e-05
	1034 1.24775015137857e-05
	1035 1.24646185213351e-05
	1036 1.24517719086725e-05
	1037 1.24390217024484e-05
	1038 1.24263078760123e-05
	1039 1.24136149679543e-05
	1040 1.24009939099778e-05
	1041 1.23883828564431e-05
	1042 1.23758463814738e-05
	1043 1.23633408293244e-05
	1044 1.23508734759525e-05
	1045 1.23384725156939e-05
	1046 1.2326108844718e-05
	1047 1.23137861010036e-05
	1048 1.23015188364661e-05
	1049 1.22893316074624e-05
	1050 1.22771543829003e-05
	1051 1.22650581033668e-05
	1052 1.22530227599782e-05
	1053 1.22410219773883e-05
	1054 1.22290466606501e-05
	1055 1.2217119547131e-05
	1056 1.22052069855272e-05
	1057 1.21933353511849e-05
	1058 1.21815201055142e-05
	1059 1.21697539725574e-05
	1060 1.21580615086714e-05
	1061 1.21464245239622e-05
	1062 1.21348230095464e-05
	1063 1.21232833407703e-05
	1064 1.21117827802664e-05
	1065 1.21003376989393e-05
	1066 1.20888926176121e-05
	1067 1.20774802780943e-05
	1068 1.20661134133115e-05
	1069 1.20547820188222e-05
	1070 1.20435297503718e-05
	1071 1.20322984002996e-05
	1072 1.20210961540579e-05
	1073 1.20099821288022e-05
	1074 1.19988808364724e-05
	1075 1.19878304758458e-05
	1076 1.19768019430921e-05
	1077 1.19658498078934e-05
	1078 1.19549122246099e-05
	1079 1.19440619528177e-05
	1080 1.19332462418242e-05
	1081 1.19224332593149e-05
	1082 1.19116693895194e-05
	1083 1.19009209811338e-05
	1084 1.18902062240522e-05
	1085 1.18795614980627e-05
	1086 1.18689413284301e-05
	1087 1.18583930088789e-05
	1088 1.18479038064834e-05
	1089 1.1837435522466e-05
	1090 1.18270081657101e-05
	1091 1.18166026368272e-05
	1092 1.18062553156051e-05
	1093 1.17959225462982e-05
	1094 1.17856197903166e-05
	1095 1.17753870654269e-05
	1096 1.17651634354843e-05
	1097 1.17549352580681e-05
	1098 1.17447980301222e-05
	1099 1.17346744445967e-05
	1100 1.1724612704711e-05
	1101 1.17145837066346e-05
	1102 1.17046201921767e-05
	1103 1.16946794150863e-05
	1104 1.16847704703105e-05
	1105 1.16749179142062e-05
	1106 1.16650799100171e-05
	1107 1.16553001134889e-05
	1108 1.16455212264555e-05
	1109 1.16358060040511e-05
	1110 1.1626087143668e-05
	1111 1.16164001155994e-05
	1112 1.1606752195803e-05
	1113 1.15971288323635e-05
	1114 1.15875654955744e-05
	1115 1.15780367195839e-05
	1116 1.15685379569186e-05
	1117 1.15590819405043e-05
	1118 1.15496550279204e-05
	1119 1.15403117888491e-05
	1120 1.15309603643254e-05
	1121 1.15216653284733e-05
	1122 1.15123630166636e-05
	1123 1.15031070890836e-05
	1124 1.14938811748289e-05
	1125 1.14847025542986e-05
	1126 1.14755548565881e-05
	1127 1.14664326247294e-05
	1128 1.14573495011427e-05
	1129 1.1448279110482e-05
	1130 1.14392905743443e-05
	1131 1.14302629299345e-05
	1132 1.14212944026804e-05
	1133 1.14123577077407e-05
	1134 1.1403461030568e-05
	1135 1.1394610737625e-05
	1136 1.13857640826609e-05
	1137 1.13769619929371e-05
	1138 1.13681717266445e-05
	1139 1.13594096546876e-05
	1140 1.13507012429181e-05
	1141 1.13420255729579e-05
	1142 1.13333953777328e-05
	1143 1.1324782462907e-05
	1144 1.13161950139329e-05
	1145 1.13076503112097e-05
	1146 1.12990901470766e-05
	1147 1.12905481728376e-05
	1148 1.12820707727224e-05
	1149 1.12736142909853e-05
	1150 1.12651750896475e-05
	1151 1.12567768155714e-05
	1152 1.12484322016826e-05
	1153 1.12401112346561e-05
	1154 1.12318130049971e-05
	1155 1.12235593405785e-05
	1156 1.12153147711069e-05
	1157 1.12070902105188e-05
	1158 1.11989002107293e-05
	1159 1.11907429527491e-05
	1160 1.11826111606206e-05
	1161 1.11745002868702e-05
	1162 1.11664203359396e-05
	1163 1.11583285615779e-05
	1164 1.11502686195308e-05
	1165 1.11422486952506e-05
	1166 1.11342769741896e-05
	1167 1.11263425424113e-05
	1168 1.11184372144635e-05
	1169 1.11105528048938e-05
	1170 1.11026938611758e-05
	1171 1.10948349174578e-05
	1172 1.10870078060543e-05
	1173 1.10792079794919e-05
	1174 1.10714290713076e-05
	1175 1.10636556200916e-05
	1176 1.10559358290629e-05
	1177 1.10482760646846e-05
	1178 1.10406053863699e-05
	1179 1.10329820017796e-05
	1180 1.10253567981999e-05
	1181 1.10177661554189e-05
	1182 1.10101909740479e-05
	1183 1.10026330730761e-05
	1184 1.09951079139137e-05
	1185 1.09875782072777e-05
	1186 1.09800830614404e-05
	1187 1.09726424852852e-05
	1188 1.09652137325611e-05
	1189 1.09578168121516e-05
	1190 1.09504399006255e-05
	1191 1.09430784505093e-05
	1192 1.09357379187713e-05
	1193 1.09284264908638e-05
	1194 1.09211214294191e-05
	1195 1.09138381958473e-05
	1196 1.09066104414524e-05
	1197 1.08993799585733e-05
	1198 1.08921758510405e-05
	1199 1.08849872049177e-05
	1200 1.08778149296995e-05
	1201 1.08706753962906e-05
	1202 1.08635604192386e-05
	1203 1.08564790934906e-05
	1204 1.08494068626896e-05
	1205 1.08423901110655e-05
	1206 1.08353788164095e-05
	1207 1.08283520603436e-05
	1208 1.08213716885075e-05
	1209 1.08143849502085e-05
	1210 1.08073891169624e-05
	1211 1.08004815047025e-05
	1212 1.07935829873895e-05
	1213 1.07866935650236e-05
	1214 1.07798223325517e-05
	1215 1.07729592855321e-05
	1216 1.07661307993112e-05
	1217 1.07593205029843e-05
	1218 1.07525647763396e-05
	1219 1.07458063212107e-05
	1220 1.07390615085023e-05
	1221 1.07323294287198e-05
	1222 1.07256473711459e-05
	1223 1.07189516711514e-05
	1224 1.07122777990298e-05
	1225 1.0705616659834e-05
	1226 1.06989991763839e-05
	1227 1.06923871499021e-05
	1228 1.0685787856346e-05
	1229 1.06791922007687e-05
	1230 1.0672628377506e-05
	1231 1.06660791061586e-05
	1232 1.06595534816734e-05
	1233 1.06530451375875e-05
	1234 1.06465458884486e-05
	1235 1.06400848380872e-05
	1236 1.06336819953867e-05
	1237 1.06272782431915e-05
	1238 1.06208935903851e-05
	1239 1.06145034806104e-05
	1240 1.06081233752775e-05
	1241 1.06017660073121e-05
	1242 1.05954031823785e-05
	1243 1.05890603663283e-05
	1244 1.05827411971404e-05
	1245 1.05764502222883e-05
	1246 1.05701737993513e-05
	1247 1.05639010143932e-05
	1248 1.05576546047814e-05
	1249 1.05514309325372e-05
	1250 1.05452299976605e-05
	1251 1.05390308817732e-05
	1252 1.05328517747694e-05
	1253 1.05267045000801e-05
	1254 1.05205472209491e-05
	1255 1.05144499684684e-05
	1256 1.05083190646837e-05
	1257 1.05022163552349e-05
	1258 1.04961272882065e-05
	1259 1.04900345831993e-05
	1260 1.04840046333265e-05
	1261 1.04779910543584e-05
	1262 1.04719683804433e-05
	1263 1.04659602584434e-05
	1264 1.04599639598746e-05
	1265 1.04539958556416e-05
	1266 1.04480423033237e-05
	1267 1.04421033029212e-05
	1268 1.04361943158437e-05
	1269 1.04303144325968e-05
	1270 1.04243945315829e-05
	1271 1.04185164673254e-05
	1272 1.04126229416579e-05
	1273 1.0406736691948e-05
	1274 1.04008968264679e-05
	1275 1.03950205811998e-05
	1276 1.03891716207727e-05
	1277 1.03833654065966e-05
	1278 1.03775882962509e-05
	1279 1.03718421087251e-05
	1280 1.03661104731145e-05
	1281 1.03604133983026e-05
	1282 1.03546908576391e-05
	1283 1.0348981959396e-05
	1284 1.034329761751e-05
	1285 1.03376214610762e-05
	1286 1.03319453046424e-05
	1287 1.03262691482087e-05
	1288 1.03205993582378e-05
	1289 1.03149131973623e-05
	1290 1.03092716017272e-05
	1291 1.03036463769968e-05
	1292 1.02980393421603e-05
	1293 1.02924623206491e-05
	1294 1.02869116744841e-05
	1295 1.02813446574146e-05
	1296 1.02758494904265e-05
	1297 1.02703334050602e-05
	1298 1.02648255051463e-05
	1299 1.02593357951264e-05
	1300 1.02538269857178e-05
	1301 1.02483109003515e-05
	1302 1.02428421087097e-05
	1303 1.02373669506051e-05
	1304 1.02318826975534e-05
	1305 1.02264430097421e-05
	1306 1.0220999683952e-05
	1307 1.02155763670453e-05
	1308 1.02101803349797e-05
	1309 1.02048052212922e-05
	1310 1.0199438293057e-05
	1311 1.0194074093306e-05
	1312 1.01887335404172e-05
	1313 1.01834129964118e-05
	1314 1.01781124612899e-05
	1315 1.01728101071785e-05
	1316 1.01674950201414e-05
	1317 1.01621681096731e-05
	1318 1.01568784884876e-05
	1319 1.01515661299345e-05
	1320 1.01462865131907e-05
	1321 1.01410159913939e-05
	1322 1.01357527455548e-05
	1323 1.01304885902209e-05
	1324 1.01252780950745e-05
	1325 1.01200712379068e-05
	1326 1.01148980320431e-05
	1327 1.010972300719e-05
	1328 1.01045907285879e-05
	1329 1.00994257081766e-05
	1330 1.00942752396804e-05
	1331 1.00891456895624e-05
	1332 1.00840197774232e-05
	1333 1.00788993222523e-05
	1334 1.00737652246607e-05
	1335 1.0068634765048e-05
	1336 1.00635061244247e-05
	1337 1.00584256870206e-05
	1338 1.00533306977013e-05
	1339 1.00482684501912e-05
	1340 1.0043185284303e-05
	1341 1.00381421361817e-05
	1342 1.00331035355339e-05
	1343 1.00280913102324e-05
	1344 1.00231081887614e-05
	1345 1.00181396192056e-05
	1346 1.00131537692505e-05
	1347 1.00081560958643e-05
	1348 1.00031775218667e-05
	1349 9.99819349090103e-06
	1350 9.99318672256777e-06
	1351 9.98819268716034e-06
	1352 9.98320956568932e-06
	1353 9.97823917714413e-06
	1354 9.97331881080754e-06
	1355 9.96842027234379e-06
	1356 9.96354356175289e-06
	1357 9.95866685116198e-06
	1358 9.95381105894921e-06
	1359 9.94896981865168e-06
	1360 9.94413039734354e-06
	1361 9.93930552795064e-06
	1362 9.93448520603124e-06
	1363 9.92966761259595e-06
	1364 9.92484183370834e-06
	1365 9.9200005934108e-06
	1366 9.91515025816625e-06
	1367 9.9103035609005e-06
	1368 9.90543958323542e-06
	1369 9.9005565061816e-06
	1370 9.89570435194764e-06
	1371 9.8908585641766e-06
	1372 9.88607916951878e-06
	1373 9.88130886980798e-06
	1374 9.87658677331638e-06
	1375 9.87189378065523e-06
	1376 9.86721897788811e-06
	1377 9.86255236057332e-06
	1378 9.85788574325852e-06
	1379 9.85323640634306e-06
	1380 9.8485897979117e-06
	1381 9.84393227554392e-06
	1382 9.8392529253033e-06
	1383 9.8345844889991e-06
	1384 9.82989058684325e-06
	1385 9.82519486569799e-06
	1386 9.8204773166799e-06
	1387 9.81575340119889e-06
	1388 9.81101948127616e-06
	1389 9.80629647528986e-06
	1390 9.80163531494327e-06
	1391 9.79701417236356e-06
	1392 9.7924184956355e-06
	1393 9.78783373284386e-06
	1394 9.78327534539858e-06
	1395 9.77873787633143e-06
	1396 9.77422951109475e-06
	1397 9.76970204646932e-06
	1398 9.76518913375912e-06
	1399 9.76068258751184e-06
	1400 9.75614329945529e-06
	1401 9.75161037786165e-06
	1402 9.74707472778391e-06
	1403 9.74254271568498e-06
	1404 9.73797523329267e-06
	1405 9.73341411736328e-06
	1406 9.72885391092859e-06
	1407 9.72427824308397e-06
	1408 9.71971167018637e-06
	1409 9.71516692516161e-06
	1410 9.71065219346201e-06
	1411 9.70616929407697e-06
	1412 9.70171458902769e-06
	1413 9.69728625932476e-06
	1414 9.6928988568834e-06
	1415 9.68849508353742e-06
	1416 9.68413496593712e-06
	1417 9.6797666628845e-06
	1418 9.67541745922063e-06
	1419 9.67106461757794e-06
	1420 9.66670449997764e-06
	1421 9.66229254117934e-06
	1422 9.6579124146956e-06
	1423 9.65351136983372e-06
	1424 9.64907121669967e-06
	1425 9.6446246971027e-06
	1426 9.64012451731833e-06
	1427 9.6356461654068e-06
	1428 9.63115508056944e-06
	1429 9.62668309512082e-06
	1430 9.62222748057684e-06
	1431 9.61783098318847e-06
	1432 9.6134872364928e-06
	1433 9.60921170189977e-06
	1434 9.60496527113719e-06
	1435 9.60076158662559e-06
	1436 9.59660519583849e-06
	1437 9.59249427978648e-06
	1438 9.58838245423976e-06
	1439 9.58425243879901e-06
	1440 9.58011241891654e-06
	1441 9.5759332907619e-06
	1442 9.57168867898872e-06
	1443 9.56742678681621e-06
	1444 9.563060302753e-06
	1445 9.55864470597589e-06
	1446 9.55417544901138e-06
	1447 9.54969345912104e-06
	1448 9.545158718538e-06
	1449 9.540642167849e-06
	1450 9.53613835008582e-06
	1451 9.53169546846766e-06
	1452 9.52733262238326e-06
	1453 9.52305617829552e-06
	1454 9.51886340772035e-06
	1455 9.51479978539282e-06
	1456 9.51085257838713e-06
	1457 9.50701632973505e-06
	1458 9.50326375459554e-06
	1459 9.49957575357985e-06
	1460 9.49594596022507e-06
	1461 9.49229615798686e-06
	1462 9.4885745056672e-06
	1463 9.48473189055221e-06
	1464 9.48074466577964e-06
	1465 9.47655462368857e-06
	1466 9.47214630286908e-06
	1467 9.46753243624698e-06
	1468 9.46273303270573e-06
	1469 9.45778901950689e-06
	1470 9.45274223340675e-06
	1471 9.44764178711921e-06
	1472 9.44257499213563e-06
	1473 9.43764098337851e-06
	1474 9.43291979638161e-06
	1475 9.42846691032173e-06
	1476 9.42437145567965e-06
	1477 9.42066435527522e-06
	1478 9.41740290727466e-06
	1479 9.41454254643759e-06
	1480 9.41205780691234e-06
	1481 9.40985046327114e-06
	1482 9.40783866099082e-06
	1483 9.40587142395088e-06
	1484 9.40373683988582e-06
	1485 9.40124664339237e-06
	1486 9.39823166845599e-06
	1487 9.39450910664164e-06
	1488 9.38991252041887e-06
	1489 9.38441462494666e-06
	1490 9.378024515172e-06
	1491 9.37081949814456e-06
	1492 9.36298965825699e-06
	1493 9.35471962293377e-06
	1494 9.34634590521455e-06
	1495 9.33819865167607e-06
	1496 9.33056253415998e-06
	1497 9.32381681195693e-06
	1498 9.31822796701454e-06
	1499 9.31413251237245e-06
	1500 9.31175691221142e-06
	1501 9.31132308323868e-06
	1502 9.31294016481843e-06
	1503 9.3166181613924e-06
	1504 9.32217790250434e-06
	1505 9.32921375351725e-06
	1506 9.33702540351078e-06
	1507 9.34466152102686e-06
	1508 9.35085790843004e-06
	1509 9.3541721071233e-06
	1510 9.35323259909637e-06
	1511 9.34700801735744e-06
	1512 9.3349844974e-06
	1513 9.31744580157101e-06
	1514 9.29526322579477e-06
	1515 9.26994198380271e-06
	1516 9.24303367355606e-06
	1517 9.21616265259217e-06
	1518 9.19055219128495e-06
	1519 9.16716999199707e-06
	1520 9.14659995032707e-06
	1521 9.12926316232188e-06
	1522 9.11555252969265e-06
	1523 9.10598464543e-06
	1524 9.10143535293173e-06
	1525 9.10346898308489e-06
	1526 9.11467941477895e-06
	1527 9.13923668122152e-06
	1528 9.18335172173101e-06
	1529 9.25491804082412e-06
	1530 9.36159176490037e-06
	1531 9.50466528593097e-06
	1532 9.66838615568122e-06
	1533 9.81010998657439e-06
	1534 9.86809300229652e-06
	1535 9.79945343715372e-06
	1536 9.62167177931406e-06
	1537 9.40593326959061e-06
	1538 9.22426534089027e-06
	1539 9.10835751710692e-06
	1540 9.05190245248377e-06
	1541 9.03628279047552e-06
	1542 9.05040087673115e-06
	1543 9.09658137970837e-06
	1544 9.18872592592379e-06
	1545 9.35059597395593e-06
	1546 9.6147678050329e-06
	1547 1.00161396403564e-05
	1548 1.05645685835043e-05
	1549 1.11789568109089e-05
	1550 1.16116270874045e-05
	1551 1.15213615572429e-05
	1552 1.08183039628784e-05
	1553 9.89581531030126e-06
	1554 9.31703380047111e-06
	1555 9.36880860535894e-06
	1556 9.91130491456715e-06
	1557 1.04631726571824e-05
	1558 1.05204371720902e-05
	1559 1.00444813142531e-05
	1560 9.48195156524889e-06
	1561 9.15301916393219e-06
	1562 9.02346073416993e-06
	1563 8.9740969997365e-06
	1564 8.95255743671441e-06
	1565 8.94760432856856e-06
	1566 8.95381435839226e-06
	1567 8.9654749899637e-06
	1568 8.97812060429715e-06
	1569 8.98854978004238e-06
	1570 8.99464430403896e-06
	1571 8.99567749002017e-06
	1572 8.99234510143287e-06
	1573 8.98618509381777e-06
	1574 8.97900463314727e-06
	1575 8.97216341400053e-06
	1576 8.96640995051712e-06
	1577 8.96197525435127e-06
	1578 8.95872653927654e-06
	1579 8.95641551323934e-06
	1580 8.95470020623179e-06
	1581 8.95328776095994e-06
	1582 8.95202265382977e-06
	1583 8.95076391316252e-06
	1584 8.94944150786614e-06
	1585 8.94803179107839e-06
	1586 8.94650202099001e-06
	1587 8.94491313374601e-06
	1588 8.94321055966429e-06
	1589 8.94144341145875e-06
	1590 8.93958895176183e-06
	1591 8.93770447873976e-06
	1592 8.93579272087663e-06
	1593 8.93384185474133e-06
	1594 8.93186643224908e-06
	1595 8.92985462996876e-06
	1596 8.92782918526791e-06
	1597 8.92574644240085e-06
	1598 8.92361185833579e-06
	1599 8.92147272679722e-06
	1600 8.91929630597588e-06
	1601 8.91705622052541e-06
	1602 8.91478794073919e-06
	1603 8.9124841906596e-06
	1604 8.91010586201446e-06
	1605 8.90769842953887e-06
	1606 8.90524643182289e-06
	1607 8.90276442078175e-06
	1608 8.90023511601612e-06
	1609 8.89770399226109e-06
	1610 8.89510465640342e-06
	1611 8.89248258317821e-06
	1612 8.88985232450068e-06
	1613 8.88718659552978e-06
	1614 8.88448812474962e-06
	1615 8.88178783498006e-06
	1616 8.87904297997011e-06
	1617 8.87630358192837e-06
	1618 8.873547812982e-06
	1619 8.87078567757271e-06
	1620 8.86798261490185e-06
	1621 8.86515044840053e-06
	1622 8.86233010533033e-06
	1623 8.85948975337669e-06
	1624 8.8566184786032e-06
	1625 8.85374083736679e-06
	1626 8.85085682966746e-06
	1627 8.84795827005291e-06
	1628 8.84504606801784e-06
	1629 8.84214477991918e-06
	1630 8.83924076333642e-06
	1631 8.83633583725896e-06
	1632 8.83339180290932e-06
	1633 8.830455954012e-06
	1634 8.82751828612527e-06
	1635 8.82454332895577e-06
	1636 8.82156200532336e-06
	1637 8.81856794876512e-06
	1638 8.81557843968039e-06
	1639 8.81258165463805e-06
	1640 8.80955758475466e-06
	1641 8.80653988133417e-06
	1642 8.80350125953555e-06
	1643 8.80047446116805e-06
	1644 8.79746676218929e-06
	1645 8.79444905876881e-06
	1646 8.79143499332713e-06
	1647 8.78841547091724e-06
	1648 8.78539322002325e-06
	1649 8.78236460266635e-06
	1650 8.77933962328825e-06
	1651 8.77632282936247e-06
	1652 8.77329966897378e-06
	1653 8.77026832313277e-06
	1654 8.76724061527057e-06
	1655 8.76421563589247e-06
	1656 8.76117883308325e-06
	1657 8.75814566825284e-06
	1658 8.75507430464495e-06
	1659 8.75200476002647e-06
	1660 8.74891065905103e-06
	1661 8.74580746312859e-06
	1662 8.74269153428031e-06
	1663 8.73954832059098e-06
	1664 8.73641965881689e-06
	1665 8.73327917361166e-06
	1666 8.7301195890177e-06
	1667 8.72698547027539e-06
	1668 8.72381860972382e-06
	1669 8.7206808530027e-06
	1670 8.71755401021801e-06
	1671 8.71445536176907e-06
	1672 8.71141128300223e-06
	1673 8.70840904099168e-06
	1674 8.70545227371622e-06
	1675 8.70254007168114e-06
	1676 8.69969426275929e-06
	1677 8.69689210958313e-06
	1678 8.69412542670034e-06
	1679 8.69134146341821e-06
	1680 8.68855568114668e-06
	1681 8.68576717039105e-06
	1682 8.68291499500629e-06
	1683 8.67995277076261e-06
	1684 8.67686139827128e-06
	1685 8.67359085532371e-06
	1686 8.67016206029803e-06
	1687 8.66653408593265e-06
	1688 8.66267964738654e-06
	1689 8.6586060206173e-06
	1690 8.65433139551897e-06
	1691 8.64990943227895e-06
	1692 8.64539015310584e-06
	1693 8.64082448970294e-06
	1694 8.63630339154042e-06
	1695 8.63191507960437e-06
	1696 8.62783781485632e-06
	1697 8.62414344737772e-06
	1698 8.62104116094997e-06
	1699 8.61864828038961e-06
	1700 8.61714761413168e-06
	1701 8.61668377183378e-06
	1702 8.61738408275414e-06
	1703 8.61937860463513e-06
	1704 8.62268188939197e-06
	1705 8.62722572492203e-06
	1706 8.63284185470548e-06
	1707 8.63915465743048e-06
	1708 8.64563298819121e-06
	1709 8.6514264694415e-06
	1710 8.65547463035909e-06
	1711 8.65657329995884e-06
	1712 8.65345919009997e-06
	1713 8.64503908815095e-06
	1714 8.63057539390866e-06
	1715 8.60989530337974e-06
	1716 8.58342082210584e-06
	1717 8.55209873407148e-06
	1718 8.51727600093e-06
	1719 8.48050876811612e-06
	1720 8.44343776407186e-06
	1721 8.40795019030338e-06
	1722 8.37658626551274e-06
	1723 8.35391074360814e-06
	1724 8.34939601190854e-06
	1725 8.38453888718504e-06
	1726 8.50979085953441e-06
	1727 8.8454362412449e-06
	1728 9.66792867984623e-06
	1729 1.15058455776307e-05
	1730 1.47326991282171e-05
	1731 1.73502794496017e-05
	1732 1.53228393173777e-05
	1733 1.117674219131e-05
	1734 9.41756024985807e-06
	1735 1.03485426734551e-05
	1736 1.06000052255695e-05
	1737 9.33055162022356e-06
	1738 8.54765039548511e-06
	1739 8.30732824397273e-06
	1740 8.24833750812104e-06
	1741 8.24161088530673e-06
	1742 8.24516246211715e-06
	1743 8.25504230306251e-06
	1744 8.26909035822609e-06
	1745 8.2802262113546e-06
	1746 8.28405518404907e-06
	1747 8.28154770715628e-06
	1748 8.27625626698136e-06
	1749 8.27106850920245e-06
	1750 8.26743234938476e-06
	1751 8.26562882139115e-06
	1752 8.26542054710444e-06
	1753 8.26632822281681e-06
	1754 8.26786981633632e-06
	1755 8.26960149424849e-06
	1756 8.27124131319579e-06
	1757 8.2726865002769e-06
	1758 8.2739652498276e-06
	1759 8.27512849355116e-06
	1760 8.27624080557143e-06
	1761 8.27739768283209e-06
	1762 8.27860822028015e-06
	1763 8.27987878437852e-06
	1764 8.28122938401066e-06
	1765 8.28263182484079e-06
	1766 8.2840742834378e-06
	1767 8.28550309961429e-06
	1768 8.28695556265302e-06
	1769 8.28836800792487e-06
	1770 8.28974316391395e-06
	1771 8.29107284516795e-06
	1772 8.29231794341467e-06
	1773 8.29350301501108e-06
	1774 8.29460532258963e-06
	1775 8.29563668958144e-06
	1776 8.29655164125143e-06
	1777 8.29739929031348e-06
	1778 8.29813870950602e-06
	1779 8.29878263175488e-06
	1780 8.29933105706004e-06
	1781 8.29973669169703e-06
	1782 8.30005501484266e-06
	1783 8.30022418085719e-06
	1784 8.30028966447571e-06
	1785 8.30024418974062e-06
	1786 8.30011231300887e-06
	1787 8.29982400318841e-06
	1788 8.29943382996134e-06
	1789 8.29892269393895e-06
	1790 8.29830332804704e-06
	1791 8.29755754239159e-06
	1792 8.29672171676066e-06
	1793 8.29576401883969e-06
	1794 8.29473083285848e-06
	1795 8.29359760246007e-06
	1796 8.29240798339015e-06
	1797 8.29113741929177e-06
	1798 8.28977317723911e-06
	1799 8.28831070975866e-06
	1800 8.2867900346173e-06
	1801 8.28519478091039e-06
	1802 8.28354768600548e-06
	1803 8.28185784484958e-06
	1804 8.28011616249569e-06
	1805 8.27832172944909e-06
	1806 8.27647545520449e-06
	1807 8.274580068246e-06
	1808 8.27264011604711e-06
	1809 8.27062740427209e-06
	1810 8.26858376967721e-06
	1811 8.26653104013531e-06
	1812 8.26443010737421e-06
	1813 8.26230643724557e-06
	1814 8.26015275379177e-06
	1815 8.25798633741215e-06
	1816 8.2557571658981e-06
	1817 8.25351889943704e-06
	1818 8.25120969238924e-06
	1819 8.24889684736263e-06
	1820 8.24657672637841e-06
	1821 8.24425933387829e-06
	1822 8.24189191916957e-06
	1823 8.23956725071184e-06
	1824 8.23720347398194e-06
	1825 8.23484151624143e-06
	1826 8.23247683001682e-06
	1827 8.23008576844586e-06
	1828 8.22770834929543e-06
	1829 8.22532001620857e-06
	1830 8.22292167868e-06
	1831 8.22049969428917e-06
	1832 8.21806770545663e-06
	1833 8.21563116915058e-06
	1834 8.21318462840281e-06
	1835 8.21070807432989e-06
	1836 8.20819786895299e-06
	1837 8.20568402559729e-06
	1838 8.20316017779987e-06
	1839 8.20059540274087e-06
	1840 8.19804790808121e-06
	1841 8.19544584373944e-06
	1842 8.19282740849303e-06
	1843 8.19018350739498e-06
	1844 8.18752232589759e-06
	1845 8.18485113995848e-06
	1846 8.18218450149288e-06
	1847 8.17955333332065e-06
	1848 8.17692398413783e-06
	1849 8.17433010524837e-06
	1850 8.17182171886088e-06
	1851 8.16938245407073e-06
	1852 8.1670077634044e-06
	1853 8.16474675957579e-06
	1854 8.16263127489947e-06
	1855 8.16065221442841e-06
	1856 8.1587986642262e-06
	1857 8.15713428892195e-06
	1858 8.15561179479118e-06
	1859 8.15427847555839e-06
	1860 8.15306611912092e-06
	1861 8.15200382930925e-06
	1862 8.15095881989691e-06
	1863 8.14990653452696e-06
	1864 8.1487442002981e-06
	1865 8.14733448351035e-06
	1866 8.14553368400084e-06
	1867 8.14316808828153e-06
	1868 8.14004397398094e-06
	1869 8.13592487247661e-06
	1870 8.13061888038646e-06
	1871 8.12385951576289e-06
	1872 8.11544305179268e-06
	1873 8.10518122307258e-06
	1874 8.09290941106156e-06
	1875 8.07851210993249e-06
	1876 8.06188836577348e-06
	1877 8.04298633738654e-06
	1878 8.02179965830874e-06
	1879 7.99836561782286e-06
	1880 7.97292159404606e-06
	1881 7.94603420217754e-06
	1882 7.91935781307984e-06
	1883 7.89720070315525e-06
	1884 7.89082969276933e-06
	1885 7.92996979726013e-06
	1886 8.09425273473607e-06
	1887 8.59852843859699e-06
	1888 9.99247549771098e-06
	1889 1.33297799038701e-05
	1890 1.84853615792235e-05
	1891 1.92789229913615e-05
	1892 1.34632737172069e-05
	1893 9.35870957619045e-06
	1894 1.03328820841853e-05
	1895 1.04240543805645e-05
	1896 8.78133778314805e-06
	1897 8.00234465714311e-06
	1898 7.80619211582234e-06
	1899 7.77302830101689e-06
	1900 7.76347314968007e-06
	1901 7.76228989707306e-06
	1902 7.76917022449197e-06
	1903 7.77586319600232e-06
	1904 7.77732566348277e-06
	1905 7.77424702391727e-06
	1906 7.76957585912896e-06
	1907 7.76558135839878e-06
	1908 7.76307479100069e-06
	1909 7.76206798036583e-06
	1910 7.76220076659229e-06
	1911 7.7630775194848e-06
	1912 7.76426804804942e-06
	1913 7.76550405134913e-06
	1914 7.76665183366276e-06
	1915 7.76769957155921e-06
	1916 7.76874730945565e-06
	1917 7.76987235440174e-06
	1918 7.77113109506899e-06
	1919 7.77257992012892e-06
	1920 7.77422610553913e-06
	1921 7.77603781898506e-06
	1922 7.77801960794022e-06
	1923 7.78015419200528e-06
	1924 7.78239791543456e-06
	1925 7.78471166995587e-06
	1926 7.78710000304272e-06
	1927 7.78953835833818e-06
	1928 7.79204492573626e-06
	1929 7.79460242483765e-06
	1930 7.79718448029598e-06
	1931 7.79978836362716e-06
	1932 7.80239042796893e-06
	1933 7.80502250563586e-06
	1934 7.80766640673392e-06
	1935 7.81025664764456e-06
	1936 7.81284961703932e-06
	1937 7.81543076300295e-06
	1938 7.81792186899111e-06
	1939 7.82037750468589e-06
	1940 7.82272945798468e-06
	1941 7.82501047069672e-06
	1942 7.8272059909068e-06
	1943 7.82932784204604e-06
	1944 7.83134964876808e-06
	1945 7.83324503572658e-06
	1946 7.83502764534205e-06
	1947 7.83667746873107e-06
	1948 7.83820087235654e-06
	1949 7.83956966188271e-06
	1950 7.84080839366652e-06
	1951 7.84189705882454e-06
	1952 7.84280837251572e-06
	1953 7.84358508099103e-06
	1954 7.84419626143062e-06
	1955 7.84462190495105e-06
	1956 7.84491021477152e-06
	1957 7.84501389716752e-06
	1958 7.8449529610225e-06
	1959 7.84470375947421e-06
	1960 7.8442872109008e-06
	1961 7.84374151407974e-06
	1962 7.84304847911699e-06
	1963 7.84224084782181e-06
	1964 7.84128405939555e-06
	1965 7.84023632149911e-06
	1966 7.83909854362719e-06
	1967 7.83788527769502e-06
	1968 7.83660107117612e-06
	1969 7.83526320446981e-06
	1970 7.83389532443834e-06
	1971 7.83256473368965e-06
	1972 7.83124505687738e-06
	1973 7.82993538450683e-06
	1974 7.82864844950382e-06
	1975 7.8274088082253e-06
	1976 7.82623737904942e-06
	1977 7.82514052843908e-06
	1978 7.82410916144727e-06
	1979 7.8231496445369e-06
	1980 7.82220649853116e-06
	1981 7.82128427090356e-06
	1982 7.82037477620179e-06
	1983 7.81942526373314e-06
	1984 7.81839662522543e-06
	1985 7.81725884735351e-06
	1986 7.81596099841408e-06
	1987 7.81438848207472e-06
	1988 7.81251765147317e-06
	1989 7.81021662987769e-06
	1990 7.80741811468033e-06
	1991 7.80404752731556e-06
	1992 7.79997662903043e-06
	1993 7.79517267801566e-06
	1994 7.78957473812625e-06
	1995 7.78310914029134e-06
	1996 7.77574678068049e-06
	1997 7.76745673647383e-06
	1998 7.7582599260495e-06
	1999 7.74808086134726e-06
};
\addlegendentry{Test}
\end{groupplot}
\end{tikzpicture}

		% This file was created by tikzplotlib v0.9.6.
\begin{tikzpicture}

\begin{groupplot}[
group style={group size=1 by 8},
legend cell align={left},
legend style={fill opacity=1, draw opacity=1, text opacity=1, draw=white},
log basis y={10},
tick align=outside,
tick pos=left,
title style={at={(0.43,0.85)},anchor=north},
x grid style={white!69.0196078431373!black},
xlabel={Epoch},
x label style={yshift=13pt},
xmin=-49.95, xmax=2048.95,
xtick style={color=black},
xtick = {0,500,1500,2000},
y grid style={white!69.0196078431373!black},
ylabel={MSE Loss},
ymode=log,
ytick style={color=black},
width=.45\textwidth,
height=.25\textwidth
]


\nextgroupplot[
title={SiLU/SiLU},
ymin=3.50471141586029e-06, ymax=0.001
]
\addplot [semithick, black, dashed]
table {%
0 0.0758203740697354
1 0.0746741194743663
2 0.073540803976357
3 0.0724140885286033
4 0.0712843500077724
5 0.070142098236829
6 0.0689770118333399
7 0.0677746469154954
8 0.0665099218022078
9 0.0651238069403917
10 0.0634385615121573
11 0.0609005892183632
12 0.0561403660103679
13 0.0487685482949018
14 0.0433073583990335
15 0.0390697466209531
16 0.0356445896904916
17 0.0327380419475958
18 0.0302093963837251
19 0.0279845524346456
20 0.0260237253387459
21 0.0242935576243326
22 0.0227625634288415
23 0.0214021718129516
24 0.0201880914974026
25 0.0190999399055727
26 0.0181201834930107
27 0.0172334585804492
28 0.0164263357874006
29 0.0156872553634457
30 0.0150064563495107
31 0.014375834580278
32 0.0137887305754703
33 0.0132397042179946
34 0.0127242775633931
35 0.0122387606243137
36 0.0117800756997894
37 0.011345628357958
38 0.0109332102583721
39 0.0105409192910884
40 0.0101671121519757
41 0.00981035229051486
42 0.00946937799744774
43 0.00914306759659667
44 0.00883043286739849
45 0.00853059299697634
46 0.0082427642919356
47 0.00796623680798803
48 0.00770037718757521
49 0.00744462011789437
50 0.00719845328421798
51 0.00696140767831821
52 0.00673305992677342
53 0.00651302001642762
54 0.00630092638311908
55 0.00609644359064987
56 0.00589925915119238
57 0.00570907760993578
58 0.00552562316443073
59 0.00534863686334575
60 0.00517787317221519
61 0.00501309897663305
62 0.00485409619432176
63 0.00470066180787398
64 0.00455259987575118
65 0.00440971996431472
66 0.00427183055217029
67 0.00413873467550729
68 0.00401022962978459
69 0.00388611790913274
70 0.00376620443057618
71 0.0036503030496533
72 0.00353823458499392
73 0.00342982863730867
74 0.0033249209409405
75 0.00322335255805228
76 0.00312497088089003
77 0.00302963219837693
78 0.00293720059744373
79 0.00284754818494548
80 0.00276055631547933
81 0.00267611576236959
82 0.00259412580271601
83 0.00251449531242542
84 0.00243714398311567
85 0.00236199850769481
86 0.0022889986021255
87 0.00221809125832806
88 0.00214923209205153
89 0.00208238587310916
90 0.00201752604243666
91 0.00195463161799125
92 0.00189368753581221
93 0.00183468422073929
94 0.0017776144131858
95 0.001722471974972
96 0.00166925160465325
97 0.0016179449557967
98 0.00156854050146649
99 0.00152102071024274
100 0.00147536194754139
101 0.00143153283670472
102 0.00138949559959656
103 0.00134920604159561
104 0.00131061180672987
105 0.00127365590014961
106 0.00123827571906077
107 0.00120440540467825
108 0.00117197666804714
109 0.00114092060448456
110 0.00111116681910062
111 0.00108264649043122
112 0.00105529272559579
113 0.00102904143841442
114 0.00100383107610469
115 0.000979603370069526
116 0.000956303745169862
117 0.000933881312903395
118 0.000912288189738319
119 0.000891479789061123
120 0.000871414841185469
121 0.000852054510687594
122 0.000833363286346867
123 0.000815307522771036
124 0.000797856321241852
125 0.000780980531089881
126 0.000764652797442977
127 0.000748847583508905
128 0.000733540972760238
129 0.000718710460660077
130 0.000704334718193422
131 0.000690393742388551
132 0.000676868713526346
133 0.000663741494463466
134 0.000650995076739491
135 0.000638613349110528
136 0.000626580790139997
137 0.000614882812897122
138 0.000603505423896422
139 0.00059243498185424
140 0.000581658839791999
141 0.000571164460325235
142 0.000560940174182178
143 0.000550974800489712
144 0.000541257673376094
145 0.00053177873542154
146 0.000522528469900863
147 0.000513498016061931
148 0.000504678755305576
149 0.00049606290963311
150 0.000487642936150223
151 0.000479411761261872
152 0.000471362651296658
153 0.000463488817786128
154 0.000455783553206857
155 0.000448240357400209
156 0.000440852388464918
157 0.000433612836559405
158 0.000426514983018933
159 0.000419552030280101
160 0.00041271738064097
161 0.000406004979595309
162 0.000399408895646047
163 0.000392923556546521
164 0.000386544080583917
165 0.000380266212459901
166 0.000374085577959704
167 0.000367998699175587
168 0.000362002884912727
169 0.000356095462393569
170 0.000350274344327772
171 0.000344537812225099
172 0.000338884696475361
173 0.000333313870441998
174 0.000327824531495935
175 0.000322416118933688
176 0.000317087660732795
177 0.00031183842338578
178 0.000306666999676963
179 0.000301571780710219
180 0.000296551080509744
181 0.000291603068035329
182 0.000286725933165144
183 0.000281918398286507
184 0.00027717979355657
185 0.000272510042123031
186 0.000267909334525029
187 0.00026337876784055
188 0.000258919472571506
189 0.000254532891972303
190 0.000250220841735427
191 0.000245985168135121
192 0.00024182776564885
193 0.000237750590258656
194 0.000233755806902991
195 0.000229845422040853
196 0.000226021632670381
197 0.000222286661369253
198 0.000218642409663516
199 0.000215090543406404
200 0.000211632963782904
201 0.000208271029350726
202 0.000205005709858597
203 0.000201837646386593
204 0.000198767231154307
205 0.000195794325577481
206 0.000192918204788839
207 0.000190138149491759
208 0.000187452603995553
209 0.000184859874366339
210 0.000182357767585017
211 0.000179944041747149
212 0.000177615851043811
213 0.000175370440842926
214 0.000173204776046987
215 0.000171115662283228
216 0.000169100007269662
217 0.0001671545114732
218 0.000165276001780512
219 0.000163461264037323
220 0.000161707364668473
221 0.000160011170351027
222 0.000158369936684721
223 0.000156780876068296
224 0.000155241343122725
225 0.000153748997604453
226 0.00015230130566124
227 0.000150896183299665
228 0.000149531430338357
229 0.000148205207551655
230 0.000146915569075645
231 0.000145660817793214
232 0.000144439281257291
233 0.000143249472557727
234 0.000142089871587814
235 0.00014095925030233
236 0.000139856162718388
237 0.000138779407507172
238 0.000137727924482078
239 0.000136700569100867
240 0.000135696283052766
241 0.000134714151840853
242 0.000133753218506172
243 0.000132812579977326
244 0.000131891459545841
245 0.000130989030196815
246 0.000130104536708586
247 0.000129237284568262
248 0.000128386521112134
249 0.000127551752086674
250 0.000126732188050482
251 0.000125927428769046
252 0.000125136750057209
253 0.000124359801930041
254 0.000123595894933715
255 0.000122844757413532
256 0.000122105717309751
257 0.00012137860102257
258 0.000120662799190541
259 0.000119958204265913
260 0.000119264105762795
261 0.000118580522098455
262 0.000117906777404642
263 0.000117243040961057
264 0.00011658850121421
265 0.000115943452840384
266 0.000115307104692874
267 0.000114679803118634
268 0.000114060715702635
269 0.000113450322828612
270 0.000112847695618257
271 0.000112253464351397
272 0.000111666595671522
273 0.000111087774030239
274 0.000110515899962138
275 0.000109951896661187
276 0.000109394289495413
277 0.000108844292924459
278 0.000108300171348219
279 0.000107763484720635
280 0.000107232085895248
281 0.000106707764359726
282 0.000106188215397651
283 0.000105675297220387
284 0.000105166618766361
285 0.000104664001298715
286 0.0001041653696916
287 0.000103671984106768
288 0.000103182345696951
289 0.000102697010106567
290 0.000102215489334867
291 0.000101737368765953
292 0.000101262961607063
293 0.000100791425609259
294 0.000100323268782176
295 9.98577264681444e-05
296 9.93951343275512e-05
297 9.89351902376256e-05
298 9.84776742143367e-05
299 9.80229052629511e-05
300 9.75703647441151e-05
301 9.71204132156345e-05
302 9.66727751006147e-05
303 9.62274105802408e-05
304 9.57846185940525e-05
305 9.53439088391406e-05
306 9.49057143202481e-05
307 9.4469712450973e-05
308 9.4036032578515e-05
309 9.3604843243611e-05
310 9.31758038973385e-05
311 9.27492814355446e-05
312 9.23250375137741e-05
313 9.1903133835558e-05
314 9.14836520848894e-05
315 9.10663617901264e-05
316 9.0651538982911e-05
317 9.02390184478463e-05
318 8.9828702698469e-05
319 8.94206852706247e-05
320 8.90149040913002e-05
321 8.86112771354419e-05
322 8.82098393901742e-05
323 8.78104751791398e-05
324 8.74130874990442e-05
325 8.70177487684032e-05
326 8.66242348251944e-05
327 8.62324256161173e-05
328 8.58423730392133e-05
329 8.54537880741191e-05
330 8.50666306178027e-05
331 8.46807199081923e-05
332 8.42958757800716e-05
333 8.39120519913195e-05
334 8.3528893071616e-05
335 8.31462847941111e-05
336 8.27640307932143e-05
337 8.2381919440877e-05
338 8.19997513588078e-05
339 8.16173508439988e-05
340 8.12344546545773e-05
341 8.08509230694199e-05
342 8.04665856435349e-05
343 8.00813810428735e-05
344 7.96951417783021e-05
345 7.93078674519165e-05
346 7.89193889545459e-05
347 7.85297096115301e-05
348 7.81390135955462e-05
349 7.77473229334191e-05
350 7.73547985772893e-05
351 7.69615924980371e-05
352 7.65678704226502e-05
353 7.61738354810859e-05
354 7.57798972301771e-05
355 7.53861875750772e-05
356 7.49930668746401e-05
357 7.46008209659976e-05
358 7.42097811894382e-05
359 7.38201652268344e-05
360 7.3432266248119e-05
361 7.30464441289769e-05
362 7.26628126983542e-05
363 7.22815627227646e-05
364 7.19030140032828e-05
365 7.15272558977631e-05
366 7.11544807074915e-05
367 7.07847160867914e-05
368 7.04181055368736e-05
369 7.00547577707766e-05
370 6.96945970730667e-05
371 6.93377236586912e-05
372 6.89841729411e-05
373 6.8633927639894e-05
374 6.82869836907685e-05
375 6.79432584149708e-05
376 6.76027841564064e-05
377 6.72655437767844e-05
378 6.69313549224171e-05
379 6.66002530920196e-05
380 6.62722486026723e-05
381 6.59471307926651e-05
382 6.56249601433956e-05
383 6.53055853945261e-05
384 6.49889691715089e-05
385 6.46750688133579e-05
386 6.43637345802972e-05
387 6.40550377539739e-05
388 6.37487743944121e-05
389 6.34449278749116e-05
390 6.31435002986791e-05
391 6.28443806220957e-05
392 6.25475592528346e-05
393 6.22529526168591e-05
394 6.19604562217546e-05
395 6.16700689306526e-05
396 6.13817370265224e-05
397 6.10954550097631e-05
398 6.08111128457267e-05
399 6.05287216330908e-05
400 6.02481857896464e-05
401 5.99694855907273e-05
402 5.96925769826839e-05
403 5.94174119612489e-05
404 5.91439996355803e-05
405 5.88722360532756e-05
406 5.86020998554204e-05
407 5.83335588828504e-05
408 5.80666651188722e-05
409 5.78013060135163e-05
410 5.75374809983487e-05
411 5.72751662843984e-05
412 5.70143155158576e-05
413 5.67549189582905e-05
414 5.64968525651466e-05
415 5.62401479697883e-05
416 5.59848038506061e-05
417 5.57307627957471e-05
418 5.54780081358786e-05
419 5.52265922948436e-05
420 5.4976400562623e-05
421 5.47274761828476e-05
422 5.44796955779248e-05
423 5.42330866011298e-05
424 5.39876648417703e-05
425 5.37433622156414e-05
426 5.35001591259743e-05
427 5.32581046712721e-05
428 5.30171224824016e-05
429 5.27771864682336e-05
430 5.25382760372395e-05
431 5.23004044481468e-05
432 5.20635589396079e-05
433 5.18276590781852e-05
434 5.15927632847024e-05
435 5.1358831413495e-05
436 5.11257895539075e-05
437 5.0893659903295e-05
438 5.06624821099422e-05
439 5.04321733814095e-05
440 5.02028139095501e-05
441 4.99742323825103e-05
442 4.97465481714698e-05
443 4.95197566152683e-05
444 4.92937733298504e-05
445 4.90685677334568e-05
446 4.88441819896934e-05
447 4.86205788234884e-05
448 4.83977663918722e-05
449 4.81757146104655e-05
450 4.7954426349861e-05
451 4.7733891889834e-05
452 4.75141619205033e-05
453 4.72951395380505e-05
454 4.70769068101617e-05
455 4.68593949278784e-05
456 4.66425840528473e-05
457 4.64266173452188e-05
458 4.62113297317046e-05
459 4.59967607753242e-05
460 4.5782976954456e-05
461 4.5569918981414e-05
462 4.53576356420626e-05
463 4.51460727788344e-05
464 4.49352941700454e-05
465 4.4725307986937e-05
466 4.45160057580551e-05
467 4.43075225717848e-05
468 4.40999063044956e-05
469 4.38930164818885e-05
470 4.36869885760416e-05
471 4.3481850553917e-05
472 4.32775295280408e-05
473 4.30741186647765e-05
474 4.28715806322089e-05
475 4.26699439941558e-05
476 4.24692283189643e-05
477 4.22694772197474e-05
478 4.20706360557688e-05
479 4.18727871362989e-05
480 4.16759395136523e-05
481 4.14801261143793e-05
482 4.12853239737387e-05
483 4.10915524327038e-05
484 4.08988005062838e-05
485 4.0707091841341e-05
486 4.05164603307639e-05
487 4.03268863635731e-05
488 4.01383673818145e-05
489 3.99509958839417e-05
490 3.97646930281326e-05
491 3.95794539258532e-05
492 3.93952991259994e-05
493 3.92122446157828e-05
494 3.90302023447475e-05
495 3.88492478151647e-05
496 3.86694098892804e-05
497 3.84905393104873e-05
498 3.83127384537829e-05
499 3.81359857470898e-05
500 3.79602045796901e-05
501 3.77854388489141e-05
502 3.7611778125779e-05
503 3.74389640001027e-05
504 3.72671581345685e-05
505 3.70963397386959e-05
506 3.69265031707755e-05
507 3.6757515545105e-05
508 3.65894744334128e-05
509 3.64224010809266e-05
510 3.62561289932728e-05
511 3.60907610144068e-05
512 3.59263361104922e-05
513 3.57627209979228e-05
514 3.55999117118699e-05
515 3.54380645433139e-05
516 3.52769851730272e-05
517 3.51167161340982e-05
518 3.49572799365205e-05
519 3.47986459416916e-05
520 3.46408643707719e-05
521 3.44838402526193e-05
522 3.4327578227078e-05
523 3.41721614205426e-05
524 3.40175158655143e-05
525 3.3863598233097e-05
526 3.37105128878079e-05
527 3.35581514079308e-05
528 3.34065697700225e-05
529 3.32557748379259e-05
530 3.310575113602e-05
531 3.29564156800188e-05
532 3.2807902321963e-05
533 3.26600712625691e-05
534 3.25130664933226e-05
535 3.23668283002121e-05
536 3.22212914909414e-05
537 3.20765730279504e-05
538 3.19325432869277e-05
539 3.17893023549232e-05
540 3.16467654997155e-05
541 3.1505057293657e-05
542 3.13640130400472e-05
543 3.12237714297225e-05
544 3.10842861637184e-05
545 3.09455177855966e-05
546 3.08075211776782e-05
547 3.06701955850031e-05
548 3.05338019614965e-05
549 3.03979958360401e-05
550 3.02629928796705e-05
551 3.01288308364178e-05
552 2.99952982913965e-05
553 2.98625833963229e-05
554 2.97305574150641e-05
555 2.95993196530731e-05
556 2.94688379725017e-05
557 2.93390636159074e-05
558 2.92100384555738e-05
559 2.90817524586373e-05
560 2.89542061153725e-05
561 2.88274228239516e-05
562 2.87013872721786e-05
563 2.85760181100159e-05
564 2.84514831179195e-05
565 2.8327612085377e-05
566 2.82045071102743e-05
567 2.80821600640024e-05
568 2.79604643935727e-05
569 2.78395852788549e-05
570 2.77193074680326e-05
571 2.75998179830594e-05
572 2.74810262013148e-05
573 2.73629588392055e-05
574 2.72456304131197e-05
575 2.7128953462352e-05
576 2.70130097916876e-05
577 2.68977573298912e-05
578 2.67832462910178e-05
579 2.66693754582548e-05
580 2.65562528412033e-05
581 2.64437105386151e-05
582 2.63318760573839e-05
583 2.62206973502543e-05
584 2.611025951893e-05
585 2.6000501065937e-05
586 2.58913005666273e-05
587 2.57828810532601e-05
588 2.56749611011742e-05
589 2.55678893665845e-05
590 2.54612437657897e-05
591 2.53553381881488e-05
592 2.52500124275912e-05
593 2.51453597854834e-05
594 2.5041351939592e-05
595 2.49378084689056e-05
596 2.48351006817416e-05
597 2.47326771685152e-05
598 2.46311941012323e-05
599 2.45299412071631e-05
600 2.44297447409281e-05
601 2.43296013380245e-05
602 2.42304572637408e-05
603 2.4131512603276e-05
604 2.4033584530514e-05
605 2.39356427940152e-05
606 2.3838891337391e-05
607 2.37421163973295e-05
608 2.36463786862373e-05
609 2.35506838563992e-05
610 2.34561505578768e-05
611 2.33614008209315e-05
612 2.32681059344486e-05
613 2.31743333074519e-05
614 2.30821563960149e-05
615 2.29892996372882e-05
616 2.28983845929065e-05
617 2.28063982987692e-05
618 2.27166590462957e-05
619 2.26254944664106e-05
620 2.25371204365388e-05
621 2.24466075309238e-05
622 2.23596113926305e-05
623 2.22697547513917e-05
624 2.21842484435797e-05
625 2.20946673081812e-05
626 2.20110329109957e-05
627 2.19215044907628e-05
628 2.18399080296194e-05
629 2.17501328947378e-05
630 2.16711202369879e-05
631 2.15804297667432e-05
632 2.15046656819595e-05
633 2.14123020043644e-05
634 2.13409384528518e-05
635 2.12454747554602e-05
636 2.11800516538574e-05
637 2.10799613356016e-05
638 2.10230481201279e-05
639 2.09154300776504e-05
640 2.08711250238025e-05
641 2.07522653212777e-05
642 2.07270515346636e-05
643 2.05913677078229e-05
644 2.05957446723914e-05
645 2.04358398718796e-05
646 2.04868668092217e-05
647 2.02925922181407e-05
648 2.04147224707185e-05
649 2.01697315276306e-05
650 2.03842137977972e-05
651 2.00527216733803e-05
652 2.03296293221911e-05
653 1.98587550030993e-05
654 2.00824113250064e-05
655 1.95566496685728e-05
656 1.9664929553187e-05
657 1.93535755954599e-05
658 1.93510510939632e-05
659 1.92611770799545e-05
660 1.91744703599284e-05
661 1.91405879661488e-05
662 1.90496719838507e-05
663 1.90069121899228e-05
664 1.89342326990527e-05
665 1.88822442837022e-05
666 1.88178439586295e-05
667 1.87638528998946e-05
668 1.87016068764478e-05
669 1.86477056871581e-05
670 1.85862634793921e-05
671 1.85325064592234e-05
672 1.84715445570305e-05
673 1.84181946849549e-05
674 1.8357144597303e-05
675 1.83044031984991e-05
676 1.82429654316252e-05
677 1.81912041767873e-05
678 1.81289283531783e-05
679 1.80786761418972e-05
680 1.80151389770344e-05
681 1.79670765163564e-05
682 1.79013455152699e-05
683 1.78567881690128e-05
684 1.7787508269862e-05
685 1.77482269165807e-05
686 1.76734226897679e-05
687 1.7642895436154e-05
688 1.75592478015574e-05
689 1.75431076456789e-05
690 1.74458400437061e-05
691 1.74545495994494e-05
692 1.73368075593316e-05
693 1.73910920864273e-05
694 1.7244267134231e-05
695 1.7384912315066e-05
696 1.71985859154233e-05
697 1.74938224297705e-05
698 1.72254094437108e-05
699 1.76898031476469e-05
700 1.71407775155785e-05
701 1.74844726252843e-05
702 1.66404041976875e-05
703 1.675544315205e-05
704 1.64709794034934e-05
705 1.64031438032453e-05
706 1.65042672151117e-05
707 1.62695743846086e-05
708 1.63175857572639e-05
709 1.62296335552981e-05
710 1.61796398501224e-05
711 1.61749487048723e-05
712 1.61002720346914e-05
713 1.60899520480484e-05
714 1.60352300895283e-05
715 1.60105164148661e-05
716 1.596712035834e-05
717 1.59380724262803e-05
718 1.58970860084651e-05
719 1.58675614798653e-05
720 1.5826816252229e-05
721 1.57974793175697e-05
722 1.57561383780092e-05
723 1.5727528619891e-05
724 1.56847425536455e-05
725 1.56576946963582e-05
726 1.56123747316883e-05
727 1.55881252013046e-05
728 1.55387889222425e-05
729 1.5519648304263e-05
730 1.5463868315635e-05
731 1.54540891834642e-05
732 1.53879998876505e-05
733 1.53960495730132e-05
734 1.53142216596791e-05
735 1.5358786875197e-05
736 1.52560042074867e-05
737 1.53820819512873e-05
738 1.52616685369367e-05
739 1.55778963133457e-05
740 1.544326963554e-05
741 1.60940657636388e-05
742 1.56623671259126e-05
743 1.61857858493875e-05
744 1.49076482074406e-05
745 1.49682141383778e-05
746 1.47139524528939e-05
747 1.4557823014627e-05
748 1.47697212398157e-05
749 1.4379873281456e-05
750 1.44071261409806e-05
751 1.45249274545733e-05
752 1.42698578642353e-05
753 1.43601545339322e-05
754 1.42788714576625e-05
755 1.42243098117945e-05
756 1.42691223032898e-05
757 1.41772614483671e-05
758 1.41887869844481e-05
759 1.41554037078606e-05
760 1.41297944402652e-05
761 1.41181627739684e-05
762 1.40877614462909e-05
763 1.40752323751769e-05
764 1.40468951244088e-05
765 1.40338459182487e-05
766 1.40048843384477e-05
767 1.39932266129961e-05
768 1.39619804357949e-05
769 1.39530134717347e-05
770 1.39182748029043e-05
771 1.39141144224197e-05
772 1.38740737334331e-05
773 1.38787395087547e-05
774 1.38311956625614e-05
775 1.38526618016499e-05
776 1.37959680621691e-05
777 1.38515979237752e-05
778 1.37887600146769e-05
779 1.39174831979005e-05
780 1.38685741148947e-05
781 1.41493186234243e-05
782 1.41569676230802e-05
783 1.46651515358087e-05
784 1.46478573554987e-05
785 1.51929931782036e-05
786 1.46442008812642e-05
787 1.481747746368e-05
788 1.35522887418915e-05
789 1.34986450390784e-05
790 1.31323475933698e-05
791 1.30635867847673e-05
792 1.3286831297421e-05
793 1.29140996882882e-05
794 1.29843091585258e-05
795 1.29768916607986e-05
796 1.28688100460295e-05
797 1.29410309241962e-05
798 1.28461408230862e-05
799 1.28583090983625e-05
800 1.28378630535053e-05
801 1.28191488997231e-05
802 1.28119397189153e-05
803 1.27949824353379e-05
804 1.27858166010242e-05
805 1.27711361130878e-05
806 1.27622065413391e-05
807 1.27473842006509e-05
808 1.27390701933905e-05
809 1.27234793012576e-05
810 1.27161089267247e-05
811 1.26988738244904e-05
812 1.2693169832545e-05
813 1.26730309517598e-05
814 1.26709155985338e-05
815 1.26462927028115e-05
816 1.26521454646422e-05
817 1.2621709799987e-05
818 1.26482583979737e-05
819 1.26173771093363e-05
820 1.27088651389329e-05
821 1.27217624132925e-05
822 1.3053260001783e-05
823 1.32726523212057e-05
824 1.43058147656916e-05
825 1.4192691381254e-05
826 1.45603000945016e-05
827 1.2418440523021e-05
828 1.25778417228162e-05
829 1.32330619848631e-05
830 1.20751436156752e-05
831 1.23129184856907e-05
832 1.27947481658452e-05
833 1.19053509202161e-05
834 1.21040912866022e-05
835 1.2488122344223e-05
836 1.17809251953815e-05
837 1.19417058073168e-05
838 1.2260704952638e-05
839 1.16909775584872e-05
840 1.18205838219865e-05
841 1.20939969363576e-05
842 1.16209234164444e-05
843 1.1732655352148e-05
844 1.19679907619741e-05
845 1.15643972300461e-05
846 1.16704226549302e-05
847 1.18624393046218e-05
848 1.15232745478977e-05
849 1.16292827136988e-05
850 1.17569146169672e-05
851 1.15026923666051e-05
852 1.16087556953914e-05
853 1.16432003380851e-05
854 1.15007295065084e-05
855 1.16020661344862e-05
856 1.15458356120257e-05
857 1.15073126174536e-05
858 1.15785176220129e-05
859 1.15033044885138e-05
860 1.15197011218982e-05
861 1.15287899866701e-05
862 1.15013290340471e-05
863 1.1516383125354e-05
864 1.15014276786951e-05
865 1.15023941695824e-05
866 1.14955415462248e-05
867 1.14937574977603e-05
868 1.14866513847289e-05
869 1.14845024903332e-05
870 1.14768298793422e-05
871 1.14743256816041e-05
872 1.14653193818981e-05
873 1.14632162251382e-05
874 1.14521824912117e-05
875 1.14515390734482e-05
876 1.14375752318097e-05
877 1.14416221066449e-05
878 1.14244796947105e-05
879 1.14437741878248e-05
880 1.14318615125342e-05
881 1.15093877148809e-05
882 1.15673757541401e-05
883 1.18969973605942e-05
884 1.23406071779186e-05
885 1.34556952247067e-05
886 1.4330005249974e-05
887 1.47752667913892e-05
888 1.29604785072956e-05
889 1.17951583149534e-05
890 1.11726334282025e-05
891 1.11442087380453e-05
892 1.13613487258135e-05
893 1.08047873048633e-05
894 1.08665914062556e-05
895 1.11452704913972e-05
896 1.06406625803857e-05
897 1.07262711139811e-05
898 1.09655749440662e-05
899 1.05544959261294e-05
900 1.06432900714992e-05
901 1.07896704193422e-05
902 1.05231163232133e-05
903 1.06111894808691e-05
904 1.06237669861287e-05
905 1.05184176923956e-05
906 1.06029735924551e-05
907 1.05315469731693e-05
908 1.05275074062661e-05
909 1.0564269203428e-05
910 1.05246997534891e-05
911 1.0545963920805e-05
912 1.05377863022227e-05
913 1.05429234693588e-05
914 1.05431246169019e-05
915 1.05485758830071e-05
916 1.05488244912522e-05
917 1.0554660562434e-05
918 1.05553492701915e-05
919 1.05612860785698e-05
920 1.05616411687492e-05
921 1.05681267470459e-05
922 1.05678821711308e-05
923 1.05754113590706e-05
924 1.05755751889092e-05
925 1.05866640929264e-05
926 1.05941631112216e-05
927 1.06219355480164e-05
928 1.06728514417398e-05
929 1.07930102970499e-05
930 1.10573221299148e-05
931 1.16550427726736e-05
932 1.2486103813103e-05
933 1.38073781883463e-05
934 1.21617203134861e-05
935 1.12546996788865e-05
936 1.13115863946689e-05
937 1.06512808315529e-05
938 1.07279568091201e-05
939 1.1050446616423e-05
940 1.02457032795655e-05
941 1.04343274109908e-05
942 1.07539500504572e-05
943 1.00610407187673e-05
944 1.02201689138326e-05
945 1.04663466693466e-05
946 9.99616336017084e-06
947 1.00616475897652e-05
948 1.02544966118501e-05
949 9.9698345508159e-06
950 9.95231251721407e-06
951 1.01164016541588e-05
952 9.94894200445628e-06
953 9.88306918259241e-06
954 1.00293879761182e-05
955 9.92362029705873e-06
956 9.84360955413877e-06
957 9.97574030492387e-06
958 9.89155855180002e-06
959 9.82561643780855e-06
960 9.943874349716e-06
961 9.85431172040308e-06
962 9.82248347369818e-06
963 9.9247198299679e-06
964 9.82230620749647e-06
965 9.82854669828725e-06
966 9.90502260123094e-06
967 9.81331672633701e-06
968 9.8425750820752e-06
969 9.87508331817821e-06
970 9.82987553399539e-06
971 9.86078154951997e-06
972 9.85556704335977e-06
973 9.85312070866939e-06
974 9.86600026919859e-06
975 9.86302914540715e-06
976 9.86933983071481e-06
977 9.87120363049598e-06
978 9.8754085939845e-06
979 9.87707224808787e-06
980 9.88065012563766e-06
981 9.88207487040427e-06
982 9.88458071304876e-06
983 9.88581783190057e-06
984 9.88702831961064e-06
985 9.8889019852777e-06
986 9.88845482297052e-06
987 9.89404438200836e-06
988 9.89318360922198e-06
989 9.9149656591635e-06
990 9.92925821563517e-06
991 1.00272977547888e-05
992 1.01796058977754e-05
993 1.06573764995233e-05
994 1.16278489201704e-05
995 1.29492414089327e-05
996 1.40304065610053e-05
997 1.16469707158728e-05
998 1.0459185368461e-05
999 1.0845458824349e-05
1000 9.68731047024107e-06
1001 9.96910684136765e-06
1002 1.03610963897438e-05
1003 9.55278405356808e-06
1004 9.73405040483044e-06
1005 9.9567018878588e-06
1006 9.53196239095178e-06
1007 9.50928252763106e-06
1008 9.67336008628195e-06
1009 9.52176689494877e-06
1010 9.34993230572445e-06
1011 9.50752333750415e-06
1012 9.4976713675976e-06
1013 9.25991076883292e-06
1014 9.41165358980811e-06
1015 9.46397944900923e-06
1016 9.21481951365877e-06
1017 9.35221893882954e-06
1018 9.42686822469341e-06
1019 9.19295799306497e-06
1020 9.31290352212955e-06
1021 9.38903779967859e-06
1022 9.18276703743004e-06
1023 9.28627340712751e-06
1024 9.3493743023032e-06
1025 9.18147908635092e-06
1026 9.26949457280557e-06
1027 9.3052897511825e-06
1028 9.19067837656939e-06
1029 9.26230153908136e-06
1030 9.25965790798955e-06
1031 9.20818662208944e-06
1032 9.26153261815443e-06
1033 9.22994258800713e-06
1034 9.22662715296951e-06
1035 9.25502964932434e-06
1036 9.23067720037807e-06
1037 9.24300679727708e-06
1038 9.24634202092989e-06
1039 9.24464264073777e-06
1040 9.25064179924107e-06
1041 9.25130219187054e-06
1042 9.25422704156631e-06
1043 9.25586050826155e-06
1044 9.25785526817435e-06
1045 9.25914656235705e-06
1046 9.26010686086443e-06
1047 9.26131582090761e-06
1048 9.26062843831232e-06
1049 9.26249096977472e-06
1050 9.25930323347757e-06
1051 9.26392285194311e-06
1052 9.25691511000082e-06
1053 9.27143211626458e-06
1054 9.26180477023308e-06
1055 9.31458487585246e-06
1056 9.33657328516801e-06
1057 9.57033275383878e-06
1058 9.91797265115224e-06
1059 1.09902195148948e-05
1060 1.26230468353583e-05
1061 1.32101467897883e-05
1062 1.17232933369849e-05
1063 9.66429998427998e-06
1064 9.67189162892623e-06
1065 9.90005579026842e-06
1066 9.23789851370316e-06
1067 9.33496209043483e-06
1068 9.51082894928845e-06
1069 9.26432727155202e-06
1070 9.09443437180357e-06
1071 9.28381225762109e-06
1072 9.28143320422237e-06
1073 8.94352473324034e-06
1074 9.16778496673487e-06
1075 9.2682896415397e-06
1076 8.8849577508654e-06
1077 9.1123537799831e-06
1078 9.24228677945393e-06
1079 8.89003026927071e-06
1080 9.08762896401072e-06
1081 9.22149027005048e-06
1082 8.92814315633927e-06
1083 9.08350213535414e-06
1084 9.21520664576292e-06
1085 8.98017086115033e-06
1086 9.0992055099548e-06
1087 9.22649372014916e-06
1088 9.03821779374425e-06
1089 9.13551946624125e-06
1090 9.25519280059461e-06
1091 9.10219469574258e-06
1092 9.19249006869904e-06
1093 9.30056684111946e-06
1094 9.17695072288893e-06
1095 9.26951325297409e-06
1096 9.36169238840989e-06
1097 9.26909057241687e-06
1098 9.36562044984157e-06
1099 9.43813690668094e-06
1100 9.38286406082511e-06
1101 9.47913123994226e-06
1102 9.53022161453987e-06
1103 9.51528191350803e-06
1104 9.6042869266455e-06
1105 9.63646250795591e-06
1106 9.65373738992525e-06
1107 9.72683830724463e-06
1108 9.74474064818764e-06
1109 9.77454974204761e-06
1110 9.82114988090643e-06
1111 9.82434495000462e-06
1112 9.84261888348215e-06
1113 9.85216596660621e-06
1114 9.83314703972837e-06
1115 9.82101599333873e-06
1116 9.78978496135596e-06
1117 9.74323141988975e-06
1118 9.69632646530272e-06
1119 9.63316907842682e-06
1120 9.56550074526774e-06
1121 9.49575354880494e-06
1122 9.41914147922773e-06
1123 9.34459435875112e-06
1124 9.27003078210475e-06
1125 9.19667038878913e-06
1126 9.12757790771934e-06
1127 9.0613237038184e-06
1128 8.99959245614923e-06
1129 8.94222408831524e-06
1130 8.88928181730364e-06
1131 8.84098972342429e-06
1132 8.79680546184147e-06
1133 8.75692121660165e-06
1134 8.72089483117122e-06
1135 8.68869426540186e-06
1136 8.65993869680892e-06
1137 8.63438449982823e-06
1138 8.61190027023895e-06
1139 8.59220639171099e-06
1140 8.57506947404829e-06
1141 8.56036082730327e-06
1142 8.54783706394358e-06
1143 8.53732120731365e-06
1144 8.52859303002163e-06
1145 8.52158819597548e-06
1146 8.51588700534478e-06
1147 8.51183691885637e-06
1148 8.50834140031509e-06
1149 8.50681093389483e-06
1150 8.50461587731388e-06
1151 8.50580542888224e-06
1152 8.50381258565847e-06
1153 8.51003873592049e-06
1154 8.50832125109946e-06
1155 8.53213815865672e-06
1156 8.54528054716752e-06
1157 8.65810050854066e-06
1158 8.81487803283676e-06
1159 9.39263628652043e-06
1160 1.0185313350064e-05
1161 1.14540975530986e-05
1162 1.13963160401909e-05
1163 1.06296661570582e-05
1164 9.90357983710055e-06
1165 8.25094093315215e-06
1166 8.48936736019823e-06
1167 9.02172638461707e-06
1168 8.05954291749345e-06
1169 8.22935251498791e-06
1170 8.67274873073143e-06
1171 7.96999825602995e-06
1172 8.08581028266531e-06
1173 8.42076278750881e-06
1174 7.91870744265566e-06
1175 7.98319831929462e-06
1176 8.23521430959318e-06
1177 7.8818674893455e-06
1178 7.90874733169744e-06
1179 8.10577156507009e-06
1180 7.85018100835089e-06
1181 7.86139258757146e-06
1182 8.02234769636812e-06
1183 7.82379243169373e-06
1184 7.84125680830527e-06
1185 7.97261841434249e-06
1186 7.81231472046784e-06
1187 7.84684004884184e-06
1188 7.94216443900098e-06
1189 7.8291788927487e-06
1190 7.87605270069491e-06
1191 7.9227188276576e-06
1192 7.8743558677985e-06
1193 7.9203876595102e-06
1194 7.9271073065712e-06
1195 7.93050568326237e-06
1196 7.96006133896299e-06
1197 7.96482857623459e-06
1198 7.98404063573344e-06
1199 7.99857331124088e-06
1200 8.01338156897202e-06
1201 8.02845789316109e-06
1202 8.04283217803459e-06
1203 8.05685156102243e-06
1204 8.06999821989507e-06
1205 8.08298823429254e-06
1206 8.09434504311923e-06
1207 8.10602544554229e-06
1208 8.11500889419392e-06
1209 8.12549435380561e-06
1210 8.13108543518126e-06
1211 8.1412798476066e-06
1212 8.14198809706568e-06
1213 8.15510960805454e-06
1214 8.14858925579642e-06
1215 8.17533505248491e-06
1216 8.16199216124858e-06
1217 8.24310904690151e-06
1218 8.25860287712032e-06
1219 8.56369895529951e-06
1220 8.86836307500971e-06
1221 9.86710733386076e-06
1222 1.09071940315175e-05
1223 1.04746530666944e-05
1224 9.30898418260995e-06
1225 8.27753562759881e-06
1226 8.19676097840727e-06
1227 8.35321101355646e-06
1228 8.12197995259112e-06
1229 7.82098219964666e-06
1230 7.94908537748995e-06
1231 8.06572911926651e-06
1232 7.62195187320458e-06
1233 7.7394580593193e-06
1234 7.96442325068369e-06
1235 7.52591346397935e-06
1236 7.62807766996332e-06
1237 7.83442766838505e-06
1238 7.48597150490582e-06
1239 7.55521925555058e-06
1240 7.71537116328602e-06
1241 7.46815116237087e-06
1242 7.5027659978133e-06
1243 7.62679947818867e-06
1244 7.45503754018273e-06
1245 7.47054717464835e-06
1246 7.56940225699054e-06
1247 7.44456492896006e-06
1248 7.46028107734276e-06
1249 7.53719110768714e-06
1250 7.44523200957303e-06
1251 7.47086687979959e-06
1252 7.52225172284682e-06
1253 7.4672528747044e-06
1254 7.49925467680157e-06
1255 7.52223970756916e-06
1256 7.50879154409745e-06
1257 7.53711940681967e-06
1258 7.54509079392562e-06
1259 7.55713462297081e-06
1260 7.57527495487409e-06
1261 7.58752947405128e-06
1262 7.60357301921033e-06
1263 7.61806563431833e-06
1264 7.6331337393043e-06
1265 7.64769226968554e-06
1266 7.66176862398993e-06
1267 7.67622512221067e-06
1268 7.68848690491097e-06
1269 7.70291611296159e-06
1270 7.71222522111259e-06
1271 7.72768484402775e-06
1272 7.73205119841691e-06
1273 7.75230186889075e-06
1274 7.74873237574525e-06
1275 7.78661819289539e-06
1276 7.77885204250595e-06
1277 7.89068083406619e-06
1278 7.95663146213599e-06
1279 8.42961152613952e-06
1280 9.04561497350187e-06
1281 1.04864696837126e-05
1282 1.0964496134136e-05
1283 1.02672869566334e-05
1284 8.53319152049892e-06
1285 7.56980104910099e-06
1286 7.75992695345451e-06
1287 7.89277912893738e-06
1288 7.34661040091567e-06
1289 7.47157514702224e-06
1290 7.62137255883033e-06
1291 7.25029859971471e-06
1292 7.33438114153273e-06
1293 7.43804380576307e-06
1294 7.19794382142425e-06
1295 7.25272843382641e-06
1296 7.30397577264341e-06
1297 7.17515441017724e-06
1298 7.20825751088228e-06
1299 7.21015174498518e-06
1300 7.16832148661695e-06
1301 7.18670970201174e-06
1302 7.17044069276085e-06
1303 7.16764157715488e-06
1304 7.17323056242947e-06
1305 7.17095815438995e-06
1306 7.17517017889691e-06
1307 7.18038492664164e-06
1308 7.18600215776632e-06
1309 7.1949329072396e-06
1310 7.20284690558515e-06
1311 7.21575105977479e-06
1312 7.22460446667128e-06
1313 7.24252898542943e-06
1314 7.24993586764811e-06
1315 7.2764249026136e-06
1316 7.27772742337152e-06
1317 7.32448924800622e-06
1318 7.31264265141363e-06
1319 7.42207881465617e-06
1320 7.40729310066968e-06
1321 7.75304872568938e-06
1322 7.92991392017939e-06
1323 9.09724327158301e-06
1324 1.00523791246587e-05
1325 1.01251516113621e-05
1326 8.78043316632215e-06
1327 7.87201528140713e-06
1328 7.87248123401696e-06
1329 7.91135738964499e-06
1330 7.90505629666427e-06
1331 7.39919299874714e-06
1332 7.63784677815238e-06
1333 7.84237016304701e-06
1334 7.252400051172e-06
1335 7.4752467646988e-06
1336 7.61975304008899e-06
1337 7.31524333730249e-06
1338 7.29406404786914e-06
1339 7.44367031302318e-06
1340 7.39110870995319e-06
1341 7.17628786262026e-06
1342 7.35587532929571e-06
1343 7.40601689308562e-06
1344 7.1580243634628e-06
1345 7.29857502967945e-06
1346 7.37929232741408e-06
1347 7.19649399272271e-06
1348 7.25004190371692e-06
1349 7.34707365523946e-06
1350 7.24465898116478e-06
1351 7.21947488813157e-06
1352 7.32419828253228e-06
1353 7.279486984757e-06
1354 7.21267977965567e-06
1355 7.3112425109656e-06
1356 7.2983059027365e-06
1357 7.22454492674274e-06
1358 7.30646036117832e-06
1359 7.30693795247817e-06
1360 7.2466017098094e-06
1361 7.30869902199061e-06
1362 7.31142990062494e-06
1363 7.27239807574165e-06
1364 7.31616819571457e-06
1365 7.31588188429555e-06
1366 7.29769937102276e-06
1367 7.32581112394826e-06
1368 7.3223416361401e-06
1369 7.31917242369207e-06
1370 7.33435475908095e-06
1371 7.33102207739478e-06
1372 7.33418734100155e-06
1373 7.3399733704349e-06
1374 7.33892231963296e-06
1375 7.34166570737216e-06
1376 7.34262061108382e-06
1377 7.34240168043243e-06
1378 7.34245352340679e-06
1379 7.34142216174405e-06
1380 7.33987303291883e-06
1381 7.33770495031649e-06
1382 7.33498657368159e-06
1383 7.33160971932989e-06
1384 7.32770917011294e-06
1385 7.32325301555647e-06
1386 7.31818234633863e-06
1387 7.31276334775544e-06
1388 7.3065336092526e-06
1389 7.30064614806736e-06
1390 7.29321935821758e-06
1391 7.28836995911308e-06
1392 7.2802710651132e-06
1393 7.28364300783824e-06
1394 7.28183947451555e-06
1395 7.32959489546658e-06
1396 7.3936426581156e-06
1397 7.67171980520232e-06
1398 8.12615279244255e-06
1399 8.98792882608745e-06
1400 9.78492112935214e-06
1401 9.12262101593342e-06
1402 8.80246711076893e-06
1403 7.13740313784683e-06
1404 7.22357534499452e-06
1405 7.26920417370991e-06
1406 6.96676328537649e-06
1407 7.04353036518057e-06
1408 7.04420339125988e-06
1409 6.90934674096866e-06
1410 6.95398044925355e-06
1411 6.91735617586176e-06
1412 6.87521691311588e-06
1413 6.8966622954747e-06
1414 6.8586370574053e-06
1415 6.85019254298425e-06
1416 6.85208471828957e-06
1417 6.83434977588604e-06
1418 6.83142940793857e-06
1419 6.82627880621567e-06
1420 6.8206527750192e-06
1421 6.81883563125041e-06
1422 6.81642066169275e-06
1423 6.81646377387324e-06
1424 6.81707629901496e-06
1425 6.8197297942163e-06
1426 6.82323804035434e-06
1427 6.8286895462677e-06
1428 6.83518529598359e-06
1429 6.843269870771e-06
1430 6.85266157951503e-06
1431 6.86296574947676e-06
1432 6.87525238163289e-06
1433 6.8868361289276e-06
1434 6.90229068567305e-06
1435 6.91330246382904e-06
1436 6.93346926716742e-06
1437 6.93991299982599e-06
1438 6.97061884480377e-06
1439 6.96412716116868e-06
1440 7.0255084452242e-06
1441 6.99380795232685e-06
1442 7.16796857780366e-06
1443 7.1547828124352e-06
1444 7.83758952849212e-06
1445 8.39438195399111e-06
1446 1.04811446064446e-05
1447 1.11528930517579e-05
1448 8.81530802843145e-06
1449 7.7964759057636e-06
1450 8.30625086933878e-06
1451 7.14441724092296e-06
1452 7.5010207432058e-06
1453 7.6763544072378e-06
1454 7.269187328518e-06
1455 7.01368890432263e-06
1456 7.22761199156707e-06
1457 7.26731131805991e-06
1458 6.86029243546216e-06
1459 7.01333227048906e-06
1460 7.10220326638478e-06
1461 6.93632137682698e-06
1462 6.84635818437584e-06
1463 7.00645071738393e-06
1464 6.98328278225802e-06
1465 6.82109309479983e-06
1466 6.92159668069792e-06
1467 6.98544078581165e-06
1468 6.87911024144228e-06
1469 6.85287057677897e-06
1470 6.96289113832904e-06
1471 6.93580127908433e-06
1472 6.84476732892847e-06
1473 6.92153879100488e-06
1474 6.96088182650101e-06
1475 6.88169840401542e-06
1476 6.88482753830044e-06
1477 6.95794348359868e-06
1478 6.92381393818664e-06
1479 6.87454549641586e-06
1480 6.93808081209113e-06
1481 6.94801670597656e-06
1482 6.88771354084849e-06
1483 6.91398779295582e-06
1484 6.9508244777694e-06
1485 6.90656987956118e-06
1486 6.89529882436091e-06
1487 6.93735232282222e-06
1488 6.91596664736949e-06
1489 6.88273041937748e-06
1490 6.91363044325044e-06
1491 6.90978896678018e-06
1492 6.87076338934389e-06
1493 6.88412821148177e-06
1494 6.88874171927978e-06
1495 6.85311742820716e-06
1496 6.85106169484584e-06
1497 6.85626589813637e-06
1498 6.82679848829082e-06
1499 6.81517486889049e-06
1500 6.81665921575814e-06
1501 6.79249800761283e-06
1502 6.77691407169334e-06
1503 6.77361571632673e-06
1504 6.75334310606956e-06
1505 6.73760750657948e-06
1506 6.73043530419193e-06
1507 6.7129753471562e-06
1508 6.69899311134259e-06
1509 6.68960502103744e-06
1510 6.67484650129779e-06
1511 6.66306325847188e-06
1512 6.65304807156986e-06
1513 6.64111266601708e-06
1514 6.6313653181993e-06
1515 6.62204502965835e-06
1516 6.61280240876749e-06
1517 6.60489870085712e-06
1518 6.59723470697315e-06
1519 6.59031029037749e-06
1520 6.58411814313808e-06
1521 6.57847116514176e-06
1522 6.57337348464182e-06
1523 6.56899895723484e-06
1524 6.56499910256514e-06
1525 6.56172219315465e-06
1526 6.55869456878122e-06
1527 6.55635551005673e-06
1528 6.55420913631133e-06
1529 6.55267520244252e-06
1530 6.55143407612968e-06
1531 6.55037626273725e-06
1532 6.55054008191769e-06
1533 6.54930142829357e-06
1534 6.55400526738958e-06
1535 6.55214476097399e-06
1536 6.57841427376127e-06
1537 6.59017731319267e-06
1538 6.74011138990238e-06
1539 6.93178992783317e-06
1540 7.71906926821941e-06
1541 8.71966185655992e-06
1542 9.57011660318585e-06
1543 9.76893783644073e-06
1544 7.04333528389611e-06
1545 6.8625526186139e-06
1546 6.97899805146562e-06
1547 6.38775379613321e-06
1548 6.65870797078583e-06
1549 6.72576019056237e-06
1550 6.27264966190921e-06
1551 6.54224549734295e-06
1552 6.49607884462e-06
1553 6.27111553086479e-06
1554 6.38151271736831e-06
1555 6.36393734509966e-06
1556 6.28054463369665e-06
1557 6.24308269436824e-06
1558 6.30053719419266e-06
1559 6.26213237531204e-06
1560 6.16692082999748e-06
1561 6.25298108936079e-06
1562 6.2263504290172e-06
1563 6.14072540550126e-06
1564 6.21069908746108e-06
1565 6.19490528919187e-06
1566 6.14270332555122e-06
1567 6.18617218783868e-06
1568 6.18057186407839e-06
1569 6.16198979841442e-06
1570 6.1869510030732e-06
1571 6.18964833876134e-06
1572 6.19478590557776e-06
1573 6.21192761585121e-06
1574 6.22382676773725e-06
1575 6.23950161582343e-06
1576 6.25786486097013e-06
1577 6.2770971478443e-06
1578 6.2973363785801e-06
1579 6.32006711853705e-06
1580 6.34217143691274e-06
1581 6.36684005961285e-06
1582 6.3900225750757e-06
1583 6.41580779436879e-06
1584 6.43873454109212e-06
1585 6.46472921239649e-06
1586 6.48608211406554e-06
1587 6.51138240570504e-06
1588 6.52946945400856e-06
1589 6.55310543073995e-06
1590 6.56630193951457e-06
1591 6.58754735027856e-06
1592 6.59501139921304e-06
1593 6.61351032604784e-06
1594 6.61969835924481e-06
1595 6.63819417034972e-06
1596 6.66848981545343e-06
1597 6.70687925641289e-06
1598 6.86416094719533e-06
1599 7.01094362121069e-06
1600 7.53679797327322e-06
1601 7.91934185784271e-06
1602 8.70561384047619e-06
1603 9.07541570072112e-06
1604 7.33755386050916e-06
1605 6.93459179501588e-06
1606 7.00183364443774e-06
1607 6.38170416245032e-06
1608 6.81232520172159e-06
1609 6.77329928855386e-06
1610 6.30363651943355e-06
1611 6.6406669319008e-06
1612 6.59540368985745e-06
1613 6.43142926826101e-06
1614 6.33277402428689e-06
1615 6.61364133058839e-06
1616 6.50116760958497e-06
1617 6.25098349615882e-06
1618 6.53585831145165e-06
1619 6.54767291585756e-06
1620 6.35323785047603e-06
1621 6.37127697267204e-06
1622 6.60129813390142e-06
1623 6.48847119499862e-06
1624 6.33317860909699e-06
1625 6.57939173187572e-06
1626 6.60350008097055e-06
1627 6.43506848341246e-06
1628 6.514581286865e-06
1629 6.68752431209896e-06
1630 6.58601195091535e-06
1631 6.51400656970225e-06
1632 6.71939844032465e-06
1633 6.72412536140143e-06
1634 6.60194475798903e-06
1635 6.71557201670225e-06
1636 6.82581461930454e-06
1637 6.73221555658188e-06
1638 6.72697745507378e-06
1639 6.87958653600163e-06
1640 6.85518318910283e-06
1641 6.77958898265274e-06
1642 6.88930379943997e-06
1643 6.94026306113926e-06
1644 6.8555523924374e-06
1645 6.8779956086118e-06
1646 6.96824180312916e-06
1647 6.91576292055629e-06
1648 6.86243484171456e-06
1649 6.92997365003833e-06
1650 6.92208968011698e-06
1651 6.83382800215071e-06
1652 6.83507141285844e-06
1653 6.85275041423239e-06
1654 6.77067986032398e-06
1655 6.70935623947599e-06
1656 6.71809634322784e-06
1657 6.6664009592543e-06
1658 6.57926124336683e-06
1659 6.55847893860084e-06
1660 6.5379788818376e-06
1661 6.4627722462518e-06
1662 6.41719089777837e-06
1663 6.41185749294948e-06
1664 6.36785775753879e-06
1665 6.3159140637481e-06
1666 6.30939698531563e-06
1667 6.29574993382676e-06
1668 6.254389782967e-06
1669 6.24071577526308e-06
1670 6.24506705371175e-06
1671 6.22284340678902e-06
1672 6.20509707616179e-06
1673 6.21447443904799e-06
1674 6.21084839380615e-06
1675 6.19538383794804e-06
1676 6.20205277002128e-06
1677 6.21072571149739e-06
1678 6.20217046609639e-06
1679 6.20437886489356e-06
1680 6.21783595100567e-06
1681 6.21708135639665e-06
1682 6.21623045660158e-06
1683 6.22866201993588e-06
1684 6.23354428697098e-06
1685 6.23169685098901e-06
1686 6.24011889982512e-06
1687 6.2471848707446e-06
1688 6.24532292192725e-06
1689 6.24894669876142e-06
1690 6.25509273710634e-06
1691 6.2531172524416e-06
1692 6.25253803576697e-06
1693 6.25559965072853e-06
1694 6.25306472556986e-06
1695 6.24914854974179e-06
1696 6.24882465771748e-06
1697 6.24505054602764e-06
1698 6.23892901785439e-06
1699 6.23549657063904e-06
1700 6.23049121450947e-06
1701 6.22319530307891e-06
1702 6.21774415954235e-06
1703 6.21175289516884e-06
1704 6.20401025486217e-06
1705 6.19763810583152e-06
1706 6.19119878209062e-06
1707 6.18391184392664e-06
1708 6.17712712269736e-06
1709 6.17178420636577e-06
1710 6.16411692533347e-06
1711 6.16102511674654e-06
1712 6.15421364358326e-06
1713 6.15872883358293e-06
1714 6.15618651345073e-06
1715 6.19331210582885e-06
1716 6.22911639425183e-06
1717 6.40343197222393e-06
1718 6.64650844228731e-06
1719 7.11489657589937e-06
1720 7.53954699028725e-06
1721 7.09618812777535e-06
1722 6.96440756797756e-06
1723 6.15298764206074e-06
1724 6.17015231618723e-06
1725 6.06580469408868e-06
1726 6.02636120028421e-06
1727 6.03724369518943e-06
1728 6.03879585447942e-06
1729 6.00867540079264e-06
1730 6.01954334378263e-06
1731 6.02306436192634e-06
1732 6.00822859730954e-06
1733 6.0149516745156e-06
1734 6.02108616831742e-06
1735 6.01479785089509e-06
1736 6.02244224445769e-06
1737 6.02881493172447e-06
1738 6.02971499041871e-06
1739 6.0384271964864e-06
1740 6.04595907738315e-06
1741 6.05178049895017e-06
1742 6.06163229477374e-06
1743 6.07059949153665e-06
1744 6.07963609944306e-06
1745 6.0902669272167e-06
1746 6.10034405035265e-06
1747 6.11080653278862e-06
1748 6.1214382558461e-06
1749 6.13170268337626e-06
1750 6.14158685685595e-06
1751 6.15118302604856e-06
1752 6.15989844732212e-06
1753 6.16785616003312e-06
1754 6.17462410978931e-06
1755 6.18033923771577e-06
1756 6.18458750878403e-06
1757 6.18740836522136e-06
1758 6.18863936630021e-06
1759 6.18820914422002e-06
1760 6.18615069836181e-06
1761 6.18249547468253e-06
1762 6.17700826222034e-06
1763 6.17029025740123e-06
1764 6.16175940937325e-06
1765 6.15250343471985e-06
1766 6.14118489838944e-06
1767 6.13027990770121e-06
1768 6.11659295035594e-06
1769 6.10519787080932e-06
1770 6.08923872391642e-06
1771 6.07934468099813e-06
1772 6.06028003424797e-06
1773 6.05577664636314e-06
1774 6.0310683700493e-06
1775 6.04153545946673e-06
1776 6.00675526918337e-06
1777 6.06231768962573e-06
1778 6.02111743397415e-06
1779 6.23084993556233e-06
1780 6.26441460482852e-06
1781 6.92349278175186e-06
1782 7.16401302014447e-06
1783 7.75136859942904e-06
1784 7.27276931122844e-06
1785 6.32536238853731e-06
1786 6.16425566590806e-06
1787 5.85222287519116e-06
1788 5.9288053062545e-06
1789 5.93802257320419e-06
1790 5.83605052106861e-06
1791 5.86513893097163e-06
1792 5.88498632581036e-06
1793 5.82850112529343e-06
1794 5.85459534008237e-06
1795 5.85910712658233e-06
1796 5.83804175491309e-06
1797 5.8534122917564e-06
1798 5.85449612877653e-06
1799 5.85134329611492e-06
1800 5.86114985701158e-06
1801 5.86437699645614e-06
1802 5.8702895131546e-06
1803 5.87898883530613e-06
1804 5.88728280348505e-06
1805 5.89772733050609e-06
1806 5.90946882894627e-06
1807 5.92256514408973e-06
1808 5.9370948672921e-06
1809 5.95316152995906e-06
1810 5.97045691108633e-06
1811 5.98907485382938e-06
1812 6.00885623036618e-06
1813 6.02945476124717e-06
1814 6.05083432514419e-06
1815 6.07246992601063e-06
1816 6.09426214026598e-06
1817 6.11560208518824e-06
1818 6.13625288004016e-06
1819 6.15560763339573e-06
1820 6.17327619689689e-06
1821 6.18899368731718e-06
1822 6.20175087018282e-06
1823 6.21218408447533e-06
1824 6.21820958812691e-06
1825 6.22239578262906e-06
1826 6.22028159469323e-06
1827 6.21778378295801e-06
1828 6.20699491271637e-06
1829 6.19885478148063e-06
1830 6.17891496013101e-06
1831 6.16782099616131e-06
1832 6.13855183928536e-06
1833 6.12939387512057e-06
1834 6.08978428928708e-06
1835 6.09131021178655e-06
1836 6.038734333913e-06
1837 6.06863436303229e-06
1838 5.9994773398131e-06
1839 6.09762748826626e-06
1840 6.01560198454365e-06
1841 6.26676306492158e-06
1842 6.19797171808045e-06
1843 6.65988340387003e-06
1844 6.54225197571634e-06
1845 6.75073039246854e-06
1846 6.36388847574665e-06
1847 5.99557422376051e-06
1848 5.98362474057978e-06
1849 5.7960416279812e-06
1850 5.84493990984924e-06
1851 5.79638247089065e-06
1852 5.78684598373513e-06
1853 5.75210598707798e-06
1854 5.76224544879267e-06
1855 5.72565261069968e-06
1856 5.73760746647167e-06
1857 5.71944398153335e-06
1858 5.72280750610332e-06
1859 5.71457765108363e-06
1860 5.71561354334449e-06
1861 5.71413836603085e-06
1862 5.71356252621058e-06
1863 5.72117073005529e-06
1864 5.71817107619665e-06
1865 5.73805730930133e-06
1866 5.73111531032566e-06
1867 5.7694507011874e-06
1868 5.75568947525085e-06
1869 5.82363682788412e-06
1870 5.79883303863937e-06
1871 5.91511677061618e-06
1872 5.87411640395885e-06
1873 6.0627023144022e-06
1874 5.99708413062672e-06
1875 6.25571250179036e-06
1876 6.14052607872395e-06
1877 6.37461572416953e-06
1878 6.18587310619034e-06
1879 6.26744422405068e-06
1880 6.09721810107544e-06
1881 6.08004214619484e-06
1882 6.02246250736016e-06
1883 6.00386111759832e-06
1884 5.99182878602278e-06
1885 5.98189852940578e-06
1886 5.97234343224784e-06
1887 5.96213545556168e-06
1888 5.95080431686767e-06
1889 5.93862949838098e-06
1890 5.92469157112419e-06
1891 5.91207758393608e-06
1892 5.89522381844176e-06
1893 5.88361623687916e-06
1894 5.86455691120591e-06
1895 5.8552588253491e-06
1896 5.83446410118427e-06
1897 5.82955938277507e-06
1898 5.80717352782756e-06
1899 5.80897893698307e-06
1900 5.78488382174669e-06
1901 5.7963261577143e-06
1902 5.76982033884121e-06
1903 5.79522287846856e-06
1904 5.76490834802712e-06
1905 5.80912795733468e-06
1906 5.77281216429526e-06
1907 5.83953756638778e-06
1908 5.79288446012072e-06
1909 5.87909346183579e-06
1910 5.81447350178621e-06
1911 5.90585123294574e-06
1912 5.81540823674231e-06
1913 5.8930369197796e-06
1914 5.78116838667597e-06
1915 5.83981915447396e-06
1916 5.72698497158797e-06
1917 5.77651943522994e-06
1918 5.68042275439495e-06
1919 5.73030121575613e-06
1920 5.65262847551651e-06
1921 5.70536919841658e-06
1922 5.64016011850299e-06
1923 5.6952225131468e-06
1924 5.63750836590771e-06
1925 5.69475390310714e-06
1926 5.64165100591651e-06
1927 5.70118565690336e-06
1928 5.65222489790074e-06
1929 5.71341027821859e-06
1930 5.67164833942257e-06
1931 5.73314980734096e-06
1932 5.70981217773436e-06
1933 5.7709151377594e-06
1934 5.79603415751251e-06
1935 5.86288040516081e-06
1936 5.99443444482262e-06
1937 6.07428193255544e-06
1938 6.31256180128048e-06
1939 6.30824790093243e-06
1940 6.40538433316351e-06
1941 6.17148092985076e-06
1942 6.09674958695905e-06
1943 5.86182374640032e-06
1944 5.82166181573029e-06
1945 5.75028967908509e-06
1946 5.74290098320773e-06
1947 5.72335152959624e-06
1948 5.72068817916715e-06
1949 5.70783884334958e-06
1950 5.70457478410447e-06
1951 5.69735848010566e-06
1952 5.69380975967704e-06
1953 5.69008382811376e-06
1954 5.68816502344305e-06
1955 5.68670495049162e-06
1956 5.68668606959477e-06
1957 5.68732221317703e-06
1958 5.68915111109902e-06
1959 5.69181356979698e-06
1960 5.69540397066248e-06
1961 5.69995189714234e-06
1962 5.70543683409142e-06
1963 5.7117677592089e-06
1964 5.71912122637741e-06
1965 5.72729772407854e-06
1966 5.7364854786357e-06
1967 5.74659482488471e-06
1968 5.75755699294689e-06
1969 5.76946666619449e-06
1970 5.78242350890434e-06
1971 5.79606618700268e-06
1972 5.81097009266784e-06
1973 5.82635025914158e-06
1974 5.84307604079015e-06
1975 5.86040796690668e-06
1976 5.87878319358026e-06
1977 5.89799011052605e-06
1978 5.91829121709964e-06
1979 5.93935888737462e-06
1980 5.96169541466907e-06
1981 5.98508201399284e-06
1982 6.0098885104054e-06
1983 6.03620018857498e-06
1984 6.06467891728357e-06
1985 6.09538938256549e-06
1986 6.12959344614694e-06
1987 6.16737572300963e-06
1988 6.21114493881691e-06
1989 6.26149048166269e-06
1990 6.32181104087692e-06
1991 6.39509630673274e-06
1992 6.48724657814626e-06
1993 6.60565713417327e-06
1994 6.76539794142172e-06
1995 6.99040331042511e-06
1996 7.32934220337711e-06
1997 7.88667268381005e-06
1998 8.90751482884866e-06
1999 1.0978647729587e-05
};
\addlegendentry{Train}
\addplot [semithick, black]
table {%
0 0.0747663080692291
1 0.0736388191580772
2 0.072521798312664
3 0.0714064538478851
4 0.0702831372618675
5 0.0691424906253815
6 0.0679727643728256
7 0.0667558461427689
8 0.0654556900262833
9 0.0639745816588402
10 0.062002532184124
11 0.0586205050349236
12 0.052093505859375
13 0.0455041639506817
14 0.0408899411559105
15 0.0372091867029667
16 0.0341332592070103
17 0.0314731001853943
18 0.029142314568162
19 0.0270908493548632
20 0.0252841487526894
21 0.0236893240362406
22 0.0222762413322926
23 0.0210187006741762
24 0.0198947675526142
25 0.0188857316970825
26 0.0179752502590418
27 0.0171489845961332
28 0.0163945239037275
29 0.0157013274729252
30 0.0150606045499444
31 0.0144651392474771
32 0.0139090437442064
33 0.0133874984458089
34 0.0128965452313423
35 0.0124329086393118
36 0.0119938440620899
37 0.0115770334377885
38 0.0111804818734527
39 0.0108024748042226
40 0.0104415230453014
41 0.0100963255390525
42 0.00976572092622519
43 0.00944868940860033
44 0.00914433598518372
45 0.00885184947401285
46 0.00857050996273756
47 0.00829966831952333
48 0.00803874805569649
49 0.00778722669929266
50 0.00754463952034712
51 0.00731055112555623
52 0.00708457455039024
53 0.00686635123565793
54 0.00665554264560342
55 0.00645183911547065
56 0.00625494960695505
57 0.00606459891423583
58 0.0058805332519114
59 0.00570251047611237
60 0.00553030613809824
61 0.00536370323970914
62 0.00520251085981727
63 0.00504654226824641
64 0.00489561865106225
65 0.00474956491962075
66 0.00460819713771343
67 0.0044713169336319
68 0.00433873292058706
69 0.00421025091782212
70 0.00408568698912859
71 0.00396486883983016
72 0.00384763162583113
73 0.00373381795361638
74 0.00362327881157398
75 0.00351586914621294
76 0.00341145345009863
77 0.00330990483053029
78 0.0032111038453877
79 0.00311494269408286
80 0.00302132032811642
81 0.00293014897033572
82 0.00284134829416871
83 0.00275484961457551
84 0.0026705942582339
85 0.00258853286504745
86 0.00250862981192768
87 0.00243085483089089
88 0.00235518696717918
89 0.00228161807172
90 0.00221014185808599
91 0.00214076065458357
92 0.00207348191179335
93 0.00200831377878785
94 0.00194526684936136
95 0.00188434938900173
96 0.00182556733489037
97 0.00176892161834985
98 0.00171440409030765
99 0.0016619983362034
100 0.00161167734768242
101 0.00156340294051915
102 0.00151713041123003
103 0.00147280213423073
104 0.00143035326618701
105 0.00138971174601465
106 0.00135080015752465
107 0.00131353770848364
108 0.00127784186042845
109 0.00124362879432738
110 0.00121081480756402
111 0.00117932027205825
112 0.00114906765520573
113 0.00111998373176903
114 0.00109199900180101
115 0.00106504873838276
116 0.00103907356970012
117 0.0010140179656446
118 0.00098983000498265
119 0.000966463063377887
120 0.000943873543292284
121 0.000922020990401506
122 0.00090086879208684
123 0.000880382489413023
124 0.000860530359204859
125 0.000841283588670194
126 0.000822613423224539
127 0.000804495008196682
128 0.000786904420237988
129 0.000769818958360702
130 0.000753217609599233
131 0.000737081049010158
132 0.000721391290426254
133 0.000706130755133927
134 0.000691283610649407
135 0.000676834955811501
136 0.000662770995404571
137 0.000649078167043626
138 0.000635743723250926
139 0.000622755964286625
140 0.000610103888902813
141 0.000597776379436255
142 0.000585763365961611
143 0.000574055302422494
144 0.000562643108423799
145 0.000551517936401069
146 0.000540671753697097
147 0.000530096120201051
148 0.000519783759955317
149 0.000509727105963975
150 0.000499918824061751
151 0.000490352336782962
152 0.000481019756989554
153 0.000471914187073708
154 0.000463028001831844
155 0.00045435331412591
156 0.00044588212040253
157 0.000437606795458123
158 0.000429518549935892
159 0.000421609584009275
160 0.000413871806813404
161 0.000406298291636631
162 0.000398881820729002
163 0.000391615525586531
164 0.000384494022000581
165 0.00037751195486635
166 0.000370664056390524
167 0.000363946775905788
168 0.000357355922460556
169 0.000350889167748392
170 0.000344543281244114
171 0.000338316516717896
172 0.000332206953316927
173 0.000326213019434363
174 0.000320333405397832
175 0.000314567150780931
176 0.000308912189211696
177 0.000303367240121588
178 0.000297929858788848
179 0.000292597367661074
180 0.000287367758573964
181 0.00028223791741766
182 0.000277205399470404
183 0.000272268720436841
184 0.000267426687059924
185 0.00026267790235579
186 0.000258022540947422
187 0.000253460108069703
188 0.000248991244006902
189 0.000244616036070511
190 0.000240335124544799
191 0.000236149324337021
192 0.000232059144764207
193 0.000228065444389358
194 0.000224168848944828
195 0.000220369707676582
196 0.000216669170185924
197 0.000213067862205207
198 0.000209565201657824
199 0.000206161756068468
200 0.000202857161639258
201 0.000199651054572314
202 0.000196541775949299
203 0.000193528961972333
204 0.00019061061902903
205 0.000187784695299342
206 0.000185049153515138
207 0.000182402131031267
208 0.000179840004420839
209 0.000177360168891028
210 0.000174960325239226
211 0.000172636879142374
212 0.000170386745594442
213 0.000168207086971961
214 0.000166094789165072
215 0.000164047043654136
216 0.000162061100127175
217 0.000160133859026246
218 0.000158263108460233
219 0.000156446476466954
220 0.000154681576532312
221 0.000152966531459242
222 0.000151298852870241
223 0.000149677012814209
224 0.000148099366924725
225 0.000146564038004726
226 0.000145069789141417
227 0.00014361493231263
228 0.000142197968671098
229 0.000140817937790416
230 0.000139473209856078
231 0.000138162708026357
232 0.00013688528269995
233 0.000135639813379385
234 0.000134425106807612
235 0.000133240260765888
236 0.000132084038341418
237 0.000130955566419289
238 0.000129853840917349
239 0.0001287780032726
240 0.000127726860227995
241 0.000126699800603092
242 0.000125695849419571
243 0.000124714206322096
244 0.000123754027299583
245 0.000122814599308185
246 0.000121895143820439
247 0.000120995006000157
248 0.000120113480079453
249 0.000119249969429802
250 0.000118403804663103
251 0.000117574389150832
252 0.000116761344543193
253 0.000115964052383788
254 0.000115181981527712
255 0.000114414877316449
256 0.000113662179501262
257 0.000112923713459168
258 0.000112198918941431
259 0.000111487781396136
260 0.000110789704194758
261 0.000110104796476662
262 0.000109432621684391
263 0.000108773216197733
264 0.000108126281702425
265 0.000107491978269536
266 0.000106869883893523
267 0.000106260311440565
268 0.000105662977148313
269 0.00010507812112337
270 0.000104505321360193
271 0.000103945065347943
272 0.000103396901977248
273 0.000102861245977692
274 0.000102337631687988
275 0.000101826393802185
276 0.000101326775620691
277 0.000100839410151821
278 0.000100363249657676
279 9.98990872176364e-05
280 9.94451984297484e-05
281 9.90026164799929e-05
282 9.85694205155596e-05
283 9.81467892415822e-05
284 9.7732205176726e-05
285 9.73271016846411e-05
286 9.69288812484592e-05
287 9.653879533289e-05
288 9.61544137680903e-05
289 9.57767333602533e-05
290 9.54038114286959e-05
291 9.5036048151087e-05
292 9.46723812376149e-05
293 9.43124396144412e-05
294 9.39561068662442e-05
295 9.36023716349155e-05
296 9.32517723413184e-05
297 9.29033412830904e-05
298 9.25572458072565e-05
299 9.22130711842328e-05
300 9.18711157282814e-05
301 9.15310738491826e-05
302 9.11925744730979e-05
303 9.08562433323823e-05
304 9.0521527454257e-05
305 9.01885869097896e-05
306 8.98575017345138e-05
307 8.95280827535316e-05
308 8.92004682100378e-05
309 8.88746217242442e-05
310 8.85506524355151e-05
311 8.82282693055458e-05
312 8.79075887496583e-05
313 8.75889163580723e-05
314 8.72713935677893e-05
315 8.69553186930716e-05
316 8.6640939116478e-05
317 8.63276800373569e-05
318 8.60154759720899e-05
319 8.57039340189658e-05
320 8.53933888720348e-05
321 8.50830911076628e-05
322 8.47731498652138e-05
323 8.446322317468e-05
324 8.4152867202647e-05
325 8.38422129163519e-05
326 8.35304090287536e-05
327 8.32170553621836e-05
328 8.29022174002603e-05
329 8.25850074761547e-05
330 8.22651854832657e-05
331 8.19421838968992e-05
332 8.16156825749204e-05
333 8.12850485090166e-05
334 8.09496996225789e-05
335 8.06090683909133e-05
336 8.02629801910371e-05
337 7.99106783233583e-05
338 7.95515807112679e-05
339 7.91858401498757e-05
340 7.88126781117171e-05
341 7.84318399382755e-05
342 7.80437185312621e-05
343 7.76477172621526e-05
344 7.72442581364885e-05
345 7.68334793974645e-05
346 7.64155702199787e-05
347 7.59905815357342e-05
348 7.55598812247626e-05
349 7.51233310438693e-05
350 7.4682458944153e-05
351 7.42377233109437e-05
352 7.37895388738252e-05
353 7.33396373107098e-05
354 7.28884915588424e-05
355 7.24373676348478e-05
356 7.19869276508689e-05
357 7.15381247573532e-05
358 7.10921594873071e-05
359 7.0649228291586e-05
360 7.02108809491619e-05
361 6.97773939464241e-05
362 6.93490874255076e-05
363 6.89266162225977e-05
364 6.8510533310473e-05
365 6.81009260006249e-05
366 6.769868923584e-05
367 6.7302979005035e-05
368 6.69143337290734e-05
369 6.65332918288186e-05
370 6.61589583614841e-05
371 6.5791726228781e-05
372 6.5431377151981e-05
373 6.50781439617276e-05
374 6.47312772343867e-05
375 6.43910534563474e-05
376 6.40570797258988e-05
377 6.37291304883547e-05
378 6.34065072517842e-05
379 6.30897557130083e-05
380 6.27780391369015e-05
381 6.24714011792094e-05
382 6.2169499869924e-05
383 6.18718404439278e-05
384 6.15783428656869e-05
385 6.12888397881761e-05
386 6.1002927395748e-05
387 6.07206166023389e-05
388 6.0441128880484e-05
389 6.01648862357251e-05
390 5.98914084548596e-05
391 5.962053546682e-05
392 5.9352139942348e-05
393 5.90861673117615e-05
394 5.88220100325998e-05
395 5.85599227633793e-05
396 5.82994398428127e-05
397 5.80407249799464e-05
398 5.77837636228651e-05
399 5.75281483179424e-05
400 5.72736425965559e-05
401 5.70205265830737e-05
402 5.67685419810005e-05
403 5.65174086659681e-05
404 5.6267403124366e-05
405 5.60182670596987e-05
406 5.57695602765307e-05
407 5.55217957298737e-05
408 5.52748861082364e-05
409 5.5028467613738e-05
410 5.47825693502091e-05
411 5.45373441127595e-05
412 5.42927118658554e-05
413 5.40484179509804e-05
414 5.38043495907914e-05
415 5.35606959601864e-05
416 5.33173733856529e-05
417 5.30741090187803e-05
418 5.28313030372374e-05
419 5.25889154232573e-05
420 5.23466696904507e-05
421 5.21047222719062e-05
422 5.18626693519764e-05
423 5.16209402121603e-05
424 5.13792692800052e-05
425 5.11375947098713e-05
426 5.08963203174062e-05
427 5.06549113197252e-05
428 5.04137460666243e-05
429 5.01725917274598e-05
430 4.99313937325496e-05
431 4.96902102895547e-05
432 4.94491287099663e-05
433 4.92078943352681e-05
434 4.89665762870573e-05
435 4.87253637402318e-05
436 4.84838819829747e-05
437 4.82424075016752e-05
438 4.80007547594141e-05
439 4.77588982903399e-05
440 4.75170199933928e-05
441 4.72747560706921e-05
442 4.70323248009663e-05
443 4.67899299110286e-05
444 4.65469311166089e-05
445 4.63035867142025e-05
446 4.60599694633856e-05
447 4.58157664979808e-05
448 4.55712543043774e-05
449 4.53263928648084e-05
450 4.5080974814482e-05
451 4.4835040171165e-05
452 4.45886689703912e-05
453 4.43418066424783e-05
454 4.40945477748755e-05
455 4.38465394836385e-05
456 4.35982292401604e-05
457 4.33493478340097e-05
458 4.30996115028393e-05
459 4.28495768574066e-05
460 4.25991056545172e-05
461 4.2348157876404e-05
462 4.20967044192366e-05
463 4.18446034018416e-05
464 4.15925569541287e-05
465 4.13396410294808e-05
466 4.10867178288754e-05
467 4.08335763495415e-05
468 4.05801984015852e-05
469 4.0326649468625e-05
470 4.00733806600329e-05
471 3.9820162783144e-05
472 3.95671122532804e-05
473 3.93144073314033e-05
474 3.90622444683686e-05
475 3.8810267142253e-05
476 3.8559315726161e-05
477 3.83088008675259e-05
478 3.80593883164693e-05
479 3.78108888980933e-05
480 3.75635027012322e-05
481 3.73174480046146e-05
482 3.70729212590959e-05
483 3.68296423403081e-05
484 3.65876949217636e-05
485 3.63478793588001e-05
486 3.61098282155581e-05
487 3.58733414032031e-05
488 3.56389828084502e-05
489 3.54068979504518e-05
490 3.51768321706913e-05
491 3.49489273503423e-05
492 3.47233108186629e-05
493 3.44997679349035e-05
494 3.42785751854535e-05
495 3.40599726769142e-05
496 3.3843705750769e-05
497 3.36297489411663e-05
498 3.34182259393856e-05
499 3.32090421579778e-05
500 3.30022703565191e-05
501 3.2798147003632e-05
502 3.2596210076008e-05
503 3.23966123687569e-05
504 3.21994484693278e-05
505 3.20047256536782e-05
506 3.18122220051009e-05
507 3.16218975058291e-05
508 3.14339849865064e-05
509 3.12480697175488e-05
510 3.10644099954516e-05
511 3.08830130961724e-05
512 3.07036316371523e-05
513 3.05261892208364e-05
514 3.03508277283981e-05
515 3.01776763080852e-05
516 3.00062711175997e-05
517 2.98369704978541e-05
518 2.96694324788405e-05
519 2.95038807962555e-05
520 2.93401717499364e-05
521 2.91779815597693e-05
522 2.90177104034228e-05
523 2.88593291770667e-05
524 2.87024249701062e-05
525 2.85472360701533e-05
526 2.83937424683245e-05
527 2.82419023278635e-05
528 2.80916883639293e-05
529 2.79428477369947e-05
530 2.77956951322267e-05
531 2.76500213658437e-05
532 2.75060774583835e-05
533 2.73631721938727e-05
534 2.72221805062145e-05
535 2.7082409360446e-05
536 2.69442916760454e-05
537 2.68073254119372e-05
538 2.6671707018977e-05
539 2.65376329480205e-05
540 2.64049467659788e-05
541 2.62734793068375e-05
542 2.61434688582085e-05
543 2.60145898209885e-05
544 2.58872260019416e-05
545 2.57608116953634e-05
546 2.56358671322232e-05
547 2.55122922681039e-05
548 2.53898469964042e-05
549 2.52685895247851e-05
550 2.5148707209155e-05
551 2.50299781328067e-05
552 2.49122622335562e-05
553 2.47960051638074e-05
554 2.46807794610504e-05
555 2.45664705289528e-05
556 2.44535003730562e-05
557 2.43416743614944e-05
558 2.42309452005429e-05
559 2.41212128457846e-05
560 2.40126810240326e-05
561 2.39053624682128e-05
562 2.37989006564021e-05
563 2.36934156419011e-05
564 2.35890820476925e-05
565 2.34856397582917e-05
566 2.33834944083355e-05
567 2.32820511882892e-05
568 2.31816065934254e-05
569 2.30823425226845e-05
570 2.29836732614785e-05
571 2.28863045776961e-05
572 2.27898090088274e-05
573 2.269407195854e-05
574 2.25993680942338e-05
575 2.25056137423962e-05
576 2.24126033572247e-05
577 2.23204369831365e-05
578 2.22293710976373e-05
579 2.21386580960825e-05
580 2.2049145627534e-05
581 2.1960293452139e-05
582 2.18722725549014e-05
583 2.17848955799127e-05
584 2.16985717997886e-05
585 2.16130065382458e-05
586 2.15278032555943e-05
587 2.14435294765281e-05
588 2.13599942071596e-05
589 2.12771683436586e-05
590 2.11948936339468e-05
591 2.11134902201593e-05
592 2.10324815270724e-05
593 2.09521822398528e-05
594 2.08727906283457e-05
595 2.07936100196093e-05
596 2.07151588256238e-05
597 2.06373260880355e-05
598 2.05599735636497e-05
599 2.04832322197035e-05
600 2.0407052943483e-05
601 2.03312210942386e-05
602 2.02561222977238e-05
603 2.0181438230793e-05
604 2.01070706680184e-05
605 2.00334688997827e-05
606 1.99600981432013e-05
607 1.9887111193384e-05
608 1.98149737116182e-05
609 1.97427143575624e-05
610 1.96712808246957e-05
611 1.96001801668899e-05
612 1.95291304407874e-05
613 1.94590611499734e-05
614 1.93888354260707e-05
615 1.93194409803255e-05
616 1.92500374396332e-05
617 1.91812778211897e-05
618 1.91123999684351e-05
619 1.90444134204881e-05
620 1.89763104572194e-05
621 1.8909209757112e-05
622 1.88413159776246e-05
623 1.87751829798799e-05
624 1.8707620256464e-05
625 1.86424622370396e-05
626 1.85751614480978e-05
627 1.85112403414678e-05
628 1.84438395081088e-05
629 1.83812662726268e-05
630 1.8313636246603e-05
631 1.82530002348358e-05
632 1.81842151505407e-05
633 1.81263185368152e-05
634 1.80558345164172e-05
635 1.80012011696817e-05
636 1.79278285941109e-05
637 1.78783375304192e-05
638 1.77999481820734e-05
639 1.77583333424991e-05
640 1.76714347617235e-05
641 1.76418034243397e-05
642 1.75418899743818e-05
643 1.7530099285068e-05
644 1.74097149283625e-05
645 1.74246179085458e-05
646 1.72745312738698e-05
647 1.73261669260683e-05
648 1.71378378581721e-05
649 1.72293348441599e-05
650 1.70087714650435e-05
651 1.71043939189985e-05
652 1.6888152458705e-05
653 1.68753067555372e-05
654 1.67370635608677e-05
655 1.65392248163698e-05
656 1.65673045557924e-05
657 1.632827843423e-05
658 1.64135071827332e-05
659 1.62786109285662e-05
660 1.62741216627182e-05
661 1.62358737725299e-05
662 1.6182604667847e-05
663 1.61661864694906e-05
664 1.61183343152516e-05
665 1.6092801160994e-05
666 1.60559193318477e-05
667 1.60235958901467e-05
668 1.59907158376882e-05
669 1.59559967869427e-05
670 1.59236988110933e-05
671 1.5887317204033e-05
672 1.58552411448909e-05
673 1.58174952957779e-05
674 1.57854847202543e-05
675 1.57457707246067e-05
676 1.57142440002644e-05
677 1.56723417603644e-05
678 1.56417536345543e-05
679 1.55970592459198e-05
680 1.55686739162775e-05
681 1.5519990483881e-05
682 1.54954723257106e-05
683 1.54405115608824e-05
684 1.54227054736111e-05
685 1.53582332131919e-05
686 1.53520468302304e-05
687 1.52714128489606e-05
688 1.52861466631293e-05
689 1.51780241139932e-05
690 1.52288957906421e-05
691 1.50740006574779e-05
692 1.51870917761698e-05
693 1.49532461364288e-05
694 1.51726835611043e-05
695 1.48110948430258e-05
696 1.51991143866326e-05
697 1.46658012454282e-05
698 1.52486109072925e-05
699 1.45790554597625e-05
700 1.51022813952295e-05
701 1.44614286909928e-05
702 1.44427112900303e-05
703 1.43501310958527e-05
704 1.40802794703632e-05
705 1.42896533361636e-05
706 1.41104574140627e-05
707 1.40979582283762e-05
708 1.4123210348771e-05
709 1.40331148941186e-05
710 1.40861420732108e-05
711 1.40311640279833e-05
712 1.40476577144e-05
713 1.40208385346341e-05
714 1.40267220558599e-05
715 1.40033807838336e-05
716 1.40090523927938e-05
717 1.39850253617624e-05
718 1.39896419568686e-05
719 1.39638059408753e-05
720 1.39684761961689e-05
721 1.39385756483534e-05
722 1.39451321956585e-05
723 1.39086396302446e-05
724 1.39198937176843e-05
725 1.38729401442106e-05
726 1.38935774884885e-05
727 1.38297200464876e-05
728 1.38680570671568e-05
729 1.37769638968166e-05
730 1.38477407745086e-05
731 1.37093375087716e-05
732 1.38404220706434e-05
733 1.36172438942594e-05
734 1.38625900945044e-05
735 1.34838473968557e-05
736 1.39469502755674e-05
737 1.32853519971832e-05
738 1.41563978104386e-05
739 1.30220796563663e-05
740 1.45709300340968e-05
741 1.28591209431761e-05
742 1.49292254718603e-05
743 1.27632811199874e-05
744 1.38474388222676e-05
745 1.26065560834832e-05
746 1.30772168631665e-05
747 1.27131606859621e-05
748 1.3038627912465e-05
749 1.24010930449003e-05
750 1.30190073832637e-05
751 1.24978587336955e-05
752 1.27960502140922e-05
753 1.25871802083566e-05
754 1.2706085726677e-05
755 1.25989909065538e-05
756 1.27236035041278e-05
757 1.25851120174048e-05
758 1.27287648865604e-05
759 1.25965862025623e-05
760 1.2732925824821e-05
761 1.26033537526382e-05
762 1.2747045730066e-05
763 1.25985916383797e-05
764 1.27665816762601e-05
765 1.25842325360281e-05
766 1.27918629004853e-05
767 1.25556871353183e-05
768 1.28268493426731e-05
769 1.25069855130278e-05
770 1.28806068460108e-05
771 1.24279631563695e-05
772 1.29692316477303e-05
773 1.23033232739544e-05
774 1.31235210574232e-05
775 1.21094572023139e-05
776 1.34031861307449e-05
777 1.18171037684078e-05
778 1.39257190312492e-05
779 1.14152780952281e-05
780 1.48936569530633e-05
781 1.09985267044976e-05
782 1.64433567988453e-05
783 1.08439326140797e-05
784 1.76038793142652e-05
785 1.0971532901749e-05
786 1.59128285304178e-05
787 1.1318478755129e-05
788 1.27040921142907e-05
789 1.18227899292833e-05
790 1.15578168333741e-05
791 1.19209544209298e-05
792 1.16927549242973e-05
793 1.15507036753115e-05
794 1.17219069579733e-05
795 1.15604607344721e-05
796 1.16342180263018e-05
797 1.16244063974591e-05
798 1.16015962703386e-05
799 1.16356395665207e-05
800 1.16267920020618e-05
801 1.16486162369256e-05
802 1.16506134872907e-05
803 1.16714836622123e-05
804 1.16732280730503e-05
805 1.16951687232358e-05
806 1.16955870907987e-05
807 1.17176796265994e-05
808 1.17143554234644e-05
809 1.17389836304937e-05
810 1.17275822049123e-05
811 1.1759629160224e-05
812 1.17321797006298e-05
813 1.1781839930336e-05
814 1.17230520118028e-05
815 1.18126035886235e-05
816 1.16894425445935e-05
817 1.1869810805365e-05
818 1.16081973828841e-05
819 1.19986098070513e-05
820 1.14352433229215e-05
821 1.23145946417935e-05
822 1.11515591925127e-05
823 1.30946064018644e-05
824 1.12974075818784e-05
825 1.4122597349342e-05
826 1.14904050860787e-05
827 1.13095338747371e-05
828 1.18697207653895e-05
829 1.14533950181794e-05
830 1.05533727037255e-05
831 1.18744455903652e-05
832 1.09888005681569e-05
833 1.04900700534927e-05
834 1.15964439828531e-05
835 1.08189433376538e-05
836 1.04256760096177e-05
837 1.1387130143703e-05
838 1.07349324025563e-05
839 1.0409231435915e-05
840 1.12290554170613e-05
841 1.06933230199502e-05
842 1.04364753497066e-05
843 1.11129575088853e-05
844 1.06737161331694e-05
845 1.04987639133469e-05
846 1.10273995233001e-05
847 1.06679572127177e-05
848 1.05939698187285e-05
849 1.09547327156179e-05
850 1.06788138509728e-05
851 1.0711685717979e-05
852 1.08921649371041e-05
853 1.07146606751485e-05
854 1.08224239738774e-05
855 1.08569947769865e-05
856 1.07883515738649e-05
857 1.0884519724641e-05
858 1.08737749542342e-05
859 1.08819467641297e-05
860 1.09126840470708e-05
861 1.0933631529042e-05
862 1.09395505205612e-05
863 1.09683896880597e-05
864 1.09778320620535e-05
865 1.0994313925039e-05
866 1.10096871139831e-05
867 1.10191458588815e-05
868 1.10354030766757e-05
869 1.10366845547105e-05
870 1.10590208350914e-05
871 1.1043459380744e-05
872 1.10832670543459e-05
873 1.10340479295701e-05
874 1.11168319563149e-05
875 1.09948095996515e-05
876 1.11815716081765e-05
877 1.08959229692118e-05
878 1.13347277874709e-05
879 1.06726965896087e-05
880 1.17363460958586e-05
881 1.02205121947918e-05
882 1.28797109937295e-05
883 9.62201920629013e-06
884 1.61466850840952e-05
885 1.02896919997875e-05
886 2.01607690542005e-05
887 1.07050509541295e-05
888 1.26950681078597e-05
889 1.06402303572395e-05
890 9.65892922977218e-06
891 1.06567376860767e-05
892 1.03528200270375e-05
893 9.52251230046386e-06
894 1.03580741779297e-05
895 1.01092064141994e-05
896 9.4969363999553e-06
897 1.02283411251847e-05
898 9.92884088191204e-06
899 9.57585416472284e-06
900 1.01155983429635e-05
901 9.80177082965383e-06
902 9.73353508015862e-06
903 9.99985786620528e-06
904 9.76119463302894e-06
905 9.88796909950906e-06
906 9.92041168501601e-06
907 9.84685266303131e-06
908 9.95588743535336e-06
909 9.93194407783449e-06
910 9.97998358798213e-06
911 9.99188068817602e-06
912 1.00370825748541e-05
913 1.00498018582584e-05
914 1.0095109246322e-05
915 1.01113755590632e-05
916 1.0155690688407e-05
917 1.01688419817947e-05
918 1.02182293630904e-05
919 1.02206167866825e-05
920 1.02818175946595e-05
921 1.02620833786204e-05
922 1.03504507933394e-05
923 1.02842486739974e-05
924 1.04361924968543e-05
925 1.02670010164729e-05
926 1.05744984466583e-05
927 1.01667728813482e-05
928 1.08652175185853e-05
929 9.91149863693863e-06
930 1.16036462713964e-05
931 9.66116203926504e-06
932 1.34938609335222e-05
933 1.08478170659509e-05
934 1.20943568617804e-05
935 1.04005375760607e-05
936 1.02043431979837e-05
937 9.96578091871925e-06
938 1.0864501746255e-05
939 9.71364625002025e-06
940 9.47466287470888e-06
941 1.07331479739514e-05
942 9.60473244049354e-06
943 9.21491937333485e-06
944 1.04803511931095e-05
945 9.62822377914563e-06
946 9.15324835659703e-06
947 1.01904288385413e-05
948 9.69437314779498e-06
949 9.18719979381422e-06
950 9.96204107650556e-06
951 9.74179874901893e-06
952 9.25778931559762e-06
953 9.82341498456663e-06
954 9.75898637989303e-06
955 9.33790488488739e-06
956 9.76572118815966e-06
957 9.75484363152646e-06
958 9.42061706155073e-06
959 9.76757382886717e-06
960 9.74068643699866e-06
961 9.51215406530537e-06
962 9.79955439106561e-06
963 9.73336864262819e-06
964 9.62197282206034e-06
965 9.82941037364071e-06
966 9.75432521954644e-06
967 9.74758495431161e-06
968 9.84523012448335e-06
969 9.81778794084676e-06
970 9.85221868177177e-06
971 9.88076862995513e-06
972 9.90252010524273e-06
973 9.91692922980292e-06
974 9.95398477243725e-06
975 9.96065227809595e-06
976 9.99411440716358e-06
977 1.00006536740693e-05
978 1.00329389169929e-05
979 1.00324377854122e-05
980 1.00672496046172e-05
981 1.00576526165241e-05
982 1.00956067399238e-05
983 1.00761617431999e-05
984 1.01151736089378e-05
985 1.00922652563895e-05
986 1.01163786894176e-05
987 1.01242176242522e-05
988 1.0066579307022e-05
989 1.02371159300674e-05
990 9.86937084235251e-06
991 1.06713550849236e-05
992 9.34416311793029e-06
993 1.2415601304383e-05
994 9.21056835068157e-06
995 1.78047266672365e-05
996 1.09787133624195e-05
997 1.22479677884257e-05
998 1.03168822533917e-05
999 9.75358761934331e-06
1000 8.98358030099189e-06
1001 1.08421763798106e-05
1002 9.55101677391212e-06
1003 8.76049307407811e-06
1004 1.04856380858109e-05
1005 9.63945149123902e-06
1006 8.78938226378523e-06
1007 1.00097367976559e-05
1008 9.79873402684461e-06
1009 8.91450690687634e-06
1010 9.63512229645858e-06
1011 9.90358785202261e-06
1012 9.06173227122054e-06
1013 9.4230126705952e-06
1014 9.92650802800199e-06
1015 9.19279318623012e-06
1016 9.3368316811393e-06
1017 9.89829004538478e-06
1018 9.29047564568464e-06
1019 9.33083174459171e-06
1020 9.8495111160446e-06
1021 9.35361913434463e-06
1022 9.37717231863644e-06
1023 9.79140895651653e-06
1024 9.3963899416849e-06
1025 9.45678766584024e-06
1026 9.72730413195677e-06
1027 9.43821396504063e-06
1028 9.5516188594047e-06
1029 9.66344759945059e-06
1030 9.50096728047356e-06
1031 9.62890226219315e-06
1032 9.62640388024738e-06
1033 9.58987311605597e-06
1034 9.66023071669042e-06
1035 9.64739137998549e-06
1036 9.66643256106181e-06
1037 9.6810454124352e-06
1038 9.70216206042096e-06
1039 9.70504515862558e-06
1040 9.73136684478959e-06
1041 9.73191436060006e-06
1042 9.75605234998511e-06
1043 9.75327748165e-06
1044 9.77825038717128e-06
1045 9.76914179773303e-06
1046 9.79521701083286e-06
1047 9.77985200734111e-06
1048 9.80515233095502e-06
1049 9.78834759735037e-06
1050 9.80106142378645e-06
1051 9.80638560577063e-06
1052 9.76207047642674e-06
1053 9.87507155514322e-06
1054 9.62312697083689e-06
1055 1.01408077171072e-05
1056 9.22579056350514e-06
1057 1.11890913103707e-05
1058 8.60667387314606e-06
1059 1.53702712850645e-05
1060 1.05831913970178e-05
1061 1.80812749022152e-05
1062 9.17679517442593e-06
1063 9.08387119125109e-06
1064 1.10713954200037e-05
1065 9.77368381427368e-06
1066 8.74138822837267e-06
1067 1.03665870483383e-05
1068 1.00622773970827e-05
1069 8.86628367879894e-06
1070 9.88413557934109e-06
1071 1.03457996374345e-05
1072 9.0993653429905e-06
1073 9.5934310593293e-06
1074 1.05571843960206e-05
1075 9.39252367970766e-06
1076 9.50321918935515e-06
1077 1.06763900475926e-05
1078 9.70791097643087e-06
1079 9.57093197939685e-06
1080 1.07437545011635e-05
1081 1.0014411600423e-05
1082 9.74135855358327e-06
1083 1.08116237242939e-05
1084 1.02944786704029e-05
1085 9.97533879854018e-06
1086 1.09144548332551e-05
1087 1.05471826827852e-05
1088 1.0252719221171e-05
1089 1.10639730337425e-05
1090 1.07833393485635e-05
1091 1.05653343780432e-05
1092 1.12576108222129e-05
1093 1.10187165773823e-05
1094 1.09117818283266e-05
1095 1.14835829663207e-05
1096 1.127087034547e-05
1097 1.12888228613883e-05
1098 1.17274876174633e-05
1099 1.15532666313811e-05
1100 1.1680078387144e-05
1101 1.19760088637122e-05
1102 1.18628822747269e-05
1103 1.2049753422616e-05
1104 1.22119326988468e-05
1105 1.21661196317291e-05
1106 1.23418622024474e-05
1107 1.23968939078622e-05
1108 1.23871695905109e-05
1109 1.24866046462557e-05
1110 1.245493604074e-05
1111 1.24261214295984e-05
1112 1.24099324239069e-05
1113 1.23015361168655e-05
1114 1.22050087156822e-05
1115 1.20784470709623e-05
1116 1.190462262457e-05
1117 1.17389572551474e-05
1118 1.15425236799638e-05
1119 1.13394680738566e-05
1120 1.11400095192948e-05
1121 1.09345619421219e-05
1122 1.0739596291387e-05
1123 1.05539020296419e-05
1124 1.03774127637735e-05
1125 1.02171206890489e-05
1126 1.00680026662303e-05
1127 9.93469802779146e-06
1128 9.81289758783532e-06
1129 9.70567725744331e-06
1130 9.60873421718134e-06
1131 9.52438676904421e-06
1132 9.44814655667869e-06
1133 9.38298489927547e-06
1134 9.32433886191575e-06
1135 9.27513337956043e-06
1136 9.2305899670464e-06
1137 9.19465765036875e-06
1138 9.16185672394931e-06
1139 9.13726853468688e-06
1140 9.11341157916468e-06
1141 9.09953541849973e-06
1142 9.0822341007879e-06
1143 9.07857065612916e-06
1144 9.06447712623049e-06
1145 9.07267076399876e-06
1146 9.05619708646554e-06
1147 9.08165839064168e-06
1148 9.05117303773295e-06
1149 9.10882044991013e-06
1150 9.03793807083275e-06
1151 9.16694261832163e-06
1152 8.99212136573624e-06
1153 9.29737325350288e-06
1154 8.85784902493469e-06
1155 9.63307775236899e-06
1156 8.53767687658546e-06
1157 1.06585503090173e-05
1158 8.14174654806266e-06
1159 1.40382317113108e-05
1160 9.36136530071963e-06
1161 1.76350586116314e-05
1162 8.93047308636596e-06
1163 1.06983625300927e-05
1164 9.06739842321258e-06
1165 7.56489453124232e-06
1166 9.20542424864834e-06
1167 8.32323530630674e-06
1168 7.48787488191738e-06
1169 8.84154906088952e-06
1170 8.08412369224243e-06
1171 7.42274278309196e-06
1172 8.61126045492711e-06
1173 7.99733788880985e-06
1174 7.41392750569503e-06
1175 8.42388817545725e-06
1176 7.97216762293829e-06
1177 7.45182387618115e-06
1178 8.28376778372331e-06
1179 7.96880431153113e-06
1180 7.52941332393675e-06
1181 8.20646437205141e-06
1182 7.97081429482205e-06
1183 7.64939795772079e-06
1184 8.19098659121664e-06
1185 7.98434030002682e-06
1186 7.82033657742431e-06
1187 8.21467619971372e-06
1188 8.03357215772849e-06
1189 8.03953935246682e-06
1190 8.25376355351182e-06
1191 8.14950271887938e-06
1192 8.26192808744963e-06
1193 8.31971647130558e-06
1194 8.3372206063359e-06
1195 8.42547524371184e-06
1196 8.45278373162728e-06
1197 8.52711855259258e-06
1198 8.55965390655911e-06
1199 8.62984052218962e-06
1200 8.66110713104717e-06
1201 8.72586861078162e-06
1202 8.75356727192411e-06
1203 8.81393043528078e-06
1204 8.83314714883454e-06
1205 8.89216153154848e-06
1206 8.89729653863469e-06
1207 8.95937955647241e-06
1208 8.94235472514993e-06
1209 9.01708790479461e-06
1210 8.96204801392742e-06
1211 9.07181583897909e-06
1212 8.94521690497641e-06
1213 9.14179690880701e-06
1214 8.86768975760788e-06
1215 9.27667952055344e-06
1216 8.6799482232891e-06
1217 9.62205922405701e-06
1218 8.31955185276456e-06
1219 1.06307925307192e-05
1220 8.01746500656009e-06
1221 1.3172834769648e-05
1222 8.99719543667743e-06
1223 1.19000033009797e-05
1224 8.42839108372573e-06
1225 7.53159656596836e-06
1226 9.0258554337197e-06
1227 8.43654834170593e-06
1228 7.41720532460022e-06
1229 8.19007163954666e-06
1230 8.50628111948026e-06
1231 7.4777312875085e-06
1232 7.70532642491162e-06
1233 8.54740937938914e-06
1234 7.58624491936644e-06
1235 7.47980311643914e-06
1236 8.47297178552253e-06
1237 7.70227597968187e-06
1238 7.42470456316369e-06
1239 8.3253698903718e-06
1240 7.79675247031264e-06
1241 7.46249770600116e-06
1242 8.18264470581198e-06
1243 7.85169595474144e-06
1244 7.54863003749051e-06
1245 8.09102584753418e-06
1246 7.87464523455128e-06
1247 7.66805351304356e-06
1248 8.05661602498731e-06
1249 7.89488967711804e-06
1250 7.8201228461694e-06
1251 8.06078514870023e-06
1252 7.95242885942571e-06
1253 7.99142071628012e-06
1254 8.09471293905517e-06
1255 8.07244578027166e-06
1256 8.14481700217584e-06
1257 8.18015087133972e-06
1258 8.22971742309164e-06
1259 8.26961331767961e-06
1260 8.32479418022558e-06
1261 8.36375147628132e-06
1262 8.41630571812857e-06
1263 8.45561044116039e-06
1264 8.50463948154356e-06
1265 8.54167956276797e-06
1266 8.58477142173797e-06
1267 8.62113665789366e-06
1268 8.65184028953081e-06
1269 8.69431823957711e-06
1270 8.6981071945047e-06
1271 8.76865942700533e-06
1272 8.70676558406558e-06
1273 8.86981342773652e-06
1274 8.6370118879131e-06
1275 9.08209403860383e-06
1276 8.40329630591441e-06
1277 9.71614826994482e-06
1278 7.98415476310765e-06
1279 1.20104696179624e-05
1280 8.64376397657907e-06
1281 1.6886624507606e-05
1282 9.59017506829696e-06
1283 1.07607938844012e-05
1284 8.43012639961671e-06
1285 7.31640739104478e-06
1286 8.74197485245531e-06
1287 7.96498898125719e-06
1288 7.3084665928036e-06
1289 8.27875828690594e-06
1290 7.73928604758112e-06
1291 7.35947332941578e-06
1292 8.03247530711815e-06
1293 7.62293302614125e-06
1294 7.45172201277455e-06
1295 7.85562679084251e-06
1296 7.55422934162198e-06
1297 7.56104282118031e-06
1298 7.71109625929967e-06
1299 7.5494726843317e-06
1300 7.63530897529563e-06
1301 7.6249566518527e-06
1302 7.61367300583515e-06
1303 7.64734613767359e-06
1304 7.64249307394493e-06
1305 7.67789879319025e-06
1306 7.68065001466312e-06
1307 7.7250178946997e-06
1308 7.73395913711283e-06
1309 7.78785397415049e-06
1310 7.7988361226744e-06
1311 7.86843793321168e-06
1312 7.8709626905038e-06
1313 7.96785116108367e-06
1314 7.94190327724209e-06
1315 8.09318407846149e-06
1316 7.99301506049233e-06
1317 8.26992709335173e-06
1318 7.9844367064652e-06
1319 8.58458588481881e-06
1320 7.84341045800829e-06
1321 9.34385479922639e-06
1322 7.61726460041245e-06
1323 1.15491639007814e-05
1324 8.57569102663547e-06
1325 1.19640972116031e-05
1326 8.58702333061956e-06
1327 7.66438643040601e-06
1328 9.15472901397152e-06
1329 9.19296599022346e-06
1330 7.68857262301026e-06
1331 8.0197805800708e-06
1332 9.55507766775554e-06
1333 8.01085025159409e-06
1334 7.53024187361007e-06
1335 9.34112540562637e-06
1336 8.50515334605007e-06
1337 7.59262275096262e-06
1338 8.73992121341871e-06
1339 8.96932124305749e-06
1340 7.90399553807219e-06
1341 8.27202075015521e-06
1342 9.16489625524264e-06
1343 8.27701387606794e-06
1344 8.1113466876559e-06
1345 9.09392383618979e-06
1346 8.61189255374484e-06
1347 8.1645994214341e-06
1348 8.92377465788741e-06
1349 8.84732253325637e-06
1350 8.31097258924274e-06
1351 8.78933315107133e-06
1352 8.97110930964118e-06
1353 8.47370210976806e-06
1354 8.73574026627466e-06
1355 9.0121075118077e-06
1356 8.61555326991947e-06
1357 8.75070327310823e-06
1358 9.01080056792125e-06
1359 8.72519922268111e-06
1360 8.80545030668145e-06
1361 8.99577389645856e-06
1362 8.80706920725061e-06
1363 8.8727283582557e-06
1364 8.9807635959005e-06
1365 8.87177520780824e-06
1366 8.92990829015616e-06
1367 8.97168865776621e-06
1368 8.92338721314445e-06
1369 8.96504297998035e-06
1370 8.9701989054447e-06
1371 8.95950142876245e-06
1372 8.97659356269287e-06
1373 8.97258905752096e-06
1374 8.97279642231297e-06
1375 8.97242716746405e-06
1376 8.96940673555946e-06
1377 8.96333767741453e-06
1378 8.95911125553539e-06
1379 8.94852200872265e-06
1380 8.94257846084656e-06
1381 8.92679508979199e-06
1382 8.92101161298342e-06
1383 8.89789225766435e-06
1384 8.89573220774764e-06
1385 8.86058114701882e-06
1386 8.86972702573985e-06
1387 8.81185769685544e-06
1388 8.84971996129025e-06
1389 8.74340821610531e-06
1390 8.85194094735198e-06
1391 8.63585682964185e-06
1392 8.92135267349659e-06
1393 8.44953046907904e-06
1394 9.19741796678863e-06
1395 8.15301154943882e-06
1396 1.0183824088017e-05
1397 8.11842073744629e-06
1398 1.32873892653151e-05
1399 9.83111567620654e-06
1400 1.4180050129653e-05
1401 9.03786713024601e-06
1402 8.75308978720568e-06
1403 8.09102311905008e-06
1404 8.61851549416315e-06
1405 7.90566809882876e-06
1406 8.04447518021334e-06
1407 8.34770980873145e-06
1408 7.83434461482102e-06
1409 8.06062507763272e-06
1410 8.11887639429187e-06
1411 7.87517274147831e-06
1412 8.0604067989043e-06
1413 7.98476139607374e-06
1414 7.95012965681963e-06
1415 8.00420184532413e-06
1416 7.95325013314141e-06
1417 7.98064957052702e-06
1418 7.95819141785614e-06
1419 7.97353368398035e-06
1420 7.96159747551428e-06
1421 7.97399843577296e-06
1422 7.97096799942665e-06
1423 7.9842793638818e-06
1424 7.98781002231408e-06
1425 8.00568614067743e-06
1426 8.01531132310629e-06
1427 8.03897637524642e-06
1428 8.05492709332611e-06
1429 8.08445292932447e-06
1430 8.10645815363387e-06
1431 8.14083159639267e-06
1432 8.16937517811311e-06
1433 8.20493642095244e-06
1434 8.24299604573753e-06
1435 8.26974519441137e-06
1436 8.33015928947134e-06
1437 8.31989564176183e-06
1438 8.44730311655439e-06
1439 8.31855686556082e-06
1440 8.65835681906901e-06
1441 8.17664113128558e-06
1442 9.21754144656006e-06
1443 7.79209130996605e-06
1444 1.12272537080571e-05
1445 8.10217443358852e-06
1446 1.61558782565407e-05
1447 9.29887846723432e-06
1448 8.63411423779326e-06
1449 1.0640978871379e-05
1450 8.03472812549444e-06
1451 7.17340071787476e-06
1452 1.00874894997105e-05
1453 8.75605564942816e-06
1454 7.23115272194264e-06
1455 8.50642300065374e-06
1456 9.53006383497268e-06
1457 7.97442316979868e-06
1458 7.67942310631042e-06
1459 9.26409302337561e-06
1460 8.8193773990497e-06
1461 7.86262626206735e-06
1462 8.52207176649245e-06
1463 9.22087838262087e-06
1464 8.41351720737293e-06
1465 8.19876731839031e-06
1466 9.03347154235234e-06
1467 8.89702005224535e-06
1468 8.31694069347577e-06
1469 8.67956168804085e-06
1470 9.08644597075181e-06
1471 8.61009630170884e-06
1472 8.50038577482337e-06
1473 8.99692804523511e-06
1474 8.86987345438683e-06
1475 8.52653647598345e-06
1476 8.81063897395507e-06
1477 8.9908180598286e-06
1478 8.65446781972423e-06
1479 8.67518247105181e-06
1480 8.97029349289369e-06
1481 8.78333139553433e-06
1482 8.62686647451483e-06
1483 8.87047735886881e-06
1484 8.85284862306435e-06
1485 8.63260720507242e-06
1486 8.75610021466855e-06
1487 8.8468759713578e-06
1488 8.64636240294203e-06
1489 8.65625679580262e-06
1490 8.77942602528492e-06
1491 8.63517743709963e-06
1492 8.57117174746236e-06
1493 8.67515791469486e-06
1494 8.58554903970798e-06
1495 8.48929812491406e-06
1496 8.55385678733001e-06
1497 8.50016294862144e-06
1498 8.40050324768526e-06
1499 8.42849840410054e-06
1500 8.3903132690466e-06
1501 8.30257522466127e-06
1502 8.30481621960644e-06
1503 8.26963150757365e-06
1504 8.19970955490135e-06
1505 8.18689477455337e-06
1506 8.15079147287179e-06
1507 8.09801986179082e-06
1508 8.07769993116381e-06
1509 8.04222509032115e-06
1510 8.00407178758178e-06
1511 7.98028759163572e-06
1512 7.94929110270459e-06
1513 7.92124683357542e-06
1514 7.89790919952793e-06
1515 7.87329645390855e-06
1516 7.85182419349439e-06
1517 7.83241830504267e-06
1518 7.81339895183919e-06
1519 7.79763286118396e-06
1520 7.78196590545122e-06
1521 7.76949764258461e-06
1522 7.75631178839831e-06
1523 7.74778800405329e-06
1524 7.73597821535077e-06
1525 7.73193096392788e-06
1526 7.71970462665195e-06
1527 7.72210751165403e-06
1528 7.7054710345692e-06
1529 7.7195281846798e-06
1530 7.68955214880407e-06
1531 7.72935709392186e-06
1532 7.66292305343086e-06
1533 7.76748493080959e-06
1534 7.60584134695819e-06
1535 7.88862962508574e-06
1536 7.48994625610067e-06
1537 8.31722991279094e-06
1538 7.4445392783673e-06
1539 1.00716488304897e-05
1540 8.98464259080356e-06
1541 1.38522518682294e-05
1542 1.06641964521259e-05
1543 8.32051591714844e-06
1544 7.41258691050461e-06
1545 8.26258929009782e-06
1546 6.91560035193106e-06
1547 6.7096016209689e-06
1548 8.28910469863331e-06
1549 7.01810495229438e-06
1550 6.49709090794204e-06
1551 7.94054813013645e-06
1552 7.28955865270109e-06
1553 6.57110649626702e-06
1554 7.45564511817065e-06
1555 7.51211473470903e-06
1556 6.73037766318885e-06
1557 7.10938047632226e-06
1558 7.55561859477893e-06
1559 6.88182262820192e-06
1560 6.96574124958715e-06
1561 7.45663828638499e-06
1562 6.98778740115813e-06
1563 6.96210099704331e-06
1564 7.32797070668312e-06
1565 7.04864487488521e-06
1566 7.03252999301185e-06
1567 7.24412166164257e-06
1568 7.10117274138611e-06
1569 7.13710323907435e-06
1570 7.22845197742572e-06
1571 7.19059835319058e-06
1572 7.25384552424657e-06
1573 7.29049133951776e-06
1574 7.32972694095224e-06
1575 7.38200378691545e-06
1576 7.43252257962013e-06
1577 7.49215632822597e-06
1578 7.54787379264599e-06
1579 7.61467481424916e-06
1580 7.67205256124726e-06
1581 7.74314230511663e-06
1582 7.79765923653031e-06
1583 7.87158751336392e-06
1584 7.91594356996939e-06
1585 7.99321333033731e-06
1586 8.0178315329249e-06
1587 8.10319761512801e-06
1588 8.09151060821023e-06
1589 8.19906017568428e-06
1590 8.12317284726305e-06
1591 8.28732208901783e-06
1592 8.0931913544191e-06
1593 8.39440963318339e-06
1594 7.98147266323213e-06
1595 8.60269756230991e-06
1596 7.80132813815726e-06
1597 9.14140218810644e-06
1598 7.77822424424812e-06
1599 1.03941165434662e-05
1600 8.49258685775567e-06
1601 1.08362319224398e-05
1602 9.30798887566198e-06
1603 8.11249446996953e-06
1604 7.79125184635632e-06
1605 8.71102747623809e-06
1606 7.09257983544376e-06
1607 6.86767998558935e-06
1608 9.04394619283266e-06
1609 7.58877513362677e-06
1610 6.75775527270162e-06
1611 8.45463364385068e-06
1612 8.47614956001053e-06
1613 7.21623791832826e-06
1614 7.62075478633051e-06
1615 9.00611757970182e-06
1616 7.9362653195858e-06
1617 7.41307576390682e-06
1618 8.71856536832638e-06
1619 8.69293853611453e-06
1620 7.74391628510784e-06
1621 8.20931927592028e-06
1622 9.10252128960565e-06
1623 8.32287969387835e-06
1624 8.0574845924275e-06
1625 9.03270847629756e-06
1626 8.91359741217457e-06
1627 8.28322663437575e-06
1628 8.79416074894834e-06
1629 9.29665475268848e-06
1630 8.71021802595351e-06
1631 8.69943141879048e-06
1632 9.39211531658657e-06
1633 9.16550379770342e-06
1634 8.81867072166642e-06
1635 9.31308659346541e-06
1636 9.50414960243506e-06
1637 9.07877074496355e-06
1638 9.22667368286056e-06
1639 9.64430000749417e-06
1640 9.37180084292777e-06
1641 9.2199188657105e-06
1642 9.59936460276367e-06
1643 9.58522923610872e-06
1644 9.28188364923699e-06
1645 9.44479052122915e-06
1646 9.62163449003128e-06
1647 9.33836963668e-06
1648 9.2467998911161e-06
1649 9.44016755966004e-06
1650 9.2908157967031e-06
1651 9.02930514712352e-06
1652 9.08802121557528e-06
1653 9.07383855519583e-06
1654 8.7897287812666e-06
1655 8.67925427883165e-06
1656 8.7151365733007e-06
1657 8.52762605063617e-06
1658 8.3207005445729e-06
1659 8.32103432912845e-06
1660 8.25802544568432e-06
1661 8.06014577392489e-06
1662 7.99575445853407e-06
1663 8.00730595074128e-06
1664 7.88670604379149e-06
1665 7.78342291596346e-06
1666 7.80434129410423e-06
1667 7.77054447098635e-06
1668 7.6717660704162e-06
1669 7.66844186728122e-06
1670 7.68937297834782e-06
1671 7.62825357014663e-06
1672 7.59959721108316e-06
1673 7.63720345275942e-06
1674 7.62136323828599e-06
1675 7.58456553739961e-06
1676 7.61327646614518e-06
1677 7.63087336963508e-06
1678 7.60294096835423e-06
1679 7.61498040446895e-06
1680 7.64715150580741e-06
1681 7.63551906857174e-06
1682 7.6339611041476e-06
1683 7.66509310778929e-06
1684 7.66804532759124e-06
1685 7.65942149882903e-06
1686 7.68097925174516e-06
1687 7.69127291277982e-06
1688 7.68111476645572e-06
1689 7.69014786783373e-06
1690 7.70159749663435e-06
1691 7.69118469179375e-06
1692 7.69076723372564e-06
1693 7.69712642068043e-06
1694 7.68878726375988e-06
1695 7.68017343943939e-06
1696 7.68307018006453e-06
1697 7.67340043239528e-06
1698 7.663712494832e-06
1699 7.6591986726271e-06
1700 7.65402364777401e-06
1701 7.63913340051658e-06
1702 7.63904245104641e-06
1703 7.62556601330289e-06
1704 7.62460513215046e-06
1705 7.60686361900298e-06
1706 7.61912133384612e-06
1707 7.58668329581269e-06
1708 7.61874298405019e-06
1709 7.5642965384759e-06
1710 7.63558819016907e-06
1711 7.52598452891107e-06
1712 7.69572852732381e-06
1713 7.46213481761515e-06
1714 7.87467979534995e-06
1715 7.39109191272291e-06
1716 8.46518923935946e-06
1717 7.61816772865131e-06
1718 1.02619569588569e-05
1719 8.97050813364331e-06
1720 1.12139923658106e-05
1721 8.19690831121989e-06
1722 8.40258780954173e-06
1723 7.24209394320496e-06
1724 7.50577873986913e-06
1725 7.42365682526724e-06
1726 7.31103864382021e-06
1727 7.38771086616907e-06
1728 7.40915447750012e-06
1729 7.32786020307685e-06
1730 7.39328606869094e-06
1731 7.41085614208714e-06
1732 7.37807295081438e-06
1733 7.41993017072673e-06
1734 7.4440245043661e-06
1735 7.43452756069019e-06
1736 7.47101148590446e-06
1737 7.493581961171e-06
1738 7.50306435293169e-06
1739 7.53612630433054e-06
1740 7.5603447839967e-06
1741 7.58142505219439e-06
1742 7.61316096031806e-06
1743 7.63982552598463e-06
1744 7.66744324209867e-06
1745 7.69799589761533e-06
1746 7.72621206124313e-06
1747 7.7550748756039e-06
1748 7.78384037403157e-06
1749 7.81054859544383e-06
1750 7.8364910223172e-06
1751 7.8602379289805e-06
1752 7.88213310443098e-06
1753 7.90073863754515e-06
1754 7.91701313573867e-06
1755 7.92935952631524e-06
1756 7.93911476648645e-06
1757 7.94371953816153e-06
1758 7.94649622548604e-06
1759 7.94279003457632e-06
1760 7.9388319136342e-06
1761 7.92715218267404e-06
1762 7.91728052718099e-06
1763 7.89768273534719e-06
1764 7.8845541793271e-06
1765 7.85704105510376e-06
1766 7.84366147854598e-06
1767 7.80739719630219e-06
1768 7.79916172177764e-06
1769 7.7506629168056e-06
1770 7.75674561737105e-06
1771 7.68662448535906e-06
1772 7.72531620896189e-06
1773 7.61227420298383e-06
1774 7.72312978369882e-06
1775 7.52135474613169e-06
1776 7.80182199378032e-06
1777 7.4271229095757e-06
1778 8.13774568086956e-06
1779 7.52379946789006e-06
1780 9.28003737499239e-06
1781 8.63111563376151e-06
1782 1.07346177173895e-05
1783 9.3281114459387e-06
1784 8.15048042568378e-06
1785 7.42616339266533e-06
1786 7.87447788752615e-06
1787 7.02531633578474e-06
1788 7.48360753277666e-06
1789 7.59996328270063e-06
1790 7.21504875400569e-06
1791 7.4536947067827e-06
1792 7.53454105506535e-06
1793 7.34733612262062e-06
1794 7.4982744990848e-06
1795 7.51326206227532e-06
1796 7.4608537943277e-06
1797 7.53797257857514e-06
1798 7.541725608462e-06
1799 7.54700204197434e-06
1800 7.58679107093485e-06
1801 7.59956674301066e-06
1802 7.62490026318119e-06
1803 7.6529395300895e-06
1804 7.67963229009183e-06
1805 7.71050872572232e-06
1806 7.74390082369791e-06
1807 7.77927471062867e-06
1808 7.81777271186002e-06
1809 7.85888278187485e-06
1810 7.90217472967925e-06
1811 7.94774769019568e-06
1812 7.99530153017258e-06
1813 8.04324827186065e-06
1814 8.09301309345756e-06
1815 8.14036138763186e-06
1816 8.18908756627934e-06
1817 8.23209120426327e-06
1818 8.27617714094231e-06
1819 8.31016859592637e-06
1820 8.34563525131671e-06
1821 8.36611070553772e-06
1822 8.39017593534663e-06
1823 8.39263793750433e-06
1824 8.40430129755987e-06
1825 8.38529558677692e-06
1826 8.38619780552108e-06
1827 8.34284219308756e-06
1828 8.33928879728774e-06
1829 8.26875020720763e-06
1830 8.27065105113434e-06
1831 8.16841566120274e-06
1832 8.19244723970769e-06
1833 8.04945102572674e-06
1834 8.1210946518695e-06
1835 7.91978254710557e-06
1836 8.08132972451858e-06
1837 7.79506717663025e-06
1838 8.11964764579898e-06
1839 7.72816201788373e-06
1840 8.32434670883231e-06
1841 7.87624867371051e-06
1842 8.73443241289351e-06
1843 8.37888183014002e-06
1844 8.72559576237109e-06
1845 8.37618063087575e-06
1846 8.01412807049928e-06
1847 7.46862042433349e-06
1848 8.01351507107029e-06
1849 7.29080556993722e-06
1850 7.75413172959816e-06
1851 7.4725990089064e-06
1852 7.57468478695955e-06
1853 7.44388671591878e-06
1854 7.56889858166687e-06
1855 7.41387066227617e-06
1856 7.54259372115484e-06
1857 7.42384190743905e-06
1858 7.51984862290556e-06
1859 7.42410566090257e-06
1860 7.51570360080223e-06
1861 7.42308338885778e-06
1862 7.52474852561136e-06
1863 7.42947122489568e-06
1864 7.55227392801316e-06
1865 7.44727367418818e-06
1866 7.60718057790655e-06
1867 7.48262527849874e-06
1868 7.70236147218384e-06
1869 7.54743041397887e-06
1870 7.85653901402839e-06
1871 7.66331413615262e-06
1872 8.08559343568049e-06
1873 7.86123291618424e-06
1874 8.35277114674682e-06
1875 8.11964946478838e-06
1876 8.47858336783247e-06
1877 8.25731331133284e-06
1878 8.32493697089376e-06
1879 8.1206853792537e-06
1880 8.15570274426136e-06
1881 7.91550337453373e-06
1882 8.10312121757306e-06
1883 7.82800907472847e-06
1884 8.04995397629682e-06
1885 7.79634592618095e-06
1886 7.98815290181665e-06
1887 7.75729677116033e-06
1888 7.92469927546335e-06
1889 7.70756560086738e-06
1890 7.85847714723786e-06
1891 7.65135609981371e-06
1892 7.79158744990127e-06
1893 7.59193153498927e-06
1894 7.72934708948014e-06
1895 7.53321319280076e-06
1896 7.67491110309493e-06
1897 7.4797785600822e-06
1898 7.63202388043283e-06
1899 7.43528289604001e-06
1900 7.6046671892982e-06
1901 7.4040917752427e-06
1902 7.59596969146514e-06
1903 7.39287270334898e-06
1904 7.60798729970702e-06
1905 7.40865516490885e-06
1906 7.63658681535162e-06
1907 7.45596707929508e-06
1908 7.66188441048143e-06
1909 7.52178721086239e-06
1910 7.64391552365851e-06
1911 7.56765712139895e-06
1912 7.55119617679156e-06
1913 7.55413020669948e-06
1914 7.4137833507848e-06
1915 7.48927141103195e-06
1916 7.30007786842179e-06
1917 7.4163481258438e-06
1918 7.23823450243799e-06
1919 7.36695528757991e-06
1920 7.21339438314317e-06
1921 7.34595505491598e-06
1922 7.20876050763763e-06
1923 7.3477363002894e-06
1924 7.21533433534205e-06
1925 7.36855417926563e-06
1926 7.22838331057574e-06
1927 7.40858467906946e-06
1928 7.24590699974215e-06
1929 7.47315107219038e-06
1930 7.26917824067641e-06
1931 7.57943826101837e-06
1932 7.31387399355299e-06
1933 7.77432524046162e-06
1934 7.4416634561203e-06
1935 8.1521329775569e-06
1936 7.80002028477611e-06
1937 8.69576342665823e-06
1938 8.34352340461919e-06
1939 8.64915818965528e-06
1940 8.30223962111631e-06
1941 7.75772787164897e-06
1942 7.84275835030712e-06
1943 7.37534537620377e-06
1944 7.536466455349e-06
1945 7.39118013370899e-06
1946 7.41713165552937e-06
1947 7.38318931325921e-06
1948 7.38703420211095e-06
1949 7.3689516284503e-06
1950 7.373926564469e-06
1951 7.36853098715073e-06
1952 7.37291611585533e-06
1953 7.37609980205889e-06
1954 7.38349626772106e-06
1955 7.39154347684234e-06
1956 7.40261066312087e-06
1957 7.41464236853062e-06
1958 7.42898691896698e-06
1959 7.44470116842422e-06
1960 7.46196337786387e-06
1961 7.48110551285208e-06
1962 7.50175559005584e-06
1963 7.52395226299996e-06
1964 7.54782968215295e-06
1965 7.57325551603572e-06
1966 7.60024477131083e-06
1967 7.62919444241561e-06
1968 7.65915865486022e-06
1969 7.69126563682221e-06
1970 7.72462044551503e-06
1971 7.75951866671676e-06
1972 7.79621404944919e-06
1973 7.83372342993971e-06
1974 7.87291992310202e-06
1975 7.91342154116137e-06
1976 7.95462256064638e-06
1977 7.99741064838599e-06
1978 8.04066348791821e-06
1979 8.08545973995933e-06
1980 8.13023325463291e-06
1981 8.17714499135036e-06
1982 8.22323727334151e-06
1983 8.27256826596567e-06
1984 8.32074147183448e-06
1985 8.37358038552338e-06
1986 8.4246339611127e-06
1987 8.48216859594686e-06
1988 8.53735673445044e-06
1989 8.60158525028965e-06
1990 8.66082154971082e-06
1991 8.73065073392354e-06
1992 8.79033450473798e-06
1993 8.85372082848335e-06
1994 8.89117109181825e-06
1995 8.90182764123892e-06
1996 8.83086340763839e-06
1997 8.65479341882747e-06
1998 8.51393360790098e-06
1999 1.03173078969121e-05
};
\addlegendentry{Test}

\nextgroupplot[
title={Tanh/Tanh},
ymin=2.75521218087108e-06, ymax=0.001,
]
\addplot [semithick, black, dashed]
table {%
0 0.0863556251861155
1 0.0844591404311359
2 0.0825606579892337
3 0.0805823332630098
4 0.078385230852291
5 0.0757820408325642
6 0.0725298142060637
7 0.0682905586436391
8 0.0627629228401929
9 0.0559107209555805
10 0.0481184471864253
11 0.040121047059074
12 0.0327093625674024
13 0.0264376227278262
14 0.0215068263933063
15 0.0178240498062223
16 0.0151406738441437
17 0.0131851125042886
18 0.0117347989871632
19 0.010631689656293
20 0.00977124547353014
21 0.00908601022092625
22 0.00853202637517825
23 0.00807968093431555
24 0.00770805956562981
25 0.00740166207833681
26 0.00714850082295015
27 0.00693900990154361
28 0.0067653901205631
29 0.00662120821652934
30 0.00650112148287008
31 0.00640067327185534
32 0.0063161057405523
33 0.00624407490977319
34 0.0061806790981791
35 0.00611320418465766
36 0.00592160573796718
37 0.00518689633099712
38 0.00355572716944152
39 0.00273756599926855
40 0.00217720900946006
41 0.00160999538456963
42 0.00123668097876362
43 0.00101338730200951
44 0.000886114861714304
45 0.000811175352282589
46 0.000763223199101049
47 0.00072876504418673
48 0.000701749937434215
49 0.000679482216582983
50 0.000660424951092864
51 0.000643668813609111
52 0.000628666514785436
53 0.000615049281805113
54 0.000602562438871246
55 0.00059102451086801
56 0.00058029869433085
57 0.000570279061321344
58 0.000560881194360263
59 0.000552035631699255
60 0.000543684585863957
61 0.000535779168785666
62 0.000528277021658141
63 0.00052114138725301
64 0.00051434025317576
65 0.000507845013089536
66 0.000501630543567444
67 0.000495674317335215
68 0.000489955835746514
69 0.000484456862068328
70 0.000479161258226668
71 0.000474053969810484
72 0.000469121425339836
73 0.000464351546270336
74 0.000459733146726649
75 0.000455256052646291
76 0.000450911237749096
77 0.000446690162334562
78 0.00044258526440899
79 0.000438589586792659
80 0.000434696693446313
81 0.000430900574428961
82 0.000427196261171048
83 0.000423578640948108
84 0.000420043129452097
85 0.000416585759467125
86 0.000413202740901397
87 0.000409890427590653
88 0.0004066456685905
89 0.000403465429371863
90 0.000400346855258249
91 0.000397287334862995
92 0.000394284432559289
93 0.000391335783660907
94 0.000388439079642922
95 0.000385592215479846
96 0.000382793243716151
97 0.000380039898345785
98 0.000377330447008717
99 0.000374662849708329
100 0.000372035098507695
101 0.00036944538123862
102 0.000366891699854932
103 0.000364372151238967
104 0.000361884704489057
105 0.00035942730073657
106 0.000356998009920062
107 0.000354594639361494
108 0.000352215121438348
109 0.000349857181618063
110 0.00034751866382976
111 0.000345197252045182
112 0.000342890563047149
113 0.000340596200089749
114 0.000338311578161665
115 0.00033603416159167
116 0.000333761362639962
117 0.000331490367443621
118 0.000329218395904718
119 0.000326942318451984
120 0.000324659249145043
121 0.00032236591033552
122 0.000320058996408079
123 0.000317735072826508
124 0.000315390580226449
125 0.000313021715271589
126 0.00031062478211652
127 0.000308195959348723
128 0.00030573137604506
129 0.000303227012807383
130 0.000300679262750236
131 0.000298084551559441
132 0.000295439532749242
133 0.000292741173211652
134 0.000289987026178551
135 0.000287175304151788
136 0.000284304843205518
137 0.000281375446718357
138 0.000278387837170158
139 0.000275343927341964
140 0.000272246796725994
141 0.000269100633317976
142 0.000265911106623662
143 0.000262684971318095
144 0.000259430033310082
145 0.000256155206585618
146 0.000252870018243812
147 0.000249584566802241
148 0.000246309167692971
149 0.000243053944615212
150 0.000239828812283349
151 0.000236642791207942
152 0.000233504053227307
153 0.000230419775135715
154 0.000227395769627492
155 0.000224436443488685
156 0.000221545258057176
157 0.000218724149107175
158 0.000215974049780243
159 0.000213295062138741
160 0.000210686432524199
161 0.000208146848393653
162 0.000205674720973548
163 0.000203267974853816
164 0.000200924572283157
165 0.000198642545313987
166 0.000196419806144377
167 0.000194254471182376
168 0.000192144831771657
169 0.000190089275378114
170 0.000188086288517297
171 0.000186134531276139
172 0.000184232828075892
173 0.000182380060209653
174 0.000180575120822368
175 0.000178817023424926
176 0.000177104689697671
177 0.000175437279068547
178 0.000173813691105806
179 0.000172232878952627
180 0.000170693923365661
181 0.000169195746565265
182 0.000167737322982475
183 0.000166317506284486
184 0.000164935196210081
185 0.000163589262285768
186 0.000162278495935197
187 0.000161001824523055
188 0.000159758145457545
189 0.000158546261332049
190 0.000157364931595794
191 0.000156213083698731
192 0.000155089588986357
193 0.000153993350807013
194 0.000152923249117976
195 0.000151878248033199
196 0.000150857247916747
197 0.000149859288598009
198 0.000148883380035159
199 0.000147928494385496
200 0.000146993740258949
201 0.000146078210008227
202 0.00014518101113481
203 0.000144301312275275
204 0.000143438180970179
205 0.000142590953799981
206 0.000141758828306138
207 0.000140941019850516
208 0.000140136885818265
209 0.000139345677666824
210 0.000138566693806297
211 0.000137799387744053
212 0.000137043022448324
213 0.000136297095025384
214 0.000135561005635054
215 0.00013483421250271
216 0.000134116085348523
217 0.000133406153196347
218 0.000132704004641937
219 0.000132009111439402
220 0.000131320942955426
221 0.000130639092276397
222 0.000129963122248
223 0.000129292575422824
224 0.000128627134102999
225 0.000127966341750607
226 0.000127309826979172
227 0.000126657171648503
228 0.000126008115472587
229 0.000125362342259905
230 0.00012471945134962
231 0.000124079152413969
232 0.00012344107943818
233 0.000122804966821377
234 0.0001221705432215
235 0.000121537539087058
236 0.00012090574378476
237 0.000120274806619136
238 0.000119644471311631
239 0.000119014603995993
240 0.00011838493844607
241 0.000117755276534126
242 0.000117125307042443
243 0.000116494916710508
244 0.000115863891295476
245 0.000115232109834551
246 0.000114599375791613
247 0.000113965487230416
248 0.000113330376876775
249 0.000112693925501617
250 0.000112055973744418
251 0.000111416399192876
252 0.000110775187323497
253 0.000110132290700449
254 0.000109487585405077
255 0.000108841091275735
256 0.000108192858021994
257 0.000107542871518262
258 0.00010689114022
259 0.000106237791669628
260 0.00010558293467966
261 0.000104926683292206
262 0.000104269189591832
263 0.00010361054636121
264 0.000102950924940615
265 0.000102290605042299
266 0.000101629722735197
267 0.000100968407494406
268 0.000100306975554076
269 9.96455592883194e-05
270 9.89842535830121e-05
271 9.83232468598771e-05
272 9.76627533049168e-05
273 9.70028484630348e-05
274 9.63436659873196e-05
275 9.56852993994062e-05
276 9.50277731277538e-05
277 9.43712274050768e-05
278 9.37157699354429e-05
279 9.30613666696445e-05
280 9.24081849831282e-05
281 9.17562129814087e-05
282 9.11056608430272e-05
283 9.04564678450015e-05
284 8.98088539571518e-05
285 8.9162915529073e-05
286 8.85187931487508e-05
287 8.78766373517692e-05
288 8.7236554392689e-05
289 8.65987174734073e-05
290 8.59634160406131e-05
291 8.53307137873571e-05
292 8.47008141846572e-05
293 8.40739807017599e-05
294 8.34503427000755e-05
295 8.28301394619757e-05
296 8.22135261984158e-05
297 8.16007958093223e-05
298 8.09921004503167e-05
299 8.03876173733897e-05
300 7.97876196401148e-05
301 7.91922360150465e-05
302 7.86016842653225e-05
303 7.80160753350856e-05
304 7.74356578148172e-05
305 7.68604833751851e-05
306 7.62907881295405e-05
307 7.57266391246958e-05
308 7.51682494097849e-05
309 7.46156377857687e-05
310 7.40690258282939e-05
311 7.35284306188078e-05
312 7.2993903543761e-05
313 7.24654978938588e-05
314 7.19432648992324e-05
315 7.14271293418278e-05
316 7.0917178177865e-05
317 7.04133508691029e-05
318 6.991569676984e-05
319 6.94241916505689e-05
320 6.8938958918352e-05
321 6.84597745674864e-05
322 6.79868901016789e-05
323 6.75201012683146e-05
324 6.70593501581607e-05
325 6.66045889516909e-05
326 6.61556054382118e-05
327 6.57122573954894e-05
328 6.52742415070406e-05
329 6.48414473829462e-05
330 6.44135812422064e-05
331 6.39906059092255e-05
332 6.35724332056498e-05
333 6.31591722850544e-05
334 6.27509578094987e-05
335 6.23480690222777e-05
336 6.19508206085584e-05
337 6.15596326127843e-05
338 6.1174877458825e-05
339 6.0796992613632e-05
340 6.04262208412365e-05
341 6.00625029534285e-05
342 5.9704921568482e-05
343 5.93517306697322e-05
344 5.89998104345568e-05
345 5.8645582129202e-05
346 5.82865418152778e-05
347 5.79242242935152e-05
348 5.75682509520448e-05
349 5.72398593448042e-05
350 5.69743149583246e-05
351 5.68214791769606e-05
352 5.68420215500964e-05
353 5.70861937632117e-05
354 5.75187969218405e-05
355 5.78676548457224e-05
356 5.75834027358724e-05
357 5.6467380844083e-05
358 5.59727582469804e-05
359 5.95002151442259e-05
360 6.78422846220883e-05
361 6.98666045479968e-05
362 6.64568877937199e-05
363 6.72929763112506e-05
364 6.06742365505397e-05
365 5.60392213770911e-05
366 5.4735556631158e-05
367 5.37047922506417e-05
368 5.28704389779477e-05
369 5.15544678023616e-05
370 5.12838097534996e-05
371 5.0990743005741e-05
372 5.06343186117419e-05
373 5.01760245015248e-05
374 4.98487212752252e-05
375 4.96645034928633e-05
376 4.94302555154036e-05
377 4.91509022566561e-05
378 4.8857662235946e-05
379 4.86094823344274e-05
380 4.83963290918155e-05
381 4.81718797971098e-05
382 4.79297625446407e-05
383 4.76812924148362e-05
384 4.74434430230986e-05
385 4.72214182849484e-05
386 4.70036305841859e-05
387 4.67814442259851e-05
388 4.6554797393128e-05
389 4.63280660056853e-05
390 4.61065517285419e-05
391 4.58914256427079e-05
392 4.56795061865023e-05
393 4.54675120522552e-05
394 4.5254415617535e-05
395 4.50411369143922e-05
396 4.48296824373529e-05
397 4.46216175831182e-05
398 4.4416891221033e-05
399 4.42141957961439e-05
400 4.40122351221817e-05
401 4.38103900393116e-05
402 4.36087644999361e-05
403 4.34082057694241e-05
404 4.32097001166198e-05
405 4.30137544000786e-05
406 4.28201190771915e-05
407 4.26281063141687e-05
408 4.24369892400023e-05
409 4.22462409872537e-05
410 4.20557032754232e-05
411 4.18657162057912e-05
412 4.16770314899395e-05
413 4.1490361915919e-05
414 4.13059890433942e-05
415 4.11236768229628e-05
416 4.09429039294196e-05
417 4.07629883625305e-05
418 4.05831131118362e-05
419 4.04024440499029e-05
420 4.02205595406713e-05
421 4.00378514910926e-05
422 3.98555479605989e-05
423 3.96749857074497e-05
424 3.94969182551108e-05
425 3.93214731673197e-05
426 3.91485877386799e-05
427 3.89786295471595e-05
428 3.88114622396074e-05
429 3.86448566587205e-05
430 3.84736890524096e-05
431 3.82921144961301e-05
432 3.80983511902855e-05
433 3.78993289942287e-05
434 3.77099046886542e-05
435 3.75440690163487e-05
436 3.74015373623138e-05
437 3.72613582797499e-05
438 3.70994131628777e-05
439 3.69358426439703e-05
440 3.68828860715098e-05
441 3.71354079788944e-05
442 3.78570964514324e-05
443 3.89351975798036e-05
444 3.96600119501045e-05
445 3.90308106901216e-05
446 3.76618133870465e-05
447 3.8800163814301e-05
448 4.39775362934824e-05
449 4.62416903914686e-05
450 4.38455476370336e-05
451 4.57381444078919e-05
452 4.38848718005147e-05
453 3.85717493145421e-05
454 3.76783619771004e-05
455 3.69178461028241e-05
456 3.61371750088324e-05
457 3.52930026679132e-05
458 3.4187366338756e-05
459 3.39325541816038e-05
460 3.3841397538481e-05
461 3.35798328734427e-05
462 3.32389303707714e-05
463 3.28796342543569e-05
464 3.27186346638086e-05
465 3.26336191136534e-05
466 3.24943225251673e-05
467 3.22984305185514e-05
468 3.20728143421434e-05
469 3.18864627928406e-05
470 3.17527592628153e-05
471 3.16265290294382e-05
472 3.14809925328063e-05
473 3.13109749185969e-05
474 3.11299949373733e-05
475 3.09620809630928e-05
476 3.08137143889553e-05
477 3.06730488119911e-05
478 3.05280680521491e-05
479 3.03725717571979e-05
480 3.02077724896321e-05
481 3.00425635089141e-05
482 2.98851039701731e-05
483 2.97357143850263e-05
484 2.95895006843239e-05
485 2.94411943215778e-05
486 2.92870765292719e-05
487 2.91269901993019e-05
488 2.89651249545386e-05
489 2.88066894427175e-05
490 2.86537107427876e-05
491 2.8504709852939e-05
492 2.83567343899449e-05
493 2.82063826730905e-05
494 2.80509324070977e-05
495 2.7890378547113e-05
496 2.77281506697591e-05
497 2.75687827411275e-05
498 2.74145238776669e-05
499 2.72650895780657e-05
500 2.71187446685417e-05
501 2.69726844379647e-05
502 2.68226990485232e-05
503 2.66650471019148e-05
504 2.65002321029328e-05
505 2.63337494033067e-05
506 2.61720544827426e-05
507 2.6018079864798e-05
508 2.58714876011368e-05
509 2.57324771055778e-05
510 2.56014683372996e-05
511 2.5473377199603e-05
512 2.53345572218677e-05
513 2.51713669356946e-05
514 2.49855596337056e-05
515 2.47987869563815e-05
516 2.46358084972087e-05
517 2.44995475036092e-05
518 2.43686718270908e-05
519 2.42334642237552e-05
520 2.41384287207325e-05
521 2.41670383474002e-05
522 2.43425010815201e-05
523 2.45256677402494e-05
524 2.4456251892957e-05
525 2.40243365503545e-05
526 2.35904582090996e-05
527 2.38800874079459e-05
528 2.51924741654364e-05
529 2.63526574393325e-05
530 2.58125249317231e-05
531 2.58152903711562e-05
532 2.97930722084061e-05
533 3.16525381052202e-05
534 2.7314516657384e-05
535 2.68524434829942e-05
536 2.79815829387076e-05
537 2.64226442041604e-05
538 2.60087165742107e-05
539 2.45975162478373e-05
540 2.25563732421108e-05
541 2.23910512389125e-05
542 2.2255964736928e-05
543 2.19509609316049e-05
544 2.14279307293452e-05
545 2.08169209088283e-05
546 2.06764948522675e-05
547 2.06897169690023e-05
548 2.06159409046336e-05
549 2.04385916653393e-05
550 2.01643331010359e-05
551 1.99810289487345e-05
552 1.99194785892587e-05
553 1.98685097458906e-05
554 1.9787904772528e-05
555 1.96544418074041e-05
556 1.94899611365429e-05
557 1.93612942034349e-05
558 1.92764745960972e-05
559 1.9203513851096e-05
560 1.91226034118586e-05
561 1.90177060162e-05
562 1.88899387936203e-05
563 1.87670717437527e-05
564 1.8666973769399e-05
565 1.85828741621208e-05
566 1.85035005202394e-05
567 1.84180555145019e-05
568 1.83155469848373e-05
569 1.81990819534406e-05
570 1.80860982084141e-05
571 1.79877866095524e-05
572 1.79020979764744e-05
573 1.7823853958987e-05
574 1.77457450689644e-05
575 1.76557846813807e-05
576 1.75480098683067e-05
577 1.74330594155947e-05
578 1.73270461445441e-05
579 1.72353260410318e-05
580 1.71554336816371e-05
581 1.70863722068759e-05
582 1.70224356423887e-05
583 1.69453090244076e-05
584 1.6839432614546e-05
585 1.67148319185628e-05
586 1.65990148452977e-05
587 1.65055282828064e-05
588 1.64263372361972e-05
589 1.63602697824672e-05
590 1.63269658166598e-05
591 1.63284345333636e-05
592 1.63095352689879e-05
593 1.62006450281638e-05
594 1.60164580087496e-05
595 1.58720047735983e-05
596 1.58561530767543e-05
597 1.58998826584877e-05
598 1.58497980216055e-05
599 1.57582823341329e-05
600 1.60058334657265e-05
601 1.68367868482733e-05
602 1.76930409168108e-05
603 1.7539679387113e-05
604 1.66215495269739e-05
605 1.69796894553542e-05
606 1.91874505475198e-05
607 2.00786513886442e-05
608 1.89189125734401e-05
609 2.08711295144326e-05
610 2.26993705894074e-05
611 1.90263456794071e-05
612 1.88093052990723e-05
613 1.89250039852595e-05
614 1.79679356619999e-05
615 1.78795254335284e-05
616 1.62324696866278e-05
617 1.51903850333923e-05
618 1.53323403040417e-05
619 1.52635170174165e-05
620 1.51817328166715e-05
621 1.46858518519366e-05
622 1.42739122210855e-05
623 1.42990704254942e-05
624 1.43393144043102e-05
625 1.43390017264267e-05
626 1.41970101807942e-05
627 1.39808372310313e-05
628 1.39057962655897e-05
629 1.39074208824752e-05
630 1.39087616410905e-05
631 1.38773871576348e-05
632 1.37735451772869e-05
633 1.36628619422652e-05
634 1.360765569558e-05
635 1.3584381882481e-05
636 1.35688999947092e-05
637 1.35382454296007e-05
638 1.34675745790958e-05
639 1.33805284967536e-05
640 1.33160584177006e-05
641 1.32763401836655e-05
642 1.32503883882862e-05
643 1.32276511521923e-05
644 1.31853268783289e-05
645 1.31136641634555e-05
646 1.30364171226915e-05
647 1.29767392529345e-05
648 1.29349277386837e-05
649 1.29072593324508e-05
650 1.28885874612905e-05
651 1.28566814492359e-05
652 1.27918577632613e-05
653 1.27090937578345e-05
654 1.26391641899204e-05
655 1.25899740783808e-05
656 1.255527399735e-05
657 1.25417074468004e-05
658 1.25488599920232e-05
659 1.25372994528306e-05
660 1.24664798004659e-05
661 1.2361571142705e-05
662 1.22887791214055e-05
663 1.22628605794262e-05
664 1.22382939853338e-05
665 1.22150502264162e-05
666 1.22833815598966e-05
667 1.24724747099236e-05
668 1.26050733353367e-05
669 1.24817383113651e-05
670 1.22341901995782e-05
671 1.22652650276223e-05
672 1.26871323615774e-05
673 1.29993164286191e-05
674 1.27802730940374e-05
675 1.28866502322467e-05
676 1.46027765133283e-05
677 1.64746164905694e-05
678 1.56597626901345e-05
679 1.47458498869213e-05
680 1.70825359546711e-05
681 1.76788797077165e-05
682 1.61269510172701e-05
683 1.76148946344767e-05
684 1.68035724854576e-05
685 1.45101428081773e-05
686 1.48759120826014e-05
687 1.4027662032845e-05
688 1.39552831939227e-05
689 1.32908077787874e-05
690 1.22445624253231e-05
691 1.23093058483903e-05
692 1.22928777486209e-05
693 1.22598454055378e-05
694 1.19191665071128e-05
695 1.15574586558864e-05
696 1.15612901474549e-05
697 1.15989871929401e-05
698 1.16044383311475e-05
699 1.14880780976989e-05
700 1.13186723318393e-05
701 1.1269454006424e-05
702 1.12762506567776e-05
703 1.12836619585721e-05
704 1.12552170001834e-05
705 1.11688978101654e-05
706 1.10948238933872e-05
707 1.10654490370621e-05
708 1.10559756691941e-05
709 1.10499007206499e-05
710 1.10209263493033e-05
711 1.09609641008035e-05
712 1.09033347861498e-05
713 1.08674890597626e-05
714 1.08473917013185e-05
715 1.0836001580472e-05
716 1.08169445827144e-05
717 1.0773723770896e-05
718 1.07177160142413e-05
719 1.06712957350652e-05
720 1.06398149490872e-05
721 1.06211118300337e-05
722 1.06118042708658e-05
723 1.05948967625835e-05
724 1.05535409691271e-05
725 1.04982581703439e-05
726 1.04523221047259e-05
727 1.04216657295098e-05
728 1.04033528476322e-05
729 1.04026884439889e-05
730 1.04129136691e-05
731 1.04005947427765e-05
732 1.03478089652498e-05
733 1.02866939464263e-05
734 1.02541240032394e-05
735 1.02394864267552e-05
736 1.02220499051953e-05
737 1.02358203193376e-05
738 1.03242350917299e-05
739 1.04245209495701e-05
740 1.04054526630648e-05
741 1.02745397452253e-05
742 1.0225240208328e-05
743 1.03728289051475e-05
744 1.05339284228734e-05
745 1.04660366240239e-05
746 1.04266945868403e-05
747 1.10841584586296e-05
748 1.22998851850298e-05
749 1.2675249486449e-05
750 1.18994640931902e-05
751 1.23275054910721e-05
752 1.45124589892021e-05
753 1.49135074245521e-05
754 1.38835228895573e-05
755 1.62592248607041e-05
756 1.64772012780645e-05
757 1.38732679886289e-05
758 1.47879855560973e-05
759 1.32674328412463e-05
760 1.29255858958288e-05
761 1.28246605513027e-05
762 1.13248205817484e-05
763 1.13089185518334e-05
764 1.10363899139543e-05
765 1.10102467658635e-05
766 1.06816924834163e-05
767 1.02404317878779e-05
768 1.02603387368561e-05
769 1.02625000515388e-05
770 1.02520790079552e-05
771 1.0056471140274e-05
772 9.88433844462122e-06
773 9.87715734979133e-06
774 9.8860224255759e-06
775 9.87853094258639e-06
776 9.79171193193906e-06
777 9.69179342469317e-06
778 9.65772814076615e-06
779 9.65051678747386e-06
780 9.64661259139632e-06
781 9.61459637771611e-06
782 9.55061976881666e-06
783 9.50045302872127e-06
784 9.47496468484132e-06
785 9.46239076426991e-06
786 9.45159998622103e-06
787 9.42149395299907e-06
788 9.37362714559242e-06
789 9.33174982264973e-06
790 9.30407712118608e-06
791 9.28798698929256e-06
792 9.27819377771755e-06
793 9.25879706947796e-06
794 9.22196013064536e-06
795 9.18201523170126e-06
796 9.1521621161661e-06
797 9.13342109321036e-06
798 9.12603460889727e-06
799 9.12392599872192e-06
800 9.10918808294525e-06
801 9.07588677456772e-06
802 9.04086874342624e-06
803 9.01684990850526e-06
804 9.00166473449815e-06
805 8.99858454062041e-06
806 9.0127678618046e-06
807 9.02362310739591e-06
808 9.00171791329285e-06
809 8.95765757036315e-06
810 8.92836915511452e-06
811 8.91735836994201e-06
812 8.90170536571588e-06
813 8.90003584430588e-06
814 8.965251602433e-06
815 9.07313225262385e-06
816 9.10389562847058e-06
817 9.01835820954489e-06
818 8.96111624193452e-06
819 9.06360314800736e-06
820 9.20645682711552e-06
821 9.16515387316963e-06
822 9.09993643283258e-06
823 9.58064212142062e-06
824 1.06069950245313e-05
825 1.10725342175044e-05
826 1.05378437940118e-05
827 1.08259287205925e-05
828 1.26851279453888e-05
829 1.33071632362203e-05
830 1.23642881888486e-05
831 1.44503542358621e-05
832 1.54856159593209e-05
833 1.28522314533086e-05
834 1.4103485337813e-05
835 1.26251208811823e-05
836 1.19349544647207e-05
837 1.21625748015219e-05
838 1.05410141681972e-05
839 1.04773102194144e-05
840 9.93608360388976e-06
841 9.87171603838988e-06
842 9.64476035747452e-06
843 9.20963047690293e-06
844 9.18157051899016e-06
845 9.10759089478574e-06
846 9.12661158025685e-06
847 8.93832251236404e-06
848 8.80834793903773e-06
849 8.79293074085297e-06
850 8.78798899783817e-06
851 8.78168008178193e-06
852 8.68004022258617e-06
853 8.60709359429279e-06
854 8.58347925536407e-06
855 8.57511813734391e-06
856 8.56768267176733e-06
857 8.51377495436623e-06
858 8.45407301675039e-06
859 8.4210866759804e-06
860 8.40336307561529e-06
861 8.39726818391284e-06
862 8.37479699988819e-06
863 8.32972230568885e-06
864 8.2913918380001e-06
865 8.26731540826131e-06
866 8.25632761980444e-06
867 8.25347467081627e-06
868 8.23689943185713e-06
869 8.204116338284e-06
870 8.17517595308459e-06
871 8.15659644359812e-06
872 8.14886947608784e-06
873 8.15266944620419e-06
874 8.15064802672794e-06
875 8.12852634624051e-06
876 8.10057708555689e-06
877 8.08066120505657e-06
878 8.06750784043686e-06
879 8.06754245807895e-06
880 8.08348771919043e-06
881 8.0894030300982e-06
882 8.06612207782109e-06
883 8.03504377699937e-06
884 8.01606938694022e-06
885 7.99985215138577e-06
886 7.99252380190296e-06
887 8.02642898101169e-06
888 8.08653945760796e-06
889 8.09909715826507e-06
890 8.04938947496225e-06
891 8.01721635745878e-06
892 8.03760536172149e-06
893 8.03640574886799e-06
894 7.99674637796954e-06
895 8.08526529283426e-06
896 8.40041055560903e-06
897 8.66941164545665e-06
898 8.58992817676096e-06
899 8.45451663167296e-06
900 8.79864170677536e-06
901 9.37452291971397e-06
902 9.33464254337935e-06
903 9.12702470579063e-06
904 1.05039008442276e-05
905 1.2224414231099e-05
906 1.14591186388679e-05
907 1.17749246015819e-05
908 1.36909252645978e-05
909 1.24924738464927e-05
910 1.26692497950387e-05
911 1.37456687774318e-05
912 1.11528839106256e-05
913 1.16084241703618e-05
914 1.03160879962871e-05
915 9.98170783717001e-06
916 9.90463738581582e-06
917 9.01648458739857e-06
918 8.90714393264602e-06
919 8.51397883394611e-06
920 8.56900184054155e-06
921 8.30724760891854e-06
922 8.12642665248831e-06
923 8.02582227699133e-06
924 7.98240843558062e-06
925 7.99568136411466e-06
926 7.84476956727787e-06
927 7.78545727708746e-06
928 7.73474683146702e-06
929 7.74357816979432e-06
930 7.73475031490278e-06
931 7.6593011657522e-06
932 7.62181972646658e-06
933 7.5941778447941e-06
934 7.60186351556058e-06
935 7.60418146228403e-06
936 7.56306763882719e-06
937 7.53209025639023e-06
938 7.51083741867831e-06
939 7.50872096588751e-06
940 7.52218321053988e-06
941 7.50734084853377e-06
942 7.47804865319779e-06
943 7.45663323975521e-06
944 7.44158763943403e-06
945 7.44806260755126e-06
946 7.46091285996897e-06
947 7.44752589998399e-06
948 7.42169327949682e-06
949 7.40053608705438e-06
950 7.38402751032652e-06
951 7.38889080587057e-06
952 7.41195337283784e-06
953 7.41334510756531e-06
954 7.38738897343438e-06
955 7.36252857169717e-06
956 7.34124913659429e-06
957 7.32745872866758e-06
958 7.35205389901239e-06
959 7.40105525309787e-06
960 7.40837388057969e-06
961 7.37087468571929e-06
962 7.34616677267041e-06
963 7.33309353506684e-06
964 7.30365913348408e-06
965 7.32478126685976e-06
966 7.46311475019468e-06
967 7.59276698047984e-06
968 7.56160892390056e-06
969 7.49323287418946e-06
970 7.57329052092359e-06
971 7.66436137666915e-06
972 7.57452217925447e-06
973 7.63430992201108e-06
974 8.27710458217723e-06
975 8.9171639654495e-06
976 8.749618320536e-06
977 8.72451975730826e-06
978 9.70770662789278e-06
979 1.00998589154244e-05
980 9.56012852171284e-06
981 1.08981111281992e-05
982 1.21173233651461e-05
983 1.06785340303617e-05
984 1.1876066555061e-05
985 1.14130920589872e-05
986 1.06043750776053e-05
987 1.14704176255032e-05
988 9.84363729017446e-06
989 9.80556820451284e-06
990 8.98069956534187e-06
991 8.66610556826686e-06
992 8.72514363514654e-06
993 8.06737358693965e-06
994 7.97644649708218e-06
995 7.60806700839112e-06
996 7.6920955951465e-06
997 7.53919052698393e-06
998 7.39166078034259e-06
999 7.27616144935439e-06
1000 7.21430351013908e-06
1001 7.27653868537459e-06
1002 7.16978547110614e-06
1003 7.11974011835537e-06
1004 7.05038280557346e-06
1005 7.05854595750566e-06
1006 7.0888983687567e-06
1007 7.03460887763185e-06
1008 7.00014456267439e-06
1009 6.95931582317399e-06
1010 6.96575247083331e-06
1011 6.9954379160464e-06
1012 6.97017811823741e-06
1013 6.93953009367476e-06
1014 6.90945278236654e-06
1015 6.89783398843957e-06
1016 6.92690405301732e-06
1017 6.93375149918296e-06
1018 6.90699199790146e-06
1019 6.88106972379643e-06
1020 6.85456860161082e-06
1021 6.85536631284123e-06
1022 6.89020885857872e-06
1023 6.89876215087537e-06
1024 6.872385271528e-06
1025 6.84642479775732e-06
1026 6.81837351734771e-06
1027 6.8103801504904e-06
1028 6.85620893392525e-06
1029 6.8988138632875e-06
1030 6.88057041919876e-06
1031 6.84808462292352e-06
1032 6.82657501727135e-06
1033 6.79176640439039e-06
1034 6.80332003799577e-06
1035 6.9175401371524e-06
1036 7.00846370982333e-06
1037 6.97210909539336e-06
1038 6.93668334328379e-06
1039 6.96778266462417e-06
1040 6.93234673043719e-06
1041 6.89570316936283e-06
1042 7.15537714679471e-06
1043 7.56028051540625e-06
1044 7.58680738677242e-06
1045 7.47273041756813e-06
1046 7.78899301678848e-06
1047 8.04446261604141e-06
1048 7.78232001152901e-06
1049 8.1933094477904e-06
1050 9.54881062043e-06
1051 9.50895558027298e-06
1052 9.38190162003139e-06
1053 1.06137411641782e-05
1054 1.00776870013419e-05
1055 1.00795289483813e-05
1056 1.13294271137931e-05
1057 9.72307357827162e-06
1058 1.01288042912984e-05
1059 9.41526562669992e-06
1060 8.9297784384712e-06
1061 9.25672499718644e-06
1062 8.21568860942534e-06
1063 8.19167921406461e-06
1064 7.54542364944655e-06
1065 7.59356753121665e-06
1066 7.49344773431915e-06
1067 7.20250751484741e-06
1068 7.05202990225473e-06
1069 6.87347981553899e-06
1070 6.98527158604634e-06
1071 6.84738579792565e-06
1072 6.77639095592042e-06
1073 6.65645344444243e-06
1074 6.65440075042767e-06
1075 6.70885834264112e-06
1076 6.63566263181536e-06
1077 6.5925843735215e-06
1078 6.52704649084512e-06
1079 6.54614089512506e-06
1080 6.58663779162794e-06
1081 6.54885853634823e-06
1082 6.51522124606174e-06
1083 6.4719603036778e-06
1084 6.47366031003571e-06
1085 6.51958146580256e-06
1086 6.51144874108667e-06
1087 6.48062117747372e-06
1088 6.44847664155179e-06
1089 6.42206614998031e-06
1090 6.45507647423216e-06
1091 6.49349210046779e-06
1092 6.47557511346264e-06
1093 6.44753651712904e-06
1094 6.41455742744768e-06
1095 6.39109306277419e-06
1096 6.43797188182305e-06
1097 6.49740686942124e-06
1098 6.48437478201203e-06
1099 6.45358564632659e-06
1100 6.42724983546117e-06
1101 6.38366850047589e-06
1102 6.41649808486022e-06
1103 6.54836170710382e-06
1104 6.6010698951402e-06
1105 6.55262622917263e-06
1106 6.55112892999909e-06
1107 6.53790341864635e-06
1108 6.46692237094015e-06
1109 6.59744280095964e-06
1110 6.94280673663172e-06
1111 7.03831253545673e-06
1112 6.94196627382127e-06
1113 7.11749094328695e-06
1114 7.23549193004658e-06
1115 7.03406608870694e-06
1116 7.4162312113657e-06
1117 8.37658302543787e-06
1118 8.33046086512468e-06
1119 8.35161507239945e-06
1120 9.214995433382e-06
1121 8.81004300712362e-06
1122 8.90649166507274e-06
1123 1.02414763887282e-05
1124 9.22717844442644e-06
1125 9.63601548420456e-06
1126 9.46859673156553e-06
1127 8.6947589359454e-06
1128 9.36028867570826e-06
1129 8.38723058294022e-06
1130 8.29394448942367e-06
1131 7.65438336269852e-06
1132 7.48979853071319e-06
1133 7.58065012540499e-06
1134 7.08362079038238e-06
1135 6.9802680311426e-06
1136 6.63627965291624e-06
1137 6.75926690973938e-06
1138 6.65492343099316e-06
1139 6.53107964332378e-06
1140 6.38650936757301e-06
1141 6.32888947205146e-06
1142 6.42180693066763e-06
1143 6.33733586408169e-06
1144 6.28458490048445e-06
1145 6.19392626433068e-06
1146 6.20789358585228e-06
1147 6.26744601106566e-06
1148 6.22101245983941e-06
1149 6.18380731953749e-06
1150 6.12611801908969e-06
1151 6.13576168362329e-06
1152 6.19476124086304e-06
1153 6.17588724960427e-06
1154 6.14455251835011e-06
1155 6.10183974103506e-06
1156 6.08195931040001e-06
1157 6.14166552281858e-06
1158 6.16720176260799e-06
1159 6.13760360046456e-06
1160 6.10974065473613e-06
1161 6.06349944742846e-06
1162 6.07387270168402e-06
1163 6.15787237379095e-06
1164 6.17732459939901e-06
1165 6.14525232922603e-06
1166 6.12197428839778e-06
1167 6.06732618457784e-06
1168 6.07689556719748e-06
1169 6.21465839056867e-06
1170 6.28093953558562e-06
1171 6.23890991757747e-06
1172 6.24069498833535e-06
1173 6.20110198212842e-06
1174 6.13751756617376e-06
1175 6.33208674205576e-06
1176 6.61747460029005e-06
1177 6.59998858054678e-06
1178 6.5979194365795e-06
1179 6.75146685047423e-06
1180 6.62706142584568e-06
1181 6.6451917426491e-06
1182 7.36551628399695e-06
1183 7.68388997407499e-06
1184 7.50183506514901e-06
1185 8.07839393068832e-06
1186 8.08182169897975e-06
1187 7.7561922608993e-06
1188 8.92503149785284e-06
1189 8.94855802968664e-06
1190 8.62670155399314e-06
1191 9.30393103182325e-06
1192 8.29596081253214e-06
1193 8.88681349309195e-06
1194 8.87907079771821e-06
1195 8.03167398544247e-06
1196 8.05009330662187e-06
1197 7.30647516533622e-06
1198 7.63028934791521e-06
1199 7.21165572326044e-06
1200 7.01932011715201e-06
1201 6.62254838879051e-06
1202 6.54705294422797e-06
1203 6.63781233356531e-06
1204 6.38267370689505e-06
1205 6.28761620014018e-06
1206 6.08467894380738e-06
1207 6.19405654855854e-06
1208 6.16057217683874e-06
1209 6.07247735562311e-06
1210 5.97350666442509e-06
1211 5.92520258546614e-06
1212 6.02171295671639e-06
1213 5.98241516147624e-06
1214 5.93762895473304e-06
1215 5.86596642637716e-06
1216 5.85484643256251e-06
1217 5.93556792871652e-06
1218 5.91742653988092e-06
1219 5.88391831701784e-06
1220 5.83146331400997e-06
1221 5.80694405538651e-06
1222 5.88509432297712e-06
1223 5.90374138198513e-06
1224 5.87165288390423e-06
1225 5.83863673231377e-06
1226 5.78343782553503e-06
1227 5.83054278013861e-06
1228 5.91830493235079e-06
1229 5.90350733098433e-06
1230 5.88116454824927e-06
1231 5.83660700215205e-06
1232 5.78607816770216e-06
1233 5.891293332283e-06
1234 6.01047808146404e-06
1235 5.98411137886501e-06
1236 5.97850214045792e-06
1237 5.94311778456813e-06
1238 5.85904675087789e-06
1239 6.02528927284851e-06
1240 6.28246747780281e-06
1241 6.25953289468129e-06
1242 6.28363830657008e-06
1243 6.35839243212644e-06
1244 6.18616402370264e-06
1245 6.36370815598752e-06
1246 6.97328269794184e-06
1247 6.98539780508156e-06
1248 7.03823801018189e-06
1249 7.43338963893336e-06
1250 7.07598954718947e-06
1251 7.33862667168239e-06
1252 8.34176051256463e-06
1253 7.85439762562845e-06
1254 8.34858825626839e-06
1255 8.38630539767848e-06
1256 7.76173845906669e-06
1257 8.71306010097328e-06
1258 8.11645729381638e-06
1259 8.07196232166518e-06
1260 7.74030675465553e-06
1261 7.31805470621794e-06
1262 7.70876942546295e-06
1263 7.03770547083593e-06
1264 7.02142864383859e-06
1265 6.47111415563728e-06
1266 6.57990732122471e-06
1267 6.55179160347075e-06
1268 6.29979354194177e-06
1269 6.12069572270002e-06
1270 5.95905909683836e-06
1271 6.11232067093681e-06
1272 5.97702160831659e-06
1273 5.90970509861677e-06
1274 5.74561570587662e-06
1275 5.78616856650171e-06
1276 5.85774921457727e-06
1277 5.77173671256759e-06
1278 5.72112200547537e-06
1279 5.62629045131047e-06
1280 5.70480891504133e-06
1281 5.74361962968339e-06
1282 5.69169192932151e-06
1283 5.64956313464648e-06
1284 5.5795332842834e-06
1285 5.65353549397685e-06
1286 5.70651907061404e-06
1287 5.66733642415329e-06
1288 5.63928806229796e-06
1289 5.5651934065537e-06
1290 5.60905846302973e-06
1291 5.7112766791434e-06
1292 5.68950252688438e-06
1293 5.67278446883535e-06
1294 5.61080709804429e-06
1295 5.57137269474595e-06
1296 5.71487848954888e-06
1297 5.78845500953662e-06
1298 5.75283447723507e-06
1299 5.7569771740873e-06
1300 5.65897872206733e-06
1301 5.67013811814121e-06
1302 5.93195347509123e-06
1303 5.99806066770014e-06
1304 5.97127465162828e-06
1305 6.04232199474808e-06
1306 5.89055356847723e-06
1307 5.94837939971171e-06
1308 6.4455975632427e-06
1309 6.50102609167647e-06
1310 6.51962583297916e-06
1311 6.77627039458173e-06
1312 6.46351847422011e-06
1313 6.70830069893213e-06
1314 7.51315608304992e-06
1315 7.19460068232536e-06
1316 7.61655757486324e-06
1317 7.64166622424511e-06
1318 7.1500334577479e-06
1319 8.11111055298852e-06
1320 7.75948260045567e-06
1321 7.75638573990989e-06
1322 7.67496085707364e-06
1323 7.12496370880444e-06
1324 7.71872155347353e-06
1325 7.10146210636253e-06
1326 7.1258783158612e-06
1327 6.56757197781133e-06
1328 6.59355401566586e-06
1329 6.70064697771977e-06
1330 6.3142656223647e-06
1331 6.19080964092689e-06
1332 5.89771926406968e-06
1333 6.09594199829644e-06
1334 5.95036250938819e-06
1335 5.8648175667031e-06
1336 5.65054405399223e-06
1337 5.67620608826758e-06
1338 5.77434009763067e-06
1339 5.65582933198527e-06
1340 5.59675519529179e-06
1341 5.47012080609477e-06
1342 5.58237850167842e-06
1343 5.60176779274002e-06
1344 5.54238362227011e-06
1345 5.47635674408298e-06
1346 5.4105519708969e-06
1347 5.53102783662496e-06
1348 5.53702204797446e-06
1349 5.50067063276316e-06
1350 5.44188062256978e-06
1351 5.38247499282107e-06
1352 5.50637797225306e-06
1353 5.53930812863257e-06
1354 5.50592151515872e-06
1355 5.47095682179588e-06
1356 5.3798553256712e-06
1357 5.48970753122546e-06
1358 5.60376910652138e-06
1359 5.56352191072307e-06
1360 5.5705939256967e-06
1361 5.46429215741995e-06
1362 5.46890493602348e-06
1363 5.71383557002036e-06
1364 5.73842749584941e-06
1365 5.73595454689269e-06
1366 5.74856209389907e-06
1367 5.5767396975881e-06
1368 5.78149341645684e-06
1369 6.12496002716512e-06
1370 6.04314942442841e-06
1371 6.19756251740711e-06
1372 6.14800114284719e-06
1373 5.94698961542406e-06
1374 6.55306351671214e-06
1375 6.75212836931394e-06
1376 6.65819901435327e-06
1377 7.05180625804047e-06
1378 6.56524997744157e-06
1379 6.93465410250838e-06
1380 7.559579508154e-06
1381 6.99794447278634e-06
1382 7.55630454118972e-06
1383 6.91888020298848e-06
1384 7.16298397129833e-06
1385 7.54774273659109e-06
1386 6.87499294649285e-06
1387 7.08833280071985e-06
1388 6.43017347634611e-06
1389 6.8187215571669e-06
1390 6.63789826926831e-06
1391 6.43504342789925e-06
1392 6.11338842748665e-06
1393 6.00447871867971e-06
1394 6.20732637202082e-06
1395 5.91495676260934e-06
1396 5.84997494179618e-06
1397 5.56997596934394e-06
1398 5.74622987237206e-06
1399 5.70473582905962e-06
1400 5.61027967016514e-06
1401 5.45079419023153e-06
1402 5.41980540891274e-06
1403 5.56970593112993e-06
1404 5.46938512169248e-06
1405 5.43457770874056e-06
1406 5.29035902285813e-06
1407 5.37696908686769e-06
1408 5.45604258572752e-06
1409 5.38679295480904e-06
1410 5.34724474388071e-06
1411 5.23839332888798e-06
1412 5.36126475392251e-06
1413 5.41714164814522e-06
1414 5.37023255375857e-06
1415 5.33416384307372e-06
1416 5.22783865886112e-06
1417 5.3623663767155e-06
1418 5.44752003417415e-06
1419 5.4030346743339e-06
1420 5.39529777299208e-06
1421 5.26166895298275e-06
1422 5.37378312426284e-06
1423 5.55556047032724e-06
1424 5.50169439605952e-06
1425 5.5448465756669e-06
1426 5.4108482556714e-06
1427 5.40241772739591e-06
1428 5.7397328534492e-06
1429 5.73762185851479e-06
1430 5.78037593612635e-06
1431 5.78659253314129e-06
1432 5.5593286090172e-06
1433 5.95348225918002e-06
1434 6.21501331288243e-06
1435 6.10021796543236e-06
1436 6.37068836706334e-06
1437 6.02307074881736e-06
1438 6.22239348224696e-06
1439 6.84162501229935e-06
1440 6.48558487759132e-06
1441 6.91706324928987e-06
1442 6.53220585711267e-06
1443 6.55408135230573e-06
1444 7.21655910851382e-06
1445 6.6365007000968e-06
1446 7.02859356671581e-06
1447 6.4256731047152e-06
1448 6.66446029917722e-06
1449 6.89447156076994e-06
1450 6.42031438502499e-06
1451 6.43666783339825e-06
1452 5.98424160180855e-06
1453 6.35397905135449e-06
1454 6.10604757422806e-06
1455 6.04530090253519e-06
1456 5.68236325282356e-06
1457 5.74903585004449e-06
1458 5.87385840322696e-06
1459 5.65641209071543e-06
1460 5.55565477267095e-06
1461 5.36811609475052e-06
1462 5.58110757786778e-06
1463 5.48323800497741e-06
1464 5.43795068619346e-06
1465 5.24875030638583e-06
1466 5.30366861983822e-06
1467 5.42765934863354e-06
1468 5.32423091126333e-06
1469 5.28372509478459e-06
1470 5.14133681051021e-06
1471 5.29517078451391e-06
1472 5.32723336688079e-06
1473 5.27754890633503e-06
1474 5.2056703019332e-06
1475 5.12008720754409e-06
1476 5.30705303170009e-06
1477 5.30380305452383e-06
1478 5.28702569013717e-06
1479 5.20231095180179e-06
1480 5.1313506714834e-06
1481 5.3521529155276e-06
1482 5.35400072410042e-06
1483 5.35251990463337e-06
1484 5.28290935264408e-06
1485 5.17178363512727e-06
1486 5.44386844580913e-06
1487 5.49599385735178e-06
1488 5.48558967849999e-06
1489 5.48439822001257e-06
1490 5.28014336342864e-06
1491 5.5871734385704e-06
1492 5.77226222375771e-06
1493 5.69852285448036e-06
1494 5.83631832729026e-06
1495 5.53049249596427e-06
1496 5.79760726360234e-06
1497 6.19290415926343e-06
1498 5.98711779931094e-06
1499 6.28474393771228e-06
1500 5.89996083188993e-06
1501 6.09950217977939e-06
1502 6.6245861702896e-06
1503 6.2456094624963e-06
1504 6.62068800671989e-06
1505 6.09356299641206e-06
1506 6.40069508150987e-06
1507 6.7233918148446e-06
1508 6.31288896180138e-06
1509 6.48188768437308e-06
1510 5.95158371829996e-06
1511 6.42951698459626e-06
1512 6.25584698088488e-06
1513 6.18658644491177e-06
1514 5.85603526559453e-06
1515 5.82375234436938e-06
1516 6.08547233760248e-06
1517 5.76286805387127e-06
1518 5.73450111129858e-06
1519 5.4133318361238e-06
1520 5.67843229859477e-06
1521 5.58413516049683e-06
1522 5.51816165472729e-06
1523 5.277219594646e-06
1524 5.32052591051269e-06
1525 5.47509542503377e-06
1526 5.32711100653671e-06
1527 5.27269068140157e-06
1528 5.09982664365083e-06
1529 5.30345663118936e-06
1530 5.27771952052447e-06
1531 5.23909301719527e-06
1532 5.09670975912258e-06
1533 5.08979144786537e-06
1534 5.28041197256357e-06
1535 5.19330312265254e-06
1536 5.19450704938862e-06
1537 5.02430153259681e-06
1538 5.13338824514875e-06
1539 5.27431633301489e-06
1540 5.19831414846905e-06
1541 5.19663862252884e-06
1542 5.02301828175789e-06
1543 5.20558515848535e-06
1544 5.32375631845383e-06
1545 5.2658225264679e-06
1546 5.26993019089872e-06
1547 5.07522426884677e-06
1548 5.3172694682857e-06
1549 5.45375884630772e-06
1550 5.39653311992794e-06
1551 5.43914857775718e-06
1552 5.19060483927802e-06
1553 5.48976336389728e-06
1554 5.6815718449954e-06
1555 5.59527497756562e-06
1556 5.7122908385665e-06
1557 5.38043989273262e-06
1558 5.74528509922345e-06
1559 5.98306977828145e-06
1560 5.84299536132704e-06
1561 6.02251628478712e-06
1562 5.59464469773729e-06
1563 6.07203364122455e-06
1564 6.21044615201072e-06
1565 6.07928643781008e-06
1566 6.15102423040526e-06
1567 5.73930944280931e-06
1568 6.3208440685969e-06
1569 6.11964524876996e-06
1570 6.19265775814171e-06
1571 5.87159735765397e-06
1572 5.84917443369193e-06
1573 6.22649199399916e-06
1574 5.84190282726382e-06
1575 5.91956436313268e-06
1576 5.49228406576674e-06
1577 5.83415410382315e-06
1578 5.75055308438266e-06
1579 5.67251052707718e-06
1580 5.3900598313561e-06
1581 5.41228882511291e-06
1582 5.62121061520315e-06
1583 5.40844416008213e-06
1584 5.3555154595486e-06
1585 5.12822175036121e-06
1586 5.38415466078135e-06
1587 5.30505521290081e-06
1588 5.27332029065519e-06
1589 5.05623794833099e-06
1590 5.12918493633663e-06
1591 5.28430555135628e-06
1592 5.16449364873495e-06
1593 5.12158373755511e-06
1594 4.95475974737758e-06
1595 5.17948642375643e-06
1596 5.16552425278149e-06
1597 5.14348538604992e-06
1598 5.0012520347309e-06
1599 4.98609066212907e-06
1600 5.21811747056233e-06
1601 5.12747345160491e-06
1602 5.15453024618751e-06
1603 4.9561614683391e-06
1604 5.07814794392658e-06
1605 5.26447474857861e-06
1606 5.17420646906075e-06
1607 5.20191354347332e-06
1608 4.98051543784683e-06
1609 5.21620394966504e-06
1610 5.3529269532504e-06
1611 5.29080079836319e-06
1612 5.30545037946695e-06
1613 5.06580322312544e-06
1614 5.41344997362359e-06
1615 5.49333424171294e-06
1616 5.47456250021838e-06
1617 5.46016470970301e-06
1618 5.21085434801449e-06
1619 5.67945757001809e-06
1620 5.65137590768927e-06
1621 5.71416004735426e-06
1622 5.59876654726565e-06
1623 5.41697539624408e-06
1624 5.96386780848945e-06
1625 5.74142265641342e-06
1626 5.93551363348155e-06
1627 5.59019027779328e-06
1628 5.68089881181066e-06
1629 6.08672860913373e-06
1630 5.75465555296972e-06
1631 5.91608224453921e-06
1632 5.45951344044227e-06
1633 5.8805642488835e-06
1634 5.8449635109703e-06
1635 5.78000922946131e-06
1636 5.53009709403085e-06
1637 5.48178538561928e-06
1638 5.80967072494332e-06
1639 5.52179929780294e-06
1640 5.54352580728334e-06
1641 5.20169983619923e-06
1642 5.51574459617399e-06
1643 5.44310310246487e-06
1644 5.39903575003109e-06
1645 5.13078842745429e-06
1646 5.20273560944418e-06
1647 5.38622311374581e-06
1648 5.22084422183156e-06
1649 5.15597702221982e-06
1650 4.97259000376005e-06
1651 5.24033986781802e-06
1652 5.1539863248351e-06
1653 5.14652524064019e-06
1654 4.92139643126777e-06
1655 5.03384362460224e-06
1656 5.18196288368244e-06
1657 5.07605910993192e-06
1658 5.02511515687587e-06
1659 4.86634023033616e-06
1660 5.12787700035489e-06
1661 5.0895788010763e-06
1662 5.09440183993348e-06
1663 4.91600920993207e-06
1664 4.94766296021965e-06
1665 5.19106675156422e-06
1666 5.08046697911269e-06
1667 5.11879484754729e-06
1668 4.88326159908681e-06
1669 5.09903577317061e-06
1670 5.23429381082963e-06
1671 5.16702160613391e-06
1672 5.14605330703688e-06
1673 4.93923026745335e-06
1674 5.29451248798551e-06
1675 5.2875204441527e-06
1676 5.32264991015552e-06
1677 5.18058905640828e-06
1678 5.08820509015351e-06
1679 5.51340419452373e-06
1680 5.3617329593969e-06
1681 5.50683132694019e-06
1682 5.20353176369781e-06
1683 5.33621113785188e-06
1684 5.68830390790254e-06
1685 5.47233589376361e-06
1686 5.61608064852948e-06
1687 5.22050622908665e-06
1688 5.63359430394428e-06
1689 5.68697934610896e-06
1690 5.63737244796414e-06
1691 5.49992275100664e-06
1692 5.34267875451633e-06
1693 5.80120902426984e-06
1694 5.52621968230582e-06
1695 5.65752603520764e-06
1696 5.23853014788855e-06
1697 5.54032507427138e-06
1698 5.62771906409409e-06
1699 5.49843875319311e-06
1700 5.3240713882019e-06
1701 5.22233573718722e-06
1702 5.55705795513006e-06
1703 5.32284530407878e-06
1704 5.34579682032188e-06
1705 5.02201388119516e-06
1706 5.31849545382101e-06
1707 5.28800259669282e-06
1708 5.24426034242964e-06
1709 4.99779364737662e-06
1710 5.06078458517578e-06
1711 5.26298199510222e-06
1712 5.10833191658122e-06
1713 5.05342325229918e-06
1714 4.86593948689773e-06
1715 5.14864485623434e-06
1716 5.06692821078403e-06
1717 5.06986611803484e-06
1718 4.83259335704034e-06
1719 4.9653499294422e-06
1720 5.11572366157509e-06
1721 5.01460048418778e-06
1722 4.95104658160272e-06
1723 4.80092366217377e-06
1724 5.0899727961351e-06
1725 5.02239873112487e-06
1726 5.04813673529725e-06
1727 4.82678564628358e-06
1728 4.93041631255409e-06
1729 5.14406205187612e-06
1730 5.03306967392092e-06
1731 5.03967966114516e-06
1732 4.81077642922045e-06
1733 5.12120587181641e-06
1734 5.13798842316504e-06
1735 5.14560113806795e-06
1736 4.99166705125731e-06
1737 4.93661216260932e-06
1738 5.30003694798609e-06
1739 5.15617301166671e-06
1740 5.26347681617523e-06
1741 4.95404008571398e-06
1742 5.17959567414294e-06
1743 5.39533914345469e-06
1744 5.27353655410678e-06
1745 5.28641262675933e-06
1746 5.01027257993059e-06
1747 5.45247749794697e-06
1748 5.36631000835897e-06
1749 5.44659492973665e-06
1750 5.16862277866181e-06
1751 5.23290699927514e-06
1752 5.58481530177346e-06
1753 5.35078324492844e-06
1754 5.43347932602245e-06
1755 5.0716549964136e-06
1756 5.48325357652146e-06
1757 5.42389978885893e-06
1758 5.43951218912042e-06
1759 5.14716949417959e-06
1760 5.22281999693064e-06
1761 5.50203228577573e-06
1762 5.28202468341732e-06
1763 5.26291026670123e-06
1764 4.99701220935123e-06
1765 5.36197019496143e-06
1766 5.22381095446889e-06
1767 5.24937887691479e-06
1768 4.92423457831848e-06
1769 5.13148598901836e-06
1770 5.24055682937785e-06
1771 5.13076674835133e-06
1772 4.97223725481888e-06
1773 4.90732784808046e-06
1774 5.1881308573698e-06
1775 5.02872475971472e-06
1776 5.02644844413425e-06
1777 4.77211045746273e-06
1778 5.04722908711841e-06
1779 5.03215290681425e-06
1780 5.0129750448491e-06
1781 4.7930980002775e-06
1782 4.86571232105604e-06
1783 5.07375015956768e-06
1784 4.95075816253632e-06
1785 4.91418665138355e-06
1786 4.72455524391791e-06
1787 5.02381634692739e-06
1788 4.96922041204328e-06
1789 4.99060990710731e-06
1790 4.75716220460498e-06
1791 4.87105138358146e-06
1792 5.07887630174508e-06
1793 4.96839125307957e-06
1794 4.94742317425079e-06
1795 4.73984097926206e-06
1796 5.07376428249273e-06
1797 5.02973577409804e-06
1798 5.07042876085606e-06
1799 4.8368331087012e-06
1800 4.91506254274299e-06
1801 5.19616811711643e-06
1802 5.05339634937485e-06
1803 5.09477788845913e-06
1804 4.81045956490789e-06
1805 5.17417888534766e-06
1806 5.17403173905251e-06
1807 5.19830352896378e-06
1808 4.98473002341626e-06
1809 4.98962749340137e-06
1810 5.34736030743943e-06
1811 5.15519902899442e-06
1812 5.24047842187514e-06
1813 4.89626703492618e-06
1814 5.27040483433439e-06
1815 5.28013716039055e-06
1816 5.27951733353405e-06
1817 5.04348671181987e-06
1818 5.06067159733448e-06
1819 5.39544933175762e-06
1820 5.18227188273102e-06
1821 5.21835490552292e-06
1822 4.9035552107668e-06
1823 5.28707853320043e-06
1824 5.20234631151695e-06
1825 5.23022276999185e-06
1826 4.91153989212734e-06
1827 5.07704773955453e-06
1828 5.26121996724527e-06
1829 5.11636442634966e-06
1830 5.00198582376044e-06
1831 4.8695900813911e-06
1832 5.19823248978923e-06
1833 5.03032301057971e-06
1834 5.0490082594834e-06
1835 4.75321749915381e-06
1836 5.04236253195955e-06
1837 5.0387326182566e-06
1838 5.012768441226e-06
1839 4.76916856495535e-06
1840 4.858916963002e-06
1841 5.06082808193753e-06
1842 4.93124240374954e-06
1843 4.85845428777054e-06
1844 4.70150710896178e-06
1845 5.0080930042995e-06
1846 4.90295132582474e-06
1847 4.9251434863784e-06
1848 4.6572343626039e-06
1849 4.87907409851829e-06
1850 4.96750615308628e-06
1851 4.91292648607811e-06
1852 4.76084321654469e-06
1853 4.72216882307919e-06
1854 5.01267061370214e-06
1855 4.88345564075843e-06
1856 4.90862994695362e-06
1857 4.6474882711145e-06
1858 4.93897627640294e-06
1859 4.96689757767399e-06
1860 4.96165769758861e-06
1861 4.76364062773627e-06
1862 4.78553174865226e-06
1863 5.07917778502787e-06
1864 4.93806021717447e-06
1865 4.96856844733884e-06
1866 4.69491341492656e-06
1867 5.0394477444371e-06
1868 5.03168514986996e-06
1869 5.05663518524102e-06
1870 4.81741773494804e-06
1871 4.88218248761996e-06
1872 5.17843485159375e-06
1873 5.01930435437004e-06
1874 5.03993631983946e-06
1875 4.76223770018436e-06
1876 5.15082105323472e-06
1877 5.08142402511069e-06
1878 5.13524665635146e-06
1879 4.83061947553765e-06
1880 4.98975026275161e-06
1881 5.21675432896274e-06
1882 5.07397103355345e-06
1883 5.00553111493218e-06
1884 4.81932196194634e-06
1885 5.20290617789243e-06
1886 5.04089602859636e-06
1887 5.10113660245892e-06
1888 4.75949019262956e-06
1889 5.05697479846745e-06
1890 5.10644891615897e-06
1891 5.06272642253336e-06
1892 4.82673016843904e-06
1893 4.87183572417393e-06
1894 5.13243612676462e-06
1895 4.96793163762277e-06
1896 4.9178921495141e-06
1897 4.71375374644367e-06
1898 5.05055670707577e-06
1899 4.92974452992456e-06
1900 4.95309975301339e-06
1901 4.65305190333609e-06
1902 4.90705922473467e-06
1903 4.96237083069673e-06
1904 4.91566399407617e-06
1905 4.70173654010253e-06
1906 4.74560224184017e-06
1907 4.98260967596309e-06
1908 4.8486292714145e-06
1909 4.79676224873771e-06
1910 4.61358033998494e-06
1911 4.92901484783204e-06
1912 4.83711056809e-06
1913 4.8609768210639e-06
1914 4.5856697266089e-06
1915 4.81132897389713e-06
1916 4.9037143110553e-06
1917 4.85290211749145e-06
1918 4.68464703207161e-06
1919 4.66803933552029e-06
1920 4.95208808004577e-06
1921 4.8208127494398e-06
1922 4.82661832990061e-06
1923 4.57961346489455e-06
1924 4.90050968871003e-06
1925 4.8756694708274e-06
1926 4.8990707415264e-06
1927 4.64290395729705e-06
1928 4.7707805843622e-06
1929 4.98889271138836e-06
1930 4.87778338698064e-06
1931 4.81913226568764e-06
1932 4.64265177768652e-06
1933 5.00580891849012e-06
1934 4.8926215114875e-06
1935 4.95510203579741e-06
1936 4.64162011981628e-06
1937 4.90110911899322e-06
1938 5.01444802925022e-06
1939 4.96125094517907e-06
1940 4.79199327330093e-06
1941 4.75008689271306e-06
1942 5.08609224780798e-06
1943 4.92024701781446e-06
1944 4.95180101456327e-06
1945 4.65667767590716e-06
1946 5.01764827642859e-06
1947 4.97170530966429e-06
1948 5.0019434869597e-06
1949 4.69958070503651e-06
1950 4.87228101286874e-06
1951 5.0600768668474e-06
1952 4.94583670729298e-06
1953 4.82001165558188e-06
1954 4.71558726111709e-06
1955 5.05264710604081e-06
1956 4.88990476199547e-06
1957 4.9085980062813e-06
1958 4.61828156073096e-06
1959 4.94891447466728e-06
1960 4.90804890063146e-06
1961 4.91728037443906e-06
1962 4.62466680906104e-06
1963 4.80459322815818e-06
1964 4.95275114076321e-06
1965 4.85973432029496e-06
1966 4.7012357811127e-06
1967 4.65517158509954e-06
1968 4.94367896575199e-06
1969 4.79928670449326e-06
1970 4.78075495369268e-06
1971 4.54797182269928e-06
1972 4.86595344728613e-06
1973 4.79994269353412e-06
1974 4.81643440908286e-06
1975 4.5349966497632e-06
1976 4.74679689732582e-06
1977 4.85290546947681e-06
1978 4.79401942854452e-06
1979 4.61604992718634e-06
1980 4.61304174592669e-06
1981 4.88246000251991e-06
1982 4.75348910855544e-06
1983 4.72927408878832e-06
1984 4.51419509506223e-06
1985 4.8384979960403e-06
1986 4.7726175078644e-06
1987 4.80228556254403e-06
1988 4.52098006276103e-06
1989 4.73377504706107e-06
1990 4.85775190828264e-06
1991 4.79994809099438e-06
1992 4.63959324292773e-06
1993 4.60554209613662e-06
1994 4.90978756584326e-06
1995 4.77445035151902e-06
1996 4.78320371977148e-06
1997 4.52546232887485e-06
1998 4.86516308750851e-06
1999 4.82433575710317e-06
};
\addlegendentry{Train}
\addplot [semithick, black]
table {%
0 0.0877174139022827
1 0.0858125761151314
2 0.0838708952069283
3 0.0817864090204239
4 0.079384408891201
5 0.0764589458703995
6 0.0727067887783051
7 0.0677694529294968
8 0.0614425018429756
9 0.0538996569812298
10 0.0457488931715488
11 0.037818718701601
12 0.0308203399181366
13 0.0251312460750341
14 0.0207843221724033
15 0.0175832323729992
16 0.0152506222948432
17 0.0135333519428968
18 0.0122409630566835
19 0.0112435985356569
20 0.0104564605280757
21 0.00982447527348995
22 0.00931101944297552
23 0.00889074150472879
24 0.00854521710425615
25 0.00826046243309975
26 0.00802547018975019
27 0.00783135835081339
28 0.00767083559185266
29 0.00753787206485868
30 0.00742744421586394
31 0.00733533455058932
32 0.00725789554417133
33 0.00719148106873035
34 0.00712931109592319
35 0.00702374335378408
36 0.0065206871367991
37 0.00477918982505798
38 0.00306714023463428
39 0.00224807346239686
40 0.00165068358182907
41 0.00130375649314374
42 0.00107261538505554
43 0.000912621500901878
44 0.000809349410701543
45 0.000744456658139825
46 0.000700216391123831
47 0.000667886633891612
48 0.000643133884295821
49 0.000623260042630136
50 0.000606707355473191
51 0.000592552532907575
52 0.000580166408326477
53 0.000569122319575399
54 0.000559127889573574
55 0.000549975840840489
56 0.000541516405064613
57 0.000533638813067228
58 0.000526258256286383
59 0.000519310007803142
60 0.000512743135914207
61 0.000506515556480736
62 0.000500593450851738
63 0.000494948122650385
64 0.000489554891828448
65 0.000484392629005015
66 0.000479442474897951
67 0.000474687461974099
68 0.000470112892799079
69 0.000465705466922373
70 0.000461452465970069
71 0.000457342743175104
72 0.000453366548754275
73 0.00044951427844353
74 0.000445777317509055
75 0.000442148389993235
76 0.000438620132626966
77 0.000435186055256054
78 0.000431840308010578
79 0.000428577041020617
80 0.000425391510361806
81 0.000422279525082558
82 0.000419236166635528
83 0.000416257476899773
84 0.000413340429076925
85 0.00041048153070733
86 0.000407677463954315
87 0.000404925231123343
88 0.0004022226203233
89 0.000399566866690293
90 0.000396955787437037
91 0.000394386937841773
92 0.000391858164221048
93 0.000389367778552696
94 0.000386913714464754
95 0.000384494138415903
96 0.000382107100449502
97 0.000379751058062539
98 0.000377424090402201
99 0.000375124684069306
100 0.000372851121937856
101 0.000370601570466533
102 0.000368374254321679
103 0.000366167660104111
104 0.00036398001248017
105 0.000361809448804706
106 0.000359654310159385
107 0.000357512617483735
108 0.000355382595444098
109 0.000353262235876173
110 0.000351149647030979
111 0.000349042791640386
112 0.000346939661540091
113 0.000344837928423658
114 0.000342735613230616
115 0.000340630096616223
116 0.000338519108481705
117 0.00033640005858615
118 0.000334270414896309
119 0.000332127296132967
120 0.000329967588186264
121 0.000327788380673155
122 0.000325586326653138
123 0.00032335813739337
124 0.000321100204018876
125 0.00031880900496617
126 0.000316480814944953
127 0.000314112025080249
128 0.000311699026497081
129 0.000309238530462608
130 0.000306727102724835
131 0.000304162007523701
132 0.000301540858345106
133 0.000298861821647733
134 0.00029612350044772
135 0.000293325691018254
136 0.000290468684397638
137 0.000287553877569735
138 0.000284583627944812
139 0.000281561224255711
140 0.000278491119388491
141 0.000275378843070939
142 0.000272230507107452
143 0.00026905327104032
144 0.000265855022007599
145 0.000262643821770325
146 0.000259428430581465
147 0.000256217317655683
148 0.000253019097726792
149 0.000249841861659661
150 0.000246693030931056
151 0.00024357957590837
152 0.000240507361013442
153 0.000237481348449364
154 0.000234505481785163
155 0.000231582773267291
156 0.000228715216508135
157 0.000225903961108997
158 0.000223149181692861
159 0.000220450965571217
160 0.000217808425077237
161 0.000215220730751753
162 0.000212686485610902
163 0.000210204350878485
164 0.000207773278816603
165 0.000205391595955007
166 0.000203058327315375
167 0.000200772512471303
168 0.000198533030925319
169 0.000196339125977829
170 0.000194189968169667
171 0.00019208513549529
172 0.000190024089533836
173 0.00018800632096827
174 0.000186031305929646
175 0.000184098767931573
176 0.000182208314072341
177 0.000180359493242577
178 0.000178551825229079
179 0.000176784800714813
180 0.000175057997694239
181 0.000173370994161814
182 0.0001717231789371
183 0.000170113824424334
184 0.000168542348546907
185 0.000167008052812889
186 0.000165510238730349
187 0.000164048295118846
188 0.000162621436174959
189 0.000161228774231859
190 0.000159869436174631
191 0.000158542738063261
192 0.000157247864990495
193 0.000155983929289505
194 0.000154750028741546
195 0.000153545435750857
196 0.000152369277202524
197 0.000151220679981634
198 0.000150098727317527
199 0.000149002706166357
200 0.000147931612445973
201 0.000146884776768275
202 0.000145861267810687
203 0.000144860212458298
204 0.000143880810355768
205 0.000142922319355421
206 0.000141983909998089
207 0.000141064738272689
208 0.000140164091135375
209 0.000139281008159742
210 0.000138414878165349
211 0.000137564755277708
212 0.000136730057420209
213 0.000135910013341345
214 0.000135103866341524
215 0.000134310670546256
216 0.000133529843878932
217 0.000132760789711028
218 0.00013200263492763
219 0.000131254812004045
220 0.000130516520584933
221 0.000129787207697518
222 0.000129066043882631
223 0.000128352505271323
224 0.000127645980683155
225 0.000126945829833858
226 0.000126251426991075
227 0.000125562160974368
228 0.000124877726193517
229 0.0001241973368451
230 0.00012352064368315
231 0.00012284709373489
232 0.000122176279546693
233 0.00012150783004472
234 0.000120841214084066
235 0.000120176213386003
236 0.000119512471428607
237 0.000118849762657192
238 0.000118187737825792
239 0.000117526287795044
240 0.000116865310701542
241 0.000116204471851233
242 0.000115543865831569
243 0.000114883398055099
244 0.000114222930278629
245 0.000113562688056845
246 0.000112902627734002
247 0.000112242982140742
248 0.000111583729449194
249 0.000110925131593831
250 0.00011026729043806
251 0.000109610504296143
252 0.00010895498417085
253 0.000108300933788996
254 0.000107648629636969
255 0.000106998282717541
256 0.000106350154965185
257 0.000105704421002883
258 0.000105061277281493
259 0.000104420803836547
260 0.000103783131635282
261 0.000103148311609402
262 0.000102516278275289
263 0.000101887068012729
264 0.000101260469818953
265 0.000100636323622894
266 0.000100014447525609
267 9.93946159724146e-05
268 9.87765015452169e-05
269 9.81598059297539e-05
270 9.75442890194245e-05
271 9.69297034316696e-05
272 9.63156999205239e-05
273 9.57020529313013e-05
274 9.50884932535701e-05
275 9.44748462643474e-05
276 9.38610173761845e-05
277 9.32467883103527e-05
278 9.26321154111065e-05
279 9.20169040909968e-05
280 9.14011325221509e-05
281 9.07847643247806e-05
282 9.01678649825044e-05
283 8.95503908395767e-05
284 8.89324874151498e-05
285 8.8314205640927e-05
286 8.76956182764843e-05
287 8.70768271852285e-05
288 8.64580506458879e-05
289 8.58393614180386e-05
290 8.5220905020833e-05
291 8.46028997329995e-05
292 8.39855201775208e-05
293 8.33689045975916e-05
294 8.27532421681099e-05
295 8.21388166514225e-05
296 8.15258317743428e-05
297 8.09144621598534e-05
298 8.03049842943437e-05
299 7.96976164565422e-05
300 7.90925478213467e-05
301 7.84899602876976e-05
302 7.78901448938996e-05
303 7.729326171102e-05
304 7.66995290177874e-05
305 7.61091723688878e-05
306 7.55224464228377e-05
307 7.49394894228317e-05
308 7.43606433388777e-05
309 7.37859809305519e-05
310 7.32156840967946e-05
311 7.26498474250548e-05
312 7.20886455383152e-05
313 7.15321293682791e-05
314 7.09804371581413e-05
315 7.04335761838593e-05
316 6.98917574482039e-05
317 6.93550828145817e-05
318 6.88236905261874e-05
319 6.82977915857919e-05
320 6.77774514770135e-05
321 6.72628302709199e-05
322 6.6754080762621e-05
323 6.62510428810492e-05
324 6.57536875223741e-05
325 6.52617673040368e-05
326 6.47749038762413e-05
327 6.42927407170646e-05
328 6.38148630969226e-05
329 6.33411400485784e-05
330 6.28712441539392e-05
331 6.24055610387586e-05
332 6.19446291239001e-05
333 6.14895325270481e-05
334 6.10415008850396e-05
335 6.06025096203666e-05
336 6.01743486186024e-05
337 5.97588405071292e-05
338 5.93572585785296e-05
339 5.89699448028114e-05
340 5.85953093832359e-05
341 5.82294742343947e-05
342 5.78649742237758e-05
343 5.7490986364428e-05
344 5.70933552808128e-05
345 5.66564849577844e-05
346 5.61660272069275e-05
347 5.56121922272723e-05
348 5.49936339666601e-05
349 5.43204878340475e-05
350 5.36189836566336e-05
351 5.29367789567914e-05
352 5.23424605489708e-05
353 5.18873675900977e-05
354 5.14983039465733e-05
355 5.09660458192229e-05
356 5.06652395415585e-05
357 5.28683885931969e-05
358 6.03480912104715e-05
359 6.79805234540254e-05
360 6.02980726398528e-05
361 5.81675267312676e-05
362 6.4058622228913e-05
363 4.97146902489476e-05
364 5.6186556321336e-05
365 5.33858874405269e-05
366 5.03852825204376e-05
367 4.89967969770078e-05
368 4.80739909107797e-05
369 4.9165897507919e-05
370 4.80748458357994e-05
371 4.7349003580166e-05
372 4.67384015792049e-05
373 4.68779217044357e-05
374 4.68082544102799e-05
375 4.63192445749883e-05
376 4.58527720184065e-05
377 4.5570224756375e-05
378 4.54812397947535e-05
379 4.53099892183673e-05
380 4.49885374109726e-05
381 4.46484009444248e-05
382 4.43831995653454e-05
383 4.41994197899476e-05
384 4.4017822801834e-05
385 4.37799499195535e-05
386 4.35063702752814e-05
387 4.32442284363788e-05
388 4.30189793405589e-05
389 4.28213752456941e-05
390 4.26219521614257e-05
391 4.24019344791304e-05
392 4.21664699388202e-05
393 4.19326606788673e-05
394 4.17138289776631e-05
395 4.15116373915225e-05
396 4.13172383559868e-05
397 4.11194123444147e-05
398 4.09127496823203e-05
399 4.07000516133849e-05
400 4.04886595788412e-05
401 4.0285329305334e-05
402 4.00925928261131e-05
403 3.99079799535684e-05
404 3.97257135773543e-05
405 3.95405222661793e-05
406 3.93503250961658e-05
407 3.91570101783145e-05
408 3.89650922443252e-05
409 3.87795189453755e-05
410 3.86033534596208e-05
411 3.84363120247144e-05
412 3.82749058189802e-05
413 3.8113856135169e-05
414 3.7948448152747e-05
415 3.77765099983662e-05
416 3.75993586203549e-05
417 3.74211376765743e-05
418 3.72474278265145e-05
419 3.70834350178484e-05
420 3.69319677702151e-05
421 3.67918473784812e-05
422 3.66576787200756e-05
423 3.65205560228787e-05
424 3.63711988029536e-05
425 3.62029022653587e-05
426 3.60137673851568e-05
427 3.58078250428662e-05
428 3.55940137524158e-05
429 3.5385357477935e-05
430 3.51978596881963e-05
431 3.50493646692485e-05
432 3.49567890225444e-05
433 3.49289111909457e-05
434 3.49557703884784e-05
435 3.49983783962671e-05
436 3.49904184986372e-05
437 3.48617941199336e-05
438 3.45807711710222e-05
439 3.41906779794954e-05
440 3.38149984600022e-05
441 3.36214106937405e-05
442 3.36693992721848e-05
443 3.35698459821288e-05
444 3.27141242451034e-05
445 3.2578689570073e-05
446 3.70703382941429e-05
447 4.38806418969762e-05
448 4.13450834457763e-05
449 3.78931981686037e-05
450 4.29120082117151e-05
451 3.61481761501636e-05
452 3.31773044308648e-05
453 3.8104408304207e-05
454 3.56035452568904e-05
455 3.31290466419887e-05
456 3.16358491545543e-05
457 3.08810449496377e-05
458 3.24579414154869e-05
459 3.19961181958206e-05
460 3.09355964418501e-05
461 3.02442549582338e-05
462 3.01399995805696e-05
463 3.05529465549625e-05
464 3.053605905734e-05
465 3.01311538351001e-05
466 2.9729109883192e-05
467 2.95288791676285e-05
468 2.95515474135755e-05
469 2.95719491987256e-05
470 2.94411893264623e-05
471 2.92139993689489e-05
472 2.89930449071107e-05
473 2.88482842734084e-05
474 2.87763432424981e-05
475 2.87092861981364e-05
476 2.85952464764705e-05
477 2.84368034044746e-05
478 2.82658984360751e-05
479 2.81156826531515e-05
480 2.8000351449009e-05
481 2.79050727840513e-05
482 2.78021507256199e-05
483 2.7675494493451e-05
484 2.75283655355452e-05
485 2.73754121735692e-05
486 2.72332417807775e-05
487 2.71105491265189e-05
488 2.70018299488584e-05
489 2.68921066890471e-05
490 2.67683735728497e-05
491 2.6627456463757e-05
492 2.64753507508431e-05
493 2.63233596342616e-05
494 2.61828245129436e-05
495 2.60588094533887e-05
496 2.59463595284615e-05
497 2.58327563642524e-05
498 2.57060110016027e-05
499 2.55611867032712e-05
500 2.5401317543583e-05
501 2.52356603596127e-05
502 2.50768662226619e-05
503 2.49372405960457e-05
504 2.48215510509908e-05
505 2.47219250013586e-05
506 2.46202216658276e-05
507 2.44982293224894e-05
508 2.43465910898522e-05
509 2.41670059040189e-05
510 2.39682685787557e-05
511 2.37650183407823e-05
512 2.35806110140402e-05
513 2.34456165344454e-05
514 2.3383203370031e-05
515 2.33847931667697e-05
516 2.33990012930008e-05
517 2.33557184401434e-05
518 2.32148468057858e-05
519 2.29931683861651e-05
520 2.27395139518194e-05
521 2.24811938096536e-05
522 2.21840145968599e-05
523 2.17865581362275e-05
524 2.14210158446804e-05
525 2.16531007026788e-05
526 2.30744353757473e-05
527 2.51413675869117e-05
528 2.5784118406591e-05
529 2.4512210075045e-05
530 2.51543806371046e-05
531 2.77776834991528e-05
532 2.61394015979022e-05
533 2.01969833142357e-05
534 2.57107021752745e-05
535 2.85362111753784e-05
536 2.39631717704469e-05
537 2.4327961000381e-05
538 2.13104995054891e-05
539 2.00701997528085e-05
540 2.2435435312218e-05
541 2.18788663914893e-05
542 2.07216362468898e-05
543 1.94767671928275e-05
544 1.90687878784956e-05
545 1.98467896552756e-05
546 2.00254562514601e-05
547 1.96297405636869e-05
548 1.91012404684443e-05
549 1.87698478839593e-05
550 1.88968970178394e-05
551 1.90720020327717e-05
552 1.90007558558136e-05
553 1.87842015293427e-05
554 1.85426833922975e-05
555 1.84163436642848e-05
556 1.84356995305279e-05
557 1.84555447049206e-05
558 1.83832944458118e-05
559 1.82426865649177e-05
560 1.80838214873802e-05
561 1.79670023499057e-05
562 1.79220151039772e-05
563 1.79030703293392e-05
564 1.78491573024075e-05
565 1.77475976670394e-05
566 1.76179710251745e-05
567 1.74895449163159e-05
568 1.73957432707539e-05
569 1.7345106243738e-05
570 1.7306039808318e-05
571 1.72411710082088e-05
572 1.71409992617555e-05
573 1.70166877069278e-05
574 1.68869555636775e-05
575 1.67780108313309e-05
576 1.67097659868887e-05
577 1.66712852660567e-05
578 1.66258159879362e-05
579 1.65457222465193e-05
580 1.64302819030127e-05
581 1.62912838277407e-05
582 1.61443476827117e-05
583 1.60180916282116e-05
584 1.59477476699976e-05
585 1.59385544975521e-05
586 1.59441879077349e-05
587 1.59036517288769e-05
588 1.57962731464067e-05
589 1.56419209815795e-05
590 1.54571534949355e-05
591 1.52446827996755e-05
592 1.50420464706258e-05
593 1.49633433466079e-05
594 1.51269368870999e-05
595 1.5478755813092e-05
596 1.57516533363378e-05
597 1.57180711539695e-05
598 1.5470326616196e-05
599 1.53109576785937e-05
600 1.53561759361764e-05
601 1.52557850015e-05
602 1.44344176078448e-05
603 1.38955265356344e-05
604 1.64324446814135e-05
605 2.02405663003447e-05
606 1.9168279322912e-05
607 1.68965907505481e-05
608 2.0116403902648e-05
609 1.95500579138752e-05
610 1.4348825061461e-05
611 1.91122544492828e-05
612 1.98110392375384e-05
613 1.7166901670862e-05
614 1.72947165992809e-05
615 1.39395324367797e-05
616 1.46355805554776e-05
617 1.58949478645809e-05
618 1.52196907947655e-05
619 1.46966485772282e-05
620 1.36576827571844e-05
621 1.37989818540518e-05
622 1.43958031912916e-05
623 1.43007873703027e-05
624 1.40294978336897e-05
625 1.36611160996836e-05
626 1.35726959342719e-05
627 1.38084515128867e-05
628 1.38780278575723e-05
629 1.37595334308571e-05
630 1.35871941893129e-05
631 1.34433612402063e-05
632 1.34554575197399e-05
633 1.35394839162473e-05
634 1.3532128832594e-05
635 1.34414331114385e-05
636 1.33242101583164e-05
637 1.32271015900187e-05
638 1.32072209453327e-05
639 1.32384748212644e-05
640 1.32322811623453e-05
641 1.31647375383181e-05
642 1.30676589833456e-05
643 1.29685486172093e-05
644 1.2903609786008e-05
645 1.28968158605858e-05
646 1.29094851217815e-05
647 1.28813207993517e-05
648 1.28034171211766e-05
649 1.27010853248066e-05
650 1.25944570754655e-05
651 1.25135684356792e-05
652 1.24954412967782e-05
653 1.25275100799627e-05
654 1.25396400108002e-05
655 1.24838434203411e-05
656 1.23761155919055e-05
657 1.22459587146295e-05
658 1.2107136171835e-05
659 1.20025433716364e-05
660 1.20161794257001e-05
661 1.21706307254499e-05
662 1.23402887766133e-05
663 1.2364122994768e-05
664 1.22288556667627e-05
665 1.20423719636165e-05
666 1.18496100185439e-05
667 1.15899338197778e-05
668 1.13137293737964e-05
669 1.14264985313639e-05
670 1.23329982670839e-05
671 1.35626314659021e-05
672 1.39608691824833e-05
673 1.33172807181836e-05
674 1.29536565509625e-05
675 1.3670782209374e-05
676 1.40369720611488e-05
677 1.18756561278133e-05
678 1.23500276458799e-05
679 1.93192263395758e-05
680 1.80683600774501e-05
681 1.37454562718631e-05
682 1.70660969160963e-05
683 1.39991607284173e-05
684 1.30637345137075e-05
685 1.69142385857413e-05
686 1.43353991006734e-05
687 1.44300975080114e-05
688 1.26803170132916e-05
689 1.22176361401216e-05
690 1.35823038363014e-05
691 1.31269598568906e-05
692 1.27181347124861e-05
693 1.19112000902533e-05
694 1.18841280709603e-05
695 1.23678155432572e-05
696 1.22703313536476e-05
697 1.20153708849102e-05
698 1.17153131213854e-05
699 1.16725586849498e-05
700 1.18685165944044e-05
701 1.18829620987526e-05
702 1.17515073725372e-05
703 1.16012761282036e-05
704 1.15166330942884e-05
705 1.15683278636425e-05
706 1.16235905807116e-05
707 1.15759130494553e-05
708 1.14773556560976e-05
709 1.13823944047908e-05
710 1.13372225314379e-05
711 1.1358917618054e-05
712 1.13768664959935e-05
713 1.13328869701945e-05
714 1.12490915853414e-05
715 1.11626386569696e-05
716 1.11033650682657e-05
717 1.10958226287039e-05
718 1.11152940007742e-05
719 1.11017161543714e-05
720 1.10363735075225e-05
721 1.0946455404337e-05
722 1.08598496808554e-05
723 1.08055228338344e-05
724 1.08121348603163e-05
725 1.08607591755572e-05
726 1.08822032416356e-05
727 1.0833812666533e-05
728 1.07354799183668e-05
729 1.06226707430324e-05
730 1.05247690953547e-05
731 1.04964965430554e-05
732 1.05935960164061e-05
733 1.07730393210659e-05
734 1.0886334166571e-05
735 1.08362019091146e-05
736 1.06700317701325e-05
737 1.04735472632456e-05
738 1.02637977761333e-05
739 1.00764382295893e-05
740 1.0126587767445e-05
741 1.06680490716826e-05
742 1.15088605525671e-05
743 1.19777614600025e-05
744 1.17172567115631e-05
745 1.12097050077864e-05
746 1.10929950096761e-05
747 1.11802764877211e-05
748 1.04763557828846e-05
749 9.57590873440495e-06
750 1.27217590488726e-05
751 1.79026810656069e-05
752 1.60794897965388e-05
753 1.27995454022312e-05
754 1.57829563249834e-05
755 1.39425337692956e-05
756 1.13354335553595e-05
757 1.7641939848545e-05
758 1.38724208227359e-05
759 1.29465133795748e-05
760 1.24535672512138e-05
761 1.10755563582643e-05
762 1.33139728859533e-05
763 1.23709214676637e-05
764 1.20469658213551e-05
765 1.11928884507506e-05
766 1.1240825187997e-05
767 1.19102069220389e-05
768 1.16271712613525e-05
769 1.13078676804435e-05
770 1.09200418592081e-05
771 1.10306391434278e-05
772 1.12514235297567e-05
773 1.11167901195586e-05
774 1.09076718217693e-05
775 1.07333489722805e-05
776 1.07566984297591e-05
777 1.08559452201007e-05
778 1.0804526937136e-05
779 1.06750903796637e-05
780 1.05602957773954e-05
781 1.05230492408737e-05
782 1.05648123280844e-05
783 1.05693088698899e-05
784 1.04942710095202e-05
785 1.03964530353551e-05
786 1.03223346741288e-05
787 1.0304170245945e-05
788 1.03261481854133e-05
789 1.03186221167562e-05
790 1.02548474387731e-05
791 1.01691084637423e-05
792 1.00984343589516e-05
793 1.00722145361942e-05
794 1.00963516160846e-05
795 1.01235391412047e-05
796 1.01004206953803e-05
797 1.00292500064825e-05
798 9.94670244836016e-06
799 9.88828105619177e-06
800 9.88997453532647e-06
801 9.95979280560277e-06
802 1.00375773399719e-05
803 1.00438564913929e-05
804 9.96640574157936e-06
805 9.85086899163434e-06
806 9.7440406534588e-06
807 9.69949996942887e-06
808 9.7849278972717e-06
809 9.9921753644594e-06
810 1.01743644336239e-05
811 1.01801524579059e-05
812 1.00169272627681e-05
813 9.79297146841418e-06
814 9.56612166191917e-06
815 9.37200820771977e-06
816 9.38983430387452e-06
817 9.87862040346954e-06
818 1.07222613223712e-05
819 1.12807892946876e-05
820 1.11056651803665e-05
821 1.05552135210019e-05
822 1.02622761914972e-05
823 1.01904988696333e-05
824 9.59162571234629e-06
825 8.8646247604629e-06
826 1.15554785224958e-05
827 1.66392874234589e-05
828 1.58580915012863e-05
829 1.22659930639202e-05
830 1.44580699270591e-05
831 1.38294517455506e-05
832 1.01167288448778e-05
833 1.74184624484042e-05
834 1.46888341987506e-05
835 1.17939080155338e-05
836 1.2117965525249e-05
837 1.01706527857459e-05
838 1.29942945932271e-05
839 1.16710434667766e-05
840 1.10403143480653e-05
841 1.0317573469365e-05
842 1.05486187749193e-05
843 1.13119394882233e-05
844 1.07812657006434e-05
845 1.04687851489871e-05
846 1.01692567113787e-05
847 1.05065482784994e-05
848 1.06665465864353e-05
849 1.04068994914996e-05
850 1.01913856269675e-05
851 1.00969464256195e-05
852 1.02553703982267e-05
853 1.03029642559704e-05
854 1.01469213404926e-05
855 9.99457552097738e-06
856 9.92374680208741e-06
857 9.98255382000934e-06
858 1.00286260931171e-05
859 9.94638230622513e-06
860 9.82332494459115e-06
861 9.73933492787182e-06
862 9.73848818830447e-06
863 9.79651031229878e-06
864 9.80245476966957e-06
865 9.72576071944786e-06
866 9.63164893619251e-06
867 9.57704924076097e-06
868 9.59899625740945e-06
869 9.6743278845679e-06
870 9.70800647337455e-06
871 9.65527397056576e-06
872 9.56350049818866e-06
873 9.49379500525538e-06
874 9.49654531723354e-06
875 9.58847249421524e-06
876 9.69230677583255e-06
877 9.70125529420329e-06
878 9.60870784183498e-06
879 9.48473461903632e-06
880 9.39484289119719e-06
881 9.40688096306985e-06
882 9.55871655605733e-06
883 9.75254260993097e-06
884 9.8139107649331e-06
885 9.69646953308256e-06
886 9.49540208239341e-06
887 9.29628458834486e-06
888 9.16814951779088e-06
889 9.25822860153858e-06
890 9.66163588600466e-06
891 1.01439782156376e-05
892 1.02945923572406e-05
893 1.00362312878133e-05
894 9.65318213275168e-06
895 9.316927389591e-06
896 8.9382356236456e-06
897 8.66185382619733e-06
898 9.35586922423681e-06
899 1.13931328087347e-05
900 1.29029476738651e-05
901 1.22205838124501e-05
902 1.10385808511637e-05
903 1.13771448013722e-05
904 1.13959995360347e-05
905 9.02071133168647e-06
906 1.2118026461394e-05
907 1.9113089365419e-05
908 1.43496599775972e-05
909 1.24428743220051e-05
910 1.35961054184008e-05
911 9.805497029447e-06
912 1.50444202517974e-05
913 1.30931675812462e-05
914 1.09160391730256e-05
915 1.03243910416495e-05
916 1.00864554042346e-05
917 1.19477927000844e-05
918 1.0516796464799e-05
919 9.9644767033169e-06
920 9.4463021014235e-06
921 1.02788480944582e-05
922 1.04544224086567e-05
923 9.81972152658273e-06
924 9.49589320953237e-06
925 9.53088692767778e-06
926 1.00109764389344e-05
927 9.90600256045582e-06
928 9.56545318331337e-06
929 9.39520214160439e-06
930 9.5157129180734e-06
931 9.78403841145337e-06
932 9.72633188212058e-06
933 9.50681351241656e-06
934 9.38283119467087e-06
935 9.4495371740777e-06
936 9.65105482464423e-06
937 9.69250322668813e-06
938 9.54744609771296e-06
939 9.40856716624694e-06
940 9.38615903578466e-06
941 9.52063146542059e-06
942 9.66410789260408e-06
943 9.63524234975921e-06
944 9.49301829678006e-06
945 9.37526510824682e-06
946 9.37005097512156e-06
947 9.5089371825452e-06
948 9.65791605267441e-06
949 9.64730224950472e-06
950 9.50434150581714e-06
951 9.35356638365192e-06
952 9.28811823541764e-06
953 9.38757966650883e-06
954 9.60705256147776e-06
955 9.73029091255739e-06
956 9.63788261287846e-06
957 9.43054510571528e-06
958 9.23220522963675e-06
959 9.13853364181705e-06
960 9.2888640210731e-06
961 9.67004598351195e-06
962 9.95191567199072e-06
963 9.87248677120078e-06
964 9.56190160650294e-06
965 9.22986691875849e-06
966 8.92545449460158e-06
967 8.82890981301898e-06
968 9.36527339945314e-06
969 1.04214395832969e-05
970 1.09879538285895e-05
971 1.05920462374343e-05
972 9.96983453660505e-06
973 9.62828562478535e-06
974 9.09544087335235e-06
975 8.38977848616196e-06
976 1.00487568488461e-05
977 1.40774054671056e-05
978 1.4264827768784e-05
979 1.16455057650455e-05
980 1.21941120596603e-05
981 1.16975943456055e-05
982 9.02731244423194e-06
983 1.55222569446778e-05
984 1.62284432008164e-05
985 1.15077555165044e-05
986 1.24180332932156e-05
987 9.4674915089854e-06
988 1.31716997202602e-05
989 1.29668205772759e-05
990 1.04037517303368e-05
991 9.81773519015405e-06
992 9.68360745901009e-06
993 1.1658782568702e-05
994 1.04250430013053e-05
995 9.57816337177064e-06
996 9.16543285711668e-06
997 1.01184859886416e-05
998 1.04866712717921e-05
999 9.7022348199971e-06
1000 9.28210192796541e-06
1001 9.36511241889093e-06
1002 1.00275628938107e-05
1003 9.99582607619232e-06
1004 9.53642938839039e-06
1005 9.29829730011988e-06
1006 9.44684143178165e-06
1007 9.85301903710933e-06
1008 9.84896541922353e-06
1009 9.54981260292698e-06
1010 9.35111620492535e-06
1011 9.40055542741902e-06
1012 9.68521362665342e-06
1013 9.81057291937759e-06
1014 9.63663842412643e-06
1015 9.41938196774572e-06
1016 9.33192768570734e-06
1017 9.47278749663383e-06
1018 9.71785721048946e-06
1019 9.75498096522642e-06
1020 9.56990334088914e-06
1021 9.365200639877e-06
1022 9.27829387364909e-06
1023 9.42012320592767e-06
1024 9.70026485447306e-06
1025 9.79683773039142e-06
1026 9.62308877205942e-06
1027 9.36989636102226e-06
1028 9.18271780392388e-06
1029 9.21025366551476e-06
1030 9.55388350121211e-06
1031 9.91795786831062e-06
1032 9.90535500022816e-06
1033 9.59609042183729e-06
1034 9.25318545341725e-06
1035 8.97103655006504e-06
1036 8.99057067726972e-06
1037 9.62453304964583e-06
1038 1.04105029095081e-05
1039 1.04835280581028e-05
1040 9.96850212686695e-06
1041 9.48427987168543e-06
1042 9.00267605175031e-06
1043 8.52413268148666e-06
1044 9.28693498281064e-06
1045 1.16407672976493e-05
1046 1.25843553178129e-05
1047 1.129052543547e-05
1048 1.0636892511684e-05
1049 1.05497101685614e-05
1050 8.88017802935792e-06
1051 1.01247833299567e-05
1052 1.60959953063866e-05
1053 1.43076031235978e-05
1054 1.16871251520934e-05
1055 1.23780546346097e-05
1056 9.26997108763317e-06
1057 1.36937314891838e-05
1058 1.45651383718359e-05
1059 1.10291903183679e-05
1060 1.07238129203324e-05
1061 9.40074551181169e-06
1062 1.2625982890313e-05
1063 1.14433887574705e-05
1064 9.96757898974465e-06
1065 9.21188529900974e-06
1066 1.01569130492862e-05
1067 1.12084317152039e-05
1068 1.004542082228e-05
1069 9.38053381105419e-06
1070 9.29662837734213e-06
1071 1.02982940006768e-05
1072 1.03322126960848e-05
1073 9.65530853136443e-06
1074 9.29941325011896e-06
1075 9.49854256759863e-06
1076 1.0086387192132e-05
1077 1.00206980278017e-05
1078 9.59760654950514e-06
1079 9.35310072236462e-06
1080 9.48142860579537e-06
1081 9.88806186796864e-06
1082 9.9509288702393e-06
1083 9.66103652899619e-06
1084 9.40596692089457e-06
1085 9.37866479944205e-06
1086 9.66881452768575e-06
1087 9.92234890873078e-06
1088 9.80343975243159e-06
1089 9.52232585405e-06
1090 9.32403145270655e-06
1091 9.37688218982657e-06
1092 9.71887038758723e-06
1093 9.94838319456903e-06
1094 9.79443120741053e-06
1095 9.48624619923066e-06
1096 9.24140204006108e-06
1097 9.24408777791541e-06
1098 9.64934133662609e-06
1099 1.00668994491571e-05
1100 9.99851909000427e-06
1101 9.62208196142456e-06
1102 9.24737014429411e-06
1103 8.99398855835898e-06
1104 9.249776667275e-06
1105 1.01033911050763e-05
1106 1.059754868038e-05
1107 1.02340572993853e-05
1108 9.67430332821095e-06
1109 9.18316345632775e-06
1110 8.68018378241686e-06
1111 9.17663965083193e-06
1112 1.1158736924699e-05
1113 1.20658114610706e-05
1114 1.10221371869557e-05
1115 1.03284237411572e-05
1116 9.91179331322201e-06
1117 8.57904797157971e-06
1118 1.02257990874932e-05
1119 1.4911135622242e-05
1120 1.3553034477809e-05
1121 1.14667200250551e-05
1122 1.17623194455518e-05
1123 9.09991103981156e-06
1124 1.27273515317938e-05
1125 1.55265934154158e-05
1126 1.16112705654814e-05
1127 1.1388642633392e-05
1128 9.26452139538014e-06
1129 1.24993257486494e-05
1130 1.27616876852699e-05
1131 1.05004464785452e-05
1132 9.64417540672002e-06
1133 9.7557913250057e-06
1134 1.17914387374185e-05
1135 1.06574316305341e-05
1136 9.67671985563356e-06
1137 9.22382423595991e-06
1138 1.03378370113205e-05
1139 1.0770132575999e-05
1140 9.92998138826806e-06
1141 9.39834626478842e-06
1142 9.48136948863976e-06
1143 1.02630147011951e-05
1144 1.02795247585163e-05
1145 9.74746035353746e-06
1146 9.41040980251273e-06
1147 9.54680308495881e-06
1148 1.00779952845187e-05
1149 1.01377208920894e-05
1150 9.75964758254122e-06
1151 9.45330612012185e-06
1152 9.46504405874293e-06
1153 9.88071860774653e-06
1154 1.01208152045729e-05
1155 9.8800883279182e-06
1156 9.53955168370157e-06
1157 9.36361266212771e-06
1158 9.58177315624198e-06
1159 1.00345287137316e-05
1160 1.00990155260661e-05
1161 9.77086256170878e-06
1162 9.42404949455522e-06
1163 9.26502252696082e-06
1164 9.58586224442115e-06
1165 1.01630275821663e-05
1166 1.02379908639705e-05
1167 9.83565223577898e-06
1168 9.40368227020372e-06
1169 9.09272966964636e-06
1170 9.34932631935226e-06
1171 1.02836384030525e-05
1172 1.07293471955927e-05
1173 1.0255128472636e-05
1174 9.67503729043528e-06
1175 9.12564610189293e-06
1176 8.80126663105329e-06
1177 9.96924154605949e-06
1178 1.17397748908843e-05
1179 1.15256452772883e-05
1180 1.04770197140169e-05
1181 1.00163142633392e-05
1182 8.96974961506203e-06
1183 9.07834146346431e-06
1184 1.30785674627987e-05
1185 1.39585508804885e-05
1186 1.15056827780791e-05
1187 1.14337444756529e-05
1188 9.65234448813135e-06
1189 1.02219819382299e-05
1190 1.56322330440162e-05
1191 1.30193666336709e-05
1192 1.16078672363074e-05
1193 1.01610166893806e-05
1194 1.05426242953399e-05
1195 1.40848187584197e-05
1196 1.15922402983415e-05
1197 1.05109102150891e-05
1198 9.2303343990352e-06
1199 1.15053990157321e-05
1200 1.18739817480673e-05
1201 1.02875710581429e-05
1202 9.44575458561303e-06
1203 9.81563243840355e-06
1204 1.12402894956176e-05
1205 1.05445778899593e-05
1206 9.71455938270083e-06
1207 9.37337881623534e-06
1208 1.02098583738552e-05
1209 1.06854231489706e-05
1210 1.00891420515836e-05
1211 9.56523672357434e-06
1212 9.51906258706003e-06
1213 1.01790183180128e-05
1214 1.0424809261167e-05
1215 9.98470386548433e-06
1216 9.57006250246195e-06
1217 9.52111804508604e-06
1218 1.00357692645048e-05
1219 1.03523680081707e-05
1220 1.00428751466097e-05
1221 9.6316989584011e-06
1222 9.44365820032544e-06
1223 9.78678326646332e-06
1224 1.03086067611002e-05
1225 1.02381818578579e-05
1226 9.80823006102582e-06
1227 9.44017301662825e-06
1228 9.42078258958645e-06
1229 1.00200604720158e-05
1230 1.0495863534743e-05
1231 1.02245012385538e-05
1232 9.71855570242042e-06
1233 9.29199268284719e-06
1234 9.30199166759849e-06
1235 1.0196120456385e-05
1236 1.08814356281073e-05
1237 1.04811751953093e-05
1238 9.8444343166193e-06
1239 9.26212487684097e-06
1240 8.97594691195991e-06
1241 1.02391031759907e-05
1242 1.17607369247708e-05
1243 1.12720472316141e-05
1244 1.0345923328714e-05
1245 9.73737860476831e-06
1246 8.73195676831529e-06
1247 1.01947316579754e-05
1248 1.35012505779741e-05
1249 1.25975184346316e-05
1250 1.11122053567669e-05
1251 1.06853203760693e-05
1252 8.79738581716083e-06
1253 1.22636402011267e-05
1254 1.52024713315768e-05
1255 1.20061940833693e-05
1256 1.15705188363791e-05
1257 9.33901992539177e-06
1258 1.20037220767699e-05
1259 1.42386406878359e-05
1260 1.1454790183052e-05
1261 1.05372928373981e-05
1262 9.42273072723765e-06
1263 1.25141632452141e-05
1264 1.18968946480891e-05
1265 1.04104274214478e-05
1266 9.38246103032725e-06
1267 1.04994414868997e-05
1268 1.16924938993179e-05
1269 1.05504959719838e-05
1270 9.69497796177166e-06
1271 9.56587609834969e-06
1272 1.08495751192095e-05
1273 1.08500998976524e-05
1274 1.00432571343845e-05
1275 9.5226341727539e-06
1276 9.88577176030958e-06
1277 1.071497626981e-05
1278 1.04887531051645e-05
1279 9.88387910183519e-06
1280 9.53907783696195e-06
1281 9.93402682070155e-06
1282 1.0589146768325e-05
1283 1.04084101621993e-05
1284 9.8815753517556e-06
1285 9.53696053329622e-06
1286 9.80253207671922e-06
1287 1.04939072116395e-05
1288 1.050575338013e-05
1289 1.00029374152655e-05
1290 9.56097119342303e-06
1291 9.53903054323746e-06
1292 1.02663325378671e-05
1293 1.07241767182131e-05
1294 1.03197589851334e-05
1295 9.76158025878249e-06
1296 9.34516447159695e-06
1297 9.66753304965096e-06
1298 1.07481582745095e-05
1299 1.09585644167964e-05
1300 1.02910489658825e-05
1301 9.66546849667793e-06
1302 9.1285073722247e-06
1303 9.78553816821659e-06
1304 1.14758395284298e-05
1305 1.14984341053059e-05
1306 1.05145209090551e-05
1307 9.84343114396324e-06
1308 8.92794287210563e-06
1309 1.00477282103384e-05
1310 1.29132404254051e-05
1311 1.23324671221781e-05
1312 1.09662532850052e-05
1313 1.03136599136633e-05
1314 8.81559572007973e-06
1315 1.20079794214689e-05
1316 1.46474003486219e-05
1317 1.19882224680623e-05
1318 1.140119820775e-05
1319 9.33348474063678e-06
1320 1.15351513159112e-05
1321 1.46603324537864e-05
1322 1.19087117127492e-05
1323 1.09950497062528e-05
1324 9.28073859540746e-06
1325 1.24947846416035e-05
1326 1.27844014059519e-05
1327 1.09326974779833e-05
1328 9.69431494013406e-06
1329 1.02675849120715e-05
1330 1.22824549180223e-05
1331 1.10900082290755e-05
1332 1.0023563845607e-05
1333 9.50310550251743e-06
1334 1.10981191028259e-05
1335 1.13147279989789e-05
1336 1.03199326986214e-05
1337 9.60545730777085e-06
1338 1.00013612609473e-05
1339 1.10868513729656e-05
1340 1.07397445390234e-05
1341 9.99874555418501e-06
1342 9.59437056735624e-06
1343 1.02381982287625e-05
1344 1.09391967271222e-05
1345 1.05188601082773e-05
1346 9.89727777778171e-06
1347 9.60870784183498e-06
1348 1.02406884252559e-05
1349 1.08852991616004e-05
1350 1.05157287180191e-05
1351 9.91646811598912e-06
1352 9.55745235842187e-06
1353 1.00636061688419e-05
1354 1.09031980173313e-05
1355 1.07101441244595e-05
1356 1.0070681128127e-05
1357 9.5396171673201e-06
1358 9.67736832535593e-06
1359 1.08046215245849e-05
1360 1.11402996481047e-05
1361 1.04515938801342e-05
1362 9.77819127001567e-06
1363 9.28640747588361e-06
1364 1.01729492598679e-05
1365 1.16377777885646e-05
1366 1.12722973426571e-05
1367 1.03723132269806e-05
1368 9.58337022893829e-06
1369 9.15426335268421e-06
1370 1.12611487566028e-05
1371 1.26881577671156e-05
1372 1.13691376100178e-05
1373 1.05591225292301e-05
1374 9.32115653995425e-06
1375 9.79695869318675e-06
1376 1.36235485115321e-05
1377 1.29613690660335e-05
1378 1.13510532173677e-05
1379 1.03397542261519e-05
1380 9.26259690459119e-06
1381 1.37217903102282e-05
1382 1.36217367980862e-05
1383 1.15758703032043e-05
1384 1.02807589428267e-05
1385 9.9264616437722e-06
1386 1.37960851134267e-05
1387 1.23818317661062e-05
1388 1.09808916022303e-05
1389 9.41299913392868e-06
1390 1.14851909529534e-05
1391 1.27144376165234e-05
1392 1.10689352368354e-05
1393 9.89371983450837e-06
1394 9.9491362561821e-06
1395 1.2003291885776e-05
1396 1.1345395250828e-05
1397 1.02519861684414e-05
1398 9.56044732447481e-06
1399 1.08424928839668e-05
1400 1.15216189442435e-05
1401 1.06223060356569e-05
1402 9.81968150881585e-06
1403 9.87932617135812e-06
1404 1.11115468826029e-05
1405 1.10834644146962e-05
1406 1.02896601674729e-05
1407 9.6994126579375e-06
1408 1.0125239896297e-05
1409 1.11326271508005e-05
1410 1.08880076368223e-05
1411 1.01651530712843e-05
1412 9.66166226135101e-06
1413 1.01656078186352e-05
1414 1.11529034256819e-05
1415 1.09055354187149e-05
1416 1.01853856904199e-05
1417 9.62197555054445e-06
1418 1.00015577118029e-05
1419 1.11983636088553e-05
1420 1.11459703475703e-05
1421 1.03658530861139e-05
1422 9.667249287304e-06
1423 9.63268121267902e-06
1424 1.11109256977215e-05
1425 1.16588880700874e-05
1426 1.07744408524013e-05
1427 9.97826464299578e-06
1428 9.28966983337887e-06
1429 1.05647695818334e-05
1430 1.23566278489307e-05
1431 1.15531429401017e-05
1432 1.05717035694397e-05
1433 9.51579204411246e-06
1434 9.63314141699811e-06
1435 1.27585108202766e-05
1436 1.27800312839099e-05
1437 1.12293500933447e-05
1438 1.02655621958547e-05
1439 9.16011595109012e-06
1440 1.28030769701581e-05
1441 1.3758140084974e-05
1442 1.16569690362667e-05
1443 1.06328589026816e-05
1444 9.32424791244557e-06
1445 1.33067333081271e-05
1446 1.33750181703363e-05
1447 1.14927142931265e-05
1448 9.97419920167886e-06
1449 1.03910224424908e-05
1450 1.34850615722826e-05
1451 1.19790274766274e-05
1452 1.06828983916785e-05
1453 9.47245098359417e-06
1454 1.19241858556052e-05
1455 1.2354659702396e-05
1456 1.09227530629141e-05
1457 9.7373213066021e-06
1458 1.04380960692652e-05
1459 1.21216899060528e-05
1460 1.12388297566213e-05
1461 1.01857376648695e-05
1462 9.72363795881392e-06
1463 1.12648667709436e-05
1464 1.1591769180086e-05
1465 1.06269208117737e-05
1466 9.80994263954926e-06
1467 1.01836212706985e-05
1468 1.14800404844573e-05
1469 1.1162976079504e-05
1470 1.03039428722695e-05
1471 9.72574889601674e-06
1472 1.05177623481723e-05
1473 1.14995291369269e-05
1474 1.09574166344828e-05
1475 1.01565510703949e-05
1476 9.69732627709163e-06
1477 1.06620100268628e-05
1478 1.15700022433884e-05
1479 1.09470420284197e-05
1480 1.01391651696758e-05
1481 9.61518890107982e-06
1482 1.06370380308363e-05
1483 1.17773097372265e-05
1484 1.11348927021027e-05
1485 1.02667927421862e-05
1486 9.51800393522717e-06
1487 1.03986958492897e-05
1488 1.21170360216638e-05
1489 1.15672200990957e-05
1490 1.05699300547712e-05
1491 9.57559132075403e-06
1492 9.94542097032536e-06
1493 1.24789685287396e-05
1494 1.22773144539678e-05
1495 1.10048167698551e-05
1496 9.90673925116425e-06
1497 9.52693881117739e-06
1498 1.27892162709031e-05
1499 1.30610569613054e-05
1500 1.14122522063553e-05
1501 1.02485064417124e-05
1502 9.48576143855462e-06
1503 1.32944687720737e-05
1504 1.32729292090517e-05
1505 1.15311158879194e-05
1506 1.00433617262752e-05
1507 1.01873592939228e-05
1508 1.38000277729589e-05
1509 1.24796115414938e-05
1510 1.11125727926265e-05
1511 9.46372165344656e-06
1512 1.1827048183477e-05
1513 1.31595252241823e-05
1514 1.14641934487736e-05
1515 1.00861752798664e-05
1516 1.01701971289003e-05
1517 1.26706536320853e-05
1518 1.18394882520079e-05
1519 1.05798608274199e-05
1520 9.65487197390758e-06
1521 1.15095363071305e-05
1522 1.21309503811062e-05
1523 1.09537586467923e-05
1524 9.89668114925735e-06
1525 1.03344891613233e-05
1526 1.19499509310117e-05
1527 1.14071654024883e-05
1528 1.03927368400036e-05
1529 9.78124171524541e-06
1530 1.10590672193212e-05
1531 1.18145899250521e-05
1532 1.09469010567409e-05
1533 1.00531187854358e-05
1534 9.98771611193661e-06
1535 1.14720669444068e-05
1536 1.1620811164903e-05
1537 1.06988609331893e-05
1538 9.87391558737727e-06
1539 1.01942969195079e-05
1540 1.17268200483522e-05
1541 1.15522798296297e-05
1542 1.06044353742618e-05
1543 9.76793035079027e-06
1544 1.02881585917203e-05
1545 1.19820442705532e-05
1546 1.16524233817472e-05
1547 1.06453817352303e-05
1548 9.70328437688295e-06
1549 1.02530320873484e-05
1550 1.23197760331095e-05
1551 1.19272617666866e-05
1552 1.08128997453605e-05
1553 9.70693417912116e-06
1554 1.01434015959967e-05
1555 1.27699058793951e-05
1556 1.23169102153042e-05
1557 1.10552273326903e-05
1558 9.76313913270133e-06
1559 1.01548448583344e-05
1560 1.33425846797763e-05
1561 1.25837768791826e-05
1562 1.12367379188072e-05
1563 9.68821132119047e-06
1564 1.06766674434766e-05
1565 1.38573013828136e-05
1566 1.2371490811347e-05
1567 1.11051185740507e-05
1568 9.43918439588742e-06
1569 1.20241356853512e-05
1570 1.35948948809528e-05
1571 1.17666786536574e-05
1572 1.0332538295188e-05
1573 1.00322313301149e-05
1574 1.31704182422254e-05
1575 1.24003336168244e-05
1576 1.09904949567863e-05
1577 9.6214898803737e-06
1578 1.17186618808773e-05
1579 1.2770741705026e-05
1580 1.13563828563201e-05
1581 1.00372717497521e-05
1582 1.04054570329026e-05
1583 1.24987491290085e-05
1584 1.17549161586794e-05
1585 1.055654593074e-05
1586 9.81675839284435e-06
1587 1.15938382805325e-05
1588 1.21434177344781e-05
1589 1.10319488157984e-05
1590 9.98713130684337e-06
1591 1.04458504210925e-05
1592 1.20895665531862e-05
1593 1.15837901830673e-05
1594 1.05407025330351e-05
1595 9.84632606559899e-06
1596 1.11734270831221e-05
1597 1.20978320410359e-05
1598 1.11875278889784e-05
1599 1.02010253613116e-05
1600 9.99418898572912e-06
1601 1.17127956400509e-05
1602 1.19923115562415e-05
1603 1.09567117760889e-05
1604 9.97611550701549e-06
1605 1.02255444289767e-05
1606 1.21327329907217e-05
1607 1.19424994409201e-05
1608 1.0850814760488e-05
1609 9.81700486590853e-06
1610 1.04676091723377e-05
1611 1.25518763525179e-05
1612 1.19758278742665e-05
1613 1.08372787508415e-05
1614 9.68425683822716e-06
1615 1.07760042737937e-05
1616 1.30258604258415e-05
1617 1.20312042781734e-05
1618 1.08598851511488e-05
1619 9.55504583544098e-06
1620 1.13354863060522e-05
1621 1.34557940327795e-05
1622 1.19827418529894e-05
1623 1.07740042949445e-05
1624 9.50478715822101e-06
1625 1.23509435070446e-05
1626 1.34684560180176e-05
1627 1.17518529805238e-05
1628 1.03173606476048e-05
1629 1.00467514130287e-05
1630 1.3437223060464e-05
1631 1.27474277178408e-05
1632 1.12856641862891e-05
1633 9.65220715443138e-06
1634 1.16863157018088e-05
1635 1.34135298139881e-05
1636 1.18139578262344e-05
1637 1.03504862636328e-05
1638 1.01944933703635e-05
1639 1.29885029309662e-05
1640 1.23233230624464e-05
1641 1.09459952000179e-05
1642 9.76188857748639e-06
1643 1.18414845928783e-05
1644 1.26931399790919e-05
1645 1.13687947305152e-05
1646 1.00675042631337e-05
1647 1.06311181298224e-05
1648 1.25900951388758e-05
1649 1.18145126180025e-05
1650 1.0605648640194e-05
1651 9.94555739453062e-06
1652 1.1826322406705e-05
1653 1.22874062071787e-05
1654 1.11423169073532e-05
1655 1.00448323792079e-05
1656 1.06502038761391e-05
1657 1.23699392133858e-05
1658 1.17468516691588e-05
1659 1.06360785139259e-05
1660 9.91667820926523e-06
1661 1.15150432975497e-05
1662 1.23797335618292e-05
1663 1.13284868348273e-05
1664 1.02347894426202e-05
1665 1.01788027677685e-05
1666 1.22015362649108e-05
1667 1.21950815810123e-05
1668 1.10414266600856e-05
1669 9.93904723145533e-06
1670 1.06824527392746e-05
1671 1.26942022689036e-05
1672 1.19925871331361e-05
1673 1.08253625512589e-05
1674 9.7519960036152e-06
1675 1.1377770533727e-05
1676 1.3020994629187e-05
1677 1.18081461550901e-05
1678 1.06002435131813e-05
1679 9.73721307673259e-06
1680 1.22841383927152e-05
1681 1.30997532323818e-05
1682 1.1620579243754e-05
1683 1.02480335044675e-05
1684 1.01507202998619e-05
1685 1.32546956592705e-05
1686 1.27777066154522e-05
1687 1.13515025077504e-05
1688 9.76362753135618e-06
1689 1.1378313502064e-05
1690 1.3676750313607e-05
1691 1.21543489512987e-05
1692 1.07433988887351e-05
1693 9.84811504167737e-06
1694 1.29840218505706e-05
1695 1.30192274809815e-05
1696 1.14844679046655e-05
1697 9.86954728432465e-06
1698 1.138881634688e-05
1699 1.34034080474521e-05
1700 1.19892165457713e-05
1701 1.05258031908306e-05
1702 1.0191942237725e-05
1703 1.29150657812716e-05
1704 1.24882735690335e-05
1705 1.10971068352228e-05
1706 9.87313796940725e-06
1707 1.18382722575916e-05
1708 1.28482142827124e-05
1709 1.15450311568566e-05
1710 1.01992191048339e-05
1711 1.06918851088267e-05
1712 1.27572566270828e-05
1713 1.20167323984788e-05
1714 1.07546538856695e-05
1715 1.00259185273899e-05
1716 1.201724717248e-05
1717 1.25144460980664e-05
1718 1.13107771539944e-05
1719 1.01221257864381e-05
1720 1.08403410195024e-05
1721 1.26587237900821e-05
1722 1.19174264909816e-05
1723 1.07236546682543e-05
1724 1.0017647582572e-05
1725 1.19159876703634e-05
1726 1.25931383081479e-05
1727 1.14160366138094e-05
1728 1.02145113487495e-05
1729 1.05357494248892e-05
1730 1.26882277982077e-05
1731 1.22287674457766e-05
1732 1.10064938780852e-05
1733 9.90865009953268e-06
1734 1.14667100206134e-05
1735 1.30048465507571e-05
1736 1.18430452857865e-05
1737 1.0582489267108e-05
1738 1.00017741715419e-05
1739 1.25325814224198e-05
1740 1.28897499962477e-05
1741 1.150586285803e-05
1742 1.00974248198327e-05
1743 1.07272671812098e-05
1744 1.33612074932898e-05
1745 1.24565785881714e-05
1746 1.11051995190792e-05
1747 9.77632225840352e-06
1748 1.21205803225166e-05
1749 1.34692572828499e-05
1750 1.1927358173125e-05
1751 1.0421579645481e-05
1752 1.03601423688815e-05
1753 1.34281845021178e-05
1754 1.27679049910512e-05
1755 1.13088071884704e-05
1756 9.81362245511264e-06
1757 1.21286511785001e-05
1758 1.34604961203877e-05
1759 1.1916272342205e-05
1760 1.03326701719197e-05
1761 1.06844508991344e-05
1762 1.33957664729678e-05
1763 1.24610405691783e-05
1764 1.10073460746207e-05
1765 9.97760980681051e-06
1766 1.25577580547542e-05
1767 1.29555683088256e-05
1768 1.15216889753356e-05
1769 1.00658517112606e-05
1770 1.13893511297647e-05
1771 1.31470678752521e-05
1772 1.19646001621732e-05
1773 1.05589670056361e-05
1774 1.0418990314065e-05
1775 1.27981093100971e-05
1776 1.24443195090862e-05
1777 1.11179642772186e-05
1778 1.0048443073174e-05
1779 1.18902889880701e-05
1780 1.2866303222836e-05
1781 1.16436794996844e-05
1782 1.03301345006912e-05
1783 1.07607002064469e-05
1784 1.28573656184017e-05
1785 1.2222792065586e-05
1786 1.09468783193734e-05
1787 1.00854022093699e-05
1788 1.20670929391054e-05
1789 1.28345900520799e-05
1790 1.15960292532691e-05
1791 1.0296768778062e-05
1792 1.07687883428298e-05
1793 1.29632944663172e-05
1794 1.23193776744301e-05
1795 1.10300843516598e-05
1796 1.00137413028278e-05
1797 1.20207851068699e-05
1798 1.30853140944964e-05
1799 1.17855570351821e-05
1800 1.04111968539655e-05
1801 1.05280569187016e-05
1802 1.31181386677781e-05
1803 1.26421209643013e-05
1804 1.12770376290428e-05
1805 9.95032314676791e-06
1806 1.18643538371543e-05
1807 1.34440870169783e-05
1808 1.20555032481207e-05
1809 1.05926510514109e-05
1810 1.03396787380916e-05
1811 1.32698605739279e-05
1812 1.29134223243454e-05
1813 1.145391161117e-05
1814 9.94379297480918e-06
1815 1.19450851343572e-05
1816 1.36058533826144e-05
1817 1.21263719847775e-05
1818 1.05580129456939e-05
1819 1.05317130874027e-05
1820 1.3461843082041e-05
1821 1.27842959045665e-05
1822 1.12958650788642e-05
1823 9.97424740489805e-06
1824 1.24758953461424e-05
1825 1.33449120767182e-05
1826 1.18468187793042e-05
1827 1.02424319265992e-05
1828 1.12403649836779e-05
1829 1.34697656903882e-05
1830 1.23164909382467e-05
1831 1.08150361484149e-05
1832 1.0354654477851e-05
1833 1.30057760543423e-05
1834 1.27877201521187e-05
1835 1.13610394691932e-05
1836 1.00969000413897e-05
1837 1.20817749120761e-05
1838 1.31557044369401e-05
1839 1.18332263809862e-05
1840 1.03830798252602e-05
1841 1.10252076410688e-05
1842 1.31672586576315e-05
1843 1.23138415801805e-05
1844 1.09298553070403e-05
1845 1.02877165772952e-05
1846 1.26131026263465e-05
1847 1.28357723951922e-05
1848 1.15036418719683e-05
1849 1.02030808193376e-05
1850 1.15562979772221e-05
1851 1.3180285350245e-05
1852 1.20704498840496e-05
1853 1.06921315818909e-05
1854 1.05000117400778e-05
1855 1.29141417346545e-05
1856 1.27170414998545e-05
1857 1.13696532935137e-05
1858 1.01363302746904e-05
1859 1.18641773951822e-05
1860 1.32789591589244e-05
1861 1.20290924314759e-05
1862 1.0604569069983e-05
1863 1.05947638076032e-05
1864 1.31747719933628e-05
1865 1.2767484804499e-05
1866 1.13696851258283e-05
1867 1.00809675132041e-05
1868 1.21037437565974e-05
1869 1.34543834064971e-05
1870 1.20704316941556e-05
1871 1.05508706838009e-05
1872 1.07140722320764e-05
1873 1.34636757138651e-05
1874 1.28027841128642e-05
1875 1.13401220005471e-05
1876 1.0058107363875e-05
1877 1.25053538795328e-05
1878 1.35034470076789e-05
1879 1.20056010928238e-05
1880 1.03761431091698e-05
1881 1.11501940409653e-05
1882 1.3692456377612e-05
1883 1.26064442156348e-05
1884 1.10714954644209e-05
1885 1.02488093034481e-05
1886 1.31115466501797e-05
1887 1.31980032165302e-05
1888 1.16850487756892e-05
1889 1.01405985333258e-05
1890 1.20346558105666e-05
1891 1.35773980218801e-05
1892 1.21735656648525e-05
1893 1.05625304058776e-05
1894 1.09658076326014e-05
1895 1.34966157929739e-05
1896 1.2632390280487e-05
1897 1.11243289211416e-05
1898 1.03163092717296e-05
1899 1.29118634504266e-05
1900 1.30804264699691e-05
1901 1.16369055831456e-05
1902 1.02304784377338e-05
1903 1.19764217743068e-05
1904 1.33834655571263e-05
1905 1.20943905130844e-05
1906 1.05842218545149e-05
1907 1.09974025690462e-05
1908 1.33203056975617e-05
1909 1.25661963465973e-05
1910 1.11294220914715e-05
1911 1.03559477793169e-05
1912 1.27462308228132e-05
1913 1.3062488505966e-05
1914 1.16831370178261e-05
1915 1.03030724858399e-05
1916 1.17441322800005e-05
1917 1.33924750116421e-05
1918 1.22198553071939e-05
1919 1.07565474536386e-05
1920 1.07224250314175e-05
1921 1.32158647829783e-05
1922 1.28102165035671e-05
1923 1.13975311251124e-05
1924 1.02376161521534e-05
1925 1.23673044072348e-05
1926 1.33783814817434e-05
1927 1.20140302897198e-05
1928 1.04991049738601e-05
1929 1.11629542516312e-05
1930 1.35397776830359e-05
1931 1.26403147078236e-05
1932 1.11675290099811e-05
1933 1.03133133961819e-05
1934 1.29466961880098e-05
1935 1.33073581309873e-05
1936 1.18395628305734e-05
1937 1.02882768260315e-05
1938 1.17664667413919e-05
1939 1.37269344122615e-05
1940 1.24407242765301e-05
1941 1.086078100343e-05
1942 1.06549696283764e-05
1943 1.35048867377918e-05
1944 1.3056087482255e-05
1945 1.15358852781355e-05
1946 1.02000803963165e-05
1947 1.26310033010668e-05
1948 1.35951904667309e-05
1949 1.21001239676843e-05
1950 1.04353666756651e-05
1951 1.15035636554239e-05
1952 1.37768829517881e-05
1953 1.25942106024013e-05
1954 1.09888405859238e-05
1955 1.06222032627556e-05
1956 1.34368892759085e-05
1957 1.30765556605184e-05
1958 1.15489192467066e-05
1959 1.02719768619863e-05
1960 1.26618861031602e-05
1961 1.34837910081842e-05
1962 1.20246240840061e-05
1963 1.04272321550525e-05
1964 1.16974351840327e-05
1965 1.36487687996123e-05
1966 1.2452215742087e-05
1967 1.08717085822718e-05
1968 1.08529875433305e-05
1969 1.342886389466e-05
1970 1.28872779896483e-05
1971 1.13936157504213e-05
1972 1.03903921626625e-05
1973 1.27974890347105e-05
1974 1.33097228172119e-05
1975 1.18884063340374e-05
1976 1.04143082353403e-05
1977 1.18765410661581e-05
1978 1.35668988150428e-05
1979 1.23578811326297e-05
1980 1.0819208000612e-05
1981 1.09581560536753e-05
1982 1.34394167616847e-05
1983 1.28547935673851e-05
1984 1.13826827146113e-05
1985 1.04104919955716e-05
1986 1.28078445413848e-05
1987 1.335948400083e-05
1988 1.19413416541647e-05
1989 1.04416512840544e-05
1990 1.17982799565652e-05
1991 1.36641037897789e-05
1992 1.2476561096264e-05
1993 1.09245438579819e-05
1994 1.08206622826401e-05
1995 1.34777437779121e-05
1996 1.30450753204059e-05
1997 1.15509938041214e-05
1998 1.0343254871259e-05
1999 1.27085595522658e-05
};
\addlegendentry{Test}

\end{groupplot}

\end{tikzpicture}
		\caption{Eight experiments with different combinations of activation functions for the model with two (left) and four (right) convolutional layers in encoder. Shown are training- and validation error over 2000 epochs.}
		\label{Fig:ActivationsC}
	\end{figure}
\end{center}
To this end the finding of hyperparameters for a convolutional autoencoder using both rarefaction levels as input data failed to produce comparable results to those obtained by the fully connected autoencoders using both rarefaction levels as separate datasets. The analysis stops here out of brevity, but is far from an end. Changing the kernel size and especially the stride width, which yields non overlapping kernel positions in all models, changing the loss function to for example \texttt{pyTorch}'s BCEWithLogitsLoss or even adding more layers and so forth are still not analyzed, but are proposed for further investigations. As a final model the two layer model trained without data augmentation is chosen. This model encompasses the smallest number of free parameters while achieving the lowest validation error.
\begin{table}[htbp!]
	\centering
	\caption{Final model.}
	\begin{tabular*}{15cm}{ @{\extracolsep{\fill}} c c c c c c @{} }
		\toprule
		Layer & Channels  & Activations & Batch size & Learning rate & Num. Epochs \\ [.5ex]
		\hline
		2    & 8,16   	  & ELU/SiLU    & 4           & \num{1e-4} & 2000 \\
		\hline
	\end{tabular*}\label{Tab:FinalC}
\end{table} 
  